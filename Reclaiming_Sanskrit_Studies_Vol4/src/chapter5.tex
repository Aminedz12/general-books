\chapter[“From Rasa Seen to Rasa Heard”: A Criticism]{“From Rasa Seen to Rasa Heard”: A Criticism}\label{chapter\thechapter:begin}
%~ \footnotetext[1]{pp.~\pageref{chapter\thechapter:begin}--\pageref{chapter\thechapter:end}. In: Kannan, K S (Ed.) (2018) {\sl Śāstra-s Through the Lens of Western Indology - A Response}. Chennai: Infinity Foundation India.}

%~ \blfootnote{}

\vskip -.5cm

\Authorline{Sreejit Datta}
\lhead[\small\thepage\quad Sreejit Datta]{}

\vskip -.5cm

\section*{Abstract}

The paper will take a close look at Prof. Sheldon Pollock’s depiction of the evolution of the idea of \textsl{rasa}\index{rasa@\textsl{rasa}} in the context of the Sanskrit tradition. The “fundamental difference” between “literature seen”\index{literature seen@“literature seen”} and “literature heard”\index{literature heard@“literature heard”} that Pollock uses almost as an axiom in his essay “From Rasa Seen to Rasa Heard” (2012) will be disputed in this paper. In his essay, Pollock tries to show that this differentiation had already occurred by the beginning of the eleventh century or even earlier. However, the present exercise will problematize this position by drawing the reader’s attention to the liminal nature of what has been known as “\textsl{sāhitya}”\index{sahitya@\textsl{sāhitya}} in pre-modern India. Pollock’s axiomatic assertions are challenged on various grounds, including the non-scriptocentricity of \textsl{sāhitya}. The distinction(s) between the Western category ‘literature’ and the Indian category ‘\textsl{sāhitya}’ underlines the epistemological differences existing between these two locations. The paper argues, with copious examples taken from both Pollock’s essay as well as original Indian sources, that “The cited (2012) article of Pollock” is essentially an exercise in peddling Western universalism.       

[{\bf Keywords:} \textsl{Rasa}, \textsl{sāhitya,} Literature, performative, pre-modern, postcolonial]

\section*{Introduction}

Epistemological divergences in the development of the two divergent intellectual traditions viz.~Western and Indian, despite all their intra-traditional heterogeneity, account for the differences in their conceptions of literature and the arts; and also about aesthetics (which the \textsl{Merriam-Webster Dictionary} defines as “a set of ideas or opinions about beauty or art; the study of beauty especially in art and literature; (1) a branch of philosophy dealing with the nature of beauty, art and taste and with the creation and appreciation of beauty; (2) a particular theory or conception of beauty or art: a particular taste for or approach to what is pleasing to the senses and especially sight”). Clearly, the term has been defined in both its general and particular senses in the English language, as far as the evidence of “the most trustworthy dictionary and thesaurus of American English” counts. 

The present exercise intends to throw light on the large gaps existing between the Indian and Western ways of looking at the concept of aesthetics by offering a close reading of Pollock 2012; especially of the way he distinguishes \textsl{Dṛśya-kāvya}\index{drsyakavya@\textsl{dṛśya-kāvya}} (which he translates as “Rasa Seen”) and \textsl{Śravya-kāvya}\index{sravya@\textsl{śravya}} (as “Rasa Heard”) in order to create an axiomatic base upon which build his arguments. 

These arguments are later taken up and used as a framework to construct an “Intellectual History of Rasa”\index{rasa@\textsl{rasa}} (this is how the introduction, to his book \textsl{A Rasa Reader}\index{A Rasa Reader@\textsl{A Rasa Reader}} is subtitled) which he claims to have achieved in (2016) the said book. The idea of \textsl{rasa} can be fairly regarded as one of the grandest contributions of Indian intellectual advancement to the progressive intellectual movement of the world in general. He posits his own theory of the evolution of the idea of \textsl{rasa}, based on the aforementioned postulate that draws a clear distinction between the allegedly two different kinds of \textsl{rasa}. This paper takes exception to his postulate (and its derivatives), and problematizes his conclusions therein by drawing upon a fresh discussion of the (literary and performative) historiography of the idea of \textsl{rasa} over the past centuries, and pointing out the Aristotelian\index{Aristotle} approach taken by him and his predecessors in their analyses of the aesthetic principles expounded in Sanskrit.

In order to do the same, I contend that there is a need to absolve the discourses on Sanskrit and aesthetics of the sin of domestication of technical terms from the Sanskrit tradition on the part of the Western scholar. To give a few clichéd (yet dangerously misleading) examples of the kind of domestication that the Western scholar of Sanskrit often resorts to, it would suffice to draw the readers’ attention to their usage of the terms ‘classical’ and ‘literature’ while translating the Indic terms \textsl{mārga/śāstrīya}\index{marga@\textsl{mārga}} and \textsl{sāhitya},\index{sahitya@\textsl{sāhitya}} in a manner which I believe is not inadvertent. There are evidentiary reasons to believe thus, as it has been a favourite design of the Orientalists of the past and The Neo-Orientalists\index{Neo-Orientalist} of the present (whose cause he has championed) to label Sanskrit as a ‘classical’ language and hence jumping to the conclusion that the entire scholarship and canons written in that language to be ‘classical’ --- dispensing it something of the stature of Classical Greek or Classical Latin. 

Such a presumption blocks the view of every new entrant to the discourse who wishes to understand the matter and contribute to the debate. It is not only a gross injustice to the multitude of people who use the Sanskrit language on a daily basis for a plethora of purposes --- both religious and secular, it is also discourteous to the Constitution of the Republic of India which has regarded Sanskrit as one of the scheduled languages as described and declared in its Eighth Schedule. 

Thus, this paper seeks to address the implications (which, through their reiterations via various channels of dispensing such products of negative knowledge, turn into insinuations) of claiming the presence of apparent discontinuities and incoherence in the evolutionary course of the grand narrative of \textsl{rasa}\index{rasa@\textsl{rasa}} Pollock 2012. In a way, this may help in setting the records of the discourses in aesthetics prevailing in India since antiquity in Sanskrit straight, and dismantle the vicious propaganda around the alleged disharmony of Indian traditional ideas and indigenousness of Sanskrit and \textsl{saṁskṛti}.\\[-21pt]

\section*{Position of the \textsl{Pūrva-pakṣin} in the Discourse}
\index{purvapaksa@\textsl{pūrvapakṣa}}

Let me briefly lay out the methodology which I shall employ here to achieve the goals mentioned previously. The “fundamental difference” between literature seen\index{literature seen@“literature seen”} and literature heard\index{literature heard@“literature heard”} that Pollock uses almost as an axiom in his discussion will be disputed by referring back to the spearheading text, the “\textsl{Gomukha}” of \textsl{rasa-śāstra},\index{rasa@\textsl{rasa}} the fountainhead of Indian aesthetic theories viz.~the \textsl{Nāṭya-śāstra}\index{Natyasastra@\textsl{Nāṭya-śāstra}} of Bharata Muni,\index{Bharatamuni} among other authoritative traditional sources. Pollock tries to show that this differentiation, and what is more, a radical opposition, between the allegedly diverse aspects of this “binary” leading to a contest for primacy, had already become “a running dispute” by the beginning of the eleventh century, and even much earlier than that (Pollock 2012:190). However, the present exercise problematizes this position by drawing the reader’s attention not just to the \textsl{kārikas} from the \textsl{Nāṭya-śāstra}\index{Natyasastra@\textsl{Nāṭya-śāstra}} that delineate a direct equivalence between \textsl{Driśya-kāvya}\index{drsyakavya@\textsl{dṛśya-kāvya}} and \textsl{Śravya-kāvya}\index{sravyakavya@\textsl{śravya-kāvya}} as well as the observations of the \textsl{arvācīna ālaṅkārika}-s, but also to the liminal nature of what has been known as “\textsl{sāhitya}”\index{sahitya@\textsl{sāhitya}} in pre-modern India.

Apart from the obvious performative tendencies (by virtue of several essentially performatory techniques such as vocal intonation, emphasis, \textsl{mudrā}\index{mudra@\textsl{mudrā}} etc. employed by the reciter or \textsl{kathaka})\index{kathaka@\textsl{kathaka}} of what Pollock calls “literature heard”,\index{literature heard@“literature heard”} which cannot exist independently of a reciter-audience interface, and which is (and not was) by and large interactively lively, a major issue that can question this axiomatic basis, is the very idea of \textsl{sāhitya} (literally, fellowship / association / combination / society / harmony) --- an idea that is radically different from literature (origin: Latin \textsl{litteratura} --- writing, grammar, learning). \textsl{Sāhitya} is a term which, by its very definition, maintains the continuum, the fluidity between categories such as drama and poetry on, one hand, and a myriad performative forms on the other. 

In other words, the contention highlighted by this paper is that the function of Indian aesthetic concepts like \textsl{kāvya},\index{kavya@\textsl{kāvya}} \textsl{sāhitya} and \textsl{rasa}\index{rasa@\textsl{rasa}} is unification of elements which are otherwise perceived as being different -- unlike the Western concepts such as drama, poetry and literature, whose primary function is to classify creative works into watertight boxed categories. The latter has been one of the major goals of Aristotle’s\index{Aristotle} \textsl{Poetics}\index{Poetics@\textsl{Poetics}} (the source text for all these Western concepts related to the discussion of aesthetics), and hence the Indian terms (whose semantic shift is a predominantly postcolonial phenomenon) should be treated according to their own distinct epistemological position: one has to be very careful especially in Anglophone discourses lest one may fall prey to Western universalism. 

It is necessary to keep in mind the major difference between these two epistemologies when discussing these issues in a comparative framework --- which is what Pollock has accomplished in his essay. This crucial difference has been blurred by Pollock’s translation strategies in particular, and his theoretical framework to interpret \textsl{rasa}\index{rasa@\textsl{rasa}} in general; thus making his approach to \textsl{rasa} vulnerable to criticism and blameworthy of subtly imposing Western universalism upon the discourse on Indian aesthetics --- an approach that would amount to bad comparativism.\\[-21pt]

\section*{Basic Empirical Evidences from the \textsl{Nāṭya-śāstra} etc.}
\begin{quote}
\textsl{mahendra-pramukhair devair uktaḥ kila pitāmahaḥ} |\\
\textsl{krīḍanīyakam icchāmo dṛśyaṁ śravyaṁ ca yad bhavet} ||\\
\textsl{na veda-vyavahāro’yaṁ saṁśrāvyaḥ śūdra-jātiṣu} |\\
\textsl{tasmāt sṛjāparaṁ vedaṁ pañcamaṁ sārvavarṇikam} ||   

\hfill(\textsl{Nāṭya-śāstra}\index{Natyasastra@\textsl{Nāṭya-śāstra}} 1.11-12)  
\end{quote}

The above \textsl{śloka} from the first chapter describing the origins of the dramatic arts (or rather the \textsl{Nāṭya-veda}) when roughly translated into English reads: 

“The Great Indra\index{Indra@Indra} and the other gods said to the Grandsire (Bhagavān Brahmā):\index{Brahma@Brahmā} we wish such an entertainment that will be both for the eyes and the ears simultaneously (1.11). Since the \textsl{Veda-s} are not for the ears of the \textsl{śūdra-}s, therefore do create a Fifth \textsl{Veda} that will be for the perusal of all the (four) \textsl{varṇas} (1.12)”. 

Of special significance to the purpose of this exercise is the use of the words ‘\textsl{dṛśya}’\index{drsya@\textsl{dṛśya}} and ‘\textsl{śravya}’\index{sravya@\textsl{śravya}} in the last pada of the first verse and the reappearance of the word ‘\textsl{saṁśrāvya}’ in the first pada of the next verse. Since Pollock is particularly anxious to draw a dividing line between \textsl{dṛśya} and \textsl{śravya} by distinguishing between his hypothesized types of \textsl{rasa}\index{rasa@\textsl{rasa}} as “Rasa Seen” and “Rasa Heard”, respectively, it becomes necessary to draw his attention to what the text has to say in this regard. By maintaining an irrefutable equivalence between the role of the ‘\textsl{dṛśya}’\index{drsya@\textsl{dṛśya}} and the ‘\textsl{śravya}’\index{sravya@\textsl{śravya}}, the \textsl{Nāṭya-śāstra}\index{Natyasastra@\textsl{Nāṭya-śāstra}} clears its stance at the very outset of its discourse between the sages and Bharata\index{Bharatamuni} on the dramatic art and the concepts pertaining to the field of aesthetics by drawing the equivalence between the twin aspects of “the Seen” and “the Heard”. The next verse again emphasizes on the \textsl{śravya} aspect of the \textsl{Veda-s}, which are collection of hymns, -- poems and prose passages -- with regard to the prohibition of the \textsl{śūdra}-s hearing them. If the Nāṭya-text did really distinguish between the two types posited of \textsl{rasa}, then the prescription of the gods led by the Great Indra\index{Indra@Indra} would not be inclusive of both kinds of the arts --- \textsl{dṛśya}\index{drsya@\textsl{dṛśya}} and \textsl{śravya}. Instead, I believe, they would be dismissive of the spectacular aspect that requires a lot more work in terms of depiction; as it includes the stage preparation, props, dresses, backdrop, painting and a lot of other branches of the dramatic art. It would be pertinent to mention here that the \textsl{Nāṭya-śāstra} would, from this juncture, move on to the descriptions and prescriptions regarding stagecraft, props, dresses and make-up in Chapters 2 (\textsl{prekṣā-gṛha-lakṣaṇa}),\index{preksagrhalaksana@\textsl{prekṣā-gṛha-lakṣaṇa}} 3 (\textsl{raṅga-devatā-pūjā})\index{rangadevatapuja@\textsl{raṅga-devatā-pūjā}} and 23 (\textsl{āhārya-abhinaya})\index{aharyaabhinaya@\textsl{āhārya-abhinaya}} respectively. 

Pollocks comments on the historicity of the \textsl{Nāṭya-śāstra} at this point, will be relevant to our discussion. In the preface to his book,\index{A Rasa Reader@\textsl{A Rasa Reader}} he states:

\begin{myquote}
“The \textsl{Treatise on Drama} [i.e.~the \textsl{Nāṭya-śāstra};\index{Natyasastra@\textsl{Nāṭya-śāstra}} it is worth noting how Pollock tends to translate even the titles of well known, Sanskrit works] was undoubtedly revised, possibly in Kashmir in the eighth century, but the work as a whole is as much as five centuries older. It therefore must come first, despite the likelihood that its earliest commentators knew nothing of some ideas it advances in the form we now have it.” 
\hfill (Pollock 2016) 
\end{myquote}

By his own accounts, Pollock therefore places the \textsl{Nāṭya-śāstra}\index{Natyasastra@\textsl{Nāṭya-śāstra}} as in the third Century C.E. and designates it to be the first treatise to be composed in India on the subject of aesthetics. This when contrasted to his claim made in his article that 

\begin{myquote}
“[w]hatever other questions may be at issue here, it should be clear that by the beginning of the eleventh century and no doubt far earlier, drama, or literature seen,\index{literature seen@“literature seen”} and poetry, or literature\index{literature heard@“literature heard”} heard, constituted two fundamentally different and differentiated forms of literature, and indeed, that there already was a dispute about the extension of \textsl{rasa} analysis from the one sphere to the other”\hfill (Pollock 2012:191) 
\end{myquote}

reveals that Pollock either grants that during the long stretch of time from no earlier than the third Century C.E. till no later than the eleventh Century C.E. there was a general agreement among the Sanskrit critics in India about the equivalence of what he calls “Rasa\index{rasa@\textsl{rasa}} Seen” and “Rasa Heard”, or he ignores the period in his proposed effort to “reconstruct[ion] of the history of the extension of aesthetical analysis from the dramatic to the non-dramatic” (Pollock 2012:189). He himself observes (in footnote 1) on \textsl{Nāṭya-śāstra} 1.11, adding “[b]ut note that \textsl{Nāṭya-śāstra} 1.11 speaks of drama itself as both \textsl{dṛśya}\index{drsya@\textsl{dṛśya}} and \textsl{śravya}.”\index{sravya@\textsl{śravya}} (Pollock 2012:189) 

Now, in order to problematize the timeline, provided by Pollock, during which the Sanskrit tradition allegedly differentiated between the two types of \textsl{rasa-}s,\index{rasa@\textsl{rasa }} we need to mention certain sources which are considered no less canonical within the same tradition, but fall within that disputed timeline of the evolution of the tradition and contradicts Pollock’s claims. The first such example is from Nandikeśvara’s\index{Nandikesvara@Nandikeśvara} \textsl{Abhinaya-darpaṇa},\index{Abhinayadarpana@\textsl{Abhinaya-darpaṇa}} which is a product of a school of thought that predates the \textsl{Nāṭya-śāstra}. According to Ramakrishna Kavi, this formidable rival of Bharata came before Bharata\index{Bharatamuni} in time. Some have even conjectured Nandikeśvara\index{Nandikesvara@Nandikeśvara} to be Bharata’s guru. Swami Prajñānānanda\index{Prajnanananda@Prajñānānanda} has quoted Alain Danielou\index{Danielou, Alain} to mention that Indian and Western scholars have placed Nandikeśvara’s school of thought between the fifth and second centuries B.C.E.; even though the final penning of this treatise was believed to have been completed only after that of the \textsl{Nāṭya-śāstra}.\index{Natyasastra@\textsl{Nāṭya-śāstra}} (Prajñānānanda 1961) See 35-36:
\begin{quote}
\textsl{āsyenālambayed gītaṁ hastenārthaṁ pradarśayet} |  \\
\textsl{cakṣurbhyāṁ darśayed bhāvaṁ pādābhyāṁ tālamādiśet} ||   \\
\textsl{yato hastas tato dṛṣṭir yato dṛṣṭis tato manah} | \\
\textsl{yato manas tato bhāvo yato bhāvas tato rasaḥ} || 
\end{quote}

These \textsl{śloka}-s neatly demonstrate the sequential causality, tracing the causal relationship in the form of a chain from \textsl{gīta} (songs which incorporate spoken words with tunes) to \textsl{rasa}. A rough translation: “the mouth is the seat of the song, hands should demonstrate the meaning, the eyes should reflect the \textsl{bhāva},\index{bhava@\textsl{bhāva}} and the legs should indicate the \textsl{tāla}.\index{tala@\textsl{tāla}} The glance follows the hand; the mind follows the glance; the \textsl{bhāva} follows the mind; and the \textsl{rasa} follows the \textsl{bhāva}. Such a theorization of the idea of \textsl{rasa} seamlessly combines the \textsl{dṛśya}\index{drsyakavya@\textsl{dṛśyakāvya}} and the \textsl{śravya}\index{sravya@\textsl{śravya}} aspects. 

But this should not necessarily suggest that the elements of \textsl{dṛśya} and \textsl{śravya} can be separately treated while dealing with the idea of \textsl{rasa}. The emphasis is rather on their indivisibility. The emphasis is not on the duality (or even multiplicity) of \textsl{rasa}. In other words, the concept of \textsl{rasa} is not a synthetic one, forcing a fusion of \textsl{dṛśya}\index{drsyakavya@\textsl{dṛśya-kāvya}} and \textsl{śravya};\index{sravya@\textsl{śravya}} it is rather an integrally unified concept. The dynamism of the concept lies in the scope it has provided to aestheticians appearing after Nandikeśvara\index{Nandikesvara@Nandikeśvara} and Bharata;\index{Bharatamuni} most of whom agree on the comparability of the experience (``\textsl{āsvādana}'',\index{asvadana@\textsl{āsvādana}} relish) of \textsl{rasa} with experience of the Self or Brahman. It is for this reason that the experience of \textsl{rasa} has been termed “\textsl{Brahmāsvāda-sahodara}”\index{brahmasvadasahodara@\textsl{brahmāsvāda-sahodara}} --- born of the same womb as the taste or experience of the \textsl{Brahman} --- such as is in the auto-commentary of Śubhaṅkara\index{Subhankara@Śubhaṅkara} on \textsl{Saṅgīta-dāmodara}\index{Sangitadamodara@\textsl{Saṅgīta-dāmodara}} 13.5.~(Mukhopadhyaya\index{Mukhopadhyaya, Mahuya} 2009). A section from Gupteshwar Prasad’s\index{Prasad, Gupteshwar} book on \textsl{Rasa} merits quotation in full in this context:

\begin{myquote}
“[Viśvanātha]\index{Visvanatha@Viśvanātha Kaviraja} assigns eight qualities to \textsl{Rasa}. It is \textsl{Akhanda}, \textsl{Sva-prakasa, Anandamaya, Cinmaya, Vedyantara-sparsa-sunya, Brahmasvada-sahodara, Loko\-ttara-camatkara-prana} and \textsl{Svakaravad-abhinna}. \textsl{Rasa}\index{rasa@\textsl{rasa}} is \textsl{Akhanda}. This means it is indivisible and is relished by all the \textsl{Sahrdayas}\index{sahrdaya@\textsl{sahṛdaya}} alike. Though \textsl{Rasa} is constituted of its component parts i.e. \textsl{Vibhava,\index{vibhava@\textsl{vibhāva}} Anubhava}\index{anubhava@\textsl{anubhāva}} and \textsl{Sancaribhava},\index{sancaribhava@\textsl{sañcāribhāva}} none of its parts can be separated from it. \textsl{Rasa} is self-effulgent or \textsl{Sva-prakasa} i.e., no device is necessary to bring it to light. \textsl{Rasa} is \textsl{Anandamaya}. This implies that the personal experience of the \textsl{Sahrdaya}\index{sahrdaya@\textsl{sahṛdaya}} takes the shape of \textsl{Rasa} which by its very nature is blissful. \textsl{Rasa} is \textsl{Cinmaya}. This means that it pervades or is permeated by consciousness. It affords us pleasure which is different from ordinary worldly pleasure. \textsl{Rasa}\index{rasa@\textsl{rasa}} is \textsl{Vedantarasparsasunya} meaning while experiencing \textsl{Rasa}, no other knowledge (\textsl{vedya} or \textsl{jnana}) intrudes and interferes in the realization of the \textsl{Sahrdaya}.\index{sahrdaya@\textsl{sahṛdaya}} \textsl{Rasa} is \textsl{Brahmasvadasahodara}. This means that for the time being the \textsl{Sahrdaya} derives similar pleasure from poetry which the Yogi gets in the communion with Brahma with the difference that \textsl{Brahmasvada} is never followed by \textsl{Laukika Vikaras} whereas \textsl{Kavyasvada} may be followed by such \textsl{Vikaras}. \textsl{Rasa} is \textsl{Lokottara-camatkaraprana}. This refers to the enlargement of the heart which is the natural result of \textsl{Ahlada}. This is to say, though the \textsl{Ahlada} of Rasa is worldly, it is basically different from other worldly \textsl{Ahladas}. The expression \textsl{Svakaravadabhinna} means that \textsl{Rasa} is relished in an integral or \textsl{Abhinna} form. Visvanatha tells us that the \textsl{Pramata} or \textsl{Sahrdaya}\index{sahrdaya@\textsl{sahṛdaya}} enjoys \textsl{Rasa} only with \textsl{Sattvodreka} i.e. when his mind is completely purged of \textsl{Rajoguna} and \textsl{Tamoguna}….[t]he word \textsl{Camatkara} seems to have been borrowed by the Sanskrit literary critics from philosophy. In \textsl{Yogavasistha}, the word \textsl{Camatkara} is used in the sense of self-flashing of thought.” 

\hfill (Prasad\index{Prasad, Gupteshwar} 1994:133) (\textsl{diacritical marks not in the original})
\end{myquote}

It is indeed true, as is evident from the above excerpt, that the development of Indian aesthetic ideas in the Sanskrit tradition grew in parallel with the evolution of philosophical ideas; and each functioned as complementary to the other’s growth. Two fine examples are the application of the terms ‘\textsl{Brahmāsvāda-sahodara}’\index{brahmasvadasahodara@\textsl{brahmāsvāda-sahodara}} and ‘\textsl{camatkāra}’ in critical exegeses, as are provided by Prasad.\index{Prasad, Gupteshwar} The \textsl{akhaṇḍatā} of \textsl{rasa}\index{rasa@\textsl{rasa}} as explained by Viśvanātha\index{Visvanatha@Viśvanātha Kaviraja} Kaviraja in \textsl{Sāhitya Darpaṇa} has not been given its due place in the historicism-oriented analysis carried out by Pollock; neither does he acknowledge the connection with philosophy, metaphysics and the spiritual dimensions. To this effect, his methodology  betrays selectivism with translation strategies that erase implications valuable to dharmic traditions.\\[-21pt]  

\section*{Further Meditations on the Antiquity of the Idea of \textsl{Rasa}}

Let us also bring Upaniṣadic voices here. 
\begin{quote}
\textsl{yad vai tat sukṛtam} | \textsl{raso vai saḥ} | \textsl{rasaṁ hy evāyaṁ labdhvānandī bhavati} | 
\hfill\hbox{(\textsl{Taittirīyopaniṣad}\index{Taittiriyopaniṣad@\textsl{Taittirīyopaniṣad}}
\index{Upanisad@\textsl{Upaniṣad}!Taittiriya@\textsl{Taittirīya}} 2.7)}

\vskip .1cm

\textsl{ānando brahmeti vyajānāt} | \textsl{ānandād dhy eva khalv imāni bhūtani jāyante} | \textsl{ānandena jātāni jīvanti} | \textsl{ānandaṁ prayanty abhisaṁviśantīti} | 
\hfill\hbox{(\textsl{Taittirīyopaniṣad}\index{Taittiriyopaniṣad@\textsl{Taittirīyopaniṣad}}\index{Upanisad@\textsl{Upaniṣad}!Taittiriya@\textsl{Taittirīya}} 3.6)}
\end{quote}

In these two passages from the \textsl{Taittirīyopaniṣad}, mention has been made of \textsl{rasa}\index{rasa@\textsl{rasa}} along with its function (i.e. production of \textsl{ānanda} in the individual’s experience) in unequivocal terms. This \textsl{Upaniṣad} forms part of the \textsl{Kṛṣṇa Yajurveda},\index{Krsna Yajurveda@\textsl{Kṛṣṇa Yajurveda}}\index{Veda-s@\textsl{Veda}-s!Yajus@\textsl{Yajus}} the \textsl{Veda} which compiles the prose mantras. The mantras of the \textsl{Yajurveda} and its prose \textsl{saṁhitā} texts have been dated between 1200 B.C.E. --- 800 B.C.E. by the most recent linguistic studies (even those that ardently support the Aryan Invasion Theory), which are also corroborated with archaeological tests of the Painted Gray Ware culture, a special style of pottery used by the elite people of the time (Witzel\index{Witzel, Michael} 2000). 

The first fragment, from \textsl{Brahmānandavallī},\index{Taittiriyopaniṣad@\textsl{Taittirīyopaniṣad}}
\index{Upanisad@\textsl{Upaniṣad}!Taittiriya@\textsl{Taittirīya}} clearly declares that that ‘\textsl{sukṛtam}’ (well-done/well-made, or self-made, \textsl{sva-kṛtam}) is nothing but \textsl{rasa}; it resides in every being in the form of essence. This word \textsl{rasa} is untranslatable; it can be approximately conveyed by such words in English as ‘essence’, ‘sap’, ‘extract’ ‘sublimity’, even ‘aesthetics’. Next it says, “\textsl{ayaṁ hi rasam eva labdhvā ānandī bhavati}”, i.e. the individual who realizes (‘\textsl{labdhvā}’, having attained the ‘\textsl{upalabdhi}’ or realization) the existence of \textsl{rasa} (in her conscious being) becomes happy (‘\textsl{ānandī}’). Therefore, the function of \textsl{rasa} is the production of elation. The \textsl{Upaniṣads} are metaphysical treatises dealing with the \textsl{pāramārthika}\index{paramarthika@\textsl{pāramārthika}} dimension of the human experience; and since \textsl{rasa} has been used not merely as a metaphorical idea but an inductive idea in it, one can assert with some confidence that \textsl{rasa}\index{rasa@\textsl{rasa}} itself has a \textsl{pāramārthika} idea which is induced (much in the same way as mathematical induction does) by the testimonies of the \textsl{vyāvahārika} experience.

The second fragment takes a step further in this induction and concludes by saying that \textsl{ānanda} itself is realised as \textsl{Brahman}. “\textsl{Ānandāt hi eva}” --- “from \textsl{ānanda} itself”, it says, “\textsl{khalu imāni bhūtāni jāyante}” i.e. “is generated all that is”. \textsl{Ānanda} nourishes them all (while they \textsl{are} in the physical plane); they depart into and finally merge into \textsl{ānanda} in the end. Therefore \textsl{ānanda} is proclaimed as the \textsl{Brahman} in this fragment, which is placed in the third \textsl{khaṇḍa} known as Bhṛguvallī, preceded by the proclamation “\textsl{raso vai sah}” in the second \textsl{khaṇḍa} of the same text. The placement of these two proclamations seem to be more than random, they visibly reveal a step-by-step building up of a logical sequence of arguments. The realization of \textsl{rasa}\index{rasa@\textsl{rasa}} begets \textsl{ānanda}, and \textsl{ānanda} is \textsl{Brahman}; therefore \textsl{rasa} --- the cause --- is equated with \textsl{ānanda} --- the effect --- in a classic Left Hand Side/Right Hand Side style-equation: 

\textsl{Sukṛta} (or \textsl{Svakṛta} or \textsl{Brahman}) = \textsl{Rasa};\index{rasa@\textsl{rasa}}

Realization of \textsl{Rasa} = Production of \textsl{Ānanda}

\textsl{Ānanda} = \textsl{Brahman} (\textsl{Sukṛta} or \textsl{Svakṛta} )

Therefore, Realisation of \textsl{Rasa}\index{rasa@\textsl{rasa}} = Realisation of \textsl{Brahman}

Indeed, Viśvanātha\index{Visvanatha@Viśvanātha Kaviraja} (1400 C.E.) in his \textsl{Sāhitya Darpaṇa}\index{Sahityadarpana@\textsl{Sāhitya Darpaṇa}} testifies to this by using many of the attributes which reflect the idiom of the \textsl{Upaniṣads} in his attempt to define \textsl{rasa} (\textsl{akhaṇḍa}, \textsl{svaprakāśa}, \textsl{ānanda-cinmaya}, \textsl{vedyāntara-sparśaśūṇya}). He finally describes it as \textsl{brahmāsvāda-sahodara}\index{brahmasvadasahodara@\textsl{brahmāsvāda-sahodara}} (akin to the taste/experience/realization of the \textsl{Brahman}) and \textsl{lokottara} (transcendental, \textsl{pāramārthika}).\index{paramarthika@\textsl{pāramārthika}} (Sinha\index{Sinha, Jadunath} 1986:163)

There are important implications of the establishment of this causal relationship between \textsl{rasa},\index{rasa@\textsl{rasa}} \textsl{ānanda} and \textsl{Brahman}. Swami Vivekananda explains the indivisibility of the binary of cause-and-effect as follows: 

“The idea of the cause we get from the idea of the effect, and if there is no effect, there will be no cause.” 

And, 

\vskip .1cm

“Nothing can be produced without a cause, and the effect is but the cause reproduced.”   

\hfill -- Swami Vivekananda (See “Soul, Nature, and God”)

\vskip .1cm

The \textsl{Brahman} is \textsl{avāṅmanasa-gocara} (not accessible to either speech or mind; “\textsl{yato vaco nivartante} | \textsl{aprāpya manasā saha} |” i.e. “from It [\textsl{Brahman}] speeches and the mind return without getting anything” --- \textsl{Taittirīyopaniṣad}\index{Taittiriyopaniṣad@\textsl{Taittirīyopaniṣad}}\index{Upanisad@\textsl{Upaniṣad}!Taittiriya@\textsl{Taittirīya}} 2.4) and hence its experience is independent of the function of the sense-organs. Now, if the experience of the \textsl{Brahman} has been equated with the experience of \textsl{rasa}, then it may be deduced that the experience of \textsl{rasa} is also independent of the functions of the senses.

Pollock hardly attaches any importance to these aspects of the \textsl{rasa} analysis carried out by most traditional luminaries of Indian aesthetics. He hardly addresses these issues and as a consequence, his analytical vision gets narrowed down, compelling him to conclude that “the concept of \textsl{rasa} was extended from literature seen\index{literature seen@“literature seen”} to literature heard”\index{literature heard@“literature heard”} or that at some point in the history of the Sanskrit tradition there occurred a “[shift in the] ontology of \textsl{rasa} where it moved from the seen to the heard.” (Pollock 2012:191) 

This approach negates even the slightest possibility of taking a dialectical method of finding the ‘truth’ about \textsl{rasa},\index{rasa@\textsl{rasa}} something which Pollock hints at by asserting the need to show that “the Sanskrit tradition differentiated between the two types of literature, or better yet, that it drew an opposition indicating that analysis applicable in one domain might not be automatically applicable in the other.” (Pollock 2012:189)\\[-21pt]

\section*{Misuse of Translation: for Obliteration}

Here arises a need to draw the reader’s attention to the kind of problem that occurs as a consequence of Pollock’s undifferentiated use of the term ‘literature’ in the context of Indic traditions. At this point I should also clarify that I am deliberately using the terms “Indic traditions” and “Sanskrit traditions” interchangeably; Such an approach, I believe, allows the freedom to equate the term with “Sanskrit traditions” --- actually more so in a discourse (conducted in the English language) on Sanskrit aesthetics, which is the fountainhead of not just literary but all artistic sensibilities in the Indian subcontinent and beyond --- with some amount of impunity. But the translation (which, in the case of any effort to translate an Indic language into English or some other European language), mostly amounts to finding very roughly replaced semantic approximations. 

It is not so in the case of translations undertaken between Indic languages (included Sanskrit), wherein both semantic as well as emotive aspects are preserved across languages to a degree far exceeding the Indic languages-to-English translations) of the Sanskrit word ‘\textsl{sāhitya}’\index{sahitya@\textsl{sāhitya}} into ‘literature’ in an academic discourse is a rather loosely done articulation. This amounts to either a lack of care for the concept of ‘\textsl{sāhitya}’, or a more devious, conscious attempt at what may be designated, following Michel de Certeau’s\index{de Certeau, Michel} work, “epistemological violence” or “epistemological pacification” (Highmore\index{Highmore, Ben} 2006:83). Vazquez\index{Vazquez, Rolando} has drawn our attention to this function of translation by reading translation as “erasure” and connected it with de Certeau’s ideas of epistemological domination of one culture by another in the name of translating and interpreting it (Vazquez 2011:27). 

Such theoretical frameworks become useful in situating what Pollock \textsl{et al} had been trying to do through a reductive approach of translation of each and every Sanskrit terminology pertaining to \textsl{alaṅkara-śāstra},\index{alankarasastra@\textsl{alaṅkara-śāstra}} in order to anglicize them -- leading to a systematic erasure of the vast differences between the Indic and anglicized terms. Modern translation studies recognizes two broad approaches to translating any text: domestication and foreignization. The latter is generally observed to occur more often in translations which are targeted to a specialist readership, such as those target language texts which are meant for scholars and specialists rather than lay reading. Such translations are naturally heavily annotated. Eugene Nida,\index{Nida, Eugene} an American translation theorist who had specialized in the translation of the Holy Bible from its original/source language into modern-day English, is hailed as the representative of the argument for domestication in translations. On the other hand, another American translation theorist, Lawrence Venuti,\index{Venuti, Lawrence} who frequently translates from Italian and Catalan into English, has championed the cause of foreignization in translation strategies. 

A careful reading of Pollock’s works on \textsl{rasa} reveals that he resorts to translating almost every technical term pertaining to \textsl{alaṅkara-śāstra} and even the Sanskrit titles of technical treatises on \textsl{rasa/alaṅkāra-śāstra}\index{rasa@\textsl{rasa}}\index{alankarasastra@\textsl{alaṅkara-śāstra}} into English --- an endeavour which amounts to, to use a simile, translating ‘\textsl{yoga}’ as ‘addition’ or ‘connection’ --- the literal English translation of that Sanskrit word. Such an exercise obliterates\index{misinterpretation!techniques of!obliteration via translation} by domesticating and expropriating a whole linguistic cosmology which is radically different from the Western cosmology. To give examples from his latest work: \textsl{A Rasa Reader},\index{A Rasa Reader@\textsl{A Rasa Reader}} Pollock translates Ānandavardhana’s\index{Anandavardhana@Ānandavardhana} famous and path-breaking concept of “\textsl{dhvani}”\index{dhvani@\textsl{dhvani}} as “implicature”. In the endnotes, he informs the reader that the term “implicature” is borrowed from H.P. Grice\index{Grice, H.P.} and that it “seems to [him] both to capture at least the linguistic (if not the aesthetic) phenomenon Ānanda [i.e. Ānandavardhana]\index{Anandavardhana@Ānandavardhana} sought to describe, and to provide a neologism comparable to Ānanda’s innovative use of \textsl{dhvani} (literally, “sound”)” (Pollock 2016). This explanation does not seem to be enough of a justification for translating the term \textsl{dhvani}\index{dhvani@\textsl{dhvani}} in the first place, keeping in mind the pivotal role played by the term \textsl{dhvani} in Ānandavardhana’s\index{Anandavardhana@Ānandavardhana} intervention in \textsl{rasa-śāstra}.\index{rasa@\textsl{rasa}} At best, Pollock’s fascination for imitating Ānandavardhana’s\index{Anandavardhana@Ānandavardhana} neologism and innovation in introducing conceptual terminology becomes apparent in his approach of translation. In that desperate attempt at imitating the great \textsl{rasika}\index{rasika@\textsl{rasika}} from Kashmir, Pollock seems to have forgotten to provide a strong basis for his comparison of Ānandavardhana’s key term \textsl{dhvani} with “implicature” beyond mere innovation and neologism (at this juncture, it will help if we remember that in Ānandavardhana’s theorization the term \textsl{dhvani}\index{dhvani@\textsl{dhvani}} is closely related to another key term from the \textsl{Dhvanyāloka},\index{Dhvanyaloka@\textsl{Dhvanyāloka}} which is \textsl{vyaṅgya}.\index{vyangya@\textsl{vyaṅgya}} Pollock uses “manifested” to translate \textsl{vyaṅgya}, again sans adequate explanation or annotation. 

Also, no mention is made of the definition of \textsl{vyaṅgya} in Viśvanātha\index{Visvanatha@Viśvanātha Kaviraja} Kaviraja’s \textsl{Sāhitya Darpaṇa}:\index{Sahityadarpana@\textsl{Sāhitya Darpaṇa}} “\textsl{vyaṅgyo} [\textsl{arthaḥ}] \textsl{vyañjanayā bodhyaḥ}”\break [\textsl{Sāhitya Darpaṇa} 2.3]. As a result, in ‘Pollockian’ translation, Ānandavardhana’s treatise \textsl{Dhvanyāloka}\index{Dhvanyaloka@\textsl{Dhvanyāloka}} becomes “\textsl{Light on Implicature}”. In doing this, Pollock plainly ignores the other title that the work had acquired viz. \textsl{Sahṛdayāloka}, owing to Ānandavardhana’s assertion that “a true poet and an ideal critic [or \textsl{sahṛdaya},\index{sahrdaya@\textsl{sahṛdaya}} or \textsl{rasika}]\index{rasika@\textsl{rasika}} share in common the gift of imagination (\textsl{pratibhā})”. (Krishnamoorthy 1983:34) He merely mentions the fact that the \textsl{Dhvanyāloka} was also known as \textsl{Sahṛdayāloka}, which he again translates as “Light for the Lover of Literature”. Then he goes on to adopt the title “Light on Implicature”, which makes the parallelism between \textsl{dhvani} --- the essence and marker of true poetry and \textsl{sahṛdaya}\index{sahrdaya@\textsl{sahṛdaya}} --- the ideal critic even further removed from its original context. He also translates “\textsl{vyaṅgya}”\index{vyangya@\textsl{vyaṅgya}} variously as “revealed”, “implied”, “suggested” and “manifested”; hardly ever alluding to the fact that “\textsl{vyaṅgya}” or “\textsl{vyañjaka}”\index{vyanjaka@\textsl{vyañjaka}} are key terminologies in understanding Ānandavardhana’s\index{Anandavardhana@Ānandavardhana} conceptual framework.

Therefore it becomes apparent that the term ‘\textsl{vyaṅgya}’\index{vyangya@\textsl{vyaṅgya}} connotes “revealed” or “manifested” and the other two translations that Pollock suggests (“implied” and “suggested”) are not satisfactory. More such cases where his translations can lead to confusion exist in the book; e.g. Pollock translates \textsl{Kāvyaprakāśa}\index{Kavyaprakasa@\textsl{Kāvyaprakāśa}} (the title of Mammaṭa’s seminal text) as Light on Poetry --- and Ānandavardhana’s\index{Anandavardhana@Ānandavardhana} \textsl{Dhvanyāloka}\index{Dhvanyaloka@\textsl{Dhvanyāloka}} as “Light on Implicature”. Such translations are not effective.

In the case of \textsl{A Rasa Reader},\index{A Rasa Reader@\textsl{A Rasa Reader}} he selectively provides footnotes/endnotes annotating the English translations of these words; but in most cases he simply uses the English replacement which he thinks suitable within quotation marks. Neither does he introduce the Sanskrit technical terms, which he translates into English, within parentheses beside his translations, as have other translators such as Manomohan Ghosh\index{Ghosh, Manomohan} (translator of The \textsl{Nāṭya-śāstra}).\index{Natyasastra@\textsl{Nāṭya-śāstra}} Ghosh provides a translator’s note in his introduction to the text, where he explains how he has used curved brackets to put the original Sanskrit technical term with his translations, repeatedly in certain cases. He also admits the untranslatability factor of certain technical terms and the two different approaches he adopts to address this issue: firstly, giving the terms in Romanised form with initial capital letters (such as \textsl{Vīthī}); and secondly, the closest English approximation of words with initial capital letters “lest these should be taken in their usual English sense” (Ghosh\index{Ghosh, Manomohan} 1995:32). 

This should provide a sound framework for explaining his approach towards translation, especially the one that is adopted in \textsl{A Rasa Reader}.\index{A Rasa Reader@\textsl{A Rasa Reader}} No such explanation is present in Pollock’s case. In \textsl{A Rasa Reader}, Pollock hardly uses any Sanskrit word (i.e.~in the main text, though not the endnotes) which is already not a part of the English lexicon. Such an approach does not hold itself accountable to either the source text or the target text and their readers; it does not feel any obligation whatsoever to provide explanations on the methodology or strategies employed in translating --- and thus interpreting --- texts which were composed in a world that differs linguistically and culturally from the translator’s. The lack of a translator’s note, which should precede any serious work of scholarship that resorts to translation heavily, speaks poorly of the author and his care for the texts which he is translating. It also amounts to being dishonest to the readership since they are kept completely in the dark so far as the different worldview of the source text is concerned. This is especially the case for a book like his \textsl{A Rasa Reader},\index{A Rasa Reader@\textsl{A Rasa Reader}} since according to its Preface, the book is intended both for general readers and students. The author of a book that has such lofty promises of providing general readers, students, comparitivists and specialist scholars an idea of the “intellectual history of \textsl{Rasa}”\index{rasa@\textsl{rasa}} through translations of and commentaries on original Sanskrit works, cannot simply excuse himself of supplying an explanation of his methodology.

Hence the book, like Pollock’s other works on \textsl{rasa},\index{rasa@\textsl{rasa}} turns out to be only one interpretation without a sufficient framework that helps the reader. Translating Sanskrit texts and its key technical-conceptual\index{misinterpretation!techniques of!obliteration via translation} terms thus has helped him divorce \textsl{kāvya}\index{kavya@\textsl{kāvya}} from \textsl{śāstra} by disregarding the spiritual aspects of \textsl{kāvya} in a wholesale manner (so they might be read using the same theoretical and critical tools which are used to read European and North American secular literature). This agenda viz.~‘secularizing’ Indian \textsl{sāhitya},\index{sahitya@\textsl{sāhitya}} of Pollock and other Neo-Orientalists\index{Neo-Orientalist} has been discussed in detail by Rajiv Malhotra\index{Malhotra, Rajiv} in \textsl{The Battle for Sanskrit}\index{Battle for Sanskrit @\textsl{Battle for Sanskrit}} under the chapter “Politicizing Indian Literature”. (Malhotra 2016) 

One may raise the question that even the national academy of letters in India, the Sahitya Akademi, centres its activities around written literature and uses the term \textsl{sāhitya}\index{sahitya@\textsl{sāhitya}} to denote written works of literature exclusively. In reply to this, it must be underlined that the semantic shift of the term \textsl{sāhitya} is a ‘post’-colonial phenomenon. I have consciously avoided the word postcolonial and supplanted the same with ‘post’-colonial, by which I wish to imply the period when direct contact between Britain and the Indian subcontinent had already began by virtue of the arrival of British merchants in the coast of Gujarat in early seventeenth century and onwards. This is juxtaposed with the clichéd ‘postcolonial’ which usually denotes the period after the colonizers had left the colony i.e.~when the colony had attained its independence (to which end, ``post-independence'' is a more suitable term). The Sanskrit term \textsl{sāhitya} signifies association, fellowship, society, togetherness, comradeship variously or all of these at the same time. Several Modern Indo-Aryan languages such as Assamese, Bengali, Bhojpuri, Hindi, Marathi, Nepali, Oriya, Rajasthani and some Dravidian languages like Malayalam uses the same word to denote both oral and written literatures in the present context (some might disagree with my use of the adjective ‘oral’ before ‘literature’ and might want to use the term ‘orature’ instead).\\[-21pt]

\section*{The \textsl{Saṅgīta-dāmodara} and the \textsl{Nāṭya-śāstra}}
\index{Natyasastra@\textsl{Nāṭya-śāstra}}

Now let us turn to what the \textsl{Saṅgīta-dāmodara}\index{Sangitadamodara@\textsl{Saṅgīta-dāmodara}} of Śubhaṅkara\index{Subhankara@Śubhaṅkara} has to say on topics related to \textsl{rasa}.\index{rasa@\textsl{rasa}} (Śubhaṅkara has been placed, by various accounts, in early sixteenth century C.E. (by Professor Dinesh Chandra Bhattacharya\index{Bhattacharya, Dinesh Chandra} who discovered and published Śubhaṅkara’s \textsl{kulapañjī}). Others have located him in as early as mid-thirteenth century C.E.; but nobody has placed him either before thirteenth century or later than sixteenth century C.E.). Śubhaṅkara\index{Subhankara@Śubhaṅkara} mentions the later additions to the types of \textsl{rasa-}s --- the \textsl{prema rasa} and the \textsl{vātsalya rasa} --- and he definitely mentions the \textsl{śānta rasa}\index{rasa@\textsl{rasa}}. In the fifth and the last chapter (or \textsl{stavaka}, as the text refers to them) of the \textsl{Saṅgītadāmodara},\index{Sangitadamodara@\textsl{Saṅgītadāmodara}} Śubhaṅkara\index{Subhankara@Śubhaṅkara} gives us a definition of \textsl{nāṭya}\index{natya@\textsl{nāṭya}} as follows:
\begin{quote}
\textsl{rasa-bhāva-samutpannāvasthānukaraṇam tu yat} |\\
\textsl{layamāna-samārabdhaṁ tan nāṭyam iti kīrtitam} || 

\hfill (\textsl{Saṅgīta-dāmodara}\index{Sangitadamodara@\textsl{Saṅgīta-dāmodara}} 5.12) 
\end{quote}

To translate: “that which is constructed by the measured \textsl{laya} and comprising the imitation of the states issuing from \textsl{rasa} and \textsl{bhāva}.”\index{bhava@\textsl{bhāva}} Here the emphasis is on the states that arise from (the experience of) \textsl{rasa}\index{rasa@\textsl{rasa}} and \textsl{bhāva}. One can hardly exclude “literature to be heard” (to borrow Pollock’s own terms) from the states that arise out of (the perusal of) \textsl{rasa} and \textsl{bhāva}\index{bhava@\textsl{bhāva}} and the factors that cause such circumstances in which those states can arise. 

Neither in this definition nor in the \textsl{ślokas} that follow has the distinction between the \textsl{dṛśya}\index{drsya@\textsl{dṛśya}} and \textsl{śravya}\index{sravya@\textsl{śravya}} aspects of \textsl{rasa} been made. Nor is there any mention of the duality of the experience of \textsl{rasa}. The experience of \textsl{rasa},\index{rasa@\textsl{rasa}} which this text equates to the manifestation of \textsl{sthāyī-bhāva} (by referring back to Bharata), is held by Śubhaṅkara\index{Subhankara@Śubhaṅkara} to be undifferentiated, unadulterated and almost equivalent to the \textsl{āsvādana}\index{asvadana@\textsl{āsvādana}} or the taste, the experience of \textsl{Brahma-jñāna} (or the Supreme Knowledge of the Self). The frequent references made to Bharata and the \textsl{Nāṭya-śāstra}\index{Natyasastra@\textsl{Nāṭya-śāstra}} also speak volumes about a continuity, rather than breaks and shifts, between the third and the thirteenth/sixteenth centuries C.E.. This is particularly significant in the light of the presence of arguments in the text in favour of the \textsl{Nāṭya-śāstra}’s dictums. Sometimes Śubhaṅkara\index{Subhankara@Śubhaṅkara} negates his own logic (as well as those of others) by drawing upon the definitions and prescriptions given by the \textsl{Nāṭya-śāstra}.       

In Ch.~22 (\textsl{vṛttivikalpa}) of the \textsl{Nāṭya-śāstra} the various \textsl{vṛtti-s} and their origins are explained. The opening \textsl{śloka} of this chapter goes like this:
\begin{quote}
\textsl{samutthānaṁ tu vṛttīnāṁ vyākhyāsyāmy anupūrvaśah} |  \\
\textsl{yathā vastūdbhavaṁ caiva kāvyānāṁ ca vikalpanam} ||

\hfill(\textsl{Nāṭya-śāstra} 22.1)
\end{quote}

Significantly, the \textsl{Nāṭya-śāstra} keeps it only as \textsl{kāvya},\index{kavya@\textsl{kāvya}} and does not specify whether it is talking about the \textsl{dṛśya-kāvya}\index{drsyakavya@\textsl{dṛśya-kāvya}} or the \textsl{śravya-kāvya}.\index{sravyakavya@\textsl{śravya-kāvya}} One can see similar usage of terminologies in the \textsl{śloka-}s from the previous chapter as well; such as:
\begin{quote}
\textsl{cekrīḍitādyaih śabdais tu kāvya-bandhā bhavanti ye} |\\
\textsl{veśyā iva na śobhante kamanḍalu-dharair-dvijaih} ||

\hfill(\textsl{Nāṭya-śāstra} 21.128)
\end{quote}

Even though the larger discourse is on the dramatic arts in general, Bharata tells the sages with regard to the origin of the concept of \textsl{vṛtti} that it was Bhagavān Brahmā\index{Brahma@Brahmā} who was the first to conceive the idea of the four \textsl{vṛtti-}s --- the first of which comes forth from the \textsl{vākya}, the spoken word or sentence. The corresponding \textsl{śloka} goes thus:
\begin{quote}
\textsl{kim idaṁ bhāratī vṛttir vāgbhir eva pravartate} |  \\
\textsl{uttarottara-saṁvṛddhā nanv imau nidhanaṁ naya} || 

\hfill(\textsl{Nāṭya-śāstra}\index{Natyasastra@\textsl{Nāṭya-śāstra}} 22.7)
\end{quote}

Here Bhagavān Brahmā is expounding the concept of \textsl{Bhāratī vṛtti},\index{bharati@\textsl{Bhāratī}}\index{vrtti-s@\textsl{vṛtti}-s!Bharati@\textsl{Bhāratī}} the first of the four \textsl{vṛtti}-s. The \textsl{Nāṭya-śāstra} narrates that during the battle between Bhagavān Viṣṇu\index{Visnu@Viṣṇu} and the twin \textsl{asura-}s Madhu and Kaiṭabha, the \textsl{asuras} had hurled insults at the Great God. Hearing such verbal insults and offensives, Brahmā\index{Brahma@Brahmā} asked the Lord if this is what is known as the \textsl{Bhāratī vṛtti},\index{bharati@\textsl{Bhāratī}}\index{vrtti-s@\textsl{vṛtti}-s!Bharati@\textsl{Bhāratī}} that which comes forth from the spoken words and thrives therefrom. Brahmā\index{Brahma@Brahmā} then implored the Lord to kill the \textsl{asura-}s. The Great Madhu-sūdana replied in the following manner:
\begin{quote}
\textsl{pitāmahavacah śrutvā provāca madhusūdanah} |\\
\textsl{bāḍhaṁ kārya-kriyā-hetorṭhā bhāratīyam vinirmitā} ||\\
\textsl{bhāṣato vākya-bhūyiṣṭhā bhāratīyaṁ bhaviṣyati} |\\
\textsl{aham etau nihanmy adya ity uktvā vacanaṁ harih} ||\\
\textsl{śuddhairavikṛtairaṅgaih sāṁgahāraistadā bhṛśam} |\\
\textsl{yodhayāmāsa tau daityau yuddha-mārga-viśāradau} ||\\
\textsl{bhūmi-saṁsthāna-saṁyogaih pada-nyāsais tadā hareh} |\\
\textsl{atibhāro’bhavad bhūmer bhāratī tatra nirmitā} ||

\hfill (\textsl{Nāṭya-śāstra}\index{Natyasastra@\textsl{Nāṭya-śāstra}} 22.8-11)
\end{quote}

Hearing what the Grandsire had to say, Madhusūdana replied --- “Yes, this \textsl{vṛtti} known as \textsl{Bhāratī}\index{bharati@\textsl{Bhāratī}}\index{vrtti-s@\textsl{vṛtti}-s!Bharati@\textsl{Bhāratī}} has been created for the purpose of the fulfillment of the work. This \textsl{vṛtti}, coming forth from the speeches of these two, will henceforth be known as \textsl{Bhāratī},\index{bharati@\textsl{Bhāratī}}\index{vrtti-s@\textsl{vṛtti}-s!Bharati@\textsl{Bhāratī}} in which speech shall be preeminent. I shall kill these today”. Speaking thus, Hari fought the two \textsl{asura-}s, experts in battle, with pristine and perfect gestures and \textsl{aṅgahāra-}s.\index{angahara@\textsl{aṅgahāra}} At that time the earth was laden with a great burden caused by the pacing of Hari on the ground and the \textsl{Bhāratī vṛtti}\index{bharati@\textsl{Bhāratī}}\index{vrtti-s@\textsl{vṛtti}-s!Bharati@\textsl{Bhāratī}} was created there. 

These four \textsl{śloka-}s tell us something important about the role of speeches, words and sentences in the dramatic arts. The \textsl{śloka-}s speak of the preeminence of the spoken word in the drama, which is evident from what the creator (Hari) of the \textsl{vṛtti-}s (sometimes translated into English as style) related to Brahma during the course of their conversation. Apart from that, the creation tale around the origins of the four \textsl{vṛtti-}s itself hints at the inseparable nature of the \textsl{dṛśya-kāvya}\index{drsyakavya@\textsl{dṛśyakāvya}} and the \textsl{śravya-kāvya}\index{sravyakavya@\textsl{śravyakāvya}} as it includes scenes from a battle (that which took place between Bhagavān Viṣṇu\index{Visnu@Viṣṇu} and the twin \textsl{asura-}s); a battle in which Hari fought with “with pristine and perfect gestures and \textsl{aṅgahāras}”.\index{angahara@\textsl{aṅgahāra}} \textsl{Aṅga-hāra-}s\index{angahara@\textsl{aṅgahāra}} are defined by the \textsl{Nāṭya-śāstra}\index{Natyasastra@\textsl{Nāṭya-śāstra}} in its fourth chapter (“Tāṇḍava-Lakṣaṇa”) in the \textsl{śloka-}s 30-34 to be a systematic arrangement of rhythmic movements of the limbs for the depiction of various meaningful gestures and situations in a dramatic performance. It is hard to believe that the concurrence of the spoken word (for the ears of the audience) and the necessity to maintain purity and perfection of gestures and \textsl{aṅgahāra-}s\index{angahara@\textsl{aṅgahāra}} (which are for their eyes) is coincidental. On the contrary, it must be an indication to the simultaneity and indivisibility of the production and enjoyment of aesthetic pleasure at the time of a performance. Even if the performance is dominated by purely verbal art that mainly appeals to the ears, the spectacular aspect cannot be divorced entirely from it. For, it is a common knowledge that even during the recitals of the \textsl{Veda}-s, \textsl{mudra}-s (gestures of the hands and fingers) play an important role, and the tradition has maintained special provision for the imparting of training in this art through the \textsl{Vedāṅga}-s. Even the \textsl{Nāṭya-śāstra}\index{Natyasastra@\textsl{Nāṭya-śāstra}} does not forget to link the origins of the \textsl{vṛtti}-s with the \textsl{Veda} recitals: 
\begin{quote}
\textsl{ṛgvedād bhāratī vṛttir yajurvedāt tu sāttvatī} |    \\
\textsl{kaiśikī sāmavedāc ca śeṣā cātharvaņāt tathā} ||

\hfill(\textsl{Nāṭya-śāstra} 22.24)
\end{quote}

This can be roughly translated into English as: “From the \textsl{Ṛgveda}\index{Rgveda@\textsl{Ṛgveda}}\index{Veda-s@\textsl{Veda}-s!Rg@\textsl{Ṛg}} comes the \textsl{Bhāratī vṛtti},\index{bharati@\textsl{Bhāratī}}\index{vrtti-s@\textsl{vṛtti}-s!Bharati@\textsl{Bhāratī}} from the \textsl{Yajurveda} the \textsl{Sāttvatī},\index{sattvati@\textsl{Sāttvatī}}\index{vrtti-s@\textsl{vṛtti}-s!Sattvati@\textsl{Sāttvatī}} from the \textsl{Sāmaveda}\index{Samaveda@\textsl{Sāmaveda}}\index{Veda-s@\textsl{Veda}-s!Sama@\textsl{Sāma}} the \textsl{Kaiśikī}\index{kaisiki@\textsl{Kaiśikī}}\index{vrtti-s@\textsl{vṛtti}-s!Kaisiki@\textsl{Kaiśikī}} and from the \textsl{Atharvan} comes the remaining one (\textsl{Ārabhaṭī})”.\index{arabhati@\textsl{Ārabhaṭī}}\index{vrtti-s@\textsl{vṛtti}-s!Arabhati@\textsl{Ārabhaṭī}} 

It is noteworthy that the \textsl{Ṛgveda}, which is hailed as the source of the \textsl{Bharati vṛtti}, is a \textsl{Veda} that is a compilation of mantras \textsl{sans} tunes --- they are meant to be recited aloud with the help of the \textsl{udātta,\index{udatta@\textsl{udātta}} anudātta}\index{anudatta@\textsl{anudātta}} and \textsl{svarita\index{svarita@\textsl{svarita}} svara-}s --- unlike the \textsl{Sāmaveda},\index{Samaveda@\textsl{Sāmaveda}}\index{Veda-s@\textsl{Veda}-s!Rg@\textsl{Ṛg}} which consists of, in the most part, mantras from the \textsl{Ṛgveda} but set into tunes. This implies that the spoken word is of paramount importance and can be recognized as a characteristic marker of the \textsl{Ṛgveda}. By associating the \textsl{Bhāratī vṛtti}\index{bharati@\textsl{Bhāratī}}\index{vrtti-s@\textsl{vṛtti}-s!Bharati@\textsl{Bhāratī}} with the \textsl{Ṛgveda}\index{Rgveda@\textsl{Ṛgveda}}\index{Veda-s@\textsl{Veda}-s!Rg@\textsl{Ṛg}}  the \textsl{Nāṭya-śāstra} clarifies its position on the equivalence drawn between the Seen and the Heard. This recurs throughout \textsl{Nāṭya-śāstra}\index{Natyasastra@\textsl{Nāṭya-śāstra}}--- to emphasize the \textsl{śravya}\index{sravya@\textsl{śravya}} aspect of performance while offering didactic discourses on the \textsl{dṛśya}\index{drsya@\textsl{ḍrśya}} aspect. No artificial distinction between the Seen and the Heard aspects of \textsl{rasa}\index{rasa@\textsl{rasa}} is therefore encouraged.

The notion of \textsl{vṛtti} is closely intertwined with the idea of application in the dramatic arts. It can be said, with some confidence, that the knowledge of \textsl{vṛtti} is imparted by the \textsl{Nāṭya-śāstra} in order to draw the trainee/director/composer’s attention to the practical aspects of dramaturgy. For example, the first of the \textsl{vṛtti-}s --- the \textsl{Bharatī vṛtti} --- relates to the compositional and enunciation techniques of a play (or any other type of composition or performance). The name \textsl{Bhāratī}\index{bharati@\textsl{Bhāratī}}\index{vrtti-s@\textsl{vṛtti}-s!Bharati@\textsl{Bhāratī}} itself speaks of the creation tale associated with this \textsl{vṛtti} and reminds us of the “weight” or “\textsl{bhāra}” the tradition (through the text) attaches to the spoken word by alluding to the “\textsl{atibhāra}” or excessive weight of Bhagavān Viṣṇu’s\index{Visnu@Viṣṇu} steps. In a way, it also offers a prescription as to the nature of the effect that should desirably be produced by the spoken word or the speeches used by the performers. Such preeminence of ‘\textsl{vāc}’ or the spoken word is central to the understanding of \textsl{rasa-śāstra}, aesthetics. Kapila Vatsyayan\index{Vatsyayan, Kapila} puts it in the following manner: “[I]n the Indian context, when one speaks of drama, dance or music, one is alluding only to the dominant or fundamental principle of the ‘word’ movement or sound and is not referring to these arts in isolation or in mutual exclusiveness.” (Vatsyayan\index{Vatsyayan, Kapila} 2005:9). Failure of understanding the basic tenet on the scholar’s part points to a deliberate distortion.\\[-21pt]       

\section*{The Case of Explaining \textsl{Rāga-rasa} through Images}
\index{rasa@\textsl{rasa}}

A strong evidence against Pollock's approach comes from \textsl{Gāndharva-vidyā},\index{gandharvavidya@\textsl{gāndharva-vidyā}} or the science of music and dance. This ancient science and art had taken upon itself the complex and challenging task of translating \textsl{bhāva} into forms and contents into both \textsl{dṛśya}\index{drsya@\textsl{dṛśya}} and \textsl{śravya}\index{sravya@\textsl{śravya}} media. One must keep in mind that music (which is dominated by \textsl{śravya} elements) and dance (where \textsl{dṛśya} elements dominate) developed in India as arts complementary to each other, the complementarity being a defining feature of their common epistemology. 

In both theory and practice, Indian art music has always devoted a special place to the art of visualizing a \textsl{rāga} by the artiste in her mind’s eye. Artistic depiction of various \textsl{rāga}-s and \textsl{ragiṇīs}, according to the moods that they evoke, has been accomplished by visual artists. Such depictions are, of course, instructed by prescriptions of the \textsl{śāstras}. Such prescriptions, formulated in \textsl{ślokas}, are known as \textsl{dhyāna-śloka-}s.\index{dhyanasloka@\textsl{dhyāna-śloka}} Each \textsl{rāga}/\textsl{rāgiṇī} has its \textsl{adhiṣṭhātrī}\index{adhisthatri@\textsl{adhiṣṭhātrī}} (roughly translated as ‘tutelary’) god/goddess, and the \textsl{dhyāna-śloka}-s evoke the specific god/goddess by enumerating the mood, the \textsl{bhāva}.\index{bhava@\textsl{bhāva}} This in turn helps the performer conceive the particular \textsl{bhāva} of the \textsl{rāga}/ \textsl{rāgiṇī}.\index{raga/ragini@\textsl{rāga/rāgiṇī}} This is a translation process, wherein an idea gets translated into twofold material manifestations: a) into the \textsl{nāda} (sonic form) and b) the \textsl{deva-deha} (godly form).

Swami Prajñānānanda\index{Prajnanananda@Prajñānānanda} asserts that this theorization and its applications have been a long-standing tradition; only it has come to be codified rather recently in a treatise titled \textsl{Rāga-vibodha}\index{Ragavibodha@\textsl{Rāgavibodha}} by Somanātha\index{Somanatha@Somanātha} in 1609 C.E.. (Prajñānānanda 1996:13) In this text, Somanatha formulates the theory in the following manner:
\begin{quote}
\textsl{uktaṁ rūpam anekaṁ tattad-rāgasya nādamayam evam} |\\
\textsl{atha devatāmayam iha kramatah kathaye tadekaikām} ||\\
\textsl{susvara-varṇa-viśeṣaṁ rupaṁ rāgasya bodhakaṁ dvedhā} |\\
\textsl{nādātmaṁ devamayaṁ tat kramato’nekam ekañ ca} || 

\hfill (\textsl{Rāga-vibodha}\index{Ragavibodha@\textsl{Rāga-vibodha}} 5.168)
\end{quote}

Enough has been said about the sonic form of \textsl{rāga}-s so far; now the godly form of \textsl{rāga}-s will be expounded one by one. They are understood to be the forms (\textsl{rūpa}) of \hbox{\textsl{rāga}-s} which are illuminated by sweet tones and letters; and forms are twofold: sonic (\textsl{nādātma}) and godly (\textsl{devamaya}). 

The \textsl{nāda} form evidently represents the \textsl{śravya}\index{sravya@\textsl{śravya}} and the \textsl{deva} form, the visual aspect of the \textsl{rāga-rūpa}.\index{raga@\textsl{rāga}} Such coupling of the two aspects side by side within theoretical paradigms for the discourse on \textsl{rasa},\index{rasa@\textsl{rasa}} in almost every application of the \textsl{Rasa}\index{Rasa Theory@\textsl{Rasa} Theory} Theory (in the fields of drama, poetry, music, dance), weakens any basis for such hypothetical assumptions as “the Sanskrit tradition differentiated between the two types of literature, or better yet, that it drew an opposition indicating that analysis applicable in the one domain might not be automatically applicable in the other” (Pollock 2012:189) taken for granted in Pollock (2012).

\section*{Conclusion}

Even Pollock has acknowledged that the reconstruction of a single and linear historical narrative of Sanskrit literary tradition is hard to achieve in the light of the revisions and contributions of authors later in the day to a text that had already seen the inception of its literary life (Pollock 2016:16). If this is seen in the light that his understanding and approach to the discourse, like everybody else, has evolved over time and he has acknowledged the futility of attempting a linear historiographical approach for Sanskrit literary-aesthetic traditions; some important questions still remain to be answered. First among them is: what about the inseparability of aesthetic concepts with their metaphysical-philosophical counterparts in the Sanskrit tradition? Viśvanātha\index{Visvanatha@Viśvanātha Kaviraja} Kavirāja, Ānandavardhana, Abhinavagupta, Śubhaṅkara\index{Subhankara@Śubhaṅkara} --- almost all the stalwarts of the tradition who have been almost unanimously placed between ninth century C.E. to sixteenth century C.E. (and not later) by international scholarship have drawn our attention to the connection of the aesthetic with the metaphysical, the spiritual. 

Why, then, is Pollock steadfast on a strategy of translation that exercises maximum domestication into the Anglo-US universe of ideas, semantics, words and terminologies? Is he not aware of the dangers of such strategies with respect to cultural misrepresentation, negation of cultural differences and criticality of otherness? In continental sociology and literary criticism, adoption of such strategies for translation has already been brought to question and they have been severely criticized for suppression of knowledge systems other than the West’s own worldviews and historicism; some eminent sociologists such as Michel de Certeau,\index{de Certeau, Michel} whose works I have drawn upon in this essay, have even gone so far as to call such stratagems in the name of translation and interpretation as “violence” brought upon specific (non-Euro-American) epistemologies. 

It is up to those scholars living \textsl{in the tradition}, or in the words of Shri Rajiv Malhotra,\index{Malhotra, Rajiv} the “insiders” (Malhotra 2016), to ponder over these vital questions relating to the present, past and future of the academia in the field of Sanskrit studies and perhaps, to raise pertinent and pointed questions about the work of Neo-Orientalist\index{Neo-Orientalist} scholarship, if not to provide concrete answers --- and build the \textsl{Uttara Pakṣa} of this discourse which has historically been heavily skewed in the direction of the \textsl{Pūrva-pakṣin},\index{purvapaksa@\textsl{pūrvapakṣa}} i.e. the typical Western Sanskritist, who uses the postmodernist, deconstructivist, feminist, or psychoanalytic framework to read and interpret traditional Indian texts. Upon scrutinizing works such as Pollock (2012) one gets a feeling that rendering \textsl{śravya}\index{sravya@\textsl{śravya}} and \textsl{dṛśya}\index{drsya@\textsl{dṛśya}} as two inherently separate categories growing independently of each other, is characteristic of the Western mind which has always sought to make sense of the world by differentiating and analyzing its manifestations into separate constituents and putting them under distinct categories. 

This can no doubt be an effective approach to understand the products of the external world, and that is exactly what the Western epistemologies have been doing --- right from the age of Aristotle through Bernhard Varen and John Ray (‘scientists’ whose work formed the basis of ‘scientific’ racism) and Carl Linnaeus (father of taxonomy, the science of classification). How far such an approach can be rendered applicable to concepts such as \textsl{rasa},\index{rasa@\textsl{rasa}} which is intrinsically psychological, is a question that should be raised more and more in the face of works pouring out from Neo-Orientalist\index{Neo-Orientalist} scholarship. It is imperative for scholars of Indic studies, and especially those of Swadeshi Indology, to recognize and understand this legacy of selectively attributing the ‘scientific’ label to approaches which understand the world by means of differentiation and refuse to acknowledge such approaches which seek synthesis, harmony, coexistence and mutual dependence.
 
\begin{thebibliography}{99}
\itemsep=2pt
\bibitem[]{chap5_item1}
{\sl\bfseries Abhinaya-darpaṇa.} See Ghosh (1934).

Also See Coomaraswamy and Duggirala (1917).

\bibitem[]{chap5_item2}
Bandyopadhyaya, Haricharana. (1978). \textsl{Baṅgīya Śabdakoṣa}. Vol. 1 \& 2. New Delhi: Sahitya Akademi. 

\bibitem[]{chap5_item3}
Bandyopadhyay, Dhirendra Nath (2000). \textsl{Samṣkṛta Sāhityera Itihāsa}. Kolkata: Pasćimabaṅga Rajya Pustak Parshat. 

\bibitem[]{chap5_item4}
Banerji, Sures Chandra. and Chakrabarti, Chanda. (Ed.) (1980). \textsl{Bharata Nāṭyaśāstra}. Calcutta: Nabapatra Prakāśana.

\bibitem[]{chap5_item5}
Coomaraswamy, Ananda K. and Duggirala, Gopala Kristnayya (1917). \textsl{The Mirror of Gesture: Being the Abhinaya Darpana of Nandikésvara}. Cambridge: Harvard University Press. 

\bibitem[]{chap5_item6}
{\sl\bfseries Dhvanyāloka with Locana}. See Ingalls (1990).

Also See Krishnamoorthy (1988).

\bibitem[]{chap5_item7}
Ghosh, Manomohan. (Ed.) (Trans.) (1934). \textsl{Nandikeśvara's Abhinaya-darpaṇam: A Manual of Gestures and Postures Used in Hindu Dance and Drama.} Calcutta: Comp. Manomohan Ghosh. Metropolitan Prtg. \& House. 

\bibitem[]{chap5_item8}
Highmore, Ben (2006). \textsl{Michel De Certeau: Analysing Culture}. London: Continuum.

\bibitem[]{chap5_item9}
Ingalls, Daniel H. H. (Trans.) (1990). \textsl{The Dhvanyāloka of Ānandavardhana with the Locana of Abhinavagupta}. Cambridge, MA: Harvard UP.

\bibitem[]{chap5_item10}
{\sl\bfseries Kaṭhopaniṣad}. See Sarvananda (1956).

\bibitem[]{chap5_item11}
Kavi, M. Ramakrishna, and Pade, Jagannathasastri. (Ed.) (1954). \textsl{Nāṭyaśāstra with the commentary of Abhinavagupta.} Baroda: Oriental Institute. 

\bibitem[]{chap5_item12}
{\sl\bfseries Kāvyādarśa}. See Mishra (1972).

\bibitem[]{chap5_item13}
Krishnamoorthy, K. (1983). “Anandavardhana”. In {\sl Cultural Leaders of India : Aestheticians}. New Delhi: Publications Division, Govt. of India. pp.~34--40.

\bibitem[]{chap5_item14}
---\kern3pt(Ed.) (1988). \textsl{Abhinavagupta's Dhvanyaloka-locana, with an Anonymous Sanskrit Commentary}. New Delhi: Meharchand Lachhmandas Publications. 

\bibitem[]{chap5_item15}
{\sl\bfseries Locana} of Abhinavagupta (on {\sl\bfseries Dhvanyāloka} of Ānandavardhana). See Ingalls (1990). 

Also See Krishnamoorthy (1988).

\bibitem[]{chap5_item16}
Malhotra, Rajiv (2016). \textsl{The Battle for Sanskrit: Is Sanskrit Political or Sacred, Opressive or Liberating, Dead or Alive?} Noida: Harper Collins.

\bibitem[]{chap5_item17}
---\kern3pt(2011). \textsl{Being Different: An Indian Challenge to Western Universalism}. New Delhi: Harper Collins India. 

\bibitem[]{chap5_item18}
Mishra, Ramchandra. (Ed.) (1972). \textsl{Kavyādarśa}. Varanasi: Chowkhamba Vidyabhawan.

\bibitem[]{chap5_item19}
Mukhopadhyaya, Mahuya. (Trans.) (2009). \textsl{Saṅgīta Dāmodara of Śubhaṅkara.} Kolkata: Asiatic Society. 

\bibitem[]{chap5_item20}
{\sl\bfseries Nāṭya-śāstra}. See Ramakrsnakavi and Pade (1954).


Also See Banerji and Cakrabarti (1980).

\bibitem[]{chap5_item21}
Pollock, Sheldon I. (2016). \textsl{A Rasa Reader: Classical Indian Aesthetics}. New York: Columbia University Press. 

\bibitem[]{chap5_item22}
---\kern3pt(2012). "From Rasa Seen to Rasa Heard." \textsl{Aux Abords De La Clairière}. pp.~187--207. 

\bibitem[]{chap5_item23}
Prajnanananda, Swami. (1961). \textsl{A History of Indian Music}. 2nd ed. Vol.~2. Calcutta: Ramakrishna Vedanta Math. 

\bibitem[]{chap5_item24}
---\kern3pt(1996). \textsl{Rāga O Rūp}. 6th ed. Vol.~1. Calcutta: Ramakrishna Vedanta Math. 

\bibitem[]{chap5_item25}
Prasad, Gupteshwar. (1994). \textsl{I.A. Richards and Indian Theory of Rasa}. New Delhi: Sarup \& Sons.

\bibitem[]{chap5_item26}
Ray, Kumudranjan. (Trans.) (1957). \textsl{Viśvanātha's Sāhitya-Darpaṇa: With English Translations and an Original Sanskrit Commentary}. Calcutta: K. Ray Publications. 

\bibitem[]{chap5_item27}
{\sl\bfseries Sāhitya Darpaṇa}. See Ray (1957).

\bibitem[]{chap5_item28}
{\sl\bfseries Saṅgīta Dāmodara}. See Mukhopadhyaya (2009).

\bibitem[]{chap5_item29}
Sarvananda, Swami. (Ed.) (Trans.) (1956). \textsl{Kaṭhopaniṣad}. Madras: Sri Ramakrishna Math.

\bibitem[]{chap5_item30}
Sen, Sukumar. (1991). \textsl{Bāṃlā Sahityer Itihās}. Vol.~1. Calcutta: Ānanda. 

\bibitem[]{chap5_item31}
Sharvananda, Swami. (Ed.) (Trans.) (1921). \textsl{Taittirīya-Upaniṣad.} Madras: Ramakrishna Mission.

\bibitem[]{chap5_item32}
Sinha, Jadunath (1986). \textsl{Indian psychology}. Vol.~2, New Delhi: Motilal Banarsidass Publishers.

\bibitem[]{chap5_item33}
“Soul, Nature and God”. Complete Works of Swami Vivekananda. Volume 2. \url{https://en.wikisource.org/wiki/The_Complete_Works_of_Swami_Vivekananda/Volume_2/Practical_Vedanta_and_other_lectures/Soul,_Nature_and_God}. Accessed on 10 January, 2017. 

\bibitem[]{chap5_item34}
{\sl\bfseries Taittiriyopaniṣad}. See Sharvananda (1921).

\bibitem[]{chap5_item35}
Vatsyayan, Kapila. (1996). \textsl{Bharata: The Nāṭyaśāstra}. Sahitya Akademi. New Delhi. 

\bibitem[]{chap5_item36}
Vázquez, Rolando. (2011). "Translation as Erasure: Thoughts on Modernity's Epistemic Violence." \textsl{Journal of Historical Sociology}. 24. pp.~27--44.

\bibitem[]{chap5_item37}
Witzel, Michael. (First version on 01 January, 2001 ) "Autochthonous Aryans? The Evidence from Old Indian and Iranian Texts." \url{http://www.people.fas.harvard.edu/~witzel/EJVS-7-3.pdf/} Accessed on 10 January, 2017. 
\end{thebibliography}
