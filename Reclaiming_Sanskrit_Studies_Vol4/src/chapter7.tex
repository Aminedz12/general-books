\chapter[{\dev रसब्रह्मसमर्थनम्}]{{\dev रसब्रह्मसमर्थनम्}}\label{chapter\thechapter:begin}
%~ \footnotetext[1]{pp.~\pageref{chapter\thechapter:begin}--\pageref{chapter\thechapter:end}. In: Kannan, K S (Ed.) (2018) {\sl Śāstra-s Through the Lens of Western Indology - A Response}. Chennai: Infinity Foundation India.}

\lhead[\small\thepage\quad {\dev शतावधानी डा॥ रा. गणेशः}]{}

\Authorline{{\dev शतावधानी डा॥ रा. गणेशः}}

%\showhyphens{helicopter}

\begin{center}
\begin{tabular}{r}
{\dev अपर्याप्तमुदाकारामामपर्यायतमास्मिताम्~।}\\
{\dev प्रपद्ये सकलास्वादां निष्कलां रसभारतीम् ॥}
\end{tabular}
\end{center}

\section*{{\dev पूर्वपीठिका}}

{\dev तदिदं `रसाध्याय'} ({\sl A Rasa Reader})\index{rasadhyaya@\textsl{rasādhyāya}} {\dev संज्ञितस्य कस्यचन ग्रन्थस्य\endnote{Editor: The reference is to Pollock (2016).} सारासारविवेचनार्थं समायोजनम्~। तथापि नास्माभिरस्यैकस्य मीमांसने मनः प्रवर्ततेतराम्~। यत ईदृशा आक्षेप\-विक्षेपा बहोः कालादारभ्य वर्तमाना एव विद्वज्जगति~। नेदमपूर्वं किञ्चिन्नाप्यभिनवम्~। अन्यच्च नात्र काप्यस्मदीयानस्मदीयविभेदव्यपदेशोऽपि लगति~। यतः पुरा हि रसस्यानुमानिकतां लोलुपतां सुखदुःखात्मकतां च कथयद्भिः कथाकृद्भिः स्वकीयं मूलच्छेदं पाण्डित्यमस्मिन्नेव भारते देशे प्रादर्शि नैकधा~। अतोऽत्र न कापि स्वजनपक्षपातपातित्यमस्मासु लिप्यते~। अपि च वैदेशिकैः कैश्चित्प्राप्तसंस्कृतविद्यागन्धैर्गन्धिलैः पौनःपुन्येन भारतीयार्षसंस्कृतेर्जीवातुभूत\-मानन्दपारम्यं निजाग्रहैरग्रहणैर्वा मुह्यद्भिर्निराकर्तुं कृता कूटपरम्परा~। तत्रभवद्भिः प्रेक्षा\-वित्कूटस्थैः प्राचार्यहिरियण्णमहोदयैः\index{rasa@\textsl{rasa}} पूर्वमेव सङ्क्षेपसुन्दरं समर्थं च समर्थनं व्यधायि रसपारम्यस्य~। तदत्रास्माभिरनुसंधेयमिति परिकरश्लोकः ---}\index{Hiriyanna, M.}
\begin{quote}
{\dev पारदृश्वा सारदृश्वा हिरियण्णमहोदयः~।}\index{rasa@\textsl{rasa}} \\
{\dev चक्रे यच्छास्त्रसङ्क्षेपं तदस्मद्ध्यानमर्हति ॥} (Hiriyanna\index{Hiriyanna, M.} 1954) 
\end{quote}

{\dev एवमेव वार्तमानिकेषु विद्वत्सु नैकतमेन पादेकल्लुनरसिंहभट्टवर्येण भारतीयसंवेदनं (Bhat\index{Narasimha Bhatta, Padekallu} 2013) कथयता कश्चनार्षो नयः प्रकाशितः~। ईदृशाः पूर्वसूरिभिरानन्दकुमारस्वामि-वासुदेवशरणाग्रवाल-गुण्डप्पप्रमुखैः\index{Coomaraswamy, A. K.} प्रपञ्चित एव~। तथाप्यलङ्कारशास्त्रदृशा प्रत्येकतया\index{Agrawala, V. S.} नरसिंहभट्टमहोदयेन\index{Gundappa, D. V.} (Bhat 2016) किमप्यनुत्तमं कल्पितं किल तदनुसारमेवं वक्तुमलमिति सङ्ग्रहः ---}\index{Narasimha Bhatta, Padekallu}
\begin{quote}
{\dev भट्टो नृसिंहनामा च यत्संवेदनमाह तत्~।}\\
{\dev पुनरत्रानुसंधेयं यतस्तत्कृत्स्नदर्शनम् ॥} (Bhat 2003)\\ 
{\dev शास्त्रजातमिदं सर्वं जीवत्सामग्र्यसत्त्ववत्~।}\\
{\dev साक्षात्कृत्य सुधीः पश्येत्समष्टिं नैव विस्मरेत् ॥}  
\end{quote}
\begin{quote}
{\dev व्यष्टिवैचित्र्यसर्वस्वं स्वानुभूतौ विलीयते~।}\index{samasti@\textsl{samaṣṭi}}\\
{\dev वृक्षवैविध्यमुग्धानां कथं कान्तारदर्शनम् ॥}
\end{quote}

{\dev सम्प्रति पूर्वपीठिकारूपेण भारतीयसौन्दर्यशास्त्रसंबद्धं किञ्चिदत्र प्रस्तूयते~। सर्वमिदं कारिका-वृत्तिमाध्यमेन निरूप्यत इति विदितवेदितव्यानामपरोक्षमेव~।}\index{vyasti@\textsl{vyaṣṭi}}
\begin{quote}
{\dev नानुमेयो परोक्षो न नाप्तवाक्यैकवेदनः~।}\\
{\dev न प्रत्यक्षोऽपि लोकार्थे ह्यपरोक्षो रसः स्मृतः ॥}
\end{quote}

{\dev भौताः पदार्थाः प्रत्यक्षानुमानाभ्यां ज्ञायन्त इति निश्चितम्~। किञ्च भावानामभौतत्वं सुप्रतिष्ठित\-मिति हेतुना भौतन्यायोऽत्र न प्रसज्यते~। भावानामभौतत्वं तु तेषामभावायापि न कल्प्यते~। यतः\ स्वसंवेदनविरुद्धोऽयं प्रसङ्गः~। अतो हि नैते भावाः परोक्षाः~। तर्हि किमवशिष्यत इति विचिकित्स्यते चेदपरोक्षा एव भावा इति  निगमनम्~। केचनाधुनिकविज्ञानिन आहुर्यद्भावा अपि काश्चन मस्तिष्कमग्नविद्युद्रासायनिकप्रक्रिया इति~। स्यादिदं तथ्यं वाऽतथ्यम्~। अनेन नास्माकं मतं प्रतिहन्यते~। यतस्तथ्यमिदं न कदाप्यस्मद्भावेषु बुध्यते~। स्यान्नाम रति-हास-क्रोध-विस्मयोत्साहादिभावप्रपञ्चः काचिद्रासायनिकी क्रिया~। परमस्मासु तथ्येनानेन प्रीतिर्वा भीतिर्वा शान्तिर्वा दान्तिर्वा न जायते~। अतः स्वसंवेदनपरिधिबाह्या ये केऽपि वैज्ञानिकवास्तवाविष्कारा मानुषसंदर्भेषु न मनागपि परिणमन्तीत्यलम्~। तस्मात्तदिदमवधेयं यद्भौतादिविषयज्ञानेन वाऽज्ञानेन भावप्रपञ्चे न किञ्चिदुपचीयते न किञ्चिद्वाऽपचीयत इति परमार्थः~। प्रायेण लोके भव-भावयोरशास्त्रीयानुदारस्वच्छन्दसाङ्कर्येण धीमन्तोऽपि जनाः क्लेशमनुभवन्ति परेषु च क्लेशमनुभावयन्तीति सर्वथा शोचनीयम्~।}
\begin{quote}
{\dev भौतादिशास्त्रजातानां सृतिवद्रैखिका नहि~।}\\
{\dev भावशास्त्रस्य चास्यर्धिश्चाक्री संविच्चिरन्तनी ॥}
\end{quote}

{\dev सामान्यतो भौतादिशास्त्रविद्याविकासं निर्वर्ण्य साम्प्रतिकपण्डिता भावाश्रितानां कलानामपि जीवव्यापाराणामपि प्रगतिः काचिद्रेखात्मिकेति भ्रान्तिमुत्पादयन्ति~। किञ्च वस्तुस्थितिरन्यैव विद्यत इति प्रेक्षावतामपरोक्षम्~। रसमीमांसा तु सर्वथा भौतातीतसरणिमनुवर्तते~। अतोऽत्र शास्त्रविकासो न सर्वथा रेखात्मकः~। तस्मादेव मुनिना भरतेन प्रोक्ते नाट्यवेदे बीजभूता नैके महाविषयाः परवर्तिनामनेकेषां विदुषां स्वानुभूतिपुरस्सराङ्गीकारार्थं समभवन्~। परन्तु तत्र तत्र स्पष्टीकरणं वा विस्तरणं वा भङ्ग्यन्तरनिरूपणं वा समपेक्षितम्~। न कदाप्यस्माभिर्मुनेर्भरतस्य सरला सुकरा वाणी शास्त्रशैशवत्वाय स्वीकार्या~। वेदोपनिषत्सु यत्तत्त्वं प्रगल्भमपि समर्थं,\index{Bharatamuni} प्रगाढमपि समग्रं सुलभवाग्लभ्यं;\index{Natyasastra@\textsl{Nāṭyaśāstra}} रामायण-महाभारतादिष्वार्षकाव्येषु\index{Ramayana@\textsl{Rāmāyaṇa}} मानुषभावभूमानुभव\-कल्पनं यद्भव्योदारमूहातीतमपि साहजिकं तत्तथैव नाट्यवेदेऽपि\index{Mahabharata@\textsl{Mahābhārata}} विज्ञेयम्~। अत्रैव महामतयोऽप्यस्मत्कालीना\index{Natyasastra@\textsl{Nāṭyaśāstra}} नैकशास्त्रेषु कृतभूरिपरिश्रमा व्यामोहिता इव दृश्यन्ते~। तदपसरणं नास्माभिरुपमातुमलम्~। केवलं तन्निरूपणमेव मेधामितम्पचानां मादृशां विनम्रो यत्न इति प्रणिपातपुरःसरं विज्ञाप्यते~।}
\begin{quote}
{\dev उदाहरणपद्यानां काव्यानां वा तथैव च~।}\\
{\dev रूपमात्रहतः कुर्यान्न तत्सिद्धान्तदूषणम् ॥}
\end{quote}
\begin{quote}
{\dev केवलं स्वर्णभूषाणां विचित्राकृतिविस्मितः~।}\\
{\dev न कोऽपि कुरुते विद्वांस्तदुपादानगर्हणम् ॥}
\end{quote}

{\dev अन्यच्च साम्प्रतिकाः साहित्यविमर्शनविदः प्रतिनिविष्टतया भारतीयकलासिद्धान्तेषु कौलीनदि\-दृक्षवः क्वचित्क्वचित्तत्तदालङ्कारिकदेशकालीनरुचिवैकट्यफलान्युदाहरणपद्यानि परिभावयन्त\-स्तद्भित्तिप्रायाणि तत्त्वान्यपि विसंवादीनीति प्रजल्पन्ति~। किञ्च न ते जानन्ति यत्प्रत्ने काञ्चन\-परिष्कारे सत्यपि रूपवैरूप्ये स्वरूपसौष्ठवं न हिनस्तीति~। अतः सर्वदास्माभिश्चिरन्तनानां सिद्धान्तमीमांसावसरे प्रतिभावद्भिर्भाव्यं ; येन न कदाचिदपि लक्ष्यदोषा लक्षणेषु नापतेयुः~। अग्निरन्नमपि पचति,\index{pratibha@\textsl{pratibhā}} पाचकमपि दहति~। न तेन कदाचिदपि दुष्यति~।}  
\begin{quote}
{\dev स्वसंवेदनमूलत्वादार्युवेदौषधं यथा~।}\\
{\dev रससिद्धान्ततत्त्वं च त्रिकालाबाधितं भवेत् ॥}
\end{quote}
\begin{quote}
{\dev केवलं मूलिकादीनां रासायनिकचिन्तनम्~।}\\
{\dev यदिष्यते तथैवात्र मनःशास्त्रीयशोधनम् ॥}
\end{quote}

{\dev आयुर्वेदे भारतीये यद्यप्यौषधानां रासायनिकी प्रक्रिया चातुराणां देहान्तस्तेषां व्यापारविधिर्न निश्चप्रचं प्रत्यपादि तच्छास्त्रविद्भिस्तथा च देहरचनामर्मसर्वस्वनिरूपणं नाकारि लोकोर्जितं तथापि प्रायेण निरपवाद इव चिकित्सा वर्वर्ति स्वास्थ्यं च जरीजागर्ति~। समासत इदमत्र वक्तव्यं यदायुर्वेदे भिषङ्मुखनिदानमेव विद्यमानमपि विनापि प्रयोगालयपरिकरव्यूहपरामर्श\-नेन भैषज्यं प्रसरत्येव;\index{Ayurveda@\textsl{Āyurveda}} गदाली निर्गलत्येव~।\index{Ayurveda@\textsl{Āyurveda}} अनेनैव न्यायेन रससिद्धान्तेऽपि विनाऽधुनिक\-मनः\-शास्त्रीयपरिशोधनं, विना तत्तद्देशकालीनसामाजिकार्थिकराजकीयसांस्कृतिकवैलक्षण्य\-समा\-कलनं केवलं मानुषास्तित्वस्य भावसामान्यस्य निस्सामान्यसार्वत्रिकानुभवपुरस्सरं यच्चिन्तनं विहितमनेनैव सर्वमपि जैविकचित्तवृत्तिविलसनं यावन्मात्रं विविधकलाविदां तत्प्रयोक्तॄणां च सौकर्याय पर्याप्तं तावन्निरूपितं निराकुलम्~। कुर्वन्तु नाम साम्प्रतिका अस्यां दिशि संशोधनं यन्मनःशास्त्रीयादिप्रतिनवतन्त्रज्ञानविज्ञानसौविध्यसाध्यम्~। यदि क्वचित्क्वचिद्रसतत्त्वमहा\-गारप्राकारसालभञ्जिकानां पादालक्तकलेपने प्रयोजकता स्यादनेन स्वागतार्हमेव सुतराम्~। परं निश्चयोऽयं यत्स महागारः सर्वथा सुदृढः सुस्थिरः सुचिरं स्थास्यत्येव परमानन्दपरमात्म\-गर्भितश्च~। यतोऽत्र दत्तस्यायुर्वेदीयदृष्टान्तस्येव न रसतत्त्वं भौतमात्रम्~। तदिदं भावपरमम्~। अतो हि नात्र कापि यातयामता कालबाह्यता वा क्लेशयत्यस्मान्~। केवलं तत्त्वस्पष्टीकर\-णार्थं स्थूलोऽप्यसावायुर्वेदीयदृष्टान्तो\index{Ayurveda@\textsl{Āyurveda}} दत्तः~। सत्यमेवानेन हीनोपमानदोषदूषिता स्यादत्र विचार\-सरणिः~। तथापि यथोपनिषत्सु वेदेषु वा तत्र तत्र हीनोपमानप्रायाणि भवन्ति वाक्यानि तथात्रा\-प्यन्वेयम्~। रसपर्यवसायिनि ध्वनिप्रतिपादनावसरे यथाऽऽनन्दवर्धनेन\index{dhvani@\textsl{dhvani}}\index{dhvani@\textsl{dhvani}!rasadhvani@\textsl{rasadhvani}}\index{rasadhvani@\textsl{rasadhvani}} तत्रभवता केवलं ध्वनि\-मात्रप्रतिष्ठापनार्थमवरोऽपि\index{Anandavardhana@Ānandavardhana} वस्तुध्वनिरादृतः स एव नयोऽत्र प्रादर्शीति स्वसमयमर्यादां स्मरन्तो वयं विरमामः~। सर्वमिदमस्मत्प्रबन्धेष्वन्यत्र सुविस्तरं निरूपितमिति वाग्विग्लापनेनालम्~।}\index{dhvani@\textsl{dhvani}}\index{dhvani@\textsl{dhvani}!vastudhvani@\textsl{vastudhvani}}\index{vastudhvani@vastudhvani@\textsl{vastudhvani}} (Ganesh 2013) 
\begin{quote}
{\dev सच्चिदानन्द-कृष्णाभ्यां वेदान्तत्वं यदीरितम्~।}\\
{\dev तदेव भरतप्रोक्ते रसतत्त्वे प्रतिष्ठितम् ॥}
\end{quote}

{\dev यथा वेदान्तशास्त्रस्य निर्विशेषसार्वत्रिकानुभवपारम्यं,\index{Bharatamuni} अध्यारोपापवादप्रक्रियानिरूपणं तत्रान्य\-तमस्यावस्थात्रयस्य मीमांसनं निस्सन्दिग्धेन विधिना स्वनामधन्येन विदुषा कृष्णस्वामिना च तच्छिष्यतल्लजेन श्रीसच्चिदानन्देन्द्रसरस्वतीस्वामिचरणेन व्यधायि तथैव चिरन्तनेन मुनिनापि तत्रभवता भरतेन महानयमौपनिषदिको नयः प्रगल्भतया समाकलितः स्वकीये नाट्याम्नाये~। नात्र काचिच्छ्रद्धाजडता, विश्वासाभासाभासता,\index{Bharatamuni} हेत्वाभासरूषिता दुस्तर्कसरणिः प्रवर्तिता~। केवलं सर्वापचेयप्रमाणैरेव सकलमपि समनुष्ठितम्~। तद्यथा ---} 
\begin{quote}
{\dev अध्यात्म-लोक-शास्त्राख्यप्रमाणत्रितयान्वितम्~।}\\
{\dev कलामीमांसनं कार्यं सम्प्रदायक्रमो ह्यसौ ॥}
\end{quote}

{\dev अत्राभिनवगुप्तपादैः स्पष्टीकृतमिदमेव स्वटीकायाम्~। रुय्यकेणापि न्यरूपि महिम्नो व्याख्यावसरे~। अस्मद्गुरुचरणैरवदातकीर्तिभिः\index{Abhinavagupta} कृष्णमूर्तिमनीषिभिः स्वप्रबन्धे सम्यगिदं प्रत्यबोधि}\index{Ruyyaka} (Krishnamoorthy\index{Krishnamoorthy, K.} 1982)  {\dev~। तत्सर्वमादृत्य कल्पयामश्चेदध्यात्मं नाम स्वसंवेदनं, लोकस्तु लोक्यमानो लोक एव यद्देशकालप्रतिष्ठितः, शास्त्रं तावदेतयोः समाहारस्वरूपं ज्ञान-विज्ञानसमन्वितं युक्तिपुरस्सरस्यानुभवस्य निश्चप्रचनिरूपणम्~।}   

{\dev पूर्वसूरीणां सरणिमेव समाकलय्य कलयामश्चेदियमत्रोपलब्धिः ---}
\begin{quote}
{\dev अवस्थात्रयनीत्यैव सच्चिदानन्दवेदनम्~।}\\
{\dev स्वात्मनस्तु यथैवास्ते रसस्यापि तथैव हि ॥}
\end{quote}
\begin{quote}
{\dev किञ्चानन्दो रसस्यास्ते स्फुरद्बाह्यकलाक्रमैः~।}\\
{\dev निमित्तहत एवायं ह्यतो ब्राह्मसहोदरः ॥}
\end{quote}
\begin{quote}
{\dev वक्रोक्तिचिन्तनं चापि श्रुत्यन्तनयसन्निभम्~।}\\
{\dev अध्यारोपापवादाभ्यां सुधीमद्भिः प्रवर्तते ॥}
\end{quote}
\begin{quote}
{\dev ध्वनिसर्वस्वविज्ञानं\index{vakrokti@\textsl{vakrokti}} पुनर्वेदान्तनीतिवत्~।}\\
{\dev नेति नेति क्रमेणैव रसपर्यवसायि हि ॥}
\end{quote}

{\dev सच्चिदानन्दघनस्य ब्रह्मणो यत्स्वरूपं तदेवात्मन इति निरुल्लङ्घ्यो वेदान्तसमयः~। किञ्च बाह्येन कलानिमित्तेन प्रत्यभिज्ञेयस्तावद्रसानन्दः सर्वथाऽवसितायां\index{dhvani@\textsl{dhvani}}\index{dhvani@\textsl{dhvani}!rasadhvani@\textsl{rasadhvani}}\index{rasadhvani@rasadhvani@\textsl{rasadhvani}} कलायां तिरोधत्ते~। अत एव निमित्तहतोऽयमानन्दः सर्वात्मना ब्रह्मास्वादसहोदरः~। तस्मादेव रसमीमांसावसरे वेदान्तिनामेव नयः समादरणीयः~। नो चेद्विचारसरणिरेव विपद्यते~। युक्तं हि सूक्ष्मं वस्तु विवेक्तुं सूक्ष्मतरं साधनं कल्प्यत इति~। वक्रोक्तिस्तावद्वक्रतापरपर्याया सकलासु कलास्वलङ्कृतिरूपा~। अलङ्कृतीनां सिद्धिस्तु सुतारामध्यारोपापवादयुक्त्यैवेति सङ्ख्यावतां विदितमेव~। यथा मुखचन्द्र इत्यत्र मुखोपरि चन्द्रस्याध्यारोप आदौ, तद्दर्शनेन सुतरामाह्लादमात्रजनकत्वतात्पर्येण तदपवादोऽपि\index{vakrokti@\textsl{vakrokti}} पश्चात्~।  एवं कलासामान्यस्य रूपसर्वस्वं वक्रतेति प्रतीयते~। ध्वनिस्तावत्परम्पराऽनुरणनरूपो\index{dhvani@\textsl{dhvani}}\index{dhvani@\textsl{dhvani}!anurananadhvani@\textsl{anuraṇanadhvani}}\index{anurananadhvani@anurananadhvani@\textsl{anuraṇanadhvani}} रसेन विना न कुत्रापि विरमति~। यथा ब्रह्मवादिनां सर्वोपाधिजातं यावत्सच्चिदानन्दघनसाधनं\index{rasa@\textsl{rasa}} तावन्नेति-नेतिक्रमेणैव निराक्रियते~। इत्थं प्रतिपदमस्माभिः कलाशास्त्रस्य वेदान्तशास्त्रोपजीवित्वं सनिदर्शनं सयुक्तिकमपि प्रपञ्चितमित्यलम्~।}
\begin{quote}
{\dev यथा वेदान्तविद्यायां जीवन्मुक्तपरम्परा~।}\\
{\dev तथा साहित्यविद्यायां महाकविपरम्परा ॥}
\end{quote}

{\dev तदिदमवधेयं यद्भारतीयपरम्परायां सर्वप्रमाणापेक्षया निस्सामान्यसार्वत्रिकानुभव एव सार्व\-भौमत्वमर्हतीति~। अनेनैव शुष्कतर्कापेक्षया स्वरसानुभवरासिक्यं वरमिति स्फुरति~। तथा च शतानां शब्दशरणानां शास्त्रज्ञंमन्यानां गड्डरिकाप्रवाहापेक्षया त्वेक एव ज्ञानविज्ञानवान्प्रश\-स्यते~। अनेनैव नयेन वेदान्तविद्यायां  वामदेव-याज्ञवल्क्य-श्रीकृष्ण-शङ्कर-रामकृष्ण-रमणा\-दीनां जीवन्मुक्तपरम्परेव साहित्यविद्यायां महाकविपरम्परा यद्व्यास-वाल्मीकि-कालिदास-शूद्रक-बाणादीनां,\index{rasa@\textsl{rasa}} होमर्-शेक्स्पियर्-दोस्तोविस्किप्रमुखाणां च साहित्यशास्त्रज्ञनामधारिणां महासमूहापेक्षया बलवत्तरेति\index{Kalidasa@Kālidāsa}\index{Vyasa@Vyāsa}\index{Valmiki@Vālmīki} निश्चीयते~। अन्यास्वपि कलासु यथा गीत-नृत्य-चित्र-शिल्पादिष्वपि सरणिरियं समूह्या~।}
\begin{quote}
{\dev आत्मतत्त्वं निराकर्तुं सच्चिदानन्दमात्रकम्~।}\\
{\dev भारतीयार्षसंस्कृत्यां व्रियन्ते बहुधा क्रमाः ॥}
\end{quote}
\begin{quote}
{\dev आनन्दद्वेषिभिर्नित्यं दुस्तर्काभासभासुरैः~।}\\
{\dev तत्रान्यतम एवासौ रसनिर्वापणो नयः ॥}
\end{quote}
\begin{quote}
{\dev शास्त्राभासान्शुष्कतर्कान्कलाभासान्दुराग्रहान्~।}\\
{\dev विहाय पश्यतां सद्यो रसतत्त्वं स्फुरत्यलम् ॥}
\end{quote}
\begin{quote}
{\dev भारतीयकलास्वादे कोविदास्तद्विनिर्मितौ~।}\\
{\dev कुशला ये च ते प्रायो रसनिर्वचने क्षमाः ॥}
\end{quote}
\begin{quote}
{\dev अथवा स्वाग्रहोन्मुक्ताः कलासामान्यशीलिताः~।}\\
{\dev स्वसंविद्रसिका अत्र समर्था इति निश्चितम् ॥}
\end{quote}

{\dev कुतोऽयं रसविध्वंसनव्यसनविधिरिति विचिन्त्यते चेत्प्रतीयत इदं यत्सर्वदा न केवलं भारते जगति सर्वत्राप्यात्मवादिभिः सह नैरात्म्यवादिनो विवदन्ते~। अमीषां नैरात्म्यवादिनां तु कण्ठ\-कुठारास्पदं तथ्यं तावज्जगतीहैव सामान्यानामपि मानवानां सदैवानुभवभाव्यं कला\-स्वादनं निरतिशयानन्दस्वरूपमिति, तदेव रस इति व्यपदिष्टमित्यपि~। ते तु कथञ्चिद्ब्रह्मनिरा\-करणेऽपि समर्थाः स्युः, यतो हि नानाविधप्रपञ्चनत्वाद्ब्रह्मस्वरूपस्य~। परं समस्तजनसेव्यमाने कलानन्दे न तादृशं भेदमुत्पादयितुं प्रभवन्ति सुतरां सामान्यानुभवविरोधित्वात्~। अत एव तन्निराकरण एव बद्धबुद्धयः खलीकुर्वन्ति स्वानुभूतिमपि~। तदर्थमूहातीता अपि दुर्वादा आद्रियन्ते~। किमुत, सात्त्विकं सुखमेव निराकुर्वन्ति, त्रिकरणैरपि द्विषन्त्यहो~। तस्माद्रसतत्त्वपरिज्ञानार्थं  प्रतिनिविष्टबुद्धिरियं दूरीकार्या~। विशिष्य भारतीयकलापुरस्सरं रसानुभूतिप्राप्त्यर्थं तत्तत्कलानां तत्त्व-प्रयोगेषु प्राञ्जलावगाहः कार्यः~। अथवा स्वस्वदेशीय-स्वस्वकालीयकलास्वादेन संभा\-वितहृदयाः स्युः~। मुक्तमात्सर्यास्त्यक्तकातर्याः स्युः~। सर्वमिदं सहृदयलक्षणनिरूपणावसरे\index{rasa@\textsl{rasa}} लोचनकृता\index{sahrdaya@\textsl{sahṛdaya}} कथितम्~। तत्पूर्वमेव नाट्यवेदप्रथमाध्याये\index{Locana@\textsl{Locana}} मुनिर्जगाद माननीयम्~।}   
\begin{quote}
{\dev चतुर्वेदमयं नाट्यं व्यपदिश्य महामुनिः~।}\\
{\dev सौन्दर्यनयसर्वस्वं ध्वनतीति सुनिश्चितम् ॥}\index{Natyasastra@\textsl{Nāṭyaśāstra}}
\end{quote}

{\dev भारतीयसौन्दर्यशास्त्रशाखिनो मूलं वेदाः~। तदिदं तथ्यं नाट्यवेदारम्भ एव मुनिना न्यरूपि~। अतो हि भारतीयकलानामात्मोपवसतिर्न कदाचन निराकृतिमर्हति~।}
\begin{quote}
{\dev पुरुषे नासदीये च रुद्राध्याये\index{Locana@\textsl{Locana}}\index{Purusasukta@Puruṣasūkta} यदूर्जितम्~।}\index{Nasadiyasukta@Nāsadīyasūkta}\\
{\dev स्कम्भोच्छिष्टादिभागे\index{Rudradhyaya@Rudrādhyāya} वा तत्रान्वेयो रसोदयः\index{Skambhasukta@Skambhasūkta} ॥}\index{Ucchistasukta@Ucchiṣṭasūkta}
\end{quote}
\begin{quote}
{\dev भूमविद्याप्रकरणे पञ्चकोशविमर्शने~।}\\
{\dev अवस्थात्रयचिन्तायां श्रुत्यन्तोक्तिर्यदूर्जिता ॥}\index{rasa@\textsl{rasa}}
\end{quote}
\begin{quote}
{\dev साप्यत्र सारतो ग्राह्या तथा श्रीकृष्णवाचिकी~।}\\
{\dev विभूतियोगसंदृब्धा नीतिश्चापि महीयसी ॥}
\end{quote}
\begin{quote}
{\dev रससिद्धान्तसंबद्धा ध्वनिसिद्धान्तयोजिताः~।}\\
{\dev सर्वे वादा दार्शनिकाः स्वानुभूतिविशोधकाः ॥}
\end{quote}

{\dev वेदेष्वेव काव्यानां च कलानां च व्यपदेशो भूरि दृश्यत इति नैकेषु प्रसिद्धेषु सूक्तेषु विश्रुत एव~। तथा चात्र परमानन्दपारम्यं निरतिशयरूपेण जागर्तीति न संशयः~। अस्मत्परम्परायामपि वेदानां काव्यमूलत्वं तत्र तत्र प्रापञ्चि\index{dhvani@\textsl{dhvani}} विज्जिका (एकोऽभून्नलिनात्...\index{Vijjika@Vijjikā} इति पद्ये)-भवभूति(उत्तर\-रामचरिते)\index{Bhavabhuti@Bhavabhūti}-राजशेखरादिभिः~। अर्वाञ्चोऽप्यरविन्द-गणपतिमुनि-कपालिशास्त्रि-वासुदेव\-\break{शरणा}\-ग्रवालादयोऽत्र\index{Rajasekhara@Rājaśekhara} समर्थानुवर्तिनः~। अतो रसतत्त्वस्य वेदमूलत्वं वेदानामनुभवमूलत्वं च निष्प्रतिहतम्~।  एतदेवोपनिषत्सु कारिकोक्तानां भूमविद्यादीनां प्रकरणव्यपदेशेन मतिरथपथ\-मावहति~। अन्यच्च श्रीमद्भगवद्गीतासु\index{Agrawala, V. S.} विभूतियोगे भगवता वासुदेवेन कलासौन्दर्यं निसर्ग\-सौन्दर्यं मानुषसौन्दर्यं च विविधाभिर्विभूतिभिः पथप्रदर्शकरूपेण सूत्रप्रायं प्राभाणि~। किमधिकं, यानि न्याय-साङ्ख्य-योग-मीमांसादीनि दर्शनानि तात्पर्येण वेदान्तदर्शनलक्ष्याणीति\index{Bhagavadgita@\textsl{Bhagavadgītā}} हिरियण्ण\-महोदयो मनुते तानि सर्वाण्यपि वेदोपवसतित्वादेव रसतत्त्वनिर्वचनेऽपि\index{Hiriyanna, M.} वावदूकतां वहन्ति~। तस्माद्रसमीमांसायां दर्शनानां प्रसक्तिः परम्पराविरोधिनीति न केनापि भाव्यम्~। तदस्त्येव मद्गुरुचरणानां कृष्णमूर्तिमहोदयानां काचिदाशङ्का स्वीयरसोल्लासकृतौ यच्छड्दर्शनप्रस्तावेन\index{Rasollasa@\textsl{Rasollāsa}} सौन्दर्यशास्त्रार्थं न किञ्चिदपि प्रयोजनमिति तत्तेषामेव परमगुरुभिर्हिरियण्णमहोदयैर्निरस्यते~। वस्तुतस्तु कृष्णमूर्तिवर्यस्य विवक्षा स्वानुभूतिविधुरशुष्कशास्त्रकोलाहलमात्रेण रसमीमांसा न कार्येति~। तत्र तस्यैव विदुषः परवर्तिन्यो नैककृतयः प्रमाणमिव समुल्लसन्ति यथा ‘भारतीय\-काव्यमीमांसे --- तत्त्व मत्तु प्रयोग’,} `Studies in Indian Aesthetics', `Indian Literary Theories: A Reappraisal' {\dev इत्याद्याः~। तथा भाषण-संवादेषु सहृदयनिष्ठत्वमेव तत्रभवता\index{sahrdaya@\textsl{sahṛdaya}} प्रत्यपादीति वयमेव साक्षिणः~।}
\begin{quote}
{\dev भारतीयकलाशास्त्रप्रस्थानानां परम्परा~।}\\
{\dev मूल्यमापनमाहात्म्याद्विश्वासार्हा चिरन्तनी ॥}
\end{quote}

{\dev न केवलं रसदर्शनस्य परमलङ्कार-गुण-रीति-मार्ग-वक्रता-ध्वनि-औचित्यादीनि सर्वाण्यपि काव्यतत्त्वानि कवि-सहृदयसंवेदनविशुद्धानि यथान्यायं स्थानमर्हन्तीति सुनिश्चितम्~। यत एतेषां तत्त्वानां प्रतिपादनपूर्वमेव\index{dhvani@\textsl{dhvani}} व्यास-वाल्मीकि\index{Vyasa@Vyāsa}\index{sahrdaya@\textsl{sahṛdaya}}-कालिदासादयो महाकवय इति मानिताः पुनरेताभिः परिकल्पनाभिः प्रत्येकतया विमृष्टा अस्खलितस्थानेष्वेव प्रतिष्ठिताः खलु! यथा त्वलङ्कारमार्गेण `उपमा\index{Valmiki@Vālmīki} कालिदासस्य'\index{Kalidasa@Kālidāsa}ति, गुण-रीत्यादीनां वर्त्मना `वैदर्भगिरां वास' इति,\index{Kalidasa@Kālidāsa} वक्रोक्तिपथेन वक्रतापरमाचार्य (`कुन्तके कालिदास' इति\index{vakrokti@\textsl{vakrokti}} कृष्णमूर्तिमहोदयस्य लेख\break उल्लेख्यः) इति, ध्वनिसरण्या\index{Krishnamoorthy, K.} ध्वनिधुरन्धर (``कालिदासप्रभृतयो द्वित्राः पञ्चषा वा महाकवय''\index{dhvani@\textsl{dhvani}} इत्यानन्द\-वर्धनः) इति, औचित्यपद्धत्या\index{Anandavardhana@Ānandavardhana} परमौचित्यकारीति (``पठेत्समस्तान्किल\index{aucitya@\textsl{aucitya}} कालिदास\-कृत\-प्रब\-न्धानि''\index{Kalidasa@Kālidāsa} ति क्षेमेन्द्रः) इति मतैक्यं वहद्भिः प्रस्थानैः किमन्यद्वा निरूप्यते? अन्यच्चाद्यापि तत्त्वान्येतानि\index{Ksemendra@Kṣemendra} रसध्वन्यौचित्यवक्रतारूपेण सर्वप्रकाराणां काव्यादिकलानां गुण-दोषविमर्शने स्वारस्यमीमांसने च समर्थानि सन्तीत्यस्मदुपज्ञोदितकृतिष्वारभ्य नैकेषां विदुषां रचनास्वपि साक्ष्यमस्तीति सूचयितुमलम्~।} 

\newpage

\section*{{\dev `रसाध्याय' विवेचनम्}}
\index{aucitya@\textsl{aucitya}}
\index{dhvani@\textsl{dhvani}}\index{dhvani@\textsl{dhvani}!rasadhvani@\textsl{rasadhvani}}\index{rasadhvani@rasadhvani@\textsl{rasadhvani}}

{\dev वयमिदानीं हृदयेकृतपूर्वपीठिकाः श्रीमतां पोल्लाक्\index{rasadhyaya@\textsl{rasādhyāya}} महोदयानां ’रसाध्याय’ग्रन्थस्य प्रास्तावि\-काव\-लोकनेन प्रस्तुतं प्रकल्पं निर्वर्तयामः~। अत्र तेषां कृतेः प्रास्ताविकमात्रपरामर्शनार्थं हेतुर्द्विधा~। तेषां स्वोपज्ञविचारबाहुल्यमस्मिन्भाग एव निक्षिप्तमित्येकः;\index{Pollock, Sheldon} अपरस्तु प्रस्तुतप्रबन्धस्य गात्र\-परिमितिः~।}

{\dev स दोषज्ञः कथयति भारतीयसौन्दर्यमीमांसायां काव्येतरकलानां सुषमाविवेचनमुपेक्षितप्राय\-मेवेति} (Pollock 2016:24){\dev~। किञ्च नायं क्षोदक्षमोऽभिप्रायः~। यतः ---}  
\begin{quote}
{\dev कलानां सकलानां तु रस एव परा गतिः~।}\\
{\dev संस्कृतेः सुषमा सर्वा तस्मिन्नेव प्रतिष्ठिता ॥}
\end{quote}
\begin{quote}
{\dev निजानां चित्तवृत्तीनां तटस्थात्मीयदर्शनम्~।}\\
{\dev कलासर्वस्वगन्तव्यं रसतत्त्वतयाऽऽदृतम् ॥}
\end{quote}
\begin{quote}
{\dev साहित्यगीतनृत्तानां चित्रशिल्पादिसंहतेः~।}\\
{\dev नाट्यं समाहृतिस्तस्मात्तद्रसः सर्वसंमतः ॥}
\end{quote}

{\dev नाट्यं तु सर्वकलानां समाहार इति स्थितम्~। अतो हि नाट्यमीमांसावसरे सर्वापि कलाभाविती त्वन्तर्हिता~। अन्यच्च यथा काव्यसौन्दर्यविवेचनार्थमलङ्कारशास्त्रमिति वा साहित्यविद्येति वा ज्ञानशाखाऽस्ति, तथैव गीत-नृत्यादीनां स्वारस्यविवेकार्थं सन्ति सङ्गीतशास्त्राणि~। विदितमेव हि वाक्यमिदं सङ्गीतरत्नाकरीयं यद् "गीतं नृत्यं तथा वाद्यं त्रयं सङ्गीतमुच्यते"~। अमरकोषोऽपि\index{Sangitaratnakara@\textsl{Saṅgītaratnākara}} "तौर्यत्रिकं नृत्यगीतं वाद्यं नाट्यमिदं त्रयम्" इति सहमतं वक्ति~। दुर्दैवादिव ‘रसाध्याय’कर्ता\index{Amarakosa@\textsl{Amarakośa}} सङ्गीतशास्त्रग्रन्थानां पल्लवग्राहिपाण्डित्यमपि नात्मनि कलयति~। अन्यच्च लेखनसौकर्यदुर्भरे मुद्रापणसौलभ्यरहिते तस्मिन् युगे समस्तकाव्यानामाद्यन्तविमर्शनं दुःशकमेवासीत्~। किमुत सकलकलानां व्याख्यानम्? अपि च साम्प्रतिका इव प्राचीनभारतीया न कदापि काव्यानामा\-न्वयिकविमर्शनद्वारा वाचकानां\index{rasadhyaya@\textsl{rasādhyāya}} सहृदयतां कदर्थयन्ति स्म~। केवलं काव्यानां स्थूलशब्दार्थता\-त्पर्यपरिज्ञानमात्रं साहाय्यं टीकाभिः कल्पयन्ति स्म~। प्रायेणाभिधावृत्तिपर्याप्ता व्युत्पत्तिनैयून्य\-समीकरणमात्रतात्पर्या भवति स्म तेषां व्याख्याक्रिया~। केवलं पाठकानां भावयित्रीप्रतिभापरि\-चर्यार्थं\index{sahrdaya@\textsl{sahṛdaya}} सर्वमिदं कार्यजातं प्रवर्तते स्म~। यतो हि निर्णिबन्धध्वननशीलतां के वा निरुन्ध्युः? को वापि\index{bhavayitripratibha@\textsl{bhāvayitrīpratibhā}}\index{pratibha@\textsl{pratibhā}!bhavayitri@\textsl{bhāvayitrī}} रसध्वनेरवधिः? अतः साम्प्रतिकविमर्शका इव तदानींतनीयाः प्रायेण स्वाभिप्रायाभि\-सन्धिभिः काव्यरसिकान्न बा(बो)धयन्ति स्म~। एतद्विहाय काव्यसामान्यतत्त्वपरिज्ञानार्थं प्रौढप्रस्थानग्रन्थाः,\index{dhvani@\textsl{dhvani}}\index{dhvani@\textsl{dhvani}!rasadhvani@\textsl{rasadhvani}}\index{rasadhvani@rasadhvani@\textsl{rasadhvani}} प्रस्निग्धपाठ्यपुस्तकानि च सहृदयरुचिरूपणार्थं प्रभवन्ति स्मेति समूहितुमलम्~। गीतनृत्यादिकलासु तावत्समस्या विभिन्ना चैतदधिका च~। यतः कला इमाः प्रयोगैकपारम्याः ; ग्रन्थस्थीकरणं तु तासां दुःशकमेव~। तथा कृते सत्यपि तद्बोधः सुतरां कठिन एवेति विद्यमानतत्तच्छास्त्रलक्षणग्रन्थैरेव ज्ञायते~। अत एव प्रायेण काव्यकलापेक्षया कलान्तराणां ग्रन्थस्थीकरणं विरलमिति वक्तुमलम्~। तथापि मुद्रापणयुगपूर्वस्मिञ्जगति पश्यामश्चेद्भारतं विहाय नान्यत्र क्वापि गीत-नृत्य-शिल्पादिकलानां ग्रन्थस्थीकरणमियति प्रमाणे दरीदृश्यते~।}

{\dev पूर्वोक्तग्रन्थकर्ता पुनश्च भारतीयसौन्दर्यशास्त्रं केवलं साहित्यरसमेव रसत्वेन मनुते} (Pollock 2016:24)  {\dev नान्यासां कलानामित्यभिप्रयति~। एतदप्यलीकभाषितम्~। नूनं सन्त्येव केचन साम्प्रतिकास्तस्य साधुवादिनोऽस्मिन्विषये~। परमिदमाक्षेपणं सर्वथा नष्टविवेकमिति निवेद्यते~। सर्वमिदं पूर्वमेवास्माभिः सविस्तरमन्यत्र प्रत्यपादीति नात्र निर्वहणैषिता क्रियते} (Ganesh 2014) {\dev~। अन्यैरपि विद्वद्भिस्तदिदं प्रपञ्चितमेव~। सङ्क्षेपेण वक्तव्यमिति चेत्सर्वाश्च कलाः कला\-कृतां चित्तवृत्तिमूला रसिकानां पुनश्चित्तवृत्तीरेव चोदयन्ति पर्यन्ते तद्दर्शन एव साफल्यमनु\-भवन्ति~। अतः सर्वेऽपि तथाकथितकाव्यरस-गीतरस-नृत्यरसमुखाश्चित्तवृत्तिमूलाश्चित्तवृत्ति\-चूलाश्चित्तवृत्यन्तराला इति निश्चिते सति को वा विवदीतु?}

{\dev विद्वानसौ कथयति भारतीयपरम्परायां प्रतिभातत्त्वविवेचनं न सम्यगभिहितमिति} (Pollock 2016:24) {\dev~। एतदप्यसदेव~।}
\begin{quote}
{\dev प्रतिभा तत्त्वतस्त्वन्ते स्वसंवेदनचेतना~।}\index{pratibha@\textsl{pratibhā}}\\
{\dev अनिर्वाच्या ह्यतस्तस्याः कार्यमेव निरूप्यते ॥}
\end{quote}
\begin{quote}
{\dev निरुपाधिक आनन्दो रसश्चापि तथैव हि~।}\\
{\dev तस्मात्तत्कार्यरूपेण प्रभेदा एव कीर्तिताः ॥}
\end{quote}
\begin{quote}
{\dev एवमेवालङ्कृतीनां गुणानां मार्गगामिनाम्~।}\\
{\dev विवृतिर्भूरिशो भाति रसस्यापेक्षया कृतौ ॥}
\end{quote}

{\dev प्रतिभा नामापूर्ववस्तुनिर्माणक्षमत्वमित्याहुः~। यत्र सर्वेषां कलाकार्याणां कारणमिति कल्पितं तदेव प्रतिभेत्यवगम्यते~। यदि कारणमेव मीमांस्यते तर्हि तत्कस्याप्यपरस्य मूलकारणस्य कार्यत्वेनावगम्यते~। एवमनवस्थादोषस्तु दुर्निवार्य एव~। अतो हि वेदान्ते ब्रह्मस्वरूपमिव केवलं स्वरूपलक्षणसङ्क्षेपकथनमात्रेण विरम्य प्रतिभातत्त्वमप्यालङ्कारिकाः प्रमेयान्तरमूरी\-कुर्वन्ति~। नैतत्तावद्दूषणं प्रत्युत भूषणमेव प्रतिभाया अचिन्त्यस्वरूपत्वात्~। विदुषा तेन पुराऽऽन्वयिककाव्यविमर्शनं\index{pratibha@\textsl{pratibhā}} दार्शनिकान्वेषणमिव नानुष्ठितमित्यपि समाक्षिप्तम्}\index{pratibha@\textsl{pratibhā}} (Pollock 2016:24){\dev~। वादोऽसावपि न क्षोदक्षम इति प्रतिभाति~। पूर्वोक्तरीत्या भारतीयकाव्यमीमांसा सहृदये बहुधा श्रद्दधाति~। न कदापि तमवगणयति~। अत एव सिद्ध्यपेक्षया स्वाद एव परमां काष्ठामधिवसति~। याश्च संस्कृतयः फलकृपणास्ता एव निजानन्दापेक्षया सर्वमन्यन्महदिति मन्वते~।\index{sahrdaya@\textsl{sahṛdaya}} अत्रापि भारते देशे केचन तथा स्युः~। परं नैतेषां कलास्वादनार्हतापि सिद्ध्यतीति समयोऽयं सौन्दर्यविदाम्~। सङ्क्षेपेण वक्तव्यं चेत्कामापेक्षया सुतरामर्थोऽवर इति~। स च कामस्तु हृदयविस्तारपरं धर्ममनुरुन्धन्मोक्षाय पूर्वपीठिकेवापि समुन्नतं पदमधिवसतीति स्वसंविद्वेदिनां साक्षात्कारः~। अन्यच्च पराग्रूपिणि द्वैतमूलिनि पाठ्यमात्रावलम्बिनि चाऽन्वयिकविमर्शने कथमिव निस्सामान्यामूर्ततत्त्वाभिमुखत्वम्? तस्मादिदं वदतोव्याघात इव प्रसज्यते~। अत एव तात्पर्यमेवम् ---}   
\begin{quote}
{\dev आस्वादापेक्षया मूल्यमापनं नैव शस्यते~।}\\
{\dev अपूपा भक्षितव्याः किं सुषिरं गणयेन्नु किम् ॥}
\end{quote}
\begin{quote}
{\dev ध्वनिमार्गेण सर्वासां कलानां मूल्यमापनम्~।}\\
{\dev सिद्धमित्यपरास्त्येव नीतिरौचित्यचञ्चुरा ॥}\index{dhvani@\textsl{dhvani}}
\end{quote}

%raghu, word 14

{\dev आन्वयिकविमर्शनार्थं काव्यानां ध्वनिसिद्धान्तस्तावद्वर्तत एव~। ध्वनि-गुणीभूतव्यङ्ग्ययो\-रालम्ब\-नेन प्रायेण सर्वाश्च कलाः सहृदयदृष्ट्या व्यवस्थापयितुमलमिति पूर्वमेव प्रतिपादितम्~। अत्र नैके पूर्वसूरिणश्च प्रबन्धाश्चास्माकं साक्ष्यमिव यच्छन्ति~। कलाविद्दृशा तु वक्रतानाम्नी महती शक्तिरस्त्येव विमर्शनार्थम्~। एतेषां सर्वेषामप्युपरि विद्यत एवौचित्यमिति व्यापकं विमर्शन\-साधनं येन देशकालानुसारिणी नानासंस्कृति-नागरकताव्यापारसहस्रवैविध्यशोधनी सौन्दर्य\-मीमांसा हस्तसाद्भवति~।}

{\dev समग्रसौन्दर्यमीमांसा कापि दार्शनिकदृशा न प्राचलदित्यप्यसत्} (Pollock 2016:24){\dev~। यतः ---}
\begin{quote}
{\dev कलामात्रे रसस्यैव मीमांसा पारमार्थिकी~।}\\
{\dev ततोऽप्युन्नतिरस्त्येव ब्रह्ममीमांसनाध्वनि ॥}
\end{quote}

{\dev विद्वद्वरेण हिरियण्णवर्येण स्वकीये} `Art Experience', `The Indian Conception of Values' {\dev चेति ग्रन्थद्वये निरुपमानविधिना वादाभासोऽयं पूर्वमेव निराकारि~। पाश्चात्यानामिव भारतीयानां कलामीमांसा साक्षाद्दर्शनशास्त्राणां प्रकरणेषु नान्तर्गतेति प्रायेण तद्विद्यासौभाग्य\-मित्येव मन्महे वयम्~। यतो हि यद्यपि सर्वदर्शनस्वारस्यपरिपुष्टापि भारतीयकलामीमांसा स्वतन्त्ररूपेण निस्सामान्यस्वानुभवमेव  प्रमाणयन्ती रसाभिव्यक्तिवादे व्यञ्जनाव्यापारविश्रान्ते सुखेन तत्त्वपरिपूर्तिमेधाञ्चकार~। अनेन हेतुना सा तु वस्तुतन्त्रपदमुपारूढा सर्वमान्यापि सम\-जनि~। नो चेन्नैकेषु दर्शनाभासेषु दृप्तलक्ष्मीवतां भवनेषु दरिद्रबान्धवधात्रीव कर्तृतन्त्रक्लेशक्लिष्टा दीनदीनं लुठति स्म~। तस्माद्वर्तमानवस्तुस्थितिरद्य भारतीयकलामीमांसायाः सर्वथा शोभन\-करीत्यवोचाम~।}

{\dev विपश्चिता तेन भरते वात्सल्यरसस्याभावः प्रदर्शितः} (Pollock 2016:28-29){\dev~। नायं नूतना\-क्षेपः~। सर्वेऽपि विवेकिनो विशदानुभवशीलिनो निर्विवादमङ्गीकुर्वन्ति यद्रससङ्ख्यामीमांसापेक्षया रससिद्धान्तः परमं गरीयानिति~। मुनिनापि भरतेन तत्र तत्र प्रादर्शि स्वागमे यच्छास्त्रविस्तरणं गच्छता कालेन यथोचितरीत्या सूरिभिः कार्यमिति~। प्रायेण विधिमिममेव शिरसिकृत्वा नैक आलङ्कारिकमूर्धन्याः स्वं स्वं विचारं यद्भरतोपज्ञोपष्टम्भकं प्राचीकटन्~। समग्रमिदं शास्त्र\-मखण्डैकवृत्त्या जङ्गमशीलतया च ग्राह्यमिति सुधीमतां मतम्~। तदिदं सर्वविद्यानामपि निर्व\-र्णनलक्षणमिति च गम्यते~। अन्यच्च रतिर्नाम स्वात्मसंतर्पणभाव इति चिन्तयामश्चेदन्यैव या कापि संविद्विच्छित्तिः प्रस्फुरेद्यथा ---}
\begin{quote}
{\dev रतिः सकामा शृङ्गारो निष्कामा वत्सलत्वभाक्~।}\\
{\dev अन्यच्च सर्वभावानां रसत्वाप्तिः क्वचिन्मता ॥}
\end{quote}
\begin{quote}
{\dev श्लोकोऽयं व्यापकं व्याख्यानमपेक्षत इति नात्र विस्तृतिरिष्यते~।}
\end{quote}

{\dev नाट्यवेदे रसस्तु कलाकृन्निष्ठो न तु सहृदयस्येति कोऽप्याक्षेपः} (Pollock 2016:29){\dev~। अत्र कृष्णमूर्तिमहोदयस्यापि सहमतिरस्तीति सूचनापि काचिच्चकास्ति~। अश्रद्धेयमिदं परमार्थतः~। यतः ---}
\begin{quote}
{\dev नाभिज्ञाते सहृदये रसः किं वा प्रयोजनम्~।}\\
{\dev कलालोकस्य सर्वस्य वक्रतावर्त्मनः परम् ॥}
\end{quote}
\begin{quote}
{\dev कलाकृद्यदि भोक्ता स्याद्रसस्यानन्यमात्रकः~।}\\
{\dev तर्हि सोऽपि भवेत्स्वस्य निर्मितेः सहृदेव हि ॥}
\end{quote}

{\dev सर्वसामान्यविवेकवेद्यं यच्चरमगत्या रसो यदि न सहृदयहृदय आविष्कृतो न भवति तर्हि सर्वदा सर्वथा च तस्य वैतथ्यं वैयर्थ्यं च सिद्धम्~। रसस्य मूलं वस्तुनि वा कलाकृति वा पात्रे वाप्यनुकर्तरि वा  यत्र कुत्रापि स्यान्नाम ; परं तत्पर्यन्तभूमिः सहृदय एव~। अस्याः सरण्याः साधनकाले प्रायेण लोल्लट-शङ्कुक-भट्टनायकादीनां वादाः प्रसृताः स्युः~। परं सर्वेऽपि तात्पर्येण सहृदयनिष्ठा इति निभालनीयम्~। यथा सर्वाण्यपि भारतीयदर्शनानि वेदान्तपर्यवसायीनीति हिरियण्णवर्यमतं तथैवात्रापि ज्ञेयम्~। किं बहुना भरतागमस्य प्रथमाध्याय एव विधातरि देवदानवमानवानां क्रीडनीयकाभ्यर्थनावधौ रूपकप्रेक्षकलक्षणनिरूपणे च तत्त्वमिदमेव स्फुरति~।}

{\dev पुनश्च पोल्लाक् वर्यः अलङ्कारशास्त्रं नाट्यमीमांसापेक्षया क्वचिदर्वाचीनमपि मनुते} (Pollock 2016:30){\dev~। एतदस्माकं विचारमूढमिति प्रतिभाति~। यतः ---}
\begin{quote}
{\dev साहित्यगीतचित्राद्याः कला नाट्ये प्रतिष्ठिताः~।}\\
{\dev अतस्तासां च मीमांसा तत्त्वदृष्ट्या सनातनी ॥}
\end{quote}
\begin{quote}
{\dev अन्यच्च लक्षणादीनां वृत्तीनां चिन्तने मुनिः~।}\\
{\dev इतिवृत्तेऽपि तद्ब्रूते छन्दसां विवृतावपि ॥}
\end{quote}

{\dev प्रायेण बहवोऽपि विपश्चिदपश्चिमाः प्राच्यभारतीयसंस्कृतौ कालानुपूर्वीविषये विविधतत्त्वानां वादानामुपजीव्योपजीवित्वप्रसक्तौ विविधविषयानधिकृत्य तत्तद्ग्रन्थेषु समुपकल्पितस्थलपरि\-माणतारतम्यविचारे च महतीं विचिकित्सां जनयन्तो विवादशालीनतायामेव स्वात्मानं धन्यं मन्यन्ते~। किञ्च भारतीयनैसर्गिकभावरीतिः काचिद्विभिन्नैव~। नूनमेतेषामुपरि निर्दिष्टानां विचा\-राणां महत्त्वं किमप्यस्त्येव~। तथापि सर्वे च ते भारतीयसंवेदनस्य हृद्भागं न कदाप्यधिवसन्तीति तथ्यम्~। केवलमङ्गभूताः खल्वेते~। विवेकमिमं भूयोऽप्युपप्लावयन्तो महान्तोऽपि मुह्यन्ति मोहयन्ति च~। तदिदमत्राप्यापतितम्~। सर्वमिदं श्रीनरसिंहभट्टेन सम्यक्प्राबोधि स्वीयकृतिषु~। नाट्यशास्त्रे षट्त्रिशल्लक्षणानां विवरणे व्याकरणालङ्कारच्छन्दोविचित्यादिषु काव्यमात्रस्य नैका\-न्यङ्गानि प्रपञ्चितानि~। तथा चेतिवृत्त-वृत्ति-रूपकप्रभेद-सन्धि-सन्ध्यङ्गादिष्वपि काव्यसंबन्धीनि भूयांसि तत्त्वानि तत्र तत्र प्रस्फुरन्तीति प्रकटमेव~।} 

{\dev मान्यलेखकः क्वचित्काव्योक्तिर्नाट्यप्रयोगापेक्षया गरीयसीति मनुते} (Pollock 2016:32)~{\dev~। अत्र तस्येङ्गितं तु श्रव्य-दृश्यकाव्ययोर्मध्ये किमपि वैमनस्यं जनयितुमिव प्रतिभाति~। रुचीनां वैविध्ये यत्किमपि कथञ्चिदपि भवितुमर्हति~। परं विशुद्धशास्त्रचिन्तनावसरे सर्वासामपि कलानां तत्तन्मात्रा कापि सार्वभौमिकी स्थितिर्जागर्तीति निश्चप्रचम्~। अन्यच्च प्रत्येकोऽपि कलाविशेषः स्वकीयं वैशिष्ट्यं यन्निजमितिमहितं बिभर्तीति दृष्टचरमेव~। यथा गाने याऽमूर्तता  विशुद्धभावैकप्रवणता वर्वर्ति सा प्रायेण शब्दार्थमयकथनात्मकसाहित्ये मनागिव कनीयसी स्यात्~। किञ्च वस्तुन एकस्य स्वभावप्रस्तावे नानाविधानां मूर्तामूर्तानां वस्तूनां वा भावानां क्रियाणां वा स्थलानां प्रस्तावमपि वक्रवाग्व्यापररूपेण सुलभतया समानेतुं साहित्ये शक्यं स्यात्~। परमिदं गीते क्वचिन्नृत्ते वा सर्वथा दुःशकमिति दृष्टचरमेव~। अथ च नृत्ये सकलानामपि सात्त्विकभावानां सूक्ष्मातिसूक्ष्मसुविशदसुनिपुणप्रकटीकरणं रससिद्धेन केनापि कलाविदा शक्यत एव~। अस्याच्छमाक्षिकरसास्वादसदृक्षस्य तीव्रानुभवस्य जानुदघ्नमपि नायाति कदाचित्साहित्यरसचर्वणा~। चित्र-शिल्पेषु या काप्यधिष्ठानसुषमा सुविस्तृता विलसति सा प्रतिक्षणोन्मीलितनिमीलितमधुरिम्णि कालप्रवाहतरलिम्नि गाने कथं वा परिलब्धुम्? अतः सर्वास्वपि कलासु या काप्यनिर्वचनीयविच्छित्तिरनन्या नानटीतीति निश्चप्रचम्~। तदेव भङ्ग्यन्तरेण ब्रूते तत्रभवान्काव्यादर्शकारः\endnote{{\dev इक्षुक्षीरगुडादीनां माधुर्यस्यान्तरं महत्~। तथापि न तदाख्यातुं सरस्वत्यापि शक्यते॥ (१.१०२)}}~। अन्यच्च मनोरङ्गस्थले नाट्यायमानत्वेन विना न कासामपि कलानां कलात्वं, तन्नाम रसास्वादत्वं, शक्यत इति लोकत्रयानुभवः~। तदेव पुनरत्र व्यपदिष्टम्~। यथा ---} 
\begin{quote}
{\dev प्रयोगत्वमनापन्ने काव्ये नास्वादसंभवः~।}\\
{\dev इति तौतीयवाक्येनाद्वैतं सर्वकलास्वपि ॥}
\end{quote}

{\dev दृढसंशयालुना विदुषा रसः कृतिगतो वा पात्रगतो वा सहृदयगतो वेति पुनश्च संशीतिमुत्थाप\-यति} (Pollock 2016:34)~{\dev~। तदिदमस्माभिः पौनःपुन्येन स्पष्टीकृतम्~। हन्त! पुनश्च परापतति प्रकारान्तरेण पिशाचिकेयं समुच्चाटनार्थम्~। उक्तं हि महतः शास्त्रस्य शतशताब्दविस्तृतस्य निर्माणान्वयकालस्य विविधानि चिन्तनानि महीयसापि विवेकेन स्वीकार्याणि परीक्ष्याणि निरू\-प्याणि च~। तदा ज्ञायत एवेदं तथ्यं यत्कविर्वा गायनो वा नटो वा चित्रशिल्पी वा स्वकलानिर्माण\-चिकीर्षायां यदा प्रेरितश्चान्तर्बहिश्च तदा बाह्यमिदं जगदान्तरमात्मनश्चित्तवृत्तीः कोऽपि सहृदय इव विविधकलाविशेषान्प्रतिमुहुर्निर्वर्ण्य तत्पश्चान्निजकलारचनायां व्यग्रो भवति~। अस्मिन्नपि काले तदा तदा कर्तृत्वेन भोक्तृत्वेन ज्ञातृत्वेन विमर्शकत्वेन चैवं बहुभूमिकामयेन भावेन स्वमेव निर्माणं परिशीलयन्ननुशीलयन्कामपि लोक-लोकोत्तरभावनामनुभवति~। तत्पश्चादवसिते सति तत्कार्ये (काव्य-चित्र-शिल्पादिषु) प्रयुज्यमाने वा (आशुकविता-गीत-नृत्यादिषु) सहृदयस्तद्रसं समास्वादयति~। सर्वेष्वेतेषु स्तरेषु कारयित्री-भावयित्रीप्रतिभयोः पर्यायपातीनि कटाक्षाणि यद्यपि प्रविलसन्ति तथापि सहृदय एव समग्रसमास्वादनसंभवः~। यतः कलाकृदेकः स्वीय\-कलाप्रेरणावधौ निर्माणावधौ वापि यदि रसानन्दं समनुभवति सहृदयास्तु नूनमसङ्ख्याका एव तस्य कलाकृतो निर्माणेन सर्वदा सर्वत्र च स्वादमास्वादयेयुः~। अनेनापि विधिना ज्ञायते यत्पर्यन्ततो रसः सहृदयनिष्ठो यतः कलाकृदपि मूलतस्तत्त्वतः सहृदय एव~। संदिहानाः केचन मन्येरन्यत्सहृदयोऽपि स्वीयभावयित्रीप्रतिभात्वात्कलाविदेव किलेति~। तदत्रास्माकं समाधानं कलाविदि द्वे अपि प्रतिभे स्थः~। किञ्च सहृदये तावदेकैव~। अनयोः सामानाधिकरण्याद्भावयि\-त्रीति ह्येकैव तिष्ठति~। अतः सर्वेऽपि सहृदयाः~। अन्यच्च सहृदयत्वेन परमार्थत आस्वाद एव चिरं स्थास्यति सत्त्वोद्रेकस्वरूपी ; न तु रजःस्पर्शमयी निर्माणशक्तिरित्यभ्युपगम्यते~। इदमेव भारतीयसंवेदनसारं यदास्वादः साधनापेक्षया वरमिति~। सर्वमिदमत्र सङ्क्षिप्तमेवम् --- }   
\begin{quote}
{\dev कवेः कल्पनकाले तु वस्तुनिष्ठा रसादृतिः~।}\\
{\dev काव्यनिर्माणवेलायां कृतिनिष्ठा भविष्यति ॥}
\end{quote}
\begin{quote}
{\dev रसिकस्य समास्वादे तन्निष्ठैवेति कथ्यते~।}\\
{\dev परं सर्वमिदं स्वस्मिन्सर्वथा सर्वदा स्थितम् ॥}
\end{quote}

{\dev सङ्ख्यावानसौ ख्यापयति क्वचिदत्र भारतीयकाव्यपरम्परा बौद्धमूलीयेति} (Pollock 2016:37){\dev~। इदं तु कौटिल्यवैकल्यभ्रान्त्यादीनां नैकेषां व्यभिचारिणां दुर्विलसितमिति सर्वथा भाव्यम्~। यतः ---}
\begin{quote}
{\dev बौद्धानां काव्यमूलत्वं न कदाचन कल्प्यते~।}\\
{\dev यतो वेदेषु तद्दृश्यं कलानामपि भूरिशः ॥}
\end{quote}
\begin{quote}
{\dev सर्वथा चित्तवृत्तीनां कलाद्वाराऽनुशीलनम्~।}\\
{\dev समस्तनरलोकाप्तिर्मतं नात्र विशिष्यते ॥}
\end{quote}

{\dev इह जगति सर्वास्वपि नागरकतासु संस्कृतिषु च मूलभूताः कलाः स्वस्वोपज्ञा एवेति प्रेक्षावता\-मपरोक्षम्~। काव्यं गीतं नृत्तं चित्रं वा प्रागैतिहासिकमानवा अपि कल्पयन्तः रञ्जयन्ति स्म स्वात्मन इति समूह्यमेव~। कापि संस्कृतिः कस्मादपि जनसमुदायात्काव्यप्रकारं वा छन्दःप्रभेदं वा वक्रोक्तिभङ्ग्यन्तरं वा स्वीकुर्यान्न तु साक्षात्काव्यकलामेव~। विशिष्य वेद-वेदाङ्ग-कथेतिहास\-वाङ्मयसमृद्धे सनातनधर्मे निवृत्तिमात्रमुग्धाद्बौद्धात्कथमिव शृङ्गारादिपुरुषार्थचतुष्टयप्रतिपाद\-करसमयस्य काव्यकलाप्रवरस्य सङ्ग्रहणसंभवः? अस्माभिः पूर्वमेव चिरन्तनतमस्य बौद्ध\-कवेरश्वघोषस्य रामायणानृण्यता प्रत्यपादि~। तथा च भारतीयकाव्यस्य सनातनधर्ममूलत्वं च प्रादर्शि नैकेषु प्रकरणेष्वित्यलम्\endnote{Ganesh, R., Ravikumar, Hari. {\sl The Bhagavad-Gita Before The Battle.}}~।}  
\begin{quote}
{\dev वेदान्तदर्शनं नूनं रससिद्धान्तहृद्गतम्~।}\\
{\dev अस्यार्वाचीनता नैव व्याख्याभिर्युज्यतेतमाम् ॥}
\end{quote}

{\dev काव्यशास्त्रकाञ्चनकषाश्ममानिना माननीयेन रसमीमांसा तु गच्छता कालेन ’वेदान्तीकृते’ति च कोऽप्यभियोगः कल्पितः} (Pollock 2016:39-40) {\dev~। इदं तावदत्यन्तमसन्मृगमरीचि\-केवेति वच्मो वयम्~। अवस्थात्रयमीमांसाकेन्द्रभूतस्य माण्डूक्योपनिषदो मूलस्रोतसस्तावद\-थर्ववेदस्य किल रसमूलत्वं मुनिना भाषितं पुरा! स्कम्भोच्छिष्टादिसूक्तेषु मुण्डकादिष्वप्युपनि\-षत्सु रस\-सिद्धान्तस्य जीवातुभूतं ब्रह्मानन्दाद्वैततत्त्वं सुप्रसिद्धमेव~। सर्वमिदं समीक्ष्य किं न केनापि याथार्थ्यं ज्ञायते? विचारोऽयं प्रथितयशोभिरानन्दकुमारस्वामिप्रमुखैर्नैकत्र नैकधा प्रपञ्चितमिति नास्माभिरत्र पुनर्विपुलीक्रियते~।}   

{\dev सहृदयंमन्येन तेन विदुषा क्वचिद्भारतीयानां रुचिभेदपरिज्ञानं नासीद्रसास्वादे लोकनीति-भाव\-नीतीनामपि संवेदनं नासीच्चेति साकूतं समाक्षिप्यते} (Pollock 2016:47, 53) {\dev~। विशि\-ष्योदाहरणरूपेण मुग्धामन्तर्वत्नीं शकुन्तलां निराकुर्वता दुष्यन्तेनानुष्ठिते निर्घृणकर्मणि रससि\-द्धान्तस्य किं वा दायित्वमिव च पृच्छति~। हन्त! सर्वेषामीदृशामाक्षेपाणां मूलं तु कलास्वाद-लोकास्वादयोरद्वैतं कल्पयतां प्रयोजनैकलोलुपानां सत्त्वातिशायिनि साक्षिभावेऽपि रजोरू\-षितकर्तृत्व-भोक्तृत्व-मदाहङ्कारहुङ्कारिणां रसवैमुख्यमेव~। तदिदं पर्यन्ततो ब्रह्मनिराकरणमेव, प्रत्युत निजानन्दन्यक्कारणमेव भवतीति पौनःपुन्येन प्रतिष्ठितमस्माभिः~। यदि विश्वस्मिन्स\-त्यस्य सौन्दर्यस्य शिवस्य चापि भारतीयार्षभावस्य विरोधि किमप्यस्ति चेत्तदिदमेव रसविद्वे\-षणं व्युत्पत्त्याभासभावनं समाजसुधारणाभासदुरहङ्कारदुर्विलसितमिति निरूपयामो यथा ---}        
\begin{quote}
{\dev रसास्वादनसंस्कारः शनैर्व्यञ्जनवर्त्मना~।}\\
{\dev व्युत्पत्तिं विविधां सूते तत्र का परिदेवना ॥}
\end{quote}
\begin{quote}
{\dev परन्त्वानन्दमात्रेण सर्वसङ्कल्पताऽदृतिः~।}\\
{\dev सिद्ध्यतीति स्वयं वेद्यं रसस्त्वव्याज एव हि ॥}
\end{quote}
\begin{quote}
{\dev नीतिः सामाजिकी रीतिर्व्याख्या वापि यदृच्छया~।}\\
{\dev रसानन्दस्य न क्वापि कर्म दुर्वार्यमुच्यते ॥}
\end{quote}
\begin{quote}
{\dev यथा ब्रह्म समस्तानां कर्मणामुत्तरं तथा~।}\\
{\dev रसानन्दोऽपि सर्वासां व्युत्पत्तीनामिति स्थितिः ॥}
\end{quote}

{\dev कदाचिदौचित्यं रसस्य नैतिक-सामाजिकमौल्यमार्गदर्शीत्यनुबध्नता साध्वनेन धीमता पुन\-रलङ्कृत्यशिरश्छेदन्याय इवास्य तत्त्वस्य कृपणतामपि कथयति} (Pollock 2016:54){\dev~। परं न कारणानि विवृणोति~। किञ्चास्मदविकलनिश्चयस्तावदौचित्यं सर्वासामपि व्युत्पत्तीनां परमायत\-नमिति~। यतो दर्शनशास्त्रेषु विवेक इव साहित्यविद्यायामौचित्यम्~। अत एव यथा नास्ति\-कास्तिकादिभेदानतिरिच्य सर्वाण्यपि दर्शनानि नित्यानित्यविवेकादिकं स्वस्वचिन्तन\-श्रेण्यां मोक्षोपायत्वेन स्वीकुर्वन्ति तथा समस्तान्यप्यालङ्कारिकप्रस्थानान्यौचित्यानौचित्य\-विवेकम्~। विवेकमतीत्य नैतिक-सामाजिकादिमौल्यानां को वान्यः प्रबलतमो विद्यते विज्ञान\-दण्डः? तस्मादभियोगोऽयमप्यपास्त एव~। तद्यथा ---} 
\begin{quote}
{\dev कार्यकारणभावानां व्युत्पत्तीनां तथैव च~।}\\
{\dev भूमिका त्वौचिती ह्येकाऽप्यनयालं विवेकिनाम् ॥}
\end{quote}

{\dev ग्रन्थारण्यगन्धेभस्तावदसौ ग्रथनकुशलः स्वस्यालङ्कारशास्त्रपाण्डित्यतपसः परमं फलमिव रससिद्धान्त एव यातयाम इति गतरस इति चतुरतया कथयति} (Pollock 2016:54-55){\dev~। तदिदं तस्य कृतिगीतस्य निगूढध्रुवपदमिव प्रतिभाति प्रेक्षावताम्~। सावधिर्विवेकः परमविवेको निरविधिरिति प्रत्यहं प्रत्यूचुरस्मदुपाध्याया महामहोपाध्यायाः श्रीरङ्गनाथशर्माणः साहित्या\-लङ्कारवेदान्तवर्माणः~। तदत्रान्वेयमित्यस्माकं भागधेयम्~। अद्यापि वाल्मीकिरस्मान्रञ्जयति व्यासो निरञ्जयति चेति कालिदासः समञ्जयति ; कतिपयसहस्रवर्षप्राचीनाः केचन मोहन-हिन्दोल-मध्यमावतीत्यादिरागा अनुरागेण निबध्नन्ति ; भारतीयं नाट्यशास्त्रं नैकनट-नर्त\-काभिनीतिभिरनुबध्नाति ; साञ्ची-एल्लोरा-बादामि-देवगडीयादीनि शिल्पानि समाह्लादयन्ति~। एव\-मेवाधुनाधुननिर्मितानि कथानकानि चलच्चित्राणि गेयानि चास्मान्सममेव सन्तोषयन्ति भूष\-यन्ति च~। सर्वत्र वयं रससूत्रेणैकेन सर्वानपि सुन्दरान्कलामणीन्स्वदेशीयान्विदेशीयान्वा समकालीनानन्यकालीनान्वा निरुपाधिकं निबध्नीमो निजहृदये च बध्नीमः~। किमिदं नालं निदर्शनं रसतत्त्वस्य सदातनत्वार्थम्? नूनमिदं दर्शनमेव~। रस-ध्वन्यौचित्य-वक्रतेति चतुः\-स्कन्धात्मकं भारतीयं सौन्दर्यशास्त्रं वस्तुतन्त्रत्वेन समग्रस्य विश्वस्यापि गर्वास्पदमिति निवेद्य विरम्यते~। तदत्र सङ्ग्रहः ---}
\begin{quote}
{\dev यातयामं गतरसं रसतत्त्वं यदि स्वयम्~।}\\
{\dev तर्हि मानुषसर्वस्वं नष्टमित्यवधार्यताम् ॥}
\end{quote}
\begin{quote}
{\dev मनसा मानवाः सर्वे मनोवृत्तिः सदातनी~।}\\
{\dev विभावा यदि भिद्यन्ते भावानां का नु वा क्षतिः ॥}
\end{quote}
\begin{quote}
{\dev विभावा अर्थमात्रा हि भावा वै कामरूपिणः~।}\\
{\dev अतः सर्वोऽप्यर्थजातः कामसेवापरायणः ॥}
\end{quote}
\begin{quote}
{\dev पाकपात्राणि भिद्यन्तां पाक एको हि सोऽप्यथा~।}\\
{\dev भिद्यते यदि चैकैव बुभुक्षा तर्पणा तथा ॥}
\end{quote}

{\dev चोद्यं तावदभियोगकण्टकमालामध्ये क्वापि निजनम्रताकम्रकुसुमान्यपि स्रगाभासपरिपूर्त्यै गुम्फति माननीयो मनीषी~।} (Pollock 2016:48)   
\begin{quote}
{\dev स्वाभिप्रायविशुद्धीनां मितिसंशीतयोऽसकृत्~।}\\
{\dev विदुषा भणिता आदाविति काचिद्विलक्ष्यता ॥}
\end{quote}
\begin{quote}
{\dev तदिदं सत्यमिति वयं श्रद्दधाना विरमामः~।}
\end{quote}

\begin{thebibliography}{99}
\itemsep=2pt
\bibitem[]{chap7_item1}
Bhat, Narasimha P. (2003). {\sl Bhāratīyaṛṣiparaṃpare mattu Saṃskṛtasāhitya}. Mangalore. 

\bibitem[]{chap7_item2}
---\kern3pt(2013). {\sl Bhāratīyasaṃvedane --- saṃvāda}. Udupi.

\bibitem[]{chap7_item3}
---\kern3pt(2016). {\sl Kāvyamīmāṃse - Hosa Hoḻahugaḻu}. Mangalore.

\bibitem[]{chap7_item4}
Ganesh, R. (2013). {\sl Bhāratīyakāvyamīmāṃse hege Bhāratīya. Ayana, Dr. T. Vasantha Kumar Felicitation Volume}. Udupi.

\bibitem[]{chap7_item5}
Ganesh (2014). ``Art Experience: A Classical Approach.'' Pratipitsa Journal. pp.~88--120.

\bibitem[]{chap7_item6}
Hiriyanna, M. (1954). {\sl Art Experience}. Mysore: Kavyalaya Publishers.

\bibitem[]{chap7_item7}
Krishnamoorthy, K. (1982). ``Pramanas: Criteria in Indian Aesthetics.'' {\sl New Bearings of Indian Literary Theories and Criticism}. Ahmedabad: B. J. Institute of Learning and Research.

\bibitem[]{chap7_item8}
Pollock, Sheldon (2016). {\sl A Rasa Reader}. New York: Columbia University Press.
\end{thebibliography}

\theendnotes
