\chapter*{Series Editorial}\label{gen_editorial}

\lhead[\small\thepage\quad K S Kannan]{}
\rhead[]{Series Editorial\quad\small\thepage}


It is a tragedy that many among even the conscientious Hindu scholars of~Sanskrit and Hinduism\index{Hinduism} still harp on Macaulay, and ignore others while accounting for the ills of the current Indian education system, and the consequent erosion of Hindu values in the Indian psyche. Of course, the machinating Macaulay brazenly declared that a single shelf of a good European library was worth the whole native literature of India, and sought accordingly to create “a class of persons, Indian in blood and colour, but English in taste, in opinions, in morals and in intellect” by means of his education system -- which the system did achieve. 

An important example of what is being ignored by most Indian scholars is the current American Orientalism\index{American Orientalism}.\index{Orientalism!American} They have failed to counter it on any significant scale. 

It was Edward Said (1935-2003) an American professor at Columbia University who called the bluff of “the European interest in studying Eastern culture and civilization” (in his book {\sl Orientalism} (1978)) by showing it to be an inherently political interest; he laid bare the subtile, hence virulent, Eurocentric prejudice aimed at twin ends – one, justifying the European colonial aspirations and two, insidiously endeavouring to distort  and delude the intellectual objectivity of even those who could be deemed to be culturally considerate towards other civilisations. Much earlier, Dr.\ Ananda Coomaraswamy\index{Coomaraswamy, Ananda K} (1877--1947) had shown the resounding hollowness of the {\sl leitmotif} of the “White Man’s Burden.” 

But it was given to Rajiv Malhotra, a leading public intellectual in America, to expose the Western conspiracy on an unprecedented scale, unearthing the {\sl modus operandi} behind the unrelenting and unhindered program for nearly two centuries now of the sabotage of our ancient civilisation yet with hardly any note of compunction.  One has only to look into Malhotra’s seminal writings -- {\sl Breaking India} (2011), {\sl Being Different} (2011), {\sl Indra’s Net} (2014), {\sl The Battle for Sanskrit} (2016), and {\sl The Academic Hinduphobia} (2016) -- for fuller details.
\vskip 1.5pt

This pentad -- preceded by {\sl Invading the Sacred} (2007) behind which, too, he was the main driving force -- goes to show the intellectual penetration of the West, into even the remotest corners (spatial/temporal/\break thematic) of our hoary heritage. There is a mixed motive in the latest Occidental enterprise,  ostensibly being carried out with pure academic concerns. For the American Orientalist doing his ``South Asian Studies'' (his new term for “Indology Studies”), Sanskrit is inherently oppressive -- especially of Dalits, Muslims and women; and as an antidote, therefore, the goal of Sanskrit studies henceforth should be, according to him, to ``exhume and exorcise the barbarism'' of social hierarchies and oppression of women happening ever since the inception of Sanskrit -- which language itself came, rather, from outside India. Another important agenda is to infuse/intensify animosities between/among votaries of Sanskrit and votaries of vernacular languages in india. A significant instrument towards this end is to influence mainstream media so that the populace is constantly fed ideas inimical to the Hindu heritage. The tools being deployed for this are the trained army of “intellectuals” -- of leftist leanings and “secular” credentials.
\vskip 1.5pt

Infinity Foundation (IF), the brainchild of Rajiv Malhotra, started 25 years ago in the US, spearheaded the movement of unmasking the “catholicity” (- and what a euphemistic word it is!) of Western academia. The profound insights provided by the ideas of ``Digestion''\index{digestion@``Digestion''} and the “U-Turn Theory” propounded by him remain unparalleled.
\vskip 1.5pt

It goes without saying that it is {\sl ultimately the Hindus in India who ought to be the real caretakers of their own heritage}; and with this end in view, {\bf Infinity Foundation India (IFI)} was started in India in 2016. IFI has been holding a series of Swadeshi Indology Conferences. 
\vskip 1.5pt

Held twice a year on an average, these conferences focus on select themes and even select Indologists of the West (sometimes of even~the East), and seek to offer refutations of mischievous  and misleading misreportages/misinterpretations bounteously brought out by these Indologists -- by way of either raising red flags at, or giving intellectual responses to, malfeasances inspired in fine by them. To employ Sanskrit terminology, the typical secessionist misrepresentations presented by the West are treated here as {\sl pūrva-pakṣa}, and our own responses/rebuttals/rectifications as {\sl uttara-pakṣa} or {\sl siddhānta}. 

The first two conferences focussed on the writings of Prof.\ Sheldon Pollock, the outstanding American Orientalist (also of Columbia University, ironically) and considered the most formidable and influential scholar of today. There can always be deeper/stronger responses than the ones that have been presented in these two conferences, or more insightful perspectives; future conferences, therefore, could also be open in general to papers on themes of prior conferences.
\bigskip

\noindent
Vijayadaśamī\hfill	{\bf Dr.~K S Kannan}\\
Hemalamba Saṁvatsara\hfill Academic Director\\
Date 30-09-2017\hfill and\\	
\phantom{.}~\hfill General Editor of the Series
            

\chapter*{Volume Editorial}\label{editorial}

\lhead[\small\thepage\quad K S Kannan]{}
\rhead[]{Volume Editorial\quad\small\thepage}


\begin{center}
\begin{tabular}{l}
\textsl{“artho’sti cen, na pada-śuddhir, athāsti sāpi}\\
\textsl{\qquad no rītir asti, yadi sā ghaṭanā kutastyā,?} |\\
\textsl{sāpy asti cen, na nava-vakra-gatis, tadetad}\\
\textsl{\qquad {\bfseries vyarthaṁ vinā rasam} -- aho gahanaṁ kavitvam!} ||”
\end{tabular}
\end{center}

A poem may have a good idea (artha), but the words therein may not be grammatically sound. It may have even this, but it may lack style (rīti). Given even this, the work may not have a proper organisation of its contents (ghaṭanā). Assuming even that, it may not be equipped with new tropes. Should that be there too, it would still be {\bf a waste if the poem is devoid of rasa} -- oh, how tough the art of poetry is!


As has been indicated in the Series Editorial, and in the Volume Editorials of the earlier volumes, Western Indology has steadily endeavoured for two centuries (and with a great deal of success) to take full control of Indic studies. Alaṅkāra-śāstra\index{alankarasastra@\textsl{alaṅkara-śāstra}} (the discipline in Sanskrit that studies the very concept of literature in its origins as well as effects) has been flourishing in India easily for over two thousand years, and the Rasa Theory\index{Rasa Theory@\textsl{Rasa} Theory} propounded by this \textsl{śāstra}, with greater and greater ramifications and clarifications through centuries, has much to contribute towards many issues in modern psychology and poetics. The fanatic votaries of Euro-centrism would of course continue either to trace everything to Greece, or proclaim that these ideas have little relevance to the present day, after all.

Prof. Sheldon Pollock\index{Pollock, Sheldon} has thus sought to show that the Theory of \textsl{Rasa} has lost its utility and is of no importance or relevance to the current complex developments in the fields of psychology/rhetorics. This volume, with contributions from over half a dozen authors, is devoted to show that his contentions have no real foundation in facts.

A synoptic view of the various papers in the volume is quite in order here.

The paper by {\bf Naresh Cuntoor (Ch.1)} entitled {\bf“\textsl{Rasa} Theory: Changes and Growth”} explores the history of the Theory of \textsl{rasa}\index{rasa@\textsl{rasa}} which has been studied under the formalisms of Mīmāṁsā, Vedānta, and Bhakti traditions. The different formalisms sensitise us to different aspects of the theory. Pollock’s perspectives on \textsl{Rasa} Theory are first provided, followed by an outline of related studies in cognitive and computational linguistics.\index{computational linguistics} Pollock’s perception of the evolution of the \textsl{Rasa} Theory\index{Rasa Theory@\textsl{Rasa} Theory} is based on the differentiation of literature seen and literature heard, and the application of the Theory as pertinent to the former, to the latter.

Even though Cuntoor remarks that “the final blow” to the existing notions of \textsl{rasa} expression, spoken of by Pollock\index{Pollock, Sheldon} as a valuable insight, it must be noted that T.N. Srikanthiah\index{Sreekantaiyya, T. N.}\footnote[1]{Sreekantaiyya, T.N. (1953). \textsl{Bhāratīya Kāvya Mīmāṁse}. Mysore: Mysore University.} (1953) has already stated this in more than one place in his immortal work (Sreekantaiyya 1953:23, 34, 24ff, 321). One may indeed make a comparative study of T. N. Sreekantaiyya (1953) and Pollock (2016).

Again, Pollock's statement that “Śrī Śaṅkuka\index{Sankuka@Śaṅkuka} was the first to argue from the spectator's point of view” is also a point noted by Sreekantaiyya (1953), who notes the issue as “the most important question”. Further the key significance of \textsl{citra-turaga-nyāya} as applicable to art in general itself was also noted by Sreekantaiyya (1953), (with a further note in the footnote that this is what shows the relationship between God's creation and the artist's creation). Cuntoor remarks that such an application across art disciplines is more striking than application across two forms of literature such as drama and poetry.

While discussing Bhaṭṭa Nāyaka,\index{Bhattanayaka@Bhaṭṭa Nāyaka} Pollock does not mention, Cuntoor notes, the Mīmāṁsā framework used in grammar by Bhaṭṭoji Dīkṣita.\index{Bhattojidiksita@Bhaṭṭoji Dīkṣita} Pollock's accusation -- that Abhinavagupta is an ungrateful disciple of Bhaṭṭa Nāyaka -- is quite unfair; for, Abhinavagupta has clearly stated that he has “seldom attacked the schools of thought of the noble [scholars that preceded him], but on the other hand, they have only been refined (\textsl{śodhita}).” \textsl{“tasmāt satām atra na dūṣitāni / matāni tāny eva tu śodhitāni”.}

Cuntoor is careful not to “infer modern scientific notions from ancient knowledge, or assert that ancient Indians discovered everything before modern science”. His motivation is to see if we can “gain new insights into \textsl{Rasa} Theory\index{Rasa Theory@\textsl{Rasa} Theory} using the perspectives of modern notions of cognitive and computational models”. Cuntoor raises the question, for example, as to whether the framework of multiple memory systems can be used to gain a better understanding of the types of \textsl{bhāva}-s.\index{bhava@\textsl{bhāva}} Also to be investigated is -- whether \textsl{Rasa} Theory could provide new principles of perceptual organisation in the context of experiencing literature; whether the study of mirror neurons in the context of imitation, self-identity and empathy can have a bearing on ideas pertaining to \textsl{karuṇa-rasa}.\index{rasa@\textsl{rasa}}\index{karunarasa@\textsl{karuṇarasa}}\index{rasa@\textsl{rasa}!\textsl{karuṇa}} Pollock's\index{Pollock, Sheldon} contentions -- of the unnaturalness of pity in man, and of his supposition of compassion as a Buddhist invention -- need also be be scrutinised. Computational aesthetics, dealing with sentiment analysis and emotion recognition, can also be tried for recognising \textsl{rasa} in literature. The technique of reductionism may perhaps be tested to its limits in \textsl{Rasa} Theory\index{Rasa Theory@\textsl{Rasa} Theory} especially. Cuntoor also refers to the absence of a detailed discussion in Pollock on \textsl{aucitya},\index{aucitya@\textsl{aucitya}} which constitutes, as Ānandavardhana\index{Anandavardhana@Ānandavardhana} says, the \textsl{parā upaniṣad} (supreme secret) of \textsl{rasa}.\index{rasa@\textsl{rasa}}

The second paper written by {\bf Ashay Naik (Ch. 2)} is on {\bf Desacralisation\index{misinterpretation!techniques of!desacralisation} of the Indian} {\sl\bfseries Rasa} {\bf Tradition}. Profanation verily may well be described as the singular agenda of Pollock, and he is accordingly on a fissiparous overdrive. Tradition linked \textsl{rasa}, the poetic relish, with the Upaniṣadic \textsl{rasa}; and presented \textsl{kāvya} as but an allotrope of the Veda inasmuch as \textsl{kāvya}\index{kavya@\textsl{kāvya}} being a \textsl{kāntā-sammita}\index{kantasammita@\textsl{kāntāsammita}} (\textsl{à la} a beloved) is kindred in spirit to the Veda which is \textsl{prabhu-sammita}\index{prabhusammita@\textsl{prabhusammita}} (\textsl{à la} a king) -- both thus seeking to subserve certain common purposes. But Pollock is frantic to drive a wedge between the Veda and the \textsl{kāvya}. Bitten by the reductivity bug, Pollock can perceive \textsl{kāvya} only as a socio-political aesthetic, divested of its religio-spiritual dimensions. And so this “Last Sanskrit Pandit” (as his hagiographers hail him) aims his arrows against Abhinavagupta,\index{Abhinavagupta} attempting to sabotage his status in the realm of Indian aesthetics. It is not Pollock's failure that arouses our pity, but his audacity. Pitting Bhoja\index{Bhoja} or Bhaṭṭa Nāyaka\index{Bhattanayaka@Bhaṭṭa Nāyaka} against Abhinavagupta betrays Pollock as a strategist, but also ultimately betrays Pollock himself. \textsl{Praśasti}-s of his own patrons notwithstanding, Pollock dutifully if brazenly attacks the \textsl{praśasti}-writers. Desacralising \textsl{Rasa} Theory thus on the one hand, and pressing Indian aesthetics to subserve Christian propaganda on the other, are but two sides of the same coin.

Speaking of Veda-s as no poetry; portrayal of the \textsl{Rāmāyaṇa}\index{Ramayana@\textsl{Rāmāyaṇa}} as essentially political in character; attempting a dichotomy between the Veda\index{Veda-s@\textsl{Veda}-s} and the \textsl{kāvya}; undermining orality of the \textsl{Rāmāyaṇa} so as to suit a late dating of the text; positing a consubstantiality of \textsl{kāvya} and \textsl{praśasti};\index{prasasti@\textsl{praśasti}} subtle sabotage of Ingalls's admonition to the Western critics of Eastern poetry; valorising Bhoja and Bhaṭṭa Nāyaka at the cost of Abhinavagupta; reading ideas of social pragmatics into the most innocent of situations; concoction of a “theological \textsl{turn}” in literary theory; projecting discrepancies with and breaches in, the tradition; positing the aim of \textsl{kāvya} as the creation of “\textsl{politically correct} subjects and subjectivities''; attributing the genesis of a “spiritualised Indian aesthetic” to royal depradations and kindred social contexts; speaking melodramatically of "an episteme that Abhinava\index{Abhinavagupta} successfully overthrew”; effectively tweaking truths subtly and ably, distorting meaning thereby; localising \textsl{rasa}\index{rasa@\textsl{rasa}} in the text, rather than in the reader; implying that Western intervention is necessary to rewrite a true history of Indian aesthetics; preferring to speak of \textsl{rasa} as a linguistic modality rather than a psychological modality; valorising a sociological hermeneutics so as to render it amenable to Marxist pigeonholing and reinterpretation etc --- are all but ploys of Pollock to usher in his own brand of Orientalism. Ashay also makes a reference to the sinister Christianisation of Bharatanāṭyam\index{Bharatanatya@\textsl{Bharatanāṭya}} and Indian aesthetics aimed at spreading the gospel of Jesus.

{\bf K Gopinath (Ch. 3)} having the caption {\bf “Towards a Computational Theory of {\sl\bfseries Rasa},} takes on squarely the contention of Pollock\index{Pollock, Sheldon} -- that Indian thinkers have neither attempted a robust theory for creativity, nor did they have a theory across \textsl{kalā}-s. Gopinath sketches a computationally inspired Theory of \textsl{Rasa}\index{Rasa Theory@\textsl{Rasa} Theory} (which is still in progress, he notes) throwing light on Indic insights in support of the theory, and buttressed by a few art forms. Pollock also complains about the absence of a settled terminology pertaining to \textsl{kāvya,\index{kavya@\textsl{kāvya}} nāṭya} and \textsl{saṅgīta}, as also \textsl{citra, pusta} and architecture, and the other \textsl{kalā}-s. Gopinath shows at the outset that the rendering of the word \textsl{pratibhā} as creativity or genius is poor, and “flash of insight” would indeed be a better one, citing verses in support from \textsl{Vākyapadīya};\index{Vakyapadiya@\textsl{Vākyapadīya}} (the same is also demonstrated in the 1923 paper (on the very key word) of Late Gopinath Kaviraj).\index{Gopinath Kaviraj}

Prof. Gopinath adds Abhinavagupta’s\index{Abhinavagupta} statement also to that effect. Gopinath adduces the testimonies of Mukund Lath,\index{Lath, Mukund} Kapila Vatsyayan,\index{Vatsyayan, Kapila} Dr.\@ V Raghavan\index{Raghavan, V.}, Manomohan Ghosh\index{Ghosh, Manomohan} and Sylvan Levi\index{Levi, Sylvan} to show the common origin or common essence, or common terminology that encompasses these. The testimonies of \textsl{Viṣṇu Dharmottara Purāṇa}\index{Purana@Purāṇa!Viṣṇudharmottara}\index{Visnudharmottara Purana@Viṣṇudharmottara Purāṇa} and Mallinātha\index{Mallinatha@Mallinātha} (the famed commentator) are also brought to bear on these issues. The academic temerity of Pollock in boldly making false statements --- as when he says God in India was generally not an artist -- is countered by the mention of the musical associations of the divinities viz. Kṛṣṇa, Sarasvatī, Nārada, Hanumān and so on. (\textsl{Saṅgīta-ratnākara}\index{Sangitaratnakara@\textsl{Saṅgītaratnākara}} has even categorical statements, in the very opening chapter, not noticed by Pollock; the most superficial glance at either Hindu sculpture or pages of Hindu mythology could have opened the purblind eyes of this critic). After all, Pollock\index{Pollock, Sheldon} has himself translated the \textsl{Rāmāyaṇa},\index{Ramayana@\textsl{Rāmāyaṇa}} and asserted Rāma’s divinity, and yet fails to note that Rāma knew music too very well: surprising; or rather, nothing so surprising.

Coming to the written text versus the oral text argument, the obsession of the West with the former, and its futility, are set forth by Gopinath by invoking the statements of stalwarts such as Vasudha,\index{Narayanan, Vasudha} Coward,\index{Coward, H. G.} Kunjunni Raja\index{Raja, Kunjunni} and others.

There is also the “intangibility” of \textsl{rasa},\index{rasa@\textsl{rasa}} as reflected, for example, in the very nomenclature of a particular type of \textsl{dhvani}\index{dhvani@\textsl{dhvani}} as \textsl{asaṁlakṣya-krama} (“of imperceptible sequence”). Gopinath hits the nail on the head when he indicates the essential complexity involved in the signification of \textsl{rasa}: \textsl{rasa} can be seen abstractly as a certain mapping of a text, performance or artefact, from a creator/actor, through a medium onto a receiver, and semantics is involved in addition to the affective part of \textsl{rasa} itself. He draws an effective analogy from science --- of the protein folding that is a complex function of a linear DNA structure, whence the message may be a complex function of the linear atomic units, but possibly without a deterministic mapping. To draw a parallel from another domain of art, the \textsl{svara}\index{svara@\textsl{svara}} to \textsl{rasa} mapping is non-trivial and may be probabilistic too. The \textsl{svara} arrangements and shapes are huge --- like the innumerable proteins; and it may not be impossible to construct a finite automaton to characterise \textsl{rāga}-s.\index{raga@\textsl{rāga}} Their ascending and descending scales defined though, yet there are factors that spell probabilistic conditions and subjective characterisations. Looking into Hidden Markov Models with possibilities of hybridisation and crossover and transpositions can show the burgeoning possibilities. It is no coincidence that the temporo-parietal junction, the location of self-referential activity in the brain, is also the region involved in musical experience. It is certainly not the case that the neuro-correlates in such instances have all been worked out yet.

Pollock’s claims of noncommonality across departments of arts are not well-substantiated; and substantiations to the contrary are available even if not very extensive and very detailed. Gopinath provides textual support, as from \textsl{Citrasūtra}\index{Citrasutra@\textsl{Citra-sūtra}} as to how there is an inextricable relationship between and amongst the different disciplines such as sculpture, painting, dance, and music --- instrumental as well as vocal on the one hand, and classical and popular on the other. Stella Kramrisch\index{Kramrisch, Stella} records the mappings between \textsl{rasa}-s and colours; and speaks of the common basis of architecture, sculpture and painting. Analogies obtain even in the philosophical ramifications across fields like Vyākaraṇa,\index{Vyakarana@Vyākaraṇa} Alaṅkāraśāstra\index{alankarasastra@\textsl{alaṅkara-śāstra}} and the Pratyabhijñā\index{Pratyabhijna@Pratyabhijñā} schools. All texts on \textsl{nāṭya} discuss the mind-body coordination and correlation. The traditional analogy of the seed and the tree with its flowers and fruits -- indicates the relationship between the various limbs of dance. The multiplicity of inputs generates a richly textured and emotionally resonant experience which is larger than the sum of its parts, as Logan Beitmen\index{Beitman, Logan} elaborates. The intimate relation between \textsl{rasa}-s,\index{rasa@\textsl{rasa}} \textsl{sthāyi-bhāva}-s\index{sthayibhava@\textsl{sthāyibhāva}} and \textsl{sañcāri-bhāva}-s\index{sancaribhava@\textsl{sañcāribhāva}} on the one hand; and the physical expression of emotions and even some psychological disorders on the other -- are worth noting. The objectivity in the taxonomy of the various \textsl{rasa}-s is borne out by the fact that they are so well corroborated by what modern psychology has identified as basic emotions.

A computational cum cognitive analysis of rasa would involve the generative and recognitive aspects. The creator and the spectator have their tasks allocated to the design time and run-time respectively -- the former involving the computationally, and the latter the cognitively, structured models (even though both normally happen unconsciously). If the cognitive and computational models are fairly well-developed,  Pollock’s\index{Pollock, Sheldon} charges can be shown to be laden with negative biases, despite his exhibition of erudition and advertised appreciation of a few aspects of Indic arts here and there.

The Indic tradition has always evinced a clear distinction between an actual emotion, and a same emotion experienced via \textsl{nāṭya}. Any system built on a finite set of rules necessarily involves iteration and recursion --- alike applicable to microscopic and macroscopic entities. The model of the Indra’s Net\index{Indra's Net} employed by the Atharvan seer (or the Buddhist sage) -- as set forth in Rajiv Malhotra’s\index{Malhotra, Rajiv} book bearing the same title -- is a telling case in point. (The very acts of recursion and reiteration after a quantitative threshold, impart upon the structure an unexpectable and inexplicable qualitative leap of sensation and perception).

Simulation of real emotions and iteration of particular patterns induce the dominant \textsl{rasa} and the subordinate \textsl{rasa}\index{rasa@\textsl{rasa}} -- mediated and spurred as they also are by memory traces and \textsl{dhvani}\index{dhvani@\textsl{dhvani}} excitations that get richer and richer -- and go in fine to trigger even affective impulses. \textsl{Sādhāraṇīkaraṇa},\index{sadharanikarana@\textsl{sādhāraṇīkaraṇa}} the Generalization, that then takes effect subtly removes the self-interest of the spectator, which removal alone activates the \textsl{rasa}-spring. Patterns of iteration and recursion generate anticipation of substructures, and thereby conduce to greater enjoyment. While on the one hand the artist of each kind is expected to acquaint himself well with many other departments of art, he also has the choice to generate new patterns\index{patterns!iteration} on the nonce (a not-easily-imagined blend of abundant constraints with yet more abundant freedom).

The art of the stage involves triple levels --- the Third Person experience (of the viewer), the Second Person enactment (by the actor), and the First Person thought (by the author), (and hence the schema here ought to be much more complex than what Pāṇini\index{Panini@Pāṇini} attempted which involves only double levels --- the speaker’s and the hearer’s). The actor and the spectator each loses his identity but in different ways. Given the complexities involved, mathematical modelling involving axiomatics may not constitute an apposite approach, and a computational  model may be nearer to the real issues. Such a model may involve generative aspects and descriptive aspects, and a particular sensitivity to Indic sensibilities. The Indic perspective looks even into ontogenic aspects (as with the Piṇḍotpatti Prakaraṇa in \textsl{Saṅgīta Ratnākara}\index{Sangitaratnakara@\textsl{Saṅgītaratnākara}} involving embryological studies), or the \textsl{sandhi} aspects in Pāṇini (involving the anatomical structures of the sound-producing organs), and the leap from the “atomic” \textsl{svara}-s to a \textsl{rāga} endowed with a “personality” of its own. An element of synaesthesis involves in the correlation of \textsl{rasa}-s and colours. Apart from sets of \textsl{sthāyi-bhāva}-s\index{sthayibhava@\textsl{sthāyibhāva}} and \textsl{rasa}-s, each eight in \textsl{Nāṭyaśāstra},\index{Natyasastra@\textsl{Nāṭyaśāstra}} there are the sets of eight \textsl{sāttvika-bhāva}-s\index{sattvikabhava@\textsl{sāttvika-bhāva}} and 33 \textsl{sañcāri-bhāva}-s\index{sancaribhava@\textsl{sañcāribhāva}} with many-to-many mappings in between. Indian art revels in the profusion of the interplay of \textsl{vyañjanā}-s,\index{vyanjana@\textsl{vyañjanā}} rather than in the reductive, fixed-and-formed entities. None, else than Hindus, excelled in extreme digitisation, as also in extreme integration, (but note on the other hand that mindless proliferation of terminology is an illness that besets modern linguistics, as Dwight Bolinger\index{Bolinger, Dwight} once noted). The magnificent juxtaposition of linguistics and music on a phenomenological basis was provided by Māgha\index{Magha@Māgha} long ago (\textsl{anantā vāṅmayasyāho geyasyeva vicitratā!}).
 
Scientists are open to the suggestion that there is a connection between the brain's biomolecular processes and the basic structure of the universe. The primacy of the sentence (in grammar) though it is constituted of its own components of diverse patterns, and the primacy of \textsl{rasa}\index{rasa@\textsl{rasa}} (in Sāhityaśāstra) though it issues out of certain combinations of its various constituent factors --- in other words of the integrality of the higher despite apparent decomposability into numerous intermediary/terminal nodes --- is an extraordinary contribution of the Hindu mind. The top-down and bottom-up approaches are looked into, and their optimisations are also worked out --- as in the two schools of Mīmāṁsā --- in the contexts as of Anvitābhidhāna-vāda\index{anvitabhidhanavada@\textsl{anvitābhidhāna-vāda}} and Abhihitānvaya-vāda.\index{abhihitanvayavada@\textsl{abhihitānvaya-vāda}}

In a given passage, there may be no element (noun or verb, adjective or adverb, or even a particle) that may not be suggestive; and even so, in a performance there may be no element (word or song, \textsl{mudrā}\index{mudra@\textsl{mudrā}} or aspect of dress etc.) that may not conduce to a particular \textsl{rasa}.\index{rasa@\textsl{rasa}}
 The \textsl{gestalt} of sense first generated by the components and subcomponents (words/phrases/clauses) of items in a sentence, and the \textsl{gestalt} of \textsl{dhvani} produced by the senses of the three types of meaning (viz. \textsl{abhidhā,\index{abhidha@\textsl{abhidhā}} lakṣaṇā},\index{laksana@\textsl{lakṣaṇā}} and \textsl{vyañjanā}),\index{vyanjana@\textsl{vyañjanā}} are presumably analogous.

Rhythms and mathematical regularities occurring in performances in sounds/metrics/gestures etc. can create a vibrational sense for the audience. What came in handy for the Hindu poets/aesthetes is the early mastery (circa 5th century B.C.E) of the requisite mathematical notions as of the Pascal’s triangle,\index{Pascal's triangle} binary computations and Fibonacci\index{Fibonacci series} series -- applicable to different realms. Even the concept of \textsl{anu-raṇana}, [re-]echoing, came to be exploited even in the nomenclature of \textsl{dhvani}\index{dhvani@\textsl{dhvani}} types. 

As to the general schema in regard to music (extensible perhaps to other arts), Rowell\index{Rowell, Lewis} says well: “A hallmark of the early Indian way of thinking about music was to identify and name all possible permutations of the basic elements, but with the realisation that only certain authorised (and far more specific) melodic constructions can become the basis for actualised music ... It was the job of the theory to provide the widest selection of possibilities, but it remained for practice to select the most pleasing of these arrangements...”.

Indian texts have also worked out many \textsl{rāga-svara}\index{raga@\textsl{rāga}} mappings, and \textsl{rasa-rāga}\index{rasa@\textsl{rasa}} mappings and even \textsl{rasa-tāla}\index{tala@\textsl{tāla}} mappings. Amazing feats in various fora -- in the realms of prosody (in metrical compositions in Sanskrit); in the \textsl{vikṛti}-s\index{vikrti@\textsl{vikṛti}} in Vedic chanting; in the various \textsl{bandha}-s\index{bandha@\textsl{bandha}} in \textsl{citra-kāvya}-s; in the construction of cryptic mnemonic verses; in the \textsl{kaṭapayādi}\index{katapayadi@\textsl{kaṭapayādi}} encoding in \textsl{rāga}-nomeclature in music; in the pyramid-like or other structures erected on foundations of odd or fractional beats in percussion instruments; in the fractal constructions in architecture; in the  astronomical rhythms captured in temple architecture; in the design formulae in Śrīcakra\index{Sricakra@Śrīcakra} or \textsl{maṇḍala}-s\index{mandala@\textsl{maṇḍala}} etc -- all betray complex mathematical patterns,\index{patterns!mathematical} progressions and symmetries that arouse a sense of wonderment at once in the mind of the lay as well as the accomplished artist and the mathematician as well. They also clearly indicate certain recurring motifs and techniques in various domains of art --- quite contrary to the biased and unsubstantiated hence irrational proclamations of polymath-pandits of the likes of Pollock.\index{Pollock, Sheldon} Hindu temples, the point of convergence indeed of all Indic arts, verily depict an evolving cosmos of growing complexity which is self-replicating, self-generating, self-similar, and dynamic; the procedures therein are recursive and generate visually complex shape from simple initial shapes through successive application of the production rules that are similar to rules for generating fractals.\index{fractals}

Wonder may be the beginning point -- for the Westerner, for all science; as for the Hindu, wonder is also the end-point of many investigations in art which also course through various sciences, (especially the Queen of Sciences). Marvel then, at the beginnings and elements of Hindu culture, and marvel again at the many peaks and consummations of Indian art – from the very design of the alphabets to productions such as the \textsl{Gītagovinda}\index{Gitagovinda@\textsl{Gītagovinda}} of Jayadeva\index{Jayadeva} or the icon of Naṭarāja, to cite but two examples --- the multi-storeyed semantics of which must all be beyond the ken of the intellectually impoverished nothing-morists.

The next chapter is from the pen of {\bf Charu Uppal} {\bf(Ch. 4)} and is entitled {\bf “From \textsl{Nāṭyaśāstra} to Bollywood''}\index{Bollywood}. The paper goes to challenge Pollock’s\index{Pollock, Sheldon} reading of the \textsl{Nāṭyaśāstra}\index{Natyasastra@\textsl{Nāṭyaśāstra}} as being rigid and frozen in time, and allowing little scope for novelty -- which features render it irrelevant to the present day context (and by implication, to the future). She is also concerned to show that even pre-Christian Greek drama had a concept of \textsl{heiropraxis}.

Whereas the Indian tradition has all along been a blend of the \textsl{laukika} and the \textsl{alaukika}, the mundane and the transcendent, Pollock\index{Pollock, Sheldon} attempts to divest it of the latter, and hence is ineligible to be an authentic interpreter of the tradition, for all his vaunted scholarship. The inappropriateness of his application of the Marxist theory of aestheticization of power and the false picture he portrays -- one of the numbing the masses into obedience by deployment of oppressive Vedic ideas --- is something that goes against the dictum of his own “preceptor” Daniel Ingalls.\index{Ingalls, Daniel H. H.} The very purpose of \textsl{Nāṭyaśāstra},\index{Natyasastra@\textsl{Nāṭyaśāstra}} as of the \textsl{Mahābhārata},\index{Mahabharata@\textsl{Mahābhārata}} is to make available to the common man the precious Vedic verities which are not easily accessible even to scholars, often. Pollock invokes chronology and authorship issues to subserve his goals, dragging the dates of ancient texts as nigh as possible to our own. Countering Gerow and Pollock, she cites V S Ramachandran\index{Ramachandran, V. S.} who speaks of artistic universals. Uppal draws attention to the role played by \textsl{rasa} in Bollywood even to this day.

The paper by {\bf Sreejit Datta (Ch. 5)} entitled {\bf “From Rasa Seen to Rasa Heard”: A Criticism}, takes a close look at Pollock’s depiction of the evolution of the idea of \textsl{rasa}.\index{rasa@\textsl{rasa}} Datta questions the very differentiation between “literature seen” and “literature heard” that Pollock starts with. He explores how literature as a Western category and \textsl{sāhitya}\index{sahitya@\textsl{sāhitya}} as an Indian category differ. Pollock’s “Rasa Seen to Rasa Heard” is essentially an exercise, he says, in peddling Western Universalism. The very etymology seems to hint at something of the nature of their content: \textsl{Literature} from Latin \textsl{“litteratura”} is something written or something pertaining to learning; whereas \textsl{sāhitya} implies a blend or fusion indicative of an integrality. The \textsl{Nāṭyaśāstra}\index{Natyasastra@\textsl{Nāṭyaśāstra}} speaks of what the gods told Brahmā --- that they want something which is \textsl{dṛśya}\index{drsyakavya@\textsl{dṛśya-kāvya}} as well as \textsl{śravya}.\index{sravyakavya@\textsl{śravya-kāvya}}

Datta also draws our attention to Pollock’s resort to translation of all technical terms in Sanskrit into English, which is tantamount to\break epistemological domination of one culture by another as indicated by\break Vazquez.\index{Vazquez, Rolando } To translate \textsl{Dhvanyāloka}\index{Dhvanyaloka@\textsl{Dhvanyāloka}} as “Light on Implicature” is atrocious. The very individuality of the original words is totally lost in the translations -- dilution and disfigurement being the invariable consequences. Much earlier (1950) Manomohan\index{Ghosh, Manomohan} Ghosh had been careful enough to provide the Sanskrit term also and with a capitalisation of the first letter of the English rendering “lest these should be taken in their usual English sense”. Recitation of the \hbox{Veda-s}, eminently the \textsl{śravya},\index{sravyakavya@\textsl{śravya-kāvya}} is also enjoined to be accompanied by \hbox{\textsl{mudrā}-s} (gestures); and the four \textsl{vṛtti}-s are related to the four Veda-s --- all emphasising once again the link between the \textsl{dṛśya}\index{drsyakavya@\textsl{dṛśya-kāvya}} and the \textsl{śravya} aspects. Kapila Vatsyayan\index{Vatsyayan, Kapila} also clarifies that the various arts are not to be referred to in isolation or in mutual exclusiveness. The sonic and deific forms of the \textsl{rāga}-s\index{raga@\textsl{rāga}} are set forth together by Somanātha\index{Somanatha@Somanātha} in his \textsl{Rāgavibodha}\index{Ragavibodha@\textsl{Rāgavibodha}} (17th century C.E.); the former being \textsl{śravya}, and the latter, \textsl{dṛśya}. To see schism where none exists, or create one where only subtle differences are shown -- is all a part of the fissiparous agenda of the West.\\[-18pt]
\begin{quote}
\textsl{ye sadartham ajānanto}\\
\textsl{vṛthā vacana-vistaraiḥ} |\\
\textsl{dūṣayanti kaveḥ kāvyaṁ}\\
\textsl{dhik tān paṇḍitamāninaḥ} ||
\end{quote}
[Fie upon the self-styled scholars who vitiate a poet's composition through their verbiage without first comprehending the good sense therein]

The paper by {\bf Shankar Rajaraman (Ch.\@ 6)} is entitled {\bf “A Critical Examination of Western Indologists’ Engagement with Sanskrit Poetic Texts”} (in the context of Translation, Editing and Analysis). Western scholarship has its own shortcomings, and the biggest of it all is, undoubtedly, its abundant prejudice: it considers its duty to be spiteful of all other civilisations, and is eminently capable of overt and covert arm-twisting. Plus, its scholarship is not always sound and unquestionable. Shankar examines in this paper a score of cases of mistranslations and cases of faulty editing and misanalysis. 

Rather than making a mere catalogue of Westerners’ errors, Shankar has classified them --- tracing them to their causes, making use of Rajiv Malhotra’s\index{Malhotra, Rajiv} four-tier model of critiquing Western Indology. He seeks to demonstrate how traditional scholarship in Sanskrit can equip one with sound analytical tools that help in detecting instances where there is inherent misunderstanding of texts. Western Indologists can be accused of not one or two errors. Shankar presents a classified list of their blunders such as --- getting the narrative wrong; non-familiarity with Indian ethos; non-familiarity with complementary bodies of knowledge; getting the semantics wrong at all possible levels --- of unitary words, compound-words, and phrases/sentences, and even failing to spot puns, (single or multiple, but as are vital for the appreciation of the verse) etc. Some of the Western translators have been blenders of these blunders --- providing unintended, unexpected, and unlimited entertainment to discerning readers. 

Literary narratives are characterised by features, one or more, of coherence, meaningfulness and emotional import, and these translators can err on all counts. Shankar illustrates mistranslations all from the CSL\index{Clay Sanskrit Library} (Clay Sanskrit Library) publication series involving big figures in Indology such as Sheldon Pollock\index{Pollock, Sheldon} (General Editor), Yigal Bronner,\index{Bronner, Yigal} Wendy Doniger,\index{Doniger, Wendy} David Shulman,\index{Schulman, David} and Gary Tubb.\index{Tubb, Garry} He has shown how Friedhelm Hardy\index{Hardy, Friedhelm} has erred in missing out on the \textsl{anvaya} of a verse from \textsl{Āryāsaptaśatī}\index{Aryasaptasati@\textsl{Āryāsaptaśatī}} of Govardhanācārya.\index{Govardhanacarya@Govardhanācārya} James Mallinson’s\index{Mallinson, Sir James} translation misses out on the sequence of events in a verse from \textsl{Pavanadūta}\index{Pavanaduta@\textsl{Pavanadūta}} of Dhoyī.\index{Dhoyi@Dhoyī} Pollock has thoroughly mistranslated a verse from \textsl{Rasataraṅgiṇī};\index{Rasatarangini@\textsl{Rasataraṅgiṇī}} he has confused trees with mountains. All the adjectival translations, therefore, have gone wrong, and so, Pollock’s lack of cultural understanding shows itself clearly. He has made many silly mistakes including translating a \textsl{lyabantāvyaya} as if it were a \textsl{tumunnantāvyaya}! In yet another verse from \textsl{Rasamañjarī},\index{Rasamanjari@\textsl{Rasamañjarī}} Pollock has mistranslated the verb itself and advertises his ignorance of what Sanskrit poets are wont to represent in the given circumstances.

The Notes added to the translations are usually meant to help readers understand the verse and its cultural context the better. In spite of a Sanskrit commentary giving the correct explanation, Wendy Doniger\index{Doniger, Wendy}
 boldly mistranslates the verse, and in effect, converts an altruistic king into a selfish one. One has to show utmost care while rendering the \textsl{nāndī} verse of a play as it is often intended to be suggestive. Wendy Doniger brazenly mistranslates the \textsl{nāndī} verse of the play \textsl{Priyadarśikā},\index{Priyadarsika@\textsl{Priyadarśikā}} and trying to show off her knowledge of mythology, renders the verse in a perverse manner. And the result: Wendy Doniger’s fixations about sexual impulses can give rise to ‘shameless’ improprieties.

A verse from \textsl{Prabodha-candrodaya}\index{Prabodha-candrodaya@\textsl{Prabodha-candrodaya}} of Kṛṣṇamiśra\index{Krsnamisra@Kṛṣnamiśra} is wrongly rendered by Kapstein,\index{Kapstein, Matthew} ignorant as he is of the role of \textsl{sindūra} at the spot of the parting of the hair of a Hindu woman; wrong dissolution of a compound word also conduces to this. He has confused a pigment to be a colour term, and missed the very force of a simile, and taken a noun as an adjective. Rendering the verse is made worse by his fictitious justification which only adds colour to the blunder. A little less elequence would have helped him, but he is bent upon advertising his ignorance.

In his translation of another verse, Mallinson falters on three counts --- ignorance of the typical and significant sporting of the lotus by Goddess Lakṣmī; rendering a word by its popular sense in a context where it is used in a specific and special sense; and seeing a pun where none exists --- and thereby laying bare his lack of knowledge of the lexicon of information that is available in the opening pages of \textsl{Amarakośa} (not some rare \textsl{kosa}, to wit)!

Adding uncalled-for footnotes helps Pollock\index{Pollock, Sheldon} show off his ignorance --- showing a visiting student as a royal priest --- in the course of his translation of a verse from \textsl{Rasataraṅgiṇī}\index{Rasatarangini@\textsl{Rasataraṅgiṇī}} of Bhānudatta;\index{Bhanudatta@Bhānudatta} where a very famous context of Kālidāsa’s\index{Kalidasa@Kālidāsa} \textsl{Raghuvaṁśa}\index{Raghuvamsa@\textsl{Raghuvaṁśa}} (the most famous poet) presents itself. What should be very familiar to a student of the celebrated Kālidāsa is not familiar to Wendy Doniger\index{Doniger, Wendy} and her notes converts wine into water! She seems to be ignorant of poetic conventions available in even Apte’s\index{Apte, V. S.} \textsl{Student’s Sanskrit-English Dictionary} --- not some recondite source!.

Hardy\index{Hardy, Friedhelm} renders, in his translation of a verse from \textsl{Āryāsaptaśatī},\index{Aryasaptasati@\textsl{Āryāsaptaśatī}} the word \textsl{pradoṣa} as ‘early morning’! verily a great blunder (\textsl{pra-doṣa})! In another verse he miscopies and misrenders \textsl{tūla as tula}; he is indeed \textsl{nistula} in his carelessness! Mallinson\index{Mallinson, Sir James} makes a king out of a brahmin by misunderstanding a simple vocable. In his translation of a verse from \textsl{Anargharāghava},\index{Anargharaghava@\textsl{Anargharāghava}} Torzsok\index{Torzsok, J} makes out a lamp as a star, thereby spoiling a wonderful poetical fancy in the description of the evening twilight.

\eject

In his translation of Dhoyī’s\index{Dhoyi@Dhoyī} \textsl{Pavanadūta},\index{Pavanaduta@\textsl{Pavanadūta}} Mallinson dissolves a Karmadhāraya\index{karmadharaya@karmadhāraya} compound as though it is a Tatpuruṣa\index{tatpurusa@tatpuruṣa} compound jeopardising the meaning of the verse as a whole.

Bronner\index{Bronner, Yigal} and Shulman\index{Shulman, David} effectively spoil the very essential idea of a verse of Vedānta Deśika\index{Vedantadesika@Vedāntadeśika} by an atrocious mistranslation that destroys the very intention of the poet. Hardy mishandles a verse from \textsl{Āryāsaptaśatī} by mistranslating two words --- thereby destroying a pun, springing from his insensitivity to grammatical subtleties in Sanskrit and his ignorance of Hindu mythology.

Numerous verses in Sanskrit abound in puns and a careless translator misses them even when they are very much present, and what is worse, “sees” puns where they just are not --- all due to lack of sensibility. Pollock has missed a beautiful pun of Bhānudatta\index{Bhanudatta@Bhānudatta} where the very beauty of the poem depended upon the pun. Sanskrit poets carefully choose words that are open to pun, and the translation that leaves out the pun on some words while taking some into account in a verse would be considered an inane translation. Notes may be the place to explain puns that cannot be easily translated, but the ignorant translator makes it clear to all discerning readers in his Notes that he is just unaware of the pun. And what is worse, the pun he has missed is a pun that is much in evidence even in the early chapters, the first few pages of \textsl{Kāvyaprakāśa}\index{Kavyaprakasa@\textsl{Kāvyaprakāśa}} --- for all his vaunted scholarship of \textsl{Alaṅkāra-śāstra}.\index{alankarasastra@\textsl{alaṅkara-śāstra}} Another mistranslation by Hardy\index{Hardy, Friedhelm} is so well done as to make a verse of Govardhana\index{Govardhana} totally unintelligible, and so, can bring infamy to the original.

As to editing: careless editors give themselves away quite often. Wendy Doniger\index{Doniger, Wendy} gives a distorted text that does violence to grammar and prosody alike, while rendering a verse from \textsl{Ratnāvalī}.\index{Ratnavali@\textsl{Ratnāvalī}} The most elementary principle -- of concord between the subject of a sentence and its verb, most elementary and almost universal in character -- is grossly violated by her. The very verse format can give clues to certain common mistakes, and a little sensitivity to prosody suffices to suspect something going amiss. Prolificity in writing is no compensation for infelicity in rendering.

Many a Western Sanskritist has no habit perhaps of reading Sanskrit texts aloud, and so is liable to miss out on the metrical felicities. Insensitive as they would be to sound and rhythm, such translators are liable to be insensitive to sense also, consequently. Rash translators misread the original Sanskrit verse, and make bold to attribute boldness to the author of the original itself. Gary Tubb\index{Tubb, Garry} deems a “violation of meter” a “bold change”, and rushes to bring out the “poetic significance” of the imagined bold change by the poet. On the other hand, nowhere have Sanskrit poeticians condoned any such violation, though so rare, of prosody. Pretending to be quite sensitive to \textsl{yati} (\textsl{caesura}), some translators have read texts too critically; but the fact is that quite on the other hand, \textsl{yati} is not quite an essential feature to certain specific meters.


Thus overdoing and underdoing their tasks as translators is by no means a small lacuna on the part of these Western scholars. The discretion to be humble is far better than the indiscretion of being supercilious.

It is a tragedy that incompetent scholars make bold to translate texts beyond their ken or without care and discipline. That a whole series is vitiated by unpardonable errors in translation reflects poorly upon the editor viz. Prof. Sheldon Pollock,\index{Pollock, Sheldon} hailed by his hagiographers as “The Last Pandit”.

A poet himself, Shankar has drawn our attention to the very many pitfalls of translators --- the Western translators in particular, that beam generally with confidence. We do not know how many texts were ultimately spoilt by Westerners who tend to think that it is their prerogative to interpret any culture on Earth.\\[-21pt]
\begin{quote}
\textsl{yadi khaṁ karaṭo gatvā}\\
\textsl{sindhor upari gāyati} |\\
\textsl{tat kiṁ sa vetti gāmbhīryaṁ}\\
\textsl{ratnāni ca tadāśaye ?} ||
\end{quote}
\vskip -5pt
[Loafing in the sky over the ocean, should a crow keep cawing, would he realise on that account, the depth of the ocean below, or cognise the gems therein?] 

The last paper by {\bf R.~Ganesh (Ch.~7)} (having the last word), entitled {\bf “Rasabrahma-samarthanam”}, counters some of the ideas of Pollock presented in the introductory portion of his \textsl{Rasa Reader}. Ganesh contrasts the views and approaches of some of the modern Sanskritists against those of Prof. Hiriyanna,\index{Hiriyanna, M.} Narasimha Bhatta\index{Narasimha Bhatta, Padekallu} and Dr. D.V. Gundappa\index{Gundappa, D. V.} (all hailing from Karnataka), and Dr. Ananda K. Coomaraswamy\index{Coomaraswamy, A. K.} and V. S. Agrawala,\index{Agrawala, V. S.} celebrated authorities in the field of Indian art.

He contests at the outset the physicality of \textsl{rasa}\index{rasa@\textsl{rasa}} which cannot be reduced to mere chemical reactions in the brain. He also speaks of the cyclicity/non-linearity of linguistic and metaphysical ideas, as against the linearity of science, and rejects the idea of Bharata’s\index{Bharatamuni} \textsl{magnum opus} as representative of a field in its stage of infancy. The \textsl{Nāṭyaśāstra}\index{Natyasastra@\textsl{Nāṭyaśāstra}} is on par with the “epics”, in terms of their naturalness yet beyond the pale of ordinary imagination. After the fashion of writers on Nyāya and Vyākaraṇa,\index{Vyakarana@Vyākaraṇa} he invokes the analogy of Ayurvedic prescriptions whose value/validity is not contestable despite advancements in anatomy/physiology/biochemistry; and asserts the validity of the \textsl{Rasa} Theory irrespective of the developments in modern psychological investigation. Drawing on an analogy of universal experience as applicable to Vedānta, one may rather deny Brahman, but not the experience of \textsl{rasa}, he says. He traces the genesis of \textsl{rasa} to early Vedic literature --- ancient portions such as the Puruṣa-sūkta, Nāsadīya-sūkta and Skambha-sūkta,\index{Skambhasukta@Skambha-sūkta}\index{Nasadiyasukta@Nāsadīya-sūkta}\index{Purusasukta@Puruṣa-sūkta} as also the Upaniṣadic portions such as the Theory of Five Sheaths (\textsl{pañca-kośa}), and even the chapter on Vibhūti-yoga in the \textsl{Bhagavad-gītā}.\index{Bhagavadgita@\textsl{Bhagavadgītā}} Even as, in the depiction of M.\@ Hiriyanna, all the \textsl{darśana}-s (schools of Indian philosophy) find their culmination in Vedānta, and are not contradictory to it, even so the \textsl{Rasa} Theory is in no contradiction with the different \textsl{darśana}-s.

His focus is on the Introduction in the \textsl{Rasa Reader}, as it is that section that teems with Pollock’s key notions. He grieves Pollock’s utter ignorance of musicological works. Speaking of the applicability of the \textsl{Rasa} Theory\index{Rasa Theory@\textsl{Rasa} Theory} to other arts such as music and dance, he refers to the preponderance of “practicals” in these realms as an important reason for a lack of discussion in books and Indian rhetorics/aesthetics. Further, they pose a few problems unique to their own fields. Pollock limits the realm of \textsl{rasa} to literature --- which is unfounded. All arts originate in the mental realm of the artist and culminate in the mental realm of the connoisseur. As Coomaraswamy states well: “The end of the work of art is the same as its beginning, for its function is ... to enable the \textsl{rasika} to identify himself in the same way with the archetype of which his work of art is an image”.

Pollock has a complaint that the principle of \textsl{pratibhā}\index{pratibha@\textsl{pratibhā}} (creativity) has not been well formulated  in the Indian tradition. It is only logical that it is so, argues Ganesh, as \textsl{pratibhā} is subjective and indescribable; and it is only in respect of its consequences that one can speak of \textsl{pratibhā}. To seek the genesis of the faculty that is at the root of all arts may well be an invitation to \textsl{anavasthā}, “endless regression” --- akin to seeking the definition of Brahman. 

Similarly, it has been shown here how relish is more valuable than critical assessment. Pollock’s\index{Pollock, Sheldon} charge on the absence of a comprehensive investigation of beauty is also baseless, as it is comparable to a similar investigation of Brahman. The objection raised by Pollock in regard to not counting \textsl{vātsalya}\index{vatsalya@\textsl{vātsalya}} as a \textsl{rasa}\index{rasa@\textsl{rasa}} is nothing new. What is more important is an investgation into rasa as such, rather than an examination of the number count of \textsl{rasa}-s.

As to the issue of the locus of \textsl{rasa} in regard to whether it is in the artist or in the creator, it has been shown how even the artist enjoys his own work as a \textsl{sahṛdaya}, and the opening chapter of \textsl{Nāṭyaśāstra}\index{Natyasastra@\textsl{Nāṭyaśāstra}} testifies to the importance of the \textsl{sahṛdaya}. The argument of Pollock that Alaṅkāraśāstra\index{alankarasastra@\textsl{alaṅkara-śāstra}} is later than \textsl{Nāṭyaśāstra} is easily answered by the fact that \textsl{nāṭya}\index{natya@\textsl{nāṭya}} is itself all-encompassing -- covering all aspects in general of poetry, picture and song. Pollock is only attempting to sow seeds of discord between \textsl{dṛśya-kāvya}\index{drsyakavya@\textsl{dṛśya-kāvya}} and \textsl{śravya-kāvya}.\index{sravyakavya@\textsl{śravya-kāvya}} As to the locus of \textsl{rasa}, regarding whether it is in the work or the actor or the \textsl{sahṛdaya},\index{sahrdaya@\textsl{sahṛdaya}} it is to be noted that even the artist dons the role of a \textsl{sahṛdaya} in the process of fine-tuning the work, and in addition has the roles of \textsl{kartṛ, bhoktṛ, jñātṛ}, and \textsl{vimarśaka}. The enjoyment of the \textsl{sahṛdaya} is post-event in the case of the creation of \textsl{kāvya\index{kavya@\textsl{kāvya}}/citra/śilpa}, but concurrent/co-event in the case of the creation of \textsl{gīta/nṛtya/āśu-kavitā} (poetry \textsl{ex tempore}). While the poet exercises \textsl{kārayitrī\index{karayitripratibha@\textsl{kārayitrī pratibhā}} pratibhā} as well as \textsl{bhāvayitrī pratibhā},\index{bhavayitripratibha@\textsl{bhāvayitrī pratibhā}} the \textsl{sahṛdaya} employs only the latter; it is thus that it is \textsl{bhāvayitrī pratibhā} that is more extensive in its role. While the function of the \textsl{kavi} is but once, that of the \textsl{sahṛdaya} can be multiple times. Viewed from the matrix of the \textsl{triguṇa}-s,\index{triguna@\textsl{triguṇa}} the poet's act is impelled by \textsl{rajas}, while that of the \textsl{sahṛdaya} is permeated by \textsl{sattva}\index{sattva@\textsl{sattva}} --- and \textsl{sattva} is discernibly superior to \textsl{rajas}.\index{rajas@\textsl{rajas}}

The charge of Pollock that Hindu poetry had its origins in Buddhism is answered by the fact that there is no Buddhist poetry as such, and that the Veda is already poetic. Further, all civilisations, including prehistoric ones, have had their share of song and dance and drawing. The wealth of literature even in early Hinduism is immense -- comprising the Veda-s,\index{Veda-s@\textsl{Veda}-s} the Vedāṅga-s, and so on; whereas Buddhism has only \textsl{nivṛtti}-oriented literarture. It may be added here that Coomaraswamy had taken strong exception to the fact that a typical Westerner would exhibit a stronger affinity towards Buddhism rather than Hinduism, even though the former concerned itself predominantly with the life of the recluse, whileas Hinduism saw life in a bigger and fuller and natural compass. The indebtedness of Aśvaghoṣa to Vālmīki is not unknown either.

Pollock’s\index{Pollock, Sheldon} argument that the Rasa Theory\index{Rasa Theory@\textsl{Rasa} Theory} has been Vedānticised holds no water. Bharata traces the various \textsl{rasa}-s to the \textsl{Atharva-veda}, in which are contained the Skambha-sūkta and the Ucchiṣṭa-sūkta\index{Skambhasukta@Skambha-sūkta}\index{Ucchisthasukta@Ucchiṣṭha-sūkta} which are permeated by poetic content. Coomaraswamy\index{Coomaraswamy, A. K.} has dwelt on these \textsl{sūkta}-s in significant detail. Pollock’s diatribe against \textsl{aucitya} after hailing its merits bespeaks rather of the maxim of \textsl{alaṅkṛta-śiraś-cheda}.

Ganesh aptly describes Pollock as a riotous elephant in the forest of books. “No \textsl{rasa},\index{rasa@\textsl{rasa}} no humanity”, asserts Ganesh. Pollock's posturing of humility is, Ganesh notes with a poet's touch of a telling simile, akin to fastening a tender flower in a garland of thorns. An exhaustive, or at least a more detailed, criticism of Pollock’s \textsl{Rasa Reader} remains a desideratum, and will help to show how the handling of a lofty theme by this American Orientalist betrays an approach which is anything but healthy and wholesome.

A few final remarks may be made here. That Abhinavagupta was by no means a \textsl{guru-drohin} towards his senior Bhaṭṭa Nāyaka (whose ideas he refined, rather than repurdiated (\textsl{matāni na dūṣitāni, kintu śodhitāni})) is just as true as Pollock is a \textsl{guru-drohin} towards his own preceptor, Prof. Daniel Ingalls (whose prime and sublime dictum was that the path of the critic of poetry must begin with poetry and not with theories of society); too, there is little of aucitya in the cavalier manner with which Pollock treats the key ideas of Alaṅkāraśāstra. Pollock is impartial in his cultivated contempt whether towards the Vedic of antiquity or towards the later rhetorical tradition.

Almost every paper has shown that a good many of the claims of Pollock are hollow and lack substance. And it goes without saying that the authors of the papers here all hold themselves responsible for the ideas they have presented.

One is reminded of a Sanskrit verse on the good and bad uses of a command over language.
\begin{quote}
\textsl{“ asthāne gamitā layaṁ hata-dhiyāṁ vāg-devatā kalpate}\\
\textsl{\qquad      dhikkārāya parābhavāya mahate tāpāya pāpāya vā} |\\
\textsl{sthāne tu vyayitā satāṁ prabhavati prakhyātaye bhūtaye}\\
\textsl{\qquad      ceto-nirvṛtaye paropakṛtaye prānte śivāvāptaye} ||”
\end{quote}
(Confer the divine faculty of speech indiscreetly upon the pervert: be sure to expect curses and humiliation, and agony and sin, unlimited. Bestow it, on the other hand, sensibly upon the noble: you may well look forward to fame and weal, bliss and benevolence, and the attainment of beatitude in fine).

\bigskip
\noindent
Cāndramāna Yugādi\hfill {\bf Dr. K. S. Kannan}\\
Śrīvilambi Saṁvatsara\hfill Academic Director\\
(18th March 2018)\hfill	and\\
~\phantom{a}\hfill  General Editor of the Series

\newpage

{\large\dev\bfseries सम्पादकीयम् }

\smallskip
 
%\textsl{\bfseries [A Sanskrit Synopsis of Ganesh's Sanskrit Paper]}

\label{editorial1}

\begin{quote}
{\dev ``निरन्तर-{\bfseries रसोद्गार}-गर्भ-सन्दर्भ-निर्भराः ।}\\[2pt]
{\dev गिरः कवीनाम् जीवन्ति न कथामात्रम् आश्रिताः ।।''}
\end{quote}
 
{\dev संस्कृतभाषया लिखितवतो विदुषो गणेशस्य लेखनस्य संक्षिप्तावलोकनम्}

\textsl{\textbf{[A Sanskrit Synopsis of Ganesh's Sanskrit Paper]}}


{\dev संपुटेऽस्मिन् प्रोफेसर् षेल्डन् पोल्लाकाख्यस्य विदुषो “रस रीडर्”} (Rasa Reader)\-{\dev -अभि\-धस्य पुस्तकस्य विमर्शनं विदुषा गणेशेन विहितमस्ति । विमर्शनमिदं  तावदूनषष्टिपुटात्मक-पोल्लाक-प्रास्ताविकमात्रं लक्ष्यीकरोति ।}

{\dev पाश्चात्त्येषु विपश्चित्सु नाम कतिचनैव कृतिनः काव्येषु शास्त्रेषु च भारतीयेषु लब्ध-प्रवेशाः; बहुलास्त्वाविल-मतय आविद्धबुद्धयो दुर्विदग्धास्स्वकीयेतरसकलजनपदकविजनशास्त्र\-कृत्स\-मुदायकुत्सनमात्रसमुत्सुकाः पौरोभाग्य\-मेवाहोभाग्यं मन्वाना दूषणैकधिषणा विराजन्ते ।\break तादृक्षेषु कौटिकेषु कूटस्थानमलंकुर्वन् पोल्लाकाख्यो विद्वत्तल्लजो वरीवर्ति ; प्रधानमल्लनिबर्ह\-णन्यायेन यदभिप्रायाद्रिपक्षविपक्षीकरणचणकुलिशायितशेमुषीका नैक इह विद्वांसो भारतीया\-श्श्रीमद्राजीवमल्होत्रप्रणीत-ग्रन्थराश्यध्ययन-चलच्चित्रक (विडियो)स्तोमपरिवीक्षणप्राप्त\break\-स्फूर्तिकाः ।}

{\dev उपक्रम एव शोभनं क्रममाद्रियमाणो गणेशो हिरियण्ण-नरसिंहभट्ट-गुण्डप्प-प्रभृतीन्\break कार्णाटान् कांश्चन  विदुषोऽन्यांश्चानन्दकुमारस्वाम्यग्रवालाद्यग्रगण्यान् मान्यान् भारतीयकला\-विषयक\-परिशोधनकृततीर्थान् संस्मृत्य, प्रतीपमतेः पोल्लाकस्य विततानि वितथानि मतजातानि प्रकर्षेण दूषयितुं प्रतिष्ठमानो नरसिंह\-भट्ट\-निदर्शितेन निशितेन नयेन प्रवक्तुमुचिततममिति प्रति\-जानीते ।}

{\dev अलङ्कारशास्त्रप्रतिपादितभावतत्त्वमधुनातना वैज्ञानिकाः केचन मस्तिकमध्यसम्भवद्विद्यु\break\-त्प्रचार}(synapses) {\dev सञ्जायमानरासायनिकक्रिया}(chemical reactions){\dev मात्रप्रपञ्चनमिति प्रचक्षते । तदाख्यानप्रत्याख्यानं विदधदयं विद्वानभ्युपगम्यापि तद्वादमस्मदनुभवविदूरभूत\break\-स्यास्य वैतथ्ययाथातथ्याभ्यामस्पृष्टत्वं शास्त्राराद्धा\-न्तस्य ख्यापयति । एवं हि नाम प्रवृद्धधिषणा अपि जनास्स्वयं क्लेशमनुभवन्तः परानपि क्लेशमनुभावयन्तीति विनिवेदयति विद्वानेषः । भौत\-शास्त्रस्तोमसरणी रैखिकेति} (linear) {\dev च भावशास्त्र\-समूहसृतिश्चाक्रिकीति} (cyclical) {\dev च विध एते विविञ्चन्ननौपम्यसारतां च पद्धत्योरमुयोरुपस्थापयन्, भौतशास्त्र} (physics) {\dev वर्तनीमति\-वर्तते रससिद्धान्तपद्धतिरिति च दर्शयति । अत एव भरतान्मुनेरर्वाचीना अपि भरतनिरूपितानि प्रमेयाणि विप्रतिपत्तिं विनैवोररीकर्तुं प्राभवन् । नहि भरतोदितनाट्य\-शास्त्रीयभाषासारल्यमेव हेतूकृत्य शास्त्रस्यैव शैशवावस्थाद्योतकत्वं तदिति मतमास्थातुमुचितं भवेत् । यतो हि तथात्व इतिहासपुराणादिष्वप्युक्तहेतुमेवाधारीकृत्य तत्त्वदर्शनचुञ्चूनामनस्ति\-त्वमलीकतमं ततः प्रती\break\-येत । यथा हि नाम वैयाकरणैर्नैयायिकैश्चायुर्वेदीयौषधविषयको महानादर उपमात्वेनोपस्था\break\-प्यते श्रद्धार्हताविषये, तथैव नयेन तेनैव रससिद्धान्तस्यापि श्रद्धेयता नामाबाधिता वरीवर्तत इति वक्ति गणॆशः । न ह्याधुनिकवैज्ञानिक- निरूपितवर्ष्मरचना}(anatomy){\dev जीवद्रासायनिक परिवर्तनका} (biochemistry) {\dev दिभिर्विषयजातैरभिज्ञानादरास्पदभूतैरायुर्वेदविहित\-मूलिका\break\-सेवनातस्सम्भवन्ती गदसमूहविनिवृत्तिर्व्याहन्येत ।}

{\dev भारतीयजीवनक्रमे तावल्लौकिकपारमार्थिकावपरस्परमिलितावनुपलभ्यौ जातुचिदपीति\break विदन्नपि पोल्लाकः कथञ्चित्तौ पृथक्चिकीर्षुः पृथग्जनमात्रादरणीयत्व एव तृप्तिमृच्छतीति नातुच्छं विभावयन्ति विमत्सरा विद्वांसः ।}

{\dev वैदिक्या हि पद्धतेस्सकलजनपदमर्दनैकपरायणतां प्रतिपिपादयिषोरस्य पोल्लाकस्य चिन्तनवर्त्म न हि सुपथा सञ्चरमाणानां वैदिकोक्त्यार्जवार्थजिघृक्षूणामनुग्रहमर्हति ।}

{\dev कलाशास्त्रस्य वेदान्तशास्त्रोपजीवितताप्युपपत्तिपुरस्सरं गणेशेन गदिता । ब्रह्मानन्दविप्रुण्मा\-त्रावभासो हि रसास्वाद इति निःसंशयमसकृच्च निगदितमभिरूपतमप्रतिभानवता शास्त्रकृन्मत\-ल्लिकेन चाभिनवगुप्तपादेन महाज्ञानिना सुधीभिस्समेभिस्सम्मानितञ्च ।}

{\dev रसवद्वस्तुप्राचुर्यं सतत्त्वार्थनिरूपणमपि वाङ्मये वैदिकेऽपि लोचनगोचरीभवति सूक्ततल्लजे पौरुषे सूक्ते, सृष्टिमूलगवेषके पुनर्नासदीये, सर्वविभूतिसमीक्षके रुद्राध्याये, द्यावापृथिव्योस्स\-ङ्गमे हृदयङ्गमे स्कम्भसूक्ते, तत्र तत्र च स्थलेष्वन्येषु । एवमेवोपनिषत्स्वपि भारतीयैः सौन्दर्य\-मीमांसा समुपेक्षितेति वितथमेव कथयता पोल्लाकेन तावत् परत्र पल्लवग्राहिपाण्डित्यमपि\break  सङ्गीतकलाशास्त्रग्रन्थविषयकमनासाद्यापि चुलुकीकृतशास्त्रराशिनेव यन्निःशङ्कमुट्टङ्कितं तन्न मनीषिभिस्सद्धिषणैरीषदप्याद्रियते । उद्रिक्तसत्त्वायां चित्तवृत्तौ कलाविदः स्वजनिमाप्नुवती कलाकृतिः खलु तन्मयीभवनरूपसहृदयचित्तसत्त्वोद्रेकसम्पादन एव साफल्यं भजत इति\break करतलामलकीकृतकलातत्त्वानामन्तश्चक्षुषो नापरोक्षम् । अथ प्रतिभातत्त्वविभावनं न सम्य\-ग्विहितं भारतीयैः कलातत्त्वविज्ञानिभिरिति ब्रुवाणेन पोल्लाकेन कलाप्रपञ्चनिर्माणमूलद्रव्य\-स्यापि मूलान्वेषणरूपेणानवस्थादोषभाज्याविष्कृतः प्रयतने बहुमानो निरर्थकः । कारणं हि सूक्ष्मरूपेण सदविभावनीयनानागुणलक्षणं सत्स्वयं च कार्यभावमापन्नं तावत् परिगणनातीत\-गुणलक्षण- समुल्लसितं विभाति बीजवृक्षनयेनेति नैतेन विभावितं विदुषा पोल्लाकेन ।}

{\dev वात्सल्यविषय उद्भावितमाक्षेपममुना सरलया सरण्या निराकरणार्हं संक्षेपेण निरूपयति\break गणेशः । अलङ्कारशास्त्रं नाट्यशास्त्रापेक्षयाऽर्वाचीनं मन्यमानः पोल्लाको गीतवाद्यनृत्यमये सङ्गीते सङ्गमं त्रितयस्य जानन्नपि न तत्त्वतो विभावितवानिव विभाति । प्रयोगत्वमनापन्नं काव्यमनास्वाद्यमव\-शिष्यत इति काव्यं सर्वमपि सर्वान्तरङ्गभूते चित्तरङ्गस्थले नाट्यायत एवेति वेदितुं नाटाट्यमानताऽपेक्ष्यते खलु । कारयित्र्या भावयित्र्याः प्रतिभाया उभय्या आलयं यद्यपि भवति कविव्यपदेशभाक्, कविसहृदयावुभावपि भावयित्र्या ह्यवश्यमाकरौ भवत इत्यतो भावयित्र्या एव प्रगाढत्वं विचक्षणानां प्रत्यक्षो विषयः। कवनसमये जागरितरजःसंस्पर्शवतः कवेरपेक्षया स्वदनसमयसमुपजातसत्त्वोत्सोदयः सहृदय एव परमामणामर्हतीति तत्त्वार्थममुं पर्यनुयोक्तुं विदितवेदितव्यः को नाम प्रभवेत् ?}

{\dev अथ च भारतीयकाव्यपरम्परा बौद्धमतप्रोद्भूतेति संभावनापि नाम पोल्लाकस्यैव कौटिल्यवैकल्य- भ्रान्त्यादीनां सूचक इति सुष्ठु सूचयति श्रीमान् गणेशः । यतो हि निवृत्तिधर्ममात्रचुञ्चुभिरभि\-हितानि तत्त्वानि कथङ्कारं काव्यकारं विनयेयुः? तथापि पाश्चात्त्यैर्बौद्धमतविषये प्रदर्शितः पक्षपातो विवेक- विहङ्गमस्य पक्षावेव विच्छिन्द्यान्ननु ? अश्वघोषोऽप्यधमर्णो निरवद्यपद्य\-निर्माणचणस्य कवेर्वाल्मीकेरिति विषये को नाम विज्ञो विप्रतिपद्येत?}

{\dev काव्यतत्त्वं च वेदान्तीकृतं भारतीयैरिति भाषमाणेन पोल्लकेन भरतकृतनाट्यशास्त्र एवाकृत्रिम\-वाङ्मयाधारत्वं नाट्यस्य प्रतिपादितं सदपि विक्षिप्तचित्ततयाऽनालोडितमिव विभाति ।}

{\dev पोल्लाकं ग्रन्थारण्यगन्धेभं ब्रुवन् गणेशस्स्वभावोक्तिरसिकत्वेनैव प्रत्यभिज्ञायते विमर्शन\-विच\-क्षणैः । रससिद्धान्तमेव यातयाममित्याक्षिपतः पोल्लाकस्य रसवैमुख्यमेव मुख्यलक्षणमिति भाति ।}

{\dev विनयवाग्भिर्ग्रन्थं समापयन् पोल्लाकः स्रगाभासपरिपूर्त्यै कण्टकमालामध्ये नम्रताकुसुमग्रथ\-नोत्सुको लक्ष्यत इति लक्षयन् स्वविमर्शमन्यादृशं समाप्तिमापयति शतावधानी गणेशः । विपक्षप्रतिक्षेपं गणेशस्येत्थं संक्षेपेण विनिवेद्य विरमाम इति शम्।।}

\medskip
\begin{center}
{\dev रसादृते, प्रविशति ह्रदयम् न तद्विदम्}\\[2pt]
{\dev मणिर् इव कृत्रिम-राग-योजितः}
\end{center}
\medskip


\bigskip
\bigskip
\noindent
{\dev चान्द्रमान-युगादिः}\hfill {\dev\bfseries डा. के. एस्. कण्णन्}\\
{\dev श्रीविलम्बि-संवत्सरः}\hfill {\dev शैक्षणिक-निर्देशकः प्रधान-सम्पादकश्च}\\
{\dev (१८ मार्च्-मासः २०१८)}

.
