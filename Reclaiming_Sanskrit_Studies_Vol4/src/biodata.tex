\makeatletter
\def\@makeschapterhead#1{%
  \vspace*{50\p@}%
  {\parindent \z@ \raggedleft
    \normalfont
    \interlinepenalty\@M
    \LARGE \bfseries  #1\par\nobreak
    \vskip 20\p@
  }}
\makeatother

\chapter*{Our Contributors\\ {\rm\sl\small (in alphabetical order of last names)}}\label{contributors}


\lhead[\small\thepage\quad Our Contributors]{}
\rhead[]{\small Our Contributors\quad \thepage}
\chead[]{}
\cfoot[]{}

\hfill {\large\bf Naresh Cuntoor}
\medskip

Dr.~Naresh Prakash Cuntoor is a Senior Research Scientist, Intelligent Automation Inc.,
Rockville, MD, US. He has an M.S. and PhD from the Department of Electrical and
Computer Engineering, University of Maryland. His research interests include human
activity recognition, scene understanding, perceptual organization and computer vision for
robotics applications. He pursues Sanskrit with keen interest and is a volunteer for
Samskrita Bharati, USA.

\bigskip
\hfill {\large\bf Sreejit Datta}
\medskip

Sreejit Datta is currently pursuing his PhD in Comparative Literature at the Centre
for Comparative Literature, Bhasha Bhavana, Visva Bharati, Shantiniketan. He is
an accomplished musician and has been obtaining training since the past 15 years
without a break He was awarded the Young Artiste Scholarship in the category
Hindustani Light Classical Music (with a specialization in Rabindrasangeet,
Tagore's songs) by the Ministry of Culture, Government of India for a two-year
advanced training in his chosen field in the year 2010.

\bigskip
\hfill {\large\bf R. Ganesh}
\medskip

Shatāvadhānī Dr.\@ R.\@ Ganesh is an Engineer and Metallurgist by training, a Sanskrit and Kannada poet, and a practitioner of the traditional art form of Avadhāna known for extempore poetry. He has performed more than 1000 Avadhāna-s till date. A scholar in Kannada and Sanskrit, he gives discourses on Indian culture and Classical Indian poetry (Sanskrit/Kannada). He has composed/translated works in Sanskrit/Kannada. He is a recipient of the Rajyotsava Prashasti (1992) and the Badarayan Vyas Samman (2003) amongst others. His D.Litt Degree is on the art of
Avadhāna in Kannada language.

\bigskip
\hfill {\large\bf K. Gopinath}
\medskip

K.~Gopinath is a Professor at Indian Institute of Science in the Computer Science and
Automation Department. His research interests are primarily in the computer systems area
(Operating Systems, Storage Systems, Systems Security and Systems Verification). He is
currently an associate editor of IEEE Letters of Computer Society (2018-); previously he was
also an associate editor of ACM Trans. on Storage (2010-2017). His education has been at
IIT-Madras (B.Tech'77), University of Wisconsin, Madison (MS'80) and Stanford University
(PhD'88). He has also worked at AMD (Sunnyvale) ('80-'82), and as a PostDoc ('88-'89) at
Stanford. He is also interested in understanding the Indic contributions in the area of
computer science and more broadly in S\&T.

\bigskip
\hfill {\large\bf Ashay Naik}
\medskip

Ashay Naik is a software developer at Matific Ltd. and has just released his first book
\textsl{Natural Enmity: Reflections on the Niti and Rasa of the Panchatantra [Book 1]}. He has a
Masters in Information Technology from the Queensland University of Technology,
Australia and an Honours in Sanskrit from the University of Sydney, Australia.

\bigskip
\hfill {\large\bf Shankar Rajaraman}
\medskip

Shankar Rajaraman is an allopathic doctor, a psychiatrist, and an award-winning Sanskrit
poet. His Sanskrit works include \textsl{Bhārāvatāra-stava} (in praise of Śiva), \textsl{Nipuṇa-prāghuṇaka} (a Bhāṇa on a contemporary theme), \textsl{Vaidyopahāsa-kalikā} (a satire on doctors) and \textsl{Devīdānavīya} (illustrating \textsl{citrabandha}) for which the Karnataka Samskrit University awarded him with the Professor M. Hiriyanna Sanskrit Works Award in 2013. \textsl{Citranaiṣadham}, his recent work, illustrates \textsl{gomūtrikābandha} throughout. He has been awarded the Badrayan Vyas Samman for the year 2016 and Bannanje Puraskara
for the year 2017. He has translated across Sanskrit, English and Kannada. His doctoral thesis is a confluence of psychology and Sanskrit poetics.

\bigskip
\hfill {\large\bf Charu Uppal}
\medskip

Charu Uppal, is a Senior Lecturer at Karlstad University in Sweden. Her research, which generally follows under the broad umbrella of media studies focuses on the role of media in bringing about social change, identity formation and mobilizing citizens towards cultural and political activism. Her work has appeared in journals such as Journal of Creative Communication, International Communication Gazette and Global Media and Communication.
