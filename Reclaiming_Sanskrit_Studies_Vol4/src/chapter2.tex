\chapter[Pollock’s Desacralization of the Indian {\sl Rasa} Tradition]{Pollock’s Desacralization of the Indian \textsl{Rasa} Tradition}\label{chapter\thechapter:begin}
%~ \footnotetext[1]{pp.~\pageref{chapter\thechapter:begin}--\pageref{chapter\thechapter:end}. In: Kannan, K S (Ed.) (2018) {\sl Śāstra-s Through the Lens of Western Indology - A Response}. Chennai: Infinity Foundation India.}
\Authorline{Ashay Naik}
\lhead[\small\thepage\quad Ashay Naik]{}

\section*{Abstract}

This essay explores the strategies employed by Pollock to distort the self-understanding of the Indian {\sl kāvya-śāstra} tradition by diminishing the importance of religious aesthetics, which forms its core part, and directing the attention of readers toward a socio-political aesthetic. He is also keen on separating Veda-s from {\sl kāvya}. While the tradition itself appears to have been more interested in {\sl kāvya} as the source of an aesthetic experience akin to the religious, Pollock is more interested in reading it as an expression of social power. His principal target in this endeavour is Abhinavagupta. While the {\sl kāvya-śāstra} tradition reveres him as a central figure for his masterful delineation of the process of {\sl rasa} arising in the reader, Pollock seeks to reduce his significance in a variety of ways. He criticizes the {\sl rasa-dhvani} school for their inattentiveness towards the sociality which allegedly forms the basis of their interpretation. He tries to show that Abhinavagupta's theory of `readerly' {\sl rasa} arose merely out of an attempt to save the {\sl rasa-dhvani} theory from the critique of Bhaṭṭa Nāyaka. Most importantly, he tries to use Bhoja as a foil against Abhinavagupta, as a better exponent of {\sl rasa} who was faithful to the tradition and duly recognized the social importance of {\sl kāvya}. This essay aims to critique Pollock's desacralization of {\sl kāvya} by showing the connection between Veda and {\sl kāvya}, elaborating on Abhinavagupta's religious aesthetics and showing how Bhoja's aesthetics was itself religious and contributed to the religious aesthetics of Rūpa Gosvāmin. It concludes by noting that while the indigenous {\sl rasa} tradition is thus under pressure of being secularized, it is being appropriated by other sacred traditions, such as Indian Christianity, for the aesthetic articulation of their religious discourse.

\section*{Introduction}

{\sl But the path of the critic of poetry must begin with poetry, not with theories of society.}

\hfill {\sl -- An Anthology of Sanskrit Court Poetry, Daniel H. H. Ingalls Sr.}

\smallskip

Religious aesthetics is an important dimension of the Indian intellectual tradition concerning {\sl rasa} --- the aesthetic delight one experiences when one reads literature or witnesses a drama. That {\sl kāvya} entails some kind of a divine experience is evident primarily from its connection with the Veda-s and other forms of literature such as the {\sl Rāmāyaṇa}, which are considered sacred even if categorized as {\sl laukika} (worldly). {\sl Rasa} as a theological category was first elaborated by the 10th century Kashmiri scholar, Abhinavagupta, and thereafter became integral to the {\sl kāvya-śāstra} or {\sl sāhitya-śāstra} tradition. Sheldon Pollock, on the other hand, follows the view that literature is a means of expressing and naturalizing socio-political dominance. The Indian understanding of {\sl kāvya} in terms of a religious aesthetic contradicts his project of depicting it as a socio-political aesthetic. Furthermore, given the current post-colonial circumstances, he would like to avoid the charge of Orientalism by showing that such a depiction is free of any Eurocentric, Christo-centric bias and compatible with the Indian understanding as such.

This essay explains how Pollock seeks to achieve his goal of desacralizing {\sl kāvya}. Firstly, it shows how Pollock tries to delink {\sl kāvya} from the Veda-s and mark its beginning from the {\sl Rāmāyaṇa}, which he reads as a political text. He also reduces {\sl kāvya} to {\sl praśasti}, the eulogies inscribed by Indian kings to articulate their political will. Secondly, the essay notes how Pollock forces a sociological interpretation on the religious aesthetic of Abhinavagupta and the language philosophy of his predecessor, Ānandavardhana. The last two sections explain how, to indigenize his views, Pollock props up {\sl kāvya-śāstra} scholars Bhoja and Bhaṭṭa Nāyaka, whose views he interprets as compatible with his own agenda, as superior to Abhinavagupta but whom the {\sl kāvya-śāstra} tradition has unfairly neglected. The essay concludes by pointing out that Pollock is not alone in his attempt at desacralizing {\sl kāvya}. Other scholars of Indian origin also share his concern. Meanwhile, the religious aesthetic of {\sl rasa} is being appropriated by Indian Christians to spread the gospel of Christ.\\[-20pt] 

\section*{Politicizing {\sl\bfseries kāvya} in relation to the Veda-s, {\sl\bfseries Rāmāyaṇa} and {\sl\bfseries praśasti}}

The relation between the Veda-s and {\sl kāvya} is complex. The determination of whether {\sl kāvya} originates in the Veda-s depends entirely on one's conceptualization of {\sl kāvya}. Inasmuch as {\sl kāvya} is understood as metrical compositions characterized by beauty expressed through figures of speech, one can trace its beginning to the Veda-s. However, if {\sl kāvya} is conceptualized as {\sl kāntā-sammita-śabda} (word of the beloved), where the expressed sense of the word is different from the intended sense, then it would have to be regarded as a literary genre distinct from the Veda-s, which is understood as {\sl prabhu-sammita-śabda} (word of the lord), where the word directly expresses the sense as a command. The former understanding i.e. of {\sl kāvya} as metrical composition characterized by beauty, is what is generally prevalent in contemporary scholarship while the latter view i.e. of {\sl kāvya} as {\sl kāntā-sammita-śabda} (word of the beloved), was held by some of the scholars of the {\sl kāvya-śāstra} tradition. Pollock (2006:3) refers to the former as {\sl pāramārthika sat} (absolute truth of philosophical reason) and the latter as {\sl vyāvahārika-sat} (certitudes people have at different stages of their history that provide the grounds for their beliefs and actions). He privileges the latter over the former for the ostentatious reason that ``we cannot understand the past until we grasp how those who made it understood what they were making, and why'' (Pollock 2006:3) but it is evident that the real cause is the opportunity it affords for suggesting a breach between the Veda-s and {\sl kāvya}. Furthermore, the {\sl kāvya-śāstra} tradition has also declared Vālmīki as the {\sl ādi-kavi} (first poet) and {\sl Rāmāyaṇa} as the {\sl ādi-kāvya} (first {\sl kāvya}). 

\newpage

Pollock exploits both these facts to accomplish a desacralization of {\sl kāvya} and its re-interpretation as a political aesthetic. Firstly, he uses the traditional discourse on the different kinds of {\sl śabda} (word) to separate {\sl kāvya} from the sacred Veda-s. Secondly, he depicts {\sl Rāmāyaṇa} as a political text and by following tradition in regarding it as the first {\sl kāvya}, establishes the political nature of {\sl kāvya}. However, his argument does not hold water and it is apparent that Pollock has simply manipulated the tradition to suit his purpose. Distinguishing {\sl kāvya} and the Veda-s on the basis of {\sl śabda} does not mean that they are saying different things or addressing separate concerns. It simply means that they employ different forms of expression --- what the Veda-s articulate as a direct command, {\sl kāvya} conveys by means of rhetorical speech. It does not follow therefrom that tradition viewed the former as belonging to a sacred realm and the latter to a secular realm. 

For Pollock the beginning of {\sl kāvya} means ``the first occurrence of a confluence of conceptual and material factors that were themselves altogether new. These include new specific norms, both formal and substantive, of expressive, workly [{\sl sic}] discourse; a new reflexive awareness of textuality; a production of new genre categories; and the application of a new storage technology, namely, writing'' (2006:77). Now, when the tradition declares {\sl Rāmāyaṇa} to be the {\sl ādi-kāvya}, it does not make the claim that the aforementioned ``conceptual and material factors'' specified by Pollock came into being. This is his own assumption --- his {\sl pāramārthika sat} as it were --- which he has superimposed on the {\sl vyāvahārika sat} of the tradition.

To be fair Pollock does refer to the traditional basis on which the {\sl Rāmāyaṇa} is revered as an {\sl ādikāvya} but he does not take it seriously. It is the {\sl śoka} (piteous cry) uttered by Vālmīki in the form of a {\sl śloka} (verse) when he observes a hunter shoot a bird in the forest. That tradition has fully endorsed this view is obvious from the ninth century {\sl Dhvanyāloka} of Ānandavardhana: 

\begin{myquote}
It is just this [{\sl rasa}] that is the soul of poetry. And so it was that, long ago, grief [{\sl śoka}], arising in the first poet [{\sl ādikavi}] from the separation of the pair of curlews, became verse [{\sl śloka}]'' (Ingalls 1990:113). 
\end{myquote}

For one who claims to be serious about {\sl vyāvahārika sat}, this explanation should be more than sufficient. But not for Pollock since this explanation is completely disconnected from any kind of politics. And so expediently he turns to {\sl pāramārthika sat}:  
\vskip 1pt

\begin{myquote}
But this may not be the only kind of newness toward which the prelude is pointing. The {\sl Rāmāyaṇa}'s highly self-conscious assertion of primacy may very likely be alluding to the fact that it was the first {\sl kāvya} to be composed in Sanskrit rather than some other form of language available in South Asia. 
\hfill (Pollock 2006:78)
\end{myquote}
\vskip 1pt

There is no basis for this assumption and Pollock offers none. Likewise, Pollock has issues with the {\sl vyāvahārika sat} i.e. the certitudes of the tradition that {\sl śloka} was Vālmīki's invention when the {\sl pāramārthika sat} is that the meter ``antedates the work by a millennium or more'' (Pollock 2006:78). And so he reads ``Vālmīki’s primacy in terms of metrics ... as a kind of synecdoche for the formal innovations of the work as a whole, and these are indeed substantial” (Pollock 2006:78). But if {\sl pāramārthika sat} is acceptable then that too agrees with the tradition in this case:
\vskip 1pt

\begin{myquote}
In Piṅgala's {\sl śāstra} this [i.e. {\sl śloka}] form is totally absent. In the Vedic literature, this word has been used in different senses. {\sl Nirukta} reads it as the synonym of the speech, of the {\sl anuṣṭubh}. In the {\sl Ṛgveda}, it is a call, or voice of the God, sound or noise. Later it is used in the sense of strophe. In {\sl Rāmāyaṇa}, it is a verse born out of sorrow, which has been echoed in Ānandavardhana's {\sl ślokaḥ śokatvamāgataḥ}''\hfill (Mitra 1989:45).
\end{myquote}
\vskip 1pt

The problem is that Pollock has not bothered to interrogate the history of Sanskrit metrics to ascertain the novelty of {\sl śloka} in the {\sl Rāmāyaṇa}, as a form connected with aesthetic emotion, because he is too fixated with studying {\sl kāvya} politically. It is also for this reason that he dismisses the orality of the text as fictitious. The very fact that the text claims to have been composed mentally by Vālmīki, then transmitted orally to Rāma's sons, who then recited it before Rāma, appears to him ``nostalgia for the oral and a desire to continue to share in its authenticity and authority'' (Pollock 2006:78). The original literacy of the text is vital for Pollock's narrative because then the date of its composition, and accordingly the commencement of {\sl kāvya} (the text being the {\sl ādikāvya}) can be located in the early first millennium when, according to Pollock, Sanskrit writing makes its first appearance. On the other hand, had it been orally composed then it could have been composed even a few centuries earlier, in which case, the date of origin for {\sl kāvya} would contradict Pollock's narrative. Thus, we see that with regards to both the Veda-s and the {\sl Rāmāyaṇa}, Pollock is merely exploiting the traditional understanding to bestow legitimacy on his view of {\sl kāvya} as a political aesthetic.

Therefore, of the variety of forms in which {\sl kāvya} expressed itself, Pollock is obsessed with only one of them --- {\sl praśasti}. Inasmuch as {\sl kāvya} and {\sl praśasti} are seen as products of the same cultural milieu, there is no issue. But it appears that Pollock assigns to {\sl praśasti} a significance far beyond the position it obtained in the {\sl kāvya-śāstra} tradition, which was nearly zero. His reduction of {\sl kāvya} to {\sl praśasti} is best expressed in the following passage:

\begin{myquote}
The {\sl praśasti} itself was intimately related to, even a subset of, a new form of language use that was coming into being in the same period and would eventually be given the name kāvya 
\hfill Pollock (2006:75) [italics mine]
\end{myquote}

However, in the {\sl kāvya-śāstra} tradition, there is very little reflection on {\sl praśasti} as such and no direction that it could be composed only in Sanskrit, as was the case, for example, with the {\sl mahākāvya}. For Pollock, {\sl praśasti} in the inscriptions and {\sl kāvya} in the court, appear to be two sides of the same coin but it is not at all clear if the {\sl kāvya-śāstra} tradition attached as much importance to the former. True, the linguistic and aesthetic analysis of {\sl kāvya} would have influenced the composition of {\sl praśasti}-s but the way Pollock presents the issue it would seem that it worked the other way around, as if the study of {\sl kāvya} was motivated by its application in, what would have been considered to be its final and sole product, {\sl praśasti}. The point of such projection is, of course, to show that {\sl kāvya} was all about politics, but this view is not borne out by the tradition itself which conceives of other purposes for {\sl kāvya}, namely, the production of {\sl alaukika ramaṇīyatā, āhlāda} and {\sl saundarya}, what could be understood as pure aesthetic delight or blissful beauty.\\[-20pt]  

\section*{Social Aesthetics of {\sl\bfseries kāvya}}

Social aesthetics refers to a reflection on the social and moral instruction that literature does or should provide, the manner in which the instruction is communicated so as to be effective, and so on. Pollock's essay {\sl The Social Aesthetic and Sanskrit Literary Theory} begins with the observation that Indology, for the most part, has not paid much attention to this aspect of Sanskrit literary theory and instead ``tends to cleave to the intellectual agenda set by the tradition itself'' (2001:197). To an extent, this essay is a critique of the position held by Daniel H. H. Ingalls Sr. on the hermeneutics of Sanskrit literary texts, whose antipathy for their sociological analysis appears to have been visceral. In {\sl An Anthology of Sanskrit Court Poetry}, Ingalls berates Western scholars who used contemporary European and American standards in their assessment of Sanskrit poetry and ignored the views of the Sanskrit literary theorists on the subject. But he reserves his most severe judgment for the Indian Marxist, D. D. Kosambi, whose ``theory of Sanskrit poetry is an application to India of Engels' and Plekhanov's theories of the class origins of literature'' (Ingalls 1965:50). Pollock states that ``thoughtful students of Sanskrit know that careful reading of the literature of others presupposes careful listening to others' theory'' and mentions Ingalls as ``pre-eminent among those'' (Pollock 2001:198-99). Having said that, the point of Pollock's essay is to tear down that view. What is striking is the diabolical way in which he goes about it. He does not reject Ingalls' prioritization of native theories in the study of native literature. Rather, he tries to prove that the native theories themselves aim to prioritize the sociological in their reflection on literature.

In Pollock's view, there was a time when the social aesthetic occupied center-stage in {\sl sāhitya-śāstra}. He provides details from the {\sl Śṛṅgāra-Prakāśa} by Bhoja, an 11th century scholar, to argue this point. But, he adds, in the 9th and 10th centuries there occurred an intellectual revolution in Sanskrit poetics, with two Kashmiri scholars, Ānandavardhana and Abhinavagupta, whose meditations on {\sl dhvani} (linguistic suggestion) and readerly {\sl rasa} (emotive experience of the reader), gave more prominence to language-philosophy and emotive experience in the study of literature. Now usually when we speak of {\sl sāhitya-śāstra}, we refer to a period lasting more than a millennium from Bharata in the early centuries of the common era to Jagannātha Paṇḍitarāja in the mid-17th century and we learn from Pollock's essay itself, that social aesthetic features in the works of most scholars in this period. But after the Kashmiri revolution, it allegedly declined in importance, and language-philosophy and emotive experience became a primary concern in literary analysis. This appears to be the case with traditional pundits up to our time and following their cue, Indologists also paid more attention to these two aspects in Sanskrit literary theory rather than the social aesthetic contained in it. This is Pollock's complaint and the main aim of his essay is to assess whether ``we can historically recuperate the social in Sanskrit literary theory'' (Pollock 2001:199). By ``recuperating'' the social, Pollock does not mean that the social aesthetic has been lost and is in need of a rediscovery but rather that its former importance needs to be reclaimed and he has cited Bhoja's work as an example of the centrality that it once enjoyed in Sanskrit literary theory. But that is not all. Pollock claims that the social aesthetic is also integral to language-philosophy and emotive experience but that it has remained obscured from the view of their proponents, fully in case of the former and partially in case of the latter. So ``recuperation'' of the social also stands for the recovery of the alleged significance of the social in these two areas. The rest of this section demonstrates how Pollock strives to desacralize {\sl kāvya} by reducing it to an aesthetic for the maintenance of social power.

In his treatise on {\sl dhvani} (linguistic suggestion) called {\sl Dhvanyāloka}, Ānandavardhana provides some examples to explain this concept. These examples are {\sl gāthā}-s (one-verse poems) drawn from {\sl Gāhā Sattasai}, an anthology of such poems written in Mahārāṣṭrī Prakṛta. They are quite enigmatic in nature as the following illustration shows:

\begin{quote}
{\sl You’re free to go wandering, holy man.}\\
{\sl The little dog was killed today}\\
{\sl by the fierce lion making its lair}\\
{\sl in the thicket on the banks of the Godā river.}
\end{quote}

Ānandavardhana has quoted this poem as an example of {\sl dhvani} where the literal sense is that of an invitation but the suggested sense, i.e. the {\sl dhvani}, is of a prohibition. In this case, limiting ourselves to the words in the poem, we can infer that someone is dissuading, albeit suggestively, a holy man from visiting a riverside thicket. Presumably, the holy man was doing so because he could not go to some other place as it had been overrun by a dog. The speaker is advising him that the dog has now been killed by a lion who has actually invaded the thicket. The literal meaning is therefore ``you are free to go wandering'' but the suggested meaning is ``do not go to the thicket.'' Now the {\sl Dhvanyāloka} is a treatise on {\sl dhvani} and by its emphasis on linguistic analysis it attempts to bring {\sl kāvya-śāstra} in league with the other highly reputed disciplines of ancient Indian thought such as {\sl vyākaraṇa, mimāṁsa} and {\sl nyāya} --- all of which are meditations on language-philosophy.

What exactly do we mean by language-philosophy? In the case of this example, it could take the form of reflecting over the linguistic use of command ({\sl vidhi}) and prohibition ({\sl niṣedha}). Thus, there are various kinds of {\sl vidhi}-s: {\sl pravartana}, where you are explicitly told to do something; {\sl atisarga}, where you are not directly commanded but an obstacle that is preventing you from doing something you were already committed to do is removed, and so on. Thus, the holy man has not been commanded to wander but the dog that was obstructing his wanderings has been removed. Further, what Ānandavardhana would like us to note is that in the very womb of that {\sl vidhi} there is concealed a {\sl niṣedha}; what makes that {\sl vidhi} possible, also makes possible that {\sl niṣedha} without having it to be stated in so many words.                

This is a glimpse of the kind of stuff that interested Sanskrit literary theoreticians. As would be obvious, nowhere in the foregoing did I need to mention the identity of the speaker or the purpose of the prohibition offered as an invitation. Neither of the two prevent us from reflecting upon {\sl dhvani} as an artifice of language. But it so happens that in his commentary on the {\sl Dhvanyāloka}, before diving into such intricacies of language-philosophy as explained above, Abhinavagupta has suggested that the verse is uttered by a woman to protect the privacy of her rendezvous with her secret lover. Other commentators have repeated that story or something similar, but nobody has explained how they obtained this ancillary information. This is probably because these facts are peripheral and ultimately not relevant to a discussion on language-philosophy and to their primary task as commentators on the {\sl Dhvanyāloka}. But this lacuna bothers Pollock very much.
\vskip 2pt

In his view, the language-philosophy cannot help us understand that the {\sl gāthā} is about a woman trying to protect her privacy. Rather, 
\bigskip

\begin{myquote}
``what we need is a social pragmatics which can explain to us that thickets by riverbanks are rendezvous spots for unmarried couples who cannot otherwise be together and the protection of their privacy means that people cannot be literally commanded to avoid them as it would reveal the liaison. Hence, the suggestion is necessary and the speaker must be a woman because the gender relations that constitute the social world of Prakrit poetry demand that it is always the woman, never the man, who organizes adultery. Only when we know such social-literary facts does the real suggestion [i.e. {\sl dhvani}] behind the poems become available, that the women speakers are sophisticated and clever, and ardent to preserve a place of lovemaking''
\hfill (Pollock 2001:207-208)
\end{myquote}

In my view, the fundamental problem with Pollock's objection is that he has not understood {\sl dhvani} at all --- or that his understanding of {\sl dhvani} is different from Ānandavardhana {\sl et al}. He transforms {\sl dhvani} from a linguistic phenomenon to a social suggestion and then complains that the social basis of the concept has not been discussed by Indian scholars. In fact, he has titled this section of his essay as ``Social Suggestion'' and as we can see from the foregoing passage, in his view, the ``real'' suggestion of the poem is purely social and alludes to the loose and devious character of women. However, such a conception of {\sl dhvani} appears to be Pollock's own innovation and would explain why Abhinavagupta {\sl et al} have not wasted any time explaining the social narrative in which they contextualized the poem. In fact, any social narrative could have been proposed to explain the same poem. For example, the speaker could be a man who buried some treasure in the thicket and was afraid someone might stumble upon it. Thus, {\sl dhvani} is a linguistic suggestion and Indian scholars approach it independent of the sociality of its occurrence. The problem with Pollock's view, however, is that it is fixated on the specific social situation described by the commentators and makes its explanation a precondition for the understanding of the {\sl dhvani}. 

Furthermore, if indeed a reflection on the social was necessary for understanding {\sl dhvani} then surely someone among the several reputed commentators of the {\sl Dhvanyāloka} would have undertaken that task. If nobody did, then one might ponder over the possibility of one's own misunderstanding. Instead, Pollock offers an explanation for the alleged lapse and points out the dire consequences of it: 

\begin{myquote}
``when both readerly expectation and theoretical concern are focused on the linguistic mechanisms of meaning, the social conditions of aesthetic suggestion escape observation let alone interrogation. The conditions for understanding this literature [i.e. Prakrit poetry] are the permanence, predictability, the common-sense of the social world, and by the very writing and reading of this and all other poetry – and this seems to be a crucial social effect --- these conditions are made all the more permanent, predictable, and commonsensical''
\hfill (Pollock 2001:208). 
\end{myquote}

Apart from its heavy-handedness, this is a bizarre conclusion. Let us say someone did explain the ``social pragmatics'' as Pollock has done. How would that change anything? What ``interrogation'' of the social situation is expected here? While it appears that it is the literary scholars who are charged, the one who is actually standing in the docks is language-philosophy itself because that is the distraction that made them absent-minded about the social. While it appears to be a description of an alleged blunder by scholars of the Sanskrit literary tradition, it is actually a prescription directed at us, that we should undertake social introspection rather than language-philosophy in the process of analyzing Sanskrit literary texts. In other words, he is suggesting that we should no longer cleave to the intellectual agenda of the currently dominant tradition because it is deficient and instead seek to revive the moribund tradition in which the social was the principal theme.

Moving from language-philosophy to aesthetic emotion, we may note that prior to Abhinavagupta, it was understood that {\sl rasa} was produced by the affective state of the character in the text. But Abhinavagupta held that {\sl rasa} was produced by the reader's experience of the text and this became the dominant view. The question of how a text produces {\sl rasa} thus transformed into a question of how the reader experiences {\sl rasa} and ``the answer was found to lie in a close analogy with religious experience'' (Pollock 2001:198). This is what Pollock refers to as the theological turn in literary theory and alleges that it ``partially'' constrained the social ground of literary theorization.

The second section of the essay, entitled ``False feelings'' that supposedly deals with this issue is the most convoluted and I will briefly summarize my understanding of it. Now corresponding to {\sl rasa} is the complementary concept of {\sl rasābhāsa}, which refers to an invalid emotive experience. So, for example, if a hero in a text behaves heroically i.e. according to the social convention of heroism, then it produces a heroic {\sl rasa}, and when he behaves un-heroically i.e. according to the social convention of un-heroic behavior, then it produces a heroic {\sl rasābhāsa}. As one can see, the normative discourse is fundamental to notions of {\sl rasa} and {\sl rasābhāsa} and Pollock points out how scholars reflected on it throughout the tradition. The concept of {\sl rasābhāsa} underwent a change along two dimensions. Initially, it was seen as a necessary part of narrative complexity but eventually it was treated as censurable, to be eschewed in good literature. The notion of readerly {\sl rasa} raised the issue of how a reader's emotive experience could be invalid. Abhinavagupta's response was that the {\sl rasābhāsa} occurs as an after-thought and not at the moment of experience itself.

\newpage

All this has been explained in excruciating detail in Pollock's essay but what is not clear at all is how was the social ground of literary theorization ``partially'' constrained by the theological turn? The essay itself makes evident that reflection on the social-moral aesthetic flourished throughout the tradition. So what is the problem? In my view, neither the theological nor the linguistic emphasis occlude any social-moral aesthetic. It is just that they make us focus our attention on theology or language-philosophy in literary criticism rather than on the social-moral aesthetic. Unfortunately, Pollock cannot say outright that we should concentrate on the latter aspect of the tradition and ignore the former because that would be contrary to the priorities of the tradition. So he wants to make the point that even the former are grounded in sociality but the sociological basis remained invisible to the tradition.
\vskip 2pt

In the history of the {\sl kāvya-śāstra} tradition, Ānandavardhana and Abhinavagupta are regarded as the most eminent scholars. Abhinavagupta was the first scholar who wrote prolifically on {\sl rasa} as a religious aesthetic, in his commentary on the work of Ānandavardhana. Pollock, on the other hand, is interested in studying {\sl kāvya} as a socio-political aesthetic. However, in the contemporary post-colonial milieu, this cannot be straightforwardly done. The tradition cannot be openly subverted anymore. Therefore, Pollock's strategy is to diminish the significance of these scholars by exposing the alleged limitations of their analyses. Alternately, he tries to valorize scholars from the {\sl kāvya-śāstra} tradition whose works are more amenable to his project.
\smallskip

\section*{Valorizing Bhoja}

In the great {\sl kāvya-śāstra} tradition of India, there are two scholars who attract Pollock's greatest admiration: firstly, Bhoja, author of {\sl Sarasvatī-kaṇṭhābharaṇa} and {\sl Śṛṅgāra-prakāśa}, and secondly, Bhaṭṭa Nāyaka, author of {\sl Hṛdayadarpaṇa}. In both cases, he laments that the tradition has been most neglectful of these two scholars and he writes with the aim of restoring to them their deserved greatness. One can read in this attempt a way of taking control of the tradition. A rift is suggested by depicting some scholars as having been unjustly ignored by the tradition. By valorising them and, conversely, by downgrading the importance of those scholars, such as Abhinavagupta, whom the tradition itself has considered as significant, one can give a new direction to the tradition.

Evidently, Pollock revels in projecting discrepancies and breaches with\-in the tradition rather than emphasizing continuity and coherence. For example, he (Pollock 1998:119-120) picks up on Sivaprasad Bhattacharyya's remark that ``Bhoja's discourse on rasa is the most detailed and provocative we have, and the most unusual, differing often essentially from both Bharata and those who follow him'' but argues that Bhattacharyya has not ``acknowledged or ... recognized the depth of this disagreement'' and adds further:

\begin{myquote}
As for those who followed Bhoja in time, what neither Bhattacharyya nor anyone else has clearly spelled out is just how fundamental the differences between them are 
\hfill (Pollock 1998:119-120).  
\end{myquote}

The {\sl kāvya-śāstra} tradition is thus projected as lacking in sufficient critical thinking --- a lacuna that Pollock and scholars trained by him will allegedly fill.

Bhoja is held up as the proper representative of the Indian {\sl kāvya-śāstra} tradition and pitted against the Kashmiri scholars, Ānandavardhana and Abhinavagupta, whose works shaped the course of the tradition from the 10th century onwards. Pollock attempts to diminish their significance thus:

\begin{myquote}
The more one works through [Bhoja's] complex analysis, and the stunning range of examples that he seems so effortlessly, and always so appositely, to adduce in support of his argument, the stronger is the impression one gets that, while Kashmiri speculation on the philosophical and theological aesthetics of reader-response is all very fine, it may be Bhoja {\sl who best tells us how literature was made to work in premodern India} 
\hfill Pollock (1998:140) [{\sl italics mine}].
\end{myquote}

Why is Bhoja so important to Pollock and why is he so eager to make him the focal point of our understanding of the {\sl kāvya-śāstra} tradition instead of the Kashmiri scholars whose priority has been established by the tradition itself? The reason is that it is the ``social effects'' of {\sl kāvya}, instead of its language-philosophical or religious-aesthetic dimension, which matters to Pollock. The overall purpose of his interpretation of {\sl kāvya} appears to be a demonstration of how it served to assert political will and maintain an oppressive social structure. Furthermore, in the contemporary post-colonial milieu, it needs to be shown that such a view was self-consciously held by the Indians themselves rather than an interpretation superimposed by a Western lens. Thus, Pollock does not fail to point out that Bhoja, considered himself ``a great king appointed by his elders to protect all that has been inherited, and who in this [first] verse [of the {\sl Śṛṅgāra-prakāśa}] beseeches God that there should be no violation against the established order ({\sl sthita}) and practices of estates and stages of life while he is engaged in the composition of this book'' (Pollock 1998:140).

While the focal point of the {\sl kāvya-śāstra} tradition is generally on the beauty ({\sl saundarya}) of {\sl kāvya} and how it produces aesthetic delight ({\sl āhlāda}) in the reader, Bhoja can be exploited to show (whether he intended it or not) that the purpose of {\sl kāvya} is the maintenance of socio-political order:

\begin{myquote}
The whole point of the [{\sl Śṛṅgāra-prakāśa}], for its part, is to discipline and correct the reading of Sanskrit literature, and by creating readers who thereby come to understand what they should and should not do in the peculiar lifeworld constituted by this literature, {\sl it aims to create politically correct subjects and subjectivities}. 
\hfill Pollock (1998:141) [{\sl italics mine}]
\end{myquote}

According to Pollock, {\sl kāvya} was the product of a courtly-civic ethos which allegedly collapsed in Kashmir from the eleventh-twelfth century onwards. This factor, he claims, brought about the shift from a socio-political aesthetic, which was the norm, to a religio-philosophical aesthetic, which was an aberration:

\begin{myquote}
This was a world rocked by royal depredations, impiety, madness, and suicide, where poets were forced to seek patronage outside the Valley ... or if they remained, began to ridicule the very idea of writing for the court. And it was a world that would eventually, after the twelfth century, permanently terminate Sanskrit literary creativity in Kashmir. One may well ask whether it was this erosion that contributed to the production of the more inward-looking, even spiritualized Indian aesthetic, one that, despite the fact that historically {\sl it constitutes a serious deviation in the tradition}, has succeeded in banishing all other forms from memory
\hfill Pollock (1998:141) [{\sl italics mine}]
\end{myquote}

This thesis has been elaborated in the essay {\sl The Death of Sanskrit} and has been effectively critiqued and demolished by Manogna Sastry in {\sl Pollock's Paper on the Death of Sanskrit} submitted for the first conference of the Swadeshi Indology series.
\eject

To understand the larger context of Pollock's attempt at valorisation of Bhoja and the corresponding diminution of the Kashmir tradition, we must take note that {\sl rasa} as the affective dimension of the literary text can be expressed internally or externally. In the former case, {\sl rasa} is embedded in the text and manifested by the character through his or her actions. In the latter case, it lies in the reader and is manifested by the awakening of the latent mental traces ({\sl vāsana}). According to Pollock, the {\sl kāvya-śāstra} tradition, including Bhoja, understood {\sl rasa} as internal to the text and this understanding persisted until Abhinavagupta decisively transferred the locus externally to the reader. Pollock seeks to valorize the former and diminish the importance of the latter precisely because the former scenario lends itself to a socio-political interpretation which becomes impossible in the latter case wherein {\sl rasa} is concerned with the spiritual development of the reader. Pollock's complaint is that subsequent to this shift the earlier tradition was altogether forgotten and ``the presuppositions derived from the justly admired Kashmiri tradition, especially as promulgated by its most sophisticated representative, Abhinavagupta, ... are often taken to represent {\sl rasa}-doctrine {\sl tout court} and transhistorically'' (Pollock 1998:125-126). He concedes that ``there is little point denying that the Kashmiri innovation produced an analysis of literary experience more engaging both to medieval and contemporary readers'' but he importunes that the earlier tradition was ``a no less serious order of analysis, which awards conceptual primacy to the textual organization of aesthetic effects rather than to those effects themselves'' (Pollock 1998:138). Eventually, he concludes:

\smallskip

\begin{myquote}
{{\sl If ... there is a glaring fault to be found in the Indian tradition ... it may rather be that of the Kashmiri thinkers}}. For what they left out in their analysis of reader response was the possibility of difference --- the problem that preoccupied Kant, how a judgement of taste is rationally justified, cannot be asked if all {\sl sahṛdayas} qua {\sl sahṛdayas} respond the same, as they appear to do for Abhinava --- and all the troublesome issues, such as authorial intention and the conflict of interpretations, that hang on such difference 
\hfill Pollock (1998:139) [{\sl italics mine}]
\end{myquote}

Bhoja, on the other hand, suggested that not only words and sentences but a literary text as a whole is endowed with an ultimate meaning that is a command. For example, in case of the Rāmāyaṇa it is to exhort the reader to be like Rāma and not like Rāvaṇa. To this end he approved of historical narratives being revised to achieve the desired effect. For example, in Bhavabhūti's {\sl Mahāvīracarita}, Vāli is slain by Rāma after a provocation. Bhoja classified passion into four types as drives towards the various {\sl puruṣārthas --- dharma, artha, kāma} and {\sl mokṣa} --- and indicated that the hero {\sl inter alia} should be depicted according to his type of passion: dignified in case of {\sl dharma}, energetic in case of {\sl artha}, romantic in case of {\sl kāma}, and serene in case of {\sl mokṣa}. The hero is, above all, a moral agent, and Bhoja's response to the controversy regarding the depiction of the virtuous enemy is worth noting. This was a matter of debate in literary theory because if the villain is described as a man of great character than what message would his eventual destruction send to the audience? On the other hand, if the villain was a man of a flawed character than that would be the cause of his destruction, and not the manly efforts of the hero. Bhoja's solution to the conundrum was that he must be shown as both flawed and virtuous --- the flaws do become the cause of his destruction but they do not become relevant in the actual combat with the hero. There the virtues of the villain dominate and reflect the glory of the hero in defeating him. Thus, Pollock concludes, ``the greatness of the hero is not just an aesthetic condition, but a social and a moral one'' (2001:222). As we can see, Bhoja is dear to Pollock because in explicitly forging the connection between the social-moral and the literal, he ``illustrates just how self-consciously literary theory could recapitulate social theory'' (2001:223) --- a development which was unfortunately arrested by the linguistic turn of Ānandavardhana and the theological emphasis on readerly {\sl rasa} by Abhinavagupta.

What is troublesome here is the manner in which Pollock has divided the tradition of Sanskrit literary theory into two fundamentally opposed camps, one in which the social-moral aesthetic enjoyed a privileged status and the other in which it remained subordinate. The impression is then created that not only does the former represent the original thinking of the tradition but that the latter suffered from a pathology of self-deception in that even here the social-moral aesthetic formed the ground but it became occluded. The paradigmatic nature of the gulf between them is driven home using such melodramatic language as ``the episteme that Abhinava successfully overthrew'' (Pollock 2001:211) or ``a new mentality produced in large part by the remarkable achievements of literary theory in Kashmir'' (Pollock 2001:198) and so on. What overthrow? What new mentality? While Ānandavardhana and Abhinavagupta are certainly the heroes of the tradition of Sanskrit literary theory, it does not appear that anyone within the tradition itself would have spoken about the significance of their works in such a way. For example, consider how Pollock explains the revolutionary nature of {\sl dhvani} theory:  

\begin{myquote}
Ānandavardhana makes a claim for scientific innovation that, viewed from a purely intellectual-historical perspective, is perhaps without precedent in India. He declares he intends to analyse a feature of literary speech that all sensitive readers grasp but that no one before him, because of its subtlety and complexity, has yet been able to theorize. 

\hfill Pollock (2001:200)
\end{myquote}

Yet what Ānandavardhana has himself stated in the opening {\sl śloka} of his work is that:

\begin{myquote}
Some have said that the soul of poetry, which has been handed down from the past by wise men as ``suggestion'' ({\sl dhvani}), does not exist; others, that it is an associated meaning ({\sl bhākta}); while some have said that its nature lies outside the scope of speech: of this [suggestion] we shall here state the true nature in order to delight the hearts of sensitive readers. 
\hfill (Ingalls 1990:47)
\end{myquote}

Just compare these two paragraphs and note for yourself the difference in the suggestion that is implicit between them. This is the problem with Pollock's literary style. He is not uttering a lie but he tweaks the truth so subtly as to completely distort its meaning. Thus, there is something deeply problematic about the way in which Pollock has divided the tradition of Sanskrit literary theory into two contesting streams, a classification that is not recognized within the tradition itself. It is analogous to the manner in which Western scholars divided the languages of India into two contesting Indo-Aryan and Dravidian families, while traditional linguistics had organized them into Prakṛta-s of different regions and Sanskrit. We are yet suffering from the repercussions of that Western intervention. Since we are dealing with an isolated and specialized topic, the mischief is likely to be contained in case of the division Pollock has fabricated in the tradition of Sanskrit literary theory, but it partakes of the same nature. Here, Bhoja is portrayed as the final upholder of a dying tradition which affirmed the central significance of a social-moral aesthetic in literary theory. And we are left with the impression that it should not have died out because it was honest to the social agenda of literature (which was, of course, the perpetuation of caste and gender oppression) and so in recovering it, we would be correcting a great historical wrong in the tradition of Sanskrit literary theory.

Yet, the delicious irony in this persistent effort of raising the estimation of Bhoja and diminishing that of Abhinavagupta and Ānandavardhana is that Bhoja's interpretation of {\sl rasa} was no less religious in nature than that of Abhinavagupta. Neal Delmonico (2016), author of an important book on Rūpa Gosvāmin, notes that while Gosvāmin's understanding of {\sl rasa} as sacred rapture is usually ascribed to the tradition of Abhinavagupta, there were important differences between them. On the other hand, it appears that Gosvāmin derived his religious aesthetics from Bhoja. As Delmonico (2016:viii) explains:

\begin{myquote}
A little more digging has revealed that a healthy variety of viewpoints on rasa existed throughout the period between Abhinavagupta and Rūpa and among those viewpoints Bhoja's was an important contender. Bhoja's work inspired and influenced a number of later writers, mostly in South India, and was incorporated into parts of a Purāṇa (the {\sl Agni Purāṇa}), the area of the dissemination of which was centred in eastern India (Bengal and Orissa). It is suggestive to note that, although Abhinavagupta's notion of {\sl rasa} eventually became the dominant one among literati throughout India, Bhoja's view bears a fairly strong resemblance to more popular views of aesthetics still extant in India.
\end{myquote}

Thus, it is not only possible to view Bhoja's {\sl rasa} theory as the precursor to an important tradition of religious aesthetic, such as developed by Śrī Rūpa Gosvāmin, it is also evident that differences between the aesthetic views of Abhinavagupta and Bhoja can be understood sympathetically, without pitting them against each other, as Pollock does. It is not that Pollock is unaware of this research. He refers to Delmonico's work but only as far as it ``correctly acknowledges, in a couple of places, Bhoja's focus on the literary character as the locus of rasa'' (Pollock 1989:118). But he regrets that Delmonico ``does not apply this in his exegesis of the work'' and that ``in the rest of his analysis I cannot follow him'' (Pollock 1989:118). Thus, Pollock wilfully ignores any research that contradicts his thesis.

\section*{Valorizing Bhaṭṭa Nāyaka}

The transformation in {\sl rasa} theory which shifted the locus of {\sl rasa} from the text to the reader, which Pollock seeks to problematize as part of his project of desacralization of {\sl kāvya}, begins with Ānandavardhana in the 9th century. Abhinavagupta, in his commentary on Ānandavardhana's {\sl Dhvanyāloka}, critically examined the {\sl rasa} theories of other thinkers, chiefly that of Bhaṭṭa Nāyaka, and then proposed his own view of readerly {\sl rasa}. As Pollock himself admits, subsequent scholars in the {\sl kāvya-śāstra} tradition like Jagannātha Paṇḍitarāja did not find any essential difference between the views of Bhaṭṭa Nāyaka and Abhinavagupta, only a change in language. But it was a significant change since the aim of Bhaṭṭa Nāyaka's critique was to demolish the {\sl rasa-dhvani} theory of Ānandavardhana, which Abhinavagupta saved by means of his counter-critique. 

Pollock is, as explained above, interested in projecting two rival schools within the {\sl kāvya-śāstra} tradition, one which interpreted {\sl rasa} as internal to the text and the other which located {\sl rasa} in the reader. The former is amenable to Marxist literary theories and Pollock seeks to make it the dominant view. Bhaṭṭa Nāyaka's role in this project is a bit complex. Ānandavardhana's {\sl rasa-dhvani} theory accords with the former position which Bhaṭṭa Nāyaka tried to demolish and in doing so established the latter position by transferring the locus of {\sl rasa} to the reader. So one would expect Pollock to support Ānandavardhana and criticize Bhaṭṭa Nāyaka. But what happened is that Abhinavagupta reinterpreted Bhaṭṭa Nāyaka's argument in a way that affirmed Ānandavardhana's view. In effect, then, Abhinavagupta made Ānandavardhana's {\sl rasa-dhvani} theory support the latter position though it was not its original aim. This is how Pollock has interpreted this slice of {\sl kāvya-śāstra} history and Bhaṭṭa Nāyaka, though he supports the latter position which is contrary to Pollock's interest, becomes his friend as with his aid, two birds can be killed with one stone: Abhinavagupta, whose aesthetic theory bestowed upon literary {\sl rasa} a spiritual dimension, and Ānandavardhana, whose {\sl rasa-dhvani} theory unintentionally became the source for the shift in the locus of {\sl rasa}. No wonder then that merely on the basis of its few surviving fragments, Pollock hails the {\sl Hṛdayadarpaṇa} of Bhaṭṭa Nāyaka as a ``masterpiece'' (Pollock 2012:233), thus suggesting that the {\sl kāvya-śāstra} tradition failed to understand itself and recognize the merit of its scholars. The implication is that Western intervention is necessary to write the proper history of the {\sl kāvya-śāstra} tradition and restore its true genius.

As an adherent of the Mīmāṁsā school, the purpose of Bhaṭṭa Nāyaka's critique was to demolish the {\sl dhvani} theory of Ānandavardhana, which depended on the {\sl vyañjanā-śakti} (suggestive power) of language, and make the {\sl abhidhā-śakti} (denotative power) of language the essential factor in the production of {\sl rasa}. As Pollock explains, up to the time of Bhaṭṭa Nāyaka, it was held that {\sl rasa} became manifest in the character of the drama and was relished by the spectator. However, the interpretation of {\sl rasa} using Mīmāṁsā theory of language led Bhaṭṭa Nāyaka to propose that {\sl rasa} was produced ({\sl bhāvanā}) in the spectator himself. Abhinavagupta reinterpreted this process in favour of Ānandavardhana's {\sl dhvani} theory by proposing that the spectator experienced {\sl rasa} on account of his own latent predispositions, and that is how later tradition understood it. On this account, Pollock refers to him as Bhaṭṭa Nāyaka's ``most ardent if most reluctant if not ungrateful disciple'' (2010:157). We need not go into the tortured arguments employed by Pollock for demonstrating that it was Bhaṭṭa Nāyaka who pioneered the hermeneutic shift from text to reader and that it involved a production ({\sl bhāvanā}) and not a manifestation ({\sl vyakti}) of {\sl rasa} in the reader, though later tradition was to conflate the two due to the reformulation of Abhinavagupta. What we need to understand is where Pollock is going with this thesis. The production of {\sl rasa} in the character (for Ānandavardhana) or in the spectator (for Bhaṭṭa Nāyaka) was in either case a matter of {\sl śabda-vṛtti} (linguistic modality) internal to the text and was transformed into a {\sl cid-vṛtti} (psychological modality), external to the text and manifested in the spectator, due to the reinterpretation of Abhinavagupta. This {\sl cid-vṛtti} is the basis for a religious aesthetic and that is what Pollock is trying to deny here by questioning the legitimacy of what Abhinavagupta has done. Furthermore, in the essay {\sl Vyakti and the History of Rasa} (2012) he takes the later scholars of the {\sl kāvya-śāstra} tradition to task for never having questioned it themselves. He thus puts the credibility of the subsequent {\sl kāvya-śāstra} tradition at stake in order to suggest that Abhinavagupta's religious aesthetic was born out of an attempt at ``turning his opponent's [i.e. Bhaṭṭa Nāyaka's] weapons directly against him'' (Pollock 2012:240).
\eject

In this essay, we are not really concerned with the actual views of Bhoja or Bhaṭṭa Nāyaka but Pollock's interpretation of them and its pernicious implication with regards to the religious aesthetics of the Indian {\sl kāvya-śāstra} tradition. The locus of {\sl rasa} can be understood as internal to the text or as lying externally in the reader's response to the text. Religious aesthetics in the Indian context arose when {\sl rasa} was understood as a psychological modality in the reader and this is the interpretation whose significance Pollock seeks to diminish as it is not amenable to an understanding of {\sl kāvya} as a socio-political aesthetic. For this purpose, {\sl rasa} needs to be understood as a linguistic modality and hence Pollock valorizes the two {\sl kāvya-śāstra} scholars whose works can be interpreted in this way. In the case of Bhoja, {\sl rasa} is internal to the text and located in the character; in case of Bhaṭṭa Nāyaka it is external to the text and produced in the reader, but it is still a matter of linguistic modality. 

\section*{Conclusion}

There was a time when Indology used to be a record of the Western experience of Indian texts and traditions, viewed through the lens of what are now pejoratively termed as Christocentric or Eurocentric categories. The scientific objectivity that it claimed for itself is condescendingly dismissed now as Orientalism, the Western imagination of India. The new intellectual orthodoxy of the post-modern, post-colonial world idealizes {\sl emic} studies and seeks to understand native culture from the perspective of the native. The contemporary Western scholar is interested in studying the cultural artefacts of a tradition in the manner in which they were received by the tradition. Now in the case of Sanskrit literature this process would be influenced by Sanskrit literary theory. However, the aspect of the production and reception of text that is of interest to Western scholars, especially of the Marxist strain, is social knowledge but it appears that the Sanskrit literary theory itself is more interested in language philosophy and emotive experience. So the question arises: what was the significance of the social-moral aesthetic in literature? How effective was the ``social effect'' of texts? If it was not a major concern for the Indians themselves, then the Western analysis of the social reception of texts by the contemporary audience is just Orientalism of a different kind.

That is, of course, what it really is but one needs to provide some kind of a cloak for it so it can pass muster. This is what I think Pollock seeks to accomplish by the recuperation of the socio-political in Sanskrit literary theory. Now as he himself shows the social was always evident in it from Bharata to Jagannātha Paṇḍitarāja. The problem, however, is that its importance was eclipsed by the emphasis on language-philosophy and emotive experience. Therefore, one might argue that such was the priority of the Indians and as good post-modern, post-colonial citizens we should respect that priority and interpret their literature in the manner in which they perceived as fit to be interpreted. 

It is here that Pollock's writings turn nasty. They strive to make the point that language-philosophy and emotive experience were themselves grounded in a socio-polity but the scholars who prioritized them remained oblivious to that socio-political ground. In other words, Pollock's writings are challenging the very legitimacy of the significance that Indians have historically attached to language-philosophy and emotive experience in the case of literary criticism. At the same time, they are also valorising a sociological hermeneutics, which is a Western priority but can now be postulated as a long-suppressed priority of the Indians as well. It is thus paving the way for a new kind of literary criticism, a new form of knowledge production, in which Sanskrit literary texts can be interpreted not in terms of their linguistic content or the emotive states they affect, but as promoting social causes, namely the sustenance of caste and gender hierarchies, and this whole study can remain free of the charge of Orientalism because --- and this is the {\sl pièce de résistance} --- it can be presented as a hermeneutics sanctioned by the Indians themselves.
\vskip 2pt

In conclusion we may note that Pollock is not alone in desacralizing {\sl kāvya} and seeking to eliminate the religious dimension of {\sl rasa}. Indian-born Saam Trivedi, Associate Professor of Philosophy at Brooklyn College, in his article {\sl Evaluating Indian Aesthetics} published online by the American Society for Aesthetics, declares that:
\smallskip

\begin{myquote}
I will set aside later commentators on Bharata (such as the tenth and eleventh century CE Kashmir Shaivite Abhinavagupta), for there is reason to think that many of these later writers may have given a religious and cosmological twist to what is at core an aesthetic theory and can be understood as such, quite apart from religion; here I disagree with writers such as Susan Schwartz who suggests that the goal of Indian aesthetics is to facilitate religious transformation. 
\hfill Trivedi (2013)
\end{myquote}

For an essay aimed at explaining the contemporary relevance of Indian {\sl rasa}, this is an unfortunate choice. In this, the author follows the work {\sl Sanskrit Criticism} by V. K. Chari who states:

\begin{myquote}
Some recent exponents, notably, A. K. Coomaraswamy, K. C. Pandey, and J. L. Masson, have given needlessly metaphysicized accounts of Indian aesthetics. Following such accounts, many people in the West have the impression that Indian art and art theories have to be studied only in their religious, transcendental setting. But ... Sanskrit criticism --- at any rate, the mainstream of it --- had nothing to do with religion or metaphysics. 
\hfill Chari (1993:6)
\end{myquote}

While some Indian scholars, presumably of dharmic persuasion, are thus keen to exclude the religious dimension in their understanding of {\sl rasa}, on the other hand, it is being appropriated by Indian Christians as a vehicle for spreading the gospel of Jesus. In {\sl Tasting the Divine: The Aesthetics of Religious Emotion in Indian Christianity}, Michelle Voss Roberts notes how Christian Bharat Nāṭyam dance organisations, {\sl Nava Sadhana Kala Kendra} in Varanasi and {\sl National Biblical Catechetical and Liturgical Centre} (NBCLC) in Bangalore, have incorporated {\sl rasa} in their evangelization and inculturation project. She explains how Abhinavagupta described the religious significance of {\sl rasa}:

\begin{myquote}
Abhinavagupta's {\sl Locana}, a commentary on Ānandavardhana's {\sl Dhvanyāloka}, develops an analogy between aesthetic relish ({\sl rasāsvāda}) and the experience of brahman ({\sl brahmāsvāda}). Like the experience of {\sl brahman, rasa} is a transcendent, universalizable, and blissful state of mind. It is unlike mundane emotions and experiences ({\sl alaukika}). Abhinavagupta calls {\sl rasa} a taste of the union of one's own nature with the divine ({\sl Locana} 2.4). In the moment of aesthetic bliss, one forgets oneself. Total immersion brings a temporary suspension of subject-object distinction, worldly concern, and sense of ordinary time and space. 
\hfill Roberts (2012:578)
\end{myquote}

Rūpa Gosvāmin further developed on this idea such that {\sl rasa} came to ``resemble the bliss of the Absolute, it is that very power of bliss, the {\sl hlādinī-śakti} of Kṛṣṇa himself, which manifests in the devotee'' (Roberts 2012:579). Roberts points out that ``later Indian Christians presuppose these two changes [of Abhinavagupta and Rūpa Gosvāmin] as they construct their devotional love for Christ'' and through the dance drama form of Bharat Nāṭyam, which has rasa as its basis, ``NBCLC and Nava Sadhana ... foster a Christian bhakti {\sl rasa}'' (Roberts 2012:579). We thus find that, one the one hand, through the works of Pollock and other scholars, the Indian {\sl rasa} tradition is under pressure of desacralization, while, on the other hand, it is being appropriated by other sacred traditions for the spread of their religious discourse.

\begin{thebibliography}{99}
\itemsep=2pt
\bibitem[]{chap2_item1}
Chari, V. K. (1993). {\sl Sanskrit Criticism}. Delhi: Motilal Banarasidass.

\bibitem[]{chap2_item2}
Delmonico, Neil. (2016). {\sl Sacred Rapture: A study of the Religious Aesthetic of Śrī Rūpa Gosvāmin}. Unpublished dissertation, University of Chicago.

\bibitem[]{chap2_item3}
Ingalls, Daniel H. H., Jeffrey Moussaieff Masson, and M. V. Patwardhan. (1990). {\sl The Dhvanyaloka of Anandavardhana with the Locana of Abhinavagupta}. Harvard University Press.

\bibitem[]{chap2_item4}
Ingalls, Daniel H. H. (1965). {\sl An Anthology of Sanskrit Court Poetry}. Harvard University Press.

\bibitem[]{chap2_item5}
Michelle Voss Roberts. (2012). ``Tasting the Divine: The Aesthetics of Religious Emotion in Indian Christianity,'' {\sl Religion} 42.4 (2012). pp 575-595.

\bibitem[]{chap2_item6}
Mitra, Arati. (1989). {\sl Origin and Development of Sanskrit Metrics}. Calcutta: The Asiatic Society.

\bibitem[]{chap2_item7}
Pollock, Sheldon. (1998). ``Bhoja's {\sl Śṛṅgāra-prakāśa} and the Problem of Rasa: A Historical Introduction and Translation.'' {\sl Asiatische Studien} 70, 1. pp 117-192. 

\bibitem[]{chap2_item8}
---\kern3pt(2001). The Social Aesthetic and Sanskrit Literary Theory. {\sl Journal of Indian Philosophy} 29, 1-2. pp 197-229.

\bibitem[]{chap2_item9}
---\kern3pt(2006). {\sl The Language of the Gods in the World of Men: Sanskrit, Culture and Power in Premodern India}. Berkeley: University of California Press.

\bibitem[]{chap2_item10}
---\kern3pt(Ed.) (2010a). {\sl Epic and Argument in Sanskrit Literary History}. Delhi: Manohar.

\bibitem[]{chap2_item11}
---\kern3pt(2010b). ``What was Bhaṭṭa Nāyaka Saying? The Hermeneutical Transformation of Indian Aesthetics.'' In Pollock (2010a). pp 143-84.

\bibitem[]{chap2_item12}
---\kern3pt(2012). Vyakti and the History of Rasa. {\sl Vimarsha, Journal of the Rashtriya Sanskrit Sansthan} (World Sanskrit Conference Special Issue), 6. pp 232-253.

\bibitem[]{chap2_item13}
Trivedi, Saam (2013). ``Evaluating Indian Aesthetics''. \url{http://aesthetics-online.org/?page=TrivediIndian}. Accessed 1 Nov 2016.
\end{thebibliography}


\label{chapter\thechapter:end}
