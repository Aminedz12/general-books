\chapter[The West on Our Poems: A Critique]{The West on Our Poems: A Critique}\label{chapter\thechapter:begin}
%~ \endnotetext[1]{pp.~\pageref{chapter\thechapter:begin}--\pageref{chapter\thechapter:end}. In: Kannan, K S (Ed.) (2018) {\sl Śāstra-s Through the Lens of Western Indology -- A Response}. Chennai: Infinity Foundation India.}

\Authorline{Shankar Rajaraman}
\lhead[\small\thepage\quad Shankar Rajaraman]{}

\section*{Abstract} 

Any Pūrvapakṣa of Western Sanskrit scholarship needs to take multiple approaches. Each of the available approaches has its own place in the larger picture. Critiquing Western understanding of Sanskrit Kāvya\index{kavya@\textsl{kāvya}} literature is one such approach. In this paper, I examine 20 examples of mistranslations, followed by two of faulty editing, and one of misanalysis by Western Sanskrit scholars. I also suggest a method of classifying the mistranslations under different heads based on the probable causes underlying them. I conclude with a short discussion on how Western Indologists must approach Sanskrit Kāvya literature. 

\medskip
\noindent
{[{\bf Key words}: 
Western Sanskrit scholarship, Sanskrit \textsl{Kāvya} literature, mistranslations, editing errors, misanalyses, Clay Sanskrit Library.]}\index{Clay Sanskrit Library} 

\section*{Introduction}  

In his four-tier model of critiquing Western Indology, Malhotra\index{Malhotra, Rajiv} (2016) explains Tier 4 as pertaining to a study of how specific Sanskrit verses are analyzed by Western Indologists \textsl{vis-a-vis} traditional Sanskrit scholars. He calls upon traditional scholars to play a role at this level noting, however, that they would be handicapped without familiarizing themselves with Tiers 2 and 3 --- tiers that are respectively about the networking strategies and theoretical frameworks employed by Western Indologists. 

In this paper, I shall demonstrate how traditional scholarship in Sanskrit can equip us with the analytical tools that are helpful not just in understanding Sanskrit texts correctly, but also in detecting instances where such understanding is inherently flawed. In other words, traditional Sanskrit scholars can summon theoretical frameworks which they are familiar with (rather than those they see as alien) in order to defend their texts against the sort of misanalysis which Western scholars subject them to. It is no doubt important that we have a thorough understanding of Western Indological thought. But deconstructing such thought must be seen as a process in which traditional knowledge (knowing our view) and knowledge of Western theories (knowing the opponent’s view) are complementary to one another, rather than the former being seen as contingent upon the latter. 

How Western Indologists analyze Sanskrit texts is closely linked to how they understand them, and how they understand them is reflected in how they translate them. Faulty understanding leads to flaws at the level of analysis and translation. With this background in mind, I shall examine 20 examples of mistranslations, two of editing errors, and one of misanalyses that reveal shortcomings in understanding Sanskrit \textsl{kāvya}\index{kavya@\textsl{kāvya}} texts. This paper is a consolidation of what I have already discussed in my blog (\url{https://hrdayasamvada.wordpress.com/}). All examples of mistranslations and editing errors that I examine have been selected from Clay Sanskrit Library's\index{Clay Sanskrit Library} publication series. According to the homepage (\url{http://www.claysanskritlibrary.org/}) of Clay Sanskrit Library (henceforth abbreviated in this article as \textbf{CSL}), it is “a series of books covering a wide spectrum of Classical Sanskrit literature spanning two millennia”. The homepage informs us that fifty-six volumes have been published under the banner since 2005. Among those associated with CSL\index{Clay Sanskrit Library} are Prof. Sheldon I. Pollock\index{Pollock, Sheldon} \hbox{(General Editor)},\break \hbox{Isabelle} Onians (Editor), and a panel of Translators such as Yigal Bronner,\index{Bronner, Yigal} Wendy Doniger,\index{Doniger, Wendy} Friedhelm Hardy,\index{Hardy, Friedhelm} Matthew Kapstein,\index{Kapstein, Matthew} Sir James Mallinson,\index{Mallinson, Sir James} David Shulman,\index{Shulman, David} and Gary Tubb,\index{Tubb, Gary} apart from the General Editor and the Editor themselves (\url{http://www.claysanskritlibrary.org/people.php}). I have included in this list the names of only those Western Sanskrit scholars whose work I shall discuss shortly. For\break examining misanalysis of Sanskrit verses, I shall refer to one example from the book Tubb (2014).\\[-21pt]

\section*{Examining Mistranslations}

In this section, I shall examine the English translations of 20 Sanskrit verses selected randomly from CSL\index{Clay Sanskrit Library} series publications. These translations can be classified thus on the basis of the type of their flaws: 
\begin{itemize}
\itemsep=2pt
\item[(a)] getting the narrative wrong; 
\item[(b)] being unfamiliar with the Indian cultural ethos; 
\item[(c)] being unfamiliar with complementary bodies of knowledge that Sanskrit \textsl{kāvya-s}\index{kavya@\textsl{kāvya}} draw upon; 
\item[(d)] getting the semantics wrong --- at the levels of individual words, compound words, sentences/phrases 
\item[(e)] failing to spot one or more puns that are important for making an overall sense of a Sanskrit verse. 
\end{itemize}

In any given instance, more than one reason may also operate. In some cases, I shall focus on explanatory notes rather than translations \textsl{per se}. I take such notes to be the logical extensions of translations because firstly, they reveal how the translator has understood the meaning of a verse over and above what he/she has translated; and secondly, how they proceed to fill gaps in the reader’s comprehension of a verse. 

\subsection*{(a) Getting the Narrative Wrong} 

Literary narratives give verbal form to a series of events (Snaevarr 2010). Such a form is characterized by \textsl{coherence} (there is some sort of causal or other types of connection between the events in a narrative), \textsl{meaningfulness }(it is possible to make sense of the way in which the narrative’s narrator and internal characters understand the events), and sometimes, \textsl{emotional import} (the narrative captures its narrator’s and internal characters’ evaluation of and emotional responses to the events) (Goldie 2004). Narratives are not just the products of culture. Culture also provides the framework within which narratives become meaningful (Brockmeler 2012). From an Indian aesthetic viewpoint, narratives can be understood in terms of the \textsl{emplotment} of \hbox{\textsl{vibhāva-s}}\index{vibhava@\textsl{vibhāva}} (Antecedent Events), \textsl{anubhāva-s}\index{anubhava@\textsl{anubhāva}} (Consequent Responses including verbal and non-verbal behaviours), and \textsl{vyabhicāri-bhava-s}\index{vyabhicaribhava@\textsl{vyabhicāribhāva}} (Transient States such as \textsl{garva}\index{garva@\textsl{garva}} (pride), \textsl{asūyā}\index{asuya@\textsl{asūyā}} (envy), \textsl{śrama}\index{srama@\textsl{śrama}} (fatigue), \textsl{vyādhi}\index{vyadhi@\textsl{vyādhi}} (physical illness), \textsl{viṣāda}\index{visada@\textsl{viṣāda}} (despondency)). Put simply, Sanskrit poets integrate \textsl{vibhāva-s}, \textsl{anubhāva-s}, and \textsl{vyabhicāri-bhava-s} in a coherent and meaningful manner within a narrative. The effect of emplotment on the reader is that his/her \textsl{sthāyi-bhāva-s}\index{sthayi-bhava@\textsl{sthāyibhāva}} (sustained egocentric mental states such as \textsl{rati}, \textsl{utsāha}\index{utsaha@\textsl{utsāha}} (perseverance), \textsl{śoka}\index{soka@\textsl{śoka}} (sorrow)) are transformed into \textsl{rasa}-s -- their pleasurable, aesthetic counterparts. According to the \textsl{Nāṭyaśāstra}\index{Natyasastra@\textsl{Nāṭyaśāstra}} of Bharatamuni,\index{Bharatamuni} dramatic narrative (\textsl{nāṭya}) must refer to the actual world for its depiction of antecedent events and consequent responses. \textsl{Vibhāva-s}\index{vibhava@\textsl{vibhāva}} and \textsl{anubhāva}-s\index{anubhava@\textsl{anubhāva}} thus have their real world correspondences in the form of \textsl{kāraṇa-s}\index{karana@\textsl{karaṇa}} and \hbox{\textsl{kārya-s}}\index{karya@\textsl{kārya}} (stimuli and responses). To know \textsl{vibhāva-s} and \textsl{anubhāva-s} is to know their corresponding real world \textsl{kāraṇa-s} and \textsl{kārya-s}. \hbox{\textsl{Vibhāva}-s}\index{vibhava@\textsl{vibhāva}} and \textsl{anubhāva}-s\index{anubhava@\textsl{anubhāva}} are therefore described by Bharatamuni\index{Bharatamuni} ((Dvivedi 1996:153) as \textsl{loka-svabhāvānugata} (compatible with what holds true in the actual world), \textsl{loka-prasiddha} (well-established in the actual world), \textsl{loka-svabhāva-saṁsiddha} (determined by what holds true in the actual world), and \textsl{loka-yātrānugāmi} (in agreement with the world of interactions). The word \textsl{loka} (world) use here refers, no doubt, to a cultural world within which \textsl{nāṭya}\index{natya@\textsl{nāṭya}} is made meaningful. 


The translator must have firsthand experience of being in a culture that endows his target narrative text with meaningfulness. If not, the translated narrative will be a poor recount of the events depicted in the original text. 


With this background, I shall examine some examples of translations\index{errors in translation!mistranslations!getting narrative wrong} that point to an improper understanding of Sanskrit versified narratives.  


1.~Hardy’s\index{Hardy, Friedhelm} (2009) translation of verse no.\@ 48 from the \textsl{Āryā-saptaśatī}\index{Aryasaptasati@\textsl{Āryāsaptaśatī}} of Govardhanācārya:\index{Govardhanacarya@Govardhanācārya} 

The verse is about a lover whose lady has purposefully kept his upper garment (\textsl{uttarīya}) with herself during his previous visit so she has the pleasure of seeing him once more when he returns to take it back. His friends ask him to stop feeling sorry for being left with a single garment because the very fact that his beloved confiscated his upper garment proves the extent to which she loves him. Given below are the original Sanskrit verse and its English translation by Hardy. 
\begin{quote}
\textsl{apy ekavāsasas tava sarva-yuvabhyo’dhikā śobhā}  ||\\
\textsl{anurakta-rāmayā punar-āgataye sthāpitottarīyasya} |
\end{quote}

\begin{myquote}
“Even though you go barebodied, wearing but a single garment, to your rendezvous with an infatuated woman, you look finer than all other young men” 
\hfill(Hardy\index{Hardy, Friedhelm} 2009:43)
\end{myquote}

Hardy seems to have taken \textsl{sthāpitottarīyasya }as an independent word and translated it as “barebodied” instead of construing it along with (\textsl{anurakta}-)\textsl{rāmayā}  “he whose upper garment has been retained” (by his beloved). Without this bit, the translated narrative does not do justice to the original narrative because there is no explanation for why the lover has to go barebodied.

2.~Mallinson’s\index{Mallinson, Sir James} (2006) translation of verse no. 37 from the \textsl{Pavana-dūta}\index{Pavanaduta@\textsl{Pavana-dūta}} of Dhoyī:\index{Dhoyi@Dhoyī}  

The scene of this verse is set in Vijayapura, the capital city of the hero, a king from the Sena dynasty. The poet describes a situation in which the women of the city are playing hide-and-seek with their lovers on the attics of the city's mansions. These women are as beautiful as the bracket figures carved on the walls of the attics and would scarcely be found out by their lovers if they were to hide among them. However, there still is a giveaway. If their lovers, in the course of their search, would perchance touch their beloveds, the latter would have goosebumps on their limbs and could thus be found out. Mallinson's\index{Mallinson, Sir James} translation doesn't do justice to the sequence of events in this narrative. I refer below to his translation along with the Sanskrit original.
\begin{quote}
\textsl{yat-saudhānām upari valabhī-sālabhañjīṣu līnāḥ}\\
\textsl{susnigdhāsu prakṛti-madhurāḥ keli-kautūhalena} |\\
\textsl{unnīyante katham api rahaḥ pāṇi-paṅkeruhāgra-}\\
\textsl{sparśodgacchat-pulaka-mukulāḥ subhruvo vallabhena} ||
\end{quote}


\begin{myquote}
"Where, in attics atop mansions, gorgeous girls of artless beauty keen for some fun play hide-and-seek among lovely wooden statues and are discovered only when the touch of the petals  of the lotuses held in their hands makes the hair on their lovers' bodies stand on end" 

\vskip .1cm

\hfill (Mallinson\index{Mallinson, Sir James} 2006:127)
\end{myquote}

According to this translation, the women who are hiding have lotuses in their hands (this is what Mallinson understands from the Sanskrit \textsl{pāṇi-paṅkeruha} though a Sanskrit compound formed of words for hand and lotus most often than not refers to a hand that is soft, pretty, and so on, hence comparable to a lotus; rather than to a lotus held in the hand) and when the petals of these lotuses touch their husbands who are searching for them, it is they (the husbands) that have goosebumps! Carrying the absurdity\index{errors in translation!mistranslations!getting narrative wrong} further, the translator makes the claim that the ladies who are hiding are discovered by their husbands when the hair on the latter's bodies stand on end.

Firstly, the ladies would be careful not to reveal their presence to their husbands and would therefore not allow the lotuses they are carrying (even if this wrong translation of \textsl{pāṇi-paṅkeruha} is accepted as correct for the sake of argument) to touch the bodies of their husbands who are searching for them. Secondly, how could it be possible that the goosebumps on one person's body give away the presence of someone else? 

3.~Pollock’s\index{Pollock, Sheldon} (2009) translation of verse no. 5. 23 from the \textsl{Rasataraṅgiṇī}\index{Rasatarangini@\textsl{Rasataraṅgiṇī}} of Bhānudatta:\index{Bhanudatta@Bhānudatta} 

This verse illustrates the \textsl{vyabhicāri-bhāva}\index{vyabhicaribhava@\textsl{vyabhicāribhāva}} of \textsl{apasmāra}\index{apasmara@\textsl{apasmāra}} (epileptic seizure). To understand the verse, one must be conversant with a minor episode in the \textsl{Rāmāyaṇa}\index{Ramayana@\textsl{Rāmāyaṇa}} in which Bharata\index{Bharata (Rama's brother)@Bharata (Rāma's brother)} shoots an arrow at Hanumat\index{Hanumat} when the latter is returning to Laṅkā carrying the mountain Droṇagiri in his hand. As the mountain falls from the hand of Hanumat, the trees on the ground shake as if out of fear. The poet fancies that the trees had a bout of seizure on seeing the falling mountain. Instead of attributing \textsl{apasmāra} to the trees on the ground, Pollock\index{Pollock, Sheldon} makes mountains (? on the ground) the subject of \textsl{apasmāra}.\index{apasmara@\textsl{apasmāra}} He thus translates “\textsl{apasmāraṁ} \textsl{dadhur bhūruhāḥ}” as “the mountains seemed possessed” (Pollock 2009:227)\endnote{The full verse is: 

\textsl{udvelan-nava-pallāvadhara-rucaḥ paryasta-śākhā-bhujaḥ}\\
\textsl{sphūrjat-koraka-phena-bindu-paṭala-vyākīrṇa-deha-śriyaḥ |}\\
\textsl{bhrāmyad-bhṛṅga-kalāpa-kuntala-juṣaḥ śvāsānilotkam}\\
\textsl{śailaṁ prekṣya kaper nipātitam apasmāraṁ dadhur bhūruhāḥ ||}}. 

Not only does Pollock’s translation alter the narrative considerably, but it is also faulty on other accounts. The translator has either carelessly rendered the word “\textsl{bhūruhāh}” as “the mountains”, or is confused about who the subject in this verse is. Because of this confusion, all compound adjectives that are actually applicable to trees are made to qualify mountains.  Furthermore, two of these adjectives are translated wrongly in a way that does not reflect a correct cultural understanding of the behavior that results from \textsl{apasmāra}.\index{apasmara@\textsl{apasmāra}} Thus \textsl{udvellan-nava-pallavādhara-rucaḥ} and \textsl{paryasta-śākhābhujāḥ} have become “the \textsl{pallava} buds their swelling lower lips” (Pollock\index{Pollock, Sheldon} 2009:227) and “the tangled branches their arms” (Pollock 2009:227) in the translation rather than “their buds, standing for pretty lower lips, quivered” and “flinging their branch-arms” (my translations). It is important to note here that \textsl{apasmāra},\index{apasmara@\textsl{apasmāra}} though described as resulting from possession by spirits, is similar in its symptomatology to seizures (compare with Māgha’s\index{Magha@Māgha} fanciful description of the ocean as suffering from \textsl{apasmāra} in \textsl{Śiśupālavadha},\index{Sisupalavadha@\textsl{Śiśupālavadha}} 3.72). The English translator must therefore imagine an episode of seizure in this context. Finally, Pollock\index{Pollock, Sheldon} reasons that the mountains were possessed “to behold the peak dropped by the monkey”. This is how he translates\index{errors in translation!mistranslations!getting narrative wrong} “\textsl{śailaṁ prekṣya kaper nipātitam}” -- thereby rendering “\textsl{prekṣya}” (“having beheld”) as if it were “\textsl{prekṣituṁ}” (“to behold”).

\vskip .1cm

4.~Pollock’s (2009) translation of verse no. 1.16 from the \textsl{Rasa-mañjarī}\index{Rasamanjari@\textsl{Rasa-mañjarī}} : 

In this verse, Pārvatī\index{Parvati@Pārvatī} is described as mistaking her reflection in the crescent moon on Śiva’s\index{Siva@Śiva} head for another woman and threatening him (presumably by rigorously shaking her hand in front of him) out of anger. Not only does Pollock translate “\textsl{tarjayāmāsa}” wrongly as “began to slap” (Pollock 2009:17)\endnote{The full verse reads:

\textsl{pratiphalam avalokya svīyam indoḥ kalāyāṁ}\\
\textsl{hara-śirasi parasyā vāsam āśaṅkamānā |}\\
\textsl{giriśam acalakanyā tarjayāmāsa kampa-}\\
\textsl{pracala-valaya-cañcat-kānti-bhājā kareṇa ||}} but in doing so, he also reveals his ignorance of behaviours (\textsl{anubhāva}-s)\index{anubhava@\textsl{anubhāva}} that Sanskrit poets regard as proper under such circumstances; and slapping (to my knowledge) is never reckoned\index{errors in translation!mistranslations!getting narrative wrong} among them.

\vskip .1cm

5.~Notes on translations rather than translations \textsl{per se} can sometimes reveal how versified narratives are misunderstood by Western Indologists. Readers are likely to be misled by such notes and get the narrative wrong. Two examples from Doniger\index{Doniger, Wendy} (2006) might suffice to clarify this point. 

Verse no. 4.3\endnote{The full verse reads:

\textsl{prāṇāḥ parityajata kāmam adakṣiṇaṁ māṁ!}\\
\textsl{re dakṣiṇā bhavata, mad-vacanaṁ kurudhvam!}\\
\textsl{śīghraṁ na yāta yadi, tan muṣitāḥ stha nūnaṁ}\\
\textsl{yātā sudūram adhunā gajagāminī sā ||}\\
The translation given is:\\
Breaths of my life, leave me; do what I ask. Oblige me more than I obliged her. If you don’t leave quickly, you’ll surely be plundered. For the woman who walks with the grace of an elephant has already gone far away.} from Harṣa’s\index{Sriharsa@Śrīharṣa} \textsl{Ratnāvalī}\index{Ratnavali@\textsl{Ratnāvalī}} is one in which the hero, king Udayana,\index{Udayana} believing that his sweetheart, Sāgarikā, is dead, orders his life-breaths to leave him, and join her before it is too late, i.e., before she has gone too far. If they delay, they would be robbed forever of the good fortune of being with her. The king wants his \textsl{prāṇa-s}, life-breaths, to enjoy what he himself is deprived of. 

Noting that this is “A difficult verse” (Doniger\index{Doniger, Wendy} 2006:493), Doniger sets about conjecturing as to what the verse means. Her reading of the word “\textsl{muṣita}” is flawed, and she attempts to fit her misreading to the context in a contorted way. “\textsl{Muṣita}” literally means “robbed of”. But like its English translation itself, it can be used in the sense of “deprived of”. The Sanskrit commentary, \textsl{Prabhā}, by Nārāyaṇaśarma explains this word as “\textsl{vañcita}”, i.e., “cheated” (“\textsl{bhavatāṁ śīghra-gamanābhāve tasyā alābhāt madvad vañcitā bhaviṣyatheti tātparyam}”--- “If you won’t go soon, you won’t be able to meet her, and will find yourself cheated like I myself am”) (Kale 1928:162). 

Doniger\index{Doniger, Wendy} translates “\textsl{muṣita}” as “plundered” (Doniger 2006:235). Plundered by whom? -- The king himself. That this is how Doniger understands the verse is clear from her notes: “The second part seems to mean that if the breaths do not leave of their own accord he will kill himself and thus steal them, to catch up with Sāgarikā —---” (Doniger 2006:490). 

The king, according to Doniger, wants to catch up with Sāgarikā and considers his life-breaths as an impediment that has to be overcome in the process. But, according to the poet, the king wants his life-breaths to attain what he himself cannot as long as he is embodied. Doniger’s king is selfish whereas the poet’s is altruistic.\index{errors in translation!mistranslations!getting narrative wrong}

\vskip .1cm

6.~The second example from Doniger\index{Doniger, Wendy} is her translation of verse no. 1.1, the \textsl{nāndī-padya}, of Harṣa’s\index{Sriharsa@Śrīharṣa} \textsl{Priyadarśikā}.\index{Priyadarsika@\textsl{Priyadarśikā}} 

This verse\endnote{The full verse reads thus:

\textsl{dhūma-vyākula-dṛṣṭir indu-kiraṇair āhlāditākṣī punaḥ}\\
\textsl{paśyantī varam utsukānata-mukhī bhūyo hriyā brahmaṇaḥ |}\\
\textsl{serṣyā pāda-nakhendu-darpaṇagate gaṅgāṁ dadhāne hare,}\\
\textsl{sparśād utpulakā kara-graha-vidhau gaurī śivāyāstu vaḥ ||}} depicts the marriage between Śiva\index{Siva@Śiva} and Pārvatī,\index{Parvati@Pārvatī} describing a series of emotional states that the latter is going through in that situation. Pārvatī, the bride, longs to have a look at the face of Śiva, the groom. But her eyes are agitated by the smoke from the sacrificial fire. The cool rays of the moon on Śiva’s head come to her rescue and comfort her reddened eyes. Just as she is about to catch a glimpse of Śiva’s face, she beholds Brahmā,\index{Brahma@Brahmā}
 the officiating priest, in their vicinity, and out of modesty bends her face down (how could she, in spite of her eagerness, directly look at the groom when another male is standing close by?). She can now see Śiva\index{Siva@Śiva} reflected in her bright toe-nails. But instead of being happy that she could manage to look at least at the reflected image of her husband, Pārvatī is filled with jealousy -- for, along with Śiva is also reflected Gaṅgā,\index{Ganga@Gaṅgā} her co-wife, whom he holds in his matted locks. Going through these emotional states, Pārvatī\index{Parvati@Pārvatī} suddenly feels the touch of Śiva’s hand on hers during the ritual of \textsl{pāṇi-grahaṇa} and is covered by goosebumps. The poet ends the verse with a prayer that Pārvatī, thus described, bring about auspiciousness.

Doniger’s\index{Doniger, Wendy} notes about why Pārvatī\index{Parvati@Pārvatī} should bend her face down when she looks at Brahmā\index{Brahma@Brahmā} are as follows (Doniger 2006:493): “She is shy of showing her face in front of Brahmā, perhaps because of the tradition, preserved in many myths, that Brahmā desired her himself at the wedding, and was punished by Śiva”.\index{Siva@Śiva}

I am not aware of such a myth. If it were popular, other Sanskrit poets would have alluded to it in their works. However, that is not the case.\index{errors in translation!mistranslations!getting narrative wrong} Kālidāsa’s\index{Kalidasa@Kālidāsa} \textsl{Kumāra-sambhava} does not mention it. Moreover, even a suggestion of such perverse love in Brahmā\index{Brahma@Brahmā} would disturb the overall beauty of the verse in which the poet has carefully brought together the descriptions of several bodily and behavioral responses (agitated eyes, bending the face down, goosebumps) and mental states (eagerness, bashfulness, envy/jealousy) to strengthen his depiction of Pārvatī’s\index{Parvati@Pārvatī} love for Śiva\index{Siva@Śiva} (In \textsl{Nāṭya-śāstric} terms, the \textsl{sthāyi-bhāva}\index{sthayi-bhava@\textsl{sthāyibhāva}} in this verse is \textsl{rati}\index{rati@\textsl{rati}} (love), which being augmented by \textsl{vyabhicāri-bhāva}-s\index{vyabhicaribhava@\textsl{vyabhicāribhāva}} such as \textsl{autsukya}\index{autsukya@\textsl{autsukya}} (eagerness), \textsl{vrīḍā}\index{vrida@\textsl{vrīḍā}} (shame/bashfulness) and \textsl{asūyā}\index{asuya@\textsl{asūyā}} (envy), and \textsl{anubhāva}-s\index{anubhava@\textsl{anubhāva}} such as agitated eyes, bending the face down, goosebumps, etc., is elevated to the state of the \textsl{rasa} viz.\@ \textsl{śṛṅgāra}\index{srngara@\textsl{śṛṅgāra}}\index{rasa@\textsl{rasa}!srngara@\textsl{śṛṅgāra}} in the reader. 

Bringing Brahmā’s\index{Brahma@Brahmā} love in the picture will be an impropriety, \textsl{anaucitya},\index{anaucitya@\textsl{anaucitya}} of the highest order. On this verse, Kale\index{Kale, M. R.} (1928:2 Notes) makes observes -- “She felt shame for fear of being observed by Brahmā who was there serving as the uniting priest” -- and further adds -- “At first she had not seen her (i.e., Gaṅgā)\index{Ganga@Gaṅgā} as she dared not look long at Śiva\index{Siva@Śiva}
 in the presence of Brahmā”.\index{Brahma@Brahmā} This, I feel, is the correct way of looking at the narrative and filling its gaps imaginatively.

\subsection*{(b) Being Unfamiliar with Indian Cultural Practices}

Two examples, one from Kapstein’s\index{Kapstein, Matthew} (2009) translation of Kṛṣṇamiśra’s\index{Krsnamisra@Kṛṣṇamiśra} allegorical play \textsl{Prabodha-candrodaya}\index{Prabodha-candrodaya@\textsl{Prabodha-candrodaya}} and another from Mallinson’s\index{Mallinson, Sir James} (2006) translation of Dhoyī’s\index{Dhoyi@Dhoyī} \textsl{Pavana-dūta},\index{Pavanaduta@\textsl{Pavana-dūta}} are analyzed below to show how something that is a part of every Hindu’s cultural knowledge can go unnoticed by Western Indologists resulting thereby in pathetic mistranslations of textual portions from Sanskrit literature. 

7.~For a Hindu, the \textsl{sindūra}\index{sindura@\textsl{sindūra}} is not merely “vermilion”. When applied in the parting of the hair by Hindu women, it also signifies that they are happily married. When someone is described in Sanskrit \textsl{kāvya-s}\index{kavya@\textsl{kāvya}} as having wiped off the \textsl{sindūra} mark of his enemies’ wives, it is a roundabout way of saying that he has annihilated his foes and rendered their wives widows. Viṣṇu\index{Visnu@Viṣṇu} is described by Kṛṣṇamiśra\index{Krsnamisra@Kṛṣnamiśra} as: 
\begin{quote}
“\textsl{vibudha-ripu-vadhū-varga-sīmanta-sindūra-sandhyā-mayūkha-cchaṭonmārjannoddāma-dhāmādhipa}” 

\hfill(\textsl{Prabodha-candrodaya},\index{Prabodha-candrodaya@\textsl{Prabodha-candrodaya}} 4.32, Kapstein\index{Kapstein, Matthew} 2009:180)
\end{quote}

Here, one must understand that Viṣṇu\index{Visnu@Viṣṇu} possesses (\textsl{adhipa}) an unhindered (\textsl{uddāma}) prowess (\textsl{dhāma}) that can rub off (\textsl{unmārjana}) the \textsl{sindūra}\index{sindura@\textsl{sindūra}} mark (\textsl{sindūra}) akin to a streak (\textsl{mayūkha}) of twilight (\textsl{sandhyā}) from the parting of the hair (\textsl{sīmanta}) of the wives (\textsl{vadhū-varga}) of \textsl{asura-s} (\textsl{vibudha-ripu}). In sum, this compound lauds Viṣṇu\index{Visnu@Viṣṇu} as the vanquisher of asura-s. 

Kapstein translates the compound as follows: “You are the sovereign whose majestic luster eclipses twilight’s rays, vermilion like the parted hair of the wives of god’s rivals” (Kapstein 2009:181). It seems the translator has divided the compound into two halves \textsl{sandhyā-mayūkha-cchaṭonmārjannoddāma-dhāmādhipa} (You are the sovereign\break whose majestic luster eclipses twilight’s rays) and \textsl{vibudha-ripu-vadhū-varga-sīmanta-sindūra} (vermilion like the parted hair of the wives of god’s rivals). 

I make the following comments on this translation: (1) it is incorrect to describe parted hair as “vermilion” because \textsl{sindūra}\index{sindura@\textsl{sindūra}} is a pigment, not a color term (unlike in English where vermilion also signifies a dark red color); (2) even if Kapstein’s\index{Kapstein, Matthew} translation is allowed for the sake of argument, it would still be a breach of Sanskrit literary practice to compare the luster of a brave warrior (which Viṣṇu\index{Visnu@Viṣṇu} is in this context) with twilight rather than with bright daylight. 

As I understand, the basic flaw in Kapstein’s translation stems from mistaking the noun \textsl{sindūra}\index{sindura@\textsl{sindūra}} for an adjective. Having wrongly decided that \textsl{sindūra} stands for dark red (vermilion), the translator uses this as a color adjective to connect the two halves of the compound. In doing so, he fails to realize that it is only the English word “vermilion” that can be used in these two senses, not the Sanskrit “\textsl{sindūra}”.

Finally, the note given for this Sanskrit compound only serves to further compound the already flawed English rendering. The note reads “Wives of the god’s rivals: the vermilion in the hair of the \textsl{asura’s }wives is visible because they are bending in submission” (Kapstein\index{Kapstein, Matthew} 2009:308). 

Firstly, the Sanskrit original makes no mention of \textsl{asura’s} wives bending in submission. Such an assumption is therefore out of context. Secondly, the translation uses “vermilion” in the sense of a color adjective whereas the note above uses it in the sense of a pigment. Thirdly, the translator gives no reason as to why the \textsl{asura’s} wives bend in submission before Viṣṇu\index{Visnu@Viṣṇu} (which in itself is a figment of the translator’s imagination). Fourthly, it is not clear what is so special about the \textsl{sindūra}\index{sindura@\textsl{sindūra}} in the hair of the \textsl{asura’s} wives as compared to that of \textsl{sura-s}, \textsl{gandharva-s}, \textsl{vidyādhara-s}, etc. In other words, there seems to be no logic, if one goes by Kapstein’s\index{Kapstein, Matthew} translation and note, for singling out the \textsl{asura’s} wives. And finally, one is clueless about who this \textsl{asura} that the note above mentions in the singular is. In summary, Kapstein’s erroneous attempt at translating the Sanskrit compound could have been avoided if he had prior cultural knowledge about the \textsl{sindūra’s}\index{sindura@\textsl{sindūra}} relationship with a Hindu woman’s marital status.

8.~Mallinson’s\index{Mallinson, Sir James} (2006:120) translation of verse no. 28\endnote{The full verse reads:

\textsl{tasmin senānvaya-nṛpatinā devarājyābhiṣikto}\\
\textsl{devaḥ suhme vasati kamalā kelikāro murāriḥ |}\\
\textsl{pānau līlā-kamalam asakṛd yat-samīpe vahantyo}\\
\textsl{lakṣmī-śaṅkāṁ prakṛti-subhagāḥ kurvate vāra-rāmāḥ ||}}: 

The verse alludes to the temple of Murāri\index{Murari@Murāri} in the Suhma\index{Udayana} province. The courtesans employed in the service of the Lord are so charming that, seeing them carry play lotuses in their hands, one would surely mistake them for Goddess Lakṣmī\index{Laksmi@Lakṣmī} herself. 

To understand this verse, the translator must be aware of the cultural detail that Lakṣmī is known by the lotus she carries in her hand. 

Mallinson\index{Mallinson, Sir James} translates \textsl{pāṇau līlā-kamalam asakṛd yat-samīpe vahantyo lakṣmī-śaṅkāṁ prakṛti-subhagāḥ kurvate vāra-nāryaḥ} as “The courtesans around the temple, with their natural beauty and the play lotuses they constantly carry in their hands, make Lakṣmī anxious”. 

The translator falters on three counts in this verse: firstly, he reveals his ignorance about the culturally significant portrayal of Lakṣmī\index{Laksmi@Lakṣmī} as holding a lotus in her hand; secondly, he wrongly understands \textsl{śaṅkāṁ}  as “anxiety” rather than as “mistaken belief”; thirdly, the note that the author provides --- “A pun is made on Lakṣmī’s name Kamalā, which means lotus” (Mallinson 2006:277) --- is both factually incorrect (lotus is \textsl{kamalaṁ}, not \textsl{kamalā)}  and irrelevant to the verse (as is evident from the translator’s failure to explain how the pun noticed by him operates in the verse). All that the note does is to make us suspect that the translator is aware of some relationship between Lakṣmī\index{Laksmi@Lakṣmī} and lotus, but not of the sort that might help us understand the verse correctly.

\subsection*{(c) Being Unfamiliar with Complementary Bodies of Knowledge that Sanskrit \textsl{Kāvya}-s draw upon}
\index{kavya@\textsl{kāvya}}

9.~Pollock’s\index{Pollock, Sheldon} (2009) note on verse no.~3.30 from Bhānudatta’s\index{Bhanudatta@Bhānudatta} \textsl{Rasataraṅgiṇī}:\index{Rasatarangini@\textsl{Rasataraṅgiṇī}} 

Anybody who has read Kālidāsa’s\index{Kalidasa@Kālidāsa} \textsl{Raghuvaṁśa}\index{Raghuvamsa@\textsl{Raghuvaṁśa}} must know the episode in which Raghu\index{Raghu} plans to go on a military expedition against Kubera\index{Kubera} so he can help Kautsa,\index{Kautsa} disciple of Varatantu, pay his \textsl{guru-dakṣiṇā}. In the said verse, Raghu speaks to Kautsa as follows: “All that I would ask of you, Kautsa, is to pause a moment”\endnote{The full verse reads thus:

\textsl{audāsyaṁ na vidhehi, gaccha na gṛhāt saṁvīkṣya mṛd-bhājanaṁ}\\
\textsl{yāce kin tu bhavantam etad akhilaṁ, kautsa, kṣaṇaṁ kṣamyatām |}\\
\textsl{dāsaś ced aham asmi ced, vasumatī sarvaiva saṁgṛhyatāṁ}\\
\textsl{svarṇaṁ ced gurudakṣiṇā, dhanapater ānīya sampādyate ||}\\
Translation by Pollock:\\
Please do not, seeing this earthen bowl of mine, leave my house in despair. All that I would ask of you, Kautsa,\index{Kautsa} is to pause a moment. If I am your slave, and still alive, take the whole world for your own. If gold is your teacher’s gift, I’ll get it from the Lord of Wealth himself.} (Pollock 2009:183). This is a translation of the Sanskrit original “\textsl{yāce kintu bhavantam etad akhilaṁ kautsa kṣaṇaṁ kṣamyatāṁ}” (Pollock 2009:182). 

Pollock’s footnote for this verse --- “King Raghu speaks to his priest at the conclusion of a sacrifice where he gave away all his wealth” (Pollock 2009:182) --- is misleading because we know from the \textsl{Raghuvaṁśa}\index{Raghuvamsa@\textsl{Raghuvaṁśa}} that Kautsa\index{Kautsa} is a student who has just completed his studies, not the priest of Raghu.

10.~Doniger’s\index{Doniger, Wendy} (2006) notes on the ancient Indian ritual of \textsl{dohada} are erroneous and mixed up. These notes have been provided for verses numbered 1.14 and 1.18 from Harṣa’s\index{Sriharsa@Śrīharṣa} \textsl{Ratnāvalī}.\index{Ratnavali@\textsl{Ratnāvalī}} The \textsl{dohada}\index{dohada@\textsl{dohada}} ritual was performed in order to fulfil the fancied wishes of specific trees so they could put forth flowers. The following verse (quoted by Apte\index{Apte, V. S.} 2005:379) summarizes the longings of several trees ---
\begin{quote}
\textsl{pādāghātād aśokas, tilaka-kurabakāv īkṣaṇāliṅganābhyāṁ,}\\
\textsl{strīṇāṁ sparśāt priyaṅgur, vikasati bakulaḥ}\\
\textsl{sīdhu-gaṇḍūṣa-sekāt,} |\\
\textsl{mandāro narma-vākyāt, paṭu-mṛdu-hasanāc campako,}\\
\textsl{vaktra-vātāc cūto, gītān-namerur vikasati ca puro}\\
\textsl{nartanāt karṇikāraḥ} ||
\end{quote}

According to this verse, the \textsl{bakula} blooms when women spit mouthfuls of wine on it and the \textsl{campaka} when women smile. 

According to Doniger’s\index{Doniger, Wendy} notes, however, “the \textsl{bakula}\index{bakula@\textsl{bakula}} ---, said to blossom when a beautiful woman sprays it with water from her mouth” (Doniger 2006:482) and “the \textsl{campaka-}s\index{campaka@\textsl{campaka}} --- enjoy the mouthfuls of wine the women have sprayed on them and they blossom when the women smile on them” (Doniger 2006:483). 

\subsection*{(d) Getting the Semantics Wrong}

(i) At the level of individual words --- 

11.~In his translation of Govardhanācārya’s\index{Govardhanacarya@Govardhanācārya} \textsl{Āryāsaptaśatī},\index{Aryasaptasati@\textsl{Āryāsaptaśatī}} Hardy\index{Hardy, Friedhelm} (2009:16) renders \textsl{pradoṣa} (verse no. 39 of the prelude) as “early morning” (Hardy 2009:17) instead of as “evening”\endnote{The full verse reads:

\textsl{sakala-kalāḥ kalpayituṁ}\\ 
\textsl{prabhuḥ prabandhasya kumuda-bandhoś ca |}\\
\textsl{sena-kula-tilaka-bhūpatir}\\
\textsl{eko rākā-pradoṣaś ca ||}\\
It has been translated as\\
They alone are capable of accomplishing the encyclopedia of arts and skills and of displaying all segments of the moon, the lotuses’ friend: the king who is the crest-jewel of the Sena lineage and the early morning of a full-moon day.}

12.~In the same aforementioned text, Hardy’s rendering of \textsl{tūla} (cotton) in verse no. 172 (Hardy 2009:82) as “\textsl{tula}” (\textsl{sic}) makes his translation “Which \textsl{tula} will have to be brought back to life that is burning in the fire of recent separation?” (Hardy 2009:83) of \textsl{nava-viraha-dahana-tūlo jīvayitavyas tvayā katamaḥ} incomprehensible. 

13.~Mallinson\index{Mallinson, Sir James} (2006:118) incorrectly translates \textsl{“bhūmidevāṅganānām”} (verse no. 27)\endnote{The full verse reads:

\textsl{gaṅgā-vīci-pluta-parisaraḥ saudha-mālāvataṁso}\\
\textsl{yāsyaty uccais tvayi rasa-mayo vismayaṁ suhma-deśaḥ |}\\
\textsl{śrotra-krīḍābharaṇa-padavīṁ bhūmi-devāṅganānāṁ}\\
\textsl{tālī-patraṁ nava-śaśi-kalā komalaṁ yatra yāti ||}\\
The translation reads:\\
The lush land of Suhma,\index{Suhma} bathed on its borders by Ganga’s\index{Ganga@Gaṅgā} waves and festooned with garlands of mansions, will be astonished at your arrival. Palm fronds as slender as the sliver of the new moon serve as ear ornaments for the king’s harem there.} in Dhoyī’s\index{Dhoyi@Dhoyī} \textsl{Pavanadūta}\index{Pavanaduta@\textsl{Pavanadūta}} as “the king’s harem”. \textsl{Bhūmideva}-s are brahmins. A king would be \textsl{nara-deva}, not \textsl{bhūmi-deva.}

14.~Torzsok’s\index{Torzsok, J} (2006:146) translation of verse no. 2.165 in Murāri’s\index{Murari (poet)@Murāri (poet)} \textsl{Anargha-rāghava}: The verse is a description of the twilight hour by Lakṣmaṇa\endnote{The verse reads:

\textsl{cūḍā-ratnaiḥ sphuradbhir viṣadhara-vivarāṇy ujjvalāny ujjvalāni}\\
\textsl{prekṣyante cakravākī-manasi niviśate sūryakāntāt kṛśānuḥ |}\\
\textsl{kiṁ cāmī śalyayantas timiram ubhayato nirbharāhas-tamisrā-}\\
\textsl{saṁghaṭṭodbhūta-sandhyānala-kiraṇa-kaṇa-spardhino bhānti dīpāḥ ||}\\
The translation has been rendered as:\\
The holes of poisonous snakes are blazing with their bright head-jewels here and there; from the sun-stones, the fire enters the hearts of the shel-duck; and the stars that pierce the darkness look like tiny sparkles of the radiant sunset, whose fire was produced by the violent friction of the day and the night on both sides.}. The crimson of the evening is fancied here as the fire that emanates when day and night rub against each other. The lamps that are lighted at this time look like sparks of this fire. The translator renders “\textsl{dīpāḥ}” as “stars” (Torzsok 2006:147) instead of as “lamps”/“diyas”. While the lamps share the golden-red hue of the evening sky and can therefore be imagined as related to the latter in some way, the stars cannot. 

(ii) At the level of compounds -

15.~In his translation of Dhoyī’s\index{Dhoyi@Dhoyī} \textsl{Pavana-dūta},\index{Pavanaduta@\textsl{Pavana-dūta}} Mallinson\index{Mallinson, Sir James} (2006:137) renders the \textsl{karmadhāraya} compound “\textsl{mukha-vidhu}” (moon-face) as “face of the moon”, thus jeopardizing the meaning of the verse as a whole. The verse describes how a \textsl{ketakī} petal (fashioned into an ear-ornament) falling from the ear of ladies during their love-sports is mistaken to be a fragment of their moon-face. The petal of \textsl{ketakī} (pandanus) is often compared to the crescent moon (and vice-versa), as in one of the benedictory verses of Bhavabhūti’s\index{Bhavabhuti@Bhavabhūti} \textsl{Mālatī-mādhava}\index{Malatimadhava@\textsl{Mālatī-mādhava}} (\textsl{ketaka-śikhā-sandigdha-mugdhendavaḥ}). I quote a part of Mallinson's translation here (2006:137) -- "connoisseurs inspect it as if a single fragment of the face of the moon were before their eyes" -- of the original “\textsl{utpaśyanti --- bhinnaṁ sākṣād iva mukha-vidhoḥ khaṇḍam ekaṁ vidagdhāḥ}” (verse no. 51, Mallinson\index{Mallinson, Sir James} 2006:136). Given that the Sanskrit compound is “\textsl{mukha-vidhu}” and not “\textsl{vidhu-mukha}”, one cannot explain away the translator’s error by assuming that he has mistaken a \textsl{karmadhāraya} compound for a \textsl{ṣaṣṭḥī-tatpuruṣa} compound. 

(iii) At the level of sentence/phrase -- 

16.~Bronner\index{Bronner, Yigal} and Shulman's\index{Bronner, Yigal} (2009:88) translation of verse no. 13\endnote{The full verse reads:

\textsl{bhagavati daye, bhavatyā vṛṣagirināthe samāplute tuṅge} |\\
\textsl{apratigha-majjanānāṁ hastālambo madāgasāṁ mṛgyaḥ} ||\\
The translation reads:\\
When you flood even the god on the peak of Bull Hill, surely my burden of evil will drown, too. Compassion, great goddess: would it be too much to ask you to give it a hand?} from Vedānta-deśika’s\index{Vedantadesika@Vedānta-deśika} \textsl{Dayā-śataka}:\index{Dayasataka@\textsl{Dayā-śataka}} The verse describes how the devotees’ sins drown in the flood of Lord Srīnivāsa’s compassion so deeply there is none who can offer them a helping hand. The purport of the verse is that the sins do not resurface. The Sanskrit original is “\textsl{hastālambo mad-āgasāṁ mṛgyaḥ}”: “my sins have to search for a helping hand that can pull them out” (my translation). That the sins cannot find such a helping hand is understood. If they had, the Lord’s compassion would have to be branded as incapable of delivering the devotee from his/her sins.

\index{Bronner, Yigal}Bronner and Shulman\index{Bronner, Yigal} (2009:89) translate the Sanskrit phrase quoted above as “would it be too much to ask you to give it (my burden of evil that will drown) a hand”. This goes completely contrary to what the poet wants to convey, implying as it does that the Lord’s Compassion (=ompassion incarnate) must give a helping hand to the devotee’s drowning sins, causing them, in effect, to surface up. 

17.~Hardy\index{Hardy, Friedhelm} (2009:184) wrongly translates the phrase \textsl{varjitā bhujaṅgena} (“freed of a serpent”) that occurs in verse no. 463 of Govardhana’s\index{Govardhanacarya@Govardhanācārya} \textsl{Āryā-saptaśatī} as “is not free from snakes”. The verse in question compares, through punning, a girl who is faithful to her husband and therefore \textsl{varjitā bhujaṅgena} --- without a paramour --- to the river Yamunā , which is also \textsl{varjitā bhujaṅgena } --  freed (by Kṛṣṇa)\index{Krsna@Kṛṣṇa} of the serpent Kāliya.\index{Kaliya@Kāliya} By translating \textsl{varjitā} as “not free” and the singular \textsl{bhujaṅgena} as “snakes”, the translator simultaneously reveals his lack of proficiency in Sanskrit grammar and Hindu mythology, apart from spoiling a happy Sanskrit pun through his translation. 

\subsection*{(e) Failing to Spot Puns}

Though I could have discussed this point in the previous section itself, I feel Sanskrit verses are so replete with puns that it is important to discuss separately the consequences of not spotting them during translation. Firstly, the very beauty of a Sanskrit verse may rely heavily on punning. Secondly, poets such as Govardhana\index{Govardhanacarya@Govardhanācārya} pun frequently. While translating the works of such poets, a translator must be prepared to encounter puns at every nook and corner. Thirdly, important as it is to spot a genuine pun, it is equally important not to overdo things and imagine a pun where none in intended. Example no. 8 discussed above is an instance of imagining a nonexistent pun. 

18.~Pollock’s\index{Pollock, Sheldon} (2009:89) half-hearted translation of verse no. 99 from Bhānudatta’s\index{Bhanudatta@Bhānudatta} \textsl{Rasa-mañjarī}:\index{Rasamanjari@\textsl{Rasa-mañjarī}} The verse is based on a series of puns that all refer simultaneously to the heroine and a lamp. Pollock has missed the pun in the line \textsl{“tasyā daiva-vaśād daśāpi caramā prāyaḥ samunmīlati”}\endnote{The full verse reads:

\textsl{cakre candramukhī pradīpa-kalikā dhātrā dharā-maṇḍale}\\
\textsl{tasyā daiva-vaśād daśāpi caramā prāyaḥ samunmīlati |}\\
\textsl{tad brūmaḥ śirasā natena, sahasā śrīkṛṣṇa nikṣipyatāṁ}\\
\textsl{snehas tatra tathā, yathā na bhavati trailokyam andhaṁ tamaḥ ||}\\
The translation has been rendered as:\\
God made the moon-faced girl the single lamp of beauty on earth, and Fate would have it that her final hour is nearly upon her. I bow my head and beg you, dear Krishna, hurry and pour a drop of love: oil in her, to keep deep darkness from engulfing the entire universe.}. The phrase “\textsl{caramā daśā}” also means “the last wick” (as in Kālidāsa’s\index{Kalidasa@Kālidāsa} “\textsl{nirviṣṭa-viṣaya-snehaḥ sa daśāntam upeyivān}”, \textsl{Raghuvaṁśa},\index{Raghuvamsa@\textsl{Raghuvaṁśa}} 12.1). His translation of this phrase as “final hour” in “Fate would have it that her final hour is nearly upon her” (Pollock 2009:89) is limited to the heroine, and bears no relation with the lamp. Either Pollock should translate all the puns occurring in a verse or give a second meaning in the notes, not translate some puns and leave out others for readers to figure out themselves.

19.~Verse no. 28 from Govardhana’s\index{Govardhanacarya@Govardhanācārya} \textsl{Āryāsapataśatī}\index{Aryasaptasati@\textsl{Āryāsaptaśatī}} seems to have eluded the imagination of the translator Hardy\index{Hardy, Friedhelm} (2009:12). The Sanskrit verse along with its English translation and notes by Hardy are given below --- 
\begin{quote}
\textsl{maṅgala-kalaśa-dvayamaya-kumbham adambhena bhajata gajavadanam} |\\
\textsl{yad-dāna-toya-taralais tila-tulanālambi rolambaiḥ} ||
\end{quote}

\begin{myquote}
Translation: “Be devoted without arrogance to Him with the Elephant’s Face! He has two frontal lobes that resemble auspicious vessels, and the bees, agitating for his ichor, become like sesame seeds” 

\hfill(Hardy 2009:13).
\end{myquote}

\newpage

Notes: “The meaning or significance of \textsl{tila} (also “mole”) is not quite clear. The ichor is so fragrant and so abundant that masses of black bees gather around his temples, so maybe the comparison is with a vessel filled with black sesame seeds” (Hardy\index{Hardy, Friedhelm} 2009:272).

The translator appears to have missed out on a second meaning of \textsl{dāna-toya}. This compound word not only means “ichor”, the fluid that flows out of the temples of an elephant in rut, but also “the ritual water that is poured into the hands of one who receives \textsl{dāna}, gifts”. The punning use of \textsl{dāna} in these two senses is very common in Sanskrit literature (e.g., “\textsl{dānaṁ dadaty api jalaiḥ sahasādhirūḍhe}”, \textsl{Śiśupāla-vadha}\index{Sisupala-vadha@\textsl{Śiśupālavadha}} 5.37). What the poet wants to convey is that the frontal lobes of Lord Gaṇeśa\index{Ganesa@Gaṇeśa} are like two vessels filled with the ritual water for \textsl{dāna} (that happens to be the ichor) and the bees clinging to it are like sesame seeds mixed with this water. Hardy’s\index{Hardy, Friedhelm} ignorance about the second meaning of \textsl{dāna}, the culturally prescribed use of water during the ritual of \textsl{dāna}, and the ingredients that are mixed with this water -- have all contributed towards making his translation ineffective. It is important to note that a common Sanskrit word for ichor is \textsl{mada} or \textsl{mada-jala}. When a poet, such as Govardhana,\index{Govardhanacarya@Govardhanācārya} with a penchant for puns, employs \textsl{dāna} (that is more commonly understood as gift and less commonly as ichor) in place of \textsl{mada}, the translator should immediately suspect that there is a pun lurking underneath. 

20.~Hardy (2009:124) glosses over a pun in verse no.\@ 293 of Govardhana’s\index{Govardhanacarya@Govardhanācārya} \textsl{Āryā-saptaśatī}\index{Aryasaptasati@\textsl{Āryā-saptaśatī}} with the result that his translation makes no sense. The verse and its translation are given below:
\begin{quote}
\textsl{duṣṭa-graheṇa gehini tena ku-putreṇa kiṁ prajātena} |\\
\textsl{bhaumeneva nijaṁ kulam aṅgārakavat kṛtaṁ yena} ||
\end{quote}

\begin{myquote}
Translation: “O house-wife, what good is that bad son, born to you under an unfavorable asterism, who, like Mars, has reduced his own family to coals” 
\hfill(Hardy\index{Hardy, Friedhelm} 2009:125).
\end{myquote}

“\textsl{Ku-putra}” is not just a “bad son” but also the planet Mars, who is referred to as “the son of \textsl{Ku}, Earth”. Similarly “\textsl{aṅgārakavat kṛtaṁ}” is not just “reduced --- to coals” but also “made into one that has \textsl{aṅgāraka}” (Aṅgāraka is yet another name for Mars). 

Since the translator doesn’t provide any notes that enlighten the reader about these other meanings of \textsl{kuputra} and \textsl{aṅgāraka}, readers are left in the dark as to what makes the poet compare the bad son to Mars. The comparison is based not on any concrete attributes common to both but on mere wordplay. 

The translation also seems to imply that what is common to the bad son and Mars is that both reduced their family to coals. However, there is nothing in Hindu mythology to suggest that Mars brought about a destruction of his own family. Even if such a story existed, comparing two people on the grounds that both reduced their families to coals is hardly poetic. 

\section*{Examining Editing Errors}

In this section, I shall examine two examples of editing errors. Both examples have been selected from the CSL\index{Clay Sanskrit Library} series. The errors are a result of the editors’ inattention to grammar and metrical details
\begin{enumerate}
\item Verse no. 1.4 in Harṣa’s\index{Sriharsa@Śrīharṣa} \textsl{Ratnāvalī},\index{Ratnavali@\textsl{Ratnāvalī}}
 edited by Doniger\index{Doniger, Wendy} (2006:70): Instead of “{\sl\bfseries bhavatu} \textsl{ca pṛthivī samṛddha-sasyā}” we have “\textsl{bhavantu} \textsl{ca pṛthivī samrūddhasasyā}” as the first line of the verse, a mistake on both accounts --- grammar and metrics. 

According to the principles of Sanskrit grammar, the subject of a sentence must agree with its verb both in person and number. However, in the edited line given above, the subject \textsl{pṛthivī} is in the singular form and the verb \textsl{bhavantu} that goes along with it is in the plural form. 

From a metrical point of view, it suffices to say that the second syllable of the edited line is long though it should have been short in keeping with the rules of the meter Puṣpitāgrā in which the verse is composed.  

\item Verse no.\@ 4.32 in Kṛṣnamiśra’s\index{Krsnamisra@Kṛṣṇamiśra} allegorical play \textsl{Prabodha-candrodaya},\index{Prabodha-candrodaya@\textsl{Prabodha-candrodaya}} edited by Kapstein\index{Kapstein, Matthew} (2009:180):  In this verse, that is composed in the Daṇḍaka\index{Dandaka@Daṇḍaka} meter, every line starts with 6 short syllables (e.g., \textsl{tri-bhu-va-na-ri-pu}) followed by a specific number of triads each of which has the following pattern: long syllable --- short syllable --- long syllable (e.g., \textsl{kai-ṭa-bho}). 

Since a Daṇḍaka’s\index{Dandaka@Daṇḍaka} beauty is fully evident only when read out aloud, Kapstein should have been extra careful not to allow a metrical error to mar it. However, he allows the metrically flawed word “\textsl{vallarī}” to creep into the compound “---\textsl{śrībhujā-}\textsl{vallarī-}\textsl{saṁśleṣa-saṅkrānta}—“. With a basic sense of Sanskrit metrical aesthetics, one can easily conclude that the correct reading is “---\textsl{śrībhujā-}\textsl{valli-}\textsl{saṁśleṣa-saṅkrānta}—“. 

As to the meaning of the compound itself, it makes no difference if \textsl{vallarī} is replaces \textsl{valli}. However, this replacement makes a lot of difference to the rhythmic structure of the verse. Editors of Sanskrit poetic texts shouldn’t compromise for meaning at the expense of metrical perfection. They should respect structure and sense alike\endnote{The passage reads:

\textsl{tribhuvana-ripu-kaiṭabhoddaṇḍa-kaṇṭhāsthi-kūṭa-sphuṭo\-nmārjitodātta-cakra-sphuraj-jyotir-ulkā-śatoḍḍāmaroddaṇḍa-khaṇḍendu-cūḍa-priya! prauḍha-dordaṇḍa-vibhrānta-manthācala-kṣubdha-dugdhāmbudhi-protthita-śrībhujāvallarī-saṁśleṣa-saṁkrānta-pīna-stanābhoga-patrāvalī-lāñchitoraḥ-sthala! sthūla-muktā-phalottāra-hāra-prabhā-maṇḍala-prasphurat-kaṇṭha! vaikuṇṭha! bhaktasya lokasya saṁsāra-mohacchidaṁ dehi bodhodayaṁ deva! tubhyaṁ namaḥ!}}.
\end{enumerate}

\section*{Examining Misanalysis}

Here I shall examine Tubb’s\index{Tubb, Gary} analysis of Bāṇa’s\index{Bana@Bāṇa} verses in the Śaśivadanā\index{Sasivadana@Śaśivadanā} meter as part of the article “On the boldness of Bāṇa” (Tubb 2014:308-354). Tubb considers Bāṇa’s verses as “based on the (21-syllable) Śaśivadanā meter” (Tubb 2014:335) which they no doubt are. The pattern of long and short syllables in this verse is as follows: UUUU\_U\_UUU\_UU\_UU\_U\_U\_ (“U” stands for short syllable and “\_” for long syllable). This is the famous Campakamālā\index{Campakamala@Campakamālā} meter in which many Kannada and Telugu poets have composed their verses.

There are two points which Tubb\index{Tubb, Gary} raises in his article:

Point one: 

Verses that are ascribed to Bāṇa\index{Bana@Bāṇa} in anthologies (e.g., \textsl{Subhāṣita-ratna-koṣa}), for example, the following verse --- 
\begin{quote}
\textsl{rajani-purandhrir oḍhra-tilakas timira-dvipa-yūtha-kesarī}\\
\qquad \textsl{     rajatamayo’bhiṣeka-kalaśaḥ kusumāyudha-medinīpateḥ} |\\
\textsl{ayam udayācalaika-cūḍāmaṇir abhinava-darpaṇo diśām}\\
\qquad\textsl{udayati gagana-sarasī-haṁsasya hasann iva vibhramaṁ śaśī} ||

\hfill (verse no. 930 in \textsl{Subhāṣita-ratna-koṣa}) 
\end{quote}
go against the prosodic pattern of short and long syllables prescribed for this meter. In the example given above, the first two lines obey the rule but the last two lines do not. This, Tubb\index{Tubb, Gary} considers a “bold change” (Tubb 2014:338). 

Nowhere in the history of Sanskrit literature have poets tried to be original by breaking metrical rules in the manner described above. Breaking metrical rules is considered as a flaw rather than as something positive. Furthermore, Tubb mentions poetic reasons for substantiating the metrical flaw, commenting how “the combined effect of these three surprises (i.e., three instances when the metrical pattern has been broken) is to place the strongest possible emphasis on the beginning of the word \textsl{haṁsasya}, --- an emphasis that serves several poetic purposes simultaneously. It stresses the action of laughing expressed by the verbal root \textsl{has/haṁs}—---” (Tubb\index{Tubb, Gary} 2014:339). 

To my knowledge, traditional commentators never look out for a suggested meaning, a \textsl{dhvani}, in instances where a metrical flaw is evident. In that case, all verses quoted by Sanskrit aestheticians for illustrating various poetic flaws may have to be interpreted as examples of \textsl{dhvani}.

Point two:

Tubb makes a lot about the instances in which poets (Bāṇa,\index{Bana@Bāṇa} Māgha)\index{Magha@Māgha} have not cared to follow the rule of \textsl{yati}\index{yati@\textsl{yati}} (caesura) in verses that are composed in the aforementioned meter. He painstakingly notes the place of \textsl{yati} in each line of the exemplary verses that he has chosen, and finds out that they don’t match even within the same verse, leave alone when one verse is compared with another composed in the same meter. 

He must know that Caṁpakamālā/Śaśivadanā\index{Sasivadana@Śaśivadanā}\index{Campakamala@Campakamālā} is a meter with a weak \textsl{yati}. Breaking a \textsl{yati} where none practically exists is no bold step. When we compare two verses composed in meters such as Indravajrā,\index{Indravajra@Indravajrā} Upendravajrā,\index{Upendravajra@Upendravajrā} or Vasantatilakā,\index{Vasantatilaka@Vasantatilakā} we can observe that the \textsl{yati}-rule\index{yati@\textsl{yati}} is scarcely obeyed. So is the case with Caṁpakamālā/Śaśivadanā.\index{Sasivadana@Śaśivadanā}\index{Campakamala@Campakamālā}

\section*{Conclusion}

Translators desire to communicate through their translations a hitherto hidden cultural world to their audience. However, the extent to which they themselves are familiar with that world and can make sense of happenings in it, plays an important role in determining how effective their communication will turn out to be. Literary texts in a classical language such as Sanskrit describe a cultural world whose continuity with contemporary times is scarcely visible even to an Indian, let alone, a Western, Sanskritist. Western Sanskrit scholars are in a sense twice as disadvantaged as their Indian counterparts in understanding and appreciating this cultural world since they are removed from it both spatially and temporally. 

As my examination of mistranslations demonstrates, Western scholars often err in their translation of Sanskrit verses because they are not conversant with something that is part and parcel of every Hindu’s culturally acquired knowledge. It is therefore important that they approach Sanskrit \textsl{kāvya}\index{kavya@\textsl{kāvya}} literature with humility and a healthy sense of uncertainty rather than with surety, born out of arrogance, that some basic grounding in the Sanskrit language and a couple of dictionaries is all that is needed to make poems from a hoary past reveal their deepest secrets. 

\begin{thebibliography}{99}
\itemsep=2pt
\bibitem[]{chap6_item1}
Apte, V. S. (2005). \textsl{The Student’s Sanskrit-English Dictionary.} Delhi: Motilal Banarsidass.

\bibitem[]{chap6_item2}
Brockmeler, J. (2012). “Narrative Scenarios: Towards a Culturally Thick Notion of Narrative”. In J. Valsiner (Ed), \textsl{The Oxford Handbook of Culture and Psychology} (pp.~439--467). New York: Oxford University Press.

\bibitem[]{chap6_item3}
Bronner, Y., \& Shulman, D. (2009). “Self-Surrender” “Peace” “Compassion” \& “The Mission of the Goose”: \textsl{Poems and Prayers from South India by Appayya Dikshita, Nilakantha Dikshita \& Vedanta Deshika.} New York: New York University Press, JJC Foundation.

\bibitem[]{chap6_item4}
Doniger, W. (2006). \textsl{The Lady of the Jewel Necklace and The Lady who Shows her Love by Harsha.} New York: New York University Press, JJC Foundation.

\bibitem[]{chap6_item5}
Dvivedī, Dr. Pārasanātha (Ed.) (Trans.) (1996). \textsl{Nāṭyaśāstra of Śrī Bharata Muni} (Part 2 of five parts) with the Commentaries  \textsl{Abhinava-bhāratī} by Śrī Abhinavaguptācārya \& \textsl{Manoramā} (Hindi Commentary of the Editor). Varanasi: Sampurnanand Sanskrit University.

\bibitem[]{chap6_item6}
Goldie, P. (2004). \textsl{The Mess Inside: Narrative, Emotion, \& the Mind.} United Kingdom: Oxford University Press.

\bibitem[]{chap6_item7}
Hardy, F. (2009). \textsl{Seven Hundred Elegant Verses by Govardhana.} New York: New York University Press, JJC Foundation.

\bibitem[]{chap6_item8}
Kale, M. R. (1928). \textsl{The Priyadarsika of Sri Harsha-deva.} Bombay: Gopal Narayan \& Co.

\bibitem[]{chap6_item9}
Kapstein, M. T. (2009). \textsl{The Rise of Wisdom Moon by Krishnamishra.} New York: New York University Press, JJC Foundation.

\bibitem[]{chap6_item10}
Malhotra, R. (2016, May 5). Interview with Rajiv Malhotra: A point by point response to R. Ganesh [Web log post]. Retrieved from   \url{http://www.speakingtree.in/blog/interview-with-rajiv-malhotra-a-point-by-point-response-to-r-ganesh}

\bibitem[]{chap6_item11}
Mallinson, S. R. (2006). \textsl{Messenger Poems by Kalidasa, Dhoyi \& Rupa Gosvamin}. New York: New York University Press, JJC Foundation.

\bibitem[]{chap6_item12}
\textsl{\textbf{Nāṭyaśāstra} of Śrī Bharata Muni} See Dvivedī.

\bibitem[]{chap6_item13}
Pollock, S. I. (2009). \textsl{Bouquet of Rasa \& River of Rasa by Bhanu-datta.} New York: New York University Press, JJC Foundation.

\bibitem[]{chap6_item14}
\textsl{\textbf{Ratnāvalī} Nāṭikā} See ??Name of the Editor??

\bibitem[]{chap6_item15}
??Name of the Editor? (Ed.) (1925). \textsl{Mahākavi-śrīharṣadeva-viracitā Ratnāvalī Nāṭikā} (3rd ed.). Mumbai: Nirnaya Sagar Press.

\bibitem[]{chap6_item16}
Snaevarr, S. (2010). \textsl{Metaphors, Narratives, Emotions: Their Interplay and Impact}

\bibitem[]{chap6_item17}
Torzsok, J. (2006). \textsl{Rama Beyond Price by Murari.} New York: New York University Press, JJC Foundation.

\bibitem[]{chap6_item18}
Tubb, Gary. (2014). “On the boldness of Bāṇa”. In Y. Bronner, D. Shulman, and G. Tubb (Eds.), \textsl{Innovation and Turning Points: Towards a History of Kavya Literature} (pp.~308--354). New Delhi: Oxford University Press.
\end{thebibliography}

\theendnotes
