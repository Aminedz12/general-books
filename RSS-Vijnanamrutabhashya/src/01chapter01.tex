\chapter{}\label{chap01}

%\Authorline{B. Sankareswari}

\begin{center}
\dev{\large\bfseries श्रीसरस्वत्यै नमः}

\dev{\large\bfseries श्रीगणेशाय नमः}
\end{center}

\begin{center}
\begin{tabular}{l}
\dev{सर्वत्र यो यत्र सर्वं यश्च सर्वमतो भवेत् ।}\\
\dev{चिदचिच्छक्तये तस्मै नमश्चिन्मात्ररूपिणे ॥ १ ॥}\\[2pt]
\dev{अन्तर्यामिगुरूद्दिष्टज्ञानविज्ञानभिक्षुणा ।}\\
\dev{ब्रह्मसूत्रऋजुव्याख्या क्रियते गुरुदक्षिणा ॥ २ ॥}\\[2pt]
\dev{श्रुतिस्मृतिन्यायवच:क्षीराब्धिमथनोद्धृतम् ।}\\
\dev{ज्ञानामृतं गुरोः प्रीत्यै भूदेवेभ्यो नु दीयते ॥ ३ ॥}\\[2pt]
\dev{परिविषय्य सद्बुद्ध्या मोहिन्येवाथ दानवान्}\\
\dev{कुतर्कान् वञ्चयित्वेदं पीयताममृतेप्सुभिः ॥ ४ ॥}\\[2pt]
\dev{पीत्वैतद् बलवन्तस्ते पाखण्डासुरयूथपान् ।}\\
\dev{विजित्य ज्ञानकर्मभ्यां यान्तु श्रीमद्गुरोः पदम् ॥ ५ ॥}
\end{tabular}
\end{center}


Bhikṣu starts his work as is the custom with some benedictory verses.
\begin{enumerate}
\item He that is everywhere, where everything is, and He that becomes everything eventually, to That (Supreme) of the nature of pure conscienceness, having the powers of consciousness  (jīvas) and insentience (prakṛti)(cidacicchaktaye), I bow down.\endnote{In Bhikṣu's philosophypuruṣa and prakṛti are śaktis of Īśvara}

\item This Ṛjuvyākhyā interpretation of the Brahmasūtra is being offered as gurudakṣiṇā  by Bhikṣu who is a jñānavijñāna, keeping in mind his guru as
          the internal self.
          
\item This Jñānāmṛta which has been obtained after churning the ocean of śruti, smṛti and Nyāya reasoning is being offered to the devas on earth for pleasing (my) guru.

\item Examining well the demons through Mohinī in the form of truthful intelligence (and) tricking them through false reasoning those desiring nectar may drink this.

\item Having drunk this, those powerful ones after conquering the ignorant groups of demons and winning (liberation) through the means of both karma and knowledge may they reach the feet of the guru.
\end{enumerate}

\dev{ब्रह्मविदाप्नोति परम्, ब्रह्म वेद ब्रह्मैव भवति, तमेव विदित्वाति मृत्युमेति'' इत्यादिश्रुतिसिद्धपरमपुरुषार्थसाधनताके ब्रह्मज्ञाने विधिः श्रूयते --- ``आत्मेत्येवोपासीत, स म आत्मेति विद्यात् तमेव धीरो विज्ञाय प्रज्ञां कुर्वीत ब्राह्मणः'' इत्यादिरूपः । तत्र किं ब्रह्म, किं वा तस्य ब्रह्मतानिर्वाहकं गुणजातम्, कीदृशं वा तस्य ज्ञानम् , कीदृशं वा तस्य फलमित्यादिकं विशिष्य मुमुक्षूणां जिज्ञासितं भवति, श्रुतिष्वापाततोऽन्योन्यविरुद्धार्थतायाः शाखाभेदेन प्रतिभासनादिति। अतस्तन्निर्णयाय ब्रह्ममीमांसाशास्त्रमपेक्षितम्}

In order to attain the highest puruṣārtha which is the realization of Brahman established and stated in śruti as ``brahmviḍāpnoti param, brahma veda brahmaiva bhavati, etc'', one hears of imperative statements such as ``ātmetyevopā\-sīta, sa ma ātmeti vidyāt'', etc. Therein many questions arise such as: What is Brahman, what are the qualities that  determine its having the nature of Brahman (brahmatā\-nirvā\-hakam guṇajātam), what is the nature of its knowledge, what is the nature of its (knowledge's) result, all of these is desired to be known, especially for those who desire mokṣa; this is because in the sacred texts there appears in the first instance(āpātataḥ) mutually contradictory meanings (anyonyaviruddhārthatāyāḥ) due to the difference (of beliefs)in the śākhās (śākhābhedena) (they follow). Therefore there is a need for an authoritative text (śāstram) on brahmamīmāmsā, in order to resolve (tannirṇayāya) these (differences).

\dev{नन्व “थातो धर्मजिज्ञासे” त्यादिपूर्वमीमांसयैव ब्रह्मज्ञानरूपधर्मस्यापि मीमांसितत्वान्नास्ति पुनराकाङ्क्षा । ब्रह्मज्ञानस्य च धर्मत्वं चोदनालक्षणत्वात् सिद्धम् “अयं तु परमो धर्मो यद् योगेनात्मदर्शन” मित्यादिस्मृतेश्च । वक्ष्यति चाचार्यः । सर्वासु वेदान्तविद्यासु चोदनां “सर्ववेदान्तप्रत्ययं चोदनाद्यविशेषा” दिति सूत्रेण, तत्र कुतर्कजातस्य च पश्चान्निराकरिष्यमाणत्वादिति। मैवम् । सामान्यतो धर्मत्वादिना
       निरूपणेऽपि अशेषविशेषनिर्धारणार्थं कल्पसूत्रादिवद् ब्रह्ममीमांसाया अप्यपेक्षितत्वात् ।}
       
\dev{ननु तथापि “सत्यं ज्ञानमनन्तं ब्रह्म, बिज्ञानमानन्दं ब्रह्मे” त्यादिश्रुतिसिद्धत्वाद् ब्रह्मस्वरूपे जिज्ञासा नोपपद्यत इति चेन्न, तदेव ज्ञानं किं सांख्ये सिद्धं 
जीवचैतन्यं किं वा चैतन्यान्तरमित्येवंस्वरूपजिज्ञासासत्त्वादिति । तदेवमाकाङ्क्षितत्वाद् ब्र ह्ममीमांसाशास्त्रस्यारम्भं प्रतिजानीते भगवान् वेदव्यासः—}

\textbf{Ques:} But as through pūrvamīmāmsā sūtras such as “athāto dharmajijñāsā” etc., there has been reflection on the dharma of the nature of knowledge of Brahman there cannot be that desire again. The knowledge of Brahman being dharma is also established because of having the quality of being known through the Vedas; there are also statements in the smṛtis such as “ayam tu paramo…yogenātmadarśanam”\endnote{Bhikṣu’s pet preference for Yoga comes out very early in the work}. Ācārya Bādarāyaṇa will also mention the injunction common to all meditation pertaining to Vedānta (sarvāsu vedāntavidyāsu) through the sūtra “sarvavedānta…viśeṣa” (BS III.3.1) with the intention of rejecting the illogical reasoning later.  

\textbf{Ans:} It is not so; even though, in general, there is decisions based on dharma, in order to reflect on the many special features (aśeṣaviśeṣanirdhāraṇārtham), similar to that of the Kalpasūtras, there is the need for the Brahmamīmāsā śāstra as well (brahmamīmāṁsāyā\break apyapekṣitattvāt).

\textbf{Ques:} Even so such statements as “satyam jñānamanantam brahma, vijñānamānandam brahma” are established in śruti and so it cannot be said that the desire to know about the nature of Brahman is not proper (brahmasvarūpe jijñāsā nopapadyate it cet)

\textbf{Ans:} It is not so. Is it the same knowledge already determined in Sāmkhya (philosophy) or is the consciousness of jīva of a different nature? Such kind of desire to know the real nature (of Brahman) is there. Thus since there is such a desire Bhagavān Vedavyāsa declares (pratijānīte) the commencement of the śāstra of brahmamīmāṁsā.

\section*{\dev{अथातो ब्रह्मजिञासा}}

\dev{अत्राथशब्द उच्चारणमात्रेण मङ्गलरूपोऽधिकारवाचकः । अधिकारश्च प्रकरणं ग्र  ग्रन्थान्येन निरूपणमिति यावत् । अतो ब्रह्मशेषतयाऽन्येषामपि मीमांसनमर्थाक्षिप्तम् । तथा प्रत्यधिकरणं ब्रह्मशब्दाभावेऽपि प्रकरणितया ब्रह्मैव \hbox{लब्धव्यमित्यादिकमथशब्दस्यप्रयोजनम्} । अत इत्यत्रेदमा प्रकृतं सूत्रमुच्यते, पञ्चमी चावधी, तथा च इदं सूत्रमारभ्येत्यर्थः । अर्थज्ञानात् प्रागेव सूत्रस्योपस्थितत्वान्न तत्परामर्शानुपपत्तिः, वक्ष्यमाणस्यापीदमा परामर्शदर्शनात् उत्तरसूत्रमारभ्येत्यर्थस्वतिसमीचीनः । तथा च यथा ग्रन्थशेषेऽ“नावृत्तिः शब्दादि” ति समग्रसूत्रद्विरावृत्तिः शास्त्रस्योत्तरावधिसूचिका तथैवात इति शब्दोऽपि तत्पूर्वावधिवाचकः । एवं हि शास्त्रपूर्वापरान्तद्वयावधारणे सति ग्रन्थमहावाक्यार्थबोधाय वाक्यान्तराकाङ्क्षया शिष्याणां विलम्बो भवतीत्याशयेन शास्त्रकृद्भिराद्यन्तावधी परिच्छिद्येते । अवश्यं चा ‘थातः’ शब्दयोर्यथोक्तार्थतैव शास्त्रान्तरेष्वभ्युपेया, यथा “अथातो व्रतमीमांसे” त्यादिश्रुतौ, अथातो गोभिलोक्तानामन्येषां चैव कर्मणाम् । अस्पष्टानां विधिं सम्यग् दर्शयिष्ये प्रदीपवत् ।। इत्यादिस्मृतौ । न हि तत्राधिकारावधी विहाया ‘थातः’ शब्दयोरन्यार्थता सम्भवति ।}

Here by the very utterance of the word ‘atha’ is indicated auspiciousness and denotes the authority (to write the work). ‘adhikāra’ indicates mainly the examination of a topic (prakaraṇam). Therefore since Brahman is the only residue (brahmaśeṣatayā) there is the need for reflection on all the other subjects as well (anyeṣāmapi). Even if the word Brahman is absent in every adhikaraṇa (of the BS), since that is the topic (prakaraṇitayā) one understands, therefore the usage of the word ‘atha’ is to indicate that the purpose (prayojanam) of the (discussion) is Brahman alone. 

The word ‘ata’ declares (ucyate) that here is the first sūtra; the fifth case is to denote the limit; thus it means starting from this place onwards. Even though the sūtra comes even before understanding the meaning (of  Brahman), it is not  unreasonable to inquire (parāmarśa) about it (Brahman), since one sees that what is going to be discussed (in the coming sūtras) is the same (topic); therefore the meaning ‘from the next sūtra onwards’ is most appropriate. Thus just as at the end of the work there is the indication of the end of the śāstra through the repetition of the sūtra “anāvṛttiḥ śabdādanāvṛttiḥ śabdāt” (BS. IV.4.22), so also the word ‘ata’ also denotes the starting point (of the śāstra). Thus in this way, (even) when the two limits of the beginning and the end of the śāstra are stated (clearly), since there is a delay because disciples/students desire to understand the meaning of the mahāvakyas through different vākyas, with this in mind (āśayena) the authors of the śāstra have divided the beginning and the end. It is necessary that the meanings of the words ‘athātaḥ’ are understood in the same manner in the other śāstras as well like “athāto brahmamīmāṁsā” and in smṛtis like “athāto gobhilo…pradīpavat”(the first śloka in the Karmapradīpa written by Kātyāyana; cited in Tripathi p.2 fn.1). Therein it is not possible to have any other meaning for ‘athātaḥ’ than the rule of limits (adhikārāvadhī).

\dev{ब्रह्मणो ब्रह्मशब्दार्थस्य जिज्ञासा ब्रह्मजिज्ञासा, अतो विशिष्य पूर्वं ब्रह्मज्ञानाभावेऽपि शिष्याणां न सूत्रवाक्यार्थबोधासंभवः । ब्राह्मणवेदहिरण्यगर्भादिषु ब्रह्मशब्दस्य गौणत्वेन न ब्रह्मशब्दार्थतेति वक्ष्यामः । जिज्ञासा चात्र \hbox{विचारो} मीमांसापरनामकः, जिज्ञासाशब्दस्य मीमांसाशब्दवद् विचारे रूढत्वात्। अथातो धर्मजिज्ञासेत्याद्यनेकशास्त्रेषु \hbox{जिज्ञासां} प्रतिज्ञाय विचारकरणदर्शनात्। श्रुतावपि ब्रह्मज्ञानेच्छयोपसन्नं शिष्यं प्रत्यपि “तद्विजिज्ञासस्व तद्ब्रह्मेति” पुनर्जिज्ञासोपदेशाच्च ।}

\dev{अत एव “अजिज्ञासितसद्धर्मो गुरुं मुनिमुपव्रजेदि”त्यादिवाक्येषु विचार एव जिज्ञासाशब्दः प्रयुज्यमानो दृश्यते, तत्रेच्छार्थकत्वासम्भवात् । तस्माद् योगेन  रूढतया च प्रकरणभेदेन जिज्ञासाशब्देन विचारेच्छयोरुभयोरेव वाचक इति बोध्यम् । विचारश्च विवरणं निर्णयहेतुभूतं लिङ्गाद्यवधारणम् । निर्णयश्चोक्तो न्यायाचार्यैः– “विमृश्य पक्षप्रतिपक्षाभ्यामर्थावधारणं निर्णयः” इति । स च निर्णयः वेदान्तैरेवेति वक्ष्यति “शास्त्रयोनित्वादि” ति सूत्रेण । तथा चायं सूत्रार्थ:— इदं सूत्रमारभ्य प्राधान्येन ब्रह्मविचारः तच्छास्त्रमस्माभिः क्रियत इति । यदि च जिज्ञासाशब्देन तच्छास्त्रं न लक्ष्यते तदा आचार्येण पूर्वमेव ब्रह्मणो निर्णीतत्वाद् विचारप्रतिज्ञा नोपपद्यते नोपपद्यते  च विचारं प्रतिज्ञाय सूत्रपरम्परारचनमिति ।}

The desire to learn the meaning of the word Brahman is ‘brahmajijñāsā’; ‘ataḥ’= even if there is the absence of knowledge especially regarding Brahman earlier, it is not impossible to instruct the śiṣyas the meanings of the sūtra vākyas (sūtravākyārthabodhāsambhavaḥ). Since the word Brahman is used in a secondary sense (gauṇatvena) in the Brāhmaṇas, Vedas, in Hiraṇyagarbha etc., we consider that they do not convey the right meaning of the word Brahman (na brahmaśabdārthatā). ‘jijñāsā’ here stands for reflection (and) is another word for mīmāṁsā, as similar to the word mīmāmsā the word jijñāsā also conventionally means contemplation/reflection; in many such śāstra texts as “athāto dharmajijñāsā” (one sees) that after declaring (pratijñāya) the desire to know, one sees engagement in thought (about the subject) (vicārakaraṇadarśanāt). In śruti as well the disciple who has approached (the guru) with the desire to learn about Brahman is again advised to have the desire to know in such statements as “tadvijijñā\-sasva tadbrahmeti” etc. That is the reason why in such statements as “ajijñāsitasaddharmo gurum munimupavrajet” it is seen that the word jijñāsā is used for thought process alone, since there is no possibility of the meaning of desire therein. Thus both etymologically and by convention the word jijñāsā denotes both (just) thinking/thought as well as desire to know, according to the context (prakaraṇabhedena). Thinking or a thought process is explaining (vivaraṇam) the cause such as the sign etc., based on which leads to a decision or conclusion (nirṇayahetubhūtam liṅ\-gādyavadhāraṇam). And a conclusion has been mentioned by Nyāya ācāryas as “vimṛṣya pakṣapratipakṣābhyām…nirṇayaḥ” (NS. I.1.41). By the sūtra “śāstrayonitvāt” it will be said that decision (nirṇaya) is through Vedānta (utterances) alone. Thus the meaning of this sūtra is ‘starting with this sūtra the main topic in general is reflection on Brahman (and) we are writing  the śāstra connected with that. If by the word jijñāsā that śāstra is not meant (tacchāstram na lakṣyate) then since ācārya (Bādarāyaṇa) has already determined (the nature of) Brahman it does not stand to reason to declare reflection (again on it); having promised reflection it is also not right to not start composing a set of sūtras (nopapadyate ca vicāram pratijñāya sūtraparamparāracanamiti).

\dev{आधुनिकास्तु प्रौढ्या सूत्रमिदमेवं व्याचक्षते — अधीतस्वाध्यायैर्विचारितकर्मकाण्डैरप्यकर्तृ\-त्वादिरूपेणा\-त्मनोऽनव\-धृतत्वात् तन्निर्धारणे चाविद्यानिवृत्त्या पुरुषार्थसिद्धेस्तन्निर्धारणायात्मनो ब्रह्मणो जिज्ञासा तदुपलक्षितो विचारः शिष्याणां कर्तव्यतया शास्त्रस्यादौ विधीयते “अथातो ब्रह्मजिज्ञासे”ति । अथशब्दो नित्यानित्यवस्तुविवेकेहामुत्रफलभोगविरागशमदमादिसंपन्मुमुक्षुत्वरूपसाधनचतुष्टयानन्तर्यमाह । अतः-शब्दश्च वक्ष्यमाणहेतुवाचकः । ज्ञातुं साक्षात्कर्तुमिच्छा जिज्ञासा तद्धेतुको विचारः । तथा चायमर्थः— यस्मादग्निहोत्रादिकमनित्यफलकं ब्रह्मज्ञानं चानन्तफलकमतः सर्वकर्माणि संन्यस्य शमदमादिसाधनचतुष्टयसंपन्नेन विविदिषुणा ब्रह्मविचार. षड्विधलिङ्गैर्वेदान्ततात्पर्यावधारणरूपो ब्रह्मसाक्षात्काराय कर्तव्य इति । “तद्विजिज्ञासस्व तद् ब्रह्मे”ति श्रुतेः । लिङ्गानि च पूर्वाचार्यैरुक्तानि— }

\begin{center}
\begin{tabular}{l}
\dev{उपक्रमोपसंहारावभ्यासोऽपूर्वता फलम् ।} \\
\dev{अर्थवादोपपत्ती च लिङ्गं तात्पर्यनिर्णये ॥ इति ॥}\\
\dev{तथा च विधिपरं सूत्रमिति वदन्ति । }
\end{tabular}
\end{center}

Modern day (vedāntins) through arrogance (prauḍhyā) explain this sūtra as follows:  Through instruction under a guru (svādhyāyaiḥ) and through  performing karma/rituals since one does not come to realize the non-doership nature of the ātman, in order to  realize that and achieve the final goal in life through the cessation of avidyā, there is the desire of ātman to know about Brahman and reflection pertaining to that (tadupalakṣito vicāraḥ); therefore at the start of the śāstra the sūtra “athāto brahmajijñāsā” is prescribed as a duty for disciples. They also say that the word “atha” denotes coming after the four fold prerequisites like “niyānityavastuviveka…mumukṣutvam”. The word “ataḥ” (according to them) denotes the reason which will be mentioned later; “jijñāsā” is the desire to realize (Brahman) directly and thought/reflection pertaining to that is the reason (for the composition of the sūtras) (jñātum sākṣātkartumicchā jijñāsā taddhetuko vicāraḥ). Thus the meaning (of the sūtra according to them) is that since rituals such as agnihotra etc., lead to results which are impermanent and knowledge of Brahman leads to a permanent result (anantaphalakam) so abandoning all rituals and  equipped with the four requisites like ‘śamadama’ etc., the one desiring to know (vividiṣuṇā) should engage in reflection on Brahman based on the sixfold marks (ṣaḍvidhaliṅgaiḥ) for determining the intended meaning of Vedānta, for the direct realization of Brahman (brahmasākṣātkā\-rā\-ya). The authority for this is the śruti “tadivjijñāsasva tadbrahmeti”. The sixfold marks mentioned by earlier ācāryas are as follows: upakramopasahārā\-bhyāsa…\-tātparyanirṇa\-yane”. Thus they say it is a sūtra which indicates a prescription (vidhiparam sūtramiti vadanti).

\dev{तत्रेदमुच्यते—कथं पुनः सर्वतन्त्रसाधारणमपेक्षितं  चोक्तार्थं हित्वा सामान्याभ्यामथात:शब्दाभ्या\-मेतादृ\-शोऽर्थ\-विशेषोऽवधारित इति । साधनाध्यायसूत्रेभ्य इति चेत्, अलमत्र तत्सूत्रणेन, एकाक्षरलाघवा ह्याचार्याः पुत्रोत्सवं मन्यन्ते । न वा साधनाध्याये सर्वकर्मत्यागं वा विचाराङ्गत्वेन शमदमादीन् वा विधास्यति, किन्तु विद्याया अकर्मशेषतया पुरुषार्थहेतुत्वं प्रतिज्ञाय तत्साधकत्वेन समाधिनिष्ठानां न्यायसिद्धं कामतः कर्मत्यागमनुवदति “उपमर्दं चेति” सूत्रेण । तथा विद्याप्रधाने संन्यासाश्रमे विधिं व्यवस्थापयिष्यति न सर्वकर्मत्यागे । तथा सम्प्रज्ञातयोगब्रह्मसाक्षात्काररूपायामेव विद्यायां फलपर्यवसायिन्यां शमदमाद्यन्तरङ्गसाधनं वक्ष्यति “शमदमाद्युपेतः स्यात्तथापि तु तद्विधेस्तदङ्गतया तेषामवश्यानुष्ठेयत्वादि” ति सूत्रेण । “तस्मादेवंविच्छान्तो दान्त उपरतस्तितिक्षु: समाहितो भूत्वा आत्मन्येवात्मानं पश्यती”त्यादिश्रुतौ संप्रज्ञातजसाक्षात्काराङ्गतयैव शमदमादिविधानात् । नारदीये नित्यानित्यविवेकादिसाधनचतुष्टयप्रतिपादनानन्तरम् —}
\begin{center}
\begin{tabular}{l}
\dev{चतुर्भिः साधनैरेतैर्विशुद्धमतिरच्युतम् ।}\\
\dev{सर्वगं भावयेद् विप्राः सर्वभूतदयापर: ॥}
\end{tabular}
\end{center}

\dev{इति वाक्येन साधनचतुष्टयस्य ध्यानाङ्गताया एव लाभाच्च । तत्र च यदि अङ्गाङ्गतया बहिरङ्गैरग्नीन्धनादिभिः कर्माङ्गैर्विक्षेपात् शमादिर्न संभवति तदा अन्तरङ्गानुरोधेन बाह्याङ्गान्यग्नीन्धनादीनि नापेक्षणीयानीत्येव वक्ष्यति “अत एव चाग्नीन्धनाद्यनपेक्षे”ति सूत्रेण, न तु कर्मत्यागं तेनापि सूत्रेण विधास्यति । तथा सति “सर्वापेक्षा च यज्ञादिश्रुतेरश्ववत्, सहकारित्वेन चे” त्युत्तरसूत्रविरोधात् “कर्मानपेक्षे” त्यस्यैव वक्तव्यतौचित्याच्च ।}

In that connection this is being stated: How is it that abandoning the intended meaning accepted by all disciplines (sarvatantrasādhāraṇa\-mapekṣitam coktārtham hitvā) you have determined such a special meaning of the common words “atha” and “ataḥ”. If it is said that it is understood from the sūtras in the sādhanādhyāya, enough of this argument using sūtras. Ācāryas consider the saving of even one letter as equivalent to the birth of a son [the implication being that they will not be using the sūtras which are composed with more than many akṣaras, to further their case].\endnote{Ekamātralāghavam putrotsavam manyante vaiyākaraṇāḥ (\dev{एकमात्रालाघवं पुत्रोत्सवं मन्यन्ते वैया\-करणाः}).} In the chapter on sādhanā (BS 3rd  adhyāya), neither the total giving up of all rituals (sarvakarmatyāgam) nor śama, dama etc., have been mentioned as part of the reflection (on Brahman). On the other hand due to the absence of the residual karma it is assured (pratijñāya) as being a cause for the attainment of one’s goal (puruṣārthahetutvam); and that being attainable for those dedicated to samādhi (tatsādhakatvena samādhiniṣṭhānām) there follows of its own accord the cessation of karma (karmatyāgamanuvadati) as a logical end (nyāyasiddham); this is in accordance with the sūtra “upamardam ca” (BS.III.4.16). So with reference to the āśrama of saṁnyāsa with its emphasis on meditation/knowledge the injunction will be this (sūtra) and not on the abandonment of all karma. Thus the sūtra “śamadamādyupetaḥ syāt…avaśyānuṣṭheyatvāt” (III.4.27) will mention that in relation to the result which occurs through the practice of samprajñātayoga-meditation for the direct realization of Brahman1 the practice of śama, dama etc., are prescribed as subsidiaries of knowledge. Thus in such śruti statements as “tasmādevamvit…paśyati”, śama, dama etc., have been prescribed as part of the direct realization (of Brahman) through samprajñāta yoga. After mentioning the four fold means like nityānityaviveka there is the verse (vākyena) “caturbhiḥ sādhanaiḥ…sarvabhūta\-dayāparaḥ” in the Naradīya, which includes the fourfold means as part of dhyāna (Bṛhannāradīya.39.54.cited in Tripathi p.3.fn1).

%page 8

\theendnotes
