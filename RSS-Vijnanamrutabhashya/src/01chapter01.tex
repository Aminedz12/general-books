\chapter{}\label{chap01}

%\Authorline{B. Sankareswari}

\begin{center}
\dev{\large\bfseries श्रीसरस्वत्यै नमः}

\dev{\large\bfseries श्रीगणेशाय नमः}
\end{center}

\begin{center}
\begin{tabular}{l}
\dev{सर्वत्र यो यत्र सर्वं यश्च सर्वमतो भवेत् ।}\\
\dev{चिदचिच्छक्तये तस्मै नमश्चिन्मात्ररूपिणे ॥ १ ॥}\\[2pt]
\dev{अन्तर्यामिगुरूद्दिष्टज्ञानविज्ञानभिक्षुणा ।}\\
\dev{ब्रह्मसूत्रऋजुव्याख्या क्रियते गुरुदक्षिणा ॥ २ ॥}\\[2pt]
\dev{श्रुतिस्मृतिन्यायवच:क्षीराब्धिमथनोद्धृतम् ।}\\
\dev{ज्ञानामृतं गुरोः प्रीत्यै भूदेवेभ्यो नु दीयते ॥ ३ ॥}\\[2pt]
\dev{परिविषय्य सद्बुद्ध्या मोहिन्येवाथ दानवान्}\\
\dev{कुतर्कान् वञ्चयित्वेदं पीयताममृतेप्सुभिः ॥ ४ ॥}\\[2pt]
\dev{पीत्वैतद् बलवन्तस्ते पाखण्डासुरयूथपान् ।}\\
\dev{विजित्य ज्ञानकर्मभ्यां यान्तु श्रीमद्गुरोः पदम् ॥ ५ ॥}
\end{tabular}
\end{center}


Bhikṣu starts his work as is the custom with some benedictory verses.
\begin{enumerate}
\item He that is everywhere, where everything is, and He that becomes everything eventually, to That (Supreme) of the nature of pure conscienceness, having the powers of consciousness  (jīvas) and insentience (prakṛti)(cidacicchaktaye), I bow down.\footnote{In Bhikṣu's avibhāga-vedānta  philosophypuruṣa and prakṛti are śaktis of Īśvara}

\item This Ṛjuvyākhyā interpretation of the Brahmasūtra is being offered as gurudakṣiṇā  by Bhikṣu who is a jñānavijñāna, keeping in mind his guru as the internal self.
          
\item This Jñānāmṛta which has been obtained after churning the ocean of śruti, smṛti and Nyāya reasoning is being offered to the devas on earth as gurudaksiṇā.

\item Examining well the demons through Mohinī in the form of truthful intelligence (and) tricking them through false reasoning those desiring nectar may drink this.

\item Having drunk this, those powerful ones after conquering the ignorant groups of demons and winning (liberation) through the means of both karma and knowledge may they reach the feet of the guru.
\end{enumerate}

\dev{ब्रह्मविदाप्नोति परम्, ब्रह्म वेद ब्रह्मैव भवति, तमेव विदित्वाति मृत्युमेति'' इत्यादिश्रुतिसिद्धपरमपुरुषार्थसाधनताके ब्रह्मज्ञाने विधिः श्रूयते --- ``आत्मेत्येवोपासीत, स म आत्मेति विद्यात् तमेव धीरो विज्ञाय प्रज्ञां कुर्वीत ब्राह्मणः'' इत्यादिरूपः । तत्र किं ब्रह्म, किं वा तस्य ब्रह्मतानिर्वाहकं गुणजातम्, कीदृशं वा तस्य ज्ञानम् , कीदृशं वा तस्य फलमित्यादिकं विशिष्य मुमुक्षूणां जिज्ञासितं भवति, श्रुतिष्वापाततोऽन्योन्यविरुद्धार्थतायाः शाखाभेदेन प्रतिभासनादिति। अतस्तन्निर्णयाय ब्रह्ममीमांसाशास्त्रमपेक्षितम्}

In order to attain the highest puruṣārtha which is the realization of Brahman established and stated in śruti as ``brahmviḍāpnoti param, brahma veda brahmaiva bhavati, etc'', one hears of imperative statements such as ``ātmetyevopā\-sīta, sa ma ātmeti vidyāt'', etc. Therein many questions arise such as: What is Brahman, what are the qualities that  determine its having the nature of Brahman (brahmatā\-nirvā\-hakam guṇajātam), what is the nature of its knowledge what is the nature of its (knowledge's) result, all of these is desired to be known, especially for those who desire mokṣa; this is because in the sacred texts there appears in the first instance(āpātataḥ) mutually contradictory meanings (anyonyaviruddhārthatāyāḥ) due to the difference (of beliefs)in the śākhās (śākhābhedena) (they follow). Therefore there is a need for an authoritative text (śāstram) on brahmamīmāmsā, in order to resolve (tannirṇayāya) these (differences).

\dev{नन्व “थातो धर्मजिज्ञासे” त्यादिपूर्वमीमांसयैव ब्रह्मज्ञानरूपधर्मस्यापि मीमांसितत्वान्नास्ति पुनराकाङ्क्षा । ब्रह्मज्ञानस्य च धर्मत्वं चोदनालक्षणत्वात् सिद्धम् “अयं तु परमो धर्मो यद् योगेनात्मदर्शन” मित्यादिस्मृतेश्च । वक्ष्यति चाचार्यः । सर्वासु वेदान्तविद्यासु चोदनां “सर्ववेदान्तप्रत्ययं चोदनाद्यविशेषा” दिति सूत्रेण, तत्र कुतर्कजातस्य च पश्चान्निराकरिष्यमाणत्वादिति। मैवम् । सामान्यतो धर्मत्वादिना}

\dev{निरूपणेऽपि अशेषविशेषनिर्धारणार्थं कल्पसूत्रादिवद् ब्रह्ममीमांसाया अप्यपेक्षितत्वात् ।}
       
\dev{ननु तथापि “सत्यं ज्ञानमनन्तं ब्रह्म, बिज्ञानमानन्दं ब्रह्मे” त्यादिश्रुतिसिद्धत्वाद् ब्रह्मस्वरूपे जिज्ञासा नोपपद्यत इति चेन्न, तदेव ज्ञानं किं सांख्ये सिद्धं जीवचैतन्यं किं वा चैतन्यान्तरमित्येवंस्वरूपजिज्ञासासत्त्वादिति । तदेवमाकाङ्क्षितत्वाद् ब्रह्ममीमांसाशास्त्रस्यारम्भं प्रतिजानीते भगवान् वेदव्यासः—}

\textbf{Ques:} But as through pūrvamīmāmsā sūtras such as “athāto dharmajijñāsā” etc., there has been reflection on the dharma of the nature of knowledge of Brahman there cannot be that desire again. The know\-ledge of Brahman being dharma is also established because of having the quality of being known through the Vedas; there are also statements in the smṛtis such as “ayam tu paramo…yogenātmadarśanam”\footnote{Bhikṣu’s pet preference for Yoga comes out very early in the work}. Ācārya Bādarāyaṇa will also mention the injunction common to all meditation pertaining to Vedānta (sarvāsu vedāntavidyāsu) through the sūtra “sarvavedānta…viśeṣa” (BS III.3.1) with the intention of rejecting the illogical reasoning later.

\textbf{Ans:} It is not so; even though, in general, there is decisions based on dharma, in order to reflect on the many special features (aśeṣaviśeṣanirdhāraṇārtham), similar to that of the Kalpasūtras, there is the need for the Brahmamīmāsā śāstra as well (brahmamīmāṁsāyā\break apyapekṣitattvāt).

\textbf{Ques:} Even so such statements as “satyam jñānamanantam brahma, vijñānamānandam brahma” are established in śruti and so it cannot be said that the desire to know about the nature of Brahman is not proper (brahmasvarūpe jijñāsā nopapadyate iticet)

\textbf{Ans:} It is not so. Is it the same knowledge already determined in Sāmkhya (philosophy) or is the consciousness of jīva of a different nature? Such kind of desire to know the real nature (of Brahman) is there. Thus since there is such a desire Bhagavān Vedavyāsa declares (pratijānīte) the commencement of the śāstra of brahmamīmāṁsā.

\section*{BS.I.1.1}

\section*{\dev{अथातो ब्रह्मजिञासा}}

\dev{अत्राथशब्द उच्चारणमात्रेण मङ्गलरूपोऽधिकारवाचकः । अधिकारश्च प्रकरणं ग्रन्थान्येन निरूपणमिति यावत् । अतो ब्रह्मशेषतयाऽन्येषामपि मीमांसनमर्थाक्षिप्तम् । तथा प्रत्यधिकरणं ब्रह्मशब्दाभावेऽपि प्रकरणितया ब्रह्मैव \hbox{लब्धव्यमित्यादिकमथशब्दस्यप्रयोजनम्} । अत इत्यत्रेदमा प्रकृतं सूत्रमुच्यते, पञ्चमी चावधी, तथा च इदं सूत्रमारभ्येत्यर्थः । अर्थज्ञानात् प्रागेव सूत्रस्योपस्थितत्वान्न तत्परामर्शानुपपत्तिः, वक्ष्यमाणस्यापीदमा परामर्शदर्शनात् उत्तरसूत्रमारभ्येत्यर्थस्वतिसमीचीनः । तथा च यथा ग्रन्थशेषेऽ“नावृत्तिः शब्दादि” ति समग्रसूत्रद्विरावृत्तिः शास्त्रस्योत्तरावधिसूचिका तथैवात इति शब्दोऽपि तत्पूर्वावधिवाचकः । एवं हि शास्त्रपूर्वापरान्तद्वयावधारणे सति ग्रन्थमहावाक्यार्थबोधाय वाक्यान्तराकाङ्क्षया शिष्याणां विलम्बो भवतीत्याशयेन शास्त्रकृद्भिराद्यन्तावधी परिच्छिद्येते । अवश्यं चा ‘थातः’ शब्दयोर्यथोक्तार्थतैव शास्त्रान्तरेष्वभ्युपेया, यथा “अथातो व्रतमीमांसे” त्यादिश्रुतौ,}
\begin{verse}
\dev{अथातो गोभिलोक्तानामन्येषां चैव कर्मणाम् । अस्पष्टानां विधिं सम्यग् दर्शयिष्ये प्रदीपवत् ।।}\\
\dev{इत्यादिस्मृतौ । न हि तत्राधिकारावधी विहाया ‘थातः’ शब्दयोरन्यार्थता सम्भवति ।}
\end{verse}

Here by the very utterance of the word ‘atha’ is indicated auspiciousness and denotes the authority (to write the work). ‘adhikāra’ indicates mainly the examination of a topic (prakaraṇam). Therefore since Brahman is the only residue (brahmaśeṣatayā) there is the need for reflection on all the other subjects as well (anyeṣāmapi). Even if the word Brahman is absent in every adhikaraṇa (of the BS), since that is the topic (prakaraṇitayā) one understands, therefore the usage of the word ‘atha’ is to indicate that the purpose (prayojanam) of the (discussion) is Brahman alone. 

The word ‘ata’ declares (ucyate) that here is the first sūtra; the fifth case is to denote the limit; thus it means starting from this place onwards. Even though the sūtra comes even before understanding the meaning (of  Brahman), it is not  unreasonable to inquire (parāmarśa) about it (Brahman), since one sees that what is going to be discussed (in the coming sūtras) is the same (topic); therefore the meaning ‘from the next sūtra onwards’ is most appropriate. Thus just as at the end of the work there is the indication of the end of the śāstra through the repetition of the sūtra “anāvṛttiḥ śabdādanāvṛttiḥ śabdāt” (BS. IV.4.22), so also the word ‘ata’ also denotes the starting point (of the śāstra). Thus in this way, (even) when the two limits of the beginning and the end of the śāstra are stated (clearly), since there is a delay because disciples/students desire to understand the meaning of the mahāvakyas through different vākyas, with this in mind (āśayena) the authors of the śāstra have divided the beginning and the end. It is necessary that the meanings of the words ‘athātaḥ’ are understood in the same manner in the other śāstras as well like “athāto brahmamīmāṁsā” and in smṛtis like “athāto gobhilo…pradīpavat”(the first śloka in the Karmapradīpa written by Kātyāyana; cited in Tripathi p.2 fn.1). Therein it is not possible to have any other meaning for ‘athātaḥ’ than the rule of limits (adhikārāvadhī).

\dev{ब्रह्मणो ब्रह्मशब्दार्थस्य जिज्ञासा ब्रह्मजिज्ञासा, अतो विशिष्य पूर्वं ब्रह्मज्ञानाभावेऽपि शिष्याणां न सूत्रवाक्यार्थबोधासंभवः । ब्राह्मणवेदहिरण्यगर्भादिषु ब्रह्मशब्दस्य गौणत्वेन न ब्रह्मशब्दार्थतेति वक्ष्यामः । जिज्ञासा चात्र \hbox{विचारो} मीमांसापरनामकः, जिज्ञासाशब्दस्य मीमांसाशब्दवद् विचारे रूढत्वात्। अथातो धर्मजिज्ञासेत्याद्यनेकशास्त्रेषु \hbox{जिज्ञासां} प्रतिज्ञाय विचारकरणदर्शनात्। श्रुतावपि ब्रह्मज्ञानेच्छयोपसन्नं शिष्यं प्रत्यपि “तद्विजिज्ञासस्व तद्ब्रह्मेति” पुनर्जिज्ञासोपदेशाच्च ।}

\dev{अत एव “अजिज्ञासितसद्धर्मो गुरुं मुनिमुपव्रजेदि”त्यादिवाक्येषु विचार एव जिज्ञासाशब्दः प्रयुज्यमानो दृश्यते, तत्रेच्छार्थकत्वासम्भवात् । तस्माद् योगेन  रूढतया च प्रकरणभेदेन जिज्ञासाशब्देन विचारेच्छयोरुभयोरेव वाचक इति बोध्यम् । विचारश्च विवरणं निर्णयहेतुभूतं लिङ्गाद्यवधारणम् । निर्णयश्चोक्तो न्यायाचार्यैः– “विमृश्य पक्षप्रतिपक्षाभ्यामर्थावधारणं निर्णयः” इति । स च निर्णयः वेदान्तैरेवेति वक्ष्यति “शास्त्रयोनित्वादि” ति सूत्रेण । तथा चायं सूत्रार्थ:— इदं सूत्रमारभ्य प्राधान्येन ब्रह्मविचारः तच्छास्त्रमस्माभिः क्रियत इति । यदि च जिज्ञासाशब्देन तच्छास्त्रं न लक्ष्यते तदा आचार्येण पूर्वमेव ब्रह्मणो निर्णीतत्वाद् विचारप्रतिज्ञा नोपपद्यते नोपपद्यते  च विचारं प्रतिज्ञाय सूत्रपरम्परारचनमिति ।}

The desire to learn the meaning of the word Brahman is ‘brahmajijñāsā’; ‘ataḥ’= even if there is the absence of knowledge especially regarding Brahman earlier, it is not impossible to instruct the śiṣyas the meanings of the sūtra vākyas (sūtravākyārthabodhāsambhavaḥ). Since the word Brahman is used in a secondary sense (gauṇatvena) in the Brāhmaṇas, Vedas, in Hiraṇyagarbha etc., we consider that they do not convey the right meaning of the word Brahman (na brahmaśabdārthatā). ‘jijñāsā’ here stands for reflection (and) is another word for mīmāṁsā, as similar to the word mīmāmsā the word jijñāsā also conventionally means contemplation/reflection; in many such śāstra texts as “athāto dharmajijñāsā” (one sees) that after declaring (pratijñāya) the desire to know, one sees engagement in thought (about the subject) (vicārakaraṇadarśanāt). In śruti as well the disciple who has approached (the guru) with the desire to learn about Brahman is again advised to have the desire to know in such statements as “tadvijijñā\-sasva tadbrahmeti” etc. That is the reason why in such statements as “ajijñāsitasaddharmo gurum munimupavrajet” it is seen that the word jijñāsā is used for thought process alone, since there is no possibility of the meaning of desire therein. Thus both etymologically and by convention the word jijñāsā denotes both (just) thinking/thought as well as desire to know, according to the context (prakaraṇabhedena). Thinking or a thought process is explaining (vivaraṇam) the cause such as the sign etc., based on which leads to a decision or conclusion (nirṇayahetubhūtam liṅ\-gādyavadhāraṇam). And a conclusion has been mentioned by Nyāya ācāryas as “vimṛṣya pakṣapratipakṣābhyām…nirṇayaḥ” (NS. I.1.41). By the sūtra “śāstrayonitvāt” it will be said that decision (nirṇaya) is through Vedānta (utterances) alone.\break Thus the meaning of this sūtra is ‘starting with this sūtra the main topic in general is reflection on Brahman (and) we are writing  the śāstra connected with that. If by the word jijñāsā that śāstra is not meant (tacchāstram na lakṣyate) then since ācārya (Bādarāyaṇa) has already determined (the nature of) Brahman it does not stand to reason to declare reflection (again on it); having promised reflection it is also not right to not start composing a set of sūtras (nopapadyate ca vicāram pratijñāya sūtraparamparāracanamiti).

\dev{आधुनिकास्तु प्रौढ्या सूत्रमिदमेवं व्याचक्षते — अधीतस्वाध्यायैर्विचारितकर्मकाण्डैरप्यकर्तृ\-त्वादिरूपेणा\-त्मनोऽनव\-धृतत्वात् तन्निर्धारणे चाविद्यानिवृत्त्या पुरुषार्थसिद्धेस्तन्निर्धारणायात्मनो ब्रह्मणो जिज्ञासा तदुपलक्षितो विचारः शिष्याणां कर्तव्यतया शास्त्रस्यादौ विधीयते “अथातो ब्रह्मजिज्ञासे”ति । अथशब्दो नित्यानित्यवस्तुविवेकेहामुत्रफलभोगविरागशमदमादिसंपन्मुमुक्षुत्वरूपसाधनचतुष्टयानन्तर्यमाह । अतः-शब्दश्च वक्ष्यमाणहेतुवाचकः । ज्ञातुं साक्षात्कर्तुमिच्छा जिज्ञासा तद्धेतुको विचारः । तथा चायमर्थः— यस्मादग्निहोत्रादिकमनित्यफलकं ब्रह्मज्ञानं चानन्तफलकमतः सर्वकर्माणि संन्यस्य शमदमादिसाधनचतुष्टयसंपन्नेन विविदिषुणा ब्रह्मविचार. षड्विधलिङ्गैर्वेदान्ततात्पर्यावधारणरूपो ब्रह्मसाक्षात्काराय कर्तव्य इति । “तद्विजिज्ञासस्व तद् ब्रह्मे”ति श्रुतेः । लिङ्गानि च\break पूर्वाचार्यैरुक्तानि— }
\begin{verse}
\dev{उपक्रमोपसंहारावभ्यासोऽपूर्वता फलम् ।} \\
\dev{अर्थवादोपपत्ती च लिङ्गं तात्पर्यनिर्णये ॥ इति ॥}\\
\end{verse}
\dev{तथा च विधिपरं सूत्रमिति वदन्ति । }

Modern day (vedāntins) through arrogance (prauḍhyā) explain this sūtra as follows:  Through instruction under a guru (svādhyāyaiḥ) and through  performing karma/rituals since one does not come to realize the non-doership nature of the ātman, in order to  realize that and achieve the final goal in life through the cessation of avidyā, there is the desire of ātman to know about Brahman and reflection pertaining to that (tadupalakṣito vicāraḥ); therefore at the start of the śāstra the sūtra “athāto brahmajijñāsā” is prescribed as a duty for disciples. They also say that the word “atha” denotes coming after the four fold prerequisites like “niyānityavastuviveka…mumukṣutvam”. The word “ataḥ” (according to them) denotes the reason which will be mentioned later; “jijñāsā” is the desire to realize (Brahman) directly and thought/reflection pertaining to that is the reason (for the composition of the sūtras) (jñātum sākṣātkartumicchā jijñāsā taddhetuko vicāraḥ). Thus the meaning (of the sūtra according to them) is that since rituals such as agnihotra etc., lead to results which are impermanent and knowledge of Brahman leads to a permanent result (anantaphalakam) so abandoning all rituals and  equipped with the four requisites like ‘śamadama’ etc., the one desiring to know (vividiṣuṇā) should engage in reflection on Brahman based on the sixfold marks (ṣaḍvidhaliṅgaiḥ) for determining the intended meaning of Vedānta, for the direct realization of Brahman (brahmasākṣātkā\-rā\-ya). The authority for this is the śruti “tadivjijñāsasva tadbrahmeti”. The sixfold marks mentioned by earlier ācāryas are as follows: upakramopasahārā\-bhyāsa…\-tātparyanirṇa\-yane”. Thus they say it is a sūtra which indicates a prescription (vidhiparam sūtramiti vadanti).

\dev{तत्रेदमुच्यते—कथं पुनः सर्वतन्त्रसाधारणमपेक्षितं  चोक्तार्थं हित्वा सामान्याभ्यामथात:शब्दाभ्या\-मेतादृ\-शोऽर्थ\-विशेषोऽवधारित इति । साधनाध्यायसूत्रेभ्य इति चेत्, अलमत्र तत्सूत्रणेन, एकाक्षरलाघवा ह्याचार्याः पुत्रोत्सवं मन्यन्ते । न वा साधनाध्याये सर्वकर्मत्यागं वा विचाराङ्गत्वेन शमदमादीन् वा विधास्यति, किन्तु विद्याया अकर्मशेषतया पुरुषार्थहेतुत्वं प्रतिज्ञाय तत्साधकत्वेन समाधिनिष्ठानां न्यायसिद्धं कामतः कर्मत्यागमनुवदति “उपमर्दं चेति” सूत्रेण । तथा विद्याप्रधाने संन्यासाश्रमे विधिं व्यवस्थापयिष्यति न सर्वकर्मत्यागे । तथा सम्प्रज्ञातयोगब्रह्मसाक्षात्काररूपायामेव विद्यायां फलपर्यवसायिन्यां शमदमाद्यन्तरङ्गसाधनं वक्ष्यति “शमदमाद्युपेतः स्यात्तथापि तु तद्विधेस्तदङ्गतया तेषामवश्यानुष्ठेयत्वादि” ति सूत्रेण । “तस्मादेवंविच्छान्तो दान्त उपरतस्तितिक्षु: समाहितो भूत्वा आत्मन्येवात्मानं पश्यती”त्यादिश्रुतौ संप्रज्ञातजसाक्षात्काराङ्गतयैव शमदमादिविधानात् । नारदीये नित्यानित्यविवेकादिसाधनचतुष्टयप्रतिपादनानन्तरम् —}
\begin{verse}
\dev{चतुर्भिः साधनैरेतैर्विशुद्धमतिरच्युतम् ।}\\
\dev{सर्वगं भावयेद् विप्राः सर्वभूतदयापर: ॥}
\end{verse}

\dev{इति वाक्येन साधनचतुष्टयस्य ध्यानाङ्गताया एव लाभाच्च । तत्र च यदि अङ्गाङ्गतया बहिरङ्गैरग्नीन्धनादिभिः कर्माङ्गैर्विक्षेपात् शमादिर्न संभवति तदा अन्तरङ्गानुरोधेन बाह्याङ्गान्यग्नीन्धनादीनि नापेक्षणीयानीत्येव वक्ष्यति “अत एव चाग्नीन्धनाद्यनपेक्षे”ति सूत्रेण, न तु कर्मत्यागं तेनापि सूत्रेण विधास्यति । तथा सति “सर्वापेक्षा च यज्ञादिश्रुतेरश्ववत्, सहकारित्वेन चे” त्युत्तरसूत्रविरोधात् “कर्मानपेक्षे” त्यस्यैव वक्तव्यतौचित्याच्च ।}

In that connection this is being stated: How is it that abandoning the intended meaning accepted by all disciplines (sarvatantrasādhāraṇa\-mapekṣitam coktārtham hitvā) you have determined such a special meaning of the common words “atha” and “ataḥ”. If it is said that it is understood from the sūtras in the sādhanādhyāya, enough of this argument using sūtras. Ācāryas consider the saving of even one letter as equivalent to the birth of a son [the implication being that they will not be using the sūtras which are composed with more than many akṣaras, to further their case].\footnote{Ekamātralāghavam putrotsavam manyante vaiyākaraṇāḥ (\dev{एकमात्रालाघवं पुत्रोत्सवं मन्यन्ते वैया\-करणाः}).} In the chapter on sādhanā (BS 3rd  adhyāya), neither the total giving up of all rituals (sarvakarmatyāgam) nor śama, dama etc., have been mentioned as part of the reflection (on Brahman). On the other hand due to the absence of the residual karma it is assured (pratijñāya) as being a cause for the attainment of one’s goal (puruṣārthahetutvam); and that being attainable for those dedicated to samādhi (tatsādhakatvena samādhiniṣṭhānām) there follows of its own accord the cessation of karma (karmatyāgamanuvadati) as a logical end (nyāyasiddham); this is in accordance with the sūtra “upamardam ca” (BS.III.4.16). So with reference to the āśrama of saṁnyāsa with its emphasis on meditation/knowledge the injunction will be this (sūtra) and not on the abandonment of all karma. Thus the sūtra “śamadamādyupetaḥ syāt…avaśyānuṣṭheyatvāt” (III.4.27) will mention that in relation to the result which occurs through the practice of samprajñātayoga-meditation for the direct realization of Brahman\footnote{Bhikṣu loses no opportunity to bring in his partiality towards yoga whenever he gets a chance} the practice of śama, dama etc., are prescribed as subsidiaries of knowledge. Thus in such śruti statements as “tasmādevamvit…paśyati”, śama, dama etc., have been prescribed as part of the direct realization (of Brahman) through samprajñāta yoga. After mentioning the four fold means like nityānityaviveka there is the verse (vākyena) “caturbhiḥ sādhanaiḥ…sarvabhūta\-dayāparaḥ” in the Naradīya, which includes the fourfold means as part of dhyāna (Bṛhannāradīya.39.54.cited in Tripathi p.3.fn1).

In that context by rejection of the several parts of the outer means of rituals like kindling the fire etc., if śama, dama etc., do not occur  (śamadamādirna sambhavati) then in compliance with the internal requisites, there is no need for external requisites such as kindling of the fire etc., this will be mentioned as “ata eva cāgnīndhanādyanapekṣā” (BS.III.4.25). When that is so then since it is in contradiction to the next sūtra “sarvāpekṣā…sahakāritvena ca” (ibid. III.4.26) it would have been appropriate to mention ‘not requiring any karma’ (karmānapekṣā).

\dev{नन्वेवमपि सूत्रद्वयोक्तयोरग्न्याद्यपेक्षानपेक्षयोर्विरोध एवेति चेन्न, संन्यासिनां बाह्याग्न्याद्यभावेऽपि स्वसमारोपितानां बाह्याग्न्यादीनामन्तरग्निहोत्रकालेऽपेक्षणीयत्वेनाऽविरोधात् । तथा च विष्णुपुराणे संन्यासप्रकरणे— }
\begin{verse}
\dev{कृत्वाग्निहोत्रं स्वशरीरसंस्थं, शारीरमग्निं स्वमुखे जुहोति ।}\\
\dev{विप्रस्तु भैक्ष्योपचितैर्हविर्भिश्चिताग्निना स व्रजति स्वलोकान्  ।। इति}
\end{verse}
\dev{मोक्षधर्मे च संन्यासप्रकरणे}
\begin{verse}
\dev{प्रादेशमात्रे हृदि निष्ठितं यत् तस्मिन् प्राणानात्मयाजी जुहोति ।}\\
\dev{तस्याग्निहोत्रं हुतमात्मसंस्थं सर्वेषु लोकेषु सदैवकेषु ।।}\\
\dev{उत्तान आस्येन हविर्जुहोति लोकस्य नाभिर्जगतः प्रतिष्ठा ।}\\
\dev{तस्याङ्गभङ्गानि कृताकृतं च वैश्वानरः सर्वमिदं प्रपेदे  ।। इति}
\end{verse}
\dev{न्यायाचार्यैश्च “समारोपणादात्मन्यप्रतिषेध” इत्युक्तम् । अग्न्यादीनामात्मन्यारोपणात् कर्मणामप्रतिषेध इत्यर्थः । अत्र मोक्षधर्मवाक्ये अङ्गहानावप्यन्तर्यागो योगिनां न दुष्यतीत्युक्तम् ।}

\textbf{Ques:}  Even so if it is said that there is a conflict between the two sūtras one prescribing rituals and other not doing so, then (the answer is);

\textbf{Ans:} Even if samnyāsins have no external (rituals) like fire kindling etc., since there is the necessity of internalizing within themselves the external (ritual) fire at the time of the agnihotra sacrifice there is no contradiction (avirodhāt). Thus in the Viṣṇu Purāṇa Saṁnyāsa chapter it says “kṛtvāgnihotram…vrajati svalokān” (III.7.30 cited in Tripathi p.4.fn.1). So also in the Mokṣadharmaparvan, under the saṁnyāsa chapter, it says “prādeśamātrehṛdi niṣṭhitam…sarvamidam prapede”   (Mokṣa.254.27. cited in ibid p.4.fn.2). The Nyāyācāryas have said “samāropaṇādātmanyapratiṣedha”.\footnote{NS.IV.1.60} It means that since the sacrificial fires have been internalized (samāropaṇāt) in the ātman, liberation is assured (apratiṣedhaḥ). In the (above) statement of the Mokṣaparvan, even if there is the cessation of the limbs of yoga, the internal sacrifice is not futile for the saṁnyāsins.

\dev{एतेन संन्यासिनां सर्वकर्मत्यागोऽशास्त्रार्थः। तथा च श्रुतिरपि “अत ऊर्ध्वममन्त्रवदाचेरदि”ति तथा च श्रुत्यन्तरं “सन्धिं समाधावात्मन्याचरेदि”ति च । आत्मनि स्वशरीरे समाधिमात्रेण देवैः सह सन्धिं सन्ध्याख्यं कर्माचरेदित्यर्थः । सान्ध्यम् सन्ध्येति । परमात्मात्मनोरेकत्वे  विज्ञानेन तयोर्भेद एव विभग्नः । तथा मनौ— }
\begin{verse}
\dev{सर्वभूतस्थमात्मानं सर्वभूतानि चात्मनि ।}\\
\dev{समं पश्यन्नात्मयाजी स्वाराज्यमधिगच्छति ।।}
\end{verse}
\dev{इत्यनेन ज्ञानस्यात्मयागेन सह समुच्चय उक्तः । आत्मयागश्च स्वशरीरस्थानामेव सर्वदेवानां स्वकीयस्नानाहारादिकालेषु स्वभोगैरेव मन्त्रादिनैरपेक्ष्येण यजनम् ।}
\begin{verse}
\dev{एतानेके महायज्ञान् यज्ञशास्त्रविदो जनाः ।}\\
\dev{अनीहमानाः सततमिन्द्रियेष्वेव जुह्वति ।।}\\[5pt]
\dev{आत्मैव देवताः सर्वाः सर्वमात्मन्यवस्थितम् ।}\\
\dev{खं संनिवेशयेत् खेषु चेष्टनस्पर्शनेऽनिलम् ।।}
\end{verse}
\dev{इत्यादिमनुवाक्यान्तरात् । तस्य चात्मयागस्य मनोमात्रसाध्यत्वमाह वशिष्ठः वैदिकं कर्म प्रकृत्य— }
\begin{verse}
\dev{बाह्यामाभ्यन्तरं चेति प्रत्येकं मुक्तिसाधनम् ।}\\
\dev{बाह्यं बहिः क्रियाभिश्च यत्तद् विहितसाधनैः ॥}\\
\dev{आभ्यन्तरं तु मनसा विध्यनुष्ठानमात्मनि ।}\\
\dev{तयोरन्यतरत् कुर्यान्नित्यं कर्म यथाविधि ।।}
\end{verse}
\dev{मोक्षधर्मे च —}
\begin{verse}
\dev{शान्तियज्ञरतो नित्यं ब्रह्मयज्ञरतो मुनिः ।}\\
\dev{वाङ्मनः काययज्ञैश्च भविष्याम्युरुगायन ।। इति ।}
\end{verse}
\dev{गौतमीयतन्त्रे च —}
\begin{verse}
\dev{केवलम्  सततं श्रीमच्चरणाम्भोजभाजिनाम् ।}\\
\dev{संन्यासिनां मुमुक्षूणां मानसः कथितो विधिः ॥}
\end{verse}
\dev{इत्युक्तम् । केवलमित्यनेन बाह्यकर्मणामेव त्यागो लब्धः, मानसकर्मविधानात् । वसिष्ठेनात्मयागस्य प्रशंसा च कृता — }
\begin{verse}
\dev{यष्टुमात्मन्यशक्तश्चेत् यजेद् बाह्येषु सर्वदा ।}\\
\dev{स्वयमुत्पन्नलिङ्गे वा स्थापिते वा विशेषतः ।।}
\end{verse}
\dev{इत्यात्मयागाशक्तावेव बाह्यस्यावश्यकत्वमित्यर्थः । एवं च सति संन्यासिनामप्यन्तरग्निहोत्रं गायत्र्यर्थब्रह्मात्मताध्यानरूपसंध्यादिकं वास्तीति न “यावज्जीवमग्निहोत्रं जुहोती” त्यादिश्रुतिविरोधः। तथा —}
\begin{verse}
\dev{नियतस्य तु संन्यासः कर्मणो नोपपद्यते ।}\\
\dev{मोहात्तस्य परित्यागस्तामसः परिकीर्तितः ।।}\\[3pt]
\dev{कर्मणां नियतानान्तु त्यागो नैव विधीयते ।}\\
\dev{तेषां कर्मफलत्यागः संन्यास इति चोच्यते ॥}
\end{verse}
\dev{इत्येवंविधवक्यान्यपि संन्यासविधिना न विरुध्यन्त इति मन्तव्यम् ।}

All this means that giving up all karma for saṁnyāsins is not the intention of the śāstras (śāstrārthaḥ). Thus there is the śruti which says “ata ūrdhvamamantravadācaret”\footnote{Āruṇeyopaniṣad.2}   there is also another śruti statement “sandhim samādhāvātmanyācaret” . It means that within oneself or within one’s own body through samādhi alone “sandhim”=one should do karma known as sandhyā; it means the twilight sandhyā (ritual). 

Paramātman and ātman being declared to be identical the statement of their being different is an aberration/obstruction (vibhaṅgaḥ). Manu’s following statement “sarvabhūtamātmānam…adhigacchati”\footnote{Manu Smṛti 12.91} indicates that there is a confluence of the sacrifice of the ātman and jñāna (jñānasya ātmayāgena saha samuccaya uktaḥ). Ātmayoga is the sacrifice (yajanam) to all the devas situated in one’s body during all one’s experiences like bathing, eating etc., without any mantras (mantrādinairapekṣyeṇa). There are other Manusmṛti statements which express the same idea such as “etāneke mahāyajñān…juhvati”\footnote{Ibid.4.22} and “ātmaiva devatāḥ sarvāḥ…ceṣtanasparśane’nilam”\footnote{Ibid.12.119  (in Jagadīśalal Śāstri (JŚ. edition the second line is the first line of 12.120)}.

Vasiṣṭha mentions that this sacrifice of the ātman is possible only\break through the mind as follows “bāhyābhyantaram ceti pratyekam mukti\-sādhanam…\-tayoranyatarat kuryānnityam karma yathāvidhi”. So also in the Mokṣadharmaparvan we have “śāntiyajñarato…bhaviṣyāmyuru\-gāyana” (175.32 cited in Tripathi p.5. fn.1).  The Gautamīyantra mentions the same idea as follows “kevalam satata…kathito vidhiḥ”. The word “kevalam” indicates the giving up only the external activities (karma) and by prescribing mental activities Vasiṣṭha has praised the yoga of the self (ātmayogasya praśaṁsā kṛtā) in the following manner “yaṣṭumātmanyaśaktaścet…viśeṣataḥ) i.e. only when one is incapable of practising ātmayoga there is the need for external (karma). When, in this manner, there is mention of kindling the agnihotra sacrifice internally or performing the sandhyākarma like meditating internally on the meaning of the Gāyatrī mantra which denotes Brahman even in the case of saṁnyāsins, it is not in contradiction to the śruti statement “yāvajjīvamagnihotram juhoti” (the agnihotra sacrifice is performed as long as one lives). So also such statements as “niyatasya tu samnyāsaḥ…samnyāsa iti cocyate” should be considered as not contrary to the prescribed duties of saṁnyāsins\footnote{Bhikṣu has given this long justification for jñānakarasamuccaya quoting from many sacred texts; it probably indicates the life he chose as a samnyāsin/bhikṣu for himself.}.

\dev{मोहादकर्तव्यताज्ञानात् कर्मत्यागस्य वैधत्वे तु तत्तच्छ्रुतिस्मृतयः संकुच्येरन् । असंकोचेनोपपत्तौ संकोचश्चान्याय्यो वैधस्य रात्रिश्राद्धस्येव प्रतिषेधानुपपत्तिश्च । प्राप्ताप्राप्तविकल्पग्रासात् पर्युदासाश्रयणे च “रात्रौ श्राद्धं न कुर्वीते” त्यादाविव कर्मत्यागवाक्येष्वपि लक्षणाभ्युपगमेन गौरवात् , कर्मत्यागवाक्यानाम् उपदर्शितविकल्पाद्यनुसारेण बाह्यकर्मत्यागपरत्वस्यैवौचित्याच्च । एतेन —}
\begin{verse}
\dev{सर्वाणि भूतानि सुखे रमन्ते, सर्वाणि दुःखेषु तथोद्विजन्ति ।}\\
\dev{तेषां भयोत्पादनजातखेदः, कुर्यान्न कर्माणि हि जातवेद:॥}
\end{verse}
\dev{इत्यादीनि जातवेदानां विदुषां कर्मत्यागविधायकवाक्यानि बाह्यकर्मत्यागपराण्येवावगन्तव्यानि, “आत्मक्रीड आत्मरतिः क्रियावानेष ब्रह्मविदां वरिष्ठ” इत्यादि श्रुतिभिरात्मारामस्यापि विदुषः कर्मावगमात् ।}
\begin{verse}
\dev{ज्ञानिनाऽज्ञानिना वापि यावद्देहस्य धारण}\\
\dev{तावद् वर्णाश्रमप्रोक्तं कर्तव्यं कर्म मुक्तये ॥}\\
\dev{ज्ञानेनैव सहैतानि नित्यकर्माणि कुर्वतः ।}\\
\dev{निवृत्तफलतृप्तस्य मुक्तिस्तस्य करे स्थिता ॥}
\end{verse}
\dev{इति वशिष्ठकौमार्दिस्मृतिभिर्विदुषोऽपि कर्मावश्यकत्वस्मरणाच्च । तथा कर्मत्यागहेतोर्हिंसाविक्षेपादेरान्तरकर्मण्यसंभवाच्च । संन्यासिनां कर्माभावे स्वशरीरेऽग्न्यादीनां देवादीनां वारोपणस्य वैफल्यापत्तेश्च ।}

If out of delusion or ignorance as to knowing what should not be done there is prescription of giving up karma then it will narrow (samkucyeran) the meaning of śruti and smṛtis. As it is proper to avoid narrowing the meaning and narrowing the meaning of something prescribed (vaidhasya) is not in conformance to reason (anyyāyo) and it is also not reasonable (to do something that is prohibited like the performance of śrāddha in the night (rātriśrāddhasyeva pratiṣedhānupapattiśca). Divided between what is meant or not meant or meant optionally (prāptāprāptavikalpagrāsāt) to lean towards an exception (paryudāsāśrayaṇe) and like (the rule) “one should not perform śrāddha in the night” by taking recourse to a secondary meaning in the statements mentioning the giving up of karma is cumbersome (karmatyāgavākyeṣvapi lakṣaṇābhyupagamena gauravāt); following the option of giving up karma in statements dealing with giving up karma it is also appropriate to understand (that it is) with reference to external karma alone (bāhyakarmatyāgaparatvasyaucityācca). Thus the utterances of learned scholars (vidvāns) like Jātaveda etc., “sarvāni bhūtāni…karmāṇi hi jātavedaḥ” (Mbh.Mokṣa. 244.25 Tri.p.5 fn.2) prescribing the abandonment of karma are to be understood as towards external karma alone; through such śruti statements as “ātmakrīḍa ātmaratiḥ kriyāvāneṣa brahmavidām variṣṭha”\footnote{Muṇḍaka.Up.III.1.4} one understands that even one who is blissful in the realization of the ātman (ātmārāmasyāpi) perform karma (karmāvagamāt). Through such verses as “jñāninā’\break jñāninā vāpi yāvaddehasya dhāraṇam…muktistasya kare sthitā” in\break smṛtis such as Vaśiṣṭha, Kūrma etc., learned scholars have reminded about the necessity of karma. Also in the case of internal karma there is no need to cite the reason for giving up karma being the discarding of violence involved (in external karma). Since there is no karma (prescribed) for samnyāsins to internalize in their bodies either the devas or agni etc., may also have the danger of being useless. 

\dev{ननु भवतामप्यारोपणमद्दृष्टार्थमेवाग्न्यादीनां देहे सत्त्वादिति चेन्न, बाह्याग्निसूर्याद्याधारकोपासनापरित्यागजदोषपरीहारस्यैव बाह्यग्न्याद्यारोपफलत्वात् । एवं ह्यन्तरे समष्टिव्यष्टिभेदेन सर्वदेवानां बाह्ययागोऽपि संपद्यत इति । एवमन्यान्यपि वाक्यानि “किमर्था वयं यक्षामहे” “अग्निहोत्रं न जुहवाञ्चक्रिरे”, “तस्मात् कर्म न कुर्वन्ति यतयः पारदर्शिनः । यदिदं वेदवचनं कुरु कर्म त्यजेति च ।।” इत्येवमादीनि विदुषां हिंसाविशेषादिदोषदुष्टबाह्यकर्मत्यागविधायकान्येव, न त्वावश्यकभिक्षादावप्यन्तर्यागस्य नमस्कारजपादेर्वा प्रतिषेधकानि । कर्मत्यागस्य शमादिपालनाय “गुणलोपो न गुणिन” इति न्यायेन दृष्टार्थकत्वात् । अत एव,}
\begin{verse}
\dev{ऋतुं प्राप्य यथा वृक्षः पत्रं त्यजति निस्पृहः ।}\\
\dev{तत्त्वं प्राप्य तथा योगी त्यजेत् कर्मपरिग्रहम् ।।}
\end{verse}
\dev{इत्यादिस्मृतिष्वपि कर्मपरिग्रहस्य कर्मोपकरणस्यैव त्यागोऽवगम्यते, अन्यथा त्यजेत् कर्म, इत्येव वक्तुमौचित्यात् । प्रणवजपे च शिखायज्ञोपवीताद्यभावेऽप्यधिकारोऽस्ति।  तापनोये “अशिखा अयज्ञोपवीता” इति परमहंसं प्रकृत्य “प्रणवे एव पर्यवसिता” इति श्रवणात् । यत्तु—}
\begin{verse}
\dev{यस्त्वात्मरतिरेव स्यादात्मतृप्तश्च मानवः ।}\\
\dev{आत्मन्येव च सन्तुष्टस्तस्य कार्यं न विद्यते ।।}\\
\dev{न केवलेन योगेन प्राप्यते परमं पदम् ।}\\
\dev{ज्ञानं तु केवलं सम्यगपवर्गफलप्रदम् ।।}
\end{verse}
\dev{इति भगवद्गीताकूर्मयोर्वाक्यं तन्निष्पन्नसमाधियोगिपरम् , तेषां बाह्यसंवेदनाभावेन कर्मसामान्याभावेऽदोषात् । तदुक्तं गीतायाम्—}
\begin{verse}
\dev{सर्वधर्मान् परित्यज्य मामेकं शरणं व्रज ।}\\
\dev{अहं त्वा सर्वपापेभ्यो मोक्षयिष्यामि मा शुचः ।। इति}
\end{verse}
\dev{तस्मात् स्वयमुद्भवत्समाधिं परित्यज्यान्तरमपि कर्म न कर्तव्यं प्रधानानुरोधेन गुणत्यागस्य न्यायसिद्धत्वात् । अत एवानेकसंवत्सरं व्याप्यापि बाह्याभ्यन्तरसर्वकर्मपरित्यागेन समाधाववस्थानं परमर्षीणां श्रूयत इति ।  तैरपि च व्युत्थानकाले   यथाशक्तिभिक्षादावन्तर्यागादिः क्रियत एवेति ।}

\textbf{Ques:}  If it is said that even in your case the internalization of fire etc., present in the body is for the purpose of adṛṣṭa then, we say;  

\textbf{Ans:} It is not so; it is for the sake of removing the defects caused by giving up meditation on the external fire which has for its support the sun, that the kindling of the external fire etc., within oneself is done (bāhyāgnisūryādyādhārakopāsanaparityāgajadoṣaprīhārasyaiva bāhyāgnyādyāropaphalatvāt). When it is so, one accomplishes the external sacrifice for all devas both collectively and individually (samaṣṭivyaṣṭibhedena). Thus other statements also by the learned like “kimarthā vayam yakṣyāmahe”, “agnihotram…cakrire”, “tasmāt karma…tyajeti ca” are prescribed for the giving up of external karma which are contaminated by defects such as violence and not against practice of internal sacrifice such as namaskāra, prayer etc., even during bhikṣā etc., which is a necessity.

In keeping with the maxim “guṇalopo na guṇina” it has a seen result. Therefore such smṛti statements like “ṛtum prāpya yathā vṛkṣaḥ…tyajet karmaparigraham” “karmaparigraha” is understood as the giving up of the tools of rituals, otherwise it would have been proper to say “tyajet karma”. The (samnyāsin) has the adhikāra to do japa with praṇava (Om) even when he has no śikhā, yajñopavīta etc.  With reference to the paramahamsa (samnyāsin) it is known that “praṇave eva paryavasitā”. Statements such as “yastvātmaratireva… samyagapavargaphalapradam” in the Bhagavadgītā (Gītā) and Kūrma Purāṇa are with reference to a yogī who has already achieved samādhi; in their case, as there is the absence of external feelings, no defect is there in the absence of common karma. Thus the Gītā says: “sarvadharmān parityajya…mā śucaḥ”\footnote{Gītā 18. 66}. Therefore, even in between, when there is abandonment of the automatic rise of samādhi (svayamudbhavatsamādhim) one need not do karma, as it is logical (in conformity to the principle) that the subsidiary is given up when the important one is present (pradhanānurodhena guṇatyāgasya nyāyasiddhatvāt). That is the reason why enveloping even the lapse of many years (anekasamvatsaram vyāpyāpi) it is heard of great ṛṣis being absorbed in samādhi having given up both external and internal karma.They also practice internal sacrifice during the time of coming out of samādhi (vyutthānakāle) and during bhikṣā etc., as much as possible (yathāśakti).

\dev{यत्तु—}
\begin{verse}
\dev{“यथोक्तान्यपि कर्माणि परिहाय द्विजोत्तमः ।}\\
\dev{आत्मज्ञाने शमे च स्याद्वेदाभ्यासे च यत्नवान् ॥”}
\end{verse}
\dev{इति मनुवाक्यं तदशक्तपरं बोध्यम्। तत्र हि कर्मत्यागो न विधीयते किन्तु रोगाद्यशक्त्या कर्माभावे प्रसक्ते विद्वानात्मज्ञानादितत्परो भवेदित्येव विधीयते ।}

\dev{केचित्तु—}
\begin{verse}
\dev{“लोकेऽस्मिन् द्विविधा निष्ठा पुरा प्रोक्ता मयानघ।}\\
\dev{ज्ञानयोगेन सांख्यानां कर्मयोगेन योगिनाम् ।।”}
\end{verse}
\dev{इत्यादिवाक्येभ्यो विद्वदविद्वद्विषयभेदेन कर्मत्यागकर्मणोर्व्यवस्थेति वदन्ति, तन्न, उत्पन्नज्ञानस्यापि कर्मविधिश्रवणेन ज्ञानयोगशब्दस्य समाधिवाचकत्वात् । तथा च एकाग्रतालक्षणज्ञानयोगेन समाध्याख्येन सांख्यानां विवेकाभ्यासिनां निष्ठा ज्ञाननिष्पत्तिः । कर्मयोगेन तु योगिनां योगाभ्यासिनां निष्ठा योगारूढता भवतीत्यर्थः । तथा चोक्तं गीतायामेव—}
\begin{verse}
\dev{आरुरुक्षोर्मुनेर्योगं कर्म कारणमुच्यते ।}\\
\dev{योगारूढस्य तस्यैव शमः कारणमुच्यते ।। इति ।}
\end{verse}
\dev{नन्वेवं कर्मत्यागस्याविधेयत्वे “विधिर्वा धारणवत्” इति साधनाध्यायस्थसूत्रेण किं विधेयमुक्तमिति चेत् , संन्यासाश्रम इत्यवेहि । कः पुनरयं संन्यासशब्दार्थः १ उच्यते, “आत्मन्यग्नीन् समारोप्य ब्राह्राणः प्रव्रजेद् गृहाद्” इति मन्वादिवाक्योक्तः स्वशरीरेऽग्न्याद्यारोपणपूर्वकः पुत्रगृहाद्यभिमानत्यागेन गृहात् प्रव्रजनरूप आश्रम इति । अपरश्च केवलकर्माभिमानकर्मफलत्यागादिः “कुर्यात् कर्मसंन्यासचिन्तनम् । काम्यानां कर्मणां न्यासं संन्यासं कवयो विदुः” इत्यादियाज्ञवल्क्यगीताद्युक्तो गौणसंन्यास एव । नैष्कर्म्यसिद्धिं परमां संन्यासेनाधिगच्छति” इति गीतावाक्येन कर्मफलसंन्यासफलस्य नैष्कर्मसिद्धिरूपस्य मन्वाद्युक्तसंन्यासाश्रमस्य पूर्वोक्तफलसंन्यासापेक्षया परमत्ववचनादिति । नैष्कर्म्यशब्दश्च अनुगीतादौ संन्यासविशेषवाची दृष्टो, यथा—}

\dev{अभयं सर्वभूतेभ्यो दत्त्वा नैष्कर्म्यमाचरेत् । सर्वभूतसमो मैत्रः सर्वेन्द्रिययतो मुनि:  इति कर्माभिमानतत्फलत्यागयोश्च गौणसंन्यासत्वेऽपि मुख्यकल्पत्वं मन्तव्यम् । “तयोस्तु कर्मसंन्यासात् कर्मयोगो विशिष्यते” इति गीतावाक्ये अभिमानफलत्यागपूर्वकस्य
कर्मयोगस्यैव सर्वकर्मत्यागापेक्षया श्रेष्ठत्ववचनात् । समाधिप्रयुक्ताऽशक्त्या रोगदारिद्र्याद्यशक्त्या चैव हि बाह्यकर्मत्यागसिद्धिरिति । अत एव श्रुतिः “आत्मरतिः क्रियावानेष ब्रह्मविदां वरिष्ठः” इति । अत एव जनकादिषु वसिष्ठादिषु च ज्ञानकर्मसमुच्चय एव दृश्यत इति । अधिकं तु साधनाध्याये वक्ष्यामः ।}

The statement of Manu: “yathoktāni karmāni…yatnavān” (MS. 12.92) that should be understood as referring to one who is incapable (of performing karma). Therein there is no prescription of giving up karma but in the event of illness etc., when there is the absence of (performance of) karma then the vidvān can apply himself to knowledge of the self. Some say that according to such verses as “loke’smin dvividhā…karmayogena yoginām” (Gītā.3.3) by dividing the two types of people as vidvān and avidvān there is the arrangement of the action of giving up karma; that is not so; even in the case of one who has attained knowledge (of the Self)  due to the prescription of karma (utpannajñānasyāpi karmavidhiśravaṇena), the word jñānayoga (in the above verse) denotes samādhi. Thus by jñānayoga\footnote{Bhikṣu interprets jñānayoga as samādhi. He is reluctant to give up the paramount importance of yoga for achievement of liberation to any other means.} characterized by one pointedness known as samādhi (ekāgratālakṣaṇajñānayogena samādhyakhyena) those followers of Sāṅkhya who are steadfast in the practice of discriminate discernment (vivekābhyāsinām) for them the purpose is the attainment of knowledge. On the other hand through karmayoga the yogīs who practice yoga have the purpose of climbing the heights of yoga. Thus there is the  statement in the Gītā: “ārurukśormuneryogam… śamaḥ kāraṇamucyate” (Gītā.6.3).

\textbf{Ques:} But then, in this manner, when giving up of ritual is not prescribed then what is prescribed by the śūtra “vidhirvā dhāraṇavat” (BS.III.4.20) in the chapter on the means.

\textbf{Ans:} Please know that it is only the saṁnyāsāśrama that is prescribed here (in the sūtra). 

\textbf{Ques:} What then is the meaning of this saṁnyāsa word; 

\textbf{Ans:} In accordance with Manu’s statement  “Ātmanyagnīn samāropya\-…gṛhād” (not traced) it means that it is the āśrama of the nature of wandering around after giving up the sense of possession (ego) towards the son, home etc., having internalized within oneself the sacred fires. The other (meaning ) according to Yājñavalkya, Gītā etc., is just the giving up of the sense of agency (ego) in the results of karma etc., (or) “one should think of giving up ritual action” (not traced) “as the sages (kavayo) understand (viduḥ) that saṁnyāsa is the giving up of ritual which has desired results” (kāmyānām karmaṇām nyāsam saṁnyāsam kavayo viduḥ ) (Gītā 18.2); this is known as a lower form of saṁnyāsa (gauṇa saṁnyāsa eva). By the Gītā statement “naiṣkarmyasiddhim paramām saṁnyāsenādhigacchati”(18.49) it  is known that by the use of the word ‘paramām’ the attainment of the result of total abandonment of karma due to abandoning the result of karma done, is superior to the saṁnyāsāśrama mentioned already by Manu.\footnote{Karmayoga where one does karma without any expectation of result is superior to Manu's restricted definition of giving up only the desired karma (kāmyakarma).} And the word naiṣkarmya has a special saṁnyāsa connotation as seen in the Anugītā etc. Thus “abhayam sarvabhūtebhyo…sarvendriyayato muniḥ”. (Aśva-Parvan: chap 45 cited in Tripathi.p.7.fn.3). Even though the giving up the expectation of the result of action as well as the sense of one’s agency regarding the action is present in gauṇa-saṁnyāsa as well one needs to understand it as of prime importance (mukhyakalpatvam mantavyam).  The Gītā statement “tayostu…karmayogo viśiṣyate” (5.2) mentions that as compared to the abandonment of all action (sarvakarmatyāgāpekṣayā) karmayoga preceded by giving up a sense of agency as well as expectancy of results is superior. There is also the giving up of external ritual due to the lack of  attention or being ill or not having the means (to perform external karma) (samādhiprayuktā’śaktyā rogadāridryādyaśaktyā caiva). Thus we have the śruti statement “ātmaratiḥ kriyāvāneṣa brahmavidām variṣṭhaḥ” (not traced). That is the reason one sees a combination of both ritual and knowledge in people such as Vasiṣṭha and Janaka. We shall discuss this further in the Sādhanapāda.

\dev{तस्मात् श्रवणार्थमविदुषां सर्वकर्मत्यागो बाह्यकर्मत्यागो वा शक्तानां न शास्त्रार्थः ।}

\dev{आतुरस्य तु बाह्यकर्माशक्तस्यान्तरकर्मानुष्ठानपूर्वकं जाबालोक्तसंन्यासं नैव निराकुर्मः। }
\begin{verse}
\dev{प्रकर्तुमसमर्थश्चेज्जुहोति-यजतिक्रियाः ।।}\\
\dev{अन्धः पङ्गुर्दरिद्रो वा विरक्तः संन्यसेद् द्विजः  ।। इति कूर्मवाक्यादिति।}
\end{verse}
\dev{यत्र श्रौतोऽयं विविदिषूणामपि सर्वकर्मसंन्यासः “परीक्ष्य लोकान् कर्मचितान् ब्राह्मणो निर्वेदमायान्नास्त्यकृतः कृतेन तद्विज्ञानार्थं स गुरुमेवाभिगच्छेत् समित्पाणिः श्रोत्रियं ब्रह्मनिष्ठं तस्मै स विद्वानुपसन्नाय सम्यक्प्रशान्तचित्ताय शमान्विताय येनाक्षरं पुरुषं वेद सत्यं प्रोवाच तां तत्त्वतो ब्रह्मविद्यामि” ति श्रुतेः । अत्र गुरुमेवेत्येवकारादिभिः सर्वकर्मत्यागावगमादिति । तत्र एवकारादगुरुभावेनोपगमव्यावृत्तेरेव लाभान्न तु कर्मव्यावृत्तेः । “क्रियावन्तः श्रोत्रिया ब्रह्मनिष्ठा स्वयं जुह्वन्त एक ऋषिं श्रद्धयन्तः तेषामेवैतां ब्रह्मविद्यां वदेदि” ति वाक्यशेषे क्रियावतामेवोद्देश्यत्वविधानात् समित्पाणित्वलिङ्गेन होमावगमाच्च । समित्पाणिशब्दस्योपहारपाण्यर्थकत्वोपवर्णनन्तु स्वाज्ञानवर्णनमेव । नास्त्यकृतः कृतेनेत्यनेन च कर्मणां मोक्षसाधनत्वमेव निराकृतम् , फलवत्सन्निधावफलस्यैवाङ्गत्वेन तस्य मोक्षासाधनत्वात् , न तु कर्मत्यागो विधीयत इति । प्रशान्तचित्तायेत्यनेनापि विद्याग्रहणोपयोग्येव प्रशम उक्तः न तु कर्मत्यागं, अर्जुनादिकर्मिभ्योऽपि ब्रह्मविद्योपदेशदर्शनादिति ।}

\dev{तथा “न कर्मणा न प्रजया न धनेन” इत्यादिश्रुतिरपि कर्मादीनामङ्गानामभावेऽपि एकेषां जडभरतादीनां मोक्षमेवानुवदति न तु कर्मत्यागं विदधाति । या च पुत्रैषणादित्यागश्रुतिः, साऽपि पुत्रादीच्छानामेव त्यागं विदधाति न तु कर्मणः ।}
\begin{verse}
\dev{त्यज धर्ममधर्मञ्च उभे सत्यानृते त्यज ।}\\
\dev{उभे सत्यानृते त्यक्त्वा येन त्यजसि तत्त्यज ।।}
\end{verse}
\dev{इति श्रुतिश्च धर्मादेस्तत्त्यागहेतुविवेकख्यातेश्च रागद्वेषादित्यागं विदधाति । “एतान्यपि तु कर्माणि सङ्गं त्यक्त्वा फलानि च । कर्तव्यानी” त्यादिगीताद्येकवाक्यत्वात् । अन्यथा जीवतां धर्माधर्मादित्यागासंभवात् , मूत्रोत्सर्गादावप्यन्ततः पिपीलिकादीनां विनाशात् ,}
\begin{verse}
\dev{न हि देहभृता शक्यं त्यक्तुं कर्माण्यशेषतः ।}\\
\dev{यस्तु कर्मफलत्यागी स त्यागीत्यभिधीयते ।।}
\end{verse}
\dev{इति गीतावाक्यादिति ।।}

Therefore, for the (understanding) of the common folk (aviduṣām)\break giving up all karma (sarvakarmatyāga) or giving up external karma (bāhyakarmatyāgo vā) in the case of those who can (śaktānām), is not the meaning of the śastra. For one who is ill, who is not capable of performing external karma we need not reject the saṁnyāsa preceded by observing internal karma mentioned by Jābāla. Thus there is this statement in the Kūrma. P “prakartumasamarthaścet…saṁnyased dvijaḥ” (Kūrma.P.adh.3.10 cited in Tri.p.8.fn.2) .  According to śruti regarding giving up karma even for those seeking to know about Brahman (vividuṣūṇāmapi) it says “parīkṣya lokān…brahmavidyām”\break (Muṇḍ.Up.I.2.12). 

\textbf{Ques:} In this case by stating ‘gurumevābhigacchet’ (approach only the guru) one understands the giving up of all karma. 

\textbf{Ans:} Therein the use of the word ‘eva’ only means to exclude approaching with the feeling of not having a guru and not of excluding ritual (karma). This is also understood from the remaining portion “kriyāvantaḥ śrotriyā…vadet” iti, and also by the use of the word ‘samit in the hand’ (samipāṇitvaliṅgena) one understands a homa (fire sacrifice). To understand the word ‘samitpāṇi’ as having some gift in hand only reveals one’s own ignorance. Through the words “nāstyakṛtaḥ kṛtena” (in the above śruti quotation) there is rejection of only karma being a means to mokṣa; since not having a result in the presence of the attainment of a result means (phalavat sannidhāvaphalasyaivāṅgatvena) it is not considered a means to mokṣa; however giving up karma is not prescribed; this is also demonstrated by the advice regarding knowledge of Brahman given even to those devoted to karma like Arjuna etc.

Even śruti words “na karmaṇā na prajayā na dhanena” etc., (Mahānā.Up. 8.14) only reiterate (anuvadati)  the mokṣa of persons such as Jaḍabharata etc., even when there is absence of ritual action etc., and do not advocate the giving up of ritual (karmatyāgam). The śruti statements such as giving up the desire for progeny etc., (putraiṣaṇādi tyāgaśrutiḥ) only mention the giving up the desire for putra etc., but not that of ritual karma. The śruti statement: “tyaja dharmamadharmañca…tyajasi tattyaja”(not traced), prescribe the giving up of attachment and hatred and also mention the giving up of discriminate discernment which is the cause of abandoning dharma etc. This is in concordance with the meaning given in the Gītā statement “etānyapi tu…tyaktvā phalāni ca” (Gītā 18.6). It is not possible while living to give up dharma and adharma, as even when one urinates or defecates there is destruction of ants etc.\footnote{The idea seems to be that since there is killing even in ordinary activities like urinating etc., one is subject to both dharma and adharma even while living. Adharma  is because of killing insects like ants etc., and dharma is prāyaścitta for such adhārmic acts I suppose. } This is made clear in the Gītā statement “na hi dehabhṛta…tyāgītyabhidīyate” (Gītā.18.11).

\dev{तस्मात् “विद्यां चाविद्यां च यस्तद्वेदोभयं सहे” त्यादिपरब्रह्मप्रकरणस्थवाक्यविरोधात् सर्वकर्माणि सन्न्यस्य श्रवणं कुर्यादित्याधुनिकानां विधिकल्पनं कुकल्पनमेवेति द्रष्टव्यम् । अतएव श्रवणानन्तरमेव मनुना संन्यासाश्रमो विहितः “वेदान्तान् विधिवत् श्रुत्वा संन्यसेदनृणो द्विज” इति । तस्मात् संन्यस्य श्रवणं कुर्यादित्यस्यायमेवार्थोऽज्ञस्याशक्त्या बाह्यकर्मत्यागरूपसंन्यासे सति श्रवणमवश्यं कर्तव्यमिति ।}

\dev{ज्ञानकर्मसमुच्चये चायं विशेषः “आरुरुक्षोर्मुनेर्योगं कर्म कारणमुच्यते । योगारूढस्य तस्यैव शमः कारणमुच्यते”\break इत्यादिवाक्यानुसारेणावगन्तव्यः— आरुरुक्षुत्वेनोत्सर्गतो ब्रह्मचारिगृहस्थयोः कर्माधिक्येनानुष्ठेयं ज्ञानं तु तदुपसर्जनम् , युज्यमानस्य तथा च वानप्रस्थस्योभे एष समे समच्चिते, योगारूढतया तु परिव्राजकस्य ज्ञानं प्रधानम् आन्तरं कर्म तदुपसर्जनमिति । एतेषु प्रकारेषूत्तरोत्तरमायासबाहुल्याद् हिंसादिदोषह्रासाच्च मोक्षाख्यफलेऽप्याश्वाशुतरादिरूपेण विशेषो मन्तव्यः ।}
\begin{verse}
\dev{जन्मान्तरैरभ्यसतो मुक्तिः पूर्वस्य जायते ।}\\
\dev{विनिष्पन्नसमाधिस्तु मुक्तिम् तत्रैव जन्मनि ।।}
\end{verse}
\dev{प्राप्नोतीति विष्णुपुराणादिति । पूर्वस्य पूर्वोक्तस्य युज्यमानस्येत्यर्थः । यथोक्तव्यवस्थया च प्रत्येकप्राधान्यसमसमुच्चयबोधकवाक्यानां पूर्वोत्तरमीमांसयोश्चाविरोध इति । प्रत्येकप्राधान्यसमुच्चयवाक्यं च यथा मोक्षधर्मे—}
\begin{verse}
\dev{ज्ञाननिष्ठां वदन्त्येके मोक्षशास्त्रविदो जनाः ।}\\
\dev{कर्मनिष्ठां वदन्त्यन्ये यतयः सूक्ष्मदर्शिनः ।।}\\
\dev{प्रहायोभयमप्येतज्ज्ञानं कर्म च केवलम् ।}\\
\dev{तृतीयेयं समाख्याता निष्ठा पञ्चशिखेन मे ।।}\\
\dev{तेनाहं सांख्यमुख्येन स्वदृष्टार्थेन तत्त्वतः ।}\\
\dev{श्रावितस्त्रिविधं मोक्षं न च राज्याद् विचालितः ।।}
\end{verse}
\dev{इति जनकेनोक्तम् । तत्र कर्मप्राधान्यसमसमुच्चयज्ञानप्राधान्यानां फलरूपं त्रिविधं मोक्षं क्रमेणाह तत्त्वसमासाख्यसांख्यस्य भाष्ये पञ्चशिखाचार्यः—}
\begin{verse}
\dev{आदौ तु मोक्षो ज्ञानेन द्वितीयो रागसंक्षयात् ।}\\
\dev{कृच्छ्रक्षयात्तृतीयस्तु व्याख्यातं मोक्षलक्षणम् ।।}
\end{verse}
\dev{इति । कृच्छ्रक्षय आत्यन्तिकदुःखाभावः ।}

Therefore, one should know that there is the mistaken view (kukalpana\-meveti) of the modern day Vedāntins misinterpreting the sentence in the chapter on the supreme Brahman like “vidyām cāvidyām ca yastadvedobhayam saha” (Īśa.Up.11; Mait.Up.7.9) as giving up all ritual karma and prescribing only listening to śruti (śravaṇam kuryāt). That is why Manu has prescribed the saṁnyāsāśrama only after listening to śruti etc.,\footnote{This is a reference to Manu’s prescribing the succession of the āśramas one after another (brahmacarya, gṛhastha, vānaprastha and then saṁnyāsa).} in the statement “vedāntān vidhivat śrutvā saṁyasedanṛṇo dvija” (not traced ).

There is this special characteristic in combining knowledge and ritual (jñānakarma samuccaye) and it has to be understood according to the statement “ārurukṣormuneryogam karma kāraṇamucyate; yogārūdhasya tasyaiva śamaḥ kāraṇamucyate” (Gītā. 6.3)\footnote{Bhikṣu uses this quote generously in all his works and is a firm believer in the combination of karma and jñāna till the very last stage of yoga.}. Generally for a brahmacārī and a gṛhastha who desire to become saṁnyāsins (muneryogam) there are a number of rituals to be observed and knowledge is subsidiary to karma (jñānam tu tadupasarjaman); for one who is striving and for one in the vānaprastha stage both karma and jñāna are equally important; whereas for the saṁnyāsin who has reached the summit of yoga knowledge is important and internal karma is its subsidiary (āntaram karma tadupasarjanam). In this succession of stages since there is great exertion in later stages and there is a diminishing of violence (in later stages) and also a growing confidence in the attainment of mokṣa, it can be considered as having this special quality (mokṣākhyaphale’pyāśvāśutarādirūpeṇa viśeṣo mantavyaḥ).

Thus from the Viṣṇu Purāṇa statement “janmāntarairabhyasato…muk\-tim tatraiva janmani” we understand that one attains (mukti there itself). The word “pūrvasya” in the above quotation refers to the one engaged in yoga (yujyamānasya). By the above mentioned methodology of explaining each mainly as an equal combination (of jñāna and karma) (yathoktavyavasthayā ca pratyekaprādhānyasamasamuccayabodhakavākyānām) it is not in opposition to pūrva and uttaramīmāṁsā. Thus this equal combination mainly of (jñāna and karma) is stated by Janaka in the Mokṣadharmaparvan as follows: “jñananiṣṭhām vadantyeke mokṣaśāstravido janāḥ…śrāvitastrividham mokṣam na ca rājyād vicālitaḥ”. There the threefold result of liberation of mainly the equal combination of karma and jñāna is mentioned in sequence (krameṇāha) as follows by Pañcaśikhācārya in the commentary on the Sāmkhya work Tattvasamāsa: “ādau tu mokṣo…mokṣalakṣaṇam”. “kṛcchrakṣaya” in the above quotation means absolute absence of sorrow.\footnote{This is Bhikṣu’s idea of liberation which he also reiterates in many of his works like the Yogavārttika and the Kaṭhopaniṣadāloka commentary.}  

\dev{इदानीं समाधिरहितस्यापि विविदिषुतामात्रेण सर्वकर्मत्यागे परोक्तकुतर्का अपि परिह्रियन्ते—}

\dev{ननु श्रुतिस्मृतीनां कर्मत्यागाविधायकत्वेऽपि ज्ञानिनामर्थादेव कर्मत्यागो लभ्यते ज्ञानकर्मणोरेकदा विरोधादिति, मैवम्, यदि ज्ञानशब्देन समाधिरुच्यते तदा एकदा विरोधेऽपि व्युत्थानदशायां भिक्षाटनात्मयागसंभवात् ,  कालाद्यङ्गहानावपि योगिनामदोषस्योक्तत्वात् । यदि पुनर्ज्ञानशब्देनानुभवमात्रमुच्यते तदा तेन सह कर्माभिमानस्येव विरोधान्न तु कर्मणः जडभरतादीनां भिक्षाटनादेर्भवतामप्यभ्युपगमात् , “कुर्याद् विद्वांस्तथाऽशक्तश्चिकीर्षुर्लोकसंग्रहमि” त्यादि वाक्यैर्वैदिककर्मणामपि विदुषिविधानाच्च । न चाभिमानाद्यभावेन ज्ञानिनां कर्म निष्फलमेवेति तत्त्याग एवोचित इति वाच्यम्,}
\begin{verse}
\dev{उत्सिदेयुरिमे लोका न कुर्यां कर्म चेदहम् ।}\\
\dev{सङ्करस्य च कर्ता स्यामुपहन्यामिमाः प्रजाः ।।}
\end{verse}
\dev{इत्यादिवाक्यतो लोकविनाशनजप्रत्यवायानुत्पत्तेरेव फलत्वसिद्धेः । “योगिनः कर्म कुर्वन्ति सङ्गं त्यक्त्वाऽऽत्मशुद्धये” इत्यादिवाक्यैः सत्त्वशुद्ध्याख्यपापक्षयहेतुतयैवाभिमानाख्यसङ्गशून्यकर्मणां साफल्याच्च । अन्यथा पापजन्यरोगादिना शरीरनाशादिभिर्ज्ञानपरिपाके प्रारब्धाक्षयेनैव प्रतिबन्धसंभवात् “कृतनियमलङ्घनादानर्थक्यमिति’’ सांख्यसूत्रादिभ्य आनर्थक्यं ज्ञानस्येति प्रकृतत्वाल्लभ्यते ।}

\dev{नन्वविद्याया अदृष्टहेतुत्वादविद्यानिवृत्तौ पापोत्पत्तिरेव न सम्भवतीति कुतो मोक्षप्रतिबन्धरशङ्केति चेन्न, ज्ञानिभ्योऽपि कर्मविधानाद् वृथाकर्मत्यागजपापातिरिक्तादृष्टेष्वेवाऽविद्याया हेतुत्वावधारणात् । }

\dev{अतएव “आप्रायणात् तथाहि दृष्टमि” ति सूत्रेण मरणपर्यन्तमेवानुभविनोऽपि विद्यावृत्तिं वक्ष्यति प्रतिबन्धनिरासार्थमिति अतोऽपि चानिर्दिष्टकालविशेषाणां ज्ञानाङ्गकर्मणां ज्ञानकालिकतया मरणपर्यन्तता लभ्यत इति ।}

Now the false argument of others of giving up all karma even by those not attaining samādhi (samādhirahitasyāpi) but just desiring (to attain knowledge of the Self) (vividiṣutāmātreṇa) is removed (as follows)—

\textbf{Ques:} But then, even though śruti and smṛti (texts) do not prescribe giving up of karma still by the very meaning of ‘those possessed of knowledge’ (jñānināmarthādeva) giving up of karma is implied, as there is a contradiction in both having karma and jñāna at the same time.

\textbf{Ans:} No it is not so; if by the word jñāna, samādhi is implied then even though there is a contradiction of both (karma and jñāna) coexisting simultaneously, during the intervals of activity (vyuthānadaśāyām) self-sacrifice is possible when wandering about for alms (bhikṣāṭanātmatyāgasaṁbhavāt); even when due to passage of time the limbs are weak no blame is there for the yogīs. If then by the word jñāna, only the experience (of realization) is intended then it is only in contradiction to the sense of agency of karma and not to karma per se. Since you also accept the wandering for alms of people like Jadabharata etc., and through such statements as “kuryād vidvāmstathā…lokasangraham” (Gītā.3.25) even in matters pertaining to the Vedas (vaidika) lack of blame is prescribed in one who knows.\footnote{As action is done without a sense of agency.}  It cannot be said that the actions of wise people (jñāninām karma) done without any pride (abhimānādyabhāvena) is not without a result. From such statements as “utsideyurime…prajāḥ” (ibid.3.24) it is understood that the result is the non-arising of impediments which lead to the destruction of the world.\footnote{The Vedic belief that the good actions of all humans contribute to the maintenance of dharma and stability
of the world is echoed here.} Statements such as “yoginaḥ…ātmaśuddhaye” (ibid.5.11) indicate the expected good results of deeds done without any sense of agency/pride due to the decline of pāpa due to the cause known as\break sattva-śuddhi (sattvaśuddhyākhyapāpakṣayahetutayaiva abhimānā\-khyasaṅgaśūnyakarmaṇām sāphalyācca)\footnote{Bhikṣu is committed to Sāṁkhya-Yoga metaphysics and epistemology. He therefore talks about the gradual
sattva-śuddhi of the intellect of selfless action. All schools of Indian philosophy believe in selfless action
(niṣkāma-karma) as a preliminary step in the quest for liberation.}. Otherwise because of the body weakening due to diseases caused by pāpa, since the fruition of knowledge can only happen when opposed by the weakening of the prārabdhakarma, one understands from Sāṁkhyasūtras like “kṛtaniyamalaṅghanādānarthakyam”etc., that knowledge which is understood from what is mentioned (prakṛtatvāllabhyate) is useless.

\textbf{Ques:} But then since avidyā is the cause of adṛṣṭa, when avidyā disappears (in the case of a jñānī) (avidyānivṛttau) there can be no rise of pāpa; then where is the doubt of having an obstacle for mokṣa.

\textbf{Ans:} That is not so; even in the case of those who are jñānīs since karma is prescribed, it is useless to assign avidyā as a cause for only adṛṣṭa apart from the pāpa arising from the giving up of karma. That is why by the sūtra “āprāyaṇāt tathāhi dṛṣtam” (BS.4.1.12) even for those who have experienced the truth (anubhavino’pi) the observance of vidyā will be mentioned up to the time of death for the sake of removal of obstacles. Moreover engaging in karma which are the limbs of knowledge, having no fixed times (anirdiṣṭakālaviśeṣam) one understands the limit for the rise of knowledge is up to the time of death. 

\dev{ननु अभिमानाभावेनात्मनि नियोज्यत्वप्रत्ययासम्भवाद् विदुषां विद्याकिंकरता न सम्भवतीति चेन्न, अभिमानाऽभावेऽप्युपाधौ नियोज्यत्वप्रत्ययसम्भवात् अन्यथा भिक्षादावपि नियोज्यत्वप्रत्ययो न स्यात् ।}

\dev{स्यादेतत्, प्रवृत्तौ स्वीयसुखादिसाधनताज्ञानं कारणं, परसुखादिषु तत्साधने वा हानोपादानादर्शनात् । तथा च सुखादीनामनात्मधर्मतया तेषु स्वीयत्वज्ञानाऽसम्भवात् कथं तत्साधने कर्मादौ विदुषां प्रवृत्तिः स्यादिति ? उच्यते, सांख्यादिभिः स्वभोग्यत्वमेव धनादिष्विव सुखदुःखयोरपि स्वीयत्वमुच्यते, न तु नैयायिकादिवत् स्वसमवेतत्वम् । स्वभोग्यत्वं च स्वसाक्ष्यत्वम्। साक्ष्यत्वं चोपाधिवृत्तिविषयत्वं विना भास्यत्वमतो न योगिनां परमेश्वरस्य वा परसुखादिषु स्वीयत्वम् । एतेन ज्ञानिनां स्वकृतिसाध्यतादिज्ञानमप्युपपादितम् । स्वोपाधिकृतावपि साक्ष्यत्वरूपस्वीयत्वसंभवात्, संस्कारादौ धनादौ च स्वीयसुखादिसाधनत्वात् स्वीयत्वम् । नन्वेचं नाहं सुखीत्यादिविवेको नोपपद्यतेति चेत् न, समवायसम्बन्धेन अहं सुखीत्यादिप्रत्ययस्यैवाविद्यात्वेन समवायसम्बन्धावच्छिन्नतदभाचप्रत्ययस्यैव विवेकशब्दार्थत्वात्, अधिकारित्वेनैव शास्त्रेषु बुद्ध्यादिभ्य आत्मनो विवेचनाच्चेति । तस्माद् विदुषामपि स्वीयसुखादिसाधनताज्ञानं स्वोपाधिप्रवृत्तेर्हेतुः संम्भवत्येव ।}

\textbf{Ques:} If it is said that due to absence of ego there is no possibility of the idea of being one charged with any duty in the ātman so there is no possibility for knowledge  to have the capacity to accomplish its purpose for the wise (viduṣām kimkaratā na sambhavatiti) then (there is the following)

\textbf{Ans:} It is not so; even when a sense/knowledge of agency is absent,  in the limitation there is the sense of being charged with duty (upādhau niyojyatvapratyayasaṁbhavāt); otherwise even in the act of wandering for alms etc., there will not be the idea of being charged with a duty.

\textbf{Ques:} Let this be. In any activity the cause is the knowledge of its being a means to one’s own happiness (svīyasukhādisādhanatājñānam kāraṇam); one does not see either acceptance or rejection with regard to another’s pleasure or in its being an instrument for this (parasukhādiṣu tatsādhanevā hānopādānādarśanāt). Thus since pleasure etc., being qualities of the non-ātman it is not possible to have a sense of them belonging to oneself (svīyatvajñānā’saṁbhavāt); so how can the wise be engaged in karma which accomplishes such results (tatsādhane karmādau viduṣam pravṛttiḥ syāditi?).

\textbf{Ans:} It is said by Sāmkhya believers that both pleasure and pain are also to be enjoyed by oneself (svabhogyatvameva svīyatvamucyate) just as wealth etc., is fit to be enjoyed (by oneself. It is not like the view of the Naiyāyikas that it is in a relationship of inherence (svasamvetatvam). And being able to enjoy oneself means the quality of being a witness oneself. And the quality of being a witness is being able to possess knowledge without the quality of being an object of the modification of the limitation (i.e. the mind) (copādhivṛttiviṣayatvam vinā bhāsyatvam); thus neither the yogīs nor Parameśvara have the sense of the pleasure etc., of others belonging to itself. The result of this is that it also makes it possible for the wise to have knowledge etc., which accomplishes one’s own purpose (svakṛtisādhyatādijñanamapi upapāditam). Even though achieved by the limitation, as there is the possibility of possessing it as one’s own in the form of the witness (sākṣyatvarūpasvīyatvasaṁbhavāt), therefore both in the case of saṁskāras etc., or wealth and so on, since it is able to accomplish the idea of pleasure belonging to oneself, one calls it as belonging to oneself.

\textbf{Ques:} But then if it is said that there will not be the rise of the wisdom in the form ‘I am not the repository of pleasure’ etc.,(nāham sukhītyādiviveko nopapadyeta) then the answer is:

\textbf{Ans:}  No; due to the relationship of inherence, since the thought such as “I am happy” etc., are due only through having ignorance (avidyātvena), the word meaning ‘insight’ indicates the absence of thought which has the delimitation of the relationship of inherence (samavāyasaṁbandhāvacchinnatadabhāvapratyayasyaiva vivekaśabdārthatvāt). Moreover in the authoritative texts (śāstreṣu) ātman has been differentiated from the intellect etc., as not being subject to any change (avikāritvenaiva buddhyādibhya ātmano vivecanācceti). Therefore even in the case of the wise the knowledge of the means of one’s own pleasure etc., is only possible due to the cause of the activity of one’s own limitation. 

\dev{अथ तथापि आत्मतृप्ततया ज्ञानिनां सुखादाविच्छैव नास्तीति चेन्न, रागरूपाया अविद्याजन्येच्छाया एव ज्ञानिनां शास्त्रेषु प्रतिषेधात् “दुःखजन्मप्रवृत्तिदोषमिथ्याज्ञानानामुत्तरोत्तरापाये तदनन्तरापायादपवर्ग” इति न्यायसूत्रादिभिः, न तु लीलारूपादीच्छा प्रतिषिद्धा भिक्षादिदर्शनविरोधात् । ननु तथापि देहेन्द्रियादिभिः सम्बन्धाभावाद् विदुषां कथं प्रवृत्तिः ? न ह्यसङ्गस्यात्मनो देहादिभिः सहाभिमानातिरिक्तः सम्बन्धः सम्भवतीति चेन्न, विदुषां ज्ञानोपपत्यर्थम् असङ्गवाक्यैर्लोपाख्यस्य विकारहेतुसंयोगस्यैव निषेधात् न तु पुष्करपत्रे जलस्येवासङ्गेऽपि चेतने स्योपाधेः संयोगविशेषः प्रतिषिध्यते “आत्मेन्द्रियमनोयुक्तं भोक्तेत्याहुर्मनीषिणः” इति श्रुतेः । स च संयोगविशेषः प्रारब्धकर्मक्षयादेव नश्यतीति सर्वैरेवाभ्युपेयम् ; अन्यथा ज्ञानाद्यनुपपत्तरिति ।}

\dev{विषयैः प्रतिबिम्वरूपस्य बन्धस्यापि च ज्ञानिसाधारण्याद् विषयभोगोऽपि ज्ञानिनामुपपन्नः । स्वप्रतिबिम्बितस्योपाधिसुखस्य भानमेव भोग इति वक्ष्यमाणत्वात् ।}

\dev{[न च “सति मूले तद् विपाको जात्यायुर्भोगाः” इतिपातञ्जलसूत्रेण क्लेशसत्त्व एव कर्मविपाको भवतीति वचनात् कथं विदुषां भोगः स्यादिति वाच्यम् ‘‘क्लेशाभावे कर्मविपाकारम्भो न भवती” ति तत्सूत्रभाष्यतो विपाकारम्भे क्लेशहेतुताया एव तत्सूत्रार्थत्वात्, ज्ञानिना च प्रारब्धविपाकमेव भुज्यत इति ।]}

\textbf{Ques:} But then if it is said that due to being satisfied within oneself (knowing the truth about the self), the wise have no desire for pleasure etc., (ātmatṛptatayā jñāninām sukhādāvicchaiva nāstiti cenna) then the answer is:

\textbf{Ans:} It is not so; it is only desire in the form of attachment which is desire that arises out of avidyā which is prohibited in the śāstras; through such Nyāyasūtras as “duḥkhajanma…tadanantarāpāyādapavarga”\break (I.1.2) there is no prohibition of desire which is of the nature of sport (na tu līlārūpādīcchā pratiṣiddhā), as that will be in contradiction to what we see as roaming around for alms.

\textbf{Ques:} But even then how can there be any activity of the wise in the absence of a relationship with the body, sense organs etc. If it is said that the non-attached ātman has a relationship apart from that associated with a sense of agency (abhimānātiriktaḥ saṁbandhaḥ saṁbhavatīti) then the answer is:

\textbf{Ans:} It is not so; for the sake of rise of knowledge in the wise through statements which denote detachment, there is rejection of any contact with the cause for change, due to what is known as any impurity (asaṅgavākyairlepākhyasya vikārasaṁyogasyaiva niṣedhāt); nor is there rejection of a special contact of caitanya with its limitation (cetane svopadheḥ saṁyogaviśeṣah niṣidhyate); even if there is no contact like a lotus leaf with the water on it (na tu puṣkarapatre jalasyevāsaṅge’pi), there is no rejection of a special contact of caitanya with its limiation . This is in accordance with the śruti statement: “ātmendriyamanoyuktam…manīṣiṇaḥ” (Kaṭh.Up.I.3.4). And all agree that special contact will only be destroyed when the prārabdha-karma comes to an end; otherwise there will be an absence of reasonable ground for (the rise of) knowledge etc.

Bondage through objects in the form of reflection being common, it is also logical for the wise to have experience (viṣayabhogo’pi jñānināmupapannaḥ). This is so since it will be mentioned that experience is only the knowing/illumination of pleasure of the limitation which is reflected by oneself (svapratibimbitasyopādhisukhasya bhānameva boga iti vakṣyamāṇatvāt).\footnote{This is in accordance with the epistemology of SY} 

\textbf{Ques:} Since through Patañjali’s sūtra “sati mūle tadvipāko jātyāyurbhogāḥ” (YS.II.13) we know that only when there is kleśa will there ensue the result of karma (the question is) how can there be experience in the case of those who are wise (who have realized the truth due to absence of kleśa). 

\textbf{Ans:} “kleśābhāve…na bhavati”—this bhāṣya on that sūtra clarifies that the meaning of that sūtra is that in the rise of vipāka the cause is kleśa; therefore it means that the vipāka of the prārabdhakarma alone is experienced by the knowing one (jñaninā).

\dev{ये स्वाभिमानमेव बुद्ध्यात्मनोः सम्बन्धं मन्यन्ते, तेषामेव ज्ञानेन सवासनाज्ञाननिवृत्त्या विदुषां भोगानुपपत्तिः । ज्ञानपरिपाकोत्तरमपि यावद्देहपातमविद्यालेशस्वीकारे च स लेशो न नश्येतैव, देहारम्भककर्मनाशस्य वासनानाशकत्वे प्रमाणाभावात्, अविदुषोऽपि पुनर्जन्मासम्भवाच्च । स्वयं विनाशे कदाचित् ज्ञानं विनाऽपि मुक्तिः स्यात् । अज्ञानज्ञानयोर्नाश्यनाशकभावे व्यभिचारप्रसङ्गाच्च । कार्यतावच्छेदकविशेषस्य च दुर्वचत्वात् ।}

\dev{किञ्च बुद्ध्यात्मनोरध्यासरूपः सम्बन्धः किमहङ्करोमीत्यादिविशिष्टबुधिनियामकतया कल्प्यते? किं वा बुद्धिप्रवृत्तिनियामकतया ? नाद्यः, अहङ्करोमीत्यादिप्रतीतेः सम्बन्धविधया स्वविषयत्वे कर्मकर्तृविरोधात्, स्वजनकत्वे तु आत्माश्रयात्,  अधिकाधिकाङ्गीकारे चान्योन्याश्रयचक्रकानवस्थादिप्रसङ्गात् , अहं कर्तेत्यादि विशिष्टबुद्धिविवेकज्ञानाऽनाश्यत्वप्रसङ्गाच्च मिथ्याज्ञानरूपसम्बन्धस्य समवायादिवत् पारमार्थिकत्वादिति । न द्वितीयः, ईश्वरस्याज्ञानाभावेनोपाधिसम्बन्धाभावप्रसक्त्या तस्य विश्वनिर्मातृत्वानुपपतिः। ननु प्रपञ्चदर्शनानुपपत्त्येश्वरस्याप्यगत्या बाधितार्थाभिमान आहार्यज्ञानरूपः कल्पनीय इति चेन्न, चेतने प्रतिबिम्बितस्यैव विषयभानहेतुताया वक्ष्यमाणत्वान्न त्वज्ञानस्य । अन्यथेश्वरस्य क्लेशकर्मादिशून्यताप्रतिपादकश्रुतिस्मृतिविरोधापत्तेः । नन्वेवमसत्प्रपञ्चाकारा वृत्तिरेवेश्वरोपाधावज्ञानं स्यादेवेति चेन्न, प्रपञ्चस्याऽत्यन्ततुच्छताया निरकरिष्यमाणत्वात् । तस्मात् किमर्थमभिमानस्य सम्बन्धत्वमिति न विद्मः ।}

\dev{अत आत्मानात्मनोः परस्परप्रतिबिम्बो ज्ञाननियामकः सम्बन्धः । बुद्धेः प्रवृत्तिहेतुस्तु पद्मपत्रजलयोरिव संयोगविशेषः सम्बन्धः,}
\begin{verse}
\dev{यथाकाशस्थितो नित्यं वायुः सर्वत्रगो महान् ।}\\
\dev{तथा सर्वाणि भूतानि मत्स्थानीत्युपधारय ।।}
\end{verse}
\dev{इत्यादिवाक्यर्लेपसङ्गादिनियामकस्य स्वाश्रयविकारहेतोः संयोगविशेषस्यैव प्रतिषेधावगमादिति आत्मानात्मनोरभेदबुद्धिस्तु भ्रम एवेति न तदर्थं सम्बन्धापेक्षेति ।}

Those who believe that the relationship of the ātman and the intellect is (in) the sense of agency, for them alone, is there a contradiction in understanding logically the knowledge of experience of knowers due to the cessation of knowledge accompanied by subtle impressions (jñānena savāsanājñānanivṛttyā viduṣām bhogānupapattiḥ). Even after knowledge has reached fruition, when one accepts a trace of avidyā left, that trace is not destroyed till the time of the fall of the body as there is no authority which states that when there is the destruction of the subtle impressions of karma then the karma that gave rise to the body is destroyed (dehārambhakakarmanāśasya vāsanānāśakatve pramāṇābhavāt).\footnote{The destruction of the subtle impressions of karma will ensure that one has realized the truth; but that does not entail that the body will also fall since the prārabdhakarma which gave rise to that particular body has to come to an end.} (In that case) even for those who have not realized the truth, reincarnation is not possible. When it is destroyed by itself, then even without realizing the truth, liberation can occur; when there is the relationship of destroyed and destroyer between ignorance and knowledge there can be a contingency regarding exceptions also (ajñānajyanayornāśyanāśakabhāve vyabhicāraprasaṅgācca). It is also difficult to describe the special quality of the delimitation of that which contains the effect (kāryatāvacchedakaviśeṣasya ca durvacat\-vāt)\footnote{Bhikṣu uses navya nyāya terminology in his works and which is evident here as well as in his other works as the Yogavārttikā for instance.}. 

Moreover is the relationship of the intellect and ātman of the nature of a superimposition imagined by the defining rule of special knowledge like “what is it that I am doing” etc., or is it defined by the activity of the intellect? It is not the first; as in the knowledge of the form “what is it that I am doing” etc., due to the rule of relationship when the object is for oneself (svaviṣayatve) there is the contradiction of the object and subject being the same; if given rise to by oneself it has the fallacy of being dependent on oneself; if one accepts more and more intermediaries then one is caught in the circle of mutual dependency and inconsistency of its being without any end (adhikādhikāṅgīkāre cānyonyāsrayacakrakānavasthādiprasaṅāt). Since there is also the contingency of the non-destruction of the special insightful knowledge as “I am the doer” (viśiṣṭabuddhivivekajñanā’nāsyatvaprasaṅgācca) the relationship in the form of an illusory knowledge is also existent just as the relation of inherence etc., (aham kartetyādi viśiṣṭabuddhivivekajñā’nāśyatvaprasṅgācca mithyājñānarūpasaṁbandhasya samvāyādivat pāramārthikatvāditi).

It cannot be the second; since there is absence of ignorance in Īśvara and due to the possibility of absence of relationship with a limitation there will be the incontingency of the creation of the universe (īśvarasyājñānābhāvenopādhisambandhābhāvaprasaktyā tasya viśvanirmātṛtvānupapattiḥ)\footnote{Bhikṣu believes that Īśvara is the efficient cause of the universe}

\textbf{Ques:} Since there is the contingency of the visibility of the universe so without any other explanation/resource available (for that) we have to imagine a contradictory meaning of agency in the form of a supporting knowledge even of Īśvara.

\textbf{Ans:} It is not so; since it will be mentioned that the cause for knowledge of the object is its being reflected in the consciousness and not of ignorance. Otherwise there will be a contradiction in the declaration of the śruti and smṛti of Īśvara being devoid of kleśa, karma etc\footnote{The reference is to YS I.24 “kleśakarmavipākāśayairaparāmṛṣṭah Īsaraḥ”.}.

\textbf{Ques:} In that case the modification in the form of the shape of the illusory universe is itself ignorance in the limitation of Īśvara (asatprapañcākārā vṛttireveśvaropādhānam syādeveti cet) then the answer is:

\textbf{Ans:} It is not so as the absolute illusory nature of the universe will be rejected. Therefore we do not understand the reason for a relationship of agency (mentioned earlier).

Herein the relationship is the mutual reflection (parasparapratibimbo) between the ātman (self) and anātman (not-self) which is the regulator of knowledge. The cause for the activity of the intellect is a special contact-relationship (saṁyogaviśeṣaḥ saṁbandhaḥ) like that between water and lotus leaf/leaves. In statements such as “yathākāśasthito…matsthānītyupadhāraya”(Gītā.9.6) one only learns the rejection of any special relationship which is the cause of change of that dependent on oneself which is the regulator of the contact of any stain (of ignorance) (lepasaṅgādiniyāmakasya svāśrayāvikārahetoḥ saṁyogaviśeṣasya pratiṣedhāvagamāt). The understanding of the relationship between the self and not-self as one of identity is a delusion only (bhrama eveti); thus there is no need for a relationship between them.

\dev{स्यादेतत् , ज्ञानेन शुक्तिरजतवत् प्रपञ्चस्य बाधे किंगोचरा किमर्था वा प्रवृत्तिः स्यात् ? न हि सुप्तोत्थितस्य स्वयमविषयगोचरा प्रवृत्तिर्दृश्यते । तस्मादविद्यावद्विषयाण्येव सर्वाणि प्रमाणानि शास्त्राणि प्रवृत्तिनिवृत्त्यादयश्चेति । अत्रोच्यते—ईश्वरस्य ज्ञानिनां च प्रवृत्तिभोगादिश्रवणादेव शुक्तिरजतादिविलक्षणव्यावहारिकसत्ता कार्याणामभ्युप\-गम्यते—}
\begin{verse}
\dev{“सद्भाव एषो भवते मयोक्तो, ज्ञानं यथा सत्यमसत्यमन्यत् ।}\\
\dev{एतच्च यत् संव्यवहारभूतं, तत्रापि चोक्तं भुवनाश्रितं ते  ॥”}
\end{verse}
\dev{इति विष्णुपुराणादिषु । सदसद्रूपत्वमेव च व्यावहारिकसत्त्वं प्रकृतितत्कार्यसाधारणम् । एकधर्मेण सत्तादशायामपि विकारिपदार्थानां धर्मान्तरेण सदैवासत्तानियमात् । कूटस्थनित्यस्य चात्मनो नास्ति धर्मतोऽप्यसत्त्वमिति स एव परमार्थसन्निति शास्त्रमर्यादा। तथा चोक्ताऽसत्तावधारणमेव बाधः, स च न प्रवृत्त्यादिविरोधी, तस्य विवेकवैराग्यमात्रहेतुत्वादिति ।}

\vskip .2cm

\dev{“वाचारम्भणं विकारो नामधेयं मृत्तिकेत्येव सत्यमि” त्यादिश्रुतिरपि दृष्टान्तमुखेनेदृशे एव सत्त्वासत्वे ब्रह्मविकारयोः प्रतिपादयति । तस्याश्चायमर्थः—यतः प्रलये च ब्रह्मणि सर्वं विकारजातमव्याकृतरूपामसत्तां गच्छति, अतो वाचां कार्यो नाममात्रशेषश्चेत्यसत्यो भङ्गुरो विकारः, कारणं ब्रह्मैव सत्यं नित्यमिति । अतीतानागतावस्थयोः कार्याणां नाममात्रेणावशेषता तु “नामैवैनं न जहाती” ति श्रवणात् । ननु  (न तु) नामधेयशब्दस्यात्यन्ततुच्छत्वमर्थः मृदविकारस्य  तुच्छता (अतुच्छतायाः) शतशः साधितत्वेन दृष्टान्तत्वासंभवात् पक्षसमत्वात् । “अपागादग्नेरग्नित्वम्” इति वाक्यशेषे विनाशमात्रश्रवणाच्चेति स्वयमूह्यम् ।}

\vskip .2cm

\textbf{Ques:} Let it be; when there is a refutation of the world through knowledge just as in the (false) knowledge of mother-of-pearl being silver, what is the object known and what is the purpose of the activity involved  (jñānena śuktirajatvat prapañcasya bādhe kimgocarā kimarthā va pravṛttiḥ syāt)? One does not witness activity in a person who has woken up from sleep towards a non-object of the senses (svayamaviṣayagocarā pravṛttirdṛśyate). Therefore all objects are of the nature of ignorance as also all the means of knowledge (sarvāṇi pramāṇāni śastrāṇi) as also activity and cessation from activity. 

\vskip .2cm

\textbf{Ans:} It is said: it is only by listening to the activity and experience (pravṛttibhogādiśravaṇāt) of Īśvara and the realized persons (īśvarasya jñāninām ca) that one accepts the reality of unusual worldly objects like the mother of pearl appearing as silver. Thus there are (supporting) statements in the Viṣṇu Purāṇa like: “sadbhāva eṣo bhavate mayo\-kto…tatrāpi coktam bhuvanāśritam te”. Worldly objects are of the nature of being both real and unreal (sadasadrūpatvameva ca vyāvahāri\-kasattvam) which is common to prakṛti and its effects. Even when objects which are subject to change exist with one characteristic there is always the rule of their non-being/non-existence in other qualities\break (ekadharmeṇa sattādaśāyāmapi vikāripadārthānām dharmāntareṇa\break saddaivāsattāniyamāt). As for the immutable, permanent ātman, it has no non-being due to any dharma as it has absolute being/existence according to the established śāstra-rule (sa eva parmārthasanniti śāstramaryādā)\footnote{The dharma here refes to the three times i.e. past, present and future. Thus the absolute is trikālābādhita}. Thus it only refutes the said understanding of non-being, and that is not in contradiction to its activity etc., as its cause is only discriminate discernment and detachment (tasya vivekavairāgyamātrahetutvāditi).

“vācāraṁbhaṇam vikāro nāmadheyam mṛttiketyeva satyam” (Chānd.\break Up. VI.1.4) such śruti statements also indicate through such examples the being and non-being of Brahman and the transformation. Its meaning is as follows: since during dissolution the entire collection of transformations (objects) attain the state of non-being of the nature of non-differentiation in Brahman, therefore the effects of śabdas (i.e. objects denoted by words) only remain in name and so (the resultant) change is non-real and perishable; the causal Brahman is alone real and permanent (kāraṇam brahmaiva satyam nityamiti). The stages of being past and not-as-yet-manifest of the effects (things) remains through names alone which is known through such sayings as “namaivainam na jahāti”. It is not that by the word nāmadheya the meaning understood is that it is totally useless; the change in clay being useless (or not useless) has been proved hundreds of times; since there is no example,  the logic is equally applicable to the other side as well (dṛṣṭāntāsaṁbhavāt pakṣasamatvam)\footnote{Whether one says that the changes are useless or useful (atucchatā) the reasoning is the same.}. One infers that it is so from hearing only the destruction from such statements as “apāgādagneragnitvam” (Chānd.Up. 6.4.1).  

\dev{“सत्ता सर्वपदार्थानां नान्या संवेदनादृते” “भूतं च सिद्धं च परेण यद्यत्तदेव तत्स्यादिति मे मनीषा” इत्यादिवाक्यानि चानुभवकारणाभ्यां विभागेनानुभूयमानकार्ययोरसत्त्वमेव बोधयन्ति न तु तयोरत्यन्तासत्त्वमेव, तथा सति ते न स्त इत्येवोच्येत न तु ताभ्यां सह तयोरभेद इति, सदसतोरभेदायेागात् । एवमेवान्या अप्येवंविधाः श्रुतिस्मृतयो व्याख्येया इति ।}

\dev{अपि च भवतु ज्ञानकर्मणोर्यथाकथञ्चिद् विरोधस्तथापि विविदिषूणां सर्वकर्मत्यागोऽनुचित एव । विचारेण सह कर्मणां विरोधस्य भवतामप्यनभ्युपगमात् , तदानीमभिमानस्य सत्त्वात् । प्रत्युतानुत्पन्नज्ञानस्य सत्यां शक्तौ देवताराधनरूपबाह्याभ्यन्तरसर्वकर्मत्यागे ज्ञानमेव नोत्पद्यत देवकृतविघ्नसम्भवात् , “तदेतद्देवानां न प्रियं यदेतन्मनुष्या विद्युरि” ति श्रुतेः ।}

\dev{ननु “विविदिषन्ति यज्ञेन दानेने” त्यादिवाक्याज्जिज्ञासाद्वारैव ज्ञाने कर्मणामुपयोग इति जिज्ञासानन्तरमेव तत्त्याज्यमिति चेन्न, तेन वाक्येन यज्ञादीनां ज्ञानसाधनत्वस्यैवावगमात् नत्विच्छासाधनत्वस्य, सर्वत्रैव नामधातुस्थले मूलधात्वर्थन सहैव कारकाणामन्वयव्युत्पत्तेः । अन्यथा रथेन जिगमिषतीत्यादौ रथादीनामिच्छा साधनत्वप्रतीतिप्रसङ्गात्, न चैवं संभवति ।}

\dev{न च,}
\begin{verse}
\dev{“कषायपक्तिः कर्माणि ज्ञानं च परमा गतिः }\\
\dev{कषाये कर्मभिः पक्वे ततो ज्ञानं प्रवर्तते ।।}\\
\dev{पापक्षयाच्छुद्धमतिर्वाञ्छति ज्ञानमुत्तमम् ।।”}
\end{verse}
\dev{इत्यादिवाक्येभ्यो वैराग्यजिज्ञासाद्वारैव कर्मणां ज्ञानाङ्गत्वमवसीयत इति वाच्यम्, तथाविधवाक्यानां वैराग्यजिज्ञासाद्वारपरत्वेऽपि द्वारान्तराप्रतिषेधकत्वात् ।}

The statements such as “sattā…saṁvedanādṛte” (Mahopaniṣad.Up.\break 5.47), “bhūtam…me manīṣā” (not traced), by separation of the experience and one’s experience and the causes (that give rise to those effects) only leads to the conclusion of their non-being but not to their total non-existence (kāryayorasattvameva bodhayanti na tu tayoratyantāsattvameva); if that were so then one would have to say that they are not present\footnote{This seems to refer to the denial of both being and non-being of objects} and not that they are identical with them both (i.e. being and non-being) as being and non-being being identical is unreasonable. It is in this way that other similar statements in śruti and smṛti need to be explained.

Moreover it is possible that there is opposition between knowledge and karma at times; even then it is improper for those who are seeking the truth (vividiṣūṇām) to give up all karma. You also do not accept that there is opposition of karma with knowledge as therein there is a sense of agency. However in one in whom knowledge has not as yet risen, when all external and internal karma  such as worshipping devas etc., is given up, knowledge itself will not arise due the possibility of obstruction caused by devas (superstition); thus there is the supporting  śruti statement “tadetaddevānām…vidyuḥ”.

\textbf{Ques:} If it is said that such statements as “vividiṣanti yajñena dānena” etc., indicate that the utility of karma in knowledge is only through the pathway of having the desire to know and after that desire (arises) it needs to be discarded\footnote{Once the mind is cleansed through karma (and bhakti) and the desire to know about Brahman arises karma can be discarded.} then the answer is:

\textbf{Ans:} That is not so; by that statement one only learns that sacrifices are the means for giving rise to knowledge (of Brahman) and not for the purpose of giving rise to the desire to know (Brahman). In all cases in place of nāmadhātu (a verb derived from a noun) it is customary to analyse the etymology/derivation with reference to the meaning of the main verb.\footnote{Thus it is not for the desire to know that sacrifices are done but to give rise to knowledge of Brahman.} Otherwise in such uses as ‘ by a chariot desires to go’ (to go desire) (literal meaning of ‘rathena jigamiṣati’) there will be the undesirable result of understanding its being  a means of the desire of the chariot; but it does not happen thus.

Through such statements like: “kaṣāyapaktiḥ…jñānamuttamam”\break (Mbh.Mokṣa.264 cited in Tri. P.13 fn.1) one does not conclude that only through generating detachment and desire to enquire (about Brahman) karma serves as a limb of jñāna (vairāgyajijñāsādvāraiva karmaṇām jñānāṅgatvamavasīyata iti vācyam). Even if such statements are indicative of their partiality towards detachment and desire to enquire (about Brahman) they do not reject other means (as well).
\begin{verse}
\dev{कर्मणा सहिताज्ज्ञानात् सम्यग्योगोऽभिजायते ।}\\
\dev{ज्ञानं च कर्म सहितं जायते दोषवर्जितम् ।।}
\end{verse}
\dev{इति  कौर्मादिवाक्यैर्जिज्ञासोत्तरं जातेऽपि ज्ञाने सम्यग्योगाख्यसम्प्रज्ञातसमाधौ ज्ञाननिर्दोषत्वे च ज्ञानाभ्याससहभावेन कर्मोपयोगश्रवणात् । दोषाः रागद्वेषमोहाः पापानि च । तथा,}
\begin{verse}
\dev{ज्ञानमुत्पद्यते पुंसः क्षयात् पापस्य कर्मणः ।}\\
\dev{यथादर्शतलप्रख्ये पश्यत्यात्मानमात्मनि ।।}
\end{verse}
\dev{इत्यादिवाक्यैर्ज्ञानप्रतिबन्धकपापक्षयद्वाराऽपि कर्मणां ज्ञानहेतुत्वं बोध्यते । तथा    “कर्माणीश्वरतुष्ट्यर्थं कुर्यान्नैष्कर्म्यमाप्नुयादि” ति वसिष्ठवाक्यान्नैष्कर्म्याख्यसमाधिहेतुत्वमीश्वरप्रीतिद्वाराऽपि कर्मणः सिद्धम्   ।}

\dev{एवं ज्ञानसमुच्चयवाक्येभ्यो मोक्षप्रतिबन्धकपापक्षयद्वारा मोक्षहेतुत्वमपि कर्मणामनुमेयम्,    अन्यथा“सहकारित्वेन च, अग्निहोत्रादि तु तत्कार्यायैव” इति वक्ष्यमाणसूत्रद्वयविरोधात् । तथा च जिज्ञासामात्रेण न कर्मत्यागः, जिज्ञासोत्तरमपि ज्ञाने योगे मोक्षे च प्रतिबन्धकसम्भवादिति “जिज्ञासुरपि योगस्य शब्दब्रह्मातिवर्तते” इति वाक्यं च परम्परया विधिकिंकरत्वाभावं फलं बोधयति “अनेकजन्मसंसिद्धस्ततो याति परां गतिमि” ति वाक्यशेषात् , ननु (न तु) योगजिज्ञासामात्रेण कर्मत्यागम् । “सर्वं कर्माखिलं पार्थ ज्ञाने परिसमाप्यत” इति च कर्मणां ज्ञानाङ्गत्वं बोधयति “अभ्यासेऽप्यसमर्थोऽसि मत्कर्मपरमो भवे” ति वाक्यं चाऽभ्यासाऽसमर्थस्य केवलं कर्म विदधाति परमशब्दादिति । तस्मात् सर्वकर्माणि संन्यस्य श्रवणं कुर्यादित्यपसिद्धान्तः कलिकृत एव—}
\begin{verse}
\dev{पुंसां जटाधरणमौढ्यवतां वृथैव}\\
\dev{मोघाशिनामखिलशौचबहिष्कृतानाम् ।}\\
\dev{पिण्डप्रदानपितृतोयविवर्जितानां ।}\\
\dev{सम्भाषणादपि नरा नरकं प्रयान्ति  ।।}
\end{verse}
\dev{इति विष्णुपुराणात् । वृथा समाधिरोगाद्यशक्तिं विनैवेत्यर्थः । तथा—}
\begin{verse}
\dev{दुःखमित्येव यत्कर्म कायक्लेशभयात् त्यजेत् ।}\\
\dev{स कृत्वा राजसं त्याग नैव त्यागफलं लभेत् ॥}
\end{verse}
\dev{इत्यादिस्मृतेश्चेत्यवधेयम् ।।}

\dev{यच्चान्यैरुक्तम्—विचारविधिपरमिदं सूत्रमिति, तदपि मन्दम्, श्रुतावेव “तद्विजिज्ञासस्व, आत्मा वा अरे श्रोतव्यः” इति विचारे निःसन्दिग्धविधिसत्वेन तत्र सूत्रवैफल्यात्, शास्त्रादावभिधेयप्रतिज्ञायामुन्मत्तप्रलापवदुपेक्षणीयतापत्तेः, सर्वशास्त्रदृष्टप्रतिज्ञापरत्वसम्भवे तत्त्यागानौचित्याच्च । अत एव पूर्वपक्षसूत्रवदस्य शिष्यप्रश्नसूत्रत्वमपि नोचितम्, अभ्यर्हिताया ग्रन्थारम्भप्रतिज्ञाया एव संभवादिति ।}

\dev{ननु यदि शमदमादिसंम्पन्नः सर्व कर्मसंन्यासी नात्राधिकारी तदाऽस्या ब्रह्ममीमांसाया कीदृशोऽधिकारीति वक्तव्यम् । उच्यते—शास्त्रार्थग्रहणोपयोगिशमादियुक्तो गुरुभक्तिनित्यकर्मतप आदिसम्पन्नो जिज्ञासुः कर्मफलविरक्त इति । “परीक्ष्य लोकान् कर्मचितान् ब्राह्मणो निर्वेदमायादि” त्याद्युक्तश्रुतेः,}
\begin{verse}
\dev{यस्य देवे परा भक्तिर्यथा देवे तथा गुरौ ।}\\
\dev{तस्यैते कथिता ह्यर्थाः प्रकाशन्ते महात्मनः ।।}
\end{verse}
\dev{इत्यादिश्रुतेश्च ।।}
\begin{verse}
\dev{इदं ते नातपस्काय नाभक्ताय कदाचन ।}\\
\dev{न चाशुश्रुषवे वाच्यं न च मां योऽभ्यसूयति ।।}
\end{verse}
\dev{इत्यादिस्मृतेश्चेति ।}

Even if such Kūrma Purāṇa statements like: “karmaṇā sahitājjñānāt…\-doṣavarjitām” (adh. 3.23 cited in ibid. p.13. fn.2) indicate the rise of knowledge after the desire to enquire (about Brahman) it is also learnt that karma is useful by being associated with the repeated practice of knowledge in removing the defects of knowledge in the state of saṁprajñāta samādhi known as samyagyoga.\footnote{Bhikṣu comes back to his favourite obsession which is yoga.} The defects are attachment, hatred and delusion which are vices (papāni). Thus according to sayings like “jñanamutpadyate…ātmani” (Mokṣa.204.8 cited in ibid. p.13. fn.3) it is learnt that karma is a cause for knowledge through weakening the vices which are obstacles to the rise of knowledge. Thus through the utterance of Vasiṣṭha “karmāṇīśvaratuṣṭyartham kuryānnaiṣkaryamāpnuyāt” it can be shown that karma is a means as being the cause for samādhi known as total detachment from karma through pleasing Īśvara.

In this manner through sentences combining jñāna and karma one can infer that karma is also a cause for the rise of mokṣa through weakening the vices caused by the obstacles to mokṣa. Otherwise it will contradict the two sūtras “sahakāritvena ca (BS.3.4.33); and agnihotrādi…tatkāryāyaiva” (ibid.4.1.16) which will be mentioned. Thus one should not give up karma only because of the desire to enquire; even after having the desire to enquire there can be obstacles in knowledge, yoga and mokṣa. The sentence “jijñāsurapi…ativartate” (Gītā. 6.44) informs us of the absence of dependence on vidhi (injunction) for the result to follow (paramparayā vidhikimkaratvābhāvam phalam bodhayati) through the next sentence “anekajanmasaṁsiddha…gatim (ibid.6.45). However one cannot give up karma just by the desire to enquire about yoga.\footnote{Since the context in the Gītā is on yoga and it is also Bhikṣu’s attachment to yoga that is reflected in this sentence.} The statement “sarvam…parisamāpyate” (ibid. 4.33) also indicate that karma is a limb of jñāna. So also the statement “abhyāse…matkarmaparamo bhava” (ibid. 12.10) indicates that for one who is incapable of continuous meditation on God only karma is prescribed by the use of the word “parama” (in the above verse). Therefore the statement that giving up all karma one should just listen to śruti (Upaniṣads) is a wrong conclusion made in the Kaliyuga (tasmāt sarvakarmāṇi sanyasya śravaṇam kuryādityapasiddhāntaḥ kalikṛta eva)\footnote{Referenc to Advaita Vedānta as a false doctrine.}. 

The word “vṛthā” in the following statement from the Viṣ.P: “pumsām jaṭādhāraṇamauḍhyavatām vṛthaiva…narā narakam prayānti” (3.28.103 cited in Tri. p.14.fn.3) (refers to those who give up karma) excluding those incapable because of being in samādhi or being sick.  So also attention must be paid to the following smṛti statements : “yastu vidyābhimānena…tyāgaphalam labhet” (not traced) which say the same thing.

\dev{(यच्चन्यैरुक्रम्) शास्त्रोक्तविद्याभ्यासे पुनरधिकारी योगशास्त्रोक्तयोगाङ्गादिसम्पन्नः श्रवणमननाभ्यां कोमलकण्टकन्यायेनोत्पन्नज्ञानो नारदीयोक्तनित्यानित्यविवेकादिसाधनचतुष्कवान् “शान्तो दान्त उपरत” इत्याद्युक्तश्रुतेः । तत्र चोपरतिर्योगविरोधिकर्मभ्य उपरम इति । तत्रापि मन्दाधिकारी गृहस्थादिस्रिदण्डिपर्यन्तः । उत्तमाधिकारी च परमहंसः,}
\begin{verse}
\dev{चतुर्विधा भिक्षवः स्युः कुटीचकबहूदकौ ।}\\
\dev{हंसः परमहंसश्च श्रेयांश्चैषां यथोत्तरम्।}\\
\dev{आत्मनिष्ठः स्वसंसक्तः त्यक्तसर्वपरिग्रहः ।}\\
\dev{चतुर्थोऽयं महानेषां ध्यानभिक्षुरुदाहृतः ।।}
\end{verse}
\dev{इति विष्णुधर्मसंहितादिवाक्यात् । परमश्चासौ हंसश्चेति परमहंसः परमात्मा “यमाहुः परमहंसमि” ति स्मृतेः । तन्निष्ठत्वाद् यतिरपि परमहंस उच्यते । हंसशब्दश्चात्मवाची—}
\begin{verse}
\dev{सकारेण बहिर्याति हकारेण विशेत् पुनः ।}\\
\dev{हंस हंसेति वै मन्त्रं जीवो जपति सर्वदा ॥}
\end{verse}
\dev{इतिस्मरणात् । हंसाश्रमस्तु केवलजीवोपयोगीति विवेकः । ते त्वाद्यास्त्रिदण्डिविशेषास्तेषाँ लिङ्गानि धर्माश्च तत्रैव विष्णुधर्मसंहितायामुक्ता विस्तरभयान्न लिख्यन्ते ।}

The statement by others that this sūtra (reference to BS. I.1.1) refers to the injunction of reflection  (vicāra) has no weight (mandam). In śruti itself there are sayings like “tadvijiñāsasva” (Taitt.Up.III.1.1) “atmā vā are śrotavyaḥ” (Bṛh. Up. IV.5.6) which prescribe reflection clearly (niḥsandigdhavidhisattvena) so the sūtra (BS. I.1) is useless for that purpose. In śāstra texts if there is no deliberation on what has been declared as the topic, there is the danger of its being ignored like the ramblings of a madman; when it is possible to be engaged in what is declared (as the topic) as seen by all the śāstras, it is also improper to give that up (sarvaśātradṛṣṭapratijñāparatvasambhave tattyāgānaucityācca). That is why to consider, like a pūrvapakṣasūtra, this sūtra as being the question of a disciple is also not correct;\footnote{Bhikṣu is perhaps referring to the four requisites laid down in Advaita in the BSBh as a question raised by a student for the competence of a disciple to pursue Advaita.} the competence (of a disciple) is decided by the declared suitable topic at the beginning of the grantha (book) itself (abhyarhitāyā granthārambhapratijñāyā eva sambhavāditi).

\textbf{Ques:} If one who is proficient in śama, dama etc., and who has given up all karma is not an adhikārī here (nātrādhikārī) then please tell me who is a fit adhikārī for this Brahmamīmāṁsā (knowledge pertaining to Brahman).\footnote{This seems a counter question by the Advaitin.}

\textbf{Ans:} One who is proficient in śama, dama etc., which enables one to grasp the meaning of the śāstras, who has accomplished successfully gurubhakti (devotion towards the guru), obligatory karma (nityakarma), austerity (tapaḥ) etc., one who has a deep desire to know (and) who is detached from the result of the action (karma).\footnote{Such a one is the competent adhikārī needs to be added to complete the answer.} This is in accordance with such śruti sayings like “parīkṣya…nirvedamayāt” (Muṇḍ.Up. I.2.12) and “yasya deve parā…prakāśante mahātmanaḥ” (Śvet. Up. 6.23) and the smṛti statement “idam te nātapaskāya…mām yo’bhyasūyati” (Gītā.18.67). 

Again with reference to learning in accordance with the instruction of the śāstras the adhikārī is one who is proficient in the yogāṅgas (limbs of yoga) mentioned in the YS;\footnote{Bhikṣu inserts his pet preference for yoga whenever he gets an opportunity.} through listening (to the śāstras) and reflection (on them) when knowledge arises through reasoning similar to the komalakaṇḍakanyāya\footnote{Using a soft thorn to remove a painful thorn; in other words it is through the listening and reflection on the śāstras that knowledge arises from the same śāstras.} then he is one who has achieved the fourfold requisites of distinguishing between what is permanent and that which is temporal etc., (nityānityaviveka) mentioned by Nārada; he is one who is “śānto dānta uparata” etc., according to śruti.\footnote{The complete quotation “śānto dānta uparatastitikṣuḥ... samāhito bhūtvā’’tmanyevā’’tmānam paśyati” Subālopaniṣad 9.14}  Therein the word “uparati” means desisting from deeds that are against the dictates of yoga\footnote{This is a special meaning given by Bhikṣu because of his attachment to yoga. uparati in general means abstaining from prescribed deeds.}; even there (there are the) dull aspirants (mandādhikārī) such as householders and including those carrying the three daṇḍa whereas the best aspirants are the paramahaṁsas.\footnote{Bhikṣu must have been a ‘ekadaṇḍin’ i.e. carrying a single stick as all advaitins. He also shows contempt for the other sannyāsins who carry three sticks.} Thus the Viṣṇu Dharmasaṁhitā says “caturvidhā bhikṣavaḥ…dhyānabhikṣurudāhṛtaḥ”.  A pramahaṁsa is the great ātman and its etymology is ‘paramaścāsau hamsaśca iti paramahamsaḥ’ i.e. paramātmā in accordance with the smṛti statement “yamāhuḥ paramahaṁsam”. Because an ‘yati’ is also established in that state he is also called a ‘paramahaṁsa’. The word ‘haṁsa’ denotes ātman as one recalls the following: “sakāreṇa bahiryāti…jīvo japati sarvadā”. One understands that the āśrama of a haṁsa\footnote{There is no hamsāśrama as such but Bhikṣau probably calls living as a paramahaṁsa in the world as a haṁsāśrama.} is only for the sake of leading one’s life (haṁsāsramastu kevalajīvopayogīti vivekaḥ). The characteristics and the duties (dharmās) of the first three who carry three daṇḍas (the kuṭīcaka, bahūdaka and the haṁsa) mentioned in the above verse) are mentioned in the Viṣṇu Dharmasaṁhitā; it is not given (here) out of fear of increasing   the length of the work (vistarabhayānna likhyante). 

\dev{तत्र विविदिषुसंन्यास आद्ययोरेव तयोस्तपःप्रधानत्वात् “तपसा ब्रह्म विजिज्ञासस्वे” ति श्रुतेः । मनौ—}
\begin{verse}
\dev{वेदसंन्यासिकानान्तु कर्मयोगं निबोधत ।}\\
\dev{संन्यस्य सर्वकर्माणि कर्मदोषानपानुदत् ।}\\
\dev{नियतो वेदमभ्यस्य पुत्रैश्वर्ये सुखं वसेत् ।।}
\end{verse}
\dev{इत्यादीनां वेदसंन्यासस्यापि स्मरणाच्च । विद्वत्संन्यासस्त्वन्त्ययोः जीवपरमात्मनिष्ठताभेदेनेत्यपि विवेक्तव्यम् ।}
\begin{verse}
\dev{भौतिकी भावना पूर्वे सांख्ये त्वक्षरभावना ।}\\
\dev{तृतीये चान्तिमा प्रोक्ता भावना पारमेश्वरी ।।}
\end{verse}
\dev{इत्यादिनाम्  कौर्मे त्रिविधयोगिकथनात् । वैश्वानरादिभावना भौतिकी, सा चाद्ययोः संन्यासिनोरिति तत्र परमहंसा जाबालश्रुतौ परिगणिताः ‘‘तत्र परमहंसा नाम संवर्तकारुणिश्वेतकेतुदुर्वासऋभुनिदाढ्यजडभरतदत्तात्रेयरैवतकप्रभृतय”इति । ननु भरतस्य यज्ञोपवीतश्रवणादस्य कथं परमहंसत्वम् ?}
\begin{verse}
\dev{त्रिदण्डं कुण्डिकां चैव सूत्रं चापि कपालिकाम् ।}\\
\dev{जन्तूनां वारणं वस्त्रं सर्व भिक्षुरिदं त्यजेत् ।।}
\end{verse}
\dev{इति परमहंसप्रकरणस्थविष्णुधर्मवाक्यादिति । [न]}
\begin{verse}
\dev{आत्मन्येवात्मना बुद्ध्या न्यस्तसर्वपरिग्रहः ।}\\
\dev{अव्यक्तलिङ्गोऽव्यक्तश्च चरेद् भिक्षुः समाहितः ।।}
\end{verse}
\dev{इति विष्णुधर्मवाक्योक्ताव्यक्तलिङ्गतासम्पादनाय मनसा यज्ञोपवीतादीनामात्मन्यारोपणेऽपि सूत्राभासधारणसंभवादिति । तस्मात् सूत्रस्य विपरीतार्थकल्पनया सर्वकर्मत्यागे शिष्यो न प्रवर्तनीयः। किन्तु मुमुक्षवे ज्ञानमुपदिश्य कर्मार्थं समाधिभङ्गो न कर्तव्यो “गुणलोपो न गुणिन” इति न्यायात्, जडभरतादिशिष्टाचाराच्चेत्येवोपदेश्यम् । समाध्याविर्भावे च कर्मत्यागस्तद्विरोधेन स्वयमेव क्रमेण भवति “एतद्ध स्म वै तद्विद्वांस आहुः ऋषयः कावषेयाः किमर्था वयमध्येष्यामहे किमर्था वयं यक्ष्यामह” इति श्रुतेः,}
\begin{verse}
\dev{न कर्माणि त्यजेद्योगी कर्मभिस्त्यज्यते ह्यसौ ।}\\
\dev{विदिते परतत्त्वे तु समस्तैनियमैरलम् ।।}\\
\dev{तालवृन्तेन किं कार्यं लब्धे मलयमारुते ।}\\
\dev{ज्ञानामृतरसो येन सकृदास्वादितो भवेत् ।।}\\
\dev{स सर्वकार्यमुत्सृज्य तत्रैव परिधावति । इत्यादि स्मृतेश्चेति ।}
\end{verse}
In that context the samnyāsin who has the desire to enquire (into the truth) is included in the first two (kuṭīcaka and bahūdaka) as for them there is the importance of austerity according to the śruti statement “tapasābrahma vijijñāsasva” (Taitt.Up. 3.2,4,5). One also hears about Vedasaṁnyāsa from the MS as “vedasamnyāsikanāntu…putraiśvarye sukham vaset” (MS.6.86)\footnote{Nowadays one does not hear about Vedasaṁnyāsa.}. Whereas a vidvatsaṁnyāsin needs to be distinguished from the last two (presumably the hamsa and paramahamsa) as they are intent on (realizing) jīva as ātman. The Kūrma Purāṇa mentions three kinds of yogīs as:  “bhautikī bhāvanā…bhāvanā pārameśvarī” (2.86; cited in Tripathi p.15.fn. 3). Meditation on Vaiśvānara etc., is of a gross nature (bhautikī); and that belongs to the first two samnyāsins (kuṭīcaka and bahūdaka). Paramahaṁsas are recoun\-ted in the Jābāla Up as “tatra paramahamsā…raivatakaprabhṛtaya”\break (Jābāla. 6).

\textbf{Ques:} Since one hears of Bharata’s thread ceremony (yajñopavīta) how can he be a paramhaṁsa (this is obviously not the Rāmāyaṇa Bharata)\footnote{Bharata is not mentioned as a paramahaṁsa in the above quote from the Jābāla Up. But this must be well known for Bhikṣu to raise the issue here.}. That is not in accordance with the Viṣṇudharma statement:  “tridaṇ\-ḍam…bhikṣuridam tyajet”. In order (for a paramhamsa) to qualify by having no clear marks/symbols according to the Viṣṇudharma  statement: “ātmanyevātmanā…avyatkaliṅga…samāhitaḥ” even if one internalizes mentally yajñopavīta etc., it is possible that there will be the semblance of the thread/yajñopavīta.

Therefore by imagining an opposite meaning of the sūtra (sūtrasya) the disciple should not engage in giving up all karma. But instructing about knowledge to one desiring mokṣa, there should be no breaking with samādhi for the sake of performance of karma (karmārtham samādhibhaṅgo na kartavyo); this follows the maxim of “guṇalopo na guṇina” (instruction should also emphasize the) excellent (spiritual) observances of ācāryas like Jaḍabharata etc. When samādhi comes into being then there is the giving up of karma gradually on its own accord as it is contradictory to it (samādhi); thus there is the saying “etaddha sma vai…vayam yakṣyāmahe”. Smṛti also says: “na karmāni tyajedyogī…tatraiva paridhāvati.”

\dev{यच्चान्यत् परैरुच्यते—अस्य शास्त्रस्य जीवब्रह्मैक्यं विषयः, तज्ज्ञानस्य च न कर्मशेषत्वमिति तदपि न, अस्य शास्त्रस्य जीवब्रह्मैक्यविषयत्वे लिङ्गाद्यभावात्: “ब्रह्मसूत्रपदैश्चैव हेतुमद्भिर्विनिश्चितै” रिति गीतावाक्येन सूत्राणां ब्रह्मविषयतामात्रावगमात्, तथा सत्य “थातो जीवनह्मैक्यजिज्ञासे” त्येव सूत्रणौचित्याच्च । शास्त्रमहावाक्यार्थ परित्यज्य तदेकदेशप्रतिज्ञानौचित्यात् । ब्रह्मसूत्ररूपैरधिकरणैर्यैर्युक्तिमद्भिरसंदिग्धैर्गीतमित्यर्थः । जीवनिरूपणं चात्र ब्रह्मशेषतयैव प्राणादिनिरूपणवदिति जीवप्रकरणे वक्ष्यामः । यद्यपि ब्रह्मात्मतैवान्न शास्त्रमहावाक्यार्थः, तथापि ब्रह्मत्वेनैवात्मत्वमाक्षिप्तमित्याशयः । “बृहत्त्वाद् बृंहणत्वाच्च आत्मा ब्रह्मेति गीयत” इति स्मृत्यादिभिरात्मब्रह्मशब्दयोरर्थैक्यं वा । अतोऽस्मन्मते सूत्रन्यूनता न शङ्कनीयेति । यश्चास्य विषयो ब्रह्म तज्ज्ञानस्य कर्मशेषत्वमप्यस्येव,}

\eject

\begin{verse}
\dev{ब्रह्मण्याधाय कर्माणि सङ्गं त्यक्त्वा करोति यः ।}\\
\dev{लिप्यते न स पापेन पद्मपत्रमिवाम्भसा ।।}
\end{verse}
\dev{इत्यादिवाक्येभ्य इति दिक् ।}

\dev{सपरिकरं ब्रह्म विचार्यमित्युद्दिष्टं, तत्र ब्रह्मलक्षणं ब्रह्मशब्दप्रवृत्तिनिमित्तञ्च प्रकृतिपुरुषादिव्यावृत्तमाह—}

\textbf{Ques:} There is this said by others: the subject matter of this śāstra is the identity of jīva and Brahman (and) for realizing it there is no need of karma, as there is no characteristic mark with reference to the identity of jīva and Brahman in this śāstra; by the Gītā statement “brahmasūtrapadaiścaiva…viniścitaiḥ” (Gītā 13.4)  one knows that its subject is only pertaining to Brahman.

\textbf{Ans:} In that context then it would be appropriate for the sūtra to be “athāto jīvabrahmaikyajijñāsā”.\footnote{Bhikṣu states if the identity of Brahman and jīva was the intention of the BS then the first sūtra should have been “athāto jīvabrahmaikyajijñāsā” and not “athāto brahmajijñāsā”.} Abandoning the meaning of the mahavākyas of the śāstras it is inappropriate to understand its meaning partially (in part).\footnote{The reference is perhaps to the mahāvākya “tat tvam asi” where one has to understand the meaning by giving up the identity not wholly but partially by substraction and non-substraction of meaning called as “jahajjajjahajlakṣanā”.} It is thus spelt out clearly in the adhikaraṇas of the Brahmasūtras. We shall also mention later in the chapter dealing with jīva that there is the definition of the jīva as being a part of Brahman just as the definition of prāṇa etc.

Even though the meaning of the śāstra-mahāvākya is the identity of Brahman and ātman, still the intention pointed out is that having the  characteristic of ātman (ātmatvam)is through having the characteristic of Brahman (brahmatvenaiva ātmatvamākṣiptam ityāśayaḥ).  Or this could be interpreted as the identity of the meaning of the words Brahman and ātman as mentioned in smṛti texts as “bṛhattvād bṛhaṇat\-vācca ātmā brahmeti gīyate”. Therefore in our understanding one need not doubt that the sūtra has omitted anything. Thus the subject matter of the sūtra is Brahman and its knowledge is the result of karma/karma\-yoga (tajjñānasya karmaśeṣatvamapyastyeva). This is understood from such statements as “brahmaṇyādhāya karmāni…pāpena padmapatra\-mivāmbhasā”(Gītā 5.10). The intention (of the sūtra) was to examine Brahman in all details (saparikaram brahma); in that context the characteristics of Brahman, the activity and purpose of the use of the word Brahman (which) differentiates it from prakṛti and puruṣa is mentioned (in the sūtra):

\section*{BS. I.1.2}

\section*{\dev{जन्माद्यस्य यतः ॥} I.\dev{२ ॥}}

\dev{अस्य जगतो नामरूपाभ्यां व्याकृतस्य चेतनाचेतनरूपस्य प्रतिनियतदेशकाल- संस्थाव्यापारादिमतोऽचिन्त्यरचनात्मकस्य जायतेऽस्ति वर्धते विपरिणमतेऽपक्षीयते विनश्यतीत्येवंरूपं जन्मादिषट्कं यतः परमेश्वरादन्तर्लीनप्रकृतिपुरुषाद्यखिलशक्तिकात् स्वतश्चिन्मात्राद् विशुद्धसत्त्वाख्यमायोपाधिकात् क्लेशकर्मविपाकाशयैरपरामृष्टाच्चेतनविशेषाद् भवति, आकाशादिव महावायुर्महाजलादिव च पृथिवी, प्रथिव्या इव च स्थावरजङ्गमादिकं, तद् ब्रह्मेति वाक्यशेषः । अत्र च ‘एतद्यत’ इत्यनुक्त्वा ‘जन्माद्यस्य यत’ इति वचनादव्यक्तरूपेण जगन्नित्यमेवेत्याचार्याशयोऽवगन्तव्यः । यत इति पञ्चमी चात्राधिष्ठानकारणत्वे महदाद्यखिलजगदधिष्ठानकारणत्वं च ब्रह्मण एव “आधारमानन्दमखण्डबोधं यस्मिन् लयं याति पुरत्रयं च, एतस्माज्जायते प्राणो मनः सर्वेन्द्रियाणि चे” त्यादिश्रुतेरिति न, (न) प्रकृतिपुरुषादिष्वतिव्याप्तेः (तिव्याप्तिः)।}

This world having various names and forms, which has both sentient and insentient entities, which has its own rules of space, time, place and action which is constructed in a manner which cannot be fathomed (acintyaracanātmakasya), has the nature of coming into existence, then exists, grows, changes and then declines and is (finally) destroyed; this world having the sixfold nature beginning with its birth comes into being by itself (svataḥ) from the powers of all the puruṣas and prakṛti hidden within Parameśvara through the special consciousness which is untained by kleśa, karma and vipāka (and) which has the limitation called māyā composed of pure sattva. The sentence needs to be completed by adding ‘just like the great wind from ākāśa, just like earth from the great waters, just like the inanimate and animate entities from the earth’ Brahman is that. By not mentioning “etadyata” (from that) through the words “janmādyasya yataḥ” one should understand that the ācārya’s (Bādarayaṇa) intention is clearly that the world is eternal.\footnote{This flows logically from the metaphysics of Sāṇkhya-Yoga.}  The word “yataḥ” in the fifth case (pañcamī) is used in the sense of being the causal support (adhiṣṭhānakāraṇa). It is Brahman alone who is the causal support of the whole world composed of mahat etc. Thus the śruti statement “ādhāram…sarvendriyāṇi ca” supports it; it indicates that there is no over-pervasion (ativyāptiḥ) with regard to prakṛti and puruṣa.\footnote{According to Bhikṣu there should not be any doubt regarding who is the final cause. But when both prakṛti and the puruṣas are declared to be vibhu and permanent it does raise some problems.} 

\vskip .2cm

\dev{`किं पुनरधिष्ठानकारणत्वम् ? उच्यते—तदेवाधिष्ठानकारणं यत्राविभक्तं येनोपष्टब्धं च सदुपादानकारणं कार्याकारेण परिणमते । यथा सर्गादौ जलाविभक्ताः पार्थिवसूक्ष्मांशास्तन्मात्राख्याः जलेनैवोपष्टम्भात् पृथिव्याकारेण परिणमन्त इत्यतो जलं महापृथिव्या अधिष्ठानकारणमिति । तथा च स्मर्यते—}
\begin{verse}
\dev{यस्य यत् कारणं प्रोक्तं तस्य साक्षान्महेश्वरः।}\\
\dev{अधिष्ठानतया स्थित्वा सदैवोपकरोति हि ।। इति ।}
\end{verse}
\dev{तथा चैतादृशकारणत्वमेवाधिष्ठानकारणत्वमिति मूलकारणत्वमिति चोच्यते । ब्रह्मणश्च स्वाविभक्तप्रकृत्याद्युपष्टम्भकत्वं साक्षितामात्रेणेति जगत्कारणत्वेऽपि न ब्रह्मणो विकारित्वं न वा प्रकृतिपुरुषादिष्वतिप्रसङ्गः, सर्गात् पूर्वमन्येषां साक्षित्वासम्भवात् । अत एवाविकारचिन्मात्रत्वेऽपि ब्रह्मणो जगदुपादानत्वं जगदभेदश्चोपपद्यते । विकारिकारणवदधिष्ठानकारणस्याप्युपादानत्वव्यवहारात् ।   कार्याविभागाधारत्वस्यैवोपादानसामान्यलक्षणत्वात् । अविभागश्चाधारतावत् स्वरूपसम्बन्धविशेषोऽत्यन्तसंमिश्रणरूपो दुग्धजलाद्येकताप्रत्ययनियामकः । }

\vskip .2cm

\dev{तत्र समवायसम्बन्धेन यत्राविभागस्तद्विकारिकारणम्। यत्र च कार्यस्य कारणाविभागेनाविभागस्तदधिष्ठानकारणम्, यथा जलं पृथिव्या इति । न हि जलस्य साक्षादेव पृथिवी विकारस्तन्मात्राणां भूतप्रकृतित्वश्रुतिस्मृतिविरोधात् । न च द्वयोरेवोपादानत्वम्, विजातीयानामनारम्भकत्वात् । एवमाकाशादीनां वाय्वाद्युपादानत्वमप्यधिष्ठानतयैव द्रष्टव्यम्। संभवत्यविरोधे सृष्टिप्रक्रियायां वैशेषिकसांख्ययोरुभयोरप्यत्र विरोधानौचित्यादिति वैशेषिकादिभिरपीदृशं। ब्रह्मणः कारणत्वमिष्यत एव । परन्तु तैरिदमपि निमित्तकारणतेति परिभाष्यते । अस्माभिस्तु समवाय्यसमवायिभ्यामुदासीनं निमित्तकारणेभ्यश्च विलक्षणतया चतुर्थमाधारकारणत्वमिति । तदेतत् सर्वं “तत्तु समन्वयादि” ति सूत्रेणाऽऽचार्यो वदिष्यति, शिष्यव्युत्पत्त्यर्थं त्वत्राप्यस्माभिः किञ्चिदुक्तमिति। इमं चार्थं तत्रैव प्रपञ्चयिष्यामः।ब्रह्मणः साक्षात् परिणामवादं विवर्तवादं च तत्रैव निराकरिष्यामः।}

\vskip .2cm

In answer to the question as to what is a causal support he says: That is itself called a resting place wherein without separation and being closely connected with it, the material cause transforms itself into the form of effects. This is like the subtle elements of earth called as tanmātras are not separated from water at the start of evolution/manifestation, and staying within water change into the shape of earth and so water is (called) the causal support of the great earth. Thus one recalls “yasya yat kāraṇam…sadaivopakaroti hi” (not traced). Similarly the causal support is this kind of causal support and it is also called the principal support. And Brahman sees prakṛti etc., which stays inseparably within itself just as a witness; even in being the cause for the world there is no change in Brahman nor does it extend (have an unwarrantable stretch) to prakṛti and puruṣas (na vā prakṛtipuruṣādiṣvatiprasaṅgaḥ)\footnote{In other words all three Brahman, prakṛti and puruṣas who are vibhu stay without any change in close proximity is what Bhikṣu says. That seems to be a stretch of the imagination when all three are all pervasive.}. Prior to evolution there is no witnessing of anything. That is why even if Brahman of the form of pure consciousness has no change it is appropriate that Brahman is the material cause of the world as well as not being different from the world (jagadupādānatvam jagadabhedaścopapadyate). Just as the transforming cause (vikārikāraṇavat) the causal support can also possess the function of a material cause.  

The general characteristic of a material cause is that it is a non-sepa\-rable support of the effect (kāryāvibhāgādhāratvasyaivopādanasāmān\-yalakṣaṇatvāt). And non-separation, similar to that of a support, is a special relationship called svarūpa-sambandha; it is of a very mixed nature; it regulates the knowledge of identity (like that) between milk and water.\footnote{This makes no sense; when water and milk are mixed together they become one and they are not seen separately. But in the case of Brahman, prakṛti and the puruṣas, Bhikṣu has already stated that there is no close contact between them.} Therein when there is non-separation by the relationship of inherence that becomes the cause for change. Where the effect is non-separate due to being non-separate from the cause then it is a causal support (adhiṣṭhānakāraṇa) like water in the case of the earth.\footnote{One can see that Bhikṣu is not happy with the svarūpa sambhandha and Brahman being an adhiṣṭhāna kāraṇa he has advocated; he therefore tries to explain it in various ways.} The earth is not a direct transformation of water as it will contradict śruti and smṛti (statements which mention) the subtle elements of (water) being of the nature of earth. Nor can the two be the material cause as there cannot be the rise of (objects )with different properties (vijātīyānāmanārambhakatvāt). Thus one should also view ākāśa etc., being the material cause of wind etc., in the sense of being a support (adhiṣṭhānatayaiva draṣṭavyam). Since it is possible for both the Vaiśeṣika and Sāṁkhya philosophers to assent to this process of creation/evolution it is improper for them to oppose it.\footnote{Bhikṣu at heart is a syncretist and even though a staunch Yoga advocate would still like to reconcile his views with that of the other orthodox systems of philosophy. His awoved battle is mainlly with Śaṅkara and his advaita philosophy.} Even the Vaiśeṣikas desire this kind of cause as Brahman. But it is defined by them as an efficient cause. We are indifferent to causes such as samavāyī and asamavāyī and (also) being distinguished (different) from efficient causes as well (nimittakāraṇebhyaśca vilakṣaṇatayā)\break (and thus) this fourth cause is a causal support (caturthamādhārakāra\-ṇatvamiti)\footnote{There is not one singular characteristic that can be mentioned for an efficient cause.}. All this will be mentioned by the ācārya under the sūtra “tattu samanvayāt” (BS.I.1.4). We also have mentioned it briefly here for the sake of the learning of the disciple (śiṣyavyutpattyartham). We shall explain this meaning there itself. We shall also reject the pariṇāma theory (real transformation) and vivartavāda theory (transformation as an appearance) of Brahman there itself. 

\dev{ब्रह्मणश्च जगत्कर्तृत्वं स्वोपाधिमायोपाधिकम्, परिणामित्वरूपोपादानत्वं च प्रकृतितत्कार्याद्यौपाधिकमपीष्यत एव । तथा चोक्तम्—}
\begin{verse}
\dev{सर्वशक्तिमयो ह्यात्मा शक्तिमण्डलताण्डवैः ।}\\
\dev{सम्सारं तन्निवृत्तिं च करोत्यविरतोदयम् ।। इति }
\end{verse}
\dev{“यस्मिन् यतो यर्हि येन यस्य यस्मै यद् यो यथा कुरुते कार्यते वा । परावरेषां परमं प्राक् स्वसिद्धं तद् ब्रह्म   तद्धेतुरनन्यदेकम्” इति च  अस्मिश्च कारणताद्वये कुलालोर्णनाभौ दृष्टान्ताविति । एवञ्च जगतः सर्वप्रकारकारणत्वमपि ब्रह्मलक्षणं कर्तुं शक्यते । प्रकृतिपुरुषादिषु शक्तिषु प्रत्येकमुपादानत्वादिरूपेण प्रतिनियतमेव कारणत्वम् । ब्रह्मणस्तु सर्वशक्तिकत्वात् तत्तदुपाधिभिः सर्वकारणत्वम् । यथा चक्षुरादीनां दर्शनादिकारणत्वं यत्प्रत्येकमस्ति तत्सर्वं सर्वाध्यक्षस्य जीवस्य भवतीति । एतेन जगतोऽभिन्ननिमित्तोपादानत्वं व्याख्यातम् । अस्मिंश्च शास्त्रे सृष्टिप्रक्रिया महदादिक्रमेणैव सांख्ययोगयोरिव वक्ष्यते वियदादिपादे सृष्टिप्रकरणे।}

The creation of the world by Brahman is dependent on its own limitation (called) māyā. Being the material cause in the form of having the quality of change is aso desired as belonging to the limitation of prakṛti and its effects. Thus it is said “sarvaśaktimayo hyātmā…karotyavirato\-dayam”. There is also the saying: “yasmin yato…tad brahma taddheturananyadekam”. In these two instances of having causal efficiency (asminśca kāraṇatādvaye) (mentioned in the quote “yasmin…brahma “) examples are that of the potter (kulāla) and the spider (ūrṇanābhi). In this manner even if the world has causes of all kinds (sarvaprakārakāra\-ṇatvamapi) it can be established as a distinctive sign of Brahman. Being a cause is fixed (pratiniyatameva) in the powers centred in prakṛti and puruṣas in the form of being a material cause (prakṛtipuruṣādiṣu śaktiṣu pratyekamupādāntvādirūpeṇa pratiniyatameva kāraṇatvam).\footnote{In Bhikṣu’s philosophy puruśas and prakṛti are śaktis of Parameśvara and help in the process of evolution of the world.} Brahman possessed of all powers (sarvaśaktikatvāt) is a cause for\break everything (sarvakāraṇatvam) through their respective limitations\break (tatadupādibhiḥ);  this is like the eye etc.,(i.e. all the senses are implied here) each possessing the quality of being causes for the act of seeing etc., respectively, (but) all those activities belong to the all-supervising jīva. In this manner Brahman being both a material and also an efficient cause has been explained.\footnote{As the causal support in the form of adhiśṭhāna-kāraṇa closely associated with the śaktis, prakṛṭi and puruṣas, Brahman is the material cause (upādānakāraṇa) of the world. So also Brahman with its upādhi of pure sattva is the efficient cause of the world. Thus Brahman is both the material and efficient cause of jagat.}  It will be mentioned later in the chapter on viyat etc., dealing with creation, that in this śāstra (Bhikṣu’s avibhāga-vedānta) the function of creation happens with the rise of mahat etc as in Sāṁkhya and Yoga\footnote{Bhikṣu sticks to the cosmology of SY.}. 

\dev{विशेषस्त्वत्रोच्यते—प्रकृतिस्वातन्त्र्यवादिभ्यां सांख्ययोगिभ्यां पुरुषार्थप्रयुक्ता प्रवृत्तिः   (प्रकृतिः) स्वयमेव पुरुषेण आद्यजीवेन संयुज्यत इत्यभ्युपगम्यते अयस्कान्तेन लोहवत् । अस्माभिस्तु प्रकृतिपुरुषसंयोग ईश्वरेण क्रियत इत्यभ्युपगम्यते, “आदिः स संयोगनिमित्तहेतुः परस्त्रिकालादकलोऽपि दृष्ट” “प्रकृतिं पुरुषं चैव प्रविश्यात्मेच्छया हरिः ।क्षोभयामास सम्प्राप्ते सर्गकाले व्ययाव्ययौ  ।। इति स्मृतेश्चेति ।}
\begin{verse}
\dev{पुरुषोऽत्र जीवः “चित्यात्मा गृह्यते यस्तु बुद्ध्यवस्थित आत्मनः ।}\\
\dev{पुरुषाख्यः स विज्ञेयो भोक्तृभावः स उच्यते ।।}
\end{verse}
\dev{इति योगियाज्ञवल्क्यात् । परमेश्वरे च पुरुषशब्दः उपाधिसम्बन्धमात्रेण गौणः ।}

\dev{ननु संयोगविशेषहेतुः क्रियाविशेषः क्षोभो विभ्वोः प्रकृतिपुरुषयोर्न संभवतीति चेन्न, प्रकृतेर्गुणत्रयरूपतया परिच्छिन्नगुणांशेन क्षोभसंभवात्, पुरुषस्य च तदौपाधिकक्षोभात् आकाशस्य वाय्वौपाधिकक्षोभवत् । अथवा संयोगोन्मुखत्वेन पुरुषे क्षोभोपचारः अत- एव “गुणेभ्यः क्षोभ्यमाणेभ्यस्त्रयो देवा विजज्ञिरे” इत्यादिश्रुतिषु गुणानामेव क्षोभः श्रूयत इति, न तु पुरुषस्येति । प्रकृतिपुरुषयोश्चेश्वरस्य प्रवेशः शास्त्रवदवधानमात्रमिति । विभावपि वेश्वरोपाधौ विभोरीश्वरस्य नित्य एव संयोग इत्यगत्याभ्युपेयम् ! नित्यसंयुक्तयोरपि वैधर्म्यात् जीवतदुपाधिदृष्टान्तेन भेदसिद्धिरिति ।}

Something special is mentioned here:

The Sāṁkhya and Yoga philosophers who believe in the independence of prakṛti believe that prakṛti used for the purpose of accomplishing the goal of puruṣa gets connected by itself with the first jīva, like iron getting connected to a magnet.  In our view, on the other hand, contact between prakṛti and puruṣa is brought about by Īśvara (in accordance with the śruti saying “ādiḥ sa samyoganimittahetuḥ parastrikālādakalo’pi dṛṣṭa” (Śvet. 6.5). Smṛti also mentions the following: “prakṛtim puruṣam caiva…sargakāle vyayāvyayau” (Kūrma.P 4.13; also Viṣ.P. 1.2.27 cited in Tripathi p.18.fn.3). The creation of the world by Brahman is dependent on (belongs to) its own limitation (called) māyā. Puruṣa here (in the above verse), is the same as jīva; this is understood from Yogi Yājñavalya’s statement: “cityātmā gṛhyate…bhokṛbhāvaḥ sa ucyate”. When puruṣa is used for Parameśvara it is in a secondary sense (and) only when it is associated with its limitation.

\textbf{Ques:} The cause for the special contact (between Hari, puruṣa and prakṛti after entering them mentioned in the verse above) is activity of a special nature i.e disturbance (kṣobhaḥ); that is not possible between the all pervading prakṛti and puruṣa.

\textbf{Ans:} That is not so. Since prakṛti is of the nature of three guṇas it is possible to have disturbance in the portion which is divided;\footnote{What Bhikṣu means by this is not clear at all. Prakṛti which is all-pervading is of the nature of the three constituents. How can there be a partial division of that?} (there is disturbance in puruṣa) due to the disturbance of the limitation of puruṣa just as there is the disturbance of ākāśa due to (the disturbance of) the limitation of wind (vāyvaupādhikakṣobhavat). Or due to having the intent of contact, there is this activity of disturbance in puruṣa.\footnote{Bhikṣu has no confidence in what he stated earlier and by this only weakens his stand. Contrast this style with the definiteness with which Śaṅkarācārya states his theories in the BSBh} That is why in such śruti sayings as “guṇebhyaḥ kṣobhymāṇebhyastrayo devāvijajñire” one hears only the disturbance of the guṇas and not that of puruṣa.  According to śāstra, the entry of Īśvara into prakṛti and puruṣa is only an intention (śāstravadavadhānamātramiti). Thus one necessarily understands (agatyābhyupeyam) that even though Īśvara and the limitation of Īśvara are (both) all-pervasive there is a permanent contact with the all-pervading Īśvara. Even when there is eternal contact between the two, through the example of the jīva and its upādhi, the difference is established of the permanent entities\footnote{Bhikṣu uses the contact of the conditional upādhi of the jīva with the jīva, to justify the contact with the nitya puruṣa and prakṛti with Īśvara. But one can see that it is a very weak argument.}. 

\dev{एतच्च जगज्जन्मादिकारणत्वं ब्रह्मशब्दप्रवृत्तिनिमित्तमपि बोध्यम् । मूलकारणस्यैव निरतिशयबृहत्त्वात् । ब्रह्मशब्दश्च पङ्कजादिवद् योगरूढः । अतो न जीवादिर्मुख्यो ब्रह्मशब्दार्थः । तथा चोक्तं नारसिंहे—}
\begin{verse}
\dev{आदिसर्गमहं तावत् कथयामि द्विजोत्तम।}\\
\dev{रहस्यैर्ज्ञायते येन परमात्मा सनातनः ।।}\\
\dev{प्राक्सृष्टेः प्रलयादूर्ध्वं नासीत् किञ्चद्द्विजोत्तम।}\\
\dev{ब्रह्मसंज्ञमभूदेकं ज्योतिर्यत् सर्वकारणम् ।।}\\
\dev{नित्यं निरञ्जनं शान्तं निर्गुणं नित्यनिर्मलम्।}\\
\dev{आनन्दसागरं स्वच्छं यत्काङ्क्षन्ति मुमुक्षवः ।।}
\end{verse}
\dev{सर्वज्ञं ज्ञानरूपत्वादच्युतं व्यापक महत्। सर्गकाले तु संप्राप्ते ज्ञात्वा तं कालरूपकम् अन्तर्लीनविकारञ्च तत्स्रष्टुमुपचक्रमे। तस्मात् प्रधानमुद्भूतं ततश्चापि महानभूत् ।। इति । रहस्यैः संगुप्तशक्तिवर्गैः सहेत्यर्थः । नासीदिति विरतव्यापारतया कारणरूपेण गर्तस्थमृतसर्पवद् विलीनमासीदित्यर्थः । अन्यथाऽन्तर्लीनविकारं चेत्युत्तरासङ्गतेः, अत्यन्तासत्त्वे “सत्त्वाच्चावरस्ये” त्यागामिसूत्रविरोधापत्तेश्च । शान्तम् रागादिरहितम् औपाधिकव्यापारशून्यं च, न तु सुषुप्तवद् विषयसंवेदनरहितम्, सर्वज्ञमित्युत्तरात् । निर्गुणं नित्यमेव गुणानभिमानेन गुणासङ्गेन च गुणातीतं, गुणानां विलयाद्वा निर्गुणत्वम् । नित्यनिर्मलमिति जीवव्यावृत्तिजीवानामौपाधिककादाचित्कमालिन्यात् । मलाश्च क्लेशकर्मविपाकाशयाः । सर्वज्ञं ज्ञानरूपत्वात् इत्यनेन साधननैरपेक्ष्यमीशस्य सर्वाकारवृत्तेरुक्तम्। ज्ञानरूपत्वमत्र निरावरणसत्त्वमूर्तिकत्वं विवक्षितम् । प्रधानस्योत्पत्तिश्च प्रकृतिपुरुषसंयोगेनाभिव्यक्तिर्गौणीति बोध्यम्। “संयोगलक्षणोत्पत्तिः कथ्यते कर्मज्ञानयोः” इति मात्स्यात् ।}

One needs to understand that this cause for the rise of the world is also the reason for the activity denoted by the word Brahman. Only the primary cause has unsurpassed expansion. The word Brahman has an etymological and conventional meaning (of unsurpassed expansion) like that of the word pañkaja (denoting a lotus). Therefore jīva etc., is not the main meaning of the word Brahman.\footnote{Bhikṣu understands that the word Brahman itself implies the meaning that it gives rise to the world etc. It therefore cannot mean the jīva as the advaitin claims.} Thus it is said in the Nāradasimha: “ādisargamaham tāvat kathayāmi dvijottamarahasyaiḥ…tasmāt pradhanamudbhūtam tataścāpi mahānabhūt”. “rahasyaiḥ” in the above verses means with the varieties of powers which are hidden within Īśvara. “nāsīt” (above) means, in the form of a cause through activity without attachment, it was hidden like a dead serpent in a hole. Otherwise the hidden change will be in contradiction to what is said later; if absolutely non-existent then there is the danger of its being incompatible with the later sūtra “sattvāccāvarasya” (BS. II.1.16).\footnote{This is satkāryavāda where the effect is believed to be existent in the cause before it comes into existence.} “śāntam” (above) means devoid of qualities such as attachment etc., and also devoid of activity associated with its limitation. It is not like deep sleep devoid of being conscious of objects; this has been made clear by the word “sarvajñam” (above). “nirguṇam”= it is eternally without qualities; it surpasses the guṇas (guṇātītam) by not being attached to the guṇas and by not having any sense of pride in the guṇas; or because of the absorption of the guṇas it has no characteristic of the guṇas. “nityanirmalam”=It (Brahman) is different from jīva since the jīvas are subject to defects due to being associated at times with their upādhis (limitations). And the defects are deposits (of karma) such as kleśa, karma and vipāka. “sravajñam”=being of the very nature of consciousness Īśa, without the aid of any instruments (of knowledge) has modifications (of the mind) of all shapes (sarvākāravṛtteruktam). Here the idea of the nature of consciousness is desired as being a personification of sattva without any obstacles. One should know that the manifestation of the evolution existing in pradhāna (pradhānasthotpattiśca) is due to the contact between puruṣa and prakṛti which is secondary. Thus it is said in the Matsya Purāṇa “samyogalakṣaṇotppattiḥ kathyate karmajñānayoḥ” (MBh. Mokṣa.216.11; cited in Tripathi p.19, fn.2). “anayoḥ” means of the above mentioned puruṣa and prakṛti. 

\dev{अनयोः पूर्वोक्तपुरुषयोः । तथाऽनयोर्लयोऽपि वियोगरूप एव कौर्मे प्रोक्तः—}
\begin{verse}
\dev{वियोजयत्यथान्योन्यं प्रधानपुरुषावुभौ ।}\\
\dev{प्रधाप्नपुंसोरनयोरेष संहार ईरितः  ॥}
\end{verse}
\dev{अत्र वाक्ये ब्रह्मसंज्ञमित्यनेन परब्रह्मण्येव रूढिः ब्रह्मशब्दस्योक्ता। तथा विष्णुपुराणेऽपि परमेश्वरे एव ब्रह्मशक्तिरुक्ता\-—}
\begin{verse}
\dev{न सन्ति यत्र सर्वेशे नामजात्यादिकल्पनाः ।}\\
\dev{सत्तामात्रात्मके ये ज्ञानात्मन्यात्मनः परे ।।}\\
\dev{तद्ब्रह्म परमं धाम स चात्मा परमेश्वरः ।}\\
\dev{स विष्णुः सर्वमेवेदं यतो नावर्तते यति  ।। इति ।}
\end{verse}
\dev{आत्मनः पर इत्यनेन जीवस्याब्रह्मत्वमुक्तम् । विष्णुर्महाविष्णुः । “बृहत्वाद् बृंहणत्वाच्च आत्मा ब्रह्मेति गीयत” इत्यादिवाक्यं चांशांश्यविभागेनात्मसामान्यपरं परमात्मपरं वा, जीवे सर्वशक्तिबृंहितत्वाभावादिति । यस्मात् परमेश्वर एव मुख्यो ब्रह्मशब्दार्थः, “परं जैमिनिर्मुख्यत्वादि” त्यागामिसूत्रात् । हिरण्यगर्भे त्वपरब्रह्मणि ब्रह्मात्मन्यूनशक्तितया तदव्यवहितकार्यत्वादिना ब्रह्मशब्द गौण इति वक्ष्यति “सामीप्यात्तु तद्व्यपदेशः” इति सूत्रेण । अत एव मनौ—}
\begin{verse}
\dev{यत्तत् कारणमव्यक्तं नित्यं सद्सदात्मकम् ।}\\
\dev{तद्विसृष्टः स पुरुषो लोके ब्रह्मेति गीयते ।। इति ।}
\end{verse}
\dev{“एतद्वै तद्ब्रह्म परमपरं चे” त्यादौ श्रुतौ गौणमुख्यभेदेन ब्रह्मद्वयवचनं बोध्यम् । अन्यथा “यन्मनसा न मनुते येनाहुर्मनो मतम्, तदेव ब्रह्म त्वं विद्धि नेदं यदिदमुपासत” इत्यादिश्रुतिविरोधात् । अन्यजीवेषु तु ब्रह्मशब्दप्रयोगोंऽशांश्यभेदाद् विभुत्वसर्वाधारत्वादिगुणयोगाद् वेति बोध्यम् ।}

“anayoḥ” means, the above mentioned puruṣa and prakṛti. Similarly their dissolution is mentioned in the Kūrma P. as their separation (viyogarūpa eva): “viyojayatyathānyonyam…eṣa samhāra īritaḥ” (Uttarārdha.48.19; cited in ibid p.19. fn.3). Similarly in the Viṣṇu P. as well the power of Brahman is mentioned as situated in Parameśvara alone: “na santi yatra sarveśe nāmajātyādikalpanāḥ…sa viṣṇuḥ sarvamevedam yato nāvartate yatiḥ” (6.4.36; ibid. p.20. fn.1). “ātmanaḥ para”= through this (phrase) it is indicated that the jīva is other than Brahman.  “viṣṇuḥ”=denotes Mahā Viṣṇu. The sentence: “bṛhattvād bṛhaṇattvācca ātmā brahmeti gīyate” denotes in general the ātman or the paramātman in the sense of non-separation due to the nature of being part and whole, as there is the absence of increase of all powers in jīva. That is why Parameśvara alone is primarily denoted by the word Brahman by the following BS “param jaiminirmukhyatvāt” (BS. 4.3.12).

But with reference to Hiraṇyagarbha , the other (inferior) Brahman, which being less powerful than Brahman and also due to its being directly connected to effects (such as the world) the word Brahman has been used in a secondary sense. This will be mentioned in the sūtra  “sāmīpyāttu tadvyapadeśaḥ” (BS.4.3.9). That is why Manu states: “yattat kāraṇamayaktam…brahmeti gīyate” (I.11; the reading in my edition has “kīrtyate” instead of “gīyate” as the last word). In such śruti statements as “etadvai tadbrahma paramaparam ca” one understands a twofold Brahman different (from each other) as primary and secondary. Otherwise it will contradict such śruti statements as  “yanmanasā na manute…nedam yadidamupāsate” (Kena.Up. 1.6). The use of the word Brahman when referring to other jīvas is understood in the sense of non-difference between the part and whole or in the sense of all pervading (ātman) and as the support of everything (Brahman).

\dev{आधुनिकास्तु जीवब्रह्मणोरखण्डतया जीवेऽपि ब्रह्मशब्दो मुख्य एव आकाशशब्द इव घटाकाशे । जीवस्याब्रह्मत्वं त्वज्ञानकल्पितम्। तथाहि—“तत्त्वमसि, अहं ब्रह्मास्मि, अनेन जीवनात्मनाऽनुप्रविश्य नामरूपे व्याकरवाणि, नान्यदतोऽस्ति द्रष्टा” इत्याद्यभेदश्रुतिशतेभ्यो जीवोऽपि ब्रह्मैव चिन्मात्रत्वाविशेषात् । ऐश्वर्यबन्धयोश्चोपाधिद्वयधर्मत्वात् । न च “द्वा सुपर्णा सयुजा सखाया समानं वृक्षं परिषस्वजाते, नित्यो नित्यानां चेतनश्चेतनानामेका बहूनां यो विदधाति कामान् । तमात्मस्थं येऽनुपश्यन्ति धीरास्तेषां शान्तिः शाश्वता नेतरेषाम्,  आत्मनि तिष्ठन्  आत्मनोऽन्तरः स मे आत्मेति विद्यात्, त्रिषु धामसु यद् भोग्यं भोक्ता भोगश्च यद्भवेत् । तेभ्यो विलक्षणः साक्षी चिन्मात्रोऽहं सदाशिवः ।।” इत्यादिभेदश्रुतिशतानुपपत्तिरिति वाच्यम्, औपाधिकभेदानुवादकत्वेन तादृशवाक्योपपत्तेः । यथा हि घटाकाशादुपाधिपरिछिन्नमहाकाशोऽन्य इति व्यवह्रियते तथैव बुद्ध्यवच्छिन्नचैतन्याज्जीवादन्यः परमेश्वर इति श्रुतिषु व्यवह्रियते । अथवा यथा मायाविनः खड्गचर्मधरात् सूत्रेणाकाशमधिरोहतः सकाशात् स एव मायावी परमार्थभूतो भूमिष्ठोऽन्यस्तथैवाविद्याकल्पितात् कर्तृभोक्तृलक्षणाज्जीवादन्यः परमेश्वरोऽस्तु । तदवमवच्छेदभेदेन बिम्बम्प्रतिबिम्वरूपेण वा जीवेश्वरयोर्भेदः । तथा श्रूयतेऽपि—“आकाशमेकं हि यथा घटादिषु पृथग्भवेत्। तथात्मैको ह्यनेकस्थो जलाधारेष्विवांशुमान्” इत्यादिष्विति वदन्ति ।}

\eject

\textbf{Ques:} Modern advaitins\footnote{Bhikṣu castigates Śaṅkara as a modern day Vedāntin which is the same as advaita for Bhikṣu. This also can mean that there is an implicit approval of the other schools of Vedānta.} consider that the word Brahman is used in a primary sense due to the immutable nature of both jīva and Brahman, like the word ākāśa with reference to ākāśa in the delimited pot. The idea of jīva not being Brahman is imagined due to ignorance (according to them). Thus through hundreds of śruti statements like “tattvamasi” (Chānd.UP.6.8.7; 6.9.4; 6.14.3), “aham brahmāsi” (Bṛ.Up. I.4.10), “anena jīvenātmanā’nupraviśya namarūpe vyākaravāṇi” (Chānd.Up. 6.3.2), “nānyadato’sti draṣṭā”(Bṛ.Up.3.7.23) (it is established that) the jīva is also Brahman alone, as there is the common characteristic of consciousness (cinmātratvāviśeṣāt). And both have upādhidharmas (characterised by) power and bondage (aiśvaryabandhayoścopādhidvayadharmatvāt). They also say it is not in opposition to other hundreds of śruti statements that declare difference such as “dvā suparṇā\-…netareṣām”, “ātmani tiṣṭhan…vidyāt” (not traced in śruti; however according to Tripathi p.20.fn.2 it is in the Liṅga.P) , “triṣu dhāmasu…\-sadāśivaḥ”(Kaivalya.Up.18). By explaining the possession of different limitations the above sentences can make sense. Just as the great ākāśa is considered different from the limited pot-ākāsa so also in śrutis, Parameśvara is considered to be different from the jīva-consciousness limited by the intellect. Or just as the same māyāvī (magician) who ascends to the sky through a string from the shield holding the sword of the māyāvin, appears to be different from the one standing in truth on the earth, so also let Parameśvara be different from the jīva characterized by being an agent, an experiencer etc., which comes about because of the imagination of avidyā (ignorance). Thus in this manner due to   different limitation(s), or like the mirror and the reflection in it, there is a difference between the jīva and Īśvara. Thus one hears the saying “ākāśamekam hi yathā ghaṭādiṣu…jalādhāreṣvivāṁśumān” (not traced). 

\dev{तत्रोच्यते— अभेदवाक्यानुरोधेन भेदवाक्यानामौपाधिकभेदपरत्यं यथा कल्यते,}

\dev{तथा भेदवाक्यानुरोधेनाभेदवाक्यानामविभागादिलक्षणाभेदपरत्वं कथं न कल्प्यते ? अविरोधस्योभयथैध सम्भवात् । श्रूयते चाविभागादिरूपाभेदोऽपि “यथोदकं शुद्धे शुद्धम: क्षिप्तं तादृगेव भवति एवं मुनेर्विजानत आत्मा भवति गौतम, न तु तद्द्वितीयमस्ति ततोऽन्यद् विभक्तमित्यादिश्रुतिषु । स्मृतिषु च—}
\begin{verse}
\dev{अविभक्तं च भूतेषु विभक्तमिव च स्थितम् ।}\\
\dev{व्यक्तं स एव वाऽव्यक्तं स एव पुरुषः परः  ॥ इत्यादिषु ।}
\end{verse}
\dev{प्रत्युतविभागादिलक्षणाभेदस्य पारमार्थिकतया तत्परत्वमेवोचितम् । औपाधिकभेदस्य तु मिथ्यात्वेन तत्परत्वं नोचितमिति । न चाविभागपरत्वे सत्यभेदशब्दे लक्षणाऽस्ति, भिदिर्विदारणे इति विभागेऽपि भिदिधातोरनुशासनात्।}

\dev{ननु जीवेऽपीश्वरभेदस्य स्वानुभवसिद्धतया तत्र श्रुतेर्न प्रामाण्यं किन्तूपासनार्थेऽनुवादतामात्रम् । प्रमाणान्तरानधिगतत्वात्तु तयोरभेद एव तात्पर्यमिति चेन्न, अभेदोपासनवाक्येन प्रसक्तस्य भेदाभिभवस्यैव विवेकवाक्यैः प्रतिषेधस्यास्माभिरभ्युपगमात्, दुःखभोगादिदोषाणामीश्वरे प्रतिषेधात्।}

\dev{ननु मोक्षफलश्रवणादभेदवाक्यानामेव सम्यग् ज्ञानपरत्वमसङ्कोचश्चेति युक्तमिति चेन्न “पृथगात्मानं प्रेरितारञ्च मत्वा जुष्टस्ततस्तेनामृतत्वमेती” त्यादि श्रुतिभिर्भेदज्ञानस्यापि मोक्षहेतुत्वश्रवणात् । भेदज्ञानस्य विवेकज्ञानतया अविद्यानिवर्तकस्यैव “तमेव विदित्वाऽति मृत्युमेती” त्यादिभेदवाक्येष्वाधिक्येन मोक्षफलश्रवणात्। किं च सम्यग्ज्ञानत्वेन हेतुना भेदाख्यविवेकज्ञानस्यैव साक्षान्मोक्षहेतुत्वं श्रुतिसिद्धम्। “अस्थूलमनणु अह्रस्वम् , न तदश्नाति किंचन, त्रिषु धामसु यद् भोग्यं भोक्ता भोगश्च यद्भवेत् तेभ्यो विलक्षणः, अथात आदेशो नेति नेति तर्ह्येतस्मादिति नेत्यन्यत् परमस्ती” त्यादिश्रुतिषु,}
\begin{verse}
\dev{“प्रधानपुरुषव्यक्तकालानां परमं हि यत् ।}\\
\dev{पश्यन्ति सूरयः शुद्धं तद् विष्णोः परमं पदम् ।।}
\end{verse}
\dev{इत्यादिस्मृतिषु च भेदज्ञानस्यैव सम्यग्ज्ञानत्वं गम्यते ।}

\textbf{Ans:} It is said: that just as in accordance with the statements supporting identity (between Brahman and ātman) one imagines that the statements that mention difference  is due to the difference in limitations (of the two), so also with regard to the statements supporting difference/non-identity why cannot one imagine identity to be in the form of non-separation (of the two) in statements supporting identity? Non-contradiction is possible both ways. One hears of identity in the form of non-separation in śruti sayings like “yathodakam…gautama” (Kaṭho.Up. II.1.15); “na tu taddvidīyamasti…vibhaktam” (Bṛ.Up.4.3.27), and also in smṛtis like: “avibhaktam ca bhūteṣu…sa eva puruṣaḥ paraḥ” (Viṣ.P. 6.4.44 cited in Tripathi. p.21.fn. 1). On the contrary, it is proper that identity in the form of non-separation is what (these statements) are inclined towards in truth (pratyuta avibhāgādilakṣaṇābhedasya\break pāramārthikatayā tatparatvamevocitam). The difference in limitations by being false is not in favour of that. Also when the word abheda is interpreted as non-separation it is not in a secondary sense (na cāvibhāgaparatve satyabhedaśabde lakṣaṇā’sti). Even when the root “bhid” denotes breaking (bhidirvidhāraṇe) there is (also) instruction regarding the root “bhid” to mean separation\footnote{Bhikṣu thus justifies that even grammatically the root bhid has the meaning of separation. So abheda can also mean non-separation.}.

\textbf{Ques:} Even in the jīva one experiences the difference from Īśvara, so there is no need for the authority of the śruti (for this purpose).  However,to say that  since there is no other pramāṇa available for this purpose, so its intention is only identity,

\textbf{Ans:} It is not right. We accept only what is rejected through sentences that reflect and overpower statements of difference (bheda) which are connected to identity statements which have devotion as intent, this is because there is rejection of defects such as sorrow, experience etc., in Īśvara.

\textbf{Ques:} If it is said that since freedom is declared as the result, sentences of identity alone lend themselves amenably to knowledge without any restriction (then) the answer is: 

\textbf{Ans:} It is not so; śruti statements such as “pṛthagātmānam…tenāmṛ\-tatvameti” (Śvet. Up.I.6) declare that even statements of difference can be the cause for freedom (mokṣahetutvaśravaṇāt). Through discrimination of knowledge pertaining to knowledge of difference (bhedajñānasya vivekajñānatayā) one hears mainly of the result of freedom through the removal of ignorance in sentences declaring difference such as: “tameva viditvā’ti mṛtyumeti” (Śvet. Up.III.8; VI.15). Moreover  since the cause  is having right knowledge, only discriminating knowledge known as difference is the direct cause of freedom which is established by śruti.\footnote{Bhikṣu seems to stress the fact that discrimination assumes a difference between two entities for it to function. This seems to also support the SY theory of insight into the difference between puruṣa and prakṛti being the criterion for moka/apavarga.} In such śruti satements as: “asthūlamanaṇu ahrasvam…” (Bṛ.Up.3.8.8), “natadaśnāti kiṁcana…” (Bṛ.Up. 3.8.8; Subāla. Up. 3.2), “triṣu dhāmasu yadbhogyam bhoktā bhogaśca yad bhavet tebhyo vilakṣaṇaḥ…(Kaiv. Up.18), “athāta ādeśo…netyanyat paramasti” (Bṛ. Up. 2.3.6) and in smṛti statements as “pradhānapuruṣavyaktakālānām…viṣṇoḥ paramam padam” only statements of difference lead to correct knowledge. 

\dev{“सत्येन लभ्यस्तपसा ह्येष आत्मा सम्यग्ज्ञानेन ब्रह्मचर्येण नित्यमित्यादिश्रुतिभ्यः, ततो मां तत्त्वतो ज्ञात्वा विशते तदनन्तरमि” त्यादिस्मृतिभ्यश्च सम्यग् ज्ञानादेव मोक्षः श्रूयत इति अभेदवाक्यान्यपि साक्षादविद्यानिवर्तकत्वासम्भवेन ब्रह्मात्मतावाक्यानमेव शेषभूतानि । न ह्यभेदज्ञानं साक्षादेवाहं दःखीत्यादिलक्षणामविद्यामुच्छेत्तुमर्हति । एकास्मिन्नेवाकाशेऽवच्छेदभदेन शब्दतदभाववदेकस्मिन्नेवात्मनि कार्यकारणलक्षणावच्छेदभेदेन दुःखादितदभावसंभवादिति । किं च ब्रह्माभेदस्य जडेष्वपि श्रवणान्न तेन दुःखादिशून्यतासिद्धिः। तस्माद् विवेकवाक्यरूपतया भेदवाक्यान्येव बलवन्ति, तद्विरोधेन चाभेदवाक्यान्यविभागपरतयैव संकोच्यानि ।}

\dev{ननु “य एतस्मिन्नुदरमन्तरं कुरुतेऽथतस्य भयं भवति” त्यादि श्रुतौ}
\begin{verse}
\dev{“तस्यात्मपरदेहेषु सतोऽप्येकमयं हि यत् ।}\\
\dev{विज्ञानं परमार्थोऽसो द्वैतिनोऽतशयदर्शिनः ॥}
\end{verse}
\dev{इत्यादिस्मृतौ च भेदनिन्दाश्रवणान्न भेदपरत्वं श्रुतीनां सम्भवतीति चेन्न, अभेदवाक्यानामविभागपरतया भेदनिन्दावाक्यानामपि विभागलक्षणमेदपरत्वात् प्रतिपाद्यविपरीतस्यैव निन्दार्हत्वात् । अन्यथा “मनसैवेदमाप्तव्यं नेह नानास्ति किंचन, मृत्योः स मृत्युमाप्नोति य इह नानेव पश्यतो” त्यादिश्रतिषु जडवर्गेष्वपि भेदनिन्दनादभेदः स्यात् विवेकादिवाक्यान्न भवतीति चेत्, तुल्यं जीवेऽपि । जीवादपि ब्रह्मणो विवेकस्योक्तत्वात्। “तस्यात्मपरदेहेष्वि” त्यादिवाक्यं च “एकमयमि” ति शब्दादवैधर्म्यलक्षणभेदपरं प्रकरणाद् ब्रह्मात्मैक्यपरमेव वेति ।}

From such śruti sayings like: “satyena labhyastapasā…brahmacaryeṇa nityam” (Muṇḍ. Up.3.1.5) and also from smṛti statements like: “tato mām tatvato jñātva viśate tadanantaram” one hears that mokṣa is attained only by correct/right knowledge. Statements of identity also, by not being able to remove directly ignorance, only end up by meaning Brahman of the nature of ātman (brahmātmatāvākyānāmeva śeṣabhūtāni). The knowledge of identity cannot directly remove ignorance of the form ‘I am hungry’ etc. Just as in one ākāśa it is possible to have sound and its absence so also in one ātman, it is possible through different limitations characterized by cause and effect to have the form of sorrow and its absence. Moreover, as one hears the identity of Brahman even with regard to inanimate things it is not proven that there is absence of duḥkha through that (kiṁca brahmābhedasya jaḍeṣvapi śravaṇānna tena duḥkhādiśūnyatāsiddhiḥ). Therefore  having the nature of discrimination alone, statements which advocate difference are more powerful; as against them sentences that advocate identity need to be narrowed down in the sense of non-separation (tadvirodhena cābhedavākyānyavibhāga paratayaiva saṁkocyāni).

\textbf{Ques:} In such śruti statements as:  “ya etasmin…bhayam bhavati” (not traced) and in smṛti sayings like: tasyātmaparadeheṣu…dvaitino’ta\-thyadarśinaḥ” (Viṣ.P. 2.14.31; cited in Tripathi p.22.fn.1) since difference is found fault with, it cannot be said that it is not possible to interpret śruti statements as favouring difference (between Brahman and ātman).

\textbf{Ans:} The fault needs to be removed by pointing out that statements of identity are in the sense of non-separation and statements  which find fault with difference are (to be understood) in the sense of difference which has the characteristic  of separation; in this way it is the contradiction that needs to be blamed (viparītasyaiva nindārhatvāt). Otherwise in such śruti statements as: “manasaivedamāptavyam…iha nāneva paśyati” (Kaṭho. Up. II.1.11) by faulting difference even in inanimate things one will have to accept identity with them. 

\textbf{Ques:} If it is said that because of statements of difference it will not occur with regard to them (inanimate things), then with reference to them the answer is:

\textbf{Ans:} the same logic applies to the jīva as well (tulyam jīve’pi); there are statements that proclaim the difference of Brahman from jīvas. Sayings like: “tasyātmaparadeheṣu” (not traced), “ekamayam” (not traced) etc., are inclined towards identity characterized by difference (avaidharmyalakṣaṇābhedaparam) or towards identity of Brahman and ātman depending on the context (prakaraṇād brahmātmaikyaparameva veti).

\dev{नन्वेवमपि लाघवमैकात्म्यश्रुतेर्बलमस्त्विति चेन्न,     बन्धमोक्षव्यवस्थानुपपत्तिप्रमाणसिद्धतया  आत्मनानात्वगौरवस्यैवादर्तव्यत्वात् । या च प्रतिबिम्बावच्छेदरूपाभ्यां बन्धमोक्षादिव्यवस्था रचिता सा  स्वशिष्यमोहनमात्रं प्रतिबिम्बस्य तुच्छतया बन्धमोक्षानौचित्यात् । ज्ञानेनोपाधिवियोगे जीवनाशप्रसङ्गात्, तत्त्वमसीत्यादिवाक्यार्थतयाऽभ्युपगतस्याखण्डत्वस्य विरोधाच्च सदसतोरभेदानुपपत्तेः । न च प्रतिबिम्बोपाधिना बिम्बस्यैव जीवत्वं वाच्यम्, तथा सत्यवच्छेदभेद एव पर्यवसानात् किमिति प्रतिबिम्बवादः पृथङ् निर्मीयते ।}

\dev{अथैवमुच्यते बिम्बप्रतिबिम्बयोः परमार्थतो नास्ति भेदः किन्तु द्विचन्द्रदर्शनवदेकस्मिन्नेव वस्तुनि भेदभ्रममात्रमिति, तदपि न विचारसहम्, बन्धमोक्षानुपपत्तितादवस्थात् । व्यवस्थातत्प्रतिपादकश्रुत्यादिकं च सर्वमेवात्मातिरिक्तम् अज्ञानकल्पितमेव वक्तव्यमिति चेन्न, एवमपि प्रमाणस्यापि बाधेन ब्रह्मातिरिक्तनिषेधस्य श्रुतस्यापि पुनः संशयापत्तेः, स्वाप्नशब्दस्य जाग्रति बाधे   पुनस्तच्छब्दबोधितार्थसंशयवत्। तथा बन्धमोक्षादिक सर्वथा नास्तीति श्रवणानन्तरं मननादिषु प्रवृत्त्यनुपपत्तेश्च, आप्तवाक्यतः पुरुषार्थाभावनिर्णयादित्यादीन्यनेकानि दूषणानि ।}

\dev{किं चात्मभेदादिकमज्ञानकल्पितं चेत्तदज्ञानं कस्येत्युच्यताम् । ब्रह्मणो भ्रान्तत्वे “ज्ञाज्ञौ द्वावजावीशानीशौ, नाविद्यानुभवात्मनि स्वप्रकाशे, अभयं भ्रान्तिरहितमनिद्रमजरामरम्, }
\begin{verse}
\dev{परः पराणां सकला न यत्र ।}\\
\dev{क्लेशादयः सन्ति परावरेशे ।}
\end{verse}
\dev{इत्यादिवाक्यैर्ब्रह्मण्यज्ञानप्रतिषेधस्य विरोधात् । जीवस्य भ्रान्तत्वे चान्योन्याश्रयात् । भ्रमेण बिम्बप्रतिबिम्बभेदसिद्धौ जीवसिद्धिर्जीवसिद्धौ च तदाश्रयस्य भ्रमस्य सिद्धिरिति ।}

\textbf{Ques:} Even then in the interest of simplicity let the śruti statements of identity be accepted then the answer is, 

\textbf{Ans:} No; due to the proof of contradiction of states of bondage and mokṣa (existing together) one needs to support the  heavier theory of the existence of many jīvas/ātmans.\footnote{The standard criticism that one does not see all jīvas attaining mokṣa when one jīva attains it. This automatically will support the theory of many jīvas} As for the explanation of the states of bondage and liberation being of the nature of reflection or delimitation (pratibimbāvacchedarūpābhyām) that is only for the sake of confusing one’s disciples; a reflection being without substance, bondage and liberation are inappropriate.  Through knowledge when the limitation is disjointed there is the danger of the destruction of the jīva; through statements like “tatvamasi” etc., the meaning obtained of being complete/identical is also contradicted, as identity between what exists and what is non-existent is illogical. Nor can one aver jīvatva (the quality of having the property of jīva) of the reflection like the prototype (bimbasyeva). Thus since what results is only the difference of limitation, why is this theory of reflection manufactured as different.

\textbf{Ques:} If then it is said that there is no difference between the prototype and the reflection in truth but like seeing two moons it is an illusion of division in one object alone, 

\textbf{Ans:} even then it is not fit for discussion (vicārasaham) due to the contradiction of bondage and freedom of those states.

\textbf{Ques:} If it is said that the state of the ātman and śruti sayings which point that out (vyavasthātatpratipādakaśrutyādikam ca) mention that everything other than ātman is to be known as imagined by ignorance, then the answer is:

\textbf{Ans:} it is not so. In this way when pramāṇa (testimony) also is contradicted, even śruti which rejects anything other than Brahman will be subject to doubt. Just as there is a contradiction of the use of the word ‘dream state’ when in the ‘waking state’ so also there arises the doubt as to  the meaning of what the word itself denotes (svāpnaśabdasya jāgrati bādhe punastacchabdabodhitārthasaṁśayavat). Thus having heard that (in truth) there is no bondage and liberation at all times, it will be illogical to engage in practices such as reflection etc., since from the words of the trustworthy, one decides that there is an absence of any puruṣārtha (goal to work for).\footnote{If the goal/puruṣārtha is liberation, then since the śruti says that there is in reality no bondage or liberation then there is nothing to work for.} Thus there are many defects in this approach.

\eject

Moreover if the difference in ātmans is imagined due to ignorance, then one needs to answer the question as to whose is ignorance. If Brahman has ignorance it contradicts statements such as “jñājñau dvāvajāvīsānīśau” (Śvet. Up. 1.9), “nāvidyānubhavātmani svaprakāśe” Nṛsim.~9.6), “abhayam bhrāntirahitamanidramjarāmaram” (not\break traced), “paraḥ parāṇām…parāvareśe” (not traced) which reject ignorance situated in Brahman. If ignorance belongs to jīva it will suffer from the defect of ‘anyonyāśraya’ (mutual dependence on each other.\footnote{Jīva itself comes into being due to avidyā and avidyā is dependent on jīva; thus there is the defect of anyonyāśrayabhāva.} If the difference as prototype and reflection is established due to ignorance, then even in the case of the existence or non-existence of the jīva there is the establishment of ignorance on which it depends. 

\dev{अथ ब्रह्मण्येवाज्ञानम्, अज्ञानप्रतिषेधवाक्यानि चानिर्वचनीयाज्ञानप्रतिषेधेन परमार्थपराणीति चेन्न, व्यवहारभूमावपि ब्रह्मण्यज्ञानासम्भवात्। “ज्ञाज्ञौ द्वावजावीशानीशावित्यादिवाक्यैर्जीवेऽज्ञान- व्यवहारदशायामेव ब्रह्मण्यज्ञानप्रतिषेधात् । किं च प्रतिविम्बवादे अंशश्रुतिस्मृतिसूत्राणां विरोधः स्यात्, प्रतिबिम्बे- ऽम्शव्यवहाराभावात् । यत्त्वा “भास एवे” ति जीवप्रकरणस्थसूत्रं, तत्राभासशब्दो न प्रतिबिम्बवाची तथा प्रयोगादर्शनात् “अंशो नानाव्यपदेशादि” ति सूत्रेणांशत्वमुक्त्वा पुनः सूत्रान्तरेण तद्विरुद्धप्रतिबिम्बतायाः सूत्रणानौचित्याच्च। किन्तु प्रकाशवाची “सर्वेन्द्रियगुणाभासमिति । अस्तु वा हेत्वाभासवदात्माभासवाची जीवस्यापि पारमार्थिकात्मत्वस्य प्रतिषेध्यमानत्वात् प्रकाशे प्रयोगदर्शनादिति, न तत्स्वाभासं दृश्यत्वादिति च । अथैवं प्रतिबिम्बदृष्टान्तः कथमुपपद्येतेति चेत्, अंशांश्यविभागेन किरणसूर्ययोरिव जीवब्रह्मणोरेकपिण्डीभावेन जीवेन अंशैः परमात्मनो नानाबुद्धिप्रतिबिम्बनादित्यवेहि}

Then let ignorance be in Brahman.\footnote{If it is not in jīva then it can only be in Brahman} 

\textbf{Ques:} Then if sayings that reject ignorance and due to rejection of ignorance which is ‘anirvacanīya’ (indefinable) they are said to be leaning towards Brahman, then the answer is: 

\textbf{Ans:} it is not so; even then in the world of activity (vyavahārabhūmau) it will not be possible to have ignorance regarding Brahman.  Through statements such as “jñājñau dvāvajau…” (Śvet.Up. I.9) only in the state of (worldly) activity through ignorance in jīva, is there rejection of ignorance in Brahman.\footnote{Jīva with its apparatus of avidyā-limitation (upādhi)strives to gt rid off avidyā and attain Brahman and realize the true nature of Brahman without ignorance.} Moreover in the theory of reflection there is opposition of the śruti and smṛti sūtras which proclaim that (jīva is a) part (of Brahman), as there is no partial activity in a reflection;  the word “ābhāsa” in the sūtra “ābhāsa eva ca” (BS.2.3.50) in the section dealing with the jīva does not denote a reflection, as one does not see it being used in that sense; having mentioned its (jīva’s) being a part through the sūtra “amśo nānāvyapadeśat…” (2.3.43) then by another sūtra to mention reflection that is opposed to it is not correct. However it denotes illumination (prakāśavācī) as in the saying “sarvendriyaguṇābhāsam…” (Gītā.13.14). Or like a fallacy (hetvābhāsavat/ semblance of reason) it can denote a semblance of the jīva to the ātman;  since there is rejection of jīva being of the nature of the ultimate ātman, it points to its use in the act of illumination\footnote{Jīvātman resembling paramātman has usefulness in the act of illuminating objects.}; it is also not self-illuminating (svābhāsam) as it is an object of experience (dṛśyatvāt).   Then how can the example of reflection fit in? 

\textbf{Ans:} By non-separation of the parts and the whole similar to the sun and its rays, jīva and Brahman by uniting as one jīva, through parts is reflected in the many intellects of paramātman (jīvabrahmaṇorekapiṇḍībhāvena); it can be understood thus.

\dev{नन्वेवं मा भवतु प्रतिविम्बवादोऽवच्छेदवादस्तु स्यादिति, मैवम्, अवच्छेदवादेऽपि “तयोरन्यः पिप्पलं स्वाद्वत्त्यनश्नन्नन्योऽभिचाकशी” त्यादिविभागानुपपत्तेः, धर्मिण एकत्वात् । अथोपाधिविशिष्टयोरेव जीवेश्वरत्वे वाच्ये, तथा च विशेषणभेदाद् भेदः स्यादिति चेन्न, विशिष्टस्यातिरेकानतिरेकयोरुभयतः पाशात् । विशिष्टस्यातिरेके भवदभिमतस्य “तत्त्वमसी” त्यादिवाक्यार्थस्याखण्डत्वस्यानुपपत्तेः । न च लक्षणया विशेषणद्वयं परित्यज्य केवलचैतन्यपरत्वं तत्त्वंपदयोर्वक्तव्यमिति वाच्यम्, वाक्यार्थयोर्विशिष्टयोरतिरिक्ततया केवलचैतन्ये तटस्थलक्षणापत्तौ जीवस्य देहाद्यभिमाननिवृत्त्यसम्भवात्, सर्वपदार्थपरित्यागेन लक्षणया “तत्त्वमसी”- त्यभेद- वाक्यस्य जीवब्रह्मभेदवादिनाप्युपपादयितुं शक्यत्वाच्च।}

\dev{किं च मोक्षावस्थायां विशेषणनाशेन जीवनाशप्रसङ्गः, ब्रह्माणि प्रपञ्चाध्यारोपा- पवादयोरविवेकरूपयोर्वैयधिकरण्यापत्तिश्च । तथा लक्षणां विनैवास्माभिस्तत्त्वमस्यादिवाक्यानां व्याख्येयतया लक्षणानौचित्यं च ।}

\dev{यदि च विशिष्टमनतिरिक्तमुच्यते तदा एकस्मिन्नेवात्मन्यवच्छेदभेदेन बन्धमोक्षैश्वर्यादिप्रसक्त्या “ये तद्विदुरमृतास्ते भवन्त्यथेतरे दुःखमेवापियन्ति” समाने वृक्षे पुरुषो निमग्नोऽनीशया शोचति  मुह्यमानः । जुष्टं यदा पश्यत्यन्यमीशमस्य महिमानमिति वीतशोक” इत्यादिविभागानुपपत्तिः । न कस्मिन्नेव वृक्षेऽवच्छेदभेदेन कपिसंयोगतदभाववति एको वृक्षः कपिसंयोगवानन्यश्च नेत्यमूढैः प्रयुज्यते । न च लौकिकभेदानुवादेन तादृशवाक्यान्युपपादयितुं शक्यन्ते, ज्ञानफलस्याज्ञानदोषस्य चापारमार्थिकत्वे तदर्थकप्रवृत्याद्यनौचित्यात् । अस्मन्मते चात्मनि दुःखभोगतदभावयोः कालभेदेन पारमार्थिकत्वस्योपपाद्यत्वात् ।}

\textbf{Ques:} If it is then said: let the reflection theory be discarded and let us accept the theory of limitation; then the answer is: 

\textbf{Ans:} even in the theory of delimitation it contradicts statements of separation such as “tayoranyaḥ pippalam…abhicākaśīti” (Muṇḍ.Up. 3.1.1) as the dharmī (one possessing the characteristic) is one (dharmiṇaḥ ekatvāt). 

\eject

\textbf{Ques:} If it is said that both jīva and Īśvara qualified by limitations are only meant (athopādhiviśiṣṭayoreva jīveśvartve vācye)  and then, due to the difference in characteristics  there is difference (between the two)\footnote{This is a reference to the jahadajajallakṣaṇā device used to distinguish Brahman from ātman. For a detailed discussion on this lakṣaṇā see \textit{The Naiṣkarmyasiddhi of Sureśvara} by R.Balasubramanian, pp.163-164; 225-228} then the answer is:

\textbf{Ans:} It is not so; the defect of having excellence or not affects both the qualified entities.  When the qualified (entities) have excellence then in your preferred meaning (bhavadabhimatasya) of the sentence “tatvamasi”, there will be a contradiction in knowing it as without division (as complete) (vākyārthasyākhaṇḍatvasyānupapattiḥ). Nor can it be said that by giving up the two qualifiers as having secondary meaning, the words “tat” and “tvam” stand only for ‘consciousness’, as, apart from the qualified meaning of the sentences, in the complete singular consciousness (kevalacaitanye), there is the difficulty of having other qualifications (taṭasthalakṣaṇāpattau);\footnote{Taṭasthalakṣaṇā is a property of a thing which is distinct from its nature and still it is known by that property like the property of smell in the case of earth (Apte’s Practical Sanskrit English Dictionary).} (and) in the case of the jīva, it is (also) not possible to get rid of the sense of agency/pride of the body etc., by giving up all things in a secondary sense;  however, it is possible to reconcile the identity of the sentence “tattvamasi” with those who advocate the difference between Brahman and jīva.\footnote{It is easier to accommodate these ideas when Brahman and ātman are viewed differently.} 

Moreover in the stage of mokṣa by giving up qualifications there will be the contingency of destruction of the jīva; there is also the danger of the superimposition of the world and its refutation being in different case relations   in Brahman (brahmaṇi prapañcādhyaropapavādayoravivekarūpayorvaiyadhikaraṇyāpattiśca). Thus since we can explain sentences like “tattvamasi” even without (resorting to) any secondary sense (tathā lakṣaṇām vinā) it is not correct/proper to use a secondary sense.

\textbf{Ques:} If then one says that there is no reduction of excellence in the qualified one then the answer is:

\textbf{Ans:} in one ātman itself, due to the difference in limitations there will be the inconsistency of connection with bondage, liberation, possession of power etc. It will also not be consistent with statements of separation such as “ye tadviduramṛtāste…   duḥkhamevāpiyanti” (not traced), “samāne vṛkṣe…muhyamānaḥ” (Muṇḍ.Up.3.1.2; Śvet.Up. 4.7), “juṣṭam yada…vītaśokaḥ” (ibid). When due to difference in limitation in one tree itself, it has both contact with a monkey and also it has no contact (with it), then those who know (amūḍhaiḥ) do not say that the one tree has contact with the monkey and the other has no contact.\footnote{There is only one tree and when in contact with the monkey it cannot be said to be another tree not having contact with the monkey.} Nor can such sayings be made intelligible by explaining it as following the meaning of difference common in worldly usage. As the result of knowledge and the defect of ignorance have great significance of meaning, (apāramārthikatve), activity for the sake of achieving it is not appropriate. In our view the experience of sorrow and its absence in ātman is due to the change in time (and) thus is applicable to (achieving) the highest truth (asmanmate cātmani duḥkhabhogatadabhavayoḥ kālabhedena pāramārthikasyopapādyatvāt).\footnote{Bhikṣu seems to be still dealing with the different limitations on the same tree with reference to the monkey. Since both truth and ignorance cannot be compatible with the real truth he uses that for demolishing the avacchedakavāda of Śaṅkara who is his main obsession.} 

\dev{किं चाखण्डैकात्म्ये सति मुक्तस्य पुनर्बन्धापत्तिः, एकान्तःकरणवियोगेऽपि मुक्तांश एवान्तःकरणान्तरसम्भवात् । यथैकघटावच्छिन्नाकाशस्य तद्धटभङ्गेऽपि घटान्तरेण पुनः सम्वन्धो भवति तद्वत् । न च तेनावच्छेदेनान्तःकरणसम्बन्ध एव न भवतीति वाच्यम्, तथा सति योगिनां सर्वगतत्वानुपपत्तेरिति । अन्तःकरण-}

\dev{गणोऽस्माभिरेव कल्पित इति न शङ्कनीयम् । कपिलादिभिरपि श्रुतिद्वैधे यथोक्ततर्काणामेव निर्णायकत्वेनोक्तत्वात् । यथा कपिलसूत्राणि—“जन्मादिव्यवस्थातः पुरुषबहुत्वम्, उपाधिभेदेऽप्येकस्य नानायोग आकाशस्येव घटादिभिः, उपाधिर्भिद्यते न तु तद्वान्, एवमेकत्वेन परिवर्तमानस्य न विरुद्धधर्माध्यासः, नाद्वैतश्रुतिविरोधो जातिपरत्वात्” इति । एतेषां पञ्चसूत्राणामयमर्थः—}

\dev{आत्मैक्ये सत्यौपाधिकानां जन्ममरणादीनामनौपाधिकानां च भोगादीनां श्रुतिस्मृतिसिद्धा आश्रयविभागव्यवस्था न स्यात् “अयं जातोऽयं मृत” इत्यादिरूपा । अतश्चेतना बहव एव न तु लाघवादाकाशवदेकत्वमित्याद्यसूत्रार्थः । श्रुतौ च भेदवदभेदस्याप्यवगमात् तर्केणैवात्र व्यवस्थेत्याशयः कपिलाचार्याणाम् ।}

\dev{ननूपाधिभेदेन व्यवस्था कर्तव्या लाघवतर्कानुग्रहेणाभेदश्रुतेर्बलवत्त्वादित्याशङ्कां समाधत्ते—द्वितीयसूत्रेण । उपाधिभेदे सत्यपि एकस्यैवात्मनो नानोपाधियोगः स्यादतो न व्यवस्थेत्यर्थः ।}

\dev{नन्वेवं विशिष्टमतिरिक्तमेव वक्तव्यं तत्राह—तृतीयसूत्रम् उपाधिरित्यादि । उपाधिरेव भिन्नो वक्तव्यो नतूपाधिमान् उपाधिनाशे जीवनाशप्रसङ्गात्, विशिष्टाऽहं पदार्थे बुद्धिविवेकानुपपत्तेश्च ।}

\dev{उक्तदूषणमुपसंहरति चतुर्थसूत्रेण “एवमि” त्यादिना । एवमेकत्वेनावस्थितस्यात्मनो न विरुद्धधर्माध्यासः औपाधिकानोपाधिकविरूद्धधर्मसम्बन्धः संभवति, विरोधस्यासंभवात्। अविरोधे चाव्यवस्थेत्यर्थः ।}

Moreover when everything is one complete single ātman (akhaṇḍaikātmye) there is the contingency of the liberated ātman being subject to bondage; even when the one internal organ is separated (ekāntaḥkara\-ṇaviyoge’pi) it is possible for the liberated portion to be (connected) to another internal organ. This is similar to (what happens) to the space limited in one pot that can get connected with another pot when the earlier one breaks. Nor can it be said that limitation has no connection whatsoever with the internal organ; if it is so then there will be inconsistency with the yogīs’ capacity of travelling to all places (mentioned in the YS).\footnote{This is a strange example and only confirms Bhikṣu’s commitment to yoga and its siddhis.} There is no need to doubt that the collection of internal organs (mentioned by the YS)  is imagined by us. This has been stated by even sages like Kapila who have determined this after following prescribed logic. Thus it is said in the Kapilasūtras: “janmādivyavasthātaḥ puruṣabahutvam, upādhibhede’pyekasya nānāyoga ākāśasyeva ghaṭā\-dibhiḥ, upādhirbhidyate na tu tadvān, evamekatvena parivartamānasya na viruddhadharmādhyāsaḥ, nādvaitaśrutivirodho jātiparatvāt”. 

The meaning of the (above) five sūtras are as follows: When there is identity of ātman   which is established by śruti and smṛti, birth and death of those with limitations and the experiences of those without limitations will not be possible in the form ‘this person is born, this person is dead’ (ātmaikye satyaupādhikānām janmamaraṇādīnāmana\-upādhikānām ca bhogādīnām śrutismṛtisiddhā āśrayavibhāgavyava\-sthā na syāt “ayam jāto’yam mṛta” ityādirupā)\footnote{The inconsistency of some jīvas having experience even efter the destruction of the limitation of those who are dead.}. Therefore consciousness is many; it cannot be one; for the sake of brevity of expression (lāghavāt) the expression (one) is used like ākāśa; that is the meaning of the first sūtra. In śruti since one understands difference as well as identity (of ātman and Brahman) it needs to be established by logic according to ācārya Kapila (tarkeṇaivātra vyavasthetyāśayaḥ kapilācāryāṇām).

\textbf{Ques:} If it is said that one has to establish (the fact) through the difference in limitation (of the ātman) and the śruti utterances declaring identity are more powerful due to brevity of expression (in logic) then the answer is: 

\textbf{Ans:} that doubt is settled through the second sūtra. Even if there is a difference in limitation there will be many limitations for the one single ātman, so there will be no definiteness, (that is the meaning of the second sūtra).

\textbf{Ques:} In that case one can talk about something that surpasses what is qualified; 

\textbf{Ans:} so the third sūtra says “upādhiḥ…”. One needs to only say that only the limitation is different and not the one which has the limitation. 

\textbf{Ques:} then there is the inconsistency of the destruction of the jīva once the upādhi is destroyed; then in the qualified ‘aham padārtha’ (tattva/principle) of the I-sense there will be the inconsistency of difference of the intellect (from itself).

\textbf{Ans:} The fourth sūtra resolves the defect as follows: “evam…”. Thus when the ātman is established as one, there cannot be superimposition of contradictory qualities; it is not possible to have a relation with contradictory qualities such as having a limitation and not having a limitation; opposite (qualities) do not coexist. If there is no contradiction then there will be confusion (cāvyavasthetyarthaḥ).

\dev{आत्माद्वैतश्रुतिमुपपादयति पञ्चमेन “नाद्वैते” त्यादिना । जातिपरत्वात् चित्सामान्याद्वैतपरत्वात् श्रुतीनामित्यर्थः । सर्ववस्तूनां सामान्यविशेषात्मकत्वेन विशेषरूपं धर्मं परित्यज्य सामान्यरूपेण धर्मिणामात्माद्वैतं प्रतिपाद्यते, वैधर्म्याणामवास्तवत्वप्रतिपादनायेति । यथा च विशिष्टयोस्तत्त्वंपदार्थतामभ्युपगम्य परैर्लक्षणया केवलविशेष्यार्थकत्वं पदयोरुच्यते, तद्वत् सांख्यैरपि । विशेषस्त्वियान् यत्तैर्लक्षणया न क्रियते “स एवायं गकारः” इत्यादाविवेति बोध्यम् । साँख्यैरीश्वरानभ्युपगमात् सामान्याभेद एवोक्तः । ब्रह्म मीमांसायान्तु अग्निस्फुलिङ्गवदंशांश्यभेदोऽप्यविभागलक्षणो वक्ष्यते “प्रकाशाश्रयवद्वा तेजस्त्वात्, अविभागेन दृष्टत्वात्” इत्यादिसूत्रैरिति । तस्मादैकात्म्ये लाघवेऽपि बन्धमोक्षादिव्यवस्थानुपपत्त्यात्मनानात्वगौरवमाश्रयणीयमिति । अपि चौपाधिकमात्रभेदेन श्रतिस्मृती अपि नोपपद्यते । तथाहि, “निरञ्जनः परमं साम्यमुपैति” “यथाग्निरग्नौ संक्षिप्तः समानत्वमनुव्रजेत् । तथात्मसाम्यमस्येति योगिनः परमात्मना” इत्यादौ मोक्षकालेऽपि भेदघटितं साम्यं श्रूयते, तदानीं चौपाधिकभेदो नास्तीति ।}

\dev{‘‘कालः स्वभावो नियतिर्यदृच्छा   भूतानि योनिः पुरुष इति चिन्त्यम्” इत्यनेन मूलकारणे चिन्तां प्रकृत्याम्नायते—}
\begin{verse}
\dev{ते ध्यानयोगानुगता अपश्यन् देवात्मशक्तिं स्वगुणैर्निगूढाम् ।}\\
\dev{यः करणानि निखिलानि तानि कालात्मयुक्तान्यधितिष्ठत्येक: ।। इति ।}
\end{verse}
\dev{अत्र सृष्टेः प्रागपीश्वराधिष्ठेयो जीवोऽस्तीत्यवगम्यते ।  तथा,}
\begin{verse}
\dev{प्रकृतिं पुरुषं चापि प्रविश्यात्मेच्छया हरिः ।}\\
\dev{क्षोभयामास सम्प्राप्ते सर्गकाले व्ययाव्ययौ ॥}
\end{verse}
\dev{इति स्मृतेरपि ।}

\dev{न चायमुपाधिसंबन्धात् पूर्वमधिष्ठेयाधिष्ठातृभावो निरंशस्यात्मनः स्वरूपभेदं विनोपपद्यत   इति।}

By the fifth sūtra “nādvaita” etc. he explains that śruti mentioning advaita is due to its favouring a common generic property (jātiparatvāt) and being intent on non-duality based on a commonality of consciousness (citsāmānyādvaitparatvāt). Since all things have a common and particular property, giving up the particular property he points out the intrinsic non-dual nature common to all qualified things, in order to point out the false nature of contradictory qualities. Just as by accepting only the word meaning of the qualified words   “tat” and “tvam” and through the  secondary meaning of the other words one mentions only the qualified meaning of the words, so do the Sāṁkhyas as well.\footnote{Bhikṣu’s partiality towards SY comes up very often.} There is this difference alone: they do not take recourse to a secondary meaning but just as saying “sa evāyam gakāraḥ” they explain it as lack of understanding (aviveketi bodhyam). Since Īśvara is not accepted by the Sāṁkhyas they mention nonduality as only a common property\footnote{The common property (jāti) of consciousness is taken to be in the sense of non-duality.}. In Brahmamīmāṁsā through such sūtras as “prakāśāśrayavadvā tejastvāt” (BS. 3.2.28), “avibhāgena dṛṣṭatvāt” (BS.4.4.4) it will also be mentioned that like fire and its sparks there is non-difference like a whole and its parts of the nature of non-separation. Therefore even if there is parsimony (of logic) in  (accepting) identity, due to lack of consistency regarding the state of bondage and liberation one needs to depend on the heavier argument of many ātmans. 

Nor can śruti and smṛti statements fit in with just difference in limitations (api caupādhikamātrabhedena śrutismṛtī api nopapadyete). Thus in such statements as “nirañjanaḥ paramam sāmyamupaiti” (Muṇḍ. Up. 3.1.3), “yathāgniragnau saṁkṣiptaḥ samānatvamanuvrajet, tathātmasāmyamasyeti yoginaḥ paramātmanā” (not traced); even at the time of liberation one hears of similarity associated with difference; at that time there is no difference of limitation\footnote{In liberation there are no limitations and hence this cannot be explained due to difference in limitation. Bhikṣu seems to suggest that SY is in truth not against the Upaniṣadic philosophy of Brahman being One alone.}. 

Thus in the Śvet.Up. by the saying: “kālaḥ svabhāvo… iti cintyam” (Śvet.Up.1.2; Nāra.Pa. Up. 9.1) there is instruction to think about the main cause in a natural manner (pravṛttyāmnāyate seems correct). So also from “te dhyānayogānugatā apaśyan…kālātmayuktānyadhitiṣṭha\-tyekaḥ” (Śvet.Up.1.3) one understands that even before creation/\-manifestation Īśvara is the support of jīva. The same is said in the smṛti as: “prakṛtim puruṣam cāpi…vyayāvyayau”. Nor is it logical to think (reasonable to think) that before this connection with the limitation, the relationship of supporter and supported can happen without a change in the partless ātman itself. 
\begin{verse}
\dev{अन्यश्च राजन् प्रवरस्तथान्यः पञ्चविंशकः ।}\\
\dev{तच्छ्रुत्वा चानुपश्यन्ति एक एवेति साधवः” ॥}
\end{verse}
\dev{इति मोक्षधर्मादौ स्वरूपभेदमुक्त्वाऽधिष्ठेयाधिष्ठात्रोरविभागलक्षणमैक्यमुक्तम् । तथा,  “सर्वगत्वादनन्तस्य स एवायमहं स्थितः।” इति विष्णुपुराणादावपि । }

\dev{अतोऽवगम्यते— }

\dev{श्रुतयोऽप्यभेदमविभागलक्षणमेव बोधयन्ति, एकवाक्यत्वात् “ऐतदात्म्यमिदं सर्वमि” ति श्रुतौ जडवर्गाभेदस्याविभागादिरूपस्यैव वक्तव्यतया जीवाभेदस्याप्यविभागादिरूपत्वसिद्धेश्च, अन्यथाऽर्धजरतीयन्यायापत्तेः ।}

\dev{व्यतिरिक्तं न यस्यास्ति व्यतिरिक्तोऽखिलस्य यः   । इति विष्णुपुराणादिभ्योऽप्यत्यन्ताभेदोऽपि न शास्त्रार्थः ब्रह्मणि पुरुषव्यतिरेकानुपपत्तेः ।}

\dev{तथा विष्णुपुराण एव—}
\begin{verse}
\dev{परस्य ब्रह्मणो रूपं पुरुषः प्रकृतेः परः । }\\
\dev{व्यक्ताव्यक्ते तथैवान्ये रूपे कालस्तथा परः  ॥ इति ।}
\end{verse}
\dev{व्यवहारमुक्त्वा परमार्थमाह—}
\begin{verse}
\dev{प्रधानपुरुषव्यक्तकालानां परमं हि यत् ।}\\
\dev{पश्यन्ति सूरयः शुद्धं तद्विष्णोः परमं पदम् ॥}
\end{verse}
\dev{इत्यादिनापि नात्यन्ताभेदः । नन्वेवं }
\begin{verse}
\dev{तस्यैव नित्यतृप्तस्य सदानन्दमयात्मनः ।}\\
\dev{अवच्छिन्नस्य जीवस्य संसृतिः कथ्यते बुधैः” ।।}
\end{verse}
\dev{इति वसिष्ठसंहितासूक्तोऽवच्छेदोऽनुपपन्न इति चेन्न, किरणसूर्यादिवदंशांशिनोर्जीव ब्रह्मणोरेकपिण्डीभावेन तस्य जीवरूपैरंशैरवच्छेदवादस्यापि प्रतिबिम्बवादवदेवोपपत्तेः । यश्च “आकाशमेकं हि यथा घटादिषु पृथग्भवेदि” त्यादिदृष्टान्तः, सोऽप्यविभागरूपैक्यमात्रांशेन पुनरखण्डत्वेऽपि न्यायानुग्रहेण बलवत्तरस्य सखण्डतादृष्टान्तस्यैवादर्तव्यत्वात् }

Thus in the Mokṣadharma : “anyaśca rājan…eka eveti sādhavaḥ” (Mokṣa.P. 318.78, cited in Tripathi p.25.fn.1)  having mentioned the change in oneself, the relationship of identity of the nature of supporter and supported characterized by non-separation is mentioned.  This is also said in the Viṣṇu.P as “sarvagatatvādanantasya sa evāyamaham sthitaḥ” (Viṣ.P.I.19.885 cited in ibid.p.25.~fn.2).   Thus one knows that even śruti instructs identity of the nature of non-separation alone, due to having consistency in meaning (ekavākyatvāt). In the śruti statement: “aitadātmyamidam sarvam” (Chānd.Up.6.8.7) since the identity of inanimate things is mentioned as of the nature of non-separation it follows that there is identity of the nature of non-separation of jīvas as well; otherwise this will suffer from the defect of the principle of being a half-widow\footnote{Well known as the ardhajarjarī nyāya i.e one cannot be a half widow. Similarly what applies to the inanimate things must equally apply to the animate world as well; otherwise it will suffer from the defect known as ‘ardhajarjarī-nyāya’}. 

Thus such statements in the Viṣṇu P. like: “vyatiriktam na yasyāsti…akhilasya yaḥ” ( Viṣ.P. I.19.78. cited in ibid.p.25.fn.3) also do not pronounce total identity as the meaning of the śāstra, since puruṣa being separate from Brahman it (total identity) is illogical (brahmaṇi puruṣavyatirekānupapatteḥ). Similarly in the Viṣ. P again it is said: “parasya Brahmano rūpam…kālastathā paraḥ” (ibid.1.2.15 cited in ibid. p.25.fn.4). Having spoken about the world he mentions the ultimate truth as: “pradhānapuruṣavyaktakalānām…viṣṇoḥ paramam padam; viṣṇoḥ svarūpāt…pradhāna puruṣaśca vipra” (ibid. 1.2.24; cited in ibid. p.25.fn.5); all the others such as the manifested (world) and kāla (time) are different from the true nature of Viṣṇu and so are pradhāna and puruṣa; by this also it says that there is no absolute identity. 

\textbf{Ques:} If it is said that the statement in the Vas.Saṁ: “tasyaiva nityatṛptasya…samsṛtiḥ kathyate budhaiḥ” (not traced) mentioning the worldly life of the conditioned jīva is not correct then the answer is: 

\textbf{Ans:} it is not so. Just like the sun and its rays being whole and possessed of parts, the jīva and Brahman being united as one whole, the argument of limitation in the form of the jīva parts is only as reasonable as the theory of reflection.

As for the simile: “ākāśamekam hi yathā ghaṭādiṣu pṛthagbhavet” that also denotes a unity only in the form of non-separation of the part, and can logically be supportive of the stronger simile of having parts (so’pyavibhāgarūpaikyamātrāśena punarakhaṇḍatve’pi nyāyānugra\-heṇa balavattarasya sakhaṇḍatādṛśṭāntasyaivādartavyatvā

\dev{यथा,}
\begin{verse}
\dev{“वायुर्यथैको भुवनं प्रविष्टो रूपं रूपं प्रतिरूपो बभूव ।}\\
\dev{एकस्तथा सर्वभूतान्तरात्मा रूपं रूपं प्रतिरूपो बहिश्च।।}
\end{verse}
\dev{“अग्नियर्थैको भुवनं प्रविष्ट” इत्यादिः सखण्डतादृष्टान्तः । वायुतदवयवाग्निस्फुलिङ्गादिष्वप्यवयवावयविनोरन्योन्याभावलक्षणो भेदः अविभागलक्षण एव चाभेद इति जीवब्रह्मणोः साम्यमिति ।}

\dev{अपि च चन्द्रजलचन्द्राकाशघटाकाशाग्निविस्फुलिङ्गच्छायातपस्त्रीपुरुषादिदृष्टान्तैः प्रतिबिम्बावरछेदांशादिवादाः परस्परविरोधेन सर्वे न सम्भवन्तीत्येक एव वाद आश्रयणीयः । इतरास्तु विवक्षिततत्तदंशमात्रे दृष्टान्ता इत्यभ्युपेयम् । तथा च सति अंशवाद एवाश्रयितुं युक्तः “अंशो नानाव्यपदेशादि” त्यादिसूत्रेणाचार्यैरंशत्वस्यैव न्यायतो मीमांस्यत्वात् । प्रतिबिम्बादिभावेनाखण्डत्वे स्पष्टसूत्राभावात् । प्रत्युत प्रत्यधिकरणं जीवब्रह्मभेदस्यैव स्पष्टं सूत्रणीयत्वात् “अधिकं तु भेदनिर्देशादि” त्यादिभिः । अंशसूत्रे च भेदाभेदयोर्वक्ष्यमाणत्वादिति । तस्माद् भेदाभेदाभ्यां जीवब्रह्मणोरशांशिभाव एव ब्रह्ममीमांसासिद्धान्तोऽवधारणीयः, “अंशो नानाव्यपदेशात्, अन्यथा चापि दाशकितवादित्वमधीयत एक” इति वक्ष्यमाणसूत्रादिति ।}

It is like the saying: “vāyuryathaiko bhuvam praviṣṭo…pratirūpo bahiśca” (Kaṭh. Up. II.2.10). The line “agniryathaiko bhuvanam praviṣṭa” (ibid. II.2.9) etc., is a simile denoting having parts.  In vāyu and its parts as also in fire and its sparks which denote parts and whole, there is only the difference in the form of mutual absence, and non-difference is non-separation alone; thus it is similar to that between the jīva and Brahman.

Those who argue for theories of reflection and and limited parts (pratibimbāvacchedāmśādivādāḥ) by such examples as moon reflected in water and moon in the sky, space limited in a pot, fire and its sparks, shadow and sunlight, woman and man, (will know) all that is not possible as they are mutually contradictory. Therefore one needs to adopt only one argument. The other arguments should be understood as only examples for those respective parts which is intended (by them). When it is so, it is appropriate to depend on the argument of parts (and whole).  By the sūtra “amśo nānāvyapadeśāt” (BS. 2.3.43) possessing parts has been argued logically by the ācāryas (aṁśatvasyaiva nyāyato mīmāmsyatvāt). There is absence of any definite sūtra stating unity through being a reflection etc. On the other hand every adhikaraṇa (section) has sūtras clearly stating the difference between jīva and Brahman through such sūtras as “adhikam tu bhedanirdeśāt” (2.1.22). And in the sūtra “amśa…” (BS. 2.3.43) the difference/non-difference relationship will be  mentioned. Therefore one should conclude from the upcoming sūtra “amśo nānāvypadeśāt…eke”(BS.2.3.43) that the siddhānta (conclusion) of Vedānta (Brahmamīmāmsā) is that jīva and Brahman which are different have a relationship of part and whole.

\dev{अंशांशिनोश्च भेदाभेदौ विभागाविभागरूपौ कालभेदेनाविरुद्धौ । अन्योन्याभावश्च जीवब्रह्मणोरात्यन्तिक एव, तथा शक्तिशक्तिमदविभागोऽपि नित्य एवेति मन्तव्यम् । तथा च स्मृति—“पृथग् विभक्ता प्रलये च गोप्ते” ति, ‘तदात्मकं तदन्यत् स्याद्यद्रूपं भासकं विदुरि’ ति चेति । तदात्मकं पूर्ववाक्योक्तप्रधानपुरुषकालात्मकमित्यर्थः । अंशत्वं च सजातीयत्वे सति अविभागप्रतियोगित्वम् तदनुयोगित्वं चांशित्वम् । येन च रूपेणांशता यत्र विवक्ष्यते तेनैव रूपेण साजात्यं तत्र ग्राह्यं, यथा आत्मांशलक्षणे आत्मत्वेनैव साजात्यं सदंशादिलक्षणेषु च सत्त्वादिरूपेणैवेत्यतो नातिप्रसङ्गः । अथवा द्रव्यत्वसाक्षाद्व्याप्यजात्या साजात्यं ग्राह्यम् । विभागश्च लक्षणान्यत्वम् अभिव्यक्तधर्मभेद इति यावत् । तद्भावश्चाविभागः । अथवास्तु अविभागः संयोगविशेषः स्वरूपसम्बन्धो वा आधेयत्वादिवत्। जलस्य दध्नि लवणस्य समुद्रे अविभागव्यवहारस्यापलपितुमशक्यत्वात् । “न तु तद्वितीयमस्ति ततोऽन्यद् विभक्तमि”ति श्रुत्या}

\dev{“सति सम्पद्यते  न विदुः सति सम्पद्यामह” इत्यादिश्रुत्या च जीवस्यापि ब्रह्मण्यविभागश्रवणात् । “या प्रकृतिः पुरुषश्चोभौ लीयेते परमात्मनी” त्यादिस्मृतेश्च । “पदगतौ, लिङ् श्लेषण” इत्याद्यनुशासनेभ्यःसम्पत्तिलयाप्ययादिशब्दानामविभागार्थत्वात्। तथा “अविभागो वा वचनादि” त्यागामिसूत्राच्चेति । }

\dev{ननु निरवयवस्य ब्रह्मणः कथं मुख्योंऽशः स्यादिति चेन्न, यथोक्तलक्षणांशत्वस्यावयवत्वाभावेऽपि दर्शनात् । यथा शरीरस्य केशादिरंशो, राशेश्चैकदेशोंऽशः, पितुश्च पुत्र इति । सर्वे च जीवाः पितरि पुत्रचेतना इव चिन्मात्रे ब्रह्मणि नित्यसर्वावभासके विषयभासनरूपं स्वलक्षणं विहाय प्रलये लक्षणानन्यत्वं गच्छन्ति । सर्गकाले च तदिच्छया तत एव लब्धचैतन्यफलोपधाना आविर्भवन्ति पितुरिव पुत्राः । अतो जीवा ब्रह्मांशा भवन्ति । “आत्मा वै जायते पुत्रः” इति श्रुत्या पुत्रे पितुरविभागलक्षणाभेदवज्जीवेऽपि ब्रह्मणोऽविभागलक्षणाभेदस्य “बहुस्यां प्रजायेये” त्यादिश्रुत्या सिद्धेरिति । अतो जीवा ब्रह्मांशा मुख्या एव भवन्ति ।}

The difference and non-difference relationship of part and whole is of the nature of separation and non-separation and is non-contradictory due to change of time. Mutual negation between the jīva and Brahman is absolute; similarly one should know that non-separation between power and the one possessing power is also eternal. Thus there is the smṛti: “pṛthag vibhaktā pralaye ca goptā” (not traced), as also: “tadātmakam tadanyat…bhāsakam viduḥ” (not traced). The word “tadātmakam” (in the above quotation) refers to the nature of definite time of pradhāna and puruṣa. Having parts when belonging to the same generic property is the counter-part of non-separation, and its adjunct is being its whole (tadanuyogitvam cāmśitvam). In which ever form one desires to speak of being a part, one needs to understand its having the same generic property in that form itself; in the  characteristic mark of the ātman, the part has the common generic property of the nature of ātmatva (being ātman), then in the characteristics of the parts also it will be in the form of sattva etc., and thus there is no contradiction in that.\footnote{In other words the amśas also will have the same characteristics of sattva etc., as the amśī (whole).} Or one can understand the common generic quality as that pervaded directly by the characteristic of dravyatva. Vibhāga (separation) is a different characteristic and is another manifested quality. Having that nature is non-separation. 

Or let avibhāga (non-separation) be a special kind of contact or an intrinsic relationship (svarūpasambadho vā) like that which has the action of a support (ādheyatvādivat). It is not possible to deny the phenomenon of non-separation of water in milk, and salt in the ocean. Through the śruti statement:  “na tu…vibhaktam” (Bṛ.Up. 4.3.24) and by the śruti saying: “sati sampadya…sampadyāmaha” one hears of non-separation of the jīva in Brahman. This is also stated in such smṛti statements as: “yāprakṛtiḥ…paramātmani” (not traced).   Through such (grammatical) rules as: “padagatau, liṅ śleṣaṇe” avibhāga has the meanings of words like stability, absorption and increase. So also the later sūtra “avibhāgo vā vacanāt” (support avibhāga) (BS.4.2.16).

\textbf{Ques:} If it is said how can the partless Brahman have important parts then the answer is: 

\textbf{Ans:} It is not so. The said characteristic of having parts is felt even in the absence of having parts. (It is like) hair etc., being  a part of the body, like being a part of a whole collection and like the son being part of the father. Like the cetanā of the son in the father, all jīvas giving up one’s own characteristic of illuminating objects during pralaya get absorbed totally without any characteristic in Brahman of the nature of pure consciousness, which eternally is the illuminator of all (objects).\footnote{The use of cetanā suggests that the son here is still very young and cannot take decisions on his own.} And during the time of evolution through his (Brahman’s) desire, having obtained therein the excellent result of consciousness (jīvas) come into being, like sons of the father. Therefore jīvas are parts of Brahman. By the śruti statement: “ātmā vai jāyate putraḥ”,  like the identity of the nature of non-separation of the father in the son, in the jīva also the identity of the nature of non-separation of Brahman is established by the śruti: “bahusyām prajāyeya” (Chānd.Up.6.2.3; Taitt.Up. 2.6). Therefore jīvas are mainly parts of Brahman. 

\eject

\dev{नन्वंशस्तत्र स्यात्, आकाशवदंशांशिभावो नार्थः (किन्तु अग्निविस्फुलिङ्ग पितापुत्रादिवदेवांशांशिभावोनार्थः किं तु अग्निविस्फुलिङ्गपितापुत्रादिवदेवांशांशिभावोऽभिप्रेत इति कथं निर्धारणीयमिति चेत्, उच्यते—आकाशवदंशांशि\-भावाश्रयणेंऽशशब्दस्य गौणत्वं स्यात् । घटाकाशो ह्याकाशाद् विभक्तो न भवति, लक्षणान्यत्वाभावात् घटाकाशधर्माणामप्याकाशधर्मत्वात् अवच्छिन्ने चांशशब्दो गौणोऽवयवादिशब्दवदिति । किं च एवमपि जीवश्चिन्मात्रस्यैवांशः स्यान्न तु सर्वकर्तुरीश्वरस्य, परैरुपाधिविशिष्टस्यैवेश्वरत्वकल्पनात् । न चेश्वरोपाधेरंशो जीवोपाधिरीश्वरस्यापि तद् द्वारा भवन्मते संसारप्रसंगात् कायव्यूहादाविवेति । अपि च पितापुत्रवदंशांशिभावो व्यासाभिप्रेत इति मोक्ष धर्मस्थाच्छिष्यवैशम्पायनवाक्यादेवावधार्यते । अतो नैवाकाशवदंशत्वं ब्रह्ममीमांसार्थः यथा, मोक्षधर्मे—}

\dev{ “बहवः पुरुषा ब्रह्मन्नुताहो एक एव तु” इति प्रश्ने—}
\begin{verse}
\dev{बहवः पुरुषा राजन् सांख्ययोगविचारिणाम् ।}\\
\dev{नैवमिच्छन्ति पुरुषमेकं कुरुकुलोद्वह ।।}
\end{verse}
\dev{इत्यनेन पुरुषनानात्वमेव विचारतो व्यवस्थाप्य व्यासोकं पुरुषैक्यं पितापत्रवदविभागेनोपपादितम्—}
\begin{verse}
\dev{समासतस्तु यद्व्यासः पुरुषैकत्वमुक्तवान् ।}\\
\dev{तत्राहं संप्रवक्ष्यामि प्रसादादमितौजसः ।।}\\
\dev{बहूनां पुरुषाणां हि यथैका योनिरिष्यते ।}\\
\dev{तथा  तम्पुरुषं विश्वमाख्यास्यामि गुणाधिकम् ॥ इति ।}
\end{verse}
\dev{अस्यार्थः—यत्तु व्यासः समासतो जीवजातस्य ब्रह्मणि प्रक्षेपत आत्मैक्यमुक्तवानर्थात् ब्रह्ममीमांसादौ तत्तुभ्यं जनमेजयाय तच्छिष्यवैशम्पायन आह, तदुपदेशप्रसादाद्वक्ष्यामि । तदेवाह ‘बहूनामिति श्लोकेनेति । उपसंहारे च “एवं बहुविधः प्रोक्तः पुरुषस्ते यथाक्रममि” त्यनेन पुरुषबहुत्वमेव सिद्धान्तितम् ।}

\dev{यत्तु तत्रैव प्रसंगान्तरे वाक्यान्तरम्- }
\begin{verse}
\dev{बहवः पुरुषाः पुत्र यत्त्वया समुदाहृताः ।}\\
\dev{एवमेतर्हि संवृत्तं द्रष्टव्यं नैवमित्यपि ।। इति,}
\end{verse}
\dev{तेन च पुरुषविभागस्य विकारवत् वाचारम्भणत्वमभिप्रेत्य जीवात्मताप्रतिषेधतः एकस्यैव ब्रह्मणः पारमार्थिकात्मत्वमुक्तं, न तु जीवात्मत्वं स्थापयित्वा पुरुषबहुत्वं निराकृतम् । तस्मात् पितापुत्रवदंशत्वमेवागाम्यंशसूत्रस्यार्थः । तथा च स्पष्टे श्रतिस्मृती अंशत्व एव भवतः ।}
\begin{verse}
\dev{“मायां तु प्रकृतिं विद्यान्मायिनं तु महेश्वरम् ।}\\
\dev{अस्यावयवभूतैस्तु व्याप्तं सर्वमिदं जगत् ।।”}
\end{verse}
\dev{तत्सां (तस्यां) शोऽयं यश्चैतामात्राः प्रतिपुरुषं क्षेत्रज्ञाः”}
\begin{verse}
\dev{“यथा सुदीप्तात् पावकाद् विस्फुलिङ्काः सहस्रशः प्रभवन्ते सरूपाः}\\
\dev{तथाक्षराद् विविधाः सौम्य भावाः प्रजायन्ते तत्र चैवापियन्ति ।।}
\end{verse}
\dev{इत्यादिश्रुतिः । स्मृतिश्च—}
\begin{verse}
\dev{“ममैवांशो जीवलोके जीवभूतः सनातनः” इत्यादिरिति ।}
\end{verse}
\textbf{Ques:}  But then let there be parts, but (you say that) the meaning cannot be that it is like part and whole similar to space. However how can it be determined that the part and whole is like fire and its sparks or like that of father and son? Then the answer is: 

\textbf{Ans:} when one depends on the meaning of part and whole as like that of space, then the meaning of the part becomes secondary. The pot-space is not divided from space (as such) since it does not possess another (different) quality (lakṣaṇānyatvābhāvāt); the characteristics of the pot-space also have the same characteristic as space (itself). When the word denoting a part is limited, it has a secondary meaning like the words denoting parts. Moreover in this state the jīva becomes a part of consciousness alone and not of Īśvara who is the creator of all; others imagine īśvaratva (state of being īśvara) as that qualified by a limitation. The limitation of jīva is also not a part of the limitation of Īśvara, as through that, in your view, there will be the contingency of Īśvara having (connection with) samsāra  through (having) an assemblage of body etc. Moreover the relationship of part and whole, like that between father and son, is desired by Vyāsa which is known from the words of his disciple Vaiśampāyana in the Mokṣadharma section (of the MBh). Therefore the meaning of being a part with reference to the jīva in the BS, is not like being a part like that of space (ākāsa) limited by pot.Thus in answer to the question “bahavaḥ puruṣā brahmannutāho eka eva tu” it is said “bahavaḥ puruṣā rājan sāmkhyayogavicāriṇām naivamicchanti puruṣamekam kurukulodvaha”, thus establishing, after reflection, the existence of many puruṣas; it is (also) shown by Vyāsa that the identity of puruṣas is a non-separation (relationship) similar to that of father and son as “samāsatastu yadvyāsaḥ puruṣaikatvamuktavān…guṇādhikam”.

The meaning of this is: “that which has been mentioned briefly by Vyāsa projecting the identity of the collection of jīvas in Brahman in such works as Brahmamīmāṁsā (Vedāntasūtras) that has been conveyed to you, Janamejaya, by his disciple Vaiśampāyana, (and) I shall tell you that through the grace of his teaching” (tadupadeśaprasādādvakṣyāmi). That same (idea) is mentioned through the verse (given above) “bahūnām” etc. In conclusion the existence of many puruṣas is established through the verse: “evam bahuvidhaḥ proktaḥ puruṣaste yathākramam”. In that same context there is the statement at the end of the discussion (prasaṅgāntare): “bahavaḥ puruṣāḥ…naivamityapi”; desiring through that (verse) the separation of puruṣas to be (only) a matter of speech (and) rejecting the jīvas’ ātman-nature (jīvatmatāpratiṣedhataḥ) the absolute ātman-nature of the single Brahman is mentioned; nor has the existence of many puruṣas been abandoned after having established the ātman-nature of the jīvas.\footnote{In other words there is a hierarchy accepted. While the jīvas share the ātman-nature with paramātman they are not the ultimate absolute ātman; similarly, having accepted the ātman-nature of the jīvas does not preclude from accepting the existence of many puruṣas as well. Brahman is the only supreme singular Ātman. The use of ātman for jīvas is only in a secondary sense.} Therefore the meaning of the upcoming sūtra “aṁśo” etc., (BS. 2.3.43) is that the relationship between Brahman and the jīvas is like that between a father and son. Thus the śruti and smṛti statements like: “māyām tu prakṛtim vidyāt…vyāptam sarvamidam jagat” (Śvet.Up 4.10), “tasyāmśo’yam…kṣetrajñānāḥ”, “yathā sudīptāt…prajāyante atra caivāpiyanti” (Muṇḍ.Up.2.1.1), and the smṛti: “mamaivāṁśo jīvaloke jīvabhūtaḥ sanātanaḥ”are clearly about (jīvas) being a part (of Brahman). 

\dev{यद्यपि जीवा अपि ब्रह्मवदेव  विभुचिन्मात्ररूपास्तथाप्युपाध्यवच्छेदेनैवाभिव्यक्तपरिच्छिन्नचैतन्यतया विस्फुलिङ्गतुल्या भवन्ति “बुद्धेर्गुणेनात्मगुणेन चैव ह्याराग्रमात्रो ह्यवरोऽपि दृष्टः” “बालाग्रशतभागस्य शतधा कल्पितस्य च । भागो जीवः स विज्ञेयः स चानन्त्याय कल्पत” इत्यादिश्रुतेरिति ।}

\dev{अस्य चांशस्य भेद ( अभेद ) प्रतिपादनस्य फलम्, अंशांशिनोरुत्सर्गतः एकरूपतया जीवस्याप्यसंसारित्वविभुत्वसर्वाधारत्वादिज्ञापनम्, ब्रह्मणश्चोपाधिविवेकेन चिन्मात्रत्वज्ञापनं, तथा जीवेंऽशेऽभेददृष्ट्या ब्रह्मोपासनं तथा ब्रह्मात्मत्वोपपादनं चेत्यादि । एतेषु व्रात्मत्वोपपादनं मुख्यं फलं ब्रहात्मतापरत्वात् सर्ववेदान्तानाम् । अत एवाचार्यो वक्ष्यति—“आत्मेति तूपयन्ति ग्राह्यन्ति च” इति । ब्रह्मात्मज्ञानादेवौपाधिकसंसारवज्जीवमारभ्य स्थूलदेहपर्यन्तेष्वभिमाननिवृत्तेः । अत इदं ब्रह्मात्मज्ञानं विविक्त- जीवज्ञानातू सांख्योक्तादपि श्रेष्ठं, नातोऽधिकं ज्ञानमस्ति ।}

\dev{नन्वंशांशिभावज्ञानात् कथं ब्रह्मात्मज्ञानं स्यात्? उच्यते—यो यत आगत्य यदधिष्ठितं यत्र जीवित्वा यत्र लीयते समुद्रतरंगादिवत् जीवात् तद्बुद्ध्यादिवच्च स तस्यामा भवति, }
\begin{verse}
\dev{यच्चाप्नोति यदादत्ते यच्चात्ति विषयानिह ।}\\
\dev{यच्चास्य सन्ततोभावस्तस्मादात्मेति कथ्यते ।।}
\end{verse}
\dev{इति आत्मलक्षणस्मृतेः । यथा हि देहेन्द्रियादीनां बुद्धिपर्यन्तानामुत्पत्तिलयाधारतया तत्साक्षित्वेनाधिष्ठातृत्वादिना च जीवस्तेषामात्मा तत्स्वरूपज्ञानाच्च तेष्वहमित्यभिमानो निवर्तते “नाहं बुद्ध्यादिरि” ति विद्ययेति सांख्यसिद्धान्तः । तथैव जीवानां चिन्मात्रस्वरूपाणामप्युत्पत्तिलयाधारतया तत्साक्षित्वेन तदधिष्ठातृत्वादिना चेश्वरस्तेषामप्यात्मा तत्स्वरूपज्ञानादेव च जीवेष्वहमित्यभिमानो निवर्तते ब्रह्मात्मविद्ययेति ब्रह्ममीमांसासिद्धान्तः । “तमेवैकं जानथ आत्मानं, नान्योऽतोऽस्ति द्रष्टा नान्योऽतोऽस्ति श्रोता, यदेव साक्षादपरोक्षाद् ब्रह्म, य आत्मा सर्वान्तरः, य आत्मनि तिष्ठन्नात्मानमन्तरो यमयति, यस्यात्मा शरीरम्, नित्यो नित्यानां चेतनश्चेतनानामि” त्यादिश्रुतेः। }
\begin{verse}
\dev{तत्त्वैः सम्पादितं भुङ्क्ते पुरुषः पञ्चविंशकः ।}\\
\dev{ईश्वरेच्छावशात् सोऽपि जडात्मा कथ्यते बुधैः ।।}\\
\dev{तवान्तरात्मा मम च ये चान्ये देहसंज्ञिताः ।}\\
\dev{सर्वेषां साक्षिभूतोऽसौ न ग्राह्यः केनचित् क्वचित् ।।}\\
\dev{अन्यश्च राजन् प्रवरस्तथान्यः पञ्चविंशकः ।}\\
\dev{तत्स्थत्वाच्चानुपश्यन्ति एक एवेति साधवः ।।}\\
\dev{ते चैनं नाभिनन्दन्ति पञ्चविंशकमप्युत ।}\\
\dev{षड्विंशमनुपश्यन्तः शुचयस्तत्परायणाः ।।}\\
\dev{तमो रजश्च सत्त्वञ्च विद्धि बुद्धिगुणानिमान् ।}\\
\dev{बुद्धिमात्मगुणं विद्यादात्मानं परमात्मनः ।।}
\end{verse}
\dev{इत्यादिस्मृतेश्च । “आत्मेति तूपयन्ती” त्यादिसूत्राच्च,  तथा “आत्मा भोक्तुरि” ति पूर्वाचार्यवाक्याच्चेति ।।}

Even though the jīvas are all pervading (and) of the nature of pure consciousness still, since it is a limited consciousness manifested through the delimitation by conditions they are like sparks (of fire). Thus śruti says: “buddherguṇenātmaguṇena…dṛṣṭaḥ” (Śvet.Up. 5.8), “bālāgra\-śatabhāgasya…kalpate” (not traced). The result of pointing out the difference (or identity) between the part and the whole is in order to inform about the non-worldly, all-pervasive, all-supportive nature of the jīva as well, and to inform that because of the difference in limitation, Brahman is of the nature of consciousness alone (brahmaṇaścopādhivivekeṇa cinmātratvajñāpanam). Thus in the part that is the jīva, there is devotion to Brahman keeping the focus on identity as also establishing the āmatva of Brahaman (tathā jīvem’śe’bhedadṛṣṭyā brahmopāsanam tathā brahmātmatvopapādanam cetyādi). In these statements the  establishment of Brahman as having ātmatva (nature of ātman) is the main purpose, since all the Vedānta systems are in favour of Brahman being ātman. Thus Śaṅkarācārya  will mention: “ātmeti tūpayanti grāhayanti ca”.\footnote{Available readings have “ātmeti tūpagacchanti grāhayanti ca” ( BS. 4.1.3)}   Starting a jīva’s life in saṁsāra (and) possessing (one’s) limitation throughout one’s gross bodies, till the end of the sense of agency, there is an end to it only through the knowledge of Brahman as being ātman (brahmātmajñānādevaupādhikasamsāravajjīvamārabhya sthūladehaparyanteṣvabhimānanivṛtteḥ). Thus this knowledge of Brahman being ātman is better than the knowledge of jīva’s being different (from prakṛti) which is mentioned by Sāmkhya as there is no more (important) knowledge than that (to be achieved).\footnote{Bhikṣu is willing to concede that Sāṁkhya’s apavarga/mokṣa is inferior to that of Vedānta knowledge. This is inspite of his maintenance of the cosmogony of Sāṁkhya and the Yoga sādhana for attaining mokṣa.}  

\textbf{Ques:} If it is said how can the knowledge of Brahman being ātman be got by knowledge of (jīvas being) part of the complete (Brahman), then the answer is:

\textbf{Ans:} that which comes from some source on which it is supported, (and) having lived therein gets absorbed there itself just like the waves of an ocean, is like the intellect etc., that comes from jīva,. Thus the Ātmalakṣaṇasmṛti states: “yaccāpnoti yadādatte…tasmādātmeti kathyate”(not traced).

Just as the jīva is the ātman of the body, the sense organs etc., right upto the intellect etc., due to its being the support for their rise and absorption (therein), and by being the witness (of all that happens) and by being their support and when it realizes its true nature due to (correct) knowledge (vidyayā), the sense of identity in them is destroyed in the form “nāham buddhyādiḥ” (I am not the intellect etc.). This is the Sāṁkhya doctrine. In a similar manner by being the support for the rise and absorption of the jīvas of the nature of pure consciousness, by being the witness (of everything), by being their supporter etc., Īśvara is also their ātman and only by realizing its true nature will the sense of agency as ‘I’ in the jīvas disappear through the knowledge of Brahman as ātman; this is the doctrine of Brahmamīmāṁsā (Vedānta).

Thus the śruti statements say: “tamevaikam jānatha ātmānam”\break (Muṇḍ.Up.2.2.5), “nānyato’sti drāṣṭā…śrotā” (Bṛ.Up.3.7.23), “yadeva\break sākṣādaparokṣāt” (Bṛ.Up 3.4.2), ``ya ātmā sarvāntaraḥ” (Bṛ.Up. 3.4.1)”,\break ``ya ātmani tiṣṭhan…gamayati”, “yasyatmā śarīram”, “nityo nityānām cetanaścetanānām” (Śvet.Up.6.13). Smṛti statements like “tatvaiḥ sampāditam bhuṅkte puruṣaḥ pañcavimśakaḥ…buddhimātmaguṇam vidyādātmānam paramātmanaḥ” also support this. Sūtras such as “ātmeti tūpayanti” (BS.4.1.3) and the ācārya’s statement “ātmā sa bhoktuḥ” support this. 

\dev{आत्मनानात्वं च चैतन्ययोग्यतामात्रेण शास्त्रेषुक्तम् । बहिर्जीवस्य चित्शक्तिमत्त्वरूपमात्मत्वं नाभ्युपगम्यते “आदरादलोप” इति सूत्रविरोधात् सुखदुःखभोगानुपपत्तेश्च। किन्तु जीवेषु चैतन्यफलोपधानं कादाचित्कतया वाचारम्भणमात्रमीश्वरपरतन्त्रं सदल्पं चेति। जीवाश्चिच्छक्तिगुणयोगाद् गौणात्मान एव, यथाऽध्यक्षत्वगुणयोगेन प्राणः करणानामात्मा, तद्वत्। मुख्यस्त्वात्मेश्वर एव, सदा सर्वज्ञत्वापरतन्त्रत्वाच्चेत्येवाभ्युपगम्यते “नान्योऽतोऽस्ति द्रष्टे” त्यादिश्रुतेरिति । तथा च नारदीये गौणमुख्यभेदेनात्मद्वयमुक्तम्—}
\begin{verse}
\dev{आत्मानं द्विविधं प्राहुः परापरविभेदतः ।}\\
\dev{परस्तु निर्गुणः प्रोक्तो ह्यहङ्कारयुतोऽपरः ।। इति ।}
\end{verse}
\dev{परापरौ श्रेष्ठाश्रेष्ठौ मुख्यगौणत्वाभ्यामिति भावः ।}

\dev{इदमेव खण्डैकात्म्यमबुद्ध्वा आधुनिका अखण्डैकात्म्युपपत्तये जीवानां प्रति बिम्बावच्छेदादिरूपैः कुकल्पनां कुर्वन्ति । सांख्यनैयायिकादिभिश्चैतन्यफलयोग्यता रूपमात्मत्वं जीवस्योच्यत इति न तदुविरोधः । यथा   चातेक्षिकात्मनो जीवस्य तत्वज्ञानादप्यहंकर्तेत्याद्यभिमाननिवृत्त्या मोक्षोभवति तथा विद्यपादे वक्ष्यामः । यदि वा शेषशेषिरूपेणात्मद्वयमेवाभ्युपगम्यते तदा ब्रह्मात्मताज्ञानस्याविद्यानिवृत्तिहेतुत्वं न साक्षादभ्युपगन्तुं शक्यते ब्रह्मात्मताज्ञानानिवर्तकत्ववद् देहाद्यात्मताज्ञानानिवर्तकत्वस्याप्यौचित्यात् । अपि त्वदृष्टेश्वरानुग्रहादिद्वारैवाविद्यानिवृत्तौ मोक्षे च हेतुत्वमभ्युपगन्तव्यम् । तथा च सति —}
\begin{verse}
\dev{परमात्मा हरिः स्वामी दासोऽहं तस्य सर्वदा ।}\\
\dev{स्वेच्छया विनियोक्तातस्तस्यैवात्मेश्वरस्य हि ।।}\\
\dev{अहंकृतिर्मकारः स्यान्नकारस्तन्निषेधकः ।}\\
\dev{तस्मात्तन्तमसैवातश्चाहङ्कारविमोचनम् ।।}
\end{verse}
\dev{इति यमपुराणयुक्त्या ईश्वरात्मज्ञानात् साक्षात् सङ्गाताभिमाननिवृत्तिर्जीवात्मत्वाभावश्च न घटेतेति । एतेन “एष त आत्मान्तर्याम्यमृतः स म आत्मेति विद्यात्, तत्त्वमसी” त्यादिश्रुतयो व्याख्याताः जीवभेदेऽपीति मन्तव्यम् ।}

The statement of many ātmans/puruṣas in the śāstras is only due to the capacity of consciousness; the ātmatva of the external power of  jīva, due to having the nature of consciousness is not appropriate, as it is in contradiction to the sūtra “ādarādalopa”(BS.3.3.40) and it is also incompatible with the experience of pleasure and pain.\footnote{The meaning of the sūtra is: “There can be no omission (of the performance of the agnihotra to Prāṇa) on account of the respect shown (in the Upaniṣad)” (trans.Swami Gambhirananda). Similarly the use of the word ‘ātman’ for jīva/puruṣa is not appropriate but is used out of respect for the term used in the Upaniṣads.} However, the bestowing of the result of consciousness being intermittent is only meant for worldly purposes, and being dependent on Īśvara it is meagre (kintu jīveṣu caitanyaphalopādhānam kādācitkadayā vacāraṁbhaṇamātram īśvaraparatantram sadalpam ceti). Jīvas by association with the quality of the power of consciousness are only subsidiary ātmans; this is like prāṇa being the ātman of the senses by having the quality of being the supervisor. The main ātman is Īśvara alone, as he is at all times omniscient and independent; this is in accordance with śruti statements such as “nānyato’sti draṣṭā” etc (Bṛ.Up.3.7.23). So also in the Nārada P. two ātmans have been mentioned by dividing them as subsidiary and primary, due to being superior and inferior, excellent and not excellent as well as being primary and subsidary: “ātmānam dvidham…aparaḥ”(Nār.P. 31.57 cited in Tripathi p.30. fn.1). 

Not knowing this partial identity, modern Vedāntins argue wrongly for a single ātman in the form of the theory of reflection and limitation. The Sāmkhyas and Naiyāyikas mention the ātman nature of the jīva in the form of having the capacity of the result of consciousness which is not contradictory. We shall mention in the section on vidyā how jīva of the nature of seeing/knowing, through giving up the sense of agency such as ‘I am the doer’ etc., attains liberation. If one accepts the two fold ātman in the form of śeṣa (servant/subsidiary) and śeṣī (master/primary) then knowledge of Brahman as ātman acquired by removal of ignorance is not possible directly (tadā brahmātmatājñānasyāvidyānivṛttihetutvam na sākṣādabhyupagantum śakyate); just as the non-removal of the knowledge of Brahman as ātman, it is proper that there is non-removal of knowledge of the body being ātman. However, only when, through the blessings of the unseen Īsvara, ignorance is removed can one understand its cause for the achievement of mokṣa.\footnote{Bhikṣu's theistic leanings are evident here.} 

Following the statement in the Yama Purāṇa : “paramātmā hariḥ svāmī dāso’ham tasya sarvadā…tasmāttantamasaivātaścāhaṅkāravimoca\-nam” (see also Tripathi.p.30 fn.3) the statement that, through the knowledge of Īśvara as ātman, there is the direct removal of the collective sense of ego/agency and the absence of the ātmatva of the jīva does not make sense. The śruti statements like “eṣa ta ātmāntaryāmyamṛtaḥ” (Bṛ.Up 3.7.3; 4.2-3), “tattvamasi” (Chānd.Up 6.8.7 and in many other places) one should know that they have also been explained in terms of difference in jīvas.

\dev{ननु “तत्त्वमसि” वाक्यस्यांशांश्यविभागेनाप्युपपत्तेः कथमस्य ब्रह्मतापरत्वमवधारणीयमिति चेत्, अविद्यानिवर्तकतयाऽभ्यर्हितत्वेन बाधकाभावे सर्वत्रैवयाभेदवाक्येषु ।}

\dev{ब्रह्मात्मतापरत्वस्यौत्सर्गिकत्वात् “तं त्वौपनिषदं पुरुषं पृच्छामी” तिश्रुतेः, वेदा ब्रह्मात्मविषया इति स्मृतेश्च । अंशांश्यभेदस्यापि ब्रह्मात्मतावगतिफलत्वात् । न च “ऐतदात्म्यमिदम् ” सर्वं स आत्मे” ति पूर्वभागेनैव ब्रह्मात्मत्वं लब्धमिति वाच्यम्, यत “ऐतदात्म्यमि” त्यनेन प्रपञ्चरस्य ब्रह्मणि स्वरूपत्वलक्षणमेवात्मत्वमविभागलक्षणाभेदो वा घटस्य मृदात्मत्ववत्, न त्वध्यक्षत्वरूपं चेतनत्वरूपं वा, “स आत्मे” त्यनेन वा साक्षित्वरूपमात्मत्वमुक्तम् । ताभ्यां च हेतुभ्यां “तत्त्वमसी” त्यनेन चाध्यक्षत्वरूपमात्मत्वमिति विभागः। अध्यक्षत्वरूपात्मत्वस्यैव त्वमहंशब्दार्थत्वात्। अहंममेति शब्दाभ्यां स्वस्वामिभावावगमात् । अत एव पातञ्चले “स्वस्वामिशक्त्योः स्वरूपोपलब्धिहेतुः संयोगः” इति स्वस्वामिशब्दाभ्यां बुद्ध्यात्मानावुक्ताविति । अतएव च “अथात आत्मादेशोऽथातोऽहङ्कारादेश” इति श्रुत्योर्न पौनरुक्त्यम्, अर्थभेदात् । अत एव च “नास्मि न मे नाहमित्यपरिशेषम्” इति सांख्यकारिकायामपि न पौनरुक्त्यं च, नास्मीत्यनेन साक्षित्वस्य नाहमित्यनेन स्वाम्यस्य प्रतिषेधादिति चेन्न एवं च कार्यकारणसंघाताध्यक्षो भवतीति कृत्वा स एवाहमित्युच्यते, यथा प्राणा इन्द्रियाध्यक्षतयैव तेषामात्मोच्यते । न चेश्वरस्य सम्बोध्यत्वप्रयोक्तृत्वाभावात् कथं त्वमहंशब्दार्थता स्यादिति वाच्यम्, वागिन्द्रियद्वारा जीवस्य प्रयोक्तृत्ववज्जीवाख्यकरणद्वारा ब्रह्मण एव संबोध्यत्वादिसकलव्यवहारप्रतिपादनायैव तत्त्वमसीत्युपदेशादिति । तथा च श्रुत्यन्तरम्— “नान्योऽतोऽस्ति द्रष्टा श्रोता मन्ता बोद्धे” त्यादिना परमात्मन एव दर्शनश्रवणादिसर्वव्यवहारकर्तृत्वमाह ।}

\textbf{Ques:} If the sentence “tattvamasi” is explained as the non-separation of the parts and whole how can one understand the non-separate nature of Brahman? Then the answer is: 

\textbf{Ans:} By the removal of ignorance. Since that is the most suited (abhyarhitatvena) in all statements which lean towards identity of Brahman being ātman in the absence of obstacles (bādhakābhāve sarvatraivābhedavākyeṣu brahmātmatāparatvasyautsargikatvāt). Thus the śruti statement “tam tvaupaniṣadam puruṣam pṛcchāmi” (Bṛ.Up.3.9.26) supports this. Smṛti also says that the Vedas deal with the subject of Brahman and ātman. The purpose of the non-difference between the parts and the whole is also to understand the ātman nature of Brahman. Nor can it be said that through “aitadātmyamidam sarvam sa ātmā…” (Chānd.Up 6.8.7 and in many other places in Chānd.Up) the ātman nature of Brahman is understood from the initial statement itself, since by saying “āitadātmyam” it can mean only that the intrinsic nature of the world in Brahman has ātmatva, or is of the nature of identity in the sense of non-separation like the pot possessing the essence of clay, and not of the nature of superintendence or of the nature of consciousness; nor by the statement “sa ātmā” is there a reference to ātmatva in the sense of being a witness. It is because of those reasons that by the statement “tattvamasi” there is a division of its having ātmatva in the sense of superintendence. The meaning of the words “tvam” and “aham” denotes ātmatva in the sense of having the power of superintendence. The meaning of the word “aham” denotes being ātman having the nature of superintendence. One understands the words “aham”, “mama” to have the sense of ‘being the possessed’ and ‘being the possessor’. That is why in Patañjali’s YS “svasvāmiśaktyoḥ…samyogaḥ” (YS.II.23) the intellect and the ātman have been denoted by the words “sva” and “svāmī”. That is why one does not have the repetition of the śruti in the form “athāta ātmādeśo’thāto’haṅkārādeśaḥ” (Chānd.Up. 7.25.2) due to the difference in meaning.\footnote{Bhikṣu seems to suggest that if the little self (ahaṁkāra) was intended to be the purport then the ādeśa should also repeat the ādeśa for ahamkāra (little self) as well; but it does not do so.} That is why again in the Sāṁkhyakārikā (SK) “nāsmi…\-apariśeṣam” (not traced) is not repeated; since by the phrase “nāsmi” the state of being a witness, and by the phrase “nāham” the state of being a master (svāmyasya) is denied.\footnote{It seems that Bhikṣu had access to this SK which is not found in the extant SK.} In this manner by being the superintendent of the collection of cause and effect it is said ‘I am he’. This is like saying that prāṇa by being in charge of the sense-organs is their self (teṣāmātmocyate). 

\textbf{Ques:} Nor can it be said that since Īśvara cannot be addressed or be used as an agent, how can it denote the meaning of the word “aham”. 

\textbf{Ans:} just as the jīva has agency through the sense organ of speech, for the sake of indicating all worldly activity such as being addressed etc., it is Brahman alone that is being addressed through the instrument known as the jīva; this is the (meaning of) the instruction “tattvamasi”. Thus another śruti states through “nānyato’ato’sti…boddhā” (Bṛ.Up.3.7.23) that paramātman alone has agency of worldly activities such as seeing, hearing etc.

\dev{यत्तु आधुनिकास्तत्त्वमसीत्यादिवाक्ये जीवस्यैव त्वमहंपदार्थ निर्णये सति “को न आत्मे” त्यादिश्रुत्यन्तरानुसारिण्या लोकानुसारिण्याश्चाकाङ्क्षाया अनुपपत्तेः ।लोके हि कोऽहमित्याकाङ्क्षयैवामुकस्त्वमसीत्युत्तरं दृश्यते नान्यथेति । अस्मन्मते तु अहंशब्दार्थत्वेन प्रयोक्तृसंघाताध्यक्षत्वादिना सामान्यरूपेणैवाकाङ्क्षायां तदित्यादिविशेषरूपेणोपदेश इति लौकिकी शब्दमर्यादा न हीयत इति । यदपि तत् त्वमेव त्वमेव तत् इति परस्परव्यतिहारवाक्यं तदन्योन्यवैधर्म्यलक्षणं भेदं निवर्तयति जीवस्यासंसारित्वप्रतिपादनाय परमात्मस्वरूपप्रदर्शनाय च, अन्यथा व्यतिहारवैयर्थ्यात्। “यच्चाप्येवं सकलं जातमपि सर्वं प्रतिष्ठितम् , स एव जीवः सुखदु:खभोक्ते” त्यादिवाक्यं तदविभागेनोपासनां विदधाति । यच्च— }
\begin{verse}
\dev{विभेदजनकेऽज्ञाने नाशमात्यन्तिकं गते ।}\\
\dev{आत्मनो ब्रह्मणा भेदमसन्तं कः करिष्यति ।।  इति विष्णुपुराणं.}
\end{verse}
\dev{तस्यायमर्थः— धर्माधर्मादिद्वारा विभागजनकेऽज्ञाने देहाद्यभिमाने ज्ञानेनात्यन्तमुत्सन्ने सति असन्तं कदाचित्कत्वेन वाचारम्भणमात्रं जीवब्रह्मविभागं पुनः कः करिष्यति कारणनाशादिति}

\dev{यच्च—}
\begin{verse}
\dev{यदि जीवः पराद् भिन्नः कार्यतामेति सुव्रत ।}\\
\dev{अचित्त्वं च प्रसज्येत घटवत् पण्डितोत्तम ।।}
\end{verse}
\dev{इति गौतमीयतन्त्रम्, तस्याप्ययमर्थः— यदि जीवः परा भिन्नः परादत्यन्तं भिन्नः परस्यानंश इति यावत् तथा प्रलयकालीनचेतनाद्वैतश्रुत्यनुरोधेन जीवस्य कार्यत्वं स्यात् प्रकृत्यादिवदभेदे च घटवज्जडत्वमेव प्रसज्यत इत्यर्थः । “क्षेत्रज्ञं चापि मां विद्धि” इत्यादिवाक्यं चांशांश्यविभागपरम् । एवमन्यान्यपि जीवब्रह्माभेदवाक्यान्यनयैव दिशा यथायोग्यं प्रकरणानुसारेणाशांश्यभेद- ब्रह्मात्मत्व-सामान्याभेदैस्तृतीयसूत्रे वक्ष्यमाणेन शक्तिशक्तिमदभेदेन वा व्याख्येयानीति दिक् ।}

However when modern (vedāntins) in making the decision in the meaning of the words ‘tvam’ and ‘aham’ as referring to    the word jīva itself is the meaning of “aham” in the sentence “tattvamasi” then it is in contradiction to other śruti statements such as “ko na ātmā”\footnote{This quote is probably from the Aitareya Up (Ait.Up 3.1.1)} (not traced) as also to the desire of people to know (about the ātman)  which is in accordance with worldly behaviour (śrutyantarānusāriṇyā lokānusāriṇyāścākāṅkṣāyā anupapatteḥ). In the world it is with the desire to know who ‘I am’ (`am I') the answer ‘you are so and so’ is given not otherwise. In our view the meaning of the word ‘aham’ does not abandon the worldly usage (laukikī śabdamaryādā) of being a superintendent who directs a collective in general; thus with that desire the instruction (upadeśa) is given in the special form as “tat”\footnote{The one who supervises and directs all the functions is denoted by the word “tat” in “tattvamasi”} Even though such sentences as “tat tvameva”, “tvameva tat” are mutually interchangeable it only accomplishes the division of the nature of mutually different properties in order to indicate the unworldly nature of the jīva and the intrinsic nature of paramātman; otherwise the reciprocity will be useless. The statement: “yaccāpyevam sakalam…sukhaduḥkhabhoktā” (not traced) accomplishes worship through (the idea of) non-separation.

The meaning of the verse “vibhedajanake’jñane…kaḥ kariṣyati” in the Viṣṇu P  (6.7.94 cited in Tripathi p.31.fn 3) is as follows: Due to dharma and adharma when ignorance having the sense of agency in the body etc., gives rise to separation , then when the separation of jīva and Brahman which is just in the form of worldly activity (kādācittkatvena vācāraṁbhaṇamātram) intermittently, which is not exhausted, is destroyed totally by (correct) knowledge then who will bring about the separation of jīva and Brahman as the cause (for the separation) is destroyed.

Also the verse in the Gautamīyatantram: “yadi jīvaḥ…paṇḍitottama” has the same meaning. Its meaning is as follows: If the jīva is totally different i.e. not a part of the absolute (parād bhinnaḥ), then in keeping with the advaita texts, which mention consciousness at the time of dissolution (pralayakālīnacetanādvaitaśrutyanurodhena) jīva will become a kārya (effect); if there is identity like prakṛti (during dissolution) then like pots etc., there is the danger of its being insentient\footnote{When different it can suffer the result of becoming like any other effect/object (kāryatvam) during pralaya. If identical like prakṛti in pralaya then there is the contingency of its being insentient like a pot etc.}. Such statements as “kṣetrajñam cāpi mām viddhi” (Gītā.13.3) are inclined towards non-separation like a whole and its parts. So also in the same way statements of identity between jīva and Brahman are mentioned in the same manner appropriately, depending on the context, in the upcoming third sūtra (BS.I.1.3) as identity like that between the whole and its parts; or they can be explained as the identity relationship between power and the one who possesses power (śaktiśaktimadabhedena vā vyākhyeyānīti dik). 

\dev{ये तु तार्किका अविभागलक्षणाभेदमपि त्यक्त्वा उपासनामात्रपरत्वेनैव अभेद वाक्यानि नयन्ति, तन्मते भेदनिन्दाश्रुत्यनुपपत्तिः, अन्योन्याभावस्य पारमार्थिकत्वाद् विभागाविभागरूपयोश्च भेदाभेदयोः श्रुत्यर्थत्वानभ्युपगमादिति । अधिकं वा भेदसाधकभेदसूत्रेषु वक्ष्यामः । तस्मात् सिद्धौ जीवेश्वरयोरंशांशिभावेन भेदाभेदौ विभागाविभागरूपौ । तत्राप्यविभाग एव आद्यन्तयोरनुगतत्वात् स्वाभाविकत्वात् नित्यत्वाच्च सत्यः । विभागस्तु मध्ये स्वल्पावच्छेदेन नैमित्तिको विकारान्तरवद् वाचारम्भणमात्रमिति विशेषः ।} 

\dev{तदेवमात्माद्वैतं व्याख्यातं सामान्यतो ब्रह्माद्वैतवाक्यानि च तृतीयसूत्रे व्याख्यास्यामः । तदेवमन्योन्याभावलक्षणभेदेन जीवादत्यन्तभिन्न एवेश्वरो ब्रह्मशब्दार्थ इति सिद्धम् ।}

\dev{तत्राप्ययं विशेष—जीवव्यावर्तनायैव जगज्जन्मादिकारणं तथा नित्येच्छादितन्मायाख्यशक्त्यौपाधिकमैश्वर्यं चोपलक्षणमेव ब्रह्मणः, ननु ( न तु ) ऐश्वर्यमपि ब्रह्मशब्दार्थान्तर्गतं “सत्यं ज्ञानमनन्तं ब्रह्म, तदेव ब्रह्म त्वं विद्धि नेदं यदिदमुपासते, साक्षी चेता केवलो निर्गुणश्च, अथात आदेशो नेति, अकर्ता चैतन्यं चिन्मानं सदि” इत्यादिस्मृतिभिः,}
\begin{verse}
\dev{ज्ञानमेव परं ब्रह्म ज्ञानं बन्धाय वै (ने) ष्यते ।}
\end{verse}
\dev{ज्ञानात्मकमिदं विश्वं न ज्ञानाद् भिद्यते परम् ।। इत्यादिस्मृतिभिः, “चितितन्मात्रेण तदात्मकत्वादित्यौडुलोमि” रित्यागामिसूत्रेण च चिन्मात्रस्यैव ब्रह्मजीवशब्दार्थत्वावगमात् उपाधिविशिष्टे शक्ति कल्पयित्वा केवले लक्षणाकल्पनायां गौरवाच्च। ब्रह्मणीच्छादिव्यवहारस्य तु स्वस्वामितासम्बन्धेन अविवेकन चोपपत्तेः । एवमेव परमात्मपरमेश्वरादिशब्दा अपि चैतन्यविशेष एव शक्ताः ।}

\dev{“वदन्ति तत्तत्त्वविदः तत्त्वं यज्ज्ञानमद्वयं ब्रह्मेति परमात्मेति भगवानिति शब्द्यते” इत्यादिश्रुतिभ्यः।}

\dev{“अनामरूपश्चिन्मात्र” इत्यादिवाक्यैस्तु जन्मानिमित्तकानामेव निषिद्धमिति ।}

Those logicians (tārkikāḥ) abandoning identity as a non-separation relationship, take the identity sentences to be inclined towards meditation/devotion (upāsanāmātraparatvena); in their view it is in contradiction to śruti that condemns difference; (according to them) since mutual absence is true (anyonyābhāvasya pāramārthikatvāt), to understand śruti utterances of difference and non-difference as of the nature of separation and non-separation is not reasonable. We will explain this further in the bheda-sūtras that (try to) establish difference. Thus the difference-nondifference relationship of jīva and Īśvara as part and whole is of the nature of being that of separation and non-separation. There also since non-separation alone follows both the beginning and the end, it being natural and eternal, it is the truth. Separation happens in between due to some minor delimitation; it is like some change due to some cause and is only meant for worldly usage (vibhāgastu madhye svalpāvacchedena naimittiko vikārāntavad vācāraṁbhaṇamātramiti viśeṣaḥ).

Thus the identity of the non-dual ātman has been generally explained; we will explain under the third sūtra (BS.I.1.3) sentences mentioning the non-duality of Brahman. In this manner, through difference of the nature of mutual absence it is established that Īśvara which is the meaning of the word Brahman, is absolutely different from jīva.\footnote{For Bhikṣu Īśvara is Brahman and the supreme ātman.} 

Even there, there is this special purpose: in order to separate jīva alone, Brahman (is mentioned) as the cause of the origin of the world, and its limitation, known as the power of māyā of the nature of eternal desire etc. having supremacy as being, is only an implication (jīvavyāvartanayaiva jagajjanmādikāraṇam tathā nityecchāditanmāyākhyaśak\-\break\hbox{tyau\-pādhikamaiśvaryam} copalakṣaṇameva brahmaṇaḥ).

\textbf{Ques:} But since the word ‘aiśvaryam” is included in the meaning of the word Brahman in such śruti sentences “satyam jñānamanantam brahma” (Taitt. Up. 2.1.1), “tadeva brahma…yadidamupāsate”  (Kena.\break Up. I.5-9), “sākṣī cetā kevalo nirguṇaśca” (Śvet. Up 6.11), “athāta ādeśo neti” (Bṛ.Up. 2.3.6), “akartā caitanyam cinmātram sat” (not traced) and in smṛti statements like “jñānameva param brahma…na jñānād bhidyate param” and in the upcoming sūtra: “cititanmātreṇa tadātmakatvādityaudulomi” (BS.4.4.6) one understands the meaning of the words Brahman and jīva as only consciousness. Imagining power in one who  is qualified by a limitation (and) imagining an indication (of power) in the (isolated) One (kevale) only makes it cumbersome (kevale lakṣaṇākalpanāyām gauravācca). The worldly usage of desire etc., in Brahman due to lack of insight) is reasonable (avivekena copapatteḥ, due to the relationship of master/possessor and servant/possessed (and) in this manner alone, can the words Paramātman, Parameśvara etc., be powerful having consciousness as an attribute. Thus śruti says: “vadanti tattatvavidaḥ tattvam…bhagavāniti śabdyate” (Bhā.P I.2.11).\footnote{It is interesting to note that Bhikṣu mentions this quotation from the Bhā.P as śruti statement. This is in keeping with the belief of 16 th century Bengal School of Vaiṣṇavism that the Bhā.P is a śruti. Even Madhusūdana Sarasvati accords an exalted position to the Bhā.P.} By such sentences as “anāmarūpaścinmātra”, even its being an efficient cause for the birth (of the world) is rejected.\footnote{According to Tripathi this sentence is not in the original text (Tripathi p.32 fn.2)} 

\dev{चैतन्यं चात्मनो न गुणः, किन्तु द्रव्यविशेष एव धर्मधर्मिविभागशून्यश्चेतन इति चैतन्यमिति चोच्यते चैतन्यधर्मकत्वप्रतिषेधाय । यथा तेजो द्रव्यं प्रकाशकं प्रकाश इति चोच्यते, सदा सहोपलम्भेन लाघवादेकत्वस्येव न्याय्यत्वात् । अहं जानामीति प्रत्ययस्यैव लौकिकत्वेनाधाराधेयभावांशे प्रमात्वमेव  न भ्रमत्वं, लोकानां संघातेष्वेवाहमिति भ्रमात् । संघातस्य च चित्प्रकाशधर्मकत्वमस्त्येव, यथोल्मूकस्य तेजोधर्मकत्वं घटादेर्वा छिद्रधर्मकत्वमिति । विवेकिनां त्वहं जानामीति प्रत्ययो न भवत्येव, तत्प्रत्ययस्य श्रुतिस्मृतिमायानुसारित्वात् । विवेकिनामपि तथा व्यवहारस्तु राहोः शिर इतिवद् विकल्पमात्रो लोकानुसारीति न काप्यनुपपत्तिः । अधिकं तूपदेशरत्नमालाख्यप्रकरणे द्रष्टव्यम् । एवमेव वा शास्त्रेषु सर्वज्ञत्वादिवचनं व्यवहारानुसारेणोपपन्नमिति । तथा च पातञ्जलसूत्रम्—“द्रष्टा दृशिमात्र” इति । तदेतत्तृतीयाध्याये स्वयं वक्ष्यति ‘‘प्रकाशादिवच्चावैशेष्यमि” ति सूत्रेणेति । एतेन ब्रह्मण आनन्दरूपत्वमपास्तम् “नैकस्यानन्दचिद्रूपत्वे विरोधादि” ति सांख्यसूत्रोक्तन्यायाच्च । आनन्दो हि दुःखवत् स्वगोचरवृत्तिं विनाऽपि दृश्यत्वादप्रकाशरूपः, चैतन्यन्तु धर्मिग्राहकमानेन प्रकाशरूपतयैव सिद्धमिति प्रकाशाप्रकाशरूपतयोभयोर्विरोधः चैतन्यस्य तु बुद्धिवृत्तिद्वारैव स्वविषयत्वं चैतन्यगोचरचैतन्यान्तराङ्गीकारेऽनवस्थानात्, साक्षात् स्वविषयत्वे च कर्मकर्तृविरोधादतश्चैतन्यस्य प्रकाशत्वमुपपन्नमिति । यदि वा चैतन्यवत् सुखस्यापि वृत्तिद्वारैव भानमभ्युपगम्यायं विरोधः परिह्रियते तथाऽपि एकज्ञानेऽन्याज्ञानाद् विरोधः स्यात् ज्ञातत्वाज्ञातत्वयोरेकदा विरोधात्, दुःखानुव्यवसायकाले सुखाज्ञानात् सानन्दसमाध्यादौ च चैतन्याज्ञानात् ज्ञानत्वसुखत्वरूपप्रकारभेदश्च त्वयाऽपि नेष्यते एकरसत्वश्रुतिविरोधात् । किं चैवं दुःखत्वमप्यात्मनः स्याल्लाघवादिति ।}

Consciousness is not an attribute of ātman, but it is a special substance. ‘Cetana’ is devoid of the separation as qualifier and qualified; cetane is also called caitanya (dharmadharmivibhāgaśūnyaścetana iti caitanyamiti cocyate) in order to reject (the idea) that it has the property of caitanya (caitanyadharmakatvapratiṣedhāya).\footnote{In other words cetana and caitanya are the same. Caitanya does not mean it has the quality of cetana but is a special substance and can be called cetana or caitanya.} It is like light which is a substance that has the property of illumination and is called light; since it is always accompanied (by illumination) for the sake of being non-cumbersome there is the logic of using just one word to denote the same thing. When there is the worldly thought like ‘I know’ it is correct knowledge as far as the part of support and supporter is concerned and it is not an illusion;\footnote{There is one who knows (supporter) and the knowledge (supported); this is given in worldly experience argues Bhikṣu} (however) there is illusion in the minds of people regarding the notion of ‘I’ in the collection (buddhi, ahaṅkāra,  manas etc.) The collection does have the quality of revealing consciousness (saṅghātasya ca citprakāśadharmakatvamastyeva), just as fire has the quality of heat, or like pots have the quality of holes. As for the wise the thought like ‘I know’ does not happen in accordance with the belief in māyā of śruti and smṛti. Even in the case of the wise, when there is such worldly behaviour it is a misapprehension like the saying ‘Rāhu’s head’\footnote{Probably a reference to the headless Rāhu after the head was cut off by Viṣṇu but which survived without the body because he had tasted a little bit of the nectar.} and  there is no contradiction as it follows worldly behaviour. More on this can be seen in the work Upadeśaratnamālā.\footnote{Thus the Upadeśaratnamālā has been written prior to this bhāṣya.} In a similar way the expressions omniscient etc., (sarvajñatvādivacanam) mentioned in the śāstras are appropriate in accordance with worldly usage (vyavahārānusāreṇopapannam). Thus we have Patañjali’s sūtra “draṣṭā dṛśimātra” (YS. 2.20). This will be mentioned by the author himself in the third chapter by the sūtra “prakāśādivaccāvaiśeṣyam” (BS.3.2.25).

By the above statement, Brahman being of the nature of ānanda has been rejected; this is also because of the logic given in the Sāṁkhyasūtra: “naikasyānandacidrūpatve virodhāt”. Ānanda (bliss) is like pain and is capable of being experienced without being a modification of one’s own  mind (and) is of the nature of non-illumination. Consciousness, on the other hand, is of the nature of illumination alone, having the purpose of grasping the object (dharmigrāhakamānena prakāśarūpatayaiva siddhamiti) and is established as of the nature of illumination; therefore there is a contradiction between being of the nature of having the quality of illumination and not having the quality of illumination (parkāśāprakāśarūpatayobhayorvirodhaḥ).\footnote{By its intrinsic nature caitanya grasps or knows the object and so it does not have illumination and non-illumination as qualities} It is only through the modification of the intellect (citta) that consciousness experiences its object (and so) possesses the object as its own; if one were to accept consciousness as an internal object of consciousness there will be lack of certainty (it will lead to arguing ad infinitum); also if one has one’s own self directly as an object there will the contradiction of the subject and object being the same; therefore it is correct to accept consciousness of the nature of illumination. 

\textbf{Ques:} If it is said that, like consciousness, by admitting the knowledge of pleasure also through a modification of the mind this contradiction can be removed, then the answer is: 

\textbf{Ans:} even then there will be a contradiction, as within one knowledge of an object (there will be the presence of) another unknown object, leading to the contradiction of (something) being known and something being unknown at the same time. Since at the time of apperception of pain (time of consciousness of pain) one is not conscious of pleasure, (so also)  in sānanda-samādhi (samādhi accompanied by bliss) etc., at the start, consciousness is not known.\footnote{The four types of saṁprajñāta samādhi has been explained in detail in the Yogavārttika under sūtra I.17. (See Rukmani 1981:104ff)} You (the opponent) also do not  desire difference of qualities of the nature of consciousness, pleasure etc., as it contradicts śruti admitting having only one sentiment (ekarasatvaśrutivirodhāt). Moreover in this way even pain can be admitted in ātman in the interest of parsimony of reasoning (caivam duḥkhatvamapyātmanaḥ syāllāghavāditi). 

\dev{ननु “आनन्दो ब्रह्मेति व्यजानाद् विज्ञानमानन्दं ब्रह्म, आनन्दाद्ध्येव खल्विमानि भूतानि जायन्त” इत्यादिश्रुतिबलत्वात् तर्कस्याप्रयोजकत्वं स्यादिति चेन्न, आनन्दादात्मनि भेदस्यापि … श्रवणात् तर्कस्यैवादर्तव्यत्वात् । “यस्तर्केणानुसन्धत्ते स धर्मं वेद नेतर” इति मनुवाक्येन संशयस्थले तर्कं विनाऽर्थावधारणस्य निन्दितत्वात् । भेदश्रुतयश्च “आनन्दं ब्रह्मणो विद्वान् न विभेति कुतश्चन, स एको ब्रह्मण आनन्दः, विज्ञानमयादन्योऽन्तर आत्मा आनन्दमय” इत्याद्याः किं बहुना, साक्षादेवानन्दरूपत्वप्रतिषेधोऽपि श्रूयते “नानन्दं न निरानन्दं, विद्वान् हर्षशोकौ जहाति” इत्यादिश्रुतिषु, स्मृतिषु च—}
\begin{verse}
\dev{“अदुःखमसुखं ब्रह्म भूतभव्यभवात्मकम् ।}\\
\dev{तत्सन्तु चेतस्यथवापि देहे}\\
\dev{सुखानि दुःखानि च किं ममात्र ।}\\
\dev{मनसः परिणामोऽयं सुखदुःखोपलक्षणम् ।” इत्याद्यासु ।}
\end{verse}
\dev{अत्र नानन्दमित्यानन्दरूपताप्रतिषेधः, न निरानन्दमिति चौपाधिकानन्दधर्मकत्वानुमतिः । विद्वानिति वाक्ये च यत् सुखहानं श्रूयते तदात्मनः सुखरूपत्वे सति न घटते, आत्मनो हानोपादानासम्भवात् । यद्यपि विद्वानिति वाक्ये जीवस्यैव सुखहानं गम्यते तथाप्यंशांशिनोरेकरूपत्वात् “तत्त्वमेव त्वमेव तत, इत्यादिवाक्यैर्जीवब्रह्मणोरत्यन्तमवैधर्म्य- प्रतिपादनाच्च जीवस्य सुखप्रतिषेधेन ब्रह्मण्यपि सुखप्रतिषेधोऽवगम्यते । अदुःखमसुखमित्यत्र च कर्मधारय एव, बहुव्रीहौ लक्षणाप्रसङ्गात् ।—सुखदुःखोपलक्षणः सुखदुःखधर्मक इत्यर्थः । अथवा सुखदुःखप्रभृतिरित्यर्थः ।}

\dev{एतेन आत्मन आनन्दरूपताप्रतिषेधान्मोक्षकाले सुखप्रतिपादकं वाक्यजातं दुःखनिवृत्तौ गौणं बोध्यम्, “तृष्णाक्षयसुखस्यैते नार्हतः षोडशीं कलामि” त्यादिप्रयोगदर्शनेन दु:खनिवृत्तौ सुखशब्दस्य निरूढलक्षणासिद्धेः । तथा चोतं कपिलचार्यैः “दु:खनिवृत्तेर्गौणः, विमुक्तः (क्त) प्रशंसा मन्दानामि” ति सूत्राभ्यामिति । अथवा “सुखं दुःखसुखात्यय” इति परिभाषया सुखशब्दोऽत्र न गौणः । तस्मादिच्छिादिवदेव नित्य आनन्दोऽपीश्वरे मायोपाधिक एव जीव इव बुद्ध्यौपाधिकः । आत्मनो निरुपाधिप्रियत्वं वा, सुखत्ववदात्मत्वस्यापि प्रेमप्रयोजकत्वात् दु:खनिवृत्तिरूपत्वाद् वा बोध्यम् ।}

\textbf{Ques:} But if it is said that due to the strong statements in śruti like “ānando brahmeti vyajānāt…imāni bhūtāi jāyante” (Taitt. Up. III.6.1), logic is without purpose, then the answer is: 

\textbf{Ans:} It is not so, since one hears that there is difference due to ānanda in ātman (so) one needs to take recourse to logic. Thus according to Manu: “yastarkeṇānusandhate sa dharmam veda netara”; i.e. Manu has disapproved of deciding the meaning without the help of logic when there is a doubt. So also we have differenct śruti statements like “ānandam brahmaṇo…kutaścana” (Taitt. Up. 2.4.1),  “sa eko brahmaṇa ānandaḥ” (ibid. II.8.1),  “vijñānamayādanyo…ānandamayaḥ” (ibid. II.5.1). In short, even when there is rejection of (Brahman being of the nature of) ānanda one also hears śruti statements like “nānandam na nirānandam” (Maho.Up.98), “vidvān harṣaśokau jahāti” (Kaṭh.Up. I. 2.12; in this verse the last line reads as matvā dhīro harṣaśokau jahāti)  there are also smṛti sayings like: “aduḥkhamasukham brahma…manasaḥ pariṇāmo’yam sukhaduḥkhopalakṣaṇam”. In the above, the statement “nānandam” rejects (Brahman) having ānanda, and “na nirānandam” allows having the quality of ānanda as a limitation. In the statement “vidvān harṣaśokau jahāti” etc., what one hears as the destruction of pleasure does not fit in with the ātman having the form of pleasure (as an experience), since the acquisition or abandonment (of pleasure or pain) is not possible in ātman. Even if in the statement “vidvān”etc., there is the result of the removal of pleasure of jīva alone still, since the part and whole are of the same nature, and in such statements as “tattvameva, tvameva tat” etc., since there is indication of complete non-difference    between jīva and Brahman, through the rejection of pleasure in jīva there is rejection of pleasure in Brahman as well. The expressions “asukham, aduḥkham” etc.,(in the above quote) are karmadhāraya (compounds) since, if they are accepted as bahuvrīhi (compounds), there will be the contingency (of having those qualities). The expression “sukhaduḥkhopalakṣaṇaḥ” (the quotation has sukhaduḥkhopalakṣaṇam) means having the quality of sukha, duḥkha etc (as a secondary attribute in the mind); or having sukha, duḥkha etc. 

By the above statement, since the ātman being of the nature of ānanda has been refuted, one understands that at the time of mokṣa the collection of expressions that indicate pleasure need to be understood as (just) secondary in the removal of duḥkha (sukhapratipādakam vākyajātam duḥkhanivṛttau gauṇam bodhyam). By observing expressions such as “tṛṣṇākṣayasukhasyaite nārhataḥ ṣoḍaśīm kalām” it is proven that in the removal of duḥkha (pain) the word sukha (pleasure) is not the accepted characteristic (duḥkhanivṛttau sukhaśabdasya nirūḍhalakṣaṇāsiddheḥ). Thus ācārya Kapila says in the two sūtras: “duḥkhanivṛttergauṇaḥ” and “vimukti praśamsā mandānām”. Or by the definition “sukham duḥkhasukhātyayaḥ” (that pleasure is the absence of both pain and pleasure) the word ‘sukha’ is not used in a secondary sense.\footnote{He is refuting the earlier statement of sukha being used in a gauṇa sense.} Therefore like desire etc., eternal ānanda also is a limitation of māyā in Īśvara, and it is like the limitation of the intellect in jīva. Or this can be understood as the desire of ātman to be free of limitations or, like having sukha (sukhatvavat), there is a purpose served by having desire of ātmatva (having ātman-nature), in the form of the removal of duḥkha (sorrow) (sukhatvavadātmatvasyāpi premaprayojakatvāt duḥkhanivṛttirūpatvād vā bodhyam).\footnote{The end of duḥkha (sorrow) much like the Buddhist is the main feature of mokṣa in SY. Again one sees Bhikṣu’s commitment towards SY.} 

\dev{यश्च ब्रह्मण आनन्दरूपताबोधकः शब्दः स “कामादिकं मन एवे” त्यादिश्रुतिवद् धर्म- धर्म्यभेदपरः । अथवा परमप्रियताबोधनाय रूपकमात्रः पूर्णानन्दत्वादिबोधनायाकाशशब्दवत् । आत्मनश्च परमप्रियत्वं “मा न भूवं भूयास” मित्यनन्यशेषोऽनुरागइति मन्तव्यम् । अत एवाचार्यो वक्ष्यति “आनन्दादयः प्रधानस्ये” ति कूर्मपुराणे “शिवस्य परमा शक्तिर्नित्यानन्दमयी ह्यहमि” ति प्रकृतिदेवतायाः पार्वत्या वाक्यमिति । न च प्रकृतावपि पृथक् सुखान्तरं कल्प्यं तथा सति तेनैव सुखानुभवोपपत्तौ सत्यामात्मसुखान्तरकल्पनावैयर्थ्यात् । बुद्धिवृत्तिरूपज्ञानादतिरिक्तं पुनरात्मस्वरूपं ज्ञानं कल्प्यमेव, अन्यथा बुद्धिवृत्तेः साक्ष्यत्वासम्भवादिति ।}

\dev{एतेन मोक्षस्य सुखरूपत्वमप्यपास्तम्, आत्मनश्च सुखासंबधात् उपाधिधर्माणा- मत्यन्तोच्छेदाच्च। किं चात्मनः सुखरूपत्वेऽपि मोक्षावस्थायां तस्य भोग्यत्वानुपपत्त्या पुरुषार्थत्वं न संभवति, स्वस्य साक्षात् स्वविषयत्वे कर्मकर्तृविरोधात्, उपाधेश्चात्यन्तं विलयादिति । मोक्षेऽप्यानन्दभोगवचनं तु ब्रह्मलोकाख्यगौणमुक्तिपरं बोघ्यम् ।}

The word that informs Brahman being of the nature of ānanda points to the identity of the nature of non-difference  between the characteristic and the possessor of the characteristics similar to śruti sentences “kāmādikam mana eva”. Or in order to indicate excellent pleasure it is used just as a metaphor for informing complete bliss like the word ākāśa. The supreme pleasure of ātman must be thought of as unique contentment without any residue (by the expression) “mā na bhūvam bhūyāsam” (Vyāsabhāṣya under YS.II.9).\footnote{This is mentioned as proof for the kleśa called abhiniveśa or clinging to life. The meaning is “Would that I would never cease; may I live”. Bhikṣu uses this desire for immortality as proof of the supreme pleasure of ātman which he will mention as the attribute of prakṛti reflected in ātman in the state of bondage.} That is why the ācārya says “ānandādayaḥ pradhānasya” iti. In the Kūrma P. the following is mentioned as the utterance of prakṛtidevī by Pārvatī :“śivasya paramā śaktirnityānandamayī hyaham”. Nor should pleasure be imagined separately even in prakṛti; in that case since there is the contingency of experiencing pleasure it will be useless to imagine another pleasure-experience of the ātman. Again one needs to imagine knowledge of the true nature of ātman over and above the knowledge which occurs due to the modification of the intellect (buddhivṛttirūpajñānādatiriktam punarātmasvarūpam jñānam kalpyameva); otherwise there will not be the direct perception of the modification of the intellect (anyathā buddhivṛtteḥ sākṣyatvāsambhavāditi).\footnote{Bhikṣu seems to refer to the direct experience of sākṣī or puruṣa in samprajñāta samādhi which he will mention later. This modification is not of an object outside but it is puruṣa’s own image in the modification.} 

\dev{ननु तर्ह्यात्यन्तिकमोक्षे कः पुरुषार्थः स्यात् ? दुःखनिवृत्तिरेव पुरुषार्थ स्यात् “तरति शोकमात्मवित्, अशरीरं वावसन्तं प्रियाप्रिये न स्पृशतः, विद्वान् हर्षशोकौ जहाती” स्यादिश्रुतेः । अविद्यानिवृत्त्यादीनां पुरुषार्थत्ववचनं च तद्घेतुतयैव, लोके सुखदुःखाभावयोरेव स्वत इष्यमाणत्वादिति । सुखदुःखसाक्षात्काररूपो गौणभोगोऽपि बुद्धेः संभवति, तस्या अचेतनत्वान् पुरुषकल्पनावैयर्थ्यापत्तेश्चेति । अतएव सांख्याः “पुरुषोऽस्ति भोक्तृभावादि” त्याहुः । अत एव च गीतायाम्—}

\dev{ननु दुःखाभावस्यापि पुरुषार्थत्यं न संभवति, दुःखस्यानात्मधर्मत्वेनात्मनो नित्यनिर्दुःखत्वादिति, उच्यते—सांख्यादिमते भोग्यत्वेनैव सुखदुःखयोर्हेयोपादेयत्वं स्वनिष्ठत्वेन अहं सुखं भुञ्जीयेत्यादिकामनादर्शनात् । भोगश्च साक्षात् स्यप्रतिबिम्बितयोः सुखदुःखयोः साक्षात्कारः स चात्मनिष्ठ एव । तस्य वा ज्ञानेनात्यन्तच्छेदः संभवत्येव, उपाधिविलयेन उपाधिधर्मदुःखादिप्रतिविम्बासंभवादिति ।}

\dev{ननु “बुद्धेर्भोग इवात्मनी” त्यादिवचनाद् बुद्धेरेव भोगो नात्मन इति चेन्न, भोगस्य द्वैविध्यात् । भोगो हि ‘भुजि पालनाभ्यवहारयो’रित्यनुशासनात् स्वबाह्यवस्तुना स्वपुष्टिः, यथाऽन्नेन स्थूलदेहस्य भोगः । तथा त्वसर्ववतूनां सुखदुःखमोहात्मकत्वाद् विषयगतसुखदुखाभ्यां बुद्धेः सुखदुःखयोः पोषणं बुद्धेर्भोगः, दुग्धमाधुर्यस्येव शर्करामाधुर्यण पोषणम् । अयमेव वा भोगश्चेतने प्रतिबिम्बति, तस्याविकारित्वात्त्, स एव पुरुषस्य भोगः सुखदुःखसाक्षात्काररूपो भवति । न तु सुखदुःखसाक्षात्काररूपो गौणभोगोऽपि बुद्धेः संभवति, तस्या अचेतनत्वान् पुरुषकल्पनावैयर्थ्यापत्तेश्चेति । अतएव सांख्याः “पुरुषोऽस्ति भोक्तृभावादि” त्याहुः । अत एव च गीतायाम्—}
\begin{verse}
\dev{कार्यकारणकर्तृत्वे हेतुः प्रकृतिरुच्यते ।}\\
\dev{पुरुषः सुखदुःखानां भोक्तृत्वे हेतुरुच्यते ।। इति ।}
\end{verse}
\dev{नन्वस्तु जीवस्य दुःखभोगस्तथापि तन्निवृत्तेः पुरुषार्थत्वं भवन्मते न घटते, परमार्थतो जीवस्याप्यनहत्वस्वीकारादिति मैवम्—स्वभोक्तृप्रयोजनज्ञानमेव बुद्धेः प्रवृत्तौ कारणम् । अन्यथा अस्य चेतनस्येदमुपकारकमितीदन्तादिरूपेण ज्ञानात् प्रवृत्त्यनुपपत्तेः । अतः परमार्थतोऽनात्मत्वेऽपि भाक्तृत्वानपापात् तत्तज्जीवार्थं तत्तद्बुद्वेः प्रवृत्तिर्युक्तैवेति । तस्माच्चिन्मात्रस्वरूपे एवांशांशिनी जीवब्रह्मणी इति सिद्धम् । }

\textbf{Ques:} In that case what is the puruṣārtha (gained) in endless mokṣa? 

\textbf{Ans:} It is the removal of pain itself which is the puruṣārtha due to śruti utterances like “tarati śokamātmavit” (Chānd. Up 7.1.3), “aśarīram vāvasantam…spṛśataḥ” (ibid.8.12.1), “vidvān harṣaśokau jahāti”. The mention of removal of ignorance etc., being a puruṣārtha is just to indicate its causal function (towards removal of sorrow) since (even) in the world the absence of pleasure and pain is desired naturally (loke sukhaduḥkābhāvayoreva svata iṣyamāṇatvāditi”.

\textbf{Ques:} But then, the absence of sorrow being a puruṣārtha, cannot be possible, because sorrow not being a quality belonging to ātman, ātman is forever without sorrow.

\textbf{Ans:} In the view of philosophers like the Sāṁkhyas it is only through experience that both pleasure and sorrow are destroyed or gained; one does not see the desire for pleasure and pain established in themselves like ‘I may experience pleasure’ etc (sukhaduḥkhayorheyopādeyatvam svaniṣṭhatvena aham sukham bujñīyettyādikāmanādarśanāt). Experience is the direct perception of pleasure and pain of their own reflected selves directly and that is established in the ātman. And it is possible for that to be completely destroyed by knowledge alone; by the disappearance of the limitation there is no possibility of reflection of the qualities of the limitation like sorrow etc.

\textbf{Ques:} If it said that due to sayings such as “buddherbhoga ivātmani” experience is of the intellect and not of ātman, then the answer is : 

\textbf{Ans:} Experience is of two kinds. Experience is understood through the śāstra (grammar) as “bhuji pālanābhyavahāryoḥ” as nourishing oneself through outside objects just as the experience (of eating) by the gross body (bhogo hi bhuji pālanābhyavahārayoritiyanuśāsanāt svabāhyavastunā svapuṣṭiḥ, yathā’nnena sthūladehasya bhogaḥ). Since all objects are of the nature of pleasure, pain and delusion the experience of the intellect of the pain and pleasure situated in the objects, is the nourishment of pleasure and pain in the intellect; it is like the support of the sweetness of sugar for sweetness of milk (dugdhamādhuryasyeva śarkarāmādhuryeṇa poṣaṇam). Or it is the reflection in consciousness of this experience, since it is unchanged; that is itself the experience of puruṣa in the form of direct perception of pleasure and pain. Even a secondary experience of the nature of direct perception of pleasure and pain does not happen for the intellect, due to its insentience and due to the danger of imagining puruṣa becoming useless. That is why the Sāṁkhya philosophers say “puruṣo’sti bhoktṛbhāvāt” iti. That is why the Gītā says “kāryakaraṇakartṛtve…bhokṛtve heturucyate” (Gītā.13.20).

\textbf{Ques :} Even if jīva has experience of pain still the aim in life to get rid of it does not fit in with your view (tathāpi tannivṛtteḥ puruṣārthatvam bhavanmate na ghaṭate) as in truth you accept that the jīva has no sense of ego (jīvasyāpyanahantvasvīkārāditi). 

\textbf{Ans:} No it is not so; the cause for the activity of the intellect is knowledge for the sake of one’s own (puruṣa’s) experience. Otherwise this assistance (of the intellect) for consciousness in the form of an identical nature will not be consistent with the activity which results from that knowledge (anyathāasya cetanasyedamupakārakamitīdantātirū\-peṇa jñānāt pravṛttyanupatteḥ).\footnote{The resultant activity is assumed to be based on the image in the intellect illumined by consciousness which then results in the appropriate activity according to SY. This is being used to refute the advaitin’s view that the jīva has no pain. In the world the jīva has both pain and pleasure and the aim is to get rid of it.} Thus in truth even though they are (not pure) having the nature of experience the activity of the respective intellects for their respective jīvas is appropriate. Therefore it is established that in the nature of pure consciousness (the relationship) of jīva and Brahman are that of parts and whole alone (tasmāccinmātrasvarūpe evāmśāmśino jīvabrahmaṇo iti siddham).

\dev{सोऽयं ब्रह्ममीमांसासिद्धान्तः संक्षेपेण पुनः स्मार्यते—“तत्त्वमसि अहं ब्रह्मास्मि”}
 
\dev{इत्येवंविधान्येव वेदान्तमहावाक्यानि च न जीब्रह्माभेदं बोधयन्ति । नाप्येतेषु वाक्येषु त्वमहंशब्दार्थो जीवः, किन्तु सन्धानान्तर्गतस्य षड्विंशतितत्त्वस्य मध्ये कोऽहंशब्दाथ इत्याकाक्षायां प्रवृत्ततया ‘तत्त्वमस्यादि बाक्यानि ब्रह्मादिश्य त्वमहेशब्दार्थत्वं । विदधति, को घटः? कम्बुग्रीवादिमान् घट इत्यादिवाक्यवत् । तदनन्तरमेव च शिष्यस्त्वम- हम्शब्दार्थमवधारयति, ब्रह्मैव मुख्यस्त्वमहंशब्दार्थो न पञ्चविंशतितत्त्वानीति । न पुनर्वाक्यप्रवृत्तेः पूर्वं विशिष्य त्वमहंशब्दार्थोऽवधृतः । तथा सति कोऽहमित्याकाङ्क्षानुपपत्तेः, “को न आत्मा किं ब्रह्मे” - त्यादिश्रौतजिज्ञासावाक्यानुपपत्तेश्च । उद्देश्यस्य तु ब्रह्मणो ज्ञानं पूर्वमपेक्षत एव कम्बुग्रीवादिमद्वस्तुज्ञानवदिति ।}
 
\dev{तत्र च उद्देश्यस्वरूपावधारकतया विवेकरूपाणि जीवब्रह्मवाक्यानि महावाक्यानां शेषभूतानि। अंशांश्यभेदवाक्यानि तु विधेयस्यात्मत्वस्योपपादकानि । तथा महावाक्यार्थसाक्षात्कारे सत्त्वशुद्धिद्वारकोपासनाद्वाराप्यभेदवाक्यानि शेषाणीति मन्तव्यम् । अन्यथा वेदान्तानां स्वतन्त्रनानार्थपरत्वेनार्थभेदापतेः, अभेदवाक्यानां स्वातन्त्र्यानुपपत्तेश्च । साक्षादविद्यानिवर्तकानुपपत्त्या स्वातन्त्र्येण मोक्षाख्यफलानुपयोगात् । अत एव त्वाचार्यो ब्रह्मण्यात्मत्वज्ञानमेव विद्यात्वेन सिद्धान्तयिष्यति “आत्मेति तूपयन्ति चे” ति सूत्रेण, न तु ज्ञानान्तरमिति ।}
 
\dev{सांख्यादिभ्यश्चास्य दर्शनस्यायं विशेषः, यत् श्रुत्यवान्तरवाक्यार्थो व्यावहारिकात्मा जोवस्तेषां विषयः । अस्य तु जीवाद्यात्मत्वबाधेन पारमार्थिक आत्मा विषय इति, “ नान्योऽतोऽस्ति द्रष्टे” त्यादिश्रुतिशतेभ्यः, स्वामित्वस्वतन्त्रत्वक्षेत्रज्ञत्वादिरूपस्य ब्रह्मण्येव देशकालभेदेनासंकुचितत्वाच्चेति ।}

\dev{उद्दिष्टं लक्षितञ्च ब्रह्म प्रमाणेन परीक्षणीयमित्यादौ सामान्यतः प्रमाणमुपन्यस्यते—}

That same siddhānta (doctrine) of Brahmamīmāmsā is again reiterated briefly as- “tattvamasi”, “aham brahmāsmi” (which) do not advocate identity of jīva and Brahman. Nor does“aham”,”tvam” mean ‘jīva’ in these sentences ; with the desire to know the meaning of the word ‘aham’ in such sentences as “tattvamasi” etc., within the collection of the 26 tattvas, (they) declare the meaning of ‘tvam’, ‘aham’ as having reference to Brahman. It is like the answer ‘it has a conch shaped neck’ (kambugrīvādimān) when asked the question ‘what is a pot like’. It is only then that the disciple understands the meanings of the words ‘tvam’, ‘aham’ as denoting principally Brahman and not the 25 tattvas. Before engaging with these statements he had not correctly determined in particular the meaning of the words ‘tvam’, ‘aham’ (na punarvākyapravṛtteḥ pūrvam viśiṣya tvamahamśabdārtho’vadhṛtaḥ).  In this situation the desire to know ‘who am I’ is not reasonable.\footnote{This is like placing the cart before the horse. First one needs to know the simple meaning of the word ‘aham’, ‘tvam’ etc., and only then can one go on to probe further into the meaning of ‘aham’.} So also the desire to know śruti statements like “ko na ātmā kim brahma” is also not reasonable. The knowledge of Brahman which is what is intended is first required just like knowledge of a thing (pot described) as (having) ‘a conch shaped neck’. 

In that context in order to ascertain the real nature of what is intended, mahāvākyas which distinguish the difference between jīva and Brahman are the supplement (śeṣabhūtāni). Statements mentioning the non-difference/identity of parts and whole point to the ātmatva of what is prescribed (vidheyasyātmatvasyopapādakāni). Thus one should think that when the meaning of the mahāvākyas is directly perceived (then) through devotion by the path of purification of the intellect by sattva also (sattvaśuddhidrāka upāsanādvārāpi),  what remains are statements that denote non-difference/identity.\footnote{Bhikṣu provides two options; one through the path of knowledge and the other through the path of bhakti for mokṣa.} Otherwise since the Vedāntavākyas are capable of multiple meanings, there is the danger of contradiction in meaning; so it is not correct/reasonable to interpret the identity statements independently (without reference to other statements). Since ignorance cannot be destroyed directly by itself, the result known as mokṣa is useless. That is the reason why ācārya (Badarāyaṇa) declares that vidyā (knowledge) is the knowledge of being ātman in Brahman (through the sūtra): “ātmeti tūpayanti) (BS. 4.1.3); it is not another kind of knowledge.\footnote{Jīva being in Brahman in an inseparable way (avibhāga) is the real meaning of ātmatva and not absolute identity.} 

The speciality of this philosophy as against that of the Sāṁkhyas etc., is that the topic of the others is that jīva which is participating in the world (vyavahārikātmā jīvasteṣām viṣayaḥ) is the other (extraneous) meaning of the śruti statements. The topic of this (Vedānta) philosophy is the supreme ātman through instruction of jīva etc., being of the nature of ātman. Through hundreds of śruti sentences like “nānyadato’sti draṣṭā” (Br.Up.3.8.11) (Brahman is described as) possessing power, being independent, knower of the body, and in Brahman alone is there no diminuition due to space and time.

Brahman which is intended and which has been indicated by characteristics needs to be examined through means of proof; with this idea, to begin with, he mentions proof in general as:

\section*{\dev{शास्त्रयोनित्वात् ॥३॥}}

\dev{शास्त्रं योनिर्मूलप्रमाणं यस्मिन्निति शास्त्रयोनि । तज्जगज्जन्मादिकारणं ब्रह्म शास्त्रमूलकत्वात् सिध्यतीत्यर्थः । शास्त्रं वेदान्ताः तच्छेषभूताः स्मृतयश्च, तैरत्र ब्रह्मणो विचारणीयत्वात्। तानि च शास्त्राणि “यतो वा इमानि भूतानि जायन्ते येन जातानि जीवन्ति यत् प्रयन्त्यभिसंविशन्ति तद् विजिज्ञासस्व तद्ब्रह्मे” त्याद्याः श्रुतयः, स्मृतयश्च—}
\begin{verse}
\dev{परः   पराणाम् परमः परमात्मात्मसंस्थितः ।}\\
\dev{नामरूपादिनिर्देशविशेषणविवर्जितः ॥}\\
\dev{तद्ब्रह्म परमं नित्यमजमव्ययमक्षयम् ।}\\
\dev{एकस्वरूपं च सदा हेयाभावाच्च निर्मलम् ।।}\\
\dev{स च विष्णुः परं ब्रह्म यतः सर्वमिदं जगत् ।}\\
\dev{जगच्च यो यत्र चेदं यस्मिंश्च लयमेष्यति ।।}
\end{verse}
\dev{इत्याद्याः । अत्र शास्त्रादिति वक्तव्ये शास्त्रयोनित्वादित्युक्तं शास्त्राविरुद्धानुमानादीनां ग्रहणाय । शास्त्रं हि ब्रह्मणि मूलं प्रमाणम्” तं त्यौपनिषदं पुरुषं पृच्छामि” इत्यादिश्रुतेः । औपनिषदम् उपनिषन्मूलकप्रमाणकम् उपनिषन्महातात्पर्यविषयं वा, “सर्वे वेदा यत् पदमामनन्ति, वेदैश्च सवैरहमेव वेद्यः” इति श्रुतिस्मृतिभ्याम् । जीवा  नाञ्चावान्तरतात्पर्यविषयत्वात् प्रत्यक्षानुमानादिविषयत्वाच्च नोपनिषदत्वम् । शास्त्रानन्तरञ्च तन्मूलकमनुमानं योगिप्रत्यक्षादिकमपि प्रमाणं प्रमाणं भवति “आत्मा वा अरे दृष्टव्यः” श्रोतव्यो मन्तव्यो निदिध्यासितव्य” इत्यादिश्रुतेः । एभिस्त्रिभिरेव प्रमाणैरस्माभिर्ब्रह्म परीक्षणीयमित्यर्थः । तत्र श्रुतयः स्मृतयश्च प्रायेण प्रमाणत्वेनोपन्यसनीयाः, कदाचिच्च संशयस्थलेऽनुमानमप्युपन्यसनीयम्,“उपपत्तेश्चे’ त्यादिसूत्रैः । अनुमानेनाप्यनवगम्ये च सन्दिग्धार्थे योगिप्रत्यक्षमप्याचारादिमुखेनोपन्यसनीयम् , “आचारदर्शनाच्चे” त्यादिसूत्रैरिति ।}

\section*{BS.I.1.3}

\subsection*{Śāstrayonitvāt}

That śāstra in which there is proof of the place of birth (origin) is “śāstrayoni”. That Brahman is the cause for the rise etc., of the world is established since it is the place of origin of the śāstra. Śāstra is the utterances of Vedānta, the rest are the smṛtis, since Brahman is discussed herein through them (through Vedānta statements and through smṛtis) (tairatra brahmaṇo vicāraṇīyatvāt). Those śāstras are śruti statements such as “yato vā imāni bhūtāni jāyante…tad vijijñāsasva tadbrahma” (Taiit.Up.3.1) as well as smṛti utterances like “paraḥ parāṇām paramaḥ paramātmāsamsthitaḥ…jagacca yo yatra cedam yasminśca layameṣyati” (not traced). Herein when one says ‘śāstrāt’ the meaning is, ‘that which has śāstra as its basis (origin)’. This is in order to accept (pramāṇas) such as inference that are not in contradiction to śāstra. Śāstra is the main proof for Brahman through statements such as “tam tvaupaniṣadam puruṣam pṛcchāmi” (Bṛ.Up. 3.9.26). “aupaniṣadam”= is proof which has the Upaniṣads as the basis; or it can mean the most important intended topic of the Upaniṣads; (this is known) from śruti and smṛti statements like “sarve vedā yat padamāmananti, vedaiśca sarvairahameva vedyaḥ” (Kaṭha.Up. I.2.15).

Since another intended topic/object is also the jīvas and also because it deals with objets of direct perception, inference etc., they are not the subject  of the Upaniṣads. After the śāstras, inference based on them (śāstras) as well as the direct perception of yogīs are authority for (the subject of knowing) Brahman from the saying: “ātmā vā are draṣṭavyaḥ śrotavyo mantavyo nididhyāsitavyaḥ” (Bṛ.Up. 4.5.6). It means that Brahman needs to be investigated by us only through these three proofs. Therein śruti and smṛti need to be used as a means of proof generally; sometimes, when there is a doubt, inference also needs to be used as mentioned in the sūtra “upapatteśca” etc (BS. 3.2.35). When it cannot be resolved even by inference and the meaning is (still) in doubt one can employ the direct perception of yogīs (as proof) as known from the behaviour (of knowers of Brahman) (yogipratyakṣamapyācārādimukhenopanyasanīyam) as śruti states “ācāradarśanāt” (BS.3.4.3).\footnote{Yogī-perception is placed on par with śāstra-evidence In many places. But here Bhikṣu elevates yogi-
perception to a much higher level. He seems to base it on the presence of such individuals in society.} 

\dev{तत्र ब्रह्मानुमानस्यायं प्रकार:—बुद्धिप्रभृतिकार्यम् उपादानगोचरप्रत्यक्षजन्यम् कार्यत्वाद् घटादिवदिति । अत्र चोपादानगोचरप्रत्यक्षवृत्तेरेव कार्ये हेतुत्वं लाघवात् न तु चैतन्यस्यापि गौरवात् । तदा बुद्ध्यादीनामुपादानगोचरवृत्तीच्छाकृतिमज्जन्यत्वेन कारणसत्त्वस्य सिद्ध्यनन्तरं, तत्कारणसत्त्वं परस्य भोग्यम् इच्छादिमत्त्वाज्जीवोपाधिवदिति कारणसत्त्यभोक्तृतयेश्वरोऽनुमेय इति ।}

\dev{ननु “यतो वाचो निवर्तन्ते अप्राप्य मनसा सहे” त्यादिश्रुतिभिर्ब्रह्मणः शब्दाद्यगोचरत्ववचनात् कथं ब्रह्मणः शास्त्रादिप्रमाणकत्वमुच्यत इति, उच्यते—वागादीनां लौकिकमनसश्च ब्रह्माशेषगोचरत्वं प्रतिषेधति तथाविधं शास्रजातम् न तु शब्दादीनां सामान्यगोचरत्वमपि । नापि योगजधर्मसंस्कृतमनोजन्यसाक्षात्कारगोचरत्वमपि ।}

\dev{“स यथा हि हननं प्रमुच्य प्रब्रूयादेतां दिशं गान्धारा एतां दिशं व्रजेति, स ग्रामाद् ग्रामान्तरं व्रजन् पण्डितो मेधावी गान्धारानेवोपसंपद्येतैवमेवेहाचार्यवान् पुरुषो वैदे” ति श्रुती शब्दादिमात्रेण सामान्यतो ज्ञाने जाते योगेन भूमिकाक्रमात् स्वयं मनसा ब्रह्म अशेषविशेषतः साक्षात्क्रियत इत्यवगमात् । अन्यथा “वदैश्च संर्वैरहमेव वेद्यः, मनसैवानुद्रष्टव्यमेतदप्रमेयं ध्रुवं तद् विष्णोः परमं पदं सदा पश्यन्ति सूरय” इत्यादिश्रुतिविरोधापत्तेरिति ।}

The process of inference of Brahman is as follows: the purpose\break (kāryam) of the activity of the intellect etc., arises by direct perception of the material object as that is its purpose like that of the perception of pot.\footnote{It can be stated as follows:

1) The effect is accomplished by the intellect etc., which is direct perception of the material object (pratijñâ).

2) Because that is its task (hetu). 3) like that of the perception of pot (udāharaṇa).} In this case reason (hetu) for the result (pratijñā) is the direct-perception-modification of the material object itself (upādānagocarapratyakṣavṛttereva kārye hetutvam) out of logical parsimony (lāghavāt) and not that of consciousness also (na tu caitanyasyā\-pi) due to logical cumbersomeness (gauravāt).\footnote{Contrasting the SY perception theory with that of the advaita theory.} Then since there\break arises modifications of the material objects of the intellect etc., having the images (of the objects) (buddhyādīnāmupādānagocaravṛttīncchakṛtimajjanyatvena), after accomplishing the purpose of the causal sattva, that causal sattva becomes the object of experience of another sattva by having desire etc ., due to the limitation of jīva. Thus one can infer Īśvara as the experiencer of the causal sattva.

\textbf{Ques:} But then by such śruti sayings as “yato vāco nivartante…manasā saha” (Taitt.Up.2.4.5) Brahman is mentioned as inaccessible to words; so how can it be said that  śāstra is proof for Brahman (katham brahmaṇaḥ śāstrādipramāṇakatvamucyate).

\textbf{Ans:} The answer is that, the entire śāstra prohibits Brahman being an object of perception totally without residue, as being beyond the perception of speech and the worldly mind (vāgādīnām laukikamanasaśca brahmāśeṣagocaratvam pratiṣedhati tathāvidham śāstrajātam); but\break they do not reject the perception of objects of the senses like śabda etc., (sound) in a general sense also (na tu śabdādīnām sāmānyagocaratvamapi). Nor do they (reject) objects of direct perception that arise in a mind that has been refined by the qualities that come into being due to yoga (nāpi yogajadharmasamskṛtamanojanyasākṣātkāragocaratvamapi). Through śruti sayings such as “sa yathā hi… diśam vrajeti…sa grāmād grāmāntaram vrajan…gāndhārānevopasampadyeta\break evameveha ācāryavānpuruṣo veda” having gained a general knowledge only through words, then progressing step by step through yoga (yogena bhūmikākramāt) one learns that there is direct perception totally of Brahman by the mind (yogena bhūmikākramāt svayam manasā brahma aśeṣaviśeṣataḥ sākṣātkriyata ityavagamāt). Otherwise it will be in contradiction to śruti sayings like “vedaiśca sarvaira hameva vedyaḥ” (Gītā.15.15)\footnote{Gītā, strictly speaking, is not a śruti but a smṛti text.}, “manasaivānudraṣṭavyam” (Bṛ.Up.4.4.19),\break “etadaprameyam dhruvam” (ibid 4.4.20), “tad viṣṇoḥ padam…sūrayaḥ” (Subāla Up. 6.1; Katha.Up. 3.9).

\dev{ननु शब्दादेरशेषज्ञानं क्वापि प्रसक्तमेव नास्ति, शब्दानुमानयोः सामान्यमात्रगोचरत्वात्, तत्कथं “यतो वाचो निवर्तन्त” इति ब्रह्मण्यसाधारण्येन वचनमिति चेत्, सत्यम्, तथापि पुरोवर्तिष्वनेकवस्तुषु मध्ये को घट इति जिज्ञासोरयं घट इति वृद्धवाक्यतो घटस्याशेषविशेषज्ञानं जायते । ब्रह्मणस्त्वतीन्द्रियतया तादृशमपि ज्ञानं न संभवतीति श्रुतेराशयः । यथोक्तविभागश्च श्रुत्यैव कृतः । यथा “न चक्षषा गृह्यते नापि वाचा नान्यैर्देवैस्तपसा कर्मणा वा । ज्ञानप्रसादेन विशुद्धसत्त्वस्ततस्तु तं पश्यति ।}

\dev{निष्कलं ध्यायमान” इति । देवैर्मन आदिभिरित्यर्थः, शक्तिशक्तिमतोरभेदादिति । तमिमं विभागं सूत्रकारोऽपि करिष्यति “तदव्यक्तमाह हि, अपि संराधने प्रत्यक्षानुमानाभ्यामि” ति सूत्रद्वयेन । तदेवं श्रतिस्मृतिन्यायसिद्धविभागेन श्रुत्योरन्योन्यमविरोधे सति वृत्तिव्याप्यत्वफलव्याप्यात्वभेदेन स्वकल्पितेन यदाधुनिकानामविरोधकल्पनं तदनुचितमिति प्रतीमः ।}

\dev{ननु निर्विशेषचिन्मात्रे ब्रह्माणि कथं योगिमनोग्राह्यो विशेष इत्युच्यत इति चेत् न, सर्वसंकरापत्त्या चेतनान्तरप्रधानतद्गुणादिष्ववृत्तस्वरूपविशेषस्य योगिगम्यस्यावश्यमभ्युपेयत्वात्, अन्ततः स्वस्वोपाधिप्रतिबिम्बानामेवातीतादिसाधारणानां चेतनेषु विशेषत्वाच्च । तैश्चावस्तुभूतैर्न निर्विशेषत्वक्षतिः । निर्विशेषवाक्यानां जीवविलक्षणगुणादिरूपविशेषप्रतिषेधकत्वाद् देशकालावस्थादिनिमित्तकविशेषप्रतिषेधकत्वाद् वा ।}

\dev{नन्वस्तु विशेषस्तथापि स विशेषः शब्दागोचरोऽतीन्द्रियश्च कथं योगिमनसाऽपि गृह्यतेति चेन्न, योगजधर्माणां शास्त्रसिद्धाचिन्त्यशक्तित्वात् धर्मान्तरवत् । कथमन्यथा “भुवनज्ञानं सूर्ये संयमात्” इति योगसूत्रमुपपद्यते । ननु संयोगादिप्रत्यासत्त्यभावात् कथं व्यवहितादिज्ञानं योगिनां स्यादिति चेन्न, मनोवैभवमते सर्वत्रैव मनःसंयोगसत्त्वात् योगजधर्मस्य च वृत्तिप्रतिबन्धक- तमोमात्रनाशकत्वात्। मनःपरिच्छिन्नतामते च योगजधर्मस्यैव प्रत्यासत्तित्वकल्पनादिति।}

\textbf{Ques:}  But then, no where is there complete knowledge connected only through words, as words and inference (through them) only have general features (of the knowledge of ) objects.Therefore how can (the words) “yato vāco nivartanta” denote the specific characteristic of Brahman. Then the answer is: 

\textbf{Ans:} True, even then in the midst of many objects in front of one when one has the desire to know which one is a pot, one gets the total specific knowledge of a pot from the words ‘this is a pot’ of a reliable wise person (vṛddhavākyataḥ). Śruti’s idea is that since Brahman is beyond the perception of the senses one cannot have that kind of a knowledge. Śruti itself has made the above mentioned separation/distinction. Thus there is the statement:  “na cakṣuṣā gṛhyate..tam paśyati niṣkalam dhyāyamāna” (Muṇḍ.Up. 3.1.8). That means by devas through the minds etc., as there is non-difference between power and one possessing power (devairmana ādibhirityarthaḥ, śaktiśaktimatorabhedāditi)\footnote{Bhikṣu seems to suggest that all these sense-organs being devas themselves they are incapable of having knowledge of themselves as that would be kartṛ-karma-virodha i.e subject and object being the same.}. The sūtrakāra (Bādarāyaṇa) also will mention this difference through the two sūtras “tadavyaktamāha hi” (BS.III.2.23) and “api samrādhane pratyakṣānumānābhyām” (ibid. III.2.24). In this manner when through division based on logical reasoning established by both śruti and smṛti, when there is no conflict between the statements of śruti, the self imagined division of the twofold modification as ‘vṛttivyāpyatva’ and ‘phalavyāpyatva’ of the modern day Vedāntins (advaitins) to avoid an imagined contradiction, appears to be incorrect.\footnote{The advaitins draw a distinction between the two one referring to ordinary modification of the mind when faced with worldly objects and the other failing to do so when the object itself is Brahman which is the result.}

\textbf{Ques:}  But when Brahman is pure consciousness without any characteristics how can the yogī’s mind grasp any specific characteristic? Then the answer is: 

\textbf{Ans:} it is not so; due to the contingency of the mixing up of all (vṛttis) there is need for the yogī’s knowledge (for (knowing) the nature of that specific intrinsic quality within consciousness amongst pradhāna and its qualities which is a non-modification (of the mind) (sarvasamkarāpattyā cetanāntarapradhānatadguṇādiṣvavṛttasvarūpaviśeṣasya yogi\-gamyasyāvaśyamabhyupeyatvāt); in short it is only the general features of past reflections of the respective limitations (of the individual selves) that are objects in consciousness. Being without any specific characteristic, there is no harm done through those which are without objects, (taiśca avastubhūtairna nirviśeṣatvakṣatiḥ). As for non-qualified sentences, due to the rejection of characteristics such as guṇa etc., different from that of the jīva or due to rejection of qualities caused by space, time or state of being (there is no harm done).

\textbf{Ques:}  Let there be this speciality; even then how can that special quality which cannot be understood by sound (of its name) and is beyond the senses like sound etc., be grasped by the mind of a yogī? Then the answer is: 

\textbf{Ans:} It is not so; the qualities that arise from yoga have powers that cannot be thought of but are attested by the śāstras (śāstrasiddhācintyaśaktimatvāt) and are of a different kind of dharma (quality). Otherwise how can one explain the Yogasūtra “bhuvanjñānam sūrye samyamāt” (YS.3.26).\footnote{Patañjali sates in this sūtra that the of knowledge of all worlds will come to one who practises samyama on sūrya (sun). This is considered a siddhi (power) which comes to one who practises yoga diligently.} 

\textbf{Ques:} If it is said how can yogīs have knowledge of things that are faraway (separated) when there is no close proximity (with the object) through contact etc., then the answer is: 

\textbf{Ans:} It is not so; where there is belief in the power of the mind, then the mind has contact everywhere (with everything) as the power born of yoga has the capacity to destroy exclusively tamas which is an obstruction to the modification of the mind (yogajadharmasya ca vṛttipratibandhakatamomātranāśakatvāt). Even in those who believe in the limited nature of the mind (like the Naiyāyikas) it can be imagined that the power born of yoga itself has the capacity to get close (to the object).

\dev{आधुनिकास्तु “यतो वाचो निवर्तन्त” इत्यादिवाक्यैरनुभवाख्यफलव्याप्यत्वं प्रतिषिध्यते। मनसैवानुद्रष्टव्यमित्यादिभिश्चाज्ञाननाशाय ब्रह्मणि वृत्तिव्याप्यत्वं स्वीक्रियत इत्यविरोध इत्याहुः ।}

\dev{तन्न , एतादृशकल्पनायां प्रमाणादर्शनात् । किं च सव्यवहारत्वावच्छेदेनैवानुभवविषयकत्वमेव प्रयोजकं, बुद्धिवृत्तिगोचरव्यवहारेऽनवस्थाभयेन वृत्तिविषयत्वस्याप्रयोजकत्वात् । घटादिभानार्थं चेतने तत्संबन्धायैव वृत्तिसिद्धेः न तु घटादिव्यवहारार्थमपि । तस्या अपि बाह्यव्यवहारा जनकत्वकल्पनेन च गौरवात्, जडत्वेन घटाविशेषाच्च । तथा च ब्रह्माणोऽनुभवव्याप्यत्वानङ्गीकारे विदुषां ब्रह्मणि व्यवहारो न स्यादिति ।}

\dev{यत्तु स्वप्रकाशत्वादेव ब्रह्मणि व्यवहारो भवति तद्विषयकज्ञानस्य तद्व्यवहारहेतुत्वकल्पने गौरवात् । वृत्तिव्याप्यत्वन्तु व्यवहारप्रतिबन्धकाज्ञाननिवृत्त्यर्थमपेक्ष्यत इत्यच्यते, तदप्यसारम्, समानविषयकत्वप्रत्यासत्त्या ज्ञानत्वेनैव व्यवहारत्वावच्छिन्नं प्रति कारणत्वात् । दृष्टं विहायात्मव्यवहारे अदृष्टकल्पनानौचित्याच्च । अज्ञानाख्यतमसो व्यवहारप्रतिबन्धकल्पने प्रमाणाभावाच्च, ज्ञानाभावादेव व्यवहाराभावसंभवात् ।सुषुप्त्यादौ ज्ञानानुत्पत्त्यर्थं तमसो वृत्तिप्रतिबन्धकत्वस्यावश्यकल्प्यतया पुनर्व्यवहारप्रति- बन्धकत्वकल्पनावैयर्थ्याच्चेति ।}

\dev{स्यादेतत्, ब्रह्मणः स्वप्रकाशत्वान्यथानुपपत्त्या कर्मकर्तृविरोधेन च ब्रह्मणोऽनुभवरूपस्य व्यवहारे स्वविषयानुभवो न हेतुः संभवतीति, अत्रोच्यते,  अज्ञेयत्वरूपं तावत् स्वप्रकाशत्वमस्माभिर्नाभ्युपगम्यते । नापि जीवब्रह्मणोरखण्डत्वं येन जीवस्य ब्रह्मज्ञाने कर्मकर्तृविरोधः स्यात् । अखण्डैकात्म्यमतेऽपि सांख्ययोगोक्तया जीवस्य स्वसाक्षात्कारप्रक्रिययैव ब्रह्मसाक्षात्कारेऽपि नास्ति कर्मकर्तृविरोधः, स्वशास्त्रे ज्ञानोत्पत्तिप्रक्रियाया अभावेन तयोः प्रक्रियैव ग्राह्याऽन्तरङ्गत्वात् ।}

Modern day (Vedāntins) reject the pervasion of the result of experience in such statements as “yato vāco nivartante”. Whereas by sayings such as “mansaivānudraṣṭavyam” (Bṛ.Up.4.4.19) they accept the pervasion of the modification (of the mind) in Brahman for the sake of removal of ignorance (and) declare that as non-contradicted. That is not so; in such imagination one sees no proof. Moreover its purpose is to have an object of experience only through defining its (worldly) usage (kim ca vyavahāratvāvacchedenaivānubhavaviṣayatvameva prayojakatvam); afraid of the lack of finitude in the activity pertaining to having the modification of the mind/buddhi as object (they consider) there is no use for having he   modification (of the mind) as an object. For the sake of knowing the pot there is modification (of the mind as pot) for the sake of connection (of that) in consciousness (cetane); it is not for the sake of usage of the pot as well (na tu ghaṭādivyavahārārtha\-mapi)\footnote{One cannot have a modification of the mind regarding the usage of the pot as well.}. It is also cumbersome to imagine that it will not give rise to outside activity by being inanimate and being non- different from a pot. Thus by not accepting the pervasion of experience in Brahman (they maintain) there is no worldly experience of the wise pertaining to Brahman.\footnote{The reason for not accepting phalavyāpyatva by the ādhunika advaitins according to Bhikṣu.} 

That there is experience in Brahman is due to its being self-luminous; thus imagining the knowledge regarding the object being a reason for its worldly activity is superfluous. It is said that the modification being pervaded by consciousness   is for the sake of removal of ignorance which is an obstacle; that is also without substance. Due to the proximity of a similar object, by its knowledge alone, it becomes the cause for that which is limited by possessing activity (samānaviṣayakatvaptyāsattyā jñānatvenaiva vyavahāratvāvacchinnam prati kāraṇatvāt).\footnote{Bhikṣu’s refutation and answer to the advaitin.} In the case of the experience of ātman it is not proper to abandon what is visible and imagine what is not visible. There is also no proof for imagining tamas (tāmasa guṇa) as an impediment in activity; it is only in the absence of knowledge is there the absence of activity. Just because in the state of suṣupti (dreamless sleep) where there is no rise of knowledge there is the necessity of imagining tamas as an obstacle for the modification (of the mind); to imagine (tamas) being an obstacle in worldly activity is useless.

\textbf{Ques:} Let it be; as there is absence of reasonable grounds (for denying) the self luminosity of Brahman and due to the contradiction between the subject and object being the same (karmakartṛvirodhena) in the experience of the activity of the direct experience of Brahman as object in the world it is not possible that experience of oneself can(not) be the reason. Then the answer is: 

\textbf{Ans:} The state of not knowing is because we cannot access the nature of self-luminosity. Nor can we (access) the wholeness/oneness of jīva and Brahman by which there will be the contradiction of subject and object being the same;  (nāpi jīvabrahmaṇorakhaṇḍatvam yena jīvasya brahmajñāne karmakartṛvirodhaḥ syāt). Even for those who believe in the identity of the two ātmans (jīvātman and paramātman), following the Sāṁkhya-Yoga (SY) idea that  knowing Brahman only through the action of witnessing directly one’s own self, there is no contradiction of the subject and object being the same; in this śāstra as there is the absence of the manner in which knowledge arises, one needs to accept their method alone as it is internal (within oneself); therefore the way in which knowledge arises itself is the object of perception, as it is internal (svaśāstre jñānotpattiprakriyāyā abhāvena tayoḥ prakriyaiva grāhyā’ntaraṅgatvāt).\footnote{Again drawing attention to the way in which knowledge arises in advaita and contrasting that to that of SY.}

\dev{सांख्ययोगयोश्चेत्थं ज्ञानोत्पत्तिप्रक्रिया— विषेयेन्द्रियसंयोगादिना वस्त्रमालिन्यस्येव बौद्धरामोद्रव्यस्यापसारणे सति यथा निर्मलवस्त्रे कुसुम्भसंयोगात्तु कुसुम्भाकारता भवति, तथैव निर्मले बुद्धिसत्त्वे विषयसंयोगाद् विषयाकारता भवति । सा बुद्ध्यवस्था विषयाकारा बुद्धिवृत्तिरित्युच्यते । सा तु स्वप्नध्यानादौ बाह्यविषयाभावकाले विषयतुल्या बुद्ध्यवस्थाऽन्तरेवोपलभ्यमाना केनाप्यपलपितुं न शक्यते । सा वृत्तिः प्रमाणम् “प्रमाणविपर्ययविकल्पनिद्रास्मृतयः” इति योगसूत्रेण वृत्तिविभजनात् ।}

\dev{वृत्तेश्च प्रमाणत्वमनुभवाख्यफलायोग्यवच्छिन्नत्वात्, रथारूढस्य ग्रामसम्बन्धे रथस्येव चेतनस्य विषयसंबन्धे द्वारत्वाच्च । द्रष्टा दृशिमात्रः शुद्धोऽपि प्रत्ययानुपश्य” इति योगसूत्रात् विभुत्वेऽप्यात्मनो नित्यचैतन्याविकारिणो मलिनबुद्धिप्रतिबिम्बेन प्रतिबन्धाद् विशुद्धबुद्धिद्वारतां विना विषयान्तरप्रतिबिम्बधारणाऽसामर्थ्यात् । अतएवात्मावरणार्थं चित्तविभुत्वमभ्युपगतं पतञ्जलिना। तथा च श्रुतिः “सुषुप्तिकाले सकले विलीने तमोऽभिभूतःसुखरूपमेती” त्यादिः  । सा च वृत्तिर्जायमानेव चेतने प्रतिबिम्बिता भवति “वृत्तिसारूप्यमितरत्रे” ति योगसूत्रात् । ततश्च स्वप्रतिबिम्बितां विषयोपरक्तां तां बुद्धिवृत्तिं स एव प्रतिबिम्बाधिष्ठानचेतनोऽवभासयति, यथा स्वप्रतिबिम्बितकुसुम्भरक्तवस्त्रं स्फाटिकभित्तिः “चितेरप्रतिसंक्रमायास्तदाकारापत्तौ स्वबुद्धिसंवेदनमि” ति योगसूत्रात् । तदाकारापत्तौ बुद्धिसमाकारापत्तौ सत्याम् अप्रतिसंक्रमाया बुद्धावसञ्चारिण्या असङ्गाया3 अपि चितेः स्वीयोपाधिबुद्धिसंवेदनं भवतीत्यर्थः । तदुक्तम्— }
\begin{verse}
\dev{तस्मिंश्चिद्दर्पणे स्फारे समस्ता बस्तुदृष्टयः ।}\\
\dev{इमास्ताः प्रतिबिम्बन्ति सरसीव तटद्रुमाः ।। इति ।}
\end{verse}
\dev{तथा च चेतने प्रतिबिम्बनादेव बुद्धिवृत्तितद्विषययोर्भानं मरुमरीचिकाध्यस्तजलस्येव मरीचिना प्रकाशनमिति । न ह्यन्यथा विभोश्चित्स्वरूपस्यात्मनः सदा सर्वसंयोगेन कादाचित्कं परिच्छिन्नं च विषयसंवेदनं केनाप्युपपादयितुमञ्जसा शक्यते । अस्मदुक्तप्रकारे तु बुद्धिवृत्तेः कादाचित्कत्वेनाप्रतिबन्धात् विषयप्रतिबिम्बस्य कादाचित्कत्वं परिच्छिन्नत्वं चोपपद्यते। उपाधीनां च यावदात्ममाबिम्बिततया तत्प्रतिबिम्बस्यानादित्वमिति । यच्च बुद्धौ चेतनस्य प्रतिबिम्बनं तद् बुद्धावहमितिप्रत्ययस्य विषयः । तथा सत्यात्मनि बुद्धिवृत्त्यारोपे च प्रयोजकं दर्पणादौ प्रतिबिम्बद्वारैव मुखादिषु दर्पणमालिन्याद्यारोपदर्शनात् । सत्त्वोपाधौ चेतनस्य प्रतिबिम्बो न बुद्धितद्विषयभाने प्रयोजकः  ।}

\dev{यथा च सन्निकर्षसाम्येऽपि चक्षूरूपमेव गृह्णाति न रसं तथैव सर्वसंयुक्तोऽप्यात्मा स्वबुद्धिवृत्तिमेव साक्षाद्गृह्णाति नेतरदिति फलबलात् क्लृप्तम् । अर्थाकारतयैव चार्थग्रहणं, सा च कूटस्थे प्रतिबिम्बान्नान्यद्युक्ता। अतो बुद्धिवृत्तेरेव चेतने साक्षात् प्रतिबिम्बनसामर्थ्यं नेतरस्येति । तस्मादर्थोपरक्तवृत्तिस्फुरणार्थं चेतने तत्प्रतिबिम्बः सिध्यति ।}

The manner in which knowledge arises in SY is as follows: through to contact of the sense organ and the object, when there is removal of the substance of the tamas guṇa in the intellect like that of dirt in a cloth, then just like through contact of the clean cloth with safflower the cloth takes on the colour of safflower, so also in the pure sattva, intellect due to contact with the object there is the shape of the object. That state of the intellect of the shape of the object is known as the modification of the intellect/mind.\footnote{Unlike advaita SY is not particular in making a distinction between the intellect and the mind. Words like budhhi, manas, citta are all used intercheageably.} That, when there is the absence of an outside object like during  the dream state and in the state of meditation the state of the intellect equal to the object which is present internally, cannot be denied by anyone. That modification is a proof of knowledge by the classification of modification by the YS as: “pramāṇaviparyayavikalpanidrāsmṛtayaḥ” (YS.I.6). The proof of the modification (of the mind) specified as connection with what is known as experience, is like the connection of a chariot to the village when the person is sitting in the chariot; that is the path of consciousness with reference to connection with the object.\footnote{In other words consciousness need not be an experiencer as well according to SY.} 

By the sūtra “draṣṭā dṛśimātraḥ śuddho’pi pratyayānupaśyati” (YS.\break II.20), even though the immutable ātman of the nature of eternal consciousness is all-pervading, being obstructed through the obstacle of the reflection of the tainted buddhi (intellect), and not having access through the untainted intellect, it is not capable of supporting the reflection of another object (vibhutve’pyātmano nityacaitanyāvikāriṇo malinabuddhipratibimbena pratibandhād viśuddhabuddhidvāratām\break vinā viṣayāntarapratibimbadhāraṇā’sāmarthyāt). The reason why\break Patañjali has supported the all-pervasive character of the mind (citta) is for the sake of covering the ātman (ata evātmāvaraṇārtham cittavibhutvamabhyupagatam patañjalinā). Thus there is the śruti “suṣuptikāle sakale vilīne tamo’bhibhūtaḥsukharūpameti” (Kaiv.Up.i.13).\break While arising itself that modification (of the mind) gets reflected in the consciousness by the YS statement “vṛttisārūpyamitaratra” (YS. I.4). Then consciousness which is the support of that reflection, which is the self-reflected modification coloured by the object, causes the knowledge of that modification,\footnote{Bhikṣu believes in a double reflection of firstly the reflection of the image of the object in caitanya and then secondly caitanya itself through its action transforming that as knowledge. This has been dealt with in detail in the Yovārttika. (Also see my paper on “Bhikṣu’s double reflection theory in the Journal of Indian Philosophy, Reidal Holland, October, 1988.)} as the self reflected cloth coloured by safflower in the crystal surface. 

This is understood from the YS “citerapratisamkramāyāstadākārāpattau svabuddhisamvedanam” (YS.4.22). “tadākārāpattau”= when it\break (consciousness) attains the same shape as the modification of the intellect, “apratisamkramāyāḥ”= consciousness which does not move\break within the intellect and thus has no contact (with it), has knowledge of the intellect which is its limitation (buddhāvasaṅcāriṇyā asaṅgāyā api citeḥ svīyopādhibuddhisamvedanam bhavatītyarthaḥ). Thus it is said: “tasminściddarpaṇe sphāre…sarasīva taṭadrumāḥ” (Annapūrṇa. Up 4.7). Thus in consciousness it is only through reflection that there is the modification of the intellect and knowledge of the object (tathā ca cetane pratibimbanādeva buddhivṛttitadviṣayayorbhānam); it is like the desert making a beam of light appear as water superimposed on a beam of light (marumarīcikādhyastajalasyeva marīcinā prakāśanamiti). In no other way is it possible for anyone to understand truly the limited knowledge of an object occasionally of the ātman which is all pervading, of the intrinsic nature of (pure) consciousness, which is at all times in contact with all things. 

However, following the path mentioned by us, through absence occasionally of obstacles of the modification of the intellect, the limitation occasionally of reflection of the object can be understood.\footnote{This is the SY theory of the attenuation of both rajas and tamas in the act of knowledge of the intellect which is then capable of knowing the object through the sattva mind/intellect.} When the limitations are reflected by the prototype (bimba) ātman, then those reflections are without beginning. That reflection of the ātman in buddhi is the object of the thought as ‘I’ in the intellect.\footnote{Again a reiteration of the double reflection theory of Bhikṣu.}  Thus, when there is the superimposition of the modification of the intellect in the ātman, the purpose served is like (what happens in mirrors);  in the mirrors it is only through reflection that one sees the superimposition of the dirt of the mirror on one’s face. In the presence of the limitation of sattva, the reflection of consciousness does not serve the purpose of knowing the intellect and its object.

Just as, in the proximity of the object, the eyes grasp only the colour and not the taste, so also even though connected to everything the ātman only directly perceives the modification of its own intellect and not anything else; this is decided from the result (of knowledge of the object). It is only by assuming the shape of the object that there is the perception of the object; and that cannot be other than the reflection in the immutable (ātman). Therefore it is only the modification of the intellect that has the capacity to reflect itself directly in consciousness and not any other. Therefore it is established that in order to activate the modification coloured by the object (that has the shape of the object) there is its reflection in consciousness  (tasmādarthoparaktavṛttisphuraṇārtham cetane tatpratibimbaḥ siddhyati).

\dev{ननु वृत्तिरेच चैतन्यस्य विषयताख्यः सम्बन्धोऽस्तु लाघवात्, किमर्थं चेतने प्रतिबिम्बः कल्प्यत इति चेन्न, अर्थसारूप्यस्यैव ग्रहणरूपतया बुद्धिस्थले सिद्धत्वाच्च । किं चोक्तैर्योगसूत्रादिभिः प्रतिबिम्बसिद्धौ केवलतर्का अप्रतिष्ठादोषपराहता एव, सन्दिग्धार्थे एव न्यायप्रवृत्तेरिति । तदेवं चैतन्ये वृत्ति प्रतिबिम्बःसिद्धः । यच्च बुद्धौ चेतनस्य प्रतिबिम्बनं तद् बुद्धेश्चिच्छायापत्तिरित्युच्यते । तच्च बुद्धेर्घटाद्याकारवच्चैतन्याकारपरिणामः । स च चैतन्यभानार्थं कल्प्यते, साक्षात् स्वदर्शने कर्मकर्तृविरोधात् । तदेव चात्मनि बुद्धिवृत्त्यारोपे च प्रयोजकं, दर्पणादौ मुखादिप्रतिबिम्बद्वारैव मुखादिषु दर्पणमालिन्याद्यारोपदर्शनात् । उपाधौ चेतनस्य प्रतिबिम्बस्तद्विषययोर्भासक इति केचित्   । तत्र प्रतिबिम्बस्य तुच्छतया भासकत्वायोगात्, आरोपितवह्नेर्दहनपचनादिकार्यादर्शनात्। यश्च सूर्यादिप्रतिबिम्बेऽपि प्रकाशो दृश्यते स किरणोपाधिकः ।}

\dev{न च बुद्धिगतप्रतिबिम्ब एव चेतनविषययोः सम्बन्धविधया भानप्रयोजकोऽस्त्विति वाच्यम्, जलगतप्रतिबिम्बत्वादीनां सूर्यतत्प्रकाशयोः संबन्धत्वादर्शनात् । प्रकाशके प्रकाश्यप्रतिबिम्बस्य सम्बन्धविधया भानहेतुत्वं तु मरीच्युदकादौ । अतः संशयस्थले अणोरपि विशेषस्याध्यवसायकरत्वात् दृष्टानुसारेण चेतनगतप्रतिबिम्ब एव भानहेतुरिति । जीवानां च सर्वेषामेव विभुत्वेऽपि स्वस्वोपाधिभिरेव परस्परप्रतिबिम्ब-}

\dev{बिम्ब इत्यत्रानादित्वान्न काप्यनुपपत्तिरिति । तदेतद् बुद्धिपुरुषयोः परस्परप्रतिबिम्बनं वेदान्तिभिः परस्पराध्यास इत्युच्यते ।}

\textbf{Ques:} Then let there be connection with the modification known as of the nature of the object in the interest of parsimony (of logic); why do you imagine reflection in consciousness? Then the answer is: 

\textbf{Ans:} It is not so; it is (already) established that it is only the similarity of the object in the intellect (with itself) that is capable of being known. Moreover, when through the many yogasūtras that have been mentioned, the reflection is established, just arguing (about it) is only attacking defects which have not been proven; logic is  employed (as is well kown) only when there is ground for doubt. Thus the modification (of the intellect) in consciousness is established. The reflection of consciousness in the intellect is known as reflection of ātman in the intellect (cicchāyāpattirityucyate). That is having the change of the form of consciousness like the intellect having the (change of the) shape of a pot etc. That is  imagined for the sake of the illumination of consciousness/ātman, as in knowing oneself by oneself directly there will be contradiction in the subject and the object being the same (karmakartṛvirodhāt). That itself is the purpose of the superimposition of the modification of the intellect in consciousness; we have seen that it is only through the reflection of one’s face in the mirror etc., that there is the superimposition of the dirt in the mirror on the faces. Some say that in the limitation there is  the reflection of both consciousness and the object. Therein because of the insignificance of the reflection it is not proper for it to be the illuminator; one does not see such effects as burning or cooking of an adventitious fire. The light seen even in the reflection of the sun is a limitation of the sun’s rays.

\textbf{Ques:} Nor can it be said that the reflection pertaining to the intellect through being connected to consciousness and object is useful for their knowledge; 

\textbf{Ans:} We do not feel in the reflection (of the sun) in the water, (connection of) the sun and its light. The reflection of the illuminated in the illuminator through a connection is the cause for knowledge (as seen in the case) of water and the beam of light. Therefore when there is a doubt regarding even the smallest object, it is only the reflection in consciousness that is the cause of knowledge as seen (ataḥ samśayasthale aṇorapi viśeṣasyādhyavasāyakaratvāt dṛṣṭānusāreṇa cetanagatapratibimba eva bhānaheturiti). Even though all jīvas are all-pervading still, it is through their respective limitations that there is mutual reflection; therefore, in this context there is no contradiction, even when being beginningless. This mutual reflection between the intellect and puruṣa is spoken of as mutual superimposition by Vedāntins.

\dev{यत्तु अनयैव रीत्या चैतन्येऽध्यस्ततया जगतः प्रकृतिपर्यन्तस्य सिद्धत्वाच्चिदेवैकं}

\dev{तत्त्वम्, अन्यत् सर्वं दृश्यजातं मरीच्युदकादिवत्तुच्छमित्याधुनिका विवर्तवादिनोऽभ्युपगच्छन्ति, तन्न, तथापि बिम्वरूपाणां विषयाणामेव प्रतिबिम्वरूपेणैव भानात् प्रतिबिम्बरूपतया चैतन्ये बाधेऽपि सर्वतो बाधाभावादिति । तदुक्तं सांख्ये “सदसद्बाधाबाधाभ्याम्” इति ।}

\dev{एतेनात्मस्वरूपमपि करतलामलकवत् प्रतिपादितम् । तथाहि—यदेतद् घटपटाद्याकारैरवभासमानं चैतन्यम् एतदेवाकारविवेकेनात्मतत्त्वं भवति स्वप्रतिबिम्बित- मुखादिविवेकेन दर्पणतत्त्वमिव “तदा द्रष्टुः स्वरूपेऽवस्थानमि” ति योगसूत्रात् । तदा वृत्तिनिरोधकाले स्वरूपावस्थितिश्चेतनस्येत्यर्थः । तथा चोक्तम्—}
\begin{verse}
\dev{अनाप्ताखिलशैलादिप्रतिबिम्बे हि यादृशी ।}\\
\dev{स्याद्दर्पणे दर्पणजा केवलात्मस्वरूपिणी ।।}\\
\dev{जगत्त्वमहमित्यादौ प्रशान्ते दृश्यसंभ्रमे ।}\\
\dev{स्यात्तादृशी केवलता स्थिते द्रष्टर्यवीक्षणे ।। इति ।}
\end{verse}
\dev{दृश्यसम्भ्रमे दृश्यादरे दृश्याकारबुद्धिवृत्ताविति यावत् । यच्चैतत् पुरुषे विषयोपरक्तं बुद्धिस्फुरणमुक्तम्, एतदेव प्रमाणस्य फलं प्रमेति चोच्यते । पुरुषार्थं हि कारणानां प्रवृत्तिरिति । तदेतदुक्तं पातञ्जलभाष्ये व्यासपादैः—“फलमविशिष्टः पौरुषेयश्चित्तवृत्तिबोध” इति । अविशिष्टो वृत्त्या सह समानाकारो वृत्तेः प्रकाशः पुरुषनिष्ठः फलमित्यर्थम् । इदं च फलं सांख्यादिभिश्चैतन्य- प्राधान्येनोच्यत इति कृत्वा तैर्ज्ञानस्वरूप एवात्मोच्यते । वैशेषिकादिभिस्तु ज्ञानस्योत्पत्तिविनाशप्रत्ययानुरोधात् यथोक्तप्रतिबिम्बप्राधान्येनैव ज्ञानाख्यं फलमुक्त्वा तस्यात्मधर्मत्वमुच्यते “अहं जानामीति” प्रत्ययानुरोधात्, अस्मदुक्तफलं चानुव्यवसाय इति तैरुच्यते इति विशेषो मन्तव्यः ।}

\eject

\textbf{Ques:} Thus in this manner, through being superimposed on consciousness, when everything in the world upto prakṛti is established, consciousness alone is the single truth (tattvam). All the others are insignificant as they are born due to being perceived just like water from a beam of light (in the desert); thus the modern day Vedāntins who believe in the theory of vivarta (vivartavādinaḥ) understand (everything).

\vskip .2cm

\textbf{Ans:} That is not so; even then, since one knows the prototype of objects only through the nature of (their) reflections as such, even if they are contradicted in consciousness, in themselves they are free of contradiction (tathāpi bimbarūpāṇām viṣayāṇāmeva pratibimbarūpe\-ṇaivabhānāt pratibimbarūpatayā caitanye bādhe’pi svato bādhābhāvā\-diti). Thus it is said in Sāṁkhya “sadasadabādhābādhābhyām”.

\vskip .2cm

Through the above, the intrinsic nature of the ātman has been shown like a gooseberry (placed) on the palm of the hand.\footnote{A simile used to indicate the clarity of the object.} That consciousness shining in the form of the pot, cloth etc., (ghaṭapaṭādyākāraiḥ) is itself understood to be the ātmatattva due to knowing the difference of the modifications of the mind (from the ātman); it is like knowing the mirror due to knowing the difference of the reflection of one’s own face (from the mirror). Thus the YS says: “tadā draṣṭuḥ svarūpe’vasthānam” (YS.I.3).\footnote{Instead of following any Vedānta model of knowledge suggested in the Upaniṣads Bhikṣu follows the SY epistemology in toto.} “tadā”=at the time of the restraint of the modifications of the mind, consciousness abides in itself.

\vskip .2cm

Thus it is said: “anāptākhilaśailādipratibimbe hi yādṛśī…jagattvamaha\-mityādau praśānte dṛśyasaṃbhrame syāttādṛśī kevalatā sthite draṣṭa\-ryavīkṣaṇe”. “dṛśyasambhramē” (in the above quote) means in the modification of the mind in the shape of the object. And this which is connected with the object (of perception) in puruṣa is known as the stimulant in the buddhi, (and) that is itself the result of the proof (of perception of the object) and is known as knowledge. And the activity of the sense organs is for the sake of achieving the goal of puruṣa (puruṣārtham hi karaṇānām pravṛttiriti). Thus it is said in the commentary by Vyāsa on the Pātañjala YS: “phalamaviśiṣṭaḥ pauruṣeyaścittavṛttibodhaḥ” (under YS.I.7).\footnote{The result of perception is the knowledge of the modification of the mind as belonging to puruṣa which is identical (with the modification of the mind) (cf.Rukmani 1981, p.60)} Assuming a similar form with the general (unqualified) modification (of the mind) (and) the illumination of that modification situated in puruṣa is the result (known as knowledge). This result is mentioned mainly as in the form of consciousness by the Sānkhya philosophers, therefore it is characterized as ātman of the essence of knowledge (tairjñānasvarūpa evātmocyate). Vaiśeṣika philosophers on the other hand, in accordance with their theory of the rise and fall of knowledge, having mentioned the result as knowledge mainly in accordance with the reflection theory, mention it as a quality of the ātman in accordance with the (meaning of the statement) “aham jānāmi”. What we call the result of knowledge is mentioned as certainty (after thought, anuvyavasāya) by them; this should be known as the difference.

\dev{या चेयं ज्ञानोत्पत्तिप्रक्रिया बहिर्विषयेष्ववधारिता एषैवात्मन्यपि प्रत्येतव्या,}

\dev{एकरूपकल्पनायाः सर्वथौचित्यात् । तथाहि, शास्त्रादिना योगजधर्मेण वा बुद्धेस्तमोऽभिभवे सति निर्मलायां बुद्धावात्माकारा वृत्तिर्जायते “अहं ब्रह्मास्मी” त्यादिरूपा । सा प्रमाणम् । सा च वृत्तिर्जायमानैव चैतने प्रतिबिम्बिता सती भासते । तदेतत् पुरुषनिष्ठं फलमात्मज्ञानमुच्यते । घटादिज्ञानाच्चात्मज्ञानस्यायं विशेषो यदात्मतत्त्वाकारवृत्त्या अभिमानो निवर्त्यते, घटाद्याकारबुद्धिवृत्तिभिस्तु मिथ्याज्ञानं न नियमेन निवर्त्यत इति । चेतनप्रतिबिम्बितबुद्धिवृत्तिव्याप्यत्वमेव बाह्याभ्यन्तरसाधारण ज्ञेयत्वं बोध्यम् ।}

\dev{तत्रैवं ज्ञानप्रक्रियायां कथमात्मज्ञाने कर्मकर्तृविरोधः स्यात् एकस्मिन्नपि यथोक्तज्ञातृज्ञेयभावस्य संभवात् । साक्षाच्चैतन्यव्याप्यत्वस्य च घटादावप्यनङ्गीकारात्,}

\dev{तस्याप्यात्मनि घटादाविव प्रकारभेदेनाविरोधाच्च । यथा सूर्यो बिम्वरूपेण प्रकाशकः प्रतिबिम्वरूपेण च प्रकाश्य इति न कर्मकर्तृविरोधः। एवमात्मापि बिम्बरूपेन ज्ञाता भवति स्वगतस्वप्रतिबिम्बरूपेण च ज्ञेय इति प्रकारभेदान्न कर्मकर्तृविरोध इति ।}

This process of acquiring knowledge is dependent on outside objects; this is to be similarly understood with reference to the ātman as well as it is proper always to imagine (happenings) to be similar (eṣaivātmani pratyetavyā, ekarūpakalpanāyāḥ sarvathaucityāt). Thus, through śāstra or through the quality born of yoga, when the tamas-guṇa of the intellect is overpowered then, in the taintless intellect, there is the modification in the form of the ātman of the nature like “aham brahmāsmi”. That is a proof/means of proof. That modification as it arises shines as reflected in consciousness. That is this result situated in puruṣa mentioned as knowledge of ātman. This is the difference between ātman-knowledge and knowledge of pots etc., i.e. due to the modification (of the mind) into the form of the essence of ātman, the sense of pride/agency disappears (ātmatattvākāra vṛttyā abhimāno nivartyate); through modifications of the mind such as a pot etc., ignorance (false knowledge) is not removed as a rule. Being pervaded in general, by the modification of the mind generally, reflected in consciousness of both internal and outside (objects), which is itself understood as the object known (cetanpratibimbitabuddhivṛttivyāpyatvameva bāhyābhyantarasādhāraṇam jeyatvam bodhyam). 

Thus in this process of acquiring knowledge how can there be any contradiction between the subject and object being the same in the knowledge of the self;  even in one (support) it is possible to have the said state of knower and known (ekasminnapi yathoktajñātṛjñeyabhāvasya sambhavāt). There is no acceptance of being directly pervaded by consciousness even in (the modifications of) the pot etc., similarly, with regards to the ātman also, just as with regards to pots etc., there is no contradiction because it is of a different nature (tasyāpyātmani ghaṭādāviva prakārabhedenāvirodhācca). There is no contradiction between the subject and object being the same, just as (in the case of) the sun being the illuminator as the prototype and   the one illuminated as the reflection. In the same way ātman as the protoptype is the knower and in the form of the reflection in oneself is the known; thus their being a difference in nature there is no contradiction.

\dev{कर्मकर्तृविरोधस्तु नास्तिकमतेष्वेव पातञ्जलसूत्रेणोपन्यस्तो “न तत्स्वाभासंदृश्यत्वादि” ति। तस्य सूत्रस्यायमर्थः—तच्चित्तं न स्वाभासं न स्वगोचरवृत्तिं विना स्वग्राह्यं दृश्यत्वात् शब्दादिवदिति । अतो न पुरुषे व्यभिचारः। विपक्षबाधकश्चात्र प्रकारभेदं विना साक्षात् स्वव्याप्यत्वे कर्मकर्तृविरोध इति। यदि चित्तस्यापि स्वज्ञानार्थं करणान्तरकल्पनया विरोधः परिह्रियते तदाऽऽत्मन एव चित्तनामकरणादस्मन्मतानुप्रवेश इति पतञ्जलेराशयः । अत एव यत् प्राचीनाः दृश्यत्वेनात्मत्यं वदन्ति तत्रापीत्थमेव प्रयोगः कर्तव्यः—चित्तं नात्मा करणं विना दृश्यत्वात् इच्छादिवदिति । अथवा घटादयः सर्वे नात्मानो लौकिकप्रमितिविषयत्वात्। तस्मादात्मनः फलव्याप्यत्वेऽपि न कर्मकर्तृविरोध इति सिद्धम् ।}

\dev{यच्च परैरुक्तम् ; आत्मनो ज्ञेयत्वे स्वप्रकाशत्वहानिरिति, तदपि तुच्छम्, आत्मसामान्येऽन्तःकरणादिव्यावृत्त्यर्थमनौपाधिकप्रकाशत्वात् । }
\begin{verse}
\dev{यथा प्रदीपः पुरतः प्रदीप्तः प्रकाशमन्यस्य करोति दीप्यन् ।}\\
\dev{तथेन्द्रियं चेन्द्रियदीपवृक्षाः ज्ञानप्रदीप्ताः परवन्त एव ।।}
\end{verse}
\dev{इति मोक्षधर्मादिष्ववगमात् । ब्रह्मणश्चासाधारण्येन स्वप्रकाशत्वमनन्याधीनप्रकाशफलोपधानकत्वम् “न तत्र सूर्यो भाति न चन्द्रतारके नेमा विद्युतो भान्ति कुतोऽयमग्निः । तमेव भान्तमनुभानि सर्वं तस्य भासा सर्वमिदं विभाती” त्यादिश्रुतिषु तथावगमादिति । ननु सुर्याद्यात्मनो}  $^{1}$\dev{दृष्टान्तदार्ष्टान्तिकयोरेकस्य प्रकाशलक्षणस्याभावात् कथं दृष्टान्तः शास्त्रेषूपपद्यत इति चेत्, उभयत्रैव प्रकाशरूपतानुभवात्  उभयसाधारणं प्रकाशत्वमखण्डोपाधिरेकोऽभ्युपगन्तव्य इति दृष्टान्तोपपत्तिः । परे तु तमोनिवर्तकत्वमेवोभयसाधारणं प्रकाशकत्वमिति । तत्र चैतन्ये तमोनिवर्तकत्वाभावस्य वक्ष्यमाणत्वात् बाह्याभ्यन्तरोभयसाधारणतमस्त्वस्यभावाच्च । चाक्षुषवृत्तिप्रतिबन्धकत्वादेर्दू- रत्वादिदोषेष्वतिप्रसङ्गादिति ।}

According to the YS of Patañjali, it is only in the view of the nāstikas (those who do not believe in the vedas) that there is contradiction between the subject and the object being the same.  “na tatsvābhāsam dṛśyatvāt” (YS.4.19). The meaning of that sūtra is as follows: That consciousness is not self-illuminating, as without its own modification of the mind into the object (of perception) it cannot perceive it like (perceiving) sound etc., since it is an object of perception (taccittam na svābhāsam na svagocaravṛttim vinā svagrāhyam dṛśyatvāt śabdādivaditi). Therefore there is no violation (of the rule) in the case of puruṣa.   Obstruction in the opponent’s view is because without any change in the nature (of the object) when one is pervaded directly (by oneself) there is a contradiction in the subject and the object being the same. The idea of Patañjali is:  If for the sake of the mind knowing itself one can avoid the contradiction by imagining another cause, then by calling the ātman itself as citta one can agree with our idea (yadi cittasyāpi svajñānārtham karaṇāntarakalpanayā virodhaḥ parihriyate tadā’’tmana eva cittanāmakaraṇādasmanmatānupraveśa iti patañjalerāsayaḥ). That is why when the old school of Naiyayikas mention the ātman as an object of perception there also the following logic needs to be employed: Ātman cannot be an object of perception without the mind being the cause, just as desire etc. (cittam icchādivaditi).  Otherwise all (objects) such as pots etc., being objects of worldly knowledge will not belong to ātman. Therefore it is established that even when there is pervasion of the result of ātman there is no contradiction between the subject and object being the same.

That which is said by others that in ātman being known, there is a defect of self-illumination is also absurd. Ātman being common (in all knowledge) shines without the help of any limitation on objects which are reverted by the internal organ (citta) etc (ātmasāmānye’ntakaraṇā\-divyāvṛttyarthamanaupādhikaprakāśyatvāt). Thus it is known from the Mokṣa.P: “yathā pradīpaḥ purataḥ pradīptaḥ…jñānadīptāḥ paravanta eva”. Due to Brahman being unique it is self-illuminating i.e it is capable of achieving the result of illuminating itself without dependence on anything else. One knows this from śruti sayings like: “na tatra sūryo bhāti…tasya bhāsāsarvamidam vibhāti” (Kaṭha.Up. 2.2.15). 

\textbf{Ques:} In case it is said that using the example of the sun for the ātman being both the   example and the object of the  example, when there is an absence of the characteristic of illumination, how can this example be accepted in the śāstras.\footnote{The sun does not have the capacity of self illumination (svaprakāśa); so how can it be compared to ātman.} Then the answer is: 

\textbf{Ans:} Since one experiences the nature of illumination commonly in both places, one needs to accept that there is a single complete limitation of the nature of illumination. Thus the example is appropriate.

In the others’ view the removal of tamas alone is common to both which is (called) illumination (knowledge). Therein, since it will be mentioned that there is the absence of the removal of tamas in ātman, tamas common to both the internal and external (objects)\footnote{The whole discussion is based on how knowledge comes into being in SY as opposed to advaita Vedānta. In advaita ignorance (avidyā/tamas) of the object being removed in contact with ātman knowledge of the object occurs. In SY the reflection of the ātman/puruṣa in the intellect with the modification, will illuminate the object as well and lead to knowledge.} is also absent with reference to it. In the case of defects such as distance with reference to the modification of the mind connected with the eye that is stretching it too far.

\dev{नन्वात्मनो ज्ञेयत्वे “विज्ञातारमरे केन विजानीयादिति” श्रुतिविरोध इात चेन्न, अस्य वाक्यस्य “केन प्रकाशान्तरेण विज्ञातारं लोकं प्रकाशयेदि” त्यर्थ इति पातञ्जलभाष्ये व्यासपादैरर्थत उक्तत्वात् । तथाहि, “परार्थात् स्वार्थसंयमात् पुरुषज्ञानमि” ति सूत्रस्य भाष्ये पुरुष एव तु तं प्रत्ययं स्वात्मालम्बनं पश्यति न तु पुरुषप्रत्ययेन बुद्धिसत्त्वात्मना पुरुषो दृश्यते “विज्ञातारमरे केन विजानीयादि” ति श्रुतेरितीति।}

\dev{आधुनिकास्तु स्वकल्पितमात्मारोपत्वं सिद्धान्तीकृत्य स्वप्रकाशलक्षणं रचयन्ति। “अवेद्यत्वे सत्यपरोक्षव्यवहारयोग्यत्वमि” ति । तत्रेदमुच्यते—एवं स्वप्रकाशत्वमामनः श्रुत्यादिष्वदर्शनादसिद्ध- मिदं लक्षणम्। स्वप्रकाशशब्दस्यास्मिन्नर्थ यौगिकत्वमपि न संभवति, स्वज्ञेयत्वस्यैव योगतो लाभात्। अपि च वेदनस्य सर्वव्यवहारकारणतया वेद्यत्वाभावे व्यवहारासंभवादप्यसिद्ध लक्षणम् । अस्माभिर्वेदनस्योपपादितत्वेन सर्वशास्त्रविरुद्धकल्पनासहस्रस्यानौचित्यात्, “य एवं वेद, सदा पश्यन्ति सूरयः इत्यादिश्रुतीनां यथाश्रुतार्थपरित्यागानौचित्याच्च। किं चावेद्यत्वे सति तत्र श्रुत्यादीनां प्रामाण्यं न स्यात्, पुरुषगतप्रमाफलाजनकत्वादिति ।}

\dev{स्यादेतद्, ब्रह्माकारवृत्तेरनुभवाख्यफलाभावेऽपि पुरुषनिष्ठाया अघियाया निवृत्तिरेव पुरुषनिष्ठं फलं स्यादिति । अत्रोच्यते—प्रमाकरणस्यैव प्रमाणशब्दार्थतया अविद्यानिवृत्तिरूपफलेन प्रमाणत्वासंभवः । किं चेयमविद्या ? किं भ्रमरूपा वृत्तिः, किं वा भ्रमकारणवासना, उत आवरकं तमोद्रव्यं चक्षुषस्तिमिरादिवत्, अथवा बौद्धतमः-प्रतिबिम्बः, उताविद्याशब्देन भवद्भिः परिभाषितं जगत्कारणमिति ।}

\textbf{Ques:} But then is it is said that in knowing the ātman, by the śruti statement “vijñātāramare kena vijānīyāt” (Bṛ.Up 2.4.14; 4.5.15) there is a contradiction (of subject and object being the same), then the answer is:

\textbf{Ans:} that is not so; the meaning of this statement is as stated by Vyāsa in his Pātañjalabhāṣya (as follows): “through which other light will the world be known to the knower”. Thus the bhāṣya under the sūtra “parārthāt svārthasaṁyamāt puruṣajñānam”\footnote{YS.III.35. The entire sūtra reads “sattva-puruṣayor-atyanta-asamkīrṇayoḥ pratyaya - aviśeṣo bhogaḥ parārthatvāt svārtha-samyamāt puruṣa-jňānam.”} says that it is the puruṣa alone that sees/knows that thought which is the support of its own self; it is not through the thought of puruṣa    of the nature of the sattva intellect that puruṣa is seen (sūtrasya bhāṣye puruṣa eva tu tam praytayam svātmālambanam paśyati na tu puruṣapratyayena buddhisattvātmanā puruṣo dṛśyate). That is the meaning of the śṛuti saying “vijñātāramare kena vijānīyāt”.

Modern day (Vedāntins) on the other hand, basing it on their imagined superimposition of the ātman explain the nature of self-luminosity as: “avedyatve satyaparokṣavyavahārayogyatvam”. In that context (we) say that this characteristic of self-luminosity of the ātman is not proved as it is not seen mentioned in śruti. It is also not possible to understand the meaning of self-luminosity in this sense even etymologically (yaugikatvamapi na sambhavati), as knowing one’s own self is itself achieved through yoga. Also since knowledge is the cause for all actions, when there is absence of a known object there is no action possible; so the characteristic (of self luminosity) is not established.   Since thousands of such imaginary statements are inappropriate as they are opposed to all śāstras, we, by accepting knowledge, consider it inappropriate to abandon the meaning understood from such śruti utterances as “ya evam veda” (Br. Up. 5.7.1 and in many other Upaniṣads as well), “sadā paśyanti sūrayaḥ”. Moreover when there is nothing to be known then śruti sayings will have no authority, since knowledge associated with humans will not be capable of yielding any result (kim cāvedyatve sati tatra śrutyādīnām prāmāṇyam na syāt, puruṣagatapramāphalājanakatvāditi).

\textbf{Ques:} Let it be so; even if the modification of the mind into that of Brahman does not have the result called experience, let the removal of ignorance situated in humans be the human result (puruṣaniṣṭhāyā avidyāyā nivṛttireva puruṣaniṣṭham phalam syāditi). Then it is said: 

\textbf{Ans:} Since giving rise to knowledge itself is the meaning of the word ‘pramāṇa’ (means of knowledge) it is not possible to be a pramāṇa by the result of removal of ignorance     (pramākaraṇasyaiva pramāśabdārthatayā avidyānivṛttirūpaphalena pramāṇatvāsambhavaḥ). (Also) what is this avidyā? Is it a modification of the nature of an illusion, or is it an impression having illusion as a cause, or yet again is it a covering of the substance tamas like the timira disease of the eyes, or is it the reflection of the tamas quality of the intellect? Or is the word avidyā as defined by you the cause of the world?

\dev{नाद्यः, तस्या बुद्धिधर्मत्वेन तन्नाशस्यापि बुद्धिधर्मतया पुरुषनिष्ठत्वाभावात् । न द्वितीयः, [“एते प्रधानस्य गुणास्त्रयः स्युरनपायिनः” इत्यादिवाक्यैः सत्त्वादिगुणत्रयस्यानात्मनिष्ठत्वेन तन्निवृत्तेरप्यनात्मनिष्ठत्वात्, तैमिरिकादिवत् तमोरूपस्य दोषस्य करणनिष्ठत्वाभावाच्च। नापि तृतीयः, तन्निवृत्तेः पूर्वोत्पन्नफलत्वासम्भवात् । वृत्तिप्रतिबन्धकतमोद्रव्यनिरासानन्तरमेव बुद्धिवृत्तेरुदयेनात्मनि बौद्धतमः प्रतिबिम्बस्यापि पूर्वमेव निवृत्तः, न हि बिम्बापगमे प्रतिबिम्बं स्थातुमर्हति ।}

\dev{एतेनावरणभङ्गो विद्यायाः फलमित्यपास्तम्। दोषनिवृत्त्यनन्तरमेव विद्योदयात्, नयनादिरूपकरणानामेव लोके आवरणदोषदर्शनेन बुद्धेरेवावरणदोषाच्च । एतेन चाविद्याया आत्मनिष्ठत्वमप्युपास्तम् । विद्याविद्ययोः सामानाधिकरण्यानौचित्याच्च । तथा चिन्मात्रत्वश्रुतिविरोधात् “कामादिकं सर्वं मन एवे”त्यादिश्रुतिविरोधाच्च । तथा “दुःखाज्ञानमला धर्माः प्रकृतेस्ते तु नात्मनः” इत्यादिस्मृतिविरोधाच्च । अन्यथा दुःखादीनामप्यात्मधर्मत्वापत्तेश्चेति । “आत्मन्यविद्ये” त्यादि श्रुतश्च कामादिवदौपाधिकाज्ञानपरा इति ।}

It is not the first (modification of the form of an illusion), since it is the quality of the intellect\footnote{He gives the reason as to why avidyā cannot be the first option i.e a modification of the nature of delusion, as it belongs to the intellect.}; so in the event of its destruction also, being a quality of the intellect, it has the absence of being established in puruṣa/ātman.\footnote{Therefore the advaitin’s contention that knowledge arises on the destruction of avidyā does not hold water.} It is also not the second (i.e. an impression,   which is the cause for the illusion). Through such statements as : “ete pradhānasya guṇāstrayaḥ syuranapāyinaḥ”, since the three guṇas sattva etc.,  being situated in the anātman, since their destruction  also is due to  their being situated in the non-self, it is unlike the timira disease situated in the eye . The defect in the form of tamas (in the three constituents of sattva, rajas and tamas) is like darkness being absent in the cause. It s also not the third (a covering by the tamas-substance like the disease timira covering the eye), since when it is removed it is not possible to have connection with the result already arisen earlier (nāpi tṛtīyaḥ, tannivṛtteḥ pūrvotpannaphalatvāsambhavāt). It is only after the removal of the tamas-substance, which is the obstacle for the modification, that the modification of the intellect arises; thus in the ātman also the substance-tamas of the intellect is removed before the reflection; when the prototype has disappeared the reflection has no reason to stay (vṛttipratibandhakatamodravyanirāsanānantarameva buddhivṛtterudayenātmani bauddhatamaḥ pratibimba\-syāpi pūrvameva nivṛtteḥ, na hi bimbāpagame pratibimbam sthātumarhati).  

By the above (reasoning) (the statement that) vidyā comes into being as a result of the destruction of the covering (of avidyā) is rejected. Since it is only after the removal of the defect that vidyā (knowledge) arises (and) since it is seen in the world that there are defects like obstructions only in the instrumental sense-organs such as eyes etc., (so) the defect of the covering is of the intellect alone. By this the theory of avidyā residing in ātman is also rejected. It is also incorrect to treat vidyā and avidyā as of the same type. It contradicts the śruti statement (that ātman is) only consciousness (and) “kāmādikam sarvam mana eva” (not traced). It is also in contradiction to such smṛti statements as “duḥkhājñānamalā dharmāḥ prakṛteste tu nātmanaḥ). Moreover there will be the undesirable result of ātman possessing dharmas (qualities) like sorrow etc. Śruti statements such as “ātmanyavidyā” point to the limitation of avidyā (ignorance) like desire etc (kāmādivadaupādhikājñānaparā iti).

\dev{नापि चतुर्थः, ज्ञानस्य जगत्कारणनाशकत्वे लौकिकनिदर्शनाभावात् श्रवणाभावाच्च । न वा तत्सम्भवति, एकपुरुषस्यात्मज्ञानेन जगन्नाशेऽन्यपुरुषाणां तद्दर्शनानुपपत्तेः; तस्यैव पुरुषस्य जीवन्मुक्तिदुशायां प्रपञ्चदर्शनानुपपत्तेश्च, कारणनाशे कार्यास्थानानौचित्यात् । जीवन्मुक्तस्याप्यविद्यालेशस्वीकारे ज्ञानाज्ञानयोर्नाश्यनाशकभावादिकल्पनायां व्यभिचारात् दोषात् । योऽपि जगत्कारणेऽविद्याशब्दः क्वचित् श्रूयते सोऽप्यविद्याकारणत्वेनैव, असङ्गे कूटस्थे निर्गुणे चात्मनि भ्रमवत् तद्वासनाया अप्यनुपपत्तेः, असङ्गे तिमिरादितुल्यावरकासंभवात् । किं चावरणं वृत्तिप्रतिबिम्बरूपमेव चक्षुरादीनां तिमिरमेघादिभिर्दृष्टं तच्च कूटस्थनित्यस्य चेतनस्य न घटते । न च चैतन्यसंबन्धकमेवावरकत्वं वाच्यं, तस्यापि सम्बन्धस्य नित्यत्वे प्रतिबन्धासंभवात्, कार्यत्वे च लाघवात् बुद्धिवृत्तेस्तत्कारणत्वस्यैवौचित्यात् “तद्धेतोरेव तदस्त्विति” न्यायात्, न तु चैतन्यार्थयोः सम्बन्धे तमः प्रतिबन्धकं कल्पयित्वा तमोनिवृत्तौ बुद्धिवृत्तेर्हेतुत्वं कल्पनीयम्, गौरवात् । किं च सुषुप्तिसंमोहवार्धकादौ वृत्तिप्रतिबन्धकतया बुद्धाववश्यं तम आवरण कल्प्यम्—}
\begin{verse}
\dev{“सत्त्वाज्जागरणं विद्याद्रजसा स्वप्नमादिशेत् ।}\\
\dev{प्रस्वापनं तु तमसा तुरीयं त्रिषु सन्ततम् ।।}
\end{verse}
\dev{इत्यादिमृतेः । चक्षुषस्तिमिरादिवत्तमोरूपस्य दोषस्य करणनिष्ठत्वानुमानाच्च । अत उभयत्र तमःकल्पनावैयर्थ्याद् बुद्धावेव तमो नात्मनीति ।}

\dev{नापि चतुर्थः, तन्निवृत्तेः पूर्वोत्पन्नत्वेन फलत्वासंभवात् । वृत्तिप्रतिबन्धकतमो द्रव्यनिरासानन्तरमेव बुद्धिवृत्तेरुदयेना\-त्मनिबौद्धतमःप्रतिबिम्बस्यापि वृतेः पूर्वमेव निवृत्तेः । न हि बिम्बापगमे प्रतिबिम्बं स्थातुमर्हति । नापि वृत्त्यतिरिक्तस्यात्मनि प्रतिबिम्बं स्थातुमर्हति । नापि वृत्त्यतिरिक्तस्यात्मनि प्रतिबिम्बो भवति, संस्कारादेरपि प्रतिबिम्बापत्त्या साक्षिभास्यतापत्तेः । तमोऽपि च घटादिवद् बुद्धिवृत्त्यारूढतयैव चेतने भासते न तु साक्षिभास्यम् । अन्यथा प्रलयेऽपि तमोऽनुभवापत्त्या तत्स्मरणापत्तेरिति । एतेनावरणान्तरभङ्गो विद्यायाः फलमित्यपास्तम्, आवरणदोषनिवृत्त्यनन्तरमेव विद्योदयात्।}
\begin{verse}
\dev{आत्मा नित्योऽक्षरोऽव्यक्त एकः क्षेत्रज्ञ आश्रयः ।}\\
\dev{अविक्रियः स्वदृग्घेतुर्व्यापकोऽसङग्यनावृतः ।।}
\end{verse}
\dev{इत्यादिवाक्यैरात्मनोऽनावृतत्वश्रवणाच्च । नयनादिरूपकरणानामेव लोके आवरणदोषदर्शनेन बुद्धेरेवावरणदोषाच्च । एतेनाविद्याया आत्मनिष्ठत्वमप्यपास्तम्, विद्याविद्ययोः सामानाधिकरण्यानौचित्याच्च । तथा चिन्मात्रत्वश्रुतिविरोधात्, “कामादिकं सर्वं मन एवे” त्यादिश्रुतिविरोधाच्च । तथा “दुःखाज्ञानमला धर्मा:प्रकृतेस्ते तु नात्मन” इत्यादिस्मृतिविरोधाच्च, अन्यथा दुःखादीनामप्यात्मधर्मत्वापत्तेश्चेति । आत्मन्यविद्याश्रुतयश्च कामादिवदौपाधिकाज्ञानपरा इति।}

Neither (is avidyā) the fourth. When the cause for knowledge of the world is destroyed, as there is absence of basis for worldly activity, there is also absence of (the sense of activity) like hearing etc.\footnote{If avidyā is defined as ignorance as in advaita Vedanta, then this will be the result when ignorance Is destroyed by knowledge which adaita advocates for mokṣa. In SY the senses all come into existence from the tāmasa guṇa of the antaḥkaraṇa; thus when that is destroyed there is no way to sense the existence of the world nor will the senses like hearing etc., be in existence.} Nor does that happen; if the world is destroyed through one person’s realization of the ātman it fails to explain the seeing of the world by others (na vā tatsambhavati, ekapuruṣasyātmajñānena jagannāśe anyapuruṣāṇam taddarśanānupapatteḥ); it also fails to explain the seeing/experiencingof the world by the same person in the stage of jīvanmukti (liberation while embodied itself); when the cause is destroyed the presence of the effect is not (logically) correct. If one accepts a touch of avidyā in the case of a jīvanmukta, even then, there is the defect of inconsistency by imagining the relationship between knowledge and ignorance as that between the destroyer and the destroyed\footnote{Since a jīvanmukta is one who has knowledge of Brahman such a one cannot have any ignorance.} When sometimes the word ‘avidyā’ (ignorance) is heard as the cause of the world that is also due to ignorance alone (so’pyavidyākāraṇatvenaiva). In the nirguṇa, (without qualities), immutable, non-attached ātman it is also unreasonable to assume avidyā and the impression of it (avidyā) (in something which is unattached); it is also not possible to have a covering similar to the eye disease (of the eye) (asaṅge timirāditulyāvarakāsambhavāt). 

Moreover a covering is an obstacle, in the form of a reflection of a modification (of the mind) by/of the eyes etc., seen through the cloud of darkness; that also cannot be logical in the case of the eternal, immutable consciousness. Nor can one say that a covering has only a relationship with consciousness since that relationship also being eternal there is no possibility of its being an obstacle. When there is an effect it is proper (to accept) a modification of the mind as being its cause by the reasoning “taddhetoreva tadastu” because it is logically concise (kāryatve ca lāghavāt buddhivṛttestatkāraṇatvasyaivaucityāt “taddhetoreva tadastu” iti nyāyāt). It is cumbersome to imagine an obstacle of darkness in the relationship between consciousness and its object and then imagine the cause for the removal of that darkness as a modification of the mind/intellect. Morover in instances of deep sleep, fainting, old age etc., one needs to necessarily imagine the obstacle of tamas of the modification of the mind in the intellect. Smṛti also says: “sattvājjāgaraṇam vidyādrajasā svapnamādiśet…turīyam triṣu santatam”.  Just like the inference of the eye disease of the eye (in the eye) (timira) one infers the defect as situated in the instrument of action (cakṣuṣastimirādiva tamorūpasya doṣasya karaṇaniṣṭhatvānumānācca). Thus since it is useless to imagine the presence of  darkness (tamas) in both places, the defect is only in the intellect and not in the ātman (atah ubhayatra tamaḥkalpanāvaiyarthyād buddhāveva tamo nātmanīti).

It is neither the fourth\footnote{Reflection of the tamas quality in the intellect. Some of the reasons stated above are reiterated here as well} as when it (avidyā) is destroyed, having come into being earlier it is not possible for the result to materialize. Since it is only after the destruction of the tamas-substance which is an obstacle to the rise of modification, that the modification of the intellect takes place; (thus) in the case of the reflection, the tamas belonging to the intellect is also destroyed earlier to the modification. It is not possible for the reflection to exist after the disappearance of the prototype.\footnote{He repeats the same arguments ditto as in the third option} Nor is it proper for a reflection apart from that of the modification to exist in the ātman (nāpi vṛttyatiriktasyātmani pratibimbam sthātumarhati). Nor is there a reflection other than that of the modification (of the intellect) in ātman; since there is the danger of having reflections of mental impressions etc., (saṁskārādeḥ) there is the undesirable consequence of its being reflected by the witness. Even tamas (a constituent of sattva) like pot etc., shines in consciousness through being superimposed as a modification of the intellect and it is not reflected (directly) by the witness; otherwise even in dissolution (pralaye) there will be the danger of experiencing tamas and the unreasonable consequence of its remembrance.

By the above argument the statement that vidyā has the result of destroying the internal covering (of ignorance) is rejected. It is only after the removal of the defect of the covering (of ignorance) that knowledge arises. One also hears of the state of the ātman being without any covering in such statements as: “ātmā nityo’kṣaro’vyakta…vyāpako’saṅgyanāvṛtaḥ” (not traced). In the world one witnesses defects of covering of only instruments of action (karaṇānāmeva) like the eyes etc., (so) also the defect of covering is only of the intellect (nayanādirūpakaranānāmeva loke āvaraṇadoṣadarśanena (uddherevāvaraṇadoṣācca). For this reason avidyā being situated in ātman is rejected. Moreover it is incongruous to have the same location for both vidyā and avidyā. Thus it is in contradiction to the śruti statement “kāmādikam sarvam mana eva”; it also contradicts smṛti statements such as “duḥkhājñānamalā dharmaḥprakṛteste tu nātmanaḥ”; otherwise there will be the undesirable consequence of ātman having such vices as sorrow etc. Śruti statements that mention avidyā in ātman are meant to be limitations of avidyā similar to desire etc.

\dev{नापि पञ्चमः, ज्ञानस्य जगत्कारणनाशकत्वे प्रमाणाभावात्, लोकसिद्धेन मिथ्याज्ञाननिवृत्तिद्वारेणैव ज्ञानान्मोक्षसिद्धौ जगत्कारणनाशाख्यदृष्टद्वारकल्पनागौरवाच्च । न वा तत्संभवति, एकपुरुषस्यात्मज्ञानेन जगन्नाशेऽन्यपुरुषाणां तदर्शनानुपपत्तेः, तस्यैव पुरुषस्य जीवन्मुक्तिदशायां प्रपञ्चदर्शनानुपपत्तेश्च, कारणनाशे कार्यावस्थानानौचित्यात् । जीवन्मुक्तस्याप्यविद्यालेशस्वीकारे ज्ञानाज्ञानयोर्नाश्यनाशकभावादिकल्पनायां व्यभिचारादिदोषाच्च । परममुक्त्यनन्तरमपि तदुपभुक्तपृथिव्यादीनां पुरुषान्तरभोग्यतायाः श्रुतिस्मृतिप्रत्यभिज्ञाभिः सिद्धत्वाच्च पुरुषभेदेनाविद्याख्यकारणभेदात् प्रतिपुरुषं तत्कार्यजगद्भेदेनानन्तजगत्कल्पनागौरवाच्च । किं चैवं प्रत्यभिज्ञादिबाधे क्षणिकविज्ञानशून्यादिवादपर्यवसानमागामिसूत्रादिविरोधश्चेति । नेदं व्यासदर्शनमपि तु सप्तमं प्रच्छन्नबौद्धदर्शनमेवेति । अथाद्वैतश्रुत्यनुरोधाज्जगत्कारणज्ञाननाश्यमिति कल्प्यमिति चेन्न, पूर्वापरवाक्यालोचनयाऽद्वैतश्रुतेर्व्याख्यास्यमानत्वादिति । योऽपि जगत्कारणेऽविद्याशब्दः सोऽप्यविद्याशक्तिमत्त्वेनात्मनि संसर्गतोऽध्यस्तत्वेन वा मन्तव्यः। “अनित्याशुचिदुःखानात्मसु नित्यशुचिसुखात्मख्यातिरविद्ये” ति योगसूत्रादिभिर्वृत्तिविशेष एवाविद्याशब्दार्थत्वावधारणादतो जगत्कारणे द्रव्यविशेषेऽविद्याशब्दसङ्केतमात्रेण तस्यज्ञाननाश्यत्वं न सिध्यतीति दिक् ।}

Nor is it the fifth; there is no proof for knowledge being the one that destroys the cause of the world (jñānasya jagatkāraṇatve prmāṇābhāvāt). By established worldly norms since it is only through the removal of false knowledge that through correct knowledge liberation is attained, it is cumbersome to imagine of what is known as the destruction of the world through this seen means (i.e.knowledge). Nor does that happen; by one person’s attainment of self-realization if the world is destroyed it will be unreasonable for others to see it; so also in the state of jīvanmukti of that very same person the sight of the world is also not logical as when the cause is destroyed it is not logical for the effect to stay. If one accepts the presence of a tinge of avidyā then there will be the defect of inconsistency etc., in imagining the relationship of destroyer and destroyed between knowledge and ignorance\footnote{Since knowledge is supposed to be the destroyer of ignorance totally there will be a contradiction if a tinge of ignorance is accepted in the jīvanmukti stage. This may be a reference to the view of Sureśvara mentioned in the Naiṣkarmyasiddhi (cf. Naiṣkarmyasiddhi verses 60--61)} (jīvanmuktasyāpi avidyāleśasvīkāre jñānājñānayornāśyanāśakabhāvā\-dikalpanāyām vyabhicārādidoṣācca). As there is a difference in each person of what is known as the causal avidyā, the effectual world of each puruṣa being different there will be the added stretched imagining of infinite worlds (tatkāryajagadbhedenānantajagatkalpanāgaura\-vācca). Moreover, in this way, since recognition is obstructed it will end up as the doctrine of momentariness (kṣaṇikavāda), stream of\break knowledge (vijñānavāda) and emptiness (śūnyavāda) and that is in contradiction to what is mentioned in the next sūtra (“tattu samanvayāt” BS.I.1.4). Neither is this the philosophy of Vyāsa (Bādarāyaṇa author of the BS); it is the seventh hidden philosophy of Buddha.\footnote{This being totally different from the already existing six schools of darśana; this is a further dig at Śaṅkara’s advaita Vedānta toward which Bhikṣu has an ongoing battle.}

\textbf{Ques:} If it is said that in accordance with the advaita śruti texts one needs to imagine the destruction (of avidyā) which is the cause of the world, through knowledge then the answer is:

\textbf{Ans:} it is not so. One needs to explain the śruti statements by reflecting on the preceding and following statements. The word ‘avidyā’ used in connection with (being) the cause of the world is to be understood in the sense of possessing the power of avidyā through connection, due to superimposition in ātman (yo’pi jagatkāraṇe’vidyāśabdaḥ so’pyavidyāśaktimatvenātmani samsargato’dhyastatvena vā manta-\break\-vyaḥ). Understanding the meaning of the word avidyā as a special modification (of the mind) through sūch Yogasūtra statements as “anityāśuciduḥkhānātmasu nityaśucisukhātmakhyātiravidyā” (YS.II.5)\break (then) by taking the meaning of the word avidyā conventionally as a special substance in the cause of the world, one cannot establish its destruction through knowledge; this is the general direction (of the reasoning).

\dev{तस्मादविद्यानिवृत्तिर्ज्ञानस्य न पुरुषगतं फलं भवति, ज्ञानप्रतिबन्धकत्वेनाविद्यानिवृत्त्यनन्तरमेव ज्ञानोदयात्, अविद्यायाः पुरुषनिष्ठत्वाभावाच्च । किन्तु विद्याख्यबुद्धिवृत्तेः पुरुषनिष्ठोऽनुभव एव फलम् । ततश्च पुरुषार्थसमाप्तितश्चित्तनिवृत्त्या मिथ्याज्ञाननिवृत्तिस्ततश्च धर्माधर्मानुत्पत्त्या मुक्तिरिति स्मर्तव्यम् ।}

\dev{यच्च— }
\begin{verse}
\dev{तेषामेषानुकम्पार्थमहमज्ञानजं तमः ।}\\
\dev{नाशयाम्यात्मभावस्थो ज्ञानदीपेन भास्वत ।।}
\end{verse}
\dev{इति गीतावाक्यं तमसो ज्ञाननाश्यत्वं वदति, तत्राज्ञानशब्दो मिथ्याज्ञानवासनावाची तम:शब्दो मिथ्याज्ञानवाची, कार्यकारणयोरभेदात् अविद्यादिक्लेशपञ्चकेऽपि ज्ञानावरकत्वेन तम आदिसंज्ञापञ्चकस्य स्मृतिप्रसिद्धत्वाच्च ।}

\dev{यद्यपि—}
\begin{verse}
\dev{अज्ञानेनावृतं ज्ञानं तेन मुह्यन्ति जन्तवः ॥}\\
\dev{ज्ञानेन तु तदज्ञानं येषां नाशितमात्मनः ।}\\
\dev{तेषामादित्यवज्ज्ञानं प्रकाशयति तत्परम् ।।}
\end{verse}
\dev{इत्यादिवाक्यम्, तस्यायमर्थः—अज्ञानेन तमोद्रव्येण ज्ञानं सत्त्वद्रव्यमावृतं प्रतिबद्धवृत्तिकं भवति मलेनेव दर्पणप्रकाशः । अतो लोका मुह्यन्ति तमोवृत्तिभ्रमवन्तो भवन्ति। येषां तु ज्ञानेन सत्त्वेन तत्तमोऽभिभूयते तेषां तत्सत्त्वं परमात्मानं प्रकाशयति करणविधयाचैतन्यव्याप्यीकरोतीति । ज्ञानाज्ञानादिशब्दास्तु वृत्तिष्विव तत्कारणसत्त्वादिष्वपि शास्त्रे प्रयुज्यन्ते “सत्त्वं ज्ञानं रजः कर्म तमोऽज्ञानमिहोच्यत” इति मन्वादिवचनादिति ।}

\dev{एतेन,}
\begin{verse}
\dev{“अन्धं तम इवाज्ञानं दीपवच्चेन्द्रियोद्भवम् ।}\\
\dev{यथा सूर्यस्तथा ज्ञानं यद्विप्रर्षे विवेकजम् ।।”}
\end{verse}
\dev{इत्यत्रापि ज्ञानाज्ञानशब्दौ सत्त्वतमोगुणयोर्द्रव्ययोर्वाचकाविति ।}

\dev{[एवं च सति यान्यन्यान्यपि वाक्यान्यनुमानानि च अनाद्यखण्डाविद्यायामाधुनिकरुपन्यस्यन्ते तानि सर्वाणि तमोगुणपराण्येव । तस्य च तमोगुणस्य ज्ञानेन}

\eject

\dev{न  नाशो गुणानां नित्यत्वस्य—}
\begin{verse}
\dev{“विकारजननीं मायामष्टरूपामजां ध्रुवाम् ।}\\
\dev{त्रिगुण तज्जगद्द्योनिरनादिप्रभवाप्ययम् ।।”}
\end{verse}
\dev{“कृतार्थं प्रति नष्टमप्यनष्ट तदन्यसाधारणत्वादि” त्यादिश्रुतिस्मृतियोगसूत्र सिद्धत्वात् । किन्तु विषयेन्द्रियसंयोगादिना तस्योपसारण क्रियते । तथा तत्त्वज्ञानिपुरुषं प्रति करणादिविधया या गुणानां विकारशक्तिस्तस्या दाह एव भवतीति दिक् ।}

Therefore the removal of avidyā does not result in knowledge in puruṣa  (tasmādavidyānivṛttirjñānasya na puruṣagatam phalam); since it is an obstacle to knowledge it is only after the disappearance of avidyā that knowledge arises; also because avidyā is not situated in puruṣa.  But the modification of the intellect known as knowledge has the result of experience situated in puruṣa. Therefore at the end of the goal of human life\footnote{Having attained the last puruṣārtha which is mokṣa.} (mokṣa) due to the disapearance of the mind, there is the disappearance of false knowledge and then due to the nonrising of dharma or adharma there is liberation; this should be recalled. The Gītā verse: “teṣāmevānukambārtham…jñānadīpena bhāsvatā” (Gītā.\-X.11) mentions the capacity for destruction of knowledge of ignorance (tamasaḥ); therein the word ignorance (ajñāna) denotes the subliminal impression of false knowledge (and) the word tamaḥ denotes false knowledge; due the relationship of non-difference between the cause and the effect, even in the enumeration of the five kleśas as avidyā etc., (YS. II.3) memory is well known as the fifth called tamas (ignorance) etc., due to the covering of knowledge   (kāryakāraṇayorabhedāt avidyādikleśapañcake’pi jñānāvarakatvena tama ādisamjñāpañcakasya smṛtiprasiddhatvācca). Though there are sayings like: “ajñānenāvṛtam jñānam tena muhyanti jatavaḥ…teṣāmādityavajjñānam prakāśayati tatparam” its meaning is as follows: “ajñānena”= through the tamas substance the “jñānam”= sattva substance is covered and its modification is obstructed just as dirt covers the brightness of the mirror.  Through that people “muhyanti”= become deluded by the modification of tamas. Of those “jñānena tu”= who conquer that tamas through sattva, for them that sattva reveals the paramātman through the sense\break organ (karaṇavidhayā) which is pervaded by (pure) consciousness\break (vyāpyīkarotīti). The words “jñāna” and “ajñāna” used with reference to the modifications are used in śāstra with reference to their cause such as sattva etc., as mentioned in Manu: “sattvam jñānam rajaḥ\break karma tamo’jñānamihocyate” (MS.12.26). By this even in the verse: “andham tama ivājñānam…jñānam yadviprarṣe vivekajam” the words “jñāna” and “ajñāna” denote the substances sattva and tamas guṇas.

When it is so, all other statements as well as inferences which modern day Vedāntins declare as referring to a singular beginningless ignorance, all of them refer to the tamas guṇa alone. And that tamas guṇa is not destroyed by knowledge as guṇas are permanent\footnote{Bhikṣu exhibits his commitment to SY philosophy whenever he gets an opportunity.} as delared by the following śruti, smṛti and YS utterances: “vikārajananīm māyām…anādiprabhavāpyayam” (not traced), “kṛtārtham prati naṣṭamapyanaṣṭam tadanyasādhāraṇatvāt” (YS.II.22). However, through contact between the object and sense organs its removal is achieved. Then with regard to the wise, realized puruṣa, the power to induce change of the guṇas through the senses etc., is burnt; this is the general idea (tathā tattvajñanipuruṣam prati karaṇādividhayā yā guṇānām vikāraśaktistasyā dāha eva bhavatīti dik).

\dev{तस्मात् सत्त्ववृतेः प्रागेव सत्त्वेन स्ववृत्तिप्रतिबन्धकतमोगुणाभिभवो न तु वृत्त्येति सिद्धम्, योगभाष्ये व्यासैरेवमवधृतत्वात् । तथा च सांख्यकारिका—“अन्योन्याभिभवाश्रयजननमिथुनवृत्तयश्च गुणाः” इति । या चेयं योगसांख्ययोः प्रमाणतत्फलादिप्रक्रिया प्रदर्शिता एषेवात्रापि दुर्शने ग्राह्या, प्रमाणप्रमादिप्रक्रियाया अत्रासूत्रणादविरोधाच्च। स्वकीयतन्त्रानुक्तेऽर्थे स्वतन्त्राविरुद्धस्य परतन्त्रसिद्धान्तस्यैव स्वतन्त्रसिद्धान्तत्वात्, अन्यथा निर्णयासंभवात्,   परोक्तमविरोधि चेति न्यायाच्च,}

\dev{प्रतितन्त्रसिद्धान्ततयैतादृशसिद्धान्तस्य न्यायाचार्यैरपि सूत्रितत्वाच्च “समानतन्त्रसिद्धः परतन्त्रासिद्धः प्रतितन्त्रसिद्धान्त” इति सूत्रेण । अत्र प्रतितन्त्रसिद्धान्त इति लक्ष्यनिर्देशः, शेषं लक्षणम् । परतन्त्रासिद्ध इति पूर्वसूत्रलक्षितात् सर्वतन्त्रसिद्धान्ताद् व्यावर्तकमात्रम् । अस्माकं च नास्तिकतन्त्रस्यैव परतन्त्रता न त्वास्तिकतन्त्रस्येति बोध्यम् । अतो ज्ञानोत्पत्तिप्रक्रियायामात्मनात्मविवेके पदार्थनिरूपणादौ सांख्ययोगक्तिसिद्धान्त एव मन्वादिसमादृततया ग्राह्योऽस्मद्दर्शनसूत्राभावात्, तदविरोधेन च न्यायादिसिद्धान्तोऽपीति न स्वतन्त्रा कापि रचनात्र}$^{2}$ \dev{निर्मूला कर्तव्या ।}
\begin{verse}
\dev{“यः शास्त्रविधिमुत्सृज्य वर्तते कामकारतः ।}\\
\dev{न स सिद्धिमवाप्नोति न सुखं न परां गतिम् ।।}
\end{verse}
\dev{इति स्मृतेः । ऋषिभिश्च वेदान्तानां समुद्र इव दुरवगाहतामवधार्य वेदान्तार्था एव संक्षिप्य स्पष्टीकृत्य च भागशः प्रोक्ताः, न तु तेष्वा (तेप्या) धुनिकवत् स्वतन्त्रा अज्ञा वा । अतस्तेषां परस्पराऽविरोधेन$^{3}$ परस्परानुसारेण चैव वेदान्तमीमांसा व्याससूत्रैरिति विज्ञप्तिः,}
\begin{verse}
\dev{एवमेकं सांख्ययोगं वेदारण्यकमेव च ।}\\
\dev{परस्पराऽङ्कान्येतानि पाञ्चरात्रं च कथ्यते ।।}
\end{verse}
\dev{इति मोक्षधर्मवाक्यादिति । सांख्ययोगमिति समाहारः, एवं वेदारण्यकमिति वेदवेदान्तावित्यर्थः । अन्यथा एतानीति बहुवचनासङ्गते । तथा—}
\begin{verse}
\dev{इतिहासपुराणाभ्यां वेदान् समुपबृंहयेत् ।}\\
\dev{बिभेत्यल्पश्रुताद् वेदो मामयं प्रहरिष्यति ।।}
\end{verse}
\dev{इत्यादिभारतादिवाक्याच्च}

Therefore it is established that even before the modification of sattvaguṇa (sattvavṛtteḥ prāgeva), there is the overpowering by sattva of tamas guṇa which is an obstacle to its (sattva’s) modification and it is not through the vṛtti (that there is overpowering) (sattvena svavṛttipratibandhakatamoguṇābhibhavo na tu vṛttyeti siddham). Thus the SK says: “anyonyābhibhavāśrayajananamithunavṛttayaśca guṇāḥ” (SK.12).  That which has been demonstrated as the process of the means of knowledge and its result in Sāṁkhya and Yoga (yā ceyam yogasāṁkhyayogayoḥ pramāṇatatphalādiprakriyā pradarśitā) that itself can   be accepted in this darśana (philosophy of Vedānta) as well, since there is no sūtra (in BS) that describes the process of means of knowledge and its result and it is also not in opposition (to it).\footnote{Yoga has spelt out the process of how knowledge is acquired. Thus Bhikṣu advocated the acceptance of this theory in Vedānta as well. Moreover this process has some commonality with Vedānta. Bhikṣu’s commitment to SY is also on display.}  

When there is non-mention of the meaning (of a topic) in one’s own śāstra, then that view of another śāstra which is not in opposition to one’s own śāstra is to be accepted as the view of one’s own śāstra; otherwise there will the impossibility of (being able to come to) any conclusion. This is also in consonance with the principle that one can accept something that is in another śāstra if it is not in opposition (to one’s own view). Since every śāstra has its doctrines such a principle has also been mentioned in sūtras written by Nyāya scholars as: “samānatantrasiddhaḥ paratantrāsiddhaḥ pratitantrasiddhānta) (NS.I.1.29). Herein the phrase “pratitantrasiddhāntaḥ” points to what is intended (lakṣyanirdeśaḥ), the rest “samānatantrasiddhaḥ paratantrāsiddhaḥ” are qualifications of that. Since the earlier sūtra (NS.I.i.28) has indicated the sense of “paratantrāsiddhaḥ” as unopposed by any school this only mentions that which separates it from all other scools (sarvasiddhāntād vyāvartakamātram). It should also be known that we consider the opposite school as the school of the nāstikas and not the āstika schools.\footnote{It seems that Bhikṣu wants to club advaita in the nāstika category.} Therefore in the process of the rise of knowledge, in the distinction between ātman and anātman, in deciding what is a substance etc., it is what is said in the school of SY (and) which finds support in Manu etc., which is to be accepted as there is an absence of any sūtra in one’s own darśana  (for that purpose).  As Nyāya also is not in opposition (to it) one should not manufacture something independently which has no basis (nirmūlā). Thus smṛti says: “yaḥśāstravidhimutsṛjya…na sukham na parām gatim”.  

Determining that the Vedānta ocean is difficult to fathom, the meaning of Vedānta has been clarified briefly by the ṛṣis and mentioned in parts; however unlike modern day Vedāntins they are not free (interpreting in their own way) nor are they ignorant.\footnote{They keep the underlying meaning in tact and do not freely give their own interpretation to the Vedānta śāstra. This is a direct hit at the theories of Śaṅkara’s Advaita Vedānta.} Therefore (we) respectfully state that without any mutual contradiction and by mutual agreement (with them) the inquiry into Vedānta is to be done of the sūtras of Vyāsa; this is the meaning of the Mokṣa.P saying: “evamekam sāmkhyayogam vedāraṇyakam eva ca, parasparā’ṅgānyetāni pāñcarātram ca kathyate”. The word “sāṅkhyayogam” is a samāhāradvan\-dva compound; similarly “vedāraṇyakam” means veda and vedānta; otherwise the use of the word “etāni” does not make sense. This is also what is understood from the MBh vākya (sentence): “itihāsapurāṇābhyām vedān samupabṛmhayet…vedo māmayam prahariṣyati”.\footnote{MBH. 1,1.267}

\dev{पुरुषमतिदोषात्त्वापात एव विरोधो दृश्यते इति । ईश्वरप्रतिषेधादिना सांख्यादिभिः सह विरोधं पश्चान्निराकरिष्यामः । ये तु वेदान्तदर्शने सांख्यादिभिः सह पारमार्थिकं विरोधं मन्यन्ते, तेषां सांख्याद्यनुसारिश्रुतिभिः सहापि विरोधो दुर्निवार इति । तत्रैते सङ्ग्रहश्लोकाः—}
\begin{verse}
\dev{व्यवहारे वयं योगास्तत्त्वषड्विंशगोचरे ।}\\
\dev{भेदस्तु वीक्ष्यते बालैरुपासावाक्यमोहितैः ।।}\\
\dev{पुंप्रकृत्योर्विवेकेन जीवत्वैश्वर्यबाधतः ।}\\
\dev{चितिमात्रैकरूपात्मवादे सांख्याश्च यादृशाः ।।}\\
\dev{शक्त्यात्मैक्यानुवादेन परजीवानुमानयोः ।}\\
\dev{त्रिगुणप्रविभागे च न्यायवैशेषिकानुगाः ।।}\\
\dev{वेदान्तोक्तिविचारे तु वयं जैमिनिपक्षगाः ।}\\
\dev{पञ्चशिंतितत्त्वानां षड्विंशे प्रविलापनात् ॥}\\
\dev{महाप्रलयकालीनं ब्रह्माद्वैतं तदात्मता ।}\\
\dev{ब्रह्मणः सृष्टिरित्यादि चास्माकमधिकं मतम् ॥}\\
\dev{आर्षज्ञानानि सर्वाणि वेदान्तानां कलाः स्मृताः ।}\\
\dev{श्रुत्यवान्तरवाक्यानामृते नास्ति ह्यसङ्गतिः ।।}\\
\dev{न केवलं ह्यास्तिकानां वादो वेदेष्वपेक्ष्यते ।}\\
\dev{उपासनार्थे देहात्मवादो बौद्धोऽप्यपेक्षितः ॥}\\
\dev{याथार्थ्यनीश्वरत्वाभ्यां संवादोऽत्र हि दृश्यते ।}\\
\dev{तस्मात् स्वतन्त्ररचना त्यक्तव्यात्र मुमुक्षुभिः ।। इति ।}
\end{verse}
\dev{प्रमात्रादिसंग्रहश्लोकाश्च—}
\begin{verse}
\dev{प्रमाता चेतनः शुद्धः प्रमाणं वृत्तिरेव च ।}\\
\dev{प्रमार्थाकारवृत्तीनां चेतने प्रतिबिम्बनम् ॥}\\
\dev{प्रतिबिम्बितवृत्तीनां विषयो मेय उच्यते ।}\\
\dev{वृत्तयः साक्षिभास्याः स्युः करणस्यानपेक्षणात् ॥}\\
\dev{साक्षाद्द्रष्टृत्वरूपं च साक्षिलक्षणमीरितम् ।}\\
\dev{सांख्यसूत्रे ततो गौणं सर्वसाक्षित्वमीश्वरे ।।}\\
\dev{इन्द्रियाव्यवधानेन तथा रागाद्यभावतः ।}\\
\dev{अथवेशप्रमा नित्या करणेन न जन्यते ।।}\\
\dev{करणाजन्यबोधाख्या साक्षितेशेऽखिलेष्वतः ।  इति}
\end{verse}
\dev{ये त्विदानीन्तना वेदान्तिनः सांख्ययोगादीन् विहाय भट्टादिमतानुसारेण व्यवहारं रचयन्ति $^{1}$स्वपञ्चसूत्रानुक्तमना\-द्यखण्डाविद्यादिकं कल्पयन्ति  ते कालदोषण विपरीततया त्याक्तानादिसेतवोऽन्धगोलाङ्गूलं गृहीत्वाऽतिगम्भीरमपारवेदान्ताब्धिं विविधबौद्धग्राहाकुलं समुत्तितीर्षन्ति ।}

Puruṣa (ātman), due to many faults for the time being alone appears to be different from itself (displays contradictions from its own true self) (puruṣamatidoṣāttvāpāta eva virodha dṛśyate). We shall later remove  the opposition towards Sāṁkhya and other philosophers who have rejected Īśvara etc. Those philosophers who, along with Sāṁkhya philosophers etc., consider opposition to the highest truth in Vedānta for them opposition to the śruti statements of Sāṁkhya philosophy is also difficult to avoid.\footnote{This probably is the reference to the presence of different Sāṁkhya schools in the milieu.} The following verses are relevant in that context: “vyavahāre vayam yogāstattvaṣadvimśagocare bhedastu vīkṣyate bālairupāsāvākyamohitaiḥ…yāthārthyānīśvaratvābhyām samvā-\-\break\hbox{do’\-tra} hi dṛśyate tasmāt svatantraracanā tyaktavyātra mumukṣibhiḥ”.

There are also the following verses regarding the knower: “pramātā cetanaḥ śuddhaḥ pramānam vṛttireva ca pramārthākāravṛttīnām cetane pratibimbanam…athaveśapramā nityā karaṇena na janyate karaṇājanyabodhākhyā sākṣite’khileṣvataḥ”.

Those present day Vedāntins, giving up Sāṁkhya and Yoga philosophies, who interpret worldly activity following Kumārilabhaṭṭa (one school of Pūrvamīmāṁsā the other being Prābhākara school) imagine a singular, beginningless avidyā etc., not mentioned in one’s own (the original) five sūtras (of the BS)\footnote{Bhikṣu consideres the fifth sūtra (BS.I.5) as important as the first four BS sūtras. This is where prakṛti figures prominently and is crucial to Bhikṣu’s commitment to S-Y philosophy and his interpretation of Vedānta accordingly.} they, having rejected the ancient bridges of knowledge and advocating contrary views (viparītatayā) due to the defect of changing times, taking hold of the tail of a blind cow, desire to cross the very serious unbounded ocean of Vedānta which is disturbed by different Buddhist sharks (te kāladoṣeṇa  viparītatayā tya\-ktānādisetavo’andhagolāṅgūlam gṛhītvā’tigambhīramapāravedāntā-\-\break{bdhim} vividhabauddhagrāhākulam samuttitīrṣanti).\footnote{Vijňānabhikṣhu seems to draw a veiled reference to the practice of ‘giving a cow as gift’ in the last rites (antyesṭi). The belief is that the deceased will hold the tail of the cow and cross the river of death (called vaitariṇī); in other words, the cow will be the swim-boat for the deceased to cross over the tortuous deadly river called vaitariṇī.}

\dev{अत्रादौ चतुरध्यायीवाक्यार्थः संक्षेपेणोच्यते-प्रथमाध्याये प्रथमपादेन प्रायशोऽस्पष्टलिङ्गैर्वेदान्तवाक्यैरधिष्ठानरूपमुपाध्युपहितं ब्रह्म विचारणीयम् । द्वितीयपादेन च स्पष्टलिङ्गैस्तथाविधमेव ब्रह्म । तृतीयपादेन चोपाधिविवेकाविवेकौदासीन्येन सर्वाधिष्ठानं ब्रह्म। एवं हि “सर्वभूतस्थमात्मानं सर्वभूतानि चात्मनि” त्यादिश्रुतिरमृतयो व्याख्याताः स्युः । यथा “एष त आत्मा अन्तर्याम्यमृतः” “तवान्तरात्मा मम च सर्वेषामेव देहिनामि” त्यादिश्रतिस्मृत्युक्तं ब्रह्मात्मत्वं चोपपादितं भविष्यतीत्याशयः । यद्धि यस्याधिष्ठातृ आधारश्च भवति तत्तस्यात्मेत्युच्यते अध्यक्षत्वात् स्वरूपत्वात्, यथा जीवो लिङ्गदेहस्येति । चतुर्थपादेन च प्रधानादीनां स्वातन्त्र्येण जगत्कारणत्वाभावः प्रधानादिशक्तिवर्गश्च ब्रह्मणः प्रतिपादयिष्यते इति प्रथमाध्यायार्थः ।}

\dev{द्वितीयाध्याये च ब्रह्मकारणत्वे प्रमाणान्तरविरोधपरिहारः, बौद्धमतनिराकरणं,}

\dev{सृष्टिस्थितिसंहारनिरूपणं च । तृतीयाध्याये च जीवस्य संसारानर्थप्रतिपादनपूर्वकतन्निवृत्त्युपायं सोपकरणब्रह्मविद्यां, तद्विषयभूतमुपाधिविविक्तं ब्रह्मतत्त्वं च विचारयिष्यति । चतुर्थाध्याये च सद्योमुक्तिक्रममुक्तिभेदेन द्विविधं ब्रह्मात्मविद्याफलमिति । तदेवं प्रमाणाविरोधसाधनफलप्रतिपादकाश्चत्वारोऽध्याया इति ।}

Now at the start I shall state briefly the sense (meaning of the four adhyāyas (chapters) (of the BS). In the first quarter (prathamapādena) of the first adhyāya, Brahman that is superimposed with limitation (and) is of the nature of support, is discussed in general, through Vedānta statements using indistinct marks/signs (aspaṣṭaliṅgaiḥ).\footnote{This is the way Bhikṣu wants us to understand his approach to the topics of the BS in his VijBh.} In the second  quarter the same Brahman (is discussed) using definite signs. In the third quarter (is discussed) Brahman which is the support of everything, through (an analysis of) the difference/non-\-difference and neutral characteristic of the limitation.  In this way śruti and smṛti statements such as “sarvabhūtasthamātmānam sarvabhūtāni cātmani” (Kaiv.Up.10) have been explained. When such śruti and smṛti sayings such as “eṣa ta ātmā antaryāmyamṛtaḥ” Bṛ.Up. 3.7.3; 4.2.3), “tavāntarātmā mama ca sarveṣāmeva dehinām” state that Brahman possesses the essence of ātman (ātmatva), it hopes that (this realization) will happen (in the future when striven for). That which is the supervisor and support of something, that is mentioned as the essence of that (tattasyātmetyucyate) through possessing superintendence and being of the same nature (adhyakṣatvāt svarūpatvāt) just as when one says ‘the jīva is the (supervisor) of the subtle body. In the fourth quarter pradhāna etc., independently not being the cause of the world and the powers of pradhāna etc., of Brahman will be discussed; this is the purpose of the first chaper (of the BS). 

In the second adhyāya there is removal (abandonment) of different opposing proofs against Brahman being the cause (of the universe), rejection of the Buddhist doctrine and investigating the creation, sustenance and destruction of the world. In the third adhyāya after pointing out the meaninglessness of the world, will be discussed the means for jīva to give up the world, such as the useful brahmavidyā and its subject matter which is the essence of Brahman devoid of any limitation (tṛtīyādhyāye ca jīvasya samsārānarthapratipādanapūrvakatannivṛttupāyam sopakaraṇabrahmavidyām, tadviṣayabhūtamupādhivi\-viktam brahmatattvam ca vicārayiṣyati). In the fourth adhyāya (is discussed) the result of brahmātmavidyā (liberation) dividing it into two types called immediate liberation and gradual liberation. Thus the four chapters demonstrate proof, non-opposition, means and the result of Brahman realization respectively.

\dev{ननु  ब्रह्मणो जगत्कारणत्वं शास्त्रयोनीति कथमुच्यते? निर्विकारचिन्मात्रताप्रतिपादकश्रुतिस्मृतिभ्यामुपादानत्वकर्तृत्वयोरुभयोरपि प्रतिषेधात् । यथा—}
\begin{verse}
\dev{“निष्क्रियं निष्कलं शान्तं निरवयं निरञ्जनम् ।}\\
\dev{अमृतस्य परं सेतुं दग्धेन्धनमिवानलम् ।।}
\end{verse}
\dev{साक्षी चेता केवलो निर्गुणश्चे” त्याद्याः श्रुतयो विकारकृत्यादिकं ब्रह्मणि प्रतिषेधन्ति । स्मृतयश्च—}
\begin{verse}
\dev{सन्निधानाद् यथाकाशकालाद्याः साधनं तरोः ।}\\
\dev{तथैवापरिणामेन विश्वस्य भगवान् हरिः ।}\\
\dev{प्रोच्यते परमेशानो यः शुद्धोप्युपचारतः ।}\\
\dev{प्रसीदतु स नो विष्णुरात्मा यः सर्वदेहिना इति ।}
\end{verse}
\dev{न च कालादिवन्निमित्तकारणत्वमेव जन्मादिसूत्रार्थः संभवति, निमित्तकारणतायाः कालादिसाधारण्येन ब्रह्मलक्षणत्वासंभवात् । [ यत इति पञ्चम्या प्रकृतित्वलाभाच्च, जनिकर्तुः प्रकृतिरित्यनुशासनेन प्रकृतावपादानसंज्ञायाम् अस्या अपादानपञ्चमीत्वात्, सूत्रमूलभूतोदाहृतश्रुतौ “यत्प्रयन्त्यभिसंविशन्ती” त्यनेन प्रकृतित्वसिद्धेश्च उपादान एव कार्यलयादिति । ]}

\dev{न च ब्रह्मणो व्यहारतो विकारत्वेऽपि विकारस्य वाचारम्भणमात्रतया विवर्तत्वेन परमार्थतो निर्विकारत्वमेवेति वाच्यम् । व्यावहारिकविकारस्यापि प्राकृततयाऽऽत्मनि प्रतिषेधात्,}
\begin{verse}
\dev{“प्रकृत्यैव च कर्माणि क्रियमाणानि सर्वशः ।}\\
\dev{यः पश्यति तथात्मानमकर्तारं स पश्यति ।।”}
\end{verse}
\dev{इत्यादिवाक्यैः । किं च विकारमिथ्यात्वेनाविकारित्वं प्रकृतेरप्यस्तीति कूटस्थनित्य- परिणामिनित्यविभागसिद्धान्ताद्यनुपपत्तिरिति ।}

\dev{अपि च वाचारम्भणवाक्येन विकारस्यात्यन्ततुच्छत्वं नोच्यते किन्त्वनित्यतया पारमार्थिकसत्यत्वाभाव एव, यतो विकारो वाचारम्भणं नाम कार्यः पश्चाच्च नाममात्रावशेषो भवति “वेदशब्देभ्य एवादौ पृथक् संस्थाश्च निर्ममे, नामैवैनं जहात्ति” इति श्रुतिस्मृतिभ्याम् । अतो विकारो मृत्तिकेत्येव कारणरूपेणैव सत्यं नित्यं न तु विकार इतीति तद्वाक्यार्थत्यादिति । तथा चैवंविधवाक्यैरनित्यत्वमात्रबोधनान्न विवर्तवादः, किं तु परिणाम एवेति ब्रह्मण उपादानत्वं न संभवति। तस्मात् स्वोपाधेः सूर्यवत साक्षितामात्रेण ज्ञानद्वारैव निमित्तकारणं ब्रह्म । सर्वाणि चोपदानत्वादिवाक्यानि प्रधानादिपराण्येवेति न ब्रह्मकारणत्वं शब्दयोनीति । तमिमं पूर्वपक्षं समाधत्ते—}

\textbf{Ques:}  But then how can you state that Brahman being the cause of the world is because of being the cause of śāstra. Brahman who is presented by śruti and smṛti as being without change and of the nature of pure consciousness has been rejected as both the material and efficient cause. Thus śruti statements like “niṣkriyam niṣkalam śāntam…dagdhendhanamivānalam” (not traced); “sākṣī cetā kevalo nirguṇaśca” (Śvet.Up.6.11) have rejected change, action etc in Brahman. So also smṛti statements such as: “sannidhānād yathākāśalakādyāḥ sādhanam taroḥ…prasīdatu sa no viṣṇurātmā yaḥ sarvadehinām”.

Also, the meaning of the sūtra “janṁādyasya yataḥ” (BS.I.1.2) similar to time etc., cannot be interpreted as causal efficiency (na ca kālādivannimittakāraṇatvameva janmādisūtrārthaḥ sambhavati); since time etc., are common (to all change) it is not possible to be a characteristic of Brahman. Also the use of the word “yataḥ” has the additional (meaning) of denoting a natural process; the agent of creation is called prakṛti according to śāstra (and) the ablative case is used in yataḥ (interpreted as prakṛti); thus due to the use of pañcamī vibhakti in yataḥ, (and) through the expression “yatprayantyabhisamviśanti” (Taitti.Up.3.1) used as an example in the original sūtra since its being prakṛti is established, (one concludes that) the effect (world) lies within the material cause i.e. prakṛti.\footnote{Bhikṣu thus introduces his theory of prakṛti beng the upādānakāra(material cause) of the world which he will develop further in more detail under BS. I.1.5} 

Moreover even though change in Brahman is used customarily, since change is only of name and form due to being superimposed (vikārasya vācārambhaṇamātratayā vivartatvena) (Brahman) being changeless is to be mentioned as in fact the truth; also because worldly change is part of nature and it is rejected in the ātman through such sayings as: “prakṛtyaiva ca karmāṇi…tathātmānamakartāram sa paśyati” (Gītā.13.29). Moreover, change being false, prakṛti is also unchanging (possesses non-change; (vikāramithyātvenāvikāritvam prakṛterapyastīti). This also logically fits in with the doctrine of avibhāga (Bhikṣu’s own interpretation of Vedānta) which believes in the eternal nature of that which changes (i.e.prakṛti) as well as the eternal immutable (Brahman/ātman).

Moreover by the sentence “vācārambhaṇa” change has not been mentioned as totally insignificant; but being impermanent it has the absence of ultimate truth (it does not possess absolute truth) (anityatayā pāramārthikasatyātvābhāva eva); therefore change is only the effect known as word and form which in the end remains just in name (paścācca nāmamātraśeṣo bhavati); thus śruti and smṛti state “vedaśabdebhya evādau pṛthak samsthāśca nirmame, nāmaivainem jahāti”. Therefore the meaning of that is ‘change in truth is like clay (mṛttiketyeva) (and) is true only in the form of being the cause and not in the change’. Thus through such statements, since only the impermanence is advocated, it is not the doctrine of superimposition but it is only the doctrine of transformation and so Brahman cannot possibly be the material cause.\footnote{This is an attack by Bhikṣu on the advaita view of Brahman being both the material and efficient cause and advocating his own theory of pariṇāma (transformation).} Therefore Brahman is just the efficient cause through knowledge/consciousness, by being just the witness of its own limitation like the sun (tasmāt svopādheḥ sūryavat sākṣitāmātreṇa jñānadvāraiva nimittakāraṇam brahma). And all the statements mentioning (Brahman being a) material cause etc., are inclined towards prakṛti alone and “śabdayoni” (in BS.I.1.3) does not indicate that Brahman is the cause. Thus the prima facie view (pūrvapakṣam) has been presented (samādhatte).

\section*{\dev{तत्तु समन्वयात् ॥ ४ ॥}}

\dev{तत्पदं व्यवहितपरामर्शार्थं, तु शब्दः पूर्वपक्षप्रतिषेधार्थः । तद्ब्रह्मणो जगज्जन्मादिकारणत्वं तु समन्वयात् अस्य जगतः कारणरूपे ब्रह्मणि समनुगतत्वात्, ब्राह्मणश्च कारणरूपेण जगति समनुगतत्वादुपपद्यत इति शेषः । तथा चाधिष्ठानरूपोपादानकारणत्वं जन्मादिसूत्रेणोक्तमित्याशयः। यत्राविभक्तं येनोपष्टब्धं सदुपादानकारणं कार्याकारेण परिणमते तदेवाधिष्ठानकारणम् । तच्चाविकारिणोऽपि जन्मादिसूत्रे गगनवाय्वादिदृष्टान्तेनास्माभिः प्रतिपादितम् । ब्रह्मण औपाधिके च कर्तृत्वविकारित्वे पश्चादुपपादयिष्यामः ।}

\dev{नन्वस्य सूत्रस्य यथोक्तार्थः कथमवधृत इति चेत्, उत्तरसूत्रात् । उत्तरसूत्रे हि}

\dev{“तदैक्षते” तिश्रुत्या ब्रह्म प्रसाधयिष्यति । अतस्तच्छ्रुत्युपक्रमवाक्यस्य “सदेव}

\dev{सोम्येदमग्र आसीदि” त्यस्यार्थ एव तत्रत्यप्रयोजनेनात्रोच्यत इत्युन्नीयते, आकाङ्क्षितत्वात् । तेन च वाक्येन प्रलये सदनुगततया जगदवस्थानेन वक्ष्यमाणजगत्कारणत्वं कूटस्थस्यापि ब्रह्मण उपपादितम्। अतोऽनेनापि सुत्रेण तेनैव हेतुना तदेवोपपद्यते इत्युचितम् । 
किं च “जन्माद्यस्य यतोऽन्वयादितरतश्चे” ति भागवतोपक्रमवाक्येऽपि अन्वयहेतुना ब्रह्मणो जगत्कारणत्वमुपपादितम् । अतोऽत्र तत्समानार्थतोचिता। तद्वाक्यस्य चायमर्थः—प्रलये तत्रैवान्वयात् सर्गकाले च तत एव विभागादस्य जगतो जन्मदि यतः यदुपादानकं तत्सत्यं धीमहि इति । अत्र सूत्रे समुपसर्गात् आकाङ्क्षायोग्यतायां च परस्परान्वये तात्पर्यमिति विशेषः । परस्परान्वये च श्रुतयः—“सर्वभूताधिवासं यद् भूतेषु च वसत्यपि, सर्वानुग्राहकत्वेन तदस्म्यहं वासुदेवः, यत्परं ब्रह्म सर्वात्मा सर्वस्यायतनं महत्, यः सर्वस्मिंस्तिष्ठन् सर्वान्तर” इत्याद्याः। स्मृतयश्च—}
\begin{verse}
\dev{“यदा भूतपृथग्भावमेकस्थमनुपश्यति ।}\\
\dev{तत एव च विस्तारं ब्रह्म संपद्यते तदा ।।}\\
\dev{मयाऽध्यक्षेण प्रकृतिः सूयते सचराचरम् ।“ इत्याद्याः ।}\\
\dev{ब्रह्मणि प्रपञ्चसमन्वयप्रकारश्चोक्तो ब्रह्माण्डपुराणे—}\\
\dev{“स्थूलं विलाप्य करणे करण निदाने,}\\
\dev{तत्करणं करणकारणवर्जिते च ।}\\
\dev{इत्थं विलाप्य यमिनः प्रविशन्ति यत्र,}\\
\dev{तं त्वां हरिं विमलबोधघनं नमामि ॥ इत्यनेन ।}
\end{verse}
\dev{अस्यायमर्थः—निर्गुणब्रह्मचिन्तनकाले योगिन आदौ स्थूलं विकारजातं करणे महदादिसप्तके प्रकृतिविकृतिरूपे मनसा विलापयन्ति । तेषां करणत्वं च स्थूलोत्पादने प्रकृतेर्द्वरत्वात् । ततस्तत्करणजातं निदाने आदिकारणे प्रकृतौ विलापयन्ति लय चिन्तयन्तीत्यर्थः । एवं सर्वत्र । तां च प्रकृति गुणत्रयरूपिणीं नित्यामपि निर्व्यापारतया गर्तस्थमृतसर्पवत् करणकारणविविक्ते चिन्मत्रे ब्रह्मणि विलापयन्ति । एवं जडवर्ग विलाप्य स्वयमप्यंशांश्यविभागेन मनसा विलीयन्ते समुद्रादौ नद्यादिवत् । उपाधिविलये तत्कृतविभागनिवृत्तेरिति । यमिन इति पुरुषसामान्योपलक्षणम् । अत्र लयो नात्यन्तोच्छेदः, किन्तु विकाराणां प्रकृतावव्यक्ततयाऽवस्थानं, प्रकृतिपुरुषयोश्च ब्रह्मणि सुषुप्तवन्निर्व्यापारतयाऽवस्थानमिति बोध्यम् । “मधु मधुकृतो निष्टिम्यन्ति नानात्मनां वृक्षाणां रसान् समवहारमेकतां रसं गमयन्ति, ते यथा तत्र न विवेकं लभन्त” इत्यादिश्रौतदृष्टान्तैः प्रलयकाले ब्रह्मण्यविभक्ततया सववस्त्वस्थानसिद्धेः ।}
\begin{verse}
\dev{ब्रह्माप्येति प्रपञ्चोऽयं रूपं हित्वा तु वैकृतम् ।}\\
\dev{जहाति कठिनावस्थां जलधौ लवणं यथा ।।}\\
\dev{आसीदिदं तमोभूतमप्रज्ञातमलक्षणम् ।}\\
\dev{अप्रतर्क्यमविज्ञेयं प्रस्तुप्तमिव सर्वतः ।।}
\end{verse}
\dev{इत्यादिस्मृतेश्चेति ।}

\section*{“Tattu samanvayāt” (BS.I.1.4)}

The word “tat”\footnote{As is Bhikṣu’s style witnessed in most of his works he starts off by explaining each word of the sūtra at the commencement of his commentary in detail.} is to recollect what is interrupted; the word “tu” means the rejection of the prima facie view (pūrvapakṣapratiṣedhā\-rthaḥ). “samanvayāt”= since Brahman being the cause of the birth etc., (sustenance and destruction) of the world is uniformly understood (naturally connected) in (the Upaniṣads); (as is his wont Bhikṣu gives a word by word meaning of the sūtra). In the (world), as Brahman is connected to the world in the form of being its cause, it is reasonable (to think as above) (brahmaṇaśca kāraṇarūpeṇa jagati samanugatatvāt upapadyate iti śeṣaḥ). Thus, being the material cause in the form of being a support, has already been mentioned in the sūtra “janmādi” etc (BS.I.1.2); that is the idea. That material cause which being undivided through being closely connected, transforms itself into the shape of the effect is alone  a supportive cause (yatrāvibhaktam yenopaṣṭabdham sadupādānakāraṇam kāryākāreṇa pariṇamate tadevādhiṣṭhānakāraṇam).  And that can also belong to that (Brahman) which is immutable; we have already mentioned that under the sūtra “janmādyasyataḥ) through the example of sky, space etc. We shall later show how Brahman, when it has a limitation, is without change in the act of creation (brahmaṇa aupādhike ca kartṛtvāvikāritve paścādupapādayiṣyāmaḥ).

\textbf{Ques:} If it questioned as to how the above mentioned meaning of this sūtra is arrived at then the answer is: 

\textbf{Ans:} from the next sūtra. In the next sūtra (īkṣaternāśabdam, BS.I.1.5) through the śruti “tadaikṣata” (Chānd.Up. 6.2.3), Brahman will be established (brahma prasādhayiṣyati).  Therefore since the purpose of the commencing sentence of that sūtra (śruti): i.e. “sadeva somyedamagra āsīt” (Chānd. Up.6.2.1) is relevant, therein it is conjectured (inferred) that the same meaning is intended here as well (tatratyaprayojanenātrocyata ityunnīyate); this is out of a desire to know (ākāṅkṣitatvāt). By that utterance, existence being present in pralaya and the world abiding as well, it is shown that the cause of the world, which will be mentioned, is in the immutable Brahman as well (tena ca vākyena pralaye sadanugatatayā jagadavasthānena vakṣyamāṇajagatkāraṇatvam kūṭasthasyāpi brahmaṇah upapāditam). Therefore by this sūtra as well (BS.I.2.4), that same view has been accepted due to that reason alone, as it is correct to do so (ato’nenāpi sūtreṇa tenaiva hetunā tadevopapādyata ityucitam).

Moreover in the commencing verse “janmādyasya yato’nvayāditarata\-śca…” of the Bhāgavata Purāṇa (Bhā.P I.1.1) also, due to the word (anvaya) meaning ‘following’, Brahman being the cause of the world has been accepted. Therefore, here (also) it is proper to have the same meaning as it is there. The meaning of that sentence (Bhā.P I.1.1) is as follows:  Since this world accompanies (Brahman) in dissolution and in evolution it is also separated from it; I meditate on that Truth (Brahman) which is the originator/material cause (yadupādānakam) of this (world) (and) from which the birth etc of this world occurs.   

Desiring to know the usage of the prefix ‘sam’ (upasargāt) in this sūtra  (tattu samanvayāt), (one finds that) there is the special intention to understand it as a mutual association.\footnote{The mutuality is between the world and Brahman i.e. the world is in Brahman and Brahman is in the world as well. This is explained by Bhikṣu himself.} There are śruti statements that declare this mutual inclusion like: “sarvabhūtādhivāsam yad\break bhūteṣu vasatyapi” (Brah.Bind.Up. verse 22); “sarvānugrāhakatvena tadasmyaham vāsudevaḥ” (ibid.verse 23); “yatparam Brahman sarvā\-tmā sarvasyāyatanam mahat” (Kaiv.Up. 16); “yaḥ sarvasminstiṣṭhan sarvāntara” (not traced). There are also (similar) smṛti sayings like: “yadā bhūtapṛthagbhāvamekasthamanupaśyati…brahma sampadyate tadā” (not traced); “mayā’dhyakṣeṇa prakṛtiḥ sūyate sacarācaram” (Gītā. 9.10). In the Brahmāṇḍa P. also the way the world is embedded in Brahman is mentioned as: “sthūlam vilāpya karaṇe karaṇa nidāne…tam tvām harim vimalabodhaghanam namāmi”. The meaning of this is as follows: When the yogī is reflecting (meditating) on the nirguṇa Brahman, at the start, he causes the absorption, through the mind, of the gross instrumental sevenfold change such as mahat etc.,\footnote{Mahat, asmitā, ahaṅkāra, manas, five tanmātras of the senses of knowledge (jñānedriyas), five tanmātras of the karmendriyas (senses of action), and five tanmātras of the gross element (bhūtas).} of the form of change in prakṛti (nirguṇabrahmacintanakāle yogini ādau sthūlam vikārajātam karaṇe mahadādi  saptake prakṛtivikṛtirūpe manasā vilāpayanti); their instrumentality is because they are the path for prakṛti to create gross products (teṣām karaṇatvam ca sthūlotpādane prakṛterdvāratvāt).Then that collection of instruments is made to be absorbed “nidāne”= in prakṛti (prakṛtau), the first cause (mūlakāraṇe); it means that he meditates on its absorption. 

This (method) is applied everywhere. And that eternal prakṛti having the three guṇas as constituents, is also absorbed in Brahman the pure consciousness, like a dead serpent lying in a hole, when it is devoid of instrument and cause (karaṇakāraṇavivikte) since it is without any action. Having absorbed in this manner the inanimate he absorbs himself\footnote{This applies to all yogis, SY.} as well through the mind due to the non-separation of the part from the whole; this is like the rivers etc., getting merged with the ocean. Since the limitation is absorbed the changes it brought about is removed. “yaminaḥ” (in the above smṛti quotation) stands in general for anyone who controls the senses (yamina iti puruṣasāmānyopalakṣaṇam). Here absorption does not signify total destruction (atyantocchedaḥ) but the changes being situated without any differentiation in prakṛti (kintu vikārāṇām prakṛtāvyaktayā’vasthānam). So also one should understand that prakṛti and puruṣa are situated in Brahman, actionless (nirvyāpātayā) as in the stage of deep sleep.\footnote{This is the way that Bhikṣu interprets the Vedānta liberation. This is an accommodation to his commitment to SY philosophy.} Through such examples of śruti as: “madhu madhukṛtā…rasam gamayanti” (Chānd.Up. 6.9.1); “te yathā tatra vivekam labhante” (ibid.6.9.2) it is known that at the time of dissolution all things exist in Brahman in a state of non-separation (brahmaṇyavibhaktatayā sarvavastvavasthānasiddheḥ). This is also known from smṛti sayings such as: “brahmāpyeti prapañco’yam rūpam hitvā tu vaikṛtam…apratarkyamavijñeyam prastuptamiva sarvataḥ”.

\dev{ब्रह्मचेतनस्य तु लयो नास्ति । “आत्मा वा इमेक एवाग्र आसीन्नान्यत् किंचनमिषदि” त्यादिश्रुतिभिर्ब्रह्मणो विश्वावभासनरूपव्यापारस्य प्रलयेऽपि सत्त्वात् । ईश्वरोपाधौ विश्वाकारवृत्तेर्नित्यत्वात् इच्छाकृतिवत्, अन्यथा विश्वनिर्माणानुपपत्तेः ।}

\dev{ईश्वरे ज्ञानेच्छादीनां कर्तृभावेऽपि स्वयमुत्पत्त्यङ्गीकारे प्रकृतिस्वातन्त्र्यप्रसङ्गः ईश्वराभ्युपगमवयर्थ्यं च, सर्गाद्युत्पन्नजीवस्यैव स्वयमिच्छाद्युतत्पत्त्यभ्युपगमौचित्यात् । न चैवं प्रलयेऽपीश्वरोपाधौ वृत्तीच्छाद्यभ्युपगमे तेषां तदाधारप्रकृतेश्च कथं लय उपपद्यतेति वाच्यम्, सृष्टिप्रभृतिव्यापारशून्यतया मृतसर्पवत् ब्रह्मचेतनेऽवस्थानादिति । ब्रह्म\-सूत्रविज्ञानामृतभाष्ये-}

\dev{नन्वीश्वरस्योपाधिरस्तीत्यत्र किं प्रमाणमिति चेत्, उच्यते—}
\begin{verse}
\dev{“कार्योपाधिरयं जीवः कारणोपाधिरीश्वरः ।}\\
\dev{न तस्य कार्य करणं च विद्यते, स्वाभाविकी ज्ञानबलक्रिया ।।}\\
\dev{मायां तु प्रकृतिं विद्यान्मायिनं तु महेश्वरम् ।}\\
\dev{अस्यावयवभूतैस्तु व्याप्तं सर्वमिदं जगत् ।।”}
\end{verse}
\dev{इत्यादिश्रुतयः । कौर्मवाक्यञ्च—}
\begin{verse}
\dev{एका शक्तिः शिवैकोऽपि शक्तिमानुच्यते शिवः ।}\\
\dev{शक्तयः शक्तिमन्तोऽन्ये सर्वे शक्तिसमुद्भवाः ।।}\\
\dev{यामधिष्ठाय योगेशो देवदेवो निरञ्जनः ।}\\
\dev{प्रधानार्थं जगत्कृत्स्नं करोति विकरोति च ।}\\
\dev{प्रधानपुरुषे सैषा महादेवैकसाक्षिणी ।}\\
\dev{सदा शिवा वियन्मूर्तिविश्वमूर्तिरमूर्तिका ।।}\\
\dev{शक्तिशक्तिमतोर्भेदं पश्यन्ति परमार्थतः ।}\\
\dev{अभेदं चानुपश्यन्ति योगिनस्तत्त्वचिन्तकाः ।। इति ।}
\end{verse}
\dev{तथा—}
\begin{verse}
\dev{प्रोच्यते परमेशानो यः शुद्धोऽप्युपचारतः ।}\\
\dev{प्रसीवतु स ना विष्णुरात्मा यः सर्वदेहिनाम् ।।}\\
\dev{चतुर्विंशतितत्त्वानि मायकर्मगुणा इति ।}\\
\dev{एते पाशाः पशुपतेः क्लेशाश्च पशुबन्धनाः ।।}\\
\dev{एतेषामेव पाशानां माया कारणमुच्यते ।}\\
\dev{मूलप्रकृतिरित्युक्ता सा शक्तिर्मयि तिष्ठति ॥}\\
\dev{शिवस्य परमा शक्तिर्नित्यानन्दमयी ह्यहम् ।}\\
\dev{स्वेच्छयैवावतारोऽत्र नैव चान्यवशात्मतः ।।}\\
\dev{इच्छाशक्तिरहं राजन् ज्ञानशक्तिरहं पुनः ।}\\
\dev{क्रियाशक्तिः प्राणशक्तिः शक्तिमान् भगनेत्रहा ।।}
\end{verse}
\dev{इत्यादीनि पराशरशिवप्रकृत्यभिमानिपार्वतीनां बाक्यानि च । तथा “प्रकृष्ठसत्त्वोपादानादीश्वरस्य शाश्वतिक उत्कर्षः” इति योगभाष्यं च ब्रह्मण उपाधी प्रमाणानि । तथा, “साक्षी चेता केवलो निर्गुणश्च” द्रष्टा दृशिमात्रः” “आह च तन्मात्रम्” इत्यादिश्रुतियोगसूत्रवेदान्तसूत्राणि च । न हि चिन्मात्रस्योपाधिद्वारतां विना इच्छादिकम् “सोऽकामयत तदात्मानं स्वयमकुरुते” त्यादि प्रत्युक्तं संभवतीति । स्वाभाविकीति स्वीय भावः पदार्थः । स्वभावः प्रकृतिः “स्वभावस्तु प्रवर्तत” इति स्मृतेस्तदीयेत्यर्थः। शिववाक्ये चांशभेदेनैकस्या एव प्रकृतेर्द्वित्वमभिप्रेत्य चतुर्विंशतितत्त्वाद् भेदेनेश्वरोपाधिर्मायाख्य उक्तः । जीवेश्वरोपाध्योरत्यन्तभेदे “अस्यावयवभूतैस्तु व्याप्तं सर्वमिदं जगदि” त्युक्तवाक्यस्याञ्जस्येनानुपपत्तिः । तत्र क्लेशादिवासनामलिनसत्त्वं जीवोपाधिः । तस्य नित्यत्वेऽपि बाह्यसत्त्वसंभेदेन कार्यतया परिणतस्यैव ज्ञानादिहेतुत्वाज्जीवस्य कार्योपाधिः । नित्यविशुद्धसत्त्वं चेश्वरोपाधिरिति विभागः ।}
\begin{verse}
\dev{सत्त्वादयो न सन्तीशे यत्र च (न) प्राकृता गुणाः ।}\\
\dev{स शुद्धः सर्वशुद्धेभ्यः पुमानाद्यः प्रसीदत् ।।}
\end{verse}
\dev{इति स्मृतेः ।}
\begin{verse}
\dev{प्राकृताः प्रकृतिकार्यभूता इत्यर्थः ।}\\
\dev{तत्र यः परमात्मा हि स नित्यं निर्गुणः स्मृतः ॥}\\
\dev{कर्मात्मा पुरुषो योऽसौ बन्धमोक्षः स युज्यते ।}\\
\dev{ससप्तदशकेनापि राशिना युज्यते च सः ।।}
\end{verse}
\dev{इति मोक्षधर्मादिश्च गुणाभिमानप्रतिषेधकः  न त्विच्छादिप्रतिषेधकः, ब्रह्मणीच्छादिश्रवणादिति । मायाशब्दश्च प्रकृतिवाची “मायां तु प्रकृतिं विद्यादि” त्यादिश्रुतिभ्यः ।}
\begin{verse}
\dev{“प्रकृत्या सर्वमेवेदं जगन्धीकृतं विभो ।,}\\
\dev{त्वं वैष्णवी शक्तिरनन्तवीर्या विश्वस्य बीजं परमासि मया ।,}\\
\dev{तस्य सर्वजगन्मूर्तिः शक्तिमयेति विश्रुता”   ।}
\end{verse}
\dev{इति मोक्षधर्ममार्कण्डेयकूर्मपुराणेभ्यश्च । लोके व्यामोहकशक्तिपु मन्त्रादिष्वेव माया  शब्दप्रयोगाच्चेति ।}\footnote{This is a very long passage to be quoted in toto in one place. But since the maning itself is repetitive I decided to quote it all at one stretch.}

There is no absorption of the consciousness of Brahman. “ātmā vā idameka evāgra āsīnnānyat kiñcana miṣat” (Ait.Up.1.1.) by such śruti statements it is known that even during the time of dissolution the activity of Brahman of the nature of illuminating everything exists. In the limitation of Īśvara the modification in the form of all things (world) is eternal like the appearance of desire and action, otherwise one cannot reasonably account for the creation of the world (anyathā viśvanirmāṇānupapatteḥ). Even if knowledge, desire etc., present in Īśvara has the nature of (his) being a creator, if one accepts the origin of that intrinsically by itself, it will lead to uselessness of the independence of prakṛti and also the acceptance of Īśvara becomes useless; so it is proper to accept the coming into being of desire etc., in the jīva that comes into being in evolution.

\textbf{Ques:} It cannot be said that, in this way, even when one accepts modification of desire etc., in the limitation of Īśvara during dissolution, still how can we explain the absorption of them and their support which is prakṛti ; then the answer  (is): 

\textbf{Ans:} Since there is the absence of the activity of creation etc., it is like the lying of a dead serpent skin in the consciousness of Brahman.

\textbf{Ques:} If it is asked what is the proof that Īśvara possesses limitation then the answer is: 

\textbf{Ans:} In śruti statements like “kāryopādhirayam jīvaḥ karaṇopādhirīśvaraḥ…asyāvayavabhūtaistu vyāptam sarvamidam jagat”(Tri.M.Nā.\-\break{Up.4.8}) one finds the answer. The Kūrma P. also says: “ekā śaktiḥ śivaiko’pi śaktimānucyate śivaḥ…abhedam cānupaśyanti yoginastattvacintakāḥ”; so also there are statements of Pārvatī followers who believe in agency of Parāśara, Śiva, Prakṛti\footnote{Does it mean devotees of Pārvatī? The meaning is not clear.} such as: “procyate parameśāno yaḥ śuddho’pyupcārataḥ…kriyāśaktiḥ prāṇaśaktiḥ śaktimān bhaganetrahā”.  So also the yogabhāṣya: “prakṛṣṭasattvopādādīśvarasya śāśvatika utkarṣaḥ” (YBh on YS.I.24) is proof of the limitation in Brahman. There are also śruti, YS and Vedāntasūtras like “sākṣī cetā kevalo nirguṇaśca” (Svet.Up.6.11); “draṣṭā dṛśimātraḥ” YS. II.20); ``āha ca tanmātram” (which say the same thing). Without an opening for pure consciousness like a limitation, is it possible to answer (pratyuktam sambhavati) such statements as: “so’kāmayata” (Taitt.Up.II.6.1); “tadātmānam svayamakurute” (ibid.II.7.1). The meaning of the word “svābhāvikī” is one’s own state (svīyo bhāvaḥ). ‘svabhāva’ is prakṛti and that is the meaning in the smṛti saying “svabhāvastu pravartate” (Gītā. 5.14).  According to Śiva’s words (in the above quotation) by partial division, imagining the single prakṛti as twofold, Īśvara’s limitation as māyā is differentiated from the 24 tattvas (of Sāṁkhya) which evolve from prakṛti’s sattva.

While there is absolute difference between the limitations of the jīva and Īśvara respectively, there is a contradiction with the statement “asyāvayavabhūtaistu vyāptam sarvamidam jagat” being truthful (āñjasyenā\-nupapattiḥ). Therein sattva tainted by subliminal impressions of afflictions etc.,(kleśādivāsanāmalinatvam) is the limitation of the jīva. Even though it (jīva) is eternal, due to the division of external sattva which causes change like knowledge etc., the limitation of jīva is known as ‘kāryopādhi’ (tasya nityatve’pi bāhyasattvasambhedena kāryatayā pariṇatasyaiva jñānādihetutvājjīvasya kāryopādhiḥ). Īśvara’s limitation is separate as it is eternally pure sattva. Thus the smṛti statement: “sattvādayo na santīśe yatra ca na prākṛtā guṇāḥ, sa śuddhaḥ sarvaśuddebhyaḥ pumānādyaḥ prasīdatu”. “prākṛtā guṇāḥ” (above) means those that have become effects of prakṛti. 

In the following Mokṣadharma.P saying : “atra yaḥ paramātmā hi sa nityam nirguṇaḥ smṛtaḥ…sasaptadaśakenāpi rāśinā yujyate ca saḥ”there is the rejection of agency associated with guṇas but not the rejection of desire etc., as one hears of desire etc., in Brahman. The word māyā denotes prakṛti as seen in śruti: “māyām tu prakṛtim vidyāt…” (Guhya.Kā.Up.15). This is also known from the statements in the Mokṣa.P, Mārkaṇḍeya P and Kūrma P (Kūrma.P.12.19; cited in Tripāṭhi p.53.fn.4) as well like: “prakṛtyā sarvamevedam jagadandhīkṛtam vibho, tvam vaiṣṇavī…paramāsi māyā, “tasya sarvajaganmūrtiḥ śaktirmāyeti viśrutaḥ”. The word māyā is also used in the world in such mantras alone that denote powers of confusion (loke vyāmohakaśaktiṣu mantrādiṣveva māyaśabdaprayogācceti).

\dev{ननु सत्त्वस्य कथमिच्छादिधर्मकत्वमुपपद्येत सत्त्वादीनां सुखादिरूपत्वेनाद्रव्यत्वादिति चेन्न, “सत्त्वं लघु प्रकाशकमिष्टमुपष्टम्भकं चलं च रजः । गुरु वरणकमेव तमः प्रदीपवच्चार्थतो वृत्तिः ।।” इति सांख्यकारिकायां गुरुत्वादिगुणैः सत्त्वानां द्रव्यत्वसिद्धेरिति । तदेवं समन्वयसूत्रेणाविकारिचिन्मात्रस्यापि ब्रह्मणोऽधिष्ठानत्वेन जगदुपादानत्वमुपपादितम् । इदमेव मूलकारणत्वं (त्व) त्व [स्व] तन्त्रकारणत्वादिशब्दैरुच्यत इति ।। ननु तथापि “तदात्मानं स्वयमकुरुत, तन्नामरूपाभ्यामेव व्याक्रियत” इति श्रुतिभ्यामुक्तयोर्विकारकर्तृत्वयोरनुपपत्तिरिति चेन्न, तादृशवाक्यानामौ\-पाधिकविकारकृत्ययो (कर्तृत्वयो) स्तात्पर्यात् । जीववत् परमात्मनोऽपि शरीरद्वयवत्त्वात् “यस्य सर्वे शरीरमि” त्यादिश्रुतीनां गौणत्वानौचित्यात्।}

\dev{तत्र कृत्याधारतया नित्यं विशुद्धसत्त्वं सूक्ष्मशरीरं, साक्षात् प्रयत्नजन्यक्रियारूपचेष्ट}

\dev{अयतया चेतरसत्वादयः स्थूलशरीरम् “चेष्टाश्रयः शरीरमि” ति न्यायसूत्रात्। चेष्टाश्रयत्वं सुखरूपार्थाश्रयत्वं वा शरीरत्वमिति तरसूत्रार्थः । शाखेषु चेन्द्रियाश्रयरूपमेव ब्रह्मणः शरीरं प्रतिषिध्यते अनर्थहेतुत्वात् कार्यत्वाच्चेति । ]}

\dev{परे तु ब्रह्मणः साक्षादुपादानत्वेऽपि कार्याणां शुक्तिरजतवद् विवर्तरूपत ब्रह्मणो न विकारित्वमित्याचक्षते । तथाहि कार्याणि तावत् परमार्थतो न सन्त्येव “वाचारम्भणं विकारो नामधेयं मृत्तिकेत्येव सत्यं, न निरोधो न चोत्पत्तिर्न बद्धो  न  साधकः । न मुमुक्षुर्न वे मुक्त इत्येषा परमार्थता” इत्यादिश्रुतेः ।।}
\begin{verse}
\dev{तस्माद् विज्ञानमेवास्ति न प्रपञ्चो न संसृतिः ।}\\
\dev{अज्ञानेनावृतं लोके विज्ञानं तेन मुह्यति ।।}\\
\dev{एवं छत्रशलाकादिपृथग्भाधो विमृष्यताम् ।।}\\
\dev{क्व यातं छत्रमित्येष न्यायस्त्वयि तथा मयि ।।}
\end{verse}
\dev{इति कूर्मविष्णुपुराणादिभ्यः ।।}
\begin{verse}
\dev{गुणानां परमं रूपं न दृष्टिपथमृच्छति ।}\\
\dev{यत्तु दृष्टिपथं प्राप्तं तन्माये (यै) व सुतुच्छकम् ॥}
\end{verse}
\dev{इति पातञ्जलभाष्यस्थात् व्यासधृतवाक्याच्च । अतएव न्यायाचायरपि विकारमिथ्यात्वं विचारतः प्रतिपादितम् “बुद्ध्या विवेचनात्तु भावानां याथात्म्यानुपलब्धिस्तन्त्वपकर्षणे पटसद्भावानुपलब्धिवत् तदनुपलब्धिः, स्वप्नविषयाभिमानवयं प्रमाणप्रमेयाभिमानः, मायागन्धर्वनगरमृगतृष्णावद्वा, बुद्धिश्चैवं  निमित्तसद्भावोपलम्भात्, तत्त्वप्रधानभेदाच्च मिथ्याबुद्धेद्वैविद्योपपत्तिरि” त्यादिसूत्रैरिति । तत्त्वतो मिथ्याबुद्धिः कार्येषु, कार्याणा व्यावहारिकसत्त्वेऽपि परमार्थतोऽसत्त्वात् । प्रधानतो मुख्यतस्तु मिथ्याबुद्धिः शशशृङ्गादिषु, तेषां सर्वथैवासत्त्वादिति विभाग इत्यर्थः । एतेषां सूत्राणां पूर्वपक्षतया व्याख्यानमपव्याख्यानमेव, एषां सूत्राणां श्रुतिस्मृत्यनुसारित्वात् । स्पष्टं सिद्धान्तसूत्रानुपलम्भाच्चेति।}

\dev{यच्च द्वितीयाध्याये गौतमाचार्यैरवयवी । व्यवस्थापितः, तद्वव्यवहारत एव। उदाहृतसूत्राणि च तत्त्वज्ञानपरीक्षाप्रकरणस्थानीति न पूर्वापरविरोधाशङ्का । तथा चोक्तम्—“न सूरयो हि व्यहारमेतं तत्त्वावमर्षेण सहामनन्ति” इति । तस्मात् श्रुतिस्मतिन्यायैर्विकारमिथ्यात्वं सिद्धम् । अतो न मिथ्याविकारैर्विवर्तरूपैर्ब्रह्मणो निर्विकारचिन्साञतायाः पारमार्थिक्या विरोध इति । अतो मदुक्तावतारिकां परित्यज्य स्वकल्पितसिद्धान्तानुसारेण सुत्रमिदं तेऽन्यथा व्याचक्षते ।}

\dev{तत्र तां व्याख्याम् पश्चान्निराकरिष्यामः ।}

\textbf{Ques:} But then how is it proper for sattva etc., to have attributes like desire etc., since sattva etc., being of the nature of pleasure etc., are not substances (adravyatvāditi); then the answer is:

\textbf{Ans:} It is not so; In the SK “sattvam laghu…pradīpavaccārthato vṛttiḥ” (SK.13) by associating qualities such as heaviness etc., to the guṇas they are establishd as substances. Similarly through the ‘samanvaya’ sūtra (BS.I.1.4) Brahman of the nature of changeless pure consciousness is also mentioned as being the material cause of the world because of being its support. Sayings such as “idameva mūlakāraṇatvam svatantrakāraṇatvādiśabdairucyata iti” (also support that).

\textbf{Ques:} Even then, through such śruti statements as “tadātmānam svayamakuruta” (Taitt.Up.2.7); “tannāmarūpbhyāmeva vyākriyata” which\break mention that change and creation (agency), will not stand to reason; then the answer is:

\textbf{Ans:} that is not so, as such sentences are intended to denote change and creation (agency) of the limitation. Just like the jīva, the paramātman also having two bodies, it is not correct to interpret statements like “yasya sarve śarīram” as having a secondary sense. Therein, being the support for action, the subtle body is eternal/permanent and of the essence of pure sattva; the other sattva etc., is the gross body as it is the supporter of movement in the form of effort which gives rise to action (sākṣāt prayatnajanyakriyārūpaceṣṭāśrayatayā cetarasattvādyaḥ sthūlasarīram). This is learnt from the NS “ceṣṭāśrayaḥ śarīram” (The complete sūtra is “ceṣṭendriyārthāśrayaḥ śarīram” NS.I.1.11). The\break meaning of that sūtra is ‘being a body means being the support of effort or being the support of objects like pleasure’. In the śāstras the body of Brahman in the form of supporting the senses alone is rejected as it is useless and also because it has action (kāryatvācca).

Others (advaitins mainly) say that though (Brahman is) the material cause, the effects being like the mother of pearl being mistaken for silver due to superimposition, Brahman has no change (in truth). Thus in truth there are no effects according to śruti: “vācāraṁbhaṇam vikāro nāmadheyam mṛttiketyeva satyam” (Chānd.Up.6.1.4); “nanirodho na cotpattirna baddho na ca sādhakaḥ” (Ātma.Up. verse 31); “na mumukṣurna vai mukta ityeṣā paramārthatā” (Ātma.Up. prose 3).\footnote{These references are from Upaniṣadvākyamahākośaḥ.} So also there are statements in the  Kūrma.P,Viṣṇu P, and others like: “tasmād vijñānamevāsti…viñnānam tena muhyati” (Kūr.P.2.28; cited in\break Tri.p.54.fn.1);  “evam cchatraśalākādipṛthakbhāvo…yātam cchatramityeṣa nyāyastvayi tathā mayi” (Viṣ.P.2.13.2. ibid.fn.2). This is also understood from Vyāsa’s bhāṣya statement: “guṇānām paramam rūpam\-…tanmāyeva sutucchakam” (VyBh. under YS.4.13). That is the reason why Nyāya ācāryas reflecting on the falsehood of change mention: “buddhyā vivecanāttu bhāvānām...paṭasadbhāvānupalabdhivat tadanupalabdhiḥ” (NS.4.2.26); “svapnaviṣayābhimānavadyam pramā\-ṇaprameyābhimānaḥ” (NS.4.2.31); “māyāgandharvanagaramṛgatṛṣṇi\-kāvad vā” (ibid. 4.2.32); “buddheścaivam nimittasadbhāvolalaṁbhāt” (ibid.4.2.36). ``tattvapradhānabhedācca mithyābuddherdvaividhyopapattiḥ'' (ibid. 4.2.37). In truth there is delusion with regard to effects/objects produced (kāryeṣu); even though the objects exist in the world of activity (vyāvahārikasattve’pi) in truth they do not exist. The main delusion from pradhāna is in instances like the horn of a rabbit; the division is because they are non-existent at all times (pradhānato mukhyatastu mithyābuddhiḥ śaśaśṛṅgādiṣu, teṣām sarvathaivāsattvāditi vibhāga ityarthaḥ). The interpretation of these sūtras from the pūrvapakṣa position is (in fact) defective interpretation;  since these sūtras follow śruti and smṛti. It is clear that it is (due to) lack of understanding the doctrine (siddhānta) of the sūtras.

The establishment of the whole in the second adhyāya (of the NS) by ācārya Gautama is only from the point of view of worldly activity.\footnote{Perhaps the reference is to sūtras NS. I.2.33-37} The sūtras used as examples are also situated in the section investigating the knowledge of what is the truth, (and) so there can be no doubt regarding the contradiction of what is stated earlier with what is said later (na pūrvāparavirodhāśaṅkā). Thus it is said: “na sūrayo hi vyaharametam tattvāvamarṣeṇa sahāmananti”. Thus through śruti and smṛti reasoning/logic the falsity of change (in Brahman) is established. Therefore through false changes of the form of superimposition on Brahman (who is) of the nature of changeless consciousness there is no contradiction in truth (ato na mithyāvikārairvivartarūpairbrahmaṇo nirvikāracinmātratāyāḥ pāramārthikyā virodha iti). Thus abandoning the introduction given by me, in consonance with their imagined doctrine, they explain this sūtra in a different manner. Therein we shall reject that explanation later.

\dev{यत्तावद् विकारासत्यत्वमुक्तं तदस्माभिरप्यभ्युपगम्यते, कूटस्थनित्यस्यैव पारमार्थिकसत्त्वात् सर्वथैवासत्ताविरहात्, न तु प्रकृतितद्विकारयोः ।}
\begin{verse}
\dev{“यत्तु कल्पान्तरेणापि नान्यसंज्ञामुपैति वै ।}\\
\dev{परिणामादिसंभूतां तद्वस्तुवृत्तयः तच्च$^{1}$ ॥}\\
\dev{वस्त्वस्ति किं कुत्रचिदादिमध्यपर्यन्तहीनं सततैकरूपम् ।}\\
\dev{यच्चान्यथा त्वं द्विज याति भूयो न तत्तथा तत्र कुतो हि तत्त्वम् ।।}
\end{verse}
\dev{इति विष्णुपुराणादिभ्यः ।}

\dev{विशेषस्त्वियान्, यद् भवद्भिर्विकाराणां शुक्तिरजततुल्यत्वमिष्यतेऽस्माभिस्तु असत्तामात्रांशे शुक्तिरजतस्वप्नादेर्दृष्टान्तत्वमिष्यते न त्वत्यन्तासत्त्वेऽपि, “वैधर्म्याच्च न स्वप्नादिवदि” त्यागामिसूत्रात्, स्वाप्नवस्तूनामपि कनककुण्डलयन्मनः परिणामतया तुच्छत्वस्य निराकरिष्यमाणत्वाच्च । बाह्यविषयसत्त्वासत्त्वाभ्यामेव च जाग्रत्स्वप्नयोर्वैधर्म्यमिति । “सदसत्ख्यातिर्बाधाबाधाम्यामि” ति कपिलसूत्रादपि प्रपञ्चस्य नात्यन्ततुच्छत्वम् । सदसद्रूपत्वमेव च व्यावहारिकसत्त्वम् । इदमेव च व्यावहारिकसत्त्वं नैयायिकादयोऽनित्यत्वपरिणामित्वलयवत्त्वादिकं वदन्ति । प्रपञ्चस्यानिर्वचनीयत्वं तु पारमार्थिकसदसद्भ्यां कूटस्थशशशृङ्गाभ्यां विलक्षणत्वमिति  ।}

\dev{ननु विरुद्धं सदसत्त्वं कथमेकत्रोपपद्येतेति चेन्न, एकधर्मेण सत्त्वशायां परिणामिवस्तुनामतीतानागतधर्मेणासत्त्वात् । तथा हि वक्ष्यत्याचार्यः “सत्त्वाच्चावरस्य, असदव्यपदेशादिति चेन्न धर्मान्तरेण वाक्यशेषादि” ति सूत्राभ्याम् । अत एव गौतमाचार्योऽपि मिथ्याबुद्धेर्द्वैविध्यमुक्तवान्। घटादयो हि अनागताद्यवस्थासु व्यक्ताद्यवस्थाभिर्बाध्यन्त इति घटादयो मिथ्याशब्देनोच्यन्ते । विद्यमानधर्मैश्च तदानीं न बाध्यन्त इति सत्या इत्यप्युच्यन्ते । एतेन सत्कार्यवादोऽप्युपपादितः, सर्वेषां सदसद्रूपत्वादिति । या च वाचारम्भणश्रुतिः सा अवयवविभागे सति विकाराणां नाममात्रत्वं वदति, उपादानस्यैव च तदपेक्षया तदानीं सत्यत्वं वदति । अन्यथा मृद्विकारादि दृष्टान्तासिद्धिदोषात् । “न निरोध” इति श्रुतिरपि पारमार्थिकसत्यत्वमेव प्रतिषेधति ।}

\dev{आत्मनोऽनिरोधादिकं वा प्रकरणात् । एवं न्यायसूत्राण्यपि नित्यत्वरूपमेव याथार्थ्यं निराकुर्वन्ति, अन्यथाकार्यकारणभावादिप्रतिपादनविरोधात् । यानि च वाक्यानिप्रपञ्चज्ञानस्य भ्रमत्वं वदन्ति, तान्यपि कारणेषु आद्यन्तयो विकाराभावादुपपद्यन्ते। तद्भाववति तत्प्रकारकज्ञानस्यैव भ्रमत्वादिति । एतेन—}
\begin{verse}
\dev{सद्भाव एषो भवते मयोक्तो, ज्ञानं यथा सत्यमसत्यमन्यत् ।}\\
\dev{एतच्च यत् संव्यवहारभूतं, तत्रापि चोक्तं भुवनाश्रितं ते ।।}\\
\dev{ज्ञानस्वरूपमत्यन्तनिर्मलं परमार्थतः ।}\\
\dev{तथैवार्थस्वरूपेण भ्रान्तिदर्शनतः स्थितम् ।।}\\
\dev{चिदिहास्ति च चिन्मात्रमिदं चिन्मयमेव तु ।}\\
\dev{चित्त्वं चिदहमेते च लोकाश्चिदिति संग्रहः ॥}
\end{verse}
\dev{इति विष्णुपुराणादिवाक्यान्यपि व्याख्यातानीति बोध्यम् ।}

The mention of change as being false is also accepted by us, as the existence of absolute truth belongs only to the eternal immutable one which is at all times devoid of non- being and it does not belong to prakṛti and its transformations/changes.

Viṣṇu P statements like: “yattu kalpāntareṇāpi…tadvastuvṛttayaḥ tacca”; “vastvasti kim kutracidādimadhyaparyantahīnam…yāti bhūyo na tattathā tatra kuto hi tattvam” (support this).

Moreover there is this which is special: whereas you desire to call changes as being like the appearance of silver in mother of pearl, we desire to equate the non-being part alone to examples such as mother of pearl, silver, dream etc., and not to absolute non-existence due to the forthcoming sūtra: “vaidharmyācca na svapnādivat” (BS.II.2.29).\footnote{In other words there is only denial of mother of pearl being silver or dream state being waking state and there is no absolute denial of all things.} Since like gold earrings (change of gold into earrings) there is change of the mind even for dream objects the rejection of their insignificance is to be rejected. The difference between waking and dreaming is only the existence and non-existence of outside objects.\footnote{In the waking stage the objects are very much existent outside of oneself whereas in a dream state the objects are within the mind itself and they do not exist outside.} The world not being totally insignificant is also declared in Kapila’s sūtra as: “sadasatkhyātirbādhābādhābhyām”.     Worldly existence is only of the nature of being (in existence) and not-being. It is this worldly existence that Naiyāyikas and others describe as being non-eternal, having change and possessing an end (idameva ca vyavahārikasattvam naiyāyikādayo’nityatvapariṇāmitvalayavattvādikam vadanti). The indescribable nature of the world is the difference between being in truth and being false, like the difference between what is immutable and a hare’s horn.

\textbf{Ques:} But then how can existence (being) and non-existence (non-being), which oppose each other, stay in the same place? Then the answer is: 

\textbf{Ans:} That is not so; when the transformed object exists in one form, it does not exist in the state of what is past and what is the future.\footnote{This can be compared to YS. 3.13. and YS.3.14 : “bhūtendriyeṣu dharmalaksaṇāavasthāpariṇāmā vyākhyātāḥ”. (also see YS. 3.9-16) (see also Rukmani:1987 pp.15 -74)} Thus the ācārya (Bādarāyaṇa) will state: “sattvāccāvarasya” (BS.II.1.16)\footnote{The (cause and effecr are non-different) because the posterior one has earlier existence (in the cause) (trans.Swami Gambhirananda)} and “asadvyapadeśāditi cenna dharmāntareṇa vākyaśeṣāt” (ibid. II.1.17)\footnote{If it be argued that the effect did not exist before creation since it is declared (in the Upaniṣad) as noneexistent then we say, no, because from the complementary portion it is known that the word is used from the standpoint of differences of characteristics. (trans.ibid.)}. That is the reason why ācārya Gautama has explained false idea as being of two types; pots etc., in the state of ‘not as yet come into being’ etc., are obstructed by ‘the state of  the manifested’ (pots) etc., therefore pots etc., are spoken of as unreal; at that time, by also not being obstructed by the existent dharmas, it is also mentioned  that the other (dharmas like past and not as yet manifested) are existing (within it)( satyā). By this reasoning the doctrine of the effect being present in the cause (satkāryavādaḥ) is also mentioned since all things have the state of being and non-being. 

The śruti which mentions all things being just name and form (i.e.false) occurs in the section dealing with parts (sā avayavavibhāge sati) only state only the names of the changes; in relation to that it mentions the reality at that time of the material cause (upādānasyaiva ca tadapekṣayā tadānīm satyatvam vadati). Otherwise it will suffer from the defect of non-estalishment of examples such as changes of clay (into pot etc). The śruti “na nirodha” (Gauḍ.Vaithya.32) also rejects only possessing absolute truth (of phenomena). Or according to the context it can mean that ātman has no bondage (ātmano’nirodhādikam vā prakaraṇāt). In this way many nyāya sūtras also reject truth in the form of being permanent, otherwise there will be contradiction in mentioning the state of being a cause or effect (nyāyasūtrāṇyapi nityatvarūpameva yāthārthyam nirākurvanti, anyathākāryakāraṇabhāvādipratipādanavirodhāt). Those sentences which state worldly knowledge as a delusion, can also happen due to the absence in change between the beginning and end in the causes; this is the delusion caused, when in the absence of something there is the knowledge of something like that (tadabhāvati tatprakārakajñānasyaiva bhramatvāditi). By this the statements in the Viṣ.P like “sadbhāva eśo bhavate mayokto…tatrapi coktam bhuvanāśritam te”;  “jñānasvarūpmatyantanirmalam paramārthataḥ…\-\break{cittvam} cidahamete ca lokāściditi saṅgrahaḥ” should be understood as explained.

\dev{ज्ञानं ज्ञानस्वरूपः परमात्मा, स एव सत्यः जीवाश्चांशतया अंशिन्येकीभूताः । अथवा लयवत्त्वेन परमात्मापेक्षया तेऽप्यसन्तः “यथा वृक्षो वनस्पतिस्तथैव पुरुषो मृषे” ति श्रुतेः । यदपि “न यत् पुरस्तान्न च तत्परस्तान्मध्ये च तत्र व्यपदेशमात्रमि” त्यादिवाक्ये प्रपञ्चस्य कालत्रयेऽप्यसत्त्वमुक्तं तदाद्यन्तयोरसत्त्वेन प्रतिक्षणपरिणामसिद्ध्या पूर्वपूर्वरूपैः सदैवा सत्त्वमभिप्रेत्य । तथा च स्मर्यते—}
\begin{verse}
\dev{नित्यदा ह्यङ्गभूतानि भवन्ति न भवन्ति च ।}\\
\dev{कालेनालक्ष्यवेगेन, सूक्ष्मत्वात्तन्न दृश्यते ।। इति ।}
\end{verse}
\dev{ये तु रज्जुसर्पादिवत् प्रपञ्चस्यात्यन्ततुच्छत्वमिच्छन्ति, ते तु बौद्धप्रभेदा एव, “मायावादमसच्छास्त्रं प्रछन्नं बौद्धमेव वा” इत्यादिपद्मपुराणवाक्यात् “असत्यमप्रतिष्ठं ते जगदाहुरनीश्वरमि” त्यादिगीतावाक्याच्च । किं च बन्धमोक्षादिकं सर्वं मिथ्येति वेदान्ता आहुरिति गुरुमुखादापाततः श्रवणादेवाकृतसाक्षात्कारस्यापि शिष्यस्य सर्वत्रैवानाश्वासात् श्रवणमननादौ साक्षात्काराय प्रवृत्तिरेव न स्यान् महानास्तिकत्वसंपादनात्, बह्वायाससाध्ये फलनिश्चयस्यैव प्रवृत्तिहेतुत्वादिति । अपि च चैतन्यातिरिक्तस्य सर्वस्यायन्तासत्त्वं येन प्रमाणेन साधनीयं, तत्सदसद्वा ? आद्ये तेनैव सर्वमिथ्यात्वबाधः, अन्त्ये असतोऽप्यर्थसाधकत्वे असता प्रमाणेन सर्वसत्यत्वमपि सिध्यतु । यच्च तैरुच्यते—अस्माभिः पराभिप्रेतप्रमाणैरेव पराभिप्रेतार्थाः खण्ड्यन्ते स्वमते च प्रमाणाभावादेव न किञ्चित् सिध्यतीति, तदप्यसत्, प्रमाणाभावेऽप्यर्थसंशयस्यैवोदयात्, द्वैतयोः  (द्वैताद्वैतयोः) प्रत्ययादिभिः प्रतीयमानत्वात् ।}

\dev{अत्र च कक्षानुकक्षायां ग्रन्थबाहुल्यमात्रमित्युपेक्षितम् । किं च अस्मदुक्तरीत्या सुष्ट्यादिवाक्यानामपि प्रामाण्यसंभवे भ्रमानुवादमात्रत्वकल्पनानौचित्यं चेति । यदि चानुभषरूपा सिद्धिरेव सत्तेत्युच्यते “सत्ता सर्वपदार्थानां नान्या संवेदनादृते” इत्यभियुक्तवाक्यात् । तथा चानुभवरूप आत्मैव सः, राहोः शिर इतिवद् धर्मधर्मिभावस्य विकल्पमात्रत्वादिति, तदा तादृशमसत्त्वं जडवर्गस्यास्माभिरपीष्यत एव, अर्थक्रियाकारित्वरूपस्य भावत्वरूपस्य सत्त्वस्यैवाभ्युपगमादिति । तस्माल्लयाव्यक्तत्वादिपर्यायमनित्यत्वमेवासत्त्वम् ‘सत्ता नित्यताऽभावो नाश’ इतिवृद्धवाक्यात् । अत्र प्रपञ्चमिथ्यात्ववाक्यैर्वेदान्त गोचरै “र्दृश्यमानमिदं सर्वमनित्यमिति चोच्यत” इति पद्मवाक्याच्चेति ।}
\begin{verse}
\dev{अत एव “न सन्ति पितरश्चेति कृत्वा मनसि यो नरः ।}\\
\dev{न तर्पयति तान् भक्त्या तस्य रक्तं पिबन्ति ते ।।” इति}
\end{verse}
\dev{प्रपञ्चासत्यतादृष्टिनिन्दास्मृतिपरम्परापीति दिक् ।}

“jñānam” (in the above verse)= The paramātman is of the nature of pure knowledge/consciousness; that alone is true; being parts the jīvas are one with the whole. Alternately because they are subject to dissolution in comparison to the paramātman they also do not exist; this is in accordance with the śruti saying: “yathā vṛkṣo vanaspatistathaiva puruṣo mṛṣe” (Br.Up. III.9.28). Even though in such statements as: ``na yat purastānna tatparastānmadhye ca tatra vyapadeśamātram ityādivākye'' the non-existence of the world in all three times (past, present and future) is mentioned, that is keeping in mind (abhipretya) the non-existence of it at the beginning (before creation) and the end (after dissolution); thus as it is shown that there is change every moment (pratkṣaṇapariṇāmasiddhyā),  as there is constant being of  preceding forms (pūrvapūrvarūpaiḥ sadaivāsattvamabhipretya). Thus the smṛti statement: “nityadā hyaṅgabhūtāni…sūkṣmatvāttanna dṛśyate”.

Those who desire the absolute insignificance like a serpent in the rope etc., they follow a type of Buddhist doctrine in accordance with the Padma Purāṇa saying: “māyāvādamasacchāstram pracchannam bauddhameva vā”;\footnote{This quote from the Padma Purāṇa is often used to equate Śaṅkra’s advaita Vedānta as Buddhism in disguise.} this is also in accordance with the statement in the Gītā: “asatyamapratiṣṭham te jagadāhuranīśvaram” (Gītā. 16.8). Moreover, happening to hear from the guru that Vedānta states that bondage and liberation is all a delusion, even the disciple who has not tried to directly experience (liberation) lacking trust totally, does not  engage in listening to and reflection ( on Vedānta vākyas) because of developing great disbelief in the Vedas (mahānāstikatvasampādanāt); this is because the reason for action is the firm belief that the result will be accomplished by strenuous effort (bahyāyāsasādhye phalaniścayasyaiva pravṛttihetutvāditi).

Moreover through what means of proof (yena pramāṇēna) can there be the establishment that everything other than consciousness is absolutely non-existent; (moreover does that proof) exist or does not exist (api ca caitanyātiriktasya sarvasyātyantāsattvam yena pramāṇena sādhanīyam, tatsadasadvā). If one accepts the first option (it is existent) then by that argument there is opposition to the statement that everything is false; if then it is the latter (proof is non-existent) since it has the capacity to produce results, through a non-existent proof let there be the existence of everything (antye asato’pyarthasādhakatve asatā pramāṇena sarvasatyatvamapi siddhyatu).\footnote{If the proof is non-existent then there can be absence of proof for the existence of everything, even when one sees that things do exist.} As to what is said by others that ‘we demolish the desired meaning of the others through proofs which are desired by them, and in our view nothing is proven to exist due to the absence of proof that is also not correct; even in the absence of proof as doubt arises (regarding the meaning alone), both non-duality and duality appear (to be true) through conviction.\footnote{Finally one bows down to the respective philosophical stand as the doubt is not with regard to what exists but its authenticity as truly existent or apparently existent.} 

This is ignored, thinking these are just many texts divided by sections and subsections.  Moreover, by our reasoning, since there is lack of proof for statements regarding creation etc., it is not correct to imagine just delusion by continuous repetition. If you say that existence (of proof) is of the nature of experience alone by the statement: “sattā sarvapadārthānām nānyā samvedanādṛte” (Maho.Up.5.47), then he is ātman of the nature of experience alone like the head of Rāhu,;\footnote{This is a reference to the attempt at stealing of nectar by Rāhu and when the deceit was discovered his head was severed from the body. But since he had tasted a little of the nectar his head became immortal.} one understands being the form of being the dharma and being a dharmin (possessing the dharma) as only a mental construction, (tathā cāunubhavarūpa ātmaiva saḥ, rāhoḥ śira itivad dharmadharmibhāvasya\break vikalpamātratvāditi).\footnote{The YS definition of vikalpa which is one of the modifications of the mind is “śabdajñānānupātī vastuśunyo vikalpaḥ”i.e vikalpa is purely knowledge which arises from words and has no corresponding object to which it refers (see also Rukmani, 1981, pp 75-79} Then that kind of a non-being is also desired of the non-sentient by us; it is proper to have being of the nature of existence working towards a purpose (arthakriyākāritvarūpasya bhāvatvarūpasya sattvasyāivābhyupagamātditi). Therefore the non-perma\-nence of the chain of dissolution and non-differentiation is itself non-existent by the wise saying (vṛddhavākyāt): “sattā nityatā’bhāvo nāśa”. In this context the Padma.P states that (same point) through Vedānta statements (declaring) the world as false like”dṛśyamānamidam sarvamanityamiti cocyate”. That is why such statements as: “na santi pitaraśceti…tasya raktam pibanti te” in the smṛti tradition condemn the idea of the unreality of the world.

\dev{स्यादेतत्, प्रलये ब्रह्मणि प्रकृतिद्वारा प्रपञ्चसमन्वयाभ्युपगमे “सदेव सोम्येदमग्र आसीदेकमेवाद्वितीयमि” ति ब्रह्माद्वैतश्रुतिबिरोधः, तेन वाक्येन सजातीयाविजातीयस्वगतभेदानाम् “एकमि” त्यादिपत्रयेण निरासात् । न च पारमार्थिकसत्तया ब्रह्माद्वैतं तद्वाक्यार्थ इति वाच्यम्, तथा सति सर्गकालेऽपि ब्रह्मण एव पारमार्थिकसत्यादग्र आसीदित्यतीतप्रयोगानौचित्यात् । “बहु स्यामि” त्युत्तरवाक्यासंगतेय, बहुभवनस्य वाचारम्भणतया पारमार्थिकसत्त्वाभावादिति । तस्मादनुपपन्नं सदेवेत्यादिवाक्यमिति ।}

\dev{अत्रोच्यते— “आप एवेदमग्र आसुः, ब्रह्म वा इदमग्र आसीदेकमेव तदेकं सन्न व्यभवत्तत् श्रेयोरूपमत्यसृजत क्षेत्रमि” त्यादिश्रुताबिथोकश्रुतावपि अविभागलक्षणा भेदेनाद्वैतबोधनात्, समानवाक्यानुसारेणैव तात्पर्यावधारणस्यौत्सर्गिकत्वात्, अन्यथा चेदंशब्दवैयर्थ्यात्, “सत्प्रतिष्ठा” इति वाक्यशेषस्याधाराधेयभावानुपपत्तेश्च । तथा “यथा मधु मधुकृतो निष्टिम्यन्ति नानात्ययानां वृक्षाणां रसान् समवहारमेकता रसं गमयन्ति” इत्यादि दृष्टान्तावगताविभागस्य परित्यागानौचित्याच्च। यथा च सर्गादौ पृथिव्याः जले सूक्ष्मरूपेणाविभागाद् “वैशेष्यात्चद्वाद’” इति न्यायेन “आप एवेदमग्रआसुरि” त्युच्यते, जलभेदेन च तदानीं पृथिव्यादिकमसदित्युच्यते । थथा वा क्षत्रियादिसर्गात् पूर्वं क्षत्रियादीनां ब्राह्मणशरीरे सूक्ष्मरूपैरविभागादू वैशेष्येण “ब्रह्म वा इदमग्र आसीदेकमेवे” त्युच्यते, तथैव प्राकृतप्रलये प्रकृतितत्कार्यजीयानां ब्रहणि सूक्ष्मरूपेणाविभागद वैशेष्यन्यायेन सदाख्यं “ब्रह्मैवेदमेकमेवाद्वितीयमासीदि” त्युच्यते । वैशेष्यं च स्वकार्यकारित्वेन बोध्यम् ।}

\dev{ब्रह्मचैतन्यं हि विश्वावभासनरूपं स्वकार्यं प्रलयेऽपीदानीमिवैव कुर्वदास्ते प्रकृतिपुरुषादयस्त सृष्ट्यादिलक्षणस्वकार्येभ्य उपरता एव सुषुप्तास्तिष्ठन्तीति ।पुरुषार्थकियाकारित्वमेव च लोके सत्त्वमिति व्यवह्रियते । तथा चोक्तं प्रश्नोप\-निषदि—“स यथा इमा नद्यः समुद्रायणाः समुद्रं प्राप्यास्तं गच्छन्ति भिद्येते तासां नामरूपे समुद्र इत्येव प्रोच्यते, एवमेवास्य परिद्रष्टुरिमाः षोडशकलाः पुरुषायणाः पुरुषं प्राप्यास्तं गच्छन्ति  भिद्येते चासां नामरूपे पुरुष इत्येव प्रोच्यते, स एषोऽमृतोऽकलो भवति, एतदेवा “हं परं ब्रह्म वेद नातः परमस्ती”ति । स एष परिद्रष्टा जीवोऽकलः कलाभिः सूक्ष्मशरीराख्यैः एकादशेन्द्रियपञ्चभूतरूपैः षोडशभिर्वियुक्तः सन् परमात्मा पुरुषो भवति । सूर्यकिरणवदंशानामंशिन्यविभागादित्यर्थः । “यथा नद्यः स्यन्दमानाः समुद्रेऽस्तं गच्छन्ति नामरूपे विहाय, तथा विद्वान् नामरूपाद् विमुक्तः परात् परं पुरुषमुपैति दिव्यमि” ति मुण्डकैकवाक्यत्वादिति । एवं च सति सदेवेत्यादिवाक्यस्थमेवकारादिपदत्रयं सजातीयविजातीयस्वगतविभागशून्यत्वपरमेवेति मन्तव्यम् ।   स्वगतश्च लक्षणान्यत्वरूपो विभागोऽवच्छेदभेदेनाकाशादिषु गुहाकूपश्रोत्रादिरूपैः प्रसिद्ध इति । यदेतत् प्रलयाद्वैतमुक्तम्, एतदेव विभागस्य वाचारम्भणत्वेनाव्यक्तावस्थायाश्च स्वाभाविकतया नित्यत्वेन च पारमार्थिक}$^{4}$ \dev{मित्युक्तं कौर्मादिषु—}
\begin{verse}
\dev{तस्मादद्वैतमेवाहुर्मुनयः परमार्थतः ।}\\
\dev{भेदो व्यक्तः (व्यक्तस्व) भावेन सा च मायात्मसंश्रया ।।}
\end{verse}
\dev{इत्यादिभिः । अस्यार्थः—परमात्मचैतन्याद्वैतमेव पारमार्थिकं मुनय आहः। भेदो विभागलक्षणद्वैतं तु व्यक्तकार्यरूपेण । सा च व्यक्तावस्था आत्माधिष्ठाना मायैव मायाकार्यत्वात् इन्द्रजालवत् क्षणभङ्गुरत्वादिति ।}

\textbf{Ques:} Let it be so; in dissolution when through prakṛti the world is in Brahman then it contradicts the śruti statement which states the non-duality of Brahman as: “sadeva somyedamagra āsīdekamevādvitīyam” (Chānd.Up. 6.2.1). By the three words “ekam eva advitīyam” it rejects the three kinds of differences such as sajātīya, vijātīya and svagata.\footnote{Sajātīya is the difference that exists between objects of the same class like one table being different from another table; the second is called vijātīya where for instance a tree differs from a table which are different classes of objects. Then there is what is known as svagatabheda where one can distinguish between the leaves, trunk etc., of a tree which is contained within one object itself i.e they are internal differences.} Nor can it be said that by absolute reality the meaning of the sentence is non-duality of Brahman; if that were the case then, even during the time of evolution, since only Brahman has absolute reality the usage of the past tense in “agra āsīt” will be improper. It will also not fit in with the next statement “bahu syām” (Taitt.Up II.6.1) as a lot of produced things (bahubhavanasya) are just name and form due to the absence of absolute reality. Therefore statements such as “sadeva” etc., are not correct.

\textbf{Ans:} Similar to such śruti sayings as “āpa evedamagra āsuḥ” (Bṛ.Up. V.5.1), “brahma vā idamagra āsīt” (ibid. I.4.1); “ekameva tadekam sannavyabhavattat…atyasṛjata kṣatram” (ibid.I.4.11) even in the above mentioned śruti statements, there is instruction of advaita (non-duality) of the nature of non-separation (ityādiśrutāvivoktaśrutāvapi avibhāgalakṣaṇābhedenādvaitabodhanāt). It is natural to understand the intention of the meaning by following the rule of similar sentences (having the same meaning), otherwise the use of the word “idam” (in the above śrutivākyas) is meaningless and the state of having the relationship of supporter and supported in the saying“satpratiṣṭhā” is contradicted. Therefore it is not wise to give up (the meaning of) non-separation which is known from such examples as “the bees collecting the juice from many dying trees purify the honey making it into one juice” (tathā “yathā madhu madhukṛto niṣṭimyanti nānātyayanām vṛkṣāṇām rasān samahāramekatām rasam gamyanti”).\footnote{The honey from many sources exist in a relationship of non-separation as it is not possible to separate honey gathered from one source and another source in that collective state. So also the state of Brahman.} Just as at the start of evolution earth (exists) in water in a subtle form by non-separation through the logic “vaiśeṣyāttadvāda” so also it is said that “āpa evedamagra āsuḥ”, i.e. earth etc., do not exist as separated from water at that time. Or it is like the kṣatriyas etc., prior to evolution, mentioned as existing distinctively (vaiśeṣyaṇyāyena sadākhyam) in the Brāhmaṇa body in a subtle form because of non-separation mentioned as: “brahmaivedamekamevādvitīyamāsīt ekameva” (Bṛ.Up.I.4.11). Similarly during the primary dissolution, prakṛti and its jīva effects through the logic of distinction in an avibhāga state, are known to exist in Brahman in a subtle form. That is mentioned as: “brahmaivedamekamevādvidīyamāsīt”. The distinction is to be understood by their capacity to accomplish their own duty (vaiśeṣyam ca svakāryakāritvena bodhyam). 

As for the consciousness of Brahman it continues to perform its task in the form of illuminating everything even in dissolution as it does now; whereas prakṛti and the puruṣas ceasing from their activities like creation etc., stay in a state of deep sleep (uparatā eva suṣuptāstiṣṭhantīti). It is in the world alone that there is existent the capacity to  perform activities like pursuing one’s goal (puruṣārtha)  which is pursued. Thus the Praś.Up says: “sa yathā imā nadyaḥ samudrāyaṇāḥ…sa eṣo’mṛto’kalo bhavati” (VI.5).This alone am I: “aham param brahma veda nātaḥ paramasti” (ibid. VI.7; the quote is not exactly as it is in extant Praś.Up. text)\footnote{Most extant Praś. Up texts have “tān hovācaitāvadevāhametat param brahma veda, nātaḥ paramastīti”}. “sa eṣa”= this jīva which is a  seer, “akalaḥ”= is devoid of parts called the subtle bodies of the form of the eleven sense organs and the five elements; thus separated from the sixteen (subtle bodies) puruṣa becomes the paramātman. It is like the rays which are parts of the sun, staying in a non-separated manner in the whole sun. This is stated in the Muṇḍ.Up as: “yathā nadyaḥ syandamānāḥ samudre’stam gacchanti…parāt param puruṣamupaiti divyam”(III.2.8). When it is so, it should be understood that the three words in the sentence starting with “sadeva” (Chānd.UP. 6.2.1) is indicative of the absence of the three kinds of differences (sajatīya, vijātīya and svagatabheda).\footnote{Refer to n.142 above} 

Internal difference is a separation of the form of having another characteristic; it is like separation due to the difference of the conditions; it is well known in space like a cave, well, ear etc. That which is mentioned as the non-dual state in dissolution, which is natural to the undifferentiated state, is characterized by name and form in the state of separation, and since it is eternal it is called the supreme truth in the Kūr.P etc (yadetat pralayādvaitamuktam, etadeva vibhāgasya vācāram\-bhaṇatvenāvyaktāvasthāyāśca svābhāvikatayā nityatvena ca pāramā\-rthikamityuktam kaurmādiṣu) as: “tasmādadvaitamevāhurmunayaḥ…\-sā ca māyatmasamsrayā” (Kūr.P. Utta. 2.22; cited in Tri. p.58, fn.6).  The meaning of this is ‘non-duality of the consciousness of paramātman alone is the supreme (truth); so say the munis. Difference is a duality characterized by separation of the nature of definite effect of differentiation. And that state of differentiaton/separation is māyā alone which rests on the ātman, as it is the doing of māyā since it is ephemeral like Indra’s net.\footnote{``Indra's net'' is probably a reference to the large number nets belonging to Indra which hangs over his palace. These nets had a multifaceted jewel at each vertex, which reflected in all of the other jewels. In this metaphor the idea conveyed is the utter confusion that occurs which is similar to that caused by māya.} 

\dev{ननु अद्वितीयशब्दस्याविभागार्थकत्वं श्रुतिस्मृत्योः कथमवधारितमिति चेत्, “न तु तद्वितीयमस्ति ततोऽन्यद् विभक्तमि” त्यादिश्रुतेः ।}
\begin{verse}
\dev{पृथग् विभक्ता प्रलये च गोप्ता ।}\\
\dev{अविभक्तं च भूतेषु विभक्तमिव च स्थितम् ।}\\
\dev{उभयोस्त्वविभागेन साध्यसाधनयोरिह ।}\\
\dev{विज्ञानमद्वैतमयं तद् वाग्रोधो मयोदितः ।}\\
\dev{क्षितिपरमाणवोऽनिलान्ते, पुनरपि यान्ति तथैकतां धरित्र्या ।}\\
\dev{सुरपशुमनुजादयस्तथान्ते, गुणकलुषेण सनातनेन तेन ।।}
\end{verse}
\dev{इत्यादिस्मृतेः, “अविभागो वचनादि” त्यादिसूत्रैश्चेति गृहाण । एतेन—}
\begin{verse}
\dev{ज्ञानस्वरूपमखिलं जगदेतदबुद्धयः ।}\\
\dev{अर्थस्वरूपं पश्यन्तः भ्राम्यन्ते मोहसंप्लवे ।।}
\end{verse}
\dev{इत्यादिवाक्येष्वपि ज्ञानस्वरूपे परमात्मनि प्रलीनतया तदविभक्तत्वमेव तत्स्वरूपत्वं बोध्यम् । प्रलये हि पुंप्रकृत्यादिकं ज्ञानस्वरूपेणैव रूप्यते न त्वर्थरूपेण, अर्थतो व्यञ्जकव्यापाराभावात्। यथा स्वसर्गादौ पृथिवी जलरूपेणैव व्यज्यते न तु पृथिवीरूपेण, गन्धकाठिन्यादीनां व्यञ्जकानामभावादिति । नन्वेतादृशाविभागज्ञानस्य किं फलमिति चेत्, समस्तपञ्चविंशतितत्त्वविलापनेन विविक्तब्रह्मात्मताज्ञानस्यानुरूपस्तद्भावो मुक्तिःफलमिति  दिक् ।}

\dev{ननु अविभागो लक्षणानन्यत्वं प्रलयकाल एव भवति, तेन कथं सर्गव्यवहारकालेऽपि ब्रह्माद्वैतश्रुतिस्मृतय उपपद्येरन्? यथा “ब्रह्मैवेदं सर्वम्, आत्मैवेदं सर्वम्, स एव सर्वं यद्भूतं यच्च भव्यं सनातनम्, नेह नानास्ति किंचने” त्याद्याः श्रुतयः प्रकृतितत्कार्ययोः पुरुषान्तरस्य सत्तामप्रतिषिध्यापि ब्रह्माद्वैतं बोधयन्ति । तथा स्मृतयश्च—}
\begin{verse}
\dev{“अहं हरिः सर्वमिदं जनार्दनः,}\\
\dev{नान्यत्ततः कारणकार्यजातम् ।}\\
\dev{व्यक्तं स एव चाव्यक्तं स एव पुरुषः परः ।}\\
\dev{वासुदेवः सर्वमि” त्याद्याः” इति,}
\end{verse}
\dev{उच्यते—उक्तवाक्यानि शक्तिमत्कार्यकारणाभेदेनैव ब्रह्माद्वैतं बोधयन्ति । “तदनु प्रविश्य सच्च त्याच्चाभवत्, सर्वं समाप्नोषि ततोऽसि सर्वः,}
\begin{verse}
\dev{अन्यश्च राजन् प्रबरस्तथान्यः पञ्चविंशकः ।}\\
\dev{तत्स्थत्वाच्चानुपश्यन्ति एक एवेति साधवः ।।}
\end{verse}
\dev{इत्यादिश्रुतिस्मृत्येकवाक्यत्वात् ।}

\textbf{Ques:} If it is questioned as to how the meaning of the word ‘advitīyam’ used in śruti and smṛti has been understood as non-separation, then the answer is: 

\textbf{Ans:} There is the śruti saying: “na tu taddvidīyamasti tato’nyad vibhaktam” (Bṛ.Up. IV.3.23). There is also the smṛti statement: “pṛthag vibhaktā pralaye ca goptā…guṇakaluṣeṇa sanātanena tena”.It is also to be understood from the sūtras such as: “avibhāgo vacanāt” (BS.IV.2.16). Also through such sayings as: “jñānasvarūpamakhilam…bhrāmyante mohasamplave” it is learnt that by being absorbed in the paramātman which is of the nature of pure consciousness its non-separation is itself its intrinsic nature (jñānasvarūpe paramātmani pralīnatayā tadavibhaktatvameva tatsvarūpatvam bodhyam). In dissolution puruṣa, prakṛti etc., are transformed into the form of pure consciousness (of paramātman) and not in the form of the objects, as there is the absence of activity which is indicative of proceeding from an object (arthato vyañjakavyāpārābhāvāt). This is just like when at the beginning of evolution earth is manifested in the form of water and not as earth due to the absence of distinguishing marks like smell, hardness etc. 

\textbf{Ques:} If the question is ‘what is the benefit in the knowledge of this kind of a non-separation (doctrine)’ then the answer is: 

\textbf{Ans:} By the absorption of all the 25 tattvas the state of having the corresponding knowledge of the separated Brahman (puruṣa), having the same nature of ātman is liberation (samastapañcavimśatitattvavilāpanena viviktabrahmātmatājñānasyānurūpastadbhāvo muktiḥ phalamiti dik)\footnote{There are two significant points made here: one Bhikṣu’s sticking to the evolutionary model of the 25 tattvas of Sāṃkhya and the second his idea of mukti being the realization of ātman being like Brahman. This seems to be the first kind of mukti Bhikṣu talks about in all his works.}

\textbf{Ques:} The non-separation of charateristics only happens during dissolution, so how is it that during the time of worldly evolution as well will the śruti and smṛti vākyas declaring the non-duality of Brahman be explained. 

\textbf{Ans:} Such śruti quotations as: “brahmaivedam sarvam” (Nṛsim.Up. 7.3), “ātmaivedam sarvam” (Nārā.Utta.Tā.I.5), “sa eva sarvam yadbhūtam yacca bhavyam sanātanam” (Kaiv.Up.9) , “neha nanāsti kimcana” (Kaṭh.Up. 4.11), even while not rejecting the otherness of puruṣa from prakṛti and its effects, instruct on the non-duality of Brahman. So also do the smṛti sayings like: “aham hariḥ sarvamidam janārdano…vāsudevaḥ sarvam”. It is said that the above mentioned sentences instruct the non-duality through a non-separation of the effects and the cause like the one possessing power (and the powers) (śaktimatkāryakāraṇābhedenaiva brahmādvaitam bodhayanti). “tadanu praviśya sacca tyaccābhavat” (Taitt.Up. 2.6), “sarvam samāpno’ṣi tato’si sarvaḥ”, “anyaśca rājan…eka eveti sādhavaḥ” such śruti and smṛti statements speak in one voice.

\dev{अयं च सार्वकालो ब्रह्मणि प्रपञ्चाभेदो बुद्धिपुरुषयोरिव गुणप्रधानभावेनात्यन्तसंमिश्रणरूपः संयोगादिविशेषः स्वरूपसम्बन्धविशेष वाऽस्तु, विवेकिनामपि शर्करादुग्धयोरेकीभावव्यवहारात् ।}
\begin{verse}
\dev{शक्तिशक्तिमतोर्भेदं पश्यन्ति परमार्थतः ।}\\
\dev{अभेदं वानुपश्यन्ति योगिनस्तत्त्वचिन्तकाः ।।}\\
\dev{अन्तर्यामी जगदूपी सर्वसाक्षी निरञ्जनः ।}\\
\dev{भिन्नाभिन्नस्वरूपेण स्थितोऽसौ परमेश्वरः ।}
\end{verse}
\dev{इति कूर्मनारदादिवाक्येनान्योन्याभावसंमिश्रणरूपयोर्भेदाभेदयोरेव पारमार्थिकत्व वचनाच्चेति ।}

\dev{अतएवोक्तम्—}
\begin{verse}
\dev{त एते भगवद्रूपं विश्वं सदसदात्मकम् ।}\\
\dev{आत्मनोऽव्यतिरेकेण पश्यन्तो व्यचरन् महीम् ।।}
\end{verse}
\dev{इति । एवमेव कार्यकारणयोर्धर्मधर्मिणोश्चोभयोरेव लक्षणभेदसत्त्वेऽपि संमिश्रणरूप एवाभेदो बोध्यः “सर्वं खल्विदं ब्रह्म तज्जलानि” ति श्रुतेः, “कामादिकं मन एवे” त्यादिश्रुतेश्चेति ।}
\begin{verse}
\dev{इन्द्रियाणि मनो बुद्धिः सत्त्वं तेजो बलं धृतिः ।}\\
\dev{वासुदेवात्मकान्याहुः क्षेत्रं क्षेत्रज्ञ एव च ।।}
\end{verse}
\dev{इत्यादिवाक्यानि च सर्वेषां ब्रह्मैवात्मेति बोधयन्ति नत्वत्यन्ताभेदमिति । एतेन सर्वाणि ब्रह्माद्वैतवाक्यानि व्याख्यातानि । तथा विभुत्वेन जीवानामपि सर्वाभेदप्रतिपादिकाः श्रुतिस्मृतयो व्याख्याता इति दिक् ।}

\dev{इदानीं परेषां सूत्रव्याख्यानमपि निराक्रियते । ते चैवमिदं सूत्रं व्याचक्षते—}

\dev{ननु ब्रह्मणः शास्त्रयोनित्वं न संभवति यत “आम्नायस्य क्रियार्थत्वादानर्थक्यमतदर्थानामि” ति पूर्वमीमांसायां कार्यपरत्वमवधृतम् । अतः कथमविशेषे सिद्धे ब्रह्मणि वेदान्तानां प्रामाण्यं स्यादित्याशङ्कां समाधत्ते—“तत्तु समन्वयात् । तु शब्दः शङ्काव्यवच्छेदार्थः । सिद्धस्यापि ब्रह्मणस्तत् शास्त्रयोनित्वं समन्वयादवधार्यं “सदेव सोम्येदमग्र असीदि” त्यादिवाक्यानामविधिप्रकरणस्थानामुपक्रमोपसंहारादिलिङ्गषट्केन सिद्ध एव ब्रह्मणि तात्पर्यं संबन्धावधारणादित्यर्थ इति ।}

\dev{तत्रेदमुच्यते—भवेदेवं पूर्वपक्षः सिद्धान्तश्च यदि ब्रह्मणो न विधिशेषत्वं स्यात्, न त्वेवम् “आत्मेत्येवोपासीत, आत्मा वारे द्रष्टव्यः” इत्यादिविधिशेषत्वाद् ब्रह्मण । आचार्येणापि “सर्ववेदान्तप्रत्ययं चोदनाद्यविशेषात्” इत्यादिसूत्रैः सगुणनिर्गुणसाधारण्येन सर्ववेदान्तविद्यास्वेव विधिर्वक्ष्यते, सर्वशब्दस्य सगुणविद्यापरत्वेन संकोचस्य बाधकाभावेनानौचित्यादिति । अत एव यस्यां विद्यायां विधिर्न श्रूयते तत्रापि विधिः कल्पनीयः । जैमिनिप्रदर्शितन्यायाच्च । तथाहि, पूर्वमीमांसावाक्यानि— ‘दृष्टो हि तस्यार्थः कर्माधबोधनं, चोदनेति क्रियायाः प्रवर्तकं वचनं, तस्य ज्ञानमुपदेशः, तद्भूतार्थानां क्रियार्थेन समाम्नायः, आम्नायस्य क्रियार्थत्वादानर्थक्यमतदर्थानामि” त्यादीनि । न चैतानि निर्गुणविद्यातिरिक्तपरत्वेन संकोचमर्हन्ति, बाधकाभावात् । नापि संकोचः संभवति, “आत्मा वा अरे द्रष्टव्यः”\break इत्यादिना निर्गुणप्रकरणेऽपि विधिश्रवणादिति ।}

And let this non-separation be a special relationship of contact (samyogādiviśeṣaḥ) or a special intrinsic relationship (svarūpasambandhaviśeṣo vā’stu) of the world at all times in Brahman similar to that of the intellect and puruṣa, existing like the guṇas and pradhāna in a state of a complete blending/mixing; it works like the oneness of sugar and milk even for separate things (like milk and sugar).\footnote{Suggesting perhaps that the existence of the world is a complete blending of two different things like sugar and milk. The emphasis is on the impossiblity of being able to separate one from the other such is their complete mixing.} This is also known from the following Kūr.P and Nārada.P sayings: “śaktiśaktimattorbhedam…sthito’sau parameśvaraḥ” also which mention the ultimate truth as of the form of mutual blending of separation and non-separation (of śaktimat and śakti: powers and that possessing the powers)) (anyonyābhāvasammiśraṇarūpayoḥ).

That is why it is said: “ta ete bhagavadrūpam…paśyanto vyacaran mahīm”. In this manner even though there is a difference in both the characteristics of both cause and effect, and of the characteristics and the one possessing the characteristics, one has to understand the identity (non-separation) in the form of a mixing (sammiśraṇarūpa evābhedo bodhyaḥ). Thus śruti says: “sarvam khalvidam brahma tajjalāniti” (Chānd.Up.III.14.1), “kāmādikam mana eva”. Statements such as: “indriyāṇi mano buddhiḥ…kṣetram kṣetrajña eva ca” also instruct everyone that Brahman is itself ātman and not (that there is) complete identity (na tvātyantābhedamiti). By (the above arguments) all statements of the identity of Brahman (with ātman) has been explained. Similarly śruti nd smṛti statements which advocate identity of jīvas through all-pervasiveness have been explained.

Now the explanations of the sūtras by others is being rejected.\footnote{Since PM is mentioned below, this is with reference to their argument that the Vedas are apauruṣeya and Brahman cannot be the source of the Vedas.} They explain this sūtra as follows: Brahman cannot be the source of the Vedas because: “āmnāyasya kriyāṛthatvādānarthakyamatadarthā\-nām”. Pūrvamīmāmsā interprets the Vedas as having the purpose of karma. 

\textbf{Ques:} Then, when Brahman is proven to be without any qualities (like karma) how can there be any Vedānta proof?\footnote{Since the nirguṇa Brahman cannot have any activity (karma) it cannot be the source of the Vedas.}

\textbf{Ans:} This doubt is resolved as “tattu samanvayāt”. The word “tu” stands for removal of the doubt. One needs to understand that the established Brahman being the source of the Vedas is due to their being its implicit meaning (being its intended purport). The sayings such as “sadeva somyedamagra āsīt” (Chānd.Up.6.2.1) which are found in sections not under the injunctions (avidhiprakaraṇasthānām) have established, by applying the six logical methods beginning with upakrama and ending with upasamhāra, that the intended meaning is Brahman, by determining the connection\footnote{The six means mentioned by Pūrvamīmāmsā and used extensively in justifying one’s interpretation of the meaning of Brahman from the Upaniṣad texts are: upakrama and upasamhāra statements at the commencement and end having the same meaning, abhyāsa (repetition or frequent mention of the same in various places), apūrvatā (presenting of unique evidence), phala (purpose), arthavāda (eulogy) and upapatti (reasoning supporting one’s stand)} (of Brahman with these logical tools) (saṃbandhāvadhāraṇādityarthaḥ).

\textbf{Ques:} In that connection if it is said: let it be that the siddhānta (doctrine) and its opposition be that Brahman is not what is the left over meaning of the injunction, then the answer is: 

\textbf{Ans:} then in such statements as “ātmetyevopāsīta”, “ātmā vā are dra\-ṣṭavyaḥ” (Bṛ.Up 2.4.5; 4.5.6) will be the left over meaning of the injunction. The ācārya also will mention through such sūtras as: “sarvavedāntapratyayamcodanādyaviśeṣāt” (BS.3.3.1) that injunctions are common to all vidyās both saguṇa and nirguṇa; since the word “sarva” (in the above sūtra) refers especially to saguṇa-vidyā (knowledge/\-meditation)\footnote{Since the sūtra occurs in the third pāda which deals specifically with upāsanā/\-meditation it is appropriate to undestand its meaning as such.} and since there is the absence of any restriction (of\break meaning) it is not proper (to restrict the meaning of the injunction) (sarvaśabdasya saguṇavidyāparatvena samkocasya bādhakābhāvenānaucityāditi). That is why when no injunction is heard/mentioned regarding a particular meditation one needs to imagine an injunction there as well (tatrāpi vidhiḥ kalpanīyaḥ). This is also in keeping with the logic presented by Jaiminī (in his sūtras). Thus (there are) the  Pūrvamīmāmsā statements such as: “dṛṣṭo hi tasyārthaḥ karmāvabodhanam”, “codaneti kriyāyāḥ pravartakam vacanam”, “tasya jñānam upadeśaḥ”,”tadbhūtārthānām kriyārthena samāmnayaḥ” (Jaim.MS.\break\-I.1.25), “āmnāyasya kriyārthatvādānarthakyamatadarthānām” (ibid.\break\-I.2.1). And these do not deserve to be restricted (in meaning) as they incline towards (something) other than on nirguṇa meditation and there is an absence of any obstacle (as well). Nor will there be any restriction as through the śruti statement: “ātmā vā are draṣṭavyaḥ” (Bṛ.Up.2.4.5) even in the section dealing with nirguṇa Brahman, one hears of the injunction.

\dev{स्यादेतत् निर्गुणप्रकरणस्थानि ‘द्रष्टव्य’ इत्यादीनि विधिप्रतिरूपकाण्येव न तु विधयः, रज्जुरेवायं न सर्पोऽतोऽत्र भयं त्यजेदित्यादिलौकिकवाक्यवत् । तथाहि, ज्ञातेऽज्ञाते च विधिर्न संभवति, ज्ञाते पुरुषार्थसमाप्त्या विधिवैयर्थ्यात्, अज्ञातः च विध्यर्थज्ञानानुपपत्तेः । न हि यागमजानतो यजेदिति विध्यर्थज्ञानं संभवति, विशेषणज्ञानसाध्यत्वाद् विशिष्टज्ञानस्येति । अपि च ब्रह्मज्ञाने सति समस्तप्रपञ्चस्य बाधितत्वात् किमर्थं केन करणेन कः प्रवर्तते ? । किं च यत् कर्तुमकर्तुमन्यथाकर्तुं वा शक्यते, तत्र च नयागविषयप्रतिपादनानन्तरं पुरुषस्य नियोग उचितः । न चाधिकारिणो वेदान्तश्रवणे सति ब्रह्मज्ञाने पुरुषस्यैवं स्वातन्त्र्यं संभवति । चक्षुरुन्मीलने सति सन्निकृष्टघटसाक्षात्कारपद् वेदान्तश्रवणमात्रादेव नित्यसंनिकृष्टब्रह्मात्मसाक्षात्कारस्यावश्यकत्वात् । अतोऽपि तत्त्वमस्यादिवाक्यैः प्रतिपादिते ब्रह्मात्मत्वे दर्शनं न विधानमर्हति विधातुं युज्यते, शब्देनैव साक्षात्कारोत्पत्तरिति । अत्रोच्यते—}

\dev{यदि शब्दादेव साक्षात्कारोऽसंप्रज्ञातसमाधिश्च भवति तदोक्तदोषैर्ब्रह्मज्ञाने विधिर्न स्यान्न त्वेवं “श्रुतानुमानप्रज्ञाभ्यामन्यविषया विशेषार्थत्वादि” ति योगसूत्रेण साक्षात्काररूपप्रज्ञाविषयस्य विशेषस्य शब्दानुमानागोचरत्वावगमात् । शब्दानुमाने हि सामान्यमात्रविषयके पदार्थतावच्छेदकव्यापकतावच्छेदकप्रकाराभ्यामेव तयोज्ञानजनकत्वात् अन्यथाऽतिप्रसङ्गात् ।}
\begin{verse}
\dev{ब्रह्मणि च गुणकर्मादिरूपविशेषाऽभावेऽपि उपाधिवृत्तिप्रतिबिम्बरूपेऽप्यन्ततो}\\
\dev{जीवचैतन्यादिभ्यो विशेषः योगिप्रत्ययगम्योऽस्ति, अन्यथा भेदानुपपत्तेरिति  ।}
\end{verse}
\dev{दशमस्त्वमसीत्यादिवाक्येष्वपि प्रत्यक्षसामग्र्या सहार्थः समाजः सूक्ष्मत्वेन कालक्रमानाकलनं वा। न च ब्रह्मज्ञानेऽप्यार्थसमाजादिकमस्त्विति वाच्यम्, सवासनमिथ्याज्ञानेन प्रतिवन्धतस्तत्त्वसाक्षा- त्कारासंभवात् । सामान्यगोचरस्य शब्दज्ञानस्य विशेषगोचरसाक्षात्कारप्रतिबन्धक- दोषनिरासकत्वासंभवात् । योगजधर्माणां चेतरधर्माणामिवाऽचिन्त्यन्यसामर्थ्यतया सामान्यगोचरयोगस्यापि विशेषगोचरसाक्षात्कारहेतुत्वं शास्त्रप्रामाण्यतोऽवधार्यते ।}
\begin{verse}
\dev{श्रोतव्यः श्रुतिवाक्येभ्यो मन्तव्यश्चोपपत्तिभिः ।}\\
\dev{मत्वा च सततं ध्येय एते दर्शनहेतवः ।।}
\end{verse}
\dev{इति स्मृतेः । न च प्रतिबन्धकस्य सवासनमिथ्याज्ञानस्य सत्त्वात् कथं योगेनापि साक्षात्कारः स्यादिति वाच्यम्, शास्त्रप्रामाण्येन योगजधर्मस्योत्तेजकत्वसंभवात् । शब्दानुमानज्ञानयोश्च प्रमाणाभावेन लोके व्यभिचारदर्शनेन  चोत्तेजकत्वकल्पनानवकाश इति । अपि च ब्रह्माणोऽतीन्द्रियतया न तत्र योगजधर्मं विना साक्षात्कारः संभवति ।}

\textbf{Ques:}  Let it be that sayings such as “draṣṭavyaḥ” etc., in the section on nirguṇa Brahman look like injunctions but are not injunctions (vidhipratirūpakāṇyeva na tu vidhayaḥ) like worldly utterances ‘this is only a rope not a serpent therefore give up fear’. Thus there can be no injunction for something known and unknown, for when something is known since the purpose is served (puruṣārthasamāptyā) an injunction is useless and regarding something unknown there is no logic in knowing the meaning of the injunction.  In the case of one who does not understand the meaning of sacrifice,  it is not possible to understand the meaning of the injunction ‘do not perform the sacrifice’ ; there is the knowledge of the qualified through understanding the meaning of the qualification (viśeṣaṇajñānasādhyatvād viśiṣṭajñānasyeti).  In the case of knowledge of Brahman when the entire world is affected then, for what purpose will someone act and with what means (kimartham kena karaṇena kaḥ pravartate). Moreover when it is possible (as Śaṅkara says) (in the case of a sacrifice to have the option) to perform it, not to perform it or do it differently\footnote{ŚBh. Under BS I.1.2} then, in that context, after pointing out the matter pertaining to engaging in the sacrifice it is proper for engaging in the performance (of the sacrifice). And in the case of the (Vedānta) aspirant, when (it is mentioned that) there is knowledge of Brahman through listening (to Vedānta) then puruṣa cannot have this kind of independence. Like perception of the pot that is closeby when the eyes are open, it is necessary that there is direct perception of Brahman/ātman that is permanently present closeby, by just listening to Vedānta. Moreover by such statements as “tat tvamasi” (Chānd.Up.6.8.7 and in many more places) when it is pointed out that Brahman is ātman, it is not proper to have such a prescription for the perception of it;  (nor) is it proper to prescribe direct perception (of Brahman/ātman). Since it is possible to have (direct prception) just by listening to Vedānta/Upaniṣads (ato’pi tattvamasyādivākyaiḥ pratipādite brahmātmatve darśanam na vidhānamarhati vidhātum yujyate; śabdenaiva sākṣātkārotpatteriti).

If by hearing (the Upaniṣads) alone direct perception and asaṁprajñānata samādhi happens\footnote{Bhikṣu allows Brahman/ātman perception only in the stage of asaṁprajñānta samādhi as a committed yogī.} then due to the above mentioned defects there should be no injunction (regarding listening) pertaining to Brahmajñāna, then the answer is:

\textbf{Ans:}  it is not so.  By the YS: “śrutānumānaprajñābhyāmanyaviṣayā viśeṣārthatvāt” (YS.I.49) one learns that the special object of perception of prajñā is of the nature of direct perception of an object of direct perception and inference. (In general) through words and inference, knowledge is generated only of the object in a general way by the quality limited by space and limited by objectness, otherwise it will exceed reason.\footnote{Bhikṣu stresses the difference between ordinary perception (pratyakṣa) and inference (anumāna) from that of the prajñā-pratyakṣa of a yogī} Even though there is the absence of anything special such as quality and action in Brahman/ātman, there is a  special distinction/difference from the jīva-consciousness in the nature of the reflection of the modification,which is accessible to the yogī, otherwise there will be the difficulty of explaining the difference (upādhivṛttipratibimbarūpe’pyantato jīvacaitanyādibhyo viśeṣaḥ yogipratyayagamyo’sti, anyatha bhedānupapatteḥ). Even in sentences like “you are the tenth” the collective meaning is arrived at through perception etc., in totality in a subtle maner, or it is by not counting (the passage of) the time (sūkṣmatvena kālakramānākalanam vā).

\textbf{Ques:} It can be said that in knowledge of Brahman also let there be meaning in totality, then the answer is: 

\textbf{Ans:} because of the indwelling impressions being an obstacle of false knowledge, it is not possible to have direct perception of Brahman. It is not possible for the general knowledge from words to be able to oppose the defect of the obstacle to direct perception of the special object (Brahman/ātman).The qualities that arise from yoga like the other qualities, having the capacity/skill which cannot be fathomed (acintyasāmarthyatayā) is the cause of even yoga based on a general object being capable of  having the direct perception of a special object.\footnote{The exalted status of Yoga and its capabilities as stated by Bhikṣu here as well in many other places again bear testimony to his total commitment to Yoga. It is only through asamprajñata yoga that Brahman can be accessed and by no other means is his refrain throughout his many works.} This is known from the smṛti śāstra: “śrotavyaḥ śrutivākyebhyo mantavyaścopapattibhiḥ matvā ca satatam dhyeya ete darśanahetavaḥ”.

\textbf{Ques:} It can be said that since there is this indwelling impressions of false knowledge being an obstacle how there can be direct perception (of the ātman) even by the yogī? Then the answer is:

\textbf{Ans:} By the testimony of the śāstras it is possible due to heightened stimulation of the qualities arising from yoga (yogajadharmasyottejakatvasambhavāt). As there is the absence of proof (for this result to occur) (pramāṇābhāvena) from knowledge arising from perception and inference and one witnesses errors (from perception and inference) in the world, there is no opportunity to imagine any heightened (awareness). Moreover as Brahman is beyond the reach of the sense-organs it is not possible to perceive directly that (Brahman) without the quality arising from yoga.

\dev{न च ब्रह्मणोऽतीन्द्रियत्वे कथं तत्र प्रपञ्चभ्रमः शास्त्रेण प्रतिपाद्यते “विज्ञानमेकं निजकर्मभेदविभिन्नचित्तैर्बहुधाभ्युपेतमि” त्यादिनेति वाच्यम् अतीन्द्रियेऽप्याकाशे मालिन्यादिभ्रमवद् बौद्धानामणुषु घटभ्रमवद्वा ब्रह्मणि प्रपञ्चभ्रमसंभवात् । उपासकानां ब्रह्मणि भ्रमसत्त्वाच्च। तथा च ब्रह्मणोऽतीन्द्रियत्वं लौकिकप्रत्यक्षाविष-}

\dev{यत्वमेव, अहमिति सर्वदा भासत्वात् । अतो भवदुक्तः शब्दात् ब्रह्मतत्वसाक्षात्कार एव नोपपद्यते। अयं घट इत्यादिप्रत्ययस्त्विदन्त्वादिरूपैर्ब्रह्मणि भ्रमरूपः संम्भवत्येव,  अहं संसारीत्यादिभ्रमवत् । रूपवत्त्वादिकं तु चाक्षुषादिप्रमायामेव कारणं भ्रमेषु तु दोष एवेति न ब्रह्मणि प्रपञ्चभ्रमानुपपत्तिः ।}

\dev{ब्रह्मणि प्रपञ्चभ्रमश्च ब्रह्मणि पारमार्थिकत्वेन प्रपञ्चज्ञानं, स्वप्नपदार्थभ्रमवदित्यवोचोम ।   तस्मात् शब्दतः सामान्यतो ज्ञातेऽपि ब्रह्मणि विशेषज्ञानरूपसाक्षात्काराय विधिर्युक्त इति ।}

\dev{अथ तथाप्यविद्यानिवृत्त्याख्यादृष्टद्वारा मोक्षहेतुत्वात् ज्ञानविधिर्विफल इति चेन्न, अवघातस्येव दृष्टादृष्टोभयद्वारत्वसंभवात् । प्रायश्चित्तस्येव ज्ञानस्याप्यदृष्टद्वारा पूर्वकृतपापादिक्षयहेतुत्वात् “तदधिगम उत्तरपूर्वाघयोरश्लेषविनाशावि’” त्यागामिसूत्रात्, विधिबलेन शीघ्रं प्रारब्धक्षये ब्रह्मप्राप्त्यादेरप्यदृष्टकार्यत्वसंभवाच्च । अत एवागामिना “कार्याख्यानादपूर्वमि” ति सूत्रेण ब्रह्मविद्यातोऽप्यपूर्वं वक्ष्यति ।।}

\dev{ननु तथापि “श्रोतव्यो मन्तयो निदिध्यासितव्य” इति विधित्रयेणैवार्थात् साक्षात्कारसिद्धे र्द्रष्टव्य इति विधिर्व्यर्थ इति चेन्न, पुरुषार्थसाधनतया द्रष्टव्य इत्यस्य विधेः प्राथमिकत्वात् तत्र साधनाकाङ्क्षयैव श्रवणादिरूपाङ्गविधित्रयप्रवृत्तेरिति । किं च जातेऽपि साक्षात्कारे साक्षात्कारसन्तानरूपसंप्रज्ञातसमाधौ द्रष्टव्य इत्यादिरूपविधिः संभवति, तस्य च फलमसंप्रज्ञातसमाधिरिति पातञ्जले स्पष्टम्,}
\begin{verse}
\dev{आगमेनानुमानेन ध्यानाभ्यासरसेन च ।}\\
\dev{त्रिधा प्रकल्पयन् प्रज्ञां लभते योगमुत्तमम् ।।}
\end{verse}
\dev{इत्यादिस्मृतिषु चेति । उत्तमं योगमसंप्रज्ञातम् । एतेन शब्दस्य साक्षात्कारजनकत्वेऽपि न तावन्मात्रेण कृतकृत्यता किं त्वसंप्रज्ञातसमाधिनैवेति मन्तव्यम् ।}

\textbf{Ques:} If Brahman is beyond the reach of the sense organs how is it that śāstras declare worldly delusion therein like: “vijñānamekam…bahu\-dhābhyupetam” (not traced); then the answer is:

\textbf{Ans:} Even though not accessible to the sense-organs, it is possible to have the delusion of the existence of the world in Brahman like the delusion of dirt etc., in the sky or like the delusion of a pot in the atoms of the Buddhists\footnote{Maybe a reference to the theory of momentariness (pratītyasamutpāda) of the Buddhists creating the existence of a pot in spite of a series of momentary existence of the atoms.}. It is also due to the devotees’ confusion with regards to Brahman (upāsakānām brahmaṇi bhramasattvācca). Thus Brahman being inaccessible to the sense-organs is only its inaccessibility to perception in a worldly sense (laukikapratyakṣāviṣayatvameva) as it is always shining as the ego (ahamiti bhātatvāt). Therefore the direct perception of the Brahmatattva through (vedānta śabda) mentioned by you does not stand to reason (brahmatattvasākṣātkāra eva nopapadyate). Thoughts such as ‘this is a pot’ in the form of sameness in Brahman (idantvādirūpairbrahmaṇi) occurs as only a delusion like the delusion of ‘I am a worldly creature’. The shape/form etc.,(of objects) is only in knowledge generated through the eyes; the cause for delusion is the defect alone; so there is no lack of reasoning for the worldly delusion in Brahman. The worldly delusion in Brahman is the knowledge of the world as true in Brahman; we say that it is like the delusion of a dream-object. Therefore even though it is known in general through words, for the direct perception in the form of special knowledge, it is proper to have an injunction.\footnote{This seems to be a justification for śabda (śrotavyaḥ) being an injunction in the sense of special knowledge through śruti which is accessible to a yogī.}

\textbf{Ques:} Even then, if it is said that since through adṛṣṭa (unknown prior samskāras like dharma and adharma) there is the removal of what is known as avidyā which causes mokṣa, the injunction for knowledge is without any purpose; then the answer is:

\textbf{Ans:} It is not so; like the case of hurting/killing (avaghātsyeva) the opening for both seen and unseen result (dṛṭṣṭādṛṣṭobhayadvāratva\-sambhavāt) is  possible. Like expiation/atonement, knowledge also, through the path of adṛṣṭa (unknown samskāra) is the cause for the attenuation of pāpa (vice) done previously; this is by the sūtra: “tadhigama uttarapūrvāghayoraśleṣavināśau tadvyapadeśāt” (BS.IV.1.13). By the strength of injunction when the prārabdha-karma (karma that has started to give result) is weakened soon,\footnote{Bhikṣu is a strong advocate of jñānakarmasamuccaya and believes that it is the apūrva from Brahman knowledge that hastens the removal of prārabdha-karma.} the attainment of Brahman etc., can be the effect of adṛṣṭa. That is why through the coming sūtra: “kāryākhyānādapūrvam” (BS. III.3.18) there will be mention of the effect of ‘apūrva’ even from the knowledge of Brahman.

\textbf{Ques:} Even so through the meaning of the three fold injunction: “śrotavyo mantavyo nididhyāsitavya” (Bṛ.Up.4.5.6) the direct perception (of Brahman) being established  the (first word) “draṣṭavyaḥ” is without purpose;\footnote{In the dialogue between Yājñavalkya and Maitreyī this instruction is as follows: “ātmā vā are draṣṭavyaḥśrotavyo mantavyo nididhyāsitavyo maitreyi”. Thus the objection is raised that the word draṣṭavyaḥ serves no purpose.} then the answer is:

\textbf{Ans:} That is not so; due to being the means for achieving one’s goal, the injunction of draṣṭavyaḥ has primary importance, (and) therein it is only with the desire to know the means that the practice of the three-limbed injunction in the form of listening etc., (is mentioned) (sādhanākāṅkṣyaiva śravaṇādirūpāṅgavidhitrayapravṛtteriti). Moreover, even when there is direct perception in samprajñāta samādhi which has the nature of a stream of direct perceptions, the injunction in the form of “draṣṭavyaḥ” happens. And its result is asamprajñāna samādhi as made clear in  Patañjali’s YS.\footnote{As a committed yogī for whom asamprajñāta samādhi alone achieves the highest mokṣa, Bhikṣu never lets go any opportunity to reiterate his partiality for yoga alone being the means for attainment of mokṣa.} There is also the following smṛti: “āgamenānumānena   dhyānābhyāsarasena ca tridhā prakal\-payan prajñām labhate yogamuttamam”. The best (for this purpose) is asamprajñāta (samadhi).\footnote{This is explaining the words ‘yogamuttamam’ in the smṛti statement.} Even if through words there is direct perception (of Brahman) there is no contentment just by that (etena śabdasya sākṣātkārajanakatve’pi na tāvanmātreṇa kṛtakṛtyatā); however it is to be known that it is possible only through asamprajñāta samādhi (kimtvasamprajñātasamādhināiveti mantavyam).\footnote{As a true yogī committed to the yoga sādhanā for final emancipation Bhikṣu does not want to admit any other means to be superior to it. He grudgingly admits the efficacy of śravaṇa etc., for that but places it as less important especially since there is no contentment derived but only a goal reached. He is almost advocating an additional psychological component to achieving kaivalya.}

\dev{यच्चान्यदुक्तं साक्षात्कारात् प्रपञ्चबाधे ज्ञानसाधनाभावात् कथं ज्ञानस्य विधेयत्वं स्यादिति, तदपि तुच्छम् बाधो हि न नाशो ज्ञानिनां प्रारब्धभोगासंभवात् । अत एव न शशशृङ्गादिवदत्यन्तसत्त्वावधारणमपि “नाभाव उपलब्धेरि” ति सूत्रात्। भिक्षादावपि प्रवृत्त्यसंभवापत्तेश्च । किंतु प्रतिक्षणभङ्गुरतया मनःपरिणामेषु स्वप्नादिपदार्थेष्विव बाह्येष्वपि पारमार्थिकसत्वाभावनिश्चय एव बाधः, तादृशबाधेऽपि च सति ज्ञानसाधनादीनां व्यावहारिकसत्त्वात् पूर्ववदेव व्यवहारो ज्ञानिनामपि संभवत्येवेति । पारमार्थिकसत्त्वाभावज्ञानं च प्रकृतिपुरुषयोर्विवेकज्ञानं तथा समाधिसाधनं ब्रह्मणि रतिमन्यत्र च वैराग्यं च जनयति, अतो न तस्य वैफल्यमिति । एतेन साक्षात्कारे तत्सन्तानरूपं (पे) संप्रज्ञातसमाधौ च पुरुषस्य स्वातन्त्र्यमपि  स्पष्टम् । मनननिदिध्यासनविषयान्तरसञ्चारादिभिः कर्तुमकर्तुमन्यथाकर्तुं च शक्यत्वादिति ।}

\dev{साक्षात्कृतिसाध्यत्वस्यैव विध्यर्थत्वे तु यजेतेत्यादिरपि विधिर्न स्यात्, यागस्येच्छाविशेषत्वादिच्छायां चेष्टसाधनज्ञानादेरेव साक्षाद्धेतुत्वादतः साक्षात्परम्परासाधारण्येन कृतिसाध्यत्वमेव च विध्यर्थो लाघवाच्च । अन्यथा जगन्नाथो दृष्टव्य इत्यादिविधिषु दर्शनादिहेतुव्यापारे धातुलक्षणागौरवाच्च, दर्शनाद्यनुकूलव्यापाराणां साक्षादिष्टसाधनत्वाभावेन विधेयत्वानौचित्याच्च ।}

\dev{किं च यथा श्रोतव्य इत्यादौ शाब्दबोधाद्यनुकूलव्यापारे धातुलक्षणया विधित्वं परैः समर्थ्यते तथैव द्रष्टव्य इत्यादावप्यात्मदर्शनानुकूलव्यापारलक्षणया विधित्वं समर्थ्यताम् । अपि च श्रोतव्य इति विधिशेषतयाऽपि ब्रह्मप्रतिपादनमस्मदभिमतं सेत्स्यत्येवेति सिद्धम् । ब्रह्मणो विध्यंधपरम्परासाधारण्येनेष्टसाधने विधिरस्त्विति चेत् श्रुतहान्यश्रुतकल्पनागौरवेण कृतिसाध्यत्व एव परम्परासाधारण्यौचित्यादिति । शेषत्वं (दितिशेषत्वम् । ) तत्र च कुतार्किकाणां कक्षापरम्परानिरासे ग्रन्थबाहुल्यभयादस्माभिरुपरम्यते । सच्छिष्यैस्त्वनयैव दिशा कक्षान्तरमपि स्वयं निरसनीयम् । तदयं श्लोकः—}
\begin{verse}
\dev{सामान्यतः शुद्धपन्था अस्माभिः संप्रदर्श्यते ।}\\
\dev{सन्मार्गे विदिते विज्ञैर्विमार्गो ज्ञास्यते स्वयम् ।। इति ।}
\end{verse}
\dev{यद्भयाच्च तैः सिद्धार्थप्रामाण्यमपसिद्धान्तोऽपि स्वीक्रियते, तद्भयमप्यस्थान एव । तथाहि, तेषां भयबीजम्—\break वेदान्तानां ब्रह्मज्ञानविधिपरत्वे ब्रह्म न सिध्येत्, निरवकाशप्रत्यक्षादिप्रमाणविरोधेन “वाचं धेनुमुपासीते” त्यादिश्रुतिवत् “तत्त्वमस्या” दिवाक्यानामुपासापरत्वेनैवोपपत्तेरित । तत्र तन्न, ब्रह्मणोऽतीन्द्रियत्वेन प्रत्यक्षप्रमाणाविरोधात् । ब्रह्मजीवयोरखण्डत्वस्य तु श्रुतिविरोधेनैव हेयत्वात् न तु प्रत्यक्षविरोधेन । अनुमानशब्दादीनां च ब्रह्मबोधकानामनिर्णयात् । तथा च धियामौत्सर्गिकं प्रामाण्यमिति न्यायेनोपासादिविधिशेषत्वेऽपि सेत्स्यत्येव ब्रह्म यागविधिशेषत्वेन स्वर्गादिवत्। न च फलप्रतिपादकतया स्वर्गादिवाक्यस्य स्वपरत्वमिति वाच्यम्, “ब्रह्म वेद ब्रह्मैव भवति, सोऽश्नुते सर्वान् कामान् सह ब्रह्मणा विपश्चिता, ब्रह्मनिर्वाणमृच्छती” ति फलश्रुतिबलाद् ब्रह्मवाक्यस्यापि स्वपरत्वसिद्धेरिति ।}

Another thing mentioned is that through direct perception (of Brahman) the world disappearing/being obstructed (due to tealizing that it is an appearance/delusion), there is the unavailability of knowledge as a means, so how can there be the injunction for knowledge (as the means to get rid off prārabdha karma); that (objection) is also not significant; disappearance does not mean total destruction as there will be no possibility for the knowers to experience prārabdha-karma. That is why there is no understanding of it as absolute absence like the rabbit’s horn; through the sūtra: “nābhāva upalabdheḥ” (BS.II.2.28) there will (also be) the danger of inactivity regarding bhikṣā (going out for alms etc).\footnote{The sūtra affirms the existence of outside objects. The example highlights the fact that going out for alms is to collect food which satisfies hunger. If there is no purpose served by food then why engage in that activity at all}  Moreover when there is destruction at every moment of the changes in the mind , there will be lack of belief in the certainty of the existence of real objects outside, similar to the objects in a dream. Even when there is that type of obstruction, since there will be available worldly means (of activity) such as knowledge etc., there will be activity just as before even in the case of those who follow the path of jñāna. And the knowledge of the absence of existence of the real truth is the knowledge of the difference between puruṣa and prakṛti;\footnote{The knowledge of the difference between material reality and spiritual reality (puruṣa and prakṛti) makes one realize the non-reality of the material world (things emanating from prakṛti).} thus the practice of samādhi creats attachment to Brahman and detachment towards other (things); therefore it is not a loss of effort.\footnote{Bhikṣu relentlessly asserts the paramountcy of yoga sādhanā at every opportunity} In this way when there is direct perception of puruṣa as a stream/series in samprajñāta samādhi there is clarity of the independence of puruṣa as well. Through roaming in different objects through reflecting and meditating it is possible to practice, not practice and practice suitably (samprajñāta samādhi).\footnote{While Śaṅkarācārya uses this independence to perform, not perform or perform differently with reference to karma (ritual) Bhikṣu uses the same rule for the practice of samādhi.} 

If only that which can be accomplished by an action is known as an injunction (vidhyarthatve syāt) then the utterance “yajeta” also cannot be an injunction, since a sacrifice has the special quality of a desire (to perform) and in the desire to perform, the means for sacrifice and the knowledge of (the yāga) are inherent as the direct cause (yāgasyecchāviśeṣatvādicchāyām ceṣṭasādhajñānadereva sākṣāddhetutvāt);  this is for the sake of logcal conciseness as well (lāghavācca),  thus the accomplishment of the ritual (action) in a direct sequential manner is the meaning of an injunction. Otherwise in such injunctions as ‘jagannātho draṣṭavyaḥ’ when the cause for action is seeing etc., there will be overstretching the meaning of verbs as well (dhātulakṣaṇāgauravācca) since (in such instances), there is an absence directly of the means for accomplishing the desired end (i.e. seeing); so it is not correct to make it an injunction. 

Moreover when in statements such as “śrotavyaḥ” others attest an injunction, when one understands activity in consonance with the denoted meaning, similarly in sayings such as “draṣṭavyaḥ” also one can attest an injunction in consonance with the nature of activity that is conducive to self realization. Moreover “śrotavyaḥ” fits in with what we desire as being a supplementary injunction for establishing Brahman.

\textbf{Ques:} If due to following a blind common tradition of needing a vidhi for Brahman (knowledge) (it is said) ‘let there be an injunction for the desired means’ then the answer is:

\textbf{Ans:} (Since  this suffers) from the tedious (cumbersome) logic of violating what is heard and of imagining what is unheard, (śrutahānyaśrutakalpanāgauravāt) it is correct  (to understand vidhi) as that which is competent for ccomplishing the task in accordance with the common tradition of action alone (kṛtisādhyatva eva pramparāsādhāraṇyaucityāditi). 

In that context, out of fear for expansion of the text/commentary, we are withdrawing from refuting the stream of false objections to the traditional arguments (tatra ca kutārkikāṇām kakṣāparamparānirāse granthabāhulyabhayādasmābhiruparamyate). By this method alone true disciples should reject other objections themselves. In that connection there is this verse: “sāmānyataḥ śuddhapanthā…vimārgo jñāsyate svayam”(not traced).

The acceptance by them of a false doctrine through fear as established through proof, that fear is also improper (out of place). (yadbhayācca taiḥ siddhārthaprāmāṇyamapasiddhānto’pi svīkriyate tadbhayamapyasthāna eva). Thus their source of fear is: 

\textbf{Ques:} if Vedānta injunctions, like the śruti vākya “vācam dhenumupā\-sīta” (Bṛ.Up. V.8.1)\footnote{This quote from the Bṛ.Up advocates meditating on speech by tracing new meanings for the words. Thus Bhikṣu advocates that sentences such as ‘tat tvam asi’ are mainly for meditation purposes.}  having as its object knowledge of Brahman, does not result in Brahman (realization) due to the contradiction in the means of knowledge like perception (niravakāśapratyakṣādipramāṇa\-virodhena) then it is reasonable to accept that sentences like: “tat tvam asi” are meant only for meditation purposes.\footnote{Inspite of declaring he will not to continue the arguments out of fear that the text will unnecessarily be too long Bhikṣu cannot resist some more attacks on advaita. There is nothing new added however to the old arguments already stated in many other places. It is well known that direct perception such as “tat tvam asi” is not possible by the said pramāṇas and so there is a mismatch in the stated pramāṇas and their end result and hence has to be understood differently.} 

\eject

\textbf{Ans:} In that context, because of Brahman being beyond the reach of the sense-organs there is no contradiction in the said pramāṇa.\footnote{Then ‘draṣṭavyah’ can be interpreted as meditating on Brahman and not direct percepton of Brahman.} The complete wholeness of jīva and Brahman (brahmajīvayoḥ akhaṇḍatvasya) is to be rejected only by pointing out the contradiction in śruti and not by direct perception. This is because testimonies like inference and śabda (words) that instruct about Brahman have no capacity to instruct with certainty. Therefore even if meditation etc., are the remaining vidhis, through the logic that one’s thought of the self has proof in general\footnote{One is reminded of Kālidāsa’s statement in Duṣyanta’s words: “satām hi sandehapadeṣu vastuṣu pramāṇantaḥkaraṇapravṛttayaḥ” (Abhi.Śāk. I.20). In matters of doubt one’s own internal thought is proof for deciding its correctness or otherwise.}, Brahman can be admitted (allowed) as that which remains after sacrifice, similar to svarga after the injunction.\footnote{Like the attainment of svarga by the injunction ‘savrgakāmo yajeta’} Also by declaring the result it does not mean that sentences mentioning svarga have svarga as its subject. By the strength of such śruti sayings as “brahma veda brahmaiva bhavati”, “so’śnute sarvān kāmān saha brahmaṇā vipaścitā”, “brahmanirvāṇamṛcchatī” declaring the result, these sayings on Brahman also have the achievement of itself as the subject (iti phalaśrutibalād brahmavākyasyāpi svaparatvasiddheriti).\footnote{Bhikṣu is willing to equate the attainment of Brahman with that of the attainment of one’s self.}

\dev{किं च धर्मादिवदेव श्रुत्यनुगृहीतानुमानाज्जगत्कारणं सेत्स्यति, अतः सिद्धार्थप्रामाण्येऽस्माकं न निर्बन्ध इति ।}

\dev{यश्च परैः सिद्धब्रह्मपरत्वे वेदान्तानामुपक्रमादिभिस्तात्पर्यसमन्वय उक्तः, सोऽपि विपरीत एव । तथाहि —}
\begin{verse}
\dev{उपक्रमोपसंहारावभ्यासोऽपूर्वता फलम् ।}\\
\dev{अर्थवादोपपत्ती च लिङ्गं तात्पर्य निर्णयः ॥}
\end{verse}
\dev{इति तात्पर्यग्राहकलिङ्गानां मध्ये फलार्थवादयोरसिद्धेर्ब्रह्मविज्ञान एवं संभवो न तु ब्रह्मणि, ब्रह्मज्ञानस्यैव “य एवं वेदे” त्यादिना फलश्रवणाच्च । “नातः परं वेदितव्यं हि किंचिदि” त्याद्यर्थवादस्यापि ज्ञान एव श्रवणाच्चेति । तस्माद् ब्रह्मज्ञान एव तात्पर्यग्राहक¹ समग्रलिङ्गोपपत्तेः ब्रह्मज्ञानाख्यकार्यपरत्वमेव वेदान्तानाम् । अपि च “अधीहि भगवन् ब्रह्मविद्यां वरिष्ठामि” त्यादिश्रुतिषु,}
\begin{verse}
\dev{ज्ञानं तेऽहं सविज्ञानमिदं वक्ष्याम्यशेषतः ।}\\
\dev{परं भूयः प्रवक्ष्यामि ज्ञानानां ज्ञानमुत्तमम् ॥}
\end{verse}
\dev{इत्यादिस्मृतिषु च ज्ञानस्यैव मुख्यतः उपक्रमादिस्तात्पर्यग्राहकोऽस्ति ।  अतस्तत्रैव तात्पर्यसिद्धौ तच्छेषतया ब्रह्माणोऽप्युपक्रमादिः क्वाचित्क उपपद्यते । ब्रह्मणि तु तात्पर्ये तदशेषतया ज्ञानं (ज्ञाने) सर्वथैवोपक्रमादिरनुपपन्न।}

\dev{किं च यैः कार्यपरत्वं वेदानामुच्यते तैरकार्यत्वेन तात्पर्याभावस्यैव साधनात्, तेषामकार्यत्वलिङ्गेन भवतां समन्वयस्य सत्प्रतिपक्षितत्वाद्वा न समन्वयेन सिद्धपरत्वमवधारयितुं शक्यत इति संक्षेपः ।}

\dev{यच्चैतदस्माभिर्ब्रह्मज्ञाने विधिर्व्यवस्थापितः तद् ब्रह्मज्ञानविधिरूपभगवदाज्ञापालनेनापि विदुषां ब्रह्मार्चनं भवतीति प्रतिपादयितुमेव, “श्रुतिस्मृती ममैवाज्ञे” “स्वकर्मणा तमभ्यर्च्य सिद्धिं विन्दति मानव” इत्यादिस्मृतेरित्यवधेयम् ।}

Moreover the cause for the (origin of the) world is accepted (favoured) through inference (which is) favoured by śruti, (which is in accordance with śruti) just like dharma etc., (dharmādivadeva śrutyanugṛhītānu\-mānājjagatkāraṇam setsyati).\footnote{The existence of dharma and adharma is accepted by inference alone as there is no direct proof for it.} Therefore with regard to proof whose purpose is established there is no need to press us. Thus: the mention by others of the collective intention of the six liṅgas beginning with upakrama etc., “upakramopasamhārāvabhyāso’pūrvatā phalam, arthavādopapattī ca liṅgam tātparyanirṇaye” as referring to the subject of the established Brahman is also not correct. When one tries to grasp the intention of the (six) liṅgas, amongst them the result and eulogy (phala and arthavāda) is only possible in distinguishing Brahman (from the rest) and not in (attainment of) Brahman as one only hears of the knowledge of Brahman as result (phalaśravaṇāt)in statements like: “ya evam veda” etc.(Bṛ.Up.5.7.1).\footnote{As Brahman is beyond the each of the senses its result and its eulogy can only be indirect as distinguishing it from known objects of experience such as sharing their characteristics etc. Thus the Bṛ.Up talks about Brahman as lightning because it removes darkness. In other words it is not possible to know Brahman directly as one would any other object. Therefore two out of the six signs upakrama etc., fail in the case of Brahman.} In such eulogical statements as: “nātaḥparam veditavyam hi kimcit” (Śvet.Up.I.12) one only hears of knowledge (of Brahman and not its attainment). Therefore, grasping only knowledge (of Brahman) is the intention of all the liṅgas (the six liṅgas) like upakrama etc., and Vedānta has the purpose of what is known as realization of Brahman alone (tasmād brahmajñāna eva tātparyagrāhaka samagraliṅgopapatteḥ brahmākhyakāryaparatvameva vedāntanām). Moreover in śruti statements such as: “adhīhi bhagavan brahmavidyām variṣṭhām” (Kaiv.Up.1) and smṛti sayings like: “jñānam te’hamsavijñānamidam vakṣyāmyaśeṣataḥ…jñānānām jñānamuttamam” it is mainly knowledge about (Brahman) that is grasped as the intention of the (six liṅgas) upakrama etc.\footnote{When knowledge about Brahman as opposed to the realization of Brahman, which is again knowledge is mentioned in this context, it distinguishes the descriptive way of conveying the knowledge of Brahman as opposed to the immediate awareness of the attainment of Brahman as an experience.} and only rarely as a supplementary (meaning) for the sake of Brahman also. The liṅgas ‘upakrama’ etc., are in no way applicable  (sarvathaivavopakramādiranupapannaḥ) when the intention is  complete knowledge of it (Brahman).

Moreover those who mention the subject matter of the Vedās as being inclined towards action,  by their not performing any action, only accomplishes the absence of intention by them\footnote{Bhikṣu is continuously arguing for non-abandonment of rituals being the subject matter of the Vedas.} (kim ca yaiḥ kāryaparatvam vedānāmucyate tairrakāryatvena tātparyābhāvasyaiva sādhanāt); not being possible to understand that the entire Veda establishes a fixed subject matter, since their proofs do not having the subject matter of action intended (teṣāmakāryatvaliṅgena),  or (on the\break other hand) by your mention of (Brahman) being the entire meaning (of the Vedas) it suffers from the hetvābhāsa (fallacy) of (satpratipakṣabhāva) having equal reasoning for the opposite view,  (bhavatām samanvayasya satpratipakṣitatvādvā na samanvayena siddhaparatvamavadhārayitum śakyata iti saṃkṣepaḥ).\footnote{satpratipakṣabhāva is mentioned in Nyāya as one of the fallacies of logic. It is when a reason proving a fact is vitiated because there is an equal reason for proving the opposite. Thus “sound is eternal because it is not heard” and “sound is non-eternal because it is produced” are both valid in Indian philosophy. In this case neither have the six liṅgas been able to prove that their intention is Brahman-knowledge nor have the Vedas proven to have Brahman as their full import.} 

This injunction for realization/knowledge of Brahman that has been laid down by us is in the form of a vidhi for knowledge of Brahman  only to show that through following the commands of Bhagavān as well it will (result) in the worship of Brahman of the wise (brahmajñānavidhirūpabhagavadājñāpālanenāpi viduṣām brahmārcanam bhava\-tīti pratipādayitumeva).\footnote{Along with being a committed yogi, Bhikṣu also is a jñāni-bhakta whose devotion to Īśvara pervades all his works.} One should know that this is the (purport of the) smṛtis such as “śrutismṛtī mamaivājñe”, “svakarmaṇā tamabhyarcya siddhim vindati mānava”(Gītā. 18.46)

\dev{ननु यदि प्रलये ब्रह्मणि प्रधानादिसमन्वयोऽभ्युपगतः, तर्हि तस्यैव जगज्जन्मादिकारणत्व (त्वं) श्रुत्यर्थोऽस्तु, भवदभिमतब्रह्मकल्पने तस्य जगदधिष्ठानकारणत्वकल्पने च गौरवात् । तथा चेश्वरप्रतिषेधे सांख्यसूत्रद्वयम् “प्रमाणाभावान्न तत्सिद्धिः, श्रुतिरपि प्रधानकार्यत्वस्ये” ति । ब्रह्मशब्दोऽपि व्यापकत्वात् प्रधानजीवयोरुपपद्यत एव । अधिष्ठानकारणत्वं च जीवनामेवास्तु । समस्तकार्याणामदृष्टद्वारा जीवकार्यत्वस्य अदृष्टवदात्मसंयोगजन्यतया जीवाधेयत्वस्य वा सर्वास्तिकसंमतत्वात् । अत एव श्रुतिस्मृती जीवप्रकरणेऽपि स्तः । “सर्वं तं परादाद्योऽन्यत्रात्मनः सर्वं वेद” । इतिश्रुतिः,}
\begin{verse}
\dev{“सर्वभूतस्थमात्मानं सर्वभूतानि चात्मनि ।}\\
\dev{येन भूतान्यशेषेण द्रक्ष्यस्यात्मन्यथो मयि ।।}
\end{verse}
\dev{इति स्मृतिश्चेति । तथा चास्मच्छास्त्रस्य नित्येश्वरपरत्वे शास्त्रयोनित्वादिति हेतुरसिद्धः, कार्यब्रह्मपरत्वे च पृथक् शास्त्रारम्भो विफलः, सांख्यादिभिरेव तथाविधब्रह्मनिरूपणादिति तामिमामाशङ्कामपाकरोति— ईक्षतेर्नाशब्दम् ॥}

\textbf{Ques:} If in dissolution when pradhāna etc., with everything collectively is accomodated in Brahman, then let the meaning of śruti be that it (Brahman of that nature) is the cause of the origin etc., of the world. It is logically cumbersome in imagining Brahman (as the cause) which is desired by you and also in imagining its (Brahman’s) being the supporting cause (according to us) (bhavadabhimatabrahmakalpane tasya jagadadhiṣṭhānakāraṇatvakalpane ca gauravāt). Thus the two Sānkhyasūtras which reject Īśvara: “pramāṇābhāvānna tatsiddhiḥ”, “śrutirapi pradhānakāryatvasya” (are relevant here). Because the word Brahman also means ‘expanse’ it can also be applied to prakṛti and jīva. Let (Brahman) be the supporting cause of the jīvas alone. There is all round consent amongst the āstikas that through all effects (coming into being) through adṛṣṭa, the rise of the jīva effect  having adṛṣṭa is born through contact with the ātman,  or by (ātman) being the underlying support of jīva  (samastakāryāṇām adṛṣṭadvārā jīvakāryatvasya adṛṣṭavadātmasamyogajanyatayā jīvādheyatvasya vā sarvāstikatvasammatatvāt).\footnote{The idea is to state that all things have Brahman/ātman as the originator by contact and also that Brahman is the underlying support of the jīva.} That is the reason why both śruti and smṛti also occur in the section dealing with the jīva. Thus śruti says: “sarvam tam parādādyo’nyatrātmanaḥ sarvam veda” (Bṛ.Up.IV.5.7);\footnote{The literal translation of this line which is part of a larger passage is “all defeats one who understands it as different from the Self”. The idea is thus to emphasize that all (everything) is the Self.} so does the smṛti statement: “sarvabhūtasthamātmānam…drakṣyasyātmanyatho mayi”. Thus since our śāstra (Bhikṣu’s doctrine) believes in the eternal nature of Īśvara, the statement that śāstra is the proof of the place of birth (origin of Īśvara) or (Īśvara is the source of śāstra) is not established. It is also useless to talk about the ‘kārya Brahman’ through starting another śāstra, since Sāmkhya philosophers have already come to the conclusion of such a Brahman.\footnote{This is reference to Brahman with its power (śakti) of prakṛti being the origin of the universe. Though Sāṃkhya does not call prakṛti as the śakti of Brahman and according to them the contact between puruṣa and pṝakṛti like that between a magnet and iron filing starts the process of evolution of the world.} That doubt is hereby removed (by the following sūtra “Īkṣaternāśabdam” BS.I.1,5):

\section*{BS.I.1.5 Īkṣaternāśabdam}

(Pradhāna) is not the cause of the world as it is not mentioned in śabda (Vedas/Upaniṣads) as it is clear from (its not having the quality of) seeing (visualization.)

\dev{विवादास्पदप्रधानादिभ्योऽतिरिक्तं ब्रह्म न अशब्दम् नाऽशास्त्रयोनि, न जगत्कारण- श्रुत्यप्रतिपाद्यमिति यावत् । कुतः ? ईक्षतेः इतरावृत्तिश्रुत्युक्तकारणविशेषणेक्षणवत्त्वादित्यर्थः । हेतावीक्षणमिदं स्वरूपाख्यानं न तु हेतौ प्रवेशनीयं वैयर्थ्यात् । अत्र यत् इतरावृत्तिश्रुत्युक्तजगत्कारणविशेषणवद् भवति, तत् जगत्कारणतया प्रतिपाद्य भवति । यत्तद्भ्यां सामान्यतो व्याप्तौ सांख्यशास्त्रोक्तं प्रधानं दृष्टान्तः, वैशेषिकाणां चाणवः । पक्षे ब्रह्मेति स्वरूपाख्यानमात्रमतो न पक्षासिद्धिः ।}

\dev{अत्रासिद्धिवैयर्थ्यादिपरिहाराय प्रयोगोऽन्यथाकर्तव्यः । यथा, जगत्कारणश्रुतिः पराभिमतप्रधानादिभ्योऽन्यविषया, तदवृत्तिधर्मप्रकारेण बोधकत्वात्, आकाशाद्वायुरित्यादि- श्रुतिवदिति ।अथवा तद्ब्रह्म चेतनो अचेतनो वा, चेतनत्वेऽपि स्वयम्भूस्तदतिरिक्तो वेति विशेषाकाङ्क्षायामिदं सूत्रं प्रववृते “ईक्षतेर्नाशब्दमि” ति । तज्जगत्कारणं ब्रह्म ना पुरुषः, कुतः ईक्षतेः ईक्षणश्रुतिगोचरत्वात् । अत एवेक्षणात् तद्ब्रह्म अशब्दं शब्दब्रह्मणो हिरण्यगर्भादति4 रिक्तं च भवति, सुतरां तु पुरुषान्तरेभ्य इत्यर्थः । हिरण्यगर्भस्य च वेदमयत्वेन वेदगर्भत्ववच्छब्दब्रह्मत्वमपि स्मर्यते—}
\begin{verse}
\dev{पूर्वस्यादौ परार्धस्य ब्राह्मो नाम महानभूत् ।}\\
\dev{कल्पो यत्राभवद् ब्रह्मा शब्दब्रह्मेति यद् विदुः ।।}
\end{verse}
\dev{इति भागवतादाविति ।}

\dev{जगत्कारणस्येक्षणे श्रुतयश्च “तदैक्षत बहुस्यां प्रजायेये” त्याद्याः । स्मृतयश्च—}
\begin{verse}
\dev{सर्गकाले तु संप्राप्ते ज्ञात्वा तं कालरूपकम् ।}\\
\dev{अन्तर्लीनविकारश्च तत्स्रष्टुमुपचक्रमे ॥}\\
\dev{तस्मादव्यक्तमुत्पन्नं ततश्चापि महानभूत् ।}
\end{verse}
\dev{इत्याद्या इति । नहीदमीक्षणं प्रधानादीनामचेतनानां संभवति, ईक्षणध्यानचिन्तनादिशब्देषु चैतन्यस्य विशेषत्वात् । ईक्षणादिशब्दस्योपाधिवृत्तिमात्रवाचकत्वेऽपि प्रकृतिस्वातन्त्र्यवादिभिस्तत्पूर्विका प्रधानप्रवृत्ति 1 र्नाभ्युपगम्यत एव । नापि जीवानां, महदादिसृष्टेः पूर्वं कारणाभावेन जीवानां चैतन्यफलोपधानाभावात् । न चेयमीक्षणपूर्विका सृष्टिर्महत्तत्त्वसृष्टेः पश्चादिति वक्तुं शक्यते, ईक्षणात् पूर्वमपि द्वैतापत्त्या  “सदेव सौम्येदमग्र आसीदेकमेवाद्वितीयमि” तीक्षणाव्यवहितपूर्वश्रुत्यनुपपत्तेः, “ततश्चापि महानभूदि”ति स्मृतिविरोधाच्च । नापि पूर्वसर्गीयमीक्षणं वक्तुं शक्यते, “सर्गकाले तु संप्राप्ते ज्ञात्वे”ति सर्गसमकालीनज्ञानावगमादिति ।}

\dev{यच्च महदादिसृष्टेस्तमोमयत्वं स्मर्यते “अबुद्धिपूर्वकः सर्गः प्रादुर्भुतस्तमोमयः” इति विष्णुपुराणादिषु, तदीश्वराधिष्ठिततमोगुणपरिणामत्वेनैवोपपद्यते । अबुद्धि ( बुद्धि ) पूर्वकत्वमपि कार्यसत्त्वस्यैव बुद्धिशब्देन परिभाषितत्वादुपपद्यत इति । “तमसैवेदमावृतमासीत्, आसीदिदं तमोभूतमप्रज्ञातमलक्षणमि” त्यादि वाक्यान्य-  पीश्वरज्ञाननित्यत्वेप्युपपन्नानि । तमसि लीनतया स्वाज्ञानत्वेनैव सर्वप्राण्यवस्थानस्य तदर्थत्वात् । तथा च श्रुतिः ‘‘तमो वा इदमेकमासत तत्परे स्यात् तत्परेणेरितं विषमत्वं प्रयाती” ति, सति संपद्य न विदुः सति सम्पद्यामह इति” इति च ।}

Brahman is different from pradhāna etc.,  which is the subject of dispute which is discussed in this sūtra\footnote{This sūtra raises the question as to whether Brahman is different from pradhāna etc and further lays down in what way it is different. It also will mention that pradhāna need not be ruled out as the cause of the world in a Sāṃkhyan sense.}; it (Brahman) is not non-men\-tioned in the śāstras/Upaniṣads nor is śāstra not the source of it, i.e it is not undeclared in the śruti that mentions the cause of the world (vivādāspadapradhānādibhyo’tiriktam brahma na aśabdam nā’śāstra\-yoni, na jagatkāraṇaśrutyapratipādyamiti yāvat). How? Because the word “īkṣateḥ” (in the sūtra) is a repetition again of what is stated in śruti as a qualification for the cause (of the world) i.e. because of having the quality of ‘seeing’ (īkṣateḥ itarāvṛttiśrutyuktajagatkāraṇaviśe\-ṣaṇekṣaṇavattvādityarthaḥ).\footnote{This can perhaps be understood as fulfilling the ‘abhyāsa’ mark given as one of the six marks beginning with “upakrama” etc.} “īkṣaṇa” used as  ‘hetu’ (reason) here is in the sense of announcing its intrinsic nature (hetāvīkṣaṇamidam svarūpākhyānam) and one should not argue about its being used in the sense of a mark, as it is useless (to do so) (hetāvīkṣaṇamidamsvarūpā\-khyānam na tu hetau praveśanīyam vaiyarthyāt). Here the repetition of having the qualification mentioned elsewhere for the cause of the world is to be shown as being the cause of the world. When there is all-pervasion by those two  generally, (there are) examples like pradhāna mentioned by Sāṇkhya as also the atoms mentioned by the Vaiśeṣikas (which have those two overlapping).\footnote{Bhikṣu is pointing out that this is a case of hetugarbhaviśeṣaṇa where the reason is built into the result. There is nothing special about only Brahman being the cause of the world argues Bhikṣu as both the qualifications mentioned are common to both pradhāna and the atoms.} The partisan view is only that it is the intrinsic nature of Brahman (pakṣe brahmeti svarūpākhyānamātram); therefore it is not that another view is not established.

Herein in order to remove (reasons like) uselessness, non-establish\-ment etc., (of another view) there must be application in a different way. Just as śruti stating the cause of the world is something different from pradhāna etc.,that is desired by others, due to the understanding of it (Brahman) as having the property of non-action, (and considering the rise of the world spontaneously) similar to the rise of wind out of space etc., according to śruti (ākāśādvāyutiyādiśrutivaditi) (there must be another way of understanding).\footnote{One of the arguments is that there is a spontaeous manifestation of things like ākāśa, vāyu, tejas, āpas and pṛthivī (space, wind, light, water and earth, each arising out of the antecedent subtle particles). In contrast prakṛti and atoms have action and so cannot be supreme.} 

Maybe that Brahman is either sentient or non-sentient; even if it is sentient is it svayambhūḥ or is it different from it; (it is) with this special desire to know that this sūtra “īkṣaternāśabdam” has been started. That Brahman which the cause of the world is not ‘apuruṣa’ (non-puruṣa). Why? “īkṣateḥ”=since it has śruti as the object of seeing. Therefore, it is because of seeing/thinking  that Brahman is also “aśabdam”different from Hiraṇyagarbha which is śabda-Brahman; it is greatly different from other puruṣas ((sutarām tu puruṣāntarebhya ityarthaḥ). Just like Hiraṇyagarbha because of containing the Vedas within itself is spoken of as being filled with the Vedas, one is reminded of sayings of ‘śabdabrahmatvam’also (hiraṇyagarbhasya ca vedamayatvena vedagarbhatvavacchabdabrahmatvamapi smaryate). Thus there is the saying: “pūrvasyādau parārdhasya brāhmo nāma mahanabhūt, kalpo yatrabhavad brahma śabdabrahmeti yad viduḥ” from the Bhā.P (III.11.34). There are also śrutis with regard to seeing/thinking like: “tadaikṣata bahusyām prajāyeya” etc (Chānd.Up. 6.2.3). There are also smṛti statements like: “sargakāle tu samprāpte…tataścāpi mahānabhūt”. 

Such seeing is not possible in entities like pradhāna etc., that are insentient, since in words that denote seeing, meditation, thinking etc., there is a special quality of consciousness. Even if the words thinking etc., denote only the modification of the limitation (of puruṣa/Brah\-man), for those who advocate the independence of prakṛti, activity preceding that is not agreeable (prakṛtisvātantryavādibhistatpūrvikā pradhānapravṛttirnābhyupagamyata eva). Nor is it possible for the jīvas since before the creation of tattvas (principles) such as mahat etc., there being the absence of any cause there is no production of any result of consciousness for the jīvas\footnote{Unless there is an object of perception there can be no knowledge.} (nāpi jīvānām, mahadādisrṣṭeḥ pūrvam kāraṇābhāvena jīvānam caitanyaphalopdhānābhāvāt). Nor can it be said that this creation after seeing/deciding, was after the creation of mahat etc. Then there will be the contradiction of there being duality even before deciding (to create) and it will contradict the earlier śruti statement of (creation) following immediately after seeing : “sadeva saumyedamevāgra āsīdekamevādvitīyam” (Chānd.Up.\break 6.2.1). It will also contradict the smṛti saying: “tataśca mahānabhūt”. Nor can one say that it is the seeing of the earlier sarga (evolution) as (it is against the saying): “sargakāle tu samprāpte jñātvā” that the knowledge/seeing is at the same time as the evolution (sargasamakālīnajñānāvagamāt”.

Moreover statements in smṛtis like Viṣṇu.P etc., that the creation of mahat etc., is full of tamas (guṇa): “abuddhipūrvakaḥ sargaḥ prādurbhūtastamomayaḥ” (I.5.4. cited in Tri. p.66.fn.) fits in with the transformation of the ‘tamoguṇa’ that is situated in Īśvara. Precedence of non-intelligence is also proper as by definition the word ‘intelligence’ (buddhiśabdena) denotes the effective sattva. “tamasaivedamāvṛta\-māsīt”, “āsīdidam tamobhūtamaprajñātamalakṣaṇam” such sayings are also correct in the context of the eternal nature of sentience of Īśvara. It means that because of one’s own ignorance, being hidden in darkness is the state of all beings. Thus śruti says: “tamo vā idamekamāsata\-…viṣamatvam prayāti” (Mait.Up.5.2), “sati sampadya na viduḥ sati sampadyāmaha iti” (not traced).

\dev{यद्यपि प्राकृतमानसदे हिकरूपेण त्रिविधा सृष्टिः पुराणादौ स्मर्यते, तदपि आद्यशरीरिणं विराड्जीवमधिकृत्यैव न परमेश्वरं, प्रयत्नवदात्मसंयोगादपि जगदुत्पत्तेरिति ।}

\dev{ये त्विदं सूत्रं प्रधानादीनां जगत्कारणताप्रतिषेधकमितीच्छन्ति, तेषां मते एवं सूत्रं व्याख्येयम्— अंशब्दं प्रधानादि न जगत्कारणं मूलकारणविशेषणेक्षणभावादिति । अत्राशब्दमिति स्वरूपाख्यानं न तु अशब्दत्वेनैव पक्षता, मूलकारणश्रुत्यप्रतिपाद्यत्वरूपस्याशब्दत्वस्य प्रागसिद्धेः । मूलकारणत्वं च स्वतन्त्रकारणत्वं परेच्छानधीनकारणत्वमिति यावत् । हेतौ चात्रापि ईक्षणशब्दः स्वरूपाख्यानमिति। आधुनिकास्तु सूत्रमिदमेवं व्याचक्षते—ईक्षणश्रुतेरशब्दं प्रधानादिकं न जगत्कारणं हेतुगर्भविशेषणत्वेनाशब्दत्वादित्येव हेतुः । नास्ति शब्दो यस्मिन्निति विग्रहेणाश्रौतत्वादित्यर्थः । अशब्दत्वे च हेतुरीक्षतेरिति। श्रुत्युक्तकारणविशेषणस्येक्षणस्याभावादित्यर्थः इति। तत्र (तन्न), ईक्षणाभावेन “तदैक्षते” त्यादिश्रुत्यर्थत्वाभावस्यैव सिद्धेः, न तु सामान्यतः श्रुत्यर्थत्वाभावस्य । “प्रधानक्षेत्रज्ञपतिर्गणेशः, अजामेकाम्, यस्तन्तुनाभ इव तन्तुभिः प्रधानजैः स्वभावतः, मायां तु प्रकृतिं विद्यादि” त्यादिश्रुतिभिः प्रधानस्यापि श्रौतत्वात् । शक्तिविधया प्रधानस्य जगत्कारणत्वश्रवणाच्च “बह्वीः प्रजाः सृजमानां सरूपाः” इत्यादिश्रुतिष्विति ।}

\dev{कथं वाऽशब्दत्वमपि, स्मृतिशतात्, सांख्योक्तानुमानत, ईश्वराविकारित्वान्यथानुपपत्त्यादिभिश्च प्रधानहेतुत्वसिद्धेः । किं च अश्रौतत्वहेतोर्जीवेश्वरत्वसिद्धिः । न चात्र प्रधानमन्त्रस्य जगत्कारणत्वं निषिध्यत इति वाच्यम्, न्यूनतापत्तेः । उत्तरोत्तराधिकरणेषु जीवानामुत्पत्तिनिराकरणदर्शनादिति ।}

\dev{यत्तु “मायां तु प्रकृतिं विद्यादि” त्यादिश्रुतेरर्थं कल्पयन्ति मायात्राविद्या, तामेव प्रकृतिं जगत्कारणं विद्यादिति, तदप्यसारम् “अस्मान्मायी सृजते विश्वमेतत्तस्मिंश्चान्यो मायया सन्निरुद्धः” इत्यव्यवहितपूर्ववाक्ये का माया ? को वा मायया बद्ध ? इत्याकाङ्क्षायां प्रवृत्तस्य “मायां त्वि” त्यादिवाक्यस्य यथोक्तार्थतानौचित्यात् । प्रकृत्यादिकमुद्दिश्य मायादिशब्दवाच्यत्वप्रतिपादनं विना आकाङ्क्षानिवृत्त्यसंभवात् शुक्तिरजताद्यविद्यायां मायाशब्दाप्रयोगाच्च । ऐन्द्रजालिकमन्त्रादावेव व्यामोहकशक्तौ लोके मायव्यवहाराच्च । कार्ये कारणाभेदात्तु तद्रचितपदार्थेष्वपि मायाशब्दप्रयोगः । “भूयश्चान्ते विश्वमायानिवृत्तिरि” त्यादौ च मायाया व्यापारनिवृत्तिरेवावगम्यते न नाश इति । तस्मात्— }
\begin{verse}
\dev{मायां तु प्रकृतिं विद्यान्मायिनं तु महेश्वरम् ।}\\
\dev{अस्यावयवभूतैस्तु व्याप्तं सर्वमिदं जगत् ।।}
\end{verse}
\dev{इति श्रुतौ श्रुत्यन्तरस्मृतिप्रसिद्धप्रकृतेरेव मायाशब्दवाच्यत्वं विधीयत इति ।}

Even though one hears of the threefold creation in Purāṇas  like natural, mental, and physical (prākṛtamānasadehikarūpeṇa trividhā sṛṣṭiḥ) that is also with reference to Virāṭ Jīva (Puruṣa) the first one who possessed a body and not with reference to Parameśvara; just as having volition  (prayatnavat) there is contact with ātman (for rise of experience in the world) so also through volition/perseverence with adṛṣṭa (accumulated karma) also there can be contact with ātman for the creation of the world (adṛṣṭavadātmasamyogādapi jagadutpatteḥ).\footnote{A comparison is drawn with the way contact of ātman with the intellect takes place for experience in the world and contact with adṛṣṭa (collected dharma/ adharma) giving rise to the world from the Virāṭ Puruṣa.} 

\textbf{Ques:} In the view of those who desire (to explain) this sūtra (BS.I.1.5) as rejecting pradhāna etc., being the cause for the world, this sūtra needs to be explained, as follows : pradhāna etc., not mentioned in the Vedas/Upaniṣads is not the cause of the world as there is absence of the main qualifying adjective of seeing/visualizing. 

\textbf{Ans:} Here the word “aśabdam” (in the sūtra) is referring to its intrinsic nature (of non-occurance in śruti); but because of  that intrinsic nature it cannot be used to adhere to one side (of the meaning) (na tu aśabdatvenaiva pakṣatā); the nature of the main cause not having the nature of being specified by śruti (not mentioned in the Upaniṣads), it cannot be established in advance (mūlakāraṇaśrutyapratipādyatvarūpasyāśabdatvasya prāgasiddheḥ).\footnote{Bhikṣu points to the fact that aśabdam’ only denotes ‘not mentioned in śruti’. He therefore warns against reading too much into that and wants to stick to the intrinsic meaning of ‘aśabdam’. The same logic applies to ‘īkṣaṇa’ as well.} Being the main cause is being an independent cause not under the influence of another’s desire (mūlakā\-raṇatvam ca svatantrakāraṇatvam parechānadhīnakāraṇatvamiti yāvat). Because of that here also the word ‘īkṣaṇa’ used as a reason (hetau) only means its intrinsic nature. 

\textbf{The advaitin’s view according to Bhikṣu:} Modern day (Vedāntins/\-advaitins) explain this sūtra in the following manner: Pradhāna etc., which do not occur in the ‘īkṣaṇa’ śruti of the Vedas/Upaniṣads is not the cause of the world, as the qualification has the reason within itself (hetugarbhaviśeṣaṇatvena  aśabdatvādityeva hetuḥ).  The etymology of the word as ‘nāsti śabdo yasminniti’ means that in it there is (not the authority of) śruti.\footnote{The vigraha can also be “not śruti”(na śabdam iti) but that will not serve Bhikṣu’s purpose} The reason for its being absent in śruti (the Upaniṣads) is because of ‘visualizing’ i.e the special cause mentioned by the Upaniṣad as visualizing/seeing is absent. 

\textbf{Counter view by Bhikṣu:} Therein, because of having an absence of (the quality of) sight (thought) the (expression) “tadaikṣata” only establishes the absence of being the meaning of śruti (as seeing in this instance) and not the absence of that meaning of śruti in general.\footnote{Bhikṣu points out that this meaning applicable to a specific instance is not true of śruti in general as there are many instances where this meaning does not hold.} Thus: “pradhānakṣetrajñapatirguṇeśaḥ” (Śvet.Up.VI.16), “ajāmekām”(Mahānā.Up.8.4), “yastantunābha iva tantubhiḥ pradhānajaiḥ svabhavataḥ” (Bhasam. Up. 2.47),”māyām tu prakṛtim vidyāt” (Śvet.Up. 4.10) through all these śruti statements, pradhāna is also part of the śruti literature (pradhānasyāpi śrautatvāt). One hears of pradhāna being the cause of the world through śakti in such śruti statements as: “baḥvīhiḥ prajāḥ sṛjamānām sarūpāḥ” (Śvet.Up.4.5).

Because of hundreds of smṛtis (smṛtiśatāt) (mentioning prakṛti),\break because of the (statements establishing prakṛti) through inference in Sāṇkhya philosophy, because of the inactivity of Īśvara (Parameśvara) it is not reasonably possible to reject pradhāna being the reason/cause (for the rise of the world) even if it is not mentioned in śruti(katham vā aśabdatvamapi, smṛtiśatāt, sāṇkhyoktānumānāt, Īśvarāvikāritvānya\-thānupapatyādibhiśca pradhānahetvasiddheḥ). Moreover, due to its not being (mentioned as such) in śruti, there is establishment of jīva being the same nature like Īśvara (kim ca aśrautatvahetorjīveśvaratvasiddheḥ). Nor can it be said that it is as if only pradhāna being the cause of the world is being rejected as it will suffer from the danger of reducing (the scope). In the upcoming adhikaraṇas (sections of the BS) the rejection of the coming into being of the jīvas is seen.

The imagined/constructed meaning of such śrutis like: “māyam tu prakṛtim vidyāt” (Śvet.Up.4.10)  as māyā here means avidyā, know that alone (“vidyāt) as “prakṛti” the cause of the world, that is also without meaning. Then what is (the meaning of) māyā (mentioned) in the immediate sentence above as: “asmānmāyī sṛjate viśvametattasminścānyo māyayā sanniruddhaḥ” (ibid.4.9). It is with the desire to know the answer to what is restrained by māyā that this “māyām tu” etc., was started; therefore, it is not proper to give the meaning (of prakṛti) as stated above. Without indicating that the meaning denoted by the word “māyā” etc., is with reference to prakṛti etc., it is not possible to remove that desire. Moreover the word ‘māyā’ is also not used in the error/ignorance of mistaking a mother of pearl for silver (śuktirajatādyavidyāyām māyaśabdāprayogācca).

It is only in mantras used in sacrifices connected with Indra (aindrajālikamantrādāveva) that (the word) māyā is used with regard to the power of delusion/confusion in the world. Since the effect is non- different from the cause (in satkāryavāda) the word māyā is used with reference to the products created by it.\footnote{Since all products evolve from prakṛti (according to SY) they share in the same characteristics of sattva, rajas and tamas and the  term māya can legitimately apply to them as well.} In “bhūyaścānte viśvamāyānivṛttiḥ” (Śvet.1.10) etc., one only understands the withdrawal of the activity of māyā and not its destruction.Therefore the śruti statement: “māyām tu prakṛtim vidyānmāyinam tu maheśvaram…vyāptam sarvamidam jagat” only establishes the meaning denoted of māya as prakṛti which is well known in other śruti texts called smṛti.\footnote{The importance that Bhikṣu attaches to the smṛti texts and the Purāṇas are very well known as he often cites them as on par with śruti texts themselves.}

\dev{तदेवमीक्षापूर्वकमहदाद्यखिलसृष्टिश्रवणात् केवलप्रधानादिजडवर्गात् हिरण्यगर्भादिभ्यश्चातिरिक्तं सदा सर्वज्ञं पर ब्रह्मेति सिद्धम् । तदुपाधौ विश्वाकारनित्यवृत्त्यनङ्गीकारे सर्वकर्तृत्वानुपपत्तेः, स्वोपाधिवृत्तावेव स्वस्य अकर्तृत्वात्। तथा च “स विश्वसृग् विश्वविदात्मयनिरि” त्यादिश्रुतिविरोधः स्यात् । आत्मयोनितया तु ब्रह्मापि क्वचित् आत्मयो\-निरित्यंशांशिनोरभेदेनोच्यते आत्मनो जीवचेतनस्य योनित्वाद् वेति बोध्यम् । धारावाहिकानन्तवृत्तिस्वीकारे च गौरवमात्रमधिकं स्यात् ।}

\dev{न चैवं नित्यवृत्तिसत्त्वेन प्रकृतेः प्रतिक्षणपरिणामित्वभङ्ग इति वाच्यम् परिमाणादिवत् वृत्तिविशेषस्य नित्यत्वेऽपि जीववृत्त्यादिभिः प्रतिक्षणपरिणामसंभवाच्च । ईशोपाघेरपरिणामित्वेऽपि जडत्वसाधर्म्येण प्रकृतिमध्ये प्रवेशसंभवाच्चेति । अथैवं “स ईक्षाञ्चक्रे, सोऽकामयते” त्यादिश्रुतिषु ब्रह्मणो ज्ञानकामनाद्युत्पत्तिवचनमनुपपन्नमिति चेन्न, स्वकार्यजनकसामग्रीसमवधाननिमित्तकाभिव्यक्तिरूपोत्पत्तेः लोकव्यवहार अनुवादात् । लोकेऽप्यभिव्यञ्जकस्य तैलसंबन्धस्यानन्तरं व्यवह्रियते । “कुङ्कुमगन्ध इदानीं जात” इति । अथवाऽमुककालेऽमुकं पश्यन् करिष्यामीतीक्षण संकल्पाद्यवच्छेदककालाद्युत्पत्त्यैवेक्षणाद्युत्पत्तिरुपचर्यते, “शिखी जात” इत्यादिवदिति मन्तव्यम् ।}

\dev{जीवज्ञानादिवदीश्वरज्ञानादेरनित्यत्वाङ्गीकारे च—}
\begin{verse}
\dev{नैवाहस्तस्य न निशा नित्यस्य परमात्मनः ।}\\
\dev{उपचारात्तथाप्येतत्तस्येशस्य द्विजोच्यते ।।}\\
\dev{अभयं भ्रान्तिरहितमनिद्रमजरामरम्।।}\\
\end{verse}
\dev{इति विष्णुनारसिंहपुराणादाववगम्यमानमीश्वरस्य जीवव्यावृत्तं ज्ञानोपरमादिनिरूप निशादिशून्यत्वं नोपपद्येत । उपचारश्चाभिव्यक्त्यादिरूपो व्याख्यात एवेति ।}

Thus in this manner, by hearing the creation of everything starting with mahat preceded by seeing/thought it is established that only the omniscient, supreme Brahman is different from the singular, dull pradhāna etc., (kevalapradhanādijaḍavargāt) and from entities such as Hiraṇyagarbha etc at all times. Since in the upādhi (limitation of pure sattva) of Brahman there is no acceptance of an eternal mental-modification of the image of the world/universe viśvākāranityavṛttyanaṅgīkāre) it is unreasonable to (assume the) creatorship of everything. It is not possible for oneself to function in one’s own upādhi.\footnote{This is the favourite argument of Bhikṣu i.e. the ‘kartṛkarma virodha’ the contradiction of the agent and action being the same.} Thus it will contradict such śruti statements as: “sa viśvasṛg viśvavidātmayoniḥ” (Śvet.Up.6.16). By the expression “ātmayoni” it means that Brahman at some point in time was its own source of origin through the identity of the part and the whole; or it can be understood as due to the origin of the consciousness of the jīva (explanation of ātmayoniḥ in the quote above). If one accepts endless modifications as a stream there will only be a lot of cumbersome (explanations).

\textbf{Ques:} Nor can it be said that in this way, having eternal modifications there is discontinuity of prakṛti at every moment (na caivam nityavṛttisattvena prakṛteḥ pratikṣaṇapariṇāmitvabhaṅga iti vācyam). 

\textbf{Ans:} Just like weight etc., even though the special vṛtti is eternal it is possible to have changes every moment through the modifications of the jīvas.\footnote{Bhikṣu thus draws a distinction between the eternal vṛtti and the jīvavṛttis and allows the changes through the jīvavṛttis.} Even if the limitation of Īsa is not subject to change, due to sharing the same quality of insentience it can be included under prakṛti. 

\textbf{Ques:} Then, if it is said that in such śrutis as: “sa īṣāñcakre” (Praś.Up 6.3)”,  “so’kāmayata” (Taitt.Up. 2.6; Bṛ.Up.1.2.4 and many other places) the mention of the rise of desire, knowledge etc., with reference to Brahman is not compatible with reason; then the answer is: 

\textbf{Ans:} that is not so; it is only a reiteration of the way things are accomplished in the world in the form of manifestation of a thing for which due attention is paid to the collection (of things) needed for accomplishment of one’s desire (svakāryajanakasāmagrīsamavadhānanimittakābhivyaktirūpotpatteḥ lokavyavahārasiddhāyā anuvādāt). Even in the world after the manifestation of something which is connected with oil it is said:  “kuṅkumagandha idānīm jāta”\footnote{Kumkuma is actually saffron; maybe it was considered an oil and so the phrase tailasambandhasyānantaram is used.}. Or ‘by seeing this person at this time I will do this’ (such utterances) are used figuratively to denote the act of seeing etc., only at the time delimited by the decision of seeing; one should comprehend it like the example “śikhī jātaḥ”.\footnote{Here the śikhī probably conveys the sense of a learned person who keeps a tuft of hair (sikhā) as the brahmins normally did and do so even today. Maybe the idea is to convey that one who will be very wise in the future is born. The import of this analogy is not very clear.} 

If like the knowledge of jīva etc., one accepts the ephemeral nature of Īśvara’s knowledge etc., it will not be in consonance with what is learnt in the sayings of the Viṣṇu, Nārasimha and other Purāṇas as: “nāivāhastasya na niśā…bhrāntirahitamanidramajarāmaram” i.e. that the nature of the cessation of knowledge of Īśvara which is different from that of the jīva in the form of cessation of knowledge devoid of darkness, will not stand to reason. Explanation has been given in a figurative manner in the form of manifestation etc., (upacāraścābhivyaktyādirūpo vyākhyāta eveti).

\dev{स्यादेतत्, ईक्षणनिमित्तकप्रकृतिप्रवृत्तिस्वीकारे प्रकृतिस्वातन्त्र्यप्रतिपादकयो:      सांख्ययोगयोरप्रामाण्यप्रसङ्गः । तौ हि पुरुषार्थमात्रप्रयुक्तायाः प्रकृतेः स्वयं पुरुषेण संयुक्तायाः सकाशान्महदाद्युत्पत्तिं मन्येते। सांख्ययोगयोश्चायमेव विशेषो यत् तात्पतिं मन्येते । त्रिविधोऽपि  सात्त्विकराजसतामसरूपो महान् सांख्यमते पूर्वकल्पसिद्धजीव एव, तत एव च ब्रह्मविष्णुमहेश्वरा इत्युच्यन्ते, न पुनर्नित्य ईश्वरोऽस्तीति । योगमते तु महतः सात्त्विको भागो नित्यैश्वर्यशक्तिमान् नित्यमुक्तः “क्लेशकर्मविपाकाशयैरपरामृष्टः परुषविशेष ईश्वर” इति योगसूत्रात् सत्त्वाख्यकार्योपाधिकत्वं जन्यज्ञानत्वं परेषामप्यनुमतमिति ।}

\dev{न चैवम् “आदिः स संयोगनिमित्तहेतुः परस्त्रिकालादकलोऽपि दृष्टः” इत्यादिश्रुतिविरोधात् सांख्ययोगयोरप्रामाण्यमेवास्त्विति वाच्यम्, “तत्कारणं सांख्ययोगाधिगम्यं ज्ञात्वा देवं मुच्यते सर्वपाशैरि” ति श्वेताश्वतरादिश्रतिभ्यः, “नास्ति सांख्यसमं ज्ञानं नास्ति योगसमं बलमि” त्यादिस्मृतिभ्यश्च तत्प्रामाण्यसिद्धेः । यच्च सांख्ययोगशब्दयो रूढिं परित्यज्यान्यार्थतावर्णनं, तदपि न, “सांख्यस्य वक्ता कपिलः, हिरण्यगर्भो योगस्य वक्ता नान्यः कदाचने” ति योगियाज्ञवल्क्यादेः,}
\begin{verse}
\dev{संख्यां प्रकुर्वते सांख्याः प्रकृतिं च प्रचक्षते ।}\\
\dev{तत्त्वानि च चतुर्विंशत् तेन सांख्या इति स्मृताः ।।}
\end{verse}
\dev{इति मोक्षधर्मादेश्च विरोधादिति । अत्रोच्यते— केवलजीवात्मज्ञानादपि मोक्षो भवतीति प्रतिपादयितुं सांख्या अनीश्वरबौद्धमताभ्युपगमवादेन प्रतिज्ञातमात्मनात्मविवेकं प्रतिपादयन्ति, ईश्वरव्यवस्थापनस्य स्वशास्त्रेऽनुपयोगात्, श्रुतिभ्यो ब्रह्मविष्णुशिवातिरिक्तेश्वरसाधने प्रयासबाहुल्यात् ब्रह्ममीमांसयैव तत्साधनस्य कृतत्वाच्च । अभ्युपगमवादश्च विष्णुपुराणादौ दृष्टः —}
\begin{verse}
\dev{एते भिन्नदृशां दैत्या विकल्पाः कथिता मया ।}\\
\dev{कृत्वाभ्युपगमं तत्र संक्षेपः श्रूयतां मम ।।}
\end{verse}
\dev{इत्यादिषु । न पुनरीश्वरमेव परमार्थतोऽपि सांख्या नाङ्गीकर्बन्ति “तत्कारणं सांख्ययोगाधिगम्यमि”—}

\dev{त्युक्तश्रुत्यादिभिस्तदभ्युपगमावगमात्, “जगदाहुरनीश्वरमि” त्या दिना स्मृतिष्वासुरत्वेन निरीश्वरवादस्यात्यन्तनिन्दितत्वाच्चेति ।}

Let it be so; if one accepts the activity ofprakṛti preceded by the cause of seeing  (being the cause of) then there will be the difficulty of rejecting the proofs of Sāṇkhya and Yoga which  declare the independence of prakṛti. They (SY) consider that from prakṛti which is employed for the sake of only achieving the goals of puruṣa (puruṣārthamātraprayuktāyāḥ prakṛteḥ) through that proximity with puruṣa by itself (svayam puruṣeṇa samyuktāyāḥ) there is the origin of mahat etc. There is only this difference between Sāṇkhya and Yoga: According to Sāṇkhya the threefold mahat of the nature of sattva, rajas and tamas is a great siddhajīva which existed in a previous age (mahān sāṇkhyamate pūrvakalpasiddhajīva eva). It is also from that (mahat) that Brahma, Viṣṇu and Maheśvara come forth.\footnote{The supreme Parameśvara is not the same as this Maheśvara. Śiva is what is meant here which are the avatāras. According to Bhikṣu they are not supreme.} So also (according to Sāṇkhya) there is no eternal Īśvara. In Yoga, on the other hand, the sāttvika part of mahat is eternally, supremely powerful, eternally free and according to the YS it is : “kleśakarmavipākāśayairaparāmṛṣṭaḥ puruṣaviśeṣa īśvara” (YS.I.24);\footnote{Īśvara is a special puruṣa untouched by kleśa (afflictions), karma (good and evil actions) and vipāka (result of actions)} its having the effect-limitation called sattva (intellect) (sattvākhyakāryopādhikatvam) which generates knowledge is also accepted by others.\footnote{Advaita for instance also accepts the production of knowledge through the witness consciousness in the antaḥkaraṇa which is similar to this.} 

Then, let it not be said that since it is in contradiction to such śruti statements as: “ādiḥ sa…parastrikālādakalo’pi dṛṣṭaḥ” (Śvet.Up. 6.5) SY has no authoritative proof.  From such śruti sayings as: “tatkāraṇam śaṇkhyayogādhigamyam…sarvapāśaiḥ” (ibid.6.13) in the Śvet.Up.etc., and from such smṛti sayings as: “nāsti sāṇhkyasamam jñānam nāsti yogasamam balam” that proof has been established. Moreover, giving up the conventional meaning of Sāṇkhya and Yoga and describing it through another meaning is also not right, since it is said in the Yogiyājñavalkya: “sāṇkhyasya vaktā kapilaḥ, hiraṇyagarbho yogasya vaktā nanyaḥ kadācana” and it is also in contradiction to what is mentioned in the Mokṣadharma.P as: “saṇkhyām prakurvate sāṇkhyāh…caturvimśat tena sāṇkhyā iti smṛtah”.

It is said: In order to demonstrate that even through knowledge of jīvātman alone mokṣa can be attained, the Sāṇkhya philosophers, taking recourse to reasoning obtained from the Buddhists who do not believe in Īśvara, have declared that the (knowledge of differentiation) (an insight into the difference) between ātman (puruṣa) and non-ātman (prakṛti) is sufficient  for mokṣa, since there is no use for the establishment of Īsvara in their śāstra.\footnote{The reason that Sāṁkhyas do not mention Īśvara is because there is no role for Īśvara in their philosophy according to Bhikṣu.} There is a lot of effort for the establishment of Īśvara apart from Brahmā, Viṣṇu and Śiva; also that task has been accomplished by the Brahmasūtra itself. The argument for its acceptance is seen in Viṣṇu P. etc., like: “ete bhinnadṛśām daityā…tatra saṇkṣepaḥ śrūyatām mama”. Moreover, even though the Sāṇkhya philosophers do not agree to Īśvara, though real, we understand through such śruti statements as:  “tatkāraṇam sāṇkhyayogādhigamyam” etc., that it is accepted.\footnote{Even in the face of Sāṇkhya statements not accepting Īśvara, Bhikṣu a staunch theist does not want to concede it.} (We understand this) also because of the smṛti sayings like: “jagadāhuranīśvaram” etc., wherein the non-Īśvara argument is totally condemned as an argument of the asuras.

\dev{योगास्तु ईश्वरमभ्युपगम्यापि तस्य वैषम्यनैर्घृण्यनिरासायासलाघवाय स्वभाववादाभ्युपगमवादेन प्रकृतिस्वातन्त्र्यमनूद्यैवेश्वरप्रणिधानात् योगादिकं प्रतिपादयन्ति, ईश्वरस्य भक्तभूतानुग्रहमात्रप्रयोजनस्वीकारेण अन्यथा प्रकृतिपुरुषयोः स्वत एव संयोग “आदिः स संयोगनिमित्तहेतरि” त्यादिश्रुतिविरोधेन योगाप्रामाण्यानौचित्यादिति दिक् ।}

\dev{इदं पुनरिहावधेयम्—रजस्तमःसंभिन्नतया मलिनं कार्यसत्त्वं परमेश्वरस्य नोपाधिः, किन्तु केवलं नित्यज्ञानेच्छानन्दादिमत् सदैकरूपं कारणसत्त्वमेव तस्योपाधिः “न स्थानतोऽपि परस्योभयलिङ्गमि” त्यागामिसूत्रात् । सृष्टिस्थितिसंहारेष्वधिकारिणस्तु सत्त्वादिकार्योपाधिकास्त्रयो देवा ब्रह्मविष्णुमहेश्वराः महदाख्या अवान्तरेश्वरा एव । न च “मत्तः परतरं नान्यत् किंचिदस्ति धनञ्जये” त्यादिगीतादिविरोध इति वाच्यम्, तथाविधवाक्यानां व्यक्तपरत्वात् ईश्वरस्य वा (चा) व्यक्ततया व्यवहार्यत्वाभावात् । एतेन तेषां परमेश्वरत्वादिकमपि व्यक्तापेक्षया बोध्यम् । नित्यत्वं च “आभूतसम्प्लवं स्थानममृतत्वं हि भाष्यत” इति स्मृतेर्बोध्यम् ।}
\begin{verse}
\dev{ब्रह्मादीनां त्रयाणां तु स्वहेतौ प्रकृतौ लयः ।}\\
\dev{प्रोच्यते कालयोगेन पुनरेव समुद्भवः ।।}
\end{verse}
\dev{इति मात्स्यादिभ्यस्तेषामप्युत्पत्तिलयसिद्धेरिति । अथवा “तद् यो यो देवानां प्रत्यबुध्यत स एव तदभवदि” - ति श्रुतेर्विष्णुदेवताया औत्पत्तिकब्रह्मात्मभावात् “मत्त: परतरं नास्ति” इत्यादिवचनमुपपद्यते। तथा च वक्ष्यत्याचार्यः— शास्त्रदृष्ट्या तूपदेशो वामदेववदि” ति । अत एवानुगीतायां “परं हि ब्रह्म कथितं योगयुक्तेन तन्मये” ति श्रीकृष्णवचनाद् भगवद्गीतायां परं ब्रह्मैव कार्यव्रह्मणा श्रीकृष्णेन अहमित्युपदिष्टमिति निर्णीतम् । अन्येषामपि ज्ञानिनां ब्रह्मभावेऽपि विष्णावेव मुख्यतो ब्रह्मभावाविर्भावात् ‘‘मत्तः परतरं नास्ती” ति वचनं युक्तं नान्यस्य (नान्यत्र) । तदुक्तं मोक्षधर्मे तस्य सर्वज्ञानिमुख्यत्वम्— }
\begin{verse}
\dev{अनाद्यन्तं परं ब्रह्म न देवा नर्षयो विदुः ।}\\
\dev{एकस्तद् वेद भगवान् धाता नारायणः प्रभुः ।।}
\end{verse}
\dev{नारायणात् सृष्टिगण ” इत्यादिना । तथा शिवस्यापि “मत्त: परतरं नास्तीति” युक्ति (उक्ति) र्युक्ता ।}

Even though the yoga philosophers accept Īśvara, in order to reduce (the criticism of) injustice, cruelty, contradiction (and) fatigue, by accepting the logic of naturalists (svabhāvavādābhyupagavādena), after mentioning the independence of prakṛti itself, they point to yoga etc., through (mentioning) devotion to Īśvara,\footnote{This is the same argument that Śaṇkara presents under BS.II.1.34 : vaiṣamyanairghuṇyaena na sāpekṣatvāt tathā hi darśayati. The well known criticism of ‘if there is an all-knowing, benevolent Īśvara how can there be so much injustice and suffering’ is dealt with in a different manner. Here Īśvara only serves the purpose of yoga through devotion to Īśvara . “Īśvarapraṇidhānādvā” (YS.I.23).} by admitting the purpose of only ‘anugraha’ (blessing of Īśvara) towards those who practice devotion. Alternately (admitting) the contact of prakṛti and puruṣa by itself (svata eva samyoge), which goes against such śruti sayings as: “ādiḥ sa saṁyoganimittahetuḥ” (Śvet.6.5) it is not correct to declare the non-authoritativeness of yoga philosophy.\footnote{This is possibly a reference to BS. II.1.3 and Śāṅkarabhāṣya under it. Bhikṣu defends Yoga as the criticism of its Īśvara cannot hold since it is just another puruṣa and does not give rise to the world.} 

One needs to pay attention to (the fact that) the effect-sattva which is tainted, since it is mixed with rajas and tamas is not a limitation of Parameśvara. However the causal-sattva, having eternal knowledge, desire and bliss, always of the same form, is its limitation.\footnote{Distinguishing the śuddha sattva upādhi of Īśvara from the kāraṇa sattva.} This is (known) from the up coming sūtra: “na sthānato’pi parasyobhayaliṅgam” (BS.3.2.11).\footnote{This sūtra declares that Brahman is unchangeable even when having upādhis and compares It to a transparent crystal not changing its essential nature even in the presence of a red flower for instance next to it.} The three devas- Brahmā, Viṣṇu and Śiva- in charge of creation, sustenance and destruction- having the limitation of the effective-sattva known as mahat are only intermediate Īśvaras (kāryopādhikāstrayaḥ devāḥ…mahadākhyā avāntareśvarā eva).\footnote{This is indeed a bold statement of Bhikṣu demoting these three worshipped as ‘supreme avatārapuruṣas’ to just an intermediate position.} This should not be viewed as a contradiction to what is stated in the Gītā: “mattaḥ parataram nānyat kiñcidasti dhanañjaya” (7.7). Such sentences have clear reference to the solely manifested individual or to Īśvara where there is definitely absence of any worldly activity.\footnote{Seems a very poor argument and unconvincing and also not spelt out clearly what he intends to say.} Through this, their being of the nature of Parameśvara in their manifested form is also indicated.\footnote{Seems a very convoluted argument and totally unconvincing. Having dismissed Brahmā, Viṣṇu and Śiva as inferior and then to elevate Śrī Kṛṣṇa in the Gītā smacks of ‘jarjarīnyāya’.} Being eternal is made clear by the smṛti saying: “ “ābhūtasamplavam sthanamamṛtatvam hi bhāṣyato”.  

By statements In Purāṇas such as Matsya etc.,: “brahmādīnām trayāṇām…kālayogena punareva samudbhavaḥ” it is established that they also have an origin and disappearance. Or by the śruti: “tad yo yo devanām pratyabudhyata sa eva tadabhavat” (Bṛ.Up.I.4.10)  since Viṣnu devatā has the state of Brahmātman innately the saying: “mattaḥ parataram nāsti’ is proper Therefore the ācārya will state: “śāstradṛṣṭyā tadupadeśo vāmadevavat” (BS.I.1.30).\footnote{Using the example of Vāmadeva this sūtra asserts that instruction is based on the experience of a seer which agrees with what the śāstras say. The Bṛ.Up I.4.10 declares that Vāmadeva who had realized the Self also knew that he was Manu and the sun as well.In other words this is an argument to accept the validity of the utterances of the śāstras and in this case that of Kṛṣṇa being equal to Īśvara .} Therefore it is decided in the Anugītā through Kṛṣṇa’s words: “param hi brahma kathitam yogayuktena tanmayā” Āśva.P.16.13 that Śrī Kṛṣṇa the effect-Brahman in the Bhagavadgītā has used ‘aham’ in the sense of Brahman itself. In the case of other wise men also when they reach the state of Brahman, as there is mainly the rise of the state of Brahman like Viṣṇu, the statement “mattaḥ parataram nāsti” is correct, and not elsewhere. Thus the importance of those wise individuals is stated in the Mokṣadharma as: “anādyantam param brahma na devā…bhagavān dhātā nārāyaṇaḥ prabhuḥ”. By Nārāyaṇa is meant the creator group etc (Mokṣa.210.23; cited in Tri. p.70. fn.1). In a similar manner the statement: “mattaḥ parataram nāsti” applies logically to Śiva also.

\dev{ये हि जीवाः पूर्वपूर्वसर्गेषु करणवर्गेण सहैव सायुज्यमुक्त्या परमेश्वरतां गताः, ते वासुदेवव्यूहे अन्तर्भवन्ति । तत्र च व्यूहे एक एव वासुदेवो नित्येश्वरः, इतरे तदंशाः वासुदेवाः तथा संकर्षणप्रद्युम्नानिरुद्धाख्यव्यूहरूपिणः विभूतिगणाः पूर्वसर्गसिद्धाः, त एते यथायोग्यं सन्ति, महदादिविराडन्तरूपेण ब्रह्मविष्णुरुद्रादिरूपेण चांशावताराः परमेश्वरस्य भवन्ति “आगच्छन्ति यथाकालं गुरोः सन्देशकारिणा (कारिणः)” इति मोक्षधर्मात् । तथा च ये हरिहरादयः परमेश्वरकोटयस्तेषां “मत्तः परतरं नास्ती” ति तद्वचनमुपपद्यत एव । यद्यपि तेषां जगद्व्यापारवर्जमेवैश्वर्यं तथापि परमेश्वरात्मकतया सर्वस्रष्टृत्वसर्वाधारत्वाद्युपदेशोऽपि तेषु युज्यते इति। तेषां च कार्योपाधित्वम्—}

\dev{गुणेभ्यः क्षोभ्यमाणेभ्यस्त्रयो देवा विजज्ञिरे ।}
\begin{verse}
\dev{एका मूर्तिस्त्रयो देवा ब्रह्मविष्णुमहेश्वराः ।।}\\
\dev{सात्त्विको राजसश्चैव तामसश्च त्रिधा महान् ।}
\end{verse}
\dev{इति मात्स्यादिभ्य इति ।}

\dev{बुद्ध्याख्यं समष्टिर्महत्तत्त्वं हि त्रिगुणात्मकमेकमेव सत्त्वाद्यंशभेदेन त्रयाणां देवानां सूक्ष्मशरीरं भवतीत्येका मूर्तिरुच्यते हरिहरात्मकदेहवत् । अतएव सूक्ष्मशरीराधारभूतं पिण्डाख्यं स्थूलशरीरमप्येषामेकमेवेति त्रयाणामेव विश्वरूपत्वमुपपद्यते । त्रयश्च देवाश्चेतनरूपा भिन्ना एवेत्यर्थः । अतएव—}
\begin{verse}
\dev{अन्योन्यमनुरक्तास्ते अन्योन्यमनुजीविताः ।}\\
\dev{अन्योन्यं प्रगताश्चैव लीलया परमेश्वराः ।।}
\end{verse}
\dev{इति मात्स्यादिवाक्यमप्युपपन्नं त्रयाणामेकपिण्डतया परस्परसापेक्षत्वात् वातपित्तकफवत् । एतेन त्रयाणां भेदाभेदावपि व्याख्यातौ वेदितव्याविति ।}

Those jīvas, who in many previous evolutionary cycles along with the group of instrumental causes have reached Parameśvara through sāyujya liberation (becoming one with Parameśvara) they are included in the group of vyūhas of Vāsudeva.\footnote{Pāñcarātra texts mention four vyūhas i.e. Vāsudeva, Saṅkarṣaṇa, Pradyumna and Aniruddha. Vāsudeva has all the six guṇas- jñāna, aiśvarya, śakti, bala, vīrya and tejas-essential for creation. Vāsudeva has all the six guṇas while the succeeding vyūhas have two each of the guṇas. It means that each of the vyūhas is Viṣṇu himself with six guṇas of which only two are manifest (For more see Rukmani, \textit{A Critical Study of the Bhāgavata Purāṇa} :200)}  In the vyūhas only Vāsudeva is eternally supreme (nityeśvaraḥ); the others are parts of Vāsudeva (tadamśāḥ vāsudevāḥ); thus the others in the form of vyūhas-Samkarṣaṇa, Pradyumna and Aniruddha-belong to the group of vibhūtis who are siddhas of the previous evolution. They are manifest as appropriate in the form of mahat etc., Virāṭ, etc., in the form of Brahmā, Viṣṇu,  Rudra etc., all of them being partial avatāras of Parameśvara as mentioned in the Mokṣadharma: “āgacchanti yathākālam guroḥ sandeśakāriṇaḥ”. Thus it is appropriate to use the statement: “mattaḥ parataram nāsti” to those Hari,Hara etc., who are included within Parameśvara’s group. Even though their supremacy does not include their creation of the world (jagadvyāparavarjamevaiśvaryam) still since they are of the essence of Parameśvara, messages/instruction such as havng the quality of being the creator of everything, having the quality of being the support of everything with reference to them, is appropriate. They have the effect-limitation (teṣām ca kāryopādhitvam) as stated in the Matsya.P: “guṇebhyaḥ kṣobhyamāṇebhyaḥ…sāttviko rājasaścaiva tāmasaśca tridhā mahān” (Matsya.P. 4.16. cited in Tri. p.71. fn.1)\footnote{Purāṇas are as much śāstra as śruti and smṛti for Bhikṣu.} 

The collective principle of Mahat called Buddhi composed of the three constituents is one alone; by division in part as sattva etc., they constitute the subtle body of the three devas; so they are called as one embodiment/manifestation like the embodiment in essence of Hari and Hara. That is the reason why based on the support of the subtle body the gross body called piṇḍa (solid) of these are just one and all three possess universal form. All the three devas of the nature of consciousness are different. Thus the saying in the Matsya P.: “anyonyamanuraktāste…līlayā parameśvaaḥ” is correct; since the three possess one solid manifestation they are mutually dependent like vāta, pitta and kapha.\footnote{These three constituents are inherent in the body of each individual according to Āyurveda and the prominence of one amongst the three defines the characteristic of each individual as one being a vāta (wind) or pitta (bile) or kapha (phlegm) individual. This is however a strange analogy to characterize the three devas.} By this (statement) one should understand that the difference and non-differece of the three has been explained.

\dev{ननु कार्योपाधिकत्वेऽपि योगिकायव्यूहवत् तेषामीश्वरलीलावतारत्वमेवास्तु, किमर्थं विष्ण्वादीनां चेतनान्तरत्वं कल्प्यते, “त्रिधा कृत्वात्मनो देहं सोऽन्तर्यामीश्वरः स्थितः” इति स्मृतेरिति चेन्न, कूर्मविष्णुपुराणादिषु पुरुषान्तरेण सह प्रकृतिं संयोज्यैव परमेश्वरस्तदुभयात्मकं देवतात्रयरूपं महान्तं सृजतीत्यवगमात् ।}

\dev{यथा कौर्मे—}
\begin{verse}
\dev{प्रधानं पुरुषं चैव प्रविश्याशु महेश्वरः ।}\\
\dev{क्षोभयामास योगेन परेण परमेश्वरः ।।}\\
\dev{प्रधानात् क्षोभ्यमाणाच्च तथा पुंसः पुरातनात् ।}\\
\dev{प्रादुरासीन्महद्बीजं प्रधानपुरुषात्मकम् ॥ इति ।}
\end{verse}
\dev{न हि विष्ण्वादिदेवतायाः कृष्णाद्यवतारेषु पुरुषान्तरप्रवेशः श्रूयते फलवान् वा भवति । अतो विष्ण्वादिदेवानां न साक्षादीश्वरावतारत्वं किं त्वंशावतारत्वम् , अतएव ऋष्यादीन् प्रकृत्य स्मर्यंते— एते}
\begin{verse}
\dev{चांशकलाः पुंसः कृष्णस्तु भगवान् स्वयम् ।}\\
\dev{इन्द्रमायाकुलं लोकं मृडयन्ति युगे युगे ।।}
\end{verse}
\dev{इति । अत्र कृष्णो विष्णुः स्वयं परमेश्वरस्तस्य पुत्रवत् साक्षादंश इत्यर्थः, ऋष्यादीनामंशांशिवचनादिति । ईश्वरस्य प्रवेशश्च ज्ञानेच्छाप्रयत्नैर्व्यापनं, जीवस्य क्षोभश्च गुणद्वारा, योगश्च ऐकाग्र्यमिति । अयं च क्षोभः संयोगविशेषद्वारा महतो हेतुः । तथा—}
\begin{verse}
\dev{गुणोपाधिकभेदेषु त्रिष्वेतेषु सनातनः ।}\\
\dev{संयोज्यात्मानमखिलं जगत्कार्यं करोति यः ।।}\\
\dev{यस्याज्ञया जगत्सर्वं ब्रह्मा सृजति नित्यशः ।}\\
\dev{हरिश्च पालको रुद्रो नाशकः स हि मोक्षदः ।।}\\
\dev{मायाया गुणभेदेन विष्णुं रुद्रं पितामहम् ।}\\
\dev{सृष्ट्वाऽनुप्राविशच्चैषामन्तर्यामितया परः ।।}\\
\dev{ब्रह्माद्या मुनिशार्दूल पराभेदेन केवलम् ।}\\
\dev{कुर्वन्ति सर्गस्थित्यन्तान् परतत्त्ववदेव हि ॥}\\
\dev{मायाकार्यगुणछत्रा (च्छन्ना) ब्रह्मविष्णुमहेश्वराः ।}\\
\dev{मायोपाधिपराग्रूपा न जीवव्यूहसंस्थिताः ।।}\\
\dev{अनाद्य तं (द्यन्तं )परं ब्रह्म न देवा नर्षयो विदुः ।}\\
\dev{एकस्तद् वेद भगवान् धाता नारायणः प्रभुः ।।}\\
\dev{नारायणाद् ऋषिगणास्तथा मुख्याः सुरासुराः ।}\\
\dev{राजर्षयः पवित्राश्च परमं दुःखभेषजम् ।। इति,}\\
\dev{ब्रह्मविष्णुशिवादीनां यः परः स महेश्वरः ।}
\end{verse}
\dev{इति च नारदीयस्कान्दमोक्षधर्ममात्स्यवाक्येभ्योऽपि न साक्षादवतारत्वम्}

\textbf{Ques:}  But then, in spite of possessing the effect-limitation let them possess the incarnation in sport of Īśvara, like the assemblage of the body of an yogī; what is the need for imagining another consciousness like Viṣṇu etc., following the smṛti which says: “tridhā kṛtvātmano deham so’ntaryāmīśvaraḥ sthitaḥ”? Then the answer is: 

\textbf{Ans:} it is not so; one gets to know from Purāṇas like Kūrma, Viṣnu etc., that it is only by the connection of prakṛti with another puruṣa that Parameśvara creates that great embodiment of the form of the three devas of the twofold essence. Thus according to the Kūrma P: “pradhānam puruṣam caiva praviśyāśu maheśvaraḥ…pradhanapuruṣātmakam” one does not hear the incarnations of Viṣṇudevatā etc., such as Kṛṇṣa etc., directly nor is it fruitful (in creation).\footnote{In Bhikṣu’s avibhāga-advaita Parameśvara enters both puruṣa and prakṛti and in the churning through yoga comes forth the manifestation of the world.} Therefore, Viṣṇu and other devas are not direct incarnations of Īśvara (Parameśvara)\footnote{Bhikṣu uses both Īsvara and Parameśvara to denote the highest truth.} but they are only partial incarnations. That is why the ṛṣis are remembered separately by calling them: “ete cāmśakalāḥ…mrḍayati yuge yuge” (Bha.P.I.3.28). By mentioning that the ṛṣis are parts of the whole it means that Kṛṣṇa/Viṣṇu is himself Parameśvara, being a part directly like his son. The entry of Īśvara means its enveloping through knowledge, desire and effort;   the disturbance of the jīva is through the guṇas; yoga is one-pointedness (ekāgryam).\footnote{This is an explanation of the different words in the above quotation: “pradhānam puruṣam caiva...”} This disturbance due to a special contact is the cause for (the rise of) mahat. Thus it is known from the Nāradīya P, Skanda P, Mokṣadharma and Matsya P statements: “guṇopādhikabhedeṣu triṣveteṣu sanātanaḥ…brahmaviṣṇuśivādīnām yaḥ paraḥ sa maheśvaraḥ” that there is no direct incarnation (of Viṣṇu).

\dev{तथा मनुयाज्ञवल्क्ययोरपि ‘‘खं संनिवेशयेत् खेष्वि” त्यादिना “प्रजने च प्रजापतिमि” त्यन्तेन सर्वशरीरेषु ब्रह्मविष्णुरुद्रपर्यन्तान् देवानुक्त्वा तेषां विष्ण्यादीनामप्यन्तर्यामितया पुरुषान्तरमुक्तम्,}
\begin{verse}
\dev{प्रशासितारं सर्वेषामणीयांसमणोरपि ।}\\
\dev{रुक्माभं स्वप्रधीगम्यं विद्यात्तं पुरुषं परम् ।।}
\end{verse}
\dev{इत्यादिना । तथा मनुनैव—}
\begin{verse}
\dev{यत्तत् कारणमव्यक्तं नित्यं सदसदात्मकम् ।}\\
\dev{तद्विसृष्टः स पुरुषो लोके ब्रह्मेति गीयते ।।}
\end{verse}
\dev{इत्यनेनाद्यपुरुषस्याव्यक्तब्रह्मकार्यस्यापि लौकिकब्रह्मत्वमुक्त्वा तस्यैव नारायणत्वमुक्तम्-}
\begin{verse}
\dev{आपो नारा इति प्रोक्ता आपो वै नरसूनवः ।}\\
\dev{ता यदस्यायनं पूर्वं तेन नारायणः स्मृतः ।।}
\end{verse}
\dev{इति । अतो}

\dev{नारायणादप्यतिरिक्तः परमेश्वर इति । नारायणादिशब्दानामुपाधिमात्रपरत्वे च लक्षणापत्तिरिति। अपि च सुर्यादिसाहचर्येण विभूतित्वस्मरणादपि ईश्वरस्य विष्ण्वादीनां भेदोऽवगम्यते । तथा मात्स्ये— }
\begin{verse}
\dev{पूजयेद् ब्रह्मविष्ण्वर्करुद्रविश्वात्मकं शिवम् ।}\\
\dev{व्रतोपवासनियमैः श्रद्धया च विमत्सर: ॥}\\
\dev{ब्रह्मा विष्णुश्च भगवान् मार्तण्डो वृषवाहनः ।}\\
\dev{इमा विभूतयः प्रोक्ताश्चराचरसमन्विताः ।। इति ।}
\end{verse}
\dev{नन्वेवं कथं गीतादिषु विष्ण्वादिदेवतैश्वर्यमेवोक्तं न तु परमेश्वरस्तस्यैश्वर्यं वेति चेत्}
\begin{verse}
\dev{एवं सततयुक्ता ये भक्तास्त्वां पुर्युपासते ।}\\
\dev{ये चाप्यक्षरमव्यक्तं तेषां के योगवित्तमाः ।।}
\end{verse}
\dev{इत्यादि प्रश्नप्रतिवचनाभ्यां विष्णुदेवाद् भेदेन परमेश्वरकथनात् । “अनादि मत्परं ब्रह्म न सत् तन्नासदुच्यते” इत्यादिना च परमेश्वरस्यैश्वर्यादिकथनादिति ।}

\dev{नन्वेवं कथमीश्वरो विष्णुरिति शिव इति चोच्यते नान्य इतीति चेत्, विष्ण्वादीनां स्वाभाविकब्रह्मात्मभावात्, साक्षादंशत्वान्मुख्यशक्तित्वाच्चेत्यवेहि  “तत् यो यो देवानां प्रत्यबुध्यत स एव तदभवदि’’ ति श्रुतेः;}
\begin{verse}
\dev{एकः शुद्धोऽक्षरो नित्यः सर्वव्यापी तथा पुमान् ।}\\
\dev{सोऽप्यंशः सर्वभूतस्य मैत्रेय परमात्मनः ।। इति,}
\end{verse}
\dev{ब्रह्मविष्णुशिवा ब्रह्म सन् प्रधाना ब्रद्दाशक्तयः । इति च स्मृतेश्च त्रयाणामपि च मध्ये विष्णुरेव ब्रह्मणो मुख्या शक्तिः—}
\begin{verse}
\dev{सर्वशक्तिमयो विष्णुः स्वरूपं ब्रह्मणः परम् ।}\\
\dev{स परः सर्वशक्तीनां ब्रह्मणः समनन्तरः ।।}
\end{verse}
\dev{इति पराशरवचनात् ।}

\dev{ननु त्रिषु मध्ये शिव एवास्य मुख्या शक्तिरित्यपि शैवपुराणेषूच्यत इति चेत् सत्यम्— विवक्षितगुणभेदेन कल्पभेदेन वा तस्य चोपपत्तिः । तदुक्तं कौर्मे—}
\begin{verse}
\dev{सात्विकेष्वथ कल्पेषु माहात्म्यमधिकं हरेः ।}\\
\dev{तामसेषु शिवस्योक्तं राजसेषु प्रजापतेः ।। इति दिक्}
\end{verse}
\dev{विष्णोर्ब्रह्मा ब्रह्मणश्च रुद्र इति सृष्टिक्रमश्च स्थूलदेहरूपेण पृथक् पृथगाविर्भाव एव बोध्यः, पुराणेषु तथा दर्शनादिति।}

Similarly statements in Manu and Yājñavakya starting with: “kham sanniveśayet kheṣvu” (MS.XII.120) and ending with: “prajane ca prajāpatim” (this is not quoted here by Bhikṣu but this is MS.XII.121) stating that in all bodies there are Brahmā, Viṣṇu and Rudra, also mention the other puruṣa in the form of the antaryāmin of Viṣṇu etc., as: “praśāsitāram sarveṣāmaṇīyāmsamaṇorapi rukmābham svapradhīgamyam vidyāttam puruṣam param” (MS.XII.122). Again Manu’s statement: “yattat kāraṇamayaktam…sa puruṣo loke brahmeti gīyate” (MS.I. 11) stating the essence of the worldly Brahman as the activity of the primal Puruṣa and the action of the undifferentiated Brahman, mentions that itself as the essence of Nārāyaṇa in the verse: “āpo nārā iti proktā…tea nārāyaṇaḥ smṛtaḥ” (MS.I.10). Thus it is clear that Parameśvara is different from Nārāyaṇa even. The words Nārāyaṇa etc., being subject only to upādhis there is the danger of (having) characteristics. Moreover due to association with the sun etc., recalling its splendour one learns that Īśvara is different from Viṣṇu etc. Thus the Matsya P. says: “pūjayet brahmaviṣṇvarkarudraviśātmakam śivam…imā vibhūtayaḥ proktāścarācarasamanvitaḥ”.\footnote{There is too much verbiage using mainly Purāṇic sources which again confirm Bhikṣu using Purāṇic sources to advance his arguments. This could be one reason why Bhikṣu did not gain a place amongst the respected Vedānta ācāryas such as Śaṅkara, Rāmānuja et al.}

\textbf{Ques:} If that is so then how is it that in the Gītā etc., only the supremacy (aiśvaryam) of Viṣṇu devatā etc., is mentioned and there is no mention of Parameśvara or his supremacy. Then the answer is: 

\textbf{Ans:} Through questions and answers in verses such as: “evam satatayuktā…teṣām ke yogavittamāḥ” (Gīta.12.1) there is mention of Parameśvara different from Viṣṇudevatā. So also through (such verses as): “anādi matparam brahma na sat tannasaducyate” (Gītā.13.12? check) there is mention of the supremacy of Parameśvara.

\textbf{Ques:} But then how is it that Īśvara is mentioned as Viṣṇu, as Śiva and not any other? 

\textbf{Ans:} It is because there is the state of Brahmātman naturally (svābhāvikabrahmātmabhāvāt) with regard to Viṣṇu etc., (they) also are  directly a part (of Īśvara) as also  they have the main power (of Īśvara) (sākṣādamśatvānmukhyaśaktitvācca). Thus śruti says: “tat yo yo devānām pratyabuddhyata sa eva tadabhavat”. “ekaḥ śuddho’kṣaro nityaḥ sarvyāpī tathā pumān, so’pyamśaḥ sarvabhūtasya maitreya paramātmanaḥ”.  Smṛtis also state that Brahmā, Viṣṇu and Śiva are Brahman’s important powers. And amongst the three Viṣṇu is alone the important power of Brahman.\footnote{This makes it amply clear that Bhikṣu was writing at the time of the ascendency of Vaiṣṇavism.} This is also learnt from the statement of Parāśara: “sarvaśaktimayo viṣṇuḥ svarūpam brahmaṇaḥ param, sa paraḥ sarvaśaktīnām brahmaṇaḥ samanantaraḥ”.

\textbf{Ques:} But it is stated in the Śaiva Purāṇas that amongst the three, Śiva alone is the main power (of Brahman). 

\textbf{Ans:} That is true. According to the guṇa-division desired or by the division of the kalpas (satya, treta, dvāpara and kali) that is logical. Thus the Kūrma P. states: “sāttvikeṣvatha kalpeṣu māhātmyamadhikam hareḥ, tāmaseṣu śivasyoktam rājaseṣu prajāpateḥ”; this is the general understanding. After Viṣṇu comes Brahmā, after Brahmā comes Rudra; it is to be understood that the order of creation in the form of gross bodies/substances arises separately as it is given in the Purāṇas.

\dev{ये तु ईश्वरस्यान्तर्याम्यतिरिक्तविधयाऽपि योगिनामिव लीलाशरीरमिच्छन्ति तेषां मते2 “न तस्य कार्यं करणं च विद्यते, अप्राणो ह्यमनाः शुभ्रो अक्षरात् परतः परः” इत्यादिश्रुतिविरोधः “रूपवदेव हि तत्प्रधानत्वादि” त्यागामिसूत्रविरोधश्च । कार्यं शरीरं करणं बुद्ध्यादि, बुद्ध्यादिना हि शरीरं प्रेर्यत इति। ब्रह्माण्डमधिकृत्य—}
\begin{verse}
\dev{“तस्मिन् कार्यं च करणं संसिद्धं परमेष्ठिनः ।”}
\end{verse}
\dev{इति स्मरणात् ।}

\dev{तथा—}
\begin{verse}
\dev{“अशरीरः शरीरेषु सर्वेषु निवसत्यसौ ।” इति,}\\
\dev{“देहद्वयस्थितो नित्यः सर्वदेहविवर्जितः ।”}
\end{verse}
\dev{इति च भारतादिष्वीश्वरस्य शरीरद्वयं प्रतिषिद्धम् ।}

\dev{तस्माद् विष्ण्वादिदेवानामेव स्थित्यादिलक्षणस्वाधिकारपालनार्थं मत्स्यादयो लीलावताराः । ते च परब्रह्मणः प्रकृष्टशक्तितया आवेशावतारतया वा परमेश्वरत्वेनोपास्या इति । तथा च श्रुतिः “यन्मनसा न मनुते येनाहुर्मनो मतं तदेव ब्रह्म त्वं विद्धि नेदं यदिदमुपासते । यस्यामतं तस्य मतं मतं यस्य न वेद सः, अविज्ञातं विज्ञानताम् विज्ञातमविजानताम्” इति ।}

\dev{ननु}
\begin{verse}
\dev{“नाहं प्रकाशः सर्वस्य योगमायासमावृतः ।}\\
\dev{मूढोऽयं नाभिजानाति लोके मामजमव्ययम् ।}
\end{verse}
\dev{इति विष्णुदेवतावाक्यात् तस्याप्यमतत्वादिकमस्तीति चेत् विष्णुदेवतायामपि महत्तत्त्वरूपोपाधिनेव (नैव) सर्वव्यवहारात्, तस्य च पराप्रत्यक्षत्वात् । तथापि “यन्मनसा न मनुते” इत्यस्य विष्णुदेवतायामसंभवः— }
\begin{verse}
\dev{मनसैव जगत्सृष्टिं संहारं च करोति यः ।}\\
\dev{तस्यारिपक्षक्षपणे कियानुद्यमविस्तरः ।।}
\end{verse}
\dev{इति विष्णुपुराणादौ विष्णुदेवताया मनोऽवगमादिति ।}

\dev{ननु विष्णुदेवतायाः परमेश्वराद् भेदे सति—}
\begin{verse}
\dev{भूमिरापोऽनलो वायुः खं मनो बुद्धिरेव च ।}\\
\dev{अहंकार इतीयं में भिन्न प्रकृतिरष्टधा ॥}\\
\dev{अपरेयमितस्त्वन्यां प्रकृतिं विद्धि मे पराम् ।}\\
\dev{जीवभूतां महाबाहो ययेदं धार्यते जगत् ॥}
\end{verse}
\dev{इति विष्णुदेवतावाक्यं नोपपद्यत, परब्रह्मण एव प्रकृतिपुरुषाधिष्ठातृत्वस्वीकारादिति, मैवम् अत्र वाक्ये भूम्यादिरूपविकाराणामेव प्रकृतित्वस्मरणात् । अत एव जीवशब्दस्य व्यष्टिजीवपरत्वादिति ।  अथवा तादृशानि सर्वाण्येव वाक्यानि परब्रह्मात्मताभिप्रायेणेति ।}

\dev{नन्वेवं परब्रह्मणः सकाशाद् भेदे सति कथं विष्णुशिवप्रकरणस्थानि सृष्ट्यादिवाक्यानि ब्रह्मनिरूपणे साधकतयोपन्यस्यन्ते विचारकैरिति चेत्, तयोरीश्वरकोटितया । परमेश्वरेण सह व्यवहारसाम्यादिति गृहाण ।   अतएव—त्रयाणामेक\-भावानां यो वै भिदामि” त्यादिस्मृतिभिस्त्रयाणामेव देवानां स्वाभाविकब्रह्मात्मभावसाम्यात्, सर्वोपनिषत्स्वत एव ब्रह्मतयोपदिश्यन्ते उपासनार्थमिति  । परब्रह्मणश्च स्वतो, नामरूपाद्यभावान् मुख्यविकारयोर्हरिहरयोर्नामरूपाभ्यामेव शास्त्रेषु प्रायशो व्यपदेशः अतएव वैष्णवाः शैवाश्चाविवेकिनो विष्ण्वाद्यतिरिक्तं परमेश्वरमविद्वांसो ब्रह्ममीमांसाशास्त्रं विष्ण्वादिपरतया व्याचक्षत इति मन्तव्यम् । अनामरूपत्वं च परब्रह्मणो विष्णुपुराणादावुक्तम्— }
\begin{verse}
\dev{न सन्ति यत्र सर्वेशे नामजात्यादिकल्पनाः ।}\\
\dev{सत्तामात्रात्मके ज्ञेये ज्ञानात्मन्यात्मनः परे ।।}\\
\dev{नामरूपे न यस्यैको योऽस्तित्वेनोपलभ्यते ।}
\end{verse}
\dev{इत्यादिनेति । आत्मनः परे जीवादतिरिक्त इत्यर्थः ।}
\begin{verse}
\dev{पञ्चसूत्र्यां समासेन पूर्वाचार्यैः प्रवर्तितः ।}\\
\dev{वेदान्तनगरीमार्गो विज्ञानार्केण दर्शितः ।।}
\end{verse}

Those who desire a body of Īśvara created in sport like the yogīs through a different mode as the antaryāmin, in their view it will be going against śruti such as: “na tasya kāryam karaṇam ca vidyate”, and it is against the forthcoming sūtra: “rūpavadeva hi tatpradhānatvāt” (BS. II.2.15).\footnote{BS. II.2.15 reads as follows: rūpādimattvācca viparyayo darśanāt”. Either Bhikṣu had this reading In front of him or he slightly modified it to convey the same sense; one remembers however, that this was stated for refuting the Viśeṣika theory of atoms by Śaṅkaracarya.} The effect is the body and the instruments are the intellect etc., since it is through the intellect that the body is goaded towards action. One recalls the Brahmāṇḍa P. statement: “tasmin kāryam ca karaṇam samsiddham parameṣṭhinaḥ”.

So also: “aśarīraḥ sarīreṣu sarveṣu nivasatyasau” and: “dehadvayasthito nityaḥ sarvadehavivarjitaḥ” in texts like the Bhārata (Mahābhārata) there is rejection of two bodies of Īśvara.  Therefore there are descents/\-incarnations such as fish etc., in sport, only for the sake of protecting their authority characterized by sustenance etc., of Viṣṇu and others that. They as the supreme power of parabrahman or as a descent through the entrance (of parabrahman) need to be worshipped as having the essence of Parameśvara. So also the śruti statement: “yanmanasā na manute…tadeva brahma tvamviddhi nedam yadidamupā\-sate” (Kena.Up.I.6); “yasyāmatam tasya matam…avijñātam vijānatām vijñātamavijānatām” (ibid.II.3).

\textbf{Ques:} But then by Viṣṇu’s words: “nāham prakāśaḥ sarvasya…loke māmajamavyayam” (Gītā 7.25) if one says through the words of Viṣṇu, that he also has lack of knowledge etc., then the answer is: 

\textbf{Ans:} even In Viṣṇudevatā all activity is through the limitation of the principle of mahat; for him also the supreme is not directly perceived.  

\textbf{Ques:} Still the statement: “yatmanasā na manute” is not possible with regard to Viṣṇudevatā. 

\textbf{Ans:} One learns about the mind of Viṣṇudevatā through the statement: “manasaiva jagatsṛṣṭim saṁhāram ca…kiyānudyamavistaraḥ” in the Viṣṇu P. etc. 

\textbf{Ques:} Since there is a difference of Viṣṇudevatā from Parameśvara the words by Viṣṇudevatā: “bhūmirāpo’nalo vāyuḥ kham mano buddhireva ca…jīvabhūtām mahābāho yayedam dhāryate jagat”does not fit as it is accepted that it is only Parabrahman who is the supporter of prakṛti and puruṣa.

\textbf{Ans:} it is not so; in the above sentence it is only the changes in the form of bhūmi etc., that is recalled as being the essence of prakṛti (bhūmyādirūpavikārāṇāmeva prakṛtitvasmaraṇāt). That is the reason that the word ‘jīva’ is meant in the sense of the individual jīva. Or all such sentences are used with the intention of being the essence of paramātman (athavā tādṛśāni sarvāṇyeva vākyāni parabrahmātmatābhiprāyeṇeti).

\textbf{Ques:} If then, when there is a difference from parabrahman how is it that  wise persons explain such statements as creation etc, located in chapters dealing with Viṣṇu, Śiva etc., as fulfilling the task of ascertainment of Brahman? Then the answer is:

\textbf{Ans:} one needs to accept them as belonging to the group of Īśvaras as there is a similarity in activity with Parameśvara. That is the reason (there are) such smṛti statements like: “trayāṇāmekabhāvānām yo vai bhidāmi” due to the natural similarity of Brahman and ātman; that is the reason that in all Upaniṣads they are declared as being Brahman in themselves for the sake of meditation/worship. Since parabrahman by itself is without name and form there is representation in general, of the main transformations Hari and Hara, through name and form. One should understand the reason why those unwise people both  Vaiṣṇavas and Śaivas, explain Parameśvara to be different from Viṣṇu (and) the unlearned explain the Brahmamīmāṁsāśāstra (Upaniṣads) as dealing with Viṣṇu as the highest being. Para-Brahman being without name and form is mentioned in the Viṣṇu P. in verses like: “na santi yatra sarveśe nāmajātyādikalpanāh…nāmarūpe na yasyaiko yo’stitvenopalabhyate”. “ātmanaḥ pare” (in the above quote) means which is different from jīva.

The sun in the form of Vijñāna (stands for Bhikṣu himself) has shown the path to the city of Vedānta which has been stated briefly by our earlier ācāryas in the (first) five sūtras (of the BS).

Even though Bhikṣu’s commentary on the five sūtras ends here his ongoing debate with the advaitin regarding the status of pradhāna continues in the fourth section (pāda) of the first adhyāya (chapter).  That has been added as an appendix to this work.
