\chapter*{Appendix}\label{appendix}

This section of the BS (I.4.1-28) has been added as an appendix just to demonstrate how commentators explain the sūtras to fit their own standpoints. When it comes to one’s own preferences in fundamentals, be it advaita, Sāṃkhya- Yoga (SY), Nyāya or otherwise there is no compromise amongst the different commentators each one staunchly defending the philosophical school which he prefers and which he identifies himself with.\footnote{Vācaspati Miśra thought was an exceptionally gifted scholar who wrote commentaries objectively on Sāmkhya, Yoga, Nyāya,  etc, and is therefore known as \textit{sarva-tantra-svatantra} (one  who was able to identify with all  the philosophical schools he wrote commentaries on ). } This needs to be emphasized in the context of considering Bhikṣu a syncretist which I have discussed in the Introduction to this work. 

The debate on pradhāna versus Brahman as the cause for the origin of the world is an ongoing one in the BS. Even though it seemed that there was a closure to that topic as it was dealt with comprehensively in  Śaṅkara’s commentary on BS.I.1.5 as well as in his commentaries on BS.II.2.1-10,  Bādarāyaṇa opens the door for the same discussion of what is it that causes the world  by his statement that according to some it is based on inference (ānumānikamapyekeṣāmiti cenna śarīrararūpakavinyastagṛhīterdarśayati ca BS. I.4.1). Commentaries on this sūtra by Śaṅkara the advaitin and Bhikṣu the avibhāgādvaitin hold completely opposite views on what the cause for the world is. They are arraigned on opposite sides arguing against each other’s position and that  leads to a lengthy discussion with the advaitin Śaṅkara arguing for the ultimate Brahman as the cause and Bhikṣu predictably arguing for pradhāna being the one to be inferred as the cause of the world. This discussion thus continues in the 4$^{\text{th}}$ pāda (section) of the first chapter (adhyāya) sūtras I.4.1-28 of the BS. both in Śaṅkara’s BSBh (commentary of the BS) and in Bhikṣu’s discussion on pradhāna, in his commentary Vijñānāmṛtabhāṣya on these BS. I.4.1-28 sūtras. 

However, if read by themselves the commentary on these sūtras will not be very clear as it is a tit for tat between Śaṅkara and Bhikṣu that is going on here and sometimes descending to vitaṇḍa-vāda, each trying to score points for his own view.. Therefore one needs to be aware of the context in which each of these commentators is stating his counter-opinion. Imagining this too to be in the form of a śāstra-carcā (discussion on what is contained in the śāstra texts) I have tried to present this back and forth as if it is taking place between a  putative advaitin and one of the SY persuasion, as far as I could make out. I try to point this out in the footnotes given.

As mentioned in the introduction to this work as my main aim is to explore the many ways in which Bhikṣu maintains his support for the SY principle of prakṛti/pradhāna being the originator of the world he continues that discussion in this section as well in a more detailed form, though some of the points raised are, as is to be expected, a repetition of what has already been mentioned in the commentaries on the first five sūtras of the BS (VijBh on BS.I.1.1-5). But knowing Bhikṣu’s style he is not the one to shy away from repeating any number of times the same arguments to further his cause of pradhāna playing a role in the origin of the world.

Amongst the 28 sūtras in this section some are not directly related to the topic at hand. I have therefore not included     the commentaries on sūtras I.4.18-22.\footnote{The last para of sūtra I.4.22 however has been included as it is relevant to the discussion.} The reader will know which ones are excluded by the numbering of the sūtras themselves. Being a rejoinder to Śankara’s commentary on each of these sūtras it will be useful if the reader can connect what Śankara has stated under each of these sūtras to what Bhikṣu says as a rejoinder to that in order to get a wholesome picture.  Since the advaitin’s contention is that pradhāna being insentient cannot have a role in the origin of the universe Bhikṣu argues otherwise, quoting mainly from the Śvetāśvatara. Up to support his contention of pradhāna having a role to play in the origin of the world, as he is a staunch advocate of the SY philosophy (also see my paper  “Vijñānabhiksu: The Sāmkhya-Yoga-Vedāntācārya in ICPR July-Sept. 2010, Vol. 27).  The footnotes are an attempt to help the reader to connect what is said in the text to be either the advaita view of Śaṅkara or the avibhāgādvaita view  of   Bhikṣu.  With that brief explanation we start the appendix with the first sūtra in BS fourth pāda of the first chapter (BS.I.4.1).

\dev{अथ प्रथमाध्यायस्य चतुर्थः पादः}

\dev{तदेवं पादत्रयेण जगज्जन्मादिकारणस्य ब्रह्मण: शास्त्रयोनित्वं प्रतिपादितम्, ब्रह्मपरत्वेन सन्दिह्यमानानां वाक्यानां ब्रह्मपरत्वावधारणेन तत्र तत्रोक्ता ब्रह्मगुणाश्चानन्दमयत्वान्तर्यामित्वादय उपासनोपयोगिनोऽवधृता: , तथा “तत्तु समन्वयादि” त्यनेनोक्तो जगद्ब्रह्मणोरन्योन्यसमन्वयश्चाकाशस्तल्लिङ्गादित्याद्यधिकरणै: प्रसाधित, प्रसङ्गाच्च ब्रह्मविद्यादिकार्य्यादयोऽपि विचारिता: । इदानीं ब्रह्मणो जगज्जन्मादिकारणत्वे श्रुतिविरोधस्य परिहारमुखेन जगज्जन्मादि कारणत्वनिर्वाहिकां ब्रह्मण: शक्तिं प्रधानादिमवधारयति चतुर्थपादेन ----}

\textbf{Now (begins) the fourth pāda of the first adhyāya}

In this manner it was demonstrated in the previous three sections\break (pādatrayeṇa) that Brahman who is the source of the birth etc., of the world is the source of śāstra.  Wherever there were utterances that expressed doubt regarding statements inclined towards Brahman, those qualities of Brahman such as ‘ānandamaya’, ‘antaryāmitva’ (indwelling quality) were all understood as fit for devotion towards Brahman, paying attention to their  inclination towards Brahman (brahmaparatvena sandihyamānānām vākyānām brahmaparatvāvadhāraṇena tatra\break tatroktā brahmaguṇāścānandamayatvādaya upāsanopayogino’\-vadhṛ\-tāḥ). Thus through the (different) sections (adhikaraṇaiḥ) have been explained the coexistence of the world and Brahman, (as also) ākāśa because of its mark etc; also based on the context topics such as Brahman, knowledge, the competent person to pursue Vedānta etc., have also been discussed. Now with the idea of removing the opposition with śruti regarding Brahman being the cause of the origin of the world etc., pradhāna, the power of Brahman, which accomplishes the task of being the cause for the origin etc., of the world is being attended to in the fourth pāda (section of the first adhyāya).\footnote{Bhikṣu states his thesis right at the beginning of the discussion, that pradhāna as the power of Brahman has a role to play in the origin of the universe.}

\textbf{I.4.1}
 
\begin{verse}
\textbf{ānumānikamaypyekeṣāmiti cenna} \\
\textbf{śarīrarūpakavinyastagṛhīterdarśayati}
\end{verse}

\dev{ननु ब्रह्मणो जगत्कारणत्वादिकं कथं श्रुत्यर्थ: स्यात् यत एकेषां शाखिनां श्वेताश्वतरादीनामानुमानिकं प्रधानमपि जगज्जन्मादिकारणत्वेन श्रूयते “अजामेकां लोहितशुक्लकृष्णां बह्वी: प्रजा:सृजमानां सरूपा” इत्यादिषु। नन्वेवमुभयोरेव कपालद्वयवद् विश्वोपादानत्वमस्तीति, मैवम्, स्वातन्त्र्येणोभयो: कारणत्वे सति ब्रह्मणो विकारित्वापत्तेरिति पूर्वपक्षबलार्थ: । तमिमं पूर्वपक्षं शक्तिशक्तिमद्भावेनोभयोरेव कारणत्वेन समाधत्ते – न शरीररूपकविन्यस्तगृहीतेरिति । वक्ष्यमाणवाक्ये शरीरेण रूपकेण दृष्टान्तेन विन्यस्तमुपन्यस्तं यन्मायाख्यं प्रधानं तस्यैवाजावाक्ये स्रष्टृत्वेन ग्रहणात्, न तु स्वतन्त्रकारणस्य कस्यचिदित्यर्थ: । एतदुक्तं भवति----- नाजावाक्ये स्वातन्त्र्येण प्रधानस्य कारणत्वं प्रोक्तं येन विरोध: स्यात्, किन्तु यथा अजाख्यच्छागीयशरीरस्य तज्जीवशक्तिविधया सजातीयबहुप्रजाकारणत्वं, तथैव प्रधानस्यापि ब्रह्मशक्तिविधया प्रधानादीनामुपादानत्वादिकं बोध्यते, तद्विना निर्विकारस्य ब्रह्मण उपादानत्वानुपपत्ते: । ब्रह्मवाक्यैश्च शरीरिवच्छक्तिमद्विधया शक्तिद्वारा ब्रह्मणोऽधिष्ठानकारणत्वं प्रतिपाद्यत इति श्रुतीनामविरोध इति ।}

The SY follower raises the question of how the meaning of śruti can be that Brahman is the cause etc., of the world since in some recensions/collections (śākhinām) such as the Śvetāśvatara etc., (ekeṣām śākhinām śvetāśvatarādīnām) one hears of pradhāna also as being the cause of the world etc., in such statements as “ajāmekām lohitaśuklakṛṣṇām baḥvīhiḥ prajāḥsṛjamānām sarūpāḥ”.\footnote{Bhikṣu’s contention is that Brahman cannot be stated to be unilaterally mentioned by all śruti statements as being the cause of the world as there are other statements such as in the Śvet.Up which say otherwise.} Then (as a compromise) let both  (Brahman and Pradhāna) like the two parts of a pot be the material cause of the universe. Then the answer is:

 It cannot be so (according to the advaitin); if both of them independently have causal efficiency then there will be the inconsistency of Brahman having change; that is the meaning of the first part of the prima facie view (pūrvapakṣadalārthaḥ).\footnote{This is the answer of the advaitin to that }
 
(As a compromise) that prima facie view is being resolved by stating that both have causal efficiency through the state of being the possessor of śakti and being śakti (respectively) (tamimam pūrvapakṣam śaktiśaktimadbhāvenobhayoreva kāraṇaatvena samādhatte)  through the statement “na śarīrarūpakavinyastagṛhīteḥ”.  In this (śarīrarūpakavin\-yastagṛhīteḥ) statement, by the use of the example of the metaphor of the body, “vinyastam”=pradhāna known as māyā is explained and that is itself to be accepted/grasped as having creatorship (sraṣtṛtvena) in the sentence “ajā” (probably referring to the verse Śvet.Up. I.9)\footnote{This is the whole verse: “ajām ekām lohita-śukla-kṛṣṇām bahvīḥ prajāḥ sṛjamānām sarūpāḥ ajo hi eko juṣamāṇo’nuśete jahātyenām bhuktabhogām ajo’nyaḥ (“Śvet.Up. 4.5).  Using this verse as example Bhikṣu’s reply to the advaitin is that Brahman as the śaktimān having the śakti of pradhāna is the cause for the origin of the world.} and not in the sense of an independent causal agent. 

\newpage

In the sentence “ajā” etc., it has not been stated that pradhāna independently is the cause as, in that case, there will be a contradiction. But when the unborn body known in the form of a goat ( (ajākhyācchāgīyaśarīrasya) is the cause of many individuals of the same class (tajjīvaśaktividhayā sajātīyabahuprajaṇatvam) through the power of the jīva of that (body) so also pradhāna is mentioned as being the cause through (being) the power of Brahman according to the simile which will be mentioned, as it is desired that there is a similarity in the metaphor and the object of  the metaphor. Thus through sentences ajā etc., (ajādi vākyaiḥ) like the body (being used by jīva) through the path of being the power (of Brahman) pradhāna etc., is explained as being the material cause etc (of the world). Without this explanation Brahman which is without change cannot be logically the material cause (of the world). And through sentences denoting Brahman, by possessing the power (of pradhāna) similar to the body, through that power is explained Brahman being the underlying supporting cause; thus there is no conflict between the śruti vākyas (iti śrutīnāmavirodhaḥ).

\dev{ननु भवेदेवं यद्यजानिरूपितप्रधानशक्तिकत्वेनात्र ब्रह्मोक्तं स्यात्, न त्वेवम्, अजाशरीरित्वेन ब्रह्मण: प्रतिपाद्यत्वे अन्यभोगत्वहेयत्ववचनानौचित्यात् । न च “अजो ह्येको जुषमाणोऽनुशेते जहात्येनां भुक्तभोगामजोऽन्य” इत्युत्तरार्धे अजाधिष्ठातृ ब्रह्म कथितमिति वाच्यम्, उत्तरार्धे `ज्ञानजीवयोरेव प्रतिपादनात् न तु ब्रह्मण:, “जहात्येनां भुक्तभोगामजोऽन्य” इति भोगत्यागस्य जीवलिङ्गत्वात् । तत्कथमजाया ब्रह्मशरीरदृष्टान्तताऽवधृतेत्याशङ्कायामाह---दर्शयति चेति । चो हेतौ, दर्शयति ह्यर्थाद् ब्रह्मशरीरतुल्यत्वं प्रधानस्य श्रुति: ------}

\dev{“यस्तन्तुनाभ इव तन्तुभि: प्रधानजै: स्वभावतो देव- एक: समावृणोति स नो दधातु ब्रह्माव्ययम् ।  अस्मान्मायी सृजते विश्वमेतत् तस्मिंश्चान्यो मायया सन्निरुद्ध: । एतज्ज्ञेयं नित्यमेवात्मसंस्थं नात: परं वेदितव्यं हि किञ्चित् । भोक्ता भोग्यं प्रेरितारं च मत्वा सर्वं प्रोक्तं त्रिविधं ब्रह्ममेतत् । मायां तु प्रकृतिं विद्यान्मायिनन्तु महेश्वरम्” ।}
			
\dev{इत्यादि  वाक्यशेषैरित्यर्थ: । अत्र तन्तुनाभदृष्टान्तादिभिस्तन्तुनाभशरीरतुल्यत्वं प्रधानस्य लब्धमिति ।}

\dev{नचैवं प्रधानस्यानुमानिकत्वं कथमुच्यते श्रुत्यामपि तत्कारणतासिद्धेरिति वाच्यम्, स्मृतिषु प्राधान्येन विशिष्य च प्रतिपादनत: स्मार्तानुमानिकादिवचनाद् ब्रह्मण औपनिषदत्ववचनवदिति ॥}

\dev{इदानीं तस्य प्रधानाख्यशक्ते: स्वरूपं दर्शयन्नपि शरीरवदेषु परमात्मन: प्रधानं न स्वतन्त्रमित्याह -------}

SY: Let it be that Brahman is here mentioned as having the pradhāna-śakti  pictured as a ‘she’-goat.

Advaitin: that is not so; then in this way if Brahman is depicted with the body of a she-goat then it will be improper to mention the experience and non-experience of any other form.  Nor can it be said that the meaning of the latter half (of the same verse: “ajo hyeko juṣamāṇo’nuśete jahātyenām bhuktabhogāmajo’nyaḥ” (Śvet.Up. 4.5)\footnote{The meaning of this line is: “the one unborn (male) lies there very happy; another unborn one gives her up after  having had his enjoyment”.} has to be understood as that the unborn Brahman-support is mentioned (ityuttarārdhe ajādhiṣṭhātṛ brahma kathitamiti vācyam). To that the answer is:

SY: in the latter half only the knowing (one) and the jīva have been mentioned and not Brahman as in the sentence “jahātyenām bhuktabhogāmajo’nya” the giving up of experience has the mark of jīva.\footnote{Only jīva can experience through the limitation of the antaḥkaraṇa, thus the liṅga indicates jīva.}

In answer to the doubt (of the advaitin) as to how one has decided about the example of the unborn body of Brahman, he says through the meaning of such śruti utterances as:  “ darśayati ca” (in the sūtra I.4.1). “ca” is used in the sense of a reason, “darśayati” means the equivalence of the body of Brahman will be shown by the śruti: “yastantunābha iva tantubhiḥ pradhānajaiḥ svabhāvato deva ekaḥ smāvṛṇot” (Bhavam.2.47)\footnote{Bhikṣu has “samāvṛṇoti” instead of “samāvṛṇot”. The second line “sa no dadhātu brahmāvyayam” is not traced and does not occur as a continuation to “samāvṛṇoti”.}; “sa no dadhātu brahmāvyayam” (not traced); “ asmānmāyī sṛjate viśvam...sanniriddhaḥ” (Śvet.Up.4.9); “etajñeyam nityamevātmasaṃstham nātaḥ param veditavyam hi kiñcit” (not traced); “bhoktā bhogyam preritāram ca matvā sarvam proktam trividham brahmametat” (Śvet.Up. 1.12); “māyām tu prakṛtim vidyānmāyinam tu maheśvaram” (ibid.4.10).  Through such examples as the spider and the web (from the spider’s navel) the similarity of the body of pradhāna (with that of the body of the spider) with the thread from the navel (tatra tantunābhaśarīratulyatvam pradhānasya labdhamiti) is obtained.

Advatin: However how can you mention the inference of pradhāna in this manner, since even śruti has not established its causal status (tatkāraṇatāsiddheḥ).

SY: Because it has been mentioned both in a primary and in a qualified sense in the smṛtis; the statements of the smṛtis and inference etc., (regarding pradhāna is like Brahman being mentioned in the Upaniṣads) .\footnote{It seems that Bhikṣu is equating the smṛtis and inference based on them to  Upaniṣadic utterances}

\newpage

Now even while describing the true nature of that power known as pradhāna (tasya pradhānākhyaśakteḥ svarūpam darśayan) (he says), in these (quotes)  just like the body the pradhāna of paramātman is not independent (śarīravadeva paramātmanaḥ pradhānam na svatantram). 

\textbf{\dev{सूक्ष्मन्तु तदर्हत्वात् ॥२॥}}

\dev{तदानुमानिकं प्रधानं जडप्रपञ्चस्य सूक्ष्मं तु सूक्ष्मतामात्रविशेषवज्जडमेवाभ्युपमन्तव्यम्, कुत:? अर्हत्वात् योग्यत्वात्, कार्यस्वरूपस्यैव कारणस्यौचित्यादित्यर्थ: । सृजमानां सरूपा इति श्रुतेश्चेति भाव: । तथा च स्थूलप्रपञ्चस्य देहादेश्चेतनाधिष्ठानेनैव कार्यकारित्वदर्शनात् कारणस्य सूक्ष्मस्यापि जडत्वाद्यविशेषेण चेतनाधिष्ठितत्वमनुमीयते न स्वातन्त्र्यं परेच्छाननुविधायित्वरूपमित्यर्थ: ।}

\dev{अथवेदं सूत्रमेवं व्याख्येयम्- ननु प्रधानाङ्गीकारे प्रलयकालीनस्य ब्रह्मचिन्मात्राद्वैतस्य क्षतिरित्याशङ्कामपाकरोति “सूक्ष्मं तु तदर्हत्वात्”। तु शब्द: शङ्काभिनिवृत्त्यर्थ:। तत् प्रधानं   समुद्रविलीनसैन्धववत् सूक्ष्ममर्हत्वात्, अप्रत्यक्षत्वाद्युपपत्तये तथौचित्यादित्यर्थ: । तथा च विलीनावस्थसैन्धवेन समुद्रस्येव साम्यावस्थारूपेण प्रधानादिनाऽपि ब्रह्मणो न द्वैतं किन्त्वैक्यमेव समुद्रसैन्धवयोरिवेति भाव: । व्युत्पादितं चेदं विस्तरतस्तत्तु समन्वयाधिकरणे विस्तारयिष्यते च तदनन्यत्वाधिकरण इति ॥}

\dev{ननु प्रलयाद्वैतानुरोधेन प्रधानापलाप एव कथं न क्रियते मुख्याद्वैतपरिग्रहस्यैवौचित्यादित्याशङ्कायामाह ---}

\textbf{BS.1.4.2}

\textbf{sūkṣmantu tadarhatvāt}

Advaitin: That inferred pradhāna is to be understood as “sūkṣmam tu”=   having only the quality of the essence of subtlety of the world (sūkṣmatāmātraviśeṣvajjaḍamevābhyupagantavyam); why is that so? “arhatvāt”= because it has that capacity; it is proper that the cause has the true form of the effect (kāryasvarūpasyaiva); it is also the idea of the śruti statement “sṛjamānām sarūpāḥ” (Śvet.Up.4.5). Thus like in the gross world actions are seen to be accomplished by the bodies only through the support of consciousness, so also of the subtle cause (i.e. prakṛti) irrespective of the quality of being insentient etc., one infers that it has the support of consciousness; it means that one infers that independence (of prakṛti)  is of the nature of being not non-conforming to another’s desire (i.e. conforming to another’s desire in this case Brahman’s desire) (kāraṇasya sūkṣmasyāpi jaḍatvādyaviśeṣeṇa cetanādhiṣṭhitatvamnumīyate na svātantryam parecchānanuvidhāyitvarūpamityarthaḥ)\footnote{This is the fulcrum of Bhikṣu’s argument i.e. prakṛti as the śakti of Brahman is the cause for the origin of the world. Thus he meets the insentient nature of prakṛti by stating that through the consciousness of Brahman the insentient prakṛti as Brahman’s śakti is the cause of the origin of the world.}

SY: this sūtra needs to be interpreted as follows:

If one accepts pradhāna (as the cause of the universe) then there will be a disruption (to the belief that) Brahman is the sole /singular/non-dual consciousness during the time of dissolution. This doubt is removed by the sūtra: 

“sūkṣmantu tadarhatvāt”. “tu”= the word “tu” (in the sūtra) is for the sake of removing the doubt. Since it is not perceived, that pradhāna, deserves to be treated as subtle like salt dissolved in the sea ``sūkṣmam arhatvāt"; it is appropriate that it be so considered due to the logic of being imperceptible  (apratyakṣatvādyupapattaye tathaucityādityarthaḥ). In a similar fashion, like the sea in the state of absorption being one with the salt, pradhāna also in a state of equilibrium is not different (na dvaitam) from Brahman, but it is one with (Brahman) like the sea and salt. This has been demonstrated in detail in the section on “tattu samanvayāt” (BS.I.1.4) and it will be explained in detail in the section dealing with “tadananyatvamārambhaṇaśabdādibhyaḥ” (BS.II.1.14).\footnote{Using the Br.up statement - ``tam vittyākarmaṇā...pūrvaprajňāta", Bhikṣu argues that like rivers non separate in the ocean, the jīva is neither absorbed totally in Brahman during pralaya nor is it totally separate. This implies that even in dissolution the jīva exists separate from Brahman according to Bhikṣu.}

Advaitin: In accordance with the rule that (only) the non-duality Brahman exits in dissolution why cannot pradhāna (also) be removed/con\-cealed (during pralaya) since it is appropriate to uphold the main theory of advaita? Then the answer is (the next sūtra): “tadadhīnatvādarthavat” (I.4.3).

\textbf{\dev{तदधीनत्वादर्थवत्॥३॥}}

\dev{सृष्ट्यादीनां प्रधानाधीनत्वात् प्रधानमर्थवत् प्रयोजनवदित्यर्थ: । न हि शक्तिं विना केवलादेकस्मादसङ्गचिन्मात्राद् ब्रह्मणोऽविकारिणो वै(वि) चित्रासंख्यविश्वनिर्माणं कादाचित्कं संभवति । अत: प्रकृतिपुरुषादिरूपा ब्रह्मशक्तिरिष्टेति । अत् एव विष्णुपुराणे ----}

\begin{verse}
\dev{निर्गुणस्याप्रमेयस्य शुद्धस्याप्यमलात्मन: ।}\\
\dev{कथं सर्गादिकर्तृत्वं ब्रह्मणो ह्युपपद्यते ॥}
\end{verse}

\dev{इत्याशङ्कायां प्रत्युत्तरम् -----}

\begin{verse}
\dev{शक्तय: सर्वभावानामचिन्त्यज्ञानगोचरा: ।}\\
\dev{यतोऽतो ब्रह्मणस्तास्तु सर्गाद्या भावशक्तय: ॥ इति ।}
\end{verse}

\dev{अस्यायमर्थ: --- यत: सर्वभावानां शक्तय: सर्वपदार्थेषु प्रकृतिपुरुषादिषु स्वस्वकार्यजननसामर्थ्यानि अचिन्त्यानि योगिप्रत्यक्षाणि च यथायोग्यं तिष्ठन्ति, अतस्ता: पूर्वोक्ता: सर्गाद्या: ब्रह्मणो भावशक्तय: ब्रह्मोपकरणभूताखिलपदार्थानां सामर्थ्यानि न तु केवल ब्रह्मण इति ।}

\textbf{BS.I.4.3}

\textbf{Tadadhīnatvādarthavat}

SY:The meaning (of this sūtra) is: Since creation/manifestation is dependent on pradhāna, “arthavat”= it has a purpose/meaning. It is not possible that this variegated innumerable universe can be created from the singular unchanging Brahman characterized by consciousness alone without power at any time. Therefore the powers of Brahman in the form of puruṣa and prakṛti are desired.

That is the reason why the Viṣ.P has the answer: “śaktayaḥ sarvabhāvānāmacintyajñānagocarāḥ, yato’to brahmaṇastāstu sargādyā bhāvaśaktayaḥ” for the question: “nirguṇasyāprameyasya śuddhasyāpyamalātmanaḥ, katham sargādkartṛtvam brahmaṇo hyupapadyate”\break (Viś.P. 2.3.2-3 cited in Trip. p.130.fn 1). The meaning of that is, since there is the capacity in all substances (and) in prakṛti and puruṣas (sarvabhāvānām śaktayaḥ sarvapadārtheṣu prakṛtipuruṣādiṣu), to generate their respective tasks as well as the unheard of direct perception of yogīs (acintyāni yogipratyakṣāṇi),\footnote{Bhikṣu brings in the yogī’s extraordinary power of subtle vision even when there is no occasion for it in this context because of his special affiliation to yoga.} therefore the afore-mentioned existential powers of Brahman at the start of the evolution/manifesta\-tion (of the world etc) are what help (to generate) the (inherent) capacity of all substances and not just Brahman alone (that does it).

\dev{ननु ब्रह्माद्वैतवाक्यानामाञ्जस्येनोपपत्यर्थं भवतु प्रपञ्चो विवर्तस्याविद्यानिमित्तताभ्युपगमेनाविद्ययाद्वैतहानिता\-दस्थ्यात् । न चाविद्यापि प्रलये नासीत् किं तु केवलादेव ब्रह्मणो विवर्त इत्येवाभ्युपगन्तव्यमिति वाच्यम्, एवं चेन्मुक्तस्यापि पुनर्विवर्तरूपसंसारप्रसङ्गात् चिन्मात्रस्य ब्रह्मणो नित्यत्वात् । अथ बन्धमोक्षादिकमपि कल्पितमेव, परमार्थतस्तदपि नास्तीत्येष श्रुतिमहावाक्यार्थो यथाकथञ्चिद् यदि व्यवस्थाप्यते, तदा बन्धमोक्षादिकं किमपि नास्तीति वेदान्तैरुच्यत इत्यापातज्ञानादेव मोक्षार्थं ब्रह्मसाक्षात्कारायाज्ञानां श्रवणादिकं याज्ञवल्क्यादीनां विद्वत्संन्यासादिकं च न स्यात्, फलनिश्चयाभावात् । बहुवित्तव्ययायासाध्ये हि कर्मणि फलनिश्चयादेवं लोकानां प्रवृत्तिरुत्पद्यते । न च भवन्मते महति संन्यासादिकर्मणि मोक्षाख्यफलनिश्चय: संभवति । मोक्षावधारणं मोक्षशास्त्राद् वेदान्तादेव भवति नेतरस्मात् । स च वेदान्तो न मोक्षादिपरो भवतोच्यते, मोक्षाद्यभावपरत्वाभ्युपगमादिति ।}

Advaitin: Let it be that in order to make sense of all the advaita statements, if one accepts the world as an appearance due to avidyā as cause, then the harm done to advaita by avidyā will (have no leg) to stand on. It is not as if avidyā does not exist in dissolution also\footnote{According to Vācaspati miśra the author of the Bhāmatī commentary on Śaṅkara’s BSBh, avidyā in its pure undeveloped form subsists during the period of pralaya. It is in that avidyā that all individual antahkaraṇas along with their impressions of past deeds continue to exist in the dormant form}, but it should be understood that it is only from the single/singular Brahman that there is vivarta (apparent transformation). (The advaitin offers this as a via media to solve the issue). Then the answer:

SY: If that is so then even in the case of one who is liberated there is the contingency of having connection with the world in the form of apparent transformation, as Brahman of the nature of pure consciousness is eternal. Therefore bondage and liberation are also imagined;  in truth  the meaning of the mahāvākyas is that they do not exist;  even when established somehow, happening to understand that vedānta-ācāryas say that there is nothing like bondage or liberation, (yathākathañcid yadi vyavasthāpyate, tadābandhamokṣādim kimapi nāstīti veḍantairucyate ityāpātajñānādeva mokṣārtham brahmasākṣāt\-kārāyājñānām śravaṇādikam, phalaniścayābhāvāt), then, since there is no certainty regarding the result, (phalaniścayābhāvāt) there will be no instructions for ignorant persons like śravaṇa etc., by acāryās like Yājñavalkya and there will be no vidvatsaṃnyāsins etc., (yājñavalkyā\-dīnām vidvatsaṃnyāsādikam ca na syāt).\footnote{The social fabric of attaching meaning to life by striving for a purpose such as liberation will be jeopardized. There will also be a loss of faith in vidvatsaṃnyāsins who are striving for liberation. And for the advaitin the worst thing can happen is that there will be a belief in the efficacy of rituals.} Then since there is certainty of the result through the performance of ritual involving a lot of wealth and effort it is reasonable that people engage in performance of rituals.\footnote{Bhikṣu’s contention is that in the case of rituals at least one is told the result which will  follow its performance. In Vedānta especially advaita the contention being that one is always free the motivation to follow śravaṇa etc., is absent.} Nor is there that certainty in your view that, the grand rituals connected with saṃnyāsins etc., will end with the result called ‘mokṣa’.\footnote{Bhikṣu again roots for his combination of rituls and jñāna jñānakarmasamuccaya as opposed to advaita’s jñāna alone approach to mokṣa.} The understanding of mokṣa is obtained only from vedānta-śāstra which is mokṣa-śāstra and not from any other (śāstra). And you say that vedānta-śāstra is not partial to mokṣa, as it is reasonable to understand it as partial to the absence of mokṣa (sa ca vedānto na mokṣādiparo bhavatocyate, mokṣādyabhāvaparatvābhyupagamāditi).\footnote{Maybe this a reference to the advaita statement that mokṣa is a state that is always present and so is not something to be sought after but to be experienced.}

\dev{ननु  प्रलयेऽप्यविद्यां व्यावहारिकीमभ्युपगम्य तया व्यावहारिको बन्धमोक्षादिरभ्युपेयो, न च तेन पारमार्थिकाद्वैतक्षतिरिति चेत्, प्रधानेऽपीदं तुल्यं प्रधानेऽर्थक्रियाकारित्वरूपव्यावहारिक सत्त्वस्यैवास्माकमिष्टत्वात् । ननु नित्यस्य प्रधानस्य कथमपारमार्थिकत्वं स्यात्, उच्यते, अपारमार्थिकत्वं यद्यनित्यत्वं, यदि वा स्वतः सिद्धत्वाभावः, यदि वा सदभिव्यक्त्यभावो, यदि वा सर्वकालेष्वभावो, यदि वान्यत्, तत्सर्वं प्रधानतत्कार्ययोर्नित्यानित्ययोः समानं, प्रधानस्य परिणामिनित्यत्वेऽपि घटस्य नित्यत्वाभावात् । स्वत: सिद्धत्वाभावादीनां च सर्वाचेतनसाधारणत्वात् , प्रधानादीनां रूपान्तरेण मध्येऽपि रूपान्तरैरतीतानागतै: सदैवाभावाच्च ।}

Advaitin: If it is said that even in dissolution accepting avidyā as pertaining to worldly activity one can understand bondage and liberation as a worldly activity,  so by that (argument) no harm will come to the advaita belief in the highest truth (as always permanent) then the answer is:

SY: This (permanence) is equally true of pradhāna; we only desire sattva in pradhāna in the form in which it is capable of accomplishing what is desired in the world (arthakriyākāritvarūpavyāvahārikasattvasyaiva asmākamiṣṭhatvāt).

Advaitin: How can the permanent pradhāna have the quality of endless materiality (nityasya pradhānasya kathamapāramārthikatvam syāt)? Then the answer is:

SY: If having the quality of endless materiality indicates impermanence, (or) if there is absence of accomplishing something oneself, (or) if there is absence of manifestation of existence (sadabhivyaktyabhāvo), (or) if there is absence at all times, or if there is something else (to be said here) all that is the same as pradhāna and its effects being permanent or impermanent (tatsarvam pradhānatatkāryayornityānityayoḥ samānam); even when the changing pradhāna is eternal there is absence of permanence of the pot (the effect/change). And in those that are not self-manifested lack of consciousness is common; in the case of pradhāna etc., even within the change of form (pradhānādīnām rūpān\-tareṇa madhy’pi) it was always absent through change of form such as the past and the future (forms) (rūpaṇtarairatītānāgataiḥ sadaivābhāvācc).\footnote{Bhikṣu’s contention in all his works has been that the essence of prakṛti continues in all changes and as such it is permanent though not in the advaita sense. (See also YS. III.12-15. Also see Rukmani: Yogavārttika:1987 pp.21-72).}

\dev{न चात्यन्तविनाशित्वमेव व्यावहारिकसत्त्वमिति वाच्यम्, अत्यन्तविशेषणवैयर्थ्यात्, पञ्चसूत्र्यां प्रदर्शितवाक्यै: परिणामित्वस्यैव व्यावहारिकसत्तात्वेन लक्षितत्वाच्चेति, व्यावहारिकसत्त्वाभ्युपगमेन शशशृङ्गादिवदत्यन्तासत्वं चानित्येष्वपि भवता त्यक्तमिति । यथा च प्रधाननित्यत्वेऽपि ब्रह्मचिन्मात्राद्वैतं तथा व्याख्यातं “तत्तु समन्वयादि” त्यत्रेति त्यज्यतां सकलश्रुतिस्मृत्यनुसारिषु प्रधानवादिषु प्रद्वेषेण सूत्राणां कष्टव्याख्यानं, त्यज्यतां च “प्रधानक्षेत्रज्ञपतिर्गुणेश:, क्षरं प्रधानममृताक्षरं हर:, क्षरात्मानावीशते देव एक:, मायां तु प्रकृतिं विद्यान्मायिनं तु महेश्वरमि” त्यादि श्रुतिषु “कार्यकारणकर्तृत्वे हेतु: प्रकृतिरुच्यत” इत्यादिसकलस्मृतिषु च प्रधानप्रकृत्यादि शब्देषु रुच्यर्थं चेहोपलक्षणादिभि: शिष्यव्यामोहनमिति दिक् ॥}

\dev{श्रुतीनां शरीरवच्छक्तिविधया अजास्रष्टृत्वपरत्वे हेत्वन्तरमाह---}

As the qualification of ‘total’ is useless (atyantaviśeṣaṇavaiyarthyāt) one cannot say that worldly existence is of the nature of total destruction alone (atyantavināśitvameva vyāvahārikikasattvamiti vācyam); as demonstrated in the statements in the first five sūtras (BS.I.1.1-5) it is indicated that only that which has change (pariṇāmitvasyaiva) has the capacity for action (vyāvahārikasattātvena lakṣitattvācca). By accepting worldly existence you have also abandoned absolute non-existence, like the rabbit’s horn,\footnote{Probably a reference to the three level reality accepted by advaita i.e prātibhāsika, vyāvahārika and pāramārthika; thus dream objects are accepted as having existence in the dream world and not rejected as totlaly non-existent in the world.} in impermanent things.  Just as Brahman as pure consciousness is non-dual even when pradhāna is permanent, (without a second), which has also been explained under the sūtra “tattu samanvayāt”, so (it is time to) give up this labored (difficult) interpretation (of pradhāna) out of hatred for those who following all śrutis and smṛtis argue in support of pradhāna. Also stop interpreting the meaning of the words like pradhāna, prakṛti etc., in śruti statements like “pradhānakṣetrajñapatirguṇeśaḥ” (Śvet.Up. 6.16)\footnote{The other half of this quotation is “saṃsāra-mokṣa-sthiti-bandha-hetuḥ”}, “kṣaram pradhānamamṛtākṣaram haraḥ, kṣarātmānāvīśate deva ekaḥ” (ibid. 1.10), “māyām tu prakṛtim vidyānmāyinam tu maheśvaram”(ibid. 4.10), and in smṛti statements like  “kāryakāraṇa kartṛtve hetuḥprakṛti\-rucyate” (Gītā. 13.21) as it pleases you and through using secondary meaning confusing the minds of the disciples (pradhānaprakṛtyādiśabdeṣu rucyartham cehopalakṣaṇādibhiḥ śiṣyavyāmohanamiti dik).

Another cause in favour of possessing continuous (constant) creation through power like the body in the context of śrutis is mentioned as “jñeyatvāvacanāt” (BS.I.4.4).\footnote{(According to the advaitin) this sūtra means that avyakta is mentioned as an entity to be known and therefore it cannot denote pradhāna. The reason is that in SY mokṣa/apavarga is a state when the diffference between puruṣa and prakṛti is known and not knowledge of either puruṣa or prakṛti. This is unlike advaita wherein the nature of Brahman is experienced and is therefore interpreted as known.}

\textbf{\dev{ज्ञेयत्वावचनाच्च ॥४॥}}

\dev{च शब्दो “दर्शयति चे” त्युक्तहेतुसमुच्चयार्थ: । अस्मिन् प्रकरणे प्रधानं ज्ञेयत्वेनोपास्यत्वेन च (न) प्रोक्तम्, न चाकार्याज्ञेयानुपास्यस्य प्रतिपादनं श्रुतिषु स्वपरं भवति “आम्नायस्य क्रियार्थत्वादानर्थक्यमतदर्थानामि” ति न्यायात्, किन्त्वन्यशेषमेव तद् “भूतार्थानां क्रियार्थेन समाम्नाय” इति न्यायात् । तथा च ज्ञेयत्वेनोक्तस्य परमात्मनो जगत्कारणत्वेन प्रकृतस्य तदुपपत्तये प्रधानं(न) कारणत्व(त्वं) शरीरवच्छक्तिविषयैवोच्यते न स्वातन्त्र्येणेत्यवधार्यत इत्यर्थ: ।।}

\dev{ननु “एतज्ज्ञेयं नित्यमेवात्मसंस्थं नात: परं वेदितव्यं हि किञ्चित् । भोक्ता भोग्यं प्रेरितारं च मत्वा सर्वं प्रोक्तं त्रिविधं ब्रह्ममेतत्” इत्यनेनास्मिन् प्रकरणे प्रधानस्य भोग्यशब्दवाच्यस्य ज्ञेयत्वं श्रुतिर्वदतीत्याश्ङ्क्य समाधत्ते -----}

SY:\footnote{Bhikṣu counters that argument in a different way. This contention of the advaitin has not been  proven as can be seen from the BS.III.3.4 (“darśayati ca”). He connects the ‘ca’ in this sūtra (BS.I.4.4) with the ‘ca’ in sūtra BS. III.3.4. and interprets it differently in his favour. }  The word “ca” is to be understood as having a combined meaning with the reason mentioned as “darśayati ca” (BS.III.3.4). 

Advaitin: In this section (prakaraṇe; BS.III.3.4)) pradhāna has not been mentioned as fit to be known or as fit to be meditated upon (jñeyatvenopāsyatvena cāproktam); in the śrutis there is no presentation of anything that is without action, (that is) not fit to be known and (that is) not fit to be meditated upon, which solely refers to itself (na cākāryājñeyānupāsyaya pratipādanam śrutiṣu svaparam bhavati). This is because of the rule “āmnāyasya kriyārthatvādānarthakyamtadarthānā\-miti” (PMS I.2.1)\footnote{In this PMS stresses the purpose of the Veda is only to enjoin ritual/action; therefore the other parts which do not have this purpose are useless and those parts can be considered to be impermanent.}; thus those portions (which do not deal with action) are only a supplementary statement to the others; this is because of the rule: “bhūtārthānām kriyārthena samāmnāyaḥ”.\footnote{This reiterates the tradition of the purpose of the Veda being action.} Thus paramātman, being the cause of the world  and mentioned as that to be known, pradhāna has no causal role in the coming forth of the world, (and it acts) like the body (which acts through the power of the jīva consciousness) through the power (of Brahman) and not independently, thus it is understood (tathā ca jñeyatvenoktasya paramātmano jagatkāraṇatvena prakṛtasya tadupapattaye pradhānam na kāraṇatvam  śarīravacchaktividhayaivocyate, na svātantryeṇetyavadhāryata ityarthaḥ).

SY: By the statement: “etajjñeyam  nityamevātmasaṃstham...kiñcit, bhoktā bhogyam preritāram ca matvā...trividham brahmametat” (Śvet.I.12)  due to the doubt that in this section, śruti mentions that pradhāna should be known through the meaning denoted by the word  “bhogya”, he (the advaitin) clears (that doubt) through (the sūtra): “vadatīti cenna prājño hi prakaraṇāt” (BS.I.4.5).

\newpage

\textbf{\dev{वदतीति चेन्न प्राज्ञो हि प्रकरणात् ॥५॥}}

\dev{यदुक्तं तत्र (तन्न) प्राज्ञ ईश्वर एव ह्यत्र ज्ञेयतयोपदिश्य तं तत्त्वेतच्छब्देन भोक्त्रादित्रयं परामृष्य तस्य ज्ञेयत्वमुच्यते~। कुत:? प्रकरणात् ब्रह्मण: प्रकरणित्वेन तस्यैवेतच्छब्देन परामर्षस्य युक्तत्वादित्यर्थ: । भोक्त्रादित्रयं तु मत्वेत्यनेनैवान्वेति । भोक्तेत्यस्य विभक्तिव्यत्यासेन भोक्तारमित्यर्थ: । प्रेरिता देवतावर्ग: ॥}

“vadatīti cenna prājño hi prakaraṇāt”

Advaitin: That which is said is not so;\footnote{According to Tripathi ‘tanna’ is the right reading and not ‘tatra’; I have translated it accordingly} having mentioned Īśvara alone to be understood by the word “prājña” then, reflecting on the (meaning of the) three words bhoktā (bhogyam, preritāram) it is mentioned by the word “etat” (in Śvet.I.12) that Īśvara is fit to be known. 

 SY: How do you say that? 
 
Advaitin: Since Brahman is the topic in the section it is appropriate that the subject of reflection is It (Brahman) itself  denoted by the word “etat” (prakaraṇāt brahmaṇaḥ prakaraṇitvena tasyaivaitacchabdena parāmarṣasya yuktatvādityarthaḥ).The three (words)  “bhoktā” etc., are connected in sequence to  the word “matvā” itself. The word “bhoktā” having a different case (vibhaktivyatyāsena) means “bhoktāram”. “preritāram” stands for a group of devatās.\footnote{The meaning is not clear. Perhaps it refers to Brahman being the mover (preritāram)}

\textbf{\dev{त्रयाणामपि चैवमुपन्यास: प्रश्नश्च ॥६॥}}

\dev{किं चैवं सति ब्रह्मवत् त्रयाणां भोक्तृभोग्यप्रेरकदेवानामपि वाक्योपक्रमे प्रश्न: प्रतिवचनाख्य उपन्यासश्च प्रसज्येयातां न त्वेवं दृश्यते “किं कारणं ब्रह्म कुत: स्म जाता” इत्यादिना केवलब्रह्मण एव प्रश्नदर्शनात्, “य: कारणानि निखिलानि तानि कालात्मयुक्तान्यधितिष्ठत्येक” इत्यादिना केवलब्रह्मण एवोपन्यासदर्शनाच्चेत्यर्थ: ।}

\dev{यच्च “ते ध्यानयोगानुगता अपश्यन् देवतात्मशक्तिं स्वगुणैर्निगूढामि” ति पूर्वार्धश्लोकेनादौ शक्तिदर्शनमुक्तं, तच्छक्तिमतोऽधिष्ठातृत्वदर्शनोपयोगितयेति मन्तव्यम्, स्वतन्त्रज्ञेयत्वे शक्तित्ववचनानुपपत्तेरिति ॥}

\dev{दृष्टान्ताच्च प्रधानं न स्वतन्त्रं किन्तु शरीररूपकविन्यस्तमेवेत्याह -----}

SY: If it is so, then in the case of (other) devas who have a threefold  form as bhoktṛ, bhogya and preraka, similar to Brahman, there will be the contingency of explaining at the beginning the question and the answer; but one does not see this as the case. “kim kāraṇam brahma kutaḥ sma jātā” (Śvet. Up. I.1), in such cases one only sees the question (raised) with reference to Brahman, (and) through “yaḥ kāraṇāni likhitāni tāni kālātmayuktānyadhitiṣṭhatyeka” (ibid. I.3) one sees reflection only on Brahman alone. 

Thus in the first half of the verse “te dhyānayogānugatā apaśyan devātmaśaktim svaguṇairnigūḍhām” (ibid. I.3) it says that knowing one’s own power is mentioned at first (ādau śaktidarśanamuktam); one should think that this is for  indicating being the support which possesses the power (tacchaktimato’dhiṣṭhātṛtvadarśanopayogitayeti mantavyam; (if it is to be known independently) then the mention of possessing the power is not reasonable (svatantrajñeyatve śaktitvavacanānupapattiriti).

Even through the example (one knows that) pradhāna is not independent but it is explained only in the form of the body, thus he says: 

\textbf{\dev{महद्वच्च ॥७॥}}

\dev{यथा  महत्तत्त्वमन्त:करणविशेष: आत्मन: चेष्टाश्रयलक्षणशरीरमेव सचक्षुरादिभि: संयुज्य स्वकार्यजनने क्षमं भवति न स्वतन्त्रमिति सर्वैरभ्युपगम्यते, तथैव प्रधानमप्यचेतनत्वाविशेषादुचितमित्यर्थ: ।}
\begin{verse}
\dev{इन्द्रियेभ्य: परा ह्यर्था अर्थेभ्यश्च परं मन: ।}\\
\dev{मनसस्तु परा बुद्धिर्बुद्धेरात्मा महान् पर: ॥}\\
\dev{महत: परमव्यक्तमव्यक्तात् पुरुष: पर: ।}
\end{verse}

\dev{इत्याद्या: श्रुतय: । स्मृत्यनुमिते(तय)श्च । न चात्र वाक्ये आत्मशब्दश्रवणवन्महच्छब्दो जीवार्थक एवास्त्विति वाच्यम्, अचेतन्स्यात्मपरत्वानुपपत्ते:, पुरुषशब्देनैवात्मनोऽग्रे निर्देक्ष्यमाणत्वाच्च । तथा अचेतनवर्गमध्ये पाठेन महतोऽप्यचेतनौचित्यात्, घटस्यात्मा  पिण्ड इतिवत् कारणेऽप्यात्मशब्दोपपत्ते:, महच्छव्दस्य स्मृतिषु महत्तत्त्वे रूढत्वाच्च, स्मृतिषु व्यवहारस्यापि श्रुतिमूलकतया तासु स्वतन्त्रसङ्केतकल्पनाया बाधकं विनाऽन्याय्यत्वाच्च । अस्यां च श्रुताववान्तरभेदेन बुद्धिमहतोर्भेद उक्तो महच्छब्दस्य चित्तवाचितयेति न स्वस्य स्वपरत्वानुपपत्ति: । तथा चोक्तं “यदाहुर्वासुदेवाख्यं चित्तं तन्महदात्मकमि” ति । प्रायशश्च स्मृतिषु चित्तबुद्धी एकीकृत्य बुद्धिशब्देन महत्तत्त्वमुच्यते।}

\dev{ननु “अस्य महतो भूतस्य नि:श्वसितमेतदि” त्यादिबहुश्रुतिषु महच्छब्द आत्मन्यप्युच्यत इति चेत्, सत्यं , महत्तत्त्वोपाधिनैव चात्मन्यपि मानुषादिशब्दवन्महच्छब्दप्रयोगस्यौचित्यात् आत्मन: स्वतो विशुद्धचिन्मात्रत्वेनाविशिष्टतया महत्त्वाल्पत्वविभागायोगादिति ॥}

\dev{परशेषतया प्रधानकारणताप्रतिपादने श्रौतदृष्टान्तमाह ----}

%\textbf{Mahadvacca (I.4.7)}

%\textbf{\dev{महद्वच्च ॥७॥}}

%\dev{यथा  महत्तत्त्वमन्त:करणविशेष: आत्मन: चेष्टाश्रयलक्षणशरीरमेव सचक्षुरादिभि: संयुज्य स्वकार्यजनने क्षमं भवति न स्वतन्त्रमिति सर्वैरभ्युपगम्यते, तथैव प्रधानमप्यचेतनत्वाविशेषादुचितमित्यर्थ: ।}
%\begin{verse}
%\dev{इन्द्रियेभ्य: परा ह्यर्था अर्थेभ्यश्च परं मन: ।}\\
%\dev{मनसस्तु परा बुद्धिर्बुद्धेरात्मा महान् पर: ॥}\\
%\dev{महत: परमव्यक्तमव्यक्तात् पुरुष: पर: ।}
%\end{verse}

%\dev{इत्याद्या: श्रुतय: । स्मृत्यनुमिते(तय)श्च । न चात्र वाक्ये आत्मशब्दश्रवणवन्महच्छब्दो जीवार्थक एवास्त्विति वाच्यम्, अचेतन्स्यात्मपरत्वानुपपत्ते:, पुरुषशब्देनैवात्मनोऽग्रे निर्देक्ष्यमाणत्वाच्च । तथा अचेतनवर्गमध्ये पाठेन महतोऽप्यचेतनौचित्यात्, घटस्यात्मा  पिण्ड इतिवत् कारणेऽप्यात्मशब्दोपपत्ते:, महच्छव्दस्य स्मृतिषु महत्तत्त्वे रूढत्वाच्च, स्मृतिषु व्यवहारस्यापि श्रुतिमूलकतया तासु स्वतन्त्रसङ्केतकल्पनाया बाधकं विनाऽन्याय्यत्वाच्च । अस्यां च श्रुताववान्तरभेदेन बुद्धिमहतोर्भेद उक्तो महच्छब्दस्य चित्तवाचितयेति न स्वस्य स्वपरत्वानुपपत्ति: । तथा चोक्तं “ यदाहुर्वासुदेवाख्यं चित्तं तन्महदात्मकमि” ति । प्रायशश्च स्मृतिषु चित्तबुद्धी एकीकृत्य बुद्धिशब्देन महत्तत्त्वमुच्यते।}

%\dev{ननु “अस्य महतो भूतस्य नि:श्वसितमेतदि” त्यादिबहुश्रुतिषु महच्छब्द आत्मन्यप्युच्यत इति चेत्, सत्यं , महत्तत्त्वोपाधिनैव चात्मन्यपि मानुषादिशब्दवन्महच्छब्दप्रयोगस्यौचित्यात् आत्मन: स्वतो विशुद्धचिन्मात्रत्वेनाविशिष्टतया महत्त्वाल्पत्वविभागायोगादिति ॥ }

%\dev{परशेषतया प्रधानकारणताप्रतिपादने श्रौतदृष्टान्तमाह ----}

\textbf{Mahadvacca (I.4.7)}

SY: Everyone agrees that the mahat-tattva is the special quality of the internal organ (yathā mahattatvamantaḥkaraṇaviśeṣaḥ),  and is the body charaterized as being the support of ātman for action by being connected with the eyes etc., (i.e the other sense-organs)  and it is not independent. In a similar way it means that pradhāna also, not qualified by sentience, is also fit (for the role).\footnote{Since pradhāna is connected to paramātman as its śakti it can also have the action of creation just like the jīva and its body.} There are śruti statements (which support that) like: “indriyebhyaḥ parā hyarthā arthebhyaśca param manaḥ manasastu parā buddhirbuddherātmā mahān paraḥ...avyaktamayaktāt puruṣaḥ ``agrees paraḥ” (Kaṭha.Up.3.10-11).  Smṛtis (also) infer this.

Advaitin: It cannot be said that just like hearing the word “ātman” in this statement let the word ‘mahat’ mean the jīva.

It is not reasonable that something which is insentient is connected to ātman; also ātman is mentioned later by the word “puruṣa (agre nirdekṣyamāṇatvācca). Therefore since mahat is included in the group of insentient (principles) it is reasonable to consider mahat also as insentient. Just as (one says that) the essence of the pot is the clay (ghaṭasyātmā piṇḍa itivat) it is appropriate to use the word ātman in the sense of ‘cause’; the word ‘mahat’ in smṛtis is conventionally understood to mean the ‘mahat tattva’\footnote{This is the first evolute from prakṛti; this is the SY cosmology.} and action in the smṛtis having its basis in the śrutis, imagining  independence in them is not possible without contradiction (of what is stated in the śrutis); it is against the rule as well (anyāyyatvācca). 

In this śruti by another division, a difference between buddhi (intellect) and mahat has been mentioned;\footnote{This is mentioned, as in SY, buddhi and mahat are more or less synonymous.} the word mahat having the denotation of mind (citta) there is no contradiction (of the nature of) oneself having connection with oneself. Thus it is said: “yadāhurvāsudevākhyam cittam tanmahadātmakam” iti. In general in the smṛtis, by the word ‘buddhi’ both citta and buddhi combined  are mentioned as the mahat-tattva.

SY: If it is said that in many such śruti statements as: “asya mahato bhūtasya niḥśvasitametat” (Bṛ.Up. 2.4.10; 4.5.11; Mait.Up. 6.32) the word ‘mahat’ is used to denote ātman, then the answer is:

Advaitin: True; like the words mānuṣa etc., being used for ātman due to the limitation of mahat alone (mahattattvopādhinaiva cātmanyapi mānuṣādiśabdavat), it is proper to use the word mahat for ātman; ātman in itself (its pure state) (svataḥ) is of the nature of pure consciousness alone, it is unqualified (aviśiṣṭatayā) to be divided as small or big (viśuddhacinmātratvenāviśiṣtatayā mahattvālpatvavibhāgāyogāt) 

As ‘para’ (consciousness) is what remains, in order to show the causal role of pradhāna, an example from śruti is mentioned:

\textbf{\dev{चमसवदविशेषात् ॥८॥}}

\dev{यथा “अर्वाग् बिलश्चमस ऊर्ध्वबुध्न” इत्यस्मिन्मन्त्रे चमसो यागशरीरत्वेनैव प्रतिपादितो न स्वातन्त्र्येण तथा प्रधानमपि जगत्कारणब्रह्मशरीरत्वेनैवात्र प्रतिपादितं न स्वातन्त्र्येण , कुत:? अविशेषात् परप्रकरणे स्वातन्त्र्येण प्रतिपादनस्यार्थकत्वाविशेषादित्यर्थ: ।}

\dev{औपनिषदमेव दृष्टान्तान्तरमाह-----}

SY: Just as in the mantra “arvāg bilaścamasa ūrdhvabudhna” (Bṛ.Up. 2.2.3)\footnote{The literal meaning of `camasa' is `a special sacrificial vessel/bowl which has the bottom up and the mouth below.} the  vessel has been depicted as being the body of the sacrifice itself and not as being independent, so also pradhāna also has been shown to be the cause of the world through having the body of Brahman and not independently (asminmantre yāgaśarīratvenaiva pratipādito na svātantryeṇa tathā pradhānamapi jagatkāraṇabrahmaśarīratvenaivātra pratipāditam na svātantryeṇa. Why is it so? “aviśeṣāt”=  it is useless to point out (that the origin of the world was by pradhāna independently) in the section dealing with the supreme (Brahman).

He mentions another example from the Upaniṣads as:

\textbf{\dev{ज्योतिरुपक्रमात्तु तथा ह्यधीयत एके ॥९॥}}

\dev{तु शब्द: पुनरर्थे । ज्योतिरुपक्रमात् तेज आद्या तेजोऽबन्नरूपा शक्ति: पुनस्तथा चेतनशरीरविधयैव सृष्टो हेतु:, हि यस्मात् एके कौथुमा अधीयते “ तत्तेज ऐक्षत बहुस्यां प्रजायेय तदपोऽसृजते” त्यादि । अत्रेक्षितृचेतनस्य शरीरविधयैव तेज आदीनां स्रष्टृत्वं पठ्यते इत्यर्थ: । अतस्तद्दृष्टान्तेनाजास्रष्टृत्वमपि चेतनब्रह्मशरीरविधयैव ग्राह्यमिति। न चाजाख्य प्रधानाभिमानिदेवता शरीरविधयाऽजाया: सकलस्रष्टृत्वं संभवति, देवानां बुद्ध्युपाधिकतया बुद्धिपूर्वकमहत्तत्त्वसृष्टावकारणत्वादिति ।}

\dev{आधुनिकास्तु आनुमानिकमप्येकेषामित्यारभ्य सर्वाणि सूत्राणि सांख्ययोगसिद्धानां प्रधानमहदादि पदार्थानां स्वरूपप्रतिषेधपरतया यथाकथाञ्चिद् व्याचक्षते, तत्तु श्रुतिस्मृतिन्यायादिविरोधादुपेक्षणीयं मुमुक्षुभि:। यच्च तेषामजाशब्दस्य भूतत्रयपरतया व्याख्यानं, तदपि हेयम्,}

\begin{verse}
\dev{अशक्य: सोऽन्यथा द्रष्टुं ज्ञायमान: कुमारका: ।}\\
\dev{विकारजननीं मायामष्टरूपामजां ध्रुवाम् ॥}
\end{verse}

\dev{इति चुलिकोपनिषदादौ प्रधान एवाजात्ववचनात् । “मायां तु प्रकृतिं विद्यादि” त्याद्यव्यवहितवाक्यशेषतो "देवात्मशक्तिं स्वगुणैर्निगूढामि” ति वाक्योपक्रमतश्चाजामेकामित्यादि वाक्येऽप्यजाशब्दस्य मूलकारणवाचित्वावधारणाच्च । तथाऽनादिसृष्ट्युपपादकतया अजाशब्दे यौगिकार्थस्यापि प्रकृते विवक्षितत्वाच्चेति ॥}

\textbf{BS.I.4.9}

\textbf{Jyotirupakramāttu tathā hyadhīyata eke}

The word “tu” is used in the sense of repetition. “jyotirupakramāt”= fire is the first ; it is a lower kind of power and again through (being) the body of consciousness is the cause for the origin of the world. That is the reason why one learns from a descendant of the teacher Kuthumin: “tatteja aikṣata bahusyām prajāyeya tadāpo’sṛjata” ityādi (Chānd.Up.62.3).\footnote{In other Upaniṣads the order of creation starts with  space and so Bhikṣu  mentions that the branch associated with the teacher Kuthumin has this order of creation.} In this context it is through the medium of the body of the seer-consciousness (consciousness in the form of the seer) that the creation of fire etc., is mentioned. Therefore through that example, the creativity of ‘ajā’ (prakṛti) is also to be understood through the medium of being the body of Brahman-consciousness.  Nor is it possible that through the medium of the body of pradhāna known as ajā, having a sense of agency, it can be the creator of everything (na cājākhyapradhānābhimānidevatāśarīravidhayā’ajāyāḥ sakalasraṣṭṛtvam sambhavati).\footnote{This is possibly a reference to the intellect/buddhi which has a sense of agency associated with the evolute ‘asmitā/ahaṅkāra’.} This is because the deities through the limitation of the intellect are not causes in the creation by mahat preceded by the intellect.

Modern Vedāntins starting with (the sūtra) “ānumānikampyekeṣām...” (I.4.1)  interpret all the sūtras somehow as rejecting the real nature of the Sāmkhy-Yoga principles (padārthas) such as pradhāna, mahat etc.,; since that goes against śruti, smṛti and logic it needs to be ignored by those who seek mokṣa (tattu śrutismṛti nyāyādi virodhādupekṣaṇī\-yam mumukṣubhiḥ). Their interpretation of the word’ajā’ as connected to the three elements needs to be rejected as well.\footnote{This goes back to the controversy of the meaning of the word ‘ajā’ which occurs in Śvet.Up. 4.5. The trans. runs as follows: The One Unborn, red, white and black, who produces manifold offspring similar in form (to herself), there lies the one unborn (male) delighting. Another unborn gives her up, having had his enjoyment (trans. S.Radhakrishnan). While the SY philosophers translate red, white and black as associated with the three guṇas rajas, sattva and tamas the advaitins asssociate it with the three elements fire, water and food. Thus the advaitin questions the SY identification of ajā with pradhāna} In statements like the Cūlika Up.: “ aśakyaḥ so’nyathā draṣṭum..māyāmaṣṭarūpāmajām dhruvām” (Mantrika.Up.3; Cūlika is also called as Mantrika.Up).\footnote{This is a minor Upaniṣad called by both names and is attached to the Śukla YV.}

Statements such as “māyām tu prakṛtim vidyāt etc.,” (Śvet. Up. 4.10) and from its conclusion immediately of the statement (ibid)\footnote{The verse continues and ends as  follows: “māyinam tu maheśvaram, tasyāvayavabhūtaistu vyāptam sarvamidam jagat”. It basically says that Maheśvara is the one who works through māyā. All beings (in the world) are parts of Him and pervade the whole world. Thus it basically says that prakṛti, through Maheśvara, has a big role to play in creation.}, and from the verse starting  as:  “devātmaśaktim svaguṇairnigūḍhām” (Śvet. Up. I.3) with reference to the sentence: “ajāmekām etc” (Śvet.Up. 4.5) one understands that the word ‘ajā’ denotes the main cause. Thus due to bringing about creation from beginningless times, it is desired that in the present context, the conventional meaning also be associated with the word ‘ajā’.

Advaitin: How is it said that the beginingless power of Īśvara is indicated by the word ‘ajā’ in the sentence (ajām  ekām lohitaśuklakṛṣṇām etc) since there is an absence of any colour in prakṛti, which is the topic of discussion, it is not possible to have the form of red, white and black (tasyāḥ lohitaśuklakṛṣṇarūpatvāsambhavāt prakṛtau rūpādyabhāvāt). This doubt is removed by (the next sūtra): “kalpanopadeśācca madhvādivadavirodhaḥ”

\textbf{\dev{कल्पनोपदेशाच्च मध्वादिवदविरोधः ।।१०।।}}

\dev{च शब्दः पुनरर्थे।रूपादिवचनाविरोधः पुनरत्र मध्वादिविद्यायामिव कल्पनयोपदेशादित्यर्थः । यथा हि आदित्यस्यामधुनोऽपि मधुत्व वाचश्चाधेनोरपि धेनुत्वं भोग्यपदत्वादिगुणलाभार्थं रूपकल्पयोपदिश्यते श्रुतिभि: “असौ वा आदित्यो देवमधुवाचं धेनुमुपासीते” त्यादिभि: तथैव प्रधानशब्दवाच्यानां सत्त्वादिगुणानां नीरूपाणामपि लोहितादिरूपाणि रागप्रकाशावरणरूपतालाभार्थं रूपकल्पनयोपदिष्टानीति ॥}

\dev{विरोधान्तरं परिहरति----}

\textbf{BS.I.4.10}

\textbf{kalpanopadeśācca madhvādivadavirodhaḥ}

The word “ca” in the sūtra (I.4.10) is in the sense of repetition. There is no contradiction in the use of the word “rūpādi” as like the meditation on vidyā etc., it has been taught through imagination. Just as the sun not being honey is  imagined to be  honey, (yathā hi ādityasyāmadhuno’pi madhutvam), just as speech which is not a cow is spoken of as cow, so also in order to possess the quality of something fit to be enjoyed, the śrutis teach by imagining the colours (rūpakalpanayopadiśyate śrutibhiḥ) through statements like “asau vā ādityo devamadhu” (Chānd.Up. 3.1.1), “vācam dhenumupāsīta” (Bṛ.5.8.1); so also the qualities sattva (rajas and tamas) which are denoted by the word pradhāna, even if without colour, are taught the colours red etc.,, by imagining the colours in order to obtain the forms of emotion, illumination and illusion (respectively).\footnote{Rāga,  energy (red/lohita) associated with rajas, (śukla) prakāśa, brightness or illumination with sattva and  (kṛṣṇa) āvaraṇa, delusion connected with tamas.} He avoids another objection through the sūtra “na saṃkhyopasaṅgrahādapi nānābhāvādatirekācca” (BS.I.4.11)

\textbf{\dev{न संख्योपसंग्रहादपि नानाभावादतिरेकाच्च ॥११॥}}

\dev{विरोध इतिच्छेदेनानुवर्तते । अथ वा पूर्वसूत्रे निषेधार्थकपृथक्पदमेव वा शब्द: । तथा च् अजावाक्ये एकत्वसंख्याग्रहणाद् यो विरोध: प्रसज्यते सोऽपि नेत्यर्थ: । अत्र हेतुस्त्वतिविस्तरभयात् पृथक्सूत्रैर्वक्ष्यते । नन्वजामेकामित्येकत्वस्य कुतो विरोध: प्रसक्त इति तत्राह---- नानाभावादिति । काण्वादिशाखान्तरे “यस्मिन् पञ्चजना आकाशश्च प्रतिष्ठित: तमेवमन्य आत्मानं विद्वान् ब्रह्मामृतोऽमृतमि” त्यादिना ब्रह्मशक्तेर्नानात्वावगमादित्यर्थ: । अत्र हि पञ्च पञ्चजना इत्यादिना शक्तिविशेषाणां प्रलयकालीनानामेव ग्रहणं वाच्यं, कार्यशक्तिपरत्वे तासामनन्तत्वेन पञ्च पञ्चेत्याकाशश्चेति विशेषवचनासामञ्जस्यात्, तत्कथमुच्यते? “ एकामजां ब्रह्वी: प्रजा: सृजमानामि” ति ॥}

\newpage

\dev{ननु सांख्योक्तिं पञ्चविंशतितत्त्वपरतया अत्र वाक्ये मुख्यकार्यमादायावान्तरभेदेनैकस्या एवाजायाश्चतुर्विंशतिसंख्योपपादनीयेति शङ्कायां दूषणान्तरमाह ------ अतिरेकाच्चेति । पञ्चविंशत्यतिरिक्तस्याकाशस्य शक्तिमध्ये पाठाच्चेत्यर्थ:॥}

\dev{पूर्वसूत्रेण प्रतिज्ञाते विरोधाभावे हेतुमाह -----}

By splitting the word as ‘virodha’ and (antaram) (the next sūtra) follows (anuvartate). Or it is a separate word meaning after rejection by the previous sūtra (athavā pūrvasūtre niṣedhārthakapṛthakpadameva vā).

Thus in the sentence “ajā”etc., by counting it as being one, the contradiction that emerges is also not there. Out of fear for the lengthiness of stating the reason for (this) here itself, it is being mentioned by different sūtras.  

Ques: How can there be a possibility of contradiction of its (ajā) being one in the sentence “ajāmekām; then the answer is:

Ans: “nānābhāvāt”= in the Kāṇvaśākhā by the statement beginning: “yasmin pañcajanā ākāśaśca pratiṣṭhitaḥ tamevamanya ātmānam vidvān brahmāmṛto’mṛtamiti” (Bṛ.Up.4.4.17) one understands that the power of Brahman is many (Brahmaśakteḥ nānātvāvagamāt). Herein by the words “pañca pañcajanā” one needs to understand that only the special powers which are present at the time of dissolution has been mentioned. As far as their connection with effects (evolving effects) is concerned, as they are without limit, it is not right to mention the special ākāśa (ether) along with the five groups of five (atra hi pañca pañcajanā  ityādinā śaktiviśeṣāṇām pralayakālīnānāmeva graha vācyam, kāryaparatve tāsāmanantatvena pañca pañcetyākāśaśceti viśeṣavacanāsāmañjasyāt).\footnote{This has reference to the controversy of enumeration of the 25 tattvas (categories)  between the Sāṃkhya philosophers and the advaitins. According to Śaṅkara the 25 are mentioned in Bṛ.Up.4.4.17 as the five celestial ministrels (gandharvas), the manes (pitṛs),the devas, the asuras and the rākṣasas.; alternately they are enumerated as the four castes along with the fifth niṣādas This seems a weak argument as the straightforward explanation is the 25 categories of Sāṃkhya.}

Ques: How do you say that?

Ans: Because of the statement “ekamajām bahvīḥ prajāḥ sṛjamānāmiti”.

\eject

Ques: Another defect called ‘atirekācca’ (exceeding the number 24)\footnote{If one includes puruṣa and space mentioned in the Bṛ.Up. quotation (4.4.17)the number exceeds twentyfour. The verse in the Bṛ.Up runs as follows: “yasmin pañca pañcajanā ākāśaśca pratiṣṭhitaḥ...”. So the objection is to the number exceeding 24. Even if puruṣa can be explained by the Sāṃkhya acceptance of the 25 tattvas space has still to be explained as an excess according to the advaitin.} is mentioned with the doubt that in this statement of Sāṃkhya taking into consideration the main effect one needs to accept the 24 categories of a single ‘ajā connected with the 25 tattvas,’  through differences in between (avāntarabhedena).  This doubt is removed by the answer:

Ans: Space which exceeds the 25 tattvas is included under the powers.

The reason for the absence of the defect declared/stated in the previous sūtra is mentioned as: “prāṇādayo vākyaśeṣāt” (BS.I.4.12)

\textbf{\dev{प्राणादयो वाक्यशेषात् ॥१२॥}}

\dev{यद्यप्यत्र सूत्रे हेतुतावाचक: शब्दो नास्ति, तथापि “वासस्ते मलिनं राजकुलं मा गा”” इत्यादाविव हेतुता लभ्यते तृतीयसूत्रपर्यालोचनयेति बोध्यम् । तथाचायं सूत्रार्थ: ------- यस्मिन् पञ्च पञ्च जना इत्युदाहृतवाक्ये प्राणादय: पञ्चैव पुरुषभेदेन व्यक्तिभेदात् पञ्च पञ्चेति वीप्सयोक्ता:, ते च कार्यभूता एव शक्तय इति । तथा च तेषां नानात्वेऽपि न मूलशक्तेरजाया एकत्वं विरुध्यत इति तृतीयसूत्रेण वक्ष्यति । हेतुमाह ---- वाक्यशेषादिति । “प्राणस्य प्राणमुत चक्षुषश्चक्षुरुत श्रोत्रस्य श्रोत्रमन्नस्यान्नं मनसो ये मनो विदुरि” ति अव्यवहितवाक्यशेषात् तत्रोक्ता:षष्ठ्यन्तप्राणादिशब्दार्था एव प्राणादय: पञ्च प्रतिपुरुषव्यक्तिभेदेन वीप्सया पूर्ववाक्य प्रयुक्ता इत्यर्थ: । अत्र वाक्ये प्राणस्य प्राणमित्यादिवचनात् पूर्ववाक्येऽपि शक्तिमति ब्रह्मणि प्रतिष्ठितत्वं।प्राणादेर्लभ्यते, शेषस्य प्राणादिशब्दपञ्चकस्य प्रेरकस्य प्रेरकतामात्रार्थकत्वात् । केनेषितोपनिषदि “केनेषितं पतति प्रेषितं मन:” इत्यादि प्रश्नस्यात्रैव “श्रोत्रस्य श्रोत्रं मनसो मनो यदि”त्यादि प्रत्युत्तरदर्शनात् “प्राण:कम्पनादि” त्यादिगतसूत्रेभ्यश्चेति ॥}

\dev{ननु भवत्येवं (त्वेवं) माध्यन्दिनशाखायां, काण्वे तु चतुर्णामेव प्राणचक्षु: श्रोत्रमनसां च वाक्यशेषे पाठात् पञ्चत्वानुपपत्तिरित्याशाङ्क्य काण्वेऽपि पञ्चत्वमुपपादयति ----}

Even though there is no word to denote the reason in this sūtra still like the statement “vāsaste malinam rājakulam mā gā” etc.,\footnote{The sentence means “your clothes are dirty, do not go to the king’s presence”; in this sentence the reason is obtained by thinking why should I not go to the king’s presence. The answer contains the reason. So also in this instance} one understands the reason by thinking about the third sūtra (BS.I.4.3). Thus the meaning of this sūtra is: In the sentence “yasmin pañca pañca janā” used as an example it is the five prāṇa etc., repeated as five, five, due to the difference of individuals because of puruṣas being different;\footnote{Since there are many puruṣas in SY each associated with its own body there are many individuals} they are also śaktis (powers) of the nature of effects. Thus even though they are many it does not contradict the oneness of the main power ‘ajā’; this will be mentioned by the third sūtra. The reason is mentioned through what is implied by the sentence  (unfinished sentence) (hetumāha-vākyaśeṣāditi). By the immediate following sentence: “prāṇasya prāṇamuta cakṣuṣaścakṣuruta...manaso ye mano viduḥ” (Bṛ.Up.4.4.18) the meaning of the words ‘prāṇa’ etc., in the sixth case mentioned therein are the five prāṇas; due to  individual difference pertaining to each puruṣa the earlier sentence is used (twice) by repetition.

In this sentence by stating “prāṇasya prāṇam” etc., one understands that in the earlier sentence also prāṇa etc., are situated in the Brahman possessing power (pūrvavākye’pi śaktimati brahmaṇi pratiṣṭhitatvam prāṇāderlabhyate); as for the remaining fivefold prāṇa (apāna, samāna, vyāna, and udāna) it only has the sense of the power of motivation of that which motivates.\footnote{This whole para is so convoluted that there is no clear cut meaning conveyed.} In answer to the questions raised in the Kena.Up: “keneṣitam patati preṣitam manaḥ” (I.1) etc.,  the answer is given here itself as “śrotrasya śrotram manaso mano yat”; it is also stated in the earlier sūtras such as “prāṇaḥ kampanāt” etc. 

Ques: this is the reading in the Mādhyandina branch of the Bṛ.Up; whereas in the  Kāṇva branch reading the sentence has only four  prāṇa, cakṣuḥ, śrotra and manas; so there is an inconsistency in counting it as ‘five’;  having this doubt he declares a fivefold division in the Kāṇva branch also as “jyotiṣaikeṣāmasatyanne” (BS.I.1.13)

\textbf{\dev{ज्योतिषैकेषामसत्यन्ने ॥१३॥}}

\dev{एकेषां काण्वानामन्नेऽसति ज्योतिषा पञ्चसंख्या पूरणीया “त्तद्देवा ज्योतिषां ज्योतिरि” ति पूर्वमुक्तत्वादित्यर्थ: । माध्यन्दिने तु पञ्चसंख्याया: प्राणादिवाक्येनैव पूरणादत्र ज्योतिर्वाक्यस्य सस्यसाम्येऽपि तदाऽऽकाङ्क्षेति भाव: । एतेन “यस्मिन् पञ्च पञ्च जना” इति वाक्यस्य कार्यशक्तिपरतया प्रधानातिरिक्तस्याकाशस्य शक्तिमध्यपाठेऽपि न प्रधानमात्रस्य मूलशक्तेरेकत्वविरोध इत्यप्युक्तप्रायं वेदितव्यम् ॥}

\dev{ननु तथापि एकशब्दस्याद्वितीयार्थकतया प्रधानमात्रशक्तित्व (त्वे) विरोध एव, विष्णुपुराणादिषु जीवकालयोरपि शक्तित्वावगमादिति, तत्राह-----}

In the case of the Kāṇva school the fifth which has an absence of food, is to be made up through light stated previously as: “taddevā jyotiṣām jyotiḥ” (Bṛ.Up. 4.4.16). In the Mādhyandina recension the number five is completed by the word prāṇa itself whereas here since the sentence jyoti etc., has resemblance (has something in common) to sattva there is that desire to know. By this the sentence “yasmin pañca pañca janā” having connection with power for (bringing about) effects (kāryaśaktiparatayā) does not contradict the power of pradhāna being the only main power, though mentioned in the context of the power of ākāśa different from pradhāna; it should be understood as thus said.

Quest: Even then since the word ‘eka’ (one) has the meaning of being ‘not-two’ there is contradiction in the statement of pradhāna alone having the power since one understands from the Viṣ. P, that even jīva and kāla possess power. In that context it is said: “kāraṇatvena cākāśādiṣu yathāvyapadiṣṭokteḥ” (BS.I.4.14).\footnote{The discussion non centres on the statement in the Viṣṇu P. that even jīva and kāla possess power to create. So pradhāna alone (as stated by ``ekām" in the Śvet. Up) cannot posses that role.}

\textbf{\dev{कारणत्वेन चाकाशादिषु यथात्यपदिष्टोक्ते: ॥१४॥}}

\dev{आकाशादिषु संकल्पकार्येषु कारणत्वेन मूलप्रकृतित्वेनैव हेतुना यथाव्यपदिष्टस्याजामेकामिति वाक्यस्य प्रयोगादित्यर्थ: । सृजमानाया उपादान कारणस्यैवाजाया एकत्वमुक्तं, जीवकालौ च नाकाशाद्युपादानमतो नोपादानशक्तेरेकत्वविरोध इति भाव: । ननु तथापि अजाख्यप्रकृते: सत्त्वादिगुणत्रयरूपतयैकत्वसंख्याविरोध इति चेत्, न, एकस्मिन्नीश्वरस्य सत्त्वे अपरेषां गुणानां प्रलयेनैकत्वोपपत्ते: । तदुक्तं मोक्षधर्मे—“तदैका प्रकृतिर्भवेदि” ति । अथवा धर्मभेदेन पुरुषभेदेन च भेदाभाव एवाजाया एकत्वमिह विवक्षितमजाश्रुतिस्वरसात् । “इन्द्रो मायाभि: पुरुरूप ईयत” इति श्रुत्यविरोधाय चेति ॥}

\dev{ब्रह्मणोऽद्वितीया मूलशक्तिरिति प्रसाधितम् । इदानीं यदुक्तं “यस्मिन्  पञ्च पञ्च जना “ इति वाक्यस्य कार्यपरत्वे पञ्च पञ्चेत्याकाश इति च विशेषवचनप्र (म) युक्तमितरेषामपि ब्रह्मादिरूपकार्याणां ब्रह्मशक्तित्वात् ब्रह्मप्रतिष्ठितत्वाच्चेति, तत् समाधीयते–}

Ans: In the case of effects such as ākāśa etc., since being the cause is due to prakṛti having the property of being the main cause, there is the usage of the sentence “ajāmekām” (Śvet. Up. 4. 5) as already mentioned. The unborn material cause which creates, is alone mentioned as being one. As for the jīva and kāla (mentioned in the Viṣ.P) they are not the material cause of ākāśa etc., therefore there is no contradiction to the material cause being one (nopādānaśakterekatvavirodha iti bhāvaḥ).

Ques: Even so prakṛti, having the three qualities sattva etc, being called as ‘one’, is a contradiction; then the answer is:

Ans: That is not so; it is logical as in the one sattvaguṇa of Īśvara all the other qualities become one/dissolve (pralayenaikatvopapatteḥ). Thus it is said in the Mokṣadharama P: “tadekā prakṛtirbhavet” iti. Or due to difference in dharmas and due to difference in puruṣas, being one is absence of difference in ‘ajā’ which is desired to be said by the śruti in keeping with the sense (svarasena). It is also to avoid opposition to the śruti “indro māyābhiḥ pururūpa īyata” (Bṛ.Up. II.5.19). 

It has been established that the main śakti of Brahman is non-dual (brahmano’dvitīyā mūlaśaktiriti prasādhitam). Now the said statement “yasmin pañca pañca janā” being connected with effects, it is not correct to say pañca pañca janā denotes ākāśa especially, since other effects having the form of Brahman, and having have the power of Brahman are also rooted in Brahman. That is being reconciled by the statement: “samākarṣāt” (BS. I.4.15)

\textbf{\dev{समाकर्षात् ॥१५॥}}

\dev{मुख्याभ्यां प्राणादिपञ्चकाकाशाभ्यामितरकार्याणामप्युपलक्षणतया समाकर्षात् ग्रहणादुक्तार्थोपपत्तिरित्यर्थ: ॥}

\dev{कथं समाकर्षणं, तत्राह----}

“samākarṣāt”= by accepting (grahaṇāt) the principal five prāṇa etc., (and) the (five) ākāśa etc., as well as the other effects by implication (upalakṣaṇatayā), the said meaning can be logical. 

Ques: What kind of acceptance? Then the answer is:

Ans: “jagadvācitvāt” (BS. I.4.16)

\textbf{\dev{जगद्वाचित्वात् ॥१६॥}}

\dev{षष्ठ्यन्तानां वाक्यशेषस्थप्राणादिशब्दानामुपलक्षकतया जगद्वाचित्वात् समस्तकार्यवाचित्वादित्यर्थ: । एतेन परमात्मन: कार्यशक्तयोऽपि मीमांसिता इति मन्तव्यम् ॥ }

The words prāṇa etc., at the end of the incomplete sentence in the sixth case denote the world by implication since it denotes all effects (samastakāryavācitvāt).\footnote{This has reference to the previous “prāṇasya prāṇamuta...” mentioned under BS.I.4.12} One should know that by this, the powers of paramātman to bring about effects has also been discussed. 

\textbf{\dev{जीवमुख्यप्राणलिङ्गान्नेति चेत् तद्व्याख्यातम् ॥१७॥}}

\dev{ननु यदुक्तमजावाक्ये ब्रह्मशक्तिविधया प्रधानप्रतिपादनं, यच्चोक्तं “प्राणस्य प्राण” इति वाक्ये ब्रह्मशक्तितया प्राणादिप्रतिपादनं तदुभयमपि न, कुतो? जीवमुख्यप्राणलिङ्गात् । अयमर्थ:--- भवेदेवं यद्यजाप्राणवाक्ययोर्ब्रह्मप्रकरणस्थत्वं सिध्येत्, तदेव तु न, अजावाक्यवति श्वेताश्वतरे जीवलिङ्गं हि दृश्यते “रुद्र यत्ते दक्षिणं मुखं तेन मां पाहि नित्यमि” त्यादिवाक्यशेषेषु । नन्वेवंविधवाक्येभ्य एव रुद्र: परमेश्वर एव भवत्विति चेन्न, ईश्वरवादिभि: श्रुत्यन्तरेभ्य: स्मृतिभ्यश्च ब्रह्मादिदेवतात्रयादतिरिक्ततयैवेश्वराभ्युपगमात् । एवं प्राणवाक्यस्याप्यव्यवधानेनादौ प्राणलिङ्गं दृश्यते “तद्देवा ज्योतिषां ज्योतिरायुर्होपासतेऽमृतमि”त्यादिना । प्राणस्य चायुष्ट्वं श्रुत्यन्तरे दृष्टं “प्राणो हि भूतानामायुस्तस्मात् सर्वायुषमुच्यते” इति ।}

\dev{तस्मादजाख्या प्रकृति: सर्गाद्युत्पन्नस्य महापुरुषस्यैव शक्ति: न परमेश्वरस्य । तस्याश्च कर्मवशादेवाद्या प्रवृत्तिर्नाधिष्ठातारमपेक्षते । अत: स्वतन्त्रैव प्रकृति: सृष्ट्यादिकारणम् । एवं प्राणादयोऽपि मुख्यप्राणस्यैव शक्तिरिति । न च प्राणस्य प्राणत्वं कथं मुख्यप्रमाणे स्यादिति वाच्यम्, प्राणस्येत्यर्थ: संभवात् (प्राणस्येत्यस्य घ्राणस्येत्यर्थ संभवात्) । प्राणशब्दस्य घ्राणवाचकताया: श्रुत्यन्तरे दृष्टत्वादिति । तदित्थं पूर्वपक्षे सिद्धान्तमाह- इति चेत् तद् व्याख्यातमिति । तल्लिङ्गद्वयं ब्रह्मण्यपि व्याख्यातमुपपादितम् प्रथमपादे “उपासात्रैविध्यादाश्रितत्त्वादिह तद् योगादि”ति सूत्रार्धेनेत्यर्थ: । }

\textbf{Jīvamukhyaprāṇaliṅgānneti cet tadvyākhyātam}

Ques: The mention that in the sentence “ajā” pradhāna has been pointed out through (being) the power of Brahman and also the statement that in the sentence “prāṇasya prāṇa” one understands that it is through the power of Brahman (brahmaśaktitayā) that prāṇa etc., is understood, then both these statements are not right. Why?  Because prāṇa is the main mark/sign of the jīva (jīvamukhyaprāṇaliṅgāt). The meaning is as follows: Let it be established that the sentences “ajā” and “prāṇa” etc., are situated in Brahman, but that is not the case. 

In the Śvet.Up which has the statement “ajā...” (4.5) one only sees the sign (of the jīva) in the verse : “rudra yatte dakṣiṇam mukham tena mām pāhi nityam” (ibid.4.21) from the meaning (meaning which is unsaid but implied) (ityādi vākyaśeṣeṣu). If you say from such statements let Rudra himself be Parameśvara that is not correct. From the śrutis and smṛtis one knows that those who believe in Īśvara only accept Īśvara over and above the three Brahmā (Viṣṇu and Śiva) (brahmādidevatrayādatiriktatayaiveśvarābhyupagamāt). In this manner one observes in the sentence immediately following, i.e “taddevā jyotiṣām jyotirāyurhopāsate’mṛtamiti” (Bṛ.Up. 4.4.16) firstly the sign of prāṇa (ādau prāṇaliṅgam dṛśyate).\footnote{This is perhaps a reference to the verse “yasminpañca pañcajanā” etc., which immediately precedes this. The discussion is still concerned with the number 25 which is under dispute between the  advaitins and SY.} The association of prāṇa with the life of beings (prāṇasya cāyuṣtvam) is mentioned in another śruti as : “prāṇo hi bhūtānām āyuḥ tasmāt sarvāyuṣamucyate” (Taitt.Up. II.3.1).
 
Therefore that which is known as “ajā” (mentioned in the Śvet. Up) is prakṛti. Śakti belongs to that mahāpuruṣa which has evolved at the start of evolution (sargādyutpannasya mahāpuruṣasyaiva) and not to Mahēśvara. The initial activity of that (prakṛti) is due to karma alone and it does not depend on the one who supports it (i.e.Īśvara). Therefore prakṛti is independently the originator of creation. Similarly prāṇa (and the other four apāna etc.,) are also the power of the main prāṇa.

Ques: It cannot be asked how can the essence (the generic property) of prāṇa be present in the main prāṇa?\footnote{According to Nyāya philosophy, if there is only one entity it cannot have a generic property (sāmānyaguṇa).}

\newpage

Ans: The meaning of prāṇasya can be ‘smelling. This is possible as it is seen that prāṇa denotes smelling in other śruti texts (see Tripāthi.p.136. fn.1).

Thus, if it is said that the prima facie doctrine is stated then the answer is that it is only interpreted (in this manner). The two marks/signs have also been mentioned in Brahman in the first section (pāda) through the half verse (ślokārdhena) as: “upāsātraividhyādāśritatvādiha tadyogāt”.

\dev{अयं भाव:---- श्वेताश्वतरे परमात्मैव महाप्रकरणो “यो ब्रह्माणं विदधाति पूर्वं यो व वेदांश्च प्रहिणोति तस्मै”; “क्षरं प्रधानममृताक्षरं हर: क्षरात्मानावीशते देव एक” इत्यादि वाक्येभ्यो ब्रह्मारुद्राद्यतिरिक्त: । तस्मात्तदन्तर्यामिण एव प्रकरणित्वात् । अतो अजाप्रकरणे यज्जीवलिङ्गमस्ति तत्प्रकरणानुरोधेन शक्तिशक्तिमदभेदेनोपासनार्थं ब्रह्मण्येव बोध्यम्, “अनेन जीवेनानुप्रविश्ये” ति श्रुत्या ब्रह्मणो जीवरूपित्वसिद्धे: । “स एव जीव: सुखदु:खभोक्ते” त्यादिषु स्फुटं तथा वचनाच्च । अन्यथा हि प्रकृतहान्यप्रकृतप्रक्रिये प्रसज्येयातामिति । एवं प्राणवाक्येऽपि यन्मुखप्राणलिङ्गमस्ति, तदपि प्रकरणानुरोधेन प्राणशक्तिद्वारा ब्रह्मण्येव बोध्यम् । }

\begin{verse}
\dev{नृप मूर्तान्यमूर्तानि यान्यत्रान्यत्र वा क्वचित् ।}\\
\dev{सन्ति वै वस्तुजातानि तानि सर्वाणि तद्वपु: ॥}
\end{verse}

\dev{इत्यादि वाक्येषु ब्रह्मण: प्राणादिरूपित्वसिद्धेरिति । अस्मिन् सूत्रे पूर्वपक्षनिराकरणमुखेनैव जीवस्यापि ब्रह्मशक्तित्वमुक्तमतो न शक्तिमीमांसान्यूनतेति ॥}

This is the idea: In the Śvet.Up paramātmā alone is the main topic (mahāprakaraṇī). (In such statements as) “yo brahmāṇam vidadhāti pūrvam yo va vedāmśca prahiṇoti  tasmai”\footnote{This is the first line of Śvet.Up. 6.18.}; “kṣaram pradhānamamṛtākṣaram haraḥ kṣarātmānāvīśate deva eka”\footnote{“kṣaram pradhānam...” is  the first line of  Śvet.Up.1.10} one understands that it is different from Brahmā, Rudra etc. Thus the topic is just the antaryāmin. Therefore in the section/topic under “ajā”, in keeping with the topic at hand, the jīva-symbol is to be understood as meant for devotion on Brahman characterized as the undivided śakti and possessor of śakti (yajjīvaliṅgamasti tatprakaraṇānurodhenopāsānārtham śaktiśaktimadabhedena brahmaṇyeva bodhyam). By the statement : “anena jīvenānupraviśya”. (Chānd. Up.6.3.2) the jīva form of Brahman is established. It is also clearly mentioned in statements like: “sa eva jīvaḥ sukhaduḥkhabhoktā”.\footnote{Kaiv.Up. 13” has “svapne tu jīvas sukha duhkha bhoktā...”} Otherwise it can hurt the topic under discussion and will give rise to the contingency of attributing action to something not under discussion.  In this way in the sentence “prāṇa” also dealing with that which is the main prāṇa as a symbol, is to be understood as in Brahman alone, through the power of prāṇa, in keeping with the topic being discussed. In such statements as : “nṛpa mūrtānyamūrtāni yānyatrānyatra vā kvacit...tāni sarvāṇi tadvapuḥ” the form of Brahman as prāṇa is established. In this sūtra jīva being the power of Brahman is mentioned with the idea of refuting the view of the opponent (pūrvapakṣanirākaraṇamukhenaiva jīvasyāpi brahma uktam); therefore there is no deficiency in the discussion of power.\footnote{This whole discussion on “pañca pañca janāḥ” is so contrived both from the advaitin Śaṅkara  and the avibhāga-advaitin Bhikṣu.}

%\textbf{\dev{अन्यार्थं तु जैमिनि: प्रश्नव्याख्यानाभ्यामपि चैवमेके ॥१८॥}}

%\dev{जैमिनिस्त्वाचार्य उपासनार्थं न मन्यते । किं तु परमसूक्ष्मत्वेन स्वतो दुष्प्रतिपाद्यतया क्रमेण ब्रह्मस्वरूपप्रतिपादनार्थं शाखाचन्द्रन्यायतो जीवमुख्यप्राणादिशक्तिरूपेणादौ सर्वान्तर्यामि ब्रह्म प्रतिपादयितुं ब्रह्मप्रकरणेषु जीवमुख्यप्राणादिकं प्रतिपाद्यत इति मन्यते, कुत: ? प्रश्नप्रतिवचनाभ्यां प्रश्नप्रत्युत्तराभ्यां ब्रह्मप्रकरणित्वादित्यर्थ: ।}

%\dev{अपि चेत्यादि । अपि च एके शाखिन: काण्वादय: एवम् अन्यार्थमेवातिस्पष्टं जीवादिकं ब्रह्मप्रकरणे पठन्ति इत्यर्थ: । यथा काण्वे चतुर्थे “सहोवाचा जातशत्रुर्यत्रैव एतत् सुप्तोऽभूद् य एष विज्ञानमयस्तदैषां प्राणानां विज्ञानेन विज्ञानमादाये”त्यादिना जीवतदवस्थादिपठनं तत्स्वापाधारब्रह्मप्रतिपादनार्थमिति “ स एष इह प्रविष्ट आनखाग्रेभ्यस्तं न पश्यत्यकृत्स्नो हि स इत्, द्वे वाव ब्रह्मणो रूपे मूर्तं चामूर्तं च, अथात आदेशो नेति नेतीति ॥}

%\dev{अपि चैवमेक इत्यत्र हेतुमाह ------}

%\textbf{BS. I.4.18}

%\textbf{anyārtham tu jaiminiḥ praśnavyākhyābhyāmapi caivameke}

%Jaimini ācārya  does not consider this is for the sake of devotion;  but by great subtlety pointing to its own defect, gradually following the maxim of ‘śākhācandranyāya’\footnote{The maxim of the bough and the moon. When a faraway object is spoken of as near by connecting it with a close object like the expression “the moon on the bough”.   (Also see V.S.Apte’s The Practical Sanskrit English Dictionary under nyāyas)} in order to gradually point out the real nature of Brahman, to begin with it  points to the inner self Brahman (sarvāntaryāmi brahma) as the form of jīva, main prāṇa etc in the form of powers (śaktirūpeṇa) in the sections devoted to the topic of Brahman (brahmaprakaraṇeṣu). Why is this so? Because through questions and discussions (and) through questions and answers Brahman has been discussed/explained.  

%‘api ca eke’\footnote{“caivameke” in the sūtra “anyārtham...”}=Also some like those who follow the Kāṇva branch read  “aṇvārtham”= as this is very clearly jīva etc.,which is discussed under the topic Brahman. Just as in the fourth Kāṇva reading: “sahovājātaśatruryatraiṣa etat supto’bhūd ya eṣa...vijñānena vijñanamādāya” one understands that it stands for the  jīva and its states and  is for the sake of showing that Brahman is its (jīva’s) support in the state of dreaming. “sa eṣa iha ānakhāgrebhyaḥ...akṛtsno hi saḥ” (not traced), “dve vāva brahmaṇo rūpe mūrtam cāmūrtam ca” (Bṛ.Up. 2.3.1; Mait.Up. 6.3); “athāta ādeśo neti neti” iti (Bṛ.Up. 2.3.6).\footnote{This is part of a larger quotation.}

%The reason for the sentence/phrase “api caivameke” (in sūtra I.4.18) is mentioned as- “vākyānvayāt” (BS.I.4.19)

%\textbf{BS.I.4.19}

%\textbf{\dev{वाक्यान्वयात् ॥१९॥}}

%\dev{विधिनिषेधरूपपूर्वापरवाक्यैकवाक्यतया काण्वादिष्वन्यार्थत्वावधारणमित्यर्थ: । शब्दाभावाच्च नोपासापरत्वमिति भाव: ।}

%\textbf{BS.I.4.19}

%\textbf{vākyānvayāt}

%Through consistency in meaning of earlier and later sentences which are in the form of prescriptions and in the form of prohibitions (of actions) one should understand a different  meaning in the Kāṇva recension (vidhiniṣedharūpapūrvāpara-vākyaikavākyatayā kāṇvādiṣanyārthatvāvadhāraṇam). As there is the absence of any word (for devotion) it is not connected to upāsanā (devotion).

%\textbf{\dev{प्रतिज्ञासिद्धेर्लिङ्गमित्याश्मरथ्य: ॥२०॥}}

%\dev{आश्मरथ्यरूपाचार्यो यथोक्तं लिङ्गद्वयं जीवप्राणयोरिव ब्रह्मणोऽपि साधारण्यमेव मन्यते, कुत:? प्रतिज्ञया ब्रह्मणि लिङ्गद्वयस्य सिद्धत्वात् । अजावाक्यादौ तावत् “ य एको वर्णो बहुधा शक्तियोगाद् वर्णाननेकान् निहितान्नो दधाति । विचैति चास्ते विश्वमादौ स देव: स नो बुद्ध्या शुभया संयुनक्तु ॥ तदेवाग्निस्तदादितयस्तदु चन्द्रमा: तदेव शुक्रं तद् ब्रह्म तदापस्तत् प्रजापतिरि” ति प्रतिज्ञया कार्यकारणभेदेन सर्वे जीवादिधर्मा: पञ्चमुखत्वादयोऽपि ब्रह्मणि सिद्धा: । प्राणवाक्यस्य च शेषे “मनसैवानुद्रष्टव्यं नेह नानास्ति किंचन मृत्यो: स मृत्युमाप्नोति य इह नानेव पश्यति” इति प्रतिज्ञया सर्वस्यात्मनि कारणे च ब्रह्मणि सर्वे प्राणादिधर्मा: परमायुष्ट्वादयोऽपि सिद्धा इत्यर्थ: । व्यासमतोच्चास्य मतस्यायं विशेषो यद् व्यासाचार्यै: शक्तिशक्तिमदभेदोपासनापरत्वेन लिङ्गद्वयमुपपादितम् ब्रह्मणि, आश्मरथ्याचार्यै स्तूपासनां विनापि कार्यकारणभेदेन कार्यधर्म इति लिङ्गद्वयमुपपादितमिति ॥}

%Just as ācāryas like Āśmarathya have mentioned two signs (for jīva), so also it is common for Brahman also to have that just like jīva and prāṇa. Why is that so? Since it is so asserted (pratijñayā) the two signs of Brahman is established. At the beginning of the sentence “ajā” by stating: “ya eko’varṇo bahudhā śaktiyogād varṇānanekān...sa no buddhyā śubhayā saṃyunaktu” (Śvet.Up.4.1); “tadevāgnistatadātyatastadu...tad brahma tadāpastat prajāpatiḥ” (ibid.4.2) by asserting that all dharmas such as the jīvas without any difference as cause and effect, as also the five-faced Brahmā etc., are all established in Brahman. After the sentence dealing with prāṇa by asserting that : “manasaivānudraṣṭavyam neha nānāsti kiñcana...ya iha nāneva paśyati” (Bṛ.Up.4.4.19) it is known that everything including the dharmas prāṇa etc., (prāṇādidharmāḥ) and the great eight lives are all established in Brahman, the ātman in all beings and the cause. The difference between the view of Vyāsa (the author of the BS) and this view is that Vyāsa connecting devotion to (Brahman) being identical with the one possessing the power and  the power (śaktiśaktimadabhedopāsanāparatvena) has pointed the two marks (liṅgadvayamupapāditam) in Brahman whereas, even without devotion through an identity between the cause and the effect  the two marks have been indicated by ācārya Āśmarathya.

%\textbf{\dev{उत्क्रमिष्यत एवं भावादित्यौडुलोमि: ॥२१॥}}

%\dev{औडुलोमिस्त्वाचार्यो मन्यते परमात्मन्यपि जीवप्राणयोर्लिङ्गमुपपद्यते उत्क्रमणारम्भकाले जीवप्राणयोरेवंभावात् परमात्मभावात् यथा मैथुनकाले स्त्रीपुंसयोरेकीभाव इत्यर्थ: । तथा च मरणकाले परमात्माविभागादन्यदापि तदभेदव्यवहारेण जीवप्राणधर्मौ ब्रह्मण्युच्येते, लोके तथा व्यवहारादिति भाव: । “अथास्य पुरुषस्य प्रयतो वाङ्मनसि संपद्यते मन: प्राणे प्राणस्तेजसि तेज: परस्यां देवतायामि” त्यादि श्रुतयश्च जीवप्राणयोर्ब्रह्मणि संपत्याख्यमैक्यं बोधयन्ति ॥}
 
%\textbf{BS.I.4.21}

%\textbf{Utkramiṣyata evam bhāvādityauḍulomiḥ}

%Ācārya Auḍulomi thinks that it is proper that the marks/signs of jīvas and prāṇa are present in the supreme ātman; at the time of the rise of creation since and prāṇa are in this state  there is the presence of the supreme being.; it is like the identity of a woman and man at the time of intercourse.  Similarly at the time of death, due to non-separation from the paramātman and otherwise due to functioning without division there is the presence of the qualities of jīva and prāṇa in Brahman. The idea is that is so at the time of worldly activity. “athāsya puruṣasya prayato vāṅmanasi...tejaḥ parasyām devatāyām” such śruti statements also (not traced) proclaim/teach the identity known as abundance (saṃpattyākhyamaikyam) of jīva and prāṇa in Brahman. 

\textbf{\dev{अवस्थितेरिति काशकृत्स्न: ॥ २२ ॥}}

\dev{काशकृत्स्नस्त्वाचार्यो मन्यते, प्रलयकाले जीवादीनां सर्वेषामेव ब्रह्मण्यवस्थितेर्विलयादत्यन्तमेकी - भावेनान्यदापि तदभेदस्यैव पारमार्थिकत्वम्, जीवब्रह्मविभागस्य नैमित्तिकत्वेन स्वाभिकाविभागापेक्षया वाचारम्भणमात्रत्वात् । अतो यथोक्तलिङ्गत्रयं ब्रह्मण्युपपद्यत इति । औडुलोमिमताश्चास्यायं विशेषो यदौडुलोमिना स्त्रीपुरुषयोरिव मिथुनीभावरूपं मरणप्रारम्भ कालीनमैक्यमादाय लिङ्गद्वयमुपपादितम् । काशकृत्स्नेन च नदीसमुद्रयोरिवैकतरस्यात्यन्त विलयत: प्रलयकालीनमैक्यमादाय लिङ्गद्वयमुपपादितमिति । यथा च सुषुप्तिमरणप्रारम्भयोरीषत् ज्ञानं प्रलयव्यावृत्तं तिष्ठति जीवानामीश्वरे संसृष्टानां तथा वक्ष्यामो जीवावस्थाप्रकरणे । श्रुती चात्र भवत: “त्रिषु धामसु यद्भोग्यमि” ति, “हृदयस्याग्रं प्रद्योतते तेन निष्क्रामति” ति च । तस्माद् युक्त उभयोर्मत भेद इति । एते च मतभेदा: सर्व एवाविरुद्धा: श्रुतिसिद्धाश्च भवन्ति प्रकृतपूर्वपक्षसमाधानाय केवलं मुनिभि: पृथक् पृथगुपन्यस्ता इति बोध्यम् ॥}

\dev{इदमिदानीं विचार्यते--- यदिदमानुमानिकं प्रधानमीश्वर ’शक्तिविधया’ विश्वस्रष्टृ प्रोक्तम्, किमेतदेव राज्ञो मन्त्रिवज्जीवस्यान्त:करणवश्च परमात्मन: संकल्पाद्याधारभूता स्वभावाख्या प्रकृतिरन्तरङ्गशक्ति:? अथ वेदं शरीरस्थानीया बहि:शक्तिरेव एतदधिष्ठानार्थं शक्त्यन्तरमस्तीश्वरस्य ? किं वा शक्त्यन्तरनैरपेक्ष्येणैव नैयायिकानामिव स्वत एवेश्वर: प्रधानपुरुषादिकमधितिष्ठतीति? तत्राद्यं पक्षं सिद्धान्तयति------}

\textbf{BS.I.4.22}

\textbf{Avasthiteriti Kāśakṛtsnaḥ}

%Ācārya Kāśakṛtsna thinks that since during the time of dissolution jīvas and everything is situated in Brahman, there is total identity due to the dissolution at other times as well that identity is the absolute   truth. With reference to the natural non-separation,  the separation of the jīva and Brahman because of (Brahman) being the efficient cause,   is only a name/ modification arising from speech (jīvabrahmavibhāggasya naimittikatvena svābhāvikāvibhāgāpekṣayā vācāraṃbhaṇamāratvāt). Thus the above mentioned three marks is proper in Brahman. The difference between the view of Auḍulomi and this is that by taking identity of the nature of intercourse in the form of a woman and man which starts at the time of death he mentions two marks. Kāśakṛtsna, on the other hand, using the example of the ocean and a river for identity at the time of dissolution, wherein one (the river, in this case jīva and prāṇa) is totally immersed, has mentioned two marks/signs. Just as at the time of deep sleep and at the beginning stage of death a little knowledge stands separated from the deluge, it is like that in the case of the jīvas attached to Īśvara which we  will mention in the section dealing with the state of the jīva (vakṣyāmo jīvāvasthāprakaraṇe). The śruti statements in this context are: “triṣu dhāmasu yadbhogyam...” (Kaiv.Up.18.)\footnote{This is just the beginning of that quote. The full verse is: “triṣu dhāmasu yad bhogyam bhoktā bhogaśca yad bhavet, tebhyo vilakṣaṇaḥ sākṣī cinmātro’ham sadāśivaḥ”}; “hṛdayasyāgram pradyotate tena niṣkrāmati” (not traced). Thus the different views of both are logical. All these different view points are mutually non-contradictory and are established by śruti. One should know that these have been stated each as separate (prakṛtapūrvapakṣasamādhānāya) by the sages, in order to reconcile the opponents to the subject in hand.

We will now discuss this: Pradhāna, due to being the power of Īśvara through inference, is mentioned as the creator of the world (recalling sūtra I.4.1). Is this similar to the minister using (the support) for the king, as the internal organ (the support) for the jīva, being by being the support for the decisions etc., of paramātman by prakṛti called natural inclination (kimetadeva rājño mantrivajjīvasyāntaḥkaraṇavacca paramātmanaḥ saṃkalpādyādhārabhūtā svabhākhyā prakṛtirantaraṅgaśaktiḥ). Or is this an external power situated in the body of Īśvara alone and is a separate power of Īśvara (athavedam śarīrasthānīyā bahiḥśaktireva etadadhiṣṭhānārtham śaktyantantaramastiśvarasya). Or is it like Īśvara by itself, without dependence on any other power, similar to (Īśvara) mentioned by the Nayāyikas ruling over pradhāna and puruṣa (kim vā śaktyantaranairapekṣyeṇaiva naiyāyikānāmiva svata eveśvaraḥ pradhādhānapuruṣādikam adhitiṣṭati iti?)\footnote{The last option is attributing Īśvara’s power independently to influence both puruṣa and prakṛti in the process of creation.}

\textbf{\dev{प्रकृतिश्च प्रतिज्ञादृष्टान्तानुपरोधात् ॥२३॥}}

\begin{verse}
\dev{मायां तु प्रकृतिं विद्यान्मायिनन्तु महेश्वरम् ।}\\
\dev{अस्यावयभूतैस्तु व्याप्तं सर्वमिदं जगत् ॥}
\end{verse}

\dev{इति प्रतिज्ञाया: , “तन्तुनाभ इव तन्तुभि: प्रधानजै: स्वभावत: देव एक: समावृणोति स नो दधातु ब्रह्माव्ययमि” ति दृष्टान्तस्य च वाक्यशेषस्थयोरनुपरोधार्थमाञ्जस्येनोपपत्यर्थमित्यर्थ: । प्रकृतित ईश्वरोपाधेरत्यन्त भेदे तथेश्वरस्य निरुपाधित्वे च हि तयो: पीडा स्यादिति । विष्णुपुराणादौ त्वपरिणामित्व रूपवैधर्म्येणेश्वरोपाधि: प्रधानादतिरिक्त एवोक्त: । अत्र तु समानतन्त्रपातञ्जलसिद्धान्तत्वात् “आनन्दादय: प्रधानस्ये” त्यागामिसूत्राच्च ईश्वरोपाधे: प्रधाने प्रवेश एव सूत्रकाराभिप्रेतो जडत्वसाधर्म्येणेत्यवगम्यत इति। }

\dev{यद्यपि “स्वभावस्तु प्रवर्तते, कार्यकारणकर्तृत्वे हेतु: प्रकृतिरुच्यत” इत्यादिप्रयोगात् प्रकृतिशब्दस्य स्वभावकारणोभयवाचित्वं, तथाप्यत्र प्रकृति: स्वभाव एव बोध्य: । “आनुमानिकमप्येकेषामि” त्यादिना सूत्रेणैव परमात्मशेषतया जगत्कारणत्वस्य प्रधाने सिद्धत्वेन “ पुनस्तत्प्रतिपादनवैयर्थ्यात् तथा “आकाशस्तल्लिङ्गादि” त्यादि सूत्रवदिदं सूत्रं न प्रकृत्याख्यशक्तिरूपेण ब्रह्ममीमांसकमपि भवति “सर्वत्र प्रसिद्धोपदेशादि” ति सामान्यसूत्रेणैव गतार्थत्वात् तथा सति पूर्वपादत्रय एवैतदधिकरणौचित्याच्चेति ।}

\newpage

%raghu, page 33

\textbf{BS.I.4.23}

\textbf{prakṛtiśca pratijñādṛṣṭāntānuparodhāt}

The  declaration: “māyām tu prakṛtim vidyānmāyinantu maheśvaram, asyāvayavabhūtaistu vyāptam sarvamidam jagat” (Śvet.Up. 4.10) and the two similes “tantunābha iva tantubhiḥ pradhānjaiḥ...sa no dadhātu brahmāvyayam” implied by the sentence fit in completely without any impediment the logical meaning.\footnote{The two similies being ‘the navel (of the spider) being composed of the  threads  of the web and Īśvara being covered naturally by that born from pradhāna} Since the limitation of Īśvara is totally different from prakṛti and Īśvara being without any limitation there is some opposition (pīḍā) between them. In the Viṣṇu P.etc., the limitation of Īśvara is mentioned as different from prakṛti due to its difference in the form of not having any change (tvapariṇāmatvarūpavaidharmyeṇeśvaropādhiḥ pradhānātirikta eva proktaḥ).  Since in the similar philosophy of Patañjali’s Yoga  (samānatantrapātañjalasiddhāntattvāt) and by the upcoming sūtra “ānandādayaḥ pradhānasya”(BS.3.3.11) one knows that it is the desire of the author of the sūtras (Bādarāyaṇa) to include Īśvara’s limitation in pradhāna due to its similarity of being insentient.

Even though by such usages as “svabhāvastu pravartte” (Gītā.5.14), “kāryakāraṇakartṛtve hetuḥ prakṛtirucyate” (ibid. 13.20) prakṛti has been mentioned as the cause for both one’s own nature as well as the cause, still in this context prakṛti I to be understood as  one’s owm nature itself (atra prakṛtiḥ svabhāva eva bodhyaḥ). By the sūtra “ānumā\-nikamapyekeṣām” (BS.I.4.1) itself pradhāna as the implied remainder of paramātman being the cause for the world is established sūtreṇaiva paramātmaśeṣatayā jagatkāraṇatvasya pradhāne siddhatvena) it is futile to repeat it again.  Thus similar to sūtras like “ākāśāstalliṅgāt” (BS.I.1.22) this sūtra also is reflective of Brahman through its power called ‘prakṛti’ (prakṛtyākhyaśaktirūpeṇa); this is obtained by the common sūtra “sarvatra prasiddhopadeśāt”(BS.I.2.1). That being so it\break would be appropriate for this section to be included in the previous three pādas.

\dev{तथा साक्षाद् ब्रह्मणो जगत्प्रकृतित्वमसि नास्य सूत्रस्यार्थ: , अस्मिन् पादे शक्तेरेव प्रकृतत्वात्,”जन्माद्यस्य यत:, तत्तु समन्वयात्, तदधोनत्वादर्थवत्” इत्यादि सूत्रैरेवाघिष्ठानकारणत्वस्य ब्रह्मणि लब्धत्वाच्च । स्वत: प्रकृतित्वस्य तु निर्विकारश्रुति स्मृतिविरुद्धत्वात् ।}

\newpage

\dev{ननु विकाराणां विवर्तत्वेनायं दोष: परिहर्तव्य इति चेन्न , यत “आत्मकृते: परिणामादि” त्यागामिसूत्रादेवात्रापि दर्शने औपाधिकपरिणाम इति चेन्न, यत् आत्मकृते: परिणामादि” त्यागामिसूत्रादेवात्रापि दर्शने औपाधिकपरिणाम एव सिद्धान्त:, विवर्तप्रतिज्ञाय़ां सूत्राभावाच्च । किं च विकाराभ्युपगमदशायामेव प्रकृतेरेव विकारित्वं नात्मन इति सकलश्रुतिस्मृतिष्ववगम्यत इति । तथा च गीता ------}

\begin{verse}
\dev{न कर्तृत्वं न कर्माणि लोकस्य सृजति प्रभु: ।}\\
\dev{न कर्मफलसंयोगं स्वभावस्तु प्रवर्तते ॥ इति ।}
\end{verse}

\dev{स्वीयो भाव: पदार्थ उपाधिरित्यर्थ: । उपाध्युपाधिमतोश्चानादि: स्वस्वामिभाव इति। यदि चात्र सूत्रे ब्रह्मण्: साक्षात् प्रकृतित्वमेवार्थं इष्यते तर्ह्यत्र प्रकृतित्वमधिष्टानकारणत्वं यदि वा मूलविकारित्वमेवात्र प्रकृतित्वं तर्हि आकाशादिरूपं तावद् ब्रह्मण: प्रकृतिरूपत्वमेवात्रार्थोऽस्तु शक्तिशक्तिमदभेदात्। तदुक्तं विष्णुपुराणादिषु “ स एव मूलप्रकृतिर्व्यक्तरूपी जगच्च स” इत्यादिरिति दिक् ।}

\dev{ईश्वरस्य उपाधिरूपा प्रकृतिरस्तीत्यत्र हेत्वन्तरमाह} $\underline{\qquad}$

Thus the meaning of this sūtra is not that Brahman directly has the nature of prakṛti (sākṣād brahmaṇo jagatprakṛtitvamapi nāsya sūtrasyārthaḥ), as in this pāda śakti is the topic (prakṛtatvāt) and through the sūtras “janmādyasya yataḥ” (BS.I.1.2), “tattu samanvayāt” (ibid.I.1.4), “tadadḥīnatvādarthavat” (ibid. I.4.3) themselves one understands that Brahman is the supporting cause. If (Brahman) is itself of the nature of prakṛti then it will contradict the śruti and smṛti statements that mention (Brahman) to be without change.

Ques: If it is said that this question of change can be removed/rectfied  through vivarta (the apparent appearance) then the answer is:

Ans: No (it cannot be done so) because of the coming sūtra alone“ātmakṛteḥ pariṇāmāt” (BS.I.4.26) here also in this philosophy (Vedānta philosophy) change due to the limitation is the doctrine (aupadhikapriṇāma eva siddhāntaḥ) as there is also an absence of any sūtra declaring vivarta. Moreover one learns from all the śrutis and smṛtis that at the stage of undergoing change itself the change is of prakṛti alone and not of the ātman. Thus the Gītā says: “na kartṛtvam na karmāṇi lokasya sṛjati prabhuḥ, na karmaphalasaṃyogam svabhāvastu pravartate”(5.14). A matter in its own form (svīyo bhāvaḥ padārthaḥ) limitation is its limitation. And the limitation and the one  having the limitation has a beginningless state of (relation of)being the possessor and the possessed  (upādhyupādhimatoścānādiḥ svasvāmibhāva iti). If one desires to have the meaning of Brahman having the nature of prakṛti directly then it will be that the form of prakṛti is the supporting cause; if on the other hand the form of prakṛti is, the form of the main change, then the change into the form of ākāśa etc., then the meaning will be that the form of prakṛti is (of Brahman) due to non-difference/identity between the power and the one possessing the power. Thus it is mentioned in Viṣṇu Purāṇa etc., that : “sa eva mūlaprakṛtirvyaktarūpī jagacca sa”, this is the general direction.

He mentions another reason as to why Īśvara has a limitation in the form of prakṛti as: 

\textbf{\dev{अभिध्योपदेशाच्च ॥२४॥}}

\dev{अभिध्या संकल्प:, तदुपदेशादपि परमात्मन: स्वभाव: सिध्यति । ईश्वरस्य निरुपाधित्वे “बहुस्यामि” त्यादि संकल्पोपदेशानुपपत्तेरित्यर्थ: । “आह च तन्मात्रमि” त्यादिसूत्रेषु ब्रह्मणश्चिन्मात्रस्वरूपताया वक्ष्यमाणत्वादिति भाव: ॥}

\dev{ननु यदिदमानुमानिकं प्रधानमीश्वरोपाधिरपीत्युच्यते, किमिदं जीवस्येव विकारभावापन्नं शुद्धसत्वाख्यं सदीश्वरस्योपाधिर्भवति, किं वा साक्षादेवेति संशये निर्णयमाह} $\underline{\qquad}$

\textbf{BS.I.4.24}

\textbf{Abhidhyopadeśācca}

“abhidhyā”= decision, “tadupadeśāt”\footnote{Recalling the sutra BS.I.2.1}= (through the teaching of Brahman being present everywhere) the nature of paramātman is established. When there is the absence of any limitation of Īśvara the statement “bahusyām” (Chānd.Up) (which indicates) a decision  (saṃkalpa) becomes illogical.\footnote{If there is no limitation, Brahman will not be able to take the decision of “I may become many”.} In such  the sūtras as “āha ca tanmātram” (BS.3.2.16) it also will mention that Brahman is of the nature of pure consciousness alone.\footnote{This is also in contradiction to having the samkalpa to become many (bahusyām)}

Ques: Even if it is admitted that this pradhāna that is inferred is also the limitation of Īśvara then does it mean that like (in the case of) the jīva having some change/transformation that called the pure sattva (śuddha sattva) becomes the limitation of Īśvara, or does it directly (become its limitation); with this doubt he gives the decision as : 

\textbf{\dev{साक्षाच्चोभयाम्नानात्॥२५॥}}

\dev{तदानुमानिकं न केवलं प्रकृति:, अपि तु साक्षादपि विकारभावानापन्नमेव नित्यशुद्धसत्त्वांशेन ईश्वरस्य प्रकृतिर्भवति, कुत:? उभयाम्नानात् । उत्तरसूत्रेण वक्ष्यमाणयोरुभयो: कारणकार्यभावापन्नयो: कृतिपरिणमयोस्तत्राम्नानादि\-त्यर्थ:॥}

\dev{उभयाम्नानमेवाह}

\textbf{\dev{आत्मकृते: परिणामात् ॥२६॥}}

\dev{“तदात्मानं स्वयमकुरुते” ति श्रुत्या स्वकृतित एवेश्वरस्य सर्वपरिणमावगमादित्यर्थ: । तथा चेश्वरस्य कार्योपाधित्वे तत्कृतित: समस्तपरिणामो नोपपद्यत, स्वोपाधे: पूर्वं स्वकृत्यभावेन उपाधिरूपपरिणामस्य स्वकृति जन्यत्वासंभवात् । अत ईश्वरोपाधिर्नित्य एवेति ॥}

\dev{तस्यैवानुमानिकस्यापरं विशेषमाह} $\underline{\qquad}$

\textbf{BS.I.4.26}

\textbf{Ātmakṛteḥ pariṇāmāt}

By the śruti statement “tadātmānam svayamakuruta” (Taitt. Up. 2.7) one learns that every change (all changes) is brought about by itself by Īśvara. Then it is not logical to say that through possessing the limitation with the capacity of bringing about the effects, through its action (the action of the upādhi) all changes takes place tathā ceśvarasya kāryopādhitve tatkṛtitaḥ samastapariṇāmo nopapadyeta).\footnote{Since Īśvara brings about all changes that cannot be attributed to Īśvara with the limitation since the limitation also is included in ‘everything’.} As there is absence of action by itself (by Īśvara) before possessing its (one’s) own limitation, the change in the form of a limitation is not possible by the action of oneself.\footnote{In other words there is no way that Īśvara, being without change, can bring about the limitation-change  as well. It needs to be there permanently is Bhikṣu’s contention. Thus Bhikṣu proves that Īśvara has his śuddha sattva limitation permanently.} Therefore the limitation of Īśvara is permamnent (ata īśvaropādhirnitya eveti).

Another special feature of that inference is:

\textbf{\dev{योनिश्च हि गीयते ॥२७॥}}

\dev{तदानुमानिकमीश्वरस्य योनिरपि जीवाख्यगर्भाधानस्थानमपि अंशभेदेन भवति । हि यस्मात् तथा गीयते “ यो योनिं योनिमधितिष्ठत्येक” इत्यादि श्रुतिभि:,}

\begin{verse}
\dev{मम योनिर्महद् ब्रह्म तस्मिन् गर्भं दधाम्यहम् ।}\\
\dev{संभव: सर्वभूतानां ततो भवति भारत ॥}
\end{verse}

\dev{इत्यादि स्मृतिभिश्चेत्यर्थ: । अजाप्रकरणस्थत्वात् योनिं योनिमिति वीप्सा अंशभेदेन बहुत्वमभिप्रेत्येति बोध्यम्, महद्ब्रह्मेति महतो ब्रह्मेत्यर्थ्स्तत्कारणत्वात् । तदयं सिद्धान्त:$\underline{\qquad}$ प्रकृतिर्नित्यशुद्धकेवलस्त्त्वांशेन नित्यज्ञानेच्छादिमतीश्वरोपाधि:, सैव च प्रकृतिर्मलिनसत्त्वविशेषरूपैरंशान्तरे रजस्तम:संभिन्नै: पुरुषसंयोगेन महत्तत्त्वरूपत: परिणता सती जीवो य(जीवीय) उपाधिर्भवति” कार्योपाधिरयं जीव: कारणोपाधिरीश्वर:” इति श्रुते: । (ततोऽपि निकृष्टैरंशान्तरै: सर्वमन्यद् विकारजातमुत्पादयतीति ) विभाग इति ॥}

\textbf{BS.I.4.27}

\textbf{Yoniśca hi gīyate}

Thus one infers that the womb of Īśvara is also the place where the foetus called the jīva is contained through a partial division (īśvarasya yonirapi jīvākhyagarbhādhānasthānamapi amśabhedena bhavati). Thus it is stated in the śrutis: “yo yonim yonimadhitiṣṭhatye...” (Śvet.Up. 4.11; 5.2); this is also stated in smṛtis as: “mamayonirmahad brahma tasmin garbham dadhāmyaham, sambhavaḥ sarvabhūtānām tato bhavati bhārata” (Gītā.14.3). Since it is placed in the section on “ajā” (Śvet.\-Up.4.5)\footnote{The verse “yonim,yonim...” occurs in the same chapter as that of “ajām ekām lohitaśuklakṛṣṇām...” in the Śvet.Up and that is probably the reference point.} the repetition of the word “yonim, yonim” is with the idea of expressing multiple divisions, it should be so understood. “mahad brahma” in the above Gītā verse denotes big since it is the cause. Thus the doctrine is as follows: prakṛti in its eternal pure sattva alone is the limitation of Īśvara having eternal knowledge and desire; that same prakṛti along with tainted sattva in the form of other parts mixed with rajas and tamas by contact with puruṣa changing into the form of the mahat sattva\footnote{This is the first principle to evolve from the coming together of puruṣa and prakṛti in SY} is the jīva i.e. is the limitation of jīva. This is in accordance with the śruti statement: “kāryopādhirayam jīvaḥ kāraṇopādhīśvaraḥ” (Tri.M.Nā.Up4.8; Śu.R.3.12). And hence  through those lower changes close all the other group of changes are produced this is the separation (tato’pi nikṛṣṭairamśāntaraiḥ sarvamanyad vikārajātam utpādayati)( iti vibhāga iti).

\textbf{\dev{एतेन सर्वे व्याख्याता व्याख्याता ॥२८॥}}

\dev{एतेन प्रधानप्राणादीनां ब्रह्मशक्तित्वप्रतिपादनेनान्येऽप्यानुमानिका: कालादृष्टादयो ब्रह्मशक्तितया व्याख्याता वेदितव्या:, न तु स्वतन्त्रं किमप्यन्यदस्ति । अतो न तत्कारण्ताप्रतिपादकश्रुतिरपि, ब्रह्मकारणतावाक्यानां विरोध इत्यर्थ: । तत्र कालादिकारणवर्गेश्वरशक्तिताप्रतिपादकश्रुतिर्यथा श्वेताश्वतरे---- काल: स्वभावो नियतिर्यदृच्छा भूतानि योनि: पुरुष इति चिन्त्यमि” ति कालादीनि प्रकृत्य “य: कारणानि निखिलानि तानि कालात्मयुक्तान्यधितिष्ठेत्येक” इति । कालात्मयुक्तानि कालादीन्यात्मान्तानीत्यर्थ: । तेषां चाधिष्टेयत्ववचनाच्छक्तित्वं सिद्धं । प्रलये च भूतादीनां परमा सूक्ष्मावस्था सत्त्वादिशब्दवाच्येति मन्तव्यम् ॥}

\textbf{BS. I.4.28}

\textbf{Etena sarve vyākhyātā vyākhyātā}

By this (discussion) one should understand that just like pradhāna, prāṇa etc., obtaining the power of Brahman, others also (like) time,  samskāras (the unknown) etc., which are inferred have been explained through the power of Brahman, there is nothing else which is independent. Therefore śruti that mentions their being a cause is in contradiction to the statements that mention Brahma as the cause. Therein the śruti in the Śvet.Up that declares the group of kāla etc., obtaining the power of Īśvara:”kālaḥ svabhāvo niyatiryadṛcchā bhūtāni yoniḥ puruṣa iti cintyam” (I.2)  mentions with reference to kāla etc., (kālādīni prakṛtya): “yaḥ kāraṇāni nikhilāni tāni kālātmayuktānyadhitiṣṭhatyeka” iti. “kālātmayuktāni” (in the above quote) means starting with kāla and ending with the ātman. By mentioning that (Īśvara) is their support their having the power is established (teṣām cādhiṣṭheyatvavacanācchaktitvam siddham). And in dissolution one must understand that all beings are in a supreme subtle state which is denoted by the word sattva etc (pralaye ca bhūtādīnām paramā sūkṣmāvasthā sattvādiśabdavācyeti mantavyam).

\dev{तथा “तेनेशितं कर्म विवर्तते ह पृथ्व्यप्तेजोऽनिलखानि नित्यम् एतस्यवाणिम्न एतावान् न्यग्रोधस्तिष्ठति, रथान् रथयोगान् पथ: सृजत “ इत्यादिश्रुतयोऽपि कर्माणुजीवादिकारणतायामुदाहार्य्यं शक्तित्वेन समाधेया: । विवर्तते विकरोतीत्यर्थ: ।}

\dev{अथवेदं सूत्रमेवं व्याख्येयम् $\underline{\quad}$ एतेन पादचतुष्टयकृतब्रह्मव्याख्यानेनान्येऽपि सर्वे वेदान्ता व्याख्याता वेदितव्या:, एभिरेव तर्कैस्तेषामपि निर्णेतुं शक्यत्वादित्यर्थ: । व्याख्याता इत्यस्य द्वि: पाठोऽध्यायसमाप्तिसूचनार्थ:॥}

\begin{verse}
\textbf{\dev{जगज्जन्मादिहेतुत्वे द्वाराणिश्वरशक्तय: ।}}\\
\textbf{\dev{आक्षेपोद्धारमुखत: पादेऽत्र प्रतिपादिता: ॥}}
\end{verse}

\dev{इति श्री विज्ञानयतिविरचिते विज्ञानामृताख्य ऋजुभाष्ये प्रथमाध्यायस्य चतुर्थ: पाद: ।}

\dev{।समाप्तश्चाध्याय:॥}

Thus the śruti statements like: “teneśitam karma vivartate...etāvān nyagrodhasthiṣṭhati” (); “rathān rathayogān pathaḥ sṛjata” (Bṛ.Up. 4.3.10) using examples which mention karma, atoms, jīva etc., with reference to cause, need to be understood as due to being the power (of Brahman) (ityādiśrutayo’pi karmāṇujīvādikāraṇatāyāmudāhāryya śaktitvena samādheyāḥ). “vivartate”=changes.\footnote{Bhikṣu makes reference to the vivarta concept which advaitin consider as an appearance and not a real change. According to Bhikṣu it is real  chang (vikāra).} 

Therefore this sutra needs to be explained as follows: By this explanation of Brahman done through the four sections (pādas of the first adhyāya) all other Vedānta statements have to be understood as   explained, as it is possible to decide them also by the same logic. “vyā\-khyātā” used twice is to indicate the end of the chapter. “jagajjanmādihetutve dvārāṇīśvaraśaktayaḥ, ākṣepoddhāramukhataḥ pāde’tra pratipāditāḥ” Thus ends the fourth pāda of the first chapter in the ṛjubhāṣya called the Vijñānāmṛta written by the saṃnyāsin called vijñāna (vijñānayativiracite vijñānāmṛtākhyaṛjubhāṣye prathamādhyā\-yasya caturthaḥ pādaḥ). The first chapter ends.
