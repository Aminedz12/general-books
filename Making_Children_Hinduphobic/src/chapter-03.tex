\chapter{Minimizing Scope Of The Harappans}


\begin{longtable}{|>{\raggedleft}p{1.5cm}|p{8.5cm}|}
\multicolumn{2}{c}{\textbf{Table: 1}}\\ 
\hline
\textbf{Page \#}  &  \textbf{McGraw Hill Text}\tabularnewline
\hline
248 & The first civilizations of ancient India developed in the Indus Valley.  \tabularnewline
\hline
253 & Thousands of years ago, India’s first civilization began in the valley around the Indus River. The Indus Valley civilization, \tabularnewline
\hline
253 & the Indus Valley civilization developed near a great river system. \tabularnewline
\hline
253 & The Indus people prospered and built cities \tabularnewline
\hline
259 & 3. COMPARING What characteristics did the Indus Valley cities have in common? \tabularnewline
\hline
259 & 6. PRESENTING Imagine you and your family live in a village in the Indus Valley. You visit a large city nearby for the first time. Write a letter to a friend describing what you experience. Share your letter with a classmate or in small groups and then collaborate in writing one letter for the group. \tabularnewline
\hline
253 & Indus Valley people  \tabularnewline
\hline
254 & Indus Valley culture \tabularnewline
\hline
254 & Indus Valley people  \tabularnewline
\hline
254 & 1. IDENTIFYING How did most Indus Valley people earn a living? \tabularnewline
\hline
EM62, EM63\footnote{EM stands for \textit{Explorer 	Magazine}.} & Indus city \tabularnewline
\hline
EM66 & Indus people \tabularnewline
\hline
EM66 & Indus valley \tabularnewline
\hline
EM67 & Indus River cities \tabularnewline
\hline
IJ166\footnote{IJ stands for \textit{Inquiry Journal}.} & Indus Valley cities, \tabularnewline
\hline
IJ166 & Indus Valley civilization \tabularnewline
\hline
\end{longtable}

\section*{Analysis and Critique} 

The above are in violation of California Education Codes 51501 and 60044—adverse reflection—and evaluation criteria clauses pertaining to historical inaccuracy and non-inclusion of current research on the matter.

It is incorrect to state that ancient India developed in the Indus Valley. It developed in the Indus and Sarasvati river valley systems. The Sarasvati River is mentioned in the HSS Framework and accounts for 60\% of the sites of the Harappan Civilization. Why can’t the Indian civilization originate between two rivers when the Mesopotamian can originate between Tigris and Euphrates? Moreover, whereas at least thousand sites are in the Sarasvati plains, only 50 are on the Indus plain itself. Therefore, it is misleading and antiquated to refer to only the Indus River (for information on the Sarasvati River, see Chakrabarti and Saini 2009; Danino 2010). Jane McIntosh has the following to say: 

\begin{quote}
Suddenly it became apparent that the “Indus” Civilization was a misnomer—although the Indus had played a major role in the development of the civilization, the “lost Saraswati” River, judging by the density of settlement along its banks, had contributed an equal or greater part to its prosperity. Many people today refer to this early state as the “Indus-Saraswati Civilization” and continuing references to the “Indus Civilization” should be an abbreviation in which the “Saraswati” is implied.\footnote{Jane 	R. McIntosh, \textit{A Peaceful Realm: The Rise and Fall of the Indus Civilization} (Boulder: Westview Press, 2002), 24.}

There are some fifty sites known along the Indus whereas the Saraswati has almost 1,000. This is misleading figure because erosion and alluviation has between them destroyed or deeply buried the greater part of settlements in the Indus Valley itself, but there can be no doubt that the Saraswati system did yield a high proportion of the Indus people’s agricultural produce.\footnote{Jane R. McIntosh, \textit{A Peaceful Realm: The Rise and Fall of the Indus Civilization} (Boulder: Westview Press, 2002), 53}
\end{quote}

Most western archaeologists now understand that the civilization largely existed outside the Indus valley. Therefore, some of them, for instance Dilip Chakrabarti and Gregory Possehl, omit the word “Valley” and call it “Indus Civilization,” whereas others use “Indus-Sarasvati Civilization.” 

The latter is the preferred term used by a vast majority of archaeologists based within India. It also highlights the fact that the Sarasvati River formed the epicenter of this culture. Most Indian archaeologists, who are doing the actual work of excavation in India, have already adopted this term. A neutral term used alternately everywhere is “Harappan Culture” following archaeological conventions. By limiting the expanse of the Harappan Culture, McGraw Hill is creating adverse reflection on the Indian heritage and consequently, indoctrinating the students against it. 

\begin{thebibliography}{99}
\bibitem{ch3_key1} Chakrabarti, Dilip, and Sukhdev Saini.\textit{The Problem of the Sarasvati River and Notes on the Archaeological Geography of Haryana and Indian Punjab.} New Delhi: Aryan Books International, 2009.
\bibitem{ch3_key2} Danino, Michel. \textit{The Lost River: On the Trail of the Sarasvati}. New Delhi: Penguin Books, 2010.
\end{thebibliography}
\vskip -10pt

\begin{longtable}{|>{\raggedleft}p{1.5cm}|p{8.5cm}|}
\multicolumn{2}{c}{\textbf{Table: 2}}\\ 
\hline
\textbf{Page \#}  &  \textbf{McGraw Hill Text}\tabularnewline
\hline
253 & The Indus culture flourished between 2600 B.C.E. and 1900 B.C.E.\tabularnewline
\hline
\end{longtable}
\vskip -12pt

\section*{Analysis and Critique} 
\vskip -2pt

This is in violation of California Education Codes 51501 and 60044—adverse reflection—and evaluation criterion clause pertaining to historical inaccuracy.

The Indus or Harappan or Indus-Sarasvati Civilization (all three designations are in use), in its urban or “Mature” phase, is dated to 2600–1900 BCE. However, it was preceded by a millennium-long “Early” phase, which saw the convergence of important concepts and technologies, and had further antecedents reaching back to about 7000 BCE, a Neolithic culture at Mehargarh, which shows a continuous evolution all the way to the Mature phase.

Rao states the following and is supported by Sarkar et al.\ (2016), who also refer to a proposal by Possehl (1999, 2002) and to various radiocarbon dates from other sites:

\begin{tabular}{l}
7000 – 3300 BCE = Pre-Harappan (Mehargarh)\\
3300 – 2600 BCE = Early Harappan\\
2600 - 2500 BCE = Early Mature Harappan\\
2500 – 1900 BCE = Mature Harappan\\
1900 – 1300 BCE = Post-urban Harappan.\footnote{K.N. Dikshit, “Origin of Early Harappan Cultures in the Sarasvati 	Valley: Recent Archaeological Evidence and Radiometric Dates,” 	\textit{Journal of Indian Ocean Archaeology}  9 (2013): 132.}
\end{tabular}

By 1900BCE, the civilization started breaking down. The archeological evidence shows consequent migration of the population eastwards. Mohenjo-Daro completely vanished by 1900 BCE but Harappa shows continuity till about 1300 BCE.

Some other sites in Gujarat show continuity till 1000 BCE (see Kenoyer 1998). Many aspects and cultural practices, including the city building plan, of the Harappans continued till the historic times as evidenced in the excavation of a six hundred BCE city of Shishupalgarh by B. B. Lal (2011) and its incorporation in the \textit{Arthashastra of Kautilya} (1992), who apparently was the mentor and teacher of Chandragupta Maurya. 

If the publisher would have given the complete picture of the civilization beginning in 7000 BCE, it would have not only been historically accurate but also helped the Indian-American children develop pride regarding their heritage and ancestry. 

\begin{thebibliography}{99}
\bibitem{chap3_key3} Gupta, S. P. “The Dawn of Civilization.” In \textit{History of Science, Philosophy and Culture in Indian Civilization: Volume I: Part 1}, edited by G. C. Pandey and D. P. Chattopadhyaya. New Delhi: Centre for Studies in Civilizations, 1999.
\bibitem{chap3_key4} Kautilya. \textit{The Arthashastra}. Translated and edited by L. N. Rangarajan. New Delhi: Penguin Books, 1992.
\bibitem{chap3_key5} Kenoyer, Jonathan Mark. \textit{Ancient Cities of the Indus Civilization}. New Delhi: Oxford University Press, 1998. 
\bibitem{chap3_key6} Lal, B. B. \textit{Piecing Together: Memoirs of an Archaeologist.} New Delhi: Aryan International Books, 2011.
\bibitem{chap3_key7} McIntosh, Jane R. \textit{A Peaceful Realm: The Rise and Fall of the Indus Civilization.} Boulder: Westview Press, 2002.
\bibitem{chap3_key8} Possehl, Gregory L. \textit{Indus Age: The Beginnings}. Philadelphia: University of Pennsylvania Press, 1999.
\bibitem{chap3_key9} Possehl, Gregory L. \textit{The Indus Civilization}. Walnut Creek, CA: Alta Mira Press, 2002.
\bibitem{chap3_key10} Sarkar, Anindya, Arati Deshpande Mukherjee, M. K. Bera, B. Das, Navin Juyal, P. Morthekai, R. D. Deshpande, V. S. Shinde, and L. S. Rao. “Oxygen Isotope in Archaeological Bioapatites from India: Implications to Climate Change and Decline of Bronze Age Harappan civilization.” \textit{Nature Scientific Reports} 6 (May 2016): 1--9. doi:10.1038/srep26555.
\end{thebibliography}

\begin{longtable}{|>{\raggedleft}p{1.5cm}|p{8.5cm}|}
\multicolumn{2}{c}{\textbf{Table: 3}}\\ 
\hline
\textbf{Page \#}  &  \textbf{McGraw Hill Text} \tabularnewline
\hline
253 &  Indus civilization spread over much of western India and Pakistan\tabularnewline
\hline
\end{longtable}

\section*{Analysis and Critique} 

This is in violation of evaluation criterion clause pertaining to historical inaccuracy.

Pakistan didn't exist at the time. During this period, the entire region was referred to as India or ancient India.
