\chapter{Mcgraw Hill Ignores 7 Topics From History Social Science Framework}

The California HSS Framework, while improved from 2006 remains deeply flawed. That being said, publishers were still required to implement it by the state board of education.

The McGraw Hill text is missing 7 topic areas from the History Social Science framework. These areas are highlighted as they refer to positive additions to the framework. Specifically:

\begin{longtable}{|c|p{3.5cm}|p{5.5cm}|}
\hline 
\# & HSS framework Content & Topic / Description\tabularnewline
\hline
1 & Ch 10, line 828 – 832 &  Discuss features that are all present in modern Hinduism i.e., Pashupati seal, clay figurine showing ‘Namaste’\tabularnewline
\hline
2 & Ch 10, 857 – 859 & Spread of Indian languages including alternate theories (i.e., it originated in India)\tabularnewline 
\hline
3 & Ch 10, line 866 – 867 & Non-Brahmin sages like Valmiki and Vyasa\tabularnewline
\hline
4 & Ch 10, line 868 – 871 & Oneness of all living beings\tabularnewline
\hline
5 & Ch 10, line 872 & Relationship between Deities \& Brahman\tabularnewline
\hline
6 & Ch 10, lines 881 – 883, 926 – 935 & Moral teachings\tabularnewline
\hline
7 & Ch 10, lines 889 – 892 & Add discussion of central practices of Hinduism today, including yoga and meditation, festivals, pilgrimage, respect for saints 
and gurus.\tabularnewline
\hline
\end{longtable}

