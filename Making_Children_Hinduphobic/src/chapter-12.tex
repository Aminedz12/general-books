\chapter[Perverting History and Hinduism\\ in the Post-gupta Period]{Perverting History and Hinduism in the Post-gupta Period}

\begin{longtable}{|>{\raggedleft}p{1.5cm}|p{8.5cm}|}
\multicolumn{2}{c}{\textbf{Table: 1}}\\ 
\hline
\textbf{Page \#} & \textbf{McGraw Hill Text} \tabularnewline
\hline 
157 & The Gupta emperors who followed Chandra Gupta II faced several problems. From the mid-400s, internal uprisings and invasions from foreign enemies greatly weakened their authority. An invasion of the Huns from Central Asia was especially destructive and drained the empire’s resources. \tabularnewline
\hline
\end{longtable}

\section*{Analysis and Critique} 

These sentences are in violation of California Education Codes 51501, 60040(b) and 60044(a) involving Ethnic and Cultural groups—related to adverse reflection and instilling pride in the heritage of the child—through not adhering to evaluation criteria clauses involving historical inaccuracy and perpetrating Eurocentric and Ethnocentric history.

The present text is a very subjective statement with doubtful historical evidence. In fact, the Guptas and Yaśodharma repulsed the Hun attacks successfully and prevented them from entering the interior of India, but the protracted warfare sapped the strength of the Guptas (see Fleet 1888; Rao 2006). 

The colonial British history has represented inaccurately the Indian history as one of invasions in which the Indians consistently lost. This was deliberately manufactured to instill a sense of inferiority complex within the Indians—quite logical in that an internalization of inferiority by the colonized always helped the colonizer. The regurgitation of this colonial history continues, and as progressive people we must put a stop to this. There is always an implication of racial inferiority when an ethnic population is consistently shown to be on the losing side. A colonial history is always a racist history, and the authors of this book do not shy away from perpetuating a racist history, despite facts being the contrary.

\begin{thebibliography}{99}
\bibitem{chap12-key1} Fleet, John F. \textit{Corpus Inscriptionum Indicarum: Inscriptions of the Early Guptas. Vol.\ III.} Calcutta: Government of India, Central Publications Branch, 1888: 150 
\bibitem{chap12-key2} Rao, Narayan Singh. \textit{Tribal Culture, Faith, History and Literature.} New Delhi: Mittal Publications, 2006:18. 
\end{thebibliography}

\begin{longtable}{|>{\raggedleft}p{1.5cm}|p{8.5cm}|}
\multicolumn{2}{c}{\textbf{Table: 2}}\\ 
\hline
\textbf{Page \#} & \textbf{McGraw Hill Text} \tabularnewline
\hline 
156 & Monasticism is a religious way of life in which one rejects worldly pursuits to devote oneself fully to spiritual work. This practice flourished among Buddhists and among Hindus during the Gupta period. \tabularnewline
\hline
\end{longtable}

\section*{Analysis and Critique} 

This violates the evaluation criterion pertaining to accuracy, and while distorting the history of Hinduism it violates Education Codes pertaining to adverse reflection against Hinduism (51501 and 60044). 
\vskip 2pt

The practice of monasticism was well known in Hinduism and India at the time of the Gupta period. Monasticism has existed in Hinduism right from the Vedic period. In the fourth stage of one’s life, one was expected to take \textit{sanyāsa}, the vow of monasticism, and retire to the forest. The early Vedic literature mentions \textit{muni} (monk, mendicant, holy person) with characteristics that mirror those found in \textit{Sanyāsin}-s and \hbox{\textit{Sanyāsinī}-s}. \textit{Ṛgveda}, for example, in Book 10, Chapter 136, mentions mendicants as those with \textit{keśin} (long haired) and \textit{mala} clothes (dirty, soil-colored, yellow, orange, saffron) engaged in the affairs of \textit{mananat} (mind, meditation). \textit{Ṛgveda}, however, refers to these people as \textit{muni} and vati (monks who beg).
\vskip 2pt

These \textit{muni}-s, their lifestyle and spiritual pursuit, likely influenced the \textit{sanyāsa} concept. One class of \textit{muni}-s was associated with \textit{Rudra} or Shiva whereas the other was one of \textit{vratyas}—individuals who have taken vows. When the Buddha left home looking for enlightenment, he also encountered these renunciants in the forest. 

The current presentation leads students to think that the practice originated with Buddhists as an improvement over Hindu practices and then was copied by the Hindus. This is adverse reflection on Hinduism by subtly stating that Buddhism is superior to Hinduism.

\begin{thebibliography}{99}
\itemsep=1pt
\bibitem{chap12-key3} Basham, Arthur Llewellyn. \textit{The Origins and Development of Classical Hinduism}. Oxford: Oxford University Press, 1991.

\bibitem{chap12-key4} Ghurye, G. S. “Ascetic Origins,” \textit{Sociological Bulletin} 1, no.\ 2 (May 1952). 162--184.

\bibitem{chap12-key5} Michaels, Axel, \textit{Hinduism: Past and Present}. Princeton: Princeton University Press, 2004.
\end{thebibliography}
\vskip -10pt

\begin{longtable}{|>{\raggedleft}p{1.5cm}|p{8.5cm}|}
\multicolumn{2}{c}{\textbf{Table: 3}}\\ 
\hline
\textbf{Page \#} & \textbf{McGraw Hill Text} \tabularnewline
\hline 
160 & The bhakti movements emphasized the spiritual equality of all believers. They also challenged certain religious traditions, such as the power held by elite priests in society. \tabularnewline
\hline
\end{longtable}
\vskip -10pt

\section*{Analysis and Critique} 
\vskip -6pt

This is in violation of California Education Codes 51501 and 60044—adverse reflection—and evaluation criteria clauses pertaining to inaccuracy, perpetrating Eurocentric and Ethnocentric writing, not remaining neutral in matters of religion (continuing with the missionary and imperialistic writings on Hinduism), and thereby instilling prejudice in Hindu and non-Hindu children against Hinduism.

The spiritual equality of all is an old Upanishadic concept, which comes from the understanding that \textit{ātman} (loosely translated as soul) is Brahman—the ultimate Reality or Divine. Given that every individual has \textit{ātman} and each of those \textit{ātman} is Brahman, it fundamentally means that everyone is same and equal—in fact, everyone is equally Divine. This also happens to be the basis of the Hindu greeting of \textit{namaste} in which one bows to the Divine in others. The bhakti movement did not “put limits” on anyone. Rather, it emphasized bhakti or devotion over rituals (which were conducted by priests). This lead to the decline of ritual worship.

The British manufactured the Indian history as one of internal strife. With its policy of “Divide and Rule,” it created in the Indian narrative as many fissures as it could. We have already seen how the differences between Buddhism and Hinduism only relate to ontology and that there are more similarities than differences. 

The colonial Indian history also represented the bhakti movement as one of strife between commoners and priests. Medieval India was not Medieval Europe, where Protestantism arose against the all-consuming power and control of the priests. It is ignorant to conflate two different contexts and make the bhakti movement look like Lutheran Protestantism. Bhakti movement emphasized a direct relationship and experience of the Divine through devotion, love, and surrender. There could have been isolated instances of couple of bhakti saints, who clashed with the brahmin priests of the temple but they should not be generalized to make an abiding feature of the bhakti movement. In other words, the authors of the McGraw Hill text are regurgitating an outdated colonial history of India, making children look for strife and dissension where there are none.

\begin{longtable}{|>{\raggedleft}p{1.5cm}|p{8.5cm}|}
\multicolumn{2}{c}{\textbf{Table: 4}}\\ 
\hline
\textbf{Page \#} & \textbf{McGraw Hill Text} \tabularnewline
\hline 
160 & The bhakti movements made Hinduism more accessible to people in every social group and built bridges between India’s different social and cultural communities. Even though medieval India was not unified into one empire or a single religion, a cultural unity began to emerge \tabularnewline
\hline
\end{longtable}

\section*{Analysis and Critique} 

This is in violation of evaluation criterion clause pertaining to inaccuracy.

It is incorrect to state that the cultural unity began to emerge at this period in history. India was culturally united in ancient times (pre-dating the Mauryan Empire). The Vedas with the Upanishads laid the foundation of cultural unity of India—though the dates of the Vedas are in dispute, they certainly predate the Mauryan Empire. One example is the cultural development continued during the post-Mauryan phase and the pre-Gupta Empire era.

When the sentences in the above two tables are read more critically, biases of the textbook authors become even more implicit. The \textit{Islamophillic} and Marxist scholars have created the wrong notion that bhakti or devotion towards the Divine entered the Indian subcontinent through Islam, which created the conditions for the rise of cultural unity. The truth is that the advent of Islam has been one of the bloodiest calamities to befall the non-Islamic population of medieval India. Abraham Eraly, while accounting the history of the Delhi Sultanate, has called this period the “Age of Wrath” and has titled his book with the same name—more on this later when we discuss the coverage of Islam by McGraw Hill textbook. 

\begin{longtable}{|>{\raggedleft}p{1.5cm}|p{8.5cm}|}
\multicolumn{2}{c}{\textbf{Table: 5}}\\ 
\hline
\textbf{Page \#} & \textbf{McGraw Hill Text} \tabularnewline
\hline 
170 & GITA GOVINDA (SONG OF THE DIVINE HERDSMAN) \tabularnewline
\hline
\end{longtable}

\section*{Analysis and Critique} 

This is in violation of California Education Codes 51501 and 60044—adverse reflection. 

It is derogatory to translate the title as Song of the Divine Herdsman. It is the song of Govinda (Lord Krishna). The name is a proper noun.

\begin{longtable}{|>{\raggedleft}p{1.5cm}|p{8.5cm}|}
\multicolumn{2}{c}{\textbf{Table: 6}}\\ 
\hline
\textbf{Page \#} & \textbf{McGraw Hill Text} \tabularnewline
\hline 
156 & gods \tabularnewline
\hline
160, 161, 170 & god \tabularnewline
\hline
160 & deities \tabularnewline
\hline
\end{longtable}

\section*{Analysis and Critique} 

These are in violation of California Education Codes 51501 and 60044 - adverse reflection.

The CAPEEM Lawsuit ruled against the State of CA in 2006 with a statement that lower-case use of “d” for Deities and “g” for Gods is discriminatory and indicates that Hindu Deities and Gods are inferior to the Abrahamic God. 

The use of lower case for Deities and Gods basically conflates Hinduism with polytheism, and in the binary of monotheism/polytheism, privileges monotheism over polytheism.
