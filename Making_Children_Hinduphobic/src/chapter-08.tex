\chapter{Buddhism is not Hinduism-Improved}

\begin{longtable}{|>{\raggedleft}p{1.5cm}|p{8.5cm}|}
\multicolumn{2}{c}{\textbf{Table: 1}}\\ 
\hline
\textbf{Page \#} & \textbf{McGraw Hill Text} \tabularnewline
\hline 
264 & During the 500s B.C.E., some Indians felt unhappy with the many ceremonies of the Hindu religion. They wanted a simpler, more spiritual faith. They left their homes and looked for peace in the hills and forests. Many trained their minds to focus and think in positive ways. This training was called meditation. Some seekers developed new ideas and became religious teachers. One of these teachers was Siddhartha Gautama (sih•DAHR•tuh GOW•tah•muh). He became known as the Buddha (BOO•dah). He founded a new religion called Buddhism (BOO•dih•zuhm).\tabularnewline
\hline
\end{longtable}

\section*{Analysis and Critique} 

This is in violation of California Education Codes 51501 and 60044—adverse reflection—and evaluation criteria clauses pertaining to historical inaccuracy, not remaining neutral in matters of religion, and thereby instilling prejudice in Hindu and non-Hindu children against Hinduism.

The present text indulges in adverse reflection stating that people left Hinduism and started looking for a simpler faith and Buddhism emerged by the efforts of one such seeker. It shows Buddhism as an improvement upon Hinduism. It overlooks the fact that Buddhism was one of the many traditions of ancient India, including \textit{Sāṁkhyā}. The Buddha himself did not claim that he was inventing a new path; rather he claimed to have revived an ancient teaching. Barua (1921) highlights how the Buddhist teaching was a continuation of the ancient teachings. 

Meditation and thinking positively have strong and old roots in Hinduism pre-dating (and borrowed by) Buddhism. Many people went to the forest during the \textit{vānaprastha} and \textit{sanyāsa} stages of their life to focus their life on spirituality (as taught to do so by Hindu scriptures and teachers when they reached these stages of their life). Further, this is not new or unique to the 500s BCE; this had already been happening for hundreds of years prior to this period and continues to happen to-date.

\begin{thebibliography}{99}
\bibitem{chap8-key1} Barua, Benimadhab. \textit{A History of Pre-Buddhistic Indian Philosophy}. Delhi: Motilal Banarsidass, 1921.
\end{thebibliography}

\begin{longtable}{|>{\raggedleft}p{1.5cm}|p{8.5cm}|}
\multicolumn{2}{c}{\textbf{Table: 2}}\\ 
\hline
\textbf{Page \#} & \textbf{McGraw Hill Text} \tabularnewline
\hline
265 & Like Hindus, the Buddha believed in reincarnation, but in a different way. He taught that people could end the cycle of rebirth by following the Eightfold Path rather than their dharma.\tabularnewline
\hline
\end{longtable}

\section*{Analysis and Critique} 

This is in violation of California Education Codes 51501 and 60044—adverse reflection—and evaluation criteria clauses pertaining to historical inaccuracy, not remaining neutral in matters of religion, advocating Buddhism over Hinduism, subtly deriding Hinduism, and thereby instilling prejudice in Hindu and non-Hindu children against Hinduism.

Incorrect statement regarding ending reincarnation in Hinduism! Exiting the cycle of reincarnation is done by exhausting the fruits of karma. The dharma in Hinduism prepares one for liberation as it does in Buddhism. In fact, Buddhism is called the Buddha dharma.

We see a strong effort on the part of the author or authors of the textbook to put Buddhism at loggerheads with Hinduism. Buddhism and Hinduism share vast amounts of similarities. There is only one difference, which has separated Buddhism and Hinduism (mainly \textit{Vedānta}): Whereas \textit{Vedānta} of Hinduism believes in ultimate reality called Brahman, Buddhism (later Buddhism because the Buddha refused to venture into any metaphysical questions such as ultimate reality and so on and so forth) does not believe in Brahman. Consequently, Buddhism does not believe in \textit{ātman} or soul (consequently because in \textit{Vedānta} \textit{ātman} is Brahman). Buddhism therefore speaks about \textit{anātman} rather than \textit{ātman}. Creating differences where there may not be any tantamount to sowing seeds of religious disharmony where there may not be any.
