\chapter{Whitewashing the Brutal History of Islam in India}

\begin{longtable}{|>{\raggedleft}p{1.5cm}|p{8.5cm}|}
\multicolumn{2}{c}{\textbf{Table: 1}}\\ 
\hline
\textbf{Page \#} & \textbf{McGraw Hill Text} \tabularnewline
\hline
166 & After Muhammad’s death in 632, the Ummayad Dynasty ruled the Muslim Empire, which stretched from the Atlantic Ocean to the Indus Valley in India. In 711 an Ummayad general named Muhammad bin Qasim conquered a region of India named Sind, bringing Islam to India and the Indus Valley for the first time. Arab and Persian traders who had visited Indian ports for centuries were now Muslims. Some formed small Muslim communities in cities along India’s west coast. As these communities became larger, the number of conversions increased. In this way, Islam became an important minority religion in Western India. \tabularnewline
\hline
166 & The descendants of these traders benefited from the emergence of a Muslim community, which stretched from Europe to the Indus River. Its size enabled Muslims to engage in trade networks between India and western lands. \tabularnewline
\hline
166 & While Islam entered Western India peacefully for the most part \tabularnewline
\hline
\end{longtable}

\section*{Analysis and Critique} 

This violates the evaluation criterion pertaining to accuracy, and by whitewashing the Indian history in favor of Islam, it violates Education Codes related to adverse reflection and indoctrination against Hinduism (51501 and 60044). As the following discussion will show, Islam was also highly destructive towards universities, temples, and all institutions where Hinduism, Buddhism, and Jainism were nurtured. By whitewashing this history of destruction, the McGraw Hill also violates Education Codes (51501, 60040b, and 60044a) involving non-Muslim Indian Americans as an Ethnic and Cultural Group by not instilling pride in their heritage. Further by not telling the history of destruction of their heritage, it adversely reflects on it. 

The text gives the impression that Islam spread in India due to Merchants spreading it. The truth is that Islam spread due to forced conversion (either through violence, intimidation or extra-taxation for non-believers known as jizya). 

Will Durant in \textit{Our Oriental Heritage} states the following:
\begin{quotation}
\noindent The Mohammedan Conquest of India is probably the bloodiest story in history. It is a discouraging tale, for its evident moral is that civilization is a precarious thing, whose delicate complex of order and liberty, culture and peace may at any time be overthrown by barbarians invading from without or multiplying within…. The first Moslem attack was a passing raid upon Multan, in the western Punjab (664 A.D.)… But the real Moslem conquest of India did not come till…. the year 997…. when Mahmud…. swept across the frontier with a force inspired by a pious aspiration for booty. He met the unprepared Hindus at Bhimnagar, slaughtered them, pillaged their cities, destroyed their temples, and carried away the accumulated treasures of centuries. Returning to Ghazni he astonished the ambassadors of foreign powers by displaying “jewels and unbored pearls and rubies shining like sparks, or like wine congealed with ice, and emeralds like fresh sprigs of myrtle, and diamonds in size and weight like pomegranates.” Each winter Mahmud descended into India, filled his treasure chest with spoils, and amused his men with full freedom to pillage and kill; each spring he returned to his capital richer than before. At Mathura (on the Jumna) he took from the temple its statues of gold encrusted with precious stones, and emptied its coffers of a vast quantity of gold, silver and jewelry; he expressed his admiration for the architecture of the great shrine, judged that its duplication would cost one hundred million \textit{dinars} and the labor of two hundred years, and then ordered it to be soaked with naphtha and burnt to the ground. Six years later he sacked another opulent city of northern India, Somnath, killed all its fifty thousand inhabitants, and dragged its wealth to Ghazni. In the end he became, perhaps, the richest king that history has ever known. Sometimes he spared the population of the ravaged cities, and took them home to be sold as slaves; but so great was the number of such captives that after some years no one could be found to offer more than a few shillings for a slave. Before every important engagement Mahmud knelt in prayer, and asked the blessing of God upon his arms. He reigned for a third of a century; and when he died, full of years and honors, Moslem historians ranked him as the greatest monarch of his time, and one of the greatest sovereigns of any age.
\medskip

\indent
Seeing the canonization that success had brought to this magnificent thief, other Moslem rulers profited by his example, though none succeeded in bettering his instruction. In 1186 the Ghuri, a Turkish tribe of Afghanistan, invaded India, captured the city of Delhi, destroyed its temples, confiscated its wealth, and settled down in its palaces to establish the Sultanate of Delhi—an alien despotism fastened upon northern India for three centuries, and checked only by assassination and revolt. The first of these bloody sultans, Kutb-d Din Aibak, was a normal specimen of his kind—fanatical, ferocious and merciless. His gifts, as the Mohammedan historian tells us, “were bestowed by hundreds of thousands, and his slaughters likewise were by hundreds of thousands. “In one victory of this warrior (who had been purchased as a slave), “fifty thousand men came under the collar of slavery, and the plain became black as pitch with Hindus.” Another sultan, Balban, punished rebels and brigands by casting them under the feet of elephants, removing their skins, stuffing these with straw and hanging them from the gates of Delhi. When some Mongolian habitants who had settled in Delhi, and had been converted to Islam, attempted arising, Sultan Alau-d-din (the conqueror of Chitor) had all the males—from fifteen to thirty thousand of them-slaughtered in one day. Sultan Muhammad bin Tughlak acquired the throne by murdering his father, became a great scholar and an elegant writer, dabbled in mathematics, physics and Greek philosophy, surpassed his predecessors in bloodshed and brutality, fed the flesh of a rebel nephew to the rebel’s wife and children, ruined the country with reckless inflation, and laid it waste with pillage and murder till the inhabitants fled to the jungle. He killed so many Hindus that, in the words of a Moslem historian, “there was constantly in front of his royal pavilion and his Civil Court a mound of dead bodies and a heap of corpses, while the sweepers and executioners were wearied out by their work of dragging the victims and putting them to death in crowds.” In order to found a new capital at Daulatabad he drove every inhabitant from Delhi and left it a desert; and hearing that a blind man had stayed behind in Delhi, he ordered him to be dragged from the old to the new capital, so that only a leg remained of the wretch when his last journey was finished. The Sultan complained that the people did not love him, or recognize his undeviating justice. He ruled India for a quarter of a century, and died in bed. His successor, Firoz Shah, invaded Bengal, offered a reward for every Hindu head, paid for 180,000 of them, raided Hindu villages for slaves, and died at the ripe age of eighty. Sultan Ahmad Shah feasted for three days whenever the number of defenseless Hindus slain in his territories in one day reached twenty thousand.
\medskip

\indent
These rulers were often men of ability, and their followers were gifted with fierce courage and industry; only so can we understand how they could have maintained their rule among a hostile people so overwhelmingly outnumbering them. All of them were armed with a religion militaristic in operation, but far superior in its stoical monotheism to any of the popular cults of India; they concealed its attractiveness by making the public exercise of the Hindu religions illegal, and thereby driving them more deeply into the Hindu soul. Some of these thirsty despots had culture as well as ability; they patronized the arts, and engaged artists and artisans–usually of Hindu origin– to build for them magnificent mosques and tombs; some of them were scholars, and delighted in converse with historians, poets and scientists. One of the greatest scholars of Asia, Alberuni, accompanied Mahmud of Ghazni to India and wrote a scientific survey of India comparable to Pliny’s \textit{Natural History} and Humboldt’s \textit{Cosmos}. The Moslem historians were almost as numerous as the generals, and yielded nothing to them in the enjoyment of bloodshed and war. The Sultans drew from the people every rupee of tribute that could be exacted by the ancient art of taxation, as well as by straightforward robbery; but they stayed in India, spent their spoils in India, and thereby turned them back into India’s economic life. Nevertheless, their terrorism and exploitation advanced that weakening of Hindu physique and morale, which had been begun by an exhausting climate, an inadequate diet, political disunity, and pessimistic religions. 
\medskip

\indent
The usual policy of the Sultans was clearly sketched by Alau-d-din, who required his advisers to draw up “rules and regulations for grinding down the Hindus, and for depriving them of that wealth and property which fosters disaffection and rebellion.” Half of the gross produce of the soil was collected by the government; native rulers had taken one-sixth. “No Hindu,” says a Moslem historian, “could hold up his head, and in their houses no sign of gold or silver…or of any superfluity was to be seen…. Blows, confinement in the stocks, imprisonment and chains, were all employed to enforce payment.” When one of his own advisers protested against this policy, Alau-d-din answered: “Oh Doctor, thou art a learned man, but thou hast no experience; I am an unlettered man, but I have a great deal. Be assured, then, that the Hindus will never become submissive and obedient till they are reduced to poverty. I have therefore given orders that just sufficient shall be left to them from year to year of corn, milk and curds, but that they shall not be allowed to accumulate any property.”\footnote{Will Durant, \textit{The Story of Civilization: Part 1: Our Oriental Heritage} (New York: Simon and Schuster, 1954), 459--462.} 
\end{quotation}
Abraham Eraly in \textit{The Age of Wrath: The History of the Delhi Sultanate} has similar things to say. This is precisely because both Will Durant and Abraham Eraly have based their contentions on the records left by the historians of the various Sultans mentioned in the above.
\bigskip

\begin{thebibliography}{99}
\bibitem{chap14-key1}  Eraly, Abraham. \textit{The Age of Wrath: The History of the Delhi Sultanate.} New York: Penguin Books, 2014. 
\end{thebibliography}
\newpage

\begin{longtable}{|>{\raggedleft}p{1.5cm}|p{8.5cm}|}
\multicolumn{2}{c}{\textbf{Table: 2}}\\ 
\hline
\textbf{Page \#} & \textbf{McGraw Hill Text} \tabularnewline
\hline
167 & At its height in the 1300s, the Delhi Sultanate was one of the strongest kingdoms in the Islamic world. The Delhi sultans built mosques and fortresses throughout their kingdom. Many of these sultans permitted the practices of Hinduism and Buddhism, while others attempted to force Islam on their Indian subjects \tabularnewline
\hline
167 & Attempts by the Delhi sultans to establish power in India’s Deccan Plateau failed. However, it did encourage some Muslim conversions in the region. \tabularnewline
\hline
167 & Even so, it contributed substantially to cultural development in the subcontinent. Its sultans introduced Islam to the people in the Deccan Plateau, South India, and part of eastern India. \tabularnewline
\hline
167 & After Delhi fell to a warlord in 1398, northern India crumbled into a number of small kingdoms. \tabularnewline
\hline
\end{longtable}

\section*{Analysis and Critique} 

This violates the evaluation criterion pertaining to accuracy, and by whitewashing the Indian history in favor of Islam, it violates Education Codes related to adverse reflection and indoctrination against Hinduism (51501 and 60044). As the following discussion will show, Islam was also highly destructive towards universities, temples, and all institutions where Hinduism, Buddhism, and Jainism were nurtured. By whitewashing this history of destruction, the McGraw Hill also violates Education Codes (51501, 60040b, and 60044a) involving non-Muslim Indian Americans as an Ethnic and Cultural Group by not instilling pride in their heritage. Further by not telling the history of destruction of their heritage, it adversely reflects on it. 

To state “others attempted to force Islam on their Indian subjects” is akin to nothing but perpetuating falsehood. All the Sultans were brutal towards the Hindus—killing them, bringing them under subservience, selling the women and children as sex slaves and slaves (much like what ISIS did with respect to non-Muslim minorities), destroying their temples, pillaging their cities and villages, and bringing them under heavy taxation. The genocide committed by each ruler of the Delhi Sultanate is well recorded by scholars and primary records of that era. The brutality of the Sultans on non-Muslims, in particular Hindus and Buddhists, has been of such staggering nature that Abraham Eraly, reviewing this period based on the records left by historians or chroniclers residing in the courts of Sultans, has named it as the \textit{Age of Wrath.} We are providing a sample of the brutality, which will speak for itself. 

Delhi Sultanate, which extended over 320 years (1206--1526 CE), was preceded with raids and invasions by Muhammad of Ghor. The Sultanate witnessed a period of extensive religious violence in various parts of India at the hands of the Sultan’s army. The perpetrators were Sunni Muslims and the primary victims were Hindus, but not exclusively as Buddhists, Jains, Shia, and Sufi Muslims were targeted as well. Religious violence became state sponsored with the start of Delhi Sultanate and it continued through the Mughal Empire. Hindus who converted to Islam were not immune from persecution, which was illustrated by the Muslim Caste System in India as established by Ziauddin al-Barani in\textit{ Fatawa-i-Jahandari}.

Mohammed Ghori destroyed Hindu temples and idols starting from his first attack in 1194 (Elliot 1953). His successor, Qutb-din-Aibak, who laid the foundation of the Delhi Sultanate in 1206, built the first mosque in Delhi from the remains of twenty Hindu and Jain temples that he demolished.\footnote{“Qutub Minar and Its Monuments, Delhi,” World Heritage 	Convention, UNESCO, accessed February 12, 2018, \url{http://whc.unesco.org/en/list/233}.} This is also noted by Welch and Crane (1983) who state that the Quwwatu'l-Islam was built with the remains of demolished Hindu and Jain temples. Maulana Hakim Saiyid Abdul Hai too records this in \textit{Hindustan Islami Ahad Mein (Hindustan under Islamic rule)}.\footnote{Maulana 	Hakim Saiyid Abdul Hai, \textit{Hindustan Islami Ahad Mein (Hindustan under Islamic rule)}, trans.\ Maulana Abdul Hasan Nadwi (Nadwatul-Ulama, 1973).} 

The slave dynasty was succeeded by the Khilji dynasty. Khiljis and their commanders regularly attacked, killed, and enslaved Hindus across India (Holt, Lambton, and Lewis 1970; Hunter 2013; Habib 1978; Levi 2002; Elliot, and Downson 2010; Kulke, and Rothermund 2016; Donkin 1998; Narasimhacharya 2004; Rao, and Reddi 1976; Aiyangar 1991). 

Amir Khusrow, who accompanied Allauddin Khilji—the second of in the line of Khiljis, who had ascended to the throne after killing his uncle Jalaluddin Khilji—in many of the raids, in\textit{ Táríkh-i 'Aláí} comments: 
\begin{quote}
The [Muslim] army left Delhi ... [in] Nov.\ 1310.... After crossing those rivers, hills and many depths, ... elephants [were sent], ... in order that the inhabitants of Ma'bar might be aware that the day of resurrection had arrived amongst them; and that all the burnt Hindus would be dispatched by the sword to their brothers in hell, so that fire, the improper object of their worship, might mete out proper punishment to them. The sea-resembling army moved swiftly, like a hurricane, to Ghurganw. Everywhere, ... the people who were destroyed were like trunks carried along in the torrent of the Jihun, or like straw tossed up and down in a whirlwind.\footnote{John Dowson, \textit{The History of India, as Told by its Own Historians: Edited from the Posthumous Papers of the Late Sir H. M. Elliot: Volume III} (London: Trubner \& Co., 1871), 86--87.}
\end{quote}
These campaigns of murder, killing, violence, abasement, and humiliation were not the works of the Alauddin Khiljis’s army alone (meaning that they did not result from the animosity between the invaded and invader). The priests—the mullahs, kazis, muftis, and court officials—of Allauddin Khilji, recommended them citing religious reasons. Scholars, who want to dispute the above, should look up Koran for its recommendations for treatment to be meted out to “idol worshippers” (\textit{kafirs} in its phraseology). Qazi Mughisuddin of Bayánah asked Allauddin to “keep Hindus in subjection, in abasement, as a religious duty, because they are the most inveterate enemies of the Prophet, and because the Prophet has commanded us to slay them, plunder them, and make them captive, saying ‘Convert them to Islam or kill them, enslave them and spoil their wealth and property.’\,”\footnote{John Dowson, \textit{The History of India, as Told by its Own Historians: Edited from the Posthumous Papers of the Late Sir H. M. Elliot: Volume III} (London: Trubner \& Co., 1871), 184.}

The general of Alauddin Khilji, Malik Kafur, attacked the central and southern parts of India taking the violent campaigns as far as Madurai in Tamil Nadu. These attacks took place between 1309 and 1311 in which three Hindu kingdoms of Deogiri (Maharashtra), Warangal (Telangana) and Madurai (Tamil Nadu) were attacked. The slaughter took place in thousands. The magnificent Halebid temple was destroyed, and numerous temples, cities, and villages were ransacked, plundered, and ruined—the ruins of some of the temples still stand to tell the tale of destructions. The loot from Warangal itself was so large that a thousand camels had to be used to carry it to Delhi. In the loot was also the famed Koh-i-Noor diamond, which after changing a few hands, landed in the crown of Queen Victoria in 1877.

In his second expedition, Malik Kafur, who was already known as Malik Naib by then, attacked and won Dvarasamudra, the capital city of the Hoysala king Vir Ballala. Defeating him, he moved on to Tamil Nadu ransacking the temple cities of Chidambaram, Srirangam, and Madurai. In 1311, Malik Kafur took over the Srirangam temple, massacred the Brahmin priests of the temple who resisted the invasion for three days, ransacked the treasury and storehouse of the temple while desecrating and destroying numerous religious icons. Kafur returned to Delhi in the October of 1311 and was received by Allauddin Khilji with great honor. Writes Abraham Eraly:
\begin{quote}
Kafur had brought with him an immense booty—“612 elephants, 96000 mans\footnote{One “mana” is 40 kilograms or approximately 88 lbs.} of gold, several boxes of jewels and pearls, and 20,000 horses,” according to Barani. The quantity of the booty brought by Kafur astonished the people of Delhi. “No one,” comments Barani “could remember anything like it, nor was there anything like it recorded in history.”\footnote{Abraham Eraly, in \textit{The Age of Wrath: A History of the Delhi Sulttanate} (New York: Penguin Books, 2015), 117.}
\end{quote}
Given that Koran itself had made it explicit, how the idol-worshippers, aka the Hindus, had to be treated, the butchery against the Hindus continued with the Tughlaqs as well—the dynasty that succeeded the Khiljis. A new round of invasions of the kingdoms of Southern India began under Ulugh Khan in 1323. Srirangam was attacked again; about 12000 of the unarmed ascetics were killed and the shrine was desecrated. Sri Vedanta Desika, the Vaishnava philosopher, hid himself in the pile of corpses, and in the process saved the sole manuscript of \textit{Srutaprakasika}, the magnum opus of Sri Sudarsana Suri whose eyes, along with those of his two sons, were put out (see Narasimhacharya 2004; Rao, and Reddi 1976; Aiyangar 1991; Renganathan 2013; Young 1988). 

In the line of the Tughlaqs, Firoz Shah Tughlaq was the third. The \textit{Tarikh-i-Firoz Shahi}, the historical record written in his times by Ziauddin Barani, attests to the continued and systematic persecution of the Hindus. Enslavement and capture of the Hindus was as widespread as ever. When the Sultan died, all the slaves in his service were killed and piled up in a heap (Bannerjee 1967). He was particularly severe on Brahmins, who would refuse to convert. Making a case in point, Ziauddin Barani in \textit{Tarikh-i-Firoz Shahi} recounts the tale of a Brahmin, who, in defiance of prohibition against the worship of Deities (idols in the terminology of Islam), had constructed a wooden tablet on which he had painted the Hindu Deities. His worship attracted other people to his house as well. When the Sultan came to know of the Brahmin,
\begin{quote}
an order was accordingly given that the Brahman, with his tablet, should be brought into the presence of the Sultan.... The true faith was declared to the Brahman and the right course pointed out, but he refused to accept it.... The Brahman was tied hand and foot and cast into it [a pile of brushwood]; the tablet was thrown on the top and the pile was lighted.... The tablet of the Brahman was lighted in two places, at his head and at his feet.... The fire first reached his feet, and drew from him a cry, but the flames quickly enveloped his head and consumed him. Behold the Sultan's strict adherence to law and rectitude.\footnote{John Dowson, \textit{The History of India, as Told by its Own Historians: Edited from the Posthumous Papers of the Late Sir H. M. Elliot: Volume III} (London: Trubner \& Co., 1871), 365.}
\end{quote}
Firoz Shah Tughlaq ensured that the Hindus paid a mandatory and high Jizya tax (a recommendation that had come to the believers of Islam from the Koran itself) and they were kept under surveillance. Any Hindu, who practiced any aspect of his or her religion in public, was arrested, persecuted, \textit{and} executed. This comes to us from the autobiography of the Sultan himself. He writes the following in \textit{Futuhat-i-Firoz Shahi:} 
\begin{quote}
Some Hindus had erected a new idol-temple in the village of Kohana, and the idolaters used to assemble there and perform their idolatrous rites. These people were seized and brought before me. I ordered that the perverse conduct of this wickedness be publicly proclaimed and they should be put to death before the gate of the palace. I also ordered that the infidel books, the idols, and the vessels used in their worship should all be publicly burnt. The others were restrained by threats and punishments, as a warning to all men, that no \textit{zimmi} could follow such wicked practices in a Musulman country.\footnote{John Dowson, \textit{The History of India, as Told by its Own Historians: Edited from the Posthumous Papers of the Late Sir H. M. Elliot: Volume III} (London: Trubner \& Co., 1871), 381.}
\end{quote}
If the persecution of the Hindus under the Delhi Sultans were not enough, Timur decide to invade India in 1398 and 1399 CE. It is incorrect to call Timur a warlord. He was a Muslim Turko-Mongol ruler and founder of the Timurid Empire. Scholars estimate that his military campaigns caused the deaths of 17 million people, amounting to about 5\% of the world population at the time. Timur writes in his autobiography that he wanted to wage war against the infidel Hindus, for at the age of 62, he wanted to become a “Ghazi.” Quoting him, Abraham Eraly writes: 
\begin{quote}
‘My object in the invasion of Hindustan,’ he told the assembled nobles, ‘is to lead an expedition against the infidels, so that … we may convert to the true faith the people of that country, and purify the land itself of the filth of infidelity and polytheism, and we may overthrow their temples and idols and become \textit{ghazis} and \textit{mujahids}.’\footnote{Abraham Eraly, in \textit{The Age of Wrath: A History of the Delhi Sulttanate} (New York: Penguin Books, 2015), 193--194.}
\end{quote}
Eraly expands further:
\begin{quote}
‘My principal object in coming to Hindustan, and in undergoing all this toil and hardship, has been to accomplish two things,’ he candidly stated once amid his Indian campaign. ‘The first is to wage war against the infidels … and by this religious warfare to acquire some claim to reward in the life to come. The other was a worldly object, that the army of Islam might gain something by plundering the wealth and valuables of the infidels. Plunder in war is as lawful as their mothers’ milk to Muslims who wage war for their faith.’ He would, he decided, invade India, slaughter its infidels, plunder the land, and return home.\footnote{Abraham Eraly, in \textit{The Age of Wrath: A History of the Delhi Sulttanate} (New York: Penguin Books, 2015), 194.}
\end{quote}
And true to his words, the massacre of the Hindu population he did wherever he went with his army. On his way to Delhi, he attacked, killed, burned towns and cities, plundered, looted and destroyed temples, and took men, women and children as slaves. Just before invading Delhi, fearing rebellion among the Hindu slaves, he killed and had one hundred thousand of them killed (Dowson 1871; Burgan 2009; Rayachaudhuri, and Habib 2004). Describing the sacking of Delhi, Sharafuddin Yazdi in \textit{Zafarnama} writes: 
\begin{quote}
[Timur's] soldiers grew more eager for plunder and destruction. On that Friday night there were about 15,000 men in the city who were engaged from early eve till morning in plundering and burning the houses. In many places the impure infidel \textit{gabrs} made resistance…. Every soldier obtained more than twenty persons as slaves, and some brought as many as fifty or a hundred men, women and children as slaves of the city. The other plunder and spoils were immense, gems and jewels of all sorts, rubies, diamonds, stuffs and fabrics, vases and vessels of gold and silver…. On the 19th of the month Old Delhi was thought of, for many Hindus had fled thither…. Amir Shah Malik and Ali Sultan Tawachi, with 500 trusty men, proceeded against them, and falling upon them with the sword despatched them to hell. High towers were built with the heads of the Hindus, and their bodies became food of ravenous beasts and birds. On the same day all Old Delhi was plundered.\footnote{John Dowson, \textit{The History of India, as Told by its Own Historians: Edited from the Posthumous Papers of the Late Sir H. M. Elliot: Volume III} (London: Trubner \& Co., 1871), 503--504.} 
\end{quote}
Smith (1923) and Keay (2011) also record the genocide committed by Timur. 

The extreme north of the Indian subcontinent, Kashmir, had so far been spared the eye of the Sultanate. It was time now for Sultan Sikander to direct his efforts to bring it under Islamic subjugation. Such was his violence and destruction on the Hindu and Buddhist population that he earned the title of \textit{But-shikan} or idol-breaker (Houtsma 2010). The temples, shrines, monasteries, ashrams, universities, and hermitages were systematically destroyed (Dawson 1875; Haig 1928). Tutored by the Isalmic theologian, Muhammad Hamadani, the Sharia laws banning dance, music, religious festivals, consumption of wine, etc were enforced. To escape the religious violence during his reign, many Hindus converted to Islam and many left Kashmir. Numerous others were killed (Lawrence 2005).

The Tughlaqs were succeeded by the Sayyid dynasty, which ruled between 1414 and 1451 CE. A historical record, \textit{Tarikh-i Mubarak-Shahi}, by Yahya bin Ahmad describe the similar tale of oppression of the Hindus. He records that the Muslim commanders persecuted and “chastised” the Hindus of Ahar, Khur, Kampila, Gwalior, Seori, Chandawar, Etawa, Sirhind, Bail, Katehr and Rahtors (Dowson 1872).

The Lodi dynasty also continued this pattern of genocide in their reign. Hindus experienced burning and killing in the regions of Uttar Pradesh, Bihar, and Bengal (Haig 1928). Such has been the iconoclasm of the Sultanate rulers that none of the ancient temples or Buddhist monasteries survive in these regions. The state of Bihar earlier was called such because of the numerous Viharas or monasteries that existed in the region. The ruins of three universities within a hundred mile radius tell the tale of what a great center for learning India was before the Islamic invaders descended on India. These universities attracted scholars from all over the world as numerous Chinese scholars who travelled to India for studies have recorded. 

The religious violence of the Lodi dynasty is summarized in \textit{Tarikh-i-Daudi}. Sultan Sikandar Lodi
\begin{quote}
was so zealous of a Musulman that he utterly destroyed diverse places of worship of the infidels, and left not a vestige of them. He entirely ruined the shrines of Mathura, the minefield of heathenism and turned their places of worship into caravanserais and colleges. Their stone images were given to the butchers to use them as meat weights, and all the Hindus in Mathura were strictly prohibited from shaving their heads and beards, and performing ablutions. He thus put and end to all the idolatrous rites of the infidels there…. Every city thus conformed as he desired to the customs of Islam.\footnote{John Dowson, \textit{The History of India, as Told by its Own Historians: Edited from the Posthumous Papers of the Late Sir H. M. Elliot: Volume IV} (London: Trubner \& Co., 1872), 447.} 
\end{quote}
\begin{thebibliography}{99}
\itemsep=1.1pt
\bibitem{chap14-key2} Aiyangar, S. Krishnaswami.  \textit{South India and her Muhammadan Invaders}. New Delhi: Asian Educational Services, 1991. 

\bibitem{chap14-key3} Bannerjee, Jamini. \textit{History of Firuz Shah Tughluq}. New Delhi: Munshiram Manoharlal, 1967. 

\bibitem{chap14-key4} Burgan, Michael. \textit{ Empire of the Mongols.} Infobase Publishing. New York: Chelsea House, 2009. 

\bibitem{chap14-key5} Donkin, Robin A. \textit{Beyond Price: Pearls and Pearl-fishing: Origins to the Age of Discoveries.} Philadelphia: American Philosophical Society, 1998.

\bibitem{chap14-key6} Dowson, John. \textit{The History of India, as Told by its Own Historians: Edited from the Posthumous Papers of the Late Sir H. M. Elliot: Volume III.} London: Trubner \& Co., 1871.

\bibitem{chap14-key7} Dowson, John. \textit{The History of India, as Told by its Own Historians: Edited from the Posthumous Papers of the Late Sir H. M. Elliot: Volume IV.} London: Trubner \& Co., 1872.

\bibitem{chap14-key8} Dowson, John. \textit{The History of India, as Told by its Own Historians: Edited from the Posthumous Papers of the Late Sir H. M. Elliot: Volume VI.} London: Trubner \& Co., 1875.

\bibitem{chap14-key9} Elliot, Henry Miers. \textit{The History of India, as Told by its Own Historians: The Muhammadan period.} Ann Arbor: University of Michigan, 1953.

\bibitem{chap14-key10} Elliot, Henry Miers, and John Dowson. \textit{The History of India. As Told by its own Historians: The Mohammadan Period. Volume 2.} Charleston: Nabu Press, 2010.

\bibitem{chap14-key11} Habib, Irfan. “Economic History of the Delhi Sultanate: An Essay in Interpretation.” \textit{Indian Historical Review} 4, no.\ 1 (June 1978): 287--302.

\bibitem{chap14-key12} Hai, Maulana Hakim Saiyid Abdul. \textit{Hindustan Islami Ahad Mein (Hindustan under Islamic rule)}. Translated by Maulana Abdul Hasan Nadwi. Nadwatul-Ulama, 1973.

\bibitem{chap14-key13} Haig, Wolseley. \textit{The Cambridge History of India}. London: Cambridge University Press, 1928. 

\bibitem{chap14-key14} Holt, P. M., Ann K. S. Lambton, and Bernard Lewis, eds. \textit{The Cambridge History of Islam: The Indian Su-continent, Southeast Asia, Africa and the Muslim West} : Cambridge: Cambridge University Press, 1970. 

\bibitem{chap14-key15} Houtsma, E. J. \textit{Brill's First Encyclopaedia of Islam, 1913--1936, Volume 4}. Leiden: Brill, 2010.

\bibitem{chap14-key16} Hunter, William Wilson. \textit{The Indian Empire: Its Peoples, History, and Products}. New York: Routledge, 2013.

\bibitem{chap14-key17} Keay, John. \textit{India: A History.} New York: Grove Press, 2011. 

\bibitem{chap14-key18} Kulke, Hermann, and Dietmar Rothermund. \textit{A History of India}. $6^{\rm th}$ ed. New York: Routledge, 2016.

\bibitem{chap14-key19} Lawrence, Walter Roper. \textit{The Valley of Kashmir}. New Delhi: Asian Educational Services, 2005. 

\bibitem{chap14-key20} Levi, Scott C. “Hindus Beyond the Hindu Kush: Indians in the Central Asian Slave Trade.” \textit{Journal of the Royal Asiatic Society} 12, no.\ 3 (November 2002): 277--288.

\bibitem{chap14-key21} Nandakumar, Prema.  “Koil Ozhugu, Authentic Documentation of History.” \textit{The Hindu}.  Accessed on February 12, 2018. \url{http://www.thehindu.com/news/cities/Tiruchirapalli/koil-ozhugu-authentic-documentation-of-history/article2774682.ece.}

\bibitem{chap14-key22} Narasimhachary, M. \textit{ Śrī Vedānta Deśika}. New Delhi: Sahitya Academi, 2004. 

\bibitem{chap14-key23} Rao, V. N. Hari, and V. M. Reddi. \textit{ History of the Śrīrangam Temple}. Tirupati: Sri Venkateswara University, 1976. 

\bibitem{chap14-key24} Raychaudhuri, Tapan, and Irfan Habib. \textit{Cambridge Economic History of India: Volume 1}. Cambridge: Cambridge University Press, 2004.  

\bibitem{chap14-key25} Renganathan, L. “Regal Glorification for Lord Ranganatha at Srirangam.” \textit{ The Hindu}. Accessed February 12, 2018.  \url{http://www.thehindu.com/news/cities/Tiruchirapalli/regal-glorification-for-lord-ranganatha-at-srirangam/article4347622.ece.} 

\bibitem{chap14-key26} Smite, Vincent Arthur (1923). \textit{The Oxford History of India, from the Earliest Times to 1911}. Oxford: Oxford University Press, 1923. 

\bibitem{chap14-key27} Welch, Anthony, and Howard Crane. “The Tughluqs: Master Builders of the Delhi Sultanate.” \textit{Muqarnas: An Annual on Islamic Art and Architecture}  I (1983): 123--166.
\end{thebibliography}

\begin{longtable}{|>{\raggedleft}p{1.5cm}|p{8.5cm}|}
\multicolumn{2}{c}{\textbf{Table: 3}}\\ 
\hline
\textbf{Page \#} & \textbf{McGraw Hill Text} \tabularnewline
\hline
150 & The Mughal Empire ruled India from 1526 to 1857. \tabularnewline
\hline
\end{longtable}

\section*{Analysis and Critique} 

This is in violation of evaluation criterion clause pertaining to accuracy.

It is inaccurate to state that the Mughal Empire ruled India over this entire period. They only ruled over India during their peak during the reign of Aurangzeb (1658--1707). The Mughal empire was also briefly interrupted by the Sur Empire (1540--1555). Besides, many Rajput kingdoms in earlier Rajputana and current Rajasthan as well as certain kingdoms in the South never came under the Mughal rule. Besides, there were the Marathas in the western part of India, who between 1674 and 1818 controlled most parts of the Indian subcontinent. 

\begin{thebibliography}{99}
\bibitem{chap14-key28} Richards, John F. \textit{The Mughal Empire.} Cambridge: Cambridge University Press, 1993.
\end{thebibliography}

\begin{longtable}{|>{\raggedleft}p{1.5cm}|p{8.5cm}|}
\multicolumn{2}{c}{\textbf{Table: 4}}\\ 
\hline
\textbf{Page \#} & \textbf{McGraw Hill Text} \tabularnewline
\hline
169 & Emperors such as Aurangzeb prohibited the building of Hindu temples and even forced Hindus to convert to Islam.\tabularnewline
\hline
\end{longtable}

\section*{Analysis and Critique} 

This violates the evaluation criterion pertaining to accuracy, and by whitewashing the Indian history in favor of Islam, it violates Education Codes related to adverse reflection and indoctrination against Hinduism (51501 and 60044). As the following discussion will show, Islam was also highly destructive towards universities, temples, and all institutions where Hinduism, Buddhism, and Jainism were nurtured. By whitewashing this history of destruction, the McGraw Hill also violates Education Codes (51501, 60040b, and 60044a) involving non-Muslim Indian Americans as an Ethnic and Cultural Group by not instilling pride in their heritage. Further by not telling the history of destruction of their heritage, it adversely reflects on it.

This is incorrect. Aurangzeb did not just force Hindus to convert to islam and prohibit building of Hindu temples. The reign of Aurangzeb witnessed one of the strongest campaigns of religious violence in the Mughal Empire's history. Aurangzeb banned Diwali, re-introduced jizya (tax mandated by Koran on infidels) on non-Muslims, led numerous campaigns of attacks against non-Muslims, forcibly converted Hindus to Islam and destroyed Hindu temples.

Aurangzeb issued orders in 1669, to all his governors of provinces to “destroy with a willing hand the schools and temples of the infidels, and that they were strictly enjoined to put an entire stop to the teaching and practice of idolatrous forms of worship.”\footnote{Vincent Smith, \textit{The Oxford History of India} (Oxford: Oxford University Press, 1919), 437.} These orders and his own initiative in implementing them led to the destruction of numerous temples, contributing to the list of temples destroyed during Islamic rule of India. Some temples were destroyed entirely; in other cases mosques were built on their foundations, sometimes using the same stones. \textit{Murtis} or icons (mistranslated as idols) in temples were smashed, and the city of Mathura was temporarily renamed as Islamabad in local official documents. Historian Matthew White claims an estimated 4.6 million people were killed under his reign. 

\begin{thebibliography}{99}
\bibitem{chap14-key28a} Avari, Burjor. \textit{Islamic Civilization in South Asia: A History of Muslim power and Presence in the Indian subcontinent.} New York: Routledge. 2013.

\bibitem{chap14-key29} Batabyal, Rakesh. \textit{Communalism in Bengal: From Famine to Noakhali}, \textit{1943--47}. Thousand Oaks: Sage, 2005. 

\bibitem{chap14-key30} Braudel, Fernand. \textit{A History of Civilizations}. Translated by Richard Mayne. New York: Penguin Books, 1995.

\bibitem{chap14-key31} Chakrabarty, Bidyut. \textit{Partition of Bengal and Assam, 1932--194}. New York: Routledge. 2004. 

\bibitem{chap14-key32} Chatterji, Joya. \textit{Bengal Divided: Hindu Communalism and Partition, 1932--1947}. Cambridge: Cambridge University Press, 2002.

\bibitem{chap14-key33} Eaton, Richard M. “Temple Desecration and Indo-Muslim States.” \textit{Journal of Islamic Studies} 11, no.\ 3 (2000): 283--319.

\bibitem{chap14-key34} Fraser, Bashabi. \textit{Bengal Partition Stories: An Unclosed Chapter}. New Delhi: Anthem Press. 2008. 

\bibitem{chap14-key35} Prabhu, Alan Machado. \textit{Sarasvati's Children: A History of the Mangalorean Christians}. Bangalore: I.J.A. Publications, 1999.

\bibitem{chap14-key36} Talbot, Cynthia. “Inscribing the Other, Inscribing the self: Hindu-Muslim identities in Pre-Colonial India.” \textit{Comparative Studies in Society and History} 37, no.\ 4 (October 1995): 692--722.
\end{thebibliography}
