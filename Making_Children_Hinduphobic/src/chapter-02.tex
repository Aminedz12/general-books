\chapter{Unfolding the Issues}
%~ \addcontentsline{toc}{chapter}{\uppercase{\chaptername\ TWO: Unfolding the Issues}}

In the previous chapter, we gave an overview of the problems that we have identified with the concerned McGraw Hill texts. We will now go into greater details discussing the issues, distortions, and California Education Codes violations. In this chapter, we will give a summary of the issues identified in the grade six and grade seven texts so that in subsequent chapters we graduate towards discussing the problematic texts put together in a tabular form. The tables have been created around clusters that refer to one particular problem so that they could all be addressed under “analysis and critique.”  
\vskip -10pt

\section*{These textbooks violate California Education Codes 51501 and 60044—adverse reflection.} 
\vskip -6pt

The narrative in the grade-six textbook, when it comes to Ancient India and Hinduism, is deeply flawed. It has egregious violations of California Education Codes and results in content that reflects adversely on Hinduism and India and serves to indoctrinate students in racist and orientalist perspectives.

The narrative on the Harrapan civilization ignores the last 50 years of scholarship. It uses antiquated terminology “Indus Valley Civilization” instead of the Framework mandated Harappan Civilization or Indus Civilization. The text discusses two major cities (Mohenjo-Daro and Harappa) ignoring the other four major cities that have been discovered: Rakhigarhi, Kalibangan, Dholavira, and Lothal. McGraw Hill ignores findings that connect this civilization to modern day Hinduism by ignoring most of the archeological finds over the intervening five decades: figurines showing the traditional Hindu greeting Namaste, yogic poses compiled in the Yoga sutras of Patanjali, fire alters used in Vedic ceremonies, seals depicting the Hindu Deity Shiva, iconic representation of Shiva in the form of \textit{Śivaliṅga}  (in most Indian temples it is the iconic representation of Shiva that is used for worship even today). The above show the Vedic and indigenous roots of the Harappan culture (and many of which are required by the HSS Framework) rather than what the text argues for. It further pushes the “Aryan Migration Theory” as fact, which at best could be considered hotly contested and at worst racist, orientalist, antiquated and disproven. The text makes factually inaccurate statements regarding the mores and the values of the people by citing phrases like “royal palaces and temples” indicating the “importance of both religion and government” when in fact one of the things that have surprised archeologists is the absence of these two types of buildings. This entire narrative violates California Education Codes 51501 and 60044—adverse reflection.  

This section also has a historically well-accepted narrative incorrect regarding the origins of the Vedas. It is widely accepted that the Vedas were developed in an oral tradition before they were written down. Per the textbook, the Aryan people developed a written language called Sanskrit and subsequently composed the Vedas. Sanskrit initially was an oral language which has had several scripts attributed to it over time. In the beginning, the Brāhmi script was used to write Sanskrit (other scripts were adopted at a later point in history). Narrative problems continue when the “Aryans” are shown to be distinct from the “Dravidians” or people from Southern India, who are stated to have a distinct culture and religion. All these theories, the Aryan Invasion Theory or the Aryan Migration Theory or the Aryan/Dravidian Theory are theories that are based on unscientific and outmoded Eurocentric nineteenth century race-based paradigms. These race-based theories have been given a fresh lease of life by racist Linguists, who essentially argue for the same race-based theories under politically correct paradigms. For instance, the race-based earlier Aryan/Dravidian divide is now being argued on linguistic lines. Similarly, the earlier Aryan Invasion Theory, in which the superior Aryan race—having its homeland in Europe (it is a different matter that over a period of time, almost all the European nationalities have claimed the Aryan homeland in their respective countries)—invading the original aborigines in the Indian subcontinent has now been replaced with the Aryan Migration Theory in which the Aryans, speaking a language having roots in the Indo-European group of languages, migrate to the Indian subcontinent by way of Iran. Further, Hinduism has developed over centuries (with clear linkages to the Harappan culture) and with contributions from many people from across the length and breadth of India. To state that it came from a mix of culture from the “Aryan” and “Dravidian” people is deeply problematic, in addition to perpetuating the discredited race-based paradigm. Overall, three pages are devoted to this topic and all that are written there are either discredited or hotly contested contentions. This section violates California Education Codes 51501 and 60044—adverse reflection on Hinduism and India.  

In the previous chapter, we have already discussed the conflation of Hinduism with Caste, and consequently with hierarchy and oppression. Since we have already discussed the above in the context of the violations of California Education Code sections 51501, 60040(b), 60044(a) and 60044(b), we will not dwell on the topic any further. We will, however, discuss this issue at length when we come to the tables below.  
%\vskip 2pt

The McGraw Hill takes a biased and skewed view to gender roles. The textbook minimizes the role of women in Hindu society and does not incorporate the umpteen instances where their status was much superior to women in comparison to some other ancient societies.  
%\vskip 2pt

They had similar education as men, and participated with men in philosophical debates. They were among the seers who are responsible for composing the Vedas. Some were \textit{brahmavadinis},  women who devoted their lives to scriptural study and many of them contributed directly to the scriptures (i.e., approximately twenty women authored parts of the Rig Veda); others received martial arts and arms training.
%\vskip 2pt

In the Vedic times, men and women were equal as far as education and religion were concerned. In addition, they could offer Vedic sacrifices to Deities on their own without the requirement of males around them. Women participated in public sacrifices alongside men. Panini distinguishes women preceptors and teachers versus the wife of male preceptors and teachers, clearly indicating that women could be either one of the two and not just the “wife.”\footnote{A. S. Altekar, \textit{Education in Ancient India} (New Delhi: Gyaan Books, 2010)} They could marry on their own, could get divorced, and remarry if widowed.\footnote{A. S. Altekar, \textit{The Position of Women in Hindu Civilization} (Delhi: Motilal Banarasidass, 2005), 4} We find widespread evidence of women education till the closing of the Vedic age. There continues to be evidence of women philosophers, authors, and poetesses, administrators, and queens (including governors or regents). “later declined with internal decline and external aggression (but that is not the period this section is meant to cover).
%\vskip 2pt

Instead, the narrative is interspersed with historically inaccurate statements with regards to where women could study (school vs. home), ability to take on a guru, etc. These treatments therefore violate California Education Code sections 51501 (fair representation of gender roles), 60040 and 60044 (adverse reflection).  
%\vskip 2pt

The text continues to distort and demean Hinduism by making false conjectures such as “people’s practice of religion was limited to rituals. Over time, … took on more meaning … and became a part of daily life,” implying that Hindu rituals in their inception were imaginative and meaningless incursions of non-rational people. Additionally, there are two related problems associated with the above: 1.\ Historically, ritual-based religions have been considered inferior to religions like Christianity and Islam. By conflating Hinduism to rituals primarily while simultaneously negating the profound metaphysics discussed in the Vedas right from the very beginning (metaphysics is discussed in the Upanishads and the Upanishads form the fourth section of the Vedas) the textbook, along with perpetrating distortion and falsehood on Hinduism, is also making it subtly inferior to other predominant religions of the world. 2.\ The statement subtly shows that Hinduism has roots in irrationality and mumbo jumbo—this is even more problematic than the first because Hindus were routinely told in the colonial times that their most sacred of the texts—the Vedas—are creations of irrational and savage people, who simply collected irrational and illogical mutterings into books. The above quote from the textbook is far more insidious than what appears because it is based in the paradigm of racial superiority of Europeans over Indians. To make matters worse, the text mentions “Brahmanism” which is a colonial term that attempts to divorce Ancient Hinduism from present day Hinduism. This term is also derogatory to Hindus. This violates California Education Codes 51501 and 60044—adverse reflection.  

The narrative on the Hindu marriage customs is also butchered. It describes one of eight types of marriage defined in Hindu scriptures and found in practice (to varying degrees).\footnote{See “Vivāha,” Hindupedia, the Hindu Encyclopedia, accessed 	February 22, \url{http://www.hindupedia.com/en/Vivāha}.} Instead of focusing on the types of marriage and values associated with marriage, the publisher choses to discuss the lack of divorce (which also is incorrect for if one looks up Kautilya’s \textit{Arthaśāstra}, written in fourth century BCE, one finds the conditions under which divorce could happen) and the age of marriage to create a derogatory description of a holy institution violating California Education Codes 51501 and 60044—adverse reflection.  

Hindu origins and beliefs are also massacred in this module. It states that Vedas had to be memorized by Brahmin priests and were later blended with ideas of other people of India which eventually became Hinduism. This statement is based on an antiquated colonial narrative as already discussed in the previous chapter. It is widely accepted that the Vedas were an oral tradition and its transmission was done by both ascetics/sages/rishis as well as Brahmins (these two categories are not the same—the Brahmins were part of the four-fold \textit{varṇa}  order whereas the ascetics, sages, and ṛṣi-s were individuals who had left the four-fold \textit{varṇa}  order to exclusively focus on finding realization in the Divine). Further, there was a continual evolution of Hindu thought that resulted in the creation of the Upanishads and other texts (numbering in the thousands) with contributions from hundreds if not thousands of people.\footnote{There is an alternate view that holds that the four section of the Vedas—the \textit{Saṁhita}, \textit{Brāhmaṇa}, \textit{Āraṇyaka}, and \textit{Upaniṣads} — were to be mastered in respectively four stages of life, \textit{Brahmacarya} (life of studentship and celibacy), \textit{Gṛhastha} (life of house holder), \textit{Vānaprastha} (life in which one gradually prepares to become a renunciate), and \textit{Sanyāsa} (life of a renunciate)} To say that the Vedas were mixed with other beliefs is patently incorrect and to arbitrarily state that the Vedas were not Hindu (only the later mixed beliefs became known as Hinduism) is similarly false and derogatory. The Vedas from the first one—the Ṛg Veda—has laid the foundation for diverse and plural beliefs to co-exist. As the textbook implies, Hinduism has not been crafted through the machinations of certain people who decided to synthetically bind numerous belief systems. The coexistence of various Hindu beliefs comes from this Ṛg-vedic idea that in this universe and beyond, there is nothing but One, and that One manifests in multiplicity and plurality. It makes derogatory and inaccurate statements that most ancient Indians could not easily understand the idea of Brahman and believed in many different Deities that were more “like people.” Such statements are false and denigrate both the intelligence of the ancient Hindus as well as make Hinduism anthropomorphic. This section tosses out terms from the HSS Framework without any discussion such as worship in home, temples, yoga, and meditation without any discussion or elaboration. Further, a similar lack of understanding can be found in the section on reincarnation, mokṣa, and karma. The text links these terms to provide a derogatory perspective on Hindu social structure to justify the publisher’s perspective that it is an elitist structure (“status in life is not an accident,” acceptance of the jāti system and belief that a higher jāti was superior and its members deserved their status,” etc.). This section violates California Education Codes 51501 and 60044—adverse reflection. We will elaborate on this section in significant detail down below.  

The introduction of Buddhism has adverse reflection on Hinduism stating that Indians were unhappy with Hinduism and wanted a simpler spiritual faith, which caused them to leave for the forests. One of those people became the Buddha. Not only is this derogatory but it is also historically inaccurate. As mentioned in the footnote below, the life of an ancient Hindu was divided in four stages: \textit{Brahmacarya}  (life of studentship and celibacy), \textit{Gṛhastha}  (life of house holder), \textit{Vānaprastha}  (life in which one gradually prepares to become a renunciate), and \textit{Sanyāsa}  (life of a renunciate). Leaving for the forest (\textit{Sanyāsa}) or intermittently visiting it in a stage earlier (\textit{Vānaprastha})  was a well-accepted practice in Ancient India. And at the same time, if one became consumed with the idea of finding one’s oneness with the Divine or finding the truths of one’s existence, one could step out of the four-stage arrangement at any stage of one’s life after having provided for the family. As described in the previous chapter, the Buddha underwent an “existential crisis,” and wanting to find the ultimate truths of ones’ existence set out to the forest. After a few years of intense spiritual practice, he found the Truth for himself, which he preached to the population that he came across traveling from one place to another on foot. In the ancient Indian way, he was simply a teacher in a long tradition of teachers, sages, and enlightened souls. The textbook description violates California Education Codes 51501 and 60044—adverse reflection.  

The section on the Mauryan Empire is littered with errors of omission and commission. It positions Alexander’s retreat from India as a turning back due to “homesick troops,” incorrectly positions the Mauryan Empire as India’s first empire, and describes Chandragupta as a cowardly despot. It ignores the role of Chanakya, his chief advisor and the author of the famed \textit{Arthaśāstra}  without whom any description of the Mauryan Empire is incomplete. It overstates the achievements of Ashoka and incorrectly emphasizes the role of Buddhism in changing his perspective. It also incorrectly attributes the public work efforts to Ashoka’s Buddhist beliefs when they had been ongoing for three generations starting with Chandragupta if not earlier. This violates the evaluation criteria of accuracy and “history is well told and based on current scholarship.”\footnote{“Criteria for Evaluating Instructional Materials: Kindergarten through Grade Eight,” in \textit{California History-Social Science Framework: Chapter 23}, California Department of Education, accessed February 13, 2018, \url{https://www.cde.ca.gov/ci/hs/cf/documents/hssfwchapter23.pdf}, 623} Given the implicit and blatant privileging of Buddhism over Hinduism, it also violates California Education Codes 51501 and 60044—adverse reflection.  

The narrative in the following section is deeply flawed as it simplifies the extent of literature to include epics that teach moral lessons and historical texts. It ignores the medical, mathematical, astronomical, philosophical, and other scientific treatises also authored during this era. It also ignores a rich tradition of texts that were centered on drama and fiction. This section is also filled with errors of omission and commission. It then butchers the summary of the Bhagavad Gītā and Rāmāyaṇa in a way that at best can be considered derogatory (two of Hinduism’s holiest scriptures which are ready and studied daily by millions of people around the world). This section violates California Education Codes 51501 and 60044—adverse reflection.

The choice of source materials (covering excerpts from the Bhagavad Gītā, “Laws of Manu,” i.e., Manu Smṛti, and Vālmīki Rāmāyaṇa) can hardly be called primary source materials. These are archaic English translations, poorly done with derogatory language. This section violates California Education Codes 51501 and 60044—adverse reflection.

The text creates a false narrative around arts and architecture implying that very little of ancient Indian art and architecture survives and that the clear majority of what remains is Buddhist in nature (rock carvings, Ashoka’s pillars, etc.). There are three problems here: 1.\ It ignores the existence of 1300 rock art sites with over a quarter of a million figures and figurines, the art of the Indus-Sarasvati Civilization, terracotta figurines dating up to the Mauryan era, sculpture of the Gupta period, temple art, etc. all of which can be connected to Hinduism (some of which is also related to Buddhism and Jainism). 2.\ Though it is true that only a little of art and architecture that existed in the entire Ancient India remains, it is still quite substantial—particularly in the Southern India and Rajasthan. Islamic empires did not rule these regions extensively and thus the architecture was not destroyed completely. The ruins attest to the fact that destruction of the art and architecture of Hindus, Buddhists, and Jains (they were all considered Hindus by the Islamic invaders) is considered a holy act in Islam, for the logical reason that Islam is against “idol worship” and given that Mohammed himself destroyed idols after the invasion of Kaaba, the invading Islamic armies considered it to be their religious duty to break “idols” and temples of the Hindus. 3.\ It is incorrect in that a vast majority of the Buddhist architecture survives. The truth is that the vast majority of the ones that survive are Hindu. The Buddhist shrines were mainly located in the regions where the Delhi Sultanate established its complete dominance leading to a complete eradication of pre-Islamic Hindu and Buddhist architecture in the region—one only finds, if any, ruins. The Arabic term for “idol-worship” is \textit{but-parasti} —the \textit{but}  (pronounced boot) in \textit{but-parasti}  comes from word Buddha. Such has been the Islamic hatred for destroying the Buddhist art that only about a decade ago, the Taliban tank bombed the Bamiyan Buddhas—finishing what was not fully accomplished in the medieval times, for the Bamiyan Buddhas were disfigured back then as well, possibly through the use of cannons. This section violates California Education Codes 51501 and 60044—adverse reflection through false representation of availability of Hindu art implying Hindus were incapable of protecting their art history vis-à-vis Buddhists.
\eject

Our review of the seventh-grade-textbook sections pertaining to Islam, India and Hindusim shows serious violations of California Education Codes 51501 and 60044. The overall narrative is deeply flawed and should be considered \textit{\textbf{Hinduphobic}}. It whitewashes, ignores, or falsely presents the genocide of Hindus in India by a large number of Muslim invaders over a multi-hundred-year period. Instead, it represents Islam as having grown by trade and that rulers “encouraged” conversion and “sometimes they forced conversion or added some extra taxes.” These are blatantly incorrect contentions if we look at the records that have been left by the historians or chroniclers residing in the courts of Sultans of the Delhi Sultanate and the emperors of the Mughal Empire. During the Sultanate period, the violence against and the persecution of non-Muslims in India has been of such staggering scale that Abraham Eraly in his book has titled it as the \textit{Age of Wrath}.\footnote{Abraham Eraly, \textit{The Age of Wrath: A History of the Delhi Sultunate} (New York: Penguin Books, 2015)}. William Durant, in the \textit{Our Oriental Heritage}  has the following to say:
\vskip -15pt

\begin{quote}
The Mohammedan Conquest of India is probably the bloodiest story in history. It is a discouraging tale, for its evident moral is that civilization is a precarious thing, whose delicate complex of order and liberty, culture and peace may at any time be overthrown by barbarians invading from without or multiplying within.\footnote{Will Durant, \textit{The Story of Civilization: Part 1: Our Oriental Heritage} (New York: Simon and Schuster, 1954), 459}.  
\end{quote}
\vskip -8pt
This is a view shared by a large number of scholars. Even Islamic biographers provide details of their conquest and slaughter of Hindus in a great deal of detail.  

In addition, while Muslims did propagate advanced mathematical and scientific ideas across the world, they were not their inventors for most, particularly the ones that have an Indian stamp. However, a reading of the text will indicate the exact opposite. All Hindu achievements are shown as Islamic inventions (i.e., Algebra). Even a portrayal of the sciences and mathematics based on purely Islamic sources would be more accurate than what is presented in this chapter. The inaccuracies result in both adverse reflection on Hinduism as well the heritage of Indian-American children, leading to indoctrination of school children about the superiority of other faiths vis-à-vis Hinduism.

We will now frame the problematic McGraw Hill text in a tabular form, organized around various themes, offering our analysis and critique.  

