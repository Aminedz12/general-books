\chapter{Distorting Buddhism}

\begin{longtable}{|>{\raggedleft}p{1.5cm}|p{8.5cm}|}
\multicolumn{2}{c}{\textbf{Table: 1}}\\
\hline
\textbf{Page \#} & \textbf{McGraw Hill Text} \tabularnewline
\hline 
260 & Siddhartha had lived a privileged life from day one. Born into a wealthy family, he wanted for nothing. He had money and riches. He had good looks and a beautiful family. Even so, there was something missing. The young man had a deep spiritual longing to understand the suffering of others and to seek inner peace. He knew that he could not do those things without leaving behind the life that he knew so well. Before embarking on his journey to find the meaning of life, Siddhartha prepared to say goodbye one last time.\tabularnewline
\hline
\end{longtable}
\vskip -10pt

\section*{Analysis and Critique} 
\vskip -5pt

This is a violation of evaluation criterion pertaining to historical accuracy.

Factually incorrect statement on Siddhartha's life! Siddhartha was brought up by his mother's younger sister, Maha Pajapati. By tradition, he is said to have been destined by birth to the life of a prince and had three palaces (for seasonal occupation) built for him. His father, said to be King Śuddhodana, wishing for his son to be a great king, is said to have shielded him from religious teachings and from knowledge of human suffering. The legend has it that astrologers at the time of his birth had predicted that his destiny as a great king could be interrupted and changed by encountering suffering, which would make him lead the life of a renunciate. His father thus wanted to protect him from every suffering and ensured that he was surrounded by the best things of life.

When he reached the age of 16, his father arranged his marriage to Yaśodharā. According to the traditional accounts, she gave birth to a son, named Rāhula. Siddhartha is said to have spent 29 years as a prince in Kapilavastu. 

At the age of 29, Siddhartha left his palace to meet his subjects. Despite his father's efforts to hide from him the sick, aged, and suffering, Siddhartha was said to have seen an old man. When his charioteer Channa explained to him that all people grow old, the prince went on further trips beyond the palace. On these he encountered a diseased man, a decaying corpse, and an ascetic. These depressed him, and he initially strove to overcome ageing, sickness, and death by living the life of an ascetic. Accompanied by Channa and riding his horse Kanthaka, Gautama quit his palace for the life of a mendicant.

\begin{thebibliography}{99}
\bibitem{chap7-key1} Narada. \textit{A Manual of Buddhism}. Taipei: Buddha Educational Foundation, 1992.

\bibitem{chap7-key2} Conze, Edward\textit{. Buddhist Scriptures}, London: Penguin, 1959.
\end{thebibliography}

\begin{longtable}{|>{\raggedleft}p{1.5cm}|p{8.5cm}|}
\multicolumn{2}{c}{\textbf{Table: 2}}\\ 
\hline
\textbf{Page \#} & \textbf{McGraw Hill Text} \tabularnewline
\hline 
264 & Dressed in a yellow robe, he traveled the country, stopping to meditate, or think deeply.\tabularnewline
\hline
\end{longtable}

\section*{Analysis and Critique} 

This statement violates the evaluation criterion pertaining to accuracy.

Meditation does not make one “think deeply.” It is defined as a practice where an individual operates or trains the mind or induces a mode of consciousness, either to realize some benefit or for the mind to simply acknowledge its content without becoming identified with that content, or as an end. In fact, meditation is undertaken to stop thinking completely and get to states of consciousness that are beyond mind and thought. 

\begin{thebibliography}{99}
\bibitem{chap7-key3} Lutz, Antoine, Heleen A. Slagter, John D. Dunne, and Richard J. Davidson. “Attention regulation and monitoring in meditation.” \textit{Trends in Cognitive Sciences} 12, no.\ 4 (April 2008): 163–169. DOI: \url{https://doi.org/10.1016/j.tics.2008.01.005}

\bibitem{chap7-key4} Singh, Kundan. \textit{The Evolution of Integral Yoga: Sri Ramakrishna, Swami Vivekananda, and Sri Aurobindo.} Saarbrucken, Germany: VDM Verlag, 2008.
\end{thebibliography}


\begin{longtable}{|>{\raggedleft}p{1.5cm}|p{8.5cm}|}
\multicolumn{2}{c}{\textbf{Table: 3}}\\ 
\hline
\textbf{Page \#} & \textbf{McGraw Hill Text} \tabularnewline
\hline 
EM61 & This statue of him in Delhi, India, shows his hands cupped to form an alms bowl for begging.\tabularnewline
\hline
\end{longtable}

\section*{Analysis and Critique} 

This is in violation of evaluation criterion pertaining to inaccuracy

This is a factually incorrect description. The “mudra” the Buddha is showing in this image is not representative of a begging bowl.
