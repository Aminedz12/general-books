\chapter{Hinduism is Caste, Hierarchy, And Oppression}

\begin{longtable}{|>{\raggedleft}p{1.5cm}|p{8.5cm}|}
\multicolumn{2}{c}{\textbf{Table: 1}}\\ 
\hline
\textbf{Page \#} & \textbf{McGraw Hill Text}\tabularnewline
\hline
258 & People were considered members of a varna based on their jobs and personal behavior, but mostly based on which varna they were born into. The most powerful varnas were the Brahmins (BRAH•mihns) and Kshatriyas (KSHA•tree•uhs). \tabularnewline
\hline
258 & Next were the Vaisyas (VYSH•yuhs), or commoners. \tabularnewline
\hline
258 & Below the Vaisyas came the Sudras (SOO•druhs). Sudras were manual workers and servants. Most Indians were in the Sudra varna. \tabularnewline
\hline
258 & By 500 C.E. or earlier there existed a community outside the jati system called the Dalits. Sometimes called the Untouchables, Dalits did work that jati Indians would not do, such as sanitation, disposal of dead animals, and cremation or burning of the dead. \tabularnewline
\hline
\end{longtable}
\vskip -10pt

\section*{Analysis and Critique} 
\vskip -4pt

This is in violation of California Education Codes 51501 and 60044—adverse reflection—and evaluation criteria clauses pertaining to historical inaccuracy, not including variety of perspectives and debates, perpetrating Eurocentric and Ethnocentric history, not remaining neutral in matters of religion (continuing with the missionary and imperialistic writings on Hinduism), and thereby instilling prejudice in Hindu and non-Hindu children against origins of Hinduism and Hinduism per se. 

The \textit{varṇa} system as described in the Ṛg Veda and expanded upon in the Bhagavad Gītā clearly mentions that the \textit{varṇa} of a person is determined by her or his temperament. There is no mention of birth-based classification of people during this period. The statement that Brahmins and Kṣatriya-s were the most powerful is also contentious. Kṣatriya-s were the kings/warriors and had physical strength. Brahmins were held in the highest esteem (which is different from power) because knowledge and its distribution were highly valued. Vaiśya-s were the business people. Śūdra-s were the workers. These four were considered four wings of the society and for an effective functioning of the society, it was important that the four wings coordinated effectively with one another. A due care was taken to see that there was not a concentration of physical strength, knowledge, or wealth in one institution or class of the society. Let us explain the above in a greater detail:
\vskip 3.2pt

The four-fold division of the society in terms of a \textit{varṇa} system, mentioned in the Ṛg Veda and Bhagavad Gītā, should not be equated with the caste system. But was this a birth-based rigid hierarchical system that this textbook makes it out to be? Our researched opinion is that it was not birth based in its inception, conception, and philosophical underpinnings and that it was not hierarchy, which was the defining feature of this system. \textit{The ancients lived in a different world and it is important to understand the world to represent them correctly.} 
\vskip 3.2pt

The ancients saw that there were people in society with different inner constitutions and for the larger good of the individual and the society, they felt that it was imperative that people took up vocations that were in tune with the inner constitution—their predominant psychological make up. The ancient Indian civilization from the Ṛg-Vedic times believed in two things: \textit{ŗta} and dharma. Dharma has many different levels of meaning, basically coming from the root word \textit{dhŗ},  which means to hold or to keep something together. To put it extremely briefly, its meaning includes responsibilities towards the society, members of a family, and oneself. When dharma is geared towards oneself, it means \textit{swadharama} comprising of two words—dharma and \textit{swa},  which means oneself. \textit{Ŗta} loosely translated means the flow—it can be loosely compared with the Chinese Tao. 
\vskip 3.2pt

Accordingly, the ancients sought alignment of different kinds to dharma with \textit{ŗta} and therefore it was important for them to align \textit{swadharma} or inner constitution with \textit{ŗta}. And they held that one honors \textit{swadharma} if one takes a vocation according to one’s inner constitution. Now, inner constitutions of people vary and hence they felt that it was imperative that people had different vocations. They mainly divided the inner constitutions into four categories and assigned different tasks for different people in the society. \textit{Bṛhadāraṇyaka} Upanishad clearly links inner constitution and actions with \textit{varṇa}. 

Even today, we have come to realize that people who have vocations that are in tune with their inner aspirations are some of the happiest people, who contribute in many positive ways to the society. 

Their emphasis was less on hierarchy and more on diversity and plurality—the importance of which the current consciousness is increasingly realizing. So, saying that the Ṛg-Vedic people instituted a rigidly hierarchical system is far from truth. In fact, the \textit{Bṛhadāraṇyaka} Upanishad clearly upsets the hierarchical description of Brahmins at the top and Śūdra-s at the bottom (see Sharma 2017 for details). That the Indian society became hierarchical over a period is because of many historical factors, which we will discuss down below. There are other profound and complex issues associated with the binary of equality and hierarchy, which we will avoid for keeping the complexity of this topic simple.\footnote{For those interested in exploring the discussion where the binary of hierarchy and equality surrounding the \textit{varṇa} 	system, as described in the Vedas and Upanishads, gets problematized and dissolved, we refer them to the work of Arvind Sharma (2017).}

That this system was not birth-based and that it had direct bearing and relevance to inner constitution can be understood more by looking at the \textit{Bhagavad Gītā} and the philosophy of Sāṁkhyā, which the Gītā is stated to have integrated within its fold. A śloka in the fourth chapter of the \textit{Bhagavad Gītā} states:

\begin{quote}
\textit{cāturvarṇyam mayā sṛṣṭam guṇa karma vibhāgaśaḥ} 
\end{quote}
Sri Krishna (as an incarnation of the divine) addressing Arjuna says the above, which is translated as follows: “I have created the four \textit{varṇa-s} through the classification of the \textit{guṇa-s} and the duties.”

Therefore, there is no doubt whatsoever that Hinduism ratifies the \textit{varṇa} system. The \textit{varṇa} system, however, does not have connections with heredity but has explicit connections with \textit{guṇa-s},  which have direct linkages with the inner constitution and psychological make up of an individual. And to understand the \textit{guṇa-s},  we will need to refer here to the Sāṁkhyā philosophy. 

According to this philosophy, the psychological make up of individuals is based on the interaction of twin principles: Puruṣa and Prakṛti. The Puruṣa essentially is a witness consciousness that supports and gives assent to the workings of the Prakṛti. However, in the ordinary human existence the Puruṣa becomes subject to the workings of Prakṛti and loses its witness capacity. 

The Prakṛti  has three modes of operation, the three \textit{guṇa-s} or qualities—\textit{sattva},  \textit{rajas},   and \textit{tamas}. \textit{Sattva} essentially is a force of equilibrium that translates in the human qualities like seeking harmony, happiness, and light. \textit{Rajas} is a force of kinesis that translates into qualities like passion, action, and effort. \textit{Tamas} is a force that translates in qualities like sloth, laziness, inertia, and inaction. These three \textit{guṇa-s} of Prakṛti  are in a flux; however, the predominance of one over the others determines the \textit{varṇa} of the individual


The person who has the predominance of \textit{sattva} is someone who will seek knowledge, who will seek to understand the secrets of existence, who will seek to know Brahman (\textit{Braḥma}), the ultimate reality; consequently, he or she belongs to the \textit{varṇa} of Brahmin (\textit{Braḥmaṇ}). People who have the predominance of \textit{rajas} will be driven by will-to-power, desire, and action—they form the \textit{Kṣhatriya} clan who by their very disposition will be rulers, politicians, and statesmen or stateswomen. The person with the mixture of \textit{rajas} and \textit{tamas} belongs to the \textit{Vaiśya} category, who by his or her orientation will be driven towards making money. The person with the predominance of \textit{tamas} is \textit{Śudra},  and such a person by his or her nature will be guided by inertia, lack of effort, laziness, inaction, and sloth. A person who goes beyond the three \textit{guṇa-s} is called \textit{triguaṇātita},  and he or she is essentially a yogi or a mystic—therefore a yogi has no \textit{varṇa}. “Traditional” India does not inquire about the \textit{varṇa} of a yogi even today. The Mahābhārata and the Śukranīti also link \textit{varṇa} to the actions or the vocations that one performs. In addition, \textit{varṇa} comes from the root word \textit{vri} which means (1) to choose, and (2) to surround or cover. \textit{Varṇa} essentially means a group that one chooses to belong.

That this system was primarily geared towards promoting unity-in-diversity within the society can be seen from its practice that it left wealth in the hands of \textit{vaiśya-s},  politics and governance in the hands of the \hbox{\textit{Kṣatriya-s}},  knowledge in the hands of the Brahmins, and service in the hands of the \textit{Śūdra-s}. All these different categories of people were supposed to work in tandem for the larger good of the society. The deterioration of the system began when it began to become hereditary and birth-based. However, the Hindu or Vedic religion never intended it to become birth-based and hereditary. The deterioration of the system again is due to internal decline and external aggression. There is nothing inherent in the tradition for it to acquire the characteristics that it acquired over a period but for this the religion cannot be faulted.
\eject

As for the fluidity of the social class system in the Vedic period, see the following:

\begin{itemize} 
\item I am a reciter of hymns, my father is a doctor, my mother a grinder of corn. We desire to obtain wealth in various actions.\footnote{Ṛgveda 9.112.3}
\item Indra, fond of Soma, would you make me the protector of people, or would you make me a ruler, or would you make me a Sage who has consumed Soma, or would you bestow infinite wealth to me?\footnote{Ṛgveda 3.44.5} 
\end{itemize}

From the second verse, it is clear without a doubt that the seeker as he/she is invoking Indra is wondering whether he/she would be become a \textit{kṣatriya} (protector of people or ruler), sage or \textit{triguṇātīta},  or a \textit{vaiśya} (a person who acquires a lot of wealth). If his/her \textit{varṇa} were fixed based on birth, then such a prayer would not have happened. Similarly in the first verse, if the \textit{varṇa} or \textit{jāti} was fixed by birth, then the composer of the hymn would have been a doctor, not a reciter of hymns. 

The Ṛgveda has 10,552 mantras but only 1 mentions all the four \textit{varṇa-s} simultaneously, and not more than 20 mantras (0.2\%) mention the different \textit{varṇa-s} individually. The Saamaveda has even a lower percentage of its 1875 mantras dealing with “caste.” The Yajurveda in all its recitations has very few (less than 3--4\%) portions dealing with “caste.” The Atharvaveda with almost 6,000 mantras (or 8,000 in the \textit{Paippalāda} version) likewise has very few references to \textit{varṇa}. Ignoring the above emic perspective on how the \textit{varṇa} system has been described and understood philosophically and by taking a few scattered references of \textit{varṇa} in the Vedas, the McGraw Hill essentializes the narrative of Hinduism around caste and goes to every length to show that Hinduism is fundamentally a regressive religion which has hierarchy and oppression written all over it. 

Further, in the ancient times in India, the untouchables were the people, who were banished from the mainstream of the society and were asked to live on the fringes of a settlement—whether rural or urban. They were found to have committed heinous crimes (such as incest) and in gross violation of the social mores of the times. Instead of imprisoning this category of people, the ancient Indian society felt more comfortable in ostracizing them and limiting their contacts with the mainstream. The “untouchables” were called \textit{antāvasāyin},  which literally means one who lives on the periphery of a settlement (Kautilya 1992, 35)

When the above explanation is not given, untouchability and oppression become endemic to Hinduism. Besides, none of the spiritual texts like the Vedas, Upanishads, or Bhagavad-Gītā speak about or ratify the practice of untouchability. The practice of untouchability had a social sanction, given the context and mores of the times, just like Islam recommended the killing of infidels, “idol-worshippers,” and non-believers and Christianity sanctioned slavery. If such obnoxious practices of Christianity and Islam are not included in McGraw Hill’s descriptions, singling out Hinduism for oppression and discrimination is akin to practicing discrimination against it. 

The narrative on Hinduism in that it is inherently hierarchical, inflexible, and oppressive was created by the missionaries and imperialists for the stated goals of converting and civilizing Hindus respectively, and that is why they never took into account the complex discussion around Upanishads, Brāhmaṇa-s, Sāṁkhyā, Gītā, Varṇa, Guṇa, etc. As Sharma (2017) discusses in significant detail, it was done with an us vs. them mentality, where the explicit aim was to create the description of \textit{Homo Equalis} for themselves (Europeans) at the expense of the characterization of \textit{Homo Hierarchicas} for the Indians. The McGraw Hill reproduces such discourse, and in doing so it not only is complicit in creating adverse reflection on Hinduism but also is complicit in indoctrinating students against it.

\begin{longtable}{|>{\raggedleft}p{1.5cm}|p{8.5cm}|}
\multicolumn{2}{c}{\textbf{Table: 2}}\\ 
\hline
\textbf{Page \#} & \textbf{McGraw Hill Text}\tabularnewline
\hline
258 & Priests, leaders, and other elites used religion to justify their high place in society. \tabularnewline
\hline
\end{longtable}

\section*{Analysis and Critique} 

This is in violation of California Education Codes 51501 and 60044—adverse reflection—and evaluation criteria clauses pertaining to historical inaccuracy, not including variety of perspectives and debates, perpetrating Eurocentric and Ethnocentric history, not remaining neutral in matters of religion (continuing with the missionary and imperialistic writings on Hinduism), and thereby instilling prejudice in Hindu and non-Hindu children against origins of Hinduism and Hinduism per se.

An example of how the narrative on Hinduism links it to caste, hierarchy and oppression. The text is blatant in suggesting collusion between priests and kings for status and control (the inference of oppression is a natural corollary).

\begin{longtable}{|>{\raggedleft}p{1.5cm}|p{8.5cm}|}
\multicolumn{2}{c}{\textbf{Table: 3}}\\ 
\hline
\textbf{Page \#} & \textbf{McGraw Hill Text}\tabularnewline
\hline
258 & Scholars refer to the jati system as a caste (KAST) system. In such a system, people remain in the same social group for life. \tabularnewline
\hline
\end{longtable}

\section*{Analysis and Critique} 

This is in violation of California Education Codes 51501 and 60044—adverse reflection—and evaluation criteria clauses pertaining to historical inaccuracy, not including variety of perspectives and debates, perpetrating Eurocentric and Ethnocentric history, not remaining neutral in matters of religion (continuing with the missionary and imperialistic writings on Hinduism), and thereby instilling prejudice in Hindu and non-Hindu children regarding origins of Hinduism and Hinduism per se.

This section confuses the term jāti with caste and when in history the term came into vogue. There was no such thing as caste in the Middle Ages as the Portuguese introduced the caste terminology in the 18th century, given that they could not understand the complex social structure they found in India (spanning varṇa and jāti).

The term caste is not even an Indian word. According to the Oxford English Dictionary, it is derived from the Portuguese \textit{casta},  meaning “race, lineage, breed” and, originally, “pure or unmixed (stock or breed).” There is no exact translation in Indian languages, but varṇa and jāti are the two most proximate terms. Caste and cast have similar roots, and therefore the European experience and understanding of caste is derived from cast, which elucidates, defines, and characterizes fixity. 

Therefore in the issue of the European description of the caste system, there are a couple of things that are involved: 1. When the European contact began with India, they projected their own experience with division-based society (caste was widely prevalent in Europe at that point in time) onto India. 2. When Britain gained control over India, through a systematic application of measures turned the Indian society into a rigid “caste” based society (for details see Dirks 2001).

With regards to the inherent inflexibility of the Hindu social structure, the text makes absolutely an incorrect statement. As we mentioned earlier, \textit{varṇa} comes from the root word \textit{vri} which means (1) to choose, and (2) to surround or cover. \textit{Varṇa} essentially means a group that on choses to belong. Forget about ancient India, even in the last century, after so much decline that the Hindu society suffered through Islamic invasions and British imperialism, the jāti system has not been inflexible. It is important that M.N. Srinivas’s seminal work \textit{Social Change in Modern India} is a must read in this regard. \textit{Jāti-s} or social groups have moved around the “caste” scale even during the time period in which the Hindu cosmology came under severe attack by the invaders, whose part of the colonial agenda was also to suppress the Hindu worldview. 

The McGraw Hill description is an adverse reflection on Hinduism stating that it is hierarchical, inflexible, and oppressive religion. The colonial and missionary writings were heavily invested in creating this impression, and McGraw Hill text reproduces such representation. 

\begin{thebibliography}{99}
\bibitem{chap5-key1} Corbridge, Stuart, John Harris, and Craig Jeffery. \textit{India Today: Economy, Politics and Society}. New York: Polity Press, 2012. 

\bibitem{chap5-key2} Dirks, Nicholas. \textit{Castes of Mind: Colonialism and the Making of Modern India}. Princeton: Princeton University Press, 2001.

\bibitem{chap5-key3} Jaiswal, Suvira. \textit{Caste: Origin, Function and Dimensions of Change}. New Delhi: Manohar Books, 1998.

\bibitem{chap5-key4} Maheshwari, Krishna. \textit{Theory of Varna}. Hindupedia, the Hindu Encyclopedia. Accessed February 13, 2018. \url{http://www.hindupedia.com/en/Theory_of_Varna}.

\bibitem{chap5-key5} Sharma, Arvind. \textit{The Ruler’s Gaze: A Study of British Rule over India from a Saidian Perspective}. London: HarperCollins \textit{Publishers}.

\bibitem{chap5-key6} Srinivas, M. N. \textit{Social Change in Modern India}. Berkeley: University of California Press, 1966.
\end{thebibliography}

\begin{longtable}{|>{\raggedleft}p{1.5cm}|p{8.5cm}|}
\multicolumn{2}{c}{\textbf{Table: 4}}\\ 
\hline
\textbf{Page \#} & \textbf{McGraw Hill Text} \tabularnewline
\hline
258 & Higher classes came to be seen as purer than lower ones. \tabularnewline
\hline
\end{longtable}

\section*{Analysis and Critique} 

This is in violation of California Education Codes 51501 and 60044—adverse reflection—and evaluation criteria clauses pertaining to historical inaccuracy, not including variety of perspectives and debates, perpetrating Eurocentric and Ethnocentric history, not remaining neutral in matters of religion (continuing with the missionary and imperialistic writings on Hinduism), and thereby instilling prejudice in Hindu and non-Hindu children against origins of Hinduism and Hinduism per se.

The \textit{varṇa} system, as must be evident from the above discussion, is very complex and it cannot be understood in the simple binary of pure/impure just like it cannot be understood in the binary of hierarchy/equality. The entire social system of ancient Hindus was geared towards a gradual god realization, culminating in mokṣa; however, it took into account people’s difference in constitution and created categories where people could take up vocations and spiritual practices in accordance with their inner constitution. From the latter perspective, no \textit{varṇa} was better or worse. It is therefore that in the Gita, Sri Krishna in chapter 3, verse 35 says the following: 

\begin{quote}
\textit{śreyān sva-dharmo viguṇaḥ}\\
\textit{para-dharmāt svanuṣṭhitāt}\\
\textit{svadharme nidhanaṁ śreyaḥ}\\
\textit{para-dharmo bhayā vahaḥ}
\end{quote}
Translation: It is better to follow one’s own \textit{svadharma}  [following one’s inner constitution based on the predominant guṇa that one has] however faulty it may be and die doing it rather than doing someone else’s \textit{swadharma}, however perfectly done it may be. Actions pertaining to the \textit{swadharma} of others are fraught with danger. 

And at the same time, it was held that the predominance of  \textit{sattva}   which made an individual fall into the category of  \textit{brahmin}   enhanced the possibility of gaining  \textit{mokṣa}   (god realization or liberation). But again here is a catch: In order to gain  \textit{mokṣa}, one had to go beyond not only the  \textit{sattva}  guṇa but all guṇa-s, including  \textit{rajas}  and  \textit{tamas}. A yogi therefore, as we mentioned earlier is called  \textit{triguṇātīta} or someone who has gone beyond all the three \hbox{guṇa-s}. Therefore everyone, whether one had a predominance of tamas, rajas, or sattva,    meaning a śūdra, kṣatriya, or brāhmaṇa, had an equal chance of becoming a yogi, who was willing to transcend the guṇa-s through spiritual practices. It is therefore one finds yogis right from the very beginning of the ancient Indian society coming from all the  \textit{varṇa-s}. So let us reiterate here: the  \textit{varṇa}   system cannot be understood from the simple binary of pure/impure or hierarchy/equality. It is far more complex than how the McGraw Hill text has presented it. This simplistic presentation results in adverse reflection on Hindu practices and cosmology. 

\begin{longtable}{|>{\raggedleft}p{1.5cm}|p{8.5cm}|}
\multicolumn{2}{c}{\textbf{Table: 5}}\\ 
\hline
\textbf{Page \#} & \textbf{McGraw Hill Text} \tabularnewline
\hline
258 & Many customs evolved to keep different groups from socializing with one another. This kept social groups largely separate from one another in daily life. \tabularnewline
\hline
\end{longtable}

\section*{Analysis and Critique} 

This is in violation of California Education Codes 51501 and 60044—adverse reflection—and evaluation criteria clauses pertaining to historical inaccuracy, not including variety of perspectives and debates, perpetrating Eurocentric and Ethnocentric history, not remaining neutral in matters of religion (continuing with the missionary and imperialistic writings on Hinduism), and thereby instilling prejudice in Hindu and non-Hindu children against origins of Hinduism and Hinduism per se.

This is inaccurate. All four varṇa-s and myriads jāti-s provided services to members of other groups. There was a tremendous mutual-interdependencies across groups. It may be worthwhile to quote Sharma (2017) \textit{en bloc} here: 

\begin{quote}
According to the Brāhmaṇa-s, the Ratnis played a key role in the investiture of the king and “one of the Ratnis was always a Shudra” in the description of such a ceremony in later times, in the Nītimayūkha of Nilakantha, all the four varṇa-s participated in the consecration of the king; sudras were present at the coronation of Yudhiṣṭhira, while four brāhmaṇa-s and three śūdra-s, along with others, are to hold ministerial positions according to the Śāntiparva of Mahābhārata; Maitrāyaṇī Saṁhita and Pañcavimśa Brāhmaṇa speak of wealthy sudras while “Shudras were members of the two political assemblies of the ancient times, namely the Janapada and Paura and as a member of the Shudra was entitled to special respect even by a Brahmin. (183)
\end{quote}
This is part of the larger racist, imperialistic, Eurocentric, and orientalist story that the jāti groups remained separate, and thus the evil Brahmins created a society where they could remain on the top while allowing for wholesale discrimination of one social group by the others. This is adverse reflection on Hinduism as it states that the belief that Hinduism is an elitist, hierarchical, and oppressive religion in which discrimination and prejudice is the norm.
\vskip -4pt

\begin{longtable}{|>{\raggedleft}p{1.5cm}|p{8.5cm}|}
\multicolumn{2}{c}{\textbf{Table: 6}}\\ 
\hline
\textbf{Page \#} & \textbf{McGraw Hill Text} \tabularnewline
\hline
262 & In Hinduism, the idea of reincarnation is closely related to another idea known as karma (KAHR•muh). According to karma, people’s status in life is not an accident. It is based on what they did in past lives. In addition, the things people do in this life determine how they will be reborn. \tabularnewline
\hline
\end{longtable}
\vskip -15pt

\section*{Analysis and Critique} 
\vskip -6pt

These are in violation of California Education Codes 51501 and 60044—adverse reflection—and evaluation criterion clause pertaining to accuracy. 

Making 1:1 correlation of reincarnation and karma is incorrect. They are not fully dependent concepts. Karma is a fundamental concept, which has no direct translation into English. It has multiple levels of applicability and can approximately be defined as the natural order of action and each action has an associated result (often called the fruit of karma). 

\noindent
\textit{Individual Karma} 

In the cycle of its evolution, the jīva has two movements—pravṛtti and nivṛtti. During pravṛtti impressions or saṁskāra-s are accumulated. One is recommended to do noble actions to reap their sweet fruits. During nivṛtti, one tries to get rid of prārabdha or accumulated karma and exhausts karma by experiencing its fruits (karma phala) to break the cycle of life and death. One is recommended to perform actions without attachment, so that its fruit or impression does not add to the baggage of one's own karma. When one performs detached actions, one only performs action as long as the previous karma phala is not nullified. One performs the highest kind of action at this stage, and such action always results in the benefit of surroundings (loka kalyāṇa). 

Mokṣa is through total karma nivṛtti and transcending the action-fruit cycle. This is possible if one realizes and discriminates between ātman and non-ātman (body, mind, etc.). One can get beyond the ambit of karma by experiencing the One beyond qualities (beyond triguṇa-s—sattva, rajas, and tamas).

Akarma is a state where an action is not bound by karma/phala. This is the kind of action performed by a liberated person. Akarma is not inaction, but sterilized action.

\noindent
\textit{Freewill} 

Fate and freewill both are significant in one's actions. While many factors like daivabala (destiny or God-will), prārabdha (one's own previous actions) affect the fruit of action, it is human effort (puruṣakara) that predominates action. Human is said to be the master of his actions (destiny), though not wholly the owner of the fruits of the actions.

\noindent
\textit{Group Karma} 

When a group of individuals do actions that affect each other, it results in a group karma. This could be a collectivity or persons closely attached to each other. In the latter case the group is called a group soul (yakśa). In the former, the persons do not get combined as a group soul but reap the fruit of collective action. This kind of karma drives the lifecycle of a society.

Besides, Karma is not foundational to Hinduism only. It is foundational to Buddhism and Jainism as well. Further this narrative is not innocuous. By explaining Karma in the manner it does and making it foundational to Hinduism, the textbook basically connects it to caste, which it then uses to substantiate hierarchy and oppression as a defining feature of Hinduism, and as we pointed out in the first chapter singles out Hinduism around the narrative of oppression and hierarchy.

Such is the bias and prejudice of the textbook writers that Buddhism and Jainism, which too have karma as a fundamental concept, are spared the treatment that Hinduism is given.

\begin{thebibliography}{99}
\bibitem{chap5-key7} Aurobindo, Sri: \textit{The Problem of Rebirth}. Puducherry: Sri Aurobindo Ashram Press, 2002.

\bibitem{chap5-key8} Aurobindo, Sri. \textit{Rebirth and Karma}. Puducherry: Sri Aurobindo Ashram Press, 2009.

\bibitem{chap5-key6a} Khandavalli, Shankara Bharadwaj. “Karma,” Hindupedia, the Hindu Encyclopedia. Accessed February 16, 2018. \url{http://www.hindupedia.com/en/Karma}.
\end{thebibliography}


\begin{longtable}{|>{\raggedleft}p{1.5cm}|p{8.5cm}|}
\multicolumn{2}{c}{\textbf{Table: 7}}\\ 
\hline
\textbf{Page \#} & \textbf{McGraw Hill Text} \tabularnewline
\hline
263 & Beliefs such as reincarnation also made many Indians more accepting of the jati system. A devout Hindu believed that the people in a higher jati were superior and deserved their status. At the same time, the belief in reincarnation gave hope to people from every walk of life. A person who leads a good life is reborn into a higher jati. \tabularnewline
\hline
283 & PREDICTING How might a belief in karma and jati influence the way a Hindu lives his or her life? \tabularnewline
\hline
\end{longtable}

\section*{Analysis and Critique} 

This is in violation of California Education Codes 51501 and 60044—adverse reflection—and evaluation criteria clauses pertaining to historical inaccuracy, not including variety of perspectives and debates, perpetrating Eurocentric and Ethnocentric history, not remaining neutral in matters of religion (continuing with the missionary and imperialistic writings on Hinduism), and thereby instilling prejudice in Hindu and non-Hindu children against origins of Hinduism and Hinduism per se. 

Concepts of reincarnation and jāti do not have a complete correlation with reincarnation pre-dating the concepts of jāti. The varṇa of the individual, from which the concept of jāti came about, has been closely entwined with his or her predominant constitution, derived from the \hbox{guṇa-s} –sattva, rajas, and tamas. These guṇa-s are discussed extensively in the Sāṁkhyā philosophy and mentioned in the Bhagavad Gītā. The predominance of one of the guṇa-s or the combination of two determined the temperament that an individual would be most predisposed towards. The idea was that an individual should engage himself or herself in a vocation, which was most suited to his or her temperament—current psychological research supports the harmony between the choice of a vocation and the natural temperament or predisposition of an individual. The rebirth of an individual is not dependent on the past karma alone but also on conditions that will lead to its evolution and closeness to the Divine—and these conditions are individual specific. There is no one rule that fits all. If harsh conditions would lead to the spiritual evolution of the individual, he or she would take birth in such conditions. If pleasant and rich conditions would lead to the spiritual evolution for another individual, he or she would be born in such conditions. Karma is not the sole determining factor in the conditions surrounding the birth of an individual, the trajectory of his or he spiritual growth is also a crucial factor (Sri Aurobindo 2009). 

None of the spiritual texts of Hinduism, called \textit{śruti-s},  support family-of-birth based varṇa system. The ossified jāti system in which the jāti of an individual became dependent on family of birth is a later development. The Muslim invasions of India, as suggested by Ronald Inden (1990), and interventions of the British, as discussed by Nicholas Dirks (2001), have had their own contributions towards this later development in Indian society. 

The textbook description mentioned in the table above exemplifies how Karma as a foundational concept in Hinduism is being conflated with hierarchy and consequently with oppression, and in doing so the publisher further substantiates the violations that we have pointed out in the beginning chapter. Not to belabor the point but showing Hinduism as hierarchical and oppressive justified the colonial and missionary intervention in India. Such misrepresentations on Hinduism, which completely negate how the Hindus have seen the tradition or how its texts have described the tradition, are the hallmark of this McGraw Hill text.

To sum up, through a discourse on caste and its conflation with karma, the text has shown Hinduism as hierarchical, inflexible, freedom denying, and oppressive. Karma is a foundational concept in Hinduism as it is in Buddhism and Jainism. By linking Karma to jāti, it is yet again solidifying its discourse on Hinduism as oppressive and hierarchical. Simultaneously, the text, as we will see down below, deals with karma in Buddhism differently. The representation of Hinduism as hierarchical and oppressive is the singular handiwork of British imperialists and Christian Missionaries, which the McGraw Hill text perpetuates from as many angles as possible.

\begin{thebibliography}{99}
\bibitem{chap5-key9} Aurobindo, Sri. \textit{Rebirth and Karma}. Puducherry: Sri Aurobindo Ashram Press, 2009.
\bibitem{chap5-key10} Dirks, Nicholas. \textit{Castes of Mind: Colonialism and the Making of Modern India}. Princeton: Princeton University Press, 2001.
\bibitem{chap5-key11} Inden, Ronald. Imagining India. Bloomington: Indiana University Press, 1990.
\end{thebibliography}
\newpage

\begin{longtable}{|>{\raggedleft}p{1.5cm}|p{8.5cm}|}
\multicolumn{2}{c}{\textbf{Table: 8}}\\ 
\hline
\textbf{Page \#} & \textbf{McGraw Hill Text} \tabularnewline
\hline
IJ165 & DETERMINING POINT OF VIEW — Discuss this excerpt with a partner. Whose point of view is being proclaimed? Does this suggest that there are other points of view to be considered? Use evidence from the text to support your answers. 2. How did the Laws of Manu impact the economy of the Sudra? Would there have also been an impact for the greater economy of India? Support your answer with details using the text and the excerpt. 3. ANALYZING TEXT How does the first sentence of instruction (91) contribute to the development of these rules for the Sudra? Explain, citing references in the text. 4. DETERMINING MEANING Based on the excerpt, what is the only hope of the Sudra who follows the Laws of Manu? Use details to support your answer. \tabularnewline
\hline
IJ168 & IDENTIFYING CAUSE AND EFFECT How did the Hindu belief in reincarnation contribute to people’s acceptance of the jati system? \tabularnewline
\hline
\end{longtable}

\section*{Analysis and Critique} 

These are in violation of California Education Codes 51501 and 60044—adverse reflection—and evaluation criterion clause pertaining to prohibition of role play, not remaining neutral in matters of religion, and thereby instilling prejudice in Hindu and non-Hindu children against Hinduism.

This question requires a student to role-play of topics related to religion, which is explicitly not allowed by the California Department of Education. The evaluation criteria require discussion to be clearly in the historical context (whereas the questions require a student to compare a religious view with present day US economic beliefs). The text explicitly explains varṇa and jāti as Hindu concepts and contends that the “Laws of Manu” come from Hindu beliefs (providing an excerpt from the Manu smṛti focused on śūdra-s as a “primary source”).

This is yet another example how the McGraw Hill text is heavily invested in creating a hierarchical and oppressive picture of Hinduism as it absolves other Abrahamic religions as well as Indian religions, for that matter, of any hierarchy or oppression. In singling out Hinduism for such discourse, it surrenders its neutrality towards religions, instilling prejudice in Hindu as well as non-Hindu children against Hinduism. 

\begin{longtable}{|>{\raggedleft}p{1.5cm}|p{8.5cm}|}
\multicolumn{2}{c}{\textbf{Table: 9}}\\
\hline
\textbf{Page \#} & \textbf{McGraw Hill Text} \tabularnewline
\hline
259 & Men attended school or became priests, while women were educated at home. Both men and women attended religious ceremonies and celebrations, but not as equals. \tabularnewline
\hline
\end{longtable}

\section*{Analysis and Critique} 

This is in violation of California Education Codes 51501 and 60044—adverse reflection—and evaluation criteria clauses pertaining to historical inaccuracy, not including variety of perspectives and debates, perpetrating Eurocentric and Ethnocentric history, not remaining neutral in matters of religion (continuing with the missionary and imperialistic writings on Hinduism), and thereby instilling prejudice in Hindu and non-Hindu children against origins of Hinduism and Hinduism per se.

These are inaccurate statements. The education to women was available without discriminating them against men. Women have authored parts of the Vedas. As quoted in the beginning of this chapter, the men and women participated in religious ceremonies as equal. The misportrayal of the position of women in ancient Hindu society was inspired by the agenda of the “white man’s burden” where the Jungian heroes wanted to liberate the colonized women in distress. It is true that position of women in Indian society was at an all-time low when the Europeans encountered the Indian society—thanks to the regressive practices undertaken by Delhi Sultans and Moghuls. However, for their own imperialistic and missionary agendas, the Europeans froze the representation of situation of women in Indian society and generalized it over a time period, which included their position in ancient India. These statements here apart from being inaccurate not only lack sufficient nuance but also are representative of Eurocentric, missionary, and colonial agenda. In as many ways as possible, the McGraw textbook authors are committed in showing that Hinduism since its inception is regressive, hierarchical, and oppressive having evolved structures where it could oppress majority of its own adherents and women. For further details, please see the following:

\begin{thebibliography}{99}
\bibitem{chap5-key12} Chaudhuri, J. B. \textit{The Position of Women in the Vedic Ritual}. New Delhi: Asian Educational Services, 1956

\bibitem{chap5-key13} Saraswati, Swamini Atmaprajnananda. \textit{Source for Rishikas: Rishikas of the Rigveda}. New Delhi: Kaveri Books, 2013.

\bibitem{chap5-key14} Shastri, S. R. \textit{Women in the Vedic Age}. Bombay: Bharatiya Vidya Bhavan, 1969.

\bibitem{chap5-key15} Upadhyaya, B. S. \textit{Women in Rgveda}. New Delhi: S Chand and Co., 1974.
\end{thebibliography}

\begin{longtable}{|>{\raggedleft}p{1.5cm}|p{8.5cm}|}
\multicolumn{2}{c}{\textbf{Table: 10}}\\ 
\hline
\textbf{Page \#} & \textbf{McGraw Hill Text} \tabularnewline
\hline
259 & In early India, boys and girls often married in their teens. People could not get divorced. \tabularnewline
\hline
\end{longtable}

\section*{Analysis and Critique} 

This is in violation of California Education Codes 51501 and 60044—adverse reflection—and evaluation criteria clauses pertaining to historical inaccuracy, not including variety of perspectives and debates, perpetrating Eurocentric and Ethnocentric history, not remaining neutral in matters of religion (continuing with the missionary and imperialistic writings on Hinduism), and thereby instilling prejudice in Hindu and non-Hindu children against origins of Hinduism and Hinduism per se. 

Social and religious commentary to create adverse reflection on marriages in Ancient India and Hinduism! In ancient India, marriages took place after men and women had been properly educated and were physically mature. In addition, divorce did happen in ancient India (see Kautilya 1992). This is another example in which this text shows how the ancient Hindus created an inflexible system, which lacked freedom. We want to ask a simple question, would any child want to associate with a religion whose narrative is built on inflexibility, rigidity, hierarchy and oppression? Besides, this narrative is wrong and incorrect for we scholars of the tradition know it and see it differently. 
