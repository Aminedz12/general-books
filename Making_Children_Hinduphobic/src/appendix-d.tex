\chapter{Appendix D: State Board Of Education Evaluation Criteria Map}

The California State Board of Education adopted and made available the “History-Social Science Evaluation Criteria Map” which was to be used in evaluating draft instructional materials (like McGraw’s) during the textbook revision process in 2017. It contains 5 categories with a number of sub-categories. However, all the criteria in Category 1 must be met in full for the program to be eligible for adoption. We have reproduced this document (modified for formatting below).

\textbf{Category 1:History–Social Science Content/Alignment with Standards} 

\begin{enumerate}
\item 
Instructional materials, as defined in Education Code Section 60010(h), support instruction designed to ensure that students master all the History–Social Science Content Standards for the intended grade level. Analysis skills of the pertinent grade span must be covered at each grade level. This instruction must be included in the student edition of the instructional materials; while there can be direction in materials for the teacher to support instruction in the standards, this cannot be in lieu of content in the student edition. The standards themselves must be included in their entirety in the student materials, either at point of instruction or collected together at another location.
\item 
Instructional materials reflect and incorporate the content of the History–Social Science Framework.
\item 
Instructional materials shall use proper grammar and spelling (Education Code Section 60045).
\item 
Instructional materials present accurate, detailed content and a variety of perspectives and encourage student inquiry.
\item 
History is presented as a story well told, with continuity and narrative coherence (a beginning, a middle, and an end), and based on the best recent scholarship. Without sacrificing historical accuracy, the narrative is rich with the forceful personalities, controversies, and issues of the time. Primary sources, such as letters, diaries, documents, and photographs, are incorporated into the narrative to present an accurate and vivid picture of the times in order to enrich student inquiry.
\item 
Materials include sufficient use of primary sources appropriate to the age level of students so that students understand from the words of the authors the way people saw themselves, their work, their ideas and values, their assumptions, their fears and dreams, and their interpretation of their own times. These sources are to be integral to the program and are carefully selected to exemplify the topic. They serve as a voice from the past, conveying an accurate and thorough sense of the period. When only an excerpt of a source is included in the materials, the students and teachers are referred to the entire primary source. The materials present different perspectives of participants, both ordinary and extraordinary people, in world and U.S. history, and further student inquiry.
\item 
Materials include the study of issues and historical and social science debates. Students are presented with different perspectives and come to understand the importance of reasoned debate and reliable evidence, recognizing that people in a democratic society have the right to disagree.
\item 
Throughout the instructional resources, the importance of the variables of time and place— history and geography—is stressed repeatedly. In examining the past and present, the instructional resources consistently help students recognize that events and changes occur in a specific time and place. Instructional resources also consistently help students judge the significance of the relative location of place.
\item 
The history–social science curriculum is enriched with various genres of fiction and nonfiction literature of and about the historical period. Forms of literature such as diaries, essays, biographies, autobiographies, myths, legends, historical tales, oral literature, poetry, and religious literature richly describe the issues or the events studied as well as the life of the people, including both work and leisure activities. The literary selections are broadly representative of varied cultures, ethnic groups, men, women, and children and, where appropriate, provide meaningful connections to the content standards in English–language arts, mathematics, science, and visual and performing arts.
\item 
Materials on religious subject matter remain neutral; do not advocate one religion over another; do not include simulation or role playing of religious ceremonies or beliefs; do not include derogatory language about a religion or use examples from sacred texts or other religious literature that are derogatory, accusatory, or instill prejudice against other religions or those who believe in other religions. Religious matters, both belief and nonbelief, must be treated respectfully and be explained as protected by the U.S. Constitution. Instructional materials, where appropriate and called for in the standards, include examples of religious and secular thinkers in history. When the standards call for explanation of belief systems, they are presented in historical context. Events and figures detailed in religious texts are presented as beliefs held by members of that religion, are clearly identified as such, and should not be presented as fact unless there is independent historical evidence justifying that presentation. All materials must be in accordance with the guidance provided in the updated History–Social Science Framework, Appendix C, “Religion and the Teaching of History–Social Science,” and Education Code sections 51500, 51501, 51511, and 51513.
\item 
Numerous examples are presented of women and men from different demographic groups who used their learning and intelligence to make important contributions to democratic practices and society and to science and technology. Materials emphasize the importance of education in a democratic society.
\end{enumerate}
