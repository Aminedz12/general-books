\chapter{Misrepresenting Hindu Epics}

\begin{longtable}{|>{\raggedleft}p{1.5cm}|p{8.5cm}|}
\multicolumn{2}{c}{\textbf{Table: 1}}\\ 
\hline
\textbf{Page \#} & \textbf{McGraw Hill Text} \tabularnewline
\hline 
274 & The Mahabharata describes a struggle for control of an Indian kingdom that took place about 1100 B.C.E. \tabularnewline
\hline
274 & In it, the deity Krishna goes with a prince into battle. The prince does not want to fight because members of his family are on the other side. Krishna reminds the prince to obey his duty as a warrior. The prince makes the painful choice to fight his family \tabularnewline
\hline
\end{longtable}

\section*{Analysis and Critique} 

As far as the first quote is concerned, it violates evaluation criteria pertaining to accuracy and inclusion of best recent scholarship. All the dates regarding the Indian history, particularly prehistory, are extremely controversial. It will be advisable to not mention the dates of ancient works like the \textit{Mahābhārata} war, at all.

Regarding the second quote, it is in violation of California Education Codes 51501 and 60044—adverse reflection—and evaluation criteria clauses pertaining to perpetrating Eurocentric and Ethnocentric views (continuing with the missionary and imperialistic writings on Hinduism), not remaining neutral in matters of religion, and thereby instilling prejudice in Hindu and non-Hindu children against Hinduism.

This statement over-simplifies the \textit{Bhagavad Gītā}, one of Hinduism's most prominent scriptures. Krishna tells the prince the nature of the \textit{ātman}, the mechanism through which enlightenment can be achieved, concepts of dharma and karma, the three types of yoga, and many profound ideas that make the text a principle one for Hindus to guide their lives and lead a life centered in the Divine. 

The author(s) of the McGraw Hill text show one of the most prominent Gods of Hinduism—Krishna—as a war hungry God, inciting Arjuna to fight, and as an unethical and immoral God, prodding Arjuna to fight his own family members, thereby equating Hinduism with violence, war, immorality, and unethical behavior. So much for representing properly one of the two most prominent epics of the Hindus!

\begin{longtable}{|>{\raggedleft}p{1.5cm}|p{8.5cm}|}
\multicolumn{2}{c}{\textbf{Table: 2}}\\ 
\hline
\textbf{Page \#} & \textbf{McGraw Hill Text} \tabularnewline
\hline 
274 & Second epic, the Ramayana, is a poem that grew to about 25,000 verses before it was written down. It tells the story of Rama, the perfect king, and Sita, his faithful wife. When Sita is kidnapped by an evil king, Rama rushes to her rescue with the help of friends. \tabularnewline
\hline
\end{longtable}

\section*{Analysis and Critique} 

This is in violation of California Education Codes 51501 and 60044—adverse reflection—and evaluation criteria clauses pertaining to historical inaccuracy, not including variety of perspectives and debates, not including best recent scholarship, perpetrating Eurocentric and Ethnocentric history, not remaining neutral in matters of religion (continuing with the missionary and imperialistic writings on Hinduism), and thereby instilling prejudice in Hindu and non-Hindu children against origins of Hinduism and Hinduism per se.

It is inaccurate to state that the poem grew to about 25,000 verses before it was written down. It was authored by sage Vālmīki (and hence, is also known as the \textit{Vālmīki Rāmāyaṇa}). Over 300 different versions of the poem are known to exist—each of which draw from Vālmīki's original text.

The description provided of the \textit{Rāmāyaṇa} belittles a scripture that is loved, worshipped, and recited by millions of Hindus around the world and is very inaccurate. Sita was not kidnapped when Rama was on the throne (as is implied) nor did he rush to her rescue with the help of his friends. It took him approximately one year to find her and he had allies who supported the invasion of Laṅka, the kingdom of the evil king, and helped him win the resulting war. In fact, he did not have his kingdom (Kosala) or its army at his side at all, for he was in exile then. 

German Indologists and European missionaries have particularly been invested in claiming that none of the ancient sacred writings of the Hindus, including the \textit{Rāmāyaṇa, Mahābhārata}, and \textit{Gītā} have single authorship. All throughout this claim has been contested by Indian Indologists, most recently by Adluri and Bagchee (2014).

\begin{thebibliography}{99}
\itemsep=0pt
\bibitem{chap11-key1} Adluri, Vishwa, and Joydeep Bagchee. \textit{The Nay Science: A History of German Indology.} New York: Oxford University Press, 2014.
\end{thebibliography}
\vskip -10pt

\begin{longtable}{|>{\raggedleft}p{1.5cm}|p{8.5cm}|}
\multicolumn{2}{c}{\textbf{Table: 3}}\\ 
\hline
\textbf{Page \#} & \textbf{McGraw Hill Text} \tabularnewline
\hline 
280 & The Bhagavad Gita is the best-known section of the religious epic called the Mahabharata. Scholars are uncertain as to who the original author of the Bhagavad Gita or the complete Mahabharata was. \tabularnewline
\hline
\end{longtable}
\vskip -30pt

\section*{Analysis and Critique} 
\vskip -7pt

This is in violation of California Education Codes 51501 and 60044—adverse reflection—and evaluation criteria clauses pertaining to historical inaccuracy, not including variety of perspectives and debates, not including best recent scholarship, perpetrating Eurocentric and Ethnocentric history, not remaining neutral in matters of religion (continuing with the missionary and imperialistic writings on Hinduism), and thereby instilling prejudice in Hindu and non-Hindu children against Hinduism].

This is inaccurate on multiple accounts. The \textit{Bhagavad Gītā} is well known to have been authored by sage Veda Vyāsa (who also authored the \textit{Mahābhārata}). 

Further, the \textit{Mahābhārata} is an “\textit{itihāsa}” or “historical” text (within the pantheon of texts considered as Hindu scripture). Thus, to call it a “religious epic” is inaccurate.

Both statements are derogatory to Hindus. In addition, German Indologists and European Christian missionaries have particularly been invested in claiming that none of the ancient sacred writings of the Hindus, including the \textit{Rāmāyaṇa} and \textit{Mahābhārata}, have single authorship. All throughout this claim has been contested by Indian Indologists, most recently by Adluri and Bagchee (2014).

\begin{thebibliography}{99}
\bibitem{chap11-key2} Adluri, Vishwa, and Joydeep Bagchee. \textit{The Nay Science: A History of German Indology.} New York: Oxford University Press, 2014.
\end{thebibliography}

\begin{longtable}{|>{\raggedleft}p{1.5cm}|p{8.5cm}|}
\multicolumn{2}{c}{\textbf{Table: 4}}\\ 
\hline
\textbf{Page \#} & \textbf{McGraw Hill Text} \tabularnewline
\hline 
280 & So spake Arjuna to the Lord of Hearts. And sighing, “I will not fight!” held silence then To whom, with tender smile, (O Bharata!) While the Prince wept despairing ‘twixt [between] those hosts, Krishna made answer in divinest verse: KRISHNA: Thou grievest where no grief should be! thou speak’st Words lacking wisdom! for the wise in heart Mourn [feel sadness] not for those that live, nor those that die.Nor I, nor thou, nor any one of these, Ever was not, nor ever will not be, For ever and for ever afterwards. All, that doth [do] live, lives always! To man’s frame As there come infancy and youth and age, So come there raisings-up and layings-down Of other and of other life-abodes [homes], Which the wise know, and fear not. This that irks [annoys]-Thy sense-life, thrilling to the elements-Bringing thee heat and cold, sorrows and joys,‘Tis brief and mutable [changeable]! Bear with it, Prince! As the wise bear. The soul which is not moved, The soul that with a strong and constant calm Takes sorrow and takes joy indifferently, Lives in the life undying! That which is Can never cease to be; that which is not Will not exist. \tabularnewline
\hline
\end{longtable}

\section*{Analysis and Critique} 

This is in violation of California Education Codes 51501 and 60044—adverse reflection—and evaluation criteria clauses pertaining to inaccuracy, not including best recent scholarship, perpetrating Eurocentric and Ethnocentric writing, not remaining neutral in matters of religion (continuing with the missionary and imperialistic writings on Hinduism), and thereby instilling prejudice in Hindu and non-Hindu children against Hinduism.\textbf{} 

First of all, this translation is archaic, which was done by Sir Edwin Arnold in the nineteenth century. The translation apart from being inaccurate is also terse, which will not make sense to a sixth-grade student. There are good English translations available by scholar-practitioner Hindus, whose translations are not only accurate but also contemporary. Sir Edwin Arnold is not considered to be an authority of the Gita and provides a colonial perspective on the text. It is also inappropriate to have a translation of a Hindu scripture by a self-proclaimed Christian be presented as a primary source on Hinduism. This is particularly important when missionary Christianity has not shied away from considering Hinduism as a pagan and heathen religion and has a stated aim of converting Hindus to Christianity.

\begin{longtable}{|>{\raggedleft}p{1.5cm}|p{8.5cm}|}
\multicolumn{2}{c}{\textbf{Table: 5}}\\ 
\hline
\textbf{Page \#} & \textbf{McGraw Hill Text} \tabularnewline
\hline 
281 & 1. I will declare the duties of kings, (and) show how a king should conduct himself, how he was created, and how (he can obtain) highest success. 2. A Kshatriya, who has received according to the rule the sacrament prescribed by the Veda, must duly protect this whole (world). 3. For, when these creatures, being without a king, through fear dispersed in all directions, the Lord created a king for the protection of this whole (creation). . . .5.\ Because a king has been formed of particles of those lords of the gods, he therefore surpasses all created beings in lustre [shine or glow]; . . .14.\ For the (king’s) sake the Lord formerly created his own son, Punishment, the protector of all creatures, (an incarnation of) the law, formed of Brahman’s glory. . . .16.\ Having fully considered the time and the place (of the offence), the strength and the knowledge (of the offender), let him justly inflict that (punishment) on men who act unjustly.17. Punishment is (in reality) the king (and) the male, that the manager of affairs, that the ruler, and that is called the surety for the four orders’ obedience to the law. \tabularnewline
\hline
\end{longtable}

\section*{Analysis and Critique} 

This is in violation of California Education Codes 51501 and 60044—adverse reflection—and evaluation criteria clauses pertaining to inaccuracy, not including best recent scholarship, perpetrating Eurocentric and Ethnocentric writing, not remaining neutral in matters of religion (continuing with the missionary and imperialistic writings on Hinduism), and thereby instilling prejudice in Hindu and non-Hindu children against Hinduism.
\newpage

This is a translation of \textit{Manusmṛti} by Georg Bühler, who did it in the nineteenth century. Apart from the translation being inaccurate, it also is archaic. The publisher could have included more recent and contemporary scholarship. Citing the work of Inden (1990) we have argued that part of the colonial and orientalist project was to show Indian rulers as oppressive, savage, and absolutist. There are many passages in \textit{Manusmṛti} that show that there were checks and balances on the rulers and that they could not act arbitrarily according to their whims and fancies. The authors of this McGraw Hill text have used those portions of \textit{Manusmṛti} that validate and substantiate the imperialistic and orientalist project. \textit{Manusmṛti} is considered as one of the important texts for social governance that Hindus used—it is a different matter that there are more illumined and liberal texts, which imperialists, orientalists, and missionaries have continually ignored. By selectively using these translations, there is a deliberate attempt on part of McGraw Hill to show Hindus and Hinduism in poor light, and therefore it invites the charge of “adverse reflection” and instilling prejudice among Hindu and non-Hindu children against Hinduism.

\begin{longtable}{|>{\raggedleft}p{1.5cm}|p{8.5cm}|}
\multicolumn{2}{c}{\textbf{Table: 6}}\\ 
\hline
\textbf{Page \#} & \textbf{McGraw Hill Text} \tabularnewline
\hline 
281 & \raggedright CANTO XIX: RÁMA’S PROMISE FROM RÁMÁYAN OF VÁLMÍKI\\ Yea, for my father’s promise sake I to the wood my way will take, And dwell a lonely exile there In hermit dress with matted hair. One thing alone I fain [gladly] would learn, Why is the king this day so stern [serious]? Why is the scourge [terror] of foes so cold, Nor gives me greeting as of old? Now let not anger flush thy cheek: Before thy face the truth I speak, In hermit’s coat with matted hair To the wild wood will I repair. How can I fail his will to do, Friend, master, grateful sovereign too? One only pang consumes my breast. That his own lips have not expressed His will, nor made his longing known That Bharat should ascend the throne\tabularnewline
\hline
\end{longtable}

\section*{Analysis and Critique} 

This is in violation of evaluation criteria clauses pertaining to not including best recent scholarship and denial of voice to ethnic groups whose tradition is being represented.

This translation is archaic, which was done by Ralph T. H. Griffith in the nineteenth century. There are better English translations available by scholar-practitioner Hindus, whose translations are contemporary.

\begin{longtable}{|>{\raggedleft}p{1.5cm}|p{8.5cm}|}
\multicolumn{2}{c}{\textbf{Table: 7}}\\ 
\hline
\textbf{Page \#} & \textbf{McGraw Hill Text} \tabularnewline
\hline 
EM66 & When a king named Dharmaputra died, his faithful dog followed him all the way to heaven. “You may come in, but your pet must stay behind,” the gatekeeper said. Dharmaputra so loved his dog that he refused to enter without it. \tabularnewline
\hline
\end{longtable}

\section*{Analysis and Critique} 

This is in violation of evaluation criterion clause pertaining to accuracy.

The name is of the king is incorrect for this is a story from the \textit{Mahābhārata}. The king being referred to as “Dharmaputra” is Yudhiṣṭhira. The story itself is also incorrectly summarized and the explanation of why the king refused to enter heaven without the dog is also incorrect. Please see the summary of the story (with explanation):

Upon the onset of the Kali yuga and the departure of Krishna, Yudhiṣṭhira and his brothers retired, leaving the throne to their only descendant to have survived the war of Kurukshetra: Arjuna's grandson, Parīkśit. Giving up all their belongings and ties, the Pāṇḍava-s, accompanied by a dog, made their final journey of pilgrimage to the Himalayas.

The journey was arduous and all the family members of Yudhiṣṭhira died on way barring the dog. On reaching the top, Indra asked him to abandon the dog before entering the heaven. But Yudhiṣṭhira refused to do so, citing the dog's unflinching loyalty as the reason. It turned out that the dog was his god-father Dharma in disguise (see Agarwal 2002).

\begin{thebibliography}{99}
\bibitem{chap11-key3} Agarwal, Satya P. \textit{Selections from the Mahabharata: Re-Affirming Gita's Call for the Good of All}. Delhi: Motilal Banarsidass, 2002.
\end{thebibliography}
