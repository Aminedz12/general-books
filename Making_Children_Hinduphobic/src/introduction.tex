\chapter*{Introduction}
\addcontentsline{toc}{chapter}{\uppercase{Introduction}}

\lhead[\small\thepage\quad Kundan Singh, PhD and Krishna Maheshwari]{}
\rhead[]{Introduction\quad\small\thepage}

The Hindu community in the United States, by and large, is familiar with the negative portrayals of India and Hinduism in the school textbooks. It usually happens through a rude shock when school-going children, after they enter grade six or seven, come home and begin complaining how they do not want to be identified as Hindu or Indian, asking questions about caste, oppression by Brahmins of all other castes, untouchables and women, about the creation of the Indian civilization by Indo-Europeans, about the monkey God Hanuman, poverty and filth in India and so on and so forth. And the parents begin to think, “well! there is more to India and Hinduism than caste, oppression, poverty and filth,” and as Rajiv Malhotra put it more than two decades ago “caste, cows, and curry.”

Not many but some begin to do their research and soon realize that it is basically the History textbooks that their children are reading which is giving them ideas. The responses are different: 1.\ Some agree with the descriptions in the textbooks and encourage their children to remove themselves from the Hindu and Indian heritage, as they themselves are prone to doing. They feel that they have come out of the savagery and it is time for them to raise their children in civilization. 2.\ Others are at a loss because they cannot answer the questions of their children, for they themselves do not know the answers—understandably so because their educational training pertaining to technology, medicine, or business administration and later their involvement in their respective professions did not give them time to explore deeply either Hinduism or the Indian culture and history. They perhaps did not even feel the need for it because psycho-spiritually one can be so close to a culture or religion that one does not feel the need to explore them. Some of these parents tend to brush aside the questions or concerns of their children and ask them to focus on those disciplines, which would help them achieve a good material life. Others try to educate their children on the broader Indian heritage and culture asking them to set aside these types of materials. Only in rare instances, are there parents that reach out to the teachers to change the discourse (but it if often too late as the chapter is complete). 3.\ There are however others, who identify that this is a problem of misrepresentation and that they need to take corrective actions before the materials are taught in order to change the flawed narrative. 

Parents of non-Hindu/Indian children, however, are unaware of teaching of these materials as they often know little to nothing of the topic. As a result, these children end up with life-long biases that impact their ability to interact with Indian-Americans.

I was introduced to the Hinduphobic topics in the US school curriculum through Rajiv Malhotra and the \textit{indictraditions egroup} that he had started. He would talk about these issues in that forum from time to time. But, for me the deft problems in the textbooks came to a head when late Prof. Shiva Bajpai, the then professor of History at the California State University, Northridge, took the matters in hand in 2005 and strove extremely hard to bring about a discourse change during the \textit{History-Social Science (HSS) Framework} revision process in California. 

The California legislature determines the “Content Standards” for the HSS curriculum, which is passed on to the California Department of Education for the creation of HSS Framework. This framework serves as a guideline for publishers to create textbooks. The California Department of Education does this work through the State Board of Education (SBE), the Instructional Quality Commission (IQC) and specifically, the Instructional Quality Commission’s History Social Science subject matter committee. The HSS Framework gets revised every 10 years through hearings in which anyone including students, parents, teachers, professors, and community organizations are allowed to attend. Prof. Shiva Bajpai was invited as an expert and scholar to support the IQC in 2005. 

I was a doctoral student at the California Institute of Integral Studies working on my dissertation at the time. I completed my dissertation and began teaching at Sofia University. In no time a decade went by and the HSS Framework came up for another round of revision. In March of 2016, I was contacted by a local Bay Area foundation, asking me to become part of a coalition of faculty members, foundations, and community leaders to address the suggestion of replacing India by South Asia in the HSS Framework by a faculty group based in California. Soon I teamed up with other Bay Area faculty in the creation of “Scholars for People” petition, which challenged the replacement, garnering 25,782 signatures. I wrote to a letter to IQC, giving evidence from Greek and Roman accounts with regards to how India since antiquity has always existed as a geographical entity (the letter is attached as appendix A). I also became part of a coalition of academics titled Social Sciences and Religion Faculty Group (SSRFG) and was a key individual in getting a letter drafted. This letter was submitted to the IQC before the hearing in May 2016, which I followed up through an in-person presentation. 

In the course of working on the SSRFG letter, I realized how difficult it was to bring a strong defense of Hinduism and Indian culture, despite substantial materials present in academia today pertaining to Orientalism, Post colonialism, emic perspectives on Hinduism, evidence against the Aryan Invasion/Migration Theory and so on and so forth due to timidity amongst scholars and the foundations that support these scholars. I also realized that even when little gains in the Framework revision process would be made, there would be a lot of drum beating to take credit by organizations supporting the Hindu cause. It began to become quite clear to me that there were too many vested interests involved to successfully and completely overhaul the HSS Framework. Further, there remained a strong lobby in mainstream academia which is not only rabidly invested in maintaining the status quo but also in seeing a ongoing production of Hinduphobic literature in textbooks. 

The IQC completed its hearing and submitted the revised HSS Framework which was adopted by SBE on July 14, 2016. Publishers were expected to submit their textbook drafts for approval by SBE in the May of 2017 based on the revised Framework. When the textbook drafts were made public, I was not surprised to see an extreme Hinduphobic slant in most of the textbooks. This was because the HSS Framework was deeply flawed. If the root is damaged, it is futile to expect healthy stems, branches, and leaves. Whereas organizations and foundations, including Hindupedia, began to concentrate on challenging the textbook drafts for their stereotypical and prejudicial characterization of India and Hinduism, I decided to address the rot in the Framework itself—the involvement of Hindupedia in this saga will be narrated to us soon by Krishna Maheshwari. 

I focused on the California Education Code Violations (see Appendix C) as mentioned in \textit{Standards for Evaluating Instructional Materials for Social Content} in the following two documents: \textit{History-Social Science Content Standards for California Public Schools: Kindergarten Through Grade Twelve} (henceforth called Content Standards), and the \textit{History-Social Science Framework} or HSS Framework. I sent my analysis to the IQC and SBE. The letter is attached as appendix B. 

In the meantime through the colossal efforts of Krishna Maheshwari and his team at Hindupedia, Houghton Mifflin Harcourt’s two draft programs covering K-6 and 6-8 got rejected by the IQC. When all this was happening my father had taken critically ill in India and I was busy nursing him back to health. I was however keeping a tab on the developments that were taking place in California. In order to firm up the rejection of the HMH textbooks, I wrote another letter to SBC and IQC. The letter focused on orientalist issues, conflation of Hinduism with caste, hierarchy and oppression, and privileging of monotheism over polytheism with a subtle conjoining of Hinduism with polytheism. Over to Krishna Maheshwari regarding his involvement in the textbook issue before we describe how we came together for a joint project on McGraw Hill texts! 

\section*{2005 California textbook revision process and Hindupedia}

I was introduced to issues related to India and Hinduism in textbooks in 2006 after the close of the textbook revision process in California. At that time, I was a student at Harvard Business School and invited to debate Prof. Michael Witzel (a Tenured Professor at Harvard and California State Board of Education appointed expert on India and Hinduism after the resignation of Prof. Shiva Bajpai).

The debate went well with Prof. Witzel all but admitting defeat in a private email to the RISA (Religions in South Asia) mailing list afterwards. It also made me realize that there were insufficient emic materials on India and Hinduism and without them, we as a community would struggle to make forward progress on issues like the ones we face in textbooks. This realization led me to launch Hindupedia, the online Encyclopedia of Hinduism in 2007 with the goal of ensuring that we have a citable and academic level emic presentation of Hindu Dharma available in English for budding scholars and the public at large. 

\section*{2015--17 California History Social Science Framework revision process}

The SBE through the IQC worked on revising the HSS Framework from 2015--2017.

I got engaged in this process in 2016 after an academic coalition based in California stole control of the discourse by presenting themselves as the only mainstream faculty on India and Hinduism. 62 of their 76 recommendations were accepted by the History Social Science Project—run out of University of California Davis and appointed in the initial stages of the process by the IQC to review public comment and make recommendations to the IQC. This academic group vilified all others who did not adhere to their point of view as fringe Hindu fanatics. They pushed the IQC to remove India and Hinduism from textbooks and replace them with the terms “South Asia” and “Religions of South Asia” respectively. At that time, my involvement was to support the Hindu advocacy groups that were already engaged with the IQC. 

I worked with Vishal Agarwal to create a nineteen-page point by point rebuttal of their requested changes with some input from members of the Hindupedia team and submitted it in March 24th under the Hindupedia letterhead (with 11 academics, a school teacher and 4 scholars signing). I also partnered with Vishal to consolidate the submissions from various Hindu groups into a single package which resulted in a 196-page submission under the aegis of \textit{Scholars for People}. This letter was signed by 10 academics and 4 scholars at the time of submission but ended up with many others signing after submission (it was documented at the time on the “Scholars for People” website). In addition to the “Scholars for People” submission, Vishal and I authored a complementary 74-page submission which addressed caste issues (which were not included in the “Scholars for People” letter) as well as provided evidence countering specific issues mentioned in subsequent letters from the aforementioned academic group.

The output of the Framework round left every group claiming victory but no group feeling that they achieved their objectives. The output of the process resulted in a deeply flawed HSS Framework. However, it was less deeply flawed as compared to the output from the 2006 process.

\section*{2017 IQC California History Social Science draft textbook review process}

The draft textbook review process spanned several IQC meetings as well as a weeklong Instructional Materials Reviewer/Content Review Expert (IMRCRE) panel discussion. Draft textbooks were made available for public review on May 12, 2017. 

Vishal and I turned around a 50-page initial review of materials from four publishers and submitted it on May $17^{\rm th}$, ahead of the May $18^{\rm th}$ IQC meeting under the Hindupedia letterhead. This initial review was signed by 14 academics and five scholars. 

I later expanded this document into a 289-page document that had in-depth analysis of the draft textbooks from Houghton Mifflin Harcourt, Discovery Education, National Geographic, McGraw Hill, and Studies Weekly. This document was shared with the Instructional Material Reviewer/Content Review Expert week-long meeting in July 2017. It was signed by 15 academics and 4 scholars. It was dutifully accepted but ultimately ignored by the panel (except for the Studies Weekly Panel where one panelist used the materials). These materials were also used as part of written and oral testimony submitted by hundreds of parents who attended the hearings and discussions.

The IQC in their final hearings on November 17 and 18 reviewed submissions from Hindupedia and others. During the final deliberations, they rejected Houghton Mifflin Harcourt’s two programs (“Kids Discovery” and “Social Studies” spanning K-8) and in their rejection referenced the Hindupedia submission and iterated their agreement with 54 reasons for rejection that we had outlined. 


\section*{2017 final State Board of Education hearing: closing out the process}

I authored a 22-page letter that reframed the entire discussion solely around violations of the History Social Science Framework. This was different from previous approaches as it used a black and white criteria rubric. Specifically, did the instructional materials include all content from the adopted HSS Standards and the HSS Framework? The SBE was held accountable by California Department of Education for ensuring adoption of the Standards and Framework. This letter also included 35 edits and corrections to McGraw Hill’s draft textbook which would make those materials less Hinduphobic. This letter was signed by 40 academics and a handful of scholars. 

This letter met opposition with the other advocacy and activist groups involved in the effort and was eventually split into two. A subsection of the McGraw Hill review was reframed and submitted under Prof. Long’s letter head while the remaining was submitted under the Hindupedia letterhead as a supplementary letter.


\section*{Testimony, public, and organizational support throughout the process}

My involvement in the textbook issue since 2016 extended beyond writing and sharing written analysis with the SBE, IQC, and IMRCRE panels. 

I attended every hearing following a submission and on numerous occasions gave additional testimony on topics unrelated to India and Hinduism. I also made it a point to talk to the commissioners during breaks and after hearings. Doing so resulted in many of them recognizing me by name as an expert on India and Hinduism. This recognition extended beyond the immediate commissioners to many of their staff as well. During one of the IQC meetings, I initiated a partnership with the FAIR act alliance which resulted in them providing us some support.

I also worked to get support from Bay Area temples and community organizations including ISKCON Silicon Valley, ISKCON Berkeley, ISKCON Sacramento, India Heritage Foundation and Bay Area Vaishnav Parivar. This included giving talks to educate parents, encouraging them to write organizational letters of support and to send people to the hearings in Sacramento. I also demonstrated the effectiveness of letter campaigns driving hundreds of letters to the SBE/IQC which was later picked up and driven to scale by other involved Hindu organizations.

Our submitted materials, selected testimony and a more detailed discussion of the textbook process in California, have been published on Hindupedia at 

{\small\url{http://www.hindupedia.com/en/Hinduism_in_California_Textbooks}}

Though the HMH textbooks were rejected by the State Board of Education, the McGraw Hill’s materials were approved largely retaining all of their Hinduphobic content. Now that the textbooks have entered the adoption phase, Kundan and I decided to challenge the adoption of the McGraw Hill textbook at the school district levels. 

We did a detailed analysis of the McGraw Hill texts that have been approved by the State Board of Education. The analysis hovered around the perpetuation of prejudicial and stereotypical materials on Hinduism, Education Code violations, and violations pertaining to the evaluation criteria set by the California Department of Education. These criteria were used as California only holds districts accountable for adhering to its Educational Codes and not to the HSS Standards and Framework. The cover letter to the analysis document was signed by 39 academics (27 for the United States, 9 from India, and 3 from Canada, the UK, and Austria), though a few of the academics read and endorsed the entire document. The document has been circulated with all 1,100 school districts in California by Hindupedia and engagement with the districts is ongoing.

\section*{Organization of the book}
\vskip -5pt

This book is based on our 95-page analysis of McGraw Hill’s textbook that was approved the SBE and shared with the officials in all 1,100 districts in California. 

The cover letter of the analysis, signed by the aforementioned academics, has become chapter one—we have included their names and institutional affiliations in a separate appendix (Appendix G). Appendices 3 and 4 of the original letter plus some materials which we had not included in the letter, particularly pertaining to Islam’s relationship with India and Hinduism, have become subsequent chapters in the book. 

The second chapter unfolds and unpacks the myriad Hinduphobic issues that are present in the McGraw Hill texts.

Chapters three to eleven analyze and critique the problematic discourse present in the sixth grade textbook and accompanying instructional material. They pertain to controversial claims about the following:
\begin{enumerate}
\item Origins of the Indian civilization which is done through limiting the discourse on the Harappan civilization by effacing the findings of the last four to five decades.
\item Origins of the Hinduism by perpetuating the Aryan Migration Theory, the politically correct version of the orientalist and racist Aryan Invasion Theory. 
\item Conflation of Hinduism with caste, and thereby characterizing it as a religion where hierarchy and oppression rule the roost. 
\item Distortion of Hinduism and Buddhism in significant ways to the extent that both scholar-practitioner as well as emic insiders to the traditions will disagree with the claims being made. 
\item Showing Buddhism as an improvement over Hinduism and making all possible attempts to pit Buddhism against Hinduism. Fissures between the traditions are created where none exist while divergences are magnified to make them incompatible. This is also done by distorting the history of Ancient India, particularly around the conversion of Ashoka to Buddhism. 
\item Distorting the discussion and representation on Hindu Epics such as Rāmāyaṇa and Mahābhārata to show Hinduism in poor light. 
\end{enumerate}
Chapters twelve to fourteen deal with the distortion of narrative on Hinduism and India in the post-Gupta period, Islamic appropriation of the Indian knowledge systems in particular Mathematics, and the whitewashing of the brutal history of Islam in India. 
