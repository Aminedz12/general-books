\chapter{Distorting Ancient History}

\begin{longtable}{|>{\raggedleft}p{1.5cm}|p{8.5cm}|}
\multicolumn{2}{|c|{\textbf{Table: 1}} 
\hline
\multicolumn{1}{|l|}{\textbf{Page #}} & \multicolumn{1}{|l|}{\textbf{McGraw Hill Text}} \tabularnewline
\hline 
270 & By the 500s B.C.E, India was divided into many small kingdoms. Conflict over land and trade weakened the kingdoms, leaving them open to foreign invasion. First, Persian armies conquered the Indus Valley in the 500s B.C.E and made it part of the Persian Empire. The Greeks, under Alexander the Great, then defeated the Persians. Alexander entered India but turned back in 325 B.C.E, when his homesick troops threatened to rebel. \tabularnewline
\hline
\end{longtable}

\section*{Analysis and Critique} 

These violate evaluation criteria pertaining to accuracy.

It is historically inaccurate to state that Alexander turned back because his homesick troops threatened to rebel. The troops threatened to rebel because they were too afraid of moving forward further into India.

The Indian campaign of Alexander the Great began in 326 BCE. After conquering the Achaemenid Empire of Persia, the Macedonian king (and now the great king of the Persian Empire), Alexander, launched a campaign into India. The Battle of the Jhelum river against a regional Indian King, Porus, is considered by many historians, Peter Connolly (1981) being one of them, as the most costly battle fought by the armies of Alexander.

Plutarch also wrote that the bitter fighting of the Hydaspes (Jhelum in Greek) made Alexander's men hesitant to continue on with the conquest of India, considering that they would potentially face far larger armies than those of Porus if they were to cross the Ganges River. As per Plutarch:
\begin{quote}
As for the Macedonians, however, their struggle with Porus blunted their courage and stayed their further advance into India. For having had all they could do to repulse an enemy who mustered only twenty thousand infantry and two thousand horse, they violently opposed Alexander when he insisted on crossing the river Ganges also, the width of which, as they learned, was thirty-two furlongs, its depth a hundred fathoms, while its banks on the further side were covered with multitudes of men-at-arms and horsemen and elephants. For they were told that the kings of the Ganderites and Praesii were awaiting them with eighty thousand horsemen, two hundred thousand footmen, eight thousand chariots, and six thousand fighting elephants.\footnote{Arrian, Plutarch, and Quintus Curtius Rufus, \textit{The Brief Life and Towering Exploits of History's Greatest Conqueror: As Told By His Original Biographers}, eds.\ Tania Gergel and Michael Wood (New York: Penguin Books, 2004), 120.}
\end{quote}
As per  Megasthenes:
\begin{quote}
Gangaridai, a nation which possesses a vast force of the largest-sized elephants. Owing to this, their country has never been conquered by any foreign king: for all other nations dread the overwhelming number and strength of these animals. Thus Alexander the Macedonian, after conquering all Asia, did not make war upon the Gangaridai, as he did on all others; for when he had arrived with all his troops at the river Ganges, he abandoned as hopeless an invasion of the Gangaridai when he learned that they possessed four thousand elephants well trained and equipped for war.\footnote{Megasthenes, \textit{Ancient India as Described by Megasthenes and Arrian}, trans.\ and ed.\ J. W. McCrindle (London: Forgotten Books, 2017), 33.}
\end{quote}

\begin{thebibliography}{99}
\bibitem{chap9-key1} Connolly, Peter. \textit{Greece and Rome At War}. London: Macdonald Phoebus Ltd, 1981.
\end{thebibliography}

\begin{longtable}{|>{\raggedleft}p{1.5cm}|p{8.5cm}|}
\multicolumn{2}{|c|{\textbf{Table: 2}} 
\hline
\multicolumn{1}{|l|}{\textbf{Page #}} & \multicolumn{1}{|l|}{\textbf{McGraw Hill Text}} \tabularnewline
\hline 
270 & After Alexander left India, an Indian military officer named Chandragupta Maurya built a strong army. He knew that only a large and powerful empire could defend India against invasion. In 321 B.C.E., Chandragupta set out to conquer northern India and unify the region under his rule. \tabularnewline
\hline
\end{longtable}

\section*{Analysis and Critique} 

These violate evaluation criterion pertaining to accuracy.

It is inaccurate to call him an &quot;Indian military officer.” He was born in a humble family, orphaned and abandoned, raised as a son by another pastoral family, then picked up, taught, and counseled by Chanakya (also known as Kautilya of the \textit{Arthashastra} fame). He never served in an army (other than when he was leading it to create the Mauryan empire). Chandragupta also did not set out to conquer northern India; rather, he set out to conquer all of India.

\begin{longtable}{|>{\raggedleft}p{1.5cm}|p{8.5cm}|}
\multicolumn{2}{|c|{\textbf{Table: 3}} 
\hline
\multicolumn{1}{|l|}{\textbf{Page #}} & \multicolumn{1}{|l|}{\textbf{McGraw Hill Text}} \tabularnewline
\hline 
270 & He was afraid of being poisoned, so he had servants taste his food before he ate it. He was so concerned about being attacked that he never slept two nights in a row in the same bed \tabularnewline
\hline
\end{longtable}

\section*{Analysis and Critique} 

These violate evaluation criteria pertaining to accuracy and inclusion of best recent scholarship.

This is an Orientalist discourse to show one of the most powerful rulers of India as weak, oppressive, and paranoid. Even if the statement were true, it is a deliberate insertion to underline the above. The demonization of the governance structure of ancient India is a characteristic orientalist discourse where its rulers are necessarily shown as despots.

\begin{longtable}{|>{\raggedleft}p{1.5cm}|p{8.5cm}|}
\multicolumn{2}{|c|{\textbf{Table: 4}} 
\hline
\multicolumn{1}{|l|}{\textbf{Page #}} & \multicolumn{1}{|l|}{\textbf{McGraw Hill Text}} \tabularnewline
\hline 
271 & The Mauryan dynasty built the first great Indian empire \tabularnewline
\hline
\end{longtable}

\section*{Analysis and Critique} 

These violate evaluation criteria pertaining to accuracy and inclusion of best recent scholarship.

The Mauryan Empire was not the first historical Indian Empire. In fact, it was a successor to the Nanda Empire (in Magadha) that was founded a century earlier. The Nandas had initiated the task of conquering and destroying numerous ancient republics and kingdoms in Northern India, and Chandragupta furthered this task (Sastri 1967; Allen 2012).

\begin{thebibliography}{99}
\bibitem{chap9-key2} Allen, Charles. \textit{Ashoka: The Search for India’s Lost Emperor.} London: Abacus, 2012.

\bibitem{chap9-key3} Sastri, K. A. Nilakanta, editor. \textit{Age of the Nandas and Mauryas.} New Delhi: Motilal Banarsidass, 1967.
\end{thebibliography}

\begin{longtable}{|>{\raggedleft}p{1.5cm}|p{8.5cm}|}
\multicolumn{2}{|c|{\textbf{Table: 5}} 
\hline
\multicolumn{1}{|l|}{\textbf{Page #}} & \multicolumn{1}{|l|}{\textbf{McGraw Hill Text}} \tabularnewline
\hline 
274 & The Gupta dynasty founded the second great Indian empire. \tabularnewline
\hline
\end{longtable}

\section*{Analysis and Critique} 

This violates evaluation criterion pertaining to accuracy.

It is historically inaccurate to state that the Gupta dynasty founded the &quot;second&quot; great Indian empire. There are several earlier empires that are considered great predating the Gupta Dynasty. Two are Nanda and Maurya.

\begin{longtable}{|>{\raggedleft}p{1.5cm}|p{8.5cm}|}
\multicolumn{2}{|c|{\textbf{Table: 6}} 
\hline
\multicolumn{1}{|l|}{\textbf{Page #}} & \multicolumn{1}{|l|}{\textbf{McGraw Hill Text}} \tabularnewline
\hline 
275 & The most important structures in early India were the rulers’ palaces and the temples used for religious worship. \tabularnewline
\hline
\end{longtable}

\section*{Analysis and Critique} 

This is in violation of California Education Codes 51501 and 60044—adverse reflection—and evaluation criteria clauses pertaining to historical inaccuracy, not including variety of perspectives and debates, not including best recent scholarship, perpetrating Eurocentric and Ethnocentric history, not remaining neutral in matters of religion (continuing with the missionary and imperialistic writings on Hinduism), and thereby instilling prejudice in Hindu and non-Hindu children against Hinduism. 

This is an unsubstantiated statement with little basis in reality! Temples were considered the most important structures in early India. Palaces would be a distant second. In addition, this statement basically shows that Hindu kings and priests were wealth hungry individuals, who would arrogate to themselves the lion’s share of the wealth of the populace. Describing ancient India—its rulers and priests—as savage was the design of the orientalists under the colonial rule (see Inden 1990). The knowledge production of the orientalists was essentially racist in nature, designed to legitimize what the postcolonial literature now calls the “white man’s burden.” 

\begin{thebibliography}{99}
\bibitem{chap9-key4} Inden, Ronald. \textit{Imagining India}. Bloomington: Indiana University Press, 1990.
\end{thebibliography}
