\chapter*{FOREWORD: HOW \textit{INFINITY FOUNDATION} GOT INVOLVED}
\addcontentsline{toc}{chapter}{FOREWORD: HOW \textit{INFINITY FOUNDATION} GOT INVOLVED}
\lhead[\small\thepage\quad Rajiv Malhotra]{}
\rhead[]{Foreword: How \textit{infinity Foundation} Got Involved\quad\small\thepage}

This book represents an important milestone in a cause that I started in the mid-1990s. The following events over the past 25 years will provide a good outline of how we have arrived at this point:

\begin{enumerate}
\item \textbf{Princeton Day School:} During the 1990s both my children went to this liberal private school which was one of the earliest in USA to offer a course on World Religions. This was taught by a teacher with sympathy toward all the faiths. I regularly interacted with the faculty on their approach to Indian civilization with the goal of upgrading their curriculum. 

Interestingly, the school was very receptive to my inputs, but the parents of other Indian students wanted nothing to do with my ‘controversial’ intervention. There were few Indian students in this school at that time, and the parents’ main interest was to get them good academic performance and college admission. They did not want to rock the boat.

Infinity Foundation proposed a new program at the school to take students of the World Religion course to India for an experience of India’s pluralism. Before the school accepted this idea, they wanted me to take a couple of their senior teachers to India to evaluate whether this would be suitable for the children. This was an amazing tour which took us to places of spiritual significance in India, including Rishikesh, Gaumukh as well as Nepal. The school approved the idea and soon we were leading tours of students and teachers to India and Nepal. Many students wrote amazing articles on their experiences. One of the teachers formally converted to Hinduism, and he remains a practicing Hindu today.

Our foundation had gifted a large collection of books on Hinduism, Buddhism, Jainism and Sikhism to the school library. Suddenly, I got my first wakeup call on US school textbook biases. The teacher who had converted to Hinduism got a call from a professor involved in the American Academy of Religion (AAR), telling him that the books about Sri Ramakrishna and Swami Vivekananda we had donated to the library ought to be removed. I was stunned by this and decided to investigate.

What I learned was a game-changing experience for me: According to the academic research claimed by AAR, Ramakrishna/Vivekananda were involved in a sexually abusive relationship. This escalated my intervention with powerful American scholars of Hinduism, which resulted in many iconic fights ever since.

I discovered that the bias was deep. Infinity Foundation funded major reports looking at the portrayals of India and Hinduism in print and television as well as education. I started attending school board meetings in various states to gather data and formulate my thesis.

I coined the term ‘Hinduphobia’ to describe this problem. Ironically, nobody I approached wanted to help me fight this Hinduphobia: not any guru, not anyone in the Indian Embassy, not the Friends of BJP, VHP America, or Hindu Swayamsevak Sangh, nor any other Indian/Hindu group in USA. Yet these folks were all parading as champions of India and Hinduism whenever there was a forum to put themselves on display. But asked to their necks out, there were not interested. I was declared ‘controversial’. (Yet, many years later, after it became a badge of honor to speak out against Hinduphobia, many of the same leaders started recycling my research findings from these early years.)

\item \textbf{Swami Dayananda Saraswati \& Arsha Vidya Gurukulam:} The first\break important person who took my investigations seriously and wanted to lend support was the late Swami Dayananda Saraswati. He regularly invited me to give talks at his Arsha Vidya Gurukulam in Pennsylvania. I started presenting workshops with research data and analysis.

Had it not been for his very strong and consistent support from the beginning, I am convinced this movement would not have gathered the momentum it eventually did.

Swamiji invited me to speak at important events and specifically wanted me to be bold and detailed in articulating the problems: including specific examples in textbooks and an analysis of the underlying infrastructure that supports Hinduphobia.

\item \textbf{Infinity Foundation’s panel at Asian Studies Conference:} We started\break putting a team of academic scholars together who I felt could be\break relied upon to argue our case for fair treatment in US school textbooks. An important milestone was when Infinity Foundation sent its delegation to speak at the Asian Studies Conference in the 1990s. Mrs. Kamlesh Kapur, Yvette Rosser, David Grey and Madhu Kishwar – all funded by IF – comprised the panel. There was a lot of backlash from the audience and the organizers because the panelists spoke candidly about the biases. Until then nobody had stirred the hornets’ nest by criticizing the establishment. The pushback and anger we experienced at several similar events convinced me to take on this fight. We launched this movement without support from the large, established Indian/Hindu organizations based in USA.

\item \textbf{Association of Asian Studies:} I discovered a magazine called \textit{EDUCATION ABOUT ASIA} that was the largest educational magazine on Asia sent to 20,000 school teachers of history/culture across USA. Until then it was dominated by articles that spoke well of Japan, China and Middle East/Islam. India was absent. There were very rare articles on India – this was the period when India was seldom included in any forum dealing with Asia. It was as if it did not exist.

I negotiated with the publisher of this magazine that Infinity Foundation would give a grant of US \$100,000 to them for a special issue to be focused on India. Additionally, we gave grants to several scholars to write the articles for that issue, including: Robert Thurman, Arvind Sharma, Yvette Rosser and Madhu Kishwar. The magazine also included a few authors of their choice to bring ‘balance’.

Yvette Rosser’s very direct review of biases generated a backlash in the letters to the editor in the subsequent issues of the magazine. We negotiated that Dr Rosser be given a chance to respond, which she did. It became clear to our small team that the problem was deeply entrenched, and the opponents were not going to change or walk away easily.

\item \textbf{Infinity Foundation funds many academic programs in USA:} I discovered how the school textbooks and curriculum on Hinduism were being controlled by a network of academic scholars based in prestigious universities across USA. To counter this, we funded Arvind Sharma as visiting professor at Harvard University with a grant in excess of \$100,000/year, and later we repeated this for Ashok Aklujkar as visiting professor to Harvard as well. We funded the Harvard Indology Roundtable for many years to bring top scholars from India, Pakistan, USA and Europe on what was then being called the Indus Civilization. I was personally engaged in all the material that would be presented and developed a hands-on familiarity with the discourse and the politics surrounding it.

Infinity Foundation funded numerous programs on India and Hinduism at: Columbia University, University of California, Penn State, UPenn, Univ. of Hawaii and many others. We were recognized during the period 1994 to 2003 as the foremost sponsor of India related academic programs in USA. I personally received many invitations and solicitations to ‘join the system’, but I saw it as an attempt to buy me out.

This decade long investment on funding the US academic system cost Infinity Foundation about \$4 million. In the process, I became well educated on how the system works, who is who among its powerful individuals, the ideological nexuses involved, how and why money and influence travel.

\item \textbf{Appointment as Chairman of Asian Studies for New Jersey:} In 2002-04, there was a Hindu-friendly governor of the state of New Jersey, and I was appointed Chairman for Asian Studies in the state schools. This was the result of many years of petitioning and lobbying about textbooks in state run schools. A few important Indian Americans who supported this governor had convinced him to make this appointment. My commission for formulating a brand-new curriculum in New Jersey ran for a couple of years and produced a document ready for adoption in the state. There were many pioneering ideas in it – which I would routinely explain in my talks to the Indian diaspora across the US.

Unfortunately for our cause, just prior to the governor signing this proposal into adoption in the state education system, he was involved in some sex scandal (with his chauffer) and had to resign. The new governor was not interested in developing such an educational program for teaching Asia in the schools. The commission was abandoned. 

But the lessons learned by this experience became a part of my advice to other Hindus across USA. By now, similar initiatives started emerging in many states. 

\item \textbf{Support for California Parents for the Equalization of Educational Materials (CAPEEM):} One of the Hindu diaspora activists who had followed my work was Arvind Kumar. He later established CAPEEM (around 2006) to fight the California textbook case. Infinity Foundation was its first major funding source as well as the organization providing moral support and spreading awareness of CAPEEM’s work to the community.

In the past dozen years, various groups have entered the California textbook case, made heroic claims about their work, and then abandoned the scene. Besides causing confusion with false and exaggerated claims, they ended up spoiling our cause in many ways. But CAPEEM has persisted and is the most consistent activist in this matter.

\item \textbf{Advisor to Uberoi Foundation:} I was approached by Shri Ved Nanda to advise him on the strategic direction of the newly established Uberoi Foundation in 2008. Using my experience with Infinity Foundation, I emphasized a few principles: That they should not turn over leadership to Western scholars or hand them control of the money; that they should not scatter by chasing many issues and pick one issue and focus on it. I advised specifically that they should focus on school education in the US, both in the training of American teachers in the Dharma traditions, as well as in advocacy for textbook reform. Unfortunately, Uberoi Foundation pursued the path of holding high-profile expensive annual meetings of ‘experts’ on all sorts of fields, without any focus. But after spending a lot of money without any impact, they have finally started to focus on school curriculum reform and teacher training – exactly as I had recommended a decade ago.
\end{enumerate}

The above is a very concise summary of a few major milestones in the matter of US school textbook and curriculum intervention. Independently of this, I have known and supported the work of Kundan Singh in other topics since 2000. Therefore, when I learned that he was now involved in the California textbook issue, it was music to my ears. I decided immediately to support his work because he is continuing what I started in the 1990s. I led this fight for the first decade, but since the past several years my interests have expanded, and I have not been able to do justice to the school textbook issue. This is why I am delighted that some other young scholars have picked up where I left off. They must continue the good fight. I wish Kundan and his collaborators the very best in this endeavor. 
\bigskip

\noindent
\textbf{Rajiv Malhotra}\\
Princeton\\
July 30, 2018. 
