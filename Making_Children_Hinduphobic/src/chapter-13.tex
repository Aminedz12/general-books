\chapter[Dishonest attribution of Indic\\ Knowledge Systems to Islam]{Dishonest attribution of Indic Knowledge Systems to Islam}
 

\begin{longtable}{|>{\raggedleft}p{1.5cm}|p{8.5cm}|}
\multicolumn{2}{c}{\textbf{Table: 1}}\\ 
\hline
\textbf{Page \#} & \textbf{McGraw Hill Text} \tabularnewline
\hline 
138 & At the Baghdad observatory founded by Mamun, Muslim astronomers studied the skies. These studies helped them create mathematical models of the universe. They correctly described the sun’s eclipses and proved that the moon affects ocean tides. \tabularnewline
\hline
\end{longtable}

\section*{Analysis and Critique} 

This violates the evaluation criterion pertaining to accuracy, and by whitewashing Indian history in favor of Islam, it violates Education Codes 51501 and 60044 pertaining to adverse reflection against Hinduism.

It is inaccurate to state that the Muslim astronomers described~the sun's eclipse and proved that the moon affects ocean tides. They confirmed (for themselves) these findings from mathematical astronomy text \textit{Brahmasphuṭa\-siddhānta} of Brahmagupta (598--668 CE) which was received in the court of Al-Mansur. It was translated by Alfazari into Arabic as \textit{Az-Zīj `alā Sinī al-`Arab} popularly called \textit{Sindhind} (Kennedy 1956). This translation was how the Hindu numerals were transmitted from the Sanskrit to the Arabic tradition (Smith, and Karpinsiki 2013). Per Al-Biruni, “As Sindh was under the actual rule of the Khalif Mansur (753–774 CE), there came embassies from that part of India to Bagdad and among them scholars, who brought with them two books.”\footnote{Edward C. Sachau, trans.\ and ed., \textit{Alberuni's India: An Account of the Religion, Philosophy, Literature, Geography, Chronology, Astronomy, Customs, Laws and Astrology of India about A. D. 1030} (New Delhi: Rupa \& Co., 2002), xxxiii.}

Alberuni's translator and editor Edward Sachau (2002) wrote: “It is Brahmagupta who taught Arabs mathematics before they got acquainted with Greek science.”\footnote{Edward C. Sachau, trans.\ and ed., \textit{Alberuni's India: An Account of the Religion, Philosophy, Literature, Geography, Chronology, Astronomy, Customs, Laws and Astrology of India about A. D. 1030} (New Delhi: Rupa \& Co., 2002), xxxiii.} 

Alfazari also translated the \textit{Khaṇḍakhādyaka} (\textit{Arakand}) of Brahmagupta (Kennedy 1956). “With the help of these pandits Alfazari, perhaps also Yaqūb ibn Tāriq, translated them. Both works have been largely used, and have exercised a great influence. It was on this occasion that the Arabs first became acquainted with a scientific system of astronomy. They learned from Brahmagupta earlier than Ptolemy,”\footnote{Edward C. Sachau, trans.\ and ed., \textit{Alberuni's 	India: An Account of the Religion, Philosophy, Literature, 	Geography, Chronology, Astronomy, Customs, Laws and Astrology of India about A. D. 1030}  (New Delhi: Rupa \& Co., 2002), xxxiii.} and through the resulting Arabic translations known as \textit{Sindhind} and \textit{Arakand}, the use of Indian numerals became established in the Islamic world (Avari 2007).

The appropriation of Indic knowledge system reflects adversely on\break Hinduism because of the following reasons: The long lineage of mathematicians that we find in the ancient India running up to the end of the medieval times is because of a tradition that comes out directly from the Vedas. In the Vedas, one finds mention of \textit{gaṇaka} (or calculator or elementary mathematician) and \textit{nakṣatra dṛṣṭa} (or star gazer). The elementary form of\break astronomy as \textit{nakṣatra vidyā} also finds a mention. At the close of the Vedic period were composed \textit{vedāṅga}-s or limbs of the Vedas, considered auxiliary to the Vedas. Two of the six divisions consist of astronomy (\textit{jyotiśa}) and\break ceremonials (\textit{Kalpa}). Part of the \textit{Kalpa} was \textit{Śulba} or the method of construction of the altar for performing the Vedic sacrifices. These altars at times had complicated geometrical forms, involving squares, rectangles, circles, quadrilaterals, trapezoids, and their combinations. It was held that for the sacrifices to be effective, the measurements had to be perfect. The \textit{Śulbasūtra}-s therefore give theorems (including what is now called as Pythagoras theorem) and\break algorithms involving linear, simultaneous, and even indeterminate equations for the construction of these sacrificial altars. Interest in geometry led to the development of trigonometry—Āryabhaṭṭa, being the first to develop the\break table of sines unlike the table of chords by the Greeks. The most famous theorems by an author or the school (after the said author) of these \textit{Śulbasūtra}-s are those by Baudhāyana. They are divided into three chapters with a total of 272 passages or 519 aphorisms.
\eject

Similarly, \textit{jyotiśa} or astronomy found sanction from the Vedas themselves, and as we mentioned earlier, it was one of the six limbs of the \hbox{Vedā}-s. For the Vedic sacrifices to be effective, it was held that the time for their performance has to be auspicious. This led the ancient Indians to have a deep interest in the movements of planets, sun, and moon and measurements involving stars and planets, including the equinoxes, solstices, waxing and waning of the moon, eclipses etc. The gazing of the stars involved angles and measurement of speed. This led to the development of geometry and trigonometry and later calculus. As the Indian civilization progressed and evolved, there came into being specialized fields of mathematics like algebra, arithmetic, geometry, trigonometry, and even calculus. Many developments in Europe pertaining to Calculus were anticipated by at least three centuries in India, predominantly through the work of the Kerala school of mathematicians.

\begin{thebibliography}{99}
\itemsep=1pt
\bibitem{chap13-key1} Avari, Burjor. \textit{India: The Ancient Past: A History of the Indian Sub-Continent from C. 7000 BC to AD 1200.} New York: Routledge, 2007.
\bibitem{chap13-key2} Kennedy, E. S. “A Survey of Astronomical Tables,” \textit{Transactions of the American Philosophical Society, New Series} 46, no. 2 (1956). 
\bibitem{chap13-key3} Smith, David Eugene, and Louis Charles Karpinsky. \textit{The Hindu-Arabic Numerals}. Mineola: Dover, 2013.
\end{thebibliography}

\begin{longtable}{|>{\raggedleft}p{1.5cm}|p{8.5cm}|}
\multicolumn{2}{c}{\textbf{Table: 2}}\\ 
\hline
\textbf{Page \#} & \textbf{McGraw Hill Text} \tabularnewline
\hline 
138 & Persian scholar al-Khawarizmi (ahl-khwa•RIHZ•meh) invented algebra \tabularnewline
\hline
167 & Muslim scholars applied this base-ten numerical system to the study of algebra. \tabularnewline
\hline
\end{longtable}

\section*{Analysis and Critique} 

This violates the evaluation criterion pertaining to accuracy, and by whitewashing the Indian history in favor of Islam, it violates Education Codes 51501 and 60044 pertaining to adverse reflection against Hinduism.

It is incorrect to state that Al-Khawarizmi invented algebra. Algebra was already quite developed in India by the time Muslims arrived as evidenced by Āryabhaṭa (476–550 CE) who authored \textit{Āryabhaṭīya}. 
\begin{quote}
He gave more elegant rules for the sum of the squares and cubes of an initial segment of the positive integers. The sixth part of the product of three quantities consisting of the number of terms, the number of terms plus one, and twice the number of terms plus one is the sum of the squares. The square of the sum of the series is the sum of the cubes.\footnote{Roger Cooke, \textit{A History of Mathematics: A Brief Course} (Hoboken: John Wiley \& Sons, 1991), 207.}
\end{quote}
Brahmagupta (598--668 CE) was an Indian mathematician who authored \textit{Brahmasphuṭasiddhānta}. In his work Brahmagupta solves the general quadratic equation for both positive and negative roots. He was the first to give a general solution to the linear Diophantine equation ax + by = c, where a, b, and c are integers. Unlike Diophantus who only gave one solution to an indeterminate equation, Brahmagupta provided a method to identify \textit{all} integer solutions.\footnote{Carl B. Boyer, “China 	and India” in \textit{A History of Mathematics}  (Hoboken: John Wiley \& Sons, 1991), 221.}

There are three theories about the origins of Arabic Algebra. The first emphasizes the Hindu influence; the second emphasizes Mesopotamian or Persian-Syriac influence, and the third underlines Greek influence. Many scholars believe that it is the result of a combination of all three sources.\footnote{Carl B. Boyer, “The 	Arabic Hagemony” in \textit{A History of Mathematics} (Hoboken: John Wiley \& Sons, 1991).} 

Throughout their time in power, before the fall of the Islamic civilization, the Arabs used a fully rhetorical algebra, where often even the numbers were spelled out in words. The Arabs would eventually replace spelled out numbers (e.g.\ twenty-two) with Arabic numerals (e.g.\ 22), but the Arabs did not adopt or develop a syncopated or symbolic algebra. Boyer further writes:
\begin{quote}
Al-Khwarizmi's work had a serious deficiency that had to be removed before it could serve its purpose effectively in the modern world: a symbolic notation had to be developed to replace the rhetorical form. This step the Arabs never took, except for the replacement of number words by number signs…. Thabit was the founder of a school of translators, especially from Greek and Syriac, and to him we owe an immense debt for translations into Arabic of works by Euclid, Archimedes, Apollonius, Ptolemy, and Eutocius.\footnote{Carl B. Boyer, “The Arabic Hagemony” in \textit{A History of Mathematics} (Hoboken: John Wiley \& Sons, 1991), 234.}
\end{quote}
