\chapter{Summarizing The Hinduphobic Issues In Mcgraw Hill’s Text Books}

We meticulously reviewed the final version of McGraw Hill’s textbooks approved in November 2017 by the State Board of Education for adoption in California (\textit{Impact: World History and Geography: Ancient Civilizations and Impact: World History and Geography: Medieval and Early Modern Times) and their supplementary materials (Inquiry Journal denoted as IJ and Explore Magazine denoted as EM)}. It is our researched view that the adoption of the McGraw Hill materials by the California school districts for grades 6 and 7 is tantamount to willingly and knowingly adopting materials that are known to cause students of Indian heritage or belonging to the Hindu faith to feel \textbf{insecure in their faith and heritage} as well as adopting materials that are out-of-date and filled with historical and cultural inaccuracies. Our reasons are the following:

\begin{itemize} 
\item \textbf{They are in violation of California education codes 51501}, \textbf{60040(b), 60044(a) and 60044(b)} by having significant materials that adversely reflect on India and Hinduism and are known to cause students to feel insecure in their faith and heritage, exposing them to ridicule by their peers of other religious backgrounds. 
\item They \textbf{do not adhere to the History Social Science Standards (1998), the History Social Science Framework (2016)}. Specifically, \textit{they fail to include 1 of 7 topics from the History Social Science (HSS) Standards (adopted by the State of California in 1998) and 7 topic areas from the History Social Science Framework} (adopted by the California State Board of Education in 2016). They are listed as appendices E and F. 
\item They have \textbf{significant adverse reflection on India and Hinduism} by using materials that caricature and stereotype Hinduism, displaying bias and prejudice.
\item They are \textbf{filled with inaccurate content on Indian history and Hinduism as a religion.} 
\end{itemize}

The California Education Code section 51501 mandates that the “state board and any governing board \textbf{shall not adopt any textbooks} or other instructional materials for use in the public schools \textbf{that contain any matter reflecting adversely} upon persons on the basis of race or \textbf{ethnicity}, gender, \textbf{religion}, disability, \textbf{nationality}, or sexual orientation.” \textbf{McGraw Hill’s textbooks do} \underline{\textbf{not comply} } \textbf{with the code section}. 

California Education Code 60044(a) clarifies that a “governing board \textbf{shall not adopt any instructional materials} for use in the schools that, in its determination, contain any matter \textbf{reflecting adversely} upon persons on the basis of race or \textbf{ethnicity}, gender, \textbf{religion}, disability, \textbf{nationality}, or sexual orientation, occupation.”\footnote{“Standards for Evaluating Instructional Materials for Social Content: 2013 Edition,” California Department of Education, accessed February 13, 2018, \url{https://www.cde.ca.gov/ci/cr/cf/documents/socialcontent2013.doc}, \textit{17}.}

\textit{The Standards for Evaluating Instructional Materials for Social Content} adds that textbooks need to “project the cultural diversity of society; \textbf{instill in each child a sense of pride in his or her heritage} ”\footnote{“Standards for Evaluating Instructional Materials for Social Content: 2013 Edition,” California Department of Education, accessed February 13, 2018, \url{https://www.cde.ca.gov/ci/cr/cf/documents/socialcontent2013.doc}, \textit{5}.} and “develop a feeling of self-worth to \textbf{enable all students to become aware and accepting of religious diversity} while being allowed to \textbf{remain secure in any religious beliefs} they may already have.”\footnote{“Standards for Evaluating Instructional Materials for Social Content: 2013 Edition,” California Department of Education, accessed February 13, 2018, \url{https://www.cde.ca.gov/ci/cr/cf/documents/socialcontent2013.doc}, \textit{10}.}

\textbf{McGraw Hill’s textbooks} \underline{\textbf{do not comply} } \textbf{with any of these code sections}. There have already been two lawsuits against the Department of Education related to this and a third is ongoing, and names the Cupertino, Fremont, and Pleasanton school districts for denigrating Hinduism and causing $6^{\rm th}$ and $7^{\rm th}$ graders to feel that there is purposeful discrimination against their religion and heritage.

We find it unfortunate that we have found myriad issues in McGraw Hill’s texts despite the feedback and participation of scholars, teachers, and members of the Hindu community over a four-year period with the Instructional Quality Commission and the State Board of Education. Specifically, 
\begin{itemize} 
\item Gavin Newsom Lt Governor, Tulsi Gabbard, and 15 other elected representatives expressed their concerns by writing to the Board of Education.
\item $\textasciitilde 7,000$ out of $\textasciitilde 11,000$ edits submitted in the HSS Framework adoption during 2014-2016, were on Hinduism and India.
\item More than 70 professors and scholars in various universities, colleges and research institutions, along with numerous teachers and community at large, provided inputs for bringing fairness in sections on India, Hinduism, and other Indian religions.
\item More than 1,500 parents participated in person to express their concerns through public hearings in Sacramento.
\item More than 300 school children across California expressed their view in front of IQC/SBE in Sacramento in 10+ hearings during the 2014-2017 period.
\item More than 42,000 signatures were submitted to improve the representation of India and Hinduism across multiple petitions.
\end{itemize}
McGraw Hill’s representatives saw and heard all of this, and yet we find that the McGraw Hill textbooks continue to have a significantly \textbf{flawed and racist narrative} on India and Hinduism.

We consistently found that the text in discussing Hinduism
\begin{itemize} 
\item essentializes the narrative of Hinduism with caste. Caste is presented as a hierarchy and thus oppression. The explicit and blatant connection that the text makes is to present Hinduism as a religion that oppressed a majority of its own adherents 
\item conflates Hinduism with caste, providing skewed, slanted, and exaggerated emphasis on caste in contrast to positive discussions on other religions
\item links the foundational concept of karma to caste, explicitly suggesting that hierarchy and oppression are in-built in the structure of Hinduism
\item takes an extremely narrow view of the concept of \textit{dharma} (which fundamentally is about performing different kinds of religiosities towards the divine, society, humanity, family, nature, and one another) and ties it around caste
\item caste, hierarchy, and oppression is the fulcrum around which the difference between Hinduism and Buddhism is placed. 
\end{itemize}
The conflation of Hinduism with caste hierarchy and oppression causes shame, impacting the healthy identity of Hindu children. There are innumerable instances where the Hindu children have either distanced themselves from the heritage of their parents and ancestors or have lived a life of double identity, walking extra miles to dis-identify themselves from their tradition in their own social and public sphere. Many students testified to this during the IQC and SBE hearings from 2014-2017. \textit{\textbf{The textbook is in violation of the educational codes, section 51501, 60040(b), and 60044(a)}}, pertaining to matters of “Ethnic and Cultural Groups.” The \textit{Standards for Evaluating Instructional Materials for Social Content} outlining the purpose of standards states the following: 
\begin{quote}
The standards project the cultural diversity of society; instill in each child a sense of pride in his or her heritage; develop a feeling of self-worth related to equality of opportunity; eradicate the roots of prejudice; and thereby encourage the optimal individual development of each student.\footnote{“Standards for Evaluating Instructional Materials for Social Content: 2013 Edition,” California Department of Education, accessed February 13, 2018, \url{https://www.cde.ca.gov/ci/cr/cf/documents/socialcontent2013.doc}, \textit{5}.}
\end{quote}
The conjoining of Hinduism with caste, far from instilling pride, instills shame and a lack of self-worth. Children become targets of prejudice among their peers, and their healthy development is adversely impacted. The publisher must have been familiar with the consequences, as many children, discussed this during many SBE/IQC hearing spanning the three-year period when the revision of the HSS Framework was an agenda topic. They testified in front of the commission regarding the negative consequences that they experience due to the negative and skewed portrayal of Hinduism. Thus, the current discourse on Hinduism in the textbook consequently violates the “adverse reflection” clause of the \textit{Standards} under the “Ethnic and Social Groups.”
\begin{quote}
\textit{Adverse reflection}. Descriptions, depictions, labels, or rejoinders that tend to demean, stereotype, or patronize minority groups are prohibited.\footnote{“Standards for Evaluating Instructional Materials for Social Content: 2013 Edition,” California Department of Education, accessed February 13, 2018, \url{https://www.cde.ca.gov/ci/cr/cf/documents/socialcontent2013.doc}, \textit{6}.}
\end{quote}
Further, the \textit{Standards} pertaining to religious matter citing Education Code Sections 51501, 60044(a) and (b) specifically state the following:
\begin{quote}
The standards enable all students to become aware and accepting of religious diversity while being allowed to remain secure in any religious beliefs they may already have.\footnote{“Standards for Evaluating Instructional Materials for Social Content: 2013 Edition,” California Department of Education, accessed February 13, 2018, \url{https://www.cde.ca.gov/ci/cr/cf/documents/socialcontent2013.doc}, \textit{10}.}
\end{quote}
And that
\begin{quote}
The standards will be achieved by depicting, when appropriate, the diversity of religious beliefs held in the United States and California, as well as in other societies, without displaying bias toward or prejudice against any of those beliefs or religious beliefs in general.\footnote{“Standards for Evaluating Instructional Materials for Social Content: 2013 Edition,” California Department of Education, accessed February 13, 2018, \url{https://www.cde.ca.gov/ci/cr/cf/documents/socialcontent2013.doc}, \textit{10}.}
\end{quote}
Given that the above \textit{Standards} emanate from the constitutions of the United States and California, the document mandates compliance insisting
\begin{quote}
1. \textit{Adverse reflection}. No religious belief or practice may be held up to ridicule and no religious group may be portrayed as inferior.

2. \textit{Indoctrination}. Any explanation or description of a religious belief or practice should be presented in a manner that does not encourage or discourage belief or indoctrinate the student in any religious belief.\footnote{“Standards for Evaluating Instructional Materials for Social Content: 2013 Edition,” California Department of Education, accessed February 13, 2018, \url{https://www.cde.ca.gov/ci/cr/cf/documents/socialcontent2013.doc}, \textit{10}.}
\end{quote}
Based on the testimonies of the Hindu children provided to the IQC and SBE, it should be more than apparent that the discourse in the \textbf{McGraw Hill textbooks do not allow them to remain secure in the beliefs} of their tradition or the practices of their parents. Caste oriented textbooks like \textbf{McGraw’s have already been proven to lead to bullying of Hindu kids} by non-Hindu kids and \textbf{violation of the judgment provided in the CAPEEM lawsuit of 2009\footnote{“Excerpts from Judgment in Favor of CAPEEM (Discriminatory Intent and Disparate Treatment) Lawsuit Challenging California’s Curriculum,” California Parents for the Equalization of Educational Materials (CAPEEM), accessed February 14, 2018, \url{http://capeem.org/wp-content/uploads/2017/09/sumjudgrulingexcerpts.pdf}.}} by the US District Court. To make matters worse, McGraw Hill ignores the History Social Science Framework, which requires the discussion of Sages Valmiki and Vyasa who were from under-privileged backgrounds and rose to the highest levels of respect within Hinduism. Including their stories would have punctured the \textbf{“hierarchy and oppressive caste equaling Hinduism” narrative} that McGraw Hill has woven.

Further, these textbooks, while denigrating Hinduism, simultaneously speak positively about other religions such as Judaism, Christianity, Islam, Buddhism, and Jainism. This is indoctrination against Hinduism in favor of the other religions.

The textbook singles out Hinduism for this derogatory treatment, exposing Hindu students to ridicule by portraying it as inherently oppressive. In the contemporary world, no child would want to be associated with a system of belief, which is being represented as hierarchical and oppressive. The Bible regards Gentiles or non-believers as inferior, and the Koran too contains negative attitudes against what it considers infidels. The Bible and the Koran permit slavery and the latter has also been used to allow enslavement of “infidels” as war booty. Textbooks \textit{\textbf{rightfully and rightly}} do not emphasize these facts and do not make the ensuing social and religious discrimination in traditional Christian and Islamic societies based on religion as their defining features—yet it singles out Hinduism for this improper treatment. It is therefore discriminatory and prejudicial to implicate Hinduism for caste inequities if Islam and Christianity are not being implicated for slavery, slave trade, and killing of non-believers of their respective faiths. California law states that elementary and middle school textbooks are not the right place to demonize the faiths in which children are raised. \textit{\textbf{Singling Hinduism out for negative portrayals tantamount to prejudice and discrimination, and hence the violation of Education codes 60044(a) and (b).}} 

This treatment of Hinduism on its own is sufficient to reject McGraw Hill’s texts for violations of the state education codes. Unfortunately, the issues with these textbooks are not solely related to this narrative alone.

The textbook continues to fail to recognize the last 50 years of scholarship on the Harrapan Civilization often referred to in India (where the scholarship is predominantly centered post partition of the Indian subcontinent) as the Sindhu-Sarasvati Civilization or the Indus-Sarasvati civilization. It fails to recognize
\begin{itemize} 
\item its breadth (4,000 sites identified, with 2,500 sites along the Sarasvati Riverbed)
\item some of the largest sites (Rakhigarhi, Dholavira, Kalibangan, Bhirrana etc.)
\item Numerous findings — the Namaste gesture, the worship of the \textit{Ficus religiosa} (\textit{peepal}) tree, the swastika on tablets, evidence of practices of yoga, fire altars, etc. — testify to a strong continuity with later Indic religious systems and traditions some of which are included in the History Social Science (HSS) Framework
\item discussion of the significance of the Sarasvati river (as discussed in the HSS Framework)
\end{itemize}
Additional problems include the following:
\begin{enumerate} 
\item It creates an entire narrative around the Indo-European and Aryan people ignoring alternate perspectives, and the fact that any semblance of the Aryans having invaded the Indian Subcontinent from the northwest has been removed from the HSS Framework. Though the HSS Framework discusses the migration of Aryans into the subcontinent, it mentions that there is a minority view that supports the migration of the “Aryans” out of India itself—we would like to add that this “minority” view has quite a few takers, and there are numerous archaeologists and Sanskritists, who have conclusively shown that the there is no evidence whatsoever of the “Aryans” either invading or migrating to India from outside the subcontinent. A non-biased text would have discussed the alternative view. \textit{\textbf{This is particularly important because the conceptualization of Aryans as a superior race bears squarely the imprint of racist scholarship, developed, harbored, fostered, and nurtured during the European colonial period. It was blatantly used for the subjugation of the colonized Indian people, and ultimately resulted in the genocide of the Jewish people.}} 

\item It completely distorts the narrative on the Mauryan empire (through a large number of errors of omission and commission—see the tables for details), beginning with discredited theories of why Alexander retreated from India to the reason for the downfall of the Mauryan Empire. These omissions are of a serious nature because they underline the Orientalist (and by default the colonialist and the racist) project of describing the “other.” The identity of the post-Renaissance Europe is closely entwined with the glory of Greece and the greatness of Greeks—particularly in relationship to Persians and people living east of Persia. It was in this backdrop the narrative around Alexander was developed in that his soldiers, after the Battle of Hydaspes, grew tired and homesick and wanted to return home because of which they rebelled. The fact is that after the fierce battle, Alexander’s troops became scared of what they would encounter in deeper recess of India where they heard of bigger armies and braver soldiers—this is based on the accounts of Plutarch, Arrian, and Megasthenese; the historians of that period.\footnote{Arrian, Plutarch, and Quintus Curtius Rufus, \textit{The Brief Life and Towering Exploits of History's Greatest Conqueror: As Told By His Original Biographers}, eds.\ Tania Gergel and Michael Wood (New York: Penguin Books, 2004).} 

\item The narrative on Indian literature is wrong and problematic on many levels—limiting it to moral lessons and some historical texts while ignoring the numerous medical, mathematical, astronomical, philosophical, and other scientific treatises. The orientalist descriptions of India denied it any contributions in science and mathematics. McGraw Hill textbooks further them. This exemplifies the nexus of orientalism and racism. At the advent of post-Renaissance Enlightenment, when Reason emerged supreme in the mainstream \textit{zeitgeist} of Europe, the European consciousness appropriated it for its own “race,” and simultaneously denied it to the rest of the population of the world or, in the language of the colonial Europe, to other “races” of the world. Consequently, Indian consciousness became mystical, metaphysical, poetic, and romantic but never rational and scientific (see Sen 2005).\footnote{Amartya Sen, \textit{The Argumentative Indian: Writings on Indian Culture, History and Identity} (London: Penguin Books, 2005).} The denial of rational and scientific contributions, manifesting in medical, mathematical, astronomical, philosophical, and scientific contributions in the McGraw Hill textbooks come from this prejudice. 

\item In the colonial period, James Mill, in \textit{History of British India}, working under the umbrella of “divide and rule,” developed this theory that Hinduism is a rag-tag collocation of many different religions. Though it is true that Hinduism comprises of many different traditions, but since antiquity it has had an overarching umbrella—coming from the Vedas and Upanishads—that has held its numerous traditions in a bouquet of “unity-in-diversity.” McGraw Hill text regurgitates the colonial construct of James Mill in that the early Hinduism was a number of disparate religions.

\item It vivisects the Hindu society stating that Brahmins (one social category in Hindu society) had a religion apart from everyone else. As mentioned earlier, creating as many fissures as possible in the Indian society was an explicit aim of the colonial British. The McGraw Hill text, it seems is continuing the British objective of keeping the Indian and the Hindu population divided. The truth is that the Brahmins were the keepers and preservers of the tradition, which was followed by members of other social categories. In fact, when Hiuen Tsiang, a Chinese monk, visited India between 680-700 CE, he notes that Brahmins were the preservers of the Buddhist tradition as well.\footnote{Samuel Beal, \textit{Si-yu-ki: Buddhist Records of the Western World: Translated from the Chinese of Hiuen Tsiang: Volume 1} (London: Kegan Paul, Trench, Trubner &amp; Co. Ltd., 1906).} 

\item It butchers the historically well-accepted narrative around the origins of the Vedas and development of subsequent scriptures, the oral tradition, and development of Sanskrit as a language. 

\item \textit{It creates} \textit{\textbf{adverse reflection}} \textit{ on Hinduism} by making Buddhism as one of the many efforts by people dissatisfied with Hinduism to create new religions. In addition, this is also false. McGraw Hill is perpetuating the colonial project of keeping the traditions of India divided and fighting among one another. The spiritual philosophy of India, right from the very beginning, has never created a final cap on the nature of religious Truth. It has what has been referred to as “open architecture” (Malhotra 2014).\footnote{Rajiv Malhotra, \textit{Indra’s Net: Defending Hinduism’s Philosophical Unity} (Noida: HarperCollins \textit{Publishers } India, 2014).} Buddha founded Buddhism not because he was dissatisfied with Hinduism; he was not even looking for discovering or inventing a new path. After having been shielded from seeing or experiencing suffering (see his story discussed below for details), in one of his travels, he saw a diseased person, an old person, and a corpse. He also encountered a monk, who had no trace of suffering on his face. That triggered in him what we will call in current parlance “existential crisis,” and he set out on a path to end suffering and find enlightenment—the same enlightenment, which every Hindu was expected to seek in the final stage of his life through \textit{sannyasa}. He did become enlightened, and preached to people the ways to end suffering. In his sermons on the “Four Noble Truths” and the “Noble Eightfold Path,” he is not speaking about his dissatisfaction with Hinduism. What he is saying is that life is suffering and there is a way to end that suffering. 

\item It avoids discussing topics included in the HSS Framework including male and female Deities, Yoga, and meditation.
\item It summarizes revered scriptures like the Bhagavad Gita and Ramayana in a derogatory manner.
\end{enumerate}
In the context of today’s society, it is the \textbf{responsibility of educators to teach classroom appropriate materials which remove old stereotypes and reduce inter-ethnic and inter-religious hatred—not propagate them.} The concerned textbooks of McGraw Hill do exactly the opposite as they violate the HSS Standards, HSS Framework, and the California Education Code sections 51501, 60040(b), 60044(a) and 60044(b). Adopting McGraw Hill’s materials for grade 6 and 7 is tantamount to willingly and knowingly adopting materials that are known to cause students of Indian heritage or belonging to the Hindu faith to feel \textbf{insecure in their faith and heritage} as well as adopting materials that are out-of-date and filled with historical and cultural inaccuracies. It will remain to be seen how many of the California school districts will adopt the books. 

