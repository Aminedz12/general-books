\chapter[Wrongly Crediting Buddhism for\\ Mauryan Policies]{Wrongly Crediting Buddhism for Mauryan Policies}

\begin{longtable}{|>{\raggedleft}p{1.5cm}|p{8.5cm}|}
\multicolumn{2}{c}{\textbf{Table: 1}}\\ 
\hline
\textbf{Page \#} & \textbf{McGraw Hill Text} \tabularnewline
\hline 
271 & After one battle, he [Ashoka] looked at the fields covered with dead and wounded soldiers. He was horrified by what he saw. He decided that he would follow Buddhist teachings and become a man of peace. \tabularnewline
\hline
\end{longtable}

\section*{Analysis and Critique} 

This is in violation of California Education Codes 51501 and 60044—adverse reflection—and evaluation criteria clauses pertaining to historical inaccuracy, not remaining neutral in matters of religion, advocating Buddhism over Hinduism, subtly deriding Hinduism, and thereby instilling prejudice in Hindu and non-Hindu children against Hinduism.

When this statement is mentioned in the backdrop of Hinduism, Hinduism becomes a religion promoting war and warfare. Besides Ashoka had already embraced Buddhism before the war in question—the Kalinga war.

\begin{longtable}{|>{\raggedleft}p{1.5cm}|p{8.5cm}|}
\multicolumn{2}{c}{\textbf{Table: 2}}\\ 
\hline
\textbf{Page \#} & \textbf{McGraw Hill Text} \tabularnewline
\hline 
271 & Ashoka kept his promise. During the rest of his life, he tried to improve the lives of his people. Ashoka made laws that encouraged good deeds, family harmony, nonviolence, and toleration of other religions. He created hospitals for people and for animals. He built fine roads, with rest houses and shade trees for the travelers’ comfort. \tabularnewline
\hline
\end{longtable}

\section*{Analysis and Critique} 

This is in violation of California Education Codes 51501 and 60044—adverse reflection—and evaluation criteria clauses pertaining to historical inaccuracy, not remaining neutral in matters of religion, advocating Buddhism over Hinduism, subtly deriding Hinduism, and thereby instilling prejudice in Hindu and non-Hindu children against Hinduism.

It is inaccurate to state that he only did these things after the war of Kalinga. Ashoka, like his father and grandfather, sponsored the construction of thousands of roads, waterways, canals, rest houses, hospitals, and other types of infrastructure throughout his reign. The above deeds were part of responsibilities and duties of the kings of the time. \textit{Arthaśāstra} that was a compendium for governance in those times—written by Kautilya, who was the teacher and mentor of Ashoka’s grandfather Chandragupta—mandated that a king performed them. \textit{Arthaśāstra} also mandates a king to not interfere with religious beliefs of people and promote tolerance towards and diversity and plurality among religious traditions. 

The authors of McGraw Hill have explicitly and subtly shown Hinduism in a bad light as they have highlighted the beauty of Buddhism. The good deeds of Ashoka were equally possible through the “Hindu” text \textit{Arthaśāstra}. There is a space where both Hinduism and Buddhism can be shown in a beautiful light without speaking ill of the other. That space has not been touched upon by the McGraw author(s).

\begin{thebibliography}{99}
\bibitem{chap10-key1} Kautilya. \textit{The Arthaśāstra}. Translated and edited by L. N. Rangarajan. New Delhi: Penguin Books, 1992.
\end{thebibliography}

\begin{longtable}{|>{\raggedleft}p{1.5cm}|p{8.5cm}|}
\multicolumn{2}{c}{\textbf{Table: 3}}\\ 
\hline
\textbf{Page \#} & \textbf{McGraw Hill Text} \tabularnewline
\hline 
272 & Ashoka’s able leadership helped the Mauryan Empire prosper. India’s good roads helped it become the center of a large trade network that stretched to the Mediterranean Sea. \tabularnewline
\hline
\end{longtable}

\section*{Analysis and Critique} 

These violate evaluation criteria pertaining to accuracy and inclusion of best recent scholarship. 

Attributing these things as unique to Ashoka is historically inaccurate. Chandragupta, Ashoka's grandfather, established links with the western Hellenistic states. Road building was an ongoing exercise which, if not continuing from earlier times, started with the Nanda and was continued by Chandragupta (Ashoka’s grandfather), Bindusāra (Ashoka's father) and Ashoka. Trade flourished during Chandragupta's and Bindusāra's rule as well as Ashoka's rule. Kautilya’s \textit{Arthaśāstra} that mandates the kings to build roads was an \textit{Arthaśāstra} in a long lineage of \textit{Arthaśāstra}-s. It would not be surprising that the earlier \textit{Arthaśāstra}-s —none of which are extant now—made building roads as one of the many responsibilities of the king. How does one know however that earlier \textit{Arthaśāstra}-s existed? \textit{Arthaśāstra} of Kautilya mentions of the continued lineage of which it is a part.
