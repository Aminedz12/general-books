\chapter{ಧ್ಯಾನ}\label{chap4}

ಪರಮಾರ್ಥಕ್ಕೆ ಸಾಧನವಾಗಿ ಕರ್ಮ, ಭಕ್ತಿ ಮತ್ತು ಜ್ಞಾನ ಎಂಬ ಮೂರು ಮಾರ್ಗಗಳು ಇವೆ. ಇವುಗಳಲ್ಲಿ ಉಪಾಸನೆ ಎನ್ನುವುದು ಭಕ್ತಿಯಲ್ಲಿ ಸೇರುತ್ತದೆ. ಕರ್ಮ ಮತ್ತು ಭಕ್ತಿಯ ವಿಷಯವಾಗಿ ಈಗ ವಿಚಾರ ಮಾಡಲಾಗುವುದಿಲ್ಲ. ಉಪಾಸನೆ ಎಂದರೇನು? ಈ ವಿಷಯವನ್ನು ಕುರಿತು ಶಂಕರಭಗವತ್ಪಾದರು ತಾವು ಬರೆದ ಛಾಂದೋಗ್ಯೋಪನಿಷತ್ ಭಾಷ್ಯದ ಪ್ರಾರಂಭದಲ್ಲಿ ಚೆನ್ನಾಗಿ ಸ್ಪಷ್ಟ ಪಡಿಸಿದ್ದಾರೆ. ಉಪಾಸನೆ ಎನ್ನುವುದು ಒಂದು ಮನೋಧರ್ಮ. ಜ್ಞಾನ ಎನ್ನುವುದು ಕೂಡ ಒಂದು ಮನೋಧರ್ಮ. ಹೀಗಿರುವಾಗ ಜ್ಞಾನಕ್ಕೂ ಉಪಾಸನೆಗೂ ವ್ಯತ್ಯಾಸವೇನು? ಉಪಾಸನೆ ಮಾಡುವಾಗಲೂ ಒಂದು ಜ್ಞಾನ ಇದ್ದೇ ಇರಬೇಕು. ಉದಾಹರಣೆಗೆ ದೇವಿಯನ್ನು ಧ್ಯಾನಿಸುತ್ತೇವೆ. ಹಾಗೆ ದೇವಿಯನ್ನು ಧ್ಯಾನಿಸುವಾಗ ದೇವಿಗೆ ಸಂಬಂಧಪಟ್ಟ ಜ್ಞಾನ ಉಂಟಾಗುತ್ತದೆ. ಏಕೆಂದರೆ ಜ್ಞಾನವಿಲ್ಲದೆ ಉಪಾಸನೆ ಹೇಗೆ ಸಾಧ್ಯ? ಈ ಆಕ್ಷೇಪವನ್ನು ಮನಸ್ಸಿನಲ್ಲಿಟ್ಟುಕೊಂಡೇ ಜ್ಞಾನಕ್ಕೂ ಉಪಾಸನೆಗೂ ವ್ಯತ್ಯಾಸವೇನು ಎನ್ನುವ ಪ್ರಶ್ನೆ ಉಂಟಾಯಿತು. ವ್ಯತ್ಯಾಸವಿಲ್ಲ ಎನ್ನುವ ಒಂದು ಆಕ್ಷೇಪವನ್ನು ಸಾಮಾನ್ಯ ದೃಷ್ಟಿಯಿಂದ ಭಗವತ್ಪಾದರು ತೆಗೆದುಕೊಂಡು, `ಕಸ್ತರ್ಹಿ ಅದ್ವೈತ ಜ್ಞಾನಸ್ಯ ಉಪಾಸನಾನಾಂ ಚ ವಿಶೇಷಃ'? ಎಂದು ಕೇಳಿದರು. `ಅದ್ವೈತ' ಜ್ಞಾನಕ್ಕೂ ಉಪಾಸನೆಗೂ ಏನು ಭೇದ' ಎನ್ನುವುದು ತಾತ್ಪರ್ಯ. ಇದಕ್ಕೆ ಉತ್ತರವಾಗಿ `ಉಚ್ಯತೇ-ಸ್ವಾಭಾವಿಕಸ್ಯಾತ್ಮನ್ಯಕ್ರಿಯೇಽಧ್ಯಾರೋಪಿತಸ್ಯ ಕರ್ತ್ರಾದಿಕಾರಕಕ್ರಿಯಾ ಫಲಭೇದ ವಿಜ್ಞಾನಸ್ಯ ನಿವರ್ತಕಮದ್ವೈತ ವಿಜ್ಞಾನಮ್, ರಜ್ಜ್ವಾದಾವಿವ ಸರ್ಪಾದ್ಯಧ್ಯಾರೋಪಲಕ್ಷಣ ಜ್ಞಾನಸ್ಯ ರಜ್ಜ್ವಾದಿ ಸ್ವರೂಪನಿಶ್ಚಯಃ ಪ್ರಕಾಶನಿಮಿತ್ತಃ|-ಎಂದು ಅವರು ಸ್ಪಷ್ಟ ಪಡಿಸಿದರು.

ಆದ್ವೈತಜ್ಞಾನಕ್ಕೂ ಉಪಾಸನೆಗೂ ಬಹಳಷ್ಟು ವ್ಯತ್ಯಾಸವಿದೆ. ಆತ್ಮನು ಆಕ್ರಿಯ. ಹೀಗಿದ್ದರೂ `ನಾನು ಮಾಡುತ್ತೇನೆ' ಎನ್ನುವ ಕರ್ತೃತ್ವ, `ನಾನು ಕ್ರಿಯೆಗೆ ಕರ್ತೃವಾಗಿದ್ದೇನೆ.' ಎನ್ನುವ ಕರಣತ್ವ, `ನನ್ನಿಂದ ಕೆಲಸವಾಗುತ್ತದೆ' ಎನ್ನುವ ಅಧಿಷ್ಠಾನತ್ವ ಇಂಥ ಶಾಸ್ತ್ರದಲ್ಲಿ ಕಾರಕವೆಂದು ಹೇಳಲಾಗಿರುವ ಆರೂ ಆತ್ಮನ ಮೇಲೆ ಆರೋಪಣೆ ಮಾಡಲ್ಪಡುತ್ತವೆ. ಇವುಗಳು ನಿಜಕ್ಕೂ ಆತ್ಮನಲ್ಲಿ ಇಲ್ಲ. ಆದ್ದರಿಂದಲೇ ಇವುಗಳು ಆತ್ಮನ ಮೇಲೆ ಆರೋಪಿತವಾಗಿದೆಯೆಂದು ಹೇಳಲಾಗಿದೆ. ಇದಕ್ಕೆ ಉಪಮಾನವೇನು? ಹೆಚ್ಚಾಗಿ ಬೆಳಕಿಲ್ಲದ ಜಾಗದಲ್ಲಿ ಹಗ್ಗ ಹಾವಾಗಿ ತೋರಬಹುದು. ಇದೇ ರೀತಿ ಇಲ್ಲಿ, ಹಗ್ಗದಲ್ಲಿ ಹಾವು ಕಲ್ಪಿಸಿದಂತೆ ಆತ್ಮನಲ್ಲಿ ಕರ್ತೃತ್ವದಂತಹ ಕಾರಕಗಳೂ ಮತ್ತು ಕ್ರಿಯೆ, ಫಲ ಇಂತಹವುಗಳ ಕಲ್ಪನೆಯನ್ನು ಯಾವುದು ದೂರಮಾಡುವುದೋ ಅದೇ ಅದ್ವೈತ ಜ್ಞಾನವೆನ್ನಲಾಗುವುದು. ಹೀಗಿರುವಾಗ ಉಪಾಸನೆ ಅಥವಾ ಧ್ಯಾನ ಎನ್ನುವುದು ಯಾವುದು?

`ಉಪಾಸನಂ ತು ಯಥಾ ಶಾಸ್ತ್ರಸಮರ್ಥಿತಂ ಕಿಂಚದಾಲಂಬನಮುಪಾದಾಯ ತಸ್ಮಿನ್ ಸಮಾನಚಿತ್ತವೃತ್ತಿ ಸಂತಾನಕರಣಂ ತದ್ವಿಲಕ್ಷಣಪ್ರತ್ಯಯಾನನ್ತರಿತಮಿತಿ ವಿಶೇಷಃ| ತಾನಿ ಏತಾನಿ ಉಪಾಸನಾನಿ ಸತ್ತ್ವ ಶುದ್ಧಿಕರತ್ವೇನ ವಸ್ತುತತ್ತ್ವಾವ ಭಾಸಕತ್ವಾದದ್ವೈತಜ್ಞಾನೋಪಕಾರಕಾಣಿ|'

-ಎನ್ನುವುದು ಭಗವತ್ಪಾದರು ಕೊಟ್ಟ ಉತ್ತರ. ಒಂದು ವಸ್ತು ಒಂದು ವಿಧವಾಗಿ ಇಲ್ಲವೆಂದುಕೊಳ್ಳೋಣ. ಹಾಗಿದ್ದರೂ ಆ ವಸ್ತು ಇಲ್ಲದಿರುವ ರೀತಿಯಲ್ಲಿ ಅದು ಇರುವಂತೆ ಚಿಂತನೆ ಮಾಡಬಹುದು. ಹೀಗೆ ಚಿಂತನೆ ಮಾಡಿದರೆ ಯಾವುದಾದರೂ ಪ್ರಯೋಜನವಿದೆಯೆ ಎಂದು ಕೇಳಿದರೆ ವಿಶೇಷವಾದ ಫಲವಿದೆ ಎನ್ನುವುದು ಉತ್ತರವಾಗುತ್ತದೆ. ಈ ಚಿಂತನೆ ಯಾವ ವಿಧವಾಗಿರಬೇಕು ಎನ್ನುವುದಕ್ಕೆ `ಯಥಾಶಾಸ್ತ್ರಂ' ಶಾಸ್ತ್ರದಲ್ಲಿ ಹೇಳಿದಂತೆ ಎಂದು ಹೇಳಿದರು. ಉದಾಹರಣೆಗೆ ಶಾಸ್ತ್ರದಲ್ಲಿ.- 

`ಮನೋ ಬ್ರಹ್ಮೇತಿ ಉಪಾಸೀತ' ಎಂದಿದೆ. ಮನಸ್ಸೆನ್ನುವುದು ಎಲ್ಲೆಯುಳ್ಳದ್ದು. ಆದರೆ ಬ್ರಹ್ಮವೆನ್ನುವುದು ಎಲ್ಲೆ ಇಲ್ಲದಿರುವುದು. ಆದರೂ ಇಂಥ ಮನಸ್ಸನ್ನು ಬ್ರಹ್ಮವೆಂದು ಉಪಾಸನೆ ಮಾಡಿಕೊಂಡು ಬಾ ಎಂದು ಮೇಲೆ ಹೇಳಿದ ಶಾಸ್ತ್ರವಾಕ್ಯದಲ್ಲಿರುವ ಅಪ್ಪಣೆ. ಹೀಗೆ ಮಾಡಿದರೆ ಒಳ್ಳೆಯ ಫಲವುಂಟು. ಈ ಧ್ಯಾನ ಜ್ಞಾನವಾಗುವುದಿಲ್ಲ. ಇದರ ಬಗ್ಗೆ' `ಅಲ್ಪಾಲಂಬನ ತಿರಸ್ಕಾರೇಣ ಉತ್ಕೃಷ್ಟ ವಸ್ತ್ವಭೇದಧ್ಯಾನಂ ಸಂಪತ್' ಎಂದು ಹೇಳಲ್ಪಟ್ಟಿದೆ. ಮನಸ್ಸೆನ್ನುವುದು ಒಂದು ಅಲ್ಪವಾದ ಅವಲಂಬ. ಇದನ್ನು ತಿರಸ್ಕರಿಸಿ ಉತ್ತಮವಾದ ಬ್ರಹ್ಮವನ್ನು ಧ್ಯಾನ ಮಾಡುವುದನ್ನು `ಸಂಪತ್' ಎನ್ನಲಾಗುವುದು. ಶಾಸ್ತ್ರಗಳಲ್ಲಿ ಪ್ರತೀಕೋಪಾಸನೆಯ ಬಗ್ಗೆ ವಿವರವಿದೆ. ಅದರ ಪ್ರಕಾರ ನಾವು ಒಂದು ಸಾಲಿಗ್ರಾಮವನ್ನು ವಿಷ್ಣುವೆಂದುಕೊಳ್ಳುತ್ತೇವೆ. ನಮಗೆ ಸಾಲಿಗ್ರಾಮ ಒಂದು ಕಲ್ಲೆಂದು ಗೊತ್ತು. ಆದರೂ ಅದರಲ್ಲಿ ವಿಷ್ಣು ಭಾವನೆಯನ್ನು ಇಟ್ಟುಕೊಳ್ಳುತ್ತೇವೆ. ಹೀಗೆ ಮಾಡುವುದಕ್ಕೆ ಪ್ರತೀಕೋಪಾಸನೆ ಎಂದು ಹೆಸರು. ಇದೇ ರೀತಿ ಅಂಗಾವಬದ್ಧ ಉಪಾಸನೆ ಎನ್ನುವುದೂ ಇದೆ. ಒಂದು ಯಜ್ಞದಲ್ಲಿ ಒಂದು ಸಾಮಗಾನ ಮಾಡಬೇಕಾಗಿ ಬರುತ್ತದೆ. ಆ ಸಾಮಕ್ಕೆ ಐದು ಭಾಗಗಳಿವೆ. ಆ ಒಂದೊಂದು ಭಾಗದಲ್ಲೂ ಒಂದೊಂದು ವಿಧವಾದ ದೃಷ್ಟಿಯನ್ನು ಇಟ್ಟುಕೊಂಡಿರಬೇಕು. ಹೀಗೆ ಮಾಡುವುದಕ್ಕೆ ಅಂಗಾವಬದ್ಧ ಉಪಾಸನೆಯೆಂದು ಹೆಸರು. ಈ ಉಪಾಸನೆ ಏತಕ್ಕಾಗಿ ಮಾಡಬೇಕು? ಯಾಗವನ್ನು ಮಾಡಿದರೆ ಫಲವುಂಟು. ಆದರೆ ಆ ಉಪಾಸನೆಯನ್ನೂ ಮಾಡಿದರೆ ಒಂದು ವಿಶೇಷವಾದ ಫಲವುಂಟೆನ್ನುವುದು ಇದಕ್ಕೆ ಉತ್ತರ.

ಇವುಗಳಿಂದ ಜ್ಞಾನಕ್ಕೂ ಉಪಾಸನೆಗೂ ವ್ಯತ್ಯಾಸವಿದೆಯೆಂದು ಸ್ಪಷ್ಟವಾಗುತ್ತದೆ. ಹೀಗಿದ್ದರೂ ಉಪಾಸನೆ ಜ್ಞಾನಕ್ಕೆ ಸಾಧನವಾಗಿದೆ. ಆದ್ದರಿಂದ ಮನುಷ್ಯ ಜನ್ಮವನ್ನು ಪಡೆದವನು ಯಾವುದಾದರೂ ಉಪಾಸನೆಯನ್ನು ಮಾಡಿ ಶ್ರೇಯಸ್ಸು ಪಡೆಯುವುದು ಒಳ್ಳೆಯದು.

ಉಪಾಸನೆ ಮಾಡುವುದು ಆವಶ್ಯಕವೆಂದು ನೋಡಿಯಾಯಿತು. ಇದು ಮಾಡಬಹುದೇ? ಈ ವಿಷಯವನ್ನು ಕುರಿತು ಯೋಗಶಾಸ್ತ್ರದಲ್ಲಿ ಉತ್ತರ ಹೇಳಿದೆ. ಅಲ್ಲಿ, ಮನಸ್ಸು ಐದು ವಿಧವಾಗಿ ವಿಂಗಡಿಸಲ್ಪಟ್ಟಿದೆ. ಮನಸ್ಸಿನ ಸ್ಥಿತಿಯನ್ನು ಕ್ಷಿಪ್ತ, ಮೂಢ, ವಿಕ್ಷಿಪ್ತ, ಏಕಾಗ್ರ ಮತ್ತು ನಿರುದ್ಧವೆಂದು ಐದು ವಿಧವಾಗಿ ವಿಂಗಡಿಸಿದರು. ಐದನೆಯದಾಗಿ ಹೇಳಲ್ಪಟ್ಟ ನಿರುದ್ಧವೆನ್ನುವ ಮನಃಸ್ಥಿತಿ ಪ್ರಾಪ್ತವಾದರೆ ಬೇರೆ ಯಾವುದೂ ಬೇಕಾಗಿಲ್ಲ. ಹೀಗೆ ಐದು ವಿಧವಾದ ಮನಃಸ್ಥಿತಿಗಳಿದ್ದರೂ ನಾವು ಸಾಮಾನ್ಯವಾಗಿ ಕ್ಷಿಪ್ತವಾಗಿರುವ ಮನಸ್ಸನ್ನೇ ನೋಡುತ್ತೇವೆ. ಕ್ಷಿಪ್ತ ಸ್ಥಿತಿಯಲ್ಲಿರುವ ಮನಸ್ಸನ್ನು ಕುರಿತು.

`ವಿಷಯೇಷು ಜ್ಞಾಪಕಂ ಮನಃ ಚಿತ್ತಮಿತ್ಯಬಿಧೀಯತೇ|' (ವಿಷಯಗಳನ್ನು ಅರಿಯುವ ಮನಸ್ಸಿಗೆ ಚಿತ್ತವೆಂದು ಹೆಸರು) ಎಂದು ಯೋಗ ಶಾಸ್ತ್ರದಲ್ಲಿರುವಂತೆಯೇ ಹೇಳಬೇಕು. ಯಾವಾಗಲೂ ಮನಸ್ಸಿಗೆ ಹೊರಗಿನ ವಸ್ತುಗಳ ಬಗ್ಗೆ ಯಾವುದಾದರೂ ವಿಷಯ ತೋರುತ್ತಲೇ ಇರುತ್ತದೆ. ಜಾಗ್ರತ್, ಸ್ವಪ್ನ ಈ ಎರಡು ಸ್ಥಿತಿಗಳಲ್ಲಿಯೂ ವಿಷಯವಿಲ್ಲದ ಸಮಯವೆನ್ನುವುದು ಅಪರೂಪ. ಇದು ಕ್ಷಿಪ್ತವಾಗಿರುವ ಮನಃಸ್ಥಿತಿ, ಅದಾದಮೇಲೆ ಹೇಳಿರುವುದು ಮೂಢಸ್ಥಿತಿ. ದುಃಖದಲ್ಲೊ ಸೋಮಾರಿತನದಲ್ಲೊ ಈ ಸ್ಥಿತಿಗೆ ಪ್ರಿಯತ್ವವಿರುತ್ತದೆ. ಸೋಮಾರಿ ತನ್ನ ದುಃಖ ಮುಂತಾದವುಗಳಲ್ಲಿರುವ ಮನಃಸ್ಥಿತಿಗೆ `ಮೂಢ' ಎನ್ನಲ್ಪಡುತ್ತದೆ. ಕ್ಷಿಪ್ತದಂತೆ ಮೂಢ ಸ್ಥಿತಿಯಲ್ಲಿರುವ ಮನಸ್ಸಿನಿಂದ ಪರವಸ್ತುವನ್ನು ಪಡೆಯಲು ಸಾಧ್ಯವಿಲ್ಲ. ಆದ್ದರಿಂದ ಮನಸ್ಸನ್ನು ಸ್ವಲ್ಪವಾದರೂ ಶುದ್ಧಪಡಿಸಿಕೊಳ್ಳಬೇಕೆಂದು ಹಿಂದಿನವರು ಹೇಳಿದರು. ಹೀಗೆ, ಸ್ವಲ್ಪವೇ ಶುದ್ಧವಾದ ಮನಸ್ಸಿನ ಸ್ಥಿತಿ ವಿಕ್ಷಿಪ್ತವೆನ್ನಲ್ಪಡುವುದು. ಯಾವುದೋ ಒಂದು ವಿಕ್ಷೇಪವಿರುವುದರಿಂದ ಇದಕ್ಕೆ `ವಿಕ್ಷಿಪ್ತ'ವೆನ್ನುವ ಹೆಸರು ಬಂದಿದೆಯೆಂದೂ ಹೇಳಬಹುದು. ಇಂಥ ಮನಸ್ಸು, ಯಾವುದರ ಬಗ್ಗೆ ಚಿಂತಿಸಬೇಕೋ ಅದರ ಬಗ್ಗೆ ಚಿಂತನೆ ಮಾಡಲು ಯೋಗ್ಯವಾದುದು. ಆದರೆ, ಮಧ್ಯೆ ಮಧ್ಯೆ ಬೇರೆ ಯಾವುದಾದರೂ ವಿಚಾರ ಬರುತ್ತಿರುತ್ತದೆ. ನಾವು ಧ್ಯಾನ ಮಾಡುತ್ತಿರುತ್ತೇವೆ. ಆದರೆ ಮಧ್ಯೆ ಮಧ್ಯೆ ಆಫೀಸಿನ ವಿಷಯ ನೆನಪಿಗೆ ಬರುತ್ತದೆ. ಹಾಗೆ ನೆನಪಿಗೆ ಬಂದರೂ ಮತ್ತೆ ಮತ್ತೆ ಧ್ಯಾನ ಮಾಡುತ್ತೇವೆ. ಹೀಗೆ ಮಧ್ಯೆ ಮಧ್ಯೆ ಬೇರೆ ವಿಷಯಗಳು ಬರುತ್ತಿದ್ದರೂ ಈ ಸ್ಥಿತಿ ಕ್ಷಿಪ್ತ ಮತ್ತು ಮೂಢಕ್ಕಿಂತಲೂ ಮೇಲಾದುದು. ಹೀಗಿದ್ದರೂ ವಿಕ್ಷಿಪ್ತ ಸ್ಥಿತಿಯಲ್ಲಿರುವ ಮನಸ್ಸು ಪರವಸ್ತುವನ್ನು ಪಡೆಯಲು ಉಪಯೋಗವಾಗುವುದಿಲ್ಲ. ಅದಕ್ಕೆ ಕಾರಣ, ಹೀಗೆ ಧ್ಯಾನಿಸಲ್ಪಡುವ ವಿಷಯಕ್ಕಿಂತಲೂ ಬೇರೆ ವಿಷಯಗಳಲ್ಲಿ ವಿಕ್ಷಿಪ್ತತೆಯುಳ್ಳ ಮನಸ್ಸು ತನ್ನ ಕೆಲಸವನ್ನು ಸಾಧಿಸಲು ಸಾಮರ್ಥ್ಯವಿಲ್ಲದುದೇ ಆಗುತ್ತದೆ. ಇಂಥ ಮನಸ್ಸು ಒಂದು ದಿನ ಧ್ಯಾನ ಮಾಡುತ್ತದೆ, ಮತ್ತೊಂದು ದಿನ ಧ್ಯಾನ ಮಾಡುವುದನ್ನು ಬಿಟ್ಟು ಬೇರೆ ಏನಾದರೂ ಮಾಡುವುದು. ಎಂಥ ಮನಸ್ಸು ಪರವಸ್ತುವನ್ನು ಪಡೆಯಲು ಉಪಯೋಗವಾಗುತ್ತದೆ ಎನ್ನುವುದಕ್ಕೆ ಉತ್ತರವಾಗಿ ಏಕಾಗ್ರ ಸ್ಥಿತಿಯಲ್ಲಿರುವ ಮನಸ್ಸು ಉಪಯೋಗಪಡುತ್ತದೆ ಎನ್ನುತ್ತಾರೆ-ಕೈವಲ್ಯೋಪನಿಷತ್ತಿನಲ್ಲಿ,

\begin{shloka}
`ಉಮಾಸಹಾಯಂ ಪರಮೇಶ್ವರಂ ಪ್ರಭುಂ ತ್ರಿಲೋಚನಂ ನೀಲಕಂಠಂ ಪ್ರಶಾಂತಮ್|\\
ಧ್ಯಾತ್ವಾ ಮುನಿರ್ಗಚ್ಛತಿ ಭೂತಯೋನಿಂ ಸಮಸ್ತಸಾಕ್ಷಿಣಂ ತಮಸಃ ಪರಸ್ತಾತ್||
\end{shloka}

-ಎಂದಿದೆ. ದೇವಿ (`ಲಲಿತಾಸಹಸ್ರನಾಮ'ದಲ್ಲಿ) `ಅಂತರ್ಮುಖ ಸಮಾರಾಧ್ಯಾ' ಮತ್ತು `ಬಹಿರ್ಮುಖ ಸುದುರ್ಲಭಾ' ಎಂದು ಸ್ತುತಿಸಲ್ಪಟ್ಟಿದ್ದಾಳೆ. ಅದೇ ರೀತಿ ಭಗವಂತನನ್ನು ಅಂತರ್ಮುಖ ಮನಸ್ಸಿನಿಂದ ಧ್ಯಾನ ಮಾಡಬೇಕು.

ಈ ಸಂದರ್ಭದಲ್ಲಿ ನನಗೆ ನನ್ನ ಆಚಾರ್ಯರ (ಶ್ರೀ ಶ್ರೀಚಂದ್ರಶೇಖರ ಭಾರತೀ ಮಹಾಸ್ವಾಮಿಗಳವರ) ಬಗ್ಗೆ ನೆನಪು ಬರುತ್ತದೆ. ಅವರು ಸಂನ್ಯಾಸವನ್ನು ತೆಗೆದುಕೊಳ್ಳುವುದಕ್ಕೆ ಮುಂಚೆ,

\begin{shloka}
`ಪೂರ್ವಾಭ್ಯಾಸೇನ ತೇನೈವ ಹ್ರಿಯತೇ ಹ್ಯವಶೋಽಪಿ ಸಃ'
\end{shloka}

(ತನ್ನ ಹಿಂದಿನ ಜನ್ಮದ ಅಭ್ಯಾಸದಿಂದ ತಾನು ಇಷ್ಟಪಡದಿದ್ದರೂ ಒಳ್ಳೆಯ ದಾರಿಯಲ್ಲಿ ಎಳೆಯಲ್ಪಡುತ್ತಾನೆ) ಎಂದು ಭಗವಂತನು ಗೀತೆಯಲ್ಲಿ ಹೇಳುವಂತೆ ಅವರು ಪಾಠಶಾಲೆಯಲ್ಲಿ ಓದುವಾಗ ಅವರದೊಂದು ಸ್ವಭಾವವಿತ್ತು. ಪ್ರತಿ ಪ್ರದೋಷದಲ್ಲೂ ವಿದ್ಯಾಶಂಕರ ದೇವಾಲಯದಲ್ಲಿರುವ ಒಂದು ಕಂಬದ ಹತ್ತಿರ ಕುಳಿತು. `ರತ್ನೈಃ ಕಲ್ಪಿತಮಾಸನಂ' (ಭಗವಂತನಿಗೆ ರತ್ನಗಳಿಂದ ಕಲ್ಪಿತವಾದ ಆಸನವನ್ನು ಸಮರ್ಪಿಸುವೆನು) ಎಂದು ಹೇಗೆ `ಶಿವಮಾನಸಪೂಜಾ ಸ್ತೋತ್ರ' ದಲ್ಲಿ ಹೇಳಿದೆಯೋ ಹಾಗೆಯೇ ಭಗವಂತನಿಗೆ ಆಸನ, ಅರ್ಘ್ಯ, ಪಾದ್ಯ ಮುಂತಾದವನ್ನು ಸಮರ್ಪಣೆ ಮಾಡಿ ನೈವೇದ್ಯ, ನೀರಾಜನ ಮಾಡಿ, ಮಂತ್ರ ಪುಷ್ಪ ಸಮರ್ಪಿಸಿ ಒಂದೇ ವಿಧವಾದ ಚಿತ್ತದಿಂದ ಮಾನಸಿಕವಾಗಿ ಪೂಜೆ ಮಾಡುವ ಪದ್ಧತಿ ಅವರಿಗಿತ್ತು. ಪೂಜೆ ಮುಗಿಯುವವರೆಗೆ ಅದರಲ್ಲೇ ಅವರ ಮನಸ್ಸು ತೊಡಗಿರುತ್ತಿತ್ತು. ಆ ವಿಷಯವನ್ನು ಕುರಿತು ಅವರು ನನಗೆ ಹೇಳಿದ್ದರು. ಇದು ಮನಸ್ಸಿನ ಏಕಾಗ್ರತೆಗೆ ನಿದರ್ಶನ. ಮೇಲೆ ಹೇಳಿದ ಕೈವಲ್ಯೋಪನಿಷತ್ತಿನಲ್ಲಿ ಹೇಳಲ್ಪಟ್ಟ `ಉಮಾಸಹಾಯಂ' ಎನ್ನುವುದನ್ನು ಕುರಿತು ಚಿಂತಿಸೋಣ. ನಾವು ಸಗುಣನಾದ ಭಗವಂತನನ್ನು ಚಿಂತಿಸುವಾಗ `ಉಮಾಸಹಾಯಂ ಪರಮೇಶ್ವರಂ' ಎಂದು ಹೇಳಿದ ಪ್ರಕಾರ ಶಕ್ತಿಯೊಡನೆ ಪರಮೇಶ್ವರನನ್ನು ಚಿಂತಿಸುವುದು ವಿಶೇಷ. ಸಕಲ ಜಗತ್ತಿಗೆ ನಾಯಕನಾಗಿರುವುದರಿಂದ `ಪ್ರಭು' ಎನ್ನುತ್ತಾರೆ. `ತ್ರಿಲೋಚನಂ' - ಭಗವಂತನಿಗೆ ಮೂರು ಕಣ್ಣುಗಳು. ಧ್ಯಾನವೇ ಮೂರನೆಯ ಕಣ್ಣು. ಅದನ್ನು ತಿಳಿಸುವುದಕ್ಕಾಗಿಯೇ ಭಗವಂತನು ಮೂರು ಕಣ್ಣುಗಳುಳ್ಳವನೆಂದು ನಾವು ಹೇಳುತ್ತೇವೆ. `ನೀಲಕಂಠಂ' - ಲೋಕಕ್ಕೆ ಒಂದು ದೊಡ್ಡ ಆಪತ್ತು ಉಂಟಾದಾಗ ಭಗವಂತನು ನಮ್ಮನ್ನು ಕಾಪಾಡಿದನು. `ನಗಿಲಿತಂ ನೋದ್ಗೀರ್ಣಮೇವ ತ್ವಯಾ' (ಆ ವಿಷವನ್ನು ಬಾಯಿಂದ ಹೊರಗೂ ಬಿಡಲಿಲ್ಲ, ನುಂಗಲೂ ಇಲ್ಲ) ಎಂದು ಹೇಳಿ - 

\begin{shloka}
`ಜ್ವಾಲೋಗ್ರಃ ಸಕಲಾಮರಾತಿಭಯದಃ ಕ್ಷ್ವೇಲಃ ಕಥಂ ವಾ ತ್ವಯಾ\\
ದೃಷ್ಟಃ ಕಿಂ ಚ ಕರೇ ಧೃತಃ ಕರತಲೇ ಕಿಂ ಪಕ್ವಜಂಬೂಫಲಮ್|\\
ಜಿಹ್ವಾಯಾಂ ನಿಹಿತಶ್ಚ ಸಿದ್ಧಘುಟಿಕಾ ವಾ ಕಂಠದೇಶೇ ಭೃತಃ\\
ಕಿಂ ತೇ ನೀಲಮಣಿರ್ವಿಭೂಷಣಮಯಂ ಶಂಭೋ ಮಹಾತ್ಮನ್ ವದ||'
\end{shloka}

(ಆ ಜ್ವಲಿಸುವ, ಎಲ್ಲ ದೇವತೆಗಳಿಗೂ ಭಯವನ್ನುಂಟುಮಾಡುವ ವಿಷ ನಿನ್ನಿಂದ ಹೇಗೆ ನೋಡಲಾಯಿತು? ನೇರಳೆ ಹಣ್ಣಿನಂತೆ ಇದನ್ನು ನೀನು ಕೈಯಲ್ಲಿಟ್ಟುಕೊಂಡೆಯಾ? ಸಿದ್ಧ ಔಷಧಿಯಂತೆ ಬಗೆದು ನಾಲಿಗೆಯ ಮೇಲೆ ಈ ವಿಷವನ್ನು ಹಾಕಿಕೊಂಡೆಯಾ? ಎಲೈ ಶಂಬುವೇ! ಮಹಾತ್ಮನೇ! ವಿಷವನ್ನು ನೀಲಿ ಆಭರಣದಂತೆ ಕಂಠದಲ್ಲಿ ಧರಿಸಿದೆಯಾ? ಹೇಳು.) - ಈ ವಿಧವಾಗಿ ಭಗವತ್ಪಾದರು ನೀಲಕಂಠನನ್ನು ಸ್ತುತಿಸಿದರು. ಒಮ್ಮೆ ದೇವತೆಗಳೂ ರಾಕ್ಷಸರು ಸಮುದ್ರವನ್ನು ಕಡೆಯಲು ತೊಡಗಿದರು. ಒಂದು ದೊಡ್ಡ ವಿಚಾರಕ್ಕೆ ಕುಳಿತುಕೊಂಡರೆ ಮೊದಲು ಏನೋ ತೋರಿ ಕೊನೆಯಲ್ಲಿ ಸಿದ್ಧಾಂತ ತೀರ್ಮಾನವಾಗುವಂತೆ ಸಮುದ್ರವನ್ನು ಕಡೆಯುವಾಗ ಮೊದಲು ವಿಷ ಬಂದು ಬಿಟ್ಟಿತು. ಏನು ಮಾಡಬೇಕೆಂದು ಯಾರಿಗೂ ತೋರಲಿಲ್ಲ. ಆ ವಿಷದಿಂದ ದೊಡ್ಡ ಜ್ವಾಲೆ ಉಂಟಾಯಿತು. ಆಗ ಭಗವಂತನು ಬಂದು ಅದನ್ನು ತೆಗೆದುಕೊಂಡು ಬಾಯಿಯಲ್ಲಿ ಹಾಕಿಕೊಂಡುಬಿಟ್ಟನು. ಅದನ್ನು ನುಂಗದೆ ತನ್ನ ಗಂಟಲಿನಲ್ಲೇ ಇಟ್ಟುಕೊಂಡನು. ಪುರಾಣ ಹೇಳುವವರು `ಭಗವಂತನಲ್ಲಿ ಒಳಗೂ ಹೊರಗೂ ಬ್ರಹ್ಮಾಂಡವಿದೆ. ಆ ವಿಷವನ್ನು ಹೊರಗಿಟ್ಟರೆ ಹೊರಗಿನ ಬ್ರಹ್ಮಾಂಡ ಸುಟ್ಟು ಹೋಗುತ್ತದೆ, ಒಳಗಿಟ್ಟರೆ ಒಳಗಿರುವ ಬ್ರಹ್ಮಾಂಡ ಸುಟ್ಟು ಹೋಗುತ್ತದೆ. ಆದ್ದರಿಂದಲೇ ಗಂಟಲಿನಲ್ಲೇ ಇಟ್ಟುಕೊಂಡನು' ಎಂದು ಕಥೆ ಹೇಳುತ್ತಾರೆ.

`ನೀಲಕಂಠಂ ಪ್ರಶಾಂತಮ್' - ನೀಲಕಂಠನೆಂದು ನಾವು ನೆನೆಯುತ್ತಲೇ ಭಗವಂತನು ಎಲ್ಲಾ ಲೋಕಗಳನ್ನೂ ಕಾಪಾಡುವನೆಂದು ಜ್ಞಾನವಾಗುತ್ತದೆ.

`ಧ್ಯಾತ್ವಾ ಮುನಿರ್ಗಚ್ಚತಿ ಭೂತಯೋನಿಂ' - ಹಾಗೆ ಯಾರು ಚಿಂತನೆ ಮಾಡುವವರು, `ತಮಸಃ ಪರಸ್ತಾತ್ ಭೂತಯೋನಿಂ' ಎಂದು ಹೇಳಿದಂತೆ ಜ್ಯೋತಿರ್ಮಯವಾಗಿರುವ ಈ ಸಕಲ ಜಗತ್ತಿಗೂ ಕಾರಣನಾದ ಪರಮೇಶ್ವರನನ್ನು ಪಡೆಯುತ್ತಾರೆ.

`ಮಾಮುಪೇತ್ಯ ತು ಕೌಂತೇಯ ಪುನರ್ಜನ್ಮ ನ ವಿದ್ಯತೇ' ಎಂದು ಹೇಳಿದಂತೆ ಅವನನ್ನು ಪಡೆದ ಮೇಲೆ ಮತ್ತೆ ಜನ್ಮವಿಲ್ಲ. `ಧ್ಯಾತ್ವಾಮುನಿಃ' (ಮುನಿ ಧ್ಯಾನವನ್ನು ಮಾಡಿ)-ಧ್ಯಾನವೆಂದರೇನು?

ಕ್ಷಿಪ್ತ, ಮೂಢ, ವಿಕ್ಷಿಪ್ತ - ಈ ಮೂರು ಚಿತ್ತವನ್ನೂ ಇಟ್ಟುಕೊಂಡು ನಾವು ಧ್ಯಾನ ಮಾಡಲಾಗುವುದಿಲ್ಲವೆಂದು ನೋಡಿದೆವು. ಆದ್ದರಿಂದ ಧ್ಯಾನ ಮಾಡುವುದಕ್ಕೆ ಶಾಸ್ತ್ರದಲ್ಲಿ ಏಕಾಗ್ರವಾದ ಚಿತ್ತವಿರಬೇಕೆಂದು ಹೇಳಲ್ಪಟ್ಟಿದೆ. ಏಕಾಗ್ರವೆಂದರೆ - `ವಿಜಾತೀಯ ಪ್ರತ್ಯಯ ತಿರಸ್ಕೃತ ಸಜಾತೀಯ ಪ್ರತ್ಯಯ ಪ್ರವಾಹಃ ಧ್ಯಾನಮ್' ಎಂದು ನಾವು ಸಾಮಾನ್ಯವಾಗಿ ಹೇಳುತ್ತೇವೆ.

`ಪ್ರೋಷಿತಭರ್ತೃಕಾ ಪತಿಂ ಧ್ಯಾಯತೀವ' ಎಂದು ಒಬ್ಬರು ಹೇಳಿದರು. ಗಂಡ ಊರಿನಲ್ಲಿಲ್ಲ. ಹೆಂಡತಿಗೆ ಗಂಡನ ಮೇಲೆ ಪ್ರೀತಿ. ಅವಳು ತನ್ನ ಗಂಡನನ್ನು ಕುರಿತೇ ಚಿಂತಿಸುತ್ತಾಳೆ. ಹೇಗೆ ಚಿಂತಿಸುತ್ತಾಳೆ ಎಂದರೆ ನಾವು ಶಾಕುಂತಲ ನಾಟಕದಲ್ಲಿ ನೋಡಿದರೆ ಗೊತ್ತಾಗುತ್ತದೆ. ದುರ್ವಾಸರ ಶಾಪ ಏತಕ್ಕೆ ಶಕುಂತಲೆಗೆ ಬಂದಿತು? ತನ್ನ ಗಂಡನನ್ನು ಕುರಿತು ಯೋಚಿಸುತ್ತಿದ್ದ ಶಕುಂತಲೆ ದುರ್ವಾಸರನ್ನು ನೋಡಲಿಲ್ಲ. ಆಗ ದುರ್ವಾಸರು, `ನೀನು ಯಾರನ್ನು ಕುರಿತು ಯೋಚಿಸುತ್ತಿದ್ದು ಎದುರಿಗೆ ಬಂದ ಋಷಿಯನ್ನು ಗೌರವಿಸಲಿಲ್ಲವೋ, ಅವನು ನಿನ್ನನ್ನು ಮರೆತುಬಿಡಲಿ' ಎಂದು ಶಪಿಸಿದರು.

ಹೀಗೆ ಏಕಾಗ್ರಚಿತ್ತವಿದ್ದರೆ ಎದುರಿಗೆ ಯಾವುದಾದರೂ ವಸ್ತುವಿದ್ದರೂ ಅದು ಕಣ್ಣಿಗೆ ತೋರುವುದಿಲ್ಲ. ಅದಕ್ಕೆ ಶಾಸ್ತ್ರದಲ್ಲಿ `ವಿಜಾತೀಯ ಪ್ರತ್ಯಯ ತಿರಸ್ಕೃತ' ಎಂದು ಹೇಳುತ್ತಾರೆ. ಭಗವಂತನ ರೂಪದ ಜ್ಞಾನವಿರುವಾಗ ಬೇರೆ ಯಾವ ವಿಧವಾದ ಜ್ಞಾನವೂ ಇರುವುದಿಲ್ಲ. ಒಂದು ಚಿತ್ರವನ್ನು ನೋಡಿದರೆ, ಆ ಚಿತ್ರ ಮನಸ್ಸಿನಲ್ಲಿ ನಾಟಿಕೊಂಡಿರುವಾಗ ಅದು ಮಾತ್ರ ಮನಸ್ಸಿನಲ್ಲಿರುತ್ತದೆಯೇ ಹೊರತು ಬೇರೆ ವಿಚಾರವೇ ಇರುವುದಿಲ್ಲ. ಇದನ್ನೇ ಶಾಸ್ತ್ರದಲ್ಲಿ `ವಿಜಾತೀಯ ಪ್ರತ್ಯಯ ತಿರಸ್ಕೃತ ಸ್ವಜಾತೀಯ ಪ್ರತ್ಯಯ ಪ್ರವಾಹಃ' ಎಂದು ಹೇಳಿರುವುದು. ಇದು ಧ್ಯಾನವಾಗುತ್ತದೆ.

ಉಪಾಸನೆಯ ಉಚ್ಚಸ್ಥಿತಿ ಯೋಗವೆನ್ನಲ್ಪಡುತ್ತದೆ. ಈಗ ಯೋಗವನ್ನು ಕುರಿತು ನೋಡೋಣ. ಮಹರ್ಷಿ ಪತಂಜಲಿ ತಮ್ಮ ಯೋಗಸೂತ್ರದಲ್ಲಿ `ಯೋಗಶ್ಚಿತ್ತವೃತ್ತಿ ನಿರೋಧಃ' ಎಂದಿದ್ದಾರೆ. ಚಿತ್ತ ವೃತ್ತಿಯನ್ನು ನಿರೋಧಿಸುವುದು (ತಡೆಯುವುದು) ಯೋಗ. ಯೋಗಕ್ಕೆ ಸಮಾಧಿ ಎಂದು ಅರ್ಥ. ಸಮಾಧಿ-ಸಂಪ್ರಜ್ಞಾತ ಸಮಾಧಿ. ಅಸಂಪ್ರಜ್ಞಾತ ಸಮಾಧಿ ಎಂದು ಎರಡು ಬಗೆ. ಏಕಾಗ್ರ ಮನಃಸ್ಥಿತಿಗೆ ಸಂಬಂಧಪಟ್ಟಿದ್ದು ಸಂಪ್ರಜ್ಞಾತ ಸಮಾಧಿ. ನಿರುದ್ಧ ಸ್ಥಿತಿಗೆ ಸಂಬಂಧಪಟ್ಟಿದ್ದು ಅಸಂಪ್ರಜ್ಞಾತ ಸಮಾಧಿ. ಮೊದಲು ಸಂಪ್ರಜ್ಞಾತ ಸಮಾಧಿಯನ್ನು ಕುರಿತು ಚಿಂತಿಸೋಣ. ಯಾವುದಾದರೂ ಆಲಂಬನ (ಆಧಾರ)ವನ್ನು ಇಟ್ಟುಕೊಂಡು ಅದರ ಮೇಲೆ ಮನಸ್ಸನ್ನು ಸ್ಥಿರಪಡಿಸಿದರೆ ಅದು ಸಂಪ್ರಜ್ಞಾತ ಯೋಗವಾಗುತ್ತದೆ. ಹೀಗೆ ಏಕಾಗ್ರತೆಯಿಂದ ಕೂಡಿದ ಮನಸ್ಸಿನೊಡನೆ ಮಾಡುವ ಧ್ಯಾನದಲ್ಲಿ ಒಂದು ವಿಶೇಷವಿದೆ. ಇದನ್ನು ವಿವರಿಸುವುದಕ್ಕೆ ಮೊದಲು `ತ್ರಿಪುಟಿ ಭಾನಂ' ಎನ್ನುವುದನ್ನು ಕುರಿತು ಗಮನ ಸೆಳೆಯಬೇಕು. ಜ್ಞಾನಂ, ಜ್ಞೇಯಂ, ಜ್ಞಾತಾ ಎಂದು ಮೂರು ತೋರುವಿಕೆಗೆ `ತ್ರಿಪುಟಿ ಭಾನಂ' ಎಂದು ಹೆಸರು. ಜ್ಞಾನವೆನ್ನುವುದು ಅರಿವು, ಜ್ಞಾನದ ವಸ್ತು ಜ್ಞೇಯ, ಈ ಜ್ಞಾನಕ್ಕೆ ಆಧಾರವಾಗಿರುವವನು ಜ್ಞಾತಾ. ನಾವು ಮತ್ತೊಬ್ಬನೊಡನೆ ಮಾತನಾಡುವಾಗ `ನಾನು ಇದ್ದೇನೆ' ಎನ್ನುವ ತೀರ್ಮಾನ ನಮಗೆ ಯಾವಾಗಲೂ ಇರುತ್ತದೆ. ಏಕೆಂದರೆ `ನಾನು ಇದ್ದೇನೆ' ಎನ್ನುವ ಜ್ಞಾನ ಸೂಕ್ಷ್ಮವಾಗಿ ಯಾವಾಗಲೂ ಇದ್ದೇ ಇರುತ್ತದೆ. ಅದರಿಂದ ಆತ್ಮನು ತಾನೇ ಪ್ರಕಾಶಿಸುತ್ತಾನೆಂದು ತಿಳಿಯಬಹುದು. ಜ್ಞಾನವಿರುವಾಗ ವಿಷಯ ತೋರುತ್ತದೆ. ಇಲ್ಲಿ ಜ್ಞೇಯವನ್ನು ವಿಷಯವಾಗುಳ್ಳದ್ದು ಜ್ಞಾನ. ಧ್ಯಾನ ಮಾಡುವಾಗ `ಧ್ಯಾನ ಮಾಡುತ್ತೇನೆ' ಎನ್ನುವ ಅರಿವು ಇದ್ದೇ ಇರುತ್ತದೆ, ಧ್ಯಾನ ಮಾಡುವಾಗ ಈ ತ್ರಿಪುಟಿ ಭಾನವಿದ್ದು ಧ್ಯಾನದ ಉಚ್ಚಸ್ಥಿತಿಯಲ್ಲಿ ಬರುವ ಈ ಏಕಾಗ್ರವೃತ್ತಿ ಸಂಪ್ರಜ್ಞಾತ ಸಮಾಧಿ ಎನ್ನುತ್ತೇವೆ. ಭಗವಂತನನ್ನು ಚಿಂತಿಸುವಾಗ `ನಾನು ಇದ್ದೇನೆ' ಎಂದೂ `ಭಗವಂತನನ್ನು ಚಿಂತಿಸುತ್ತಿದ್ದೇನೆ' ಎಂದೂ ಸೂಕ್ಷ್ಮವಾಗಿ ತೋರುತ್ತಿರುತ್ತದೆ. ಹೀಗೆ ಮಾಡುವ ಧ್ಯಾನದ ಉತ್ತಮವಾದ ಸ್ಥಿತಿಗೆ ಸಂಪ್ರಜ್ಞಾತ ಸಮಾಧಿಯೆಂದು ಹೆಸರು. ಏಕಾಗ್ರ ಮನಃಸ್ಥಿತಿಗೂ ಮೇಲಾದ ಸ್ಥಿತಿ ನಿರುದ್ಧ. ಈ ಸ್ಥಿತಿಯಲ್ಲಿ, `ನಾನು ಜ್ಞಾನವನ್ನು ಅನುಭವಿಸುತ್ತಿದ್ದೇನೆ' ಎಂದು ಈ ವಿಧವಾಗಿ ತ್ರಿಪುಟಿ ಭಾನವಿಲ್ಲದೆ ಕೇವಲ ವಸ್ತ್ವಾಕಾರ ವಿಷಯ ಜ್ಞಾನ (ಧ್ಯಾನ ಮಾಡುವ ವಸ್ತುವಿನ ಬಗ್ಗೆ ಇರುವ ಅರಿವು) ವಿದ್ದು ಅದು ಅಸಂಪ್ರಜ್ಞಾತ ಸಮಾಧಿ ಎನ್ನಲಾಗುವುದು. ನಾವು ಇದನ್ನು ಅನುಭವದಲ್ಲಿ ಕಾಣಬಹುದು. ಕೆಲವೊಮ್ಮೆ ಈ ಜ್ಞಾನ ಬಂದಿದೆಯೆಂದು ತೋರುತ್ತದೆ. ಕೆಲವೊಮ್ಮೆ ಜ್ಞಾನ ಬರುತ್ತದೆಯೆಂದು ತೋರುತ್ತದೆ. ಇದಕ್ಕೂ ಸುಷುಪ್ತಿಗೂ ಏನು ಭೇದವೆಂದು ಕೇಳಿದರೆ ಸುಷುಪ್ತಿಯಲ್ಲಿ ಅಜ್ಞಾನವಿರುತ್ತದೆ, ವಿಷಯವಿಲ್ಲ. ಅಸಂಪ್ರಜ್ಞಾತ ಸಮಾಧಿಯಲ್ಲಿ ವಿಷಯ ತೋರುತ್ತಿರುತ್ತದೆ, ಆದರೆ ವಿಷಯ ತೋರುತ್ತಿದೆ ಎನ್ನುವುದು ಮಾತ್ರ ತೋರುವುದಿಲ್ಲ. ಅಂದರೆ ತರ್ಕದಲ್ಲಿ ಹೇಳುವ ಅನುವ್ಯವಸಾಯವೆನ್ನುವುದು ಇಲ್ಲ. ಹೀಗೆ ಸಮಾಧಿಯನ್ನು ಸಂಪ್ರಜ್ಞಾತ ಸಮಾಧಿ, ಅಸಂಪ್ರಜ್ಞಾತ ಸಮಾಧಿಯೆಂದು ಎರಡು ವಿಧವಾಗಿ ವಿಭಜಿಸಿ, ಏಕಾಗ್ರವಾದ ಮನಸ್ಸಿನಿಂದ ಚಿಂತನೆ ಮಾಡುವಾಗ ಸಂಪ್ರಜ್ಞಾತ ಸಮಾಧಿ ಎನ್ನುವ ನಿರುದ್ಧ ಸ್ಥಿತಿಯಲ್ಲಿ ಅಸಂಪ್ರಜ್ಞಾತ ಸಮಾಧಿಯನ್ನು ಹೇಳಿದರು. ಚಿತ್ತ ವೃತ್ತಿನಿರೋಧವೆಂದರೆ ಏನು? ಚಿತ್ತಕ್ಕೆ ಆತ್ಮನಾದಿಯಾಗಿ ಯಾವ ವಿಷಯವೂ ಇಲ್ಲದೆ ನಿರೋಧವೆನ್ನುವುದು ಇಲ್ಲ. ಚಿತ್ತಕ್ಕೆ ವಿಷಯವಿದೆ. ಆದರೆ ಅದು ಬೇರೆ ವಿಷಯಗಳಲ್ಲಿ ಚಲನವಿಲ್ಲದೆ ಒಂದೇ ವಿಷಯದಲ್ಲಿರುತ್ತದೆ ಎನ್ನುವುದನ್ನೇ `ಯೋಗಶ್ಚಿತ್ತವೃತ್ತಿ ನಿರೋಧಃ' ಎಂದು ಹೇಳಿದ್ದಾರೆ. ಈ ವಿಧವಾಗಿ ಮನಸ್ಸನ್ನು ವಶದಲ್ಲಿಟ್ಟುಕೊಂಡು ಈಶ್ವರನನ್ನು ಧ್ಯಾನಿಸುತ್ತೇವೆ.

ಜಪ ಮಾಡುವಾಗ ಮನಸ್ಸು ಏಕಾಗ್ರವಾಗಿರುವುದಲ್ಲದೆ ನಿರುದ್ಧವಾಗಿರುವುದಿಲ್ಲ. ಮಂತ್ರಜಪದಲ್ಲಿ ಮಂತ್ರದ ವಸ್ತುಜ್ಞಾನವೂ, ಶಬ್ದಜ್ಞಾನವೂ ಇರುತ್ತದೆ. ಪ್ರಣವ ಗಾಯತ್ರಿಯಂಥ ಮಂತ್ರಗಳನ್ನು ಮನಸ್ಸು ಉಚ್ಚರಿಸಿಕೊಂಡಿರುವುದಕ್ಕೆ `ಶಬ್ದಾನುವಿದ್ದಮ್' ಎಂದು ಹೆಸರು. ಕೆಲವೊಮ್ಮೆ ಭಗವಂತನ ರೂಪವನ್ನು ಚಿಂತಿಸುತ್ತೇವೆ. ಅದು `ರೂಪಾನುವಿದ್ಧಮ್' ಎನ್ನಲ್ಪಡುತ್ತದೆ. ಈ ಎರಡೂ ಇಲ್ಲದೆ ನಿರ್ಗುಣ ಬ್ರಹ್ಮವನ್ನು ಚಿಂತನೆ ಮಾಡಲು ಸಾಧ್ಯವೇ ಎನ್ನುವ ಪ್ರಶ್ನೆಗೆ, ಸಾಧ್ಯ ಎನ್ನುವುದೆ ಸಿದ್ಧಾಂತವಾಗುತ್ತದೆ. ನಿರ್ಗುಣ ಚಿಂತನೆಯಲ್ಲಿ ಜ್ಞಾನ ಬರುವುದೇ ಹೊರತು ಸಮಾಧಿ ಹೇಗೆ ಬರುತ್ತದೆಯೆಂದು ಕೇಳಿದರೆ, ಬರುತ್ತದೆಯೆಂದೇ ಉತ್ತರ ಹೇಳಿದ್ದಾರೆ, ಜ್ಞಾನ ಬೇರೆ, ಧ್ಯಾನ ಬೇರೆ. ಧ್ಯಾನವೆಂದರೆ ಪರಬ್ರಹ್ಮ ಸಾಕ್ಷಾತ್ಕಾರವಾಗಿ ಧ್ಯಾನಾತಿಶಯದಲ್ಲಿ ಒಂದು ಸ್ಫುರಣೆಯಾಗುತ್ತದೆ. ಆ ಸ್ಫುರಣೆಯಾದರೆ ಮತ್ತೆ ಧ್ಯಾನ ಮಾಡಬೇಕಾಗಿಲ್ಲ. ಆ `ಸ್ಫುರಣೆ' ಹಾಗೆಯೇ ಇದ್ದು ಬಿಡುತ್ತದೆ. ಧ್ಯಾನ ಮಾಡಬೇಕೆನ್ನುವ ತೀರ್ಮಾನವನ್ನು ಮನಸ್ಸಿನಲ್ಲಿಟ್ಟುಕೊಂಡು ಚಿಂತನೆ ಮಾಡುವುದು ಧ್ಯಾನವಾಗುತ್ತದೆ. ಹೀಗೆ ಮಾಡುವಾಗ ಮನಸ್ಸಿನಲ್ಲಿ ಪ್ರಕಾಶ ತೋರುವವರೆಗೆ ಪ್ರಯತ್ನ ಮಾಡಬೇಕು. ಪ್ರಕಾಶ ತೋರಿದ ಮೇಲೆ ವಸ್ತು ತಾನಾಗಿಯೇ ತೋರುತ್ತವೆ. ಹಾಗಿರುವುದು ಅಸಂಪ್ರಜ್ಞಾತ ಸಮಾಧಿ ಆಗುತ್ತದೆ. ನಾವು ಅಸಂಪ್ರಜ್ಞಾತ ಸಮಾಧಿಯಲ್ಲಿದ್ದುಕೊಂಡು ಬ್ರಹ್ಮ ಸಾಕ್ಷಾತ್ಕಾರವನ್ನು ಪಡೆಯಬೇಕು. ಈ ಸ್ಥಿತಿಯನ್ನು ಪಡೆದರೆ, 

\begin{shloka}
`ಯೋಗಾಭ್ಯಾಸವಶೀಕೃತೇನ ಮನಸಾ ತನ್ನಿರ್ಗುಣಂ ನಿಷ್ಕ್ರಿಯಮ್|\\
ಜ್ಯೋತಿಃ ಕಿಂಚನ ಯೋಗಿನೋ ಪಶ್ಯನ್ತಿ.......||'
\end{shloka}

(ಯೋಗಾಭ್ಯಾಸದಿಂದ ಮನಸ್ಸನ್ನು ವಶಪಡಿಸಿಕೊಂಡು ಯೋಗಿಗಳು ನಿರ್ಗುಣವೂ, ನಿಷ್ಕ್ರಿಯವೂ ಆದ ಜ್ಯೋತಿಯನ್ನು ಕಾಣುತ್ತಾರೆ) ಎಂದು ಹೇಳಿದಂತೆ ಬೇರೆ ಎಲ್ಲವನ್ನೂ ಬಿಟ್ಟುಬಿಡುತ್ತಾರೆ. ಹೀಗೆ ನಿರುದ್ಧವಾದ ಮನಸ್ಸಿನಿಂದ ಧ್ಯಾನಮಾಡಲು ಸಾಧ್ಯವಾಗದೇ ಹೋದರೆ ಮನಸ್ಸನ್ನು ಏಕಾಗ್ರವನ್ನಾಗಿ ಮಾಡಿಕೊಂಡು ಪರಮಾತ್ಮನನ್ನು ಚಿಂತಿಸಿದರೆ ಅದಕ್ಕೆ ಉಪಾಸನೆ ಎನ್ನುತ್ತಾರೆ.

ಈ ಉಪಾಸನೆ ಸಿದ್ಧಿಸಬೇಕಾದರೆ ಯಮ, ನಿಯಮಗಳು ಅವಶ್ಯಕ. ಈ ಯಮ-ನಿಯಮಗಳಿಲ್ಲದೆ `ನಾನು ಧ್ಯಾನ ಮಾಡುತ್ತಿದ್ದೇನೆ, ನಾನು ಯಾವಾಗಲೂ ನಿರುದ್ಧವಾದ ಚಿತ್ತದಲ್ಲಿ, ಸಮಾಧಿಯಲ್ಲಿದ್ದೇನೆ' ಎಂದು ಯಾರಾದರೂ ಹೇಳಿದರೆ ಅವುಗಳು ಕೇವಲ ಮಾತುಗಳೇ ಹೊರತು ನಿಜವಲ್ಲ. ನನ್ನಲ್ಲಿ ಕೋಟಿ ರೂಪಾಯಿಗಳಿವೆ ಎಂದು ಹೇಳಿಕೊಂಡು ಗುಡಿಸಿಲಿನಲ್ಲಿ ಕುಳಿತ್ತಿದ್ದರೆ ಅವನ ಬಳಿ ಕೋಟಿ ರೂಪಾಯಿಗಳಿವೆಯೆಂದು ಯಾರಾದರೂ ನಂಬುತ್ತಾರೆಯೇ? ಅದಕ್ಕೆ ತಕ್ಕಂತೆ ಯಾವುದಾದರು ಒಂದು ದೊಡ್ಡ ಮನೆಯಲ್ಲಿ ಇದ್ದರೆ ಮಾತ್ರ ನಂಬುತ್ತಾರೆ.

ಯಮ-ನಿಯಮ ಎಂದರೇನು?

\begin{shloka}
`ಅಹಿಂಸಾಸತ್ಯಾಸ್ತೇಯಬ್ರಹ್ಮಚರ್ಯಾಪರಿಗ್ರಹಾ ಯಮಾಃ|'
\end{shloka}

ಎಂದು ಹೇಳಲ್ಪಟ್ಟಿದೆ.

\begin{shloka}
`ಅಹಿಂಸಾಪ್ರತಿಷ್ಠಾಯಾಂ ತತ್ಸನ್ನಿಧೌ ವೈರತ್ಯಾಗಃ|'
\end{shloka}

(ಅಹಿಂಸೆಯಲ್ಲಿ ಪ್ರತಿಷ್ಠೆ ಪಡೆದವನ ಸನ್ನಿಧಿಯಲ್ಲಿ ವಿರೋಧಿತನವನ್ನು ಬಿಟ್ಟು ಬಿಡುತ್ತಾರೆ.) ಅಹಿಂಸಾ ಪ್ರತಿಷ್ಠೆ ಎನ್ನುವುದು ಬೇರೆ, ಅಹಿಂಸೆ ಎನ್ನುವುದು ಬೇರೆ. `ನನ್ನ ಬಾಳಿಗೆ ಬೇಕಾದ ಹಿಂಸೆಯಾದರೆ ಅದು ಮಾಡಬಹುದು. ಆದರೆ, ಅನಾವಶ್ಯಕವಾಗಿ ಹಿಂಸೆ ಮಾಡುವುದಿಲ್ಲ' ಎನ್ನುವ ಭಾವನೆ ಇದ್ದರೆ ಅದು ಅಹಿಂಸೆ. ಯಾವ ಕಾಲದಲ್ಲಾದರೂ, ಯಾವ ಜಾಗದಲ್ಲಾದರೂ ಯಾವುದೊಂದು ಪ್ರಾಣಿಗೂ ಹಿಂಸೆಯಾಗುವುದಿಲ್ಲ ಎನ್ನುವ ತೀರ್ಮಾನವನ್ನು ಮಾಡಿಕೊಂಡು ಅದರ ಪ್ರಕಾರ ನಡೆಯುತ್ತಾ ಬಂದರೆ ಅದು ಅಹಿಂಸಾ ಪ್ರತಿಷ್ಠೆಯಾಗುತ್ತದೆ. ನಾವು ಪುರಾಣಗಳಲ್ಲಿ ಅಹಿಂಸಾ ಪ್ರತಿಷ್ಠೆಯನ್ನು ಎಲ್ಲೆ ಮುಟ್ಟಿದ ಋಷಿಗಳಲ್ಲಿ ಆಶ್ರಮದಲ್ಲಿ ಪರಸ್ಪರ ವಿರೋಧಿಗಳಾಗಿದ್ದ ಪ್ರಾಣಿಗಳೂ ತಮ್ಮ ಹಿಂದಿನ ವಿರೋಧವನ್ನು ಬಿಟ್ಟು ಸ್ವಾಭಾವಿಕವಾಗಿ ಹೊಂದಿಕೊಂಡು ಬಂದಿವೆಯೆಂದು ನೋಡುತ್ತೇವೆ. ಹೀಗೆ ಅಹಿಂಸೆಯಲ್ಲಿ ಸ್ಥಿರವಾಗಿದ್ದುದರಿಂದ ಅವರ ಅಂತಃಕರಣದಲ್ಲಿ ಅಂಥ ಶಕ್ತಿ ಉಂಟಾಯಿತು. ನಮಗೆಲ್ಲ ಅಹಿಂಸಾ ಪ್ರತಿಷ್ಠೆ ಬಂದು ಬಿಟ್ಟಿದೆಯೆಂದು ಹೇಳಲಾಗುವುದಿಲ್ಲ, ಏಕೆಂದರೆ `ಯಾವುದಾದರೂ ಒಂದು ಕೆಲಸ ನಡೆಯುವುದಕ್ಕೆ ಚಿಕ್ಕ ಹಿಂಸೆ ಮಾಡಿದರೂ ಕೆಲಸ ಆದರೆ ಸಾಕು' ಎಂದೆನಿಸುತ್ತದೆ. ಅಹಿಂಸಾ ಪ್ರತಿಷ್ಠೆ ಬಂದರೆ ಮಾತ್ರ ಅಜ್ಞಾನಕ್ಕೆ ದೂರವಿರುವ ಪರಮೇಶ್ವರನನ್ನು ಪಡೆಯಬಹುದು. ಇದೇ ಮೊದಲು ನುಡಿದ `ಉಮಾ ಸಹಾಯಂ.........' ಎಂದು ಆರಂಭಿಸಿದ ಉಪನಿಷತ್ತಿನ ವಾಕ್ಯದಲ್ಲಿರುವ `ತಮಸಃ ಪರಸ್ತಾತ್' ಎನ್ನುವ ಮಾತುಗಳಿಂದ ಸೂಚಿತವಾಗಿದೆ. ಆದರೆ ಎಷ್ಟೋ ಮಂದಿ ಹಾಗೆ ಇಷ್ಟಪಡದೇ ಲೌಕಿಕವಾದ ಫಲಕ್ಕಾಗಿ ಧ್ಯಾನ ಮಾಡುತ್ತಾರೆ, ಅದಕ್ಕೆ ಅಹಿಂಸೆಯಾದರೂ ಇರಬೇಕು. ಅಹಿಂಸೆ ಇದ್ದರೆ ಮಾತ್ರ ಶಾಸ್ತ್ರ ವಿಹಿತವಾದ ಕರ್ಮಗಳನ್ನು ಮಾಡುತ್ತಾ ಭಗವಂತನನ್ನು ಚಿಂತಿಸಬಹುದು.

\begin{shloka}
`ಪಥ್ಯಮೌಷಧಸೇವಾ ಚ ಕ್ರಿಯತೇ ಯೇನ ರೋಗಿಣಾ|\\
ಆರೋಗ್ಯ ಸಿದ್ಧಿರ್ದೃಷ್ಟಾಸ್ಯ ...................................||'
\end{shloka}

ಔಷಧಿ ಕುಡಿದು ಪಥ್ಯವಿರುವುದಿಲ್ಲ ಎಂದರೆ ಔಷಧಿ ಸರಿಯಾಗಿ ಕೆಲಸ ಮಾಡುವುದಿಲ್ಲ. ಕೆಲವು ವೇಳೆ ಔಷಧಿ ಪೂರ್ತಿ ಕೆಲಸ ಮಾಡುವುದಿಲ್ಲ. ಅದೇ ರೀತಿ ಪರಮಾರ್ಥವನ್ನು ಪಡೆಯಲು ನಾವು ಮಾಡುವ ಪ್ರಯತ್ನಗಳು ಸಿದ್ಧಿಸಬೇಕಾದರೆ ಅಹಿಂಸೆಯನ್ನು ಅವಲಂಬಿಸಬೇಕು.

ಅನಂತರ `ಸತ್ಯ' ಎನ್ನುವುದು ಹೇಳಲಾಗಿದೆ. `ನಿಜ ಹೇಳುತ್ತೇನೆ' ಎಂದು ಎಲ್ಲರೂ ಹೇಳಬಹುದು. ಆದರೆ ಪೂರ್ತಿಯಾಗಿ ಸತ್ಯವಾಗಿರುವುದು ಸ್ವಲ್ಪ ಕಷ್ಟವೇ. ತಮಾಷೆಯಾಗಿ ಒಂದು ಕಥೆ ಹೇಳುವುದುಂಟು. `ನಾನು ಸುಳ್ಳು ಹೇಳುವುದಿಲ್ಲ' ಎನ್ನುವ ಒಬ್ಬನು ಸಾಕ್ಷಿ ಹೇಳುವುದಕ್ಕಾಗಿ ಹೋದನು. ನ್ಯಾಯಾಧೀಶನು ಅವನಿಗೆ `ಸತ್ಯವನ್ನು ಹೇಳಬೇಕು' ಎಂದು ಹೇಳಿದಾಗ ಅವನು, `ನನ್ನ ಕೈಯಲ್ಲಿ ಪ್ರಾಣವಿರುವವರೆಗೂ ನಾನು ಸತ್ಯವನ್ನೇ ಹೇಳುತ್ತೇನೆ' ಎಂದನು. `ಪ್ರಾಣ(ಜೀವ) ಇರುವವರೆಗೆ ಅವನು ಸತ್ಯವನ್ನೇ ಹೇಳುತ್ತಾನೆ' ಎಂದು ನ್ಯಾಯಾಧೀಶನು ಭಾವಿಸಿದನು. ಆದರೆ ಸಾಕ್ಷಿ ಹೇಳುವವನು ಹೇಳಿದ್ದೆಲ್ಲಾ ಸುಳ್ಳು. `ಜೀವ ಇರುವವರೆಗೆ ಸತ್ಯವನ್ನೇ ಹೇಳುತ್ತೇನೆಂದು ಹೇಳಿದೆಯಲ್ಲಾ!' ಎಂದು ಕೇಳಿದಾಗ ಅವನು, `ಹೌದು ಸ್ವಾಮಿ, ಅದಕ್ಕಾಗಿಯೇ ಕೈಯಲ್ಲಿ ಒಂದು ಹುಳು ಇಟ್ಟುಕೊಂಡಿದ್ದೆ. ಸುಳ್ಳು ಹೇಳಬೇಕು ಎಂದು ತೋರಿದಾಗ ಹುಳುವನ್ನು ಸಾಯಿಸಿಬಿಟ್ಟೇ ಇಷ್ಟು ಸುಳ್ಳನ್ನು ಹೇಳಿದೆ. ಕೈಯಲ್ಲಿ ಜೀವ ಇರುವವರೆಗೆ ಸುಳ್ಳು ಹೇಳಲಿಲ್ಲ' ಎಂದನು. ಇದು ಸತ್ಯವೇ? ಸತ್ಯವೇ ಅಲ್ಲ.

ಯಾವ ಸಮಯದಲ್ಲೂ ಬೇರೊಬ್ಬರಿಗೆ ಯಾವ ವಿಷಯವನ್ನೂ ಬೇರೆ ಅರ್ಥ ಬರುವಂತೆ ಹೇಳಬಾರದು. ಬೇರೆ ಅರ್ಥ ಬರುವಂತೆ ಒಂದು ಮಾತನ್ನು ಹೇಳಿ ತಾನು ಸರಿಯಾಗಿಯೇ ಹೇಳಿದಂತೆ ತೋರಿಸಿಕೊಂಡರೆ ಅದಕ್ಕೆ ಛಲವೆಂದು ಹೆಸರು. ಅದು ಇರಬಾರದು.

ಅನಂತರ `ಅಸ್ತೇಯ' ಎನ್ನುತ್ತಾರೆ. ಮನಸಾ, ವಾಚಾ, ಕರ್ಮಣಾ ಯಾರೂ ಕಳ್ಳತನ ಮಾಡಬಾರದು. ವಸ್ತುವನ್ನು ಕಳ್ಳತನ ಮಾಡಬೇಕೆಂದು ತೋರುತ್ತದೆಂದು ಮತ್ತೊಬ್ಬರೊಡನೆ ಹೇಳಿದರೆ ಅದು ಮಾತಿನಿಂದ (ವಾಚಾ) ಮಾಡಲ್ಪಟ್ಟ ಕಳ್ಳತನ. ಕಳ್ಳತನ ಮಾಡುವುದು ತಪ್ಪೆಂದು ಎಲ್ಲರಿಗೂ ಗೊತ್ತು. ಏಕೆಂದರೆ ಕಳ್ಳತನ ಮಾಡಿದುದು ತಿಳಿದರೆ ಪೋಲೀಸನು ಹಿಡಿದುಕೊಳ್ಳುತ್ತಾನೆ. ಆದರೆ ಮಾತಿನಿಂದಲೂ, ಮನಸ್ಸಿನಿಂದಲೂ ಮಾಡಬಹುದಾದ ಕಳ್ಳತನವನ್ನು ತಪ್ಪೆಂದು ಎಷ್ಟೋ ಜನ ಭಾವಿಸುವುದಿಲ್ಲ. ಆದರೆ, ಅದೂ ತಪ್ಪು. ಹೀಗೆ ಮನಸಾ, ವಾಚಾ, ಕರ್ಮಣಾ ಉಂಟಾಗುವ ಕಳ್ಳತನವಿಲ್ಲದೆ ಇದ್ದರೆ ಒಂದು ವಿಶೇಷವಾದ ಸಂಸ್ಕಾರ ಮತ್ತು ಶಕ್ತಿ ಇರುತ್ತದೆ. ನಾವು ಯಾರಾದರೂ ಹಾಗೆ ಕಳ್ಳತನ ಮಾಡುತ್ತೇವೆಯೇ ಎಂದರೆ, ಒಂದು ಕರ್ಚೀಫು ಎಲ್ಲಾದರೂ ಸಿಕ್ಕಿದರೆ ಯಾರೂ ಕಳ್ಳತನ ಮಾಡುವುದಿಲ್ಲ. ಆದರೆ ಒಂದು ತೋಡ ಒಂದು ಉಂಗುರ ಯಾರೂ ಇಲ್ಲದ ಸಮಯದಲ್ಲಿ ಸಿಕ್ಕಿದರೆ, ಯಾರೂ ಇಲ್ಲ, ತೆಗೆದುಕೊಳ್ಳೋಣ ಎನಿಸುತ್ತದೆ. ಕರ್ಚೀಫಿನ ವಿಷಯದಲ್ಲಿ ತೋರುವಂತೆಯೇ ಉಂಗುರದ ವಿಷಯದಲ್ಲಿಯೂ ತೋರಿದರೆ ಆವಾಗಲೇ ನಿಜವಾಗಿಯೂ ಅಸ್ತೇಯವಿದೆಯೆಂದು ಹೇಳಬಹುದು. ಅದನ್ನು ಬಿಟ್ಟು, ಸಾಮಾನ್ಯವಾದ ವಸ್ತುವಿನ ವಿಷಯದಲ್ಲಿ `ಇದು ಬೇಡ' ಎಂದು ತೋರಿ, ತಾನು ಸಂಪಾದಿಸುವುದಕ್ಕೆ ಆಗದೇ ಇರುವ ವಸ್ತುವಿನ ವಿಷಯದಲ್ಲಿ `ಇದು ಮಾತ್ರ ಪರವಾಗಿಲ್ಲ' ಎಂದು ತೋರಿದರೆ ಅದು ಅಸ್ತೇಯವಲ್ಲ. ಅಸ್ತೇಯವೆನ್ನುವುದು ಇದ್ದರೆ ಏನು ಫಲ ಎಂದರೆ-

\begin{shloka}
`ಅಸ್ತೇಯ ಪ್ರತಿಷ್ಠಾಯಾಂ ಸರ್ವರತ್ನೋಪಸ್ಥಾನಮ್|'
\end{shloka}

(ಅಸ್ತೇಯದಲ್ಲಿ ಪ್ರತಿಷ್ಠೆ ಉಂಟಾದರೆ ಎಲ್ಲಾ ರತ್ನಗಳ ಪ್ರಾಪ್ತಿಯೂ ಆಗುತ್ತದೆ.) ಎಂದು ಹೇಳಿದ್ದಾರೆ.

\begin{shloka}
`ಬ್ರಹ್ಮಚರ್ಯಪ್ರತಿಷ್ಠಾಯಾಂ ವೀರ್ಯಲಾಭಃ|'
\end{shloka}

ಬ್ರಹ್ಮಚರ್ಯ ಪ್ರತಿಷ್ಠೆ ಉಂಟಾದರೆ ಸಾಮರ್ಥ್ಯ ಉಂಟಾಗುತ್ತದೆ. ಬ್ರಹ್ಮಚರ್ಯವೆಂದರೆ ಇಂದ್ರಿಯ ನಿಗ್ರಹ. ಅದು ಶಾಸ್ತ್ರೀಯವಾದುದು. ಗೃಹಸ್ಥನು ಬ್ರಹ್ಮಚಾರಿಯೇ ಅಲ್ಲವೇ ಎನ್ನುವ ಪ್ರಶ್ನೆ ಇದೆ. ಒಂದು ಜಾಗದಲ್ಲಿ ಆಪಸ್ತಂಬರು, ಗೃಹಸ್ಥನು ಪುತ್ರನನ್ನು ಪಡೆಯುವುದು ಧರ್ಮ. ಅನಂತರ ಅವನಿಗೆ ವಿಷಯ ಭೋಗ ಬೇಕೆ, ಬೇಡವೆ ಎಂದು ಕೇಳಿದರೆ `ಅನ್ಯೇತು ಕಾಮಜಾಃ ಪ್ರೋಕ್ತಾಃ' ಎಂದು ಹೇಳಿದರು. ಮೊದಲು ಅವನು ಪುತ್ರನನ್ನು ಪಡೆಯುವುದು ನಿಯಮ. ಅನಂತರ ಅವನಿಗೆ ಪುತ್ರಪ್ರಾಪ್ತಿಯಾದರೆ ಅದು ಅವನ ಇಷ್ಟದಿಂದಲೇ ಆದದ್ದು. ಧರ್ಮದಿಂದ ಮಾತ್ರ ಬಂದುದು ಎನ್ನುವುದಿಲ್ಲ. ಆದ್ದರಿಂದ ಗೃಹಸ್ಥನು ಈ ರೀತಿ ನಡೆದುಕೊಂಡರೆ ಅವನೂ ಬ್ರಹ್ಮಚಾರಿಯೇ ಎಂದು ಹೇಳಿದರು.

ಕೊನೆಯಲ್ಲಿ ಅಪರಿಗ್ರಹ ಎನ್ನುವುದನ್ನು ಹೇಳಿದರು. ಮತ್ತೊಬ್ಬರ ವಸ್ತುಗಳಿಗೆ ಆಸೆ ಪಡಬಾರದು, ಅಯಾಚಿತ ಭಟ್ಟಾಚಾರ್ಯರೆನ್ನುವವರು ಒಬ್ಬರಿದ್ದರು. ಅವರು ಯಾಚನೆ ಮಾಡುತ್ತಿರಲಿಲ್ಲ. ಆದರೆ ಯಾರು ಏನು ಕೊಟ್ಟರೂ ತೆಗೆದುಕೊಳ್ಳುತ್ತಿದ್ದರು. ಅದಕ್ಕೆ ಒಬ್ಬರು, `ಇದು ಪ್ರಯೋಜನವಿಲ್ಲ. ನೀನು ಅಪರಿಗ್ರಹ ಭಟ್ಟಾಚಾರ್ಯನಾಗಬೇಕು' ಎಂದರು. ನಮಗೆ ಯಾರು ಏನು ಕೊಟ್ಟರೂ ತೆಗೆದುಕೊಳ್ಳಬೇಕೆಂಬ ಆಸೆ ಮನಸ್ಸಿನಲ್ಲಿದೆ. ನಾವಾಗಿಯೇ ಕೇಳಿದರೆ ಇತರರು ಕೊಡುತ್ತಾರೆಯೋ ಇಲ್ಲವೋ ಎನ್ನುವ ಸಂಕೋಚ. ಹಾಗಿರಬಾರದು. ಕೊಟ್ಟರೂ ಬೇಡ ಎಂದಿರಬೇಕು. ಸಂನ್ಯಾಸಿಗಳಾಗಿರುವವರಿಗೆ ಅಪರಿಗ್ರಹ ವ್ರತವಾಗುತ್ತದೆ. ಸಂನ್ಯಾಸಿ ತನ್ನ ಬದುಕಿಗೆ ಬೇಕಾದ ಕಮಂಡಲು, ಕಾಷಾಯ ವಸ್ತ್ರ, ಆಸನಗಳಂಥ ವಸ್ತುಗಳನ್ನು ಇಟ್ಟುಕೊಳ್ಳಬಹುದು. ಗೃಹಸ್ಥನೂ ತನ್ನ ಜೀವನಕ್ಕೆ ಬೇಕಾದವುಗಳನ್ನು ಮಾತ್ರ ಸಂಪಾದಿಸಿಕೊಂಡರೆ ಅದು ನಿಜವಾದ ಅಪರಿಗ್ರಹವಾಗುತ್ತದೆ.

ಕೆಲವರಿಗೆ ಪ್ರಪಂಚದಲ್ಲಿರುವ ಎಲ್ಲಾ ವಸ್ತುಗಳನ್ನು ಸಂಪಾದಿಸಿದರೂ ಕೂಡ ಎಲ್ಲಿಯಾದರೂ ನನ್ನ ಹತ್ತಿರ ಇಲ್ಲದ ವಸ್ತುಗಳಿರಬಹುದು, ಅವು ನನಗೆ ದೊರೆಯಲಿಲ್ಲವಲ್ಲಾ ಎನ್ನುವ ವ್ಯಥೆ. ಇದಕ್ಕೆ ಸಂಬಂಧಪಟ್ಟಂತೆ ಒಂದು ಕಥೆ ಇದೆ. ಶುಂಭ, ನಿಶುಂಭರಿಗೆ `ನಾವು ದೊಡ್ಡ ಅಸುರರು `ಪೃಥಿವ್ಯಾಂ ಯಾನಿ ರತ್ನಾನಿ' (ಎಂದು ಹೇಳಿದಂತೆ) ಯಾವ ಯಾವ ರತ್ನಗಳಿವೆಯೋ ಅವುಗಳೆಲ್ಲ ನಮಗೆ ಬೇಕು' ಎನ್ನುವ ಅಭಿಪ್ರಾಯವಿತ್ತು. `ಯತೋ ರತ್ನ ಭುಜೋ ವಯಂ'-ನಾವು ಎಲ್ಲ ರತ್ನಗಳನ್ನು ಅನುಭವಿಸುವವರು. `ನೀನು ಸ್ತ್ರೀ ರತ್ನವಾಗಿದ್ದೀಯೆ. ಆದ್ದರಿಂದ ನೀನು ನಮ್ಮ ಹತ್ತಿರ ಬಂದು ಬಿಡು' ಎಂದು ಅವರು ದೇವಿಗೆ ಹೇಳಿಕಳುಹಿಸಿದರು. ದೇವಿ `ನೀವು ಹೇಳುವುದು ನ್ಯಾಯವೇ. ರತ್ನಗಳನ್ನು ಅನುಭವಿಸುವವರು ಎನ್ನುವುದು ಸರಿ. ನಾನು ರತ್ನವೆನ್ನುವುದೂ ಸರಿ. ನಾನು ನಿಮ್ಮ ಹತ್ತಿರ ಬರಬೇಕೆಂದೂ ನೀವು ಆಸೆ ಪಡುವುದೂ ಸರಿ. ಆದರೆ, ನಾನು ಪ್ರತಿಜ್ಞೆಯೊಂದು ಮಾಡಿದ್ದೇನೆ. `ಯೋ ಮೇ ದರ್ಪಂ ವ್ಯಪೋಹತಿ'-ನನಗೆ ಸ್ವಲ್ಪ ದರ್ಪ (ಗರ್ವ)ವಿದೆ. ಆ ದರ್ಪದಿಂದ `ಯೋ ಮೇ ಪ್ರತಿಬಲೋ ಲೋಕೇ' ಎನ್ನುವಂತೆ ನನಗೆ ಎದುರಾಗಿ ನಿಂತು ಯಾರು ನನ್ನನ್ನು ಯುದ್ಧದಲ್ಲಿ ಜಯಿಸುತ್ತಾರೋ ಅವರನ್ನು ಮದುವೆಯಾಗುವುದೆಂದು ಪ್ರತಿಜ್ಞೆ ಮಾಡಿದ್ದೇನೆ. ಆದ್ದರಿಂದ ಶುಂಭನೋ ನಿಶುಂಭನೋ ಯಾರಾದರೂ ಸರಿಯೆ ನನ್ನೊಡನೆ ಹೋರಾಡಲು ಬರಬಹುದು' ಎಂದಳು. ಈ ಕಥೆ ಏತಕ್ಕೆಂದರೆ `ಯತೋ ರತ್ನ ಭುಜೋ ವಯಂ' ಎನ್ನುವ ಭ್ರಾಂತಿಯನ್ನಿಟ್ಟುಕೊಂಡು ಸ್ತ್ರೀ ರತ್ನ ಭೂತಾ ಚಾರ್ವಂಗೀ-ಸ್ತ್ರೀ ರತ್ನವಾಗಿ ಪ್ರಕಾಶಿಸುವವಳೂ, ಸುಂದರವಾದ ಅಂಗಗಳುಳ್ಳವಳಾಗಿಯೂ ಇರುವ, ಜಗನ್ಮಾತೆಯನ್ನು ಶುಂಭ-ನಿಶುಂಭರು ಅಪೇಕ್ಷಿಸಿದರು. ಈ ಬುದ್ಧಿ ಇದ್ದರೆ ಇದು ಅಪರಿಗ್ರಹವಾಗುತ್ತದೆಯೇ?

ನಮ್ಮ ಜೀವನಕ್ಕೆ ಏನು ಬೇಕೋ ಅದು ಬಿಟ್ಟು ಬೇರೆ ಯಾವುದನ್ನೂ ಸೇರಿಸಿಟ್ಟುಕೊಳ್ಳುವುದಿಲ್ಲ ಎಂದಿರಬೇಕು. ಇದುವರೆಗೆ `ಯಮ' ವೆನ್ನುವುದು ಯಾವುದೆಂದು ನೋಡಿದೆವು. ಇದೇ ರೀತಿ `ನಿಯಮ' ಯಾವುದು ಎನ್ನುವುದಕ್ಕೆ `ಶೌಚಃ ಸಂತೋಷ ತಪಃ ಸ್ವಾಧ್ಯಾಯೇಶ್ವರಪ್ರಣಿಧಾನಾನಿ ನಿಯಮಾಃ' ಎಂದು ಹೇಳಿದರು. ಶೌಚವೆನ್ನುವುದು ಒಂದು ನಿಯಮ. `ಮೃಜ್ಜಲಾಭ್ಯಾಂ ಬಹಿಃ ಶೌಚಂ' ಮಣ್ಣಿನಿಂದಲೂ ನೀರಿನಿಂದಲೂ ಹೊರಗೆ ಇರುವಿಕೆ ಶೌಚ. ಮಣ್ಣಿನಲ್ಲಿ ಒಂದು ವಿಶೇಷವಾದ ಗುಣವಿದೆ. ಯಾವುದಾದರೂ ಅಶುದ್ಧಿಯಾದರೆ, ಆ ಮಣ್ಣನ್ನು ಮುಟ್ಟಿಸಿ ನೀರನ್ನು ಹಾಕಿದರೆ ಆ ಅಶುದ್ಧಿ ಹೊರಟು ಹೋಗುತ್ತದೆ. ಮರಳು ಸೇರಿರುವುದಾಗಲಿ, ಜೇಡಿ ಮಣ್ಣು ಸೇರಿರುವುದಾಗಲಿ,-ಹೀಗೆಲ್ಲಾ ಉಪಯೋಗ ಮಾಡಕೂಡದು. ಇದೆಲ್ಲವನ್ನು ಶೌಚವನ್ನು ಕುರಿತು ಹೇಳುವಾಗ ಹೇಳಿದ್ದಾರೆ.

ಅನಂತರ `ಸಂತೋಷ'ವನ್ನು ಕುರಿತು ಹೇಳಿದರು. ಅಶುದ್ಧವಾದ ಮನಸ್ಸುಳ್ಳವನು ಐದು ನಿಮಿಷಗಳು ಕೂಡ ಏನೂ ಮಾಡದೆ ಕುಳಿತುಕೊಳ್ಳಲಾರನು. ಅವನ ಮನಸ್ಸು ಎಲ್ಲೆಲ್ಲೋ ತಿರುಗಾಡುತ್ತಿರುತ್ತದೆ. ಆದ್ದರಿಂದ `ತೃಪ್ತಿ' ಎನ್ನುವುದು ಬೇಕು. `ಅಸಂತುಷ್ಟೋ ದ್ವಿಜೋ ನಷ್ಟಃ' (ತೃಪ್ತಿ ಇಲ್ಲದ ಬ್ರಾಹ್ಮಣನು ನಾಶ ಹೊಂದುತ್ತಾನೆ). ಬ್ರಾಹ್ಮಣನಾದವನಿಗೆ ಸಂತೋಷವೇ ಉತ್ತಮವಾದ ಐಶ್ವರ್ಯ. ಇರುವದನ್ನು ಇಟ್ಟುಕೊಂಡು ಸಂತೋಷವಾಗಿದ್ದರೆ ಮನಸ್ಸು ಏಕಾಗ್ರವಾಗಿರುವುದು.

`ತಪಸ್ಸು' ಎಂದರೆ ವ್ರತಗಳು. ಅದೇ ರೀತಿ `ನಾನಶನಾತ್ ಪರಂ ತಪಃ' ಏಕಾದಶಿಯಂದು ನಾವು ಊಟಮಾಡದೆ ಇರುವುದು ದೊಡ್ಡ ತಪಸ್ಸು. ಸ್ವಾಧ್ಯಾಯ ಪ್ರವಚನ (ತಮ್ಮ ತಮ್ಮ ವೇದವನ್ನು ಅಧ್ಯಯನ ಮಾಡುವುದು), ತಪಸ್ಸು.

`ಈಶ್ವರಪ್ರಣಿಧಾನಾನಿ'-ಈಶ್ವರನು ನಾವು ಮನಸ್ಸಿನಲ್ಲಿ ಚಿಂತಿಸಬೇಕು. ಮೇಲೆ ಹೇಳಿದ ಯಮ-ನಿಯಮಗಳೊಡನೆ ಈಶ್ವರನನ್ನು ಆರಾಧಿಸಿದರೆ ಅದು ಬಹಳ ವಿಶೇಷ. ಅದನ್ನು ಧ್ಯಾನವೆನ್ನುತ್ತಾರೆ. ಹೀಗೆ ಧ್ಯಾನ ಮಾಡಿದರೆ ಅದು ಉಪಾಸನೆ ಎನ್ನಲ್ಪಡುತ್ತದೆ. ಯಮ-ನಿಯಮಗಳಿಲ್ಲದ ನನಗೆ ಷೋಡಶಾಕ್ಷರೀ, ಪಂಚಾಕ್ಷರೀ, ಬಾಲಾ, ಗಣಪತಿ ಇಂಥ ಮಂತ್ರಗಳು ಉಪದೇಶವಾಗಿವೆಯೆಂದು ಹೇಳಿಕೊಂಡು ಜಪ ಮಾಲೆಯನ್ನು ಕೈಯಲ್ಲಿ ಮಾತ್ರ ಇಟ್ಟುಕೊಂಡಿದ್ದರೆ, ಅದು ಕೂಡ ಅವಕಾಶವಿದ್ದಾಗ ಮಾತ್ರ ಮಾಡಿಕೊಂಡಿದ್ದರೆ, ಅದು ಉಪಾಸನೆ ಆಗುವುದಿಲ್ಲ. ಸರಿಯಾಗಿ ಉಪಾಸನೆ ಮಾಡಿದರೆ ಲೌಕಿಕ ಫಲವನ್ನು ಸಂಪಾದಿಸಬಹುದು. ಈ ವಿಷಯವನ್ನು ಭಗವಂತನು ಗೀತೆಯಲ್ಲಿ ಹೇಳಿದ್ದಾನೆ-

\begin{shloka}
`ಯೋಗೀ ಯುಂಜೀತ ಸತತಮಾತ್ಮಾನಂ ರಹಸಿ ಸ್ಥಿತಃ|\\
ಏಕಾಕೀ ಯುತಚಿತ್ತಾತ್ಮಾ ನಿರಾಶೀರಪರಿಗ್ರಹಃ||\\
ಶುಚೌ ದೇಶೇ ಪ್ರತಿಷ್ಠಾಪ್ಯ ಸ್ಥಿರಮಾಸನಮಾತ್ಮನಃ|\\
ನಾತ್ಯುಚ್ಛ್ರಿತಂ ನಾತಿನೀಚಂ ಚೈಲಾಜಿನಕುಶೋತ್ತರಮ್||\\
ತತ್ರೈಕಾಗ್ರ್ಯಂ ಮನಃ ಕೃತ್ವಾ ಯತ ಚಿತ್ತೇಂದ್ರಿಯಕ್ರಿಯಃ|\\
ಉಪವಿಶ್ಯಾಸನೇ ಯುಂಜ್ಯಾದ್ಯೋಗಮಾತ್ಮವಿಶುದ್ಧಯೇ||\\
ಸಮಂ ಕಾಯಶಿರೋಗ್ರೀವಂ ಧಾರಯನ್ನಚಲಂ ಸ್ಥಿರಃ|\\
ಸಂಪ್ರೇಕ್ಷ್ಯ ನಾಸಿಕಾಗ್ರಂ ಸ್ವಂ ದಿಶಶ್ಚಾನವಲೋಕಯನ್||\\
ಪ್ರಶಾಂತಾತ್ಮಾ ವಿಗತಭೀಃ ಬ್ರಹ್ಮಚಾರಿವ್ರತೇ ಸ್ಥಿತಃ|\\
ಮನಃ ಸಂಯಮ್ಯಮಚ್ಚಿತ್ತೋ ಯುಕ್ತ ಆಸೀತ ಮತ್ಪರಃ||
\end{shloka}

ಯೋಗೀ ಯುಂಜೀತ ಸತತಮಾತ್ಮಾನಂ ರಹಸಿ ಸ್ಥಿತಃ-`ಆತ್ಮಾನಂ' ಎಂದರೆ ಅಂತಃಕರಣ. `ಯುಂಜೀತ' ಎಂದರೆ ಏಕಾಗ್ರವಾಗಿ ಅಂತಃಕರಣವನ್ನು ಇಟ್ಟುಕೊಂಡು ಚಿಂತನೆ ಮಾಡುವುದು. ಕೆಲವರು ಸಭೆಯಲ್ಲಿ ಕುಳಿತು ಧ್ಯಾನ ಮಾಡುತ್ತಾರೆ. ಹಾಗೆ ಮಾಡಿದರೆ ಒಳ್ಳೆಯ ಯೋಗೀಶ್ವರರೆಂಬ ಹೆಸರು ಬರುತ್ತದೆ. ಹೀಗಿರಬಾರದು. `ಏಕಾಕಿ' ಒಬ್ಬಂಟಿಗನಾಗಿ, ಎನ್ನುತ್ತಾರೆ. ಭಗವಂತನ ಚಿಂತನೆ ಮಾಡಬೇಕಾದರೆ ಒಬ್ಬಂಟಿಗನಾಗಿ ಕುಳಿತುಕೊಳ್ಳಬೇಕು. ಇದನ್ನೇ `ರಹಸಿ ಸ್ಥಿತಃ' ಎನ್ನುತ್ತಾರೆ. ಚಿತ್ತವು ಸ್ವಾಧೀನವಾಗಿರಬೇಕು. ಶರೀರವೂ ಸ್ವಾಧೀನವಾಗಿರಬೇಕು. ಶರೀರ ಸ್ವಾಧೀನವಾಗಿದ್ದರೇನೇ ಚಿತ್ತ ಸ್ವಾಧೀನದಲ್ಲಿರುತ್ತದೆ. ಆದ್ದರಿಂದ `ಯತ ಚಿತ್ತಾತ್ಮಾ' ಎನ್ನುತ್ತಾರೆ. ಅದು ಬೇಕು, ಇದು ಬೇಕು ಎನ್ನುವ ಆಸೆಗಳಿಲ್ಲದೆ ಇರಬೇಕೆನ್ನುವುದಕ್ಕೆ `ನಿರಾಶೀರಪರಿಗ್ರಹಃ (ಆಸೆ ಇಲ್ಲದೆಯೂ, ವಸ್ತುಸಂಗ್ರಹ ಮಾಡದೆಯೂ) ಎನ್ನುತ್ತಾರೆ.

`ಶುಚೌ ದೇಶೇ ಪ್ರತಿಷ್ಠಾಪ್ಯ ಸ್ಥಿರಮಾಸನಮಾತ್ಮನಃ'-ಜಾಗ ಸರಿಯಾಗಿದ್ದರೇನೇ ಅದು ಧ್ಯಾನಕ್ಕೆ ಸರಿಯಾಗುತ್ತದೆ. ತರ್ಕಶಾಸ್ತ್ರದಲ್ಲಿ ಒಂದು ಕಡೆ ಶುಚಿಯಾಗಿರುವುದನ್ನು ಕುರಿತು ಹೇಳುವಾಗ, ಪೂರ್ವಪಕ್ಷವನ್ನು ಮನಸ್ಸಿನಲ್ಲಿಟ್ಟುಕೊಂಡು ತಪ್ಪಾದ ಯುಕ್ತಿಯನ್ನು ಮೊದಲು ತೋರಿಸಿದರು. `ನರಚಿರಃ ಕಪಾಲಂ ಶುಚಿ ಪ್ರಾಣ್ಯಂಗತ್ವಾತ್ ಶಂಖವತ್'-ಒಂದು ಪ್ರಾಣಿಯ ಅವಯವವಾದ ಶಂಖವನ್ನು ನಾವು ಪೂಜೆಗೆ ಉಪಯೋಗಿಸುತ್ತೇವೆ. ಏಕೆಂದರೆ ಅದನ್ನು ಶುಚಿಯೆಂದುಕೊಳ್ಳುತ್ತೇವೆ. ತಲೆ ಒಂದು ಓಡುವ ಪ್ರಾಣಿಯ (ಮನುಷ್ಯನ) ಅಂಗ ತಾನೇ! ಆದ್ದರಿಂದ ಅದನ್ನೂ ಶುಚಿಯೆಂದೇ ಇಟ್ಟುಕೊಳ್ಳಬಹುದು. ಹೀಗಿರುವಾಗ ಮೂಳೆಯನ್ನು ಮುಟ್ಟಿದರೆ ಏಕೆ ಸ್ನಾನ ಮಾಡಬೇಕು ಎನ್ನುವ ಪ್ರಶ್ನೆ ಕೆಲವರಿಗೆ. ಈ ಯುಕ್ತಿ ಸರಿಯಾಗಿಲ್ಲ ಎಂದು ತೋರಿಸುವದಕ್ಕೆ `ಶಾಸ್ತ್ರ ನಿಷಿದ್ಧಾತ್ವಾತ್' ಎಂದು ಉತ್ತರ ಹೇಳಿದರು. ಮನುಷ್ಯನ ಅಸ್ಥಿಯನ್ನು ಮುಟ್ಟಿದರೆ `ಸಚೇಲಸ್ನಾನ ಮಾಚರೇತ್' (ಉಟ್ಟಿರುವ ಬಟ್ಟೆಯೊಡನೆ ಸ್ನಾನ ಮಾಡಬೇಕು) ಎಂದು ಹೇಳಿದ್ದಾರೆ. ಸಾಮಾನ್ಯವಾದ ಮೂಳೆಯನ್ನು ಮುಟ್ಟಿದರೆ ಸ್ನಾನ ಮಾಡಿದರೆ ಸಾಕು. ಆದರೆ (ಸತ್ತು) ಕೆಲವು ದಿನಗಳು ಮಾತ್ರವೇ ಆಗಿದ್ದು ಮೂಳೆಯಲ್ಲಿ ಮಾಂಸದಂಥವು ಇರುವುದನ್ನು ಮುಟ್ಟಿದರೆ ಸಚೇಲ ಸ್ನಾನ ಮಾಡಬೇಕೆಂದು ಹೇಳಿದ್ದಾರೆ.

ಆದ್ದರಿಂದ ಶಾಸ್ತ್ರದಲ್ಲಿ ಹೇಳಿದಂತೆ ಇರುವುದೇ ನಿಜವಾದ ಶುಚಿ. ಇದು ಇರಬೇಕೆನ್ನುತ್ತಾರೆ. ದೇವಾಲಯಗಳು, ಪರ್ವತಗಳು, ನದೀತೀರ, ಮನೆಯಲ್ಲಿರುವ ಪ್ರತಿ ದಿನವೂ ಪೂಜೆ ಮಾಡುವ ಜಾಗ ಇವುಗಳೆಲ್ಲವನ್ನೂ ಶುಚಿ ಎನ್ನುತ್ತಾರೆ. ಈ ಕಾಲದಲ್ಲಿ `ಎಲ್ಲವನ್ನೂ ಸುಂದರವಾಗಿ ಶುದ್ಧಿ ಮಾಡಿಬಿಟ್ಟೆ. ಎಲ್ಲವೂ ಶುದ್ಧ' ಎಂದು ಹೇಳುತ್ತಾರೆ. ಹಾಗಲ್ಲ, ಯಾವ ಯಾವ ಜಾಗವನ್ನು ಶುದ್ಧವೆಂದು ಶಾಸ್ತ್ರದಲ್ಲಿ ಹೇಳಿದ್ದಾರೆ. ಎಂಜಲು ಅಶುದ್ಧ ಎನ್ನುತ್ತಾರೆ. ಏಕೆ ಅದು ಅಶುದ್ಧ ಎಂದರೆ ಅದು ಅಶುದ್ಧ ಅಷ್ಟು ಮಾತ್ರ ಹೇಳಬಹುದು. ಎಂಜಲಾದ ಬಾಯನ್ನು ನಾವು ಮುಟ್ಟುತ್ತೇವೆ. `ಅದು ಅಶುದ್ಧವಲ್ಲವೇ?' ಎಂದು ಕೇಳಿದರೆ ಎಂಜಲು ಒಳಗಡೆ ಇದ್ದರೆ ಶುದ್ಧ, ಹೊರಗಡೆ ಬಂದರೆ ಅಶುದ್ಧ, ಅಷ್ಟೇ. ಆದ್ದರಿಂದ ಶಾಸ್ತ್ರದಲ್ಲಿ ಹೇಳಲ್ಪಟ್ಟ ಶುಚಿಯಾದ ಜಾಗದಲ್ಲಿಯೇ ನಾವು ಕುಳಿತುಕೊಳ್ಳಬೇಕು. ಇದನ್ನು ಮನಸ್ಸಿನಲ್ಲಿಟ್ಟುಕೊಂಡೇ `ಶುಚೌದೇಶೇ ಪ್ರತಿಷ್ಠಾಪ್ಯ ಸ್ಥಿರಮಾಸನಮಾತ್ಮನಃ' ಎನ್ನುತ್ತಾರೆ. ನಮ್ಮ ಆಸನವನ್ನು ಹಾಕಿಕೊಂಡು ನಾವು ಕುಳಿತುಕೊಳ್ಳಬೇಕೆಂದು ಅದರಲ್ಲಿ ಬರುತ್ತದೆ. ಇದರ ತಾತ್ಪರ್ಯವೇನೆಂದರೆ, ಇನ್ನೊಬ್ಬರ ಆಸನದಲ್ಲಿ ನಾವು ಕುಳಿತುಕೊಳ್ಳಬಾರದು. ಈ ಕಾಲದಲ್ಲಿ ಎಲ್ಲರೂ ಮತ್ತೊಬ್ಬರ ಆಸನಗಳಲ್ಲಿ ಕುಳಿತುಕೊಳ್ಳಬೇಕಾಗುವುದು. ಮತ್ತೊಬ್ಬರು ಕುಳಿತ ಆಸನದಲ್ಲಿ ಕುಳಿತರೆ ಅದು ಅಶುಚಿ. ಅದೇ ರೀತಿ ನಮ್ಮ ಕಮಂಡಲುವನ್ನು ಮತ್ತೊಬ್ಬರು ಉಪಯೋಗಿಸಿದರೆ ಅದೂ ಅಶುಚಿ. ಆದ್ದರಿಂದಲೇ ಶಿಷ್ಯರು ತಮಗಾಗಿಯೇ ನೀರಿನ ಪಾತ್ರೆಯನ್ನು ಇಟ್ಟುಕೊಂಡಿರುತ್ತಾರೆ. ಅದನ್ನು ನೋಡಿ ಅವರಿಗೆ ಬೇರೆಯವರ ಮೇಲೆ ದ್ವೇಷವಿದೆಯೆಂದು ಹೇಳುವ ಹಾಗಿಲ್ಲ. ಆರೋಗ್ಯ ಶಾಸ್ತ್ರದಲ್ಲಿಯೂ ಇದಕ್ಕೆ ಕಾರಣಗಳಿವೆ. ಈಗ ಇವುಗಳನ್ನು ಕುರಿತು ನಾವು ಸಂಶೋಧನೆ ಮಾಡುವುದಿಲ್ಲ. ಏಕೆಂದರೆ ಮತ್ತೊಬ್ಬರ ಆಸನವನ್ನು ಉಪಯೋಗಿಸಿದರೂ ರೋಗ ಬರಲಿಲ್ಲವಲ್ಲಾ ಎಂದು ಕೆಲವರು ಕುತರ್ಕ ಮಾಡಲು ಅವಕಾಶವಿದೆ. `ಶಾಸ್ತ್ರದಲ್ಲಿ ಪವಿತ್ರತೆ ಇಲ್ಲ' ಎಂದು ಒಂದೇ ಮಾತಿನಲ್ಲಿ ಮುಗಿಸಿಬಿಡುತ್ತಾರೆ. ಈ ಕಾಲದಲ್ಲಿ ಯಾರದೋ ಬಟ್ಟೆಯನ್ನು ಯಾರೋ ಹಾಕಿಕೊಂಡು ಹೋಗುತ್ತಾರೆ. ಆ ಸಮಯದಲ್ಲಿ ಯಾವುದೋ ಒಂದು ಬಟ್ಟೆ ಸಿಕ್ಕಿದರೆ ಸಾಕು ಎನ್ನುವ ಭಾವನೆ! ಐವತ್ತು ವರ್ಷಗಳಿಗೆ ಹಿಂದೆ ಹೀಗಿರಲಿಲ್ಲ. ಆದರೆ ಬಹಳ ಯೋಗ್ಯರಾದ ದೊಡ್ಡವರು ಉಪಯೋಗಿಸಿದುದು ಸಿಕ್ಕಿದರೆ ಅದನ್ನು ಪವಿತ್ರವೆಂದು ಇಟ್ಟುಕೊಂಡು ವಿಶೇಷ ಕಾಲದಲ್ಲಿ ಮಾತ್ರ ಹಾಕಿಕೊಳ್ಳುತ್ತಾರೆ. ಇದು ಪರಂಪರೆಯಾಗಿ ಬಂದ ವಿಶೇಷ. ಅನಂತರ ಭಗವಂತನು `ನಾತ್ಯುಚ್ಛ್ರಿತಂ ನಾತಿ ನೀಚಂ ಚೈಲಾಜಿನ ಕುಶೋತ್ತರಮ್' ಎಂದನು. ಎಂಥ ಜಾಗ? ದೊಡ್ಡ ಮೇಜಿನ ಮೇಲೆ ಕುಳಿತು ಜಪ ಮಾಡಬಾರದು. ಭೂಮಿಯ ಮೇಲೂ ಕುಳಿತುಕೊಳ್ಳಬಾರದು. ಭೂಮಿಯ ಮೇಲೆ ಕುಳಿತುಕೊಂಡರೆ ಅದರಲ್ಲಿರುವ ಶೈತ್ಯಾದಿಗಳು ನಮಗೆ ಬಂದು ಬಿಡುತ್ತವೆ. ಬಹಳ ಎತ್ತರವಾದ ಜಾಗದಲ್ಲಿ ಕುಳಿತು ಸಮಾಧಿ ಸ್ಥಿತಿಗೆ ಬಂದವನು ಕೆಳಗೆ ಬಿದ್ದರೆ ಅಪಾಯವಾಗುತ್ತದೆ. ಆದ್ದರಿಂದ `ನಾತ್ಯುಚ್ಛ್ರಿತಂ ನಾತಿ ನೀಚಂ'-ಎತ್ತರವಾದ ಜಾಗವೂ ಅಲ್ಲ, ಕೆಳಗಿನ ಜಾಗವೂ ಅಲ್ಲ ಎನ್ನುತ್ತಾರೆ. ದರ್ಭೆಯ ಮೇಲೆ ಕೃಷ್ಣಾಜಿನ ಹಾಕಿಕೊಂಡು ಅದರ ಮೇಲೆ ವಸ್ತ್ರವನ್ನು ಹಾಕಿಕೊಂಡು ಕುಳಿತುಕೊಂಡರೆ ಒಂದು ವಿಧವಾದ ದೋಷವೂ ಇರುವುದಿಲ್ಲ. ಆದ್ದರಿಂದಲೇ `ಚೈಲಾಚಿನ ಕುಶೋತ್ತರಂ' ಎನ್ನುತ್ತಾರೆ. ಮನಸ್ಸು ಕೂಡ ಶಾಂತವಾಗಿರುತ್ತದೆ. ದರ್ಭೆ ಹಾಕಿಕೊಳ್ಳುವುದರಿಂದ ಶೈತ್ಯದೋಷ ಬರುವುದಿಲ್ಲ. ಕೃಷ್ಣಾಜಿನ ಸ್ವಲ್ಪ ರಕ್ಷೆ ಕೊಡುತ್ತದೆ. ವಸ್ತ್ರ ಹಾಕಿಕೊಂಡು ಅದರ ಮೇಲೆ ಕುಳಿತುಕೊಂಡರೆ ಇನ್ನೂ ವಿಶೇಷ. ಹೀಗೆ ಆಸನದ ಮೇಲೆ ಕುಳಿತು ಏನು ಮಾಡಬೇಕು ಎನ್ನುವುದಕ್ಕೆ ಭಗವಂತನು,

`ತತ್ರೈಕಾಗ್ರ್ಯಂ ಮನಃ ಕೃತ್ವಾ ಯತ ಚಿತ್ತೇಂದ್ರಿಯಕ್ರಿಯಃ' ಎಂದನು. ಇಂದ್ರಿಯಗಳು ಇಲ್ಲಿ-ಅಲ್ಲಿ ಅಲುಗಾಡಕೂಡದು. ಇಂದ್ರಿಯಗಳು, ಚಿತ್ತವೂ ನಮ್ಮ ವಶದಲ್ಲಿರಬೇಕು. ದೇಹದ ಕ್ರಿಯೆಯೂ ಹೀಗೆಯೇ ಇರಬೇಕು. ಅಂದರೆ ಶರೀರವನ್ನು ಯಾವಾಗಲೂ ಅಲ್ಲಾಡಿಸುತ್ತಾ ಇರಬಾರದು. ಶರೀರವನ್ನು ಅಲ್ಲಾಡಿಸುತ್ತಿದ್ದರೆ ಚಿತ್ತ ಒಂದೇ ಜಾಗದಲ್ಲಿ ನಿಲ್ಲುವುದಿಲ್ಲ. `ಸಮಂ ಕಾಯಶಿರೋಗ್ರೀವಂ ಧಾರಯನ್ನಚಲಂ ಸ್ಥಿರಃ'-ತಲೆ, ಕತ್ತು, ದೇಹ ಈ ಮೂರೂ ಸಮವಾಗಿರಬೇಕು. ಕೃಷ್ಣನನ್ನು ವರ್ಣಿಸುವಾಗ `ಮೂರು ಜಾಗಗಳಲ್ಲಿ ತಿರುವು ಇರುವವನು' ಎನ್ನುತ್ತಾರೆ. ಧ್ಯಾನ ಮಾಡುವವನು ಹೀಗೆ ಕುಳಿತುಕೊಂಡರೆ ಅವನು ಫೋಟೋವಿಗಾಗಿ ಕುಳಿತುಕೊಂಡಿದ್ದಾನೆಂದೇ ತೋರುವುದು.

`ಸಂಪ್ರೇಕ್ಷ್ಯ ನಾಸಿಕಾಗ್ರಂ ಸ್ವಂ ದಿಶಶ್ಚಾನವಲೋಕಯನ್' ಎಂದರು. ನಾಸಿಕಾಗ್ರವೆಂದರೆ ನಾವು ಮೂಗಿನ ತುದಿಯನ್ನು ನೋಡಬೇಕೆಂಬುದು ಅರ್ಥ. ಹಾಗೆ ಮಾಡಿದರೆ ಅದು ಧ್ಯಾನವೇ ಆಗುತ್ತದೆ. ಮೂಗಿನ ತುದಿಯನ್ನು ನಾವು ನೋಡುತ್ತಿದ್ದರೆ ಹೇಗಿರುತ್ತದೋ ಅದೇ ರೀತಿ ಕಣ್ಣುಗಳು ಸ್ಥಿರವಾಗಿರಬೇಕೆನ್ನುವುದನ್ನು `ನಾಸಿಕಾಗ್ರಂ ಪಶ್ಯನ್ನಿವ' ಎನ್ನುತ್ತಾರೆ. ಅನಂತರ `ಪ್ರಶಾಂತಾತ್ಮಾ ವಿಗತಭೀಃ ಬ್ರಹ್ಮಚಾರಿವ್ರತೇ ಸ್ಥಿತಃ'-ಇದಕ್ಕೆ `ಶಾಂತವಾಗಿಯೂ, ಭಯವಿಲ್ಲದೆಯೂ, ಬ್ರಹ್ಮಚಾರಿ ವ್ರತದಲ್ಲಿ ಸ್ಥಿತವಾಗಿರುವವನಾಗಿಯೂ' ಎಂದು ತಾತ್ಪರ್ಯ.

`ಮನಃ ಸಂಯಮ್ಯ ಮಚ್ಚಿತ್ತೋ ಯುಕ್ತ ಆಸೀತ ಮತ್ಪರಃ-ಮನಸ್ಸನ್ನು ನಿಗ್ರಹಿಸಿ ಹೀಗೆ ಯುಕ್ತನಾದವನು `ಮತ್ಪರ' ನಾಗಿ ಇರಬೇಕು. ನಾವು ಬೇರೆಯವರು ಧ್ಯಾನದಲ್ಲಿರುವುದನ್ನು ನೋಡಿ ಒಂದೆರಡು ದಿನಗಳು ಪ್ರಯತ್ನಪಟ್ಟು ಆಮೇಲೆ ಬಿಟ್ಟುಬಿಡೋಣವೆಂದುಕೊಂಡರೆ ಅದು ಧ್ಯಾನವಾಗುವುದಿಲ್ಲ.

`ಆಹಮೇವ ಉತ್ಕೃಷ್ಟಃ ಯಸ್ಯ ಸಃ ಮತ್ಪರಃ' [ಯಾರಿಗೆ ನಾನೇ (ಪರವಸ್ತು) ಉತ್ತಮವಾಗಿದ್ದೇನೋ ಅವನು ಮತ್ಪರಃ] ಹೀಗೆ ಇರುವವನಿಗೆ ಯಾವ ಫಲ ದೊರೆಯುವುದೆಂದರೆ,

\begin{shloka}
`ಯುಂಜನ್ನೇವಂ ಸದಾಽಽತ್ಮಾನಂ ಯೋಗೀ ನಿಯತಮಾನಸಃ|\\
ಶಾಂತಿಂ ನಿರ್ವಾಣ ಪರಮಾಂ ಮತ್ಸಂಸ್ಥಾಮಧಿಗಚ್ಛತಿ||'\\
\hfill{ಗೀತೆ-\eng{vi. 15}}
\end{shloka}

`ಹೀಗೆ ಮನೋನಿಗ್ರಹವಿರುವ ಯೋಗಿ ಯಾವಾಗಲೂ ಮನಸ್ಸನ್ನು ವಶಪಡಿಸಿಕೊಂಡು ಬಂದರೆ ಮೋಕ್ಷವನ್ನು ಪರಿಣಾಮವಾಗುಳ್ಳ ಶಾಂತಿಯಾದ ನನ್ನ ಸ್ಥಿತಿಯನ್ನು ಪಡೆಯುತ್ತಾನೆಂದು ಕೃಷ್ಣನು ಹೇಳಿದುದು. ಆದ್ದರಿಂದ ದೇವತೆಗಳನ್ನು ಧ್ಯಾನಮಾಡುತ್ತಾ ಬಂದರೆ ಅದು ಸಂಪ್ರಜ್ಞಾತ ಸಮಾಧಿವರೆಗಾದರೂ ಹೋಗಬೇಕು. ಕೆಲವರಿಗೆ ಸಗುಣದಲ್ಲಿರುವ ರುಚಿಯನ್ನು ಮೀರಿ ಅದಕ್ಕೆ ಮೇಲೆ ಹೋಗಬೇಕೆಂದಿರುತ್ತದೆ. ಅವರು ನಿರ್ಗುಣ ಧ್ಯಾನ ಮಾಡಲು ಸಾಧ್ಯವೇ ಎಂದು ಕೇಳುತ್ತಾರೆ. ಹಾಗೆ ಮಾಡಬಹುದು. ಹೇಗೆ ಮಾಡಬೇಕು?

ಮೊದಲು ಒಂದು ನಿಮಿಷ ಪ್ರಕಾಶವನ್ನು ನೋಡಬೇಕು. ಹಾಗೆ ನೋಡುವಾಗ ಪ್ರಪಂಚ ಮರೆತು ಹೋಗುತ್ತದೆ. ಮರೆತು ಹೋದಾಗ `ನಾನು ಮರೆಯುತ್ತಿದ್ದೇನೆ' ಎನ್ನುವ ಅರಿವು ಇರುವುದಿಲ್ಲ. ಹಾಗೆ ಮರೆತಿರುವ ಸಮಯದಲ್ಲಿ ಕಣ್ಣುಗಳನ್ನು ಮುಚ್ಚಿಕೊಳ್ಳಬೇಕು. ಅನಂತರ ಎರಡು ಕಣ್ಣುಗಳೂ ಅಂತರ್ಮುಖವಾಗಿ ಹೋಗುತ್ತಿವೆಯೆಂಬುದಾಗಿ ಚಿಂತನೆ ಮಾಡಬೇಕು. ಆಗ ಒಂದು ಪ್ರಕಾಶ ತಾನಾಗಿಯೇ ತೋರುತ್ತದೆ. ಆ ಪ್ರಕಾಶ ನಾವು ಲೋಕದಲ್ಲಿ ನೋಡುವ ಪ್ರಕಾಶದಂತೆ ಇರದೆ ನೀಲಿಯಾಗಿರುತ್ತದೆ. ಕೆಲವು ವೇಳೆ ಚಂದ್ರನ ಪ್ರಕಾಶದಂತೆ ಇರುತ್ತದೆ. ಆ ಪ್ರಕಾಶ ತೋರಿದೊಡನೆ ಕಣ್ಣುಗಳು ಒಳಗೆ ನೋಡುತ್ತಿವೆಯೆಂದು ಭಾವಿಸಿ ನಾನು ಶುದ್ಧ ಚೈತನ್ಯನಾಗಿರುವವನು ಎನ್ನುವ ಭಾವವನ್ನು ಸ್ಥಿರಪಡಿಸಿಕೊಳ್ಳಬೇಕು. `ಚೈತ್ಯಂ ಸರ್ವವ್ಯಾಪಿ' ಎಂದು ಕಾಣಬೇಕು. ಹೀಗೆ ಮಾಡುತ್ತಾ ಬಂದರೆ ಅದು ಅಸಂಪ್ರಜ್ಞಾತ ಸಮಾಧಿಯಲ್ಲಿ ಕೊನೆಗೊಳ್ಳುತ್ತದೆ. ಅದರಿಂದ ಆತ್ಮಸಾಕ್ಷಾತ್ಕಾರ ಪಡೆಯಬಹುದು. ಸಂಪ್ರಜ್ಞಾತ ಸಮಾಧಿ ಪಡೆಯಬೇಕಾದರೆ ಕಣ್ಣುಗಳನ್ನು ಮುಚ್ಚಿಕೊಂಡು ಭಗವಂತನ ಸ್ವರೂಪವನ್ನು ಧ್ಯಾನಮಾಡಬೇಕು. ಹೀಗೆ ಅಭ್ಯಾಸ ಮಾಡಿಕೊಂಡೇ ಬಂದರೆ ಒಂದು ದಿನ ಸ್ವರೂಪ ಎದುರಿಗೆ ಬಂದು ನಿಂತಂತಾಗುತ್ತದೆ. ಕಣ್ಣುಗಳನ್ನು ತೆರೆದರೂ ಕೂಡ ಆ ಸ್ವರೂಪವನ್ನು ಕಾಣಬಹುದು. ಇದನ್ನೇ `ದೇವತಾ ಸಾಕ್ಷಾತ್ಕಾರ'ವೆನ್ನುತ್ತಾರೆ. ಈ ಸ್ಥಿತಿ ಉಂಟಾದ ಮೇಲೆ ಆ ಕಾಲದಲ್ಲಿ ಮನಸ್ಸಿನಲ್ಲಿ ಯಾವುದಾದರೊಂದು ಚಿಂತನೆ ಮಾಡಿದರೆ ಅದಕ್ಕೆ ಉತ್ತರ ಕೂಡ ದೊರೆಯುತ್ತದೆ. ಆ ಉತ್ತರ ದೈವ ವಾಕ್ಕಾಗುತ್ತದೆ. ಹೀಗಿರುವವನು ತನ್ನನ್ನೂ ಇತರರನ್ನೂ ಉದ್ಧಾರ ಮಾಡಬಲ್ಲನು. ಇದನ್ನು ಸಾಧಿಸುವುದರಲ್ಲಿ ಸ್ವಲ್ಪ ವಿಳಂಬವಾದರೂ ಇದನ್ನೂ ಖಂಡಿತ ಸಾಧಿಸಬಹುದು. ಆದ್ದರಿಂದ `ಶಾಂತಿಂ ನಿರ್ವಾಣ ಪರಮಾಂ ಮತ್ಸಂಸ್ಥಾಮಧಿಗಚ್ಛತಿ' ಎಂದು ಕೃಷ್ಣನು ಹೇಳಿದ ರೀತಿಯಲ್ಲಿ, ಅದನ್ನು ಇಷ್ಟ ಪಟ್ಟೇ ನಾವು ಧ್ಯಾನ ಮೊದಲಾದವುಗಳನ್ನು ಮಾಡುತ್ತೇವೆ. ಇದಕ್ಕೆ ಬೇಕಾದ ಸಾಧನ-ಸಂಪತ್ತುಗಳನ್ನು ಇಟ್ಟುಕೊಂಡು ನಾವು ಧ್ಯಾನ ಮಾಡುತ್ತಾ ಬಂದರೆ ಶ್ರೇಯಸ್ಸನ್ನು ಪಡೆಯಬಹುದು.
