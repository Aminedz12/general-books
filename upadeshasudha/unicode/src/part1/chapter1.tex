{\newcommand\toptitle{}
\makeatletter
\patchcmd{\@makechapterhead}{\if@mainmatter}{\if@mainmatter\Large\bf\toptitle\vskip 20pt\par}{}{}
\makeatother

\renewcommand\toptitle{ಶ್ರೀ ಶ್ರೀ ಜಗದ್ಗುರು ಮಹಾಸ್ವಾಮಿಗಳವರ ಉಪದೇಶ}



\chapter{ಬಾಳಿನ ದಾರಿ}\label{chap1}}


\begin{shloka}
ವಿಶುದ್ಧ ಜ್ಞಾನ ದೇಹಾಯ ತ್ರೀವೇದೀ ದಿವ್ಯ ಚಕ್ಷುಷೇ|\\
ಶ್ರೇಯಃ ಪ್ರಾಪ್ತಿನಿಮಿತ್ತಾಯ ನಮಃ ಸೋಮಾರ್ಧಧಾರಿಣೇ||\\
ನಮಾಮಿ ಯಾಮಿನೀನಾಥಲೇಖಾಲಂಕೃತಕುಂತಲಾಮ್|\\
ಭವಾನೀಂ ಭವಸಂತಾಪನಿರ್ವಾಪಣಸುಧಾನದೀಮ್||
\end{shloka}


ದೇವಿಯನ್ನು ಸ್ತುತಿಸುವಾಗ ಆಕೆಯ ಸ್ವರೂಪವನ್ನು ಯಾವ ವಿಧವಾಗಿ ಭಾವಿಸಬೇಕು? `ದೇವಿಯನ್ನು ಪೂಜಿಸಿ ನಾವು ಯಾವ 
ವಿಧವಾದ ಫಲವನ್ನು ಪಡೆಯಬಹುದು' ಎನ್ನುವುದು ಮೇಲೆ ಹೇಳಿದ ಶ್ಲೋಕದಲ್ಲಿ ಸ್ಪಷ್ಟಪಡಿಸಲಾಗಿದೆ. ದೇವಿಗೆ 
ಮಾಲೆ, ಬಳೆ ಇಂತಹ ಅಲಂಕಾರಗಳು ಇದ್ದರೂ ಒಂದು ಅಲಂಕಾರ ವಿಶೇಷವಾಗಿ ಇದೆ. ಅದು ಯಾವುದು ಎನ್ನುವ ಪ್ರಶ್ನೆಗೆ 
ಉತ್ತರವಾಗಿ `ಯಾಮಿನೀನಾಥ ಲೇಖಾಲಂಕೃತ ಕುಂತಲಾಮ್' ಎಂದು ಶ್ಲೋಕದಲ್ಲಿ ಇದೆ. `ಯಾಮಿನೀನಾಥ' 
ಎಂದರೆ ಚಂದ್ರ. `ಲೇಖಾ' ಎಂದರೆ ಕಲೆ. ಆದ್ದರಿಂದ `ಯಾಮಿನೀನಾಥಲೇಖಾಲಂಕೃತ ಕುಂತಲಾಮ್' ಎನ್ನುವುದಕ್ಕೆ 
ಚಂದ್ರಕಲೆಯಿಂದ ಅಲಂಕೃತವಾದ ಶ್ರೀಮುಡಿಯುಳ್ಳವಳು ಎಂದು ಅರ್ಥ. ಈ ಆಭರಣದ ವಿಶೇಷವೇನು? ಈ 
ಪ್ರಪಂಚದಲ್ಲಿ ಯಾರೂ ಚಂದ್ರನನ್ನು ತಲೆಯಲ್ಲಿ ಇಟ್ಟುಕೊಳ್ಳಲಾರರು. ಆದರೆ ದೇವಿ ಚಂದ್ರನನ್ನು ಹೇಗೆ ಇಟ್ಟುಕೊಂಡಿದ್ದಾಳೆ? ಚಂದ್ರ 
ಎಲ್ಲಿ ಇಡಲ್ಪಟ್ಟಿದ್ದಾನೆ ಎಂದು ಹೇಳುವುದರಲ್ಲಿ ಮುಖ್ಯವಾದ ಭಾವವಿದೆ. ಪರಮಶಿವನೂ ದೇವಿಯೂ ಒಟ್ಟಾಗಿ 
ಜ್ಞಾನ ಕಾಯರಾಗಿದ್ದಾರೆ. ಅವರಿಗೆ ಒಂದು ವಿಧವಾದ ಶರೀರ ನಿಜಕ್ಕೂ ಇಲ್ಲ. ಆದರೂ ಭಕ್ತರನ್ನು ಅನುಗ್ರಹಿಸಲು 
ಅವರು ಹಲವು ಶರೀರಗಳನ್ನು ಪಡೆಯುವುದುಂಟು. ಈಶ್ವರನು ಭಕ್ತನನ್ನು ಅನುಗ್ರಹಿಸಲು ಮೊದಲು ಯಾವ ಶರೀರವನ್ನು 
ಪಡೆದನೋ ಅದು ನಮಗೆ ಈಗ ಉಪಾಸನೆಗೆ ಸಹಾಯ ಮಾಡುತ್ತದೆ. ದೇವಿಯ ತಲೆಯ ಮೇಲಿರುವ ಚಂದ್ರನ 
ಭಾವ ಹೀಗಿರುತ್ತದೆ-ಚಂದ್ರ ಪ್ರಕಾಶವನ್ನು ಹರಡುತ್ತಾನೆ. ಅದೇ ವಿಧವಾಗಿ ಜ್ಞಾನ ಪ್ರಕಾಶ ಹರಡುತ್ತದೆ. ಆದ್ದರಿಂದ ದೇವಿ 
ತಲೆಯ ಮೇಲೆ ಚಂದ್ರನನ್ನು ಇಟ್ಟುಕೊಂಡಿದ್ದಾಳೆ ಎನ್ನುವುದಕ್ಕೆ ಆಕೆ ಜ್ಞಾನದಿಂದ ಕೂಡಿದ್ದಾಳೆ ಎನ್ನುವುದು ಭಾವ. ಹೀಗೆ 
ನಿಜವಾಗಿಯೂ ಇದ್ದರೂ ಕೂಡ ಸಗುಣೋಪಾಸನೆ ಮಾಡುವಾಗ ಆಕೆಯನ್ನು ತಲೆಯ ಮೇಲೆ ಚಂದ್ರನನ್ನು ಇಟ್ಟುಕೊಂಡಂತೆ 
ಸ್ಮರಿಸುವುದು ಆವಶ್ಯಕ. ಚಂದ್ರನನ್ನು ತಲೆಯ ಮೇಲೆ ಇಟ್ಟುಕೊಂಡಿರುವುದರ ಭಾವ, ತಲೆ ಶ್ರೇಷ್ಠವಾದ ಜ್ಞಾನದಿಂದ 
ಕೂಡಿದೆ ಎಂದು. ದೇವಿಯಂತೆಯೇ ಈಶ್ವರನೂ ಕೂಡ `ಸೋಮಾರ್ಧಧಾರಣೇ' ಎಂದರೆ ತಲೆಯಲ್ಲಿ ಚಂದ್ರಕಲೆಯನ್ನು ಧರಿಸಿರುವವನು ಎಂದು ಸ್ತುತಿಸಲ್ಪಟ್ಟಿದ್ದಾನೆ.

ಶ್ಲೋಕದಲ್ಲಿ ಹೇಳಲ್ಪಟ್ಟ ದೇವಿಯ ಎರಡನೆಯ ವಿಶೇಷಣ `ಭವ ಸಂತಾಪ ನಿರ್ವಾಪಣಸುಧಾನದೀಮ್' ಎನ್ನುವುದು. ಈ ಪ್ರಪಂಚದಲ್ಲಿ 
ನಾವು ಹೊಂದುವ ಸುಖ ಶಾಶ್ವತವಾದುದಲ್ಲ. ಅಲ್ಲದೆ, ನಾವು ಪಡೆಯುವ ಸುಖ ದುಃಖಗಳಿಂದ ಕೂಡಿರುತ್ತದೆ. ದುಃಖವಿಲ್ಲದ 
ಸುಖ ಪ್ರಪಂಚದಲ್ಲಿಲ್ಲ. `ಯೌವನಂ ಜರಯಾಗ್ರಸ್ತಂ' ಎಂದು ಹೇಳಲಾಗಿದೆ. ಯೌವನ ಸುಖ ಪಡೆಯುವುದಕ್ಕೆ ಒಂದು 
ಕಾರಣವಾಗಿದೆ. ಮುಪ್ಪು ಇದನ್ನು ಕಬಳಿಸಿ ಬಿಡುತ್ತದೆ. ಈ ರೀತಿ ಒಂದೊಂದು ಸುಖವೂ ತನ್ನ ಶತ್ರುವಾದ ದುಃಖದಿಂದ 
ನಾಶವಾಗುತ್ತದೆ. ಪರಿಶೀಲಿಸಿ ನೋಡಿದರೆ ನಾವು ಯಾವುದನ್ನು ಸುಖವೆಂದು ಭಾವಿಸುತ್ತೇವೋ 
ಹಾಗೂ ಯಾವುದನ್ನು ದುಃಖವೆಂದು ಭಾವಿಸುತ್ತೇವೋ ಅವೆರಡೂ ದುಃಖವೇ ಆಗುತ್ತವೆಂದು ತಿಳಿಯುತ್ತೇವೆ. ದುಃಖಗಳನ್ನು 
ಯಾವಾಗಲೂ ಅನುಭವಿಸಿದರೆ ನಮಗೆ ಒಂದು `ತಾಪ' ಅಥವಾ ಬಿಸಿ ಉಂಟಾಗುತ್ತದೆ. ಇದನ್ನು ದೂರ 
ಮಾಡುವುದಕ್ಕೆ ದಾರಿ ಇದೆಯೇ? ಹೌದು, ಭವಾನಿಯೇ ಆ ದಾರಿ. ಬೇಸಿಗೆಯಲ್ಲಿ ಬೇಗೆಯಿಂದಾಗಿ 
ನಮಗೆ ಶರೀರದಲ್ಲಿ ತಾಪ ಉಂಟಾಗುತ್ತದೆ. ನಾವು ಹೊಳೆಯಲ್ಲಿ ಸ್ನಾನ ಮಾಡಿದರೆ ದುಃಖ ದೂರವಾಗಿ 
ಆನಂದವಾಗುತ್ತದೆ. ಅದೇ ರೀತಿ ಭವಸಾಗರದಲ್ಲಿ ತಾಪವನ್ನು ಅನುಭವಿಸುತ್ತಿರುವ ನಮಗೆ ದೇವಿ 
ತಾಪವನ್ನು ದೂರ ಮಾಡುವ ಹೊಳೆಯಂತೆ ಇದ್ದಾಳೆ. ಯಾವ ವಿಧವಾದ ಹೊಳೆಯಂತೆ ಇದ್ದಾಳೆ? 
`ಸುಧಾನದೀಮ್'-ಅಮೃತದ ಹೊಳೆಯಂತೆ ಇದ್ದಾಳೆ. ಸಾಮಾನ್ಯವಾದ ಹೊಳೆಯಲ್ಲಿ ಮುಳುಗಿದರೇನೆ 
ನಮಗೆ ಆನಂದವಾಗುತ್ತದೆ ಎಂದ ಮೇಲೆ ಅಮೃತದ ಹೊಳೆಯಲ್ಲಿ ಮುಳುಗಿದರೆ ಆನಂದವೂ, ಶಾಂತಿಯೂ, 
ತೃಪ್ತಿಯೂ ನಾವು ಪಡೆಯಬಹುದು ಎನ್ನುವುದರಲ್ಲಿ ಸಂದೇಹವಾದರೂ ಇದೆಯೇ?

ದುಃಖವನ್ನು ದೂರ ಮಾಡಲು ಪಡೆದ ಪರವಸ್ತುವಿನ ಆಕಾರದ ಹೆಸರೇನು ಎನ್ನುವ ಪ್ರಶ್ನೆಗೆ ಉತ್ತರವಾಗಿ ಶ್ಲೋಕದಲ್ಲಿ 
`ಭವಾನೀಂ' ಎಂದು ತಿಳಿಸಲಾಗಿದೆ. ಅದೇನೆಂದರೆ ಸ್ತ್ರೀ ರೂಪದಲ್ಲಿ ದರ್ಶನವನ್ನು ಕೊಡುವ ಪರವಸ್ತು ಇಲ್ಲಿ `ಭವಾನಿ' ಎಂದು 
ಕರೆಯಲ್ಪಟ್ಟಿದ್ದಾಳೆ. 
ಆಕೆಯನ್ನು ವಂದಿಸಿ ನಾವು ತಾಪದಿಂದ ಬಿಡುಗಡೆ ಪಡೆದು ಜ್ಞಾನವನ್ನೂ, ಶಾಂತಿಯನ್ನೂ ಸುಖವನ್ನೂ ಪಡೆಯುವುದು ಅವಶ್ಯಕವೆನ್ನುವುದು ಶ್ಲೋಕದ ಭಾವ.

ವಿದ್ಯೆಯನ್ನು ಏತಕ್ಕಾಗಿ ಕಲಿಯಬೇಕು ಎನ್ನುವ ಪ್ರಶ್ನೆ  ಕೆಲವರಿಗೆ ಉಂಟಾಗಬಹುದು. ಹುಲಿ, ಆನೆಯಂತಹ ಬಲವುಳ್ಳ ಪ್ರಾಣಿಗಳು ಜೀವಿಸುತ್ತವೆ. ಅವುಗಳಂತೆಯೇ 
ಏಕೆ ಜೀವಿಸಬಾರದು ಎಂದು ಕೂಡ ಕೆಲವರಿಗೆ ತೋರಬಹುದು. ಭಗವಂತನು ಬಲವುಳ್ಳ ಪ್ರಾಣಿಗಳನ್ನು ಸೃಷ್ಟಿಸಿದರೂ ಅವುಗಳು ವಿವೇಕಹೀನತೆಯಿಂದಾಗಿ 
ಭಗವಂತನನ್ನು ಪಡೆದುಕೊಳ್ಳುವ ಯೋಗ್ಯತೆಯನ್ನು ಪಡೆದಿಲ್ಲ. ಮನುಷ್ಯನೂ ಪಾಮರರಂತೆ ಜೀವಿಸಿ ಬಾಳನ್ನು ವ್ಯರ್ಥಮಾಡಿಕೊಳ್ಳಬಹುದು. 
ಹಾಗಲ್ಲದೆ ವಿವೇಕದಿಂದ ಬಾಳಿ ಪರವಸ್ತುವಿನ ದರ್ಶನವನ್ನು ಪಡೆದು ಜನ್ಮ ಸಮುದ್ರದಿಂದ ಬಿಡುಗಡೆ ಪಡೆಯಬಹುದು. ಈ ಶಕ್ತಿಯನ್ನು ದೇವರು 
ಮನುಷ್ಯನಿಗೆ ಕೊಟ್ಟಿದ್ದಾನೆ. ಪ್ರಾಣಿಗಳಿಗೂ ಬುದ್ಧಿ ಇದೆ. ಆದರೆ ಅವುಗಳ ಬುದ್ಧಿಗೆ ನಿಶ್ಚಯವಾಗಿಯೂ ಒಂದು ಎಲ್ಲೆ ಇದೆ. ಇದಕ್ಕೆ ವಿರೋಧವಾಗಿ 
ಮನುಷ್ಯನ ಬುದ್ಧಿಗೆ ಎಲ್ಲೆ ಇಲ್ಲ. ಅವನು `ನಾನು ಯಾವ ವಿಧವಾಗಿ ಬಾಳಿದರೆ ಒಳ್ಳೆಯದನ್ನು ಪಡೆಯುವೆನು' ಎಂದು ಯೋಚಿಸಿ ಅದಕ್ಕೆ ಸರಿಯಾದ 
ರೀತಿಯಲ್ಲಿ ಬಾಳುವ ಶಕ್ತಿಯುಳ್ಳವನು. `ನಾನು ನಿಜಕ್ಕೂ ಯಾರು? ಎಲ್ಲಿಂದ ಬಂದೆ? ಎಲ್ಲಿಗೆ ಹೋಗುವೆನು?' ಎಂದು ಮನುಷ್ಯನು ಯೋಚಿಸಬಲ್ಲನು. 
ಹೀಗೆ ಚಿಂತಿಸುವಿಕೆಯು ಅವನನ್ನು ಇತರ ಪ್ರಾಣಿಗಳಿಂದ ಬೇರ್ಪಡಿಸುವುದು. ಚಿಂತಿಸುವುದನ್ನೇ ಅಲ್ಲದೆ ಜ್ಞಾನವಂತನಾಗಿಯೂ ಪರವಸ್ತುವನ್ನು ಪಡೆಯುವ 
ಯೋಗ್ಯತೆಯು ಅವನಿಗೆ ಇದೆ. ಆದ್ದರಿಂದ `ಜಂತೂನಾಂ ನರಜನ್ಮ ದುರ್ಲಭಂ' ಎಂದು ಹೇಳಲಾಗಿದೆ, ಅಂದರೆ ಜಂತುಗಳಾಗಿ ಹುಟ್ಟುವುದಕ್ಕಿಂತ ಮನುಷ್ಯ 
ಜನ್ಮ ಹೆಚ್ಚು ಎಂದು ಭಾವ. ಪ್ರಾಣಿಗಳು ತಮ್ಮ ಬುದ್ಧಿಯಿಂದ ಯಾವುದನ್ನು ಹುಡುಕುತ್ತವೆ? ಈ ಪ್ರಶ್ನೆಗೆ ವಿಶ್ರಾಂತಿ, ಆಹಾರ, ತಿರುಗುವಿಕೆ ಎಂದೇ ಉತ್ತರವಾಗುತ್ತದೆ.

ಮನುಷ್ಯನೂ ಇವುಗಳಲ್ಲಿಯೇ ತೊಡಗಿರುವವನಾದರೆ ಅವನಿಗೂ ಪ್ರಾಣಿಗಳಿಗೂ ವ್ಯತ್ಯಾಸವೇ ಇರುವುದಿಲ್ಲ. ಹಾಗಿದ್ದರೆ ಮನುಷ್ಯ ಜನ್ಮ ವ್ಯರ್ಥ 
ಮಾಡಿಕೊಂಡಂತೇ ಸರಿ. ಆದ್ದರಿಂದ ನಾವು ಆ ರೀತಿ ಇರಬಾರದು. ನಾವು ಜ್ಞಾನ ಪಡೆಯಲು ಖಂಡಿತ ಪ್ರಯತ್ನಪಡಬೇಕು. ಜ್ಞಾನ ಪಡೆಯುವುದು ಹೇಗೆ? ಕೆಲವರು 
ತಾವಾಗಿಯೇ, ಗುರುವಿನ ಉಪದೇಶವಿಲ್ಲದೆ, ವಿದ್ಯೆ  ಪಡೆದಿದ್ದೇವೆಂದು ಹೇಳುತ್ತಾರೆ. ಆದರೆ ಹಾಗೆ ಮಾಡುವುದು ಸರಿಯಲ್ಲ. ನಾವು ಈಗ ಮಾತನಾಡುತ್ತೇವೆ. ನಾವು 
ಚಿಕ್ಕವರಾಗಿದ್ದಾಗ ನಮ್ಮ ತಾಯಿಯೋ, ತಂದೆಯೋ, ಅಲ್ಲಿದ್ದವರೋ ಮಾತನಾಡದೆ ಇದ್ದಿದ್ದರೆ ನಮಗೆ ಮಾತನಾಡಲು ಬರುತ್ತಿರಲಿಲ್ಲ. ಆದ್ದರಿಂದ 
ಅವರ ಮಾತುಗಳನ್ನು ಕೇಳಿಯೇ ನಾವು ಮಾತನಾಡುವುದನ್ನು ಕಲಿತುಕೊಂಡಿದ್ದೇವೆ. `ಯಾರ ಸಹಾಯವೂ ಇಲ್ಲದೆ ನಾನು ದೊಡ್ಡ ವಿದ್ವಾಂಸನಾಗಿ ಬಿಟ್ಟೆ' ಎಂದು 
ಹೇಳಿಕೊಂಡರೆ ಅದು ಸುಳ್ಳೇ ಆಗುತ್ತದೆ. ನಾವು ಕಲಿಯಬೇಕಾದುವು ಅನೇಕವಿವೆ. ಹೇಗೆ?

\begin{shloka}
ಜಾತ್ಯಂಧಾ ಜಾತಿಬಧಿರಾ ಜಾತಿಮೂಕಾಶ್ಚ ತೇ ಜನಾಃ|\\
ಸಮ್ಯಗಾರಾಧಿತಾ ಯೈರ್ನ ಸಂತೋ ವಿಜ್ಞಾನ ಚಿಂತವಃ||
\end{shloka}

ಎಂದು ಹೇಳಲ್ಪಟ್ಟಿದೆ. ಯಾರು ಜ್ಞಾನ ಸಮುದ್ರವಾಗಿ ಪ್ರಕಾಶಿಸುವ ಮಹಾತ್ಮರ ದರ್ಶನ ಪಡೆಯಲಿಲ್ಲವೋ, ಅವರ ಉಪದೇಶವನ್ನು ಕೇಳಲಿಲ್ಲವೋ, 
ಅವರ ಮಾತಿನಂತೆ ಬಾಳನ್ನು ತಿದ್ದುಕೊಳ್ಳಲಿಲ್ಲವೋ, ಅಂಥಹವರು ಹುಟ್ಟಿನಿಂದಲೇ ಕುರುಡರು; ಹುಟ್ಟಿನಿಂದಲೇ ಕಿವುಡರು; ಹುಟ್ಟಿನಿಂದಲೇ 
ಮೂಕರು; ಎಂದು ಅರ್ಥ. ನಾವು ಈ ರೀತಿ ಕುರುಡರಾಗಿಯೋ, ಕಿವುಡರಾಗಿಯೋ, ಮೂಕರಾಗಿಯೋ, ಇರಬಾರದು. ನಾವು ಜ್ಞಾನ 
ಸಮುದ್ರರಾಗಿ ಪ್ರಕಾಶಿಸುವ ಮಹಾತ್ಮರನ್ನು ಪೂಜಿಸಿ ಜ್ಞಾನ ಪಡೆಯಬೇಕು.

ಸುರಂಗದಿಂದ ವಜ್ರವನ್ನು ತೆಗೆದರೆ ಅದು ವಜ್ರವಾದರೂ ಕೂಡ ಹೊಳೆಯುವುದಿಲ್ಲ. ಅದನ್ನು ಸಾಣೆ ಹಿಡಿದರೆ ಅದು ಬಹಳವಾಗಿ ಪ್ರಕಾಶಿಸುತ್ತದೆ. 
ಸಾಮಾನ್ಯವಾದ ಕಲ್ಲನ್ನು ಸಾಣೆ ಹಿಡಿದರೆ ಅದು ಸವೆದು ಹೋಗುತ್ತದೆ. ಅಲ್ಲದೆ, ವಜ್ರ ಸಹಜವಾಗಿ ಸಿಕ್ಕಿದರೂ ಅದನ್ನು ಹರಿತ ಮಾಡುವವರೆಗೆ ಗಾಜನ್ನು 
ಕಡಿಯಲು ಅದರಿಂದ ಸಾಧ್ಯವಾಗುವುದಿಲ್ಲ. ಅದೇ ರೀತಿ ಮನುಷ್ಯನು ಸಹಜವಾಗಿಯೇ ಸ್ವಲ್ಪ ಮಟ್ಟಿಗೆ ಜ್ಞಾನಿ. ಶಾಸ್ತ್ರಗಳು ಮತ್ತು ಗುರುವಿನ 
ಉಪದೇಶದಿಂದ ವಿಶೇಷವಾಗಿ ಜ್ಞಾನ ಉಂಟಾಗುತ್ತದೆ. ಇದು ಸಾಣೆ ಹಿಡಿದ ವಜ್ರದಂತೆ ಪ್ರಕಾಶಿಸುತ್ತದೆ.

\begin{shloka}
ಏಕಂ ಹಿ ಚಕ್ಷುರಮಲಂ ಸಹಜಾವಬೋಧಃ\\
ವಿದ್ವದ್ಭಿರೇವ ಸಹ ಸಂವಸತಿರ್ದ್ವಿತೀಯಮ್|\\
ಯಸ್ಯಾಸ್ತಿ ನ ದ್ವಯಮಿದಂ ಸ್ಫುಟಮೇವ ಸೋಽನ್ತಃ\\
ತಸ್ಯಾಪ್ಯಮಾರ್ಗಚಲನೇ ವದ ಕೋಽಪರಾಧಃ||
\end{shloka}

ಎಂದು ಒಬ್ಬರು ಹೇಳಿದರು. ಮನುಷ್ಯನಿಗೆ ಎರಡು ಕಣ್ಣುಗಳಿವೆ. ಅವುಗಳಲ್ಲಿ ಒಂದು ಸಾಮಾನ್ಯ ಬುದ್ಧಿಯುಳ್ಳದ್ದು. ಕೆಲವರಿಗೆ ಸಾಮಾನ್ಯ ಬುದ್ಧಿಯೇ 
ಇರುವುದಿಲ್ಲ. ಅವರಿಗೆ ಹೇಳಿಕೊಟ್ಟರೂ ತಿಳಿಯುವುದಿಲ್ಲ. ಹಾಗಿದ್ದರೆ ಕಷ್ಟ. `ವಿದ್ವದ್ಭಿರೇವ ಸಹ ಸಂವಸತಿಃ ದ್ವಿತೀಯಮ್'-`ವಿದ್ವದ್ಭಿಃ' ವಿದ್ವಾಂಸರು ಅಂದರೆ ಜ್ಞಾನ 
ಸಮುದ್ರರಾಗಿ ಪ್ರಕಾಶಿಸುವ ಮಹಾತ್ಮರ ಸಹವಾಸವೇ ಆವಶ್ಯಕ. ಅವರನ್ನು ಸೇವಿಸಿ ಅವರು ಮಾಡುವ ಉಪದೇಶವನ್ನು ಕೇಳಿ, ಮನಸಿನಲ್ಲಿಟ್ಟುಕೊಂಡು, ಆ 
ರೀತಿ ನಡೆಯುತ್ತಾ ಜ್ಞಾನ ಪಡೆಯುವುದು ಎರಡನೆಯ ಕಣ್ಣಾಗುತ್ತದೆ. ಎರಡು ಕಣ್ಣುಗಳೂ ಮನುಷ್ಯನಿಗೆ ಬೇಕು. ಸಾಮಾನ್ಯ ಬುದ್ಧಿ ಇಲ್ಲದೆ 
ಇದ್ದು ಕೇವಲ ಗುರು ಹೇಳಿದ್ದನ್ನು ಕೇಳಿಕೊಂಡು ಇರಬಾರದೆ ಎಂದರೆ ಅದಕ್ಕೆ ಒಂದು ಕಥೆ ನೆನಪಿಗೆ ಬರುತ್ತದೆ. ನಾಲ್ಕು ಮಂದಿ ಒಬ್ಬ ಗುರುವಿನ 
ಹತ್ತಿರ ಮಂತ್ರಶಾಸ್ತ್ರವನ್ನು ಕಲಿತುಕೊಂಡರು. ಪಾಠ ಮುಗಿದ ಮೇಲೆ, ತಮ್ಮ ತಮ್ಮ ಮನೆಗಳಿಗೆ ಹೋಗುವ ದಾರಿಯಲ್ಲಿ ಒಂದು ಕಾಡು 
ಇದ್ದಿತು. ದಾರಿಯಲ್ಲಿ ಸತ್ತು ಹೋದ ಒಂದು ಹುಲಿಯ ಶವವನ್ನು ನೋಡಿದರು. ಕೆಲವರ ಮನಸ್ಸಿನಲ್ಲಿ `ನಾವು ಶವಕ್ಕೆ ಜೀವ ಕೊಡುವ ಸಂಜೀವಿನಿ 
ಮಂತ್ರವನ್ನು ಕಲಿತಿದ್ದೇವಲ್ಲಾ, ಅದನ್ನು ಪರೀಕ್ಷಿಸಬೇಕು' ಎನ್ನುವ ವಿಚಾರ ಬಂದಿತು. ಆದರೆ ನಾಲ್ಕನೆಯ ವಿದ್ಯಾರ್ಥಿಗೆ ಅದು 
ಅಪಾಯವನ್ನುಂಟುಮಾಡುತ್ತದೆಂದು ತೋರಿತು. ಅವನು ಪ್ರಯತ್ನ ಪಡದೆ ಇದ್ದರೂ, ಇತರರು ಅವನು ಕೊಟ್ಟ ಎಚ್ಚರಿಕೆಯನ್ನು ಗಮನಿಸಲಿಲ್ಲ. ಆದ್ದರಿಂದ ಅವನು ಸ್ವಲ್ಪ ದೂರ ಸರಿದು ಒಂದು ಮರವನ್ನು ಹತ್ತಿ ಕುಳಿತುಕೊಂಡನು. ಇತರ ಮೂರು ಮಂದಿ ತಾವು ಕಲಿತುಕೊಂಡಿದ್ದ ಮಂತ್ರದ ಶಕ್ತಿಯನ್ನು ತಿಳಿಯುವುದಕ್ಕಾಗಿ ಹುಲಿಗೆ ಜೀವಕೊಟ್ಟರು. ಉತ್ತಮವಾದ ಕೆಲಸವನ್ನು ಮಾಡಿದರು! ಏಕೆಂದರೆ ಬದುಕಿದ ಹುಲಿ ಅವರನ್ನು ತನ್ನ ಹಸಿವು ತೀರಿಸಿ ಕೊಳ್ಳುವುದಕ್ಕಾಗಿ ಕೊಂದು ಬಿಟ್ಟಿತು. ಈ ಮೂರು ಮಂದಿ ವಿದ್ಯಾರ್ಥಿಗಳು ಗುರುವಿನ ಬಳಿ ಚೆನ್ನಾಗಿ ಮಂತ್ರಶಾಸ್ತ್ರವನ್ನು ಕಲಿತಿದ್ದರೂ ಸಾಮಾನ್ಯ ಬುದ್ಧಿ ಇಲ್ಲದ್ದರಿಂದ ಆಪತ್ತಿಗೆ ಒಳಗಾದರು. 

ಇದರಿಂದ ಎರಡು ವಿಧವಾದ ಜ್ಞಾನ ಪ್ರತಿಯೊಬ್ಬನಿಗೂ ಆವಶ್ಯಕವೆಂದು ತಿಳಿಯುತ್ತದೆ. ಒಂದು ಕತ್ತಿ ಇದೆ. ಅದನ್ನು ಹರಿತ ಮಾಡದೆ ಇದ್ದರೆ 
ಕಡಿಯಲಾಗುವುದಿಲ್ಲ. ಅದೇ ರೀತಿ ಗುರುವಿನ ಬಳಿ ಕಲಿಯದಿದ್ದರೆ ನಮ್ಮ ವಿದ್ಯೆ ಪ್ರಕಾಶಿಸುವುದಿಲ್ಲ. ಯಾರಿಗೆ ಈ ವಿಧವಾದ ಎರಡು ಜ್ಞಾನಗಳು 
ಇಲ್ಲವೋ ಅವನು ಕುರುಡನಂತೆ. ಅವನು ತಪ್ಪು ಮಾಡಿದರೆ ಆಶ್ಚರ್ಯ ಪಡಬೇಕಾಗಿಲ್ಲ. ನಾವು ತಪ್ಪು ಮಾಡದೆ ಬಾಳಬೇಕಾದರೆ ಎರಡು ವಿಧ ಜ್ಞಾನವನ್ನು ಪಡೆಯಬೇಕು.

ವಿದ್ಯೆಯನ್ನು ಯಾವ ವಯಸ್ಸಿನಲ್ಲಿ ಕಲಿಯಬೇಕು? ಕೆಲವರು `ನಾಳೆ ಕಲಿಯೋಣ, ವಯಸ್ಸಾದ ಮೇಲೆ ವಿದ್ಯೆ ಕಲಿಯೋಣ' ಎಂದು ಮುಂದಕ್ಕೆ 
ಹಾಕಿಕೊಂಡೇ ಹೋದರೆ ಪ್ರಯತ್ನಗಳು ಫಲಿಸುವುದಿಲ್ಲ. ಸಂಸ್ಕೃತದಲ್ಲಿ ವಿದ್ಯೆ ಕಲಿಯಬೇಕಾದ ವಯಸ್ಸನ್ನು ಕುರಿತು `ಲಾಲಯೇತ್‌ಪಂಚ 
ವರ್ಷಾಣಿ ದಶ ವರ್ಷಾಣಿ ದಂಡಯೇತ್' ಎಂದು ಹೇಳಲಾಗಿದೆ. ಮೊದಲು ಐದು ವರ್ಷಗಳು ಮಕ್ಕಳನ್ನು ಬಹಳ ಪ್ರೀತಿಯಿಂದ ಸಾಕಬೇಕು ಏಕೆ? ಏಕೆಂದರೆ, 
ಮಕ್ಕಳಿಗೆ ಶರೀರದಲ್ಲಿ ಹೆಚ್ಚಾಗಿ ಬಲವಿರುವುದಿಲ್ಲ. ಐದು ವರ್ಷಗಳಾದ ಮೇಲೆ ಅವರಿಗೆ ಮಾತನಾಡುವುದಕ್ಕೂ, ಕೆಲಸಗಳನ್ನು ಮಾಡುವುದಕ್ಕೂ 
ಶಕ್ತಿ ಇರುತ್ತದೆ. ಆದಾದಮೇಲೆ ಹತ್ತು ವರ್ಷಗಳು `ದಶ ವರ್ಷಾಣಿ ದಂಡಯೇತ್' ಎಂದರೆ ದೊಣ್ಣೆಯಿಂದ ಹೊಡೆಯಬೇಕೆಂದಲ್ಲ. ಆದರೆ 
ಚೆನ್ನಾಗಿ ಉಪದೇಶಿಸಬೇಕೆಂದು ಭಾವ. ಒಬ್ಬ ಶಾಲೆಗೆ ಹೋಗುವ ವಿದ್ಯಾರ್ಥಿ ಬೆಳಗ್ಗೆ ಐದು ಘಂಟೆಯಾದ ಮೇಲೆ ಮಲಗಿರಲು ಇಷ್ಟಪಟ್ಟರೆ ತಂದೆ ಅವನನ್ನು 
`ನೀನು ಚೆನ್ನಾಗಿ ಓದಿಕೊಳ್ಳಬೇಕು ಈ ವಿಧವಾಗಿ ಕಾಲಹರಣ ಮಾಡಬಾರದೆಂದು ತಿಳಿಯ ಹೇಳಿ ಎಬ್ಬಿಸಬೇಕು ಅದೇ ರೀತಿ ವಿದ್ಯಾಬ್ಯಾಸ ಮಾಡುವ ಜಾಗದಲ್ಲೂ ಗುರು 
ವಿದ್ಯಾರ್ಥಿಯನ್ನು ಚೆನ್ನಾಗಿ ಗಮನವಿಟ್ಟು ಕಲಿಯುವಂತೆ ಮಾಡಬೇಕು.' ಹೀಗೆ ಮಾಡುವುದೇ ಶಿಕ್ಷೆಯಾಗುತ್ತದೆ.    

ಮೇಲೆ ಹೇಳಿದಂತೆ ನಾವು ಓದಿದರೆ ವಿದ್ಯೆಯನ್ನು ಚೆನ್ನಾಗಿ ಪಡೆಯಬಹುದು. ವಿದ್ಯೆಯಲ್ಲೂ ಅನೇಕ ವಿಧವಿದೆ. ಲೌಕಿಕ ವಿದ್ಯೆ, ವೈದಿಕ ವಿದ್ಯೆ ಮತ್ತು 
ಆಧ್ಯಾತ್ಮಿಕ ವಿದ್ಯಾ  ಎಂದು ವಿಭಾಗ ಮಾಡುತ್ತಾರೆ. ಭಗವಂತನು ಗೀತೆಯಲ್ಲಿ `ಆಧ್ಯಾತ್ಮ ವಿದ್ಯಾ ವಿದ್ಯಾ ನಾಂ' ಎಂದು ಹೇಳಿದ್ದಾನೆ. ಅಂದರೆ ಅನೇಕ 
ವಿದ್ಯೆಗಳಲ್ಲಿ ಶ್ರೇಷ್ಠವಾದುದು ಆಧ್ಯಾತ್ಮ ವಿದ್ಯೆ. `ನಾವು ಶ್ರೇಷ್ಠವಾದ ವಿದ್ಯೆಯನ್ನಲ್ಲವೇ ಕಲಿತುಕೊಳ್ಳಬೇಕು. ಆದ್ದರಿಂದ ಆಧ್ಯಾತ್ಮ ವಿದ್ಯೆ ಕಲಿಯೋಣ' 
ಎಂದರೆ ಅದಕ್ಕೆ ಬೇಕಾದ ಯೋಗ್ಯತೆಗಳೆಲ್ಲವನ್ನೂ ಪಡೆದೆ ನಾವು ಹಾಗೆ ಕಲಿಯಬಹುದು. ಆದರೆ ಅದು ಕಷ್ಟ.

`ಪರೀಕ್ಷ್ಯ ಲೋಕಾನ್ ಕರ್ಮಚಿತಾನ್' ಎಂದು ಉಪದೇಶದಲ್ಲಿ ಹೇಳಲಾಗಿದೆ. ಅಂದರೆ ಲೋಕವನ್ನು ಮನಸ್ಸಿನಲ್ಲಿ ಪರೀಕ್ಷಿಸಿ ನಂತರ ಎಲ್ಲವನ್ನೂ ಬಿಟ್ಟುಬಿಡಬೇಕೆಂದು ಅರ್ಥ. 
ನಾವು ಈ ರೀತಿ ಪರೀಕ್ಷೆ ಮಾಡದೆ ಬಿಟ್ಟು ಬಿಟ್ಟವೆಂದರೆ, ಬಿಟ್ಟಿದ್ದು ಫಲವನ್ನು ಕೊಡುವುದಿಲ್ಲ. ಏಕೆಂದರೆ ಮನಸ್ಸು ಪಕ್ವವಾಗಿರುವುದಿಲ್ಲ. ಉದಾಹರಣೆಗೆ - ಆಸೆಯುಳ್ಳವನು 
ಸಂನ್ಯಾಸವನ್ನು ತೆಗೆದುಕೊಳ್ಳಬಹುದು. ಆದರೂ ಅವನು ಸಂಸ್ಕಾರಗಳ ಬಲದಿಂದಾಗಿ ಸಂನ್ಯಾಸವನ್ನು ತೆಗೆದುಕೊಂಡ 
ಮೇಲೂ ಹೋಟಲಿಗೆ ಹೋಗಿ ಊಟ ಮಾಡಬಹುದೇ? ಹಾಗೆ ಸಂನ್ಯಾಸ ತೆಗೆದುಕೊಂಡರೆ ಏನು ಪ್ರಯೋಜನ?

ಆದ್ದರಿಂದ ಮೊದಲು ಲೌಕಿಕ ವಿದ್ಯೆ ಕಲಿಯಬೇಕು. ಲೌಕಿಕವಾದ ವಿದ್ಯೆ ಎಂದರೇನು? ನಾವು ಹೇಗೆ ನಡೆದುಕೊಂಡರೆ ಪ್ರಪಂಚಕ್ಕೆ ಒಪ್ಪಿಗೆಯಾಗುತ್ತದೋ, ನಾವು 
ಹೇಗೆ ನಡೆದುಕೊಂಡರೆ ಪ್ರಪಂಚಕ್ಕೆ ಕಷ್ಟಕೊಟ್ಟಂತೆ ಆಗುವುದಿಲ್ಲವೋ, ಅದೇ ಲೌಕಿಕವಾದ ವಿದ್ಯೆ. ಉದಾಹರಣೆಗೆ ನಾವು ಒಂದು ಮನೆಯನ್ನೇ ಕಟ್ಟಿಸಬಹುದು, 
ಹಣವನ್ನೇ ಸಂಪಾದಿಸಬಹುದು ರಾಜ್ಯವನ್ನಾಳಬಹುದು, ತಪ್ಪು ಮಾಡಿದವರನ್ನು ಶಿಕ್ಷಿಸುವ ಸಂದರ್ಭ ಬರಬಹುದು. ಹೀಗಿರತಕ್ಕಂತಹ ವಿಷಯಗಳಲ್ಲಿ 
ಅವುಗಳಿಗೆ ತಕ್ಕಂತೆ ನಡೆದುಕೊಳ್ಳುವ ಬುದ್ಧಿ ಇಲ್ಲದಿದ್ದರೆ ಹೇಗೆ ಕೆಲಸಗಳನ್ನು ಮಾಡುವುದು? ಆದ್ದರಿಂದಲೇ ನಮ್ಮ ಪೂರ್ವಿಕರು ಅರ್ಥಶಾಸ್ತ್ರ, 
ದಂಡನೀತಿಯಂಥವುಗಳನ್ನು ಹೇಳಿಕೊಟ್ಟಿದ್ದಾರೆ, ಇಲ್ಲಿ ಮೇಲೆ ಹೇಳಿದ ಲೌಕಿಕ ವಿದ್ಯೆ ಮಾತ್ರ ನಮಗೆ ಸಾಕಾಗುವುದಿಲ್ಲ. ನಾವು ಕೆಲವೊಮ್ಮೆ 
ನೋಡುತ್ತೇವೆ. ನಾವು ಮಾಡುವುದು ಒಂದೂ ನಡೆಯುವುದಿಲ್ಲ. ನಾವು ಮಾಡುವ ಪ್ರಯತ್ನಗಳು ಉತ್ತಮವಾದರೂ ಅವುಗಳು ಫಲಿಸುವುದಿಲ್ಲ. ಇದಕ್ಕೆ 
ಕಾರಣವೇನೆಂದರೆ `ದೈವಕೃಪೆ' ಇಲ್ಲ ಎನ್ನುವುದು. ಆ ಭಗವಂತನ ದಯೆ ಪಡೆಯಲು ನಾವು ಕೆಲವು ಅರ್ಚನೆ ಕೆಲಸಗಳನ್ನು ಮಾಡಬೇಕು. ದೇವಸ್ಥಾನಕ್ಕೆ 
ಹೋಗಿ ಪೂಜೆ, ಮಾಡುವುದು, ಮನೆಯಲ್ಲಿಯೇ ವಿಷ್ಣು ಸಹಸ್ರನಾಮವನ್ನೋ ಶಿವಸಹಸ್ರನಾಮವನ್ನೋ ಪಾರಾಯಣ ಮಾಡುವುದು ಇಂಥವುಗಳನ್ನೂ 
ಮಾಡಬಹುದು. ಹೆಚ್ಚು ಬುದ್ಧಿಯುಳ್ಳವರು ಕೆಲವರು ಸಂಧ್ಯಾವಂದನೆಯನ್ನೂ, ವಿಶೇಷವಾಗಿ ಹೋಮ ಮೊದಲಾದವುಗಳನ್ನೂ ಮಾಡುವರು. 
ಇವುಗಳೆಲ್ಲ ವೈದಿಕವಾದ ವಿದ್ಯೆ. ಇದು `ಇಹ ಪರ' ವಾದ ಸುಖ-ಶಾಂತಿಗಳನ್ನು ಕೊಡುವುದಾಗಿ ಇರುವುದು. ಲೌಕಿಕವಿದ್ಯೆ ಕೇವಲ ಇಹ ಫಲ ನೀಡತಕ್ಕದ್ದು.

ಆಧ್ಯಾತ್ಮಿಕ ವಿದ್ಯೆಯೆನ್ನುವುದು `ನಾವು ಇಷ್ಟೆಲ್ಲಾ ಮಾಡುತ್ತಿದ್ದೇವಲ್ಲಾ, ನಾವು ಎಷ್ಟು ದಿನಗಳು ಇರುವೆವು? ನಾವು ಶಾಶ್ವತವೇ? ಅಥವಾ ಶಾಶ್ವತವಲ್ಲವೇ? ಶಾಶ್ವತವಾದರೆ 
ನಾವು ಸಾಯುವುದಾದರೂ ಏಕೆ? ಸತ್ತಮೇಲೆ ನಮ್ಮ ರೂಪ ಹೇಗಿರುತ್ತದೆ? ಹುಟ್ಟು ಸಾವು ಇಲ್ಲದೆ ನಾವು ಶಾಶ್ವತವಾಗಿರುವುದಕ್ಕೆ ಯಾವ 
ತತ್ತ್ವವನ್ನು ತಿಳಿಯಬೇಕು?' ಇವೆಲ್ಲ ಸೇರುವಂತಹುದು. ಇವುಗಳೆಲ್ಲವನ್ನೂ ಪಡೆಯುವುದಕ್ಕೆ ಮನುಷ್ಯನ ಜನ್ಮ ಉಪಯೋಗವಾದುದರಿಂದ ಮನುಷ್ಯ 
ಜನ್ಮ ಶ್ರೇಷ್ಠವಾದುದು. ಆದರೆ ನಾವು ವಿದ್ಯೆ ಕಲಿಯಬೇಕು ಎನ್ನುವುದು ಬಹಳ ಮುಖ್ಯವಾದುದು. ನಮ್ಮ ಹತ್ತಿರ ಬಹಳವಾಗಿ ಔಷಧಗಳು ಇರಬಹುದು. 
ಇವುಗಳಲ್ಲಿ ಯಾವುದಕ್ಕೆ ಯಾವುದನ್ನು ಉಪಯೋಗಿಸಿದರೆ ಫಲ ಶ್ರೇಷ್ಠವಾಗಿರುತ್ತದೆ ಎನ್ನುವ ಯೋಚನೆ ಬೇಕು. ಅದನ್ನು ಬೇರೆ ಕೆಲಸಕ್ಕೆ ಉಪಯೋಗಿಸಿದರೆ 
ಅಸಹಾಯವಾಗಬಹುದು. ಅದೇ ರೀತಿ ವಿದ್ಯೆಯ ವಿಷಯವೂ ಕೂಡ. ಅಲ್ಲದೆ, ವಿದ್ಯೆ ಕಲಿತವನಿಗೆ `ವಿನಯ' ಎಂದು ಹೇಳಲಾಗುವ ಸಂಯಮ ಅಥವಾ 
ಗುಣ ಇರಬೇಕು. `ಶೀಲ' 'ಎನ್ನುವುದು ಬೇಕು, ಇವೆರಡೂ ಇಲ್ಲದೆ ವಿದ್ಯೆ ವಿದ್ಯೆಯಲ್ಲ ಎಂದು ಒಬ್ಬರು ಹೇಳುತ್ತಾರೆ.

ವಿದ್ಯೆಯಿಂದ ನಾವು ಪಡೆದಿರಬೇಕಾದುವು ಯಾವವು ಎನ್ನುವುದಕ್ಕೆ ಗೌತಮ ಮಹರ್ಷಿಗಳು `ದಯಾ ಸರ್ವಭೂತೇಷು, ಕ್ಷಾಂತಿಃ', ಅನಸೂಯ, ಶೌಚಂ, 
ಅನಾಯಸಃ, ಮಂಗಲಂ, ಅಕಾರ್ಪಣ್ಯಂ, ಅಸ್ಪೃಹಾ, ಎಂದು ಹೇಳುತ್ತಾರೆ.

ಭಗವಂತನು ವೈಕುಂಠದಿಂದ ಕೆಳಗಿಳಿದು ಹಲವು ವಿಧವಾದ ಅವತಾರಗಳನ್ನು ಮಾಡಿದುದು ಅವನಿಗೆ ಸಂಬಂಧಿಸಿದಂತೆ ಅವಶ್ಯಕವಿಲ್ಲದಿದ್ದರೂ, ಶ್ರಮಪಟ್ಟು ಜನರಿಗೆ 
ಒಳ್ಳೆಯದನ್ನು ಮಾಡಬೇಕೆಂಬ ಒಂದೇ ವಿಚಾರದಿಂದ ಕೂಡಿದವನಾಗಿ ಕರುಣೆಯಿಂದ ಅವತಾರ ಮಾಡಿದನು. ಭಗವಂತನು ನಮಗೆ ಇತರರು ಕಷ್ಟ 
ಕೊಡುವಾಗ ಅದನ್ನು ದೂರಮಾಡುವ ಶಕ್ತಿಯನ್ನು ಕೊಟ್ಟಿದ್ದಾನೆ. ನಾವು ಆ ಶಕ್ತಿಯನ್ನು ನಮಗೆ `ದಯೆ' ಅಥವಾ `ಕರುಣೆ' ಇದ್ದರೆ ಮಾತ್ರ ಉಪಯೋಗಿಸುತ್ತೇವೆ. ದಯೆ 
ಎಂದರೇನು? ಇತರರು ಕಷ್ಟ ಕೊಡುವಾಗ ಅದನ್ನು ದೂರ ಮಾಡಬೇಕೆಂಬ ವಿಚಾರ ಉಂಟಾದರೆ ಅದೇ ದಯೆ. ಬೇರೆ ರೀತಿಯಲ್ಲಿದ್ದರೆ ಅವನನ್ನು `ದಯೆ ಇಲ್ಲದವನು' ಎಂದು 
ಹೇಳುತ್ತಾರೆ. ಮನುಷ್ಯನ ಶ್ರೇಷ್ಠವಾದ ಗುಣ ದಯೆ. ಹಾಗಿರುವ ದಯೆ ಎನ್ನುವ ಕರುಣೆಯನ್ನು ನಮ್ಮಲ್ಲಿ ವೃದ್ಧಿಪಡಿಸಿಕೊಳ್ಳಬೇಕು. ಕೆಲವರಿಗೆ ಸಹಜವಾಗಿಯೇ 
ವಿಶೇಷವಾಗಿ ಕರುಣೆ ಇರುತ್ತದೆ. ಕೆಲವರಿಗೆ ಒಳ್ಳೆಯವರ ಸಹವಾಸದಿಂದ ಅವರಂತೆ ತಾವೂ ಇರಬೇಕೆನ್ನುವ ಭಾವನೆ ಉಂಟಾಗಿರುವುದರಿಂದ ದಯೆ 
ಉಂಟಾಗುತ್ತದೆ. ಆದ್ದರಿಂದ ನಾವು ದಯೆಯನ್ನು ಹೆಚ್ಚಿಸಿಕೊಳ್ಳಬೇಕು. ಶಾಲೆಯಲ್ಲಿ ಓದಿನಲ್ಲಿ ಹಿಂದಿರುವ ಹುಡುಗರ ಮೇಲೆ ದಯೆತೋರಿ, ತನಗೆ ತಿಳಿದಷ್ಟು 
ಅವರಿಗೆ ಹೇಳಿಕೊಟ್ಟು ಅವರನ್ನು ಮುಂದಕ್ಕೆ ತರಬೇಕಾದುದು ಒಂದು ಶ್ರೇಷ್ಠವಾದ ಗುಣವಾಗುತ್ತದೆ. ಅದು ಕರುಣೆಯಾಗುತ್ತದೆ. ಅದನ್ನು ನೀವು ಕಲಿತು ಬಾಳಿನಲ್ಲಿ 
ಆಚರಣೆಗೆ ತರವೆವೂ. ಶಾಲೆಯ ವಿದ್ಯಾರ್ಥಿಗಳ ವಿಷಯವಾಗಿಯೇ ಅಲ್ಲದೆ `ಸರ್ವಭೂತೇಷು' ಎಂದು ಹೇಳಿರುವುದರಿಂದ ಇತರರ ವಿಷಯವಾಗಿಯೂ ಅದೇ ರೀತಿ ಕರುಣೆಯುಳ್ಳವರಾಗಿ ಇರಬೇಕು. 

ಆನಂತರ `ಕ್ಷಾಂತಿಃ' ಎನ್ನುವ ಗುಣ. ತನಗೆ ಹಿಡಿಸದ ಮಾತುಗಳನ್ನು ಯಾರಾದರೂ ಆಡಿದರೆ, ತನಗೆ ಹಿಡಿಸದ ಕೆಲಸವನ್ನು ಯಾರಾದರೂ 
ಮಾಡಿದರೆ, ಮನುಷ್ಯನು ಅಂಥಹವರನ್ನು ಹೊಡೆಯಲು ಹೋಗುತ್ತಾನೆ. ಇದಕ್ಕೆ ಶಕ್ತಿ ಇಲ್ಲದೆ ಹೋದರೆ ಬೇರೆ ಕೆಲಸಗಳನ್ನು ಮಾಡುತ್ತಾನೆ. 
ತಾನೇ ಎದುರಿಗೆ ನಿಲ್ಲದೆ ಮರೆಯಲ್ಲಿದ್ದು ಮಾಡುತ್ತಾನೆ. `ಅವನು ಮಾಡಿದ, ಆದ್ದರಿಂದ ನಾನೂ ಮಾಡುತ್ತೇನೆ' ಎಂದು ಜವಾಬು ಹೇಳುತ್ತಾನೆ. ಇದು ಒಳ್ಳೆಯ 
ಮನುಷ್ಯನ ಲಕ್ಷಣವಲ್ಲ. ತನ್ನ ಬಗ್ಗೆ ಯಾರಾದರೂ ಕ್ಷುದ್ರವಾಗಿ ಮಾತನಾಡಿದರೆ ಕ್ಷುದ್ರವಾಗಿ ಏನಾದರೂ ಮಾಡಿದರೆ, ಒಳ್ಳೆಯ ಮನುಷ್ಯನಾದವನು ಇದರಿಂದ 
ಏನಾಗುತ್ತದೆ? ಅವನಿಗೆ ನಾಳೆ ಬುದ್ಧಿ ಬರುತ್ತದೆ ಎಂದುಕೊಂಡು ಕ್ಷಮಿಸಿಬಿಡುತ್ತಾನೆ. ಕ್ಷುದ್ರವಾಗಿ ನಡೆದುಕೊಂಡವನು ಇವನ ಒಳ್ಳೆಯತನವನ್ನು ನೋಡಿ 
ಕ್ಷಮೆ ಕೇಳಿಕೊಳ್ಳುವ ಸ್ಥಿತಿಯೂ ಬಂದುಬಿಡುತ್ತದೆ. ಆದ್ದರಿಂದ ನಾವು ಕ್ಷಾಂತಿಃ ಎಂದು ಹೇಳಲ್ಪಟ್ಟ ಈ ಗುಣವನ್ನು ಬೆಳಸಿಕೊಳ್ಳಬೇಕು.

ಆನಂತರದ ಗುಣ `ಅನಸೂಯ' ಎನ್ನುವುದು. ಮನುಷ್ಯನು ತಾನು ಗುಣವಂತನಾಗಿ ಇಲ್ಲದಿದ್ದರೂ ಇನ್ನೊಬ್ಬ ಗುಣವಂತನನ್ನು ಕಂಡು ಅವನ ಗುಣಗಳ ಬಗ್ಗೆ ಸರಿಯಾಗಿ 
ಯೋಚಿಸದೆ ಅದರಲ್ಲಿ ಕೊರತೆಯನ್ನು ನೋಡಿ ಅದನ್ನು ಬಹಿರಂಗಪಡಿಸುವುದು ಸಹಜ. ಒಬ್ಬನು ಚೆನ್ನಾಗಿ ಓದುತ್ತಾನೆ ಎಂದರೆ ಇನ್ನೊಬ್ಬ ಓದದೆ ಇರುವ 
ಹುಡುಗ, ಇವನಿಗೆ ಅನೇಕ ಸಲ ಓದಿದರೇನೆ ಬರುತ್ತೆ. ಆದ್ದರಿಂದಲೇ ಓದುತ್ತಾನೆ. ಇಲ್ಲದೆ ಇದ್ದರೆ ನಾನೂ ಓದುತ್ತೇನೆ. ಇನ್ನು ಮೇಲೆ ಓದುತ್ತೇನೆ ಎನ್ನುತ್ತಾನೆ. 
ಆದರೆ ಅವನು ಹೇಳುವಂತೆ ಮಾಡುವುದಿಲ್ಲ. ಓದಿನಲ್ಲಿ ತನ್ನ ಸೋಮಾರಿತನವನ್ನು ಮರೆಸುವುದಕ್ಕಾಗಿ ಇನ್ನೊಬ್ಬ ಚೆನ್ನಾಗಿ ಓದುವುದನ್ನೇ ಕೊರತೆಯಾಗಿ ಎಣಿಸಿ `ಅವನು 
ರಾತ್ರಿ ಹಗಲು ಓದಿದರೂ ಪ್ರಯೋಜನವಿಲ್ಲ' ಎನ್ನುತ್ತಾನೆ. ಯಾವ ಗುಣವಿದೆಯೋ ಅದನ್ನೇ ನಾವು ಹೇಳಬೇಕೇ ವಿನಹ ಸುಮ್ಮನೆ `ದೋಷಾರೋಪಣೆ' (ವ್ಯರ್ಥವಾಗಿ ತಪ್ಪು ಪ್ರಕಟಿಸುವುದು) ಮಾಡಬಾರದು.

ಅನಂತರದ ಗುಣ `ಶೌಚಮ್' ಎನ್ನಲಾಗುವ ಶುದ್ಧತೆ. ಮನುಷ್ಯನಿಗೆ ಮುಖ್ಯವಾದುದು ಶೌಚವೆನ್ನುವ ದೊಡ್ಡಗುಣ. ಶೌಚವೆಂದರೇನು? ನಮ್ಮ ಎದುರಿಗೆ 
ಒಬ್ಬನು ಬಾಯಲ್ಲಿ ಬೆಟ್ಟು ಇಟ್ಟುಕೊಂಡು ಅನಂತರ ನಮ್ಮನ್ನು ಮುಟ್ಟಿದರೆ ನಮಗೆ ಬೇಸರವಾಗುತ್ತದೆ, `ಅವನು ನಮ್ಮನ್ನು ಮುಟ್ಟಿದನಲ್ಲಾ' 
ಎಂದು ನಮಗೆ ತೋರಿದರೂ ಅದು ಅವನಿಗೆ ಸಹಜವಾಗಿರುತ್ತದೆ. ಇದೇ ರೀತಿ ತಿಂಗಳುಗಟ್ಟಲೆ ಸ್ನಾನ ಮಾಡದೆ ಇರುವವನು ಹತ್ತಿರಕ್ಕೆ ಬಂದರೆ ನಾವು 
ಮೂಗು ಮುಚ್ಚಿಕೊಳ್ಳಬೇಕಾಗಬಹುದು, ಆದರೆ ಅಭ್ಯಾಸವಾಗಿರುವ ಅವನಿಗೆ ಏನೂ ತಿಳಿಯುವುದಿಲ್ಲ. ಕೆಲವರು ಬಟ್ಟೆಗಳನ್ನು ಒಗೆದು ಉಟ್ಟುಕೊಳ್ಳುವುದು 
ಸಂಪ್ರದಾಯವಿದೆ (!!) ಎಂದು ಹೇಳಿ ತಿಂಗಳಿಗೋ ವರ್ಷಕ್ಕೋ ಒಮ್ಮೆ ಬಟ್ಟೆಗಳನ್ನು ಒಗೆದರೆ, ಆ ಬಟ್ಟೆಗಳನ್ನು ನೋಡಿದರೇನೆ ಸಂಕಟವಾಗುತ್ತದೆ, ನಾವು ಎಂಜಲು 
ಮಾಡಬಾರದು, ಎಂಜಲು ಮೂಲಕ ಒಬ್ಬರಿಂದ ಇನ್ನೊಬ್ಬರಿಗೆ ವ್ಯಾಧಿಗಳು ಉಂಟಾಗುತ್ತವೆ. ಪ್ರತಿದಿನವೂ ಸ್ನಾನಮಾಡಿ ಒಗೆದ ಬಟ್ಟೆಗಳನ್ನು ಉಟ್ಟುಕೊಳ್ಳಬೇಕು. ಇದೇ ಶೌಚ.

`ಅನಾಯಾಸಃ' ಅಥವಾ ಸೋಮಾರಿತನವಿಲ್ಲದಿರುವಿಕೆ ಎನ್ನುವುದು. ಇದಾದ ಮೇಲೆ ಕೆಲವರು `ಬೆಳಗ್ಗೆ' ಓದುವುದಕ್ಕೆ ಹೇಳಿದರೆ `ಬಹಳಕಷ್ಟ' ಎನ್ನುತ್ತಾರೆ. 
`ಮಧ್ಯಾಹ್ನ' ಎಂದರೆ `ನನಗೇನೂ ಆಗಲೇ ಇಲ್ಲ' ಎನ್ನುತ್ತಾರೆ. ಚಿಕ್ಕಕೆಲಸವಾದರೂ, ದೊಡ್ಡ ಕೆಲಸವಾದರೂ `ಬಹಳ ಕಷ್ಟ' ಎನ್ನುತ್ತಾರೆ. ನಿಜಕ್ಕೂ `ಕಷ್ಟ' ಎನ್ನುವುದು ಒಂದೂ ಇಲ್ಲ. `ನಾವು 
ಮಾಡಬಹುದು' ಎನ್ನುವ ನಿಶ್ಚಯಕ್ಕೆ ಬಂದು, ಉತ್ಸಾಹದಿಂದ ತೊಡಗುವುದು ಮನುಷ್ಯನ ಕರ್ತವ್ಯ. ಆದ್ದರಿಂದ ಆಯಾಸಕ್ಕೆ  (ಸೋಮಾರಿತನಕ್ಕೆ) ಅವಕಾಶ ಕೊಡಕೂಡದು.

`ಮಂಗಲ' ಎನ್ನುವುದು ಇದಾದ ಮೇಲಿನ ಗುಣ. ಕೆಲವರು ಯಾವಾಗಲೂ ಕೆಳಮಟ್ಟದ ಶಬ್ದಗಳನ್ನೇ ಬಳಸುತ್ತಾರೆ. ನಾವು ಯಾರಾದರೂ ದೊಡ್ಡವರೊಡನೆ ಮಾತನಾಡುವಾಗ 
ಅವರಿಗೆ ತಕ್ಕಂತೆ, ವಯಸ್ಸಾಗಿರುವವರಾದರೆ ``ಬನ್ನಿ" ಎನ್ನುವುದಕ್ಕೆ ಬದಲು ``ದಯಮಾಡಿ" ಎಂದರೆ, ಅವರಿಗೆ ಬಹಳ ಸಂತೋಷವಾಗಿ ನಮ್ಮ ಮೇಲೆ ಪ್ರೀತಿ ಹೆಚ್ಚಾಗುತ್ತದೆ. 
`ಆಸನವನ್ನು ಅಲಂಕರಿಸಿ' ಎಂದರೆ ಮಂಗಳವಾದ ಮಾತು. `ಅಲ್ಲಿ ಹೋಗಿ ಕೂತುಕೋ' ಎಂದರೆ ಸಾಧಾರಣವಾದ ಮಾತು. ಮಂಗಳವಾದ ಮುಖಭಾವವೂ ಇರಬೇಕು. ಇದನ್ನು ನಾವು ಕಲಿಯಬೇಕು.

ಅನಂತರ `ಅಕಾರ್ಪಣ್ಯ'. ಮನುಷ್ಯನಲ್ಲಿ ಕಾಮ, ಕ್ರೋಧ, ಲೋಭ ಮುಂತಾದ ಕೆಟ್ಟ ಗುಣಗಳು ಇರುತ್ತವೆ. ಒಬ್ಬೊಬ್ಬರಿಗೆ ಶಕ್ತಿ ಇಲ್ಲದೆ ಹೋಗುವುದರಿಂದ 
ಕಾಮ-ಕ್ರೋಧಗಳು ತಾವಾಗಿಯೇ ಬರದೆ ಇದ್ದರೂ ಲೋಭವನ್ನು ಗೆಲ್ಲುವುದು ಸ್ವಲ್ಪ ಕಠಿಣವಾಗುತ್ತದೆ. ಎಷ್ಟು ವಯಸ್ಸಾದರೂ ಇನ್ನೂ ಹಣವನ್ನು 
ಸೇರಿಸಬೇಕೆಂಬ ವಿಚಾರ ಲೋಭವಾಗುತ್ತದೆ. ಜೇನು ನೊಣಗಳು ಎಷ್ಟು ಸೇರಿಸಿದರೂ ಯಾರೋ ಜೇನುಗೂಡಿನಿಂದ ಜೇನುತುಪ್ಪವನ್ನೆಲ್ಲ 
ತೆಗೆದುಕೊಂಡು ಹೋಗುವ ಹಾಗೆ. ನಾವು ಏಕೆ ಸೇರಿಸಿಟ್ಟ ಅದನ್ನು ಯಾರೋ ತೆಗೆದುಕೊಂಡು ಹೋಗುವಂತೆ ಮಾಡಬೇಕು? ಒಬ್ಬ ಬಡವ 
ಬೇಕಾಗಿದೆಯೆಂದು ಕೇಳಿದರೆ ಅದು ಸರಿಯೆಂದು ನಮಗೆ ತೋರಿದರೆ ಸಾಧ್ಯವಾದ ಮಟ್ಟಿಗೆ ಸಹಾಯ ಮಾಡಬಹುದೇ ವಿನಹ ಲೋಭಿಯಾಗಬಾರದು. ಇದೇ `ಅಕಾರ್ಪಣ್ಯ' ವೆಂದು ಹೇಳಲ್ಪಟ್ಟಿದೆ.

ಅದಾದ ಮೇಲೆ `ಅಸ್ಪೃಹಾ' ಆಸೆಗಳು ಹೆಚ್ಚಾಗುತ್ತಾ ಹೆಚ್ಚಾಗುತ್ತಾ ತಪ್ಪುಗಳನ್ನು ಮಾಡುವನು. ಹೊಟ್ಟೆಗೆ ಇಲ್ಲದವನು ಕಳ್ಳತನ ಮಾಡುವುದಕ್ಕಿಂತಲೂ 
ಹಣವುಳ್ಳವನು ಕಳ್ಳತನ ಮಾಡುವುದೇ ಹೆಚ್ಚು. ಏಕೆಂದರೆ ಅವನಿಗೆ ಆಸೆ ಹೆಚ್ಚು. ಆಸೆಗಳನ್ನು ಒಂದು ಮಿತಿಯಲ್ಲಿಟ್ಟುಕೊಂಡು ನಾವು ವ್ಯವಹರಿಸಿದರೆ ನಾವು 
ದೋಷರಹಿತರಾಗುತ್ತೇವೆ. ಎಲ್ಲ ಪಾಪಗಳಿಗೂ ಮೂಲ ಸ್ಥಾನವಾದ ಆಸೆಯನ್ನು ನಾವು ನಿಗ್ರಹಿಸಿ ನಾವು ಶ್ರೇಯಸ್ಸನ್ನು ಪಡೆಯಬೇಕು.

ವಿದ್ಯೆಯಿಂದ ನಾವು ಈ ಎಂಟು ಗುಣಗಳನ್ನು ಖಂಡಿತವಾಗಿಯೂ ಪಡೆಯಬೇಕು. ವಿದ್ಯೆ ಕಲಿತೂ ಈ ಗುಣಗಳನ್ನು ನಾವು ಪಡೆಯದೆ ಹೋದರೆ ಕತ್ತಿ ಅಥವಾ 
ಚಾಕುವನ್ನು ಇಟ್ಟುಕೊಂಡು ತರಕಾರಿಯನ್ನು ಕತ್ತರಿಸದೆ ನಮ್ಮ ಶರೀರದ ಅಂಗಗಳನ್ನೇ ಕತ್ತರಿಸಿಕೊಂಡಂತೆಯೇ ಸರಿ. ಈ ರೀತಿ ನಾವು ನಮ್ಮ ಬಾಳನ್ನು 
ಹಾಳು ಮಾಡಿಕೊಳ್ಳಬಾರದು. ಆದ್ದರಿಂದ ಆ ನಿಗ್ರಹ, ಗುಣ ಮೊದಲಾದವುಗಳನ್ನು ನಮ್ಮ ದೃಢನಿಶ್ಚಯವಾಗಿ ಸ್ವೀಕರಿಸಿ ವಿದ್ಯಾವಂತರಾಗಿದ್ದರೆ ಮಾತ್ರ 
ದೇಶದ ಒಳ್ಳೆಯ ನಾಗರೀಕರಾಗಿರಬಲ್ಲೆವು. ಇತರರಿಗೂ ದಾರಿ ತೋರಿಸಬಹುದು, ಶ್ರೇಯಸ್ಸು ಪಡೆಯಬಹುದು.

ನೀವೆಲ್ಲರೂ ಗುಣಶೀಲಗಳನ್ನು ಪಡೆದು ಬಾಳಬೇಕೆಂದು ನಾವು ಪ್ರಾರ್ಥಿಸುತ್ತೇವೆ. ಈಶ್ವರನು ನಿಮಗೆಲ್ಲರಿಗೂ ಒಳ್ಳೆಯದನ್ನು ಮಾಡಲಿ. 

 

