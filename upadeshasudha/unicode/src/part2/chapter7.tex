\chapter{ಮನಶ್ಶಾಂತಿಗಾಗಿ ಆಹಾರ}

\begin{shloka}
ಶ್ರುತಿಶಾಖಿ ಪಲ್ಲವಾಭ್ಯಾಂ\\
ಸ್ಮೃತಿಮಾತ್ರಪ್ರಾಪಿತಾರ್ಥಿಸೌಖ್ಯಾಭ್ಯಾಮ್|\\
ಗತಿಜಿತ ಮತ್ತಗಜಾಭ್ಯಾಂ\\
ನತಿರಸ್ತ್ವೇಷಾ ಗಣೇಶ ಚರಣಾಭ್ಯಾಮ್||
\end{shloka}

ಮನುಷ್ಯನಿಗೆ ಆಹಾರ ನಿದ್ರೆಗಳು ಅವಶ್ಯಕವಾದುವು. ಯಾವ ನಿಯಮಕ್ಕೂ ಒಳಗಾಗದೆ ತನ್ನ ಇಷ್ಟದಂತೆ ನಡೆದುಕೊಂಡರೆ ಹಾಗೆ ಮಾಡುವುದು ಅಧರ್ಮವಾಗುತ್ತದೆ. ಆಹಾರದಲ್ಲಿ ಯಾವುವು ನಿಯಮಗಳು? ಯಾವುದಕ್ಕೆ ಆ ನಿಯಮಗಳು ಎನ್ನುವುದನ್ನು ಕುರಿತು ಈಗ ನೋಡೋಣ.

ಸಾಮಾನ್ಯವಾಗಿ ನಾವು ಏಕೆ ಊಟಮಾಡುತ್ತೇವೆ? ಹಸಿವನ್ನು ದೂರಮಾಡಲು ಊಟಮಾಡುತ್ತೇವೆ. ಆದರೆ ಆ ಒಂದೇ ಲಕ್ಷ್ಯಕ್ಕಾಗಿ ಮಾತ್ರವೇ ಅಥವಾ ಬೇರೆ ಯಾವುದಾದರೂ ಲಕ್ಷ್ಯಕ್ಕಾಗಿಯೇ ಎಂದು ನಾವು ಯೋಚಿಸಬೇಕು. ಭಗವಂತನು ನಮಗೆ ಈ ಶರೀರವನ್ನು ಕೊಟ್ಟಿದ್ದಾನೆ. ನಮ್ಮ ಪೂರ್ವಪುಣ್ಯದ ಕಾರಣವಾಗಿ ನಮಗೆ ಈಗ ಮನುಷ್ಯಜನ್ಮ ದೊರಕಿದೆ. ಈ ಮನುಷ್ಯಜನ್ಮದಲ್ಲಿ ನಾವು ಮಾಡಬೇಕಾದುದನ್ನು ಹೇಳುವಾಗ ನಮ್ಮ ಅಂತಃಕರಣವನ್ನು ಶುದ್ಧವಾಗಿಟ್ಟುಕೊಳ್ಳಬೇಕೆಂದು ಶಾಸ್ತ್ರಗಳು ಘೋಷಿಸುತ್ತವೆ.

ಅಂತಃಕರಣ ಯಾವಾಗ ಶುದ್ಧಿಯಾಗಿರುತ್ತದೆ?

\begin{shloka}
`ಅನ್ನಮಯಂ ಹಿ ಸೌಮ್ಯ ಮನಃ\\
(ಮನಸ್ಸು ಅನ್ನದಿಂದ ಕೂಡಿರುತ್ತದೆ)
\end{shloka}

-ಎಂದು ಒಂದು ಕಡೆ ಹೇಳಲ್ಪಟ್ಟಿದೆ. ನಾವು ತಿನ್ನುವ ಆಹಾರಕ್ಕೆ ತಕ್ಕ ಮನಸ್ಸು, ಗುಣಗಳನ್ನು ಒಬ್ಬನಲ್ಲಿ ಇರುವುದನ್ನು ನಾವು ಅನುಭವದಲ್ಲಿ ನೋಡಬಹುದು. ಮತ್ತೆ ಮತ್ತೆ ಕೋಪಗೊಳ್ಳುವವರನ್ನು ನೋಡಿ, `ಅವರು ಕೋಪ ಪೂರ್ತಿಯಾಗುವಂತೆ ಊಟಮಾಡುತ್ತಾರೆಯೇ?' ಎಂದು ಕೇಳುವುದನ್ನು ನೋಡುತ್ತೇವೆ. ಅವರೂ ಅದನ್ನು ಒಪ್ಪಿಕೊಳ್ಳುವುದುಂಟು. ಆದ್ದರಿಂದ ಒಬ್ಬನು ಬಹಳ ಕೋಪಿಷ್ಠನಾಗಿದ್ದರೆ, ಅವನು ಉಪ್ಪು, ಖಾರ ಮುಂತಾದವುಗಳನ್ನು ಹೆಚ್ಚಾಗಿ ತಿನ್ನುತ್ತಾನೆಂದು ತಿಳಿಯುತ್ತದೆ. ಇನ್ನೊಂದು ವಿಧವಾದ ಆಹಾರವೂ ಇದೆ, ಅದು ತಾಮಸವಾದ ಆಹಾರವೆನ್ನಲಾಗುವುದು. ಇಂಥ ಆಹಾರವನ್ನು ತಿನ್ನುವವನಿಗೆ ತಿಂದ ಹತ್ತು ನಿಮಿಷಗಳಲ್ಲಿಯೇ, ತಾನು ಈ ಪ್ರಪಂಚದಲ್ಲಿ ಇದ್ದಾನೋ ಇಲ್ಲವೋ ತಿಳಿಯುವುದಿಲ್ಲ. ತನ್ನ ಬಾಯಿಗೆ ಬಂದಂತೆ ಏನೋ ಮಾತನಾಡುತ್ತಾನೆ. ಅಥವಾ ಮನುಷ್ಯನೇ ಇವನು ಎನ್ನುವಂತೆ ಪ್ರಜ್ಞೆ ಇಲ್ಲದೆ ಇರುತ್ತಾನೆ.

ಇನ್ನೊಂದು ವಿಧವಾದ ಆಹಾರವಿದೆ. ತಿಂದೊಡನೆ ಶರೀರದ ಬಲ, ಮನಸ್ಸಿನ ತೃಪ್ತಿ, ಚಿಂತಿಸುವ ರೀತಿ ಇವುಗಳು ಹೆಚ್ಚಾಗುತ್ತವೆ.

\begin{shloka}
`ಅನ್ನಮಯಂ ಹಿ ಸೌಮ್ಯಮನಃ'
\end{shloka}

-ಎಂದು ಶಾಸ್ತ್ರದಲ್ಲಿ ಹೇಳಿರುವುದು ತಪ್ಪಾಗುವುದಿಲ್ಲ. `ಆಹಾರ' ವೆನ್ನುವುದು ಪವಿತ್ರವಾಗಿದ್ದರೆ ನಮ್ಮ ಮನಸ್ಸಿನಲ್ಲೂ ಪವಿತ್ರವಾದ ವಿಷಯಗಳೇ ಇರುತ್ತವೆಂದು ತಿಳಿಯುತ್ತದೆ. ಹಾಗಲ್ಲದೆ ಆಹಾರವೆನ್ನುವುದು ಅಶಾಂತಿಕರವಾಗಿದ್ದರೆ ನಮ್ಮ ಮನಸ್ಸುಕೂಡ ಅದೇ ರೀತಿ ಅಶಾಂತಿಕರವಾಗಿರುತ್ತದೆ. ಅದನ್ನು ಸ್ಪಷ್ಟಪಡಿಸಲು ಒಂದು ಕಥೆ ಹೇಳುವುದುಂಟು. ಒಬ್ಬ ದೊಡ್ಡ ಯೋಗೀಶ್ವರರು ಒಂದು ಆಶ್ರಮ ಏರ್ಪಡಿಸಿಕೊಂಡು ಅಲ್ಲಿ ವಾಸವಾಗಿದ್ದರು. ದಿನವೂ ಭಿಕ್ಷೆಗೆ ಹೋಗಿ ತಮಗೆ ದೊರೆಯುವ ಆಹಾರವನ್ನು ಸೇವಿಸುವ ಅಭ್ಯಾಸವನ್ನು ಇಟ್ಟುಕೊಂಡಿದ್ದರು. ಅವರ ಮನಸ್ಸು ಶಾಂತಿಯಿಂದ ಕೂಡಿದ್ದಿತು. ಯಾವ ತಪ್ಪುಕೆಲಸಕ್ಕೂ ಒಳಗಾಗದೆ ಬಾಳುತ್ತಿದ್ದರು. ಆ ದೇಶದ ಅರಸನಿಗೆ ಅವರ ವಿಷಯವಾಗಿ ಬಹು ಗೌರವವಿದ್ದಿತು. ಅರಸನು ಒಂದು ದಿನ ಅವರನ್ನು ತನ್ನ ಅರಮನೆಗೆ ಭೋಜನಕ್ಕಾಗಿ ಆಹ್ವಾನಿಸಿದನು. ಅವರಾದರೋ `ಅಯ್ಯಾ ನಾನು ಪ್ರತಿದಿನವೂ ಭಿಕ್ಷೆ ಬೇಡಿ ಊಟ ಮಾಡುತ್ತೇನೆ. ನಿನ್ನ ಮನೆಗೆ ಬಂದು ನಾನು ಊಟ ಮಾಡಿದರೆ ನನ್ನ ಮನಸ್ಸು ಏನಾಗುತ್ತದೋ ಗೊತ್ತಿಲ್ಲ. ಆದ್ದರಿಂದ ನಾನು ನಿನ್ನ ಮನೆಯಲ್ಲಿ ಊಟಮಾಡುವುದು ಅಷ್ಟು  ಒಳ್ಳೆಯದಲ್ಲ' ಎಂದರು. ಆದರೂ ಅರಸನು ಬಲವಂತ ಮಾಡಿದ್ದರಿಂದ ಅವನ ಅರಮನೆಗೆ ಬಂದು ಕುಳಿತುಕೊಂಡರು. ಔತಣವಾದ ಮೇಲೆ ತಮ್ಮ ಕೈಯನ್ನು ತೊಳೆಯಲು ಹೋದಾಗ ಅಲ್ಲಿ ಒಂದು ಮುತ್ತಿನ ಸರವನ್ನು ನೋಡಿದರು. ಅಕ್ಕ ಪಕ್ಕದಲ್ಲಿ ಯಾರೂ ಇಲ್ಲದಿರುವುದನ್ನು ಕಂಡರು. ಕಮಂಡಲು ಮಾತ್ರ ಕೈಯಲ್ಲಿದ್ದಿತು. ಕಮಂಡಲುನಲ್ಲಿ ಮುತ್ತಿನ ಹಾರವನ್ನು ಇಟ್ಟುಕೊಂಡು ಅರಸನಿಗೆ ಆಶೀರ್ವಾದ ಹೇಳಿ ತಮ್ಮ ಆಶ್ರಮಕ್ಕೆ ಬಂದು ಬಿಟ್ಟರು. ಅದಾದ ಮೇಲೆ ಅವರಿಗೆ ಒಂದೇ ಯೋಚನೆ -`ನಾನು ಏತಕ್ಕೆ ಇಂಥಾ ಕೆಲಸ ಮಾಡಿದೆನು?' ಹೀಗಿರುವಾಗ ಅರಮನೆಯಲ್ಲಿ ಅರಸನ ಹೆಂಡತಿ ಅರಸನೊಡನೆ ತಾನು ಸ್ನಾನಾಗಾರದಲ್ಲಿ ಇಟ್ಟಿದ್ದ ಮುತ್ತಿನ ಸರ ಕಾಣದೆ ಇರುವುದನ್ನು ತಿಳಿಸಿದಳು. ಅರಸನು ಯೋಚಿಸಿ. ಅಲ್ಲಿರುವ ಕೆಲಸಗಾರರೇ ಯಾರಾದರೂ ಕಳ್ಳತನ ಮಾಡಿರಬಹುದೆಂದು ತೀರ್ಮಾನಿಸಿ ಏಟು, ಒದೆತಗಳ ಮೂಲಕ ಅವರಿಂದ ಸತ್ಯವನ್ನು ನುಡಿಸಲು ಪ್ರಯತ್ನಿಸಿದನು. ಎಲ್ಲರೂ, `ನಾನಲ್ಲ, ನಾನಲ್ಲ' ಎಂದು ಹೇಳಿ ಏಟು ತಿಂದು, ಆ ಯೋಗೀಶ್ವರರೇ ಒಳಗೆ ಹೋಗಿದ್ದರು, ಅವರೇ ತೆಗೆದುಕೊಂಡು ಹೋಗಿರಬೇಕು' ಎಂದು ಹೇಳುವುದನ್ನು ಕೇಳಿ ಅರಸನಿಗೆ ಮತ್ತಷ್ಟು ಕೋಪ ಉಂಟಾಯಿತು. ಪುನಃ ಹೊಡೆದನು. ಅವರು ಏಟಿನ ಮೇಲೆ ಏಟು ತಿನ್ನುತ್ತಿದ್ದರು.

ಆಶ್ರಮದಲ್ಲಾದರೂ ಯೋಗೀಶ್ವರರಿಗೆ ಒಂದೇ ಯೋಚನೆ. `ಇಂದು ನಾನು ಎಂಥಾ ಕೆಲಸ ಮಾಡಿಬಿಟ್ಟೆನು. ಮುತ್ತಿನ ಸರವನ್ನು ಕದ್ದು ತಂದಿದ್ದೇನೆ. ಇದನ್ನು ಯಾರಿಗೆ ಹಾಕಬೇಕು? ಇದನ್ನು ನಾನು ಹಾಕಿಕೊಂಡು ಹೋದರೆ ನಾಳೆ ಯಾರೂ ಭಿಕ್ಷೆ ಹಾಕುವುದಿಲ್ಲ. ಇಲ್ಲೇ ಇಟ್ಟು ಹೋದರೆ ಯಾರಾದರೂ ಕದ್ದುಕೊಂಡು ಹೋದರೆ ಏನು ಮಾಡುವುದು? ನಾನು ದೊಡ್ಡ ಅಪಾಯಕ್ಕೆ ಒಳಗಾದೆನಲ್ಲಾ! ಸರಿ, ಈ ಅಪಾಯವನ್ನು ಏತಕ್ಕೆ ತಂದುಕೊಂಡೆನು? ನನ್ನ ಮನಸ್ಸು ಕೆಟ್ಟುಹೋಯಿತು. ಇಷ್ಟು ದಿನಗಳು ಎಷ್ಟೋ ಸಲ ಎಷ್ಟೋ ಮುತ್ತಿನ ಸರಗಳನ್ನು ಕಂಡಿದ್ದರೂ ನನಗೆ ಒಂದೂ ದಿನವೂ ಆಸೆ ಉಂಟಾಗಿರಲಿಲ್ಲ. ಇಂದು ಈ ಆಸೆ ಉಂಟಾಗಿದೆಯೆಂದರೆ ಮನಸ್ಸು ಕೆಟ್ಟು ಹೋಗಿದೆ ಎನ್ನುವದು ನಿಶ್ಚಯ. ಯಾವುದಕ್ಕಾಗಿ ಮನಸ್ಸು ಕೆಟ್ಟಿರಬಹುದು' ಎಂದು ಯೋಚಿಸಿದರು. `ಈ ಅನ್ನ ಹೊಟ್ಟೆಯಲ್ಲೇ ಇರಬಾರದು' ಎಂದುಕೊಂಡು ಆ ಅನ್ನವನ್ನು ವಾಂತಿ ಮಾಡಿ ಬಿಟ್ಟರು. ಅನಂತರ ಕಮಂಡಲುವನ್ನು ತೆಗೆದುಕೊಂಡು ನೇರವಾಗಿ ಅರಸನ ಹತ್ತಿರಕ್ಕೆ ಬಂದರು. ಅಲ್ಲಿ ದೊಡ್ಡರಗಳೆ ಆಗುತ್ತಿರುವುದನ್ನು, ಕಂಡು ಯೋಗೀಶ್ವರರು ಅರಸನನ್ನು, `ಏನು ಇಷ್ಟುರಗಳೆ' ಎಂದು ಕೇಳಿದರು. ಅರಸನು ಅದಕ್ಕೆ `ಇಲ್ಲಿ ಕದ್ದವರು ಯಾರೂ ನಿಜವನ್ನು ಒಪ್ಪಿಕೊಳ್ಳುವುದಿಲ್ಲವೆನ್ನುತ್ತಾರೆ' ಎಂದನು. ಅದಕ್ಕೆ ಆ ಮುನಿ, `ಯಾರು ಕದ್ದಿದ್ದಾನೋ ಅವನೇ ತಾನೇ ಕಳ್ಳ, ನೀನು ಯಾರನ್ನೋ ಹಿಡಿದುಕೊಂಡು ಹೊಡೆದರೆ ಅವನು ಹೇಗೆ ಒಪ್ಪಿಕೊಳ್ಳುವನು. ಹಾಗೆ ಒಂದು ವೇಳೆ ಒಪ್ಪಿಕೊಂಡರೂ ಸರವನ್ನು ಹೇಗೆ ತಂದು ಕೊಡಲು ಸಾಧ್ಯವಾಗುವುದು? ಕಳ್ಳನು ಬೇರೆ ಯಾರೂ ಅಲ್ಲ' ಎಂದರು. ಅರಸನು, `ಸ್ವಾಮಿಗಳೇ, ನೀವು ಹೇಳುವುದು ಒಂದೂ ತಿಳಿಯಲಿಲ್ಲ' ಎಂದು ಹೇಳಲಾಗಿ ಮುನಿ, `ಈ ಸರವನ್ನು ತೆಗೆದುಕೊ, ಕಳ್ಳತನ ಮಾಡಿದ ನಾನೇ ನೇರವಾಗಿ ತಂದುಕೊಟ್ಟಿದ್ದೇನೆ' ಎಂದು ಸರವನ್ನು ಕೊಟ್ಟರು. ಅರಸನಿಗೆ ಬಹಳ ಆಶ್ಚರ್ಯವಾಯಿತು! `ಏನು ಮಹಾಪ್ರಭುಗಳೇ! ನೀವು ದೊಡ್ಡ ಮುನಿಗಳು, ನೀವು ಈ ಸರವನ್ನು ತೆಗೆದುಕೊಂಡು ಹೋಗಬಹುದೇ? ತೆಗೆದುಕೊಂಡು ಹೋಗಿದ್ದರೂ ಅದನ್ನು ಪುನಃ ತರಬಹುದೇ? ಏನೂ ತಿಳಿಯಲಿಲ್ಲವಲ್ಲಾ! ಬಹಳ ಆಶ್ಚರ್ಯವಾಗಿದೆಯಲ್ಲಾ!' ಎಂದನು. ಮುನಿ ತಕ್ಷಣವೇ, `ನಿನ್ನ ಮನೆಯ ಭಿಕ್ಷೆ ನನಗೆ ಬೇಡವೆಂದು ನಾನು ಮೊದಲೇ ಹೇಳಿದ್ದೆನು. ನೀನು ಬಲವಂತ ಮಾಡಿದೆ. ಅದರ ಫಲ ಇಷ್ಟುಮಂದಿಗೆ ಏಟು. ಅಷ್ಟೇ ಅಲ್ಲದೆ, ನನಗೆ `ಕಳ್ಳ'ನೆನ್ನುವ ಹೆಸರು ದೊರೆಯಿತು' ಎಂದರು. ಅರಸನು ಇನ್ನೂ ಸ್ಪಷ್ಟವಾಗಿ ಹೇಳಬೇಕೆಂದು ಕೇಳಲು ಮುನಿ, `ಅರಸನೇ! ನಿನ್ನ ಬೊಕ್ಕಸದಲ್ಲಿ ಇಷ್ಟೊಂದು ಅಕ್ಕಿಯನ್ನು ತಂದು ಇಟ್ಟಿದ್ದೀಯಲ್ಲಾ! ನ್ಯಾಯವಾದ ಮಾರ್ಗದಲ್ಲಾ, ಯೋಚಿಸಿ ನೋಡು!' ಎಂದೊಡನೆಯೇ ಅರಸನೂ, `ಹೌದು! ಎಷ್ಟೋ ಅನ್ಯಾಯವಾದ ಮಾರ್ಗಗಳ ಮೂಲಕವೇ ಇಷ್ಟು ಅಕ್ಕಿ ಇಲ್ಲಿ ಸೇರಿದೆ' ಎಂದನು. ಮುನಿ ಇಷ್ಟು ಅಕ್ಕಿಯನ್ನು ನಾನು ಒಂದೇ ಒಂದು ದಿನ ತಿಂದೆನು. ಅದಕ್ಕೆ ನನಗೆ ಅಂಥಾ ಕೆಟ್ಟ ಬುದ್ಧಿ ಉಂಟಾಯಿತು. ಹಾಗಿರುವಾಗ ಪ್ರತಿದಿನವೂ ಈ ಅಕ್ಕಿಯನ್ನು ತಿನ್ನುತ್ತಿರುವ ನಿನ್ನ ಬುದ್ಧಿ ಹೇಗಿರುತ್ತದೆನ್ನುವುದನ್ನು ಊಹಿಸಲಾಗುವುದಿಲ್ಲ. ಆದ್ದರಿಂದ, ಇಂದಿನಿಂದ ನೀನು ನನ್ನನ್ನು ಊಟಕ್ಕೆ ಕರೆಯಬೇಡ. ನಾನು ಹೊರಗೇನೇ ಭಿಕ್ಷೆ ಮಾಡಿ ಊಟ ಮಾಡುವೆನು' ಎಂದರು. `ಭಿಕ್ಷೆ ಮಾಡಿ ತಿಂದರೆ ಅದು ಅನ್ಯಾಯವಾಗುವುದಿಲ್ಲವೇ' ಎಂದು ಅರಸನು ಕೇಳಿದ್ದಕ್ಕೆ ಅವರು, `ನನ್ನ ಪಾತ್ರೆಯಲ್ಲಿ ಹಾಕುವವರೆಗೂ ಅದು ಅಶುದ್ಧವೇ. ಪಾತ್ರೆಯಲ್ಲಿ ಹಾಕಿದೊಡನೆ ಪವಿತ್ರವಾಗಿ ಬಿಡುವುದೆನ್ನುವುದು ಶಾಸ್ತ್ರ. ಹಾಗೆಯೇ, `ಭವತಿ ಭಿಕ್ಷಾಂ ದೇಹಿ' ಎಂದು ಹೇಳಿ ಒಬ್ಬನು ಯಾಚಿಸುವಾಗ ಕೊಡುವವನು ಯಾರೆಂದು ನೋಡಬೇಕಾಗಿಲ್ಲ. ಪಾತ್ರೆಯಲ್ಲಿ ಬಿದ್ದೊಡನೆ ಅದೆಲ್ಲಾ ಶುದ್ಧವೇ. ಆದ್ದರಿಂದ ಶುದ್ಧವಾದ ಅನ್ನವನ್ನು ತಿಂದರೆ ಮನಸ್ಸು ಶುದ್ಧವಾಗಿರುವುದು. ಅಪವಿತ್ರವಾದ ಅನ್ನಕ್ಕೆ ಮೊದಲು ಹೇಳಿದಂತೆ ಗುಣವಿರುತ್ತದೆ. ಭಗವತ್ಪಾದರು ಹೇಗೆ ನಾವು ಊಟ ಮಾಡಬೇಕೆಂಬುದನ್ನು ಕುರಿತು ಹೇಳಿದ್ದಾರೆ. ಅದೇ ವಿಷಯವನ್ನೇ ಗೀತಾಚಾರ್ಯನೂ,

\begin{shloka}
ಆಹಾರಸ್ತ್ವಪಿ ಸರ್ವಸ್ಯ ತ್ರಿವಿಧೋ ಭವತಿ ಪ್ರಿಯಃ |\\
ಯಜ್ಞಸ್ತಪಸ್ತಥಾ ದಾನಂ ತೇಷಾಂ ಭೇದಮಿಮಂ ಶೃಣು ||
\end{shloka}

(ಒಬ್ಬೊಬ್ಬರಿಗೂ ಪ್ರಿಯವಾದ ಆಹಾರವೂ ಮೂರು ವಿಧವಾಗಿರುತ್ತದೆ. ಯಾಗ, ತಪಸ್ಸು, ದಾನ ಹೀಗೆ ಇವುಗಳ ಭೇದಗಳನ್ನು ಕೇಳು.) ಎಂದು ಹೇಳಿದ್ದಾನೆ.


ಆಹಾರದಲ್ಲಿ ಮೂರು ವಿಧ. ಶ್ರದ್ಧೆ, ತಪಸ್ಸು ಮುಂತಾದವನ್ನು  ಕೃಷ್ಣನು ಮೂರು ವಿಧವಾಗಿ ವಿಂಗಡಿಸಿ ಹೇಳುವಂತೆ ಆಹಾರವನ್ನೂ ಮೂರು ವಿಧವಾಗಿ ವಿಂಗಡಿಸಿ ಹೇಳುತ್ತಾನೆ. ಮೊದಲನೆಯದು ಸಾತ್ತ್ವಿಕಪ್ರಿಯ- ಸಾತ್ತ್ವಿಕವಾಗಿರುವ ಮನುಷ್ಯನಿಗೆ ಪ್ರಿಯವಾದ ಆಹಾರ.

\begin{shloka}
ರಸ್ಯಾಸ್ನಿಗ್ಧಾಃ ಸ್ಥಿರಾ ಹೃದ್ಯಾ ಆಹಾರಾಃ ಸಾತ್ತ್ವಿಕಪ್ರಿಯಾಃ ||
\end{shloka}

(ರಸವುಳ್ಳವು, ಸ್ನಿಗ್ಧವಾಗಿರುವುವು, ಸ್ಥಿರವೂ ಹೃದ್ಯವೂ ಆಗಿರುವ ಆಹಾರಗಳು ಸಾತ್ತ್ವಿಕಪ್ರಿಯವಾದವು.)

ಕೆಲವು ವಸ್ತುಗಳನ್ನು ತಿನ್ನುವಾಗಲೇ ಬಾಯಿಯೆಲ್ಲಾ ಉರಿಯುತ್ತದೆ. ಅವುಗಳನ್ನು ನಾವು ತಿನ್ನಲು ಸಾಧ್ಯವಾಗುವುದಿಲ್ಲ. ಕೆಲವರು ಮೆಣಸಿನಕಾಯನ್ನೇ ತಿನ್ನುವರು. ಮಧುರವಾಗಿರುವುದೂ, ಶರೀರಕ್ಕೆ ಹಿತವಾಗಿರುವುದೂ ಆದ ಆಹಾರವೇ ಸಾತ್ತ್ವಿಕವಾದ ಆಹಾರವೆನ್ನಲ್ಪಡುವುದು.

ಅನಂತರ, `ಸ್ನಿಗ್ಧಾ' ಎಂದು ಹೇಳಲ್ಪಟ್ಟಿದೆ. ನಾವು ನಮ್ಮ ಜೀವನದಲ್ಲಿ ಹಸುವಿಗೆ ಪ್ರತ್ಯೇಕವಾದ ಗೌರವವನ್ನು ಕೊಟ್ಟಿದ್ದೇವೆ. ಏಕೆಂದರೆ, ಹಸು ಕೊಡುವ ಹಾಲು ಮನುಷ್ಯನ ದೇಹಕ್ಕೂ, ಮನಸ್ಸಿಗೂ ಹೆಚ್ಚು ಒಳ್ಳೆಯದನ್ನು ಮಾಡುತ್ತದೆ. ಎಂಥ ಉಪವಾಸದ ದಿನವಾದರೂ ಕೂಡ ಅಂದು ಹಾಲನ್ನು ಸೇವಿಸಿದರೆ ಉಪವಾಸವ್ರತಕ್ಕೆ ಭಂಗ ಉಂಟಾಗುವುದಿಲ್ಲವೆನ್ನುತ್ತದೆ ಶಾಸ್ತ್ರ. ಅಂಥ ಉತ್ತಮವಾದ ಪದಾರ್ಥ ಹಾಲು. ಹಸುವಿನ ತುಪ್ಪ, ಹಾಲು ಮುಂತಾದವು ಶುದ್ಧವಾದವೆಂದು ಹೇಳಲ್ಪಟ್ಟಿವೆ. ಇವುಗಳಲ್ಲಿ ಶರೀರಕ್ಕೆ ಪುಷ್ಟಿಯನ್ನು ಕೊಡತಕ್ಕ ಅಂಶ ಒಂದಿದೆ. ಅದನ್ನು `ಸ್ನೇಹ' ವೆಂದು ಸಂಸ್ಕೃತದಲ್ಲಿ ಹೇಳಬಹುದು.

`ಸ್ಥಿರಾ' ಎಂದು ಅದಾದ ಮೇಲೆ ಹೇಳಲ್ಪಟ್ಟಿದೆ. ಅಸ್ಥಿರವಾದ ಆಹಾರ, ಸ್ಥಿರವಾದ ಆಹಾರ ಎನ್ನುವ ಭೇದವಿದೆಯೇ ಎಂದು ಕೇಳಿದರೆ, ಕೆಲವು ಆಹಾರಗಳು ತಯಾರಾದ ಸ್ವಲ್ಪ ಹೊತ್ತಿನಲ್ಲಿಯೇ ಕೆಟ್ಟು ಹೋಗುತ್ತವೆ. ಸ್ವಲ್ಪ ಹೊತ್ತು ಮಾತ್ರ ಸ್ಥಿರವಾಗಿರುವುವು.

ಅನಂತರ `ಹೃದ್ಯಾಃ' ಎಂದು ಹೇಳಲ್ಪಟ್ಟಿದೆ. ಕೆಲವು ಆಹಾರಗಳನ್ನು ತಿನ್ನುವಾಗಲೇ ತೊಂದರೆಯಾಗುತ್ತದೆ. ಎಷ್ಟೋ ಮಂದಿ ಹೊಗೆಸೊಪ್ಪು ಹಾಕಿಕೊಳ್ಳುತ್ತಾರೆ. ಹೊಗೆಸೊಪ್ಪು ಹಾಕಿಕೊಳ್ಳುವುದು ಸುಖಕರವಾಗಿರುತ್ತದೆಯೇ ಎಂದರೆ, ಅದನ್ನು ಹಾಕಿಕೊಳ್ಳುವುದಕ್ಕೆ ಹದಿನೈದು ದಿನಗಳಾದರೂ ಅಭ್ಯಾಸಬೇಕು. ಅಭ್ಯಾಸವಾದ ಮೇಲೆ ಹಾಕಿಕೊಳ್ಳಬಹುದು. ಕೆಲವರಿಗೆ ಹೊಗೆಸೊಪ್ಪು ಹಾಕಿಕೊಳ್ಳುವಾಗ ವಾಂತಿಯೇ ಆಗಿಬಿಡುತ್ತದೆ. ಸ್ವಲ್ಪ ಸ್ವಲ್ಪವಾಗಿ ಅಭ್ಯಾಸ ಮಾಡಿಕೊಂಡು ಬಂದರೆ ಕೊನೆಗೆ ಹೊಗೆಸೊಪ್ಪು ಹಾಕದೆ ಇದ್ದರೆ ವಾಂತಿ ಬಂದು ಬಿಡುತ್ತದೆ ಎನ್ನುವ ಸ್ಥಿತಿ ಉಂಟಾಗುತ್ತದೆ! ಸಾಧಾರಣವಾಗಿ ಹೊಗೆಸೊಪ್ಪನ್ನು ನೋಡದವನಿಗೆ ಅದನ್ನು ಕಂಡರೇನೇ ವಾಂತಿ ಬಂದು ಬಿಡುತ್ತದೆ. ಆದ್ದರಿಂದಲೇ `ಹೃದ್ಯಾಃ' ಎಂದು ಹೇಳಿರುವುದು. ನಾವು ತಿನ್ನುವ ವಸ್ತು, ತಿನ್ನುವಾಗ ಆನಂದವನ್ನು ಕೊಡುವಂತೆ ಇರಬೇಕು. ಮಕ್ಕಳಿಗೆ ನಾವು ಯಾವುದನ್ನಾದರೂ ಪ್ರೀತಿಯಿಂದ ಕೊಡಬೇಕೆಂದಿದ್ದರೆ ಕಲ್ಲುಸಕ್ಕರೆ, ಮಿಠಾಯಿ ಅಥವಾ ಹಾಲನ್ನು ಕೊಡುತ್ತೇವೆ. ಎರಡು ಹಸಿಮೆಣಸಿನಕಾಯನ್ನಾಗಲಿ, ಉಪ್ಪನ್ನಾಗಲಿ ಯಾರಾದರೂ ಕೊಡುತ್ತಾರೆಯೇ? ಆದರೆ ನಮ್ಮ ಅಭ್ಯಾಸದ ಮೂಲ ಮೆಣಸಿನಕಾಯಿ ಮುಂತಾದವು. ಅವುಗಳನ್ನು ನಾವು ತಿನ್ನುವ ಪದಾರ್ಥಗಳನ್ನಾಗಿ ಮಾಡಿಕೊಂಡಿದ್ದೇವೆ. ಭಗವಂತನ ದೃಷ್ಟಿಯಲ್ಲಿ ಸ್ವಾಭಾವಿಕವಾಗಿ ಮನುಷ್ಯನಿಗೆ ಬೇಕಾದ ವಸ್ತುವನ್ನು ಅವನು ತಿಂದರೆ ಸರಿ. ಮೆಣಸಿನಕಾಯಿ ತಿನ್ನದೆ ಮನುಷ್ಯನು ಸಾಧಾರಣವಾಗಿ ದೃಢಕಾಯನಾಗಿ ಇರಬಹುದು. ಆದರೆ ಬಾಯಿಗೆ ಹಲವು ವಿಧವಾದ ರುಚಿಗಳೂ ಇವೆ. ಅರ್ಧಂಬರ್ಧ ರುಚಿ ಇರುವ ಪದಾರ್ಥಗಳೂ ಮನುಷ್ಯನಿಗೆ ಅಭ್ಯಾಸವಾಗಿ ಬಿಟ್ಟಿವೆ. ಆದರೆ ಭಗವಂತನು `ಆಹಾರಾಃ ಸಾತ್ತ್ವಿಕ ಪ್ರಿಯಾಃ' ಎಂದು ಹೇಳಿ ಸಾತ್ತ್ವಿಕವಾದ ಆಹಾರಗಳನ್ನು ನಮಗೆ ತಿಳಿಸಿರುತ್ತಾನೆ.

ಅನಂತರ,

\begin{shloka}
ಕಟ್ವಾಮ್ಲಲವಣಾತ್ಯುಷ್ಣ ತೀಕ್ಷ್ಣ ರೂಪ ವಿದಾಹಿನಃ |\\
ಆಹಾರಾ ರಾಜಸಾಸ್ಯೇಷ್ಟಾ ದುಃಖಶೋಕಮಯಪ್ರದಾಃ ||
\end{shloka}

(ವಗಚು, ಹುಳಿ, ಉಪ್ಪು, ಹೆಚ್ಚು ಬಿಸಿಯಾಗಿರುವುದು, ಖಾರವಾಗಿರುವುದು, ರೂಕ್ಷವಾದುದು ಉರಿಯನ್ನುಂಟು ಮಾಡುವುದು, ದುಃಖವನ್ನು ಕೊಡುವುದು-ಇಂಥ ಆಹಾರಗಳು ರಜೋಗುಣವುಳ್ಳವರಿಗೆ ಪ್ರಿಯವಾಗುತ್ತವೆ.) ಎಂದಿದ್ದಾನೆ.

ಆಹಾರಗಳಲ್ಲಿ ಎರಡನೆಯ ವಿಧವಾಗಿ `ರಾಜಸ ಆಹಾರ' ಹೇಳಲ್ಪಟ್ಟಿದೆ. ಮೇಲೆ ಹೇಳಿದ ಶ್ಲೋಕದಲ್ಲಿ `ಕಟು' ಎಂದಿದೆ. ಕೆಲವರು ಮೆಣಸಿನಕಾಯಿ ಇಷ್ಟಪಟ್ಟು ತಿನ್ನುತ್ತಾರೆ. ಆದರೆ ಹೆಚ್ಚಾಗಿ ತಿಂದರೆ ಅದರಿಂದ ತೊಂದರೆಯಾಗುವುದೆನ್ನುವ ಭಯವಿರುತ್ತದೆ. ಖಾರ ಹೆಚ್ಚಾಗಿ ತಿನ್ನುವುದರಿಂದ ಎಷ್ಟೋ ರೋಗಗಳು ಹೆಚ್ಚಾಗುತ್ತವೆ. ಹಾಗೆಯೇ ನಿಂಬೇಹಣ್ಣು ಹೆಚ್ಚಾಗಿ ತಿನ್ನುವವರೂ ಇದ್ದಾರೆ. ಉಪ್ಪನ್ನು ಹೆಚ್ಚಾಗಿ ತಿನ್ನುವವರೂ ಇದ್ದಾರೆ. ಇನ್ನೂ ಕೆಲವರು ಊಟ ಮಾಡುವಾಗ ಪಾತ್ರೆಯನ್ನು ಬಟ್ಟೆಯಲ್ಲಿ ಹಿಡಿದುಕೊಳ್ಳುತ್ತಾರೆ. ಏಕೆಂದರೆ ಅದು ಬಿಸಿಯಾಗಿದೆ ಎನ್ನುತಾರೆ. ಆದರೆ ಅವರ ನಾಲಿಗೆಗೆ ಆ ಬಿಸಿ ತಿಳಿಯುವುದೇ ಇಲ್ಲ. ಏಕೆಂದರೆ ನಾಲಿಗೆಗೆ ಅಭ್ಯಾಸವಾಗಿ ಬಿಟ್ಟಿದೆ. ಕುದಿಯುತ್ತಿರುವ ಅನ್ನವನ್ನು ಹಾಕಿದರೂ, ನಾಲಿಗೆಯಲ್ಲಿ ನರಗಳು ಸತ್ತು ಹೋದರೂ ಕೂಡ ಪ್ರಜ್ಞೆಯೇ ಇರುವುದಿಲ್ಲ. ಭಗವಂತನು ನಮಗೆ ಏತಕ್ಕೆ ನಾಲಿಗೆ ಕೊಟ್ಟಿದ್ದಾನೆ? ಯಾವುದೇ ಒಂದು ಪದಾರ್ಥವನ್ನು ತಿನ್ನುವುದಕ್ಕೆ ಮೊದಲು ಅದು ಬಿಸಿಯಾಗಿದೆಯೇ, ಬಹಳ ಖಾರವಾಗಿದೆಯೇ ಅಥವಾ ಹುಳಿಯಾಗಿದೆಯೇ ಎಂದು ನೋಡಿ ಅನಂತರ ತಿನ್ನುವುದಕ್ಕಾಗಿಯೇ ನಾಲಿಗೆಯನ್ನು ಕೊಟ್ಟಿದ್ದಾನೆ. ನಾಲಿಗೆಯನ್ನು ಸಂಯಮದಲ್ಲಿಟ್ಟುಕೊಳ್ಳದೆ ಖಾರದಂತಹವುಗಳನ್ನೇ ತಿನ್ನುತ್ತಿದ್ದರೆ ಶರೀರ ಬೇಗ ಕೆಟ್ಟು ಹೋಗುತ್ತದೆ. ಎರಡನೆಯದಾಗಿ ಇದಕ್ಕೆ ತಕ್ಕಂತೆ ಮನಸ್ಸು ಯಾವಾಗಲೂ ಬಿಸಿಯಾಗಿಯೇ ಇರುವುದು.

ಒಂದು ಹುಲಿಯ ಸ್ವಭಾವವನ್ನೂ ಒಂದು ಜಿಂಕೆಯ ಸ್ವಭಾವವನ್ನೂ ಹೋಲಿಸಿ ನೋಡಬಹುದು. ಜಿಂಕೆಯ ಸ್ವಭಾವ ಶಾಂತವಾಗಿರುತ್ತದೆ. ಅದನ್ನು ನೋಡುತ್ತಲೇ ಮನಸ್ಸಿಗೆ ಒಂದು ವಿಧವಾದ ಶಾಂತಿಯಾಗುತ್ತದೆ. ಆದರೆ ಒಂದು ಹುಲಿಯನ್ನು ಕಂಡರೆ ಆನಂದವಾಗಬಹುದು, ಶಾಂತಿ ಉಂಟಾಗುವುದಿಲ್ಲ. ಹುಲಿಯನ್ನು ಕಂಡರೆ, `ಆಹಾ! ಬಹಳ ಶಾಂತವಾಗಿದೆ' ಎಂದು ಯಾರೂ ಹೇಳುವುದಿಲ್ಲ. `ಆಹಾ! ಬಹಳ ಜೋರಾಗಿದೆ' ಎಂದು ಬೇಕಾದರೆ ಹೇಳಬಹುದು. ಆ ಕಾರಣದಿಂದ ನಾವು ಅದರ ಹತ್ತಿರಕ್ಕೆ ಹೋದರೆ ಅದು ತನ್ನ ಉಗುರುಗಳನ್ನು ತೋರಿಸುತ್ತದೆ. ತಕ್ಷಣ ಅದನ್ನು `ಇದು ಬಹಳ ದುಷ್ಟ ಮೃಗ' ಎನ್ನುತ್ತೇವೆ. ಅದೇ ರೀತಿ ನಾವು ಸಾತ್ತ್ವಿಕವಾದ ಆಹಾರಗಳನ್ನು ತೆಗೆದುಕೊಳ್ಳದೆ ರಾಜಸ ಆಹಾರಗಳನ್ನೇ ತೆಗೆದುಕೊಳ್ಳುತ್ತಿದ್ದರೆ ಹುಲಿ ಮುಂತಾದವುಗಳಂತೆ,


\begin{shloka}
`ಕಟ್ವಾಮ್ಲಲವಣಾತ್ಯುಷ್ಣತೀಕ್ಷ್ಣ ರೂಕ್ಷವಿದಾಹಿ ನಃ |'
\end{shloka}

-ಎಂದು ಹೇಳಿದಂತೆ ನಾವು ತಿನ್ನುತ್ತಿದ್ದರೆ ರಾಜಸವಾದ ವಿಚಾರಗಳೇ ಬರುತ್ತಿರುತ್ತವೆ. ಗೀತೆಯಲ್ಲಿ ಇನ್ನೂ ಹೀಗಿದೆ-

\begin{shloka}
`ಯಾತ ಯಾಮಂ ಗತರಸಂ ಪೂತಿ ಪರ್ಯುಷಿತಂ ಚ ಯತ್ |\\
ಉಚ್ಛಿಷ್ಟಮಪಿ ಚಾಮೇಧ್ಯಂ ಭೋಜನಂ ತಾಮಸಪ್ರಿಯಂ ||'
\end{shloka}

(ಯಾಮ ಕಳೆದುದು ಸ್ವಾದರಹಿತವಾದುದು, ದುರ್ಗಂಧ ಯುಕ್ತವಾದುದು, ತಂಗಳು, ಎಂಜಲು, ಅಶುದ್ಧವಾದುದು, ಹೀಗಿರುವ ಆಹಾರ ತಮೋ ಗುಣವುಳ್ಳವರಿಗೆ ಪ್ರಿಯವಾದುದು)

ಈ ಕಾಲದಲ್ಲಿ ಅನೇಕ ಜನರಿಗೆ ತಂಗಳು ಅನ್ನ ತಿನ್ನುವ ಅಭ್ಯಾಸವಿದೆ. ತಂಗಳು ಅನ್ನ ತಿನ್ನುವುದು ತಪ್ಪೆಂದು ಶಾಸ್ತ್ರ ಹೇಳುತ್ತದೆ. ಅದನ್ನು ತಿಂದರೆ ಬಲ ಉಂಟಾಗುತ್ತದೆ ಎನ್ನುತ್ತಾರೆ. ಆದರೆ ಎಂಥ ಬಲ ಉಂಟಾಗುತ್ತದೆಂದರೆ ತಾಮಸವಾದ ಬಲ ಉಂಟಾಗುತ್ತದೆ. ತಂಗಳು ಅನ್ನ ತಿಂದು ಓದುವುದಕ್ಕೆ ಕುಳಿತುಕೊಂಡರೆ ಚೆನ್ನಾಗಿ ನಿದ್ದೆ ಬರುತ್ತದೆ. ನಾವು ಯಾವ ಉದ್ದೇಶಕ್ಕಾಗಿ ಕುಳಿತುಕೊಂಡೆವೋ ಅದು ಆಗುವುದಿಲ್ಲ. ತಯಾರು ಮಾಡಿ ಬಹಳ ದಿನಗಳಾದ, ಹಳೆಯ ತಿಂಡಿತಿನಿಸುಗಳನ್ನು ತಿನ್ನುವ ಅಭ್ಯಾಸಮಾಡಿಕೊಂಡರೆ ನಿದ್ದೆ ಮಾತ್ರವೇ ಅಲ್ಲ, ಎದುರಿಗಿರುವುದೆಲ್ಲಾ ಸ್ವರ್ಗವಾಗಿಯೋ, ಅಥವಾ ನರಕವಾಗಿಯೋ ಕೂಡ ತೋರಬಹುದು. ಹಾಕಿಕೊಳ್ಳುವುದಕ್ಕೆ ಬಟ್ಟೆ ಬೇಕೆ ಬೇಡವೇ ಎನ್ನುವ ಪ್ರಜ್ಞೆಯೂ ಇಲ್ಲದೆ ಹೋಗುವುದು. ಆದ್ದರಿಂದಲೇ ಭಗವಂತನು,

\begin{shloka}
`ಯಾತ ಯಾಮಂ ಗತರಸಂ'
\end{shloka}


-ಎಂದಿದ್ದಾನೆ. ಅಡಿಗೆ ಮಾಡಿ ನಾಲ್ಕು ಗಂಟೆಗಳಾಗಿ ಬಿಟ್ಟರೆ ಆ ಆಹಾರದಲ್ಲಿ ಪ್ರಾಣಕ್ಕೆ ದೊರೆಯಬೇಕಾದ ರಸವೂ ಹೋಗಿಬಿಡುವುದು. ದಿನಗಳು ಆಗುತ್ತಾ ಆಗುತ್ತಾ `ಪರ್ಯುಷಿತಂ' (ನಾಶೋನ್ಮುಖತೆ) ಹೆಚ್ಚಾಗುತ್ತಾ ಇರುತ್ತದೆ. ಆದರೆ ಅದರಲ್ಲೇ ಅಭ್ಯಾಸವಿರುವವನಿಗೆ ಎಷ್ಟು ದಿನಗಳು ವಾಸನೆ ಹೆಚ್ಚಾಗುತ್ತದೆಯೋ, ಅಷ್ಟೇ ಅದರಲ್ಲಿ ರಸ ಹೆಚ್ಚಾಗಿ ಇರುವುದಾಗಿ ತೋರುವುದು. ಆದರೆ ಮನಸ್ಸು ಬಹಳ ತಾಮಸವಾಗಿ ಬಿಡುವುದು. ಮನಸ್ಸು ಬಹಳ ತಮೋಗುಣದಿಂದ ಕೂಡಿದ್ದರೆ ಅವನ ಸಹವಾಸವೂ, ಸ್ವಭಾವವೂ ಅದೇ ರೀತಿ ಆಗಿ ಬಿಡುವುದು. ಆದರೆ ಅವನು ಸಂತೋಷವಾಗಿದ್ದಾನಲ್ಲಾ ಎಂದರೆ ಆ ಸಂತೋಷದಿಂದ ಏನು ಪಡೆಯಬಹುದು? ಹಲವು ಅನರ್ಥಗಳನ್ನೇ ಅವನು ಪಡೆಯುವನು. ಅವನ ಬಳಿ ಇರುವ ಆಸ್ತಿಯೆಲ್ಲಾ ನಾಶವಾಗಿ ಬಿಡುವುದು. ಬೀದಿಯಲ್ಲಿ ಹೋಗುವಾಗ ಎಲ್ಲಾದರೂ ಕೊಳಚೆಯಲ್ಲಿ ಬಿದ್ದು ಬಿಡುವನು. ಅಲ್ಲದೆ, ಬೀದಿಯಲ್ಲಿ ಬಿದ್ದು ಬಿಟ್ಟನೆನ್ನುವ ಕೆಟ್ಟ ಹೆಸರೂ ಅವನಿಗೆ ದೊರೆಯುವುದು.

\begin{shloka}
`ಕ್ರುದ್ಧೋ ಹನ್ಯಾತ್ ಗುರೂನಪಿ'
\end{shloka}

-ಎಂದು ಹೇಳಿದಂತೆ ಅಂಥವನು ಕೋಪ ಬಂದರೆ ಗುರುವನ್ನೂ ಹೊಡೆದುಬಿಡುವನಂತೆ. ಬುದ್ಧಿ ಸರಿಯಾಗಿಲ್ಲದವನು ಹೆಂಡತಿಯನ್ನು ಹೊಡೆಯುವುದು ಸಹಜ; ತನ್ನನ್ನೂ ಹೊಡೆದು ಕೊಳ್ಳುತ್ತಾನೆ. ಆಮೇಲೆ ಅವನಿಗೆ ತನ್ನನ್ನು ತಾನು ಹೊಡೆದುಕೊಂಡಿದ್ದು ಅರಿವಾಗುತ್ತದೆ. ಇಂಥ ಬುದ್ಧಿಯನ್ನು ಉಂಟುಮಾಡುವುದು ತಾಮಸವಾದ ಆಹಾರವೆಂದು ಭಗವಂತನು ಹೇಳಿದ್ದಾನೆ. ಸಾತ್ತ್ವಿಕವಾದ ಆಹಾರವನ್ನೇ ನಾನು ತೆಗೆದುಕೊಳ್ಳಬೇಕು. ರಾಜಸವಾದ ಆಹಾರ, ಒಂದುವೇಳೆ, ಪ್ರಾಪಂಚಿಕ ವ್ಯವಹಾರಗಳಲ್ಲಿ ತೊಡಗಬೇಕಾಗಿರುವುದರಿಂದ, ತೆಗೆದುಕೊಂಡರೆ ಬಾಧಕವಿಲ್ಲ. ಆದರೆ ತಾಮಸವಾದ ಆಹಾರವನ್ನು ಎಂದೂ ತೆಗೆದುಕೊಳ್ಳಬಾರದು. ಇಷ್ಟೇ ಅಲ್ಲ, ವೇದ ಹೇಳುತ್ತದೆ-

\begin{shloka}
`ಮೋಘಮನ್ನಂ ವಿನ್ದತೇ ಅಪ್ರಚೇತಾಃ\\
ಸತ್ಯಂ ಬ್ರವೀಮಿ ವದ ಇತ್ಸತಸ್ಯ |\\
ನಾರ್ಯಮಣಂ ಪುಷ್ಯತಿ ನೋ ಸಕಾಯಂ\\
ಕೇವಲಾಘೋ ಭವತಿ ಕೇವಲಾದೀ ||'
\end{shloka}

ಊಟ ಮಾಡುವಾಗ ಹೇಗೆ ಊಟ ಮಾಡುತ್ತೇವೆ? ಮೊದಲು, ಮನೆಯಲ್ಲಿ ಅಡಿಗೆ ಮಾಡಿಟ್ಟು, ಬೀದಿ ಬಾಗಿಲಿಗೆ ಬಂದು ನಿಂತು ಯಾರಾದರೂ ಅತಿಥಿ ಬಂದಿದ್ದಾರೆಯೇ ಎಂದು ನೋಡಿ, ಅವರನ್ನು ಒಳಗೆ ಕರೆದುಕೊಂಡು ಹೋಗಿ ಕೂರಿಸಿ, ಅವರಿಗೆ ಭೋಜನ ಮಾಡಿಸಬಹುದೇ ಎಂದು ನೋಡಿ `ಸ್ವಾಮಿ, ಇಂಥ ಭಾಗ್ಯ ನನಗೆ ದೊರೆಯಬೇಕಲ್ಲಾ!' ಎಂದು ಭಾವಿಸುವ ಕಾಲ ಒಂದಿತ್ತು. ಆದರೆ ಈ ಕಾಲದಲ್ಲಿ ಊಟ ಮಾಡುವ ಸಮಯವಾದರೆ ಬಾಗಿಲನ್ನು ಹಾಕಿ ಬಿಡುತ್ತಾರೆ. ಏಕೆಂದರೆ ಯಾರಾದರೂ ಒಳಗೆ ಬಂದುಬಿಟ್ಟರೆ ಕಷ್ಟ. `ನಾವು ಇರುವುದಾದರೋ ನಾಲ್ಕು ಜನ. ಇರುವುದು ನಮಗೆ ಸಾಕಾಗುವುದಿಲ್ಲ. ಇದರಲ್ಲಿ ಐದನೆಯವರಿಗೆ ಎಲ್ಲಿ?' ಎನ್ನುವ ಭಾವನೆ ಈ ಕಾಲದಲ್ಲಿದೆ. ಸುಮಾರು ಐವತ್ತು ವರ್ಷಗಳಿಗೆ ಹಿಂದೆ ಕೂಡ ಯಾರಾದರೂ ಅತಿಥಿ ಊಟಕ್ಕೆ ಬರುವರೇ ಎಂದು ಎದುರು ನೋಡುತ್ತಿದ್ದ ಕಾಲವಿತ್ತು. ಆದರೆ ಕೆಲವರ ಅಭಿಪ್ರಾಯವೇನೆಂದರೆ, `ಮೇಲೆ ಹೇಳಿದಂತೆ ಇದ್ದರೆ ದೇಶ ಕೆಟ್ಟು ಹೋಗುತ್ತದೆ. ಏಕೆಂದರೆ ಎಲ್ಲರೂ ಪರಾನ್ನಕ್ಕೆ ಆಸೆ ಪಡುವರಾಗಿ ಬಿಡುತ್ತಾರೆ. ಅದರಿಂದ ಎಲ್ಲರಿಗೂ ಏಕೆ ಸಂಪಾದಿಸಬೇಕು ಎನ್ನುವ ಭಾವನೆ ಉಂಟಾಗುವ ಕಾಲ ಬರುತ್ತದೆ. ಆದ್ದರಿಂದ ಅದು ಒಳ್ಳೆಯ ಅಭ್ಯಾಸವಲ್ಲ.' ಆದರೆ ಮೇಲೆ ಹೇಳಿದಂತೆ ನುಡಿದ ಶಾಸ್ತ್ರ ಪರಾನ್ನವನ್ನು ತಿನ್ನುವುದು ಪಾಪವೆಂದೂ ಕೂಡ ತಿಳಿಸುತ್ತದೆ. ಅದೇನೆಂದರೆ ಒಂದು ಜಾಗದಲ್ಲಿ ಏಕಾಂತವಾಗಿ ಊಟ ಮಾಡುವುದು ಪಾಪವೆಂದು ಹೇಳುವ ಶಾಸ್ತ್ರ, ಅದರೊಡನೆಯೇ ಪರಾನ್ನ ತಿನ್ನುವುದೂ ಪಾಪವೆಂದು ಹೇಳಿದೆ. ಇದರಿಂದ ಏನು ತಿಳಿಯುತ್ತದೆಂದರೆ ಹಣವಂತನಾದವನು ತನ್ನೊಡನೆ ಊಟ ಮಾಡಲು ಯಾರಾದರೂ ಬರುತ್ತಾರೆಯೇ ಎಂದು ಎದುರು ನೋಡಬೇಕು, ಊಟ ಮಾಡುವವನು -`ಇಂದು ನಾನು ಏಕೆ ವ್ಯರ್ಥವಾಗಿ ಪರಾನ್ನ ತಿನ್ನಬೇಕು? ಇದನ್ನು ತಿನ್ನದೆ ಇರಲು ಸಾಧ್ಯವಾಗುವುದಿಲ್ಲವೇ?' ಎಂದು ಯೋಚಿಸಬೇಕು. ಹೀಗೆ ಒಬ್ಬನು ಊಟ ಮಾಡದೆ ಇರಲು ಸಾಧ್ಯವಾಗದೇ ಇದ್ದರೆ ಬೇರೆ ಮಾರ್ಗವಿಲ್ಲದೆ `ಇಂದು ಪವಿತ್ರವಾದ ಅನ್ನವನ್ನು ತಿನ್ನಬೇಕು' ಎನ್ನುವ ಭಾವನೆಯಿಂದ ಒಂದು ಜಾಗದಲ್ಲಿ ಊಟ ಮಾಡಬಹುದು. ಈ ಸ್ಥಿತಿಯಲ್ಲಿದ್ದರೆ ಹಣವಂತನಿಗೂ ಬಡವನಿಗೂ ಹೊಂದಾಣಿಕೆ ಹೆಚ್ಚಾಗುತ್ತದೆ. ಆದರೆ ಈ ಕಾಲದಲ್ಲಿ ಇಬ್ಬರಿಗೂ ಜಗಳ. ಹೀಗೆ ಜಗಳವಾಡುವುದು ಸರಿಯಲ್ಲ. ಎರಡು ಕಡೆಯವರೂ ಅನ್ಯೋನ್ಯವಾಗಿದ್ದು, `ನಾನು ದಾನ ತೆಗೆದುಕೊಳ್ಳುವುದಕ್ಕೆ ಏನು ಮಾಡಿದ್ದೇನೆ' ಎಂದು ತೆಗೆದುಕೊಳ್ಳುವವನೂ, `ನಾನು ನನ್ನ ಹಣವನ್ನು ಕೊಡದೆ ಹೋದರೆ ಸಂಪಾದಿಸಿದುದೆಲ್ಲಾ ವ್ಯರ್ಥವಾಗಿ ಹೋಗುವುದೇ ಅಲ್ಲದೆ ಪಾಪವನ್ನೂ ಕಟ್ಟಿಕೊಳ್ಳಬೇಕಾಗುವುದು' ಎಂದು ಕೊಡುವವನೂ ಭಾವಿಸಬೇಕು. ಕೊಡುವವನು ಅವನೊಡನೆ ಸ್ವಲ್ಪವಾದರೂ ಸ್ವೀಕರಿಸಬೇಕೆಂದು ಕೋರಬೇಕು. ತೆಗೆದುಕೊಳ್ಳುವವನು `ನಿರ್ಬಂಧವಾಗಿ ನೀವು ಕೊಡುತ್ತಿರುವಿರಲ್ಲಾ! ನನಗೆ ತೆಗೆದುಕೊಳ್ಳಲು ಅಧಿಕಾರವಿದೆ, ಸರಿ ಇರಲಿ' ಎಂದು ಹೇಳಿ ತೆಗೆದುಕೊಳ್ಳಬೇಕು. ಇದನ್ನು ಇಟ್ಟುಕೊಂಡು ವೇದ,

\begin{shloka}
`ಮೋಘಮನ್ನಂ ವಿದ್ದುತೇ ಅಪ್ರಚೇತಾಃ'
\end{shloka}

-ಎಂದು ಹೇಳಿದಂತೆ ಯಾರು ತನಗಾಗಿ ಮಾತ್ರ ಅಡಿಗೆ ಮಾಡಿಕೊಂಡು ಊಟ ಮಾಡುತ್ತಾನೋ ಅದು ವ್ಯರ್ಥವೇ. `ಅಮೋಘ' ಬೆಲೆ ಕಟ್ಟಲು ಸಾಧ್ಯವಾಗದ ವಸ್ತು. `ಮೋಘ' ವೆಂದರೆ, ಅದು ವ್ಯರ್ಥವೆಂದು ಅರ್ಥ. `ಅಪ್ರಚೇತಾಃ' ಅಂಥವನಿಗೆ ಚೈತನ್ಯವಿದ್ದರೂ ಚೈತನ್ಯವಿಲ್ಲದವನಂತೆಯೇ ಅವನು. ಹಾಗಿರುವವನು ದೇವತೆಗಳಿಗೆ ನೈವೇದ್ಯವಿಡುವುದರಿಂದ ಸಮಾಜಕ್ಕೆ ಅವನ ಸಂಪತ್ತು ಉಪಯೋಗವಾಗುವುದಿಲ್ಲ. ಆದ್ದರಿಂದಲೇ `ನಾರ್ಯಮಣಂ ಪುಷ್ಯತಿ ನೋ ಸಕಾಯಂ' ಎಂದು ಹೇಳಲ್ಪಟ್ಟಿದೆ. ಹಾಗಿರುವವನು `ಕೇವಲಾಘೋ ಕೇವಲಾದೀ' ಎಂದು ಹೇಳಿರುವಂತೆ, ಪಾಪವನ್ನು ಮಾತ್ರ ತಿನ್ನುತ್ತಿದ್ದಾನೆ. ಅಂದರೆ, ಒಬ್ಬನು ತನಗಾಗಿ, ಮಾತ್ರ ತಿನ್ನುತ್ತಿದ್ದರೆ ಅವನು ಪಾಪವನ್ನು ಮಾತ್ರ ತಿನ್ನುತ್ತಿದ್ದಾನೆಂದು ವೇದ ಹೇಳುತ್ತದೆ. ಇದು ವೇದದ ತೀರ್ಮಾನ.

ಭಗವತ್ಪಾದರು ವೇದದಲ್ಲಿ ಹೇಳಿರುವ ಅನೇಕ ಉಪದೇಶಗಳನ್ನು `ಶತ ಶ್ಲೋಕೀ' ಎನ್ನುವ ಗ್ರಂಥದಲ್ಲಿ ಸುಂದರವಾಗಿ ಬರೆದಿದ್ದಾರೆ. ಅದರಲ್ಲಿ

\begin{shloka}
`ಅನ್ನಂ ದೇವಾತಿಥಿಭ್ಯೋಽರ್ಪಿತಮಮೃತಮಿದಂ ಚಾನ್ಯಥಾ\\
\hspace{5.3cm} ಮೋಘಮನ್ನಮ್\\
ಯಶ್ಚಾತ್ಮಾರ್ಥೇ ವಿಧತ್ತೇ ತದಿಹ ನಿಗದಿತಂ ಮೃತ್ಯರೂಪಂ ಹಿ ತಸ್ಯ |\\
ಲೋಕೋಽಸೌ ಕೇವಲಾಘೋ ಭವತಿ ತನುಭೃತಾಂ ಕೇವಲಾದೀ ಚ ಯಃ\\
\hspace{5.3cm} ಸ್ಯಾತ್\\
ತ್ಯಕ್ತ್ವಾ ಪ್ರಾಣಾಗ್ನಿಹೋತ್ರಂ ವಿಧಿವದನುದಿನಂ ಯೋಽಶ್ನುತೇ ಸೋಽಪಿ\\
\hspace{5.3cm} ಮರ್ತ್ಯಃ ||'
\end{shloka}

(ದೇವತೆಗಳಿಗೂ, ಅತಿಥಿಗಳಿಗೂ ಅರ್ಪಿತವಾದ ಅನ್ನ ಅಮೃತವೆ ಆಗುತ್ತದೆ, ಬೇರೆ ವಿಧವಾದ ಅನ್ನ ವ್ಯರ್ಥವೇ ಸರಿ. ಅವನು ತನ್ನ ಪ್ರಯೋಜನವನ್ನು ಮಾತ್ರ ಉದ್ದೇಶಿಸಿ ಮಾಡುವ ಅನ್ನ ಇಲ್ಲಿಯೇ `ಮೃತ್ಯು' ರೂಪವಾಗಿರುವುದೆಂದು ಹೇಳಿರುವುದು ಪ್ರಸಿದ್ಧವಾಗಿಯೇ ಇದೆ. ತಾನು ಪ್ರತ್ಯೇಕವಾಗಿಯೇ ಊಟ ಮಾಡುವವನು ಈ ಪ್ರಪಂಚದಲ್ಲಿ ಪಾಪವನ್ನೇ ಮೂಟೆ ಕಟ್ಟಿಕೊಳ್ಳುತ್ತಾನೆ. ವೇದದಲ್ಲಿ ಹೇಳಲ್ಪಟ್ಟ ಪ್ರಾಣಾಗ್ನಿಹೋತ್ರವನ್ನು ಬಿಟ್ಟು ಬಿಟ್ಟು ಊಟಮಾಡುವವನೂ ಮರಣಕ್ಕೆ ಗುರಿಯಾಗುತ್ತಾನೆ.)

\begin{shloka}
`ತೈರ್ದತ್ತಾ ನ ಪ್ರದಾಯೋಭ್ಯಃ ಯೋ ಭುಂಕ್ತೆ ಸ್ತೇನ ಏವ ಸಃ'
\end{shloka}

(ದೇವತೆಗಳಿಂದ ಕೊಡಲ್ಪಟ್ಟಿದ್ದರೂ ಅವರಿಗೆ ಕೊಡದೆ ತಿನ್ನುವವನು ಕಳ್ಳನೇ ಆಗುತ್ತಾನೆ.) -ಎಂದು ಭಗವಂತನು ಗೀತೆಯಲ್ಲಿ ಹೇಳಿದ್ದಾನೆ. ಭಗವಂತನು ಅನ್ನವನ್ನು ಕೊಟ್ಟನು,

\begin{shloka}
`ಅನ್ನಂ ದೇವಾತಿಥಿಭ್ಯಃ ಅರ್ಪಿತಂ ಅಮೃತಮಿದಮ್'
\end{shloka}

ಆದ್ದರಿಂದ ದೇವತೆಗಳು, ಅತಿಥಿಗಳು ಇವರಿಗೆ ಕೊಟ್ಟು ಉಳಿದುದು ಅಮೃತವೆನ್ನಲ್ಪಡುತ್ತದೆಂದು ಶ್ಲೋಕ ಹೇಳುತ್ತದೆ.

\begin{shloka}
`ಅನ್ನಂ ದೇವಾತಿಥಿಭ್ಯೋಽರ್ಪಿತಮಮೃತಮಿದಮ್'
\end{shloka}

ಎನ್ನುವಂತೆ ಇಲ್ಲದೆ ಇದ್ದರೆ,

\begin{shloka}
`ಅನ್ಯಥಾ ಮೋಘಮನ್ನಂ'
\end{shloka}

-ಎಂದು ಹೇಳಲಾಗುವುದು. ಭಗವಂತನಿಗೆ ನೈವೇದ್ಯವಾಗದೆ ಇರುವ ಅನ್ನವಾದರೆ ಅದು `ಮೋಘಮನ್ನಂ'-ವ್ಯರ್ಥವಾಗಿರುವುದು. ಅಂದರೆ, ಭಗವಂತನಿಗೆ ಅರ್ಪಣೆ ಮಾಡದೆ ಊಟಮಾಡುವ ಅನ್ನ ವ್ಯರ್ಥವಾಗುವುದು.

ನಾವು ಅನ್ನವನ್ನು ಪಾಕ ಮಾಡಿದ್ದರೆ ಅದು ದೇವತೆಗಳಿಗೆ ಅರ್ಪಿತವಾಗಬೇಕು. ಹಾಗೆ ಅರ್ಪಣೆ ಮಾಡಿದರೆ ಅದು ಪ್ರಸಾದವೆಂದಾಗುತ್ತದೆ. ಆ ಪ್ರಸಾದವನ್ನು ನಾವು ತೆಗೆದುಕೊಳ್ಳಬೇಕು ಎನ್ನುವ ಭಾವನೆಯಿಂದ ತೆಗೆದುಕೊಳ್ಳಬೇಕೇ ಹೊರತು ಅನ್ನವನ್ನು ಮಾಡುವಾಗಲೇ `ನಾವು ಊಟ ಮಾಡುವುದಕ್ಕಾಗಿ' ಎನ್ನುವ ಭಾವನೆ ಇರಬಾರದು.

\begin{shloka}
`ಯಶ್ಚಾತ್ಮಾರ್ಥೇ ವಿಧತ್ತೇ ತದಿಹ ನಿಗದಿತಂ ಮೃತ್ಯುರೂಪಂ ಹಿ ತಸ್ಯ'
\end{shloka}

`ನನಗಾಗಿಯೇ ನಾನು ತೆಗೆದುಕೊಳ್ಳುತ್ತೇನೆ' ಎನ್ನುವ ಭಾವನೆಯಿಂದ ಕುಳಿತುಕೊಂಡರೆ ಆ ಅನ್ನ ನಮಗೆ ಮೃತ್ಯುವಂತೆ ಆಗುವುದು. ಅದು ಅನ್ನವೇ ಆಗುವುದಿಲ್ಲ. ಆದ್ದರಿಂದ ನಾವು ಅಡಿಗೆ ಮಾಡುವಾಗಲೇ ಭಗವಂತನಿಗೆ ನೈವೇದ್ಯವಿಡಬೇಕೆನ್ನುವ ಭಾವನೆಯಿಂದ ಮಾಡಬೇಕು. ಭಗವಂತನಿಗೆ ಕೊಟ್ಟು ಬಿಟ್ಟರೆ ನಾವು ಹೇಗೆ ಊಟ ಮಾಡುವುದು ಎಂದರೆ, ನಾವು ಅನ್ನವನ್ನು ಭಗವಂತನಿಗೆ ಸಮರ್ಪಿಸಿ, ಅದನ್ನು  ಪ್ರಸಾದ ರೂಪದಲ್ಲಿ ತೆಗೆದುಕೊಳ್ಳಬೇಕು. ಹಾಗೆ ಮಾಡಿದರೆ ಮೊದಲು ಅನ್ನವಾಗಿದ್ದದ್ದು  ಪ್ರಸಾದವಾಗಿ ಬಿಡುವುದು. ಹೀಗೆ ಊಟಮಾಡುವುದರಿಂದ ಯಾವ ತಪ್ಪೂ ಉಂಟಾಗುವುದಿಲ್ಲ.

\begin{shloka}
`ಲೋಕೇಽಸೌ ಕೇವಲಾಘೋ ಭವತಿ ತನುಭೃತಾಂ ಕೇವಲಾದೀ ಚ ಯಃ ಸ್ಯಾತ್'
\end{shloka}

ಸರಿ, ದೇವತೆಗಳಿಗಾಗಿ ನಾವು ಅನ್ನವನ್ನು ಮಾಡಿದ್ದಾಯಿತು. ಅನಂತರ ತಿನ್ನುವುದಕ್ಕೆ ಕುಳಿತುಕೊಂಡಾಗ ದೇವತೆಗಳಿಗೆ ನೈವೇದ್ಯ ಮಾಡಿ ಬಿಟ್ಟೆವು ಎನ್ನುವುದರಿಂದ ನಾವು ಮಾತ್ರ ಒಬ್ಬರೇ ಊಟಮಾಡುವುದಕ್ಕಾಗಿ ಬಾಗಿಲು ಮುಚ್ಚಿಕೊಂಡು ಕುಳಿತುಕೊಂಡರೆ, ಹಾಗೆ ಮಾಡುವುದು ತಪ್ಪೆಂದು ಹೇಳಲ್ಪಟ್ಟಿದೆ. ಪ್ರಪಂಚದಲ್ಲಿ ಹಾಗೆ ಒಬ್ಬನೆ ಕುಳಿತು ತಿನ್ನುವವನು ಪಾಪಿಯಾಗುತ್ತಾನೆ. ಆದ್ದರಿಂದ ಪ್ರಸಾದವೆಂದು ಹೆಸರು ಇಟ್ಟುಕೊಂಡರೂ, ಅದನ್ನು ಒಬ್ಬನೇ ತಿನ್ನಬಾರದು. ಹಾಗೆ ಒಂದು ವೇಳೆ ಹತ್ತಿರ ಯಾರೂ ಇಲ್ಲದಿದ್ದರೆ ನಾವು ಒಬ್ಬರೇ ಊಟಮಾಡುವುದರಲ್ಲಿ ತಪ್ಪಿಲ್ಲ. ಅದರಿಂದ ಪಾಪ ಉಂಟಾಗುವುದಿಲ್ಲ. ಆದರೆ ಸಾಧಾರಣವಾಗಿ ಉಪನಯನವಾದ ಯಾವ ವಟುವಾದರೂ `ಪ್ರಾಣಾಯಸ್ವಾಹಾ, ಅಪಾನಾಯ ಸ್ವಾಹಾ, ವ್ಯಾನಾಯಸ್ವಾಹಾ, ಉದಾನಾಯ ಸ್ವಾಹಾ, ಸಮಾನಾಯ ಸ್ವಾಹಾ, ಬ್ರಹ್ಮಣೇ ಸ್ವಾಹಾ, ಎಂದು ಹೇಳಿ ಪಂಚಾಹುತಿಗಳನ್ನು ಹಾಕಿಕೊಳ್ಳುವನು. ಭಗವಂತನಿಗೆ ನೈವೇದ್ಯ ಮಾಡುವಾಗ ಗಾಯತ್ರೀ ಮಂತ್ರವನ್ನು ಹೇಳಿ, ನೀರನ್ನು ಪ್ರೋಕ್ಷಿಸಿ ಸುತ್ತ ಕಟ್ಟಿ ಅನ್ನದ ಮೇಲೆ ಹಾಕಿ, `ಪ್ರಾಣಾಯ ಸ್ವಾಹಾ' ಎನ್ನುವ ಐದು ಮಂತ್ರಗಳಿಂದ ಹೇಗೆ ಹಗಲು, ಸಾಯಂಕಾಲ ಎರಡು ವೇಳೆಗಳಲ್ಲಿಯೂ,

\begin{shloka}
`ಸತ್ಯಂತ್ವರ್ತೇನ ಪರಿಷಿಂಚಾಮಿ|\\
ಋತಂ ತ್ವಾ ಸತ್ಯೇನ ಪರಿಷಿಂಚಾಮಿ||'
\end{shloka}

-ಎಂದು ವಿಹಿತ ಮಂತ್ರಗಳನ್ನು ಉಚ್ಚರಿಸುತ್ತೇವೆ. ಹೀಗೆ ಮಾಡುವ ಉದ್ದೇಶವೇನು? ನಮ್ಮ ಶರೀರದಲ್ಲಿ ಐದು ವಿಧವಾದ ಪ್ರಾಣಗಳಿವೆ. ಅವುಗಳನ್ನೇ ಪ್ರಾಣ, ಅಪಾನ, ವ್ಯಾನ, ಉದಾನ, ಮತ್ತು ಸಮಾನವೆಂದು ಹೇಳುವುದು.

ವಾಯುವಿನ ಅಭಿಮಾನ ದೇವತೆಗೆ ಮೊದಲು ಆಹುತಿ ಕೊಡುವುದೆಂಬುದು ಇದರ ಅರ್ಥ. ಒಂದೊಂದು (ಮಂತ್ರದ) ಹೆಸರಿನ ಕೊನೆಯಲ್ಲಿಯೂ `ಸ್ವಾಹಾ' ಎಂದು ಹೇಳುವುದರ ಅಭಿಪ್ರಾಯವೇನೆಂದರೆ ನಾನು ತಿನ್ನುವುದಿಲ್ಲ, `ಪ್ರಾಣವಾಗಿರುವ ದೇವತೆಗೆ ಆಹುತಿಯನ್ನು ಕೊಡುತ್ತೇನೆ. ಅಪಾನವಾಗಿರುವ ದೇವತೆಗೆ ಒಂದು ಆಹುತಿಯನ್ನು ಕೊಡುತ್ತೇನೆ. ವ್ಯಾನವಾಗಿರುವ ದೇವತೆಗೆ ಒಂದು ಆಹುತಿಯನ್ನು ಕೊಡುತ್ತೇನೆ. ಉದಾನವಾಗಿರುವ ದೇವತೆಗೆ ಒಂದು ಆಹುತಿಯನ್ನು ಕೊಡುತ್ತೇನೆ. ಸಮಾನವಾಗಿರುವ ದೇವತೆಗೆ ಒಂದು ಆಹುತಿಯನ್ನು ಕೊಡುತ್ತೇನೆ' ಎಂದಾಗುತ್ತದೆ. ಈ ಭಾವನೆಯಿಂದ ನಾವು ಆಗ ಊಟ ಮಾಡಬೇಕು. ಇದರ ಮೂಲಕ ನಾವು ಆ ದೇವತೆಗಳಿಗೆ ಆಹುತಿ ಕೊಟ್ಟ ಮೇಲೆ ಉಳಿದ ಅನ್ನ ಪ್ರಸಾದವಾಗಿ ಬಿಡುವುದು. ಇದೇ ಅನ್ನವನ್ನು ನಾವು ಮೊದಲು ದೇವತೆಗಳಿಗೆ ನೈವೇದ್ಯ ಮಾಡಿರಬಹುದು, ಮಾಡಲಾರದೆಯೇ ಇರಬಹುದು. ಅತಿಥಿಗೆ ಊಟ ಹಾಕಿರಬಹುದು, ಅಥವಾ ಹಾಕಲು ಸಾಧ್ಯವಾಗದೆ ಇರಬಹುದು. ಹೇಗಾದರೂ ಆಗಲಿ, ಈ ಐದು ದೇವತೆಗಳಿಗೆ ಆಹಾರವನ್ನು ಕೊಟ್ಟ ಮೇಲೆ ಆ ಆಹಾರ ಪ್ರಸಾದವಾಗಿ ಬಿಡುವುದು.

\begin{shloka}
`ನಾಯಮಾತ್ಮಾ ಬಲಹೀನೇನ ಲಭ್ಯಃ'
\end{shloka}

(ಬಲಹೀನನಾದವನಿಂದ ಈ ಆತ್ಮನು ಲಭ್ಯನಲ್ಲ.)-ಎಂದು ಹೇಳಿರುವಂತೆ,

`ನಾವು ಮಾಡಬೇಕಾದ ಕರ್ಮಗಳಿಗಾಗಿಯೂ, ವೇದಾಂತ ವಿಚಾರವಾಗಿಯೂ ಭಗವಂತನಿಗೆ ಪೂಜೆ ಮಾಡುವುದಕ್ಕಾಗಿಯೂ ಶರೀರಕ್ಕೆ ಶಕ್ತಿಬೇಕು. ಮನಸ್ಸಿನಲ್ಲಿ ಶಕ್ತಿ ಇರಬೇಕು. ಅದಕ್ಕಾಗಿ ನಾವು ಊಟಮಾಡಬೇಕು' ಎನ್ನುವ ಭಾವನೆ ಇರಬೇಕು. ಅದಲ್ಲದೆ ಈ ಕಾಲದಲ್ಲಿ `ಆ ಟಾನಿಕ್ ತೆಗೆದುಕೊಳ್ಳಬೇಕು. ಈ ಟಾನಿಕ್ ತೆಗೆದುಕೊಳ್ಳಬೇಕು. ಮಾಂಸ ಸೇರಿದ ಪದಾರ್ಥಗಳನ್ನು ತೆಗೆದುಕೊಳ್ಳಬೇಕು' ಎನ್ನುವ ಭಾವನೆಯಿಂದ ಯಾವುದು ಹೊಟ್ಟೆಗೆ ಹೋದರೂ' ಅದು ಪವಿತ್ರವೇ' ಎನ್ನುವ ಅಭಿಪ್ರಾಯದಿಂದ ಜನರು ಕೂಡಿದ್ದಾರೆ.

\begin{shloka}
`ಅಢ್ಯೋಽಭಿಜನವಾನಸ್ಮಿ ಕೋಽನ್ಯೋಽಸ್ತಿ ಸದೃಶೋ ಮಯಾ|\\
ಯಕ್ಷ್ಯೇ ದಾಸ್ಯಾಮಿ ಮೋದಿಷ್ಯೇ ಇತ್ಯಜ್ಞಾನವಿಮೋಹಿತಾಃ||'
\end{shloka}

(ಐಶ್ವರ್ಯವನ್ನು ಹೊಂದಿ ದೊಡ್ಡ ಕುಲದವನಾಗಿದ್ದೇನೆ. ನನಗೆ ಸಮನಾದವನು ಯಾರಿದ್ದಾನೆ? ಯಾಗವನ್ನು ಮಾಡುತ್ತೇನೆ, ದಾನ ಕೊಡುತ್ತೇನೆ, ಸಂತೋಷಪಡುತ್ತೇನೆ ಎಂದುಕೊಳ್ಳುತ್ತಾರೆ, ಅಜ್ಞಾನದಲ್ಲಿ ಭ್ರಾಂತರಾಗಿರುವವರು.)

\begin{shloka}
`ಅನೇಕ ಚಿತ್ತವಿಭ್ರಾಂತಾ ಮೋಹಜಾಲಸಮಾವೃತಾಃ'
\end{shloka}

(ಹಲವು ಭಾವನೆಗಳಿಂದ ಭ್ರಾಂತರಾಗಿರುವವರು, ಮೋಹಜಾಲದಲ್ಲಿ ಆವೃತ\-ರಾಗಿರುವವರು)-

\begin{shloka}
`ಮಾಮಪ್ರಾಷ್ಯೈವ ಕೌಂತೇಯ ತತೋ ಯಾನ್ತ್ಯಧಮಾಂಗತಿಂ ||'
\end{shloka}

(ನನ್ನನ್ನು ಪಡೆಯದೆ ಇನ್ನೂ ಕೆಳಮಟ್ಟದ ಗತಿಯನ್ನು ಪಡೆಯುವವರು.)

`ಮನುಷ್ಯನ ಜನ್ಮ ದೊರತಮೇಲೆ ಅದನ್ನು ವ್ಯರ್ಥ ಮಾಡಿಕೊಳ್ಳಬಹುದೇ?

\begin{shloka}
`ಮಾಮುಪೇತ್ಯ ತು ಕೌಂತೇಯ ಪುನರ್ಜನ್ಮ ನ ವಿದ್ಯತೇ||'
\end{shloka}

(ನನ್ನನ್ನು ಪಡೆದವನಿಗೆ ಪುನರ್ಜನ್ಮವಿಲ್ಲ. ಅರ್ಜುನ!)

-ಎಂದು ಹೇಳಿರುವುದರಿಂದ ಪರಮಾತ್ಮನ ಸಾಕ್ಷಾತ್ಕಾರವನ್ನು ಪಡೆದು ಈ ಹುಟ್ಟುಸಾವು ಚಕ್ರದಿಂದ ಬಿಡುಗಡೆ ಹೊಂದಬೇಕೇ, ಎಂದು ಯೋಚನೆ ಮಾಡಿದರೆ -ಈ ಹುಟ್ಟು-ಸಾವು ಚಕ್ರಬೇಡವೆಂದೆ ಎಲ್ಲರೂ ಭಾವಿಸುವರು. ಮನಸ್ಸನ್ನು ಪವಿತ್ರವಾಗಿಟ್ಟುಕೊಂಡು ವಿಚಾರಮಾಡಿದರೇನೇ ಆ ವಿಚಾರ ನಮಗೆ ಫಲವನ್ನು ಕೊಡುವುದು. ಮನಸ್ಸು ಅಪವಿತ್ರವಾಗಿದ್ದರೆ ಒಬ್ಬನು ಮಾಡುವ ಕೆಲಸಗಳಿಂದ ಸರಿಯಾದ ಫಲ ದೊರೆಯುವುದಿಲ್ಲ. ವಿಚಾರವೂ ಸರಿಯಾಗಿ ಉಂಟಾಗುವುದಿಲ್ಲ.

\begin{shloka}
`ಸತ್ಸಂಗತ್ವೇ ನಿಸ್ಸಂಗತ್ವಂ\\
ನಿಸ್ಸಂಗತ್ವೇ ನಿರ್ಮೋಹತ್ವಮ್\\
ನಿರ್ಮೋಹತ್ವೇ ನಿಶ್ಚಲ ತತ್ತ್ವಂ\\
ನಿಶ್ಚಲ ತತ್ತ್ವೇ ಜೀವನ್ಮುಕ್ತಿಃ||'
\end{shloka}

(ಸತ್ಸಂಗತ್ವದಿಂದ ನಿಸ್ಸಂಗತ್ವ, ನಿಸ್ಸಂಗತ್ವದಿಂದ ನಿರ್ಮೋಹತ್ವ, ನಿರ್ಮೋಹತ್ವದಿಂದ ನಿಶ್ಚಲವಾದ ಜ್ಞಾನ, ನಿಶ್ಚಲವಾದ ಜ್ಞಾನದಿಂದ ಜೀವನ್ಮುಕ್ತಿ ಉಂಟಾಗುತ್ತದೆ.)

-ಎನ್ನುವ ಶ್ಲೋಕದಲ್ಲಿ `ಸತ್ಸಂಗತ್ವ' ಬೇಕೆಂದು ಹೇಳಲಾಗಿದೆ. ಇಂಥ ನಿಯಮಗಳನ್ನು ಬಿಟ್ಟು  ಸಿಕ್ಕಿಸಿಕ್ಕಿದ ಜಾಗದಲ್ಲಿ ಕುಡಿಯುವುದೋ ತಿನ್ನುವುದೋ, ಅದೇ ರೀತಿ ಯಾವುದು ಯಾವುದನ್ನೋ ಕುಡಿಯುವುದೋ, ತಿನ್ನುವುದೋ ಮಾಡುತ್ತಿದ್ದರೆ ನಮ್ಮ ಮನಸ್ಸು ಅದಕ್ಕೆ ತಕ್ಕಂತೆ ತನ್ನ ಇಷ್ಟ ಬಂದಂತೆ ಅಲೆಯುವುದು. ಕೋತಿಯ ಬುದ್ಧಿ ಹೇಗಿರುತ್ತದೋ ಹಾಗೆ ಅದು ಓಡಾಡುತ್ತದೆ. ಆದರೆ ಸತ್ಪುರುಷರು ಹೇಗಿರುತ್ತಾರೆಂದರೆ, ಸತ್ಪುರುಷರಿಗೆ ಎಲ್ಲಿ ಜಾಗವಿರುವುದೋ ಅಲ್ಲಿ ಅವರು ಇರುವರು. ಆದ್ದರಿಂದ ನಾವೆಲ್ಲರೂ ಸತ್ಸಾಂಗತ್ಯದಲ್ಲಿದ್ದು, ಸದಾಹಾರ(ಪವಿತ್ರವಾದ ಆಹಾರ)ವನ್ನು ತೆಗೆದು ಕೊಳ್ಳುತ್ತಾ ಬಂದರೆ ನಮ್ಮ ಬುದ್ಧಿಯು ಕೂಡ ಬಹಳ ಶುದ್ಧವಾಗಿರುವುದು.

ನಾವು ಜೀವನದಲ್ಲಿ ಬೇಕಾದಷ್ಟು ಸಂಪಾದನೆ ಮಾಡಬಹುದು. ಹಣ ಸಂಪಾದನೆ ಮಾಡಿ ಏನು ಪ್ರಯೋಜನ? ಪ್ರಪಂಚದಲ್ಲಿ ಎಷ್ಟೋ ಮಂದಿ ಹಣವಂತರೂ ಇದ್ದಾರೆ. ಅವರಲ್ಲಿ ಸುಖವಾಗಿಯೂ, ಶಾಂತಿಯಾಗಿಯೂ ಇರುವ ಹಣವಂತರನ್ನು ವಿರಳವಾಗಿ ಕಾಣಬಹುದು, ಅವರಿಗೆ ಇರುವ ಯೋಚನೆಗಳಿಗೆ ಪರಮಾತ್ಮನೊಬ್ಬನೆ ಎಲ್ಲೆ ಕಾಣಿಸಬಲ್ಲನು. ನಾನು ಅನೇಕ ಜನರನ್ನು ನೋಡಿದ್ದೇನೆ. ದೊಡ್ಡಕಾರಿನಲ್ಲಿ ಒಬ್ಬರು ಹೋಗುತ್ತಿದ್ದಾರೆ, ಆದರೆ ಮನಸ್ಸಿನಲ್ಲಿ ಯಾವ ಯಾವುದೋ ಚಿಂತೆಗಳು. ನಾಳೆ ಏನು ಗತಿ ಎನ್ನುವ ಯೋಚನೆಯಲ್ಲಿ ಇರುತ್ತಾರೆ. ಅದರಲ್ಲೂ ಮಕ್ಕಳು ವಯಸ್ಸಿಗೆ ಬಂದವರಾದರೆ ಇನ್ನು ಹೇಳಬೇಕಾಗಿಲ್ಲ.

\begin{shloka}
`ಪುತ್ರಾದಪಿ ಧನಭಾಜಾಂ ಭೀತಿಃ\\
ಸರ್ವತ್ರೈಷಾ ವಿಹಿತಾ ರೀತಿಃ'
\end{shloka}

(ಮಕ್ಕಳಿಂದಲೂ ಕೂಡ ಹಣವಂತರಿಗೆ ಭಯವಿದೆ. ಎಲ್ಲೆಲ್ಲಿಯೂ ಇದೇ ರೀತಿ ನಡೆದುಕೊಂಡು ಬರುತ್ತಿದೆ.)

ಮಕ್ಕಳಿಗೆ ಎಲ್ಲಾ ಕೊಡಬೇಕಲ್ಲಾ ಎಂದು ತಂದೆಗೆ ಚಿಂತೆ. ಅವರು ಎಲ್ಲವನ್ನೂ ಕೊಡಬೇಕಲ್ಲಾ ಎಂದು ಮಕ್ಕಳಿಗೆ ಚಿಂತೆ. ಹಾಗೆಯೇ ತಂದೆಗೆ, `ಮಕ್ಕಳಿಗೆ ಇಷ್ಟು ಕೊಟ್ಟರೂ ತೃಪ್ತಿ ಇಲ್ಲವಲ್ಲಾ' ಎನ್ನುವ ಭಾವನೆ. ಮಕ್ಕಳಿಗಾದರೋ, `ತಂದೆ ಇಷ್ಟೆಲ್ಲಾ ಇಟ್ಟುಕೊಂಡು ಏನು ಮಾಡುತ್ತಾರೆ' ಎನ್ನುವ ಭಾವನೆ. ಹಣವಂತರಿಗೆ ಶಾಂತಿ ಇಲ್ಲವೆಂದು ತೋರಿಸುವುದಕ್ಕೆ ಈ ನಿದರ್ಶನ ಒಂದೇ ಸಾಕು. ಅಲ್ಲದೆ, ಮಗನನ್ನು ನೋಡಿದರೆ ತಂದೆಗೆ ಆಗುವುದಿಲ್ಲ. ಮಗನಿಗೆ ತಂದೆಯನ್ನು ಕಂಡರೆ ಆಗುವುದಿಲ್ಲ. ಆದರೆ ಇಬ್ಬರೂ ಒಟ್ಟಿಗೆ ಕಾರಿನಲ್ಲಿ ಕುಳಿತುಕೊಂಡು ಹೋಗುವರು. ಹೊರಗೆ ನೋಡುವವರಿಗೆ `ಆಹಾ! ಎಷ್ಟು ಆನಂದವಾಗಿದ್ದಾರೆ' ಎಂದೆನಿಸುತ್ತದೆ. ಆದರೆ ಒಳಗೆ ಹೋಗಿ ನೋಡಿದರೆ ಮಾತ್ರ ನಿಜ ಗೊತ್ತಾಗುತ್ತದೆ. ಸಾತ್ತ್ವಿಕವಾದ ಬಾಳನ್ನು ನಡೆಸುವವರಿಗೆ ಶತ್ರುವನ್ನು ಕಂಡರೂ ಭಯವಿಲ್ಲ. ಅವನು ಹೊಡೆದರೂ ಕೂಡ, `ಏಕಪ್ಪಾ ಹೊಡೆಯುತ್ತೀಯೆ? ನಿನಗೆ ನಾನು ಏನು ಮಾಡಿದೆನು?' ಎಂದು ಕೇಳುವಷ್ಟು ಧೈರ್ಯವನ್ನು ಹೊಂದಿರುತ್ತಾನೆ, ಇತರರಿಗೆ ಈ ಧೈರ್ಯವಿರುವುದಿಲ್ಲ, ರಜೋಗುಣ, ತಮೋಗುಣವುಳ್ಳವರಿಗೆ ಯಾರನ್ನು ಕಂಡರೂ ಭಯ. ನಾವು ಸಾತ್ತ್ವಿಕವಾದ ಬಾಳನ್ನು ಬಾಳಬೇಕು. ಅದಕ್ಕೆ ಸಾತ್ತ್ವಿಕವಾದ ಹಾಗೂ ಪವಿತ್ರವಾದ ಆಹಾರ ಮುಖ್ಯ. ಅದು ಮಂತ್ರಪೂರ್ವಕವಾಗಿದ್ದರೆ ಇನ್ನೂ ವಿಶೇಷ. ಇಂಥ ವಿಷಯಗಳು ವೇದದಲ್ಲೂ, ಮನು ಸ್ಮೃತಿಯಂಥ ಧರ್ಮಶಾಸ್ತ್ರಗಳಲ್ಲಿಯೂ ಹೇಳಲ್ಪಟ್ಟಿವೆ. ಭಗವತ್ಪಾದರು, `ನಿರ್ಗುಣ ಪರವಸ್ತುವಿಗೆ ಗುಣವಿಲ್ಲ, ರೂಪವಿಲ್ಲ' ಎಂದು ಹೇಳುವರಾದ್ದರಿಂದ ಪೂಜೆಗಳ ಬಗ್ಗೆ ತಿಳಿಸಿಲ್ಲವೆಂದು ಕೆಲವರು ಹೇಳುತ್ತಾರೆ. ಸೂತ್ರಭಾಷ್ಯದಲ್ಲಿಯೇ ಕೆಲವೆಡೆಗಳಲ್ಲಿ ಭಗವತ್ಪಾದರು,

\begin{shloka}
`ಯಥಾ ಶಾಲಿಗ್ರಾಮೇ ವಿಷ್ಣು ಬುದ್ಧಿಃ'\\
(ಸಾಲಿಗ್ರಾಮದಲ್ಲಿ ವಿಷ್ಣುವೆನ್ನುವ ಭಾವನೆ)-ಎಂದೂ\\
`ಪ್ರತಿಮಾಯಾಂ ವಿಷ್ಣು ಬುದ್ಧಿಃ'\\
(ಪ್ರತಿಮೆಯಲ್ಲಿ ವಿಷ್ಣುವೆನ್ನುವ ಭಾವನೆ)
\end{shloka}

-ಎಂದೂ ಹೇಳಿದ್ದಾರೆ. ಇದಕ್ಕೆ ಏನು ಅರ್ಥ? `ಸಾಲಿಗ್ರಾಮದಲ್ಲಿ ವಿಷ್ಣುವೆನ್ನುವ ಭಾವನೆ ಇರುವಂತೆ' ಎನ್ನುವುದೇ ಆಗುತ್ತದೆ. ಸಾಲಿಗ್ರಾಮವನ್ನು ಬರೀ ಕಲ್ಲೆಂದು ತಿಳಿಯಬಾರದು. ಅದರಲ್ಲಿ ವಿಷ್ಣು ಭಗವಂತನೆನ್ನುವ `ಭಾವನೆ' ಇರಬೇಕೆನ್ನುವುದನ್ನು ಭಗವತ್ಪಾದರು ಒತ್ತಿ ಹೇಳಿದ್ದಾರೆ. ಆದ್ದರಿಂದ ಭಗವತ್ಪಾದರು ಸಗುಣವಾದ ಉಪಾಸನೆಯನ್ನು ಒಪ್ಪುತ್ತಾರೆಯೇ ಇಲ್ಲವೇ ಎಂದರೆ, ಇದೇ ರೀತಿ ಸಗುಣವಾದ ಉಪಾಸನೆಯನ್ನು ಮಾಡಿದರೆ, ನಿನಗೆ `ನಿಶ್ಚಲತ್ವ ಬಂದರೆ, ಕರ್ಮ ತಾನಾಗಿಯೇ ಬಿಟ್ಟು ಹೋಗುವುದು. ಈಗ ಸಗುಣ ಆರಾಧನೆಯನ್ನು ಬಿಟ್ಟು ಬಿಡಬೇಕೆಂದೋ, ಅಥವಾ ಆಗ ಬಿಟ್ಟು ಹೋಯಿತಲ್ಲಾ ಎಂದೋ ಚಿಂತಿಸಬೇಕಾದ ಆವಶ್ಯಕತೆ ಇಲ್ಲ. ಮತ್ತೆ ಬಂದು ಸೇರಿದರೆ, ನಾನು ಬಿಟ್ಟು ಬಿಟ್ಟೆನು! ಮತ್ತೆ ಬಂದು ಸೇರಿತಲ್ಲಾ' ಎಂದೂ ಚಿಂತಿಸಬೇಕಾಗಿಲ್ಲ. ಪ್ರಾರಬ್ಧಕರ್ಮ ಎನ್ನುವುದು ಒಂದಿದೆ. ಹೇಗೋ ಈ ಶರೀರವನ್ನು ಕಾಪಾಡಿಕೊಳ್ಳುವುದನ್ನು ಮಾತ್ರ ಮನಸ್ಸಿನಲ್ಲಿ ಇಟ್ಟುಕೊಳ್ಳಬೇಕು. ಆ ಪರವಸ್ತುವಿನಿಂದ ಮಾತ್ರ ಮನಸ್ಸನ್ನು ವಿಚಲಿತಗೊಳಿಸಬೇಕಾಗಿಲ್ಲ.

\begin{shloka}
`ಪ್ರಾರಬ್ಧಂ ಪುಷ್ಯತಿ ವಪುರಿತಿ ನಿಶ್ಚಲಃ......'
\end{shloka}

(ಪ್ರಾರಬ್ಧ ಶರೀರವನ್ನು ಪೋಷಿಸುತ್ತದೆ ಎನ್ನುವ ನಿಶ್ಚಯದೊಡನೆ ಚಂಚಲವಾಗಿಲ್ಲದೆ ಇರುವವನು......)

ಎಂದು ಹೇಳಲ್ಪಟ್ಟಿದೆ.

\begin{shloka}
`ಪ್ರತಿಮಾಸು ಅಲ್ಪ ಬುದ್ಧಿಃ'
\end{shloka}

-ಎಂದು ಶಂಕರರು ಅಲ್ಪ ಬುದ್ಧಿಯನ್ನು ಹೇಳಿಲ್ಲ. ಯಾರಿಗೆ ನಿರ್ಗುಣ, ನಿರಾಕಾರ ಬ್ರಹ್ಮದಲ್ಲಿ ಮನಸ್ಸನ್ನು ನಿಲ್ಲಿಸಲು, ಸಾಧ್ಯವಿಲ್ಲವೋ ಅವರೆಲ್ಲರೂ ಕೂಡ

\begin{shloka}
`ಯಥಾ ಶಾಲಿಗ್ರಾಮೇ ವಿಷ್ಣು ಬುದ್ಧಿಃ'
\end{shloka}

-ಎನ್ನುವಂತೆ ಮಾರ್ಗವನ್ನು ಅನುಸರಿಸಬಹುದು. ಇದನ್ನು ಅವರು ಅನೇಕ ಕಡೆ ಹೇಳಿದ್ದಾರೆ. ಇದನ್ನು ನೋಡಿದರೆ ಭಗವತ್ಪಾದರೂ ಪರಮ ವೈಷ್ಣವರಾಗಿದ್ದಾರಲ್ಲಾ ಎಂದು ಕೂಡ ಕೆಲವರಿಗೆ ತೋರುವುದು. ಎಲ್ಲಾ ಒಂದೇ ಎಂದರು. ಇನ್ನೂ ಕೂಡ ಮಹೇಶ್ವರನನ್ನು ಕುರಿತು


\begin{shloka}
(`ಯಥಾ ಬಾಣಲಿಂಗೇ ಮಹೇಶ್ವರ ಬುದ್ಧಿಃ')\\
(ಬಾಣಲಿಂಗದಲ್ಲಿ ಮಹೇಶ್ವರನೆನ್ನುವ ಭಾವನೆ.)
\end{shloka}

-ಎನ್ನುವದೇ ಆಗುವುದು. ಆದ್ದರಿಂದ ಮಹೇಶ್ವರನನ್ನು ಆರಾಧನೆ ಮಾಡಬೇಡ ಎನ್ನುವುದು ತಾತ್ಪರ್ಯವಲ್ಲ. ವಾಸುದೇವನಿಗೂ ವಾಮದೇವನಿಗೂ ಭೇದವನ್ನು ಎಣಿಸಬೇಕಾಗಿಲ್ಲ. ವಾಸುದೇವ, ವಾಮದೇವ ಇವರಲ್ಲಿ ವ್ಯತ್ಯಾಸವಾಗುವ ಅಕ್ಷರಗಳೆಂದರೆ ಸು-ಮ. `ಸುಮ' ವೆಂದರೆ ಪುಷ್ಪ. ತಾತ್ಪರ್ಯವೇನೆಂದರೆ, ಪುಷ್ಪ ಬೇರೆಯಾದರೂ ಪಡೆಯಬೇಕಾದ ಫಲ (ಹಣ್ಣು) ಒಂದೇ. ಆದ್ದರಿಂದ ದೇವತೆಗಳಲ್ಲಿ ಭೇದವನ್ನು ಎಣಿಸಬೇಕಾಗಿಲ್ಲ; ದೊರೆಯುವ ಫಲ ಒಂದೇ. ಈ ಬುದ್ಧಿಯೊಡನೆ ಸಾತ್ತ್ವಿಕವಾದ ಆಹಾರವನ್ನು ತೆಗೆದುಕೊಳ್ಳುತ್ತಾ, ಸಾತ್ತ್ವಿಕವಾದ ಬಾಳನ್ನು ನಡೆಸಲು ಪ್ರಯತ್ನಿಸಬೇಕೆಂದು ಹೇಳಿ ಇಲ್ಲಿಗೆ ಮುಗಿಸುತ್ತಿದ್ದೇನೆ.















































