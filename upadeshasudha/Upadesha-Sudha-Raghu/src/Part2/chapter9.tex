\chapter{ಮನುಷ್ಯಜನ್ಮ ಮತ್ತು ಶ್ರೇಷ್ಠ ಯೋಗ್ಯತೆಗಳು}\label{chap9}

\begin{shloka}
ವಿಶುದ್ಧಜ್ಞಾನದೇಹಾಯ ತ್ರಿವೇದೀ ದಿವ್ಯಚಕ್ಷುಷೇ |\\
ಶ್ರೇಯಃಪ್ರಾಪ್ತಿನಿಮಿತ್ತಾಯ ನಮಸ್ಸೋಮಾರ್ಧಧಾರಿಣೇ ||\\
ನಮಾಮಿ ಯಾಮಿನೀನಾಥಲೇಖಾಲಂಕೃತಕುಂತಲಾಮ್ |\\
ಭವಾನೀಂ ಭವಸನ್ತಾಪನಿರ್ವಾಪನ ಸುಧಾನದೀಮ್ ||
\end{shloka}

``ಯಾರಿಗೆ ಭಗವಂತನ ವಿಷಯದಲ್ಲಿ ಎಲ್ಲೆ ಇಲ್ಲದಷ್ಟು ಭಕ್ತಿ ಇರುವುದೋ, ಭಗವಂತನ ವಿಷಯದಲ್ಲಿರುವ ಭಕ್ಕ್ತಿಯಂತೆಯೇ ಗುರುವಿನ ವಿಷಯದಲ್ಲಿಯೂ ಭಕ್ತಿ (ಯಾರಿಗೆ) ಇರುವುದೋ, ಅಂಥ ಮಾಹಾತ್ಮರಲ್ಲಿ, ಹೇಳಲ್ಪಟ್ಟ ಅರ್ಥಗಳೂ, ಹೇಳದೇ ಇರುವ ಅರ್ಥಗಳೂ (ಶಾಸ್ತ್ರಗಳು) ತಾನಾಗಿಯೇ ಪ್ರಕಾಶಿಸುತ್ತವೆ'' - ಎನ್ನುವ ವಾಕ್ಯ ಉಪನಿಷತ್ತಿನಲ್ಲಿ ಬರುತ್ತದೆ. ಇದರ ತಾತ್ಪರ್ಯವೇನೆಂದರೆ ಭಕ್ತಿ ಎನ್ನುವುದು ಜ್ಞಾನಕ್ಕೆ ಸಾಧನವಾಗಿದೆ. ಭಕ್ತಿಗೆ ಅವಲಂಬನವಾಗಿ ದೈವ ಇರಬೇಕೆಂದು ನಾವು ಇಟ್ಟುಕೊಳ್ಳುತ್ತೇವೆ. ಭಕ್ತಿಯಲ್ಲಿ ಅರ್ತ, ಅರ್ಥಾರ್ಥಿಗೆ ಸಂಬಂಧಪಟ್ಟ ಎರಡು ಭೇದಗಳು ಇವೆ. ಅಂಥ ಭಕ್ತಿಗಳೆಲ್ಲವೂ ಒಂದು ವಿಧವಾದ ವ್ಯಾಪಾರವೆಂದೇ ನಾವು ಹೇಳಬೇಕು. ಆಗ ನಾವು ಮಾಡುವ ಭಕ್ತಿಗೆ ಭಗವಂತನು ಫಲವನ್ನು ಕೊಟ್ಟರೆ ನಾವು ಭಗವಂತನಲ್ಲಿ ಭಕ್ತಿಯುಳ್ಳವರಾಗಿರಬಹುದು. ಇಲ್ಲದಿದ್ದರೆ ಭಕ್ತಿಯೂ ಹೋಗಿಬಿಡಬಹುದು. ಆದರೆ ಜ್ಞಾನವನ್ನು ಅಪೇಕ್ಷೆಪಡುವವನು ಭಗವಂತನೊಡನೆ ವ್ಯಾಪಾರಿಯಂತೆ ಇರದೆ ವಿಶೇಷವಾಗಿ ಭಕ್ತಿಯುಳ್ಳವನಾಗಿರುವನು. ಅಂಥ ಭಕ್ತಿಗೆ ಫಲವೇನೆಂದರೆ ಜ್ಞಾನವೇ ಫಲ.

ಜ್ಞಾನ ಉಂಟಾದಮೇಲೆ

\begin{shloka}
``ಜ್ಞಾನೀತ್ವಾತ್ಮೈವ ಮೇ ಮತಮ್''
\end{shloka}

(ಜ್ಞಾನಿ ನಾನೇ ಎನ್ನುವುದು ನನ್ನ ಅಭಿಮತ) -ಎಂದು ಹೇಳಿದಂತೆ, ಭಕ್ತನಿಗೂ ಭಗವಂತನಿಗೂ ಯಾವ ವಿಧವಾದ ವ್ಯತ್ಯಾಸವೂ ಇರುವುದಿಲ್ಲ. ``ಸಿಂಧುಸ್ಸರಿದ್ವಲ್ಲಭಂ'' ಎಂದು ಹೇಳಿದಂತೆ ಹೇಗೆ ನದಿಗಳು ಸಮುದ್ರವನ್ನು ಸೇರಿದಮೇಲೆ ಅವುಗಳಲ್ಲಿ ವ್ಯತ್ಯಾಸವಿರುವುದಿಲ್ಲವೋ, ಹಾಗೆಯೇ ಭಕ್ತನಾದವನು ಭಗವಂತನೊಡನೆ ಒಂದಾಗಿ ಬಿಡುತ್ತಾನೆ. ಆದ್ದರಿಂದ ಯಾವುದಾದರೂ ಒಂದು ಫಲವನ್ನು ಎದುರು ನೋಡುತ್ತಾ ನಾವು ಮಾಡುವ ಭಕ್ತಿ ಶ್ರೇಷ್ಠವಾದುದಲ್ಲ. ವ್ಯಾಪಾರದಂತಿರುವ ಭಕ್ತಿಯನ್ನು ತೆಗೆದುಕೊಳ್ಳುವವರು ತಾವು ಯಾವುದಾದರೂ ಫಲಕ್ಕಾಗಿ ಭಗವಂತನಿಗೆ ಶರಣಾಗಿ, ಅದರಿಂದ ಫಲವನ್ನು ಪಡೆದ ಮೇಲೆ ``ನಾವು ಭಗವಂತನಿಂದ ಇನ್ನು ಯಾವ ಫಲವನ್ನು ಪಡೆಯಬೇಕು. ನಮ್ಮ ಮಕ್ಕಳೆಲ್ಲಾ ಒಳ್ಳೆಯ ಕೆಲಸದಲ್ಲಿದ್ದಾರೆ. ನಾವೂ ಒಳ್ಳೆಯ ಆರೋಗ್ಯವಂತರಾಗಿದ್ದೇವೆ. ಹಾಗಿರುವಾಗ ಏತಕ್ಕೆ ಭಗವಂತನನ್ನು ಪ್ರಾರ್ಥನೆ ಮಾಡಬೇಕು'' -ಎಂದು ಯೋಚಿಸುವರು. ಆದರೆ ಜ್ಞಾನವನ್ನು ಪಡೆಯಬೇಕೆಂದುಕೊಳ್ಳುವವನು ಹಾಗಲ್ಲ. ಒಂದು ಸಕ್ಕರೆ ಬೊಂಬೆ ಸಮುದ್ರದ ಆಳವನ್ನು ನೋಡಲು ಅದರಲ್ಲಿ ಇಳಿಯಿತು. ಅನಂತರ ಆ ಬೊಂಬೆಯನ್ನು ನಾವು ಹೇಗೆ ನೋಡಬಹುದು? ಸಮುದ್ರದ ಆಳವನ್ನು ನೋಡಲು ಹೋದ ಆ ಬೊಂಬೆ ಕೆಳಗೆ ಹೋಗುತ್ತಾ ತಾನು ಚಿಕ್ಕದಾಗುತ್ತಲೇ ಹೋಗುತ್ತದೆ. ಕೊನೆಗೆ ಆ ಬೊಂಬೆ ಕಾಣುವುದಿಲ್ಲ. ಅದು ಸಮುದ್ರದಲ್ಲಿ ಕರಗಿಹೋಗುತ್ತದೆ. ಇದೇ ಜೀವನ ಸ್ಥಿತಿ. ಆದ್ದರಿಂದ ಭಕ್ತನಾದವನು ಭಗವಂತನ ಭಕ್ತ ಎನ್ನುವ ಸಮುದ್ರಕ್ಕೆ ಹೋದರೆ, ಆ ಬೊಂಬೆ ಸಮುದ್ರವನ್ನು ಬಿಟ್ಟು ಹೊರಗೆ ಬಾರದಂತೆ, ಜ್ಞಾನವನ್ನು ಅಪೇಕ್ಷಿಸುವವನು ಜ್ಞಾನವನ್ನು ಪಡೆದ ಮೇಲೆ ಭಗವಂತನೊಡನೆ ಒಂದಾಗಿ ಬಿಡುತ್ತಾನೆ. ಆದ್ದರಿಂದ ಜ್ಞಾನವನ್ನು ಅಪೇಕ್ಷಿಸುವ ಜಿಜ್ಞಾಸುವಿನಲ್ಲಿರುವ ಭಕ್ತಿ ಬಹಳ ಶ್ರೇಷ್ಠವಾದುದು. ಅಂಥ ಜಿಜ್ಞಾಸುವಿಗೆ ``ಪರಾ ಭಕ್ತಿ'' ಎನ್ನುವುದು ಇದೆ, ಜ್ಞಾನವನ್ನು ಪಡೆಯಬೇಕೆನ್ನುವ ಅವನ ಅಪೇಕ್ಷೆ ಪೂರ್ತಿಯಾಗಿ ನೆರವೇರಬೇಕಾದರೆ ``ಯಥಾ ದೇವೇ ತಥಾ ಗುರೌ'' -ಎಂದು ಹೇಳಿರುವುದರಿಂದ ಹೇಗೆ ದೇವರ ಮೇಲೆ ಭಕ್ತಿ ಇಟ್ಟುಕೊಂಡಿದ್ದಾನೋ ಹಾಗೆಯೇ ಗುರುವಿನ ಮೇಲೂ ಭಕ್ತಿಯನ್ನು ಇಟ್ಟುಕೊಂಡಿರಬೇಕೆಂದು ಹೇಳಲಾಗಿದೆ. ಇಂಥ ಭಕ್ತಿ ಇಟ್ಟುಕೊಂಡಿರುವವನಿಗೆ ಯಾವ ಫಲ ದೊರೆಯುತ್ತದೆ ಎಂದು ಕೇಳಿದರೆ ಅವನ ಜಿಜ್ಞಾಸೆ ಪೂರ್ಣವಾಗುತ್ತದೆ.

\begin{shloka}
``ತಸ್ಮೈತೇ ಕಥಿತಾಹ್ಯರ್ಥಾಃ ಪ್ರಕಾಶನ್ತೇ ಮಹಾತ್ಮನಃ''
\end{shloka}

ಗುರು ಶಿಷ್ಯನಿಗೆ ಕಲಿಸುತ್ತಾರೆ. ಶಿಷ್ಯನ ಭಕ್ತಿ ವಿಶೇಷವಾಗಿದ್ದರೆ ಗುರು ಕಲಿಸುವ ಅವಶ್ಯಕತೆ ಇಲ್ಲದೆ ಇದ್ದರೂ ಅವರು ಆಶೀರ್ವಾದ ಮಾಡುತ್ತಲೇ ತತ್ತ್ವದ ತಾತ್ಪರ್ಯ ಶಿಷ್ಯನಿಗೆ ತಿಳಿಯುತ್ತದೆ. ಇದಕ್ಕೆ ಉದಾಹರಣೆಯಾಗಿ ನಾವು ಏಕಲವ್ಯನ ಕಥೆಯನ್ನು ತೆಗೆದುಕೊಳ್ಳಬಹುದು. ಏಕಲವ್ಯನು ಗುರುವಿನ ಮೇಲೆ ಅಪಾರವಾಗಿ ಭಕ್ತಿ ಇಟ್ಟುಕೊಂಡಿದ್ದನು. ಆದರೆ ಗುರುವಿಗೆ ಶಿಷ್ಯನ ಮೇಲೆ ಅಷ್ಟು ವಿಶ್ವಾಸವಿರಲಿಲ್ಲ. ಗುರುವಿಗೆ ಮೊದಲು ಅರ್ಜುನನು ಸಿಕ್ಕಿದ್ದರೂ. ತಾನು ಅರ್ಜುನನಿಗೆ ಯಾವ ರೀತಿಯಲ್ಲೂ ಕಡಿಮೆಯಲ್ಲವೆಂದು ತೋರಿಸಿದನು. ಗುರುಭಕ್ತಿಯಿಂದ ಎಂಥ ಫಲ ದೊರೆಯುತ್ತದೆ ಎನ್ನುವುದನ್ನು ನಾವು ಸಾಧಾರಣವಾಗಿ ಲೌಕಿಕ ವಿಷಯದಲ್ಲೇ ನೋಡುವಾಗ ಆಧ್ಯಾತ್ಮ ವಿಷಯದ ಬಗ್ಗೆ ಹೇಳಬೇಕಾಗಿಲ್ಲ.


\begin{shloka}
%page 253
\end{shloka}














