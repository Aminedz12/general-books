\chapter{ಗುರು ಎನ್ನುವವರು ಯಾರು?}\label{chap12}

\begin{shloka}
ಗಂಗಾಪೂರಪ್ರಚಲಿತ ಜಟಾವಸ್ತ್ರ ಭೋಗೀಂದ್ರಭೀತಾಂ\\
ಆಲಿಂಗತೀಮಚಲತನಯಾಂ ಸಸ್ಮಿತಂ ವೀಕ್ಷಮಾಣಃ|\\
ಲೀಲಾಪಾಂಗೈಃ ಪ್ರಣತಜನತಾಂ ನಂದಯಶ್ಚಂದ್ರಮೌಲಿಃ\\
ಮೋಹಧ್ವಾನ್ತಂ ಹರತು ಪರಮಾನಂದಮೂರ್ತಿಶ್ಶಿವೋ ನಃ||
\end{shloka}

ಶ್ರೀ ಶಂಕರ ಭಗವತ್ಪಾದರು ಗುರು ಎಂದರೆ ಯಾರು? ಎಂದು ತಾವೇ ಪ್ರಶ್ನಿಸಿಕೊಂಡು `ಪ್ರಶ್ನೋತ್ತರಮಾಲಿಕಾ'ಎನ್ನುವ ಅರವತ್ತಏಳು ಶ್ಲೋಕಗಳುಳ್ಳ ಗ್ರಂಥದಲ್ಲಿ ಅದಕ್ಕೆ ಜವಾಬನ್ನು ಹೇಳಿರುತ್ತಾರೆ. ಅಷ್ಟು ಚಿಕ್ಕ ಗ್ರಂಥದಲ್ಲಿ ವ್ಯಾವಹಾರಿಕವಾಗಿರುವ ವಿಷಯಗಳೂ, ಪಾರಮಾರ್ಥಿಕವಾಗಿರುವ ವಿಷಯಗಳೂ ಸೇರುವಂತೆ ಮಾಡಿದ್ದಾರೆ. ಸಂಸ್ಕೃತದಲ್ಲಿ ಸ್ವಲ್ಪ ಜ್ಞಾನವಿದ್ದು, ರಘುವಂಶದಲ್ಲಿ ಸ್ವಲ್ಪ ಓದಿದ್ದರೂ ಅಂಥಹವನಿಗೂ ಅರ್ಥವಾಗುವಂತೆ ಆ ಚಿಕ್ಕ ಗ್ರಂಥ ರಚಿಸಲ್ಪಟ್ಟಿದೆ. 

ಅದರಲ್ಲಿ ಶಂಕರರು ಒಂದು ಪ್ರಶ್ನೆ ಕೇಳಿದರು.

\begin{shloka}
``ಕೋ ಗುರುಃ?"
\end{shloka}

(ಗುರು ಎನ್ನುವವರು ಯಾರು?)

ಎನ್ನುವುದೇ ಅದು. 

\begin{shloka}
``ಅಧಿಗತ ತತ್ತ್ವಃ"
\end{shloka}

(ತತ್ತ್ವವನ್ನು ತಿಳಿದವನು.)

ಎಂದು ಮೊದಲನೆಯ ಪಾದವನ್ನು ಪೂರ್ತಿ ಮಾಡಿದ್ದಾರೆ. ತತ್ತ್ವವೆಂದರೆ ``ಪ್ರಪಂಚ ಎಂತಹುದು? ಅದು ಕೊನೆಗೆ ಎಲ್ಲಿಗೆ ಹೋಗಿ ಸೇರುವುದು? ಅದು ಸ್ವರೂಪ ಎಂತಹುದು?" ಎಂದು ಯಾರು ತಿಳಿದುಕೊಂಡಿರುವರೋ ಅವರೇ ತತ್ತ್ವವನ್ನು ತಿಳಿದು ಕೊಂಡವರು. 

\begin{shloka}
``ಯತೋ ವಾ ಇಮಾನಿ ಭೂತಾನಿ ಜಾಯನ್ತೇ\\
ಯೇನ ಜಾತಾನಿ ಜೀವನ್ತಿ ಯತ್ಪ್ರಯನ್ತ್ಯಭಿಸಂವಿಶನ್ತಿ ತತ್ ವಿಜಿಜ್ಞಾಸಸ್ವ|"
\end{shloka}

(ಯಾವುದರಿಂದ ಎಲ್ಲಾ ಜೀವರುಗಳು ಹುಟ್ಟುತ್ತವೋ, ಯಾವುದರಿಂದ ಹುಟ್ಟಿದಮೇಲೆ ಅವುಗಳು ಬಾಳುತ್ತವೋ, ಯಾವುದನ್ನು ಉದ್ದೇಶಿಸಿ ಅವು ಚಲಿಸುತ್ತವೋ, ಯಾವುದರಲ್ಲಿ ಒಂದಾಗಿ ಬಿಡುತ್ತವೋ, ಅದನ್ನು ತಿಳಿದುಕೊಳ್ಳುವವನಾಗು)

-ಎನ್ನುವಂತೆ ಈ ಪ್ರಪಂಚವೆಲ್ಲಾ ಪರಬ್ರಹ್ಮದಿಂದಲೇ ಬಂದಿರುವುದು, ಪರವಸ್ತುವಿನ ಅನುಗ್ರಹದಿಂದಲೇ ಇದು ಇರುವುದು. ಇದು ಕೊನೆಗೆ ಹೋಗಿ ಸೇರುವುದು ಕೂಡ ಪರವಸ್ತುವಿನಲ್ಲಿಯೇ. ಆದ್ದರಿಂದ ಈ ಪ್ರಪಂಚದಲ್ಲಿ ಪಾರಮಾರ್ಥಿಕ ತತ್ತ್ವ ಪರವಸ್ತುವಾದ ಶಿವರೂಪವನ್ನು ಬಿಟ್ಟು ಬೇರೆ ಯಾವುದೂ ಅಲ್ಲ. ಈ ಶಿವನೇ ಪ್ರಪಂಚವಾಗಿ ತಾಂಡವ ಮಾಡುತ್ತಿದ್ದಾನೆ. ಅವನು ಸ್ವರೂಪವನ್ನು ಪಡೆಯುವುದರಿಂದ ಯಾವುದನ್ನೂ ತನಗಾಗಿ ಸಾಧಿಸಿಕೊಳ್ಳುವುದಿಲ್ಲ. ಆದ್ದರಿಂದಲೇ `ಪ್ರಪಂಚವೆಲ್ಲಾ ಮಿಥ್ಯೆ' ಎಂದು ಅದ್ವೈತಶಾಸ್ತ್ರದಲ್ಲಿ ಹೇಳುವುದು. `ಏಕೆ ಪ್ರಪಂಚ ಮಿಥ್ಯೆ? ನಾನು ಮಾತನಾಡುತ್ತಿರುವುದು ಸುಳ್ಳೆ? ಇವುಗಳು ಹೇಗೆ ಸುಳ್ಳಾಗಲು ಸಾಧ್ಯ? ನೀನು ಊಟ ಮಾಡುತ್ತಿರುವೆ. ನಾನು ನೋಡುತ್ತಿರುವೆನು. ನೀನು ಮಾತನಾಡುತ್ತಿದ್ದೀಯೆ. ನಾನು ಕೇಳುತ್ತಿದ್ದೇನೆ' -ಹೀಗೆ ಒಬ್ಬನು ಕೇಳಿದರೂ ಇವುಗಳೆಲ್ಲವೂ ಪರಶಿವನ ಶಕ್ತಿಯೇ ಹೊರತು ಬೇರೆ ಯಾವುದೂ ಅಲ್ಲ. ಈ ಪರಮಶಿವನು ಕರ್ಮಗಳ ಪ್ರಕಾರ `ಎಷ್ಟು ಕಾಲ ಈ ಪ್ರಪಂಚವನ್ನು ನಡೆಸಬೇಕು' ಎಂದು ಸಂಕಲ್ಪವನ್ನು ಮಾಡಿದ್ದಾನೋ ಅಷ್ಟುಕಾಲ ನಡೆಯುವುದು. ಮನುಷ್ಯನು `ನಾನೇ ಸ್ಥಿರವಾಗಿ ಇರುವವನು' ಎಂದು ಭಾವಿಸಿದರೆ ಅವನು ಹಾಗೆ ಇರಲಾಗುವುದಿಲ್ಲ. ಅವನು ಇರುವುದಕ್ಕೆ ಮೊದಲು ಜಾಗವೇ ಇಲ್ಲ. ಇದು ಹೇಗೆ ಸಾಧ್ಯ? ಒಂದು ಉದಾಹರಣೆ ನೋಡೋಣ -ಒಬ್ಬ ಐಂದ್ರಜಾಲಿಕನು, ಸಾಧಾರಣವಾಗಿ ನೋಡಿದರೆ ಏನೂ ಇಲ್ಲದ ಒಂದು ಜಾಗದಲ್ಲಿ ಮಾವಿನಮರ ಒಂದನ್ನು ತೋರಿಸುತ್ತಾನೆ. ಅಷ್ಟೇ ಅಲ್ಲದೆ, ಅದರಲ್ಲಿ ಹಣ್ಣುಗಳಿರುವುದನ್ನೂ ತೋರಿಸುತ್ತಾನೆ. ಆ ಹಣ್ಣುಗಳನ್ನು ತಿನ್ನುವ ಗಿಣಿಗಳನ್ನೂ ತೋರಿಸುತ್ತಾನೆ. ಆ ಗಿಣಿಗಳು ಮಾಡುವ ಶಬ್ದವೂ ನಮಗೆ ಕೇಳಿಬರುತ್ತದೆ. ಇಷ್ಟೆಲ್ಲವನ್ನೂ ನಾವು ಒಂದು ಟಾರು ರಸ್ತೆ ಮಧ್ಯದಲ್ಲಿ ಕಾಣುತ್ತೇವೆ. ಟಾರು ಹಾಕಿರುವ ರಸ್ತೆಯ ಮಧ್ಯದಲ್ಲಿ ಮಾವಿನ ಮರ ಹೇಗೆ ಬಂದೀತು? ಅದರಲ್ಲಿ ಹಣ್ಣುಗಳು ಹೇಗೆ ಬಂದವು?  ಅವುಗಳಲ್ಲಿ ಗಿಣಿಗಳು ಹೇಗೆ? -ಇವುಗಳೆಲ್ಲವೂ ಐಂದ್ರಜಾಲಿಕನ ಸಾಮರ್ಥ್ಯದಿಂದಾಗಿ ಸೃಷ್ಟಿಸಲ್ಪಟ್ಟವು. ಅವನು ಎಷ್ಟು ಹೊತ್ತು `ಎಲ್ಲರಿಗೂ ಕಾಣುವಂತಾಗಲಿ' ಎಂದು ಸಂಕಲ್ಪವನ್ನು ಮಾಡಿರುತ್ತಾನೋ ಅಲ್ಲಿಯವರೆಗೆ ನೋಡುವವರಿಗೆ ನೋಟ ಕಾಣಿಸುತ್ತದೆ. ಅವನು ಅದಕ್ಕಾಗಿ ಮಂತ್ರಗಳನ್ನೋ, ಮೂಲಿಕೆಗಳನ್ನೋ ಇಟ್ಟುಕೊಂಡಿರುತ್ತಾನೆ. ಅದೇ ರೀತಿ ಭಗವಂತನ ಶಕ್ತಿಗೆ ಮಿಗಿಲಾದುದು ಯಾವುದೂ ಇಲ್ಲ. ಅವನ ಮಾಯಾ ವಿಲಾಸದಿಂದಾಗಿ ಈ ಪ್ರಪಂಚವೇ ಇದೆ. ನಾವು ನೋಡುತ್ತಿರುವವರೆಗೆ ಈ ಪ್ರಪಂಚ ಮಿಥ್ಯೆಯಲ್ಲ. ಆದರೆ ನಿಜಕ್ಕೂ ಇದು ಪರಮಶಿವನನ್ನು ಬಿಟ್ಟು ಯಾವ ತತ್ತ್ವವೂ ಇಲ್ಲ. ಅದೇ ಮುಖ್ಯವಾದ ವಸ್ತು. ಅದನ್ನು ಬಿಟ್ಟು ಇರಬಲ್ಲ ವಸ್ತು ಯಾವುದೂ ಇಲ್ಲ. 

ಅಂಥ ಪರವಸ್ತುವನ್ನು ಅರಿತವನೇ ಗುರುವೆಂದು ಶಂಕರರು ಹೇಳಿದರು. ಗುರು ತಾನು ಅರಿತಿದ್ದರೆ ಸಾಕೇ? ಬೇರೆ ಏನಾದರೂ ವಿಶೇಷವಿದೆಯೇ? -ಎಂದು ಕೇಳಿದರೆ ಅದಕ್ಕೆ 

\begin{shloka}
``ಶಿಷ್ಯಹಿತಾಯ ಉದ್ಯತಃ ಸತತಂ"
\end{shloka}

(ಶಿಷ್ಯ ಹಿತಕ್ಕಾಗಿ ಯಾವಾಗಲೂ ಪ್ರಯತ್ನಪಡುವವನು.)

-ಎಂದು ಹೇಳಿದರು. ಶಂಕರ ಭಗವತ್ಪಾದರು `ಮಹಾತ್ಮನು ಯಾರು' ಎನ್ನುವುದಕ್ಕೆ ವಿವರಣೆ ಹೀಗೆ ಕೊಡುತ್ತಾರೆ-

\begin{shloka}
``ಶಾನ್ತಾ ಮಹಾನ್ತೋ ನಿವಸನ್ತಿ ಸನ್ತೋ\\
ವಸನ್ತವಲ್ಲೋಕಹಿತಂ ಚರನ್ತಃ|\\
ತೀರ್ಣಾ ಸ್ಸ್ವಯಂ ಭೀಮಭವಾರ್ಣವಂ ಜನಾನ್\\
ಅಹೇತುನಾಽನ್ಯಾನಪಿ ತಾರಯನ್ತಃ||''
\end{shloka}

[ವಸಂತಕಾಲದಂತೆ ಪ್ರಪಂಚಕ್ಕೆ ಒಳ್ಳೆಯದನ್ನು ಮಾಡುತ್ತಾ, ತಾವು ಸ್ವತಃ (ಭವಸಮುದ್ರವನ್ನು) ದಾಟಿದವರು ಆದರೂ ಕೂಡ, ಕಾರಣವಿಲ್ಲದೇನೆ ಇತರ ಜನರನ್ನೂ ಸಹ ಭಯಂಕರವಾದ ಭವಸಮುದ್ರದಿಂದ ದಾಟಿಸುತ್ತಾ  ಶಾಂತಸ್ವಭಾವವುಳ್ಳವರಾಗಿ ಸಾಧುಗಳಾದ ಮಹಾತ್ಮರು (ಪ್ರಪಂಚದಲ್ಲಿ) ಇರುತ್ತಾರೆ.]

ಕೆಟ್ಟದಾಗಲಿ, ಒಳ್ಳೆಯದಾಗಲಿ ಯಾರಿಗಾದರೂ, ಯಾವುದಾದರೂ ತಿಳಿದಿದ್ದರೆ ಅದನ್ನು ಇನ್ನೊಬ್ಬರಿಗೆ ಹೇಳಿಕೊಡಬೇಕೆನ್ನುವ ಭಾವನೆ ಸಾಮಾನ್ಯವಾಗಿ ಮನಸ್ಸಿಗೆ ಬರುತ್ತದೆ. ಅದನ್ನು ಕೆಟ್ಟ ವಿಷಯದಲ್ಲಿ ಹೇಳಿದರೆ ಕೇಳುವುದಕ್ಕೆ ಮಧುರವಾಗಿರಬಹುದು; ಒಳ್ಳೆಯ ವಿಷಯದಲ್ಲಿ ಹೇಳಿದರೆ ಕೇಳುವುದಕ್ಕೆ ಮಧುರವಲ್ಲದೆಯೂ ಇರಬಹುದು.

ಒಬ್ಬ ಕೆಟ್ಟ ಹುಡುಗನಿದ್ದನು. ಅವನು ಒಂದು ಆಂಜನೇಯ ವಿಗ್ರಹವನ್ನು ನೋಡುತ್ತಿದ್ದನು. ಅವನು ನೋಡಿದನು ವಿಗ್ರಹದ ಬಾಲ ಉದ್ದವಾಗಿದೆ. ಬಾಲದ ಕೊನೆಯಲ್ಲಿ ಒಂದು ಮಣಿ ಇದೆ. ಆ ಮಣಿಯಲ್ಲಿ ಏನಿದೆ ಎಂದು ನೋಡಲು ಕೈಯನ್ನು ಒಳಗೆ  ಹಾಕಿದನು. `ಅಯ್ಯೋ' ಎಂದು ಕಿರುಚುತ್ತಾ ಕೈಯನ್ನು ಹೊರಗೆ ತೆಗೆದನು. ಅದನ್ನು ಕಂಡ ಇನ್ನೊಬ್ಬ ಹುಡುಗನು `ಏಕೆ ಕಿರುಚುತ್ತೀಯೆ?' ಎಂದು ಕೇಳಿದನು. ಅದಕ್ಕೆ ಆ ಹುಡುಗನು `ಕೈಯನ್ನು ಹಾಕಿ ನೋಡಿದರೆ ಬಹಳ ಚೆನ್ನಾಗಿರುತ್ತದೆ, ನೀನೂ, ಹಾಕಿ ನೋಡು, ಎಂದನು. ಆ ಹುಡುಗನೂ ಕೈಯನ್ನು ಹಾಕಿದನು. ಕೊನೆಗೆ ಅವನೂ `ಅಯ್ಯೋ' ಎಂದು ಕಿರುಚುತ್ತಾ ಕೈಯನ್ನು ಹೊರಗೆ ತೆಗೆದನು, ಏಕೆಂದರೆ ಒಳಗೆ ಒಂದು ಚೇಳು ಇದ್ದಿತು. ಕೆಟ್ಟ ಹುಡುಗನು ತಾನು ಕಷ್ಟವನ್ನು ಅನುಭವಿಸಿದುದು ಸಾಲದೆಂದು ಮತ್ತೊಬ್ಬನಿಗೂ ಕಷ್ಟವನ್ನು ಕೊಟ್ಟನು. ಸತ್ಪುರುಷನಾಗಿದ್ದರೆ ಮಾತ್ರ ಕಷ್ಟವನ್ನು ಅನುಭವಿಸಿ ಇನ್ನೊಬ್ಬನಿಗೆ ಹಾಗೆ ಆಗಬಾರದೆಂದುಕೊಳ್ಳುವನು. ಮನುಷ್ಯನ ಸ್ವಭಾವ ``ತಾನು ಕಷ್ಟವನ್ನು ಅನುಭವಿಸಿದರೆ ಇನ್ನೊಬ್ಬನೂ ಕಷ್ಟವನ್ನು ಅನುಭವಿಸಲಿ'' ಎನ್ನುವುದು. ಉತ್ತಮವಾಗಿರುವ ಜ್ಞಾನಿಗೆ ``ತಾನು ಆನಂದವನ್ನು ಅನುಭವಿಸಿ ಆಯಿತು. ಇನ್ನೊಬ್ಬನಿಗೆ ಆ ಆನಂದವನ್ನು ಅನುಭವಿಸುವಂತೆ ಮಾಡಬೇಕು'' ಎನ್ನುವ ಸ್ವಾಭಾವಿಕವಾದ ಭಾವನೆ ಇರುತ್ತದೆ. 

\begin{shloka}
``ಅಹೇತುಕದಯಾಸಿಂಧುಃ ಬಂಧುರಾನಮತಾಂ ಸತಾಮ್ ||''
\end{shloka}

(ಗುರು ಕಾರಣವಿಲ್ಲದೆ ಕರುಣೆಯನ್ನು ತೋರಿಸುವ ದಯಾಸಿಂಧು; ಅವರನ್ನು ನಮಸ್ಕರಿಸುವ ಪೂತಾತ್ಮರಿಗೆ ಅವರು ಮಿತ್ರರು.)

-ಎಂದು ವಿವೇಕಚೂಡಾಮಣಿಯಲ್ಲಿ ಹೇಳಿದೆ. ಯಾರಿಗಾದರೂ ಗುರು ಒಂದು ವಿಷಯವನ್ನು ತಿಳಿಸಬೇಕೆಂದಿದ್ದರೆ ಅವರ ದಯೆಗೆ ಕಾರಣವೇ ಇಲ್ಲ. ಶಿಷ್ಯನು ಅವರ ಬಳಿಗೆ ಶರಣಾಗತನಾಗಿ ಬಂದರೆ ಆ ಕಾರಣ ಒಂದೇ ಸಾಕು.

ಸತ್ಪುರುಷರು ಯಾರು? ಮಹಾತ್ಮರು ಯಾರು? ಸತ್ಪುರುಷರೆಂದು ಹೇಗೆ ತಿಳಿದುಕೊಳ್ಳುವುದು?

ಮೊದಲು, `ಶಾನ್ತಾಃ' ಎಂದು ಹೇಳಿದೆ. ಒಂದು ವಿಷಯವನ್ನು ನೋಡಬಹುದು. ಮಹಾತ್ಮನ ಮುಖವನ್ನು ನೋಡಿದರೆ ಶಾಂತಿ ತಿಳಿಯುವುದು. ಅವನಲ್ಲಿ ಸತ್ಪುರುಷನ ಲಕ್ಷಣವನ್ನು ಕಾಣಬಹುದು. ಶಿಷ್ಯನು ಮೊದಲು ಶಾಂತನಾಗಿರುವವರ ಹತ್ತಿರವೇ ಶರಣಾಗತಿ ಪಡೆಯಬೇಕು. ಅವರೇ ಮಹಾತ್ಮರು. ಅವರ ಸ್ವಭಾವ

\begin{shloka}
`ವಸಂತವಲ್ಲೋಕಹಿತಂ ಚರನ್ತಃ'
\end{shloka}

-ಎನ್ನುವಂತೆ ಇರುವುದು. ವಸಂತ ಋತುವಿನಲ್ಲಿ ಎಲ್ಲಿನೋಡಿದರೂ ಗಿಡ-ಮರಗಳು ಹಸುರಾಗಿರುತ್ತವೆ. ಎಲ್ಲೆಲ್ಲೂ ಆನಂದವಿರುತ್ತದೆ. ಹೂವುಗಳು ಅರಳಿ ನೋಡಲು ಸಂತೋಷವಾಗುವಂಥ ಕಾಲ ಅದು. ಹತ್ತು ಜನರನ್ನು ಸಂತೋಷಪಡಿಸುವುದರಿಂದ ಆ ಸಮಯಕ್ಕೆ ಯಾವ ಪ್ರಯೋಜನವೂ ಇಲ್ಲದಿದ್ದರೂ, ಹಾಗೆ ಮಾಡುವುದು ಅದರ ಸ್ವಭಾವ. ಅದರಂತೆಯೇ ಸತ್ಪುರುಷರು ಹತ್ತು ಜನ ಸುಖವಾಗಿರಲಿ ಎನ್ನುವ ಸ್ವಭಾವವುಳ್ಳವರೂ ಅವರು.-

\begin{shloka}
``ತೀರ್ಣಾಃ ಸ್ವಯಂ.... ತಾರಯನ್ತಃ"
\end{shloka}

-ಎಂದು ಹೇಳಿದಂತೆ ದುಃಖ-ಜನನ-ಮರಣ-ಶೋಕ-ಮೋಹ ಹೀಗೆ ಪದೆ ಪದೆ ಬರುವ ಭವಸಮುದ್ರವನ್ನು ದಾಟಿದವನು. ಅವನು ತನ್ನಂತಯೇ ಶಿಷ್ಯನಿಗೂ ಜ್ಞಾನವನ್ನು ಉಂಟುಮಾಡಲು ಪ್ರಯಾಸಪಡುವವರು. ಶಿಷ್ಯನಿಗಾಗಿ ಹೀಗೆ ಯಾರು ದಯೆಯಿಂದ ಪ್ರಯತ್ನ ಮಾಡುವರೋ ಅವರೇ ಮಹಾತ್ಮರು. ಮಹಾತ್ಮರಾಗಿರುವ ಒಬ್ಬರನ್ನು ನಾವು `ಗುರು' ಎಂದು ಕರೆಯಬಹುದ್. ಕೆಲವರು ತಾವು ಯಾವುದಾದರೂ ಶಾಸ್ತ್ರಗಳನ್ನು ಕಲಿತು ಒಳ್ಳೆಯ ಯೋಗ್ಯತೆ ಪಡೆದಿದ್ದರೂ ಕೂಡ ಇತರರಿಗೆ ಹೇಳಿ ಕೊಡುವುದೇ ಇಲ್ಲ. ಏಕೆಂದರೆ `ಹೇಳಿಕೊಟ್ಟರೆ ಇತರರೂ ತಿಳಿದುಕೊಂಡು ಬಿಡುತ್ತಾರೆ' ಎನ್ನುವ ಸಂಕುಚಿತ ಭಾವನೆ ಅವರಿಗೆ ಇರುತ್ತದೆ. ಇಂಥವರು ಗುರು ಆಗಲಾರರು.

ಆದರೆ ಕೆಲವರು ಅಸೂಯೆ ಇಲ್ಲದಿದ್ದರೂ ಇತರರಿಗೆ ಹೇಳಿಕೊಡುವುದಿಲ್ಲ. (ಏಕೆಂದರೆ ಅವರಿಗೆ ಅನುಭವವಿರುವುದಿಲ್ಲ.) ಮಹಾಭಾರತದಲ್ಲಿ -ಅಶ್ವತ್ಥಾಮನಿಗೆ ಒಂದು ಅಸ್ತ್ರವಿದ್ಯೆ ಗೊತ್ತು. ಆದರೆ ಅಸ್ತ್ರವನ್ನು ಬಿಟ್ಟಮೇಲೆ ಅದನ್ನು ತಾನೇ ತಡೆದು ನಿಲ್ಲಿಸುವಂಥ ಶಕ್ತಿ ಇದ್ದರೇನೆ ಆ ಅಸ್ತ್ರವನ್ನು ಪ್ರಯೋಗ ಮಾಡಲು ಸಾಧ್ಯ. ಆ ಶಕ್ತಿ ಇಲ್ಲದೆ ಅದನ್ನು ಪ್ರಯೋಗಿಸಲು ಸಾಧ್ಯವಿಲ್ಲ. ಮಹಾಭಾರತಯುದ್ಧ ಆದ ಮೇಲೆ ತನ್ನ ಪಕ್ಷ ಸೋತುಹೋಯಿತೆಂದು ಕೋಪಗೊಂಡ ಅಶ್ವತ್ಥಾಮನು ಆ ಅಸ್ತ್ರವನ್ನು ಪ್ರಯೋಗಿಸಿದನು. ಆ ಅಸ್ತ್ರದಿಂದ ಎಲ್ಲರಿಗೂ ಕಷ್ಟವಾಯಿತು. ಕೃಷ್ಣನೂ, ಭೀಷ್ಮರೂ ಎಲ್ಲರೂ ಇದಕ್ಕಾಗಿ ಅವನ ಮೇಲೆ ಕೋಪಗೊಂಡರು. `ಯಾವಾಗಲೂ ತಲೆನೋವು ಉಂಟಾಗುವುದು' ಎನ್ನುವ ಶಾಪವು ಅವನಿಗೆ ಬಂದಿತು. ಆದ್ದರಿಂದ ಅಶ್ವತ್ಥಾಮನಂತೆ ಯೋಗ್ಯತೆ ಇಲ್ಲದೆ ಗುರುವಾಗಲು ಸಾಧ್ಯವಿಲ್ಲ. ನಾವು ಮಕ್ಕಳ ಕೈಗೆ ಒಂದು ಪೆನ್ನು ಕೊಟ್ಟರೆ ಅದರಲ್ಲಿ ಹೇಗೆ ಬರೆಯುವುದೆಂದು ನೋಡಿ ಕೊನೆಗೆ ತೆಗೆದುಕೊಂಡು ಹೋಗಿ ಅದನ್ನು ಒಲೆಗೆ ಹಾಕಿಬಿಡುತ್ತಾರೆ. ಆದ್ದರಿಂದ ಅಧಿಕಾರವಿಲ್ಲದವರ ಹತ್ತಿರ ನಾವು ಜಾಗರೂಕರಾಗಿರಬೇಕು. ಹಾಗೆಯೇ, ಯೋಗ್ಯತೆ ಇದ್ದರೂ ಕೂಡ ತನ್ನಿಂದ ಇನ್ನೊಬ್ಬನು ಸರಿಯಾದ ಮಾರ್ಗಕ್ಕೆ ಬಂದು ಶ್ರೇಯಸ್ಸನ್ನು ಪಡೆಯುತ್ತಾನಲ್ಲಾ ಎನ್ನುವ ಸಂಕುಚಿತ ಮನೋಭಾವನೆ ಇದ್ದರೆ ಅಂಥವನೂ ಗುರು ಆಗಲಾರನು. ಆದ್ದರಿಂದಲೇ-

\begin{shloka} 
``ಶಿಷ್ಯಹಿತಾಯ ಉದ್ಯತತಃ ಸತತಮ್'' 
\end{shloka}

-ಎಂದು ಹೇಳಿರುವುದು. ಯಾವಾಗಲೂ ಶಿಷ್ಯನ ಹಿತವನ್ನು ಯಾರು ಅಪೇಕ್ಷಿಸುವರೋ ಅವರೇ ಗುರು ಆಗುವರು. ಅಂಥ ಗುರುವನ್ನು ಹುಡುಕಿ, ಅವರಿಗೆ ಶರಣಾಗಿ ವಿದ್ಯೆಯನ್ನು ಸಂಪಾದಿಸಿಕೊಳ್ಳಬೇಕು. ಗೀತೆಯಲ್ಲಿ ಭಗವಂತನು-

\begin{shloka}
``ತದ್ವಿದ್ಧಿ ಪ್ರಣಿಪಾತೇನ ಪರಿಪ್ರಶ್ನೇನ ಸೇವಯಾ |\\
ಉಪದೇಕ್ಷ್ಯನ್ತಿ ತೇಜ್ಞಾನಂ ಜ್ಞಾನಿನಃ ತತ್ತ್ವದರ್ಶಿನಃ ||''
\end{shloka}

(ಅದನ್ನು ನಮಸ್ಕಾರ, ಪ್ರಶ್ನೆ ಸೇವೆ ಇವುಗಳ ಮೂಲಕ ತಿಳಿಯುವವನಾಗು. ತತ್ತ್ವವನ್ನು ತಿಳಿದ ಜ್ಞಾನಿಗಳು ನಿನಗೆ ತತ್ತ್ವೋಪದೇಶ ಮಾಡುವರು.)

-ಎಂದಿದ್ದಾನೆ. ಜ್ಞಾನವನ್ನು ಪಡೆಯಬೇಕಾದರೆ ತತ್ತ್ವದರ್ಶಿಗಳಾಗಿರುವವರ ಹತ್ತಿರ ಹೋಗಿ ಜ್ಞಾನವನ್ನು ಪಡೆಯಬೇಕು.

\begin{shloka}
``ಪರಿಪ್ರಶ್ನೇನ ಸೇವಯಾ''
\end{shloka}

ಅವರ ಬಳಿಗೆ ಹೋಗಿ ಮೊದಲು ಕೇಳಬೇಕು, ಏನು ಕೇಳಬೇಕು? `ಏನು ಸಾರ್! ಏನು ಸಮಾಚಾರ?' ಎಂದು ಕೇಳಿದರೆ `ಏನೂ ಇಲ್ಲ ಹೋಗು!' ಎಂದು ಬಿಡುವರು. ಆದ್ದರಿಂದ ಅವರ ಮನಸ್ಸು ಪ್ರಸನ್ನವಾಗಿರುವ ವೇಳೆಯಲ್ಲಿ, ಸೇವೆಗಳಿಂದ ಅವರನ್ನು ತೃಪ್ತಿಪಡಿಸಿದ ನಂತರ ಶಿಷ್ಯನು ಕೇಳಿದನೆಂದರೆ ಗುರು ಯಥಾರ್ಥಾವಾಗಿರುವ ತತ್ತ್ವವನ್ನು ಕಲಿಸುವರು. 

\begin{shloka}
``ಜ್ಞಾನಿನಃ ತತ್ತ್ವದರ್ಶಿನಃ"
\end{shloka}

-ಎಂದು ಹೇಳಿರುವುದರಿಂದ ನಾವು ತತ್ತ್ವದರ್ಶಿಗಳಾದ ಜ್ಞಾನಿಗಳ ಹತ್ತಿರವೇ ಹೋಗಬೇಕು.

ಎಂಥ ಜ್ಞಾನವನ್ನು ಪಡೆಯಬೇಕು? ವಿದ್ಯೆಗಳಲ್ಲಿ ಹಲವು ವಿಧವಾದ ವಿದ್ಯೆಗಳಿವೆ. `ಚತುಷಷ್ಟಿ ವಿದ್ಯೆ' -ಎಂದು ಹೇಳಲ್ಪಡುವ ಅರವತ್ತುನಾಲ್ಕು ವಿದ್ಯೆಗಳಿವೆ. ಜೂಜು ಆಡುವುದು, ಇಂಥವು ಕೂಡ ಅರವತ್ತುನಾಲ್ಕು ಕಲೆಗಳಲ್ಲಿ ಸೇರುತ್ತೇವೆ. ಯಾವ ಕಲೆಯನ್ನು ಹೇಳಿ ಕೊಡಲು ಗುರು ಎನ್ನುವರು ಬೇಕು? ಗುರು ಇಲ್ಲದೆ ಯಾವುದೂ ಫಲಿಸುವುದಿಲ್ಲ. ಹೀಗೆ ಹಲವು ವಿದ್ಯೆಗಳು ಇದ್ದರೂ ಎಂಥ ಜ್ಞಾನವನ್ನು ನಾವು ಪಡೆಯಬೇಕು? 

\begin{shloka}
`ತತ್ ಜ್ಞಾನಂ ಪ್ರಶಮಕರಂ ಯದಿಂದ್ರಿಯಾಣಾಮ್" 
\end{shloka}

(ಯಾವುದು ಇಂದ್ರಿಯಗಳಿಗೆ ಶಾಂತಿಯನ್ನು ಕೊಡುತ್ತದೆಯೋ ಆ ಜ್ಞಾನ)

-ಎಂದು ಹೇಳಿದೆ. ಯಾವುದರಿಂದ ಶಾಂತಿ ಉಂಟಾಗುವುದೋ ಅಂಥ ಜ್ಞಾನವನ್ನು ಪಡೆಯಬೇಕು. ಯಾವುದೋ ಒಂದು ಕಲೆಯನ್ನು ಕಲಿತು, ಆದರಿಂದ ಸ್ವಲ್ಪವೂ ಶಾಂತಿ ಇಲ್ಲವೆಂದು ಹೇಳಿದರೆ ಅಂಥ ಜ್ಞಾನ ಜ್ಞಾನವೇ ಅಲ್ಲ. 

\begin{shloka}
``ತತ್ ಜ್ಞಾನಂ ಯದುಪನಿಷತ್ಸು ನಿಶ್ಚಿತಾರ್ಥಮ್''
\end{shloka}

(ಉಪನಿಷತ್ತಿನಲ್ಲಿ ಯಾವುದು ನಿಶ್ಚಯವಾಗಿ ಹೇಳಲ್ಪಟ್ಟಿದಯೋ ಅದನ್ನು ತಿಳಿದುಕೊಳ್ಳಬೇಕು.)

-ಎಂದಿರುವುದರಿಂದ ಉಪನಿಷತ್ತಿನಲ್ಲಿ ಹೇಳಿರುವುದನ್ನೇ ನಾವು ತಿಳಿದುಕೊಳ್ಳಬೇಕು. ಆ ಜ್ಞಾನವನ್ನು ಪಡೆದರೇನೇ ನಮ್ಮ ಇಂದ್ರಿಯಗಳಿಗೆಲ್ಲಾ ಶಾಂತಿಯುಂಟಾಗುತ್ತದೆ; ಸಮಾಧಾನವಾಗುತ್ತದೆ. ಅನಂತರ ನಾವು ಯಾವ ವಸ್ತುವಿನ ಹಿಂದೆಯೂ ತಿರುಗಾಡಬೇಕಾಗಿಲ್ಲ. ಅಂಥ ಜ್ಞಾನವನ್ನು ಪಡೆಯುವುದಕ್ಕಾಗಿ ನಾವು ಗುರುವಿನ ಬಳಿಗೆ ಹೋಗಬೇಕು. ಆದ್ದರಿಂದ ಆ ಗುರುವೂ ತತ್ತ್ವವನ್ನು ತಿಳಿದವರಾಗಿರಬೇಕು. ಅಂಥ ಗುರುವಿನ ಮಹಾತ್ಮ್ಯೆಯನ್ನು ಶಂಕರರು-

\begin{shloka}
``ದೃಷ್ಟಾಂತೋ ನೈವ ದೃಷ್ಟಃ ತ್ರಿಭುವನ ಜಠರೇ ಸದ್ಗುರೋ ರ್ಜ್ಞಾನದಾತುಃ\\
ಸ್ಪರ್ಶಶ್ಚೇತ್ತತ್ರಕಲ್ಪ್ಯಃ ಸ ನಯತಿ ಯದಹೋ ಸ್ವರ್ಣತಾಮಶ್ಮ ಸಾರಮ್ |\\
ನ ಸ್ಪರ್ಶತ್ವಂ ತಥಾಪಿ ಶ್ರಿತಚರಣಯುಗೇ ಸದ್ಗುರುಃ ಸ್ವೀಯ ಶಿಷ್ಯೇ\\
ಸ್ವೀಯಂ ಸಾಮ್ಯಂ ವಿಧತ್ತೇ ನಿರುಪಮಃ ತೇನ ವಾ ಲೌಕಿಕೋಽಪಿ ||" 
\end{shloka}

[ಜ್ಞಾನವನ್ನು ಕೊಡು ಸದ್ಗುರುವಿಗೆ ಮೂರು ಲೋಕಗಳಲ್ಲಿ ಯೂ ಉಪಮಾನವಿಲ್ಲ. ಸ್ಪರ್ಶಮಣಿಯನ್ನು ಉಪಮಾನವಾಗಿ ಹೇಳೋಣವೆಂದರೆ ಅದು ಕಬ್ಭಿಣವನ್ನು ಚಿನ್ನವನ್ನಾಗಿ ಮಾಡುತ್ತದೆಯೇ ಹೊರತು ಇನ್ನೊಂದು ಸ್ಪರ್ಶಮಣಿಯನ್ನು ಮಾಡಲಾರದು; ಆದರೆ ಗುರು ಆದವನು ತನ್ನನ್ನು ಆಶ್ರಯಿಸಿದ ಶಿಷ್ಯನಿಗೆ ತನ್ನ ಸ್ಥಿತಿಯನ್ನೇ ಕೊಡುವನು; ಆದ್ದರಿಂದ ಆತನಿಗೆ ಸಮಾನ ಯಾರೂ ಇಲ್ಲ; ಆತನು ನಿರುಪಮಾನನು. ಲೌಕಿಕ ಗುರುವಿಗೆ ಉಪಮಾನವನ್ನು ಹೇಳಲಾರದೆ ಇರುವಾಗ ಸದ್ಗುರುವಿಗೆ ಹೇಗೆ ಹೇಳುವುದು?]

-ಎನ್ನುವ ಶ್ಲೋಕದಲ್ಲಿ ವರ್ಣಿಸುತ್ತಾರೆ. ``ಒಂದು ವಸ್ತು ಹೀಗಿದೆ" ಎನ್ನುವುದಕ್ಕೆ ದೃಷ್ಟಾಂತ (ಉಪಮಾನ) ಕೊಡಬೇಕು. ಒಬ್ಬನು ಸಮುದ್ರವನ್ನೇ ಕಂಡಿಲ್ಲ. ಅವನಿಗೆ ಸಮುದ್ರವನ್ನು ಕುರಿತು ಹೇಗೆ ಹೇಳುವುದು? ``ನೋಡು! ನಮ್ಮ ಊರಿನಲ್ಲಿ ಒಂದು ಕೊಳವಿದೆಯಲ್ಲಾ ಅದಕ್ಕಿಂತಲೂ ಸಮುದ್ರ ದೊಡ್ಡದಾಗಿರುತ್ತದೆ. ಕೊಳದಲ್ಲಿರುವ ನೀರಿನಂತಲ್ಲದೆ ಸಮುದ್ರದಲ್ಲಿ ದೊಡ್ಡ ದೊಡ್ಡ ಅಲೆಗಳು ಬರುತ್ತವೆ. ಆದ್ದರಿಂದ ಒಂದು ದೊಡ್ಡ ಕೊಳವೆಂದು ಭಾವಿಸು" ಎಂದು ಹೇಳಬಹುದು. ಹಾಗೆ ಹೇಳದೆ, ``ಸಮುದ್ರ ಇಷ್ಟು ಉದ್ಧವಾಗಿರುತ್ತದೆ, ಇಷ್ಟು ಅಗಲವಾಗಿರುತ್ತದೆ, ಇಷ್ಟು ದೊಡ್ಡದಾಗಿರುತ್ತದೆ" ಎಂದು ಹೇಳಿದರೆ ಕೇಳುವವನಿಗೆ ಅದು ತಿಳಿಯುವುದೇ ಇಲ್ಲ. ಆದ್ದರಿಂದ ಯಾವ ವಸ್ತುವಿಗಾಗಲಿ ದೃಷ್ಟಾಂತವನ್ನು ಕೊಟ್ಟೇ ವಿವರಿಸಬೇಕು. `ಗುರು ಎಂದರೆ ಯಾರು? ಅವರು ಎಂಥವರು ಎಂದು ಹೇಳಲು ಯಾವುದಾದರೂ ದೃಷ್ಟಾಂತವನ್ನು ಹೇಳಬಹುದೇ?' ಎಂದು ಕೇಳಿದರೆ ``ಅವರಿಗೆ ದೃಷ್ಟಾಂತವಾಗಿ ಯಾವುದನ್ನು ಹೇಳುವುದು ಎಂದು ತಿಳಿಯಲಿಲ್ಲ'' ಎನ್ನುತ್ತಾರೆ ಶಂಕರರು-

\begin{shloka}
``ದೃಷ್ಟಾಂತೋ........ ಸ್ವರ್ಣತಾಮಶ್ಮಸಾರಮ್ |'
\end{shloka}

ಸ್ಪರ್ಶಮಣಿ ಎಂದು ಹೇಳಲ್ಪಡುವ ಒಂದು ಮಣಿ ಇದೆ. ಆ ಮಣಿಯ ಸ್ವರೂಪವೇನೆಂದರೆ ಅದನ್ನು ಮುಟ್ಟಿದ ಯಾವುದೇ ಕಬ್ಬಿಣವಾಗಲಿ ಚಿನ್ನವಾಗಿಬಿಡುತ್ತದೆ. ಕಬ್ಬಿಣದ ಬೆಲೆ ಬಹಳ ಕಡಿಮೆ. ಅಂಥ ಕಬ್ಭಿಣವನ್ನು ಚಿನ್ನವನ್ನಾಗಿ ಮಾರ್ಪಡಿಸುವ ಆ ಸ್ಪರ್ಶಮಣಿಗೆ ಎಂಥ ಒಂದು ದೊಡ್ಡ ಶಕ್ತಿ ಇದೆ ಎನ್ನುವುದನ್ನು ನಾವು ಹೇಳದೆಯೇ ತಿಳಿದುಕೊಳ್ಳಬಹುದು. ಇಂಥ ಉತ್ತಮವಾದ ವಸ್ತುವಾಗಿ ಸ್ಪರ್ಶಮಣಿ ಇರುವಂತೆ ಸದ್ಗುರುವು ಏನೂ ತಿಳುವಳಿಕೆ ಇಲ್ಲದೆ, ನಿಂತುಕೊಂಡೇ ಮೂತ್ರವಿಸರ್ಜನೆ ಮಾಡುವ ಅಭ್ಯಾಸವುಳ್ಳ, `ಯಾವುದನ್ನು ತಿನ್ನಬೇಕು' `ಯಾವುದನ್ನು ತಿನ್ನಬಾರದು' ಎನ್ನುವುದು ತಿಳಿಯದೆ ಎಲ್ಲಿ ಎಂದರೆ ಅಲ್ಲಿ ತಿನ್ನುವುವವನೂ, ಯಾವುದು ಆರೋಗ್ಯಕರ, ಯಾವುದು ಅಲ್ಲ ಎಂದು ತಿಳಿಯದವನೂ, ಪ್ರಪಂಚದಲ್ಲಿ ಹತ್ತು ಜನರೊಡನೆ ಹೇಗೆ ಇರಬೇಕು ಎನ್ನುವುದು ತಿಳಿಯದೆ ಚಿಕ್ಕಮಗವಿನಂತಿರುವ ಶಿಷ್ಯನನ್ನು ಪ್ರಪಂಚದಲ್ಲಿ ಜನರೆಲ್ಲರೂ ಹೊಗಳುವಂತೆ ಪ್ರಯೋಜನಕಾರಿಯನ್ನಾಗಿ ಮಾಡಿಬಿಡುತ್ತಾರೆ. ಅಂಥ ಶಕ್ತಿ ಗುರುವಿಗೆ ಇದೆ.

ಈ ಕಾಲದಲ್ಲಿ ಸ್ಕೂಲು ಅಧ್ಯಾಪಕರುಗಳಿಗೆ ಸಂಬಳ ಕಡಮೆಯೆಂದು ನನಗೆ ಬಹಳ ಮಂದಿ ಹೇಳಿದ್ದಾರೆ. `ಸಾಧಾರಣ ವಿದ್ಯಾರ್ಥಿಗಳನ್ನು ಉತ್ತಮ ಪ್ರಜೆಯನ್ನಾಗಿ ರೂಪಿಸುವ ನಮಗೆ ಸಂಬಳ ಬಹಳ ಕಡಮೆ ಎಂದು ಅವರು ವ್ಯಥೆ ಪಡುತ್ತಾರೆಂದು ಕೇಳಿದ್ದೇನೆ. ಆದ್ದರಿಂದ ಹೇಗೆ ಕಬ್ಬಿಣವನ್ನು ಚಿನ್ನವನ್ನಾಗಿ ಮಾರ್ಪಡಿಸುವ ಶಕ್ತಿ ಸ್ಪರ್ಶಮಣಿಗೆ ಇದೆಯೋ, ಹಾಗೆಯೇ ಏನೂ ತಿಳಿಯದ ತಿಳಿವಳಿಕೆಯೇ ಇಲ್ಲದವನನ್ನು ಉತ್ತಮ ಜ್ಞಾನಿಯನ್ನಾಗಿ ಮಾಡುತ್ತಾರೆ ಗುರು. ಅವರಿಗೆ ಉಪಮಾನವಾಗಿ ಈ ಸ್ಪರ್ಶಮಣಿಯನ್ನು  ಹೇಳೋಣವೇ ಎಂದರೆ ಅದು ಆಗುವುದಿಲ್ಲ. ಗುರುವಿನಲ್ಲಿ ಈ ಸ್ಪರ್ಶಮಣಿಗಿಂತಲೂ ಹೆಚ್ಚು ಗುಣವಿದೆ. ಸ್ಪರ್ಶಮಣಿಯಿಂದ ಯಾವ ಕಬ್ಬಿಣ ಚಿನ್ನವಾಗಿ ಮಾರ್ಪಟ್ಟಿತೋ ಅದಕ್ಕೆ ಇನ್ನೊಂದು ಕಬ್ಬಿಣವನ್ನು ಚಿನ್ನವನ್ನಾಗಿ ಮಾರ್ಪಡಿಸುವ ಶಕ್ತಿ ಇಲ್ಲ. ಶ್ರೇಷ್ಠ ಆಚಾರ್ಯರಾಗಿರುವವರು ತಮ್ಮಂತಯೇ ಸಾವಿರಾರು ಜನರನ್ನು ಆಚಾರ್ಯರನ್ನಾಗಿ ಮಾಡಬಲ್ಲರು (ಕಾವಿ ಧರಿಸಿರುವರೇ ಆಚಾರ್ಯರೆಂದು ಅರ್ಥವಲ್ಲ). ಆದ್ದರಿಂದ ಗುರುವಿಗೆ ಸ್ಪರ್ಶಮಣಿಯ ದೃಷ್ಟಾಂತವನ್ನು ಕೊಡಲು ಸಾಧ್ಯವಾಗುವುದಿಲ್ಲ. ಪ್ರಾಪಂಚಿಕ ವಿಷಯಗಳಲ್ಲಿ ಗುರುವಾಗಿರುವವನಿಗೆ ಕೂಡ ದೃಷ್ಟಾಂತವನ್ನು ಕೊಡುವುದು ಸಾಧ್ಯವಾಗದೆ ಇರುವಾಗ ಆತ್ಮಜ್ಞಾನವನ್ನು ಕೊಡುವ ಭವಬಂಧನದಿಂದ ಬಿಡುಗಡೆಮಾಡುವ ಗುರುವಿಗೆ ಏನು ದೃಷ್ಟಾಂತ ಕೊಡುವುದು? ಅವರಿಗೆ ದೃಷ್ಟಾಂತ ಕೊಡಲು ಸಾಧ್ಯತೇ ಇಲ್ಲ. ಆದ್ದರಿಂದ ಸ್ಪರ್ಶಮಣಿಗಿಂತಲೂ ಗುರುವನ್ನು ಅಧಿಕವಾಗಿ  ಹೊಗಳಬೇಕು. ಶ್ಲೋಕದಲ್ಲಿ ಅಸದ್ಗುರುವಿಗೆ ಉದಾಹರಣೆ ಕೊಟ್ಟಲ್ಲ. ಕೆಟ್ಟ ವಿಷಯಗಳನ್ನು ಹೇಳಿಕೊಡಬಹುದು. ಮಧ್ಯವನ್ನು ಪಾನಮಾಡುವ ಅಭ್ಯಾಸವನ್ನು ಮಾಡಿಸುವವರೂ ಇದ್ದಾರೆ. ಅವರೂ ಒಂದು ವಿಷಯದಲ್ಲಿ ಗುರುವೇ. ಏಕೆಂದರೆ, ಯಾವುದಾದರೂ ಕಲೆಯನ್ನು ಕಲಿಸಿದರೆ ಅವನೂ ಗುರು. ತಂದೆಯೂ ಗುರುವೇ-

\begin{shloka}
`ಏಕಾಕ್ಷರ ಪ್ರದಾತಾ ಚ ಗುರುರಿತ್ಯಭಿದೀಯತೇ |\\
(ಒಂದು ಅಕ್ಷರವನ್ನು ಕಲಿಸಿದವನು ಗುರುವೆಂದೇ ಹೇಳಲ್ಪಡುತ್ತಾನೆ.)
\end{shloka} 

ಹೀಗೆ ಹಲವು ವಿಧವಾಗಿ ಗುರು ಇರಬಹುದು; ಏಕೆಂದರೆ ಯಾರು ಯಾರಿಂದ ನಾವು ಅರಿವನ್ನು ಪಡೆಯುತ್ತೇವೋ ಅವರೆಲ್ಲರೂ `ಗುರು'ವಾಗುತ್ತಾರೆ. ಆದ್ದರಿಂದ ನಾವು ಕೆಟ್ಟ ಅರಿವನ್ನು ಸಂಪಾದಿಸಿಕೊಳ್ಳದೆ ಇರುವುದಕ್ಕಾಗಿ ಜ್ಞಾನವನ್ನು ಕೊಡುವ ಸದ್ಗುರುವನ್ನು ಕುರಿತು ಶ್ಲೋಕದಲ್ಲಿ ಶಂಕರರು ಹೇಳಿದ್ದಾರೆ.

ಆದರೂ ಆ ಸ್ಪರ್ಶಮಣಿಯ ಸಕಲ ಗುಣಗಳು ಇಲ್ಲಿ ಇರುವುದಿಲ್ಲ. 


\begin{shloka}
`ಸ್ವೀಯಂ ಸಾಮ್ಯಂ ವಿಧತ್ತೇ'
\end{shloka}

ಮೊದಲೇ ಹೇಳಿದಂತೆ ಗುರುವಾದವರು ಶಿಷ್ಯನನ್ನು ತನ್ನ ಸ್ಥಿತಿಗೆ ತಂದು ಬಿಡುತ್ತಾರೆ. `ಮುದ್ರಾರಾಕ್ಷಸ' ಎನ್ನುವ ಒಂದು ನಾಟಕ ಹಳೆಯ ಕಾಲದ ರಾಜಕೀಯ ಸ್ಥಿತಿಯನ್ನು ಚೆನ್ನಾಗಿ ತಿಳಿಸುತ್ತದೆ. ಅದರಲ್ಲಿ ಒಂದು ಜಾಗದಲ್ಲಿ `ಶಿಷ್ಯನು ಯಾರಾದರೂ ಕೆಟ್ಟವನಾಗಿದ್ದರೆ ಅದಕ್ಕೆ ಕಾರಣ ಗುರುವೇ ಆಗುತ್ತಾನೆ. ಗುರುವೇ ಕೆಡಿಸಿಬಿಟ್ಟರು' ಎಂದು ಹೇಳಿದೆ. ಕೆಟ್ಟ ಉಪಾಧ್ಯಾಯರ ಸಹವಾಸದಿಂದ ಶಿಷ್ಯನೂ ಕೆಟ್ಟು ಹೋಗುತ್ತಾನೆ.

\begin{shloka}
`ಯಾದೃಶೈಃ ಸನ್ನಿವಸತಿ ಯಾದೃಶಾಂ ಚೋಪಸೇವತೇ |\\
ಯಾದೃಗಿಚ್ಛೇಶ್ಚ ಭವಿತುಂ ತಾದೃಗ್ಭವತಿ ಪೂರುಷಃ ||
\end{shloka}

 ಎಂಥ ಜನರೊಡನೆ ಒಬ್ಬನು ಸಹವಾಸ ಮಾಡುತ್ತಾನೋ, ಎಂಥವರನ್ನು ದೊಡ್ಡವರೆಂದು ಭಾವಿಸಿ ಸೇವಿಸುತ್ತಾನೋ, ತಾನು ಹೇಗೆ ಆಗಬೇಕೆಂದು ಬಯಸುತ್ತಾನೋ ಹಾಗೆಯೇ ಅವನು ಆಗುವನೆಂದು ಹೇಳಲ್ಪಟ್ಟಿದೆ. ಸೇರುವಿಕೆ, ಸೇವೆ ಎರಡೂ ಇದ್ದರೂ ಕೂಡ, ದೃಢವಾದ ಮನಸ್ಸಿನಿಂದ ಇದ್ದರೆ ಒಬ್ಬನು ಕೆಟ್ಟದಾರಿಯಲ್ಲಿ ನಡೆಯಲಾರನು. 
 
 ಯಾರಿಗೆ ನಾವು ಸೇವೆ ಮಾಡುತ್ತೇವೋ, ಅವರು ಕೆಟ್ಟವರಾಗಿದ್ದರೆ, ನಾವೂ ಕೆಟ್ಟವರಾಗಬೇಕಾದುದೇ. ಆದ್ದರಿಂದ ಮನಸ್ಸಿನಲ್ಲಿರುವ ಫಲವನ್ನು ಪರೀಕ್ಷಿಸಿ ನೋಡಬೇಕು.
 
ಕೆಲವರು ಚಲನಚಿತ್ರವನ್ನು ನೋಡಿ ಕೆಟ್ಟು ಹೋಗುವುದುಂಟು, ಏಕೆಂದರೆ ಅದರಲ್ಲಿ ಬರುವ ಕಳ್ಳನನ್ನು ಕಂಡು ನೋಡುವವನು ಕಳ್ಳತನವನ್ನು ಕಲಿತುಕೊಂಡು ಕಳ್ಳತನ ಮಾಡುವನು, ಹಲವು ದುಷ್ಟ ಕೆಲಸಗಳನ್ನು ಮಾಡುವನು. ಲಾಯರ್ ಆಗಿರುವವರು ಕೇಳದೆ ಇರುವ ಕಳ್ಳತನವೇ ಇಲ್ಲ. ಅವರ ಹತ್ತಿರ ಬರುವ ಕಳ್ಳರು, `ನಾನು ಹೀಗೆ ಮಾಡಿದೆನು. ನನ್ನನ್ನು ಕಾಪಾಡಿ' ಎಂದು ಕೇಳಿಕೊಳ್ಳತ್ತಾರೆ. ಲಾಯರ್ ಕೆಲಸ ಏನು? ಅವರ ಹತ್ತಿರ ಬರುವವರ ಕೇಸನ್ನು ಗೆದ್ದು ಕೊಡುವುದೇ. ಆದ್ದರಿಂದ ಅವರು ಎಷ್ಟು ಕೆಟ್ಟ ಕೆಲಸಗಳನ್ನು ನೋಡಿದ್ದರೂ, ಅವರು ಮಾತ್ರ ಕೆಟ್ಟ ಕೆಲಸವನ್ನು ಕಲಿತುಕೊಳ್ಳವುದಿಲ್ಲ. ಲಾಯರ್‌ಗೆ ಇಂಥ ಜನರೊಡನೆ ಇದ್ದು ಅಭ್ಯಾಸವಾಗಿದೆ. `ಈ ಉದ್ಯೋಗವನ್ನು ಹಣ ಸಂಪಾದಿಸುವುದಕ್ಕಾಗಿ ಮಾಡುತ್ತಿದ್ದೇನೆ. ಇದು ನನಗೆ ಸಂಬಂಧಪಟ್ಟಿಲ್ಲ' ಎನ್ನುವ ನಿರ್ಣಯ ಭಾವನೆ ಅವರಲ್ಲಿದೆ. 

ಚಿಕ್ಕ ಮಕ್ಕಳಿಗೆ ಮನೋಬಲವಿರುವುದಿಲ್ಲ. ಯಾರಿಗೆ ಶಾಸ್ತ್ರ ಸದ್ಗುರು ಇಂಥವುಗಳ ಸಹಾಯದಿಂದ ಮನೋಬಲವಿರುತ್ತದೋ, ಅಂಥವರು ಇತರರನನ್ನು ತಿದ್ದಲು ಹೋಗಬೇಕು. ಅದನ್ನು ಬಿಟ್ಟು ಸಾಮಾನ್ಯ ಜನರು ಕೆಟ್ಟವರನ್ನು ಸರಿಮಾಡುತ್ತೇವೆಂದುಕೊಂಡ ಅವರ ಹತ್ತಿರಕ್ಕೆ ಹೋದರೆ ಅವರನ್ನು ಸರಿಮಾಡಲು ಹೋದವರು ಕೂಡ ಕೆಟ್ಟವರಾಗಿಬಿಡುತ್ತಾರೆ. ಆದ್ದರಿಂದಲೇ ಕೆಟ್ಟ ಸಹವಾಸ  ಕೂಡದೆಂದು ಹೇಳಿರುವುದು.

ಕೆಲವರು ಒಂದು ಪ್ರಶ್ನೆ ಕೇಳುತ್ತಾರೆ. `ಕೆಟ್ಟವರು ಯಾವಾಗಲು ಕೆಟ್ಟವರಾಗಿಯೇ ಇರಬೇಕೇ? ನಾವೆಲ್ಲರೂ ಸೇರಿ ಅವರನ್ನು ಒಳ್ಳೆಯತನಕ್ಕೆ ಕೊಂಡೊಯ್ಯುತ್ತೇವೆ' ಎನ್ನುತ್ತಾರೆ. ಮೊದಲು ಅವರು ತಮ್ಮಲ್ಲಿ ಎಷ್ಟು ಬಲವಿದೆ ಎನ್ನುವುದನ್ನು  ತಿಳಿದುಕೊಳ್ಳಬೇಕು. ಹೆಂಡ ಕುಡಿಯುವನನ್ನು ಸರಿಯಾದ ದಾರಿಗೆ ತರುತ್ತೇವೆಂದು ಹೇಳಿ ನಾವೇ ಅದರ ಅಭ್ಯಾಸವನು ಮಾಡಿಕೊಂಡರೆ ಅದು ಅಪಮಾನಕರವಾದ ವಿಷಯವಾಗುತ್ತದೆ. ಆದ್ದರಿಂದ ಸಹವಾಸ ದೋಷವೆನ್ನುವುದು ಇದೆ. ಕೆಟ್ಟ ದಾರಿಯಲ್ಲಿ ಹೋಗಬಾರದೆಂದು ಅದಕ್ಕೇ ಹೇಳಿರುವುದು. ಪ್ರಪಂಚ ಕೆಟ್ಟು ಹೋಯಿತೆಂದು ಹೇಳಿ ಇತರರೂ ಕೆಟ್ಟು ಹೋಗುತ್ತಾರೆ. ಆದ್ದರಿಂದ ಗುರುವಾಗಿರುವವನು ಪ್ರಪಂಚ ಹೇಗಿದ್ದರೂ ಸರಿ. ತಾನು ಇತರರಿಗೆ ಒಳ್ಳೆಯ ದಾರಿಯನ್ನು ತೋರಿಸುವ ಸಾಮರ್ಥ್ಯವುಳ್ಳವರಾಗಿರಬೇಕು. ಅಂಥವರನ್ನೇ ಶಾಸ್ತ್ರದಲ್ಲಿ `ಆಚಾರ್ಯ'ರೆಂದು ಹೇಳಿರುವುದು. 

\begin{shloka} 
`ಆಚಿನೋತಿ ಹಿ ಶಾಸ್ತ್ರರ್ಥಂ ಆಚಾರೇ ಸ್ಥಾಪಯತ್ಯಪಿ |\\
ಸ್ವಯಮಾಚರತೇ ಯಸ್ಮಾತ್ ಆಚಾರ್ಯಃ ಸಃ ನಿಗದ್ಯತೇ ||'
\end{shloka}
 
(ತಾವು ಶಾಸ್ತ್ರವನ್ನು ಸಂಗ್ರಹಿಸುವವರೂ, ಇತರನ್ನೂ ಅನುಸರಿಸುವಂತೆ ಮಾಡುವವರೂ, ತಾವು ಅನುಸರಿಸುವವರೂ ಅದುದರಿಂದ ಅವರು ಅಚಾರ್ಯರೆಂದು ಕರೆಯಲ್ಪಡುತ್ತಾರೆ.)

-ಎಂದು ಹೇಳಲ್ಪಟ್ಟಿದೆ. ತಾನು ಶಾಸ್ತ್ರವನ್ನು, ಪ್ರಪಂಚವನ್ನು ಎರಡನ್ನೂ ನೋಡಿ ಪ್ರಪಂಚ ಸರಿಯಾಗಿದೆಯೇ ಎಂದು ಶೋಧಿಸಿ ಗುರು ಒಂದು ತೀರ್ಮಾನಕ್ಕೆ ಬರಬೇಕು. ಅನಂತರ ಶಿಷ್ಯನಿಗೆ ದಾರಿ ತೋರಿಸಿ ಅವನು ಯಾವ ರೀತಿ ಇರಬೇಕೆನ್ನುವುದನ್ನು ತೋರಿಸಬೇಕು. ಆದರೆ ಕೆಲವರು ಹೀಗೆ ಇರುವುದಿಲ್ಲ. 


\begin{shloka}
`ಪಿತೃಭಿಃ ಕಲಹಾಯನ್ತೇ ಪುತ್ರಾನುಪದಿಶನ್ತಿ ಪಿತೃಭಕ್ತಿಮ್ |\\
ಪರದಾರಾನುಪಯನ್ತಃ ಪಠನ್ತಿ ಶಾಸ್ತ್ರಾಣಿ ದಾರೇಷು ||'
\end{shloka}

(ತಮ್ಮ ತಂದೆ-ತಾಯಿಯವರೊಡನೆ ಜಗಳವಾಡುತ್ತಾರೆ. ಆದರೆ ತಮ್ಮ ಮಕ್ಕಳಿಗೆ ಹೆತ್ತವರಿಗೆ ಭಕ್ತಿತೋರಿಸಬೇಕೆಂದು ಉಪದೇಶಿಸುತ್ತಾರೆ. ಕೆಲವರು ಪರಸ್ತೀಯರೊಡನೆ ಸಂಬಂಧವನ್ನು ಇಟ್ಟುಕೊಂಡಿರುತ್ತಾರೆ. ಮನೆಯಲ್ಲಿ ತಮ್ಮ ಹೆಂಗಸರಿಗೆ ಶಾಸ್ತ್ರಗಳನ್ನು ಓದುತ್ತಾರೆ.)

-ಎಂದು ನೀಲಕಂಠ ದೀಕ್ಷಿತರು ಹೇಳಿದ್ದಾರೆ. ಒಬ್ಬರು ತನ್ನ ಮಗನಿಗೆ  `ತಂದೆಗೆ ಮರ್ಯಾದೆ ಕೊಡಬೇಕು. ಅವರು ಏನು ಹೇಳಿದರೂ ಎದುರು ಜವಾಬು ಕೊಡಬಾರದು. ಸರಿ, ಸರಿ ಎಂದು ತಲೆ ಅಲ್ಲಾಡಿಸಬೇಕು' ಎನ್ನುತ್ತಾರೆ. ಆದರೆ ಮಗನು ದಿನವೂ ಮನೆಯಲ್ಲಿ ನೋಡುತ್ತಿರುವ ದೃಶ್ಯವೇ ಬೇರೆ. ತಂದೆ ತನ್ನ ತಂದೆಯನ್ನು ನೋಡಿ, `ಏನು ಮಾಡುವುದು! ಇನ್ನೂ ಈ ಮುದುಕ ಸಾಯಲಿಲ್ಲವಲ್ಲಾ! ಎಂಬತ್ತು ವರ್ಷ ವಯಸ್ಸಾಯಿತು. ನಮ್ಮನ್ನು  ಜಗಳವಾಡುತ್ತಿರುತ್ತಾರೆ. ಆದರೆ ತಮ್ಮ ಮಗನಿಗೆ ಮಾತ್ರ-

\begin{shloka}
`ಮಾತೃದೇವೋ ಭವ ಪಿತೃದೇವೋ ಭವ|'
\end{shloka}

(ನಿನಗೆ ನಿನ್ನ ತಾಯಿ ದೇವತೆಯಾಗಲಿ, ನಿನಗೆ ನಿನ್ನ ತಂದೆ ದೇವತೆಯಾಗಲಿ ಎಂದು ಹೇಳುತ್ತಿರುತ್ತಾರೆ.)

ಒಬ್ಬರು ತನ್ನ ಹೆಂಡತಿಗೆ `ಸೀತೆ ಹೇಗೆಲ್ಲಾ ಇದ್ದಳು. ನೀನೂ ಹಾಗೆಯೇ, ಇದ್ದರೆ ನಿನ್ನ ಮಾಂಗಲ್ಯದಿಂದ ನಮ್ಮ ಸಂಸಾರ ಚೆನ್ನಾಗಿರುತ್ತದೆ' ಎನ್ನುತ್ತಾರೆ. ಆಗ ಹೆಂಡತಿ `ನೀವೂ ರಾಮನು ಹೇಗಿದ್ದನು, ವಶಿಷ್ಠರು ಹೇಗಿದ್ದರು ಎಂದೆಲ್ಲವನ್ನೂ ನೋಡಿ ಹಾಗೆ ಉತ್ತಮ ರೀತಿಯಲ್ಲಿರಿ' -ಎಂದು ಹೇಳಿದರೆ ಆಗ ಎಷ್ಟು  ಸ್ವಾರಸ್ಯವಾಗಿರುತ್ತದೆ! ಮನುಷ್ಯನ ಸ್ವಭಾವವೆಂದರೆ ಒಳ್ಳೆಯ ದಾರಿ ಇನ್ನೊಬ್ಬನಿಗೆ ಹೇಳುವುದು. ಆದರೆ ತಾನು ಮಾತ್ರ ಏನೂ ಮಾಡುವುದಿಲ್ಲ. ಅನಾರ್ಯರಾಗಿರುವವರನ್ನು ಎಲ್ಲಿ ಬಿಟ್ಟರೂ ಕೂಡ ಅವರು ತಮ್ಮ ಸ್ಥಿತಿಯಿಂದ ಸ್ವಲ್ಪವೂ ಬದಲಾಗುವುದಿಲ್ಲ. ಅಂಥವರೇ ಆಚಾರ್ಯರು, ಅಂಥವರಿಗೆ ನಾವು ಉದಾಹರಣೆಯನ್ನು ಹೇಳಬಹುದು. ಅವರನ್ನು ನಾವು ಸೇವಿಸಬೇಕು. ಆದ್ದರಿಂದ ಗುರು ಎಂಥವರೆಂದು ಶಿಷ್ಯನೂ, ಶಿಷ್ಯನು ಎಂಥವನೆಂದು ಗುರುವೂ ನೋಡಿಕೊಳ್ಳಬಹುದು. ಅನಾರ್ಯನಾಗಿರುವವನನ್ನು ಹೊಗಳುವುದೂ ತಪ್ಪು, ಅನಾರ್ಯನಾಗಿರುವವನನ್ನು ಹೊಗಳದೆ ಇರುವುದೂ ತಪ್ಪು :


ಈಗ ಆರ್ಯನೆಂದೂ, ದ್ರಾವಿಡನೆಂದೂ ವಿಭಾಗ ಮಾಡಿ ಹೇಳುತ್ತಾರೆ. `ಆರ್ಯನೆಂದರೆ ಬೇರೆ ದೇಶದಿಂದ ಬಂದವನೆಂದೂ, ದ್ರಾವಿಡನೆಂದರೆ ಇಲ್ಲಿಯೇ ಇದ್ದವನೆಂದೂ, ಈ ಆರ್ಯ ದ್ರಾವಿಡರಿಗೆ ಸಂಬಂಧವಿಲ್ಲವೆಂದೂ ಹೇಳುತ್ತಾರೆ. `ಯಾರು ದ್ರಾವಿಡರು' ಎಂದರೆ  `ಇಲ್ಲಿಯೇ ಇದ್ದವರು'  ಎನ್ನುತ್ತಾರೆ. ನನಗೆ ಬಹಳ ಸಂತೋಷವಾಗಿದೆ. ಜನಿವಾರ ಹಾಕಿಕೊಳ್ಳುವವರು ಆರ್ಯರೇ? ಅಥವಾ ಜನಿವಾರ ಹಾಕಿಕೊಳ್ಳದವರು ಆರ್ಯರೇ? ಅವರು ವೇದಾಭ್ಯಾಸ ಮಾಡಿ, ಇತರರಿಗೂ ಹೇಳಿಕೊಟ್ಟು ಹೋಮಗಳನ್ನು ಮಾಡಿ ಭಗವಂತನನ್ನು ತೃಪ್ತಪಡಿಸುವುದರಿಂದ ಆರ್ಯರೇ? ಹಾಗೆ ಒಂದು ವೇಳೆ ಬ್ರಾಹ್ಮಣರೇ ಆರ್ಯರೆಂದು ಭಾವಿಸುವುದಾದರೆ ಅವರು ಬರುವಾಗ ಮಾತ್ರ ಬಂದಿದ್ದರೆಂದು ಹೇಳಲು ಸಾಧ್ಯವಿಲ್ಲ. ಅವರಿಗೆ ಎಂದೂ ನೇಗಿಲು ಹಿಡಿದು ವ್ಯವಸಾಯ ಮಾಡುವ ಅಭ್ಯಾಸವಿಲ್ಲ. ಅವರು ಬರುವಾಗ ತಮಗೆ ಸಹಾಯಕ್ಕಾಗಿ ಇತರರನ್ನೂ ಕರೆದುಕೊಂಡೇ ಬಂದಿರಬೇಕು. ಹಾಗೆ ನೋಡಿದರೆ ಬ್ರಾಹ್ಮಣರು ಮಾತ್ರ ಬಂದಿರಲು ಸಾಧ್ಯವಿಲ್ಲ. ಇತರ ಜಾತಿಯವರನ್ನೂ ಕರೆದುಕೊಂಡು ಬಂದಿರಬೇಕು. ಆದ್ದರಿಂದ ಬ್ರಾಹ್ಮಣರು ಆರ್ಯರೆಂದು, ಬರುವಾಗ ಅವರು ಮಾತ್ರ ಬಂದಿರಲಿಲ್ಲ. ಬಂದಿದ್ದರೆ ಬ್ರಾಹ್ಮಣರೊಡನೆ ಬೇರೆಯವರೂ ಬಂದಿರಬೇಕು. ಬರಲಿಲ್ಲವೆಂದರೆ ಬ್ರಾಹ್ಮಣರೊಡನೆ ಬೇರೆಯವರೂ ಇಲ್ಲಿಯೇ ಇದ್ದವರು.

ಚರಿತ್ರೆ ಬರೆಯುವವರಲ್ಲಿ ಜಗಳವಾದರೇನೇ ಅವರು ಜೀವಿಸಲು ಸಾಧ್ಯ. ಆದರಿಂದ ಅವರು, `ದ್ರಾವಿಡರು ಮೊದಲು ಬಂದರು, ಆನಂತರ ಆರ್ಯರು ಬಂದರು' ಎಂದು ಹೇಳಿದರು. ಇನ್ನೊಬ್ಬ ಚರಿತ್ರಕಾರನು, `ಹಾಗಲ್ಲ; ದ್ರಾವಿಡರು ಮೊದಲು ಇಲ್ಲಿದ್ದರು. ಆನಂತರ ಆರ್ಯರು ಬಂದರು' ಎನ್ನುತ್ತಾರೆ. ಆರ್ಯರಿಗೂ ದ್ರಾವಿಡರಿಗೂ ಜಗಳ ಉಂಟಾಗಲಿ ಎಂದುಕೊಂಡು ಅವರು ಹಾಗೆ ಹೇಳಿದರು. ಮೂರನೆಯವನಾಗಿ ಬಂದ ಹೊರದೇಶದವನು, `ಆರ್ಯರಾಗಿರುವ ನೀವೂ ಹೊರದೇಶದಿಂದ ಬಂದವರು. ದ್ರಾವಿಡರೂ ಬಂದವರು! ನೀವಿಬ್ಬರೂ ನಡೆಸುವ ಅಧಿಕಾರಕ್ಕಿಂತಲೂ ಹೆಚ್ಚಾಗಿ ಅಧಿಕಾರಿ ನಾನು ಮಾಡುವೆನು' ಎನ್ನುತ್ತಾನೆ.

ಈ ಕಥೆಯೆಲ್ಲವನ್ನೂ ಯಾರೂ ನಂಬಬಾರದು. ನಮ್ಮ ದೇಶದಲ್ಲಿ ರಾಮೇಶ್ವರದಿಂದ ಹಿಡಿದು ಹಿಮಾಲಯದವರೆಗೆ ಭಗವಂತನಾದ ಶಿವನೂ, ವಿಷ್ಣುವೂ ಹಲವು ವಿಧವಾದ ಲೀಲೆಗಳನ್ನು ತೋರಿದ್ದಾರೆ. ಯಾವ ಕಾವ್ಯವನ್ನಾಗಲಿ, ಪುರಾಣವನ್ನಾಗಲಿ ತೆಗೆದು ನೋಡಿದರೆ ಎಲ್ಲಾ ಜಾತಿಯವರೂ ಇಲ್ಲಿ ಇದ್ದದ್ದನ್ನು ನಾವು ಕಾಣುತ್ತೇವೆ. ಎಲ್ಲಾ ಜಾತಿಯವರೂ ಎಲ್ಲಾ ಕಾಲದಲ್ಲಿಯೂ ಇದ್ದುದರಿಂದ `ಕೆಲವರು ಇಲ್ಲಿದ್ದರು' `ಕೆಲವರು ಮಾತ್ರ ಬಂದರು' ಎನ್ನುವ ಮಾತಿಗೆ ಅವಕಾಶವೇ ಇಲ್ಲ. ಆದರೂ ಜನರ ಇಂಥ ಕಥೆಗಳನ್ನು ನಂಬಿಬಿಡುತ್ತಾರೆ. ಇದುತಪ್ಪು. 

ನಾವೆಲ್ಲರೂ ಭಗವಂತನ ಕೃಪೆಯಿಂದ ಒಂದಾಗಿ ಬಾಳುತ್ತಿದ್ದೇವೆ. ದಕ್ಷಿಣದಿಂದ ಗುಜರಾತ್ ಮುಂತಾದ ಸ್ಥಳಗಳಿಗೆ ಹೋಗಿ ಕೆಲವರು ಇದ್ದಾರೆ. ಇದಕ್ಕಾಗಿ ಅಲ್ಲಿರುವವರು ಇವರನ್ನು ಕತ್ತು ಹಿಡಿದು ಹೊರಕ್ಕೆ ಹಾಕಲಿಲ್ಲ. ಆದ್ದರಿಂದ ಎಲ್ಲರೂ ಒಟ್ಟಿಗೆ ಸೇರಿಯೇ ಇದ್ದೇವೆ. ಇಲ್ಲಿಂದ ಹತ್ತು ಜನ ಹೊರ ದೇಶಕ್ಕೆ ಹೋಗಿರಬಹುದು. ಅಲ್ಲಿಂದ ಹತ್ತು ಜನ ಇಲ್ಲಿಗೆ ಬಂದಿರಬಹುದು. ಆದ್ದರಿಂದ ಒಳಕ್ಕೆ (ಇಲ್ಲಿಗೆ ) ಬಂದವರೆಲ್ಲರನ್ನೂ ಆರ್ಯರೆಂದು ಹೇಳುವುದು ತಪ್ಪು. 

ಡೆಲ್ಲಿಗೆ ಹೋಗಿ ನೋಡಿದರೆ ಅಲ್ಲಿ ದಕ್ಷಿಣದವರು ಇರುವ ಜಾಗವಿದೆ. ಅದನ್ನು ನೋಡಿದರೆ ದಕ್ಷಿಣ ಭಾರತದಂತೆಯೇ ಇರುವುದು. ಅದನ್ನು `ಡೆಲ್ಲಿ' ಎನ್ನುವುದೇ ಕಷ್ಟ, ಹಾಗೆಯೇ ಬೊಂಬಾಯಿಯಲ್ಲಿ `ಮಾತಂಗಾ'ಗೆ ಹೋಗಿ ನೋಡಿದರೆ ಅಲ್ಲಿರುವ ಉತ್ತರ ಭಾರತದವರೂ ತಮಿಳು ಮಾತನಾಡಲು ಕಲಿತಿದ್ದಾರೆ. ಆದ್ದರಿಂದ ಅವರನ್ನು ಆ ಸ್ಥಳದವರೆಲ್ಲವೆಂದು ಹೇಳಲು ಸಾಧ್ಯವೇ? ಕೆಲವರು ಇಲ್ಲಿಂದ ಹೋದುದರಿಂದ ಮೊದಲು ಬೊಂಬಾಯಿಯಲ್ಲಿ ಯಾರೂ ಇರಲಿಲ್ಲವೆಂದು ಹೇಳಲು ಸಾಧ್ಯವೇ? ಅಲ್ಲಿ ಕೆಲವರು ಬ್ರಾಹ್ಮಣರು ಇರುವುದರಿಂದ ಬ್ರಾಹ್ಮಣರೆಲ್ಲರೂ ಇಲ್ಲಿಂದ ಹೋದವರೆಂದು ಹೇಳಲು ಸಾಧ್ಯವೆ? ಬ್ರಾಹ್ಮಣರೂ ಹೋದರು. ಇತರರೂ ಹೋದರು. ಅದೇ ರೀತಿ ಅಲ್ಲಿದ್ದವರು ಕೆಲವರು ಇಲ್ಲಿಗೆ ಬಂದಿರಬಹುದು. ಆದ್ದರಿಂದ ಆರ್ಯದ್ರಾವಿಡ ಎನ್ನುವ ಭೇದಭಾವವನ್ನು ಎಲ್ಲಾರೂ ಬಿಟ್ಟುಬಿಡಬೇಕು. ಆರ್ಯನೂ ದ್ರಾವಿಡನೂ ಪರಸ್ಪರ ಜಗಳವಾಡುತ್ತಾರೆ. ಕೊನೆಗೆ ಹೊರದೇಶದವನು ಅಧಿಕಾರವನ್ನು ಕಿತ್ತುಕೊಳ್ಳುತ್ತಾನೆ. ಅದೇ ಮುಕ್ತಾಯವಾಗುತ್ತದೆ. ಆರ್ಯನೆಂದರೆ ಔನ್ನತ್ಯವುಳ್ಳವನು. ಉತ್ತಮವಾದ ಬಾಳನ್ನು ನಡೆಸುತ್ತಾ ಯಾರು ಭಗವಂತನ ಕೃಪೆಯನ್ನು ಸಂಪಾದಿಸಿದ್ದಾನೋ ಅವನೇ ಆರ್ಯನು. 

`ಮುದ್ರಾರಾಕ್ಷಸಂ' ಎನ್ನುವ ನಾಟಕದಲ್ಲಿ ಅಮಾತ್ಯ ರಾಕ್ಷಸನ್ನೆನ್ನುವ ಕಥಾಪಾತ್ರವಿದೆ. ಅವನು `ಆರ್ಯ' ನೆಂದು ಕರೆಯಲ್ಪಟ್ಟವನು. ಆದರೆ ಅವನ ಮೇಲೆ ವ್ಯರ್ಥವಾಗಿ ಸೇಡಿನ ಆಪಾದನೆಯಾಯಿತು. ಅನಂತರ ಅವನು `ಅನಾರ್ಯ'ನಾಗಿಬಿಟ್ಟನು. ಆದ್ದರಿಂದ ಆರ್ಯನಾಗಿರುವವನು ಅನಾರ್ಯನಾಗಬಹುದು. ಔನ್ನತ್ಯದಲ್ಲಿರುವವನು ಕೆಟ್ಟಕೆಲಸ ಮಾಡಿದರೆ ಅವನು ಅನಾರ್ಯನಾಗುವನು. ಅವನು ಶ್ರೇಷ್ಠನಾಗಿದ್ದರೆ `ಆರ್ಯ'ನೆಂದು ಹೇಳಲ್ಪಡುವನು. ಆದ್ದರಿಂದ ಶ್ರೇಷ್ಠನಾಗಿರುವವನನ್ನು ಸೇವಿಸದೆ ಇರುವುದೂ, ಶ್ರೇಷ್ಠನಲ್ಲದವನನ್ನು ಸೇವಿಸುವುದೂ ತಪ್ಪು. 

ಹಾವು ಕಡಿದರೆ ಕಾಸು ತೆಗೆದುಕೊಂಡು ಒಂದು ಔಷಧವನ್ನು ಒಬ್ಬನು ಕೊಡುವನು. ಚೇಳುಕಡಿತಕ್ಕೂ ಔಷಧ ಕೊಡುವನು. ಇವರು ಕೂಡ `ಗುರು' ಎನ್ನುವ ಶ್ರೇಣಿಗೆ ಸೇರಿದರೂ ಸಹ ಸದ್ಗುರುವಿನ ಮಹಿಮೆ ಬಹಳ ದೊಡ್ಡದು. ಗುರುವನ್ನು ಅರಾಧಿಸಿದರೆ ಯಾವ ವಿಶೇಷವಾದ ಫಲ ಉಂಟಾಗುವುದು ಎನ್ನುವುದಕ್ಕೆ-

\begin{shloka} 
`ಯಸ್ಯ ದೇವೇ ಪರಾಭಕ್ತಿಃ ಯಥಾ ದೇವೇ ತಥಾ ಗುರೌ |\\
ತಸ್ಯೈತೇ ಕಥಿತಾಹ್ಯರ್ಥಾಃ ಪ್ರಕಾಶನ್ತೇ ಮಹಾತ್ಮನಃ ||
\end{shloka}

(ಭಗವಂತನ ವಿಷಯದಲ್ಲಿ ಎಲ್ಲೆ ಇಲ್ಲದಷ್ಟು ಭಕ್ತಿ ಯಾರಿಗೆ ಇದೆಯೋ, ಅದೇ ರೀತಿ ಗುರುತಿನ ವಿಷಯದಲ್ಲಿಯೂ ಯಾರಿಗೆ ಭಕ್ತಿ ಇರುವುದೋ, `ಅಂಥ ಮಹಾತ್ಮನ ಹತ್ತಿರ' ಹೇಳಿದ ಅರ್ಥಗಳೆಲ್ಲವೂ ಪ್ರಕಾಶಿಸುತ್ತವೆ.)
 
 ಯಾರು ಗುರುವಿನ ಮೇಲೆ ವಿಶೇಷವಾದ ಭಕ್ತಿಯನ್ನು ಇಟ್ಟುಕೊಂಡು, ಅವರನ್ನು ಆರಾಧಿಸುತ್ತಾ ಬರುತ್ತಾರೋ, ಅವರಿಗೆ ಗುರು ಏನೂ ಹೇಳಿಕೊಡದೆ ಇದ್ದರೂ ಸಹ ಎಲ್ಲ ವಿದ್ಯೆಗಳೂ ಬಂದು ಬಿಡುವುದು, ಇದಕ್ಕೆ ಸಂಬಂಧಪಟ್ಟಂತೆ ಕಥೆ ಹೇಳುವುದುಂಟು. 
 
 ಮಹರ್ಷಿ ಒಬ್ಬರು ಇದ್ದರು. ಅವರಿಗೆ ಹಲವುಮಂದಿ ಶಿಷ್ಯರಿದ್ದರು. ಅಂಥವರಲ್ಲಿ ಒಬ್ಬರನ್ನು `ಇವನಿಗೆ ಎಷ್ಟು ಗುರುಭಕ್ತಿ ಇದೆ' ಎಂದು ಪರೀಕ್ಷೆ ಮಾಡಲು ಯೋಚಿಸಿದರು. ಆದ್ದರಿಂದ ಅವರು ಪ್ರತಿನಿತ್ಯವೂ ಅವನಿಗೆ ದನ ಕಾಯುವಂತೆ ಹೇಳಿದರು, `ಒಂದು ನಿಮಿಷವೂ ಕೂಡ ಸಮ್ಮನೆ ಇರಬಾರದು' ಎಂದು ಹೇಳಿದರು. ಅವನು ಪ್ರತಿನಿತ್ಯವೂ ಅದೇ ರೀತಿ ಮಾಡಿದನು, ಹಾಗದರೂ ಅವನು ಆರೋಗ್ಯವಾಗಿದ್ದನು. ಗುರು ಕೇಳಿದಾಗ ಅವನು ಗುರುವಿನೊಡನೆ `ನಾನು ಭಿಕ್ಷಾನ್ನ ತಿನ್ನುತಿದ್ದೇನೆ' ಎಂದನು. ಗುರು `ಇನ್ನು ಮುಂದೆ ನೀನು ನನಗೆ ಭಿಕ್ಷಾನ್ನವನ್ನು ಕೊಟ್ಟು ಬಿಡಬೇಕು' ಎಂದರು. ಶಿಷ್ಯನು ಮೊದಲಿನಂತಯೇ ಆರೋಗ್ಯವಾಗಿಯೇ ಇದ್ದನು. ಗುರುವು ಕೇಳಿದ್ದಕ್ಕೆ  `ಈಗ ಎರಡನೆಯ ಸಲ ಭಿಕ್ಷಾನ್ನ ಮಾಡಿ ಊಟ ಮಾಡುತ್ತೇನೆ' ಎಂದನು, ಅದಕ್ಕೆ ಗುರು `ಇನ್ನು ಮೇಲೆ ಎರಡನೆಯ ಸಲ ಭಿಕ್ಷಾನ್ನ ಮಾಡಕೂಡದು, ಅದು ತಪ್ಪು' ಎಂದರು. 
 
 
 ಆದಾದ ಕೆಲವು ದಿನಗಳ ಮೇಲೂ ಅವನು ಆರೋಗ್ಯವಾಗಿದ್ದನು, ಗುರು ಅದಕ್ಕೆ ಕಾರಣ ಕೇಳಿದಾಗ `ಹಸುಗಳನ್ನು ಕಾಯುವುದರಿಂದ ಹಸುಗಳ ಹಾಲನ್ನು ಕುಡಿಯುತ್ತೇನೆ' ಎಂದನು. ಗುರು `ಇನ್ನು ಮೇಲೆ ನೀನು ಹಸುವಿನ ಹಾಲು ಕುಡಿಯಬಾರದು, ನನಗೆ ಬರಬೇಕಾದ ಹಾಲನ್ನು ಕುಡಿದರೆ ಅದು ಕಳ್ಳತನ ಮಾಡಿದಂತೆ' ಎಂದರು. 
 
 
 ಅನಂತರ ಶಿಷ್ಯನು ಹಸುವಿನ ಹಾಲನ್ನು ಕುಡಿಯಲಿಲ್ಲ.  ಕರುಗಳು ಹಾಲು ಕುಡಿದಮೇಲೆ ಇವನ ಮೇಲಿದ್ದ ಪ್ರೀತಿಯಿಂದಾಗಿ ನೋರೆಗಳನ್ನು ಇಟ್ಟುಕೊಳ್ಳುತ್ತಿದ್ದವು. ಆದ್ದರಿಂದ ಆ ನೊರೆಗಳನ್ನು ಕುಡಿದು, ಬರುತ್ತಿದ್ದನು. ಇದನ್ನು ಅರಿತ ಗುರು `ನೊರೆಯನ್ನೂ ನೀನು ಕುಡಿಯಬಾರದು' ಎಂದರು. ಕೊನೆಯಲ್ಲಿ ಅವನು ಎಕ್ಕದ ಎಲೆಗಳನ್ನು ತಿನ್ನುತ್ತಾ ಬಂದನು. ಎಕ್ಕದ ಎಲೆಯನ್ನು ತಿಂದಿದುರಿಂದ ಅವನ ಕಣ್ಣುಗಳು ಕುರುಡಾಗಿ ಅವನು ನಡೆದು ಹೋಗುವಾಗ ಒಂದು ಹಳೆಯ ಬಾವಿಯಲ್ಲಿ ಬಿದ್ದುಬಿಟ್ಟನು. ಹಸುಗಳೆಲ್ಲಾ ಇವನ ಇಲ್ಲದೆಯೇ ಮನೆಗೆ ಹೋದವು. 
 
 ಶಿಷ್ಯನು ಬರಲಿಲ್ಲವಲ್ಲಾ ಎಂದು ಗುರುವಿಗೆ ಬಹಳ ಚಿಂತೆಯಾಯಿತು. `ನಾನು ಪರೀಕ್ಷೆ ಮಾಡಬೇಕೆಂದುಕೊಂಡರೆ ಕೊನೆಗೆ ಶಿಷ್ಯನೇ ಬರಲಿಲ್ಲವಲ್ಲಾ. ಇದರಲ್ಲಿ ಏನು ತಪ್ಪಾಯಿತೋ ಎಂದುಕೊಂಡು ಅವನನ್ನು ಹುಡುಕುತ್ತಾ ಅವರು ಹೊರಟರು. `ಎಲ್ಲಿದ್ದೀಯೇ ಎಲ್ಲದ್ದೀಯೇ' ಎಂದು ಕೇಳುವ ಗುರುವಿನ ಕಂಠಧ್ವನಿಯನ್ನು ಕೇಳಿ ಶಿಷ್ಯನು ಬಾವಿಯಿಂದ, `ನಾನು ಇಲ್ಲೇ ಇದ್ದೇನೆ' ಎಂದನು, ಗುರು ಬಾವಿಯ ಹತ್ತಿರಕ್ಕೆ ಬಂದರು. `ಏಕಪ್ಪಾ ಬಾವಿಯಲ್ಲಿ ಬಿದ್ದಿದ್ದೀಯೇ?' ಎಂದು ಕೇಳಿದರು. `ಏನೂ ಇಲ್ಲ ನಿಮ್ಮ ಆಜ್ಞೆಯನ್ನು ಪರಿಪಾಲಿಸಲಿಲ್ಲ. ಎಕ್ಕದ ಎಲೆಯನ್ನು ತಿಂದೆನು. ಆದರಿಂದ ಕಣ್ಣು ಕಾಣದೇಹೋಯಿತು, ಎಂದು ಶಿಷ್ಯನು ನುಡಿದನು. ಗುರು ಅವನಿಗೆ `ವೇದದಲ್ಲಿ ಒಂದು  ನಿರ್ದಿಷ್ಟವಾದ ಭಾಗವನ್ನು ನೀನು ಹೇಳು. ಅದು ಅಶ್ವಿನೀ ದೇವತೆಗಳನ್ನು ಸ್ತುತಿಸುವ ರೂಪದಲ್ಲಿದೆ. ಅದನ್ನು ಹೇಳಿದರೆ ನಿನಗೆ ಕಣ್ಣುಗಳು ಬರುತ್ತವೆ' ಎಂದರು. 
 
 ಶಿಷ್ಯನು ಅದೇ ರೀತಿ ಮಾಡಿದನು.  ಅಶ್ವಿನೀಕುಮಾರರು ಪ್ರತ್ಯಕ್ಷರಾಗಿ ಗುರುವಿನ ಬೇಡಿಕೆಯನ್ನು  ಮನ್ನಿಸಿದರು. ಶಿಷ್ಯನು ಕಳೆದುಕೊಂಡ ಕಣ್ಣುಗಳನ್ನು ಮತ್ತೆ ಪಡೆದನು. ಬಾವಿಯಿಂದ ಮೇಲೆಕ್ಕೆ ಬಂದು ಅವನು `ನನ್ನ ಪಾಪದಿಂದ ಕಳೆದುಕೊಂಡ ಕಣ್ಣುಗಳನ್ನು ನಿಮ್ಮ ದಯೆಯಿಂದ ಮತ್ತೆ  ಪಡೆದನು' ಎಂದನು. ಗುರು `ನಿನ್ನ ಪಾಪ ಇದಕ್ಕೆ ಕಾರಣವಲ್ಲ, ನಾನು ಪರೀಕ್ಷೆ ಮಾಡಿದುದೇ ಕಾರಣ. ನೀನು ಇನ್ನು ಮೇಲೆ ಓದುಬೇಕಾಗಿಯೂ ಇಲ್ಲ. ನನ್ನನ್ನು ಸೇವಿಸಬೇಕಾಗಿಯೂ ಇಲ್ಲ. ನಿನಗೆ ಚತುರ್ದಶ  ವಿದ್ಯೆಗಳು ತಾವಾಗಿಯೇ  ಬಂದುಬಿಡುವೆವು'ಎಂದರು. ಆದ್ದರಿಂದ ನಾವು ಯಾವಾಗಲೂ ಸದ್ಗುರುವಿಗೆ ಶರಣಾಗಬೇಕು.
 
 \begin{shloka}
 `ನೈಷಾ ತರ್ಕೇಣ ಮತಿರಾಪನೇಯಾ\\
 ಪ್ರೋಕ್ತಾನ್ಯೇನೈವ ಸುಜ್ಞಾನಾಯ ಷ್ರೇಷ್ಠ!''
 \end{shloka}
 
 
 ಪ್ರೀತಿಯುಳ್ಳವನೇ! ಇತರರಿಂದ ಉಪದೇಶಿಸಲ್ಪಟ್ಟ ಅನಂತರವೇ ಉತ್ತಮವಾದ ಜ್ಞಾನಕ್ಕೆ ಮಾರ್ಗವಾಗಿರುವ ನಿನ್ನ ಅರಿವು. ತರ್ಕದಿಂದ ಪಡೆಯಲಾಗುವುದಿಲ್ಲ.
 
 ನಮಗೆ ತರ್ಕದಿಂದ ತಿಳಿಯುವಂಥ ಸಾರ್ಮರ್ಥ್ಯ ಮೂಲತಃ ತಾನಾಗಿಯೇ ಇರುವುದರಿಂದ ಪರವಸ್ತುವನ್ನು ತಿಳಿಯಬಲ್ಲೆವು ಎಂದುಕೊಂಡರೆ ಅದು ತಪ್ಪು, ಹಾಗೆ ತಿಳಿಯುವ ಶಿಷ್ಯನಿಗೆ, ಅವನು ಬೇರೆ ಯಾವುದಾದರೂ ಅರಿವನ್ನು ಸಂಪಾದಿಸಿದ್ದರೂ ಆ ಅರಿವು ಮರೆತು ಹೋಗುವುದು. 
 
 ಪರವಸ್ತುವನ್ನು ಅರಿತ ಆಚಾರ್ಯರ ಹತ್ತಿರ ಪಡೆದ ಜ್ಞಾನವೇ ಸ್ಥಿರವಾದ ಜ್ಞಾನವಾಗುವುದು.  ಆದ್ದರಿಂದ ಸದ್ಗುರುವಿನ  ಹತ್ತಿರ ನಾವು ಜ್ಞಾನವನ್ನು ಪಡೆದುಕೊಂಡು, ಅವರ ಬಗ್ಗೆ  ಕೃತಜ್ಞರಾಗಿರಬೇಕು.
 
  


