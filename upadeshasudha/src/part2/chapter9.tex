\chapter{ಮನುಷ್ಯಜನ್ಮ ಮತ್ತು ಶ್ರೇಷ್ಠ ಯೋಗ್ಯತೆಗಳು}\label{chap9}

\begin{shloka}
ವಿಶುದ್ಧಜ್ಞಾನದೇಹಾಯ ತ್ರಿವೇದೀ ದಿವ್ಯಚಕ್ಷುಷೇ |\\
ಶ್ರೇಯಃಪ್ರಾಪ್ತಿನಿಮಿತ್ತಾಯ ನಮಸ್ಸೋಮಾರ್ಧಧಾರಿಣೇ ||\\
ನಮಾಮಿ ಯಾಮಿನೀನಾಥಲೇಖಾಲಂಕೃತಕುಂತಲಾಮ್ |\\
ಭವಾನೀಂ ಭವಸನ್ತಾಪನಿರ್ವಾಪನ ಸುಧಾನದೀಮ್ ||
\end{shloka}

``ಯಾರಿಗೆ ಭಗವಂತನ ವಿಷಯದಲ್ಲಿ ಎಲ್ಲೆ ಇಲ್ಲದಷ್ಟು ಭಕ್ತಿ ಇರುವುದೋ, ಭಗವಂತನ ವಿಷಯದಲ್ಲಿರುವ ಭಕ್ಕ್ತಿಯಂತೆಯೇ ಗುರುವಿನ ವಿಷಯದಲ್ಲಿಯೂ ಭಕ್ತಿ (ಯಾರಿಗೆ) ಇರುವುದೋ, ಅಂಥ ಮಾಹಾತ್ಮರಲ್ಲಿ, ಹೇಳಲ್ಪಟ್ಟ ಅರ್ಥಗಳೂ, ಹೇಳದೇ ಇರುವ ಅರ್ಥಗಳೂ (ಶಾಸ್ತ್ರಗಳು) ತಾನಾಗಿಯೇ ಪ್ರಕಾಶಿಸುತ್ತವೆ'' - ಎನ್ನುವ ವಾಕ್ಯ ಉಪನಿಷತ್ತಿನಲ್ಲಿ ಬರುತ್ತದೆ. ಇದರ ತಾತ್ಪರ್ಯವೇನೆಂದರೆ ಭಕ್ತಿ ಎನ್ನುವುದು ಜ್ಞಾನಕ್ಕೆ ಸಾಧನವಾಗಿದೆ. ಭಕ್ತಿಗೆ ಅವಲಂಬನವಾಗಿ ದೈವ ಇರಬೇಕೆಂದು ನಾವು ಇಟ್ಟುಕೊಳ್ಳುತ್ತೇವೆ. ಭಕ್ತಿಯಲ್ಲಿ ಅರ್ತ, ಅರ್ಥಾರ್ಥಿಗೆ ಸಂಬಂಧಪಟ್ಟ ಎರಡು ಭೇದಗಳು ಇವೆ. ಅಂಥ ಭಕ್ತಿಗಳೆಲ್ಲವೂ ಒಂದು ವಿಧವಾದ ವ್ಯಾಪಾರವೆಂದೇ ನಾವು ಹೇಳಬೇಕು. ಆಗ ನಾವು ಮಾಡುವ ಭಕ್ತಿಗೆ ಭಗವಂತನು ಫಲವನ್ನು ಕೊಟ್ಟರೆ ನಾವು ಭಗವಂತನಲ್ಲಿ ಭಕ್ತಿಯುಳ್ಳವರಾಗಿರಬಹುದು. ಇಲ್ಲದಿದ್ದರೆ ಭಕ್ತಿಯೂ ಹೋಗಿಬಿಡಬಹುದು. ಆದರೆ ಜ್ಞಾನವನ್ನು ಅಪೇಕ್ಷೆಪಡುವವನು ಭಗವಂತನೊಡನೆ ವ್ಯಾಪಾರಿಯಂತೆ ಇರದೆ ವಿಶೇಷವಾಗಿ ಭಕ್ತಿಯುಳ್ಳವನಾಗಿರುವನು. ಅಂಥ ಭಕ್ತಿಗೆ ಫಲವೇನೆಂದರೆ ಜ್ಞಾನವೇ ಫಲ.

ಜ್ಞಾನ ಉಂಟಾದಮೇಲೆ

\begin{shloka}
``ಜ್ಞಾನೀತ್ವಾತ್ಮೈವ ಮೇ ಮತಮ್''
\end{shloka}

(ಜ್ಞಾನಿ ನಾನೇ ಎನ್ನುವುದು ನನ್ನ ಅಭಿಮತ) -ಎಂದು ಹೇಳಿದಂತೆ, ಭಕ್ತನಿಗೂ ಭಗವಂತನಿಗೂ ಯಾವ ವಿಧವಾದ ವ್ಯತ್ಯಾಸವೂ ಇರುವುದಿಲ್ಲ. ``ಸಿಂಧುಸ್ಸರಿದ್ವಲ್ಲಭಂ'' ಎಂದು ಹೇಳಿದಂತೆ ಹೇಗೆ ನದಿಗಳು ಸಮುದ್ರವನ್ನು ಸೇರಿದಮೇಲೆ ಅವುಗಳಲ್ಲಿ ವ್ಯತ್ಯಾಸವಿರುವುದಿಲ್ಲವೋ, ಹಾಗೆಯೇ ಭಕ್ತನಾದವನು ಭಗವಂತನೊಡನೆ ಒಂದಾಗಿ ಬಿಡುತ್ತಾನೆ. ಆದ್ದರಿಂದ ಯಾವುದಾದರೂ ಒಂದು ಫಲವನ್ನು ಎದುರು ನೋಡುತ್ತಾ ನಾವು ಮಾಡುವ ಭಕ್ತಿ ಶ್ರೇಷ್ಠವಾದುದಲ್ಲ. ವ್ಯಾಪಾರದಂತಿರುವ ಭಕ್ತಿಯನ್ನು ತೆಗೆದುಕೊಳ್ಳುವವರು ತಾವು ಯಾವುದಾದರೂ ಫಲಕ್ಕಾಗಿ ಭಗವಂತನಿಗೆ ಶರಣಾಗಿ, ಅದರಿಂದ ಫಲವನ್ನು ಪಡೆದ ಮೇಲೆ ``ನಾವು ಭಗವಂತನಿಂದ ಇನ್ನು ಯಾವ ಫಲವನ್ನು ಪಡೆಯಬೇಕು. ನಮ್ಮ ಮಕ್ಕಳೆಲ್ಲಾ ಒಳ್ಳೆಯ ಕೆಲಸದಲ್ಲಿದ್ದಾರೆ. ನಾವೂ ಒಳ್ಳೆಯ ಆರೋಗ್ಯವಂತರಾಗಿದ್ದೇವೆ. ಹಾಗಿರುವಾಗ ಏತಕ್ಕೆ ಭಗವಂತನನ್ನು ಪ್ರಾರ್ಥನೆ ಮಾಡಬೇಕು'' -ಎಂದು ಯೋಚಿಸುವರು. ಆದರೆ ಜ್ಞಾನವನ್ನು ಪಡೆಯಬೇಕೆಂದುಕೊಳ್ಳುವವನು ಹಾಗಲ್ಲ. ಒಂದು ಸಕ್ಕರೆ ಬೊಂಬೆ ಸಮುದ್ರದ ಆಳವನ್ನು ನೋಡಲು ಅದರಲ್ಲಿ ಇಳಿಯಿತು. ಅನಂತರ ಆ ಬೊಂಬೆಯನ್ನು ನಾವು ಹೇಗೆ ನೋಡಬಹುದು? ಸಮುದ್ರದ ಆಳವನ್ನು ನೋಡಲು ಹೋದ ಆ ಬೊಂಬೆ ಕೆಳಗೆ ಹೋಗುತ್ತಾ ತಾನು ಚಿಕ್ಕದಾಗುತ್ತಲೇ ಹೋಗುತ್ತದೆ. ಕೊನೆಗೆ ಆ ಬೊಂಬೆ ಕಾಣುವುದಿಲ್ಲ. ಅದು ಸಮುದ್ರದಲ್ಲಿ ಕರಗಿಹೋಗುತ್ತದೆ. ಇದೇ ಜೀವನ ಸ್ಥಿತಿ. ಆದ್ದರಿಂದ ಭಕ್ತನಾದವನು ಭಗವಂತನ ಭಕ್ತ ಎನ್ನುವ ಸಮುದ್ರಕ್ಕೆ ಹೋದರೆ, ಆ ಬೊಂಬೆ ಸಮುದ್ರವನ್ನು ಬಿಟ್ಟು ಹೊರಗೆ ಬಾರದಂತೆ, ಜ್ಞಾನವನ್ನು ಅಪೇಕ್ಷಿಸುವವನು ಜ್ಞಾನವನ್ನು ಪಡೆದ ಮೇಲೆ ಭಗವಂತನೊಡನೆ ಒಂದಾಗಿ ಬಿಡುತ್ತಾನೆ. ಆದ್ದರಿಂದ ಜ್ಞಾನವನ್ನು ಅಪೇಕ್ಷಿಸುವ ಜಿಜ್ಞಾಸುವಿನಲ್ಲಿರುವ ಭಕ್ತಿ ಬಹಳ ಶ್ರೇಷ್ಠವಾದುದು. ಅಂಥ ಜಿಜ್ಞಾಸುವಿಗೆ ``ಪರಾ ಭಕ್ತಿ'' ಎನ್ನುವುದು ಇದೆ, ಜ್ಞಾನವನ್ನು ಪಡೆಯಬೇಕೆನ್ನುವ ಅವನ ಅಪೇಕ್ಷೆ ಪೂರ್ತಿಯಾಗಿ ನೆರವೇರಬೇಕಾದರೆ ``ಯಥಾ ದೇವೇ ತಥಾ ಗುರೌ'' -ಎಂದು ಹೇಳಿರುವುದರಿಂದ ಹೇಗೆ ದೇವರ ಮೇಲೆ ಭಕ್ತಿ ಇಟ್ಟುಕೊಂಡಿದ್ದಾನೋ ಹಾಗೆಯೇ ಗುರುವಿನ ಮೇಲೂ ಭಕ್ತಿಯನ್ನು ಇಟ್ಟುಕೊಂಡಿರಬೇಕೆಂದು ಹೇಳಲಾಗಿದೆ. ಇಂಥ ಭಕ್ತಿ ಇಟ್ಟುಕೊಂಡಿರುವವನಿಗೆ ಯಾವ ಫಲ ದೊರೆಯುತ್ತದೆ ಎಂದು ಕೇಳಿದರೆ ಅವನ ಜಿಜ್ಞಾಸೆ ಪೂರ್ಣವಾಗುತ್ತದೆ.

\begin{shloka}
``ತಸ್ಮೈತೇ ಕಥಿತಾಹ್ಯರ್ಥಾಃ ಪ್ರಕಾಶನ್ತೇ ಮಹಾತ್ಮನಃ''
\end{shloka}

ಗುರು ಶಿಷ್ಯನಿಗೆ ಕಲಿಸುತ್ತಾರೆ. ಶಿಷ್ಯನ ಭಕ್ತಿ ವಿಶೇಷವಾಗಿದ್ದರೆ ಗುರು ಕಲಿಸುವ ಅವಶ್ಯಕತೆ ಇಲ್ಲದೆ ಇದ್ದರೂ ಅವರು ಆಶೀರ್ವಾದ ಮಾಡುತ್ತಲೇ ತತ್ತ್ವದ ತಾತ್ಪರ್ಯ ಶಿಷ್ಯನಿಗೆ ತಿಳಿಯುತ್ತದೆ. ಇದಕ್ಕೆ ಉದಾಹರಣೆಯಾಗಿ ನಾವು ಏಕಲವ್ಯನ ಕಥೆಯನ್ನು ತೆಗೆದುಕೊಳ್ಳಬಹುದು. ಏಕಲವ್ಯನು ಗುರುವಿನ ಮೇಲೆ ಅಪಾರವಾಗಿ ಭಕ್ತಿ ಇಟ್ಟುಕೊಂಡಿದ್ದನು. ಆದರೆ ಗುರುವಿಗೆ ಶಿಷ್ಯನ ಮೇಲೆ ಅಷ್ಟು ವಿಶ್ವಾಸವಿರಲಿಲ್ಲ. ಗುರುವಿಗೆ ಮೊದಲು ಅರ್ಜುನನು ಸಿಕ್ಕಿದ್ದರೂ. ತಾನು ಅರ್ಜುನನಿಗೆ ಯಾವ ರೀತಿಯಲ್ಲೂ ಕಡಿಮೆಯಲ್ಲವೆಂದು ತೋರಿಸಿದನು. ಗುರುಭಕ್ತಿಯಿಂದ ಎಂಥ ಫಲ ದೊರೆಯುತ್ತದೆ ಎನ್ನುವುದನ್ನು ನಾವು ಸಾಧಾರಣವಾಗಿ ಲೌಕಿಕ ವಿಷಯದಲ್ಲೇ ನೋಡುವಾಗ ಆಧ್ಯಾತ್ಮ ವಿಷಯದ ಬಗ್ಗೆ ಹೇಳಬೇಕಾಗಿಲ್ಲ.


\begin{shloka}
``ತಸ್ಯೈತೇ ಕಥಿತಾಹ್ಯರ್ಥಾಃ ಪ್ರಕಾಶನ್ತೇ ಮಹಾತ್ಮನಃ''
\end{shloka}

ಎಂದು ಹೇಳಿದಂತೆ ಇಂಥ ಗುರುಭಕ್ತಿಯುಳ್ಳವರನ್ನು ನಾವು ಅಷ್ಟು ಸುಲಭವಾಗಿ ನೋಡಲು ಸಾಧ್ಯಾವಾಗುವುದಿಲ್ಲ. ಹುಡಿಕಿ ನೋಡಿದರೆ ಒಬ್ಬರೋ ಇಬ್ಬರೋ ದೊರೆಯುತ್ತಾರೆ. ಭಗವತ್ಪಾದರು,

\begin{shloka}
``ಶಾಸ್ತ್ರಸ್ಯ ಗುರುವಾಕ್ಯಸ್ಯ ಸತ್ಯಬುದ್ಧ್ಯಾವತಾರಣಾ |\\
ಸಾ ಶ್ರದ್ಧಾ ಕಥಿತಾ ಸದ್ಭಿಃ ಯಯಾ ವಸ್ತೂಪಲಭ್ಯತೇ ||
\end{shloka}

(ಯಾವುದರಿಂದ ನಿಜವಾದ ವಸ್ತು ತಿಳಿಯುವುದೊ, ಆ ಶಾಸ್ತ್ರವೂ, ಗುರುವಿನ ಉಪದೇಶವೂ ಸತ್ಯವೆನ್ನುವ ಭಾವನೆಯಿಂದಲೇ ಉಂಟಾಗುವ ನಿಶ್ಚಿತವಾದ ನಂಬಿಕೆ `ಶ್ರದ್ಧೆ'ಯೆಂದು ಒಳ್ಳೆಯವರಿಂದ ಹೇಳಲ್ಪಡುತ್ತದೆ.)

-ಎಂದು ಹೇಳಿದಂತೆ ಶಾಸ್ತ್ರವೂ, ಗುರುವಾಕ್ಯವೂ ಈ ಎರಡೂ ಸತ್ಯವೆನ್ನುವ ಭಾವನೆ ಒಬ್ಬನಿಗೆ ಇರಬೇಕು. ನಾವು ಅದನ್ನು ಬಿಟ್ಟು ಈ ದೇಹವನ್ನೇ ಆತ್ಮವೆಂದು ಬಹಳ ನಿಜವಾಗಿ ನಂಬಿಕೊಂಡಿದ್ದೇವೆ.

\begin{shloka}
``ದೇಹಾತ್ಮಾಪ್ರತ್ಯಯೋ ಯದ್ವತ್ ಪ್ರಮಾಣತ್ವೇನ ಕಲ್ಪಿತಃ |\\
ಲೌಕಿಕಂ ತದ್ವದೇವೇದಂ ಪ್ರಮಾಣಂ ತ್ವಾತ್ಮ ನಿಶ್ಚಯಾತ್ ||'
\end{shloka}

(ಹೇಗೆ ಪ್ರಮಾಣದಿಂದ ಭಾವಿಸಿಕೊಂಡು ದೇಹವೇ ಆತ್ಮವೆಂದು ತೀರ್ಮಾನಿಸಲಾಗಿದೆಯೋ ಹಾಗೆಯೇ ಈ ಲೋಕವೂ ಕೂಡ ಆತ್ಮಜ್ಞಾನ ಉಂಟಾಗುವವರಿಗೆ ಪ್ರಮಾಣದಿಂದ ನಿಶ್ಚಯಿಸಲ್ಪಟ್ಟಿವೆ.)

ಒಬ್ಬನನ್ನು ಹೊಡೆದುಬಿಟ್ಟು ``ನಾನು ನಿನ್ನನ್ನು ಹೊಡೆದಿಲ್ಲ. ನೀನು ಆ ದೇಹಕ್ಕೆ ದೂರವಿರುವವನು. ನಾನು ದೇಹವನ್ನು ಮಾತ್ರ ಹೊಡೆದೆನೆಂದು ಹೇಳಿದರೆ ಅವನು ಒಪ್ಪುವುದಿಲ್ಲ. ``ಇಲ್ಲ, ಇಲ್ಲ, ನೀನು ನನ್ನನ್ನೇ ಹೊಡೆದದ್ದು, ನೀನು ಎರಡು ಏಟು ಹೊಡೆದರೆ ನಾನು ನಿನಗೆ ಎರಡು ಎರಡು ಏಟು ಹೊಡೆಯುತ್ತೇನೆ'' ಎಂದು ಅವನು ಹೇಳುತ್ತಾನೆ. ಆದ್ದರಿಂದ ಹೇಗೆ ಮನುಷ್ಯನಿಗೆ ಈ ಶರೀರವೇ ಆತ್ಮ ಎನ್ನುವ ನಂಬಿಕೆ ಇದೆಯೋ ಹಾಗೆಯೇ ಶಾಸ್ತ್ರದಲ್ಲಿಯೂ, ಗುರುವಾಕ್ಯದಲ್ಲಿಯೂ ಸತ್ಯನಿಶ್ಚಯವಿರಬೇಕು. ವಜ್ರವನ್ನು ಪರೀಕ್ಷಿಸುವುದಕ್ಕ್ಕಾಗಿ ನಾವು ವಿಚಾರಿಸುತ್ತಿರುತ್ತೇವೆ. ವಜ್ರವನ್ನು ನಾವು ಇಟ್ಟುಕೊಂಡು ಪರಿಶೋಧನೆ ಮಾಡಬೇಕೆಂದುಕೊಳ್ಳುವಾಗ ಅದನ್ನು ಪರೀಕ್ಷಿಸುವ ನಿಪುಣನು, ``ಇದರಲ್ಲಿ ಚುಕ್ಕಿ ಗುರುತಿದೆ, ಅದು ಕೊನೆಯಲ್ಲಿದೆ, ಹೀಗೆಲ್ಲಾ ಇರಬಾರದು'' ಎಂದೆಲ್ಲಾ ಅವನು ಹೇಳುವಾಗ ನಾವು ಅಪನಂಬಿಕೆಯಿಂದ ನೋಡಿದರೆ ಅದೆಲ್ಲಾ ನಮ್ಮ ಕಣ್ಣಿಗೆ ಕಾಣುವುದೇ ಇಲ್ಲ. ಮೊದಲು ಅವನು ಹೇಳುವುದನ್ನು ಕೇಳಬೇಕು. ಮನಗೆ ಬಂದು ನೋಡಿದರೆ ಆ ಅವಲಕ್ಷಣ ನಮಗೆ ಚೆನ್ನಾಗಿ ತೋರುತ್ತದೆ. ಮೊದಲು ಅವನು ಹೇಳುವುದನ್ನು ಸ್ವೀಕರಿಸದೆ ಇದ್ದರೆ, ``ಆ ವಜ್ರ `ಫಳ ಫಳ'' ಇದೆ, ಚೆನ್ನಾಗಿದೆ'' ಎಂದು ನಾವು ಹೇಳುತ್ತೇವೆ. ಹೀಗೆಯೇ ಆತ್ಮತತ್ತ್ವವನ್ನು ತಿಳಿಯಬೇಕೆಂದುಕೊಳ್ಳುವ ಶಿಷ್ಯನೂ ಕೂಡ ಶಾಸ್ತ್ರ ವಿಷಯದಲ್ಲಿ ಬಹಳ ಶ್ರದ್ಧೆಯುಳ್ಳವನಾಗಿರಬೇಕು. ಆತ್ಮತತ್ತ್ವವನ್ನು ತಿಳಿದುಕೊಂಡ ಮೇಲೆ ಒಬ್ಬನಿಗೆ ವೇದಾಂತ ಬೇಕೇ ಎಂದು ಕೇಳಿದರೆ,

\begin{shloka}
``ಪಲಾಲಮಿವ ಧಾನ್ಯಾರ್ಥೀ ತ್ಯಜೇತ್ ಗ್ರಂಥಮಶೇಷತಃ''
\end{shloka}

(ಧಾನ್ಯವನ್ನು ಅಪೇಕ್ಷಿಸುವವನು ಹೊಟ್ಟನ್ನು ಬಿಟ್ಟುಬಿಡುವಂತೆ ಒಂದು ಹಂತಕ್ಕೆ ಬಂದಮೇಲೆ ಗ್ರಂಥಗಳೆಲ್ಲವನ್ನು ಬಿಟ್ಟುಬಿಡಬಹುದು ಅಂದರೆ ಓದುವುದನ್ನು ಬಿಟ್ಟುಬಿಡಬಹುದು.)

ನಾವು ಅಕ್ಕಿಯನ್ನು ಮಾಡಲು ಭತ್ತವನ್ನು ತರುತ್ತೇವೆ. ಅಕ್ಕಿಯನ್ನು ಮಾಡಿದ ಮೇಲೆ ಅದರ ಮೇಲಿರುವ ಹೊಟ್ಟನ್ನು ನಾವು ದೂರ ಮಾಡುತ್ತೇವೆ. ಶಾಸ್ತ್ರ ವಿಷಯಗಳೂ ಜ್ಞಾನಕ್ಕಾಗಿಯೇ ನಮಗೆ ಬೇಕು. ಜ್ಞಾನ ಬಂದಮೇಲೆ ಶಾಸ್ತ್ರ ಹೇಗೆ ಹೇಳಿದರೂ ಅದರಲ್ಲಿ ಯಾವ ವಿಧವಾದ ಯೋಚನೆಯೂ ಇರಬೇಕಾಗಿಲ್ಲ. ಆದರೆ ನಾವು ಇನ್ನೂ ಅಕ್ಕಿಯನ್ನು ಮಾಡಿಯೇ ಇಲ್ಲ. ಅಕ್ಕಿ ಹೊಟ್ಟು ಎರಡೂ ಸೇರಿಕೊಂಡಿದ್ದರೆ ನಾವು ಊಟಮಾಡಲು ಆಗುವುದಿಲ್ಲ, ಈಗ ಮನುಷ್ಯರಿಗೆಲ್ಲಾ ಆತ್ಮ ಜ್ಞಾನ ಉಂಟಾಗದೆ ಇರುವುದರಿಂದ ಶಾಸ್ತ್ರಗಳನ್ನು ಓದಬೇಕು ಎನ್ನುವುದು ಬಹಳ ಅವಶ್ಯಕ.

ನಾವು ಯಾವಾಗಲೂ ಪ್ರಥಮ ಪಂಡಿತರಾಗಿರುವವರಿಗೆ ಮರ್ಯಾದೆ ಕೊಡಬೇಕು. ಅದನ್ನು ಬಿಟ್ಟು `ನಮ್ಮ ಬುದ್ಧಿಗೆ ತೋರುವ ಸಿದ್ಧಾಂತಕ್ಕೆ ಅನುಗುಣವಾಗಿರುವ ದೊಡ್ಡವರಿಗೆ ಮಾತ್ರ ಮರ್ಯಾದೆ ಕೊಡುತ್ತೇನೆಂದು' ಒಬ್ಬನು ಹೇಳಿದರೆ,

\begin{shloka}
`ವಾಟ್ಯೈ ಪ್ರದೀಯತಾಂ ವೀಟೀ\\
(ಅಜ್ಜಿಗೆ ತಾಂಬೂಲವನ್ನು ಕೊಡಲಿ)
\end{shloka}

ಅವನು ತನ್ನ ಅಜ್ಜಿಗೇನೆ ಮರ್ಯಾದೆ ಕೊಡಬಹುದು.

ಮುಖ್ಯವಾಗಿ ಮನುಷ್ಯರು ಮಾಡಬೇಕಾದುದೇನೆಂದರೆ, ಶಂಕರ ಭಗವತ್ಪಾದರು ತಮ್ಮ ವಿವೇಕ ಚೂಡಾಮಣಿ ಗ್ರಂಥದಲ್ಲಿ ಮೊದಲನೆಯ ಶ್ಲೋಕದಲ್ಲಿ ಹಲವಾರು ಅಂಶಗಳನ್ನು ಸೇರಿಸಿ ಹೇಳಿದರು.

\begin{shloka}
``ಜಂತೂನಾಂ ನರಜನ್ಮ ದುರ್ಲಭಮತಃ ಪುಂಸ್ತ್ವಂ ತತೋ ವಿಪ್ತತಾ\\
ತಸ್ಮಾ ದ್ವೈದಿಕಧರ್ಮಮಾರ್ಗಪರತಾ ವಿದ್ವತ್ತ್ವಮಸ್ಮಾತ್ಪರಮ್ |\\
ಆತ್ಮಾನಾತ್ಮವಿವೇಚನಂ ಸ್ವನುಭವೋ ಬ್ರಹ್ಮಾತ್ಮನಾ ಸಂಸ್ಥಿತಿಃ\\
ಮುಕ್ತಿರ್ನೋ ಶತಕೋಟಿಜನ್ಮಸುಕೃತ್ಯೆಃ ಪುಣೈರ್ವಿನಾ ಲಭ್ಯತೇ ||
\end{shloka}

(ಹುಟ್ಟಿದವರಿಗೆ ಮನುಷ್ಯಜನ್ಮ ಬಹಳ ಶ್ರಮದಿಂದ ದೊರೆಯುವಮ್ತಹುದು; ಅದರಲ್ಲೂ ಪುರುಷನಾಗುವುದು, ಬ್ರಾಹ್ಮಣನಾಗುವುದು. ಆದ್ದರಿಂದ ವೇದದಲ್ಲಿ ಹೇಳಿರುವ ಪ್ರವೃತ್ತಿ-ನಿವೃತ್ತಿ ಧರ್ಮಮಾರ್ಗದಲ್ಲಿ ನಡೆಯುವುದು, ವಿದ್ವಾಂಸನಾಗಿರುವುದು, ಅದಾದ ಮೇಲೆ ಆತ್ಮ-ಅನಾತ್ಮ-ವಿವೇಚನೆ, ಒಳ್ಳೆಯ ಸಾಕ್ಷಾತ್ಕಾರ ಅನುಭವ, ಬ್ರಹ್ಮ ಸ್ವರೂಪವಾಗಿ ಸುಸ್ಥಿರವಾಗುವ ಮೋಕ್ಷ ಸ್ಥಿತಿ ನೂರು ಕೋಟಿ ಜನ್ಮಗಳಲ್ಲಿ ಮಾಡಿದ ಪುಣ್ಯಗಳಿಂದಲ್ಲದೆ ದೊರೆಯುವುದಿಲ್ಲ.)

ಮೊದಲ ಪಂಕ್ತಿಯಲ್ಲಿ `ಜನ್ತೂನಾಂ ನರಜನ್ಮ ದುರ್ಲಭಂ' ಎನ್ನುತ್ತಾರೆ. ಈ ಮನುಷ್ಯಜನ್ಮ ಬಹಲ ದುರ್ಲಭವಾದುದು. ಕೆಲವರಿಗೆ ಮನುಷ್ಯ ಜನ್ಮಕ್ಕಿಂತಲೂ ನಾಯಿ ಜನ್ಮ ಶ್ರೇಷ್ಠವಾದುದೆನ್ನುವ ಭಾವನೆ ಇದೆ. ಏಕೆಂದರೆ ಒಬ್ಬ ದೊಡ್ಡವೈಸ್ರಾಯ್ ಮನೆಯಲ್ಲಿ ನಾಯಿಯಾಗಿ ಹುಟ್ಟಿದರೆ ಅದನ್ನು ಗಾಳಿಯಲ್ಲಿ ತಿರುಗಾಡಲು ಕರೆದುಕೊಂಡು ಹೋಗುವುದಕ್ಕೆ ಇಬ್ಬರು ಆಳುಗಳು ನಿಯಮಿಸಲ್ಪಡುತ್ತಾರೆ. ನಾವು ಕುಡಿಯುವುದು ಆಡಿನ ಹಾಲೋ, ಹಸುವಿನ ಹಾಲೋ ತಿಳಿಯದು, ಅದು ಬೆಳ್ಳಗೆ ಇದ್ದರೆ ಹಾಲೆನ್ನುವ ನಂಬಿಕೆಯಿಂದ ಕುಡಿದು ಬಿಡುತ್ತೇವೆ. ಆದರೆ ಆ ನಾಯಿಗೆ ಹಾಗಲ್ಲ. (ಹಸುವಿನ ಹಾಲೇ ದೊರೆಯುತ್ತದೆ!) ನಾವೆಲ್ಲರೂ ವೈಸ್ರಾಯ್ ಅವರನ್ನು ನೋಡಬೇಕಾದರೆ ಕಾದು ಕುಳಿತಿರಬೇಕು. ಆದರೆ ಆ ನಾಯಿಗೆ ಹಾಗಲ್ಲ. ಅದು ಆ ವೈಸ್ರಾಯ್ ಜೊತೆಯಲ್ಲಿಯೇ ಮಲಗುವುದು. ಆ ವೈಸ್ರಾಯ್ ನಾಯಿಯನ್ನು ಬಿಡುವುದಿಲ್ಲ. ಎಂಥ ಪುಣ್ಯ ಮಾಡಿದರಲ್ಲವೆ ಇಂಥ ಸ್ಥಿತಿಯನ್ನು ನಾಯಿ ಪಡೆಯಬಲ್ಲದು! ಆದ್ದರಿಂದ ಮನುಷ್ಯ ಜನ್ಮ ಶ್ರೇಷ್ಠವಾದುದೇ? ನಾಯಿಯ ಜನ್ಮ ಶ್ರೇಷ್ಠವಾದುದೇ? ನಾವೆಲ್ಲರೂ ಕಷ್ಟಪಡುತ್ತಿದ್ದೇವೆ. ಸಾಮಾನ್ಯವಾಗಿ ನೋಡಿದರೆ ಆ ನಾಯಿಯ ಜನ್ಮವೇ ಶ್ರೇಷ್ಠವೆಂದು ಕೆಲವರು ವಾದಿಸುವರು. ಸ್ವಲ್ಪ ಶೋಧನೆ ಮಾಡಿ ನೋಡಿದರೆ ನಮಗೆ ಹಲವು ವಿಷಯಗಳು ಸ್ಪಷ್ಟವಾಗುವುವು.

\begin{shloka}
`ನ ಹಿ ಜ್ಞಾನೇನ ಸದೃಶಂ ಪವಿತ್ರಮಿಹ ವಿದ್ಯತೇ'
\end{shloka}

(ಇಲ್ಲಿ ಜ್ಞಾನಕ್ಕೆ ಸಮನಾಗಿ ಪವಿತ್ರವಾದುದು ಯಾವುದೂ ಇಲ್ಲ.)

-ಎಂದು ಹೇಳಿದಂತೆ ಪವಿತ್ರವಾದ ಜ್ಞಾನವನ್ನು ಸಂಪಾದಿಸಬೇಕಾದರೆ ಅಂಥ ಯೋಗ್ಯತೆ ನಾಯಿಯ ಜನ್ಮಕ್ಕೆ ಇಲ್ಲ. ಯಾವುದೋ ಹಿಂದಿನ ಜನ್ಮದಲ್ಲಿ ಮಾಡಿದ ಪುಣ್ಯದಿಂದ ಆ ನಾಯಿ ಈಗ ಒಳ್ಳೆಯ ಜಾಗದಲ್ಲಿ ಹುಟ್ಟಿ ಸುಖವನ್ನು ಅನುಭವಿಸುತ್ತಿದೆ. ಇನ್ನೊಂದು ನಾಯಿ ಯಾವಾಗಲೂ ದುಃಖವನ್ನು ಅನುಭವಿಸುತ್ತಿದೆ. ಎರಡು ನಾಯಿಗಳೂ ಯಾವ ಯಾವ ಪದಾರ್ಥಗಳು ಸಿಕ್ಕುವುವೋ ಅವುಗಳನ್ನು ತಿನ್ನುತ್ತೇವೆ. ಸಿಕ್ಕಲಿಲ್ಲವೆಂದರೆ ತಿಪ್ಪೆಯಲ್ಲಿ ಬಂದು ಮಲಗಿಕೊಳ್ಳಬೇಕು. ಮನುಷ್ಯನ ವಿಷಯ ಹಾಗಲ್ಲ.

\begin{shloka}
`ಆತ್ಮಾವೈ ಶಕ್ಯತೇ ತ್ರಾತುಂ ಕರ್ಮಭಿಃ ಶುಭ ಲಕ್ಷಣೈಃ\\
(ಒಳ್ಳೆಯ ಕರ್ಮಗಳಿಂದ ಒಬ್ಬನು ತನ್ನನ್ನು ರಕ್ಷಿಸಿಕೊಳ್ಳಬಹುದು.)
\end{shloka}

-ಎಂದು ಹೇಳಿದಂತೆ ಯಾವುದು ಒಳ್ಳೆಯದು? ಯಾವುದು ಕೆಟ್ಟದು? ಎಂದು ನೋಡುವ ಶಕ್ತಿ ಮನುಷ್ಯನಿಗಿದೆ. ಅದನ್ನು ತಿಳಿದುಕೊಂಡ ಮೇಲೆ ಅಂಥ ಪರಿಸ್ಥಿಗೆ ತಕ್ಕಂತೆ ಮನುಷ್ಯನು ತನ್ನ ಬಾಳನ್ನು ನಡೆಸಬಹುದು. ನಾಯಿಗಳ ವಿಷಯದಲ್ಲಿ ಅಂಥ ಅರಿವು ಇಲ್ಲ. ತನಗೆ ಸ್ವಾಭಾವಿಕವಾಗಿ ಬಂದಿರುವ ಶಕ್ತಿಯಿಂದ ಅದು ಯಾವುದಾದರೂ ಕೆಲಸವನ್ನು ಮಾಡುತ್ತಿರುತ್ತದೆ. ಈಗ ಯಾವುದು ದೊರೆಯುತ್ತಿದೆಯೋ ಅದನ್ನು ಇಟ್ಟುಕೊಂಡು ದಿನಗಳನ್ನು ಕಳೆಯುತ್ತಿದೆ. ಮನುಷ್ಯನ ಬಾಳು ಹಾಗಲ್ಲ. ಆದ್ದರಿಂದಲೇ ಮನುಷ್ಯನ ಬಾಳು ದೊಡ್ಡದೆಂದು ನಾವು ಭಾವಿಸಬೇಕು. ಹೀಗೆ ಹೇಳಿ ಮನುಷ್ಯನ ಜನ್ಮ ದೊರತುದು ಶ್ರೇಷ್ಠವೆಂದುಕೊಂಡು, ಬೇರೆ ಏನೂ ಮಾಡಬೇಕಾಗಿಲ್ಲ ಎಂದುಕೊಂಡರೆ ಅದು ತಪ್ಪು ಭಾವನೆ. ನಮ್ಮ ಪೂರ್ವ ಪುಣ್ಯದ ಫಲದಿಂದ ಈ ಜನ್ಮವನ್ನು ಪಡೆದಿದ್ದೇವೆ. ಆದರೆ ಉತ್ತಮ ಫಲವನ್ನು ಪಡೆಯದೆ ಇದ್ದರೆ, ನಾವು ಮೃಗಗಳಿಗಿಂತಲೂ ಕೀಳಾಗಿಬಿಡುತ್ತೇವೆ. ಆಗ ಮೃಗಗಳೇ ಮನುಷ್ಯರಿಗಿಂತಲೂ ಮೇಲಾಗಿ ಬಿಡುವುದು. ಏಕೆಂದರೆ ಪ್ರಾಣಿಗಳ ವಿಷಯದಲ್ಲಿ,

\begin{shloka}
`ಪ್ರತ್ಯವಾಯ ಅಪ್ರವರ್ತನಾತ್'
\end{shloka}

(ಪಾಪದಲ್ಲಿ ತೊಡಗದೆ ಇರುವುದರಿಂದ)

-ಎನ್ನುವುದರಿಂದ ಒಂದು ನಾಯಿ ಯಾರನ್ನಾದರೂ ಕಡಿದರೆ ನಾವು ಅದಕ್ಕೆ ವಿರುದ್ಧವಾಗಿ ಕೇಸು ಹಾಕುವುದಿಲ್ಲ. ಆ ನಾಯಿಯನ್ನು ಹೊಡೆಯಲು ಸಾಧ್ಯವಾದರೆ ಎರಡು ಏಟು ಹೊಡೆಯುತ್ತವೆ, ಇಲ್ಲದಿದ್ದರೆ `ಹಾಳಾಗಿ ಹೋಗಲಿ' ಎಂದು ಬಿಟ್ಟು ಬಿಡುತ್ತೇವೆ. ಆದರೆ ಅದೇ ಕೆಲಸವನ್ನು ಮನುಷ್ಯನು ಮಾಡಿದರೆ ಅವನಿಗೆ ಬುದ್ಧಿ ಇದೆಯೇ ಎಂದು ಕೇಳುತ್ತೇವೆ. ಮನುಷ್ಯನು ಹೀಗೆ ಮಾಡಿದರೆ ಅವನನ್ನು ಕಂಡು ಹಿಡಿದು ಪೋಲೀಸಿಗೆ ಒಪ್ಪಿಸಿ ಅವನ ಮೇಲೆ ಕೇಸು ಹಾಕಿ ತೀರ್ಪು ಆಗುವವರೆಗೆ ನಾವು ಸುಮ್ಮನೆ ಇರುವುದಿಲ್ಲ. ನಾಯಿಗೆ ಇಂಥ ದಂಡನೆ ಕೊಡಲಿಲ್ಲವಲ್ಲಾ! ಅದಕ್ಕೆ ವಿರೋಧವಾಗಿ ಕೇಸು ಹಾಕಲಿಲ್ಲವಲ್ಲಾ! ಆದ್ದರಿಂದ ನಾಯಿಯ ಜನ್ಮ ಶ್ರೇಷ್ಠವೆಂದು ಯಾರಾದರೂ ಹೇಳಿದರೆ ಅವನಿಗೆ ವಿವೇಕವಿಲ್ಲವೆಂದು ಹೇಳುವೆವು. ಮನುಷ್ಯನಿಗೆ ವಿವೇಕವಿದೆ. ಅದಕ್ಕೆ ತಕ್ಕಂತೆ ಅವನು ನಡೆದುಕೊಳ್ಳದಿದ್ದರೆ ಅದಕ್ಕೆ ಅವನೇ ಜವಾಬ್ದಾರಿ. ಇಂಥ ಜವಾಬ್ದಾರಿಯೆಲ್ಲಾ ನಾಯಿಗೆ ಇಲ್ಲ. ಒಂದು ಕೆಲಸವನ್ನು ಮಾಡಲಿಲ್ಲವೆಂದರೆ ಅದಕ್ಕೆ ದಂಡನೆ ಇಲ್ಲ. ಮನುಷ್ಯನಿಗೆ ಒಂದು ಕೆಲಸವನ್ನು ಮಾಡಲು ಬರದಿದ್ದರೆ ಅವನಿಗೆ ನಾಲ್ಕು ಏಟು ಹಾಕಿಯಾದರೂ ``ಮಾಡಬೇಕಾದ ಕೆಲಸಗಳನ್ನು ಚೆನ್ನಾಗಿ ಮಾಡು' ಎಂದು ಬುದ್ಧಿವಾದ ಹೇಳುವೆವು. ಆದ್ದರಿಂದ ಮನುಷ್ಯ ಜನ್ಮಕ್ಕೆ ಜವಾಬ್ದಾರಿ ಉಂಟು.

\begin{shloka}
`ಕರ್ತುಂ ಅಕರ್ತುಂ ಅನ್ಯಥಾ ಕರ್ತುಂ ಶಕ್ಯತೇ'
\end{shloka}

[(ಮನುಷ್ಯನಿಗೆ ಒಂದು ಕೆಲಸ) ಮಾಡಲು ಮಾಡದೆ ಇರಲು ಬೇರೆ ವಿಧವಾಗಿ ಮಾಡಲು ಆಗುತ್ತದೆ.]

\begin{shloka}
`ಪ್ರಾಪ್ತಂ ಪ್ರಾಪ್ತಂಚ ಭೋಕ್ತುಂ ಶಕ್ನೋತಿ'
\end{shloka}

(ಪಡೆಯುವುದಕ್ಕೂ, ಪಡೆದುದನ್ನು ಅನುಭವಿಸುವುದಕ್ಕೂ ಅವನಿಗೆ ಆಗುತ್ತದೆ.) ಆದ್ದರಿಂದ `ಜನ್ತೂನಾಂ' ಎಂದು ವಿವೇಕ ಚೂಡಾಮಣಿಯಲ್ಲಿ ಹೇಳಿದಂತೆ ಮನುಷ್ಯ ಜನ್ಮ ಬಹಳ ದುರ್ಲಭವೆಂದು ಸ್ಪಷ್ಟವಾಗುತ್ತದೆ

\begin{shloka}
`ನಾನಾಯೋನಿ ಸಹಸ್ರಸಂಭವವಶಾತ್ ಜಾತಾ ಜನನ್ಯಃ ಕತಿ\\
ಪ್ರಖ್ಯಾತಾ ಜನಕಾಃ ಕಿಯನ್ತ ಇತಿ ಮೇ ಸೇತ್ಸ್ಯನ್ತಿ ಚಾಗ್ರೇ ಕತಿ |
\end{shloka}

(ಎಷ್ಟೋ ಸಾವಿರಾರು ಯೋನಿಗಳಲ್ಲಿ ಹುಟ್ಟಿದುದರಿಂದ ಎಷ್ಟೋ ಮಂದಿ ತಾಯಂದಿರು ಆದರು, ಎಷ್ಟೋ ಮಂದಿ ತಂದೆಯರು ಆದರು, ನಮಗೆ ಮುಂದೆ ಎಷ್ಟೋ ಮಂದಿ ಹೋಗುತ್ತಿದ್ದಾರೆ.)

ಹೀಗಿರುವಾಗ ನಾವು ಈಗಿನ ಜನ್ಮವನ್ನು ಮನುಷ್ಯನ ಜನ್ಮವಾಗಿ ಪಡೆದುಕೊಂಡಿದ್ದೇವೆ. ಹೀಗೆ ದೊರತ ಶರೀರ `ನಾನು' ಅಲ್ಲ; ಈ ಶರೀರ ಸ್ಥಿರವಲ್ಲ ಎನ್ನುವಂಥ ತೀರ್ಮಾನಕ್ಕೆ ನಾವು ಬರಬೇಕು.

ಅನಂತರ ಶ್ಲೋಕದಲ್ಲಿ `ಪುಂಸ್ತ್ವಂ' (ಗಂಡಸಾಗಿ ಹುಟ್ಟುವುದು)- ಎಂದು ಹೇಳಲ್ಪಟ್ಟಿದೆ, ಮನುಷ್ಯಜನ್ಮ ದೊರೆಯುವುದು ದುರ್ಲಭ. ಅದರಲ್ಲೂ ಗಂಡಸಾಗಿ ಹುಟ್ಟುವುದು ಇನ್ನೂ ವಿಶೇಷವೆಂದು ಹೇಳಲ್ಪಟ್ಟಿದೆ. ಹೀಗೆ ಹೇಳಿರುವುದರಿಂದ ಕೆಲವರಿಗೆ ಕೋಪ ಉಂಟಾಗಬಹುದು. ಅದಕ್ಕಾಗಿ ನಾವು ಶ್ಲೋಕವನ್ನು ಬದಲಾವಣೆ ಮಾಡಲಾರೆವು. ಅದರಲ್ಲಿರುವುದನ್ನು ಹಾಗೆಯೇ ಹೇಳಬೇಕು.

\begin{shloka}
ಸ್ತ್ರೀಪುಂಸಾವಾತ್ಮಭಾಗೌ ಭಿನ್ನಮೂರ್ತೇಃ ಸಿಸೃಕ್ಷಯಾ'\\
(ಸೃಜಿಸಲು ಅಪೇಕ್ಷಿಸಿದ ನಿನ್ನ ಭಿನ್ನಮೂರ್ತಿಯ ಭಾಗಗಳು\\
ಸ್ತ್ರೀ-ಪುರುಷರು.)
\end{shloka}

ಪ್ರಪಂಚವನ್ನು ಸೃಷ್ಟಿಸಬೇಕೆನ್ನುವ ಭಾವನೆ ಬರುತ್ತಲೇ ಪರಮಾತ್ಮನು ತನ್ನನ್ನೇ ಎರಡಾಗಿ ವಿಭಜಿಸಿದನು. `ಎರಡಾಗಿ ಇಟ್ಟುಕೊಂಡರೆ ಜಗಳ ಬರುವುದಲ್ಲಾ! ಆದ್ದರಿಂದ ಒಬ್ಬನಿಗೆ ಈ ಕೆಲಸ, ಇನ್ನೊಬ್ಬನಿಗೆ ಇನ್ನೊಂದು ಕೆಲಸವೆಂದು ಇಟ್ಟುಕೊಂಡರೆ ಜಗಳ ಬರುವುದಿಲ್ಲ' ಎಂದುಕೊಂಡನು. ಆದ್ದರಿಂದ `ಮನೆಯನ್ನು ಚೆನ್ನಾಗಿಟ್ಟುಕೊಂಡು ಕುಟುಂಬವನ್ನು ಸರಿಯಾಗಿ ಇಟ್ಟುಕೊಳ್ಳುವುದು ಹೆಂಗಸರ ಕೆಲಸ' -ಎಂದು ತೀರ್ಮಾನಿಸಿದನು. ಅದೇ ರೀತಿ `ಹೊರಕ್ಕೆ ಹೋಗಿ ಕೆಲಸಗಳನ್ನು ಮಾಡಿ ಸಂಪಾದಿಸುವುದು ಗಂಡಸಿನ ಕರ್ತವ್ಯ'ವೆಂದು ತೀರ್ಮಾನಿಸಿದನು. ಹೀಗೆ ಅಲ್ಲದೆ ಹೆಂಗಸು ಮನೆಯನ್ನು ಸರಿಯಾಗಿ ನೋಡಿಕೊಳ್ಳದೆ ಇದ್ದರೆ ಹೆಂಗಸನ್ನು ನಾವು ಆಕ್ಷೇಪಿಸುತ್ತೇವೆ. ಹಾಗೆಯೇ ಗಂಡಸಾದವನು ಹೊರಗೆ ಕೆಲಸ ಮಾಡದೆ ಇದ್ದರೆ ಹೆಂಗಸು, `ಇವನನ್ನು ಮದುವೆಯಾಗಿ ಹೇಗೆ ಬಾಳು ಸಾಗಿಸುವುದು' ಎಂದು ಕಳವಳ ಪಡುವಳು. ಇಬ್ಬರೂ ವ್ಯತ್ಯಾಸವಾಗಿ ಬಿಟ್ಟರೆ ಬೇರೆ ರಗಳೆಯೇ ಇಲ್ಲ. ಗಂಡಸು ಹೆಂಗಸು ಇಬ್ಬರೂ ಮನೆಯಲ್ಲಿ ಕುಳಿತು ಬಿಟ್ಟರೂ ಅಥವಾ ಇಬ್ಬರೂ ಆಫೀಸಿಗೆ ಹೋಗಿ, ಹೋಟಲಿನಲ್ಲಿ ಊಟ ಮಾಡುತ್ತಿದ್ದರೂ ಬಾಳಿನಲ್ಲಿ ಏನು ಮಾಧುರ್ಯವೆನ್ನುವುದನ್ನು ನಾವು ಮನಸ್ಸನ್ನು ಬಿಚ್ಚಿ ಯೋಚಿಸಿದರೆ, ಅಂಥ ಬಾಳಿನಲ್ಲಿ ಮಾಧುರ್ಯವೇ ಇಲ್ಲವೆನ್ನುವುದು ತೀರ್ಮಾನವಾಗಿ ತಿಳಿಯುವುದು.

ಗಂಡಸಾದವನು ಪ್ರೀತಿ ಇಲ್ಲದೆಯೇ ಬಾಳನ್ನು ನಡೆಸಿಬಿಡಬಹುದು. ಒಬ್ಬ ಹೆಂಗಸು ಹಾಗೆ ಇರಲಾಗುವುದಿಲ್ಲ. ಏಕೆಂದರೆ ಭಗವಂತನು ಹೆಂಗಸಿಗೆ ಒಂದು ವಿಶೇಷವಾದ ಶಕ್ತಿಯನ್ನು ಕೊಟ್ಟಿದ್ದಾನೆ. ಹೆಂಗಸು ತನ್ನಂತೆ ಇನ್ನೊಂದು ಶರೀರವನ್ನು ಸೃಷ್ಟಿ ಮಾಡಲು ಭಗವಂತನಿಂದ ಶಕ್ತಿಯನ್ನು ಪಡೆದಿದ್ದಾಳೆ. ಮಗು ಹುಟ್ಟಿತೆಂದರೆ ತಾಯಿಗೆ ಎಲ್ಲಿಂದ ಅಭಿಮಾನ ಬಂದು ಬಿಡುವುದೋ ತಿಳಿಯದು! ತಾಯಿ ತನ್ನನ್ನು ಮದುವೆಯಾದ ಗಂಡನಿಗಿಂತಲೂ ಹೆಚ್ಚಾಗಿ ಪ್ರೀತಿಸಲು ತಯಾರಾಗಿರುತ್ತಾಳೆ. ಆದರೆ ಮಗುವಿಗಿಂತಲೂ ಹೆಚ್ಚಾಗಿ (ಬೇರೆ ಯಾರನ್ನೂ) ಪ್ರೀತಿಸಲು ತಯಾರಾಗಿರುವುದಿಲ್ಲವೆನ್ನುವುದನ್ನು ಕಾಣುತ್ತೇವೆ. ಭಗವಂತನು ಹೇಗೆ ಗಂಡು, ಹೆಣ್ಣು ರೂಪವಾಗಿಬಿಟ್ಟನೆಂದು ಹೇಳುವುವೋ ಹಾಗೆಯೇ ಹೆಂಗಸು ಕೂಡ ಒಂದು ಮಗುವನ್ನು ಹೆತ್ತು ತಾಯಿಯಾದರೆ ಆ ಮಗುವಿನ ಮೇಲೆ ಆ ಹೆಂಗಸಿಗೆ ಬಹಲ ಪ್ರೇಮ ಉಂಟಾಗಿ ಬಿಡುವುದು. ಹಾಗೆ ಪ್ರೇಮ ಉಂಟಾಗುತ್ತಲೇ ತನಗೆ ಆಹಾರವಿಲ್ಲದಿದ್ದರೂ ಮಗು ತಿನ್ನುವುದಾದರೆ ಆ ತಾಯಿ ಎಲ್ಲಿಂದಲಾದರೂ ಭಿಕ್ಷೆಯನ್ನಾದರೂ ಕೇಳಿತಂದು ಆ ಮಗುವಿಗೆ ಆಹಾರ ಕೊಡುವಳು.

\begin{shloka}
`ಆತ್ಮಾ ವೈ ಪುತ್ರ ನಾಮಾಸಿ'\\
(ತಾನೇ ಮಗನಾಗಿದ್ದಾನೆ)
\end{shloka}

ಆದ್ದರಿಂದ ಹೆಂಗಸಿಗೆ ತನ್ನ ಮಗುವಿನ ಮೇಲೆ ಬಹಳ ಪ್ರೇಮವಿರುತ್ತದೆ. ಹೀಗೆ ಪ್ರೇಮವಿದ್ದರೆ ಹೆಂಗಸಾದವಳಿಗೆ ಆತ್ಮಚಿಂತನೆ ಮಾಡಲು ಅಷ್ಟಾಗಿ ಅವಕಾಶವೇ ಇರುವುದಿಲ್ಲ. ಆತ್ಮಚಿಂತನೆ ಮಾಡುವವನು ಬೇರೆ ಚಿಂತನೆ ಮಾಡುವುದೇ ಇಲ್ಲ. ನಾವೆಲ್ಲರೂ ಆತ್ಮಚಿಂತನೆ ಮಾಡುವುದಕ್ಕೋಸ್ಕರವೇ ಹುಟ್ಟಿದ್ದೇವೆ. ಆದರೂ ಹಣೆಬರಹ, ಸ್ವಲ್ಪಮಟ್ಟಿಗೆ ಆತ್ಮಚಿಂತನೆ ಮಾಡಿ ಉಳಿದ ಸಮಯವನ್ನು ಈ ಪ್ರಾಪಂಚಿಕ ವಿಷಯದಲ್ಲಿ ಕಳೆಯುತ್ತೇವೆ. ಹೆಂಗಸರಿಗೆ ಹಲವು ವಿಷಯಗಳ ಬಗ್ಗೆ ಚಿಂತಿಸಬೇಕಾದ ಸ್ಥಿತಿಯೂ, ಸ್ವಾಭಾವಿಕವಾಗಿಯೇ ಅವರಲ್ಲಿ ಪ್ರೇಮವೂ ಇರುವುದರಿಂದ ಆತ್ಮಚಿಂತನೆಯಂತಹ ವಿಷಯಗಳಲ್ಲಿ ಗಮನವನ್ನು ಹರಿಸುವುದಕ್ಕೆ, ಗಂಡಸರಿಗಿಂತ ಬಹಳ ಕಡಿಮೆ ಅವಕಾಶವು ಇದೆ. ಮೈತ್ರೇಯೀ ಎನ್ನುವವಳು ಆತ್ಮಚಿಂತನೆ ಯಾವಾಗಲೂ ಮಾಡುತ್ತಿದ್ದಳೆಂದು ಉಪನಿಷತ್ತಿನಲ್ಲಿ ಓದುತ್ತೇವೆ. ಎಲ್ಲಾ ಹೆಂಗಸರನ್ನು ನಾವು ಆ ಮೈತ್ರೇಯಿಯ ಗೋಷ್ಠಿಯಲ್ಲಿ ಸೇರಿಸಲಾಗುವುದಿಲ್ಲ. ಅದೇ ರೀತಿ, ಗಂಡಸರೆಲ್ಲರೂ ``ಆತ್ಮ ವಿಷಯದಲ್ಲಿ ಮನಸ್ಸು ಇತ್ಟಿರುತ್ತಾರೆ''-ಎಂದು ಹೇಳಲು ಸಾಧ್ಯವಿಲ್ಲದಿದ್ದರೂ, ಒಬ್ಬ ಹೆಂಗಸಿಗೆ, ಒಬ್ಬ ಗಂಡಸಿಗೆ ಸಮ ಪರಿಮಾಣದಲ್ಲಿ ದುಃಖವನ್ನೂ ಸುಖವನ್ನೂ ಕೊಟ್ಟರೆ ಇವರಿಬ್ಬರಲ್ಲಿ ಗಂಡಸು ಇವುಗಳಲೆಲ್ಲಾ ಮನಸ್ಸನ್ನು ಇಡದೆ, ಒಬ್ಬನೆ ಇದ್ದು, ಯಾವಾಗಲೂ ಆತ್ಮಚಿಂತನೆ ಮಾಡಿ, ಬಾಳನ್ನು ನಡೆಸಲು ಬಹಳ ಅವಕಾಶವಿರುವುದರಿಂದ ಒಬ್ಬ ಗಂಡಸಿನ ಜನ್ಮ ಉತ್ತಮವಾದುದೆಂದು ಶಂಕರರು ಹೇಳಿದರು. ಆದ್ದರಿಂದಲ್ಲೆ ಅವರು ಶ್ಲೋಕದಲ್ಲಿ ``ಪುಂಸ್ತ್ವಂ'' ಎಂದು ಹೇಳಿದರು. ಅದೇನೆಂದರೆ ಬ್ರಾಹ್ಮಣನಾಗಿ ಹುಟ್ಟಿದರೆ ಇನ್ನೂ ವಿಶೇಷವೆಂದು ಹೇಳಲ್ಪಟ್ಟಿದೆ. ಒಬ್ಬನು ಬ್ರಾಹ್ಮಣ್ಯದಿಂದ ಇರುವವರೆಗೆ ಮಾತ್ರ ಬ್ರಾಹ್ಮಣ್ಯವೆನ್ನುವುದು ಉತ್ತಮವೆಂದು ಹೇಳಲ್ಪಟ್ಟಿದೆ. ಬ್ರಾಹ್ಮಣ್ಯವನ್ನೇ ಬಿಟ್ಟವನಿಗೆ ``ಅವನು ಬ್ರಾಹ್ಮಣನಾದರೂ ಅವನು ಉತ್ತಮವಾದವನಲ್ಲ''ವೆಂದೇ ಹೇಳಬಹುದು. ಆದ್ದರಿಂದಲೇ ಉತ್ತಮವಾದ ಪರಂಪರೆ ಎನ್ನುವುದು ಬಹಳ ಮುಖ್ಯ. ಭಗವಂತನು ``ಗುಣ ಕರ್ಮವಿಭಾಗಶಃ''-ಎನ್ನುವ ರೀತಿಯಲ್ಲಿ ವಿಭಾಗ ಮಾಡಿದನು.

ಕುದುರೆಗಳನ್ನು ಬೆಲೆಗೆ ತೆಗೆದುಕೊಳ್ಳಬೇಕೆಂದುಕೊಳ್ಳುವವನು ಆ ಕುದುರೆಯನ್ನು ನೋಡುವುದಕ್ಕಿಂತಲೂ ಹೆಚ್ಚಾಗಿ ಆ ಕುದುರೆಯ ಪರಂಪರೆಯನ್ನು ವಿಚಾರಿಸುತ್ತಾನೆ. ಈ ಕುದುರೆ ತಾಯಿ, ತಂದೆ ಯಾರು? ಎಂದೆಲ್ಲವನ್ನೂ ವಿಚಾರಿಸಿ ಕುದುರೆಗೆ ಕೊಡಬೇಕಾದ ಬೆಲೆಯನ್ನು ಕುದುರೆಯ ಮಾಲೀಕನಿಗೆ ಕೊಟ್ಟು ಬಿಡುತ್ತಾನೆ. ಆದರೂ ಒಬ್ಬನು ತನ್ನ ತಾತನಂತೆಯೇ ಇರಬೇಕೆಂದಿಲ್ಲ. ತಂದೆಯಂತೆ ಮಗ ಈಗ ಇರುವುದೇ ಇಲ್ಲ. ಏಕೆಂದರೆ ಕಾಲ ಬದಲಾವಣೆ ಆಗುತ್ತಲೇ ಇದೆ. ಅಷ್ಟೇ ಅಲ್ಲದೆ ಮನುಷ್ಯನಿಗೆ ಸ್ವಾತಂತ್ರ್ಯ ಎನ್ನುವುದಿದೆ. ಮೃಗಗಳಿಗೆ ಅಂಥ ಸ್ವಾತಂತ್ರ್ಯವಿಲ್ಲ. ಪ್ರಾಣಿಗಳು ತಮ್ಮನ್ನು ತಾವು ಗಮನಿಸದೆ ಇದ್ದರೂ ಅವುಗಳನ್ನು ಗಮನಿಸಲು ಬೇರೆಯವರಿದ್ದಾರೆ. ಮನುಷ್ಯನಿಗೆ ಹಾಗಲ್ಲ. ಮನುಷ್ಯನು ತನ್ನನು ತಾನು ರಕ್ಷಿಸಿಕೊಳ್ಳಬೇಕು. ಆದ್ದರಿಂದ ನಾವು ನಮಗೆ ವಿಧಿಸಲ್ಪಟ್ಟ ಅನುಷ್ಠಾನಗಳನ್ನು ಮಾಡುವುದೂ, ನಮ್ಮ ಪರಂಪರೆಯನ್ನು ಕಾಪಾಡಿಕೊಂಡು ಬರುವುದೂ ಬಹಳ ಅವಶ್ಯಕ. ಮನುಷ್ಯನಿಗೆ ಭಗವಂತನು ಸ್ವಾತಂತ್ರ್ಯವನ್ನು ಕೊಟ್ಟಿರುವುದರಿಂದ, ಅದನ್ನು ಇಟ್ಟುಕೊಂಡು ನಾವು ನಮ್ಮನ್ನು ಕಾಪಾಡಿಕೊಳ್ಳುವ ಮಾರ್ಗವನ್ನು ಹುಡುಕಬೇಕು. ಭಗವಂತನೇ ಎದುರಿಗೆ ಬಂದು ``ನಾನು ನಿಮ್ಮನ್ನು ಕಾಪಾಡುತ್ತೇನೆ'' ಎಂದು ಹೇಳಿದರೂ ಯಾರೂ ಕೂಡ ಅಂಥ ಭಗವಂತನನ್ನು ನಂಬುವುದಿಲ್ಲ! ಆದ್ದರಿಂದಲೇ ಭಗವಂತನು ನಮಗೆಲ್ಲಾ ಬುದ್ಧಿ ಎನ್ನುವುದನ್ನು ಕೊಟ್ಟು ನಮ್ಮನ್ನು ನಾವು ಕಾಪಾಡಿಕೊಳ್ಳಲು ದಾರಿ ತೋರಿಸಿದ್ದಾನೆ-

\begin{shloka}
``ನ ದೇವಾ ದಂಡಮಾದಾಯ ರಕ್ಷನ್ತಿ ಪಶುಪಾಲವತ್ |\\
ಯಂ ಹಿ ರಕ್ಷಿತುವಿಚ್ಛನ್ತಿ ಬುದ್ಧ್ಯಾ ಸಂಯೋಜಯನ್ತಿ ತಮ್ ||''
\end{shloka}

(ದೇವತೆಗಳು ಪ್ರಾಣಿಗಳನ್ನು ಕಾಪಾಡುವವನಂತೆ ಕೋಲನ್ನು ಹಿಡಿದು ಕಾಪಾಡುವುದಿಲ್ಲ: ಕಾಪಾಡಬೇಕಾದವನಿಗೆ ಒಳ್ಳೆಯ ಬುದ್ಧಿಯನ್ನು ಕೊಡುತ್ತಾರೆ.)


ಒಬ್ಬ ಅರ್ಚಕರ ಮಗ ಚೆನ್ನಾಗಿ ವೇದಾಧ್ಯಯನ ಮಾಡಿದ್ದಾನೆ. ಆದರೆ `ಆತನು ಸಂಧ್ಯಾವಂದನೆ ಮಾಡುತ್ತಾನೆಯೇ' -ಎಂದು ಕೇಳಿದರೆ `ಹಾಗೆ ಮಾಡಲಿಲ್ಲ' -ವೆಂದು ನಮಗೆ ಜವಾಬು ದೊರೆತರೆ ಅದು, ಯಾವ ವಿಧವಾಗಿಯೂ ಒಳ್ಳೆಯದೇ ಅಲ್ಲ. ಅವನ ಭಾವನೆಯಲ್ಲಿ ಸಂಧ್ಯಾವಂದನೆ ಮಾಡಿದರೆ ಯಾರೂ ಹಣಕೊಡುವುದಿಲ್ಲ. ಆದರೆ ಅದೇ ಸಮಯದಲ್ಲಿ ಇತರರಿಗೆ ವೇದವನ್ನು ಹೇಳಿಕೊಟ್ಟರೆ ಅಥವಾ ಯಾವುದಾದರೂ ಪಾರಾಯಣ ಮಾಡಿದರೆ ಅದಕ್ಕೆ ಹಣ ದೊರೆಯುವುದು. ಹಾಗಿರುವಾಗ ಸಂಧ್ಯಾವಂದನೆ ಏಕೆ ಮಾಡಬೇಕು?

\begin{shloka}
`ಪ್ರಯೋಜನಮನುದ್ದಿಶ್ಯ ನ ಮಂದೋಽಪಿ ಪ್ರವರ್ತತೇ |'
\end{shloka}

(ಪ್ರಯೋಜನವಿಲ್ಲದೆ ಮೂರ್ಖನು ಕೂಡ ಯಾವ ಕೆಲಸವನ್ನೂ ಮಾಡುವುದಿಲ್ಲ.)

-ಎನ್ನುವಂತೆ ಪ್ರಯೋಜನವನ್ನು ಎದುರು ನೋಡಿಯೇ ಒಬ್ಬ ಮುಟ್ಠಾಳನು ಕೂಡ ಎಂದೂ ಕೆಲಸದಲ್ಲೂ ತೊಡಗುವುದಿಲ್ಲ. ಹಾಗಿರುವಾಗ ಬುದ್ಧಿಶಾಲಿಯಾದ ನಾನು ಯಾವ ವಿಧವಾದ ಪ್ರಯೋಜನವೂ ಇಲ್ಲದೆ ಸಂಧ್ಯಾವಂದನೆ ಏಕೆ ಮಾಡಬೇಕೆಂದು ಯೋಚಿಸಿದರೆ, ಅಂಥವನ ಬ್ರಾಹ್ಮಣ ಬಹಳ ಕೆಳಮಟ್ಟದ್ದು ಆಗುತ್ತದೆ, ಅದು ಮೇಲ್ಪಟ್ಟಿದ್ದು ಅಲ್ಲವೇ ಅಲ್ಲ. ಆದ್ದರಿಂದಲೇ ಶಂಕರರು,

\begin{shloka}
``ವೈದಿಕಧರ್ಮಮಾರ್ಗಪರತಾ''
\end{shloka}

-ಎಂದಿದ್ದಾರೆ. ಮನುಷ್ಯ ಜನ್ಮದಂತಹವುಗಳನ್ನೆಲ್ಲಾ ಪಡೆದ ಮೇಲೂ ಒಬ್ಬನು ವೈದಿಕ ಮಾರ್ಗದಲ್ಲಿ ಅಂದರೆ ಧರ್ಮಮಾರ್ಗದಲ್ಲಿ ನಡೆದುಕೊಂಡು ಬರಬೇಕೆಂದು ಹೇಳಲಾಗಿದೆ. ಯೋಗ್ಯತೆ ಇರುವವರು ವೇದವನ್ನು ಅಧ್ಯಯನ ಮಾಡಿ ಸಂಧ್ಯಾವಂದನೆ, ಜಪ, ಹೋಮ ಇವುಗಳನ್ನು ಸರಿಯಾಗಿ ಮಾಡುತ್ತಾ ಬಾಳನ್ನು ನಡೆಸಿದರೆ ಅದು ವೇದದಲ್ಲಿ ಹೇಳಿದ ಧರ್ಮಮಾರ್ಗವಾಗುವುದು.

ದಾರಿಯಲ್ಲಿ ಯಾವುದೋ ಜಗಳವಾಯಿತು. ಪೋಲಿಸ್ ಅಧಿಕಾರಿ ಒಬ್ಬರು, ಜಗಳವಾಡುವ ಒಬ್ಬ ತಪ್ಪಿತಸ್ಥನಿಗೆ ಗುಂಡಿಕ್ಕಲು ತನ್ನೊಡನೆ ಬಂದಿದ್ದ ಪೋಲಿಸ್ ಪೇದೆಗಳಿಗೆ ಆಜ್ಞಾಪಿಸಿದರು. ಅವರು ಕೂಡ ಅಗತ್ಯವಾದರೆ ಆ ದಂಡನೆ ಕೊಡಲು ಸಿದ್ಧರಾಗಿದ್ದರು. ಒಂದು ವೇಳೆ ಅಂಥ ಘಟನೆ ನಡೆದಿದ್ದರೆ ಅವರು ಗುಂಡಿಕ್ಕಿ ಅದರಿಂದ ಯಾವ ವಿಧವಾದ ಬಾಧಕವೂ ಉಂಟಾಗದಂತೆ ಮನೆಗೆ ಬಂದು ಮಲಗುತ್ತಿದ್ದರು. ಹೀಗೆ ಗುಂಡಿಕ್ಕುವುದಕ್ಕೆ ಆಜ್ಞೆ ಅವರಿಗೆ ಸರಕಾರ ಕೊಟ್ಟಿದೆ. ಆ ಯೋಗ್ಯತೆಯೇ ಇಲ್ಲದಿರುವವನೊಬ್ಬನು ದಾರಿಯಲ್ಲಿ ನಿಂತು ಪೋಲಿಸರಿಗೆ ಆಜ್ಞೆ ಮಾಡಿದರೆ ಪೋಲಿಸರು ಅವನನ್ನು ಕೋರ್ಟಿಗೆ, ಜೈಲಿಗೆ ಕರೆದುಕೊಂಡು ಹೋಗುವರು. ಏಕೆಂದರೆ `ಯೋಗ್ಯತೆ ಇಲ್ಲ' -ಎನ್ನುವುದೇ ಕಾರಣ. ಅವನು ಯೋಗ್ಯತೆಯನ್ನು ಪಡೆದರೆ ಅವನೂ ಇತರ ಪೇದೆಗಳಿಗೆ ಆಜ್ಞಾಪಿಸಬಹುದು. ಈಗ ಈ ಜನ್ಮದಲ್ಲಿ ಯೋಗ್ಯತೆ ದೊರೆತಿದ್ದರೆ ``ವೇದದಲ್ಲಿ ಹೇಳಿರುವ ಕರ್ಮಗಳನ್ನು ವಿಧಿಯಾಗಿ ಮಾಡಬೇಕು'' ಎನ್ನುವುದನ್ನು ಶಂಕರರು ಸ್ಪಷ್ಟಪಡಿಸಿದ್ದಾರೆ. ಇಂಥ ಅಧಿಕಾರವನ್ನು ಪಡೆದೂ ಏನೂ ಮಾಡದೆ ಇದ್ದರೆ, ಬಾಳನ್ನು ನಿದ್ದೆಯಲ್ಲಿ ಕಳೆದರೆ ಅದು ಬಹಳ ಮೂರ್ಖತನವಾಗುತ್ತದೆ. ಏಕೆಂದರೆ

\begin{shloka}
``ಬ್ರಾಹ್ಮಣಸ್ಯ ಹಿ ದೇಹೋಽಯಂ ನೋಪಭೋಗಾಯ ಕಲ್ಪತೇ |\\
ಇಹ ಕ್ಲೇಶಾಯ ಮಹತೇ ಪ್ರೇತ್ಯ ಅನಂತ ಸುಖಾಯ ಚ ||''
\end{shloka}

(ಬ್ರಾಹ್ಮಣನ ಈ ದೇಹ ಭೋಗಕ್ಕಾಗಿ ಏರ್ಪಟ್ಟಿಲ್ಲ. ಹೆಚ್ಚು ಕಷ್ಟಗಳನ್ನು ಇಲ್ಲಿ ಅನುಭವಿಸಿ ಅನಂತರ ಅನಂತ ಸುಖವನ್ನು ಪಡೆಯುವುದಕ್ಕೆ ಇದು ಇದೆ.)

ಎಂದು ಒಂದು ಕಡೆ ತಿಳಿಸಲ್ಪಟ್ಟಿದೆ. ಒಂದು ದಿನ ಗ್ರಹಣ ಬಂದಿತು. ರಾತ್ರಿ ಮೂರುಘಂಟೆಗೆ ಚಂದ್ರಗ್ರಹಣ ಅದು. ಮಾಘಮಾಸ ಚಳಿಯಲ್ಲಿ ಹೊದ್ದು ಕೊಂಡಿರುವ ಹೊದಿಕೆಯನ್ನು ತೆಗೆಯಲು ಮನಸ್ಸೇ ಇಲ್ಲ. ಆದರೆ ಶಾಸ್ತ್ರವೋ ``ನೀನು ನದಿಗೆ ಹೋಗಿ ಸ್ನಾನಮಾಡಿ ಗ್ರಹಣ ನಾಲ್ಕು ಘಂಟೆಯವರೆಗೆ ಇರುವುದರಿಂದ ನದೀತೀರದಲ್ಲಿ ಕುಳಿತು ಜಪ ಮಾಡು'' ಎಂದೆಲ್ಲಾ ಹೇಳುತ್ತದೆ. ಆದರೆ ಈ ಶಾಸ್ತ್ರದಲ್ಲಿ ವಿರಕ್ತಿ ಉಂಟಾಗಿ ``ಚಳಿಗಾಲದಲ್ಲಿ ಯಾರು ಹೋಗಿ ಜಪ ಮಾಡುವುದು'' -ಎಂದು ಯೋಗ್ಯತೆ ಇರುವವನೊಬ್ಬನು ಹೇಳಿದರೆ ಅದರಿಂದ ಅವನಿಗೆ ಪಾಪ ಉಂಟಾಗುತ್ತದೆಯೇ ಹೊರತು ಬೇರೆ ಯಾವ ರೀತಿಯಲ್ಲೂ ಅವನಿಗೆ ಒಳ್ಳೆಯದಾಗುವುದಿಲ್ಲ.

ಧರ್ಮವನ್ನು ಮಾಡುವುದರಿಂದ ಈಗ ಸ್ವಲ್ಪ ಮಾತ್ರ ದುಃಖ ಉಂಟಾದರೂ ಕೂಡ ಅನಂತರ ಸುಖವೇ ಆಗುವುದು. ಆದ್ದರಿಂದ ಯಾವಾಗಲೂ ವೈದಿಕ ಧರ್ಮ ಮಾರ್ಗದಲ್ಲಿ ನಡೆಯಬೇಕೆಂದು ಹೇಳಲಾಗಿದೆ.

ಸಂಧ್ಯಾವಂದನೆ ಮಾಡುವಾಗ, ``ಆಪೋಹಿಷ್ಠಾ ಮಯೋ ಭುವಃ ತಾ ನ ಊರ್ಜೇ ದಧಾತನ'' (ಜಲ ದೇವತೆಗಳೇ! ಯಾವ ರೀತಿಯಲ್ಲಿ ಸ್ನಾನ, ಪಾನ ಇವುಗಳಿಗೆ ಕಾರಣವಾಗಿದ್ದು ಸುಖವನ್ನು ಕೊಡುತ್ತೀರೋ, ಅಂಥ ನೀವು ನಿಮ್ಮ ಭಕ್ತರಾದ ನಮಗೆ ಅನ್ನವನ್ನು ಕೊಡಬೇಕು.) -ಎನ್ನುವ ಇದರ ಅರ್ಥ ಎಷ್ಟೋಮಂದಿಗೆ ಗೊತ್ತಿಲ್ಲ. ಕೆಲವರಿಗೆ ಸರಿಯಾಗಿ ಉಚ್ಚರಿಸುವುದಕ್ಕೆ ಬರುವುದಿಲ್ಲ. ಏಕೆಂದರೆ ಅವರಿಗೆ ಹೇಳಿಕೊಟ್ಟಂತೆಯೇ ಅವರು ಹೇಳುವರು. ಶಂಕರ ಭಗವತ್ಪಾದರು ಇದರ ಬಗ್ಗೆ-

\begin{shloka}
`ವಿದ್ವತ್ವಂ ಅಸ್ಮಾತ್‌ಪರಂ'
\end{shloka}

-ಎಂದಿದ್ದಾರೆ. `ನಾನು ಯಾವ ಮಂತ್ರವನ್ನು ಹೇಳುತ್ತಿದ್ದೇನೆ? ಇದಕ್ಕೆ ಅರ್ಥವೇನು? ಅರ್ಥ ತಿಳಿಯದೆ ನಾನು ಏನೇನೋ ಹೇಳುತ್ತಿದ್ದೇನೆಂದುಕೊಂಡು ಒಬ್ಬನು ಹೇಳುವ ಮಂತ್ರದ ಅರ್ಥವನ್ನು ಸರಿಯಾಗಿ ತಿಳಿಯಲು ಪ್ರಯತ್ನಪಡಬೇಕು.

ಮದುವೆಯಲ್ಲಿ ಮದುವೆ ಮಾಡಿಸುವ ಪುರೋಹಿತರು ವಧೂ-ವರರನ್ನು ನೋಡಿ-

\begin{shloka}
``ಧರ್ಮೇ ಚಾರ್ಥೇ ಚ ಕಾಮೇ ಚ ನಾತಿಚರಿತವ್ಯಂ ತ್ವಯಾ''
\end{shloka}

(ಧರ್ಮ, ಅರ್ಥ, ಕಾಮ ಇವುಗಳಲ್ಲಿ ಹೆಚ್ಚಾಗಿ ಸಂಬಂಧ ಪಡಕೂಡದು.)

``ಈ ಮಂತ್ರವನ್ನು ಹೇಳಿ''-ಎಂದು ಹೇಳುವರು. ವರನು ಮಂತ್ರವನ್ನು ಹೇಳದೆ ಮಾತನಾಡದೆ ಕುಳಿತಿರುವನು. ಪುರೋಹಿತರು ಹೇಳುತ್ತಾ ಹೋಗುವರು. ವರನು ಹೇಳದೆ ಇದ್ದದರಿಂದ ಅವರು ತಮಗೆ ತಾವೇ ಮದುವೆ ಮಾಡಿಕೊಂಡುತಾಗುವುದು. ಮಂತ್ರವನ್ನು ಹೇಳದೆ ಇರುವುದರಿಂದ ವಧೂ-ವರರಿಬ್ಬರಿಗೂ ಯಾವ ವಿಧವಾದ ಪ್ರತಿಜ್ಞೆಯೂ ಇಲ್ಲ. ಆದ್ದರಿಂದ ಮಂತ್ರಗಳನ್ನು ಅತಿಕ್ರಮಿಸಿ ನಡೆಯುವುದರಿಂದ ಯಾವ ವಿಧವಾದ ದೋಷವೂ ಇಲ್ಲ! ಏಕೆಂದರೆ ಅವರಿಗೆ ಮಂತ್ರಗಳ ಅರ್ಥ ಗೊತ್ತಿದ್ದರೆ ತಾನೇ!

\begin{shloka}
`ಅಗ್ನಿಮುಖಾ ವೈ ದೇವಾಃ'
\end{shloka}

(ದೇವತೆಗಳಿಗೆ ಅಗ್ನಿ ಬಾಯಿ ಇದ್ದಂತೆ ಇದ್ದಾನೆ.) -ಎಂದು ಹೇಳಿದಂತೆ ```ಅಗ್ನಿಸಾಕ್ಷಿಯಾಗಿಯೇ ಮಹಾಜನಗಳ ಮುಂದೆ ಇಂಥ ಪ್ರತಿಜ್ಞೆ ಮಾಡಿದೆನು. ಒಬ್ಬರಿಗೊಬ್ಬರು ದ್ರ್ರೋಹವನ್ನು ಬಯಸುವುದು ನ್ಯಾಯವಲ್ಲ''-ಎನ್ನುವ ಭಾವನೆ ಬರಬೇಕಾದರೆ ವಿವಾಹಕಾಲದಲ್ಲಿ ವರನಿಗೆ ತಾನು ಹೇಳುವ ಮಂತ್ರಗಳ ಅರ್ಥ ಗೊತ್ತಿರಬೇಕು. ಇದೆಲ್ಲಾ ಇಲ್ಲದೆ ಕೇವಲ ಸಾಕ್ಷಿಯಂತೆ ಮದುವೆ ನೋಡುತ್ತ್ತಾ ಕುಳಿತಿದ್ದರೆ ವರನಿಗೆ ``ಮಂತ್ರಗಳನ್ನು ಪುರೋಹಿತರು ತಾನೇ ಹೇಳುತ್ತಿದ್ದರು! ಇದರಿಂದ ತಪ್ಪಿದರೂ ಯಾವ ವಿಧವಾದ ದೋಷವೂ ಇಲ್ಲ'' ಎಂದು ತೋರಬಹುದು. ಆದರೆ ಅರ್ಥವನ್ನು ತಿಳಿದು ನಾವು ಮಂತ್ರಗಳನ್ನು ಹೇಳಿದರೆ ಅದು ಹೆಚ್ಚು ಫಲವನ್ನು ಕೊಡುವುದೆಂದು ಹೇಳಲ್ಪಟ್ಟಿದೆ.

\begin{shloka}
``ಯದೇವ ವಿದ್ಯಯಾ ಕರೋತಿ ಶ್ರದ್ಧಯಾ\\
ಉಪನಿಷದಾ ತದೇವ ವೀರ್ಯವತ್ತರಂ ಭವತಿ ||
\end{shloka}

(ಯಾವುದು ತಿಳಿದು, ಶ್ರದ್ಧೆಯಿಂದ, ಧ್ಯಾನದಿಂದ ಮಾಡಲ್ಪಡುವುದೋ ಅದೇ ತಿಳಿಯದೆ ಮಾಡಲ್ಪಟ್ಟಿದ್ದಕ್ಕಿಂತಲೂ ಬಹಳವಾಗಿ ಫಲವನ್ನು ಕೊಡಬಲ್ಲದು.)

ಸಿನೆಮಾ ನೋಡುವಾಗ, ಒಂದು ಹಿಂದೀ ಚಿತ್ರವಾದರೆ ಅದು ನಮಗೆ ತಿಳಿಯದ ಭಾಷೆಯಾದರೂ ಕೂಡ ``ದೃಶ್ಯಗಳು ಬಹಳ ಚೆನ್ನಾಗಿವೆ'' ಎಂದು ಹೇಳುತ್ತೇವೆ. ಹಿಂದೀ ತಿಳಿದವರಾದರೆ ಅದನ್ನು ತಿಳಿದುಕೊಂಡು ಸಂತೋಷಪಡುವರು. ಸಣ್ಣವಿಷಯಗಳಿಗೇ ನಾವು ಮಾಡುವ ಕೆಲಸಗಳ ಅರ್ಥ ಹೀಗಿದ್ದರೆ ಮಂತ್ರಗಳ ಅರ್ಥವನ್ನು ತಿಳಿಯುವುದು ಎಷ್ಟು ಆವಶ್ಯಕವೆಂದು ನಾವು ಹೇಳದೆಯೇ ತಿಳಿದುಕೊಳ್ಳಬಹುದು.

`ವಿದ್ವತ್ವಂ ಅಸ್ಮಾತ್ ಪರಂ' ಎಂದು ಹೇಳಿದೆ. ದಿನವೂ ಒಬ್ಬನು ಸಂಧ್ಯಾವಂದನೆ ಮಾಡುತ್ತಾನೆ. ಅಗ್ನಿಹೋತ್ರ ಮಾಡುತ್ತಾನೆ. ಯಾಗಗಳನ್ನು ಕೈಗೊಳ್ಳುತ್ತಾನೆ. ಪ್ರತಿರಾತ್ರಿಯೂ ಹಗಲು ಬದಲಾಗುತ್ತಾ ಬರುತ್ತದೆ. ದೈನಿಕಕಾರ್ಯಗಳನ್ನು ಮಾಡುತ್ತಾನೆ. ಊಟ ಮಾಡುತ್ತಾನೆ. ಹೀಗೆ ಎಷ್ಟು ದಿನಗಳು ಮಾಡಿದ ಕೆಲಸವನ್ನು ಪದೇ ಪದೇ ಮಾಡುವುದು?

\begin{shloka}
`ರಾತ್ರಿಃ ಸೈವಾ ಸ ಏವ ದಿವಸಃ'\\
(ಅದೇ ರಾತ್ರಿ, ಅದೇ ಹಗಲು)
\end{shloka}

ಹೀಗೆ ಬರುತ್ತಲೇ ಇರುತ್ತದೆ.

\begin{shloka}
`ಅಪಿಧ್ಯಾಯನ್ ವರ್ಣರತಿಪ್ರಮೋದಾನ್\\
ಅತಿದೀರ್ಘೇ ಜೀವಿತೇ ಕೋ ರಮತೇ |'
\end{shloka}

(ಸಂಗೀತ, ಭೋಗ ಅಪ್ಸರಸ್ತ್ರೀಗಳಿಂದ ಉಂಟಾಗಿರುವ ಸುಖ ಇವುಗಳು ನಾಶವುಳ್ಳದ್ದು ಎಂದು ಚಿಂತಿಸುವವನು ದೀರ್ಘ ಜೀವನದಲ್ಲಿ ಹೇಗೆ ರುಚಿಯನ್ನು ಕಾಣುವನು?)

ಆದ್ದರಿಂದ ಇಂಥ ಜೀವನದಲ್ಲಿ ರಸವೇ ಇಲ್ಲವೆಂದು ತಿಳಿಯುವುದು. ಆದ್ದರಿಂದಲೇ `ಆತ್ಮಾನಾತ್ಮ ವಿವೇಕಂ'-ಮಾಡಬೇಕೆಂದು ಶ್ಲೋಕದಲ್ಲಿ ಅನಂತರ ಹೇಳಿದೆ. `ಆತ್ಮಾನಾತ್ಮ ವಿವೇಕಂ' ಎಂದರೆ ಏನು? ಶರೀರಕ್ಕಿಂತ ಬೇರೆಯಾದ ಒಂದು ವಸ್ತುವಿದೆ. ಆ ವಸ್ತುವೇ `ನಾನು' ಎಂದು ಶಾಸ್ತ್ರ ಹೇಳುತ್ತದೆ. ಅನಂತರ ಶಾಸ್ತ್ರವನ್ನು ಇಟ್ಟುಕೊಂಡು ನಮ್ಮ ಯುಕ್ತಿಯಿಂದ ನಾವು ಇದೇ ರೀತಿ ತೀರ್ಮಾನ ಮಾಡಬಹುದು. ಅದಾದ ಮೇಲೆ ಅನುಭವ ಬೇರೆ ಬರಬೇಕಾಗಿದೆ. ಆದ್ದರಿಂದಲೇ-

\begin{shloka}
`ಆತ್ಮಾ ವಾ ಅರೇ ದ್ರಷ್ಟವ್ಯಃ ಶ್ರೋತವ್ಯೋ ಮನ್ತವೋ ನಿದಿಧ್ಯಾಸಿತವ್ಯಃ'
\end{shloka}

(ಪ್ರಿಯಳೇ ! ಆತ್ಮನನ್ನು ಕಾಣಬೇಕು, ಅದನ್ನು ಕುರಿತು ಕೇಳಬೇಕು, ಚಿಂತನೆ ಮಾಡಬೇಕು. ಏಕಾಗ್ರಚಿತ್ತದಿಂದ ಧ್ಯಾನ ಮಾಡಬೇಕು)

-ಎಂದು ಹೇಳಿದೆ. ``ಆತ್ಮಾನಾತ್ಮ ವಿವೇಚನೆ' ಮಾಡುವುದು ಸುಲಭವಾಗಿದ್ದರೂ ಅದನ್ನು ಅನುಭವಕ್ಕೆ ತೆಗೆದುಕೊಂಡು ಬಂದರೆ ಆಗ ಯಾವ ಚಿಂತೆಯೂ ಇಲ್ಲ.

ಹತ್ತು ರೂಪಾಯಿಗಳು ಮಾತ್ರ ಸಂಬಳ ಬರುವ ಒಬ್ಬ ಕೆಲಸಗಾರನನ್ನು ನೋಡಿ, `ನೀನು ಸುಖವಾಗಿದ್ದೀಯಾ?' ಎಂದು ಕೇಳಿದರೆ `ನಾನು ಸುಖವಾಗಿಯೇ ಇದ್ದೇನೆ' ಎಂದು ಹೇಳುವನು. ಮತ್ತೆ ಕೇಳಿದರೆ ``ಏಕೆ ಹೀಗೆ ಕೇಳುತ್ತೀರಿ? ಕಷ್ಟವಿಲ್ಲದೆ ನಾನು ಬಾಳನ್ನು ನಡೆಸಲು ಹೇಗೆ ಸಾಧ್ಯ? ಸುಖ-ದುಃಖ ಎರಡೂ ನನ್ನ ಬಾಳಿನಲ್ಲಿವೆ. ಏನೋ ನನ್ನ ಶಕ್ತಿಗೆ ತಕ್ಕಂತೆ ಬಾಳನ್ನು ನಡೆಸುತ್ತಿದ್ದೇನೆ. ಇದ್ದಿದ್ದರಲ್ಲಿ ನನಗೆ ಸುಖವೂ ಇದೆ' ಎನ್ನುತ್ತಾನೆ. ಇದೇ ರೀತಿ ಸರಕಾರದ ಕೆಲಸದಲ್ಲಿರುವ ಒಬ್ಬನನ್ನು `ನೀನು ಸುಖವಾಗಿದ್ದೀಯಾ' ಎಂದು ಕೇಳಿದರೆ, ಸುಖವಾಗಿಯೇ ಇದ್ದೇನೆ. ಏಕೆ ಹೀಗೆ ಕೇಳುತ್ತೀರಿ? ಮನುಷ್ಯನಾಗಿ ಹುಟ್ಟಿದವನಿಗೆ ಕಷ್ಟವೇ ಇಲ್ಲದೆ ಇರಲು ಸಾಧ್ಯವೇ? ಎಷ್ಟೋ ಕಷ್ಟಗಳಿವೆ. ನನ್ನ ಡೈರೆಕ್ಟರ್ ಬಹಳ ಕೆಟ್ಟವರು. ನಾನು ಹೇಳಿದಂತೆ ಕೇಳುವುದೇ ಇಲ್ಲ. ಆದರೆ ನಾನು ಹೇಳಿದಂತೆ ಮಾಡಿದರೇನೇ ವ್ಯಾಪಾರದಲ್ಲಿ ಅವರಿಗೆ ಲಾಭ ದೊರೆಯುವುದು. ಆದರೂ ಅವರು ನನ್ನ ಮಾತನ್ನು ಕೇಳುವುದೇ ಇಲ್ಲ. ಇದಕ್ಕಾಗಿ ನಾನು ಒಂದು ಕೆಲಸವನ್ನೂ ಮಾಡದೆ ಇದ್ದರೆ ನನ್ನನ್ನು ಕೆಲಸದಿಂದ ತೆಗೆದುಬಿಡುವರು. ಆದರೆ ಅವರು ಹೇಳುವುದನ್ನು ಕೇಳುವುದಕ್ಕಾಗಿ ಸ್ವಲ್ಪ ಹೊತ್ತು, ನಾನು ಪಡುವ ಶ್ರಮಕ್ಕೆ ಲಾಭ ಸಿಕ್ಕಬೇಕು ಎನ್ನುವುದಕ್ಕಾಗಿ ಸ್ವಲ್ಪ ಹೊತ್ತೂ ನಾನು ವಿನಿಯೋಗಿಸಬೇಕಾಗಿದೆ. ಹೇಗೋ ಚೆನ್ನಾಗಿ ಲಾಭಪಡೆಯಬೇಕೆಂದಿದೆ. ಕೆಲಸ ಬೇಡವೆಂದು ತೋರಿದರೂ ಇದನ್ನು ಬಿಟ್ಟರೆ ಸಂಬಳ ಸಿಕ್ಕುವುದಿಲ್ಲ. ಆದ್ದರಿಂದ ಹೇಗೋ ನೋಡಬೇಕಲ್ಲಾ ಎನ್ನುವುದಕ್ಕಾಗಿ ಹೋಗುತ್ತಿದ್ದೇನೆ. ಹೇಗೋ ನೋಡುತ್ತಿದ್ದೇನೆ'' -ಎಂದೇ ಹೇಳುವನು. ಹತ್ತು ಸಾವಿರರೂಪಾಯಿಗಳು ಸಂಬಳ ತೆಗೆದುಕೊಳ್ಳುವವನೂ ಇದೇ ರಾಗವನ್ನೇ ಹಾಡುತ್ತಾನೆ.

ಹೀಗೆ ಎಲ್ಲರೂ ಹೇಳಿಕೊಂಡಿರುವ ಕಾಲದಲ್ಲಿ ಇದರಿಂದ ಬಿಡುಗಡೆ ಬೇಕಾದರೆ ಮೋಕ್ಷವೆನ್ನುವುದು ಬೇಕು. ಆದ್ದರಿಂದಲೇ `ಮುಕ್ತಿ' ಎನ್ನುವುದು ಹೇಳಲ್ಪಟ್ಟಿದೆ.

\begin{shloka}
`ಶ್ರೋತಿಯಸ್ಯ ಚಾಕಾಮಹತಸ್ಯ'\\
(ವೇದವನ್ನು ತಿಳಿದವನಿಗೂ, ಕಾಮವಿಲ್ಲದವನಿಗೂ...........)
\end{shloka}

-ಎನ್ನುವಂತೆ ಕಾಮವಿಲ್ಲದವನಿಗೆ ತೃಪ್ತಿ ಇರುವುದು. ಆದ್ದರಿಂದ ಪ್ರಪಂಚದಲ್ಲಿರುವುದನ್ನೆಲ್ಲಾ

\begin{shloka}
`ಆಸಾರಮೇವ ಸಂಸಾರಂ ದೃಷ್ಟ್ಯ್ವಾ'
\end{shloka}

(`ಸಂಸಾರ'ವೆನ್ನುವುದು ಸಾರವಿಲ್ಲದುದೆಂದು ಕಂಡು)-ಎಂದು ಭಾವಿಸಬೇಕು.

\begin{shloka}
`ಕಾರ್ಯಮಿತ್ಯೇವ ಯತ್ಕರ್ಮ ನಿಯತಂ ಕ್ರಿಯುತೇಽರ್ಜುನ |'
\end{shloka}

(ಅರ್ಜುನ! ಯಾವುದು ಮಾಡಬೇಕೆಂದೇ ನಿತ್ಯ ಕರ್ಮವಿದೆಯೋ......ಹಾಗೆ ಇದ್ದರೆ)

`ಏನೋ ಗೃಹಸ್ಥನೆನ್ನುವ ಹೆಸರು ಬಂದಾಯಿತು, ಕೆಲಸ ಆದರೆ ಸಂತೋಷ. ಕೆಲಸ ಆಗದಿದ್ದರೂ ಅದರಿಂದ ಚಿಂತೆ ಇಲ್ಲ. ಏಕೆಂದರೆ ಪ್ರಪಂಚದ ಸ್ವಭಾವವೇ ಇದು ಎನ್ನುವ ಭಾವನೆಯಿಂದ ಮನುಷ್ಯರು ಪ್ರಪಂಚದಲ್ಲಿ ಬಾಳುತ್ತಿದ್ದರೆ ಆತ್ಮಾನಾತ್ಮ ವಿವೇಚನೆಯೂ ಅದರಿಂದ ಉಂಟಾಗುವ ಅನುಭವವೂ ಬಂದು ಕೊನೆಗೆ ಮುಕ್ತಿಯೂ ದೊರೆಯುವುದು. ಇಂಥ ಅರಿವಿನ ಯೋಗ್ಯತೆ ನಮಗೆ ಉಂಟಾಗಬೇಕಾದರೆ,

\begin{shloka}
`ಶತಕೋಟಿ ಜನ್ಮ ಸುಕೃತೇಃ ಪುಣ್ಯೈರ್ವಿನಾ ಲಭ್ಯತೇ ||'
\end{shloka}

-ಎಂದು ಹೇಳಿದಂತೆ ಎಷ್ಟೋ ಕೋಟಿ ಜನ್ಮಗಳಲ್ಲಿ ಮಾಡಿದ ಪುಣ್ಯದಿಂದಲೂ ಸಾಧನೆಯಿಂದಲೂ ಈಗ ಇವುಗಳೆಲ್ಲವೂ ದೊರೆಯಲು ಸಾಧ್ಯವಾಯಿತೆಂದು ಶಂಕರ ಭಗವತ್ಪಾದರು `ವಿವೇಕಚೂಡಾಮಣಿ'ಯಲ್ಲಿ ಹೇಳಿದ್ದಾರೆ.

ಈಗಲೇ ನಾವು ಮುಕ್ತಿ ಪಡೆಯುತ್ತೇವೆಯೇ ಎಂದು ಕೇಳಿದರೆ, ಎಲ್ಲರೂ {\eng I.A.S.} ಪಾಸ್ ಮಾಡಲು ಸಾಧ್ಯವಿಲ್ಲ. ನಾವು ಮೊದಲು ಪ್ರಾಥಮಿಕ ಶಾಲೆಯಲ್ಲಿ ಓದಬೇಕಾಗಿದೆ. ಅಲ್ಲಿ ಓದಿದ ಮಾತ್ರಕ್ಕೆ ನಮಗೆ {\eng I.A.S.} ಸರ್ಟಿಫೀಕೇಟ್ ಕೊಡುವುದಿಲ್ಲ. `{\eng I.A.S.} ಪಾಸ್ ಮಾಡುವುದಕ್ಕೇನಾ ಹಾಗೆ ಓದಿದ್ದು' ಎಂದು ಕೇಳಿದರೆ, ಪ್ರಾಥಮಿಕ ಶಾಲೆಯಲ್ಲಿ ಪಾಸ್ ಮಾಡಿದ್ದರೂ {\eng I.A.S.} ಪಾಸ್ ಆಗದೆ ಇರಬಹುದು. ಸಂಬಂಧ ಪಟ್ಟ ಶಾಲೆಗಳು ಎಷ್ಟೋ ಇವೆ. ಅವುಗಳಲ್ಲಿ ಒಂದು ಲಕ್ಷ ಮಂದಿ ಓದುತ್ತಾರೆಂದರೆ ಅವರುಗಳಲ್ಲಿ {\eng I.A.S.} ಪರೀಕ್ಷೆಯಲ್ಲಿ ತೇರ್ಗಡೆಯಾಗುವವರು ಎಷ್ಟು ಮಂದಿ ಎಂದು ಲೆಕ್ಕ ಹಾಕಿದರೆ ಸುಮಾರು ಹತ್ತುಜನ ಮಾತ್ರ {\eng I.A.S.} ಪರೀಕ್ಷೆಯಲ್ಲಿ ತೇರ್ಗಡೆಯಾಗುತ್ತಾರೆಂದರೆ ಅದು ಎಷ್ಟು ಕಷ್ಟವಾದ ಪರೀಕ್ಷೆಯೆಂದು ಗೊತ್ತಾಗುತ್ತದೆ. ಅದೇ ರೀತಿ `ಮುಕ್ತಿಯು' ವಿಷಯದಲ್ಲಿ ಆಸೆ ಇರುವವರೆಲ್ಲಾ ಪ್ರಯತ್ನ ಮಾಡಬೇಕು. ಇಷ್ಟೆಲ್ಲಾ ಸಾಧನೆ ಮಾಡಿದರೂ ಈ ಜನ್ಮದಲ್ಲಿ ಅದು ದೊರೆಯದೆ ಇದ್ದರೆ.-

\begin{shloka}
`ಮನುಷ್ಯಣಾಂ ಸಹಸ್ರೇಷು ಕಶ್ಚಿತ್ ಯತತಿ ಸಿದ್ಧಯೇ |\\
ಯತತಾಮಪಿ ಸಿದ್ಧಾನಾಂ ಕಶ್ಚಿನ್ಮಾಂ ವೇತ್ತಿ ತತ್ತ್ವತಃ ||'
\end{shloka}

(ಸಾವಿರಾರು ಜನರಲ್ಲಿ ಯಾರೋ ಒಬ್ಬನು ಸಿದ್ಧಿಗಾಗಿ ಪ್ರಯತ್ನ ಪಡುತ್ತಾನೆ. ಹಾಗೆ ಪ್ರಯತ್ನಪಡುವವರಲ್ಲಿ ಯಾರೋ ಒಬ್ಬನು ನನ್ನನ್ನು ಇದ್ದಂತೆ ತಿಳಿಯುತ್ತಾನೆ.)

ಈಗಲೇ ಮುಕ್ತಿಯನ್ನು ಸಾಧಿಸಲು ಸಾಧ್ಯವಾಗದೆ ಹೋದರೆ-

\begin{shloka}
`ನಾಸ್ತಿ ಚೇನ್ನಾಸ್ತಿ ನೋ ಹಾನಿಃ'
\end{shloka}

(ಇಲ್ಲವೆಂದರೆ ಅದರಿಂದ ಹಾನಿಯೂ ಇಲ್ಲ.)

-ಎನ್ನುವಂತೆ ಮತ್ತೆ ಜನ್ಮವಿದೆ. ಈ ಜನ್ಮದಲ್ಲಿ ನಾವು ಮಾಡುವ ಪ್ರಯತ್ನದ ಸಂಸ್ಕಾರಗಳು ಮುಂದಿನ ಜನ್ಮದಲ್ಲಿಯೂ ಬಂದು ನಮಗೆ ಒಳ್ಳೆಯ ಫಲಗಳು ಉಂಟಾಗುವುದಕ್ಕಾದರೂ ನಾವು ಈ ಜನ್ಮದಲ್ಲಿ ಇಂಥ ಪ್ರಯತ್ನಗಳೆಲ್ಲವನ್ನೂ ಮಾಡಬೇಕಾಗಿದೆ. ಇದರಿಂದ `ವಿವೇಕ ಚೂಡಾಮಣೀ' ಎನ್ನುವ ಉತ್ತಮ ಗ್ರಂಥವನ್ನು ಎಲ್ಲರೂ ಓದಬೇಕೆಂದು ನಾನು ಹೇಳುವೆನು.

\begin{shloka}
`ದಿನೇ ದಿನೇ ಚ ವೇದಾಂತಶ್ರವಣಾತ್ ಭಕ್ತಿ ಸಂಯುತಾತ್'
\end{shloka}

[ಪ್ರತಿದಿನವೂ ಭಕ್ತಿಯಿಂದ ವೇದಾಂತ ಶ್ರವಣ ಮಾಡಿದರೆ (ಒಳ್ಳೆಯದಾಗುತ್ತದೆ)]

ವೇದಾಂತವೆನ್ನುವುದು ಪಾಪಗಳನ್ನು ದೂರಮಾಡುವ ಪ್ರಾಯಶ್ಚಿತ್ತಗಳಂತೆ ಫಲ ಕೊಡುವುದೆಂದು ನಮಗೆ ಸ್ಪಷ್ಟವಾಗುತ್ತದೆ. ಇಂಥ ಶ್ರವನ ಮುಂತಾದವುಗಳನ್ನು ನಾವು ಸರಿಯಾಗಿ ಮಾಡಿ ನಮ್ಮ ಬಾಳನ್ನು ಸಾರ್ಥಕಮಾಡಿಕೊಳ್ಳಬೇಕೆಂದು ಹೇಳಿ ಈ ವಿಷಯವನ್ನು ಮುಗಿಸುತ್ತೇನೆ.

































































































































































































