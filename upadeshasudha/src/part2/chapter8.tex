\chapter{ಭಗವಂತನು ಇದ್ದಾನೆಯೇ?}\label{chap8}

\begin{shloka}
ಗಂಗಾಪೂರ ಪ್ರಚಲಿತ ಜಟಾಸ್ರಸ್ತ ಭೋಗೀಂದ್ರಭೀತಾಂ\\
ಆಲಿಂಗತೀಮಚಲತನಯಾಂ ಸಸ್ಮಿತಂ ವೀಕ್ಷ್ಯಮಾಣಃ |\\
ಲೀಲಾಪಾಂಗೈಃ ಪ್ರಣತಜನತಾಂ ನಂದಯಂಶ್ಚಂದ್ರಮೌಲಿಃ\\
ಮೋಹಧ್ವಾಂತಂ ಹರತು ಪರಮಾನಂದಮೂರ್ತಿಃ ಶಿವೋ ನಃ‌ ||
\end{shloka}

ಉದಯನಾಚಾರ್ಯರೆಂಬ ಒಬ್ಬ ಶ್ರೇಷ್ಠ ತಾರ್ಕಿಕರಿದ್ದರು. `ಪರಮಶಿವನು, ಇದ್ದಾನೆ'-ಎಂದು ನಿರೂಪಿಸುವುದಕ್ಕಾಗಿಯೇ ಅವರು ಒಂದು ಗ್ರಂಥವನ್ನು ಬರೆದರು. ಆ ಗ್ರಂಥದಲ್ಲಿ ಪೂರ್ತಿಯಾಗಿ ಪರಮಶಿವನು ಇದ್ದಾನೆಂದೇ ಹೇಳಿದ್ದಾರೆ. ಪರಮ ಶಿವನೆಂದರೆ ಅವರು ವಿಷ್ಣುವನ್ನಾಗಲಿ, ಬ್ರಹ್ಮನನ್ನಾಗಲಿ ಹೇಳಿಲ್ಲ. ಪ್ರಪಂಚದ ಸೃಷ್ಟಿ, ಸ್ಥಿತಿ, ಲಯಗಳಿಗೆ ಕಾರಣವಾಗಿರುವ ಒಂದು ಚೈತನ್ಯ ಇದಯೇ, ಇಲ್ಲವೇ ಎನ್ನುವುದನ್ನು ಅವರು ವಿಮರ್ಶಿಸುತ್ತಾರೆ. ನಾವು ಆ ಚೈತನ್ಯಕ್ಕೆ ವಿಷ್ಣುವೆಂದೋ, ಶಿವನೆಂದೋ ಹೆಸರು ಕೊಡಬಹುದು. ಇದಕ್ಕೆ ಯಾವ ಆಕ್ಷೇಪಣೆಯೂ ಇಲ್ಲ. ಆದರೆ ಪ್ರಪಂಚವನ್ನು ಸೃಷ್ಟಿಸಿ, ಅದನ್ನು ಕಾಪಾಡಿ, ಅದರ ಲಯಕ್ಕೆ ಕಾರಣವಾದ ಒಂದು ವಸ್ತು ಇದೆ ಎನ್ನುವುದು ನಿಜ.

\begin{shloka}
`ಮಮೈವಾಂಶೋ ಜೀವಲೋಕೇ ಜೀವಭೂತಃ ಸನಾತನಃ\\
(ಈ ದೇಹದಲ್ಲಿ ಸನಾತನ ಜೀವಾತ್ಮನು ನನ್ನದ್ದೇ ಆದ ಅಂಶ)
\end{shloka}

-ಎಂದು ಗೀತೆಯಲ್ಲಿ ಭಗವಂತನು ಹೇಳಿದ್ದಾನೆ.

ಪರಮಶಿವನಿಂದಲೇ ನಾವು ಬಂದಿರುವುದು. ಬಂದ ನಾವು ಅವನನ್ನೇ ಮರೆತಿದ್ದೇವೆ. ಮತ್ತೆ ಅವನ ಪಾದಾರವಿಂದಗಳಲ್ಲಿ ಸೇರಿ `ಅವನು' ಆಗುವವರೆಗೂ ನಮಗೆ ಹುಟ್ಟು-ಸಾವು ಎನ್ನುವ ಚಕ್ರ ಇದ್ದೇ ಇದೆ. ಹಾಗಿರುವ. ಪರಮಶಿವನ ವಿಷಯದಲ್ಲಿ ಜನರು ಯಾವ ವಿಧವಾದ ಭಾವನೆಯನ್ನು ಇಟ್ಟುಕೊಂಡಿದ್ದಾರೆ ಎನ್ನುವುದನ್ನು ಈಗ ನೋಡೋಣ.

ಪುಷ್ಪದಂತನೆಂಬ ಒಬ್ಬ ಪ್ರಸಿದ್ಧ ಕವಿ ಇದ್ದನು. ಅವನು ಒಬ್ಬ ಗಂಧರ್ವ. ಆ ಗಂಧರ್ವನು ಒಮ್ಮೆ ಭಗವಂತನಿಗೆ ಪೂಜೆ ಮಾಡಿದ ನಿರ್ಮಾಲ್ಯವನ್ನು ದಾಟಿಬಿಟ್ಟನಂತೆ. ಅದರಿಂದ ಅವನಿಗೆ ಅಂತರ್ಧಾನ ಶಕ್ತಿಗೆ ಹೋಗಿಬಿಟ್ಟಿತು. ಆಗ ಅವನು ಪರಶಿವನನ್ನು ಸ್ತ್ರೋತ್ರ ಮಾಡಿದನು. ಏತಕ್ಕೆ ಸ್ತ್ರೋತ್ರ ಮಾಡಿದನೆಂದರೆ, `ನನ್ನ ಶಕ್ತಿಗಳನ್ನು ಕಳೆದುಕೊಂಡೆನು, ನನಗೆ ಮತ್ತೆ ಆ ಶಕ್ತಿಗಳು ಬರಬೇಕು' ಎನ್ನುವ ಆಸೆಯಿಂದ ಶಾಸ್ತ್ರದಲ್ಲಿರುವ ಎಲ್ಲಾ ತಾತ್ಪರ್ಯವನ್ನೂ ಸೇರಿಸಿಕೊಂಡು ಸುಂದರವಾಗಿ ಸ್ತ್ರೋತ್ರವನ್ನು ಮಾಡಿದನು.

\begin{shloka}
`ತವೈಶ್ವರ್ಯಂ ಯತ್ತಜ್ಜಗದುದಯ ರಕ್ಷಾಪ್ರಲಯ ಕೃತ್\\
ತ್ರಯೀವಸ್ತು ವ್ಯಸ್ತಂ ತಿಸೃಷು ಗುಣಭಿನ್ನಾಸು ತನುಷು |\\
ಅಭವ್ಯಾನಾಮಸ್ಮಿನ್ ವರದ ರಮಣೀಯಾಮರಣೀಮ್\\
ವಿಹನ್ತುಂ ವ್ಯಾಕ್ರೋಶೀಂ ವಿದಧಥ ಇಹೈಕೇ ಜಡಧಿಯಃ ||'
\end{shloka}

ಪ್ರಪಂಚದಲ್ಲಿ ಬುದ್ಧಿಶಾಲಿಗಳೂ ಇದ್ದಾರೆ, ಮಂದ ಬುದ್ಧಿಗಳೂ ಇದ್ದಾರೆ. ಮಂದಬುದ್ಧಿಗಳಿಗೆ ಎಲ್ಲಾ ವಿಷಯದಲ್ಲೂ ಮಂದಬುದ್ಧಿ ಇಲ್ಲ. ಒಳ್ಳೆಯ ವಿಷಯವನ್ನು ತಿಳಿದುಕೊಳ್ಳುವುದರಲ್ಲಿ ಮಾತ್ರ ಮಂದಬುದ್ಧಿ. ಕೆಟ್ಟ ವಿಷಯಗಳನ್ನು ಹೇಳಬೇಕಾಗಿದ್ದರೆ ಅವರಿಗೆ ಬಹಳ ಬುದ್ಧಿಯೇ ಇರುತ್ತದೆ. ಆದರೆ ಯಾವುದಾದರೂ ಒಳ್ಳೆಯ ವಿಷಯವನ್ನು ಕೇಳಿದರು ಆ ವಿಷಯದಲ್ಲಿ ಮಾತ್ರ ಅವರು ಮಂದಬುದ್ಧಿಗಳು.

ಭಗವಂತನ ಐಶ್ವರ್ಯ ಎಂತಹುದೆಂದರೆ,

\begin{shloka}
`ತವೈಶ್ವರ್ಯಂ ಯತ್ತಜ್ಜಗದುದಯ ರಕ್ಷಾಪ್ರಲಯ ಕೃತ್'
\end{shloka}

-ಎಂದು ಹೇಳಿದೆ.

ವ್ಯಾಸರು, `ಮೊದಲು ಪರಮಾತ್ಮನನ್ನು ತಿಳಿಯಲು ಇಚ್ಚೆ ಇರಬೇಕೆಂದು' ಹೇಳಿದರು. ಪರಮಾತ್ಮನನ್ನು ಕುರಿತು ವಿಚಾರ ಮಾಡಬೇಕು. ಪರಮಾತ್ಮನು ಇದ್ದಾನೆಯೇ ಇಲ್ಲವೇ ಎನ್ನುವ ಪ್ರಶ್ನೆಗಳಿಗೆಲ್ಲಾ ಭಗವತ್ಪಾದರೂ `ಇದ್ದಾನೆ' ಎಂದು ನಿರೂಪಣೆ ಮಾಡಿದ್ದಾರೆ. ಅವನು ಇದ್ದಾನೆ ಎಂದರೆ,

\begin{shloka}
`ಲಕ್ಷಣಪ್ರಮಾಣಾಭ್ಯಾಂಹಿ ವಸ್ತು ಸ್ಥಿತಿಃ
\end{shloka}

(ಲಕ್ಷಣ ಮತ್ತು ಪ್ರಮಾಣಗಳ ಮೂಲಕ ವಸ್ತುವಿನ ಸಿದ್ಧಿ ಉಂಟಾಗುತ್ತದೆ.)' ಎಂದು ಶಾಸ್ತ್ರ ಹೇಳುವಂತೆ ಇರಬೇಕು.

ಯಾವುದಾದರೂ ವಸ್ತು ಹೊಸದಾಗಿ ಬಂದರೆ, ಅದು ಎಂಥ ವಸ್ತು ಎಂದು ಕೇಳುತ್ತೇವೆ. ಅದನ್ನು ನೋಡಿದವನು ಯಾರು ಎಂದು ಕೇಳುತ್ತೇವೆ. ಒಂದು ವಸ್ತು ಇಂತಹದು ಎಂದು ನವು ಹೇಳುವಾಗ ಅದನ್ನು ಆ ವಸ್ತುವನ್ನು ಯಾರಾದರೂ ನೋಡಿದ್ದರೆ ಅದಕ್ಕೆ ಪ್ರಮಾಣವೆನ್ನುತ್ತೇವೆ. ನಾವೇ ನಮ್ಮ ಕಣ್ಣುಗಳಿಂದ ನೋಡಿ ಹೇಳಿದರೆ `ಪ್ರತ್ಯಕ್ಷ ಪ್ರಮಾಣ' ಆಗುತ್ತದೆ. ಒಂದು ವಸ್ತುವನ್ನು ಒಬ್ಬನು ಕೇಳಿ ತಿಳಿದುಕೊಂಡು ಅವನ ಮೂಲಕ ಇನ್ನೊಬ್ಬನು ತಿಳಿದುಕೊಂಡರೆ ಅದು `ಶಬ್ದಪ್ರಮಾಣ' ವೆನ್ನಲಾಗುವುದು. ಇದೇ ರೀತಿ `ಅನುಮಾನ'ದಂಥ ಅನೇಕ ಪ್ರಮಾಣಗಳಿವೆ. ಅವುಗಳಲ್ಲಿ ವಿಶೇಷವಾಗಿ ಪ್ರತ್ಯಕ್ಷ, ಅನುಮಾನ, ಶಬ್ದ - ಈ ಮೂರು ಪ್ರಮಾಣಗಳೂ ಹೇಳಲ್ಪಟ್ಟಿವೆ. ಇಂಥ ಪ್ರಮಾಣಗಳಲ್ಲಿ ಯಾವ ಪ್ರಮಾಣದ ಮೂಲಕ ಭಗವಂತನು ಇದ್ದಾನೆಂದು ಹೇಳಬಹುದೆಂದು ನಾವು ನೋಡಬೇಕು. ಮೇಲೆ ಹೇಳಿದ ಶ್ಲೋಕದಲ್ಲಿ ಪರಮಾತ್ಮನಿಗೆ ಲಕ್ಷಣವಿದೆಯೆಂದು ಕವಿ ಹೇಳುತ್ತಾನೆ.

ಭಗವಂತನ ಐಶ್ವರ್ಯ ಸಾಮರ್ಥ್ಯವಾಗುತ್ತದೆ. ಭಗವಂತನ ಸಾಮರ್ಥ್ಯ ಎಂತಹದು?

ನಾವು ನೋಡುತ್ತಿರುವ ನಕ್ಷತ್ರ ಮಂಡಲವೋ, ಭೂಮಂಡಲವೋ, ಈ ಸೂರ್ಯನೋ, ಬೇರೆ ಯಾವುದೆಲ್ಲವನ್ನೂ ನಾವು ನೋಡುತ್ತೇವೋ, ಯಾವುದೆಲ್ಲವನ್ನೂ ಕೇಳುತ್ತೇವೋ, ಯಾವ ವಸ್ತುಗಳೆಲ್ಲವನ್ನೂ ಯಾವುದಾದರೂ ಒಂದು ಇಂದ್ರಿಯದಿಂದ ನಾವು ತಿಳಿಯುತ್ತೇವೆಯೋ, ಅವುಗಳೆಲ್ಲವೂ ಉಂಟಾಗಿರುವುದಾಗಿ ಎಲ್ಲರೂ ಒಪ್ಪುತ್ತಾರೆ. ಯಾರೂ ಪ್ರಪಂಚ ಉಂಟಾಗದೆ ಇರುವುದೊಂದು ಎಂದು ಹೇಳುವುದಿಲ್ಲ. ಆದ್ದರಿಂದ ಪ್ರಪಂಚ ಉಂಟಾಗಿರುವುದು ಎನ್ನುವುದು ತೀರ್ಮಾನವೇ. ಅದು ಯಾವಾಗ ಉಂಟಾಯಿತು ಎನ್ನುವುದನ್ನೆಲ್ಲಾ ನಾವು ಈಗ ಸಂಶೋಧನೆ ಮಾಡಲು ಹೊರಟಿಲ್ಲ. ಈ ಪ್ರಪಂಚದ ಸೃಷ್ಟಿಯೊಡನೆ, ಅದು ಅಲ್ಲಿಗೆ ನಿಲ್ಲದೆ, ಅದು ಕಾಪಾಡಲ್ಪಡುತ್ತಲೂ ಇದೆ. ನಾವು ಒಂದು ವಸ್ತುವನ್ನು ಉಂಟುಮಾಡುತ್ತೇವೆ. ಅದನ್ನು ತೆಗೆದು ಭದ್ರವಾಗಿ ಒಂದು ಪೆಟ್ಟಿಗೆಯಲ್ಲಿಡುತ್ತೇವೆ. ಅವಶ್ಯಕವಾದಾಗ ಅದನ್ನು ತೆಗೆದು ಉಪಯೋಗಿಸುತ್ತೇವೆ. ಅದನ್ನು ಉಪಯೋಗಿಸುವ ರೀತಿಯಲ್ಲಿ ಉಪಯೋಗಿಸುತ್ತೇವೆ. ಹಾಗಲ್ಲದೆ ಯಾರೋ ತಯಾರು ಮಾಡಿ ಅದನ್ನು ಹೊರಗೆ ಹಾಕಿದರೆ, ಅದು ಬಿಸಿಲಿನಿಂದಲೋ, ಮಳೆಯಿಂದಲೋ ಅಥವಾ ಗಾಳಿಯಿಂದಲೋ ಕೆಟ್ಟು ಹೋಗುತ್ತದೆ. ಹೀಗಲ್ಲದೆ ಈ ಪ್ರಪಂಚ ಚೆನ್ನಾಗಿ ನಡೆಯುತ್ತಿರುವುದನ್ನು ನೋಡುತ್ತೇವೆ. ಒಂದು ಮಾವಿನ ಓಟೆಯನ್ನು ಹಾಕಿದರೆ ಅದರಿಂದ ಮಾವಿನ ಮರದ ಮೊಳಕೆಯೇ ಬರುತ್ತದೆ. ಭತ್ತದ ಪೈರು ಹಾಕಿದರೆ ಭತ್ತವೇ ಬರುತ್ತದೆ. `ಇಂದು ಭತ್ತ ಬರುತ್ತದೆ, ನಾಳೆ ಬೇರೆ ಪೈರು ಬರುತ್ತದೆ' ಎನ್ನುವುದು ಇಲ್ಲ. ಎಲ್ಲವೂ ಒಂದು ನಿಯಮದಿಂದ ನಡೆಯುತ್ತಿರುವುದನ್ನು ನಾವು ಪ್ರಪಂಚದಲ್ಲಿ ನೋಡುತ್ತಿದ್ದೇವೆ. ಇದಕ್ಕೆ ರಕ್ಷಣೆ ಎಂದು ಹೆಸರು. ಮನುಷ್ಯನಿಗೆ ಮೊದಲು ಬಾಲ್ಯ, ಅನಂತರ ಯೌವನ, ಅನಂತರ ವಾರ್ಧಕ್ಯ (ಮುಪ್ಪು) ಇವು ಕ್ರಮವಾಗಿ ಬರುತ್ತವೆ. ಹಾಗಲ್ಲದೆ ಮೊದಲು ವಾರ್ಧಕ್ಯ, ಅನಂತರ ಯೌವನ ಹೀಗೆ ಎಂದೂ ಬರುವುದಿಲ್ಲ. ಅದೇ ಮೊದಲು ಜನ್ಮ, ಅನಂತರ ಸ್ಥಿತಿ, ಅದಾದ ಮೇಲೆ ಲಯ ಎಂದು ಕ್ರಮವಾಗಿದೆ.

ಯಾಸ್ಕರು ನಿರುಕ್ತದಲ್ಲಿ ಹೇಳುವಂತೆ `ಜಾಯತೇ' (ಉಂಟಾಗುತ್ತದೆ) `ಆಸ್ತಿ' (ಇರುತ್ತದೆ) `ವರ್ಧತೇ' (ಬೆಳೆಯುತ್ತದೆ) `ವಿಪರಿಣಮತೇ' (ಬದಲಾಗುತ್ತದೆ) `ಅಪಕ್ಷೀಯತೇ (ಕೆಟ್ಟು ಹೋಗುತ್ತದೆ), `ನಶ್ಯತಿ' (ನಾಶವಾಗುತ್ತದೆ) ಎಂದು ಆರು ವಿಧಗಳಿವೆ.

ಮೊದಲು ಒಂದು ವಸ್ತು ಉಂಟಾಗಬೇಕು. ಉಂಟಾದರೆ ತಾನೇ ಅದು ಇರಲು ಸಾಧ್ಯ? ಉಂಟಾಗದೆ ಒಂದು ವಸ್ತುವೂ ಇರಲು ಸಾಧ್ಯವಿಲ್ಲ, ಆದ್ದರಿಂದ ಉಂಟಾಗುವುದು ಒಂದು ಸ್ಥಿತಿ. ಇರುವುದು ಎನ್ನುವುದು ಒಂದು ಸ್ಥಿತಿ. ಅನಂತರ `ವಿಪರಿಣಮತೇ' ಎಂದು ಹೇಳಿರುವಂತೆ ನಾವು {\eng 50} ವರ್ಷಗಳವರೆಗೆ ಬೆಳೆಯುತ್ತಲೆ ಬರುತ್ತೇವೆ. ಅನಂತರ ನಮಗೆ ಶೈಥಿಲ್ಯ ಉಂಟಾಗಿದೆ ಎನ್ನುತ್ತೇವೆ. ಇನ್ನೂ ಕೆಲವು ದಿನಗಳು ಕಳೆದ ಮೇಲೆ ನಮ್ಮ ಹಲ್ಲುಗಳು ಬಿದ್ದುಹೋದವು ಎನ್ನುತ್ತೇವೆ. ಕನ್ನಡಕ ಹೊಸದಾಗಿ ಹಾಕಿಕೊಳ್ಳುತ್ತೇವೆ ಎನ್ನುವುದು, ಅದಾದ ಮೇಲಿನದು `ಕ್ಯಾಟ್‌ರಾಕ್ಟ್' ಬಂದಿದೆ ಎನ್ನುವುದು ಇನ್ನೊಂದು, ಅನಂತರ `ಅಪಕ್ಷೀಯತೇ' ಎಂದು ಹೇಳಿದೆ. ಶಕ್ತಿಯಾದ ಯಾವ ಔಷಧವನ್ನು ಸೇವಿಸಿದರೂ ಕೂಡ ಶರೀರ ಆಯಾಸಗೊಂಡು ಇರುವುದೇ ವಿನಹ ಆ ಔಷಧ ಶರೀರವನ್ನು ಬಲವುಳ್ಳದ್ದಾಗಿ ಮಾಡುವುದೇ ಇಲ್ಲ. ಇವೆಲ್ಲಾ ಆದ ಮೇಲೆ `ನಶ್ಯತಿ' ಎಂದು ಹೇಳಲ್ಪಟ್ಟಿದೆ. ಆದ್ದರಿಂದ ಹೀಗೆ ಒಬ್ಬನು ನಾಶಹೊಂದುತ್ತಾನೆ.


ಹೀಗೆ ಆರು ಸ್ಥಿತಿಗಳನ್ನು ಅವರು ಹೇಳಿದರು. ವ್ಯಾಸರು ವೇದದಲ್ಲಿರುವ ತಾತ್ಪರ್ಯವನ್ನು ಹೇಳಲು ಬಂದರೆ ವಿನಹ ಪ್ರಪಂಚದಲ್ಲಿ ತಾವು ನೋಡಿದುದನ್ನೆಲ್ಲಾ ತಮ್ಮ ಸೂತ್ರಗಳಲ್ಲಿ ಬರೆಯಬೇಕೆನ್ನುವುದು ಅವರ ಅಭಿಪ್ರಾಯವಲ್ಲ. ಅವರು,


\begin{shloka}
`ಜನ್ಮಾದ್ಯಸ್ಯ ಯತಃ'
\end{shloka}

-ಎಂದು ಹೇಳುವುದನ್ನು ಭಗವತ್ಪಾದರು,

\begin{shloka}
`ಜನ್ಮ ಸ್ಥಿತಿಂ ಭಂಗಂ ಸಮಾಸಾರ್ಥತಃ'
\end{shloka}

ಜನ್ಮ, ಸ್ಥಿತಿ, ನಾಶ (ಈ ಮೂರು ಸೇರಿ) ಸಮಾಸ (ಸೇರುವಿಕೆ)ದ ಅರ್ಥ- ಎಂದು ಬರೆದಿದ್ದಾರೆ. ಅದರಲ್ಲಿ ಜನ್ಮ, ಸ್ಥಿತಿ, ಭಂಗ ಎಂದು ಹೇಳಲಾಗಿದೆ, ಯಾಸ್ಕರು ಹೇಳಿದ ಆ ಆರು (ವಿಧಗಳನ್ನು) ತೆಗೆದುಕೊಳ್ಳಬಾರದೆ ಎಂದು ಕೇಳಿದರೆ, ಯಾಸ್ಕರು ಯಾವಾಗ ಬರೆದರು? ಪ್ರಪಂಚ ಉಂಟಾಗಿ ಎಷ್ಟೋ ದಿನಗಳು ಕಳೆದ ಮೇಲೆ ಅವರು ಹುಟ್ಟಿ, ಪ್ರಪಂಚವನ್ನು ನೋಡಿ-

`ಜಾಯತೇ, ಆಸ್ತಿ, ವರ್ಧತೇ, ವಿಪರಿಣಮತೇ, ಅಪಕ್ಷೀಯತೇ, ನಶ್ಯತಿ' ಎಂದು ಹೇಳಿದರು.

ಆದರೆ ವ್ಯಾಸರು ವೇದಾತಾತ್ಪರ್ಯವನ್ನು ಇಲ್ಲೇ ಹೇಳ ಬಯಸಿದರು.

\begin{shloka}
`ಯತೋ ವಾ ಇಮಾನಿ ಭೂತಾನಿ ಜಾಯನ್ತೇ\\
ಯೇನ ಜಾತಾನಿ ಜೀವನ್ತಿ ಯತ್ತ್ರಯನ್ತ್ಯಭಿಸಂವಿಶನ್ತಿ'
\end{shloka}

(ಯಾವುದರಿಂದ ಎಲ್ಲಾ ಭೂತಗಳು ಹುಟ್ಟುವಿಕೆಯನ್ನು ಪಡೆದಿರುವೆಯೋ ಯಾವುದರಿಂದ ಹುಟ್ಟಿದ ಮೇಲೆ ಅವುಗಳ ಬಾಳುತ್ತವೆಯೋ, ಯಾವುದನ್ನು ಉದ್ದೇಶಿಸಿ ನಡೆದು ಯಾವುದರಲ್ಲಿ ಒಂದಾಗಿ ಬಿಡುವುದೋ)- ಎಂದು ಹೇಳಿದಂತೆ, ವೇದದಲ್ಲಿ `ಜನ್ಮ ಸ್ಥಿತಿ ನಾಶಃ' ಎನ್ನುವ ಮೂರೂ ಹೇಳಲ್ಪಟ್ಟಿವೆ. ಬೇರೆ ಯಾವುದೂ ಹೇಳಲ್ಪಟ್ಟಿಲ್ಲ. ಅದಕ್ಕೆ ಬೇರೆ ಯಾವುದೂ ಇಲ್ಲವೆಂದೂ ಅರ್ಥವಿಲ್ಲ.

\begin{shloka}
`ಜನ್ಮ ಸ್ಥಿತಿ ಭಂಗಂ ಸಮಾಸಾರ್ಥತಃ'
\end{shloka}

-ಎಂದು ಹೇಳಿರುವುದರಿಂದ ಅದನ್ನು ಇಟ್ಟುಕೊಂಡು ಪುಷ್ಪದಂತನು `ಎಲೈ ಭಗವಂತನೇ ನಿನ್ನ ಸಾಮರ್ಥ್ಯ ಜನ್ಮ ಭಂಗಕ್ಕೆ ಕಾರಣವಾಗಿರುವುದು' ಎಂದನು. ಶ್ಲೋಕದಲ್ಲಿ `ಯತ್ತತ್' ಎನ್ನುತ್ತಾನೆ. `ಅವನೇ ಇವನು' ಎಂದರೆ ಏನು ಅರ್ಥ? ಭಗವಂತನನ್ನು ಪ್ರತ್ಯಕ್ಷವಾಗಿ ನೋಡಬಹುದೆಂದು ಅರ್ಥ. `ಯತ್ತತ್' ಎಂದರೆ ಮೊದಲು ಯಾವುದು ಇದ್ದೀತೋ `ಅದೇ ಆಗುವುದು, ಯಾರು ಹಾಗೆ ನೋಡಿದುದು?

`ಧ್ಯಾನಾಭ್ಯಾಸೇನ ವಶೀಕೃತೇನ ಮನಸಾ ತನ್ನಿರ್ಗುಣಂ ನಿಷ್ಕ್ರಯಮ್' (ಧ್ಯಾನಾಭ್ಯಾಸದಿಂದ ವಶೀಕೃತವಾದ ಮನಸ್ಸಿನಿಂದ ಯೋಗಿಗಳು ಆ ನಿರ್ಗುಣವು ನಿಷ್ಕ್ರಯವೂ ಆದ.....)

-ಎಂದು ಹೇಳುವಂತೆ, ಯಾರು ಯೋಗಾಭ್ಯಾಸ ಮತ್ತು ಭಗವಂತನ ಉಪಾಸನೆಯನ್ನು ಚೆನ್ನಾಗಿ ಮಾಡಿ ತಮ್ಮ ಬಾಳನ್ನು ಲಕ್ಷ್ಯವುಳ್ಳೆದ್ದಾಗಿ ಮಾಡಿಕೊಳ್ಳಬೇಕೆನ್ನುವ ತೀರ್ಮಾನದೊಡನೆ ಬಾಳನ್ನು ನಡೆಸುತ್ತಾರೆಯೋ ಅಂಥವರಿಗೆ ಭಗವಂತನ ಸಾಕ್ಷಾತ್ಕಾರವಾಗುತ್ತದೆ.

ಭಗವಂತನು ಗೀತೆಯಲ್ಲಿ,

\begin{shloka}
`ತೇಷಾಮೇವಾನುಕಂಪಾರ್ಥಮಹಮಜ್ಞಾನಜಂ ತಮಃ |\\
ನಾಶಯಾಮ್ಯಾತ್ಮ ಭಾವಸ್ಥೋ ಜ್ಞಾನದೀಪೇನ ಭಾಸ್ವತಾ ||'
\end{shloka}

ಅವರನ್ನು ಅನುಗ್ರಹಿಸಲು ಅವರ ಅಂತಃಕರಣದಲ್ಲಿದ್ದು ನಾನು ಸ್ವತಃ ಅವರ ಅಜ್ಞಾನ ಜನಿತವಾದ ಅಂಧಕಾರವನ್ನು ಪ್ರಕಾಶಮಯ ತತ್ತ್ವರೂಪ ದೀಪದ ಮೂಲಕ ಹೋಗಲಾಡಿಸುತ್ತೇನೆ ಎಂದು ಹೇಳಿದನು. ಭಗವಂತನು ಅಂಥವರ ಹತ್ತಿರ ಪ್ರತ್ಯಕ್ಷವಾಗಿ ಇದ್ದುಬಿಡುತ್ತಾನಂತೆ. ಇದಕ್ಕೆ ಒಂದು ಉದಾಹರಣೆಯನ್ನು ನೋಡಬಹುದು. ಅರ್ಜುನನು ಪರಮಾತ್ಮನನ್ನು ನೋಡುತ್ತಿದ್ದರೂ ಪರಮಾತ್ಮನೆನ್ನುವ ನೆನಪೇ ಅವನಿಗೆ ಬರಲಿಲ್ಲ. ಅದು ಕೃಷ್ಣ ಪರಮಾತ್ಮನಿಗೆ ಗೊತ್ತು.

\begin{shloka}
`ಸಖೇತಿ ಮತ್ವಾ ಪ್ರಸಭಂ ಯದುಕ್ತಂ\\
ಹೇ ಕೃಷ್ಣ ಹೇ ಯಾದವ ಹೇ ಸಖೇತಿ
\end{shloka}

(ಸ್ನೇಹಿತನೆಂದು ತಿಳಿದು `ಎಲೈ ಕೃಷ್ಣನೇ ಎಲೈ ಯಾದವನೇ, ಎಲೈ ಮಿತ್ರನೇ, ಎಂದು ಹಠಾತ್ತನೇ ಹೇಳಿದುದನ್ನು)

-ಎಂದು ಅರ್ಜುನನಿಂದ ಹೇಳಲ್ಪಟ್ಟಿದ್ದರಿಂದ ಭಗವಂತನನ್ನು ಅವನು ತನ್ನ ಸ್ನೇಹಿತನೆಂದುಕೊಳ್ಳುತ್ತಿದ್ದನೆ ಹೊರತು ಪರಮಾತ್ಮನೆಂದಲ್ಲ, ಆದರೆ ಪರಮಾತ್ಮನಿಗೆ ಇವನನ್ನು ಅನುಗ್ರಹಿಸಬೇಕೆಂದು ತೋರಿತು. ಆದ್ದರಿಂದ,

\begin{shloka}
`ದಿವ್ಯಂ ದದಾಮಿ ತೇ ಚಕ್ಷುಃ ಪಶ್ಯ ಮೇ ಯೋಗಮೈಶ್ವರಮ್''
\end{shloka}

(ನಿನಗೆ ಜ್ಞಾನ ಕಣ್ಣುಗಳನ್ನು ಕೊಡುತ್ತೇನೆ, ನನ್ನ ಈಶ್ವರ ಯೋಗವನ್ನು ನೋಡು) ಎಂದನು.

`ನಾನು ಒಂದು ಕಣ್ಣು ಕೊಡುತ್ತೇನೆ' ಎಂದು ಭಗವಂತನು ಹೇಳಿದನು. ಪರಮೇಶ್ವರನಿಗೆ ಇರುವಂತೆ ಕೊಟ್ಟನೇ ಎಂದರೆ ಯಾರೂ ಅರ್ಜುನನಿಗೆ ಮೂರು ಕಣ್ಣುಗಳಿಂದು ಹೇಳುವುದನ್ನು ನಾವು ಕೇಳಿಲ್ಲ. ಅವನಿಗೆ ಎರಡು ಕಣ್ಣುಗಳೇ ಇದ್ದವು. ಆದರೆ ಭಗವಂತನು ಮಾತ್ರ `ದಿವ್ಯ ಚಕ್ಷುಸ್' ಕೊಡುವುದಾಗಿ ಹೇಳಿದನು.

ಇದಕ್ಕೆ ಅರ್ಥವೇನೆಂದರೆ ಪರಮಾತ್ಮನ ಕರುಣೆಯಾದುದರಿಂದ ಅವನಿಗೆ ಮೂರನೆಯ ಕಣ್ಣಾದ ಜ್ಞಾನದ ಕಣ್ಣು ಬಂದಿತು. ಆಗ ಅರ್ಜುನನಿಗೆ ಭಗವಂತನು ಇಂಥವನೆಂದು ತಿಳಿದುಕೊಳ್ಳಲು ಸಾಧ್ಯವಾಯಿತು, ಮೊದಲು ಅವನು `ಕೃಷ್ಣನು ನನಗಿಂತಲೂ ಬುದ್ಧಿಶಾಲಿ, ನನಗೆ ಯಾವುದಾದರೂ ಆಪತ್ತು ಬಂದರೆ ಆ ಆಪತ್ತಿನಿಂದ ನನ್ನನು ಕಾಪಾಡುತ್ತಾನೆ' ಎನ್ನುವ ತೀರ್ಮಾನದಿಂದ ಮಾತ್ರ ಇದ್ದನು. ಆದರೆ ಭಗವಂತನ ದರ್ಶನವಾದ ಮೇಲೆ-

\begin{shloka}
`ಪಿತಾಽಸಿ ಲೋಕಸ್ಯ ಚರಾಚರಸ್ಯ ತ್ವಮಸ್ಯ ಪೂಜ್ಯಶ್ಚ ಗರುರ್ಗರೀಯಾನ್ |\\
ನ ತ್ವತ್ಸಮೋಽಸ್ತ್ಯಪ್ಯಧಿಕಃ ಕುತೋಽನ್ಯೋ ಲೋಕಾತ್ರಯೇಽಪ್ಯ\\
\hspace{5.3cm} ಪ್ರತಿಮಪ್ರಭಾವ ||
\end{shloka}

(ನೀನೇ ಚರಾಚರ ಲೋಕದ ತಂದೆ. ನಿನಗೆ ಯಾರೂ ಸಮನಾಗಿಲ್ಲದೆ ಇರುವುದರಿಂದ ಶ್ರೇಷ್ಠನಾದ ಗುರುವಾದ ನೀನು ಈ ಪ್ರಪಂಚದಿಂದ ಪೂಜಿಸಲ್ಪಡಲು ಯೋಗ್ಯನಾಗಿದ್ದೇಯೆ, ಅಸಮಪ್ರಭಾವನೇ! ಮೂರು ಲೋಕಗಳಲ್ಲಿಯೂ ನಿನಗಿಂತಲೂ ಶ್ರೇಷ್ಠರಾದರು ಯಾರು ತಾನೇ ಇರಲು ಸಾಧ್ಯ?)

ಈ ಶ್ಲೋಕವನ್ನು ಭಗವಂತನು ಜ್ಞಾನದ ಕಣ್ಣುಕೊಟ್ಟ ಮೇಲೆ ಅರ್ಜುನನು ಹೇಳಿದನು. ಆದ್ದರಿಂದ ಭಗವಂತನನ್ನು ಯಾರು ನೋಡಬಹುದು ಎಂದರೆ ಯಾರು ಭಗವಂತನ ಕೃಪೆಗೆ ಪಾತ್ರನಾಗುತ್ತಾನೋ, ಅವನೇ ನೋಡಬಲ್ಲನು. ಅವನ ಕೃಪೆಗೆ ಯಾರು ಪಾತ್ರನಾಗುತ್ತಾನೋ, ಅವನೇ ನೋಡಬಲ್ಲನು. ಅವನ ಕೃಪೆಗೆ ಯಾರು ಪಾತ್ರರಾದವರು? ಯಾರು ಭಗವಂತನು ಹೇಳಿದಂತೆ ನಡೆದು ಕೊಳ್ಳುತ್ತಾರೋ, ಹಾಗಿರುವವರೇ ಭಗವಂತನ ಕೃಪೆಗೆ ಪಾತ್ರರಾಗುತ್ತಾರೆ. ಆದ್ದರಿಂದಲೇ ಕವಿ `ಯತ್ತತ್' ಎಂದು ಹೇಳಿದನು.

`ಯತ್ತತ್' ಎಂದರೆ ಪ್ರತ್ಯಕ್ಷವಾಗಿಯೇ ನೋಡುವುದು ಎಂದು ಅರ್ಥ. `ಭಗವಂತನ ಐಶ್ವರ್ಯವನ್ನು ಅಂಥ ಕೃಪೆಯನ್ನು ಪಡೆದಿರುವವರು ನೋಡಿರಬಹುದು. ನಾವು ಹೇಗೆ ತಿಳಿದುಕೊಳ್ಳುವುದು? ನಾವು ಪಾಮರರಲ್ಲವೇ' ಎಂದರೆ-

\begin{shloka}
`ತ್ರಯೀ ವಸ್ತು'
\end{shloka}

ಎಂದು ಹೇಳಲ್ಪಟ್ಟಿದೆ.

`ತ್ರಯೀ' ಎಂದರೆ ಋಕ್-ಯಜುಃ-ಸಾಮವೆಂಬ ಮೂರು ವೇದಗಳು. ಈ ಮೂರು `ತ್ರಯಸ್ತ್ರಯೀ' ಎನ್ನಲ್ಪಡುತ್ತವೆ. ತ್ರಯೀ ಯಾವುದೋ, ಭಗವಂತನು ಅದರ ತಾತ್ಪರ್ಯವಾಗಿರಬೇಕೆಂದು ಶ್ಲೋಕದಲ್ಲಿ ಹೇಳಿದೆ. ವೇದ ಒಂದು ಕಡೆ,

\begin{shloka}
`ಅಜಾಯಮಾನೋ ಬಹುಧಾ ವಿಜಾಯತೇ'\\
(ಹುಟ್ಟದೆ ಹಲವು ವಿಧವಾಗಿ ತೋರುತ್ತಾನೆ.)
\end{shloka}

-ಎಂದು ಹೇಳಿದೆ. ಅವತಾರ ಭಗವಂತನಿಗೆ ಇದೆಯೇ ಇಲ್ಲವೇ ಎಂದರೆ, ಹೀಗೆ ವೇದ ಹೇಳಿದೆ. ಯಾರೋ ಒಬ್ಬರು ನನ್ನ ಬಳಿಗೆ ಬಂದು, `ಈ ವಾಕ್ಯ ಪುರುಷಸೂಕ್ತದಲ್ಲಿ ತಾನೇ ಬರುವುದು. ವೇದದಲ್ಲಿ ಎಲ್ಲಿ ಹೇಳಿದೆ?' ಎಂದು ಕೇಳಿದರು. ನಾನು ಅವರಿಗೆ `ಪುರುಷಸೂಕ್ತವೂ ವೇದವೂ ಒಂದೇ' ಎಂದು ಹೇಳಿದೆನು. ವೇದದ ಬ್ರಾಹ್ಮಣ ಭಾಗದಲ್ಲಿ ಅದು ಇದೆ. ಯಜುರ್ವೇದದಲ್ಲಿ ಅರಣ್ಯಕದಲ್ಲಿ ಬರುತ್ತದೆ. ಆರಣ್ಯಕ ಬ್ರಾಹ್ಮಣ ಭಾಗಕ್ಕೆ ಸೇರಿದುದು. ಋಗ್ವೇದದಲ್ಲೂ ಇದೇ ವಾಕ್ಯ ಬರುತ್ತದೆ. ಆದ್ದರಿಂದ ಅನಾದಿಯಾಗಿರುವ ವೇದವೇ, `ಪರಮಾತ್ಮ'ನಿದ್ದಾನೆಂದು ಹೇಳಿದೆ.

\begin{shloka}
`ಅಣೋರಣೀಯಾನ್ ಮಹತೋ ಮಹೀಯಾನ್'
\end{shloka}

ಎಂದು ವೇದ ಹೇಳುತ್ತದೆ.

ಭಗವಂತನು ಎಂತಹವನು? ಬಹಳ ಸೂಕ್ಷ್ಮವಾಗಿರುವವನೆಂದು ಹೇಳಲ್ಪಟ್ಟಿದೆ. ಯಾವವಸ್ತುವನ್ನು ಅಣುವೆಂದುಕೊಳ್ಳುತ್ತಿದ್ದೇವೋ ಅದಕ್ಕಿಂತಲೂ ಬಹಳ ಸೂಕ್ಷ್ಮವಾಗಿರುವವನು ಭಗವಂತನು. ನಾವು ಆಕಾಶ ಬಹಳ ದೊಡ್ಡದೆಂದು ಕೊಂಡಿದ್ದರೆ 

\begin{shloka}
`ಪಾದೋಽಸ್ಯ ವಿಶ್ವಾ ಭೂತಾನಿ ತ್ರಿಪಾದಸ್ಯಾಮೃತಂ ದಿವಿ\\
ತ್ರಿಪಾದೂರ್ಧ್ವ ಉದೈತ್ಪುರುಷಃ‌ |'
\end{shloka}

ಪ್ರಪಂಚವೆಲ್ಲಾ ಪರವಸ್ತುವಿನ ಒಂದು ಭಾಗವೇ ಆಗಿವೆ. ಪರವಸ್ತುವಿನ ಮುಕ್ಕಾಲು ಭಾಗ ನಾಶವಾಗಿದೆ ಇದ್ದು ಸ್ವಪ್ರಕಾಶನಾದ ಪರಮಾತ್ಮನ ಹತ್ತಿರ ಪ್ರಕಾಶಿಸುತ್ತವೆ.) -ಎಂದು ವೇದದಲ್ಲಿಯೂ,

\begin{shloka}
`ವಿಷ್ಟಭ್ಯಾಹಮಿದಂ ಕೃತ್ಸ್ನಂ ಏಕಾಂಶೇನ ಸ್ಥಿತೋ ಜಗತ್'\\
(ನನ್ನ ಒಂದು ಅಂಶದಿಂದ ಪ್ರಪಂಚವನ್ನೆಲ್ಲಾ ನಾನು ಧರಿಸುತ್ತೇನೆ.)
\end{shloka}

-ಎಂದು ಗೀತೆಯಲ್ಲಿಯೂ ಹೇಳಲ್ಪಟ್ಟಿದೆ. ಅವನ ಒಂದು ಚಿಕ್ಕ ಭಾಗ ಪ್ರಪಂಚವಾಗಿ ವ್ಯಾಪ್ತವಾಗಿರುವುದು. ಅದಕ್ಕೆ ಮೇಲೂ ಅವನು ಇದ್ದಾನೆ. ಪ್ರಪಂಚದಿಂದ ನೋಡಿದರೇನೇ `ಮೇಲೆ, ಕೆಳಗೆ' ಎನ್ನುವ ವ್ಯವಹಾರವೆಲ್ಲಾ ಬರುತ್ತದೆ. ಆದರೆ ಒಬ್ಬನೇ ಇರುವುದು. ಹಾಗಿರುವ ಪರವಸ್ತು ತ್ರಯೀ ವಸ್ತುವಿನಿಂದ ಪಡೆಯಲಾಗುವುದು. ಮೂರು ವೇದಗಳಿಗೂ ತಿಳಿಯಲ್ಪಡುವವನಾಗಿರುವವನು ಭಗವಂತನೇ. ಆದರೆ, ಹಾಗೆ ಒಂದೇ ವಸ್ತುವಾಗಿದ್ದರೂ ಕೂಡ,

\begin{shloka}
`ವ್ಯಸ್ತಂ ತಿಸೃಷು ಗುಣಭಿನ್ನಾಸು ತನುಷು'
\end{shloka}

ಎಂದು ಹೇಳಲ್ಪಟ್ಟಿದೆ.

ಭಗವಂತನಿಗೆ `ಸತ್ಯಂ ಜ್ಞಾನಂ ಅನಂತಂ ಬ್ರಹ್ಮ' ಎನ್ನುವುದೇ ಲಕ್ಷಣ.

\begin{shloka}
`ಅಜೋಽಪಿ ಸನ್ನವ್ಯಯತ್ಮಾ ಭೂತಾನಾಮೀಶ್ವರೋಽಪಿ ಸನ್ |\\
ಪ್ರಕೃತಿಂ ಸ್ವಾಮಧಿಷ್ಠಾಯ ಸಂಭವಾಮ್ಯಾತ್ಮಮಾಯಯಾ ||'
\end{shloka}

(ನಾನು ಹುಟ್ಟು ಇಲ್ಲದವನು, ಜೀವರುಗಳಿಗೆಲ್ಲಾ ಈಶ್ವರನು. ಆದರೂ ನನ್ನ ಪ್ರಕೃತಿಯನ್ನು ವಶಪಡಿಸಿಕೊಂಡು ಆತ್ಮಮಾಯೆಯಿಂದ ಅವತರಿಸುತ್ತೇನೆ).


ಭಗವಂತನಿಗೆ ಜನ್ಮವಿಲ್ಲ. ಆದರೂ ಜನ್ಮವನ್ನು ತೆಗೆದುಕೊಳ್ಳುತ್ತಾನೆ. ನಾವು ಜನ್ಮವನ್ನು ಪಡೆಯುತ್ತೇವೆ. ಹಾಗಾದರೆ ಏನು ವ್ಯತ್ಯಾಸ? ಅವನು ಇಷ್ಟಪಟ್ಟಾಗ ಜನ್ಮವನ್ನು ತೆಗೆದುಕೊಳ್ಳುತ್ತಾನೆ. ನಾವಾದರೋ, ನಮ್ಮ ಕರ್ಮ ನಮ್ಮನ್ನು ಹೇಗೆ ತಳ್ಳುತ್ತದೆಯೋ ಹಾಗೆ ಜನ್ಮವನು ಪಡೆಯುತ್ತೇವೆ.

\begin{shloka}
`ತತೈಹ ರಮಹೀಯ ಚರಣಾಃ..........ರಮಣೀಯಾಂ ಯೋನಿಂ\\
ಅಪದ್ಯೇರನ್..........ಕಪೂಯ ಚರಣಾ ಕಪೂಯಂ ಯೋನಿಮಾ\\
ಪದ್ಯೇರನ್'
\end{shloka}

(ಹೇಗೆ ಇಲ್ಲಿ ಒಳ್ಳೆಯ ಕೆಲಸಗಳನ್ನು ಮಾಡುತ್ತಾರೆಯೋ, ಒಳ್ಳೆಯ ಜನ್ಮಗಳನ್ನು ಪಡೆಯುತ್ತಾರೆಯೋ, ಹಾಗೆಯೇ ಕೆಟ್ಟ ಕೆಲಸಗಳನ್ನು ಮಾಡುವವರು ಕೆಟ್ಟ ಜನ್ಮಗಳನ್ನು ಪಡೆಯುತ್ತಾರೆ.)

ಎಂದು ಹೇಳಿದೆ,

\begin{shloka}
`ಉಭ್ಯಾಭ್ಯಾಂ ಪುಣ್ಯಪಾಪಾಭ್ಯಾಂ ಮಾನುಷ್ಯಂ ಲಭತೇ ವಶಃ'
\end{shloka}

(ಪುಣ್ಯ, ಪಾಪ ಇವೆರಡನ್ನೂ ಮಾಡಿದರೆ ಮನುಷ್ಯ ಜನ್ಮ ಉಂಟಾಗುತ್ತದೆ)

ನಾವು ಮಾಡಿದ ಪಾಪಕ್ಕೆ ತಕ್ಕಂತೆ ಜನ್ಮ ಬರುತ್ತದೆ. ಪುಣ್ಯವನ್ನು ಮಾಡಿದರೆ ಅದಕ್ಕೆ ತಕ್ಕಂತೆ ಜನ್ಮ ಬರುತ್ತದೆ.

ಭಗವಂತನು ತನ್ನ ಶಕ್ತಿಯಿಂದ, ತಾನು ಹುಟ್ಟಿದಂತೆ ತೋರಿಸಬಹುದು. ಆದರೆ ಅವನಿಗೆ ಹುಟ್ಟುವಿಕೆ ಇಲ್ಲ. `ನನಗೆ ಜನ್ಮವಿಲ್ಲದಿದ್ದರೂ ನಾನು ಹುಟ್ಟುತ್ತೇನೆ' ಎಂದು ಭಗವಂತನು ಗೀತೆಯಲ್ಲಿ ಹೇಳಿರುವುದರ ಅರ್ಥವೇನು? ಭಗವಂತನಿಗೆ ನಿಜಕ್ಕೂ ಜನ್ಮವಿಲ್ಲವೆಂದು ಅರ್ಥ.

ಭಗವತ್ಪಾದರು,

\begin{shloka}
`ದೇವಕ್ಯಾಂ ವಸುದೇವಾದಂಶೇನ ಜಾತ ಇವ'\\
(ದೇವಕಿ-ವಸುದೇವರ ಮೂಲಕ ಹುಟ್ಟಿದಂತೆ)
\end{shloka}

ಎನ್ನುವಲ್ಲಿ `ಹುಟ್ಟುವಂತೆ' ಎಂದು ಹೇಳಿದ್ದಾರೆ.

`ಲೋಕಾನುಗ್ರಹಂ ಕುರ್ವನ್' ಎಂದು ಹೇಳಿರುವಂತೆ `ಇಷ್ಟು ಜನರು ಆತ್ಮ ತತ್ತ್ವವನ್ನು ತಿಳಿಯದೆ ತಳಮಳಿಸುತ್ತಿದ್ದಾರಲ್ಲಾ ಎಂದುಕೊಂಡು, ಇವರಿಗೆ ಹೇಗಾದರೂ ಉಪದೇಶ ಮಾಡಿದರೆ ಇದರ ಕಿವಿಗೆ ಮುಟ್ಟುತ್ತದೆಯೋ ಎಂದುಕೊಂಡು ದಯೆಯಿಂದ ಭಗವಂತನು ಅನುಗ್ರಹ ಮಾಡಲು ಬಂದನು' ಎಂದು ಭಗವತ್ಪಾದರು ಹೇಳಿದ್ದಾರೆ.


ಭಗವಂತನು ಸೃಷ್ಟಿಸುವಾಗ ಬ್ರಹ್ಮನೆಂದು ಹೆಸರಿಟ್ಟುಕೊಂಡು ಸೃಷ್ಟಿಸಿದನು. ರಕ್ಷಣೆ ಮಾಡುವಾಗ ವಿಷ್ಣು ಎಂದು ಹೆಸರು ಇಟ್ಟುಕೊಂಡು ರಕ್ಷಣೆ ಮಾಡುತ್ತಾ ಬರುತ್ತಾನೆ. ನಾಶ ಮಾಡುವಾಗ ರುದ್ರನಾಗಿದ್ದು ನಾಶಮಾಡುತ್ತಾನೆ. ಶಕ್ತಿ ಎನ್ನುವುದು ಒಂದೇ ಒಂದು ಚೈತನ್ಯ. ಅದರ ಶಕ್ತಿಯಿಂದ ಯಾವ ಕಾಲದಲ್ಲಿ ಯಾವ ರೂಪವನ್ನು ಧರಿಸಿದರೆ ಕೆಲಸವಾಗುತ್ತದೆಯೋ ಆ ಕಾಲದಲ್ಲಿ ಆ ರೂಪವನ್ನು ಧರಿಸುವನು. ಶಿವನಲ್ಲಿ ಮಾತ್ರ ನಂಬಿಕೆ ಇರುವವನು ಬ್ರಹ್ಮ, ವಿಷ್ಣು ಈ ಎರಡು ಹೆಸರುಗಳನ್ನು ಒಪ್ಪುವುದಿಲ್ಲ.


\begin{shloka}
`ಬಹುರಜಸೇ ವಿಶ್ವೋತ್ಪತ್ತೌ ಭವಾಯ ನಮೋ ನಮಃ\\
ಪ್ರಬಲ ತಮಸೇ ತತ್ಸಂಹಾರೇ ಹರಾಯ ನಮೋನಮಃ‌ |\\
ಜನಸುಕೃತೇ ಸತ್ತ್ವೋದ್ರಿಕ್ತೌ ಮೃಡಾಯ ನಮೋ ನಮಃ\\
ಪ್ರಮಹಸಿ ಪತೇ ನಿಸ್ತ್ರೈಗುಣ್ಯೇಶಿವಾಯ ನಮೋ ನಮಃ ||'
\end{shloka}

(ಪ್ರಪಂಚವನ್ನು ಸೃಷ್ಟಿ ಮಾಡುವುದಕ್ಕಾಗಿ ರಜಸ್ಸನ್ನು ಹೆಚ್ಚಾಗಿಸಿಕೊಂಡಿರುವವನಿಗೆ ನಮಸ್ಕಾರ. ಪ್ರಪಂಚವನ್ನು ನಾಶಮಾಡುವುದಕ್ಕಾಗಿ ತಮಸ್ಸನ್ನು ಹೆಚ್ಚಾಗಿಸಿಕೊಂಡಿರುವ ಹರನಿಗೆ ನಮಸ್ಕಾರ, ಜನರಿಗೆ ಆನಂದವನ್ನು ನೀಡಲು ಸತ್ತ್ವವನ್ನು ಹೆಚ್ಚಾಗಿರಿಸಿಕೊಂಡಿರುವ ಮೃಡನಿಗೆ ನಮಸ್ಕಾರ, ಮೂರು ಗುಣಗಳಿಗೂ ಮೇಲೆ ಪ್ರಕಾಶಿಸುವ ಶಿವನಿಗೆ ನಮಸ್ಕಾರ.)


`ಒಬ್ಬನೇ ಶಿವನು ಸೃಷ್ಟಿ ಮಾಡುವಾಗ `ಭವ'ನೆಂದು ಹೆಸರು ಇಟ್ಟುಕೊಳ್ಳುತ್ತಾನೆ. ಅವನೇ ಕಾಪಾಡುವಾಗ `ಮೃಡ' ನೆಂದು ಹೆಸರು ಇಟ್ಟುಕೊಳ್ಳುತ್ತಾನೆ. ಸಂಹಾರ ಮಾಡುವಾಗ `ಹರ'ನೆಂದು ಹೆಸರು ಇಟ್ಟುಕೊಳ್ಳುತ್ತಾನೆ. ಆದರೆ ಅವನು ಯಾವಾಗಲೂ `ಶಿವನೇ' ಎಂದು ಶೈವನು ಹೇಳುತ್ತಾನೆ. `ಶಿವ' ಎಂದರೆ ಈ ಮೂರು ರೂಪಗಳೂ ಬರುತ್ತವೆ. `ಶಿವ' ಎಂದರೆ ಬ್ರಹ್ಮ. ಸಂಸ್ಕೃತದಲ್ಲಿ `ಶಿವ' ಶಬ್ದಕ್ಕೆ ಪುಲ್ಲಿಂಗವೂ ಉಂಟು, ನಪುಂಸಕಲಿಂಗವೂ ಉಂಟು. ನಪುಂಸಕಲಿಂಗದಲ್ಲಿ ಅದನ್ನು ಉಪಯೋಗಿಸಿದರೆ ಅದಕ್ಕೆ ಮಂಗಳವೆಂದು ಅರ್ಥ. ಮಂಗಳದಲ್ಲಿ ಮಂಗಳವಾಗಿ `ಮಂಗಲಾನಾಂ ಚ ಮಂಗಲಂ' ಎನ್ನುವಂತೆ ಇರುವ ಪರಬ್ರಹ್ಮ ಒಂದೇ ಮಂಗಳ. ಯಾವುದಾದರೂ ರೂಪವನ್ನು ಇಟ್ಟುಕೊಂಡರೇ ಆಗ `ಶಿವಃ; ಎಂದು ಹೇಳಬೇಕು. ಆಗ ಯಾವ ರೂಪವನ್ನು ಬೇಕಾದರೂ ಪಡೆಯಬಹುದು. ಹಾಗಲ್ಲದೆ ನಪುಂಸಕ ಬಂದರೆ `ಒಂದು ವಿಧವಾದ ಗುಣವೂ ಇಲ್ಲ' ಎಂದು ಅರ್ಥ. ಆದ್ದರಿಂದಲೇ `ಬ್ರಹ್ಮ' ಎಂದರೆ `ಚತುರ್ಮುಖ' ಬ್ರಹ್ಮನಿಗೆ ಹೆಸರು. `ಬ್ರಹ್ಮಜಿಜ್ಞಾಸಾ' ಎನ್ನುವ ಜಾಗದಲ್ಲಿ ನಿರ್ಗುಣಬ್ರಹ್ಮವನ್ನೇ ಅದು ಸೂಚಿಸುತ್ತದೆ. ಪುಲ್ಲಿಂಗವಾದ ಬ್ರಹ್ಮ ಶಬ್ದವನ್ನು ತೆಗೆದುಕೊಂಡರೆ ಅದು ನಿರ್ಗುಣವಲ್ಲ. ಅದು ಗುಣವೇ. ನಿರ್ಗುಣವಾದ ಬ್ರಹ್ಮ ಶಬ್ದವನ್ನು ತೆಗೆದುಕೊಂಡು ನಾವು `ಬ್ರಹ್ಮಣೋ ಜಿಜ್ಞಾಸಾ' (ಬ್ರಹ್ಮವನ್ನು ಕುರಿತು ತಿಳಿದುಕೊಳ್ಳೆಬೇಕೆನ್ನುವ ಆಸೆ) ಎಂದು ಹೇಳುತ್ತೇವೆ. ಆದರೆ ಒಂದು ವಸ್ತು ಆಯಾಕಾಲದಲ್ಲಿ ಅವತಾರವನ್ನು ಪಡೆದು ಜನರನ್ನು ಅನುಗ್ರಹಿಸುತ್ತದೆ. ಅವತಾರ ಮೂರೇ ಇರಬೇಕೆಂಬ ನಿಯಮವಲ್ಲ. ಅದು ಮೂವತ್ತು ಆಗಬಹುದು, ಮೂವತ್ತು ಲಕ್ಷವಾಗಬಹುದು ಅಥವಾ ಮೂರು ಕೋಟಿ ಆಗಬಹುದು.

\begin{shloka}
`ಯದ್ಯದ್ವಿಭೂತಿಮತ್ಸತ್ತ್ವಂ ಶ್ರೀಮದೂರ್ಜಿತಮೇವ ವಾ‌|\\
ತತ್ತದೇವಾವಗಚ್ಛ ತ್ವಂ ಮಮ ತೇಜೋಂಶ ಸಂಭವಮ್ ||'
\end{shloka}

(ಮಹಿಮೆಯೂ, ಸೌಂದರ್ಯವೂ, ಶಕ್ತಿಯೂ ಇರುವುದು ಯಾವುದೆಲ್ಲವೂ ಇದೆಯೋ, ಅದೆಲ್ಲವೂ ನನ್ನ ತೇಜಸ್ಸಿನ ಅಂಶದಿಂದ ಉಂಟಾದುದೆಂದು ತಿಳಿ.)- ಎನ್ನುವಲ್ಲಿ,

`ವಿಶೇಷವಾದ ಕ್ರಿಯಾಶಕ್ತಿ ಯಾವುದು ಇದ್ದರೂ ಅದು ನನ್ನ ಅಂಶ' ಎಂದು ಭಗವಂತನು ಹೇಳಿದ್ದಾನೆ, ಹಾಗಿರುವಾಗ ರಾಮ, ಕೃಷ್ಣ, ನರಸಿಂಹ ಈ ಅವತಾರಗಳೆಲ್ಲವೂ ಭಗವಂತನ ರೂಪಗಳೇ ಆಗುತ್ತವೆ. ಆದ್ದರಿಂದಲೇ, `ಭಗವಂತನ ರೂಪಗಳು ಮೂರು' ಎಂದು ಇಟ್ಟುಕೊಳ್ಳಬೇಕಾಗಿಲ್ಲ. ಅವುಗಳು ಅನೇಕ' ವೆಂದು ಹೇಳಿ ಅಲ್ಲಿ ವ್ಯಾಖ್ಯಾನ ಮಾಡಲ್ಪಟ್ಟಿದೆ. `ತ್ರಯೀ' ಎಂದು ಹೇಳಿರುವುದರಿಂದ ವೇದಕ್ಕೆ ಭಗವಂತನು ವಿಷಯವಾಗಿದ್ದಾನೆಂದು ಹೇಳಲ್ಪಟ್ಟಿದೆ. ಹಲವರು ವೇದಗಳನ್ನು ಓದುತ್ತಾರೆ. ಹಾಗಿರುವಾಗ ಭಗವಂತನ ಐಶ್ವರ್ಯ ಇಲ್ಲವೆಂದು ಹೇಗೆ ಹೇಳುವುದು? ಈ ಪ್ರಶ್ನೆಗೆ ಒಂದು ಉತ್ತರವಿದೆ. ವೇದ ಹೇಳುವುದನ್ನು ಪ್ರಮಾಣವೆಂದು ಇಟ್ಟುಕೊಳ್ಳುವವನು ತಾನು ಅದನ್ನು  ಸ್ವೀಕರಿಸಬಹುದು. ವೇದವೇ ಅಪ್ರಮಾಣ ಎನ್ನುವವನ ಹತ್ತಿರ ವೇದ ಹೇಳುವುದು ಸರಿ ಎಂದು ಹೇಗೆ ಹೇಳುವುದು? ಪ್ರಮಾಣವೆಂದು ಒಪ್ಪಿಕೊಳ್ಳದವನಿಗೆ ಅದರ ಪ್ರಕಾರ ದಂಡನೆ ಕೊಡಲು ಸಾಧ್ಯವೇ? ಆದರೆ ಪ್ರಪಂಚದಲ್ಲಿ ತಪ್ಪು ಮಾಡಿದವರಿಗೆಲ್ಲಾ ದಂಡನೆ ಕೊಡುವೆವು. ಆ ವಿಷಯವೇ ಬೇರೆ. `ದೈವ ನಿಂತು ಕೊಲ್ಲುವುದು' ಎಂದು ಗಾದೆ ಇದೆ. ನಾವು ಭಗವಂತನು ಏನು ಮಾಡಬಲ್ಲನು ಎನ್ನುವ ಭಾವನೆಯಲ್ಲಿದ್ದೇವೆ. ಆದರೆ ಭಗವಂತನು ಯಾರಿಗೆ ಯಾವ ಕಾಲದಲ್ಲಿ ಏನು ದಂಡನೆ ಕೊಡಬೇಕೋ, ಅದನ್ನು ಕೊಡುತ್ತಾನೆ.

`ತ್ರಯೀ ವಸ್ತು' ಎಂದು ನೀವು ಹೇಳಿದರೂ ನಾನು ಒಪ್ಪಲಾಗುವುದಿಲ್ಲ. ಯೋಗಿಗಳು ಪ್ರತ್ಯಕ್ಷವೆಂದು ಹೇಳಿದರೂ ಯೋಗಿಗಳನ್ನು ನೋಡಿದವರು ಯಾರು? ಯೋಗಿಯೂ ನನ್ನಂತೆಯೇ ಎಂದು ನಾನು ಭಾವಿಸುತ್ತೇನೆ. ಅವನು ಒಬ್ಬ ಇಂದ್ರಜಾಲ ಮಾಡುವವನೆಂದು ನಾನು ಭಾವಿಸುತ್ತೇನೆ. ಇಂದ್ರಜಾಲದಲ್ಲಿ ಮಾಡಲ್ಪಡುವದನ್ನೆಲ್ಲಾ ಕಂಡು ಯಾರಾದರೂ ಅವನ ಹಿಂದೆ ಓಡುವುದುಂಟೇ? ಅದೇ ರೀತಿ ನಾನು ಈಶ್ವರನನ್ನು ಒಪ್ಪಿಕೊಳ್ಳಲು ಸಾಧ್ಯವಿಲ್ಲ, ಏಕೆಂದರೆ ನಾನು ಯುಕ್ತಿಯನ್ನು ಹೇಳುವವನು. ನಾನು ಈಶ್ವರನನ್ನು ಕುರಿತು ಎಷ್ಟೋ ಚಿಂತನೆ ಮಾಡಿದ್ದೇನೆ. ಹೇಗೆ ನೋಡಿದರೂ ಈಶ್ವರನು ಇಲ್ಲ ಎನ್ನುವ ಕೊನೆಯ ತೀರ್ಮಾನಕ್ಕೆ ಬರಬೇಕಾಗಿದೆ' ಎಂದು ನಾಸ್ತಿಕನು ಹೇಳುತ್ತಾನೆ.

ಪ್ರಪಂಚದಲ್ಲಿ ನಾಸ್ತಿಕವಾದಗಳು ಇವೆ. ಅವುಗಳು ಕೆಲವರಿಗೆ ರಮಣೀಯವಾಗಿ ತೋರುತ್ತವೆ. ಯಾರಿಗೇ? ಭಾಗ್ಯವಿಲ್ಲದವನಿಗೆ ವಾದಗಳು ರಮಣೀಯವಾಗಿ ತೋರುತ್ತವೆ. ಮಕ್ಕಳಿಗೆ ನಶ್ಯ ಕೊಟ್ಟರೆ ಅವರಿಗೆ `ಆಹಾ! ಭೇಷ್, ನಾನೂ ದೊಡ್ಡವನಾಗಿಬಿಟ್ಟೆ' ಎಂದು ತೋರುತ್ತದಂತೆ. ಚಿಕ್ಕವನು ನಶ್ಯಹಾಕಿಕೊಂಡರೆ `ಅದು ದೊಡ್ಡವರು ಮಾಡುವ ಕೆಲಸ. ಆದ್ದರಿಂದ ನಾನೂ ದೊಡ್ಡವನು' ಎಂದು ಭಾವಿಸುತ್ತಾನೆ. ಹಾಗೆಯೇ ಚುಟ್ಟ ಸೇದಿದರೆ, `ಅದು ದೊಡ್ಡವರಿಗಾಗಿ ಇರುವುದು. ಅದನ್ನು ಸೇದಿದರೆ ನಾನು ದೊಡ್ಡವನಾಗಿ ಬಿಡುವೆನು' ಎನ್ನುವ ಭಾವನೆಯಿಂದ ನಾಲ್ಕು ದಿನಗಳು ಆ ಕೆಲಸವನ್ನು ಮಾಡುವನು. ಅದಾದ ಮೇಲೆ ನೋಡಬೇಕು ಅವನ ಅವಸ್ಥೆ. ನಶ್ಯ ಇಲ್ಲದಿದ್ದರೆ ಐದು ಪೈಸೆಯನ್ನಾದರೂ ಕದ್ದು ನಶ್ಯ ಹಾಕಿಕೊಳ್ಳಬೇಕೆನಿಸುತ್ತದೆ. `ಏನು ಹೀಗೆ ಆಗಿ ಬಿಟ್ಟೆಯಲ್ಲಾ' ಎಂದು ಕೇಳಿದರೆ `ಕೆಟ್ಟ ಅಭ್ಯಾಸದಲ್ಲಿ ಬಿದ್ದಾಯಿತು. ಇದನ್ನು ಇನ್ನು ಬಿಡಲು ಸಾಧ್ಯವಿಲ್ಲ' ಎಂದು ಅವನು ಹೇಳುವನು.

\begin{shloka}
`ಅಭವ್ಯಾನಾಮಸ್ಮಿನ್ ವರದ ರಮಣೀಯಾಮರಮಣೀಮ್ |\\
ವಿಹನ್ತುಂ ವ್ಯಾಕ್ರೋಶೀಂ ವಿದಧಥ ಇಹೈಕೇ ಜಡಧಿಯಃ ||'
\end{shloka}

ಎಂದು ಹೇಳುವಂತೆ, ನಾಸ್ತಿಕವಾದವನ್ನು ಹಲವರು ಹೇಳುತ್ತಿದ್ದರೆ. ಅವರು `ಜಡಧೀಯಃ' (ಮಂದಬುದ್ಧಿಯುಳ್ಳವರು) ಆಗುವರು.

ಭಗವಂತನ ಐಶ್ವರ್ಯವನ್ನು ತಿಳಿದುಕೊಳ್ಳದೆ ಅದನ್ನು ನಿರಾಕರಿಸುವುದಕ್ಕಾಗಿ ಭಗವಂತನು ಇಲ್ಲವೆಂದು ನಿರ್ಣಯಿಸುತ್ತಾನೆ. ಅದರಿಂದ ಏನು ಪ್ರಯೋಜನ?

\begin{shloka}
`ನಾಸ್ತಿ ಚೇನ್ನಾಸ್ತಿ ನೋ ಹಾನಿಃ\\
ಅಸ್ತಿಚೇನ್ನಾಸ್ತಿಕೋ ಹತಃ'
\end{shloka}

(ಇಲ್ಲದಿದ್ದರೆ ಅದರಿಂದ ನಷ್ಟವೇನಿಲ್ಲ, ಇರುವುದಾದರೆ ನಾಸ್ತಿಕನು ನಾಶವಾಗುತ್ತಾನೆ)

-ಎನ್ನುವ ವಾಕ್ಯ ಒಂದು ಕಡೆ ಬರುತ್ತದೆ. ಎಂದೋ ಒಂದು ದಿನ ಗ್ರಹಣ ಬರುತ್ತದೆ. ಅದು ಮಧ್ಯಾಹ್ನದ ಸಮಯದಲ್ಲಿ ಬಂದರೆ ನಾವು ಸುಖವಾಗಿ ಸ್ನಾನಮಾಡಿ ಊಟ ಮಾಡಬಹುದು. ಆದರೆ ನಮ್ಮ ಪ್ರಾರಬ್ದ(?) ಅದು ರಾತ್ರಿ ಹನ್ನೊಂದು ಗಂಟೆಗೆ ಬರುತ್ತದೆ. ಗ್ರಹನಕ್ಕೆ ಸ್ವಲ್ಪ ಹೊತ್ತು ಮುಂಚೆ ಏನೂ ತಿನ್ನಬಾರದು. ಹನ್ನೊಂದು ಗಂಟೆಯಾದ ಮೇಲೆ ಮೂರು ಗಂಟೆ ಕಾಲ ಗ್ರಹಣ ಇರುವುದು. ಅದುವರೆಗೆ ಏನೂ ತಿನ್ನಲು ಸಾಧ್ಯವಿಲ್ಲ. ಜಪ ಮಾಡುತ್ತಾ ಕುಳಿತಿರಬೇಕು. ಕೆಲವರಿಗೆ `ಇದು ಬಹಳ ಕಷ್ಟದ ಕೆಲಸವಾಯಿತಲ್ಲಾ! ಯಾವುದಾದರೂ ಸಿನಿಮಾಗೆ ಹೋಗಿ ಕುಳಿತುಕೊಂಡರೆ ಹೇಗಾದರೂ ಕಾಲವನ್ನು ಕಳೆಯಬಹುದು. ಜಪ ಮಾಡುತ್ತಾ ಕುಳಿತು ಕೊಳ್ಳಬೇಕೆಂದರೆ ದುಃಖವಾಗುತ್ತದಲ್ಲಾ!' ಎಂದು ತೋರುತ್ತದೆ. ನಾಸ್ತಿಕನಾಗಿ ಇರುವವನಿಗೆ `ನಾಸ್ತಿಕವಾದಗಳು ಬಹಳ ಸರಿ' ಎಂದು ತೋರುವುದು. ಎಲ್ಲಿಯವರೆಗೆ ಅವನು ನಾಸ್ತಿಕನಾಗಿರುವನು? ಅವನಿಗೆ ಯಾವುದಾದರೂ ಒಂದು ಕಷ್ಟ ಬರುವವರೆಗೆ ಮಾತ್ರ ಅವನು ನಾಸ್ತಿಕನಾಗಿರುವನು. ಆದರೆ ಸಂಕಟ ಬಂದರೆ ವೆಂಕಟರಮಣನನ್ನು ಕೂಗುವನು. ಹೀಗೆ ನಾವು ಲೋಕದಲ್ಲಿ ನೋಡುತ್ತೇವೆ.

ಒಂದು ಕಡೆ ಉದಯನಾಚಾರ್ಯರು, ಭಗವಂತನೇ, ನೀನು ಎಂದೂ ಈ ನಾಸ್ತಿಕನನ್ನು ರಕ್ಷಿಸುವ ಭಾರವನ್ನು ವಹಿಸಬೇಕಾಗಿಲ್ಲ. ಅವನು ತನ್ನನ್ನು ತಾನು ಕಾಪಾಡಿಕೊಳ್ಳುವನು. ನಿನ್ನ ಚಿಂತೆ ಅವನಿಗೆ ಬೇಕಾಗಿಲ್ಲ. ಅವನ ಚಿಂತೆ ನಿನಗೆ ಬೇಕಾಗಿಲ್ಲ. ಆದರೆ ಯಾವುದೋ ಒಂದು ಸಮಯದಲ್ಲಿ ಅವನು ತನ್ನನ್ನು ತಾನು ಕಾಪಾಡಿಕೊಳ್ಳಲಾರನು. ಆಗ ಮಾತ್ರ ನೀನು ಅವನನ್ನು ಕೈ ಬಿಡಬೇಡ. ಅವನು ನಾಸ್ತಿಕನಾಗಿದ್ದು ನಿನ್ನನ್ನು ನಿಂದಿಸುತ್ತಿದ್ದನು, ಎನ್ನುವುದರಿಂದ ನೀನು ಕೈ ಬಿಡುವುದೇ ನ್ಯಾಯವಲ್ಲವೇ? ಭಕ್ತನನ್ನಲ್ಲವೇ ನೀನು ಕಾಪಾಡಬೇಕು? ಭಕ್ತನಲ್ಲವೇ ಭಗವಂತನಿದ್ದನೆಂದು ನಿನಗೆ ಶರಣಾಗಿದ್ದಾನೆ ಎಂದುಕೊಂಡು ಅವನನ್ನು ನೀನು ಕೈ ಬಿಡಬೇಡ. ಭಕ್ತನು ಭಗವಂತನಿದ್ದಾನೆ ಎನ್ನುವುದಕ್ಕಾಗಿ ಭಗವಂತನ ಚಿಂತೆ ಎಷ್ಟು ಮಾಡಿದನೋ. ಅದೇ ರೀತಿ ನೀನು ಇಲ್ಲ ಎನ್ನುವುದಕ್ಕಾಗಿ ನಾಸ್ತಿಕನು ನಿನ್ನನ್ನು ಕುರಿತು ಚಿಂತಿಸಿದ್ದಾನೆ. ಆದ್ದರಿಂದ ನೀನು ಅವನನ್ನು ಕೈ ಬಿಡಬೇಡ. ಆ ನಿರ್ಭಾಗ್ಯವಂತನಿಗೆ ಬಹಳ ಚೆನ್ನಾಗಿರುವ ಮಾತುಗಳಿಂದ ನಿನ್ನ ಐಶ್ವರ್ಯವಿಲ್ಲವೆಂದು ಹೇಳುವುದಕ್ಕೆ ಅನೇಕ ಮಂದಿ ಪ್ರಯತ್ನ ಮಾಡುತ್ತಿದ್ದಾರೆ' ಎನ್ನುತ್ತಾರೆ.

ಭಗವಂತನು ಇಲ್ಲವೆಂದು ಹೇಗೆ ಹೇಳುವುದು? ನಾವು ಇಷ್ಟು ಜನರನ್ನು ನೋಡಿದ್ದೇವೆ. ಕಾಶ್ಮೀರದಲ್ಲಿರುವ ಮನುಷ್ಯನು ಹೇಗಿದ್ದಾನೆಂದರೆ, ಅವನಿಗೆ ಎರಡು ಕಣ್ಣುಗಳು, ಎರಡು ಕೈಗಳು ಮುಂತಾದವು ಇವೆಯೆಂದು ಹೇಳುತ್ತೇವೆ. ಹಾಗಲ್ಲದೆ ನಾಲ್ಕು ಕಣ್ಣುಗಳು, ಎರಡು ಕೊಂಬುಗಳು ಇವೆಯೆಂದು ಹೇಳಬಲ್ಲವೇ? ನಾವು ಹೇಗೆ ನೋಡಿದ್ದೇವೋ ಹಾಗೆಯೇ ಕಲ್ಪನೆ ಮಾಡಲು ಸಾಧ್ಯ. ಆಸ್ತಿಕರಾಗಿರುವ ನೀವುಗಳು ಈ ಬ್ರಹ್ಮಾಂಡವನ್ನು ಭಗವಂತನೇ ಸೃಷ್ಟಿ ಮಾಡಿದನೆನ್ನುತ್ತೀರಿ.

`ಇದನ್ನು ಒಪ್ಪುತ್ತೇನೆ, ಇದೇ ಕಾರಣದಿಂದಲೇ ಭಗವಂತನು ಇಲ್ಲ ಎನ್ನುವುದು ನನ್ನ ತೀರ್ಮಾನ. ಯಾವಾಗ ಭಗವಂತನು ಸೃಷ್ಟಿ ಮಾಡಿದನೆಂದು ಆಸ್ತಿಕರಾದ ನೀವು ಹೇಳುತ್ತೀರೋ ಆಗಲೇ ಭಗವಂತನು ಇಲ್ಲವೆಂದು ತೀರ್ಮಾನ' ಎನ್ನುತ್ತಾನೆ ನಾಸ್ತಿಕನು. `ನಾವು ಮನುಷ್ಯರು ಹಲವರನ್ನು ನೋಡಿದ್ದೇವೆ. ಕಾಶ್ಮೀರದಲ್ಲಿರುವ ಮನುಷ್ಯರು ನಮ್ಮಂತೆಯೇ ಎರಡು ಕೈಗಳನ್ನೂ, ಕಣ್ಣುಗಳನ್ನೂ ಉಳ್ಳವರೇ ಹೊರತು ನಾಲ್ಕು ಕೈಗಳು, ಎರಡು ಕೊಂಬುಗಳು ಉಳ್ಳವರೆಂದು ಹೇಳುವುದಿಲ್ಲ. ನಾವು ಹೇಗೆ ನೋಡುತ್ತೇವೋ ಹಾಗೆಯೇ ಕಲ್ಪನೆಮಾಡಲು ಸಾಧ್ಯ.

\begin{shloka}
`ದೃಷ್ಟತ್ವ ಅದೃಷ್ಟ ಕಲ್ಪನಮ್'
\end{shloka}

(ನೋಡಿದಂತೆ ನೋಡದದು ಕಲ್ಪಿತವಾಗುವುದು)-ಎಂದು ಒಂದು ಕಡೆ ಬರುತ್ತದೆ. ಹಾಗೆಯೇ ಆಸ್ತಿಕರಾದ ನೀವು ಭಗವಂತನೇ ಪ್ರಪಂಚವನ್ನು ಸೃಷ್ಟಿಮಾಡಿದನೆಂದು ಹೇಳುತ್ತೀರಿ. (ಕ್ರಿಯೆ ಇರುವವನಿಗೆ ಖಂಡಿತ ನಾಶವಿದೆ). ಆದ್ದರಿಂದ `ಭಗವಂತನೇ ಪ್ರಪಂಚವನ್ನು ಸೃಷ್ಟಿಸಿದನು, ಆದ್ದರಿಂದ ಅವನು ಇದ್ದಾನೆ' ಎಂದು ನೀವು ಹೇಳುವುದರ ಆಧಾರದ ಮೇಲೆ ಅವನು ಇಲ್ಲವೆಂದು ನಾನು ನಿರೂಪಿಸುತ್ತೇನೆ' ಎನ್ನುತ್ತಾನೆ ಅವನು.

ಹೀಗೆ ಪರಮಾತ್ಮನೇ ಇಲ್ಲ ಇನ್ನುವ ವಿಷಯದಲ್ಲಿಯೂ ಸುಂದರವಾದ ತರ್ಕವಿದೆ. ಆದರೆ ಅಂಥ ತರ್ಕ ತರ್ಕವಲ್ಲ. ಆ ಕುತರ್ಕದಿಂದ ಏನಾಗುತ್ತದೆಂದರೆ ವಿವೇಕವಿಲ್ಲದ ಜನರು ಮೋಹಗೊಂಡು ಅದನ್ನು ನಿಜವೆಂದುಕೊಳ್ಳುವರು.

ನಾಸ್ತಿಕನ ವಿರೋಧವನ್ನು ಈಗ ನೋಡೋಣ. ಒಬ್ಬ ಕುಂಬಾರನು ಒಂದು ಕೊಡವನ್ನು ಮಾಡುವನು. ಕೊಡ ಮಾಡುವುದು ಎಂದರೆ, `ಎಲೈ ಕೊಡವೇ! ನೀನು ಎದುರಿಗೆ ಬಂದು ನಿಲ್ಲು' ಎಂದರೆ ಅದು ಬಂದು ನಿಲ್ಲುವುದೇ? ನಾನು ಹೀಗೆ ಇದುವರೆಗೆ ನೋಡಿಲ್ಲ. ಗುದ್ದಲಿ ತೆಗೆದುಕೊಂಡು ಹೋಗಿ ಮಣ್ಣನ್ನು ಅಗೆದು ತೆಗೆದು, ನೀರು ಹಾಕಿ ಅದರಲ್ಲಿರುವ ಕಲ್ಲುಗಳನ್ನು ತೆಗೆದು ಅದು ಮೃದುವಾದ ಮೇಲೆ ಕುಂಬಾರನು ಕೊಡವನ್ನು ಮಾಡುವನು. ಅದೇ ರೀತಿ ಈ ಭಗವಂತನೆನ್ನುವವನು ಯಾವುದಾದರೂ ಕ್ರಿಯೆಯನ್ನು ಮಾಡುವುದನ್ನು ನೀವು ನೋಡಿರುವಿರಾ? ಹಾಗೆ ಅವನು ಏನೇನು ಮಾಡಿದನು? ನೀವು ಭಗವಂತನನ್ನು ಕುರಿತು, `ನಿರ್ಗುಣ, ನಿಷ್ಕಲ, ನಿಷ್ಕ್ರಿಯ, ಶಾಂತ -' ಎಂದು ಹೇಳುತ್ತೀರಲ್ಲಾ! ನಿಷ್ಕ್ರಿಯವಾಗಿರುವ ವಸ್ತು ಎಂದು ಹೇಳಿ, ಪ್ರಪಂಚವನ್ನು ಸೃಷ್ಟಿಮಾಡಿದನೆಂದು ಹೇಳುವಿರಲ್ಲಾ ಇವುಗಳಲ್ಲಿ ಒಂದಕ್ಕೊಂದು ಸಂಬಂಧವೇ ಇಲ್ಲವಲ್ಲಾ! ಈಶ್ವರನಿಂದ ಯಾವ ವಿಧವಾದ ಕ್ರಿಯೆಯೂ ನಡೆಯಲು ಸಾಧ್ಯವಿಲ್ಲ. ಅಷ್ಟೇ ಅಲ್ಲ. ಭಗವಂತನ ಸ್ವರೂಪ ಏನು ಎಂದು ಹೇಳುವಿರಾ?


\begin{shloka}
`ಅಕಾಯಮವ್ರಣಮ್'
\end{shloka}

-ಎಂದು ಹೇಳುವಂತೆ ಅವನಿಗೆ ಶರೀರವು ಇಲ್ಲ, ವ್ರಣವೂ ಇಲ್ಲ ಎಂದು ಹೇಳುವಿರಿ. ಶರೀರವಿಲ್ಲದವನು ಹೇಗೆ ಕ್ರಿಯೆಗಳನ್ನು ಮಾಡಲು, ಸಾಧ್ಯ? ಕ್ರಿಯೆಗಳಿಲ್ಲದೆ ವಸ್ತು ಹೇಗೆ ಬರುವುದು? ಒಬ್ಬ ಕುಂಬಾರನಿಗೆ ಶರೀರವಿದೆ. ನಮ್ಮಂತೆಯೇ ಅವನೂ ಒಬ್ಬ ಮನುಷ್ಯ. ನಿಮ್ಮ ಭಗವಂತನಿಗೆ ಕೈ ಇದೆಯೇ? ಕಾಲು ಇದೆಯೇ? ಎಲ್ಲಿದೆ?

\begin{shloka}
ಕಿಂಮೀಹಃ ಕಿಂ ಕಾಯಃ ಸ ಖಲು ಕಿಮುಪಾಯಃ ತ್ರಿಭುವನಂ |\\
ಕಿಮಾಧಾರೋ ಧಾತಾ ಸ್ವಜಾತಿ ಕಿಮುಪಾದಾನ ಇತಿ ಚ ||
\end{shloka}

(ಸೃಷ್ಟಿ ಮಾಡುವವನು ಮೂರು ಲೋಕಗಳನ್ನೂ ಯಾವ ಆಸೆಯಿಂದ, ಯಾವ ಶರೀರದಿಂದ, ಯಾವ ಉಪಾಯದಿಂದ, ಯಾವ ಆಧಾರದಿಂದ, ಯಾವ ಉಪಾದಾನದಿಂದ ಸೃಷ್ಟಿಸುತ್ತಾನೆ.)

`ಈ ಕುಂಬಾರನು ಒಂದು ಮಡಕೆಯನ್ನು ಮಾಡಬೇಕಾದರೆ ಅವನಿಗೆ ಮಣ್ಣು, ಚಕ್ರ ಇವುಗಳು ಬೇಕು. ಭಗವಂತನು ಯಾವ ಮಣ್ನನ್ನು ಇಟ್ಟಿಕೊಂಡಿದ್ದಾನೆ. ಯಾವ ಚಕ್ರವನ್ನು ಇಟ್ಟುಕೊಂಡಿದ್ದಾನೆ? ಅದೆಲ್ಲಾ ಸರಿ. ಕುಳಿತುಕೊಳ್ಳಲು ಜಾಗ ದೊರತರೆ ತಾನೇ ನಾವು ಕ್ರಿಯೆಯನ್ನು ಮಾಡಲು ಸಾಧ್ಯ. ನಿಮ್ಮ ಭಗವಂತನಿಗೆ ಕುಳಿತುಕೊಳ್ಳಲು ಜಾಗವೇ ಇಲ್ಲವಲ್ಲ. ಅವನು ಎಲ್ಲಿ ಕುಳಿತುಕೊಳ್ಳುವನು? ಸೃಷ್ಟಿಮಾಡುವುದಕ್ಕೆ ಮೊದಲು ಎಲ್ಲಿ ಕುಳಿತಿದ್ದನು? ಕುಳಿತುಕೊಳ್ಳೆಲು ಜಾಗವಿಲ್ಲದವನು ಇಷ್ಟು ದೊಡ್ಡ ಪ್ರಪಂಚವನ್ನೆಲ್ಲಾ ಸೃಷ್ಟಿಮಾಡಲು ಸಾಧ್ಯವಾಯಿತೆಂದು ನೀವು ಹೇಳುವುದಾದರೆ ಅದನ್ನು ನಾವು ನಂಬಲು ಸಾಧ್ಯವೇ ಇಲ್ಲ. ಅಲ್ಲದೇ, ಮಣ್ಣು ಇದ್ದರೆ ಕೊಡವನ್ನು ಮಾಡಬಹುದು. ಮಣ್ಣೇ ಇಲ್ಲದೆ ಕೊಡವನ್ನು ಹೇಗೆ ಮಾಡುವುದು? ನಾವು ಯಾವುದಾದರೂ ಕಾರ್ಖಾನೆಗೆ ಹೋದರೆ ಅಲ್ಲಿರುವವರು ಅದನ್ನು ನಮಗೆ ತಿಳಿಸುವರು. ಭಗವಂತನು ಇಷ್ಟು ದೊಡ್ಡ ಸೃಷ್ಟಿಯನ್ನು ಮಾಡಿದನೆಂದು ಹೇಳುತ್ತೀರಿ. ಇದಕ್ಕೆಲ್ಲಾ ಮೂಲವಸ್ತು ಯಾವುದು? ಕಲ್ಲೇ? ಮಣ್ಣೇ? ಮೂಲವಸ್ತು ಇದ್ದರೆ ಭಗವಂತನು ಎನ್ನುವವನು ಬೇಕಾಗಿಲ್ಲ. ಅದು ಇಲ್ಲ ಎಂದರೆ ಭಗವಂತನು ಹೇಗೆ ಸೃಷ್ಟಿಮಾಡಿದನು?' ಹೀಗೆ ನಾಸ್ತಿಕರು ಕೇಳುತ್ತಾರೆ.

ಇವುಗಳೆಲ್ಲವೂ ತರ್ಕಗಳೇ. ಕೆಲವು ವಿಷಯಗಳನ್ನು ಕುರಿತು ನಾವು ತರ್ಕ ಮಾಡಬಹುದು. ಕೆಲವು ವಿಷಯಗಳನ್ನು ಕುರಿತು ನಾವು ತರ್ಕಮಾಡಬಾರದು. ಏಕೆ ತರ್ಕಮಾಡಬಾರದು? ಒಬ್ಬ ಎಲಕ್ಟ್ರಿಕಲ್ ಇಂಜಿನಿಯರ್, `ಈ ತಂತಿಯ ಮೇಲೆ ಕೈ ಇಡಬೇಡಿ' ಎನ್ನುತ್ತಾನೆ. ಇನ್ನೊಬ್ಬನು `ನೀನು ಹೇಳಿದರೆ ನಾನು ಕೈ ಇಡಕೂಡದೇ?' ಎಂದು ಕೇಳಿದರೆ, ಸಾವಿರ ವೋಲ್ಟೇಜ್ ಇರುವ ವಿದ್ಯುತ್ತಿನಲ್ಲಿ ಕೈ ಇಟ್ಟರೆ ಅವನ ಗತಿ ಅಷ್ಟೇ. ಇದು ಆ ಎಲಕ್ಟ್ರಿಕಲ್ ಇಂಜಿನಿಯರ್‌ಗೆ ಗೊತ್ತು. ಸಾಧಾರಣ ಜನರಿಗೆ ಗೊತ್ತಿಲ್ಲ. ಆದ್ದರಿಂದ ಕೆಲವು ವಿಷಯಗಳನ್ನು ತರ್ಕದಿಂದ ತಿಳಿಯಬಹುದು. ಕೆಲವು ವಿಷಯಗಳನ್ನು ನಂಬಿಕೆಯಿಂದ ತಿಳಿಯಬಹುದು. ಭಗವಂತನಿದ್ದಾನೆ ಎನ್ನುವುದಕ್ಕೆ ತರ್ಕವೂ ಇದೆ. ತರ್ಕವಿಲ್ಲ ಎಂದಲ್ಲ. ಆದರೆ ನಂಬಿಕೆಯೂ ಇರಬೇಕು.

`ನೀನು ಯಾರ ಮಗ' ಎಂದು ಕೇಳಿದರೆ `ನಾನು ರಾಮಾಶಾಸ್ತ್ರಿಗಳ ಮಗ' ಎಂದು ಹೇಳುತ್ತೇನೆ. `ನೀನು ಹುಟ್ಟುವುದಕ್ಕೆ ಮುಂಚೆ ನಿನ್ನ ತಂದೆ ರಾಮಾಶಾಸ್ತ್ರಿ ಎಂದು ನಿನಗೆ ಗೊತ್ತೆ?' ಎಂದು ಕೇಳಿದರೆ `ಗೊತ್ತಿಲ್ಲ' ಎಂದು ಹೇಳುತ್ತೇನೆ.

ನಾನು ಹುಟ್ಟುವುದಕ್ಕೆ ಮುಂಚೆ ನನ್ನ ತಂದೆ ರಾಮಾಶಾಸ್ತ್ರಿ ಎಂದು ನನಗೆ ಹೇಗೆ ಗೊತ್ತು? ಇನ್ನೂ ಯಾರಾದರೂ, ಹೀಗೆ ಕೇಳಿದರೆ, `ಈ ವಿಷಯವನ್ನು ತನ್ನ ತಾಯಿ, ತಂದೆ ಹೇಳಿದರು' ಎಂದು ಹೇಳುವೆನು. ಅದಕ್ಕೆ ಒಬ್ಬನು `ಅವರು ಸುಳ್ಳು ಹೇಳಿರಬಹುದಲ್ಲಾ' ಎಂದು ನನ್ನನ್ನು ಕೇಳಿದರೆ, `ಹೇಳುವುದಿಲ್ಲಾ' ಎಂದು ನಾನು ನಂಬುವುದಾಗಿ ಹೇಳುವೆನು. ನನ್ನ ತಾತ, ಅಜ್ಜಿ, ಹತ್ತಿರ ಇರುವವರು, ಚಿಕ್ಕಪ್ಪ ಮುಂತಾದವರೆಲ್ಲರೂ ನಾನು ರಾಮಾಶಾಸ್ತ್ರಿಗಳ ಮಗನೇ ಎಂದು ಹೇಳಿದ್ದಾರೆ. ಯಾರಾದರು ನನ್ನನ್ನು `ನೀನು ನೋಡಿದೆಯಾ? ನೀನು ಕೇಳಿದೆಯಾ?' ಎಂದು ಕೇಳಿದರೆ ನಾನು ಇದಕ್ಕೆ ಏನು ಹೇಳಬಲ್ಲೆನು? ಇದಕ್ಕಾಗಿ ನಾನು ನನ್ನ ರಕ್ತವನ್ನೂ, ನನ್ನ ತಂದೆಯ ರಕ್ತವನ್ನೂ ಪರಿಶೋಧನೆ ಮಾಡಬೇಕಾಗಿಲ್ಲ. ಯಾವ ನ್ಯಾಯಾಲಯದಲ್ಲಿಯೂ ಈ ರೀತಿ ಕೇಳುವುದಿಲ್ಲ. ನ್ಯಾಯಾಲಯದಲ್ಲಿ `ನೀನು ಯಾರ ಮಗ?' ಎಂದು ಕೇಳುತ್ತಾರೆ. `ಸಾಕ್ಷಿ ಯಾರು?' ಎಂದರೆ ನಾನು ಪಕ್ಕದ ಮನೆಯವರನ್ನು ಸಾಕ್ಷಿಯಾಗಿ ತೋರಿಸಬಹುದು. ಈ ಸಾಮಾನ್ಯ ವಿಷಯದಲ್ಲಿ ನಂಬಿಕೆ ಅವಶ್ಯಕವಾಗಿರುವಾಗ ಆ ಪರವಸ್ತುವಿನ ವಿಷಯದಲ್ಲಿ ನಂಬಿಕೆ ಇಲ್ಲದೆ ಕೇವಲ ತರ್ಕದಿಂದಲೇ ಹೇಗೆ ತೀರ್ಮಾನ ಮಾಡಲು ಸಾಧ್ಯ?

\begin{shloka}
`ಅಂಚಿತ್ಯಾಃ ಖಲು ಯೇ ಭಾವಾ ನ ತಾನ್ ತರ್ಕೇಣ ಯೋಜಯೇತ್‌ |\\
ಪ್ರಕೃತಿಭ್ಯಃ ಪರಂ ಯಚ್ಚ ತದಚಿಂತ್ಯಸ್ಯ ಲಕ್ಷಣಮ್ ||'
\end{shloka}

(ಚಿಂತನೆಗೆ ಮೀರಿದ ವಿಷಯಗಳ ಬಗ್ಗೆ ತರ್ಕವನ್ನು ಮಾಡಬಾರದು. ಪ್ರಕೃತಿಗೆ ಮೇಲ್ಪಟ್ಟುದ್ದು ಯಾವುದು ಇದೆಯೋ ಅದು ಅಚಿಂತ್ಯದ ಲಕ್ಷಣ.)

ಭಗವಂತನು ಐಶ್ವರ್ಯ ಇಂತಹುದೇ, ಆದರೆ ಇದನ್ನು ಅರಿಯದೆ ತರ್ಕವನ್ನು ಮಾತ್ರ ನಾಸ್ತಿಕನು ಕೊಡುತ್ತಾನೆ. ಇಂಥ ತರ್ಕಕ್ಕೂ ಸಹ ನಂಬಿಕೆಯೇ ಕಾರಣ. ಒಬ್ಬನು ತನ್ನ ತಂದೆ, ತಾಯಿ `ನೀನು ನಮ್ಮ ಮಗ' ಎಂದು ಹೇಳಿದ್ದರೂ ಅದರಲ್ಲಿ ನಂಬಿಕೆ ಇಲ್ಲದೆ ಪಕ್ಕದ ಮನೆಯವರನ್ನು ಕೇಳುತ್ತಾನೆ. ಅವರೂ ಅದೇ ಹೇಳುತ್ತಾರೆ. ಅದರಲ್ಲೂ ನಂಬಿಕೆ ಇರುವುದಿಲ್ಲ. ಆದರೆ ಹಲವರು ಅದೇ ರೀತಿ ಹೇಳುವುದನ್ನು ಕೇಳಿ, `ಎಲ್ಲರೂ ಇದನ್ನೇ ಹೇಳುತ್ತೀರಿ. ನಾನು ಇದನ್ನು ಹೇಳುವುದು ಒಳ್ಳೆಯದು' ಎನ್ನುವ ಪರಿಣಾಮಕ್ಕೆ ಬರುತ್ತಾನೆ.

\begin{shloka}
`ಯತೋ ವಾಚೋ ನಿವರ್ತಂತೇ ಅಪ್ರಾಪ್ಯ ಮನಸಾ ಸಹ |'
\end{shloka}

(ಯಾವುದನ್ನು ವಾಕ್ಕಿನಿಂದಲೂ, ಮನಸ್ಸಿನಿಂದಲ್ಲು ಪಡೆಯಲು ಸಾಧ್ಯವಿಲ್ಲವೋ....)

-ಎಂದು ಹೇಳಿರುವಂತೆ ಭಗವಂತನನ್ನು ನಾವು ಇನ್ನು ತರ್ಕದ ಮೂಲಕವಾಗಿ ನಿರೂಪಿಸಲೂ ಆಗುವುದಿಲ್ಲ.

\begin{shloka}
`ಕುತರ್ಕೋಽಯಂ ಕಾಚಿನ್ ಮುಖರಯತಿ ಮೋಹಾಯ ಜಗತಃ'
\end{shloka}

ಪ್ರಪಂಚದಲ್ಲಿ ಮೋಹವನ್ನುಂಟುಮಾಡುವುದಕ್ಕಾಗಿ ಬುದ್ಧಿ ಇಲ್ಲದವರ ನಾಸ್ತಿಕವಾದ ತರ್ಕ ಹರಡುತ್ತದೆ. ಆದ್ದರಿಂದ ನಾಸ್ತಿಕರ ತರ್ಕ ಕುತರ್ಕವೇ ಸರಿ.

\begin{shloka}
`ಅಜನ್ಮಾನೋ ಲೋಕಾಃ ಕಿಮವಯವವನ್ತೋಽಪಿ ಜಗತಾಂ\\
ಅಧಿಷ್ಠಾತಾರಂ ಕಿಂ ಭವವಿಧಿರನಾದೃತ್ಯ ಭವತಿ |\\
ಅನೀಶೋವಾ ಕುರ್ಯಾತ್ ಭುವನ ಜನನೇ ಕಃ ಪರಿಕರಂ\\
ಯತೋ ಮಂದಾಸ್ತ್ವಾಂ ಪ್ರತ್ಯಮರವರ ಸಂಶೇರತ ಇಮೇ ||'
\end{shloka}

(ಶ್ರೇಷ್ಠ ನಾಯಕನೇ ! ಲೋಕಗಳಿಗೆ ಆಧಾರವಿಲ್ಲವೆಂದು ಹೇಗೆ ತಾನೇ ಹೇಳುವುದು? ಲೋಕದ ಸೃಷ್ಟಿ, ಸೃಷ್ಟಿಸುವವನು ಒಬ್ಬನು ಇಲ್ಲದೆ ಆಗುವುದೇ? ಭಗವಂತನಲ್ಲದೆ ಬೇರೆ ಯಾರು ತಾನೆ ಲೋಕಗಳನ್ನು ಸೃಷ್ಟಿಸಲು ಸಾಧ್ಯ? ಅವರು ಮೂರ್ಖರಾಗಿದ್ದಾರೆ `ನೀನು ಇದ್ದೀಯ' ಎನ್ನುವ ವಿಷಯದಲ್ಲಿ ಸಂದೇಹ ವ್ಯಕ್ತ ಪಡಿಸುತ್ತಾರೆ.)

ಹೀಗೆ ಒಳ್ಳೆಯ ತರ್ಕ ಯಾವುದೆಂದು ಹೇಳಲ್ಪಟ್ಟಿದೆ. ಆದ್ದರಿಂದ ಭಗವಂತನು ಇದ್ದಾನೆ ಎನ್ನುವುದಕ್ಕೆ ಶ್ರುತಿಯೂ, ಯುಕ್ತಿಯೂ, ಪ್ರತ್ಯಕ್ಷವೂ ಅಲ್ಲದೆ ಅನುಭವವೂ ಇದೆ. ನಾವು ಕೆಲವೊಮ್ಮೆ ಭಗವಂತನನ್ನು ನೋಡುವುದುಂಟು. ಅವನು ಯಾವ ರೂಪದಲ್ಲಿ ಬಂದು ಅನುಗ್ರಹಿಸುತ್ತಾನೆಂದು ಹೇಳಲು ಸಾಧ್ಯವಿಲ್ಲ. ಕೆಲವೊಮ್ಮೆ ಯಾರೋ ಬಂದು ಅನುಗ್ರಹಿಸಿ ಹೋಗುತ್ತಾರೆ. ಆದರೆ ಅವರು ಯಾರೆಂದು ನಮಗೆ ಗೊತ್ತಿಲ್ಲ. ಇದನ್ನು ನಾವು ಪ್ರಪಂಚದಲ್ಲಿ ನೋಡುತ್ತೇವೆ. ಅಷ್ಟೇ ಅಲ್ಲ. ಯಾರೋ ಒಬ್ಬ ಸಾಮಾನ್ಯ ಮನುಷ್ಯನು ಆಶೀರ್ವದಿಸಿದರೂ ಅದು ಫಲಿಸುವುದು. ಆದರೆ ಅವನಿಗೆ ಆ ಶಕ್ತಿಯೇ ಇದ್ದುದಿಲ್ಲ. ಭಗವಂತನು ಅವನ ರೂಪದಲ್ಲಿ ಬಂದು ಅನುಗ್ರಹ ಮಾಡಿದನೆಂದು ನಾವು ನೋಡುತ್ತೇವೆ. ವೇದದಲ್ಲಿ, ಪ್ರಮಾಣವೆನ್ನುವ ನಂಬಿಕೆ ಇಲ್ಲದವರಿಗಾಗಿ `ಭಗವಂತನು ಇದ್ದಾನೆಂದು ಹೇಗೆ ಯುಕ್ತಿಯಿಂದ ನಿರೂಪಿಸಲು ಸಾಧ್ಯ? ಅವನ ರೂಪವೇನು?' ಎನ್ನುವುದು ಇನ್ನು ವಿವರವಾಗಿ ನೋಡೋಣ.

\begin{shloka}
`ಅಚಿಂತ್ಯಾ ಖಲು ಯೇ ಭಾವಾ ನ ತಾನ್ ತರ್ಕೇಣ ಯೋಜಯೇತ್ |\\
ಪ್ರಕೃತಿಭ್ಯಃ ಪರಂ ಯಚ್ಚ ತದಚಿಂತ್ಯಸ್ಯ ಲಕ್ಷಣಮ್ ||'
\end{shloka}

-ಎಂದು ಒಂದು ಜಾಗದಲ್ಲಿ ಬರುವುದನ್ನು ಮೊದಲೇ ನೋಡಿದೆವು. ಕೆಲವು ವಸ್ತುಗಳು ತರ್ಕಕ್ಕೆ ಒಳಪಡುವುದಿಲ್ಲ. ಅವುಗಳಿಗೆ ತರ್ಕಬೇಕಾಗಿಲ್ಲ. `ಆಪ್ತವಾಕ್ಯ' ಪ್ರಮಾಣವಾಗುತ್ತದೆ. ಆಪ್ತನು ಸರ್ವೇಶ್ವರ. ಅವನ ವಾಕ್ಯ ವೇದ. ಅದು ಸಾಕು. ಆದರೆ ಅದರಲ್ಲಿ ನಂಬಿಕೆ ಉಂಟಾಗುವವರೆಗೆ ತರ್ಕ ಖಂಡಿತವಾಗಿಯೂ ಇದೆ. ನಂಬಿಕೆ ಉಂಟಾದ ಮೇಲೆ ತರ್ಕವಿಲ್ಲದೆಯೂ, ನಾವು ವೇದವಾಕ್ಯದಿಂದ

\begin{shloka}
`ತಂ ತ್ವೌಪನಿಷದಂ ಪುರುಷಂ ಪೃಚ್ಛಾಮಿ'
\end{shloka}

(ಉಪನಿಷತ್ತುಗಳಲ್ಲಿ ಹೇಳಲಾಗಿರುವ ಪುರುಷನ ಬಗೆಗೆ ಕೇಳುತ್ತೇನೆ.)

\begin{shloka}
`ಉಪನಿಷದಾ ಏವ ಸಮಧಿಗಮ್ಯತೇ'
\end{shloka}

-ಎಂದು ಹೇಳಿದಂತೆ ಉಪನಿಷತ್ತಿನಿಂದಲೇ ನಾವು ಭಗವಂತನನ್ನು ತಿಳಿಯಲು ಸಾಧ್ಯ.

\begin{shloka}
`ನಾವೇದವಿತ್ ಮನುತೇ ತಂ ಬೃಹಂತಮ್'
\end{shloka}

(ವೇದವನ್ನು ಅರಿಯದವನು ಪರವಸ್ತುವನ್ನು ಚಿಂತಿಸುವುದಿಲ್ಲ.)

-ಎಂದು ಒಂದು ಕಡೆ ತಿಳಿಸಲಾಗದೆ.

ನಾವು ನಮ್ಮ ಬುದ್ಧಿಗೆ ತಕ್ಕಂತೆ ಯಾವುದೋ ಒಂದು ರೀತಿ ತಿಳಿದು ಕೊಳ್ಳಬಹುದು. ಕೆಲವರನ್ನು `ಭಗವಂತನು ಎಲ್ಲಿದ್ದಾನೆ' ಎಂದು ಕೇಳಿದರೆ, `ಅವನು ವೈಕುಂಠದಲ್ಲಿದಾನೆ' ಎಂದು ಅವರು ಹೇಳುತ್ತಾರೆ. ಇನ್ನೂ ಕೆಲವರನ್ನು ಕೇಳಿದರೆ, `ಯಾವುದಾದರೂ ಆಕಾಶದಲ್ಲಿದ್ದಾನೆ' ಎನ್ನುತ್ತಾರೆ. ಎಷ್ಟು ಆಕಾಶಗಳು(!) ಎನ್ನುವುದು ಅವರಿಗೆ ಗೊತ್ತಿಲ್ಲ. ಆದ್ದರಿಂದ ಮನುಷ್ಯನ ಬುದ್ಧಿಯಿಂದಲೇ ಪರಮ ಶಿವನನ್ನು ತಿಳಿದುಕೊಳ್ಳುವುದು ಎಂದಾದರೆ ಅದಕ್ಕೆ ಒಂದು ಎಲ್ಲೆಯೇ ಇಲ್ಲದಂತಾಗುತ್ತದೆ. ಆದರೆ ಪರಮೇಶ್ವರನ ಸ್ವರೂಪ ಹೇಗಿದೆಯೋ ಹಾಗೆ ನಾವು ತಿಳಿದುಕೊಳ್ಳಲು ಆಗುವುದಿಲ್ಲ. ಅದಕ್ಕೆ ವೇದವೇ ಸಾಧನವೆಂದು ನಮ್ಮ ಹಿರಿಯರ ನಿರ್ಣಯ. ಆದರೆ ಮೊದಲು ಅದನ್ನು ಹೇಳಿದರೆ ನಾಸ್ತಿಕನಿಗೆ ಅದರಲ್ಲಿ ನಂಬಿಕೆಯೇ ಇರುವುದಿಲ್ಲ. ಆದ್ದರಿಂದ ನಾನು ತರ್ಕದಲ್ಲೇ ಜವಾಬು ಕೊಡುತ್ತ್ತೇನೆ. ಆದರೆ ಈ ತರ್ಕ ಎಲ್ಲಾ ಜಾಗದಲ್ಲೂ ಪೂರ್ತಿಯಾಗಿ ಇರುವುದಿಲ್ಲ. ಕೊನೆಯಲ್ಲಿ ನಾವು ವೇದಕ್ಕೆ ಶರಣಾಗಬೇಕು. ಅನಂತರ ಜನರು, `ನಾವು ಯೋಚನೆ ಮಾಡಿದುದೂ, ವೇದದಲ್ಲಿ ಹೇಳಲ್ಪಟ್ಟಿರುವುದೂ, ಸಮಾನವಾಗಿದೆ. ಆದ್ದರಿಂದ ನಾವು ವೇದಕ್ಕೆ ಶರಣಾಗಬೇಕು' ಎನ್ನುವ ಮನೋಭಾವನೆಗೆ ಒಳಗಾಗಬಹುದು. ಆದ್ದರಿಂದ ಮೊದಲು ತರ್ಕವನ್ನು ಹೇಳಬೇಕು. ಶಂಕರ ಭಗವತ್ಪಾದರು ವೇದವಾಕ್ಯಗಳನ್ನು ತೆಗೆದುಕೊಂಡು ಅವುಗಳ ತಾತ್ಪರ್ಯವನ್ನು ಹೇಳಲು ಬಂದರೆ ಹೊರತು ಯುಕ್ತಿಯನ್ನು ಮಾತ್ರ ತಾರ್ಕಿಕರಂತೆ ಇಟ್ಟುಕೊಂಡು ಪರಮೇಶ್ವರನಿದ್ದಾನೆಂದು ನಿರೂಪಿಸಲು ಬರಲಿಲ್ಲ. ಆದ್ದರಿಂದ ಯುಕ್ತಿಯನ್ನು ಮಾತ್ರ ಇಟ್ಟುಕೊಂಡು ನಾವು ಹಾಗೆ ಹೇಳಲಾರೆವು; ಏಕೆಂದರೆ, 

\begin{shloka}
`ಯತ್ನೇನಾನುಮಿತೋಽಪ್ಯರ್ಥಃ ಕುಶಲೈರನುಮಾತೃಭಿಃ |\\
ಅಭಿಯುಕ್ತತರೈಃ ಅನ್ಯೈಃ ಅನ್ಯಥೈವೋಪದ್ಯತೇ ||
\end{shloka}

(ಇಂದು ಬುದ್ಧಿಶಾಲಿಗಳು ಪ್ರಯತ್ನ ಪಟ್ಟು ಯಾವುದನ್ನು ತೀರ್ಮಾನ ಮಾಡುತ್ತಾರೋ, ಇವರುಗಳಿಗಿಂತಲೂ ಬುದ್ಧಿಶಾಲಿಗಳು ಇದನ್ನು ಬದಿಗಿಟ್ಟು ಬೇರೆ ಒಂದನ್ನು ತೀರ್ಮಾನ ಮಾಡುವರು)

-ಎಂದು ವಾಚಸ್ಪತಿ ಮಿಶ್ರರು ಹೇಳಿದ್ದಾರೆ. `ನಾವು ಇಂದು ಕೋರ್ಟಿನಲ್ಲಿ ಸೋತುಹೋದೆವು. ಏಕೆಂದರೆ ನಮ್ಮ ಲಾಯರಿಗೆ ಬುದ್ಧಿ ಇಲ್ಲ. ಸ್ವಲ್ಪ ಹೆಚ್ಚು ಹಣವನ್ನು ತೆಗೆದುಕೊಳ್ಳುವ ಲಾಯರ್ ಹತ್ತಿರ ಹೋಗಿದ್ದರೆ ಅವನು ಹೇಗಾದರೂ ಮಾಡಿ ಗೆಲ್ಲಿಸುತ್ತಿದ್ದನು' ಎಂದು ತೋರುವುದನ್ನು ನಾವು ನೋಡಬಹುದು. ಆದರೆ `ಕೇಸು ಸರಿಯಾದುದೇ? ಲಾಯರ್ ಸರಿಯಾದವನೇ?' ಎನ್ನುವ ವಿಷಯದಲ್ಲಿ ನಮಗೆ ಸಂದೇಹವೇ ಉಂಟಾಗುತ್ತದೆ. ಆದ್ದರಿಂದ ಕೇವಲ ಯುಕ್ತಿಯಿಂದ ಈ ವಿಷಯದಲ್ಲಿ ಸಿದ್ಧಾಂತವನ್ನು ಮಾಡಲು ಸಾಧ್ಯವಿಲ್ಲ. ನಮ್ಮ ಯುಕ್ತಿಯೂ ಸಹ ವೇದಕ್ಕೆ ಅನುಕೂಲವಾಗಿರಬೇಕೆಂದು-

\begin{shloka}
`ವಾಕ್ಯರ್ಥಶ್ಚ ವಿಚಾರ್ಯತಾಂ ಶ್ರುತಿಶಿರಃ ಪಕ್ಷಃ ಸಮಾಶ್ರೀಯತಾಮ್ ||'
\end{shloka}

(ವಾಕ್ಯಾರ್ಥವನ್ನು ವಿಚಾರ ಮಾಡು, ಉಪನಿಷತ್ತಿನ ಅಭಿಪ್ರಾಯಕ್ಕೇ ಶರಣುಹೋಗು.)

\begin{shloka}
`ದುಸ್ತರ್ಕಾತ್ ಸುವಿರಮ್ಯತಾಂ ಶ್ರುತಿಮತಸ್ತರ್ಕೋಽನುಸಂಧೀಯ ತಾಮ್|'
\end{shloka}

(ಕೆಟ್ಟ ತರ್ಕಗಳಿಂದ ದೂರವಾಗಿರು, ವೇದಕ್ಕೆ ಸಹಕಾರಿಯಾದ ತರ್ಕವನ್ನು ಅನುಸಂಧಾನಮಾಡು.)

-ಹೇಳಲ್ಪಟ್ಟಿದೆ. ವೇದಕ್ಕೆ ಸಹಕಾರಿಯಾದ ತರ್ಕವನ್ನು ನಾವು ತೆಗೆದುಕೊಳ್ಳಬೇಕು, ಹಾಗಿಲ್ಲದ ತರ್ಕವನ್ನು ಬಿಟ್ಟು ಬಿಡಬೇಕು. ಅಂಥ ತರ್ಕವನ್ನು ಕವಿ,

\begin{shloka}
`ಅಜನ್ಮಾನೋ..................ಇಮೇ'
\end{shloka}

ಎನ್ನುವ ಶ್ಲೋಕದಲ್ಲಿ ಅದು ಮೂರ್ಖರು ಎತ್ತುವ ತರ್ಕವೆಂದು ವರ್ಣಿಸಿದ್ದಾನೆ.

ಜನರಿಗೆ ಒಂದು ಸಂದೇಹ ಉಂಟಾಗಿದೆ, `ಏಕೆ ಸಂದೇಹ ಉಂಟಾಗಿದೆ?' `ಮಂದಾಃ' ಎಂದು ಹೇಳಿರುವುದಕ್ಕೆ ಕಾರಣ ಬುದ್ಧಿ ಪರಿಶುದ್ಧವಾಗಿಲ್ಲ. ಹೊರ ಪ್ರಪಂಚದ ವಾಸನೆಗಳಿಂದಾಗಿ ಮಂದವಾಗಿ, ಜಡವಾಗಿ, ಜಡವಾಗಿರುವ ಅಂತಃಕರಣದಿಂದ ಯೋಚನೆ ಮಾಡುವುದರಿಂದಾಗಿ ಸಂಶಯ (ಸಂದೇಹ)ವೇ ಒಬ್ಬನಲ್ಲಿ ಇದ್ದು ಬಿಡುತ್ತದೆ. ಪರಮೇಶ್ವರನು ಇದ್ದಾನೆಯೇ ಇಲ್ಲವೇ ಎಂದರೆ `ಇಲ್ಲ' ಎಂದು ಹೇಳುವುದಕ್ಕೆ ಅವನಿಗೆ ಧೈರ್ಯವಿಲ್ಲ. `ಇದ್ದಾನೆ' ಎಂದು ಹೇಳುವುದಕ್ಕೆ ಮನಸ್ಸಿಲ್ಲ. ಆದ್ದರಿಂದ `ಪರಮೇಶ್ವರನು ಇದ್ದರೂ ಇರಬಹುದು, ಇಲ್ಲದಿದ್ದರೂ ಇರಬಹುದು' ಎನ್ನುತ್ತಾನೆ.

\begin{shloka}
`ಸಂಶಯಾತ್ಮಾ ವಿನಶ್ಶತಿ'\\
(ಸಂದೇಹ ಪಡುವವನು ನಾಶವಾಗುತ್ತಾನೆ.)
\end{shloka}

ನೀಲಕಂಠ ದೀಕ್ಷಿತರು ಹೇಳುವಂತೆ ಒಬ್ಬನು ನಾಸ್ತಿಕನಾಗಿದ್ದರೆ, ಅವನು ತನ್ನ ಹಣವನ್ನೆಲ್ಲಾ ಚೆನ್ನಾಗಿ ಅನುಭವಿಸಿ ಪ್ರಪಂಚದಲ್ಲಿ ಯಾವುದನ್ನೋ ಸಾಧಿಸಿದಂತೆ ತೃಪ್ತಿಯನ್ನು ಪಡೆಯುವನು. ಒಬ್ಬನು ಆಸ್ತಿಕನಾಗಿದ್ದರೆ, ಧರ್ಮಕಾರ್ಯಗಳಲ್ಲಿ ತನ್ನ ಹಣವನ್ನೆಲ್ಲಾ ವಿನಿಯೋಗಿಸಿ ಇಹ-ಪರ ಎರಡನ್ನೂ ಸಾಧಿಸಿಕೊಳ್ಳುತ್ತಾನೆ. ಕೆಲವರು ತಮಗಾಗಿಯೂ ಖರ್ಚು ಮಾಡುವುದಿಲ್ಲ. ಇತರರಿಗೂ ಕೊಡುವದಿಲ್ಲ. ಹಾಗಿರುವವರು ಆಸ್ತಿಕರೇ? ನಾಸ್ತಿಕರೇ? ಇದಕ್ಕೆ ಕಥೆಯೊಂದನ್ನು ಹೇಳುವುದುಂಟು.

\begin{shloka}
`ಅರ್ಧಂ ಪಾಕಾಯ ಅರ್ಧಂ ಪ್ರಸವಾಯ'\\
(ಅರ್ಧ ಅಡಿಗೆಗೆ ಅರ್ಧ ಪ್ರಸವಕ್ಕೆ)
\end{shloka}

ಎನ್ನುವುದೇ ಆ ಕಥೆ, ಒಬ್ಬನು ಕೋಳಿಮೊಟ್ಟೆ ಇಟ್ಟುಕೊಂಡಿದ್ದನು. ಅವನ ಮನಸ್ಸಿನಲ್ಲಿ `ಮೊಟ್ಟೆಯಿಂದ ಮರಿಗಳು ಬಂದರೆ ಸಂತತಿ ಹೆಚ್ಚಾಗುತ್ತದೆ' ಎಂದು ತೋರಿದಂತೆ. ಆದ್ದರಿಂದ ಮೊಟ್ಟೆಯನ್ನು ಹಾಗೆಯೇ ಇಟ್ಟುಕೊಳ್ಳಬೇಕೆಂದು ಕೊಂಡನು. ಆದರೆ ತೆಗೆದುಕೊಂಡು ಬಂದುದನ್ನು ರುಚಿ ನೋಡದೆ ಇದ್ದರೆ ಪ್ರಯೋಜನವಿಲ್ಲವೆಂದುಕೊಂಡು ಅವನು ಮೊಟ್ಟೆಯಲ್ಲಿ ಅರ್ಧವನ್ನು ತಿಂದು ಉಳಿದ ಅರ್ಧವನ್ನು ಮರಿಗಳಿಗಾಗಿ ಇಟ್ಟುಕೊಂಡನಂತೆ. ಆದ್ದರಿಂದ,

\begin{shloka}
`ಅರ್ಧಂ ಪಾಕಾಯ ಅರ್ಧಂ ಪ್ರಸವಾಯ'
\end{shloka}

-ಎಂದು ಹೇಳಲಾಗಿದೆ. `ಸಂಶಯಾತ್ಮಾ ವಿನಶ್ಯತಿ' ಎನ್ನುವುದರಿಂದ ಸಂಶಯವನ್ನು ಇಟ್ಟುಕೊಳ್ಳಲೇಬಾರದು.

ಇನ್ನು ನಮ್ಮ ಹತ್ತಿರ ಯಾವ ತರ್ಕವಿದೆಯೆಂದು ನೋಡೋಣ.

\begin{shloka}
`ಅಜನ್ಮಾನೋ ಲೋಕಾಃ ಕಿಮವಯವವನ್ತೋಪಿಽಜಗತಾಂ'
\end{shloka}

-ಎಂದು ಮೊದಲು ಹೇಳಲ್ಪಟ್ಟಿದೆ, ನಾವು ಪ್ರಪಂಚದಲ್ಲಿ ನೋಡುವ ಪ್ರತಿಯೊಂದು ವಸ್ತುವೂ ಅನೇಕ ವಸ್ತುಗಳು ಸೇರಿ ಆ ವಸ್ತುವಾಗಿ ಮಾರ್ಪಟ್ಟಿದೆ. ಒಂದು ಮನೆಯನ್ನು ತೆಗೆದುಕೊಂಡರೆ ಅದಕ್ಕೆ ಒಂದು ಕಂಬ ಬೇಕು. ಮೇಲೆ ಹಾಕುವುದಕ್ಕೆ `ಶೀಟ್' ಬೇಕು. ಆ `ಶೀಟ್' ನಿಲ್ಲುವುದಕ್ಕೆ ತೊಲೆಬೇಕು. ನೆಲಮಾಡಬೇಕು. ಇವುಗಳೆಲ್ಲಾ ಆ ಮನೆಯ ಅವಯವಗಳು. ಒಬ್ಬ ಮನುಷ್ಯನನ್ನು ತೆಗೆದುಕೊಂಡರೆ ಅವನಿಗೆ ಕೈ, ಕಾಲು, ಮೂಗು, ಮುಂತಾದವು ಇರುತ್ತವೆಯೆಂದು ನೋಡುತ್ತೇವೆ. ಈ ಪ್ರಪಂಚದಲ್ಲಿ ಇರುವ ಯಾವ ವಸ್ತುವಿಗೇ ಆಗಲಿ ಅವಯವಗಳು ಇವೆ. ಅವಯವಗಳಿರುವ ವಸ್ತುಗಳೆಲ್ಲಾ ಉಂಟಾಗಿರುವುವೇ ಎಂದು ನಾವು ಸಾಮಾನ್ಯವಾಗಿ ನೋಡಬಹುದು.

\begin{shloka}
`ಅಂಶಿನಃ ಸ್ವಂಶಕಾತ್ಯಂತಾಭಾವಸ್ಯ ಪ್ರತಿಯೋಗಿನಃ |\\
ಅಂಶಿತ್ವಾತ್ ಇತರಾಂಶೀವ ಧಿಗೇಶೈವ ಗುಣಾದಿಷು ||'
\end{shloka}

ಅವಯವಗಳಿರುವ ವಸ್ತುಗಳನ್ನು ಕುರಿತು ತರ್ಕಶಾಸ್ತ್ರ ಹೀಗೆ ತಿಳಿಸುತ್ತದೆ. ಆದ್ದರಿಂದ ಯಾವ ಯಾವುದಕ್ಕೆ ಅವಯವವನ್ನು ಹೇಳುತ್ತೇವೋ, ಅವುಗಳೆಲ್ಲವೂ ಉಂಟಾದುವೆಂದೇ ತೀರ್ಮಾನ. ಒಂದು ಪರ್ವತದಲ್ಲಿ ನಾವು ಬೆಂಕಿಯನ್ನು ನೋಡದಿದ್ದರೂ ಅಲ್ಲಿ ಬೆಂಕಿ ಇದೆಯೆಂದು ಅಲ್ಲಿ ಹೊಗೆ ಹೋಗುವುದನ್ನು ನೋಡಿ ಹೇಳಿ ಬಿಡುತ್ತೇವೆ. ಅದನ್ನು ಅನುಮಾನವೆಂದು ಹೇಳುತ್ತೇವೆ. ಅದೇ ರೀತಿ ಪ್ರಪಂಚದಲ್ಲಿರುವ ವಸ್ತುಗಳು ನಮ್ಮಿಂದಲೋ, ನಮ್ಮ ಹಿಂದಿನವರಿಂದಲೋ ಮಾಡಲ್ಪಟ್ಟಿವೆ. ಆದ್ದರಿಂದ ಒಂದು ಮನೆಯನ್ನೋ ಒಂದು ಪುಸ್ತಕವನ್ನೋ ಅಥವಾ ಒಂದು ಗಡಿಯಾರವನ್ನೋ ತೆಗೆದುಕೊಂಡರೂ ಇವೆಲ್ಲವೂ ಅವಯವಗಳು ಇರುವವೇ. ಇವುಗಳೆಲ್ಲವೂ ಉಂಟುಮಾಡಲ್ಪಟ್ಟವೆಂದು ನಾವು ತಿಳಿಯುತ್ತೇವೆ. ಹಾಗೆಯೇ ಈ ದೊಡ್ಡ ಭೂಮಿಯೂ, ನಕ್ಷತ್ರ ಮಂಡಲವೂ ಅವಯವಗಳುಳ್ಳವು. ಹಲವು ವಿಧವಾದ ಅಂಶಗಳು ಸೇರಿ ಇವು ಒಂದು ದೊಡ್ಡ ವಸ್ತುವಾಗಿ ನಮಗೆ ತೋರುತ್ತವೆ. ಸೇರಿದ ವಸ್ತ್ತು `ಉಂಟಾಗಿರುವುದು' ಎಂದು ನಾವು ಮೊದಲೇ ತೀರ್ಮಾನ ಮಾಡಿದೆವು. ಪ್ರಪಂಚ ಮೊದಲಾದವು ಸೇರಿದ ವಸ್ತುಗಳು. ಆದ್ದರಿಂದ ಇವುಗಳೆಲ್ಲವೂ ಉಂಟಾಗಿರುವುವೆಂದು ತೀರ್ಮಾನವಾಗಿ ನಾವು ನಿಶ್ಚಯಿಸಬೇಕು.

\begin{shloka}
`ಅಧಿಷ್ಠಾತಾರಂ ಕಿಂ ಭವವಿಧಿರನಾದೃತ್ಯ ಭವತಿ |'
\end{shloka}

ನಾವು ನೋಡುವ ಎಲ್ಲಾ ವಸ್ತುಗಳನ್ನೂ ಒಬ್ಬನು ಮಾಡಿರಬೇಕು. ಹೀಗಲ್ಲದೆ ಯಾವ ವಸ್ತುವೂ ಆಗಿರುವುದನ್ನು ನಾವು ನೋಡಿಲ್ಲ. ನನ್ನ ಎದುರಿಗಿರುವ ಮೈಕ್ ಯಾರೋ ಒಬ್ಬರು ಮಾಡಿ ಇಟ್ಟಿರುವುದು ಆಗಿದೆ. ಯಾರು ತಯಾರು ಮಾಡಿದವರೆಂದು ಅಂಗಡಿಯವರು ಹೇಳುವರು. ಆದ್ದರಿಂದ ಇದು ತಯಾರು ಮಾಡಲ್ಪಟ್ಟಿದೆ ಎನ್ನುವುದು ತೀರ್ಮಾನ. ಅದೇ ರೀತಿ ಪ್ರಪಂಚವೂ ಸೃಷ್ಟಿಸಲ್ಪಟ್ಟು ಇದಕ್ಕೆ ಒಬ್ಬ ಸೃಷ್ಟಿಕರ್ತ (ಅಧಿಷ್ಠಾತ)ನಿದ್ದು `ಪ್ರಪಂಚವನ್ನು ಸೃಷ್ಟಿಸಬೇಕು- ಎನ್ನುವ ಜ್ಞಾನ ಅವನಿಗೆ ಉಂಟಾಗಿ, ಅನಂತರ ಹೇಗೆ ಸೃಷ್ಟಿಸಬೇಕೆಂದು ಯೋಚನೆಮಾಡಿ ಸೃಷ್ಟಿಯನ್ನು ಮಾಡಿದನು. ಒಬ್ಬನು ಉಂಟುಮಾಡದೆ ಯಾವ ವಸ್ತುವೂ ಉಂಟಾಗುವುದಕ್ಕೆ ಸಾಧ್ಯವೇ ಇಲ್ಲ. ಆದ್ದರಿಂದ ನಾವು ನೋಡುವ ಈ ಪ್ರಪಂಚಕ್ಕೆ ಒಬ್ಬನು ಕರ್ತೃ ಇದ್ದಾನೆ ಎನ್ನುವುದು ತೀರ್ಮಾನವೆಂದು ಹೇಳಲಾಗಿದೆ. ನಾಸ್ತಿಕನು ಮೊದಲು ಹಲವು ಪ್ರಶ್ನೆಗಳನ್ನು ಕೇಳಿದ್ದನು.

\begin{shloka}
`ಕಿಮೀಹಃ ಕಿಂ ಕಾಯಃ ಸ ಖಲು ಕಿಮುಪಾಯಸ್ತ್ರಿಭುವನಮ್'
\end{shloka}

ಎನ್ನುತ್ತ್ತಾನೆ ನಾಸ್ತಿಕನು. ಭಗವಂತನಿಗೆ ಎಂಥ ಕೆಲಸಗಳಿವೆ? ಅವನಿಗೆ ಯಾವ ಶರೀರವಿದೆ? ಅವನಿಗೆ ಪ್ರಪಂಚವನ್ನು ಸೃಷ್ಟಿಮಾಡಲು ಯಾವ ಮೂಲವಸ್ತುವಿದೆ? ಆ ಮೂಲವಸ್ತುವನ್ನು ಯಾರು ಸೃಷ್ಟಿ ಮಾಡಿದರು? ಅವನೇ ಈಶ್ವರನೇ? ಈ ರೀತಿ ಅನೇಕ ಪ್ರಶ್ನೆಗಳನ್ನು ನಾಸ್ತಿಕನು ಕೇಳಿದ್ದನಲ್ಲವೇ? ಅದಕ್ಕೆ ಪುಷ್ಪದಂತನು ಉತ್ತರ ಕೊಡುವ ರೀತಿಯಲ್ಲಿ-

\begin{shloka}
`ಅನಿಶೋ ವಾ ಕುರ್ಯಾತ್'
\end{shloka}

ಎಂದು ಶ್ಲೋಕವನ್ನು ಕೇಳಿದ್ದನು.

ಯಾವುದಾದರೂ ಒಂದು ಸಾಧಾರಣವಾದ ಮಡಕೆಯನ್ನು ಮಾಡಬೇಕಾಗಿದ್ದರೂ, ಆ ಕೆಲಸದಲ್ಲಿ ನೈಪುಣ್ಯವಿರುವವನೇ ಅದನ್ನು ಮಾಡಲು ಸಾಧ್ಯ. ನಾವು ಮಡಕೆಯನ್ನು ಮಾಡುತ್ತೇವೆಂದು ಹೇಳಿ ಮಣ್ನನ್ನೂ, ಚಕ್ರವನ್ನೂ ಇಟ್ಟುಕೊಂಡು ಕುಳಿತರೆ ಮಾಡುವುದಕ್ಕೆ ಒಂದು ನೈಪುಣ್ಯವಿರಬೇಕು. ಆದ್ದರಿಂದ ಇಷ್ಟು ದೊಡ್ಡ ಪ್ರಪಂಚವನ್ನು ಮಾಡಬೇಕೆಂದರೆ ಅದನ್ನು ಮಾಡಿದವನು ಎಂಥ ನಿಪುಣನಾಗಿರಬೇಕು? ಆ ನೈಪುಣ್ಯವಿಲ್ಲದವನು ಸೃಷ್ಟಿಯನ್ನು ಮಾಡಿದನೆಂದು ನಾವು ತಿಳಿಯುವಂತಿಲ್ಲ. ಇಷ್ಟು ನೈಪುಣ್ಯವಿರುವ ಪರಮೇಶ್ವರನು ಯಾವ ಮೂಲ ವಸ್ತುವನ್ನು ಇಟ್ಟುಕೊಂಡು ಸೃಷ್ಟಿಯನ್ನು ಮಾಡಲು ಸಾಧ್ಯವಾಯಿತೆಂದು ನಾಸ್ತಿಕನು ಕೇಳುತ್ತಾನೆ. ಮೂಲವಸ್ತು ಇಲ್ಲದೆ ಯಾವುದಾದರೂ ಒಂದು ವಸ್ತುವನ್ನು ತಯಾರು ಮಾಡಲು ಸಾಧ್ಯವೆ? ಇಲ್ಲವೇ?- ಎನ್ನುವುದನ್ನು ನಾವು ಈಗ ನೋಡಬಹುದು. ಒಬ್ಬ ಇಂದ್ರಜಾಲವನ್ನು ಮಾಡುವವನು ಕೈಯಲ್ಲಿ ಏನೂ ಇಲ್ಲದಿದ್ದರೂ, `ಲಾಡು' ಬೇಕು ಎಂದರೆ ಅದನ್ನು ತೋರಿಸುತ್ತಾನೆ. ಲಾಡು ಎಲ್ಲಿಂದ ಬಂದಿತು? ಅವನು ಅದಕ್ಕೆ ಸಂಬಂಧಪಟ್ಟದ್ದನ್ನು ಎಲ್ಲಿ ಇಟ್ಟುಕೊಂಡಿದ್ದರೂ ನಮ್ಮ ನೋಟಕ್ಕೆ ಒಂದೂ ಇಲ್ಲ. ಅವನು ಒಬ್ಬನನ್ನು ಕರೆದು `ನೀನು ನಿಲ್ಲು. ನಿನ್ನ ಕೈಯನ್ನು ಅಲ್ಲಾಡಿಸಬೇಡ! ತಾನಾಗಿಯೇ ನಿನಗೆ ಒಂದು ಚೀಟಿ ಸಿಕ್ಕುವುದು' ಎನ್ನುತ್ತಾನೆ. ಆಗ ಚೀಟಿ ಬರುತ್ತದೆ. ಎಲ್ಲಿಂದ ಬಂದಿತು -ಎಂದರೆ ನಾವು ಜವಾಬು ಹೇಳಲಾಗುವುದಿಲ್ಲ. ಇಂದ್ರಜಾಲವಿದ್ಯೆಯನ್ನು ಅರಿತವರು ಕೆಲವರು ಹಣವನ್ನು ಕೂಡ ಬರುವಂತೆ ಮಾಡುತ್ತಾರೆ. ಹಣವನ್ನು ಬರುವಂತೆ ಮಾಡುವಾಗ ನೋಡುವುದಕ್ಕೆ ಅದು ಬಹಳ ವಿಚಿತ್ರವಾಗಿರುತ್ತದೆ. ಆದರೆ ರೂಪಾಯಿ ಮಾತ್ರ `ಫಳಫಳ' ಹೊಳೆಯುತ್ತದೆ. ಈ ರೂಪಾಯಿ ನಮಗೆ ಉಪಯೋಗಕ್ಕೆ ಬರುವುದಿಲ್ಲ. ಆದರೂ ಇಂದ್ರಜಾಲ ಮಾಡುವವನು ಹಣವನ್ನು ಮಾಡಿದನೆಂದು ನಾವು ನೋಡಿದೆವು. ರೂಪಾಯಿಗೆ ಬೇಕಾದ ಬೆಳ್ಳಿಯನ್ನು `ಅವನು ಎಲ್ಲಿಂದ ತಂದನು?' ಅವನಿಗೆ ಬೆಳ್ಳಿ ಬೇಕಾಗಿಲ್ಲ. ಮಂತ್ರಶಕ್ತಿ ಅಥವಾ ಯಾವುದಾದರೂ ಕೆಲವು ವಸ್ತುಗಳನ್ನು ಸೇರಿಸುವುದರಿಂದ ನಿಜವಲ್ಲದ ವಸ್ತುಗಳನ್ನು ಅವನು ಮಾಡುವುದನ್ನು ನಾವು ಕಾಣಬಹುದು. ಕೆಲವು ವಿಧವಾದ ಉಪಾಸನೆಗಳನ್ನು ಮಾಡಿದರೆ, ಅಂಥವನು ಇಷ್ಟಪಟ್ಟರೆ ಒಂದು ವಸ್ತು ಉಂಟುಮಾಡುವಂತೆ ನೋಡುವವರಿಗೆ ತೋರುತ್ತದೆ. ಆದರೆ ನಿಜಕ್ಕೂ ಅದು ಉಂಟಾದ ವಸ್ತುವೇ ಅಲ್ಲ. ಸಾಮಾನ್ಯವಾದ ಇಂದ್ರಜಾಲಗಾರರಿಗೆ ಒಂದು ಮೂಲವಸ್ತುವೂ ಇಲ್ಲದೆ ಸೃಷ್ಟಿಮಾಡುವ ನೈಪುಣ್ಯವಿರುವಾಗ, ಇಷ್ಟು ದೊಡ್ಡ ಪ್ರಪಂಚವನ್ನು ಸೃಷ್ಟಿಮಾಡಿದ ನಿಪುಣನಾದ ಭಗವಂತನಿಗೆ ಮೂಲವಸ್ತು ಇಲ್ಲದೆ ಸೃಷ್ಟಿಮಾಡಲು ನೈಪುಣ್ಯವಿಲ್ಲವೇ?

ಭಗವಂತನೇ ಗೀತೆಯಲ್ಲಿ.

\begin{shloka}
`ಪ್ರಕೃತಿಂ ಸ್ವಾಮಧಿಷ್ಠಾಯ ಸಂಭವಾಮ್ಯಾತ್ಮಮಾಯಮಾ'
\end{shloka}

(ನನ್ನ ಪ್ರಕೃತಿಯನ್ನು ವಶಪಡಿಸಿಕೊಂಡು ನನ್ನ ಮಾಯೆಯಿಂದಲೇ ಅವತರಿಸುತ್ತೇನೆ.)' ಎಂದಿದ್ದಾನೆ. ಭಗವಂತನಶಕ್ತಿ ಚಿಂತನೆಗೆ ಸಿಕ್ಕುವುದಿಲ್ಲ. ಅಂಥ ಭಗವಂತನಶಕ್ತಿಯಿಂದ ಇಷ್ಟು ಸೃಷ್ಟಿಯನ್ನು ಅವನು ಮಾಡಬಲ್ಲನು. ಆದ್ದರಿಂದಲೇ ಅವನನ್ನು ಪಡೆಯಲು ಯಾವ ಉಪಾಯವೆಂದು ನಾಸ್ತಿಕನು ಕೇಳಬೇಕಾಗಿಲ್ಲ.

\begin{shloka}
`ಯತೋ ಮಂದಾಸ್ತ್ವಾಂ ಪ್ರತ್ಯಮರವರ ಸಂಶೇರತೇ ಇಮೇ ||'
\end{shloka}

ಎಂದು ಶ್ಲೋಕದ ಕೊನೆಯಲ್ಲಿ ಬರುತ್ತದೆ. ಭಗವಂತನ ವಿಷಯದಲ್ಲಿ ಮಂದಬುದ್ಧಿಯುಳ್ಳವನಿಗೆ ಮಾತ್ರ ಸಂದೇಹ ಉಂಟಾಗುತ್ತದೆ. ಒಬ್ಬನು ಹಲವು ಜನ್ಮಗಳಿಂದ ಬರುವ ಕೆಟ್ಟ ಸಂಸ್ಕಾರಗಳನ್ನು ಇಟ್ಟುಕೊಂಡು ವೇದಾಂತವನ್ನು ಕೇಳಿದರೆ, ಎಷ್ಟುಮಂದಿ ಹಾಗೆ ಬ್ರಹ್ಮಜ್ಞಾನವನ್ನು ಪಡೆಯಬಹುದು? ಅಂಥವರು ಬ್ರಹ್ಮಜ್ಞಾನವನ್ನು ಪಡೆದರೆಂದು ಹೇಳುವುದಕ್ಕೆ ಬದಲಾಗಿ ಹಣವನ್ನು ಪಡೆದರೆಂದು ಹೇಳುವುದು ಒಳ್ಳೆಯದು. ಭಗವತ್ಪ್ಪಾದರು.

\begin{shloka}
`ವಾಗ್ವೈಖರೀ ಶಬ್ದಝರೀ ಶಾಸ್ತ್ರವ್ಯಾಖ್ಯಾನ ಕೌಶಲಮ್ |\\
ವೈದುಷ್ಯಂ ವಿದುಷಾಂ ತದ್ವತ್ ಭುಕ್ತಯೇ ನ ತು ಮುಕ್ತಯೇ ||'
\end{shloka}

(ಸರಳವಾಗಿ ಮಾತನಾಡುವ ರೀತಿ, ನಿರರ್ಗಳವಾಗಿ ಶಬ್ದಗಳನ್ನು ಬಳಸುವ ನೈಪುಣ್ಯ, ಶಾಸ್ತ್ರಗಳನ್ನು ವ್ಯಾಖ್ಯಾನ ಮಾಡುವ ಕೌಶಲ, ವಿದ್ವಾಂಸರ ವಿದ್ವತ್ತು, ಹಾಗೆಯೇ (ವೀಣೆಯನ್ನು ನುಡಿಸಿ ಕೀರ್ತಿ, ಹಣ ಮುಂತಾದವುಗಳನ್ನು ಪಡೆಯುವುದು ಭೋಗವನ್ನು ಪಡೆಯುವುದಕ್ಕೇ ಹೊರತು ಸಂಸಾರ ಬಂಧನದಿಂದ ಬಿಡುಗಡೆ ಪಡೆಯಲು ಸ್ವಲ್ಪವೂ ಉಪಯೋಗಕ್ಕೆ ಬರುವುದಿಲ್ಲ.)' ಎಂದು (`ವಿವೇಕ ಚೂಡಾಮಣಿ'ಯಲ್ಲಿ) ಹೇಳಿದಂತೆ ಭುಕ್ತಿಗೆ ಸಾಧನವಾಗಬಹುದು ಮುಕ್ತಿಗೆ ಸಾಧನವಾಗುವುದಿಲ್ಲ. ಏಕೆಂದರೆ,

\begin{shloka}
`ಅಧಿಕಾರಿಣಮಾಶಾಸ್ತೇ ಫಲಸಿದ್ಧಿರ್ವಿಶೇಷತಃ'\\
(ಫಲಸಿದ್ಧಿ ಮುಖ್ಯವಾಗಿ ಅಧಿಕಾರಿಯನ್ನು ಎದುರು ನೋಡುತ್ತದೆ.)
\end{shloka}

ಒಬ್ಬನು ಒಂದು ಅರ್ಜಿಯನ್ನು ಇಂಜಿನಿಯರ್ ಕೆಲಸಬೇಕೆಂದು ಹೇಳಿದ್ದರಿಂದ ಕಳುಹಿಸಿದ್ದನು. ಅವನು ಕಾಲೇಜಿನಲ್ಲಿ ಓದಿರಲಿಲ್ಲ. ಆದರೆ ಅರ್ಜಿ ಮಾತ್ರ ಕೊಟ್ಟನು. ಅರ್ಜಿಯಲ್ಲಿ ಮೆಟ್ರಿಕ್ಯುಲೇಷನ್ ಮಾತ್ರ ಪಾಸ್ ಮಾಡಿರುವುದಾಗಿ ತಿಳಿಸಿದ್ದರೆ ಇಂಜಿನಿಯರ್ ಕೆಲಸ ಕೊಡುವರೇ? ಕೊಡುವುದಿಲ್ಲ.

ಆದ್ದರಿಂದ ಒಬ್ಬನಿಗೆ ವಿವೇಕ, ವೈರಾಗ್ಯ, ಶಮಾದಿ ಷಟ್ಕಸಂಪತ್ತಿ, ಮುಮುಕ್ಷುತ್ವ, ಎನ್ನುವುವು ಬೇಕು. ನಮಗೆ ಮುಮುಕ್ಷತ್ವವಿದೆ. ಯಾವಾಗ? ಯಾವುದಾದರೂ ದುಃಖ ಉಂಟಾದರೆ `ಅಯ್ಯೋ ಈ ಪ್ರಪಂಚವೇ ಬೇಡ' ಎಂದು ನಮಗೆ ತೋರುತ್ತದೆ. ಅದು ಸ್ವಲ್ಪ ದೂರವಾದರೆ, `ಇನ್ನು ನಾಲ್ಕು ವರ್ಷಗಳು ಪ್ರಪಂಚದಲ್ಲಿ ಇರಲಾರವೇ' ಎಂದು ತೋರುತ್ತದೆ. ಇದು ಮುಮುಕ್ಷತ್ವವೇ ಅಲ್ಲ. ಉತ್ತಮವಾದ ಮುಮುಕ್ಷುತ್ವಕ್ಕೆ,

\begin{shloka}
`ಪ್ರದೀಪ್ತಶಿರಃ ಜಲಂ ಪ್ರವಿವಿಕ್ಷುರಿವ'\\
(ತಲೆಗೆ ಬೆಂಕಿ ಹಚ್ಚಿಕೊಂಡವನು ತಣ್ಣೀರಿನಲ್ಲಿ ಪ್ರವೇಶಿಸುವಂತೆ)
\end{shloka}

ಎಂದು ಉದಾಹರಣೆ ಕೊಡಲ್ಪಟ್ಟಿದೆ.

ಮನೆಗೆ ಬೆಂಕಿ ಹಚ್ಚಿಕೊಂಡರೆ ಬಿಂದಿಗೆಯಲ್ಲಿ ನೀರು ತಂದು ಹಾಕುತ್ತೇವೆ ಬಟ್ಟೆಗೆ ಬೆಂಕಿ ಹತ್ತಿಕೊಂಡರೆ ಒಡನೆ ಬಟ್ಟೆಯನ್ನು ತೆಗೆದು ಬಿಡುತ್ತೇವೆ. ಆದರೆ ತಲೆಗೆ ಬೆಂಕಿ ಹಚ್ಚಿಕೊಂಡರೆ ಕೊಳವನ್ನು ಕಂಡ ಒಡನೆ, ಈಜುವುದಕ್ಕೆ ಬರುವುದೋ, ಇಲ್ಲವೋ, ಅದರಲ್ಲಿ ಹಾರಿಬಿಡುವೆವು. ಏಕೆಂದರೆ, `ಶರೀರಕ್ಕೆ ಬೆಂಕಿ ಹಚ್ಚಿಕೊಂಡಿದೆ, ಆದ್ದರಿಂದ ಮೊದಲು ಈ ಬೆಂಕಿಯನ್ನು ದೂರ ಮಾಡಿದರೆ ಆನಂತರ ಏನಾದರೂ ಪರವಾಗಿಲ್ಲ.' ಎಂದು ಇರುವೆವು.

ಇಂಥ ಮುಮುಕ್ಷುತ್ವವಿರುವವನು ಉತ್ತಮ ಅಧಿಕಾರಿಯಾಗುತ್ತಾನೆ. `ಹಾಗಾದರೆ ಇತರರು ವೇದಾಂತ ಶ್ರವಣ ಮಾಡಿದರೆ ಏನೂ ಫಲವಿಲ್ಲವೇ? - ಎಂದು ಕೇಳಿದರೆ ಅದು ಎಂದೂ ಫಲವಿಲ್ಲದೆ ಹೋಗುವುದಿಲ್ಲ.

\begin{shloka}
`ದಿನೇ ದಿನೇ ಚ ವೇದಾನ್ತಶ್ರವಣಾತ್ ಭಕ್ತಿಸಂಯುತಾತ್\\
ಗುರುಶುಶ್ರೂಷಯಾ ಲಭ್ಯಾತ್'............
\end{shloka}

ದಿನವೂ ಭಕ್ತಿಯೂ, ವೇದಾಂತ ಶ್ರವಣವೂ ಕೂಡಿದ ಗುರುಸೇವೆ ಮಾಡಿದರೆ (ಒಳ್ಳೆಯದಾಗುತ್ತದೆ.)

ನಾವು ಅನೇಕ ಪ್ರಾಯಶ್ಚಿತ್ತಗಳನ್ನು ಮಾಡುವುದುಂಟು. ಯಾವುದೋ ಒಂದು ಪಾಪವನ್ನು ಮಾಡಿದ್ದರಿಂದಲೇ ದುಃಖವನ್ನು ಅನುಭವಿಸುತ್ತೇವೆಂದು ತೋರುವಾಗ ಪ್ರಾಯಶ್ಚಿತ್ತಗಳನ್ನು ಹೇಳುತ್ತೇವೆ. ವೇದಾಂತಶ್ರವಣವೆನ್ನುವುದು ಪಾಪಗಳನ್ನು ಹೋಗಲಾಡಿಸುವ ಒಂದು ಪ್ರಾಯಶ್ಚಿತ್ತವೆಂದು ಹೇಳಲಾಗಿದೆ. ಅದನ್ನು ಹೇಗೆ ಮಾಡುವುದು? ಅನೇಕ ಮಂದಿ ವೇದಾಂತ ಶ್ರವಣ ಮಾಡುತ್ತಾರೆ. ನೀಲಕಂಠ ದೀಕ್ಷಿತರು ಬಹಳ ಸ್ವಾರಸ್ಯವಾಗಿ ಒಂದು ಕಡೆ,

\begin{shloka}
`ಅಕೌಮಾರಾತ್ ಗುರುಚರಣ ಶುಶ್ರೂಷಯಾ ಬ್ರಹ್ಮವಿದ್ಯಾ-\\
ಸ್ವಾಸ್ಥ್ಯಾಯಾಯಾಸ್ಥಾಮಹ ಮಹಿತೀಮರ್ಜಿತಂ ಕೌಶಲಂ ಯತ್ |\\
ನಿದ್ರಾಹೇತೋಃ ನಿಶಿ ನಿಶಿ ಕಥಾಃ ಶೃಣ್ವತಾಂ ಪಾರ್ಥಿವಾನಾಮ್\\
ಕಾಲಕ್ಷೇಪೌಪಯಿಕಮಿದಮಪ್ಯಾಃ ಕಥಂ ಪರ್ಯಣಂಸೀತ್ ||'
\end{shloka}

ಎನ್ನುತ್ತಾರೆ. ಶ್ಲೋಕದ ಅಭಿಪ್ರಾಯವನ್ನು ವಿವರವಾಗಿ ನೋಡೋಣ.

`ಚಿಕ್ಕ ವಯಸ್ಸಿನಿಂದಲೇ ಒಳ್ಳೆಯ ಗುರುವಿಗೆ ಶುಶ್ರೂಷೆಗಳನ್ನು ಮಾಡಿ ಶಾಸ್ತ್ರದಲ್ಲಿ ಒಳ್ಳೆಯ ಪಾಂಡಿತ್ಯವನ್ನು ಸಂಪಾದಿಸಿಬಿಟ್ಟನು. ಆ ಒಳ್ಳೆಯ ಜ್ಞಾನವನ್ನು ಯಾವುದಕ್ಕೆ ಉಪಯೋಗಪಡಿಸಿಕೊಳ್ಳುತ್ತಿದ್ದೀನಿ? `ರಿಟೈರ್' ಆದವರಿಗೆ ವಯಸ್ಸಾದ ಕಾರಣದಿಂದಾಗಿ ದುಃಖ ಉಂಟಾಗುವುದಿಲ್ಲ. ಆಗ ಅವರು ಸತ್ಕಥಾಕಾಲಕ್ಷೇಪಕ್ಕೆ ಹೋಗಲು ಯೋಚಿಸುತ್ತಾರೆ. ಸತ್ಕಥಾಕಾಲಕ್ಷೇಪಕ್ಕೆ ಹೋಗಲು ಶಕ್ತಿ ಇಲ್ಲದವನು ಮಲಗಿಕೊಂಡೋ ಕುಳಿತುಕೊಂಡೋ ಇರುವನು. ಆಗ ಅವನು ವೇದಾಂತವನ್ನು ಹೇಳಿ' ಎಂದರೆ ನಾನು ಹೇಳುತ್ತಲೇ ಇರುವೆನು. ಏತಕ್ಕೆಂದರೆ ಅವನಿಗೆ ದುಃಖ ಬರುವುದಿಲ್ಲ. ಬೇರೆ ಔಷಧಗಳು ಬಲವನ್ನು ಕೊಡದಿದ್ದರೂ ಈ ವೇದಾಂತ ಶ್ರವಣ ಅವನಿಗೆ ಸ್ವಲ್ಪ ಬಲವನ್ನು ಕೊಡುತ್ತದೆ. ಆದ್ದರಿಂದ ವ್ಯರ್ಥವಾಗಿ ಸಮಯವನ್ನು ಕಳೆಯುವುದಕ್ಕೆ ಒಂದು ಸಾಧನವಾಗಿ ಬಿಟ್ಟಿತು ವೇದಾಂತ' ಎನ್ನುತ್ತಾರೆ ಅವರು. ಆದ್ದರಿಂದ,

\begin{shloka}
`ಅಧಿಕಾರಣಾಮಾಶಾಸ್ತೇ ಫಲಸಿದ್ಧಿರ್ವಿಶೇಷತಃ'
\end{shloka}

-ಎಂದು ಭಗವಂತನು ಹೇಳಿದಂತೆ ಅಧಿಕಾರವೆನ್ನುವುದು ಇರಬೇಕು. ನಾವು ಪರಮಶಿವನ ಸಾಕ್ಷಾತ್ಕಾರವನ್ನು ಪಡೆಯಬೇಕೆಂದುಕೊಂಡರೆ ಏನು ಮಾಡಬೇಕು? ಸ್ವಾರಾಜ್ಯಸಿದ್ಧಿ ಎನ್ನುವ ಗ್ರಂಥದಲ್ಲಿ.

\begin{shloka}
`ಸ್ಮಾರಂ ಸ್ಮಾರಂ ಜನಿಪದನುಜಂ ಜಾತ ನಿರ್ವೇದವೃತ್ತಿಃ'
\end{shloka}

-ಎಂದು ಹೇಳಲ್ಪಟ್ಟಿದೆ. `ಜನಿಪದಂ' ಎಂದರೆ ಪ್ರಪಂಚ. ಅದರಲ್ಲಿರುವ ದುಃಖವನ್ನು ಕಂಡು ಪ್ರಪಂಚದಲ್ಲಿ ಅನುಭವಿಸಿದುದು ಸಾಕು ಎನ್ನುವ ತೃಪ್ತಿ ಬರಬೇಕು. ಆದ್ದರಿಂದಲೇ ಮೊದಲು ಬ್ರಹ್ಮಚಾರಿಯಾಗಿದ್ದು, ಗೃಹಸ್ಥನಾಗಿದ್ದು, ವಾನಪ್ರಸ್ಥನಾಗಿದ್ದು, ಆನಂತರ ಸಂನ್ಯಾಸವನ್ನು ತೆಗೆದುಕೊಳ್ಳಬಹುದು ಎನ್ನುವುದು ಒಂದು ದಾರಿ ಇದೆ. ಒಬ್ಬನು ಪ್ರಪಂಚವನ್ನು ಕಂಡಿದ್ದೇ ಇಲ್ಲ. ಆದರೆ ಸಂನ್ಯಾಸವನ್ನು ತೆಗೆದುಕೊಂಡು ಬಿಟ್ಟನು. ಅನಂತರ ಪ್ರಪಂಚದಲ್ಲಿರುವುದನ್ನು ನೋಡಿ ಚಪಲ ಉಂಟಾದರೆ ಅಂಥ ಸಂನ್ಯಾಸದಿಂದ ಏನು ಪ್ರಯೋಜನ? ಮೇಲೆ ಇದ್ದವನು ಕೆಳಗೆ ಬಿದ್ದಂತಾಗುತ್ತದೆ. ಈ ಕಾಲದಲ್ಲಿ ಹೀಗಾಗುವುದು ತಪ್ಪೆಂದೇ ಭಾವಿಸಲ್ಪಡುವುದಿಲ್ಲ. ಏಕೆಂದರೆ ಮೊದಲು `ಗವರ್ನರ್'ಆಗಿದ್ದ ಒಬ್ಬನು `ಮಿನಿಷ್ಟರ್' ಆಗಿ `ಕಮೀಷನರ್' ಆಗಿ, ಕೊನೆಯಲ್ಲಿ ಸಾಮಾನ್ಯವಾದ ಒಂದು ಕಮಿಟಿಯಲ್ಲಿ ಮೆಂಬರ್ ಆಗುತ್ತಾನೆ. ಹಿಂದಿನ ಕಾಲದಲ್ಲಿ ಮೇಲಕ್ಕೆ ಹೋದವರು ಕೆಳಗೆ ಬರುತ್ತಲೇ ಇರಲಿಲ್ಲ. `ರಿಟೈರ್' ಆದರೆ ಬೇರೆ ಕೆಲಸಕ್ಕೆ ಹೋಗುತ್ತಿರಲಿಲ್ಲ. ಏಕೆಂದರೆ, `ನಾನು ಡೆಪ್ಯುಟಿ ಕಮಿಷನರ್ ಆಗಿ ಇದ್ದವನು! ಈ ಕೆಳಮಟ್ಟದ ಕೆಲಸಕ್ಕೆ ಹೋದರೆ ಯಾರಾದರೂ ಏನಾದರೂ ಭಾವಿಸುತ್ತಾರೆ' ಎನ್ನುವ ಭಯ. ಈ ಕಾಲದಲ್ಲಿ ಇಂಥ ಭಯವಿಲ್ಲ. ಹೇಗಾದರೂ ಸಂಪಾದನೆ ಮಾಡಬೇಕೆನ್ನುವುದೇ ಗುರಿ. ಅಂದು ನ್ಯಾಯಾಧೀಶನಾಗಿದ್ದರೂ ಇಂದು ವಕೀಲನಾಗಿದ್ದರೆ ಏನು ನಷ್ಟ ಎಂದು ಒಬ್ಬನು ಯೋಚಿಸುತ್ತಾನೆ. `ಪ್ರಭುವೇ ಎಂದು ಇತರರು ನಮ್ಮನ್ನು ನೋಡಿ ಹೇಳಿದರು. ಈಗ ನಾವು ಇತರರನ್ನು ನೋಡಿ ಪ್ರಭುವೇ, ! ಎಂದು ಹೇಳುವುದರಲ್ಲಿ ಏನು ಕಷ್ಟ' ಎಂದು ಅವನು ಹೇಳುತ್ತಾನೆ. ಯಾರು ಪ್ರಭುವೋ, ಯಾರು ಪ್ರಭು ಅಲ್ಲವೋ ಭಗವಂತನಿಗೇ ಗೊತ್ತು. ಕುಳಿತಿದ್ದರೆ ಪ್ರಭು, ಕೆಳಗಿದ್ದರೆ ಆಳು ಅಲ್ಲವೇ! ಹಿಂದಿನ ಕಾಲದಲ್ಲಿ ಹಾಗಲ್ಲ. ಒಂದೊಂದು ಆಶ್ರಮವಾಗಿ ದಾಟಿಕೊಂಡು ಹೋಗುವವನು ಮೇಲಕ್ಕೆ ಹೋಗಬೇಕೇ ಹೊರತು ಕೆಳಕ್ಕೆ ಬರಕೂಡದು ಎನ್ನುವುದು ನಿಯಮ. ಆದ್ದರಿಂದ, 

`ಸ್ಮಾರಂ ಸ್ಮಾರಂ ಜನಿಪದನುಜಂ ಜಾತ ನಿರ್ವೇದವೃತ್ತಿಃ' ಎಂದು ಹೇಳಲ್ಪಟ್ಟಿದೆ.

\begin{shloka}
`ಪರೀಕ್ಷ್ಯ ಲೋಕಾನ್ ಕರ್ಮಚಿತಾನ್ ಬ್ರಾಹ್ಮಣೋ\\
ನಿರ್ವೇದಮಾಯಾನ್ನಾಸ್ತ್ಯ ಕೃತಃ ಕೃತೇನ |'
\end{shloka}

ಒಬ್ಬ ಬ್ರಾಹ್ಮಣನು ಕರ್ಮಗಳಿಂದ ಉಂಟಾಗುವ ಲೋಕಗಳನ್ನು ಚೆನ್ನಾಗಿ ಪರೀಕ್ಷೆ ಮಾಡಿ, (ಇಲ್ಲಿ) ಕರ್ಮದ ಫಲವಾಗಿ ಇಲ್ಲದುದು ಒಂದೂ ಇಲ್ಲ. ಆದ್ದರಿಂದ ಯಾವುದಕ್ಕಾಗಿ ಕರ್ಮ ಮಾಡಬೇಕು' ಎನ್ನುವ ಭಾವನೆಯಿಂದ ಸಂನ್ಯಾಸವನ್ನು ತೆಗೆದುಕೊಳ್ಳಬೇಕೆಂದು ವೇದದಲ್ಲಿ ಬರುತ್ತದೆ.

`ಆಕೃತ' ವೆಂದರೆ ಮೋಕ್ಷವೆಂದು ಅರ್ಥ. ಯಜ್ಞವನ್ನು ಮಾಡಿದರೆ ಸ್ವರ್ಗ ಲಭಿಸುತ್ತದೆ. ಸ್ವರ್ಗ ಲಭಿಸಿದರೆ ಅಲ್ಲಿ ಆನಂದವಾಗಿರಬಹುದು. ಇಲ್ಲ ಎಂದರೆ,

\begin{shloka}
`ಕ್ಷಿಣೀ ಪುಣೇ ಮರ್ತ್ಯಲೋಕೇ ವಿಶನ್ತಿ'\\
(ಪುಣ್ಯ ಮುಗಿಯುತ್ತಲೇ ಮನುಷ್ಯ ಲೋಕಕ್ಕೆ ಬರುತ್ತಾನೆ.)
\end{shloka}

-ಎಂದು ಹೇಳಿದಂತೆ, ಮೇಲೆ ಹೋಗಿ ಕುಳಿತಿರಬಹುದೆಂದು ಕುಳಿತುಕೊಂಡರೆ ಕೆಲವು ದಿನಗಳು ಕುಳಿಸಿರುತ್ತಾರಂತೆ. ಅನಂತರ ನಮ್ಮ ಫಲಗಳೆಲ್ಲಾ ಮುಗಿದು ಹೋಯಿತೆಂದು ಹೇಳಿ ನಮ್ಮನ್ನು ಕೆಳಕ್ಕೆ ಕಳುಹಿಸಿಬಿಡುತ್ತಾರೆ. ಆದ್ದರಿಂದ ಯಾವ ಕರ್ಮದಿಂದಲೂ ನಾವು ಪುರುಷಾರ್ಥವನ್ನು ಸಾಧಿಸಲು ಸಾಧ್ಯವಿಲ್ಲ. ಅದಕ್ಕೆ ಜ್ಞಾನಸಾಧನ. ಜ್ಞಾನ ಸ್ಥಿರವಾಗಿ ಹೇಗೆ ಉಂಟಾಗುತ್ತದೆ? ಅದಕ್ಕೆ ಮೊದಲು ವೈರಾಗ್ಯ ಬೇಕು. ವೈರಾಗ್ಯ ಹೇಗೆ ಬರುವುದು?

\begin{shloka}
`ಭುಕ್ತಾಬಹವೋ ದಾರಾ ಲಬ್ಧಾಃ ಪುತ್ರಾಶ್ಚ ಪೌತ್ರಾಶ್ಚ |\\
ನೀತಂ ಶತಮಪ್ಯಾಯುಃ ಸತ್ಯಂ ವದ ಮರ್ತುಮಸ್ತಿ ಮನಃ ||'
\end{shloka}

[ಹಲವು ಹೆಂಡತಿಯರನ್ನು ಅನುಭವಿಸಿ ಆಯಿತು, ಮಕ್ಕಳು-ಮೊಮ್ಮಕ್ಕಳು ಹುಟ್ಟು ಆಯಿತು. ನೂರು ವರ್ಷ ಜೀವಿಸಿದ್ದು ಆಯಿತು. ಆದರೆ ನಿಜ ಹೇಳು,  (ನಿನ್ನ) ಮನಸ್ಸು ಮರಣಕ್ಕೆ ಸಿದ್ಧವಾಗಿದೆಯೇ? (ಇಲ್ಲವೇ ಇಲ್ಲ.)] - ಎಂದು ನೀಲಕಂಠ ದೀಕ್ಷಿತರು ಹೇಳುತ್ತಾರೆ.

ಒಂದು ಮದುವೆಯಲ್ಲ, ಎರಡು ಮೂರು ಮಾಡಿಕೊಂಡಾಯಿತು. ಅನಂತರ ಮಕ್ಕಳು-ಮೊಮ್ಮಕ್ಕಳು ಎಲ್ಲರೂ ಹುಟ್ಟಿ ಆಯಿತು. ಹೀಗೆಯೇ ನೂರು ವರ್ಷಗಳು ಕಳೆದವು. `ಪ್ರಪಂಚದಲ್ಲಿ ಎಲ್ಲವನ್ನೂ ನೋಡಿ ಆಯಿತು. ಮತ್ತೆ ಏನು ಬೇಕು?' ಎಂದು ನಾವು ಅಂಥವನನ್ನು ಕೇಳಿದರೆ, `ಇನ್ನೂ ಎರಡು ವರ್ಷಬೇಕು' ಎನ್ನುತ್ತಾನೆ. ಭಗವಂತನು ಎರಡು ವರ್ಷ `{\eng Extension}' ಕೊಟ್ಟರೆ ಅದರಿಂದ ಒಂದು ತೃಪ್ತಿಯನ್ನು ಪಡೆಯುತ್ತಾನೆ,

\begin{shloka}
`ಜಾತಸ್ಯ ಹಿ ಧ್ರುವೋ ಮೃತ್ಯುಃ ಧ್ರುವಂ ಜನ್ಮ ಮೃತಸ್ಯ ಚ ||'
\end{shloka}

(ಹುಟ್ಟಿದವನಿಗೆ ಸಾವು ಖಂಡಿತ, ಸತ್ತವನಿಗೆ ಹುಟ್ಟು ಖಂಡಿತ.)

-ಎಂದು ದಿನವೂ ಹೇಳಿಕೊಂಡಿರುತ್ತೇವೆ. ಹಾಗೆ ಆಗುತ್ತಿರುವುದನ್ನು ದಿನವೂ ನೋಡುತ್ತಿದ್ದೇವೆ. ಪ್ರಪಂಚದಲ್ಲಿ ಆಶ್ಚರ್ಯಕರವಾದ ವಿಷಯ ಯಾವುದೆಂದು ಧರ್ಮರಾಜನನ್ನು ಕೇಳಿದಾಗ,

\begin{shloka}
`ಅಹನ್ಯಹನಿ ಭೂತಾನಿ ಪ್ರವಿಶನ್ತಿ ಯಮಾಲಯಮ್ |\\
ಶೇಷಾಃ ಸ್ಥಾವರಮಿಚ್ಛನ್ತಿ ಕಿಮಾಶ್ಚರ್ಯಮತಃ ಪರಮ್ ||'
\end{shloka}

(ದಿನವೂ ಪ್ರಾಣಿಗಳು ಯಮಲೋಕಕ್ಕೆ ಹೋಗುತ್ತಿವೆ; ಆದರೂ ಇತರರು ಸ್ಥಿರವಾಗಿರಬೇಕೆಂದು ಕೊಳ್ಳುವರು, ಇದಕ್ಕಿಂತಲೂ ಆಶ್ಚರ್ಯ ಯಾವುದು!)

-ಎಂದು ಹೇಳಿದನು. `ವಿಮಾನ ಬಂದುದು ದೊಡ್ಡ ಆಶ್ಚರ್ಯ' ಎನ್ನುತ್ತಾರೆ ಕೆಲವರು, ಕೆಲವರು `ರೇಡಿಯೋ ಬಂದುದು ಆಶ್ಚರ್ಯ' ಎನ್ನುತ್ತಾರೆ. ಕೆಲವರು `ಮನೆಯಲ್ಲಿಯೇ ಸಿನಿಮಾ ನೋಡಬಹುದು. ಇದು ಆಶ್ಚರ್ಯ' ಎನ್ನುತ್ತಾರೆ. ಧರ್ಮರಾಜನಿಗೆ ಇದೆಲ್ಲಾ ಆಶ್ಚರ್ಯವಾಗಿ ತೋರಲಿಲ್ಲ.

ಪ್ರತಿ ನಿತ್ಯವೂ ನಾವು ನೋಡುತ್ತಿದ್ದೇವೆ. ಹಲವರು ಯಮನ ಹತ್ತಿರಕ್ಕೆ ದಿನವೂ ಹೋಗುತ್ತಿದ್ದಾರೆ. ಅದನ್ನು ನೋಡಿದರೆ ನಮಗೆ, `ನಾವೂ ಒಂದು ದಿನ ಹೀಗೆ ಹೋಗಬೇಕಾಗುತ್ತದೆ' ಎಂದು ತೋರುತ್ತದೆ. ಆದರೆ ಯಾವುದೋ ಒಂದು ಒಳ್ಳೆಯ ಔಷಧವೋ, ಒಳ್ಳೆಯ ವೈದ್ಯರೋ ದೊರೆತರೆ ನಾವು ಸಾಯದೆ ಇರಬಹುದೆನ್ನುವ ಯೋಚನೆಯನ್ನು ಮಾತ್ರ ಮನಸ್ಸು ಮಾಡುತ್ತಲೇ ಇರುತ್ತದೆ. ನೀಲಕಂಠ ದೀಕ್ಷಿತರು ಹೇಳುತ್ತಾರೆ-

\begin{shloka}
ಚಕ್ಷುಸ್ವನ್ತೇ ಚಲತಿ ದಶನೇ ಶ್ಮಶ್ರುಣಿ ಶ್ವೇತಮಾನೇ\\
ಸೀದತ್ಯಂಗೇ ಮನಸಿ ಕಲುಷೇ ಕಂಪಮಾನೇ ಕರಾಗ್ರೇ |\\
ದೂತೈರೇತೈರ್ದಿನಕರಭುವಃ ಶಶ್ವದುದ್ಬೋಧ್ಯಮಾನಾಃ\\
ತ್ರಾತುಂ ದೇಹಂ ತದಪಿ ಭಿಷಜಾಮೇವ ಸಾಂತ್ವಂ ವದಾಮಃ ||
\end{shloka}

ವಯಸ್ಸಾಯಿತು. `ಚಕ್ಷುಸ್ವನ್ತೇ', ಕಣ್ಣುಗಳು ಸರಿಯಾಗಿ ಕಾಣುವುದಿಲ್ಲ. ಚೆನ್ನಾಗಿ ನೋಡುತ್ತಾನೆ, ಆದರೆ ಏನೂ ಗೊತ್ತಾಗುವುದಿಲ್ಲ. `ಚಲತಿ ದಶನೇ' ಹಲ್ಲುಗಳೆಲ್ಲವೂ ಅಲ್ಲಾಡುತ್ತವೆ. `ಶ್ಮಶ್ರುಣಿ ಶ್ವೇತಮಾನೆ' ಶರೀರದಲ್ಲಿರುವ ರೋಮಗಳು ನೆರತು ಹೋದವು. `ಸೀದತ್ಯಂಗೇ' ದೃಢವಾಗಿದ್ದ ಶರೀರ ಶಿಥಲವಾಯಿತು. ಏನಾದರೂ ಬರೆಯಬೇಕೆಂದುಕೊಂಡಾಗ ಪೆನ್ನು ಕೈಗೆ ತೆಗೆದುಕೊಳ್ಳುತ್ತಾನೆ. ಯಾವುದೋ ಒಂದು ಸಮಯದಲ್ಲಿ ಪೆನ್ನು ಕೆಳಗೆ ಇಟ್ಟು ಬಿಡುತ್ತಾನೆ. ಮತ್ತೆ ಬರೆಯಬೇಕೆಂದು ತೋರುತ್ತದೆ. ಪೆನ್ನು ಎಲ್ಲಿ ಇಟ್ಟು ಬಿಟ್ಟೆನೆಂದು ಹುಡುಕುತ್ತಾನೆ. ಹಾಗೂ ಹೀಗೂ ಹುಡುಕಿದ ಮೇಲೆ ಪೆನ್ನು ಇಲ್ಲೇ ಇಟ್ಟೆನೆಂದು ಜ್ಞಾಪಕಕ್ಕೆ ಬರುತ್ತದೆ. ವಯಸ್ಸಾಯಿತು ಎಂದರೆ ಎರಡು, ಮೂರು ಸಲ ಮಾಡಿದ ಕೆಲಸಗಳನ್ನೇ ಮಾಡಬೇಕಾಗಬಹುದು. ನಾವು ಅರ್ಘ್ಯ ಕೊಡುವಾಗ ಸಂಕಲ್ಪ ಮಾಡುತ್ತೇವೆ. ಹೀಗಿರುವಾಗ `ಓಂ ಭೂಃ' ಎಂದು ಆರಂಭಿಸುತ್ತೇವೆ. ಏಕಾಏಕಿ ``ಈಗ ಅರ್ಘ್ಯ ಕೊಡಬೇಕೇ? ಗಾಯಿತ್ರಿ ಜಪ ಮಾಡಬೇಡವೇ?' ಎನ್ನುವ ಸಂದೇಹ ಉಂಟಾಗುತ್ತದೆ. ಈ ಸಂದೇಹವನ್ನೇ ಹೇಳಿ `ನಾನು ಏನು ಮಾಡುವುದು' ಎಂದು ಕೆಲವರು ನನ್ನನ್ನು ಕೇಳಿದರು. ಈ ಮರೆಯುವಿಕೆಯಿಂದಾಗಿ ಸಂಧ್ಯಾವಂದನೆ ಮುಗಿಯುವುದು ಕೆಲವು ವೇಲೆ ಮುಕ್ಕಾಲು ಘಂಟೆ ಆದರೂ ಆಗಬಹುದು. ಕೆಲವು ವೇಳೆ ಸಂಧ್ಯಾವಂದನೆ ಮಾಡಿ ಆಯಿತು ಎಂದುಕೊಂಡು ಐದು ನಿಮಿಷಗಳಲ್ಲಿಯೇ ಒಬ್ಬನು ಎದ್ದು ಹೋದರೂ ಹೋಗುತ್ತಾನೆ. ಅನಂತರ ಗಡಿಯಾರವನ್ನು ನೋಡಿ `ಇಷ್ಟು ಬೇಗಮುಗಿಸಿ ಬಿಟ್ಟೆನಲ್ಲಾ' ಎಂದು ಭಾವಿಸುತ್ತಾನೆ. ಅದಕ್ಕೆ ನಾನು ಒಂದು ಉಪಾಯವನ್ನು ಹೇಳಿಕೊಟ್ಟೆನು. ಮೊದಲು ಜಪವಾದಮೇಲೆ ಉದ್ಧರಣೆಯಲ್ಲಿ ಸ್ವಲ್ಪ ನೀರು ತೆಗದಿಡಬೇಕು. ಉದ್ಧರಣೆಯಲ್ಲಿ ನೀರು ಇರುವವರೆಗೆ ಮಾಡಬೇಕಾದ ಕೆಲಸವೇ ಆಗುತ್ತಿದೆಯೆಂದು ತಿಳಿಯುತ್ತದೆ. ಅದು ಮುಗಿದ ಮೇಲೆ ನೀರನ್ನು ಹಾಕಿ ಬಿಡಬೇಕು. ಉದ್ಧರಣೆ ಖಾಲಿಯಾಗಿ ಇರುವುದರಿಂದ ಈ ಕರ್ಮ ಮುಗಿಯಿತೆಂದು ನಿರ್ಣಯ. ಇದೇ ರೀತಿ ಜಪಕ್ಕೆ ಮುಂಚೆ ಪೂರ್ವನ್ಯಾಸ ಮಾಡಬೇಕು. ಜಪ ಮಾಡಿದ ಮೇಲೆ ಉತ್ತರನ್ಯಾಸ ಮಾಡಬೇಕು. ಪೂರ್ವನ್ಯಾಸ ಮುಗಿಸುವುದಕ್ಕೆ ಮುಂಚೆ ಒಬ್ಬನು ನೆನಪಿಲ್ಲದೆ `ದಿಗ್ಭಂಧಃ' ಎಂದು ಹೇಳಬೇಕೇ `ದಿಗ್ವಮೋಕಃ' ಎಂದು ಹೇಳಬೇಕೆ? ಎಂದು ತಡವರಿಸುವುದು ಸಹಜ. ಇದನ್ನು ತಮಾಷೆಯಾಗಿ ಹೇಳುತ್ತಿಲ್ಲ. ವಯಸ್ಸಾದವನಿಗೆ ಅಷ್ಟು ನೆನಪಿಲ್ಲದೆ ಹೋಗುತ್ತದೆ. ಅದಕ್ಕೆ ಏನು ಮಾಡಬೇಕು? ಧ್ಯಾನ ಶ್ಲೋಕವನ್ನು ಹೇಳಲು ಪ್ರಾರಂಭಿಸಿದೊಡನೆಯೇ, ನ್ಯಾಸವನ್ನು ಪ್ರಾರಂಭಿಸುವುದಕ್ಕೆ ಮುಂಚೆ ಯಾವುದನ್ನಾದರೂ ಒಂದನ್ನು ಗುರುತಾಗಿ ಇಟ್ಟುಕೊಳ್ಳಬೇಕು. ಹಾಗೆ ಇಟ್ಟುಕೊಂಡರೆ, `ಭೂರ್ಭುವಸ್ಸುವರೋಂ ಇತಿ ದಿಗ್ಭಂಧಃ' ಎನ್ನುವುದರಲ್ಲಿ ಜಪ ಮುಗಿದಿಲ್ಲವೆಂದು ಜ್ಞಾಪಕಕ್ಕೆ ಬರುವುದು. ಜಪಮುಗಿದಾದ ಮೇಲೆ ಗುರುತನ್ನು ಪಕ್ಕದಲ್ಲಿಟ್ಟುಕೊಂಡಿರಬೇಕು. ಆಗ ಮತ್ತೆ ನ್ಯಾಸವನ್ನು ಪ್ರಾರಂಭಿಸುವಾಗ ``ಭೂರ್ಭುವಸ್ಸುವರೋಮ್ ಇತಿ ದಿಗ್ವಿಮೋಕಃ' ಎಂದು ಹೇಳಬೇಕೆಂದು ತಿಳಿಯುವುದು. ಹೀಗೆ ಧ್ಯಾನ ಶ್ಲೋಕವನ್ನು ಹೇಳುವಾಗಲೂ ಗುರುತು ಇಟ್ಟುಕೊಂಡರೆ ಸಂಧ್ಯಾವಂದನೆಯನ್ನಾಗಲಿ, ಜಪವನ್ನಾಗಲಿ ಸರಿಯಾಗಿ ಮಾಡಬಹುದು, ಹೀಗಿಲ್ಲದಿದ್ದರೆ ಒಬ್ಬನು ಮಾಡಿದ ಕರ್ಮವನ್ನೇ ಮತ್ತೆ ಮತ್ತೆ ಮಾಡುತ್ತಿರುತ್ತಾನೆ. ಇಲ್ಲದಿದ್ದರೆ ಮಾಡಿಬಿಟ್ಟೆನೆಂದುಕೊಂಡು ಜಪವನ್ನೇ ಮರೆತು ಬಿಡುತ್ತಾನೆ. ಕೆಲವರು ಪಂಚೋಪಚಾರ ಪೂಜೆಯನ್ನು ಮಾಡಿ (ಜಪ ಮುಗಿಸದೆಯೇ) ಮರೆತು `.......ಜಪಂ ಸಮರ್ಪಯಾಮಿ' ಎಂದು ಹೇಳಿ ಬಿಡುತ್ತಾರೆ. ಆದರೆ `ಇವನು ಜಪಮಾಡಲಿಲ್ಲವಲ್ಲಾ! ನಾನು ಯಾವುದನ್ನು ತೆಗೆದುಕೊಳ್ಳುವುದು?' - ಎಂದು ಭಗವಂತನು ಯೋಚನೆ ಮಾಡುವ ಸ್ಥಿತಿ ಉಂಟಾಗುತ್ತದೆ. ಅದೇ ರೀತಿ ಆಚಮನ ಮಾಡಬೇಕೆಂದು ನೀರು ತೆಗೆದುಕೊಂಡು ಹೋದರೆ ನೀರು ಇರುವುದು, ಕೈ ನಡುಗುತ್ತದೆ! ಏನು ಮಾಡುವುದು? ಮಂಗಳಾರತಿ ಮಾಡಬೇಕೆಂದುಕೊಂಡರೆ ಕರ್ಪೂರ ಕೆಳಗೆ ಬಿದ್ದುಹೋಗುತ್ತದೆ. ಹೀಗೆ ಅನುಭವಿಸುವ ಸ್ಥಿತಿ ಉಂಟಾಗುತ್ತದೆ. ವಯಸ್ಸಾಗಿರುವುದರಿಂದ ಯಮನು ಒಬ್ಬ ದೂತನನ್ನು ಕಳುಹಿಸಿದ್ದಾನೆ. ಯಮಲೋಕಕ್ಕೆ ಬರಬೇಕಾದ ಸಮಯ ಆಯಿತೆಂದು ತೋರಿಸುವುದಕ್ಕಾಗಿ ಕಳುಹಿಸಿದ್ದಾನೆ. ಆಗಲೂ ಮನುಷ್ಯನು ತನ್ನ ಶ್ರೇಯಸ್ಸಿಗೆ ಏನು ಬೇಕೆಂದು ನೋಡದೆ ಇದ್ದರೆ ಇನ್ನೊಬ್ಬ ದೂತನನ್ನು ಕಳುಹಿಸುತ್ತಾನೆ. ಹೀಗೆ ನಾಲ್ಕು ಐದು ದೂತರುಗಳನ್ನು ಕಳುಹಿಸಿದರೂ ಕೂಡ ಆ ಮನುಷ್ಯನ ಯೋಚನೆಯೇ ಬೇರೆ. `ಬೆಂಗಳೂರಿನಲ್ಲಿರುವ ಡಾಕ್ಟರ್ ಉಪಯೋಗವಿಲ್ಲ. ಮದ್ರಾಸಿನಲ್ಲಿರುವ ಡಾಕ್ಟರ್ ಬಹಳ ಬುದ್ಧಿವಂತ. ಅವರಿಗಿಂತಲೂ ಕಲ್ಕತ್ತಾದಲ್ಲಿರುವ ಡಾಕ್ಟರ್ ಬುದ್ಧಿವಂತ. ಅವರಿಗಿಂತಲೂ ಲಂಡನ್ನಿನಲ್ಲಿರುವ ಡಾಕ್ಟರ್ ಬುದ್ಧಿವಂತ' ಎಂದೇ ಅವನು ಯೋಚಿಸುತ್ತಾನೆ. ಹೀಗೆ ಅವನು ಓಡುತ್ತಿದ್ದರೆ ಆ ಡಾಕ್ಟರ್ ಎಲ್ಲಿಗೆ ಓಡುತ್ತಾರೆ? ಅವರಿಗೂ ಅದೇ ರೀತಿ ಆಗುವಾಗ ಅವರು ಎಲ್ಲಿಗೆ ಓಡುವರು? ಆದ್ದರಿಂದ ಎಲ್ಲಿಗೂ ಯಾರೂ ಓಡಬೇಕಾಗಿಲ್ಲ. ನಮ್ಮ ಶರೀರ ಅಳಿಯುವಂತಹದು ಎನ್ನುವ ನೆನಪು ನಮಗೆ ಇದ್ದರೆ ಸಾಕು. ಒಂದು ಗಡಿಯಾರವನ್ನು ನಾಲ್ಕು ಅಥವಾ ಐದು ಸಾರಿ ಸರಿ ಮಾಡಿಸಬಹುದು. ಮತ್ತೆ ಮತ್ತೆ ಅಂಗಡಿಯವನ ಹತ್ತಿರ ತೆಗೆದುಕೊಂಡು ಹೋಗುತ್ತಿದ್ದರೆ ಅವನು, `ಇದನ್ನು ತರುತ್ತಿದ್ದೀರಲ್ಲಾ! ನಾನು ಎಷ್ಟು ಸಲ ರಿಪೇರಿ ಮಾಡುವುದು' ಎನ್ನುತ್ತಾನೆ.

\begin{shloka}
`ತಥಾ ಶರೀರಾಣಿ ವಿಹಾಯ ಜೀರ್ಣಾನ್ಯನ್ಯಾನಿ ಸಂಯಾಂತಿ ನವಾನಿ\\
\hspace{5.5cm} ದೇಹೀ ||'
\end{shloka}

(ಅದೇ ರೀತಿ, ಆತ್ಮನು ಹಳೆಯ ಶರೀರವನ್ನು ಬಿಟ್ಟು ಬಿಟ್ಟು ಹೊಸ ಶರೀರವನ್ನು ಪ್ರವೇಶಿಸುವುದು.)


-ಎಂದು ಭಗವಂತನು ಹೇಳಿದಂತೆ ಇದನ್ನು ಬಿಟ್ಟು ಬಿಟ್ಟರೂ ಇನ್ನೊಂದು ಶರೀರ ಬರುತ್ತದೆ. ಆ ಧೈರ್ಯದಿಂದ ಇದ್ದರೆ ಆಗ ವಿರಕ್ತಿ ಇದೆಯೆಂದು ಅರ್ಥ. `ನಿರ್ವೇದ' ಬಂದ ಮೇಲೆ ಏನು ಮಾಡಬೇಕು?

\begin{shloka}
`ಧ್ಯಾಯ ಧ್ಯಾಯಂ ಪಶುಪತಿಮುಮಾಕಾಂತಮನ್ತರ್ನಿಷಣ್ಣಮ್'
\end{shloka}

ಭಗವಂತನನ್ನು ಹುಡುಕಿಕೊಂಡು ಎಲ್ಲಿಗೋ ದೇವಾಲಯಕ್ಕೆ ಹೋಗಬೇಕಾಗಿಲ್ಲ. ಸ್ವಲ್ಪ ಹೊತ್ತು ಭಗವಂತನ ಚಿತ್ರವನ್ನು ನಿರ್ನಿಮೇಷ ನೇತ್ರಗಳಿಂದ ನೋಡಿ ಸ್ಥಿರವಾದ ಚಿತ್ತದಿಂದ ಮನಸ್ಸಿನಲ್ಲೇ ಆ ಮೂರ್ತಿ ಬಂದಂತೆ ಭಾವನೆ ಮಾಡಿಕೊಳ್ಳಬೇಕು. ಆನಂತರ ಕಣ್ಣುಗಳನ್ನು ಮುಚ್ಚಿಕೊಳ್ಳಬೇಕು. ಹೀಗೆ ಚಿಂತನೆ ಮಾಡಿದರೆ,

\begin{shloka}
`ಪಶುಪತಿಮುಮಾಕಾನ್ತಮನ್ತರ್ನಿಷಣ್ಣಮ್'
\end{shloka}

ಎಂದು ಹೇಳಿದಂತೆ ಭಗವಂತನು ಒಳಗೆ ಇದ್ದಾನೆಂದು ತೋರುವುದು.

\begin{shloka}
`ಈಶ್ವರಃ ಸರ್ವಭೂತಾನಾಂ ಹೃದ್ದೇಶೇರ್ಜುನ ತಿಷ್ಠತಿ |\\
ಭ್ರಾಮಯನ್ ಸರ್ವಭೂತಾನಿ ಯಂತ್ರಾರೂಢಾನಿ ಮಾಯಯಾ ||'
\end{shloka}

(ಭಗವಂತನು ಎಲ್ಲಾ ಜೀವಿಗಳ ಹೃದಯದಲ್ಲೂ ಇದ್ದಾನೆ. ಅರ್ಜುನ! ಅವನ ಮಾಯೆಯಿಂದಾಗಿ ಯಂತ್ರಾರೂಢರಂತೆ ಎಲ್ಲಾ ಜೀವಿಗಳನ್ನು ಭ್ರಮಿಸುವಂತೆ ಮಾಡುತ್ತಾನೆ.)

-ಎಂದು ಹೇಳಿದಂತೆ ಅವನನ್ನು ಚಿಂತನೆ ಮಾಡಬೇಕು. ಚಿಂತನೆ ಮಾಡಿದರೆ ಏನು ಪ್ರಯೋಜನ?

\begin{shloka}
`ಪಾಯಂ ಪಾಯಂ ಸಪರಿಚಾರಮಾನಂದ ಪೀಯೂಷಧಾರಾಮ್ ||
\end{shloka}

ಒಬ್ಬನು ಅಮೃತವನ್ನು ಪಡೆಯಲು ಸಾಧ್ಯ. ಏಕೆಂದರೆ ಆ ಆನಂದ ಬರುತ್ತಲೇ ಇರುವುದು. ಹೀಗೆ ನಾವು ಪರಮ ಶಿವನನ್ನು ಧ್ಯಾನಮಾಡಿ ಆ ಆನಂದವನ್ನು ಅನುಭವಿಸುತ್ತಾ ಇದ್ದು ಎಲ್ಲದಕ್ಕೂ ಕಾರಣವಾಗಿರುವ ನಮ್ಮ ಗುರುಗಳ ಪಾದಾರವಿಂದಗಳನ್ನು ನಾನು ಚಿಂತಿಸುತ್ತೇನೆಂದು ಕವಿ ಹೇಳುತ್ತಾರೆ. ಆದ್ದರಿಂದ ಭಗವಂತನು ಇದ್ದಾನೆ ಎನ್ನುವುದಕ್ಕೆ ಅನುಭವವೇ ಸಾಕ್ಷಿ.

\begin{shloka}
`ಕ್ಷಿತ್ಯಾದೀನಾಂ ಅವಯವವತಾಂ ನಿಶ್ಚಿತಾಂ ಜನ್ಮತಾವತ್\\
ತನ್ನಾಸ್ತ್ಯೇವ ಕ್ವಚನ ಕಲಿತಂ ಕರ್ತ್ರಧಿಷ್ಠಾನ ಹೀನಮ್ |\\
ನಾಧಿಷ್ಠಾತುಂ ಪ್ರಭವತ ಜಡಃ ನಾಪ್ಯನೀಶಶ್ಚ ಭಾವಃ\\
ತಸ್ಮಾದಾದ್ಯಃ ತ್ವಮಸಿ ಜನತಾಂ ನಾಥ ಜಾನೇ ವಿಧಾತಾ ||'
\end{shloka}

(ಭೂಮಿಯಂಥವುಗಳಿಗೆ ಅವಯವಗಳಿರುವುದರಿಂದ ಜನ್ಮ ಇದೆ ಎನ್ನುವುದು ನಿಶ್ಚಯ. ಇಲ್ಲಿ ಕರ್ತೃ ಎನ್ನುವವನು ಒಬ್ಬನೂ ಇಲ್ಲ ಎನ್ನುವುದು ಇಲ್ಲ. ಮೂರ್ಖನಾಗಿರುವವನು ಅಧಿಷ್ಠಾನವನ್ನು ಹೇಳುವುದಿಲ್ಲ. ಈಶ್ವರನು ಎಂದೂ ಹೇಳುವುದಿಲ್ಲ. ಆದ್ದರಿಂದ ಮೊದಲಾದವನೂ ನೀನೇ! ಉಂಟು ಮಾಡಿದವನೂ ನೀನೇ! ಸ್ವಾಮಿಯೇ, ನಿನ್ನನ್ನು ಹಾಗೆ ನಾನು ತಿಳಿದುಕೊಂಡಿದ್ದೇನೆ.)

ಪುಷ್ಪದಂತನು ತನ್ನ ಶ್ಲೋಕದಲ್ಲಿ ಸುಲಭವಾಗಿ ಹೇಳಿದ ವಿಷಯಗಳು ಇಲ್ಲಿ ಹೇಳಲ್ಪಟ್ಟಿವೆ.

\begin{shloka}
`ಕ್ಷಿತ್ಯಾದೀನಾಂ ಅವಯವವತಾಂ ನಿಶ್ಚಿತಂ ಜನ್ಮ ತಾವತ್'\\
ಭೂಮಿಯಂಥವುಗಳೆಲ್ಲ ಉಂಟಾಗಿದೆ ಎನ್ನುವುದು ನಿಶ್ಚಯ,\\
`ತನ್ನಾಸ್ತ್ಯೇವ ಕ್ವಚನ ಕಲಿತಂ ಕರ್ತ್ರಧಿಷ್ಠಾನ ಹೀನಮ್ |'
\end{shloka}

ಅದಕ್ಕೆ ಅಧಿಷ್ಠಾನವಾದ (ಪ್ರಪಂಚಕ್ಕೆ ಆಧಾರವಾದ) ವಸ್ತು ಜಡವಾಗಿರಲು ಸಾಧ್ಯವಿಲ್ಲ. ಆದ್ದರಿಂದ.

\begin{shloka}
`ತಸ್ಮಾದಾದ್ಯಃ ತ್ವಮಸಿ ಜನತಾ ನಾಥ ಜಾನೇ ವಿಧಾತಾ ||'
\end{shloka}

ಎಂದು ಹೇಳಲ್ಪಟ್ಟಿದೆ.

\begin{shloka}
ಪರಮಾತ್ಮನು ಒಬ್ಬನೇ? ಅಥವಾ ಹಲವು ವಿಧವಾದವನೇ?\\[5pt]
ಇಂದ್ರಂ ಮಿತ್ರಂ ವರುಣಮನಿಲಂ ಪದ್ಮಜಂ ವಿಷ್ಣುಮೀಶಮ್\\
ಪ್ರಾಹುಸ್ತೇ ತೇ ಪರಮಶಿವ ತೇ ಮಾಯಯಾ ಮೋಹಿತಾಸ್ತ್ವಮ್ |\\[5pt]
ಏತೈಃ ಸಾರ್ಧಂ ಸಕಲಮಪಿ ಶಕ್ತಿಲೇಶೇ ಸಮಾಪ್ತಂ\\
ಸತ್ವಂ ದೇವಃ ಶ್ರುತಿಷು ಶಂಭುರಿತ್ಯಾದಿ ದೇವಃ ||
\end{shloka}

ಇಂದ್ರ, ಮಿತ್ರ, ವರುಣ, ಅನಿಲ, ಬ್ರಹ್ಮ, ವಿಷ್ಣು, ಈಶ ಎನ್ನುವುದೆಲ್ಲಾ ನಿನ್ನ ಮಾಯೆಯಿಂದ ಮೋಹಗೊಂಡು ಅವರವರು ಹೇಳುತ್ತಾರೆ. ಇವುಗಳೆಲ್ಲವೂ ನಿನ್ನ ಶಕ್ತಿಯ ಒಂದು ಕಣದಲ್ಲಿಯೇ ಅಡಗಿ ಹೋಗುವುದು. ಅಂಥ ನೀನು ವೇದಗಳಲ್ಲಿ `ಶಂಭುವಾದ ದೇವ'ನೆಂದು ಹೇಳಲ್ಪಟ್ಟಿದ್ದೀಯೆ)

-ಎಂದು ಶ್ಲೋಕದಲ್ಲಿ ಹೇಳಿರುವಂತೆ ಎಷ್ಟು ಹೆಸರುಗಳನ್ನು ಹೇಳಿದರೇನು? ನಾವು ಚಂದ್ರನನ್ನೋ, ವರುಣನನ್ನೋ, ಸುಬ್ರಹ್ಮಣ್ಯನನ್ನೋ, ಗಣಪತಿಯನ್ನೋ ಉಪಾಸನೆ ಮಾಡಿದರೂ ಪರಮಾತ್ಮನು ಒಬ್ಬನೇ. ಆದ್ದರಿಂದ ಭಗವಂತನು ಇದ್ದಾನೆನ್ನುವುದನ್ನು ತೀರ್ಮಾನವಾಗಿ ತಿಳಿದುಕೊಳ್ಳಬೇಕು. ಭಗವಂತನು ಕಣ್ಣಿಗೆ ಕಾಣುವುದಿಲ್ಲವೆನ್ನುವುದು ಹಲವರ ವಿರೋಧ. ಗಾಳಿಯನ್ನು ನೋಡಿದ್ದಿಲ್ಲ ಎನ್ನುವುದಕ್ಕಾಗಿ ಗಾಳಿ ಇಲ್ಲ ಎಂದು ಹೇಳಲಾಗುವುದಿಲ್ಲ. ಗಾಳಿ ಇದೆ ಎನ್ನುವುದನ್ನು ಹೇಗೆ ತಿಳಿದುಕೊಳ್ಳಬಹುದು? ಬಲವಾಗಿ ಮೂಗನ್ನೂ, ಬಾಯಿಯನ್ನೂ ಮುಚ್ಚಿಕೊಂಡರೆ ಆಗ ಗಾಳಿ ಇದೆಯೇ, ಇಲ್ಲವೇ ಎನ್ನುವುದು ತಿಳಿಯುವುದು. ಹಾಗೆಯೇ ಶರೀರ ಆರೋಗ್ಯವಾಗಿದ್ದು, ಎಲ್ಲಾ ಚೆನ್ನಾಗಿದ್ದರೆ ಭಗವಂತನು ಇರುವುದು ತಿಳಿಯುವುದಿಲ್ಲ. ಕಷ್ಟ ಬಂದಾಗ ಎಲ್ಲರಿಗೂ ಭಗವಂತ ಇದ್ದಾನೆಯೇ, ಇಲ್ಲ್ಲವೇ ತಿಳಿಯುವುದು. ಆ ಸ್ಥಿತಿಯಲ್ಲಿ ಒಬ್ಬನು `ಭಗವಂತನು ಹೇಗಿದ್ದಾನೋ ತಿಳಿಯದು. ಆದರೆ ಅವನು ಇದ್ದಾನೆ' ಎನ್ನುತ್ತಾನೆ. ಆದ್ದರಿಂದ ಭಗವಂತನು ಇದ್ದೇ ಇರುತ್ತಾನೆ. ಅನ್ಯಾಯ ಮಾಡುವವರು ಚೆನ್ನಾಗಿದ್ದಾರಲ್ಲಾ ಎಂದು ಕೆಲವರಿಗೆ ಸಂದೇಹ. ಅಂಥವರು ಕಾರಿನಲ್ಲಿ ಸುಖವಾಗಿ ಹೋಗುತ್ತಿದ್ದಾರಲ್ಲಾ ಎನ್ನುವುದು ಕೆಲವರಿಗೆ ಸಂದೇಹ. ನಿಜಕ್ಕೂ ಅವರಿಗೆ ಇರುವ ಮನೋವೇದನೆ ಹಲವಾರು. ಆದ್ದರಿಂದ ಭಗವಂತನಿದ್ದಾನೆ ಎನ್ನುವ ವಿಷಯದಲ್ಲಿ ಯಾರಿಗೂ ಸಂದೇಹವಿರಬಾರದು. `ಕುಸುಮಾಂಜಲಿ'ಯನ್ನು ಬರೆದವರು-

\begin{shloka}
`ಇತ್ಯೇವಂ ಶ್ರುತಿನೀತಿಸಂಪ್ಲವಜಲೈಃ ಭೂಯೋಭಿರಾಕ್ಷಾಲಿತೇ\\
ಯೇಷಾಂ ನೋಪತಮಾ ತಥಾಸಿ ಹೃದಯೇ ತೇ ಶೈಲಸಾರಾಶಯಾಃ |\\
ಕಿನ್ತು ಪ್ರಸ್ತುತ ವಿಪ್ರದೀಪಮತಯೋಭ್ಯುಚ್ಚೈಃ ಭವಚ್ಚಿನ್ತಕಾಃ\\
ಕಾಲೇ ಕಾರುಣಿಕ ತ್ವಯೈವ ಕೃಪಯಾ ತೇ ರಕ್ಷಣಿಯಾ ಜನಾಃ ||'
\end{shloka}

-ಎಂದಿದ್ದಾರೆ. ವೇದವೆನ್ನುವ ಪ್ರಮಾಣ ಭಗವಂತನಿದ್ದಾನೆಂದು ಹೇಳುತ್ತದೆ. ನೀತಿ ಎಂದರೆ ತರ್ಕವೆಂದು ಅರ್ಥ. ನಾವು ಕನ್ನಡಿಯಲ್ಲಿ ಮುಖವನ್ನು ನೋಡುತ್ತೇವೆ. ಸರಿಯಾಗಿ ಕಾಣಲಿಲ್ಲವೆಂದರೆ ಅದನ್ನು ಚೆನ್ನಾಗಿ ಒರೆಸಿ ನೋಡಬಹುದು.

\begin{shloka}
`ಶ್ರುತಿನೀತಿ ಸಂಪ್ಲವ ಜಲೈಃ ಭಯೋಭಿರಾಕ್ಷಾಲಿತೇ'
\end{shloka}

ನಾಸ್ತಿಕನ ಹೃದಯವನ್ನೂ ಶ್ರುತಿ-ನೀತಿ ಎನ್ನುವ ನೀರಿನಿಂದ ಚೆನ್ನಾಗಿ ತೊಳೆದು ಇಡುತ್ತೇನೆ. ಆದರೆ, ಕನ್ನಡಿ ಆದರೆ ಅದನ್ನು ನೀರಿನಲ್ಲಿ ತೊಳೆದರೆ ಮುಖವನ್ನು ಚೆನ್ನಾಗಿ ನೋಡಬಹುದು. ಅದನ್ನು ಬಿಟ್ಟು ಒಂದು ಕಪ್ಪಾದ ತುಂಡನ್ನು ತೆಗೆದುಕೊಂಡು ಅದನ್ನು ಚೆನ್ನಾಗಿ ಒರೆಸಿದರೆ, ಅದರಲ್ಲಿ ಮುಖವನ್ನು ನೋಡಿದರೆ, ಎಷ್ಟು ಒರೆಸಿದೆವೋ ಅಷ್ಟು ಅದು ಕಪ್ಪಾಗಿ ಬದಲಾಗುತ್ತಲೇ ಇರುವುದು. ಅದೇ ರೀತಿ-

\begin{shloka}
`ಯೇಷಾಂ ನೋಪತಮಾ ತಥಾಸಿ ಹೃದಯೇ ತೇ ಶೈಲಸಾರಾಶಯಾಃ'
\end{shloka}

ನಾಸ್ತಿಕನ ಹೃದಯ ಉಕ್ಕಿನಿಂದ ಮಾಡಲ್ಪಟ್ಟಿದ್ದು, ಆದರೂ ಅದು `{\eng Stainless Steel}' ಅಲ್ಲ, ಅದು ಕಪ್ಪಾಗಿ ಇರುತ್ತದೆ. ಅದನ್ನು ಎಷ್ಟು ಶುದ್ಧಿ ಪಡಿಸಿದರೂ ಅದು ಇನ್ನೂ ಕಪ್ಪಾಗಿಯೇ ಬದಲಾಗುತ್ತದೆ. ಅದರಲ್ಲಿ ಬಿಂಬವನ್ನು ಸರಿಯಾಗಿ ನೋಡಲಾಗುವುದಿಲ್ಲ. ನೀರು ಇರುವಾಗ ಯಾವುದೋ ಇರುವಂತೆ ತೋರುವುದು. ನೀರು ಹೋಯಿತೆಂದರೆ ಮತ್ತೆ ಏನೂ ಕಾಣುವುದಿಲ್ಲ. `ಆದರೆ ನಾನು ಉಳಿದುದನ್ನು ಮಾಡಿಬಿಟ್ಟೆನು. ಅವನಲ್ಲಿ (ನಾಸ್ತಿಕನಲ್ಲಿ) ಭಗವಂತನಾದ ನೀನು ಇರಬೇಕು. ಅವನೂ ನನ್ನಂತೆಯೇ `ನೀನು ಇಲ್ಲ ಎಂದು ಹೇಳುವುದರ ಮೂಲಕ ಚಿಂತನೆ ಮಾಡುತ್ತಲೇ ಇದ್ದಾನೆ. ನೀನು ಅವನನ್ನು ಮರೆತು ಬಿಡಕೂಡದು' ಎಂದು ನೀಲಕಂಠದೀಕ್ಷಿತರು ಹೇಳಿದರು. ಏಕೆಂದರೆ

\begin{shloka}
`ಸರ್ವಂ ವ್ಯರ್ಥಂ ಮರಣಸಮಯೇ ಸಾಂಬ ಏಕಃ ಸಹಾಯಃ'
\end{shloka}

(ಮರಣಕಾಲದಲ್ಲಿ ಎಲ್ಲವೂ ವ್ಯರ್ಥ, ನಿಜವಾದ ಭಗವಂತನೊಬ್ಬನೇ ಸಹಾಯಕನು.)

-ಎನ್ನುವಂತೆ ಐಶ್ವರ್ಯವೂ, ನೆಂಟರೂ ಇರುವಾಗ ಒಬ್ಬನಿಗೆ ಯಾರೂ ಬೇಡದಿದ್ದರೂ ಇವುಗಳು ಹೋದಮೇಲೆ ಅವನನ್ನು ಭಗವಂತನೇ ಕಾಪಾಡಬೇಕೆನ್ನುವುದು ತಾತ್ಪರ್ಯ. ಆದ್ದರಿಂದ ಭಗವಂತನಿದ್ದಾನೆ ಎನ್ನುವುದು ನಿರ್ಣಯ.



























































































































































































































































































































