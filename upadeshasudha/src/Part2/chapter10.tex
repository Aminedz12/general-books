\chapter{ಇಂದ್ರಿಯಾಧೀನತೆ ನಾಶಕ್ಕೆ ಕಾರಣ}\label{chap10}

\begin{shloka}
ಗಂಗಾಪೂರ ಪ್ರಚಲಿತ ಜಟಾಸ್ರಸ್ತ ಭೋಗೀಂದ್ರಭೀತಾಂ\\
ಅಲಿಂಗಂತೀಮಚಲತನಯಾಂ ಸಸ್ಮಿತಂ ವೀಕ್ಷಮಾಣಃ |\\
ಲೀಲಾಪಾಂಗೈಃ ಪ್ರಣತಜನತಾಂ ನಂದಯಂಶ್ಚಂದ್ರಮೌಲಿಃ\\
ಮೋಹಧ್ವಾಂತಂ ಹರತು ಪರಮಾನಂದ ಮೂರ್ತಿಶ್ಶಿವೋ ನಃ ||
\end{shloka}

ಎಲ್ಲರಿಗೂ ಭಗವಂತನ ಪಾದಾರವಿಂದಗಳಲ್ಲಿ ಮನಸ್ಸನ್ನು ಇಡುವುದು ಬಹಳ ಇಷ್ಟ. ಆದರೆ `ಎಷ್ಟು ದಿವಸ ಆಗಮಿಕ ಸಾಧನೆಗಳನ್ನೂ ಅನುಷ್ಠಾನಗಳನ್ನೂ ಮಾಡಿದರೂ ಕೂಡ ಇನ್ನೂ ಭಗವಂತನ ದಯೆ ದೊರೆಯಲಿಲ್ಲವಲ್ಲಾ, ನಾನು ಏನು ಪರಿಹಾರ ಮಾಡಬೇಕು' ಎಂದು ಒಬ್ಬನು ಕೇಳಬಹುದು. ಇದರ ಕಾರನವನ್ನು ಕುರಿತು ಈಗ ವಿಚಾರ ಮಾಡೋಣ. ಅನೇಕರಿಗೆ ಭಗವಂತನ ಸಾಕ್ಷಾತ್ಕಾರವನ್ನು ಪಡೆಯಬೇಕೆಂಬ ಆಸೆ ಇದೆ. ಅದಕ್ಕಾಗಿ ಯಾವುದಾದರೂ ಸಾಧನೆಯನ್ನು ಕೈಗೊಳ್ಳಬೇಕೆಂದು ಆಸೆಯೂ ಇದೆ. ಆದರೂ ಅವರು ತಮ್ಮ ಲಕ್ಷ್ಯವನ್ನು ಪಡೆಯಲು ಸಾಧ್ಯವಾಗಲಿಲ್ಲ. ಇದನ್ನು ಕುರಿತು ನೀಲಕಂಠದೀಕ್ಷಿತರು ಒಂದು ಕಡೆ,

\begin{shloka}
ಕಾಮಂ ಜನಾಃ ಸ್ಮಯನ್ತೇ ಕೈಲಾಸವಿಲಾಸವರ್ಣನಾವಸರೇ |\\
ಸಾಧನಕಥನಾವಸರೇ ಸಾಚೀಕುರ್ವನ್ತಿ ವಕ್ತ್ರಾಣಿ ||
\end{shloka}

(ಕೈಲಾಸವನ್ನು ವರ್ಣಿಸುವಾಗ ಜನರು ಸಂತೋಷದಿಂದಿದ್ದಾರೆ. ಆದರೆ `ಸಾಧನೆಗಳು' ವರ್ಣಿಸಲ್ಪಡುವಾಗ ಅವರು ತಮ್ಮ ಮುಖಗಳನ್ನು ತಿರುಗಿಸಿಕೊಂಡು ಬಿಡುತ್ತಾರೆ.)

-ಎಂದಿದ್ದಾರೆ. ಪುರಾಣಗಳಲ್ಲಿ ನಾವು ಕೈಲಾಸನಾಥನನ್ನು ಕುರಿತು ಓದುವಗ ನಮ್ಮ ಮನಸ್ಸಿನಲ್ಲಿ ಆನಂದ ಉಂಟಾಗುತ್ತದೆ. ಹೀಗಿರುವ ಕೈಲಾಸನಾಥನನ್ನು ಒಮ್ಮೆಯಾದರೂ ನಾವು ನೋಡಲಾಗುವುದಿಲ್ಲವೇ ಎಂಬ ಆಸೆಯೂ ಉಂಟಾಗುತ್ತದೆ. ಆದ್ದರಿಂದ ಕೈಲಾಸವನ್ನು ಕುರಿತು ಕೇಳುವಾಗ ಜನರು ಬಹಳವಾಗಿ ತೃಪ್ತರಾಗುತ್ತಾರೆ. ಆದರೆ ಅಂತ ಕೈಲಾಸನಾಥನ ಸಾಕ್ಷಾತ್ಕಾರ ದೊರೆಯಬೇಕಾದರೆ ನಾವು ಮಾಡಬೇಕಾದುದು ಏನು ಎನ್ನುವುದನ್ನು ಹೇಳುವಾಗ-

\begin{shloka}
`ಸಾಧನಕಥನಾವಸರೇ ಸಾಚೀಕುರ್ವನ್ತಿ ವಕ್ತ್ರಾಣಿ'
\end{shloka}

-ಎಂದು ಹೇಳಿದಂತೆ `ಇವೆಲ್ಲಾ ನಡೆಯುವ ಸಾಧನೆಗಳಲ್ಲ, ಇವುಗಳು ನಮ್ಮಿಂದ ಆಗುವವಲ್ಲ' ಎನ್ನುವ ತೀರ್ಮಾನಕ್ಕೆ ಜನರು ಬರುತ್ತಾರೆ. ಏಕೆ ಅವರು ಅಂಥಾ ತೀರ್ಮಾನಕ್ಕೆ ಬರುತ್ತಾರೆಂದರೆ,

\begin{shloka}
ಇಂದ್ರಿಯಸ್ಯೇಂದ್ರಿಯಾರ್ಥೇ ರಾಗದ್ವೇಷೌ ವ್ಯವಸ್ಥಿತೌ |\\
ತರ್ಯೋನವಶಮಾಗಚ್ಛೇತ್ ತೌ ಹ್ಯಸ್ಯ ಪರಿಪಂಥಿನೌ ||
\end{shloka}

(ಇಂದ್ರಿಯಗಳಿಗೆ ವಿಷಯಗಳಲ್ಲಿ ಆಸಕ್ತಿಯೂ, ಅನಾಸಕ್ತಿಯೂ ಇದೆ. ಅವುಗಳಿಗೆ ವಶವಾಗಬಾರದು. ಅವುಗಳೇ ಅವನಿಗೆ ಶತ್ರುಗಳು.)

-ಎಂದು ಗೀತೆಯಲ್ಲಿ ಭಗವಂತನು ಹೇಳಿದಂತೆ ಇಂದ್ರಿಯಗಳಿಗೆ ಆಸಕ್ತಿ ಅನಾಸಕ್ತಿಗಳು (ರಾಗದ್ವೇಷಗಳು) -ಎರಡೂ ಇರುವುದನ್ನು ನೋಡಿದವರಾGಇ ಮನುಷ್ಯರಿದ್ದಾರೆ.

ಶಾಸ್ತ್ರಗಳಲ್ಲಿ ಅವರವರಿಗೆ ವಿಧಿಸಲ್ಪಟ್ಟ ಕರ್ಮಗಳನ್ನು ಉಪಯುಕ್ತವಾದ ಸಮಯದಲ್ಲಿ ಮಾಡಬೇಕೆಂದಿದ್ದರೂ ಅವುಗಳನ್ನು ಮಾಡುವುದರಲ್ಲಿ ಜನರಿಗೆ ಅಷ್ಟಾಗಿ ಮನಸ್ಸಿಲ್ಲ. ಉದಾಹರಣೆಗೆ- ಸೂರ್ಯಾಸ್ತಮಾನ ಕಾಲದ ಸಂಧ್ಯಾವಂದನೆ ಮಾಡಬೇಕೆನ್ನುವುದು ಶಾಸ್ತ್ರದಲ್ಲಿ ಹೇಳಿದ್ದರೆ, ಅವರು ಅದೇ ಕಾಲದಲ್ಲೇ ಬೇರೆ ಕೆಲಸಗಳಲ್ಲಿ ತೊಡಗಿರುವರು. ಏಕೆಂದರೆ ಸಂಧ್ಯಾವಂದನೆ ಮಾಡುವುದಕ್ಕೆ ಮನಸ್ಸಿಲ್ಲ, `ನಿಮಗೆ ಸಮಯ ಸಿಕ್ಕಿದಾಗ ಮಾಡಬಹುದು' -ಎಂದು ಹೇಳಿದರೆ ಅವರು ಒಂದು ವೇಳೆ ಮಾಡಬಹುದು. ಆದರೆ ನಮ್ಮ ಹಿಂದಿನವರು.-

\begin{shloka}
`ಸಂಧೌ ಸಂಧ್ಯಾಮುಪಾಸೀತ'
\end{shloka}

(ಸಂಧ್ಯೆಯನ್ನು ಸಂಧಿಕಾಲದಲ್ಲಿ ಮಾಡಬೇಕು)

-ಎಂದು ಹೇಳಿದ್ದಾರೆ. ಆದರೆ ಆಗ ಅವರೂ ಯಾವುದಾದರೂ ಸಿನಿಮಾಗೆ ಹೋಗಬೇಕಾದರೆ, ನಾಟಕಕ್ಕೆ ಹೋಗಬೇಕಾಗಿದ್ದರೆ ಅಥವಾ ಯಾವುದಾದರೂ ಸಂಗೀತ ಕಛೇರಿಗೆ ಹೋಗಬೇಕಾಗಿದ್ದರೆ ಅವುಗಳಲ್ಲಿಯೇ ಅವರಿಗೆ ಸಮಯವಿರುವುದೇ ಹೊರತು ಸಂಧ್ಯಾವಂದನೆ ಮಾಡುವುದರಲ್ಲಿ ಅಥವಾ ಅಂಥಹುದರಲ್ಲಿ ಮನಸ್ಸನ್ನು ಇಡಲಾರರು.

\begin{shloka}
ಇಂದ್ರಿಯಸ್ಯೇಂದ್ರಿಯಾರ್ಥೇ
\end{shloka}

-ಎಂದು ಭಗವಂತನು ಹೇಳಿದಂತೆ, ಒಬ್ಬನು ತನ್ನ ಇಂದ್ರಿಯಗಳನ್ನೆಲ್ಲಾ ಸಂಯಮದಲ್ಲಿಟ್ಟುಕೊಂಡು ತನ್ನ ಶ್ರೇಯಸ್ಸಿಗಾಗಿ ಪ್ರಯತ್ನಪಡಬೇಕೆಂದು ತಿಳುವಳಿಕೆ ಕೊಟ್ಟರೂ, ಕೇಳುವುದಕ್ಕೆ ಕೆಲವರು ತಯಾರಾಗಿಲ್ಲ. ನಾವು ಕೇಳುವುದಕ್ಕೆ ತಯಾರಾಗಿದ್ದೇವೆ. ಆದರೆ ಯಾರು ಯಾವ ಉಪದೇಶವನ್ನು ಕೊಟ್ಟರೂ ಕೂಡ `ಕೇಳುವಾಗ ಬಹಳ ಚೆನ್ನಾಗಿರುತ್ತದೆ' ಎಂದು ಹೇಳುತ್ತೇವೆ. ಅದನ್ನು ಹಾಗೆ ಅನುಷ್ಠಾನ ಮಾಡುವಾಗ `ಅವುಗಳು ಬಹಳ ಕಷ್ಟವಾದವು; ಈ ಕಾಲದಲ್ಲಿ ಇದನ್ನು ಯಾರು ಮಾಡಬಲ್ಲರು? ಸ್ವಲ್ಪ ಸುಲಭವಾಗಿರುವುದನ್ನು ಮಾತ್ರ ಮಾಡಿದರೆ ಸಾಲದೇ? ಆದ್ದರಿಂದ ಸುಲಭವಾಗಿ ಕರ್ಮಗಳನ್ನು ಮಾತ್ರ ಮಾಡುತ್ತೇನೆ' ಎಂದು ಹೇಳುತ್ತೇವೆ. ಇದರ ಬಗ್ಗೆ ಅಪ್ಪಯ್ಯ ದೀಕ್ಷಿತರು ಒಂದು ಶ್ಲೋಕದಲ್ಲಿ,

\begin{shloka}
``ಉಪಾಯಂ ತೇ ಚೇತಃ ಪರಮಮುಪದಶಾಮಿ ಸ್ಮರರಿಪೋ\\
ಪದಾಂಭೋಜದ್ವಂದ್ವೇ ಪರವಶತಯಾ ರಜ್ಯಸಿ ಯತಃ |\\
ಅವಶ್ಯಂ ಭೋಕ್ತವ್ಯಾಂ ಅನುದಿನಂ ಅತರ್ಕ್ಯೋಽಪಗಮನಾಂ\\
ದಶಾಮಂತ್ಯಾಮಂತಃ ಸಪದಿ ಗಣಯ ಉಪಸ್ಥಿತತರಾಮ್ ||'
\end{shloka}

(ಮನಸ್ಸೇ! ನಿನಗೆ ಉತ್ತಮವಾದ ದಾರಿಯನ್ನು ಹೇಳುತ್ತೇನೆ. ನೀನು ಕಾಮದೇವನ ಶತ್ರುವಿನ ಪಾದಕಮಲಗಳಲ್ಲಿ ಬೇರೆ ವಿಷಯಗಳಲ್ಲಿ ಚಿಂತನೆಯುಳ್ಳವನಾಗಿದ್ದು, ಆಸಕ್ತಿಯುಳ್ಳವನಾಗಿದ್ದೀಯೇ. ಆದ್ದರಿಂದಲೆ, ಪ್ರತಿದಿನವೂ ಅವಶ್ಯವಾಗಿ, ಚಿಂತಿಸಲಾಗದ ಗತಿಯನ್ನು ಅನುಭವಿಸಿಯೇ ಆಗಬೇಕು. ಬೇರೆ ಸ್ಥಿತಿಯನ್ನು ಎಣಿಸುತ್ತಿರುವೆ.)-ಎಂದಿದ್ದಾರೆ.

ಇದೆ ವಿಷಯವನ್ನು ಕುರಿತು ನೀಲಕಂಠ ದೀಕ್ಷಿತರು ಬಹಳ ಸ್ವಾರಸ್ಯಕರವಾಗಿ ಶ್ಲೋಕವೊಂದನ್ನು ಕೊಟ್ಟಿದ್ದಾರೆ.

\begin{shloka}
`ವೇದಾ ವಾ ಸ್ಯುಃ ವಿತತರಚನಾಃ ವಿಸ್ಮರೇದೀಶ್ವರೋ ವಾ\\
ಧರ್ಮಾಧರ್ಮಸ್ಥಿತಿ ವಿರಚನಾಂ ಅಂತಕೋ ವಾ ಮೃಷಾಸ್ಯಾತ್ |\\
ನಿತ್ಯೋ ವಾ ಸ್ಯಾಂ ಅಹಮಿತಿ ಬಹೂನುಲ್ಲಿಖನ್ತಃ ಸಮಾಧೀನ್\\
ವೇದೋವೃತ್ತ್ಯಾ ಮುದಿತಮನಸಃ ಸರ್ವತೋ ನಿವೃತಾಃ ಸ್ಮಃ ||'
\end{shloka}

(ವೇದಗಳು ವ್ಯರ್ಥವಾದ ರಚನೆಗಳಾಗಬಹುದು; ಧರ್ಮವನ್ನೂ ಅಧರ್ಮವನ್ನೂ ಹೇಗೆ ವಿಂಗಡಿಸುವುದು ಎನ್ನುವುದನ್ನು ಈಶ್ವರನೂ ಮರೆಯಬಹುದು. ಅಥವಾ ಯಮನು ಸುಳ್ಳಾಗಿಯೂ, ನಾನು ನಿತ್ಯವಾಗಿಯೂ ಇರಬಹುದು; ಹೀಗೆ ಹಲವಾರು ವಿವರಣೆ ಕೊಡುತ್ತಾರೆ; ಸಂತೋಷದಿಂದ ಕೂಡಿದ ಮನಸ್ಸುಳ್ಳವರಾದ ನಾವು(ಇಂಥ) ಎಲ್ಲಾ ಎಡೆಗಳಿಂದಲೂ ತಿರಿಗಿಬಿಡುತ್ತೇವೆ.)

ಒಬ್ಬನು `ನಾನು ಪರಮಶಿವನನ್ನು ಕುರಿತು ಚಿಂತನೆಯೇ ಮಾಡಲಿಲ್ಲ. ಆದರೆ ನಾನು ತೃಪ್ತಿಯಾಗಿದ್ದೇನೆ. ತೃಪ್ತಿಯಾಗಿರುವನು ಏಕೆ ಪರಮಶಿವನನ್ನು ಚಿಂತಿಸಬೇಕು? ಪ್ರಪಂಚದಲ್ಲಿ ತೃಪ್ತಿ ಇಲ್ಲದವರು ಮಾತ್ರ ಪರಮಶಿವನನ್ನು ಚಿಂತಿಸಬೇಕು. ನಾನಾದರೋ ತೃಪ್ತಿಯಿಂದಿದ್ದೇನೆ. ಆದ್ದರಿಂದ ನಾನು ಏತಕ್ಕೆ ಚಿಂತನೆ ಮಾಡಬೇಕು'-ಎಂದು ಕೇಳುತ್ತಾನೆ. ಆದರೆ ಅವನ ಈ ಪ್ರಶ್ನೆ ಬಹಳ ತಪ್ಪಾದುದೆಂದು ನಾವು ಈಗ ನೋಡುವೆವು.

ಪ್ರಪಂಚದಲ್ಲಿ ಎಷ್ಟೋ ಕಷ್ಟಗಳನ್ನು ಮನುಷ್ಯರು ಪ್ರತಿದಿನವೂ ಅನುಭವಿಸುತ್ತಿದ್ದಾರೆ. ವೇದದಲ್ಲಿ,

\begin{shloka}
`ಅಹರಹಃ ಸಂಧ್ಯಾಮುಪಾಸೀತ'
\end{shloka}

(ಪ್ರತಿದಿನವೂ ಸಂಧ್ಯೋಪಾಸನೆಯನ್ನು ಮಾಡಬೇಕು.)

\begin{shloka}
`ಸತ್ಯಾನ್ನ ಪ್ರಮದಿತವ್ಯಮ್ |ಧರ್ಮಾನ್ನ ಪ್ರಮದಿತವ್ಯಮ್|'
\end{shloka}

(ಸತ್ಯದ ವಿಷಯದಲ್ಲಿ ಪ್ರಮಾದವಾಗಬಾರದು, ಧರ್ಮಕಾರ್ಯಗಳಿಂದ ಹಿಂಜರಿಯಬಾರದು.)

\begin{shloka}
``ಸ್ವಾಧ್ಯಾಯ ಪ್ರವಚನಾಭ್ಯಾಂ ನ ಪ್ರಮದಿತವ್ಯಮ್''
\end{shloka}

`ಕಲಿಯುವುದರಲ್ಲೂ ಕಲಿಸುವುದರಲ್ಲೂ ಗಮನ ಕಡಮೆಯಾಗಬಾರದು)

ಎಂದು ಹಲವು ವಿಧ ಕರ್ಮಗಳು ಮನುಷ್ಯರಿಗೆ ವಿಧಿಸಲ್ಪಟ್ಟಿವೆ. ಸರಕಾರ ಹಲವು ಕಾನೂನುಗಳನ್ನು ವಿಧಿಸಿದೆ. ಅದೇ ರೀತಿ ಭಗವಂತನು ಎಲ್ಲಾ ಸರಕಾರಗಳಿಗೂ ಮೇಲೆ ಸರಕಾರವಾದ್ದರಿಂದ, ಅವನು ಹೇಳಿರುವ ವೇದವನ್ನು ಒಬ್ಬನು ಅನುಸರಿಸಲಿಲ್ಲವೆಂದರೆ ಭಗವಂತನು ಸರಕಾರದಂತೆಯೇ ದಂಡನೆ ಕೊಟ್ಟೇ ಕೊಡುತ್ತಾನೆ. ಇದನ್ನು ಒಬ್ಬನು ಯೋಚನೆ ಮಾಡುವುದೇ ಇಲ್ಲ. ನಾವು ಈಗ ಬಾಳುವ ದೇಶದಲ್ಲಿರುವ ಸರಕಾರದ ಮಾತಿನಂತೆ ನಡೆಯದೆ ಇದ್ದರೆ, ಅನಂತರ ಈ ರಾಜ್ಯದಿಂದಲೇ ನಮ್ಮನ್ನು ಹೊರಹಾಕುತ್ತಾರೆನ್ನುವ ಭಯದಿಂದ `ಕಾನೂನುಗಳನ್ನು ಒಪ್ಪಲು ಸಾಧ್ಯವಿಲ್ಲ'ವೆಂದು ನಾವು ಹೇಳಲಾರೆವು. ಆದರೆ ಭಗವಂತನ ವಿಷಯದಲ್ಲಿ ಮಾತ್ರ ಜನರು `ಏಕೆ ಅವನ ವಿಧಿ-ನಿಯಮಗಳನ್ನು ಪಾಲಿಸಬೇಕು'-ಎನ್ನುವ ಭಾವನೆಯುಳ್ಳವರಾಗಿದ್ದಾರೆ. ಏಕೆಂದರೆ ಭಗವಂತನ ವಿಧಿ-ನಿಯಮಗಳೆಲ್ಲವೂ ವ್ಯರ್ಥವೆಂದು ಅಂಥಾ ಮನುಷ್ಯರ ಭಾವನೆಯಾಗಿದೆ.

ಸುಳ್ಳಾದ ಶಾಸ್ತ್ರಗಳನ್ನು ಏಕೆ ಒಪ್ಪಿಕೊಳ್ಳಬೇಕು. ಅವುಗಳೆಲ್ಲವೂ ಅನಾವಶ್ಯಕವೆಂದು ಅವರು ಭಾವಿಸುತ್ತಾರೆ.

ಭಗವಂತನು ಈಗತಾನೇ ಹುಟ್ಟಿದನೆಂದಿಲ್ಲ. ಅವನ ಹುಟ್ಟು ಅನಾದಿಯಾದುದು. ಆದರೂ ಕೆಲವರಿಗೆ ಒಂದು ಪ್ರಶ್ನೆ. `ಸಾಧಾರಣವಾಗಿ ಪ್ರಪಂಚದಲ್ಲಿರುವ ಮನುಷ್ಯರಿಗೆಲ್ಲಾ ಐವತ್ತು ಅಥವಾ ಅರವತ್ತು ವರ್ಷ ವಯಸ್ಸಾಗುತ್ತಲೇ ಮರೆಯುವಿಕೆ ಎನ್ನುವುದು ಉಂಟಾಗುತ್ತದೆ. ಆದರೆ ಐವತ್ತು ವರ್ಷ ವಯಸ್ಸಾಗುವವರಿಗೆ ಮರೆಯುವಿಕೆ ಎನ್ನುವುದು ಸಹಜವೆಂದಾದರೆ ಬಹಳ ಪುರಾತನನಾದ ಅವನಿಗೆ ಏಕೆ ಮರೆಯುವಿಕೆ ಉಂಟಾಗಬಾರದು? ಏನೋ ಸೃಷ್ಟಿಸಲು ಪ್ರಾರಂಭಿಸಿ ಭಗವಂತನು ವೇದಗಳನ್ನೂ ಉಪದೇಶಿಸಿಬಿಟ್ಟನು. ಈ ಸ್ಥಿತಿಯಲ್ಲಿ ಅವನು ಎಷ್ಟು ವಿಷಯಗಳನ್ನು ತನ್ನ ನೆನಪಿನಲ್ಲಿಟ್ಟುಕೊಂಡಿದ್ದಾನೆಂದು ಹೇಗೆ ಹೇಳುವುದು? ಭಗವಂತನಿಗೆ ತಾನು ಮಾಡಿದ ವಿಧಿ-ನಿಯಮಗಳೆಲ್ಲವೂ ನೆನಪಿನಲ್ಲಿವೆ ಎನ್ನುವ ಪರಿಸ್ಥಿತಿ ಇದ್ದರೆ ಅವನು ತಪ್ಪಿತಸ್ಥರಿಗೆ ದಂಡನೆ ಕೊಟ್ಟರೆ ಪರವಾಗಿಲ್ಲ, ಅವನಿಗೆ ತಾನು ಮಾಡಿದ ವಿಧಿ-ನಿಯಮಗಳು ಮರೆತು ಹೋಗುವುದಿಲ್ಲವೇ?' ಎನ್ನುವುದೇ ಪ್ರಶ್ನೆ.

ಭಗವಂತನಿಗೆ ಎಲ್ಲವೂ ಜ್ಞಾಪಕವಿರುತ್ತದೆ, ಅವನು ಎಲ್ಲವನ್ನೂ ಮಾಡಲು ಶಕ್ತಿಯುಳ್ಳವನು. ಅವನು `ಸರ್ವಸಮರ್ಥ'ನೆಂದು ಹೇಳಲ್ಪಟ್ಟಿದೆ. ಸಮರ್ಥನಾಗಿರುವ ಈಶ್ವರನಿಗೆ ಜ್ಞಾಪಕ ಮರೆಯುವಿಕೆ ಹೇಗೆ ಉಂಟಾಗುವುದು? ಕೆಲವರು ಐವತ್ತು ವರ್ಷಗಳಾಗುತ್ತಲೇ ತಮಗೆ ಮರೆಯುವಿಕೆ ಉಂಟಾಗಿದೆಯೆಂದು ಹೇಳುತ್ತಾರೆ ಆದರೆ ನಾವು ಕೆಲವರನ್ನು ನೋಡುತ್ತೇವೆ. ಅವರಿಗೆ ಮರೆಯುವಿಕೆ ಎನ್ನುವುದು ಎಂಭತ್ತು ವಯಸಾದರೂ ಸ್ವಲ್ಪವೂ ಇರುವುದಿಲ್ಲ. ಸಾಮಾನ್ಯರಾದ ಇಂಥಾ ಮನುಷ್ಯರಿಗೆ ವಯಸ್ಸಾದರೂ ಹೀಗೆ ಇರಲು ಸಾಧ್ಯವಾಗುವುದಾದರೆ ಭಗವಂತನಿಗೆ ಮರೆಯುವಿಕೆ ಏಕೆ ಉಂಟಾಗಬೇಕು? ನಾವೆಲ್ಲರೂ ಸಮರ್ಥರು. ಅಂದರೆ ಸಾಮರ್ಥ್ಯವುಳ್ಳವರೆಮ್ದು ಹೆಸರು ಇಟ್ಟುಕೊಂಡರೆ ಭಗವಂತನು ಸರ್ವಸಮರ್ಥನು, ಅಂದರೆ ಎಲ್ಲವನ್ನೂ ಮಾಡಲು ಶಕ್ತಿಯುಳ್ಳವನು. ಆದ್ದರಿಂದ ಮರೆಯುವಿಕೆ ಎನ್ನುವುದು ಇಲ್ಲ.

ಆದ್ದರಿಂದ ಭಗವಂತನು, `ನನ್ನ ವಿಧಿ-ನಿಯಮಗಳನ್ನು ಅನುಸರಿಸಿ ನೀನು ನಡೆಯುವುದಿಲ್ಲ. ಆದ್ದರಿಂದ ನೀನು ದಂಡನೆಯನ್ನು ಅನುಭವಿಸಬೇಕು' -ಎಂದು ಧಾರಾಳವಾಗಿ ಹೇಳಬಹುದು. ಇನ್ನೊಂದು ಪ್ರತಿಕೂಲಪ್ರಶ್ನೆ. ನ್ಯಾಯಾಲಯದಲ್ಲಿ ಯಾವ ತೀರ್ಪನ್ನಾದರೂ ಕೊಂಡಬಹುದು. ಆದರೆ ನ್ಯಾಯಾಲಯದಲ್ಲಿ ತೀರ್ಪನ್ನು ಕೊಟ್ಟಮೇಲೆ ಆ ನ್ಯಾಯಾಲಯದಲ್ಲಿರುವ ಅಧಿಕಾರಿಗೆ ಇನ್ನೇನು ಮಾಡಲು ಸಾಧ್ಯವಿಲ್ಲ, ಪೋಲಿಸಿನವರು ಆ ತಪ್ಪಿತಸ್ಥನನ್ನು ಕರೆದುಕೊಂಡು ಜೈಲಿಗೆ ಹೋದ ಮೇಲೆ ಇವನಿಗೆ ದಂಡನೆ ಪ್ರಾರಂಭವಾಗುತ್ತದೆ. ಪೋಲಿಸಿನವರು ಕರೆದುಕೊಂಡುಹೋಗಲಿಲ್ಲವೆಂದರೆ ಇವನಿಗೆ ಸಂಬಂಧಪಟ್ಟ ತೀರ್ಪು ಪುಸ್ತಕದಲ್ಲಿ ಮಾತ್ರ ಇರುವುದು. ಅದೇ ರೀತಿ ಭಗವಂತನು ದಂಡನೆ ವಿಧಿಸಿದರೂ ಅದನ್ನು ಕೊಡುವವರು ಯಾರಿದ್ದಾರೆ ಎನ್ನುವುದೇ ಇನ್ನೊಂದು ಪ್ರತಿಕೂಲ ಪ್ರಶ್ನೆ.

ಭಗವಂತನು ವಿಧಿಸಿದ ದಂಡನೆಯನ್ನು ನಮಗೆಲ್ಲಾ ಕೊಡುವುದಕ್ಕಾಗಿಯೇ ಯಮಧರ್ಮರಾಜನು ಇರುವುದು. ಆ ಯಮಧರ್ಮರಾಜನು,

\begin{shloka}
`ಅಯಂ ಲೋಕೋ ನಾಸ್ತಿ ಪರ ಇತಿ ಮಾನೀ\\
ಪುನಃ ಪುನರ್ವಶಮಾಪದ್ಯತೇ ಮೇ |'
\end{shloka}

(ಯಾರು `ಈ ಪ್ರಪಂಚ ಮಾತ್ರವಿದೆ, ಬೇರೆ ಪ್ರಪಂಚವಿಲ್ಲ' ಎಂದು ಭಾವಿಸುತ್ತಾನೋ ಅವನು ಮತ್ತೆ ಮತ್ತೆ ನನ್ನ ವಶಕ್ಕೆ ಬರುತ್ತಾನೆ.)

\begin{shloka}
ಅಯಂ ಲೋಕೋ ನಾಸ್ತಿ ಪರ ಇತಿ ಮಾನೀ'
\end{shloka}

-ಎಂದಂತೆ ಕಳ್ಳನೂ ಭಾವಿಸುತ್ತಾನೆ. ನನ್ನ ಬುದ್ಧಿವಂತಿಕೆ ಮತ್ತು ಸಾಮರ್ಥ್ಯದ ಮುಂದೆ ಪೋಲೀಸಿನವನ ಬುದ್ಧಿವಂತಿಕೆ ಮತ್ತು ಸಾಮರ್ಥ್ಯ ಏನಿದೆ? ಎಷ್ಟೋಸಲ ಪೋಲೀಸಿನವನ ಕಣ್ಣಿಗೆ ಮಣ್ಣು ಎರಚಿದ್ದೇನೆಂದು ಕಳ್ಳನು ಹೇಳುತ್ತಲೇ ಇರುತ್ತಾನೆ. ಆದರೆ ಒಂದು ವಿಷಯವನ್ನು ತಿಳಿದುಕೊಳ್ಳಬೇಕು. ಒಮ್ದು ಗಾದೆ ಇದೆ `ಕಳ್ಳನ ಹೆಂಡತಿ ಎಂದಿದ್ದರೂ ಮುಂಡೆ' ಎಂದು. ಏಕೆಂದರೆ, ಕಳ್ಳನು ಹತ್ತುಸಲ ಕಳ್ಳತನ ಮಾಡುತ್ತಾನೆ ಆಮೇಲೆ ಸಿಕ್ಕಿಕೊಳ್ಳುತ್ತಾನೆ. ಅದರಂತೆಯೇ ಈ ಪ್ರಪಂಚದಲ್ಲಿರುವರೂ ಕೂಡ ಇಹಲೋಕದಲ್ಲಿ ಅನುಭವಿಸುವ ಕರ್ಮ ಪ್ರಬಲವಾಗಿರುವವರಿಗೆ ಅವರು ಯಮನ ಹಿಡಿತಕ್ಕೆ ಸಿಕ್ಕದೆ ಇರಬಹುದು. ಒಬ್ಬನು ತಾನು ನಿತ್ಯವಾಗಿಯೇ ಇದ್ದು ಬಿಟ್ಟರೆ, ಆಗ ಯಮನ ಹಿಡಿತಕ್ಕೆ ಹೇಗೆ ಉಂಟಾಗುತ್ತದೆಂದು ಕೇಳಬಹುದು. ಆದರೆ ನಾವು ನಿತ್ಯರಾಗಲು ಸಾಧ್ಯವೇ ಇಲ್ಲ. ರೋಗಿಗಳು ಡಾಕ್ಟರ್‌ಗೆ ಹಣ ಕೊಡುತ್ತಲೆ ಇದ್ದಾರೆ. ಎಲ್ಲಿಯವರೆಗೆ? ಒಬ್ಬ ಡಾಕ್ಟರ್ ಸಾಲದೆಂದು ಭಾವಿಸಿ ಮದ್ರಾಸಿಗೆ ಬರುವನು. ಮದ್ರಾಸಿನಲ್ಲಿರುವ ಡಾಕ್ಟರ್‌ಗೆ ಹಣ ಕೊಟ್ಟು ನಂತರ ಲಂಡನ್‌ಗೆ ಹೋಗಬೇಕೆನ್ನುತ್ತಾನೆ. ಆದರೆ ಎಷ್ಟರವರೆಗೆ ಹಣ ಕೊಡುತ್ತಿರುವುದು? ಹಣ ಮುಗಿದು ಹೋದರೆ ಯಾವ ವೈದ್ಯನೂ ಇವನಿಗೆ ಉಪಯೋಗವಿಲ್ಲವೆನ್ನುವ ಸ್ಥಿತಿ ಉಂಟಾಗುತ್ತದೆ. ಇವನು ತನ್ನ ಶರೀರವನ್ನು ಬಿಡಬೇಕಾದುದೇ. ಆದರೆ ಮನುಷ್ಯರು `ನಿತ್ಯೋವಾಸ್ಯಾಸಿ' -ಎಂದು ಹೇಳಿದಂತೆ ತಾವು ಯಾವಾಗಲೂ ಸ್ಥಿರವಾಗಿರುವರೆಂದುಕೊಂಡಿದ್ದರೂ ಕೂಡ ಯಮನು ಒಂದು ದಿನ ಅವರಿಗೆ ದುಃಖವನ್ನು ಕೊಟ್ಟೇಕೊಡುವನು. ನಾವು ಎಷ್ಟು ಮಂದಿ ವೈದ್ಯರನ್ನು ನಮ್ಮ ಹತ್ತಿರ ಇಟ್ಟುಕೊಂಡು ನಮಗೆ ಚಿಕಿತ್ಸೆ ಮಾಡುವಂತೆ ಹೇಳಿದರೂ ಯಮನು ನಮ್ಮನ್ನು ತೆಗೆದುಕೊಂಡು ಹೋಗದೆ ಬಿಡುವುದಿಲ್ಲ. ಏಕೆಂದರೆ ಆ ಯಮನೇ ಆ ವೈದ್ಯರುಗಳನ್ನೂ ಸಹ ಒಂದು ದಿನ ತೆಗೆದುಕೊಂಡು ಹೋಗಲು ತಯಾರಾಗಿದ್ದಾನೆ. ಅಂಥ ಯಮನು ನಮ್ಮನ್ನು ಮಾತ್ರ ಏಕೆ ಬಿಟ್ಟು ಬಿಡುತ್ತಾನೆ?

ಮನೆಯಲ್ಲಿ ಇಟ್ಟಿದ್ದ ವಡೆಯೊಂದನ್ನು ಒಂದು ಚಿಕ್ಕ ಪ್ರಾಣಿ ತಿನ್ನುತ್ತಿತ್ತು; ಅದನ್ನು ನೋಡಿದ ದೋಡ ಇಲಿಯೊಂದು ಆ ವಡೆಯನ್ನು ತಿಂದು ಬಿಟ್ಟಿತು. ಅದನ್ನು ತಿನ್ನುತ್ತಿದ್ದ ಪ್ರಾಣಿಯನ್ನೂ ತಿಂದುಬಿಟ್ಟಿತು. ಅದೇ ರೀತಿಯಲ್ಲಿ ಯಮ ನಾವು ಯಾರಿಂದ ಕಾಪಾಡಲ್ಪಡುತ್ತೇವೆಂದು ಭಾವಿಸುತ್ತಿರುತ್ತೇವೋ, ಅಂಥ ವೈದ್ಯರನ್ನು ಮಾತ್ರ ಬಿಟ್ಟು ಬಿಡುವುದಿಲ್ಲ. ಅವರನ್ನೂ, ಅವರು ಚಿಕಿತ್ಸೆಗೆ ಒಳಪಡಿಸುವ ರೋಗಿಗಳನ್ನೂ ಅವನು ಒಂದು ದಿನ ಕರೆದುಕೊಂಡು ಹೋಗೇ ಹೋಗುತ್ತಾನೆ.; ಆದರೆ ಅಂಥ ಯಮನು ಒಬ್ಬನನ್ನು ಮಾತ್ರ ಏನೂ ಮಾಡಲಾರನು. ಆ ಒಬ್ಬನು ಯಾರು?

\begin{shloka}
`ಕಾಮಮಸ್ತು ಜಗತ್ಸರ್ವಂ ಕಾಲಸ್ಯಾಸ್ಯ ವಶಂವದಮ್ |\\
ಕಾಲಕಾಲಂ ಪ್ರಪನ್ನಾನಾಂ ಕಾಲಃ ಕಿಂ ನ ಕರಿಷ್ಯತಿ ||'
\end{shloka}

(ಈ ಕಲಿಕಾಲದ ವಶಕ್ಕೆ ಬಂದಿರುವ ಪ್ರಪಂಚವೆಲ್ಲಾ ಹೇಗೆ ಬೇಕಾದರೂ ಇರಲಿ. ಕಾಲನಿಗೆ ಕಾಲನಾಗಿರುವ ಕಾಲನು ಏನು ಮಾಡಬಹುದು?) ಯಾರು ಕಾಲನನ್ನೂ ಕೂಡ ಸುಟ್ಟು ಹಾಕಿದನೋ ಅಂಥ ಕಾಲಕಾಲನಿಗೇ ನಾವು ಶರಣಾದರೆ ಯಮ ಧರ್ಮರಾಜನು ಅಂಥವನನ್ನು ಏನೂ ಮಾಡಲಾರನು. ಆದರೆ ಅದಕ್ಕಾಗಿ ಯಾರೂ ಪ್ರಯತ್ನ ಪಡುವಂತೆ ತೋರುವುದಿಲ್ಲ.

ಪ್ರಪಂಚದಲ್ಲಿ ಮನುಷ್ಯರಿಗೆ ಹಲವು ಆಯುಧಗಳಿವೆ. ಅವುಗಳು ಯಾವುವು ಎಂದರೆ,

\begin{shloka}
ಶಬ್ದಾದಿಭಿಃ ಪಂಚಭಿರೇವ ಪಂಚ ಪಂಚತ್ವಮಾಪುಃ ಸ್ವಗುಣೇನ ಬದ್ಧಾಃ |\\
ಕುರುಂಗ-ಮಾತಂಗ-ಪತಂಗ-ಮೀನ-ಭೃಂಗಾ ನರಃ ಪಂಚಭಿರಂಚಿತಃ\\
\hspace{5.5cm} ಕಿಮ್ ||'
\end{shloka}

(ಜಿಂಕೆ, ಆನೆ, ದೀಪದ ಹುಳು, ಮೀನು, ಭೃಂಗ ಇವುಗಳು ಶಬ್ದಾದಿ ಪಂಚದಲ್ಲಿ ಸ್ವಗುಣದಿಂದ ಬದ್ಧರಾಗಿರುವುದರಿಂದ ನಾಶವನ್ನು ಪಡೆಯುತ್ತವೆ. ಇದರಲ್ಲಿಯೂ ಸಂಬಂಧವನ್ನು ಇಟ್ಟುಕೊಂಡಿರುವ ಮನುಷ್ಯನ ಗತಿ ಏನು?)

-ಎಂದು ಶಂಕರಭಗವತ್ಪಾದರು ಹೇಳಿದ್ದಾರೆ. ಸಾಮಾನ್ಯವಾಗಿ ಒಂದೊಂದು ವಿಷಯದಲಿಯೂ ಒಬ್ಬ ಜೀವನು ಆಸಕ್ತಿಯುಳ್ಳವನಾಗಿದ್ದು ಅದರಿಂದಲೇ ತನ್ನ ಪ್ರಾಣವನ್ನು ಕಳೆದುಕೊಳ್ಳುತ್ತಾನೆಂದು ಶ್ಲೋಕದಲ್ಲಿ ಅವರು ಹೇಳಿರುವುದನ್ನು ನಾವು ನೋಡುತ್ತೇವೆ.

ಜಿಂಕೆಯನ್ನು ಹಿಡಿಯಲು ಹೋಗುವ ಬೇಡನು, ಮೊದಲು ಕಾಡಿನಲ್ಲಿ ಮಧುರವಾದ ಸಂಗೀತಕ್ಕೆ ಏರ್ಪಾಡು ಮಾಡುವನು. ಜಿಂಕೆ ತನ್ನ ಕಿವಿಗಳಿಗೆ ಅಂಥ ಸಂಗೀತ ಬಹಳ ಅದ್ಭುತವಾಗಿದೆಯೆಂದುಕೊಂಡು ಶಬ್ದ ಬರುವ ಜಾಗಕ್ಕೆ ಓಡಿ ಬರುವುದು. ಆದರೆ ಈ ಜಾಗದಲ್ಲಿ ಬೇಡನು ಬಲೆಯೊಂದನ್ನು ಹಾಸಿರುವನು. ಶಬ್ದವನ್ನು ಕೇಳುವ ಆಸೆಯಿಂದ ಬರುವ ಜಿಂಕೆಗಳಿಗೆ ಕೆಳಗೆ ಹಾಸಿರುವ ಬಲೆ ಕಣ್ಣಿಗೆ ಬೀಳುವುದಿಲ್ಲ. ಯಾವುದೋ ಮಧುರವಾದ ಶಬ್ದ ಕೇಳಿ ಬರುತ್ತದಲ್ಲಾ ಎಂದು ಅದರಲ್ಲೇ ತೃಪ್ತಿ ಪಡೆಯುವ ಜಿಂಕೆಗಳು ಕೆಳಗೆ ಹಾಸಿರುವ ಬಲೆಯನ್ನು ನೋಡಲು ಬುದ್ಧಿ ಇಲ್ಲದವಾಗಿ ಬಿಡುತ್ತವೆ. ಅನಂತರ ನೇರವಾಗಿ ಬಲೆಯಲ್ಲಿ ಬಿದ್ದು ಸಿಕ್ಕಿಕೊಳ್ಳುವುವು. ಶಬ್ದವೆನ್ನುವ ಒಂದು ಇಂದ್ರಿಯಾಕರ್ಷಣೆಯಿಂದಾಗಿ ತಮ್ಮ ಪ್ರಾಣವನ್ನೇ ಕೊಡಬೇಕಾಗುತ್ತದೆ.

ಆನೆಗಳನ್ನು ಪಳಗಿಸುವುದರಲ್ಲಿ, ಹೊಸದಾಗಿ ಯಾವುದಾದರೂ ಆನೆ ದೊರೆತರೆ ಆ ಆನೆಯನ್ನು ಹಳ್ಳದಲ್ಲಿ ಹಾಕಿ ಬಿಡುತ್ತಾರೆ. ಆನೆಯನ್ನು ಹಳ್ಳದಲ್ಲಿ ಹಾಕಿದರೂ ಕೂಡ ಮನುಷ್ಯನು ಆ ಆನೆಯೊದನೆ ಹೋಗಿ ತಕ್ಷಣ ಹತ್ತಿರ ಪಳಗಿಸಿಕೊಂಡು ಬಿಡಲು ಸಾಧ್ಯವಿಲ್ಲ. ಅದು ಮನುಷ್ಯನ ಹತ್ತಿರಕ್ಕೆ ಬರಬೇಕಾದರೆ ಮೊದಲೇ ಪಳಗಿದ ಒಂದು ಹೆಣ್ನಾನೆಯನ್ನು ಇದಕ್ಕಾಗಿ ಆರಿಸಿ ಇಟ್ಟುಕೊಳ್ಳುತ್ತಾರೆ. ಹೇಗೆಂದರೆ ಈ ಹೆಣ್ಣಾನೆಯ ಹೊಟ್ಟೆಯ ಹತ್ತಿರ ಮಾವಟಿಗನು ತನ್ನನ್ನು ಒಂದು ಬಟ್ಟೆಯಿಂದ ಕಟ್ಟಿಕೊಂಡು ಈ ಹೆಣ್ಣಾನೆಯನ್ನು ಗಂಡಾನೆಯ ಹತ್ತಿರ ಹೋಗುವಂತೆ ಮಾಡುತ್ತಾನೆ. ಆ ಹೆಣ್ಣಾನೆ ಗಂಡಾನೆಯ ಹತ್ತಿರಕ್ಕೆ ಹೋದರೂ ಸಹ ಗಂಡಾನೆ ಏನೂ ಮಾಡುವುದಿಲ್ಲ. ಏಕೆಂದರೆ ತನ್ನ ಹತ್ತಿರ ಬರುತ್ತದೆಂದು ಸಂತೋಷವಾಗಿರುವುದು. ಪಳಗಿಸುವ ಮನುಷ್ಯನು ಹೆಣ್ಣಾನೆಗೆ `ಒಂದು ಸರಪಣಿಯನ್ನು ತೆಗೆದು ಗಂಡಾನೆಯ ಮೇಲೆ ಹಾಕು' ಎನ್ನುತ್ತಾನೆ. ಹೆಣ್ನಾನೆ ಹಾಗೆಯೇ ಮಾಡುವುದು. ಆಗ ಆ ಗಂಡಾನೆಗೆ ಅದು ಯಾವ ವಿಧವಾದ ಕೋಪವನ್ನೂ ಉಂಟುಮಾಡುವುದಿಲ್ಲ. ಏಕೆಂದರೆ, ಅದರ ಮಟ್ಟಿಗೆ ಹೆಣ್ಣಾನೆ ತಾನೆ ತನ್ನ ಕಾಲಿಗೆ ಸರಪಣಿ ಹಾಕುತ್ತದೆ ಎನ್ನುವುದರಿಂದ ಅದು ಸುಮ್ಮನೆ ಇರುವುದು, ಅನಂತರ ಹೆಣ್ಣಾನೆ, ಮಾವಟಿಗನ ಆಜ್ಞೆ ಪ್ರಕಾರ ಈ ಗಂಡಾನೆಯನ್ನು ಹಿಡಿದು ಎಳೆದರೂ ಕೂಡ ಹೆಣ್ಣಾನೆ ಎಲ್ಲೆಲ್ಲಿ ಓಡುವುದೋ, ಅಲ್ಲೆಲ್ಲಾ ಅದೂ ಓಡುತ್ತಿರುತ್ತದೆ. ಹೀಗೆ ಕೆಲವು ದಿನಗಳಲ್ಲಿ ಗಂಡಾನೆಯೂ ಮಾವಟಿಗನ ವಶಕ್ಕೆ ಬಂದು ಬಿಡುವುದು. ಇದೇ ರೀತಿ ಒಂದು ಹೆಣ್ಣಾನೆಯನ್ನು ಪಳಗಿಸಬೇಕಾದರೂ ಕೂಡ ಗಂಡಾನೆಯ ಸಹಾಯವನ್ನು ಪಡೆದು ಹೀಗೆ ಮಾಡುವರು. ಆದ್ದರಿಂದ, ಒಂದು ಹೆಣ್ಣಾನೆಯನ್ನು ಕಂಡರೆ `ಅದು ನನ್ನ ಸಂಗಾತಿ' ಎಂದುಕೊಂಡು ತನ್ನ ಶಕ್ತಿ ಎಷ್ಟಿದ್ದರೂ ಅದೆಲ್ಲವನ್ನು ಮರೆತು, ಇದಾದ ಮೇಲೆ ತನಗೆ ಮನುಷ್ಯನಿಂದ ಉಂಟಾಗುವ ಎಲ್ಲಾ ವಿಪತ್ತುಗಳನ್ನು ಮರೆತು, ತನ್ನ ಜೀವನವನ್ನು ಪೂರ್ತಿ ಆ ಆನೆ ಮಾವಟಿಗನಿಗೆ ಒಪ್ಪಿಸಿ ಬಿಡುತ್ತದೆ. ಇದಕ್ಕೆಲ್ಲಾ ಕಾರನ ಆನೆ; ಸ್ಪರ್ಶ ಸುಖದಲ್ಲಿ ಬಿದ್ದು ತನ್ನ ಬಾಳನ್ನೇ ಒಬ್ಬ ಮನುಷ್ಯನಿಗೆ ಅರ್ಪಣೆಮಾಡಿ ಬಿಡುವುದು.

ನಾವು ಮನೆಯಲ್ಲಿ ದೀಪದ ಹುಳುವನ್ನು ನೋಡಿದ್ದೇವೆ. ಎಲ್ಲಾದರೂ ದೀಪದ ಪ್ರಕಾಶವನ್ನು ಕಂಡರೆ ಆ ಹುಳು `ಈ ದೀಪ ನೋಡುವುದಕ್ಕೆ ಬಹಳ ಸುಂದರವಾಗಿದೆ' ಎಂದು ಭಾವಿಸಿ ಆ ದೀಪವನ್ನು ಹೊಡದು ಹೊಡದು ಕೊನೆಗೆ ಆ ದೀಪದಲ್ಲಿಯೇ ಬಿದ್ದು ತನ್ನ ಪ್ರಾಣವನ್ನು ಕಳೆದುಕೊಂಡುಬಿಡುತ್ತದೆ. ಇದಕ್ಕೆ ಕಾರಣವನ್ನು ನೋಡುವಾಗ ಬಹಳ ಸುಂದರವಾಗಿದೆಯೆಂದು ಭಾವಿಸಿ ಆ ಮನೆಯಲ್ಲಿ ಹುಳು ಆ ವಸ್ತುವಿನಿಂದಲೇ ತನ್ನ ಪ್ರಾಣವನ್ನು ಕಳೆದುಕೊಂಡು ಬಿಡುವುದು.

ಬೆಸ್ತನು ಹೇಗೆ ಮೀನು ಹಿಡಿಯುತ್ತಾನೆ? ನೆಲದ ಮೇಲಿರುವ ಮಣ್ಣುಹುಳುವನ್ನು ಅವನು ತೆಗೆದುಕೊಂಡು ಹೋಗುತ್ತಾನೆ. ಕೋಲಿನಲ್ಲಿ ಒಂದು ಮುಳ್ಳು ಇರುವುದು. ಹುಳುವಿನೊಡನೆ ಕೂಡಿದ ಕೋಲನ್ನು ಅವನು ನೀರಿಗೆ ಹಾಕುತ್ತಾನೆ, ಅದರಲ್ಲಿ ಸಿಕ್ಕಿಕೊಂಡಿರುವ ಹುಳುವನ್ನು ಕಂಡು ಮೀನುಗಳು ಹುಳುವನ್ನು ತಿನ್ನುವುದಕ್ಕಾಗಿ ತಮ್ಮ ಬಾಯಿಯನ್ನು ತೆರೆದುಕೊಂಡು ಹತ್ತಿರ ಬರುವಾಗ ಆ ಹುಳುಗಳನ್ನು ಅವುಗಳು ನುಂಗಲು ಸಾಧ್ಯವಾಗುವುದಿಲ್ಲ. ಏಕೆಂದರೆ ಹುಳುವನ್ನು ನುಂಗಲು ಪ್ರಯತ್ನಪಡುವಾಗ ಕೋಲಿನಲ್ಲಿರುವ ಮುಳ್ಳಿಗೆ ಅದು ಸಿಕ್ಕಿಕೊಂಡುಬಿಡುವುದು. ಕೊನೆಗೆ ಮೀನು ಸ್ವಾದಕ್ಕೆ ಆಸೆಪಟ್ಟು ಬೆಸ್ತನ ಕೋಲಿಗೆ ಸಿಕ್ಕಿಕೊಂಡು ತನ್ನ ಪ್ರಾಣವನ್ನೇ ಕಳೆದುಕೊಂಡುಬಿಡುತ್ತದೆ.

ದುಂಬಿ ಹೂಗಳಲ್ಲಿರುವ ರಸವನ್ನು ಆಸ್ವಾದಿಸುವುದಕ್ಕಾಗಿ ಹೂಗಳಲ್ಲಿಯೇ ಇರಬಯಸುತ್ತದೆ. ಒಮ್ಮೆ ಒಂದು ದುಂಬಿ ಒಂದು ತಾವರೆ ಹೂವಿನಲ್ಲಿ ಕುಳಿತು ಅದರ ರಸವನ್ನು ಆಸ್ವಾದಿಸುತ್ತಿತ್ತು. ಆದರೆ ರಾತ್ರಿಯಾಯಿತು. ಇನ್ನೂ ಅದರಲ್ಲೇ ಕುಳಿತಿರಬೇಕೆನ್ನುವ ಆಸೆಯಿಂದ ದುಂಬಿ ಇದ್ದಾಗ ತಾವರೆ ಹೂವು ಮುಚ್ಚಿಕೊಂಡಿತು.

\begin{shloka}
``ರಾತ್ರಿರ್ಗಮಿಷ್ಯತಿ ಭವಿಷ್ಯತಿ ಸುಪ್ರಭಾತಂ\\
ಭಾಸ್ವಾನುದೇಷ್ಯತಿ ಹಸಿಷ್ಯತಿ ಪಂಕಜಶ್ರೀಃ‌ |\\
ಇತ್ಥಂ ವಿಚಿಂತಯತಿ ಕೋಶಗತೇ ದ್ವಿರೇಫೇ\\
ಹಾ ಹನ್ತ ಹನ್ತ ನಲಿನೀಂ ಗಜ ಉಜ್ಜಹಾರ ||''
\end{shloka}

(``ರಾತ್ರಿ ಕಳೆದು ಹೋಗುವುದು, ಸೂರ್ಯೋದಯವಾಗುವುದು, ತಾವರೆ ಅರಳುವುದು' ಹೀಗೆ ಹೂವಿನೊಳಗಿದ್ದ ದುಂಬಿ ಯೋಚಿಸಿತು. ಅಯ್ಯೋ! ಅಯ್ಯೋ! ಆನೆ ತಾವರೆಯನ್ನು ಕಿತ್ತೆಸೆಯಿತು.)

%page 283





































































































