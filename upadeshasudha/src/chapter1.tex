\chapter*{ಶ್ರೀ ಶ್ರೀ ಜಗದ್ಗುರು ಮಹಾಸ್ವಾಮಿಗಳವರ ಉಪದೇಶ}

\section{ಬಾಳಿನ ದಾರಿ}


\begin{shloka}
ವಿಶುದ್ಧ ಜ್ಞಾನ ದೇಹಾಯ ತ್ರೀವೇದೀ ದಿವ್ಯ ಚಕ್ಷುಷೇ|\\
ಶ್ರೇಯಃ ಪ್ರಾಪ್ತಿನಿಮಿತ್ತಾಯ ನಮಃ ಸೋಮಾರ್ಧಧಾರಿಣೇ||\\
ನಮಾಮಿ ಯಾಮಿನೀನಾಥಲೇಖಾಲಂಕೃತಕುಂತಲಾಮ್|\\
ಭವಾನೀಂ ಭವಸಂತಾಪನಿರ್ವಾಪಣಸುಧಾನದೀಮ್||
\end{shloka}


ದೇವಿಯನ್ನು ಸ್ತುತಿಸುವಾಗ ಆಕೆಯ ಸ್ವರೂಪವನ್ನು ಯಾವ ವಿಧವಾಗಿ ಭಾವಿಸಬೇಕು? `ದೇವಿಯನ್ನು ಪೂಜಿಸಿ ನಾವು ಯಾವ 
ವಿಧವಾದ ಫಲವನ್ನು ಪಡೆಯಬಹುದು' ಎನ್ನುವುದು ಮೇಲೆ ಹೇಳಿದ ಶ್ಲೋಕದಲ್ಲಿ ಸ್ಪಷ್ಟಪಡಿಸಲಾಗಿದೆ. ದೇವಿಗೆ 
ಮಾಲೆ, ಬಳೆ ಇಂತಹ ಅಲಂಕಾರಗಳು ಇದ್ದರೂ ಒಂದು ಅಲಂಕಾರ ವಿಶೇಷವಾಗಿ ಇದೆ. ಅದು ಯಾವುದು ಎನ್ನುವ ಪ್ರಶ್ನೆಗೆ 
ಉತ್ತರವಾಗಿ `ಯಾಮಿನೀನಾಥ ಲೇಖಾಲಂಕೃತ ಕುಂತಲಾಮ್' ಎಂದು ಶ್ಲೋಕದಲ್ಲಿ ಇದೆ. `ಯಾಮಿನೀನಾಥ' 
ಎಂದರೆ ಚಂದ್ರ. `ಲೇಖಾ' ಎಂದರೆ ಕಲೆ. ಆದ್ದರಿಂದ `ಯಾಮಿನೀನಾಥಲೇಖಾಲಂಕೃತ ಕುಂತಲಾಮ್' ಎನ್ನುವುದಕ್ಕೆ 
ಚಂದ್ರಕಲೆಯಿಂದ ಅಲಂಕೃತವಾದ ಶ್ರೀಮುಡಿಯುಳ್ಳವಳು ಎಂದು ಅರ್ಥ. ಈ ಆಭರಣದ ವಿಶೇಷವೇನು? ಈ 
ಪ್ರಪಂಚದಲ್ಲಿ ಯಾರೂ ಚಂದ್ರನನ್ನು ತಲೆಯಲ್ಲಿ ಇಟ್ಟುಕೊಳ್ಳಲಾರರು. ಆದರೆ ದೇವಿ ಚಂದ್ರನನ್ನು ಹೇಗೆ ಇಟ್ಟುಕೊಂಡಿದ್ದಾಳೆ? ಚಂದ್ರ 
ಎಲ್ಲಿ ಇಡಲ್ಪಟ್ಟಿದ್ದಾನೆ ಎಂದು ಹೇಳುವುದರಲ್ಲಿ ಮುಖ್ಯವಾದ ಭಾವವಿದೆ. ಪರಮಶಿವನೂ ದೇವಿಯೂ ಒಟ್ಟಾಗಿ 
ಜ್ಞಾನ ಕಾಯರಾಗಿದ್ದಾರೆ. ಅವರಿಗೆ ಒಂದು ವಿಧವಾದ ಶರೀರ ನಿಜಕ್ಕೂ ಇಲ್ಲ. ಆದರೂ ಭಕ್ತರನ್ನು ಅನುಗ್ರಹಿಸಲು 
ಅವರು ಹಲವು ಶರೀರಗಳನ್ನು ಪಡೆಯುವುದುಂಟು. ಈಶ್ವರನು ಭಕ್ತನನ್ನು ಅನುಗ್ರಹಿಸಲು ಮೊದಲು ಯಾವ ಶರೀರವನ್ನು 
ಪಡೆದನೋ ಅದು ನಮಗೆ ಈಗ ಉಪಾಸನೆಗೆ ಸಹಾಯ ಮಾಡುತ್ತದೆ. ದೇವಿಯ ತಲೆಯ ಮೇಲಿರುವ ಚಂದ್ರನ 
ಭಾವ ಹೀಗಿರುತ್ತದೆ-ಚಂದ್ರ ಪ್ರಕಾಶವನ್ನು ಹರಡುತ್ತಾನೆ. ಅದೇ ವಿಧವಾಗಿ ಜ್ಞಾನ ಪ್ರಕಾಶ ಹರಡುತ್ತದೆ. ಆದ್ದರಿಂದ ದೇವಿ 
ತಲೆಯ ಮೇಲೆ ಚಂದ್ರನನ್ನು ಇಟ್ಟುಕೊಂಡಿದ್ದಾಳೆ ಎನ್ನುವುದಕ್ಕೆ ಆಕೆ ಜ್ಞಾನದಿಂದ ಕೂಡಿದ್ದಾಳೆ ಎನ್ನುವುದು ಭಾವ. ಹೀಗೆ 
ನಿಜವಾಗಿಯೂ ಇದ್ದರೂ ಕೂಡ ಸಗುಣೋಪಾಸನೆ ಮಾಡುವಾಗ ಆಕೆಯನ್ನು ತಲೆಯ ಮೇಲೆ ಚಂದ್ರನನ್ನು ಇಟ್ಟುಕೊಂಡಂತೆ 
ಸ್ಮರಿಸುವುದು ಆವಶ್ಯಕ. ಚಂದ್ರನನ್ನು ತಲೆಯ ಮೇಲೆ ಇಟ್ಟುಕೊಂಡಿರುವುದರ ಭಾವ, ತಲೆ ಶ್ರೇಷ್ಠವಾದ ಜ್ಞಾನದಿಂದ 
ಕೂಡಿದೆ ಎಂದು. ದೇವಿಯಂತೆಯೇ ಈಶ್ವರನೂ ಕೂಡ `ಸೋಮಾರ್ಧಧಾರಣೇ' ಎಂದರೆ ತಲೆಯಲ್ಲಿ ಚಂದ್ರಕಲೆಯನ್ನು ಧರಿಸಿರುವವನು ಎಂದು ಸ್ತುತಿಸಲ್ಪಟ್ಟಿದ್ದಾನೆ.

ಶ್ಲೋಕದಲ್ಲಿ ಹೇಳಲ್ಪಟ್ಟ ದೇವಿಯ ಎರಡನೆಯ ವಿಶೇಷಣ `ಭವ ಸಂತಾಪ ನಿರ್ವಾಪಣಸುಧಾನದೀಮ್' ಎನ್ನುವುದು. ಈ ಪ್ರಪಂಚದಲ್ಲಿ 
ನಾವು ಹೊಂದುವ ಸುಖ ಶಾಶ್ವತವಾದುದಲ್ಲ. ಅಲ್ಲದೆ, ನಾವು ಪಡೆಯುವ ಸುಖ ದುಃಖಗಳಿಂದ ಕೂಡಿರುತ್ತದೆ. ದುಃಖವಿಲ್ಲದ 
ಸುಖ ಪ್ರಪಂಚದಲ್ಲಿಲ್ಲ. `ಯೌವನಂ ಜರಯಾಗ್ರಸ್ತಂ' ಎಂದು ಹೇಳಲಾಗಿದೆ. ಯೌವನ ಸುಖ ಪಡೆಯುವುದಕ್ಕೆ ಒಂದು 
ಕಾರಣವಾಗಿದೆ. ಮುಪ್ಪು ಇದನ್ನು ಕಬಳಿಸಿ ಬಿಡುತ್ತದೆ. ಈ ರೀತಿ ಒಂದೊಂದು ಸುಖವೂ ತನ್ನ ಶತ್ರುವಾದ ದುಃಖದಿಂದ 
ನಾಶವಾಗುತ್ತದೆ. ಪರಿಶೀಲಿಸಿ ನೋಡಿದರೆ ನಾವು ಯಾವುದನ್ನು ಸುಖವೆಂದು ಭಾವಿಸುತ್ತೇವೋ 
ಹಾಗೂ ಯಾವುದನ್ನು ದುಃಖವೆಂದು ಭಾವಿಸುತ್ತೇವೋ ಅವೆರಡೂ ದುಃಖವೇ ಆಗುತ್ತವೆಂದು ತಿಳಿಯುತ್ತೇವೆ. ದುಃಖಗಳನ್ನು 
ಯಾವಾಗಲೂ ಅನುಭವಿಸಿದರೆ ನಮಗೆ ಒಂದು `ತಾಪ' ಅಥವಾ ಬಿಸಿ ಉಂಟಾಗುತ್ತದೆ. ಇದನ್ನು ದೂರ 
ಮಾಡುವುದಕ್ಕೆ ದಾರಿ ಇದೆಯೇ? ಹೌದು, ಭವಾನಿಯೇ ಆ ದಾರಿ. ಬೇಸಿಗೆಯಲ್ಲಿ ಬೇಗೆಯಿಂದಾಗಿ 
ನಮಗೆ ಶರೀರದಲ್ಲಿ ತಾಪ ಉಂಟಾಗುತ್ತದೆ. ನಾವು ಹೊಳೆಯಲ್ಲಿ ಸ್ನಾನ ಮಾಡಿದರೆ ದುಃಖ ದೂರವಾಗಿ 
ಆನಂದವಾಗುತ್ತದೆ. ಅದೇ ರೀತಿ ಭವಸಾಗರದಲ್ಲಿ ತಾಪವನ್ನು ಅನುಭವಿಸುತ್ತಿರುವ ನಮಗೆ ದೇವಿ 
ತಾಪವನ್ನು ದೂರ ಮಾಡುವ ಹೊಳೆಯಂತೆ ಇದ್ದಾಳೆ. ಯಾವ ವಿಧವಾದ ಹೊಳೆಯಂತೆ ಇದ್ದಾಳೆ? 
`ಸುಧಾನದೀಮ್'-ಅಮೃತದ ಹೊಳೆಯಂತೆ ಇದ್ದಾಳೆ. ಸಾಮಾನ್ಯವಾದ ಹೊಳೆಯಲ್ಲಿ ಮುಳುಗಿದರೇನೆ 
ನಮಗೆ ಆನಂದವಾಗುತ್ತದೆ ಎಂದ ಮೇಲೆ ಅಮೃತದ ಹೊಳೆಯಲ್ಲಿ ಮುಳುಗಿದರೆ ಆನಂದವೂ, ಶಾಂತಿಯೂ, 
ತೃಪ್ತಿಯೂ ನಾವು ಪಡೆಯಬಹುದು ಎನ್ನುವುದರಲ್ಲಿ ಸಂದೇಹವಾದರೂ ಇದೆಯೇ?

ದುಃಖವನ್ನು ದೂರ ಮಾಡಲು ಪಡೆದ ಪರವಸ್ತುವಿನ ಆಕಾರದ ಹೆಸರೇನು ಎನ್ನುವ ಪ್ರಶ್ನೆಗೆ ಉತ್ತರವಾಗಿ ಶ್ಲೋಕದಲ್ಲಿ 
`ಭವಾನೀಂ' ಎಂದು ತಿಳಿಸಲಾಗಿದೆ. ಅದೇನೆಂದರೆ ಸ್ತ್ರೀ ರೂಪದಲ್ಲಿ ದರ್ಶನವನ್ನು ಕೊಡುವ ಪರವಸ್ತು ಇಲ್ಲಿ `ಭವಾನಿ' ಎಂದು 
ಕರೆಯಲ್ಪಟ್ಟಿದ್ದಾಳೆ. 
ಆಕೆಯನ್ನು ವಂದಿಸಿ ನಾವು ತಾಪದಿಂದ ಬಿಡುಗಡೆ ಪಡೆದು ಜ್ಞಾನವನ್ನೂ, ಶಾಂತಿಯನ್ನೂ ಸುಖವನ್ನೂ ಪಡೆಯುವುದು ಅವಶ್ಯಕವೆನ್ನುವುದು ಶ್ಲೋಕದ ಭಾವ.

ವಿದ್ಯೆಯನ್ನು ಏತಕ್ಕಾಗಿ 
ಕಲಿಯಬೇಕು ಎನ್ನುವ ಪ್ರಶ್ನೆ 
ಕೆಲವರಿಗೆ ಉಂಟಾಗಬಹುದು. ಹುಲಿ, 
ಆನೆಯಂತಹ ಬಲವುಳ್ಳ ಪ್ರಾಣಿಗಳು 
ಜೀವಿಸುತ್ತವೆ. ಅವುಗಳಂತೆಯೇ 
ಏಕೆ ಜೀವಿಸಬಾರದು ಎಂದು ಕೂಡ 
ಕೆಲವರಿಗೆ ತೋರಬಹುದು. 
ಭಗವಂತನು ಬಲವುಳ್ಳ 
ಪ್ರಾಣಿಗಳನ್ನು ಸೃಷ್ಟಿಸಿದರೂ 
ಅವುಗಳು ವಿವೇಕಹೀನತೆಯಿಂದಾಗಿ 
ಭಗವಂತನನ್ನು ಪಡೆದುಕೊಳ್ಳುವ 
ಯೋಗ್ಯತೆಯನ್ನು ಪಡೆದಿಲ್ಲ. 
ಮನುಷ್ಯನೂ ಪಾಮರರಂತೆ ಜೀವಿಸಿ 
ಬಾಳನ್ನು 
ವ್ಯರ್ಥಮಾಡಿಕೊಳ್ಳಬಹುದು. 
ಹಾಗಲ್ಲದೆ ವಿವೇಕದಿಂದ ಬಾಳಿ 
ಪರವಸ್ತುವಿನ ದರ್ಶನವನ್ನು 
ಪಡೆದು ಜನ್ಮ ಸಮುದ್ರದಿಂದ 
ಬಿಡುಗಡೆ ಪಡೆಯಬಹುದು. ಈ 
ಶಕ್ತಿಯನ್ನು ದೇವರು 
ಮನುಷ್ಯನಿಗೆ ಕೊಟ್ಟಿದ್ದಾನೆ. 
ಪ್ರಾಣಿಗಳಿಗೂ ಬುದ್ಧಿ ಇದೆ. 
ಆದರೆ ಅವುಗಳ ಬುದ್ಧಿಗೆ 
ನಿಶ್ಚಯವಾಗಿಯೂ ಒಂದು ಎಲ್ಲೆ 
ಇದೆ. ಇದಕ್ಕೆ ವಿರೋಧವಾಗಿ 
ಮನುಷ್ಯನ ಬುದ್ಧಿಗೆ ಎಲ್ಲೆ 
ಇಲ್ಲ. ಅವನು `ನಾನು ಯಾವ 
ವಿಧವಾಗಿ ಬಾಳಿದರೆ 
ಒಳ್ಳೆಯದನ್ನು ಪಡೆಯುವೆನು' 
ಎಂದು ಯೋಚಿಸಿ ಅದಕ್ಕೆ ಸರಿಯಾದ 
ರೀತಿಯಲ್ಲಿ ಬಾಳುವ 
ಶಕ್ತಿಯುಳ್ಳವನು. `ನಾನು 
ನಿಜಕ್ಕೂ ಯಾರು? ಎಲ್ಲಿಂದ ಬಂದೆ? 
ಎಲ್ಲಿಗೆ ಹೋಗುವೆನು?' ಎಂದು 
ಮನುಷ್ಯನು ಯೋಚಿಸಬಲ್ಲನು. 
ಹೀಗೆ ಚಿಂತಿಸುವಿಕೆಯು 
ಅವನನ್ನು ಇತರ ಪ್ರಾಣಿಗಳಿಂದ 
ಬೇರ್ಪಡಿಸುವುದು. 
ಚಿಂತಿಸುವುದನ್ನೇ ಅಲ್ಲದೆ 
ಜ್ಞಾನವಂತನಾಗಿಯೂ 
ಪರವಸ್ತುವನ್ನು ಪಡೆಯುವ 
ಯೋಗ್ಯತೆಯು ಅವನಿಗೆ ಇದೆ. 
ಆದ್ದರಿಂದ `ಜಂತೂನಾಂ ನರಜನ್ಮ 
ದುರ್ಲಭಂ' ಎಂದು ಹೇಳಲಾಗಿದೆ, 
ಅಂದರೆ ಜಂತುಗಳಾಗಿ 
ಹುಟ್ಟುವುದಕ್ಕಿಂತ ಮನುಷ್ಯ 
ಜನ್ಮ ಹೆಚ್ಚು ಎಂದು ಭಾವ. 
ಪ್ರಾಣಿಗಳು ತಮ್ಮ ಬುದ್ಧಿಯಿಂದ 
ಯಾವುದನ್ನು ಹುಡುಕುತ್ತವೆ? ಈ 
ಪ್ರಶ್ನೆಗೆ ವಿಶ್ರಾಂತಿ, ಆಹಾರ, ತಿರುಗುವಿಕೆ ಎಂದೇ ಉತ್ತರವಾಗುತ್ತದೆ.

ಮನುಷ್ಯನೂ ಇವುಗಳಲ್ಲಿಯೇ 
ತೊಡಗಿರುವವನಾದರೆ ಅವನಿಗೂ 
ಪ್ರಾಣಿಗಳಿಗೂ ವ್ಯತ್ಯಾಸವೇ 
ಇರುವುದಿಲ್ಲ. ಹಾಗಿದ್ದರೆ 
ಮನುಷ್ಯ ಜನ್ಮ ವ್ಯರ್ಥ 
ಮಾಡಿಕೊಂಡಂತೇ ಸರಿ. ಆದ್ದರಿಂದ 
ನಾವು ಆ ರೀತಿ ಇರಬಾರದು. ನಾವು 
ಜ್ಞಾನ ಪಡೆಯಲು ಖಂಡಿತ 
ಪ್ರಯತ್ನಪಡಬೇಕು. ಜ್ಞಾನ 
ಪಡೆಯುವುದು ಹೇಗೆ? ಕೆಲವರು 
ತಾವಾಗಿಯೇ, ಗುರುವಿನ 
ಉಪದೇಶವಿಲ್ಲದೆ, ವಿದ್ಯೆ  
ಪಡೆದಿದ್ದೇವೆಂದು 
ಹೇಳುತ್ತಾರೆ. ಆದರೆ ಹಾಗೆ 
ಮಾಡುವುದು ಸರಿಯಲ್ಲ. ನಾವು ಈಗ 
ಮಾತನಾಡುತ್ತೇವೆ. ನಾವು 
ಚಿಕ್ಕವರಾಗಿದ್ದಾಗ ನಮ್ಮ 
ತಾಯಿಯೋ, ತಂದೆಯೋ, 
ಅಲ್ಲಿದ್ದವರೋ ಮಾತನಾಡದೆ 
ಇದ್ದಿದ್ದರೆ ನಮಗೆ ಮಾತನಾಡಲು 
ಬರುತ್ತಿರಲಿಲ್ಲ. ಆದ್ದರಿಂದ 
ಅವರ ಮಾತುಗಳನ್ನು ಕೇಳಿಯೇ 
ನಾವು ಮಾತನಾಡುವುದನ್ನು 
ಕಲಿತುಕೊಂಡಿದ್ದೇವೆ. `ಯಾರ 
ಸಹಾಯವೂ ಇಲ್ಲದೆ ನಾನು ದೊಡ್ಡ 
ವಿದ್ವಾಂಸನಾಗಿ ಬಿಟ್ಟೆ' ಎಂದು 
ಹೇಳಿಕೊಂಡರೆ ಅದು ಸುಳ್ಳೇ ಆಗುತ್ತದೆ. ನಾವು ಕಲಿಯಬೇಕಾದುವು ಅನೇಕವಿವೆ. ಹೇಗೆ?

\begin{shloka}
ಜಾತ್ಯಂಧಾ ಜಾತಿಬಧಿರಾ ಜಾತಿಮೂಕಾಶ್ಚ ತೇ ಜನಾಃ|\\
ಸಮ್ಯಗಾರಾಧಿತಾ ಯೈರ್ನ ಸಂತೋ ವಿಜ್ಞಾನ ಚಿಂತವಃ||
\end{shloka}

ಎಂದು ಹೇಳಲ್ಪಟ್ಟಿದೆ. ಯಾರು 
ಜ್ಞಾನ ಸಮುದ್ರವಾಗಿ 
ಪ್ರಕಾಶಿಸುವ ಮಹಾತ್ಮರ ದರ್ಶನ 
ಪಡೆಯಲಿಲ್ಲವೋ, ಅವರ 
ಉಪದೇಶವನ್ನು ಕೇಳಲಿಲ್ಲವೋ, 
ಅವರ ಮಾತಿನಂತೆ ಬಾಳನ್ನು 
ತಿದ್ದುಕೊಳ್ಳಲಿಲ್ಲವೋ, 
ಅಂಥಹವರು ಹುಟ್ಟಿನಿಂದಲೇ 
ಕುರುಡರು; ಹುಟ್ಟಿನಿಂದಲೇ 
ಕಿವುಡರು; ಹುಟ್ಟಿನಿಂದಲೇ 
ಮೂಕರು; ಎಂದು ಅರ್ಥ. ನಾವು ಈ 
ರೀತಿ ಕುರುಡರಾಗಿಯೋ, 
ಕಿವುಡರಾಗಿಯೋ, ಮೂಕರಾಗಿಯೋ, 
ಇರಬಾರದು. ನಾವು ಜ್ಞಾನ 
ಸಮುದ್ರರಾಗಿ ಪ್ರಕಾಶಿಸುವ ಮಹಾತ್ಮರನ್ನು ಪೂಜಿಸಿ ಜ್ಞಾನ ಪಡೆಯಬೇಕು.

ಸುರಂಗದಿಂದ ವಜ್ರವನ್ನು 
ತೆಗೆದರೆ ಅದು ವಜ್ರವಾದರೂ ಕೂಡ 
ಹೊಳೆಯುವುದಿಲ್ಲ. ಅದನ್ನು 
ಸಾಣೆ ಹಿಡಿದರೆ ಅದು ಬಹಳವಾಗಿ 
ಪ್ರಕಾಶಿಸುತ್ತದೆ. 
ಸಾಮಾನ್ಯವಾದ ಕಲ್ಲನ್ನು ಸಾಣೆ 
ಹಿಡಿದರೆ ಅದು ಸವೆದು 
ಹೋಗುತ್ತದೆ. ಅಲ್ಲದೆ, ವಜ್ರ 
ಸಹಜವಾಗಿ ಸಿಕ್ಕಿದರೂ ಅದನ್ನು 
ಹರಿತ ಮಾಡುವವರೆಗೆ ಗಾಜನ್ನು 
ಕಡಿಯಲು ಅದರಿಂದ 
ಸಾಧ್ಯವಾಗುವುದಿಲ್ಲ. ಅದೇ 
ರೀತಿ ಮನುಷ್ಯನು ಸಹಜವಾಗಿಯೇ 
ಸ್ವಲ್ಪ ಮಟ್ಟಿಗೆ ಜ್ಞಾನಿ. 
ಶಾಸ್ತ್ರಗಳು ಮತ್ತು ಗುರುವಿನ 
ಉಪದೇಶದಿಂದ ವಿಶೇಷವಾಗಿ ಜ್ಞಾನ 
ಉಂಟಾಗುತ್ತದೆ. ಇದು ಸಾಣೆ ಹಿಡಿದ ವಜ್ರದಂತೆ ಪ್ರಕಾಶಿಸುತ್ತದೆ.

\begin{shloka}
ಏಕಂ ಹಿ ಚಕ್ಷುರಮಲಂ ಸಹಜಾವಬೋಧಃ\\
ವಿದ್ವದ್ಭಿರೇವ ಸಹ ಸಂವಸತಿರ್ದ್ವಿತೀಯಮ್|\\
ಯಸ್ಯಾಸ್ತಿ ನ ದ್ವಯಮಿದಂ ಸ್ಫುಟಮೇವ ಸೋಽನ್ತಃ\\
ತಸ್ಯಾಪ್ಯಮಾರ್ಗಚಲನೇ ವದ ಕೋಽಪರಾಧಃ||
\end{shloka}

ಎಂದು ಒಬ್ಬರು ಹೇಳಿದರು. 
ಮನುಷ್ಯನಿಗೆ ಎರಡು 
ಕಣ್ಣುಗಳಿವೆ. ಅವುಗಳಲ್ಲಿ 
ಒಂದು ಸಾಮಾನ್ಯ 
ಬುದ್ಧಿಯುಳ್ಳದ್ದು. ಕೆಲವರಿಗೆ 
ಸಾಮಾನ್ಯ ಬುದ್ಧಿಯೇ 
ಇರುವುದಿಲ್ಲ. ಅವರಿಗೆ 
ಹೇಳಿಕೊಟ್ಟರೂ 
ತಿಳಿಯುವುದಿಲ್ಲ. ಹಾಗಿದ್ದರೆ 
ಕಷ್ಟ. `ವಿದ್ವದ್ಭಿರೇವ ಸಹ 
ಸಂವಸತಿಃ 
ದ್ವಿತೀಯಮ್'-`ವಿದ್ವದ್ಭಿಃ' 
ವಿದ್ವಾಂಸರು ಅಂದರೆ ಜ್ಞಾನ 
ಸಮುದ್ರರಾಗಿ ಪ್ರಕಾಶಿಸುವ 
ಮಹಾತ್ಮರ ಸಹವಾಸವೇ ಆವಶ್ಯಕ. 
ಅವರನ್ನು ಸೇವಿಸಿ ಅವರು ಮಾಡುವ 
ಉಪದೇಶವನ್ನು ಕೇಳಿ, 
ಮನಸಿನಲ್ಲಿಟ್ಟುಕೊಂಡು, ಆ 
ರೀತಿ ನಡೆಯುತ್ತಾ ಜ್ಞಾನ 
ಪಡೆಯುವುದು ಎರಡನೆಯ 
ಕಣ್ಣಾಗುತ್ತದೆ. ಎರಡು 
ಕಣ್ಣುಗಳೂ ಮನುಷ್ಯನಿಗೆ ಬೇಕು. 
ಸಾಮಾನ್ಯ ಬುದ್ಧಿ ಇಲ್ಲದೆ 
ಇದ್ದು ಕೇವಲ ಗುರು 
ಹೇಳಿದ್ದನ್ನು ಕೇಳಿಕೊಂಡು 
ಇರಬಾರದೆ ಎಂದರೆ ಅದಕ್ಕೆ ಒಂದು 
ಕಥೆ ನೆನಪಿಗೆ ಬರುತ್ತದೆ. 
ನಾಲ್ಕು ಮಂದಿ ಒಬ್ಬ ಗುರುವಿನ 
ಹತ್ತಿರ ಮಂತ್ರಶಾಸ್ತ್ರವನ್ನು 
ಕಲಿತುಕೊಂಡರು. ಪಾಠ ಮುಗಿದ 
ಮೇಲೆ, ತಮ್ಮ ತಮ್ಮ ಮನೆಗಳಿಗೆ 
ಹೋಗುವ ದಾರಿಯಲ್ಲಿ ಒಂದು ಕಾಡು 
ಇದ್ದಿತು. ದಾರಿಯಲ್ಲಿ ಸತ್ತು 
ಹೋದ ಒಂದು ಹುಲಿಯ ಶವವನ್ನು 
ನೋಡಿದರು. ಕೆಲವರ 
ಮನಸ್ಸಿನಲ್ಲಿ `ನಾವು ಶವಕ್ಕೆ 
ಜೀವ ಕೊಡುವ ಸಂಜೀವಿನಿ 
ಮಂತ್ರವನ್ನು 
ಕಲಿತಿದ್ದೇವಲ್ಲಾ, ಅದನ್ನು 
ಪರೀಕ್ಷಿಸಬೇಕು' ಎನ್ನುವ 
ವಿಚಾರ ಬಂದಿತು. ಆದರೆ 
ನಾಲ್ಕನೆಯ ವಿದ್ಯಾರ್ಥಿಗೆ ಅದು 
ಅಪಾಯವನ್ನುಂಟುಮಾಡುತ್ತದೆಂದು ತೋರಿತು. ಅವನು ಪ್ರಯತ್ನ ಪಡದೆ ಇದ್ದರೂ, ಇತರರು ಅವನು ಕೊಟ್ಟ ಎಚ್ಚರಿಕೆಯನ್ನು ಗಮನಿಸಲಿಲ್ಲ. ಆದ್ದರಿಂದ ಅವನು ಸ್ವಲ್ಪ ದೂರ ಸರಿದು ಒಂದು ಮರವನ್ನು ಹತ್ತಿ ಕುಳಿತುಕೊಂಡನು. ಇತರ ಮೂರು ಮಂದಿ ತಾವು ಕಲಿತುಕೊಂಡಿದ್ದ ಮಂತ್ರದ ಶಕ್ತಿಯನ್ನು ತಿಳಿಯುವುದಕ್ಕಾಗಿ ಹುಲಿಗೆ ಜೀವಕೊಟ್ಟರು. ಉತ್ತಮವಾದ ಕೆಲಸವನ್ನು ಮಾಡಿದರು! ಏಕೆಂದರೆ ಬದುಕಿದ ಹುಲಿ ಅವರನ್ನು ತನ್ನ ಹಸಿವು ತೀರಿಸಿ ಕೊಳ್ಳುವುದಕ್ಕಾಗಿ ಕೊಂದು ಬಿಟ್ಟಿತು. ಈ ಮೂರು ಮಂದಿ ವಿದ್ಯಾರ್ಥಿಗಳು ಗುರುವಿನ ಬಳಿ ಚೆನ್ನಾಗಿ ಮಂತ್ರಶಾಸ್ತ್ರವನ್ನು ಕಲಿತಿದ್ದರೂ ಸಾಮಾನ್ಯ ಬುದ್ಧಿ ಇಲ್ಲದ್ದರಿಂದ ಆಪತ್ತಿಗೆ ಒಳಗಾದರು. 

ಇದರಿಂದ ಎರಡು ವಿಧವಾದ ಜ್ಞಾನ 
ಪ್ರತಿಯೊಬ್ಬನಿಗೂ 
ಆವಶ್ಯಕವೆಂದು ತಿಳಿಯುತ್ತದೆ. 
ಒಂದು ಕತ್ತಿ ಇದೆ. ಅದನ್ನು 
ಹರಿತ ಮಾಡದೆ ಇದ್ದರೆ 
ಕಡಿಯಲಾಗುವುದಿಲ್ಲ. ಅದೇ ರೀತಿ 
ಗುರುವಿನ ಬಳಿ ಕಲಿಯದಿದ್ದರೆ 
ನಮ್ಮ ವಿದ್ಯೆ 
ಪ್ರಕಾಶಿಸುವುದಿಲ್ಲ. ಯಾರಿಗೆ 
ಈ ವಿಧವಾದ ಎರಡು ಜ್ಞಾನಗಳು 
ಇಲ್ಲವೋ ಅವನು ಕುರುಡನಂತೆ. 
ಅವನು ತಪ್ಪು ಮಾಡಿದರೆ 
ಆಶ್ಚರ್ಯ ಪಡಬೇಕಾಗಿಲ್ಲ. ನಾವು 
ತಪ್ಪು ಮಾಡದೆ ಬಾಳಬೇಕಾದರೆ ಎರಡು ವಿಧ ಜ್ಞಾನವನ್ನು ಪಡೆಯಬೇಕು.

ವಿದ್ಯೆಯನ್ನು ಯಾವ 
ವಯಸ್ಸಿನಲ್ಲಿ ಕಲಿಯಬೇಕು? 
ಕೆಲವರು `ನಾಳೆ ಕಲಿಯೋಣ, 
ವಯಸ್ಸಾದ ಮೇಲೆ ವಿದ್ಯೆ 
ಕಲಿಯೋಣ' ಎಂದು ಮುಂದಕ್ಕೆ 
ಹಾಕಿಕೊಂಡೇ ಹೋದರೆ 
ಪ್ರಯತ್ನಗಳು ಫಲಿಸುವುದಿಲ್ಲ. 
ಸಂಸ್ಕೃತದಲ್ಲಿ ವಿದ್ಯೆ 
ಕಲಿಯಬೇಕಾದ ವಯಸ್ಸನ್ನು 
ಕುರಿತು `ಲಾಲಯೇತ್‌ಪಂಚ 
ವರ್ಷಾಣಿ ದಶ ವರ್ಷಾಣಿ 
ದಂಡಯೇತ್' ಎಂದು ಹೇಳಲಾಗಿದೆ. 
ಮೊದಲು ಐದು ವರ್ಷಗಳು 
ಮಕ್ಕಳನ್ನು ಬಹಳ ಪ್ರೀತಿಯಿಂದ 
ಸಾಕಬೇಕು ಏಕೆ? ಏಕೆಂದರೆ, 
ಮಕ್ಕಳಿಗೆ ಶರೀರದಲ್ಲಿ 
ಹೆಚ್ಚಾಗಿ ಬಲವಿರುವುದಿಲ್ಲ. 
ಐದು ವರ್ಷಗಳಾದ ಮೇಲೆ ಅವರಿಗೆ 
ಮಾತನಾಡುವುದಕ್ಕೂ, 
ಕೆಲಸಗಳನ್ನು ಮಾಡುವುದಕ್ಕೂ 
ಶಕ್ತಿ ಇರುತ್ತದೆ. ಆದಾದಮೇಲೆ 
ಹತ್ತು ವರ್ಷಗಳು `ದಶ ವರ್ಷಾಣಿ 
ದಂಡಯೇತ್' ಎಂದರೆ ದೊಣ್ಣೆಯಿಂದ 
ಹೊಡೆಯಬೇಕೆಂದಲ್ಲ. ಆದರೆ 
ಚೆನ್ನಾಗಿ ಉಪದೇಶಿಸಬೇಕೆಂದು 
ಭಾವ. ಒಬ್ಬ ಶಾಲೆಗೆ ಹೋಗುವ 
ವಿದ್ಯಾರ್ಥಿ ಬೆಳಗ್ಗೆ ಐದು 
ಘಂಟೆಯಾದ ಮೇಲೆ ಮಲಗಿರಲು 
ಇಷ್ಟಪಟ್ಟರೆ ತಂದೆ ಅವನನ್ನು 
`ನೀನು ಚೆನ್ನಾಗಿ 
ಓದಿಕೊಳ್ಳಬೇಕು ಈ ವಿಧವಾಗಿ 
ಕಾಲಹರಣ ಮಾಡಬಾರದೆಂದು ತಿಳಿಯ 
ಹೇಳಿ ಎಬ್ಬಿಸಬೇಕು ಅದೇ ರೀತಿ 
ವಿದ್ಯಾಬ್ಯಾಸ ಮಾಡುವ 
ಜಾಗದಲ್ಲೂ ಗುರು 
ವಿದ್ಯಾರ್ಥಿಯನ್ನು ಚೆನ್ನಾಗಿ 
ಗಮನವಿಟ್ಟು ಕಲಿಯುವಂತೆ ಮಾಡಬೇಕು.' ಹೀಗೆ ಮಾಡುವುದೇ ಶಿಕ್ಷೆಯಾಗುತ್ತದೆ.    

ಮೇಲೆ ಹೇಳಿದಂತೆ ನಾವು ಓದಿದರೆ 
ವಿದ್ಯೆಯನ್ನು ಚೆನ್ನಾಗಿ 
ಪಡೆಯಬಹುದು. ವಿದ್ಯೆಯಲ್ಲೂ 
ಅನೇಕ ವಿಧವಿದೆ. ಲೌಕಿಕ ವಿದ್ಯೆ, 
ವೈದಿಕ ವಿದ್ಯೆ ಮತ್ತು 
ಆಧ್ಯಾತ್ಮಿಕ ವಿದ್ಯಾ  ಎಂದು 
ವಿಭಾಗ ಮಾಡುತ್ತಾರೆ. ಭಗವಂತನು 
ಗೀತೆಯಲ್ಲಿ `ಆಧ್ಯಾತ್ಮ 
ವಿದ್ಯಾ ವಿದ್ಯಾ ನಾಂ' ಎಂದು 
ಹೇಳಿದ್ದಾನೆ. ಅಂದರೆ ಅನೇಕ 
ವಿದ್ಯೆಗಳಲ್ಲಿ 
ಶ್ರೇಷ್ಠವಾದುದು ಆಧ್ಯಾತ್ಮ 
ವಿದ್ಯೆ. `ನಾವು ಶ್ರೇಷ್ಠವಾದ 
ವಿದ್ಯೆಯನ್ನಲ್ಲವೇ 
ಕಲಿತುಕೊಳ್ಳಬೇಕು. ಆದ್ದರಿಂದ 
ಆಧ್ಯಾತ್ಮ ವಿದ್ಯೆ ಕಲಿಯೋಣ' 
ಎಂದರೆ ಅದಕ್ಕೆ ಬೇಕಾದ 
ಯೋಗ್ಯತೆಗಳೆಲ್ಲವನ್ನೂ ಪಡೆದೆ ನಾವು ಹಾಗೆ ಕಲಿಯಬಹುದು. ಆದರೆ ಅದು ಕಷ್ಟ.

`ಪರೀಕ್ಷ್ಯ ಲೋಕಾನ್ 
ಕರ್ಮಚಿತಾನ್' ಎಂದು 
ಉಪದೇಶದಲ್ಲಿ ಹೇಳಲಾಗಿದೆ. 
ಅಂದರೆ ಲೋಕವನ್ನು 
ಮನಸ್ಸಿನಲ್ಲಿ ಪರೀಕ್ಷಿಸಿ 
ನಂತರ ಎಲ್ಲವನ್ನೂ 
ಬಿಟ್ಟುಬಿಡಬೇಕೆಂದು ಅರ್ಥ. 
ನಾವು ಈ ರೀತಿ ಪರೀಕ್ಷೆ ಮಾಡದೆ 
ಬಿಟ್ಟು ಬಿಟ್ಟವೆಂದರೆ, 
ಬಿಟ್ಟಿದ್ದು ಫಲವನ್ನು 
ಕೊಡುವುದಿಲ್ಲ. ಏಕೆಂದರೆ 
ಮನಸ್ಸು ಪಕ್ವವಾಗಿರುವುದಿಲ್ಲ. 
ಉದಾಹರಣೆಗೆ - ಆಸೆಯುಳ್ಳವನು 
ಸಂನ್ಯಾಸವನ್ನು 
ತೆಗೆದುಕೊಳ್ಳಬಹುದು. ಆದರೂ 
ಅವನು ಸಂಸ್ಕಾರಗಳ ಬಲದಿಂದಾಗಿ 
ಸಂನ್ಯಾಸವನ್ನು ತೆಗೆದುಕೊಂಡ 
ಮೇಲೂ ಹೋಟಲಿಗೆ ಹೋಗಿ ಊಟ ಮಾಡಬಹುದೇ? ಹಾಗೆ ಸಂನ್ಯಾಸ ತೆಗೆದುಕೊಂಡರೆ ಏನು ಪ್ರಯೋಜನ?

ಆದ್ದರಿಂದ ಮೊದಲು ಲೌಕಿಕ 
ವಿದ್ಯೆ ಕಲಿಯಬೇಕು. ಲೌಕಿಕವಾದ 
ವಿದ್ಯೆ ಎಂದರೇನು? ನಾವು ಹೇಗೆ 
ನಡೆದುಕೊಂಡರೆ ಪ್ರಪಂಚಕ್ಕೆ 
ಒಪ್ಪಿಗೆಯಾಗುತ್ತದೋ, ನಾವು 
ಹೇಗೆ ನಡೆದುಕೊಂಡರೆ 
ಪ್ರಪಂಚಕ್ಕೆ ಕಷ್ಟಕೊಟ್ಟಂತೆ 
ಆಗುವುದಿಲ್ಲವೋ, ಅದೇ 
ಲೌಕಿಕವಾದ ವಿದ್ಯೆ. 
ಉದಾಹರಣೆಗೆ ನಾವು ಒಂದು 
ಮನೆಯನ್ನೇ ಕಟ್ಟಿಸಬಹುದು, 
ಹಣವನ್ನೇ ಸಂಪಾದಿಸಬಹುದು 
ರಾಜ್ಯವನ್ನಾಳಬಹುದು, ತಪ್ಪು 
ಮಾಡಿದವರನ್ನು ಶಿಕ್ಷಿಸುವ 
ಸಂದರ್ಭ ಬರಬಹುದು. 
ಹೀಗಿರತಕ್ಕಂತಹ ವಿಷಯಗಳಲ್ಲಿ 
ಅವುಗಳಿಗೆ ತಕ್ಕಂತೆ 
ನಡೆದುಕೊಳ್ಳುವ ಬುದ್ಧಿ 
ಇಲ್ಲದಿದ್ದರೆ ಹೇಗೆ 
ಕೆಲಸಗಳನ್ನು ಮಾಡುವುದು? 
ಆದ್ದರಿಂದಲೇ ನಮ್ಮ ಪೂರ್ವಿಕರು 
ಅರ್ಥಶಾಸ್ತ್ರ, 
ದಂಡನೀತಿಯಂಥವುಗಳನ್ನು 
ಹೇಳಿಕೊಟ್ಟಿದ್ದಾರೆ, ಇಲ್ಲಿ 
ಮೇಲೆ ಹೇಳಿದ ಲೌಕಿಕ ವಿದ್ಯೆ 
ಮಾತ್ರ ನಮಗೆ ಸಾಕಾಗುವುದಿಲ್ಲ. 
ನಾವು ಕೆಲವೊಮ್ಮೆ 
ನೋಡುತ್ತೇವೆ. ನಾವು ಮಾಡುವುದು 
ಒಂದೂ ನಡೆಯುವುದಿಲ್ಲ. ನಾವು 
ಮಾಡುವ ಪ್ರಯತ್ನಗಳು 
ಉತ್ತಮವಾದರೂ ಅವುಗಳು 
ಫಲಿಸುವುದಿಲ್ಲ. ಇದಕ್ಕೆ 
ಕಾರಣವೇನೆಂದರೆ `ದೈವಕೃಪೆ' 
ಇಲ್ಲ ಎನ್ನುವುದು. ಆ ಭಗವಂತನ 
ದಯೆ ಪಡೆಯಲು ನಾವು ಕೆಲವು 
ಅರ್ಚನೆ ಕೆಲಸಗಳನ್ನು 
ಮಾಡಬೇಕು. ದೇವಸ್ಥಾನಕ್ಕೆ 
ಹೋಗಿ ಪೂಜೆ, ಮಾಡುವುದು, 
ಮನೆಯಲ್ಲಿಯೇ ವಿಷ್ಣು 
ಸಹಸ್ರನಾಮವನ್ನೋ 
ಶಿವಸಹಸ್ರನಾಮವನ್ನೋ ಪಾರಾಯಣ 
ಮಾಡುವುದು ಇಂಥವುಗಳನ್ನೂ 
ಮಾಡಬಹುದು. ಹೆಚ್ಚು 
ಬುದ್ಧಿಯುಳ್ಳವರು ಕೆಲವರು 
ಸಂಧ್ಯಾವಂದನೆಯನ್ನೂ, 
ವಿಶೇಷವಾಗಿ ಹೋಮ 
ಮೊದಲಾದವುಗಳನ್ನೂ ಮಾಡುವರು. 
ಇವುಗಳೆಲ್ಲ ವೈದಿಕವಾದ 
ವಿದ್ಯೆ. ಇದು `ಇಹ ಪರ' ವಾದ 
ಸುಖ-ಶಾಂತಿಗಳನ್ನು 
ಕೊಡುವುದಾಗಿ ಇರುವುದು. ಲೌಕಿಕವಿದ್ಯೆ ಕೇವಲ ಇಹ ಫಲ ನೀಡತಕ್ಕದ್ದು.

ಆಧ್ಯಾತ್ಮಿಕ 
ವಿದ್ಯೆಯೆನ್ನುವುದು `ನಾವು 
ಇಷ್ಟೆಲ್ಲಾ 
ಮಾಡುತ್ತಿದ್ದೇವಲ್ಲಾ, ನಾವು 
ಎಷ್ಟು ದಿನಗಳು ಇರುವೆವು? ನಾವು 
ಶಾಶ್ವತವೇ? ಅಥವಾ 
ಶಾಶ್ವತವಲ್ಲವೇ? ಶಾಶ್ವತವಾದರೆ 
ನಾವು ಸಾಯುವುದಾದರೂ ಏಕೆ? 
ಸತ್ತಮೇಲೆ ನಮ್ಮ ರೂಪ 
ಹೇಗಿರುತ್ತದೆ? ಹುಟ್ಟು ಸಾವು 
ಇಲ್ಲದೆ ನಾವು 
ಶಾಶ್ವತವಾಗಿರುವುದಕ್ಕೆ ಯಾವ 
ತತ್ತ್ವವನ್ನು ತಿಳಿಯಬೇಕು?' 
ಇವೆಲ್ಲ ಸೇರುವಂತಹುದು. 
ಇವುಗಳೆಲ್ಲವನ್ನೂ 
ಪಡೆಯುವುದಕ್ಕೆ ಮನುಷ್ಯನ ಜನ್ಮ 
ಉಪಯೋಗವಾದುದರಿಂದ ಮನುಷ್ಯ 
ಜನ್ಮ ಶ್ರೇಷ್ಠವಾದುದು. ಆದರೆ 
ನಾವು ವಿದ್ಯೆ ಕಲಿಯಬೇಕು 
ಎನ್ನುವುದು ಬಹಳ 
ಮುಖ್ಯವಾದುದು. ನಮ್ಮ ಹತ್ತಿರ 
ಬಹಳವಾಗಿ ಔಷಧಗಳು ಇರಬಹುದು. 
ಇವುಗಳಲ್ಲಿ ಯಾವುದಕ್ಕೆ 
ಯಾವುದನ್ನು ಉಪಯೋಗಿಸಿದರೆ ಫಲ 
ಶ್ರೇಷ್ಠವಾಗಿರುತ್ತದೆ 
ಎನ್ನುವ ಯೋಚನೆ ಬೇಕು. ಅದನ್ನು 
ಬೇರೆ ಕೆಲಸಕ್ಕೆ ಉಪಯೋಗಿಸಿದರೆ 
ಅಸಹಾಯವಾಗಬಹುದು. ಅದೇ ರೀತಿ 
ವಿದ್ಯೆಯ ವಿಷಯವೂ ಕೂಡ. ಅಲ್ಲದೆ, 
ವಿದ್ಯೆ ಕಲಿತವನಿಗೆ `ವಿನಯ' 
ಎಂದು ಹೇಳಲಾಗುವ ಸಂಯಮ ಅಥವಾ 
ಗುಣ ಇರಬೇಕು. `ಶೀಲ' 'ಎನ್ನುವುದು 
ಬೇಕು, ಇವೆರಡೂ ಇಲ್ಲದೆ ವಿದ್ಯೆ ವಿದ್ಯೆಯಲ್ಲ ಎಂದು ಒಬ್ಬರು ಹೇಳುತ್ತಾರೆ.

ವಿದ್ಯೆಯಿಂದ ನಾವು 
ಪಡೆದಿರಬೇಕಾದುವು ಯಾವವು 
ಎನ್ನುವುದಕ್ಕೆ ಗೌತಮ 
ಮಹರ್ಷಿಗಳು `ದಯಾ ಸರ್ವಭೂತೇಷು, 
ಕ್ಷಾಂತಿಃ', ಅನಸೂಯ, ಶೌಚಂ, 
ಅನಾಯಸಃ, ಮಂಗಲಂ, ಅಕಾರ್ಪಣ್ಯಂ, ಅಸ್ಪೃಹಾ, ಎಂದು ಹೇಳುತ್ತಾರೆ.

ಭಗವಂತನು ವೈಕುಂಠದಿಂದ 
ಕೆಳಗಿಳಿದು ಹಲವು ವಿಧವಾದ 
ಅವತಾರಗಳನ್ನು ಮಾಡಿದುದು 
ಅವನಿಗೆ ಸಂಬಂಧಿಸಿದಂತೆ 
ಅವಶ್ಯಕವಿಲ್ಲದಿದ್ದರೂ, 
ಶ್ರಮಪಟ್ಟು ಜನರಿಗೆ 
ಒಳ್ಳೆಯದನ್ನು ಮಾಡಬೇಕೆಂಬ 
ಒಂದೇ ವಿಚಾರದಿಂದ ಕೂಡಿದವನಾಗಿ 
ಕರುಣೆಯಿಂದ ಅವತಾರ ಮಾಡಿದನು. 
ಭಗವಂತನು ನಮಗೆ ಇತರರು ಕಷ್ಟ 
ಕೊಡುವಾಗ ಅದನ್ನು ದೂರಮಾಡುವ 
ಶಕ್ತಿಯನ್ನು ಕೊಟ್ಟಿದ್ದಾನೆ. 
ನಾವು ಆ ಶಕ್ತಿಯನ್ನು ನಮಗೆ 
`ದಯೆ' ಅಥವಾ `ಕರುಣೆ' ಇದ್ದರೆ 
ಮಾತ್ರ ಉಪಯೋಗಿಸುತ್ತೇವೆ. ದಯೆ 
ಎಂದರೇನು? ಇತರರು ಕಷ್ಟ 
ಕೊಡುವಾಗ ಅದನ್ನು ದೂರ 
ಮಾಡಬೇಕೆಂಬ ವಿಚಾರ ಉಂಟಾದರೆ 
ಅದೇ ದಯೆ. ಬೇರೆ 
ರೀತಿಯಲ್ಲಿದ್ದರೆ ಅವನನ್ನು 
`ದಯೆ ಇಲ್ಲದವನು' ಎಂದು 
ಹೇಳುತ್ತಾರೆ. ಮನುಷ್ಯನ 
ಶ್ರೇಷ್ಠವಾದ ಗುಣ ದಯೆ. 
ಹಾಗಿರುವ ದಯೆ ಎನ್ನುವ 
ಕರುಣೆಯನ್ನು ನಮ್ಮಲ್ಲಿ 
ವೃದ್ಧಿಪಡಿಸಿಕೊಳ್ಳಬೇಕು. 
ಕೆಲವರಿಗೆ ಸಹಜವಾಗಿಯೇ 
ವಿಶೇಷವಾಗಿ ಕರುಣೆ ಇರುತ್ತದೆ. 
ಕೆಲವರಿಗೆ ಒಳ್ಳೆಯವರ 
ಸಹವಾಸದಿಂದ ಅವರಂತೆ ತಾವೂ 
ಇರಬೇಕೆನ್ನುವ ಭಾವನೆ 
ಉಂಟಾಗಿರುವುದರಿಂದ ದಯೆ 
ಉಂಟಾಗುತ್ತದೆ. ಆದ್ದರಿಂದ 
ನಾವು ದಯೆಯನ್ನು 
ಹೆಚ್ಚಿಸಿಕೊಳ್ಳಬೇಕು. 
ಶಾಲೆಯಲ್ಲಿ ಓದಿನಲ್ಲಿ 
ಹಿಂದಿರುವ ಹುಡುಗರ ಮೇಲೆ 
ದಯೆತೋರಿ, ತನಗೆ ತಿಳಿದಷ್ಟು 
ಅವರಿಗೆ ಹೇಳಿಕೊಟ್ಟು ಅವರನ್ನು 
ಮುಂದಕ್ಕೆ ತರಬೇಕಾದುದು ಒಂದು 
ಶ್ರೇಷ್ಠವಾದ ಗುಣವಾಗುತ್ತದೆ. 
ಅದು ಕರುಣೆಯಾಗುತ್ತದೆ. 
ಅದನ್ನು ನೀವು ಕಲಿತು ಬಾಳಿನಲ್ಲಿ 
ಆಚರಣೆಗೆ ತರವೆವೂ. ಶಾಲೆಯ 
ವಿದ್ಯಾರ್ಥಿಗಳ ವಿಷಯವಾಗಿಯೇ 
ಅಲ್ಲದೆ `ಸರ್ವಭೂತೇಷು' ಎಂದು 
ಹೇಳಿರುವುದರಿಂದ ಇತರರ ವಿಷಯವಾಗಿಯೂ ಅದೇ ರೀತಿ ಕರುಣೆಯುಳ್ಳವರಾಗಿ ಇರಬೇಕು. 

ಆನಂತರ `ಕ್ಷಾಂತಿಃ' ಎನ್ನುವ 
ಗುಣ. ತನಗೆ ಹಿಡಿಸದ 
ಮಾತುಗಳನ್ನು ಯಾರಾದರೂ 
ಆಡಿದರೆ, ತನಗೆ ಹಿಡಿಸದ 
ಕೆಲಸವನ್ನು ಯಾರಾದರೂ 
ಮಾಡಿದರೆ, ಮನುಷ್ಯನು 
ಅಂಥಹವರನ್ನು ಹೊಡೆಯಲು 
ಹೋಗುತ್ತಾನೆ. ಇದಕ್ಕೆ ಶಕ್ತಿ 
ಇಲ್ಲದೆ ಹೋದರೆ ಬೇರೆ 
ಕೆಲಸಗಳನ್ನು ಮಾಡುತ್ತಾನೆ. 
ತಾನೇ ಎದುರಿಗೆ ನಿಲ್ಲದೆ 
ಮರೆಯಲ್ಲಿದ್ದು ಮಾಡುತ್ತಾನೆ. 
`ಅವನು ಮಾಡಿದ, ಆದ್ದರಿಂದ ನಾನೂ 
ಮಾಡುತ್ತೇನೆ' ಎಂದು ಜವಾಬು 
ಹೇಳುತ್ತಾನೆ. ಇದು ಒಳ್ಳೆಯ 
ಮನುಷ್ಯನ ಲಕ್ಷಣವಲ್ಲ. ತನ್ನ 
ಬಗ್ಗೆ ಯಾರಾದರೂ ಕ್ಷುದ್ರವಾಗಿ 
ಮಾತನಾಡಿದರೆ ಕ್ಷುದ್ರವಾಗಿ 
ಏನಾದರೂ ಮಾಡಿದರೆ, ಒಳ್ಳೆಯ 
ಮನುಷ್ಯನಾದವನು ಇದರಿಂದ 
ಏನಾಗುತ್ತದೆ? ಅವನಿಗೆ ನಾಳೆ 
ಬುದ್ಧಿ ಬರುತ್ತದೆ ಎಂದುಕೊಂಡು 
ಕ್ಷಮಿಸಿಬಿಡುತ್ತಾನೆ. 
ಕ್ಷುದ್ರವಾಗಿ ನಡೆದುಕೊಂಡವನು 
ಇವನ ಒಳ್ಳೆಯತನವನ್ನು ನೋಡಿ 
ಕ್ಷಮೆ ಕೇಳಿಕೊಳ್ಳುವ 
ಸ್ಥಿತಿಯೂ ಬಂದುಬಿಡುತ್ತದೆ. 
ಆದ್ದರಿಂದ ನಾವು ಕ್ಷಾಂತಿಃ ಎಂದು ಹೇಳಲ್ಪಟ್ಟ ಈ ಗುಣವನ್ನು ಬೆಳಸಿಕೊಳ್ಳಬೇಕು.

ಆನಂತರದ ಗುಣ `ಅನಸೂಯ' 
ಎನ್ನುವುದು. ಮನುಷ್ಯನು ತಾನು 
ಗುಣವಂತನಾಗಿ ಇಲ್ಲದಿದ್ದರೂ 
ಇನ್ನೊಬ್ಬ ಗುಣವಂತನನ್ನು ಕಂಡು 
ಅವನ ಗುಣಗಳ ಬಗ್ಗೆ ಸರಿಯಾಗಿ 
ಯೋಚಿಸದೆ ಅದರಲ್ಲಿ 
ಕೊರತೆಯನ್ನು ನೋಡಿ ಅದನ್ನು 
ಬಹಿರಂಗಪಡಿಸುವುದು ಸಹಜ. 
ಒಬ್ಬನು ಚೆನ್ನಾಗಿ ಓದುತ್ತಾನೆ 
ಎಂದರೆ ಇನ್ನೊಬ್ಬ ಓದದೆ ಇರುವ 
ಹುಡುಗ, ಇವನಿಗೆ ಅನೇಕ ಸಲ 
ಓದಿದರೇನೆ ಬರುತ್ತೆ. 
ಆದ್ದರಿಂದಲೇ ಓದುತ್ತಾನೆ. 
ಇಲ್ಲದೆ ಇದ್ದರೆ ನಾನೂ 
ಓದುತ್ತೇನೆ. ಇನ್ನು ಮೇಲೆ 
ಓದುತ್ತೇನೆ ಎನ್ನುತ್ತಾನೆ. 
ಆದರೆ ಅವನು ಹೇಳುವಂತೆ 
ಮಾಡುವುದಿಲ್ಲ. ಓದಿನಲ್ಲಿ 
ತನ್ನ ಸೋಮಾರಿತನವನ್ನು 
ಮರೆಸುವುದಕ್ಕಾಗಿ ಇನ್ನೊಬ್ಬ 
ಚೆನ್ನಾಗಿ ಓದುವುದನ್ನೇ 
ಕೊರತೆಯಾಗಿ ಎಣಿಸಿ `ಅವನು 
ರಾತ್ರಿ ಹಗಲು ಓದಿದರೂ 
ಪ್ರಯೋಜನವಿಲ್ಲ' 
ಎನ್ನುತ್ತಾನೆ. ಯಾವ 
ಗುಣವಿದೆಯೋ ಅದನ್ನೇ ನಾವು 
ಹೇಳಬೇಕೇ ವಿನಹ ಸುಮ್ಮನೆ 
`ದೋಷಾರೋಪಣೆ' (ವ್ಯರ್ಥವಾಗಿ ತಪ್ಪು ಪ್ರಕಟಿಸುವುದು) ಮಾಡಬಾರದು.

ಅನಂತರದ ಗುಣ `ಶೌಚಮ್' 
ಎನ್ನಲಾಗುವ ಶುದ್ಧತೆ. 
ಮನುಷ್ಯನಿಗೆ ಮುಖ್ಯವಾದುದು 
ಶೌಚವೆನ್ನುವ ದೊಡ್ಡಗುಣ. 
ಶೌಚವೆಂದರೇನು? ನಮ್ಮ ಎದುರಿಗೆ 
ಒಬ್ಬನು ಬಾಯಲ್ಲಿ ಬೆಟ್ಟು 
ಇಟ್ಟುಕೊಂಡು ಅನಂತರ ನಮ್ಮನ್ನು 
ಮುಟ್ಟಿದರೆ ನಮಗೆ 
ಬೇಸರವಾಗುತ್ತದೆ, `ಅವನು 
ನಮ್ಮನ್ನು ಮುಟ್ಟಿದನಲ್ಲಾ' 
ಎಂದು ನಮಗೆ ತೋರಿದರೂ ಅದು 
ಅವನಿಗೆ ಸಹಜವಾಗಿರುತ್ತದೆ. 
ಇದೇ ರೀತಿ ತಿಂಗಳುಗಟ್ಟಲೆ 
ಸ್ನಾನ ಮಾಡದೆ ಇರುವವನು 
ಹತ್ತಿರಕ್ಕೆ ಬಂದರೆ ನಾವು 
ಮೂಗು 
ಮುಚ್ಚಿಕೊಳ್ಳಬೇಕಾಗಬಹುದು, 
ಆದರೆ ಅಭ್ಯಾಸವಾಗಿರುವ ಅವನಿಗೆ 
ಏನೂ ತಿಳಿಯುವುದಿಲ್ಲ. ಕೆಲವರು 
ಬಟ್ಟೆಗಳನ್ನು ಒಗೆದು 
ಉಟ್ಟುಕೊಳ್ಳುವುದು 
ಸಂಪ್ರದಾಯವಿದೆ (!!) ಎಂದು ಹೇಳಿ 
ತಿಂಗಳಿಗೋ ವರ್ಷಕ್ಕೋ ಒಮ್ಮೆ 
ಬಟ್ಟೆಗಳನ್ನು ಒಗೆದರೆ, ಆ 
ಬಟ್ಟೆಗಳನ್ನು ನೋಡಿದರೇನೆ 
ಸಂಕಟವಾಗುತ್ತದೆ, ನಾವು ಎಂಜಲು 
ಮಾಡಬಾರದು, ಎಂಜಲು ಮೂಲಕ 
ಒಬ್ಬರಿಂದ ಇನ್ನೊಬ್ಬರಿಗೆ 
ವ್ಯಾಧಿಗಳು ಉಂಟಾಗುತ್ತವೆ. 
ಪ್ರತಿದಿನವೂ ಸ್ನಾನಮಾಡಿ ಒಗೆದ ಬಟ್ಟೆಗಳನ್ನು ಉಟ್ಟುಕೊಳ್ಳಬೇಕು. ಇದೇ ಶೌಚ.

`ಅನಾಯಾಸಃ' ಅಥವಾ 
ಸೋಮಾರಿತನವಿಲ್ಲದಿರುವಿಕೆ 
ಎನ್ನುವುದು. ಇದಾದ ಮೇಲೆ 
ಕೆಲವರು `ಬೆಳಗ್ಗೆ' 
ಓದುವುದಕ್ಕೆ ಹೇಳಿದರೆ 
`ಬಹಳಕಷ್ಟ' ಎನ್ನುತ್ತಾರೆ. 
`ಮಧ್ಯಾಹ್ನ' ಎಂದರೆ `ನನಗೇನೂ 
ಆಗಲೇ ಇಲ್ಲ' ಎನ್ನುತ್ತಾರೆ. 
ಚಿಕ್ಕಕೆಲಸವಾದರೂ, ದೊಡ್ಡ 
ಕೆಲಸವಾದರೂ `ಬಹಳ ಕಷ್ಟ' 
ಎನ್ನುತ್ತಾರೆ. ನಿಜಕ್ಕೂ `ಕಷ್ಟ' 
ಎನ್ನುವುದು ಒಂದೂ ಇಲ್ಲ. `ನಾವು 
ಮಾಡಬಹುದು' ಎನ್ನುವ 
ನಿಶ್ಚಯಕ್ಕೆ ಬಂದು, 
ಉತ್ಸಾಹದಿಂದ ತೊಡಗುವುದು 
ಮನುಷ್ಯನ ಕರ್ತವ್ಯ. ಆದ್ದರಿಂದ ಆಯಾಸಕ್ಕೆ  (ಸೋಮಾರಿತನಕ್ಕೆ) ಅವಕಾಶ ಕೊಡಕೂಡದು.

`ಮಂಗಲ' ಎನ್ನುವುದು ಇದಾದ ಮೇಲಿನ 
ಗುಣ. ಕೆಲವರು ಯಾವಾಗಲೂ 
ಕೆಳಮಟ್ಟದ ಶಬ್ದಗಳನ್ನೇ 
ಬಳಸುತ್ತಾರೆ. ನಾವು ಯಾರಾದರೂ 
ದೊಡ್ಡವರೊಡನೆ ಮಾತನಾಡುವಾಗ 
ಅವರಿಗೆ ತಕ್ಕಂತೆ, 
ವಯಸ್ಸಾಗಿರುವವರಾದರೆ ``ಬನ್ನಿ" 
ಎನ್ನುವುದಕ್ಕೆ ಬದಲು ``ದಯಮಾಡಿ" 
ಎಂದರೆ, ಅವರಿಗೆ ಬಹಳ 
ಸಂತೋಷವಾಗಿ ನಮ್ಮ ಮೇಲೆ 
ಪ್ರೀತಿ ಹೆಚ್ಚಾಗುತ್ತದೆ. 
`ಆಸನವನ್ನು ಅಲಂಕರಿಸಿ' ಎಂದರೆ 
ಮಂಗಳವಾದ ಮಾತು. `ಅಲ್ಲಿ ಹೋಗಿ 
ಕೂತುಕೋ' ಎಂದರೆ ಸಾಧಾರಣವಾದ 
ಮಾತು. ಮಂಗಳವಾದ ಮುಖಭಾವವೂ ಇರಬೇಕು. ಇದನ್ನು ನಾವು ಕಲಿಯಬೇಕು.

ಅನಂತರ `ಅಕಾರ್ಪಣ್ಯ'. 
ಮನುಷ್ಯನಲ್ಲಿ ಕಾಮ, ಕ್ರೋಧ, ಲೋಭ 
ಮುಂತಾದ ಕೆಟ್ಟ ಗುಣಗಳು 
ಇರುತ್ತವೆ. ಒಬ್ಬೊಬ್ಬರಿಗೆ 
ಶಕ್ತಿ ಇಲ್ಲದೆ ಹೋಗುವುದರಿಂದ 
ಕಾಮ-ಕ್ರೋಧಗಳು ತಾವಾಗಿಯೇ 
ಬರದೆ ಇದ್ದರೂ ಲೋಭವನ್ನು 
ಗೆಲ್ಲುವುದು ಸ್ವಲ್ಪ 
ಕಠಿಣವಾಗುತ್ತದೆ. ಎಷ್ಟು 
ವಯಸ್ಸಾದರೂ ಇನ್ನೂ ಹಣವನ್ನು 
ಸೇರಿಸಬೇಕೆಂಬ ವಿಚಾರ 
ಲೋಭವಾಗುತ್ತದೆ. ಜೇನು ನೊಣಗಳು 
ಎಷ್ಟು ಸೇರಿಸಿದರೂ ಯಾರೋ 
ಜೇನುಗೂಡಿನಿಂದ 
ಜೇನುತುಪ್ಪವನ್ನೆಲ್ಲ 
ತೆಗೆದುಕೊಂಡು ಹೋಗುವ ಹಾಗೆ. 
ನಾವು ಏಕೆ ಸೇರಿಸಿಟ್ಟ ಅದನ್ನು 
ಯಾರೋ ತೆಗೆದುಕೊಂಡು ಹೋಗುವಂತೆ 
ಮಾಡಬೇಕು? ಒಬ್ಬ ಬಡವ 
ಬೇಕಾಗಿದೆಯೆಂದು ಕೇಳಿದರೆ ಅದು 
ಸರಿಯೆಂದು ನಮಗೆ ತೋರಿದರೆ 
ಸಾಧ್ಯವಾದ ಮಟ್ಟಿಗೆ ಸಹಾಯ 
ಮಾಡಬಹುದೇ ವಿನಹ ಲೋಭಿಯಾಗಬಾರದು. ಇದೇ `ಅಕಾರ್ಪಣ್ಯ' ವೆಂದು ಹೇಳಲ್ಪಟ್ಟಿದೆ.

ಅದಾದ ಮೇಲೆ `ಅಸ್ಪೃಹಾ' ಆಸೆಗಳು 
ಹೆಚ್ಚಾಗುತ್ತಾ 
ಹೆಚ್ಚಾಗುತ್ತಾ ತಪ್ಪುಗಳನ್ನು 
ಮಾಡುವನು. ಹೊಟ್ಟೆಗೆ 
ಇಲ್ಲದವನು ಕಳ್ಳತನ 
ಮಾಡುವುದಕ್ಕಿಂತಲೂ 
ಹಣವುಳ್ಳವನು ಕಳ್ಳತನ 
ಮಾಡುವುದೇ ಹೆಚ್ಚು. ಏಕೆಂದರೆ 
ಅವನಿಗೆ ಆಸೆ ಹೆಚ್ಚು. 
ಆಸೆಗಳನ್ನು ಒಂದು 
ಮಿತಿಯಲ್ಲಿಟ್ಟುಕೊಂಡು ನಾವು 
ವ್ಯವಹರಿಸಿದರೆ ನಾವು 
ದೋಷರಹಿತರಾಗುತ್ತೇವೆ. ಎಲ್ಲ 
ಪಾಪಗಳಿಗೂ ಮೂಲ ಸ್ಥಾನವಾದ 
ಆಸೆಯನ್ನು ನಾವು ನಿಗ್ರಹಿಸಿ ನಾವು ಶ್ರೇಯಸ್ಸನ್ನು ಪಡೆಯಬೇಕು.

ವಿದ್ಯೆಯಿಂದ ನಾವು ಈ ಎಂಟು 
ಗುಣಗಳನ್ನು ಖಂಡಿತವಾಗಿಯೂ 
ಪಡೆಯಬೇಕು. ವಿದ್ಯೆ ಕಲಿತೂ ಈ 
ಗುಣಗಳನ್ನು ನಾವು ಪಡೆಯದೆ 
ಹೋದರೆ ಕತ್ತಿ ಅಥವಾ 
ಚಾಕುವನ್ನು ಇಟ್ಟುಕೊಂಡು 
ತರಕಾರಿಯನ್ನು ಕತ್ತರಿಸದೆ 
ನಮ್ಮ ಶರೀರದ ಅಂಗಗಳನ್ನೇ 
ಕತ್ತರಿಸಿಕೊಂಡಂತೆಯೇ ಸರಿ. ಈ 
ರೀತಿ ನಾವು ನಮ್ಮ ಬಾಳನ್ನು 
ಹಾಳು ಮಾಡಿಕೊಳ್ಳಬಾರದು. 
ಆದ್ದರಿಂದ ಆ ನಿಗ್ರಹ, ಗುಣ 
ಮೊದಲಾದವುಗಳನ್ನು ನಮ್ಮ 
ದೃಢನಿಶ್ಚಯವಾಗಿ ಸ್ವೀಕರಿಸಿ 
ವಿದ್ಯಾವಂತರಾಗಿದ್ದರೆ ಮಾತ್ರ 
ದೇಶದ ಒಳ್ಳೆಯ 
ನಾಗರೀಕರಾಗಿರಬಲ್ಲೆವು. ಇತರರಿಗೂ ದಾರಿ ತೋರಿಸಬಹುದು, ಶ್ರೇಯಸ್ಸು ಪಡೆಯಬಹುದು.

ನೀವೆಲ್ಲರೂ ಗುಣಶೀಲಗಳನ್ನು 
ಪಡೆದು ಬಾಳಬೇಕೆಂದು ನಾವು 
ಪ್ರಾರ್ಥಿಸುತ್ತೇವೆ. ಈಶ್ವರನು ನಿಮಗೆಲ್ಲರಿಗೂ ಒಳ್ಳೆಯದನ್ನು ಮಾಡಲಿ. 

\newpage

\section{ಭಕ್ತಿ}

\begin{shloka}
ತ್ರಯೀವೇದ್ಯಂ ಹೃದ್ಯಂ ತ್ರಿಪುರಹರಮಾದ್ಯಂ ತ್ರಿನಯನಂ\\
ಜಟಾಭಾರೋದಾರಂ ಚಲದುರಗಹಾರಂ ಮೃಗಧರಮ್|\\
ಮಹಾದೇವಂ ದೇವಂ ಮಯಿ ಸದಯಭಾವಂ ಪಶುಪತಿಂ\\
ಚಿದಾಲಂಬಂ ಸಾಂಬಂ ಶಿವಮತಿವಿಡಂಬಂ ಹೃದಿ ಭಜೇ||
\end{shloka}

ಜ್ಞಾನವನ್ನು ಪಡೆಯಲು ಮೂರು 
ಮಾರ್ಗಗಳು ಹೇಳಲ್ಪಟ್ಟಿವೆ. 
ತೀವ್ರವಾದ 
ವೈರಾಗ್ಯವಿರುವುದರಿಂದ ಯಾರ 
ಮನಸ್ಸು ತಮ್ಮ ವಶದಲ್ಲಿರುವುದೋ 
ಅವರು ಜ್ಞಾನಮಾರ್ಗವನ್ನು 
ಅವಲಂಬಿಸಬಹುದು. ಕೆಲವರಿಗೆ 
ಶಾಸ್ತ್ರಗಳಲ್ಲಿ ಹೇಳಲ್ಪಟ್ಟ 
ಕರ್ಮಗಳ ಬಗ್ಗೆ ಹೆಚ್ಚು 
ತಿಳುವಳಿಕೆ. ಆ ಕರ್ಮಗಳ 
ವಿಷಯದಲ್ಲಿ ಹೆಚ್ಚು ಶ್ರದ್ಧೆ 
ಇರುತ್ತದೆ. ಇಂಥಹವರು 
ಕರ್ಮಮಾರ್ಗದಲ್ಲಿ ನಡೆಯಬಹುದು. 
ಮತ್ತೆ ಕೆಲವರಿಗೆ 
ಶಾಸ್ತ್ರಗಳಲ್ಲಿ ಹೇಳಲ್ಪಟ್ಟ 
ಕರ್ಮಗಳ ಬಗ್ಗೆ ಹೆಚ್ಚಾಗಿ 
ತಿಳುವಳಿಕೆಯಾಗಲಿ ಮೇಲ್ಮಟ್ಟದ 
ವೈರಾಗ್ಯವಾಗಲಿ ಇಲ್ಲದೆ ಹೋದರೆ 
ಅಂಥಹವರಿಗೆ ಏನು ದಾರಿ? ಇದಕ್ಕೆ 
`ನೀನು ಭಗವಂತನನ್ನು 
ಭಜಿಸುತ್ತಾ ಇರು. ಅದು ನಿನಗೆ 
ಉತ್ತಮವಾದ ದಾರಿಯಾಗುತ್ತದೆ' ಎಂದು ನಮ್ಮ ಹಿಂದಿನವರು ಹೇಳಿದ್ದಾರೆ.

`ಜ್ಞಾನಂ ಕರ್ಮ ಚ ಭಕ್ತಿಶ್ಚ 
ನೋಪಾಯೋನ್ಯೋಽಸ್ತಿ 
ಕುತ್ರಚಿತ್|' ಎಂದು ಇದೆ. 
ಜ್ಞಾನಮಾರ್ಗ, ಕರ್ಮಮಾರ್ಗ, 
ಭಕ್ತಿಮಾರ್ಗ ಎಂದು ಮೂರು 
ಮಾರ್ಗಗಳಿವೆ. ಈ ಮೂರು 
ಮಾರ್ಗಗಳನ್ನು ಬಿಟ್ಟು ಬೇರೆ 
ಯಾವ ಮಾರ್ಗವೂ ಇಲ್ಲ. ಇದೇ 
ಸಾರಾಂಶ. ಎಲ್ಲಾ ವಿಧವಾದ 
ಬಂಧಗಳಿಂದ ಬಿಡುಗಡೆ ಪಡೆದು 
ಪರವಸ್ತುವನ್ನು ಅರಿತ 
ಜೀವನ್ಮುಕ್ತರೂ ಸಗುಣ ಪರಮೇಶ್ವರನನ್ನು ಆರಾಧಿಸುವುದು ಉಂಟೆಂದು ಭಾಗವತದಲ್ಲಿ ಹೇಳಿದೆ.

\begin{shloka}
ಆತ್ಮಾರಾಮಾಶ್ಚ ಮುನಯೋ ನಿರ್ಗ್ರಂಥಾ ಅಪ್ಯರುಕ್ರಮೇ|\\
ಕುರ್ವಂತ್ಯಹೈತುಕೀಂ ಭಕ್ತಿಂ ಇತ್ಥಂ ಭೂತಗುಣೋ ಹರಿಃ||
\end{shloka}

ಭಗವಂತನ ಗುಣಗಳು ಹೇಗಿರುವುವು 
ಎಂದರೆ ಯಾರು-ನಿರ್ಗುಣ 
ಪರವಸ್ತುವಿನ ವಾಸ್ತವಿಕವಾದ 
ಸ್ವರೂಪವನ್ನು 
ತಿಳಿದವರಾಗಿದ್ದಾರೋ ಅಂಥಹವರು 
ಕೂಡ ಸಗುಣ ಪರಮೇಶ್ವರನ 
ಸ್ವರೂಪವನ್ನು 
ಆರಾಧಿಸುತ್ತಾರೆಂದು 
ಹೇಳಲ್ಪಟ್ಟಿದೆ. ಏಕೆಂದರೆ 
ಭಗವಂತನ ಆ ಕಲ್ಯಾಣ ಗುಣಗಳು 
ಇಂಥಹವು. ಯಾರಾದರೂ ಅವನನ್ನು ಬಿಟ್ಟುಬಿಡಲು ಇಷ್ಟಪಟ್ಟರೂ ಅದು ಬೇಡವೆಂದು ತೋರುತ್ತದೆ.

\begin{shloka}
ತ್ವಯಿ ಜನಾರ್ದನ ಭಕ್ತಿರಚಂಚಲಾ ಯದಿ ಭವೇತಫಲಪ್ರವಣಾಮಮ|\\
ಅಭಿಲಷಾಮ್ಯಪವರ್ಗ ಪರಾಙ್ಮುಖಃ ಪುನರಪೀಹ ಶರೀರ ಪರಿಗ್ರಹಮ್||
\end{shloka}

ಮೋಕ್ಷ ಲಾಭವಾದವನಿಗೆ 
ಶರೀರವೆನ್ನುವುದು 
ಇರುವುದಿಲ್ಲ. ಅನಂತವಾದ 
ಚೈತನ್ಯ ಸ್ವರೂಪನಾಗಿ 
ಅವನಿರುತ್ತಾನೆ. ಆಗ ಅವನು ಆನಂದ 
ರೂಪನಾಗಿರುತ್ತಾನೆಂದು 
ಶಾಸ್ತ್ರ ಹೇಳುತ್ತಿದ್ದರೂ 
ಭಕ್ತರಿಗೆ ಭಗವದ್ಭಕ್ತಿಯಿಂದ 
ಉಂಟಾದ ಆನಂದವನ್ನು, ಅನುಭವಿಸಿ 
ಏನೆಂದು ತೋರುತ್ತದೆ 
ಎನ್ನುವುದನ್ನು ಕುರಿತು, 
`ತ್ವಯಿ ಜನಾರ್ದನ ಭಕ್ತಿರಚಂಚಲಾ ಯದಿ ಭವೇತಫಲಪ್ರವಣಾ ಮಮ' ಎಂದು ಹೇಳಲ್ಪಟ್ಟಿದೆ.

`ಎಲೈ ಭಗವಂತನೇ! ನಿನ್ನ ಮೇಲೆ 
ಅಚಂಚಲವಾದ ಭಕ್ತಿ ಇದ್ದರೆ 
ಮಾತ್ರ ಸಾಕು. ಯಾವ ಫಲವನ್ನು 
ಇಷ್ಟಪಟ್ಟು ನಾನು ಭಕ್ತಿಯನ್ನು 
ಮಾಡುವಂತಾಗದಿರಲಿ' - ಒಂದು 
ಫಲಾಪೇಕ್ಷೆಯೂ ಇಲ್ಲದೆ ನಾನು 
ನಿನಗಾಗಿ ಎಂದೇ ಭಕ್ತಿ 
ಮಾಡುತ್ತೇನೆ. ಎಂದರೆ 
ಅಭಿಲಾಷಾಮ್ಯಪವರ್ಗಪರಾಙ್ಮುಖಃ ಪುನರಪೀಹ ಶರೀರ ಪರಿಗ್ರಹಮ್ `ಮೋಕ್ಷವೆನ್ನುವುದೂ ಬೇಡ. (ಮತ್ತೆ ನಿನಗೆ ಏನು ಬೇಕು?) ಮತ್ತೆ ಮತ್ತೆ ನನಗೆ ಶರೀರ ಉಂಟಾಗುತ್ತಾ ನಿನಗೆ ಸೇವೆ ಮಾಡುತ್ತಾ ಹೋದರೆ ಅಷ್ಟೇ ನನಗೆ ಸಾಕು.'

ಭಕ್ತಿ ಮಾರ್ಗದಲ್ಲಿರುವವನಿಗೆ 
ಯಾವಾಗಲೂ ಆನಂದವಿದ್ದೇ 
ಇರುತ್ತದೆ. ಆದರೆ 
ಜ್ಞಾನಮಾರ್ಗದಲ್ಲಿರುವವರಿಗೆ 
ಸಾಧನೆ ಮಾಡುವ ಕಾಲದಲ್ಲಿ 
ಆನಂದವಾಗಿರುವುದಿಲ್ಲ. 
ಯಾವಾಗಲೂ ಅವರು 
ಇಂದ್ರಿಯಗಳನ್ನೂ ಮನಸ್ಸನ್ನೂ 
ತಮ್ಮ ವಶದಲ್ಲಿಟ್ಟುಕೊಂಡಿರಬೇಕು. ಆದರೆ ಇವುಗಳನ್ನು ನಿಗ್ರಹಿಸುವುದು ಕಷ್ಟವಾದುದೇ.

`ಅವ್ಯಕ್ತಾ ಹಿ ಗತಿರ್ದುಃಖಂ 
ದೇಹವದ್ಭಿರವಾಪ್ಯತೇ' 
ಎಂದಲ್ಲವೇ ಭಗವಂತನು 
ಹೇಳಿರುವುದು! ಆ ಅವ್ಯಕ್ತವಾದ 
ಪರವಸ್ತುವನ್ನು ಚಿಂತಿಸುವುದು 
ಕಷ್ಟ. ಹೀಗಿದ್ದರೂ 
ಜ್ಞಾನಮಾರ್ಗದಲ್ಲಿ 
ನಡೆಯುವವನು ಪ್ರಯತ್ನಿಸುತ್ತಾ 
ಮನಸ್ಸನ್ನು ದೃಢಪಡಿಸಿಕೊಂಡು 
ಮುಂದುವರೆಯಬೇಕು. ಆದರೆ 
ಭಗವಂತನ ಆರಾಧನೆ ಹಾಗಲ್ಲ. 
ಭಗವಂತನ ಲೀಲೆಗಳು ಮನಸ್ಸನ್ನು ಆಕರ್ಷಿಸಿ ಅದನ್ನು ಸ್ಥಿರಪಡಿಸುತ್ತವೆ.

ಮಧುಸೂದನ ಸರಸ್ವತಿಯವರು 
ಅದ್ವೈತ ಮಾರ್ಗದಲ್ಲಿ ಬಹಳ 
ಸಿದ್ಧರು. ಅವರು ಶುದ್ಧವಾದ 
ಚೈತನ್ಯವನ್ನು ಬಿಟ್ಟು ಬೇರೆ 
ಯಾವುದೂ ಸತ್ಯವಾಗಿಯೂ 
ಇಲ್ಲವೆನ್ನುವುದನ್ನು 
ಸ್ಪಷ್ಟಪಡಿಸುವ `ಅದ್ವೈತ 
ಸಿದ್ಧಿ' ಎನ್ನುವ ಗ್ರಂಥವನ್ನು 
ಬರೆದಿದ್ದಾರೆ. ಅಂಥಹವರು ಕೂಡ 
ಒಂದು ಕಡೆ ಏನು ಹೇಳಿದ್ದಾರೆ 
ಎನ್ನುವುದು ಗಮನಾರ್ಹ. 
`ನಿರ್ಗುಣವಾಗಿಯೂ, 
ನಿಷ್ಕಲವಾಗಿಯೂ, ಆನಂದವಾಗಿಯೂ 
ಇರುವ ಯಾವುದೋ ಚೈತನ್ಯವನ್ನು 
ಯೋಗಿಗಳು ಧ್ಯಾನದಲ್ಲಿ ಕಾಣುತ್ತಾರೆ. ಅದನ್ನು ಅವರು ನೋಡಲಿ. ಹೀಗೆಂದು ಹೇಳಿ -

\begin{shloka}
`ಅಸ್ಮಾ ಕಂ ತು ತದೇವ ಲೋಚನ ಚಮತ್ಕಾರಾಯ ಭೂಯಾಚ್ಚಿರಮ್|\\
ಕಾಲಿಂದೀಪುಲಿನೇಷು ತತ್ಕಿಮಪಿ ಯನ್ನೀಲಂ ಮಹೋ ಧಾವತಿ||'
\end{shloka}

ಎಂದು ತಮ್ಮ ಅಭಿಪ್ರಾಯವನ್ನು 
ಪ್ರಕಟಿಸಿದ್ದಾರೆ. `ನಮ್ಮ 
ಕಣ್ಣುಗಳಿಗೆ ಯಮುನಾ ದಡದಲ್ಲಿ 
ಆಟವಾಡುತ್ತಿರುವ ಶ್ಯಾಮಲ 
ಗಾತ್ರನಾದ ಕೃಷ್ಣನು ಯಾವಾಗಲೂ 
ಕಾಣುತ್ತಾನೆ' ಎನ್ನುವುದು 
ತಾತ್ಪರ್ಯ. ಸಗುಣೋಪಾಸನೆ 
ಮಾಡುವವರಿಗೆ, ಭಗವಂತನ 
ಭಕ್ತಿಮಾರ್ಗದಲ್ಲಿರುವವರಿಗೆ 
ಆನಂದವೆನ್ನುವುದು ಯಾವಾಗಲೂ 
ಇರುತ್ತದೆ. ಮನಸ್ಸು 
ದೃಢವಿಲ್ಲದಿರುವವರಿಗೆ 
ಭಗವದ್ಭಕ್ತಿಯಿಂದ ತಾನಾಗಿಯೇ 
ಮನಸ್ಸು ಸ್ಥಿರವಾಗಿ ಆ 
ನಿರ್ಗುಣವಾದ ಆತ್ಮ ಸಾಕ್ಷಾತ್ಕಾರ ಭಗವಂತನ ದಯೆಯಿಂದ ತಾನಾಗಿಯೇ ಆಗುವುದು.

\begin{shloka}
`ತೇಷಾಮೇವಾನುಕಂಪಾರ್ಥಮಹಮಜ್ಞಾನಜಂ ತಮಃ|\\
ನಾಶಯಾಮ್ಯಾತ್ಮಭಾವಸ್ಥೋ ಜ್ಞಾನದೀಪೇನ ಭಾಸ್ವತಾ||
\end{shloka}

ಎಂದು ಗೀತೆಯಲ್ಲಿ ಭಗವಂತನು 
ತಿಳಿಸಿದ್ದಾನೆ. ಭಗವಂತನು 
ಭಕ್ತನ ಮೇಲೆ ಅನುಗ್ರಹಿಸಿ 
ಹೃದಯದಲ್ಲಿ ಆಸೀನನಾದನೆಂದರೆ 
ಭಗವಂತನು ತಾನಾಗಿ ಜ್ಞಾನ 
ದೀಪವನ್ನು ಪ್ರಕಾಶಿಸಿ 
ಅಜ್ಞಾನವನ್ನು ನಾಶ ಮಾಡುತ್ತಾನೆ. ಇದು ಭಗವಂತನ ಅಭಿಪ್ರಾಯ. ಭಕ್ತಿಯ ಹಿರಿಮೆ ಇಂಥಹದು.

ಭಕ್ತಿ ಎನ್ನುವುದು ಏನು 
ಎನ್ನುವುದಕ್ಕೆ `ಸಾತ್ವಸ್ಮಿನ್ 
ಪರಮಪ್ರೇಮ ರೂಪಾ'-ಅದು ಭಗವಂತನ 
ವಿಷಯದಲ್ಲಿ ಶ್ರೇಷ್ಠವಾದ ಪ್ರೇಮರೂಪ ಎಂದು ನಾರದರು ಹೇಳಿದರು.

ಶಾಂಡಿಲ್ಯರು ಒಬ್ಬ ಮಹರ್ಷಿ, 
ಅವರು ಭಕ್ತಿಯನ್ನು ಕುರಿತು 
ಒಂದು ಗ್ರಂಥವನ್ನು 
ಬರೆದಿದ್ದಾರೆ. `ಈ ಭಕ್ತಿ 
ಎಲ್ಲರೂ ಮಾಡುತ್ತಿರುವ ವಿಷಯ. 
ಆದರೆ ಇದಕ್ಕೆ ಆಧಾರ ಬೇರೆ 
ಬೇರೆಯಾಗಿರುವುದರಿಂದ ಯಾವುದೋ 
ರೂಪದಲ್ಲಿ ಅದು ಬದಲಾಗುತ್ತದೆ' 
ಎನ್ನುವ ಅಭಿಪ್ರಾಯವನ್ನು ಅವರು 
ತಿಳಿಸಿದ್ದಾರೆ. ಮನುಷ್ಯನಾಗಿ 
ಹುಟ್ಟಿದ ಪ್ರತಿಯೊಬ್ಬನೂ 
ಯಾವದಾದರೂ ಒಂದು ವಿಷಯದಲ್ಲಿ 
ಪ್ರೀತಿ ಇಟ್ಟುಕೊಂಡವನಾಗಿಯೇ 
ಬಾಳುತ್ತಾನೆ ಹೊರತು ಒಂದರಲ್ಲೂ ಪ್ರೀತಿ ಇಲ್ಲದೆ ಬಾಳುವುದಿಲ್ಲ.

\begin{shloka}
`ಬಾಲಸ್ತಾವತ್ ಕ್ರೀಡಾ ಸಕ್ತ-\\
ಸ್ತರುಣಸ್ತಾವತ್ತರುಣೀ ಸಕ್ತಃ|\\
ವೃದ್ಧಸ್ತಾವಚ್ಚಿಂತಾಸಕ್ತಃ||'
\end{shloka}

ಎನ್ನುವಂತೆ ಎಲ್ಲರಿಗೂ 
ಪ್ರಿಯವಾಗಿರುವಂಥಹ ಒಂದು 
ವಸ್ತು ಇರುತ್ತದೆ. 
`ಬಾಲಸ್ತಾವತ್ 
ಕ್ರೀಡಾಸಕ್ತಃ'-ಮಕ್ಕಳಿಗೆ 
ಯಾವಾಗಲೂ ಆಡುವುದೆಂದರೆ ಇಷ್ಟ. 
ಶರೀರಕ್ಕೆ ಆಹಾರವೂ ಬೇಡ. 
ಆಟದಲ್ಲಿ ಅಷ್ಟು ಆಸಕ್ತಿ. ಇದು 
ಇಲ್ಲದಿದ್ದರೆ ಮಕ್ಕಳ 
ಬೆಳವಣಿಗೆ ಕೆಟ್ಟುಹೋಗುತ್ತದೆ. 
ಆದ್ದರಿಂದಲೇ ಮಕ್ಕಳು 
ಆಡುತ್ತಿದ್ದರೆ ಅವರನ್ನು ಯಾರು 
ತಡೆಯುವುದಿಲ್ಲ. 
`ತರುಣಸ್ತಾವತ್ತರುಣೀಸಕ್ತಃ'-ಯೌವನದಲ್ಲಿ ಹೆಣ್ಣು ಇಂಥಹವುಗಳ ಮೇಲೆ ಆಸಕ್ತಿ ಇರುವುದುಂಟು. `ವೃದ್ಧಸ್ತಾವಚ್ಚಿಂತಾ ಸಕ್ತಃ'-ಮುಪ್ಪಿನಲ್ಲಿ ಚಿಂತೆ ಇರುವುದು ಸಹಜ. `ಅದು ಹಾಗೆ ಆಗಲಿಲ್ಲ, ಇದು ಹೀಗೆ ಆಗಲಿಲ್ಲ. ಅದು ಆಗಲಿಲ್ಲ, ಇದು ಆಗಲಿಲ್ಲ' ಎಂದು ಚಿಂತೆ ಮಾಡುತ್ತಿರುವುದೇ ವಯಸ್ಸಾದವರಿಗೆ ಇಷ್ಟ. ಹಾಗೆ ಚಿಂತೆ ಇಲ್ಲದೆ ಅವರು ಇರಲಾರರು.

ಹೀಗೆ ಮನುಷ್ಯನ ಜೀವನದಲ್ಲಿ 
ಯಾವುದಾದರೂ ವಿಷಯದಲ್ಲಿ 
ಪ್ರೇಮವಿರುತ್ತದೆ. ಎಲ್ಲವನ್ನೂ 
ಬಿಟ್ಟುಬಿಟ್ಟು ಕಾವಿ ಬಟ್ಟೆ 
ಧರಿಸಿದ 
ಸಂನ್ಯಾಸಿಗಳಾಗಿರುವವರಿಗೆ 
ಯಾವ ವಿಷಯದಲ್ಲಿ ಪ್ರೇಮ ಎಂದರೆ 
ಅವರಿಗೆ ಆತ್ಮವನ್ನು 
ಪಡೆಯಬೇಕೆನ್ನುವ ವಿಷಯದಲ್ಲಿ 
ಪ್ರೇಮ. ಪ್ರಿಯವಾಗಿರುವ ಈ ಆತ್ಮ 
ಒಂದಿರುವಾಗ ಯಾವುದಕ್ಕಾಗಿ 
ಸಾವಿರಾರು ವಿಷಯಗಳಲ್ಲಿ 
ಪ್ರೀತಿಯನ್ನಿಟ್ಟುಕೊಳ್ಳಬೇಕು? ಇದು ಒಂದು ಸಾಕು ಎಂದು ಸ್ಥಿರವಾಗಿದ್ದು ಇದರ ಸಾಕ್ಷಾತ್ಕಾರ ಪಡೆಯಬೇಕೆಂದು ಸಂನ್ಯಾಸಿಗಳೂ ಪ್ರೇಮ ಇಟ್ಟುಕೊಂಡೇ ಇದ್ದಾರೆ. ಆದ್ದರಿಂದ ಪ್ರೇಮವಿಲ್ಲದೆ ಯಾರೂ ಇಲ್ಲ. ಮನುಷ್ಯನಿಗೆ ಯಾವುದಾದರೂ ಒಂದು ವಿಷಯದಲ್ಲಿ ಪ್ರೇಮ ಇದ್ದೇ ಇರುತ್ತದೆ. ಮಕ್ಕಳು-ಮರಿ ಇಲ್ಲದವನು ಒಂದು ಬೆಕ್ಕನ್ನೋ, ನಾಯಿಯನ್ನೋ ಸಾಕುತ್ತಾನೆ. ಹಿಂದಿನ ಕಾಲದಲ್ಲಿ ಹಸುವನ್ನು ಸಾಕುವುದು ಪುಣ್ಯವೆಂದುಕೊಂಡು ಸಾಕುತ್ತಿದ್ದರು. ಈಗಲೋ, ನಾಯಿ, ಬೆಕ್ಕು ಇವುಗಳಾದರೆ ಯಾವಾಗಲೂ ಪಕ್ಕದಲ್ಲಿಯೇ ಇಟ್ಟುಕೊಳ್ಳಬಹುದೆಂದುಕೊಂಡು ಸಾಕುತ್ತಾರೆ. ಏಕೆಂದರೆ ಮನುಷ್ಯನಿಗೆ ಯಾವುದಾದರೂ ಒಂದು ಚೈತನ್ಯದಲ್ಲಿ ವಿಶ್ವಾಸವಿರುತ್ತದೆ. ಆ ವಿಶ್ವಾಸವೇ ಪ್ರೇಮ. ಆ ಪ್ರೇಮ ಸ್ವಾಭಾವಿಕವಾದುದರಿಂದ ಋಷಿಗಳು `ಇವನು ಈ ರೀತಿ ಪ್ರೀತಿ ಇಟ್ಟುಕೊಂಡಿದ್ದಾನಲ್ಲಾ, ಇದು ಇವನ ಸ್ವಭಾವವಾಗಿದೆ. ಇವನಿಗೆ ಒಂದು ಒಳ್ಳೆಯ ದಾರಿ ತೋರಿಸಬೇಕು' ಎಂದುಕೊಂಡರು. ತಾಮ್ರಪರ್ಣಿ ನದಿಯಲ್ಲಿ ಎಷ್ಟೋ ನೀರು ಹರಿಯುತ್ತಿದ್ದರೂ ಎಲ್ಲಾ ಸಮುದ್ರಕ್ಕೆ ಸೇರುತ್ತಿತ್ತು. ಇಂಜಿನಿಯರುಗಳು ಏನು ಮಾಡಿದರು! ದಾರಿಯಲ್ಲೇ ಸೇತುವೆ ಕಟ್ಟಿದರೆ ಎಷ್ಟೋ ಉಪಯೋಗವಾಗುತ್ತದೆಂದು ಯೋಚಿಸಿದರು. ಅದೇ ರೀತಿ ನಮ್ಮ ಋಷಿಗಳು, `ನೀನು ಒಂದು ವಿಷಯದಲ್ಲಿ ತಾನೆ ಮುಖ್ಯವಾಗಿ ಪ್ರೇಮವನ್ನಿಟ್ಟುಕೊಳ್ಳುವೆ. ಅದನ್ನು ನೀನು ಭಗವಂತನ ವಿಷಯದಲ್ಲಿಟ್ಟುಕೋ. ಹಾಗೆ ಅದನ್ನು ಭಗವಂತನ ಕಡೆಗೆ ತಿರುಗಿಸಿಬಿಡು. ಹಾಗೆ ತಿರುಗಿಸಿಬಿಟ್ಟರೆ, ಭಗವಂತನ ವಿಷಯವಾಗಿ ಮನಸ್ಸು ಆಗಿ ಬಿಟ್ಟರೆ ಯಾವಾಗಲೂ ಮನಸ್ಸಿನಲ್ಲಿ ಸಚ್ಚಿಂತನೆ, ಸತ್ಕಥನ ಎನ್ನುವಂತೆ ಭಗವಂತನ ಪಾದಾರವಿಂದಗಳಲ್ಲಿ ಮನಸ್ಸು ತಾನಾಗಿಯೇ ಬಂದು ಬಿಡುತ್ತದೆ. ಅವನಲ್ಲಿ ಉತ್ತಮವಾದ ಪ್ರೇಮ ಉಂಟಾಗಿ ಎಂದೂ ಅದನ್ನು ಬಿಡಲಾಗುವುದಿಲ್ಲ. ಆದ್ದರಿಂದ ನಾರದರು, 

\begin{shloka}
`ಸಾತ್ವಸ್ಮಿನ್ ಪರಮಪ್ರೇಮರೂಪಾ'
\end{shloka}

-`ಭಕ್ತಿ ಎನ್ನುವುದು 
ಪರವಸ್ತುವಿನ ವಿಷಯದಲ್ಲಿ 
ಉತ್ತಮವಾದ ಪ್ರೇಮರೂಪಾ' ಎಂದು 
ಹೇಳಿದರು. ಈ ಭಕ್ತಿಗೆ 
``ಪರಮಪ್ರೇಮರೂಪಾ" ಎನ್ನುವ 
ಹೆಸರು ಏಕೆ ಬಂದಿದೆ 
ಎನ್ನುವುದನ್ನು ಯೋಚಿಸೋಣ. 
ಯಾವುದೋ ಸಂಕಟ ಬಂದ ಸಮಯದಲ್ಲಿ 
`ವೆಂಕಟರಮಣ' ಎಂದು ಹೇಳುತ್ತಾರೆ. 
ಸಂಕಟ ಹೋದ ಮೇಲೆ `ವೆಂಕಟರಮಣ' 
ಮರೆತು ಹೋಗುತ್ತದೆ. ಅಂಥಹ 
ಭಕ್ತಿಯುಳ್ಳ ಭಕ್ತನನ್ನು 
ಕುರಿತು ಪ್ರಹ್ಲಾದನು, `ನ ಸ 
ಭೃತ್ಯಃ ಸ ವೈ ವಣಿಕ್' 
ಎಂದಿದ್ದಾನೆ. ಅಂದರೆ, ಯಾರಾದರೂ 
ಭಗವಂತನ ಮೇಲೆ ಯಾವುದಾದರೂ 
ಕಷ್ಟ ದೂರವಾಗುವವರೆಗೋ, ಬೇರೆ 
ಯಾವುದಾದರೂ ಕಾರಣದಿಂದಲೋ 
ಪ್ರೀತಿ ಇಟ್ಟುಕೊಂಡಿದ್ದರೆ 
ಅವನು ಭಗವಂತನೊಡನೆ ವ್ಯಾಪಾರ 
ಮಾಡುವ ಒಬ್ಬ ವ್ಯಾಪಾರಿಯೇ 
ಆಗುತ್ತಾನೆ ಹೊರತು ಭಗವಂತನ 
ನಿಜವಾದ ದಾಸನಲ್ಲ. ವ್ಯಾಪಾರಿ 
ಏನು ಮಾಡುತ್ತಾನೆ? ಹಣ ಕೊಟ್ಟರೆ 
ಸಾಮಾನು ಕೊಡುತ್ತಾನೆ. ಅದೇ 
ರೀತಿ ನಾವೂ ಭಕ್ತಿ ಎನ್ನುವ 
ಹಣವನ್ನು ಕೊಟ್ಟು ಭಗವಂತನ 
ಹತ್ತಿರ ಸಾಮಾನು ಕೊಳ್ಳಲು 
ಹೋದರೆ ನಾವು 
ವ್ಯಾಪಾರಿಗಳೇ-ಹಾಗೇ ಆಗಬಾರದು. 
ಭಕ್ತಿ ಯಾವುದಕ್ಕೆ ಮಾಡುವುದು? 
ಭಗವಂತನಿಗಾಗಿಯೇ ಭಕ್ತಿ 
ಮಾಡಬೇಕು. ಆದ್ದರಿಂದಲೇ ನಾರದರು ಭಕ್ತಿಯನ್ನು `ಪರಮ ಪ್ರೇಮ ರೂಪಾ' ಎಂದಿದ್ದಾರೆ.

ಆತ್ಮನ ಮೇಲೆ ಎಲ್ಲರಿಗೂ 
ಪರಮಪ್ರೇಮವಿರುತ್ತದೆ. 
ಏಕೆಂದರೆ `ನಾನು ನನಗೆ 
ಪ್ರಿಯವಿಲ್ಲದೆ ಮತ್ತೊಬ್ಬನೇ 
ನನಗೆ ಪ್ರಿಯ? ನಾನು ಯಾವಾಗಲೂ 
ನನಗೆ ಪ್ರಿಯವೇ. ನಾನು ನನಗೆ ಅಪ್ರಿಯವಾಗಿರುವುದೇ ಇಲ್ಲ' ಎಂದೇ ಯಾರಾದರೂ ಹೇಳುತ್ತಾರೆ.

ಹಿರಣ್ಯಕಶಿಪುವಿನ ವಧೆಯಾಯಿತು. 
ಅದಾದ ಮೇಲೆ ನೃಸಿಂಹಸ್ವಾಮಿ 
ಪ್ರಹ್ಲಾದನನ್ನು ಕುರಿತು 
`ನಿನ್ನ ಮನಸ್ಸಿನಲ್ಲಿ 
ಇಷ್ಟವೇನೋ ಅದನ್ನು ವರವಾಗಿ 
ಕೇಳು' ಎಂದನು. ಆಗ ಬೇರೆ ಯಾವುದನ್ನು ವರವಾಗಿ ಕೇಳದೆ ಪ್ರಹ್ಲಾದನು,

\begin{shloka}
`ಯಾ ಪ್ರೀತಿರವಿವೇಕಾನಾಂ ವಿಷಯೇಷ್ವನಪಾಯಿನೀ|\\
ತ್ವಾಮನುಸ್ಮರತಃ ಸಾ ಮೇ ಹೃದಯಾನ್ಮಾಽಪಸರ್ಪತು||'
\end{shloka}

`ಯಾ ಪ್ರೀತಿರವಿವೇಕಾನಾಂ 
ವಿಷಯೇಷ್ವನಪಾಯಿನೀ'-ಲೌಕಿಕ 
ವಸ್ತುಗಳಲ್ಲಿ ಪ್ರೀತಿ 
ಇಟ್ಟುಕೊಂಡಿರುವವರಿಗೆ, 
ಅವುಗಳನ್ನು ಬಿಟ್ಟರೆ ಬೇರೆ 
ಯಾವುದರ ಬಗ್ಗೆಯೂ ಚಿಂತೆಯೇ 
ಇರುವುದಿಲ್ಲ. ಒಬ್ಬನಿಗೆ ಒಂದು 
ಮಗುವಿನ ಮೇಲೆ ಹೆಚ್ಚಾಗಿ 
ಪ್ರೀತಿ ಇರುತ್ತದೆ. 
ಹಾಗಿರುವಾಗ ಆ ಮಗು ಯಾವುದಾದರೂ 
ಕಾರಣದಿಂದ ಮರಣವನ್ನು ಪಡೆದರೆ 
ಬೇರೆ ಯಾವುದಾದರೂ ಒಳ್ಳೆಯ 
ಜನ್ಮವನ್ನು ಪಡೆಯಬಹುದು. ಆದರೆ 
ಜೀವಿಸುವ ಇವನು `ಮಗು 
ಹೋಗಿಬಿಟ್ಟಿತು, ನಾನು 
ಹೋಗಲಿಲ್ಲವಲ್ಲಾ' ಎಂದು 
ವ್ಯಥೆಪಡುತ್ತಿರುತ್ತಾನೆ. 
ವಸ್ತುಗಳ ಮೇಲೆ ಪ್ರೀತಿ 
ಇಟ್ಟುಕೊಂಡಿರುವವರು 
ಯಾವುದಾದರೂ ವಿಷಯದಲ್ಲಿ 
ದೃಢವಾಗಿ ಪ್ರೇಮ ಉಂಟಾದರೆ ಅವರು ಅದರಲ್ಲೇ ಹುಚ್ಚರಾಗಿರುವಂತೆ ಆಗಿ ಬಿಡುತ್ತಾರೆ.

`ತ್ವಾಮನುಸ್ಮರತಃ ಸಾ ಮೇ 
ಹೃದಯಾನ್ಮಾಽಪಸರ್ಪತು'. `ಪಾಮರ 
ಜನರಲ್ಲಿ ಇಹ ವಸ್ತುಗಳಲ್ಲಿ 
ಆಸೆ ಎನ್ನುವ ಪ್ರೇಮ ಇರುವಂತೆ?' 
ನಿನ್ನ ವಿಷಯವಾಗಿ ಪ್ರೇಮ ನನ್ನ 
ಹೃದಯದಿಂದ ಎಂದೂ ಹೋಗದಿರಲಿ' 
ಎಂದು ಪ್ರಹ್ಲಾದನು 
ಪ್ರಾರ್ಥಿಸಿದನು. ಅದೇ `ಪರಮ 
ಪ್ರೇಮರೂಪಾ' ಎಂದು ಹೇಳಲ್ಪಟ್ಟ 
ಭಕ್ತಿಯೆಂದು ಹೇಳುತ್ತೇವೆ. 
ಆದರೆ, ಸಾಮಾನ್ಯವಾಗಿ ನಾವು 
ಭಜನೆ ಮಾಡುವುದನ್ನೂ, ದೈವಿಕ 
ಕಥೆ ಓದುವುದನ್ನೂ ಭಕ್ತಿ 
ಎನ್ನುತ್ತೇವೆ. ಇವುಗಳೆಲ್ಲವೂ 
ಭಕ್ತಿಗೆ 
ಸಾಧನವಾಗಿರುವುದರಿಂದ 
ಇವುಗಳನ್ನು ಭಕ್ತಿಯೆಂದೇ 
ಹೇಳಬಹುದು. ಎಲ್ಲರೂ 
ಜೀವನ್ಮುಕ್ತರಲ್ಲ. ಆದರೂ, 
ಜೀವನ್ಮುಕ್ತನಾಗಲು ಯಾರಾದರೂ 
ಸಾಧನೆ ಮಾಡುತ್ತಿದ್ದರೆ 
ಅವನನ್ನು, `ಏನಪ್ಪಾ! ದೊಡ್ಡ 
ಜ್ಞಾನಿ' ಎನ್ನುತ್ತೇವೆ. 
ಜ್ಞಾನಮಾರ್ಗದಲ್ಲಿ 
ನಡೆಯುವುದರಿಂದ, ಜ್ಞಾನ ಸಾಧನ 
ಅನುಷ್ಠಾನದಿಂದ, ಅಂಥಹವನನ್ನು 
ಜ್ಞಾನಿ ಎನ್ನುತ್ತೇವೆ. ಆದ್ದರಿಂದ ಭಾಗವತದಲ್ಲಿಯೂ-

\begin{shloka}
`ಭಕ್ತ್ಯಾ ಸಂಜಾತಯಾ ಭಕ್ತ್ಯಾ ವಿಪ್ರತ್ಯುತ್ಪುಲಕಾಂತನುಮ್'
\end{shloka}

ಎಂದಿದೆ. ಸಾಮಾನ್ಯ ಭಕ್ತಿಯಿಂದ 
ಉಂಟಾಗುವ ಫಲವೂ ಇಲ್ಲಿ ಹೇಳಿದೆ. 
ಅದೇನೆಂದರೆ ಸಾಧನವಾಗಿರುವ 
ಭಕ್ತಿಯಿಂದ ಸಾಧ್ಯವಾಗಿರುವ 
ಭಕ್ತಿ ಉಂಟಾಗುತ್ತದೆ. 
ಸಾಧನವಾದ ಭಕ್ತಿ ಯಾವುದು? 
ಭಗವಂತನ ಲೀಲಾವಿಭೂತಿಗಳನ್ನು 
ಕುರಿತು ಭಾಗವತ, ಸ್ಕಾಂದ 
ಪುರಾಣಗಳಂಥಹ ಪುರಾಣಗಳನ್ನು 
ಹಲವು ಬಾರಿ ಓದಿದರೆ ಮನಸ್ಸು 
ಅಲ್ಲೇ ಮಗ್ನವಾಗಿ ಬಿಡುತ್ತದೆ. 
ಹಾಗೆ ಮನಸ್ಸು ಮಗ್ನವಾದ ಮೇಲೆ 
ಆನಂದವಾಗುತ್ತದೆ. ಆ 
ಪ್ರೇಮದಲ್ಲಿ ವಿಶೇಷವಾಗಿ 
ಇರುವುದಾದರೂ ಏನು? 
`ವಿಪ್ರತ್ಯುತ್ಪುಲಕಾಂ 
ತನುಮ್-ಆ ಭಕ್ತಿಸಾಧನದಿಂದ 
ಉಂಟಾಗುವ ಸಾಧ್ಯಭಕ್ತಿ ಇರುವಾಗ 
ಭಕ್ತನ ಶರೀರದಲ್ಲಿ ರೋಮಾಂಚನವಾಗುತ್ತದೆ. ಸಾಧನಾ ಭಕ್ತಿಯನ್ನು ಕುರಿತು,

\begin{shloka}
ಶ್ರವಣಂ ಕೀರ್ತನಂ ವಿಷ್ಣೋಃ ಸ್ಮರಣಂ ಪಾದಸೇವನಮ್|\\
ಅರ್ಚನಂ ವಂದನಂ ದಾಸ್ಯಂ ಸಖ್ಯಮಾತ್ಮ ನಿವೇದನಮ್||\\
ಇತಿ ಪುಂಸಾರ್ಪಿತಾ ವಿಷ್ಣೌ ಭಕ್ತಿಶ್ಚೇನ್ನವಲಕ್ಷಣಾ|\\
ಕ್ರಿಯತೇ ಭಗವತ್ಯಧ್ಧಾ ತನ್ಮನ್ಯೇಽಧೀತಮುತ್ತಮಮ್||
\end{shloka}

ಪ್ರಹ್ಲಾದನು ಕೊಟ್ಟ ವಿವರಣೆ 
ಇದು. ಪ್ರಹ್ಲಾದನನ್ನು 
ಹಿರಣ್ಯಕಶಿಪು `ಇಷ್ಟು ದಿನಗಳು 
ಗುರುಕುಲದಲ್ಲಿ ಯಾವ 
ವಿದ್ಯೆಯನ್ನು ಕಲಿತುಕೊಂಡು 
ಬಂದೆ' ಎಂದು ಕೇಳಿದಾಗ ಅವನು, `ಇದೇ ಉತ್ತಮವಾಗಿರುವ ವಿದ್ಯೆ' ಎಂದನು, ಅದು ಯಾವುದು?

\begin{shloka}
`ಶ್ರವಣಂ ಕೀರ್ತನಂ ವಿಷ್ಣೋಃ ಸ್ಮರಣಂ ಪಾದಸೇವನಮ್|\\
ಅರ್ಚನಂ ವಂದನಂ ದಾಸ್ಯಂ ಸಖ್ಯಮಾತ್ಮ ನಿವೇದನಮ್||
\end{shloka}

ಪರಮೇಶ್ವರನ ಬಗ್ಗೆ ಕೇಳುವುದು, 
ಅವನನ್ನು ಸ್ಮರಿಸುವುದು, 
ಸೇವಿಸುವುದು, 
ನಮಸ್ಕರಿಸುವುದು, ಅವನ 
ದಾಸನಾಗಿರುವುದು, ಅವನನ್ನು 
ಸ್ನೇಹಿತನನ್ನಾಗಿ 
ಮಾಡಿಕೊಳ್ಳುವುದು, ಅವನಿಗೆ 
ಆತ್ಮ ನಿವೇದನೆ 
ಮಾಡಿಕೊಳ್ಳುವುದು ಇದೆ 
ಉತ್ತಮವಾದ ವಿದ್ಯೆ. 
ಇವುಗಳಲ್ಲಿ 
ಕೊನೆಯಲ್ಲಿರುವುದು 
ಆತ್ಮನಿವೇದನೆ. ಅದು ಹೇಗೆಂದರೆ 
ಅಲ್ಲೇ ಹೋಗಿ ಸೇರಿಬಿಡುವುದು. 
ಹಾಗೆ ಸೇರಿದ ಮೇಲೆ ಭಗವಂತನು 
ಯಾರು, ತಾನು ಯಾರು ಎನ್ನುವುದೇ 
ಅವನಿಗೆ ತಿಳಿಯುವುದಿಲ್ಲ. 
ಆದ್ದರಿಂದಲೇ ಭಗವತ್ಪಾದರು, 
ಭಕ್ತಿಯಲ್ಲಿರುವ 
ಮೆಟ್ಟಲುಗಳನ್ನು 
ತೋರಿಸುವುದಕ್ಕಾಗಿ ಶಿವಾನಂದಲಹರಿಯಲ್ಲಿ ಹೀಗೆ ಒಂದು ಶ್ಲೋಕವನ್ನು ಹಾಡಿದರು.

\begin{shloka}
`ಅಂಕೋಲಂ ನಿಜಬೀಜಸಂತತಿರಯಸ್ಕಾಂತೋಪಲಂ ಸೂಚಿಕಾ\\
ಸಾಧ್ವೀ ನೈಜವಿಭುಂ ಲತಾ ಕ್ಷಿತಿರುಹಂ ಸಿಂಧುಃ ಸರಿದ್ವಲ್ಲಭಮ್|\\
ಪ್ರಾಪ್ನೋತೀಹ ಯಥಾ ತಥಾ ಪಶುಪತೇಃ ಪಾದಾರವಿಂದದ್ವಯಂ\\
ಚೇತೋವೃತ್ತಿರುಪೇತ್ಯ ತಿಷ್ಠತಿ ಸದಾ ಸಾ ಭಕ್ತಿರಿತ್ಯುಚ್ಯತೇ||'
\end{shloka}

ನಮ್ಮ ಮನೋಭಾವ ಹೀಗಿದ್ದರೆ 
ಅದಕ್ಕೆ ಭಕ್ತಿ ಎಂದು ಹೆಸರು 
ಎನ್ನುತ್ತಾರೆ. ಅದು 
ಹೇಗಿರಬೇಕೆಂದರೆ ಮೊಟ್ಟ ಮೊದಲು 
ಭಗವಂತನು ಬಂದು, `ನಾನೇ ಭಗವಂತ, 
ನನ್ನನ್ನು ನೀನು ಆರಾಧಿಸು' ಎಂದು ಹೇಳಿದರೂ ಕೂಡ ಯಾರೂ ಕೇಳುವುದಿಲ್ಲ. 

ಅರ್ಜುನನು ಕೃಷ್ಣಪರಮಾತ್ಮನೊಡನೆ ಇದ್ದನು. ಹಾಗಿದ್ದರೂ-

\begin{shloka}
`ಸಖೇತಿ ಮತ್ವಾ ಪ್ರಸಭಂ ಯುದುಕ್ತಂ\\
ಹೇ ಕೃಷ್ಣ ಹೇ ಯಾದವ ಹೇ ಸಖೇತಿ|'
\end{shloka}

ಎಂದು ಅವನು ಹೇಳುವ ರೀತಿಯನ್ನು 
ನೋಡಿದರೆ ಅವನು ಭಗವಂತನನ್ನು 
ಕೇವಲ 
ಸ್ನೇತಿತನೆಂದುಕೊಂಡುಬಿಟ್ಟಿದ್ದನು. ವಿಶ್ವರೂಪದರ್ಶನವಾಗುವವರೆಗೆ `ನಾನು ತಪ್ಪು ಮಾಡಿಬಿಟ್ಟೆ. ಭಗವಂತನೇ ನಮ್ಮೊಡನೆ ಹೀಗೆ ಬಂದು ಆಟವಾಡುತ್ತಿದ್ದಾನೆ' ಎಂದು ಅನಂತರ ಭಾವಿಸುತ್ತಾನೆ. ಆದ್ದರಿಂದ, ಈ ಕಾಲದಲ್ಲಿ, ಸಾಮಾನ್ಯ ಜನರೂ, ಭಗವಂತನೇ ಬಂದು ಹೇಳಿದರೂ ಕೂಡ `ಅವನು ಭಗವಂತ'ನೆಂದು ಭಾವಿಸುವುದಿಲ್ಲ. ಆದ್ದರಿಂದ, ಆಚಾರ್ಯರಾಗಿರುವವರು ಭಾಗವತದಂತಹ ಗ್ರಂಥಗಳನ್ನು ಹೇಳಿ ಜನರ ಮನಸ್ಸಿನಲ್ಲಿ ಒಂದು ಉತ್ತಮವಾದ ಮನೋಭಾವನೆ ಉಂಟುಮಾಡಬೇಕು. ಮೇಲೆ ಹೇಳಿದ ಭಗವತ್ಪಾದರ ಶ್ಲೋಕದಲ್ಲಿ-

`ಅಂಕೋಲಂ ನಿಜ ಬೀಜ ಸಂತತಿಃ' 
ಎಂದು ಮೊದಲು ಕಾಣುತ್ತದೆ. 
ಅಂಕೋಲವೆನ್ನುವುದು ಒಂದು 
ವೃಕ್ಷ. ಆ ವೃಕ್ಷದ 
ಸ್ವರೂಪವೇನೆಂದರೆ, ಬೀಜ ಬಂದರೆ 
ಆ ಮರಕ್ಕೆ ಹೋಗಿ ಹಾಗೆಯೇ 
ಹಿಡಿದುಕೊಂಡುಬಿಡುತ್ತದೆ. 
ಅದರಲ್ಲಿ ಸ್ವಲ್ಪ 
ಗೋಂದಿನಂತಿರುತ್ತದೆ. ಮರ ಒಂದು 
ಕೆಲಸವನ್ನಾದರೂ ಮಾಡುವುದಿಲ್ಲ. 
ಬೀಜ ಗಾಳಿಯ ಮೂಲಕ ತಾನಾಗಿಯೇ 
ಮರದ ಹತ್ತಿರ ಹೋಗುತ್ತದೆ. 
ಮತ್ತೆ ತಾನೇ ಮರವನ್ನು 
ಹಿಡಿದುಕೊಳ್ಳುತ್ತದೆ ಅಷ್ಟೇ. ಅದೇ ರೀತಿ ಮೊಟ್ಟಮೊದಲು ಭಕ್ತನ ಭಾಗ್ಯ ವಿಶೇಷದಿಂದ ಅವನು ತಾನಾಗಿಯೇ ಹೋಗಿ ಭಗವಂತನನ್ನು ಹಿಡಿದುಕೊಳ್ಳುತ್ತಾನೆ ಎನ್ನುತ್ತಾರೆ. 

ಅವನ ಮನೋವೃತ್ತಿ ಎನ್ನುವುದು ಭಗವದಾಕಾರವಾಗಿ ಹೋಗುತ್ತದೆ. ಅನಂತರ `ಆಯಸ್ಕಾಂತೋಪಲಂ ಸೂಚಿಕಾ' ಎಂದು ಹೇಳುತ್ತಾರೆ. ಭಗವಂತನಿಗೆ ಕರುಣೆ ಇದೆ. `ನನ್ನ ಬಳಿಗೆ ಬಂದು ಆಶ್ರಯಿಸಿದ್ದಾನೆ. ಇವನನ್ನು ನಾನು ಅನುಗ್ರಹಿಸಬೇಕು' ಎಂದು ಭಗವಂತನಿಗೆ ಕರುಣೆ ಉಂಟಾಗುತ್ತದೆ. ಕರುಣೆ ಉಂಟಾದ ಮೇಲೆ ಭಗವಂತನ ಹತ್ತಿರ ಬಂದು ಸೇರಿದವನು ಮತ್ತೆ ಬಿಟ್ಟು ಹೋಗಬೇಕೆಂದು ಕೊಂಡರೂ ಬಿಟ್ಟು ಹೋಗಲಾಗುವುದಿಲ್ಲ. ಭಗವಂತನ ಲೀಲಾ ವಿಭೂತಿಗಳ ಮಹಿಮೆ ಏನೆಂದರೆ ಮೊದಲ ದಿವಸ ಭಾಗವತ ಗೋಷ್ಠಿಗೆ ಹೋಗಿ ಸೇರಿದರೆ ಮಾರನೆಯ ದಿನವೂ ಅದೇ ಗೋಷ್ಠಿಗೆ ಹೋಗಿ ಸೇರಬೇಕೆಂದು ತೋರುತ್ತದೆ. ಸಿನಿಮಾ ಒಂದು ದಿನ ಹೋಗಿ ನೋಡಿ ಹುಚ್ಚು ಹಿಡಿಯಿತು ಎಂದರೆ ಮತ್ತೆ ನಾಲ್ಕು ದಿನಗಳು ಆಗಲಿಲ್ಲವೆಂದರೆ ಸಾಲ ತೆಗೆದುಕೊಂಡೋ ಅಥವಾ ಮನೆಯಿಂದ ಕಳ್ಳತನ ಮಾಡಿಯೋ ಸಿನಿಮಾ ನೋಡುತ್ತಾನೆ. ಇದೇ ರೀತಿ ಭಗವದ್ವಿಷಯದಲ್ಲೂ ಹುಚ್ಚು ಹಿಡಿದು ಬಿಡುತ್ತದೆ ಎನ್ನುತ್ತಾರೆ. ಹೇಗೆಂದರೆ `ಅಯಸ್ಕಾಂತೋಪಲಂ ಸೂಚಿಕಾ' ಅಯಸ್ಕಾಂತವೆನ್ನುವುದು ಕಬ್ಬಿಣವನ್ನು ತನ್ನೊಡನೆ ಹಿಡಿದುಕೊಳ್ಳುತ್ತದೆ. ಹಾಗೆ ಹಿಡಿದುಕೊಂಡ ಮೇಲೆ ಆ ಕಬ್ಬಿಣವನ್ನು ಎಳೆಯುವುದು ಕಷ್ಟ. ತಾನಾಗಿಯೇ ಅಯಸ್ಕಾಂತ ಬಿಟ್ಟುಬಿಡುವುದಿಲ್ಲ. ಇನ್ನೊಬ್ಬರು ಬಲಾತ್ಕಾರವಾಗಿ ಅದನ್ನು ಹಿಡಿದು ಎಳೆದರೆ ಮಾತ್ರ ಕಬ್ಬಿಣ ಬೇರೆಯಾಗಿ ಬರಬಲ್ಲದು. ಆದ್ದರಿಂದ ಭಕ್ತನು ಭಗವಂತನ ಹತ್ತಿರ ಬಂದ ಮೇಲೆ ಭಗವಂತನು ಭಕ್ತನನ್ನು ಬಲವಾಗಿ ಹಿಡಿದುಕೊಳ್ಳುವನು. 

ಮತ್ತೆ `ಸಾಧ್ವೀ ನೈಜವಿಭುಂ'-ಹೇಗೆಂದರೆ, ಪತಿವ್ರತೆಯಾದ ಹೆಂಗಸು ಎಷ್ಟು ವಿಶ್ವಾಸದಿಂದ ಪತಿಯನ್ನು ಆರಾಧಿಸಿಕೊಂಡು ಬರುವಳೋ ಅದೇ ರೀತಿ ಪತಿವ್ರತೆಯಾಗಿರುವ ಸ್ತ್ರೀಯನ್ನು ಪತಿ ಎಷ್ಟು ವಿಶ್ವಾಸದಿಂದಲೂ, ಪ್ರೇಮದಿಂದಲೂ, ಕಾಣುತ್ತಾ ಬರುತ್ತಾನೋ, ಹಾಗೆಯೇ ಭಗವಂತನು ಭಕ್ತನನ್ನು ತನ್ನವನೆಂದುಕೊಳ್ಳುತ್ತಾನೆ. ಇವನೂ ಭಗವಂತನು ನನ್ನವನೆಂದುಕೊಳ್ಳುತ್ತಾನೆ. ಹೀಗೆ ಅನ್ಯೋನ್ಯವಾದ ಪ್ರೇಮದಿಂದ ಬಂಧ ಉಂಟಾಗುವುದು ಮೂರನೆಯ ಮೆಟ್ಟಲು ಎನ್ನುತ್ತಾರೆ.

ಅನಂತರ `ಲತಾ ಕ್ಷಿತಿರುಹಂ'-ನಾಲ್ಕನೆಯ ಮೆಟ್ಟಲಲ್ಲಿ ಆಗುವುದು ಹೀಗೆ. ಭಗವಂತನು ಮರದಂತೆ ಆಶ್ರಯವಾಗಿರುವವನು. ಆ ಮರವನ್ನಾಶ್ರಯಿಸಿದ ಬಳ್ಳಿಯಂತೆ ಮೇಲಕ್ಕೆ ಹೋಗುತ್ತಿರುವವನು ಭಕ್ತ. ಒಂದು ಮರಕ್ಕೆ ಉತ್ತಮವಾದ ಮಲ್ಲಿಗೆ ಬಳ್ಳಿ ಹಬ್ಬಿಕೊಂಡು ಹೋದರೆ ಆ ಮರಕ್ಕೆ ಎಷ್ಟು ಶೋಭೆ ಉಂಟಾಗುತ್ತದೆ! ಅದೇ ರೀತಿ ಭಕ್ತನಿಂದ ಭಗವಂತನಿಗೆ ಶೋಭೆ ಉಂಟಾಗುತ್ತದೆ. ಅದು ಎಂಥ ಶೋಭೆ?

ಭಗವಂತನು ಭಕ್ತರನ್ನು ಅನುಗ್ರಹಿಸಲು ಅನೇಕ ರೂಪಗಳನ್ನು ಧರಿಸುತ್ತಾನೆ. 

\begin{shloka}
`ಆವಜಾನನ್ತಿ ಮಾಂ ಮೂಢ ಮಾನುಷೀಂ ತನುಮಾಶ್ರಿತಮ್|'\\
`ಪರಂ ಭಾವಮಜಾನನ್ತೋ ಮಮ ಭೂತಮಹೇಶ್ವರಮ್||'
\end{shloka}

ನಿರ್ಗುಣವಾದ ಪರವಸ್ತು ಹಲವು ವಿಧವಾಗಿ ದರ್ಶನವನ್ನು ಕೊಡುವುದರಿಂದ ಭಗವಂತನಿಗೇ ಹಿರಿಮೆಯುಂಟಾಗುತ್ತದೆ. ಭಕ್ತನಿಲ್ಲದಿದ್ದರೆ ಭಗವಂತನು ದರ್ಶನ ಕೊಡುವ ಅವಶ್ಯಕತೆ ಇಲ್ಲ. ದರ್ಶನ ಕೊಡುವ ಸಮಯದಲ್ಲಿ ಲೀಲೆಗಳನ್ನೆಲ್ಲಾ ತೋರಿಸಬೇಕಾಗುತ್ತದೆ. ಲೀಲೆಗಳನ್ನು ತೋರಿಸಲಿಲ್ಲವೆಂದರೆ ಸಾಮಾನ್ಯ ಜನರಿಗೆ ಭಗವಂತನ ಸ್ಮರಣೆ ಹೇಗಾಗಬೇಕು? ಆದ್ದರಿಂದ ಭಗವಂತನಿಗೆ ಭಕ್ತ ಒಂದು ಅಲಂಕಾರವಾಗಿದ್ದಾನೆಂದು ಹೇಳಿಬಿಟ್ಟರು. ಒಬ್ಬ ಪ್ರಹ್ಲಾದನು ಹುಟ್ಟದೆ ಇದ್ದಿದ್ದರೆ ಈ ನೃಸಿಂಹಾವತಾರ ಲೀಲೆ ನಡೆಯುತ್ತಿತ್ತೇ? ನಾವು ವಿಷ್ಣುವನ್ನು ವೈಕುಂಠವಾಸಿಯೆಂದು ಮಾತ್ರ ಹೇಳಿದ್ದೇವೆ ವಿನಾ ನೃಸಿಂಹರೂಪದಲ್ಲಿ ಪೂಜೆ ಮಾಡುತ್ತಿದ್ದೆವೋ? ಪ್ರಹ್ಲಾದನನ್ನು ಅನುಗ್ರಹಿಸುವುದಕ್ಕಾಗಿ ಭಗವಂತನು ಬಂದನು. ಆದ್ದರಿಂದ ನೃಸಿಂಹ ರೂಪದಲ್ಲಿ ಅವನನ್ನು ಆರಾಧಿಸುವೆವು. ಎಲ್ಲಾ ರೂಪಗಳೂ ಹೀಗೆಯೇ. ಕೃಷ್ಣಾವತಾರದಲ್ಲಿ ಪಂಚಪಾಂಡವರು ಹುಟ್ಟದೆ ಹೋಗಿದ್ದರೆ, ಅನುಗ್ರಹಿಸಲ್ಪಡಬೇಕಾದವರೇ ಇಲ್ಲದೆ ಹೋದರೆ ಕೃಷ್ಣಾವತಾರವಾಗಬೇಕಾದ ಆವಶ್ಯಕತೆ ಇಲ್ಲ. ಆದ್ದರಿಂದ ಭಗವಂತನು ಭಕ್ತನನ್ನು ಅನುಗ್ರಹಿಸಲು ಬರುವುದರಿಂದ, ಭಕ್ತನಿಂದ ಭಗವಂತನಿಗೆ ಒಂದು ಅಲಂಕಾರ ಉಂಟಾಗುತ್ತದೆ ಎನ್ನುತ್ತಾರೆ.

ಕೊನೆಯಲ್ಲಿ, `ಸಿಂಧುಃ ಸರಿದ್ವಲ್ಲಭಮ್'. ಮೊದಲು ಹೇಳಿದ ಭಕ್ತನ ಮನಸ್ಸೆನ್ನುವ ಹೊಳೆ ಭಗವಂತನೆನ್ನುವ ಸಮುದ್ರದ ಕಡೆಗೆ ಹರಿದುಕೊಂಡೇ ಹೋಗುತ್ತದೆ. ಹೊಳೆ ಎಷ್ಟು ದೂರ ಹರಿದು ಹೋಗುತ್ತದೆ? ಸಮುದ್ರ ಸಿಕ್ಕುತ್ತಲೇ ಅದೂ ಸಮುದ್ರವಾಗಿಯೇ ಆಗಬೇಕು. ಅದೇ ರೀತಿ ಭಕ್ತನ ಈ ಮನೋಭಾವನೆ ಬರ-ಬರುತ್ತಾ ಭಗವತ್ಸ್ವರೂಪವಾಗಿ ಬಿಡುತ್ತದೆಂದು ಹೇಳಿದರು. ಆಗ ಮನಸ್ಸೆನ್ನುವುದು ಇರುವುದಿಲ್ಲ. ಬೇರೆ ಆತ್ಮನೂ ಇಲ್ಲ.

\begin{shloka}
`ಮಮೈವಾಂಶೋ ಜೀವಲೋಕೇ ಜೀವಭೂತಃ ಸನಾತನಃ'
\end{shloka}

`ಜೀವ' ಎಂದರೆ ಬೇರೆ ಎಂದು ತಿಳಿಯಬೇಕಾಗಿಲ್ಲ. `ಅವನು ನನ್ನ ಒಂದು ಅಂಶ. ನನ್ನ ಒಂದುಭಾಗ' ಎನ್ನುತ್ತಾನೆ ಭಗವಂತನು. `ಭಾಗವಿದೆ' ಎಂದರೆ ಅದು ಹೋಗಿ ಸೇರುವವರೆಗೆ ಮಾತ್ರ ಭಾಗ. ಸೇರಿದ ಮೇಲೆ ಒಂದೇ. ಒಂದು ದೊಡ್ಡ ಕೊಳದಿಂದ ನೀರು ತೆಗೆದು ನಾವು ಪಾತ್ರೆಯಲ್ಲಿಟ್ಟರೆ ಅದು ಬೇರೆ, ಇದು ಬೇರೆ. ಮತ್ತೆ ನೀರು ಆ ಕೊಳಕ್ಕೆ ಬಿದ್ದು ಬಿಟ್ಟರೆ ಆ ರೀತಿಯಾದ ಬೇರೆ ತನ ಯಾವುದೂ ಇಲ್ಲ. ಬೇರೆ ಬೇರೆ ನೀರಾಗಿದ್ದುದು ಈಗ ಒಂದಾಗಿ ಸೇರಿಬಿಟ್ಟಿತು. ಹಾಗೆಯೇ ಭಗವದಾಕಾರವನ್ನು ಮನಸ್ಸು ಪಡೆದು ಬಿಟ್ಟಿತೆಂದರೆ ಆಗ ಜ್ಞಾನ ಉಂಟಾಗುತ್ತದೆ. ಅದನ್ನು ಗಮನಿಸಿಯೇ `ಸಖ್ಯಂ ಆತ್ಮ ನಿವೇದನಂ' ಎಂದು ಪ್ರಹ್ಲಾದನು ಹೇಳಿದನು. ಆತ್ಮನಿವೇದನೆ ಎನ್ನುವುದು ಬಂದು ಬಿಟ್ಟರೆ ಜ್ಞಾನ ಉಂಟಾಗುತ್ತದೆ. ಇಂಥಹ ಪ್ರೇಮವಾದ ಆ ಭಕ್ತಿ ಬರ ಬರುತ್ತಾ ಜ್ಞಾನಮಾರ್ಗವನ್ನು ಅವಲಂಬಿಸುತ್ತದೆಂದು ಹೇಳಿದ್ದಾರೆ. ಭಕ್ತಿ ಮಾರ್ಗಕ್ಕೆ ಜ್ಞಾನಮಾರ್ಗಕ್ಕಿಂತ ಹೆಚ್ಚಾಗಿರುವ ವಿಶೇಷವೆಂದರೆ ಸಾಧನಾಕಾಲದಲ್ಲಿ ಸುಖದ ಅನುಭವ. ಜ್ಞಾನಮಾರ್ಗದಲ್ಲಿ ಅದು ಅಷ್ಟಾಗಿಲ್ಲ. ಭಕ್ತನಿಗೆ ಸಾಧನಾಕಾಲದಲ್ಲಿಯೂ ಸುಖಾನುಭವವಿದೆ ಎನ್ನುತ್ತಾರೆ ಕೆಲವರು. ದೇವರನ್ನು ಉದ್ದೇಶಿಸಿ, `ನಾನು ಹುಲ್ಲಾಗಿದ್ದರೆ ನಿನಗೆ ಸೇವೆ ಮಾಡುವ ಹಸುವಿಗೆ ಒಂದು ಹಿಡಿ ಹುಲ್ಲಾಗಿ ಬಿಡುವೆನು. ಒಂದು ಕಲ್ಲಾಗಿದ್ದರೆ ಮೆಟ್ಟಲು ಮೇಲೆ ಹತ್ತಿ ಭಗವಂತನ ದರ್ಶನಕ್ಕೆ ಹೋಗುವವರಿಗಾಗಿ ಮಾಡಲ್ಪಡುವ ಮೆಟ್ಟಲಾದರೆ ಜನ್ಮ ಸಾರ್ಥಕ' ಎನ್ನುತ್ತಾರೆ.

ಒಬ್ಬರು, `ನಾನು ಇಷ್ಟು ದಿನಗಳೂ ಭಗವದ್ಭಕ್ತನಾಗಿದ್ದೆ. ಭಗವಂತನ ಆರಾಧನೆಗಾಗಿ ಎಲ್ಲವನ್ನೂ ವಿನಿಯೋಗಿಸಿ ಆಯಿತು. ಈಗ ಪ್ರಾಣ ಕೊಡುವ ಕಾಲ ಬಂದಿದೆ. ಈ ದೇಹ ಹೇಗೆ ಭಗವಂತನಿಗೆ ಉಪಯೋಗವಾಗುತ್ತದೆಂದು ಚಿಂತಿಸಿ,

\begin{shloka}
ಪಂಚತ್ವಂ ತನುರೇಷ ಭೂತನಿವಹಾಃ ಸ್ವಾಂಶೇ ವಿಶಂತು ಸ್ಫುಟಂ\\
ದಾತಾರಂ ಪ್ರಣಿಪತ್ಯ ಹಂತ ಶಿರಸಾ ತತ್ರಾಪಿ ಯಾಚೇ ವರಮ್|\\
ದತ್ವಾಪೀಷು ಪಯಃ ತದೀಯಮುಕುರೋ ಜ್ಯೋತಿಸ್ತದೀಯಾಂಕಣ\\
ವ್ಯೋಮ್ನಿವ್ಯೋಮ ತದೀಯ ವರ್ತ್ಮನಿ ಧರಾ ತತ್ತಾಲವೃಂತೇಽನಿಲಃ||
\end{shloka}

ಎಂದು ಪ್ರಾರ್ಥಿಸಿದರು. `ಈ ಶರೀರ ಪಂಚಭೂತಗಳಿಂದ ಏರ್ಪಡುವುದು. ಸತ್ತಮೇಲೆ ಪಂಚಭೂತಗಳು ಏನಾಗುವುವು? ಈ ಶರೀರದಲ್ಲಿರುವ ಪೃಥ್ವಿಯ ಅಂಶ ದೊಡ್ಡ ಆಕಾಶದಲ್ಲಿ ಹೋಗಿ ಸೇರುವುದು. ಶರೀರದಲ್ಲಿರುವ ತೇಜೋ ಅಂಶ ತೇಜಸ್ಸಿನಲ್ಲಿ ಹೋಗಿ ಸೇರಿಬಿಡುವುದು. ಗಾಳಿಯ ಅಂಶ ಗಾಳಿಯಲ್ಲಿ ಹೋಗಿ ಸೇರುವುದು. ಸತ್ತಮೇಲೆ ಈ ಪಂಚಭೂತಗಳೆಲ್ಲ ಎಲ್ಲೋ ಹೋಗಿ ಸೇರುವುದು ಬೇಡ. ಕೃಷ್ಣಪರಮಾತ್ಮನು ಯಾವ ಬಾವಿಯಿಂದ ನೀರು ತೆಗೆದುಕೊಂಡು ಕುಡಿಯುತ್ತಾನೋ ಆ ಬಾವಿಯಲ್ಲಿ ನೀರಿನ ಅಂಶ ಹೋಗಿ ಸೇರಲಿ. ಭಗವಂತನು ಮುಖವನ್ನು ನೋಡುವುದಕ್ಕಾಗಿ ಕನ್ನಡಿ ಹಿಡಿದು ನೋಡುತ್ತಾನೆ, ಈ ಕನ್ನಡಿಯಲ್ಲೇ ನನ್ನ ತೇಜೋಂಶ-ಪ್ರಕಾಶ ಅಂಶ ಸೇರಲಿ. ಭಗವಂತನು ತಿರುಗಾಡಲು ದೊಡ್ಡ ಮೈದಾನವಿದೆಯಂತೆ. ಆ ಮೈದಾನದಲ್ಲೆ ಆಕಾಶ ಅಂಶ ಹೋಗಿ ಸೇರಲಿ. ಭಗವಂತನು ವೃಂದಾವನದಲ್ಲಿ ಸಂಚಾರ ಮಾಡುವಾಗ ಅವನಿಗೆ ಭೂಮಿ ಅವಶ್ಯಕವಾಗುತ್ತದೆ. (ಈ ಕಾಲದಂತೆ ವಿಮಾನದಲ್ಲಿ ಹೋಗುವುದಿಲ್ಲ. ಭೂಮಿಯ ಮೇಲೆ ಹೋಗಬೇಕಾಗುತ್ತದೆ.) ಆ ಭೂಮಿಯಲ್ಲಿ ಪೃಥ್ವಿಯ ಅಂಶ ಹೋಗಿ ಸೇರಲಿ. ಬಿಸಿಲು ಕಾಲದಲ್ಲಿ ಭಗವಂತನು ಬೀಸಣಿಗೆಯಲ್ಲಿ ಬೀಸಿಕೊಳ್ಳುವಾಗ ಆ ಗಾಳಿಯಲ್ಲಿ ನನ್ನ ಗಾಳಿ ಅಂಶ ಹೋಗಿ ಸೇರಲಿ. ನನ್ನ ಮನಸ್ಸು ಆತ್ಮ ಎರಡೂ ಭಗವಂತನಲ್ಲಿ ಸೇರಿಬಿಟ್ಟಿವೆ. ಪಂಚಭೂತಗಳಲ್ಲಿ ಸೇರುವ ನನ್ನ ಅಂಶಗಳೂ ಹೀಗೆ ಸೇರಿದರೆ ಅವನ ಸೇವೆ ಯಾವಾಗಲೂ ನಡೆಯುತ್ತಿರುತ್ತದೆ' ಎನ್ನುವುದು ಪ್ರಾರ್ಥನೆಯ ಭಾವ. ಭಗವಂತನ ಹತ್ತಿರ ಸೇವೆ ಮಾಡುವವರು ಅದರ ರುಚಿ ಎಂಥಹದೆಂಬುದನ್ನು ಹೇಳುವರು. ಭಗವದ್ಭಕ್ತರ ಆನಂದಾನುಭವ ಎಲ್ಲೆಯಿಲ್ಲದ ಆನಂದಾನುಭವ. ಆದ್ದರಿಂದ ಎಲ್ಲರೂ ಆ ಭಕ್ತಿಮಾರ್ಗದಲ್ಲಿ ನಡೆಯಬೇಕು. ಮಾರ್ಗಗಳು ಮೂರು. ಯಾರಿಗೆ ವಿಶೇಷವಾದ ವೈರಾಗ್ಯ, ವಿವೇಕ ಮುಂತಾದವು ಇರುವುದಿಲ್ಲವೋ ಅಂಥಹವರು ಭಕ್ತಿಮಾರ್ಗದಲ್ಲಿ ಹೋಗಬಹುದೆಂದು ಹೇಳಿರುವೆವು. ಕರ್ಮ ಮಾರ್ಗದಲ್ಲಿ ಹೋಗುವುದಕ್ಕೆ ವಿಶೇಷವಾದ ಸಾಮರ್ಥ್ಯವಿರಬೇಕು. ಭಕ್ತಿಮಾರ್ಗದಲ್ಲಿ ಹೋಗುವುದಕ್ಕೆ ಯಾರಾದರೂ ಹಣವಂತನಾಗಿಯೋ, ವಿದ್ಯಾವಂತನಾಗಿಯೋ, ಬುದ್ಧಿಶಾಲಿಯಾಗಿಯೋ ಇರಬೇಕೆ ಎನ್ನುವುದಕ್ಕೆ-

\begin{shloka}
`ವ್ಯಾಧಸ್ಯಾಚರಣಂ ಧ್ರುವಸ್ಯ ಚ ವಯಃ ವಿದ್ಯಾಗಜೇಂದ್ರಸ್ಯ ಕಾ\\
ಕಾ ಜಾತಿರ್ವಿದುರಸ್ಯ ಯಾದವಪತೇರುಗ್ರಸ್ಯ ಕಿಂ ಪೌರುಷಮ್|\\
ಕುಬ್ಜಾಯಾಃ ಕಮನೀಯರೂಪಮಧಿಕಂ ಕಿಂ ವಾ ಸುದಾಮ್ನೋ ಧನಂ\\
ಭಕ್ತ್ಯಾ ತುಷ್ಯತಿ ಕೇವಲಂ ನ ತು ಗುಣೈಃ ಭಕ್ತಿಪ್ರಿಯೋ ಮಾಧವಃ||'
\end{shloka}

ಎಂದು ಉತ್ತರವಾಗುತ್ತದೆ. ಭಗವಂತನು ಎಲ್ಲ ರೀತಿಯ ಭಕ್ತರನ್ನು ಅನುಗ್ರಹಿಸುವುದುಂಟು. ಒಂದು ಕಥೆಯನ್ನು ನೋಡಿ ಅನಂತರ ಆ ಶ್ಲೋಕವನ್ನು ನೋಡೋಣ.

ಒಂದು ಕಾಲದಲ್ಲಿ ಒಂದು ಜಾಗದಲ್ಲಿ ನದಿ ಪ್ರವಾಹ ಉಂಟಾಯಿತು. ಆಗ ಆ ದೇಶದ ರಾಜನು, `ಜನರೆಲ್ಲ ನೀರಿನಲ್ಲಿ ಒಂದೊಂದು ಜಾಗವಾಗಿ ಅಲ್ಲಲ್ಲಿ ಮಣ್ಣು ತುಂಬಿಬಿಡಿ' ಎಂದು ಹೇಳಿದನು. ಹೀಗೆ ಅವನು ಊರಿನಲ್ಲಿರುವವರಿಗೆಲ್ಲ ಕೆಲಸ ಕೊಟ್ಟಾಯಿತು. ಅಲ್ಲಿ ಒಬ್ಬ ಮುದುಕಿ ಇದ್ದಳು. ಅವಳ ಕೆಲಸ ದೋಸೆ ಮಾಡಿ ಅದನ್ನು ಮಾರಿ ಹಣ ಸಂಪಾದಿಸುವುದು. ಭಗವಂತನು ಆಗ ಒಂದು ಲೀಲೆ ಮಾಡಬೇಕೆಂದುಕೊಂಡನು. ಅವಳಿಗೆ ಗುದ್ದಲಿಯಂಥಹುದನ್ನು ತೆಗೆದುಕೊಂಡು ಮಣ್ಣನ್ನು ಗೊಡೆಯಲ್ಲಿ ಹಾಕುವುದಕ್ಕೂ ಆಗುತ್ತಿರಲಿಲ್ಲ. ದೋಸೆ ಮಾಡಿ ಮಾರುವುದು ಮಾತ್ರ ಅವಳಿಗೆ ಸಾಧ್ಯ. ಭಗವಂತನು ಅವಳ ಬಳಿಗೆ ಬಂದು, `ಏನಮ್ಮ, ನೀನು ಏನೂ ಕೆಲಸ ಮಾಡಲಿಲ್ಲವೇ?' ಎಂದು ಕೇಳಿದನು. ಅದಕ್ಕೆ ಅವಳು, `ನಾನೇನು ಮಾಡುವುದು? ಈ ದೋಸೆಯೆಲ್ಲಾ ಇಂದು ಮಾರದೇ ಇದ್ದರೆ ನಾನು ಜೀವಿಸಲಾಗದ ಕಷ್ಟಪಡಬೇಕು. ಆದರೆ ನಾನು ಹೋಗದೆಯೂ ಇರಲಾಗುವುದಿಲ್ಲ. ಅದು ರಾಜಾಜ್ಞೆ' ಎಂದಳು. ಭಗವಂತನು `ಆ ಕೆಲಸವೆಲ್ಲ ನಾನು ಮಾಡುತ್ತೇನೆ. ನಿನ್ನ ದೋಸೆಗಳೆಲ್ಲ ಒಣಗಿರುವುದನ್ನು ಯಾರೂ ತೆಗೆದುಕೊಂಡು ಹೋಗುವುದಿಲ್ಲ. ಅದನ್ನು ನೀನು ನನಗೆ ಕೊಟ್ಟರೆ ನನ್ನ ಕೂಲಿಗೆ ಆಗುತ್ತೆ' ಎಂದನು. ಹೀಗೆ ಚಮತ್ಕಾರವಾಗಿ ಮಾತನಾಡುತ್ತಾ ಅವನು ಚೆನ್ನಾಗಿರುವ ದೋಸೆಗಳನ್ನು ತಿಂದು ಬಿಟ್ಟನು. ಮಧ್ಯಾಹ್ನವಾಯಿತು. ದೋಸೆಗಳೆಲ್ಲವನ್ನೂ ತಿನ್ನುವುದು ಆಯಿತು. ತಲೆಯ ಮೇಲೆ ಗೊಡೆ, ಕೈಯಲ್ಲಿ ಗುದ್ದಲಿ ಹಿಡಿದುಕೊಂಡಿದ್ದ ಅವನು ಬರುವವರೊಡನೆ ಮಾತನಾಡುತ್ತಿದ್ದನು. ಕೆಲಸವೇನೂ ಮಾಡಲಿಲ್ಲ. ಸ್ವಲ್ಪ ಹೊತ್ತಾದ ಮೇಲೆ ನೋಡಿದರೆ ಒಂದೇ ಒಂದು ಕಡೆಯಿಂದ ಮಾತ್ರ ನೀರು ಬರುತ್ತಿರುವುದನ್ನು ಜನರು ಕಂಡರು. ರಾಜನಿಗೆ ಸಮಾಚಾರ ತಲುಪಿತು. ಎಲ್ಲರೂ ರಾಜನ ಬಳಿ, ಇವನು, ನಾವು ಯಾವುದಾದರೂ ಹೇಳಿದರೆ ಚಮತ್ಕಾರವಾಗಿ ಮಾತನಾಡಿ ಬಿಡುತ್ತಾನೆ. ಇವನ ಮುಖವನ್ನು ನೋಡಿದರೆ ನಮಗೆಲ್ಲ ಮೋಹ ಉಂಟಾಗುತ್ತದೆ. ಇವನನ್ನು ಹೊಡೆಯಬೇಕೆನಿಸುತ್ತದೆ. ಆದರೆ ಕೈ ಬರುವುದಿಲ್ಲ. ತಾವೇ ಶಿಕ್ಷಿಸಬೇಕು? ಎಂದರು. ಇವನನ್ನು ನೋಡಿದ ರಾಜನು ಒಂದು ಏಟು ಹೊಡೆದನು. ಆದರೆ ಆ ಏಟು ಹೊಡೆದೊಡನೆ ಎಲ್ಲರಿಗೂ ಏಟು ಬಿದ್ದಿತು. ಎಲ್ಲರೂ ರಾಜನು ತಮ್ಮನ್ನೇ ಹೊಡೆದುಬಿಟ್ಟನೇನೋ ಎಂದು ಮುಟ್ಟಿ ಮುಟ್ಟಿ ನೋಡಿಕೊಂಡರು. ನೋಡಿದರೆ ರಾಜನಿಗೂ ಏಟು ಬಿದ್ದಿತು. ಇತರರ ಬಳಿ ರಾಜನು ಇಲ್ಲದಿದ್ದರೂ ಅವನು ಭಗವಂತನಿಗೆ ಹೊಡೆದ ಏಟು ಮಾತ್ರ ಎಲ್ಲರಿಗೂ ಬಿದ್ದು ಬಿಟ್ಟಿತು.

ಇದನ್ನು ಕುರಿತು ಚಮತ್ಕಾರವಾಗಿ, `ಭಗವಂತನಾದ ನೀನು ಎಲ್ಲರ ಆತ್ಮನಾಗಿದ್ದೀಯೆ, ನೀನು ಸರ್ವಾತ್ಮಕನೆಂದು ತೋರಿಸುವುದೂ ಬಹಳ ಸ್ವಾರಸ್ಯಕರವಾಗಿ ತೋರಿಸಿದೆ. ದೋಸೆ ತಿನ್ನುವ ವೇಳೆಯಲ್ಲಿ ಸರ್ವಾತ್ಮಕತ್ವವನ್ನು ತೋರಿಸಬೇಕೆನಿಸಲಿಲ್ಲವೇ! ನೀನು ದೋಸೆ ತಿಂದು ಹಸಿವು ತೀರಿದೊಡನೆ ಎಲ್ಲರಿಗೂ ಹಸಿವು ತೀರಿದಂತೆ ತೋರಿಸಲಿಲ್ಲವಲ್ಲಾ! ಈಗ ಏಟು ಕೊಟ್ಟ ಸಮಯದಲ್ಲಿ ಮಾತ್ರ ಎಲ್ಲರಿಗೂ ಸರ್ವಾತ್ಮಕತ್ವವನ್ನು ತೋರಿಸುವೆಯಾ?' ಎನ್ನುತ್ತಾರೆ. `ಏಕೆಂದರೆ ಏಟು ಬಿದ್ದರೇನೇ ಭಗವಂತನ ನೆನಪು ಬರುತ್ತದೆ. ಹಸಿವು ಉಂಟಾಗಲಿಲ್ಲವೆಂದರೆ ಯಾರೂ ಹಸಿವಾಗಲಿಲ್ಲವೆಂದು ಚಿಂತಿಸುವುದಿಲ್ಲ. ಆದರೆ ಏಟು ಬಿದ್ದರೆ, `ಅಯ್ಯೋ! ಏನು ಪ್ರಾರಬ್ಧ, ಏಟು ಬಿದ್ದು ಬಿಟ್ಟಿತು.!' ಎಂದು ನಿನ್ನನ್ನು ಎಲ್ಲರೂ ಸ್ಮರಿಸುವುದಕ್ಕಾಗಿ ನಿನ್ನ ಸರ್ವಾತ್ಮಕತ್ವವನ್ನು ಈ ರೀತಿ ನೀನು ತೋರಿಸಿದೆ!' ಎಂದು ಪರಮ ಭಕ್ತರಾದ ನೀಲಕಂಠದೀಕ್ಷಿತರು ಹೇಳಿದರು. ಆದ್ದರಿಂದ ಭಗವಂತನು ಸರ್ವತ್ರ ಎಲ್ಲರನ್ನೂ ನೋಡುತ್ತಿದ್ದಾನೆ. ಆ ಮುದುಕಿಗೆ ಅವನು ಒಬ್ಬ ಕೂಲಿಯವನಾಗಿ ದರ್ಶನವನ್ನು ಕೊಟ್ಟನು. ಇವೆಲ್ಲಾ ಪುರಾಣಗಳಲ್ಲಿ ಇರುವವೇ ಆಗಿವೆ.

ನಾನು ಮೊದಲು ಹೇಳಿದ ಶ್ಲೋಕದಲ್ಲಿ ಮೊದಲು `ವ್ಯಾಧಸ್ಯಾಚರಣಂ' ಎಂದಿದೆ. ಧರ್ಮವ್ಯಾಧನೆಂಬುವನು ಒಬ್ಬನಿದ್ದನು. ಅವನು ಬೇಡರ ಕುಲದಲ್ಲಿ ಹುಟ್ಟಿದವನಾದರೂ ಅವನಿಗೆ ಭಗವತ್ಸಾಕ್ಷಾತ್ಕಾರವಾಯಿತು. ಆದ್ದರಿಂದಲೇ ಅವನಿಗೆ ಧರ್ಮವ್ಯಾಧನೆಂದು ಹೆಸರೇ ಬಂದಿತು. ಧರ್ಮವ್ಯಾಧನಿಗೆ ಆಚಾರ ಒಂದೂ ವಿಶೇಷವಾಗಿ ಇರಲಿಲ್ಲ. ಆದರೂ ಅವನಿಗೆ ಭಗವಂತನ ದಯೆ ಲಭಿಸಿತು. ಇದರಿಂದ ವಿಶೇಷವಾದ ಆಚಾರವಿದ್ದರೇ ಮಾತ್ರ ಭಗವಂತನಲ್ಲಿ ಭಕ್ತಿ ಇರಬೇಕೆಂದಿಲ್ಲವೆಂದು ತಿಳಿಯುವುದು.

ಅನಂತರ `ಧ್ರುವಸ್ಯ ಚ ವಯಃ'-ಧ್ರುವನು ಒಬ್ಬ ಚಿಕ್ಕ ಹುಡುಗ. ಅವನಿಗೆ ಎರಡು ವರ್ಷಗಳು ಕೂಡ ಆಗಿರಲಿಲ್ಲ. ಅಂಥಹವನು ಯಾವುದೋ ಒಂದು ಸಿಂಹಾಸನ ಬೇಕೆಂದು ತಪಸ್ಸು ಮಾಡಿದನು. ಅವನಿಗೆ ಭಗವಂತನು ಮೇಲಾದ ಶಾಶ್ವತವಾದ ಉತ್ತಮ ಸ್ಥಾನವನ್ನು ಕೊಟ್ಟನು.

`ವಿದ್ಯಾ ಗಜೇಂದ್ರಸ್ಯಕಾ'-ಗಜೇಂದ್ರನಿಗೆ ಭಗವಂತನು ವೈಕುಂಠ ಸಾಮ್ರಾಜ್ಯವನ್ನು ಕೊಟ್ಟನು. ಬುದ್ಧಿಯಲ್ಲಿ ಅವನು ಕೆಳಮಟ್ಟದ ಬುದ್ಧಿವುಳ್ಳ ಮೊಸಳೆಯೊಡನೆ ತನ್ನ ಬಲವನ್ನೆಲ್ಲಾ ಪ್ರಯೋಗಿಸಿದರೂ ಉಪಯೋಗವಾಗಲಿಲ್ಲ. ಒಡನೆ ಒಂದು ಕಮಲವನ್ನು ತೆಗೆದುಕೊಂಡು ಭಗವಂತನಿಗೆ ಸಮರ್ಪಿಸಿ, `ನೀನೇ ನನ್ನನ್ನು ಕಾಪಾಡಬೇಕು' ಎಂದನು. ಒಡನೆ ಭಗವಂತನು ವೈಕುಂಠವೆಲ್ಲಿದೆ, ಲಕ್ಷ್ಮಿ ಎಲ್ಲಿದ್ದಾಳೆ ಎನ್ನುವುದನ್ನು ಒಂದನ್ನೂ ಗಮನಿಸದೆ ಒಡನೆ ಬಂದು ಕಾಪಾಡಿದನು.

`ಕಾ ಜಾತಿರ್ವಿದುರಸ್ಯ'-ವಿದುರನು ಒಮ್ಮೆ ಭಗವಂತನನ್ನು ತನ್ನ ಮನೆಗೆ ಸ್ವಾಗತಿಸಿದನು. ಭಗವಂತನು ದುರ್ಯೋಧನನ ಮನೆಗೆ ಹೋಗುವ ದಾರಿಯಲ್ಲಿ ವಿದುರನ ಮನೆಗೆ ಹೋಗಿ ಅಲ್ಲಿ ಉಪಚಾರವನ್ನು ಸ್ವೀಕರಿಸಿದನು. ಇತರರೆಲ್ಲರನ್ನೂ ಬಿಟ್ಟು ಭಗವಂತನು ವಿದುರನ ಮನೆಗೆ ಹೋದ ಕಾರಣವೇನೆಂದರೆ ವಿದುರನ ವಿಷಯವಾಗಿ ಭಗವಂತನಿಗೆ ಪರಮ ಪ್ರೇಮವಿದ್ದಿತು.

`ಯಾದವಪತೇರುಗ್ರಸ್ಯ ಕಿಂ ಪೌರುಷಮ್'-ಉಗ್ರಸೇನನು ಯಾದವರ ಅರಸನಾಗಿದ್ದನು. ಆದರೆ ಅವನ ಮಗನಾದ ಕಂಸನು ಅವನನ್ನು ಬಂಧಿಸಿದನು. ಹೀಗಿದ್ದಾಗ ಆ ಉಗ್ರಸೇನನಿಗೆ ಎಷ್ಟು ಶೂರತ್ವವಿರಬೇಕೆಂದು ನಾವೇ ಪರಿಗಣಿಸಬಹುದು. ಹಾಗಿದ್ದರೂ ಭಗವಂತನು ಅವನನ್ನು ಅನುಗ್ರಹಿಸಿದನು.

`ಕುಬ್ಜಾಯಾಃ ಕಮನೀಯ ರೂಪಮಧಿಕಮ್'-ಕುಬ್ಜೆ ಎನ್ನುವವಳಿಗೆ ಸೌಂದರ್ಯವೇ ಇರಲಿಲ್ಲ. ಆದರೂ ಅವಳ ಸೇವೆಯನ್ನು ಕಂಡು ತೃಪ್ತಿಪಟ್ಟ ಕೃಷ್ಣಪರಮಾತ್ಮನು ಅವಳನ್ನು ಅನುಗ್ರಹಿಸಿ ಅವಳ ಶರೀರ ದೋಷವನ್ನು ದೂರ ಮಾಡಿದನು.

`ಕಿಂ ವಾ ಸುದಾಮ್ನೋ ಧನಂ'-ಗುರುಕುಲದಲ್ಲಿ ಕೃಷ್ಣನ ಸ್ನೇಹಿತನಾಗಿದ್ದವನು ಸುದಾಮ. ತರುವಾಯ ಪರಮ ದರಿದ್ರನಾದ ಅವನು ತನ್ನ ಹೆಂಡತಿಯ ಆಸೆಗಳಿಗೆ ಒಳಪಟ್ಟು ಕೃಷ್ಣನ ಹತ್ತಿರ ಹಣ ಪಡೆಯಲು ಹೋದನು. ಕೃಷ್ಣನು ಪ್ರೇಮದಿಂದ ಅವನು ಕೊಟ್ಟ ಅವಲಕ್ಕಿಯನ್ನು ಸ್ವೀಕರಿಸಿ ಅಮಿತ ಧನವನ್ನು ಆಶೀರ್ವದಿಸಿದನು. ಇದರಿಂದ ಹಣವಂತನೇ ಭಕ್ತ ಮಾರ್ಗದಲ್ಲಿ ಹೋಗಬೇಕೆಂದಿಲ್ಲವೆಂದು ತಿಳಿಯುತ್ತದೆ. ಹಾಗಾದರೆ ಭಗವಂತನಿಗೆ ಮುಖ್ಯವಾಗಿ ಬೇಕಾದುದು ಏನು?

`ಭಕ್ತ್ಯಾತುಷ್ಯತಿ ಕೇವಲಂ ನ ತು ಗುಣೈಃಭಕ್ತಿಪ್ರಿಯೋ ಮಾಧವಃ'-ಭಕ್ತನ ಅಂತಃಕರಣ ಶುದ್ಧವಾಗಿದೆಯೇ ಇಲ್ಲವೇ ಎನ್ನುವುದನ್ನೇ ಅವನು ನೋಡುತ್ತಾನೆ. ಭಕ್ತನ ಅಂತಃಕರಣ ಶುದ್ಧವಾಗಿದ್ದರೆ ಅವನ ಜಾತಿಯಾಗಲಿ, ಕುಲವಾಗಲಿ, ಐಶ್ವರ್ಯವಾಗಲಿ ಯಾವುದನ್ನೂ ನೋಡುವುದಿಲ್ಲ. ಆದ್ದರಿಂದ ಎಲ್ಲರೂ ಸುಲಭವಾಗಿ ಆರಾಧಿಸಿ ಮನುಷ್ಯ ಜನ್ಮವನ್ನು ಸಾರ್ಥಕ ಮಾಡಿಕೊಳ್ಳಲು ಮುಖ್ಯವಾಗಿರುವಂಥಹದು ಭಕ್ತಿ ಮಾರ್ಗವೇ ಎಂದು ಹಿಂದಿನವರು ಹೇಳಿದ್ದಾರೆ. ಆದರೆ ಈ ಕಾಲದಲ್ಲಿ ಕೆಲವರಿಗೆ ಒಂದು ದೃಢವಾದ ಮನೋಭಾವನೆ ಇದೆ. ಅದು ಏನೆಂದರೆ-

\begin{shloka}
`ಅಪಿ ಚೇತ್ ಸುದುರಾಚಾರಃ ಭಜತೇ ಮಾಮನನ್ಯಭಾಕ್|\\
ಸಾಧುರೇವ ಸ ಮಂತವ್ಯಃ ಸಮ್ಯಕ್ ವ್ಯವಸಿತೋ ಹಿ ಸಃ||'
\end{shloka}

ಎಂದಲ್ಲವೇ ಭಗವಂತನು ಹೇಳಿದ್ದಾನೆ.

ಕೆಟ್ಟ ಆಚಾರವಂತನಾದರೂ ನನ್ನನ್ನು ತೀವ್ರವಾಗಿ ಭಕ್ತಿಯಿಂದ ಆರಾಧಿಸಿದರೆ ಅವನನ್ನು ಸಾಧುವೆಂದು ತಿಳಿಯಬೇಕು. ಏಕೆಂದರೆ ನನ್ನನ್ನು ಆರಾಧಿಸಬೇಕೆಂಬ ಒಳ್ಳೆಯ ನಿರ್ಧಾರಕ್ಕೆ ಬಂದಿದ್ದಾನೆ'-ಇದು ಶ್ಲೋಕದ ತಾತ್ಪರ್ಯ. ಇದನ್ನು ಇಟ್ಟುಕೊಂಡು `ದುರಾಚಾರ'ವಂತನಾದರೂ ಕೂಡ ಭಗವದ್ಭಕ್ತನೆಂದು ಬೋರ್ಡು ಹಾಕಿಕೊಂಡು ಬಿಟ್ಟರೆ ಅವನು ಯಾವಾಗಲೂ ಯೋಗ್ಯನೆಂದೇ ಕೆಲವರು ನಿರ್ಧಾರಕ್ಕೆ ಬಂದು ಬಿಟ್ಟಿದ್ದಾರೆ. ಇದು ಸರಿಯಾದ ತೀರ್ಮಾನವಲ್ಲ.

ಈ ವಿಷಯದ ಬಗ್ಗೆ ಸುರೇಶ್ವರರು-

\begin{shloka}
`ಬುದ್ಧಾದ್ವೈತ ಸತ್ತತ್ತ್ವಸ್ಯ ಯಥೇಷ್ಟಾಚರಣಂ ಯದಿ|\\
ಶುನಾಂ ತತ್ತ್ವದೃಶಾಂ ಚೈವ ಕೋ ಭೇದೋಽಶುಚಿಭಕ್ಷಣೇ||'
\end{shloka}

ಎಂದು ಕೇಳಿದ್ದಾರೆ.

ಅದ್ವೈತ ತತ್ತ್ವವನ್ನು ತಿಳಿದವನು ಹೇಗೆ ಬೇಕಾದರೂ ಜೀವನವನ್ನು ನಡೆಸಿದರೆ ಅವನಿಗೂ ನಾಯಿಗೂ ಅಶುದ್ಧವಾದುದನ್ನು ತೆಗೆದುಕೊಳ್ಳುವುದರಲ್ಲಿ ವ್ಯತ್ಯಾಸವೇನೆಂದು ಕೇಳುತ್ತಾರೆ. ನಾಯಿಗೆ ಲೋಕ ಜ್ಞಾನವಿರುವುದಿಲ್ಲ. ಈ ಜ್ಞಾನಿಗೆ ಲೋಕ ಬಿಟ್ಟು ಹೋಗಿದೆ. ಆದರೆ ಜ್ಞಾನ ಉಂಟಾಗುವುದಕ್ಕೆ ಮುಂಚೆ ಬಹಳ ಕಾಲ ಉತ್ತಮವಾದ ದಾರಿಯಲ್ಲಿ ಜ್ಞಾನಿ ಹೋಗುತ್ತಿದ್ದುದರಿಂದ ಯಾವಾಗಲೂ ಅವನು ಶಾಸ್ತ್ರಮಾರ್ಗವನ್ನು ಬಿಟ್ಟು ಹೋಗುವುದಿಲ್ಲ; ಹಾಗೆ ಜೀವನವನ್ನು ನಡೆಸುವುದರಿಂದ ಯಾವ ವಿಧವಾದ ಪ್ರಯೋಜನವೂ ಅವನಿಗೆ ಇಲ್ಲದಿದ್ದರೂ ಲೋಕದ ಕಲ್ಯಾಣಕ್ಕಾಗಿ ಹಾಗೆ ನಡೆದುಕೊಂಡು ಬರುತ್ತಾನೆ. ಈ ವಿಷಯವನ್ನು ಭಗವಂತನೂ ಗೀತೆಯಲ್ಲಿ ಸ್ಪಷ್ಟಪಡಿಸಿದ್ದಾನೆ. ಕೆಟ್ಟ ಆಚಾರವುಳ್ಳ ಭಕ್ತನನ್ನು ಸಾಧುವೆಂದು ತಿಳಿಯಬೇಕೆನ್ನುವ ಭಗವಂತನ ಅಭಿಪ್ರಾಯವೇನೆಂಬುದನ್ನು ಗಮನಿಸಬೇಕು. ಅವನು `ಅಪಿ ಚೇತ್ ಸುದುರಾಚಾರಃ' ಎಂದು ಮಾತ್ರ ಹೇಳಿದ್ದಾನೆ. ಇದರಿಂದ `ಕೆಟ್ಟ ಆಚಾರ ಒಂದು ವೇಳೆ ಇದ್ದರೂ' ಎಂದು ಮಾತ್ರ ತಿಳಿಯಬೇಕು. ಇದನ್ನು ಬಿಟ್ಟು ಭಗವಂತನು ದುರಾಚಾರಕ್ಕೆ ಅನುಮತಿ ನೀಡಿದ್ದಾನೆಂದು ತಿಳಿದರೆ ಒಂದು ವಿಧವಾದ ಒಳ್ಳೆಯ ಫಲವನ್ನು ಪಡೆಯಲಾಗುವುದಿಲ್ಲ ಏಕೆಂದರೆ ಭಗವಂತನ ನಾಮಸ್ಮರಣೆ ಮಾಡುವಾಗ ಹತ್ತು ವಿಧವಾದ ಅಪರಾಧಗಳಾಗಿರಬಹುದೆಂದು ಒಂದು ಕಡೆ ಹೇಳಿದ್ದಾರೆ (ಒಳ್ಳೆಯವರ ಬಗ್ಗೆ ತಪ್ಪನ್ನು ಹೇಳುವುದು, ದುಷ್ಟರ ಬಳಿ ನಾಮ ಮಹಿಮೆಯನ್ನು ಕುರಿತು ಹೇಳುವುದು, ಶಿವನೂ ವಿಷ್ಣುವೂ ಬೇರೆಯೆಂದು ತಿಳಿಯುವುದು, ವೇದ-ಶಾಸ್ತ್ರ-ಗುರುವಿನ ಮಾತು ಮುಂತಾದವುಗಳಲ್ಲಿ ನಂಬಿಕೆ ಇಲ್ಲದೆ ಇರುವುದು, ಭಗವಂತನ ಹೆಸರುಗಳು ನಿಜವಾಗಿಯೂ ಶಕ್ತಿ ಇರುವುದಾಗಿ ಇಲ್ಲದಿದ್ದರೂ ಹಾಗೆ ಕಲ್ಪಿಸಲ್ಪಟ್ಟವೆಂದು ಭ್ರಮೆಯಿಂದಿರುವುದು, ನಾಮವಿದೆಯೆನ್ನುವ ಧೈರ್ಯದಿಂದ ಮಾಡಬಾರದ ಕೆಲಸಗಳನ್ನು ಮಾಡುವುದು, ವಿಹಿತ ಕರ್ಮಗಳನ್ನು ಮಾಡದೆ ಇರುವುದು, ಇತರ ಧರ್ಮಗಳೊಡನೆ ನಾಮವನ್ನು ಸಮವಾಗಿ ಪರಿಗಣಿಸುವುದು - ಇವು ಹತ್ತು ವಿಧವಾದ ಅಪರಾಧಗಳು). ಆ ಅಪರಾಧಗಳು ಯಾವುವೂ ಇಲ್ಲದಿದ್ದರೆ ಮಾತ್ರ ನಾಮವೆನ್ನುವುದು ಫಲಿಸುತ್ತದೆ. ಬೆಂಕಿ ತಿಳಿದು ಮುಟ್ಟಿದರೂ ಸುಡುತ್ತದೆ, ತಿಳಿಯದೆ ಮುಟ್ಟಿದರೂ ಸುಡುತ್ತದೆ. ಅದೇ ರೀತಿ, ನಾಮವೆನ್ನುವುದು ಪಾಪಕ್ಷಯ ಮಾಡುತ್ತದೆಂದು ತಿಳಿದು ಜಪಿಸಿದರೂ ಹಾಗಾಗುತ್ತದೆ, ತಿಳಿಯದೆ ಜಪಿಸಿದರೂ ಹಾಗಾಗುತ್ತದೆ. ಹೀಗಿದ್ದರೂ ಪಾಪಕ್ಷಯ ಮಾಡುವ ಶಕ್ತಿ ನಾಮಕ್ಕಿದೆಯೆಂದುಕೊಂಡು ಪಾಪವನ್ನು ಮಾಡುವುದೇ ಕೆಲಸವಾಗಿಟ್ಟುಕೊಂಡರೆ ಆಗ ಆ ನಾಮ ಅವನನ್ನು ಬಿಟ್ಟು ಬಿಡುತ್ತದೆ. ಏಕೆಂದರೆ, `ಇವನಿಂದ ಪ್ರಪಂಚವೇ ಹಾಳಾಗುತ್ತದೆ. ಪ್ರಪಂಚದ ಕ್ಷೇಮಕ್ಕಾಗಿ ನಾವು ಈ ನಾಮಕ್ಕೆ ಪ್ರಭಾವವನ್ನು ನೀಡಿದರೆ ಪ್ರಪಂಚವನ್ನು ಕೆಡಿಸಲು ನೀನು ಈ ನಾಮವನ್ನು ಉಪಯೋಗಪಡಿಸುತ್ತಾ ಬಂದೆ-ಎಂದರೆ ನಿನಗೆ ಆ ನಾಮದಿಂದ ಉಂಟಾಗುವ ಪ್ರಭಾವವೇ ಇಲ್ಲ'-ಎಂದು ಭಗವಂತನೇ ಶಾಪವನ್ನು ಕೊಟ್ಟು ಬಿಡುತ್ತಾನೆ. ಆದ್ದರಿಂದ ನಾವು ಭಗವಂತನ ನಾಮಸ್ಮರಣೆ, ಈಶ್ವರಾರಾಧನೆ ಮುಂತಾದವುಗಳನ್ನು ಮಾಡುತ್ತಾ ಇದ್ದು ಶಾಸ್ತ್ರವು ಹೇಳುವ ದಾರಿಯಲ್ಲಿ ನಡೆದುಕೊಂಡು ಬಂದರೆ ನಮ್ಮ ಬಾಳಿನಲ್ಲಿ ಯಾವ ವಿಧವಾದ ಕೊರತೆ ಇರುವುದಿಲ್ಲ. ಈ ಭಕ್ತಿಗೆ ಯಾರು ಅಧಿಕಾರಿ, ಯಾರು ಅಧಿಕಾರಿಯಲ್ಲ ಎಂದರೆ ಎಲ್ಲಾರೂ ಅಧಿಕಾರಿಗಳೇ. ಅಧಿಕಾರಿಗಳಾದ ಮೇಲೆ ಭಗವಂತನು ಹೇಳಿದ ರೀತಿಯಲ್ಲಿ ನಡೆಯುತ್ತಾ ಪರಂಪರೆಯಾಗಿ ನಮಗೆ ತಿಳಿದುಬಂದ ದಾರಿಯಲ್ಲಿ ಹೋಗಿ ಭಗವಂತನ ಕಟಾಕ್ಷವನ್ನು ಸಂಪಾದಿಸಬೇಕು.

\newpage

\section{ಶ್ರೀ ಶಂಕರ ಭಗವತ್ಪಾದಾಚಾರ್ಯರು}

ಅನಾದಿಯಿಂದ ಮಾನವನ ಆಧ್ಯಾತ್ಮಿಕ, ಭೌತಿಕ ಪ್ರಯೋಜನಕ್ಕಾಗಿ ಈಶ್ವರ ನಿಃಶ್ವಾಸರೂಪವಾದ ವೇದಗಳಿವೆ. ವೇದಗಳ ಆಧಾರದ ಮೇಲೆ ಸ್ಮೃತಿಗಳು ಬೆಳಕಿಗೆ ಬಂದವು. ಅನಾದಿಯಾಗಿ ಎಂದರೆ ಮಾನವನ ಅವತರಣ ಕಾಲದಿಂದಲೂ ಎಂದು ಭಾವ. ಶಾಸ್ತ್ರಗಳು, ಸಂಪ್ರದಾಯ ನಮಗೆ ತಿಳಿಸುವುದರ ಪ್ರಕಾರ ನಮ್ಮ ಅಭ್ಯುನ್ನತಿಗಾಗಿ ಈಶ್ವರನು ನಮಗೆ ವೇದಗಳನ್ನು ನೀಡಿದನು. ವೇದಗಳು ನಾವು ಅನುಷ್ಠಾನ ಮಾಡಬೇಕಾದ ಕರ್ಮಗಳನ್ನು ತಿಳಿಸುತ್ತವೆ. ಜಗತ್ತಿನ ಸ್ಥಿತಿಯನ್ನು ನಿಯಮಿಸುವ ಶಾಸನಗಳನ್ನು ಉಪಾಸಿಸಲು ಅವು ಹೇಳುತ್ತವೆ. ಪ್ರಪಂಚದ ಸೃಷ್ಟಿಯನ್ನು ಮಾಡಿದವನ ಬಗ್ಗೆ ಯಥಾರ್ಥ ಜ್ಞಾನವನ್ನು ಅವು ತಿಳಿಸುತ್ತವೆ. ಭಿನ್ನ ಭಿನ್ನ ಸ್ಥಳಗಳಲ್ಲಿ ಈ ಕರ್ಮ, ಉಪಾಸನೆ ಮತ್ತು ಜ್ಞಾನದ ಪ್ರಾಮುಖ್ಯ ಹೇಳಲಾಗಿದೆ. ಒಂದರ ಪ್ರಾಮುಖ್ಯವನ್ನು ಕುರಿತು ಹೇಳುವಾಗ ಉಳಿದ ಎರಡರ ಪ್ರಾಮುಖ್ಯವನ್ನು ಹೇಳುವ ಹಾಗಿಲ್ಲ.

ಭೌತಿಕ ಸುಖಾದಿಗಳು, ಐಶ್ವರ್ಯಾದಿಗಳು ಬೇಕೆನ್ನುವವನು ಕರ್ಮಗಳನ್ನೂ ಉಪಾಸನೆಯನ್ನೂ ಅವಲಂಬಿಸಬೇಕು. ಆದರೆ ಜ್ಞಾನ ಸರ್ವೋತ್ತಮವಾದುದು. ಕರ್ಮ-ಉಪಾಸನೆಗಳನ್ನು ಸೋಪಾನವಾಗಿ ಉಪಯೋಗಿಸುವುದರಿಂದಲೇ ಜ್ಞಾನವನ್ನು ಪಡೆಯಲು ಸಾಧ್ಯವಾಗುವುದು. ಕರ್ಮೋಪಾಸನೆಯಲ್ಲಿಯೇ ನಿಲ್ಲುವವನಿಗೆ ಜ್ಞಾನ ಲಭಿಸುವುದಿಲ್ಲ. ಅನೇಕ ಶತಾಬ್ದಿಗಳವರೆಗೆ ಕರ್ಮೋಪಾಸನೆಗಳಿಗೆ ಪ್ರಾಮುಖ್ಯ ಕೊಡಲಾಗಿತ್ತು. ಜ್ಞಾನಹೇತುವಾದ ಶ್ರವಣ, ಮನನ, ನಿದಿಧ್ಯಾಸನಗಳನ್ನು ಮರೆತಂತಾಯಿತು. ಜನರು ಯಾಂತ್ರಿಕವಾದ ಬಾಳಿಗೆ ಅಭ್ಯಸ್ತರಾದರು. ತಮ್ಮ ತಾರ್ಕಿಕ ಲಕ್ಷ್ಯದಿಂದ ಶ್ರೇಯಸ್ಸನ್ನು ಸಾಧಿಸಬಹುದೆಂದು ಭಾವಿಸಿದರು. ಅದರಿಂದಾಗಿ ಮಾನವ ನಿರ್ಮಿತವಾದ ವೇದದೂರವಾದ ಮತಗಳು ಪ್ರಾದುರ್ಭವಿಸಿದುವು. ಇದಕ್ಕೆ ಅನೇಕ ಕಾರಣಗಳಿವೆ.

`ಭಕ್ಷ್ಯಾದ್ಯನಿಯಮ ಇತಿ ರಾಗಿಣಿಃ ಸ್ವೇಚ್ಛಯಾ ದಾರಪರಿಗ್ರಹ ಇತಿ ಕುತರ್ಕಾಭ್ಯಾಸಿನಃ ಕರ್ಮಲಾಘವಮಿತ್ಯಲಸಾಃ, ಇತಃ ಪತಿತಾನಾಮಪ್ಯಸ್ತ್ಯನುಪ್ರವೇಶ ಇತಿ ಅನನ್ಯ ಗತಿಕಾಃ|'

ಇಚ್ಛೆಗಳಿರುವವರು ಮತಾಂತರನ್ನು ಮಾಡಿಕೊಳ್ಳಲು ಕಾರಣ ಆಹಾರ ವಿಷಯವಾದ ಸ್ವಾತಂತ್ರ್ಯ. ಕುತರ್ಕ ಮಾಡುವವರ ಮತಾಂತರಕ್ಕೆ ಕಾರಣ ದಾರಪರಿಗ್ರಹಣದಲ್ಲಿ ಸ್ವೇಚ್ಚೆ. ಸೋಮಾರಿಗಳ ಮತಾಂತರಕ್ಕೆ ಕಾರಣ ಕರ್ಮಭಾರವಿಲ್ಲದೆ ಹೋದುದು. ಅನನ್ಯ ಗತಿಕರ ಮತಾಂತರಕ್ಕೆ ಕಾರಣ ಪತಿತರಿಗೆ ಕೂಡ ಪ್ರವೇಶವಿದೆ ಎನ್ನುವುದು.

ಈ ಕಾರಣಗಳಿಗಾಗಿ ಜನರು ತಪ್ಪು ದಾರಿಯನ್ನು ಹಿಡಿದರು. ಪ್ರತಿಯೊಬ್ಬರೂ ತಮಗೆ ತೋಚಿದಂತೆ ತೀರ್ಮಾನ ಮಾಡಿದರೆ ಪ್ರಮಾಣವಾದ ನಿರ್ಣಯವನ್ನು ಮಾಡುವುದಾದರೂ ಹೇಗೆ?

\begin{shloka}
`ಕೈಶ್ಚಿದಭಿಯುಕ್ತ್ಯರ್ಯತ್ನೇನೋತ್ಪ್ರೇಕ್ಷಿತಾಸ್ತರ್ಕಾ\\
ಅಭಿಯುಕ್ತತರೈರನ್ಯೈರಾಭಾಸ್ಯಮಾನಾ ದೃಶ್ಯಂತೇ||'
\end{shloka}

ಕೆಲವರು ಬುದ್ಧಿವಂತರು ತಮ್ಮ ಬುದ್ಧಿಯ ಮೂಲಕ ಯಾವುದನ್ನು ಸರಿಯೆಂದು ತೋರಿಸಿದರೋ ಅದನ್ನೇ ಹೆಚ್ಚು ಬುದ್ಧಿವಂತರು ತಪ್ಪೆಂದು ತೋರಿಸುತ್ತಿದ್ದರು.

ಮುಖ್ಯವಾಗಿ ನಾವು ಗಮನಿಸಬೇಕಾದುದು - ಭಗವಂತನಿದ್ದಾನೆ. ಅವನು ಸೃಷ್ಟಿ ಮಾಡಿದಾಗಲೇ ತಾನು ಸೃಷ್ಟಿಸಿದ ಮಾನವರಿಗಾಗಿ ನಿಯಮಗಳನ್ನು ಮಾಡಿರಬೇಕು. ಪರಿಪಾಲಿಸುವ ರಾಜನು ಕೂಡ ತನ್ನ ನಿಯಮಗಳನ್ನು ಪ್ರಕಾಶಿಸಬೇಕು. ನಮಗೆ ನಮ್ಮ ಶಾಸ್ತ್ರಗಳಿರಬೇಕು. ಶಾಸ್ತ್ರಗಳ ಪ್ರಕಾರ ಕರ್ಮಫಲವಿರಬೇಕು. ಈಶ್ವರನು ಮಾನವರನ್ನು ಪಾಲಿಸಲು ಶಾಸ್ತ್ರಗಳಲ್ಲಿ ನಿಯಮಗಳನ್ನು ಪ್ರಕಾಶಿಸಿದನು. ಅವುಗಳಿಲ್ಲದಿದ್ದರೆ ಅಯುಕ್ತವಾಗುತ್ತಿತ್ತು. ಕುದುರೆಯ ಮೇಲೆ ಲಗಾಮಿಲ್ಲದೆ ಸವಾರಿ ಮಾಡಿದಂತೆ.

ವೇದಗಳ ಮೇಲೆ ನಂಬಿಕೆ ಇಲ್ಲದಿರುವ ಮತಗಳ ಪ್ರಾಬಲ್ಯದಿಂದಾಗಿ ವೇದಮತ ಕ್ಷೀಣವಾಯಿತು. ಆ ಸಮಯದಲ್ಲಿ ಪೂರ್ಣಾನದಿ ತೀರದಲ್ಲಿ, ಬಾಲಕೃಷ್ಣ ದೇವಾಲಯದ ಹತ್ತಿರ ಪವಿತ್ರ ಕುಟುಂಬದಲ್ಲಿ ಶ್ರೀಶಂಕರ ಭಗವತ್ಪಾದರು ಪ್ರಾದುರ್ಭವಿಸಿದರು. ಅವರು ಬಹಳ ಚಿಕ್ಕ ವಯಸ್ಸಿನಲ್ಲೆ ಶಾಸ್ತ್ರಾಧ್ಯಯನವನ್ನು ಪೂರ್ತಿಗೊಳಿಸಿದರು. ಅವರು ಪಡೆದ ಜ್ಞಾನದ ಕಾರಣದಿಂದಾಗಿ ಐಹಿಕ ಸುಖಭೋಗಗಳಲ್ಲಿ ಅವರಿಗೆ ಇಚ್ಛೆ ಇಲ್ಲದೆ ಹೋಯಿತು. ವೇದಪ್ರಮಾಣವಲ್ಲದ ಮತಗಳು ಸಾರಹೀನವಾದುವೆಂದೂ, ಅಂಥಹವುಗಳನ್ನು ನಿರೋಧಿಸಬೇಕೆಂದೂ ಅವರು ಭಾವಿಸಿದರು. ಅವರು ತಮ್ಮ ತಂದೆ-ತಾಯಿಗೆ ಒಬ್ಬರೇ ಸಂತಾನ. ತಂದೆ ಸ್ವರ್ಗಸ್ಥರಾದರು. ಶ್ರೀಶಂಕರರು ಲೋಕಸಂಗ್ರಹವನ್ನು ವಾಂಛಿಸಿದರು. ಅವರು ಮಹರ್ಷಿ.

\begin{shloka}
ಅಯಂ ನಿಜಃ ಪರೋವೇತಿ ಗಣನಾ ಲಘುಚೇತಸಾಮ್|\\
ಉದಾರಚರಿತಾನಾಂ ತು ವಸುಧೈವ ಕುಟುಂಬಕಮ್||
\end{shloka}

ಅವನು ನನ್ನವನು. ಇವನು ಬೇರೆಯವನು. ಹೀಗೆ ಸಂಕುಚಿತ ಭಾವನೆಯುಳ್ಳವರು ಭಾವಿಸುತ್ತಾರೆ. ಉದಾರ ಚರಿತರಿಗೆ ಲೋಕವೆಲ್ಲಾ ಕುಟುಂಬವೇ

ಇದಕ್ಕೆ ಶ್ರೀಭಗವತ್ಪಾದರು ಉದಾಹರಣೆ, ಈ ಅಭಿಯುಕ್ತೋಕ್ತಿ ಶ್ರೀಭಗವತ್ಪಾದರ ರೂಪದಲ್ಲಿ ಅವತರಿಸಿತೆಂದು ಹೇಳಬೇಕು.

ಅವರು ತಮ್ಮ ತಾಯಿಯ ಮನಸ್ಸಿಗೆ ಕಷ್ಟಕೊಡಲು ಇಷ್ಟಪಡಲಿಲ್ಲ. ಪೂರ್ಣಾನದಿಯಲ್ಲಿ ಸ್ನಾನ ಮಾಡುತ್ತಿದ್ದಾಗ ಒಂದು ಮೊಸಳೆ ಅವರನ್ನು ಹಿಡಿದುಕೊಂಡಿತು. ಅವರು ಸಹಾಯಕ್ಕಾಗಿ ಕೂಗಿದರು. ತಾಯಿ ಬಂದು ನೋಡಿ ಗಾಬರಿಗೊಂಡಳು. ಆಕೆಯೊಡನೆ ಅವರು `ಸಂನ್ಯಾಸವೆಂದರೆ ಎಲ್ಲರಿಗೂ ಅಭಯವನ್ನು ಕೊಡುವುದು, ನನ್ನಿಂದ ಭಯಗೊಂಡು ಮೊಸಳೆ ನನ್ನನ್ನು ಹಿಡಿದುಕೊಂಡಿದೆ. ನನ್ನಿಂದ ಎಲ್ಲರಿಗೂ ಅಭಯವಾದರೆ ಇದು ನನ್ನನ್ನು ಬಿಟ್ಟು ಬಿಡುವುದು ಆದ್ದರಿಂದ ನಾನು ಸಂನ್ಯಾಸ ಸ್ವೀಕರಿಸಬೇಕು' ಎಂದರು. ನಿಸ್ಸಹಾಯಳಾದ ಆಕೆ ಒಪ್ಪಿಕೊಂಡರು. ಅವರು ಗುರುವನ್ನು ಹುಡುಕಿಕೊಂಡು ನರ್ಮದಾ ತೀರಕ್ಕೆ ಬಂದರು. ಅಲ್ಲಿ ಶ್ರೀ ಗೋವಿಂದ ಭಗವತ್ಪಾದರಿಂದ ಕ್ರಮ ಸಂನ್ಯಾಸವನ್ನು ಸ್ವೀಕರಿಸಿದರು. ಆ ಗುರುವಿನಲ್ಲಿ ಪಡೆದ ವಿದ್ಯೆಯೇ ಫಲಿಸಿತು.

ಶ್ರೀಶಂಕರರ ಶಕ್ತಿಸಾಮರ್ಥ್ಯಗಳನ್ನು ಕಂಡು ಅವರ ಗುರುಗಳು ವಿಸ್ಮಯಾನಂದಚಕಿತರಾದರು. ಲೋಕಸಂಗ್ರಹಕ್ಕಾಗಿ ಬ್ರಹ್ಮಸೂತ್ರಗಳಿಗೆ ಭಾಷ್ಯ ಬರೆಯುವಂತೆ ಗುರುಗಳು ಅವರಿಗೆ ಅಪ್ಪಣೆ ಕೊಟ್ಟರು. ಗುರುಗಳ ಅಪ್ಪಣೆ ಪ್ರಕಾರ ಶ್ರೀಶಂಕರರು ಬ್ರಹ್ಮಸೂತ್ರ-ಭಗವದ್ಗೀತಾ ಉಪನಿಷದ್ಭಾಷ್ಯಗಳನ್ನು ರಚಿಸಿದರು. ಶ್ವೇತಾಶ್ವತರೋಪನಿಷತ್ತಿಗೂ ಭಾಷ್ಯವನ್ನು ಬರೆದರು. ಮಹಾಭಾರತದಲ್ಲಿರುವ ಸನತ್ಸುಜಾತೀಯಕ್ಕೆ ಕೂಡ ವ್ಯಾಖ್ಯೆ ಬರೆದರು. ಮತ್ತೂ ಎಷ್ಟೋ ಗ್ರಂಥಗಳನ್ನು ರಚಿಸಿದರು. ಸರ್ವವ್ಯಾಪಿಯಾದ ಬ್ರಹ್ಮವೊಂದೇ ಸತ್ಯವೆಂದು ದೃಢಪಡಿಸಿದರು, ಅದೇ ತುರೀಯವೆಂದರು.

ಶ್ರೀಶಂಕರರ ಭಾಷ್ಯಗಳಲ್ಲಿ ಸುಂದರವಾದ ಉದಾಹರಣೆಗಳಿವೆ. ಶಾಸ್ತ್ರಜ್ಞಾನವಿರುವವರಿಗೆ ಶ್ರೀಶಂಕರರ ಗ್ರಂಥಗಳು ಚೆನ್ನಾಗಿ ಅರ್ಥವಾಗುತ್ತವೆ. ತಾರ್ಕಿಕರ ವಾದಗಳಿಗೆಲ್ಲ ಅವರು ಸಮಾಧಾನ ಹೇಳಿ ಅವರ ವಾದಗಳನ್ನು ಪೂರ್ವಪಕ್ಷ ಮಾಡಿದರು. ಅವರ ಗ್ರಂಥಗಳು ಪಂಡಿತರಿಗೆ ಎಷ್ಟೋ ಮೋದವನ್ನುಂಟು ಮಾಡುತ್ತವೆ. ಆದರೆ ಶ್ರೀ ಶಂಕರರು ಪಂಡಿತರಿಗಾಗಿ ಮಾತ್ರವಲ್ಲ ಭೂಮಿಯ ಮೇಲೆ ಅವತರಿಸಿದುದು ಪ್ರತಿಯೊಬ್ಬರಿಗೂ ಪ್ರಯೋಜನವಾಗಬೇಕು. ಪ್ರತಿಯೊಬ್ಬರೂ ಸಂಸಾರ ಸಾಗರವನ್ನು ದಾಟಬೇಕು ಎನ್ನುವುದು ಅವರ ಭಾವನೆ. ಅದಕ್ಕಾಗಿ ಅವರು ವಿವೇಕಚೂಡಾಮಣಿ, ಉಪದೇಶಸಾಹಸ್ರೀ, ಪ್ರಬೋಧಸುಧಾಕರ ಮುಂತಾದ ಗ್ರಂಥಗಳನ್ನು ರಚಿಸಿದರು.

ಈ ಗ್ರಂಥಗಳನ್ನು ಓದಲು ಬಹಳ ದಿನಗಳು ಬೇಕಾಗಿಲ್ಲ. ಈ ಗ್ರಂಥಗಳನ್ನು ಅರ್ಥಮಾಡಿಕೊಳ್ಳಲು ಶಾಸ್ತ್ರಜ್ಞಾನ ಕೂಡ ಅವಶ್ಯಕವಿಲ್ಲ. ಸ್ವಲ್ಪ ಶ್ರದ್ಧೆ ಮತ್ತು ಸ್ವಲ್ಪ ಸಂಸ್ಕೃತ ಭಾಷಾಜ್ಞಾನವಿದ್ದರೆ ಸಾಕು.

ಶ್ರೀಶಂಕರರ ಜನ್ಮಸ್ಥಾನ ಕೇರಳ. ಆ ಪ್ರಾಂತೀಯ ಭಾಷೆ ಅವರ ಮಾತೃಭಾಷೆಯಾಗಿದ್ದಿರಬಹುದು. ಆದರೆ ಅವರು ತಮ್ಮ ಮಾತೃಭಾಷೆಯಲ್ಲಿ ಗ್ರಂಥಗಳನ್ನು ಬರೆಯಲಿಲ್ಲ. ವಿಶಾಲ-ದೃಷ್ಟಿಯಿಂದ ಸಂಸ್ಕೃತ ಆ ಕಾಲದಲ್ಲಿ ಅಖಿಲ ಭಾರತ ವಿದ್ವದ್ಭಾಷೆಯಾದುದರಿಂದ ಆ ಭಾಷೆಯಲ್ಲಿ ಎಲ್ಲರಿಗೂ ಉಪದೇಶ ಯೋಗ್ಯವಾದ ತಮ್ಮ ಗ್ರಂಥಗಳನ್ನು ಬರೆದರು. ಎಲ್ಲ ಧರ್ಮಗಳಿಗೂ ವೇದಗಳು ಮೂಲ, ವೇದಗಳು ಸಂಸ್ಕೃತದಲ್ಲಿವೆ. ಆದ್ದರಿಂದ ಸಂಸ್ಕೃತದಲ್ಲಿ ಅವರು ಗ್ರಂಥರಚನೆ ಮಾಡಿದುದು ಸಮುಚಿತವೇ.

ಶ್ರೀಶಂಕರರು ಒಂದು ಹೊಸ ಮತವನ್ನು ಪ್ರಚಾರಮಾಡಲು ಪ್ರಯತ್ನಿಸಲಿಲ್ಲ. ಹಾಗೆ ಮಾಡುವುದು ನಿಷ್ಪ್ರಯೋಜನಕರ. `ವೇದ ಸಾರವೇನು? ವೇದವಿಹಿತ ಧರ್ಮಗಳನ್ನು ಬಿಡಬಹುದೇ? ಅವುಗಳನ್ನು ವಿಸ್ಮರಿಸಿದರೆ ನಮಗೆ ಶ್ರೇಯಸ್ಸಿಲ್ಲ. ಸತ್ಯದ ಬಗ್ಗೆ ನಮಗೆ ತಿಳಿಯದು'-ಇವು ಅವರ ಉಪದೇಶ. ಇದು ಹೊಸ ಮತವಲ್ಲ. ಹೊಸ ಆಂದೋಳನವಲ್ಲ. ವೇದಗಳು ಸಂಸ್ಕೃತದಲ್ಲಿರುವಾಗ ವೇದಸಾರವನ್ನು ಬೇರೆ ಭಾಷೆಯಲ್ಲಿ ಹೇಳುವುದು ಶಂಕರರಿಗೆ ಹೇಗೆ ಸಮಂಜಸ? ಪ್ರತಿಯೊಬ್ಬರಲ್ಲೂ ನಿತ್ಯಾನಿತ್ಯ ವಸ್ತುವಿವೇಕ, ವೈರಾಗ್ಯ, ಶಮಾದಿಷಟ್ಕ ಮತ್ತು ಮೋಕ್ಷೇಚ್ಛೆ ಇರುವುದಿಲ್ಲ. ಅವು ಇಲ್ಲದವರಿಗೆ ವೇದಾಂತ ಶ್ರವಣಯೋಗ್ಯತೆ ಇಲ್ಲ. ಇನ್ನು ಮನನ, ನಿದಿಧ್ಯಾಸಗಳು ಹೇಗೆ? ಅವರಿಗೆ ವೇದಜ್ಞಾನ ಲಾಭಯೋಗ್ಯತೆ ಇಲ್ಲ. ಅವರಿಗೂ ಒಳ್ಳೆಯದು ಆಗಬೇಕು. ಭಗವಂತನ ಅನುಗ್ರಹವಾದರೆ ಎಲ್ಲವನ್ನೂ ಸಾಧಿಸಬಹುದು. ಅದಕ್ಕಾಗಿ ಶ್ರೀಶಂಕರರು `ಪ್ರಪಂಚಸಾರ'ವೆನ್ನುವ ಮಂತ್ರ ಶಾಸ್ತ್ರಗ್ರಂಥ ಒಂದನ್ನು ರಚಿಸಿದರು. ದೇವತೆಗಳನ್ನು ಎಲ್ಲರೂ ಪ್ರಾರ್ಥಿಸಬೇಕೆನ್ನುವುದು ಅವರ ಸದಭಿಪ್ರಾಯ. ಹಾಗೆ ಪ್ರಾರ್ಥಿಸಿ ಭಗವದನುಗ್ರಹವನ್ನು ಸಂಪಾದಿಸಿ ಶ್ರೇಯ-ಪ್ರೇಯಸ್ಸುಗಳನ್ನು ಸಾಧಿಸಿಕೊಳ್ಳಲೆಂದೂ ದೇವತಾನುಗ್ರಹದಿಂದ ಪ್ರತಿಯೊಬ್ಬರೂ ವೇದಾಂತ ತತ್ತ್ವಗಳನ್ನು ತಿಳಿಯಬಲ್ಲರೆಂದೂ ಅವರು ಭಾವಿಸಿದರು. ಇಂಥ ದೊಡ್ಡ ಔದಾರ್ಯ ಭಾವದಿಂದ ಅವರು ರಚಿಸಿರುವ ಸ್ತೋತ್ರಗಳನ್ನು ಪಠಿಸಿ ಯಾರು ಬೇಕಾದರೂ ಅವುಗಳ ಪ್ರಯೋಜನವನ್ನು ಪಡೆಯಬಹುದು.

ಶ್ರೀಭಗವತ್ಪಾದರು ಆಸೇತು ಹಿಮಾಚಲ ಪರ್ಯಂತ ಎರಡು ಸಾರಿ ಪರ್ಯಟನೆ ಮಾಡಿದರು. ಅವರು ತಮ್ಮ ವಿದ್ವತ್ತಿನಿಂದ ಪಂಡಿತರೆಲ್ಲರೂ ಅವರ ವಾದವನ್ನು ಅಂಗೀಕರಿಸುವಂತೆ ಮಾಡಿದರು. ಆ ದಿನಗಳಲ್ಲಿ ಎಷ್ಟೋ ಚಿಕ್ಕ ಚಿಕ್ಕ ಮತಗಳು ಹುಟ್ಟಿಕೊಂಡಿದ್ದವು. ಅಂಥವರನ್ನು ವೇದಮತಾನುಯಾಯಿಗಳನ್ನಾಗಿ, ಮಾಡಿ - `ನೀನು ಶಿವನನ್ನಾಗಲಿ, ದೇವಿಯನ್ನಾಗಲಿ, ಗಣಪತಿಯನ್ನಾಗಲಿ, ಸೂರ್ಯನನ್ನಾಗಲಿ, ಸುಬ್ರಹ್ಮಣ್ಯನನ್ನಾಗಲಿ, ಬೇರೆ ಯಾವ ದೇವರನ್ನಾಗಲಿ ಪೂಜಿಸಿದರೂ ನಿಜಕ್ಕೂ ಸಚ್ಚಿದಾನಂದ ರೂಪನಾದ ಪರ ಬ್ರಹ್ಮನನ್ನೇ ಆರಾಧಿಸುತ್ತಿರುವೆ' ಎಂದು ಅವರು ಬೋಧಿಸಿದರು. ಯಾರಾದರೂ ತಮ್ಮ ಅಭಿರುಚಿಗೆ ಅನುಗುಣವಾಗಿ ಆರಾಧಿಸಬಹುದು. `ಏಕಂ ಸದ್ವಿಪ್ರಾ ಬಹುಧಾ ವದನ್ತಿ' - ಬ್ರಹ್ಮ ಒಂದೇ, ಜ್ಞಾನಿಗಳು ಹಲವು ವಿಧವಾಗಿ ಹೇಳುತ್ತಾರೆ.

ಇದನ್ನು ನಾವು ಮರೆಯಬಾರದು. `ಧರ್ಮಭೂಮಿ' ಎಂದರೆ ನಾವು ವಾಸಿಸುವ ಪಟ್ಟ‌ಣ, ಸುತ್ತ ಮುತ್ತಿನ ಪ್ರದೇಶ ಮಾತ್ರವೇ ಅಲ್ಲ. ಭಾರತ ದೇಶವೆಲ್ಲ `ಧರ್ಮಭೂಮಿ' ಎಂದು ಶ್ರೀಶಂಕರರು ಹೇಳಿದರು. ಅವರು ಭಾರತ ದೇಶದ ನಾಲ್ಕು ದಿಕ್ಕುಗಳಲ್ಲಿಯೂ ಪ್ರಾಕ್ ದಕ್ಷಿಣ ಪೂರ್ವೋತ್ತರಗಳಲ್ಲಿ ನಾಲ್ಕು ಮಠಗಳನ್ನು ಸ್ಥಾಪಿಸಿ ಅವುಗಳಿಗೆ ನಾಲ್ಕು ಮಹಾವಾಕ್ಯಗಳನ್ನು ಕೊಟ್ಟರು. ಈಗಲೂ ಆ ಮಠಗಳನ್ನು ನೋಡಬಹುದು.

ಅವರ ಕಾರಣದಿಂದಾಗಿ ವೇದಗಳು ಈಗಲೂ ಸ್ಥಿರವಾಗಿವೆ. ಶಿವ-ವಿಷ್ಣು ದೇವತೆಗಳ ದೇವಾಲಯಗಳು ನಿಂತಿವೆ. ಅವರು ಇಲ್ಲದೆ ಹೋಗಿದ್ದರೆ ಧರ್ಮದ ಸ್ವರೂಪ ನಮಗೆ ತಿಳಿಯುತ್ತಿರಲಿಲ್ಲ. ಸಂಸಾರ ಬದ್ಧನಾದ ಜೀವನು ವಿಮುಕ್ತಿಯನ್ನು ಪಡೆದು ಕೈವಲ್ಯವನ್ನು ಸಾಧಿಸುವ ಮಾರ್ಗವನ್ನು ಅವರು ನಮಗೆ ಬೋಧಿಸಿದರು, ಅನುಗ್ರಹಿಸಿದರು. ಅವರು ರಚನೆಯಾದ `ಮೋಹ ಮುದ್ಗರ' ದಲ್ಲಿನ ಈ ಶ್ಲೋಕ ಮನನೀಯವಾಗಿದೆ.

\begin{shloka}
ತ್ವಯಿ ಮಯಿ ಚಾನ್ಯತ್ರೈಕೋ ವಿಷ್ಣುಃ ವ್ಯರ್ಥಂ ಕುಪ್ಯಸಿ ಮಯ್ಯಸಹಿಷ್ಣುಃ|\\
ಸರ್ವಸ್ಮಿನ್ನಪಿ ಪಶ್ಯಾತ್ಮಾನಂ ಸರ್ವತ್ರೋಸ್ತೃಜ ಭೇದ ಜ್ಞಾನಮ್||
\end{shloka}

ನಿನ್ನಲ್ಲಿ, ನನ್ನಲ್ಲಿ ಎಲ್ಲರಲ್ಲಿಯೂ ಒಬ್ಬನೇ ಪ್ರಭುವಿದ್ದಾನೆ, ವ್ಯರ್ಥವಾಗಿ ನನ್ನ ಮೇಲೆ ಕೋಪಿಸಿಕೊಳ್ಳಬೇಡ. ಎಲ್ಲರಲ್ಲಿಯೂ ಆತ್ಮನನ್ನು ನೋಡು, ಭೇದಜ್ಞಾನವನ್ನು ಬಿಡು.


\newpage


\section{ಧ್ಯಾನ}

ಪರಮಾರ್ಥಕ್ಕೆ ಸಾಧನವಾಗಿ ಕರ್ಮ, ಭಕ್ತಿ ಮತ್ತು ಜ್ಞಾನ ಎಂಬ ಮೂರು ಮಾರ್ಗಗಳು ಇವೆ. ಇವುಗಳಲ್ಲಿ ಉಪಾಸನೆ ಎನ್ನುವುದು ಭಕ್ತಿಯಲ್ಲಿ ಸೇರುತ್ತದೆ. ಕರ್ಮ ಮತ್ತು ಭಕ್ತಿಯ ವಿಷಯವಾಗಿ ಈಗ ವಿಚಾರ ಮಾಡಲಾಗುವುದಿಲ್ಲ. ಉಪಾಸನೆ ಎಂದರೇನು? ಈ ವಿಷಯವನ್ನು ಕುರಿತು ಶಂಕರಭಗವತ್ಪಾದರು ತಾವು ಬರೆದ ಛಾಂದೋಗ್ಯೋಪನಿಷತ್ ಭಾಷ್ಯದ ಪ್ರಾರಂಭದಲ್ಲಿ ಚೆನ್ನಾಗಿ ಸ್ಪಷ್ಟ ಪಡಿಸಿದ್ದಾರೆ. ಉಪಾಸನೆ ಎನ್ನುವುದು ಒಂದು ಮನೋಧರ್ಮ. ಜ್ಞಾನ ಎನ್ನುವುದು ಕೂಡ ಒಂದು ಮನೋಧರ್ಮ. ಹೀಗಿರುವಾಗ ಜ್ಞಾನಕ್ಕೂ ಉಪಾಸನೆಗೂ ವ್ಯತ್ಯಾಸವೇನು? ಉಪಾಸನೆ ಮಾಡುವಾಗಲೂ ಒಂದು ಜ್ಞಾನ ಇದ್ದೇ ಇರಬೇಕು. ಉದಾಹರಣೆಗೆ ದೇವಿಯನ್ನು ಧ್ಯಾನಿಸುತ್ತೇವೆ. ಹಾಗೆ ದೇವಿಯನ್ನು ಧ್ಯಾನಿಸುವಾಗ ದೇವಿಗೆ ಸಂಬಂಧಪಟ್ಟ ಜ್ಞಾನ ಉಂಟಾಗುತ್ತದೆ. ಏಕೆಂದರೆ ಜ್ಞಾನವಿಲ್ಲದೆ ಉಪಾಸನೆ ಹೇಗೆ ಸಾಧ್ಯ? ಈ ಆಕ್ಷೇಪವನ್ನು ಮನಸ್ಸಿನಲ್ಲಿಟ್ಟುಕೊಂಡೇ ಜ್ಞಾನಕ್ಕೂ ಉಪಾಸನೆಗೂ ವ್ಯತ್ಯಾಸವೇನು ಎನ್ನುವ ಪ್ರಶ್ನೆ ಉಂಟಾಯಿತು. ವ್ಯತ್ಯಾಸವಿಲ್ಲ ಎನ್ನುವ ಒಂದು ಆಕ್ಷೇಪವನ್ನು ಸಾಮಾನ್ಯ ದೃಷ್ಟಿಯಿಂದ ಭಗವತ್ಪಾದರು ತೆಗೆದುಕೊಂಡು, `ಕಸ್ತರ್ಹಿ ಅದ್ವೈತ ಜ್ಞಾನಸ್ಯ ಉಪಾಸನಾನಾಂ ಚ ವಿಶೇಷಃ'? ಎಂದು ಕೇಳಿದರು. `ಅದ್ವೈತ' ಜ್ಞಾನಕ್ಕೂ ಉಪಾಸನೆಗೂ ಏನು ಭೇದ' ಎನ್ನುವುದು ತಾತ್ಪರ್ಯ. ಇದಕ್ಕೆ ಉತ್ತರವಾಗಿ `ಉಚ್ಯತೇ-ಸ್ವಾಭಾವಿಕಸ್ಯಾತ್ಮನ್ಯಕ್ರಿಯೇಽಧ್ಯಾರೋಪಿತಸ್ಯ ಕರ್ತ್ರಾದಿಕಾರಕಕ್ರಿಯಾ ಫಲಭೇದ ವಿಜ್ಞಾನಸ್ಯ ನಿವರ್ತಕಮದ್ವೈತ ವಿಜ್ಞಾನಮ್, ರಜ್ಜ್ವಾದಾವಿವ ಸರ್ಪಾದ್ಯಧ್ಯಾರೋಪಲಕ್ಷಣ ಜ್ಞಾನಸ್ಯ ರಜ್ಜ್ವಾದಿ ಸ್ವರೂಪನಿಶ್ಚಯಃ ಪ್ರಕಾಶನಿಮಿತ್ತಃ|-ಎಂದು ಅವರು ಸ್ಪಷ್ಟ ಪಡಿಸಿದರು.

ಆದ್ವೈತಜ್ಞಾನಕ್ಕೂ ಉಪಾಸನೆಗೂ ಬಹಳಷ್ಟು ವ್ಯತ್ಯಾಸವಿದೆ. ಆತ್ಮನು ಆಕ್ರಿಯ. ಹೀಗಿದ್ದರೂ `ನಾನು ಮಾಡುತ್ತೇನೆ' ಎನ್ನುವ ಕರ್ತೃತ್ವ, `ನಾನು ಕ್ರಿಯೆಗೆ ಕರ್ತೃವಾಗಿದ್ದೇನೆ.' ಎನ್ನುವ ಕರಣತ್ವ, `ನನ್ನಿಂದ ಕೆಲಸವಾಗುತ್ತದೆ' ಎನ್ನುವ ಅಧಿಷ್ಠಾನತ್ವ ಇಂಥ ಶಾಸ್ತ್ರದಲ್ಲಿ ಕಾರಕವೆಂದು ಹೇಳಲಾಗಿರುವ ಆರೂ ಆತ್ಮನ ಮೇಲೆ ಆರೋಪಣೆ ಮಾಡಲ್ಪಡುತ್ತವೆ. ಇವುಗಳು ನಿಜಕ್ಕೂ ಆತ್ಮನಲ್ಲಿ ಇಲ್ಲ. ಆದ್ದರಿಂದಲೇ ಇವುಗಳು ಆತ್ಮನ ಮೇಲೆ ಆರೋಪಿತವಾಗಿದೆಯೆಂದು ಹೇಳಲಾಗಿದೆ. ಇದಕ್ಕೆ ಉಪಮಾನವೇನು? ಹೆಚ್ಚಾಗಿ ಬೆಳಕಿಲ್ಲದ ಜಾಗದಲ್ಲಿ ಹಗ್ಗ ಹಾವಾಗಿ ತೋರಬಹುದು. ಇದೇ ರೀತಿ ಇಲ್ಲಿ, ಹಗ್ಗದಲ್ಲಿ ಹಾವು ಕಲ್ಪಿಸಿದಂತೆ ಆತ್ಮನಲ್ಲಿ ಕರ್ತೃತ್ವದಂತಹ ಕಾರಕಗಳೂ ಮತ್ತು ಕ್ರಿಯೆ, ಫಲ ಇಂತಹವುಗಳ ಕಲ್ಪನೆಯನ್ನು ಯಾವುದು ದೂರಮಾಡುವುದೋ ಅದೇ ಅದ್ವೈತ ಜ್ಞಾನವೆನ್ನಲಾಗುವುದು. ಹೀಗಿರುವಾಗ ಉಪಾಸನೆ ಅಥವಾ ಧ್ಯಾನ ಎನ್ನುವುದು ಯಾವುದು?

`ಉಪಾಸನಂ ತು ಯಥಾ ಶಾಸ್ತ್ರಸಮರ್ಥಿತಂ ಕಿಂಚದಾಲಂಬನಮುಪಾದಾಯ ತಸ್ಮಿನ್ ಸಮಾನಚಿತ್ತವೃತ್ತಿ ಸಂತಾನಕರಣಂ ತದ್ವಿಲಕ್ಷಣಪ್ರತ್ಯಯಾನನ್ತರಿತಮಿತಿ ವಿಶೇಷಃ| ತಾನಿ ಏತಾನಿ ಉಪಾಸನಾನಿ ಸತ್ತ್ವ ಶುದ್ಧಿಕರತ್ವೇನ ವಸ್ತುತತ್ತ್ವಾವ ಭಾಸಕತ್ವಾದದ್ವೈತಜ್ಞಾನೋಪಕಾರಕಾಣಿ|'

-ಎನ್ನುವುದು ಭಗವತ್ಪಾದರು ಕೊಟ್ಟ ಉತ್ತರ. ಒಂದು ವಸ್ತು ಒಂದು ವಿಧವಾಗಿ ಇಲ್ಲವೆಂದುಕೊಳ್ಳೋಣ. ಹಾಗಿದ್ದರೂ ಆ ವಸ್ತು ಇಲ್ಲದಿರುವ ರೀತಿಯಲ್ಲಿ ಅದು ಇರುವಂತೆ ಚಿಂತನೆ ಮಾಡಬಹುದು. ಹೀಗೆ ಚಿಂತನೆ ಮಾಡಿದರೆ ಯಾವುದಾದರೂ ಪ್ರಯೋಜನವಿದೆಯೆ ಎಂದು ಕೇಳಿದರೆ ವಿಶೇಷವಾದ ಫಲವಿದೆ ಎನ್ನುವುದು ಉತ್ತರವಾಗುತ್ತದೆ. ಈ ಚಿಂತನೆ ಯಾವ ವಿಧವಾಗಿರಬೇಕು ಎನ್ನುವುದಕ್ಕೆ `ಯಥಾಶಾಸ್ತ್ರಂ' ಶಾಸ್ತ್ರದಲ್ಲಿ ಹೇಳಿದಂತೆ ಎಂದು ಹೇಳಿದರು. ಉದಾಹರಣೆಗೆ ಶಾಸ್ತ್ರದಲ್ಲಿ.- 

`ಮನೋ ಬ್ರಹ್ಮೇತಿ ಉಪಾಸೀತ' ಎಂದಿದೆ. ಮನಸ್ಸೆನ್ನುವುದು ಎಲ್ಲೆಯುಳ್ಳದ್ದು. ಆದರೆ ಬ್ರಹ್ಮವೆನ್ನುವುದು ಎಲ್ಲೆ ಇಲ್ಲದಿರುವುದು. ಆದರೂ ಇಂಥ ಮನಸ್ಸನ್ನು ಬ್ರಹ್ಮವೆಂದು ಉಪಾಸನೆ ಮಾಡಿಕೊಂಡು ಬಾ ಎಂದು ಮೇಲೆ ಹೇಳಿದ ಶಾಸ್ತ್ರವಾಕ್ಯದಲ್ಲಿರುವ ಅಪ್ಪಣೆ. ಹೀಗೆ ಮಾಡಿದರೆ ಒಳ್ಳೆಯ ಫಲವುಂಟು. ಈ ಧ್ಯಾನ ಜ್ಞಾನವಾಗುವುದಿಲ್ಲ. ಇದರ ಬಗ್ಗೆ' `ಅಲ್ಪಾಲಂಬನ ತಿರಸ್ಕಾರೇಣ ಉತ್ಕೃಷ್ಟ ವಸ್ತ್ವಭೇದಧ್ಯಾನಂ ಸಂಪತ್' ಎಂದು ಹೇಳಲ್ಪಟ್ಟಿದೆ. ಮನಸ್ಸೆನ್ನುವುದು ಒಂದು ಅಲ್ಪವಾದ ಅವಲಂಬ. ಇದನ್ನು ತಿರಸ್ಕರಿಸಿ ಉತ್ತಮವಾದ ಬ್ರಹ್ಮವನ್ನು ಧ್ಯಾನ ಮಾಡುವುದನ್ನು `ಸಂಪತ್' ಎನ್ನಲಾಗುವುದು. ಶಾಸ್ತ್ರಗಳಲ್ಲಿ ಪ್ರತೀಕೋಪಾಸನೆಯ ಬಗ್ಗೆ ವಿವರವಿದೆ. ಅದರ ಪ್ರಕಾರ ನಾವು ಒಂದು ಸಾಲಿಗ್ರಾಮವನ್ನು ವಿಷ್ಣುವೆಂದುಕೊಳ್ಳುತ್ತೇವೆ. ನಮಗೆ ಸಾಲಿಗ್ರಾಮ ಒಂದು ಕಲ್ಲೆಂದು ಗೊತ್ತು. ಆದರೂ ಅದರಲ್ಲಿ ವಿಷ್ಣು ಭಾವನೆಯನ್ನು ಇಟ್ಟುಕೊಳ್ಳುತ್ತೇವೆ. ಹೀಗೆ ಮಾಡುವುದಕ್ಕೆ ಪ್ರತೀಕೋಪಾಸನೆ ಎಂದು ಹೆಸರು. ಇದೇ ರೀತಿ ಅಂಗಾವಬದ್ಧ ಉಪಾಸನೆ ಎನ್ನುವುದೂ ಇದೆ. ಒಂದು ಯಜ್ಞದಲ್ಲಿ ಒಂದು ಸಾಮಗಾನ ಮಾಡಬೇಕಾಗಿ ಬರುತ್ತದೆ. ಆ ಸಾಮಕ್ಕೆ ಐದು ಭಾಗಗಳಿವೆ. ಆ ಒಂದೊಂದು ಭಾಗದಲ್ಲೂ ಒಂದೊಂದು ವಿಧವಾದ ದೃಷ್ಟಿಯನ್ನು ಇಟ್ಟುಕೊಂಡಿರಬೇಕು. ಹೀಗೆ ಮಾಡುವುದಕ್ಕೆ ಅಂಗಾವಬದ್ಧ ಉಪಾಸನೆಯೆಂದು ಹೆಸರು. ಈ ಉಪಾಸನೆ ಏತಕ್ಕಾಗಿ ಮಾಡಬೇಕು? ಯಾಗವನ್ನು ಮಾಡಿದರೆ ಫಲವುಂಟು. ಆದರೆ ಆ ಉಪಾಸನೆಯನ್ನೂ ಮಾಡಿದರೆ ಒಂದು ವಿಶೇಷವಾದ ಫಲವುಂಟೆನ್ನುವುದು ಇದಕ್ಕೆ ಉತ್ತರ.

ಇವುಗಳಿಂದ ಜ್ಞಾನಕ್ಕೂ ಉಪಾಸನೆಗೂ ವ್ಯತ್ಯಾಸವಿದೆಯೆಂದು ಸ್ಪಷ್ಟವಾಗುತ್ತದೆ. ಹೀಗಿದ್ದರೂ ಉಪಾಸನೆ ಜ್ಞಾನಕ್ಕೆ ಸಾಧನವಾಗಿದೆ. ಆದ್ದರಿಂದ ಮನುಷ್ಯ ಜನ್ಮವನ್ನು ಪಡೆದವನು ಯಾವುದಾದರೂ ಉಪಾಸನೆಯನ್ನು ಮಾಡಿ ಶ್ರೇಯಸ್ಸು ಪಡೆಯುವುದು ಒಳ್ಳೆಯದು.

ಉಪಾಸನೆ ಮಾಡುವುದು ಆವಶ್ಯಕವೆಂದು ನೋಡಿಯಾಯಿತು. ಇದು ಮಾಡಬಹುದೇ? ಈ ವಿಷಯವನ್ನು ಕುರಿತು ಯೋಗಶಾಸ್ತ್ರದಲ್ಲಿ ಉತ್ತರ ಹೇಳಿದೆ. ಅಲ್ಲಿ, ಮನಸ್ಸು ಐದು ವಿಧವಾಗಿ ವಿಂಗಡಿಸಲ್ಪಟ್ಟಿದೆ. ಮನಸ್ಸಿನ ಸ್ಥಿತಿಯನ್ನು ಕ್ಷಿಪ್ತ, ಮೂಢ, ವಿಕ್ಷಿಪ್ತ, ಏಕಾಗ್ರ ಮತ್ತು ನಿರುದ್ಧವೆಂದು ಐದು ವಿಧವಾಗಿ ವಿಂಗಡಿಸಿದರು. ಐದನೆಯದಾಗಿ ಹೇಳಲ್ಪಟ್ಟ ನಿರುದ್ಧವೆನ್ನುವ ಮನಃಸ್ಥಿತಿ ಪ್ರಾಪ್ತವಾದರೆ ಬೇರೆ ಯಾವುದೂ ಬೇಕಾಗಿಲ್ಲ. ಹೀಗೆ ಐದು ವಿಧವಾದ ಮನಃಸ್ಥಿತಿಗಳಿದ್ದರೂ ನಾವು ಸಾಮಾನ್ಯವಾಗಿ ಕ್ಷಿಪ್ತವಾಗಿರುವ ಮನಸ್ಸನ್ನೇ ನೋಡುತ್ತೇವೆ. ಕ್ಷಿಪ್ತ ಸ್ಥಿತಿಯಲ್ಲಿರುವ ಮನಸ್ಸನ್ನು ಕುರಿತು.

`ವಿಷಯೇಷು ಜ್ಞಾಪಕಂ ಮನಃ ಚಿತ್ತಮಿತ್ಯಬಿಧೀಯತೇ|' (ವಿಷಯಗಳನ್ನು ಅರಿಯುವ ಮನಸ್ಸಿಗೆ ಚಿತ್ತವೆಂದು ಹೆಸರು) ಎಂದು ಯೋಗ ಶಾಸ್ತ್ರದಲ್ಲಿರುವಂತೆಯೇ ಹೇಳಬೇಕು. ಯಾವಾಗಲೂ ಮನಸ್ಸಿಗೆ ಹೊರಗಿನ ವಸ್ತುಗಳ ಬಗ್ಗೆ ಯಾವುದಾದರೂ ವಿಷಯ ತೋರುತ್ತಲೇ ಇರುತ್ತದೆ. ಜಾಗ್ರತ್, ಸ್ವಪ್ನ ಈ ಎರಡು ಸ್ಥಿತಿಗಳಲ್ಲಿಯೂ ವಿಷಯವಿಲ್ಲದ ಸಮಯವೆನ್ನುವುದು ಅಪರೂಪ. ಇದು ಕ್ಷಿಪ್ತವಾಗಿರುವ ಮನಃಸ್ಥಿತಿ, ಅದಾದಮೇಲೆ ಹೇಳಿರುವುದು ಮೂಢಸ್ಥಿತಿ. ದುಃಖದಲ್ಲೊ ಸೋಮಾರಿತನದಲ್ಲೊ ಈ ಸ್ಥಿತಿಗೆ ಪ್ರಿಯತ್ವವಿರುತ್ತದೆ. ಸೋಮಾರಿ ತನ್ನ ದುಃಖ ಮುಂತಾದವುಗಳಲ್ಲಿರುವ ಮನಃಸ್ಥಿತಿಗೆ `ಮೂಢ' ಎನ್ನಲ್ಪಡುತ್ತದೆ. ಕ್ಷಿಪ್ತದಂತೆ ಮೂಢ ಸ್ಥಿತಿಯಲ್ಲಿರುವ ಮನಸ್ಸಿನಿಂದ ಪರವಸ್ತುವನ್ನು ಪಡೆಯಲು ಸಾಧ್ಯವಿಲ್ಲ. ಆದ್ದರಿಂದ ಮನಸ್ಸನ್ನು ಸ್ವಲ್ಪವಾದರೂ ಶುದ್ಧಪಡಿಸಿಕೊಳ್ಳಬೇಕೆಂದು ಹಿಂದಿನವರು ಹೇಳಿದರು. ಹೀಗೆ, ಸ್ವಲ್ಪವೇ ಶುದ್ಧವಾದ ಮನಸ್ಸಿನ ಸ್ಥಿತಿ ವಿಕ್ಷಿಪ್ತವೆನ್ನಲ್ಪಡುವುದು. ಯಾವುದೋ ಒಂದು ವಿಕ್ಷೇಪವಿರುವುದರಿಂದ ಇದಕ್ಕೆ `ವಿಕ್ಷಿಪ್ತ'ವೆನ್ನುವ ಹೆಸರು ಬಂದಿದೆಯೆಂದೂ ಹೇಳಬಹುದು. ಇಂಥ ಮನಸ್ಸು, ಯಾವುದರ ಬಗ್ಗೆ ಚಿಂತಿಸಬೇಕೋ ಅದರ ಬಗ್ಗೆ ಚಿಂತನೆ ಮಾಡಲು ಯೋಗ್ಯವಾದುದು. ಆದರೆ, ಮಧ್ಯೆ ಮಧ್ಯೆ ಬೇರೆ ಯಾವುದಾದರೂ ವಿಚಾರ ಬರುತ್ತಿರುತ್ತದೆ. ನಾವು ಧ್ಯಾನ ಮಾಡುತ್ತಿರುತ್ತೇವೆ. ಆದರೆ ಮಧ್ಯೆ ಮಧ್ಯೆ ಆಫೀಸಿನ ವಿಷಯ ನೆನಪಿಗೆ ಬರುತ್ತದೆ. ಹಾಗೆ ನೆನಪಿಗೆ ಬಂದರೂ ಮತ್ತೆ ಮತ್ತೆ ಧ್ಯಾನ ಮಾಡುತ್ತೇವೆ. ಹೀಗೆ ಮಧ್ಯೆ ಮಧ್ಯೆ ಬೇರೆ ವಿಷಯಗಳು ಬರುತ್ತಿದ್ದರೂ ಈ ಸ್ಥಿತಿ ಕ್ಷಿಪ್ತ ಮತ್ತು ಮೂಢಕ್ಕಿಂತಲೂ ಮೇಲಾದುದು. ಹೀಗಿದ್ದರೂ ವಿಕ್ಷಿಪ್ತ ಸ್ಥಿತಿಯಲ್ಲಿರುವ ಮನಸ್ಸು ಪರವಸ್ತುವನ್ನು ಪಡೆಯಲು ಉಪಯೋಗವಾಗುವುದಿಲ್ಲ. ಅದಕ್ಕೆ ಕಾರಣ, ಹೀಗೆ ಧ್ಯಾನಿಸಲ್ಪಡುವ ವಿಷಯಕ್ಕಿಂತಲೂ ಬೇರೆ ವಿಷಯಗಳಲ್ಲಿ ವಿಕ್ಷಿಪ್ತತೆಯುಳ್ಳ ಮನಸ್ಸು ತನ್ನ ಕೆಲಸವನ್ನು ಸಾಧಿಸಲು ಸಾಮರ್ಥ್ಯವಿಲ್ಲದುದೇ ಆಗುತ್ತದೆ. ಇಂಥ ಮನಸ್ಸು ಒಂದು ದಿನ ಧ್ಯಾನ ಮಾಡುತ್ತದೆ, ಮತ್ತೊಂದು ದಿನ ಧ್ಯಾನ ಮಾಡುವುದನ್ನು ಬಿಟ್ಟು ಬೇರೆ ಏನಾದರೂ ಮಾಡುವುದು. ಎಂಥ ಮನಸ್ಸು ಪರವಸ್ತುವನ್ನು ಪಡೆಯಲು ಉಪಯೋಗವಾಗುತ್ತದೆ ಎನ್ನುವುದಕ್ಕೆ ಉತ್ತರವಾಗಿ ಏಕಾಗ್ರ ಸ್ಥಿತಿಯಲ್ಲಿರುವ ಮನಸ್ಸು ಉಪಯೋಗಪಡುತ್ತದೆ ಎನ್ನುತ್ತಾರೆ-ಕೈವಲ್ಯೋಪನಿಷತ್ತಿನಲ್ಲಿ,

\begin{shloka}
`ಉಮಾಸಹಾಯಂ ಪರಮೇಶ್ವರಂ ಪ್ರಭುಂ ತ್ರಿಲೋಚನಂ ನೀಲಕಂಠಂ ಪ್ರಶಾಂತಮ್|\\
ಧ್ಯಾತ್ವಾ ಮುನಿರ್ಗಚ್ಛತಿ ಭೂತಯೋನಿಂ ಸಮಸ್ತಸಾಕ್ಷಿಣಂ ತಮಸಃ ಪರಸ್ತಾತ್||
\end{shloka}

-ಎಂದಿದೆ. ದೇವಿ (`ಲಲಿತಾಸಹಸ್ರನಾಮ'ದಲ್ಲಿ) `ಅಂತರ್ಮುಖ ಸಮಾರಾಧ್ಯಾ' ಮತ್ತು `ಬಹಿರ್ಮುಖ ಸುದುರ್ಲಭಾ' ಎಂದು ಸ್ತುತಿಸಲ್ಪಟ್ಟಿದ್ದಾಳೆ. ಅದೇ ರೀತಿ ಭಗವಂತನನ್ನು ಅಂತರ್ಮುಖ ಮನಸ್ಸಿನಿಂದ ಧ್ಯಾನ ಮಾಡಬೇಕು.

ಈ ಸಂದರ್ಭದಲ್ಲಿ ನನಗೆ ನನ್ನ ಆಚಾರ್ಯರ (ಶ್ರೀ ಶ್ರೀಚಂದ್ರಶೇಖರ ಭಾರತೀ ಮಹಾಸ್ವಾಮಿಗಳವರ) ಬಗ್ಗೆ ನೆನಪು ಬರುತ್ತದೆ. ಅವರು ಸಂನ್ಯಾಸವನ್ನು ತೆಗೆದುಕೊಳ್ಳುವುದಕ್ಕೆ ಮುಂಚೆ,

\begin{shloka}
`ಪೂರ್ವಾಭ್ಯಾಸೇನ ತೇನೈವ ಹ್ರಿಯತೇ ಹ್ಯವಶೋಽಪಿ ಸಃ'
\end{shloka}

(ತನ್ನ ಹಿಂದಿನ ಜನ್ಮದ ಅಭ್ಯಾಸದಿಂದ ತಾನು ಇಷ್ಟಪಡದಿದ್ದರೂ ಒಳ್ಳೆಯ ದಾರಿಯಲ್ಲಿ ಎಳೆಯಲ್ಪಡುತ್ತಾನೆ) ಎಂದು ಭಗವಂತನು ಗೀತೆಯಲ್ಲಿ ಹೇಳುವಂತೆ ಅವರು ಪಾಠಶಾಲೆಯಲ್ಲಿ ಓದುವಾಗ ಅವರದೊಂದು ಸ್ವಭಾವವಿತ್ತು. ಪ್ರತಿ ಪ್ರದೋಷದಲ್ಲೂ ವಿದ್ಯಾಶಂಕರ ದೇವಾಲಯದಲ್ಲಿರುವ ಒಂದು ಕಂಬದ ಹತ್ತಿರ ಕುಳಿತು. `ರತ್ನೈಃ ಕಲ್ಪಿತಮಾಸನಂ' (ಭಗವಂತನಿಗೆ ರತ್ನಗಳಿಂದ ಕಲ್ಪಿತವಾದ ಆಸನವನ್ನು ಸಮರ್ಪಿಸುವೆನು) ಎಂದು ಹೇಗೆ `ಶಿವಮಾನಸಪೂಜಾ ಸ್ತೋತ್ರ' ದಲ್ಲಿ ಹೇಳಿದೆಯೋ ಹಾಗೆಯೇ ಭಗವಂತನಿಗೆ ಆಸನ, ಅರ್ಘ್ಯ, ಪಾದ್ಯ ಮುಂತಾದವನ್ನು ಸಮರ್ಪಣೆ ಮಾಡಿ ನೈವೇದ್ಯ, ನೀರಾಜನ ಮಾಡಿ, ಮಂತ್ರ ಪುಷ್ಪ ಸಮರ್ಪಿಸಿ ಒಂದೇ ವಿಧವಾದ ಚಿತ್ತದಿಂದ ಮಾನಸಿಕವಾಗಿ ಪೂಜೆ ಮಾಡುವ ಪದ್ಧತಿ ಅವರಿಗಿತ್ತು. ಪೂಜೆ ಮುಗಿಯುವವರೆಗೆ ಅದರಲ್ಲೇ ಅವರ ಮನಸ್ಸು ತೊಡಗಿರುತ್ತಿತ್ತು. ಆ ವಿಷಯವನ್ನು ಕುರಿತು ಅವರು ನನಗೆ ಹೇಳಿದ್ದರು. ಇದು ಮನಸ್ಸಿನ ಏಕಾಗ್ರತೆಗೆ ನಿದರ್ಶನ. ಮೇಲೆ ಹೇಳಿದ ಕೈವಲ್ಯೋಪನಿಷತ್ತಿನಲ್ಲಿ ಹೇಳಲ್ಪಟ್ಟ `ಉಮಾಸಹಾಯಂ' ಎನ್ನುವುದನ್ನು ಕುರಿತು ಚಿಂತಿಸೋಣ. ನಾವು ಸಗುಣನಾದ ಭಗವಂತನನ್ನು ಚಿಂತಿಸುವಾಗ `ಉಮಾಸಹಾಯಂ ಪರಮೇಶ್ವರಂ' ಎಂದು ಹೇಳಿದ ಪ್ರಕಾರ ಶಕ್ತಿಯೊಡನೆ ಪರಮೇಶ್ವರನನ್ನು ಚಿಂತಿಸುವುದು ವಿಶೇಷ. ಸಕಲ ಜಗತ್ತಿಗೆ ನಾಯಕನಾಗಿರುವುದರಿಂದ `ಪ್ರಭು' ಎನ್ನುತ್ತಾರೆ. `ತ್ರಿಲೋಚನಂ' - ಭಗವಂತನಿಗೆ ಮೂರು ಕಣ್ಣುಗಳು. ಧ್ಯಾನವೇ ಮೂರನೆಯ ಕಣ್ಣು. ಅದನ್ನು ತಿಳಿಸುವುದಕ್ಕಾಗಿಯೇ ಭಗವಂತನು ಮೂರು ಕಣ್ಣುಗಳುಳ್ಳವನೆಂದು ನಾವು ಹೇಳುತ್ತೇವೆ. `ನೀಲಕಂಠಂ' - ಲೋಕಕ್ಕೆ ಒಂದು ದೊಡ್ಡ ಆಪತ್ತು ಉಂಟಾದಾಗ ಭಗವಂತನು ನಮ್ಮನ್ನು ಕಾಪಾಡಿದನು. `ನಗಿಲಿತಂ ನೋದ್ಗೀರ್ಣಮೇವ ತ್ವಯಾ' (ಆ ವಿಷವನ್ನು ಬಾಯಿಂದ ಹೊರಗೂ ಬಿಡಲಿಲ್ಲ, ನುಂಗಲೂ ಇಲ್ಲ) ಎಂದು ಹೇಳಿ - 

\begin{shloka}
`ಜ್ವಾಲೋಗ್ರಃ ಸಕಲಾಮರಾತಿಭಯದಃ ಕ್ಷ್ವೇಲಃ ಕಥಂ ವಾ ತ್ವಯಾ\\
ದೃಷ್ಟಃ ಕಿಂ ಚ ಕರೇ ಧೃತಃ ಕರತಲೇ ಕಿಂ ಪಕ್ವಜಂಬೂಫಲಮ್|\\
ಜಿಹ್ವಾಯಾಂ ನಿಹಿತಶ್ಚ ಸಿದ್ಧಘುಟಿಕಾ ವಾ ಕಂಠದೇಶೇ ಭೃತಃ\\
ಕಿಂ ತೇ ನೀಲಮಣಿರ್ವಿಭೂಷಣಮಯಂ ಶಂಭೋ ಮಹಾತ್ಮನ್ ವದ||'
\end{shloka}

(ಆ ಜ್ವಲಿಸುವ, ಎಲ್ಲ ದೇವತೆಗಳಿಗೂ ಭಯವನ್ನುಂಟುಮಾಡುವ ವಿಷ ನಿನ್ನಿಂದ ಹೇಗೆ ನೋಡಲಾಯಿತು? ನೇರಳೆ ಹಣ್ಣಿನಂತೆ ಇದನ್ನು ನೀನು ಕೈಯಲ್ಲಿಟ್ಟುಕೊಂಡೆಯಾ? ಸಿದ್ಧ ಔಷಧಿಯಂತೆ ಬಗೆದು ನಾಲಿಗೆಯ ಮೇಲೆ ಈ ವಿಷವನ್ನು ಹಾಕಿಕೊಂಡೆಯಾ? ಎಲೈ ಶಂಬುವೇ! ಮಹಾತ್ಮನೇ! ವಿಷವನ್ನು ನೀಲಿ ಆಭರಣದಂತೆ ಕಂಠದಲ್ಲಿ ಧರಿಸಿದೆಯಾ? ಹೇಳು.) - ಈ ವಿಧವಾಗಿ ಭಗವತ್ಪಾದರು ನೀಲಕಂಠನನ್ನು ಸ್ತುತಿಸಿದರು. ಒಮ್ಮೆ ದೇವತೆಗಳೂ ರಾಕ್ಷಸರು ಸಮುದ್ರವನ್ನು ಕಡೆಯಲು ತೊಡಗಿದರು. ಒಂದು ದೊಡ್ಡ ವಿಚಾರಕ್ಕೆ ಕುಳಿತುಕೊಂಡರೆ ಮೊದಲು ಏನೋ ತೋರಿ ಕೊನೆಯಲ್ಲಿ ಸಿದ್ಧಾಂತ ತೀರ್ಮಾನವಾಗುವಂತೆ ಸಮುದ್ರವನ್ನು ಕಡೆಯುವಾಗ ಮೊದಲು ವಿಷ ಬಂದು ಬಿಟ್ಟಿತು. ಏನು ಮಾಡಬೇಕೆಂದು ಯಾರಿಗೂ ತೋರಲಿಲ್ಲ. ಆ ವಿಷದಿಂದ ದೊಡ್ಡ ಜ್ವಾಲೆ ಉಂಟಾಯಿತು. ಆಗ ಭಗವಂತನು ಬಂದು ಅದನ್ನು ತೆಗೆದುಕೊಂಡು ಬಾಯಿಯಲ್ಲಿ ಹಾಕಿಕೊಂಡುಬಿಟ್ಟನು. ಅದನ್ನು ನುಂಗದೆ ತನ್ನ ಗಂಟಲಿನಲ್ಲೇ ಇಟ್ಟುಕೊಂಡನು. ಪುರಾಣ ಹೇಳುವವರು `ಭಗವಂತನಲ್ಲಿ ಒಳಗೂ ಹೊರಗೂ ಬ್ರಹ್ಮಾಂಡವಿದೆ. ಆ ವಿಷವನ್ನು ಹೊರಗಿಟ್ಟರೆ ಹೊರಗಿನ ಬ್ರಹ್ಮಾಂಡ ಸುಟ್ಟು ಹೋಗುತ್ತದೆ, ಒಳಗಿಟ್ಟರೆ ಒಳಗಿರುವ ಬ್ರಹ್ಮಾಂಡ ಸುಟ್ಟು ಹೋಗುತ್ತದೆ. ಆದ್ದರಿಂದಲೇ ಗಂಟಲಿನಲ್ಲೇ ಇಟ್ಟುಕೊಂಡನು' ಎಂದು ಕಥೆ ಹೇಳುತ್ತಾರೆ.

`ನೀಲಕಂಠಂ ಪ್ರಶಾಂತಮ್' - ನೀಲಕಂಠನೆಂದು ನಾವು ನೆನೆಯುತ್ತಲೇ ಭಗವಂತನು ಎಲ್ಲಾ ಲೋಕಗಳನ್ನೂ ಕಾಪಾಡುವನೆಂದು ಜ್ಞಾನವಾಗುತ್ತದೆ.

`ಧ್ಯಾತ್ವಾ ಮುನಿರ್ಗಚ್ಚತಿ ಭೂತಯೋನಿಂ' - ಹಾಗೆ ಯಾರು ಚಿಂತನೆ ಮಾಡುವವರು, `ತಮಸಃ ಪರಸ್ತಾತ್ ಭೂತಯೋನಿಂ' ಎಂದು ಹೇಳಿದಂತೆ ಜ್ಯೋತಿರ್ಮಯವಾಗಿರುವ ಈ ಸಕಲ ಜಗತ್ತಿಗೂ ಕಾರಣನಾದ ಪರಮೇಶ್ವರನನ್ನು ಪಡೆಯುತ್ತಾರೆ.

`ಮಾಮುಪೇತ್ಯ ತು ಕೌಂತೇಯ ಪುನರ್ಜನ್ಮ ನ ವಿದ್ಯತೇ' ಎಂದು ಹೇಳಿದಂತೆ ಅವನನ್ನು ಪಡೆದ ಮೇಲೆ ಮತ್ತೆ ಜನ್ಮವಿಲ್ಲ. `ಧ್ಯಾತ್ವಾಮುನಿಃ' (ಮುನಿ ಧ್ಯಾನವನ್ನು ಮಾಡಿ)-ಧ್ಯಾನವೆಂದರೇನು?

ಕ್ಷಿಪ್ತ, ಮೂಢ, ವಿಕ್ಷಿಪ್ತ - ಈ ಮೂರು ಚಿತ್ತವನ್ನೂ ಇಟ್ಟುಕೊಂಡು ನಾವು ಧ್ಯಾನ ಮಾಡಲಾಗುವುದಿಲ್ಲವೆಂದು ನೋಡಿದೆವು. ಆದ್ದರಿಂದ ಧ್ಯಾನ ಮಾಡುವುದಕ್ಕೆ ಶಾಸ್ತ್ರದಲ್ಲಿ ಏಕಾಗ್ರವಾದ ಚಿತ್ತವಿರಬೇಕೆಂದು ಹೇಳಲ್ಪಟ್ಟಿದೆ. ಏಕಾಗ್ರವೆಂದರೆ - `ವಿಜಾತೀಯ ಪ್ರತ್ಯಯ ತಿರಸ್ಕೃತ ಸಜಾತೀಯ ಪ್ರತ್ಯಯ ಪ್ರವಾಹಃ ಧ್ಯಾನಮ್' ಎಂದು ನಾವು ಸಾಮಾನ್ಯವಾಗಿ ಹೇಳುತ್ತೇವೆ.

`ಪ್ರೋಷಿತಭರ್ತೃಕಾ ಪತಿಂ ಧ್ಯಾಯತೀವ' ಎಂದು ಒಬ್ಬರು ಹೇಳಿದರು. ಗಂಡ ಊರಿನಲ್ಲಿಲ್ಲ. ಹೆಂಡತಿಗೆ ಗಂಡನ ಮೇಲೆ ಪ್ರೀತಿ. ಅವಳು ತನ್ನ ಗಂಡನನ್ನು ಕುರಿತೇ ಚಿಂತಿಸುತ್ತಾಳೆ. ಹೇಗೆ ಚಿಂತಿಸುತ್ತಾಳೆ ಎಂದರೆ ನಾವು ಶಾಕುಂತಲ ನಾಟಕದಲ್ಲಿ ನೋಡಿದರೆ ಗೊತ್ತಾಗುತ್ತದೆ. ದುರ್ವಾಸರ ಶಾಪ ಏತಕ್ಕೆ ಶಕುಂತಲೆಗೆ ಬಂದಿತು? ತನ್ನ ಗಂಡನನ್ನು ಕುರಿತು ಯೋಚಿಸುತ್ತಿದ್ದ ಶಕುಂತಲೆ ದುರ್ವಾಸರನ್ನು ನೋಡಲಿಲ್ಲ. ಆಗ ದುರ್ವಾಸರು, `ನೀನು ಯಾರನ್ನು ಕುರಿತು ಯೋಚಿಸುತ್ತಿದ್ದು ಎದುರಿಗೆ ಬಂದ ಋಷಿಯನ್ನು ಗೌರವಿಸಲಿಲ್ಲವೋ, ಅವನು ನಿನ್ನನ್ನು ಮರೆತುಬಿಡಲಿ' ಎಂದು ಶಪಿಸಿದರು.

ಹೀಗೆ ಏಕಾಗ್ರಚಿತ್ತವಿದ್ದರೆ ಎದುರಿಗೆ ಯಾವುದಾದರೂ ವಸ್ತುವಿದ್ದರೂ ಅದು ಕಣ್ಣಿಗೆ ತೋರುವುದಿಲ್ಲ. ಅದಕ್ಕೆ ಶಾಸ್ತ್ರದಲ್ಲಿ `ವಿಜಾತೀಯ ಪ್ರತ್ಯಯ ತಿರಸ್ಕೃತ' ಎಂದು ಹೇಳುತ್ತಾರೆ. ಭಗವಂತನ ರೂಪದ ಜ್ಞಾನವಿರುವಾಗ ಬೇರೆ ಯಾವ ವಿಧವಾದ ಜ್ಞಾನವೂ ಇರುವುದಿಲ್ಲ. ಒಂದು ಚಿತ್ರವನ್ನು ನೋಡಿದರೆ, ಆ ಚಿತ್ರ ಮನಸ್ಸಿನಲ್ಲಿ ನಾಟಿಕೊಂಡಿರುವಾಗ ಅದು ಮಾತ್ರ ಮನಸ್ಸಿನಲ್ಲಿರುತ್ತದೆಯೇ ಹೊರತು ಬೇರೆ ವಿಚಾರವೇ ಇರುವುದಿಲ್ಲ. ಇದನ್ನೇ ಶಾಸ್ತ್ರದಲ್ಲಿ `ವಿಜಾತೀಯ ಪ್ರತ್ಯಯ ತಿರಸ್ಕೃತ ಸ್ವಜಾತೀಯ ಪ್ರತ್ಯಯ ಪ್ರವಾಹಃ' ಎಂದು ಹೇಳಿರುವುದು. ಇದು ಧ್ಯಾನವಾಗುತ್ತದೆ.

ಉಪಾಸನೆಯ ಉಚ್ಚಸ್ಥಿತಿ ಯೋಗವೆನ್ನಲ್ಪಡುತ್ತದೆ. ಈಗ ಯೋಗವನ್ನು ಕುರಿತು ನೋಡೋಣ. ಮಹರ್ಷಿ ಪತಂಜಲಿ ತಮ್ಮ ಯೋಗಸೂತ್ರದಲ್ಲಿ `ಯೋಗಶ್ಚಿತ್ತವೃತ್ತಿ ನಿರೋಧಃ' ಎಂದಿದ್ದಾರೆ. ಚಿತ್ತ ವೃತ್ತಿಯನ್ನು ನಿರೋಧಿಸುವುದು (ತಡೆಯುವುದು) ಯೋಗ. ಯೋಗಕ್ಕೆ ಸಮಾಧಿ ಎಂದು ಅರ್ಥ. ಸಮಾಧಿ-ಸಂಪ್ರಜ್ಞಾತ ಸಮಾಧಿ. ಅಸಂಪ್ರಜ್ಞಾತ ಸಮಾಧಿ ಎಂದು ಎರಡು ಬಗೆ. ಏಕಾಗ್ರ ಮನಃಸ್ಥಿತಿಗೆ ಸಂಬಂಧಪಟ್ಟಿದ್ದು ಸಂಪ್ರಜ್ಞಾತ ಸಮಾಧಿ. ನಿರುದ್ಧ ಸ್ಥಿತಿಗೆ ಸಂಬಂಧಪಟ್ಟಿದ್ದು ಅಸಂಪ್ರಜ್ಞಾತ ಸಮಾಧಿ. ಮೊದಲು ಸಂಪ್ರಜ್ಞಾತ ಸಮಾಧಿಯನ್ನು ಕುರಿತು ಚಿಂತಿಸೋಣ. ಯಾವುದಾದರೂ ಆಲಂಬನ (ಆಧಾರ)ವನ್ನು ಇಟ್ಟುಕೊಂಡು ಅದರ ಮೇಲೆ ಮನಸ್ಸನ್ನು ಸ್ಥಿರಪಡಿಸಿದರೆ ಅದು ಸಂಪ್ರಜ್ಞಾತ ಯೋಗವಾಗುತ್ತದೆ. ಹೀಗೆ ಏಕಾಗ್ರತೆಯಿಂದ ಕೂಡಿದ ಮನಸ್ಸಿನೊಡನೆ ಮಾಡುವ ಧ್ಯಾನದಲ್ಲಿ ಒಂದು ವಿಶೇಷವಿದೆ. ಇದನ್ನು ವಿವರಿಸುವುದಕ್ಕೆ ಮೊದಲು `ತ್ರಿಪುಟಿ ಭಾನಂ' ಎನ್ನುವುದನ್ನು ಕುರಿತು ಗಮನ ಸೆಳೆಯಬೇಕು. ಜ್ಞಾನಂ, ಜ್ಞೇಯಂ, ಜ್ಞಾತಾ ಎಂದು ಮೂರು ತೋರುವಿಕೆಗೆ `ತ್ರಿಪುಟಿ ಭಾನಂ' ಎಂದು ಹೆಸರು. ಜ್ಞಾನವೆನ್ನುವುದು ಅರಿವು, ಜ್ಞಾನದ ವಸ್ತು ಜ್ಞೇಯ, ಈ ಜ್ಞಾನಕ್ಕೆ ಆಧಾರವಾಗಿರುವವನು ಜ್ಞಾತಾ. ನಾವು ಮತ್ತೊಬ್ಬನೊಡನೆ ಮಾತನಾಡುವಾಗ `ನಾನು ಇದ್ದೇನೆ' ಎನ್ನುವ ತೀರ್ಮಾನ ನಮಗೆ ಯಾವಾಗಲೂ ಇರುತ್ತದೆ. ಏಕೆಂದರೆ `ನಾನು ಇದ್ದೇನೆ' ಎನ್ನುವ ಜ್ಞಾನ ಸೂಕ್ಷ್ಮವಾಗಿ ಯಾವಾಗಲೂ ಇದ್ದೇ ಇರುತ್ತದೆ. ಅದರಿಂದ ಆತ್ಮನು ತಾನೇ ಪ್ರಕಾಶಿಸುತ್ತಾನೆಂದು ತಿಳಿಯಬಹುದು. ಜ್ಞಾನವಿರುವಾಗ ವಿಷಯ ತೋರುತ್ತದೆ. ಇಲ್ಲಿ ಜ್ಞೇಯವನ್ನು ವಿಷಯವಾಗುಳ್ಳದ್ದು ಜ್ಞಾನ. ಧ್ಯಾನ ಮಾಡುವಾಗ `ಧ್ಯಾನ ಮಾಡುತ್ತೇನೆ' ಎನ್ನುವ ಅರಿವು ಇದ್ದೇ ಇರುತ್ತದೆ, ಧ್ಯಾನ ಮಾಡುವಾಗ ಈ ತ್ರಿಪುಟಿ ಭಾನವಿದ್ದು ಧ್ಯಾನದ ಉಚ್ಚಸ್ಥಿತಿಯಲ್ಲಿ ಬರುವ ಈ ಏಕಾಗ್ರವೃತ್ತಿ ಸಂಪ್ರಜ್ಞಾತ ಸಮಾಧಿ ಎನ್ನುತ್ತೇವೆ. ಭಗವಂತನನ್ನು ಚಿಂತಿಸುವಾಗ `ನಾನು ಇದ್ದೇನೆ' ಎಂದೂ `ಭಗವಂತನನ್ನು ಚಿಂತಿಸುತ್ತಿದ್ದೇನೆ' ಎಂದೂ ಸೂಕ್ಷ್ಮವಾಗಿ ತೋರುತ್ತಿರುತ್ತದೆ. ಹೀಗೆ ಮಾಡುವ ಧ್ಯಾನದ ಉತ್ತಮವಾದ ಸ್ಥಿತಿಗೆ ಸಂಪ್ರಜ್ಞಾತ ಸಮಾಧಿಯೆಂದು ಹೆಸರು. ಏಕಾಗ್ರ ಮನಃಸ್ಥಿತಿಗೂ ಮೇಲಾದ ಸ್ಥಿತಿ ನಿರುದ್ಧ. ಈ ಸ್ಥಿತಿಯಲ್ಲಿ, `ನಾನು ಜ್ಞಾನವನ್ನು ಅನುಭವಿಸುತ್ತಿದ್ದೇನೆ' ಎಂದು ಈ ವಿಧವಾಗಿ ತ್ರಿಪುಟಿ ಭಾನವಿಲ್ಲದೆ ಕೇವಲ ವಸ್ತ್ವಾಕಾರ ವಿಷಯ ಜ್ಞಾನ (ಧ್ಯಾನ ಮಾಡುವ ವಸ್ತುವಿನ ಬಗ್ಗೆ ಇರುವ ಅರಿವು) ವಿದ್ದು ಅದು ಅಸಂಪ್ರಜ್ಞಾತ ಸಮಾಧಿ ಎನ್ನಲಾಗುವುದು. ನಾವು ಇದನ್ನು ಅನುಭವದಲ್ಲಿ ಕಾಣಬಹುದು. ಕೆಲವೊಮ್ಮೆ ಈ ಜ್ಞಾನ ಬಂದಿದೆಯೆಂದು ತೋರುತ್ತದೆ. ಕೆಲವೊಮ್ಮೆ ಜ್ಞಾನ ಬರುತ್ತದೆಯೆಂದು ತೋರುತ್ತದೆ. ಇದಕ್ಕೂ ಸುಷುಪ್ತಿಗೂ ಏನು ಭೇದವೆಂದು ಕೇಳಿದರೆ ಸುಷುಪ್ತಿಯಲ್ಲಿ ಅಜ್ಞಾನವಿರುತ್ತದೆ, ವಿಷಯವಿಲ್ಲ. ಅಸಂಪ್ರಜ್ಞಾತ ಸಮಾಧಿಯಲ್ಲಿ ವಿಷಯ ತೋರುತ್ತಿರುತ್ತದೆ, ಆದರೆ ವಿಷಯ ತೋರುತ್ತಿದೆ ಎನ್ನುವುದು ಮಾತ್ರ ತೋರುವುದಿಲ್ಲ. ಅಂದರೆ ತರ್ಕದಲ್ಲಿ ಹೇಳುವ ಅನುವ್ಯವಸಾಯವೆನ್ನುವುದು ಇಲ್ಲ. ಹೀಗೆ ಸಮಾಧಿಯನ್ನು ಸಂಪ್ರಜ್ಞಾತ ಸಮಾಧಿ, ಅಸಂಪ್ರಜ್ಞಾತ ಸಮಾಧಿಯೆಂದು ಎರಡು ವಿಧವಾಗಿ ವಿಭಜಿಸಿ, ಏಕಾಗ್ರವಾದ ಮನಸ್ಸಿನಿಂದ ಚಿಂತನೆ ಮಾಡುವಾಗ ಸಂಪ್ರಜ್ಞಾತ ಸಮಾಧಿ ಎನ್ನುವ ನಿರುದ್ಧ ಸ್ಥಿತಿಯಲ್ಲಿ ಅಸಂಪ್ರಜ್ಞಾತ ಸಮಾಧಿಯನ್ನು ಹೇಳಿದರು. ಚಿತ್ತ ವೃತ್ತಿನಿರೋಧವೆಂದರೆ ಏನು? ಚಿತ್ತಕ್ಕೆ ಆತ್ಮನಾದಿಯಾಗಿ ಯಾವ ವಿಷಯವೂ ಇಲ್ಲದೆ ನಿರೋಧವೆನ್ನುವುದು ಇಲ್ಲ. ಚಿತ್ತಕ್ಕೆ ವಿಷಯವಿದೆ. ಆದರೆ ಅದು ಬೇರೆ ವಿಷಯಗಳಲ್ಲಿ ಚಲನವಿಲ್ಲದೆ ಒಂದೇ ವಿಷಯದಲ್ಲಿರುತ್ತದೆ ಎನ್ನುವುದನ್ನೇ `ಯೋಗಶ್ಚಿತ್ತವೃತ್ತಿ ನಿರೋಧಃ' ಎಂದು ಹೇಳಿದ್ದಾರೆ. ಈ ವಿಧವಾಗಿ ಮನಸ್ಸನ್ನು ವಶದಲ್ಲಿಟ್ಟುಕೊಂಡು ಈಶ್ವರನನ್ನು ಧ್ಯಾನಿಸುತ್ತೇವೆ.

ಜಪ ಮಾಡುವಾಗ ಮನಸ್ಸು ಏಕಾಗ್ರವಾಗಿರುವುದಲ್ಲದೆ ನಿರುದ್ಧವಾಗಿರುವುದಿಲ್ಲ. ಮಂತ್ರಜಪದಲ್ಲಿ ಮಂತ್ರದ ವಸ್ತುಜ್ಞಾನವೂ, ಶಬ್ದಜ್ಞಾನವೂ ಇರುತ್ತದೆ. ಪ್ರಣವ ಗಾಯತ್ರಿಯಂಥ ಮಂತ್ರಗಳನ್ನು ಮನಸ್ಸು ಉಚ್ಚರಿಸಿಕೊಂಡಿರುವುದಕ್ಕೆ `ಶಬ್ದಾನುವಿದ್ದಮ್' ಎಂದು ಹೆಸರು. ಕೆಲವೊಮ್ಮೆ ಭಗವಂತನ ರೂಪವನ್ನು ಚಿಂತಿಸುತ್ತೇವೆ. ಅದು `ರೂಪಾನುವಿದ್ಧಮ್' ಎನ್ನಲ್ಪಡುತ್ತದೆ. ಈ ಎರಡೂ ಇಲ್ಲದೆ ನಿರ್ಗುಣ ಬ್ರಹ್ಮವನ್ನು ಚಿಂತನೆ ಮಾಡಲು ಸಾಧ್ಯವೇ ಎನ್ನುವ ಪ್ರಶ್ನೆಗೆ, ಸಾಧ್ಯ ಎನ್ನುವುದೆ ಸಿದ್ಧಾಂತವಾಗುತ್ತದೆ. ನಿರ್ಗುಣ ಚಿಂತನೆಯಲ್ಲಿ ಜ್ಞಾನ ಬರುವುದೇ ಹೊರತು ಸಮಾಧಿ ಹೇಗೆ ಬರುತ್ತದೆಯೆಂದು ಕೇಳಿದರೆ, ಬರುತ್ತದೆಯೆಂದೇ ಉತ್ತರ ಹೇಳಿದ್ದಾರೆ, ಜ್ಞಾನ ಬೇರೆ, ಧ್ಯಾನ ಬೇರೆ. ಧ್ಯಾನವೆಂದರೆ ಪರಬ್ರಹ್ಮ ಸಾಕ್ಷಾತ್ಕಾರವಾಗಿ ಧ್ಯಾನಾತಿಶಯದಲ್ಲಿ ಒಂದು ಸ್ಫುರಣೆಯಾಗುತ್ತದೆ. ಆ ಸ್ಫುರಣೆಯಾದರೆ ಮತ್ತೆ ಧ್ಯಾನ ಮಾಡಬೇಕಾಗಿಲ್ಲ. ಆ `ಸ್ಫುರಣೆ' ಹಾಗೆಯೇ ಇದ್ದು ಬಿಡುತ್ತದೆ. ಧ್ಯಾನ ಮಾಡಬೇಕೆನ್ನುವ ತೀರ್ಮಾನವನ್ನು ಮನಸ್ಸಿನಲ್ಲಿಟ್ಟುಕೊಂಡು ಚಿಂತನೆ ಮಾಡುವುದು ಧ್ಯಾನವಾಗುತ್ತದೆ. ಹೀಗೆ ಮಾಡುವಾಗ ಮನಸ್ಸಿನಲ್ಲಿ ಪ್ರಕಾಶ ತೋರುವವರೆಗೆ ಪ್ರಯತ್ನ ಮಾಡಬೇಕು. ಪ್ರಕಾಶ ತೋರಿದ ಮೇಲೆ ವಸ್ತು ತಾನಾಗಿಯೇ ತೋರುತ್ತವೆ. ಹಾಗಿರುವುದು ಅಸಂಪ್ರಜ್ಞಾತ ಸಮಾಧಿ ಆಗುತ್ತದೆ. ನಾವು ಅಸಂಪ್ರಜ್ಞಾತ ಸಮಾಧಿಯಲ್ಲಿದ್ದುಕೊಂಡು ಬ್ರಹ್ಮ ಸಾಕ್ಷಾತ್ಕಾರವನ್ನು ಪಡೆಯಬೇಕು. ಈ ಸ್ಥಿತಿಯನ್ನು ಪಡೆದರೆ, 

\begin{shloka}
`ಯೋಗಾಭ್ಯಾಸವಶೀಕೃತೇನ ಮನಸಾ ತನ್ನಿರ್ಗುಣಂ ನಿಷ್ಕ್ರಿಯಮ್|\\
ಜ್ಯೋತಿಃ ಕಿಂಚನ ಯೋಗಿನೋ ಪಶ್ಯನ್ತಿ.......||'
\end{shloka}

(ಯೋಗಾಭ್ಯಾಸದಿಂದ ಮನಸ್ಸನ್ನು ವಶಪಡಿಸಿಕೊಂಡು ಯೋಗಿಗಳು ನಿರ್ಗುಣವೂ, ನಿಷ್ಕ್ರಿಯವೂ ಆದ ಜ್ಯೋತಿಯನ್ನು ಕಾಣುತ್ತಾರೆ) ಎಂದು ಹೇಳಿದಂತೆ ಬೇರೆ ಎಲ್ಲವನ್ನೂ ಬಿಟ್ಟುಬಿಡುತ್ತಾರೆ. ಹೀಗೆ ನಿರುದ್ಧವಾದ ಮನಸ್ಸಿನಿಂದ ಧ್ಯಾನಮಾಡಲು ಸಾಧ್ಯವಾಗದೇ ಹೋದರೆ ಮನಸ್ಸನ್ನು ಏಕಾಗ್ರವನ್ನಾಗಿ ಮಾಡಿಕೊಂಡು ಪರಮಾತ್ಮನನ್ನು ಚಿಂತಿಸಿದರೆ ಅದಕ್ಕೆ ಉಪಾಸನೆ ಎನ್ನುತ್ತಾರೆ.

ಈ ಉಪಾಸನೆ ಸಿದ್ಧಿಸಬೇಕಾದರೆ ಯಮ, ನಿಯಮಗಳು ಅವಶ್ಯಕ. ಈ ಯಮ-ನಿಯಮಗಳಿಲ್ಲದೆ `ನಾನು ಧ್ಯಾನ ಮಾಡುತ್ತಿದ್ದೇನೆ, ನಾನು ಯಾವಾಗಲೂ ನಿರುದ್ಧವಾದ ಚಿತ್ತದಲ್ಲಿ, ಸಮಾಧಿಯಲ್ಲಿದ್ದೇನೆ' ಎಂದು ಯಾರಾದರೂ ಹೇಳಿದರೆ ಅವುಗಳು ಕೇವಲ ಮಾತುಗಳೇ ಹೊರತು ನಿಜವಲ್ಲ. ನನ್ನಲ್ಲಿ ಕೋಟಿ ರೂಪಾಯಿಗಳಿವೆ ಎಂದು ಹೇಳಿಕೊಂಡು ಗುಡಿಸಿಲಿನಲ್ಲಿ ಕುಳಿತ್ತಿದ್ದರೆ ಅವನ ಬಳಿ ಕೋಟಿ ರೂಪಾಯಿಗಳಿವೆಯೆಂದು ಯಾರಾದರೂ ನಂಬುತ್ತಾರೆಯೇ? ಅದಕ್ಕೆ ತಕ್ಕಂತೆ ಯಾವುದಾದರು ಒಂದು ದೊಡ್ಡ ಮನೆಯಲ್ಲಿ ಇದ್ದರೆ ಮಾತ್ರ ನಂಬುತ್ತಾರೆ.

ಯಮ-ನಿಯಮ ಎಂದರೇನು?

\begin{shloka}
`ಅಹಿಂಸಾಸತ್ಯಾಸ್ತೇಯಬ್ರಹ್ಮಚರ್ಯಾಪರಿಗ್ರಹಾ ಯಮಾಃ|'
\end{shloka}

ಎಂದು ಹೇಳಲ್ಪಟ್ಟಿದೆ.

\begin{shloka}
`ಅಹಿಂಸಾಪ್ರತಿಷ್ಠಾಯಾಂ ತತ್ಸನ್ನಿಧೌ ವೈರತ್ಯಾಗಃ|'
\end{shloka}

(ಅಹಿಂಸೆಯಲ್ಲಿ ಪ್ರತಿಷ್ಠೆ ಪಡೆದವನ ಸನ್ನಿಧಿಯಲ್ಲಿ ವಿರೋಧಿತನವನ್ನು ಬಿಟ್ಟು ಬಿಡುತ್ತಾರೆ.) ಅಹಿಂಸಾ ಪ್ರತಿಷ್ಠೆ ಎನ್ನುವುದು ಬೇರೆ, ಅಹಿಂಸೆ ಎನ್ನುವುದು ಬೇರೆ. `ನನ್ನ ಬಾಳಿಗೆ ಬೇಕಾದ ಹಿಂಸೆಯಾದರೆ ಅದು ಮಾಡಬಹುದು. ಆದರೆ, ಅನಾವಶ್ಯಕವಾಗಿ ಹಿಂಸೆ ಮಾಡುವುದಿಲ್ಲ' ಎನ್ನುವ ಭಾವನೆ ಇದ್ದರೆ ಅದು ಅಹಿಂಸೆ. ಯಾವ ಕಾಲದಲ್ಲಾದರೂ, ಯಾವ ಜಾಗದಲ್ಲಾದರೂ ಯಾವುದೊಂದು ಪ್ರಾಣಿಗೂ ಹಿಂಸೆಯಾಗುವುದಿಲ್ಲ ಎನ್ನುವ ತೀರ್ಮಾನವನ್ನು ಮಾಡಿಕೊಂಡು ಅದರ ಪ್ರಕಾರ ನಡೆಯುತ್ತಾ ಬಂದರೆ ಅದು ಅಹಿಂಸಾ ಪ್ರತಿಷ್ಠೆಯಾಗುತ್ತದೆ. ನಾವು ಪುರಾಣಗಳಲ್ಲಿ ಅಹಿಂಸಾ ಪ್ರತಿಷ್ಠೆಯನ್ನು ಎಲ್ಲೆ ಮುಟ್ಟಿದ ಋಷಿಗಳಲ್ಲಿ ಆಶ್ರಮದಲ್ಲಿ ಪರಸ್ಪರ ವಿರೋಧಿಗಳಾಗಿದ್ದ ಪ್ರಾಣಿಗಳೂ ತಮ್ಮ ಹಿಂದಿನ ವಿರೋಧವನ್ನು ಬಿಟ್ಟು ಸ್ವಾಭಾವಿಕವಾಗಿ ಹೊಂದಿಕೊಂಡು ಬಂದಿವೆಯೆಂದು ನೋಡುತ್ತೇವೆ. ಹೀಗೆ ಅಹಿಂಸೆಯಲ್ಲಿ ಸ್ಥಿರವಾಗಿದ್ದುದರಿಂದ ಅವರ ಅಂತಃಕರಣದಲ್ಲಿ ಅಂಥ ಶಕ್ತಿ ಉಂಟಾಯಿತು. ನಮಗೆಲ್ಲ ಅಹಿಂಸಾ ಪ್ರತಿಷ್ಠೆ ಬಂದು ಬಿಟ್ಟಿದೆಯೆಂದು ಹೇಳಲಾಗುವುದಿಲ್ಲ, ಏಕೆಂದರೆ `ಯಾವುದಾದರೂ ಒಂದು ಕೆಲಸ ನಡೆಯುವುದಕ್ಕೆ ಚಿಕ್ಕ ಹಿಂಸೆ ಮಾಡಿದರೂ ಕೆಲಸ ಆದರೆ ಸಾಕು' ಎಂದೆನಿಸುತ್ತದೆ. ಅಹಿಂಸಾ ಪ್ರತಿಷ್ಠೆ ಬಂದರೆ ಮಾತ್ರ ಅಜ್ಞಾನಕ್ಕೆ ದೂರವಿರುವ ಪರಮೇಶ್ವರನನ್ನು ಪಡೆಯಬಹುದು. ಇದೇ ಮೊದಲು ನುಡಿದ `ಉಮಾ ಸಹಾಯಂ.........' ಎಂದು ಆರಂಭಿಸಿದ ಉಪನಿಷತ್ತಿನ ವಾಕ್ಯದಲ್ಲಿರುವ `ತಮಸಃ ಪರಸ್ತಾತ್' ಎನ್ನುವ ಮಾತುಗಳಿಂದ ಸೂಚಿತವಾಗಿದೆ. ಆದರೆ ಎಷ್ಟೋ ಮಂದಿ ಹಾಗೆ ಇಷ್ಟಪಡದೇ ಲೌಕಿಕವಾದ ಫಲಕ್ಕಾಗಿ ಧ್ಯಾನ ಮಾಡುತ್ತಾರೆ, ಅದಕ್ಕೆ ಅಹಿಂಸೆಯಾದರೂ ಇರಬೇಕು. ಅಹಿಂಸೆ ಇದ್ದರೆ ಮಾತ್ರ ಶಾಸ್ತ್ರ ವಿಹಿತವಾದ ಕರ್ಮಗಳನ್ನು ಮಾಡುತ್ತಾ ಭಗವಂತನನ್ನು ಚಿಂತಿಸಬಹುದು.

\begin{shloka}
`ಪಥ್ಯಮೌಷಧಸೇವಾ ಚ ಕ್ರಿಯತೇ ಯೇನ ರೋಗಿಣಾ|\\
ಆರೋಗ್ಯ ಸಿದ್ಧಿರ್ದೃಷ್ಟಾಸ್ಯ ...................................||'
\end{shloka}

ಔಷಧಿ ಕುಡಿದು ಪಥ್ಯವಿರುವುದಿಲ್ಲ ಎಂದರೆ ಔಷಧಿ ಸರಿಯಾಗಿ ಕೆಲಸ ಮಾಡುವುದಿಲ್ಲ. ಕೆಲವು ವೇಳೆ ಔಷಧಿ ಪೂರ್ತಿ ಕೆಲಸ ಮಾಡುವುದಿಲ್ಲ. ಅದೇ ರೀತಿ ಪರಮಾರ್ಥವನ್ನು ಪಡೆಯಲು ನಾವು ಮಾಡುವ ಪ್ರಯತ್ನಗಳು ಸಿದ್ಧಿಸಬೇಕಾದರೆ ಅಹಿಂಸೆಯನ್ನು ಅವಲಂಬಿಸಬೇಕು.

ಅನಂತರ `ಸತ್ಯ' ಎನ್ನುವುದು ಹೇಳಲಾಗಿದೆ. `ನಿಜ ಹೇಳುತ್ತೇನೆ' ಎಂದು ಎಲ್ಲರೂ ಹೇಳಬಹುದು. ಆದರೆ ಪೂರ್ತಿಯಾಗಿ ಸತ್ಯವಾಗಿರುವುದು ಸ್ವಲ್ಪ ಕಷ್ಟವೇ. ತಮಾಷೆಯಾಗಿ ಒಂದು ಕಥೆ ಹೇಳುವುದುಂಟು. `ನಾನು ಸುಳ್ಳು ಹೇಳುವುದಿಲ್ಲ' ಎನ್ನುವ ಒಬ್ಬನು ಸಾಕ್ಷಿ ಹೇಳುವುದಕ್ಕಾಗಿ ಹೋದನು. ನ್ಯಾಯಾಧೀಶನು ಅವನಿಗೆ `ಸತ್ಯವನ್ನು ಹೇಳಬೇಕು' ಎಂದು ಹೇಳಿದಾಗ ಅವನು, `ನನ್ನ ಕೈಯಲ್ಲಿ ಪ್ರಾಣವಿರುವವರೆಗೂ ನಾನು ಸತ್ಯವನ್ನೇ ಹೇಳುತ್ತೇನೆ' ಎಂದನು. `ಪ್ರಾಣ(ಜೀವ) ಇರುವವರೆಗೆ ಅವನು ಸತ್ಯವನ್ನೇ ಹೇಳುತ್ತಾನೆ' ಎಂದು ನ್ಯಾಯಾಧೀಶನು ಭಾವಿಸಿದನು. ಆದರೆ ಸಾಕ್ಷಿ ಹೇಳುವವನು ಹೇಳಿದ್ದೆಲ್ಲಾ ಸುಳ್ಳು. `ಜೀವ ಇರುವವರೆಗೆ ಸತ್ಯವನ್ನೇ ಹೇಳುತ್ತೇನೆಂದು ಹೇಳಿದೆಯಲ್ಲಾ!' ಎಂದು ಕೇಳಿದಾಗ ಅವನು, `ಹೌದು ಸ್ವಾಮಿ, ಅದಕ್ಕಾಗಿಯೇ ಕೈಯಲ್ಲಿ ಒಂದು ಹುಳು ಇಟ್ಟುಕೊಂಡಿದ್ದೆ. ಸುಳ್ಳು ಹೇಳಬೇಕು ಎಂದು ತೋರಿದಾಗ ಹುಳುವನ್ನು ಸಾಯಿಸಿಬಿಟ್ಟೇ ಇಷ್ಟು ಸುಳ್ಳನ್ನು ಹೇಳಿದೆ. ಕೈಯಲ್ಲಿ ಜೀವ ಇರುವವರೆಗೆ ಸುಳ್ಳು ಹೇಳಲಿಲ್ಲ' ಎಂದನು. ಇದು ಸತ್ಯವೇ? ಸತ್ಯವೇ ಅಲ್ಲ.

ಯಾವ ಸಮಯದಲ್ಲೂ ಬೇರೊಬ್ಬರಿಗೆ ಯಾವ ವಿಷಯವನ್ನೂ ಬೇರೆ ಅರ್ಥ ಬರುವಂತೆ ಹೇಳಬಾರದು. ಬೇರೆ ಅರ್ಥ ಬರುವಂತೆ ಒಂದು ಮಾತನ್ನು ಹೇಳಿ ತಾನು ಸರಿಯಾಗಿಯೇ ಹೇಳಿದಂತೆ ತೋರಿಸಿಕೊಂಡರೆ ಅದಕ್ಕೆ ಛಲವೆಂದು ಹೆಸರು. ಅದು ಇರಬಾರದು.

ಅನಂತರ `ಅಸ್ತೇಯ' ಎನ್ನುತ್ತಾರೆ. ಮನಸಾ, ವಾಚಾ, ಕರ್ಮಣಾ ಯಾರೂ ಕಳ್ಳತನ ಮಾಡಬಾರದು. ವಸ್ತುವನ್ನು ಕಳ್ಳತನ ಮಾಡಬೇಕೆಂದು ತೋರುತ್ತದೆಂದು ಮತ್ತೊಬ್ಬರೊಡನೆ ಹೇಳಿದರೆ ಅದು ಮಾತಿನಿಂದ (ವಾಚಾ) ಮಾಡಲ್ಪಟ್ಟ ಕಳ್ಳತನ. ಕಳ್ಳತನ ಮಾಡುವುದು ತಪ್ಪೆಂದು ಎಲ್ಲರಿಗೂ ಗೊತ್ತು. ಏಕೆಂದರೆ ಕಳ್ಳತನ ಮಾಡಿದುದು ತಿಳಿದರೆ ಪೋಲೀಸನು ಹಿಡಿದುಕೊಳ್ಳುತ್ತಾನೆ. ಆದರೆ ಮಾತಿನಿಂದಲೂ, ಮನಸ್ಸಿನಿಂದಲೂ ಮಾಡಬಹುದಾದ ಕಳ್ಳತನವನ್ನು ತಪ್ಪೆಂದು ಎಷ್ಟೋ ಜನ ಭಾವಿಸುವುದಿಲ್ಲ. ಆದರೆ, ಅದೂ ತಪ್ಪು. ಹೀಗೆ ಮನಸಾ, ವಾಚಾ, ಕರ್ಮಣಾ ಉಂಟಾಗುವ ಕಳ್ಳತನವಿಲ್ಲದೆ ಇದ್ದರೆ ಒಂದು ವಿಶೇಷವಾದ ಸಂಸ್ಕಾರ ಮತ್ತು ಶಕ್ತಿ ಇರುತ್ತದೆ. ನಾವು ಯಾರಾದರೂ ಹಾಗೆ ಕಳ್ಳತನ ಮಾಡುತ್ತೇವೆಯೇ ಎಂದರೆ, ಒಂದು ಕರ್ಚೀಫು ಎಲ್ಲಾದರೂ ಸಿಕ್ಕಿದರೆ ಯಾರೂ ಕಳ್ಳತನ ಮಾಡುವುದಿಲ್ಲ. ಆದರೆ ಒಂದು ತೋಡ ಒಂದು ಉಂಗುರ ಯಾರೂ ಇಲ್ಲದ ಸಮಯದಲ್ಲಿ ಸಿಕ್ಕಿದರೆ, ಯಾರೂ ಇಲ್ಲ, ತೆಗೆದುಕೊಳ್ಳೋಣ ಎನಿಸುತ್ತದೆ. ಕರ್ಚೀಫಿನ ವಿಷಯದಲ್ಲಿ ತೋರುವಂತೆಯೇ ಉಂಗುರದ ವಿಷಯದಲ್ಲಿಯೂ ತೋರಿದರೆ ಆವಾಗಲೇ ನಿಜವಾಗಿಯೂ ಅಸ್ತೇಯವಿದೆಯೆಂದು ಹೇಳಬಹುದು. ಅದನ್ನು ಬಿಟ್ಟು, ಸಾಮಾನ್ಯವಾದ ವಸ್ತುವಿನ ವಿಷಯದಲ್ಲಿ `ಇದು ಬೇಡ' ಎಂದು ತೋರಿ, ತಾನು ಸಂಪಾದಿಸುವುದಕ್ಕೆ ಆಗದೇ ಇರುವ ವಸ್ತುವಿನ ವಿಷಯದಲ್ಲಿ `ಇದು ಮಾತ್ರ ಪರವಾಗಿಲ್ಲ' ಎಂದು ತೋರಿದರೆ ಅದು ಅಸ್ತೇಯವಲ್ಲ. ಅಸ್ತೇಯವೆನ್ನುವುದು ಇದ್ದರೆ ಏನು ಫಲ ಎಂದರೆ-

\begin{shloka}
`ಅಸ್ತೇಯ ಪ್ರತಿಷ್ಠಾಯಾಂ ಸರ್ವರತ್ನೋಪಸ್ಥಾನಮ್|'
\end{shloka}

(ಅಸ್ತೇಯದಲ್ಲಿ ಪ್ರತಿಷ್ಠೆ ಉಂಟಾದರೆ ಎಲ್ಲಾ ರತ್ನಗಳ ಪ್ರಾಪ್ತಿಯೂ ಆಗುತ್ತದೆ.) ಎಂದು ಹೇಳಿದ್ದಾರೆ.

\begin{shloka}
`ಬ್ರಹ್ಮಚರ್ಯಪ್ರತಿಷ್ಠಾಯಾಂ ವೀರ್ಯಲಾಭಃ|'
\end{shloka}

ಬ್ರಹ್ಮಚರ್ಯ ಪ್ರತಿಷ್ಠೆ ಉಂಟಾದರೆ ಸಾಮರ್ಥ್ಯ ಉಂಟಾಗುತ್ತದೆ. ಬ್ರಹ್ಮಚರ್ಯವೆಂದರೆ ಇಂದ್ರಿಯ ನಿಗ್ರಹ. ಅದು ಶಾಸ್ತ್ರೀಯವಾದುದು. ಗೃಹಸ್ಥನು ಬ್ರಹ್ಮಚಾರಿಯೇ ಅಲ್ಲವೇ ಎನ್ನುವ ಪ್ರಶ್ನೆ ಇದೆ. ಒಂದು ಜಾಗದಲ್ಲಿ ಆಪಸ್ತಂಬರು, ಗೃಹಸ್ಥನು ಪುತ್ರನನ್ನು ಪಡೆಯುವುದು ಧರ್ಮ. ಅನಂತರ ಅವನಿಗೆ ವಿಷಯ ಭೋಗ ಬೇಕೆ, ಬೇಡವೆ ಎಂದು ಕೇಳಿದರೆ `ಅನ್ಯೇತು ಕಾಮಜಾಃ ಪ್ರೋಕ್ತಾಃ' ಎಂದು ಹೇಳಿದರು. ಮೊದಲು ಅವನು ಪುತ್ರನನ್ನು ಪಡೆಯುವುದು ನಿಯಮ. ಅನಂತರ ಅವನಿಗೆ ಪುತ್ರಪ್ರಾಪ್ತಿಯಾದರೆ ಅದು ಅವನ ಇಷ್ಟದಿಂದಲೇ ಆದದ್ದು. ಧರ್ಮದಿಂದ ಮಾತ್ರ ಬಂದುದು ಎನ್ನುವುದಿಲ್ಲ. ಆದ್ದರಿಂದ ಗೃಹಸ್ಥನು ಈ ರೀತಿ ನಡೆದುಕೊಂಡರೆ ಅವನೂ ಬ್ರಹ್ಮಚಾರಿಯೇ ಎಂದು ಹೇಳಿದರು.

ಕೊನೆಯಲ್ಲಿ ಅಪರಿಗ್ರಹ ಎನ್ನುವುದನ್ನು ಹೇಳಿದರು. ಮತ್ತೊಬ್ಬರ ವಸ್ತುಗಳಿಗೆ ಆಸೆ ಪಡಬಾರದು, ಅಯಾಚಿತ ಭಟ್ಟಾಚಾರ್ಯರೆನ್ನುವವರು ಒಬ್ಬರಿದ್ದರು. ಅವರು ಯಾಚನೆ ಮಾಡುತ್ತಿರಲಿಲ್ಲ. ಆದರೆ ಯಾರು ಏನು ಕೊಟ್ಟರೂ ತೆಗೆದುಕೊಳ್ಳುತ್ತಿದ್ದರು. ಅದಕ್ಕೆ ಒಬ್ಬರು, `ಇದು ಪ್ರಯೋಜನವಿಲ್ಲ. ನೀನು ಅಪರಿಗ್ರಹ ಭಟ್ಟಾಚಾರ್ಯನಾಗಬೇಕು' ಎಂದರು. ನಮಗೆ ಯಾರು ಏನು ಕೊಟ್ಟರೂ ತೆಗೆದುಕೊಳ್ಳಬೇಕೆಂಬ ಆಸೆ ಮನಸ್ಸಿನಲ್ಲಿದೆ. ನಾವಾಗಿಯೇ ಕೇಳಿದರೆ ಇತರರು ಕೊಡುತ್ತಾರೆಯೋ ಇಲ್ಲವೋ ಎನ್ನುವ ಸಂಕೋಚ. ಹಾಗಿರಬಾರದು. ಕೊಟ್ಟರೂ ಬೇಡ ಎಂದಿರಬೇಕು. ಸಂನ್ಯಾಸಿಗಳಾಗಿರುವವರಿಗೆ ಅಪರಿಗ್ರಹ ವ್ರತವಾಗುತ್ತದೆ. ಸಂನ್ಯಾಸಿ ತನ್ನ ಬದುಕಿಗೆ ಬೇಕಾದ ಕಮಂಡಲು, ಕಾಷಾಯ ವಸ್ತ್ರ, ಆಸನಗಳಂಥ ವಸ್ತುಗಳನ್ನು ಇಟ್ಟುಕೊಳ್ಳಬಹುದು. ಗೃಹಸ್ಥನೂ ತನ್ನ ಜೀವನಕ್ಕೆ ಬೇಕಾದವುಗಳನ್ನು ಮಾತ್ರ ಸಂಪಾದಿಸಿಕೊಂಡರೆ ಅದು ನಿಜವಾದ ಅಪರಿಗ್ರಹವಾಗುತ್ತದೆ.

ಕೆಲವರಿಗೆ ಪ್ರಪಂಚದಲ್ಲಿರುವ ಎಲ್ಲಾ ವಸ್ತುಗಳನ್ನು ಸಂಪಾದಿಸಿದರೂ ಕೂಡ ಎಲ್ಲಿಯಾದರೂ ನನ್ನ ಹತ್ತಿರ ಇಲ್ಲದ ವಸ್ತುಗಳಿರಬಹುದು, ಅವು ನನಗೆ ದೊರೆಯಲಿಲ್ಲವಲ್ಲಾ ಎನ್ನುವ ವ್ಯಥೆ. ಇದಕ್ಕೆ ಸಂಬಂಧಪಟ್ಟಂತೆ ಒಂದು ಕಥೆ ಇದೆ. ಶುಂಭ, ನಿಶುಂಭರಿಗೆ `ನಾವು ದೊಡ್ಡ ಅಸುರರು `ಪೃಥಿವ್ಯಾಂ ಯಾನಿ ರತ್ನಾನಿ' (ಎಂದು ಹೇಳಿದಂತೆ) ಯಾವ ಯಾವ ರತ್ನಗಳಿವೆಯೋ ಅವುಗಳೆಲ್ಲ ನಮಗೆ ಬೇಕು' ಎನ್ನುವ ಅಭಿಪ್ರಾಯವಿತ್ತು. `ಯತೋ ರತ್ನ ಭುಜೋ ವಯಂ'-ನಾವು ಎಲ್ಲ ರತ್ನಗಳನ್ನು ಅನುಭವಿಸುವವರು. `ನೀನು ಸ್ತ್ರೀ ರತ್ನವಾಗಿದ್ದೀಯೆ. ಆದ್ದರಿಂದ ನೀನು ನಮ್ಮ ಹತ್ತಿರ ಬಂದು ಬಿಡು' ಎಂದು ಅವರು ದೇವಿಗೆ ಹೇಳಿಕಳುಹಿಸಿದರು. ದೇವಿ `ನೀವು ಹೇಳುವುದು ನ್ಯಾಯವೇ. ರತ್ನಗಳನ್ನು ಅನುಭವಿಸುವವರು ಎನ್ನುವುದು ಸರಿ. ನಾನು ರತ್ನವೆನ್ನುವುದೂ ಸರಿ. ನಾನು ನಿಮ್ಮ ಹತ್ತಿರ ಬರಬೇಕೆಂದೂ ನೀವು ಆಸೆ ಪಡುವುದೂ ಸರಿ. ಆದರೆ, ನಾನು ಪ್ರತಿಜ್ಞೆಯೊಂದು ಮಾಡಿದ್ದೇನೆ. `ಯೋ ಮೇ ದರ್ಪಂ ವ್ಯಪೋಹತಿ'-ನನಗೆ ಸ್ವಲ್ಪ ದರ್ಪ (ಗರ್ವ)ವಿದೆ. ಆ ದರ್ಪದಿಂದ `ಯೋ ಮೇ ಪ್ರತಿಬಲೋ ಲೋಕೇ' ಎನ್ನುವಂತೆ ನನಗೆ ಎದುರಾಗಿ ನಿಂತು ಯಾರು ನನ್ನನ್ನು ಯುದ್ಧದಲ್ಲಿ ಜಯಿಸುತ್ತಾರೋ ಅವರನ್ನು ಮದುವೆಯಾಗುವುದೆಂದು ಪ್ರತಿಜ್ಞೆ ಮಾಡಿದ್ದೇನೆ. ಆದ್ದರಿಂದ ಶುಂಭನೋ ನಿಶುಂಭನೋ ಯಾರಾದರೂ ಸರಿಯೆ ನನ್ನೊಡನೆ ಹೋರಾಡಲು ಬರಬಹುದು' ಎಂದಳು. ಈ ಕಥೆ ಏತಕ್ಕೆಂದರೆ `ಯತೋ ರತ್ನ ಭುಜೋ ವಯಂ' ಎನ್ನುವ ಭ್ರಾಂತಿಯನ್ನಿಟ್ಟುಕೊಂಡು ಸ್ತ್ರೀ ರತ್ನ ಭೂತಾ ಚಾರ್ವಂಗೀ-ಸ್ತ್ರೀ ರತ್ನವಾಗಿ ಪ್ರಕಾಶಿಸುವವಳೂ, ಸುಂದರವಾದ ಅಂಗಗಳುಳ್ಳವಳಾಗಿಯೂ ಇರುವ, ಜಗನ್ಮಾತೆಯನ್ನು ಶುಂಭ-ನಿಶುಂಭರು ಅಪೇಕ್ಷಿಸಿದರು. ಈ ಬುದ್ಧಿ ಇದ್ದರೆ ಇದು ಅಪರಿಗ್ರಹವಾಗುತ್ತದೆಯೇ?

ನಮ್ಮ ಜೀವನಕ್ಕೆ ಏನು ಬೇಕೋ ಅದು ಬಿಟ್ಟು ಬೇರೆ ಯಾವುದನ್ನೂ ಸೇರಿಸಿಟ್ಟುಕೊಳ್ಳುವುದಿಲ್ಲ ಎಂದಿರಬೇಕು. ಇದುವರೆಗೆ `ಯಮ' ವೆನ್ನುವುದು ಯಾವುದೆಂದು ನೋಡಿದೆವು. ಇದೇ ರೀತಿ `ನಿಯಮ' ಯಾವುದು ಎನ್ನುವುದಕ್ಕೆ `ಶೌಚಃ ಸಂತೋಷ ತಪಃ ಸ್ವಾಧ್ಯಾಯೇಶ್ವರಪ್ರಣಿಧಾನಾನಿ ನಿಯಮಾಃ' ಎಂದು ಹೇಳಿದರು. ಶೌಚವೆನ್ನುವುದು ಒಂದು ನಿಯಮ. `ಮೃಜ್ಜಲಾಭ್ಯಾಂ ಬಹಿಃ ಶೌಚಂ' ಮಣ್ಣಿನಿಂದಲೂ ನೀರಿನಿಂದಲೂ ಹೊರಗೆ ಇರುವಿಕೆ ಶೌಚ. ಮಣ್ಣಿನಲ್ಲಿ ಒಂದು ವಿಶೇಷವಾದ ಗುಣವಿದೆ. ಯಾವುದಾದರೂ ಅಶುದ್ಧಿಯಾದರೆ, ಆ ಮಣ್ಣನ್ನು ಮುಟ್ಟಿಸಿ ನೀರನ್ನು ಹಾಕಿದರೆ ಆ ಅಶುದ್ಧಿ ಹೊರಟು ಹೋಗುತ್ತದೆ. ಮರಳು ಸೇರಿರುವುದಾಗಲಿ, ಜೇಡಿ ಮಣ್ಣು ಸೇರಿರುವುದಾಗಲಿ,-ಹೀಗೆಲ್ಲಾ ಉಪಯೋಗ ಮಾಡಕೂಡದು. ಇದೆಲ್ಲವನ್ನು ಶೌಚವನ್ನು ಕುರಿತು ಹೇಳುವಾಗ ಹೇಳಿದ್ದಾರೆ.

ಅನಂತರ `ಸಂತೋಷ'ವನ್ನು ಕುರಿತು ಹೇಳಿದರು. ಅಶುದ್ಧವಾದ ಮನಸ್ಸುಳ್ಳವನು ಐದು ನಿಮಿಷಗಳು ಕೂಡ ಏನೂ ಮಾಡದೆ ಕುಳಿತುಕೊಳ್ಳಲಾರನು. ಅವನ ಮನಸ್ಸು ಎಲ್ಲೆಲ್ಲೋ ತಿರುಗಾಡುತ್ತಿರುತ್ತದೆ. ಆದ್ದರಿಂದ `ತೃಪ್ತಿ' ಎನ್ನುವುದು ಬೇಕು. `ಅಸಂತುಷ್ಟೋ ದ್ವಿಜೋ ನಷ್ಟಃ' (ತೃಪ್ತಿ ಇಲ್ಲದ ಬ್ರಾಹ್ಮಣನು ನಾಶ ಹೊಂದುತ್ತಾನೆ). ಬ್ರಾಹ್ಮಣನಾದವನಿಗೆ ಸಂತೋಷವೇ ಉತ್ತಮವಾದ ಐಶ್ವರ್ಯ. ಇರುವದನ್ನು ಇಟ್ಟುಕೊಂಡು ಸಂತೋಷವಾಗಿದ್ದರೆ ಮನಸ್ಸು ಏಕಾಗ್ರವಾಗಿರುವುದು.

`ತಪಸ್ಸು' ಎಂದರೆ ವ್ರತಗಳು. ಅದೇ ರೀತಿ `ನಾನಶನಾತ್ ಪರಂ ತಪಃ' ಏಕಾದಶಿಯಂದು ನಾವು ಊಟಮಾಡದೆ ಇರುವುದು ದೊಡ್ಡ ತಪಸ್ಸು. ಸ್ವಾಧ್ಯಾಯ ಪ್ರವಚನ (ತಮ್ಮ ತಮ್ಮ ವೇದವನ್ನು ಅಧ್ಯಯನ ಮಾಡುವುದು), ತಪಸ್ಸು.

`ಈಶ್ವರಪ್ರಣಿಧಾನಾನಿ'-ಈಶ್ವರನು ನಾವು ಮನಸ್ಸಿನಲ್ಲಿ ಚಿಂತಿಸಬೇಕು. ಮೇಲೆ ಹೇಳಿದ ಯಮ-ನಿಯಮಗಳೊಡನೆ ಈಶ್ವರನನ್ನು ಆರಾಧಿಸಿದರೆ ಅದು ಬಹಳ ವಿಶೇಷ. ಅದನ್ನು ಧ್ಯಾನವೆನ್ನುತ್ತಾರೆ. ಹೀಗೆ ಧ್ಯಾನ ಮಾಡಿದರೆ ಅದು ಉಪಾಸನೆ ಎನ್ನಲ್ಪಡುತ್ತದೆ. ಯಮ-ನಿಯಮಗಳಿಲ್ಲದ ನನಗೆ ಷೋಡಶಾಕ್ಷರೀ, ಪಂಚಾಕ್ಷರೀ, ಬಾಲಾ, ಗಣಪತಿ ಇಂಥ ಮಂತ್ರಗಳು ಉಪದೇಶವಾಗಿವೆಯೆಂದು ಹೇಳಿಕೊಂಡು ಜಪ ಮಾಲೆಯನ್ನು ಕೈಯಲ್ಲಿ ಮಾತ್ರ ಇಟ್ಟುಕೊಂಡಿದ್ದರೆ, ಅದು ಕೂಡ ಅವಕಾಶವಿದ್ದಾಗ ಮಾತ್ರ ಮಾಡಿಕೊಂಡಿದ್ದರೆ, ಅದು ಉಪಾಸನೆ ಆಗುವುದಿಲ್ಲ. ಸರಿಯಾಗಿ ಉಪಾಸನೆ ಮಾಡಿದರೆ ಲೌಕಿಕ ಫಲವನ್ನು ಸಂಪಾದಿಸಬಹುದು. ಈ ವಿಷಯವನ್ನು ಭಗವಂತನು ಗೀತೆಯಲ್ಲಿ ಹೇಳಿದ್ದಾನೆ-

\begin{shloka}
`ಯೋಗೀ ಯುಂಜೀತ ಸತತಮಾತ್ಮಾನಂ ರಹಸಿ ಸ್ಥಿತಃ|\\
ಏಕಾಕೀ ಯುತಚಿತ್ತಾತ್ಮಾ ನಿರಾಶೀರಪರಿಗ್ರಹಃ||\\
ಶುಚೌ ದೇಶೇ ಪ್ರತಿಷ್ಠಾಪ್ಯ ಸ್ಥಿರಮಾಸನಮಾತ್ಮನಃ|\\
ನಾತ್ಯುಚ್ಛ್ರಿತಂ ನಾತಿನೀಚಂ ಚೈಲಾಜಿನಕುಶೋತ್ತರಮ್||\\
ತತ್ರೈಕಾಗ್ರ್ಯಂ ಮನಃ ಕೃತ್ವಾ ಯತ ಚಿತ್ತೇಂದ್ರಿಯಕ್ರಿಯಃ|\\
ಉಪವಿಶ್ಯಾಸನೇ ಯುಂಜ್ಯಾದ್ಯೋಗಮಾತ್ಮವಿಶುದ್ಧಯೇ||\\
ಸಮಂ ಕಾಯಶಿರೋಗ್ರೀವಂ ಧಾರಯನ್ನಚಲಂ ಸ್ಥಿರಃ|\\
ಸಂಪ್ರೇಕ್ಷ್ಯ ನಾಸಿಕಾಗ್ರಂ ಸ್ವಂ ದಿಶಶ್ಚಾನವಲೋಕಯನ್||\\
ಪ್ರಶಾಂತಾತ್ಮಾ ವಿಗತಭೀಃ ಬ್ರಹ್ಮಚಾರಿವ್ರತೇ ಸ್ಥಿತಃ|\\
ಮನಃ ಸಂಯಮ್ಯಮಚ್ಚಿತ್ತೋ ಯುಕ್ತ ಆಸೀತ ಮತ್ಪರಃ||
\end{shloka}

ಯೋಗೀ ಯುಂಜೀತ ಸತತಮಾತ್ಮಾನಂ ರಹಸಿ ಸ್ಥಿತಃ-`ಆತ್ಮಾನಂ' ಎಂದರೆ ಅಂತಃಕರಣ. `ಯುಂಜೀತ' ಎಂದರೆ ಏಕಾಗ್ರವಾಗಿ ಅಂತಃಕರಣವನ್ನು ಇಟ್ಟುಕೊಂಡು ಚಿಂತನೆ ಮಾಡುವುದು. ಕೆಲವರು ಸಭೆಯಲ್ಲಿ ಕುಳಿತು ಧ್ಯಾನ ಮಾಡುತ್ತಾರೆ. ಹಾಗೆ ಮಾಡಿದರೆ ಒಳ್ಳೆಯ ಯೋಗೀಶ್ವರರೆಂಬ ಹೆಸರು ಬರುತ್ತದೆ. ಹೀಗಿರಬಾರದು. `ಏಕಾಕಿ' ಒಬ್ಬಂಟಿಗನಾಗಿ, ಎನ್ನುತ್ತಾರೆ. ಭಗವಂತನ ಚಿಂತನೆ ಮಾಡಬೇಕಾದರೆ ಒಬ್ಬಂಟಿಗನಾಗಿ ಕುಳಿತುಕೊಳ್ಳಬೇಕು. ಇದನ್ನೇ `ರಹಸಿ ಸ್ಥಿತಃ' ಎನ್ನುತ್ತಾರೆ. ಚಿತ್ತವು ಸ್ವಾಧೀನವಾಗಿರಬೇಕು. ಶರೀರವೂ ಸ್ವಾಧೀನವಾಗಿರಬೇಕು. ಶರೀರ ಸ್ವಾಧೀನವಾಗಿದ್ದರೇನೇ ಚಿತ್ತ ಸ್ವಾಧೀನದಲ್ಲಿರುತ್ತದೆ. ಆದ್ದರಿಂದ `ಯತ ಚಿತ್ತಾತ್ಮಾ' ಎನ್ನುತ್ತಾರೆ. ಅದು ಬೇಕು, ಇದು ಬೇಕು ಎನ್ನುವ ಆಸೆಗಳಿಲ್ಲದೆ ಇರಬೇಕೆನ್ನುವುದಕ್ಕೆ `ನಿರಾಶೀರಪರಿಗ್ರಹಃ (ಆಸೆ ಇಲ್ಲದೆಯೂ, ವಸ್ತುಸಂಗ್ರಹ ಮಾಡದೆಯೂ) ಎನ್ನುತ್ತಾರೆ.

`ಶುಚೌ ದೇಶೇ ಪ್ರತಿಷ್ಠಾಪ್ಯ ಸ್ಥಿರಮಾಸನಮಾತ್ಮನಃ'-ಜಾಗ ಸರಿಯಾಗಿದ್ದರೇನೇ ಅದು ಧ್ಯಾನಕ್ಕೆ ಸರಿಯಾಗುತ್ತದೆ. ತರ್ಕಶಾಸ್ತ್ರದಲ್ಲಿ ಒಂದು ಕಡೆ ಶುಚಿಯಾಗಿರುವುದನ್ನು ಕುರಿತು ಹೇಳುವಾಗ, ಪೂರ್ವಪಕ್ಷವನ್ನು ಮನಸ್ಸಿನಲ್ಲಿಟ್ಟುಕೊಂಡು ತಪ್ಪಾದ ಯುಕ್ತಿಯನ್ನು ಮೊದಲು ತೋರಿಸಿದರು. `ನರಚಿರಃ ಕಪಾಲಂ ಶುಚಿ ಪ್ರಾಣ್ಯಂಗತ್ವಾತ್ ಶಂಖವತ್'-ಒಂದು ಪ್ರಾಣಿಯ ಅವಯವವಾದ ಶಂಖವನ್ನು ನಾವು ಪೂಜೆಗೆ ಉಪಯೋಗಿಸುತ್ತೇವೆ. ಏಕೆಂದರೆ ಅದನ್ನು ಶುಚಿಯೆಂದುಕೊಳ್ಳುತ್ತೇವೆ. ತಲೆ ಒಂದು ಓಡುವ ಪ್ರಾಣಿಯ (ಮನುಷ್ಯನ) ಅಂಗ ತಾನೇ! ಆದ್ದರಿಂದ ಅದನ್ನೂ ಶುಚಿಯೆಂದೇ ಇಟ್ಟುಕೊಳ್ಳಬಹುದು. ಹೀಗಿರುವಾಗ ಮೂಳೆಯನ್ನು ಮುಟ್ಟಿದರೆ ಏಕೆ ಸ್ನಾನ ಮಾಡಬೇಕು ಎನ್ನುವ ಪ್ರಶ್ನೆ ಕೆಲವರಿಗೆ. ಈ ಯುಕ್ತಿ ಸರಿಯಾಗಿಲ್ಲ ಎಂದು ತೋರಿಸುವದಕ್ಕೆ `ಶಾಸ್ತ್ರ ನಿಷಿದ್ಧಾತ್ವಾತ್' ಎಂದು ಉತ್ತರ ಹೇಳಿದರು. ಮನುಷ್ಯನ ಅಸ್ಥಿಯನ್ನು ಮುಟ್ಟಿದರೆ `ಸಚೇಲಸ್ನಾನ ಮಾಚರೇತ್' (ಉಟ್ಟಿರುವ ಬಟ್ಟೆಯೊಡನೆ ಸ್ನಾನ ಮಾಡಬೇಕು) ಎಂದು ಹೇಳಿದ್ದಾರೆ. ಸಾಮಾನ್ಯವಾದ ಮೂಳೆಯನ್ನು ಮುಟ್ಟಿದರೆ ಸ್ನಾನ ಮಾಡಿದರೆ ಸಾಕು. ಆದರೆ (ಸತ್ತು) ಕೆಲವು ದಿನಗಳು ಮಾತ್ರವೇ ಆಗಿದ್ದು ಮೂಳೆಯಲ್ಲಿ ಮಾಂಸದಂಥವು ಇರುವುದನ್ನು ಮುಟ್ಟಿದರೆ ಸಚೇಲ ಸ್ನಾನ ಮಾಡಬೇಕೆಂದು ಹೇಳಿದ್ದಾರೆ.

ಆದ್ದರಿಂದ ಶಾಸ್ತ್ರದಲ್ಲಿ ಹೇಳಿದಂತೆ ಇರುವುದೇ ನಿಜವಾದ ಶುಚಿ. ಇದು ಇರಬೇಕೆನ್ನುತ್ತಾರೆ. ದೇವಾಲಯಗಳು, ಪರ್ವತಗಳು, ನದೀತೀರ, ಮನೆಯಲ್ಲಿರುವ ಪ್ರತಿ ದಿನವೂ ಪೂಜೆ ಮಾಡುವ ಜಾಗ ಇವುಗಳೆಲ್ಲವನ್ನೂ ಶುಚಿ ಎನ್ನುತ್ತಾರೆ. ಈ ಕಾಲದಲ್ಲಿ `ಎಲ್ಲವನ್ನೂ ಸುಂದರವಾಗಿ ಶುದ್ಧಿ ಮಾಡಿಬಿಟ್ಟೆ. ಎಲ್ಲವೂ ಶುದ್ಧ' ಎಂದು ಹೇಳುತ್ತಾರೆ. ಹಾಗಲ್ಲ, ಯಾವ ಯಾವ ಜಾಗವನ್ನು ಶುದ್ಧವೆಂದು ಶಾಸ್ತ್ರದಲ್ಲಿ ಹೇಳಿದ್ದಾರೆ. ಎಂಜಲು ಅಶುದ್ಧ ಎನ್ನುತ್ತಾರೆ. ಏಕೆ ಅದು ಅಶುದ್ಧ ಎಂದರೆ ಅದು ಅಶುದ್ಧ ಅಷ್ಟು ಮಾತ್ರ ಹೇಳಬಹುದು. ಎಂಜಲಾದ ಬಾಯನ್ನು ನಾವು ಮುಟ್ಟುತ್ತೇವೆ. `ಅದು ಅಶುದ್ಧವಲ್ಲವೇ?' ಎಂದು ಕೇಳಿದರೆ ಎಂಜಲು ಒಳಗಡೆ ಇದ್ದರೆ ಶುದ್ಧ, ಹೊರಗಡೆ ಬಂದರೆ ಅಶುದ್ಧ, ಅಷ್ಟೇ. ಆದ್ದರಿಂದ ಶಾಸ್ತ್ರದಲ್ಲಿ ಹೇಳಲ್ಪಟ್ಟ ಶುಚಿಯಾದ ಜಾಗದಲ್ಲಿಯೇ ನಾವು ಕುಳಿತುಕೊಳ್ಳಬೇಕು. ಇದನ್ನು ಮನಸ್ಸಿನಲ್ಲಿಟ್ಟುಕೊಂಡೇ `ಶುಚೌದೇಶೇ ಪ್ರತಿಷ್ಠಾಪ್ಯ ಸ್ಥಿರಮಾಸನಮಾತ್ಮನಃ' ಎನ್ನುತ್ತಾರೆ. ನಮ್ಮ ಆಸನವನ್ನು ಹಾಕಿಕೊಂಡು ನಾವು ಕುಳಿತುಕೊಳ್ಳಬೇಕೆಂದು ಅದರಲ್ಲಿ ಬರುತ್ತದೆ. ಇದರ ತಾತ್ಪರ್ಯವೇನೆಂದರೆ, ಇನ್ನೊಬ್ಬರ ಆಸನದಲ್ಲಿ ನಾವು ಕುಳಿತುಕೊಳ್ಳಬಾರದು. ಈ ಕಾಲದಲ್ಲಿ ಎಲ್ಲರೂ ಮತ್ತೊಬ್ಬರ ಆಸನಗಳಲ್ಲಿ ಕುಳಿತುಕೊಳ್ಳಬೇಕಾಗುವುದು. ಮತ್ತೊಬ್ಬರು ಕುಳಿತ ಆಸನದಲ್ಲಿ ಕುಳಿತರೆ ಅದು ಅಶುಚಿ. ಅದೇ ರೀತಿ ನಮ್ಮ ಕಮಂಡಲುವನ್ನು ಮತ್ತೊಬ್ಬರು ಉಪಯೋಗಿಸಿದರೆ ಅದೂ ಅಶುಚಿ. ಆದ್ದರಿಂದಲೇ ಶಿಷ್ಯರು ತಮಗಾಗಿಯೇ ನೀರಿನ ಪಾತ್ರೆಯನ್ನು ಇಟ್ಟುಕೊಂಡಿರುತ್ತಾರೆ. ಅದನ್ನು ನೋಡಿ ಅವರಿಗೆ ಬೇರೆಯವರ ಮೇಲೆ ದ್ವೇಷವಿದೆಯೆಂದು ಹೇಳುವ ಹಾಗಿಲ್ಲ. ಆರೋಗ್ಯ ಶಾಸ್ತ್ರದಲ್ಲಿಯೂ ಇದಕ್ಕೆ ಕಾರಣಗಳಿವೆ. ಈಗ ಇವುಗಳನ್ನು ಕುರಿತು ನಾವು ಸಂಶೋಧನೆ ಮಾಡುವುದಿಲ್ಲ. ಏಕೆಂದರೆ ಮತ್ತೊಬ್ಬರ ಆಸನವನ್ನು ಉಪಯೋಗಿಸಿದರೂ ರೋಗ ಬರಲಿಲ್ಲವಲ್ಲಾ ಎಂದು ಕೆಲವರು ಕುತರ್ಕ ಮಾಡಲು ಅವಕಾಶವಿದೆ. `ಶಾಸ್ತ್ರದಲ್ಲಿ ಪವಿತ್ರತೆ ಇಲ್ಲ' ಎಂದು ಒಂದೇ ಮಾತಿನಲ್ಲಿ ಮುಗಿಸಿಬಿಡುತ್ತಾರೆ. ಈ ಕಾಲದಲ್ಲಿ ಯಾರದೋ ಬಟ್ಟೆಯನ್ನು ಯಾರೋ ಹಾಕಿಕೊಂಡು ಹೋಗುತ್ತಾರೆ. ಆ ಸಮಯದಲ್ಲಿ ಯಾವುದೋ ಒಂದು ಬಟ್ಟೆ ಸಿಕ್ಕಿದರೆ ಸಾಕು ಎನ್ನುವ ಭಾವನೆ! ಐವತ್ತು ವರ್ಷಗಳಿಗೆ ಹಿಂದೆ ಹೀಗಿರಲಿಲ್ಲ. ಆದರೆ ಬಹಳ ಯೋಗ್ಯರಾದ ದೊಡ್ಡವರು ಉಪಯೋಗಿಸಿದುದು ಸಿಕ್ಕಿದರೆ ಅದನ್ನು ಪವಿತ್ರವೆಂದು ಇಟ್ಟುಕೊಂಡು ವಿಶೇಷ ಕಾಲದಲ್ಲಿ ಮಾತ್ರ ಹಾಕಿಕೊಳ್ಳುತ್ತಾರೆ. ಇದು ಪರಂಪರೆಯಾಗಿ ಬಂದ ವಿಶೇಷ. ಅನಂತರ ಭಗವಂತನು `ನಾತ್ಯುಚ್ಛ್ರಿತಂ ನಾತಿ ನೀಚಂ ಚೈಲಾಜಿನ ಕುಶೋತ್ತರಮ್' ಎಂದನು. ಎಂಥ ಜಾಗ? ದೊಡ್ಡ ಮೇಜಿನ ಮೇಲೆ ಕುಳಿತು ಜಪ ಮಾಡಬಾರದು. ಭೂಮಿಯ ಮೇಲೂ ಕುಳಿತುಕೊಳ್ಳಬಾರದು. ಭೂಮಿಯ ಮೇಲೆ ಕುಳಿತುಕೊಂಡರೆ ಅದರಲ್ಲಿರುವ ಶೈತ್ಯಾದಿಗಳು ನಮಗೆ ಬಂದು ಬಿಡುತ್ತವೆ. ಬಹಳ ಎತ್ತರವಾದ ಜಾಗದಲ್ಲಿ ಕುಳಿತು ಸಮಾಧಿ ಸ್ಥಿತಿಗೆ ಬಂದವನು ಕೆಳಗೆ ಬಿದ್ದರೆ ಅಪಾಯವಾಗುತ್ತದೆ. ಆದ್ದರಿಂದ `ನಾತ್ಯುಚ್ಛ್ರಿತಂ ನಾತಿ ನೀಚಂ'-ಎತ್ತರವಾದ ಜಾಗವೂ ಅಲ್ಲ, ಕೆಳಗಿನ ಜಾಗವೂ ಅಲ್ಲ ಎನ್ನುತ್ತಾರೆ. ದರ್ಭೆಯ ಮೇಲೆ ಕೃಷ್ಣಾಜಿನ ಹಾಕಿಕೊಂಡು ಅದರ ಮೇಲೆ ವಸ್ತ್ರವನ್ನು ಹಾಕಿಕೊಂಡು ಕುಳಿತುಕೊಂಡರೆ ಒಂದು ವಿಧವಾದ ದೋಷವೂ ಇರುವುದಿಲ್ಲ. ಆದ್ದರಿಂದಲೇ `ಚೈಲಾಚಿನ ಕುಶೋತ್ತರಂ' ಎನ್ನುತ್ತಾರೆ. ಮನಸ್ಸು ಕೂಡ ಶಾಂತವಾಗಿರುತ್ತದೆ. ದರ್ಭೆ ಹಾಕಿಕೊಳ್ಳುವುದರಿಂದ ಶೈತ್ಯದೋಷ ಬರುವುದಿಲ್ಲ. ಕೃಷ್ಣಾಜಿನ ಸ್ವಲ್ಪ ರಕ್ಷೆ ಕೊಡುತ್ತದೆ. ವಸ್ತ್ರ ಹಾಕಿಕೊಂಡು ಅದರ ಮೇಲೆ ಕುಳಿತುಕೊಂಡರೆ ಇನ್ನೂ ವಿಶೇಷ. ಹೀಗೆ ಆಸನದ ಮೇಲೆ ಕುಳಿತು ಏನು ಮಾಡಬೇಕು ಎನ್ನುವುದಕ್ಕೆ ಭಗವಂತನು,

`ತತ್ರೈಕಾಗ್ರ್ಯಂ ಮನಃ ಕೃತ್ವಾ ಯತ ಚಿತ್ತೇಂದ್ರಿಯಕ್ರಿಯಃ' ಎಂದನು. ಇಂದ್ರಿಯಗಳು ಇಲ್ಲಿ-ಅಲ್ಲಿ ಅಲುಗಾಡಕೂಡದು. ಇಂದ್ರಿಯಗಳು, ಚಿತ್ತವೂ ನಮ್ಮ ವಶದಲ್ಲಿರಬೇಕು. ದೇಹದ ಕ್ರಿಯೆಯೂ ಹೀಗೆಯೇ ಇರಬೇಕು. ಅಂದರೆ ಶರೀರವನ್ನು ಯಾವಾಗಲೂ ಅಲ್ಲಾಡಿಸುತ್ತಾ ಇರಬಾರದು. ಶರೀರವನ್ನು ಅಲ್ಲಾಡಿಸುತ್ತಿದ್ದರೆ ಚಿತ್ತ ಒಂದೇ ಜಾಗದಲ್ಲಿ ನಿಲ್ಲುವುದಿಲ್ಲ. `ಸಮಂ ಕಾಯಶಿರೋಗ್ರೀವಂ ಧಾರಯನ್ನಚಲಂ ಸ್ಥಿರಃ'-ತಲೆ, ಕತ್ತು, ದೇಹ ಈ ಮೂರೂ ಸಮವಾಗಿರಬೇಕು. ಕೃಷ್ಣನನ್ನು ವರ್ಣಿಸುವಾಗ `ಮೂರು ಜಾಗಗಳಲ್ಲಿ ತಿರುವು ಇರುವವನು' ಎನ್ನುತ್ತಾರೆ. ಧ್ಯಾನ ಮಾಡುವವನು ಹೀಗೆ ಕುಳಿತುಕೊಂಡರೆ ಅವನು ಫೋಟೋವಿಗಾಗಿ ಕುಳಿತುಕೊಂಡಿದ್ದಾನೆಂದೇ ತೋರುವುದು.

`ಸಂಪ್ರೇಕ್ಷ್ಯ ನಾಸಿಕಾಗ್ರಂ ಸ್ವಂ ದಿಶಶ್ಚಾನವಲೋಕಯನ್' ಎಂದರು. ನಾಸಿಕಾಗ್ರವೆಂದರೆ ನಾವು ಮೂಗಿನ ತುದಿಯನ್ನು ನೋಡಬೇಕೆಂಬುದು ಅರ್ಥ. ಹಾಗೆ ಮಾಡಿದರೆ ಅದು ಧ್ಯಾನವೇ ಆಗುತ್ತದೆ. ಮೂಗಿನ ತುದಿಯನ್ನು ನಾವು ನೋಡುತ್ತಿದ್ದರೆ ಹೇಗಿರುತ್ತದೋ ಅದೇ ರೀತಿ ಕಣ್ಣುಗಳು ಸ್ಥಿರವಾಗಿರಬೇಕೆನ್ನುವುದನ್ನು `ನಾಸಿಕಾಗ್ರಂ ಪಶ್ಯನ್ನಿವ' ಎನ್ನುತ್ತಾರೆ. ಅನಂತರ `ಪ್ರಶಾಂತಾತ್ಮಾ ವಿಗತಭೀಃ ಬ್ರಹ್ಮಚಾರಿವ್ರತೇ ಸ್ಥಿತಃ'-ಇದಕ್ಕೆ `ಶಾಂತವಾಗಿಯೂ, ಭಯವಿಲ್ಲದೆಯೂ, ಬ್ರಹ್ಮಚಾರಿ ವ್ರತದಲ್ಲಿ ಸ್ಥಿತವಾಗಿರುವವನಾಗಿಯೂ' ಎಂದು ತಾತ್ಪರ್ಯ.

`ಮನಃ ಸಂಯಮ್ಯ ಮಚ್ಚಿತ್ತೋ ಯುಕ್ತ ಆಸೀತ ಮತ್ಪರಃ-ಮನಸ್ಸನ್ನು ನಿಗ್ರಹಿಸಿ ಹೀಗೆ ಯುಕ್ತನಾದವನು `ಮತ್ಪರ' ನಾಗಿ ಇರಬೇಕು. ನಾವು ಬೇರೆಯವರು ಧ್ಯಾನದಲ್ಲಿರುವುದನ್ನು ನೋಡಿ ಒಂದೆರಡು ದಿನಗಳು ಪ್ರಯತ್ನಪಟ್ಟು ಆಮೇಲೆ ಬಿಟ್ಟುಬಿಡೋಣವೆಂದುಕೊಂಡರೆ ಅದು ಧ್ಯಾನವಾಗುವುದಿಲ್ಲ.

`ಆಹಮೇವ ಉತ್ಕೃಷ್ಟಃ ಯಸ್ಯ ಸಃ ಮತ್ಪರಃ' [ಯಾರಿಗೆ ನಾನೇ (ಪರವಸ್ತು) ಉತ್ತಮವಾಗಿದ್ದೇನೋ ಅವನು ಮತ್ಪರಃ] ಹೀಗೆ ಇರುವವನಿಗೆ ಯಾವ ಫಲ ದೊರೆಯುವುದೆಂದರೆ,

\begin{shloka}
`ಯುಂಜನ್ನೇವಂ ಸದಾಽಽತ್ಮಾನಂ ಯೋಗೀ ನಿಯತಮಾನಸಃ|\\
ಶಾಂತಿಂ ನಿರ್ವಾಣ ಪರಮಾಂ ಮತ್ಸಂಸ್ಥಾಮಧಿಗಚ್ಛತಿ||'\\
\hfill{ಗೀತೆ-\eng{vi. 15}}
\end{shloka}

`ಹೀಗೆ ಮನೋನಿಗ್ರಹವಿರುವ ಯೋಗಿ ಯಾವಾಗಲೂ ಮನಸ್ಸನ್ನು ವಶಪಡಿಸಿಕೊಂಡು ಬಂದರೆ ಮೋಕ್ಷವನ್ನು ಪರಿಣಾಮವಾಗುಳ್ಳ ಶಾಂತಿಯಾದ ನನ್ನ ಸ್ಥಿತಿಯನ್ನು ಪಡೆಯುತ್ತಾನೆಂದು ಕೃಷ್ಣನು ಹೇಳಿದುದು. ಆದ್ದರಿಂದ ದೇವತೆಗಳನ್ನು ಧ್ಯಾನಮಾಡುತ್ತಾ ಬಂದರೆ ಅದು ಸಂಪ್ರಜ್ಞಾತ ಸಮಾಧಿವರೆಗಾದರೂ ಹೋಗಬೇಕು. ಕೆಲವರಿಗೆ ಸಗುಣದಲ್ಲಿರುವ ರುಚಿಯನ್ನು ಮೀರಿ ಅದಕ್ಕೆ ಮೇಲೆ ಹೋಗಬೇಕೆಂದಿರುತ್ತದೆ. ಅವರು ನಿರ್ಗುಣ ಧ್ಯಾನ ಮಾಡಲು ಸಾಧ್ಯವೇ ಎಂದು ಕೇಳುತ್ತಾರೆ. ಹಾಗೆ ಮಾಡಬಹುದು. ಹೇಗೆ ಮಾಡಬೇಕು?

ಮೊದಲು ಒಂದು ನಿಮಿಷ ಪ್ರಕಾಶವನ್ನು ನೋಡಬೇಕು. ಹಾಗೆ ನೋಡುವಾಗ ಪ್ರಪಂಚ ಮರೆತು ಹೋಗುತ್ತದೆ. ಮರೆತು ಹೋದಾಗ `ನಾನು ಮರೆಯುತ್ತಿದ್ದೇನೆ' ಎನ್ನುವ ಅರಿವು ಇರುವುದಿಲ್ಲ. ಹಾಗೆ ಮರೆತಿರುವ ಸಮಯದಲ್ಲಿ ಕಣ್ಣುಗಳನ್ನು ಮುಚ್ಚಿಕೊಳ್ಳಬೇಕು. ಅನಂತರ ಎರಡು ಕಣ್ಣುಗಳೂ ಅಂತರ್ಮುಖವಾಗಿ ಹೋಗುತ್ತಿವೆಯೆಂಬುದಾಗಿ ಚಿಂತನೆ ಮಾಡಬೇಕು. ಆಗ ಒಂದು ಪ್ರಕಾಶ ತಾನಾಗಿಯೇ ತೋರುತ್ತದೆ. ಆ ಪ್ರಕಾಶ ನಾವು ಲೋಕದಲ್ಲಿ ನೋಡುವ ಪ್ರಕಾಶದಂತೆ ಇರದೆ ನೀಲಿಯಾಗಿರುತ್ತದೆ. ಕೆಲವು ವೇಳೆ ಚಂದ್ರನ ಪ್ರಕಾಶದಂತೆ ಇರುತ್ತದೆ. ಆ ಪ್ರಕಾಶ ತೋರಿದೊಡನೆ ಕಣ್ಣುಗಳು ಒಳಗೆ ನೋಡುತ್ತಿವೆಯೆಂದು ಭಾವಿಸಿ ನಾನು ಶುದ್ಧ ಚೈತನ್ಯನಾಗಿರುವವನು ಎನ್ನುವ ಭಾವವನ್ನು ಸ್ಥಿರಪಡಿಸಿಕೊಳ್ಳಬೇಕು. `ಚೈತ್ಯಂ ಸರ್ವವ್ಯಾಪಿ' ಎಂದು ಕಾಣಬೇಕು. ಹೀಗೆ ಮಾಡುತ್ತಾ ಬಂದರೆ ಅದು ಅಸಂಪ್ರಜ್ಞಾತ ಸಮಾಧಿಯಲ್ಲಿ ಕೊನೆಗೊಳ್ಳುತ್ತದೆ. ಅದರಿಂದ ಆತ್ಮಸಾಕ್ಷಾತ್ಕಾರ ಪಡೆಯಬಹುದು. ಸಂಪ್ರಜ್ಞಾತ ಸಮಾಧಿ ಪಡೆಯಬೇಕಾದರೆ ಕಣ್ಣುಗಳನ್ನು ಮುಚ್ಚಿಕೊಂಡು ಭಗವಂತನ ಸ್ವರೂಪವನ್ನು ಧ್ಯಾನಮಾಡಬೇಕು. ಹೀಗೆ ಅಭ್ಯಾಸ ಮಾಡಿಕೊಂಡೇ ಬಂದರೆ ಒಂದು ದಿನ ಸ್ವರೂಪ ಎದುರಿಗೆ ಬಂದು ನಿಂತಂತಾಗುತ್ತದೆ. ಕಣ್ಣುಗಳನ್ನು ತೆರೆದರೂ ಕೂಡ ಆ ಸ್ವರೂಪವನ್ನು ಕಾಣಬಹುದು. ಇದನ್ನೇ `ದೇವತಾ ಸಾಕ್ಷಾತ್ಕಾರ'ವೆನ್ನುತ್ತಾರೆ. ಈ ಸ್ಥಿತಿ ಉಂಟಾದ ಮೇಲೆ ಆ ಕಾಲದಲ್ಲಿ ಮನಸ್ಸಿನಲ್ಲಿ ಯಾವುದಾದರೊಂದು ಚಿಂತನೆ ಮಾಡಿದರೆ ಅದಕ್ಕೆ ಉತ್ತರ ಕೂಡ ದೊರೆಯುತ್ತದೆ. ಆ ಉತ್ತರ ದೈವ ವಾಕ್ಕಾಗುತ್ತದೆ. ಹೀಗಿರುವವನು ತನ್ನನ್ನೂ ಇತರರನ್ನೂ ಉದ್ಧಾರ ಮಾಡಬಲ್ಲನು. ಇದನ್ನು ಸಾಧಿಸುವುದರಲ್ಲಿ ಸ್ವಲ್ಪ ವಿಳಂಬವಾದರೂ ಇದನ್ನೂ ಖಂಡಿತ ಸಾಧಿಸಬಹುದು. ಆದ್ದರಿಂದ `ಶಾಂತಿಂ ನಿರ್ವಾಣ ಪರಮಾಂ ಮತ್ಸಂಸ್ಥಾಮಧಿಗಚ್ಛತಿ' ಎಂದು ಕೃಷ್ಣನು ಹೇಳಿದ ರೀತಿಯಲ್ಲಿ, ಅದನ್ನು ಇಷ್ಟ ಪಟ್ಟೇ ನಾವು ಧ್ಯಾನ ಮೊದಲಾದವುಗಳನ್ನು ಮಾಡುತ್ತೇವೆ. ಇದಕ್ಕೆ ಬೇಕಾದ ಸಾಧನ-ಸಂಪತ್ತುಗಳನ್ನು ಇಟ್ಟುಕೊಂಡು ನಾವು ಧ್ಯಾನ ಮಾಡುತ್ತಾ ಬಂದರೆ ಶ್ರೇಯಸ್ಸನ್ನು ಪಡೆಯಬಹುದು.


\newpage

\section{ಕರ್ಮ}

\begin{shloka}
ಗಂಗಾಪೂರಪ್ರಚಲಿತ ಜಟಾಸ್ರಸ್ತ ಭೋಗಿಂದ್ರ ಭೀತಾಂ\\
ಆಲಿಂಗಂತೀಮಚಲತನಯಾಂ ಸಸ್ಮಿತಂ ವೀಕ್ಷಮಾಣಃ|\\
ಲೀಲಾಪಾಂಗೈಃ ಪ್ರಣತಜನತಾಂ ನಂದಯಂಶ್ಚಂದ್ರಮೌಲಿಃ\\
ಮೋಹಧ್ವಾಂತಂ ಹರತು ಪರಮಾನಂದಮೂರ್ತಿಃ ಶಿವೋ ನಃ||
\end{shloka}

ಪರಮಾತ್ಮನಿದ್ದಾನೆ, ಅವನೇ ನಮಗೆ ಸುಖ-ದುಃಖಗಳನ್ನು ಕೊಡುತ್ತಾನೆ ಎಂದು ಒಪ್ಪುತ್ತೇವೆ. ಅವನು ಹೇಗೆ ಸುಖ ದುಃಖಗಳನ್ನು ಕೊಡುತ್ತಾನೆಂದು ಸ್ವಲ್ಪ ಯೋಚಿಸೋಣ. ಭಗವಂತನಿಗೆ ಯಾರ ವಿಷಯದಲ್ಲಿ ಪ್ರೀತಿ ಇದೆಯೋ ಅವರಿಗೆ ಸುಖವನ್ನೂ, ಯಾರ ವಿಷಯದಲ್ಲಿ ಪ್ರೀತಿ ಇಲ್ಲವೋ ಅಂಥಹವರಿಗೆ ದುಃಖವನ್ನೂ ಕೊಡುತ್ತಾನೆ. ಆದರೆ, ಭಗವಂತನಿಗೆ ಯಾವುದರಿಂದ ತಮ್ಮ ವಿಷಯದಲ್ಲಿ ಪ್ರೀತಿಯೋ ಅಪ್ರೇಮವೋ ಉಂಟಾಯಿತು ಎನ್ನುವುದನ್ನು ಮಾತ್ರ ಅವರು ಯೋಚನೆ ಮಾಡುವುದಿಲ್ಲ. ಇದಕ್ಕೆ ಸಂಬಂಧಪಟ್ಟಂತೆ ನೀಲಕಂಠ ದೀಕ್ಷಿತರು ಒಂದು ಕಡೆ-

\begin{shloka}
ಸಾಧ್ಯಾ ಶಂಭೋಃ ಕಥಮಪಿ ದಯೇತ್ಯಪ್ಯಸಾಧ್ಯೊಪದೇಶಃ\\
ಕೋಪಂ ತಸ್ಯ ಪ್ರಥಮಮಪನುತ್ಯೈವ ಸಾಧ್ಯಃ ಪ್ರಸಾದಃ|\\
ಕೋಪೋ ವರ್ಣಾಶ್ರಮ ನಿಯಮಿತಾಚಾರ ನಿರ್ಲಂಘನೋತ್ಥಃ\\
ಶಾಂತಿಂ ನೇಯಃ ಸ ಕಥಮಧುನಾಪ್ಯವ್ಯವಸ್ಥಾ ಪ್ರವೃತ್ತೈಃ||
\end{shloka}

(ಪರಮೇಶ್ವರನ ದಯೆಯನ್ನು ಹೇಗಾದರೂ ಪಡೆಯಬೇಕೆನ್ನುವುದು ಸಾಧ್ಯವಾದ ಉಪದೇಶ. ಪರಮೇಶ್ವರನಿಗೆ ನಮ್ಮ ಮೇಲಿರುವ ಕೋಪ ದೂರವಾದ ಮೇಲೆ ತಾನೆ ಅವನ ಅನುಗ್ರಹ ನಾವು ಪಡೆಯಲು ಸಾಧ್ಯ. ಆ ಕೋಪ, ವರ್ಣ ಮತ್ತು ಆಶ್ರಮಕ್ಕೆ ಸಂಬಂಧಪಟ್ಟ ಆಚಾರ-ನಿಯಮಗಳನ್ನು ಉಲ್ಲಂಘಿಸಿದ್ದರಿಂದ ಉಂಟಾದುದು. ಆ ಕೋಪವನ್ನು ಹೇಗೆ ಶಾಂತಪಡಿಸುವುದು?) ಎಂದು ನುಡಿದರು. ಭಗವಂತನಿಗೆ, ತಮ್ಮ ವಿಷಯದಲ್ಲಿ ಕೋಪವಿರುವುದಾದರೆ ಅವನ್ನು ದೂರಮಾಡಬೇಕೆಂದೇ ಎಲ್ಲರೂ ಭಾವಿಸುತ್ತಾರೆ. ಕೋಪವನ್ನು ದೂರ ಮಾಡಬೇಕಾದರೆ ಮೊದಲು, ಕೋಪ ಏತಕ್ಕೆ ಬಂದಿದೆಯೆಂದು ನೋಡಬೇಕು. ಭಗವಂತನು ಒಂದು ನಿಮಿಷ ಕೋಪದಿಂದಲೂ, ಮತ್ತೊಂದು ನಿಮಿಷ ಬೇರೆ ವಿಧವಾಗಿಯೂ ಇರುವ ಹುಚ್ಚನಂತೆ ಅಲ್ಲ. ಅವನು ಸರ್ವಜ್ಞ. ಹಾಗಿರುವವನ ಅನುಗ್ರಹ ನಮಗೆ ದೊರೆಯಬೇಕಾದರೆ ನಾವು ಆತನ ಕೋಪವನ್ನು ದೂರಮಾಡಬೇಕು.

\begin{shloka}
ಸಾಧ್ಯಾಶಂಭೋಃ ಕಥಮಪಿ ದಯೇತ್ಯಪ್ಯಸಾಧ್ಯೋಪದೇಶಃ
\end{shloka}

ಭಗವಂತನ ಅನುಗ್ರಹವನ್ನು ನಾವು ಪಡೆಯಬಹುದೆನ್ನುವುದು ಅಷ್ಟು ಸುಲಭವಾದ ಕೆಲಸವಲ್ಲ.

\begin{shloka}
ಕೋಪಂ ತಸ್ಯ ಪ್ರಥಮಮಪನುತ್ಯೈವ ಸಾಧ್ಯಃ ಪ್ರಸಾದಃ
\end{shloka}

ಭಗವಂತನಿಗೆ ಇರುವ ಕೋಪವನ್ನು ದೂರಮಾಡಿದರೆ ಅವನ ಪ್ರಸಾದವಾದ ದಯೆ ದೊರೆಯುವುದು. ಏತಕ್ಕೆ ಅವನಿಗೆ ಕೋಪ ಬಂದಿದೆ?

\begin{shloka}
ಕೋಪೋವರ್ಣಾಶ್ರಮ ನಿಯಮಿತಾಚಾರ ನಿರ್ಲಂಘನೋತ್ಥಃ
\end{shloka}

ಪ್ರಪಂಚ ಸರಿಯಾಗಿ ನಡೆಯಬೇಕೆನ್ನುವುದಕ್ಕಾಗಿ ಲೌಕಿಕ ನಿಯಮಗಳು ಇರುವಂತೆ ವೇದಗಳೂ ನಮಗೆ ದಾರಿ ತೋರಿಸುವ ನಿಯಮಗಳೇ ಆಗಿವೆ. ಆಯಾ ಆಶ್ರಮದಲ್ಲಿರುವವರೂ, ತಮ್ಮ ತಮ್ಮ ಸಂಪ್ರದಾಯದಲ್ಲಿರುವವರೂ, ಹೇಗೆ ಬಾಳು ನಡೆಸಬೇಕೆಂದು ವಿಧಿನಿಯಮವಿದೆಯೋ ಹಾಗೆಯೇ ಬಾಳು ನಡೆಸಿದರೆ ಅವರಿಗೆ ಸುಖವಾಗಿರಲು ಸಾಧ್ಯ, ಅವರ ಸುತ್ತ-ಮುತ್ತ ಇರುವವರಿಗೂ ಸುಖವಾಗಿರಲು ಸಾಧ್ಯ. ಹಾಗಿಲ್ಲದೆ ತನಗೆ ಇಷ್ಟಬಂದಂತೆ ಯಾರಾದರೂ ಬಾಳು ನಡೆಸಿದರೆ, ಭಗವಂತನ ಪ್ರಸಾದ ದೊರೆಯುವುದೇ? ಇದಕ್ಕೆ ಬದಲಾಗಿ, ಅವನ ಧರ್ಮಗಳನ್ನು ಮೀರಿದ್ದಕ್ಕಾಗಿ ಭಗವಂತನ ಕೋಪಕ್ಕೆ ಅವನು ಪಾತ್ರನಾಗುತ್ತಾನೆ.

ಒಬ್ಬ ಹುಡುಗ ತನ್ನ ತಂದೆ-ತಾಯಿಯ ಮಾತುಗಳನ್ನು ಕೇಳುವುದೇ ಇಲ್ಲ. ತಂದೆ-ತಾಯಿ ಸ್ಕೂಲಿಗೆ ಹೋಗು ಎಂದರೆ ಹುಡುಗ ಸಿನಿಮಾಗೆ ಹೋಗುತ್ತೇನೆ ಎನ್ನುತ್ತಾನೆ. ಹೀಗೆಲ್ಲ ನಿಯಮಕ್ಕೆ ವಿರೋಧವಾಗಿ ನಡೆದುಕೊಂಡು ತನ್ನ ತಂದೆಗೆ ತನ್ನನ್ನು ಕಂಡರೆ ಆಗುವುದೇ ಇಲ್ಲ ಎಂದು ಹುಡುಗ ಹೇಳಿದರೆ ತಂದೆ ಏನು ಮಾಡುತ್ತಾರೆ? ಹಾಗೆಯೇ ಭಗವಂತನ ನಿಯಮಗಳನ್ನು ನಾವು ಅತಿಕ್ರಮಿಸಿ ನಡೆದುಕೊಂಡು ನಮಗೆ ಭಗವಂತನ ಪ್ರಸಾದ ಬೇಕೆಂದರೆ ಅದು ದೊರೆಯುವುದಿಲ್ಲ. ಆದ್ದರಿಂದಲೇ ಶ್ಲೋಕದಲ್ಲಿ,

\begin{shloka}
ಶಾಂತಿಂ ನೇಯಃ ಸ ಕಥಮಧುನಾಪ್ಯವ್ಯವಸ್ಥಾ ಪ್ರವೃತ್ತೈಃ||
\end{shloka}

ಎಂದು ಹೇಳಲ್ಪಟ್ಟಿದೆ. ಇಷ್ಟು ದಿನಗಳು ಪ್ರಪಂಚವನ್ನು ನೋಡಿ, ಶಾಸ್ತ್ರವನ್ನು ಓದಿ ಸತ್ಪುರುಷರೊಡನೆ ಸಹವಾಸ ಮಾಡಿಯೂ ಕೂಡ ಶಾಸ್ತ್ರ ನಿಯಮಗಳಿಗೆ ಅನುಗುಣವಾಗಿ ಜೀವನ ನಡೆಸದೆ ಇದ್ದರೆ ಹೇಗೆ ನಾವು ಭಗವಂತನ ಕೋಪವನ್ನು ಶಾಂತ ಮಾಡಬಲ್ಲೆವು ಎಂದು ನೀಲಕಂಠ ದೀಕ್ಷಿತರು ಒಂದು ಪ್ರಶ್ನೆ ಒಡ್ಡಿದರು.

\begin{shloka}
ಸಾ ಕಥಂ ಶಾಂತಿಂ ನೇತುಂ ಶಕ್ಯಃ
\end{shloka}

(ಅದನ್ನು ಹೇಗೆ ಸಮಾಧಾನಪಡಿಸುವುದು?)

ಆದ್ದರಿಂದ ನಮ್ಮ ನಮ್ಮ ಆಶ್ರಮಗಳಿಗೆ ಯೋಗ್ಯವಾದ ಧರ್ಮಗಳನ್ನು ನಾವು ಸರಿಯಾದ ರೀತಿಯಲ್ಲಿ ಆಚರಿಸುತ್ತಾ ಇರಬೇಕು.

ವೇದಾಂತ ಶಾಸ್ತ್ರವನ್ನು ಓದಿ ನೋಡಿದರೆ ಪರಮಾತ್ಮನೇ ಜೀವಾತ್ಮನಾಗಿಬಿಟ್ಟನೆಂದು ತಿಳಿಯುತ್ತದೆ. ಯಾವಾಗ ಪರಮಾತ್ಮನು ಜೀವಾತ್ಮನಾದನೆನ್ನುವ ಪ್ರಶ್ನೆಗೆ ನಾವು ಉತ್ತರ ಕೊಡಲು ಸಾಧ್ಯವಿಲ್ಲ. ಯಾವಾಗ ಬೀಜ ಉಂಟಾಯಿತು ಎಂದು ಕೇಳಿದರೆ ಮರ ಉಂಟಾಗುವುದಕ್ಕೆ ಮೊದಲು ಎಂದು ಹೇಳಬಹುದು. ಯಾವಾಗ ಮರ ಉಂಟಾಯಿತು ಎಂದು ಕೇಳಿದರೆ ಅದಕ್ಕೆ ನಾವು ಉತ್ತರ ಹೇಳಬಹುದು. ಮೊದಲು ಬೀಜ ಉಂಟಾಯಿತು ಎಂದು ಹೇಳಿದರೆ ಮರವಿಲ್ಲದೆ ಬೀಜ ಹೇಗೆ ಉಂಟಾಗುತ್ತದೆ ಎನ್ನುವ ಪ್ರಶ್ನೆ ಏಳುತ್ತದೆ. ಮರವೇ ಮೊದಲು ಉಂಟಾಯಿತು ಎಂದರೆ ಬೀಜವಿಲ್ಲದೆ ಮರ ಹೇಗೆ ಉಂಟಾಗುತ್ತದೆ ಎಂದು ಮತ್ತೊಂದು ಪ್ರಶ್ನೆ ಏಳುತ್ತದೆ. ಆದ್ದರಿಂದ ಬೀಜ-ಮರ-ಬೀಜ ಎಂದು ಚಕ್ರದಂತೆ ಅನಾದಿಯಾಗಿ ಬರುತ್ತದೆಯೆಂದು ಮಾತ್ರ ಹೇಳಬಲ್ಲೆವು. ಅದೇ ರೀತಿ ಜೀವನು ಅನಾದಿಯಾಗಿ ಶರೀರಗಳನ್ನು ಗ್ರಹಿಸುತ್ತಿದ್ದಾನೆ. ಅದಕ್ಕೆ ಕಾರಣವಾಗಿ `ಕರ್ಮ' ಹೇಳಲ್ಪಡುವುದು, `ಕರ್ಮ' ಎನ್ನುವುದು ನಮ್ಮ ಕೆಲಸಗಳನ್ನು ಸೂಚಿಸುತ್ತದೆ.

ಕರ್ಮದ ಮಹಿಮೆಯನ್ನು ಕುರಿತು ಹೇಳುವಾಗ ಕವಿ ಒಬ್ಬರು-

\begin{shloka}
`ಬ್ರಹ್ಮಾಯೇನ ಕುಲಾಲವನ್ನಿಯಮಿತಃ ಬ್ರಹ್ಮಾಂಡಭಾಂಡೋದರೇ\\
ವಿಷ್ಣುರ್ಯೇನ ದಶಾವತಾರಗಹನೇ ಕ್ಷಿಪ್ತೋಮಹಾಸಂಕಟೇ|\\
ಶಂಭುರ್ಯೇನ ಕಪಾಲಪಾಣಿ ಪುಟಕೇ ಭೀಕ್ಷಾಟನಂ ಸೇವತೇ\\
ಸೂರ್ಯೇ ಭ್ರಾಮ್ಯತಿ ನಿತ್ಯಮೇವ ಗಗನೇ ತಸ್ಮೈ ನಮಃ ಕರ್ಮಣೇ||'
\end{shloka}

[ಯಾವುದರಿಂದ ಬ್ರಹ್ಮನು ಪ್ರಪಂಚದಲ್ಲಿ (ಸೃಷ್ಟಿಯೆನ್ನುವ ಕೆಲಸದಲ್ಲಿ) ಕುಂಬಾರನಂತೆ ನಿಯಮಿತನಾಗಿದ್ದಾನೋ, ಯಾವುದರಿಂದ ವಿಷ್ಣು ದೊಡ್ಡ ಸಂಕಟವಾದ ಹತ್ತು ಅವತಾರವೆನ್ನುವ ಕಾಡಿನಲ್ಲಿ ಹಾಕಲ್ಪಟ್ಟಿದ್ದಾನೋ, ಯಾವುದರಿಂದ ಪರಮೇಶ್ವರನು ಕೈಯಲ್ಲಿ ಕಪಾಲವನ್ನು ಹಿಡಿದು ಭಿಕ್ಷೆಯನ್ನು ತೆಗೆದುಕೊಳ್ಳುತ್ತಾನೋ ಮತ್ತು ಯಾವುದರಿಂದ ಸೂರ್ಯನು ಪ್ರತಿ ನಿತ್ಯವು ತಿರುಗುತ್ತಲೇ ಇದ್ದಾನೋ ಅಂಥ ಕರ್ಮಕ್ಕೆ ನಮ್ಮ ನಮಸ್ಕಾರ] ಎಂದು ಹೇಳಿದರು. ಆದರೆ, ಕವಿ ಹೇಳುವುದನ್ನೂ ನಾವು ಒಪ್ಪಲಾಗುವುದಿಲ್ಲ. ಅವರು ಕರ್ಮದ ಮಹಿಮೆಯನ್ನು ಹೇಳಬೇಕೆಂದಿರುವುದರಿಂದ ಭಗವಂತನು ಕೂಡ ಕರ್ಮದಿಂದ ಕಷ್ಟಪಟ್ಟನೆಂದು ತೋರಿಸುತ್ತಾರೆ.

ಆದರೆ, ನಿಜಕ್ಕೂ ಭಗವಂತನೂ ಕರ್ಮಕ್ಕೆ ಸಿಕ್ಕಿದವನೇ ಅಲ್ಲ. ಕರ್ಮಕ್ಕೆ  ಸಿಕ್ಕಿಕೊಂಡರೆ ಅವನು ಭಗವಂತನೇ ಅಲ್ಲ. ರಾಮನು ವನವಾಸ ಮಾಡಿದನು. ರಾಮನು ಭಗವಂತನೆಂದೇ ನಾವು ಖಂಡಿತವಾಗಿಯೂ ಹೇಳುತ್ತೇವೆ. ವಾಲ್ಮೀಕಿ, ರಾಮನಿಗೆ ಕರ್ಮವಿದೆ ಎಂದು ಬರೆದಿದ್ದಾರಲ್ಲಾ! ಆದ್ದರಿಂದ ಭಗವಂತನಾದ ರಾಮನು ವನವಾಸದಂಥವುಗಳನ್ನು ಅನುಭವಿಸಬೇಕಾಗಿದ್ದರಿಂದ ಅವನು ಕರ್ಮಕ್ಕೆ ಸಿಕ್ಕಿಕೊಂಡವನೇ ಎಂದು ಅದರ ತಾತ್ಪರ್ಯವನ್ನು ತಿಳಿಯದೆ ಕೆಲವರು ಕೇಳಬಹುದು. ಮನುಷ್ಯ ಜನ್ಮವನ್ನು ಪಡೆದರೆ ಹೀಗೆಲ್ಲಾ ಅನುಭವಿಸಬೇಕೆಂದು ತೋರಿಸುವುದಕ್ಕಾಗಿಯೇ ಭಗವಂತನು ಲೀಲೆ ಮಾಡಿದನು. `ದಶರಥನು ರಾಜ ಅವನ ಮಗ ರಾಮ' ಎಂದು ಸಾಮಾನ್ಯವಾಗಿ ರಾಮಾಯಣ ಓದುವವರಿಗೆ `ರಾಮ ದೊಡ್ಡ ಮಹಾತ್ಮ, ಆದರೂ ಹೀಗೆಲ್ಲ ಕಷ್ಟಪಟ್ಟನು' ಎಂದೇ ತೋರುತ್ತದೆ. ಅಲ್ಲದೆ `ಪರಮಾತ್ಮ ಎಂದು ಹೇಳಿದ ಮೇಲೆ ಅವನಿಗೆ ಇಷ್ಟೆಲ್ಲ ಕಷ್ಟವೇಕೆ?' ಎಂದು ಅವರಿಗೆ ಸಂದೇಹವೂ ಆಗುತ್ತದೆ. ಆದ್ದರಿಂದ ಸ್ವಲ್ಪ ವಿವರಗಳನ್ನು ತಿಳಿದುಕೊಂಡವರಿಗೆ ಹೇಳಿದರೆ ಭಗವಂತನು ತೋರಿದ ಲೀಲೆಯು ಅರ್ಥವಾಗುತ್ತದೆ. ಸರಿ, ಈಗ ಕವಿ ಹೇಳಿರುವ ಶ್ಲೋಕವನ್ನು ನೋಡೋಣ.

\begin{shloka}
`ಬ್ರಹ್ಮಾ ಯೇನ ಕುಲಾಲವನ್ನಿಯಮಿತಃ ಬ್ರಹ್ಮಾಂಡ ಭಾಂಡೋದರೇ'
\end{shloka}

ಬ್ರಹ್ಮನಿಗೆ ಒಂದು ಕೆಲಸವಿದೆ, ಅದೇನೆಂದರೆ ಈ ಪ್ರಪಂಚವನ್ನು ಸೃಷ್ಟಿ ಮಾಡುವುದೇ ಬ್ರಹ್ಮನ ಕೆಲಸ. ಕುಂಬಾರನಿಗೆ ಮಡಕೆಗಳನ್ನು ತಯಾರು ಮಾಡುವುದು ಕೆಲಸವೆಂದು ನಾವು ಹೇಳುವಂತೆ ಈ ಪ್ರಪಂಚವೆಲ್ಲವನ್ನೂ, ಸೃಷ್ಟಿಸುವುದು ಬ್ರಹ್ಮನ ಕೆಲಸ. ಬ್ರಹ್ಮನಿಗೆ ಏತಕ್ಕೆ ಈ ಶಿಕ್ಷೆ ಬಂದಿತೆಂದರೆ, ಅದಕ್ಕೆ ಅವನ ಕರ್ಮವೇ ಕಾರಣವೆನ್ನುತ್ತಾರೆ ಕವಿ. ಭಗವಂತನು ಗೀತೆಯಲ್ಲಿ-

\begin{shloka}
`ಅಜೋಽಪಿ ಸನ್ನವ್ಯಯತ್ಮಾ ಭೂತಾನಾಮೀಶ್ವರೋಽಪಿ ಸನ್|'\\
ಪ್ರಕೃತಿಂ ಸ್ವಾಮಧಿಷ್ಠಾಯ ಸಂಭವಾಮ್ಯಾತ್ಮ ಮಾಯಯಾ||
\end{shloka}

(ನಾನು ಅಜನಾದರೂ, ಅವ್ಯಯನಾದರೂ, ಎಲ್ಲ ಜೀವರಿಗೂ ಈಶ್ವರನಾದರೂ ನನ್ನ ಪ್ರಕೃತಿಯನ್ನು ವಶದಲ್ಲಿಟ್ಟುಕೊಂಡು, ನನ್ನ ಮಾಯೆಯಿಂದಲೇ ಅವತರಿಸುವೆನು) ಎಂದಿದ್ದಾನೆ. ನಾವೂ ಹುಟ್ಟುತ್ತೇವೆ, ಭಗವಂತನೂ ಅವತಾರ ಮಾಡುತ್ತಾನೆ. ಭಗವಂತನು ತನಗೆ ಬೇಕೆಂದು ತೋರಿದರೆ ಮಾತ್ರ ಅವತಾರ ಮಾಡುತ್ತಾನೆ. ನಮ್ಮ ವಿಷಯ ಹಾಗಲ್ಲ. ನಾವೆಲ್ಲರೂ ಹುಟ್ಟಿ ಏಕೆ ಹುಟ್ಟಿದೆವೋ ಎಂದು ಯೋಚನೆ ಮಾಡುತ್ತೇವೆ. ಆದರೆ ಭಗವಂತನು ತಾನು ಅವತಾರ ಮಾಡಬೇಕೇ, ಬೇಡವೇ ಎನ್ನುವುದನ್ನು ಮೊದಲೇ ಯೋಚಿಸುತ್ತಾನೆ. ಯಾವುದಾದರೂ ಲೀಲೆಯನ್ನು ತೋರಿಸಬೇಕೆಂದರೆ ಅವನು ಅವತಾರ ಮಾಡುತ್ತಾನೆ. ಈ ಸ್ಥಿತಿಯಲ್ಲಿ ಕರ್ಮದ ಮಹಿಮೆಯನ್ನು ಹೇಳಬೇಕೆಂದಿರುವ ಕವಿ, ಏನು ಹೇಳುತ್ತಾರೆಂದು ನೋಡೋಣ,

\begin{shloka}
`ವಿಷ್ಣುರ್ಯೇನ ದಶಾವತಾರಗಹನೇ ಕ್ಷಿಪ್ತೋ ಮಹಾಸಂಕಟೇ'
\end{shloka}

ಭಗವಾನ್ ಮಹಾವಿಷ್ಣು ದಶಾವತಾರಗಳನ್ನು ಮಾಡಿದನು. ಆ ಹತ್ತು ಅವತಾರಗಳೂ ಸಾಮಾನ್ಯವಾದ ಅವತಾರಗಳಲ್ಲ. ಅವುಗಳಲ್ಲಿ ಒಂದೊಂದರಲ್ಲಿಯೂ ಮಹಾಕಾರ್ಯಗಳನ್ನು ಅವನು ಸಾಧಿಸಿದನು. ಮತ್ಸ್ಯಾವತಾರ ಮಾಡಿದಾಗ ವೇದಗಳೆಲ್ಲವನ್ನೂ ಬ್ರಹ್ಮದೇವನಿಂದ ಹಯಗ್ರೀವನು ಅಪಹರಿಸಿದಾಗ ಭಗವಂತನು ಹಯಗ್ರೀವನನ್ನು ಕೊಂದು ವೇದಗಳನ್ನು ಮುಂದಿನ ಕಲ್ಪದಲ್ಲಿ ನಮಗೆ ಕೊಟ್ಟನು. ಅನಂತರ ಕೂರ್ಮಾವತಾರ ಮಾಡಿದಾಗ, ದೇವಾಸುರರು ಕ್ಷೀರ ಸಮುದ್ರವನ್ನು ಕಡೆದಾಗ, ಮಂದರ ಪರ್ವತವನ್ನು ಹೊತ್ತು, ಆಮೆಯ ರೂಪವನ್ನು ಧರಿಸಿದನು. ಅದರಿಂದ ದೇವತೆಗಳು ಅಮೃತವನ್ನು ಪಡೆದರು. ರಾಮಾವತಾರ, ಕೃಷ್ಣಾವತಾರಗಳ ಕಥೆಗಳನ್ನು ನಾವು ಕೇಳಿದ್ದೇವೆ. ಹೀಗೆ ಅನೇಕ ಅವತಾರಗಳನ್ನು ಭಗವಂತನು ಮಾಡಿದನು. ಕವಿ `ದಶಾವತಾರ ಗಹನೇ' ಎನ್ನುತ್ತಾರೆ. ಅಂದರೆ ದಶಾವತಾರವೆನ್ನುವ ಸಂಕಷ್ಟಕರವಾದ ಕಾಡಿನಲ್ಲಿ ಭಗವಂತನು ಸಿಕ್ಕಿಕೊಂಡನೆನ್ನುತ್ತಾರೆ. ಕವಿಯ ತಾತ್ಪರ್ಯದ ಪ್ರಕಾರ ಭಗವಂತನ ಈ ಹತ್ತು ಅವತಾರಗಳೂ ಅವನ ಕರ್ಮವೇ ಆಗುತ್ತದೆ. ಭಗವಾನ್ ವಿಷ್ಣು ಮಾಡಿದ ಅವತಾರಗಳಿಗೆ ಒಂದು ಶಾಪವೇ ಕಾರಣವೆಂದು ಒಂದು ಕಥೆಯೂ ಇದೆ. ಕವಿ ಅನಂತರ,

\begin{shloka}
`ಶಂಭುರ್ಯೇನ ಕಪಾಲಪಾಣಿಪುಟಕೇ ಭಿಕ್ಷಾಟನಂ ಸೇವತೇ'
\end{shloka}

ಎಂದು ಹೇಳಿದರು. ಭಗವಂತನಾದ ಪರಮಶಿವನು ಕೈಲಾಸಕ್ಕೆಲ್ಲಾ ಅಧಿಪತಿ. ಅನ್ನಪೂರ್ಣಾ ಅವನ ಹೆಂಡತಿ. ಆದರೂ, ಪರಮಶಿವನೂ `ಭವತಿ ಭಿಕ್ಷಾಂ ದೇಹಿ' ಎಂದು ಕೇಳಿದನಂತೆ. ಅದಕ್ಕೆ ಶನಿ `ಸಮಯ ಬಂದರೆ ನಿನಗೂ ತೊಂದರೆ ಮಾಡುತ್ತೇನೆ' ಎಂದನಂತೆ. ಪರಮಶಿವನು, `ನಿನ್ನ ಕಣ್ಣಿಗೆ ಬಿದ್ದರೆ ತಾನೆ ನೀನು ನನಗೆ ತೊಂದರೆ ಕೊಡುವುದು' ಎಂದು ಹೇಳಿ ತಾವರೆ ಹೂವಿನಲ್ಲಿ ಸಣ್ಣ ದುಂಬಿಯಾಗಿ ಅವಿತುಕೊಂಡನಂತೆ. ಶನಿಗೆ, `ಈಗ ನೀನು ನನಗೆ ಯಾವ ವಿಧವಾದ ತೊಂದರೆಯೂ ಕೊಡಲಾಗುವುದಿಲ್ಲ' ಎಂದನು. ಶನಿ ಪರಮಶಿವನಿಗೆ, `ನೀನು ನನಗಾಗಿ ಬಹಳ ಕಷ್ಟಪಟ್ಟು ಚಿಕ್ಕರೂಪವನ್ನು ಪಡೆದು ಇಷ್ಟು ಹೊತ್ತು ಇದ್ದುದೇ ನಾನು ನಿನಗೆ ಮಾಡಿದ ತೊಂದರೆ' ಎಂದನು.

ಆದ್ದರಿಂದ ಕವಿಯ ತಾತ್ಪರ್ಯದಂತೆ, ಭಗವಂತನಾದ ಪರಮಶಿವನೂ ಭಿಕ್ಷಾಟನೆ ಮಾಡುವುದಕ್ಕೆ ಅವನ ಕರ್ಮವೇ ಕಾರಣ. ಆದರೆ, ಮೊದಲೇ ಹೇಳಿದಂತೆ ನಿಜಕ್ಕೂ ಪರಮಶಿವನಿಗೆ ಕರ್ಮದ ಯಾವ ಬಂಧನವೂ ಇಲ್ಲದಿದ್ದರೂ ಪ್ರಪಂಚದಲ್ಲಿ ಕರ್ಮದ ಪ್ರಾಬಲ್ಯ ಇಷ್ಟಾಗಿದೆಯೆಂದು ತೋರಿಸುವುದಕ್ಕೇ ಅವನು ಈ ರೀತಿ ಲೀಲೆ ಮಾಡಿದನು. ಶ್ಲೋಕದ ಕೊನೆಯ ಚರಣದಲ್ಲಿ ಕವಿ,

\begin{shloka}
`ಸೂರ್ಯೋ ಭ್ರಾಮ್ಯತಿ ನಿತ್ಯಮೇವ ಗಗನೇ'
\end{shloka}

ಎಂದರು. ಎಲ್ಲರಂತೆಯೇ ಸೂರ್ಯನಿಗೆ ಕೂಡ ಏನು ಶಿಕ್ಷೆ? ಪ್ರತಿದಿನವೂ ಭೂಮಂಡಲದ ಸುತ್ತಲೂ ತಿರುಗುವುದು ಸೂರ್ಯನ ಕೆಲಸ. ಏಕೆ ಅವನು ಹೀಗೆ ಸುತ್ತುತ್ತಿರಬೇಕೆಂದರೆ, ಅದಕ್ಕೆ ಅವನ ಕರ್ಮವೇ ಕಾರಣವೆನ್ನುತ್ತಾರೆ ಕವಿ. ಕೊನೆಯಲ್ಲಿ ಕವಿ,

\begin{shloka}
`ತಸ್ಮೈ ನಮಃ ಕರ್ಮಣೇ'
\end{shloka}

ಎನ್ನುತ್ತಾರೆ. ಅಂಥ ಬಲಿಷ್ಠವಾದ ಕರ್ಮವನ್ನು ನಮಸ್ಕರಿಸುತ್ತಾರೆ.

ಕರ್ಮ ಎನ್ನುವುದು ಯಾರನ್ನು ಬಿಟ್ಟಿಲ್ಲ. ಹೇಗೆ ವ್ಯಾಧಿ ವೈದ್ಯನನ್ನೂ ಪೀಡಿಸುತ್ತದೆಯೋ ಅದೇ ರೀತಿ ಕರ್ಮ ಎಲ್ಲರನ್ನೂ ತನ್ನ ಹಿಡಿತದಲ್ಲಿಟ್ಟುಕೊಳ್ಳುತ್ತದೆ. ವೈದ್ಯನು ಕೆಮ್ಮಿಗೆ ಔಷಧಿ ಕೊಡುತ್ತಲೇ ಇರುತ್ತಾನೆ. ಆದರೆ, ಅವನೂ ಕೆಮ್ಮುತ್ತಲೇ ಇರುತ್ತಾನೆ. ಕಾರಣ ಕೇಳಿದರೆ ಎಷ್ಟೋ ಮಂದಿಗೆ ಗುಣಕಾರಿಯಾದ ಆ ಔಷಧಿ ಅವನಿಗೆ ಮಾತ್ರ ಪರಿಣಾಮಕಾರಿಯಾಗಲಿಲ್ಲ ಎನ್ನುತ್ತಾನೆ. ಹಾಗೆ ನಡೆಯುವುದನ್ನು ನಾವೇ ನೋಡಿರಬಹುದು. ಹೀಗೆ ನಡೆಯುವುದಕ್ಕೆ ಅವರವರ ಕರ್ಮವೇ ಕಾರಣ. ವೈದ್ಯನಿಗೂ, ರೋಗಿಗೂ ಆ ವ್ಯಾಧಿಯನ್ನು ಅನುಭವಿಸಬೇಕೆನ್ನುವ ಕರ್ಮವೇ ಇದಕ್ಕೆ ಕಾರಣ. ಇನ್ನು, ಕರ್ಮದ ಸ್ವರೂಪ ನೋಡೋಣ.

ಮನುಷ್ಯನು ಯಾವಾಗಲೂ ಯಾವುದಾದರೂ ಒಂದು ಕೆಲಸ ಮಾಡುವಂತೆಯೇ ತೋರುತ್ತಾನೆ. ಜಾಗ್ರತ್ ಅವಸ್ಥೆಯಲ್ಲಿ ಮನುಷ್ಯನು ಕೆಲಸ ಮಾಡುತ್ತಿದ್ದಾನೆ 


