\chapter*{ಶ್ರೀ ಶ್ರೀ ಜಗದ್ಗುರು ಮಹಾಸ್ವಾಮಿಗಳವರ ಉಪದೇಶ}

\section{ಬಾಳಿನ ದಾರಿ}


\begin{shloka}
ವಿಶುದ್ಧ ಜ್ಞಾನ ದೇಹಾಯ ತ್ರೀವೇದೀ ದಿವ್ಯ ಚಕ್ಷುಷೇ|\\
ಶ್ರೇಯಃ ಪ್ರಾಪ್ತಿನಿಮಿತ್ತಾಯ ನಮಃ ಸೋಮಾರ್ಧಧಾರಿಣೇ||\\
ನಮಾಮಿ ಯಾಮಿನೀನಾಥಲೇಖಾಲಂಕೃತಕುಂತಲಾಮ್|\\
ಭವಾನೀಂ ಭವಸಂತಾಪನಿರ್ವಾಪಣಸುಧಾನದೀಮ್||
\end{shloka}


ದೇವಿಯನ್ನು ಸ್ತುತಿಸುವಾಗ ಆಕೆಯ ಸ್ವರೂಪವನ್ನು ಯಾವ ವಿಧವಾಗಿ ಭಾವಿಸಬೇಕು? `ದೇವಿಯನ್ನು ಪೂಜಿಸಿ ನಾವು ಯಾವ 
ವಿಧವಾದ ಫಲವನ್ನು ಪಡೆಯಬಹುದು' ಎನ್ನುವುದು ಮೇಲೆ ಹೇಳಿದ ಶ್ಲೋಕದಲ್ಲಿ ಸ್ಪಷ್ಟಪಡಿಸಲಾಗಿದೆ. ದೇವಿಗೆ 
ಮಾಲೆ, ಬಳೆ ಇಂತಹ ಅಲಂಕಾರಗಳು ಇದ್ದರೂ ಒಂದು ಅಲಂಕಾರ ವಿಶೇಷವಾಗಿ ಇದೆ. ಅದು ಯಾವುದು ಎನ್ನುವ ಪ್ರಶ್ನೆಗೆ 
ಉತ್ತರವಾಗಿ `ಯಾಮಿನೀನಾಥ ಲೇಖಾಲಂಕೃತ ಕುಂತಲಾಮ್' ಎಂದು ಶ್ಲೋಕದಲ್ಲಿ ಇದೆ. `ಯಾಮಿನೀನಾಥ' 
ಎಂದರೆ ಚಂದ್ರ. `ಲೇಖಾ' ಎಂದರೆ ಕಲೆ. ಆದ್ದರಿಂದ `ಯಾಮಿನೀನಾಥಲೇಖಾಲಂಕೃತ ಕುಂತಲಾಮ್' ಎನ್ನುವುದಕ್ಕೆ 
ಚಂದ್ರಕಲೆಯಿಂದ ಅಲಂಕೃತವಾದ ಶ್ರೀಮುಡಿಯುಳ್ಳವಳು ಎಂದು ಅರ್ಥ. ಈ ಆಭರಣದ ವಿಶೇಷವೇನು? ಈ 
ಪ್ರಪಂಚದಲ್ಲಿ ಯಾರೂ ಚಂದ್ರನನ್ನು ತಲೆಯಲ್ಲಿ ಇಟ್ಟುಕೊಳ್ಳಲಾರರು. ಆದರೆ ದೇವಿ ಚಂದ್ರನನ್ನು ಹೇಗೆ ಇಟ್ಟುಕೊಂಡಿದ್ದಾಳೆ? ಚಂದ್ರ 
ಎಲ್ಲಿ ಇಡಲ್ಪಟ್ಟಿದ್ದಾನೆ ಎಂದು ಹೇಳುವುದರಲ್ಲಿ ಮುಖ್ಯವಾದ ಭಾವವಿದೆ. ಪರಮಶಿವನೂ ದೇವಿಯೂ ಒಟ್ಟಾಗಿ 
ಜ್ಞಾನ ಕಾಯರಾಗಿದ್ದಾರೆ. ಅವರಿಗೆ ಒಂದು ವಿಧವಾದ ಶರೀರ ನಿಜಕ್ಕೂ ಇಲ್ಲ. ಆದರೂ ಭಕ್ತರನ್ನು ಅನುಗ್ರಹಿಸಲು 
ಅವರು ಹಲವು ಶರೀರಗಳನ್ನು ಪಡೆಯುವುದುಂಟು. ಈಶ್ವರನು ಭಕ್ತನನ್ನು ಅನುಗ್ರಹಿಸಲು ಮೊದಲು ಯಾವ ಶರೀರವನ್ನು 
ಪಡೆದನೋ ಅದು ನಮಗೆ ಈಗ ಉಪಾಸನೆಗೆ ಸಹಾಯ ಮಾಡುತ್ತದೆ. ದೇವಿಯ ತಲೆಯ ಮೇಲಿರುವ ಚಂದ್ರನ 
ಭಾವ ಹೀಗಿರುತ್ತದೆ-ಚಂದ್ರ ಪ್ರಕಾಶವನ್ನು ಹರಡುತ್ತಾನೆ. ಅದೇ ವಿಧವಾಗಿ ಜ್ಞಾನ ಪ್ರಕಾಶ ಹರಡುತ್ತದೆ. ಆದ್ದರಿಂದ ದೇವಿ 
ತಲೆಯ ಮೇಲೆ ಚಂದ್ರನನ್ನು ಇಟ್ಟುಕೊಂಡಿದ್ದಾಳೆ ಎನ್ನುವುದಕ್ಕೆ ಆಕೆ ಜ್ಞಾನದಿಂದ ಕೂಡಿದ್ದಾಳೆ ಎನ್ನುವುದು ಭಾವ. ಹೀಗೆ 
ನಿಜವಾಗಿಯೂ ಇದ್ದರೂ ಕೂಡ ಸಗುಣೋಪಾಸನೆ ಮಾಡುವಾಗ ಆಕೆಯನ್ನು ತಲೆಯ ಮೇಲೆ ಚಂದ್ರನನ್ನು ಇಟ್ಟುಕೊಂಡಂತೆ 
ಸ್ಮರಿಸುವುದು ಆವಶ್ಯಕ. ಚಂದ್ರನನ್ನು ತಲೆಯ ಮೇಲೆ ಇಟ್ಟುಕೊಂಡಿರುವುದರ ಭಾವ, ತಲೆ ಶ್ರೇಷ್ಠವಾದ ಜ್ಞಾನದಿಂದ 
ಕೂಡಿದೆ ಎಂದು. ದೇವಿಯಂತೆಯೇ ಈಶ್ವರನೂ ಕೂಡ `ಸೋಮಾರ್ಧಧಾರಣೇ' ಎಂದರೆ ತಲೆಯಲ್ಲಿ ಚಂದ್ರಕಲೆಯನ್ನು ಧರಿಸಿರುವವನು ಎಂದು ಸ್ತುತಿಸಲ್ಪಟ್ಟಿದ್ದಾನೆ.

ಶ್ಲೋಕದಲ್ಲಿ ಹೇಳಲ್ಪಟ್ಟ ದೇವಿಯ ಎರಡನೆಯ ವಿಶೇಷಣ `ಭವ ಸಂತಾಪ ನಿರ್ವಾಪಣಸುಧಾನದೀಮ್' ಎನ್ನುವುದು. ಈ ಪ್ರಪಂಚದಲ್ಲಿ 
ನಾವು ಹೊಂದುವ ಸುಖ ಶಾಶ್ವತವಾದುದಲ್ಲ. ಅಲ್ಲದೆ, ನಾವು ಪಡೆಯುವ ಸುಖ ದುಃಖಗಳಿಂದ ಕೂಡಿರುತ್ತದೆ. ದುಃಖವಿಲ್ಲದ 
ಸುಖ ಪ್ರಪಂಚದಲ್ಲಿಲ್ಲ. `ಯೌವನಂ ಜರಯಾಗ್ರಸ್ತಂ' ಎಂದು ಹೇಳಲಾಗಿದೆ. ಯೌವನ ಸುಖ ಪಡೆಯುವುದಕ್ಕೆ ಒಂದು 
ಕಾರಣವಾಗಿದೆ. ಮುಪ್ಪು ಇದನ್ನು ಕಬಳಿಸಿ ಬಿಡುತ್ತದೆ. ಈ ರೀತಿ ಒಂದೊಂದು ಸುಖವೂ ತನ್ನ ಶತ್ರುವಾದ ದುಃಖದಿಂದ 
ನಾಶವಾಗುತ್ತದೆ. ಪರಿಶೀಲಿಸಿ ನೋಡಿದರೆ ನಾವು ಯಾವುದನ್ನು ಸುಖವೆಂದು ಭಾವಿಸುತ್ತೇವೋ 
ಹಾಗೂ ಯಾವುದನ್ನು ದುಃಖವೆಂದು ಭಾವಿಸುತ್ತೇವೋ ಅವೆರಡೂ ದುಃಖವೇ ಆಗುತ್ತವೆಂದು ತಿಳಿಯುತ್ತೇವೆ. ದುಃಖಗಳನ್ನು 
ಯಾವಾಗಲೂ ಅನುಭವಿಸಿದರೆ ನಮಗೆ ಒಂದು `ತಾಪ' ಅಥವಾ ಬಿಸಿ ಉಂಟಾಗುತ್ತದೆ. ಇದನ್ನು ದೂರ 
ಮಾಡುವುದಕ್ಕೆ ದಾರಿ ಇದೆಯೇ? ಹೌದು, ಭವಾನಿಯೇ ಆ ದಾರಿ. ಬೇಸಿಗೆಯಲ್ಲಿ ಬೇಗೆಯಿಂದಾಗಿ 
ನಮಗೆ ಶರೀರದಲ್ಲಿ ತಾಪ ಉಂಟಾಗುತ್ತದೆ. ನಾವು ಹೊಳೆಯಲ್ಲಿ ಸ್ನಾನ ಮಾಡಿದರೆ ದುಃಖ ದೂರವಾಗಿ 
ಆನಂದವಾಗುತ್ತದೆ. ಅದೇ ರೀತಿ ಭವಸಾಗರದಲ್ಲಿ ತಾಪವನ್ನು ಅನುಭವಿಸುತ್ತಿರುವ ನಮಗೆ ದೇವಿ 
ತಾಪವನ್ನು ದೂರ ಮಾಡುವ ಹೊಳೆಯಂತೆ ಇದ್ದಾಳೆ. ಯಾವ ವಿಧವಾದ ಹೊಳೆಯಂತೆ ಇದ್ದಾಳೆ? 
`ಸುಧಾನದೀಮ್'-ಅಮೃತದ ಹೊಳೆಯಂತೆ ಇದ್ದಾಳೆ. ಸಾಮಾನ್ಯವಾದ ಹೊಳೆಯಲ್ಲಿ ಮುಳುಗಿದರೇನೆ 
ನಮಗೆ ಆನಂದವಾಗುತ್ತದೆ ಎಂದ ಮೇಲೆ ಅಮೃತದ ಹೊಳೆಯಲ್ಲಿ ಮುಳುಗಿದರೆ ಆನಂದವೂ, ಶಾಂತಿಯೂ, 
ತೃಪ್ತಿಯೂ ನಾವು ಪಡೆಯಬಹುದು ಎನ್ನುವುದರಲ್ಲಿ ಸಂದೇಹವಾದರೂ ಇದೆಯೇ?

ದುಃಖವನ್ನು ದೂರ ಮಾಡಲು ಪಡೆದ ಪರವಸ್ತುವಿನ ಆಕಾರದ ಹೆಸರೇನು ಎನ್ನುವ ಪ್ರಶ್ನೆಗೆ ಉತ್ತರವಾಗಿ ಶ್ಲೋಕದಲ್ಲಿ 
`ಭವಾನೀಂ' ಎಂದು ತಿಳಿಸಲಾಗಿದೆ. ಅದೇನೆಂದರೆ ಸ್ತ್ರೀ ರೂಪದಲ್ಲಿ ದರ್ಶನವನ್ನು ಕೊಡುವ ಪರವಸ್ತು ಇಲ್ಲಿ `ಭವಾನಿ' ಎಂದು 
ಕರೆಯಲ್ಪಟ್ಟಿದ್ದಾಳೆ. 
ಆಕೆಯನ್ನು ವಂದಿಸಿ ನಾವು ತಾಪದಿಂದ ಬಿಡುಗಡೆ ಪಡೆದು ಜ್ಞಾನವನ್ನೂ, ಶಾಂತಿಯನ್ನೂ ಸುಖವನ್ನೂ ಪಡೆಯುವುದು ಅವಶ್ಯಕವೆನ್ನುವುದು ಶ್ಲೋಕದ ಭಾವ.

ವಿದ್ಯೆಯನ್ನು ಏತಕ್ಕಾಗಿ 
ಕಲಿಯಬೇಕು ಎನ್ನುವ ಪ್ರಶ್ನೆ 
ಕೆಲವರಿಗೆ ಉಂಟಾಗಬಹುದು. ಹುಲಿ, 
ಆನೆಯಂತಹ ಬಲವುಳ್ಳ ಪ್ರಾಣಿಗಳು 
ಜೀವಿಸುತ್ತವೆ. ಅವುಗಳಂತೆಯೇ 
ಏಕೆ ಜೀವಿಸಬಾರದು ಎಂದು ಕೂಡ 
ಕೆಲವರಿಗೆ ತೋರಬಹುದು. 
ಭಗವಂತನು ಬಲವುಳ್ಳ 
ಪ್ರಾಣಿಗಳನ್ನು ಸೃಷ್ಟಿಸಿದರೂ 
ಅವುಗಳು ವಿವೇಕಹೀನತೆಯಿಂದಾಗಿ 
ಭಗವಂತನನ್ನು ಪಡೆದುಕೊಳ್ಳುವ 
ಯೋಗ್ಯತೆಯನ್ನು ಪಡೆದಿಲ್ಲ. 
ಮನುಷ್ಯನೂ ಪಾಮರರಂತೆ ಜೀವಿಸಿ 
ಬಾಳನ್ನು 
ವ್ಯರ್ಥಮಾಡಿಕೊಳ್ಳಬಹುದು. 
ಹಾಗಲ್ಲದೆ ವಿವೇಕದಿಂದ ಬಾಳಿ 
ಪರವಸ್ತುವಿನ ದರ್ಶನವನ್ನು 
ಪಡೆದು ಜನ್ಮ ಸಮುದ್ರದಿಂದ 
ಬಿಡುಗಡೆ ಪಡೆಯಬಹುದು. ಈ 
ಶಕ್ತಿಯನ್ನು ದೇವರು 
ಮನುಷ್ಯನಿಗೆ ಕೊಟ್ಟಿದ್ದಾನೆ. 
ಪ್ರಾಣಿಗಳಿಗೂ ಬುದ್ಧಿ ಇದೆ. 
ಆದರೆ ಅವುಗಳ ಬುದ್ಧಿಗೆ 
ನಿಶ್ಚಯವಾಗಿಯೂ ಒಂದು ಎಲ್ಲೆ 
ಇದೆ. ಇದಕ್ಕೆ ವಿರೋಧವಾಗಿ 
ಮನುಷ್ಯನ ಬುದ್ಧಿಗೆ ಎಲ್ಲೆ 
ಇಲ್ಲ. ಅವನು `ನಾನು ಯಾವ 
ವಿಧವಾಗಿ ಬಾಳಿದರೆ 
ಒಳ್ಳೆಯದನ್ನು ಪಡೆಯುವೆನು' 
ಎಂದು ಯೋಚಿಸಿ ಅದಕ್ಕೆ ಸರಿಯಾದ 
ರೀತಿಯಲ್ಲಿ ಬಾಳುವ 
ಶಕ್ತಿಯುಳ್ಳವನು. `ನಾನು 
ನಿಜಕ್ಕೂ ಯಾರು? ಎಲ್ಲಿಂದ ಬಂದೆ? 
ಎಲ್ಲಿಗೆ ಹೋಗುವೆನು?' ಎಂದು 
ಮನುಷ್ಯನು ಯೋಚಿಸಬಲ್ಲನು. 
ಹೀಗೆ ಚಿಂತಿಸುವಿಕೆಯು 
ಅವನನ್ನು ಇತರ ಪ್ರಾಣಿಗಳಿಂದ 
ಬೇರ್ಪಡಿಸುವುದು. 
ಚಿಂತಿಸುವುದನ್ನೇ ಅಲ್ಲದೆ 
ಜ್ಞಾನವಂತನಾಗಿಯೂ 
ಪರವಸ್ತುವನ್ನು ಪಡೆಯುವ 
ಯೋಗ್ಯತೆಯು ಅವನಿಗೆ ಇದೆ. 
ಆದ್ದರಿಂದ `ಜಂತೂನಾಂ ನರಜನ್ಮ 
ದುರ್ಲಭಂ' ಎಂದು ಹೇಳಲಾಗಿದೆ, 
ಅಂದರೆ ಜಂತುಗಳಾಗಿ 
ಹುಟ್ಟುವುದಕ್ಕಿಂತ ಮನುಷ್ಯ 
ಜನ್ಮ ಹೆಚ್ಚು ಎಂದು ಭಾವ. 
ಪ್ರಾಣಿಗಳು ತಮ್ಮ ಬುದ್ಧಿಯಿಂದ 
ಯಾವುದನ್ನು ಹುಡುಕುತ್ತವೆ? ಈ 
ಪ್ರಶ್ನೆಗೆ ವಿಶ್ರಾಂತಿ, ಆಹಾರ, ತಿರುಗುವಿಕೆ ಎಂದೇ ಉತ್ತರವಾಗುತ್ತದೆ.

ಮನುಷ್ಯನೂ ಇವುಗಳಲ್ಲಿಯೇ 
ತೊಡಗಿರುವವನಾದರೆ ಅವನಿಗೂ 
ಪ್ರಾಣಿಗಳಿಗೂ ವ್ಯತ್ಯಾಸವೇ 
ಇರುವುದಿಲ್ಲ. ಹಾಗಿದ್ದರೆ 
ಮನುಷ್ಯ ಜನ್ಮ ವ್ಯರ್ಥ 
ಮಾಡಿಕೊಂಡಂತೇ ಸರಿ. ಆದ್ದರಿಂದ 
ನಾವು ಆ ರೀತಿ ಇರಬಾರದು. ನಾವು 
ಜ್ಞಾನ ಪಡೆಯಲು ಖಂಡಿತ 
ಪ್ರಯತ್ನಪಡಬೇಕು. ಜ್ಞಾನ 
ಪಡೆಯುವುದು ಹೇಗೆ? ಕೆಲವರು 
ತಾವಾಗಿಯೇ, ಗುರುವಿನ 
ಉಪದೇಶವಿಲ್ಲದೆ, ವಿದ್ಯೆ  
ಪಡೆದಿದ್ದೇವೆಂದು 
ಹೇಳುತ್ತಾರೆ. ಆದರೆ ಹಾಗೆ 
ಮಾಡುವುದು ಸರಿಯಲ್ಲ. ನಾವು ಈಗ 
ಮಾತನಾಡುತ್ತೇವೆ. ನಾವು 
ಚಿಕ್ಕವರಾಗಿದ್ದಾಗ ನಮ್ಮ 
ತಾಯಿಯೋ, ತಂದೆಯೋ, 
ಅಲ್ಲಿದ್ದವರೋ ಮಾತನಾಡದೆ 
ಇದ್ದಿದ್ದರೆ ನಮಗೆ ಮಾತನಾಡಲು 
ಬರುತ್ತಿರಲಿಲ್ಲ. ಆದ್ದರಿಂದ 
ಅವರ ಮಾತುಗಳನ್ನು ಕೇಳಿಯೇ 
ನಾವು ಮಾತನಾಡುವುದನ್ನು 
ಕಲಿತುಕೊಂಡಿದ್ದೇವೆ. `ಯಾರ 
ಸಹಾಯವೂ ಇಲ್ಲದೆ ನಾನು ದೊಡ್ಡ 
ವಿದ್ವಾಂಸನಾಗಿ ಬಿಟ್ಟೆ' ಎಂದು 
ಹೇಳಿಕೊಂಡರೆ ಅದು ಸುಳ್ಳೇ ಆಗುತ್ತದೆ. ನಾವು ಕಲಿಯಬೇಕಾದುವು ಅನೇಕವಿವೆ. ಹೇಗೆ?

\begin{shloka}
ಜಾತ್ಯಂಧಾ ಜಾತಿಬಧಿರಾ ಜಾತಿಮೂಕಾಶ್ಚ ತೇ ಜನಾಃ|\\
ಸಮ್ಯಗಾರಾಧಿತಾ ಯೈರ್ನ ಸಂತೋ ವಿಜ್ಞಾನ ಚಿಂತವಃ||
\end{shloka}

ಎಂದು ಹೇಳಲ್ಪಟ್ಟಿದೆ. ಯಾರು 
ಜ್ಞಾನ ಸಮುದ್ರವಾಗಿ 
ಪ್ರಕಾಶಿಸುವ ಮಹಾತ್ಮರ ದರ್ಶನ 
ಪಡೆಯಲಿಲ್ಲವೋ, ಅವರ 
ಉಪದೇಶವನ್ನು ಕೇಳಲಿಲ್ಲವೋ, 
ಅವರ ಮಾತಿನಂತೆ ಬಾಳನ್ನು 
ತಿದ್ದುಕೊಳ್ಳಲಿಲ್ಲವೋ, 
ಅಂಥಹವರು ಹುಟ್ಟಿನಿಂದಲೇ 
ಕುರುಡರು; ಹುಟ್ಟಿನಿಂದಲೇ 
ಕಿವುಡರು; ಹುಟ್ಟಿನಿಂದಲೇ 
ಮೂಕರು; ಎಂದು ಅರ್ಥ. ನಾವು ಈ 
ರೀತಿ ಕುರುಡರಾಗಿಯೋ, 
ಕಿವುಡರಾಗಿಯೋ, ಮೂಕರಾಗಿಯೋ, 
ಇರಬಾರದು. ನಾವು ಜ್ಞಾನ 
ಸಮುದ್ರರಾಗಿ ಪ್ರಕಾಶಿಸುವ ಮಹಾತ್ಮರನ್ನು ಪೂಜಿಸಿ ಜ್ಞಾನ ಪಡೆಯಬೇಕು.

ಸುರಂಗದಿಂದ ವಜ್ರವನ್ನು 
ತೆಗೆದರೆ ಅದು ವಜ್ರವಾದರೂ ಕೂಡ 
ಹೊಳೆಯುವುದಿಲ್ಲ. ಅದನ್ನು 
ಸಾಣೆ ಹಿಡಿದರೆ ಅದು ಬಹಳವಾಗಿ 
ಪ್ರಕಾಶಿಸುತ್ತದೆ. 
ಸಾಮಾನ್ಯವಾದ ಕಲ್ಲನ್ನು ಸಾಣೆ 
ಹಿಡಿದರೆ ಅದು ಸವೆದು 
ಹೋಗುತ್ತದೆ. ಅಲ್ಲದೆ, ವಜ್ರ 
ಸಹಜವಾಗಿ ಸಿಕ್ಕಿದರೂ ಅದನ್ನು 
ಹರಿತ ಮಾಡುವವರೆಗೆ ಗಾಜನ್ನು 
ಕಡಿಯಲು ಅದರಿಂದ 
ಸಾಧ್ಯವಾಗುವುದಿಲ್ಲ. ಅದೇ 
ರೀತಿ ಮನುಷ್ಯನು ಸಹಜವಾಗಿಯೇ 
ಸ್ವಲ್ಪ ಮಟ್ಟಿಗೆ ಜ್ಞಾನಿ. 
ಶಾಸ್ತ್ರಗಳು ಮತ್ತು ಗುರುವಿನ 
ಉಪದೇಶದಿಂದ ವಿಶೇಷವಾಗಿ ಜ್ಞಾನ 
ಉಂಟಾಗುತ್ತದೆ. ಇದು ಸಾಣೆ ಹಿಡಿದ ವಜ್ರದಂತೆ ಪ್ರಕಾಶಿಸುತ್ತದೆ.

\begin{shloka}
ಏಕಂ ಹಿ ಚಕ್ಷುರಮಲಂ ಸಹಜಾವಬೋಧಃ\\
ವಿದ್ವದ್ಭಿರೇವ ಸಹ ಸಂವಸತಿರ್ದ್ವಿತೀಯಮ್|\\
ಯಸ್ಯಾಸ್ತಿ ನ ದ್ವಯಮಿದಂ ಸ್ಫುಟಮೇವ ಸೋಽನ್ತಃ\\
ತಸ್ಯಾಪ್ಯಮಾರ್ಗಚಲನೇ ವದ ಕೋಽಪರಾಧಃ||
\end{shloka}

ಎಂದು ಒಬ್ಬರು ಹೇಳಿದರು. 
ಮನುಷ್ಯನಿಗೆ ಎರಡು 
ಕಣ್ಣುಗಳಿವೆ. ಅವುಗಳಲ್ಲಿ 
ಒಂದು ಸಾಮಾನ್ಯ 
ಬುದ್ಧಿಯುಳ್ಳದ್ದು. ಕೆಲವರಿಗೆ 
ಸಾಮಾನ್ಯ ಬುದ್ಧಿಯೇ 
ಇರುವುದಿಲ್ಲ. ಅವರಿಗೆ 
ಹೇಳಿಕೊಟ್ಟರೂ 
ತಿಳಿಯುವುದಿಲ್ಲ. ಹಾಗಿದ್ದರೆ 
ಕಷ್ಟ. `ವಿದ್ವದ್ಭಿರೇವ ಸಹ 
ಸಂವಸತಿಃ 
ದ್ವಿತೀಯಮ್'-`ವಿದ್ವದ್ಭಿಃ' 
ವಿದ್ವಾಂಸರು ಅಂದರೆ ಜ್ಞಾನ 
ಸಮುದ್ರರಾಗಿ ಪ್ರಕಾಶಿಸುವ 
ಮಹಾತ್ಮರ ಸಹವಾಸವೇ ಆವಶ್ಯಕ. 
ಅವರನ್ನು ಸೇವಿಸಿ ಅವರು ಮಾಡುವ 
ಉಪದೇಶವನ್ನು ಕೇಳಿ, 
ಮನಸಿನಲ್ಲಿಟ್ಟುಕೊಂಡು, ಆ 
ರೀತಿ ನಡೆಯುತ್ತಾ ಜ್ಞಾನ 
ಪಡೆಯುವುದು ಎರಡನೆಯ 
ಕಣ್ಣಾಗುತ್ತದೆ. ಎರಡು 
ಕಣ್ಣುಗಳೂ ಮನುಷ್ಯನಿಗೆ ಬೇಕು. 
ಸಾಮಾನ್ಯ ಬುದ್ಧಿ ಇಲ್ಲದೆ 
ಇದ್ದು ಕೇವಲ ಗುರು 
ಹೇಳಿದ್ದನ್ನು ಕೇಳಿಕೊಂಡು 
ಇರಬಾರದೆ ಎಂದರೆ ಅದಕ್ಕೆ ಒಂದು 
ಕಥೆ ನೆನಪಿಗೆ ಬರುತ್ತದೆ. 
ನಾಲ್ಕು ಮಂದಿ ಒಬ್ಬ ಗುರುವಿನ 
ಹತ್ತಿರ ಮಂತ್ರಶಾಸ್ತ್ರವನ್ನು 
ಕಲಿತುಕೊಂಡರು. ಪಾಠ ಮುಗಿದ 
ಮೇಲೆ, ತಮ್ಮ ತಮ್ಮ ಮನೆಗಳಿಗೆ 
ಹೋಗುವ ದಾರಿಯಲ್ಲಿ ಒಂದು ಕಾಡು 
ಇದ್ದಿತು. ದಾರಿಯಲ್ಲಿ ಸತ್ತು 
ಹೋದ ಒಂದು ಹುಲಿಯ ಶವವನ್ನು 
ನೋಡಿದರು. ಕೆಲವರ 
ಮನಸ್ಸಿನಲ್ಲಿ `ನಾವು ಶವಕ್ಕೆ 
ಜೀವ ಕೊಡುವ ಸಂಜೀವಿನಿ 
ಮಂತ್ರವನ್ನು 
ಕಲಿತಿದ್ದೇವಲ್ಲಾ, ಅದನ್ನು 
ಪರೀಕ್ಷಿಸಬೇಕು' ಎನ್ನುವ 
ವಿಚಾರ ಬಂದಿತು. ಆದರೆ 
ನಾಲ್ಕನೆಯ ವಿದ್ಯಾರ್ಥಿಗೆ ಅದು 
ಅಪಾಯವನ್ನುಂಟುಮಾಡುತ್ತದೆಂದು ತೋರಿತು. ಅವನು ಪ್ರಯತ್ನ ಪಡದೆ ಇದ್ದರೂ, ಇತರರು ಅವನು ಕೊಟ್ಟ ಎಚ್ಚರಿಕೆಯನ್ನು ಗಮನಿಸಲಿಲ್ಲ. ಆದ್ದರಿಂದ ಅವನು ಸ್ವಲ್ಪ ದೂರ ಸರಿದು ಒಂದು ಮರವನ್ನು ಹತ್ತಿ ಕುಳಿತುಕೊಂಡನು. ಇತರ ಮೂರು ಮಂದಿ ತಾವು ಕಲಿತುಕೊಂಡಿದ್ದ ಮಂತ್ರದ ಶಕ್ತಿಯನ್ನು ತಿಳಿಯುವುದಕ್ಕಾಗಿ ಹುಲಿಗೆ ಜೀವಕೊಟ್ಟರು. ಉತ್ತಮವಾದ ಕೆಲಸವನ್ನು ಮಾಡಿದರು! ಏಕೆಂದರೆ ಬದುಕಿದ ಹುಲಿ ಅವರನ್ನು ತನ್ನ ಹಸಿವು ತೀರಿಸಿ ಕೊಳ್ಳುವುದಕ್ಕಾಗಿ ಕೊಂದು ಬಿಟ್ಟಿತು. ಈ ಮೂರು ಮಂದಿ ವಿದ್ಯಾರ್ಥಿಗಳು ಗುರುವಿನ ಬಳಿ ಚೆನ್ನಾಗಿ ಮಂತ್ರಶಾಸ್ತ್ರವನ್ನು ಕಲಿತಿದ್ದರೂ ಸಾಮಾನ್ಯ ಬುದ್ಧಿ ಇಲ್ಲದ್ದರಿಂದ ಆಪತ್ತಿಗೆ ಒಳಗಾದರು. 

ಇದರಿಂದ ಎರಡು ವಿಧವಾದ ಜ್ಞಾನ 
ಪ್ರತಿಯೊಬ್ಬನಿಗೂ 
ಆವಶ್ಯಕವೆಂದು ತಿಳಿಯುತ್ತದೆ. 
ಒಂದು ಕತ್ತಿ ಇದೆ. ಅದನ್ನು 
ಹರಿತ ಮಾಡದೆ ಇದ್ದರೆ 
ಕಡಿಯಲಾಗುವುದಿಲ್ಲ. ಅದೇ ರೀತಿ 
ಗುರುವಿನ ಬಳಿ ಕಲಿಯದಿದ್ದರೆ 
ನಮ್ಮ ವಿದ್ಯೆ 
ಪ್ರಕಾಶಿಸುವುದಿಲ್ಲ. ಯಾರಿಗೆ 
ಈ ವಿಧವಾದ ಎರಡು ಜ್ಞಾನಗಳು 
ಇಲ್ಲವೋ ಅವನು ಕುರುಡನಂತೆ. 
ಅವನು ತಪ್ಪು ಮಾಡಿದರೆ 
ಆಶ್ಚರ್ಯ ಪಡಬೇಕಾಗಿಲ್ಲ. ನಾವು 
ತಪ್ಪು ಮಾಡದೆ ಬಾಳಬೇಕಾದರೆ ಎರಡು ವಿಧ ಜ್ಞಾನವನ್ನು ಪಡೆಯಬೇಕು.

ವಿದ್ಯೆಯನ್ನು ಯಾವ 
ವಯಸ್ಸಿನಲ್ಲಿ ಕಲಿಯಬೇಕು? 
ಕೆಲವರು `ನಾಳೆ ಕಲಿಯೋಣ, 
ವಯಸ್ಸಾದ ಮೇಲೆ ವಿದ್ಯೆ 
ಕಲಿಯೋಣ' ಎಂದು ಮುಂದಕ್ಕೆ 
ಹಾಕಿಕೊಂಡೇ ಹೋದರೆ 
ಪ್ರಯತ್ನಗಳು ಫಲಿಸುವುದಿಲ್ಲ. 
ಸಂಸ್ಕೃತದಲ್ಲಿ ವಿದ್ಯೆ 
ಕಲಿಯಬೇಕಾದ ವಯಸ್ಸನ್ನು 
ಕುರಿತು `ಲಾಲಯೇತ್‌ಪಂಚ 
ವರ್ಷಾಣಿ ದಶ ವರ್ಷಾಣಿ 
ದಂಡಯೇತ್' ಎಂದು ಹೇಳಲಾಗಿದೆ. 
ಮೊದಲು ಐದು ವರ್ಷಗಳು 
ಮಕ್ಕಳನ್ನು ಬಹಳ ಪ್ರೀತಿಯಿಂದ 
ಸಾಕಬೇಕು ಏಕೆ? ಏಕೆಂದರೆ, 
ಮಕ್ಕಳಿಗೆ ಶರೀರದಲ್ಲಿ 
ಹೆಚ್ಚಾಗಿ ಬಲವಿರುವುದಿಲ್ಲ. 
ಐದು ವರ್ಷಗಳಾದ ಮೇಲೆ ಅವರಿಗೆ 
ಮಾತನಾಡುವುದಕ್ಕೂ, 
ಕೆಲಸಗಳನ್ನು ಮಾಡುವುದಕ್ಕೂ 
ಶಕ್ತಿ ಇರುತ್ತದೆ. ಆದಾದಮೇಲೆ 
ಹತ್ತು ವರ್ಷಗಳು `ದಶ ವರ್ಷಾಣಿ 
ದಂಡಯೇತ್' ಎಂದರೆ ದೊಣ್ಣೆಯಿಂದ 
ಹೊಡೆಯಬೇಕೆಂದಲ್ಲ. ಆದರೆ 
ಚೆನ್ನಾಗಿ ಉಪದೇಶಿಸಬೇಕೆಂದು 
ಭಾವ. ಒಬ್ಬ ಶಾಲೆಗೆ ಹೋಗುವ 
ವಿದ್ಯಾರ್ಥಿ ಬೆಳಗ್ಗೆ ಐದು 
ಘಂಟೆಯಾದ ಮೇಲೆ ಮಲಗಿರಲು 
ಇಷ್ಟಪಟ್ಟರೆ ತಂದೆ ಅವನನ್ನು 
`ನೀನು ಚೆನ್ನಾಗಿ 
ಓದಿಕೊಳ್ಳಬೇಕು ಈ ವಿಧವಾಗಿ 
ಕಾಲಹರಣ ಮಾಡಬಾರದೆಂದು ತಿಳಿಯ 
ಹೇಳಿ ಎಬ್ಬಿಸಬೇಕು ಅದೇ ರೀತಿ 
ವಿದ್ಯಾಬ್ಯಾಸ ಮಾಡುವ 
ಜಾಗದಲ್ಲೂ ಗುರು 
ವಿದ್ಯಾರ್ಥಿಯನ್ನು ಚೆನ್ನಾಗಿ 
ಗಮನವಿಟ್ಟು ಕಲಿಯುವಂತೆ ಮಾಡಬೇಕು.' ಹೀಗೆ ಮಾಡುವುದೇ ಶಿಕ್ಷೆಯಾಗುತ್ತದೆ.    

ಮೇಲೆ ಹೇಳಿದಂತೆ ನಾವು ಓದಿದರೆ 
ವಿದ್ಯೆಯನ್ನು ಚೆನ್ನಾಗಿ 
ಪಡೆಯಬಹುದು. ವಿದ್ಯೆಯಲ್ಲೂ 
ಅನೇಕ ವಿಧವಿದೆ. ಲೌಕಿಕ ವಿದ್ಯೆ, 
ವೈದಿಕ ವಿದ್ಯೆ ಮತ್ತು 
ಆಧ್ಯಾತ್ಮಿಕ ವಿದ್ಯಾ  ಎಂದು 
ವಿಭಾಗ ಮಾಡುತ್ತಾರೆ. ಭಗವಂತನು 
ಗೀತೆಯಲ್ಲಿ `ಆಧ್ಯಾತ್ಮ 
ವಿದ್ಯಾ ವಿದ್ಯಾ ನಾಂ' ಎಂದು 
ಹೇಳಿದ್ದಾನೆ. ಅಂದರೆ ಅನೇಕ 
ವಿದ್ಯೆಗಳಲ್ಲಿ 
ಶ್ರೇಷ್ಠವಾದುದು ಆಧ್ಯಾತ್ಮ 
ವಿದ್ಯೆ. `ನಾವು ಶ್ರೇಷ್ಠವಾದ 
ವಿದ್ಯೆಯನ್ನಲ್ಲವೇ 
ಕಲಿತುಕೊಳ್ಳಬೇಕು. ಆದ್ದರಿಂದ 
ಆಧ್ಯಾತ್ಮ ವಿದ್ಯೆ ಕಲಿಯೋಣ' 
ಎಂದರೆ ಅದಕ್ಕೆ ಬೇಕಾದ 
ಯೋಗ್ಯತೆಗಳೆಲ್ಲವನ್ನೂ ಪಡೆದೆ ನಾವು ಹಾಗೆ ಕಲಿಯಬಹುದು. ಆದರೆ ಅದು ಕಷ್ಟ.

`ಪರೀಕ್ಷ್ಯ ಲೋಕಾನ್ 
ಕರ್ಮಚಿತಾನ್' ಎಂದು 
ಉಪದೇಶದಲ್ಲಿ ಹೇಳಲಾಗಿದೆ. 
ಅಂದರೆ ಲೋಕವನ್ನು 
ಮನಸ್ಸಿನಲ್ಲಿ ಪರೀಕ್ಷಿಸಿ 
ನಂತರ ಎಲ್ಲವನ್ನೂ 
ಬಿಟ್ಟುಬಿಡಬೇಕೆಂದು ಅರ್ಥ. 
ನಾವು ಈ ರೀತಿ ಪರೀಕ್ಷೆ ಮಾಡದೆ 
ಬಿಟ್ಟು ಬಿಟ್ಟವೆಂದರೆ, 
ಬಿಟ್ಟಿದ್ದು ಫಲವನ್ನು 
ಕೊಡುವುದಿಲ್ಲ. ಏಕೆಂದರೆ 
ಮನಸ್ಸು ಪಕ್ವವಾಗಿರುವುದಿಲ್ಲ. 
ಉದಾಹರಣೆಗೆ - ಆಸೆಯುಳ್ಳವನು 
ಸಂನ್ಯಾಸವನ್ನು 
ತೆಗೆದುಕೊಳ್ಳಬಹುದು. ಆದರೂ 
ಅವನು ಸಂಸ್ಕಾರಗಳ ಬಲದಿಂದಾಗಿ 
ಸಂನ್ಯಾಸವನ್ನು ತೆಗೆದುಕೊಂಡ 
ಮೇಲೂ ಹೋಟಲಿಗೆ ಹೋಗಿ ಊಟ ಮಾಡಬಹುದೇ? ಹಾಗೆ ಸಂನ್ಯಾಸ ತೆಗೆದುಕೊಂಡರೆ ಏನು ಪ್ರಯೋಜನ?

ಆದ್ದರಿಂದ ಮೊದಲು ಲೌಕಿಕ 
ವಿದ್ಯೆ ಕಲಿಯಬೇಕು. ಲೌಕಿಕವಾದ 
ವಿದ್ಯೆ ಎಂದರೇನು? ನಾವು ಹೇಗೆ 
ನಡೆದುಕೊಂಡರೆ ಪ್ರಪಂಚಕ್ಕೆ 
ಒಪ್ಪಿಗೆಯಾಗುತ್ತದೋ, ನಾವು 
ಹೇಗೆ ನಡೆದುಕೊಂಡರೆ 
ಪ್ರಪಂಚಕ್ಕೆ ಕಷ್ಟಕೊಟ್ಟಂತೆ 
ಆಗುವುದಿಲ್ಲವೋ, ಅದೇ 
ಲೌಕಿಕವಾದ ವಿದ್ಯೆ. 
ಉದಾಹರಣೆಗೆ ನಾವು ಒಂದು 
ಮನೆಯನ್ನೇ ಕಟ್ಟಿಸಬಹುದು, 
ಹಣವನ್ನೇ ಸಂಪಾದಿಸಬಹುದು 
ರಾಜ್ಯವನ್ನಾಳಬಹುದು, ತಪ್ಪು 
ಮಾಡಿದವರನ್ನು ಶಿಕ್ಷಿಸುವ 
ಸಂದರ್ಭ ಬರಬಹುದು. 
ಹೀಗಿರತಕ್ಕಂತಹ ವಿಷಯಗಳಲ್ಲಿ 
ಅವುಗಳಿಗೆ ತಕ್ಕಂತೆ 
ನಡೆದುಕೊಳ್ಳುವ ಬುದ್ಧಿ 
ಇಲ್ಲದಿದ್ದರೆ ಹೇಗೆ 
ಕೆಲಸಗಳನ್ನು ಮಾಡುವುದು? 
ಆದ್ದರಿಂದಲೇ ನಮ್ಮ ಪೂರ್ವಿಕರು 
ಅರ್ಥಶಾಸ್ತ್ರ, 
ದಂಡನೀತಿಯಂಥವುಗಳನ್ನು 
ಹೇಳಿಕೊಟ್ಟಿದ್ದಾರೆ, ಇಲ್ಲಿ 
ಮೇಲೆ ಹೇಳಿದ ಲೌಕಿಕ ವಿದ್ಯೆ 
ಮಾತ್ರ ನಮಗೆ ಸಾಕಾಗುವುದಿಲ್ಲ. 
ನಾವು ಕೆಲವೊಮ್ಮೆ 
ನೋಡುತ್ತೇವೆ. ನಾವು ಮಾಡುವುದು 
ಒಂದೂ ನಡೆಯುವುದಿಲ್ಲ. ನಾವು 
ಮಾಡುವ ಪ್ರಯತ್ನಗಳು 
ಉತ್ತಮವಾದರೂ ಅವುಗಳು 
ಫಲಿಸುವುದಿಲ್ಲ. ಇದಕ್ಕೆ 
ಕಾರಣವೇನೆಂದರೆ `ದೈವಕೃಪೆ' 
ಇಲ್ಲ ಎನ್ನುವುದು. ಆ ಭಗವಂತನ 
ದಯೆ ಪಡೆಯಲು ನಾವು ಕೆಲವು 
ಅರ್ಚನೆ ಕೆಲಸಗಳನ್ನು 
ಮಾಡಬೇಕು. ದೇವಸ್ಥಾನಕ್ಕೆ 
ಹೋಗಿ ಪೂಜೆ, ಮಾಡುವುದು, 
ಮನೆಯಲ್ಲಿಯೇ ವಿಷ್ಣು 
ಸಹಸ್ರನಾಮವನ್ನೋ 
ಶಿವಸಹಸ್ರನಾಮವನ್ನೋ ಪಾರಾಯಣ 
ಮಾಡುವುದು ಇಂಥವುಗಳನ್ನೂ 
ಮಾಡಬಹುದು. ಹೆಚ್ಚು 
ಬುದ್ಧಿಯುಳ್ಳವರು ಕೆಲವರು 
ಸಂಧ್ಯಾವಂದನೆಯನ್ನೂ, 
ವಿಶೇಷವಾಗಿ ಹೋಮ 
ಮೊದಲಾದವುಗಳನ್ನೂ ಮಾಡುವರು. 
ಇವುಗಳೆಲ್ಲ ವೈದಿಕವಾದ 
ವಿದ್ಯೆ. ಇದು `ಇಹ ಪರ' ವಾದ 
ಸುಖ-ಶಾಂತಿಗಳನ್ನು 
ಕೊಡುವುದಾಗಿ ಇರುವುದು. ಲೌಕಿಕವಿದ್ಯೆ ಕೇವಲ ಇಹ ಫಲ ನೀಡತಕ್ಕದ್ದು.

ಆಧ್ಯಾತ್ಮಿಕ 
ವಿದ್ಯೆಯೆನ್ನುವುದು `ನಾವು 
ಇಷ್ಟೆಲ್ಲಾ 
ಮಾಡುತ್ತಿದ್ದೇವಲ್ಲಾ, ನಾವು 
ಎಷ್ಟು ದಿನಗಳು ಇರುವೆವು? ನಾವು 
ಶಾಶ್ವತವೇ? ಅಥವಾ 
ಶಾಶ್ವತವಲ್ಲವೇ? ಶಾಶ್ವತವಾದರೆ 
ನಾವು ಸಾಯುವುದಾದರೂ ಏಕೆ? 
ಸತ್ತಮೇಲೆ ನಮ್ಮ ರೂಪ 
ಹೇಗಿರುತ್ತದೆ? ಹುಟ್ಟು ಸಾವು 
ಇಲ್ಲದೆ ನಾವು 
ಶಾಶ್ವತವಾಗಿರುವುದಕ್ಕೆ ಯಾವ 
ತತ್ತ್ವವನ್ನು ತಿಳಿಯಬೇಕು?' 
ಇವೆಲ್ಲ ಸೇರುವಂತಹುದು. 
ಇವುಗಳೆಲ್ಲವನ್ನೂ 
ಪಡೆಯುವುದಕ್ಕೆ ಮನುಷ್ಯನ ಜನ್ಮ 
ಉಪಯೋಗವಾದುದರಿಂದ ಮನುಷ್ಯ 
ಜನ್ಮ ಶ್ರೇಷ್ಠವಾದುದು. ಆದರೆ 
ನಾವು ವಿದ್ಯೆ ಕಲಿಯಬೇಕು 
ಎನ್ನುವುದು ಬಹಳ 
ಮುಖ್ಯವಾದುದು. ನಮ್ಮ ಹತ್ತಿರ 
ಬಹಳವಾಗಿ ಔಷಧಗಳು ಇರಬಹುದು. 
ಇವುಗಳಲ್ಲಿ ಯಾವುದಕ್ಕೆ 
ಯಾವುದನ್ನು ಉಪಯೋಗಿಸಿದರೆ ಫಲ 
ಶ್ರೇಷ್ಠವಾಗಿರುತ್ತದೆ 
ಎನ್ನುವ ಯೋಚನೆ ಬೇಕು. ಅದನ್ನು 
ಬೇರೆ ಕೆಲಸಕ್ಕೆ ಉಪಯೋಗಿಸಿದರೆ 
ಅಸಹಾಯವಾಗಬಹುದು. ಅದೇ ರೀತಿ 
ವಿದ್ಯೆಯ ವಿಷಯವೂ ಕೂಡ. ಅಲ್ಲದೆ, 
ವಿದ್ಯೆ ಕಲಿತವನಿಗೆ `ವಿನಯ' 
ಎಂದು ಹೇಳಲಾಗುವ ಸಂಯಮ ಅಥವಾ 
ಗುಣ ಇರಬೇಕು. `ಶೀಲ' 'ಎನ್ನುವುದು 
ಬೇಕು, ಇವೆರಡೂ ಇಲ್ಲದೆ ವಿದ್ಯೆ ವಿದ್ಯೆಯಲ್ಲ ಎಂದು ಒಬ್ಬರು ಹೇಳುತ್ತಾರೆ.

ವಿದ್ಯೆಯಿಂದ ನಾವು 
ಪಡೆದಿರಬೇಕಾದುವು ಯಾವವು 
ಎನ್ನುವುದಕ್ಕೆ ಗೌತಮ 
ಮಹರ್ಷಿಗಳು `ದಯಾ ಸರ್ವಭೂತೇಷು, 
ಕ್ಷಾಂತಿಃ', ಅನಸೂಯ, ಶೌಚಂ, 
ಅನಾಯಸಃ, ಮಂಗಲಂ, ಅಕಾರ್ಪಣ್ಯಂ, ಅಸ್ಪೃಹಾ, ಎಂದು ಹೇಳುತ್ತಾರೆ.

ಭಗವಂತನು ವೈಕುಂಠದಿಂದ 
ಕೆಳಗಿಳಿದು ಹಲವು ವಿಧವಾದ 
ಅವತಾರಗಳನ್ನು ಮಾಡಿದುದು 
ಅವನಿಗೆ ಸಂಬಂಧಿಸಿದಂತೆ 
ಅವಶ್ಯಕವಿಲ್ಲದಿದ್ದರೂ, 
ಶ್ರಮಪಟ್ಟು ಜನರಿಗೆ 
ಒಳ್ಳೆಯದನ್ನು ಮಾಡಬೇಕೆಂಬ 
ಒಂದೇ ವಿಚಾರದಿಂದ ಕೂಡಿದವನಾಗಿ 
ಕರುಣೆಯಿಂದ ಅವತಾರ ಮಾಡಿದನು. 
ಭಗವಂತನು ನಮಗೆ ಇತರರು ಕಷ್ಟ 
ಕೊಡುವಾಗ ಅದನ್ನು ದೂರಮಾಡುವ 
ಶಕ್ತಿಯನ್ನು ಕೊಟ್ಟಿದ್ದಾನೆ. 
ನಾವು ಆ ಶಕ್ತಿಯನ್ನು ನಮಗೆ 
`ದಯೆ' ಅಥವಾ `ಕರುಣೆ' ಇದ್ದರೆ 
ಮಾತ್ರ ಉಪಯೋಗಿಸುತ್ತೇವೆ. ದಯೆ 
ಎಂದರೇನು? ಇತರರು ಕಷ್ಟ 
ಕೊಡುವಾಗ ಅದನ್ನು ದೂರ 
ಮಾಡಬೇಕೆಂಬ ವಿಚಾರ ಉಂಟಾದರೆ 
ಅದೇ ದಯೆ. ಬೇರೆ 
ರೀತಿಯಲ್ಲಿದ್ದರೆ ಅವನನ್ನು 
`ದಯೆ ಇಲ್ಲದವನು' ಎಂದು 
ಹೇಳುತ್ತಾರೆ. ಮನುಷ್ಯನ 
ಶ್ರೇಷ್ಠವಾದ ಗುಣ ದಯೆ. 
ಹಾಗಿರುವ ದಯೆ ಎನ್ನುವ 
ಕರುಣೆಯನ್ನು ನಮ್ಮಲ್ಲಿ 
ವೃದ್ಧಿಪಡಿಸಿಕೊಳ್ಳಬೇಕು. 
ಕೆಲವರಿಗೆ ಸಹಜವಾಗಿಯೇ 
ವಿಶೇಷವಾಗಿ ಕರುಣೆ ಇರುತ್ತದೆ. 
ಕೆಲವರಿಗೆ ಒಳ್ಳೆಯವರ 
ಸಹವಾಸದಿಂದ ಅವರಂತೆ ತಾವೂ 
ಇರಬೇಕೆನ್ನುವ ಭಾವನೆ 
ಉಂಟಾಗಿರುವುದರಿಂದ ದಯೆ 
ಉಂಟಾಗುತ್ತದೆ. ಆದ್ದರಿಂದ 
ನಾವು ದಯೆಯನ್ನು 
ಹೆಚ್ಚಿಸಿಕೊಳ್ಳಬೇಕು. 
ಶಾಲೆಯಲ್ಲಿ ಓದಿನಲ್ಲಿ 
ಹಿಂದಿರುವ ಹುಡುಗರ ಮೇಲೆ 
ದಯೆತೋರಿ, ತನಗೆ ತಿಳಿದಷ್ಟು 
ಅವರಿಗೆ ಹೇಳಿಕೊಟ್ಟು ಅವರನ್ನು 
ಮುಂದಕ್ಕೆ ತರಬೇಕಾದುದು ಒಂದು 
ಶ್ರೇಷ್ಠವಾದ ಗುಣವಾಗುತ್ತದೆ. 
ಅದು ಕರುಣೆಯಾಗುತ್ತದೆ. 
ಅದನ್ನು ನೀವು ಕಲಿತು ಬಾಳಿನಲ್ಲಿ 
ಆಚರಣೆಗೆ ತರವೆವೂ. ಶಾಲೆಯ 
ವಿದ್ಯಾರ್ಥಿಗಳ ವಿಷಯವಾಗಿಯೇ 
ಅಲ್ಲದೆ `ಸರ್ವಭೂತೇಷು' ಎಂದು 
ಹೇಳಿರುವುದರಿಂದ ಇತರರ ವಿಷಯವಾಗಿಯೂ ಅದೇ ರೀತಿ ಕರುಣೆಯುಳ್ಳವರಾಗಿ ಇರಬೇಕು. 

ಆನಂತರ `ಕ್ಷಾಂತಿಃ' ಎನ್ನುವ 
ಗುಣ. ತನಗೆ ಹಿಡಿಸದ 
ಮಾತುಗಳನ್ನು ಯಾರಾದರೂ 
ಆಡಿದರೆ, ತನಗೆ ಹಿಡಿಸದ 
ಕೆಲಸವನ್ನು ಯಾರಾದರೂ 
ಮಾಡಿದರೆ, ಮನುಷ್ಯನು 
ಅಂಥಹವರನ್ನು ಹೊಡೆಯಲು 
ಹೋಗುತ್ತಾನೆ. ಇದಕ್ಕೆ ಶಕ್ತಿ 
ಇಲ್ಲದೆ ಹೋದರೆ ಬೇರೆ 
ಕೆಲಸಗಳನ್ನು ಮಾಡುತ್ತಾನೆ. 
ತಾನೇ ಎದುರಿಗೆ ನಿಲ್ಲದೆ 
ಮರೆಯಲ್ಲಿದ್ದು ಮಾಡುತ್ತಾನೆ. 
`ಅವನು ಮಾಡಿದ, ಆದ್ದರಿಂದ ನಾನೂ 
ಮಾಡುತ್ತೇನೆ' ಎಂದು ಜವಾಬು 
ಹೇಳುತ್ತಾನೆ. ಇದು ಒಳ್ಳೆಯ 
ಮನುಷ್ಯನ ಲಕ್ಷಣವಲ್ಲ. ತನ್ನ 
ಬಗ್ಗೆ ಯಾರಾದರೂ ಕ್ಷುದ್ರವಾಗಿ 
ಮಾತನಾಡಿದರೆ ಕ್ಷುದ್ರವಾಗಿ 
ಏನಾದರೂ ಮಾಡಿದರೆ, ಒಳ್ಳೆಯ 
ಮನುಷ್ಯನಾದವನು ಇದರಿಂದ 
ಏನಾಗುತ್ತದೆ? ಅವನಿಗೆ ನಾಳೆ 
ಬುದ್ಧಿ ಬರುತ್ತದೆ ಎಂದುಕೊಂಡು 
ಕ್ಷಮಿಸಿಬಿಡುತ್ತಾನೆ. 
ಕ್ಷುದ್ರವಾಗಿ ನಡೆದುಕೊಂಡವನು 
ಇವನ ಒಳ್ಳೆಯತನವನ್ನು ನೋಡಿ 
ಕ್ಷಮೆ ಕೇಳಿಕೊಳ್ಳುವ 
ಸ್ಥಿತಿಯೂ ಬಂದುಬಿಡುತ್ತದೆ. 
ಆದ್ದರಿಂದ ನಾವು ಕ್ಷಾಂತಿಃ ಎಂದು ಹೇಳಲ್ಪಟ್ಟ ಈ ಗುಣವನ್ನು ಬೆಳಸಿಕೊಳ್ಳಬೇಕು.

ಆನಂತರದ ಗುಣ `ಅನಸೂಯ' 
ಎನ್ನುವುದು. ಮನುಷ್ಯನು ತಾನು 
ಗುಣವಂತನಾಗಿ ಇಲ್ಲದಿದ್ದರೂ 
ಇನ್ನೊಬ್ಬ ಗುಣವಂತನನ್ನು ಕಂಡು 
ಅವನ ಗುಣಗಳ ಬಗ್ಗೆ ಸರಿಯಾಗಿ 
ಯೋಚಿಸದೆ ಅದರಲ್ಲಿ 
ಕೊರತೆಯನ್ನು ನೋಡಿ ಅದನ್ನು 
ಬಹಿರಂಗಪಡಿಸುವುದು ಸಹಜ. 
ಒಬ್ಬನು ಚೆನ್ನಾಗಿ ಓದುತ್ತಾನೆ 
ಎಂದರೆ ಇನ್ನೊಬ್ಬ ಓದದೆ ಇರುವ 
ಹುಡುಗ, ಇವನಿಗೆ ಅನೇಕ ಸಲ 
ಓದಿದರೇನೆ ಬರುತ್ತೆ. 
ಆದ್ದರಿಂದಲೇ ಓದುತ್ತಾನೆ. 
ಇಲ್ಲದೆ ಇದ್ದರೆ ನಾನೂ 
ಓದುತ್ತೇನೆ. ಇನ್ನು ಮೇಲೆ 
ಓದುತ್ತೇನೆ ಎನ್ನುತ್ತಾನೆ. 
ಆದರೆ ಅವನು ಹೇಳುವಂತೆ 
ಮಾಡುವುದಿಲ್ಲ. ಓದಿನಲ್ಲಿ 
ತನ್ನ ಸೋಮಾರಿತನವನ್ನು 
ಮರೆಸುವುದಕ್ಕಾಗಿ ಇನ್ನೊಬ್ಬ 
ಚೆನ್ನಾಗಿ ಓದುವುದನ್ನೇ 
ಕೊರತೆಯಾಗಿ ಎಣಿಸಿ `ಅವನು 
ರಾತ್ರಿ ಹಗಲು ಓದಿದರೂ 
ಪ್ರಯೋಜನವಿಲ್ಲ' 
ಎನ್ನುತ್ತಾನೆ. ಯಾವ 
ಗುಣವಿದೆಯೋ ಅದನ್ನೇ ನಾವು 
ಹೇಳಬೇಕೇ ವಿನಹ ಸುಮ್ಮನೆ 
`ದೋಷಾರೋಪಣೆ' (ವ್ಯರ್ಥವಾಗಿ ತಪ್ಪು ಪ್ರಕಟಿಸುವುದು) ಮಾಡಬಾರದು.

ಅನಂತರದ ಗುಣ `ಶೌಚಮ್' 
ಎನ್ನಲಾಗುವ ಶುದ್ಧತೆ. 
ಮನುಷ್ಯನಿಗೆ ಮುಖ್ಯವಾದುದು 
ಶೌಚವೆನ್ನುವ ದೊಡ್ಡಗುಣ. 
ಶೌಚವೆಂದರೇನು? ನಮ್ಮ ಎದುರಿಗೆ 
ಒಬ್ಬನು ಬಾಯಲ್ಲಿ ಬೆಟ್ಟು 
ಇಟ್ಟುಕೊಂಡು ಅನಂತರ ನಮ್ಮನ್ನು 
ಮುಟ್ಟಿದರೆ ನಮಗೆ 
ಬೇಸರವಾಗುತ್ತದೆ, `ಅವನು 
ನಮ್ಮನ್ನು ಮುಟ್ಟಿದನಲ್ಲಾ' 
ಎಂದು ನಮಗೆ ತೋರಿದರೂ ಅದು 
ಅವನಿಗೆ ಸಹಜವಾಗಿರುತ್ತದೆ. 
ಇದೇ ರೀತಿ ತಿಂಗಳುಗಟ್ಟಲೆ 
ಸ್ನಾನ ಮಾಡದೆ ಇರುವವನು 
ಹತ್ತಿರಕ್ಕೆ ಬಂದರೆ ನಾವು 
ಮೂಗು 
ಮುಚ್ಚಿಕೊಳ್ಳಬೇಕಾಗಬಹುದು, 
ಆದರೆ ಅಭ್ಯಾಸವಾಗಿರುವ ಅವನಿಗೆ 
ಏನೂ ತಿಳಿಯುವುದಿಲ್ಲ. ಕೆಲವರು 
ಬಟ್ಟೆಗಳನ್ನು ಒಗೆದು 
ಉಟ್ಟುಕೊಳ್ಳುವುದು 
ಸಂಪ್ರದಾಯವಿದೆ (!!) ಎಂದು ಹೇಳಿ 
ತಿಂಗಳಿಗೋ ವರ್ಷಕ್ಕೋ ಒಮ್ಮೆ 
ಬಟ್ಟೆಗಳನ್ನು ಒಗೆದರೆ, ಆ 
ಬಟ್ಟೆಗಳನ್ನು ನೋಡಿದರೇನೆ 
ಸಂಕಟವಾಗುತ್ತದೆ, ನಾವು ಎಂಜಲು 
ಮಾಡಬಾರದು, ಎಂಜಲು ಮೂಲಕ 
ಒಬ್ಬರಿಂದ ಇನ್ನೊಬ್ಬರಿಗೆ 
ವ್ಯಾಧಿಗಳು ಉಂಟಾಗುತ್ತವೆ. 
ಪ್ರತಿದಿನವೂ ಸ್ನಾನಮಾಡಿ ಒಗೆದ ಬಟ್ಟೆಗಳನ್ನು ಉಟ್ಟುಕೊಳ್ಳಬೇಕು. ಇದೇ ಶೌಚ.

`ಅನಾಯಾಸಃ' ಅಥವಾ 
ಸೋಮಾರಿತನವಿಲ್ಲದಿರುವಿಕೆ 
ಎನ್ನುವುದು. ಇದಾದ ಮೇಲೆ 
ಕೆಲವರು `ಬೆಳಗ್ಗೆ' 
ಓದುವುದಕ್ಕೆ ಹೇಳಿದರೆ 
`ಬಹಳಕಷ್ಟ' ಎನ್ನುತ್ತಾರೆ. 
`ಮಧ್ಯಾಹ್ನ' ಎಂದರೆ `ನನಗೇನೂ 
ಆಗಲೇ ಇಲ್ಲ' ಎನ್ನುತ್ತಾರೆ. 
ಚಿಕ್ಕಕೆಲಸವಾದರೂ, ದೊಡ್ಡ 
ಕೆಲಸವಾದರೂ `ಬಹಳ ಕಷ್ಟ' 
ಎನ್ನುತ್ತಾರೆ. ನಿಜಕ್ಕೂ `ಕಷ್ಟ' 
ಎನ್ನುವುದು ಒಂದೂ ಇಲ್ಲ. `ನಾವು 
ಮಾಡಬಹುದು' ಎನ್ನುವ 
ನಿಶ್ಚಯಕ್ಕೆ ಬಂದು, 
ಉತ್ಸಾಹದಿಂದ ತೊಡಗುವುದು 
ಮನುಷ್ಯನ ಕರ್ತವ್ಯ. ಆದ್ದರಿಂದ ಆಯಾಸಕ್ಕೆ  (ಸೋಮಾರಿತನಕ್ಕೆ) ಅವಕಾಶ ಕೊಡಕೂಡದು.

`ಮಂಗಲ' ಎನ್ನುವುದು ಇದಾದ ಮೇಲಿನ 
ಗುಣ. ಕೆಲವರು ಯಾವಾಗಲೂ 
ಕೆಳಮಟ್ಟದ ಶಬ್ದಗಳನ್ನೇ 
ಬಳಸುತ್ತಾರೆ. ನಾವು ಯಾರಾದರೂ 
ದೊಡ್ಡವರೊಡನೆ ಮಾತನಾಡುವಾಗ 
ಅವರಿಗೆ ತಕ್ಕಂತೆ, 
ವಯಸ್ಸಾಗಿರುವವರಾದರೆ ``ಬನ್ನಿ" 
ಎನ್ನುವುದಕ್ಕೆ ಬದಲು ``ದಯಮಾಡಿ" 
ಎಂದರೆ, ಅವರಿಗೆ ಬಹಳ 
ಸಂತೋಷವಾಗಿ ನಮ್ಮ ಮೇಲೆ 
ಪ್ರೀತಿ ಹೆಚ್ಚಾಗುತ್ತದೆ. 
`ಆಸನವನ್ನು ಅಲಂಕರಿಸಿ' ಎಂದರೆ 
ಮಂಗಳವಾದ ಮಾತು. `ಅಲ್ಲಿ ಹೋಗಿ 
ಕೂತುಕೋ' ಎಂದರೆ ಸಾಧಾರಣವಾದ 
ಮಾತು. ಮಂಗಳವಾದ ಮುಖಭಾವವೂ ಇರಬೇಕು. ಇದನ್ನು ನಾವು ಕಲಿಯಬೇಕು.

ಅನಂತರ `ಅಕಾರ್ಪಣ್ಯ'. 
ಮನುಷ್ಯನಲ್ಲಿ ಕಾಮ, ಕ್ರೋಧ, ಲೋಭ 
ಮುಂತಾದ ಕೆಟ್ಟ ಗುಣಗಳು 
ಇರುತ್ತವೆ. ಒಬ್ಬೊಬ್ಬರಿಗೆ 
ಶಕ್ತಿ ಇಲ್ಲದೆ ಹೋಗುವುದರಿಂದ 
ಕಾಮ-ಕ್ರೋಧಗಳು ತಾವಾಗಿಯೇ 
ಬರದೆ ಇದ್ದರೂ ಲೋಭವನ್ನು 
ಗೆಲ್ಲುವುದು ಸ್ವಲ್ಪ 
ಕಠಿಣವಾಗುತ್ತದೆ. ಎಷ್ಟು 
ವಯಸ್ಸಾದರೂ ಇನ್ನೂ ಹಣವನ್ನು 
ಸೇರಿಸಬೇಕೆಂಬ ವಿಚಾರ 
ಲೋಭವಾಗುತ್ತದೆ. ಜೇನು ನೊಣಗಳು 
ಎಷ್ಟು ಸೇರಿಸಿದರೂ ಯಾರೋ 
ಜೇನುಗೂಡಿನಿಂದ 
ಜೇನುತುಪ್ಪವನ್ನೆಲ್ಲ 
ತೆಗೆದುಕೊಂಡು ಹೋಗುವ ಹಾಗೆ. 
ನಾವು ಏಕೆ ಸೇರಿಸಿಟ್ಟ ಅದನ್ನು 
ಯಾರೋ ತೆಗೆದುಕೊಂಡು ಹೋಗುವಂತೆ 
ಮಾಡಬೇಕು? ಒಬ್ಬ ಬಡವ 
ಬೇಕಾಗಿದೆಯೆಂದು ಕೇಳಿದರೆ ಅದು 
ಸರಿಯೆಂದು ನಮಗೆ ತೋರಿದರೆ 
ಸಾಧ್ಯವಾದ ಮಟ್ಟಿಗೆ ಸಹಾಯ 
ಮಾಡಬಹುದೇ ವಿನಹ ಲೋಭಿಯಾಗಬಾರದು. ಇದೇ `ಅಕಾರ್ಪಣ್ಯ' ವೆಂದು ಹೇಳಲ್ಪಟ್ಟಿದೆ.

ಅದಾದ ಮೇಲೆ `ಅಸ್ಪೃಹಾ' ಆಸೆಗಳು 
ಹೆಚ್ಚಾಗುತ್ತಾ 
ಹೆಚ್ಚಾಗುತ್ತಾ ತಪ್ಪುಗಳನ್ನು 
ಮಾಡುವನು. ಹೊಟ್ಟೆಗೆ 
ಇಲ್ಲದವನು ಕಳ್ಳತನ 
ಮಾಡುವುದಕ್ಕಿಂತಲೂ 
ಹಣವುಳ್ಳವನು ಕಳ್ಳತನ 
ಮಾಡುವುದೇ ಹೆಚ್ಚು. ಏಕೆಂದರೆ 
ಅವನಿಗೆ ಆಸೆ ಹೆಚ್ಚು. 
ಆಸೆಗಳನ್ನು ಒಂದು 
ಮಿತಿಯಲ್ಲಿಟ್ಟುಕೊಂಡು ನಾವು 
ವ್ಯವಹರಿಸಿದರೆ ನಾವು 
ದೋಷರಹಿತರಾಗುತ್ತೇವೆ. ಎಲ್ಲ 
ಪಾಪಗಳಿಗೂ ಮೂಲ ಸ್ಥಾನವಾದ 
ಆಸೆಯನ್ನು ನಾವು ನಿಗ್ರಹಿಸಿ ನಾವು ಶ್ರೇಯಸ್ಸನ್ನು ಪಡೆಯಬೇಕು.

ವಿದ್ಯೆಯಿಂದ ನಾವು ಈ ಎಂಟು 
ಗುಣಗಳನ್ನು ಖಂಡಿತವಾಗಿಯೂ 
ಪಡೆಯಬೇಕು. ವಿದ್ಯೆ ಕಲಿತೂ ಈ 
ಗುಣಗಳನ್ನು ನಾವು ಪಡೆಯದೆ 
ಹೋದರೆ ಕತ್ತಿ ಅಥವಾ 
ಚಾಕುವನ್ನು ಇಟ್ಟುಕೊಂಡು 
ತರಕಾರಿಯನ್ನು ಕತ್ತರಿಸದೆ 
ನಮ್ಮ ಶರೀರದ ಅಂಗಗಳನ್ನೇ 
ಕತ್ತರಿಸಿಕೊಂಡಂತೆಯೇ ಸರಿ. ಈ 
ರೀತಿ ನಾವು ನಮ್ಮ ಬಾಳನ್ನು 
ಹಾಳು ಮಾಡಿಕೊಳ್ಳಬಾರದು. 
ಆದ್ದರಿಂದ ಆ ನಿಗ್ರಹ, ಗುಣ 
ಮೊದಲಾದವುಗಳನ್ನು ನಮ್ಮ 
ದೃಢನಿಶ್ಚಯವಾಗಿ ಸ್ವೀಕರಿಸಿ 
ವಿದ್ಯಾವಂತರಾಗಿದ್ದರೆ ಮಾತ್ರ 
ದೇಶದ ಒಳ್ಳೆಯ 
ನಾಗರೀಕರಾಗಿರಬಲ್ಲೆವು. ಇತರರಿಗೂ ದಾರಿ ತೋರಿಸಬಹುದು, ಶ್ರೇಯಸ್ಸು ಪಡೆಯಬಹುದು.

ನೀವೆಲ್ಲರೂ ಗುಣಶೀಲಗಳನ್ನು 
ಪಡೆದು ಬಾಳಬೇಕೆಂದು ನಾವು 
ಪ್ರಾರ್ಥಿಸುತ್ತೇವೆ. ಈಶ್ವರನು ನಿಮಗೆಲ್ಲರಿಗೂ ಒಳ್ಳೆಯದನ್ನು ಮಾಡಲಿ. 

\newpage

\section{ಭಕ್ತಿ}

\begin{shloka}
ತ್ರಯೀವೇದ್ಯಂ ಹೃದ್ಯಂ ತ್ರಿಪುರಹರಮಾದ್ಯಂ ತ್ರಿನಯನಂ\\
ಜಟಾಭಾರೋದಾರಂ ಚಲದುರಗಹಾರಂ ಮೃಗಧರಮ್|\\
ಮಹಾದೇವಂ ದೇವಂ ಮಯಿ ಸದಯಭಾವಂ ಪಶುಪತಿಂ\\
ಚಿದಾಲಂಬಂ ಸಾಂಬಂ ಶಿವಮತಿವಿಡಂಬಂ ಹೃದಿ ಭಜೇ||
\end{shloka}

ಜ್ಞಾನವನ್ನು ಪಡೆಯಲು ಮೂರು 
ಮಾರ್ಗಗಳು ಹೇಳಲ್ಪಟ್ಟಿವೆ. 
ತೀವ್ರವಾದ 
ವೈರಾಗ್ಯವಿರುವುದರಿಂದ ಯಾರ 
ಮನಸ್ಸು ತಮ್ಮ ವಶದಲ್ಲಿರುವುದೋ 
ಅವರು ಜ್ಞಾನಮಾರ್ಗವನ್ನು 
ಅವಲಂಬಿಸಬಹುದು. ಕೆಲವರಿಗೆ 
ಶಾಸ್ತ್ರಗಳಲ್ಲಿ ಹೇಳಲ್ಪಟ್ಟ 
ಕರ್ಮಗಳ ಬಗ್ಗೆ ಹೆಚ್ಚು 
ತಿಳುವಳಿಕೆ. ಆ ಕರ್ಮಗಳ 
ವಿಷಯದಲ್ಲಿ ಹೆಚ್ಚು ಶ್ರದ್ಧೆ 
ಇರುತ್ತದೆ. ಇಂಥಹವರು 
ಕರ್ಮಮಾರ್ಗದಲ್ಲಿ ನಡೆಯಬಹುದು. 
ಮತ್ತೆ ಕೆಲವರಿಗೆ 
ಶಾಸ್ತ್ರಗಳಲ್ಲಿ ಹೇಳಲ್ಪಟ್ಟ 
ಕರ್ಮಗಳ ಬಗ್ಗೆ ಹೆಚ್ಚಾಗಿ 
ತಿಳುವಳಿಕೆಯಾಗಲಿ ಮೇಲ್ಮಟ್ಟದ 
ವೈರಾಗ್ಯವಾಗಲಿ ಇಲ್ಲದೆ ಹೋದರೆ 
ಅಂಥಹವರಿಗೆ ಏನು ದಾರಿ? ಇದಕ್ಕೆ 
`ನೀನು ಭಗವಂತನನ್ನು 
ಭಜಿಸುತ್ತಾ ಇರು. ಅದು ನಿನಗೆ 
ಉತ್ತಮವಾದ ದಾರಿಯಾಗುತ್ತದೆ' ಎಂದು ನಮ್ಮ ಹಿಂದಿನವರು ಹೇಳಿದ್ದಾರೆ.

`ಜ್ಞಾನಂ ಕರ್ಮ ಚ ಭಕ್ತಿಶ್ಚ 
ನೋಪಾಯೋನ್ಯೋಽಸ್ತಿ 
ಕುತ್ರಚಿತ್|' ಎಂದು ಇದೆ. 
ಜ್ಞಾನಮಾರ್ಗ, ಕರ್ಮಮಾರ್ಗ, 
ಭಕ್ತಿಮಾರ್ಗ ಎಂದು ಮೂರು 
ಮಾರ್ಗಗಳಿವೆ. ಈ ಮೂರು 
ಮಾರ್ಗಗಳನ್ನು ಬಿಟ್ಟು ಬೇರೆ 
ಯಾವ ಮಾರ್ಗವೂ ಇಲ್ಲ. ಇದೇ 
ಸಾರಾಂಶ. ಎಲ್ಲಾ ವಿಧವಾದ 
ಬಂಧಗಳಿಂದ ಬಿಡುಗಡೆ ಪಡೆದು 
ಪರವಸ್ತುವನ್ನು ಅರಿತ 
ಜೀವನ್ಮುಕ್ತರೂ ಸಗುಣ ಪರಮೇಶ್ವರನನ್ನು ಆರಾಧಿಸುವುದು ಉಂಟೆಂದು ಭಾಗವತದಲ್ಲಿ ಹೇಳಿದೆ.

\begin{shloka}
ಆತ್ಮಾರಾಮಾಶ್ಚ ಮುನಯೋ ನಿರ್ಗ್ರಂಥಾ ಅಪ್ಯರುಕ್ರಮೇ|\\
ಕುರ್ವಂತ್ಯಹೈತುಕೀಂ ಭಕ್ತಿಂ ಇತ್ಥಂ ಭೂತಗುಣೋ ಹರಿಃ||
\end{shloka}

ಭಗವಂತನ ಗುಣಗಳು ಹೇಗಿರುವುವು 
ಎಂದರೆ ಯಾರು-ನಿರ್ಗುಣ 
ಪರವಸ್ತುವಿನ ವಾಸ್ತವಿಕವಾದ 
ಸ್ವರೂಪವನ್ನು 
ತಿಳಿದವರಾಗಿದ್ದಾರೋ ಅಂಥಹವರು 
ಕೂಡ ಸಗುಣ ಪರಮೇಶ್ವರನ 
ಸ್ವರೂಪವನ್ನು 
ಆರಾಧಿಸುತ್ತಾರೆಂದು 
ಹೇಳಲ್ಪಟ್ಟಿದೆ. ಏಕೆಂದರೆ 
ಭಗವಂತನ ಆ ಕಲ್ಯಾಣ ಗುಣಗಳು 
ಇಂಥಹವು. ಯಾರಾದರೂ ಅವನನ್ನು ಬಿಟ್ಟುಬಿಡಲು ಇಷ್ಟಪಟ್ಟರೂ ಅದು ಬೇಡವೆಂದು ತೋರುತ್ತದೆ.

\begin{shloka}
ತ್ವಯಿ ಜನಾರ್ದನ ಭಕ್ತಿರಚಂಚಲಾ ಯದಿ ಭವೇತಫಲಪ್ರವಣಾಮಮ|\\
ಅಭಿಲಷಾಮ್ಯಪವರ್ಗ ಪರಾಙ್ಮುಖಃ ಪುನರಪೀಹ ಶರೀರ ಪರಿಗ್ರಹಮ್||
\end{shloka}

ಮೋಕ್ಷ ಲಾಭವಾದವನಿಗೆ 
ಶರೀರವೆನ್ನುವುದು 
ಇರುವುದಿಲ್ಲ. ಅನಂತವಾದ 
ಚೈತನ್ಯ ಸ್ವರೂಪನಾಗಿ 
ಅವನಿರುತ್ತಾನೆ. ಆಗ ಅವನು ಆನಂದ 
ರೂಪನಾಗಿರುತ್ತಾನೆಂದು 
ಶಾಸ್ತ್ರ ಹೇಳುತ್ತಿದ್ದರೂ 
ಭಕ್ತರಿಗೆ ಭಗವದ್ಭಕ್ತಿಯಿಂದ 
ಉಂಟಾದ ಆನಂದವನ್ನು, ಅನುಭವಿಸಿ 
ಏನೆಂದು ತೋರುತ್ತದೆ 
ಎನ್ನುವುದನ್ನು ಕುರಿತು, 
`ತ್ವಯಿ ಜನಾರ್ದನ ಭಕ್ತಿರಚಂಚಲಾ ಯದಿ ಭವೇತಫಲಪ್ರವಣಾ ಮಮ' ಎಂದು ಹೇಳಲ್ಪಟ್ಟಿದೆ.

`ಎಲೈ ಭಗವಂತನೇ! ನಿನ್ನ ಮೇಲೆ 
ಅಚಂಚಲವಾದ ಭಕ್ತಿ ಇದ್ದರೆ 
ಮಾತ್ರ ಸಾಕು. ಯಾವ ಫಲವನ್ನು 
ಇಷ್ಟಪಟ್ಟು ನಾನು ಭಕ್ತಿಯನ್ನು 
ಮಾಡುವಂತಾಗದಿರಲಿ' - ಒಂದು 
ಫಲಾಪೇಕ್ಷೆಯೂ ಇಲ್ಲದೆ ನಾನು 
ನಿನಗಾಗಿ ಎಂದೇ ಭಕ್ತಿ 
ಮಾಡುತ್ತೇನೆ. ಎಂದರೆ 
ಅಭಿಲಾಷಾಮ್ಯಪವರ್ಗಪರಾಙ್ಮುಖಃ ಪುನರಪೀಹ ಶರೀರ ಪರಿಗ್ರಹಮ್ `ಮೋಕ್ಷವೆನ್ನುವುದೂ ಬೇಡ. (ಮತ್ತೆ ನಿನಗೆ ಏನು ಬೇಕು?) ಮತ್ತೆ ಮತ್ತೆ ನನಗೆ ಶರೀರ ಉಂಟಾಗುತ್ತಾ ನಿನಗೆ ಸೇವೆ ಮಾಡುತ್ತಾ ಹೋದರೆ ಅಷ್ಟೇ ನನಗೆ ಸಾಕು.'

ಭಕ್ತಿ ಮಾರ್ಗದಲ್ಲಿರುವವನಿಗೆ 
ಯಾವಾಗಲೂ ಆನಂದವಿದ್ದೇ 
ಇರುತ್ತದೆ. ಆದರೆ 
ಜ್ಞಾನಮಾರ್ಗದಲ್ಲಿರುವವರಿಗೆ 
ಸಾಧನೆ ಮಾಡುವ ಕಾಲದಲ್ಲಿ 
ಆನಂದವಾಗಿರುವುದಿಲ್ಲ. 
ಯಾವಾಗಲೂ ಅವರು 
ಇಂದ್ರಿಯಗಳನ್ನೂ ಮನಸ್ಸನ್ನೂ 
ತಮ್ಮ ವಶದಲ್ಲಿಟ್ಟುಕೊಂಡಿರಬೇಕು. ಆದರೆ ಇವುಗಳನ್ನು ನಿಗ್ರಹಿಸುವುದು ಕಷ್ಟವಾದುದೇ.

`ಅವ್ಯಕ್ತಾ ಹಿ ಗತಿರ್ದುಃಖಂ 
ದೇಹವದ್ಭಿರವಾಪ್ಯತೇ' 
ಎಂದಲ್ಲವೇ ಭಗವಂತನು 
ಹೇಳಿರುವುದು! ಆ ಅವ್ಯಕ್ತವಾದ 
ಪರವಸ್ತುವನ್ನು ಚಿಂತಿಸುವುದು 
ಕಷ್ಟ. ಹೀಗಿದ್ದರೂ 
ಜ್ಞಾನಮಾರ್ಗದಲ್ಲಿ 
ನಡೆಯುವವನು ಪ್ರಯತ್ನಿಸುತ್ತಾ 
ಮನಸ್ಸನ್ನು ದೃಢಪಡಿಸಿಕೊಂಡು 
ಮುಂದುವರೆಯಬೇಕು. ಆದರೆ 
ಭಗವಂತನ ಆರಾಧನೆ ಹಾಗಲ್ಲ. 
ಭಗವಂತನ ಲೀಲೆಗಳು ಮನಸ್ಸನ್ನು ಆಕರ್ಷಿಸಿ ಅದನ್ನು ಸ್ಥಿರಪಡಿಸುತ್ತವೆ.

ಮಧುಸೂದನ ಸರಸ್ವತಿಯವರು 
ಅದ್ವೈತ ಮಾರ್ಗದಲ್ಲಿ ಬಹಳ 
ಸಿದ್ಧರು. ಅವರು ಶುದ್ಧವಾದ 
ಚೈತನ್ಯವನ್ನು ಬಿಟ್ಟು ಬೇರೆ 
ಯಾವುದೂ ಸತ್ಯವಾಗಿಯೂ 
ಇಲ್ಲವೆನ್ನುವುದನ್ನು 
ಸ್ಪಷ್ಟಪಡಿಸುವ `ಅದ್ವೈತ 
ಸಿದ್ಧಿ' ಎನ್ನುವ ಗ್ರಂಥವನ್ನು 
ಬರೆದಿದ್ದಾರೆ. ಅಂಥಹವರು ಕೂಡ 
ಒಂದು ಕಡೆ ಏನು ಹೇಳಿದ್ದಾರೆ 
ಎನ್ನುವುದು ಗಮನಾರ್ಹ. 
`ನಿರ್ಗುಣವಾಗಿಯೂ, 
ನಿಷ್ಕಲವಾಗಿಯೂ, ಆನಂದವಾಗಿಯೂ 
ಇರುವ ಯಾವುದೋ ಚೈತನ್ಯವನ್ನು 
ಯೋಗಿಗಳು ಧ್ಯಾನದಲ್ಲಿ ಕಾಣುತ್ತಾರೆ. ಅದನ್ನು ಅವರು ನೋಡಲಿ. ಹೀಗೆಂದು ಹೇಳಿ -

\begin{shloka}
`ಅಸ್ಮಾ ಕಂ ತು ತದೇವ ಲೋಚನ ಚಮತ್ಕಾರಾಯ ಭೂಯಾಚ್ಚಿರಮ್|\\
ಕಾಲಿಂದೀಪುಲಿನೇಷು ತತ್ಕಿಮಪಿ ಯನ್ನೀಲಂ ಮಹೋ ಧಾವತಿ||'
\end{shloka}

ಎಂದು ತಮ್ಮ ಅಭಿಪ್ರಾಯವನ್ನು 
ಪ್ರಕಟಿಸಿದ್ದಾರೆ. `ನಮ್ಮ 
ಕಣ್ಣುಗಳಿಗೆ ಯಮುನಾ ದಡದಲ್ಲಿ 
ಆಟವಾಡುತ್ತಿರುವ ಶ್ಯಾಮಲ 
ಗಾತ್ರನಾದ ಕೃಷ್ಣನು ಯಾವಾಗಲೂ 
ಕಾಣುತ್ತಾನೆ' ಎನ್ನುವುದು 
ತಾತ್ಪರ್ಯ. ಸಗುಣೋಪಾಸನೆ 
ಮಾಡುವವರಿಗೆ, ಭಗವಂತನ 
ಭಕ್ತಿಮಾರ್ಗದಲ್ಲಿರುವವರಿಗೆ 
ಆನಂದವೆನ್ನುವುದು ಯಾವಾಗಲೂ 
ಇರುತ್ತದೆ. ಮನಸ್ಸು 
ದೃಢವಿಲ್ಲದಿರುವವರಿಗೆ 
ಭಗವದ್ಭಕ್ತಿಯಿಂದ ತಾನಾಗಿಯೇ 
ಮನಸ್ಸು ಸ್ಥಿರವಾಗಿ ಆ 
ನಿರ್ಗುಣವಾದ ಆತ್ಮ ಸಾಕ್ಷಾತ್ಕಾರ ಭಗವಂತನ ದಯೆಯಿಂದ ತಾನಾಗಿಯೇ ಆಗುವುದು.

\begin{shloka}
`ತೇಷಾಮೇವಾನುಕಂಪಾರ್ಥಮಹಮಜ್ಞಾನಜಂ ತಮಃ|\\
ನಾಶಯಾಮ್ಯಾತ್ಮಭಾವಸ್ಥೋ ಜ್ಞಾನದೀಪೇನ ಭಾಸ್ವತಾ||
\end{shloka}

ಎಂದು ಗೀತೆಯಲ್ಲಿ ಭಗವಂತನು 
ತಿಳಿಸಿದ್ದಾನೆ. ಭಗವಂತನು 
ಭಕ್ತನ ಮೇಲೆ ಅನುಗ್ರಹಿಸಿ 
ಹೃದಯದಲ್ಲಿ ಆಸೀನನಾದನೆಂದರೆ 
ಭಗವಂತನು ತಾನಾಗಿ ಜ್ಞಾನ 
ದೀಪವನ್ನು ಪ್ರಕಾಶಿಸಿ 
ಅಜ್ಞಾನವನ್ನು ನಾಶ ಮಾಡುತ್ತಾನೆ. ಇದು ಭಗವಂತನ ಅಭಿಪ್ರಾಯ. ಭಕ್ತಿಯ ಹಿರಿಮೆ ಇಂಥಹದು.

ಭಕ್ತಿ ಎನ್ನುವುದು ಏನು 
ಎನ್ನುವುದಕ್ಕೆ `ಸಾತ್ವಸ್ಮಿನ್ 
ಪರಮಪ್ರೇಮ ರೂಪಾ'-ಅದು ಭಗವಂತನ 
ವಿಷಯದಲ್ಲಿ ಶ್ರೇಷ್ಠವಾದ ಪ್ರೇಮರೂಪ ಎಂದು ನಾರದರು ಹೇಳಿದರು.

ಶಾಂಡಿಲ್ಯರು ಒಬ್ಬ ಮಹರ್ಷಿ, 
ಅವರು ಭಕ್ತಿಯನ್ನು ಕುರಿತು 
ಒಂದು ಗ್ರಂಥವನ್ನು 
ಬರೆದಿದ್ದಾರೆ. `ಈ ಭಕ್ತಿ 
ಎಲ್ಲರೂ ಮಾಡುತ್ತಿರುವ ವಿಷಯ. 
ಆದರೆ ಇದಕ್ಕೆ ಆಧಾರ ಬೇರೆ 
ಬೇರೆಯಾಗಿರುವುದರಿಂದ ಯಾವುದೋ 
ರೂಪದಲ್ಲಿ ಅದು ಬದಲಾಗುತ್ತದೆ' 
ಎನ್ನುವ ಅಭಿಪ್ರಾಯವನ್ನು ಅವರು 
ತಿಳಿಸಿದ್ದಾರೆ. ಮನುಷ್ಯನಾಗಿ 
ಹುಟ್ಟಿದ ಪ್ರತಿಯೊಬ್ಬನೂ 
ಯಾವದಾದರೂ ಒಂದು ವಿಷಯದಲ್ಲಿ 
ಪ್ರೀತಿ ಇಟ್ಟುಕೊಂಡವನಾಗಿಯೇ 
ಬಾಳುತ್ತಾನೆ ಹೊರತು ಒಂದರಲ್ಲೂ ಪ್ರೀತಿ ಇಲ್ಲದೆ ಬಾಳುವುದಿಲ್ಲ.

\begin{shloka}
`ಬಾಲಸ್ತಾವತ್ ಕ್ರೀಡಾ ಸಕ್ತ-\\
ಸ್ತರುಣಸ್ತಾವತ್ತರುಣೀ ಸಕ್ತಃ|\\
ವೃದ್ಧಸ್ತಾವಚ್ಚಿಂತಾಸಕ್ತಃ||'
\end{shloka}

ಎನ್ನುವಂತೆ ಎಲ್ಲರಿಗೂ 
ಪ್ರಿಯವಾಗಿರುವಂಥಹ ಒಂದು 
ವಸ್ತು ಇರುತ್ತದೆ. 
`ಬಾಲಸ್ತಾವತ್ 
ಕ್ರೀಡಾಸಕ್ತಃ'-ಮಕ್ಕಳಿಗೆ 
ಯಾವಾಗಲೂ ಆಡುವುದೆಂದರೆ ಇಷ್ಟ. 
ಶರೀರಕ್ಕೆ ಆಹಾರವೂ ಬೇಡ. 
ಆಟದಲ್ಲಿ ಅಷ್ಟು ಆಸಕ್ತಿ. ಇದು 
ಇಲ್ಲದಿದ್ದರೆ ಮಕ್ಕಳ 
ಬೆಳವಣಿಗೆ ಕೆಟ್ಟುಹೋಗುತ್ತದೆ. 
ಆದ್ದರಿಂದಲೇ ಮಕ್ಕಳು 
ಆಡುತ್ತಿದ್ದರೆ ಅವರನ್ನು ಯಾರು 
ತಡೆಯುವುದಿಲ್ಲ. 
`ತರುಣಸ್ತಾವತ್ತರುಣೀಸಕ್ತಃ'-ಯೌವನದಲ್ಲಿ ಹೆಣ್ಣು ಇಂಥಹವುಗಳ ಮೇಲೆ ಆಸಕ್ತಿ ಇರುವುದುಂಟು. `ವೃದ್ಧಸ್ತಾವಚ್ಚಿಂತಾ ಸಕ್ತಃ'-ಮುಪ್ಪಿನಲ್ಲಿ ಚಿಂತೆ ಇರುವುದು ಸಹಜ. `ಅದು ಹಾಗೆ ಆಗಲಿಲ್ಲ, ಇದು ಹೀಗೆ ಆಗಲಿಲ್ಲ. ಅದು ಆಗಲಿಲ್ಲ, ಇದು ಆಗಲಿಲ್ಲ' ಎಂದು ಚಿಂತೆ ಮಾಡುತ್ತಿರುವುದೇ ವಯಸ್ಸಾದವರಿಗೆ ಇಷ್ಟ. ಹಾಗೆ ಚಿಂತೆ ಇಲ್ಲದೆ ಅವರು ಇರಲಾರರು.

ಹೀಗೆ ಮನುಷ್ಯನ ಜೀವನದಲ್ಲಿ 
ಯಾವುದಾದರೂ ವಿಷಯದಲ್ಲಿ 
ಪ್ರೇಮವಿರುತ್ತದೆ. ಎಲ್ಲವನ್ನೂ 
ಬಿಟ್ಟುಬಿಟ್ಟು ಕಾವಿ ಬಟ್ಟೆ 
ಧರಿಸಿದ 
ಸಂನ್ಯಾಸಿಗಳಾಗಿರುವವರಿಗೆ 
ಯಾವ ವಿಷಯದಲ್ಲಿ ಪ್ರೇಮ ಎಂದರೆ 
ಅವರಿಗೆ ಆತ್ಮವನ್ನು 
ಪಡೆಯಬೇಕೆನ್ನುವ ವಿಷಯದಲ್ಲಿ 
ಪ್ರೇಮ. ಪ್ರಿಯವಾಗಿರುವ ಈ ಆತ್ಮ 
ಒಂದಿರುವಾಗ ಯಾವುದಕ್ಕಾಗಿ 
ಸಾವಿರಾರು ವಿಷಯಗಳಲ್ಲಿ 
ಪ್ರೀತಿಯನ್ನಿಟ್ಟುಕೊಳ್ಳಬೇಕು? ಇದು ಒಂದು ಸಾಕು ಎಂದು ಸ್ಥಿರವಾಗಿದ್ದು ಇದರ ಸಾಕ್ಷಾತ್ಕಾರ ಪಡೆಯಬೇಕೆಂದು ಸಂನ್ಯಾಸಿಗಳೂ ಪ್ರೇಮ ಇಟ್ಟುಕೊಂಡೇ ಇದ್ದಾರೆ. ಆದ್ದರಿಂದ ಪ್ರೇಮವಿಲ್ಲದೆ ಯಾರೂ ಇಲ್ಲ. ಮನುಷ್ಯನಿಗೆ ಯಾವುದಾದರೂ ಒಂದು ವಿಷಯದಲ್ಲಿ ಪ್ರೇಮ ಇದ್ದೇ ಇರುತ್ತದೆ. ಮಕ್ಕಳು-ಮರಿ ಇಲ್ಲದವನು ಒಂದು ಬೆಕ್ಕನ್ನೋ, ನಾಯಿಯನ್ನೋ ಸಾಕುತ್ತಾನೆ. ಹಿಂದಿನ ಕಾಲದಲ್ಲಿ ಹಸುವನ್ನು ಸಾಕುವುದು ಪುಣ್ಯವೆಂದುಕೊಂಡು ಸಾಕುತ್ತಿದ್ದರು. ಈಗಲೋ, ನಾಯಿ, ಬೆಕ್ಕು ಇವುಗಳಾದರೆ ಯಾವಾಗಲೂ ಪಕ್ಕದಲ್ಲಿಯೇ ಇಟ್ಟುಕೊಳ್ಳಬಹುದೆಂದುಕೊಂಡು ಸಾಕುತ್ತಾರೆ. ಏಕೆಂದರೆ ಮನುಷ್ಯನಿಗೆ ಯಾವುದಾದರೂ ಒಂದು ಚೈತನ್ಯದಲ್ಲಿ ವಿಶ್ವಾಸವಿರುತ್ತದೆ. ಆ ವಿಶ್ವಾಸವೇ ಪ್ರೇಮ. ಆ ಪ್ರೇಮ ಸ್ವಾಭಾವಿಕವಾದುದರಿಂದ ಋಷಿಗಳು `ಇವನು ಈ ರೀತಿ ಪ್ರೀತಿ ಇಟ್ಟುಕೊಂಡಿದ್ದಾನಲ್ಲಾ, ಇದು ಇವನ ಸ್ವಭಾವವಾಗಿದೆ. ಇವನಿಗೆ ಒಂದು ಒಳ್ಳೆಯ ದಾರಿ ತೋರಿಸಬೇಕು' ಎಂದುಕೊಂಡರು. ತಾಮ್ರಪರ್ಣಿ ನದಿಯಲ್ಲಿ ಎಷ್ಟೋ ನೀರು ಹರಿಯುತ್ತಿದ್ದರೂ ಎಲ್ಲಾ ಸಮುದ್ರಕ್ಕೆ ಸೇರುತ್ತಿತ್ತು. ಇಂಜಿನಿಯರುಗಳು ಏನು ಮಾಡಿದರು! ದಾರಿಯಲ್ಲೇ ಸೇತುವೆ ಕಟ್ಟಿದರೆ ಎಷ್ಟೋ ಉಪಯೋಗವಾಗುತ್ತದೆಂದು ಯೋಚಿಸಿದರು. ಅದೇ ರೀತಿ ನಮ್ಮ ಋಷಿಗಳು, `ನೀನು ಒಂದು ವಿಷಯದಲ್ಲಿ ತಾನೆ ಮುಖ್ಯವಾಗಿ ಪ್ರೇಮವನ್ನಿಟ್ಟುಕೊಳ್ಳುವೆ. ಅದನ್ನು ನೀನು ಭಗವಂತನ ವಿಷಯದಲ್ಲಿಟ್ಟುಕೋ. ಹಾಗೆ ಅದನ್ನು ಭಗವಂತನ ಕಡೆಗೆ ತಿರುಗಿಸಿಬಿಡು. ಹಾಗೆ ತಿರುಗಿಸಿಬಿಟ್ಟರೆ, ಭಗವಂತನ ವಿಷಯವಾಗಿ ಮನಸ್ಸು ಆಗಿ ಬಿಟ್ಟರೆ ಯಾವಾಗಲೂ ಮನಸ್ಸಿನಲ್ಲಿ ಸಚ್ಚಿಂತನೆ, ಸತ್ಕಥನ ಎನ್ನುವಂತೆ ಭಗವಂತನ ಪಾದಾರವಿಂದಗಳಲ್ಲಿ ಮನಸ್ಸು ತಾನಾಗಿಯೇ ಬಂದು ಬಿಡುತ್ತದೆ. ಅವನಲ್ಲಿ ಉತ್ತಮವಾದ ಪ್ರೇಮ ಉಂಟಾಗಿ ಎಂದೂ ಅದನ್ನು ಬಿಡಲಾಗುವುದಿಲ್ಲ. ಆದ್ದರಿಂದ ನಾರದರು, 

\begin{shloka}
`ಸಾತ್ವಸ್ಮಿನ್ ಪರಮಪ್ರೇಮರೂಪಾ'
\end{shloka}

-`ಭಕ್ತಿ ಎನ್ನುವುದು 
ಪರವಸ್ತುವಿನ ವಿಷಯದಲ್ಲಿ 
ಉತ್ತಮವಾದ ಪ್ರೇಮರೂಪಾ' ಎಂದು 
ಹೇಳಿದರು. ಈ ಭಕ್ತಿಗೆ 
``ಪರಮಪ್ರೇಮರೂಪಾ" ಎನ್ನುವ 
ಹೆಸರು ಏಕೆ ಬಂದಿದೆ 
ಎನ್ನುವುದನ್ನು ಯೋಚಿಸೋಣ. 
ಯಾವುದೋ ಸಂಕಟ ಬಂದ ಸಮಯದಲ್ಲಿ 
`ವೆಂಕಟರಮಣ' ಎಂದು ಹೇಳುತ್ತಾರೆ. 
ಸಂಕಟ ಹೋದ ಮೇಲೆ `ವೆಂಕಟರಮಣ' 
ಮರೆತು ಹೋಗುತ್ತದೆ. ಅಂಥಹ 
ಭಕ್ತಿಯುಳ್ಳ ಭಕ್ತನನ್ನು 
ಕುರಿತು ಪ್ರಹ್ಲಾದನು, `ನ ಸ 
ಭೃತ್ಯಃ ಸ ವೈ ವಣಿಕ್' 
ಎಂದಿದ್ದಾನೆ. ಅಂದರೆ, ಯಾರಾದರೂ 
ಭಗವಂತನ ಮೇಲೆ ಯಾವುದಾದರೂ 
ಕಷ್ಟ ದೂರವಾಗುವವರೆಗೋ, ಬೇರೆ 
ಯಾವುದಾದರೂ ಕಾರಣದಿಂದಲೋ 
ಪ್ರೀತಿ ಇಟ್ಟುಕೊಂಡಿದ್ದರೆ 
ಅವನು ಭಗವಂತನೊಡನೆ ವ್ಯಾಪಾರ 
ಮಾಡುವ ಒಬ್ಬ ವ್ಯಾಪಾರಿಯೇ 
ಆಗುತ್ತಾನೆ ಹೊರತು ಭಗವಂತನ 
ನಿಜವಾದ ದಾಸನಲ್ಲ. ವ್ಯಾಪಾರಿ 
ಏನು ಮಾಡುತ್ತಾನೆ? ಹಣ ಕೊಟ್ಟರೆ 
ಸಾಮಾನು ಕೊಡುತ್ತಾನೆ. ಅದೇ 
ರೀತಿ ನಾವೂ ಭಕ್ತಿ ಎನ್ನುವ 
ಹಣವನ್ನು ಕೊಟ್ಟು ಭಗವಂತನ 
ಹತ್ತಿರ ಸಾಮಾನು ಕೊಳ್ಳಲು 
ಹೋದರೆ ನಾವು 
ವ್ಯಾಪಾರಿಗಳೇ-ಹಾಗೇ ಆಗಬಾರದು. 
ಭಕ್ತಿ ಯಾವುದಕ್ಕೆ ಮಾಡುವುದು? 
ಭಗವಂತನಿಗಾಗಿಯೇ ಭಕ್ತಿ 
ಮಾಡಬೇಕು. ಆದ್ದರಿಂದಲೇ ನಾರದರು ಭಕ್ತಿಯನ್ನು `ಪರಮ ಪ್ರೇಮ ರೂಪಾ' ಎಂದಿದ್ದಾರೆ.

ಆತ್ಮನ ಮೇಲೆ ಎಲ್ಲರಿಗೂ 
ಪರಮಪ್ರೇಮವಿರುತ್ತದೆ. 
ಏಕೆಂದರೆ `ನಾನು ನನಗೆ 
ಪ್ರಿಯವಿಲ್ಲದೆ ಮತ್ತೊಬ್ಬನೇ 
ನನಗೆ ಪ್ರಿಯ? ನಾನು ಯಾವಾಗಲೂ 
ನನಗೆ ಪ್ರಿಯವೇ. ನಾನು ನನಗೆ ಅಪ್ರಿಯವಾಗಿರುವುದೇ ಇಲ್ಲ' ಎಂದೇ ಯಾರಾದರೂ ಹೇಳುತ್ತಾರೆ.

ಹಿರಣ್ಯಕಶಿಪುವಿನ ವಧೆಯಾಯಿತು. 
ಅದಾದ ಮೇಲೆ ನೃಸಿಂಹಸ್ವಾಮಿ 
ಪ್ರಹ್ಲಾದನನ್ನು ಕುರಿತು 
`ನಿನ್ನ ಮನಸ್ಸಿನಲ್ಲಿ 
ಇಷ್ಟವೇನೋ ಅದನ್ನು ವರವಾಗಿ 
ಕೇಳು' ಎಂದನು. ಆಗ ಬೇರೆ ಯಾವುದನ್ನು ವರವಾಗಿ ಕೇಳದೆ ಪ್ರಹ್ಲಾದನು,

\begin{shloka}
`ಯಾ ಪ್ರೀತಿರವಿವೇಕಾನಾಂ ವಿಷಯೇಷ್ವನಪಾಯಿನೀ|\\
ತ್ವಾಮನುಸ್ಮರತಃ ಸಾ ಮೇ ಹೃದಯಾನ್ಮಾಽಪಸರ್ಪತು||'
\end{shloka}

`ಯಾ ಪ್ರೀತಿರವಿವೇಕಾನಾಂ 
ವಿಷಯೇಷ್ವನಪಾಯಿನೀ'-ಲೌಕಿಕ 
ವಸ್ತುಗಳಲ್ಲಿ ಪ್ರೀತಿ 
ಇಟ್ಟುಕೊಂಡಿರುವವರಿಗೆ, 
ಅವುಗಳನ್ನು ಬಿಟ್ಟರೆ ಬೇರೆ 
ಯಾವುದರ ಬಗ್ಗೆಯೂ ಚಿಂತೆಯೇ 
ಇರುವುದಿಲ್ಲ. ಒಬ್ಬನಿಗೆ ಒಂದು 
ಮಗುವಿನ ಮೇಲೆ ಹೆಚ್ಚಾಗಿ 
ಪ್ರೀತಿ ಇರುತ್ತದೆ. 
ಹಾಗಿರುವಾಗ ಆ ಮಗು ಯಾವುದಾದರೂ 
ಕಾರಣದಿಂದ ಮರಣವನ್ನು ಪಡೆದರೆ 
ಬೇರೆ ಯಾವುದಾದರೂ ಒಳ್ಳೆಯ 
ಜನ್ಮವನ್ನು ಪಡೆಯಬಹುದು. ಆದರೆ 
ಜೀವಿಸುವ ಇವನು `ಮಗು 
ಹೋಗಿಬಿಟ್ಟಿತು, ನಾನು 
ಹೋಗಲಿಲ್ಲವಲ್ಲಾ' ಎಂದು 
ವ್ಯಥೆಪಡುತ್ತಿರುತ್ತಾನೆ. 
ವಸ್ತುಗಳ ಮೇಲೆ ಪ್ರೀತಿ 
ಇಟ್ಟುಕೊಂಡಿರುವವರು 
ಯಾವುದಾದರೂ ವಿಷಯದಲ್ಲಿ 
ದೃಢವಾಗಿ ಪ್ರೇಮ ಉಂಟಾದರೆ ಅವರು ಅದರಲ್ಲೇ ಹುಚ್ಚರಾಗಿರುವಂತೆ ಆಗಿ ಬಿಡುತ್ತಾರೆ.

`ತ್ವಾಮನುಸ್ಮರತಃ ಸಾ ಮೇ 
ಹೃದಯಾನ್ಮಾಽಪಸರ್ಪತು'. `ಪಾಮರ 
ಜನರಲ್ಲಿ ಇಹ ವಸ್ತುಗಳಲ್ಲಿ 
ಆಸೆ ಎನ್ನುವ ಪ್ರೇಮ ಇರುವಂತೆ?' 
ನಿನ್ನ ವಿಷಯವಾಗಿ ಪ್ರೇಮ ನನ್ನ 
ಹೃದಯದಿಂದ ಎಂದೂ ಹೋಗದಿರಲಿ' 
ಎಂದು ಪ್ರಹ್ಲಾದನು 
ಪ್ರಾರ್ಥಿಸಿದನು. ಅದೇ `ಪರಮ 
ಪ್ರೇಮರೂಪಾ' ಎಂದು ಹೇಳಲ್ಪಟ್ಟ 
ಭಕ್ತಿಯೆಂದು ಹೇಳುತ್ತೇವೆ. 
ಆದರೆ, ಸಾಮಾನ್ಯವಾಗಿ ನಾವು 
ಭಜನೆ ಮಾಡುವುದನ್ನೂ, ದೈವಿಕ 
ಕಥೆ ಓದುವುದನ್ನೂ ಭಕ್ತಿ 
ಎನ್ನುತ್ತೇವೆ. ಇವುಗಳೆಲ್ಲವೂ 
ಭಕ್ತಿಗೆ 
ಸಾಧನವಾಗಿರುವುದರಿಂದ 
ಇವುಗಳನ್ನು ಭಕ್ತಿಯೆಂದೇ 
ಹೇಳಬಹುದು. ಎಲ್ಲರೂ 
ಜೀವನ್ಮುಕ್ತರಲ್ಲ. ಆದರೂ, 
ಜೀವನ್ಮುಕ್ತನಾಗಲು ಯಾರಾದರೂ 
ಸಾಧನೆ ಮಾಡುತ್ತಿದ್ದರೆ 
ಅವನನ್ನು, `ಏನಪ್ಪಾ! ದೊಡ್ಡ 
ಜ್ಞಾನಿ' ಎನ್ನುತ್ತೇವೆ. 
ಜ್ಞಾನಮಾರ್ಗದಲ್ಲಿ 
ನಡೆಯುವುದರಿಂದ, ಜ್ಞಾನ ಸಾಧನ 
ಅನುಷ್ಠಾನದಿಂದ, ಅಂಥಹವನನ್ನು 
ಜ್ಞಾನಿ ಎನ್ನುತ್ತೇವೆ. ಆದ್ದರಿಂದ ಭಾಗವತದಲ್ಲಿಯೂ-

\begin{shloka}
`ಭಕ್ತ್ಯಾ ಸಂಜಾತಯಾ ಭಕ್ತ್ಯಾ ವಿಪ್ರತ್ಯುತ್ಪುಲಕಾಂತನುಮ್'
\end{shloka}

ಎಂದಿದೆ. ಸಾಮಾನ್ಯ ಭಕ್ತಿಯಿಂದ 
ಉಂಟಾಗುವ ಫಲವೂ ಇಲ್ಲಿ ಹೇಳಿದೆ. 
ಅದೇನೆಂದರೆ ಸಾಧನವಾಗಿರುವ 
ಭಕ್ತಿಯಿಂದ ಸಾಧ್ಯವಾಗಿರುವ 
ಭಕ್ತಿ ಉಂಟಾಗುತ್ತದೆ. 
ಸಾಧನವಾದ ಭಕ್ತಿ ಯಾವುದು? 
ಭಗವಂತನ ಲೀಲಾವಿಭೂತಿಗಳನ್ನು 
ಕುರಿತು ಭಾಗವತ, ಸ್ಕಾಂದ 
ಪುರಾಣಗಳಂಥಹ ಪುರಾಣಗಳನ್ನು 
ಹಲವು ಬಾರಿ ಓದಿದರೆ ಮನಸ್ಸು 
ಅಲ್ಲೇ ಮಗ್ನವಾಗಿ ಬಿಡುತ್ತದೆ. 
ಹಾಗೆ ಮನಸ್ಸು ಮಗ್ನವಾದ ಮೇಲೆ 
ಆನಂದವಾಗುತ್ತದೆ. ಆ 
ಪ್ರೇಮದಲ್ಲಿ ವಿಶೇಷವಾಗಿ 
ಇರುವುದಾದರೂ ಏನು? 
`ವಿಪ್ರತ್ಯುತ್ಪುಲಕಾಂ 
ತನುಮ್-ಆ ಭಕ್ತಿಸಾಧನದಿಂದ 
ಉಂಟಾಗುವ ಸಾಧ್ಯಭಕ್ತಿ ಇರುವಾಗ 
ಭಕ್ತನ ಶರೀರದಲ್ಲಿ ರೋಮಾಂಚನವಾಗುತ್ತದೆ. ಸಾಧನಾ ಭಕ್ತಿಯನ್ನು ಕುರಿತು,

\begin{shloka}
ಶ್ರವಣಂ ಕೀರ್ತನಂ ವಿಷ್ಣೋಃ ಸ್ಮರಣಂ ಪಾದಸೇವನಮ್|\\
ಅರ್ಚನಂ ವಂದನಂ ದಾಸ್ಯಂ ಸಖ್ಯಮಾತ್ಮ ನಿವೇದನಮ್||\\
ಇತಿ ಪುಂಸಾರ್ಪಿತಾ ವಿಷ್ಣೌ ಭಕ್ತಿಶ್ಚೇನ್ನವಲಕ್ಷಣಾ|\\
ಕ್ರಿಯತೇ ಭಗವತ್ಯಧ್ಧಾ ತನ್ಮನ್ಯೇಽಧೀತಮುತ್ತಮಮ್||
\end{shloka}

ಪ್ರಹ್ಲಾದನು ಕೊಟ್ಟ ವಿವರಣೆ 
ಇದು. ಪ್ರಹ್ಲಾದನನ್ನು 
ಹಿರಣ್ಯಕಶಿಪು `ಇಷ್ಟು ದಿನಗಳು 
ಗುರುಕುಲದಲ್ಲಿ ಯಾವ 
ವಿದ್ಯೆಯನ್ನು ಕಲಿತುಕೊಂಡು 
ಬಂದೆ' ಎಂದು ಕೇಳಿದಾಗ ಅವನು, `ಇದೇ ಉತ್ತಮವಾಗಿರುವ ವಿದ್ಯೆ' ಎಂದನು, ಅದು ಯಾವುದು?

\begin{shloka}
`ಶ್ರವಣಂ ಕೀರ್ತನಂ ವಿಷ್ಣೋಃ ಸ್ಮರಣಂ ಪಾದಸೇವನಮ್|\\
ಅರ್ಚನಂ ವಂದನಂ ದಾಸ್ಯಂ ಸಖ್ಯಮಾತ್ಮ ನಿವೇದನಮ್||
\end{shloka}

ಪರಮೇಶ್ವರನ ಬಗ್ಗೆ ಕೇಳುವುದು, 
ಅವನನ್ನು ಸ್ಮರಿಸುವುದು, 
ಸೇವಿಸುವುದು, 
ನಮಸ್ಕರಿಸುವುದು, ಅವನ 
ದಾಸನಾಗಿರುವುದು, ಅವನನ್ನು 
ಸ್ನೇಹಿತನನ್ನಾಗಿ 
ಮಾಡಿಕೊಳ್ಳುವುದು, ಅವನಿಗೆ 
ಆತ್ಮ ನಿವೇದನೆ 
ಮಾಡಿಕೊಳ್ಳುವುದು ಇದೆ 
ಉತ್ತಮವಾದ ವಿದ್ಯೆ. 
ಇವುಗಳಲ್ಲಿ 
ಕೊನೆಯಲ್ಲಿರುವುದು 
ಆತ್ಮನಿವೇದನೆ. ಅದು ಹೇಗೆಂದರೆ 
ಅಲ್ಲೇ ಹೋಗಿ ಸೇರಿಬಿಡುವುದು. 
ಹಾಗೆ ಸೇರಿದ ಮೇಲೆ ಭಗವಂತನು 
ಯಾರು, ತಾನು ಯಾರು ಎನ್ನುವುದೇ 
ಅವನಿಗೆ ತಿಳಿಯುವುದಿಲ್ಲ. 
ಆದ್ದರಿಂದಲೇ ಭಗವತ್ಪಾದರು, 
ಭಕ್ತಿಯಲ್ಲಿರುವ 
ಮೆಟ್ಟಲುಗಳನ್ನು 
ತೋರಿಸುವುದಕ್ಕಾಗಿ ಶಿವಾನಂದಲಹರಿಯಲ್ಲಿ ಹೀಗೆ ಒಂದು ಶ್ಲೋಕವನ್ನು ಹಾಡಿದರು.

\begin{shloka}
`ಅಂಕೋಲಂ ನಿಜಬೀಜಸಂತತಿರಯಸ್ಕಾಂತೋಪಲಂ ಸೂಚಿಕಾ\\
ಸಾಧ್ವೀ ನೈಜವಿಭುಂ ಲತಾ ಕ್ಷಿತಿರುಹಂ ಸಿಂಧುಃ ಸರಿದ್ವಲ್ಲಭಮ್|\\
ಪ್ರಾಪ್ನೋತೀಹ ಯಥಾ ತಥಾ ಪಶುಪತೇಃ ಪಾದಾರವಿಂದದ್ವಯಂ\\
ಚೇತೋವೃತ್ತಿರುಪೇತ್ಯ ತಿಷ್ಠತಿ ಸದಾ ಸಾ ಭಕ್ತಿರಿತ್ಯುಚ್ಯತೇ||'
\end{shloka}

ನಮ್ಮ ಮನೋಭಾವ ಹೀಗಿದ್ದರೆ 
ಅದಕ್ಕೆ ಭಕ್ತಿ ಎಂದು ಹೆಸರು 
ಎನ್ನುತ್ತಾರೆ. ಅದು 
ಹೇಗಿರಬೇಕೆಂದರೆ ಮೊಟ್ಟ ಮೊದಲು 
ಭಗವಂತನು ಬಂದು, `ನಾನೇ ಭಗವಂತ, 
ನನ್ನನ್ನು ನೀನು ಆರಾಧಿಸು' ಎಂದು ಹೇಳಿದರೂ ಕೂಡ ಯಾರೂ ಕೇಳುವುದಿಲ್ಲ. 

ಅರ್ಜುನನು ಕೃಷ್ಣಪರಮಾತ್ಮನೊಡನೆ ಇದ್ದನು. ಹಾಗಿದ್ದರೂ-

\begin{shloka}
`ಸಖೇತಿ ಮತ್ವಾ ಪ್ರಸಭಂ ಯುದುಕ್ತಂ\\
ಹೇ ಕೃಷ್ಣ ಹೇ ಯಾದವ ಹೇ ಸಖೇತಿ|'
\end{shloka}

ಎಂದು ಅವನು ಹೇಳುವ ರೀತಿಯನ್ನು 
ನೋಡಿದರೆ ಅವನು ಭಗವಂತನನ್ನು 
ಕೇವಲ 
ಸ್ನೇತಿತನೆಂದುಕೊಂಡುಬಿಟ್ಟಿದ್ದನು. ವಿಶ್ವರೂಪದರ್ಶನವಾಗುವವರೆಗೆ `ನಾನು ತಪ್ಪು ಮಾಡಿಬಿಟ್ಟೆ. ಭಗವಂತನೇ ನಮ್ಮೊಡನೆ ಹೀಗೆ ಬಂದು ಆಟವಾಡುತ್ತಿದ್ದಾನೆ' ಎಂದು ಅನಂತರ ಭಾವಿಸುತ್ತಾನೆ. ಆದ್ದರಿಂದ, ಈ ಕಾಲದಲ್ಲಿ, ಸಾಮಾನ್ಯ ಜನರೂ, ಭಗವಂತನೇ ಬಂದು ಹೇಳಿದರೂ ಕೂಡ `ಅವನು ಭಗವಂತ'ನೆಂದು ಭಾವಿಸುವುದಿಲ್ಲ. ಆದ್ದರಿಂದ, ಆಚಾರ್ಯರಾಗಿರುವವರು ಭಾಗವತದಂತಹ ಗ್ರಂಥಗಳನ್ನು ಹೇಳಿ ಜನರ ಮನಸ್ಸಿನಲ್ಲಿ ಒಂದು ಉತ್ತಮವಾದ ಮನೋಭಾವನೆ ಉಂಟುಮಾಡಬೇಕು. ಮೇಲೆ ಹೇಳಿದ ಭಗವತ್ಪಾದರ ಶ್ಲೋಕದಲ್ಲಿ-

`ಅಂಕೋಲಂ ನಿಜ ಬೀಜ ಸಂತತಿಃ' 
ಎಂದು ಮೊದಲು ಕಾಣುತ್ತದೆ. 
ಅಂಕೋಲವೆನ್ನುವುದು ಒಂದು 
ವೃಕ್ಷ. ಆ ವೃಕ್ಷದ 
ಸ್ವರೂಪವೇನೆಂದರೆ, ಬೀಜ ಬಂದರೆ 
ಆ ಮರಕ್ಕೆ ಹೋಗಿ ಹಾಗೆಯೇ 
ಹಿಡಿದುಕೊಂಡುಬಿಡುತ್ತದೆ. 
ಅದರಲ್ಲಿ ಸ್ವಲ್ಪ 
ಗೋಂದಿನಂತಿರುತ್ತದೆ. ಮರ ಒಂದು 
ಕೆಲಸವನ್ನಾದರೂ ಮಾಡುವುದಿಲ್ಲ. 
ಬೀಜ ಗಾಳಿಯ ಮೂಲಕ ತಾನಾಗಿಯೇ 
ಮರದ ಹತ್ತಿರ ಹೋಗುತ್ತದೆ. 
ಮತ್ತೆ ತಾನೇ ಮರವನ್ನು 
ಹಿಡಿದುಕೊಳ್ಳುತ್ತದೆ ಅಷ್ಟೇ. ಅದೇ ರೀತಿ ಮೊಟ್ಟಮೊದಲು ಭಕ್ತನ ಭಾಗ್ಯ ವಿಶೇಷದಿಂದ ಅವನು ತಾನಾಗಿಯೇ ಹೋಗಿ ಭಗವಂತನನ್ನು ಹಿಡಿದುಕೊಳ್ಳುತ್ತಾನೆ ಎನ್ನುತ್ತಾರೆ. 

ಅವನ ಮನೋವೃತ್ತಿ ಎನ್ನುವುದು ಭಗವದಾಕಾರವಾಗಿ ಹೋಗುತ್ತದೆ. ಅನಂತರ `ಆಯಸ್ಕಾಂತೋಪಲಂ ಸೂಚಿಕಾ' ಎಂದು ಹೇಳುತ್ತಾರೆ. ಭಗವಂತನಿಗೆ ಕರುಣೆ ಇದೆ. `ನನ್ನ ಬಳಿಗೆ ಬಂದು ಆಶ್ರಯಿಸಿದ್ದಾನೆ. ಇವನನ್ನು ನಾನು ಅನುಗ್ರಹಿಸಬೇಕು' ಎಂದು ಭಗವಂತನಿಗೆ ಕರುಣೆ ಉಂಟಾಗುತ್ತದೆ. ಕರುಣೆ ಉಂಟಾದ ಮೇಲೆ ಭಗವಂತನ ಹತ್ತಿರ ಬಂದು ಸೇರಿದವನು ಮತ್ತೆ ಬಿಟ್ಟು ಹೋಗಬೇಕೆಂದು ಕೊಂಡರೂ ಬಿಟ್ಟು ಹೋಗಲಾಗುವುದಿಲ್ಲ. ಭಗವಂತನ ಲೀಲಾ ವಿಭೂತಿಗಳ ಮಹಿಮೆ ಏನೆಂದರೆ ಮೊದಲ ದಿವಸ ಭಾಗವತ ಗೋಷ್ಠಿಗೆ ಹೋಗಿ ಸೇರಿದರೆ ಮಾರನೆಯ ದಿನವೂ ಅದೇ ಗೋಷ್ಠಿಗೆ ಹೋಗಿ ಸೇರಬೇಕೆಂದು ತೋರುತ್ತದೆ. ಸಿನಿಮಾ ಒಂದು ದಿನ ಹೋಗಿ ನೋಡಿ ಹುಚ್ಚು ಹಿಡಿಯಿತು ಎಂದರೆ ಮತ್ತೆ ನಾಲ್ಕು ದಿನಗಳು ಆಗಲಿಲ್ಲವೆಂದರೆ ಸಾಲ ತೆಗೆದುಕೊಂಡೋ ಅಥವಾ ಮನೆಯಿಂದ ಕಳ್ಳತನ ಮಾಡಿಯೋ ಸಿನಿಮಾ ನೋಡುತ್ತಾನೆ. ಇದೇ ರೀತಿ ಭಗವದ್ವಿಷಯದಲ್ಲೂ ಹುಚ್ಚು ಹಿಡಿದು ಬಿಡುತ್ತದೆ ಎನ್ನುತ್ತಾರೆ. ಹೇಗೆಂದರೆ `ಅಯಸ್ಕಾಂತೋಪಲಂ ಸೂಚಿಕಾ' ಅಯಸ್ಕಾಂತವೆನ್ನುವುದು ಕಬ್ಬಿಣವನ್ನು ತನ್ನೊಡನೆ ಹಿಡಿದುಕೊಳ್ಳುತ್ತದೆ. ಹಾಗೆ ಹಿಡಿದುಕೊಂಡ ಮೇಲೆ ಆ ಕಬ್ಬಿಣವನ್ನು ಎಳೆಯುವುದು ಕಷ್ಟ. ತಾನಾಗಿಯೇ ಅಯಸ್ಕಾಂತ ಬಿಟ್ಟುಬಿಡುವುದಿಲ್ಲ. ಇನ್ನೊಬ್ಬರು ಬಲಾತ್ಕಾರವಾಗಿ ಅದನ್ನು ಹಿಡಿದು ಎಳೆದರೆ ಮಾತ್ರ ಕಬ್ಬಿಣ ಬೇರೆಯಾಗಿ ಬರಬಲ್ಲದು. ಆದ್ದರಿಂದ ಭಕ್ತನು ಭಗವಂತನ ಹತ್ತಿರ ಬಂದ ಮೇಲೆ ಭಗವಂತನು ಭಕ್ತನನ್ನು ಬಲವಾಗಿ ಹಿಡಿದುಕೊಳ್ಳುವನು. 

ಮತ್ತೆ `ಸಾಧ್ವೀ ನೈಜವಿಭುಂ'-ಹೇಗೆಂದರೆ, ಪತಿವ್ರತೆಯಾದ ಹೆಂಗಸು ಎಷ್ಟು ವಿಶ್ವಾಸದಿಂದ ಪತಿಯನ್ನು ಆರಾಧಿಸಿಕೊಂಡು ಬರುವಳೋ ಅದೇ ರೀತಿ ಪತಿವ್ರತೆಯಾಗಿರುವ ಸ್ತ್ರೀಯನ್ನು ಪತಿ ಎಷ್ಟು ವಿಶ್ವಾಸದಿಂದಲೂ, ಪ್ರೇಮದಿಂದಲೂ, ಕಾಣುತ್ತಾ ಬರುತ್ತಾನೋ, ಹಾಗೆಯೇ ಭಗವಂತನು ಭಕ್ತನನ್ನು ತನ್ನವನೆಂದುಕೊಳ್ಳುತ್ತಾನೆ. ಇವನೂ ಭಗವಂತನು ನನ್ನವನೆಂದುಕೊಳ್ಳುತ್ತಾನೆ. ಹೀಗೆ ಅನ್ಯೋನ್ಯವಾದ ಪ್ರೇಮದಿಂದ ಬಂಧ ಉಂಟಾಗುವುದು ಮೂರನೆಯ ಮೆಟ್ಟಲು ಎನ್ನುತ್ತಾರೆ.

ಅನಂತರ `ಲತಾ ಕ್ಷಿತಿರುಹಂ'-ನಾಲ್ಕನೆಯ ಮೆಟ್ಟಲಲ್ಲಿ ಆಗುವುದು ಹೀಗೆ. ಭಗವಂತನು ಮರದಂತೆ ಆಶ್ರಯವಾಗಿರುವವನು. ಆ ಮರವನ್ನಾಶ್ರಯಿಸಿದ ಬಳ್ಳಿಯಂತೆ ಮೇಲಕ್ಕೆ ಹೋಗುತ್ತಿರುವವನು ಭಕ್ತ. ಒಂದು ಮರಕ್ಕೆ ಉತ್ತಮವಾದ ಮಲ್ಲಿಗೆ ಬಳ್ಳಿ ಹಬ್ಬಿಕೊಂಡು ಹೋದರೆ ಆ ಮರಕ್ಕೆ ಎಷ್ಟು ಶೋಭೆ ಉಂಟಾಗುತ್ತದೆ! ಅದೇ ರೀತಿ ಭಕ್ತನಿಂದ ಭಗವಂತನಿಗೆ ಶೋಭೆ ಉಂಟಾಗುತ್ತದೆ. ಅದು ಎಂಥ ಶೋಭೆ?

ಭಗವಂತನು ಭಕ್ತರನ್ನು ಅನುಗ್ರಹಿಸಲು ಅನೇಕ ರೂಪಗಳನ್ನು ಧರಿಸುತ್ತಾನೆ. 

\begin{shloka}
`ಆವಜಾನನ್ತಿ ಮಾಂ ಮೂಢ ಮಾನುಷೀಂ ತನುಮಾಶ್ರಿತಮ್|'\\
`ಪರಂ ಭಾವಮಜಾನನ್ತೋ ಮಮ ಭೂತಮಹೇಶ್ವರಮ್||'
\end{shloka}

ನಿರ್ಗುಣವಾದ ಪರವಸ್ತು ಹಲವು ವಿಧವಾಗಿ ದರ್ಶನವನ್ನು ಕೊಡುವುದರಿಂದ ಭಗವಂತನಿಗೇ ಹಿರಿಮೆಯುಂಟಾಗುತ್ತದೆ. ಭಕ್ತನಿಲ್ಲದಿದ್ದರೆ ಭಗವಂತನು ದರ್ಶನ ಕೊಡುವ ಅವಶ್ಯಕತೆ ಇಲ್ಲ. ದರ್ಶನ ಕೊಡುವ ಸಮಯದಲ್ಲಿ ಲೀಲೆಗಳನ್ನೆಲ್ಲಾ ತೋರಿಸಬೇಕಾಗುತ್ತದೆ. ಲೀಲೆಗಳನ್ನು ತೋರಿಸಲಿಲ್ಲವೆಂದರೆ ಸಾಮಾನ್ಯ ಜನರಿಗೆ ಭಗವಂತನ ಸ್ಮರಣೆ ಹೇಗಾಗಬೇಕು? ಆದ್ದರಿಂದ ಭಗವಂತನಿಗೆ ಭಕ್ತ ಒಂದು ಅಲಂಕಾರವಾಗಿದ್ದಾನೆಂದು ಹೇಳಿಬಿಟ್ಟರು. ಒಬ್ಬ ಪ್ರಹ್ಲಾದನು ಹುಟ್ಟದೆ ಇದ್ದಿದ್ದರೆ ಈ ನೃಸಿಂಹಾವತಾರ ಲೀಲೆ ನಡೆಯುತ್ತಿತ್ತೇ? ನಾವು ವಿಷ್ಣುವನ್ನು ವೈಕುಂಠವಾಸಿಯೆಂದು ಮಾತ್ರ ಹೇಳಿದ್ದೇವೆ ವಿನಾ ನೃಸಿಂಹರೂಪದಲ್ಲಿ ಪೂಜೆ ಮಾಡುತ್ತಿದ್ದೆವೋ? ಪ್ರಹ್ಲಾದನನ್ನು ಅನುಗ್ರಹಿಸುವುದಕ್ಕಾಗಿ ಭಗವಂತನು ಬಂದನು. ಆದ್ದರಿಂದ ನೃಸಿಂಹ ರೂಪದಲ್ಲಿ ಅವನನ್ನು ಆರಾಧಿಸುವೆವು. ಎಲ್ಲಾ ರೂಪಗಳೂ ಹೀಗೆಯೇ. ಕೃಷ್ಣಾವತಾರದಲ್ಲಿ ಪಂಚಪಾಂಡವರು ಹುಟ್ಟದೆ ಹೋಗಿದ್ದರೆ, ಅನುಗ್ರಹಿಸಲ್ಪಡಬೇಕಾದವರೇ ಇಲ್ಲದೆ ಹೋದರೆ ಕೃಷ್ಣಾವತಾರವಾಗಬೇಕಾದ ಆವಶ್ಯಕತೆ ಇಲ್ಲ. ಆದ್ದರಿಂದ ಭಗವಂತನು ಭಕ್ತನನ್ನು ಅನುಗ್ರಹಿಸಲು ಬರುವುದರಿಂದ, ಭಕ್ತನಿಂದ ಭಗವಂತನಿಗೆ ಒಂದು ಅಲಂಕಾರ ಉಂಟಾಗುತ್ತದೆ ಎನ್ನುತ್ತಾರೆ.

ಕೊನೆಯಲ್ಲಿ, `ಸಿಂಧುಃ ಸರಿದ್ವಲ್ಲಭಮ್'. ಮೊದಲು ಹೇಳಿದ ಭಕ್ತನ ಮನಸ್ಸೆನ್ನುವ ಹೊಳೆ ಭಗವಂತನೆನ್ನುವ ಸಮುದ್ರದ ಕಡೆಗೆ ಹರಿದುಕೊಂಡೇ ಹೋಗುತ್ತದೆ. ಹೊಳೆ ಎಷ್ಟು ದೂರ ಹರಿದು ಹೋಗುತ್ತದೆ? ಸಮುದ್ರ ಸಿಕ್ಕುತ್ತಲೇ ಅದೂ ಸಮುದ್ರವಾಗಿಯೇ ಆಗಬೇಕು. ಅದೇ ರೀತಿ ಭಕ್ತನ ಈ ಮನೋಭಾವನೆ ಬರ-ಬರುತ್ತಾ ಭಗವತ್ಸ್ವರೂಪವಾಗಿ ಬಿಡುತ್ತದೆಂದು ಹೇಳಿದರು. ಆಗ ಮನಸ್ಸೆನ್ನುವುದು ಇರುವುದಿಲ್ಲ. ಬೇರೆ ಆತ್ಮನೂ ಇಲ್ಲ.

\begin{shloka}
`ಮಮೈವಾಂಶೋ ಜೀವಲೋಕೇ ಜೀವಭೂತಃ ಸನಾತನಃ'
\end{shloka}

`ಜೀವ' ಎಂದರೆ ಬೇರೆ ಎಂದು ತಿಳಿಯಬೇಕಾಗಿಲ್ಲ. `ಅವನು ನನ್ನ ಒಂದು ಅಂಶ. ನನ್ನ ಒಂದುಭಾಗ' ಎನ್ನುತ್ತಾನೆ ಭಗವಂತನು. `ಭಾಗವಿದೆ' ಎಂದರೆ ಅದು ಹೋಗಿ ಸೇರುವವರೆಗೆ ಮಾತ್ರ ಭಾಗ. ಸೇರಿದ ಮೇಲೆ ಒಂದೇ. ಒಂದು ದೊಡ್ಡ ಕೊಳದಿಂದ ನೀರು ತೆಗೆದು ನಾವು ಪಾತ್ರೆಯಲ್ಲಿಟ್ಟರೆ ಅದು ಬೇರೆ, ಇದು ಬೇರೆ. ಮತ್ತೆ ನೀರು ಆ ಕೊಳಕ್ಕೆ ಬಿದ್ದು ಬಿಟ್ಟರೆ ಆ ರೀತಿಯಾದ ಬೇರೆ ತನ ಯಾವುದೂ ಇಲ್ಲ. ಬೇರೆ ಬೇರೆ ನೀರಾಗಿದ್ದುದು ಈಗ ಒಂದಾಗಿ ಸೇರಿಬಿಟ್ಟಿತು. ಹಾಗೆಯೇ ಭಗವದಾಕಾರವನ್ನು ಮನಸ್ಸು ಪಡೆದು ಬಿಟ್ಟಿತೆಂದರೆ ಆಗ ಜ್ಞಾನ ಉಂಟಾಗುತ್ತದೆ. ಅದನ್ನು ಗಮನಿಸಿಯೇ `ಸಖ್ಯಂ ಆತ್ಮ ನಿವೇದನಂ' ಎಂದು ಪ್ರಹ್ಲಾದನು ಹೇಳಿದನು. ಆತ್ಮನಿವೇದನೆ ಎನ್ನುವುದು ಬಂದು ಬಿಟ್ಟರೆ ಜ್ಞಾನ ಉಂಟಾಗುತ್ತದೆ. ಇಂಥಹ ಪ್ರೇಮವಾದ ಆ ಭಕ್ತಿ ಬರ ಬರುತ್ತಾ ಜ್ಞಾನಮಾರ್ಗವನ್ನು ಅವಲಂಬಿಸುತ್ತದೆಂದು ಹೇಳಿದ್ದಾರೆ. ಭಕ್ತಿ ಮಾರ್ಗಕ್ಕೆ ಜ್ಞಾನಮಾರ್ಗಕ್ಕಿಂತ ಹೆಚ್ಚಾಗಿರುವ ವಿಶೇಷವೆಂದರೆ ಸಾಧನಾಕಾಲದಲ್ಲಿ ಸುಖದ ಅನುಭವ. ಜ್ಞಾನಮಾರ್ಗದಲ್ಲಿ ಅದು ಅಷ್ಟಾಗಿಲ್ಲ. ಭಕ್ತನಿಗೆ ಸಾಧನಾಕಾಲದಲ್ಲಿಯೂ ಸುಖಾನುಭವವಿದೆ ಎನ್ನುತ್ತಾರೆ ಕೆಲವರು. ದೇವರನ್ನು ಉದ್ದೇಶಿಸಿ, `ನಾನು ಹುಲ್ಲಾಗಿದ್ದರೆ ನಿನಗೆ ಸೇವೆ ಮಾಡುವ ಹಸುವಿಗೆ ಒಂದು ಹಿಡಿ ಹುಲ್ಲಾಗಿ ಬಿಡುವೆನು. ಒಂದು ಕಲ್ಲಾಗಿದ್ದರೆ ಮೆಟ್ಟಲು ಮೇಲೆ ಹತ್ತಿ ಭಗವಂತನ ದರ್ಶನಕ್ಕೆ ಹೋಗುವವರಿಗಾಗಿ ಮಾಡಲ್ಪಡುವ ಮೆಟ್ಟಲಾದರೆ ಜನ್ಮ ಸಾರ್ಥಕ' ಎನ್ನುತ್ತಾರೆ.

ಒಬ್ಬರು, `ನಾನು ಇಷ್ಟು ದಿನಗಳೂ ಭಗವದ್ಭಕ್ತನಾಗಿದ್ದೆ. ಭಗವಂತನ ಆರಾಧನೆಗಾಗಿ ಎಲ್ಲವನ್ನೂ ವಿನಿಯೋಗಿಸಿ ಆಯಿತು. ಈಗ ಪ್ರಾಣ ಕೊಡುವ ಕಾಲ ಬಂದಿದೆ. ಈ ದೇಹ ಹೇಗೆ ಭಗವಂತನಿಗೆ ಉಪಯೋಗವಾಗುತ್ತದೆಂದು ಚಿಂತಿಸಿ,

\begin{shloka}
ಪಂಚತ್ವಂ ತನುರೇಷ ಭೂತನಿವಹಾಃ ಸ್ವಾಂಶೇ ವಿಶಂತು ಸ್ಫುಟಂ\\
ದಾತಾರಂ ಪ್ರಣಿಪತ್ಯ ಹಂತ ಶಿರಸಾ ತತ್ರಾಪಿ ಯಾಚೇ ವರಮ್|\\
ದತ್ವಾಪೀಷು ಪಯಃ ತದೀಯಮುಕುರೋ ಜ್ಯೋತಿಸ್ತದೀಯಾಂಕಣ\\
ವ್ಯೋಮ್ನಿವ್ಯೋಮ ತದೀಯ ವರ್ತ್ಮನಿ ಧರಾ ತತ್ತಾಲವೃಂತೇಽನಿಲಃ||
\end{shloka}

ಎಂದು ಪ್ರಾರ್ಥಿಸಿದರು. `ಈ ಶರೀರ ಪಂಚಭೂತಗಳಿಂದ ಏರ್ಪಡುವುದು. ಸತ್ತಮೇಲೆ ಪಂಚಭೂತಗಳು ಏನಾಗುವುವು? ಈ ಶರೀರದಲ್ಲಿರುವ ಪೃಥ್ವಿಯ ಅಂಶ ದೊಡ್ಡ ಆಕಾಶದಲ್ಲಿ ಹೋಗಿ ಸೇರುವುದು. ಶರೀರದಲ್ಲಿರುವ ತೇಜೋ ಅಂಶ ತೇಜಸ್ಸಿನಲ್ಲಿ ಹೋಗಿ ಸೇರಿಬಿಡುವುದು. ಗಾಳಿಯ ಅಂಶ ಗಾಳಿಯಲ್ಲಿ ಹೋಗಿ ಸೇರುವುದು. ಸತ್ತಮೇಲೆ ಈ ಪಂಚಭೂತಗಳೆಲ್ಲ ಎಲ್ಲೋ ಹೋಗಿ ಸೇರುವುದು ಬೇಡ. ಕೃಷ್ಣಪರಮಾತ್ಮನು ಯಾವ ಬಾವಿಯಿಂದ ನೀರು ತೆಗೆದುಕೊಂಡು ಕುಡಿಯುತ್ತಾನೋ ಆ ಬಾವಿಯಲ್ಲಿ ನೀರಿನ ಅಂಶ ಹೋಗಿ ಸೇರಲಿ. ಭಗವಂತನು ಮುಖವನ್ನು ನೋಡುವುದಕ್ಕಾಗಿ ಕನ್ನಡಿ ಹಿಡಿದು ನೋಡುತ್ತಾನೆ, ಈ ಕನ್ನಡಿಯಲ್ಲೇ ನನ್ನ ತೇಜೋಂಶ-ಪ್ರಕಾಶ ಅಂಶ ಸೇರಲಿ. ಭಗವಂತನು ತಿರುಗಾಡಲು ದೊಡ್ಡ ಮೈದಾನವಿದೆಯಂತೆ. ಆ ಮೈದಾನದಲ್ಲೆ ಆಕಾಶ ಅಂಶ ಹೋಗಿ ಸೇರಲಿ. ಭಗವಂತನು ವೃಂದಾವನದಲ್ಲಿ ಸಂಚಾರ ಮಾಡುವಾಗ ಅವನಿಗೆ ಭೂಮಿ ಅವಶ್ಯಕವಾಗುತ್ತದೆ. (ಈ ಕಾಲದಂತೆ ವಿಮಾನದಲ್ಲಿ ಹೋಗುವುದಿಲ್ಲ. ಭೂಮಿಯ ಮೇಲೆ ಹೋಗಬೇಕಾಗುತ್ತದೆ.) ಆ ಭೂಮಿಯಲ್ಲಿ ಪೃಥ್ವಿಯ ಅಂಶ ಹೋಗಿ ಸೇರಲಿ. ಬಿಸಿಲು ಕಾಲದಲ್ಲಿ ಭಗವಂತನು ಬೀಸಣಿಗೆಯಲ್ಲಿ ಬೀಸಿಕೊಳ್ಳುವಾಗ ಆ ಗಾಳಿಯಲ್ಲಿ ನನ್ನ ಗಾಳಿ ಅಂಶ ಹೋಗಿ ಸೇರಲಿ. ನನ್ನ ಮನಸ್ಸು ಆತ್ಮ ಎರಡೂ ಭಗವಂತನಲ್ಲಿ ಸೇರಿಬಿಟ್ಟಿವೆ. ಪಂಚಭೂತಗಳಲ್ಲಿ ಸೇರುವ ನನ್ನ ಅಂಶಗಳೂ ಹೀಗೆ ಸೇರಿದರೆ ಅವನ ಸೇವೆ ಯಾವಾಗಲೂ ನಡೆಯುತ್ತಿರುತ್ತದೆ' ಎನ್ನುವುದು ಪ್ರಾರ್ಥನೆಯ ಭಾವ. ಭಗವಂತನ ಹತ್ತಿರ ಸೇವೆ ಮಾಡುವವರು ಅದರ ರುಚಿ ಎಂಥಹದೆಂಬುದನ್ನು ಹೇಳುವರು. ಭಗವದ್ಭಕ್ತರ ಆನಂದಾನುಭವ ಎಲ್ಲೆಯಿಲ್ಲದ ಆನಂದಾನುಭವ. ಆದ್ದರಿಂದ ಎಲ್ಲರೂ ಆ ಭಕ್ತಿಮಾರ್ಗದಲ್ಲಿ ನಡೆಯಬೇಕು. ಮಾರ್ಗಗಳು ಮೂರು. ಯಾರಿಗೆ ವಿಶೇಷವಾದ ವೈರಾಗ್ಯ, ವಿವೇಕ ಮುಂತಾದವು ಇರುವುದಿಲ್ಲವೋ ಅಂಥಹವರು ಭಕ್ತಿಮಾರ್ಗದಲ್ಲಿ ಹೋಗಬಹುದೆಂದು ಹೇಳಿರುವೆವು. ಕರ್ಮ ಮಾರ್ಗದಲ್ಲಿ ಹೋಗುವುದಕ್ಕೆ ವಿಶೇಷವಾದ ಸಾಮರ್ಥ್ಯವಿರಬೇಕು. ಭಕ್ತಿಮಾರ್ಗದಲ್ಲಿ ಹೋಗುವುದಕ್ಕೆ ಯಾರಾದರೂ ಹಣವಂತನಾಗಿಯೋ, ವಿದ್ಯಾವಂತನಾಗಿಯೋ, ಬುದ್ಧಿಶಾಲಿಯಾಗಿಯೋ ಇರಬೇಕೆ ಎನ್ನುವುದಕ್ಕೆ-

\begin{shloka}
`ವ್ಯಾಧಸ್ಯಾಚರಣಂ ಧ್ರುವಸ್ಯ ಚ ವಯಃ ವಿದ್ಯಾಗಜೇಂದ್ರಸ್ಯ ಕಾ\\
ಕಾ ಜಾತಿರ್ವಿದುರಸ್ಯ ಯಾದವಪತೇರುಗ್ರಸ್ಯ ಕಿಂ ಪೌರುಷಮ್|\\
ಕುಬ್ಜಾಯಾಃ ಕಮನೀಯರೂಪಮಧಿಕಂ ಕಿಂ ವಾ ಸುದಾಮ್ನೋ ಧನಂ\\
ಭಕ್ತ್ಯಾ ತುಷ್ಯತಿ ಕೇವಲಂ ನ ತು ಗುಣೈಃ ಭಕ್ತಿಪ್ರಿಯೋ ಮಾಧವಃ||'
\end{shloka}

ಎಂದು ಉತ್ತರವಾಗುತ್ತದೆ. ಭಗವಂತನು ಎಲ್ಲ ರೀತಿಯ ಭಕ್ತರನ್ನು ಅನುಗ್ರಹಿಸುವುದುಂಟು. ಒಂದು ಕಥೆಯನ್ನು ನೋಡಿ ಅನಂತರ ಆ ಶ್ಲೋಕವನ್ನು ನೋಡೋಣ.

ಒಂದು ಕಾಲದಲ್ಲಿ ಒಂದು ಜಾಗದಲ್ಲಿ ನದಿ ಪ್ರವಾಹ ಉಂಟಾಯಿತು. ಆಗ ಆ ದೇಶದ ರಾಜನು, `ಜನರೆಲ್ಲ ನೀರಿನಲ್ಲಿ ಒಂದೊಂದು ಜಾಗವಾಗಿ ಅಲ್ಲಲ್ಲಿ ಮಣ್ಣು ತುಂಬಿಬಿಡಿ' ಎಂದು ಹೇಳಿದನು. ಹೀಗೆ ಅವನು ಊರಿನಲ್ಲಿರುವವರಿಗೆಲ್ಲ ಕೆಲಸ ಕೊಟ್ಟಾಯಿತು. ಅಲ್ಲಿ ಒಬ್ಬ ಮುದುಕಿ ಇದ್ದಳು. ಅವಳ ಕೆಲಸ ದೋಸೆ ಮಾಡಿ ಅದನ್ನು ಮಾರಿ ಹಣ ಸಂಪಾದಿಸುವುದು. ಭಗವಂತನು ಆಗ ಒಂದು ಲೀಲೆ ಮಾಡಬೇಕೆಂದುಕೊಂಡನು. ಅವಳಿಗೆ ಗುದ್ದಲಿಯಂಥಹುದನ್ನು ತೆಗೆದುಕೊಂಡು ಮಣ್ಣನ್ನು ಗೊಡೆಯಲ್ಲಿ ಹಾಕುವುದಕ್ಕೂ ಆಗುತ್ತಿರಲಿಲ್ಲ. ದೋಸೆ ಮಾಡಿ ಮಾರುವುದು ಮಾತ್ರ ಅವಳಿಗೆ ಸಾಧ್ಯ. ಭಗವಂತನು ಅವಳ ಬಳಿಗೆ ಬಂದು, `ಏನಮ್ಮ, ನೀನು ಏನೂ ಕೆಲಸ ಮಾಡಲಿಲ್ಲವೇ?' ಎಂದು ಕೇಳಿದನು. ಅದಕ್ಕೆ ಅವಳು, `ನಾನೇನು ಮಾಡುವುದು? ಈ ದೋಸೆಯೆಲ್ಲಾ ಇಂದು ಮಾರದೇ ಇದ್ದರೆ ನಾನು ಜೀವಿಸಲಾಗದ ಕಷ್ಟಪಡಬೇಕು. ಆದರೆ ನಾನು ಹೋಗದೆಯೂ ಇರಲಾಗುವುದಿಲ್ಲ. ಅದು ರಾಜಾಜ್ಞೆ' ಎಂದಳು. ಭಗವಂತನು `ಆ ಕೆಲಸವೆಲ್ಲ ನಾನು ಮಾಡುತ್ತೇನೆ. ನಿನ್ನ ದೋಸೆಗಳೆಲ್ಲ ಒಣಗಿರುವುದನ್ನು ಯಾರೂ ತೆಗೆದುಕೊಂಡು ಹೋಗುವುದಿಲ್ಲ. ಅದನ್ನು ನೀನು ನನಗೆ ಕೊಟ್ಟರೆ ನನ್ನ ಕೂಲಿಗೆ ಆಗುತ್ತೆ' ಎಂದನು. ಹೀಗೆ ಚಮತ್ಕಾರವಾಗಿ ಮಾತನಾಡುತ್ತಾ ಅವನು ಚೆನ್ನಾಗಿರುವ ದೋಸೆಗಳನ್ನು ತಿಂದು ಬಿಟ್ಟನು. ಮಧ್ಯಾಹ್ನವಾಯಿತು. ದೋಸೆಗಳೆಲ್ಲವನ್ನೂ ತಿನ್ನುವುದು ಆಯಿತು. ತಲೆಯ ಮೇಲೆ ಗೊಡೆ, ಕೈಯಲ್ಲಿ ಗುದ್ದಲಿ ಹಿಡಿದುಕೊಂಡಿದ್ದ ಅವನು ಬರುವವರೊಡನೆ ಮಾತನಾಡುತ್ತಿದ್ದನು. ಕೆಲಸವೇನೂ ಮಾಡಲಿಲ್ಲ. ಸ್ವಲ್ಪ ಹೊತ್ತಾದ ಮೇಲೆ ನೋಡಿದರೆ ಒಂದೇ ಒಂದು ಕಡೆಯಿಂದ ಮಾತ್ರ ನೀರು ಬರುತ್ತಿರುವುದನ್ನು ಜನರು ಕಂಡರು. ರಾಜನಿಗೆ ಸಮಾಚಾರ ತಲುಪಿತು. ಎಲ್ಲರೂ ರಾಜನ ಬಳಿ, ಇವನು, ನಾವು ಯಾವುದಾದರೂ ಹೇಳಿದರೆ ಚಮತ್ಕಾರವಾಗಿ ಮಾತನಾಡಿ ಬಿಡುತ್ತಾನೆ. ಇವನ ಮುಖವನ್ನು ನೋಡಿದರೆ ನಮಗೆಲ್ಲ ಮೋಹ ಉಂಟಾಗುತ್ತದೆ. ಇವನನ್ನು ಹೊಡೆಯಬೇಕೆನಿಸುತ್ತದೆ. ಆದರೆ ಕೈ ಬರುವುದಿಲ್ಲ. ತಾವೇ ಶಿಕ್ಷಿಸಬೇಕು? ಎಂದರು. ಇವನನ್ನು ನೋಡಿದ ರಾಜನು ಒಂದು ಏಟು ಹೊಡೆದನು. ಆದರೆ ಆ ಏಟು ಹೊಡೆದೊಡನೆ ಎಲ್ಲರಿಗೂ ಏಟು ಬಿದ್ದಿತು. ಎಲ್ಲರೂ ರಾಜನು ತಮ್ಮನ್ನೇ ಹೊಡೆದುಬಿಟ್ಟನೇನೋ ಎಂದು ಮುಟ್ಟಿ ಮುಟ್ಟಿ ನೋಡಿಕೊಂಡರು. ನೋಡಿದರೆ ರಾಜನಿಗೂ ಏಟು ಬಿದ್ದಿತು. ಇತರರ ಬಳಿ ರಾಜನು ಇಲ್ಲದಿದ್ದರೂ ಅವನು ಭಗವಂತನಿಗೆ ಹೊಡೆದ ಏಟು ಮಾತ್ರ ಎಲ್ಲರಿಗೂ ಬಿದ್ದು ಬಿಟ್ಟಿತು.

ಇದನ್ನು ಕುರಿತು ಚಮತ್ಕಾರವಾಗಿ, `ಭಗವಂತನಾದ ನೀನು ಎಲ್ಲರ ಆತ್ಮನಾಗಿದ್ದೀಯೆ, ನೀನು ಸರ್ವಾತ್ಮಕನೆಂದು ತೋರಿಸುವುದೂ ಬಹಳ ಸ್ವಾರಸ್ಯಕರವಾಗಿ ತೋರಿಸಿದೆ. ದೋಸೆ ತಿನ್ನುವ ವೇಳೆಯಲ್ಲಿ ಸರ್ವಾತ್ಮಕತ್ವವನ್ನು ತೋರಿಸಬೇಕೆನಿಸಲಿಲ್ಲವೇ! ನೀನು ದೋಸೆ ತಿಂದು ಹಸಿವು ತೀರಿದೊಡನೆ ಎಲ್ಲರಿಗೂ ಹಸಿವು ತೀರಿದಂತೆ ತೋರಿಸಲಿಲ್ಲವಲ್ಲಾ! ಈಗ ಏಟು ಕೊಟ್ಟ ಸಮಯದಲ್ಲಿ ಮಾತ್ರ ಎಲ್ಲರಿಗೂ ಸರ್ವಾತ್ಮಕತ್ವವನ್ನು ತೋರಿಸುವೆಯಾ?' ಎನ್ನುತ್ತಾರೆ. `ಏಕೆಂದರೆ ಏಟು ಬಿದ್ದರೇನೇ ಭಗವಂತನ ನೆನಪು ಬರುತ್ತದೆ. ಹಸಿವು ಉಂಟಾಗಲಿಲ್ಲವೆಂದರೆ ಯಾರೂ ಹಸಿವಾಗಲಿಲ್ಲವೆಂದು ಚಿಂತಿಸುವುದಿಲ್ಲ. ಆದರೆ ಏಟು ಬಿದ್ದರೆ, `ಅಯ್ಯೋ! ಏನು ಪ್ರಾರಬ್ಧ, ಏಟು ಬಿದ್ದು ಬಿಟ್ಟಿತು.!' ಎಂದು ನಿನ್ನನ್ನು ಎಲ್ಲರೂ ಸ್ಮರಿಸುವುದಕ್ಕಾಗಿ ನಿನ್ನ ಸರ್ವಾತ್ಮಕತ್ವವನ್ನು ಈ ರೀತಿ ನೀನು ತೋರಿಸಿದೆ!' ಎಂದು ಪರಮ ಭಕ್ತರಾದ ನೀಲಕಂಠದೀಕ್ಷಿತರು ಹೇಳಿದರು. ಆದ್ದರಿಂದ ಭಗವಂತನು ಸರ್ವತ್ರ ಎಲ್ಲರನ್ನೂ ನೋಡುತ್ತಿದ್ದಾನೆ. ಆ ಮುದುಕಿಗೆ ಅವನು ಒಬ್ಬ ಕೂಲಿಯವನಾಗಿ ದರ್ಶನವನ್ನು ಕೊಟ್ಟನು. ಇವೆಲ್ಲಾ ಪುರಾಣಗಳಲ್ಲಿ ಇರುವವೇ ಆಗಿವೆ.

ನಾನು ಮೊದಲು ಹೇಳಿದ ಶ್ಲೋಕದಲ್ಲಿ ಮೊದಲು `ವ್ಯಾಧಸ್ಯಾಚರಣಂ' ಎಂದಿದೆ. ಧರ್ಮವ್ಯಾಧನೆಂಬುವನು ಒಬ್ಬನಿದ್ದನು. ಅವನು ಬೇಡರ ಕುಲದಲ್ಲಿ ಹುಟ್ಟಿದವನಾದರೂ ಅವನಿಗೆ ಭಗವತ್ಸಾಕ್ಷಾತ್ಕಾರವಾಯಿತು. ಆದ್ದರಿಂದಲೇ ಅವನಿಗೆ ಧರ್ಮವ್ಯಾಧನೆಂದು ಹೆಸರೇ ಬಂದಿತು. ಧರ್ಮವ್ಯಾಧನಿಗೆ ಆಚಾರ ಒಂದೂ ವಿಶೇಷವಾಗಿ ಇರಲಿಲ್ಲ. ಆದರೂ ಅವನಿಗೆ ಭಗವಂತನ ದಯೆ ಲಭಿಸಿತು. ಇದರಿಂದ ವಿಶೇಷವಾದ ಆಚಾರವಿದ್ದರೇ ಮಾತ್ರ ಭಗವಂತನಲ್ಲಿ ಭಕ್ತಿ ಇರಬೇಕೆಂದಿಲ್ಲವೆಂದು ತಿಳಿಯುವುದು.

ಅನಂತರ `ಧ್ರುವಸ್ಯ ಚ ವಯಃ'-ಧ್ರುವನು ಒಬ್ಬ ಚಿಕ್ಕ ಹುಡುಗ. ಅವನಿಗೆ ಎರಡು ವರ್ಷಗಳು ಕೂಡ ಆಗಿರಲಿಲ್ಲ. ಅಂಥಹವನು ಯಾವುದೋ ಒಂದು ಸಿಂಹಾಸನ ಬೇಕೆಂದು ತಪಸ್ಸು ಮಾಡಿದನು. ಅವನಿಗೆ ಭಗವಂತನು ಮೇಲಾದ ಶಾಶ್ವತವಾದ ಉತ್ತಮ ಸ್ಥಾನವನ್ನು ಕೊಟ್ಟನು.

`ವಿದ್ಯಾ ಗಜೇಂದ್ರಸ್ಯಕಾ'-ಗಜೇಂದ್ರನಿಗೆ ಭಗವಂತನು ವೈಕುಂಠ ಸಾಮ್ರಾಜ್ಯವನ್ನು ಕೊಟ್ಟನು. ಬುದ್ಧಿಯಲ್ಲಿ ಅವನು ಕೆಳಮಟ್ಟದ ಬುದ್ಧಿವುಳ್ಳ ಮೊಸಳೆಯೊಡನೆ ತನ್ನ ಬಲವನ್ನೆಲ್ಲಾ ಪ್ರಯೋಗಿಸಿದರೂ ಉಪಯೋಗವಾಗಲಿಲ್ಲ. ಒಡನೆ ಒಂದು ಕಮಲವನ್ನು ತೆಗೆದುಕೊಂಡು ಭಗವಂತನಿಗೆ ಸಮರ್ಪಿಸಿ, `ನೀನೇ ನನ್ನನ್ನು ಕಾಪಾಡಬೇಕು' ಎಂದನು. ಒಡನೆ ಭಗವಂತನು ವೈಕುಂಠವೆಲ್ಲಿದೆ, ಲಕ್ಷ್ಮಿ ಎಲ್ಲಿದ್ದಾಳೆ ಎನ್ನುವುದನ್ನು ಒಂದನ್ನೂ ಗಮನಿಸದೆ ಒಡನೆ ಬಂದು ಕಾಪಾಡಿದನು.

`ಕಾ ಜಾತಿರ್ವಿದುರಸ್ಯ'-ವಿದುರನು ಒಮ್ಮೆ ಭಗವಂತನನ್ನು ತನ್ನ ಮನೆಗೆ ಸ್ವಾಗತಿಸಿದನು. ಭಗವಂತನು ದುರ್ಯೋಧನನ ಮನೆಗೆ ಹೋಗುವ ದಾರಿಯಲ್ಲಿ ವಿದುರನ ಮನೆಗೆ ಹೋಗಿ ಅಲ್ಲಿ ಉಪಚಾರವನ್ನು ಸ್ವೀಕರಿಸಿದನು. ಇತರರೆಲ್ಲರನ್ನೂ ಬಿಟ್ಟು ಭಗವಂತನು ವಿದುರನ ಮನೆಗೆ ಹೋದ ಕಾರಣವೇನೆಂದರೆ ವಿದುರನ ವಿಷಯವಾಗಿ ಭಗವಂತನಿಗೆ ಪರಮ ಪ್ರೇಮವಿದ್ದಿತು.

`ಯಾದವಪತೇರುಗ್ರಸ್ಯ ಕಿಂ ಪೌರುಷಮ್'-ಉಗ್ರಸೇನನು ಯಾದವರ ಅರಸನಾಗಿದ್ದನು. ಆದರೆ ಅವನ ಮಗನಾದ ಕಂಸನು ಅವನನ್ನು ಬಂಧಿಸಿದನು. ಹೀಗಿದ್ದಾಗ ಆ ಉಗ್ರಸೇನನಿಗೆ ಎಷ್ಟು ಶೂರತ್ವವಿರಬೇಕೆಂದು ನಾವೇ ಪರಿಗಣಿಸಬಹುದು. ಹಾಗಿದ್ದರೂ ಭಗವಂತನು ಅವನನ್ನು ಅನುಗ್ರಹಿಸಿದನು.

`ಕುಬ್ಜಾಯಾಃ ಕಮನೀಯ ರೂಪಮಧಿಕಮ್'-ಕುಬ್ಜೆ ಎನ್ನುವವಳಿಗೆ ಸೌಂದರ್ಯವೇ ಇರಲಿಲ್ಲ. ಆದರೂ ಅವಳ ಸೇವೆಯನ್ನು ಕಂಡು ತೃಪ್ತಿಪಟ್ಟ ಕೃಷ್ಣಪರಮಾತ್ಮನು ಅವಳನ್ನು ಅನುಗ್ರಹಿಸಿ ಅವಳ ಶರೀರ ದೋಷವನ್ನು ದೂರ ಮಾಡಿದನು.

`ಕಿಂ ವಾ ಸುದಾಮ್ನೋ ಧನಂ'-ಗುರುಕುಲದಲ್ಲಿ ಕೃಷ್ಣನ ಸ್ನೇಹಿತನಾಗಿದ್ದವನು ಸುದಾಮ. ತರುವಾಯ ಪರಮ ದರಿದ್ರನಾದ ಅವನು ತನ್ನ ಹೆಂಡತಿಯ ಆಸೆಗಳಿಗೆ ಒಳಪಟ್ಟು ಕೃಷ್ಣನ ಹತ್ತಿರ ಹಣ ಪಡೆಯಲು ಹೋದನು. ಕೃಷ್ಣನು ಪ್ರೇಮದಿಂದ ಅವನು ಕೊಟ್ಟ ಅವಲಕ್ಕಿಯನ್ನು ಸ್ವೀಕರಿಸಿ ಅಮಿತ ಧನವನ್ನು ಆಶೀರ್ವದಿಸಿದನು. ಇದರಿಂದ ಹಣವಂತನೇ ಭಕ್ತ ಮಾರ್ಗದಲ್ಲಿ ಹೋಗಬೇಕೆಂದಿಲ್ಲವೆಂದು ತಿಳಿಯುತ್ತದೆ. ಹಾಗಾದರೆ ಭಗವಂತನಿಗೆ ಮುಖ್ಯವಾಗಿ ಬೇಕಾದುದು ಏನು?

`ಭಕ್ತ್ಯಾತುಷ್ಯತಿ ಕೇವಲಂ ನ ತು ಗುಣೈಃಭಕ್ತಿಪ್ರಿಯೋ ಮಾಧವಃ'-ಭಕ್ತನ ಅಂತಃಕರಣ ಶುದ್ಧವಾಗಿದೆಯೇ ಇಲ್ಲವೇ ಎನ್ನುವುದನ್ನೇ ಅವನು ನೋಡುತ್ತಾನೆ. ಭಕ್ತನ ಅಂತಃಕರಣ ಶುದ್ಧವಾಗಿದ್ದರೆ ಅವನ ಜಾತಿಯಾಗಲಿ, ಕುಲವಾಗಲಿ, ಐಶ್ವರ್ಯವಾಗಲಿ ಯಾವುದನ್ನೂ ನೋಡುವುದಿಲ್ಲ. ಆದ್ದರಿಂದ ಎಲ್ಲರೂ ಸುಲಭವಾಗಿ ಆರಾಧಿಸಿ ಮನುಷ್ಯ ಜನ್ಮವನ್ನು ಸಾರ್ಥಕ ಮಾಡಿಕೊಳ್ಳಲು ಮುಖ್ಯವಾಗಿರುವಂಥಹದು ಭಕ್ತಿ ಮಾರ್ಗವೇ ಎಂದು ಹಿಂದಿನವರು ಹೇಳಿದ್ದಾರೆ. ಆದರೆ ಈ ಕಾಲದಲ್ಲಿ ಕೆಲವರಿಗೆ ಒಂದು ದೃಢವಾದ ಮನೋಭಾವನೆ ಇದೆ. ಅದು ಏನೆಂದರೆ-

\begin{shloka}
`ಅಪಿ ಚೇತ್ ಸುದುರಾಚಾರಃ ಭಜತೇ ಮಾಮನನ್ಯಭಾಕ್|\\
ಸಾಧುರೇವ ಸ ಮಂತವ್ಯಃ ಸಮ್ಯಕ್ ವ್ಯವಸಿತೋ ಹಿ ಸಃ||'
\end{shloka}

ಎಂದಲ್ಲವೇ ಭಗವಂತನು ಹೇಳಿದ್ದಾನೆ.

ಕೆಟ್ಟ ಆಚಾರವಂತನಾದರೂ ನನ್ನನ್ನು ತೀವ್ರವಾಗಿ ಭಕ್ತಿಯಿಂದ ಆರಾಧಿಸಿದರೆ ಅವನನ್ನು ಸಾಧುವೆಂದು ತಿಳಿಯಬೇಕು. ಏಕೆಂದರೆ ನನ್ನನ್ನು ಆರಾಧಿಸಬೇಕೆಂಬ ಒಳ್ಳೆಯ ನಿರ್ಧಾರಕ್ಕೆ ಬಂದಿದ್ದಾನೆ'-ಇದು ಶ್ಲೋಕದ ತಾತ್ಪರ್ಯ. ಇದನ್ನು ಇಟ್ಟುಕೊಂಡು `ದುರಾಚಾರ'ವಂತನಾದರೂ ಕೂಡ ಭಗವದ್ಭಕ್ತನೆಂದು ಬೋರ್ಡು ಹಾಕಿಕೊಂಡು ಬಿಟ್ಟರೆ ಅವನು ಯಾವಾಗಲೂ ಯೋಗ್ಯನೆಂದೇ ಕೆಲವರು ನಿರ್ಧಾರಕ್ಕೆ ಬಂದು ಬಿಟ್ಟಿದ್ದಾರೆ. ಇದು ಸರಿಯಾದ ತೀರ್ಮಾನವಲ್ಲ.

ಈ ವಿಷಯದ ಬಗ್ಗೆ ಸುರೇಶ್ವರರು-

\begin{shloka}
`ಬುದ್ಧಾದ್ವೈತ ಸತ್ತತ್ತ್ವಸ್ಯ ಯಥೇಷ್ಟಾಚರಣಂ ಯದಿ|\\
ಶುನಾಂ ತತ್ತ್ವದೃಶಾಂ ಚೈವ ಕೋ ಭೇದೋಽಶುಚಿಭಕ್ಷಣೇ||'
\end{shloka}

ಎಂದು ಕೇಳಿದ್ದಾರೆ.

ಅದ್ವೈತ ತತ್ತ್ವವನ್ನು ತಿಳಿದವನು ಹೇಗೆ ಬೇಕಾದರೂ ಜೀವನವನ್ನು ನಡೆಸಿದರೆ ಅವನಿಗೂ ನಾಯಿಗೂ ಅಶುದ್ಧವಾದುದನ್ನು ತೆಗೆದುಕೊಳ್ಳುವುದರಲ್ಲಿ ವ್ಯತ್ಯಾಸವೇನೆಂದು ಕೇಳುತ್ತಾರೆ. ನಾಯಿಗೆ ಲೋಕ ಜ್ಞಾನವಿರುವುದಿಲ್ಲ. ಈ ಜ್ಞಾನಿಗೆ ಲೋಕ ಬಿಟ್ಟು ಹೋಗಿದೆ. ಆದರೆ ಜ್ಞಾನ ಉಂಟಾಗುವುದಕ್ಕೆ ಮುಂಚೆ ಬಹಳ ಕಾಲ ಉತ್ತಮವಾದ ದಾರಿಯಲ್ಲಿ ಜ್ಞಾನಿ ಹೋಗುತ್ತಿದ್ದುದರಿಂದ ಯಾವಾಗಲೂ ಅವನು ಶಾಸ್ತ್ರಮಾರ್ಗವನ್ನು ಬಿಟ್ಟು ಹೋಗುವುದಿಲ್ಲ; ಹಾಗೆ ಜೀವನವನ್ನು ನಡೆಸುವುದರಿಂದ ಯಾವ ವಿಧವಾದ ಪ್ರಯೋಜನವೂ ಅವನಿಗೆ ಇಲ್ಲದಿದ್ದರೂ ಲೋಕದ ಕಲ್ಯಾಣಕ್ಕಾಗಿ ಹಾಗೆ ನಡೆದುಕೊಂಡು ಬರುತ್ತಾನೆ. ಈ ವಿಷಯವನ್ನು ಭಗವಂತನೂ ಗೀತೆಯಲ್ಲಿ ಸ್ಪಷ್ಟಪಡಿಸಿದ್ದಾನೆ. ಕೆಟ್ಟ ಆಚಾರವುಳ್ಳ ಭಕ್ತನನ್ನು ಸಾಧುವೆಂದು ತಿಳಿಯಬೇಕೆನ್ನುವ ಭಗವಂತನ ಅಭಿಪ್ರಾಯವೇನೆಂಬುದನ್ನು ಗಮನಿಸಬೇಕು. ಅವನು `ಅಪಿ ಚೇತ್ ಸುದುರಾಚಾರಃ' ಎಂದು ಮಾತ್ರ ಹೇಳಿದ್ದಾನೆ. ಇದರಿಂದ `ಕೆಟ್ಟ ಆಚಾರ ಒಂದು ವೇಳೆ ಇದ್ದರೂ' ಎಂದು ಮಾತ್ರ ತಿಳಿಯಬೇಕು. ಇದನ್ನು ಬಿಟ್ಟು ಭಗವಂತನು ದುರಾಚಾರಕ್ಕೆ ಅನುಮತಿ ನೀಡಿದ್ದಾನೆಂದು ತಿಳಿದರೆ ಒಂದು ವಿಧವಾದ ಒಳ್ಳೆಯ ಫಲವನ್ನು ಪಡೆಯಲಾಗುವುದಿಲ್ಲ ಏಕೆಂದರೆ ಭಗವಂತನ ನಾಮಸ್ಮರಣೆ ಮಾಡುವಾಗ ಹತ್ತು ವಿಧವಾದ ಅಪರಾಧಗಳಾಗಿರಬಹುದೆಂದು ಒಂದು ಕಡೆ ಹೇಳಿದ್ದಾರೆ (ಒಳ್ಳೆಯವರ ಬಗ್ಗೆ ತಪ್ಪನ್ನು ಹೇಳುವುದು, ದುಷ್ಟರ ಬಳಿ ನಾಮ ಮಹಿಮೆಯನ್ನು ಕುರಿತು ಹೇಳುವುದು, ಶಿವನೂ ವಿಷ್ಣುವೂ ಬೇರೆಯೆಂದು ತಿಳಿಯುವುದು, ವೇದ-ಶಾಸ್ತ್ರ-ಗುರುವಿನ ಮಾತು ಮುಂತಾದವುಗಳಲ್ಲಿ ನಂಬಿಕೆ ಇಲ್ಲದೆ ಇರುವುದು, ಭಗವಂತನ ಹೆಸರುಗಳು ನಿಜವಾಗಿಯೂ ಶಕ್ತಿ ಇರುವುದಾಗಿ ಇಲ್ಲದಿದ್ದರೂ ಹಾಗೆ ಕಲ್ಪಿಸಲ್ಪಟ್ಟವೆಂದು ಭ್ರಮೆಯಿಂದಿರುವುದು, ನಾಮವಿದೆಯೆನ್ನುವ ಧೈರ್ಯದಿಂದ ಮಾಡಬಾರದ ಕೆಲಸಗಳನ್ನು ಮಾಡುವುದು, ವಿಹಿತ ಕರ್ಮಗಳನ್ನು ಮಾಡದೆ ಇರುವುದು, ಇತರ ಧರ್ಮಗಳೊಡನೆ ನಾಮವನ್ನು ಸಮವಾಗಿ ಪರಿಗಣಿಸುವುದು - ಇವು ಹತ್ತು ವಿಧವಾದ ಅಪರಾಧಗಳು). ಆ ಅಪರಾಧಗಳು ಯಾವುವೂ ಇಲ್ಲದಿದ್ದರೆ ಮಾತ್ರ ನಾಮವೆನ್ನುವುದು ಫಲಿಸುತ್ತದೆ. ಬೆಂಕಿ ತಿಳಿದು ಮುಟ್ಟಿದರೂ ಸುಡುತ್ತದೆ, ತಿಳಿಯದೆ ಮುಟ್ಟಿದರೂ ಸುಡುತ್ತದೆ. ಅದೇ ರೀತಿ, ನಾಮವೆನ್ನುವುದು ಪಾಪಕ್ಷಯ ಮಾಡುತ್ತದೆಂದು ತಿಳಿದು ಜಪಿಸಿದರೂ ಹಾಗಾಗುತ್ತದೆ, ತಿಳಿಯದೆ ಜಪಿಸಿದರೂ ಹಾಗಾಗುತ್ತದೆ. ಹೀಗಿದ್ದರೂ ಪಾಪಕ್ಷಯ ಮಾಡುವ ಶಕ್ತಿ ನಾಮಕ್ಕಿದೆಯೆಂದುಕೊಂಡು ಪಾಪವನ್ನು ಮಾಡುವುದೇ ಕೆಲಸವಾಗಿಟ್ಟುಕೊಂಡರೆ ಆಗ ಆ ನಾಮ ಅವನನ್ನು ಬಿಟ್ಟು ಬಿಡುತ್ತದೆ. ಏಕೆಂದರೆ, `ಇವನಿಂದ ಪ್ರಪಂಚವೇ ಹಾಳಾಗುತ್ತದೆ. ಪ್ರಪಂಚದ ಕ್ಷೇಮಕ್ಕಾಗಿ ನಾವು ಈ ನಾಮಕ್ಕೆ ಪ್ರಭಾವವನ್ನು ನೀಡಿದರೆ ಪ್ರಪಂಚವನ್ನು ಕೆಡಿಸಲು ನೀನು ಈ ನಾಮವನ್ನು ಉಪಯೋಗಪಡಿಸುತ್ತಾ ಬಂದೆ-ಎಂದರೆ ನಿನಗೆ ಆ ನಾಮದಿಂದ ಉಂಟಾಗುವ ಪ್ರಭಾವವೇ ಇಲ್ಲ'-ಎಂದು ಭಗವಂತನೇ ಶಾಪವನ್ನು ಕೊಟ್ಟು ಬಿಡುತ್ತಾನೆ. ಆದ್ದರಿಂದ ನಾವು ಭಗವಂತನ ನಾಮಸ್ಮರಣೆ, ಈಶ್ವರಾರಾಧನೆ ಮುಂತಾದವುಗಳನ್ನು ಮಾಡುತ್ತಾ ಇದ್ದು ಶಾಸ್ತ್ರವು ಹೇಳುವ ದಾರಿಯಲ್ಲಿ ನಡೆದುಕೊಂಡು ಬಂದರೆ ನಮ್ಮ ಬಾಳಿನಲ್ಲಿ ಯಾವ ವಿಧವಾದ ಕೊರತೆ ಇರುವುದಿಲ್ಲ. ಈ ಭಕ್ತಿಗೆ ಯಾರು ಅಧಿಕಾರಿ, ಯಾರು ಅಧಿಕಾರಿಯಲ್ಲ ಎಂದರೆ ಎಲ್ಲಾರೂ ಅಧಿಕಾರಿಗಳೇ. ಅಧಿಕಾರಿಗಳಾದ ಮೇಲೆ ಭಗವಂತನು ಹೇಳಿದ ರೀತಿಯಲ್ಲಿ ನಡೆಯುತ್ತಾ ಪರಂಪರೆಯಾಗಿ ನಮಗೆ ತಿಳಿದುಬಂದ ದಾರಿಯಲ್ಲಿ ಹೋಗಿ ಭಗವಂತನ ಕಟಾಕ್ಷವನ್ನು ಸಂಪಾದಿಸಬೇಕು.

\newpage

\section{ಶ್ರೀ ಶಂಕರ ಭಗವತ್ಪಾದಾಚಾರ್ಯರು}

ಅನಾದಿಯಿಂದ ಮಾನವನ ಆಧ್ಯಾತ್ಮಿಕ, ಭೌತಿಕ ಪ್ರಯೋಜನಕ್ಕಾಗಿ ಈಶ್ವರ ನಿಃಶ್ವಾಸರೂಪವಾದ ವೇದಗಳಿವೆ. ವೇದಗಳ ಆಧಾರದ ಮೇಲೆ ಸ್ಮೃತಿಗಳು ಬೆಳಕಿಗೆ ಬಂದವು. ಅನಾದಿಯಾಗಿ ಎಂದರೆ ಮಾನವನ ಅವತರಣ ಕಾಲದಿಂದಲೂ ಎಂದು ಭಾವ. ಶಾಸ್ತ್ರಗಳು, ಸಂಪ್ರದಾಯ ನಮಗೆ ತಿಳಿಸುವುದರ ಪ್ರಕಾರ ನಮ್ಮ ಅಭ್ಯುನ್ನತಿಗಾಗಿ ಈಶ್ವರನು ನಮಗೆ ವೇದಗಳನ್ನು ನೀಡಿದನು. ವೇದಗಳು ನಾವು ಅನುಷ್ಠಾನ ಮಾಡಬೇಕಾದ ಕರ್ಮಗಳನ್ನು ತಿಳಿಸುತ್ತವೆ. ಜಗತ್ತಿನ ಸ್ಥಿತಿಯನ್ನು ನಿಯಮಿಸುವ ಶಾಸನಗಳನ್ನು ಉಪಾಸಿಸಲು ಅವು ಹೇಳುತ್ತವೆ. ಪ್ರಪಂಚದ ಸೃಷ್ಟಿಯನ್ನು ಮಾಡಿದವನ ಬಗ್ಗೆ ಯಥಾರ್ಥ ಜ್ಞಾನವನ್ನು ಅವು ತಿಳಿಸುತ್ತವೆ. ಭಿನ್ನ ಭಿನ್ನ ಸ್ಥಳಗಳಲ್ಲಿ ಈ ಕರ್ಮ, ಉಪಾಸನೆ ಮತ್ತು ಜ್ಞಾನದ ಪ್ರಾಮುಖ್ಯ ಹೇಳಲಾಗಿದೆ. ಒಂದರ ಪ್ರಾಮುಖ್ಯವನ್ನು ಕುರಿತು ಹೇಳುವಾಗ ಉಳಿದ ಎರಡರ ಪ್ರಾಮುಖ್ಯವನ್ನು ಹೇಳುವ ಹಾಗಿಲ್ಲ.

ಭೌತಿಕ ಸುಖಾದಿಗಳು, ಐಶ್ವರ್ಯಾದಿಗಳು ಬೇಕೆನ್ನುವವನು ಕರ್ಮಗಳನ್ನೂ ಉಪಾಸನೆಯನ್ನೂ ಅವಲಂಬಿಸಬೇಕು. ಆದರೆ ಜ್ಞಾನ ಸರ್ವೋತ್ತಮವಾದುದು. ಕರ್ಮ-ಉಪಾಸನೆಗಳನ್ನು ಸೋಪಾನವಾಗಿ ಉಪಯೋಗಿಸುವುದರಿಂದಲೇ ಜ್ಞಾನವನ್ನು ಪಡೆಯಲು ಸಾಧ್ಯವಾಗುವುದು. ಕರ್ಮೋಪಾಸನೆಯಲ್ಲಿಯೇ ನಿಲ್ಲುವವನಿಗೆ ಜ್ಞಾನ ಲಭಿಸುವುದಿಲ್ಲ. ಅನೇಕ ಶತಾಬ್ದಿಗಳವರೆಗೆ ಕರ್ಮೋಪಾಸನೆಗಳಿಗೆ ಪ್ರಾಮುಖ್ಯ ಕೊಡಲಾಗಿತ್ತು. ಜ್ಞಾನಹೇತುವಾದ ಶ್ರವಣ, ಮನನ, ನಿದಿಧ್ಯಾಸನಗಳನ್ನು ಮರೆತಂತಾಯಿತು. ಜನರು ಯಾಂತ್ರಿಕವಾದ ಬಾಳಿಗೆ ಅಭ್ಯಸ್ತರಾದರು. ತಮ್ಮ ತಾರ್ಕಿಕ ಲಕ್ಷ್ಯದಿಂದ ಶ್ರೇಯಸ್ಸನ್ನು ಸಾಧಿಸಬಹುದೆಂದು ಭಾವಿಸಿದರು. ಅದರಿಂದಾಗಿ ಮಾನವ ನಿರ್ಮಿತವಾದ ವೇದದೂರವಾದ ಮತಗಳು ಪ್ರಾದುರ್ಭವಿಸಿದುವು. ಇದಕ್ಕೆ ಅನೇಕ ಕಾರಣಗಳಿವೆ.

`ಭಕ್ಷ್ಯಾದ್ಯನಿಯಮ ಇತಿ ರಾಗಿಣಿಃ ಸ್ವೇಚ್ಛಯಾ ದಾರಪರಿಗ್ರಹ ಇತಿ ಕುತರ್ಕಾಭ್ಯಾಸಿನಃ ಕರ್ಮಲಾಘವಮಿತ್ಯಲಸಾಃ, ಇತಃ ಪತಿತಾನಾಮಪ್ಯಸ್ತ್ಯನುಪ್ರವೇಶ ಇತಿ ಅನನ್ಯ ಗತಿಕಾಃ|'
