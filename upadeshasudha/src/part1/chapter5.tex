\chapter{ಕರ್ಮ}\label{chap5}

\begin{shloka}
ಗಂಗಾಪೂರಪ್ರಚಲಿತ ಜಟಾಸ್ರಸ್ತ ಭೋಗಿಂದ್ರ ಭೀತಾಂ\\
ಆಲಿಂಗಂತೀಮಚಲತನಯಾಂ ಸಸ್ಮಿತಂ ವೀಕ್ಷಮಾಣಃ|\\
ಲೀಲಾಪಾಂಗೈಃ ಪ್ರಣತಜನತಾಂ ನಂದಯಂಶ್ಚಂದ್ರಮೌಲಿಃ\\
ಮೋಹಧ್ವಾಂತಂ ಹರತು ಪರಮಾನಂದಮೂರ್ತಿಃ ಶಿವೋ ನಃ||
\end{shloka}

ಪರಮಾತ್ಮನಿದ್ದಾನೆ, ಅವನೇ ನಮಗೆ ಸುಖ-ದುಃಖಗಳನ್ನು ಕೊಡುತ್ತಾನೆ ಎಂದು ಒಪ್ಪುತ್ತೇವೆ. ಅವನು ಹೇಗೆ ಸುಖ ದುಃಖಗಳನ್ನು ಕೊಡುತ್ತಾನೆಂದು ಸ್ವಲ್ಪ ಯೋಚಿಸೋಣ. ಭಗವಂತನಿಗೆ ಯಾರ ವಿಷಯದಲ್ಲಿ ಪ್ರೀತಿ ಇದೆಯೋ ಅವರಿಗೆ ಸುಖವನ್ನೂ, ಯಾರ ವಿಷಯದಲ್ಲಿ ಪ್ರೀತಿ ಇಲ್ಲವೋ ಅಂಥಹವರಿಗೆ ದುಃಖವನ್ನೂ ಕೊಡುತ್ತಾನೆ. ಆದರೆ, ಭಗವಂತನಿಗೆ ಯಾವುದರಿಂದ ತಮ್ಮ ವಿಷಯದಲ್ಲಿ ಪ್ರೀತಿಯೋ ಅಪ್ರೇಮವೋ ಉಂಟಾಯಿತು ಎನ್ನುವುದನ್ನು ಮಾತ್ರ ಅವರು ಯೋಚನೆ ಮಾಡುವುದಿಲ್ಲ. ಇದಕ್ಕೆ ಸಂಬಂಧಪಟ್ಟಂತೆ ನೀಲಕಂಠ ದೀಕ್ಷಿತರು ಒಂದು ಕಡೆ-

\begin{shloka}
ಸಾಧ್ಯಾ ಶಂಭೋಃ ಕಥಮಪಿ ದಯೇತ್ಯಪ್ಯಸಾಧ್ಯೊಪದೇಶಃ\\
ಕೋಪಂ ತಸ್ಯ ಪ್ರಥಮಮಪನುತ್ಯೈವ ಸಾಧ್ಯಃ ಪ್ರಸಾದಃ|\\
ಕೋಪೋ ವರ್ಣಾಶ್ರಮ ನಿಯಮಿತಾಚಾರ ನಿರ್ಲಂಘನೋತ್ಥಃ\\
ಶಾಂತಿಂ ನೇಯಃ ಸ ಕಥಮಧುನಾಪ್ಯವ್ಯವಸ್ಥಾ ಪ್ರವೃತ್ತೈಃ||
\end{shloka}

(ಪರಮೇಶ್ವರನ ದಯೆಯನ್ನು ಹೇಗಾದರೂ ಪಡೆಯಬೇಕೆನ್ನುವುದು ಸಾಧ್ಯವಾದ ಉಪದೇಶ. ಪರಮೇಶ್ವರನಿಗೆ ನಮ್ಮ ಮೇಲಿರುವ ಕೋಪ ದೂರವಾದ ಮೇಲೆ ತಾನೆ ಅವನ ಅನುಗ್ರಹ ನಾವು ಪಡೆಯಲು ಸಾಧ್ಯ. ಆ ಕೋಪ, ವರ್ಣ ಮತ್ತು ಆಶ್ರಮಕ್ಕೆ ಸಂಬಂಧಪಟ್ಟ ಆಚಾರ-ನಿಯಮಗಳನ್ನು ಉಲ್ಲಂಘಿಸಿದ್ದರಿಂದ ಉಂಟಾದುದು. ಆ ಕೋಪವನ್ನು ಹೇಗೆ ಶಾಂತಪಡಿಸುವುದು?) ಎಂದು ನುಡಿದರು. ಭಗವಂತನಿಗೆ, ತಮ್ಮ ವಿಷಯದಲ್ಲಿ ಕೋಪವಿರುವುದಾದರೆ ಅವನ್ನು ದೂರಮಾಡಬೇಕೆಂದೇ ಎಲ್ಲರೂ ಭಾವಿಸುತ್ತಾರೆ. ಕೋಪವನ್ನು ದೂರ ಮಾಡಬೇಕಾದರೆ ಮೊದಲು, ಕೋಪ ಏತಕ್ಕೆ ಬಂದಿದೆಯೆಂದು ನೋಡಬೇಕು. ಭಗವಂತನು ಒಂದು ನಿಮಿಷ ಕೋಪದಿಂದಲೂ, ಮತ್ತೊಂದು ನಿಮಿಷ ಬೇರೆ ವಿಧವಾಗಿಯೂ ಇರುವ ಹುಚ್ಚನಂತೆ ಅಲ್ಲ. ಅವನು ಸರ್ವಜ್ಞ. ಹಾಗಿರುವವನ ಅನುಗ್ರಹ ನಮಗೆ ದೊರೆಯಬೇಕಾದರೆ ನಾವು ಆತನ ಕೋಪವನ್ನು ದೂರಮಾಡಬೇಕು.

\begin{shloka}
ಸಾಧ್ಯಾಶಂಭೋಃ ಕಥಮಪಿ ದಯೇತ್ಯಪ್ಯಸಾಧ್ಯೋಪದೇಶಃ
\end{shloka}

ಭಗವಂತನ ಅನುಗ್ರಹವನ್ನು ನಾವು ಪಡೆಯಬಹುದೆನ್ನುವುದು ಅಷ್ಟು ಸುಲಭವಾದ ಕೆಲಸವಲ್ಲ.

\begin{shloka}
ಕೋಪಂ ತಸ್ಯ ಪ್ರಥಮಮಪನುತ್ಯೈವ ಸಾಧ್ಯಃ ಪ್ರಸಾದಃ
\end{shloka}

ಭಗವಂತನಿಗೆ ಇರುವ ಕೋಪವನ್ನು ದೂರಮಾಡಿದರೆ ಅವನ ಪ್ರಸಾದವಾದ ದಯೆ ದೊರೆಯುವುದು. ಏತಕ್ಕೆ ಅವನಿಗೆ ಕೋಪ ಬಂದಿದೆ?

\begin{shloka}
ಕೋಪೋವರ್ಣಾಶ್ರಮ ನಿಯಮಿತಾಚಾರ ನಿರ್ಲಂಘನೋತ್ಥಃ
\end{shloka}

ಪ್ರಪಂಚ ಸರಿಯಾಗಿ ನಡೆಯಬೇಕೆನ್ನುವುದಕ್ಕಾಗಿ ಲೌಕಿಕ ನಿಯಮಗಳು ಇರುವಂತೆ ವೇದಗಳೂ ನಮಗೆ ದಾರಿ ತೋರಿಸುವ ನಿಯಮಗಳೇ ಆಗಿವೆ. ಆಯಾ ಆಶ್ರಮದಲ್ಲಿರುವವರೂ, ತಮ್ಮ ತಮ್ಮ ಸಂಪ್ರದಾಯದಲ್ಲಿರುವವರೂ, ಹೇಗೆ ಬಾಳು ನಡೆಸಬೇಕೆಂದು ವಿಧಿನಿಯಮವಿದೆಯೋ ಹಾಗೆಯೇ ಬಾಳು ನಡೆಸಿದರೆ ಅವರಿಗೆ ಸುಖವಾಗಿರಲು ಸಾಧ್ಯ, ಅವರ ಸುತ್ತ-ಮುತ್ತ ಇರುವವರಿಗೂ ಸುಖವಾಗಿರಲು ಸಾಧ್ಯ. ಹಾಗಿಲ್ಲದೆ ತನಗೆ ಇಷ್ಟಬಂದಂತೆ ಯಾರಾದರೂ ಬಾಳು ನಡೆಸಿದರೆ, ಭಗವಂತನ ಪ್ರಸಾದ ದೊರೆಯುವುದೇ? ಇದಕ್ಕೆ ಬದಲಾಗಿ, ಅವನ ಧರ್ಮಗಳನ್ನು ಮೀರಿದ್ದಕ್ಕಾಗಿ ಭಗವಂತನ ಕೋಪಕ್ಕೆ ಅವನು ಪಾತ್ರನಾಗುತ್ತಾನೆ.

ಒಬ್ಬ ಹುಡುಗ ತನ್ನ ತಂದೆ-ತಾಯಿಯ ಮಾತುಗಳನ್ನು ಕೇಳುವುದೇ ಇಲ್ಲ. ತಂದೆ-ತಾಯಿ ಸ್ಕೂಲಿಗೆ ಹೋಗು ಎಂದರೆ ಹುಡುಗ ಸಿನಿಮಾಗೆ ಹೋಗುತ್ತೇನೆ ಎನ್ನುತ್ತಾನೆ. ಹೀಗೆಲ್ಲ ನಿಯಮಕ್ಕೆ ವಿರೋಧವಾಗಿ ನಡೆದುಕೊಂಡು ತನ್ನ ತಂದೆಗೆ ತನ್ನನ್ನು ಕಂಡರೆ ಆಗುವುದೇ ಇಲ್ಲ ಎಂದು ಹುಡುಗ ಹೇಳಿದರೆ ತಂದೆ ಏನು ಮಾಡುತ್ತಾರೆ? ಹಾಗೆಯೇ ಭಗವಂತನ ನಿಯಮಗಳನ್ನು ನಾವು ಅತಿಕ್ರಮಿಸಿ ನಡೆದುಕೊಂಡು ನಮಗೆ ಭಗವಂತನ ಪ್ರಸಾದ ಬೇಕೆಂದರೆ ಅದು ದೊರೆಯುವುದಿಲ್ಲ. ಆದ್ದರಿಂದಲೇ ಶ್ಲೋಕದಲ್ಲಿ,

\begin{shloka}
ಶಾಂತಿಂ ನೇಯಃ ಸ ಕಥಮಧುನಾಪ್ಯವ್ಯವಸ್ಥಾ ಪ್ರವೃತ್ತೈಃ||
\end{shloka}

ಎಂದು ಹೇಳಲ್ಪಟ್ಟಿದೆ. ಇಷ್ಟು ದಿನಗಳು ಪ್ರಪಂಚವನ್ನು ನೋಡಿ, ಶಾಸ್ತ್ರವನ್ನು ಓದಿ ಸತ್ಪುರುಷರೊಡನೆ ಸಹವಾಸ ಮಾಡಿಯೂ ಕೂಡ ಶಾಸ್ತ್ರ ನಿಯಮಗಳಿಗೆ ಅನುಗುಣವಾಗಿ ಜೀವನ ನಡೆಸದೆ ಇದ್ದರೆ ಹೇಗೆ ನಾವು ಭಗವಂತನ ಕೋಪವನ್ನು ಶಾಂತ ಮಾಡಬಲ್ಲೆವು ಎಂದು ನೀಲಕಂಠ ದೀಕ್ಷಿತರು ಒಂದು ಪ್ರಶ್ನೆ ಒಡ್ಡಿದರು.

\begin{shloka}
ಸಾ ಕಥಂ ಶಾಂತಿಂ ನೇತುಂ ಶಕ್ಯಃ
\end{shloka}

(ಅದನ್ನು ಹೇಗೆ ಸಮಾಧಾನಪಡಿಸುವುದು?)

ಆದ್ದರಿಂದ ನಮ್ಮ ನಮ್ಮ ಆಶ್ರಮಗಳಿಗೆ ಯೋಗ್ಯವಾದ ಧರ್ಮಗಳನ್ನು ನಾವು ಸರಿಯಾದ ರೀತಿಯಲ್ಲಿ ಆಚರಿಸುತ್ತಾ ಇರಬೇಕು.

ವೇದಾಂತ ಶಾಸ್ತ್ರವನ್ನು ಓದಿ ನೋಡಿದರೆ ಪರಮಾತ್ಮನೇ ಜೀವಾತ್ಮನಾಗಿಬಿಟ್ಟನೆಂದು ತಿಳಿಯುತ್ತದೆ. ಯಾವಾಗ ಪರಮಾತ್ಮನು ಜೀವಾತ್ಮನಾದನೆನ್ನುವ ಪ್ರಶ್ನೆಗೆ ನಾವು ಉತ್ತರ ಕೊಡಲು ಸಾಧ್ಯವಿಲ್ಲ. ಯಾವಾಗ ಬೀಜ ಉಂಟಾಯಿತು ಎಂದು ಕೇಳಿದರೆ ಮರ ಉಂಟಾಗುವುದಕ್ಕೆ ಮೊದಲು ಎಂದು ಹೇಳಬಹುದು. ಯಾವಾಗ ಮರ ಉಂಟಾಯಿತು ಎಂದು ಕೇಳಿದರೆ ಅದಕ್ಕೆ ನಾವು ಉತ್ತರ ಹೇಳಬಹುದು. ಮೊದಲು ಬೀಜ ಉಂಟಾಯಿತು ಎಂದು ಹೇಳಿದರೆ ಮರವಿಲ್ಲದೆ ಬೀಜ ಹೇಗೆ ಉಂಟಾಗುತ್ತದೆ ಎನ್ನುವ ಪ್ರಶ್ನೆ ಏಳುತ್ತದೆ. ಮರವೇ ಮೊದಲು ಉಂಟಾಯಿತು ಎಂದರೆ ಬೀಜವಿಲ್ಲದೆ ಮರ ಹೇಗೆ ಉಂಟಾಗುತ್ತದೆ ಎಂದು ಮತ್ತೊಂದು ಪ್ರಶ್ನೆ ಏಳುತ್ತದೆ. ಆದ್ದರಿಂದ ಬೀಜ-ಮರ-ಬೀಜ ಎಂದು ಚಕ್ರದಂತೆ ಅನಾದಿಯಾಗಿ ಬರುತ್ತದೆಯೆಂದು ಮಾತ್ರ ಹೇಳಬಲ್ಲೆವು. ಅದೇ ರೀತಿ ಜೀವನು ಅನಾದಿಯಾಗಿ ಶರೀರಗಳನ್ನು ಗ್ರಹಿಸುತ್ತಿದ್ದಾನೆ. ಅದಕ್ಕೆ ಕಾರಣವಾಗಿ `ಕರ್ಮ' ಹೇಳಲ್ಪಡುವುದು, `ಕರ್ಮ' ಎನ್ನುವುದು ನಮ್ಮ ಕೆಲಸಗಳನ್ನು ಸೂಚಿಸುತ್ತದೆ.

ಕರ್ಮದ ಮಹಿಮೆಯನ್ನು ಕುರಿತು ಹೇಳುವಾಗ ಕವಿ ಒಬ್ಬರು-

\begin{shloka}
`ಬ್ರಹ್ಮಾಯೇನ ಕುಲಾಲವನ್ನಿಯಮಿತಃ ಬ್ರಹ್ಮಾಂಡಭಾಂಡೋದರೇ\\
ವಿಷ್ಣುರ್ಯೇನ ದಶಾವತಾರಗಹನೇ ಕ್ಷಿಪ್ತೋಮಹಾಸಂಕಟೇ|\\
ಶಂಭುರ್ಯೇನ ಕಪಾಲಪಾಣಿ ಪುಟಕೇ ಭೀಕ್ಷಾಟನಂ ಸೇವತೇ\\
ಸೂರ್ಯೇ ಭ್ರಾಮ್ಯತಿ ನಿತ್ಯಮೇವ ಗಗನೇ ತಸ್ಮೈ ನಮಃ ಕರ್ಮಣೇ||'
\end{shloka}

[ಯಾವುದರಿಂದ ಬ್ರಹ್ಮನು ಪ್ರಪಂಚದಲ್ಲಿ (ಸೃಷ್ಟಿಯೆನ್ನುವ ಕೆಲಸದಲ್ಲಿ) ಕುಂಬಾರನಂತೆ ನಿಯಮಿತನಾಗಿದ್ದಾನೋ, ಯಾವುದರಿಂದ ವಿಷ್ಣು ದೊಡ್ಡ ಸಂಕಟವಾದ ಹತ್ತು ಅವತಾರವೆನ್ನುವ ಕಾಡಿನಲ್ಲಿ ಹಾಕಲ್ಪಟ್ಟಿದ್ದಾನೋ, ಯಾವುದರಿಂದ ಪರಮೇಶ್ವರನು ಕೈಯಲ್ಲಿ ಕಪಾಲವನ್ನು ಹಿಡಿದು ಭಿಕ್ಷೆಯನ್ನು ತೆಗೆದುಕೊಳ್ಳುತ್ತಾನೋ ಮತ್ತು ಯಾವುದರಿಂದ ಸೂರ್ಯನು ಪ್ರತಿ ನಿತ್ಯವು ತಿರುಗುತ್ತಲೇ ಇದ್ದಾನೋ ಅಂಥ ಕರ್ಮಕ್ಕೆ ನಮ್ಮ ನಮಸ್ಕಾರ] ಎಂದು ಹೇಳಿದರು. ಆದರೆ, ಕವಿ ಹೇಳುವುದನ್ನೂ ನಾವು ಒಪ್ಪಲಾಗುವುದಿಲ್ಲ. ಅವರು ಕರ್ಮದ ಮಹಿಮೆಯನ್ನು ಹೇಳಬೇಕೆಂದಿರುವುದರಿಂದ ಭಗವಂತನು ಕೂಡ ಕರ್ಮದಿಂದ ಕಷ್ಟಪಟ್ಟನೆಂದು ತೋರಿಸುತ್ತಾರೆ.

ಆದರೆ, ನಿಜಕ್ಕೂ ಭಗವಂತನೂ ಕರ್ಮಕ್ಕೆ ಸಿಕ್ಕಿದವನೇ ಅಲ್ಲ. ಕರ್ಮಕ್ಕೆ  ಸಿಕ್ಕಿಕೊಂಡರೆ ಅವನು ಭಗವಂತನೇ ಅಲ್ಲ. ರಾಮನು ವನವಾಸ ಮಾಡಿದನು. ರಾಮನು ಭಗವಂತನೆಂದೇ ನಾವು ಖಂಡಿತವಾಗಿಯೂ ಹೇಳುತ್ತೇವೆ. ವಾಲ್ಮೀಕಿ, ರಾಮನಿಗೆ ಕರ್ಮವಿದೆ ಎಂದು ಬರೆದಿದ್ದಾರಲ್ಲಾ! ಆದ್ದರಿಂದ ಭಗವಂತನಾದ ರಾಮನು ವನವಾಸದಂಥವುಗಳನ್ನು ಅನುಭವಿಸಬೇಕಾಗಿದ್ದರಿಂದ ಅವನು ಕರ್ಮಕ್ಕೆ ಸಿಕ್ಕಿಕೊಂಡವನೇ ಎಂದು ಅದರ ತಾತ್ಪರ್ಯವನ್ನು ತಿಳಿಯದೆ ಕೆಲವರು ಕೇಳಬಹುದು. ಮನುಷ್ಯ ಜನ್ಮವನ್ನು ಪಡೆದರೆ ಹೀಗೆಲ್ಲಾ ಅನುಭವಿಸಬೇಕೆಂದು ತೋರಿಸುವುದಕ್ಕಾಗಿಯೇ ಭಗವಂತನು ಲೀಲೆ ಮಾಡಿದನು. `ದಶರಥನು ರಾಜ ಅವನ ಮಗ ರಾಮ' ಎಂದು ಸಾಮಾನ್ಯವಾಗಿ ರಾಮಾಯಣ ಓದುವವರಿಗೆ `ರಾಮ ದೊಡ್ಡ ಮಹಾತ್ಮ, ಆದರೂ ಹೀಗೆಲ್ಲ ಕಷ್ಟಪಟ್ಟನು' ಎಂದೇ ತೋರುತ್ತದೆ. ಅಲ್ಲದೆ `ಪರಮಾತ್ಮ ಎಂದು ಹೇಳಿದ ಮೇಲೆ ಅವನಿಗೆ ಇಷ್ಟೆಲ್ಲ ಕಷ್ಟವೇಕೆ?' ಎಂದು ಅವರಿಗೆ ಸಂದೇಹವೂ ಆಗುತ್ತದೆ. ಆದ್ದರಿಂದ ಸ್ವಲ್ಪ ವಿವರಗಳನ್ನು ತಿಳಿದುಕೊಂಡವರಿಗೆ ಹೇಳಿದರೆ ಭಗವಂತನು ತೋರಿದ ಲೀಲೆಯು ಅರ್ಥವಾಗುತ್ತದೆ. ಸರಿ, ಈಗ ಕವಿ ಹೇಳಿರುವ ಶ್ಲೋಕವನ್ನು ನೋಡೋಣ.

\begin{shloka}
`ಬ್ರಹ್ಮಾ ಯೇನ ಕುಲಾಲವನ್ನಿಯಮಿತಃ ಬ್ರಹ್ಮಾಂಡ ಭಾಂಡೋದರೇ'
\end{shloka}

ಬ್ರಹ್ಮನಿಗೆ ಒಂದು ಕೆಲಸವಿದೆ, ಅದೇನೆಂದರೆ ಈ ಪ್ರಪಂಚವನ್ನು ಸೃಷ್ಟಿ ಮಾಡುವುದೇ ಬ್ರಹ್ಮನ ಕೆಲಸ. ಕುಂಬಾರನಿಗೆ ಮಡಕೆಗಳನ್ನು ತಯಾರು ಮಾಡುವುದು ಕೆಲಸವೆಂದು ನಾವು ಹೇಳುವಂತೆ ಈ ಪ್ರಪಂಚವೆಲ್ಲವನ್ನೂ, ಸೃಷ್ಟಿಸುವುದು ಬ್ರಹ್ಮನ ಕೆಲಸ. ಬ್ರಹ್ಮನಿಗೆ ಏತಕ್ಕೆ ಈ ಶಿಕ್ಷೆ ಬಂದಿತೆಂದರೆ, ಅದಕ್ಕೆ ಅವನ ಕರ್ಮವೇ ಕಾರಣವೆನ್ನುತ್ತಾರೆ ಕವಿ. ಭಗವಂತನು ಗೀತೆಯಲ್ಲಿ-

\begin{shloka}
`ಅಜೋಽಪಿ ಸನ್ನವ್ಯಯತ್ಮಾ ಭೂತಾನಾಮೀಶ್ವರೋಽಪಿ ಸನ್|'\\
ಪ್ರಕೃತಿಂ ಸ್ವಾಮಧಿಷ್ಠಾಯ ಸಂಭವಾಮ್ಯಾತ್ಮ ಮಾಯಯಾ||
\end{shloka}

(ನಾನು ಅಜನಾದರೂ, ಅವ್ಯಯನಾದರೂ, ಎಲ್ಲ ಜೀವರಿಗೂ ಈಶ್ವರನಾದರೂ ನನ್ನ ಪ್ರಕೃತಿಯನ್ನು ವಶದಲ್ಲಿಟ್ಟುಕೊಂಡು, ನನ್ನ ಮಾಯೆಯಿಂದಲೇ ಅವತರಿಸುವೆನು) ಎಂದಿದ್ದಾನೆ. ನಾವೂ ಹುಟ್ಟುತ್ತೇವೆ, ಭಗವಂತನೂ ಅವತಾರ ಮಾಡುತ್ತಾನೆ. ಭಗವಂತನು ತನಗೆ ಬೇಕೆಂದು ತೋರಿದರೆ ಮಾತ್ರ ಅವತಾರ ಮಾಡುತ್ತಾನೆ. ನಮ್ಮ ವಿಷಯ ಹಾಗಲ್ಲ. ನಾವೆಲ್ಲರೂ ಹುಟ್ಟಿ ಏಕೆ ಹುಟ್ಟಿದೆವೋ ಎಂದು ಯೋಚನೆ ಮಾಡುತ್ತೇವೆ. ಆದರೆ ಭಗವಂತನು ತಾನು ಅವತಾರ ಮಾಡಬೇಕೇ, ಬೇಡವೇ ಎನ್ನುವುದನ್ನು ಮೊದಲೇ ಯೋಚಿಸುತ್ತಾನೆ. ಯಾವುದಾದರೂ ಲೀಲೆಯನ್ನು ತೋರಿಸಬೇಕೆಂದರೆ ಅವನು ಅವತಾರ ಮಾಡುತ್ತಾನೆ. ಈ ಸ್ಥಿತಿಯಲ್ಲಿ ಕರ್ಮದ ಮಹಿಮೆಯನ್ನು ಹೇಳಬೇಕೆಂದಿರುವ ಕವಿ, ಏನು ಹೇಳುತ್ತಾರೆಂದು ನೋಡೋಣ,

\begin{shloka}
`ವಿಷ್ಣುರ್ಯೇನ ದಶಾವತಾರಗಹನೇ ಕ್ಷಿಪ್ತೋ ಮಹಾಸಂಕಟೇ'
\end{shloka}

ಭಗವಾನ್ ಮಹಾವಿಷ್ಣು ದಶಾವತಾರಗಳನ್ನು ಮಾಡಿದನು. ಆ ಹತ್ತು ಅವತಾರಗಳೂ ಸಾಮಾನ್ಯವಾದ ಅವತಾರಗಳಲ್ಲ. ಅವುಗಳಲ್ಲಿ ಒಂದೊಂದರಲ್ಲಿಯೂ ಮಹಾಕಾರ್ಯಗಳನ್ನು ಅವನು ಸಾಧಿಸಿದನು. ಮತ್ಸ್ಯಾವತಾರ ಮಾಡಿದಾಗ ವೇದಗಳೆಲ್ಲವನ್ನೂ ಬ್ರಹ್ಮದೇವನಿಂದ ಹಯಗ್ರೀವನು ಅಪಹರಿಸಿದಾಗ ಭಗವಂತನು ಹಯಗ್ರೀವನನ್ನು ಕೊಂದು ವೇದಗಳನ್ನು ಮುಂದಿನ ಕಲ್ಪದಲ್ಲಿ ನಮಗೆ ಕೊಟ್ಟನು. ಅನಂತರ ಕೂರ್ಮಾವತಾರ ಮಾಡಿದಾಗ, ದೇವಾಸುರರು ಕ್ಷೀರ ಸಮುದ್ರವನ್ನು ಕಡೆದಾಗ, ಮಂದರ ಪರ್ವತವನ್ನು ಹೊತ್ತು, ಆಮೆಯ ರೂಪವನ್ನು ಧರಿಸಿದನು. ಅದರಿಂದ ದೇವತೆಗಳು ಅಮೃತವನ್ನು ಪಡೆದರು. ರಾಮಾವತಾರ, ಕೃಷ್ಣಾವತಾರಗಳ ಕಥೆಗಳನ್ನು ನಾವು ಕೇಳಿದ್ದೇವೆ. ಹೀಗೆ ಅನೇಕ ಅವತಾರಗಳನ್ನು ಭಗವಂತನು ಮಾಡಿದನು. ಕವಿ `ದಶಾವತಾರ ಗಹನೇ' ಎನ್ನುತ್ತಾರೆ. ಅಂದರೆ ದಶಾವತಾರವೆನ್ನುವ ಸಂಕಷ್ಟಕರವಾದ ಕಾಡಿನಲ್ಲಿ ಭಗವಂತನು ಸಿಕ್ಕಿಕೊಂಡನೆನ್ನುತ್ತಾರೆ. ಕವಿಯ ತಾತ್ಪರ್ಯದ ಪ್ರಕಾರ ಭಗವಂತನ ಈ ಹತ್ತು ಅವತಾರಗಳೂ ಅವನ ಕರ್ಮವೇ ಆಗುತ್ತದೆ. ಭಗವಾನ್ ವಿಷ್ಣು ಮಾಡಿದ ಅವತಾರಗಳಿಗೆ ಒಂದು ಶಾಪವೇ ಕಾರಣವೆಂದು ಒಂದು ಕಥೆಯೂ ಇದೆ. ಕವಿ ಅನಂತರ,

\begin{shloka}
`ಶಂಭುರ್ಯೇನ ಕಪಾಲಪಾಣಿಪುಟಕೇ ಭಿಕ್ಷಾಟನಂ ಸೇವತೇ'
\end{shloka}

ಎಂದು ಹೇಳಿದರು. ಭಗವಂತನಾದ ಪರಮಶಿವನು ಕೈಲಾಸಕ್ಕೆಲ್ಲಾ ಅಧಿಪತಿ. ಅನ್ನಪೂರ್ಣಾ ಅವನ ಹೆಂಡತಿ. ಆದರೂ, ಪರಮಶಿವನೂ `ಭವತಿ ಭಿಕ್ಷಾಂ ದೇಹಿ' ಎಂದು ಕೇಳಿದನಂತೆ. ಅದಕ್ಕೆ ಶನಿ `ಸಮಯ ಬಂದರೆ ನಿನಗೂ ತೊಂದರೆ ಮಾಡುತ್ತೇನೆ' ಎಂದನಂತೆ. ಪರಮಶಿವನು, `ನಿನ್ನ ಕಣ್ಣಿಗೆ ಬಿದ್ದರೆ ತಾನೆ ನೀನು ನನಗೆ ತೊಂದರೆ ಕೊಡುವುದು' ಎಂದು ಹೇಳಿ ತಾವರೆ ಹೂವಿನಲ್ಲಿ ಸಣ್ಣ ದುಂಬಿಯಾಗಿ ಅವಿತುಕೊಂಡನಂತೆ. ಶನಿಗೆ, `ಈಗ ನೀನು ನನಗೆ ಯಾವ ವಿಧವಾದ ತೊಂದರೆಯೂ ಕೊಡಲಾಗುವುದಿಲ್ಲ' ಎಂದನು. ಶನಿ ಪರಮಶಿವನಿಗೆ, `ನೀನು ನನಗಾಗಿ ಬಹಳ ಕಷ್ಟಪಟ್ಟು ಚಿಕ್ಕರೂಪವನ್ನು ಪಡೆದು ಇಷ್ಟು ಹೊತ್ತು ಇದ್ದುದೇ ನಾನು ನಿನಗೆ ಮಾಡಿದ ತೊಂದರೆ' ಎಂದನು.

ಆದ್ದರಿಂದ ಕವಿಯ ತಾತ್ಪರ್ಯದಂತೆ, ಭಗವಂತನಾದ ಪರಮಶಿವನೂ ಭಿಕ್ಷಾಟನೆ ಮಾಡುವುದಕ್ಕೆ ಅವನ ಕರ್ಮವೇ ಕಾರಣ. ಆದರೆ, ಮೊದಲೇ ಹೇಳಿದಂತೆ ನಿಜಕ್ಕೂ ಪರಮಶಿವನಿಗೆ ಕರ್ಮದ ಯಾವ ಬಂಧನವೂ ಇಲ್ಲದಿದ್ದರೂ ಪ್ರಪಂಚದಲ್ಲಿ ಕರ್ಮದ ಪ್ರಾಬಲ್ಯ ಇಷ್ಟಾಗಿದೆಯೆಂದು ತೋರಿಸುವುದಕ್ಕೇ ಅವನು ಈ ರೀತಿ ಲೀಲೆ ಮಾಡಿದನು. ಶ್ಲೋಕದ ಕೊನೆಯ ಚರಣದಲ್ಲಿ ಕವಿ,

\begin{shloka}
`ಸೂರ್ಯೋ ಭ್ರಾಮ್ಯತಿ ನಿತ್ಯಮೇವ ಗಗನೇ'
\end{shloka}

ಎಂದರು. ಎಲ್ಲರಂತೆಯೇ ಸೂರ್ಯನಿಗೆ ಕೂಡ ಏನು ಶಿಕ್ಷೆ? ಪ್ರತಿದಿನವೂ ಭೂಮಂಡಲದ ಸುತ್ತಲೂ ತಿರುಗುವುದು ಸೂರ್ಯನ ಕೆಲಸ. ಏಕೆ ಅವನು ಹೀಗೆ ಸುತ್ತುತ್ತಿರಬೇಕೆಂದರೆ, ಅದಕ್ಕೆ ಅವನ ಕರ್ಮವೇ ಕಾರಣವೆನ್ನುತ್ತಾರೆ ಕವಿ. ಕೊನೆಯಲ್ಲಿ ಕವಿ,

\begin{shloka}
`ತಸ್ಮೈ ನಮಃ ಕರ್ಮಣೇ'
\end{shloka}

ಎನ್ನುತ್ತಾರೆ. ಅಂಥ ಬಲಿಷ್ಠವಾದ ಕರ್ಮವನ್ನು ನಮಸ್ಕರಿಸುತ್ತಾರೆ.

ಕರ್ಮ ಎನ್ನುವುದು ಯಾರನ್ನು ಬಿಟ್ಟಿಲ್ಲ. ಹೇಗೆ ವ್ಯಾಧಿ ವೈದ್ಯನನ್ನೂ ಪೀಡಿಸುತ್ತದೆಯೋ ಅದೇ ರೀತಿ ಕರ್ಮ ಎಲ್ಲರನ್ನೂ ತನ್ನ ಹಿಡಿತದಲ್ಲಿಟ್ಟುಕೊಳ್ಳುತ್ತದೆ. ವೈದ್ಯನು ಕೆಮ್ಮಿಗೆ ಔಷಧಿ ಕೊಡುತ್ತಲೇ ಇರುತ್ತಾನೆ. ಆದರೆ, ಅವನೂ ಕೆಮ್ಮುತ್ತಲೇ ಇರುತ್ತಾನೆ. ಕಾರಣ ಕೇಳಿದರೆ ಎಷ್ಟೋ ಮಂದಿಗೆ ಗುಣಕಾರಿಯಾದ ಆ ಔಷಧಿ ಅವನಿಗೆ ಮಾತ್ರ ಪರಿಣಾಮಕಾರಿಯಾಗಲಿಲ್ಲ ಎನ್ನುತ್ತಾನೆ. ಹಾಗೆ ನಡೆಯುವುದನ್ನು ನಾವೇ ನೋಡಿರಬಹುದು. ಹೀಗೆ ನಡೆಯುವುದಕ್ಕೆ ಅವರವರ ಕರ್ಮವೇ ಕಾರಣ. ವೈದ್ಯನಿಗೂ, ರೋಗಿಗೂ ಆ ವ್ಯಾಧಿಯನ್ನು ಅನುಭವಿಸಬೇಕೆನ್ನುವ ಕರ್ಮವೇ ಇದಕ್ಕೆ ಕಾರಣ. ಇನ್ನು, ಕರ್ಮದ ಸ್ವರೂಪ ನೋಡೋಣ.

ಮನುಷ್ಯನು ಯಾವಾಗಲೂ ಯಾವುದಾದರೂ ಒಂದು ಕೆಲಸ ಮಾಡುವಂತೆಯೇ ತೋರುತ್ತಾನೆ. ಜಾಗ್ರತ್ ಅವಸ್ಥೆಯಲ್ಲಿ ಮನುಷ್ಯನು ಕೆಲಸ ಮಾಡುತ್ತಿದ್ದಾನೆ ಎನ್ನುವುದನ್ನು ನಾವು ನೋಡಿ ತಿಳಿದುಕೊಳ್ಳುತ್ತೇವೆ. ಕನಸಿನಲ್ಲಿ ಮನಸ್ಸು ಏನೇನೋ ಕಲ್ಪನೆ ಮಾಡುತ್ತದೆ ಎನ್ನುವುದೂ ನಮಗೆ ಗೊತ್ತು. ಕನಸಿನಲ್ಲಿ ನಾವು ಬೆಟ್ಟವನ್ನು ದಾಟುವ ಹಾಗೆಯೂ, ನದಿಯಲ್ಲಿ ಮುಳುಗುವ ಹಾಗೆಯೂ ತೋರುತ್ತದೆ. ಕೆಲವೊಮ್ಮೆ ಆಕಾಶದಲ್ಲಿ ಹಾರುತ್ತಿರುವಂತೆ ಕನಸಿನಲ್ಲಿ ತೋರುತ್ತದೆ. ಸೂಕ್ಷ್ಮ ಶರೀರ ಹೀಗೆ ವಿಶೇಷವಾಗಿ ಕೆಲಸ ಮಾಡಿದರೂ ಸ್ಥೂಲ ಶರೀರ ಕೆಲಸ ಮಾಡುವುದಿಲ್ಲ. ಸುಷುಪ್ತಿಯಲ್ಲಿ ಯಾವ ವಿಧವಾದ ಕೆಲಸವೂ ಇಲ್ಲ ಎನ್ನುವುದನ್ನು ನಾವೆಲ್ಲರೂ ಒಪ್ಪಿಕೊಳ್ಳುತ್ತೇವೆ. ಕನಸಿನಲ್ಲಿ ನಾವು ಯಾವ ತಪ್ಪು ಮಾಡಿದರೂ ಅದು ತಪ್ಪಲ್ಲ. ಏಕೆಂದರೆ ಪ್ರಪಂಚದಲ್ಲಿ ಅದರಿಂದ ಯಾರಿಗೂ ಯಾವ ವಿಧವಾದ ಲಾಭವೂ ಇಲ್ಲ, ನಷ್ಟವೂ ಇಲ್ಲ. ಒಂದು ಮುಖ್ಯ ವಿಷಯವನ್ನು ನಾವು ಗಮನಿಸಬೇಕು. ಕನಸಿನಲ್ಲಿ ಏನೇನು ಬರುತ್ತದೆಯೋ ಅದಕ್ಕೆ ಅವನವನ ಕರ್ಮವೇ ಕಾರಣ. ಕನಸಿನಲ್ಲಿ ಒಮ್ಮೆ ತಾನು ರಾಜನಾಗಿ ಕುಳಿತುಕೊಂಡಂತೆ ತೋರುವಾಗ ಬಹಳ ಸಂತೋಷವಾಗುತ್ತದೆ. ಮತ್ತೊಮ್ಮೆ ತನ್ನನ್ನು ಪೋಲೀಸಿನವನು ಹಿಡಿದುಕೊಂಡು ಹೋಗುತ್ತಿದ್ದಂತೆ ತೋರಿದರೆ ಆ ನಿಮಿಷ ಅದರ ಬಗ್ಗೆ ವ್ಯಥೆಯಾಗುತ್ತದೆ. ಆದ್ದರಿಂದ ಒಮ್ಮೆ ಸಂತೋಷ ಮತ್ತೊಮ್ಮೆ ದುಃಖವೂ ಆಗುತ್ತದೆ. ಹೀಗೆ ಕನಸಿನಲ್ಲಿ ಉಂಟಾಗುವ ದುಃಖಗಳಿಗೂ ಕಾರಣವಾಗಿರುವುದು ಕರ್ಮ ಆಗುತ್ತದೆ.

ಇದೇ ರೀತಿ ಜಾಗ್ರತ್ ಸ್ಥಿತಿಯಲ್ಲಿ ನಡೆಯುವುದೆಲ್ಲ ಕರ್ಮದ ಕಾರಣದಿಂದಲೇ ಉಂಟಾದವು ಆಗುತ್ತವೆ. ನಾವು ಮೊದಲು ಮಾಡಿದ ಸತ್ಕರ್ಮಗಳ ಫಲವಾಗಿ ತಾತ್ಕಾಲಿಕವಾಗಿ ಸುಖವನ್ನೂ, ಕೆಟ್ಟ ಕರ್ಮದ ಫಲವಾಗಿ ದುಃಖವನ್ನೂ ಪಡೆಯುತ್ತೇವೆ. ಯಾವುದು ಒಳ್ಳೆಯ ಕೆಲಸವೆಂದು ಇಲ್ಲಿ ಹೇಗೆ ತೀರ್ಮಾನ ಮಾಡುವುದು ಎನ್ನುವ ಪ್ರಶ್ನೆ ನಮಗೆ ಉಂಟಾಗುತ್ತದೆ. ಆ ಪ್ರಶ್ನೆಗೆ ಸಮಾಧಾನ ನೋಡಲು ನಮಗೆ ವೇದವಿದೆ. ವೇದದಲ್ಲಿ ಧರ್ಮ ಯಾವುದು, ಅಧರ್ಮ ಯಾವುದು ಎಂದು ಹೇಳಲ್ಪಟ್ಟಿದೆ. ಯಾವುದು ನಮ್ಮಿಂದ ಮಾಡಲ್ಪಡಬೇಕೋ ಅದು ಧರ್ಮ. ನಮ್ಮಿಂದ ಮಾಡಲ್ಪಡಬಾರದೋ ಅದು ಅಧರ್ಮ. ಪ್ರಪಂಚದಲ್ಲಿ ನಿಯಮ ಕಾನೂನು ಎನ್ನುವುದಿದೆ. ಕಾನೂನು ಪುಸ್ತಕದಲ್ಲಿ ಯಾವು ಯಾವುದನ್ನು ನಾವು ಮಾಡಬೇಕೆಂದು ಹೇಳಿದೆ. ಕಾನೂನು ನಿಯಮಗಳನ್ನು ಉಲ್ಲಂಘಿಸಿದರೆ ನಮಗೆ ಶಿಕ್ಷೆಯಾಗುತ್ತದೆ. ನಾವು ಯಾವು ಯಾವುದನ್ನು ಮಾಡಬೇಕೆನ್ನುವುದನ್ನು ಸಾಮಾನ್ಯವಾಗಿ ಕಾನೂನು ಹೇಳುವುದಿಲ್ಲ. ಮಾಡಬೇಕಾದವುಗಳನ್ನು ನಮ್ಮ ಇಷ್ಟಕ್ಕೆ ಬಿಟ್ಟುಬಿಡುತ್ತಾರೆ. ಆದರೆ, ಕೆಲವು ವೇಳೆ ನಾವು ಏನು ಮಾಡಬೇಕು ಎನ್ನುವುದನ್ನೂ ಕಾನೂನೇ ತೀರ್ಮಾನಿಸುತ್ತದೆ. ಉದಾಹರಣೆಗೆ ನಾವು ಯಾವುದಾದರೂ ಮನೆ ಕಟ್ಟಬೇಕೆಂದು ಇಷ್ಟಪಟ್ಟರೆ ನಮಗೆ ತೋರಿದ ಜಾಗದಲ್ಲಿ ಹಾಗೆ ಕಟ್ಟಲಾಗುವುದಿಲ್ಲ. ನಾವು ಆ ಜಾಗವನ್ನು ಕೊಂಡಿರಬೇಕು. ಇಲ್ಲದಿದ್ದರೆ ಜಾಗ ನಮ್ಮ ತಂದೆಯಿಂದ ನಮಗೆ ದೊರೆತಿರಬೇಕು. ಇದೂ ಇಲ್ಲದಿದ್ದರೆ ಯಾರಾದರೂ ದಾನವಾದರೂ ಕೊಟ್ಟಿರಬೇಕು. ಕಟ್ಟಡವನ್ನು ಕಟ್ಟುವುದಕ್ಕೆ ಮೊದಲು ಪುರಸಭೆಯ ಅನುಮತಿ ಪಡೆದಿರಬೇಕು. ಹೀಗೆಲ್ಲ ಇಲ್ಲದೆ `ಜಾಗ ಇಲ್ಲಿದೆ ನಾನು ಹೋಗಿ ಮನೆ ಕಟ್ಟಿದರೆ ಏನು ತಪ್ಪು?' ಎಂದುಕೊಂಡು ಯಾರಾದರೂ ಮನೆ ಕಟ್ಟಿದರೆ ಪುರಸಭೆ ಅವನನ್ನು ಬಿಟ್ಟುಬಿಡುವುದಿಲ್ಲ.

ಮನೆಯನ್ನು ಒಡೆದು ಹಾಕುವಂತೆ ಅವನಿಗೆ ಹೇಳಿ ನೋಡುತ್ತಾರೆ. ಅವನು ಹಾಗೆ ಮಾಡದೆ ಇದ್ದರೆ ಅವರೇ ಬಂದು ಒಡೆದು ಹಾಕಿ ಅದಕ್ಕೆ ಆಗುವ ಖರ್ಚನ್ನೂ ಅವನಿಂದಲೇ ವಸೂಲು ಮಾಡುತ್ತಾರೆ. ಅದು ಅವನಿಗೆ ಆಗದೇ ಹೋದರೆ ಜೈಲಿಗೆ ಕಳುಹಿಸುತ್ತಾರೆ. ಆದ್ದರಿಂದ ಕಾನೂನು ಪುಸ್ತಕದಲ್ಲಿ ಬರೆದಿರುವಂತೆ ನಡೆಯದವರು ಇಷ್ಟು ಶಿಕ್ಷೆ ಅನುಭವಿಸಬೇಕಾಗುತ್ತದೆ. ಆ ಪುಸ್ತಕದಲ್ಲಿ ಏನು ಬರೆಯಲ್ಪಟ್ಟಿದೆ ಎನ್ನುವುದು ನನಗೆ ಗೊತ್ತಿಲ್ಲ ಎಂದು ಯಾರೂ ಹೇಳಲಾಗುವುದಿಲ್ಲ. ಏಕೆಂದರೆ ಅದನ್ನು ಕುರಿತು ತಿಳಿಯುವುದು ಅವನ ಕರ್ತವ್ಯ. ವೇದವೂ ಕಾನೂನು ಪುಸ್ತಕದಂತೆ. ನಾವು ಯಾವ ರೀತಿ ಮಾಡಬೇಕು, ಯಾವುದನ್ನು ಮಾಡಬಾರದು ಎನ್ನುವುದನ್ನು ಸ್ಪಷ್ಟವಾಗಿ ತಿಳಿಸುತ್ತದೆ.

ವೇದದಲ್ಲಿ ನಂಬಿಕೆ ಇರುವ ಆಸ್ತಿಕನಾಗಿರುವವನು ಪರಲೋಕವಿದೆ, ತಾನು ಮಾಡುವ ಕರ್ಮಕ್ಕೆ ಫಲ ಇದೆ ಎಂಬುದನ್ನು ನಂಬುತ್ತಾನೆ. ಆದ್ದರಿಂದಲೇ `ಅಸ್ತಿ ನಾಸ್ತಿ ದಿಷ್ಟಂ ಮತಿಃ'  ಎಂದು ಹೇಳಿದೆ. ದಿಷ್ಟಂ ಎಂದರೆ ಅದೃಷ್ಟ ಎಂದು ತಾತ್ಪರ್ಯ. ನಾವು ಮಾಡುವ ಕರ್ಮಗಳು ಒಂದು ಸಂಸ್ಕಾರವನ್ನು ಕೊಡುತ್ತವೆ. ಅದಕ್ಕೆ ಅದೃಷ್ಟವೆಂದು ಹೆಸರು. ಕರ್ಮದ ಇನ್ನೊಂದು ವಿಶೇಷವನ್ನು ನೋಡೋಣ.

\begin{shloka}
`ಕುರ್ವತೇ ಕರ್ಮಭೋಗಾಯ ಕರ್ಮ ಕರ್ತುಂ ಚ ಭುಂಜತೇ|'
\end{shloka}

ನಾವು ಒಂದು ಆಫೀಸಿಗೆ ಹೋಗಿ ಕೆಲಸ ಮಾಡುತ್ತೇವೆ. ಅದು ಏಕೆಂದು ಕೇಳಿದರೆ ಸಂಬಳ ಪಡೆಯುವುದಕ್ಕಾಗಿಯೇ ಆಫೀಸಿಗೆ ಹೋಗುತ್ತೇವೆ. ಸಂಬಳ ಏಕೆ ತೆಗೆದುಕೊಳ್ಳುತ್ತೇವೆಂದರೆ, ಕೆಲಸ ಮಾಡಿದುದಕ್ಕಾಗಿ ಸಂಬಳ ತೆಗೆದುಕೊಳ್ಳುತ್ತೇವೆ. ಎರಡು ಕೆಲಸಗಳೂ ಒಂದಾದ ಮೇಲೆ ಒಂದು ಬರುತ್ತವೆ, ಅದೇ ರೀತಿ,

\begin{shloka}
`ಅಗ್ನಿಹೋತ್ರಂ ಜಹುಯಾತ್ ಸ್ವರ್ಗಕಾಮಃ|\\
ಜ್ಯೋತಿಷ್ಟೋಮೇನ ಸ್ವರ್ಗಕಾಮೋ ಯಜೇತ|\\
ಪುತ್ರಕಾಮೋ ಪುತ್ರೇಷ್ಟ್ಯಾ ಯಜೇತ|
\end{shloka}

ಹೀಗಿರುವಾಗ ವೇದ ವಾಕ್ಯಗಳು, ಆಯಾ ಫಲಗಳಿಗೆ ತಕ್ಕ ಕರ್ಮಗಳನ್ನು ಮಾಡಲು ತಿಳಿಸುತ್ತವೆ. ಆದ್ದರಿಂದ ಭೋಗವೆಲ್ಲ ಬೇಕೆನ್ನುವುದಕ್ಕಾಗಿ, ಅದಕ್ಕೆ ಬೇಕಾದ ಕರ್ಮಗಳನ್ನು ನಾವು ಮಾಡಬೇಕು ಕೊನೆಗೆ ಆ ಭೋಗವೇತಕ್ಕೆ ಎಂದರೆ `ಕರ್ಮ ಕರ್ತುಂ ಚ ಭುಂಜತೇ' ಎನ್ನುವಂತೆ ಆಗುತ್ತದೆ. ಪಾಯಸವನ್ನು ತಿನ್ನದ ಒಬ್ಬನಿಗೆ ಅವನ ಆಪ್ತ ಮಿತ್ರನೊಬ್ಬನು ಪಾಯಸವನ್ನು ಕೊಟ್ಟನಂತೆ. ಪಾಯಸವನ್ನು ತಿಂದ ಅವನಿಗೆ ಪಾಯಸ ಬಹಳ ರುಚಿಯಾಗಿದೆಯೆಂದು ತಿಳಿಯಿತು. ಅನಂತರ ತನ್ನ ಮನೆಯಲ್ಲಿಯೂ ಪಾಯಸ ಮಾಡಿಸಿಕೊಂಡು ತಿನ್ನಬೇಕೆನ್ನುವ ಆಸೆ ಉಂಟಾಯಿತು. ಆದ್ದರಿಂದ ಪಾಯಸವನ್ನು ತಿಂದ ಭೋಗ ಮತ್ತೆ ಪಾಯಸವನ್ನು ತಿನ್ನಬೇಕೆನ್ನುವುದಕ್ಕೆ ಕಾರಣವಾಯಿತು. ಮೊದಲು ಪಾಯಸ ತಿನ್ನದೆ ಇದ್ದಾಗ ತನ್ನ ಮನೆಯಲ್ಲಿ ಪಾಯಸ ಮಾಡಿಸಿಯೇ ಇರಲಿಲ್ಲ. ಆದ್ದರಿಂದ ಭೋಗದಿಂದ ರುಚಿ, ರುಚಿಯಿಂದ ಮತ್ತೆ ಕರ್ಮ, ಕರ್ಮದಿಂದ ಭೋಗ ಹೀಗೆ ಇದು ಸುತ್ತಿಕೊಂಡೇ ಬರುತ್ತದೆ. ಹೀಗಿರುವ ಕರ್ಮವನ್ನೇ `ಬಂಧಕ'ವೆನ್ನುತ್ತೇವೆ. `ಬಂಧಕ'ವೆಂದರೆ ಏನು ಅರ್ಥ? ಅಂಥ ಕರ್ಮ ಬಿಡುಗಡೆ ಇಲ್ಲದಂತೆ ನಮ್ಮ ಹಿಂದೆಯೇ ಸುತ್ತಿಕೊಂಡಿರುತ್ತದೆ. ನಾವು ಪ್ರಾಣ ಬಿಟ್ಟರೂ ಅದು ನಮ್ಮನ್ನು ಬಿಡುವುದಿಲ್ಲ. ಇಂದು ನಾವು ಮಾಡುವ ಕರ್ಮ ನಮ್ಮ ಹಿಂದೆಯೇ ಬರುತ್ತದೆ, ಭಗವಂತ ಗೀತೆಯಲ್ಲೂ-

\begin{shloka}
`ನೇಹಾಭಿಕ್ರಮನಾಶೋಸ್ತಿ ಪ್ರತ್ಯವಾಯೋ ನ ವಿದ್ಯತೇ\\
ಸ್ವಲ್ಪಮಪ್ಯಸ್ಯ ಧರ್ಮಸ್ಯ ತ್ರಾಯತೇ ಮಹತೋ ಭಯಾತ್||
\end{shloka}

(ಇದರಲ್ಲಿ ಪ್ರಯತ್ನವಿಲ್ಲ. ಇದರಲ್ಲಿ ಕೆಡುಕೆನ್ನುವುದೂ ಇಲ್ಲ. ಈ ಧರ್ಮವನ್ನು ಸ್ವಲ್ಪ ಮಾಡಿದರೂ, ಇದು ಅವನನ್ನು ಭಯದಿಂದ ಕಾಪಾಡುವುದು) ಎನ್ನುತ್ತಾನೆ. ಪುಣ್ಯ ಕರ್ಮವಾದರೆ ಅದು ದೊಡ್ಡ ಭಯದಿಂದ ನಮ್ಮನ್ನು ಕಾಪಾಡುತ್ತದೆ. ಪಾಪಕರ್ಮವಾದರೆ ಅದು ದೊಡ್ಡ ಭಯವನ್ನು ಕೊಡುತ್ತದೆ. ಆದ್ದರಿಂದ ನಾವು ಪ್ರಾಣ ಬಿಟ್ಟರೂ ಕರ್ಮ ಬಿಡದೆ ನಮ್ಮನ್ನು ಹಿಂಬಾಲಿಸುತ್ತದೆ.

ಒಂದೆಡೆಯಲ್ಲಿ

\begin{shloka}
`ನದ್ಯಾಂ ಕೀಟಾ ಇವಾವರ್ತಾತಾವರ್ತಾನ್ತರಮಾಶು ತೇ|\\
ವ್ರಜನ್ತೋ ಜನ್ಮನೋ ಜನ್ಮ ಲಭನ್ತೇ ನೈವನಿರ್ವೃತಿಮ್||
\end{shloka}

ಎಂದು ಹೇಳಲ್ಪಟ್ಟಿದೆ. ನಮಗೆ ಜನ್ಮ ಬಂಧನವಿರುವವರೆಗೆ ಅಮಿತ ಶಾಂತಿ ದೊರೆಯುವುದಿಲ್ಲ. ನಾವು ಸಾಧಾರಣವಾಗಿ ಅನುಭವಿಸುವ ಶಾಂತಿ ಹೇಗಿದೆಯೆಂದರೆ ಒಂದು ಆಸೆ ಮನಸ್ಸಿನಲ್ಲಿ ಉಂಟಾಗಿ ಆ ಆಸೆ ನಿಮಿಷದಲ್ಲೇ ಮುಗಿದು ಹೋಗುತ್ತದೆ. ಆದ್ದರಿಂದ ನಾವು ಸುಖವಾಗಿರಬಹುದೆಂದುಕೊಳ್ಳುವಾಗ ಇನ್ನೊಂದು ವಸ್ತುವನ್ನು ನೋಡುತ್ತೇವೆ. ತಕ್ಷಣ ಆ ವಸ್ತು ವಿಷಯದಲ್ಲಿ ಆಸೆ ಬಂದು ಬಿಡುತ್ತದೆ. ಅನಂತರ ಆ ವಸ್ತು ದೊರೆತ ಒಡನೆ ಬೇರೆ ವಿಷಯದಲ್ಲಿ ಮನಸ್ಸು ಆಸೆಪಡಲು ಪ್ರಾರಂಭಿಸುತ್ತದೆ.

\begin{shloka}
`ತೇಷಾಂ ಶಾಂತಿಃ ಶಾಶ್ವತೀ ನೇತರೇಷಾಮ್|'
\end{shloka}

ಎಂದು ಹೇಳಿದ ಪ್ರಕಾರ ನಾವು ಪರಮಾತ್ಮನ ಸಾಕ್ಷಾತ್ಕಾರ ಪಡೆದರೆ ಮಾತ್ರ ಶಾಶ್ವತವಾದ ಶಾಂತಿ ಎನ್ನುವುದು ದೊರೆಯುವುದು,

\begin{shloka}
`ನದ್ಯಾಂ ಕೀಟಾ ಇವಾವರ್ತಾತಾವರ್ತಾನ್ತರಮಾಶು ತೇ|'
\end{shloka}

ಚಿಕ್ಕ ಹುಳು ಅಲ್ಲಿ ಇಲ್ಲಿ ಓಡಾಡುತ್ತದೆ. ಅಕಸ್ಮಾತ್ತಾಗಿ ನದಿಯಲ್ಲಿ ಬಿದ್ದು ಬಿಡುತ್ತದೆ. ನದಿಯಲ್ಲಿ ಒಂದು ಸುಳಿ ಇದ್ದಿತು. ಹುಳು ಆ ಸುಳಿಯಲ್ಲಿಯೇ ಸುತ್ತಾಡಲು ಪ್ರಾರಂಭಿಸಿತು. ಆ ಸುಳಿಯಲ್ಲಿ ತಾನು ಕಷ್ಟಪಡಬೇಕಾಗಿದೆಯಲ್ಲಾ ಎಂದು ಅದು ಚಿಂತಿಸಿತು. ಸ್ವಲ್ಪ ಹೊತ್ತಿನಲ್ಲಿ ದೈವಿಕವಾಗಿ ಅದು ಸುಳಿಯಿಂದ ತಪ್ಪಿಸಿಕೊಂಡಿತು. ಆದರೆ ಸ್ವಲ್ಪ ದೂರ ಹೋಗುವುದರೊಳಗಾಗಿ ಇನ್ನೊಂದು ಸುಳಿಯಲ್ಲಿ ಸಿಕ್ಕಿಕೊಂಡಿತು. ನಮ್ಮ ರೀತಿಯೂ ಇದೇ ರೀತಿ ಇದೆ. {\eng{84}}ಲಕ್ಷ ಜೀವರಾಶಿಗಳಲ್ಲಿ ಯಾವುದಾದರೂ ಒಂದು ಜೀವರಾಶಿಯಾಗಿ ನಾವೂ ಹುಟ್ಟುತ್ತಲೇ ಇದ್ದೇವೆ. ಈಗ ಮನುಷ್ಯ ಜನ್ಮವನ್ನು ತೆಗೆದುಕೊಂಡಿದ್ದೇವೆ.

\begin{shloka}
ಶುಭೈರಾಪ್ನೋತಿ ದೇವತ್ವಂ ನಿಷಿದ್ಧೇರ್ನಾರಕೀಂ ತನುಮ್|\\
ಉಭಾಭ್ಯಾಂ ಪುಣ್ಯಪಾಪಾಭ್ಯಾಂ ಮಾನುಷ್ಯಂ ಲಭತೇಽವಶಃ||
\end{shloka}

(ಪುಣ್ಯಕರ್ಮಗಳಿಂದ ಒಬ್ಬನು ದೈವತ್ವವನ್ನು ಪಡೆಯುತ್ತಾನೆ, ಪಾಪ ಕರ್ಮಗಳಿಂದ ನೀಚ ಜನ್ಮವನ್ನು ಪಡೆಯುತ್ತಾನೆ. ಪುಣ್ಯವೂ ಪಾಪವೂ ಸೇರಿದರೆ ನಿಶ್ಚಯವಾಗಿಯೂ ಮನುಷ್ಯ ಜನ್ಮ ಪಡೆಯುತ್ತಾನೆ.) ವಿಶೇಷವಾಗಿ ಸತ್ಕರ್ಮವನ್ನು ಮಾಡಿದವರು ದೇವತೆಗಳಾಗಿ ಹುಟ್ಟುತ್ತಾರೆ. ಆದರೆ ದೇವತೆಗಳಾಗಿ ಹುಟ್ಟಿಯೂ ಯಾವ ಫಲವೂ ಇಲ್ಲ. ದೇವತೆಗಳಾಗಿ ಹುಟ್ಟಿದರೆ ಮಾಡಬೇಕಾದುದೇ ಇಲ್ಲ. ಏಕೆಂದರೆ ಅಲ್ಲಿ ಇರುವ ಭೋಗಗಳೇ ಇದಕ್ಕೆ ಕಾರಣ. ಇನ್ನೊಂದು ವಿಶೇಷವೆಂದರೆ, ಅಲ್ಲಿರುವವರೆಗೆ ಯಾವುದೋ ಕೆಲವು ದಿನಗಳು ಸುಖವಾಗಿ ಬದುಕಬಹುದು. ಅನಂತರ,

\begin{shloka}
`ಕ್ಷೀಣೇ ಪುಣ್ಯೇ ಮರ್ತ್ಯಲೋಕಂ ವಿಶನ್ತಿ'
\end{shloka}

ಎಂದು ಹೇಳಿದಂತೆ ಪುಣ್ಯವೆಲ್ಲ ಮುಗಿದ ಮೇಲೆ ಕೆಳಗೆ ಇಳಿದು ಬರಬೇಕಾದುದೇ, ಆದರೆ ಪಾಪ ಮಾಡಿದವರಿಗೆ ನರಕ ಸಿಕ್ಕುತ್ತದೆ. ಯಾರಿಗೆ ಸತ್ಯ ಯಾವುದು, ಅಸತ್ಯ ಯಾವುದು ಎನ್ನುವ ವಿವೇಚನೆಯೇ ಇಲ್ಲವೋ, ಯಾರು ಭಗವಂತನನ್ನು ನೆನೆಯುವುದಿಲ್ಲವೋ, ಯಾರು ಇತರರ ಸ್ವತ್ತನ್ನು ಹೇಗೆ ತಾವು ತೆಗೆದುಕೊಳ್ಳುವುದೆಂದು ನೋಡುತ್ತಿರುತ್ತಾರೋ ಅವರು ಅಧರ್ಮ ಮಾಡುವವರಾಗಿದ್ದಾರೆ. ಇಷ್ಟೇ ಅಲ್ಲ, ಕೆಲವರು ವಿಶೇಷವಾದ ಅಧರ್ಮಗಳನ್ನು ಮಾಡುತ್ತಾರೆ. ಕೆಲವನ್ನು ಮಾಡಿದರೆ ಅಧರ್ಮ. ಅದೇ ರೀತಿ ಕೆಲವನ್ನು ಬಿಟ್ಟರೆ ಅಧರ್ಮ. ಹಾಗೆಯೇ ಯಾರಿಗೆ ಸಂಧ್ಯಾವಂದನೆ ಮಾಡಬೇಕೆಂದು ಹೇಳಿದೆಯಲ್ಲ, ಅದನ್ನು ಬಿಟ್ಟರೆ, ಅಧರ್ಮ. ಇಂಥ ಅಧರ್ಮಗಳೆಲ್ಲ ಮಾಡುವವನು ನರಕವನ್ನೇ ಪಡೆಯುತ್ತಾನೆ. ಆದ್ದರಿಂದ, ಮನುಷ್ಯನ ಜನ್ಮವೇ ಎಲ್ಲದಕ್ಕಿಂತಲೂ ಶ್ರೇಷ್ಠವಾದುದೆನ್ನುವುದನ್ನು ನಾವು ತಿಳಿಯಬಹುದು. ದುರ್ಲಭವಾದ ಮನುಷ್ಯ ಜನ್ಮವನ್ನು ಪಡೆದ ನಾವು ನಮ್ಮ ಹಿಂದಿನವರು ತೋರಿಸಿದ ಆಚಾರ ಪರಂಪರೆಯಂತೆ ಜೀವನವನ್ನು ನಡೆಸಬೇಕು.

ಇದಕ್ಕೆ ಸಂಬಂಧಪಟ್ಟಂತೆ ಒಂದು ಘಟನೆಯನ್ನು ನನ್ನ ಹತ್ತಿರ ಪ್ರೇಮಪಾತ್ರರೊಬ್ಬರು ಹೇಳಿದರು. ನಮ್ಮ ದೇಶದ ಒಬ್ಬರು ವಿದೇಶದಲ್ಲಿ ಕೆಲಸ ಮಾಡಿಬಂದರು. ಒಮ್ಮೆ ಆ ದೇಶಕ್ಕೆ ಇವರ ಮಾವನವರೂ, ಅತ್ತೆಯವರೂ ಹೋದರು. ಇವರು ಅವರಿಬ್ಬರನ್ನೂ ಕರೆದುಕೊಂಡು ಹೋಗಲು ಆ ದೇಶದ ವಿಮಾನ ನಿಲ್ದಾಣಕ್ಕೆ  ಬಂದಿದ್ದರು. ವಿಮಾನ ನಿಲ್ದಾಣದಲ್ಲಿ ಇವರನ್ನು ನೋಡಿದ ಇವರ ಸ್ನೇಹಿತರು `ಇದೇನು ವಿಚಿತ್ರವಾಗಿದೆ! ತಾಯಿಯನ್ನೂ, ತಂದೆಯನ್ನೂ ಸರಿಯಾಗಿ ನೋಡಿಕೊಳ್ಳದ ಈ ಕಾಲದಲ್ಲಿ ಅತ್ತೆಯವರೂ, ಮಾವನವರೂ ಬಂದಿದ್ದಾರೆಂದು ಅವರನ್ನು ಕರೆದುಕೊಂಡು ಹೋಗಲು ಬಂದಿದ್ದೀಯಲ್ಲಾ!' ಎಂದರಂತೆ. ಅವರು ಅದಕ್ಕೆ, `ಇದು ಈ ಕಾಲದಲ್ಲಿ ಉಪಯೋಗವಿಲ್ಲದ ಆಚಾರವಾಗಿದೆ. ಒಬ್ಬರಿಗೆ ವಯಸ್ಸಾದರೆ ಅವರು ಯಾರಾದರೂ ಅವರನ್ನು ಅನಾಥಾಶ್ರಮಕ್ಕೆ ಸೇರಿಸಿಬಿಡುವುದೇ ಒಳ್ಳೆಯದು. ಅನಂತರ ಒಂದು ವರ್ಷಕ್ಕೋ ಅಥವಾ ಎರಡು ವರ್ಷಕ್ಕೋ ಒಂದು ಸಾರಿ ಅಲ್ಲಿಗೆ ಹೋಗಿ ಅವರನ್ನು ನೋಡಿ ಬಂದರೆ ಸಾಕು. ಇಷ್ಟು ತೊಂದರೆ ಯಾಕೆ?' ಎಂದು ಹೇಳಿದರಂತೆ. ಇಂಥ ಘಟನೆಯನ್ನು ನೋಡಿದರೆ ನಮ್ಮ ದೇಶದ ಸಂಸ್ಕೃತಿಗೆ ಅದು ಎಷ್ಟು ಬೇರೆಯಾಗಿದೆಯೆಂದು ಸ್ಪಷ್ಟವಾಗಿ ಕಾಣಬಹುದು. ನಮ್ಮ ವೇದ-`ಮಾತೃದೇವೋ ಭವ| ಪಿತೃದೇವೋ ಭವ|' ಎಂದು ಹೇಳಿ ತಾಯಿಯನ್ನೂ ತಂದೆಯನ್ನೂ ದೇವರಂತೆ ಗೌರವಿಸಬೇಕೆಂದು ಹೇಳುತ್ತದೆ.

ಈ ಕಾಲದ ಮನುಷ್ಯರು ಏನೆಂದು ಭಾವಿಸುತ್ತಾರೆ? `ಅಮ್ಮ ಅಪ್ಪ ಬಹಳ ವಯಸ್ಸಾಗಿರುವವರು. ಅವರ ಶರೀರವೆಲ್ಲ ಶಿಥಿಲವಾಗಿ ಅವರು ಯಾವಾಗಲೂ ಕೆಮ್ಮಿಕೊಂಡೇ ಇರುತ್ತಾರೆ. ಇದನ್ನು ನೋಡುವುದಕ್ಕೆ ಅಸಹ್ಯವಾಗಿದೆ. ಅವರು ಚೆನ್ನಾಗಿದ್ದರೆ ನಾನು ಅವರನ್ನು ನನ್ನೊಡನೆ ಇಟ್ಟುಕೊಳ್ಳುತ್ತೇನೆ' ಎಂದು ಭಾವಿಸುತ್ತಾರೆ. ಹಾಗಿರುವವರು ಸ್ವಲ್ಪ ಯೋಚಿಸಬೇಕು. `ನಾವು ಚಿಕ್ಕ ಮಗುವಾಗಿದ್ದಾಗ ತಾಯಿಯ ಮಡಿಲಲ್ಲಿ ಎಷ್ಟು ಸಾರಿ ಮಲ-ಜಲ ವಿಸರ್ಜನೆ ಮಾಡಿ ಅಸಹ್ಯ ಮಾಡಿದ್ದೇವೆ! ಅದೆಲ್ಲವನ್ನು ತಾಯಿ ಸಹಿಸಿಕೊಂಡು ಎಷ್ಟು ಪ್ರೀತಿಯಿಂದ, ಎರಡೂವರೆ ವರ್ಷಗಳವರೆಗೆ ಸೇವೆ ಮಾಡಿದ್ದಾಳೆ! ಒಂದು ಮಗುವೆ ಅಲ್ಲದೇ ಹುಟ್ಟಿದ ನಾಲ್ಕು ಮಕ್ಕಳಿಗೂ ಆ ರೀತಿಯ ಸೇವೆ ಮಾಡಿದುದಕ್ಕಾಗಿ ತಾಯಿ ಕಡಿಮೆ ಪಕ್ಷ ಹತ್ತು ವರ್ಷಗಳಾದರೂ ಕಳೆದಿರುತ್ತಾಳೆ. ಆದ್ದರಿಂದ ನಾವು ನಾಲ್ಕು ಮಂದಿಯೂ ಸೇರಿ ತಲಾ ಎರಡು ವರ್ಷಗಳಾದರೂ ಆಕೆಯನ್ನು ಇಟ್ಟುಕೊಂಡು ಕಾಪಾಡಿದರೆ ಏನು ತಪ್ಪು?' ಎಂದು ನಾವು ಯೋಚಿಸಬೇಕು.

ಒಂದು ಕಡೆ ನೀಲಕಂಠ ದೀಕ್ಷಿತರು, `ಪ್ರಾದುರ್ಭೂತಾ ಸ್ವಯಮಿವ ಹಿತೇ' ಎಂದು ಹೇಳುತ್ತಾರೆ. ಸ್ವಲ್ಪ ವಯಸ್ಸಾಯಿತು, ಶರೀರದಲ್ಲಿ ಬಲವಿದ್ದಾಗ `ನಾನು ತಂದೆ-ತಾಯಿಗೆ ಹುಟ್ಟಿದವನೇ ಅಲ್ಲ. ನಾನು ನೇರವಾಗಿ ಸ್ವರ್ಗ ಲೋಕದಿಂದ ಹಾರಿ ಬಂದೆ' ಎಂದು ಭಾವಿಸುತ್ತಾರೆ. ತನ್ನ ವಿದ್ಯಾಭ್ಯಾಸಕ್ಕೆ ಗುರು ಎಷ್ಟು ಕಷ್ಟಪಟ್ಟಿದ್ದಾರೆಂಬುದನ್ನೂ ತಾಯಿಯೂ, ತಂದೆಯೂ ಎಷ್ಟು ಕಷ್ಟಪಟ್ಟು ಹಣ ವೆಚ್ಚ ಮಾಡಿದ್ದಾರೆಂಬುದನ್ನೂ ಅವರು ಮರೆತುಬಿಟ್ಟು,

\begin{shloka}
`ಪ್ರಾಕ್ತನಾದೃಷ್ಟಲಬ್ಧಪ್ರಜ್ಞೋನ್ಮೇಷಾ\\
ಇವ ಚ ತನಯಾ ನಸ್ಮರನ್ತ್ಯಾತ್ಮನೋಽಪಿ'
\end{shloka}

-ಎಂದು ಶ್ಲೋಕದಲ್ಲಿ ಬರುವಂತೆ `ಹಿಂದಿನ ಜನ್ಮದಲ್ಲಿ ನಾನು ಮಾಡಿದ ಪುಣ್ಯದ ಫಲವಾಗಿ ನಾನು ಬುದ್ಧಿವಂತನಾಗಿದ್ದೇನೆ ಹೊರತು ಈಗ ಯಾರಿಂದಲಾದರೂ ಕಲಿತುಕೊಂಡೇ ನಾನು ಬುದ್ಧಿವಂತನಾಗಬೇಕಾಗಿಲ್ಲ' ಎಂದು ಹೇಳುತ್ತಾರೆ. ನಾವು ಹೀಗೆ ನಡೆದುಕೊಳ್ಳಬಾರದು.

ಒಂದು ಸ್ವಾರಸ್ಯವಾದ ವಿಷಯವೇನೆಂದರೆ ಕೆಲವರು ಸಂನ್ಯಾಸಿಗಳು ತಮ್ಮ ಗುರುಗಳು ಯಾರೆಂದು ಹೇಳುವುದಿಲ್ಲ. ಏಕೆಂದರೆ ಅವರಿಗೆ ಗುರು ಇರಲೇ ಇಲ್ಲ. ಓದು-ಬರೆದುಕೊಂಡ ಕೆಲವರಿಗೆ, ಅವರಿಗೆ ಸಂನ್ಯಾಸ ಕೊಟ್ಟ ಗುರು ಅಷ್ಟು ಓದಿಕೊಂಡವರು ಆಗದೇ ಇದ್ದರೆ ಅವರ ಹೆಸರನ್ನು ಹೇಳಿದರೆ ಗೌರವ ಕಡಿಮೆಯಾಗುತ್ತದೆಂಬ ಭಾವನೆ ಇದೆ. ಆದ್ದರಿಂದ, ಯಾರೋ ಒಬ್ಬ ಪ್ರಸಿದ್ಧ ವ್ಯಕ್ತಿ ಸತ್ತು ಹೋಗಿದ್ದರೆ ಅವರೇ ತಮ್ಮ ಗುರು ಎಂದು ಹೇಳಿಕೊಳ್ಳುತ್ತಾರೆ. ಇದನ್ನೇ ನೀಲಕಂಠ ದೀಕ್ಷಿತರು ಒಂದು ಕಡೆ `ಪ್ರಸಿದ್ಧಶ್ಚ ಮೃತೋ ಗುರುಃ' ಎಂದು ಹೇಳಿದರು. ಏತಕ್ಕೆ ಸತ್ತುಹೋದ ಒಬ್ಬರನ್ನು ಗುರುವೆಂದು ಹೇಳಬಹುದೆಂದರೆ, ಅವರು ಮತ್ತೆ ಬಂದು ತಿರಸ್ಕರಿಸಲಾಗುವುದಿಲ್ಲ. ಆದ್ದರಿಂದ ಹೀಗೆ ಹೇಳಬಹುದೆನ್ನುತ್ತಾರೆ.

ನಾವು ಹೀಗಲ್ಲದೆ ನಮ್ಮ ಆಚಾರ ಪರಂಪರೆಯಂತೆ ನಿಯಮವಾಗಿ ಬಾಳನ್ನು ನಡೆಸುವುದೇ ಶ್ರೇಷ್ಠವಾದುದು. ಮಗು ಆಗಬೇಕೆಂದು ತಪಸ್ಸು ಮಾಡಿ ಮಗನನ್ನು ಪಡೆದ ತಂದೆ-ತಾಯಿಯರು ಎಷ್ಟೋ ಮಂದಿ ಇದ್ದಾರೆ. ತಾನು ಊಟ ಮಾಡದೆ ಇದ್ದರೂ ಮಗನಿಗೆ ತಿನ್ನಿಸಿ ಬೆಳಸಿ ಅವನನ್ನು ದೊಡ್ಡವನನ್ನಾಗಿ ಮಾಡಿ ಸಂತೋಷ ಪಡುವವರೂ ಇದ್ದಾರೆ.

ಕೆಲವರು, `ತಾಯಿ-ತಂದೆ ಕೆಟ್ಟವರಾಗಿದ್ದರೆ ಏನು ಮಾಡುವುದು' ಎನ್ನುತ್ತಾರೆ. ಯಾರು ಮಾಡುವುದು ಸರಿ, ಯಾರು ಮಾಡುವುದು ತಪ್ಪು ಎಂದು ನಾವು ತೀರ್ಮಾನ ಮಾಡಲಾಗುವುದಿಲ್ಲ. ಯಾವುದು ಹೇಗಿದ್ದರೂ ಅವರನ್ನು ಕಾಪಾಡಿದರೆ ಪಾಪವೇನೂ ಇಲ್ಲ. ಆದ್ದರಿಂದ ಕಾಪಾಡುವುದು ಕರ್ತವ್ಯ. ಇದಕ್ಕೆ ವಿರೋಧವಾಗಿ ನಾವು ನಡೆದುಕೊಂಡು ಅಧರ್ಮವನ್ನೇ ಮಾಡುತ್ತಾ ಬಂದರೆ `ವ್ರಜನ್ತೋ ಜನ್ಮನೋ ಜನ್ಮ' ಎಂದು ಹೇಳಿದಂತೆ ಒಂದು ಜನ್ಮದಿಂದ ಮತ್ತೊಂದು ಜನ್ಮವನ್ನು ಪಡೆಯುತ್ತಾ ಹೋಗಬೇಕಾದುದೇ. ಯಾವಾಗಲಾದರೂ ಒಮ್ಮೆ ಮನುಷ್ಯ ಜನ್ಮ ದೊರೆಯಬಹುದು. `ದುರ್ಲಭಂ ಮಾನುಷಂ ದೇಹಮ್' ಎಂದು ಹೇಳಿದಂತೆ ಮನುಷ್ಯ ಜನ್ಮವೆನ್ನುವುದು ದುರ್ಲಭವಾದುದು. ಈ ಮನುಷ್ಯ ಜನ್ಮದಲ್ಲಿಯೂ ನಾವು ಒಳ್ಳೆಯ ವಿಧವಾಗಿ ನಡೆದುಕೊಳ್ಳಲಿಲ್ಲವೆಂದರೆ ಮತ್ತೆ ಆಡಾಗಿಯೋ, ಆಕಳಾಗಿಯೋ ಜನ್ಮವನ್ನು ಪಡೆಯಬೇಕಾದುದೇ. ಆಡಾಗಿಯೋ, ಆಕಳಾಗಿಯೋ ಹುಟ್ಟಿದರೆ ಯಾರಾದರೂ ಹೊಡೆದರೆ ಆ ಕಡೆಯಿಂದ ಈ ಕಡೆಗೆ ಹೋಗಬಹುದು. ಕೊನೆಯಲ್ಲಿ ಯಾರಾದರೂ ಬಲಿ ಕೊಡುವುದಕ್ಕೆ ಕರೆದೊಯ್ದರೆ ಕತ್ತನ್ನು ಕೊಡಬೇಕಾಗುವುದು. ಸರಿ ಕಥೆ ಮುಗಿಯಿತು. ಆದರೆ, ಮನುಷ್ಯನಾಗಿ ಹುಟ್ಟಿದವನು, ತನ್ನ ನಿಜವಾದ ಸ್ವರೂಪವನ್ನು ಅರಿತವನಾಗಿ ಕೃತಾರ್ಥನಾಗಬಹುದು. ಆದ್ದರಿಂದ ನಮಗೆ ಮತ್ತೆ ಜನ್ಮ ಬರದೆ ಇರಲು ಏನೆಲ್ಲ ಪ್ರಯತ್ನಗಳು ಮಾಡಬೇಕೋ ಅವುಗಳೆಲ್ಲವನ್ನು ಈ ಜನ್ಮದಲ್ಲಿಯೇ ಕೈಕೊಳ್ಳಬೇಕು.

ನೀರಿನ ಸುಳಿಯಲ್ಲಿ ಮತ್ತೆ ಮತ್ತೆ ಸಿಕ್ಕಿಕೊಂಡು ಒದ್ದಾಡುತ್ತಿರುವ ಹುಳುವನ್ನು ಕರುಣೆಯಿಂದ ಪಾರು ಮಾಡಿದ ಒಬ್ಬರು, ಅದನ್ನು ನೀರಿನಿಂದ ಹೊರಗೆ ತಂದಮೇಲೆ ಅದರ ಕಷ್ಟಗಳು ದೂರವಾಗುತ್ತವೆ ಎನ್ನುವುದನ್ನು ಸ್ಪಷ್ಟಪಡಿಸುವುದಕ್ಕಾಗಿಯೇ.

`ಪ್ರಾಪ್ಯ ತೀರತರುಚ್ಛಾಯಾಂ ವಿಶ್ರಾಮ್ಯನ್ತಿಯಥಾ ಸುಖಮ್'-ಎನ್ನುತ್ತಾರೆ. ಇದೇ ರೀತಿ ಮತ್ತೆ-ಮತ್ತೆ ಜನ್ಮ ಸಮುದ್ರದಲ್ಲಿ ಸಿಕ್ಕಿಕೊಂಡು ದುಃಖಪಡುವ ಮನುಷ್ಯರಿಗೆ ಅದರಿಂದ ದೂರಾಗುವ ಮಾರ್ಗವೇನೆಂದರೆ, ಆಚಾರ್ಯರ ಹತ್ತಿರ ಹೋಗಿ, ತಮ್ಮ ನಿಜವಾದ ಸ್ವರೂಪವನ್ನು ಅರಿಯುವುದು. ಅದಕ್ಕಾಗಿ ಶಿಷ್ಯನಾದವನು ವೇದಾಂತ ಶ್ರವಣದಂಥವುಗಳನ್ನು ಮಾಡುವುದೇ ಮುಖ್ಯ.

\begin{shloka}
`ಪಂಚಕೋಶವಿವೇಕೇನ ಲಭನ್ತೇ ನಿರ್ವೃತಿಂ ಪರಮ್'
\end{shloka}

ಎಂದು ಹೇಳಿರುವುದರಿಂದ `ಪಂಚಕೋಶ ವಿವೇಕ'ಕ್ಕೆ ಕೊಟ್ಟಿರುವ ಮುಖ್ಯತ್ವವನ್ನು ತಿಳಿಯಬಹುದು. (ಜೀವನು ಬ್ರಹ್ಮವಲ್ಲದೇ ಬೇರೆಯಲ್ಲ ಎನ್ನುವುದನ್ನು ತಿಳಿಯಲು ಬ್ರಹ್ಮವನ್ನು ಜೀವನನ್ನಾಗಿ ತೋರಿಸಲು ಐದು ಕೋಶಗಳ ಸ್ವರೂಪವನ್ನು ತಿಳಿದುಕೊಂಡು ಅವುಗಳಿಂದ ಆತ್ಮನನ್ನು ಬೇರ್ಪಡಿಸಿ ತಿಳಿಯುವುದು `ಪಂಚಕೋಶ ವಿವೇಕ'ವನ್ನು ಸೂಚಿಸುತ್ತದೆ.) ಪ್ರಪಂಚದಲ್ಲೇ ಮುಳುಗಿ, ಬಾಹ್ಯವಸ್ತುಗಳ ಚಿಂತನೆಯಲ್ಲಿಯೇ ಇರುವವನಿಗೆ `ಪಂಚಕೋಶ ವಿವೇಕ' ಉಂಟಾಗುವುದು ಅಷ್ಟು ಸುಲಭವಲ್ಲ. ಆದ್ದರಿಂದಲೇ ಅಂಥವರು ತಮ್ಮ ತಮ್ಮ ಕರ್ಮಗಳನ್ನು ಸರಿಯಾಗಿ ಮಾಡಬೇಕೆಂದು ಹೇಳಿದೆ.

ಈಶಾವಾಸ್ಯೋಪನಿಷತ್ತು

\begin{shloka}
``ಕುರ್ವನ್ನೇವೇಹ ಕರ್ಮಾಣಿ ಜಿಜೀವಿಶೇಚ್ಛತಂ ಸಮಾಃ|"
\end{shloka}

ಎನ್ನುತ್ತದೆ. ಪ್ರಪಂಚದಲ್ಲಿ ಬಾಳಬೇಕು ಎನ್ನುವುದನ್ನೆಲ್ಲಾ ಆಸೆಯಾಗಿಟ್ಟುಕೊಂಡು ಕರ್ಮವನ್ನು ಸರಿಯಾಗಿ ಮಾಡುವುದಕ್ಕಿಂತಲೂ ಬೇರೆ ದಾರಿ ಇಲ್ಲ. ಇದಕ್ಕೆ ವಿರೋಧವಾಗಿ ಯಾರಿಗೆ ಪ್ರಪಂಚ ವಿಷಯಗಳಲ್ಲಿ ಆಸಕ್ತಿ ಇಲ್ಲವೊ ಅಂಥವನು ವೇದಾಂತಕ್ಕೆ ಅಧಿಕಾರಿಯಾಗುತ್ತಾನೆ.

ಇನ್ನು ಕರ್ಮದ ಬಗೆಗಳನ್ನು ನೋಡೋಣ. ಕರ್ಮವನ್ನು ಮೂರು ವಿಧವಾಗಿ ವಿಂಗಡಿಸಬಹುದು, ೧) ಇಷ್ಟ, ೨) ಪೂರ್ತಿ, ೩) ದತ್ತ.

\begin{shloka}
ಅಗ್ನಿಹೋತ್ರಂ ತಪಃ ಸತ್ಯಂ ವೇದಾನಾಂ ಚಾನುಪಾಲನಮ್|\\
ಆತಿಥ್ಯಂ ವೈಶ್ವದೇವಂ ಚ ಇಷ್ಟಮಿತ್ಯಭಿಧೀಯತೇ||
\end{shloka}

ಮೊದಲು ಶ್ಲೋಕದಲ್ಲಿ ಹೇಳಲ್ಪಟ್ಟ `ಇಷ್ಟ'ವೆನ್ನುವುದನ್ನು ನೋಡೋಣ. ಮೊದಲು ಅಗ್ನಿಹೋತ್ರವೆನ್ನುವುದು ಹೇಳಲಾಗಿದೆ. ಅಗ್ನಿಹೋತ್ರವನ್ನು ಯೋಗ್ಯತೆ ಇರುವವರು ನಿತ್ಯವೂ ಅನುಷ್ಠಾನ ಮಾಡಬೇಕೆಂದು ವೇದದಲ್ಲಿ ಹೇಳಿದೆ.

`ತಪಃ'-ಏಕಾದಶಿಯಂದು ಉಪವಾಸವಿರುತ್ತೇವೆ. ಅದೊಂದು ತಪಸ್ಸಾಗುತ್ತದೆ. ಹಲವು ವ್ರತಗಳನ್ನು ಮಾಡುತ್ತೇವೆ. ಅವುಗಳೂ ತಪಸ್ಸೆ, ಬ್ರಹ್ಮ ಯಜ್ಞ ಒಂದು ತಪಸ್ಸು.

\begin{shloka}
`ಸ್ವಾಧ್ಯಾಯ ಪ್ರವಚನಾಭ್ಯಾಂ ನ ಪ್ರಮದಿತವ್ಯಮ್'
\end{shloka}

ವೇದಾಧ್ಯಯನ ಮಾಡುವುದೂ, ಅದನ್ನು ಮಾಡಿಸುವುದೂ ಬಿಡಬಾರದು.

\begin{shloka}
`ತದ್ಧಿ ತಪಃ ತದ್ಧಿತಪಃ'
\end{shloka}

(ಅದೇ ತಪಸ್ಸು, ಅದೇ ತಪಸ್ಸು) ಎಂದು ಹೇಳಿರುವುದನ್ನು ನಾವು ನೋಡಬಹುದು.

ಅನಂತರ `ಸತ್ಯ' ಎನ್ನುವುದು ಹೇಳಲ್ಪಟ್ಟಿದೆ. ಸತ್ಯ ಹೇಳುವುದು ಎನ್ನುವುದೇ ದೊಡ್ಡ ತಪಸ್ಸು. ಒಂದು ವಿಷಯವನ್ನು ನಾವು ಯಾವ ರೀತಿಯಲ್ಲಿ ತಿಳಿದುಕೊಂಡಿದ್ದೇವೋ ಅದೇ ರೀತಿಯಲ್ಲಿ ಹೊರಗೆ ಹೇಳುವುದಕ್ಕೆ `ಸತ್ಯ'ವೆಂದು ಹೆಸರು. ಬೇರೆ ರೀತಿಯಲ್ಲಿ ಮಾತನಾಡಿದರೆ ಅದು ಸುಳ್ಳು ಆಗುತ್ತದೆ. ನಾವು ಸತ್ಯವನ್ನು ಹೇಳುವಾಗಲೂ ಜಾಗ್ರತೆಯಾಗಿರಬೇಕು. ನಾವು ಸತ್ಯವನ್ನು ನುಡಿದು ಅದರಿಂದ ಯಾರಿಗಾದರೂ ಅಪಕಾರ ಮಾಡುತ್ತೇವೆಂದರೆ

\begin{shloka}
`ಸತ್ಯಾನ್ ಮೌನಂ ವಿಶಿಷ್ಯತೇ'
\end{shloka}

ಎಂದು ಹೇಳುವುದರಿಂದ ಆ ಸಮಯದಲ್ಲಿ ಸತ್ಯಕ್ಕಿಂತಲೂ ಮೌನವನ್ನಾಚರಿಸುವುದೇ ಮೇಲು. ಅದಕ್ಕಾಗಿ ಆಗ ನಾವು ಸುಳ್ಳು ಹೇಳಬೇಕಾಗಿಲ್ಲ. ಮೌನವಾಗಿದ್ದರೇ ಸಾಕು. ಆದ್ದರಿಂದಲೇ-

\begin{shloka}
`ಸತ್ಯಂ ಬ್ರೂಯಾತ್ ಪ್ರಿಯಂ ಬ್ರೂಯಾನ್ನಬ್ರೂಯಾತ್ ಸತ್ಯಮಪ್ರಿಯಮ್|\\
ಪ್ರಿಯಂ ಚ ನಾನೃತಂ ಬ್ರೂಯಾದೇಷ ಧರ್ಮಃ ಸನಾತನಃ||
\end{shloka}

(ಸತ್ಯವನ್ನು ಹೇಳು. ಪ್ರಿಯವನ್ನು ಹೇಳು. ಪ್ರಿಯವಲ್ಲದ ಸತ್ಯವನ್ನು ಹೇಳಬೇಡ. ಪ್ರಿಯವಾದ ಅಸತ್ಯವನ್ನೂ ಹೇಳಬೇಡ. ಇದೇ ಸನಾತನ ಧರ್ಮ.) ಎಂದು ಹೇಗೆ ಸತ್ಯವನ್ನು ಹೇಳಬೇಕೆನ್ನುವುದಕ್ಕೆ ಲಕ್ಷಣಗಳನ್ನು ನಮ್ಮ ಹಿಂದಿನವರು ಹೇಳಿದ್ದಾರೆ.

ಅದಾದ ಮೇಲೆ, `ವೇದಾನಾಂ ಚ ಅನುಪಾಲನಮ್'-ವೇದವೆನ್ನುವುದು ನಮ್ಮನ್ನು ತಾಯಿಯಂತೆ ಕಾಪಾಡುವುದು. ಆ ವೇದವನ್ನೇ ನಾವು ಇಲ್ಲವೆಂದು ಹೇಳಿದರೆ ಮೂಲವಿಲ್ಲದೆ ಹೇಗೆ ಎಲೆಯೋ, ಕೊಂಬೆಯೋ ಇರಲು ಸಾಧ್ಯವಿಲ್ಲವೋ, ಹಾಗೆಯೇ ಧರ್ಮವೆನ್ನುವುದೇ ಇಲ್ಲದೇ ಆಗುತ್ತದೆ.

\begin{shloka}
`ಆತಿಥ್ಯಂ ವೈಶ್ವದೇವಂ ಚ'
\end{shloka}

ನಮ್ಮ ಹತ್ತಿರ ಯಾರು ಬಂದರೂ ಕೂಡ, ಒಡನೆ ಅವರಿಗೆ `ಇಲ್ಲ' ಎಂದು ಹೇಳಿ ಕಳುಹಿಸಬಾರದು. `ಪತ್ರಂ ಪುಷ್ಪಂ ಫಲಂ ತೋಯಂ' ಎಂದು ಗೀತೆಯಲ್ಲಿ ಭಗವಂತನು ಹೇಳಿದಂತೆ ನಮ್ಮ ಹತ್ತಿರ ಇರುವುದನ್ನು ಬಂದವರಿಗೆ ಕೊಡಬೇಕು. ಅನಂತರ `ವೈಶ್ವದೇವ' ಎಂದು ಹೇಳಲ್ಪಟ್ಟಿದೆ.

ಪ್ರಪಂಚದಲ್ಲಿ ಬಾಳಲು ಮನುಷ್ಯರಿಗೆ ಮಾತ್ರವಲ್ಲದೆ ಎಷ್ಟೋ ಜೀವರಾಶಿಗಳಿಗೆ ಕೂಡ ಭಗವಂತನು ಅವಕಾಶ ಕೊಟ್ಟಿದ್ದಾನೆ. ಆದ್ದರಿಂದ ಅವುಗಳು ಕೂಡ ಬಾಳಬೇಕಾಗಿರುವುದರಿಂದ ಪಶುಗಳಿಗೆ ಕೂಡ ನಾವು ತಿನ್ನುವುದರಲ್ಲಿ ಸ್ವಲ್ಪವಾದರೂ ಕೊಡಬೇಕು. `ವೈಶ್ವದೇವ' ಮಾಡುವವರು ಕೊನೆಯಲ್ಲಿ `ಕಾಕಬಲಿ' ಕೊಡುತ್ತಾರೆ. ಹಸುಗಳಿಗೂ ಅವರು ಕೊಡುತ್ತಾರೆ. ಅನಂತರ ಯಾರಾದರೂ ಬಂದಿದ್ದಾರೆಯೇ ಎಂದು ನೋಡುತ್ತಾರೆ. ಅವರಿಗೂ ಆಹಾರ ಕೊಡುತ್ತಾರೆ.

ಮೇಲೆ ಹೇಳಿದ ಎಲ್ಲವೂ `ಇಷ್ಟ'ವೆನ್ನುವ ವಿಭಾಗದಲ್ಲಿ ಸೇರುತ್ತವೆ.

\begin{shloka}
ವಾಪೀ ಕೂಪತಟಾಕಾದಿ ದೇವತಾಯತನಾನಿ ಚ|\\
ಅನ್ನಪ್ರದಾನಮಾರಾಮಃ ಪೂರ್ತಮಿತ್ಯಭೀಧೀಯತೇ||
\end{shloka}

ಮೇಲೆ ಹೇಳಿದ ಶ್ಲೋಕದಲ್ಲಿ `ಪೂರ್ತ'ವೆನ್ನುವುದು ತಿಳಿಸಲಾಗಿದೆ. ದೊಡ್ಡ ಬಾವಿ ತೋಡಿಸುವುದು, ದೇವರಿಗೆ ದೇವಾಲಯ ನಿರ್ಮಾಣ ಮಾಡಿಸುವುದು, ಅನ್ನದಾನ ಮಾಡುವುದು, ಛತ್ರಗಳನ್ನು ಏರ್ಪಡಿಸುವುದು ಇವುಗಳೆಲ್ಲ ಪೂರ್ತವೆನ್ನುವ ವಿಭಾಗದಲ್ಲಿ ಬರುತ್ತವೆ.

ನಾವು ಇವುಗಳನ್ನೆಲ್ಲ ಒಪ್ಪುತ್ತೇವೆ. ಆದರೆ, ಮೊದಲು `ಇಷ್ಟ' ಎನ್ನುವ ವಿಭಾಗದಲ್ಲಿ ಹೇಳಿದವುಗಳನ್ನು ಒಪ್ಪಿಕೊಳ್ಳಲು ಹಿಂಜರಿಯುತ್ತೇವೆ. ಎರಡನ್ನೂ ಹೇಳಿದವರು ಒಬ್ಬರೇ. ಆದ್ದರಿಂದಲೇ ನಾವು ಇವುಗಳನ್ನು ಕುರಿತು ವಿವೇಚನೆ ಮಾಡಿ ಒಂದು ನಿರ್ಣಯಕ್ಕೆ ಬರಬೇಕು. ವಿವೇಚನೆ ಮಾಡುವುದಕ್ಕೆ ಒಂದು ಮೂಲವೆನ್ನುವುದು ಬೇಕು. ಕಾನೂನು ಪುಸ್ತಕವನ್ನು ಇಟ್ಟುಕೊಂಡು ಕದಿಯುವುದು ಸರಿಯೇ ತಪ್ಪೇ ಎಂದು ತೀರ್ಮಾನಿಸಬಹುದು. ಕಾನೂನು ಪುಸ್ತಕವೇ ಇಲ್ಲದೆ, ಕದಿಯುವುದು ಸರಿಯೇ ಎಂದು ಯಾರಾದರೂ ಕೇಳಿದರೆ, ಒಬ್ಬನು ಅದನ್ನು `ಸರಿ' ಎಂದು ಒಪ್ಪುತ್ತಾನೆ. ಮತ್ತೊಬ್ಬನು ಅದನ್ನು `ತಪ್ಪು' ಎನ್ನುತ್ತಾನೆ. ಆದ್ದರಿಂದ ಮೂಲವಿಲ್ಲದೆ ನಾವು ವಿವೇಚನೆ ಮಾಡಬಾರದು. ಶ್ಲೋಕಗಳನ್ನು ಬರೆದವರು ಮೂಲವನ್ನು ಇಟ್ಟುಕೊಂಡೇ ಹೇಳಿದ್ದಾರೆ.

ಅನಂತರ `ದತ್ತ'ವೆನ್ನುವ ವಿಭಾಗವನ್ನು ಕುರಿತು ಹೇಳಿದ್ದಾರೆ.

\begin{shloka}
ಶರಣಾಗತ ಸಂತರಣಂ ಭೂತಾನಾಂ ಚಾಪ್ಯಹಿಂಸನಮ್|\\
ಬರ್ಹಿವೇದಿ ಚ ಯದ್ದಾನಂ ದತ್ತಮಿತ್ಯಬಿಧೀಯತೇ||
\end{shloka}

(ಶರಣಾಗತನಾದವನನ್ನು ಕಾಪಾಡುವುದು, ಪ್ರಾಣಿಗಳ ವಿಷಯದಲ್ಲಿ ಅಹಿಂಸೆ, ಯಾಗ ಸಂಬಂಧವಿಲ್ಲದೆ ದಾನಕೊಡುವುದು ಇಂಥವು ದತ್ತವೆನ್ನಲ್ಪಡುತ್ತವೆ.) ಶಾಸ್ತ್ರದಲ್ಲಿ ಹೇಳಿದ ಯಾವುದಾದರೂ ಕರ್ಮಗಳನ್ನು ಮಾಡಿದ ಮೇಲೆ ನಾವು ದಕ್ಷಿಣೆ ಕೊಡುತ್ತೇವೆ. ಈ ಕರ್ಮ ಇಷ್ಟವೆನ್ನುವ ವಿಭಾಗದಲ್ಲೇ ಸೇರುತ್ತದೆ. ಒಂದು ಮನೆಯನ್ನು ಕಟ್ಟಿಕೊಡುವುದು, ಹಸುವನ್ನು ದಾನವಾಗಿ ಕೊಡುವುದು ಇವುಗಳೆಲ್ಲ ಶಾಸ್ತ್ರ ರೀತಿಯಾಗಿ ಮಾಡಲ್ಪಡುವ ಕರ್ಮಗಳ ಭಾಗವಾದ್ದರಿಂದ ಇವುಗಳೆಲ್ಲವೂ `ಇಷ್ಟ'ವೆನ್ನುವ ವಿಭಾಗದಲ್ಲೇ ಸೇರುತ್ತವೆ.

ಮೊದಲು `ಶರಣಾಗತ ಸಂತರಣಂ'-ರಾತ್ರಿ ಸಮಯದಲ್ಲಿ ಯಾರಾದರೂ ನಮ್ಮ ಮನೆಗೆ ಬಂದು `ಸ್ವಾಮಿ, ಇರುವುದಕ್ಕೆ ಜಾಗ ಕೊಡಿ' ಎಂದು ಕೇಳಿದರೆ, `ನ ಕಂಚನ ವಸತೌ ಪ್ರತ್ಯಾಚಕ್ಷೀತ' (ಯಾವ ಅತಿಥಿಗೂ ಅನಾದರಣೆ ಮಾಡಬಾರದು) ಎಂದು ಹೇಳಿದಂತೆ, ಅವನನ್ನು ಕಳುಹಿಸದೆ ಅವನಿಗೆ ಅಭಯವನ್ನು ನೀಡಬೇಕು. 

ಅನಂತರ,

`ಭೂತಾನಾಂ ಚಾಪ್ಯಹಿಂಸನಮ್'-ಅನಾವಶ್ಯಕವಾಗಿ ಪ್ರಾಣಿಗಳಿಗೆ ಕಷ್ಟ ಕೊಡಬಾರದು. ಕೆಲವರು ತಮಾಷೆಗಾಗಿ ಪ್ರಾಣಿಗಳನ್ನು ಹಿಂಸಿಸುತ್ತಾರೆ. ಇದು ತಪ್ಪು. ಹಾಗಾದರೆ ಬೇಟೆಯಾಡುವುದನ್ನು ಧರ್ಮವೆಂದು ಹೇಳಿದ್ದಾರೆಂದು ಕೆಲವರು ಕೇಳಬಹುದು. ಕ್ರೂರ ಮೃಗಗಳು ಬಂದು ಜನರಿಗೆ ಹಿಂಸೆ ಮಾಡಿದಾಗ ಬೇಟೆಯಾಡುವುದು ಧರ್ಮ. ಹಾಗಿಲ್ಲದೆ, ಒಬ್ಬನು ಬೇಕೆಂದೇ ಕಾಡಿಗೆ ಹೋಗಿ ಬೇಟೆಯಾಡುತ್ತಲಿದ್ದರೆ ಅದು ತಪ್ಪಾಗುತ್ತದೆ. ಭಗವಂತನು ಸೃಷ್ಟಿ ಮಾಡಿದ ಪ್ರಾಣಿಗಳು ಕೊಲ್ಲುವುದಕ್ಕಾಗಿಯೇ?

ಅನಂತರ `ಬರ್ಹಿವೇದಿ ಚ ಯದ್ದಾನಂ'-ಈಗ ದಾನವನ್ನು ಕುರಿತು ನೋಡೋಣ. ಶಂಕರಭಗವತ್ಪಾದರು ದಾನವನ್ನು ಕುರಿತು ಹೀಗೆ ಹೇಳುತ್ತಾರೆ. ನಾವು ಒಂದು ಹತ್ತು ರೂಪಾಯಿ ನೋಟನ್ನು ಕೈಯಲ್ಲಿಟ್ಟುಕೊಂಡು ಹೋಗುವಾಗ ಎಲ್ಲೋ ಅದು ಬಿದ್ದುಹೋದರೆ ಹತ್ತು ರೂಪಾಯಿ ಅನ್ಯಾಯವಾಗಿ ಕಾಣದೆ ಹೋಯಿತಲ್ಲಾ ಎಂದು ಅಳುತ್ತೇವೆ. ಆದರೆ ಹತ್ತು ರೂಪಾಯಿ ನೋಟು ತೆಗೆದುಕೊಂಡು ಹೋಗುವಾಗ ನಮ್ಮ ಹತ್ತಿರಕ್ಕೆ ಒಬ್ಬ ಬಡವನು ಬಂದು, `ಸ್ವಾಮಿ' ನನ್ನ ಹತ್ತಿರ ಹಣ ಇಲ್ಲದ ಕಾರಣ ಪರೀಕ್ಷೆಗೆ ಹೋಗಲಿಲ್ಲ. `ಹತ್ತು ರೂಪಾಯಿಗಳಿದ್ದರೆ ನಾನು ಪರೀಕ್ಷೆಗೆ ಹೋಗಿ ಅದರಲ್ಲಿ ತೇರ್ಗಡೆ ಪಡೆಯಬಹುದು' ಎಂದರೆ, ನಾವು ಹತ್ತು ರೂಪಾಯಿ ಕೊಟ್ಟುಬಿಡುತ್ತೇವೆ. ಆದರೆ, ಈ ಘಟನೆಯಿಂದಾಗಿ ಮನೆಗೆ ಬಂದು ಅಳುವುದಿಲ್ಲ. `ಇಂದು ಒಂದು ಒಳ್ಳೆಯ ಕೆಲಸ ಮಾಡಿದೆ' ಎನ್ನುವ ಸಂತೋಷದಿಂದ ಇರುತ್ತೇವೆ. ಹಣ ಕಳೆದು ಹೋದಾಗ ವ್ಯಥೆ ಪಡುವ ನಾವು, ಹಣವನ್ನು ಒಂದು ಒಳ್ಳೆಯ ಕೆಲಸಕ್ಕೆ ಖರ್ಚು ಮಾಡಿದಾಗ ಸಂತೋಷಪಡುತ್ತೇವೆ. ಆದ್ದರಿಂದ ನಮ್ಮ ಸಂಪತ್ತೆಲ್ಲ `ಕೊಟ್ಟುಬಿಟ್ಟೆ' ಎಂದು ನಾವು ಹೇಳಿಕೊಳ್ಳುವಂತೆ ಖರ್ಚು ಮಾಡಬೇಕು. `ಹಣವನ್ನು ಹಾಳು ಮಾಡಿದೆ' ಎಂದು ಹೇಳಿಕೊಳ್ಳುವಂತೆ ಇರಬಾರದು. ಆದ್ದರಿಂದ, ಹಾಗೆ ಮಾಡಲ್ಪಡುವ ದಾನ ಮುಂತಾದವು `ದತ್ತ'ವೆನ್ನುವ ವಿಭಾಗದಲ್ಲಿ ಬರುತ್ತವೆ.

ಕರ್ಮಗಳನ್ನು ಬೇರೆ ಮೂರು ವಿಧವಾಗಿಯೂ ವಿಂಗಡಿಸಬಹುದು. ಅವುಗಳನ್ನು-೧) ನಿತ್ಯ, ೨) ನೈಮಿತ್ತಿಕ ಮತ್ತು ೩) ಕಾಮ್ಯ ಎನ್ನುವುವು. ಸರಿಯಾದ ಕಾಲಕ್ಕೆ ಮತ್ತೆ ಮತ್ತೆ ಬರುವ ಕರ್ಮ ನಿತ್ಯ ಕರ್ಮವೆನ್ನಲ್ಪಡುತ್ತದೆ. ನಿತ್ಯಕರ್ಮವನ್ನು ನಾವು ಮಾಡದೇ ಇದ್ದರೆ ನಮಗೆ ಪಾಪ ಬರುತ್ತದೆ. ಅನಂತರ ಹೇಳಲ್ಪಟ್ಟಿದ್ದು ನೈಮಿತ್ತಿಕ ಕರ್ಮ. ನೈಮಿತ್ತಿಕ ಕರ್ಮದ ಕಾಲ ಒಂದು ನಿಶ್ಚಿತವಾದ ನಿಯಮಕ್ಕೆ ಸೇರಿದುದು. ಉದಾಹರಣೆಗೆ, ಗ್ರಹಣ ಕಾಲದಲ್ಲಿ ಸ್ನಾನ ಮಾಡಬೇಕೆನ್ನುವುದು ಶಾಸ್ತ್ರದಲ್ಲಿ ಹೇಳಲಾದ ನಿಯಮ. ಆದರೆ, ಚಂದ್ರಗ್ರಹಣ ಒಂದೊಂದು ಪೌರ್ಣಮಿಯಂದೂ ಬರುವುದಿಲ್ಲ. ಅಧಿಕಾರಿಯಾದವರು, ಸೂರ್ಯೋದಯ ಕಾಲದಲ್ಲೂ, ಸೂರ್ಯಾಸ್ತ ಕಾಲದಲ್ಲೂ ಸಂಧ್ಯಾವಂದನೆ ಮಾಡಬೇಕೆಂದು ಹೇಳಿದೆ. ಅಲ್ಲದೆ, ಕೆಲವರಿಗೆ ಅಮಾವಾಸ್ಯೆ ದಿನ ತರ್ಪಣ ಮಾಡಬೇಕೆಂದು ವಿಧಿಸಲಾಗಿದೆ. ಇವೆಲ್ಲವೂ ನಿತ್ಯಕರ್ಮಗಳು. ಏಕೆಂದರೆ ಸೂರ್ಯೋದಯ, ಸೂರ್ಯಾಸ್ತ ಪ್ರತಿದಿನವೂ ನಡೆಯುವ ಘಟನೆಗಳೆಂದು ನಾವು ತಿಳಿದಿದ್ದೇವೆ. ಆದರೆ, ಗ್ರಹಣವೆನ್ನುವುದು ಹೇಗೆಂದರೆ, ನಮ್ಮ ದೇಶದಲ್ಲಿ ಗ್ರಹಣ ಕಾಣಿಸಿದರೇನೇ ನಾವು ಸ್ನಾನ ಮಾಡಬೇಕು. ಹಾಗಿಲ್ಲದಿದ್ದರೆ ಸ್ನಾನ ಮಾಡಬೇಕಾಗಿಲ್ಲ. ಊರಿನಲ್ಲಿ ಎಷ್ಟೋ ಜನ ದುಃಖದಿಂದ ಕೂಡಿದ್ದರೂ ನಾವು ಯಾರನ್ನಾದರೂ ಒಬ್ಬರನ್ನು ಎದುರಿಗೆ ನೋಡಿದರೆ ಅವರಿಗೆ ಮಾತ್ರ ಸಹಾಯ ಮಾಡಬೇಕೆನಿಸುತ್ತದೆ. ಎಷ್ಟೋ ಜನ ದುಃಖಪಡುವಾಗ ಅವರಿಗೆ ಮಾತ್ರ ಏಕೆ ಸಹಾಯ ಮಾಡಬೇಕೆಂದು ಯಾರಾದರೂ ಕೇಳಿದರೆ ಅದು ಮೂಢತನವಾಗುತ್ತದೆ. ಅದೇ ರೀತಿ ಗ್ರಹಣ ನಮಗೆ ಗೋಚರವಾದರೆ ಮಾತ್ರ ಸ್ನಾನ ಮಾಡುತ್ತೇವೆ. ಇದು ನೈಮಿತ್ತಿಕ ಕರ್ಮಕ್ಕೆ ಸೇರುತ್ತದೆ.

ಅನಂತರ ಕಾಮ್ಯ ಕರ್ಮ. ಯಾವ ಕರ್ಮದಲ್ಲಿ ಒಬ್ಬನು ಆಸೆಯಿಂದ ಮಾಡುತ್ತಾನೋ ಅದು ಕಾಮ್ಯಕರ್ಮ. ಮಕ್ಕಳಿಲ್ಲ ಎನ್ನುವುದಕ್ಕಾಗಿಯಾಗಲಿ, ಆಯಸ್ಸು ಹೆಚ್ಚಾಗಬೇಕು ಎನ್ನುವುದಕ್ಕಾಗಲಿ ಕರ್ಮಗಳನ್ನು ಮಾಡಿದರೆ ಅವುಗಳು ಕಾಮ್ಯಕರ್ಮಗಳೆನ್ನಲ್ಪಡುತ್ತವೆ. ನಿತ್ಯ, ನೈಮಿತ್ತಿಕ ಕರ್ಮಗಳನ್ನು ಮಾಡಿದರೆ ಮನಸ್ಸು ಶುದ್ಧಿಯಾಗುತ್ತದೆ. ಕಾಮ್ಯ ಕರ್ಮ ಮಾಡಿದರೆ, ಆ ಕಾಮ್ಯವನ್ನು ಅನುಭವಿಸಲು ಈ ಜನ್ಮದಲ್ಲಿ ಸಾಧ್ಯವಾಗದೇ ಇದ್ದರೆ,

\begin{shloka}
`ನಾಭುಕ್ತಂ ಕ್ಷೀಯತೇ ಕರ್ಮ ಕಲ್ಪಕೋಟಿಶತೈರಪಿ'
\end{shloka}

(ಭೋಗವನ್ನು ಕೊಡದೆ ಕರ್ಮ ನೂರು ಕೋಟಿ ಕಲ್ಪಗಳಿಗೂ ನಾಶ ಹೊಂದುವುದಿಲ್ಲ.) ಎಂದು ಹೇಳಿದಂತೆ ಅದು ಮುಂದಿನ ಜನ್ಮಕ್ಕೆ ಕಾರಣವಾಗುತ್ತದೆ. ಆದ್ದರಿಂದ ಹೀಗೆ ಒಂದು ಜನ್ಮದಿಂದ ಮತ್ತೊಂದು ಜನ್ಮ ಎಂದು ಇದು ಹಿಂಬಾಲಿಸುತ್ತಿರುತ್ತದೆ. ಇದು ನಮಗೆ ಒಂದು ಪ್ರಶ್ನೆಯಾಗಿ ಉಳಿಯುತ್ತದೆ. ಮನುಷ್ಯನು ಒಳ್ಳೆಯದಾಗಲಿ ಕೆಟ್ಟದಾಗಲಿ ಯಾವುದಾದರೂ ಒಂದು ಕರ್ಮವನ್ನು ಮಾಡುತ್ತಲೇ ಇದ್ದಾನೆ. ಕರ್ಮವನ್ನು ಮಾಡುತ್ತಲೇ ಇದ್ದರೆ ಜನ್ಮದಿಂದ ಜನ್ಮ ಮತ್ತೆ ಮತ್ತೆ ಪಡೆಯುವುದನ್ನು ತಡೆಯಲಾಗುವುದಿಲ್ಲ. ಅದಕ್ಕಾಗಿ, ಕರ್ಮ ಮಾಡದೆ ಇರಲೂ ಸಾಧ್ಯವಿಲ್ಲ. ಆದ್ದರಿಂದ ಜನ್ಮ ಸಮುದ್ರದಲ್ಲಿ ಮತ್ತೆ ಸಿಕ್ಕಿಕೊಳ್ಳದೆ ಇರುವುದಕ್ಕೆ ಯಾವುದು ದಾರಿ ಎನ್ನುವುದು ನಮಗೆಲ್ಲ ಪ್ರಶ್ನೆಯಾಗಿದೆ.

\begin{shloka}
ಯೋಗಸ್ಥಃ ಕುರು ಕರ್ಮಾಣಿ ಸಂಗಂ ತ್ಯಕ್ತ್ವಾ ಧನಂಜಯ|\\
ಸಿದ್ಧ್ಯಸಿದ್ಧ್ಯೋಃ ಸಮೋ ಭೂತ್ವಾ ಸಮತ್ವಂ ಯೋಗ ಉಚ್ಯತೇ||
\end{shloka}

(ಎಲೈ ಧನಂಜಯನೇ! ಹಠವಿನ್ನು ಬಿಟ್ಟು, ಜಯಾಪಜಯಗಳಲ್ಲಿ ಸಮಭಾವನೆಯುಳ್ಳವನಾಗಿ ಯೋಗದಿಂದ ಕರ್ಮಗಳನ್ನು ಮಾಡುತ್ತಾ ಬಾ. ಸಮ ಭಾವದಿಂದಿರುವುದು ಯೋಗವೆನ್ನಲ್ಪಡುತ್ತದೆ.) ಎಂದಿದ್ದಾನೆ. ಕರ್ಮವೆನ್ನುವುದು ಬಂಧಕ, ಮೋಚಕ ಎಂದು ಎರಡು ವಿಧ. ಬಂಧಕವೆನ್ನುವುದು ಯಾವ ಕರ್ಮ ನಮ್ಮ ಮುಂದಿನ ಜನ್ಮಕ್ಕೆ ಕಾರಣವಾಗಿರುವುದೋ, ಅದಾಗುತ್ತದೆ. ಭಗವಂತನು `ಯೋಗಃ ಕರ್ಮಸು ಕೌಶಲಂ' ಎನ್ನುತ್ತಾನೆ. ನಾವು `ಯೋಗದಲ್ಲಿ' ಇದ್ದು ಕರ್ಮಗಳನ್ನು ಮಾಡಬೇಕೆಂದು ಅವನು ಹೇಳಿದುದು. ಯೋಗದಲ್ಲಿದ್ದು ಕರ್ಮಗಳನ್ನು ಮಾಡಬೇಕಾದರೆ ಯೋಗದಂಡವನ್ನು ಹಿಡಿದುಕೊಳ್ಳಬೇಕೆಂದೋ, ಕಾಷಾಯ ವಸ್ತ್ರವನ್ನು ಹಾಕಿಕೊಳ್ಳಬೇಕೆಂದೋ ಅರ್ಥವಲ್ಲ.

`ಸಿದ್ಧ್ಯಸಿದ್ಧ್ಯೋಃ ಸಮೋ ಭೂತ್ವಾ ಸಮತ್ವಂ ಯೋಗ ಉಚ್ಯತೇ' - ಎಂದು ಭಗವಂತನು ಹೇಳಿದುದು.

ನಾವು ಯಾವ ಕರ್ಮವನ್ನು ಮಾಡಿದರೂ ಅದು ಭಗವದರ್ಪಣೆ ಎಂದು ಭಾವಿಸಿ ಮಾಡಬೇಕು. ಉದಾಹರಣೆಗೆ, ನಾವು ಮಾಡುವ ಸಂಧ್ಯಾವಂದನೆ ಯಾವುದಕ್ಕೆಂದರೆ, ಅದು ಭಗವದರ್ಪಣೆಯೇ ಆಗುತ್ತದೆ. ಗೃಹಸ್ಥನು ಮಕ್ಕಳನ್ನು ಕಾಪಾಡುವಾಗ, `ಮಕ್ಕಳ ಮೇಲೆ ನನಗೆ ಬಹಳ ಪ್ರೀತಿ, ಆದ್ದರಿಂದ ಕಾಪಾಡುತ್ತೇನೆ' ಎನ್ನುವ ಭಾವನೆ ಅವನಿಗೆ ಇರಬಾರದು. ಅದಕ್ಕೆ ಬದಲಾಗಿ `ಮಕ್ಕಳನ್ನು ದೇವರು ನನಗೆ ಕೊಟ್ಟಿದ್ದಾನೆ. ಅವರನ್ನು ಕಾಪಾಡುವುದು ನನ್ನ ಕರ್ತವ್ಯ. ಇದು ಕೂಡ ಒಂದು ಭಗವದರ್ಚನೆ. ನಾನು ನನ್ನ ಈ ಕರ್ತವ್ಯವನ್ನು ಸರಿಯಾಗಿ ಮಾಡದೇ ಹೋದರೆ ಭಗವಂತನು ಅಪ್ರಸನ್ನನಾಗುತ್ತಾನೆ. ಕಾಪಾಡುವುದನ್ನು ಸರಿಯಾಗಿ ಮಾಡಿದರೇನೇ ಭಗವಂತನು ಪ್ರಸನ್ನನಾಗುತ್ತಾನೆ' ಎನ್ನುವ ಭಾವನೆ ಬರಬೇಕು. 

ಭಗವಂತನು ಹೇಳಿದಂತೆ ನಾವು ಮಾಡುವ ಕೆಲಸದಲ್ಲಿ ನಮಗೆ ಫಲ ಸಿಕ್ಕಬೇಕೆನ್ನುವ ಅಪೇಕ್ಷೆಯೇ ಇರಬಾರದು. ಹಾಗೆ ನಾವು ಫಲವನ್ನು ಎದುರು ನೋಡದೆ ಕರ್ಮವನ್ನು ಮಾಡುವುದನ್ನೇ ಭಗವಂತನು `ಯೋಗಃ' ಎಂದಿದ್ದಾನೆ. ಆದ್ದರಿಂದ ಬಂಧಕವಾಗಿರುವ ಕರ್ಮಗಳನ್ನೂ ಮೋಚಕವಾಗಿರುವ ಕರ್ಮಗಳಾಗಿ ಬದಲಾಯಿಸಬಹುದು. ನಾವು ಮೋಚಕವಾಗಿರುವ ಕರ್ಮಗಳನ್ನು ಮಾಡಬೇಕೇ ಹೊರತು ಆಸೆಯಿಂದ ಮಾಡಬಾರದು. ಹಾಗಾದರೆ ಕಾಮ್ಯಕರ್ಮಗಳನ್ನು ಕುರಿತು ಶಾಸ್ತ್ರದಲ್ಲಿ ಏಕೆ ಹೇಳಿದೆಯೆಂದು ಕೆಲವರಿಗೆ ಸಂದೇಹ ಉಂಟಾಗಬಹುದು. ಇದಕ್ಕೆ  ಉತ್ತರ ಹೀಗೆ ಹೇಳಬಹುದು. ಒಬ್ಬನು ಜ್ಯೋತಿಷ್ಟೋಮ (ಸ್ವರ್ಗವನ್ನು ಪಡೆಯಬೇಕೆಂದು ಆಸೆಯಿಂದ) ಯಾಗವನ್ನು ಮಾಡುತ್ತಾನೆ. ಅನಂತರ ಕೆಲವು ದಿನಗಳು ಶಾಸ್ತ್ರವನ್ನು ಓದಿ ಓದಿ, `ಈ ಯಾಗವನ್ನೆಲ್ಲ ನಾನು ಭಗವಂತನಿಗೆ ಅರ್ಪಣೆ ಮಾಡಿರಬೇಕು. ಆದ್ದರಿಂದ ಇದೆಲ್ಲ ಭಗವದರ್ಪಣೆಯಾಗಲಿ. ಫಲದಲ್ಲಿ ನನಗೆ ಅಪೇಕ್ಷೆ ಇಲ್ಲ' ಎನ್ನುವ ತೀರ್ಮಾನಕ್ಕೆ ಬಂದು ಬಿಟ್ಟರೆ, `ಯೋಗಃ ಕರ್ಮಸು ಕೌಶಲಂ' ಎಂದು ಭಗವಂತನು ಹೇಳಿದುದು ಅರ್ಥವಾಗುತ್ತದೆ. ಹೀಗೆ ಮಾಡಿದರೆ ಬಂಧಕವಾದ ಕರ್ಮಗಳನ್ನೂ ಮೋಚಕವಾದ ಕರ್ಮಗಳಾಗಿ ನಾವು ಬದಲಾಯಿಸಿಕೊಳ್ಳಬಹುದು.

ಆದ್ದರಿಂದ ನಾವು ಕೆಟ್ಟ ಕರ್ಮಗಳನ್ನೆಲ್ಲ ಬಿಟ್ಟು ಒಳ್ಳೆಯ ಕರ್ಮಗಳನ್ನೇ ಮಾಡಬೇಕು. ಅವುಗಳನ್ನು ಮಾಡುವ ಕಾಲದಲ್ಲಿಯೂ, `ಇವುಗಳೆಲ್ಲಾ ನನ್ನ ಕರ್ತವ್ಯ. ಇವುಗಳನ್ನು ಮಾಡುವುದರ ಮೂಲಕ ನಾನು ಭಗವಂತನನ್ನು ಅರ್ಚಿಸುತ್ತೇನೆ' ಎನ್ನುವ ಭಾವನೆಯಿಂದ ಮಾಡುತ್ತಾ ಬಂದರೆ, ಅದೇ ನಮ್ಮ ಅಂತಃಕರಣಕ್ಕೆ ಶಾಂತಿಯನ್ನು ಕೊಡುತ್ತದೆ. ಶಾಂತವಾದ ಅಂತಃಕರಣವಿರುವವನು ವೇದಾಂತ ವಿಚಾರ ಮಾಡಿದರೆ ಅದು ಫಲಿಸುತ್ತದೆ.

ವಿವೇಕ ಚೂಡಾಮಣಿಯಲ್ಲಿ ಶಂಕರರು,

\begin{shloka}
`ಅಧಿಕಾರಿಣಮಾಶಾಸ್ತೇ ಫಲಸಿದ್ಧಿರ್ವಿಶೇಷತಃ|'
\end{shloka}

ಎಂದು ಹೇಳಿದಂತೆ ಅಧಿಕಾರ (ಯೋಗ್ಯತೆ) ಎನ್ನುವುದು ಬೇಕು. ಡೆಪ್ಯುಟಿ ಕಮೀಷನರ್ ಕೆಲಸಕ್ಕೆ ಒಬ್ಬನು ಅರ್ಜಿ ಹಾಕಿಕೊಂಡನು. ಆದರೆ ಆ ಕೆಲಸಕ್ಕೆ ಯೋಗ್ಯತೆ ಕೊಡುವ ಪರೀಕ್ಷೆಯನ್ನು ಅವನು ಬರೆಯಲಿಲ್ಲವೆಂದರೆ ಅವನಿಗೆ ಆ ಕೆಲಸ ಹೇಗೆ ದೊರೆಯುತ್ತದೆ? ಹಾಗೆಯೇ, ವೇದಾಂತಶಾಸ್ತ್ರ ಓದುವುದಕ್ಕೆ `ಅಂತಃಕರಣ ಶುದ್ಧಿ' ಅವಶ್ಯಕ. ಅಂತಃಕರಣ ಶುದ್ಧಿ ಎಂದರೆ ಯಾವುದೆಂದು ನೋಡೋಣ. ನಾವು ವಸ್ತುಗಳನ್ನು ನೋಡದೆ ಇರುವಾಗ ಒಂದು ಆಸೆಯೂ ಉಂಟಾಗುವುದಿಲ್ಲ. ಆದರೆ ವಸ್ತುಗಳನ್ನು ನೋಡುತ್ತಲೇ ಆಸೆ ಉಂಟಾಗುತ್ತದೆ - ಎಂದರೆ ಅಂಥ ಅಂತಃಕರಣ ವೇದಾಂತ ವಿಚಾರ ಮಾಡುವುದಕ್ಕೆ ಸರಿಯಾಗುವುದಿಲ್ಲ. ಎಲ್ಲಾ ವಸ್ತುಗಳ ವಿಷಯದಲ್ಲೂ,

\begin{shloka}
`ಬ್ರಹ್ಮಾದಿ ಸ್ಥಾವರಾಂತೇಷು ವೈರಾಗ್ಯಂ ವಿಷಯೇಷ್ವನು|\\
ಯದೈವ ಕಾಕವಿಷ್ಠಾಯಾಂ ವೈರಾಗ್ಯಂ ತದ್ಧಿ ನಿರ್ಮಲಮ್||
\end{shloka}

(ಬ್ರಹ್ಮದಿಂದ ಹಿಡಿದು ಹುಳುವಿನವರೆಗೆ ಎಲ್ಲಾ ವಸ್ತುಗಳಲ್ಲಿಯೂ ಕಾಗೆಯ ಮಲದ ವಿಷಯದಲ್ಲಿ ಉಂಟಾಗುವಂತೆ ಯಾವ ಆರುಚಿ ಇರುವುದೋ ಅದೇ ಶುದ್ಧವಾದ ವೈರಾಗ್ಯವಾಗುವುದು.) - ಎಂದು ಹೇಳಿದಂತೆ, ಹೇಗೆ ದಾರಿಯಲ್ಲಿ ಯಾವುದಾದರೂ ಪಕ್ಷಿ ಮಲದಿಂದ ಆಶುದ್ಧಿ ಮಾಡಿದ್ದರೆ ನಾವು ಅದರ ವಿಷಯವಾಗಿ ಉದಾಸೀನವಾಗಿರುವಂತೆ, ಬ್ರಹ್ಮನ ಸ್ಥಾನದಿಂದ ಹಿಡಿದು ಸಾಧಾರಣವಾಗಿರುವ ಜನ್ಮದವರೆಗೂ ಎಲ್ಲವೂ ಅನಿತ್ಯವೆಂದು ಒಂದು ನಿರ್ಣಯಕ್ಕೆ ಬರಬೇಕು. ಅನಂತರ,

\begin{shloka}
`ನ ಹ್ಯಧ್ರುವೈಃ ಪ್ರಾಪ್ಯತ್ಯೇ ಹಿ ಧ್ರುವಂ ಹಿ ತತ್'
\end{shloka}

(ಅನಿತ್ಯವಾದವುಗಳಿಂದ ನಿತ್ಯವಾದವುಗಳನ್ನು ಪಡೆಯಲಾಗುವುದಿಲ್ಲ.) - ಎಂದು ಹೇಳಿದಂತೆ ಅನಿತ್ಯವಾಗಿರುವ ವಸ್ತುಗಳನ್ನು ಇಟ್ಟುಕೊಂಡು ನಿತ್ಯವಾಗಿರುವುದನ್ನು ಹೇಗೆ ಪಡೆಯುವುದು? ಆದ್ದರಿಂದ, `ನನಗೆ ಅನಿತ್ಯವಾದ ವಸ್ತುಗಳು ಬೇಕಾಗಿಲ್ಲ. ನಾನು ನಿತ್ಯವಾದ ವಸ್ತುವನ್ನು ಸಂಪಾದಿಸುತ್ತೇನೆ' ಎನ್ನುವ ತೀರ್ಮಾನ ಉಂಟಾದಾಗಲೇ ಯಾರಾದರೂ ವೇದಾಂತಕ್ಕೆ ಅಧಿಕಾರಿ. ಆ ಅಧಿಕಾರ ಬರಬೇಕಾದರೆ-

\begin{shloka}
`ಕಷಾಯೇ ಕರ್ಮಭಿಃಪಕ್ವೇ ತತೋ ಜ್ಞಾನಂ ಪ್ರವರ್ತತೇ'
\end{shloka}

ಎನ್ನುವಂತೆ ನಡೆದುಕೊಳ್ಳಬೇಕು. ಮೊದಲು ನಾವು ಕರ್ಮಾನುಷ್ಠಾನ ಮಾಡುತ್ತಾ ಬರಬೇಕು. ಕೆಲವರಿಗೆ ಕರ್ಮಾನುಷ್ಠಾನ ಮಾಡುವುದರಲ್ಲಿಯೇ ಹುಚ್ಚು ಹಿಡಿಯುತ್ತದೆ. `ವೇದದಲ್ಲಿ ಹೇಳಿದ ಈ ಯಜ್ಞ ಮಾಡಿ ಆಯಿತು. ಇನ್ನು ಈ ಯಜ್ಞ ಮಾಡಬೇಕು' ಎನ್ನುವುದರಲ್ಲಿಯೇ ನಾವು ಇದ್ದರೆ ಮೊದಲೇ ಹೇಳಿದಂತೆ ಕಾಮ್ಯಕರ್ಮಗಳು ಚಿತ್ತಶುದ್ಧಿಯನ್ನು ಕೊಡುವುದಿಲ್ಲ. ನಾವು ಕರ್ತವ್ಯವೆಂದು, ಕೊನೆಯವರೆಗೆ ಒಂದು ಕರ್ಮವನ್ನು ಮಾಡಿದರೆ, ಅದರ ಫಲವನ್ನು ನಾವು ಅನುಭವಿಸಲೇಬೇಕಾಗುವುದು. ಆದ್ದರಿಂದ ನಿಷಿದ್ಧವಾದ ಕರ್ಮಗಳನ್ನು ನಾವು ಬಿಟ್ಟು ವಿಹಿತವಾದ ಕರ್ಮಗಳನ್ನೇ ಮಾಡುತ್ತಾ ಬಂದರೆ ಅಂತಃಕರಣದಲ್ಲಿ ಶಾಂತಿಯುಂಟಾಗುತ್ತದೆ.

ಕೆಲವರು ಜಪ ಮಾಡಬೇಕೆಂದುಕೊಂಡು ಕುಳಿತುಕೊಳ್ಳುತ್ತಾರೆ. ಜಪಮಾಲೆ ಅವರ ಕೈಯಲ್ಲಿ ತಿರುಗುತ್ತದೆ. ಅವರ ನಾಲಿಗೆ ಬಾಯಲ್ಲಿ ತಿರುಗುತ್ತಿರುತ್ತದೆ. ಆದರೆ ಮನಸ್ಸು ಮಾತ್ರ ಹತ್ತು ದಿಕ್ಕುಗಳಲ್ಲಿಯೂ ತಿರುಗುತ್ತಿರುತ್ತದೆ. ಹೀಗೆ ಜಪ ಮಾಡಿದರೆ ಅದು ಜಪವೇ ಆಗುವುದಿಲ್ಲ. ಆದರೂ ಅದು ಸ್ವಲ್ಪವಾದರೂ ಫಲವನ್ನು ಕೊಡುತ್ತದೆ. ಒಂದೇ ತರಗತಿಯಲ್ಲಿದ್ದು ಫೇಲಾಗುವ ವಿದ್ಯಾರ್ಥಿಗೆ ಎರಡು ವರ್ಷ ಅದೇ ತರಗತಿಯಲ್ಲಿ ಓದಿದರೆ ಏನೋ ಸ್ವಲ್ಪವಾದರೂ ಜ್ಞಾನ ಬರುತ್ತದೆ. ಅದಕ್ಕಾಗಿ ಸ್ಕೂಲಿಗೆ ಹೋಗಬೇಡವೆಂದು ಹೇಳಿದರೆ ಅಷ್ಟು ಕೂಡ ಜ್ಞಾನ ಬರುವುದಿಲ್ಲ. ಮನಸ್ಸು ಏಕಾಗ್ರವಾಗಿಲ್ಲ ಎನ್ನುವುದಕ್ಕಾಗಿ ಯಾರೂ ಜಪ ಮಾಡುವುದನ್ನು ನಿಲ್ಲಿಸಿ ಬಿಡಬಾರದು. ಮನಸ್ಸು ಏಕಾಗ್ರವಾದರೆ ಅದು ಬಹಳ ವಿಶೇಷವೇ.

ಆದ್ದರಿಂದ ನಾವು ಕರ್ಮಗಳನ್ನು ಸರಿಯಾಗಿ ಮಾಡಿ, ಈಶ್ವರನ ಕೋಪವನ್ನು ಹೋಗಲಾಡಿಸಿ, ಅವನ ಪ್ರಸಾದವನ್ನು ಪಡೆದು, ಸ್ಥಿರವಾದ ಮನಸ್ಸುಳ್ಳವರಾಗಿದ್ದು ಉಪಾಸನೆ ಮಾಡುತ್ತಾ ಬಂದರೆ ಅದು ಜ್ಞಾನಕ್ಕೆ ಸಾಧನವಾಗುವುದು.
