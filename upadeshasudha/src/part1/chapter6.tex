\chapter{ಮನುಷ್ಯ ಜನ್ಮವನ್ನು ವ್ಯರ್ಥ ಮಾಡಿಕೊಳ್ಳಬೇಡಿ}\label{chap6}

\begin{shloka}
ಗಂಗಾಪೂರಪ್ರಚಲಿತ ಜಟಾಸ್ರಸ್ತ ಭೋಗೀಂದ್ರಭೀತಾಂ\\
ಆಲಿಂಗಂತೀಮಚಲತನಯಾಂ ಸಸ್ಮಿತಂ ವೀಕ್ಷಮಾಣಃ|\\
ಲೀಲಾಪಾಂಗೈಃ ಪ್ರಣತಜನತಾ ನಂದಯಂಶ್ಚಂದ್ರಮೌಲಿಃ\\
ಮೋಹಧ್ವಾಂತಂ ಹರತು ಪರಮಾನಂದಮೂರ್ತಿಶ್ಶಿವೋ ನಃ||
\end{shloka}

ಮನುಷ್ಯನಾಗಿ ಹುಟ್ಟಿದ ಪ್ರತಿಯೊಬ್ಬನಿಗೂ ಶ್ರೇಯಸ್ಸು ಪಡೆಯಬೇಕೆನ್ನುವ ಆಸೆ ತೀವ್ರವಾಗಿರುವುದು ಸಹಜವೇ. ಆ ಶ್ರೇಯಸ್ಸಿನ ಬಗ್ಗೆ ಹಲವರಿಗೆ ಹಲವು ವಿಧವಾದ ಸಂದೇಹಗಳಿವೆ. ಕೆಲವರು ತಾವು ಪ್ರಪಂಚದಲ್ಲಿ ತಿನ್ನುವುದು, ನಿದ್ರಿಸುವುದು, ಸುಖವಾದ ಸೌಕರ್ಯಗಳನ್ನು ಪಡೆಯುವುದನ್ನೇ ಶ್ರೇಯಸ್ಸೆಂದು ಭಾವಿಸುತ್ತಾರೆ. ಇದು ಸರಿಯಲ್ಲ. ನಿತ್ಯವಾದ ಸುಖವೇ ಶ್ರೇಯಸ್ಸು ಎನ್ನಲ್ಪಡುತ್ತದೆ. ಇದಲ್ಲದೆ ಬೇರೆ ಯಾವ ಸುಖವಿದೆಯೋ ಅದಕ್ಕೆ ಪ್ರೇಯಸ್ಸೆಂದು ಹೆಸರು. ಶ್ರೇಯಸ್ಸಿಗೂ ಪ್ರೇಯಸ್ಸಿಗೂ ಇರುವ ವ್ಯತ್ಯಾಸವನ್ನು ತಿಳಿಯದೆ ಸಾಧಾರಣವಾದ ವೈಭವಕ್ಕೆ (ಅನಿತ್ಯವಾದ ಸುಖಕ್ಕೆ) ಶ್ರೇಯಸ್ಸೆಂದು ಹೇಳಿಬಿಡುತ್ತೇವೆ. ಕಠೋಪನಿಷತ್ತಿನಲ್ಲಿ-

\begin{shloka}
ಅನ್ಯಚ್ಛ್ರೇಯೋಽನ್ಯದುತೈವ ಪ್ರೇಯ-\\
ಸ್ತೇ ಉಭೇ ನಾನಾರ್ಥೇ ಪುರುಷಂ ಸಿನೀತಃ|\\
ತಯೋಃ ಶ್ರೇಯಃ ಆದದಾನಸ್ಯ ಸಾಧು\\
ಭವತಿ ಹೀಯತೇಽರ್ಥಾದ್ಯ ಉ ಪ್ರೇಯೋ ವೃಣೀತೇ||
\end{shloka}

(ಶ್ರೇಯಸ್ಸೆನ್ನುವುದು ಬೇರೆ. ಅದರಿಂದ ಬೇರೆಯಾದುದು ಪ್ರೇಯಸ್ಸು. ಅವುಗಳು ಬೇರೆ ಬೇರೆ ದಾರಿಯಲ್ಲಿದ್ದು ಮನುಷ್ಯರನ್ನು ತಮ್ಮ ನಿಯಮಗಳಲ್ಲಿ ಇಟ್ಟುಕೊಂಡಿರುತ್ತವೆ. ಈ ಎರಡರಲ್ಲಿ ಶ್ರೇಯಸ್ಸನ್ನು ಆರಿಸಿಕೊಳ್ಳುವವನಿಗೆ ಒಳ್ಳೆಯದಾಗುತ್ತದೆ. (ಆದರೆ) ಪ್ರೇಯಸ್ಸನ್ನು ಆರಿಸಿಕೊಳ್ಳುವವನು ನಿಜವಾದ ಲಕ್ಷ್ಯವನ್ನು ಮುಟ್ಟಲಾರನು)-ಎಂದು ಹೇಳಲ್ಪಟ್ಟಿದೆ. ಅದರಲ್ಲಿ ಶ್ರೇಯಸ್ಸು ಪ್ರೇಯಸ್ಸು ಎಂದು ಎರಡಿವೆ. ಶ್ರೇಯಸ್ಸನ್ನು ಪಡೆದವನು ಯಾವಾಗಲೂ ಅದರಲ್ಲಿ ಸ್ಥಿರವಾಗಿರುತ್ತಾನೆ. ಅಂದರೆ ಅವನಿಗೆ ನಿತ್ಯವಾದ ಸುಖವುಂಟು. ಆದರೆ ಪ್ರೇಯಸ್ಸನ್ನು ಅಪೇಕ್ಷಿಸುವವರಿಗೆ ಸುಖ ಇಂದು ಇರುತ್ತದೆ, ನಾಳೆ ಇರುವುದಿಲ್ಲ. ನಾವು ತಾತ್ಕಾಲಿಕವಾದ ಸುಖವನ್ನು ಬಯಸುವುದಿಲ್ಲ. ಸ್ಥಿರವಾದ ಸುಖವನ್ನು ಬೇಕೆಂದೇ ಬಯಸುತ್ತೇವೆ. ಆದರೆ ಸ್ಥಿರವಾದ ಸುಖವನ್ನು ಪಡೆಯಲು ಸಾಮರ್ಥ್ಯವಿಲ್ಲದ ನಾವು ತಾತ್ಕಾಲಿಕ ಸುಖದಲ್ಲೇ ಸಂತೋಷಪಟ್ಟು ಕೊಂಡವರಾಗಿ ತೃಪ್ತಿ ಪಡೆಯುತ್ತೇವೆ. ಆದ್ದರಿಂದ ಬುದ್ಧಿವಂತನಾದವನು ಶ್ರೇಯಸ್ಸನ್ನೇ ಅಪೇಕ್ಷಿಸಬೇಕು. ಏಕೆ ಹಾಗೆ ಎನ್ನುವುದನ್ನು ನೋಡೋಣ.

ಮನುಷ್ಯಜನ್ಮ ಎನ್ನುವುದು ನಮಗೆ ಈಗ ಹೊಸದಾಗಿ ಬಂದುದಲ್ಲ. ದುರ್ವಾಸ ಮಹರ್ಷಿಗಳು ಶಕ್ತಿಮಹಿಮ್ನ ಸ್ತೋತ್ರದಲ್ಲಿ,

\begin{shloka}
ನಾನಾ  ಯೋನಿಸಹಸ್ರಸಂಭವವಶಾತ್  ಜಾತಾ ಜನನ್ಯಃ ಕತಿ\\
ಪ್ರಖ್ಯಾತಾ ಜನಕಾಃ ಕಿಯನ್ತ ಇತಿ ಮೇ ಸೇತ್ಸ್ಯನ್ತಿ ಚಾಗ್ರೇ ಕತಿ!\\
ಏತೇಷಾಂ ಗಣನೈವ ನಾಸ್ತಿ..........
\end{shloka}

(ಎಷ್ಟೋ ಸಾವಿರಗಟ್ಟಲೆ ಯೋನಿಗಳಲ್ಲಿ ಜನ್ಮವನ್ನು ಪಡೆದುದರಿಂದ ಎಷ್ಟೋ ಮಂದಿ ತಾಯಿಯರು ಆಗಿ ಹೋಗಿದ್ದಾರೆ. ಎಷ್ಟೋ ಮಂದಿ ತಂದೆಯರು ಆಗಿ ಹೋಗಿದ್ದಾರೆ. ನಮ್ಮ ಮುಂದೆ ಎಷ್ಟೋ ಮಂದಿ ಹೋಗುತ್ತಿದ್ದಾರೆ. ಇವುಗಳಿಗೆಲ್ಲ ಗಣನೆಯೇ ಇಲ್ಲ.........) ಎಂದು ಹೇಳಿದ್ದಾರೆ.

ಉಪನಿಷತ್ತುಗಳಲ್ಲಿ ಆತ್ಮನು `ಸತ್ಯಂ ಜ್ಞಾನಂ ಅನನ್ತಂ ಬ್ರಹ್ಮ' ಎಂದು ವಿವರಿಸಲ್ಪಟ್ಟಿದ್ದಾನೆ. ಅಂದರೆ ಆತ್ಮನು ಸತ್ಸ್ವರೂಪಿ, ಜ್ಞಾನ ಸ್ವರೂಪಿ, ಅನಂತಸ್ವರೂಪಿ. ಅವನು ಎಲ್ಲೆ ಇಲ್ಲದಷ್ಟು ಸುಖಸ್ವರೂಪಿ.

\begin{shloka}
`ಯತ್ ಸೌಖ್ಯಾಂಬುಧಿ ಲೇಶಲೇಶತ ಇಮೇ ಶಕ್ರಾದಯೋ ನಿರ್ವೃತಾಃ|'
\end{shloka}

ಆ ಪರಮಾನಂದ ಒಂದು ತುಣುಕನ್ನು ಅನುಭವಿಸಿ ಇಂದ್ರನೇ ಮುಂತಾದವರು ತೃಪ್ತಿಗೊಂಡರು. ಇದರಿಂದ ಆ ಆನಂದಸಾಗರ ಎಂತಹದೆಂಬುದನ್ನು ನಾವು ತಿಳಿದುಕೊಳ್ಳಬಹುದು. ಈ ಆನಂದಸಾಗರವೇ ಚೈತನ್ಯ. ಈ ಚೈತನ್ಯವೇ ಜೀವರೂಪವನ್ನು ಪಡೆದು ಎಲ್ಲವನ್ನೂ ನೋಡುತ್ತಿದೆ. ಇದಕ್ಕೆ ಸಂಬಂಧಪಟ್ಟಂತೆ ಒಂದು ಉದಾಹರಣೆ ನೋಡೋಣ.

ಆಕಾಶ ಎನ್ನುವುದು ಒಂದೇ ವಿಧವಾಗಿಯೇ ಇರುತ್ತದೆ. ಆದರೆ ಕೊಡವನ್ನು ನಾವು ಇಟ್ಟರೆ, `ಕೊಡದಲ್ಲಿರುವ ಆಕಾಶ' (ಘಟಾಕಾಶ) - ಎಂದು ಅದಕ್ಕೆ ಹೆಸರು ಬಂದುಬಿಡುತ್ತದೆ. ಅದೇ ಒಂದು ಮನೆಕಟ್ಟಿದರೆ `ಗೃಹಾಕಾಶ' (ಮನೆಯ ರೀತಿಯಲ್ಲಿರುವ ಆಕಾಶ) - ಎಂದು ಹೆಸರು ಬಂದುಬಿಡುತ್ತದೆ.

\begin{shloka}
`ಸ ಏಷ ಇಹ ಆನಖಾಗ್ರೇಭ್ಯಃ ಪ್ರವಿಷ್ಟಃ'
\end{shloka}

ಅದೇ ರೀತಿ ಚೈತನ್ಯವೂ ಅನಾದಿಯಾದ ಸಂಸಾರ ಚಕ್ರದಲ್ಲಿ ಪ್ರವೇಶಿಸಿತು. ಅದರಿಂದ ಬಂಧದಲ್ಲಿ ಸಿಕ್ಕಿಕೊಂಡಂತಾಯಿತು. ಆಕಾಶ ನಿಜಕ್ಕೂ ಒಂದಾದರೂ, ಘಟಾಕಾಶವೆಂದು ನಾವು ಹೇಳುವಂತೆ ಇಲ್ಲಿಯೂ ಕೂಡ ಸಂಸಾರಚಕ್ರ ಏತಕ್ಕೆ ಅನಾದಿಯೆಂದು ಹೇಳಲ್ಪಟ್ಟಿದೆಯೆಂದು ಕೆಲವರಲ್ಲಿ ಪ್ರಶ್ನೆ ಉಂಟಾಗುತ್ತದೆ. ಸಂಸಾರಕ್ಕೆ ಮೂಲ ಹೇಳಲಾಗುವುದಿಲ್ಲ, ಕೆಲವಕ್ಕೆ ಕಾರಣ ಹೇಳಲಾರೆವು. ಮರದಿಂದ ಹಣ್ಣು ಬರುತ್ತದೆಯೇ ಅಥವಾ ಹಣ್ಣಿನಿಂದ ಮರವೇ ಎಂದು ಕೇಳಿದರೆ `ಬರುತ್ತಿದೆ' ಎಂದು ಹೇಳಬಹುದೇ ಹೊರತು ಅದಕ್ಕೆ ಮೂಲ ತೋರಿಸಲಾಗುವುದಿಲ್ಲ. ಅದೇ ರೀತಿ ಕರ್ಮದಿಂದ ಜನ್ಮ, ಜನ್ಮದಿಂದ ಸುಖಗಳು. ಮತ್ತೆ ರುಚಿ, ಪುನಃ ಕರ್ಮ ಎಂದು ಇದು ಮುಂದುವರೆಯುತ್ತಾ ಚಕ್ರದಂತೆ ಸುತ್ತಿಕೊಂಡಿರುತ್ತದೆ. ಈ ಚಕ್ರದಲ್ಲಿ ಸಿಕ್ಕಿಕೊಂಡವನಿಗೆ ಇದರಿಂದ ಬಿಡುಗಡೆಯಾಗಬೇಕೆಂಬ ಆಸೆ ಉಂಟಾಗುವುದು ಸಹಜ. ಹೇಗೆ ಬಿಡುಗಡೆ ಪಡೆಯುವುದು.

ಸಮುದ್ರದಲ್ಲಿರುವ ನೀರು ಆವಿಯಾಗಿ ಮಾರ್ಪಟ್ಟು ಮೇಘವಾಗುತ್ತದೆ. ಅನಂತರ ಅದೇ ಮಳೆಯಾಗಿ ಬಿದ್ದು ನದಿಯಾಗಿ ಹರಿದು ಕಡಲಲ್ಲಿ ಸೇರುತ್ತದೆ. ನಾವು ಸಮುದ್ರ ಸ್ಥಾನೀಯವಾಗಿ ಬ್ರಹ್ಮದಿಂದ ಬೇರ್ಪಟ್ಟವರು. ಆದರೂ ಬ್ರಹ್ಮವೇ. ಬೇರೆ ರೂಪವನ್ನು ಪಡೆದ ಮೇಲೆ ಬ್ರಹ್ಮವೆನ್ನುವ ಹೆಸರಿಲ್ಲ. ಮೇಘವಾಗಿರುವಾಗ ಅದಕ್ಕೆ ಸಮುದ್ರವೆನ್ನುವ ಹೆಸರು ಹೇಳುವುದಿಲ್ಲ. ಅದೇ ರೀತಿ ನದಿಯಾಗಿ ಇರುವಾಗಲೂ ಅದಕ್ಕೆ ಸಮುದ್ರವೆನ್ನುವ ಹೆಸರು ಇಲ್ಲ. ಆದರೆ ನದಿ ಸಮುದ್ರವನ್ನು ಸೇರುತ್ತಲೇ 

`ಸಮುದ್ರ ಇತ್ಯೇವಂ ಪ್ರೋಚ್ಯತೇ ನಾಮರೂಪೇ ವಿಹಾಯ' (ನಾಮ ರೂಪಗಳನ್ನು ಬಿಟ್ಟು ಸಮುದ್ರವೆಂದೇ ಹೇಳಲ್ಪಡುತ್ತದೆ) - ಎಂದಾಗುತ್ತದೆ. ನಾವು ಕೂಡ ಬ್ರಹ್ಮದಲ್ಲಿ ಸೇರುವವರೆಗೆ ಸುತ್ತಿಕೊಂಡೇ ಇರುತ್ತೇವೆ. ನಮ್ಮ ಲಕ್ಷ್ಯ  ಬ್ರಹ್ಮವನ್ನು ಸೇರುವುದೇ ಆಗಿದೆ. ಆಗಲೇ ಶಾಂತಿಯುಂಟಾಗುತ್ತದೆ. ಆದರೆ, ಕೆಲವರು ಅದನ್ನು ಬಿಟ್ಟು, `ನಾವು ಪರಮಶಿವನ ಲೋಕಕ್ಕೋ, ವಿಷ್ಣುವಿನ ಲೋಕಕ್ಕೋ, ಬ್ರಹ್ಮನ ಲೋಕಕ್ಕೋ ಹೋಗಿ ಇದ್ದರೆ ಅದು `ಮೋಕ್ಷ' ಎನ್ನುತ್ತಾರೆ. ಆದರೆ ಯಾವಾಗಲೂ, 

\begin{shloka}
`ದ್ವಿತೀಯಾತ್ ವೈ ಭಯಂ ಭವತಿ'
\end{shloka}

-ಎಂದು ಹೇಳಿರುವುದರಿಂದ ಎರಡನೆಯ ವಸ್ತು ಎನ್ನುವುದಿದ್ದರೆ ಅದರಿಂದ ಭಯ ಉಂಟಾಗುತ್ತದೆ. ಪ್ರಪಂಚದಲ್ಲಿ, ಎಂಥ ಬಲಶಾಲಿಯಾದರೂ ಅವನಿಗೆ ಇನ್ನೊಬ್ಬ ಬಲಶಾಲಿಯನ್ನು ಕಂಡರೆ ಭಯವಾಗುವುದನ್ನು ನಾವು ನೋಡುತ್ತೇವೆ. ತನ್ನಂತೆ ಇನ್ನೊಬ್ಬನು ಇಲ್ಲ ಎನ್ನುವ ತೀರ್ಮಾನವಾದಾಗ ಅವನಿಗೆ ಭಯ ಉಂಟಾಗಲು ಮಾರ್ಗವಿರುವುದಿಲ್ಲ. ತೀರ್ಮಾನವಾದರೂ ಎರಡನೆಯ ವಸ್ತು ಇದ್ದರೆ, ಅದು ಪ್ರಯೋಜನವಿಲ್ಲ. (ಬಂಧವೆನ್ನುವುದು ಅಜ್ಞಾನದಿಂದ ಬಂದುದು. ಆದ್ದರಿಂದ ಈ ಅಜ್ಞಾನವನ್ನು ದೂರ ಮಾಡಿದರೆ ಬಂಧನವೆನ್ನುವುದೇ ಇರುವುದಿಲ್ಲ.)

\begin{shloka}
`ದ್ವಿತೀಯಾತ್ ವೈ ಭಯಂ ಭವತಿ'
\end{shloka}

ಎನ್ನುವುದರಿಂದ ವೈಕುಂಠದಂಥ ಲೋಕಗಳಲ್ಲಿದ್ದರೂ ಎರಡನೆಯ ವಸ್ತುವಿನ ಬಗ್ಗೆ ಅರಿವು ಇರುವುದರಿಂದ ಆಗಲೂ ಭಯವಿದ್ದೇ ಇರುತ್ತದೆ. ಜಯ-ವಿಜಯರು ವೈಕುಂಠದಲ್ಲೇ ಇದ್ದರೂ ಮತ್ತೆ ಅವರು ಜನ್ಮ ಪಡೆಯುವಂತಾಯಿತು. ಆದರೆ,

\begin{shloka}
`ಅತ್ರ ಬ್ರಹ್ಮ ಸಮಶ್ನುತೇ'
\end{shloka}

-ಎನ್ನುವಂತೆ ಈ ಜನ್ಮದಲ್ಲಿಯೇ ಬ್ರಹ್ಮಸಾಕ್ಷಾತ್ಕಾರವನ್ನು ಪಡೆದವರಿಗೆ ಮತ್ತೆ ಜನ್ಮ ಬಂದಿತೆಂದು ಯಾವ ಒಂದು ಉಪನಿಷತ್ತಾಗಲಿ, ಪುರಾಣವಾಗಲಿ, ಇತಿಹಾಸವಾಗಲಿ ಹೇಳಿದಂತೆ ಎಲ್ಲಿಯೂ ಇಲ್ಲ. ಬ್ರಹ್ಮ ಸಾಕ್ಷಾತ್ಕಾರವಾದವರಿಗೆ

\begin{shloka}
`ನ ತಸ್ಯ ಪ್ರಾಣಾ ಉತ್ಕ್ರಾಮನ್ತಿ!'
\end{shloka}

(ಅವನ ಪ್ರಾಣ ಉತ್ಕ್ರಮಿಸುವುದಿಲ್ಲ)-ಎಂದು ಹೇಳಲ್ಪಟ್ಟಿದೆ. ಬೇರೆಯವರಿಗೆ ಹೀಗಲ್ಲ. ಅಲ್ಲದೆ ಜ್ಞಾನಿಯ ವಿಷಯವಾಗಿ

\begin{shloka}
`ಅತ್ರೈವ ಸಮವಲೀಯನ್ತೇ'
\end{shloka}

-ಎಂದೂ ಹೇಳಿರುವುದರಿಂದ ಯಾವ ಯಾವ ಭೂತಗಳಿಂದ ಈ ಶರೀರ ಬಂದಿದೆಯೋ, ಆ ಭೂತಗಳಲ್ಲಿಯೇ ಇದು ಹೋಗಿ ಸೇರುತ್ತದೆಂದು ತಿಳಿಯುತ್ತದೆ. ಆದ್ದರಿಂದ ಬ್ರಹ್ಮಸಾಕ್ಷಾತ್ಕಾರ ಪಡೆದರೇನೇ ನಮಗೆ ಜನ್ಮ-ಮರಣವೆನ್ನುವ ಚಕ್ರದಿಂದ ಬಿಡುಗಡೆ ಉಂಟಾಗುತ್ತದೆ.

\begin{shloka}
`ಯದೇಹ ಕರ್ಮಚಿತೋ ಲೋಕಃ ಕ್ಷೀಯತೇ ಏವಮೇವ\\
ಅಮುತ್ರ ಪುಣ್ಯಚಿತೋ ಲೋಕಃ ಕ್ಷೀಯತೇ.'
\end{shloka}

ಈ ಪ್ರಪಂಚದಲ್ಲಿ ಕರ್ಮದ ಫಲವಾಗಿರುವುವು ಕಣ್ಣುಗಳಿಗೆ ಬೀಳುತ್ತಿದ್ದಂತೆಯೇ ನಾಶವಾಗುವಂತೆ ಪುಣ್ಯದಿಂದಲೂ, ಮಹಾಯಜ್ಞಗಳಿಂದಲೂ, ಉಪಾಸನೆಗಳಿಂದಲೂ ಸಂಪಾದಿಸಲಾದ ಲೋಕಗಳೂ ನಶಿಸಿ ಹೋಗುತ್ತವೆ. ಆದ್ದರಿಂದ ನಾವು ನಮ್ಮ ಸ್ವರೂಪವನ್ನು ತಿಳಿದುಕೊಂಡು ಅಜ್ಞಾನವನ್ನು ಹೋಗಲಾಡಿಸಿಕೊಳ್ಳಬೇಕು.

\begin{shloka}
`ನೈವ ತಸ್ಯ ಕೃತೇನಾರ್ಥೋ ನಾಕೃತೇನೇಹ ಕಶ್ಚನ.'
\end{shloka}

ಜ್ಞಾನವನ್ನು ಪಡೆದವನು ಕೆಲಸಗಳನ್ನು ಮಾಡಿದರೂ, ಮಾಡದೆ ಇದ್ದರೂ ಅವನಿಗೆ ಯಾವ ವಿಧವಾದ ಫಲದಲ್ಲಿಯೂ ಅಭಿಲಾಷೆ ಇರುವುದಿಲ್ಲ. ಆದರೆ ಇತರರಿಗೆ ಅವನಿಂದ ಶ್ರೇಯಸ್ಸು ಉಂಟಾಗುವ ಯೋಗವಿದೆ. ಆ ಯೋಗದಿಂದ ಅವನು ಕೆಲಸಗಳನ್ನು ಮಾಡಿಕೊಂಡಿದ್ದಾನೆ.

\begin{shloka}
`ಜನ್ತೂನಾಂ ನರಜನ್ಮ ದುರ್ಲಭಂ'
\end{shloka}

ಜನ್ಮಗಳಲ್ಲಿ ನರಜನ್ಮ ದುರ್ಲಭವಾದುದು.

\begin{shloka}
`ನಾನಾಯೋನಿ ಸಹಸ್ರ ಸಂಭವಂ'
\end{shloka}

-ಎನ್ನುವಂತೆ ನಮಗೆ ಅನೇಕ ಜನ್ಮಗಳು ಆಗಿವೆಯೆಂದು ನೋಡಿದೆವು. ಜನ್ಮಗಳಲ್ಲಿ ಚರ-ಅಚರವೆಂದು ಎರಡು ವಿಧ. ಈ ಪ್ರಪಂಚದಲ್ಲಿ ನಾವು ನೋಡುವ ಪ್ರಾಣಿಗಳು ಚರ. ಈ ಎರಡರಲ್ಲಿ ಉತ್ತಮವಾದುದು ಮನುಷ್ಯ ಜನ್ಮವೇ. ಮನುಷ್ಯನು ತನ್ನ ಕರ್ಮಗಳಿಂದ ತನ್ನನ್ನು ತಾನು ಉತ್ತಮಗೊಳಿಸಿಕೊಳ್ಳುತ್ತಾನೆ. ಅಥವಾ ತನ್ನನ್ನು ಕೆಡಿಸಿಕೊಳ್ಳಲೂ ಬಹುದು. ಹೀಗೆ ಮಾಡಲು ಮನುಷ್ಯನಿಗೆ ಸ್ವಾತಂತ್ರ್ಯವಿದೆ. ಬೇರೆ ಯಾವ ಜೀವಕ್ಕೂ ಈ ರೀತಿಯಾದ ಸ್ವಾತಂತ್ರ್ಯವಿಲ್ಲ. `ಇದು ಒಳ್ಳೆಯದು' ಎಂದು ತಿಳಿದು ಅದನ್ನು ಮಾಡುವ ಯೋಗ್ಯತೆಯಾಗಲಿ `ಇದು ತಪ್ಪು' ಎಂದು ತಿಳಿದು ಅದನ್ನು ಬಿಡುವ ಯೋಗ್ಯತೆಯಾಗಲೀ ಇತರ ಪ್ರಾಣಿಗಳಿಗೆ ಇಲ್ಲವೇ ಇಲ್ಲ. ಅವು ಯಾವ ನಿಮಿಷದಲ್ಲಿ ಏನು ಮಾಡಬೇಕೆಂದು ತೋರುವುದೋ ಅದನ್ನು ಮಾಡುತ್ತವೆ. ಅದರಿಂದ ಯಾವ ಫಲ ದೊರೆಯುತ್ತದೆಯೋ ಅದನ್ನು ಅನುಭವಿಸುತ್ತವೆ. ಮನುಷ್ಯ ಜನ್ಮವೆನ್ನುವುದು ಹಾಗಲ್ಲ. ಇಂದು ಒಂದು ಕೆಲಸವನ್ನು ಮಾಡುವಾಗ `ಶ್ರಮವಾಗಿದ್ದರೂ ಅನಂತರ ಶ್ರೇಯಸ್ಸಾಗುತ್ತದೆ' ಎಂದು ತಿಳಿದರೆ ಮನುಷ್ಯನು ಆ ಕೆಲಸವನ್ನು ಮಾಡಬಲ್ಲನು. ಅದರಿಂದ ಮನುಷ್ಯ ಜನ್ಮವನ್ನು  ಪಡೆದ ನಾವು ಬ್ರಹ್ಮಸಾಕ್ಷಾತ್ಕಾರವನ್ನು ಪಡೆಯಲು ಪ್ರಯತ್ನಪಟ್ಟು ಅದನ್ನು ಪಡೆದು `ಕುರುಯತ್ನಮಜ್ಮನಿ' ಎಂದು ಹೇಳಿದಂತೆ ಜನ್ಮರಾಹಿತ್ಯವನ್ನು ಮಾಡಿಕೊಳ್ಳಬಹುದೆಂದು ಶಾಸ್ತ್ರದಿಂದಲೂ ಅನುಭವದಿಂದಲೂ ತಿಳಿಯುತ್ತದೆ.

ಒಂದು ಸ್ಥಳದಲ್ಲಿ ಒಬ್ಬರು, ಚೆನ್ನಾಗಿ ಊಟಮಾಡಿ, ಮನಸ್ಸು ತೃಪ್ತಿಪಡುವಂತೆ ಪಾನಪಾಡಿ, ಚೆನ್ನಾಗಿ ಸಂತೋಷಪಡಿ. ಆದರೆ ಒಂದು ಮಾತ್ರ ಖಂಡಿತವಾಗಿಯೂ ತಿಳಿದುಕೊಳ್ಳಿ. ಅದೇನೆಂದರೆ, `ಶರೀರದಿಂದ ಬೇರಾದ `ಆತ್ಮ' ಎನ್ನುವುದು ಒಂದಿದೆ. ಇದನ್ನು ಮನಸ್ಸಿನಲ್ಲಿ ಗಂಭೀರವಾಗಿ ಸ್ಥಿರವಾಗಿಟ್ಟುಕೊಳ್ಳಿ'-ಎನ್ನುತ್ತಾರೆ. ಈ ವಿಧವಾಗಿ ನಾವು ಮನಸ್ಸಿನಲ್ಲಿ ಸ್ಥಿರಪಡಿಸಿಕೊಂಡರೆ ಮಾತ್ರ ನಾವು ಮುಂದುವರಿಯಬಹುದು. ಅದಕ್ಕಾಗಿಯೇ `ದೇಹಕ್ಕೆ ಅತಿರಿಕ್ತನಾದ ಆತ್ಮ' ಎನ್ನುವುದನ್ನು ತೀರ್ಮಾನಿಸಿಕೊಳ್ಳಬೇಕೆಂದು ಹೇಳಿದರು.

ಹೀಗೆ ಅರಿವಿಲ್ಲದೆ ಇದ್ದರೆ, ನಾವು ಚಾರ್ವಾಕರಂತೆ ಶ್ರೇಯಸ್ಸಿನ ವಿಷಯದಲ್ಲಿಯೆ ನಂಬಿಕೆ ಇಲ್ಲದವರಾಗಿ ಬಿಡುತ್ತೇವೆ. ಈಗ ಕೆಲವು ಮತಗಳನ್ನು ಕುರಿತು ನೋಡೋಣ.

ತಾರ್ಕಿಕರು ದೇಹಾತಿರಿಕ್ತನಾದ ಆತ್ಮನಿದ್ದಾನೆಂದು ಹೇಳುತ್ತಾರೆ. ಆದರೆ ಅದರ ಬಗ್ಗೆ ವಿವರಿಸಲು ಹೋಗುವುದಿಲ್ಲ. ಅವರು ಆತ್ಮನಿಗೆ ಸುಖ ದುಃಖ ಎಲ್ಲಾ ಉಂಟೆನ್ನುತ್ತಾರೆ. ನಾವು ಅದನ್ನು ಒಪ್ಪಲಾಗುವುದಿಲ್ಲ. ಸುಖವೋ ದುಃಖವೋ ಉಂಟು ಎಂದರೆ ಎರಡೂ ಹೋಗಬೇಕಾದುವೇ. ಎರಡು ಸ್ಥಿತಿಗಳೂ ಹೋದರೆ ಆತ್ಮನು ಕಲ್ಲಿನಂತೆ ಆಗಿಬಿಡುತ್ತಾನೋ? ಕಲ್ಲಿಗಿಂತ ಉನ್ನತ ಸ್ಥಿತಿಯಲ್ಲಿರುವ ನಾವುಗಳು ಕಲ್ಲಾಗುವುದಕ್ಕೆ  ಪ್ರಯತ್ನಪಡಬೇಕೇ? ಎಂದು ಕೇಳಿದರೆ, ಅದಕ್ಕೆ ಅವರು ಉತ್ತರ ಕೊಡಲು ಸಾಧ್ಯವಿಲ್ಲ. ಆದ್ದರಿಂದ ತರ್ಕಶಾಸ್ತ್ರದಲ್ಲಿ ಹೇಳಲ್ಪಟ್ಟ ಮೋಕ್ಷ ಕಲ್ಲಿಗೂ ಇದೆ. ಆದ್ದರಿಂದ ಅಂಥ ಮೋಕ್ಷ ಉಪಯೋಗವಾಗುವುದಿಲ್ಲ. ಕೆಲವರೂ `ಇಲ್ಲಿ ಆತ್ಮನಿಗೆ ನಿತ್ಯಸುಖ ಬಂದುಬಿಡುತ್ತದೆ' ಎನ್ನುತ್ತಾರೆ. ಇದರಲ್ಲಿ ಒಂದು ತಪ್ಪಿದೆ. ಅದೇನೆಂದರೆ, ನಿತ್ಯಸುಖ ಬಂದು ಬಿಡುತ್ತದೆ, ಎಂದರೆ ಅದು ನಿತ್ಯಸುಖವೇ ಆಗುವುದಿಲ್ಲ. ಬಂದ ನಿತ್ಯಸುಖ ಹೋಗಲೇ ಬೇಕಾಗುತ್ತದೆ. ಹೋದಮೇಲೆ ಆತ್ಮನು ಜಡನಾಗಿರಬೇಕು. ಆದ್ದರಿಂದ ತರ್ಕಶಾಸ್ತ್ರದಲ್ಲಿ `ಆತ್ಮನಿದ್ದಾನೆ' ಎಂದು ಹೇಳಿ ಅದಾದ ಮೇಲಿನ ವಿಮರ್ಶೆಯ ಬಾಧ್ಯತೆಯನ್ನು ನಮ್ಮ ಮೇಲೆಯೇ ಹಾಕಿಬಿಟ್ಟರು.   

ಸಾಂಖ್ಯ ಮತವನ್ನು ಒಪ್ಪುವವರು, `ಆತ್ಮನು ಸುಖ ಸ್ವರೂಪಿ, ಜ್ಞಾನಸ್ವರೂಪಿ' ಎನ್ನುತ್ತಾರೆ. ಅಲ್ಲದೆ, ಅವರ ಅಭಿಪ್ರಾಯದಂತೆ ಮುಕ್ತರಾಗಿರುವ ಹಲವು ಬೇರೆ ಬೇರೆ ಆತ್ಮರು ಇದ್ದಾರೆ. ಹಲವರು ಮುಕ್ತರಾಗಿ ಇರುವುದರಿಂದ,

\begin{shloka}
`ದ್ವಿತೀಯಾತ್ ವೈ ಭಯಂ ಭವತಿ'
\end{shloka}

-ಎಂದು ಹೇಳಿದಂತೆ ಒಬ್ಬ ಮುಕ್ತರಿಗೆ, ಇಷ್ಟು ಮಂದಿ ಮುಕ್ತರಲ್ಲಿ ಯಾರು ಹೆಚ್ಚು, ಯಾರು ಕಡಿಮೆ ಎನ್ನುವ ಅಭಿಪ್ರಾಯ ಉಂಟಾಗಬಹುದು. ಭೇದದೃಷ್ಟಿ ಇರುವವರೆಗೆ ಮೋಕ್ಷವಿಲ್ಲವೆಂದು ಮೊದಲೇ ಹೇಳಿದೆವು. ಆದ್ದರಿಂದ ಈ ಮತವೂ ಸರಿಯಲ್ಲ.

\begin{shloka}
`ದ್ವಿತೀಯಾತ್ ವೈ ಭಯಂ ಭವತಿ'
\end{shloka}

-ಎನ್ನುವ ಸ್ಥಾನದಲ್ಲಿ ಒಂದನ್ನು ಗಮನಿಸಬೇಕು. ನಾವೇ ಒಂದು ಪಿಶಾಚಿಯನ್ನು ಎದುರಿಗೆ ಉಂಟುಮಾಡಿದರೆ ನಾವು ಭಯಪಡುತ್ತೇವೆಯೇ? ಭಯಪಡುವುದೇ ಇಲ್ಲ. ಹೊಸದಾಗಿ ಯಾವುದಾದರೂ ಪಿಶಾಚಿ ಬಂದರೆ ಭಯಪಡುವ ಸಂದರ್ಭ ಒದಗಬಹುದು.

ಅದ್ವೈತಮತದ ಪ್ರಕಾರ ಇಷ್ಟು ದೊಡ್ಡ ಪ್ರಪಂಚ ಪಿಶಾಚಿಯಂತೆ ಇದ್ದರೂ ನಮ್ಮ ಸ್ವರೂಪವೇ ಎಂದು ಹೇಳಲ್ಪಟ್ಟಿದೆ. ಆದ್ದರಿಂದ ನಾವು ಏಕೆ ಭಯಪಡಬೇಕು? ಆದ್ದರಿಂದ ಅದ್ವಿತೀಯವಾದ ಬ್ರಹ್ಮದಲ್ಲಿ ಇರುವ ರಸ ಬೇರೆ ಯಾವ ಮತದಲ್ಲಿಯೂ ಇಲ್ಲ.

ಎಷ್ಟೋ ಕಾಲದಿಂದ ಹುಟ್ಟು-ಸಾವುಗಳಲ್ಲಿ ಸಿಕ್ಕಿಕೊಂಡಿರುವ ನಾವು ಜ್ಞಾನವನ್ನು ಪಡೆಯದೆ ಇರುವುದರಿಂದ ಮುಟ್ಠಾಳರೇ? ಅಲ್ಲ. ಏಕೆ? ಈಗ ಭಗವಂತನ ಕೃಪೆಯಿಂದ ನಮಗೆ ಮನುಷ್ಯ ಜನ್ಮ ದೊರೆತಿದೆ. ಇದರಿಂದ ನಾವು ಶ್ರೇಯಸ್ಸನ್ನು ಪಡೆಯಬಹುದು. ಸ್ವಲ್ಪ ಶೋಧಿಸಿ ನೋಡಿದರೆ ದುಃಖವಿಲ್ಲದ ಸುಖ ಪ್ರಪಂಚದಲ್ಲಿ ಎಲ್ಲಿಯೂ ಇಲ್ಲವೆಂದು ತಿಳಿಯಬಹುದು. ಆದ್ದರಿಂದ ಹೊರ ವಸ್ತುಗಳಿಂದ ಸುಖವಿಲ್ಲವೆನ್ನುವುದನ್ನೂ ತಿಳಿಯಬಹುದು. ಇದರಿಂದ ಯಾವುದಾದರೂ ಒಂದು ಮಾರ್ಗದಿಂದ ಶ್ರೇಯಸ್ಸನ್ನು ಪಡೆಯಬೇಕೆನ್ನುವ ಆಸೆ ಉಂಟಾಗುತ್ತದೆ. ಅದಕ್ಕೆ ಮಾರ್ಗವೇನು? ಎನ್ನುವುದನ್ನು ಮುಂದೆ ಬರುವ ಶ್ಲೋಕ ಹೇಳುತ್ತದೆ.

\begin{shloka}
ಸ್ವಸ್ಯಾವಿದ್ಯಾ ಬಂಧ ಸಂಬಂಧ ಮೋಕ್ಷಾತ್\\
ಸತ್ಯ ಜ್ಞಾನಾನಂದ ರೂಪಾತ್ಮ ಲಭ್ಧೌ|\\
ಶಾಸ್ತ್ರಂ ಯುಕ್ತಿರ್ದೇಶಿಕೋಕ್ತಿಃ ಪ್ರಮಾಣಂ\\
ಚಾನ್ತಃ ಸಿದ್ಧಾ ಸ್ವಾನುಭೂತಿಃ ಪ್ರಮಾಣಮ್||
\end{shloka}

ಹಲವು ಶಾಸ್ತ್ರಗಳಲ್ಲಿ `ಮೋಕ್ಷಕ್ಕೆ ಶಾಸ್ತ್ರವೇ ಪ್ರಮಾಣ' ಎನ್ನುವ ಉತ್ತರವೇ ದೊರೆಯುತ್ತದೆ. ಭಗವತ್ಪಾದರು ಅಷ್ಟು ನಿನ್ನ ಸ್ವಂತ ಮಾತ್ರವಲ್ಲದೇ, `ನೀನು ಮೋಕ್ಷವನ್ನು ಪಡೆಯಬೇಕಾದರೆ ಅನುಭವವೂ ಪ್ರಮಾಣ'-ಎನ್ನುತ್ತಾರೆ.

`ನಮಗೆ ಎಷ್ಟೋ ಅನುಭೂತಿಗಳು ಇರಬಹುದು. ಅವುಗಳಿಂದ ಮೋಕ್ಷ ಬರುವಂತೆ ತೋರುವುದಿಲ್ಲ. ಮೋಕ್ಷವನ್ನು ಪಡೆಯಲು ಏನು ಅನುಭೂತಿಯಾಗಬೇಕು' ಎಂದು ಯಾರಾದರೂ ಕೇಳಬಹುದು. ಅಂಥ ಅನುಭೂತಿ ಉಂಟಾಗುವುದಕ್ಕೆ ಒಂದು ಪದ್ಧತಿ ಇದೆ. ಆ ಪದ್ಧತಿಯಲ್ಲೇ ನಾವು ಹೋಗಬೇಕು.

ಒಂದು ವಜ್ರವಿದೆ. ಅದು ಒಳ್ಳೆಯ ವಜ್ರವೇ? ಎಂತಹದು ಎಂದೆಲ್ಲಾ ಪರೀಕ್ಷಿಸುವುದಕ್ಕೆ ನಾವು ಇಷ್ಟಪಡಬಹುದು. ಆದರೆ ಅದು ಫಳಫಳ ಎಂದಿದೆ ಎನ್ನುವುದು ಮಾತ್ರ ನಮಗೆ ಗೊತ್ತು. ಇಲ್ಲಿಯೂ ನಮಗೆ `ಪ್ರತ್ಯಕ್ಷ ಅನುಭೂತಿ' ನಾವು ವಜ್ರವನ್ನು ನೋಡಿದರೆ ಉಂಟಾಗುತ್ತದೆ. ಆದರೆ ಆಭರಣಗಳ ವಿಷಯದಲ್ಲಿ ನಿಪುಣನಾಗಿರುವವನಿಗೆ ಆ ವಜ್ರವನ್ನು ನಾವು ಕೊಟ್ಟರೆ ಅವನು, `ಸ್ವಾಮಿ, ಈ ವಜ್ರ ಧರಿಸುವುದಕ್ಕೆ ಯೋಗ್ಯವಲ್ಲ. ಇದರಲ್ಲಿ ಒಂದು ಕಪ್ಪು ಮಚ್ಚೆ ಇದೆ' ಎನ್ನುತ್ತಾನೆ. ನಮಗೆ ಮೊದಲು ಆ ಮಚ್ಚೆ ಕಾಣಿಸಲೇ ಇಲ್ಲ. ಆ ಆಭರಣವನ್ನು ಪರೀಕ್ಷಿಸುವವನು ನಮಗೆ `ಈ ಕಡೆಯಿಂದ ನೋಡಿ! ಈ ಕಡೆಯಿಂದ ನೋಡಿ!' ಎಂದು ಹೇಳಿ ನಮಗೆ ಮಚ್ಚೆಯನ್ನು ತೋರಿಸುತ್ತಾನೆ. ಹೀಗೆ ಅವನು ತನ್ನ ಅನುಭೂತಿಯೆಲ್ಲವನ್ನೂ ನಮಗೆ ತಂದು ಕೊಟ್ಟರೆ ನಮ್ಮ ಅನುಭೂತಿಯೇ ಪ್ರಮಾಣವಾಗುತ್ತದೆ.

ಇಂಥ ಒಂದು ಚಿಕ್ಕ ಮಚ್ಚೆಯ ವಿಷಯದಲ್ಲಿ ಹೀಗಿರುವಾಗ,

\begin{shloka}
`ತಂ ದುರ್ದರ್ಷಂ ಗೂಢಮನುಪ್ರವಿಷ್ಟಂ\\
ಗುಹಾಹಿತಂ ಗಹ್ವರೇಷ್ಟಂ ಪುರಾಣಮ್'
\end{shloka}

ಎನ್ನುವ ಬ್ರಹ್ಮದ ವಿಷಯದಲ್ಲಿ ಹೇಳುವ ಅಗತ್ಯವೆ ಇಲ್ಲ. ಮಂತ್ರದಲ್ಲಿ `ದುರ್ದರ್ಷಂ' ಎಂದು ಹೇಳಿರುವಂತೆ ಅವನನ್ನು ನೋಡುವುದು ಬಹಳ ಕಷ್ಟ. `ಗುಹಾಹಿತಂ' ಎಂದರೆ ಒಳಗೆ ಇರುವವನು ಎಂದು ಅರ್ಥ.

ಗಾಳಿ ಎಲ್ಲೆಡೆಯಲ್ಲಿಯೂ ಇರುತ್ತದೆ. ಆದರೆ ಗಾಳಿ ಬರುತ್ತಿದ್ದರೇನೇ `ಗಾಳಿ ಇದೆ' ಎಂದು ಹೇಳುತ್ತೇವೆ. ಗಾಳಿ ಬರದೇ ಇದ್ದರೆ `ಗಾಳಿ ಇಲ್ಲ' ಎನ್ನುತ್ತೇವೆ. ಆದರೆ ಗಾಳಿ ಇಲ್ಲದ ಜಾಗದಲ್ಲಿ ಎರಡು ನಿಮಿಷಗಳ ಮೇಲೆ ಇರಲಾಗುವುದಿಲ್ಲ. ಗಾಳಿ ಇಲ್ಲವೆಂದು ಹೇಳುವವನ ಮೂಗು ಹಿಡಿದುಕೊಂಡರೆ ಗಾಳಿ ಇದೆಯೆಂದು ಅವನು ತಿಳಿಯುತ್ತಾನೆ. ಆದ್ದರಿಂದ `ಗುಹಾಹಿತಂ ಗಹ್ವರೇಷ್ಟಂ ಪುರಾಣಮ್' ಎಂದು ಹೇಳಲ್ಪಡುವ ಬ್ರಹ್ಮವನ್ನು ತಿಳಿದುಕೊಳ್ಳಬೇಕಾದರೆ `ಹೀಗೆ ಒಂದು ವಸ್ತುವಿದೆ' ಎಂದು ಹೇಳಲು ಒಬ್ಬ ಆಪ್ತರು ಬೇಕು. ಆ ಆಪ್ತರಿಗೆ ಅದು ಹೇಗೆ ತಿಳಿಯಿತು ಎಂದರೆ ಅವರು ಹೆಚ್ಚಾಗಿ ಪರೀಕ್ಷೆ ಮಾಡಿ ಅದನ್ನು ತಿಳಿದುಕೊಂಡಿದ್ದಾರೆ.

ನಾವಾಗಿಯೇ ವಿಚಾರಮಾಡಿ ಒಂದು ಸಿದ್ಧಾಂತವನ್ನು ಮಾಡಬಹುದೆಂದರೆ, ಇನ್ನೊಂದು ದಿನ ನಮಗಿಂತ ಬುದ್ಧಿವಂತನಾದವನು `ಅಂದಿಗೆ ಅದು ಸರಿಯಾಗಿದ್ದಿತು. ಈಗ ಅದು ಕೆಲಸಕ್ಕೆ ಬಾರದ್ದು' ಎಂದು ಹೇಳಿ ನಮ್ಮ ಸಿದ್ಧಾಂತವನ್ನು ಬದಲಾಯಿಸಿ ಬೇರೆಯೊಂದನ್ನು ಸ್ಥಿರಪಡಿಸುತ್ತಾನೆ. ಇದನ್ನೇ ವಾಚಸ್ಪತಿ ಮಿಶ್ರರು ಬಹಳ ಸ್ವಾರಸ್ಯವಾಗಿ,

\begin{shloka}
ಯತ್ನೇನಾನುಮಿತೋಽಪ್ಯರ್ಥಃ ಕುಶಲೈರನುಮಾತೃಭಿಃ|\\
ಅಭಿಯುಕ್ತ ತರೈರನ್ಯೈರನ್ಯಥೈವ ಉಪಪಾದ್ಯತೇ||
\end{shloka}

ಎಂದಿದ್ದಾರೆ.

ತರ್ಕದಲ್ಲಿ ಪಂಡಿತನು ಯುಕ್ತಿಯಿಂದ ಆತ್ಮನಿಗೆ `ಆತ್ಮನಃ ಸ್ಯೂಃ ಚತುರ್ದಶ'. ಹದಿನಾಲ್ಕು ಗುಣಗಳಾದ ರಾಗದ್ವೇಷಗಳು ಇವೆಲ್ಲವೂ ಇವೆ ಎಂದನು. ಆದರೆ ಅವನಿಗಿಂತಲೂ ಹೆಚ್ಚು ತರ್ಕಬುದ್ಧಿಯುಳ್ಳ ಸಾಂಖ್ಯ `ಆತ್ಮನ ಹೆಸರಿನಲ್ಲಿ ರಾಗ-ದ್ವೇಷ ಯಾವುದನ್ನೂ ಹೇಳಬಾರದು. ಅವುಗಳೆಲ್ಲವೂ ಪ್ರಕೃತಿಗೆ ಮಾತ್ರವಿದೆ' ಎಂದು ಹೇಳಿ ತಾರ್ಕಿಕನು ಆತ್ಮಾನುಗುಣವಾಗಿ ಹೇಳಿದ ರಾಗ-ದ್ವೇಷ ಮುಂತಾದವುಗಳನ್ನು ಸಾಂಖ್ಯನು ಪ್ರಕೃತಿಯ ಮೇಲೆ ಹಾಕಿ ಪುರುಷ (ಆತ್ಮ) ನನ್ನು ಶುದ್ಧವಾಗಿಟ್ಟುಬಿಟ್ಟನು. ಹೀಗೆ ಪ್ರಕೃತಿ ಎನ್ನುವುದು ಒಂದು ಇರುವಂತೆ ಮಾಡಿ ಅದರ ಮೇಲೆ ಪುರುಷನಿರುವಂತೆ ವಿಶ್ವಾಸವನ್ನು ಉಂಟುಮಾಡಿ ಒಂದು ದೊಡ್ಡ ಸಂಸಾರವನ್ನೇ ನಡೆಸುವಂತೆ ಮಾಡಿದನು ಸಾಂಖ್ಯನು. ಇಬ್ಬರೂ (ತಾರ್ಕಿಕನೂ, ಸಾಂಖ್ಯನೂ) ತರ್ಕದಲ್ಲಿ ಒಳ್ಳೆಯ ನೈಪುಣ್ಯ ಹೊಂದಿದ್ದರೂ, ತಾರ್ಕಿಕನು ಮನಸ್ಸಿನವರೆಗೆ ಹೋದನು. ಸಾಂಖ್ಯನು, ಅವನಿಗಿಂತಲೂ ಮುಂದೆ ಒಂದು ಹೆಜ್ಜೆ ಪ್ರಕೃತಿವರೆಗೆ ಹೋದನು.

ವೇದಾಂತಶಾಸ್ತ್ರ ಈ ಸಾಂಖ್ಯನನ್ನು ಕುರಿತು `ಆ ಪ್ರಕೃತಿ ಯಾರಿಂದ ಆಯಿತು. ನಿನಗೆ ಗೊತ್ತೇ' ಎಂದು ಕೇಳಿದರೆ, ಸಾಂಖ್ಯನು `ಅದಕ್ಕಾಗಿಯೇ ಆಯಿತು' ಎನ್ನುತ್ತಾನೆ. ಜಡವಾಗಿರುವ ವಸ್ತು ಎಲ್ಲಾದರೂ ತಾನೇ ಉಂಟಾಗುತ್ತದೆಯೇ? ಗಾಳಿ ಆಡಿದರೇನೇ ಎಲೆ ಅಲ್ಲಾಡುತ್ತದೆ. ಗಾಳಿ ಆಡದೆ ಇದ್ದರೆ ಒಂದು ಸಣ್ಣ ಎಲೆ ಕೂಡ ಅಲ್ಲಾಡುವುದಿಲ್ಲ. ಹೀಗಿರುವುದರಿಂದ,

\begin{shloka}
`ಸೈವಾತ್ಮ ಶಕ್ತಿಂ ಸ್ವಗುಣೈರ್ನಿಗೂಢಂ'
\end{shloka}

`ಅವನ ಸಂಕಲ್ಪವಿಲ್ಲದಿದ್ದರೆ ಈ ಪ್ರಪಂಚವೇ ಇಲ್ಲ. ಆದ್ದರಿಂದ ಚೈತನ್ಯವೇ ಎಲ್ಲದಕ್ಕೂ ಮೂಲ' ಎಂದೆಲ್ಲಾ ತಾರ್ಕಿಕರಿಂದ ಹೇಳಲೇ ಆಗುವುದಿಲ್ಲ. ಆದ್ದರಿಂದ ತಾರ್ಕಿಕನು ಸ್ವಲ್ಪ ಮಟ್ಟಿಗೆ ಹೇಳಿದನು. ಅವನಿಗಿಂತಲೂ ದೊಡ್ಡ ತಾರ್ಕಿಕನಾದ ಸಾಂಖ್ಯನು ಬೇರೆ ವಿಧವಾಗಿ ಹೇಳಿದನು. ಆದರೆ

\begin{shloka}
`ವೇದಾನ್ತ ವಿಜ್ಞಾನ ಸುನಿಶ್ಚಿತಾರ್ಥಾಃ'
\end{shloka}

ಎಂದು ಹೇಳಿದಂತೆ ವೇದಾಂತಿಗಳು ಮುಂದಕ್ಕೆ ಹೋಗಿ ವಿಮರ್ಶಿಸಿದರು. ಆತ್ಮದರ್ಶನ ಅವರಿಗೆ,

\begin{shloka}
`ಆತ್ಮಾವಾ ಅರೇ ದ್ರಷ್ಟವ್ಯೋ ಶ್ರೋತವ್ಯೋ ಮಂತವ್ಯೋ ನಿದಿಧ್ಯಾಸಿತವ್ಯಃ'
\end{shloka}

(`ಪ್ರಿಯಳೇ! ಆತನನ್ನು ಕಾಣಬೇಕು, ಆತ್ಮನ ಬಗ್ಗೆ ಕೇಳಬೇಕು ಚಿಂತಿಸಬೇಕು. ಸ್ಥಿರವಾದ ಮನಸ್ಸಿನಿಂದ ಧ್ಯಾನ ಮಾಡಬೇಕು'.)

ಎನ್ನುವ ಮಾರ್ಗವೆಲ್ಲಾ ಹೇಳಲ್ಪಟ್ಟಿವೆ. ಅವುಗಳಲ್ಲಿ ವೇದಾಂತ ಶ್ರವಣ ಮುಖ್ಯವಾದುದು.

\begin{shloka}
`ತಂ ತ್ವೌಪನಿಷದಂ ಪುರುಷಂ ಪೃಚ್ಛಾಮಿ'
\end{shloka}

ಉಪನಿಷತ್ತಿನಿಂದಲೇ ಅವನನ್ನು ಪಡೆಯಬಹುದು.

ಕೆಲವರು ಹೀಗೆ ಅಭ್ಯಂತರವನ್ನು ಹೇಳುತ್ತಾರೆ-

\begin{shloka}
`ಗರ್ಭಸ್ಥ ಏವ ಋಷಿರ್ವಾಮದೇವಃ ಪ್ರತಿಪೇದೇ ಅಹಂ ಮನುರಭವಂ ಸೂರ್ಯಶ್ಚ'
\end{shloka}

`ಗರ್ಭದಲ್ಲಿರುವಾಗಲೇ ವಾಮದೇವರಿಗೆ ಬ್ರಹ್ಮಜ್ಞಾನವಾಗಿ ಬಿಟ್ಟಿತು ಎಂದು ನೀವು ಹೇಳುತ್ತೀರಲ್ಲ! ಹಾಗಿರುವ ಉಪನಿಷತ್ತುಗಳನ್ನು ಓದಿಯೇ ಬ್ರಹ್ಮವನ್ನು ತಿಳಿದುಕೊಳ್ಳಬೇಕೆಂದು ಹೇಳುವಿರಲ್ಲಾ! ಇದು ಹೇಗೆ? ಈ ಪ್ರಶ್ನೆಗೆ ಒಂದು ಶ್ರೇಷ್ಠ ಮಾತನ್ನೇ ಉತ್ತರವಾಗಿ ಹೇಳಬಹುದು. ಆತ್ಮನೆನ್ನುವವನು ಈ ಶರೀರವಲ್ಲ. ಅವನು ಪ್ರತಿಯೊಂದು ಶರೀರವನ್ನೂ ತೆಗೆದುಕೊಳ್ಳುತ್ತಾನೆ ಎನ್ನುವುದೇ ಅದು.

ವಾಮದೇವರು ಗರ್ಭದಲ್ಲಿರುವಾಗ ಆತ್ಮ ಸಾಕ್ಷಾತ್ಕಾರವನ್ನು ಪಡೆದರು ಎನ್ನುವುದರಿಂದ ಅವರು ಉಪನಿಷತ್ತಿನ ವಿಚಾರವೇ ಮಾಡಲಿಲ್ಲವೆಂದು ಯಾರು ಹೇಳಿದರು? ಅವರು ಮೊದಲ ಜನ್ಮದಲ್ಲಿ ಉಪನಿಷತ್ ವಿಚಾರವನ್ನು ಮಾಡಿದ್ದರೆಂದು ನಾವು ಇಟ್ಟುಕೊಳ್ಳಬಹುದಲ್ಲವೆ. ಉಪನಿಷತ್ ವಿಚಾರ ಮಾಡಿದೊಡನೆ ಬ್ರಹ್ಮಜ್ಞಾನ ಆವಾಗಲೇ ಉಂಟಾಗಬಹುದಾಗಿತ್ತಲ್ಲಾ ಎಂದರೆ, ಕೆಲವರಿಗೆ ಅರ್ಜಿ ಹಾಕುತ್ತಲೇ ಕೆಲಸ ಸಿಕ್ಕಬಹುದು, ಕೆಲವರಿಗೆ ಆರು ತಿಂಗಳು ವಿಳಂಬವಾಗಬಹುದು. ಏಕೆಂದರೆ ಅವರಿಗೆ ಶಿಫಾರಸು ಹೆಚ್ಚಾಗಿರಬಹುದು. ಅದೇ ರೀತಿ ಒಬ್ಬನಿಗೆ ಉಪನಿಷತ್ ವಿಚಾರದಿಂದ ತಕ್ಷಣವೇ ಜ್ಞಾನ ಉಂಟಾಗಬಹುದು. ಇನ್ನೊಬ್ಬನಿಗೆ ಪ್ರಾರಬ್ಧ ಕರ್ಮವನ್ನು ಅನುಭವಿಸಿದ ಮೇಲೆ ಪೂರ್ವ ಸಂಸ್ಕಾರದಿಂದ ಜ್ಞಾನ ಉಂಟಾಗುತ್ತದೆ. ಆದ್ದರಿಂದ ಹೇಗಾದರೂ ಆಗಲಿ, ಆತ್ಮಜ್ಞಾನ ಉಪನಿಷತ್ತುಗಳ ಮೂಲಕವಾಗಿಯೇ ಸಂಪಾದಿಸಲು ಸಾಧ್ಯ. ಅನುಭೂತಿ ಎನ್ನುವುದು ಶಾಸ್ತ್ರದಿಂದಲೇ ಬರುತ್ತದೆ. ಆದ್ದರಿಂದ ಅದಕ್ಕೆ ಮೊದಲು ಪ್ರಮಾಣ ಶಾಸ್ತ್ರವೇ ಆಗುತ್ತದೆ.

ಶಾಸ್ತ್ರವೇ ಮೊದಲು ಪ್ರಮಾಣವೆಂದು ಹೇಳಿದರೆ ನಾವು ಅದನ್ನು ನಂಬಲಾರೆವು. ಶಾಸ್ತ್ರದಲ್ಲಿ ಏನೇನೋ ಹೇಳಿದ್ದಾರೆ. ಅವುಗಳೆಲ್ಲ ಅರ್ಥವಾಗುವುದಿಲ್ಲ. ಇದು ಮಾತ್ರ ಅರ್ಥವಾಗುತ್ತದೆಯೋ ಎನ್ನುತ್ತಾರೆ ಕೆಲವರು. ಅದಕ್ಕಾಗಿಯೇ ಎರಡನೆಯ `ಯುಕ್ತಿ' ಎಂದಿದ್ದಾರೆ.

ಇನ್ನು ಯುಕ್ತಿಯನ್ನು ನೋಡೋಣ. ಆತ್ಮನು ನಿತ್ಯನು ಅವನು ಸುಖಸ್ವರೂಪಿ. ಆತ್ಮನಲ್ಲದುದು ಯಾವುದೂ ನಿತ್ಯವಾಗುವುದಿಲ್ಲ. ಒಂದು ವಿಷಯವನ್ನು ತಿಳಿಯುವಾಗಲೂ, ಹಲವು ವಿಷಯಗಳನ್ನು ತಿಳಿಯುವಾಗಲೂ ಈ ಜ್ಞಾನವೆನ್ನುವುದು ಮಾತ್ರ ಯಾವಾಗಲೂ ಬದಲಾಗದೆ ಇರುತ್ತದೆ. ಸೂರ್ಯನ ಬೆಳಕು ಹಲವು ವಸ್ತುಗಳನ್ನು ಪ್ರಕಾಶಗೊಳಿಸುತ್ತದೆ. ಇದರಿಂದ ಸೂರ್ಯನ ಬೆಳಕು ಬೇರೆ ಬೇರೆಯೆಂದು ಹೇಳಬಹುದೇ? ಹೇಳಲಾಗುವುದಿಲ್ಲ. ಇನ್ನೊಂದು ಉದಾಹರಣೆ ನೋಡೋಣ.

ಒಂದೇ ಕನ್ನಡಿ. ಅದರ ಮುಂದುಗಡೆ ಇಟ್ಟ ವಸ್ತುಗಳೆಲ್ಲ ಪ್ರತಿಫಲಿಸುತ್ತವೆ. ಅದೇ ರೀತಿ ಜಾಗ್ರತ್ ಅವಸ್ಥೆಯಲ್ಲಿ ಇರುವ ಜ್ಞಾನ ಒಂದೇ ಒಂದು, ಅದು ವಿಷಯಗಳನ್ನು ಪ್ರತಿಫಲಿಸುತ್ತದೆ. ಇಲ್ಲದಿದ್ದರೆ ನಿರಾಕಾರವಾಗಿರುತ್ತದೆ. ಜಾಗ್ರತ್ ಅವಸ್ಥೆಯಂತೆ, ಸ್ವಪ್ನಾವಸ್ಥೆಯಲ್ಲಿಯೂ ಅದೇ ಜ್ಞಾನ (ಮನಸ್ಸಿನಲ್ಲಿ ತೋರುವ ವಸ್ತುಗಳನ್ನು ಪ್ರತಿಫಲಿಸಿಕೊಂಡು) ಪ್ರಕಾಶಿಸುತ್ತದೆ. ಅನಂತರ ಸುಷುಪ್ತಿ ಅವಸ್ಥೆಯಲ್ಲಿ ಜ್ಞಾನವಿರುತ್ತದೆ. ಆದರೆ ಅದು ಇದೆಯೆಂದು ತಿಳಿಯಲಾಗುವುದಿಲ್ಲ. ನಿದ್ರೆಯಿಂದ ಎದ್ದಮೇಲೆ ಒಬ್ಬನು `ನಾನು ಚೆನ್ನಾಗಿ ನಿದ್ರೆಮಾಡಿದೆ. ಆದರೆ ಏನೂ ಗೊತ್ತಾಗಲಿಲ್ಲ' ಎನ್ನುವುದನ್ನು ನೋಡುತ್ತೇವೆ. `ಚೆನ್ನಾಗಿ ಸಂತೋಷವಾಗಿ ನಿದ್ದೆಮಾಡಿದೆ' ಎನ್ನುವುದು ಒಬ್ಬನ ನೆನಪಿಗೆ ಬರಬೇಕಾದರೆ ಅವನು ನಿದ್ದೆ ಮಾಡುವಾಗ ಸುಖದಲ್ಲಿ ಮುಳುಗಿರಬೇಕು. ಅಂದರೆ ಆಗ ಉಂಟಾದ ಅನುಭವವನ್ನೇ ಅವನು ಹೇಳುತ್ತಾನೆ. ನಿದ್ರೆಮಾಡುವಾಗ ಜ್ಞಾನವಿದ್ದರೂ, `ನಾನು ಚೆನ್ನಾಗಿ ನಿದ್ರೆ ಮಾಡುತ್ತಿದ್ದೇನೆ' ಎಂದು ಅವನಿಗೆ ಆಗ ತಿಳಿಯಲಿಲ್ಲ. ಆದ್ದರಿಂದಲೇ `ಸುಷುಪ್ತಿಯಲ್ಲಿ ಅನುವ್ಯವಸಾಯವಿಲ್ಲ'ವೆಂದು ಹೇಳುತ್ತೇವೆ, ಆದ್ದರಿಂದ ಜಾಗ್ರತ್, ಸ್ವಪ್ನ, ಸುಷುಪ್ತಿ ಎನ್ನುವ ಮೂರು ಸ್ಥಿತಿಗಳಲ್ಲಿಯೂ ಕೂಡ ಒಂದೇ ಜ್ಞಾನ ಪ್ರಕಾಶಿಸುತ್ತದೆ. ವಿಷಯಗಳು ಬೇರೆಯಾಗುವಾಗ ಅದಕ್ಕೆ ದೊರೆಯುವ ಹೆಸರುಗಳು ಬೇರೆ ಬೇರೆಯಾಗುತ್ತವೆ. ಹೀಗೆ ದಿನ, ಪಕ್ಷ, ಮಾಸ, ಋತು, ಅಯನ, ವರ್ಷ, ಕಲ್ಪ ಮುಂತಾದ ಎಲ್ಲಾ ಕಾಲಗಳಲ್ಲಿಯೂ ಒಂದೇ ಜ್ಞಾನವೇ ಪ್ರಕಾಶಿಸುತ್ತಾ `ನಿತ್ಯ'ವಾಗಿ ಬೆಳಗುತ್ತದೆ. ಹೀಗೆ ಚೈತನ್ಯವೆನ್ನುವುದು `ನಿತ್ಯ'ವೆನ್ನುವುದನ್ನು ಯುಕ್ತಿಯಿಂದ ನಿರೂಪಣೆ ಮಾಡಬಹುದು.

ಚೈತನ್ಯ `ಸುಖ ಸ್ವರೂಪಿ' ಎನ್ನುವುದಕ್ಕೆ ಯುಕ್ತಿಯನ್ನು ನೋಡೋಣ,

ನಿದ್ರೆ ಮಾಡುವಾಗ ಸುಖವುಂಟೇ ಇಲ್ಲವೇ ಎಂದರೆ, ಸುಖವುಂಟು ಎನ್ನುವುದು ಅನುಭವಕ್ಕೆ ಬರುತ್ತದೆ. ಆ ಸುಖ ಎಲ್ಲಿಂದ ಬಂದುದು? ಯಾವುದಾದರೂ ತಿಂಡಿ ತಿನಸುಗಳನ್ನು ತಿಂದುದರಿಂದಲೋ, ಯಾವುದಾದರೂ ವಸ್ತುಗಳನ್ನು ನೋಡಿದುದರಿಂದಲೋ, ಯಾರನ್ನಾದರೂ ಹಾಸ್ಯ ಮಾಡಿದುದರಿಂದಲೋ ಬಂದದ್ದಲ್ಲ.

ಗಾಢನಿದ್ರೆಯಲ್ಲಿ ಯಾವ ವಸ್ತುವಿನೊಡನೆಯೂ ಸಂಬಂಧವಿಲ್ಲ. ಹೊರಗಿನಿಂದ ಸುಖ ಬರುವುದಕ್ಕೆ ಮಾರ್ಗವೇ ಇಲ್ಲ. ಆದ್ದರಿಂದ ಚೈತನ್ಯ ಸ್ವರೂಪವಾಗಿರುವ (ದೇಹದಿಂದಲೂ, ಮನಸ್ಸಿನಿಂದಲೂ ಬೇರೆಯಾದ `ನಾನು' ಎಂದು ಹೇಳಲ್ಪಡುವ) ಆ `ನನ್ನ' ಹತ್ತಿರದಿಂದಲೇಯೆ ಸ್ಫುರಿಸಲು ಸಾಧ್ಯ. ಇಲ್ಲದಿದ್ದರೆ ಬದಲಾಗುತ್ತದೆ. ಆದ್ದರಿಂದ `ನಾನು' ಎನ್ನುವುದೇ ಆನಂದಸ್ವರೂಪವಾಗಿ ಇದ್ದೇನೆ. ಇತರ ಸಮಯಗಳಲ್ಲಿ ವಿಷಯಗಳ ಸಂಬಂಧದಿಂದ ಆ ಆನಂದವೆನ್ನುವುದು ಮರೆತು ಹೋಗುತ್ತದೆ. ಅದು ಹೊರಗೆ ತೋರಬೇಕಾದರೆ ವಿಷಯಗಳ ಸಂಬಂಧವನ್ನು ತಪ್ಪಿಸಬೇಕು. ಹೇಗೆ ಕೊಳೆಯಾದ ಕನ್ನಡಿಯಲ್ಲಿ ಸರಿಯಾಗಿ ಮುಖ ತೋರಬೇಕಾದರೆ ಕೊಳೆಯನ್ನು ಒರಸಿ ಬಿಡುತ್ತೇವೋ, ಅದೇ ರೀತಿ ಸುಖಸ್ವರೂಪವನ್ನು ತಿಳಿಯ ಬಯಸಿದರೆ ವಿಷಯಗಳ ಸಂಬಂಧವನ್ನು ಬಿಡಬೇಕು. ಇಲ್ಲದಿದ್ದರೆ, `ಆತ್ಮನು ಸುಖಸ್ವರೂಪಿ' ಎನ್ನುವುದು ಸ್ಪಷ್ಟವಾಗಿ ತಿಳಿಯಲಾಗುವುದಿಲ್ಲ. ಹೀಗೆ ಯುಕ್ತಿಗಳನ್ನು ಹೇಳಿದ್ದಾರೆ.

ವೇದ ಏನೆಂದು ಹೇಳುತ್ತದೆ?

`ಸ ಯಚ್ಚಾಯಂ ಪುರುಷೇ ಯಚ್ಚಾಸಾವಾದಿತ್ಯೇ'-ಈ ಆತ್ಮನು ಒಬ್ಬನೇ ಈ ಶರೀರದಲ್ಲಿ, ಸೂರ್ಯನಲ್ಲಿ ಸರ್ವವ್ಯಾಪಿಯಾಗಿದ್ದಾನೆ.

ಕೇವಲ ಶಾಸ್ತ್ರವನ್ನು ಹೇಳಿಕೊಂಡಿದ್ದರೆ ಅದು ನಂಬಿಕೆಯನ್ನು ಉಂಟುಮಾಡುವುದಿಲ್ಲ. ಆದ್ದರಿಂದ ಯುಕ್ತಿ ಎನ್ನುವುದು ಅವಶ್ಯಕವಾಗುತ್ತದೆ. ಯುಕ್ತಿಯೂ ಇರಬಹುದು. ಶಾಸ್ತ್ರವೂ ಇರಬಹುದು. ಆದರೆ ಹೋಗಿ ನೋಡಿದವನು ಯಾರೂ ಇಲ್ಲ ಎಂದರೆ ನಮಗೆ ಹೇಗೆ ನಂಬಿಕೆಯಾಗುತ್ತದೆ? ಅದಕ್ಕಾಗಿಯೇ ಶಂಕರರು `ದೇಶಿಕೋಕ್ತಿಃ ಪ್ರಮಾಣಂ' ಎಂದರು. `ದೇಶಿಕನು' ಹೇಳುವುದು ಪ್ರಮಾಣವೇ. ದೇಶಿಕನು ಯಾರು?

\begin{shloka}
ಶಾಂತಾ ಮಹಾಂತೋ ನಿವಸನ್ತಿ ಸಂತೋ\\
ವಸಂತವಲ್ಲೋಕಹಿತಂ ಚರನ್ತಃ|\\
ತೀರ್ಣಾಃ ಸ್ವಯಂ ಭೀಮ ಭವಾರ್ಣವಂ ಜನಾನ್\\
ಅಹೇತುನಾಽನ್ಯಾನಪಿ ತಾರಯನ್ತಃ||
\end{shloka}

(ವಸಂತಕಾಲದಂತೆ ಪ್ರಪಂಚಕ್ಕೆ ಒಳ್ಳೆಯದನ್ನು ಮಾಡುತ್ತಾ ತಾವು ಸ್ವತಃ ದಾಟಿದವರಾಗಿದ್ದುಕೊಂಡು ಅಕಾರಣವಾಗಿಯೇ ಇತರ ಜನರನ್ನು ಭಯಂಕರ ಸಂಸಾರವೆನ್ನುವ ಕಡಲಿನಿಂದ ದಾಟಿಸುವವರಾಗಿ, ಶಾಂತ ಸ್ವರೂಪರಾಗಿ ಸಾಧುಗಳಾಗಿರುವ ದೊಡ್ಡವರು (ಪ್ರಪಂಚದಲ್ಲಿ) ಇರುತ್ತಲಿದ್ದಾರೆ.)

-ಎಂದು ಹೇಳಿದಂತೆ ಇರುವವರು ದೇಶಿಕರು. ಅವರು ಸಂಸಾರಸಮುದ್ರವನ್ನು ದಾಟಿರುವವರು. ಅವರ ಶರೀರವಿರುವುದಕ್ಕೆ ಸ್ವಲ್ಪಮಟ್ಟಿಗೆ ಅವರ ಕರ್ಮ ಕಾರಣವಾಗಿದೆ. ತತ್ತ್ವವನ್ನು ಅರಿತುಕೊಳ್ಳುವ ಪುಣ್ಯ ಶಿಷ್ಯನಿಗೆ. ಇದರಿಂದ,

\begin{shloka}
ಆಚಾರ್ಯಾತ್ ಲಬ್ಧಬೋಧಾ ಅಪಿ ವಿಧಿವಶತಃ ಸನ್ನಿಧೌ ಸಂಸ್ಥಿತಾನಾಂ|\\
ತ್ರೇಧಾ ತಾಪಂ ಚ ಪಾಪಂ ಸಕರುಣ ಹೃದಯಾಃ ಸ್ವೋಕ್ತಿಭಿಃ ಕ್ಷಾಲಯನ್ತಿ||
\end{shloka}

(ಆಚಾರ್ಯರ ಹತ್ತಿರವಿದ್ದು, ಮೊದಲು ಮಾಡಿದ ಪುಣ್ಯಫಲವಾಗಿ ಜ್ಞಾನವನ್ನು ಪಡೆದವರು ಕೂಡ, ತಮ್ಮ ಮುಂದಿರುವವರ ಮೂರು ವಿಧವಾದ ತಾಪಗಳನ್ನೂ, ಪಾಪಗಳನ್ನೂ ಸಕರುಣ ಹೃದಯರಾಗಿದ್ದು ತಮ್ಮ ಮಾತುಗಳಿಂದ ದೂರ ಮಾಡುತ್ತಾರೆ.)

-ಎಂದು ಶಂಕರರು ಹೇಳುವಂತೆ, ಆಚಾರ್ಯರು ಹೇಳಿಕೊಟ್ಟು ಶಿಷ್ಯರೂ ಗುರುವಾಗುತ್ತಾರೆ.

`ವಿಧಿವಶತಃ' ಗುರುವಿಗೆ `ಇದನ್ನು ಮಾಡಬೇಕು, ಅದನ್ನು ಮಾಡಬೇಕು' ಎನ್ನುವ ಆಸೆ ಇಲ್ಲ. ಆದರೆ ಶರೀರ ಇನ್ನೂ ಇರುವುದರಿಂದ ಪ್ರಾರಬ್ಧ ಕರ್ಮ ಇರುವುದರಿಂದಲೂ `ಆತ್ಮಜ್ಞಾನ ಪಡೆಯುವುದು' ಎನ್ನುವ ಯೋಗದಿಂದಲೂ ಹೇಳಿಕೊಡುತ್ತಾರೆ.

`ತ್ರೇಧಾ ತಾಪಂ ಚ ಪಾಪಂ'-ಮೂರು ವಿಧವಾದ ತಾಪವೂ, ಪಾಪವೂ ಅವರ ಯುಕ್ತಿಯಿಂದ ದೂರವಾಗುತ್ತವೆ. ಇದನ್ನೇ ಶಂಕರರು `ದೇಶಿಕೋಕ್ತಿಃ ಪ್ರಮಾಣಂ'-ಎಂದು ಹೇಳುತ್ತಾರೆ. ದೇಶಿಕರು ಹೇಳುವುದನ್ನು ಕೇಳಿಕೊಂಡೇ ಇರಬೇಕು. ಅನಂತರವೇ------`ಸ್ವಾನುಭೂತಿ ಪ್ರಮಾಣಂ' ಎನ್ನುತ್ತಾರೆ.

ನಮಗೆ ಏಕೆ ಅನುಭೂತಿ ಉಂಟಾಗಲಿಲ್ಲವೆಂದರೆ, ನಾವು ಶಾಸ್ತ್ರವನ್ನು ಓದಲಿಲ್ಲ. ಅದನ್ನು ಸರಿಯಾಗಿ ವಿಮರ್ಶಿಸಿಲ್ಲ. ನಮಗೆಲ್ಲ ನೀಲಕಂಠ ದೀಕ್ಷಿತರು ಹೇಳುವಂತೆ ಮಡಕೆ ಅನಿತ್ಯ, ಮಡಕೆಯಾಗಿರುವಿಕೆ ಅನಿತ್ಯ ಎನ್ನುವುದು ಮಾತ್ರ ಗೊತ್ತು. ಇಂಥ ವಿವೇಕ ವಿವೇಕವಲ್ಲ. ನಮ್ಮ ಅಜ್ಞಾನವನ್ನು ದೂರ ಮಾಡುವುದೇ ವಿವೇಕ. ಹಾಗೆ ನಾವು ಶಾಸ್ತ್ರವನ್ನು ಓದಿದರೂ, ದೇಶಿಕರ ಯುಕ್ತಿಯನ್ನು ಕೇಳಿದರೂ, ನಾವು ಅವುಗಳನ್ನು ಸಾಧನೆ ಮಾಡುವವರೆಗೆ ಅವುಗಳು ಪ್ರಯೋಜನವಾಗುವುದಿಲ್ಲ. ದೇಶಿಕರಾದವರು ತಮಗೆ ಅನುಭೂತಿಯಾದುದೆಲ್ಲವನ್ನೂ ಶಿಷ್ಯನಿಗೆ ಹೇಳಿದರೂ ಕೂಡ ಶಿಷ್ಯನು ಸಾಧನೆ ಮಾಡದೆ ಇದ್ದರೆ ಏನೂ ಪ್ರಯೋಜನವಾಗುವುದಿಲ್ಲ.

ಅದೇ ವೇದಾಂತ ಶಾಸ್ತ್ರವನ್ನು ಎಲ್ಲರೂ ಓದುತ್ತಾರೆ. ಹಾಗೆ ಓದುವದು ಯಾರಿಗೆ ಫಲಿಸುತ್ತದೆ ಎನ್ನುವುದನ್ನು,

\begin{shloka}
ಆಧೀತೇ ಜ್ಞಾನವ್ರತಜಪಸಮಾಧಾನ ನಿಯಮೈಃ|\\
ವಿಶುದ್ಧಸ್ವಾನ್ತಾನಾಂ ಜಗದಿದಮಸಾರಂ ವಿಮೃಶತಾಮ್||
\end{shloka}

-ಎನ್ನುವ ಶ್ಲೋಕದಲ್ಲಿ ಹೇಳಿದ್ದಾರೆ. ನಾವು ಈ ರೀತಿಯಲ್ಲಿ ಇರಬೇಕು. ಶಾಸ್ತ್ರಗಳನ್ನು ಕಲಿತು, ಅದರಂತೆ ನಮ್ಮ ವರ್ಣಾಶ್ರಮ ಧರ್ಮಗಳನ್ನು ಚೆನ್ನಾಗಿ ಪಾಲಿಸುತ್ತಾ ಅದರಿಂದ ಭಗವಂತನ ಅನುಗ್ರಹವನ್ನು ಸಂಪಾದಿಸಬೇಕು. ಇದೇ ಅಲ್ಲದೆ, ದಾನ, ವ್ರತ, ಜಪ, ಸಮಾಧಿ ಇವುಗಳೆಲ್ಲವೂ ಕೂಡ ಇರಬೇಕೆಂದು ಈ ಶ್ಲೋಕದಲ್ಲಿ ಹೇಳಿದೆ.

ಮನಸ್ಸೆನ್ನುವುದೇ ವಿಚಿತ್ರವಾದುದು. ಸಂಸ್ಕಾರವಿರುವ ಮನಸ್ಸೆನ್ನುವುದು. ಅದು ಇಲ್ಲದಿರುವ ಮನಸ್ಸಿಗಿಂತಲೂ ಪೂರ್ತಿಯಾಗಿ ಬೇರೆಯಾಗಿರುತ್ತದೆ. ಕೃಷ್ಣನ ಕಥೆ ಓದಿದರೆ ಅವನು ಬೆಣ್ಣೆಯನ್ನು ಕದ್ದು ತಿಂದನೆಂದು ಬರುತ್ತದೆ. ಇದನ್ನು ಓದಿ ನಾವು ಏಕೆ ಬೆಣ್ಣೆಯನ್ನು ಕದ್ದು ತಿನ್ನಬಾರದು ಎಂದುಕೊಂಡು ಯಾರೂ ಹಾಗೆ ಮಾಡಕೂಡದು. ಹಾಗೆ ಕಥೆ ಹೇಳುವ ತಾತ್ಪರ್ಯವೇನೆಂದರೆ, ಜನರಿಗೆ ಭಗವಂತನನ್ನು ಬೇರೆ ಬೇರೆ ರೀತಿಯಲ್ಲಿ ವರ್ಣನೆ ಮಾಡಿದರೆ ಅದರಲ್ಲಿ ಅಭಿರುಚಿ ಉಂಟಾಗುತ್ತದೆ ಎನ್ನುವುದೇ ಆಗುತ್ತದೆ. ಸಾಧಾರಣ ಜನರಿಗೆ ಭಗವಂತನ ಬಾಲಲೀಲೆಗಳು ಭಗವಂತನ ಮೇಲೆ ಪ್ರೇಮವನ್ನು ಉಂಟುಮಾಡುತ್ತವೆ. ಅಲ್ಲದೆ, ಭಕ್ತನ ಮನಸ್ಸನ್ನು ಒಳಗೆ ಮಾರ್ಪಡಿಸುವುದಕ್ಕಾಗಿ ಭಗವಂತನ ಲೀಲೆಗಳು ತೋರಿಸಲ್ಪಟ್ಟಿರಬಹುದು. ಇದು ಬಿಟ್ಟು ಭಗವಂತನು ಮಾಡಿದ್ದೆಲ್ಲವನ್ನೂ ನಾನು ಮಾಡುತ್ತೇನೆ ಎಂದರೆ ಭಗವಂತನು ಗೋವರ್ಧನ ಪರ್ವತವನ್ನು ಎತ್ತಿದಂತೆ ನೀನು ಅದನ್ನು ಎತ್ತಿ ಹಿಡಿದುಕೊಳ್ಳು ಎಂದರೆ ಆ ಮನುಷ್ಯನಿಗೆ ಅದು ಆಗದ ಕೆಲಸವಾಗುತ್ತದೆ. ಆದ್ದರಿಂದ ನಾವು ನಮ್ಮ ಅಂತಃಕರಣವನ್ನು ಶುದ್ಧವಾಗಿಟ್ಟುಕೊಳ್ಳಬೇಕು. ನಾವು ನಮ್ಮ ವರ್ಣಾಶ್ರಮ ಧರ್ಮಗಳನ್ನು ಸರಿಯಾಗಿ ಅನುಷ್ಠಾನ ಮಾಡಿದರೆ ಮನಸ್ಸು ಶುದ್ಧವಾಗಿರುತ್ತದೆ. ಆಗ ಶಾಸ್ತ್ರಗಳಲ್ಲಿ ಹೇಳಲ್ಪಟ್ಟ ತಾತ್ಪರ್ಯ ತಿಳಿಯುತ್ತದೆ. ಹೀಗಿಲ್ಲದಿದ್ದರೆ ಅವನು ಬೇರೆ ರೀತಿಯಾಗಿ ಅರ್ಥ ಮಾಡಿಕೊಳ್ಳುತ್ತಾನೆ.

ಒಂದು ಉದಾಹರಣೆ ನೋಡೋಣ. ಒಮ್ಮೆ ಒಬ್ಬನು ದೇವಸ್ಥಾನಕ್ಕೆ ಹೋದನು. ಅದುವರೆಗೆ ಅವನಿಗೆ ದೇವಸ್ಥಾನಕ್ಕೆ ಹೋಗಿ ಅಭ್ಯಾಸವೇ ಇಲ್ಲ. ದೇವಸ್ಥಾನದಲ್ಲಿ ಅವನಿಗೆ ತೀರ್ಥ ಕೊಟ್ಟರು. ಅವನು ಅದನ್ನು ತೆಗೆದುಕೊಂಡನು. ಆದರೆ ಬಾಯಲ್ಲಿ ಏನೋ ಸಿಕ್ಕಿಕೊಂಡಂತೆ ಇದ್ದಿತು. ನೋಡಿದರೆ, ಅದು ತೀರ್ಥದೊಡನೆ ಬಂದ ತುಳಸಿ ದಳ. ಅದನ್ನು ಅವನು ಅಲ್ಲಿಯೇ ಉಗಿದುಬಿಟ್ಟನು. ತಕ್ಷಣ ಅಲ್ಲಿದ್ದ ಒಬ್ಬನು `ಏನಯ್ಯ ಇದು, ದೇವಸ್ಥಾನದಲ್ಲಿ ಉಗಳಬಹುದೆ'? ಎಂದು ಕೇಳಿದನು. ಅವನು ಒಡನೆ ಆ ತುಳಸಿದಳವನ್ನು ಅಲ್ಲಿದ್ದ ಭಗವಂತನ ವಿಗ್ರಹದ ಮೇಲೆ ಇಟ್ಟುಬಿಟ್ಟನು. ಇದನ್ನು ನೋಡಿದ ಮತ್ತೊಬ್ಬನು, `ಇಲ್ಲಿ ಇಡಬೇಡ ಅಯೋಗ್ಯ' ಎಂದನು. ತುಳಸಿಯನ್ನು ತೆಗೆದು ಅವನು ಇನ್ನೊಂದು ವಿಗ್ರಹದ ಮೇಲೆ ಇಟ್ಟನು. ಅದಕ್ಕೂ ಬೈಗುಳ ತಿಂದು ಮತ್ತೆ ಕೈನಲ್ಲೇ ಹಿಡಿದುಕೊಂಡು ಹೊರಟು ಹೋದನು.

ಸಂಸ್ಕಾರವಿಲ್ಲದವನಿಗೆ ತುಳಸಿಯನ್ನು ಉಗಳಬಾರದೆಂದು ಹೇಳಿದರೆ ಅವನಿಗೆ ಅದನ್ನು ಎಲ್ಲಿಡಬೇಕೆಂದು ತಿಳಿಯುವುದಿಲ್ಲ. ಸಂಸ್ಕಾರವಿರುವವನು ತುಳಸಿಯನ್ನು ತೀರ್ಥದೊಡನೆ ಸ್ವೀಕರಿಸುತ್ತಾನೆ. ಇದೇ ರೀತಿ ವೇದಾಂತ ಶಾಸ್ತ್ರ ವಿಷಯದಲ್ಲೂ,

\begin{shloka}
ಸಾಂಸಾರಿಕಸುಖಾಸಕ್ತಂ ಅಹಂ ಬ್ರಹ್ಮೇತಿವಾದಿನಂ|\\
ಕರ್ಮಬ್ರಹ್ಮೋಭಯಭ್ರಷ್ಟಂ ತಂ ತ್ಯಜೇತ್ ಅಂತ್ಯಜಂ ಯಥಾ||
\end{shloka}

-ಎನ್ನುವಂತೆ ಸಂಸ್ಕಾರವಿಲ್ಲದವನ ನಡವಳಿಕೆ ಇರುತ್ತದೆ. ಅಂದರೆ ಅವನು ಯಾವಾಗಲೂ ವೇದಾಂತವನ್ನು ನೋಡಿ, ಯಾರು ಬಂದರೂ ಸರಳವಾಗಿ ವೇದಾಂತಕ್ಕೆ ವಿವರವನ್ನು ಹೇಳುತ್ತಾನೆ. ಮತ್ತೊಂದು ಕಡೆ, ನನಗೆ ಸಂಬಳ ಇನ್ನೂ ಹೆಚ್ಚಾಗಲಿಲ್ಲವಲ್ಲಾ! ನಾನು ಸೋಫಾಸೆಟ್ ಕೊಂಡುಕೊಳ್ಳಲಾಗಲಿಲ್ಲವಲ್ಲಾ? ಎಂದು ಯೋಚನೆ ಮಾಡುತ್ತಿರುತ್ತಾನೆ. ಆದ್ದರಿಂದ ಅವನಿಗೆ ಕರ್ಮವೂ ಇಲ್ಲ, ಬ್ರಹ್ಮವೂ ಇಲ್ಲ. ಅಂತಹವನನ್ನು ಬಿಟ್ಟು ಬಿಡಬೇಕೆಂದು ಶ್ಲೋಕ ಹೇಳುತ್ತದೆ. ಸಂಸ್ಕಾರವಿದ್ದರೆ ಅವನು ಜ್ಞಾನವನ್ನು ಪಡೆದೊಡನೆಯೇ-

\begin{shloka}
ಯದೃಚ್ಛಾಲಾಭಸಂತುಷ್ಟೋ ದ್ವಂದ್ವಾತೀತೋ ವಿಮತ್ಸರಃ|\\
ಸಮಃ ಸಿದ್ಧಾವಸಿದ್ಧೌ ಚ ಕೃತ್ವಾ ಪಿ ನ ನಿಬಧ್ಯತೇ||
\end{shloka}

(ತಾನಾಗಿಯೇ ದೊರೆತುದರಲ್ಲಿ ಸಂತೋಷಪಡುವವನೂ, ದ್ವಂದ್ವವಿಲ್ಲದವನೂ, ವಿಮತ್ಸರನೂ, ಜಯಾಪಜಯಗಳಲ್ಲಿ ಸಮನಾಗಿರುವವನೂ ಕರ್ಮಗಳನ್ನು ಮಾಡಿದರೂ ಅವುಗಳಿಗೆ ಒಳಗಾಗುವುದಿಲ್ಲ,)

-ಎನ್ನುವಂತೆ ಪ್ರಾರಬ್ಧ ಕರ್ಮದಿಂದಾಗಿ ಅವನು ದೇಹದ ಕೆಲಸಗಳನ್ನು ಮಾಡುತ್ತಾ ಹೋಗುತ್ತಾನೆ. ಆಯಾ ಸಮಯದಲ್ಲಿ ಯಾವುದಾದರೂ ದೊರೆತರೆ ಸರಿ, ಇಲ್ಲದಿದ್ದರೆ ಅದು ಕೂಡ ಭಗವಂತನ ಸಂಕಲ್ಪವೆಂದುಕೊಳ್ಳುತ್ತಾನೆ. ಇವನು ತನ್ನ ಮನಸ್ಸಿನ ಶಾಂತಿ ಸ್ಥಿತಿಯನ್ನು ಬಿಟ್ಟು ಎಂದೂ ಅಸ್ಥಿರನಾಗಿರುವುದಿಲ್ಲ. ಆದ್ದರಿಂದ ವೇದಾಂತವನ್ನು ಓದಲೇಬೇಕಾದರೆ ಮನಸ್ಸು ಸಂಸ್ಕಾರದಿಂದ ಕೂಡಿದ್ದು ವಿಚಾರಮಾಡಬೇಕು.

ಇಲ್ಲಿ ಇನ್ನೊಂದು ವಿಷಯವನ್ನೂ ಗಮನಿಸಬೇಕು. ವಿಚಾರವೆನ್ನುವುದನ್ನು ನಾವು ಶಾಸ್ತ್ರ ಓದಿ ಅದೇ ರೀತಿಯಲ್ಲಿ ಮಾಡಬೇಕು ಅಲ್ಲದೆ, ಯಾವ ವಿಚಾರಕ್ಕಾಗಲಿ ಮೂಲವೆನ್ನುವುದು ಬೇಕು. ಒಬ್ಬನು ಕಳ್ಳತನ ಮಾಡಿದನೆಂದು ಇಟ್ಟುಕೊಳ್ಳಿ. ಕೆಲವರು, `ಕಳ್ಳತನ ಮಾಡುವುದು ತಪ್ಪಲ್ಲವೆಂದು ನಾವು ಭಾವಿಸುತ್ತೇವೆ' ಎಂದು ಹೇಳಿದರೆ ಅದಕ್ಕೆ ನಾವು ಏನು ಹೇಳುತ್ತೇವೆ?. ನೀನು ಏನು ಬೇಕಾದರೂ ಹೇಳು. ಆದರೆ ಇದಕ್ಕೆಲ್ಲಾ ಕಾನೂನು ಪುಸ್ತಕವೊಂದು ಬಂದಿದೆ. ಅದರಲ್ಲಿ ಕಳ್ಳತನ ಮಾಡುವುದು ತಪ್ಪೆಂದು, ಅದಕ್ಕೆ ಆರು ತಿಂಗಳು ಜೈಲುಶಿಕ್ಷೆ ಉಂಟೆಂದು ಹೇಳಿದೆಯೆಂದು ಹೇಳುತ್ತೇವೆ. ಈ ಪುಸ್ತಕವನ್ನು ಮೂಲವಾಗಿಟ್ಟುಕೊಂಡು ಅವನು ನಿಜವಾಗಿಯೂ ಕಳ್ಳತನ ಮಾಡಿದನೆ ಎನ್ನುವುದನ್ನೆಲ್ಲಾ ವಿಚಾರಣೆ ಮಾಡಬಹುದು. ಆದ್ದರಿಂದ ಮೂಲವೆನ್ನುವುದು ಬೇಕು. ಅದೇ ಇಲ್ಲದೆ ವಿಚಾರವೇ ಮಾಡಲಾಗುವುದಿಲ್ಲ. ಅದೇ ರೀತಿ ಉಪನಿಷತ್ತು ಮೂಲವಾಗಿ ತಿಳಿಯಲ್ಪಡುವ ಪರವಸ್ತುವನ್ನು ಉಪನಿಷತ್ತೇ ಇಲ್ಲದೆ ವಿಚಾರ ಮಾಡಿದರೆ,

\begin{shloka}
ಯತ್ನೇನಾನುಮಿತೋಪ್ಯರ್ಥಃ ಕುಶಲೈರನುಮಾತೃಭಿಃ|
\end{shloka}

ಎನ್ನುವಂತೆ ಆಗುತ್ತದೆ.

ಮೊದಲು ನಾವು ವೇದಾಂತಕ್ಕೆ ಶರಣಾಗಬೇಕು. ಅದರಲ್ಲಿ ಹೇಳಿದಂತೆ ಬಾಳನ್ನು ನಡೆಸಬೇಕು. ಮನಸ್ಸು ಶುದ್ಧವಾಗುತ್ತಾ ಆಗುತ್ತಾ `ಕರ್ಮ' ತಾನಾಗಿಯೇ ಸ್ವಲ್ಪವಾಗಿ ಬಿಟ್ಟು ಹೋಗುತ್ತದೆ. ಬಿಟ್ಟುಹೋದ ಕರ್ಮಕ್ಕಾಗಿ ನಾವು ಅಳಬೇಕಾಗಿಲ್ಲ. ವಿಚಾರ ತೀವ್ರವಾಗುತ್ತಿದೆಯೇ ಎಂದು ಮಾತ್ರ ನಾವು ನೋಡಬಹುದು. ಹಾಗೆ ಅದು ತೀವ್ರವಾದರೆ,

\begin{shloka}
`ಅತ್ರ ಬ್ರಹ್ಮ ಸಮಶ್ನುತೇ'.
\end{shloka}

-ಎನ್ನುವಂತೆ ಈ ಜನ್ಮದಲ್ಲೇ ಆತ್ಮಜ್ಞಾನವನ್ನು ಸಂಪಾದಿಸಿಕೊಳ್ಳಬಹುದು. ಅದರ ಮೂಲಕ ಪುನರ್ಜನ್ಮವಿಲ್ಲದಂತೆ ಮಾಡಿಕೊಳ್ಳಬಹುದು. ಜನ್ಮವೇ ಇಲ್ಲ ಎಂದರೆ ದುಃಖವೇ ಇಲ್ಲ. ದುಃಖವಿಲ್ಲದ, ನಿತ್ಯ ಸುಖವಾಗಿರುವ ಆತ್ಮನನ್ನು ಕುರಿತು ಜ್ಞಾನವನ್ನು ಪಡೆದು `ಬ್ರಹ್ಮವಿದಾಪ್ನೋತಿ ಪರಂ' ಎನ್ನುವಂತೆ ಇರಬೇಕು. ಬ್ರಹ್ಮವನ್ನು ಬಿಟ್ಟು ಬೇರೆ ಯಾವ ವಸ್ತುವೂ ಪರ ವಸ್ತುವಲ್ಲ. ಹೀಗೆ `ಬ್ರಹ್ಮವಿದ್ ಬ್ರಹ್ಮೈವಭವತಿ' ಎನ್ನುವಂತೆ ನಾವು ಪರವಸ್ತುವನ್ನು ಪಡೆದು ನಮ್ಮ ಈ ಜನ್ಮವನ್ನು ಸಾರ್ಥಕ ಮಾಡಿಕೊಳ್ಳಬೇಕು.

ಶಂಕರಭಗವತ್ಪಾದರು ಒಂದು ಕಡೆ ಒಂದು ಉದಾಹರಣೆ ಕೊಟ್ಟಿದ್ದಾರೆ. ನಮ್ಮ ದೇಹ ಒಂದು ದೋಣಿಯಂತಿದೆ. ಇದರಲ್ಲಿ ಕುಳಿತಿರುವ ಜೀವನನ್ನು ಸೋಮಾರಿ, `ಎಷ್ಟು ನೀರು ಬಂದರೂ ಚಿಂತೆಯೇನು? ನಾನಾದರೂ ದೋಣಿಯಲ್ಲಿ ಕುಳಿತಿದ್ದೇನೆ' ಎಂದುಕೊಳ್ಳುತ್ತಾನೆ. ಆದರೆ

\begin{shloka}
ಪರಾಂಚಿ ಖಾನಿ ವ್ಯತೃಣತ್ಸ್ವಯಂಭೂಃ|\\
ತಸ್ಮಾತ್ ಪರಾಙ್‌ಪಶ್ಯತಿ ನಾನ್ತರಾತ್ಯನ್||
\end{shloka}

(ಸ್ವಯಂಭೂ ಆದ ಪರಮೇಶ್ವರನು ಹೊರಗುಳ್ಳ ವಸ್ತುಗಳನ್ನು ಕೊರೆಯುಳ್ಳದ್ದಾಗಿ ಮಾಡಿರುವುದರಿಂದ ಹೊರ ವಸ್ತುಗಳನ್ನೇ ಮನುಷ್ಯನು ಕಾಣುತ್ತಾನೆಯೇ ಹೊರತು ಅತ್ಮನನ್ನು ಅಲ್ಲ.)

-ಎಂದು ಹೇಳುವಂತೆ, ಇಂದ್ರಿಯಗಳು ಓಡುವ ಕಡೆಯೇ ನಾವೂ ಓಡುತ್ತೇವೆ. ಇಂದ್ರಿಯಗಳೆಲ್ಲ ನಮ್ಮ ದೋಣಿಯಲ್ಲಿರುವ ಛಿದ್ರಗಳು. ಆ ಛಿದ್ರಗಳನ್ನು ನಾವು ಮುಚ್ಚದೆ ಹೋದರೆ ಸ್ವಲ್ಪ ದೂರ ಹೋದಮೇಲೆ ನಮ್ಮ ದೋಣಿ ಮುಳುಗಿ ಹೋಗುತ್ತದೆ. ನಾವೂ ಮುಳುಗಿ ಹೋಗುತ್ತೇವೆ. ಆದ್ದರಿಂದ ನಾವು ಪುಣ್ಯ ಸಾಧಕವಾಗಿ, ಜ್ಞಾನಸಾಧಕವಾಗಿ ಇರುವ ಮನುಷ್ಯ ಜನ್ಮವನ್ನು ವ್ಯರ್ಥ ಮಾಡಿಕೊಳ್ಳಬಾರದು.

\begin{shloka}
ಇತಃ ಕೋಽನ್ವಸ್ತಿ ಮೂಢಾತ್ಮಾ ಯಸ್ತು ಪ್ರಮಾದ್ಯತಿ|\\
ದುರ್ಲಭಂ ಮಾನುಷಂ ದೇಹಂ ಪ್ರಾಪ್ಯತತ್ರಾಪಿ ಪೌರುಷಮ್||
\end{shloka}

[ದುರ್ಲಭವಾದ ಮನುಷ್ಯ ಶರೀರವನ್ನೂ, ಅದರಲ್ಲೂ ಪುರುಷತನವನ್ನೂ ಪಡೆದು, ತನ್ನ (ಆತ್ಮನ) ವಿಷಯವಾಗಿ ದಾರಿ ತಪ್ಪುವವನಿಗಿಂತಲೂ ಮುಟ್ಠಾಳನಾಗಿ ಯಾರಾದರೂ ಇರಬಹುದೇ?]

ಕಾಲವೆನ್ನುವುದು ದೊರೆಯುವುದು ಕಷ್ಟ. ಮತ್ತೆ ಯಾವ ವಸ್ತುವಾದರೂ ದೊರೆಯಬಹುದು. ಆದರೆ ಕಾಲವೆನ್ನುವುದು ವ್ಯರ್ಥವಾಗಿ ಹೋಯಿತೆಂದರೆ ಪುನಃ ಅದನ್ನು ಸಂಪಾದಿಸಲು ಸಾಧ್ಯವಿಲ್ಲ. ಆದ್ದರಿಂದ ನಾವು ಬಹಳ ಗಮನದಿಂದಿರಬೇಕು. ನಮ್ಮ ಆಭರಣಗಳನ್ನೆಲ್ಲಾ ಕಳ್ಳನು ಕದ್ದುಕೊಂಡು ಹೋದರೆ ನಾವು ಅಳುವಂತೆ ಕಾಲ ಸ್ವಲ್ಪ ವ್ಯರ್ಥವಾದರೂ ನಾವು ಅಳಬೇಕು. ದೊರೆತಿರುವ ಮನುಷ್ಯ ಜನ್ಮ ಬಹಳ ದುರ್ಲಭವಾದುದು. ಪ್ರಪಂಚವೆನ್ನುವುದರ ಬಗ್ಗೆ ತಕ್ಕ ಮಟ್ಟಿಗೆ ನಮಗೆ ಗೊತ್ತು. ಆದ್ದರಿಂದ ಇದರಲ್ಲಿಯೇ ನಾವು ಯಾವಾಗಲೂ ಭ್ರಮಿಸಿಕೊಂಡು ಇರಬಾರದು. ನಾವು ಸಾಧಿಸಬೇಕಾದವು ಬಹಳ ಇವೆ. `ಇತಃ ಕೋನ್ವಸ್ತಿ ........................ ಪೌರುಷಮ್' ಎನ್ನುವ ಶ್ಲೋಕದಂತೆ ಯಾರು ಮೂಢಾತ್ಮರೆಂದು ಶಂಕರರು ಹೇಳಿದರು. ಆದ್ದರಿಂದ ನಾವು ಮನುಷ್ಯ ಜನ್ಮವನ್ನು ಪಡೆದಿರುವುದರಿಂದ ಆತ್ಮಸಾಕ್ಷಾತ್ಕಾರ ಪಡೆಯಲು ಪ್ರಯತ್ನಿಸಿ ಅದನ್ನು ಪಡೆಯಬೇಕು. 

