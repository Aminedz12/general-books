{
\makeatletter
\def\@makechapterhead#1{%
  \vspace*{5\p@}%
  {\parindent \z@ \raggedright \normalfont
    \ifnum \c@secnumdepth >\m@ne
      \if@mainmatter
        \LARGE\bfseries \@chapapp\space\thechapter
        \vskip 4pt
        \par\nobreak
        \vskip 5\p@
      \fi
    \fi
    \interlinepenalty\@M
    \LARGE\bfseries #1\par\nobreak
    \vskip 15\p@
  }}
  \makeatother
 %\chapter{leVKakara binanxha}

%\lhead[{\footnotesize\fontfamily{txr}\selectfont\thepage}]{{\footnotesize\sl\bfseries leVKakara binanxha}}


\chapter{ಲೇಖಕನ ಮಾತು}



%\phantom{a}

%\vskip  -1.2cm

%\quad 
ಕಾಲ ತಾನೇ ಕಳೆದು ಹೋಗುತ್ತಿರುತ್ತದೆ. ಕಾಲಕ್ಷೇಪವೆಂದರೆ ನಮಗಿರುವ ಕಾಲದ ಸದುಪಯೋಗ, ಮನಸ್ಸಿಗೆ ಮುದನೀಡುವ, ಬುದ್ಧಿಯನ್ನು ಚುರುಕಾಗಿಸುವ, ದೈನಂದಿನ ಜಡತ್ವವನ್ನು ದೂರವಾಗಿಸುವ ಕ್ರಿಯೆ. ಸಂಗೀತ, ನೃತ್ಯ , ವಾಯುವಿಹಾರ, ಸಾಹಿತ್ಯಗೋಷ್ಠಿ ಇವೆಲ್ಲ ಕಾಲಕ್ಷೇಪಗಳೇ. ಗಣೀತವನ್ನೂ ಅಂತಹ ಸದುಪಯೋಗಿಯಾಗಿ ಮಾಡಿಕೊಳ್ಳಬಹುದು. ಈ ಹೊತ್ತಿಗೆಯು ಈ ನಿಟ್ಟಿನಲ್ಲಿ ಸಹಕಾರಿಯಾಗಬಲ್ಲದು. 

ಕರ್ನಾಟಕ ರಾಜ್ಯ ವಿಜ್ಞಾನ ಪರಿಷತ್ತು ನನ್ನ ಗಣಿತ ಬರಹಕ್ಕೆ ಬಹಳ ಪ್ರೋತ್ಸಾಹ ನೀಡುತ್ತಾ ಬಂದಿದೆ. ಈ ಕಿರು ಹೊತ್ತಿಗೆಯನ್ನು ಪ್ರಕಟಿಸುತ್ತಿದೆ. ಕರಾವಿಪದ ಅಧ್ಯಕ್ಷರು. ಉಪಾಧ್ಯಕ್ಷರು, ಕಾರ್ಯದರ್ಶಿಗಳೇ ಆದಿಯಾಗಿ ಎಲ್ಲಾ ಪದಾಧಿಕಾರಿಗಳಿಗೂ ನನ್ನ ಹೃದಯಪೂರ್ವಕ ನಮನ. ಪುಸ್ತಕ ಆಯ್ಕೆ ಸಮಿತಿಯ ಅಧ್ಯಕ್ಷರು, ಸದಸ್ಯರುಗಳಿಗೆ ಕೃತಜ್ಞತಾಪೂರ್ವಕ ಆಭಾರ. 

ಪುಸ್ತಕಕ್ಕೆ  ಚಿಕ್ಕ ಚೊಕ್ಕ ಮುನ್ನುಡಿ ಬರೆದುಕೊಟ್ಟ ಆತ್ಮೀಯ ಕಾಗದ ಕುಶಲಿ ಶ್ರೀ ವಿ. ಎಸ್. ಎಸ್.  ಶಾಸ್ತ್ರಿಯರಿಗೆ ನಮನಗಳು.

ಓದುಗರಿಗೆ ಸೂಚನೆ: 
\begin{enumerate}
\item ಓದುವ ಮುನ್ನ ನೋಟ್ ಬುಕ್, ಲೇಖನಿ ಸಿದ್ಧವಾಗಿರಲಿ 
\item ಈ ಪುಸ್ತಕದಲ್ಲಿ ಏನನ್ನೂ ಬರೆಯಬೇಡಿ 
\item ಉತ್ತರವನ್ನು ಪಡೆಯಲು ಸಾಧ್ಯವಾದಷ್ಟು ಯತ್ನಿಸಿ
\item ಉತ್ತರ ಲಭಿಸಿದ್ದಲ್ಲಿ ಸರಿಯೇ ಎಂದು ನೋಡಲು ಉತ್ತರದ ಹಾಳೆ ಓದಿ 
\item ಸಮಸ್ಯೆಗಳನ್ನು ಬಿಡಿಸಲು ಹೆಚ್ಚಿನ ಮಟ್ಟದ ಗಣಿತ ಜ್ಞಾನ ಬೇಕಿಲ್ಲ. ಪ್ರೌಢಶಾಲಾ ಮಟ್ಟ ಸಾಕಾಗುತ್ತದೆ, ಬೇಕಾಗಿರುವುದು ಆಸಕ್ತಿ, ಪರಿಶ್ರಮ 
\end{enumerate}
ಓದುಗರ ಹಿಂಬರಹಕ್ಕೆ ಲೇಖಕನ ಸ್ವಾಗತವಿದೆ 

\begin{flushright}
{\bf ಬಿ. ಕೆ. ವಿಶ್ವನಾಥರಾವ್}\\
ಗಣಿತ ಸಂವಹನಕಾರ\\
ನ:{\rm 94}, {\rm 30}ನೇ ಅಡ್ಡ ರಸ್ತೆ, \\
ಬನಶಂಕರಿ {\rm 2}ನೇ ಹಂತ, ಬೆಂಗಳೂರು: {\rm 70}\\
ದೂರವಾಣಿ: {\rm 080-26739273}
\end{flushright}


}



\newpage

%~ \begin{center}
%~ \textbf{{\LARGE pariviDi}}
%~ \end{center}
%~ 
%~ \begin{enumerate}
%~ \item[{\rm 1)}] purAtana kAlada BAratiVya gaNitajacnxru
%~ \item[{\rm 2)}] kelavu Adhunika gaNitajacnxru (shirxVnivAsa rrAmAnujanf, Di. Arf. kaperxVkara DA. si. enf. shirxVnivAsaliyayxMgArf)
%~ \item[{\rm 3)}] gaNita adara pArxmuKayx matutx boVdhisuva vidhAna
%~ \item[{\rm 4)}] viroVthoVkitxgaLu 
%~ \item[{\rm 5)}] heVtAvxBAsugaLu
%~ \item[{\rm 6)}] mAyA cwkada itihAsa matutx racane
%~ \item[{\rm 7)}] DUyxrara mAyA cwkada visheVSate
%~ \item[{\rm 8)}] samaseyxya suLiyalilx (nATaka)
%~ \item[{\rm 9)}] Ayalxrana sUtarx 
%~ \item[{\rm 10)}] pAsakxlana tirxBuja
%~ \item[{\rm 11)}] PamaRna aMtima parxmeVya
%~ \item[{\rm 12)}] visheVSa samiVkaraNa
%~ \item[{\rm 13)}] visheVSa mAyAcwka
%~ \item[{\rm 14)}] oMdu saraLa samaseyx
%~ \item[{\rm 15)}] samaseyxya suLiyalilx
%~ \item[{\rm 16)}] gaNitadalilx vidhAnaveV muKayx utatxra muKayxvalalx
%~ \item[{\rm 17)}] kutUhala samaseyxgaLu
%~ \item[{\rm 18)}] kelavu sAvxrasayxkara GaTanegaLu
%~ \item[{\rm 19)}] kananxDada gaNita garxMthagaLu
%~ \item[{\rm 20)}] BAratiVya gaNitajacnxra paTiTx
%~ \item[{\rm 21)}]  pAshAcxtayx gaNitajacnxra paTiTx
%~ \item[{\rm 22)}] veVdagaNitada sUtarxgaLu
%~ \item[{\rm 23)}] pAriBASika padagaLu
%~ \item[{\rm 24)}] garxMtha QuNa
%~ \item[{\rm 25)}] ideV leVKakara parxkaTita kaqtigaLu
%~ \end{enumerate}


