\chapter{``Sanskrit is dead'', and that’s okay}\label{chapter4}\index{Sanskrit}

\Authorline{Naresh P. Cuntoor}
\lhead[\small\thepage\quad Naresh P. Cuntoor]{}
\section*{Abstract}

In this paper a critical review of Pollock’s hypothesis of the death\index{death} of Sanskrit is presented. Using a tenuous analogy with European languages and history, his hypothesis advocates Sanskrit’s symbiotic relationship\index{symbiotic relationship} with royal power as a means of ensuring power. It claims that Indian kings exploited aspects of Sanskrit such as grammar\index{grammar} and śleṣa to enhance their political\index{power!political} status. In turn, royal patronage\index{patronage} is said to favor Sanskrit over vernacular languages. As a result Sanskrit enjoyed a dominant status until vernacular\index{vernacular} languages grew in strength and acceptability. This narrative, he avers, is inspired by the European history of cosmopolis in attempting to establish a similar status for Sanskrit, and later for Kannada.\index{Kannada} Once Sanskrit ceded ground to vernacular languages, it faced precipitous decline to face death by the turn of 17th century CE, despite valiant revival efforts\index{revival efforts}\index{Sanskrit!revival of} made by the Muslim rulers\index{Muslim!rulers} in northern India, and later by the English colonialists, he opines. Further, Sanskrit’s death is considered insignificant by him because there was little original knowledge production in Sanskrit as it is. I shall describe several instances of empirical data being sacrificed seemingly because of the necessities of narrative-building. The paper raises a few questions for discussion as part of future work.

\section*{Introduction}

One of the foundational languages of Indian civilization, Sanskrit continues to have a deep and wide influence in today’s India through language, culture and science. Its reach in all these dimensions have long transcended geographical boundaries\index{geographical!boundary} of the Indian subcontinent. Lulled perhaps by the deep and continuing influence of Sanskrit or the self-evident nature of Sanskrit, Indian scholarship has failed to articulate the position of Sanskrit with sufficient clarity. Even when its position has been articulated in a manner which defies belief and evidence (Pollock 2006), the arguments have not been addressed thoroughly. In this paper, I shall present a critical summary of Pollock’s theory regarding the death of Sanskrit based primarily on the book {\sl The Language of the Gods in the World of Men}\index{Language of the Gods in the World of Men, The@\textsl{Language of the Gods in the World of Men, The}} (Pollock 2006) and based on “The Death of Sanskrit” (Pollock 2001).

{\sl The Language of the Gods in the World of Men},\index{The Language of the Gods in the World of Men@\textsl{The Language of the Gods in the World of Men}} as the title suggests, presents the death of Sanskrit, and argues that its death is not a significant loss as far as knowledge is concerned. This is for two main reasons: there was a vigorous exchange of ideas within India and with the larger ancient world. So Indian knowledge, if any, survives in various forms. Second, Indian knowledge, in large measure was non-existent till the advent of the modern era.

This paper is divided into four main sections: 

\begin{enumerate}
\item {\sl Preliminaries}: Here I shall discuss the form of language used in Pollock (2006) and its implications, and present a brief historical background of Greek and Latin\index{Latin!(Vulgar, Medieval, Church)} development.
\item  {\sl Sanskrit as a source of power}\index{Sanskrit!as a source of power}:\index{power!Sanskrit as a source of} In this section, the language characteristics that are considered a source of power are described. 
\item  {\sl Sanskrit cosmopolis}\index{Sanskrit!cosmopolis}: The concept of cosmopolis and its Sanskrit nature are summarized.
\item  {\sl The death of Sanskrit}: Finally, the manner of Sanskrit’s death is described.
\end{enumerate}
\newpage

\section{Preliminaries}

Before addressing the substance of Pollock’s arguments, his writing style deserves notice. In academic literature, the usage of complicated writing style may be completely natural, even if it is unnecessary. I am reminded of the editor R.H.Fiske\index{Fiske, R H} (author of the {\sl Dictionary of Unendurable English}\index{Dictionary of Unendurable English@\textsl{Dictionary of Unendurable English}}) who considered such a style of writing as one that encouraged illiteracy. The reader seeks clarity when startling claims such as the death of Sanskrit\index{hypothesis!death of Sanskrit} are made. One is interested in understanding the nature of the death being described, and the evidence presented in its support. Redundant words and meaningless tautologies (e.g., “when literature {\sl became literature}”, Pollock (2006:4) hardly make the writing clear. Style cannot be ignored completely either, especially because Pollock’s writings merit serious consideration by traditional scholars who need to thoroughly understand the arguments before attempting an informed response. Using a convoluted writing style\index{convoluted!writing style} (as Pollock does) is hardly the best way of conducting a conversation.

It could be argued that making the writings accessible to traditional Indian scholars is the work of others including multilingual Indian scholars. There have been such attempts, e.g., K.V. Akshar’s\index{Akshar, K V} monograph in Kannada\index{Kannada} (Akshar 2003), which describes two of Pollock’s papers that are relevant to the present discussion. At the end of the monograph, P. Chandrashobhi writes an epilogue which re-summarizes Pollock’s arguments and makes mild criticisms at the end. Now this is not to say that writing style is the only (or the main) factor preventing sharper academic discourse. But to the extent that it does, academic discourse has remained limited and bounded to western circles. 

I shall now turn to Pollock’s hypothesis regarding the death of Sanskrit. The rather dramatic evocation of ‘death’ has been criticized by Hanneder\index{Hanneder} using several examples from Kashmir, Vijayanagara\index{Vijayanagara} and modern day India to show its untenability. I shall return to the metaphor of death later.

It is instructive to see a few examples of the immediate impact of Pollock’s work. Academic work by influential scholars inevitably creates amplifying echoes. For instance, even the title of Kaviraj(2005)\index{Kaviraj, Sudipta} makes it more dramatic by calling it “the sudden death.” The arguments quickly become accepted as “compelling” accounts of the “Sanskrit Cosmopolis” (Cox\index{Cox, Whitney} 2011). The theory once completely accepted, becomes a foundation for further theorizing. Further work (Gould\index{Gould, Rebecca} 2008) unabashedly praise the theory and its proponent as “brilliant”, “great” and in even more effusive terms which seem jarring in academic papers. These papers are mentioned as illustrative examples of how Pollock’s arguments have quickly become accepted theory (notwithstanding a few critical academic reviews). The theory never makes a pretense of being a hypothesis\index{hypothesis}\index{hypothesis!death of Sanskrit} that needs to be justified and vetted by knowledgeable peers. (Hypotheses are subject to empirical scrutiny. Those that withstand scrutiny become theories. Hypotheses that are contradicted by data are discarded. This norm seems to have been completely ignored in the present case). 
\medskip

{\bf Background: Greek and Latin}\index{Greek}\index{Latin}\index{Latin!(Vulgar, Medieval, Church)}
\smallskip

No discussion of the history of the West can ignore the importance of ancient Greece because of its contribution in the development of literature, science, rule of law and so on. In this section, I shall briefly summarize the history of Greek and Latin based on the description in Freeman\index{Freeman, Charles} (2004). In this paper, the description is superficial because the goal is limited to providing a comparative basis for the description of Sanskrit that follows. 


The Greek or Hellenic ideal is perhaps best personified by Homer’s\index{Homer} epics {\it the Iliad}\index{Iliad@\textsl{Iliad}} and {\it the Odyssey}\index{Odyssey@\textsl{Odyssey}}. Homer described the importance of virtue in citizens and its necessity for developing virtuous city-states. The Hellenic ideal\index{Hellenic ideal} placed great emphasis on virtue in all aspects of human life. At this stage of civilization, Greece was a collection of city-states ({\it poleis}). The Greek city-states retained their separate identity despite their close connection, as exemplified in Herodotus’\index{Herodotus} characterization of “the same stock and the same speech, our shared temples of the gods and religious rituals, our similar customs.”\index{temple!customs}

Among the Greek city-states, two groups of people, the Spartans and the Athenians are important because of their strength. Rivalry between the two groups resulted in the Peloponnesian War\index{Peloponnesian War} in the 5th c.\ BCE in which Athens\index{Athens} was destroyed. Sparta emerged victorious as the pre-eminent force in the region. Shortly thereafter, the nature of city-states saw a dramatic change. Following long and persistent wars, the city-states could not live up to their earlier ideal. After seeing his own teacher murdered, even Plato saw a need for a reformulation of the idea of democracy.\index{democracy} City-states lost their identity and the new political entity that took their place was larger. Citizens, whose virtues were earlier celebrated, now became subjects. Greek became the common language ({\sl Koine})\index{Koine}. This was the nature of the Greek\index{Greek} cosmopolis\index{cosmopolis} which began around the time of Alexander’s death. Power and subjugation may be said to defining characteristics of the newly evolved cosmopolis.

About a century after Alexander,\index{Alexander} the Punic Wars\index{Punic Wars} were a series of three large wars between Rome and Carthage.\index{Carthage} During the course of these wars, Rome gradually established its supremacy, became the new dominant empire and absorbed Greece (among several other parts of Europe and Africa) within it. With Roman victory, Latin spread and became the common language in the empire. A language that was earlier largely confined to central Italy, Latin gradually became the dominant language of large parts of Europe and Africa - especially the western Mediterranean. Regions that previously had no common language became linguistically connected by Latin.

On the other hand, in eastern Mediterranean\index{Mediterranean} (e.g., Syria, Egypt) the Roman Empire\index{Roman empire} built strategic alliances with local kingdoms instead of establishing their own rule. In these regions, Greek had a much deeper root and would maintain its preeminence till the arrival of Islam. In this region, Latin did not make significant inroads compared to its western counterpart.

With the fall of the Roman Empire in 5th century CE, Latin lost its political backing. Lacking central control, regional variations took hold in Latin over the next 4-5 centuries. This phase of Latin, marked by its non-standard nature, is called Vulgar Latin.\index{Latin!(Vulgar, Medieval, Church)} Gradually different branches of Vulgar Latin evolved into Romance languages\index{Romance languages} (the term refers to their Roman origin). These became today’s French, Spanish, Portuguese and other European languages\index{European!languages}. During this period of evolution of Vulgar Latin, another form of Latin, called Medieval Latin evolved mostly as a literary language. Its grammar\index{grammar} and syntax changed significantly as it had now become a scholarly language, rather than a spoken language. During the middle Ages, its usage declined further, as Romance languages began to flourish. Around this time, in the 13th century CE, Dante\index{Dante} made a clean break, shunned the use of (Medieval) Latin which had become a language of the educated, and used Italian, the language of the masses, to compose poetry. Church Latin however, continued to be used in liturgy.\index{liturgy}

Thus Latin,\index{Latin!(Vulgar, Medieval, Church)} which began as a language in a small region in Italy, acquired dominant status with Roman victories and became the common language of the Roman Empire - especially in the western part of the empire. It did not (possibly could not) seek to displace or challenge the position of Greek in eastern Mediterranean. The eastern empire which became the Byzantine Empire\index{Byzantine Empire} continued to use Greek even though the official language\index{official language} was Latin. (Latin’s official status in the Byzantine Empire was abolished in 7th c.\ CE, restoring Greek’s status). Much of this region would eventually yield to Islamic influence. In the western region, Latin eventually evolved into Romance languages because of a lack of central control.

\section{Sanskrit as a Source of Power}\index{power}\index{power!Sanskrit as a source of}

A recurring theme in Pollock (2006) is identification (or superimposition) of events in time that are marked by profound changes in the premodern Indian (South Asian) landscape. The terms “premodern India”\index{premodern!India} and “premodern South Asia”\index{premodern!South Asia} are used interchangeably, although the latter seems to be the preferred term in Pollock (2006) and Pollock 2001. Words, phrases and languages are an inherent source of power in Pollock’s writings. Sanskrit and English terms are re-interpreted as long as they allow for a predetermined conclusion. Two examples of re-interpretation\index{re-interpretation} are briefly described presently. 

\begin{enumerate}
%~ \itemsep=0pt
\item {\bf Sanskrit terms re-interpreted:} Terms such as {\sl pāramārthika sat} \index{paramarthika sat@\textsl{pāramārthika sat}} and {\sl vyāvahārika sat}\index{vyavaharika sat@\textsl{vyāvahārika sat}} are introduced in the sense of “absolute perspective of science” and “traditions of language thought” (Pollock (2006:65) without providing any reason for forcibly re-interpreting these terms to ascribe notions that are not supported by their typical use in existing literature. The purpose of using these terms in the newly interpreted sense is clear enough. The Sanskrit terms - and their stated Latin equivalents (regardless of the correctness in equivalence) - provide a veneer of plausibility to support his explanation of Buddhists’\index{Buddhist} use of Sanskrit in scriptural texts\index{texts (Sanskrit/Indic)} though it is not used in other non-scriptural texts. The two forms of {\sl sat} take on new meaning, {\sl artha}\index{artha@\textsl{artha}} becomes power (Section 3), and so on.


\item {\bf English terms re-interpreted:} Terms such as {\sl aestheticization of power\index{aestheticization of power}, orientalism,\index{Orientalism (\textsl{passim})|)} cosmopolis}, and so on are familiar to the Western audience. Aestheticization and orientalism have a long history in India-related discourse. For now, I will focus on the term {\sl cosmopolis}.\index{cosmopolis} Pollock explicitly states that the notion of ‘cosmopolis’ in its usual sense of the word (i.e., in the European context) is nothing like the sense with which he uses it in the Indian context. However, the connotation\index{connotation} of the term ‘cosmopolis’ as used in the European context is applied directly to support the hypothesis of Sanskrit as a source of power.\index{hypothesis!Sanskrit as source of power}\index{power!Sanskrit as a source of} In the description of Pollock (2006), power takes center-stage in the “Sanskrit cosmopolis”,\index{Sanskrit!cosmopolis} just as it did in the European (sociopolitical) cosmopolis. In other words, the disclaimer of terms having different meanings in the European and Indian contexts becomes a distinction without a difference. Just as in the case of {\sl pāramārthika sat} and {\sl vyāvahārika sat}, the wildly differing connotations of cosmopolis in two contexts as used make little difference.
\end{enumerate}

After ascribing new meanings to existing terms, the original meanings of the terms are invoked to arrive at startling conclusions. For example, though the term cosmopolis is used in a very different sense from its usual meaning in the European historical context, its implication - of certain empires and languages establishing domineering influence - is stated as the result of creation of the cosmopolis. Sanskrit becomes the dominating force used by Hindu kings (e.g., kings of the Vijayanagara\index{Vijayanagara} and Śrīvijaya\index{srivijaya@Śrīvijaya} kingdoms) seeking to establish their political and social supremacy by quashing existing local languages. The form of domination and power also differs from what is observed in the European context. Several caveats of how the sense of cosmopolis and power differs in the Indian and European contexts are indeed described at length. Yet the conclusion strips the caveats away and presents a narrative of Sanskrit being “an oppressive power that had to die” before new or local languages had a chance to flourish.

In the remaining part of this section, I shall discuss the nature of power, the tools of power ascribed to Sanskrit\index{Sanskrit} and the resulting implications.
\newpage


\subsection{Form of power in European and Indian contexts}
\vskip -4pt

Section 1 described the origin, spread and decline of Latin. Let us now turn to Sanskrit and the power ascribed to it. In this section, I shall first summarize related arguments made in a few different places in Pollock (2006) and then briefly discuss a few questions that arise as a result of the formulation of the hypothesis in Pollock (2006). 

No single local source can be traced as the original home of Sanskrit, neither can any military-political or military-religious force be said to be responsible for its spread (Pollock 2006:262). Further, when Medieval Latin\index{Medieval Latin}\index{Latin!(Vulgar, Medieval, Church)} faced a precipitous decline in literary production, no such fall can be detected in Sanskrit literature. Poets, philosophers and scientists continued to use Sanskrit in the second millennium. So, clearly, a straightforward equivalence between Sanskrit and Latin\index{Sanskrit!and Latin}\index{Latin!and Sanskrit} is non-existent, and no attempt at establishing such equivalence is made in Pollock (2006). But power and rupture of a different kind is described.

The first written literature in Latin is traced to 240 BCE when a Greek play was adapted into Latin by a Greek slave, Livius Andronicus.\index{Livius Andronicus} A little later, Naevius composed an epic poem about the first Punic war. The Sanskrit equivalent offered here is as follows. Vedic literature has taken form in the first millennium BCE\@. With writing having been invented in the third century BCE, a new form of literature {\sl kāvya} is said to take root. Until the first century CE, all literature is primarily Vedic, which retains an other-worldliness and isolation (Pollock 2006:39). No non-Vedic literature is allowed to take shape in Sanskrit because Sanskrit intellectuals\index{Sanskrit!intellectuals} are “bounded and limited”. While Sanskrit\index{Sanskrit} is a source of sacred power, Prakrit\index{Prakrit} is used for political and other {\sl laukika} (worldly) purposes. For example, the Brahmanical Sātavāhana kings\index{Satavahana kings@Sātavāhana kings} used Prakrit for administration and literature (e.g., {\sl Gāthāsaptaśatī}\index{Gathasaptasati@\textsl{Gāthāsaptaśatī}}). On the other hand, the Shakas, who were foreign rulers, used and promoted Sanskrit at the cost of Prakrit (Pollock 2006:72). In this sense, Sanskrit becomes a source of power\index{hypothesis!Sanskrit as source of power}\index{power!Sanskrit as a source of} that a foreign ruler can appropriate to assert authority.

Now if Sanskrit was “bounded and limited” in its scope and confined to sacral rites, it needed a catalyst to infuse newness and broaden its appeal. Foreign rulers such as Shakas,\index{Shakas} and much later, the Vijayanagara kings provided such an impetus in the political sense when they sought to occupy power using Sanskrit as a means of establishing their supremacy. And rulers such as Rudradāman and Buddhist poets\index{Buddhist!poets} provided the needed catalyst to liberate Sanskrit for its oppressive masters. Just as an outsider, Livius who was Greek and a slave, was responsible for the introduction of Latin\index{Latin!(Vulgar, Medieval, Church)} literature, outsiders such as Shakas and Buddhists were responsible for introducing Sanskrit as a medium for {\sl kāvya}.

I shall use Thapar\index{Thapar, Romila} (2015)’s description of Rudradāman\index{Rudradaman@Rudradāman} to provide context. With the advent of Kushanas,\index{Kushanas} the Shakas had moved further south into western India. Here Rudradāman was a mid-second century BCE Kshatrapa\index{Kshatrapa} king. His Junagarh inscription\index{Junagarh inscription} in Sanskrit, the earliest of its kind that has been discovered, provides a detailed account of his activities. Instead of using Prakrit, why did he use Sanskrit in his inscriptions? Thapar’s account provides at least two reasons for Rudradāman’s use of Sanskrit, without suggesting a definitive answer. He could have used Sanskrit to endear himself to orthodoxy to establish his legitimacy as a ruler. Or he could have anticipated the winds of change which was bringing about parallel patronage\index{patronage} in court circles. In Pollock (2006), no such hesitation is even hinted at in arriving at a conclusion that Rudradāman used Sanskrit to establish his legitimacy as a ruler. Similarly, Buddhist poets who had already composed the {\sl Jātaka} tales\index{Jataka tales@\textsl{Jātaka tales}} in Pali\index{Pali}, switched to Sanskrit to compose literature (e.g., Aśvaghoṣa’s {\sl Buddhacarita}\index{Buddhacarita@\textsl{Buddhacarita}}\index{Asvaghosa@Aśvaghoṣa}). 

\subsection{{\sl Praśasti} and power}\index{prasasti@\textsl{praśasti}}\index{power!Sanskrit as a source of}

Rudradāman is important for another reason. The Junagarh inscription\index{Junagarh inscription} is considered an early example of {\sl praśasti}. In this discussion, I shall juxtapose the discussion of {\sl praśasti} in Pollock (2006) and Thapar (2015). According to Thapar (2015), {\sl praśasti} was still evolving as a literary style at the time, and would find clearer form in Aśvaghoṣa’s {\sl Buddhacarita}. A marginal activity initially, it “later” became important enough to “affect the structure of the political economy”\index{political!economy} (Thapar 2015:225). (Thapar’s account does not specify the time frame to which “later” refers). Pollock considers that {\sl praśasti} attained final form by the time of Pallavas\index{Pallavas} (7th c.\ CE). It is interesting to note that {\sl praśasti} as a form of literature is accorded significant importance in Pollock (2006), whereas the historical account in Thapar is less eager to embrace its importance (“...one hesitates to take it literally” -Thapar 2015:283-4).

{\sl Praśasti} plays a central role in Pollock (2006) especially when trying to establish equivalence with Latin.\index{Latin!(Vulgar, Medieval, Church)} Moreover the occurrence of {\sl praśasti-s} is portrayed as a “sudden”, “startling, nearly simultaneous” event not just in the Indian subcontinent but in Southeast Asia as well. On the other hand, Thapar\index{Thapar, Romila} (2015) describes a gradual development of {\sl praśasti} literature. In the absence of evidence of any large scale military intervention before that of the Cholas\index{Cholas} in 11th c.\ CE, mercantile and religious interactions\index{mercantile and religious interactions} are said to be the basis for the spread of Sanskrit to South East Asia in early centuries CE\@. During this time, according to Pollock (2006), Sanskrit becomes the key to power because of the proliferation of {\sl praśasti}.\index{power!Sanskrit as a source of} The historical description of politics and trade in Thapar (2015) mentions the presence of Indian merchants in Myanmar, Oc-eo (near the Gulf of Siam)\index{Gulf of Siam} and so on. More interestingly, similarities between pre-existing cultures in Southeast Asia and South India in activities such as rice cultivation\index{rice cultivation} and burial are described. Such exchanges “evolved into trade and incipient urbanism.” Again, a deeper, more long-lasting, connection is discussed in Thapar (2015) unlike Pollock (2006). Whereas Thapar (2015) discusses a measured, historical account of events in India and Southeast Asia, Pollock (2006) presents a larger narrative without describing alternative possibilities.

Let us briefly look at the case of {\sl praśasti} in Kannada as described in Pollock (2006). The earliest evidence of writing in Kannada is available from the 5th c.\ CE, but the earliest {\sl praśasti}\index{prasasti, earliest@\textsl{praśasti}, earliest} and earliest literature is said to be from the 9th c.\ CE - both intimately tied to the royal court. In other words, analogous to the case of Sanskrit, {\sl kāvya} and {\sl praśasti} is said to emerge at the same time. In the description of the {\sl praśasti} of Krishnaraja\index{Krishnaraja} (Pollock 2006:334), it is said that this is the first of its kind in Kannada. Among the king’s titles, three types of words are seen: (1) Kannada compound words, (2) Sanskrit compound words and (3) words that are formed by compounding a Kannada word and a Sanskrit word by means of an {\sl ari-samāsa}\index{ari-samasa@\textsl{ari-samāsa}} which is generally frowned upon. This mixing of form of titles is made out as a “careful balancing” of “globalizing and localizing\index{localizing} registers”.

A late 9th c.\ CE inscription of King Erayappa\index{Erayappa (King)} is then discussed as an example of path-breaking vernacularity in Kannada country. The quoted passage of long Sanskrit words interspersed with Kannada words is described as an example of a new lexicon, a new “mode of representation” (Pollock 2006:335). The discussion does not mention however that part of the phraseology …{\sl nirmala-tārāpati}… is reminiscent of the earliest available Kannada inscription (i.e., the Halmidi inscription\index{Halmidi inscription}). (I shall briefly return to the Kannada writing discussion in Section 2.4.) 

In Pollock (2006), the description of the development of Kannada language and power may be summarized as follows: 
\begin{enumerate}
\itemsep=0pt
\item Though evidence of writing in Kannada is available from the 5th c.\ onwards, the earliest works of Kannada literature in a distinctly recognizable form is not available till the 9th century.
\item In the 9th century, two forms of literature arose almost simultaneously – {\sl kāvya}\index{kavya@\textsl{kāvya}} and {\sl praśasti}.
\item Kannada literary development and its royal patronage\index{patronage} had a symbiotic relationship.\index{symbiotic relationship} 
\end{enumerate}

\subsection{Tools of Power}

As discussed above, Sanskrit is presented as a source of legitimizing power. The nature of its power\index{power!Sanskrit as a source of} is further described by the following characteristics:
\begin{enumerate}
\itemsep=0pt
\item Lack of geographical\index{geographical} grounding\index{geographical!grounding}
\item Written literature\index{literature}
\item Aesthetics\index{aesthetics} in Sanskrit literature
\item Grammar\index{grammar}
\item {\sl śleṣa}\index{slesa@\textsl{śleṣa}}
\end{enumerate}

These factors are briefly explained next. 

Sanskrit,\index{Sanskrit} it is said, was “at home everywhere, and nowhere.” It has no single, traceable place of origin. This meant that no single group of people can claim ownership of Sanskrit. But those who want power use it where and when needed. In this sense, Sanskrit finds a home everywhere. Without a clearly identifiable geographical localization,\index{geographical!localization} on the contrary, it finds a home nowhere.

Writing plays a crucial part in Pollock’s theory. To begin with it, a new term, {\sl literization}\index{literization} is introduced to describe the phase when a language acquires a written form. ({\sl Literarization}\index{literarization}, on the other hand, refers to the phase when a language starts producing literature). Three reasons are described in the introductory chapter of Pollock (2006): (a) Writing claims an authority that orality cannot. Speaking is natural, but writing is not. (b) Writing is richer than oral,\index{oral tradition} and allows for the examination of language itself, (c) “Writing makes possible the production of history of a sort the oral is incapable of producing”. All this is stated without elaboration.

Using an idea that was promoted in the 1930’s, Pollock (2006) opines that Sanskrit - specifically works such as Kālidāsa’s {\sl Raghuvaṁśa}\index{Kālidāsa@Kalidasa}\index{Raghuvamsa@\textsl{Raghuvaṁśa}} and various {\sl praśastis} - attracted power because of their aesthetic nature. The theory of aestheticization of power\index{aestheticization of power} and politics has been criticized on several grounds (Jay  1992).\index{Jay, Martin} A general criticism of the theory however, is insufficient to test the validity of the theory as applied in the specific contexts of Sanskrit and Kannada. 

For a language to acquire power,\index{power!Sanskrit as a source of} in addition to written status, grammar\index{grammar} is said to be a prerequisite. Sanskrit managed to attract power because of its highly developed grammar. In turn the political power was interested in ensuring grammar’s place. Bhoja’s\index{Bhoja/Bhojadeva} patronage\index{patronage} of poetry and grammar is described in detail to imply that his strength and position of power was at least in part because of the patronage. Statements such as “{\sl rājani kavau sarvo lōkaḥ kaviḥ syāt}” are quoted as examples of kings using proper language as a means of power. Further, Patañjali’s\index{Patanjali@Patañjali} adage of “{\sl ekaḥ śabdaḥ saṁyak jñātaḥ śāstrānvitaḥ suprayuktaḥ svarge lōke kāmadhuk bhavati}” is used to justify the relevance of proper language to {\sl dharma}\index{dharma@\textsl{dharma}} (and hence power) (Pollock 2006:183). Languages acquire such political power when their grammar reaches maturity.

The use of {\sl śleṣa} attained an impressive level of sophistication in Sanskrit. Its evolution is then connected to the power of politics using arguments that echo the theory of aestheticization of power\index{aestheticization of power}. Similarly {\sl `alaṅkāra'\index{alankara@\textsl{alaṅkāra}}} becomes equally powerful in their ability to influence political power through their use in {\sl praśasti-s}. Sanskrit then becomes a tool to enhance reality through its unique abilities in {\sl ślēṣa}, other {\sl alaṅkāra-s} and so on. The king then exploits this power to legitimize and maintain his power.

\subsection{Some Questions}

In the analysis in Pollock (2006) described above, several factors are completely or largely ignored. I shall mention a few items below. The discussion in this section will be brief in order to limit the scope of the paper to {\sl pūrva-pakṣa} and not expand it into an {\sl uttarapakṢa}. The questions are raised here because they are an extension of the {\sl pūrvapakṣha} arguments themselves. 

\subsubsection{{\sl A long tradition of oral literature}}

Only written literature is recognized by him as literature for consideration, and not orally transmitted literature. Three reasons are stated for this treatment (Section 2.3) – authority, richness and ability to produce history. Pollock (2006) states that writing possesses these three faculties whereas oral tradition does not. If writing is to be accorded such primacy over oral literature, then it is natural for the reader to expect a careful and reasoned justification of the stated position.

{\bf Authority:} It is difficult to understate the impact of the Vedas, {\sl Rāmāyaṇa}\index{Ramayana@\textsl{Rāmāyaṇa}} and {\sl Mahābhārata}\index{Mahabharata@\textsl{Mahābhārata}} in the Indian context. All these forms of literature continued to be transmitted orally for several generations before they were written down. Did their stature or authority change in the minds of people after they were written down?  Pollock (2006) does not address this question.

{\bf Richness:} In addition to the richness of story-telling in the {\sl itihāsa-s},\index{itihasa@\textsl{itihāsa}} the {\sl sūtra} literature\index{sutra literature@\textsl{sūtra} literature} – especially those of Pāṇini, can hardly be faulted for lack of richness, imagination or creativity.

{\bf Ability to produce history:} The argument that writing, unlike oral tradition, can produce history is stated without elaboration in Pollock (2006). It would be interesting to explore this aspect more thoroughly especially in the light of attempts at building history through narratives. For example, songs of the {\sl bhakti} movement in the Dāsa tradition\index{Dasa tradition@Dāsa tradition} provide a glimpse of the society of their time. Doniger\index{Doniger, Wendy} (2009) is another example of historical perspective-taking which relies on oral tradition\index{oral tradition} such as the Veda-s.\index{Veda}
\newpage

\subsubsection{{\sl Mahābhāṣya references to existing literature}}

A central tenet of the thesis of Pollock (2006) is that Sanskrit {\sl kāvya} literature began around the 1st c.\ CE\@. References to older traditions of literature described in the {\sl Mahābhāṣya}\index{Mahabhasya@\textsl{Mahābhāṣya}} are stated in the book, but quickly set aside citing difficulties in dating the {\sl Mahābhāṣya} itself. Whatever difficulties there may be regarding its dating, even taking the late estimates of its period, we are left with more than a century and a half between references in {\sl Mahābhāṣya} to {\sl kāvya}, and to the starting period of {\sl kāvya} as per Pollock (2006). This large difference is difficult to reconcile. 

The question of whether older traditions of {\sl kāvya} to which {\sl Mahābhāṣya} alluded are to be considered as literature in the sense of Pollock, remains unresolved.

\subsubsection{{\sl Pāṇini’s reference to drama}}

In addition to {\sl Mahābhāṣya}’s literary references, Pāṇini {\sl sūtra-}s contain references to dramaturgy\index{dramaturgy} (Krishnashastri\index{Krishnashastri, A R} 2002:28). These do not appear in the analysis in Pollock (2006). Bharata’s\index{Bharata} {\sl Nāṭyaśāstra}\index{Natyasastra@\textsl{Nāṭyaśāstra}} which is the earliest extant work in its field is discussed in Pollock (2006) and its impact on development of literary analysis in Indian languages is described. Several post-Bharata treatises and their influence on analytical texts\index{texts (Sanskrit/Indic)} of vernacular language are discussed as well.

However the silence on references to older texts is somewhat perplexing. For the purposes of establishing the age of beginning of a field, even if the details of the older texts’ discussion are unknown, can the older texts be summarily dismissed, as advanced by the necessities of narrative-building?\index{narrative-building}

\subsubsection{{\sl Regional variations in Sanskrit and Prakrit}}

Sanskrit as described by Pāṇini,\index{Panini@Pāṇini} Kātyayana\index{Katyayana@Kātyayana} and `Patañjali', showed regional variations. From their descriptions, it is clear that the regional differences were not a hindrance to the development or usage of language itself. Similarly, the variety of Prakrits\index{Prakrits, variety of} are well-documented, e.g., in {\sl Śṛṅgāraprakāśa}.\index{Srngaraprakasa@\textsl{Śṛṅgāraprakāśa}} This long-standing diversity within Sanskrit\index{Sanskrit} and Prakrit does not support a centralized, uniform view of the language. Further, Vedic Sanskrit\index{Sanskrit!Vedic} exhibits even greater diversity in forms, as recognized by all-encapsulating descriptions such as “{\sl bahulam chandasi}” by the grammarians.\index{grammar} Given this diversity and equal importance accorded to different usages, the analogy of Latin\index{Latin!(Vulgar, Medieval, Church)}\index{Latin!analogy of} is puzzling as far as power and politics is concerned. 

If indeed language is to be seen through a European lens,\index{European!lens} with a strong power\index{power!Sanskrit as a source of} attribute, how does Pollock (2006) reconcile the absence of attempts to assert dominance of one form of the language over another? 
\vskip -40pt

\subsubsection{{\sl Parallel literary traditions of Sanskrit, Prakrit and Apabhramśa}}\index{Parallel literary traditions}\index{Apabhramsa@Apabhramśa}
\vskip -5pt

As seen in dramas and bilingual poems, poets did not hesitate to simultaneously employ several languages. And we see that Sanskrit,\index{Sanskrit} Prakrit\index{Prakrit} and Apabhramśa\index{Apabhramsa@Apabhramśa} flourished simultaneously. So we are left to reconcile conflicting factors. On the one hand, we observe fluidity in language usage from early times. On the other, Pollock (2006) seeks to identify rigid events of birth and death in fairly dramatic terms. 

Even in modern times, music compositions\index{music compositions} exhibit an interesting linguistic fluidity. Compositions that are generally categorized as Kannada or Telegu compositions (e.g., several compositions of Muttayya Bhāgavatar\index{Muttayya Bhagavatar@Muttayya Bhāgavatar} and Mysore Vāsudevācārya),\index{Mysore Vasudevacarya@Mysore Vāsudevācārya} are primarily in Sanskrit, except for a stray verb or so. Further, renditions of Prakrit text into Sanskrit have traditionally be called {\sl chāyā}\index{chaya@\textsl{chāyā}} rather than {\sl anuvāda}\index{anuvada@\textsl{anuvāda}} (or translation). Given this fluidity between languages and their usage, one would anticipate a much more difficult task in trying to identify events such as birth and death of their usage. Such nuanced discussion is largely missing in Pollock (2006). 

Similarly, the Kannada/Sanskrit ratios\index{Kannada/Sanskrit ratios} regarding epigraphs of various dynasties (footnote in Pollock 2006:332) requires further elaboration. How are epigraphs classified as Kannada vs Sanskrit? Does the occurrence of a single Kannada word make it a Kannada epigraph?\index{Kannada!epigraph}
\vskip -40pt

\subsubsection{{\sl A long history of Prakrit literature}}
\vskip -5pt

Pollock (2006) considers that the creation of vernacular literature was “intimately related to new conceptions of communities and places”. And its creation is treated as a significant shift that is difficult to explain. It is necessary to better understand the premise of the quest in this regard. If one is to approach the problem of identifying beginnings of literature in a particular language in fairly binary terms, it is necessary to examine the background first. Was the prevailing literature fully in Sanskrit prior to the hypothesized break in literary tradition? This question when posed in the European context seems easier to answer because European literature before Dante,\index{Dante} was composed in Latin.\index{Latin!(Vulgar, Medieval, Church)}  In the Indian context, was literature mono-linguistic before the so-called vernacular millennium?

Besides the parallel usage\index{parallel usage} of Sanskrit and Prakrit in literature, there have been several Prakrit\index{Prakrit} texts (as mentioned in Pollock (2006) as well), e.g., Vimalasūri’s\index{Vimalasuri@Vimalasūri} {\sl Paumacariya}\index{Paumacariya@\textsl{Paumacariya}} (2nd-4th c.\ CE), Guṇāḍhya’s\index{Gunadhya@Guṇāḍhya} {\sl Bṛhatkathā}\index{Brhatkatha@\textsl{Bṛhatkathā}} (earlier than 3rd c.\ CE). Given this tradition, the reason for treating literature in Kannada or other vernaculars as a startling enterprise is difficult to understand.

\subsubsection{{\sl Influence of Prakrit and Apabhramśa on Sanskrit}}\index{Sanskrit!influence of Prakrit and Apabhramsa on@influence of Prakrit and Apabhramśa on}

Pollock mentions that Prakrit and Apabhramśa influenced Sanskrit (Pollock 2006:65). If Sanskrit’s power\index{power!Sanskrit as a source of} emanated from its “proper”ness and geographic transcendence or Latinesque uniformity,\index{Latinesque uniformity} and further, if, Sanskrit intellectuals\index{Sanskrit!intellectuals} were “bounded and limited” - it seems contradictory to expect Sanskrit being influenced by Prakrit or other languages. In the context of Khmer language\index{Khmer language} for example, this is the argument presented - Sanskrit influenced Khmer language, but not the other way around (Pollock 2006:127). If the rulers were indeed trying to find legitimacy through language, they could have allowed for Khmer’s influence on Sanskrit. These factors have not been articulated. 

\subsubsection{{\sl Śṛṅgāraprakāśa’s silence on praśasti}}\index{Srngaraprakasa@\textsl{Śṛṅgāraprakāśa}}\index{prasasti@\textsl{praśasti}}

In its detailed discussion of different types of literature, {\sl Śṛṅgāraprakāśa} (which is referenced in Pollock (2006) as an authority text), does not treat {\sl praśasti} as an all-too serious form of literature. (Neither does the historian Thapar,\index{Thapar, Romila} as mentioned above). If one is to understand {\sl praśasti} as a form of literature that deserves the kind of categorization that Pollock (2006) demands, then the same could be said of other forms of literature, e.g., {\sl bhakti stotra-s}.\index{bhakti stotra@\textsl{bhakti stotra}} After all {\sl stotrakāra-s} in their {\sl phalaśruti}\index{phalasruti@\textsl{phalaśruti}} take themselves as seriously as {\sl praśastikāra-s}\index{prasastikara@\textsl{praśastikāra}} do! 
\vskip -10pt


\subsubsection {{\sl Early writing in Kannada}}
\vskip -4pt

It is surprising to see that a discussion of {\sl vīragal} is almost completely absent from the discussion of writing in Karnataka\index{Karnataka} in Pollock (2006). (It appears once in the book as a footnote (Pollock 2006:330)). Naik and Naik(1948) considers such stone inscriptions to be peculiar to Karnataka, and mentions that almost all inscriptions of its kind are found in Karnataka. The form and composition of {\sl vīragal-s} tend to follow certain structure as described in Naik and Naik. An early form of writing that is so completely distinctive in style and geography would surely merit a longer discussion when considering the history of writing and literature in a region. One is left to wonder if this omission is to suit the necessities of narrative-building\index{narrative-building} rather than empirical data. 
\vskip -10pt

\subsubsection{{\sl Knowledge systems such as mathematics and science in Sanskrit}}\index{knowledge systems}
\vskip -4pt

Literary knowledge begins and ends with {\sl kāvya} and {\sl praśasti} according to Pollock (2006). Surely {\sl śāstra} literature is a valid form of literature too. Varāha\-mihira\index{Varahamihira@Varāhamihira} is briefly mentioned in Pollock (2006). But long traditions of maths literature, their contributions and impact is largely ignored. For example, the Kerala school of mathematics\index{Kerala School of Mathematics} itself had a long tradition that spanned several centuries during the middle ages. The influence of science and maths is arguably quite important when one considers the longevity and sustainability of societies and civilization. In fact, one could argue that access to scientific and technological texts and practices could be beneficial to a king. 

Pollock (2006) does not discuss scientific literature\index{Sanskrit!scientific literature} in any detail as it may apply to creation and consolidation of power. 

\subsubsection{{\sl Rāmāyaṇa’s non-conformity with Pāṇinian forms}}\index{Ramayana@\textsl{Rāmāyaṇa}}\index{Paninian forms@Pāṇinian forms}

If grammar\index{grammar} is considered a prerequisite for literary development, and if the {\sl Rāmāyaṇa} took its final literary form in the 1st c.\ CE much after Pāṇini’s grammar,\index{grammar} why did the upholders of orthodoxy, the Sanskrit intellectuals\index{Sanskrit!intellectuals} who rendered {\sl Rāmāyaṇa}\index{Ramayana@\textsl{Rāmāyaṇa}} not correct the many non-Pāṇinian forms in the text? 

\subsubsection{{\sl Early Indonesian kingdoms}}\index{Indonesian kingdoms}

In the discussion of Sanskrit’s interaction with Southeast Asian languages, a historical account of kingdoms of the region and their relation to the ancient world is missing. Just as the historical account (e.g., Thapar’s\index{Thapar, Romila} account discussed at various points here) can add additional clarity in the Indian context, it is reasonable to expect historical analysis to better illuminate the status of languages and their relationship in Southeast Asia. 

As stated at the outset, the above list is not intended to be detailed or comprehensive because of limitations of scope of this paper. However the questions suggest that the necessities of narrative-building in Pollock (2006) seem to propel the consideration of empirical data, instead of allowing empirical data to guide the narrative and subsequent conclusions. 

\section{Sanskrit Cosmopolis}

It is true that the Sanskrit world had a kind of universalism that deserves to be studied. Pollock (2006) uses the term cosmopolis\index{cosmopolis} to describe this, despite the limitations of the term (and its historical meaning). Historically, cosmopolitan\index{cosmopolitan} referred to “a community of free males” (Pollock 2006:12). The Sanskrit cosmopolis\index{Sanskrit!cosmopolis} had no such characteristic; neither does Pollock (2006) seek to argue from that position. But the term Sanskrit cosmopolis is used to represent three factors: 
\begin{enumerate}
\itemsep=0pt
\item Geographic expanse:\index{geographic expanse} The influence of Sanskrit transcended regional and national boundaries.
\item Political dimension: The assumption is that a global identification required a political dimension. 
\item The role of Sanskrit: In establishing the political and cultural universalism\index{cultural!universalism} (to whatever extent), the thesis is that Sanskrit was a critical lynchpin.
\end{enumerate}
Ancient Indians’ concept of geography shows some degree of diversity, but some boundaries like the Himalayas\index{Himalayas} recur (Pollock 2006:191). Apart from Indians who self-identified as a people of a common land (despite living in different kingdoms), foreign visitors such as the Greeks and Chinese\index{Chinese!travellers} too, recognize this common identity. Literary sources ranging from {\sl the Mahābhārata} to {\sl Kāmasūtra}\index{Kamasutra@\textsl{Kāmasūtra}} display an awareness of cultural unity and diversity across the country. The image of the {\sl Kāvyapuruṣa}\index{Kavyapurusa@\textsl{kāvyapuruṣa}} which recognizes the holistic nature of different styles and media by giving it an anthropomorphic form\index{anthropomorphic form} is described in detail (Pollock 2006:202-3). While it readily recognizes the universal nature of the {\sl Kāvyapuruṣa} depiction, Pollock (2006) considers it to be “finite” and limiting in the sense that “no other exists outside it.” This description is reminiscent of the “limited and bounded” characterization of Sanskrit intellectuals\index{Sanskrit!intellectuals} (section 2.1).

Across the country, despite the geographical diversity\index{geographical!diversity} in which there were differences in Sanskrit usage as attested in the {\sl Mahābhāṣya},\index{Mahabhasya@\textsl{Mahābhāṣya}} Pollock (2006) argues that the uniformity in the {\sl mārga}\index{marga@\textsl{mārga}} discourse is striking. For example, the {\sl vaidarbhī} style\index{vaidarbhi style@\textsl{vaidarbhī} style} in the north was no different from the {\sl vaidarbhī} style in the south. It is not clear why language usage in the sense of preference for certain word forms / verbs as discussed in the {\sl Mahābhāṣya} should affect stylistic choices in composition. A poet using {\sl vaidarbhī} style in a certain part of the country could use word forms which would be distinct from another poet using {\sl vaidarbhī} style in a different part of the country. Pollock (2006) does not seem to make this distinction.

Indian classical musical\index{compositions, classical musical} compositions provide another example of how geographical proximity\index{geographical!proximity} may not be necessary or sufficient to influence composers’ works. The Mysore\index{Mysore} king Jayachāmarajendra Wodeyar’s\index{Wodeyar!Jayachamarajendra@Jayachāmarajendra} compositions (20th c.\ CE) bear a strong resemblance with Muttusvāmi Dīkṣitar’s\index{Muttusvami Diksitar@Muttusvāmi Dīkṣitar} who preceded him by a century. The similarity is apparent in the lyrical sense\index{lyrical sense} (e.g., preference for long compound words), and in the musical sense\index{musical sense} (e.g., keeping the {\sl sāhitya}\index{sahitya@\textsl{sāhitya}} quite tightly bound unlike say, Tyāgarāja).\index{Tyagaraja@Tyāgarāja} This style is recognizably different from Mysore Vāsudevācārya’s,\index{Mysore Vasudevacarya@Mysore Vāsudevācārya} who was the resident musician in the Mysore royal court, and stylistically more strongly related to Tyāgarāja. 

Let us return to the second feature of Sanskrit cosmopolis in Pollock (2006) - the political dimension.\index{political!dimension} The discussion uses {\sl Mahābhārata}\index{Mahabharata@\textsl{Mahābhārata}} as an example to describe the political power\index{power!political} that characterized the Sanskrit cosmopolis. It quotes half a verse from the Bhīṣmaparvan\index{Bhismaparvan@Bhīṣmaparvan} to illustrate the paradox of power. To provide clearer context to this line, two verses are quoted below (Sukthankar\index{Sukthankar, V S} 1942):
\begin{myquote}
{\sl arthasya puruṣo dāso dāsas tvartho na kasyacit |\\
iti satyaṁ mahārāja baddho'smy arthēna kauravaiḥ|| 6.41.36 ||\\
atas tvāṁ klībavad vākyaṁ bravīmi kurunandana |\\
hṛto'smyarthēna kauravyayuddhād anyat kim icchasi ||6.41.37||}
\end{myquote}

Pollock (2006:225) translates the first line as “man is slave to power, but power is slave to no one.”

Now, Bhīṣma\index{Bhisma@Bhīṣma} utters these words after Yudhiṣṭhira\index{Yudhisthira@Yudhiṣṭhira} has explained the reasons for fighting against him, and seeks his blessings to vanquish him. Bhīṣma recognizes the necessity and propriety of Yudhiṣṭhira’s fight and gives his blessing. In doing so, he explains his commitment to Duryodhana\index{Duryodhana} alluding to his prior oath to the throne. In that context (or indeed even in a general context), translating {\sl artha} as power seems rather odd. Especially because Bhīṣma says “{\sl hṛtō'smi arthena},” which may be translated as “I am compelled by {\sl artha}.” Bhīṣma was not compelled by political power to act - but Bhīṣma's own earlier commitment. Or, arguably it could be translated as wealth, in the sense that Bhīṣma earned his livelihood (though this does seem to be a rather weak and excessively literal translation). In short, Pollock seems to use a line from the {\sl Mahābhārata} and attributes a meaning that is textually and contextually difficult to sustain.

This is of course, not to deny that there is a political dimension\index{political!dimension} to the central conflict depicted in the {\sl Mahābhārata}, or in the ancient or medieval Indian kingdoms (or for that matter in modern India). The connection to Sanskrit is not clear, especially when it is justified by questionable translations.\index{questionable translations}

Another aspect of the political dimension\index{political!dimension} described is the support for recitation and propagation of {\sl the Mahābhārata} provided by rulers across the country. The Pollock (2006) itself describes the {\sl Mahābhārata} as “a veritable library of the world” because of its impact and content (Pollock 2006:225). So it would seem natural for rulers to promote such story telling (and perhaps encouraging story tellers in their kingdom to add their stories to the library).  Further, the {\sl Rāmāyaṇa}\index{Ramayana@\textsl{Rāmāyaṇa}} has also seen several variations - so much so that some of the South-east Asian versions diverge significantly from Vālmīki’s version. 

In other words, this version of the cosmopolis seems to have little or no similarities in the political sense with its existing usage. I am not suggesting that Pollock (2006) attempts to use cosmopolitan in its usual sense when applied to Sanskrit, but it would be more helpful to use words which do not require an elaborate justification of why the usual meaning is not the intended meaning for a given term. Incidentally, Hastināpura is translated as the “City of Elephants”\index{City of Elephants@“City of Elephants”} (Pollock 2006:362). Sources such as the {\sl Śabdakalpadruma}\index{Sabdakalpadruma@\textsl{Śabdakalpadruma}} explain the word (using an {\sl aluk-samāsa\index{aluk-samasa@\textsl{aluk-samāsa}}}) as a city founded by Hastin, a king of the Candravaṁśa  lineage.

Next, let us consider the third aspect of Sanskrit cosmopolis in Pollock (2006) - the role of Sanskrit. Having already described the tools of power\index{power!Sanskrit as a source of} ascribed to Sanskrit, this aspect is perhaps discussed in relation to vernacular\index{vernacular} languages, and Sanskrit’s interaction with them. I shall pick Kannada as an example because Kannada is discussed more extensively compared to other languages in Pollock (2006). The earliest writing in Kannada is available from the 5th c.\ CE\@. The first works of {\sl kāvya} in Kannada appear almost five centuries later with Pampa\index{Pampa} who is called the {\sl ādikavi}\index{adikavi (Kannada)@\textsl{ādikavi (Kannada)}} in Kannada. In the intervening five centuries, literature was confined to Sanskrit, Prakrit and Apabhramśa. This, it is argued, was because Sanskrit asserted its power and suppressed literature in vernaculars.

Later, when Sanskrit “cedes literary power to vernaculars”, it provides the framework for not just literary creativity, but also provides the basic structure for Kannada grammar\index{Kannada!grammar} ({\sl vibhakti-s}, etc.).\index{vibhakti@\textsl{vibhakti}} And the two languages shared a two-way exchange. 

Quoting Nāgavarma\index{Nagavarma@Nāgavarma} (Pollock 2006:370), he translates the passage as “...metrical species\index{metrical species} [of Kannada and other vernaculars] have arisen from the three languages Sanskrit, Prakrit and Apabhramsha, and from the half language Paishachi.” Attributing the correct analysis of this passage to TV Venkatachala Sastry,\index{Sastry, TV Venkatachala} Pollock says that it has long been misunderstood. From a plain reading of the phrase “{\sl sarva-viṣaya-bhāṣā-jātigaḷu},”\index{sarva-visaya-bhasa-jatigalu@\textsl{sarva-viṣaya-bhāṣā-jātigaḷu}} a translation of “metrical species” seems worthy of discussion. 

While he admits the {\sl tatsama-tadbhava} relationship\index{tatsama-tadbhava relationship@\textsl{tatsama-tadbhava} relationship} between Sanskrit and Kannada, it seeks to portray a much more strident break between the two languages in the emergence of Kannada literature. To that end, borrowing “metrical species” born from Sanskrit counterparts allows for a dramatic break which the older translation in which “languages born” from the three and a half languages, do not. The self-conscious note with which Kannada creates its own position is then further explained using the examples of {\sl Kavirājamārga}\index{Kavirajamarga@\textsl{Kavirājamārga}} and {\sl Śabdamaṇidarpaṇa}\index{Sabdamanidarpana@\textsl{Śabdamaṇidarpaṇa}} (SMD). Here the self-aggrandizing statement of the SMD is treated seriously, just as {\sl praśasti} statements are. 

Further, Sanskrit’s power in Vijayanagara\index{Vijayanagara} is extolled. Though Kannada\index{Kannada} and Telugu\index{Telugu} were used extensively for administrative purposes (in their respective areas of usage), Pollock (2006) says that the Vijayanagara kings promoted Sanskrit literature and not Kannada literature. Now it is true, many works of Sanskrit were produced in that period - no less than the {\sl Sāyaṇabhāṣya}\index{Sayanabhasya@\textsl{Sāyaṇabhāṣya}} of the Vedas\index{Veda} themselves. But Kannada literary works seem to be ignored (Desai 1936) - e.g., Timmanna’s {\sl Kṛṣṇarāya Bhāratakathāmañjarī}\index{Krsnaraya Bharatakathamanjari@\textsl{Kṛṣṇarāya Bhāratakathāmañjarī}}\index{Timmanna(kavi)@Timmaṇṇa(kavi)} (which is considered a landmark work of Kannada pride), Vīraśaiva\index{Virasaiva@Vīraśaiva} poets Mallaṇṇārya\index{Mallannarya@Mallaṇṇārya} ({\sl Vīraśaivapurāṇa}),\index{Virasaivapurana@\textsl{Vīraśaivapurāṇa}} Virūparāja\index{Viruparaja (poet)@Virūparāja (poet)} ({\sl Tribhuvanatilaka}),\index{Tribhuvanatilaka@\textsl{Tribhuvanatilaka}} Nanjarāju\index{Nanjaraju@Nanjarāju} ({\sl Kumāravyāsana kathe}).\index{Kumaravyasana kathe@\textsl{Kumāravyāsana kathe}}

Sanskrit is thus presented as the dominant power\index{power!Sanskrit as a source of} in the cosmopolis, the power which stifled literary production in vernacular languages.
\vskip -10pt

\section{The Death of Sanskrit}
\vskip -3pt

Finally, I shall briefly summarize Pollock’s theory of the death of Sanskrit under five main points:

\begin{enumerate}
%~ \itemsep=0pt
\item Sanskrit’s death in Kashmir\index{Kashmir}
\item Its steady decline in Vijayanagara kingdom
\item The last Sanskrit poet
\item Death before colonialism,\index{colonialism} not because of it. 
\item All of useful knowledge in Sanskrit was because of the West anyway.
\end{enumerate}

In Kashmir, it is said that Sanskrit effectively died in the 12th c.\ CE, and failed to find a strong voice again, despite the efforts of Zain-ul-'abidin\index{Zain-ul-"'abidin} in the 15th c.\ CE\@. The possible influence of Islamic invasions\index{Islamic invasion} is not considered seriously. However, the historical account in \index{Thapar, Romila}Thapar’s book tells a different story where Mahmud Ghazni\index{Ghazni, Mahmud (Muhammed)} is described as the “champion iconoclast\index{iconoclast} looted the richest temples\index{temple!loot} at an unprecedented scale” (Thapar 2015:427).\index{Thapar, Romila} Mahmud Ghuri\index{Ghuri, Mahmud (Muhammed)} followed in his footsteps in the 12th c.\ CE not to plunder, but to establish a potential kingdom.
\vskip 1.5pt

Pollock (2006) briefly mentions the possible influence and accepts a southward shift in reaction to Islam development in the north. However, even if Islam rulers occupied power\index{power!political} (the word invaders\index{invaders} is not used in Pollock (2006) in this context), Pollock (2006) does not expect any adverse impact on traditional Sanskrit pundits. The strong learning tradition it is argued, should have persisted in literary endeavors. 
\vskip 1.5pt

In the case of Vijayanagara,\index{Vijayanagara} Pollock (2006) and Pollock (2001) do not consider any Sanskrit works of this era to be original work, but merely reproductive in nature. Moreover, “the Vijayanagara cultural world seems to have produced few if any Sanskrit works” that transcended time or geography (Pollock 2001:401). Here the discussion seems a bit unclear. First, it is said that a large body of Vijayanagara literature is yet to see the light of day - because they are available only in unpublished manuscripts.\index{manuscript} Pollock (2006) echoes a wish for a change in this state of affairs and the publication of manuscripts. At the same time, Pollock (2001) asserts that Vijayanagara has not produced Sanskrit works that have stood the test of time. As a means of substantiating this line of reasoning, examples of Kannada and Telugu\index{Telugu} manuscripts that have survived since their creation during the Vijayanagara period are given. In the absence of a record of similar manuscripts in Sanskrit, it is said that literary production of Sanskrit during the Vijayanagara period was not especially impactful. The main thrust of the argument relies on three factors: availability, publishing and subjective assessment of manuscripts.
\vskip 1.5pt

The first two factors of availability and publishing of manuscripts are somewhat related. Without a full catalogue and publication of manuscripts, it is difficult to assess the validity of the argument. More publication, research and writing regarding works of this period would help to better understand the contributions of Vijayanagara. Sadly, this field of study seems to have been a victim of modern day Indian politics. On the third factor regarding the subjective assessment of quality and whether works are primarily derivative, it perhaps merits a larger discussion. Hanneder’s\index{Hanneder} review discusses the shortcomings of relying excessively on qualitative assessment to decide the status of language vibrancy. 
\vskip 1.5pt

Kṛṣṇadēvarāya’s\index{Krsnadevaraya@Kṛṣṇadevarāya} {\sl Jāmbavatīpariṇaya}\index{Jambavatiparinaya@\textsl{Jāmbavatīpariṇaya}} is discussed as an example of a Vijayanagara work with limited literary value (Pollock 2001:402-403). After briefly describing the premise of the work, Pollock (2001) dismissively insists that the dramatic adaptation of the tale contains “nothing new”. The tale is then seen through a politico-military perspective\index{politico-military perspective} by relating the work to the king’s Orissa campaign. As in the case of earlier kings who used Sanskrit to assert their power\index{power!Sanskrit as a source of} (Section 2), Kṛṣṇadēvarāya’s work too is recognized for its “mytho-political representation”\index{mytho-political representation} of the king.
\vskip 1.5pt

At this stage, Pollock (2001)\index{Pollock!shifting of goalposts} turns to the lack of personal experience reflected in Sanskrit works from the Vijayanagara period as a sign of Sanskrit’s decline. Hanneder calls out the shifting of goalposts as “a surprising statement produced by the necessities of argumentation, rather than through evidence”. The notion of personal experience as a poetic tool is further elaborated by Pollock in his discussion of Jagannātha. 
\vskip 1.5pt

According to Pollock (2001), the last Sanskrit poet was Jagannātha\index{Jagannatha, Panditaraja@Jagannātha, Paṇḍitarāja} of the 17th c.\ CE\@. The reasons stated for this are as follows. He was the last great poet to travel the length and breadth of India. His was the last great work that found transregional acceptance. His {\sl praśasti} of the kings of Udaipur and Delhi, though rooted in the literature’s historical past, had a certain newness to it. The form of newness attributed to Jagannātha in this regard is not clearly described in Pollock (2006). Finally, his own personal life story finds resonance in his works. And the interest in his life story is enhanced because of the romantic love (and agony) he found in a Muslim woman.\index{Muslim!woman} With his death, original Sanskrit literature of any appreciable value, in Pollock’s estimation, died.
\vskip 1.5pt

The death of Sanskrit was despite the Mughals’\index{Mughals} attempt at sustaining it. Examples of Akbar’s\index{Akbar} court are offered as examples of a deliberate attempt at nurturing Sanskrit knowledge production. Sanskrit learning and schools, it is said, continued well into colonial rule. And the British Raj\index{British Raj} records show the proliferation of schools in the 19th c.\ CE\@. Yet despite this, Sanskrit literary production had ceased long before. Despite the continued presence of an informed readership which was well-versed in vernaculars and Sanskrit, 17th c.\ CE saw the death of Sanskrit.

Interestingly - and this aspect is not discussed in Pollock (2006) or Pollock (2001) - 17th c.\ CE saw the disappearance of the Kerala school of mathematics\index{Kerala School of Mathematics} too. Starting with Mādhava\index{Madhava (mathematician)@Mādhava (mathematician)} (14th c.\ CE), a school of mathematics which made path-breaking contributions ends in 17th c.~CE.

Finally, as Pollock (2006) says, “Westernization is a permanent and global phenomenon.” A long list of items are offered as example of forms of knowledge or practice that India borrowed from the West - astronomy from the Greeks, Ashoka’s “very idiom of rule” from  Achaemenid Persia,\index{Achaemenid Persia} architecture from Vetruvius, even {\sl Rāmāyaṇa}\index{Ramayana@\textsl{Rāmāyaṇa}} could arguably come from Homerian source (Pollock 2006:265). (On  {\sl Rāmāyaṇa}, Pollock (2006) does not offer a conclusive statement, but states that {\sl Rāmāyaṇa} possible origin in the Greek tales is a question worthy of consideration). Now, this idea of Western origin of all Indian or Sanskrit knowledge is made in the context of denying indigenism. It is argued that one can make a similar case for Easternization,\index{Easternization} in the sense that Greek’s influence on Europe would count as Easternization of Europe. A constant exchange of ideas and knowledge is the crucial element. The implication is that even if Sanskrit died, it is not something that should especially concern Indians.

This denial of any indigenous knowledge, while extolling the virtues of constant contact and exchange of ideas, is perhaps best seen in the following context. First, a long list of items are presented as contributions of the West, unlike other regions or cultures. Even if geographically accurate, it strains credibility to think of Greece\index{Greece} as the East in the context of knowledge creation and transmission. Second, mathematics, grammar\index{grammar} or other forms of knowledge which were far more advanced in India than their western counterparts, do not find mention in this supposed equivalence in knowledge. Third, Pollock (2006) quotes extensively from Hegel. It is not difficult to see Hegel’s ideas resonating in Pollock (2006) and other Pollock’s writings on aesthetics (not discussed here). Hegel\index{Hegel, G W F} considered the “chief defect” of ancient Indians was that “they cannot grasp either the meanings themselves in their clarity, or existing reality in its own proper shape and significance.” This idea is further expanded in denying any possibility of Indian knowledge in Pollock (2006) and elsewhere. Moreover Hegel thinks the Indians “refer each and everything back to the sheerly Absolute and Divine, and to contemplate in the commonest and most sensuous things a fancifully created presence and actuality of Gods” (Pollock 2006:352-3).

Thus, Pollock (2006) argues that a series of important events occurred in geographically distinct areas such as Kashmir and Vijayanagara, which led to the death of Sanskrit across the country, despite the attempts of Muslim kings of Kashmir and the Mughals in preventing it. In 17th c.\ CE, with the last Sanskrit poet Jagannātha, Sanskrit’s death was sealed.

\section{Conclusion}

The theory of birth, growth and death of Sanskrit is certainly interesting to any student of Sanskrit. In this paper, Pollock’s hypothesis of the role of languages\index{hypothesis!role of languages} – particularly, Sanskrit and Kannada – in building and sustaining political power\index{power!political} are described. Per Pollock, Sanskrit’s linguistic, literary, grammatical and analytical maturity in ancient India became a source of power for kings who exploited it for their political gain. Having recognized its ability to accrue power, they nurtured Sanskrit literary production. This gave rise to a new literary form of {\sl kāvya}. The Sanskrit poets returned the favor by composing {\sl praśasti}, a new form of literature to praise the power of kings. This mutually beneficial ecosystem was disrupted by the rise of vernacular languages. Some vernacular languages such as Kannada then followed a similar pattern of royal patronage\index{patronage} and power accrual using Kannada. Gradually literary production in Sanskrit declined despite efforts by Muslim rulers\index{Muslim!rulers} and later by British colonialists to revive it. 

Throughout the process, Pollock employs three main techniques to form the hypothesis:\index{hypothesis!formation of} 
\begin{enumerate}
\itemsep=0pt
\item Selective data sampling:\index{data sampling} As shown in Section 2.4 and in Hanneder\index{Hanneder}(2002), empirical data\index{empirical data} is selectively employed to suit the narrative, rather than let a comprehensive analysis of data guide the formation of the narrative. Hanneder uses the term “necessities of argumentation” to describe this characteristic. In my opinion, the selectivity has a stronger purpose. It not only serves the purpose of supporting a particular argument, but forms the basis of forming the narrative itself.  

\item Jargonization:\index{jargonization}\index{misinterpretation,!techniques of,!jargonization} Terms such as literization and literarization are described in detail.  Creating new terms to describe certain phenomena is certainly important. Contextual discussion however is often lacking. Understanding whether a term which has a clear utility in the European context continues to have a clear connotation in the Indian context would help in better appreciating the hypothesis.\index{hypothesis!formation of}

\item Re-interpretive analogy-making:\index{misinterpretation,!techniques of,!reinterpretive analogy-making} Terms and historical events which may have a superficial resemblance (if any), are identified by fully stating caveats about their limited sense of similarity. The ensuing discussion however, quickly sacrifices the stated caveats in favor of building a narrative.
\end{enumerate}

Finally, it hardly needs re-emphasis that Pollock’s ideas merit serious discussion. It is unfortunate that the papers have not been addressed in similar depth or breadth. I shall conclude with some thoughts on the nature of future work that may add greater clarity to the questions raised in Pollock’s papers. A tempting form of response may be comprised of two forms: (a) a careful and detailed discussion and rebuttal of the points raised in Pollock’s writings, and (b) building a different, well-justified narrative that can explain the observed facts over the years. 

But this response may face similar challenges and shortcomings. Instead, the field may benefit by taking a more systematic, data-driven\index{data-driven} approach in first building up a large dataset of literature, their themes and cross-dependencies and strive to build detailed, quantified analysis of the questions of interest. Creation of such a dataset could be carried out through a joint effort, and by developing automated techniques to check for biases\index{bias} in the data annotation. Then questions such as birth, growth, influence and death can be posed and analyzed more definitively.
\newpage

\begin{thebibliography}{99}
\itemsep=2pt
\bibitem[]{chap4_item1}
Akshar, K.V. (2003) {\sl Viśvātmaka Deśabhāṣe} (Kannada). Translations of Sheldon Pollock’s works. Sagara, Karnataka: Akshara Prakashana.

\bibitem[]{chap4_item2}
Cox, Whitney. (2011) ``Saffron in the Rasam." In Y. Bronner, W. Cox, \& L. McCrea (Eds.), {\sl Language, culture, and power: New directions in South Asian studies}; pp.~177--201.

\bibitem[]{chap4_item3}
Desai, P. (1936) {\sl Vijaya Sāmrājya} (Kannada). Dharwar: Vijayanagara Smarakotsva Samiti.

\bibitem[]{chap4_item4}
Doniger, Wendy. (2009) {\sl The Hindus: An Alternative History}. New York: Penguin.

\bibitem[]{chap4_item5}
Freeman, Charles. (2004) {\sl Egypt, Greece and Rome: Civilizations of the Ancient Mediterranean}. New York: Oxford University Press.

\bibitem[]{chap4_item6}
Gould, Rebecca. (2008) ``How newness enters the world: The methodology of Sheldon Pollock." {\sl Comparative Studies of South Asia, Africa and the Middle East} 28, no.~3; pp.~533--557.

\bibitem[]{chap4_item7}
Hanneder, Jürgen. (2002) ``On “The Death of Sanskrit”." {\sl Indo-Iranian Journal} 45, no.~4; pp.~293--310.

\bibitem[]{chap4_item8}
Hegel, Georg Wilhelm Friedrich. (1975) {\sl Hegel's aesthetics: Lectures on fine art}. New York: Clarendon Press.

\bibitem[]{chap4_item9}
Jay, Martin. (1992) ““The Aesthetic Ideology” as Ideology; or, What Does It Mean to Aestheticize Politics?.” {\sl Cultural Critique} 21;\\ pp.~41--61.

\bibitem[]{chap4_item10}
Kaviraj, Sudipta. (2005) “The sudden death of Sanskrit knowledge.” {\sl Journal of Indian Philosophy} 33, no.~1; pp.~119--142.

\bibitem[]{chap4_item11}
Krishnashastri, A.R. (2002) {\sl Saṁskṛta Nāṭaka} (Kannada). Bengaluru: Hemantha Sahitya. (Original work published 1957).

\bibitem[]{chap4_item12}
Naik, A. V., and A. N. Naik. (1948) “Inscriptions of the Deccan: an Epigraphical Survey (circa 300 BC-1300 AD)." {\sl Bulletin of the Deccan College Research Institute} 9, no.~1/2; pp.~1--160.

\bibitem[]{chap4_item13}
Pollock, Sheldon. (2001) ``The Death of Sanskrit." {\sl Comparative Studies in Society and History} 43, no.~02; pp.~392--426.

\bibitem[]{chap4_item14}
Pollock, Sheldon. (2006) {\sl The Language of the Gods in the World of Men: Sanskrit, Culture, and Power in Premodern India. Delhi}: Univ of California Press.

\bibitem[]{chap4_item15}
Sukthankar, VS  (ed.) (1942). {\sl The Mahabharata}. Poona: Bhandarkar Oriental Research Institute.

\bibitem[]{chap4_item16}
Thapar, Romila. (2015) {\sl The Penguin history of early India: From the origins to AD 1300. London}: Penguin UK.

\end{thebibliography}
