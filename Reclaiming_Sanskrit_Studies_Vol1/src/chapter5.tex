\chapter{Sanskrit is Not Dead}\label{chapter5}

\Authorline{Satyanarayana Dasa}
\lhead[\small\thepage\quad Satyanarayana Dasa]{}

\section*{Introduction}

\begin{myquote}
\eleven
“Government feeding tubes\index{Government feeding tubes} and oxygen tanks may try to preserve the language in a state of quasi-animation, but most observers would agree that, in some crucial way, Sanskrit is dead… Sanskrit literature could hardly be said to be alive if it had ceased to function as the vehicle for living thought, thought that supplemented and not simply duplicated reality”\hfill(Pollock 2001:393,414).
\end{myquote}

Sheldon Pollock’s 2001 argument of Sanskrit’s decline is a justifiable one, but he has overstated the case by equating this decrease in production to the death of the language. He has similarly overextended the case in asserting that Sanskrit is no longer a medium for living thought. Through a case study of the Vraja\index{case study of the Vraja} region, with a special focus on the Gauḍīya Vaiṣṇava tradition,\index{Gauḍīya Vaiṣṇava tradition} we analyze in this paper the significant literary contributions to novel thought expressed in Sanskrit in the 16-17th centuries CE. We further show Sanskrit in Vraja, and within the Gauḍīya tradition,\index{Gauḍīya tradition} has possessed historical continuity\index{historical continuity} and vitality since the 16th century, and is thus not dead.
\vskip 2pt

In our {\sl pūrva-pakṣa}, we engage with the five primary arguments of Pollock’s Introduction before he launches into his four case studies. First, we argue that language cannot be divorced from its socio-political context, so whether or not Sanskrit is being deployed as a political instrument is not germane to the vitality of the language. Second, Pollock makes bold assertions about the elementary level Sanskrit education, but he does not define the terms of his rhetoric, provide statisticalevidence, nor any references for his claims. Third, in referencing the first six Sanskrit awards from the Sahitya Akademi,\index{Sahitya Akademi} Pollock argues Sanskrit’s contemporary literary potency is near extinction; we contend that while not a star performer amongst Indian languages, Sanskrit is performing adequately. Fourth, Pollock claims Sanskrit literature experienced a “momentous rupture” after a vibrant period from 1550-1750, but we point to the statistics of Sanskrit authors and literary works indicating this assertion as overstated. Finally, we analyze some of the problems of Pollock’s vague terminology to raise doubts about his conclusion that Sanskrit is dead. An important consequence of more clearly defining language death,\index{language death} using Pollock’s own metrics, allows us to think of living language as a vehicle of novel thought and imagination. 
\vskip 2pt

Our case study begins by applying a popular Sanskrit maxim from Nyāya\index{maxim from Nyāya}\index{Nyāya} to suggest that existence of particulars point to the existence of universals—a method to show that if Sanskrit is not dead in Vraja or the Gauḍīya tradition,\index{Gauḍīya tradition} more specifically, then it can be inferred that Sanskrit is not dead. The crux of our paper then focuses on what Pollock claims to be a gap in the intellectual history\index{intellectual history} of Sanskrit from the 16-18th century, in order to prove that Sanskrit literature in Vraja was making history, demonstrating vitality, and producing novel—dare we say radical—thought. We specifically reference the new ideas and associated literary production of the Gosvāmin-s\index{Gosvāmin} of Vrindāvan,\index{Vrindāvan} with some additional attention to a selection of their contemporaries and successors. Our focus then shifts to contemporary Sanskrit education and ritual performance in Vraja. These mediums intersect through the innovated ritual of Bhāgavata kathā,\index{Bhāgavata kathā@\textsl{Bhāgavata kathā}} (stories from the {\sl Bhāgavata Purāṇa}\index{Bhāgavata Purāṇa@\textsl{Bhāgavata Purāṇa}}), a driver of cultural animation and social imagination. Finally, we examine Gauḍīya Sanskrit literary production since 1800 by identifying a selection of major scholars and their works. We conclude that there is sufficient doubt in Pollock’s argument – in order to refute his claim that Sanskrit is dead. 

\section*{{\sl Pūrva-Pakṣa}}

\subsection*{Political Context}

Pollock opens his paper by locating Sanskrit in the Indian political climate since the 1990s, stating that Sanskrit is central to the rhetoric of the contemporary Indian political right, specifically the BJP\index{BJP} and VHP.\index{VHP} He refers to political propaganda citing Sanskrit’s role as a “source and preserver of world culture” and the evidence for Sanskrit’s 4000-5000 year existence to be the seals from the Indus Valley civilization\index{Indus Valley civilization} (Pollock 2001:392). We are not endorsing nor defending these claims. We suggest that since language cannot be separated from its socio-political context, the potential politicization of a language is perhaps an expected outcome, and thus, not necessarily discrediting the merits of the language’s utility and relevance (Mufwene\index{Mufwene, Salikoko} 2004:206). Regardless of the degree to which one agrees with the argument that Sanskrit is being deployed as an instrument of Hindu identity politics,\index{identity politics} Pollock’s suggestion that Sanskrit’s relevance is limited to this sphere—as rhetorical political currency—\index{rhetorical political currency} is overreaching.
\vskip -20pt

\subsection*{Education}
\vskip -4pt

In summarizing Sanskrit’s role in independent India, Pollock suggests “disparities in political inputs and cultural outcomes could be detailed across the board” (Pollock 2001:393). However, the two examples he utilizes to make his case do not have much support. First, he addresses Sanskrit’s status as an official language\index{official language} of India and its associated funding benefits, specifically at colleges and universities. Yet in the discussion of the original fifteen, now eighteen, official languages of democratic India, Pollock seems to be suggesting Sanskrit’s inclusion is based largely on statecraft rather than merit. He curiously propounds this view  without any references or statistics: “with few exceptions, however, the Sanskrit pedagogy and scholarship at these institutions\index{institutions, educational} have shown a precipitous decline from pre-Independence quality and standards, almost in inverse proportion to the amount of funding they receive.” (Pollock 2001:392) 

Pollock neglects to define the metrics he has deployed for all of the categories supporting this unsubstantiated assertion: pedagogy,\index{pedagogy} scholarship, quality, and standards. He neither provides further qualitative explanation to the difference between the pre-Independence\index{pre-Independence} and post-Independence context. However, perhaps most problematic is the absence of funding data to support his assertion of an inverse relationship between input and output. There is an additional curiosity in Pollock’s claim: to which time period exactly does “pre-Independence” refer? Pollock claims that Sanskrit’s pre-modern decline was so significant that “by 1800, the capacity of Sanskrit thought to make history had vanished” (Pollock 2001:394). He also quotes the Gujarati poet, Dalpatrām Dahyabhai,\index{Dalpatrām Dahyabhai} who in 1857 indicated that Sanskrit was dead (Pollock 2001:394). So if Sanskrit was already dead or nearly dead for more than a century before India’s Independence, how could it have shown a “precipitous” decline\index{precipitous decline} in the 70 years since? There would have been no cliff from which to fall.

\subsection*{Contemporary Literary Recognition}
 
In the second example, Pollock suggests Sanskrit’s similarly poor performance in the domain of literature through the context of historical awards granted by the Sahitya Akademi\index{Sahitya Akademi} (“The National Academy of Letters”) in comparison to the other twenty-one recognized languages. He indicates the first five recipients “were given for works in English or Hindi on Sanskrit culture” and suggests the sixth Sanskrit award was based on an “almost metaliterary genre entirely unintelligible without specialized training” (Pollock 2001:392-3). The conclusions one can draw from analyzing six awards are likely more limited than Pollock suggests. The Akademi itself concedes that in its mission “to promote the unity of Indian literature” its annual awards across the board-- not merely in Sanskrit-- have “frequently been the butt of controversy” (Gokak\index{Gokak, V K} 1985:11-12).

An analysis of the number of total awards by language in the first thirty years of the Sahitya Akademi, that is, in the years prior to 1984 and rising BJP\index{BJP} influence, shows that Sanskrit performed just below the median. Thirteen languages surpassed Sanskrit, with Hindi leading the way (Rao\index{Rao} 1985:194-210). Sanskrit equaled or exceeded the total awards of eight other languages, including English, with fifteen total awards (Rao 1985:194-210)\endnote{This is not an apples to apples comparison as different languages entered the Akademi at different times. “Sahitya Academi’s programme was originally concerned with the fourteen languages enumerated in the Constitution of India, namely Assamese, Bengali, Hindi, Gujarati, Kannada, Kashmiri, Malayalam, Marathi, Oriya, Punjabi, Sanskrit, Tamil, Telugu, and Urdu. During the first thirty years of Sahitya Academi, 8 languages were added to them— English in 1954, Sindhi in 1957, Maithili in 1965, Dogri in 1969, Manipuri and Rajasthani in 1971, Konkani and Nepali in 1975” (Rao 1985:62).}. This may be instructive considering—as of 1961— “translations, anthologies, abridgements and edited or annotated works are not eligible for Award”; one perhaps could have wrongly assumed Sanskrit awards were largely limited to translations (Gokak 1985:21, Rao 1985:50). 

Additionally, a Sanskrit work was the recipient of the prestigious Jnanpith award\index{Jnanpith award} in 2006 (Shoba\index{Shoba} 2009). The Jnanpith Award is arguably the most prestigious literary award in India and is presented annually to the single most outstanding Indian work or writer in a contemporary context (Jnanpith Award). The award is presented by the Bharatiya Jnanpith Trust,\index{Bharatiya Jnanpith Trust} but award selection is the responsibility of an independent selection board. To date, works in fifteen different languages have received the Jnanpith Award (Jnanpith Laureates). Based on the metric of literary achievement proposed by Pollock, Sanskrit is performing within the top two thirds of Indian official languages\index{official language} and hence possesses a moderate degree of language vitality\index{language vitality} in this context. If Sanskrit were in fact dead, it likely would not be winning awards at this level and frequency. 
\vskip -20pt

\subsection*{Literary Production into Modernity}
\vskip -4pt

Pollock indicates the period of 1550-1750 “constitutes one of the most innovative epochs of Sanskrit systematic thought”, and our case study of Vraja which follows supports this claim (2001:393). However, he suggests this literary vitality was halted by a “momentous rupture”\index{momentous rupture} to such a degree “that by 1800, the capacity of Sanskrit thought to make history had vanished” (2001:394). Pollock does not attempt to explain the reasons for this radical shift though he acknowledges it is an important question with significant consequences across intellectual disciplines. 

An analysis of Karl H. Potter’s\index{Potter, Karl H.} first volume of the {\sl Encyclopedia of Indian Philosophy} seems to dispute Pollock’s claim regarding the rapid decline of Sanskrit literary production by 1800, in terms of quantity of unique Sanskrit authors and works. Potter’s bibliography contains lists of authors and their works related to Indian schools of philosophy, divided into two headings. The first heading is “Primary Texts and Literature About Them or Their Authors”, and the second is “Literature about the Philosophical Systems and Indian Philosophy in General”. The first heading is divided into three parts: Part I − Listed by author’s dates, Part II − Known author but unknown text date (DU), and Part III − Author is unknown and text date is unknown (ADU). Part I is further subdivided into four subcategories; (a) BCE– 4th c., (b) 5th c. – 9th c., (c) 10th c. - 14th c., (d) 15th c.– Present. We have analyzed Part I, the list of authors whose works and dates are known. 

The results of our analysis do not support Pollock’s assertion. The number of unique Sanskrit authors has actually increased over the centuries in question. The number of authors listed between 1601CE– 1700CE is 108, between 1701CE– 1800CE is 183, between 1801CE – 1900CE is 185, and between 1901CE-2000CE is 391 (Potter 1995:594-785). We see a 69\% increase in unique authors from the 17th to 18th century and what could be considered a growth explosion\index{growth explosion} from the 19th to 20th century with a 111\% increase in unique authorship. 

The total Sanskrit works listed between 1601CE– 1700CE is 1225, between 1701CE– 1800CE is 775, between 1801CE– 1900CE is 585, and between 1901CE– 2000CE is 695 (Potter 1995:594-785). The total number of Sanskrit publications did decline during the 18th century but a 37\% decline from the highly productive previous century can hardly be considered a “momentous rupture”. As one of “the most innovative epochs of Sanskrit systematic thought” extended into the middle of the 18th century, it is not surprising that production dropped by nearly 25\% in the nineteenth century. If Sanskrit were in such a death spiral, one would expect a decrease in production in the 20th century. But instead, we see a 19\% increase. And, perhaps even more noteworthy, 20th century production is 90\% of that of the eighteenth century, fifty years of which represented a golden period of production. 

Rather than a perpetual decline towards extinction, the data seems to suggest a stabilization of Sanskrit production after an exceptionally robust 17th century. It is, however, possible that one could contend a rupture occurred between the 17th and 18th centuries, in terms of the ratio of works produced by each unique author, due to its 63\% decline. The total works produced by author is 11.34 from 1601CE-1700CE, 4.23 from 1701-1800CE, 3.16 from 1801CE-1900CE, and 1.78 between 1901CE-2000CE. However, a seemingly stronger argument would emphasize the extraordinary level of production of Sanskrit authors during the 17th century, totaling over eleven works per author! This level of production is uncanny in the context of the modern and post-modern eras. The production of two to four works per author, in the 18th through 20th centuries, is not an output indicative of a dead language. While the social impact of the literary work during these periods is not necessarily directly proportional to production, it is difficult to support Pollock’s argument of a “momentous rupture”\index{momentous rupture} to the point of death in light of the data presented above. 
\vskip-10pt

\subsection*{Terms}
\vskip -3pt

Pollock claims that “most observers would agree that, in some crucial way, Sanskrit is dead” (2001:393). In order to respond to his bold assertion, it is important to define the two important categories Pollock introduces, particularly because he does not define them himself: {\sl who are the observers in agreement and how do we define language death}?\index{language death} In analyzing the vague category of “most observers”, is Pollock referring to himself and his colleagues of South Asian Studies in the United States and perhaps Western Europe? Or Indian scholars? Or trans-continental scholastic consensus? Or is Pollock referring to popular opinion? And if so, popular opinion in the United States, or India, or South Asia, or the global population at large? Or someone else altogether? The vagueness\index{vagueness}\index{misinterpretation, techniques of, vagueness} of this characterization of “most observers” is problematic. 

While Pollock’s individual position on the matter seems clear enough and his stature and achievement in the Western academy is well noted, one could, perhaps, infer that his is the commonly held perspective amongst his peers. However, short of significant corroborating scholarship, we hesitate to draw such a conclusion. Amongst Indian scholars, however, many observers would not agree with Pollock’s assertion. For example, hundreds of scholars are actively engaged in regular dialogue concerning Sanskrit language through the well known forum of Sanskrit scholars, known as Bhāratīya Vidvat Pariṣat.\index{Bhāratīya Vidvat Pariṣat} Through the forum, there are daily questions and interactions amongst scholars concerning Sanskrit literary verse interpretations and references, amongst other topics. Most of the scholars on Bhāratīya Vidvat Pariṣat are Indian or of Indian origin. This level of engagement and collaboration suggests Sanskrit is not dead amongst this audience. 

The activities of the International Association of Sanskrit Studies\index{International Association of Sanskrit Studies} also suggests a level of scholastic engagement. The professional association has consistently held World Sanskrit Conferences\index{World Sanskrit Conferences} every three years since its inauguration in 1972. The 16th Conference,\index{16th Conference} hosted in Bangkok, Thailand,\index{Bangkok, Thailand} was organized into twenty-one fields where scholars presented papers followed by discussions (Sharma\index{Sharma, Hari Dutt} 2015:2). The distinct fields included: “Veda\index{Veda} and Vedic Literature, Epics, Purāṇas, Āgama and Tantra, Linguistics, Grammar,\index{grammar} Poetry, Drama and Aesthetics\index{aesthetics}, Buddhist and Jaina Studies\index{Buddhist studies}, Vaiṣṇavism\index{Vaiṣṇavism} and Śaivism, History of Religions and Ritual Studies, Sanskrit in Southeast Asia, Philosophy, History, Art and Architecture, Epigraphy,\index{epigraphy} Sanskrit in Relation with Regional Languages and Literatures, Sanskrit, Science and Scientific Literature,\index{Sanskrit, scientific literature} Sanskrit Pedagogy and Contemporary Sanskrit Writing, Sanskrit in the IT World, Yoga and Āyurveda\index{Ayurveda}, Sūtra, Smṛti and Śāstra, and Manuscriptology” (Sharma 2015:2). In addition to the considerable breadth of topics which may indicate a certain level of vitality within the Sanskrit medium, the field titled {\sl Sanskrit Pedagogy and Contemporary Sanskrit Writing}\index{Sanskrit Pedagogy and Contemporary Sanskrit Writing} seems to further suggest that Sanskrit is not dead.

There does not seem to be trans-continental scholastic agreement that Sanskrit is dead. Even if the unproven and hypothetical inference of agreement amongst Western scholars were in fact substantiated, the use of the unquantified “most” raises concerns in the light of the above. 

Based on Pollock’s domain as a distinguished professor of pre-modern South Asia and the academic context of his extensive scholarship over several decades, it is unlikely he is referring to popular opinion in citing “most observers”. However, in the unlikely event that he is referring to a popular consensus, we don’t have any data to confirm this potential hypothesis\index{potential hypothesis} one way or the other, what to speak of the merits of such a potential endeavor. Therefore, Pollock’s term “most observers” is highly problematic from multiple vantage points, which may suggest why he didn’t provide more specificity in defining the category. 

While Pollock utilizes the term “dead” to refer to Sanskrit, he does so in a normative way, recognizing that the “metaphor is misleading, suggesting biologistic or evolutionary beliefs about cultural change that are deeply flawed” (Pollock 2001:393). He further problematizes the category in suggesting that some may argue that “all written languages\index{written languages} are learned and learnèd,\index{learned and learnèd} and therefore in some sense frozen in time (“dead”)” (2001:393). Pollock’s nuanced framing of language death\index{language death} is useful in avoiding the potential liabilities he suggests, but it also leaves us without a clear picture of what exactly he means in pronouncing “Sanskrit is dead”.\index{Sanskrit is dead} The strongest indication Pollock provides for a definition of language death is being  “frozen in time”, which he uses specifically as a synonym. 

A robust engagement with linguistic theory is beyond the scope of this paper, but we do think it useful to offer a clear definition, without necessarily committing to its corollaries surrounding biological or cultural change. Salikoko Mufwene,\index{Mufwene, Salikoko} Professor of Linguistics at University of Chicago, states, ``language death is likewise a protracted change of state used to describe community level loss of competence in a language, it denotes a process that does not affect all speakers at the same time nor to the same extent. Under one conception of the process, it has to do with the statistical assessment of the maintenance versus loss of competence in a language variety among its speakers. Total death is declared when there are no speakers left of a particular language variety in a population that had used it" (2004:204). 

By Mufwene’s definition, no one would argue that Sanskrit has experienced a “total death” as the 2001 Indian census refers to over 14,000 Sanskrit speakers and Pollock references “Sanskrit periodicals and journals, feature films and daily newscasts on All India Radio,\index{All India Radio} school plays” (Pollock 2001:393; 2001 India Census).\index{2001 India Census} The more useful definition offered by Mufwene for our purpose relates to loss of language competence.\index{language competence} Of course, without a fixed degree of “community level loss of competence in a language” offered in the definition of language death, there can be a wide range of interpretation. 

In assessing whether or not Sanskrit is dead, we have a preliminary working definition coupling Pollock’s frozen in time and Mufwene’s loss of community language competence. Pollock adds some critical qualification in conceptualizing language and communication. He states, “the communication of new imagination, for example, is hardly less valuable in itself than the communication of new information. In fact, a language’s capacity to function as a vehicle for such imagination is one crucial measure of its social energy” (2001:394). Pollock clarifies that language vitality\index{language vitality} is not limited to the transmission of new information, but includes the impact of communicated information in the form of new imagination or experience. His use of the modifier “crucial” may be informative as it was initially used to describe “Sanskrit is dead, in some crucial way”, but here points to the impact of language-inspired imagination (2001:393). 

Therefore, our conclusion in determining if Sanskrit is dead will utilize Pollock’s own metric which places equal value on the communication of new information and imagination through a broader context of community language competence. Through our Vraja case study with special attention on the Gauḍīya Vaiṣṇava tradition,\index{Gauḍīya tradition} we will demonstrate that Sanskrit has retained continuity as an expression of living thought and imagination. The Vraja experience casts significant doubts on Pollock’s claim that “Sanskrit is dead”. 

\section*{Vraja Case Study}

Pollock utilizes four cases from different periods and geographies to illustrate his conclusion that Sanskrit is dead. We will not analyze these individual cases, but instead offer an alternative case study showing the continuity of Sanskrit vitality and literary production in the Vraja region, with special attention on the Gauḍīya Vaiṣṇava tradition. We will further demonstrate that these literary contributions are novel in content. In refutation of Pollock’s claim that Sanskrit is dead or was “stillborn”,\index{stillborn}\index{Sanskrit, stillborn} we use the popular Sanskrit maxim of the “rice in the cooking-pot”,\index{rice in the cooking-pot} {\sl sthāli-pulāka nyāya}.\index{sthāli-pulāka nyāya} According to Colonel G. A. Jacob,\index{Jacob, Colonel G. A.} the maxim means, “In a cooking-pot all the grains are equally moistened by the heated water, such that when one grain is found to be well cooked, the same may be inferred with regard to the other grains. So the maxim is used when the condition of the whole class is inferred from that of a part” (Jacob 1900:40). In the present study, the grain is analogized to the life of Sanskrit in the Vraja area and the cooking-pot is equated to the whole country of India. In Vraja, we primarily focus on the vibrant life of Sanskrit related to the school of Gauḍīya Vaiṣṇavism,\index{Gauḍīya Vaiṣṇavism} which had its inception in the first quarter of the sixteenth century and continues to flourish in the present day. If it is proven that Sanskrit was alive and continues to be so in Vraja, then one can infer, using the above-mentioned maxim, that it is also so in the rest of India. 

Here one may object to the use of the cooking-pot example, because there is no such cultural uniformity in India that is analogous to the uniform condition of heat in the pot. It is partly true. But, if we recognize the fact that India’s majority population accepts Hindu {\sl dharma},\index{Hindu dharma}\index{dharma} which is rooted primarily in Sanskrit literature and depends on it for its daily religious performance, then it can be compared to a big cooking pot of Hindu culture with different varieties of grains in it in the form of various Hindu traditions. Moreover, Vraja\index{Vraja} sustained a severe attack under the Moghul rule of Aurangzeb\index{Aurangzeb} beginning in 1669, which had a considerable impact. Monika Horstmann\index{Horstmann, Monika} describes, “Govindadevjī\index{Govindadevjī} was removed from Vrindavan together with his consort and Vṛndādevī\index{Vṛndādevī} when Aurangzeb\index{Aurangzeb} displayed an increasing intolerance towards the Hindus. The contemporary historian Sāqī Musta’idd Khān\index{Khān, Sāqī Musta’idd} reported that on 8 April 1669, the Emperor had issued orders to demolish Hindu schools and temples\index{Hindu schools and temples} and to restrain the practice of the Hindu religion” (Horstmann 1999:8)\endnote{We are not claiming Aurangzeb,\index{Aurangzeb} nor Moghul rule more generally, as some sort of catch-all category explaining the decline of Sanskrit. Growse illuminates this sort of broad misconception, “Aurangzeb, the conventional bīte noire of Indian history, who is made responsible for every act of destruction” (194). However, neither are we throwing the baby out with the bath water; the impact of Aurangzeb’s attacks in Vraja was considerable.}. If Sanskrit could survive and flourish under these conditions, then there may be reason to believe it could perform better in Indian areas of relative peace. It is like testing the hardest grain in the cooking-pot. If the hardest grain is cooked, then the softer ones would surely be cooked (Miśra\index{Miśra, Chavinātha} 1978:35)\endnote{This line of reasoning is called Kaimutika Nyāya,\index{Kaimutika Nyāya} which is literally translated as “what to speak of” and sometimes translated as a fortiori.  It is a principle used to indicate the greatness or weakness of some object or process.  For example, to show tha caliber of a guru, one can say that even a neophyte disciple of the guru defeated a famous scholar in debate, so what to speak of the the guru himself.  }. 

However, accepting the logic of case study particulars proving a universal concept could be — and arguably is — utilized by Pollock also. He also uses case studies in an attempt to prove Sanskrit’s death.\index{Sanskrit, death off} If all cases accurately portray their respective assertions, then we have a contradiction in conclusions: it would be unclear if pan-Indic Sanskrit is in fact dead or alive. Regardless, the ambiguity resulting from these rivaling conclusions raises considerable doubt to Pollock’s claim that Sanskrit is dead\endnote{In what may be an attempt to insulate his argument from such challenges, Pollock acknowledges, “the world of Sanskrit is broad and deep, and it would be unsurprising to find different domains following different historical rhythms and requiring different measures of vitality” (2001:394). However, one could apply this very same logic to his case studies to argue the opposite point: Sanskrit is not dead though it may appear to be in these instances.}. 

\subsection*{Novel Literary Content}

The Vraja case study\index{Vraja case study} is instructive in additional ways also. First, it contributes to the larger tapestry of understanding India’s intellectual history between the 16th and 18th centuries, which history, Pollock has argued elsewhere, “remains to be written, since these [Sanskrit] texts have yet to be accessed, read, and analyzed” (Pollock 2005:78-79). Pollock’s expresses concern about a gap in Indian intellectual history,\index{intellectual history} “We have no clear understanding of whether, and if so, when, Sanskrit culture ceased to make history, whether, and if so, why, it proved incapable of preserving into the present the creative vitality it displayed in earlier epochs, and what this loss of effectivity might reveal about those factors within the wider world of society and polity that had kept it vital” (Pollock 2001:393). The Vraja case can play a small part in unraveling this larger and more complex question by showing that Sanskrit culture showed a historical continuity\index{historical continuity} and vitality. 
\vskip 2pt

Second, it can point to the specific features of Sanskrit’s blossoming in the 16th - 17th centuries. The Vraja case aligns with Pollock’s assessment of this period as “one of the most innovative epochs of Sanskrit systematic thought” (2001:393). However, there is discord in the characterization of these novel developments. Sanskrit texts from this time period have received shallow analysis to date, thus requiring further research. But Pollock expresses that the initial indications that the period produced new literary styles but not new content will likely be substantiated. He states, 
\vskip 2.5pt

\begin{myquote}
\eleven
the project and significance of the self-described ‘new intellectuals’ in the sixteenth and seventeenth centuries also await detailed analysis, but some first impressions are likely to be sustained by further research. What these scholars produced was a newness of style without a newness of substance. No idiom was developed in which to articulate a new relationship to the past, let alone a critique; no new forms of knowledge—no new theory of religious identity, for example, let alone of the political—were produced in which the changed conditions of political and religious life could be conceptualized \hfill(Pollock 2001:417). 
\end{myquote}
\vskip 2.5pt

Our exposition of the Gauḍīya tradition\index{Gauḍīya tradition} will strongly refute the above mentioned “first impressions” which Pollock suggests are likely to be proven. We demonstrate that new, perhaps radically new, Sanskrit content emerged from this period. S.K De\index{De, S K} explains, “it was the inspiration and teaching of the six pious and scholarly Gosvāmin-s\index{Gosvāmin} which came to determine finally the doctrinal trend of Bengal Vaiṣṇavism\index{Bengal Vaiṣṇavism} which, however modified and supplemented in later times, dominated throughout its subsequent history” (De 1986:118). These new Sanskrit texts produced from early Gauḍīya scholars were the foundation of a new religious tradition and identity. 
\vskip 2.5pt

\begin{myquote}
\eleven
One of the most remarkable features of the Caitanya\index{Caitanya} movement\index{Caitanya movement} is its extraordinary literary activity, the power and vitality of its inspiration being evidenced by the vast literature which it produced both in the learned classical tongue and in the living language of the province. It enriched the field of Sanskrit scholarship by its more solid and laborious productions in theology, philosophy, ritualism, and Rasa-śāstra, so on the other, it poured itself out lavishly in song and story, almost creating as it did a new literary epoch \hfill (De 1986:556). 
\end{myquote}

S.K. De highlights the new literary world created by the Gauḍīya tradition, which principally occurred in Sanskrit in the region of Vraja. Vraja consists of about 84 square miles making up the district of Mathura\index{Mathura} in the state of UP, parts of Faridabad district in the state of Haryana, and part of Bharatpur district in the state of Rajasthan. In the 16-17th centuries, it was exceptionally fertile ground for literary production and novel concepts, most specifically directed to the exposition of Kṛṣṇa {\sl bhakti}. David Haberman\index{Haberman, David} (2003) describes, 
\begin{myquote}
\eleven
Other groups besides the Gauḍīyas were also actively involved in developing the region of Vraja as a new center of Kṛṣṇa worship: all these seem to have worked in an atmosphere of mutual influence. Vallabhācārya,\index{Vallabhācārya} a Tailang brāhmaṇa whose family came from what is now Andhra Pradesh, arrived in Vraja in the early years of sixteenth century, and there began what was to become the Puṣṭi Mārga, one of the most popular of the Vaiṣṇava lineages (sampradāya) centered in Vraja. Rūpa Gosvāmin\index{Rūpa Gosvāmin}\index{Gosvāmin} refers to Vallabha’s teachings directly in the Bhaktirasāmṛtasindhu.\index{Bhaktirasāmṛtasindhu@\textsl{Bhaktirasāmṛtasindhu}} A local Vraja saint by the name of Hita Harivaṁśa\index{Hita Harivaṁśa} established the Rādhāvallabha temple in Vṛndāvana in the year 1534, and composed (italics mine) passionate poems about the love-affairs of Rādhā and Kṛṣṇa that still inspire members of a small but influential sampradāya known as the Rādhāvallabhīs. Another poet-saint who took residence in Vṛndāvana at this time was Svāmī Haridāsa,\index{Svāmī Haridāsa} who established the temple image of Kuñjabihārī or Banke Bihārī. Svāmī Haridāsa is said to have been the teacher of Tansen,\index{Tansen} the legendary musician of Akabar’s court. Although the Vaiṣṇava saint Nimbārka\index{Nimbārka} was most likely born in the thirteenth century, the sampradāya, he founded also played an active role in the establishment of the new form of worship in Vraja that focused on the love affair of Rādhā Kṛṣṇa. Another key figure involved in the establishment of the new form of worship in Vraja is Mādhavendra Pūrī. It is not clear whether Mādhavendra Pūrī came from Bengal or from southern India, nonetheless all Vraja sources portray him as having a vital role in establishing the important Kṛṣṇa shrine on top of Mount Govardhana. The works of the creative leaders of the new religion centered in Vraja were then carried by others throughout northern India, thus insuring the lasting influence of the poetry, texts, and religious culture that were produced during the creative years of the early sixteenth century. For example, the works of Rūpa Gosvāmin\index{Rūpa Gosvāmin} were carried back to Bengal by such disciples as Narottama Dāsa Thākura\index{Thākura, Narottama Dāsa} and Śrīnivāsa Ācārya,\index{Srīnivāsa Ācārya@Śrīnivāsa Ācārya} and were incorporated into the widely popular Caitanya Caritāmṛta\index{Caitanya Caritāmṛta@\textsl{Caitanya Caritāmṛta}} of Kṛṣṇadāsa Kavirāja,\index{Kṛṣṇadāsa Kavirāja} thereby creating a wide popular and long-lasting audience. The Vaiṣṇava culture that began in Vraja in the sixteenth century is still vitally alive, and Vṛndāvana continues to be a major center for temple and pilgrimage\index{pilgrimage} activities today. The picture that emerges during the first half of the sixteenth century is an explosion of lively and imaginative activity initiated by various scholars, poets, and saints, and focused on Kṛṣṇa as the fully manifest form of ultimate reality in the guise of a passionate Vraja cowherd 

~\hfill(Haberman 2003:xxxiv-xxxv). 
\end{myquote}

This was a novel literary period in terms of both content and style, and one that encompassed a number of different genres. And while vernacular also contributed to this portrait, most of the scholars and poets mentioned by Haberman wrote primarily in Sanskrit, the only exception being Svāmī Haridāsa\index{Svāmī Haridāsa} who wrote only in Braja-bhāṣā.\index{Braja-bhāṣā} Many different scholars and poets contributed to this literary explosion, but perhaps the greatest inspiration was drawn from Śrī Caitanya Mahāprabhu.\index{Caitanya Mahāprabhu} 

\subsection*{Sanskrit Literature of the Gauḍīya Gosvāmin-s: 16th Century}
\index{Gosvāmin}

Śrī Caitanya Mahāprabhu,\index{Caitanya Mahāprabhu} the founder of the Gauḍīya (also referred to as Caitanya or Bengal) Vaiṣṇava school, was born at Navadvip, Bengal, in 1486 CE. He made his residence at Jagannath Puri, Orrisa, at the age of 24 after entering into the renounced order of life. From there he visited Vrindāvan in the fall of 1515 for about a month. After returning to Puri, he sent some of his erudite Bengali followers, headed by Śrī Sanātana Gosvāmin and Śrī Rūpa Gosvāmin,\index{Rūpa Gosvāmin} to settle in Vraja. They were assigned three main jobs: first, to discover the various places related to Kṛṣṇa’s {\sl līlā}-s or divine play; second, to establish temples of Kṛṣṇa worship; and third, to compose literary works delineating the path of devotion, {\sl bhakti-yoga}, as taught by the Śrī Caitanya. Tony Stewart\index{Stewart, Tony} (2010) confirms the role of Gosvāmin-s, “they were more than religious archaeologists,\index{religious archaeologists} they were scholars who had been deputed to gather and compose texts\index{texts (Sanskrit/Indic)} so they might better explain the religious devotion, {\sl bhakti}, that Caitanya had revealed” (Stewart 2010:4). These followers took the instructions of Śrī Caitanya Mahāprabhu\index{Caitanya Mahāprabhu} to heart and successfully fulfilled the instructions of their teacher. S.K De refers to the {\sl Gosvāmin-s}, “it was indeed their eminence and influence which gave a marked primacy to the Bengal school over other rival schools in the holy city associated with the name of Kṛṣṇa”(De 1986:118). The scope of this paper will focus on the third instruction of Śrī Caitanya Mahāprabhu:\index{Caitanya Mahāprabhu} composing works delineating the path of Bhakti-yoga. 

The principal followers of Śrī Caitanya Mahāprabhu\index{Caitanya Mahāprabhu} joining Sanātana and Rūpa in Vraja included their nephew Jīva,\index{Jīva [Gosvāmin]} Raghunāth dāsa, Raghunāth Bhaṭṭa, and Gopāla Bhaṭṭa. Popularly, they came to be known as the six Gosvāmin-s\index{Gosvāmin}\index{Rūpa Gosvāmin} and their literary focus was novel in comparison to other Vaiṣṇava schools who focused on producing original commentaries of the {\sl prasthāna trayī}.\index{prasthāna trayī} The Gosvāminns, and the Gauḍīya tradition which they shaped, instead focused on expositions inspired by the {\sl Bhāgavata Purāṇa},\index{Bhāgavata Purāṇa@\textsl{Bhāgavata Purāṇa}} which they viewed as Vyāsa’s natural commentary of the sūtra and the canonical text of their new tradition (SN Dasa\index{Dasa, SN} 2015: Anu\index{Anu} 19). This shift of canonical focus from the {\sl prasthāna trayī}\index{prasthāna trayī} to the {\sl Bhāgavata Purāṇa}\index{Bhāgavata Purāṇa@\textsl{Bhāgavata Purāṇa}} was a radical one, which highlights the novelty of its corresponding literary outputs. 

The Gosvāmin-s mainly wrote in Sanskrit, which may suggest the language’s prominence at the time, rather than writing in the Braja-bhāṣā, the vernacular of Vraja, or in Bengali or Oriya\endnote{An analysis of why the Gosvāmin-s\index{Gosvāmin} wrote primarily in Sanskrit is not in the scope of this project. Tony Stewart seems to align with an argument made by Sheldon Pollock in his 2005 work {\sl The Ends of Man at the End of Premodernity}. Stewart states, “I think it is clear now in retrospect that the community was engaged in a type of literary practice that in some ways mimicked the courts with their concern to cultivate a proper Sanskrit (universal) ethos for regulating actions and modulating power within the group” (Stewart 2010:ix). As an additional consideration, Rūpa\index{Rūpa Gosvāmin} composed Sanskrit works before meeting Caitanya (De 1986:148). So regardless to what degree Stewart/Pollock’s assertion may or may not be true, Rūpa’s Sanskrit literary inclination was not a development of his newfound faith.}. While the definitive dates of their literary production is difficult to discern, “we can approximately fix the period of their literary activity” from the onset of the 16th century (De 1986:160)\endnote{The most flourishing period of Rūpa’s literary activity falls between 1533-1550 CE, but it probably began as early as 1494 CE. Sanātana literary works were about the same time period and written accounts testify both he and his brother seem to have died in the same year. Sanātana’s latest work was completed in 1554 CE. Jīva’s literary works seem to have been complied between 1555-1592 CE(De 1986:160-4).}. What is definitive is they produced an immense body of literature. Edward C. Dimock\index{Dimock, Edward C.} illustrates, 
\begin{myquote}
\eleven
“Rūpa and Sanātana, and their nephew Jīva, were brilliant men, learned in the {\sl śāstras} and every conceivable category of learning from esthetics to grammar.\index{grammar} Jīva\index{Jīva} was perhaps the most brilliant of all, and he has more than twenty Sanskrit works covering grammar, poetry, poetics, ritual, theology, and philosophy to his credit, including the monumental {\sl ṣaṭ-sandarbha},\index{saṭ-sandarbha@ṣaṭ-sandarbha} which is the first full treatment of the theology of the Bengal school of Vaiṣṇavism” \hfill (Dimock 1999:24). 
\end{myquote}

Śrī Sanātana Gosvāmin wrote six Sanskrit texts,\index{texts (Sanskrit/Indic)} the most noteworthy of which is the the {\sl Śrī-bṛhad-bhāgavtāmṛtam\index{Srī-bṛhad-bhāgavtāmṛtam@Śrī-bṛhad-bhāgavtāmṛtam}\endnote{Additional Sanskrit books written by Śrī Sanātana Gosvāmin\index{Gosvāmin} include Vaiṣṇava-toṣiṇī (commentary on the tenth canto of {\sl Bhāgavata Purāṇa}) and {\sl Digdarśinī} (commentary on {\sl Hari-bhakti-vilāsa}), {\sl Līlā-stava}, {\sl Laghu-harināmāmṛta-vyākaraṇam}, {\sl Daśama-tippaṇi} (H. Dasa 1398). }}. The text utilizes a Puranic narrative style and its principle theme is the theology of Kṛṣṇa {\sl bhakti}. It is divided into two parts, the first of which charts the story of sage Nārada’s quest to locate the greatest of Kṛṣṇa devotees, thereby facilitating an explanation of the characteristics of an ideal devotee and “the different stages of devotional attainment, ending in the Madhura or erotic attitude of the Gopī-s to Kṛṣṇa” (De 1986:235). A principal and novel theme of the Gauḍīya literary tradition is the representation of the Vraja cowherd girls ({\sl gopī-s}) as the paradigmatic {\sl bhakta-s}.

The second part of the text “reverses the process described in the first part” and focuses on “Kṛṣṇa’s mode of manifesting himself to His Bhakta” (De 1986:235). Kṛṣṇa’s reciprocation of love for his {\sl bhakta-s} is another point of theological emphasis for the Gauḍīya tradition.\index{Gauḍīya tradition} The second part of Sanātana’s Gosvāmin’s\index{Gosvāmin} text concludes that “Vṛndāvana is the real paradise of Kṛṣṇa where the unmanifest eternal sport of Kṛṣṇa becomes manifest to him alone who is blessed with real Bhakti for the deity” (De 1986:238). Sanātana thus establishes Vṛndāvana as the intersection between the phenomenal and noumenal worlds\index{phenomenal and noumenal worlds} for the {\sl bhakta}. 

Śrī Rūpa Gosvāmin\index{Rūpa Gosvāmin} was a more prolific writer than his brother and his literary work on {\sl bhakti rasa}, as seen specifically in {\sl Bhakti-rasāmṛta-sindhu}\index{Bhaktirasāmṛtasindhu@\textsl{Bhaktirasāmṛtasindhu}} (devotional poetics) and {\sl Śrī-ujjvala-nīlamaṇi},\index{Srī-ujjvala-nīlamaṇi@Śrī-ujjvala-nīlamaṇi} are particularly noteworthy as they represented a novel genre through a fusion of Sanskrit poetics and {\sl bhakti} sentiment\endnote{See Appendix A, for a complete list of Sanskrit books written by Śrī Rūpa Gosvāmin\index{Rūpa Gosvāmin}}. S.K. De explains, “a new turn was thus given to the old Rasa theory of conventional Poetics but also to the religious emotion underlying older Vaiṣṇava faith” (De\index{De, S K} 1986:166). David Haberman\index{Haberman, David} expounds upon De’s recognition of Rūpa’s literary contributions of new ideas. He describes, 
\vskip 3pt

\begin{myquote}
\eleven
“Rūpa’s understanding of rasa differs greatly from those of other theoreticians who preceded him. Whereas previous writers normally restricted the rasa experience to the limited space of the theatre, he extended it to all of life. Rasa is now not understood to be simply a temporary aesthetic experience, but rather as the culminating core of genuine human life. $\ldots$ Although Rūpa will proceed to introduce variety into love, it is clear that bhakti-rasa has a single and very special Foundational Emotion.\index{Foundational Emotion} For him, then, all genuine Rasa is based on some form of love, or more specifically, some form of love for Kṛṣṇa. This is a significant point of departure from previous bhakti theoreticians, such as Bhojadeva,\index{Bhoja/Bhojadeva} who recognized the traditional nine Foundational Emotions”  \hfill 	(Haberman 2003:li-lii). 
\end{myquote}
\vskip 3pt

More so than novel form, Rūpa Gosvāmin’s literary production can be understood as novel literary content where “literary-erotic emotion transmuted into a deep and ineffable devotional sentiment, which is intensely personal and yet impersonalized into a mental condition of disinterested joy”(De 1986:169). 

Despite the significant literary contributions of his uncles, Sanātana and Rūpa,\index{Rūpa Gosvāmin} Edward Dimock suggests, “Jīva\index{Jīva [Gosvāmin]} may have been the most brilliant of all” the Gosvāmins\index{Gosvāmin} due to the breadth and depth of his literary work (24). S.K. De seems to arrive at a similar conclusion: “Jīva became the highest court of appeal in doctrinal matters so long as he lived” (1986:150). Jīva’s {\sl magnum opus} is the Ṣaṭ Sandarbha,\index{Saṭ Sandarbha@Ṣaṭ Sandarbha} a six-part systematic analysis of the Gauḍīya school, which has been reified as the authoritative doctrine of the tradition\endnote{See Appendix A, for a complete list of Sanskrit books written by Śrī Jīva Gosvāmin\index{Jīva [Gosvāmin]}}. The Sandarbhas quote heavily from the Bhāgavata Purāṇa in addition to drawing from the Upaniṣads and other Puranas, but it is “considerably original in its outlook and presentation…ideas and methods” (De 1986:256). The Sandarbhas represent perhaps the greatest novel literary achievement of the Gauḍīya tradition.  

Śrī Gopāla Bhaṭṭa Gosvāmin’s most famous work is {\sl Haribhakti-vilāsa}, which is an erudite and “authoritative exposition of most of the compulsory rites and ceremonies…that is now regarded to be the highest authority of the Bengal school of Vaiṣṇavism” (De 1986:408-9)\endnote{Additional Sanskrit books written by Śrī Gopāla Bhaṭṭa Gosvāmin’s Sanskrit include: {\sl Bhāgavata-sandarbha} (This became the basis of six Sandarbha of Jīva Gosvāmin) and {\sl Kriyā-sāra-dīpikā} (religious duties) (H. Dasa 1209).}. Consistent with the innovative nature of the Gauḍīya tradition, it largely departs from the orthodox Smṛti tradition\index{Smṛti tradition} and instead “evolves a Sṃrti of its own” based on its own canonical scriptures and influences from Bengali Tantra (De 1986:409). 

Śrī Raghunātha Dāsa Gosvāmin’s principal literary contribution is poetry titled, {\sl Muktācaritam}\endnote{Additional Sanskrit books written by Śrī Raghunātha Dāsa Gosvāmin include: {\sl Stavāvalī} (prayers) and {\sl Dāna-keli-cintāmaṇi} (poetry) (H. Dasa 1326).}. The {\sl Muktācaritam} is written in the Campū style combining prose and verse, and it focuses on Kṛṣṇa’s Vraja līla-s. In reference to Kṛṣṇa, the text demonstrates “the superiority of his free love for Radhā over his wedded love for Satyabhāmā” (De 1986:123). The theme of prioritizing spontaneous love generated by the heart over that which is socially bound is a fundamental theme of the Gauḍīya tradition, and one of its novel literary and theological contributions. 

Raghunāth Bhaṭṭa seems to have assisted the others in their literary works; no works are directly ascribed to him. He is most renowned for singing Bhagāvata Purāṇa in a vast range of different melodies. 

\subsection*{Gosvāmin Contemporaries}
\index{Gosvāmin}
\vskip -7pt

As David Haberman clarified earlier, there were numerous exemplary personalities, who were Vraja contemporaries of the Gosvāmin-s, who wrote profusely in Sanskrit. Two of the most prominent examples include Śrī Prabodhānanda Sarasvatī and Śrī Nārayaṇa Bhaṭṭa. One of the most noteworthy works of Śrī Prabodhānanda Sarasvatī\index{Srī Prabodhānanda Sarasvatī@Śrī Prabodhānanda Sarasvatī} is {\sl Caitanya-candrāmṛta},\index{Caitanya Mahāprabhu} a devotional and poetic hagiography of Caitanya\endnote{See Appendix B, for a complete list of Sanskrit books written by Śrī Prabodhānanda Sarasvatī.}. The text is novel in its Sanskrit presentation of Caitanya as an {\sl avatāra}, specifically, an incarnation of both Kṛṣṇa and Radhā. While this particular conception of Caitanya’s divinity does not seem to be explicitly endorsed by the Vraja Gosvāmin-s, it did have traction in the movement’s Bengali literature and eventually came to be incorporated into the Gauḍīya doctrine. S.K. De describes {\sl Caitanya-candrāmṛta}, “the poem undoubtedly reflects what is called the Gaura-pāramya attitude of his Navadvīpa devotees, which is not explicit in the works of the Gosvāminns, but which regards Caitanya in himself, and not as an image of Kṛṣṇa, as the highest reality or Parama Tattva” (1986:130). 

Śrī Nārayaṇa Bhaṭṭa\index{Srī Nārayaṇa Bhaṭṭa@Śrī Nārayaṇa Bhaṭṭa} was a prolific Sanskrit author who made several novel contributions of continuous impact on the social imagination of the the tradition\endnote{See Appendix B, for a complete list of Sanskrit books written by Śrī Nārayaṇa Bhaṭṭa.}. He identified and explained the various holy places surrounding the twelve forests of Vraja, which became the basis for the one month Vraja parīkrama (circumambulation) that remains a vibrant and popular annual ritual, drawing pilgrims from all parts of India. He also propagated the integration of {\sl rāsa līlā}\index{rāsa līlā} into Vraja dramas,\index{Vraja dramas} which also are central to Vraja cultural life. Lastly, he founded the {\sl Śrigi temple} in Rādha’s town of {\sl Varśana}, a primary pilgrimage\index{pilgrimage} site. 
\vskip -8pt

\subsection*{Gauḍīya Scholastics following the Gosvāmin-s: 17-18th Century}
\vskip -7pt

There were numerous scholars who wrote in Sanskrit in the immediate wake of the Gosvāmin-s, but we will focus on three of them: Śrī Kṛṣṇadāsa Kavirāja, Śrī Viśvanātha Cakravarti, and Śrī Baladeva Vidyābhuṣaṇa\index{Baladeva Vidyābhuṣaṇa}\endnote{Besides these well-known authors, there have been many lesser known scholars in Vraja who wrote in Sanskrit. Śrī Radhā-kṛṣṇa-Dāsa Gosvāmin, who wrote Sādhana-dipikā\index{Sādhana-dipikā} and {\sl Daṣaṣlokī-bhāṣya}, is such an example.}. Śrī Kṛṣṇadāsa Kavirāja\index{Srī Kṛṣṇadāsa Kavirāja@Śrī Kṛṣṇadāsa Kavirāja} “is the only figure in Gauḍīya Vaiṣṇava history known to have studied with all six of the original Gosvāmin-s of Vṛndāvana” (Stewart 190). He wrote a {\sl mahākāvya} (great poetry) about the daily activities of Kṛṣṇa and {\sl gopī-s} in Vṛndāvana called {\sl Śrī-govinda-līlāmṛtam},\index{Srī-govinda-līlāmṛtam@Śrī-govinda-līlāmṛtam} but his most famous work is {\sl Caitanya-caritāmṛta},\index{Caitanya Caritāmṛta@\textsl{Caitanya Caritāmṛta}} the sacred biography of Caitanya’s life which has become canonized by the tradition\endnote{While there is contestation to the date, De affirms 1615CE is most likely (1986:57).}. While largely written in Bengali, the interspersion of Sanskrit is so prolific that Tārāpada Mukhyopadādhyāya contends “the ‘real’ {\sl Caitanya-caritāmṛta}\index{Caitanya Mahāprabhu} was this Sanskrit skeleton that was fleshed out by the Bengali text” (Stewart 243). Tony Stewart seems to endorse Mukhyopadādhyāya’s argument that it is more accurate to define it as a “Sanskrit {\sl sūtra} with Bengali commentary serving to unpack the terse statements” (246). Regardless, we can consider {\sl Caitanya-caritāmṛta} as a “mixed text” with its “three thousand verses of Sanskrit quotation embraced by nearly twenty thousand verses of Bengali— for it was the Sanskrit that structured the text” (Stewart 246, 21). 

Śrī Viśvanātha Cakravarti\index{Srī Viśvanātha Cakravarti@Śrī Viśvanātha Cakravarti} (17th Century) was one of the greatest scholars after the Gosvāmin-s\index{Gosvāmin} of Vṛndāvana in Gauḍīya Vaiṣṇavism\endnote{See Appendix B, for a complete list of Sanskrit books written by Śrī Viśvanātha Cakravarti.}. He wrote voluminously in Sanskrit, including a commentary on {\sl Bhagavad Gītā}\index{Bhagavadgītā@\textsl{Bhagavadgītā}}. But Cakravarti is perhaps best known for his clear expositions on the writings of Rūpa Gosvāmin,\index{Rūpa Gosvāmin} specifically on {\sl rāsa} theory. An additional important contribution was his resolution of the ambiguity concerning if the Vraja {\sl gopī-s} were or were not married to Kṛṣṇa. The amorous, yet out of wedlock, relationship between Kṛṣṇa and the {\sl gopī-s} was (and is) a provocative concept, but one that is central to the Gauḍīya theology of {\sl Rāgānugā bhakti}.\index{Rāgānugā bhakti} Cakravarti’s work was crucial to expanding and clarifying this controversial and relatively novel idea. 

Like Śrī Jīva Gosvāmin,\index{Jīva [Gosvāmin]} Śrī Baladeva Vidyābhuṣaṇa\index{Baladeva Vidyābhuṣaṇa} (18th century) wrote in Sanskrit on almost all the branches of literary studies such as philosophy, poetics, rhetoric, dramaturgy,\index{dramaturgy} history, liturgy, and panegyrics\endnote{See Appendix B, for a complete list of Sanskrit books written by Śrī Baladeva Vidyābhuṣaṇa\index{Baladeva Vidyābhuṣaṇa}.}. But his greatest literary achievement was his original commentary on the {\sl prasthāna-trayī},\index{prasthāna-trayī} Vedānta’s canonical trilogy comprised of the Upaniṣad-s, Vedānta Sūtra-s, and the {\sl Bhagavad Gītā}. In order for a {\sl sampradāya} of the Vedānta to be officially recognized, it needs to prove its {\sl siddhānta} through original commentary on the {\sl prasthāna-trayī}. As the Gauḍīya literary tradition was novel to the point of radical innovation, its early scholastics did not prioritize this formality. However, following a controversy concerning worship of both Radhā and Kṛṣṇa at the Govindaji temple in Jaipur, Baladeva validated the tradition with his original Sanskrit commentaries of the {\sl prasthāna-trayī} (Kapoor\index{Kapoor} 1994:38). He was the first in the Gauḍīya tradition to comment on the Upaniṣad-s and {\sl Vedānta Sūtras}. 
\vskip -100pt

\subsection*{Contemporary Sanskrit Education}\index{Contemporary Sanskrit}
\vskip -5pt

Today Sanskrit education\index{Sanskrit education} is abundant in Vraja and is transmitted in three primary ways: through government institutions,\index{institutions, governmental} specifically universities, through private institutions,\index{institutions, private} and through the traditional {\sl guru-śiṣya} relationship. There are a considerable number of Sanskrit schools operating in Vraja and thirty of the most well-known schools can be found in Appendix C. In the institutional context, there are two primary levels following the completion of {\sl uttara madhyamā},\index{uttara madhyamā} the approximate equivalent of a high school degree. The Śastrī or Tīrtha degree\index{Tīrtha degree}\index{Śastrī degree} is awarded after the successful completion of annual written and oral examinations over a three-year period. The Śastrī\index{Sastrī@Śastrī} or Tīrtha\index{Tīrtha} degree is equivalent to a bachelor’s degree. The Ācārya\index{Acārya degree@Ācārya degree} degree is equivalent to a master’s degree and requires an additional two years of course work and annual examinations following the Śastrī/Tīrtha degree. In terms of pedagogy, students of Sanskrit grammar typically memorize the sutras, notably those of Pāṇīni. But grammar\index{grammar} is not the only subject taught in Sanskrit, one can study a wide range of subjects including but not limited to: Sāhitya, Pūraṇa, Itihāsa,\index{itihāsa} Jyautiśa, Vedas,\index{Veda} Vedānta, etc. In addition to these formal educational institutions,\index{institutions, educational} there are hundreds of {\sl āśrama-s}\index{aśrama@\textsl{āśrama}} in Vṛnḍāvana where teachers— {\sl guru-s} and {\sl ācārya-s}— teach to their students/disciples in a traditional manner. While not as widely prevalent as the pre-modern era, this method of knowledge transmission still exists. By traditional manner, there is no formal school and admission. The interested student approaches a particular guru and studies under him. I personally studied in this manner in addition to completing a Sanskrit PhD degree from Agra University. Specifically, I studied the {\sl Ṣaṭ Sandarbha} from Śri Haridāsa Śastrī Mahārāja and Nyāya\index{Nyāya} from Śri Śyāmāśaraṇa Mahārāja. For a more current example, at Jīva Institute, there are four such teachers, including myself, who teach students on request. At present, I am teaching a six- month course in Sanskrit and {\sl darśana-s} to about 50 students from different parts of the world. The traditional transmission of Sanskrit knowledge is still actively functioning. 

\subsection*{Ritual Use of Sanskrit }
\vskip -5pt

Tony Stewart speaks of the adaptability and vitality of the Gauḍīya tradition, “each generation was charged with the responsibility of revalorizing its tradition without destroying it, to make it relevant to a contemporary world without having to diverge from the general consensus of its broad normative ideas” (9). As Sanskrit is deeply embedded within the foundation of the tradition, the language continues to play an indispensable role in its living transmission. 

The continuity of Sanskrit education thus impacts the performance of rituals,\index{rituals, performance of} where the use and understanding of the classical language is still considered vital. While these various rituals do not contribute to new literary production, they play critical roles as catalysts of cultural animation\index{cultural!animation} and novel production of social imagination.\index{social imagination} As Pollock himself underscores, “the communication of new imagination, for example, is hardly less valuable in itself than the communication of new information. In fact, a language’s capacity to function as a vehicle for such imagination is one crucial measure of its social energy” (Pollock 2001:394). Sanskrit makes considerable contemporary contributions to cultural vitality\index{cultural!vitality} through ritual. 

Bhāgavata Vidyālaya-s\index{Bhāgavata Vidyālaya} are specialized schools where students exclusively study the {\sl Bhāgavata Purāṇa} with Sanskrit commentaries through a one-year course. Although the classes are taught in Hindi, basic Sanskrit grammar\index{grammar} is a pre-requisite since the primary text is written in Sanskrit. These students are prepared to become professional speakers of {\sl Bhāgavata Purāṇa},\index{Bhāgavata Purāṇa@\textsl{Bhāgavata Purāṇa}} which is very popular in northern India. Performers of Bhāgavata kathā, as the recitation of the narrative is referred to, speak on portions of the text from memory over a period of seven days. Although the primary discourse in spoken in Hindi, it includes the recitation of an abundance of Sanskrit verses from the primary text. There are hundreds of Bhāgavata kathā performers coming through Vraja, and the most popular ones, such as Krishna Chandra Sastri, Mṛdula Krishna Sastri, Devakinānda Thakur, and Pundrik Gosvami, draw tens of thousands of participants to a single recitation. Kathā takes place all year round, and popular performers may deliver one or two performances per month. 

Sometimes 108 or even 1008 reciters sit together and sing {\sl Bhāgavata Purāṇa} collectively. Bhāgavata kathā,\index{Bhāgavata kathā@\textsl{Bhāgavata kathā}} which is underpinned by and interspersed with Sanskrit, is one of the most popular and colorful forms of contemporary religious activity igniting social activity. 

Not as publically oriented as Bhāgavata kathā, Hindu religious ceremonies called {\sl saṁskāra-s}\index{saṁskāra-s}
 are recited in Sanskrit and continue to play an important role in family life. They are most commonly performed during rites of human passage such as birth, marriage, and death, but there are many more including the first tonsure of a baby, first eating of grains, beginning of education, etc. The first {\sl saṁskāra} is ideally performed on the day the child is conceived and the last one is done after death. While the observation of {\sl saṁskāra-s} varies and some Hindus do not observe them at all, they remain a relevant aspect of traditional cultural life in Vraja, and of contemporary India more generally.

The {\sl saṁskāra-s} are performed under the guidance of a priest who is trained in ceremonial rituals and the recitation of Sanskrit {\sl mantra-s}. Even if the priest is not a Sanskrit scholar, he must know basic rules of Sanskrit pronunciation. The priest may not even have gone to a Sanskrit school, but he would have learned the {\sl mantra-s}\index{mantra-s} from his father, or senior family member, or member of the community. Ceremonial knowledge\index{Ceremonial knowledge}\index{oral tradition, ceremonial knowledge} has been preserved in India for centuries through this type of oral transmission. 

An additional Vedic inspired ritual, though diminishing, is chanting of the Gāyatrī {\sl mantra}\index{Gāyatrī mantra} at twilight time, twice per day. While historically applicable to the three {\sl dvija varna-s},\index{varṇa} it is primarily only applicable for {\sl brāhmaṇa-s} in the present context for fulfillment of Hindu {\sl dharma}.\index{Hindu dharma} After receiving the sacred thread at the {\sl yajñopavīta} ceremony, one is expected to perform the ceremony, called {\sl sandhyā-vandana}, twice per day. The {\sl mantra} is typically transmitted through a family lineage of {\sl brāhmaṇa-s} and the particulars of ceremonial performance varies depending on the branch of the  Veda\index{Veda} in which the family tradition is rooted. 

A more commonplace ritual and one that is inclusive of a wider audience is the regular chanting of Sanskrit {\sl mantra}-s and {\sl stotra}-s, prayers. There is significant variety of content, and {\sl mantra}-s often reference Kṛṣṇa, Viṣṇu, Śiva, Durgā, or Ganeśa, though Kṛṣṇa is most prevalent in Vraja. Several common sources of chanting include: Vaidika Sūkta-s, Śānti-pāṭha, {\sl Bhagavad Gītā\index{Bhagavadgītā@\textsl{Bhagavadgītā}}, Viṣṇu-sahasra-nāma, Durgā-saptaśatī, Śiva-mahimnas-stotra, Gajendra-stuti, Gopāla-sahasra-nāma, Rādhā-kṛpā-kaṭākṣa-stotra, Gopī-gīta}, prayers to one’s {\sl iṣṭadevatā} and guru. 

Temples are abundant in Vraja, even in the smallest of the villages in rural areas. Vṛndāvan is a town of temples. There are approximately 5000 temples— some big and some small— in Vṛndāvan alone, and every temple has a deity. When a new temple is built, which is not uncommon, the deity is first installed through a ceremony called {\sl prāṇa-pratiṣṭhā}\index{prāṇa-pratiṣṭhā} in order to prepare it to invoke divine presence in the deity. This is an elaborate process involving an abundance of Sanskrit mantra recitation. After the installation, the deity must be worshiped at least twice a day in order to maintain its divine presence. Thus every temple has at least one priest, {\sl pūjāri} or an {\sl arcaka}. The process of deity worship involves recitation of {\sl mantra}-s and prayers called {\sl stotra}-s, which are mostly in Sanskrit. 

Lastly, Vṛndāvana has a class of priests, {\sl paṇḍa-s},\index{paṇḍa} who guide pilgrims and perform worship for them. On the pilgrim’s behalf, a {\sl paṇḍa} will recite Sanskrit {\sl mantra}-s to deities or holy places, like the Yamuna river. The pilgrim is believed to receive the spiritual benefit performed by the {\sl paṇḍa}, and the latter typically receives compensation for the services performed.
\vskip -50pt

\subsection*{Gauḍīya Literary Tradition: 1800-Present}

In discussing the Gauḍīya literary corpus,\index{Gauḍīya literary corpus} specifically that which relates to Sanskrit and vernacular biographies of Caitanya,\index{Caitanya Mahāprabhu} Tony Stewart confirms “the enormity of this textual tradition” (2010:xi). Stewart highlights that the Gauḍīya tradition,\index{Gauḍīya tradition} like other textual traditions, “depend on what I call ‘living texts’\index{living texts}\index{texts (Sanskrit/Indic)} that are routinely modified to suit the immediate needs, rather than privileging some kind of original or Ur-text” (2010:xi). Stewart underscores the vitality inherent in these texts within their traditional context. They are by his definition “living” through the continuity of their popular engagement and scholarly revision. Because Sanskrit plays a central role in this living textual tradition,\index{living textual tradition} it is not dead.

The continuous role of Sanskrit in the Gauḍīya Vaiṣṇava {\sl sampradāya}\index{Gauḍīya Vaiṣṇava sampradāya} cannot be underestimated. Like many of the medieval {\sl bhakti} movements, the use of the vernacular in various literary forms has ensured the spread of Krishna {\sl bhakti} to all strata of society. However, the Gauḍīya-s have maintained an active commitment to Sanskrit engagement. In addition to the production of new texts, the fluid scholastic interpretation of the Sanskrit canon based on time, place, and circumstance catalyzes the creation of new imagination for devotees and defines the tradition as living. We will touch on several of these important contributions, with special attention on Kedaranath Datta Bhaktivinoda\index{Bhaktivinoda, Kedarnath Dutta|(}. 

The 19th century saw the study of Sanskrit introduced in the Fort William school for British and East India Company bureaucrats as a necessity for understanding the new crown jewel of the British Empire. British scholars and researchers, along with their Indian counterparts, embarked on a search for manuscripts that had been lost or had never achieved wide fame. This development was of special interest to Gauḍīya\index{Gauḍīya tradition} Vaiṣṇava-s and by the end of the century, there were numerous efforts to find rare books and to bring them to publication. Perhaps the most important development was the publication of Rajendralal Mitra's\index{Mitra, Rajendralal} (1824-1891) {\sl Notices}, which lists hundreds of texts\index{texts (Sanskrit/Indic)} held in mostly private collections, many written by reputed authors. Rajendralal Mitra was one of the first modern Indologists of Indian origin, and his contribution triggered the publication of many literary works and the attention of the Vaiṣṇava community (Kapoor 2002:3527). 

The Radharaman Press of Berhampore (Murshidabad), funded by Raja Manindra Chandra Nandy of Cossim Bazaar and edited by Rām Nārāyan Vidyaratna, published many of the books identified through {\sl Notices} along with their Bengali translations, making them available to a wide audience for the first time. {\sl Notices} identified a large number of commentaries that had been written by unknown authors; these filled an important role for understanding the original texts, which often had no earlier commentaries. 

The availability of books in printed form also led to the writing of new commentaries or inspired new compositions. Take for example Vīracandra Gosvāmin's\index{Vīracandra Gosvāmin}\index{Gosvāmin} commentary on the {\sl Gopāla-campū},\index{Gopāla-campū} or that of Vaṁśīdhara on the {\sl Bhāgavatam (Bhāvārtha-dīpikā-prakāśa)}\index{Bhāvārtha-dīpikā-prakāśa@\textsl{Bhāvārtha-dīpikā-prakāśa}}, which were actually written in the 19th century. Kedaranath Datta Bhaktivinoda (1834-1914) wrote original Sanskrit hymns and {\sl sūtra} texts with Bengali translations and commentary to make them available to a wider audience, but he was not alone. Nitya Svarūpa Brahmacārī\index{Nitya Svarūpa Brahmacārī} of the Devakinandan Press in Vrindavan at the beginning of the 20th century is also worthy of mention.
\vskip 2pt

Bhaktivinoda\index{Bhaktivinoda} was a voluminous writer who produced over 100 literary works in Sanskrit, Bengali, and English from 1849-1907, approximately twenty of which were original\endnote{Shukavak Dasa, 8. He provides a complete list of all publication in Appendix 1, p. 283-295. The production of literary works includes those written, translated, or edited.}. {\sl Kṛṣṇa-saṁhitā}\index{Kṛṣṇa-saṁhitā} is perhaps Bhaktivinoda’s most innovative work, both in terms of content and style. This three-part volume of history and theology combines Sanskrit verse and Bengali commentary. Furthermore, it applies modern scholastic methodology to its traditional insider theological perspective in an effort to reconcile both approaches (Dasa 1999:2,9). Shukavak Dasa explains, “He offered a plausible date for the {\sl Bhāgavata} according to internal and extra-textual evidence; he pointed out corruptions in the text, and he brought attention to the human weakness of its author…Bhaktivinoda was showing that it was indeed possible to take a critical look\index{critical look} at one’s own tradition, and at that same time maintain a deep and abiding faith within that tradition” (2). In addition to the originality of the text content and hybridized language style, Bhaktivinoda’s synthesized perspective was a novel literary contribution. 
\vskip 2pt

Bhaktvinoda\index{Bhaktivinoda, Kedarnath Dutta} also utilized his hybrid language style in {\sl Śri-gaurāṅga-līlā-smaraṇa-stotram},\index{Sri-gaurāṅga-līlā-smaraṇa-stotram@Śri-gaurāṅga-līlā-smaraṇa-stotram} although this time coupling Sanskrit with English. The collection of Sanskrit verses describing Caitanya’s teachings and the theology emerging from them was prefaced by forty-seven pages of introduction written in English (Dasa 1999:91). The vitality of this text can be expressed by the extent of its circulation beyond Bengali scholastic, popular, or expatriate circles. “The work was sent to various universities and intellectuals in different parts of the world and eventually found its way onto the book shelves of McGill University in Montreal, the University of Sydney in Australia and the Royal Asiatic Society of London (Dasa 1999:91). One could perhaps contend that despite the volume of literary work produced by Bhaktvinoda, the limited geographical scope of its audience demonstrates the limited vitality of Sanskrit. However, the international and scholastic audience of the {\sl Śri-gaurāṅga-līlā-smaraṇa-stotram}, disputes this potential objection and thereby demonstrates the vitality of Bhaktvinoda’s Sanskrit literature. 
\vskip 2pt

Bhaktivinoda’s vast literary publications, much of which were in Sanskrit, also considerably impacted and expanded how the Gauḍīya tradition could view itself. Shukavak Dasa explains, “Bhaktivinoda’s approach widened the limits of human reason and rational analysis with Caitanya\index{Caitanya Mahāprabhu} Vaiṣṇavism. It challenged the balance between traditional faith and human reason by allowing a greater degree of rational and symbolic understanding than might otherwise be permitted in Vaiṣṇava religious life. Bhaktivinoda’s employment of the {\sl ādhunika-vāda}\index{adhunika-vāda@\textsl{ādhunika-vāda}} allowed the {\sl bhadralok}\index{bhadralok@\textsl{bhadralok}} to experiment with new historical perspectives that were better suited to life in the nineteenth century” (Dasa 1999:252)\endnote{{\sl Ādhunika-vāda}\index{adhunika-vāda@\textsl{ādhunika-vāda}} can be translated as “modern approach”, which refers to history as an evolutionary process. {\sl Bhadraloka}\index{bhadralok@\textsl{bhadralok}} refers to educated Bengali professionals who interacted with the British administration. (Shukavak Dasa 128-132, 16-17)}. Perhaps, Bhaktivinoda’s\index{Bhaktivinoda, Kedarnath Dutta|)} greatest achievement was his synthesis of innovation and tradition to facilitate the Gauḍīya movement’s transition into modernity. As Sanskrit was one of his primary tools in achieving this end, it is difficult to imagine how Sanskrit could be dead.
\vskip 2pt

Literary production and discovery continued well into the 20th century and to some extent still persists. Haridas Das\index{Das, Haridas} (1899-1957) was perhaps the most notable researcher in the Gauḍīya Vaiṣṇava school in the first half of the century (Brazinski 2016). He discovered original texts and commentaries and wrote many new Sanskrit commentaries to works that did not have them ({\sl Āhnika-kaumudī}\index{Ahnika-kaumudī@\textsl{Āhnika-kaumudī}} of Kavi Karṇapūra, {\sl Āścarya-rāsa-prabandha}\index{Aścarya-rāsa-prabandha@\textsl{Āścarya-rāsa-prabandha}} by Prabodhananda, {\sl Mādhava-mahotsava}\index{Mādhava-mahotsava} of Jīva Gosvāmin,\index{Jīva [Gosvāmin]} to name just a few) along with their Bengali translations. These research developments increased popular awareness of the vastness of the Gauḍīya Vaiṣṇava literary tradition, both in the vernaculars and Sanskrit, thus increasing interest in its canonical texts and later writers. Other names that should be mentioned are Puri Dāsa (Bhakti Prasād Puri) and Kṛṣṇa Dāsa Babaji of Kusum Sarovar. This awareness and interest in the Sanskrit texts\index{texts (Sanskrit/Indic)} of the Gosvāmin-s\index{Gosvāmin} and their followers continued unabated in the second half of the 20th century with new translations appearing in Hindi from the Gaur Gadadhar Press of Haridāsa Śastrī and the Harinam Press of Shyamlal Hakim in Vrindāvan\endnote{Additional Sanskrit works produced by Gauḍīya scholars in the 20th century include: Viśvambharadeva Gosvāmin: {\sl 1. Āstikya Darśanam, 2. Vedārtha-dīpikā, 3. Haribhakti-sarvasva, Śrigovonda-paricaryā, 5. Suvijñāna-ratnamālā}. Kanhaiyālāla Śāstri: {\sl Śrīmādhava-sevā} commentary on {\sl Bhāgavata Purāṇa}. Śrīgopījanaballabha Dāsa: {\sl Śrī Rasika Maìgala}. Śrīrādhānanadadeva Gosvāmin: {\sl Śrī Rādhā-govinda Kāvyam}. Śrī Rāmanārāyaṇa: {\sl Bhāva-bhāva-vibhāvikā commentary on Bhāgavata Purāṇa} (Kṛṣṇa Das Baba 4).}. Several accomplished scholars of Sāhitya also wrote noteworthy Sanskrit commentaries, specifically Shivprasad Bhattacharya Kavyatirtha\index{Bhattacharya, Shivprasad} on {\sl Alaṅkāra Kaustubha}\index{Alaṅkāra Kaustubha@\textsl{Alaṅkāra Kaustubha}}, Surendranath Shastri\index{Shastri, Surendranath} on {\sl Dāna-keli Kaumudī},\index{Dāna-keli Kaumudī} and Krishna Vihari Misra\index{Misra, Krishna Vihari} on {\sl Bhakti-rasa-vivecanam}.\index{Bhakti-rasa-vivecanam@\textsl{Bhakti-rasa-vivecanam}} 

The Gauḍīya Math,\index{Gauḍīya tradition} established by Bhaktisiddhanta Saraswati\index{Saraswati, Bhaktisiddhanta} in 1916, sought to democratize Sanskrit learning so that it no longer remained the monopoly of caste {\sl brāhmaṇa-s}. Anyone with the capacity to learn the language was encouraged to do so in order to enlist the gravitas of Sanskrit in the cause of evangelization. This democratization of Sanskrit\index{Sanskrit, democratization of} has been a feature of its continued prestige and popularity. Indeed, the spreading of Krishna {\sl bhakti} to other parts of the world has resulted in a huge upsurge of interest and scholarship, as well as in attempts to revive Sanskrit as a spoken language. Most of the books of the canon have been translated into English and have become the subject of scholarly research. 

In the 21st century, Sanskrit literary production continues, albeit not at the same rate as in the premodern era. By way of example, two Vraja based scholars have recently published Sanskirt texts.\index{texts (Sanskrit/Indic)} Pandita Ānanda Gopāla Dāsa\index{Dāsa, Pandita Ānanda Gopāla} has written a very elaborate commentary on {\sl Sarva-saṁvādinī} of Śrī Jīva Gosvāmin,\index{Gosvāmin}\index{Jīva [Gosvāmin]} the first commentary every written on this text, in addition to a four volume work on Baladeva’s {\sl Govinda Bhāṣya}\index{Baladeva Vidyābhuṣaṇa}\index{Baladeva’s Govinda Bhāṣya}\index{Govinda Bhāṣya} commentary of Vedānta Sūtra-s. Girirāja Kiśora Śāstri\index{Śāstri, Girirāja Kiśora} wrote a commentary on {\sl Bhāgavata Purāṇa}

\section*{Conclusion}

The aftermath of Pollock’s provocative argument that “Sanskrit is dead” is indicative of the complex and significant interests at stake. The ensuing debate has been useful in unearthing a more nuanced perspective of Sanskrit’s recent history, contemporary function, and prospects for the future. While we draw a different conclusion than Pollock based on our own study, we are grateful to him for initiating and inspiring deeper research into the intellectual history\index{intellectual history} of Sanskrit thought. In addition to our attempt to demonstrate that Sanskrit is not dead, the significant and novel literary contributions of the 16-17th century Gauḍīya tradition\index{Gauḍīya tradition} indicate Sanskrit’s historical vibrancy. We hope our case study of Vraja can and will play a small role in this much larger project, and we don’t doubt the variegatedness and richness such an exploration will surely produce. .


\section*{Appendix A}

The complete list of Sanskrit books written by Śrī Rūpa Gosvāmin\index{Srī Rūpa Gosvāmin@Śrī Rūpa Gosvāmin} are as follows:{\sl  Haṁsadūta {\rm (Poetry)}, Uddhava-sandeśa {\rm (poetry)}, Śrīrādhā-\Break kṛṣṇa-goṇoddeśa-dīpikā, Śrī-Kṛṣṇa-janma-tithi-vidhi {\rm (worship)}, Aṣṭādaśa-\Break līlā-chanda {\rm (Kṛṣṇa stories, poetry)}, Dāna-keli-kaumudī {\rm (play)}, Śrī-bhakti-rasāmṛta-sindhu {\rm (devotional poetics)}, Śrī-ujjvala-nīlamaṇi {\rm (Madhura-\Break rasa poetics)}, prayuktākhyāta-candrikā {\rm (Sanskrit verbs)}, Mathurā-\Break māhātmya {\rm (description of holy places in Mathurā area)}, Padyāvalī {\rm (collection of verses about Kṛṣṇa bhakti)}, Laghu-bhagavatāmṛtam {\rm (Summary of the principles found in Bṛḥad-bhāgavatām of Sanātana Gosvāmin)}, Nikuñja-rahasya-stava{\rm (description of intimate pastimes of Rādhā and Kṛṣṇa)}, Nāṭaka-candrikā {\rm (definitions related to play)}, Vidagdha-mādha\-va-nāṭaka {\rm (play)}, Lalita-mādhava-nāṭaka {\rm (play)} {\rm (H. Dasa 1351)}.}

The complete list of Sanskrit books written by Śrī Jīva Gosvāmin\index{Srī Jīva Gosvāmin@Śrī Jīva Gosvāmin} are as follows: {\sl  Gopala-campū {\rm (2 volumes, Life stories of Kṛṣṇa)}, Śrīharināmā\-mṛta-vyākaraṇam {\rm (complete grammar\index{grammar} of Sanskrit)}, Bṛhat-krama-sanda\-rbha {\rm (commentary on complete {\sl Bhāgavata Purāṇa})}, Laghu-krama-\Break sandarbha {\rm (abridged commentary on {\sl Bhāgavata Purāṇa})}, Laghu-vai\-ṣṇa\-va-toṣiṇī {\rm (Commentary on the tenth canto of {\sl Bhāgavata Purāṇa})}, Tattva-sandarbha {\rm (Essence of {\sl Bhagavata Purāṇa})}, Bhagavat-sandarbha {\rm (description of Bhagavān, the Supreme Person)}, Paramātma-sandarbha {\rm (description of the Immanent Being, Jīva, and Māyā)}, Kṛṣṇa-sandarbha {\rm (Description of Kṛṣṇa, the Original form of Supreme Person)}, Bhakti-sandarbha {\rm (description of bhakti as the prescribed process)}, Prīti-sandarbha {\rm (description of unconditional love)}, Saṅkalpa-kalpa-druma, Mādhava-maho\-tsva, Radhākṛṣṇārcana-dipikā {\rm (process of worship of Rādhā and Kṛṣṇa)}, Gopala-virudāvali {\rm (prayers)}, Rasāmṛta-śeṣa {\rm (poetics)}, Gāyatrī-vyākhyā\break {\rm (Explanation of Gāyatrī mantra)},\index{Gāyatrī mantra} Commentary on Brahma-saṁhitā, Sūtra-mālikā, Dhātu-saṅgraha {\rm (collection of Sanskrit roots)}, Commentary on\break Yoga-sāra-tattva, Śrīkṛṣṇa-pada-cihna {\rm (description of marks on the soles of Kṛṣṇa’s feet)}, Rādhikā-pada-cihna {\rm (description of marks on the soles of Radha’s feet)}, Durgamasaṅgamanī {\rm (commentary on {\sl Śrībhaktirasāmṛ\-ta-sindhu})}, Locanarocanī {\rm (commentary on {\sl Ujjvala-nīlamaṇi})}, Harināma-vyākhya  ({\rm meaning of} mahāmantra), Yugalāṣṭaka {\rm (prayer)}, Upāsanā-\Break tattva {\rm (worship)}, Anarpitacarī-śloka-vyākhyā {\rm (commentary on the\break Anarpitacari verse)}, Svarṇa-ṭīkā, Jāhnvāṣṭaka {\rm (prayer)} {\rm (H. Dasa 1249)}.} 

\section*{Appendix B}
%~ \vskip -10pt

{\rm The complete list of Sanskrit books written by Śrī Prabodhānanda\break Sarasvatī\index{Srī Prabodhānanda Sarasvatī@Śrī Prabodhānanda Sarasvatī} are as follows:} {\sl Rādhāsudhā-nidhi {\rm (Poetry)}, Caitanya-candrā\-mṛta {\rm (biography)}, Vṛndāvana-mahimāmṛta {\rm (poetry in praise of Vṛndāvana)}, Saṅgīta-madhava {\rm (Poery about Rādhā and Kṛṣṇa {\sl līlā})}, Āścarya-\Break rāsa-prabandha {\rm (poetry on {\sl rāsalīlā})}, {\rm Commentary on the 87th chapter of tenth canto of} Bhāgavata Purāṇa, {\rm Commentary on} Gīta-govinda, {\rm commentary on} Gopāla-tāpanī Upaniṣad, {\rm Commentary on} Gāyatrī-mantra} {\rm (H. Dasa 1290)}. 

The complete list of Sanskrit books written by Śrī Nārayaṇa Bhaṭṭa\index{Srī Nārayaṇa Bhaṭṭa@Śrī Nārayaṇa Bhaṭṭa} are as follows:{\sl  Vraja-bhakti-vilāsa {\rm (description about the forests of Vraja)}, Vraja-pradīpikā, Vrajotsva-candrikā {\rm (description of festivals in Vraja)},\break Vraja-mahodadhi, Vrajotsvāhlādinī, Bṛhad-vraj-guṇotsva {\rm (Description of\break the villages in Vraja)}, Vraja-prakāśa, Bhakta-bhūṣaṇa-sandarbha {\rm (Philosophy)}, Bhakti-viveka {\rm (principle of devotion)}, Bhakti-rasa-taraṅgiṇī {\rm (description of {\sl rasa}-s of {\sl bhakti})}, Sādhana-dipikā {\rm (details about the practice of devotion)}, Rasikāhlādinī {\rm (commentary on {\sl Bhāgavata Purāṇa})}, Dharma-pra\-vartinī, Premāṅkura-nāṭaka} {\rm (Play)} {\rm (H. Dasa 1272)}. 

{\rm The complete list of Sanskrit books written by Śrī Viśvanātha Cakra\-varti are as follows:}{\sl  Śrī-kṛṣṇa-bhāvanāmṛtam {\rm (poetry)}, Śrī-gaurāṅga-līlā\-mṛtam {\rm (biography)}, Aiśvarya-kādambinī {\rm (theology)}, Stavāmṛta-laharī\break {\rm (panegyrics)}, Sindhu-bindu {\rm (devotion)}, Ujjvala-kiraṇa  {\rm (devotional)},\break Bhāgavatāmṛta-kaṇā {\rm (theology)}, Rāga-vartma-candrikā {\rm (devotion)},\break Mādhurya-kādambinī {\rm (devotion)}, Gauraṅga-svarūpa-candrikā {\rm (theology)}, Camatkāra-candrikā {\rm (poetry)}, Kṣaṇadā-gīta-cintāmaṇi {\rm (poetry)}, Sārārtha-darśinī {\rm commentary on} Bhāgavata Purāṇa, Sārārtha-varśiṇī {\rm commentary on} Bhagavad Gītā\index{Bhagavadgītā@\textsl{Bhagavadgītā}}, Bha\-kti-sāra-pradarśinī {\rm commentary on} Bhakti-rasāmṛta-sindhu, Ānanda-ca\-ndri\-kā {\rm commentary on} Ujjvala-Nilamaṇi, Bhakta-harśiṇi {\rm commentary on} Gopāla-tāpanī Upaniṣad, Mahati {\rm commentary on} Dāna-keli-kaumudī, Sukhavartinī {\rm commentary on} Ānanda-vṛndāvana-campu, Subodhinī commentary on Alaṅ\-kāra-kaustubha\index{Alaṅkāra Kaustubha@\textsl{Alaṅkāra Kaustubha}}, {\rm Commentary on} Caitanya-caritā\-mṛta, {\rm Commentary on} Prema-bhakti-candrikā} {\rm (H. Dasa 1370)}. 

The complete list of Sanskrit books written by Śrī Baladeva Vidyā\-bhuṣaṇa\index{Baladeva Vidyābhuṣaṇa} is as follows: {\sl Siddhānta-ratna {\rm (Supplement to {\sl Govinda-bhāṣya}, his commentary on Vedānta-sūtra)}, Prameya-ratnāvalī {\rm (philosophy)},\break Siddhānta-darpaṇa {\rm (Philosophy)}, Vedānta-syamantaka {\rm (philosophy)},\break {\rm Commentary on} The Ten Principal Upaniṣads, {\rm Commentary on} Gopāla-tāpanī Upaniṣad, Gitā-bhuṣaṇa {\rm commentary on} Bhagavad Gītā, Nāmārtha-sudhā {\rm commentary on} Viṣṇū-sahasra-nāma, Govinda-bhāṣya {\rm commentary on} Vedā\-nta-sūtra, Vaiṣṇava-nandinī {\rm commentary on} Bhāgavata Purāṇa, Sāraṅga-raṅgadā {\rm commentary on} Laghu-bhagavatāmṛtam {\rm of Śrī Rūpa\break Gosvāmi},\index{Rūpa Gosvāmin} Rasika-raṅgadā {\rm commentary on} Padyāvalī, Stavamālā-bhuṣaṇa {\rm commentary on} Stavamālā, {\rm Commentary on} Govinda-virudāvalī, Aiṣva\-rya-kādambinī, {\rm Commentary on} Kṛṣṇa-bhāvanāmṛta, {\rm Commentary on}\break Gopāla-champū, Vai\-ya\-karaṇa-kaumudi {\rm (Sanskrit grammar)}, Kāvya-\Break kau\-stubha {\rm (rhetoric)}, Sā\-hitya-kaumudī {\rm (rhetorics)}, Tattva-dīpikā {\rm (philosophy)}, Govinda-bhāṣya-kārikā} {\rm (philosophy)} {\rm (Dasa 1986:1292)}. 

\section*{Appendix C: Contemporary Sanskrit\hfill\break Schools in Vraja}\index{Contemporary Sanskrit}\index{Vraja, sanskrit schools}
\begin{enumerate}
\itemsep=0pt
\item Śrī Caitanya Sanskrit Śikṣā Saṁsthāna, Vṛndāvana 
\item Śrī Dharma Saṅgha Sanskrit Mahavidyālaya Vṛndāvana 
\item Śrī Kuñjavihārī Sanskrit Mahavidyālaya Vṛndāvana 
\item Śrī Nimbārka Sanskrit Mahavidyālaya Vṛndāvana 
\item Śrī Raṅga-lakṣamī Sanskrit Mahavidyālaya Vṛndāvana 
\item Śrī Rāmanuja Hayagrīva Sanskrit Mahavidyālaya\hfil\break Vṛndāvana 
\item Śrī Kṛṣṇa Darśana Sanskrit Mahavidyālaya Vṛndāvana 
\item Śrī Brahma Vidyā Sanskrit Mahavidyālaya Vṛndāvana 
\item Śrī Gulaba Hari Sanskrit Mahavidyālaya Vṛndāvana 
\item Śrī Rādhā Kṛṣṇa Sanskrit Uttara Madhyamā Vidyālaya,\hfil\break Vṛndāvana 
\item Śrī Śrīnivāsa Sanskrit Uttara Madhyamā Vidyālaya,\hfil\break Vṛndāvana 
\item Śrī Nārayaṇa Veda Vedāṅga Sanskrit Uttara Madhyamā\hfil\break Vidyālaya, Vṛndāvana 
\item Śrī Vaiṣṇava Sanskrit Uttara Madhyamā Vidyālaya,\hfil\break Vṛndāvana 
\item Śrī Śyāmānanda Sanskrit Vidyālaya, Vṛndāvana 
\item Śrī Rāmānanada Sanskrit Uttara Madhyamā Vidyālaya,\hfil\break Vṛndāvana 
\item Śrī Bhakti Sanskrit Uttara Madhyamā Vidyālaya,\hfil\break Vṛndāvana 
\item Śrī Śridhara Vidyāniketana, Vṛndāvana 
\item Śrī Māthura Sanskrit Vidyālaya, Mathurā 
\item Śrī Dvārkeśa Sanskrit Uttara Madhyamā Vidyālaya,\hfil\break Mathurā 
\item Seth Biśanalāla Sanskrit Uttara Madhyamā Vidyālaya,\hfil\break Mathurā 
\item Śrī Mādhava Sanskrit Mahavidyālaya, Govardhana 
\item Śrī Sārvabhauma Sanskrit Mahavidyālaya, Govardhana 
\item Śrī Nimbārka Sanskrit Uttara Madhyamā Vidyālaya,\hfil\break Neemagaon 
\item Śrī Gājīpura Sanskrit Mahāvidyālaya, Barasānā 
\item Śrī Rājeśvarī Sanskrit Vidyālaya Barasānā 
\item Śrī Caitanya Sanskrit Śikṣā samsthāna, Radhakuṇḍa 
\item Śrī Gaurāṅga Sanskrit Vidyālaya, Rādhākuṇḍa 
\item Śrī Ratana Moti Sanskrit Mahāvidyālaya, Gokula 
\item Śrī Kārṣṇī Sanskrit Uttara Madhyamā Vidyālaya, Gokula 
\item Institute of Oreintal Philosophy, Vṛndāvana 
\end{enumerate}


\begin{thebibliography}{99}
\itemsep=2pt
\bibitem[]{chapter5_chapter5_item1} 
Brazinski, Jan. (2016) {\sl Haridas Das Babaji: Servant of Gauḍīya Sahitya}. 
\url{http://jagadanandadas.blogspot.in/2016/10/haridas-das-babaji-servant-of-gaudiya.html} 

\bibitem[]{chapter5_item2} 
Dasa, Sri Haridas. (1986) {\sl Śri Gauḍīya Vaiṣṇava Abhidhāna}. Vol~3. Navadvip: Sri Haribol Kutir. 

\bibitem[]{chapter5_item3} 
Dasa, Krishna Baba (Trans.) (1938) {\sl Vrajotsva-candrikā}. Kusum Sarovar: Self-published. 

\bibitem[]{chapter5_item4} 
Dasa, Satyanarayana (Trans.) (2015) {\sl Śrī Tattva Sandarbha: Vaiṣṇava Epistemology and Ontology}. Vrindavan: Jiva Institute of Vaishnava Studies. 

\bibitem[]{chapter5_item5} 
Dasa, Shukavak N. (1999) {\sl Hindu Encounter with Modernity}. Los Angles: Sanskrit Religions Institute. 

\bibitem[]{chapter5_item6} 
De, Sushil Kumar. (1986) {\sl Early History of the Vaishnava Faith and Movement in Bengal}. Calcutta: Firma KLM Private Limited. 

\bibitem[]{chapter5_item7} 
“Distribution of the 22 Scheduled Languages.” 2001 Census of India. Government of India: Ministry of Home Affairs. 
\url{http://www.censusindia.gov.in/Census_Data_2001/Census_Data_Online/Language/parta.htm}

\bibitem[]{chapter5_item8} 
Dimock, Jr, Edward C (Trans.) (1999) {\sl Caitanya Caritāmṛta of Kṛṣṇadāsa Kaviraj: A Translation and Commentary}. Cambridge, MA: Harvard University Department of Sanskrit and Indian Studies. 

\bibitem[]{chapter5_item9} 
Gokak, Vinayak Krishna. (1985) {\sl Foreward: Three Decades: A Short History of Sahitya Akademi 1964--1984}. New Delhi: Sahitya Akademi. 

\bibitem[]{chapter5_item10} 
Growse, F.S. (1979) {\sl Mathurā: A District Memoir}. New Delhi: Asian Educational Services. 

\bibitem[]{chapter5_item11} 
Haberman, David L. (Trans.) (2003) {\sl Bhaktirasāmṛtasindhu}. Delhi: Motilal Banarsidass Publishers Pvt. Ltd. 

\bibitem[]{chapter5_item12} 
Horstmann, Monika. (1999) {\sl In Favor of Govinddevjī: Historical documents relation to a deity of Vrindaban and Eastern Rajasthan}. New Delhi: Indira Gandhi National Centre for the Arts. 

\bibitem[]{chapter5_item13} 
Jacob, Colonel G. A., Collected. (1900) {\sl Laukika-nyāyāñjaliḥ : A Handful of Popular Maxims Current in Sanskrit Literature}. Bombay: Niranya-Sagar Press. 

\bibitem[]{chapter5_item14} 
“Jnanpith Award.” Bharatiya Jnanpith. \url{http://www.jnanpith.net/awards/jnanpith-award}.

\bibitem[]{chapter5_item15} 
“Jnanpith Laureates”. Bharatiya Jnanpith. \url{http://www.jnanpith.net/page/jnanpith-laureates}. 

\bibitem[]{chapter5_item16} 
Kapoor, O.B.L. (1994) {\sl The Philosophy and Religion of Śri Caitanya}. New Delhi: Munishiram Manoharlal Publishers. 

\bibitem[]{chapter5_item17} 
Kapoor, Subodh (Ed.) (2002) {\sl The Indian Encyclopedia: Biographical, Historical, Religious, 
Administrative, Ethnological, Commerical, and Scientific}, Vol.~11. New Delhi: Cosmo Publications. 

\bibitem[]{chapter5_item18} 
Miśra, Chavinātha. (1978) {\sl Nyāyoktikoṣaḥ}. Delhi: Ajanta Publications. 

\bibitem[]{chapter5_item19} 
Mufwene, Salikoko S. (2004) “Language Birth and Death.” {\sl Annual Review of Anthropology}. Vol.~33: 201--222. 

\bibitem[]{chapter5_item20} 
Pollock, Sheldon. (2001) “The Death of Sanskrit.” {\sl Comparative Studies in Society and History}. 43.2: 392--426. 

\bibitem[]{chapter5_item21} 
Pollock, Sheldon. (2005) {\sl The Ends of Man at the End of Premodernity}. Amsterdam: Royal Netherlands Academy of Arts and Sciences. 

\bibitem[]{chapter5_item22} 
Potter, Karl H., (Ed.) (1995) {\sl Encyclopedia of Indian Philosophies}. Vol~1. Delhi: Motilal Banarsidass. 

\bibitem[]{chapter5_item23} 
Rao, D.S. (1985) {\sl Three Decades: A Short History of Sahitya Academi 1964--1984}. New Delhi: Sahitya Academi. 

\bibitem[]{chapter5_item24} 
Sharma, Hari Dutt. (2015) {\sl Report on the 16th World Sanskrit Conference}. Bangkok, Thailand: International Association of Sanskrit Studies. 

\bibitem[]{chapter5_item25} 
Stewart, Tony K. (2010) {\sl The Final Word: The Caitanya Caritāmṛta and the Grammar of Religious Traditions}. New York: Oxford University Press. 

\bibitem[]{chapter5_item26} 
Shoba, V. (2009) “Sanskrit’s first Jnanpith winner is a ‘poet by instinct’”. {\sl The Indian Express}. The Express Group, 1/14/2009. 
{\fontsize{11}{13}\selectfont\url{http://archive.indianexpress.com/news/sanskrits-first-jnanpith-winner-is-a-poet-by-} \url{instinct/410480/0}}

\end{thebibliography}

\theendnotes
