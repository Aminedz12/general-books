\chapter{On “Death of Sanskrit”}\label{chapter8}

\Authorline{Manogna Sastry}
\lhead[\small\thepage\quad Manogna Sastry]{}

\section*{Introduction}

‘The Death of Sanskrit’ – the essay published under this rather astonishing title by the Indologist Sheldon Pollock (Pollock 2001) bids us to a dynamic re-examination and a defense of our invaluable literary instrument. The most august, venerable and determinative creations of the Indian race have been written in Sanskrit, and pronouncing death upon the language is tantamount to an inability in understanding the true role of this divine tongue. Pollock’s essay merits a clear understanding of the burden of his song. The motives for the author’s position and the method of his reasoning are studied here in order to understand why he would see efforts to promote the language as a politically biased “exercise in nostalgia”. The distinction the author makes between living and dead languages and his attempts at placing Sanskrit in the latter category are closely examined.  The four major questions the author considers as a part of his analysis in proclaiming death upon the language consist of the impact of the vernaculars upon Sanskrit, the political, social and spiritual components that played a role in the decline of the language and the factors he considers necessary for consolidating the language in today’s times. Pollock cites four specific historical instances that apparently illustrate his stance and these are critically examined.
\newpage

\begin{myquote}
India of the ages is not dead nor has she spoken her last creative word; she lives and has still something to do for herself and the human peoples.\break And that which must seek now to awake is not an anglicised Oriental people, docile pupil of the West and doomed to repeat the cycle of the Occident's success and failure, but still the ancient immemorable  Shakti recovering her deepest self, lifting her head higher towards the supreme source of light and strength and turning to discover the complete meaning and a vaster form of her Dharma.\index{dharma@\textsl{dharma}} \hfill– Sri Aurobindo\index{Aurobindo}
\end{myquote} 

Whenever one endeavours to comprehend the value of a culture and appreciate, especially a culture into which he is born and forms the source of one’s highest standards, he tends to gloss over the shortcomings and defects of his own heritage. At this juncture, it may prove helpful to have insights from an outsider to better one’s reflection. For the outsider too, there are ways of perceiving an alien culture. There is, firstly, the sympathetic viewer and there are several foreigners who have embraced Indian culture and contributed enormously to its value, from Sister Nivedita\index{Nivedita, Sister} to Romain Rolland\index{Rolland, Romain}. Then, there is the impartial and objective critic, who, without any prejudice in objective, balances the worth against the weakness and whose objective criticism\index{criticism} must be welcomed to balance one’s own study. The third is the belligerent critic, whose motives need to be carefully studied as well as his methods. And Sheldon Pollock lends himself to this third type. His overcharging declaration of the ‘death’ of Sanskrit\index{Sanskrit!revival of} summons a vigorous analysis and re-statement of the role of the language in present day India.

While Pollock’s methods\index{Pollock!methods} of analysis and recognition of the function of Sanskrit\index{Sanskrit!revival of} as the literary instrument\index{literary!instrument} of India reveals a conflict between Western and Asian approaches to understanding Oriental culture, his motives are revealed in the very first part of the essay. Pollock’s bias\index{bias} against the recent attempts at revitalising Sanskrit\index{Sanskrit!revival of} is revealed through his disdain for efforts by the Indian government as well as private organizations here for promoting the language. He holds the rise of Hindutva\index{Hindutva} and the election of the Bharatiya Janata Party to power\index{power!political} at the central levels of government responsible for reviving a nationalistic culture. Efforts to promote the language independent of the Hindutva association that have been carried out in post independent India by several individuals and organizations are not given any credit. Pollock believes that this revival is driven by political motivations, and in the process, “distorted images of India’s past” are being created including the viewpoint that Sanskrit is native to India and that it existed during the Indus Valley Civilization\index{Indus Valley Civilization}. He sets aside prevailing debates about the origin and geographical indigenousness\index{geographical!indigenousness} of the language as well as discussions of the forms of Sanskrit during the latter part of the Harappan\index{Harappan} civilsation. Pollock further betrays his lack of recognition of the true role of Sanskrit in defining the identity of the Indian soul when he scorns any championing of the language by nationalistic elements, calling them “farcical repetition of Romantic myths of primevality”. 

Indeed, it is vitally important for a nation to be cautious in ensuring that exaggerated claims about its past are not made a part of its national narrative. Yet, the process of rediscovery and reinterpretation of the Indian identity that is currently taking place, in the context of globalised and liberalised policies at all levels of the State, finds the Indian of today looking back at his history without the coloured glasses\index{coloured glasses} of subjection that his colonial oppressor forced him to wear.  It is very important for one to consider the state of the language and the field, leaving apart from the fringe elements who are occupying a small place. The number of people for whom Sanskrit is associated with the very template of the cultural fabric\index{cultural!fabric} of India is far too big; and to dismiss this section by labeling and associating the campaign with only ultra-nationalistic elements amounts to closing the debate for the majority of the section. And Pollock assails this process of reinterpretation and growth, which is marked by a feature of self-assertive power\index{power!political}, by calling it ‘the State’s anxiety’ and condemns any effort that has gone into the preservation and propagation of the language as a part of a ‘melancholic history.’ Had this condemnation arisen from a lesser mind, it would have been forgivable. For a critic of Pollock’s caliber, then, to question efforts in post independent India for the preservation and sharpening of the instrument in which the greatest canons of the land have been written, is but a pointer to the dubious and hostile motives of the man.

Pollock goes on to enlist the efforts that have gone into popularising Sanskrit since independence, and states that there have been hardly any results for all the funds and resources that have been allocated for the development of the language. While Pollock speaks about the political checks that he sees as being placed in this ‘revisionism’, he seems to be speaking of only one face of it. He makes no mention at all of how during the decades before the rise of the ‘Hindutva’\index{Hindutva}, Sanskrit had to contend with the active State and governmental neglect. In 1994, attempts were made under the P V Narasimha Rao\index{Rao, P V Narasimha} government to remove Sanskrit from the CBSE syllabus\index{CBSE syllabus}. It was the intervention of the Supreme Court that ended the Government’s foolish move. The role played by these political and judicial checks in post independent India are not even discussed by Pollock while he maintains that any effort to revive the language is futile.

Having expressed his derision and disdain for the work done in promoting the language, especially as a spoken one, Pollock finally states the crux of his essay – “Government feeding tubes\index{Government feeding tubes} and oxygen tanks may try to preserve the language in a state of quasi-animation, but most observers would agree that, in some crucial way, Sanskrit is dead” (p. 393). When does one declare a language to be dead? In answering this question, Pollock does not establish any rigorous conditions but speaks of the distinction between written and natural languages. While admitting the influence of Sanskrit across Asia, he maintains that there have not been any theories on “whether and if so, when, Sanskrit culture ceased to make history”, and why the language could not remain creatively vital.

A further study of the paper reveals that Pollock’s definition of the ‘death’ of Sanskrit is primarily a reference to the dearth of production of creative works, especially in the domain of {\sl kāvya}, over the last millennium. His method consists of singling out particular instances - over different periods of time spread over a thousand years and geographically scattered across the land - to support his thesis and overlooking those that invalidate it, pointing thus to a manner of biased\index{bias} scholarship. His consideration, of phases when emphasis was laid on documentation and “reinscription and restatement” as periods of decline and decay of the language, is debatable; and while doing so, Pollock conveniently casts away vitally important canons from these periods which readily disprove his thesis. Thus, Pollock’s method\index{Pollock!methods} of arbitrary selection\index{arbitrary selection} drives his thesis from two major angles – (a) his exclusive focus on the genre of {\sl kāvya} (excluding even {\sl stotra-s}\index{stotra literature@\textsl{stotra} literature}) as the only measure of the vitality of the language, and (b) his very small set of four instances spread over a millennium across the country during which the language declined.

In his study of the “history of Sanskrit knowledge systems,” Pollock (2001:393) remarks that “the two centuries before European colonialism\index{colonialism} decisively established itself in the sub continent around 1750 constitute one of the most innovative epochs of Sanskrit systematic thought”. Sanskrit thrived in India until then, according to him, with massive creative output in every major genre of the language. He does not make any reference at all to the effects of the Islamic invasion\index{Islamic invasion} on the social and cultural fabric of the land and particularly on the linguistic scene. Yet, by 1800, the creative ability has disappeared from the land. Thus, according to Pollock, Sanskrit was alive enough to withstand all the shocks and turmoil the land had faced in the first half of the millennium to lead to the {\sl Navya} movement\index{Navya movement@\textsl{Navya} movement} but was dead within the next fifty years, by 1800. Calling this a momentous rupture, he admits the analysis of this sudden change is complex but himself selects only the genre of {\sl kāvya} to study the decline in the language, quoting the Gujarati poet Dalpatrām Dahyabhai’s work in 1857 as further evidence and  highlights his case throughout the essay.

Without any analysis of the various roles Sanskrit has played in expressing the very psyche and mind of the Indian; without any recognition of the central role it has played in formulating some of the most profound spiritual, philosophical, intellectual and emotional conceptions of not just the Indian but of the human race; and by further not considering the various genres in which Sanskrit has continued to survive and thrive in India, Pollock’s proclaiming death on the language on account of its no longer being the first language of the majority population in the country is contrived and purposeful overcharging. His carefully chosen examples that purportedly demonstrate\index{purportedly demonstrate} the death of the language over the millennium consist of: (1) “The disappearance of Sanskrit literature in Kashmir,” (2) “its diminished power\index{power!political} in sixteenth century Vijayanagara,” (3) “its short-lived moment of modernity at the Mughal court,” and (4) “its ghostly existence in Bengal on the eve of colonialism\index{colonialism}.” The four cases are duly considered in the following sections.

\section*{The four cases:}

\makeatletter
\renewcommand\thesubsection{\@arabic\c@subsection}
\makeatother
\subsection{‘The Lady Vanishes’}

Pollock’s first case consists in analyzing the decline, or rather in his own words, “the complete disappearance”, of Sanskrit from Kashmir. Pollock begins his case by considering the scene in the valley during 1140, a time during which Kashmir served as a celebrated and revered seat for the learning of the language. Alaṅkāra\index{Alankara (poet)@Alaṅkāra (poet)} holds a gathering in honour of his brother Maṅkha\index{Mankha@Maṅkha}, the author of the epic {\sl Śrīkaṇṭhacarita}\index{Srikanthacarita@\textsl{Śrīkaṇṭhacarita}}. The assembly hosts some of the greatest names in the field, from Ruyyaka\index{Ruyyaka}, Maṅkha’s mentor, to Trailokya\index{Trailokya (poet)} and Jinduka\index{Jinduka (poet)}. Kalhaṇa\index{Kalhana@Kalhaṇa}, whose work {\sl Rājataraṅgiṇī\index{Rajatarangini@\textsl{Rājataraṅgiṇī}!(Kalhaṇa)}} is considered the most noteworthy historical poem of the language, is present in this august gathering as well. Every major branch of the language is represented through its best exponent in this exalted congregation, establishing the premier position held by Kashmir during the time. Not only is the assembly playing host to literary titans, but also to the leading names in the fields of physicians, philosophers and architects who were present at the gathering.

Pollock builds a strong case to highlight that, the period of 1140CE represented the zenith of the intellectual development attained by Kashmir. He is right in highlighting the remarkable lineage the place boasted of since the seventh century. Kṣemendra\index{Ksemendra@Kṣemendra}, the polymath and author of {\sl Bhārata Mañjarī\index{Bharata Manjari@\textsl{Bhārata Mañjarī}}} and {\sl Bṛhatkathāmañjarī\index{Brhatkathamanjari@\textsl{Bṛhatkathāmañjarī}}} represented the best of the eleventh century. The twelfth century saw Maṅkha, along with his brothers, Jayadratha\index{Jayadratha (poet)} and Kavirāja\index{Kaviraja (poet)@Kavirāja (poet)}, rise to literary acclaim with other creative titans.

Yet, within fifty years from 1140CE, the literary landscape has changed beyond recognition. Pollock highlights that production of new creations in all the major genres of the language has come to a standstill. From being a focal point of learning, Kashmir has become barren. Fifteenth century is the next age during which significant compositions emerge again. The royal patronage\index{patronage} issued to Sanskrit\index{Sanskrit!revival of} by the Sultan Zain-ul-'abidin\index{Zain-ul-"'abidin} created the necessary conditions for the revival, says Pollock. Jonarāja was the chief scholar at the Sultan’s court and he continued Kalhaṇa’s {\sl Rājataraṅgiṇī}. The gap in the narrative of the rulers of the land in the epic poem highlights the political and cultural transformations that have swept through the valley. Śrīvara’s\index{Srivara@Śrīvara} anthology, {\sl Subhāṣitāvalī}\index{Subhasitavali@\textsl{Subhāṣitāvalī}}, enlists more than 350 poets of the land, and Pollock (2001:397) asserts that even though the large number of poets enlisted can be construed to indicate that the period since 1140 did indeed produce several works in the language, the anthology does not produce any proof that works of significance were created.

Pollock then inquires into the reasons for such rupture in the creative output from the literary scene of Kashmir. He mentions that works could have been lost in the various calamities that befell the land, including frequent fires in the capital. He touches very briefly on the Mongol invasion of 1320, but makes no note at all of the remarkable transformations that took place in the social, religious and cultural fabric of the land during the reigns of Sultan Zain-ul-'abidin’s\index{Zain-ul-"'abidin} predecessors, especially Sikander Shāh Mīr\index{Sikandar Shah Mir (“Butshikhan”)@Sikandar Shāh Mīr (“Butshikhan”)} (ref. Kaw 2004:108), who went on a spree of destroying idols, temples and Hindu culture\index{temple!destruction of} in every village of the valley, as documented by Jonarāja\index{Jonaraja@Jonarāja} in {\sl Rājataraṅgiṇī}\index{Rajatarangini@\textsl{Rājataraṅgiṇī}!(Jonarāja)} as well. He points out that, even though Kashmir had faced a similar breakdown during the late ninth century, it was able to recover and bounce back to produce some of the finest works in the language through Abhinavagupta\index{Abhinavagupta}, Kuntaka\index{Kuntaka} and Mahimabhaṭṭa\index{Mahimabhatta@Mahimabhaṭṭa} But, he believes the twelfth century is different as the deterioration could never be reversed. Very briefly admitting that “the possibility exists that this picture of literary collapse is an artifact of our data,” (Pollock 2001:397) and that, the loss of works in the various calamities that befell the state could paint a different picture, Pollock sets it aside to present his interpretation of the decline. One thus clearly observes a pattern of the author to mention in passing probable causes for the decline but peremptorily dismisses all of them except the one he chooses to focus upon.

Pollock cites particular references from Kalhaṇa’s\index{Kalhana@Kalhaṇa} {\sl Rājataraṅgiṇī}\index{Rajatarangini@\textsl{Rājataraṅgiṇī}!(Kalhaṇa)} of the excesses and debauchery of the Hindu kings, especially King Harṣa\index{Harsa@Harṣa} and holds them responsible for creating a scenario where culture degenerated. “It is a direct consequence of this, one has to assume, that for poets like Maṅkha political power\index{power!political} had not only become irrelevant to their lives as creative artists and to the themes of their poetry but an impediment,” (Pollock 2001:399). At this juncture, one must wonder why Pollock conveniently leaves out the patronage\index{patronage} given by Harṣa’s\index{Harsa@Harṣa} predecessors in the Lohara dynasty\index{Lohara dynasty} to Sanskrit. 1286 is a significant year in the history of the land as complete anarchy broke out during the time. The attack by Rinchana, the Tibetan ruler who plundered the capital Srinagar\index{Srinagar}, the calamity that befell the whole land as well as the decisive victories of the Turks\index{Turks} over the Hindus, all occurring during the thirteenth century and the first three decades of the fourteenth century find no account in Pollock’s work. Having ignored these momentous changes as critical causes for the breakdown of socio-political structures in the place, and only mentioning the later invasion of 1320 in a single line, Pollock states, “But none of these possibilities seem very likely.” This is a highly contentious statement from the author. On the one hand, he ignores the most savage and formidable challenges Hindu culture as a whole faced during the Islamic invasion\index{Islamic invasion} in Kashmir\index{Kashmir}, and on the other, he emphatically places the blame on the “breakdown of the courtly-civic ethos of Kashmir” during the rule of the Hindu kings.

The stark contrast in the manner in which Pollock absolves the Turkish kings of any responsibility for the blows faced by Sanskrit literary culture is illustrated through the manner in which he condones Sultan Zain-ul-'abidin’s destruction of the idol of Goddess Śāradā\index{Goddess Sarada, destruction of the idol of@Goddess Śāradā, destruction of the idol of}. Pollock’s double standards\index{double standards} and duplicity in calling the Sultan a “pious devotee”, and describing the act of destruction as something the goddess made him do, “the goddess ‘made him smash to pieces her very own image’” is very revealing of Pollock’s methods\index{Pollock!methods} of analysis where he casts no light at all on any observation, or even the historical evidence that challenges and counters his tenuously built analysis.

Further highlighting his case, Pollock maintains that Sanskrit literature historically propagated out of Kashmir and the lack of any such spread post twelfth century is seen as validation of his argument that the language had significantly declined in quality and quantity during the period. Pollock points out that the very nature of the works is one of “culture reduced to reinscription and restatement”. Originality in any of the major branches of literature has disappeared but, continued secondary productions demonstrate the unbroken work carried out in the language. Thus, Pollock states that, as what was lost during the period was the more principal element of creating original work in literature, Sanskrit has disappeared from its once hallowed seat, viz. Kashmir. He does not mention even in passing the revival of the language in 1857 under the reign of Raṇavīra Singh\index{Ranavira Singh@Raṇavīra Singh}, who is said to have commissioned more than 30 works in all genres of Sanskrit\index{Sanskrit!revival of} literature. The king’s courtly poets including Pandit Sahibram, Viśveśvara\index{Sahibram, Visvesvara@Sahibram, Viśveśvara} generated great keenness for the language once again in the valley (Majumdar\index{Majumdar, R C} 2001a:165) and this finds no account in Pollock’s work.

\subsection{‘Sanskrit in the City of Victory and Knowledge’}

If the first case had at its heart the role played by the breakdown of the “courtly-civic ethos” and the “debauchery of the Hindu kings”, the second case has at its crux the “complicated politics of literary language and far sharper competition among literary cultures” in Vijayanagara (Pollock 2001:400), one of the greatest empires ever built in southern India. In holding the rise of the vernaculars responsible for the decline in Sanskrit, Pollock brings a dubious critical eye to the multilingual nature of life in the empire. In spite of Kṛṣṇadevarāya\index{Krsnadevaraya@Kṛṣṇadevarāya} being a Kannada\index{Kannada} king, Pollock argues that he did little to champion Kannada\index{Kannada} at the court. Timmaṇṇa\index{Timmanna(kavi)@Timmaṇṇa(kavi)} was the only court poet of the language while the Dāsa tradition\index{Dasa tradition@Dāsa tradition} flourished in the empire. Sanskrit in the state saw no new creations but then, Pollock admits that scholarship in the language reached its zenith during the period. The governors of the state too were very well versed in the language and were learned men but, of “only reproductive and not original learning,” he states. As J Hanneder analyses in his response ‘On “The Death of Sanskrit” to Pollock’s paper, “to state this of scholars like Vidyāraṇya\index{Vidyaranya@Vidyāraṇya} and Sāyaṇa\index{Sayana@Sāyaṇa} who were crucial figures in establishing through their literary activities what came to be considered the fundamental canon of Hindu religious tenets is totally unconvincing.” (Hanneder 2002:307)

In a further analysis, especially in literature that was produced in Vijayanagara, Pollock laments that the quality of {\sl kāvya} written was poor and wonders how and why the works from the court survived at all (Pollock 2001:401), after the empire was destroyed in 1565. He believes that the dynamism and spirit seen in the works of Kannada\index{Kannada} and Telugu\index{Telugu} are absent in Sanskrit. Thus, in this multilingual empire, he believes Kannada was poorly represented in the court, Sanskrit was only used for State purposes and Telugu\index{Telugu} was the main medium in which even Kṛṣṇadevarāya\index{Krsnadevaraya@Kṛṣṇadevarāya} composed. Using words such as “conflict” and “competition” between the vernaculars on the one hand and Sanskrit on the other, Pollock seems inclined to build for the reader an image that perpetuates a sharp dichotomy between the two. The scenario - where the vernaculars and Sanskrit thrived, and creative energies\index{creative energies} flowed mutually between them as against a fight for power\index{power!political} - is not given even a slight consideration by the author, thus bringing his motives under scrutiny. 

Selecting {\sl Jāmbavatīpariṇaya\index{Jambavatiparinaya@\textsl{Jāmbavatīpariṇaya}}}, King Kṛṣṇadevarāya’s\index{Krsnadevaraya@Kṛṣṇadevarāya} work written in Sanskrit, he reviews the work as one whose theme has been written about before historically with no novel element in the effort and that the significance of the work is more because of its association with the king and the “political narrative of the Vijayanagara Empire”. Despite {\sl kāvya} not being a genre in which personal expression has been a characteristic feature of the language, Pollock holds {\sl Jāmbavatīpariṇaya}\index{Jambavatiparinaya@\textsl{Jāmbavatīpariṇaya}} to a different standard of judgment. The defeat of the Orissa king Gajapati\index{Gajapati (King)} to Kṛṣṇadevarāya\index{Krsnadevaraya@Kṛṣṇadevarāya} and the spread of the empire until the Bay of Bengal on the east and the jubilation that followed this victory are seen as finding expression through the work {\sl Jāmbavatīpariṇaya}\index{Jambavatiparinaya@\textsl{Jāmbavatīpariṇaya}}. Thus, the work is not a true reflection of the prominent and creatively active role played by Sanskrit in state but is dismissed as a body that is only politically significant. Most of the work of this period in Sanskrit is an expression of the “mytho-political representation of the king’s person” and of the nature of “imperial documents”. Having ignored {\sl stotra-s} and religious works as significant examples of literary activity earlier in the essay, Pollock now points out that the vernaculars were more active in propagating religious sentiments in Vijayanagara and Sanskrit was only reserved for official state purposes. Thus, it is not the absence of works in Sanskrit that establish its decline during the period but, the nature of the works that Pollock questions, using standards that are arbitrary.

\subsection{The Last Sanskrit Poet}

Pollock’s third section, The Last Sanskrit Poet, considers three remarkable figures – Jagannātha Paṇḍitarāja\index{Jagannatha, Panditaraja@Jagannātha, Paṇḍitarāja}, the Jain poet Siddhicandra\index{Siddhicandra} and Kavīndrācārya\index{Kavindracarya Sarasvati@Kavīndrācārya Sarasvatī}. Hailing Jagannātha as the last poet to have attained canonical status in the legion of {\sl kavi}-s, Pollock laments that one does not have literary information about the life of the poet despite his period of existence being closer to our present time. He then goes on to make the startling inference that this is because “the cosmopolitan space occupied by Sanskrit literature\index{stotra literature@\textsl{stotra} literature} for much of the two preceding millennia persisted well into the seventeenth century despite what are represented as fundamental changes in the political environment with the coming of the Mughals in the previous century,” (Pollock 2001:404). Repeatedly, one finds Pollock’s method\index{Pollock!methods} - of dismissing the negative impact of the Islamic invasions\index{Islamic invasion} on the creation, preservation and restoration of the Sanskrit language in India – as a characteristic attribute of his work.

The manner in which Jagannātha’s {\sl Bhāminīvilāsa}\index{Bhaminivilasa@\textsl{Bhāminīvilāsa}} and {\sl Rasagaṅgādhara}\index{Rasagangadhara@\textsl{Rasagaṅgādhara}} achieved massively widespread dissemination is never seen again in the history of the land, and his {\sl Rasagaṅgādhara}\index{Rasagangadhara@\textsl{Rasagaṅgādhara}} remains among the finest literary treatises in the language. Briefly describing Jagannātha’s personal story - as a Brahmin born into an orthodox Telugu family, who fell in love with a Muslim woman\index{Muslim!woman}, and died by drowning in the Gaṅgā - Pollock then moves onto the study of the {\sl Navya} age\index{Navya age@\textsl{Navya} age}. The departure from the {\sl prācya} style\index{pracya@\textsl{prācya}} brought with it a novel approach and tone and methods of analysis into the language. Along with Jagannātha, some of the extraordinary exponents of this movement include Bhaṭṭoji Dīkṣita\index{Bhattoji Diksita@Bhaṭṭoji Dīkṣita}, the author of {\sl Siddhāntakaumidī}, Nīlakaṇṭha Caturdhara\index{Nilakantha Caturdhara@Nīlakaṇṭha Caturdhara}, Siddhicandra\index{Siddhicandra} the Jain poet, and Kavīndrācārya\index{Kavindracarya Sarasvati@Kavīndrācārya Sarasvatī}.

Analysing Siddhicandra’s\index{Siddhicandra} time and place in the Mughal court, Pollock presents Abul Fazal\index{Abul Fazal} and Akbar\index{Akbar} himself as men who provided patronage\index{patronage} to Siddhicandra\index{Siddhicandra} and in the process, played significant roles in his life. Siddhicandra\index{Siddhicandra} had learnt Persian\index{Persian}, and the period saw many Sanskrit works translated into Persian, while Sanskrit too was influenced by the Mughal language of the State. While he combined impressive learning in both languages, Pollock believes that the quality of his work is mediocre, and does not embody and express the transformative events of the age. Discussing Siddhicandra’s\index{Siddhicandra} debates with the Mughal emperor Jahangir\index{Jahangir} and empress Nūr Mahal, Pollock speaks about how Siddhicandra\index{Siddhicandra} was pressurised to give up his celibacy and marry, yet the poet stood resolute in his position, even at the threat of exile; Pollock suggests that this is a pointer to the idea that the prevailing literary and social idea was that “all innovation should be in service of the oldest of Jain monastic ideals,” (Pollock 2001:407). Siddhicandra’s\index{Siddhicandra} work and self identification as a {\sl navya\index{navya@\textsl{navya}} kavi} are thus indications of how the very nature of scholarship in Sanskrit has changed; and Pollock interprets this as a definitive end of the {\sl prācya\index{pracya@\textsl{prācya}}}.

Considering Kavīndra Sarasvatī’s life and work next, Pollock expounds that the massive transformations that swept through the society failed to mark the poet’s work. Highlighting that Kavīndra’s major achievement was the abolition of the jizya tax\index{jizya@\textsl{jizya}} that was imposed on Hindus during the Mughal rule, Pollock states that the work of the poet remained conventional, and bore no marks of innovation or creativity. Again dismissing the work done in the genre of hymns and restating of older problems in newer language, Pollock makes a sweepingly damning statement: “What Sanskrit learning in the seventeenth century prepared one to do, one might infer from the works of Siddhicandra and Kavīndrācārya, was to resist all other learning,” (Pollock 2001:408). Pollock reduces all the transformations taking place in the language - including his own admission in the earlier part of the essay viz. that this period corresponded with some of the most innovative departures that took place in Sanskrit - to the statement that the outcome of the literary era is an inert unresponsiveness and resistance. This is indeed a very dismissive and contemptuous position Pollock elects to take.

Returning to comparisons with Jagannātha, Pollock searches for information about the Sanskrit culture of the age in Jagannātha’s personal life, and concludes that part of this culture is “the historical fact of a literary representation linking the greatest Sanskrit poet of the age with a Muslim woman\index{Muslim!woman},” (Pollock 2001:409). The themes Jagannātha explores, such as the death of his wife and child, and the tone of his personal reference  - are all shown as features of the {\sl navya} tradition. Pollock’s mode of analysis is not only questionable but his conclusions and inferences in each of these instances of study are labored and factitious. Thus, the highlight of the literary analysis of Sanskrit in the seventeenth century that Pollock makes, pivots on the role played by Jagannātha’s marriage to a Muslim woman. It is indeed startling that this is the conclusion the acclaimed Indologist makes.

\subsection{‘Under the Shadow of The Raj’}

Pollock begins the last case by recounting the surveys commissioned during the British Rāj\index{British Raj@British Rāj}, of the state of Sanskrit learning in the Bengal and Madras presidencies. Despite Pollock’s insistence that Sanskrit as a language had lost the vital force to spread and survive, he admits that the surveys provided information to the contrary picture. Discussing the statistics from the ‘Third Report’, the survey commissioned by William Adam\index{Adam, William} in the 1830s in five districts in the Bengal presidency, one finds that there were 353 Sanskrit schools\index{Sanskrit!schools} with over 2500 students in them. He points out that most of the students in the schools belonged to the Brahman community while half the students in the Muslim schools of Burdwan district were non-Muslims, with the Brahmins forming one third of this population. Pollock notes that, “The vast majority of Sanskrit students were engaged in the study of grammar, logic or law. Other subjects, among them literature, figure far less prominently,” (Pollock 2001:412).

Admitting though, that this does not imply the absence of creation of new literature in Sanskrit, Pollock mentions that, in fact, new works were being produced in Bengal even in 1830s. One is sure to find this true across the country, with R C Majumdar\index{Majumdar, R C} writing about the numerous works that were being created throughout the nation (Majumdar 2001b:960-967). But the standards and quality of the creations were, maintains Pollock, poor and constitute “literary atrophy”. While other knowledge systems of Sanskrit, such as logic, continued to thrive with the creation of remarkable material, Pollock also states that the spread of the new creations across the country demonstrates the strength and vitality of the disseminating systems. But, as literary texts\index{texts (Sanskrit/Indic)}  weren’t a highlight of this distribution, and school syllabi carried in it outdated works, Pollock hastily concludes that, “That the literary texts\index{texts (Sanskrit/Indic)} were no longer inserted into this distributive network – and they were not – must be due to the fact that they did not merit insertion in the eyes of the Sanskrit readers themselves,” (Pollock 2001:413).

Socio-political factors such as the termination of the {\sl zamindāri} system and cessation of the patronage\index{patronage, cessation of} extended to pundits are, maintains Pollock, not the main cause of this decline in literary output. Across various genres of Sanskrit, excepting literature, scholarly inventions and accomplishments continued, however, to pour new life into the language, especially in various courts of the Hindu kings. The Maratha court of Tanjore\index{Tanjore}\index{Maratha court of Tanjore}, the courts of Krishnaraja Wodeyar\index{Wodeyar!Krishnaraja} in Mysore\index{Mysore}, and Jai Singh II\index{Jai Singh II} in Jaipur\index{Jaipur} provided patronage\index{patronage} for the development of the language, and this resulted in a capacious and extensive production of analytical texts\index{texts (Sanskrit/Indic)} and newer commentaries on older topics. “But, how did the Sanskrit literary imagination react to all this? It simply did not,” concludes Pollock.  

Numerous newer and interesting works came up during the period : works such as those of Rāghava Āpa Khanḍekar\index{Khandekar, Raghava Apa@Khanḍekar, Rāghava Āpa} in Maharashtra who wrote  a lexicon {\sl Kośāvataṁsa\index{Kosavatamsa@\textsl{Kośāvataṁsa}}}, a book on astronomy {\sl Kheṭakṛti\index{Khetakrti@\textsl{Kheṭakṛti}}} and the literary work {\sl Kṛṣṇavilāsa\index{Krsnavilasa@\textsl{Kṛṣṇavilāsa}}}; other writers such as Achyutarāya Modak\index{Modak, Achyutaraya@Modak, Achyutarāya}, Gaṅgādhara\index{Gangadhara@Gaṅgādhara} of Nagpur; writers from the Kashmir seat of learning such as Śivaśaṅkara\index{Sivasankara (Kashmiri writer)@Śivaśaṅkara (Kashmiri writer)}, Vāsudeva\index{Vasudeva (Kashmiri scholar)@Vāsudeva (Kashmiri scholar)}, Ganeśa\index{Ganesa@Ganeśa} and Lāla Paṇḍita\index{Lala Pandita@Lāla Paṇḍita}; and writers from the southern regions such as Rāmasvāmi Śāstri\index{Ramasvami Sastri@Rāmasvāmi Śāstri} and  Sundara Rāja; all these and more importantly, he ignores the changing trends that found their expression through the publication of short stories and journals through the influence of Western style, and goes on to state, “In terms of both the subjects considered acceptable and the audience it was prepared to address, Sanskrit had chosen to make itself irrelevant to the new world.” One wonders why the critic is so determined to set aside any evidence that is contrary to his position.

Taking the example of Ishwar Chandra Vidyasagar\index{Ishwar Chandra Vidyasagar}, Pollock (2001:414) asserts, “Sanskrit intellectuals seemed able to respond, or were interested in responding, only to a challenge made on their own terrain – that is, in Sanskrit.” Furthermore, “Sanskrit had ceased to function as the vehicle for living thought, thought that supplemented and not simply duplicated reality.” (2001:414). If this is indeed the state of affairs in the country, as suggested by Pollock, one must question how a song like Vande Mātaram\index{Vande Mataram@\textsl{Vande Mātaram}} - written in simple Sanskrit with a sprinkling of Bengali words, as a part of {\sl Ānandmaṭh\index{Anandmath@\textsl{Ānandmaṭh}}} by Bankim Chandra Chaṭṭopādhyay\index{Chatterji, Bankim Chandra}\index{Chattopadhyay, Bankim Chandra@Chaṭṭopādhyay, Bankim Chandra|see{Chatterji, Bankim Chandra}} - spread throughout the length and breadth of the nation, and indeed became a call for galvanizing energy during the Swadeshi movement\index{Swadeshi movement}; it was as well a call for fight for freedom, calling upon the people to take up the fight, the ones who had been oppressed by centuries of economic, political and social bondage. It was among the very first attempts at forging an identity of Motherland; and in the process, the song has given an entire people a timeless conception of one’s nation as a living force, and in a language they have seen as the mother of all their native languages. 

In this context, as Pollock sounds the death-knell for Sanskrit\index{Sanskrit!death-knell for}, during this latest period in his analysis, one feels Hanneder’s statement (Hanneder 2002:308) holds true:

\begin{myquote}
“It would be possible to add instances, where Pollock has interpreted the evidence to fit his thesis without considering other options. But let us briefly mention two examples of, if one wishes, innovations: The first is the development of a particular brand of Campū from tenth century onwards. The other is the recent adaptation in Sanskrit literature of new genres like that of the modern short story. One, in my view, particularly impressive synthesis of classical Sanskrit style and the modern social – critical short story is found in Kṣamā Rao’s\index{Rao, Ksama@Rao, Kṣamā} works. It would not be surprising to find more of this sort in the Sanskrit literature of 19th and 20th century. One should also not forget that the transformation of the Sanskrit Pandits who came in contact with or were under the influence of the British education system in India, are not examples for the power of ‘Sanskritic culture’ to adapt and interact with modernity; this innovation was even a necessary condition for the emergence of Indology\index{Indology!emergence of} itself.”
\end{myquote}

Pollock goes on to suggest “Perhaps those who are not inheritors of a two-thousand year long tradition cannot possibly know its weight – the weight of all the generations of the dead who remain contemporary and exigent” (2001:414). Firstly, the tradition is much older than the two thousand years that Pollock mentions. Secondly, it appears that Pollock is responding to William Adam, who conducted the survey in Bengal presidency in 1830, as a man who loves the language, but having made such a biased\index{bias} case for declaring Sanskrit dead, Pollock now takes the stand of a duplicitously benevolent critic. Criticizing Adams is certainly pointless, for it would amount to breaking a butterfly on a wheel but Pollock makes the truest statement of the essay in responding to the former – 

\begin{myquote}
“The love and care of language (‘complicated alliterations’), the vast and enchanting  Borgesian library of narratives (‘absurd fictions’), the profound reflections on human destiny (‘metaphysical abstractions’) are central values marking Sanskrit\index{texts (Sanskrit/Indic)} literature from its beginning, and a source of incomparable pleasure and sustenance to those with cultural training to appreciate them.”\hfill Pollock (2001:414)
\end{myquote}

Pollock then sets the stage for his conclusion with trying to understand “when and why this repertory became a practice of repetition and not renewal; when and why what had always been another absolutely central value of the tradition – the ability to make literary newness\index{literary!newness}, or as a tenth century writer put it, ‘the capacity continually to reimagine the world’ was lost to Sanskrit forever.” (2001:414).

\section*{Conclusion}

Pollock is only right when he states “It is no straightforward manner to configure these four moments of Sanskrit literary culture into a single, plausible historical narrative.” (2001:414). The reason for this difficulty is that there is no naturally connecting thread\index{connecting thread} that runs between them; and Pollock is trying to laboriously and synthetically create a case for the death of the entire Sanskrit language by highlighting disconnected moments in the genre of literary activity alone in a perpetually changing scenario over a millennium of India.  He asserts his particulars, often “anecdotal factoids\index{anecdotal factoids},” as Rajiv Malhotra\index{Malhotra, Rajiv} rightly calls them, and sometimes outright untrue statements, in such an assertive manner even while consciously ignoring evidence contrary to his claims. Thus, 
\medskip

\begin{myquote}
“in Kashmir after the thirteenth century, Sanskrit literature ceased almost entirely to be produced; in Vijayanagara, not a single Sanskrit literary work entered into transregional circulation, an achievement that signaled excellence in earlier periods; in seventeenth century Delhi, remarkable innovations found no continuation, leaving nineteenth century Sanskrit literary culture utterly unable to perpetuate itself into modernity.”\hfill Pollock (2001:414) 
\end{myquote}
\medskip

Pollock goes on to compare Sanskrit with Greek\index{Greek} and Latin\index{Latin} - two cultures which apparently have shared the same fate as Sanskrit.  Greek\index{Greek} literary activity met an end when the Academy was shut down in 529 CE; and received its final blow when Constantinople\index{Constantinople} fell to the Turks in 1453 CE\@. Comparing Sanskrit with Greek\index{Greek} is not only unfair and unreasonable, one wonders what other similarity the two languages have, apart from their classicism. With Pollock’s own examples, one can witness the resilience and indomitability\index{Sanskrit!resilience and indomitability} of the Sanskrit language - to have withstood shocks, assault, negligence and attempts to actively destroy it over a millennium; and what is more, it continues to survive and rediscover itself even in our own day and age.

In comparison with Latin,\index{Latin!(Vulgar, Medieval, Church)} Pollock (2001:415) considers the later period of the language and says “Both died slowly, and earliest as a vehicle of literary expression.” “Politics of translocal aspiration” have ‘forced’ attempts at renewal, he says, brushing into this ‘forced attempts’ all the remarkable literary work that was carried out in the courts of the Peshwas\index{Peshwas} and the Wodeyars\index{Wodeyars} in the eighteenth and nineteenth centuries. “Both came to be ever more exclusively associated with narrow forms of religion and priestcraft, despite centuries of a secular aesthetic.” (2001:415). While Pollock associates this insular characteristic to the language, one is reminded of recent efforts to study Sanskrit even by scientists at NASA\index{NASA} for its grammatical structure and the manner in which it lends itself naturally to artificial intelligence (see Briggs\index{Briggs, Rick} 1985). It is also important to consider here the words of J Hanneder\index{Hanneder} (2002:309) in his paper responding to Pollock on the matter, 

\begin{myquote}
“Perhaps the parallel with the decline of Latin\index{Latin!(Vulgar, Medieval, Church)} leads him to take the production of religious literature as less indicative of an alive Sanskrit culture, while the religious Stotra\index{stotra literature@\textsl{stotra} literature} is for this reason not a valid genre for him! Whether literary theory\index{literary!theory} is really that important for the creativity of a language remains doubtful. It is a mode of reflection attesting an intellectually sophisticated climate, but it could also be argued that the state of discussion in Alaṃkāraśāstra did not call for another 1000 years of revolutionary theories. And I imagine that not only Sanskrit poets would have protested against the notion that the real indicators of intellectual activity are the professional critic and the professor of literature, rather than the poet.” 
\end{myquote}

Pollock then mentions the differences between the two languages viz.\ Sanskrit and Latin,\index{Latin!(Vulgar, Medieval, Church)} with the first being the role of communicative expertise that has been a part of Sanskritic culture throughout its history until the time of the British, when Macaulay\index{Macaulay, T B} famously introduced his system of education (see Macaulay\index{Macaulay, T B} 1835) with the explicit aim of ending learning and communication in Sanskrit. While both Europe and India have seen vernacularisation\index{vernacularisation} over the last millennium, the two situations differ in their nature. Pollock acknowledges that “the intellectuals who promoted the transformation, certainly in its most consequential phases, were themselves learned in Sanskrit”, and this, as against the situation with Latin in Europe.  His assessment that those who could read vernacular poetry in India were also well versed in Sanskrit is valid and true. 

In his search for the causes of the “death” of Sanskrit, Pollock emphatically sets aside, without any consideration of unbiased evidence, any impact by the Islamic invasion\index{Islamic invasion} on the state of the language. He maintains, “The evidence adduced here shows this to be historically untenable,” (Pollock 2001:416), and has presented carefully selected instances that support his claim, while consciously discounting the works of so many writers who have written contrary to it. He further continues, “It was not ‘alien rule unsympathetic to {\sl kāvya}’ and a ‘desperate struggle with barbarous invaders’\index{invaders} that sapped the strength of Sanskrit literature. In fact, it was often the barbarous invader who sought to revive Sanskrit.” (2001:416). It becomes impossible to attach credibility to Pollock’s conclusions when one reads accounts of the manner in which the culture of the land, with its unparalleled structures of learning including its vast libraries,\index{libraries, destruction of} was massacred under the invasions of Muhammed Ghazni\index{Ghazni, Mahmud (Muhammed)}, Muhammed Ghori\index{Ghuri, Mahmud (Muhammed)}, and their cultural successors until Aurangzeb\index{Aurangzeb} four centuries ago. This is not to forget that there were stray instances of Muslim rulers\index{Muslim!rulers} who have attempted to bring peace and tolerance to the land – Akbar\index{Akbar} and Dara Shikoh\index{Dara Shikoh} readily come to one’s mind – but glossing over a whole major part of one’s cultural history and absolving Islamic rule\index{Islamic rule} of any responsibility of adversely impacting Sanskrit is outright willful misrepresentation\index{misinterpretation,!techniques of,!willful misrepresentation}.

Probing further with his coloured glasses\index{coloured glasses} into the reasons for the “death” of Sanskrit, Pollock enlists three major causes. The first of these is the “internal debilitation of the political\index{power!political} institutions\index{institutions!political} that had previously underwritten Sanskrit, pre-eminently the court.” (2001:416). The second is the “heightened competition among a new range of languages.” (2001:416). While comparing the two cultures of Latin and Sanskrit,\index{Latin!and Sanskrit}\index{Latin!(Vulgar, Medieval, Church)} Pollock was right in maintaining that those who encouraged the spread of the vernacular in India were themselves well versed with Sanskrit. But now, he feels that the increasing usage of the vernaculars throughout the country seemed to have adversely affected Sanskrit. He cites the example of the court of Vijayanagara to support his case once again. One can easily recount the vernacular poets Keśavdās and Bihārīlāl\index{Biharilal@Bihārīlāl} but not any Sanskrit poet, maintains Pollock. By describing Sanskrit as “the idiom of a cosmopolitan literature,” Pollock attempts to present a nonexistent divide\index{nonexistent divide} between the Vernaculars and Sanskrit. India has always been a land of multiplicity; and having both the vernaculars and Sanskrit existing simultaneously, with constant flow of ideas and energies between the two - is an unacceptable scenario to Pollock. 

The third cause for the ‘death’ is the dramatic erosion of “the civic ethos embodied in the court.” Pollock again repeats his assertion that the ethos “had more or less fully succumbed by the thirteenth century, long before consolidation of the Turkish power in the Valley.” (2001:416). One certainly hopes that a lie  repeated a thousand times does not become truth. Pointing out that Sanskrit had no establishments and structures to help it adapt, disseminate and percolate the changes more intensely into the social fabric of the land, Pollock believes that the traditional networks and collective efforts went into fighting limited and temporary goals. Once again, this position of Pollock seems conflicting with the idea expressed by many historians and critics that there was, indeed, a proliferation of publishing material towards the later parts of the nineteenth and twentieth centuries with the introduction of the printing press to India.

Pollock admits, no doubt, that newer works and invaluable critiques and interpretations did indeed develop during the Vijayanagara\index{Vijayanagara} period and later; but then, in his conclusion, he believes that the impact of the Navya phase was insignificant as it was “a newness of style without a newness of substance.” Pollock betrays the fault in his reasoning, with his arbitrary selection\index{arbitrary selection} of examples and illustrations, and further limiting to only the genre of {\sl kāvya}\index{kavya@\textsl{kāvya}} on the one hand to the sweeping, overreaching generalizations\index{overreaching generalizations} he makes for the entire language from these examples, on the other. He limits his survey while making specific points but makes the following statements for the entire field – “No idiom was developed in which to articulate a new relationship to the past, let alone a critique; no new forms of knowledge – no new theory of religious identity, for example, let alone of the political – were produced in which the changed conditions of political and religious life could be conceptualized.” (2001:417).

Pollock ends his essay with a further belabouring of the same statement - “At all events, the fact remains that well before the consolidation of colonialism\index{colonialism}, before even the establishment of the Islamicate political order, the mastery of tradition had become an end it itself for Sanskrit literary culture, and reproduction, rather than revitalization, the overriding concern.” (2001:418).  

The analysis Rajiv Malhotra\index{Malhotra, Rajiv} has performed of this particular theme of Pollock’s work in his book {\sl The Battle for Sanskrit\index{Battle for Sanskrit, The@\textsl{Battle for Sanskrit, The}}} (2016) has been invaluable for its preciseness and clarity. The systematic compilation of all the attempts that have been made over the last millennium to end any attempts to reinvigorate Sanskrit language as well as the focused rebuttals given to Pollock’s essay set a sound and apposite platform for {\sl pūrva-pakṣa\index{purva-paksa@\textsl{pūrva-pakṣa}}}. The significance of Pollock’s work\index{Pollock!work, significance of} lies in the stirring it has provided and the clarion call it has given in urging Indians to once again take up the stewardship of the creation and organisation of their cultural instruments. In heeding to the call and rising to the challenge alone would one find a commensurate response to ‘The Death of Sanskrit’.


\begin{thebibliography}{99}
\bibitem[]{chapter8_item1}
Briggs, Rick (1985) “Knowledge Representation in Sanskrit and Artificial Intelligence”. {\sl AI Magazine}, Vol.~6, No.~1, pp.~32--39

\bibitem[]{chapter8_item2}
Hanneder, J (2002). “On The Death of Sanskrit”. {\sl Indo-Iranian Journal}. 45. pp.~293--310.

\bibitem[]{chapter8_item3}
Kaw, M. K. (2004). {\sl Kashmir Education, Culture, and Science Society - Kashmir and Its People: Studies in the Evolution of Kashmiri Society}. A.P.H. 
Publishing Corporation. Jammu and Kashmir.  

\bibitem[]{chapter8_item4}
Macaulay, T. B. (1835). {\sl Minute on Education}.

\bibitem[]{chapter8_item5}
Majumdar, R.C. (Ed.) (2001a) {\sl The Struggle for Empire}. Vol.~5. 5th Edition. Bharatiya Vidya Bhavan. 

\bibitem[]{chapter8_item6}
Majumdar, R.C (Ed.) (2001b) {\sl The Struggle for Freedom}. Vol.~7, 5th Edition. Bharatiya Vidya Bhavan

\bibitem[]{chapter8_item7}
Malhotra, Rajiv (2016) {\sl The Battle for Sanskrit}. Harper Collins. New Delhi.

\bibitem[]{chapter8_item8}
Pollock, Sheldon (2001). “Death of Sanskrit”. {\sl Comparative Studies in Society and History}. Vol.~43. Issue~2. Pp~392--426. 

\bibitem[]{chapter8_item9}
Sri Aurobindo (1997). {\sl The Renaissance in India and Other Essays on Indian Culture}. The Complete Works of Sri Aurobindo – Volume 20. Sri Aurobindo 
Ashram Publications Department

\end{thebibliography}
