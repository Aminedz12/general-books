\chapter{German Indology, Sanskrit and Nazi ideology}\label{chapter1}
\vskip -10pt

\Authorline{K. Gopinath}
\lhead[\small\thepage\quad K. Gopinath]{}
\vskip -10pt

\section*{Abstract}

We first discuss Pollock’s work on the question of “German Indology, Sanskrit and Nazi ideology” in his paper (1993) ``Deep Orientalism\index{Orientalism (passim)}? Notes on Sanskrit and Power Beyond the Raj". We critically examine his thesis on complicity of Sanskrit, through German Indology, for the Nazi genocide\index{genocide}; other claims in the paper (such as German Indologists attributing degeneracy to India) are not addressed here, or briefly touched upon if at all. Using the propaganda model\index{propaganda model} of the US media proposed by Herman and Chomsky\index{Herman and Chomsky} (1988), we examine whether Pollock’s writings on this subject can best be explained from a propaganda model\index{propaganda model}.  We end the paper with a brief discussion to examine if and how traditional knowledge systems or Indic systems\index{Indic!systems} could have the protection that artistic works in principle enjoy - such as ``moral rights”, a form of IPR\@. This has implications for Indic systems as many of them are in the informal domain and hence are subject to capture, disinformation\index{disinformation} or “digestion”\index{digestion}.

\vskip -10pt


\section*{Introduction}

Sheldon Pollock’s work on the question of “German Indology, Sanskrit and Nazi ideology” is instructive in how certain questions are posed, extraneous elements subtly brought in, and selected evidence then marshalled to deduce or suggest certain desired conclusions. This is in essence how propaganda systems \index{propaganda systems} work and in this paper we attempt to show that it is indeed the case with Pollock's selective use of historical materials. I base my analysis on his 1993\endnote{available at {\scriptsize\url{https://www.academia.edu/2242722/Orientalism_and_the_postcolonial_predicament_perspectives_on_South_Asia}}); note that this pdf file has some infirmities (word spacing problems, etc) but this is not a problem for the arguments in this paper.} paper ``Deep Orientalism? Notes on Sanskrit and Power Beyond the Raj", starting first by summarising Pollock's main argument (in parts labelled {\bf A to E} below).
{\renewcommand\theenumi{\Alph{enumi}}
\renewcommand\labelenumi{{\bf\theenumi.}}
\begin{enumerate}
\item At the outset, Pollock expands the scope of the word “orientalism” very broadly and makes it equivalent to any system of domination or oppression: to wit, 
\begin{myquote}
“$\ldots$orientalism is disclosed as a species of a larger discourse of power that divides the world into ‘betters and lessers’ and thus facilitates the domination (or ‘orientalization\index{orientalization}’ or ‘colonization\index{colonization}’) of any group”. \hfill(Pollock 1993:77)
\end{myquote}

\item But then, a good part of the last 5 centuries were mostly ones of European/American domination or oppression of the rest of the world. Cleverly, Pollock takes examples of inequality in Indian society in the past and now makes it also some type of (expanded notion of) ``orientalism" because of the inequality that he perceives:
\begin{myquote}
“indigenous discourses of power - the various systematized and totalized constructions of inequality in traditional India - might be viewed as a preform of orientalism”. \hfill(Pollock 1993:77).
\end{myquote}

Hence, there is now continuity between Indian ``internal orientalism" and other types of later orientalisms. Furthermore, for Pollock, “Sanskrit knowledge presents itself to us as a major vehicle of the ideological form of social power in traditional India”, with “restriction of access to Sanskrit ‘literacy’ as a principal mode of domination”. Also, “Sanskrit studies, as an indigenous form of knowledge production [is] equally saturated with domination”, opines Pollock. Hence both Sanskrit knowledge as well as Sanskrit studies are implicated as major factors in oppression, which now is recast as internal “orientalism”.

\item He then argues that German Indology has been the most productive “orientalism" in the past and documents that it transmogrified into an area of study that had many Nazi sympathizers. Unscientifically, Pollock does not consider other disciplines and check whether practitioners in those fields had any such sympathy for Nazis. For example, Martin Heidegger\index{Heidegger, Martin} (a celebrated philosopher) and Werner Heisenberg\index{Heisenberg, Werner} (a celebrated physicist), to quote just 2 names, have also been suspected to have Nazi sympathies. Indeed, from the 1920’s, there was a “Deutsch Physik”\index{Deutsch Physik} movement led by two Nobel laureates (Philippe Lenard\index{Lenard, Philippe} and Johannes Stark\index{Stark, Johannes}) that rejected the (Jewish) physics of Einstein\index{Einstein, Albert} (relativity theory, the inadmissibility of ether, etc). During WWII many famous German scientists participated in the German atomic bomb project (e.g., Hans Geiger, Walter Gerlach, Otto Hahn, Werner Heisenberg, Carl Friedrich von Weizsäcker). Another example is the discipline of anatomy; Sabine Hildebrandt (2016) discusses (in the “front matter” of the book) the 
\begin{myquote}
“anatomists’ involvement in racial hygiene\index{racial!hygiene} as the leading science of NS ideology” and that their “complicity with the Nazi state went beyond the merely ideological. They progressed through gradual stages of ethical transgression\index{ethical transgression}, turning increasingly to victims of the regime for body procurement\index{body procurement}, as the traditional model of working with bodies of the deceased gave way, in some cases, to a new paradigm of experimentation with the ‘future dead.’ ”
\end{myquote}

(The involvement of anatomists will be discussed further in connection with the genocide\index{genocide} in Namibia\index{Namibia} and later during WWII.) In disciplines such as anthropology and genetics, racial theories were {\sl de rigueur} not just in Germany but widespread in Europe and America till the end of WWII\@. 

Furthermore, disputing the centrality or the ‘materiality’ of Indology studies in the German academia, using yearbooks of the latter used by Pollock himself, Grünendahl\index{Grunendahl@Grünendahl} (2012:192) reports that he

\begin{myquote}
“discovered that within this period [1933-38] two Indological chairs fell vacant, while the overall position of Indology remained as peripheral as it always had been in German academia, contrary to what Pollock would have us believe.”
\end{myquote}

Without comparative data of Nazi sympathies in other disci\-plines, Pollock’s thesis on the role of German Indology in the holocaust\index{holocaust!Nazi} is a house of cards; so too, his preposterous claim of Sanskrit ’s culpability. 

\item The next step for Pollock is to conflate the native meritocratic system (Aryan\index{Aryan} vs non-Aryan) with racism though the word {\sl ārya}\index{arya@\textsl{ārya}} was never used in the racial sense in India. It is surprising that even after notions of “Aryan race”\index{Aryan!race} have been debunked repeatedly, Pollock resuscitates it; note his almost casual way of posing it below without any supporting evidence with respect to “race”: native forms of oppression were 
\begin{myquote}
“creating the idea of race and concurrently legislating racial exclusivity, asserting linguistic hierarchy and claiming superiority for the language of the masters, and securing an order of domination by monopolizing ‘life chances’ such as forms of literacy”.\\[-15pt] 

~\hfill(Pollock 1993:78)
\end{myquote}

If race is really involved, how does Pollock explain the phenomenon of a {\sl saṁnyāsin's} “non-caste”hood\index{caste!non-caste hood@“non-caste” hood} as they can be initially from any {\sl varṇa}\index{varna@\textsl{varṇa}} (see, for example, how Buddha\index{Gautama Buddha} phrases this in {\sl Aggañña Sutta}); no racial system in the world, that I am aware of, would allow such a change. Similarly, Dharampal\index{Dharampal} (1971:preface) reports, while travelling in the late 50’s, that “there was no {\sl jati} on a yatra” (ie. during long pilgrimages\index{pilgrimage}, many normative practices do not obtain). The Indic tradition also has many examples of {\sl kṣatriya-s}\index{ksatriya@\textsl{kṣatriya}} (such as Viśvāmitra\index{Visvamitra@Viśvāmitra}) becoming {\sl brāhmaṇa-s;}\index{brahmana@\textsl{brāhmaṇa}} the Peshwa Baji Rao\index{Baji Rao, Peshwa} I could be considered as a recent example in the opposite direction. There was upward mobility too; example can be given of Shivaji\index{Shivaji} who categorised himself as a Kṣatriya using the flexibility in the system itself. Many {\sl jāti-s} have created genealogies or declared themselves as {\sl sūrya-vaṁśī}\index{surya-vamsi@sūrya-vaṁśī} (“belonging to the Solar Genealogy”) or {\sl candra-vaṁśī}\index{candra-vamsi@candra-vaṁśī} (“belonging to the Lunar Genealogy”) based on the social reality obtaining. There is enough evidence to argue that the {\sl jāti-varṇa} system is closer to a somewhat fluid sociological system of differentiation/inequality (that is not too different from other systems elsewhere in premodern times) and not based on biological race. 

Interestingly, the native {\sl varṇa-jāti} system was misconstrued into an inflexible ``caste" system\index{caste!system} by the British/Europeans based on their own social regimentation prevailing in Europe/Americas since at least the 15th century with, for example, castas, mestizos, and mulattos in the Americas. Furthermore, the “one-drop” rule\index{one-drop rule@“one-drop” rule} in US statutes and elsewhere all the way into the beginning of the 20th century made even a “drop” of black or Native American blood made them the inferior race; even Thomas Jefferson’s\index{Jefferson, Thomas} children through his mixed race concubine were born into slavery, with some of them having to “escape” to become free! There is still an ongoing project in many parts of the West where the {\sl varṇa\index{varna@\textsl{varṇa}}-jāti} system is to be declared as equivalent to racism with attendant stringent penalties (eg. attempted discussion/legislation in UK/EU parliaments or UN, or in the\break academia where Dalits are equated with blacks through the\break Afro-Dalit project)\index{Afro-Dalit project}.

With the racism angle thus brought in without any justification whatsoever, now one could argue that Nazis were only continuing the race discrimination that Indians were doing all along!  

In the quote cited above, Pollock also mentions linguistic hierarchy\index{linguistic hierarchy}. This is surprising to me at least as many regional languages were in a ‘give and take’ relationship\index{give and take relationship@‘give and take’ relationship} with Sanskrit; for example, almost all the well known poets of Telugu between 10th century and 16th century were also Sanskrit pandits. While prejudiced opinions are to be expected in any subjective enterprise (such as affinity for a language), to elevate them to notions of domination or superiority seem inappropriate. In the last 2 centuries, given the official neglect\index{official neglect} of Sanskrit and motivated writings of European or American Oriental scholars, such harmful ideas have gained prominence, especially in Tamil Nadu (for example, Dravidian linguistic “separatism”\index{linguistic separatism@linguistic “separatism”}). One can also use Pollock’s own argument against his own writing - that his use of “high” and convoluted English\index{convoluted!English} language in his articles is a source of oppression and domination over others.

\item Next, because of the resilience of the Sanskrit knowledge systems for more than 2 millennia, there must be, according to Pollock, something in Sanskrit (``deep structure"\index{deep structure} of order, domination and oppression?) that makes internal orientalism (see Pollock 2006) where he argues that Sanskrit cosmopolis is oppressive) and the German (Nazi) Indology orientalism variety possible. The Germans studied Sanskrit knowledge systems intensely and hence “infected” themselves with that virus of deep structure, and hence order, domination and oppression were the hallmark of Nazis. In a sense, Sanskrit or its knowledge systems molded the Nazis into murderous machines.

Going through these steps {\bf A to E}, it is obvious that there are too many leaps of logic, convenient conflating of categories and too many rhetorical devices\index{rhetorical devices} ({\sl jalpa/vitaṇḍā}\index{jalpa/vitanda@\textsl{jalpa/vitaṇḍā}}) used to, so to say, “pre-form" a conclusion. For example, in the section ``For a Critical Indology" in this paper where there is a serious effort to prove some causal connection with Sanskrit and the Nazi holocaust\index{holocaust!Nazi}, he writes
\begin{myquote}
1: "Reviewing Indology in the way we have just done, we encounter a field of knowledge whose history and object 
both have been permeated with power."\hfill (Pollock 1993:111)
\end{myquote}

Yes, in the way it has been presented by Pollock (ie. in isolation with respect to developments elsewhere), the field and the object matter does seem to revel in themes of domination and oppression. But does Pollock also weigh in other fields of knowledge, their histories and their objects? For a start, religious thought that permitted slaving societies\index{slaving societies} in Arabia or of Spain/Portugal/England/US, or the Christian models\index{Christian models} of understanding that resulted in burning down Alexandria\index{Alexandria} and its libraries, or the Islamic\index{Islamic invasion} ones that caused Nalanda's\index{Nalanda} libraries to be burnt or Bamiyan\index{Bamiyan} to be destroyed\index{libraries, destruction of} recently, or the European/Machiavellian and racial systems of thinking in the 19th century that finally resulted in the British and other European powers\index{power!political} to burn down the summer palace in Beijing\index{Beijing} along with its valuable artefacts in the late 19th century?  Their role in creating the ground for later holocausts is not even hinted at. We may well heed the following ``gentle" observation, even if not exactly apposite given Pollock’s obvious wide erudition:
\begin{myquote}
{\sl ekaṁ śāstram adhīyāno na gacchecchāśtra-nirṇayam}
\end{myquote}
If you know/study only one subject, you cannot go around saying anything authoritatively even in that subject!

Furthermore, continuing to quote from the same paper,
\begin{myquote}
2: ``From its colonial origins in Justice Sir William to its consummation in SS Obersturmführer Wüst\index{Wust, SS Obersturmfuhrer@Wüst, SS Obersturmführer} [``Nazi" Indologist], Sanskrit and Indian studies have contributed directly to consolidating and sustaining programs of domination."	\hfill(Pollock 1993:111)
\end{myquote}

Why are the studies consummated with Wüst, hardly known to most?  Why not Ingalls\index{Ingalls} or others? Bringing in Wüst is underhanded; it emphasizes the Nazi connection for no reason. The mention of William Jones\index{Jones, William} makes it clear that German Indology is not entirely the subject of discussion here.

It is also not clear why these studies have contributed to domination; I believe Pollock has vastly overstated this aspect and will be discussed further down.

Also, 
\begin{myquote}
3:~``In this (noteworthy orthogenesis\index{orthogenesis}) these studies have recapitulated the character of their subject, that indigenous discourse of power for which Sanskrit\index{power!Sanskrit as a source of} has been one major vehicle and which has shown a notable longevity and resilience." \hfill(Pollock 1993:111)
\end{myquote}
\smallskip

This quote may not be all that clear but what the sentence is likely to be saying, paraphrased, is that the subject of study (Sanskrit or Indology broadly) made its students (German Indologists) into supporters also of oppression/domination (effectively, Nazis or provide legitimacy for Nazis). The latter property is almost axiomatic of Sanskrit given the internal structure of Sanskrit itself (due to ``orthogenesis\index{orthogenesis}", the property of an inflexible internal structure that manifests itself in any environment). It is not clear why Sanskrit alone of the classical languages is a major vehicle of discourse of power\index{power!Sanskrit as a source of}. One could argue that Latin\index{Latin!(Vulgar, Medieval, Church)}, still used in Vatican liturgy\index{Vatican!liturgy}, could fit the bill. The Roman empire\index{Roman empire} after all was a large empire for 6 to 7 centuries with a large war machine; it would not give any quarter to an enemy that would not fall in, a good example being how Carthage\index{Carthage} was annihilated to the last “man”. Given that Vatican is widely believed to have cooperated (see below) with Fascism\index{fascism} and Nazism, Latin\index{Latin!(Vulgar, Medieval, Church)} is a far stronger contender, using a similar logic of argumentation, for inducing genocidal\index{genocide} tendencies in Nazis.

Next: 
\begin{myquote}
4: ``From such factors as the semantic realm of the distinction arya/anarya\index{anarya@\textsl{anārya}} and the biogenetic map of inequality\index{biogenetic map of inequality} (along with less theorized material, from Vedic and epic literature, for instance), it may seem warranted to speak about a ``pre-form of racism" in early India (Geissen 1988: 48ff.), especially in a discussion of indigenous ``orientalism," since in both its classic colonial and its National Socialist [``Nazi"] form orientalism is inseparable from racism."  \hfill(Pollock 1993:107)
\end{myquote}
\medskip

Pollock cleverly introduces “racism" (biogenetic map of inequality\index{biogenetic map of inequality}?) here to set the stage for further mental manipulations by equating indigenous “orientalism” with racism; now one can\break talk about internal orientalism as racism. He then tars most of the early literature (Vedic and epic) as racist without having to give any proof (because “less theorized”). The most obvious racism that has been practised is that between black/white, European/ Native American, Arab/Nubian, European/Asian, European/ Yellow/Chinese/Japanese, etc. But the concept of race is at best extraneous in the larger Indian context as discussed before; what obtained here as many have argued is that of a reasonably fluid {\sl varṇa\index{varna@\textsl{varṇa}}-jāti} continuum\index{varna-jati continuum@\textsl{varṇa-jāti} continuum} (Dirks 2001). Note that biogenesis as understood conventionally is that life forms are  produced by life forms (contrast with abiogenesis); it is not clear what Pollock exactly has in mind here; I am assuming this is Pollock’s convoluted way of talking about racism, just as with many other examples of such ambiguous uses in the paper.

Note that Pollock also weakens his own case:
\begin{myquote}
5: “From among the complexities of NS analysis of the {\sl Urheimat}\index{Urheimat@\textsl{Urheimat}} question it is worth calling attention to the way the nineteenth-century view expressed by Schlegel\index{Schlegel, Friedrich von} was reversed: the original Indo-Europeans were now variously relocated in regions of the Greater German Reich; German thereby became the language of the core (Binnensprache), whereas Sanskrit was transformed into one of its peripheral, ``colonial" forms.” \hfill 	(Pollock 1993:92)
\end{myquote}
\medskip

If German Indology makes Sanskrit into a peripheral form in the question of {\sl Urheimat\index{Urheimat@\textsl{Urheimat}}} (the original home of the Aryans), an issue that “had always been a scholarly question prompted and driven by the ideological demands of the European polities in which this discourse originated”, then it is strange that a peripheral form of “domination and oppression” (the Aryan\index{Aryan}/Sanskrit combine) would now be so central when it comes to holocaust\index{holocaust!Nazi} as Pollock has implied. In this connection, it is not clear why Sanskrit should be discussed at such lengths as Pollock has attempted.

What is striking is the almost complete erasure of the ``elephant" in the room: British and US role in not seriously attempting to stop the Nazis till 1939 (e.g., Chamberlain’s Munich treaty\index{Munich treaty} in 1938; this is also a credible allegation by the Communists of that era) and active collaboration by Vatican\index{Vatican} with Italian and German fascism\index{fascism}. Instead, he spends tens of pages looking at obscure sources from German Indologists who are hoping to use Indian (Sanskrit) materials to show “Indo-Germanic” superiority over other Europeans or provide some cover for Nazism. Or, find one Indian author with certain views (e.g., Bhaṭṭa Lakṣmīdhara\index{Bhatta Laksmidhara@Bhaṭṭa Lakṣmidhara}) out of the many and make him the spokesperson for all Indians and even then probably not fairly. For example: “It is in this context that Lakṣmīdhara adduces a foundational principle of the traditional Indian discourse of power\index{power!Sanskrit as a source of}: ‘Whatever act the {\sl arya-s} who know the Vedas\index{Veda} claim to be {\sl dharma}\index{dharma@\textsl{dharma}}, is {\sl dharma}\index{dharma@\textsl{dharma}}; whatever they reject is said to be {\sl adharma}.’” (Pollock 1993: 107). First, if Arya is taken to be the traditional meaning of “a noble person” instead of a member of a race that Europeans/Americans made it out to be, it is not clear what the difficulty is. Note also that the plural of Arya is used; hence it could be a “committee” of noble persons (a collegium of eminent judges?) and thus quite reasonable (just as law even now is what judges interpret it to be). Additionally, other formulations of {\sl dharma}\index{dharma@\textsl{dharma}} are widely available and more accessible; e.g., {\sl Taittirīya Upaniṣad\index{Taittiriya Upanisad@\textsl{Taittirīya Upaniṣad}}} (Śikṣā Valli, Anuvāka 11) instructs {\sl snātaka-s} (“graduates”) as follows:
\begin{quote}
{\sl yāny anavadyāni karmāṇi | tāni sevitavyāni | no itarāṇi |}\\
{\rm “Those actions that are blameless, follow only those, not any other”.}
\end{quote}
Compare with the Popperian conception of Science: {\sl “Consider hypotheses as provisionally true that cannot be refuted or falsified”}
\begin{quote}
{\sl yāny asmākaṁ sucaritāni tāni tvayopāsyāni | no itarāṇi |}\\
“What acts of ours are good, follow those, not any other”.\\[2pt]
{\sl ye ke cāsmacchreyāṁso brāhmaṇāḥ | teṣāṁ tvayā’’sanena praśvasitavyam |}\\
“If there are whosoever brāhmaṇa-s are better than ourselves, comfort them by offering a seat `don't so much as breathe aloud”!
\end{quote}
Note here the open-ended perspective (for example, the inviting of 'better' teachers) rather than the closed perspective (or authoritarian model) that Pollock wants to imply.
\end{enumerate}}

\section*{Critical Study of the Genesis of Nazi Genocide}

If a serious academic were to critically understand the ghastly Nazi phenomenon, many alternative and pertinent explanations would suggest themselves and other than what Pollock wants to suggest such as casually implicating Sanskrit's ``deep structure"\index{deep structure@“deep structure”} for a role in genocide\index{genocide}. Two major alternative explanations are as follows:

\subsection*{1. Church and persecution of Jews}
First, consider Raul Hilberg's\index{Hilberg, Raul} {\sl magnum opus} (Hilberg\index{Hilberg} 1961) on the destruction of European Jews where he shows with great force that the many canonical laws\index{canonical laws} of the church (he lists at least 22) and what the Nazis promulgated were same or very similar (see also Nicholls\index{Nicholls, William} 1993:204-206). To list only three examples (out of 22), consider:

a: “Jews and Christians\index{Christians} not permitted to eat together,”

{\sl Synod of Elvira}\index{Synod of Elvira@\textsl{Synod of Elvira}}, 306 CE

Cf. “Jews barred from dining cars,” Transport Minister to Interior Minister, 

Dec 30, 1939, {\sl Document NG\index{Document NG}}-3995. 
\medskip

Or, 

b: “Prohibition of intermarriage and of sexual intercourse between Christians and Jews,” Synod of Elvira, 306 CE 

Cf. “Law for the Protection of German Blood and Honor,” 

Sep 15, 1935 (RGB1\index{RGB1} I, 1146).
\medskip

Or,

c: “Jews not permitted to show themselves in the streets during Passion Week\index{Passion Week},” 

3rd  Synod of Orleans, 538 CE 

Cf. “Decree authorizing local authorities to bar Jews from the streets on certain days

(i.e., Nazi holidays\index{Nazi!holidays})” 

Dec 3, 1933 (RGB1 I, 1676).

Given the extensive and detailed prohibitions on the Jews by the Church, one can easily surmise that deep seated prejudices became established in the Church as well as internalized by the church goers during the approximately two millennia, as admitted by no less than a recent Pope (Pope Benedict\index{Pope Benedict} XVI, once a ``Hitler Youth''\index{Hitler!Youth} himself\endnote{See “Pope Benedict XVI reflects on life under Hitler's\index{Hitler!Adolf} Nazi Party”. }) “... it cannot be denied that a certain insufficient resistance to this atrocity on the part of Christians\index{Christians} can be explained by an inherited anti-Judaism\index{anti-Judaism} present in the hearts of not a few Christians.”\endnote{See “ADL Welcomes Election of Cardinal Ratzinger as New Pope”.} For example, Matthew 23:31-33, says of Pharisees\index{Pharisees} (“lower class” Jews), “$\ldots$ You snakes, you brood of vipers! How can you escape being sentenced to hell?”

However, for Pollock, 
\begin{myquote}
“The myth of Aryan\index{Aryan!origins} burst from the world of dream into that of reality when the process of what I suggest we think of as an internal colonization\index{colonization} of Europe began to be, so to speak, shastrically codified, within two months of the National Socialists' capturing power\index{power!political} (April 1933)”. 	\hfill (Pollock 1993: 86)
\end{myquote}

Where did the ``shastrically"\index{shastrically} notion come from? Why not say it comes from the Synods (ie. synodically\index{synodically}) as shown by Hilberg which is more direct historically and attested and not just some casually and casuistically thrown word? The mischief can be understood if one is discussing untouchability with its restrictions on sharing food in India: suppose I use the word ``synodically" in such discussions as Jews and Christians\index{Christians} were also not to share food as per the Synod of Elvira CE 306? My submission is that Pollock's use of ``shastrically" is in the same category.

Similarly, 
\begin{myquote}
``The whole weight of these early laws rested on the concept ``Aryan"\index{Aryan} (or rather, somehow significantly, at first on its negation).”\\[-15pt]

~\hfill(Pollock 1993: 87)
\end{myquote}

But, what about Hilberg's detailed analysis? Is not non-Aryan here just another name for Judaic\index{Judaic} only as it is not stressing any aspect of Aryan\index{Aryan} culture but talking about Jews in a coded way only.

There is extensive literature on the subject of Catholic (Vatican)\index{Catholic church}, Protestant\index{Protestant(ism)} and Lutheran complicity\index{Lutheran complicity} and a good summary can be found in an “occasional” paper by Ericksen\index{Ericksen, Robert P}\endnote{See also “The German Churches and The Nazi State”.}. To quote just one relevant, though longish, passage:

\begin{myquote}
“Gerhard Kittel\index{Kittel, Gerhard} [a well known German Protestant\index{Protestant(ism)} theologian during the Nazi years] assessed Adolf Hitler\index{Hitler!Adolf} quite accurately and liked what he saw. That would explain why Kittel’s antisemitism\index{anti-Semitism} grew more intemperate between 1933 and 1944 and why he never apologized after 1945, but energetically defended his own harsh attacks upon Jews. He claimed that his antisemitism\index{anti-Semitism} had been entirely consistent with his Christian faith and no harsher than the antisemitism\index{anti-Semitism} of Jesus or Paul[47]. His only concession was to acknowledge the obvious, that the death camps could not be defended. I think this understanding best explains Kittel but also helps us understand those many other church leaders, pastors, theologians, and lay people who applauded Hitler, who called 1933 a year of rebirth\index{year of rebirth}, and, in the words of Paul Althaus\index{Althaus, Paul}, considered Hitler “a gift and miracle from God”\index{Hitler!gift and miracle from God@Hitler - “gift and miracle from God”}[48]. The problem was not that they misunderstood Hitler, but that they so readily reconciled their consciences and their Christian identities to the harshness of the Nazi state.”

“Why? I have not tried to address that question here, but the short answer is this: They were so hurt by World War I and the national humiliation of the Versailles Treaty\index{Versailles Treaty}, they were so opposed to the open society\index{open society} created by democracy\index{democracy} and the Weimar Republic\index{Weimar Republic}, they were so frightened by the economic crises of hyper inflation\index{hyper inflation} and then the Great Depression\index{Great Depression}, and they were so threatened by the sociological changes of the modern world that someone as ideologically aggressive as Adolf Hitler\index{Hitler!Adolf} seemed an answer to all their problems. He was the candidate of military strength and national pride; the candidate of family values, promising, among other things, to put women back in the home where they belonged; and the candidate whose antisemitism\index{anti-Semitism} fit their own preconceptions and concern that Jews did not really belong in an ideal, unified Christian society [49]. Based upon their hopes and dreams, Christians\index{Christians} and other Germans found it easier to march behind Adolf Hitler\index{Hitler!Adolf} than we would like to think. An honest assessment of the historical record seems to make that clear. It cannot be the legitimate task of historians to bury, ignore, try to hide or try to ignore that complex reality.”
\end{myquote}
\newpage

Tragically, Pollock precisely does that (even worse, shifts the blame) and trivialises the difficult experience of the Germans just after WWI as a mindless adoption of (concocted) racial ideas from a far away land (India) and Indic thought system that had no almost relevance in their lives except in the rarefied and far removed academic world.

Another source of Nazi\index{holocaust!Nazi} thinking was the worldview of “German Christians\index{German Christians}”. Quoting from the same link for the Holocaust website (see United States Holocaust Memorial Museum\index{Memorial Museum}), “Historically the German Evangelical Church\index{German Evangelical Church} (comprised of 28 regional churches or Landeskirchen\index{Landeskirchen} that included the three major theological traditions that had emerged from the Reformation: Lutheran, Reformed, and United) viewed itself as one of the pillars of German culture and society, with a theologically grounded tradition of loyalty to the state. During the 1920s, a movement emerged within the German Evangelical Church called the Deutsche Christen\index{Deutsche Christen}, or “German Christians\index{German Christians}.” The “German Christians” embraced many of the nationalistic and racial aspects of Nazi\index{holocaust!Nazi} ideology. Once the Nazis came to power\index{power!political}, this group sought the creation of a national “Reich Church\index{Reich Church}” and supported a “nazified” version of Christianity\index{Christianity, nazified version of}.” Martin Luther in the 16th century and other German historical figures were not free from anti-Jewish racism\index{anti-Jewish} and it is not surprising that German Christians\index{Christians} were, if not actively, at least passively involved in the pogroms\index{pogroms} in Germany.

Yet another source of Nazi\index{holocaust!Nazi} thinking was the {\sl völkisch}\index{volkisch@\textsl{völkisch}} movement in Germany that had as central themes “sentimental patriotic interest in German folklore, local history and a ‘back-to-the-land’ anti-urban and anti-modernity populism. The dream was for a self-sufficient life lived with a mystical relation to the land; it was a reaction to the cultural alienation of the Industrial revolution and the ‘progressive’ liberalism\index{progressive liberalism@‘progressive’ liberalism} of the later 19th century and its urbane materialist banality.” (See “Völkisch movement” entry in Wikipedia). Unfortunately, this descended into anti-Jewish\index{anti-Jewish} racism in many instances as Jews were seen as outsiders.

\subsection*{2. European Role in Genocides Just before WWI}

Second, another likely explanation comes from the extensive experience of conducting regular genocides\index{genocide} by Europeans on the hapless natives of many countries in the 18th to early 20th century itself. Benjamin Madley\index{Madley, Benjamin} writes about how expertise in genocide\index{genocide} travelled from German campaigns in South West Africa to Auschwitz and suggests how this experience incubated ideas and methods adopted and developed by the Nazis in Eastern Europe (Madley 2005:429-464):

\begin{myquote}
``The German terms Lebensraum\index{Lebensraum} and Konzentrationslager\index{Konzentrationslager}, both widely known because of their use by the Nazis, were not coined by the Hitler\index{Hitler!Adolf} regime. These terms were minted many years earlier in reference to German South West Africa\index{German South West Africa}, now Namibia\index{Namibia}, during the first decade of the twentieth century, when Germans colonized the land and committed genocide against the local Herero\index{Herero} and Nama peoples\index{Nama (people)}. Later use of these borrowed words suggests an important question: did Wilhelmine [German Emperor Kaiser Wilhelm II\index{Kaiser Wilhelm II}, 1888-1918] colonization\index{colonization} and genocide in Namibia\index{Namibia}  influence Nazi plans to conquer and settle Eastern Europe, enslave and murder millions of Slavs\index{Slavs} and exterminate Gypsies\index{Gypsies} and Jews\index{Jews}? This article argues that the German experience in Namibia\index{Namibia} was a crucial precursor to Nazi colonialism\index{colonialism!Nazi/German} and genocide and that personal connections, literature, and public debates served as conduits for communicating colonialist and genocidal ideas and methods from the colony to Germany."
\end{myquote}

Furthermore, 
\begin{myquote}
“$\ldots$The genocide was characterized by widespread death from starvation and dehydration\index{death from starvation and dehydration} due to the prevention of the retreating Herero\index{Herero} from leaving the Namib Desert by German forces. Some sources also state that the German colonial army\index{colonial army} systematically poisoned desert water wells\index{poisoning wells} [12][13]. In 1985, the United Nations' Whitaker Report\index{Whitaker Report} classified the aftermath as an attempt to exterminate\index{extermination} the Herero\index{Herero} and Nama peoples\index{Nama (people)} of South-West Africa, and therefore one of the earliest attempts at genocide in the 20th century...” \hfill(See “Herero and Namaqua genocide” entry in Wikipedia)
\end{myquote}

Note that the genocidal campaigns\index{genocidal campaigns} against the Jews were started barely 2 decades later. At this stage, it is interesting to briefly list some additional details that are relevant to understand how the German colonialism\index{German colonialism} in Namibia\index{Namibia} directly influenced Nazi thinking (most of this information is summarized from the document collection in Bartrop and Jacobs\index{Bartrop and Jacobs} (2015) vol2, p. 997-1110; we do not give detailed references here for brevity). In 1840’s, the Rhenish Missionary Society was established in Namibia\index{Namibia} (the colonial German Southwest Africa). In 1884, Bismarck’s Berlin conference allotted spheres of influence in Africa amongst the colonial powers\index{power!political}; Germany now had the 3rd largest area after Britain/France. During 1885-1904, there was intense competition between Hereros, Namas in Namibia and German outsiders for land/cattle; there were minor battles with multiple realignments amongst the contenders. Just as in US with respect to Native Americans\index{Native Americans}, Germans wilfully broke signed treaties, stole land, and abused natives economically, racially, sexually etc. During the early part of this phase (1885-90), Heinrich Goering\index{Goering, Heinrich} (father of the Nazi leader) was the Imperial Commissioner of South W Africa. In 1904, there was a major Herero\index{Herero} rebellion\index{Herero rebellion} aided by Nama (the latter had realized by this time that Germans were a bigger danger). Gen van Trotha\index{van Trotha, Gen.}, the Commander in Chief of German colonial forces (who incidentally was earlier part of 8 nation alliance that suppressed the Boxer rebellion\index{Boxer rebellion} in China), instituted a systematic policy of extermination\index{extermination, policy of} (murder, dehydration, poisoning wells\index{poisoning wells}); there were also medical experiments on the victims. The Germans also borrowed the concept of concentration camps from the British (from Lord Kitchener's\index{Lord Kitchener} camps for Boers and Africans).

The German colonization\index{colonization} of Namibia had the unfortunate effect of making racism and eugenics\index{eugenics} fashionable at the academic, social and political levels in Germany; US was already advanced in this area given that they had to either dispossess the Native Americans, subjugate blacks or provide legitimation for such acts, with as many as 23 US states (such as Indiana\index{Indiana}, California\index{California}, Washington\index{Washington}) having the earliest compulsory sterilization\index{sterilization} programs\index{sterilization programs} in the world legally, inspired by the “scientific” discipline of “eugenics”, from 1907 onwards (See Reilly\index{Reilly, P R} 1987:153-170)\endnote{See also Sofair and Kaldjian LC (2000) and also Roelcke (2002)}. The direct academic roots of Nazi racism can be said to be through the tutelage of the German professor of medicine, anthropology and eugenics Eugen Fischer\index{Fischer, Eugen} who organized experiments on African victims of Heroro revolt (including on body parts, with organized export of skulls, etc). Due to the widespread sexual exploitation of the Namibians by the Germans, there was widespread alarm in Germany about mixed blood; his eugenics program therefore advocated “improvement of society by preventing birth of inferior races and encouraging birth of superior races”. He authored a book on mixed children of European men and Hottentot\index{Hottentot} (a derogatory term for Namibians) women in German southwest Africa. His recommendations that mixed descendants be not allowed to reproduce were followed and by 1912 interracial marriage\index{interracial marriage} was prohibited throughout the German colonies (note the later Nazi regulations on Jews). Note that California was already much ahead in “thinking” on eugenics by the turn of the 20th century; by 1933, California state\index{California state, laws of} had subjected more people to forceful sterilization\index{forceful sterilization} than all other U.S\@. states combined (the latter number a total of 65,000) (Kevles 1994:18). The forced sterilization\index{sterilization} program later engineered by the Nazis was partly inspired by California’s laws and initially had even support from Rockefeller Foundation!\index{Rockefeller Foundation}

Eugen Fischer also wrote an influential book on “principles of heredity” in 1921, the German title being closer to “Foundations of Human Hereditary Teaching and Racial Hygiene”, along with two others (Baur and Lenz).  Fischer’s part in the book concentrated on the different racial groups on earth while Lenz’s part on the topic of racial hygiene. The second edition of this book was read by Hitler when he was in prison (1923) and the ideas in this book seem to have formed the core of racial ideas in Mein Kampf\index{Mein Kampf@\textsl{Mein Kampf}} (published Oct 1924); a search of this book shows complete absence of any Indic thinking other than the use of words Aryan\index{Aryan} or Swastika but with much  discussion of  racial theories common in Europe in connection with Jews, blacks and Hottentots\index{Hottentot} (Namibians), including also the widely prevalent European notion that British rule is “beneficial” for India. In 1927, Fischer became the Director of Kaiser Wilhelm Institute of Anthropology\index{Kaiser Wilhelm Institute of Anthropology} and in 1933 started teaching racial hygiene\index{racial!hygiene} to SS doctors responsible to guard “purity of race\index{race, purity of}”, the same year Adolf Hitler\index{Hitler!Adolf} appointed him rector of the Frederick William University\index{Frederick William University} of Berlin (now Humboldt University\index{Humboldt University}). In 1935, he conducted experiments on racially mixed children in Rhine (born of German women and Asian/African soldiers during the Rhineland occupation\index{Rhineland occupation}) and enforced their sterilization\index{sterilization}. In 1943, his protégé’s student Mengele\index{Mengele} conducted infamous experiments on Jewish/Roma twins/dwarves\index{Jewish/Roma twins/dwarves}. Surprisingly, in 1952, he was appointed Hon. President of Anthropology Society!\index{Anthropology Society} (now “denazified\index{denazified}”).

In 1985, an UN report declared Herero\index{Herero} genocide\index{genocide} the first of the 20th century. With worldwide protests against apartheid\index{apartheid} regimes in South Africa and Namibia gathering steam from late 70’s, Namibia became independent in 1990; sensitive scholars in US may have come to know by now what happened in Namibia as there were also US student protests in ‘80s including at Columbia University (where some buildings were blockaded by students in 1985). In 2004, Germany’s ambassador to Namibia apologized for the gruesome genocide\index{genocide} but rejected any possibility of reparations; some medical exhibits (skulls, etc) of the victims were returned though.

One can forcefully argue that a clear racist thought system that was responsible for genocide\index{genocide} and enacted ruthlessly in Namibia a better candidate explanation for the later Nazi genocides of  Jews, Gypsies and other “undesirables” than far-fetched ideas that Pollock advances, though it might have had a minor role in the ecosystem of racial ideas prevalent in Germany, Europe or US.

Note that genocides such as in Namibia were not rare either; Madley\index{Madley, Benjamin} discusses 2 other not well known cases (Madley\index{Madley, Benjamin} 2004:167-192) in a summary article. Genocidal campaigns\index{Genocidal campaigns} were conducted in Canada, Australia resulting in the native populations totally wiped out (e.g., Tasmanians\index{Tasmanians}. See “Colonial Genocides\index{Colonial Genocides}” entry in Wikipedia) or reduced to very small populations that were forced out into inhospitable areas.  Extremely brutal regimes were common with colonialists from Netherlands and Belgium.

One instructive case is that of the Native American extermination\index{extermination} in US\@.  In {\sl American Holocaust}\index{American Holocaust@\textsl{American Holocaust}}, David Stannard\index{Stannard, David E} shows connection between American (“Columbus”) holocaust, Nazi holocaust\index{holocaust!Nazi} and also the Church, with the dead estimated in the 100s of millions (Stannard\index{Stannard} 1993). He writes that “wherever Europeans or white Americans went, the native people were caught between imported plagues and barbarous atrocities, typically resulting in the annihilation\index{annihilation} of 95 percent of their populations.” What kind of people, Stannard asks, do such horrendous things to others? His highly “provocative” answer: Christians\index{Christians}. “Digging deeply into ancient European and Christian attitudes toward sex, race, and war, the cultural ground was well prepared by the end of the Middle Ages for the centuries-long genocide\index{genocide!colonial} campaign that Europeans and their descendants launched--and in places continue to wage--against the New World's\index{New World} original inhabitants” (of interest to us: also against Indic systems\index{Indic!systems}). Stannard contends that the perpetrators of the American Holocaust\index{holocaust!American} drew on the same ideological wellsprings as did the later architects of the Nazi Holocaust.

More recently, Madley\index{Madley, Benjamin} (2016) in his book An American Genocide\index{genocide!American} writes “Under US rule, California\index{California state, laws of} Indians died at an even more astonishing rate. Between 1846 and 1870, California's Native American population plunged from perhaps 150,000 to 30,000. By 1880, census takers recorded just 16,277 California Indians. Diseases, dislocation, and starvation were important causes of these many deaths. However, abduction, {\sl de jure} and {\sl de facto} unfree labor, mass death\index{mass death} in forced confinement\index{confinement, forced} on reservations, homicides\index{homicides}, battles, and massacres also took thousands of lives and hindered reproduction. $\ldots$ The organized destruction of California's Indian peoples under US rule was not a closely guarded secret. Mid-nineteenth-century California newspapers frequently addressed, and often encouraged, what we would now call genocide\index{genocide}, as did some state and federal employees. Historians began using these and other sources to address the topic as early as 1890. That year, historian Hubert Howe Bancroft\index{Bancroft, Hubert Howe} summed up the California Indian catastrophe under US rule: ``The savages were in the way; the miners and settlers were arrogant and impatient; there were no missionaries\index{missionaries} or others present with even the poor pretense of soul-saving or civilizing. It was one of the last humanhunts\index{humanhunts} of civilization, and the basest and most brutal of them all." In 1935, US Indian Affairs commissioner John Collier\index{Collier, John} added, ``The world's annals contain few comparable instances of swift depopulation - practically, of racial massacre - at the hands of a conquering race." In 1940, historian John Walton Caughey\index{Caughey, John Walton} titled a chapter of his California history "Liquidating the Indians\index{liquidating the Indians}: ``Wars and Massacres." Three years later, Cook wrote the first major study on the topic. He quantified the violent killing of 4,556 California Indians between 1847 and 1865, concluding that, ``since the quickest and easiest way to get rid of the Northern California\index{California state, laws of} Indian was to kill him off, this procedure was adopted as standard for some years."”

Surprisingly, no obvious entity was held responsible in standard accounts till about 4-5 decades back; least of all the Church or Western powers. Columbus\index{Columbus} was widely praised as discoverer of the Americas, with resistance to observing Columbus Day\index{Columbus Day} formally starting in US only in 1992 in Davis, CA\@. Also Rajiv Malhotra\index{Malhotra, Rajiv} (2009) writes: 
\begin{myquote}
“The greatest episodes of ethnic cleansing\index{ethnic cleansing} and genocide\index{genocide} of Native Americans\index{Native Americans} occurred in the period following independence that was dominated by Thomas Jefferson\index{Jefferson, Thomas} and Andrew Jackson\index{Jackson, Andrew} $\ldots$ Because of the popular demonology\index{demonology} of Native Americans\index{Native Americans} and pseudo-scientific research\index{pseudo-scientific research} to show their innate inferiority, ironically enough, the only defenders remaining were missionaries claiming that although Native Americans were presently savages they could be rescued by converting them to Christianity. Further physical genocide\index{genocide!physical} could be prevented by
completing the cultural genocide\index{genocide!cultural}. Sadly, freedom loving Americans explained away their genocide\index{genocide} of Native Americans as the natives’ inability to adapt to civilization: 

“As American hopes of creating a policy based on Enlightenment ideas of human equality failed, and as it relentlessly drove the Indians from all areas desired by the whites, Americans transferred their own failure to the Indians and condemned the Indians racially” (Horsman\index{Horsman, Reginald} 1981, p. 207).”   	\hfill(Malhotra 2009:184,195)
\end{myquote}

The role of the British in genocidal campaigns\index{genocidal campaigns} could be of considerable interest, given that as an “Eastern” speaker of the English language, the principal language of hegemony in India currently, we can be said to feel it acutely that we need to speak an oppressors’s language\index{oppressors’s language} for economic survival, in contrast to what Pollock says, “Perhaps the western Sanskritist feels this most acutely, given that Sanskrit was the principal discursive instrument of domination in premodern India, $\ldots$”. Note that this sentiment of Pollock seems insubstantial or frivolous as he never experienced any oppression as such and his study or interest in Sanskrit can only be said to be voluntary; this may not be true for many “Eastern” English speakers or their immediate ancestors given the very recent history as we briefly discuss below.

The British engineered famines\index{famines} systematically, by levying and mindlessly enforcing unsustainable rates of taxes on land or exporting foodstuffs out even when famines were raging, in India and Ireland (Davis\index{Davis} 2001)\endnote{See Davis(2001). Also see Dutt (1900), Paul (1900), Ghosh (1944). For Ireland, see Woodham-Smith (1962).}, and fought (along with other European and American powers) with China no less than 3 ``opium wars"\index{opium wars} to be able to drug the hapless Chinese. Mike Davis\index{Davis, Mike} (ibid.) concludes that between 12 and 29 million Indians died in the 1876-78 famine\index{famine} alone and this happened when there was a surplus of grains in 1876; he holds British state policy responsible for the large scale murder. Lytton Strachey\index{Strachey, Lytton} (the godfather of the eponymic Bloomsbury bohemian intellectual) was the viceroy of British India\index{British India} at that time and he was busy organizing, during this horrific famine\index{famine}, the most colossal and expensive meal in world history for 68,000 officials, satraps and maharajas to celebrate “Kaiser-i-Hind”\index{Kaiser-i-Hind} Victoria (ibid, chapter 1). Furthermore, in 1879, Viceroy Lytton “actually overruled his entire council to accommodate Lancashire’s lobby (the Association) by removing all import duties on British-made cotton, despite India’s desperate need for more revenue in a year of widespread famine\index{famine} and tragic loss of life throughout Maharashtra” (Wolpert, 1989:248).” Due to lack of space we cannot go into further details but we will summarize some of the other insights: China and India, previous to European penetration/colonization,\index{colonization} had historically managed food shortages much better; for example, “there was no mass mortality from either starvation or disease from the 1743-44 climatic catastrophe” (--- “the spring monsoon failed two years in a row, devastating winter wheat in Zhili (Hubei) and northern Shandong; scorching winds withered crops and farmers dropped dead in their fields from sunstroke”); relief was sufficiently well organized that many deaths were avoided (ibid). Similarly, Bajaj and Srinivas\index{Bajaj and Srinivas} (1996) document in their book {\sl Annaṁ Bahukurvīta} how the dharmic ecosystem over the millennia  functioned during emergencies for relief, valorising the act of giving food as a solemn vow ({\sl tad vratam}) rather than expecting free market to work well during shortages. Shockingly, in contrast, Lytton's Anti-Charitable Contributions Act 1877 forbade, on pain of imprisonment, private relief donations that interfered with free market setting of prices during the famine\index{famine}.  Furthermore, “in contrast to the rigidity and dogmatism of British land-and-revenue settlements, both the Mughals and Marathas flexibly tailored their rule to take account of the crucial ecological relationships and unpredictable climate fluctuations of the subcontinent’s drought-prone regions. The Mughals had “laws of leather,”\index{laws of leather} wrote journalist Vaughan Nash during the famine\index{famine} of 1899, in contrast to the British “laws of iron\index{laws of iron}” (Nash 1900:92). Although the British insisted that they had rescued India from “timeless hunger,” more than one district official was jolted when Indian nationalists quoted from a 1878 study published in the prestigious Journal of the Statistical Society\index{Journal of the Statistical Society@\textsl{Journal of the Statistical Society}} that contrasted 31 serious famines\index{serious famines} in 120 years of British rule against only 17 recorded famines in the entire previous two millennia (Walford, 1878: 434–42).” ({\sl ibid}). To effect “balance of payments” of European powers (esp. British due to their massive war machine\index{war machine}: “military expenditures never comprised less than 25 percent (34 percent including police) of [British] India’s\index{British India} annual budget”; compare with typical 3--5\% defence budgets now), recourse was also taken to effect financial exchange rate manipulations that pauperized the Chinese\index{Chinese, exploitation of} and Indian people as China and India were on silver standard\index{silver standard}: the move to gold standard\index{gold standard} by major European powers (for eg, Germany in 1871), then by US, Japan caused silver to depreciate. It is said that “the gold standard stole one quarter of the purchasing power of India’s silver ornaments (Nash 1900:88)” (ibid.) this especially calamitous during the 4 major famines between 1875-1900. Taking the just the Berar region\index{Berar region}, “during the famine\index{famine} of 1899–1900, when 143,000 Beraris died directly from starvation, the province exported not only thousands of bales of cotton but an incredible 747,000 bushels of grain (Satya\index{Satya} 1994:148, 281–2, 296). Despite heavy labor immigration into Berari in the 1890s, the population fell by five percent and “life expectation at birth” twice dipped into the 15-years range before finally falling to less than ten years during the “extremely bad year” of 1900 (Dyson\index{Dyson} 1989:181–82).” Across British India\index{British India}, “in the age of Kipling - that “glorious imperial half century” from 1872 to 1921 - the life expectancy\index{life expectancy} of ordinary Indians fell by a staggering 20 percent, a deterioration in human biology probably without precedent in the subcontinent’s long history (Davis 1951:8).” 

More recent are the Bengal famine\index{Bengal famine} during WWII\index{WWII} (see, e.g., Mukerjee\index{Mukerjee, Madhushree} (2011) that details the extraordinary racism of Winston Churchill\index{Churchill, Winston} that resulted in the death of as many as 3 million Bengalis during WWII\index{WW II}; for example, official records show that Churchill ensured\index{ensured} that surplus grain in New Zealand would not be diverted to Calcutta during the famine but sent to godowns in Canada in spite of pressing entreaties of officials) and the 1971 genocide\index{genocide} by Pakistan Army (deaths estimated to be between 1.5 to 3 million, a substantial number being Indic persons) in the now Bangladesh abetted by US/UK (US and British navies made a threatening pincer against India in December 1971 with nuclear submarines) and some European powers as they were busy looking elsewhere. As Pollock and his followers casually implicate Sanskrit's deep structure\index{deep structure} and Mīmāṁsaka-s for genocide\index{genocide}, it may be of interest to know that many in Germany looked up to the obvious and demonstrated skills, for at least 1 century, of the British on how to ``pacify" native populations. For example, use of Maxim machine guns post 1880's turned British imperial conquest into a genocidal murder of a number of colonized groups of people in Africa. The German genocide\index{genocide} in Namibia through mass starvation, dehydration and plain unconscionable murder during 1904-14 was an example of how well the ``German student"\index{German student} had learnt the lessons from the ``English teacher\index{English teacher}". No wonder the Nazi Germans did not need much research into how to conduct a holocaust\index{holocaust!Nazi} barely two decades later.

The period just before the WWI can be said to be the period when Europeans and Americans had perfected the art of mass murder\index{mass murder} on native peoples worldwide in the name of convenient and unscrupulously applied ``natural principles\index{natural principles}". No wonder that WWI saw mass barbarism on a scale not known before; it requires no stressing that the Europeans had by now learnt the art of indiscriminate warfare (including use of poison gas\index{poison gas} and dropping of bombs on civilians using the new warplanes) except that WWI finally pushed the mindless violence mostly into the European continent itself.

Pollock quotes a part of {\sl Gītā}\index{Bhagavadgita@\textsl{Bhagavadgītā}} 2.63 at the beginning of his article: 

\begin{myquote}
{\sl (smṛti-bhraṁśād buddhi-nāśaḥ)}\\
{\rm “When memory is destroyed, intelligence is lost.”}
\end{myquote}
This is an apt quote: when memory of Namibia and other genocides\index{genocide} (either by Europeans or by Americans) or that of the Christian anti-Jewish\index{anti-Jewish} attitudes in history are erased, intelligent reasoning is a casualty.

\section*{Motives for Pollock’s Thesis}
Given the brief listing of the above genocides\index{genocide} and the attendant discussion, it is clear that Pollock is wide off the mark. But~the important question is why is Pollock raising such indefensible arguments?

First, it seems to require only a little thought to ascribe Indic influence on Nazism, namely use of swastika\index{swastika} (used in Nazi circles post 1920 as its symbol and adopted from swastika's use in the German völkisch\index{volkisch@\textsl{völkisch}} nationalist movements) or the use of “Aryan” racist\index{Aryan!race} paradigm\index{racist paradigm} but one can argue that this is a reflection of lazy thinking. Considering the swastika, it is useful to make a deeper analysis. While swastika is used in an Indic context in an auspicious sense almost exclusively ({\sl svastikaḥ\index{svastikah@svastikaḥ} => svasti (śubhāya hitaṁ)+ka} that which is good for auspiciousness), the swastika was used in the Nazi circles to project power or only in a political sense. The sense of “sacred” is completely absent. Moreover Nazi ideologues such as Alfred Rosenberg\index{Rosenberg, Alfred} came to Germany from Russia/Latvia/Estonia backgrounds where the swastika symbol was/is still a local cultural symbol; e.g., the Air Force Academy\index{Air Force Academy} (Finnish Air Force\index{Finnish Air Force}) has a swastika in its emblem as of now while the Finnish Air force had it till 1944 (See “Air Force Academy, Finnish Air Force”). Furthermore, the swastika has some resemblance to the Iron Cross, a military decoration\index{Iron Cross (Prussian military decoration)} in the Kingdom of Prussia\index{Prussia}, and later in the German Empire (1871–1918) except for an additional stroke at each end and the slant.

Similarly, the use of Arya in Indic context was always a term of respect, not power or race. Furthermore all the major Nazi ideologues\index{Nazi!ideologues} such as Rosenberg, Himmler\index{Rosenberg, Himmler}, or Chamberlain\index{Chamberlain, Houston} had access to Sanskrit through translations and very unlikely that the ``deep structure\index{deep structure}" of Sanskrit or the Mīmāṁsā theories (as casually thrown in by Halbfass\index{Halbfass}) was known to them. Furthermore, Elst\index{Elst, Koenraad} (2004) writes that 

\begin{myquote}
“... in spite of Himmler's\index{Himmler, Heinrich} openness to occult theories about the Aryan race\index{Aryan!race}, his plans to weed out the handicapped and to promote the procreation of superior Aryan types can very simply be derived from eugenicist ideas which were fairly widespread in the medical community of not just Germany but also of Scandinavia, the USA and other countries.”
\end{myquote}

So if Pollock is at all right about the connections between Sanskrit ``deep structure" and Nazi genocide\index{genocide}, then it has to be due to a sort of ``butterfly effect\index{butterfly effect}" in spite of the massive evidence against it (ie. a butterfly flapping its wings in, say, China causes a major hurricane in Midwest US as the weather system is a chaotic system and not amenable to easy mathematical analysis; a very small change in the initial conditions or state of a deterministic nonlinear system can result in large differences in a later state!). Or, is Pollock's explanation an ode to {\sl kākatālīya-nyāya\index{kakataliya-nyaya@\textsl{kākatālīya-nyāya}}} (the maxim of the accidental falling of the palm fruit upon a crow flying as per its fancy)?

What is surprising is that Pollock does not use the Christian Church as the ``deep structure" that could be a good explanation for the Holocaust\index{holocaust!Nazi}: its anti-semitism\index{anti-Semitism} can be said to be ``in-born" in a deep way. The Church has survived for 2 millennia as a powerful force (fully 1/3 rd of the world's population now subscribes to this “axiomatic” belief system\index{axiomatic belief system@“axiomatic” belief system} without any need for any substantiation); it is also likely to be the largest MNC\index{MNC} worldwide as of now in terms of resources and people (e.g., bishops\index{bishops, Vatican appointment} in India are appointed from Vatican\index{Vatican}), and its imprint is everywhere (in the Arctic too!). Its ``deep structure" should be impregnable as it has incorporated or ``digested" Judaism (e.g., Old Testament), paganism\index{paganism} (Greek, Roman or other European varieties), many native cultures worldwide (Native American, Ethiopian, Indian, etc). One can also argue that the English language\index{English language} fits the requirement for ``deep structure", if not at the grammatical level, at the economic, social and political level; it is the global hegemon\index{global hegemon} with many credits for subjugating hapless people worldwide. Contrast this with Sanskrit, whose “deep structure" that is said to have aided in genocide\index{genocide}, is on its last legs having been systematically driven close to extinction by the Islamic and Christian military structures and now by the ``modern secular" government in India itself. Its ``deep structure"\index{deep structure} alone unfortunately does not seem enough to keep it alive (argued by Pollock himself!) given the brutal environment it finds itself in.

What is also surprising is that Pollock and his later followers casually invert what really happened: Romas or Gypsies\index{Romas or Gypsies} (of Indic extraction) were murdered in large numbers (half a million) in Europe by the~Nazis. Instead of Indic world giving ideas for genocide\index{genocide} to the Europeans, the inverse is true with the people of Indic extraction, the~Gypsies as widely believed, getting genocided against! The victim is victimized in Pollock's analysis. Another example of this phenomenon is the following: Pollock writes

\begin{myquote}
“The fact that the production of {\sl dharma-nibandha}\index{dharma-nibandha@\textsl{dharma-nibandha}} discourse, as noted above, almost perfectly follows the path of advance of the Sultanate from the Doab to Devagiri to the Deccan (Lakṣmīdhara, Hemādri, Madhaya) suggests, on the one hand, that totalizing conceptualizations of society became possible only by juxtaposition with alternative lifeworlds, and on the other, that they became necessary only at the moment when the total form of the society was for the first time believed, by the privileged theorists of society, to be threatened.” 
\hfill(Pollock 1993: 105)
\end{myquote}

Here Islamic military structures who have gave no choice to large populations (Egypt, Persia, $\ldots$) except to accept Islamic domination (generously phrased as “alternative lifeworlds\index{alternative lifeworlds}”!) are viewed as doing a “positive” favour; to resist is to accept “totalizing conceptualizations of society” by “privileged theorists of society”. ISIS may be happy to hear of such formulations. Interestingly, in spite of many examples of wanton destruction of life and property by the Islamic armies in India, there is hardly any credible example of the reverse historically. 

So what is the explanation for Pollock's writings? The only~explanation that I can advance is that many of us, not excluding Pollock, are consciously or unconsciously embedded in powerful propaganda\break systems, and Pollock’s writings, I believe, can only be explained on such a model.  Consider the following feature that I came across in {\sl The ``Hindu"} newspaper\index{The Hindu (newspaper)@\textsl{The Hindu} (newspaper)} in 2012 (Basu 2012).

Here a very favourable report is given of a German woman, whose father was admittedly a Nazi soldier\index{Nazi!soldier} but she now is an Indian convener of the National Alliance of People's Movements,\index{National Alliance of People's Movements} and working in Tamil Nadu Theological Seminary,\index{Tamil Nadu Theological Seminary} Madurai (having first come to India with her husband on a two-year study programme at the Christian Institute for Study of Religion and Society,\index{Christian Institute for Study of Religion and Society} Bangalore). She researches Dalit atrocities, issues affecting farm workers, tobacco workers, fisherfolk and construction labourers, women activists, and ``has put her stamp on every movement radiating an uncomplicated spirit of affability".

What is surprising is her locus of work. When a massive tremor (Nazism) has struck Germany and one's own immediate father is involved (but ``we were not aware of what was happening"), the most convenient thing is to study someone else's misdeeds, even if they are not on the same scale (compare with 50-80 million deaths in WWII). It sure makes you feel better! Not just that; there is livelihood available through the same complicit (worldwide) organization! While one is not arguing that she cannot do research on any topic of her choice, what is interesting is the choice made: a ``concerned" person who escapes from family’s Nazi involvement and points fingers at others. It would have behooved her to do research and/or teach, say, on how Nazis came to power\index{power!political}, or how the Vatican\index{Vatican} helped Hitler\index{Hitler!Adolf} to assume dictatorial power etc (note that if the Catholic Centre party\index{Catholic Centre party} had voted\index{Hitler!Catholic Centre party voting for} no or even abstained on the Enabling\index{Enabling} (``dictatorship") Act, Hitler\index{Hitler!Adolf} would have been defeated but their leader Ludwig Kaas\index{Kaas, Ludwig}, a close friend and advisor to Eugenio Pacelli\index{Pacelli, Eugenio}, the future Pope Pius XII\index{Pope Pius XII}, endorsed the Enabling Act\endnote{See “ADL Welcomes Election of Cardinal Ratzinger as New Pope”.}) or even how Lutheran Church\index{Lutheran Church} funds conversions,\index{conversions} etc in India resulting in severe social stress in rural and semi-urban communities and how foreign money supports the powerful Church establishment in South India (Malhotra et al 2011), etc. 

Pollock's discourse is unfortunately not any better. Pollock's clear worldview is that ``Hindutva"\index{Hindutva} is responsible for many ills that afflict India and, as a concerned person, it his duty to fight against such tendencies. But the Hindutva Movement\index{Hindutva Movement} has arisen precisely because the Indic people have been essentially without voice for the better part of seven decades after 1947 (tragically possibly due to Hindu Mahasabha’s Nathuram Godse murdering Gandhi and the Congress Party/Nehru family\index{Congress Party/Nehru family} exploiting it to marginalize any serious Indic thinking\index{Indic!thinking}).  Given such a worldview but one without any serious basis in reality, only a propagandistic model makes sense to understand an eminent academic like Pollock; a more detailed look is given below.
\eject

By ignoring or not acknowledging recent research (e.g., the Namibian genocide\index{genocide} of the Germans as a likely blueprint for Jewish genocide\index{genocide} of the Nazis) and proposing the deep structure\index{deep structure} of Sanskrit as a possible candidate, Pollock is precisely doing a needed exercise in a propaganda system. If an obvious truth has to be made to recede into the background, non-issues that cloud one's thinking has to be casually introduced, then carefully nurtured and repeated by others so that in a few decades non-issues can take centre stage. A good example can be seen in the demonization of Modi\index{Modi, demonization of} since 2002; even if careful research shows the hand of vested or complicit interests (Congress Party, for example), the manufactured ``facts" take a life of their own. This is a specific instance of a more general anti-Indic\index{anti-Indic} (or anti-BJP\index{anti-BJP}) propaganda\index{propaganda}. Madhu Kishwar\index{Kishwar, Madhu} in her book {\sl Modi, Muslims and Media} documents these carefully but the propagandistic system present in the country does not yet allow for a careful examination of the truth (Kishwar 2013). It is interesting to know that Modi’s visit to US when he was the CM of Gujarat was thwarted by a collaboration between evangelical Christian Right in the US, the Indian (expat) Left and the Islamic interests even when not a single FIR was filed against him (and till today not filed) (Janmohamed\index{Janmohamed} 2013). Pollock was one of the signatories to this petition\endnote{See Coalition Against Genocide (2005).}; the surprise here is that, without evaluating the proper situation on the ground and going by prejudices, he seems to accept that a “genocide”\index{genocide} has taken place in Gujarat, while at the same time deflecting attention from the complicity of major genocidal actors such as Europeans/Americans or the Church with respect to what happened in Americas, Europe or Australia by targeting Sanskrit as a contributing factor. Modi was then and now is possibly seen as a threat as he is a symbol of a self-respecting nation and has proved many detractors wrong by his measured moves and hard work. When Swami Laxmanananda Saraswati\index{Saraswati, Swami Laxmanananda} in Orissa was killed, it was again a surprising collaboration between Maoists\index{Maoists} and evangelical Christians\index{Christians}. The United States Commission on International Religious Freedom\index{The United States Commission on International Religious Freedom} (which is dominated by evangelical Christians) strangely concluded in 2009 that India and Afghanistan are in the same category (``watch list of countries of particular concern") when it comes to religious freedom (and that Pakistan is actually better).

Another good example is that of Mathias Tierce\index{Tierce, Mathias}, a German author, editor and yoga enthusiast (Ghosh\index{Ghosh, Kali Charan} 2012). His books include {\sl Yoga in the Third Reich. Concepts, Contrasts, Consequences}. Probably~influenced by Pollock-type writings, this yoga enthusiast here repeats unsubstantiated charges or unbalanced critiques; Himmler\index{Himmler, Heinrich} carrying the {\sl Gītā}\index{Bhagavadgītā@\textsl{Bhagavadgītā}} with him everywhere, or misusing or misunderstanding its complex contextual reasoning is used to taint {\sl Gītā} with Nazism. {\sl Gītā's} and\break Upaniṣadic messages are for a person who has already reached a high level of ethical conduct (as they equate the human self with the cosmic Self, a dangerous doctrine in the hands of immature persons) and not for someone who needs justifications for murderous deeds. {\sl Gītā's} message\index{Gītā, message of the} is closer to ``do your right or just duty without worrying about the rewards", but not to ``do unjust (or murderous) acts with detachment (!) and expect to be free from that karma". However, this yoga enthusiast/author paraphrases supposedly one of Himmler's favourite {\sl śloka-s as} ``even if they commit evil acts, they can still remain untainted and unaffected by ones' own actions” which seems a clear fabrication. He uses this as an example of the amoral world of {\sl Gītā} and Himmler\index{Himmler, Heinrich}. Probably, he is referring to {\sl Gītā} 4.36 where it is said that even the worst sinner can cross over all the wickedness with the raft of knowledge. Or, mixing {\sl Gītā} 4.36 with 4.14 where Kṛṣṇa says that ``actions do not taint Me as I have no hankering after the results of actions"! Or, wilfully misconstruing {\sl Gītā} 5.7 where a {\sl karmayogin} is said to be untainted even though performing action by not also quoting that he has to be pure in his heart, fully conquered his mind and mastered his senses, etc. Again, Tietke is clearly fulfilling one role in a propaganda system where mud is thrown widely by (preferably) disparate actors in the hope that something will stick even if known to be untrue. Another exercise (in the opposite direction!) that attempts to disconnect (“good”) Yoga from Indic tradition and making it Western-inspired is Mark Singleton’s\index{Singleton, Mark} book on Yoga ({\sl Yoga Body: The Origins of Modern Posture Practice}); it is not discussed further due to lack of space.

Furthermore, Pollock has interpreted many Indic texts, such as the {\sl Rāmāyaṇa}\index{Ramayana@\textsl{Rāmāyaṇa}}, as reflective of social oppression and Sanskrit as complicit. Koenraad Elst\index{Elst, Koenraad} (2004) writes: 

\begin{myquote}
“Marxists never wonder whether a theory is true or not, they only care about what class interests a theory may serve.  Lenin despised a concern for universally valid truth as “bourgeois objectivity\index{bourgeois objectivity}”; in this respect, he was the forerunner of postmodern relativism\index{postmodern relativism}.”
\end{myquote}

Thus, it probably matters little if there is sufficient truth in an analysis or interpretation; it matters only if such an analysis can be used to reduce ``oppression" as refracted into the text. Thus, even such propaganda is legitimate, as the end goal is all that is important. Furthermore such propaganda is useful to reduce the discomfort sensitive souls steeped in “Western Universalism\index{Western Universalism}” would experience if informed that Nazi holocaust\index{holocaust!Nazi} “pre-formed” closer home (given already the sensitivity to oppression in Sanskrit texts\index{texts (Sanskrit/Indic)}). Hence, it is useful to look for Nazi holocaust precursors elsewhere to assuage conscience or as a part of dominant narrative (or equivalently, part of propaganda system\index{propaganda system}) and to preempt the impact of works like Stannard\index{Stannard, David E} ahead of time. This provides implicit support to “western universalism” by whitewashing European genocides\index{genocide!whitewashing European} (of Africans/Blacks, Jews, Romas, $\ldots$)  and explicit support to the demonization of Indic systems\index{Indic!systems, demonization of}. Hence it is important to understand how the propaganda systems work.
\smallskip

Interestingly, Pollock discusses this as part of the same article under consideration in this paper: 

\begin{myquote}
“In German Indology of the NS era, a largely nonscholarly mystical nativism deriving ultimately from a mixture of romanticism and protonationalism merged with that objectivism of Wissenschaft earlier described, and together they fostered the ultimate ``orientalist" project\index{orientalist project}, the legitimation of genocide\index{genocide}. Whatever other enduring lessons this may teach us, it offers a superb illustration of the empirical fact that disinterested scholarship in the human sciences, like any other social act, takes place within the realm of interests; that its objectivity is bounded by subjectivity; and that the only form of it that can appear value-free is the one that conforms fully to the dominant ideology, which alone remains, in the absence of critique, invisible as ideology."

~\hfill(Pollock 1993:96)
\end{myquote}

However, stating this does not mean Pollock himself is not serving a propaganda function, ie of tarring a local tradition with Nazism when its intellectual materials have been misconstrued by German Nazis or German Indologists for reasons of settling scores with their bugbears. Note that the word ``orientalist" ideally should have been qualified as “European orientalist” here, as historically they were most proficient at genocide (Spanish conquest\index{Spanish conquest} of Americas, slaving/extermination\index{extermination} of blacks or Africans\index{Spanish conquest of Americas, slaving/extermination of blacks or Africans}, etc). Using the ``orientalist" word colors the mind somehow into thinking that ``orientals" are somehow responsible for genocide, hence serves a propaganda function outright. Another instance of this occurs as “In dissecting what accordingly has to be seen as the dominant form of Indianist orientalism$\ldots$”; here he is actually discussing German indology and is therefore needlessly confusing.

\section*{A Propaganda Model}\index{A Propaganda Model}
A good model of a propaganda system is that of Herman and Chomsky (1988). While their model is with respect to mass media, the same analysis is broadly applicable to other manifestations of thought control\index{thought control} such as in academia or how Indic systems\index{Indic!systems} of thought are portrayed in intellectual circles. It will be helpful to examine the usefulness of this model, after making some of our own changes, for understanding the propaganda against Indic systems of thinking. They write, 
\smallskip

\begin{myquote}
``$\ldots$ based on many years of study of the workings of media, [that it] serve[s] to mobilize support for the special interests that dominate the state and the private activity, and that their choices, emphases, and omissions can often be understood best, and sometimes with striking clarity and thought, by analyzing them in such terms."

~\hfill(Herman and Chomsky 1988:1)
\end{myquote}
\smallskip

Their propaganda model\index{propaganda model} views “private media\index{private media} as businesses interested in the sale of a product—readers and audiences—to other businesses (advertisers) rather than that of quality news to the public.” The media's “societal purpose”, they write, “...[is to serve the] agenda of privileged groups" but the study of [the media] institutions\index{institutions!media} and how they function must be scrupulously ignored, apart from fringe elements or a relatively obscure scholarly literature” (see “Propaganda model”). The theory postulates five general classes of ``filters"\index{filters (in news media)} that determine the type of news that is presented in news media. These five classes are: Ownership of the medium, Medium’s funding sources\index{funding sources}, Sourcing, Flak and enforcers, Anti-communism and fear ideology\index{fear ideology} [such as ``War on Terror" and ``counter-terrorism"]. Furthermore, (p.2) ``these elements interact with and reinforce each other. The raw material of the news must pass through successive filters, leaving only the cleansed residue fit to print."
\vskip 2pt

This theory has to be certainly modified for India with a different set of filters as I am not so much concerned with media exclusively as with thought or knowledge systems (such as Indic, Christian or Islamic) represented in books, academic talks and instruction, plays, blogs etc. here, so this also needs to be modified in the model.

\section*{The filters}

First, for our purposes of understanding Pollock, one can replace ``Anti-communism" with ``Anti-Indic"\index{anti-Indic} as the concerned media’s (world-wide) funding sources\index{funding sources} is mostly from anti-Indic forces (such as many foreign NGOs\index{foreign NGOs}, Christian organizations, Islamic sources, etc.). To justify and put this in perspective, first note that funds available to Abrahamic faiths\index{Abrahamic faiths, funds available} worldwide dwarfs that available to Indic systems\index{Indic!systems}. In India itself, as a point of comparison, note that India’s total defence allocation (2011-12) was 1,64,000 crores. Foreign-funded NGOs\index{Foreign-funded NGOs} (FFNGOs) in 2011-12 received about 12,000 crores ie. about 7\% of India's then defence budget. It is widely believed that the inflow is much larger if non-legal flows as well as funds that come through UN system are also added to these numbers. It is noteworthy that funds given by UN bodies\index{UN bodies, funds from} or World Bank\index{World Bank} are not under the purview of FCRA\index{FCRA} (e.g., World Bank funding to GoI in 2013-14 was \$5.2 billion (Rs 33000 crore) compared to Rs 12,000 crore (\$ 1.8 billion) received by NGOs\index{NGOs}). Note also that the majority of donors including the top three are church-based organizations. It is therefore not surprising that any malfeasance regarding foreign NGOs or Christian/Islamic organizations in India do not find a receptive environment and gets attenuated in public discourse very quickly while anything anti-Indic can be played and replayed till it becomes centre-stage. Hence replacing the anti-communism filter with an anti-Indic filter is justifiable with respect to topics under discussion in this paper.

Let us now consider other filters\index{filters (in news media)}. First, “ownership/control of the medium”. Even to this day, the media, the judiciary, the NGO sector or the education sector in India are owned or populated by a substantial anti-Indic elite\index{anti-Indic!elite}. A good illustration is the recent “Award-wapsi”\index{Award-wapsi} events (2015) or supposed attacks on Church properties (2014) where minor or not so important events were magnified to hog all mindspace in the print and TV media.  On the contrary, substantial damage, destruction related to Indic sphere (a good example: temples\index{templ!destruction of} in Kashmir\index{Kashmir}), or killings/murders/exile of those in the Indic camp (again, Kashmiri pandits\index{Kashmiri pandits}) are/were barely covered. At the international level, it is even more pronounced: all major newspapers in US/UK, e.g., NYT/BBC\index{NYT/BBC}\endnote{For NYT, for a good summary, see
{\scriptsize\url{https://en.wikipedia.org/wiki/Anti-Indian_sentiment\#New_York_Times} }(accessed Jul 28, 2016). For BBC see, Alasdair Pinkerton (Oct 2008). ``A new kind of imperialism? The BBC, cold war broadcasting and the contested geopolitics of South Asia". Historical Journal of Film, Radio and Television 28 (4): 537–555. This is a peer-reviewed article that analyses BBC Indian coverage from independence through 2008; it concludes that BBC coverage of South Asian geopolitics and economics has had a pervasive Indophobic slant.}, regularly write highly negative or stereotyped accounts of happenings in India.

Next, the funding sources\index{funding sources} as filter. The model posits that the “people buying the media products (newspapers, magazines, TV) are the product that is sold to the businesses that buy advertising space; the “news” has only a marginal role as the product”. The situation is a bit different in India compared to US where private property is more important.  In India, due to the domination of the Nehru family\index{Nehru family} for a better part of the last 7 decades, and limited availability of advertisements or newsprint, only news that reflected this family’s interests or did not negatively affect it could be printed to continue to get Government advertisements\index{Government advertisements}, and journalists would also get privileged access\index{privileged access} (e.g., on PM’s plane on trips abroad) or freeloading\index{freeloading} in terms of 5 star hotel stays or food, or even plots of land at throwaway prices. Therefore, much of the news reported had to make the Nehru family look good at least in the first 3-4 decades after independence. Here, news could be said to be marginal product of making the political families look good.

Against this, funding related to Indic activities is either at a cottage industry level (highly localised or, worse, disorganized and possibly even harmful sometimes) or actually in the hands of the anti-Indic\index{anti-Indic!elite} elements. Much of the temple wealth\index{temple!wealth} has been neutralized as the rich temples have been taken over by the Government and the temple ecosystem\index{temple!ecosystem} has been run to the ground or actually to fund the anti-Indic forces. Additionally, Church-related or Saudi Arabian funds are highly visible; almost any drive across any major highway in South India should be sufficient visual proof.  Bollywood\index{Bollywood} seems to have been under some Dubai/Pakistan or other Islamic connections for some time; however, after the Vajpayee government declared it as an industry, bank loans have became possible and the role of tainted or black money is becoming a little less.  Furthermore, many intellectuals\index{intellectuals} seem to be under the grip of Western interests (either due to direct funding or indirectly through awards, scholarships, and such). Consider the recent award of Magsaysay award\index{Magsaysay Award} to a Karnatic musician; the award citation specifically states that it is for ‘ensuring social inclusiveness in culture’, a surprise as many {\sl varṇa-s}\index{varna@\textsl{varṇa}} and \hbox{{\sl jāti-s}} (example, MS Subbulakshmi\index{Subbulakshmi, M S}, Dwaram Venkataswamy Naidu\index{Naidu, Dwaram Venkataswamy}, K J Yesudas\index{Yesudas}, Rajaratnam Pillai\index{Pillai, Rajaratnam}, Sheik Chinna Moulana\index{Sheik Chinna Moulana}, etc) are/were stars in the field in addition to the (demonized) Brāhmaṇa-s; the citation says further that “He saw that his was a caste-dominated\index{caste!domination} art that fostered an unjust, hierarchic order by effectively excluding the lower classes from sharing in a vital part of India’s cultural legacy” (See “Thodur Madabusi Krishna”\index{Krishna, Thodur Madabusi} entry in Wikipedia). Such motivated awards are de rigueur now from Western-inspired agencies and serve to deepen social cleavages in areas where not noticeable before; note also the award winner’s use of the word “caste” in approved ways. No wonder that anti-Indic\index{anti-Indic!elite} stories get maximum play in print and TV media in India (this might be becoming just a bit less after 2 years of Modi’s government and the increasing untenability of the anti-Indic story line applied indiscriminately in news events and their consequent exposure in the social media).

With respect to the next filter (sourcing), ``the mass media are drawn into a symbiotic relationship\index{symbiotic relationship} with powerful sources of information by economic necessity and reciprocity of interest\index{reciprocity of interest}.” In our case, US universities have dedicated departments to study Indic religions but, due to convenient concerns about ``secularism", Hindu or Indic religions cannot be studied or encouraged to be studied as such in Indian universities\index{Indian universities, religion studies in}. This is probably due to the excessive caution of the executive as the Constitution of India\index{Constitution of India} lists “to develop scientific temper” as one of the fundamental duties of Indian citizens (but surprisingly seems to be restricted only to the Indic community). Hence the sourcing issue is important as (small) armies of US scholars study Indic religions with all their biases\index{bias} intact and fill up the intellectual space with their writings. Since the Indic scholars cannot be supported by the formal research organizations in India, research output in this sector in India resembles at the most a cottage industry with sporadic or poor quality writings. It is often said that Sanskrit study is better done outside India even when native scholars that remain are outstanding! The end result is that US universities decide the course of research on Indic issues and native Sanskrit/native language scholarship counts for nothing. As Rajiv Malhotra\index{Malhotra, Rajiv} and his colleagues have documented in {\sl Invading the Sacred\index{Invading the Sacred@\textsl{Invading the Sacred}}}, native informants are used sometimes by European/American academics to extract information that end up being used in surprisingly twisted ways.

Next, the flak or enforcer filter\index{enforcer filter} employed against Indic systems\index{Indic!systems}. ``If flak is produced on a large scale, or by individuals or groups with substantial resources, it can be both uncomfortable and costly to the media". Here flak is what makes receptivity difficult to all shades of thought including Indic perspectives, and how flak is orchestrated for specific purposes. Specifically, flak here is essentially the hostility to the Indic perspective or showing it up for its non-“coolness”, or that squelches uncomfortable questions about received models of understanding (e.g., ``Modi a butcher or murderer”). A good example is the invisibility and lack of positions in Indian academia for people with Indic focus\index{Indic!focus} in the social sciences area; flak ensures that they cannot survive in the hostile environment. Madhu Kishwar became {\sl persona non grata} once she started research on understanding why Modi was being demonized incessantly and wrote the book {\sl `Modi, Muslims and Media'}. As another example of flak, she found it difficult to find a publisher for her book (and had to publish it herself) or launch her book in Delhi (2013) as no Indian Government funded social research body would provide the venue or, having promised a venue, would invariably cancel at the last minute.

For fear ideology\index{fear ideology} in the model, one can list RSS/BJP demonization\index{BJP, demonization of}\index{RSS, demonization of}, “Hindutva”\index{Hindutva}, “right wing fascism\index{right wing fascism}”, human rights violations (atrocity literature\index{atrocity literature} (Malhotra et al 2011)), academic Hinduphobia (Malhotra 2016) as exemplars.

\section*{Validation of the Model}

While this is non-trivial, the last 2 decades has shown the issues in sharp detail. For the first time since 1947 political formations with sympathy for or support of Indic peoples are serious contenders for power\index{power!political} (in some states) or in power (at the centre or in some states). The propaganda system against the Indic people finds this unacceptable and has used every means available to contain it, the most notable example being the demonization of Modi starting from about 2002 and continuing even now. I discuss this briefly due to lack of space; there are also many published works on the subject (e.g., early works such as those by Dayanand Saraswati\index{Saraswati, Swami Dayananda}, Vivekananda\index{Vivekananda, Swami}, Sita Ram Goel\index{Goel, Sita Ram}, and more recent works —such as by Rajiv Malhotra  ({\sl Invading the Sacred\index{Invading the Sacred@\textsl{Invading the Sacred}}; Breaking India} (with Arvind Neelakandan); {\sl Academic Hinduphobia}); by Madhu Kishwar ({\sl Zealous Reformers; Deadly Laws: Battling Stereotypes; Modi, Muslims and Media}), or the works of S N Balagangadhara\index{Balagangadhara N}. Some well known issues are the following:
{
\begin{enumerate}
\item how Ishrat encounter case\index{Ishrat encounter case} was reported compared to Sadhvi Pragya’s\index{Sadhvi Pragya} imprisonment and torture
\item temples under government control\index{temple!government control} and their ruination. Once upon a time, temples were the centers of learning, culture and {\sl svādhyāya} (study of self). Government control has made temples vacate its role in learning and culture, leaving only a shell under an unfriendly  bureaucratic management and thus is so unsatisfactory that going to temples is now a chore, given the opportunistic developments and land grabbing around the temples\index{temple!land grabbing of} and the resulting noise and pollution. Propagation of Dharmic activity through the temples is no longer easy given that a parasitic bureaucracy\index{parasitic bureaucracy} decides what can be done.
\item the sorry state of Indic languages\index{Indic!languages}, especially Sanskrit.
\item legal interference\index{legal interference} in many Indic traditions (e.g., Sādhu-s\index{Sadhu@Sādhu} in principle could be classed as “vagrants” under still not-repealed British Raj\index{British Raj} regulations and hence confined), Jain practice of {\sl santara}\index{santara@\textsl{santara}} (voluntary death by fasting) declared illegal suddenly, patronizing judicial pronunciations against customary practices in places of worship.
\item given the “no-interference” or “no-go” policy of Nehruvian\index{Nehru!policies of} policies with respect to personal laws of “minorities” (an excessive caution given that the constitution of India recognises minority entitlements\index{minority entitlements} with respect to educational institutions\index{institutions!educational} only, minority being either linguistic or religious), NGOs\index{NGOs} and others have felt no restraint in proposing and effecting legal remedies for the Indic peoples without any balance (going against the principle of natural justice\index{natural justice} where an accused has to prove his innocence rather than the accusing party) such as mindlessly stringent dowry laws\index{dowry laws} that criminalizes the man’s side of the family without due process, attempted legislation with respect to rape in marriage or with respect to divorce, attempted legislation on communal violence\index{communal violence, legislation on} that makes only the “majority” community responsible, etc. The Indian state has been derelict in outsourcing legislative wisdom to favoured NGOs\index{favoured NGOs} with its most egregious form during the UPA rule (2004-2014) through the National Advisory Council\index{National Advisory Council}, initially not even a legal entity, that resulted in a dyarchy\index{dyarchy} where power was exercised without responsibility.
\end{enumerate}}

\section*{Pollock and the Propaganda Model}\index{propaganda model}

The following can be cited to show that the model is useful with respect to Pollock:
{
\begin{enumerate}
\item The Anti-Indic\index{anti-Indic!elite} perspective of Pollock is clear going by his writings. He has written about the oppression in the Sanskrit cosmopolis as well as being part of campaigns against Hindutva\index{Hindutva} and other issues as a “concerned Sanskritist”. There is no sense of sacred in his view of Sanskrit philology as {\sl The Battle for Sanskrit}\index{The Battle for Sanskrit@\textsl{The Battle for Sanskrit}} by Rajiv Malhotra clearly shows. While he has opined that Sanskrit is dead, he still wants to flog a “dead horse” going by his enormous output; what for?
\item The sourcing aspect (availability of intellectual materials) is clear as the major visible centers of formal intellectual Sanskrit scholarship\index{Sanskrit!scholarship} has moved to the West (US, Germany) such as at Ivy schools\index{Ivy schools}. Students from India go to these centers for higher studies without any serious connection with local Indic scholarly community.
\item The funding aspect is also clear: Government of India has for all practical purposes refused to fund “religion” based studies, especially Indic ones and the only funds available are dispersed/local. While funds for such studies are also not that easy in US/Europe, the funds available at Ivy schools are still much larger and more consistent than what are available in India.
\item The flak is contributed by many of Pollock’s supporters or his students, professional bodies like RISA/AAR\index{RISA/AAR}, Left and Abrahamic supporters in the media in India and elsewhere. Note also the example of the two Yoga instructors (Mathias Tierce and Mark Singleton) given earlier.
\end{enumerate}}

\section*{The Reality of anti-Indic Propaganda and Its Counter}\index{anti-Indic!propaganda}

Given the propaganda system in operation, it is clear that Indic systems\index{Indic!systems} will be at the receiving end. There has been and will in future be disinformation, “digestion”\index{digestion}, defamation, victimization/intimidation, attempted starving of funds, etc. Given this reality, what are the options open to us now? There are some heavy handed approaches such as trying to pass anti-defamation\index{anti-defamation}\endnote{The Indian legal system also has a defamation model but seems too stringent; hence, Subramanian Swamy’s recent attempt to make it more helpful with respect to freedom of expression. The Jewish Anti-Defamation\index{anti-defamation} League is a successful “pressure group” approach that might be useful to consider in the Indic context.}, anti-hate laws\index{anti-hate laws} just like anti-dowry\index{anti-dowry laws}, anti-SC/ST atrocities laws\index{anti-SC/ST atrocities laws}\endnote{For example, if we need to develop a proper critique of Ambedkar or other leaders in SC/ST community, we have a big problem as of now. Due to 4 (v) of Jan 1, 2016 - NO. 1 OF 2016. [31st December, 2015.] An Act to amend the Scheduled Castes and the Scheduled Tribes (Prevention of Atrocities) Act, 1989 (www.indiacode.nic.in/acts-in-pdf/2016/201601.pdf) passed only recently by both Rajya Sabha and Lok Sabha, if you ``by words either written or spoken or by any other means disrespects any late person held in high esteem by members of the Scheduled Castes or the Scheduled Tribes” then it ``shall be punishable with imprisonment for a term which shall not be less than six months but which may extend to five years and with fine."}, etc. but these are generally problematic and often indefensible (e.g., due to the burden of proof on accused, reversing standard jurisprudence\index{standard jurisprudence}). Getting sufficient support in the legislatures may also be quite difficult for the Indic case. Or one can use the country’s existing laws such as under Section 295A of Indian Penal Code\index{Indian Penal Code}, which forbids “deliberate and malicious acts intended to outrage the feelings of any religious community,” just as Batra\index{Batra} used it against Doniger’s\index{Doniger, Wendy} book; this is also problematic as such laws can also interfere with sincere and incisive discussions. While poison pills\index{poison pills} (Malhotra 2014:300) and such other interesting ideas (just like copyleft in software) are useful in protecting one set of truth claims against another, they do not protect the replication mechanisms for such ideas themselves and hence need some complementary support.

One approach also could be the use of (IPR) laws on moral rights\index{laws on moral rights} (especially the part concerning right to the integrity of the work). These guarantee that a work cannot be “modified” (for example intentionally mistranslate a book) to bring the original author into disrepute. However, these do not work as the works in question, e.g., {\sl Gītā}\index{Bhagavadgita@\textsl{Bhagavadgītā}}, are typically many centuries old and we are also discussing a philological collection. Essentially, many Indic works/thought systems are in the public domain with no tangible “author”.

The best course seems to be to work with available means for redress (e.g., with the recent California text book\index{California text book} hearings) but many systems may not be as open as necessary for redress in many cases. In such cases mobilization and protest remain the only choices.

Another approach is using the model available in the free software community. Here technical competence and meritocracy matter; if one is not able to produce the free software required for a particular purpose, that inability makes that person incapable of affecting further decisions viz “We reject kings, presidents and voting. We believe in rough consensus and running code” (Dave Clark).  Since some model of effectiveness is available in the world of (computer) code to a reasonable extent across the world, this perspective is feasible. If such a model of effectiveness is available in matters largely affecting Indic persons, those with effective solutions can be said to have “{\sl adhikāra}\index{adhikara@\textsl{adhikāra}}”.

However, in the battle against anti-Indic propaganda\index{anti-Indic!propaganda}, we typically only have ``truth claims". (If they involve only “facts”, standard methods may be good enough.) How do we judge effectiveness of an argument?  When does criticism cross the line and become propaganda? Who can speak for or provide critical perspectives on Indic models? Could this be how well such perspectives or criticisms are received by Indic populace? Is this akin to the notion of {\sl adhikāra} present already in the tradition? Or, does this mean those that help the Indic community in a significant way (e.g., Baba Ramdev, or Sri Sri Ravishankar) or those who are perceived to be effective widely have the “{\sl adhikāra}\index{adhikara@\textsl{adhikāra}}” and not others? If alternative communities without connection with ``original" ones (for example, a Yoga school outside India) are in the picture, what right do they have in interpreting tradition? Is the {\sl adhikāra} is to be negotiated, not asserted? Are the moral rights to be acknowledged? Given Rajiv Malhotra’s extensive documentation of “digestion”\index{digestion} of Indic systems\index{Indic!systems, digestion of} by powerful forces outside the Indic sphere, these are not minor issues and such issues may well actually decide the trajectory of the Indic universe.

Note that in some other areas of IPR there has been movement already such as in geographical indicators\index{geographical!indicators} (basmati rice\index{basmati rice} is a good example; it can only be product from the Indian subcontinent) or in biodiversity\index{biodiversity} (where there is some protection from exploitation of rare flora and fauna); misrepresentation attracts legal penalties. Note that this problem is also faced by traditional knowledge systems worldwide. While issues affecting traditional knowledge have been studied well in the last few decades, in India at least, some motivated researchers have minimized them or put these systems as existing in opposition to Indic systems\index{Indic!systems} due to fears of “Hindutva”\index{Hindutva} and the like (for example, see Neelakandan’s discussion with respect to criticism against “Ayurgenomics”\index{Ayurgenomics} (Neelakandan 2016)).  We may need to see how traditional knowledge (e.g., use of turmeric\index{turmeric}) has been protected and possibly explore extending an appropriate model to the Indic world also. This has major implications for Indic systems as many of them are in the informal domain and hence are subject to capture, disinformation or “digestion”. It is clear that the above questions need to be investigated carefully; we have only scratched the surface.

In Pollock’s case, does his political or ideological perspectives (e.g., Marxist-like or “Western Universalist” perspective) preclude any {\sl adhikāra\index{adhikara@\textsl{adhikāra}}}? Given the open-ended aspect of lived life, such a restrictive perspective or model also seems problematic. Caraka\index{Caraka}, for example, says

\begin{myquote}
{\sl kṛtsno hi loko buddhimatām ācāryaḥ, śatruś cābuddhimatām}

{\sl ataś cābhisamīkṣya buddhimatā’mitrasyāpi dhanyaṁ yaśasyam āyuṣyaṁ pāuṣṭikaṁ lokyam abhyupadiśato vacaḥ śrotavyam anuvidhātavyaṁ ceti}

~\hfill {\sl (Carakasaṁhitā 8.14)}

{\rm “...the whole world is teacher for the wise and an enemy for the foolish. Therefore, the wise, after careful examination, learn even from rivals about excellence, fame, long life, health, and of world affairs by listening to their words of instruction and assimilating them”.}
\end{myquote}

A better approach or perspective may therefore be that already present in {\sl Taittirīya Upaniṣad} (Śikṣā Valli, Anuvāka 11) that says (repeated here for emphasis), 


\begin{myquote}
{\sl yāny anavadyāni karmāṇi | tāni sevitavyāni | no itarāṇi |}

{\rm Follow only such actions that `are untainted', not any other.}

{\sl yāny asmākaṁ sucaritāni tāni tvayopāsyāni | no itarāṇi |}

{\rm Whatever is good in us, follow those, not any other.}

{\sl ye ke cāsmacchreyāṁso brāhmaṇāḥ | teṣāṁ tvayā’’sanena praśvasitavyam |}

{\rm If there are any better brāhmaṇa-s, offer them a seat and “don't so much as breathe”!}
\end{myquote}

\section*{Conclusions}

In this paper, we have explored one paper of Pollock that touches on German Indology, Sanskrit and the Nazi ideology and attempted to show its connection with a propaganda\index{propaganda (passim)} system as the best explanation available. Further work is needed to empirically substantiate some aspects of the model (for example, funding, sourcing aspects from funding agencies as well establish the number and nature of “flak enforcers” in news media as well newer media such as social media). The intersection of propaganda, anti-Indic\index{anti-Indic!elite} bias\index{bias} as well as legal remedies also have been sketched briefly; this needs more serious work.


\begin{thebibliography}{99}
\itemsep=2pt
\bibitem[]{chap3_item1}
“ADL Welcomes Election of Cardinal Ratzinger as New Pope” (19 Apr, 2005) \url{http://www.adl.org/PresRele/VaticanJewish_96/44698_96.htm}. Accessed on 20 May 2016. 

\bibitem[]{chap3_item2}
“Air Force Academy, Finnish Air Force” (Last modified on 10 Jun, 2016) \url{https://en.wikipedia.org/wiki/Air_Force_Academy,_Finnish_Air_Force}. Accessed 11 Feb 2017.

\bibitem[]{chap3_item3}
Alvares, Claude (2000) {\sl Collected Writings of Dharampal Series}. Mapusa: Other India Press

\bibitem[]{chap3_item4}
Bajaj, J and Srinivas, M.D. (1996) “{\sl Annaṁ bahu kurvīta}”. Madras: Centre for Policy Studies. 

\bibitem[]{chap3_item5}
Bartrop, Paul R and Jacobs, Steven Leonard (2015) (eds) {\sl Modern Genocide: The Definitive Resource and Document Collection}. Santa Barbara, CA: ABC-CLIO.

\bibitem[]{chap3_item6}
Basu, Soma (03 May, 2012) “From Germany. For India”. In {\sl The Hindu}. \url{http://www.thehindu.com/features/metroplus/society/from-germany-for-india/article3380644.ece}. Accessed 15 May 2016.

\bibitem[]{chap3_item7}
Coalition Against Genocide (24 Feb, 2005) “Faculty Letter to AAHOA”. 
\url{http://www.coalitionagainstgenocide.org/press/support/faculty.aahoa.php}. Accessed 06 Aug, 2016

\bibitem[]{chap3_item8}
“Colonial Genocides” \url{http://www.cis.yale.edu/gsp/colonial/index.html}. Accessed May 20, 2016. Now see under \url{gsp.yale.edu}.

\bibitem[]{chap3_item9}
Davis, K. (1951) {\sl Population of India and Pakistan}. Princeton: Princeton Univ Press.

\bibitem[]{chap3_item10}
Davis, Mike (2001) {\sl Late Victorian Holocausts: El Niño Famines and the Making of the Third World}. NY: Verso. 

\bibitem[]{chap3_item11}
Dharampal (1971) {\sl Collected Works of Dharampal (Vol 1.) Indian Science and Technology in the Eighteenth Century}. Mapusa: Other India Press.

\bibitem[]{chap3_item12}
Dirks, Nicholas B. (2001) {\sl Castes of Mind: Colonialism and the Making of Modern India}. Princeton, NJ: Princeton University Press.

\bibitem[]{chap3_item13}
Dutt, R C (1900) {\sl Open Letters to Lord Curzon on Famines and land assessments in India}	. London: Kegan Paul. 

\bibitem[]{chap3_item14}
Dyson, T. (1989) “The historical demography of Berar, 1881–1980”. In T. Dyson (Ed.) {\sl India’s Historical Demography: Studies in Famine, Disease and Society}. London. 

\bibitem[]{chap3_item15}
Elst, Koenraad (12 Aug, 2004) “Hinduism, Environmentalism and the Nazi Bogey”. In {\sl The Koenraad Elst Site}. \url{http://koenraadelst.bharatvani.org/articles/politics/bogey.html}. Accessed on 15 May, 2016.

\bibitem[]{chap3_item16}
Ericksen, Robert P (2009) “Christian Complicity? Changing Views on German Churches and the Holocaust”. {\sl In Holocaust Encyclopedia} website. \url{https://www.ushmm.org/m/pdfs/20091110-ericksen.pdf}. Accessed on 20 May 2016

\bibitem[]{chap3_item17}
Fischer, Eugen;  Baur, Erwin and  Lenz, Fritz  (1921--1940) {\sl Principles of Human Heredity and Race Hygiene}. Berlin.

\bibitem[]{chap3_item18}
Ghosh, Kali Charan (1944) {\sl Famines in Bengal, 1770--1943}. Calcutta : Indian Associated Publishing Co.

\bibitem[]{chap3_item19}
Ghosh, Palash (26 Mar, 2012) “Nazi Germany’s Fascination With Ancient India: The Case Of Heinrich Himmler". In {\sl International Business Times}.\\ {\tt http://www.ibtimes.com/nazi-germanys-fascination-\break ancient-india-case-heinrich-himmler-214364}. Accessed 15 May 2016.

\bibitem[]{chap3_item20}
Grünendahl, Reinhold (2012) “History in the Making: On Sheldon Pollock’s “NS Indology” and Vishwa Adluri’s “Pride and Prejudice””. {\sl International Journal of Hindu Studies} 16, 2; pp. 189--257

\bibitem[]{chap3_item21}
“Herero and Namaqua genocide” (Last modified on 10 Jan, 2017) \url{https://en.wikipedia.org/wiki/Herero_and_Namaqua_genocide}. Accessed on 20 Jan 2017.

\bibitem[]{chap3_item22}
Herman, Edward S. and Chomsky, Noam (1988) {\sl Manufacturing Consent: The Political Economy of the Mass Media.} New York: Pantheon Books.

\bibitem[]{chap3_item23}
Hilberg, Raul (1961) {\sl The Destruction of European Jews.} New Haven: Yale University Press.

\bibitem[]{chap3_item24}
Hildebrandt, Sabine (2016) {\sl The Anatomy of Murder: Ethical Transgressions and Anatomical Science during the Third Reich.} Brooklyn, NY: Berghahn Books.

\bibitem[]{chap3_item25}
Horsman, Reginald (1981) “Race and Manifest Destiny: The Origins of American Racial Anglo-Saxonism” Cambridge: Harvard Univ Press

\bibitem[]{chap3_item26}
Janmohamed, Zahir (05 Dec, 2013) “U.S. Evangelicals, Indian Expats Teamed Up to Push Through Modi Visa Ban". In The New York Times. Accessed on 06 Aug, 2016.

\bibitem[]{chap3_item27}
Kevles, Danial  (1994) “Eugenics and the Human Genome Project: Is the Past Prologue”. In Timothy F. Murphy, Marc Lappé (Ed.) {\sl Justice and the Human Genome Project.} Berkeley: University of California Press. 

\bibitem[]{chap3_item28}
Kishwar, Madhu (2013) {\sl Modi, Muslims and Media.} Manushi Publications. New Delhi.

\bibitem[]{chap3_item29}
Madley, Benjamin (2004) “Patterns of frontier genocide 1803--1910: the Aboriginal Tasmanians, the Yuki of California, and the Herero of Namibia.” {\sl Journal of Genocide Research} (2004), 6(2), June; pp.~167--192.

\bibitem[]{chap3_item30}
Madley, Benjamin (2005) “From Africa to Auschwitz: How German South West Africa Incubated Ideas and Methods Adopted and Developed by the Nazis in Eastern Europe". {\sl European History Quarterly} July 2005 vol.~35 no.~3; pp.~429--464.

\bibitem[]{chap3_item31}
Madley, Benjamin (2016) “An American Genocide: The United States and the California Indian Catastrophe, 1846-1873”  New Haven: Yale Univ Press.

\bibitem[]{chap3_item32}
Malhotra, Rajiv (2009) “American Exceptionalism and the Myth of the Frontiers”. In {\sl Rajani Kannepalli Kanth: The Challenge of Eurocentrism: Global Perspectives, Policy, and Prospects.} NY: Palgrave Macmillan.

\bibitem[]{chap3_item33}
Malhotra, Rajiv (2014) {\sl Indra’s net.} Noida: HarperCollins.

\bibitem[]{chap3_item34}
Malhotra, Rajiv (2016) {\sl Academic Hinduphobia.} New Delhi: Voice of India.

\bibitem[]{chap3_item35}
Malhotra, Rajiv and Neelakandan, Arvind. (2011) {\sl Breaking India.} New Delhi: Amaryllis. 

\bibitem[]{chap3_item36}
Mukerjee, Madhusree (2011) {\sl Churchill's Secret War: The British Empire and the Ravaging of India During World War II.} NY: Basic Book

\bibitem[]{chap3_item37}
Nash, Vaughan (1900) “The Great Famine and Its Causes” London: Longmans Green

\bibitem[]{chap3_item38}
Neelakandan, Aravindan (Aug 09, 2016) “What We Knew About The Genome”. \url{http://swarajyamag.com/magazine/what-we-knew-about-the-genome}. 
Accessed on Oct 1, 2016

\bibitem[]{chap3_item39}
Nicholls, William (1993) {\sl Christian Antisemitism: A History of Hate}. Lanham, MD: Rowman and Littlefield Publishers Inc

\bibitem[]{chap3_item40}
Pinkerton, Alasdair (2008). “A new kind of imperialism? The BBC, cold war broadcasting and the contested geopolitics of South Asia". {\sl Historical Journal of Film, Radio and Television} 28 (4); pp.~537--555.

\bibitem[]{chap3_item41}
Popper, Karl (1934) The Logic of Scientific Discovery. Vienna: Springer

\bibitem[]{chap3_item42}
Pollock, Sheldon (2006) {\sl The Language of the Gods in the World of Men: Sanskrit, Culture and Power in Premodern India.} Berkeley: University of California Press.

\bibitem[]{chap3_item43}
“Pope Benedict XVI reflects on life under Hitler's Nazi Party” (31 May, 2011)  \url{http://www.catholic.org/news/international/europe/story.php?id=41597}. Accessed 01 Jul, 2016.

\bibitem[]{chap3_item44}
“Propaganda model” (Last modified on 13 Dec, 2016) \url{https://en.wikipedia.org/wiki/Propaganda\_model}. Accessed Jun 15, 2016.

\bibitem[]{chap3_item45}
Reilly, PR (1987) “Involuntary sterilization in the United States: a surgical solution”. {\sl The Quarterly Review of Biology.} Vol~62 No.~2; pp.153--70. 

\bibitem[]{chap3_item46}
Roelcke, V. (2002) “Mentalities and sterilization laws in Europe during the 1930s”. {\sl Der Nervenarzt.} 2002 Nov; 73(11); pp.~1019--30.

\bibitem[]{chap3_item47}
Satya, L. (1994) “Cotton and Famine in Berar, 1850--1900.” Ph.D. Dissertation, Boston: Tufts University.

\bibitem[]{chap3_item48}
Sofair, A N and Kaldjian, L C (2000), “Eugenic sterilization and a qualified Nazi analogy: the United States and Germany, 1930--1945”, {\sl Annals of Internal Medicine} 2000 Feb 15;132(4); pp.~312--9. 

\bibitem[]{chap3_item49}
Stannard, David E (1993) {\sl American Holocaust: The Conquest of the New World}. NY: Oxford University Press.

\bibitem[]{chap3_item50}
“Thodur Madabusi Krishna” (2016) \url{http://sandbox.mossesgeld.com/rmaf/awardees/thodur-madabusi-krishna-2/}.\\ Accessed on 06 Aug, 2016.

\bibitem[]{chap3_item51}
United States Holocaust Memorial Museum. “The German Churches and The Nazi State”. Holocaust Encyclopedia. \url{https://www.ushmm.org/wlc/en/article.php?ModuleId=10005206}. Accessed on 20 May 2016

\bibitem[]{chap3_item52}
“Völkisch movement” (Last modified on 7 Nov, 2016) \url{https://en.wikipedia.org/wiki/Völkisch_movement}. Accessed 15 May 2016.	

\bibitem[]{chap3_item53}
Walford, Cornelius (1878),'The famines of the world: past and present', {\sl Journal of the Royal Statistical Society}. No.~41, September 1878, pp.~433--526 and No.~42, March 1879, pp.~79--265.

\bibitem[]{chap3_item54}
Woodham-Smith, Cecil (1962) The Great Hunger: Ireland 1845--1849. NY: Harper and Row.
\end{thebibliography}


\theendnotes
