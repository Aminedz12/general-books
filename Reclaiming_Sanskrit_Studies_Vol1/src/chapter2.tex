\chapter[A {\sl pūrva-pakṣa} of “Deep Orientalism?”]{A {\sl\bfseries pūrva-pakṣa} of “Deep Orientalism?”}\label{chapter2}

\Authorline{Ashay Naik}
\lhead[\small\thepage\quad Ashay Naik]{}

\section*{Abstract}

This paper is an attempt at a {\sl prima facie “pūrva-pakṣa”} analysis of “Deep Orientalism?”\endnote{The question mark in the title of Pollock’s essay is pertinent and reflects its typically amorphous and suggestive style. Ideas are often expressed in the form of possibilities which would need to be substantiated by future research or are presumed to have been established by other scholars without specifying their argument. Grünendahl\index{Grunendahl@Grünendahl} (2012) who refers to these discourse strategies as “desideratum scheme”\index{misinterpretation,!techniques of,!desideratum scheme} and “feigned factuality"\index{misinterpretation,!techniques of,!feigned factuality} (Grünendahl\index{Grunendahl@Grünendahl} 2012:201), also notes derisively “the increasing tendency” in Pollock’s acolytes “to drop (metaphorically) the invertebrate question mark, as if Pollock’s amorphous presumptions\index{presumptions} had meanwhile coagulated into hard facts” (Grünendahl\index{Grunendahl@Grünendahl} 2012:187).} – an essay authored by Sheldon Pollock and published in {\sl Orientalism and Postcolonial Predicament (1993)} – an anthology of writings concerned with the issues facing Indian studies in the wake of the Orientalist critique – edited by Carol Breckenridge\index{Breckenridge, Carol} and Peter van der Veer\index{Veer, Peter van der}. The declared aims of Pollock’s essay are two-fold. Firstly, it seeks to analyse using the framework of Orientalism the collaboration between German Indology and the Nazi regime (1933-45), on the one hand; and the role of Sanskrit knowledge in the “pre-colonial forms of domination” in India, on the other. Secondly, in the light of the alleged realization - that both Indology itself, as well as its object of study (i.e. Sanskrit knowledge) have been deeply implicated in power - it proposes a critical Indology for the future that will be resistant towards imperialism, or at least not act in collaboration with it. Thus, there is a descriptive and a prescriptive component to Pollock’s essay. The purpose of this {\sl pūrva-pakṣa} is, firstly, to explain the comparative morphology of domination attempted by Pollock and expose its limitations; secondly, to interrogate the implications of the critical Indology proposed by Pollock for the study of the Sanskritic heritage. Although Pollock has not stated it explicitly as the objective of his essay, it is alleged by some that he seeks to establish a causal connection between Sanskrit thought and the forms of domination evident in colonialism\index{colonialism} and Nazism\endnote{Consider, for example, the section “Blaming Sanskrit for Nazism” in Malhotra (2016). Also, the editors of Orientalism and Postcolonial Predicament admit: “Sheldon Pollock [shows] that there are important family resemblances between precolonial Brahmanical discourse and Orientalist scholarship. Pollock argues that the German Orientalists took a feature of Brahmanical discourse, namely its distinction between Aryans and non-Aryans, as ‘civilized’ and ‘uncivilized’ respectively, and applied this distinction to their own society in their attempt to define the Jews as non-Aryan” (Breckenridge and Veer, 1993).}. The {\sl pūrva-pakṣa} will also investigate this allegation and discuss how the comparative morphology of domination makes such a reading ineluctable even if it may not be the express intent of the author himself. 


\section*{Introduction}

Pollock announces at the outset that his essay attempts to bring together German and Sanskrit knowledge-power nexuses\index{knowledge-power nexuses} within the framework of “Orientalism” – the prototypical knowledge-power nexus which interrogates the collaboration between British Orientalism and British colonialism\index{colonialism} – as a result of which they get projected as “Orientalisms” in their own right. And in conclusion, he reflects on the challenges and promises of a new Indology in the wake of the transcended “Orientalist” one, which, inasmuch as it will be driven by American scholars like Pollock, could as well be understood as American Orientalism\index{American Orientalism}. I have thus organized the {\sl pūrva-pakṣa} in terms of these four alleged forms of Orientalism – the British, the German, the Sanskrit and the American.

\section*{British Orientalism}

There was a time when the concept of Orientalism meant simply the study of Asian languages and cultures but Edward Said\index{Said, Edward} (1978) gave the term a polemical twist while yet retaining its original sense and this inversion has been the cause of great confusion\endnote{Phillip (2014) declares: “It must be said at the outset that few specialists in Middle Eastern or Asian studies have been persuaded by Said’s appraisal of scholarly endeavours by academic Orientalists. Indeed, his book has been found to be riddled with errors, flagrant omissions, and drastic overgeneralizations.” See Phillip (2014:208-09) for a bibliography of general critiques of Said’s Orientalism. An excellent resource on the topic is the chapter Said’s Orientalism: a book and its aftermath in Lockman (2004) which gives a summary of the book and the controversial exchanges which followed its publication.} which is evident in Pollock’s essay as well. 

\subsection*{Three Senses}\index{Orientalism, three senses of}

Said offers at least three “interdependent” meanings of Orientalism:
{\eleven
\renewcommand\theenumi{\alph{enumi}}
\renewcommand\labelenumi{(\theenumi)}
\begin{enumerate}
\itemsep=0pt
\item Anyone who teaches, writes about, or researches the Orient $\ldots$ either in its specific or its general aspects, is an Orientalist, and what he or she does is Orientalism.
\item Orientalism is a style of thought based upon an ontological and epistemological distinction made between “the Orient” and (most of the time) “the Occident.”
\item Taking the late eighteenth century as {\sl a very roughly defined starting point} Orientalism can be discussed and analysed as the corporate institution for dealing with the Orient – dealing with it by making statements about it, authorizing views of it, describing it, by teaching it, settling it, ruling over it: in short, {\sl Orientalism as a Western style\index{Indology!as a Western discipline} for dominating, restructuring, and having authority over the Orient}. (Said\index{Said, Edward} 1978: 2-3, italics mine)
\end{enumerate}}
Right off the bat we can see that in the first two meanings we are talking about Orientalism as general and trans-historical, while in the third it comes across as specific and historical. The former is purely epistemological and covers the whole gamut of what has been said by the West about the East as an entity different from itself right from the time of the ancient Greeks. The latter, on the other hand, is heavily implicated in power\index{power!political}, sponsored by power\index{power!political} and intended to secure and expand power\index{power!political}. 

Said has not distinguished between these two processes of knowledge production and this apparent oversight is what lies, in my view, at the heart of all the controversy surrounding the concept of Orientalism. 

\subsection*{Three Scopes}

Corresponding to the three meanings of Orientalism, Said also provides three different scopes for the knowledge which can be considered Orientalist\index{Orientalism, three senses of}. Thus, at one point he says:
\begin{myquote}
Strictly speaking, Orientalism is a field of learned study. In the Christian West, Orientalism is considered {\sl to have commenced its formal existence} with the decision of the Church Council of Vienne\index{Church Council of Vienne} in 1312 to establish a series of chairs in “Arabic, Greek, Hebrew, and Syriac at Paris, Oxford, Bologna, Avignon, and Salamanca” \hfill(Said 1978:49-50, italics mine).
\end{myquote}

This agrees well with the first meaning of Orientalism given above. Further, with regards to the second meaning of marking a distinction between the Orient and the Occident, he says:
\begin{myquote}
A very large mass of writers $\ldots$ have accepted the basic distinction between East and West as the starting point for elaborate theories, epics, novels, social descriptions, and political accounts concerning the Orient, its people, customs, “mind,” destiny, and so on. This Orientalism can accommodate Aeschylus, say, and Victor Hugo, Dante and Karl Marx.~\hfill(Said 1978:2--3)
\end{myquote}
\newpage

This suggests that even ancient Greek works such as produced by Aeschylus can be regarded as Orientalist. On the other hand, with regard to the third meaning of Orientalism\index{Orientalism, three senses of} as a post-eighteenth century phenomenon linked with colonialism\index{colonialism}, he says:
\begin{myquote}
Britain and France were {\sl the pioneer nations in the Orient and in Oriental studies} $\ldots$ [and] these vanguard positions were held by virtue of the two greatest colonial networks in pre-twentieth-century history~\hfill(Said~1978:17,~italics~mine).
\end{myquote}
\smallskip

When we consider Said’s assertion that all the three meanings are “interdependent” it becomes evident that {\sl all} the knowledge which Europe has produced about the Orient, from the beginning of its history, is Orientalist in his view i.e. tainted by a lust for domination and asseverated from a position of superiority. I think he has concentrated specifically on British and French Orientalism in the colonial period because (to use Pollock’s own eloquent words) “it offers an extreme and often transparent instance of knowledge generating and sustaining power and the domination that defines it” (Pollock 1993:77). In addition, it connects seamlessly with American Orientalism\index{American Orientalism} which – in my view – is the real target of his book\endnote{Anglo-French Orientalism\index{Anglo-French Orientalism} also provides a neat segue to American Orientalism\index{American Orientalism}, which is the main adversary of the Arab world today. Said\index{Said, Edward} was an activist for the Palestinian cause and {\sl Orientalism} belonged to a phase when “deepening political engagement in the 1970s led him to criticize the ways in which Arabs and Muslims were often depicted in the Western media” (Lockman, 2004:183). His other books in the same period addressed “the traumatic dispossession, subordination and ongoing suppression which the Palestinians had experienced at the hands of Zionism and Israel\index{Israel}” and “what Said\index{Said, Edward} saw as distorted and pernicious US media coverage of the Iranian revolution of 1978–79 and its aftermath, and of the threat which Islam allegedly posed to the United States” ({\sl ibid}). It thus appears to me that the real target of {\sl Orientalism} was actually American Orientalism, and Anglo-French Orientalism\index{Anglo-French Orientalism} provided merely the foundation for the polemic. “From the beginning of the nineteenth century until the end of World War II France and Britain dominated the Orient and Orientalism; since World War II America has dominated the Orient, and approaches it as France and Britain once did” (Said\index{Said, Edward}, 1978:4).}. But it is evident that for Said, Orientalism is an essential feature of European knowledge production about Asia as he links it with the very character of European culture:
\begin{myquote}
It is hegemony, or rather the result of cultural hegemony\index{cultural!hegemony} at work, that gives Orientalism the durability and the strength $\ldots$ Orientalism is never far from $\ldots$ the idea of Europe, a collective notion identifying “us” Europeans against all “those” non-Europeans, and indeed it can be argued that the major component in European culture is precisely what made that culture hegemonic both in and outside Europe: the idea of European identity as a superior one in comparison with all the non-European peoples and cultures. There is in addition the hegemony of European ideas about the Orient, themselves reiterating European superiority over Oriental backwardness\index{Oriental backwardness}.\hfill (Said 1978:7)
\end{myquote}

This detour into Said’s {\sl Orientalism} was necessary to clear up some of the confusion that confronts us in Pollock’s essay right at the beginning. Pollock (1993) has understood Orientalism only in terms of its third meaning given above i.e. as a process of knowledge-production necessarily connected with colonialism\index{colonialism}. He has therefore felt it necessary to make adjustments to this concept so that it can include German Orientalism and Sanskrit\index{power!Sanskrit as a source of} knowledge, neither of which were involved in colonialism\index{colonialism}. In the case of the former, he suggests that its vector should be conceived as directed inwards “towards the colonization\index{colonization} and domination of Europe itself” (Pollock 1993:77) while, to accommodate the latter, Orientalism should be understood simply as a “discourse of power that divides the world into ‘betters and lessers’ and thus facilitates the domination $\ldots$ of any group” {\sl (ibid)}. Having thus over-generalized (and consequently trivialized) the scope of Orientalism to include knowledge produced in the service of any and every kind of domination, he suggests that “Orientalist constructions in the service of colonial domination may be only a specific historical instance of a larger, transhistorical, albeit locally inflected, interaction of knowledge and power” (Pollock 1993:76).

These two adjustments suggested by Pollock to the concept of Orientalism – the multi-directionality of its vector and its trans-historicization\index{trans-historicization} to include any discourse about “betters and  lessers”– are both misleading in my view, and arise from his failure to understand Said’s\index{Said, Edward} thesis, as outlined above. This is because Orientalism, according to Said\index{Said, Edward}, was always trans-historical and yet exclusively European. While Said is typically ambivalent about his exclusion of German Orientalism\endnote{Pollock (1993:118) claims in a footnote that Said\index{Said, Edward} has acknowledged the omission of German Orientalism in his essay {\sl Orientalism Reconsidered}\index{Orientalism Reconsidered@\textsl{Orientalism Reconsidered}} (1985). This is false. Quite to the contrary, he specifically informs his critics that “[problems] like my exclusion of German Orientalism, which no one has given any reason for me to have included – have frankly struck me as superficial or trivial, and there seems no point in even responding to them” (Said\index{Said, Edward} 1985:90).}, his central point was: “German Orientalism had in common with Anglo-French and later American Orientalism\index{American Orientalism} $\ldots$ a kind of intellectual {\sl authority} over the Orient within Western culture” (1978:19, italics original). Ultimately Orientalism is all about the {\sl authority} which the West has wielded against the East right from the very start, as a result of which, in Said’s view, the whole corpus of European knowledge about the Orient stands implicated as Orientalist. 

Pollock’s fundamental error, as of most readers of Said, is that they focus on the connection forged by the latter between Anglo-French Orientalism and Anglo-French colonialism\index{colonialism} and then wonder what to make of the Orientalisms of other European nations whose colonial enterprise was relatively limited. But one doesn’t have to go looking for the {\sl real} subject peoples of a given Orientalism such that if one cannot be found externally, then one would need to be conceived of internally, as Pollock has done in the case of German Orientalism. For Said, Orientalism was already a trans-historical European phenomenon. The Anglo-French colonialism\index{colonialism} was only a local inflection (to borrow Pollock’s term), a historical moment when the sense of domination inherent in Orientalism came into contact with real domination in the form of colonialism\index{colonialism}. But what Said\index{Said, Edward} is telling us is that the sense of domination over the Orient was always there right from the time when Europe conceived of itself as different and as superior to the Orient. The actualization of this hegemony in the form of Anglo-French colonialism\index{colonialism} only provides a fruitful ground for a study of how it shaped real domination but it does not need colonial actualization for its existence because it has always been there as “the major component in European culture.” When Germans began their study of the Orient, they did it in the same Orientalist way as the English and the French. They also objectified an ‘Orient’ and asserted their superiority over it by claiming their ‘knowledge’ of it as authoritative. This is what, according to Said\index{Said, Edward}, makes their scholarship Orientalist (in the polemical sense).


Given the exclusively European connotations of Orientalism, it cannot be universalized to include all discourses about “betters and lessers” and thus extended to Sanskrit knowledge. Said\index{Said, Edward} (1989:208) himself has used this phrase specifically to refer to the divide that yet separates the colonized\index{colonization} from their erstwhile colonizers even after the end of colonialism\index{colonialism}. Its over-generalization leads not only to loss of historical specificity (for which Pollock duly apologizes in the essay) but even more importantly, cultural specificity – the hegemony of European epistemological categories. This is not to suggest that discrimination is not to be found in Sanskrit knowledge but it would have to be understood in the specific context of Indian culture. To regard it as Orientalist is to make the error that the knowledge, and its effects, produced by a {\sl brāhmaṇa} out of a sense of superiority over the {\sl śūdra} or the {\sl mleccha}, are analogous to the knowledge, and its effects, produced by a European out of a sense of superiority over the Oriental.

None of the foregoing – I should concede in the end – should be read as a defence of Said’s\index{Said, Edward} {\sl Orientalism} from Pollock’s distortion. My point is simply that if one does use the Saidian thesis as a point of departure – as Pollock clearly does – then one cannot make the kind of adjustments he suggests, to attain the goals which he desires. In my view, both Said’s {\sl Orientalism}(1978) and Pollock’s {\sl Deep Orientalism?}(1993) suffer from the same pathology of a hermeneutic of suspicion taken to an excess. As Marchand\index{Marchand, Suzanne} (2009: xxvi) has expressed it passionately in her defence of German Orientalism against the Saidian thesis:
\newpage

\begin{myquote}
I find presumptuous and rather condescending the conception $\ldots$ that all knowledge is power, especially since the prevailing way of understanding this formulation suggests that power is something sinister and oppressive, something exerted against or over others. Of course, knowledge can be used in this way, but knowledge as understanding can also lead to appreciation, dialogue, self-critique, perspectival reorientation, and personal and cultural enrichment. Oriental studies did partake of and contribute to the exploitation and “othering” of non-westerners $\ldots$ but it also has led to positive outcomes $\ldots$ and I cannot subscribe to a philosophical stance that suggests that such things do not motivate or characterize the pursuit of knowledge.
\end{myquote}
\vskip -40pt


\section*{German Orientalism}\index{German Orientalism}
\vskip -3pt

German Indology is depicted by Pollock as having  proceeded along two lines – a romantic quest for identity and a scholarship that privileged scientific methods {\sl (Wissenschaft)\index{Wissenschaft@\textsl{Wissenschaft}}} – which interpenetrated in the period of Nazi  Germany to become its intellectual foundation. He then offers some particular instances of contributions made by Germany Indologists such as Walther Wüst and Erich  Frauwallner and the investments made by Nazi Germany in their field of study. All the data provided by Pollock to construct this narrative has been disputed by Grünendahl\index{Grunendahl@Grünendahl} (2012); and to avoid prolixity, I will not repeat his meticulous analysis here but only point out his conclusion which is that “Pollock’s master narrative $\ldots$ rest[s] on unsubstantiated claims, presumptions\index{presumptions} and misrepresentations”  (227). Instead, I will pursue a different strategy.

The connection Pollock has attempted to forge between German Indology and Nazi Germany betokens the methodological problem of his essay. Two diverse strands are here brought into juxtaposition with each other to erect the  façade of mutual reinforcement and affirmation without paying much attention to the larger whole of which they were constituent parts. This approach suffers from the problem of missing the forest for the trees by its exclusive focus on the mere trees – the inter-relationship between German Indology and Nazi Germany. However, when we do switch our attention to the forest, we realize how vast and complicated it is such that the role of German Indology becomes hardly discernible, which explains why it gets barely a mention in the historiography\index{historiography} of this period,  notwithstanding its colossal size, given the epochal consequences of Nazi Germany. It is impossible for me to contextualize within the limits of this paper in detail either German Indology or Nazi Germany; and what follows is merely a threadbare account of the reflection of various experts on this subject but hopefully it should be enough to put their interrelationship in a proper perspective.

We will focus on two areas which are of central concern in Pollock’s essay: the discourse about “betters and lessers,” which in this case takes the form of a dichotomy between Indo-German Aryan\index{Aryan} and Semite, and the inward vector of its domination which makes it “Orientalist.” First of all, we may note that Thomas Trautmann\index{Trautmann, Thomas} (1997:211-16) and Edwin Bryant\index{Bryant, Edwin} (2001:60-2) have persuasively argued against Max Müller’s interpretation of the Dasas/Dasyus\index{Dasas/Dasyus} as a nose-less, black-skinned people racially different from the Aryans. As Marchand\index{Marchand, Suzanne} points out:

\begin{myquote}
Sheldon Pollock’s argument in his essay “Deep Orientalism” that some of the racism in the term “Arya” was already imbedded in ancient Indian texts\index{texts (Sanskrit/Indic)} has been modified by Bryant and Trautmann, who have shown just how much “reading in” was necessary to wrench a racial contest between higher, lighter people, and lower, darker ones out of the ancient Indian texts”\hfill (Marchand 2009:128). 
\end{myquote}

On the other hand, antisemitism\index{anti-Semitism} has been a characteristic feature of European culture since ancient times. What we find in the modern period is only the rationalist and racist variety of an antisemitism\index{anti-Semitism} which had prevailed in Europe in the Christian and pagan times as well. The roots of modern antisemitism\index{anti-Semitism} lie not in German Indology but in the views of Enlightenment philosophers such as Voltaire\index{Voltaire}, whose ire against Judaism\index{Judaism} was a subset of his antipathy towards the Church and it is this dissatisfaction with Semitic\index{Semitic} religion which led to his projection of Vedic thought as a viable alternative and of the mythic Aryan Brahmin\index{Aryan!Brahmin} as its ideal practitioner, who was a foil against both the “plagiarist” Jew and the “degenerate” Indian (Figuera\index{Figuera, Dorothy} 2002:10ff). While Christianity was denounced, antisemitism\index{anti-Semitism} survived because its pagan variant was recalled through the discovery of Greco-Roman knowledge. 
\begin{myquote}
By bringing classical antisemitism\index{anti-Semitism} into the post-Christian rationalist thought of the eighteenth and nineteenth centuries, Voltaire enabled it to be grafted on to medieval Christian stereotypes, providing a new, international, secular anti-Jewish rhetoric\index{anti-Jewish} in the name of European culture.\hfill(Hellig\index{Hellig, Jocelyn} 2003:271)
\end{myquote}
\medskip

The cause of a German nationalism fostered by Johan Gottfried Herder’s\index{Herder, Johan Gottfried} idea of a {\sl Volksgeist\index{Volksgeist@\textsl{Volksgeist}}} (a national spirit or genius) and the alleged superiority of Aryan languages\index{Aryan!languages} over Semites, and hence of their speakers, as proposed by German Indologists such as Friedrich Schlegel\index{Schlegel, Friedrich von} and Christian Lassen\index{Lassen, Christian} were important but not the only factors favouring antisemitism\index{anti-Semitism} in this period. The French theologian Ernest Renan\index{Renan, Ernest} divested Jesus of his Jewishness and made him an Aryan\index{Aryan} figure. He declared himself to be the first at having recognized the superiority of the Indo-European over the Semite race (Hellig\index{Hellig, Jocelyn} 2003:276). Another Frenchman, Gobineau contributed significantly to the development of race theory but was not himself anti-Jewish\index{anti-Jewish}. His work, however, proved influential to the promotion of German antisemitism\index{anti-Semitism} through Richard Wagner\index{Wagner, Richard} and Houston Chamberlain\index{Chamberlain, Houston}. Likewise, with “the publication of Darwin’s ideas in 1859 $\ldots$ the biologist Ernst Haeckel\index{Haeckel, Ernst} $\ldots$ became Germany’s chief apostle of social Darwinism popularizing Darwin’s ideas by applying them to the development of civilization” (Hellig\index{Hellig, Jocelyn} 2003:280-1). Thus, we find that in the nineteenth century German Indology was only one of diverse fields – nationalism, race theory, Christian historical criticism, Darwinism – which were developing independently all over Europe and nourishing German antisemitism\index{anti-Semitism} in their own way. Further, German Indologists were not the only “bad” guys: “contributions to racist antisemitism came from many sources – professors, journalists, clergymen, statesmen, philosophers – and in Germany the antisemitic movement took to the streets, reaching a climax in 1881” (Hellig\index{Hellig, Jocelyn} 2003:271). 

Pollock mentions in passing that the dichotomization between Indo-German Aryan\index{Aryan} and Semite was “called into being by the social and economic emancipation of Jews” (Pollock 1993:82) but immediately suggests that it was “Orientalist” knowledge which made it possible. Thus, he ignores the problems relating to Jewish emancipation\index{Jewish!emancipation} as the cause of antisemitism. Jewish emancipation refers to the process of integrating Jews into the wider Christian society which occurred in light of the egalitarian ideals of Enlightenment. Previously, they held the status of tolerated aliens living on the fringes of society as self-governing communities. But emancipation carried the cost of relinquishing their Jewish particularities and adopting the cultural norms of Enlightenment society (Hellig\index{Hellig, Jocelyn} 2003:256ff). While the Jews did enter into public life and began to occupy prominent positions as equal citizens, they were unwilling or unable to cease being a distinctive community unto themselves – a nation within a nation – which was unacceptable, especially in Germany where nationalism came to be shaped not through fulfilment of rational obligations, but through participation in the Romantic ideals of the inner spirit of the {\sl Volk}. The theorization of a racial divide between Indo-German Aryans\index{Aryan} and Semites in this period was problematic not so much for the “Orientalist” inferiorization\index{inferiorization} of the latter – since in European history the Jews have always been an inferiorized people. It is only when this inferiorization took on a racial colour i.e. when the inferior qualities attributed to the Jews came to be seen as their immutable racial traits\index{racial!traits} that it became evident that emancipation could locate them within German society as equal citizens but never integrate them as such. These two limitations – adherence to their unique cultural identity on the part of the Jews and racial antisemitism on the part of the Germans – made their expulsion from the body politic seem necessary to Germans such as Hitler\index{Hitler!Adolf}; and when no nation showed willingness to grant them refuge\endnote{“Annihilation\index{annihilation} was arrived upon only after the failure of other methods, such as boycott, emigration, legislation and Aryanisation\index{Aryanisation}. Proponents of the Functionalist\index{functionalist} view argue, for example, that the Nazis\index{holocaust!Nazi} were content to be rid of the Jews through emigration until as late as 1941. The refusal of most of the rest of the countries of the world to take in the Jews gave Hitler\index{Hitler!Adolf} the green light” (Hellig 2003:44).}, total annihilation\index{annihilation} became the final solution of the Jewish question. One way to understand this problem of Jewish emancipation is to note that the earlier – Christian – understanding of the Jews, instituted a division between “betters and lessers”; and this, preposterous as it may sound, would have to be read as a kind of “Orientalism,” if we follow Pollock’s definition of the term. On the other hand, Jewish emancipation was an attempt to make them a “better”\endnote{As Hellig (2003:258) puts it so eloquently: “For Jews to obtain the ‘ticket of entrance’ to gentile society, as the young Heinrich Heine put it, they had to be made ‘better’ in some way.”} people; and what we discern is that such betterment comes at a steep cultural cost, which, if one is not willing to fully pay, transforms one from a tolerated, albeit inferiorized, outsider into an “abject” insider\endnote{The term “abject” in apposition to “subject” and “object” is borrowed from Kristeva. For its application to the Jews in modern Germany, see Fitzpatrick (2008:477).}; one who therefore comes to be seen as so dangerous as to be considered deserving of total annihilation\index{annihilation}. These are the kind of complexities involved in the inter-relationships of human groups which get glossed over by Pollock’s ridiculous adaptation of Orientalism.

Pollock has referred to “the legitimation of genocide”\index{genocide!legitimation of} as the “ultimate ‘Orientalist’ project”\index{orientalist project} (Pollock 1993:96) but as the foregoing paragraphs show “Orientalism” cannot even be regarded as a primary factor in the rise of modern antisemitism, which in turn cannot be regarded as a sufficient condition for the Holocaust. Pollock only pays lip service to the difference between Functionalist and Intentionalist\index{intentionalism/intentionalist} (Pollock uses the term “idealist”) approaches in Nazi\index{holocaust!Nazi} historiography\index{historiography} (1993:88). His contention that Intentionalist views “seem to have gained at least parity in current re-thinking in the historiography\index{historiography} of the movement $\ldots$ in part attributable to the fuller history of the Holocaust now available” {\sl (ibid)} is mere hand-waving since he would like to pin the blame fundamentally on the “ideational dimension” of Orientalist knowledge. He is careful not to inform his readers that the Functionalist\index{functionalist} historians have argued that Hitler\index{Hitler!Adolf} was a lazy and weak dictator, driven by historical circumstances, and that war and genocide\index{genocide} were the result of the “cumulative radicalization”\index{cumulative radicalization} arising from the chaotic and competitive nature of the Nazi\index{holocaust!Nazi} state apparatus rather than from a specific ideology (Mason 1995:212ff). 

Matters get even more complicated as precedents to Nazi atrocities\index{Nazi!atrocities} can be found not only in the pogroms\index{pogroms} of the Soviet {\sl kulaks}\index{Soviet kulaks} and the Armenians, but in the Herero\index{Herero} and Nama genocides\index{genocide!Nama} (1904--07) perpetrated by Germany itself in its South-West African colonies. There is, first of all, the controversy known as {\sl Historikerstreit} about whether the Nazi period (1933--45) should be cordoned off as a unique event in German history lest its historicization leads to the normalization and trivialization of the Holocaust, or whether it should be viewed as the end product of a way of modernization that was unique (Sonderweg\endnote{“The Sonderweg thesis, as the name suggests, emphasized that nineteenth- and early twentieth-century Germany suffered the stain of a uniqueness stemming from a maimed path to modernity, a path via which Germany became not another England, the putatively normative successful nation, but a ``conservative utopia" in which liberalism was only weakly anchored. Ineluctably, this narrative suggested, by reaching its historic turning point and ``failing to turn,” Germany missed the path to modernity and was only shocked into becoming a “normal” liberal state by undergoing the purgative rigours of National Socialism.” (Fitzpatrick\index{Fitzpatrick, Matthew P} 2008:481). The thesis has been disputed in {\sl The Peculiarities of German History} by Blackbourn and Eley {\sl (ibid)}.}) to Germany and different from other European nations. There is further controversy on whether the German colonial genocide\index{genocide} is to be causally connected with the Holocaust or not and whether this connection, if accepted, should be treated as supporting the Sonderweg thesis or not. Needless to say, in none of these historiographies does German Orientalism figure anywhere at all. Fitzpatrick\index{Fitzpatrick, Matthew P} (2008) who has done an excellent survey of these various positions, concludes that European colonialism\index{colonialism} produced initially a hierarchical racism based on socio-political differences but as miscegenation tended to blur the difference between ruler and ruled, a biological racism\index{biological racism} arose in the colonial milieu which treated the Mischlinge\index{Mischlinge} (mixed blood) as the abject entity which polluted the self. It was this biological racism, which developed in the context of colonialism\index{colonialism}, which was exported back to Europe. While all European colonial nations were equally affected by this problem, the German case was peculiar – not necessarily due to Sonderweg – but because its defeat in the first World War brought its colonial adventure to an abrupt halt:
\medskip

\begin{myquote}
With colonialism\index{colonialism} suppressed by the victorious French and British, it was transformed into a vehicle for a hypertrophic expansionist nationalism that sought internal as well as external grounds for the catastrophic failure of the German nation-state to maintain parity with or hegemony over other European powers $\ldots$ Upon its return to Europe, expansionism\index{expansionism} predicated on racial difference was fundamentally altered, with “inner,” biological categories of difference inferred in a colonial discourse tailored for Europe, where external markers of racial difference were not apparent and would therefore not serve the purpose of imperial social stratification\index{social stratification}. New biological categories were consolidated – a biologically inferior alterity – the “Asiatic” Slav, and an abject, polluting, debased German self, the biologically deficient but nonetheless assimilated “Germanic Jew” $\ldots$ One was to be conquered and ruled, even at the risk of a war of annihilation\index{annihilation}. The other was simply to be eliminated ruthlessly, expelled from the body politic

\hfill(Fitzpatrick\index{Fitzpatrick, Matthew P} 2008:500-1).
\end{myquote}

This is exactly the same as the inward vector of German domination specified by Pollock but the various scholars whose views Fitzpatrick has summarized – Jürgen  Zimmerer, Isabel Hull, Benjamin Madley, Pascal Grosse,  Hannah Arendt, and others – have traced the problem to German colonialism\index{colonialism!Nazi/German} and its military competition with  other European powers, rather than to German Orientalism. What becomes evident from the foregoing is that the history of modern Germany which forms the overarching context of German Indology is a highly complex subject and a myriad factors have contributed to the rise of Nazi Germany and its agenda of war and genocide\index{genocide}. By completely ignoring this context and fixating upon German Indology alone and the collaboration between some German Indologists and the Nazi regime, Pollock misleads his readers into ascribing it an exaggerated role in the history of Nazism. But maybe that is his goal – not so much to incriminate German Indology itself as the Sanskrit knowledge which was its object of study.

Pollock (1993:83) suggests that in the Nazi period, the two limbs of German Orientalism viz. romanticism and {\sl Wissenschaft}, merged to become the official worldview of the state and “fostered the ultimate ‘Orientalist’ project\index{orientalist project}, the legitimation of genocide.” What made this interpenetration possible was, according to him, the conceptual framework of a facts-based, value-free, objective scholarship  which sought to transcend political values: 

\begin{myquote}
It offers a superb illustration of the empirical fact that disinterested scholarship in the human sciences, like any other social act, takes place within the realm of interests; that its objectivity is bounded by subjectivity; and that the only form of it that can appear value-free is the one that conforms fully to the dominant ideology, which alone remains, in the absence of critique, invisible as ideology
\hfill(Pollock 1993:96).
\end{myquote}
\medskip

This denunciation of objective scholarship and the corresponding valorisation of a “morally sensitive scholarship” shows that for Pollock it is not so much important for research to be  evidence-based as it should be engaged in the politics of the underdog such as “giving priority to what has hitherto been  marginal, invisible and unheard” (Pollock 1993:114). This view, which suggests that knowledge is ineluctably political, and  therefore necessarily biased, as a result of which morality demands that it should be employed to favour the weak, explains the motivation behind some of his other works such as ‘political philology’\index{political!philology} as well as points to the fundamental weakness in his overall scholarship.
\vskip 2pt

First of all, there is the contrary opinion that knowledge is not necessarily political – passionately articulated by Marchand (2009: xxvi) as stated above. Secondly, and more importantly, in the specific context of {\sl Wissenschaft}, Pollock’s remonstrations against objective scholarship make no sense at all. Romanticism, which arose out of a sense of alienation and loss, prioritized emotion or will over reason, which was advocated by the Enlightenment. Objective scholarship essentially means that reason should be independent of will. If reason became subservient to will under the Nazi regime, as is evident from its promotion of “pseudo”-science aimed towards the achievement of preconceived results, it was not because reason was not aware of the power of will, as Pollock suggests, but because it was self-consciously brought into subservience of the will by the Nazi regime. Hitler\index{Hitler!Adolf} belaboured that the German antipathy towards the Jews was emotional and needed to be understood scientifically. Accordingly, he advocated a “scientific antisemitism”\index{Hitler! scientific antisemitism@“scientific antisemitism”} or an “antisemitism of reason” (Steinweis 2006:7). To put simply, Hitler’s point was that “will” remains ineffective unless reason serves as its instrument. It is interesting to note that Pollock is making the same point albeit on the ground that knowledge is ineluctably political i.e. to say, reason is necessarily subservient to will and so should not pretend to take an independent stand. Furthermore, Hitler too would claim that he was advocating a “morally-sensitive scholarship”\index{morally-sensitive scholarship} since, from his perspective, he was only trying to save Germany from being destroyed from within by the “Jewish bacillus”\index{Jewish!bacillus} (Fitzpatrick\index{Fitzpatrick, Matthew P}, 2008:477).

In conclusion, the superficial manner in which Pollock has dealt with the subject of German Indology and the Nazi regime – raising the issue of an allegedly objective scholarship and lengthy pronouncements by some Indologists sympathetic to the Nazi cause – suggests to me that he is not genuinely interested in understanding this complex and tragic chapter in German history whose many dimensions I have sought to outline above. Instead, it appears that the objective is to use German Orientalism as a bridge between British and Sanskrit Orientalism whose affinities are otherwise not at all sensible given that the former was engaged in the study of colonized people for the purpose of colonial rule, while the latter was nothing of the kind. The aim of projecting German Orientalism as knowledge sponsored by a state for the domination, oppression and extermination\index{extermination} of its own citizens, facilitates the conception of Sanskrit knowledge along similar lines and thus makes it amenable to a theorization as an epistemological tool meant for political ends.

\section*{Sanskrit Orientalism}

In order to theorize Sanskrit knowledge as a form of “pre-Orientalist Orientalism” collaborating with a “pre-colonial colonialism\index{colonialism},” Pollock (1993:96 ff.) disputes the post-colonialist view that Brahmanical texts\index{texts (Sanskrit/Indic)} were elevated to a legal authority under colonial rule and their formulations thus acquired an unprecedented hegemonic form. His contention is that such a transformation had already occurred in the production of the commentaries and digests on the {\sl dharma-śāstra-s}\index{dharma@\textsl{dharma}!\textsl{-śāstra}} – the {\sl dharma-nibandha}\index{dharma@\textsl{dharma}!\textsl{-nibandha}} genre – during the eleventh and twelfth centuries CE\@. He specifically points out the Hindu rulers of this period who sponsored these works and refers to their authors as Indian “Orientalists” (Pollock 1993:98). He attempts to rationalize this cultural production by suggesting that it occurred as a “special reaffirmation of {\sl dharma}”\index{dharma@\textsl{dharma}} in response to the Turkish invasion. He outlines the various restrictions imposed on the {\sl śūdra-s} in the {\sl dharma-śāstra-s}, including denial of Vedic knowledge, as an instrumental use of knowledge for the purpose of domination.
\newpage

The fact that the {\sl dharma-śāstra-s}\index{dharma@\textsl{dharma}!\textsl{-śāstra}} conceive of people belonging to different {\sl varṇa-s}\index{varna@\textsl{varṇa}} as consisting of different essential natures and assigns them, accordingly, different sets of privileges and obligations is common knowledge. But to reduce this schema to a mere form of domination without supplying the necessary context – as Pollock attempts to do – becomes just as problematic as in the case of German Orientalism, explained in the previous section. As far as the knowledge aspect is concerned, one is required to study the {\sl dharma-śāstra-s} holistically, as say Kane or Lingat\index{Lingat, Robert} has done, and if the dimension of power is of interest, then it would be necessary to consider its competition with other sources of power, mainly derived from custom, in its attempt to transform itself into law – the instrument of legitimate coercion in society. However, just as in case of his study of German Orientalism, Pollock has done neither of the two. On the one hand, he remains fixated on the apparent iniquities of {\sl śūdra-dharma}, while on the other, he considers random statements of {\sl varṇa} hierarchy as “dreams of power” with effective bearing on reality by the mere virtue of their having been dreamt (Pollock 1993:102-3).
\vskip 2pt

It was the British judges – in 19$^{\rm th}$ century India - who faced the full brunt of the problem when they allocated a juridical character to the {\sl dharma-śāstra-s}. Undoubtedly the pronouncements of the {\sl dharma-śāstra-s} were meant for gain of spiritual merit but to what extent did they acquire the force of law such that they could assume a form of domination in pre-colonial history? It would appear that even pronouncements that were juridical in character i.e. pertaining to {\sl vyavahāra}, carried a dharmic rather than a legal force. Having assessed the debates on the issue between the colonial judges, Lingat (1973) concludes:
\smallskip

\begin{myquote}
The written law of the śāstras and the customary laws of the different groups of humanity thus existed side by side, equally respected though often in notable disagreement with each other. The former acted upon the latter and restricted its mobility; but the latter also acted upon the former through the medium of interpretation. The result was an extremely variable and diverse law\hfill(Lingat~1973:141-2).
\end{myquote}
\smallskip

The pre-colonial Indian judge, then, played the role of an arbitrator “who had the power to apply as the case demanded the law of the holy {\sl ṛṣi-s} or the custom of ancestors” (ibid). Given its coercive nature, if law is one of the best indexes of the structure of power in society then we can see that it was a contested issue between the {\sl dharma-śāstra-s}\index{dharma@\textsl{dharma}!\textsl{-śāstra}} and the customs of various local groups.

As mentioned above, in order to offer a precedent for the changes which occurred in Hindu Law in the colonial period, Pollock refers to the composition of the {\sl dharma-nibandha}-s\index{dharma@\textsl{dharma}!\textsl{-nibandha}} as a sort of “pre-Oriental Orientalism” in the face of the Turkish invasion. However, systematization of the {\sl dharma-śāstra} texts\index{texts (Sanskrit/Indic)} in the form of commentaries and digests began in the ninth century with the famous gloss of Medhātithi\index{Medhatithi@Medhātithi} on the {\sl Manusmṛti}\index{Manusmrti@\textsl{Manusmṛti}} and not in response to any foreign invasion\endnote{The commentaries ({\sl bhāṣya})\index{bhasya@\textsl{bhāṣya}} and digests ({\sl nibandha})\index{nibandha@\textsl{nibandha}} can be understood as narratives in the same genre. Pollock himself makes no attempt to distinguish between the two. The differencem is only that the digests arranged their material topically while the commentaries based themselves on a particular {\sl dharmaśāstra} but they too included materials from other {\sl dharmaśāstras}\index{dharma@\textsl{dharma}!\textsl{-śāstra}} and sought to reconcile between them as they approached these master texts\index{texts (Sanskrit/Indic)} as a single corpus (see Lingat\index{Lingat, Robert} 1973, 107 ff).}. According to Lingat (1973:143-4) it was probably an attempt at rediscovery after a hiatus had passed since the composition of the last {\sl dharma-śāstra}-s.

In any case, these two events – the production of the {\sl dharma-nibandha}-s and the production of Indological works in the colonial period – in different times and under radically different circumstances, and in fact authored by different people – the Indians in the first case and the Europeans in the latter – can hardly be comparable. True, both involved scriptural study and validation, and both were sponsored by powers ruling in India, but that is only a superficial comparison. In the nineteenth century, we know that Eurocentric\index{Eurocentric} and Christocentric frameworks\index{Christocentric frameworks} were used in the study of Indian scriptures for the purpose of colonization and proselytization. Further, the interpretations were explicitly intended to be of juridical value to enable the colonial government to rule its Hindu subjects. On the other hand, {\sl dharma-nibandha} interpretation was of an exegetical nature, using the principles of Mīmāṁsā\index{Mimamsa@Mīmāṁsā} whose purpose was clearly to elucidate {\sl dharma} rather than law.


When it comes to Sanskrit knowledge, Pollock repeatedly uses the term “pre-colonial forms of domination” for the real nature of social power in ancient India is not known with certainty. In fact, a “detailed topography of pre-colonial domination” (1993:104) is a desideratum in his view but yet he declares in advance that it would consist of “not just the instrumental use of Knowledge (indeed, of {\sl veda}\index{Veda}) in the essentialization and dichotomization of the social order, but the very control of knowledge” {\sl (ibid)}. On the other hand, he complains that the post-colonialists, without having interrogated the forms of pre-colonial domination, accuse the British of having arbitrarily privileged the {\sl dharma-śāstra}-s over local custom. Thus, he refers to the essay {\sl Contentious Traditions: The Debate on Sati in Colonial India} by Lata Mani in which she contends that as an effect of the colonial discourse, Brahmanical scripture came to be privileged and constituted as the authentic cultural tradition of India. His argument is that in order to prove this point, the author does not “proceed to the logically prior question, ‘whether brahmanic texts\index{texts (Sanskrit/Indic)} [have] always been prioritized as the source of law’\index{law!sources of} $\ldots$ but to ‘a careful reading of the Parliamentary Papers’ $\ldots$ [and thus] we never leave the colonial arena in pursuit of these goals” (1993:99).

However, he does not consider the fact that the reason why Mani does not find it necessary to leave the “colonial arena” is that the evidence she is looking for is covered in the texts of the colonial period where she discerns a change in the depositions made by the pundits:
\begin{myquote}
While officials treated {\sl vyavasthās}\index{vyavastha-s@\textsl{vyavasthā}-s} (the written responses of pundits to questions put to them by colonial officials on various aspects of sati) as truthful exegeses of the scriptures in an absolute sense, it is clear from reading the {\sl vyavasthās} that the pundits issuing them believed them to be interpretive\hfill (Mani 1987:133).
\end{myquote}

As Mani explains, the Parliamentary Papers show that the {\sl vyavasthās} were tentative which would imply that the pundits issuing them were being called upon to interpret scripture in altogether different ways and for unprecedented purposes: 

\begin{myquote}
In the beginning at least, the responses of pundits appointed to the court did not reflect the kind of authority that colonial officials had assumed, both for the texts and the pundits ({\sl ibid}, 149). 

By contrast there is nothing tentative about the 1830 orthodox petition; there are no qualifiers prefacing textual excerpts $\ldots$ [and the petition was noted as being] ‘accompanied by legal documents.’ Here the equation between law and scripture is complete (150). 
\end{myquote}

What Mani’s research of the Parliamentary Papers reveals is how Indians adapted themselves as they began to understand what could and could not pass muster in the new regime as legally admissible and gradually started prioritizing scripture in their legal petitions as they realized it would prove most effective with their colonial masters. It is evident from Mani’s essay that apart from Brahmanical scripture, there were other sources of law such as caste councils\index{caste!council} and customary usages, which were ignored by the colonial administrators.
\newpage

While Lingat\index{Lingat, Robert} (1973:139 ff.) explains how the British judges such as H. Nelson, District Judge in the Madras Presidency, and Indologists such as Auguste Barth sought to rectify the undue weightage they had ascribed to the {\sl dharma\index{dharma}-śāstra}-s in legal matters due to European presumptions\index{presumptions} about the operation of law, Pollock chooses to repeat the errors of the early Orientalists. He treats the restrictions on the {\sl śūdra}-s in the {\sl dharma-śāstra}-s\index{dharma@\textsl{dharma}!\textsl{-śāstra}} as “juridical structures of inequality” (Pollock 1993:111) and speculates that “in fact, much of the discourse as we find it in the nineteenth-century Raj could easily have derived, and may have actually derived, from a text like the twelfth-century digest” (Pollock 1993:100).
\medskip

\section*{American Orientalism}\index{American Orientalism}

The purpose of Pollock’s essay is to highlight the challenges faced by an Indology confronted with the realization of its Orientalist essence and its future in a world which remains suspicious of objectivity and where critical knowledge remains subservient in unknowable ways to imperialist ambition. In the final section of his essay, titled ‘For a Critical Indology,’ Pollock explains four limitations of Indological scholarship which have become exposed in recent times and the way forward for this discipline. This discussion merits our attention because it bears repercussions for the manner in which Indological scholars will be ideally expected to approach their objects of study.
\vskip 2pt


Firstly, Indology was complicit in colonialism\index{colonialism} and Nazism but even as these forms of domination have faded away, “the rise of a new empire and its continued production and utilization of Orientalist knowledge” suggests that “neo-colonial foundations have been built in their place” (Pollock 1993:111). Without naming it, Pollock appears to be hinting at the rise of an American Orientalism\index{American Orientalism}. 
\vskip 2pt

Secondly, Pollock relates Indology to a more general “crisis of the culture of humanistic scholarship as such” (ibid). He refers to scholars in humanistic studies who collaborated directly or indirectly with repressive powers and questions the value of their scholarship to human life. Interestingly, the scholars he mentions in this context include only those who later came to be associated with fascism\index{fascism} such as Heidegger,\index{Heidegger, Martin} Paul de Man,\index{Man, Paul de} Eliade\index{Eliade, Mircea} and Dumezil. He is careful not to draw suspicion towards Western scholarship in the humanities, which was supportive of leftist repressive powers. 

Thirdly, while humanistic scholarship in the West remains critical of the dominant regime, it is at the same time sponsored by the regime itself through its institutions. This point raised by Pollock is important because it unwittingly reveals the warped mentality which critical scholarship has assumed in our times. It is but obvious that any liberal regime will allocate space for an oppositional discourse and encourage a diverse set of ideas which will serve as a check on excess. But it is only those who enter that space not so much to offer correctives to the policies of the dominant regime but to overthrow the system in its totality, who ponder in amazement at the sponsorship they receive from it and disparage it as “domestication” and “commodification” of their knowledge. 


Finally, Pollock once again alludes to the problem of objectivity in scholarship which is readily admitted in theory but is not corrected in practice. This neatly sums up the crisis faced by the humanities in the West and it is important for Indians to pay attention to it because Indology, as a Western discipline,\index{Indology!as a Western discipline} as the application of Western theories for studying Indian texts\index{texts (Sanskrit/Indic)} and traditions, is equally affected by it.

Let us now consider the tentative solutions which Pollock has proposed to overcome the foregoing problems. Firstly, with regards to the issue of objectivity, he suggests that “we should construct perspectives that $\ldots$ would include giving priority to what has hitherto been marginal, invisible and unheard” (114). While Pollock admits the difficulties of excavating this knowledge “given the radical silencing and screening out of communities effected by ‘classical’ culture” ({\sl ibid}), what is unclear is the relation of this strategy to the problem of objectivity which as he puts it has to do with “the pre-judgments, theory-ladenness, and perspectival partiality out of and with which we perceive $\ldots$ a cultural object; the way discourses serve in class-divided societies to sustain forms of domination; the purely rhetorical (rather than ontological) status of the truth claims of historical description” (Pollock 1993:113). 


The implicit suggestion here is that since scholarship cannot be objective anyway it is best to justify it as the unworthy means to the noble end of giving voice to the “marginal, invisible and unheard.” What is disquieting is that since these voices have admittedly been “radically silenced and screened out” by the “classical” culture, their recovery will likely involve nothing more than a denigration and denunciation of the “classical” culture itself and a superimposition by contemporary thinkers of their own radical views on their objects of study. We must bear in mind that according to the standard Orientalist critique, such a dominant Sanskritic “classical” culture is itself the construct {\sl par excellence} of colonial Indology. Pollock, on the other hand, appears to be suggesting that colonial Indology privileged a Sanskritic “classical” culture that had always been dominant and what future Indology should do is privilege those cultural voices which were allegedly dominated by it. Hence his desperate attempts to “save” colonial Indology from the stigma of Orientalism i.e. from the allegation of having constructed an imagined India using a Western lens. What he is effectively saying is that colonial Indology did not err in terms of the form of the dominant Sanskritic “classical” culture which it conceived; it erred rather in perpetuating that dominance when it should have deprecated it and this is the error which future Indology should strive to rectify.  


Secondly, he points out that prior to the eighteenth century, the world system was dominated by India while Europe enjoyed a peripheral position. However, the socio-economic historiography\index{historiography} of India was geared towards explaining the causes of under-development rather than the actual developments which had occurred. In a similar manner, he suggests that “a pre-emptive European conceptual framework of analysis has disabled us from probing central features of South Asian life, from pre-western forms of ‘national’ (or feminist, or communalist, or ethnic) identity or consciousness, pre-modern forms of cultural ‘modernism,’ pre-colonial forms of colonialism\index{colonialism}” (Pollock 1993:115). 

What is assumed here is that if India was a developed economy prior to the colonial conquest then it must have analogous formulations of modernity and the purpose of Indology is to discover the Indian equivalents for those categories which qualify as modernistic in the West. This is a problematic quest because it does not seek to understand India in terms of its own intellectual categories but presumes that similar historical processes were taking place in India as in the West but got arrested due to the colonial intervention. Instead of imposing the Western forms, it seeks to activate the allegedly native forms of these historical processes. These correspondences sound eerily familiar to the manner in which developments within a Western category like ‘religion’ are mapped to Indian phenomena such that the flow “Old Testament – New Testament – Catholicism – Protestantism”\index{Protestant(ism)} becomes comparable to “Veda\index{Veda} – Upaniṣads – Brahmanism or Dharmic Hinduism\index{Hinduism} – Buddhism\index{Buddhism} and other protest movements.”


Thirdly, Pollock warns sternly against “third-worldism” i.e. the projection of an unproblematized concept of tradition. He points out that “traditions $\ldots$ have been empires of oppression in their own right – against women, above all, but also against other domestic communities” (Pollock 1993:116). In the Indian case, he considers Sanskrit as the main agent of oppression. He understands the difficulties faced by the Western scholars in thematizing violence in traditions which have suffered at the hands of the West, but appears to condone the intervention in cases of “a culture’s failure to play by its own rules” and “evidence of internal opposition to its domination” ({\sl ibid}). Both these are problematic clauses as a loop-hole can easily be found to justify intervention through the production of “atrocity literature.”\index{atrocity literature} There is also the audacity of interventionism\index{interventionism} to consider here as if there is a mandate for Western scholars to meddle in the internal issues of “third-world” cultures. It is not clear if such a mandate is entrusted to Western intelligentsia in general with regards to all cultures, or if this is a prerogative merely of Western Indologists with regards to Sanskrit culture. What is, however, most alarming in this point is the advocacy of a {\sl prima facie} approach towards “tradition” as an “empire of oppression” rather than seeking to understand how it facilitates the maintenance of a decent society. On the other hand, we have seen in the previous point how the Indologist is exhorted by Pollock to highlight the modernistic impulses in Indian history, notwithstanding the fact that the most widespread and horrific instances of oppression in the last three centuries are to be found in modernity.

Finally, Pollock is critical of attempts at historicizing the “traditional domination as coded in Sanskrit” ({\sl ibid}). He quotes the view of a random woman who claims to submit to “the economic, social and emotional violence of Indian widowhood” on account of the {\sl śāstra-s} and a Dalit manifesto which declares the “first and foremost object of this cultural revolution” to be the liberation of every person from the “thraldom of the {\sl śāstra-s” (ibid)}. Clearly, in his view the {\sl śāstra-s} are highly oppressive in nature and their ill-effects are experienced to this day. He demonstrates no interest in verifying the allegations laid against the {\sl śāstra}-s or what improvements can be done to check their excesses. If we refer to his views articulated elsewhere in the essay that “none of this palliation [of counter-movements like Buddhism\index{Buddhism} or sectarian movements like Pañcarātra\index{Pancaratra@\textsl{Pañcarātra}}] makes itself felt in the totalizing constructions of the social order” (Pollock 1993:110), it is evident that he considers the situation hopeless. His intention appears to be nothing short of branding the {\sl śāstra-s} as an incurable evil and thus exhorts that “we should not resist any ‘historicization’ that serves to normalize or trivialize domination” (Pollock 1993:116).

I assume that Pollock wants to say that we {\sl should} resist ‘historicization’ and the negation is a typo because he gives here the example of {\sl Historikerstreit}\index{Historikerstreit@\textsl{Historikerstreit}} in which a critique was developed against the demand for a historicization of the Nazi\index{holocaust!Nazi} period. As we have noted above, in the previous two sections, understanding German Indology in the context of historical antisemitism, Nazi\index{holocaust!Nazi} Germany in the context of German colonialism\index{colonialism!Nazi/German}, and Sanskrit knowledge in the context of tension between {\sl dharma} and law, permits us to come to grips with the tragedy which overcame the Jewish people and the traditional Hindu legal system. On the other hand, to quarantine these historical moments, whether Nazi Germany or the Sanskrit culture, as if they were manifestations of some kind of an ahistorical, demonic evil, would reduce them simply to monuments of undying national shame, which has happened in the case of the former and which, it appears, Pollock would like to make happen in the case of the latter. What is ironical is that by drawing a comparison between Nazi Germany and Sanskrit culture, he accomplishes the very travesty he seeks to avoid – of normalizing and trivialising the crimes of Nazism. After all, at the core of {\sl Historikerstreit} lay precisely “the question of moral equivalence – the notion that by comparing atrocities and genocides\index{genocide}, the Holocaust is diminished through its linkage with other ‘lesser’ events” (Fitzpatrick,\index{Fitzpatrick, Matthew P} 2008:483).

Furthermore, it is hypocritical that Pollock does not find it problematic to historicize British colonialism\index{colonialism!British} by tracing its form of domination to a “pre-colonial colonialism.” In this matter, he willingly “jeopardizes the heuristic historical specificity of the very concept [of Orientalism]” on the pretext that “we may lose something still greater if not doing so constrains our understanding of the two other historical phenomenon [i.e. German Indology and Sanskrit knowledge]” (Pollock 1993:78).Yet elsewhere, in the process of criticizing the pursuit of objectivity in scholarship, he deplores “the decontextualization and dehistoricization of the scholarly act” pointing out that it “enabled some of the most politically deformed scholarship in history $\ldots$ to come into existence” (Pollock 1993:86-87).It thus appears that determining which historical events, concepts and texts\index{texts (Sanskrit/Indic)} should or should not be historicized, contextualized or, for that matter, “nuanced,” constitutes the very essence of the politics of knowledge production.

\section*{Conclusion}

The declared intent of Pollock’s essay is to extend the framework of Orientalism to include German Indology and Sanskrit knowledge and thus to depict them as forms of domination. However, the net result of this effort is to exonerate British Orientalism\index{British Orientalism} from misinterpreting and misrepresenting Sanskrit knowledge for the advancement of colonial rule, and project Sanskrit knowledge itself as a factor in the development of Nazi ideology. It is impressed upon the reader through reiteration that Sanskrit knowledge was a tool for expressing power\index{power!Sanskrit as a source of} in ancient India:
\begin{myquote}
Sanskrit knowledge presents itself to us as a major vehicle of the ideological form of social power in traditional India\hfill (Pollock 1993:77).
\end{myquote}

\begin{myquote}
Sanskrit was the principal discursive instrument of domination in\break pre-modern~India \hfill(Pollock 1993:116).
\end{myquote}

\begin{myquote}
[An] analysis of the object of Indology [i.e. Sanskrit knowledges] $\ldots$ as an indigenous form of knowledge production {\sl equally saturated with domination} [as Indology itself, has] important implications 

~\hfill(Pollock 1993:80, italics mine). 
\end{myquote}
However, if Sanskrit knowledge was always imbricated in power, then its projection as a form of domination in Indology cannot be treated as a distortion, which means that Indology itself cannot be considered as Orientalist in the polemical sense. Rather, it suggests that if Indological knowledge abetted in colonial domination it was because its object of study naturally lent itself to such an application for even prior to the British conquest, it was a form of exercising hegemony. Pollock admit this much as well:
\newpage

\begin{myquote}
Sanskrit and Indian studies have contributed directly to   consolidating and sustaining programs of domination. In this (noteworthy orthogenesis) {\sl these studies have recapitulated the character of their subject}, that indigenous discourse of power for which Sanskrit has been one major vehicle and which has shown a notable longevity and resilience

~\hfill  (Pollock 1993: 111, italics mine).
\end{myquote}
If the British Indologists were simply ‘recapitulating the character of their subject’ in their collaboration with colonialism, then it can well be said that the German Indologists were doing the same in their collaboration with Nazism. Pollock forges a connection between Sanskrit knowledge and Nazi ideology through such subliminal devices as referring, on the one hand, to the anti-Jewish\index{anti-Jewish} laws in Nazi Germany as {\sl śāstric} codifications (Pollock 1993:86) and advising, on the other hand, the intellectual-historical\index{intellectual history} study of the Sanskrit {\sl arya} in a manner analogous to the German {\sl Arier}\index{Arier@\textsl{Arier}} (Pollock 1993:107). He treats the restrictions on the {\sl śūdra}-s found in dharmic texts\index{texts (Sanskrit/Indic)} as “components of a program of domination whose true spirit we might begin to conjure with other comparable programs, such as the {\sl Arierparagraphen}\index{Arierparagraphen@\textsl{Arierparagraphen}} of the NS state” (Pollock 1993:111). His expostulation against the historicization of Sanskrit knowledge is also expressed through a reference to the {\sl Historikerstreit} debate suggesting that its form of domination be considered analogous to the Nazi regime. Sanskrit knowledges thus serve not only as a vehicle of oppression in their own right, but through British and German Indology become causal factors in colonialism\index{colonialism!British} and Nazism\index{colonialism!Nazi/German} as well. All these memes of oppression, historically associated with Western powers – British and Nazi – thus converge upon Sanskrit knowledge through Pollock’s slanderous project of the comparative morphology of domination.


\begin{thebibliography}{99}
\itemsep=2pt
\bibitem[]{chap2_item1} 
Adluri, Vishwa. (2011). “Pride and Prejudice: Orientalism and German Indology.” {\sl International Journal of Hindu Studies}. Vol.~15, No.~3 (DECEMBER 2011), 
pp.~253--292

\bibitem[]{chap2_item2}
Breckenridge, Carol and Peter van der Veer. (1993). {\sl Orientalism and Postcolonial Predicament}. University of Pennsylvania Press, Philadelphia.

\bibitem[]{chap2_item3}
Bryant, Edwin. (2001). {\sl The Quest for the Origins of Vedic Culture}. Oxford University Press: New York.

\bibitem[]{chap2_item4}
Figuera, Dorothy (2002). {\sl Aryans, Jews, Brahmins: Theorizing Authority through Myths of Identity}. State University of New York Press, Albany.

\bibitem[]{chap2_item5}
Fitzpatrick, Matthew P. (2008). “The Pre-History of the Holocaust? The Sonderweg and Historikerstreit Debates and the Abject Colonial Past.” 
{\sl Central European History}. Vol.~41, No.~3 (Sep., 2008),\break pp.~477--503

\bibitem[]{chap2_item6}
Grünendahl, Reinhold (2012). “History in the Making: On Sheldon Pollock’s “NS Indology” and Vishwa Adluri’s “Pride and Prejudice.”” 
{\sl International Journal of Hindu Studies}. Vol.~16, No.~2 (AUGUST 2012), pp.~189--257.

\bibitem[]{chap2_item7}
Hellig, Jocelyn (2003). {\sl The Holocaust and Antisemitism: A Short History}. Oneworld Publications: Oxford, England.

\bibitem[]{chap2_item8}
Lewis, Bernard (1993). {\sl Islam and the West}. OUP: New York.

\bibitem[]{chap2_item9}
Lingat, Robert (1973). {\sl The Classical Law of India} (tr.\ J. D. M. Derrett). University of California Press, Berkeley and Los Angeles, California. 

\bibitem[]{chap2_item10}
Lockman, Zachary (2004). {\sl Contending Visions of the Middle East: The History and Politics of Orientalism}. Cambridge University Press.

\bibitem[]{chap2_item11}
Malhotra, Rajiv (2016). {\sl The Battle for Sanskrit}. Harper Collins, India.

\bibitem[]{chap2_item12}
Mani, Lata (1987). {\sl Contentious Traditions: The Debate on Sati in Colonial India}. Cultural Critique, Autumn 1987, pp.~119--156.

\bibitem[]{chap2_item13}
Marchand, Suzanne (2009). {\sl German Orientalism in the Age of Empire}. Cambridge University Press: New York.

\bibitem[]{chap2_item14}
Mason, Timothy (1995). {\sl Nazism, Fascism and the Working Class}. Cambridge University Press: Cambridge.

\bibitem[]{chap2_item15}
Phillips, Kim (2014). {\sl Before Orientalism: Asian Peoples and Cultures in European Travel Writing}, 1245--1510. University of Pennsylvania Press: Philadelphia.

\bibitem[]{chap2_item16}
Pollock, Sheldon (1993). “Deep Orientalism? Notes on Sanskrit and Power Beyond the Raj” (pp.~76--133) in {\sl Orientalism and Postcolonial Predicament} (Ed.) Carol Breckenridge and Peter van der Veer.

\bibitem[]{chap2_item17}
Said, Edward\index{Said, Edward} (1978). {\sl Orientalism}. Vintage Books: New York.

\bibitem[]{chap2_item18}
Said, Edward\index{Said, Edward} (1985). {\sl “Orientalism Reconsidered”}. Cultural Critique, No.~1. (Autumn, 1985), pp.~89--107.

\bibitem[]{chap2_item19}
Said, Edward\index{Said, Edward} (1989). Representing the Colonized: Anthropology's Interlocutors. {\sl Critical Inquiry}. Vol.~15, No.~2 (Winter, 1989),\break pp.~205--225.

\bibitem[]{chap2_item20}
Steinweis, Alan (2006). {\sl Studying the Jew: Scholarly Antisemitism in Nazi Germany}. Harvard University Press: Cambridge, Massachusetts 

\bibitem[]{chap2_item21}
Sugirtharajah, Sharada (2003). {\sl Imagining Hinduism: A Postcolonial Perspective}. Routledge: London.

\bibitem[]{chap2_item22}
Trautmann, Thomas (1997). {\sl Aryans\index{Aryan} and British India}. University of California Press: Berkeley.
\end{thebibliography}

\theendnotes
