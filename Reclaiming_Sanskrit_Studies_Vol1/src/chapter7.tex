\chapter{Sanskrit: the Phoenix Phenomenon}\label{chapter7}

\Authorline{K S Kannan and H R Meera}
\lhead[\small\thepage\quad K S Kannan and H R Meera]{}

\section*{Introduction}

The stalwarts of Western Indology have been held in great esteem for a long time now and in many instances their word has been taken as final on various topics. The extensive studies that they have made, the intense interest they have shown in learning about an alien culture, and the inter-disciplinarity they have brought into the studies have all been, rightly, respected. This bespeaks, though, of a scholar subjecting a specimen to scrutiny where the specimen is passive, at the receiving end of the attention. 

For a true intellectual exchange to happen, however, it is necessary for both the parties – the examiner and the examined - to interact as equals, subjecting each other and each other’s methods to scrutiny. The current paper is one such attempt  - to return the attention by considering the claim of “death of Sanskrit” made by Sheldon Pollock, a leading American Orientalist,  in his paper of 2001.

This paper aims to 
{\renewcommand\theenumi{\alph{enumi}}
\renewcommand\labelenumi{(\theenumi)}
\begin{enumerate}
%~ \itemsep=0pt
\item summarize the points made by him in as dispassionate and accurate a manner as possible,
\item analyse the same from various angles,
\item discuss the propriety of the data selected and analysed by him,
\item discuss the general methodological idiosyncrasies\index{methodological idiosyncrasies} betrayed in his writings, and 
\item provide pointers to the {\sl uttarapakṣin}-s – as to where further work is necessary to possibly uncover more data, and do greater justice to the issue.
\end{enumerate}}

In the following section (1), we briefly present the issues (that Pollock touches upon) more-or-less as bullet points in order to capture their essence (words within double quotes “ ” are the ones cited verbatim from Pollock). 

\section{Synopsis of the thesis}

The introductory portion of the essay gives us a picture of Pollock’s take on the political scenario\index{power!political} (viz. the ascent to power of BJP\index{BJP}) which makes Sanskrit as the main actor\index{main actor} in its identity politics\index{identity politics}. In sum, his take is that Hindus are being fed a distorted picture of their past (“Hindutva\index{Hindutva} propagandists have sought to show...that Sanskrit is indigenous to India...Sanskrit is considered...the source and sole preserver of world culture” (Pollock 2001:392)). He traces its “melancholy” roots to the early days of independence when Sanskrit was included as one of the recognised languages of the Republic of India. Even though this has ensured that Sanskrit Universities\index{Sanskrit!Universities} and institutions\index{institutions!Sanskrit} have received a lot of funding, Sanskrit literature does not show any growth.  Even with all the awards, grants, and promotion via All India Radio, school curricula etc, Sanskrit is as good as being in a coma, kept alive artificially “as an exercise in nostalgia”. And hence “in some crucial way, Sanskrit is dead”.

Pollock points out that the long history of Sanskrit makes it difficult to have one measure of vitality\index{measure of vitality} given “the different historical rhythms” (Pollock 2001:394) but he decides on a measure in that very paragraph – viz. the production of kāvya, since “it is in itself often an argument of how language is to be used , indeed, about how life is to be lived” (Pollock 2001:394)implying that he will decide on the life of Sanskrit based solely on the quality\endnote{Pollock is very critical of {\sl citrakāvya\index{citrakavya@\textsl{citrakāvya}}} genre but it takes undoubted authority on language to be able to write one. Hanneder (2002:300) comments on this (however, we cannot agree with his full substantiation) and points out that judging the literature over a very long period of time (like making a sweeping statement that after 1200s the literature is all of lesser quality) is problematic since other scholars can hold different opinions.} and quantity of the production of {\sl kāvya}.

After quoting a Gujarati poet who talks of the ‘death of Sanskrit’ in 1857, he terms that this story is “only part, and only intuition”, which implies that he is about to provide the full story, sans an intuitional approach, and with hard evidence.

In order to prove his avowed thesis of the death of Sanskrit literary culture as a historical process, he selects 4 points in history where we see disappearance or degeneration of Sanskrit literature – viz
\begin{enumerate}
\itemsep=0pt
\item Kashmir after 13th century CE
\item Vijayanagara after 16th century CE
\item Mughal court in mid-17th century CE
\item Bengal in the 19th century CE
\end{enumerate}

We will summarize the essential points for all the four stages in history (sections 1.1 to 1.4) followed by the summary of his conclusion as the fifth part (1.5).

\makeatletter
\renewcommand\subsubsection{\@startsection{subsubsection}{3}{\z@}%
                                     {0ex\@plus 0ex \@minus 0ex}%
                                     {-0.2ex \@plus .2ex}%
                                     {\normalfont\normalsize\bfseries}}
\makeatother                                     

\subsection{Kashmir:}

\subsubsection{} Pollock gives a detailed picture - of the presentation of {\sl Śrīkaṇṭha Carita\index{Srikanthacarita@\textsl{Śrīkaṇṭhacarita}}} by Maṅkha\index{Mankha@Maṅkha} in the royal assemblage in the presence of several preeminent scholars. The description of this peak of brilliant scholarship is followed by the denouement that this was the last of Kashmir’s generation of Sanskrit poets of note.  He narrates in detail how the production of literature “reduced to {\sl stotra-s}\index{stotra literature@\textsl{stotra literature}}”, and how nothing innovative by way of literary theories got produced at all after the 11th century.

\subsubsection{} The nearly dead Sanskrit tradition (during the reign of Hindu kings) is revived during the reign of the Muslim ruler Sultan Zain-ul-‘abidin\index{Zain-ul-‘abidin} who established peace after a long period of tumult. Here did get composed Jonarāja’s\index{Jonaraja@Jonarāja} {\sl Rājataraṅgiṇī}\index{Rajatarangini@\textsl{Rājataraṅgiṇī}!\textsl{(Jonarāja)}} along the lines of Kalhaṇa’s\index{Kalhana@Kalhaṇa}, “only, much less aesthetic”, in the words of Pollock and the work of Śrīvara\index{Srivara@Śrīvara} continuing that (again, a barer chronicle than its predecessor). Pollock also mentions that even though there was no serious original work from Śrīvara, his {\sl Subhāṣitāvalī}\index{Subhasitavali@\textsl{Subhāṣitāvalī}} implies a “reasonably accomplished curatorial study of Sanskrit at the Sultan’s court or had a substantial library”. A significant conclusion of Pollock here is this: even though Śrīvara’s anthology mentions a number of poets regarding whom much is not known (and hence could have lived and produced work during that 300 year period - where Pollock claims not much work was done), the samples in the anthology do not prove that anything of significance got produced in the literary field between the 12th and 15th centuries

\subsubsection{} Pollock lists the possible reasons for this literary collapse viz. (i) loss of important texts (ii) Mongol invasion\index{Mongol invasion} of 1320 (iii) simple non-availability of texts\index{texts (Sanskrit/Indic)} to modern editors. He then dismisses them all in one stroke that they do not seem likely. His reasoning is this: important texts\index{texts (Sanskrit/Indic)} were getting written and transmitted out of Kashmir for several centuries and for the centuries under question, there is nothing such that has happened; hence, nothing got produced during the time. According to him, the texts\index{texts (Sanskrit/Indic)} of the works that have been preserved, date to 12th century or even earlier, which strengthens the case that nothing got produced after 12th century.

\subsubsection{} For the singular death of the literary culture, Pollock traces the reason to the changes in socio-political sphere. His argument goes thus: It was because of the royal patronage\index{patronage} (what he calls “courtly-civic ethos”) that Sanskrit sustained its vitality; during this period, however, there was no royal patronage,\index{patronage} and this caused Sanskrit literature to die. The Hindu kings who have been described by Kalhaṇa were breaching the limits of degeneracy and situations got much worse as time passed. He substantiates the evidence of deterioration by citing a paucity of any mention of good works, or indeed, any poets, in Jonarāja’s {\sl Rājataraṅgiṇī}\index{Rajatarangini@\textsl{Rājataraṅgiṇī}!\textsl{(Jonarāja)}}.

\subsubsection{} Concluding, Pollock draws attention to a passage in Jonarāja’s work which describes how the Sultan longed to have “an epiphany” with Goddess Śāradā, but how he ended up breaking Her image. Pollock quotes Jonarāja “This no doubt occurred...because of the presence of the barbarians. A king is held responsible for the transgressions of his underlings”. 

\subsection{Vijayanagara}

\subsubsection{} The second sample chosen is the multilingual empire of Vijayanagara which had inscriptions issued in Kannada, Telugu, Sanskrit and Tamil. He states that the 350 years of the empire saw literary productions in Telugu, Tamil and Sanskrit. His accusation here is that even though the rulers were of Tulu\index{Tulu} or Kannada lineages, they did very little to promote courtly literature in Kannada
\newpage

\subsubsection{} The example he picks up is, of course, of Kṛṣṇadevarāya\index{Krsnadevaraya@Kṛṣṇadevarāya} (who ruled from 1509-1529 CE). According to Pollock there was only Timmaṇṇakavi\index{Timmanna(kavi)@Timmaṇṇa(kavi)} as the Kannaḍa court poet whose only accomplishment was the completion of Kumāravyāsa’s {\sl Bhārata}\index{Bharata (of Kumaravyasa)@\textsl{Bhārata (of Kumāravyāsa)}}. Kṛṣṇadevarāya\index{Krsnadevaraya@Kṛṣṇadevarāya} himself wrote his work {\sl Āmukta-mālyada\index{Amukta-malyada@\textsl{Āmukta-mālyada}}} in Telugu. While the courtly poetry in Kannada was so limited, the Kannada literature outside the court was thriving – with the saint-poets like Purandaradāsa\index{Purandaradasa@Purandaradāsa} and Kanakadāsa\index{Kanakadasa@Kanakadāsa}, and poets like Lakṣmīśa\index{Laksmisa@Lakṣmīśa} whose {\sl Jaimini Bhārata\index{Jaimini Bharata@\textsl{Jaimini Bhārata}}} is very popular even to this day. The contrast, he says, is very stark.
\vskip 1.5pt

\subsubsection{} He then comments on the paradoxes in the Sanskrit literature of the time. In his opinion, while there was phenomenal scholarship in Sanskrit (“almost industrialized magnitude”\index{industrialized magnitude} p~401), the literary creativity was exhausted. The administrators of the kingdom were very learned in terms of cultural literacy as well, but according to him, their learning was only reproductive and not original.
\vskip 1.5pt

\subsubsection{} The Sanskrit compositions that were produced in that period did not get, according to him, much circulation. They saw some immediate productions and were read during their time but didn’t attract much attention – no commentator glossed on them, nor were verses selected for an anthology, nor did they become a part of some curriculum. He surmises that “much may have been lost when the city was sacked in 1565 but works of major court poets and personalities do survive”. He expresses his curiosity in learning why these survived at all.
\vskip 1.5pt

\subsubsection{} He reasons that much was written with “vital literary energies” in the vernaculars. He compares the two {\sl Bhārata-s\index{Bharata@Bhārata|see{Bhāratavarṣa}}} that were created viz. the Kannada production of Kumāravyāsa\index{Kumaravyasa@Kumāravyāsa}, which did not receive any royal patronage,\index{patronage} and the Sanskrit production of Divākara\index{Divakara@Divākara}, which did. The former was quite popular, while the latter lay mostly unread\endnote{The latter point presented rather dramatically as “the latter lay unread and unrecopied in the palace library from the moment the ink on the palm leaves was dry.” (Pollock 2001:401)}.
\vskip 1.5pt

\subsubsection{} He then examines the titles that Kṛṣṇadevarāya\index{Krsnadevaraya@Kṛṣṇadevarāya} had {\sl – kāvya-nāṭaka-marmajña, kavitā-prāvīṇya-phaṇīśa and sakala-kalā-bhoja} – and mentions the {\sl Jāmbavatī-pariṇaya} that Rāya composed as being much praised by the erudite scholars who were present for its performance (p 402). He gives a brief sketch of the story which is originally from {\sl Bhāgavata-purāṇa\index{Bhagavata Purana@\textsl{Bhāgavata Purāṇa}}} and goes on to critique that there is nothing new literarily in the adaptation. However, he finds that this work is of interest to him because of how this fits, according to him, the political narrative of Vijayanagara Empire.

The play in Sanskrit (with the female characters therein using the Prakrit language) is as usual, but the Sanskrit was no longer associated with sacredness (“sacrality...associated with Sanskrit had been neutralized centuries before”, to cite Pollock). So he considers the question as to why Kṛṣṇadevarāya chose to compose in Sanskrit while the vernacular was the more powerful. His conclusion is that the {\sl kāvya} is “profoundly historicist-political” – more so than the previously composed {\sl carita-s, vijaya-s} and {\sl abhyudaya-s} that gave poetic accounts of military conquests; and for the reason that it was so closely tied to the history of its day, it ceased to be of interest at a later date.	

\subsubsection{} He concludes with the note that Sanskrit as a mode of personal expression was on its death-bed, and what it could express was only the concerns of the empire. So, when the empire disappeared, that sounded, he says, the death knell for Sanskrit as well. 
\vskip -40pt

\subsection{Mughal Court}
\vskip -5pt

\subsubsection{} Choosing an individual for anecdotal evidence (rather than a period as he does in the aforesaid cases) Pollock calls Jagannātha Paṇḍitarāja\index{Jagannatha, Panditaraja@Jagannātha, Paṇḍitarāja} (circa 17th century CE) (Sarma\index{Sarma} 1994:6-8) “the Last Sanskrit Poet\index{the Last Sanskrit Poet}”, and embarks on tracing his life and times (“very close to us in time, and yet we have almost as little concrete evidence about him as we have about the fourth-century master Kālidāsa\index{Kalidasa@Kālidāsa}” (Pollock 2001:404)).

\subsubsection{} In Pollock’s opinion, the output of Jagannātha in his capacity of a court poet is much like any other court poet’s. {\sl Rasagaṅgādhara\index{Rasagangadhara@\textsl{Rasagaṅgādhara}}}, Jagannātha’s famous work on Alaṁkāra-śāstra\index{Alamkara-sastra@Alaṁkāra-śāstra}, has all “the same assumptions, procedures and goals” as its predecessors in the field (implying it has added nothing new to the space). Likewise do the “panegyrics to the kings of Udaipur and Delhi and Assam which remain largely indistinguishable from centuries of such productions” (Pollock 2001:404).

\subsubsection{} However, Pollock says “one senses in his lyrics and even in his scholarly works some very new sensibility, which...might fairly be called a modern subjectivity” (2001:404-405). Tracing the story of Jagannātha - of breaking away from orthodoxy and falling in love with a Muslim woman,\index{Muslim!woman} and finally ending his life by drowning in the Gaṅgā\index{Ganga@Gaṅgā} -  Pollock asserts : “Something very old died when Jagannātha died, and also something very new.”

\subsubsection{} He then brings in the data about paṇḍits who called themselves {\sl ‘navya’} and presented works with new terminology, style, and modes of analysis that was radically different from the earlier ones. He however raises a question as to whether this radical change was merely a question of details or whether it pertained to the structure. Recounting the works of Bhaṭṭoji Dīkṣita\index{Bhattoji Diksita@Bhaṭṭoji Dīkṣita}, Nīlakaṇṭha Caturdhara\index{Nilakantha Caturdhara@Nīlakaṇṭha Caturdhara}, and Gāgā Bhaṭṭa\index{Gaga Bhatta@Gāgā Bhaṭṭa}, he also draws attention to the change of the political scenario of the time viz. the rise of Maratha\index{Maratha} power.\index{power!political}

\subsubsection{} He then talks of how the past and the present (i.e. mid-17th century CE) subtly but clearly got separated though there was no overt schism between the “{\sl prācya\index{pracya@\textsl{prācya}}}” and “{\sl navya}”\index{pracya and navya@\textsl{prācya and navya}}. To look at the socio-political milieu of scholarship, Pollock picks up two traditional scholars viz. Siddhicandra\index{Siddhicandra} (1587--1666 CE), a Jain monk\index{Jain monk} in the court of Akbar\index{Akbar} and Jahangir\index{Jahangir}, and Kavīndrācārya Sarasvatī\index{Kavindracarya Sarasvati@Kavīndrācārya Sarasvatī} (1600--75), a leading {\sl paṇḍit} of Vārāṇasī\index{Varanasi@Vārāṇasī} -- and contrasts them with Jagannātha.

\subsubsection{} Siddhicandra,\index{Siddhicandra} a man of great personal beauty and for that reason a favourite with the two Mughal emperors, was a Jain monk. He has written an autobiography which is the basis for the conclusions drawn by Pollock. Writing about the general socio-political atmosphere of the day, Pollock paints it as being very open to new ideas. The evidence forwarded are : the writings of {\sl Abul Fazal}\index{Abul Fazal} and the attitudes of Akbar\index{Akbar} himself; large scale translations that were happening from Sanskrit to Persian\index{Persian}; Mughal courtiers like Abdur Rahim\index{Rahim, Abdur} who were themselves experimenting with writing in Sanskrit; and lastly, Sanskrit intellectuals,\index{Sanskrit intellectuals} including young Siddhicandra, who were learning Persian.

Even with such exposure to cosmopolitan life, the writings of Siddhicandra stay, according to Pollock, intellectually at the 12th century instead of in the 17th century. In his {\sl Kāvyaprakāśa-khaṇḍana\index{Kavyaprakasa-khandana@Kāvyaprakāśa-khaṇḍana}}, critiquing Mammaṭa’s\index{Mammata@Mammaṭa} work, he calls himself ‘{\sl navya}’ and yet the intellectual content is the same (Pollock 2001:406). Pollock generalizes next about the “newness” of the “new” intellectuals:
\begin{myquote}
\eleven
“They certainly strove for ever greater precision and sophistication of definition and analysis (in imitation, in fact, of the New Logic), but these matters of style were far more striking than any substantive innovation... Beyond such innovations in analytic idiom, however, what may be most importantly new here is the self-proclaimed newness itself, and its intimation that the past is somehow passed, even if it will not go completely away.” \hfill(Pollock 2001:407)
\end{myquote}

\subsubsection{} Turning his attention to Siddhicandra’s autobiography (which Pollock sees as centering around Siddhicandra’s celebration of his maintenance of the traditional moral vision of his own self) Pollock expatiates on the “dramatic core” of the text. To sum it up, it is how the king and queen try to persuade Siddhicandra to renounce his celibacy, and how steadfast Siddhicandra is in his adherence to his monastic vows. Pollock finds it “especially suggestive of the nature of Sanskrit literary culture at this moment that all innovation—the narrative and literary and discursive novelty—should be in service of the oldest of Jain monastic ideals” (Pollock 2001:407).

\subsubsection{} Next comes Kavīndrācārya Sarasvatī in whom, Pollock bewails, even radical alteration of social environment produced no commensurate transformation of cultural vision. The scholar earned fame for persuading Shah Jahan\index{Shah Jahan} from levying jizya\index{jizya@\textsl{jizya}} tax on the pilgrims to\break Varanasi\index{Varanasi@Vārāṇasī} and Prayag\index{Prayag}. He was patronized by Prince Dara Shikoh\index{Dara Shikoh} and later accompanied the French traveller François Bernier\index{Bernier, Francois@Bernier, François} for over three years. Yet, his literary production, according to Pollock, was very conventional: his Sanskrit work was, glossorial and hymnal while his Hindi production was of greater value. He also owned a big library 
($>2000$ manuscripts) (which was probably due to the pension from the Mughal emperor, Pollock is so quick to point out).

With this, Pollock comes to the judgement part :“What Sanskrit learning in the seventeenth century prepared one best to do, one might infer from the lives and works of Siddhicandra and Kavīndra, was to resist all other learning.” (Pollock 2001:408)

\subsubsection{} The contrast by way of Jagannātha is presented next. He too had similar exposure to cosmopolitan life; he too enjoyed the patronage\index{patronage} of Dara Shikoh like Kavīndrācārya. But, Pollock points out, the effect of these on Jagannātha’s literary production was different – that Jagannātha got inspirations from vernacular texts\index{texts (Sanskrit/Indic)}. Unlike Siddhicandra who was steadfast in his celibacy, Jagannātha had a personal life where he was linked to a Muslim woman\index{Muslim!woman}. Pollock sees the verses written by Jagannātha regarding this beloved as being like nothing else before or after. He forwards the opinion that it was all because Jagannātha’s poetry was informed by Persian literature through social interactions, and that this is a manifestation of the theme of the ever-unattainable ‘{\sl mahbūb\index{mahbub@\textsl{mahbūb}}}’, the unattainability exaggerated by ethnic difference. 

Similarly, Jagannātha introduces a personal tone in his scholarly as well as poetic works, Pollock opines. Be it responding to the scholar who insulted his guru, or providing demonstrative examples for various tropes in his {\sl Rasa-gaṅgādhara}, Jagannātha brings in an individual element. Pollock asseverates that though other poets earlier have shown individuality and projected distinctive selves, there is “still something new in what Jagannātha is doing”. Jagannātha writes about “the death of his child”, while no one has written such personal sorrow before. 

Pollock however cedes that there is difficulty in painting a naive picture of Jagannātha putting out his emotions thus – that there is inconsistency in merging the emotional poet who is venting his lament in the poem, and the detached scholar who is clinically dissecting the same poem in {\sl Rasa-gaṅgādhara}. Pollock admits that he is stumped with “the discontinuities in Jagannātha’s poetry and theory”. He considers the possible explanations – that Jagannātha considered it inauspicious to have written about the death of his wife; and are we not supposed to consider them “the expressions of his true self?”; that he had forgotten that these were the verses on the death of his wife. Pollock discards all these possibilities and concludes that the philological method, by way of critically editing {\sl Bhāminīvilāsa}, is the only way to analyse this. “Or did a stupid editor mix verses up after Jagannātha’s death?” – Pollock muses.

He concludes that the greatest Sanskrit literary critic and poet of the age composed verses on the death of his wife, who happened to be a Muslim – and even if it was not Jagannātha who wrote those verses, whoever wrote those did a ‘first’ in Sanskrit. This newness, he says, “was born - and died”.

\subsection{Bengal}
\vskip -5pt

\subsubsection{} For the fourth period, Pollock gives a report that the British Raj\index{British Raj} did of the census of institutions\index{institutions!educational} of learning in Bengal. While students who studied various {\sl śāstra}-s in the 353 Sanskrit schools\index{Sanskrit!schools}\index{Sanskrit!Colleges} were all brahmins, those studying Persian at Muslim schools had a mixture of Hindu and Muslim students of whom a fraction were brahmins. The curricula at the Sanskrit schools which popularly taught Nyāya\index{Nyaya@Nyāya}, Vyākaraṇa\index{Vyakarana@Vyākaraṇa} and other {\sl śāstra}-s with some students pursuing literature, had texts\index{texts (Sanskrit/Indic)} from 4th to 12th centuries, and only one work from contemporary Bengal.

\subsubsection{} Even though a lot of Sanskrit literature was being produced, no {\sl kāvya} situated itself, Pollock says,  in the world of early colonialism\index{colonialism}, whereas Nyāya tradition thrived by addressing the issues at hand (Viśvanātha Tarkapañcānana’s\index{Visvanatha Tarkapancanana@Viśvanātha Tarkapañcānana} {\sl Siddhānta-muktāvalī\index{Siddhanta-muktavali@\textsl{Siddhānta-muktāvalī}}} reorganized received wisdom and found acceptance all over the subcontinent). Hence Pollock concludes that the network which worked with Sanskrit intellectuals\index{Sanskrit intellectuals} across the nation was intact but no one inserted literary texts\index{texts (Sanskrit/Indic)} into this network (hence they were not worth inserting). He points out that Sanskrit imagination saw nothing new even when the production of literature was thriving. Be it in Tanjore\index{Tanjore} or Jaipur\index{Jaipur} or Mysore\index{Mysore}, the court-patronized productions never left the court, and the court-produced literature ceased to make history.

\subsubsection{} Pollock notes that whatever was necessary to impact people even outside of literature, be it the struggles against Christian missionizing or the anti-colonial pamphlets of Ishwar Chandra Vidyasagar\index{Ishwar Chandra Vidyasagar} (who was the founder of the Calcutta Sanskrit College\index{Calcutta Sanskrit College}), it was the vernacular that was used. Hence, Pollock says, Sanskrit can hardly be called alive if it was not a ‘vehicle for living thought’. 

\subsubsection{} Pollock’s final word on this is: Yes, Sanskrit paṇḍits aspire to create a literary-cultural realm where Kālidāsa\index{Kalidasa@Kālidāsa} would have been at home; someone who does not have this 2000-year legacy (such as William Adam\index{Adam, William} who conducted the aforementioned census in the 1830s) will not feel the weight of dead poets and scholars still being quite alive for them. But there is no point wasting criticism on them like Adam did. All the things that Adam criticized about the culture are actually central values of Sanskrit literature, which can only be appreciated by those who have the necessary cultural training.

However, Pollock asserts that the question is actually why and when Sanskrit literature became a practice of repetition and not renewal, and why the capacity to reimagine the world was lost to Sanskrit for ever.

\subsection{Conclusions}
\vskip -5pt

\subsubsection{} That there can be no straightforward narrative\index{straightforward narrative} to fit the above four moments into a plausible single historical narrative, Pollock readily recognizes; but he wants to recognize, in general terms, how a great tradition can die – implying that the {\sl when} of the death of Sanskrit tradition is a given and is in the past, and that we only have to figure out the {\sl how} and {\sl why}. 

\subsubsection{} He brings in the comparison with other dead languages\index{dead languages} like Greek and Latin. Greek is ruled out straightaway, but Pollock sees the later history of Latin\index{Latin!(Vulgar, Medieval, Church)} as showing great similarities to that of Sanskrit. Slow death, first losing out as the vehicle of literary expression while continuing to be expression for learned discourse; periodic renewals or forced rebirths, sometimes connected with “politics of translocal aspiration”; the way it “ever more exclusively associated with narrow forms of religion and priestcraft, despite centuries of secular aesthetic” (Pollock 2001:415) – these points are common to both. The contrast is with regard to communicative competence which was lost for Latin\index{Latin!(Vulgar, Medieval, Church)} with vernacularization, whereas Sanskrit continued to be known and studied - by the very people who promoted the vernaculars\index{vernaculars}.

\subsubsection{} Acknowledging the difference between Latin\index{Latin!(Vulgar, Medieval, Church)} and Sanskrit, he wants to locate the cause of the death of Sanskrit in the “South Asian historical experience”. He dismisses outright the decline of Sanskrit culture with Islamic rule\index{Islamic rule}, saying that it is historically untenable even though the contemporary climate (implying the political atmosphere of today) favours the theory. It is his firm conviction that it was the “barbarous invader” who was trying to revive Sanskrit, while “a set of much long-term cultural, social, and political changes” caused the death.

\subsubsection{} Paraphrasing his thoughts on the last-mentioned changes: It was the court that took care of Sanskrit, and when there was internal enfeeblement there, it naturally affected Sanskrit. Then there was the competition between vernaculars\index{vernaculars} and Sanskrit for recognition. These factors did not act in equal force everywhere. In Kashmir it took a toll on Sanskrit because of eroded ethos. By 13th century, before the Turkish rulers established themselves, it had already died and hence they can’t be held responsible for it. In Vijayanagara, the literature produced was predictable and hollow. Anything which was literarily important was said in Telugu or, outside the court, in Kannada or Tamil. Those who didn’t have such content wrote in Sanskrit and were never read. The same was the case in the North, where vernacular held the sway. Whatever was important and urgent had to be said in the vernacular and the cosmopolitan Sanskrit made less and less sense in this regional world.

\subsubsection{} Any structures that came to take collective action against incidents like abuse of pilgrims, through sending petitions never became institutionalized; instead they remained transitory, narrow goals. The “new” intellectuals had nothing new in substance to offer but only in style. No new idiom came to life to even articulate the new relationship with the past, forget critiquing it. No new knowledge, no new theory of religious, let alone political, identity was produced to reflect the changes that were being seen. No new creation of literature in a sustained manner came to be. 

Pollock ends this with 

\begin{myquote}
\eleven
“the mental and social spheres of Sanskrit literary production grew ever more constricted, and the personal and this-worldly, and eventually even the presentist-political, evaporated, until only the dry sediment of religious hymnology remained... At all events, the fact remains that well before the consolidation of colonialism\index{colonialism}, before even the establishment of the Islamicate political order, the mastery of tradition had become an end in itself for Sanskrit literary culture, and reproduction, rather than revitalization, the overriding concern... In the field of power\index{power!political} of the time, the production of Sanskrit literature had become a paradoxical form of life where prestige and exclusivity were both vital and terminal.’
\vskip -5pt

~\hfill(Pollock 2001:417-418)
\end{myquote}
\vskip -40pt

\section{Analysis of the “analysis”}
\vskip -5pt

Pollock’s acclaim as a Sanskritist leads one to expect a dispassionate presentation of facts and deductions in the subject. However he catches the reader’s attention in the very first paragraph by his overtly political tone\index{political tone} in contextualizing his writing. He describes the ‘Hindutva’\index{Hindutva} forces\index{Hindutva forces@‘Hindutva’ forces} which have captured power\index{power!political} and are bent on presenting distorted pictures of the past to the masses, essentially changing the narrative which has been set in the past decades. 

In this section, we take specific points from the summary above for analysis, and group logically linked points even if they are chronologically apart. Due to fear of prolixity, we will not do detailed analyses of all the points. Where necessary, we shall be referring to the point summed up in the synopsis in section B.
\vskip -40pt

\subsection{What is the purpose of knowledge?}
\vskip -5pt

In a culture that defines the four valid goals of life\index{life, valid goals of} – {\sl dharma,\index{dharma@\textsl{dharma}} artha\index{artha@\textsl{artha}}, kāma\index{kama@\textsl{kāma}}}, and {\sl mokṣa\index{moksa@\textsl{mokṣa}}}– and where the first three are to be subservient to the last one (even though the fourth is sometimes subsumed under the first), everything is aligned to the last of the {\sl puruṣārtha\index{purusartha@\textsl{puruṣārtha}}}-s, however mediated the link. 

The only thing that needs to be done for its own sake is, according to our tradition, the preservation of the Veda-s\index{Veda}. As Patañjali\index{Patanjali@Patañjali} says ‘{\sl brāhmaṇena niṣkāraṇaḥ vedaḥ saṣaḍaṅgo ‘dhyeyo jñeyaś ca’(Mahābhāṣya\index{Mahabhasya@\textsl{Mahābhāṣya}}}, Paspaśāhnika\index{Paspasahnika@Paspaśāhnika}) [For a {\sl brāhmaṇa}, one whose principal pursuit is knowledge, the Veda-s\index{Veda} along with all the six Vedāṅga-s, are to be studied and understood for no putative reason whatsoever]. This is so since the Veda-s\index{Veda} are considered to be the first manifestation of the Ultimate Truth whose knowledge leads to {\sl mokṣa}. Hence, even the {\sl niṣkāraṇa} part of Patañjali’s quote refers to doing so with no eye on worldly gain.

Everything is done for the sake of putting us on the path to this final goal– even {\sl kāma} in its right place, and within bounds leads to that. Each point in time is linked to this ultimate goal, from before birth to beyond death ({\sl niṣekādi-smaśānāntaḥ: Manusmṛti\index{Manusmrti@\textsl{Manusmṛti}}} 2.16), and all key actions take the form of {\sl yajña}-s\index{yajna@\textsl{yajña}}. Even the consumption of food is linked to the ultimate goal:  
\begin{center}
\begin{tabular}{>{\sl}l}
āhārārthaṁ karma kuryād anindyaṁ / \\
kuryād āhāraṁ prāṇa-sandhāraṇārtham | \\
prāṇās sandhāryās tattva-vijñāna-hetos / \\
tattvaṁ jñeyaṁ yena bhūyo na janma || (cf Bhāgavata\index{Bhagavata@Bhāgavata} {\rm 11.18.34}) 
\end{tabular}
\end{center}
[Translation (ours): Take up a blameless profession  for the sake of earning your food; partake of food but to sustain life; sustain life in order to know the Truth; and know that Truth as a consequence of which one is no more subject to rebirth]

Everyone is linked in a hierarchical manner to follow the leader, the pursuer of knowledge, so that each person is aligned to a path of his/her choice ({\sl sve sve karmaṇy abhirataḥ saṁsiddhiṁ labhate naraḥ - Bhagavadgītā\index{Bhagavadgita@\textsl{Bhagavadgītā}}} 18.45) which ultimately leads to the Summum Bonum. 

Even the kāvya\index{kavya@\textsl{kāvya}} is a part of educating oneself to align in the right direction. Called {\sl kāntā-saṁmita\index{kanta-sammita@\textsl{kāntā-saṁmita}}}, it is a part of the system to hand values to the people (the other two being the Veda-s\index{Veda} and Purāṇa-s).
\vskip -40pt

\subsection{The Measure of Vitality\index{Measure of Vitality}}
\vskip -5pt

Pollock’s narrow definition of what is a measure of the vitality of a language - viz. {\sl kāvya}-s of a certain calibre and type - excludes a lot of different types of literature ({\sl stotra} literature,\index{stotra literature@\textsl{stotra} literature} for example) which together contribute to making a literary culture vibrant. He also gauges the ‘death’ or ‘enfeebled’ nature of the Sanskritic culture,  by the non-production of any new theories in literary criticism (noted above in 1.1.1).

In such a framework of the four-fold values of life, as defined above, there is no work ‘for its own sake’ or ‘for the sake of mere pleasure’. Even pleasure can be of the right sort or the wrong sort (as kāma\index{kama@\textsl{kāma}} finds its place in both the four {\sl puruṣārtha\index{purusartha@\textsl{puruṣārtha}}}-s and the {\sl ari-ṣaḍ-varga\index{ari-sad-varga@\textsl{ari-ṣaḍ-varga}}}, the set of six [internal] foes) and this imposes a certain criterion for it to be a valid goal. Hence ‘{\sl ars gratia artis}’ (‘Art for art’s sake’)\index{ars gratia artis (‘Art for art’s sake’)@\textsl{‘ars gratia artis’} (‘Art for art’s sake’)} has never really found currency in the traditional Indian scenario (Coomaraswamy 2011:40). 

When Pollock criticizes the lack of any theory of literary criticism (see 1.1.1 above) designed to oppose or replace the one which was presented before the 12th century CE, we need to examine if there indeed was any need for such an upheaval. When the {\sl Rasa} Theory and the {\sl Dhvani} Theory came to be presented in fairly refined forms, the scholars could only debate on certain details of the theory, they must have felt no need to come up with something completely new and revolutionary simply to displace the established ones. Where a necessity was felt through experimentation and experience to debate the veracity of a point (as in philosophy), there has been no undue reverence shown simply because it is a long-established idea of a respected saint or scholar\endnote{A parallel case in point would be the acceptance of Euclidean geometry\index{Euclidean geometry} for near two millennia (from 300BCE to around 19th century CE) until non-Euclidean geometry\index{non-Euclidean geometry} was discovered. (Kulczycki1956:53)}.

The point to be understood here is that the tradition has always respected scholars for their ideas; however, when some lack is felt i.e., where theory fails to explain the experience, the debates have gone on relentlessly forward\endnote{The cropping up of the Advaita-Viśiṣṭādvaita-Dvaita\index{Advaita-Visistadvaita-Dvaita@Advaita-Viśiṣṭādvaita-Dvaita} traditions and further subdivisions thereof are cases in point.}. Of course, the human element of some personal rivalries and prejudices do reveal themselves, but the primary concern has been being more dispassionate than anything else\endnote{Famous is the Śaṅkara-Maṇḍanamiśra debate.}. By taking but one narrow section of the knowledge spectrum, only erroneous conclusions can be reached. 

When Pollock talks of the paradoxes he sees in the Sanskrit literature produced during the Vijayanagara period, or laments the lack of literature in the pre-colonial Bengal (see 1.2.3, 1.4.2 above), he has again put on the lens of what according to him constitutes a thriving literary culture. Ignoring significant portions of literature such as the detailed {\sl bhāṣya\index{bhasya@\textsl{bhāṣya}}} that Sāyaṇa\index{Sayana@Sāyaṇa} produced on all the four Veda-s;\index{Veda} dismissing the same as mere scholarship of the reproductive type without any originality\endnote{For more details on the Sanskrit works produced in the Vijayanagara empire, see Krishnamachariar (1989)}; mentioning the exciting works happening in Nyāya\index{Nyaya@Nyāya} while saying in the same breath that Sanskrit culture was dead – {\sl this} indeed is the real paradox.

The point Pollock makes, stated in (1.1.2) above, suffers again from focussing on a narrow range (viz. {\sl kāvya}) of a very vast spectrum of literature, of which {\sl kāvya} forms a small part after all as against the total intellectual output. Pollock has conveniently defined the parameters at the very outset to suit his own projections. Hanneder\index{Hanneder} (2002: 302-303) shows how this argument does not hold water since there are evidences of works which are superficially {\sl stuti-s}\index{stuti} (and hence disqualified as per Pollock’s criteria), but also express the religious developments in the Kashmir valley. 

The significance of the works in Sanskrit produced during the Vijayanagara period (see 1.2.4 supra) is measured (Pollock 2001:401) on the basis of whether they drew the attention of any commentator (curiously, and strangely, he does not consider production of a commentary to be a ‘valid’ measure of the vitality of a linguistic culture; for, had it been so, he could hardly have glossed over Sāyaṇa’s\index{Sayana@Sāyaṇa} massive commentarial work\index{commentarial work}). 

\subsection{Statistical inexactitudes}

While Pollock emphasizes the good effect of the rule of the Sultan Zain-ul-‘abidin (see 1.1.2 supra) in that he promoted Sanskrit literature production, and also emphasizes how the bad Hindu kings before that were the epitomes of degeneracy,\index{degeneracy} he fails to mention the other side of the picture – viz. the maladministration\index{maladministration} of Muslim rulers\index{Muslim!rulers} in general, as too the good patronage\index{patronage} given by most Hindu rulers. In fact, this is one of the ruling features we see in his writings - where one witnesses a systematic underplay of the negative effects of Islamic rule\index{Islamic rule} coupled with an over-emphasis of  the little good done by them, while at the same time doing the exact opposite in respect of Hindu rulers. This is elaborated upon in the next sub-section.

And as Hanneder\index{Hanneder} points out (2002:308), terming Jagannātha as “the last Sanskrit poet” is not warranted as there are works such as those of the acclaimed Ambikādatta Vyāsa,\index{Vyasa, Ambikadatta@Vyāsa, Ambikādatta} the {\sl mahākāvya} of the Nepali scholar Sukṛtidatta Panta,\index{Panta, Sukrtidatta@Panta, Sukṛtidatta} and the writings of Kṣamā Rao,\index{Rao, Ksama@Rao, Kṣamā} amongst a host of others (for more details, see Hanneder 2002:309).

\subsection{List-and-dismiss}

Even great jeopardies – such as the loss of valuable texts\index{texts (Sanskrit/Indic)} - by fires that destroyed libraries\index{libraries} in Kashmir (1.1.3 above) find but a very casual mention in his writings, as evidenced in the following sentence:
\begin{myquote}
\eleven
“important creative texts\index{texts (Sanskrit/Indic)} may have disappeared, perhaps in one of the fires that periodically engulfed the capital of Kashmir, or in the Mongol invasion of 1320, which, according to a sixteenth-century Persian chronicle, left the country in ruins. Texts\index{texts (Sanskrit/Indic)} may simply have eluded the notice of modern editors however carefully they may have combed the manuscript collections of Kashmir. But none of these possibilities seems very likely.”	\hfill Pollock (2001:398)
\end{myquote}

After listing out the possibilities of the existential threats\index{existential threats} that had faced the Kashmiris, it is all dismissed with the last sentence, without giving any sensible reason. The one ‘reason’ given is that important Sanskrit literature, particularly literary theory, used to disseminate out of Kashmir, and nothing of the sort has happened during the period. 

If we consider in general the destruction caused by Islamic invasion\index{Islamic invasion} from the perspective of the literary activities of any period, the greatest would be the destruction of libraries\index{libraries, destruction of} even though the loss of the life of the scholars is no less tragic. The former, however, would mean the loss of generations of cumulative learning and creation. 

Just taking one instance in the same {\sl Rājataraṅgiṇī} of Śrīvara\index{Rajatarangini@\textsl{Rājataraṅgiṇī}!\textsl{(Śrīvara)}}, we come across the following verse (1.5.75):
\begin{myquote}
{\sl sekandhara-dharānātho yavanaiḥ preritaḥ purā |} \\
{\sl pustakāni ca sarvāṇi tṛṇāny agir ivādahat} || 
\end{myquote}

[Translation (our own): Instigated earlier by {\sl yavana}-s (Muslims), Sikandar,\index{Sikandar Shah Mir (“Butshikhan")@Sikandar Shāh Mīr (“Butshikhan")} the sultan, burnt down all the books - just as fire would burn down grass.]

The period being talked about is that of Sikandar But-shikhan\index{Sikandar Shah Mir (“Butshikhan")@Sikandar Shāh Mīr (“Butshikhan")} (which means ‘Sikandar the Destroyer of Idols’\index{Destroyer of Idols}) (1389-1413 CE), regarding whom Śrīvara wrote (ref.\ Syed et al.\ 2011:282).

However, each of the ‘not-very-likely’ reasons that he has listed must have caused a great deal of instability in the normal life, which would undoubtedly affect any scholarly output that might have been possible at the time. Also, of the fires “that engulfed the capital” of Kashmir, one must of necessity ask - what was their cause? That was the period of the beginning of Islamic occupation - isn’t it then a part of the conflict and invasion? Śrīvara himself talks of the literature in {\sl deśa-bhāṣā} which is Persian rather than Kashmiri; doesn’t this point to a large-scale conversion?\index{conversions} These are some of the questions that arise in the light of the claims made by Pollock. More work would be required if we are to find further/clearer answers.

In the case of Vijayanagara (1.2.4 above) too, the result of an Islamic invasion\index{Islamic invasion} is downplayed considerably as in the case of Kashmir (1.1.3 above). The destruction that Vijayanagara saw after the battle of Talikota\index{Talikota, battle of} in 1565 marks a civilizational break (ref.\ Eaton\index{Eaton, Richard M} 2005:100). Sewell\index{Sewell, Robert} writes that the fallen capital was ruthlessly pillaged and destroyed for five months (Sewell\index{Sewell} 1962:200).

\subsection{Parsimony in applying Ockham’s Razor!}

While Pollock sees nothing new in content in either {\sl Rasagaṅgādhara}\index{Rasagangadhara@\textsl{Rasagaṅgādhara}} or the {\sl vilāsa}-s of Jagannātha (1.3.2 above), he submits that there is still something novel akin to the ‘modern subjectivity’ (Pollock 2001:405). He asserts (1.3.3 above) that it is Jagannātha’s personal story of love and the loss of the beloved etc., which have been voiced in {\sl Bhāminīvilāsa},\index{Bhaminivilasa@\textsl{Bhāminīvilāsa}} while most other scholars do not even raise this question of whether there is something personal about this at all.  It is just considered to be a work which is nothing but a collection of independent verses - by no means unusual in Sanskrit literature.
\newpage

The four sections of {\sl Bhāminīvilāsa} – Anyokti-vilāsa, Śṛṅgāra-vilāsa, Karuṇā-vilāsa, Śānta-vilāsa – have various poems arranged in the typical {\sl muktaka}\index{muktaka@\textsl{muktaka}} style (ref.\ Lienhard\index{Lienhard, Siegfried} 1984:103). While Pollock talks of Karuṇā-vilāsa as being reflective of Jagannātha’s personal bereavement, and tries hard to explain away the discrepancy between the emotionality of Jagannātha the poet and the detachment of Jagannātha the critic, one finds it easier to take the poem as just that – a poem which has come from the imagination of a great poet, rather than any portrayal of personal tragedies. 
\vskip 2pt

Is it not the first quality of a great poet to be able to portray his (imaginary) characters show emotions that are realistic? Does it not issue from the {\sl pratibhā}\index{pratibha@\textsl{pratibhā}} (genius) of the poet? When we come across Kālidāsa’s\index{Kalidasa@Kālidāsa} Aja bemoaning in First Person the death of his beloved wife (ref.\ {\sl Raghuvaṁśa}\index{Raghuvamsa@\textsl{Raghuvaṁśa}} Canto 8), are we to construe it as Kālidāsa’s personal grief coming through? The hair-splitting analysis made in order to explain away these superimposed discontinuities in Jagannātha’s theory and practice seems at best to be an exercise in futility.
\smallskip

\subsection{Perhaps...probably...and therefore}\index{Perhaps...probably...and therefore}

Jagannātha’s link with a Muslim woman\index{Muslim!woman} seems for Pollock, to be the sole qualifying factor for the “something new” in his poetry (1.3.9 above)! While no one disputes the existence of poetry in Persian regarding the unattainable love\index{unattainable love} (though Pollock emphasizes, that it is so on counts of ethnic difference), through ages we find in our own literature here, verses dwelling on this theme. Famous in the category is {\sl Caura-pañcāśikā}\index{Caura-pancasika@\textsl{Caura-pañcāśikā}} of Bilhaṇa\index{Bilhana@Bilhaṇa} (circa 11th c.CE), where Bilhaṇa\index{Bilhana@Bilhaṇa}, a commoner, falls in love with the princess. As for the personal tone, many works post-12th century (or even prior ones, though it is difficult to demarcate the transition) talk of their creator’s prowess, bring the personal element to ‘crush’ the opponent etc (e.g. {\sl Madhva-tantra-mukha-mardanam}\index{Madhva-tantra-mukha-mardanam@\textsl{Madhva-tantra-mukha-mardanam}} of Appayya Dīkṣita\index{Appayya Diksita@Appayya Dīkṣita} (16th c.CE)). This ‘cosmopolitan’ change in Jagannātha - that he broke the conventions, unlike Siddhicandra who remained celibate - again seems to be the driving point of seeing something new in his work. Can’t it suffice that Jagannātha was a truly brilliant poet (while Siddhicandra might not have been) and should it take a breaking of taboo\index{breaking of taboo} to make his brilliant poetry brilliant? Praising a truly great scholar for the wrong reasons is worse than “damnation by faint praise”\index{misinterpretation!techniques of!damnation by faint praise}. It would be akin to saying {\sl Mahābhārata}\index{Mahabharata@\textsl{Mahābhārata}} is truly great because it was written by the son of a fisherwoman.

In sum, by making an assumption that a disputed conjecture is indeed a fact i.e. that {\sl Bhāminīvilāsa} is but a personal narrative and that it is about his beloved who was a Muslim, and trying to explain the inconsistencies in the points of view of Jagannātha in his poetic and scholarly works, Pollock is discarding simple and plausible explanations, in favour of complicated and highly unlikely speculations.

\makeatletter
\renewcommand\subsubsection{\@startsection{subsubsection}{3}{\z@}%
                                     {3.25ex\@plus -4ex \@minus -.6ex}%
                                     {1.5ex \@plus .2ex}%
                                     {\normalfont\normalsize\bfseries}}
\makeatother

~\\[-20pt]

\subsection{Selection of data}

After considering specific kingdoms in the earlier two anecdotal samples, that he has chosen to consider now an individual for analysis as the third (1.3.1 above) is rather telling. Taking specific geographic locations in Bhārata in order to arrive at the life (or death) of a pan-Bhārata language is not logically sound, unless it is established a priori that the sample he is considering is truly representative of the whole (via {\sl sthālī-pulāka-nyāya}).\index{sthali-pulaka-nyaya@sthālī-pulāka-nyāya} 

~\\[-40pt]

\subsubsection{Spatial dimension of the problem:}
\vskip -5pt

One basic problem that we see in the anecdotal samples that are analysed to arrive at the conclusions in the paper, is the inherent contradiction between - (i) considering Sanskrit on the one hand as having pan-Indian\index{pan-Indian} (or pan-South Asian) reach and (ii) not considering the entire area of Greater India\index{Greater India} or rather Bhāratavarṣa,\index{Bharatavarsa@Bhāratavarṣa} at the four specific time periods that are being considered\endnote{There is also the inconsistency pointed out by Hanneder (2002:295): While Pollock argues for the political function of Sanskrit, the area across which that was spread extended from present-day Afghanistan to present-day Indonesia (roughly 84 lakh sqkms), without a military conquest but through intellectuals and religious persons. Pollock (1995:231) also argues that Sanskrit was never a commoner’s language. There is ample evidence to prove the contrary which shall be taken up in a later paper.}. If one focuses on a particular geographic area where conditions may or may not be conducive for cultural life to thrive (as in the case of Kashmir\index{Kashmir} during the invasion of Islamic forces, or the case of Vijayanagara after the Battle of Talikota\index{Talikota, battle of} in 1565), one is bound to see such ‘deaths’. On the contrary, if one looks at the entire Bhāratavarṣa as the locus of literary activity, we can see how Sanskrit is alive rather than being ‘dead’. 

After all, it was a common practice for scholars to move from kingdom to kingdom. ({\sl svadeśe pūjyate rājā; vidvān sarvatra pūjyate}, as the adage goes). Pollock himself mentions the ease with which Jagannātha moved from court to court, “from Andhra to Jaipur to Delhi and from Udaipur to Assam in a kind of vast “circumambulation of the quarters”” (Pollock 2001:404). To see Bhāratavarṣa as merely a collection of pieces of land that were kingdoms or {\sl janapada}-s, and nothing more, would be a grave injustice to the land which is truly the cradle of an ancient and steady civilization.

\subsubsection{Temporal dimension of the problem:}
\vskip -5pt

There are problems in choosing and considering specific slivers of time, in the long history of a language to gauge such a thing as its vitality. Such a situation obtains especially against a “problem” which is rather exclusive to the Indian civilization viz. one of the tendency for self-effacement\index{self-effacement}. As Coomaraswamy\index{Coomaraswamy, A K} (2004:175) points out, the artist is usually anonymous and it is always what is said, and not who said it, that mattered. 

While this is an excellent philosophy, it does make things extremely difficult in a scenario such as we face in the present age, where historicity of all aspects of literature matters, what with the history-centricity\index{history-centric} that is being thrust on us. (The reason for the emphasis is not hard to fathom – since the secularism which finds wide currency in today’s scholarship has its roots in the Abrahamic tradition\index{Abrahamic tradition, history-centric} which is deeply or rather essentially history-centric\index{history-centric}) (ref.\ Malhotra\index{Malhotra, Rajiv} 2011, Ch.2) . In our eventful several millennia of history, forget the absolute dating of a poet/philosopher (to locate him/her, to wit, in the grid of a linear timeline) and locating him/her in a probable geography - even relatively placing them sees too many difficulties more often than not. Commonly, it is the mention of the reigning king that helps in placing the scholar on the time line (though even that is fraught with difficulties since there can be disputes in dating the king himself, as there can be several namesakes of a king in different periods/ages). A lot of different kinds of internal evidences will have to be consulted in the absence of external evidences, in order to deduce a possible time range. For instance, in the case of Kālidāsa,\index{Kalidasa@Kālidāsa} assignments of date ranges from 1 century BCE to 8th century CE, and  with no definite idea about his place. What is worse, confusion gets confounded with multiple scholars or poets bearing the same name.

\newpage

\subsubsection{Numerical dimension of the problem}

{\leftskip=10pt\rightskip=10pt{\sl kati kavayaḥ, kati kṛtayaḥ, kati luptāḥ, kati caranti, kati śithilāḥ |\\
tad api pravartayati māṁ nigamoktākhyāna-saṁvidhāne hā ||}\par}


If we consider the number of works produced (encompassing all genres of writing from {\sl kāvya} to {\sl bhāṣya} to independent works on the various sciences) , only a fraction probably were reproduced in the manuscript\index{manuscript} form – due to various reasons such as lessening interest in the subject, natural calamities, invasions, economic conditions etc. The major reason for the losses of the onces actually created is actually destruction of libraries.\index{libraries, destruction of}  Of even that fraction, it is only a further fraction that survives till date – thanks to the tropical weather,\index{tropical weather, effect on manuscripts} ignorance in the families where they are preserved etc. Hundreds of Sanskrit texts\index{texts (Sanskrit/Indic)} are available only in Tibetan translation.\index{Tibetan translation} Of the surviving ones, even if we have some information of more than half of them, there are according to the census, over 30 million manuscripts.\index{manuscript} 
\begin{myquote}
\eleven
Sanskrit is the primary culture-bearing language\index{culture-bearing language} of India, with a continuous production of literature in all fields of human endeavor over the course of four millennia. Preceded by a strong oral tradition\index{oral tradition} of knowledge transmission, records of written Sanskrit remain in the form of inscriptions dating back to the first century B.C.E.\ Extant manuscripts\index{Extant manuscripts} in Sanskrit number over 30 million - one hundred times those in Greek and Latin\index{Latin!(Vulgar, Medieval, Church)} combined - constituting the largest cultural heritage\index{cultural!heritage} that any civilization has produced prior to the invention of the printing press. Sanskrit works include extensive epics, subtle and intricate philosophical, mathematical, and scientific treatises, and imaginative and rich literary, poetic, and dramatic texts.\index{texts (Sanskrit/Indic)}	\hfill (Goyal {\sl et al} 2012:1012)
\end{myquote}

The number of surviving manuscripts being that large, it is difficult to imagine the size of the corpus had all the works survived. 

When we look at only one particular genre such as {\sl kāvya}\endnote{The arbitrary choice of genres to determine if Sanskrit is alive or dead done where {\sl stotra}\index{stotra literature@\textsl{stotra literature}} is not considered a valid genre – this, according to Hanneder (2002:301), is only in line with the comparison with Latin\index{Latin!(Vulgar, Medieval, Church)} literature.}, if the numbers we are considering are even a third, hypothetically, of the total corpus available now considering the profusion of {\sl kāvya} literature that has been produced, there are still so many in that number which haven’t even seen the light of the day\endnote{One has only to look at Pingree’s {\sl Census of the Exact Sciences in Sanskrit}\index{Pingree’s Census of the Exact Sciences in Sanskrit@Pingree’s \textsl{Census of the Exact Sciences in Sanskrit}} to get an idea of how many scientific manuscripts have not been published yet.}. There would then be 10 million of the {\sl kāvya} manuscripts.\index{kavya manuscripts@\textsl{kāvya} manuscripts} If even a fourth of all the manuscripts ever produced have survived till date, the total number of manuscripts originally produced would be 4*30million, i.e. 120 million (another hypothetical number). Even if one factors in the repetitions (50 mss for each {\sl kāvya}), it would come to 5 lakh {\sl kāvya}-s. If this number is to be representative of the corpus that was originally produced, one sees the difficulty of extending a deduction or a conclusion from this fraction to the original whole.

We are also aware, to but a very small extent, of the losses. Many anthologies have eulogising verses that mention poets by name. In a play of Kālidāsa\index{Kalidasa@Kālidāsa} we find the mention of Bhāsa,\index{Bhasa (dramatist)@Bhāsa (dramatist)} Saumilla\index{Saumilla (poet)} and Kaviputra\index{Kaviputra (poet)}\endnote{...{\sl prathita-yaśasāṁ Bhāsa-Saumilla-Kaviputrādīnāṁ prabandhān atikramya vartamāna-kaveḥ Kālidāsasya\index{Kalidasa@Kālidāsa} kṛtau kathaṁ pariṣado bahumānaḥ} ({\sl Mālavikāgnimitram}\index{Malavikagnimitram@\textsl{Mālavikāgnimitram}} 1.1+)}. For centuries together, scholars had merely heard of Bhāsa\index{Bhasa (dramatist)@Bhāsa (dramatist)}, with none of his works being available. It was only in the beginning of the 20th century that Bhāsa’s works were traced (see Sastri 1925). And we still do not have any inkling as to who Saumilla\index{Saumilla (poet)} was or what Kaviputra\index{Kaviputra (poet)} wrote. Likewise there are so many poetesses who are known only by the appearance of their \hbox{{\sl muktaka-s}} in anthologies. Same is the case even of great writers in {\sl śāstra}-s such as Bhaṭṭa Tauta\index{Bhatta Tauta@Bhaṭṭa Tauta} (the guru of Abhinavagupta,\index{Abhinavagupta} and the author of {\sl Kāvya-kautuka-vivaraṇa}\index{Kavya-kautuka-vivarana@\textsl{Kāvya-kautuka-vivaraṇa}} which has not been traced) and Kauṭalya\index{Kautalya@Kauṭalya} (a.k.a Cāṇakya,\index{Canakya@Cāṇakya} the author of {\sl Arthaśāstra}\index{Arthasastra@\textsl{Arthaśāstra}}, which was discovered only in the early 20th century). 

It is often argued that if a work has not survived, it is because it did not have ideas that were worthy/capable of surviving. It is a very dangerous argument to make, for, this can be used as an explanation for any work disappearing from circulation, accidentally or deliberately, for various reasons\endnote{For instance, manuscripts could have been lost/stayed unpublished for any of the following reasons:
\begin{enumerate}
\item Preferring only to study past masters rather than also encourage/preserve newer works.
\item Possible repositories such as the university libraries\index{libraries, destruction of} getting destroyed in invasions/attacks
\item Newer works not following the pattern of what is trending. But when a century or two later people somehow hear about it and are interested, no copies to be found.
\item Manuscripts not surviving in the humid and hot tropical weather\index{tropical weather, effect on manuscripts} - what is the typical life expectancy of a manuscript\index{manuscript!life expectancy of} in such climates? - if it is not copied and preserved, it is gone for ever
\item Subsequent generations having lost the cultural connect with the manuscripts - dumping them in rivers or in the fire rather than allowing unregarding collectors to take it.
\end{enumerate}}.
When one is trying to base one’s conclusions on mere negative evidence\index{negative evidence} one must bear all these problems in mind\endnote{The basic thesis of Pollock (1995) in his work “the Sanskrit Cosmopolis” is to portray Sanskrit as a tool for wielding political power\index{power!political} through what he terms as ‘aestheticization of power’\index{aestheticization of power}. Hanneder (2002:297) points to the refutation that has already been written by Buhler\index{Buhler} on the matter (f14). In this thesis Pollock claims that Sanskrit lost its “political” function whereas in “the Death of Sanskrit” essay, he claims that Sanskrit lost its true life. The manuscripts available are from a period after Sanskrit has supposedly died. If one argues that those available are the elementary works and not the top-notch productions, the opponent can easily argue that you’ll need the elementary works copied at a larger scale to be used for education across the nation, and cannot definitely be held up as “negative evidence” for lack of advanced works.} and recall the oft-quoted statement of wisdom, viz. “Absence of evidence\index{Absence of evidence} is by no means an evidence of absence”. In the case at hand, simply because we haven’t found yet manuscripts of a specific genre and/or from a specific period, we are by no means entitled to conclude that there weren’t any books written – particularly in a civilization such as ours which shows such remarkable continuity and redoubtable zeal in knowledge production. All that one may conclude is – “We do not have any data from this period to theorize”. On a lighter note, one remembers Sherlock Holmes\index{Holmes, Sherlock} saying to his trusted Watson “It is a capital mistake to theorize before you have data”. But unlike Holmes who cried out, “Data! Data! I cannot make bricks without clay!” – some scholars can apparently pull off this veritable magic!

\section{Meta-analysis and pattern recognition}

In this section we try to recognize certain patterns of reasoning that emerge from reading (between the lines as well) the article of Pollock.
\begin{enumerate}
\item {\bf Choosing a narrow definition\index{misinterpretation!techniques of!narrow definition}\index{narrow definition} to determine the vitality of a tradition/language:} Choosing only that {\sl kāvya}-genre which according to Pollock is representative of ‘creative vitality’, he completely neglects the other genres of literature produced, including (a) the very substantial glosses by giants of scholars like Sāyaṇa, (b) {\sl stotra} literature\index{stotra literature@\textsl{stotra literature}}, and (c) scientific literature\index{Sanskrit!scientific literature} (ref. Hanneder 2002:298-99), to list but three.
\item {\bf Selecting data to fit a theory:} He cites Jonarāja’s listing of no poets, no good works in the period of 140 years. He mentions that the presence of Turks as insignificant. However, he fails to mention the havoc wreaked by Sikandar Butshikhan\index{Sikandar Shah Mir (“Butshikhan")@Sikandar Shāh Mīr (“Butshikhan")} who is mentioned by Śrīvara\index{Srivara@Śrīvara} (see Kaul\index{Kaul} 2001:231,233 and Haig\index{Haig, T W} 1918:454,455). Such cherry-picking of facts\index{misinterpretation!techniques of!cherry-picking facts} can only lead to a distorted picture.
\item {\bf Selective playing-up and playing-down:\index{misinterpretation!techniques of!selective playing-up \& playing-down}} We find instances of his playing down the great genocides,\index{genocide} invasions etc that happened with the coming of the Muslim invaders\index{invaders} (see 1.1.3, 1.2.4 above). He plays up the rare good things  e.g. the donation that was given by a Muslim to a temple here, a scholar supported there, and a good Sultan somewhere else – in order to project a picture of a very benevolent rule (e.g. the rule of Zain-ul-‘abidin\index{Zain-ul-‘abidin}) while not citing the enormous destruction caused (e.g.\ libraries\index{libraries, destruction of} burnt by Sikandar). His general claim is that Muslim rulers\index{Muslim!rulers} tried to save the language and were benevolent (Pollock 2001:416) which is contradicted by other records (ref.\ Bostom\index{Bostom, Andrew G} 2008:458-460). On the contrary, every case of a ‘bad’ Hindu ruler is projected as though it is a representative sample of ‘bad’ Hindu rule, while not choosing to look at the benevolent Hindu rulers and their contributions. This can be summed up with a new rule – {\sl meyādhīnā mānasiddhiḥ}\index{meyadhina manasiddhih@\textsl{meyādhīnā mānasiddhiḥ}} (Let us choose such tools and measures as suit the conclusions we want to portray!)
\item {\bf Using various terms/frameworks of social science, modern psychology, anthropology, Biblical studies\index{Biblical studies} etc to superimpose on traditional Indian thought.} The traditional scholars who are ignorant of other modern fields and the various theories therein are completely confounded and awed into submission to Pollock's methodology. There are inherent problems of using vocabulary and frameworks of fields of study {\sl completely alien} to Sanskrit studies.\index{Sanskrit studies} (For instance, {\sl Wirkungsgeschichte}\index{Wirkungsgeschichte@\textsl{Wirkungsgeschichte}} (See Pollock\index{Pollock!methods} 2001:393) is a term in Biblical studies\index{Biblical studies} coined by Gadamer\index{Gadamer} for the hermeneutic principle of “history of effects”.\index{history of effects} It means that a text is understood by taking account the effects it has produced in history, by inserting oneself in this history and dialoguing with it. See Eberhard\index{Eberhard, Philippe} 2004:90)
\item {\bf List and dismiss:}\index{List and dismiss}\index{misinterpretation!techniques of!list and dismiss} Enumerating the possible causes for a particular event (lest an opponent point out that all causes were not considered) but dismissing them without really giving even as much as substantial reason, saying no more than that they are ‘not very likely’. We see this in 1.1.3 and 1.2.4, to give but two examples. 
\item {\sl\bfseries Divida et impera:\index{misinterpretation!techniques of!divida et impera}} Though trying to deduce the fate of a pan-Bhārata language, his focus is on specific kingdoms and {\sl janapada}-s and not Bhāratavarṣa, in order to declare its death in each case. To show an apparent conflict between Sanskrit and vernaculars, he pits the production and popularity of vernacular literature against that of Sanskrit literature, drawing conclusions, and colouring them with politics.
\end{enumerate}

\section{Conclusion}

The work under scrutiny opens our eyes to the effort that has gone into a study of our literature and culture, the intentions notwithstanding. For a true debate to even begin, there needs to be an earnest effort from the insiders of this culture under scrutiny to come up with a faithful {\sl pūrva-pakṣa}\index{purva-paksa@pūrva-pakṣa} so that further efforts can be directed at coming up with a competent {\sl uttarapakṣa}\index{uttarapaksa@\textsl{uttarapakṣa}}. There are several areas which require work in order to do the latter. We indicate a few here:
\renewcommand\theenumi{\alph{enumi}}
\renewcommand\labelenumi{(\theenumi)}
\begin{enumerate}
\item There are inherent contradictions across Pollock's\index{Pollock!methods} writings which need to be investigated and critiqued. For instance, Hanneder (2002:302) points to the contention that Śrīvara’s work is almost “clerical” (1.1.2). He underlines the conflicting opinion of Pollock himself about Śrīvara who is “the most interesting intellectual at the court of Zain-ul-‘abidin” and yet “unable to create serious original work himself”. Also, his stand about the influence of Islamic forces as pointed out by Rajiv Malhotra (2016:285-286)
\item That very little promotion was done by King Kṛṣṇadevarāya for Kannada\index{Kannada} literature production (which thrived outside) and that Sanskrit, Telugu and Tamil received courtly patronage\index{patronage} (even though the Sanskrit literature of the day failed to make impact) – is a deduction which can and ought to be questioned though only after being armed with apt and sufficient data. It is only after a very thorough collection of information regarding the literature of the time (without making selection of any specific genre at the collection stage itself) can one even begin to challenge the conclusion.
\item In general, the question of how much literature (of all genres including scientific) was produced in any given period is something that cannot be easily answered given the complexity of circumstances. However, delving deep into the material that is actually available will itself probably lead to some concrete conclusions which are not arising out of emotion but are based on facts, against the questions raised in sections 1.1.2, 1.2.1, 1.2.2, 1.2.5,  1.4.1.
\end{enumerate}

Working along the lines suggested above, we should, hopefully, be in a position at some future date to reply to the issues raised. While it is necessary to attack the logic of Pollock’s arguments, one needs to essentially abstain from resorting to polemics\index{polemics} (even though that stand might not be reciprocated). 

For a language which has supposedly {\sl ‘died’ so many times}, Sanskrit definitely has a way of returning to life – a veritable phoenix that it is! However, rather than qualifying the low points in its history as ‘death’\index{death}, one can term it renewal or reconfiguration. This is because, unlike the other ‘dead’ languages, Sanskrit has a very dynamic and close relationship with the vernaculars which are in currency.

Hanneder (2002:309) rightly points to the essential cultural misunderstanding that Pollock displays when he repeatedly points to the ‘death’\index{death} of Sanskrit (whatever may be his measures). It is that here time is essentially cyclic and what was once an old style, renews itself and becomes current, whereas the Western mind essentially sees linear time\index{linear time} - where the arrow moves in but a single direction, and ‘culminating’ in the present achievements. The Indian mind does more easily handle the idea of renewal than does a Western mind.

While we have only raised pertinent questions in this paper, we intend to produce sounder arguments against Pollock in a future context.

\begin{thebibliography}{99}
\itemsep=2pt
\bibitem[]{chapter7_item1}
{\sl Bhagavadgītā} (=Ch.6.25 to 6.42 of {\sl Mahābhārata}).See Belvalkar.

\bibitem[]{chapter7_item2}
{\sl Bhāgavata}. See Tapasyananda.

\bibitem[]{chapter7_item3}
Belvalkar, S. K. (1945) Ed. {\sl Bhagavadgītā}. Bombay: Editor.

\bibitem[]{chapter7_item4}
Bostom, Andrew G. (2005) {\sl The Legacy of Jihad: Islamic Holy War and the Fate of Non-Muslims}. New York: Prometheus Books.

\bibitem[]{chapter7_item5}
{\sl Caurapañcāśikā}. See Tadpatrikar.

\bibitem[]{chapter7_item6}
Coomaraswamy, Ananda Kentish. (2004) {\sl The Essential Ananda K Coomaraswamy}. Edited by Rama P. Coomaraswamy. Bloomington: World Wisdom.

\bibitem[]{chapter7_item7}
Coomaraswamy, Ananda Kentish. (2011) {\sl The Wisdom of Ananda Coomaraswamy: Reflections on Indian Art, Life, and Religion}. Edited by S. Durai Raja Singam and Joseph A. Fitzgeral. Bloomington: World Wisdom.

\bibitem[]{chapter7_item8}
Eaton, Richard M. (2005) {\sl The New Cambridge History of India} I.8. Cambridge University Press.

\bibitem[]{chapter7_item9}
Eberhard, Philippe. (2004) {\sl The Middle Voice in Gadamer’s Hermeneutics: A Basic Interpretation with some Theological Implications}. Tubingen: Mohr Siebeck. 

\bibitem[]{chapter7_item10}
Goyal, Pawan., Gérard Huet, Amba Kulkarni, Peter Scharf, and Ralph Bunker. (2012) “A Distributed Platform for Sanskrit Processing,” {\sl Proceedings of COLING 2012: Technical Papers}, COLING 2012 (2012): 1011--1028.

\bibitem[]{chapter7_item11}
Haig, T. W. (1918) “The Chronology and Geneology of the Muhammadan Kings of Kashmir”.  {\sl The Journal of the Royal Asiatic Society of Great Britain and Ireland} (Jul.~1918), 451--468.

\bibitem[]{chapter7_item12}
Hanneder, J. (2002) “On “The Death of Sanskrit””. {\sl Indo-Iranian Journal 45}(2002): 293--310.

\bibitem[]{chapter7_item13}
Joglekar, K. M. (Ed.) (1916) {\sl Raghuvaṁśa} of Kālidāsa. Bombay: Nirnaya Sagar Press.

\bibitem[]{chapter7_item14}
Kaul, Arun. (2000) “Kesar : The cultural geography of Kashmir”. {\sl India International Centre Quarterly} Vol 27/28 (Winter 2000/Spring 2001), 226--234.

\bibitem[]{chapter7_item15}
Kaul, Srikanth (Ed.) (1966) {\sl Rājataraṅgiṇī of Śrīvara and Śuka}. Hoshiarpur: Vishveshvaranand Institute.

\bibitem[]{chapter7_item16}
Kielhorn, Franz (Ed.) (1880-1884) {\sl The Vyakarana-Mahabhashya of Patañjali} (3Vol.s). Bombay: Government Central Book Depot.

\bibitem[]{chapter7_item17}
Krishnamachariar, M. (1989) {\sl History of Classical Sanskrit Literature}. Delhi: Motilal Banarsidass.

\bibitem[]{chapter7_item18}
Kulczycki, Stefan.\index{Kulczycki, Stefan} (1956) {\sl Non-Euclidean Geometry}. (Translated from Polish by Stanislaw Knapowski). New York: Dover Publications Inc.

\bibitem[]{chapter7_item19}
Lienhard, Siegfried. (1984) {\sl The History of Classical Poetry: Sanskrit, Pali, Prakrit}. Wiesbaden: Harrassowitz.

\bibitem[]{chapter7_item20}
{\sl Mahābhāṣya} of Patañjali (on {\sl Aṣṭādhyāyī} of Pāṇini) See Keilhorn.

\bibitem[]{chapter7_item21}
{\sl Mālavikāgnimitra}. See Parab. 

\bibitem[]{chapter7_item22}
Malhotra, Rajiv. (2011) {\sl Being Different}. Noida, India: HarperCollins Publishers.

\bibitem[]{chapter7_item23}
Malhotra, Rajiv. (2016) {\sl The Battle for Sanskrit}. Noida, India: HarperCollins Publishers.

\bibitem[]{chapter7_item24}
{\sl Manusmṛti}. See Shastri.

\bibitem[]{chapter7_item25}
Parab, K P, (Ed.) (1924) {\sl Mālavikāgnimitram} of Kālidāsa. Bombay: Nirnaya Sagar Press.

\bibitem[]{chapter7_item26}
Pingree, David. (1970 et seq) {\sl Census of the Exact Sciences in Sanskrit}\break (5 Volumes). Philadelphia: American Philosphical Society.

\bibitem[]{chapter7_item27}
Pollock, Sheldon. (2001) “The Death of Sanskrit.” {\sl Society for Comparative Study of Society and History}. pp.~392--426.

\bibitem[]{chapter7_item28}
Pollock, Sheldon. (1995) “The Sanskrit Cosmopolis”, Ideology and Status of Sanskrit. {\sl Contributions to the History of the Sanskrit Language}. (Ed.) by Jan E.M. Houben, Leiden: E.J. Brill. pp.~197--247

\bibitem[]{chapter7_item29}
{\sl Raghuvaṁśa}. See Joglekar.

\bibitem[]{chapter7_item30}
{\sl Rājataraṅgiṇī}. See Kaul.

\bibitem[]{chapter7_item31}
Sarma, N. N. (1994) {\sl Jagannātha The Renowned Sanskrit Poet of Medieval India}. New Delhi: Mittal Publication.

\bibitem[]{chapter7_item33}
Sastri, T Ganapati and Bhasa. (1925) “The Works of Bhasa”. {\sl Bulletin of the School of Oriental Studies University of London} Vol.~3 No.~4.\\ pp.~627--637.

\bibitem[]{chapter7_item34}
Sewell, Robert and Fernando Nunes. (1900) {\sl A Forgotten Empire: Vijayanagar, a Contribution to the History of India}. London: S. Sonnenschein \& Co. Ltd.

\bibitem[]{chapter7_item35}
Shastri, J. L. (Ed.) (1983) {\sl Manusmṛti} of Manu. Delhi: Motilal Banarsidass.

\bibitem[]{chapter7_item36}
Syed, Muzaffar Husain (Ed.) (2011) {\sl A Concise History of Islam}. New Delhi: Vij Books India. 

\bibitem[]{chapter7_item37}
Tadpatrikar, S. N. (Ed.) (1946) {\sl Caura-pañcāśikā, an Indian Love Lament of Bilhana Kavi}. Poona: Poona Oriental Series No.~86.

\bibitem[]{chapter7_item38}
Tapasyananda, Swami (Ed.) (2003) {\sl Bhāgavatapurāṇa}. Chennai: Ramakrishna Ashram.

\end{thebibliography}

\theendnotes
