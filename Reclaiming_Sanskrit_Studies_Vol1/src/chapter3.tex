\chapter[Pollock’s Idea of a “National-Socialist Indology”]{Sheldon Pollock’s Idea of a “National-Socialist Indology”}\label{chapter3}

\Authorline{Koenraad Elst}
\lhead[\small\thepage\quad Koenraad Elst]{}

\section*{Abstract}

Sheldon Pollock is by no means the first one to build on the mythology that has overgrown the factual core of a link between racism in general, National-Socialism in particular, and the study of Indo-European and Sanskrit. In his case, the alleged National-Socialist connection of Sanskrit is heavily over-interpreted and emphatically taken to be causal, as if the interest in Sanskrit has caused the Holocaust. We verify the claims on which he erects this thesis one by one, and find them surprisingly weak or simply wrong. They could only have been made in a climate in which a vague assumption of these links (starting with the swastika, which in reality was not taken from Hinduism\index{Hinduism}) was already common. Yet, even non-specialists could easily have checked that Adolf Hitler\index{Hitler!Adolf} expressed his contempt for Hinduism, repeatedly and in writing.

Pollock’s attempt to even link the Out-of-India Theory with the Nazi\index{holocaust!Nazi} worldview is the diametrical opposite of the truth; it was the rivalling Aryan Invasion Theory\index{Aryan!Invasion Theory} (which Pollock himself upholds) that formed the cornerstone and perfect illustration of the Nazi worldview. This linking could only pass peer review because of the general animus against Hinduism and Indo-European indigenism in American academe. The whole forced attempt to associate Hinduism\index{Hinduism} with National-Socialism\index{National-Socialism} suggests a rare animosity against Hinduism.  

\section*{Sheldon Pollock’s Idea of a “National-Socialist Indology”}

Sheldon Pollock, professor of Sanskrit at Columbia University, links Sanskrit with the Holocaust, no less. Or at least, his critics (Malhotra 2016, Paranjape 2016) cite him to that effect. But according to Tony Joseph\index{Joseph, Tony} (2016), one of Pollock’s declared defenders, “the anti-Pollock campaign is based on quotes stripped out of their context”. Be that as it may in general, in the present paper we will take care to put Pollock’s utterances on the links between the Sanskrit tradition and National-Socialism\index{National-Socialism} (NS, Nazism) in context. As it happens, in this case, there is simply no context that could possibly render Pollock’s position harmless, for it is the single worst allegation possible in contemporary Western culture, viz. responsibility for the Nazi Holocaust\index{holocaust!Nazi}. Nevertheless, we will give due attention to this context.

A somewhat frivolous element from the immediate context was that NS Germany was hosting Indian freedom fighter Subhas Chandra Bose\index{Bose, Subhas Chandra}. (“Jana Gana Mana\index{Jana Gana Mana}, independent India’s national anthem\index{national!anthem}, was first performed in an imposing celebration at the inauguration of the Deutsch-Indische Gesellschaft\index{Deutsch-Indische Gesellschaft@\textsl{Deutsch-Indische Gesellschaft}} in Hamburg on 11 September 1942, for Bose had chosen it as free India’s national anthem\index{nationa!anthem}”. (Hartog\index{Hartog, Rudolf} 2001:iv))  But that context can perhaps best be left undiscussed, for the regiment that Bose had formed, was anything but Hindu traditionalist. Far from inspiring caste hierarchy\index{caste!hierarchy} into their NS hosts, Bose’s soldiers were not organized by caste\index{caste}, unlike those in the British-Indian army\index{British-Indian army}. Bose was a socialist, a progressive, on the same anti-caste wavelength as Pollock; but the Nazis never held that against him. So let us see what Pollock himself chooses to focus on.

\section*{The centuries before National-Socialism}

In his overview of pre-Nazi German Indology\index{pre-Nazi German Indology}, Pollock only follows the received wisdom when he stereotypes 19th-century German scholarship. He credits the German Orientalists with the creation of {\sl Wissenschaft\index{Wissenschaft@\textsl{Wissenschaft}},} the scholarly canons\index{scholarly canons} that became normative in academe worldwide, with an ideal of objectivity\index{objectivity} (as in the definition of historiography\index{historiography} by Leopold von Ranke\index{Ranke, Leopold von}, mentioned by Pollock (1993:84): reconstructing events “as they have been in reality”); but also with a frantic search for a German national identity. In that project, they used the Sanskrit tradition, viz. as partially preserving the “Aryan”\index{Aryan} identity\index{Aryan identity@“Aryan” identity} that, through Indo-European (Indogermanisch or Arisch\index{Indogermanisch or Arisch}) linguistic unity, could inspire any German cultural self-identification.

Like so many contemporary scholars of the field, including Hindus and Muslims, he assumes the conceptual framework of Edward Said’s\index{Said, Edward} thesis {\sl Orientalism}\index{Orientalism} (1978). There, Said\index{Said, Edward} analyses the academic discipline of Orientalism\index{Orientalism} (formally: “Oriental Philology and History”, as still on this writer’s diploma) as but an instrument for control, a method used by the colonial powers to dominate the natives of the Orient. This claim has been lambasted as riddled with factual errors and as conceptually a conspiracy theory by Irwin\index{Irwin, Robert} (2006), Warraq\index{Warraq, Ibn} (2007) and Elst\index{Elst} (2012), but has taken academe by storm and is now cited as gospel.

Pollock (1993:114) gives a nod to the existence of a “pre-orientalist orientalism”\index{pre-orientalist orientalism}, a scholarship originating in the Catholic Church’s\index{Catholic church} late-medieval attempts to reconnect with the Oriental Churches and to study its Islamic rival, when Europe had no colonies yet. Yet he formulates his own career project in Saidian terms: “Moving beyond orientalism finally presupposes moving beyond the culture of domination and the politics of coercion that have nurtured orientalism in all its varieties, and been nurtured by it in turn.” (Pollock 1993:117)

The expression “orientalism in all its varieties” refers, among other things, to the special nature of Orientalism in its mainstay, German-speaking Central Europe. Unlike France and England, Germany had initially no colonies, and later only some colonies far from the Orient. This fact should refute Said’s\index{Said, Edward} whole thesis, but he neutralizes it by keeping German Orientalism outside the purview of his book. (Later, this gap was filled by Marchand\index{Marchand, Suzanne} 2009 and, for Indology, by Adluri \& Bagchee\index{Adluri \& Bagchee} 2014) According to Pollock (1993:83), Germany practised its own Orientalism in an attempt to solve its “problem of identity”, and more consequentially, “to colonize Europe, and Germany itself, from within”, in a “German allomorph of British imperialism\index{imperialism}”. And here we see, with a big stretch, the connection between the proverbial German scholarship of the 19th century and the Nazi project of empire. Just as British Orientalism was evil by being a concoction in the service of empire-building, German Orientalism was evil by its connection with hyper-nationalism and genocide\index{genocide@\textsl{genocide}}. That, at least, is Pollock’s message.

\section*{Nazi investment in Indology}

Modern scholars devote a lot of time to studying every possible dimension of the NS polity and ideology, including the work of anyone thought to have NS ties, such as the philosopher and NS party member Martin Heidegger:\index{Heidegger, Martin} ‘Like the predicament of Indology, that of humanistic studies in general has belatedly seized the attention of scholars, as {\sl Der Fall Heidegger} [“the case of Heidegger”] demonstrates.’ (Pollock 1993:112)

And so, Pollock includes his own specialism, Indology, among the culprits. The indictment:
\begin{myquote}
“In German Indology of the NS era, a largely nonscholarly mystical nativism deriving ultimately from a mixture of romanticism and protonationalism merged with that objectivism of Wissenschaft earlier described, and together they fostered the ultimate ‘orientalist’ project\index{orientalist project}, the legitimation of genocide.”\index{genocide} \hfill(Pollock 1993:96)
\end{myquote}

The allegation is extremely serious. Yet, he never quotes any of those Indologists as actually declaring they want genocide; or that the aim of their professional choice is genocide. Since, moreover, genocide is somehow declared to be the ultimate finality of Orientalism, this seems also to indict the French and British Orientalists. Alright, he fails to prove that rather unlikely point; but what does he prove?

He claims a “substantial increase in the investment on the part of the NS state in Indology and ‘Indo-Germanistik.’ Both [Heinrich] Himmler\index{Himmler, Heinrich} and [Alfred] Rosenberg\index{Rosenberg, Alfred} sponsored institutes centrally concerned with ‘Indo-Germanische Geistesgeschichte\index{Geistesgeschichte}’.” (Pollock 1993:95) Note the sly cursory shift from Indology to “Indo-Germanistik”: the former deals with India and, as we shall see, received no increased interest from the NS regime at all; while the latter deals with Europe and especially Germany, the putative racial heir of the Indo-Europeans, and became the centre of NS attention. 
\newpage

In fact, the proof he himself adduces, proves something else than an increased German interest in Indology. He compares a total of 26 full professors of "Aryan" orientalism\index{Aryan!orientalism} in Germany with just 4 in England, the colonial metropole – but for the year 1903 (detailed in Rhys-Davids\index{Rhys Davids, T W} 1904). This had nothing to do with a NS penchant for India, for this special German interest in the Orient existed since long before (and incidentally, makes minced meat of Said’s\index{Said, Edward} linking Orientalism with the colonial entreprise).

In support, Pollock refers to a primary source, the {\sl Minerva Jahrbücher\index{Minerva Jahrbücher@\textsl{Minerva Jahrbüche}},} an annuary with academic data. But after thoroughly checking these, the contemporary German Indologist from Göttingen, Reinhold Grünendahl\index{Reinhold Grünendahl} (2012:95), shows these data to confirm an uneventful continuity with pre-NS days: 
\smallskip

\begin{myquote}
“As was the case with the 1933/34 volume and Rhys Davids’s\index{Rhys Davids’s} paper of 1904, none serve to corroborate Pollock’s presumptions\index{presumptions}. The same holds for recent evidence-based studies that in any way pertain to such issues [$\ldots$], all of which confirm that Pollock’s deep ruminations on ‘the political economy\index{political!economy} of Indology in Germany in the period 1800–1945’ (1993: 118n5) are entirely unfounded. Nevertheless, his attendant admonition that this is an ‘important question’ awaiting ‘serious analysis’ (118n5) has become a kind of gospel, recited by others [$\ldots$] with increasing confidence, but with very little to show as yet in terms of substantiation. Yet, all this while, dozens of ‘histories of German Indology’ are built on the—still unfulfilled—promises of that gospel.”
\end{myquote}
\smallskip

In his researches, Grünendahl\index{Grünendahl} (2012:194) has checked Rhys-Davids’ writings and discovered a telling example of how the racialist “NS” worldview was already present in Britain earlier:
\smallskip

\begin{myquote}
“However, a more important factor seems to me to be Rhys Davids’s racialist—or more precisely Aryanist—bias\index{bias}, documented, for example, in statements to the effect that Gautama Buddha\index{Gautama Buddha} ‘was the only man of our own race, the only Aryan, who can rank as the founder of a great religion’ and that therefore ‘the whole intellectual and religious development of which Buddhism\index{Buddhism} is the final outcome was distinctively Aryan, and Buddhism is the one essentially Aryan faith’ (1896:185), which ‘took its rise among an advancing and conquering people full of pride in their colour and their race$\ldots$  ‘ (1896:187).”
\end{myquote}
\smallskip

But Pollock (1993:94) has a point when he notes that a number of Indologists were NS party members or SS officials, e.g. Walther Wüst\index{Wüst, Walther}, Erich Frauwallner\index{Frauwallner, Erich}, Jakob Hauer\index{Hauer, Jakob}, Richard Schmidt\index{Schmidt, Richard}. He estimates these as one-third of the Indology professors. He admits that there was a silent opposition too, e.g. when Wüst became board member of the {\sl Deutsche Morgenländische Gesellschaft\index{Deutsche Morgenländische Gesellschaft}} (“German Oriental Society\index{German Oriental Society}”), “the aged Geiger [= Wilhelm Geiger, Wüst’s mentor] objected privately to the behavior of Wüst.” (Pollock 1993:123) And that may have been the tip of an iceberg, though we will not hazard a guess on the proportion of open resisters, silent resisters, fence-sitters and collaborators.

\section*{Functionalism}

Pollock gives the impression of a rather shaky grasp on NS history. It is, after all, not his field. Rather than properly delving into it in preparation of this ambitious paper, he seems to have gone by the received wisdom prevalent in his own liberal circles. The charitable explanation is that, not being a historian of WW2, he simply overstepped the boundaries of his competence. The alternative is that he deliberately forged this claim about Indology and National-Socialism\index{National-Socialism} as a weapon, in this case against the Sanskrit tradition. Some popular writers (e.g. Pennick\index{Pennick} 1981) have indeed done something similar, often after classifying Hindu ideas like reincarnation in the “occult” category. They have correctly sensed the windfall of benefits, whether political or commercial, guaranteed to whomever manages to instrumentalize references to National-Socialism\index{National-Socialism}.

He claims that “the extermination\index{extermination} of the Jews would seem to pose a serious challenge to any purely functionalist\index{functionalist} explanation of National Socialism.” (Pollock 1993:88) Perhaps there was still some room for doubt in 1993, and though marginal, the hypothesis of “intentionalism\index{intentionalism/intentionalist}”,\index{hypothesis!intentionalism} viz. that the genocide\index{genocide} of the Jews was planned since the beginning of the Nazi movement, was still in existence in erudite company. This hypothesis would imply, in a maximalist interpretation (viz. that any supporter of National-Socialism wilfully supported all of its programme), that NS Indologists like Fauwallner or Hauer supported “genocide”.

Strictly speaking, Hindus have no reason to defend these Indologists, for none of them was Hindu, and they projected a non-Hindu NS framework onto Hindu texts\index{texts (Sanskrit/Indic)}. Given the complexity of the reasons for a man’s inclinations, their interest in the Sanskrit traditions implies nothing at all about the Sanskrit tradition itself. Nevertheless, it is worth observing that even if these scholars were party members, this does not mean they supported genocide\index{genocide}, for that item was not on the party programme. Anti-Semitism\index{Anti-Semitism} is bad (and of that, they can certainly be held guilty) but genocide\index{genocide} is something else again. Even when it was later decided upon, it was still carried out in secrecy because the top Nazis knew that it would offend the German population including many party members.

Today, no historian worth his salt takes this intentionalism\index{intentionalism/intentionalist} serious anymore, though it lives on in Hollywood stereotyping. Apart from being unsupported by facts, intentionalism also sins against the reigning postmodernist canon by being “essentialist\index{essentialist}”, i.e. positing an irreducible unchanging nature to NS ideology; when even that turns out to be historical and changeble under the impact of circumstances. The “functionalist\index{functionalist}” hypothesis\index{hypothesis!functional} has won the day, viz. that the idea of genocide\index{genocide} only came about in a chain of unforeseen decisions under war circumstances in 1940-41. Even then it was carried out in secrecy: as late as 1943, Jewish Councils\index{Jewish!Councils} in occupied countries co-operated in the deportation of their own community, thinking Auschwitz\index{Auschwitz} was merely a labour camp. You could be a NS party member and support the idea of a Jew-free Europe (through emigration, as had happened in the 1930s) yet not support nor even know about genocide\index{genocide}. To be sure, ethnic cleansing\index{ethnic cleansing} is reprehensible too, but it is not genocide\index{genocide}. Scholars ought to exercise a sense of proportion.

For political campaigners living in the relative comforts of the post-war era, it is easy to laugh at the German commoners’ 1945 profession: {\sl Wir haben das nicht gewusst} (“We didn’t know about it”); but very often, it was formally the truth. Again, supporting the NS regime was bad enough; but it was something else than support for genocide\index{genocide}. Pollock, voracious quoter that he is, can at any rate not cite any NS Indologist as expressing support for genocide\index{genocide}.

But suppose, just suppose, that tomorrow, an incriminating statement by one of those Indologists gets discovered. Well, he is a human being, susceptible to all kinds of influences, not just those from his field of specialism. It would then still remain to be proven that the Sanskrit tradition which he studied, had given him this inspiration for genocide\index{genocide}. Sanskrit writers can be accused of teaching inequality through caste\index{caste!inequality}, and Pollock does indeed do so, but it has not been quoted here (nor, as far as is known, anywhere else) that the Vedas\index{Veda} or {\sl Śāstras} preach genocide\index{genocide}.

Now, if Vedic literature ever enjoined (not even just recounted, but actually enjoined) genocide\index{genocide}, there is no doubt that the Dalit movement, the missionaries\index{missionaries}, the “secularists\index{secularists}”, the Khalistanis\index{Khalistanis}, the many anti-Hindu\index{anti-Hindu} India-watchers\index{India-watchers}, and perhaps Pollock himself, would gleefully quote it. Not that they ever quote the instances of genocide\index{genocide} in the Bible or in the traditions about Mohammed, but for Hinduism\index{Hinduism} they would not be that polite. Indeed, it would have been logical to quote it very prominently in this very paper, as it would prove its entire point. But it seems not to exist.

\section*{Anti-Semitism and the Sanskrit Tradition}\index{anti-semitism}

While genocide\index{genocide} entered the mind of some top Nazis only by 1941, another element always associated with the NS ideology was intrinsic to it since the beginning, and very prominent in its writings since ca. 1920 and in policies since 1933: anti-Semitism\index{anti-Semitism}. While Pollock takes as a matter of course that to the Nazis, “Aryan”\index{Aryan!orientalism} was the opposite of “Semitic\index{Semitic}”, he doesn’t furnish any fact or quote at all that would meaningfully link this with the Sanskrit tradition.

He correctly notes that the NS view of the Vedas\index{Veda} was through an anti-Semitic lens: 
\begin{myquote}
“The {\sl Ŗg-Veda} as an Aryan\index{Aryan!orientalism} text\index{Aryan text} ‘free of any taint of Semitic contact’; the ‘almost Nordic zeal’\index{Nordic!zeal} that lies in the Buddhist\index{Buddhist} conception of the {\sl marga} [way]; the ‘Indo-Germanic religion-force’ of yoga; the sense of race\index{Nordic!race} and the ‘conscious desire for racial protection’; the ‘{\sl volksnahe} kingship’\index{volksnahe kingship} such is the meaning of the Indo-Aryan past for the National Socialist present, a present that, for Wüst, could not be understood without this past.”~\hfill(Pollock 1993:89)
\end{myquote}

Yes, that is how the Nazis saw it because they were racist and anti-Semitic\index{anti-semitism} to begin with. But the knowledge of the Sanskrit tradition could add absolutely nothing to that. There was nothing in the Vedas\index{Veda} themselves that suggested anti-Semitism, it was entirely in the eye of the beholder. Anti-Semitism\index{anti-semitism} existed in Europe ever since the people became convinced, through Christianity, that the Jews\index{Jews} had been responsible for Jesus’ death. Modern nationalism added an ideal of homogenization\index{homogenization}, so that Jews\index{Jews} were wished away as an obstacle to this ideal. By contrast, Vedic literature\index{Vedic literature} doesn’t know of Jews at all, and Hindu history has only shown a pluralistic hospitality for the Jewish communities on the Malabar coast, complete with their distinctive traditions.

In fact, that pluralism\index{pluralism} and respect for different identities could be cited as a redeeming feature of the caste system\index{caste!system} targeted so systematically by Pollock. Hindu reformers, who deny the intimate link between caste and Hinduism\index{Hinduism} during the past twenty or so centuries, are wrong. But they might start by enumerating some redeeming features, such as the sense of belonging-to (rather than of being-excluded-from), and pluralism. Within the caste framework, nobody’s comfort or identity was threatened when a foreign community was added. At any rate, the really existing caste system did not include anti-Semitism\index{anti-Semitism}.

The only thing Pollock can come up with here, in a footnote, is this: “The ratio {\sl ārya : caṇḍāla}\index{caṇḍāla} [outcaste, untouchable] :: German : Jew was made already by Nietzsche”. (Pollock 1993:119) Friedrich Nietzsche was popular among top Nazis, but they read him selectively, leaving out the positions that would not have endeared him to an NS regime. He disliked German nationalism and especially anti-Semitism\index{anti-semitism} as, at least, vulgar.

Nevertheless, it is true that Nietzsche {\sl (Twilight of the Idols\index{Twilight of the Idols}: The ‘Improvers’ of Mankind 3)} speculated on the Jews’ origin as emigrated {\sl Caṇḍālas}, Untouchables, based on more speculations by the amateur-Indologist Louis Jacolliot\index{Jacolliot, Louis}. We have discussed this question in full detail elsewhere (Elst 2008), but briefly, it all hinges on the mistranslation of the word {\sl dauścarmya}\index{dauścarmya}, “having a skin defect”, from the enumeration of {\sl Caṇḍāla} traits in the classical law code {\sl Manu Smṛti} (10:52, but based on 11:49). This word was understood as “missing a piece of skin”, hence “circumcised”. In reality, Manu nowhere mentions circumcised ones, whether Jews or Muslims.

Moreover, Nietzsche’s account doesn’t fit the neat scheme given by Pollock. Nietzsche recognized that in some ways, Jews do not fit the dirty and submissive stereotype of outcastes at all, and have been characterized by Aryan\index{Aryan!orientalism} traits ever since their entry into history. They ennobled themselves by becoming warriors and conquering their “promised land”. They are stereotypically very money-savvy, like the trader caste, and their obsessive purity rules and book-orientedness remind one of the Brahmin caste. Whereas Untouchables do the dirty work at funerals, Jewish priests or {\sl Kohanim} are expected to stay away from corpses. Jews were demonized by the Nazis, but not as low-castes\index{caste!inequality}.

Unlike the stereotype of {\sl Caṇḍālas} (more applicable to the Gypsies, known to descend from Indian low-castes\index{caste!inequality} and despised by the Nazis), the Jews were considered as rich, powerful, manipulative and extremely clever. Jews definitely did not relate to Germans the way Hindu low-castes\index{caste!inequality} relate to the upper castes\index{caste!inequality}. And anyway, the NS policy regarding the Jews was not based on this Nietzsche quote.

\section*{The “Aryan” homeland question}\index{Aryan!homeland|(}

The official birth of Indo-European linguistics by William Jones’s\index{Jones, William} famous Kolkata speech in 1786 (set on a scientific footing by Franz Bopp\index{Bopp, Franz} in 1816, as recognized by Pollock 1993:84) set in motion the search for the original homeland of this language family. The initial favourite was India, as famously stated by Friedrich von Schlegel\index{Schlegel, Friedrich von} (cited by Pollock 1993:85). In the present context, it might be a significant detail that Schlegel “married the daughter of the Jewish philosopher Moses Mendelssohn”, for which he was reproached as “missing racial instinct”. (Poliakov\index{Poliakov, Léon} 1971:217) Many other scholars from this period can be cited to the same effect, e.g. in 1810, Jakob-Joseph Görres had Abraham come from Kashmir. (Poliakov\index{Poliakov, Léon} 1971:219)

In the next decades, the putative homeland was relocated westwards, but Pollock (1993:77) claims that the Germans “continued, however subliminally, to hold the nineteenth-century conviction that the origin of European civilization was to be found in India (or at least that India constituted a genetically related sibling)”. 

This insinuates that the Nazis still believed in an Indian homeland whereas the British and their allies had long converted to the idea of a more westerly homeland. This much is true, that the NS state was intensely interested in the question of Indo-European origins: Communist states “employ myths of utopia\index{utopia}, while fascist systems employ myths of origins”. (Pollock 1993:85) But then, Pollock artfully smuggles in a continuity between the earlier Out-of-India hypothesis\index{hypothesis!out of India} and the preferred NS location of the homeland. The NS state’s legitimation by “Aryan” origins “had been provided early in nineteenth-century Indian orientalism; a benchmark is Friedrich von Schlegel's\index{Schlegel, Friedrich von} identification (1819) of the ‘Arier\index{Arier}’ as ‘our Germanic ancestors, while they were still in Asia’ (Sieferle\index{Sieferle} 1987:460). In the later NS search for authenticity, Sanskritists, like other intellectuals ‘experts in legitimation’, as Gramsci\index{Gramsci, Antonio} put it, did their part in extrapolating and deepening this discourse.” (Pollock 1993:85; Antonio Gramsci was the Italian Communist leader who, ca. 1930, theorized the acquisition of cultural hegemony as a prerequisite for political revolution.)

Note also this adroit suggestion of Indian paternity in this quotation from the NS Indologist Walther Wüst: 
\begin{myquote}
“I know of no more striking example of this hereditary, long-term tradition than the ingenious synopsis contained in the brief words of the Führer\index{Führer} and the longer confession of the great aryan personality of antiquity, the Buddha\index{Gautama Buddha}. There is only one explanation for this, and that is the basic explanation for components of the National-Socialist world-view: the circumstance, the basic fact of racial constitution\index{racial!constitution}. And thanks to fate, this was preserved through the millennia$\ldots$ [through] the holy concept of ancestral heritage [Ahnenerbe\index{Ahnenerbe}].”\hfill (Pollock: 1993:90)
\end{myquote}

Note that in all Pollock’s quotations from NS Indologists, only one Hindu is mentioned by name, repeatedly: the Buddha. He tries to make the {\sl Mīmāṁsā} thinkers with their chiseling of {\sl Śāstra} law into an inspiration to the Nazis (as if they needed {\sl Mīmāṁsā} to conceive of inequality), but never manages to find a Nazi quote about them. None, for example, about the 12th-century {\sl Śāstra} commentator Bhaṭṭa Lakṣmīdhara\index{Bhaṭṭa Lakṣmidhara}, whom he himself drags in frequently as justifying societal hierarchy. On the other hand he presents the Buddha as an antidote to Vedic inequality, yet that same Buddha turns out to be very popular among the Nazis.

There was no NS belief in an Indian homeland at all, on the contrary. To be sure, there was a belief in a racial kinship with the ancient “Aryans”\index{Aryan!homeland} in India, freshly invaded from their more westerly homeland. Of Nordic\index{Nordic!race} origin, these Aryans brought their talents into India and gave expression to them in ancient writings, and these naturally showed a kinship with Greek thought and other “Aryan” achievements in Europe.

Thus, NS Indologist Erich Frauwallner\index{Frauwallner, Erich} says, in Pollock’s account (Pollock 1993:93): 
\newpage

\begin{myquote}
“Frauwallner argued that the special meaning of Indian philosophy lay in its being ‘a typical creation of an aryan people’, that its similarities with western philosophy derived from ‘the same racially determined talent’, and that it was a principal scholarly task of Indology to d emonstrate this fact. Reiterating an axiom of NS doctrine, that ‘Wissenschaft\index{Wissenschaft@\textsl{Wissenschaft}} in the strict sense of the word is something that could be created only by nordic Indo-Germans’, Frauwallner adds, ‘From the agreement in scientific character of Indian and European philosophy, we can draw the further conclusion that philosophy as an attempt to explain the world according to scientific method is likewise a typical creation of the Aryan mind.’"
\end{myquote}

India was important to the Nazis not because they saw it as their ancestral land (they did not), but because it illustrated the Nazi worldview: 
{%\renewcommand\labelenumi{(\theenumi)}
\begin{enumerate}
\item dynamic white Aryans enter the land of indolent dark people; 
\item they subjugate them in a racially-conceived caste system\index{caste!system}, a kind of Apartheid\index{Apartheid}; 
\item in spite of trying for racial purity, they succumb too often to the charms of native women, so they racially degenerate; and therefore, 
\item they ended up overpowered by whiter races, first the Turks\index{Turks} and then, mercifully, their own purer Aryan\index{Aryan!homeland} cousins from Britain. 
\end{enumerate}}
Since the mid-19th century, this worldview had already been promoted by Britain, meanwhile it had been fortified by the ascendence of Darwinism\index{Darwinism} (“struggle for life”, “survival of the fittest”) and finally acquired an extra intensity in NS Germany.

In the first years of the renewed debate on the Indo-European homeland, it was a common confusion that the Out-of-India theory\index{Out-of-India Theory}, disappeared after Schlegel\index{Schlegel, Friedrich von} but now back in strength, had something NS about it, e.g. Zydenbos\index{Zydenbos} 1993. This confusion was deliberately fostered by some Indian “secularists”\index{secularists} trying to criminalize the Indian homeland hypothesis\index{hypothesis!Indian homeland}, e.g. Sikand\index{Sikand, Yoginder} 1993. In reality, the NS theorists as well as the NS textbooks emphasized and highlighted the putative European homeland and concomitant invasion into India. Zydenbos and Sikand themselves were in Hitler’s\index{Hitler!Adolf} camp.

The confusion centred on the word {\sl Aryan,} of Sanskrit origin. In NS discourse, this was routinely interpreted in a racial sense, though race\index{Nordic!race} was here not just skin colour but also had a cultural component. Thus, both Sanskrit and German could only be such precise and structured languages\index{structured languages} because they emanated from racially predisposed nations (and the “degeneration” of Sanskrit to the much looser modern Indian languages is thus a consequence of racial degeneration). In vulgar propaganda, the linguistic, cultural and biological dimensions were completely confounded, but this even influenced high-brow discourse: 
\begin{myquote}
“An example of this more sophisticated orientalism is the work of Paul Thieme\index{Thieme, Paul}”, Harvard Sanskritist Michael Witzel’s\index{Michael Witzel’s} revered teacher, esp. his “analysis of the Sanskrit word ārya, where at the end he adverts to the main point of his research: to go beyond India in order to catch the ‘distant echo of Indo-germanic\index{Indo-germanic} customs’.”\hfill (Pollock 1993:91)
\end{myquote}

From Adolf Hitler’s\index{Hitler!Adolf} mouth, no words in praise of Hinduism\index{Hinduism} are known. A few expressions of contempt, yes, e.g. Subhas Bose’s\index{Bose, Subhas Chandra} army: “Hitler himself ridiculed the 3000-man strong regiment of Indians.” (Hartog\index{Hartog, Rudolf} 2001:iii) But there is one statement of importance in the present context. Among Hitler’s rare utterances on the Hindus was a racial interpretation of the AIT. These are his own words (Jäckel \& Kuhn\index{Jäckel \& Kuhn} 1980:195, quoting Hitler 1920): 
\begin{myquote}
{\sl “Wir wissen, dass die Hindu in Indien ein Volk sind, gemischt aus den hochstehenden arischen Einwanderern und der dunkelschwarzen Urbevölkerung, und dass dieses Volk heute die folgen trägt; denn es ist auch das Sklavenvolk einer Rasse, die uns in vielen Punkten nahezu als zweite Judenheit erscheinen darf.”}
(“We know that the Hindus in India are a people mixed from the lofty Aryan\index{Aryan!homeland} immigrants and the dark-black aboriginal population\index{aboriginal population}, and that this people is bearing the consequences today; for it is also the slave people of a race that almost seems like a second Jewry.”)
\end{myquote}

For Grünendahl\index{Grünendahl}, this is merely an example of how the primary sources of German history contradict the free-for-all that amateurs make of it. Joseph\index{Joseph, Tony} (2016) dismisses Grünendahl’s many factual data as only to be expected from a “German”, an {\sl ad hominem} against a whole nation, as well as a covert admission that he cannot refute even one of them. Grünendahl (2012:196) had noted the same inability in an earlier Pollock defender, Vishwa Adluri\index{Adluri, Vishwa} (2011): when challenging the many facts mustered by Grünendahl, Adluri neither shows Grünendahl’s data to be incorrect, nor does he bring other facts proving Pollock’s case: “a pitiable want of judgment as well as evidence.”


But the objective finality of Pollock’s thesis is more specific, viz. to blacken the Indian homeland hypothesis\index{hypothesis!Indian homeland} by associating it with National-Socialism. Reality, however, is just the opposite: more even than other Europeans, the Nazis espoused and upheld a westerly homeland and the invasion hypothesis\index{hypothesis!invasion}. This invasion happens to be a corner-stone of Pollock’s worldview, with invader castes\index{caste!invader} guilty of expropriating and subjugating the natives, who became the lower castes. Hitler\index{Hitler!Adolf}-Pollock, same struggle!

To end this discussion on an element of nuance, however, we have to note an odd passage.
Though Pollock repeatedly affirms NS belief in what is nowadays called the Out-of-India theory,\index{Out-of-India Theory} and assumes this throughout his paper, honesty demands that we mention how one time, very cursorily, he nods to the opposite (and true) scenario: 
\begin{myquote}
“From among the complexities of NS analysis of the Urheimat question\index{Urheimat@\textsl{Urheimat}!question} it is worth calling attention to the way the nineteenth-century view expressed by Schlegel\index{Schlegel, Friedrich von} was reversed: the original Indo-Europeans were now variously relocated in regions of the Greater German Reich\index{Greater German Reich}; German thereby became the language of the core {\sl (Binnensprache),} whereas Sanskrit was transformed into one of its peripheral, ‘colonial’ forms.”~\hfill(Pollock 1993: 91-92)
\end{myquote}

Even this is not entirely true, for the dominant opinion was that the homeland was to the east or southeast of Germany, while even the SS research department {\sl Ahnenerbe\index{Ahnenerbe}} explored locating the homeland in Atlantis\index{Atlantis}: fanciful, but at least outside Germany. Still, Pollock’s statement does justice to the NS worldview by denying India the honour of being the homeland. It explains why we often see a shift in the focus of NS Indologists from India to “Indo-European” or “Aryan”\index{Aryan!homeland} issues, racially identified with Nordic\index{Nordic}. Thus, in 1934 Jakob Hauer\index{Hauer, Jakob} still wrote on “Indo-Aryan” metaphysics of struggle, but in 1937 he published on the religious history of the “Indo-Europeans”. Pollock’s own enumeration of supposedly India-related activities usually confuses “Indian” with “Indo-European”, i.e. “Aryan” or essentially “Nordic”. It is only by confusing those two that an impression of a NS orientation towards India can be created.

\section*{National-Socialism and the Sanskrit tradition}

Two factors of a seeming connection between Hinduism\index{Hinduism} and NS Germany are unavoidable: the swastika\index{swastika} and the term {\sl ārya}\index{arya@ārya}. About the swastika, the matter is simple: it does not come from India. It is a more widespread symbol, very prominent e.g.\ in Troy\index{Troy}, excavated by a German, and part of Greek history which was a decisive inspiration to Germany’s academic culture. It was also very common in the Baltic area, where German army veterans formed Freikorps militias to defeat the Bolsheviks in 1918-20. When they came home, often to join nationalist parties, they brought the swastika with them.  (More detail in Elst 2007) The NS view was that it was Nordic in origin, and that it only became common in India after having been imported by the Aryans\index{Aryan!homeland}.

As Sünner\index{Sünner} (1999:66) puts it: “the hooked cross, which the Nordic cult of light carried into the Orient”. Indeed, this was part of a general septentriocentric worldview\index{septentriocentric worldview}: “Unconcerned about historic truth, their builders were termed as ‘Indo-European Urvolk’\index{Indo-European Urvolk} and ‘heroes of the North’, who later co-founded the civilizations of Egypt, India and Persia.” (Sünner 1999:60-61) India was not in the picture, except as a distant horizon to be conquered and civilized by Norsemen.

As for Aryan\index{Aryan!homeland}: “The term ārya\index{arya@ārya} itself merits intellectual-historical\index{intellectual history} study (and I mean diachronic analysis, not static etymology) for premodern India at least of the sort {\sl Arier}\index{Arier} has received for modern Europe.” (Pollock 1993:107) True, Hindus too might learn a lot from realizing that this term is historical, that it has gone through changes, and that the classical meaning “noble” (a meaning unattested in the {\sl Ŗg-Veda}) has mundane roots too. Yet, Pollock does no more than focus on the {\sl Manu Smṛti,}\index{Manusmṛti} fairly late and worn-out as a supposed repository of caste teachings and anti-egalitarian musings about {\sl ārya}\index{arya@\textsl{ārya}} vs. {\sl anārya}\index{anārya}.

Moreover, Pollock (1993: 107-8) brings in the concept of race: 
“From such factors as the semantic realm of the distinction {\sl ārya/anārya\index{anārya@\textsl{anārya}}} and the biogenetic map of inequality\index{biogenetic map of inequality} (along with less theorized material, from Vedic and epic literature, for instance), it may seem warranted to speak about a ‘pre-form of racism’ in early India (Geissen\index{Geissen, Imanuel} 1988: 48ff.), especially in a discussion of indigenous ‘orientalism’, since in both its classic colonial and its National Socialist form orientalism is inseparable from racism.” 

That is certainly the NS reading, but from a top Indologist, we might have expected an explanation of whether this was the Indians’ own intended reading. Pollock doesn’t go into this question at all but confidently assumes an indubitably positive answer. To exonerate him, we might take this as merely a logical application of the Aryan invasion scenario, firmly established since the mid-19th century: the Aryans came in, met a different race of aboriginals, and imposed a racial Apartheid\index{apartheid} on them: the caste system\index{caste!system}. So, in a way, the case against Pollock is the case against Western Indology as a whole.

His arguments about the inequality fostered by the {\sl Śāstras} thus always has a racial dimension. Caste is too big a subject to argue out here, but let us notice that he has no problem mustering incriminating quotations excluding low-castes from Vedic knowledge. He does acknowledge feeble dissenting voices, such as the early Mīmāṁsā thinker Bādari\index{Bādari}, who argued that Śūdras too can build Vedic fires for sacrifice, since "the Śūdra desires heaven, too ($\ldots$ ) and what is it in a sacrifice that any man can do but that the Śūdra is unable to do?" (Pollock 1993:109) But the over-all picture is decidedly inegalitarian. However, inequality is a nearly equally distributed good, and for that value, the Nazis could have found inspiration in other societies, if they needed any at all, such as the Arab or colonial slave systems.

Inequality yes, but racial, no. All this depends on the racial reading of Sanskrit concepts, starting with the enumeration of four social functions (without a word about how to recruit for them) in late-Vedic hymn RV 10.90\index{RV 10.90}, “the locus classicus in the Veda”\index{Veda} for caste division. Even “the biology” is said to be “of course latent in the RV passage”, though he does not say how and other readers cannot find a trace of it. (Pollock 1993:125) All this follows from the Aryan invasion scenario, and on this point, there is no chance of refuting Pollock unless the homeland\index{Aryan!homeland|)} debate is waged all over. 

At any rate, none of these interesting musings about Indian society played any role whatsoever in NS policies. Pollock (1993:86-87) is only projecting his own focus when his Indologist’s eye recognizes caste phenomena in NS policies: 
\medskip

\begin{myquote}
“The myth of Aryan origins burst from the world of dream into that of reality when the process of what I suggest we think of as an internal colonization\index{colonization} of Europe began to be, so to speak, shastrically codified, within two months of the National Socialists' capturing power\index{power, political} (April 1933). The ‘Law on the Reconstitution of the German Civil Service’, the ‘Law on the Overcrowding of German Schools’, and a host of supplementary laws and codicils\index{codicils} of that same month were the first in a decade dense with legal measures designed to exclude Jews and other minority communities from the apparatuses of power (including ‘authoritative’ power, the schools and universities), and to regulate a wide range of social, economic, and biological activities.” 
\end{myquote}

Finally, another point of difference between NS racism\index{NS racism} and caste society\index{caste!society} is this. The relation between the Germans and the “inferior” races was conceived as hierarchical, but the NS conception of German society was not as a hierarchy: all members of the master race were deemed reasonably equal. Hitler\index{Hitler!Adolf} himself had climbed up from down below; there was no objection against social mobility. Germans were not to be held inside the class they happened to be born into, unlike the low-caste Hindus in Pollock’s oft-stated view. 

In this connection, Hitler himself objected to the idea of a hereditary priesthood\index{hereditary priesthood}. Though he doesn’t name it, this could be applied to the Brahmin caste: “The Catholic Church\index{Catholic church} recruits its clergy in principle from all classes of society, without discrimination. A simple cowherd can become cardinal. That is why the Church remains combattive.” (Pierik\index{Pierik} 2012:83, quoting Hitler’s table talk of 2 November 1941) So, even in his anti-Brahminism\index{anti-Brahminism}, Pollock finds himself in Hitler’s camp.

\section*{The banalization of criminalizing Hinduism\index{Hinduism}}\index{Hinduism, criminalizing}

In his conclusion, Pollock digresses about the future of his scholarly discipine. There, he takes for granted the connection he has earlier posited between Sanskrit and the Holocaust\index{holocaust!Nazi}. In an entirely non-polemical tone, he off-hand builds on this putative connection: “How, concretely, does one do Indology beyond the Raj and Auschwitz in a world of pretty well tattered scholarly paradigms?” (Pollock 1993:114) 

In the process of disinformation\index{disinformation}, an idea is first launched and argued in high-brow papers like the {\sl New York Review}\index{New York Review@\textsl{New York Review}} or the {\sl Economic and Political Weekly\index{Economic and Political Weekly@\textsl{Economic and Political Weekly}}}; subsequently it is presented as the received wisdom in more general media, like the {\sl Washington Post}\index{Washington Post} or the {\sl Times of India}; but the final stage is when the idea is presented as a matter of course, and conveyed through popular media, women’s magazines etc. That is what completes the instilling of false ideas in the popular mind.

Similarly, to promote an idea intended to become a fixture in our worldview, it is useful to repeat it, first as a topical proposal to be proven, then as a theorem deemed to have been proven, finally as a truism on which other proposals can safely be built.

Yet, we don’t hold Pollock as an individual guilty of this disinformation. Though his authoritative voice does its bit to determine the Zeitgeist, he was mainly surfing on an already-existing Zeitgeist.

Firstly, the tendency to project the Nazi\index{holocaust!Nazi} episode onto something morbid and unique in the more distant German past, and particularly the exoticization\index{exoticization} of 19th-century Germany’s supposed self-doubt and search for an identity, was already very common in the preceding decades (e.g. Poliakov\index{Poliakov, Léon} 1971), and still is to some extent.

Secondly, the link between Hinduism\index{Hinduism} (as well as Lamaism\index{Lamaism}) and Nazi\index{holocaust!Nazi} culture had already been proposed by a number of writers. Moreover, it converges with a widespread revulsion among Westerners against the caste system\index{caste!system}, which they liken to slavery and identify, through the Aryan invasion\index{Aryan!Invasion Theory} hypothesis\index{hypothesis!invasion}, with racism. This misses the warlike element in National-Socialism, but that is slightly made up for by all the stories about Hindu riots and by the symmetry fallacy\index{symmetry fallacy} whenever Muslim violence comes in the news: “Ah, but all religions do it; Hinduism\index{Hinduism} must have a similar terrorism\index{terrorism}.” Even then, it still doesn’t have the element of “genocide\index{genocide}”, but Pollock remains determined to read that into it.

\section*{Conclusion}

It is one thing to hold a view that, upon analysis, turns out to be mistaken. To err is but human. However, one should become extra careful when the view one expresses, is an allegation. It becomes even more serious when it is the worst allegation one can possibly make, viz. the accusation of responsibility for the Holocaust\index{holocaust!Nazi}.

The situation with allegations is simple: either you prove them, or you yourself are guilty of slander. This then can be held against Pollock: he has made a grave allegation, yet has failed to buttress it with proof, though not for lack of trying.

The question which Hindus should contemplate, then, is this one. Should the Sanskrit tradition be given in care to a professor of Sanskrit who stands by such a grave though false allegation against it?

\newpage

\begin{thebibliography}{99}
\itemsep=2pt
\bibitem[]{chap1_item1}
Adluri, Vishwa (2011) “Pride and prejudice: Orientalism and German Indology”, {\sl International Journal of Hindu Studies}, 15, p.~253--294.

\bibitem[]{chap1_item1a}
Adluri, Vishwa and Bagchee, Joydeep, (2014) {\sl The Nay Science, A History of German Indology}. New York: Oxford University Press.

\bibitem[]{chap1_item2}
Breckenridge, Carol A., and Van der Veer, Peter, (Eds.) (1993) {\sl Orientalism and the Postcolonial Predicament: Perspectives on South Asia}. University of Pennsylvania Press.

\bibitem[]{chap1_item3}
Elst, Koenraad (2007) {\sl Return of the Swastika}.  Delhi: Voice of India.

\bibitem[]{chap1_item3a}
Elst, Koenraad (2008) “Manu as a Weapon against Egalitarianism: Nietzsche and Hindu Political Philosophy”, in {\sl Siemens \& Roodt} (2008), p.~543--582.

\bibitem[]{chap1_item3b}
Elst, Koenraad (2012) “The Case for Orientalism”, in {\sl The Argumentative Hindu. Essays by a Non-Affiliated Orientalist}. Delhi: Voice of India. p.~442--455.

\bibitem[]{chap1_item4}
Geissen, Imanuel (1988) {\sl Geschichte des Rassismus}. Frankfurt: Suhrkamp.

\bibitem[]{chap1_item5}
Grünendahl, Reinhold (2012): “History in the making: on Sheldon Pollock’s ‘NS Indology’ and Vishwa Adluri’s ‘Pride and prejudice’”. 
{\sl International Journal of Hindu Studies} no.~16, p.~189--252.

\bibitem[]{chap1_item6}
Hartog, Rudolf (2001) {\sl The Sign of the Tiger. Subhas Chandra Bose and his Indian Legion in Germany}, 1941--45. Delhi: Rupa.

\bibitem[]{chap1_item7}
Irwin, Robert (2006) {\sl For Lust of Knowing: the Orientalists and Their Enemies}. London: Allan Lane.

\bibitem[]{chap1_item8}
Hale, Christopher (2003) {\sl Himmler’s Crusade. The true story of the 1938 Nazi expedition into Tibet}. London: Bantam.

\bibitem[]{chap1_item9}
Hauer, Jakob Wilhelm (1934) {\sl Eine indo-arische Metaphysik des Kampfes und der Tat: Die Bhagavad Gita in neuer Sicht}. Stuttgart: Kohlhammer.

\bibitem[]{chap1_item10}
Hauer, Jakob Wilhelm (1937) {\sl Glaubensgeschichte der Indogermanen}, Vol.~1. Stuttgart: Kohlhammer.

\bibitem[]{chap1_item11}
Hitler, Adolf (1920) {\sl “Warum sind wir Antisemiten?”}, in Jäckel \& Eberhard (1980), p.~184--204.

\bibitem[]{chap1_item12}
Jäckel, Eberhard, and Kuhn, Axel, (Eds.) (1980) {\sl Hitlers sämtliche Aufzeichnungen}, 1905--1924

\bibitem[]{chap1_item13}
Joseph, Tony (2016) “Battle against Pollock”. {\sl Business Standard}. Delhi, 3 May 2016.

\bibitem[]{chap1_item14}
Malhotra, Rajiv (2015) {\sl The Battle for Sanskrit. Is Sanskrit Political or Sacred? Oppressive or Liberating? Dead or Alive?}. Delhi: HarperCollins.

\bibitem[]{chap1_item15}
Marchand, Suzanne (2009) {\sl German Orientalism in the Age of Empire - Religion, Race and Scholarship}. Washington DC/New York: German Historical Institute and Cambridge University Press. 

\bibitem[]{chap1_item16}
Paranjape, Makarand R. (2016) “The deepest Orientalist”. {\sl Business Standard}. Delhi, 29 April 2016.

\bibitem[]{chap1_item17}
Pennick, Nigel (1981) {\sl Hitler’s Secret Sciences}. Suffolk: Nevile Spearmen.

\bibitem[]{chap1_item18}
Pierik, Perry, (Ed.), (2012, 1980$^1$) {\sl Hitlers Tafelgesprekken 1941-44} (“Hitler’s Table Talk”), Soesterberg: Aspekt.

\bibitem[]{chap1_item19}
Poliakov, Léon (1987, 1971$^1$) {\sl Le Mythe Aryen. Essai sur les Sources du Racisme et des Nationalismes}. Paris: Editions Complexe.

\bibitem[]{chap1_item20}
Pollock, Sheldon (1993) “Deep Orientalism? Notes on Sanskrit and power beyond the Raj”, in Breckenridge \& Van der Veer (1993), p.~76--133.

\bibitem[]{chap1_item21}
Quinn, Malcolm (1994) {\sl The Swastika. Constructing the Symbol}. London: Routledge.

\bibitem[]{chap1_item22}
Rhys Davids, Thomas William (1896) {\sl Buddhism: Its History and Literature}. London: G.P. Putnam’s Sons (American Lectures on the History of Religions, First Series, 1894--1895).

\bibitem[]{chap1_item22a}
Rhys Davids, Thomas William (1903-4) “Oriental Studies in England and Abroad”, {\sl Proceedings of the British Academy}. London. pp.~183--97.

\bibitem[]{chap1_item23}
Said, Edward (1995, 1978$^1$) {\sl Orientalism}. London: Penguin.

\bibitem[]{chap1_item24}
Schlegel, Friedrich (1808) {\sl Über die Sprache und Weisheit der Indier}. Heidelberg: Mohr.

\bibitem[]{chap1_item25}
Sieferle, Rolf Peter (1987) "Indien und die Arier in der Rassentheorie". {\sl Zeitschrift für Kulturaustausch} 37, pp.~444--67.

\bibitem[]{chap1_item26}
Siemens, Herman, and Roodt, Vasti, (Eds.) (2008) {\sl Nietzsche. Power and Politics}. Berlin: Walter de Gruyter.

\bibitem[]{chap1_item27}
Sikand, Yoginder (1993) “Exploding the Aryan myth”. {\sl Observer of Business and Politics}, 30-10-1993.

\bibitem[]{chap1_item28}
Sünner, Rüdiger (1999) {\sl Schwarze Sonne. Entfesselung und Missbrauch der Mythen in Nationalsozialismus und rechter Esoterik}. Freiburg im Breisgau: Herder Spektrum.

\bibitem[]{chap1_item29}
Thieme, Paul (1938) {\sl Der Fremdling im Rigveda}. Leipzig: Brockhaus.

\bibitem[]{chap1_item30}
Warraq, Ibn (2007) {\sl Defending the West. A Critique of Edward Said’s Orientalism}. Amherst NY: Prometheus.

\bibitem[]{chap1_item31}
Zydenbos, Robert (1993) “An obscurantist argument”, {\sl Indian Express}, 12-12-1993.
\end{thebibliography}
