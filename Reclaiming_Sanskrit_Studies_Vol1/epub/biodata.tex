\makeatletter
\def\@makeschapterhead#1{%
  \vspace*{50\p@}%
  {\parindent \z@ \raggedright
    \normalfont
    \interlinepenalty\@M
    \LARGE \bfseries  #1\par\nobreak
    \vskip 20\p@
  }}
\makeatother

\chapter*{Our Contributors\\ {\rm\sl\small (in alphabetical order of last names)}}\label{contributors}

\lhead[\small\thepage]{}
\rhead[]{\small\thepage}
\chead[]{}
\cfoot[]{}



\section*{Naresh Prakash Cuntoor}

Dr.~Naresh Prakash Cuntoor is a Senior Research Scientist, Intelligent Automation Inc., Rockville, MD, US\@. He has an M.S.\ and PhD from the Department of Electrical and Computer Engineering, University of Maryland. His research interests include human activity recognition, scene understanding, perceptual organization and computer vision for robotics applications. He pursues Sanskrit with keen interest and is a volunteer for Samskrita Bharati USA.

\section*{Satyanarayana Dasa}

Satyanarayana Dasa is a BTech from IIT(Delhi) in Mechanical Engineering and MTech from the same institution in Industrial Engineering. After working in this line for a while in Mumbai and then in the US, he gave up that career in favour of a quest for his spiritual roots. He took up Sanskrit studies and studied the entire literature of the Gauḍīya Vaiṣṇava tradition under the tutelage of Guru Haridasa Shastri Maharaj and completed his PhD in Sanskrit from Agra University. Dr. Dasa is the founder of the Jiva Institute of Vedic Studies in Vrindavan.

\section*{Koenraad Elst}

Dr.~Elst is a scholar of Philosophy, Chinese and Indo-Iranian Studies and is well-known for this several books and articles on various India-centric issues such as the Ayodhya Ramjanmabhoomi issue and the Aryan question as well as those reflecting a deep study of Nazism amongst other things. Amongst the books he has penned are {\sl The Saffron Swastika}, {\sl Ayodhya and After: Issues before Hindu Society}, {\sl Negationism in India – Concealing the Record of Islam} and {\sl Psychology of Prophetism – a Secular look at the Bible}. 

\section*{K Gopinath}

Prof.~Gopinath is currently employed as Professor at the Indian Institute of Science in Bangalore in the Computer Science and Automation Department. He has a doctorate from Stanford, Master’s from UW Madison, BTech from IITM and has worked at AMD and Sun Microsystems Labs.

\section*{K S Kannan}

Prof.~K S Kannan is a Visiting Professor at the Centre for Ancient History and Culture, Jain University, Bengaluru, and is the Academic Director of the Swadeshi Indology Conference Series. He is a Former Director, Karnataka Sanskrit University and has a PhD in Sanskrit. His pursuits include research into Sanskrit literature and Indian Philosophy, and he is the author of about 20 books, prominent among them being {\sl Theoretical Foundations of Āyurveda} published by FRLHT, Bengaluru in 2007 and {\sl Virūpākṣa Vasantotsava Campū}, Annotated Edition, published by Kannada Vishvavidyalaya, Hampi, Karnataka in 2001. He is also the co-translator of {\sl Vibhinnate}, the translation into Kannada (2016), of the seminal work in English {\sl Being Different} written by Rajiv Malhotra.

\section*{Jayaraman Mahadevan}

Dr.~Jayaraman Mahadevan is currently serving as Director, Research Department, Krishnamacharya Yoga Mandiram, Chennai, Scientific Industrial Research Organisation. He has a PhD in Sanskrit from the Department of Sanskrit, University of Madras, His thesis was titled The Doctrine of Tantrayukti – A Study. He has presented 20 papers in various National and International conferences and has given talks in universities, colleges and institutions of national eminence. He has also written books, organised seminars, funded projects and been the guide for students pursuing their PhD\@. His areas of interest are Yoga, Tantrayukti, Vedanta, Sanskrit Poetic Literature and Manuscript Studies.

\section*{H R Meera}

H R Meera is an engineer graduate with BE degree from BMS College of Engineering, Bangalore. After serving as a consultant and software engineer for several years, she yearned to connect to her roots and pursued study of Sanskrit, procuring an MA\@. She is currently doing her PhD in interdisciplinary work at the National Institute of Advanced Studies, Bangalore. She is the co-translator of {\sl Vibhinnate}, the Kannada translation of Rajiv Malhotra’s Being Different.

\section*{Ashay Deepak Naik}

Ashay Naik is a software developer at Matific Ltd. and has just released his first book {\sl Natural Enmity: Reflections on the Niti and Rasa of the Panchatantra} [Book 1]. He has a Masters in Information Technology from the Queensland University of Technology, Australia and an Honours in Sanskrit from the University of Sydney, Australia.

\section*{Manogna H Sastry}

Manogna H Sastry is Chief Operations Officer as well as Mentor and Research Associate at the Centre for Fundamental Research and Creative Education, Bengaluru http://www.cfrce.com/manognahshastry.htm. Manogna has an M.S. from the Indian Institute of Astrophysics, Bengaluru, where her thesis was on Inflationary Cosmology. She is a keen student of Sanskrit literature and Indic studies, a passionate environmentalist and gardener.
