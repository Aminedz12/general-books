\chapter*{Series Editorial}\label{gen_editorial}

\lhead[\small\thepage\quad Western Indology \& Its Quest For Power]{}
\rhead[]{Series Editorial\quad\small\thepage}


It is a tragedy that many among even the conscientious Hindu scholars of~Sanskrit and Hinduism\index{Hinduism} still harp on Macaulay, and ignore others while accounting for the ills of the current Indian education system, and the consequent erosion of Hindu values in the Indian psyche. Of course, the machinating Macaulay brazenly declared that a single shelf of a good European library was worth the whole native literature of India, and sought accordingly to create “a class of persons, Indian in blood and colour, but English in taste, in opinions, in morals and in intellect” by means of his education system - which the system did achieve. 

An important example of what is being ignored by most Indian scholars is the current American Orientalism\index{American Orientalism}. They have failed to counter it on any significant scale. 

It was Edward Said (1935-2003) an American professor at Columbia University who called the bluff of “the European interest in studying Eastern culture and civilization” (in his book {\sl Orientalism} (1978)) by showing it to be an inherently political interest; he laid bare the subtile, hence virulent, Eurocentric prejudice aimed at twin ends – one, justifying the European colonial aspirations and two, insidiously endeavouring to distort  and delude the intellectual objectivity of even those who could be deemed to be culturally considerate towards other civilisations. Much earlier, Dr.\ Ananda Coomaraswamy (1877--1947) had shown the resounding hollowness of the {\sl leitmotif} of the “White Man’s Burden.” 

But it was given to Rajiv Malhotra, a leading public intellectual in America, to expose the Western conspiracy on an unprecedented scale, unearthing the {\sl modus operandi} behind the unrelenting and unhindered program for nearly two centuries now of the sabotage of our ancient civilisation yet with hardly any note of compunction.  One has only to look into Malhotra’s seminal writings - {\sl Breaking India} (2011), {\sl Being Different} (2011), {\sl Indra’s Net} (2014), {\sl The Battle for Sanskrit} (2016), and {\sl The Academic Hinduphobia} (2016) - for fuller details.
\vskip 1.5pt

This pentad - preceded by {\sl Invading the Sacred} (2007) behind which, too, he was the main driving force - goes to show the intellectual penetration of the West, into even the remotest corners (spatial/temporal/\break thematic) of our hoary heritage. There is a mixed motive in the latest Occidental enterprise,  ostensibly being carried out with “pure academic concerns”. For the American Orientalist doing his ``South Asian Studies'' (his new term for “Indology Studies”), Sanskrit is inherently oppressive - especially of Dalits, Muslims and women. And as an ``antidote", therefore, the goal of Sanskrit studies henceforth should be, according to him, to ``exhume and exorcise the barbarism'' of social hierarchies and oppression of women happening ever since the inception of Sanskrit - which language itself came, rather, from outside India. Another important agenda is to infuse/intensify animosities between/among votaries of Sanskrit and votaries of vernacular languages in india. A significant instrument towards this end is to influence mainstream media so that the populace is constantly fed ideas inimical to the Hindu heritage. The tools being deployed for this are the trained army of “intellectuals” - of leftist leanings and “secular” credentials.
\vskip 1.5pt

Infinity Foundation (IF), the brainchild of Rajiv Malhotra, started 25 years ago in the US, spearheaded the movement of unmasking the “catholicity” (- and what a euphemistic word it is!) of Western academia. The profound insights provided by the ideas of ``Digestion''\index{digestion} and the “U-Turn Theory” propounded by him remain unparalleled.
\vskip 1.5pt

It goes without saying that it is {\sl ultimately the Hindus in India who ought to be the real caretakers of their own heritage}; and with this end in view, {\bf Infinity Foundation India (IFI)} was started in India in 2016. IFI has been holding a series of Swadeshi Indology Conferences. 
\vskip 1.5pt

Held twice a year on an average, these conferences focus on select themes and even select Indologists of the West (sometimes of even~the East), and seek to offer refutations of mischievous  and misleading misreportages/misinterpretations bounteously brought out by these Indologists - by way of either raising red flags at, or giving intellectual responses to, malfeasances inspired in fine by them. To employ Sanskrit terminology, the typical secessionist misrepresentations presented by the West are treated here as {\sl pūrva-pakṣa}, and our own responses/rebuttals/rectifications as {\sl uttara-pakṣa} or {\sl siddhānta}. 

The first two conferences focussed on the writings of Prof.\ Sheldon Pollock, the outstanding American Orientalist (also of Columbia University, ironically) and considered the most formidable and influential scholar of today. There can always be deeper/stronger responses than the ones that have been presented in these two conferences, or more insightful perspectives; future conferences, therefore, could also be open in general to papers on themes of prior conferences.
\bigskip

\noindent
Vijayadaśamī\hfill	{\bf Dr.~K S Kannan}\\
Hemalamba Saṃvatsara\hfill Academic Director\\
Date 30-09-2017\hfill and\\	
\phantom{.}~\hfill General Editor of the Series                  
               

\chapter*{Volume Editorial}\label{editorial}

\vskip 9pt
\lhead[\small\thepage\quad Western Indology \& Its Quest For Power]{}
\rhead[]{Volume Editorial\quad\small\thepage}




Infinity Foundation India conducted two conferences in the recent past (in July 2016 and February 2017) which examined the impact of some of the writings of Prof.\ Sheldon Pollock of Columbia University. While the first conference had four themes, the second had six more; and all these ten topics pertained to the interpretations proffered by Pollock.
\vskip 1.5pt

While there is no intention of, or point in, targeting particular individuals, the focus on Pollock was due to the fact that he is the most formidable of the American Orientalists today, and his views and interpretations appear to Hindus to be most pernicious, nevertheless most pervasive in influence: a good many contemporary scholars, especially in the West  build on his pet ideas. 
\vskip 1.5pt

Pollock is obsessed with the notions of power and politics. As Malhotra points out, the word “power” occurs 600 times in his 2006 book {\sl The Language of Gods...}; and the words “political” and “politics” occur 900 times! There is no event or utterance, be it the most innocuous, where he cannot see some vicious play of power. Pollock cannot be too proud of this idiosyncrasy, for he has lucid company in Prof.\ O’Flaherty for whom the most innocent of moves can reek of sexual ramifications. (We promise our readers to reserve a couple of volumes for her too). Eight volumes are being presented now for dealing with the ideas of Pollock: six volumes of around 250 or more pages each and two monographs authored by two young Sanskrit scholars. Of these eight,  Volume 1 is being presented now.
\vskip 1.5pt

A conspectus of the various papers in this volume is quite in order here. This volume presents eight papers in two parts – three on what is perceived by the ilks of Pollock as the diabolical influence that Sanskrit and the {\sl Śāstra}-s have had, directly or indirectly, on Nazism; and five on the vile theme of the putative death of Sanskrit.

{\bf Part 1} of the book is devoted to the treatment of the supposed Sanskrit springs of the Nazi holocaust\index{holocaust! Nazi}: the weird and convoluted links Pollock belabours to manufacture between the two stand exposed here. 

The first paper authored by K Gopinath ({\bf Ch.\ 1}) exposes the propagandist designs evidenced in Pollock’s 1993 paper “Deep Orientalism: Notes on Sanskrit and Power beyond the Raj”. Gopinath's paper shows how Pollock subtly brings in extraneous elements, invokes {\sl selective} evidence that can suggest his own {\sl prefixed} conclusions, and employs other devices – all typical protocols of  propagandism.  Pollock supposes a trans-mogrification of German Indologists into supporters of Nazism, blatantly ignoring all the same, how, for example, famed German physicists and philosophers too vouched for their Nazi sympathies. Gopinath musters evidence from writers such as Grunendahl\index{Grunendahl@Grünendahl} to show how Pollock is a past master in the suppression of inconvenient facts, and thus proves himself a prevaricator {\sl par excellence}.

The second paper by Ashay Naik ({\bf Ch.~2}) makes a {\sl pūrva-pakṣa} of Pollock’s 2001 paper “Deep Orientalism?”. Pollock endeavours to establish therein a link between German Indology and the notorious Nazi ideology. The Nazis indulged in the worst of horrors, and linking Sanskrit with the Nazis can easily make Sanskrit culpable: is not an abettor a partaker of crime, after all? Naik organises his paper under the four labels of Orientalism viz. the British, the German, the Sanskrit, and the American. Pollock has sought to implicate Sanskrit knowledge as a factor in the development of Nazi ideology. Equipped with pertinent facts and cogent logic, Ashay shows how Pollock’s comparative morphology of domination, for all its polemic, remains untenable.

The third paper by Koenraad Elst ({\bf Ch.~3}) closely considers the claims of Pollock only to find them “surprisingly weak or simply wrong”; Pollock is not, of course, the first person to exploit the links between racism, Nazism, and the study of Indo-European culture on the one hand, and Sanskrit on the other. The counterpoint viz.\ Adolf Hitler’s\index{Hitler!Adolf} contempt for Hinduism\index{Hinduism}, though well known, is cautiously concealed by Pollock, and blatant lies are made use of so as to subserve his own polemical writings. While there is, of course, a general animus against Hinduism\index{Hinduism} in American academe, Pollock’s deliberate and concocted links between Hinduism and National Socialism suggest “a rare animosity against Hinduism”.  Again, Western Indologists spare no efforts to depict the Indian caste-system\index{caste!system} as slavery and racism, for which they invoke the discredited yet handy AIT (Aryan Invasion Theory\index{Aryan!Invasion Theory}), and fantasize a Nazi parallel, notwithstanding the fact that nowhere in the long history of India do we find even a faint hint of genocide\index{genocide} in connection with the {\sl varṇa\index{varna@\textsl{varṇa}}} system. He who makes public allegations, and cannot prove them, is guilty of slander, which Pollock patently is. The author of the paper asks the rhetorical question, finally, as to whether Pollock is at all fit to be trusted to preside over the publication of Indian classics.

{\bf Part 2} of the book is devoted to the projection of the “death” of Sanskrit. The discovery {\sl non pareil} of Pollock is that Sanskrit has died multiple times, even though it was still-born (his own naive fancy, again!) Fallacies galore of Pollockian polemics are well laid bare here.

The first paper in this part by Naresh Cuntoor ({\bf Ch.~4}) prepares for us a decoction of Pollock’s concoctions. To sample but three: Sanskrit had a symbiotic relationship\index{symbiotic relationship} with royal power; Sanskrit grammar\index{grammar} and {\sl śleṣa (paranomasia)} enhanced the political\index{power! political} status of kings; royal patronage\index{patronage} favoured Sanskrit over vernacular language. Extraordinary is Pollock’s sacrifice of empirical data at the altar of his own brand of narrative building. What comes handy as a powerful tool for butchery of facts is the complicated style of writing Pollock meticulously deploys. Unfortunately for Pollock, opacity of diction cannot always shield the deviousness of his designs. Pollock freely posits theories that never once make even a pretense of being at first a hypothesis that may naturally need to be justified. All the accoutrements of posturing are well in place: like a conscientious author, Pollock starts with meticulous and detailed enumeration of various caveats; one however has only to wait to get to the section of his conclusions to realise that all those caveats are thrown into winds unceremoniously. Pollock draws, again, analogies of Sanskrit with the truly dead languages, but only fallaciously.


The second paper is by Satyanarayana Das ({\bf Ch.~5}). Simply entitled “Sanskrit is not Dead” this paper puts Pollock in the dock by presenting a case study – of the Vraja literature of the 16th-17th centuries - that betrayed, against the wishful thinking of Pollock, the remarkable historical continuity\index{historical continuity} and vitality of Sanskrit. The author shows how Pollock does not provide either statistical evidence, or even proper references for his claims. The only point of note in Pollock’s presentation of a “momentous rupture” (post the “vibrant period'' 1550-1750), is that it is just sensational-in lieu of being really sensible. As against the whimsical claims of Pollock, the paper of Das makes its own argument sound, buttressed with apt statistical evidence.

The third paper by Jayaraman Mahadevan ({\bf Ch.~6}) deals with the classic inane statement by Pollock viz. “Government feeding tubes and oxygen tanks may try to preserve the language [Sanskrit] in a state of quasi-animation .... Sanskrit is dead”. The paper sets forth pertinent information in abundance from one of the most important documents of the Government of India (which Pollock knows cannot claim ignorance of) viz. the {\sl Report of the First Sanskrit Commission} constituted by the Government of India in 1956 with Dr. Suniti Kumar Chatterji as its Chairman; Chatterji, by the way, was by no means known for his sympathies with any “Hindutva\index{Hindutva}” philosophy that Pollock is so fidgety about. This was in fact at a time when no “Hindutva force” worth the name was even heard of, much less had been shaping things, and, to speak the truth, it was during the very reign of the government headed by Jawaharlal Nehru\index{Nehru!Jawaharlal}, “secularism” incarnate, euphemism for allergy to Hinduism\index{Hinduism}, which called all the shots.

The {\sl Report} was prepared covering a cross-section of India - (the then) 14 states of India, at 56 centres, and interviewing over 1100 persons – thus representing various shades of opinion. A perusal of the {\sl Report} shows that the Britishers paid Sanskrit teachers {\sl half the salary} they paid others, and worse, the hypocritical Nehru Government unabashedly perpetuated it! It was the response from the public – from Maharajas to the ordinary folk including non-Brahmins - that sustained Sanskrit in these adverse times. The paper has ably bleached the p(f)igments of Pollock’s political imagination.

The fourth paper by Kannan and Meera ({\bf Ch.~7}) scrutinises Pollock’s paper of 2001 on the “Death of Sanskrit”. The paper draws attention to the methodological idiosyncrasies of Pollock that can do no credit to a conscionable author: there is nothing for Pollock, for example, that cannot have a bearing on power. Pollock is aghast that Hindutva propagandists have sought to show – what is an elementary fact, after all – that Sanskrit is indigenous to India.  Revival of Sanskrit\index{Sanskrit!revival of} is to him a mere “exercise in nostalgia”. From the vast canvas of several millennia of the history of Sanskrit, Pollock picks ({\sl read} cherry-picks) just four tricky points of time - to delineate the degeneration/disappearance of Sanskrit. Pollock indeed “sees things” that few others can: viz. that it is the benign Muslim kings that tried to patronize and protect Sanskrit (and not jettison and jeopardize Sanskrit), while Hindu kings were apathetic towards it! Pollock does not even attempt to camouflage his intense Islamophilia (matched only by his high Hinduphobia) when he makes light of the atrocious burning of libraries in Kashmir as but “fire accidents” – in lieu of speaking of the wanton destruction\index{libraries, destruction of} by the Muslim marauders, which no chronicler has ever made any secret of. Pollock can have few sympathisers even among American Orientalists except, of course, his own intellectual offsprings like Audrey Trushke. The other typical techniques of “List and Dismiss”, “Perhaps...Probably...And therefore”, “{\sl divida et impera}”, “Selective playing up and playing down”, and the anachronistic and unprincipled superimposition of the frameworks of modern psychology and social science on Indian traditional lore of antiquity, to mention but a few – all illustrative of the contumely of an unchallenged maverick – are tellingly set forth in this paper.

The fifth (and the last) paper by Manogna Sastry ({\bf Ch.~8}) also deals with Pollock’s “Death of Sanskrit” paper of 2001. Starting with Sri Aurobindo’s\index{Aurobindo} note on India’s distinctness from the Occident, Ms.\ Sastry notes how there is a vast continuum of Western critics – from Sister Nivedita and Romain Rolland, sympathetic and understanding towards the Indian heritage at one end of the spectrum, and, at the other, the pompous and belligerent critics exemplified by Doniger\index{Doniger, Wendy} and Pollock. American Orientalism\index{American Orientalism} has only spawned a plethora of scholars of the latter type. The facile and puerile assertion of Pollock  that Sanskrit is championed just by promoters of Hindutva\index{Hindutva},  is easily repudiated by her by way of citing the numerous independent attempts in support of Sanskrit,  of several individuals and organizations that are totally independent of or utterly unaware of the Hindutva\index{Hindutva} movement. And to no small effect has been related by her the research of Hanneder, to show how Pollock is eminently capable of interpreting all evidence to fit just his own pet theses, and by whatever means. Summoning “anecdotal factoids\index{anecdotal factoids}” to suppress key facts staring in the face, or again, reinforcing devious attempts so as to drive a wedge between Sanskrit and vernaculars (which many European Indologists have carefully and consistently cultivated for long) - are all no small feats of Pollock. Very telling are her words towards the conclusion of the paper that the opus of Pollock is not all in vain in that it has actually provided the clarion call to Indians – to resume the stewardship of the creation and organization of their own cultural instruments.

It goes without saying that the opinions expressed in the papers presented here are those of the respective authors. The authors hold themselves responsible for the veracity of their statements.

\begin{center}***\end{center}

A survey of all the papers above may be said to reveal one important trait of Pollock. While not one paper here doubts or questions the profusion~of Pollock's scholarship, every one without exception has expressed apprehensions about the integrity of Pollock reminding us thus of the {\sl subhāṣita –  “vyartham pāṇḍityam guṇa-varjitam”} (“What avails scholarship, after all, {\sl sans} integrity?”)! 
\bigskip

\noindent
Vijayadaśamī\hfill	{\bf Dr.~K S Kannan}\\
Hemalamba Saṃvatsara\hfill Academic Director\\
Date 30-09-2017\hfill and\\	
\phantom{.}~\hfill General Editor of the Series  






