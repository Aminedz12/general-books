\chapter[On “Feeding Tubes and Oxygen Tanks” for Sanskrit....]{On “Feeding Tubes and Oxygen Tanks” for Sanskrit: 
In the light of the First Sanskrit Commission Report (1956--57)}\label{chapter6}
\index{Report, First Sanskrit Commission}

\Authorline{Jayaraman Mahadevan}
\lhead[\small\thepage\quad Jayaraman Mahadevan]{}

\section*{Abstract}
 
This paper focuses upon one statement from the paper “The Death of Sanskrit” (2001) of Prof. Sheldon Pollock which is- “Government feeding tubes\index{Government feeding tubes} and oxygen tanks may try to preserve the language in a state of quasi-animation, but most observers would agree that, in some crucial way, Sanskrit is dead.” (Pollock 2001: 393)  There are two clear implications from this statement.  Firstly – he implies that without the sponsorship of the Government, even the perceived ‘quasi-animate’  state of Sanskrit would not have been possible. Secondly – that the public and non-governmental players did not have any role at all in safe-guarding Sanskrit as Sanskrit has been never the language of the masses.  

The First Sanskrit Commission report, hereafter, the Commission, assumes significance in this context. The Commission was constituted by the Government of India in 1956 with Dr.\ Suniti Kumar Chatterji as its chairman. Seven other scholars of repute from various parts of the country were its members. The Commission crisscrossed the length and breadth of the country. In the words of the Commission’s report “…to consider the question of the present state of Sanskrit Education in all its aspects” ({\sl Sanskrit Commission of India\index{Sanskrit!Commission of India Report} Report} 1957:1).What does the study of the Sanskrit Commission’s detailed report reveal? Nearly five decades prior to the statement of Pollock in question (and its implications), one finds well researched facts and observations that render Pollock’s statement redundant, or rather, ridiculous. It is also interesting to note that Pollock has taken care not mention this report in his paper. Thus, this paper endeavors to juxtapose various observations from the overlooked First Sanskrit  Commission Report, that fly in the face of Pollock’s aforementioned statement, and allow the readers to  see for themselves the flaws and blemishes of Pollock’s understanding of the status of Sanskrit.

\section{The Trigger}
 
The article “The Death of Sanskrit” (2001) by Sheldon Pollock offers various elements of {\sl pūrva-pakṣa} for a Sanskritist. But, this paper focuses upon just one statement from the paper which is - 
\smallskip
\begin{myquote}
\eleven
“Government feeding tubes\index{Government feeding tubes} and oxygen tanks may try to preserve the language in a state of quasi-animation, but most observers would agree that, in some crucial way, Sanskrit is dead.” \hfill(Pollock 2001:393)
\end{myquote}
\smallskip

This statement is found in the third paragraph of the said article. Why just one statement and why this statement?  To answer this straightaway - firstly, it has to be noted that the paper ``Death of Sanskrit'' appears in the year 2001.  In the beginning of the paper Pollock states–
\smallskip

\begin{myquote}
\eleven
“The state’s anxiety both about Sanskrit’s role in shaping the historical identity of the Hindu nation and about its contemporary vitality has manifested itself in substantial new funding for Sanskrit education, and in the declaration of 1999--2000 as the “Year of Sanskrit,” with plans for conversation camps, debate and essay competitions, drama festivals, and the like.” \hfill (Pollock 2001:392)  
\end{myquote}
\smallskip
 
Though aimed at throwing light on the anxiety of the State, eventually unease in the mind of Pollock gets exposed caused by the apparent spike in funding for Sanskrit and related activities in 1999--2000. And soon after, this triggered Pollock to conceptualize and eventually pen this paper in the year 2001. As can be observed, the above “Government feeding tubes oxygen tank” statement, directly indicates that trigger, which is at the root of the paper.

Secondly, it is also the first statement in the article in which Pollock unceremoniously states Sanskrit is dead. If this trigger-indicating statement is shown as untenable, it will symbolically indicate the relative strength of the superstructure that has been erected upon this by Pollock in the later part of his paper. Furthermore, it can also be stated that recourse has been taken to {\sl sthālīpulāka-nyāya}–\index{sthali-pulaka-nyaya@\textsl{sthālī-pulāka-nyāya}}\index{nyaya-s@\textsl{nyāya}-s!\textsl{sthālī-pulāka}} examining one grain of rice to see whether rest of the rice in the boiling pot is cooked - in evaluating Pollock’s paper.

\section{About the First Sanskrit Commission}

The First Sanskrit Commission Report, passim, naturally assumes significance in this context. The Commission was appointed by the Government of India in 1956 with Dr.\ Suniti Kumar Chatterji\index{Chatterji, Suniti Kumar} as its chairman. Seven other scholars of repute from various parts of the country were members. The objective of the Commission was, in the words of the Commission’s Report :
\smallskip

\begin{myquote}
\eleven
“…to consider the question of the present state of Sanskrit Education in all its aspects. To this end (p.8), “the tour programme of the Commission, was carried out in five laps, covered all the [then] 14 States of India and  visited 56 centres and interviewed over 1,100 persons, representing various shades of opinion.” \hfill({\sl Sanskrit Commission of India Report} 1957:1)
\end{myquote}
\smallskip

A study of the Sanskrit Commission’s report reveals that nearly five decades before Pollock’s statement in question already, the First Sanskrit Commission’s report effectively presents appropriate “responses” to this statement and its implications. It is highly improbable that he is unaware of this report of 1957. In fact he mentions the year 1949 regarding the inclusion of Sanskrit in the Eight Schedule of the Constitution.\index{Constitution, Eighth Schedule} He also makes a note of the awardees of the Sahitya Akademi since its inception in the year 1955. But conspicuously missing in this article is the mention of the Sanskrit Commission’s report of 1957 which is an important and unique document regarding the status of Sanskrit in India.  Our paper endeavors to  juxtapose various observations from the First Sanskrit Commission’s report and help the readers see for themselves the flaw in Pollock’s understanding of the status of Sanskrit.

\section{Analyzing the statement}

Pollock’s statement is analyzed in its two parts in this paper. 

Part 1:  “Government feeding tubes\index{Government feeding tubes} and oxygen tanks may try to preserve the language in a state of quasi-animation…

Part 2:  “$\ldots$but most observers would agree that, in some crucial way, Sanskrit is dead.”

There are two clear implications for the first part of the statement. Firstly – he implies that without the sponsorship of the Government, even the perceived ‘quasi-animate’ state of Sanskrit would not have been possible either. Secondly – that the public and non-governmental players did not have any role at all in safeguarding Sanskrit, as Sanskrit, has been never the language of the masses. 

The second part of the statement is very plain and is discussed as such.

\subsection{First part of Pollock’s Statement, implications and Response}

\subsubsection{The First Implication and its response - Oxygen Tanks or Slow Euthanasia?}
\index{Sanskrit!slow euthanasia of}

Has the governmental aid really helped the survival of Sanskrit since whenever it was given? To this statement one finds a remarkable answer in the Sanskrit Commission's Report. It states 

\begin{myquote}
\eleven
 “As pointed out already, the authorities did not allow the traditional system either to die out or to flourish, but, by a process of nominal assistance, retained it alongside of modem education, in an unhealthy condition, ever subject to difficulty-always, open to criticism.”\\[-15pt] 
 
 ~\hfill({\sl Sanskrit Commission of India Report} 1957:28)
\end{myquote}
 
This state of affairs is as regards the attitude of the authorities of the pre-independence era, is quite understandable. Even during the post independence years, we may note that in all these seven decades, there is no substantial change in this approach. 

The Commission records that the status of Sanskrit post independence\index{Sanskrit!status of} has not been any better than during the British rulers. It states -

\begin{myquote}
\eleven
“This Commission, in the course of its tours, could see a feeling of regret and disappointment among the people that, while no positive steps had been taken for helping Sanskrit, the measures undertaken in respect of other languages have had adverse repercussions on it. The ultimate result of this has been that Sanskrit has not been allowed to enjoy even the status and facilities it had under the British Raj.”\index{British Raj}\\[-15pt] 

~\hfill({\sl Sanskrit Commission of India Report} 1957:7)
\end{myquote}

The Sanskrit Commission report goes further to add a powerful imagery by quoting a verse in Sanskrit regarding the Government’s attitude towards Sanskrit post-independence. It states -
\begin{myquote}
\eleven
In this connection, the Sanskrit Commission would like to quote an old verse, which many Sanskritists referred to and which graphically pictured their real feeling: `` `The night will pass and the bright day will dawn; the sun will rise and the lotus will bloom in all its beauty' While the bee, imprisoned in a closed bud, was thus pondering over its future, alas, an elephant uprooted the lotus-plant itself."\\[-15pt]   

~\hfill({\sl Sanskrit Commission of India Report} 1957:8)
\end{myquote}

Further, it is important to note the fact that the Commission also makes it clear that the step-motherly attitude\index{Sanskrit!step-motherly attitude towards} of the Government towards Sanskrit does not necessarily reflect the feelings of the people at large towards Sanskrit.  It states – 

\begin{myquote}
\eleven
“On the one hand, Sanskrit scholars, members of the public, educationists and authorities were keenly alive to the importance of Sanskrit studies; and, on the other, there was one kind or another of official and administrative difficulty or lack of practical assistance which produced a sense of frustration.” \hfill({\sl Sanskrit Commission of India Report} 1957:9)
\end{myquote}

Has the Government’s indifference towards Sanskrit undergone any change now - five decades after the first Sanskrit Commission? The answer is an unfortunate no. It is evidenced by the following statement from the committee set up by Ministry of Human Resource Development,\index{Ministry of Human Resource Development} Government of India, recently (2016) to evolve a “Vision and Roadmap for the Development of Sanskrit – Ten year Perspective Plan" – 

\begin{myquote}
\eleven
“It is a fact that during the British period salary of Sanskrit teachers was half the salary of the salary of the teachers of other subjects due to which Sanskrit was looked down upon for long. Even today, in most of the states Sanskrit teachers who teach at Secondary and Higher Secondary level Vidyalayas are given Primary level teachers’ salary, teachers who teach at UG and PG level Sanskrit Mahavidyalayas are given the salary of Secondary grade teachers’. Hence these Vidyalayas and Mahavidyalayas do not attract the talented teachers and students”\\[-15pt]  

~\hfill({\sl Vision and Road Map for the Development of Sanskrit:}\index{Vision and Road Map for the Development of Sanskrit@\textsl{Vision and Road Map for the Development of Sanskrit}} {\sl Report:} 2016:5) 
\end{myquote}

Wantonly or unwittingly, successive Governments in independent India trod the path of the erstwhile British masters. This observation of the committee puts in perspective the role of government in regard to the protection and promotion of Sanskrit. 

As revealed by the above statements from the Sanskrit Commission report, the motive behind the minimal support to Sanskrit by successive governments has been brought to light. When the Governmental support has such motives as that of making Sanskrit appearing sick (though it might not be so), it leads us to the inevitable conclusion that  governmental support or the lack of it cannot be taken as a correct indicator of the real status of Sanskrit. Hence it can be safely stated that Pollock has, on the basis of his above statement, allowed himself to be misled into assessing the real status of Sanskrit through the t(a)inted glasses of Governmental support.

\subsubsection{The Second implication and its response:}

This brings us to the second and more serious aspect as to the role of people and agencies other than the government in supporting and continuing the unbroken tradition of Sanskrit learning in the country.

A perusal of the Sanskrit Commission’s report brings out the fact that even more than the role of the Government, it is the people of the country who are keeping the flame of Sanskrit knowledge burning. The Sanskrit Commission’s report states that Sanskrit is associated with the ‘cultural consciousness’\index{cultural!consciousness} ({\sl Sanskrit Commission of India Report}\index{Sanskrit!Commission of India Report} 1957:67) of the country and that the love of Sanskrit is “next only to that of patriotism towards Mother India” ({\sl Sanskrit Commission of India Report} 1957:p.65). Though the Sanskrit Commission states that the state of Sanskrit in the country leaves much to be desired – yet far from being ‘dead’ as speculated by Pollock - the Commission states that “About the enthusiasm of the people of India as a whole for Sanskrit, we have received,..., the most convincing evidence.” ({\sl Sanskrit Commission of India Report} 1957: ii)

Following are the non-governmental players\index{non-governmental players} and factors that can be identified from the report of the First Sanskrit Commission– 
\renewcommand\theenumi{\alph{enumi}}
\renewcommand\labelenumi{(\theenumi)}
\begin{enumerate}
\itemsep=0pt
\item The Maharajas\index{Maharajas} of the Princely States
\item Hindu religious institutions\index{Hindu religious institutions}\index{institutions!religious}
\item People belonging to non-Brahmin castes\index{caste!non-Brahmin c.} and other religions            
\item Individual Traditional Pandits
\item Nationalistic Spirit
\item Role of voluntary Academies and Organizations
\item People in General 
\end{enumerate}

The list pertains to the Pre-independent as well as post-independent eras. It is to be noted that the seeds sown in the pre-independent era have sprouted later. For example, the role of the kings is a pre-independent era factor, but it would be seen in the words of the Commission itself as to how those pre-independence traditions helped in the preservation of Sanskrit in the post-independence era. 

The seven factors listed above are now to be considered one by one.

\paragraph{The Maharajas}
\index{Maharajas}

The Commission records that –
\begin{myquote}
\eleven
Apart from honouring Sanskrit Pandits and musicians in their Darbars and on occasions of domestic celebrations, and national festivals, the Maharajas did two important pieces of service to Sanskrit studies - one, the organization into libraries of their Palace collections of Sanskrit\break manuscripts, and two, the setting up of Sanskrit colleges.\index{Sanskrit!Colleges}\\[-15pt]

~\hfill({\sl Sanskrit Commission of India Report} 1957:22)
\end{myquote}

The Commission enlists the Princely states that supported Sanskrit –
\begin{myquote}
\eleven
“Darbhanga,\index{Darbhanga} Vizianagaram, Baroda, Nagpur,\index{Nagpur} Jaipur,\index{Jaipur} Indore, Gwalior,\index{Gwalior} Mysore,\index{Mysore} Travancore,\index{Travancore} Kapurthala, Patiala,\index{Patiala} Jammu and Kashmir\index{Jammu and Kashmir} - to mention only the more prominent States - started their Sanskrit Colleges.” \hfill({\sl Sanskrit Commission of India Report} 1957:22)
\end{myquote}

When one looks at the above list of princely states, it becomes evident that since the pre-independence era, even while under the British Raj\index{British Raj}, it is these princes and kings who truly supported Sanskrit studies. Pre independence British Government was indifferent, and the attitude of post-independence governments has already been pointed out. 

The role of the Maharajas in fostering Sanskrit education inspired other wealthy individuals too. The Commission records - “Inspired by the example of the Princes, the Zamindars\index{Zamindars} and smaller landlords and merchants also founded Sanskrit Colleges.” ({\sl Sanskrit Commission of India Report} 1957:22)

It is the Sanskrit organizations set up by the kings which were later converted into Government Sanskrit Colleges and Government manuscript repositories.\index{Government manuscript repositories}    
%~ \vskip -40pt

\paragraph{Hindu Religious institutions}\index{Hindu religious institutions}
%~ \vskip -4pt

The role played by Hindu religious institutions\index{institutions!religious} comes next. Hinduism\index{Hinduism} is often ridiculed for its mind boggling multiplicity of schools of philosophy and religious tenets. Though all these divergences exist among these religious denominations, they still were united in preserving Sanskrit which is the common thread. The Commission observers–
\begin{myquote}
\eleven
“Maths, temples\index{temple!activities of} and other religious institutions established similar Colleges; and affluent individuals and public leaders and associations also followed, founding their own Sanskrit Colleges,\index{Sanskrit!Colleges} or, by administrative direction, helping old religious and cultural endowments\index{endowments, religious and cultural} to start such\break colleges.”\hfill({\sl Sanskrit Commission of India Report} 1957:22)
\end{myquote}

%~ ~\\[-40pt]

\paragraph{People belonging to Non-Brahmin castes and other Religions}
\index{non-Brahma(i)n!castes}
%~ \vskip -4pt

Based on the first two points some may criticize that the kings and the priests had their own motives, i.e. dominating the others in society in promoting Sanskrit.  But the Commission’s report makes it clear that even Muslims and Christians\index{Muslim!and Christians}\index{Christians} in certain parts of the country actively participated in the learning of Sanskrit. Nor was it the case that the Brahmin community alone was learning and preserving Sanskrit. Cutting across the barriers of caste\index{caste} and religion, Sanskrit was embraced by all people in India. Neither was gender a barrier in learning Sanskrit. The following statements of the Commission testify to this – 
\begin{myquote}
\eleven
“That Sanskrit does not belong to any particular community is proved by Andhra and Kerala where the entire non-Brahman classes\index{non-Brahma(i)n!classes} are imbued with Sanskrit, and speak a language highly saturated with Sanskrit. In Kerala, even Izhavas,\index{Izhavas} Thiyas,\index{Thiyas} Moplas\index{Moplas} and Christians\index{Christians} read Sanskrit. In Madhya Pradesh, we were told, a paper in Sanskrit was compulsory at the School Final Examination and even 66 Muslims took it. In a Lucknow Intermediate College, there are Muslim girls studying Sanskrit; in Gujarat, Parsis\index{Parsis} study it; in Panjab, there are several Sikhs among Sanskrit students and teachers, and Sastris and research scholars in Sanskrit. The Director of Public Instruction of Madhya Pradesh, who is a Christian, told us that he advised the Anglo-Indian students also to read Sanskrit. It was necessary that, as future citizens of India, they gained an insight into the mind and the culture of the bulk of the Indian people. And this, he added, was possible only through the study of Sanskrit.”\\[-15pt]

~\hfill ({\sl Sanskrit Commission of India Report} 1957:64)
\end{myquote}

This patronage\index{patronage} of Sanskrit by Muslims and Christians is not a new phenomena. The Commission opines that it is part of the historical continuity\index{historical continuity} of the interest of the non-Hindu communities in Sanskrit. To quote – 
\begin{myquote}
\eleven
“This aspect of Sanskrit, that it was not exclusively religious, was appreciated even by some of the Muslim rulers\index{Muslim!rulers} of India, who patronised Sanskrit literature, and, in some cases (as in Bengal and Gujarat),\index{Gujarat} had their epigraphic\index{epigraphy} records inscribed in Sanskrit. It was the scientific and secular aspect of Sanskrit literature that made the Arabs welcome Indian scholars to Baghdad\index{Baghdad} to discourse on sciences like Medicine and Astronomy, and to translate books in these subjects into Arabic. The Ayurveda\index{Ayurveda} system of medicine, until recently, was the truly National Indian System, which was practised everywhere, and access to this was through Sanskrit books, which even Muslim practitioners of the Ayurveda in Bengal studied.”  
({\sl Sanskrit Commission of India Report} 1957:79)
\end{myquote}

This is yet another statement from the Commission’s report which is relevant to be quoted here–
\begin{myquote}
\eleven
“In the course of our tours in South India, we interviewed several non-Brahmans in high position and active in public life, business, etc., and we found them all favourable to Sanskrit. In Madras City itself, we found that, both in the recognised schools and private classes, non-Brahmans, and even a few Muslims and Christians,\index{Christians} studied Sanskrit. In one of the High Schools of Chidambaram, a Muslim student was reported to have stood first in Sanskrit; and in another School, there were Harijans\index{Harijans} among the Sanskrit students. In Chidambaram we were glad to find a group of leading non-Brahman merchants of the town who appeared before us for interview as staunch supporters of Sanskrit education and culture.”\\[-15pt] 

~\hfill({\sl Sanskrit Commission of India Report} 1957:65)
\end{myquote}

The Commission further adds –
\begin{myquote}
\eleven
“In Tanjore\index{Tanjore} also, we were told by the Headmasters and Sanskrit teachers of local schools that non-Brahmans, Muslims and Christians\index{Christians} freely took Sanskrit. It was again the non- Brahmans, particularly the great benefactors belonging to the Chettiar community,\index{Chettiar community} who had, in the recent past, endowed many Pathasalas for Veda\index{Veda} and Sanskrit.”\\[-15pt] 

~\hfill ({\sl Sanskrit Commission of India Report} 1957:66)
\end{myquote}

Thus it becomes evident from the above excerpts that all religious groups and caste\index{caste} groups patronized Sanskrit. Moreover, it also becomes evident that there is also no north-south divide\index{north-south divide} in promoting Sanskrit. 

From the Commission’s report it was shown that cutting across religious, caste divisions people learn and love Sanskrit. It is very educative to note, when we see the Sanskrit Commission state that, even in the creation of Sanskrit literature over the ages, it was not the case that it was just one community was involved. The Commission notes –
\begin{myquote}
\eleven
“It must be further pointed out that the large mass of literature in Sanskrit was not produced by any particular community. Several instances can be quoted of non-Brahman\index{Sanskrit!non-brahmana authors} and non-Hindu authors\index{Sanskrit!non-Hindu authors} who have made significant contributions to Sanskrit literature. It is definitely wrong to assume that Sanskrit represents only the religious literature of the Hindus.”  \hfill ({\sl Sanskrit Commission of India Report} 1957:79)
\end{myquote}

\paragraph{Individual Traditional Pandits}

Apart from various non-governmental organizations it was finally the individual scholars who stood for the protection and perpetuation of Sanskrit. The Commission states that –
\begin{myquote}
\eleven
“In addition to these two agencies, namely, the Government-organised Sanskrit Colleges,\index{Sanskrit!Colleges} such as the Banaras and Calcutta Colleges, and the different Colleges of the Princely States and the private and religious agencies, there was also the third channel through which the Sanskrit tradition continued to flow, namely, the one-Pandit schools.\index{one-Pandit schools} In fact, this tradition of one-Pandit schools was alive in all regions of India in a greater or lesser degree, according to the past history of each place. The tempo of modernisation had not fully swept away the Pandit of the traditional type and his institutions.” 
\vskip -5pt

~\hfill({\sl Sanskrit Commission of India Report }1957:22)
\end{myquote}

The Commission truthfully records the level of scholarship\index{level of scholarship} among those Pandits. It states –
\begin{myquote}
\eleven
“We specially enquired whether the Pandits still carried on the tradition of writing new commentaries or dialectical works. We were sorry to note that the number of outstanding Pandits of the old type was generally not large; in some States, they could be counted on one's fingers. Some Pandits, however, did continue their literary activity; a few of them have, under the inspiration of modern research, produced critical and expository treatises in Sanskrit or in the regional languages on Sastraic and other general philosophical subjects.”\\[-15pt] 

~\hfill({\sl Sanskrit Commission of India Report} 1957:45)
\end{myquote}

It becomes evident from the above quote that the level of scholarship certainly has come down. But it has not died down completely. 

Further, in the context of revival of Non-formal Śāstra Learning System\index{non-formal Sastra Learning System@non-formal Śāstra Learning System} the recent 2016 MHRD Committee states the following - 
\begin{myquote}
\eleven
“Under this stream, Gurukula system\index{Gurukula system} will be revived and the expertise of traditional scholars who still adhere to their traditions and do not leave their place of residence will be utilized. Each such traditional Scholar of Shastric excellence will teach traditional Shastras in a traditional way maintaining their way of living and transmitting such knowledge”. \hfill({\sl Vision and Road Map for the Development of Sanskrit: Report}: 2016:26) 
\end{myquote}
	
This quote also clearly testifies to the existence of traditional scholars with high level of erudition in {\sl Śāstra}-s even today. The Sanskrit Commission also records the following regarding the existing level of scholarship during its period – 
\begin{myquote}
\eleven
“We also found that the Sanskrit Muse was still an inspiration and that the Pandits everywhere wrote poems and plays in Sanskrit. Of course, Sanskrit was very freely used as a means of communication\index{means of communication} and for the expression of all current ideas. We actually met some Pandits who could employ Sanskrit with eloquence and oratorical effect.”\\[-15pt]

~\hfill({\sl Sanskrit Commission of India Report} 1957:45)
\end{myquote}
\newpage

On the question of how the pandits maintain their level of scholarship and even refine it amidst the gloom of discouragement all around, the Commission states that– 
\begin{myquote}
\eleven
“Among the activities\index{temple!activities of}, which keep up the scholarly interest of the Pandits and also afford them some encouragement and help, are the Sabhas or the Sadas (learned gatherings), which are held from time to time by rulers, Zamindars,\index{Zamindars} rich men, Acharyas and public associations. The former Princely States used to hold such gatherings once a year on the occasion of some festival, like the Dasara. The religious Teachers, Acharyas, still hold such gatherings of Pandits; also whenever any Pandit from a different part of the country visits an Acharya, he is engaged in a Sastrartha\index{sastrartha@\textsl{śāstrārtha}} or is asked to lecture, and is honoured with presents and cash-gifts. There are also some private endowments\index{endowments, private} which arrange for such Pandit Sadas,\index{Pandit Sadas} once a year, on Rama-navami, Krishna-jayanti, and similar occasions. In some of the temples, Pandits are similarly invited to give expositions and are honoured. In fact, it was these public debates in Sastras which had been the main inspiration for the growth of the thought and literature in the field of Sanskrit. And it would be by their resuscitation that the old intensity of Sastra-learning could be retained and promoted.” \hfill ({\sl Sanskrit Commission of India Report} 1957:45)
\end{myquote}

The Commission records that along with the Traditional Pandits, ‘Sanskritists of the modern type’ had the capability to converse in Sanskrit. The following is the relevant quote from the report regarding this – 
\begin{myquote}
\eleven
 Just as many of the replies to the Questionnaire received by the Commission were in Sanskrit, quite a number of interviews also took place in Sanskrit. It was not the Pandits alone who gave their evidence in Sanskrit; many Sanskritists of the modern type also freely discussed with the Commission through the medium of Sanskrit. This once again proved that Sanskrit still continued to be the lingua franca\index{lingua franca@\textsl{lingua franca}} of Sanskrit scholars of this country, irrespective of the different regions to which they belonged. \hfill({\sl Sanskrit Commission of India Report} 1957:9)
\end{myquote}

Thus the Commission not only brings to light the contribution of individual Sanskrit scholars in the survival of Sanskrit but it also records the level of their knowledge as well as the {\sl modus operandi}\index{modus operandi@\textsl{modus operandi}} by which sound scholarship is maintained. As vouched by recent MHRD Committee, such observations hold good even after almost five decades of the preparation of the first Sanskrit Commission Report.  

\paragraph{Nationalistic spirit – A catalyst to Sanskrit preservation}

The Sanskrit Commission notes that the nationalistic spirit during the pre independence era also served to protect our culture and the language. The following are the words of the Sanskrit Commission in this regard  – 
\begin{myquote}
\eleven
“Two circumstances averted the rot to some extent: one, the Princely States and the native patterns of life there; and the other, the new awakening in the country of a nationalistic spirit\index{nationalistic spirit} which sought to make up for the drawbacks in the scheme of education on the cultural side by founding institutions\index{institutions!educational} of cultural importance. Thanks to both of these, a network of Sanskrit colleges\index{Sanskrit!Colleges} of a quasi-modern set-up came into being.”\\[-15pt] 

~\hfill({\sl Sanskrit Commission of India Report} 1957:28)
\end{myquote}

This indicates how Governmental support, or the lack of it, does not by itself settle something. The emotional bonding people have towards Sanskrit is what has sustained Sanskrit. The above quote might lead to an understanding that Sanskrit rides on the nationalistic spirit. But the Commission notes that the relation between Sanskrit and the nationalistic spirit is symbiotic.\index{symbiotic relationship} It states that Sanskrit has the potential to strengthen the solidarity of our nation. It observes – 

\begin{myquote}
\eleven
“$\ldots$a distinguished group of India's thought-leaders that Sanskrit can very well be rehabilitated as a pan-Indian\index{pan-Indian} speech, to strengthen the solidarity of Modern India. Indeed, to emphasise this point, a witness, appearing before the Commission, suggested that if the Sanskrit Commission had come before the States Reorganisation Commission, many of the recent bickerings in our national life could have been avoided.”  \hfill({\sl Sanskrit Commission of India Report} 1957:81)
\end{myquote}

~\\[-40pt]

\paragraph{Role of NGO-Academies in keeping up popular interest}
\index{NGOs}

The following observation by the Sanskrit Commission regarding the active role played by non-Governmental, voluntary organizations, deals a decisive blow to the ‘governmental oxygen tank’ supposition of Pollock. The Commission’s report reads  -  

\begin{myquote}
\eleven
“In all regions there are now Sanskrit Academies, Associations, Sabhas, Parisads, etc., which organise the celebration of Sanskrit Poets' Days;\index{Sanskrit!Poets' Days} lectures on Sanskrit subjects; Sanskrit classes; competitions in Sanskrit essay-writing, Sanskrit elocution, and original composition (Short Story, Poem, Play); Sanskrit Recitals and Dramas; and publication of cheap booklets in Sanskrit. All of these keep up popular interest in Sanskrit. The names of many such associations, whose representatives met us, may be seen in the lists in the Appendices.” 
“The Sanskrit Sahitya Parishad,\index{Sanskrit Sahitya Parishad, The} Calcutta, the Sanskrit Academy,\index{Sanskrit!Academy} Madras, the Bharatiya Vidya Bhavan,\index{Bharatiya Vidya Bhavan} Bombay, the Samskrita Visva Parishad\index{Samskrita Visva Parishad} which has now over 500 branches all over India, the Brahmana Sabha,\index{Brahmana Sabha} Bombay, which has a Sanskrit dramatic troupe, the Akhil Bharatiya Samskrita Sahitya Sammelan, Delhi, may be specially mentioned among the bodies which have been doing sustained work of more than a local provenance. Recently, in Nagpur and Ujjain, societies have been established for the study and propagation of Kalidasa's\index{Kalidasa@Kālidāsa} works, and we were pleased to note that the respective State Governments were helping these societies. The Kalidasa Society\index{Kalidasa Society} at Ujjain, we were told, had a fund of Rs. 1 1/4 lakhs of its own. There are several organisations in the country whose object is to popularise the study of the Gita\index{Bhagavadgita@\textsl{Bhagavadgītā}}. Establishments like the Svadhyaya Mandal,\index{Svadhyaya Mandal, role of} Pardi, and the Veda-Dharma-Paripalana-Sangham,\index{Veda-Dharma-Paripalana-Sangham, role of} Kumbhakonam, take interest in the popularisation of Vedic thought and literature. Among the modern neo-Hindu movements, the Arya Samaj\index{Arya Samaj} and the Ramakrishna Mission\index{Ramakrishna Mission} are doing excellent work for the spread of interest in Sanskrit and its knowledge. Many Sanskrit Colleges\index{Sanskrit!Colleges} and the Sanskrit Departments of Colleges have Associations, which organise regular lectures on Sanskrit subjects, and sometimes also produce Sanskrit dramas.” \hfill({\sl Sanskrit Commission of India Report} 1957:46)
\end{myquote}

The Commission also observes in this regard–
\begin{myquote}
\eleven
 “For outside these educational institutions,\index{institutions!educational} there is in the country a network of voluntary organisations. The number and the extent of planned activities of these private bodies only underline the need for supplementing what is being done for Sanskrit through the official\break set-up.”\hfill({\sl Sanskrit Commission of India Report} 1957:67)
\end{myquote}

The above comment is also an eye-opener. The government is required to supplement the work of voluntary organizations. This means that the zeal and impetus to preserve and perpetuate Sanskrit is already present in the people.

Going by the information currently available in the website of\break Rashtirya Sanskrit Sansthan\index{Rashtirya Sanskrit Sansthan} (for the year 2009-2010), 767 voluntary Sanskrit Institutions receive financial assistance from the Government. This is to give a pointer to the probable number of voluntary Sanskrit organizations that still exist in this country. This implies that there might be more organizations which do not receive or require governmental support for their activities.   

\paragraph{People in General – The Undercurrent }

Apart from various agencies which operate in an organized manner, the Sanskrit Commission indicates the presence of an undercurrent of love for Sanskrit among the people of the country which keeps alive the tradition of Sanskrit learning and dissemination.

The Commission states that– 
\begin{myquote}
\eleven
“it is possible for an Indian or a foreigner knowing no other language than Sanskrit to be able to find throughout the whole of India some persons everywhere who can communicate with him in Sanskrit”.\\[-15pt] 

~\hfill ({\sl Sanskrit Commission of India Report} 1957:81)
\end{myquote}

Apart from indicating the pan-Indian presence of Sanskrit, the above statement may also imply that speakers of Sanskrit may be far and few in between. But regarding the undercurrent of love towards Sanskrit in the heart of people of this land, the Commission makes this remarkable observation–
\begin{myquote}
\eleven
“Generally speaking, the people of India love and venerate Sanskrit with a feeling which is next only to that of patriotism towards Mother India. This feeling permeates the common man, the litterateur and the educationist, the businessman, the administrator and the politician. Everybody realises its cultural importance\index{cultural!importance} and knows that whatever one cherishes as the best and the noblest in things Indian is embedded in Sanskrit.”\hfill ({\sl Sanskrit Commission of India Report} 1957:66)
\end{myquote}

Another noteworthy observation of the Commissions about the general feeling of people regarding Sanskrit comes during the discussion of national language\index{national!language} status for Sanskrit, a race which it lost to Hindi. It states – 
\begin{myquote}
\eleven
“After Independence, the Constituent Assembly decided that the official language\index{official language} of India was to be Hindi written in Devanagari script, and this was put in the Constitution. But the proceedings of the Constituent Assembly on this question were anything but smooth, and though there was a tacit agreement in this matter, Sanskrit never ceased to loom in the background. A general feeling was there that if the binding force of Sanskrit was taken away, the people of India would cease to feel that they were parts of a single culture and a single nation. The readiness with which Hindi received the support of a large section of the Indian people was because Hindi appeared to make a stand for Sanskrit. The support of Hindi in a way meant laying stress on the unity of India through Sanskrit, even if it were through the intermediacy of Hindi. The aspirations of a free Indian people, it was thought, could be best expressed through Sanskrit, functioning through the Modern Indian Languages.”\\[-15pt] 

~\hfill ({\sl Sanskrit Commission of India Report} 1957:70)
\end{myquote}

Many state governments appointed Commissions prior to the first Sanskrit Commission to assess the interest of people in Sanskrit. These Commissions have time and again established the innate love among the masses towards Sanskrit. To quote the words of the Commission as an example - 
\begin{myquote}
\eleven
“The old Madhya Pradesh Government had appointed in 1955 a Committee to go into the question of Sanskrit institutions,\index{institutions!Sanskrit} and here again, we would like to emphasise, the verdict of the public opinion had been in favour of preserving the traditional style of Sanskrit Education with the introduction of the necessary elements of modern knowledge.”\\[-15pt]

~\hfill ({\sl Sanskrit Commission of India Report} 1957:42)
\end{myquote}

The Commission makes an interesting remark which gives a sense that English education has inadvertently sparked off renewed interest in Sanskrit and culture. The lack of imparting of cultural education\index{cultural!education} in the traditional set-up itself seems to have suggested to the educated people to turn to their roots. The Commission observes -        
\begin{myquote}
\eleven
More recently, owing to a new awakening among the educated middle class and also owing to the interest of some of the leading citizens in the locality, expositions of the epics, the Gita\index{Bhagavadgita@\textsl{Bhagavadgītā}}, the Upanisads, Vedanta, Dharma,\index{dharma@\textsl{dharma}} etc., have become a regular and organised activity in some\break places. These expositions are arranged as public lectures to large audiences or as private classes to select groups. They have, indeed, proved a great source of help to the Pandits. The Pandits are in demand also for individual tuition in the Gita or Vedanta which some well-to-do persons desire to have. This appears to be an expanding activity and augurs well for the revival of interest in Sanskrit.\\[-15pt]

~\hfill ({\sl Sanskrit Commission of India Report} 1957:45)
\end{myquote}

In the same vein the Commission further observes that – 
\begin{myquote}
\eleven
“In the course of our tours, we noticed everywhere an unmistakable awakening of the cultural consciousness of the people. There was .a keen awareness of the importance of Sanskrit among people at large; and we soon realised that a complete picture of the situation regarding Sanskrit could not be had only by visiting Schools, Colleges, Universities and Pathasalas.” \hfill ({\sl Sanskrit Commission of India Report} 1957:67)
\end{myquote}
\newpage

Thus, if the {\sl pāṭhaśāla}-s, schools and colleges themselves cannot be the true indicators of the real situation of Sanskrit in the country, then it is very clear that the government, as has been shown above, which has not so favourable an attitude towards Sanskrit can never be the indicator of the situation of Sanskrit in the country. And Pollock’s statement that only the government oxygen tank helps survival of Sanskrit is grossly incorrect and his pronouncement of death of Sanskrit based on it is untenable.

Finally, regarding the role of general masses in the revival of Sanskrit the Commission states – 
\begin{myquote}
\eleven
“These voluntary public activities in the field of Sanskrit are all-comprehensive - general and special, popular and learned, scholastic as well as artistic, literary as well as organisational. In fact, it was these activities among the general public which struck us as the most encouraging circumstance. They definitely pointed to the recapture of that spirit and atmosphere, which would help Sanskrit again to emerge with a fresh vitality and force.”
\vskip -5pt

~\hfill ({\sl Sanskrit Commission of India Report} 1957:67)
\end{myquote}

~\\[-40pt]

\subsubsection{What then is the role of the Government in promotion of Sanskrit?}
\vskip -5pt

The seven real factors for the perpetuation of the Sanskrit as brought to light by the Sanskrit Commission Report were presented above. Will this then mean that the Government has no role to play in the promotion and preservation of Sanskrit?  The detailed report of the Commission indeed is full of recommendations to the Government in that regard. But the Commission is of a considered view that Sanskrit in no way, needs government ‘oxygen tanks’ or ‘feeding tubes’.\index{Government feeding tubes} Rather, to quote the words of the Commission -
\begin{myquote}
\eleven
“Already there are large endowments\index{endowments} and other resources in the country which, with the help and direction of the Government, can be properly harnessed for the purposes set forth by us.” 
\vskip -5pt

~\hfill({\sl Sanskrit Commission of India Report} 1957:293)
\end{myquote}

Hence Pollock’s ‘government oxygen tank’ statement appears highly unwarranted and unrealistic. The Commission’s report implies that facilitation and not mere funding is what is required from the government towards preservation of Sanskrit. This is with regard to the first half of the statement of Pollock which is the main focus of this paper.   

\subsection{The Second part of Pollock’s Statement and its Response - Some observers who won’t agree!}

Let us also consider the second half of Pollock’s statement and the Sanskrit Commissions “responses” to that briefly before concluding this paper. The second half of the statement of Pollock reads –“…but most observers would agree that, in some crucial way, Sanskrit is dead”. (Pollock 2001: 393). 

\makeatletter
\renewcommand\thesubsubsection{\thesubsection.\@alph\c@subsubsection}
\makeatother

\subsubsection{Observers - 1}

At the outset, it is obvious that the highly qualified members of Sanskrit Commission are not part of the team of Pollock’s brand of ‘observers’. It is to be noted that the members of Sanskrit Commission were not mere scholars, but they were ‘observers’ too, but in a real sense. As has been stated in the introduction, the members of the Commission travelled the length and breadth of the country. The itinerary of the various scholars, by the day and by the hour even, is available as official documents. In the course of their work, they keenly “observed” not only the state of Sanskrit but also the stature of Sanskrit. Further they clearly indicate that Sanskrit is not dead. Hence there is no question of its revival. It’s just an issue of  modernization\index{modernization} of Sanskrit. The following quote captures the Sanskrit Commission’s take on whether Sanskrit is dead-
\begin{myquote}
\eleven
“Sanskrit at that time permeated all aspects of Indian life, and so there could be no question of reviving it. Only there was an attempt to modernise its study. The place of Sanskrit in Indian life and in the Indian set-up was taken for granted by the nationalist workers before Independence. When Bankim Chandra Chatterji\index{Chatterji, Bankim Chandra} composed his National Song Vande Mataram\index{Vande Mataram@\textsl{Vande Mātaram}}\index{national!song} about the year 1880, he could not have foreseen what an importance this song would later on acquire in the national movement, of which the two words, Vande Mataram, practically became the basic mantra, the Rastra-Gayatri,\index{Rastra-Gayatri} if we may say so. He composed this song in Sanskrit (with a few Bengali sentences within) as the most natural thing. The place of Sanskrit was so obvious that no one gave any special thought to it.” \hfill({\sl Sanskrit Commission of India Report} 1957:69)
\end{myquote}

The following quote from the Commission’s report is even more powerfully direct on whether Sanskrit is dead or alive  – 
\newpage

\begin{myquote}
\eleven
“Even at the present day Sanskrit is very very living, because a large number of people use Sanskrit in their conversation, when they come from different parts of the country, and composition in Sanskrit, in both prose and verse, goes on almost unabated. It has been possible to write a history of recent Sanskrit literature\index{Sanskrit!literature, recent} as it has developed, say, during the last century and a half. Entire conferences\index{conferences} are conducted wholly or at least to a very large extent through the medium of Sanskrit.\index{Sanskrit!medium of} In the popular Purana recitations, the reciters who have all the art of telling a story dramatically use by preference a highly Sanskritised Bengali, Telugu, Oriya, Kannada or Panjabi, which is largely understood even by the unlettered masses. It is not uncommon to find religious lecturers giving discourses in simple Sanskrit, and they are generally understood by people possessing a slight education in their own mother-tongues. And above all, there is a tremendous love, which is something very close to veneration, for Sanskrit.\index{Sanskrit!veneration for} And when Sanskrit is now being used even to express modern scientific or political ideas in essays or discourses on various modern subjects, it cannot be said to have closed the door to further development - it has still life in it. All these things would go to establish that Sanskrit is still a living force\index{living force} in Indian life. It would be almost suicidal to neglect and gradually to relegate into oblivion as something dead and useless this very vital source of national culture and solidarity.” \hfill({\sl Sanskrit Commission of India Report} 1957:88-89)
\end{myquote}

~\\[-30pt]
 
\subsubsection{Observers - 2}
\vskip -5pt
The Commission quotes University Education Commission (p.3) (December 1948- August 1949) and states that – “Sanskrit was best suited for spiritual training\index{spiritual training, Sanskrit in} as it embodies the element of morality in the larger sense.” ({\sl Sanskrit Commission of India Report} 1957:3)

Again, the esteemed members of University Education Commission who seem to envisage a futuristic role for Sanskrit in moral and spiritual education, also are not part of Pollock’s ‘observers’ team.

~\\[-40pt]

\subsubsection{Observers - 3}
\vskip -5pt

The members of Official Languages Commission\index{Official Languages Commission} (OLC) (June 1955-June 1956) of the Government of India, quoted in the First Sanskrit Commission report also do not seem to be in Pollock’s ‘most observers’. Because, they state (p.4)-
\begin{myquote}
\eleven
``It is hardly necessary to add that, besides the current regional languages, there is an immense amount of work which needs to be done in respect of Sanskrit, Pali, Prakrits, Apabhramsa, etc. The Sanskrit language pre-eminently and the other ancient languages in different degrees have powerfully influenced current Indian speeches and a study of these has an obvious' bearing on the study of contemporary forms of speech" (p. 218). The OLC observations quoted by Sanskrit Commission further states that (p.4) "All our languages, including what are known as the Dravidian languages,\index{Dravidian languages} have through all the centuries habitually drafted, in a greater or less degree, to meet every new situation and requirement for expression of a new idea or shade of meaning, upon that vast and inexhaustible treasure-house of vocabulary, phrase, idiom and concept comprised by the Sanskrit language and literature. The Ramayana\index{Ramayana} and the Mahabharata,\index{Mahabharata} the Puranas and the Sastras, the Classical poems, dramas and literary masterpieces of Sanskrit have served throughout those centuries not only as the reservoir of ideas, sentiments and parables to be drawn by all for the embellishment of their literary output, but also as benchmarks of literary excellence, as standards for social conduct, as examplars of morality, and, in short, as the repository of wit and wisdom of all the Indian peoples throughout the ages..."\hfill ({\sl Sanskrit Commission of India Report} 1957:4)
\end{myquote}

\subsubsection{Observers – 4}

Interestingly, even Jawaharlal Nehru\index{Nehru!Jawaharlal} does not agree that Sanskrit is dead. The Commission quotes independent India’s first prime minister in this regard. It states–
\begin{myquote}
\eleven
“When Jawaharlal Nehru made the following observations about the importance of Sanskrit in India, he only reiterated the general belief of the Indian people, and the considered views which have been expressed not only by the greatest thinkers and leaders of India, but also by foreign scholars and specialists in Indian history and civilisation who are in a position to appraise objectively the value of Sanskrit: ``If I was asked what is the greatest treasure which India possesses and what is her finest heritage, I would answer unhesitatingly it is the Sanskrit language and literature, and all that it contains. This is a magnificent inheritance, and so long as this endures and influences the life of our people, so long the basic genius of India will continue".” 
\vskip -5pt

~\hfill({\sl Sanskrit Commission of India Report} 1957:71-72)
\end{myquote}

The Commission’s report also brings in another quote from the Jawaharlal Nehru to imply that Sanskrit is not only alive, but it is the language that has the potential to even out the balance in education which is leaning towards only the sciences. The report reads  – 
\begin{myquote}
\eleven
“And specially in modern times when a sort of dangerous over-weightage is being given to Sciences and Technology, the Humanities in Sanskrit will prove greatly helpful in restoring the proper balance. It is, indeed, highly significant that, as Prime Minister Shri Nehru\index{Nehru!Jawaharlal} told this Commission, Professor Oppenheimer,\index{Oppenheimer} the great American atomic scientist, spends considerable time in reading Sanskrit and Pali.” 
\vskip -5pt

~\hfill({\sl Sanskrit Commission of India Report} 1957:76)
\end{myquote}

~\\[-45pt]

%prathibha, index from here, page no 199
\subsubsection{Observers - 5}
\vskip -3pt

It seems that even (most) Western Universities and its academia are not part of “the observers” of Pollock who declared Sanskrit as dead. The Commission’s report states  - “...the West knows India as ``Sanskrit India",\index{Sanskrit India@“Sanskrit India”} and whenever an Indian University celebrates its jubilee, a Western University normally sends its felicitations in a Sanskrit address...” ({\sl Sanskrit Commission of India Report} 1957:89)

Adding a contemporary dimension to this, even in 2007 when the then president of India APJ Abdul Kalam visited Greece he was greeted by his Greece counterpart Karolos Papoulias\index{Papoulias, Karolos} in Sanskrit. A news report on this states –
\begin{myquote}
\eleven
“It was a pleasant surprise for President A P J Abdul Kalam\index{Kalam, APJ Abdul} when his Greek counterpart Karolos Papoulias greeted him in Sanskrit at the banquet ceremony hosted in honour of the visiting dignitary.  ``Rashtrapati Mahabhaga, Suswagatam Yavana deshe Bhawatam (Mr President, welcome you in Greece)", thus began the Greek President his speech at the banquet hosted at the Presidential palace on Thursday night much to the delight of the Indian delegation” 
\vskip -5pt

~\hfill ({\sl Times of India}, Apr 27, 2007) 
\end{myquote}

Thus it becomes evident from the above that, like the first part of Pollock’s statement, the second part too, to state in the language of Tarka (Indian Logic), seems to have a {\sl sat-pratipakṣa}\index{sat-pratipaksa@\textsl{sat-pratipakṣa}} (a valid counter-example) in the light of the First Sanskrit Commission Report.
%~ \vskip -10pt

\section{Conclusion:}
%~ \vskip -5pt

Based on the views gleaned from the Sanskrit Commission’s report (1957), it could be seen that, considered whether in parts or as a whole, one of the very initial statements of Pollock in his paper Death of Sanskrit (2001), has been rendered null and void decades prior to his statement.

With respect to this statement, this redundancy might be attributed to the non-serious and frivolous outlook of Pollock, if the following might be taken as an inadvertent admission towards that end in his article “The Death of Sanskrit”. In the sentence that precedes ‘oxygen tank’ statement he states - “Its cultivation constitutes largely an exercise in nostalgia for those directly involved, and, for outsiders, a source of bemusement that such communication takes place at all.” (Pollock  2001:393)

True. For an outsider (like Pollock) Sanskrit might be a “source of bemusement”. But the insiders, whose views have been adequately presented in Sanskrit Commission’s report, look at Sanskrit as (p.279) - “…one of the bases of our national culture and solidarity” ({\sl Sanskrit Commission of India Report} 1957:279) and as the source that “will provide a base for the promotion of International Understanding in the East and the West”. ({\sl Sanskrit Commission of India Report} 1957:279)

\begin{thebibliography}{99}
\itemsep=2pt
\bibitem[]{chapter6_item1}
“Greek Prez greets Kalam in Sanskrit” (2007, Apr 27), {\sl Times of India},\index{Times of India (newspaper)@\textsl{Times of India} (newspaper)} \url{http://timesofindia.indiatimes.com/world/europe/Greek-Prez-greets-Kalam-in-Sanskrit/articleshow/1965800.cms}  Accessed on 21 March 2017. 

\bibitem[]{chapter6_item2}
Pollock, Sheldon (April 2001). ``The Death of Sanskrit" (PDF). {\sl Comparative Studies in Society and History}. 43(2): 392--426.

\bibitem[]{chapter6_item3}
{\sl Sanskrit Commission of India} (1956-57): (Report accessed from the website of Information Repository of Education in India:)\index{Information Repository of Education} \url{http://www.teindia.nic.in/mhrd/50yrsedu/u/45/3Z/Toc.htm}.\\ Accessed on 09 January, 2017.

\bibitem[]{chapter6_item4}
{\sl Vision and Road Map for the Development of Sanskrit\index{Vision and Road Map for the Development of Sanskrit@\textsl{Vision and Road Map for the Development of Sanskrit}} – Ten year Perspective Plan}: (Report accessed from the website 
of Rashtriya Sanskrit Sansthan:)~\url{http://www.sanskrit.nic.in/ 016_02_12_Vision_and_Road_ Map_ or_the_Development_of_Sanskrit_Ten_year_perspective_Plan.pdf}. Accessed on 09 January, 2017.
\end{thebibliography}
