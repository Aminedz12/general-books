\chapter{Prerequisites}
\graphicspath{{Chapter1/Chapter1Figs/EPS/}{Chapter1/Chapter1Figs/}}

\section{Introduction}\label{sec1.1}

Topology is main branch of pure Mathematics the purpose of subject is to elucidate and inspect the concept of topological spaces, their continuity and continuos mapping, within the structure of Mathematics. The study of these and their general properties leads to a formation of general topology. The fundamental structure on a topological space is not a distance function, but a collection of open sets, thinking directly in terms of open sets often leads to greater clarity as well as greater generality. Here we present an detailed study of a new kind of generalized closed sets termed ws-closed sets and their respective continuous maps, closed maps, open maps, homeomorphisms, locally closed sets, locally continuous maps and bitopological spaces.

This thesis includes an overview on mathematicans and their work on toplogical spaces for the progress of topology. Second section discussion starts with stronger and weaker forms of open sets and closed sets. In third Section deals with stronger and weaker forms of continuous functions and irresolute maps. Section 4 deals with some closed maps, open maps, section 5 elaborates homeomorphisms. Section 6 Explains the concepts seperation of axioms and in section 7 Elobarates the concept of locally closed sets, furher LC-continuous. In section 8 Bitopological spaces are explained.

Complete in thesis $\TSP$, $\TSQ$ and $\TSR$ denote the topological spaces $(\TSP,\tau)$, $(\TSQ,\sigma)$ and $(\TSR, \eta)$ respectively for which separation axioms are not assumed untill mentioned explicitly. For each subset $\clrD$ of a space $(P, \tau)$, the closure of $\clrD$, interior of $\clrD$, semi-interior of $\clrD$, semi-closure of $\clrD$, w-interior of $\clrD$, ws-closure of $\clrD$, and the complement of $\clrD$ are denoted by $\mbox{cl}(\clrD)$ or $\tau-\mbox{cl}(\clrD)$, $\mbox{int}(\clrD)$ or $\tau-\mbox{int}(\clrD)$, $\mbox{sint}(\clrD)$, $\mbox{scl}(\clrD)$, $\mbox{ws-int}(\clrD)$, $\mbox{ws-cl}(\clrD)$, and $\clrD^{\text{C}}$ or $\TSP$--$\TSD$ respectively.

\section{Stronger and weaker forms of open sets and\break closed sets}\label{sec1.2}

Stone \cite{Stone} and Velicko \cite{Velicko} have popularized and investigated stronger type open sets termed as regular, strong regular open sets respectively. Levine \cite{Levine}, Bhattacharya and Lahiri \cite{Bhattacharya}, Mashhour \cite{abd}, Biswas \cite{Biswas}, Gnanambal \cite{Gnanambal}, Veera Kumar \cite{Veerakumar}, Palanippan and Rao \cite{Palaniappan}, Sheik Jhon \cite{Sheik} and Nagaveni \cite{Nagaveni} have respectively determined generalized, semi-generalized, $\alpha$, semi, g*, regular generalized, g\#, w and weakly generalized closed sets. The complements of these different types of open (closed) sets are called the same type of closed (open) sets and so on. The following definitions are the prerequisites for the present study.

\begin{dfn}\label{dfn1.2.1}
For subset $\clrD$ of $(\TSP, \tau)$, if $\clrD \subseteq \cl$ $(\mbox{int} (\clrD))$ then it is semi open set and If int $(\cl(\clrD)) \subseteq \clrD$ then it is semi closed set \cite{Levine1}.
\end{dfn}

\begin{dfn}\label{dfn1.2.2}
For subset $\clrD$ of $(P, \tau)$, if $\clrD \subseteq \mbox{int}(\cl(\clrD))$ then it is pre-open set (1982) and if $\cl(\mbox{int}(\clrD)) \subseteq \clrD$then it is pre-closed set \cite{Arya1}. 
\end{dfn}

\begin{dfn}\label{dfn1.2.3}
For subset $\clrD$ of $(P, \tau)$, if $\clrD \subseteq\mbox{int} (\cl(\mbox{int}(\clrD)))$ then it is $\alpha$-open set \cite{Njastad} (1965) and if $\cl(\mbox{int}(\cl(\clrD)))\subseteq \clrD$ then it is $\alpha$-closed set.
\end{dfn}

\begin{dfn}\label{dfn1.2.4}
For subset $\clrD$ of $(\TSP, \tau)$, if $\clrD \subseteq \cl(\mbox{int}(\cl(\clrD))))$ then it is semi-pre-open set (1986) ($\beta$-open \cite{abd}) and if $\mbox{int}(\cl(\mbox{int}(\clrD))) \subseteq \clrD$ then it is a semi-pre closed set ($\beta$-closed). 
\end{dfn}

\begin{dfn}\label{dfn1.2.5}
For a subset $\clrD$ of $(\TSP, \tau)$, if $\clrD = \mbox{int} (\cl(\clrD))$ then it is regular open set (1937) and if $\clrD = \cl(\mbox{int}(\clrD))$ then it is a regular closed set \cite{Long}.
\end{dfn}

\begin{dfn}\label{dfn1.2.6}
For subset $\clrD$ of $(\TSP, \tau)$, if there is a regular open set $\TSP$ such that $\TSP \subseteq \clrD \subseteq \alpha \cl(\TSP)$ then it is regular semi open set (1978) \cite{Cameron}. 
\end{dfn}

\begin{dfn}\label{dfn1.2.7}
For subset $\clrD$ of a topological space $(\TSP, \tau)$ 
\end{dfn}

{\fontsize{10}{12}\selectfont
\begin{longtable}{@{}|p{.9cm}|>{\raggedright}p{5cm}|>{\centering}p{2.5cm}|>{\centering}p{1.7cm}|>{\centering}p{2.8cm}|@{}}
\hline
\textbf{Sl.no} & \textbf{Name of the set} & \textbf{if} & \textbf{whenever} & {\boldmath $\TSM$} \textbf{is}\tabularnewline
\hline
1 & Generalized closed set (g-closed) \cite{Levine} & $\cl(\clrD) \subseteq \TSM$ & $\clrD \subseteq \TSM$ & open in $(\TSP, \tau)$.\tabularnewline
\hline
2 & Generalized semi closed set (gs-closed) \cite{Arya} & $\scl(\clrD) \subseteq \TSM$ & $\clrD \subseteq \TSM$ & open in $(\TSP, \tau)$\tabularnewline
\hline
3 & Semi generalized closed set (sg-closed) \cite{Bhattacharya1} & $\scl(D) \subseteq \TSM$ & $\clrD \subseteq \TSM$ & semi open in $(\TSP, \tau)$.\tabularnewline
\hline
4 & Generalized semi pre closed set (gsp-closed) \cite{Dontchev} & $\spcl(\clrD) \subseteq \TSM$ & $\clrD \subseteq \TSM$ & open in $(\TSP, \tau)$.\tabularnewline
\hline
5 & $\alpha$-generalized closed set ($\alpha$g-closed) \cite{Maki3} & $\alpha\cl(\clrD) \subseteq \TSM$ & $\clrD\subseteq\TSM$ & open in $(\TSP, \tau)$.\tabularnewline
\hline
6 & Generalized $\alpha$-closed set (g$\alpha$ closed) \cite{Maki3} & $\alpha\cl(\clrD) \subseteq \TSM$ & $\clrD\subseteq\TSM$ & $\alpha$-open in $(\TSP, \tau)$.\tabularnewline
\hline
7 & Regular generalized closed set (rg-closed) \cite{Palaniappan} & $\cl(\clrD) \subseteq \TSM$ & $\clrD \subseteq \TSM$ & Regular-open in $(\TSP, \tau)$.\tabularnewline
\hline
8 & Generalized pre closed set (gp-closed) \cite{Noiri1} & $\pcl(\clrD) \subseteq \TSM$ & $\clrD \subseteq \TSM$ & open in $(\TSP, \tau)$.\tabularnewline
\hline
9 & Generalized pre regular closed set (gpr-closed) \cite{Gnanambal} & $\pcl(\clrD) \subseteq \TSM$ & $\clrD \subseteq \TSM$ & regular open in $(\TSP, \tau)$.\tabularnewline
\hline
10 & w-closed set \cite{Sheik} & $\cl(\clrD) \subseteq \TSM$ & $\clrD \subseteq \TSM$ & semi-open in $(\TSP, \tau)$.\tabularnewline
\hline
11 & swg-closed set \cite{Nagaveni6} & $\cl(\mbox{int}(\clrD)) \subseteq \TSM$ & $\clrD \subseteq \TSM$ & semi-open in $(\TSP, \tau)$.\tabularnewline
\hline
12 & rw-closed set \cite{Benchalli} & $\cl(\clrD) \subseteq \TSM$ & $\clrD \subseteq \TSM$ & regular semi-open in $(\TSP, \tau)$.\tabularnewline
\hline
13 & R*-closed set \cite{Janaki3} & $\rcl(\clrD) \subseteq \TSM$ & $\clrD \subseteq \TSM$ & regular semi-open in $(\TSP, \tau)$.\tabularnewline
\hline
14 & rgw-closed set \cite{Sanjay} & $\cl(\mbox{int}(\clrD))\subseteq \TSM$ & $\clrD \subseteq \TSM$ & regular semi-open in $(\TSP, \tau)$.\tabularnewline
\hline
15 & Wgr$\alpha$-closed set \cite{Janaki1} & $\cl(\mbox{int}(\clrD)) \subseteq \TSM$ & $\clrD \subseteq \TSM$ & regular $\alpha$-open in $(\TSP, \tau)$\tabularnewline
\hline
16 & pgpr-closed set \cite{Anitha} & $\pcl(\clrD)) \subseteq \TSM$ & $\clrD \subseteq \TSM$ & rg-open in $(\TSP, \tau)$\tabularnewline
\hline
17 & rps-closed set \cite{Shyala} & $\spcl(\clrD) \subseteq \TSM$ & $\clrD \subseteq \TSM$ & rg-open in $(\TSP, \tau)$.\tabularnewline
\hline
18 & gprw-closed set \cite{Sanjay1} & $\pcl(\clrD) \subseteq \TSM$ & $\clrD \subseteq \TSM$ & regular semi-open in $(\TSP, \tau)$.\tabularnewline
\hline
19 & $\alpha$rw-closed set \cite{Wali3} & $\alpha\cl(\clrD) \subseteq \TSM$ & $\clrD\subseteq\TSM$ & rw-open in $(\TSP, \tau)$.\tabularnewline
\hline
%22 & g$\alpha${*}{*}-closed set \cite{key} & $\cl(\clrD) \subseteq\mbox{int}(\cl(M))$ & $\clrD \subseteq \TSM$ & $\alpha$-open in $(\TSP,\tau)$.\tabularnewline
%\hline
20 & $\psi$-closed set \cite{Veerakumar2} & $\scl(\clrD) \subseteq \TSM$ & $\clrD \subseteq \TSM$ & sg-open in $(\TSP, \tau)$.\tabularnewline
\hline
21 & g\#-closed set \cite{Veerakumar} & $\cl(\clrD) \subseteq \TSM$ & $\clrD \subseteq \TSM$ & $\TSM is \alpha$g-open in $(\TSP, \tau)$.\tabularnewline
\hline
22 & $\alpha$gp-closed set \cite{Navalagi2} & $\cl(\clrD) \subseteq \TSM$ & $\clrD \subseteq \TSM$ & pre-open in $(\TSP, \tau)$.\tabularnewline
\hline
%26 & pgr$\alpha$-closed set \cite{key} & $\cl(\clrD) \subseteq \TSM$ & $\clrD \subseteq \TSM$ & regular $\alpha$-open in $(\TSP, \tau)$.\tabularnewline
%\hline
23 & $\beta$wg*-closed set \cite{Dhanapakyam} & $\gcl(\clrD) \subseteq \TSM$ & $\clrD \subseteq \TSM$ & $\beta$-open in $(\TSP, \tau)$.\tabularnewline
\hline
24 & *g$\alpha$-closed set \cite{Vigneshwaran} & if $\cl(\clrD) \subseteq \TSM$ & $\clrD \subseteq \TSM$ & g$\alpha$-open in $(\TSP,\tau)$.\tabularnewline
\hline
25 & {*}{*}g$\alpha$-closed set \cite{Vigneshwaran1} & $\cl(\clrD) \subseteq \TSM$ & $\clrD \subseteq \TSM$ & *g$\alpha$-open in $(\TSP, \tau)$.\tabularnewline
\hline
26 & g$\alpha$b-closed set \cite{Nagaveni7} & $\bcl(\clrD) \subseteq \TSM$ & $\clrD \subseteq \TSM$ & $\alpha$-open in $(\TSP, \tau)$.\tabularnewline
\hline
%31 & sgb-closed set \cite{Nagaveni7} & $\bcl(\clrD) \subseteq \TSM$ & $\clrD \subseteq \TSM$ & semi-open in $(\TSP, \tau)$.\tabularnewline
%\hline
27 & rgb-closed set \cite{Mariappa} & $\bcl(\clrD) \subseteq \TSM$ & $\clrD \subseteq \TSM$ & regular-open in $(\TSP, \tau)$.\tabularnewline
\hline
28 & rg*b-closed set \cite{Indirani} & $\bcl(\clrD) \subseteq \TSM$ & $\clrD \subseteq \TSM$ & rg-open in $(\TSP, \tau)$.\tabularnewline
\hline
%34 & Pre-semi-closed set \cite{key} & $\spcl(\clrD) \subseteq \TSM$ & $\clrD\subseteq\TSM$ & $\spcl(\clrD) \subseteq \TSM$\tabularnewline
%\hline
29 & r\textasciicircum{g}-closed set \cite{Janaki2} & $\gcl(\clrD) \subseteq \TSM$ & $\clrD \subseteq \TSM$ & regular-open in $(\TSP, \tau)$.\tabularnewline
\hline
%36 & $\hat{\TSg}$-closed set \cite{VeeraKumar3} & $\cl(\clrD) \subseteq \TSM$ & $\clrD \subseteq \TSM$ & semi-open in $(\TSP, \tau)$\tabularnewline
%\hline
%37 & \#gs-closed set \cite{key} & $\scl(\clrD) \subseteq \TSM$ & $\clrD \subseteq \TSM$ & *g-open in $(\TSP, \tau)$.\tabularnewline
%\hline
%38 & $\tilde{\TSg}$-closed set \cite{key} & $\cl(\clrD) \subseteq \TSM$ & $\clrD \subseteq \TSM$ & \#gs-open in $(\TSP, \tau)$.\tabularnewline
%\hline
30 & g\#$\alpha$-closed set \cite{Devi} & $\alpha\cl(\clrD) \subseteq \TSM$ & $\clrD\subseteq\TSM$ & g-open in $(\TSP, \tau)$.\tabularnewline
\hline
31 & $\alpha$gs-closed set \cite{Rajamani} & $\alpha\cl(\clrD) \subseteq \TSM$ & $\clrD \subseteq \TSM$ & semi-open in $(\TSP, \tau)$.\tabularnewline
\hline
32 & g\#s-closed set \cite{VeeraKumar1} & $\scl(\clrD) \subseteq \TSM$ & $\clrD \subseteq \TSM$ & $\TSM$ is $\alpha$g-open in $(\TSP, \tau)$.\tabularnewline
\hline
33 & gb-closed set \cite{Ahmad} & $\bcl(\clrD) \subseteq \TSM$ & $\clrD \subseteq \TSM$ & $\TSM$ is -open in $(\TSP, \tau)$.\tabularnewline
\hline
34 & rb-closed set \cite{Nagaveni2} & $\cl(\clrD) \subseteq \TSM$ & $\clrD \subseteq \TSM$ & b-open in $(\TSP, \tau)$.\tabularnewline
\hline
35 & swg*-closed set \cite{Nagaveni} & $\gcl(\clrD) \subseteq \TSM$ & $\clrD \subseteq \TSM$ & semi-open in $(\TSP, \tau)$.\tabularnewline
\hline
36 & gr-closed set \cite{Bhattacharya1} & $\rcl(\clrD) \subseteq \TSM$ & $\clrD \subseteq \TSM$ & open in $(\TSP, \tau)$.\tabularnewline
\hline
37 & $\beta$wg{*}{*}-closed set \cite{Subashini} & $\beta$wg*$\cl(\clrD) \subseteq \TSM$ & $\clrD \subseteq \TSM$ & regular-open in $(\TSP, \tau)$.\tabularnewline
\hline
%47 & $\alpha\hat{\TSg}$s-closed set \cite{key} & $\scl(\clrD) \subseteq \TSM$ & $\clrD \subseteq \TSM$ & $\alpha$gs-open in $(\TSP, \tau)$.\tabularnewline
%\hline
%48 & w$\alpha\hat{\TSg}$-closed set \cite{key} & $\cl(\mbox{int}(\clrD)) \subseteq \TSM$ & $\clrD \subseteq \TSM$ & $\alpha$gs-open in $(\TSP, \tau)$.\tabularnewline
%\hline
%49 & R-closed set \cite{key} & $\alpha \cl(\clrD) \subseteq\mbox{int}(\TSM)$ & $\clrD \subseteq \TSM$ & w-open in $(\TSP, \tau)$.\tabularnewline
%\hline
38 & g$\alpha$*-closed set \cite{Maki3} & $\alpha\cl(\clrD) \subseteq\mbox{int}(\TSM)$ & $\clrD \subseteq \TSM$ & $\alpha$-open in $(\TSP, \tau)$.\tabularnewline
\hline
39 & \"g-closed set \cite{key} & $\cl(\clrD) \subseteq \TSM$ & $\clrD \subseteq \TSM$ & sg-open in $(\TSP, \tau)$.\tabularnewline
\hline
%52 & *g$\alpha$-closed set \cite{Vigneshwaran} & $\alpha\cl(\clrD) \subseteq \TSM$ & $\clrD \subseteq \TSM$ & g$\alpha$-open in $(\TSP, \tau)$.\tabularnewline
%\hline
40 & \#g$\alpha$-closed set \cite{Maki4} & $\alpha\cl(\clrD) \subseteq \TSM$ & $\clrD \subseteq \TSM$ & g\#$\alpha$-open in $(\TSP, \tau)$.\tabularnewline
\hline
41 & g$\xi$*-closed set \cite{VivekPrabu} & $\alpha\cl(\clrD) \subseteq \TSM$ & $\clrD \subseteq \TSM$ & \#g$\alpha$-open in $(\TSP, \tau)$.\tabularnewline
\hline
42 & g\#-pre-closed set \cite{Pious} & $\pcl(\clrD) \subseteq \TSM$ & $\clrD \subseteq \TSM$ & g\#-open in $(\TSP, \tau)$.\tabularnewline
\hline
43 & gps-closed set \cite{Gnanambal1} & $\pcl(\clrD) \subseteq \TSM$ & $\clrD \subseteq \TSM$ & semi open in $(\TSP, \tau)$.\tabularnewline
\hline
44 & gspr-closed set \cite{Navalagi} & $\spcl(\clrD) \subseteq \TSM$ & $\clrD \subseteq \TSM$ & regular-open in $(\TSP, \tau)$.\tabularnewline
\hline
45 & wg-closed set \cite{Nagaveni3} & $\cl(\mbox{int}(\clrD)) \subseteq \TSM$ & $\clrD \subseteq \TSM$ & open in $(\TSP, \tau)$.\tabularnewline
\hline
46 & rg$\alpha$-closed set \cite{Vadivel} & $\alpha\cl(\clrD) \subseteq \TSM$ & $\clrD \subseteq \TSM$ & regular $\alpha$-open in $(\TSP, \tau)$.\tabularnewline
\hline
47 & w$\alpha$-closed set \cite{Benchalli1} & $\alpha\cl(\clrD) \subseteq \TSM$ & $\clrD \subseteq \TSM$ & w-open in $(\TSP, \tau)$.\tabularnewline
\hline
48 & gw$\alpha$-closed set \cite{Benchalli2} & $\alpha\cl(\clrD) \subseteq \TSM$ & $\clrD \subseteq \TSM$ & w$\alpha$-open in $(\TSP, \tau)$\tabularnewline
\hline
\end{longtable}}

The compliment of the above mentioned definition closed sets are their open sets. 

\section{ws-Continuous maps and irresolute maps}

In general topology, the concept of continuous functions plays a very important role. This section deals with continuous maps and irresolute maps contributed by few topologist namely Arya and Gupta \cite{Arya1},  Levine \cite{Levine1}, Balachandran et al \cite{Maki11}, Mashhour \cite{Abd1}, Sundaram \cite{Maki2},  Gnanambal \cite{Gnanambal}, Palaniappan \cite{Palaniappan}, Sheik John \cite{Sheik1}, and R.S. Walli \cite{Wali4} has explored regular continuous and completely, semi,  $\alpha$, pre, g.sg, gpr, rg, w, $\alpha rw$-continuous. Irresolute maps were innovated and explored by Crossely and Hildebrand \cite{Crossley}.

\begin{dfn}\label{dfn1.3.1}
In a map $\TSh: (\TSP, \tau) →(\TSQ, \sigma)$
\begin{enumerate}
\item Pretend $\TSh^{-1}(\clrD)$ is r-closed in $\TSP$ for all closed subset $\clrD$ of $\TSQ$ then it is termed as regular-continuous(r-continuous) \cite{Arya1}. 
\item Pretend $\TSh^{-1}(\clrD)$ is regular closed in $\TSP$ for all closed subset $\clrD$ of $\TSQ$ then it is termed as completely-continuous \cite{Levine2}. 
%\item Pretend $\TSh^{-1}(\clrD)$ is clopen (both open and closed) in $\TSP$ for all subset $\clrD$ of $\TSQ$ then it is termed as strongly-continuous \cite{Maki11}. 
\item Pretend $\TSh^{-1}(\clrD)$ is $\alpha$-closed in $\TSP$ for all closed subset $\clrD$ of $\TSQ$ then it is termed as $\alpha$-continuous \cite{Maki11}. 
\item Pretend $\TSh^{-1}(\clrD)$ is $\alpha$-closed in $\TSP$ for all semi-closed subset $\clrD$ of $\TSQ$ then it is termed as strongly $\alpha$-continuous \cite{Maki11}. 
\item Pretend $\TSh^{-1}(\clrD)$ is w-closed in $\TSP$ for all closed subset $\clrD$ of $\TSQ$ then it is termed as $\omega$-continuous \cite{Sheik}. 
\item Pretend $\TSh^{-1}(\clrD)$ is rw-closed in $\TSP$ for all closed subset $\clrD$ of $\TSQ$ then it is termed as $\TSr\omega$-continuous \cite{Benchalli}. 
\end{enumerate}
\end{dfn}

\begin{dfn}\label{dfn1.3.2} 
In a map $\TSh: (\TSP, \tau) \rightarrow (\TSQ, \sigma)$
\begin{enumerate}
\item Pretend $\TSh^{-1}(\clrD)$ is $\alpha$-closed in $\TSP$ for all $\alpha$-closed subset $\clrD$ of $\TSQ$ then it is termed as $\alpha$-irresolute \cite{Maki11}. 
\item Pretend $\TSh^{-1}(\clrD)$ is semi-closed in $\TSP$ for all semi-closed subset $\clrD$ of $\TSQ$ then it is termed as irresolute \cite{Crossley}.
\item Pretend $\TSh^{-1}(\clrD)$ is $\omega$-open in $\TSP$ for all w-closed subset $\clrD$ of $\TSQ$ then it is termed as contra w-irresolute \cite{Sheik1}. 
\item Pretend $\TSh^{-1}(\clrD)$ is semi-open in $\TSP$ for all semi-closed subset $\clrD$ of $\TSQ$ then it is termed as contra irresolute \cite{Baker}. 
\item Pretend $\TSh^{-1}(\clrD)$ is open in $\TSP$ for all closed subset $\clrD$ of $\TSQ$ then it is termed as contra continuous \cite{Dontchev1}. 
\end{enumerate}
\end{dfn}


\begin{dfn}\label{dfn1.3.3} 
A topological space $(\TSP, \tau)$ is termed as 
\begin{enumerate}[\rm (i)]
\item $\TST_{\sfrac{1}{2}}$ space \cite{Malghan} if each semi-closed set is closed. 
\item $\TST_{\text{ws}}$ space \cite{Basavaraj} if each ws-closed set is closed. 
\end{enumerate}
\end{dfn}

\section{Closed and open maps}

Malghan \cite{Malghan} explored and deliberated generalised closed maps. Sundaram \cite{Maki2}, Arockirani \cite{Arockiarani}, Nagaveni \cite{Nagaveni}, Sheik John \cite{Sheik1}, Benchalli \cite{Benchalli} and R. S. Walli \cite{Wali2}, Pushpalatha \cite{Pushpalatha}, Crossely and Hildebrand \cite{Crossley}, mashhour \cite{Abd1} and Gnanambal \cite{Gnanambal} has explored generalised open maps, rg-closed maps, wg closed and open maps, w-closed and open maps, rw-closed and open maps, $\alpha$rw-closed and open maps, g*-closed and open maps, pre semi open maps, $\alpha$-open maps, gpr-closed maps respectively.

\begin{dfn}\label{dfn1.4.1}
A map $\TSh: (\TSP, \tau) \rightarrow (\TSQ \sigma)$ is said to be
\begin{enumerate}
\item $\alpha$-closed map \cite{Maki10} if $\TSh(\TSN)$ is $\alpha$-closed in $\TSQ$ $\forall$ closed subset $\TSV$ of $\TSP$. 
\item gspr-closed map \cite{Navalagi} if $\TSh(\TSN)$ is gspr-closed in $\TSQ$ $\forall$ closed subset $\TSN$ of $\TSP$. 
\item semi-closed map \cite{Levine1} if $\TSh(\TSN)$ is semi-closed in $\TSQ$ $\forall$ closed subset $\TSN$ of $\TSP$. 
\item r$\omega$-closed map \cite{Benchalli} if $\TSh(\TSN)$ is rw-closed in $\TSQ \forall$ closed subset $\TSN$ of $\TSP$.
\item regular closed map \cite{Stone} if $\TSh(\TSN)$ is closed in $\TSQ$ $\forall$ regular closed set $\TSN$ of $\TSP$. 
\item Contra closed map \cite{Baker} if $\TSh(\TSN)$ is closed in $\TSQ$ $\forall$ open set $\TSN$ of $\TSP$. 
\item Contra regular closed map \cite{Stone} if $\TSh(\TSN)$ is r-closed in $\TSQ$ $\forall$ open set $\TSN$ of $\TSP$. 
\item Contra semi-closed map \cite{Navalagi3} if $\TSh(\TSN)$ is s-closed in $\TSQ$ $\forall$ open set $\TSN$ of $\TSP$. 

Homeomorphism

The concept of generalised homeomorphism explored and deliberated by Balachandran et all. \cite{Maki5}, Sheik John \cite{Sheik1}, Vadivel et al \cite{Vadivel}, 
Thangavel \cite{Shyala}, Maki has explored rwg- homeomorphism, w-homeomorphism, rg$\alpha$-homeomorphism, gs-homeomorphism and sg-homeomorphism.
\end{enumerate}
\end{dfn}


\begin{dfn}\label{dfn1.4.2}
Map $\TSh: \TSP \rightarrow \TSQ$ is said to be 
\begin{enumerate}
\item homeomorphism if $\TSh$ is both open and continuous
\item g-homeomorphism \cite{Maki5} if $\TSf$ is both g-continuous and g-open
\item gspr-homeomorphism \cite{Navalagi} if $\TSh$ is both gspr-continuous and gspr-open
\item gs-homeomorphism \cite{Maki12} if $\TSf$ is both gs-continuous and gs-open
\end{enumerate}
\end{dfn}

\section{Locally closed sets and LC-continuous maps}

Ganster and Reilly \cite{Ganster}, Sundaram \cite{Sundaram}, Arockkiarani \cite{Arockiarani}, Park \cite{Park} and Sheik John \cite{Sheik1} have respectively introduced and studied locally, generalized locally, regular generalized locally and w-locally closed sets. 

\begin{dfn}\label{dfn1.5.1}
In a subset $\clrD$ of topological space $(\TSP, \tau)$ 
\begin{enumerate}[\rm (i)]
\item If $\clrD=\TSM \cap \TSN$, where $\TSM$ is open and $\TSN$ is closed in $(\TSX,\tau)$ then it is called as locally closed (briefly lc) set \cite{Reilly1}. 
\item If $\clrD=\TSM \cap \TSN$, where $\TSM$ is $\alpha$-open and $\TSN$ is $\alpha$-closed in $(\TSX, \tau)$. then it is called as $\alpha$-locally closed (briefly $\alpha$lc) set \cite{Njastad}.
\item If $\clrD=\TSM \cap \TSN$, where $\TSM$ is w-open and $\TSN$ is w-closed in $(\TSX, \tau)$ then it is called as rw-locally closed (briefly wlc) set \cite{Sheik1}.
\item \cite{Sheik1} if $\clrD=\TSM \cap \TSN$, where $\TSM$ is open and $\TSN$ is semi-closed in $\TSX$, then it is called w-lc* set
%\item \cite{Sheik1} if $\clrD=\TSM \cap \TSN$, where $\TSM$ is open and $\TSN$ is w-closed in $\TSX$, then it is called w-lc* set
\item If $\clrD=\TSM\cap \TSN$, where $\TSM$ is open and $\TSN$ is semi-closed in $(\TSX, \tau)$ then it is called as semi - locally closed (briefly lsc) set \cite{Balachandran}.
\end{enumerate}
\end{dfn}

\begin{dfn}\label{dfn1.5.2} 
In a map $\TSh: (\TSP, \tau) \rightarrow (\TSQ, \sigma)$ 
\begin{enumerate}[\rm (i)]
\item Pretend $\TSh^{-1}(\clrD)$ is locally closed set in $(\TSP, \tau)$ for each open set $\TSP$ of $(\TSQ, \sigma)$ then it is called as LC-continuous \cite{Reilly1}.
\item Pretend $\TSh^{-1}(\clrD)$ is $\alpha$-locally closed set in $(\TSP, \tau)$ for each open set $\clrD$ of $(\TSQ, \sigma)$ then it is called as LC-continuous \cite{Njastad}.
%\item Pretend $\TSh^{-1}(\clrD)$ is $\alpha$-locally closed set in $(\TSP,\tau)$ for each open set $\clrD$ of $(\TSQ, \sigma)$ then it is called as w-LC continuous \cite{Sheik}.
\item Pretend $\TSh^{-1}(\clrD)$ is g-locally closed set in $(\TSP,\tau)$ for each open set $\TSD$ of $(\TSQ, \sigma)$ then it is called as GLC continuous \cite{Balachandran}.
\end{enumerate}
\end{dfn}

\begin{dfn}\label{dfn1.5.3} 
A map $\TSh: (\TSP, \tau)\rightarrow (\TSQ, \sigma)$ is termed
\begin{enumerate}[\rm (i)]
\item LC-irresolute \cite{Reilly1} if $\TSh^{-1}(\TSN)$ is a lc set in $(\TSP,\tau)$ $\forall$ lc-set $\TSN$ in $(\TSQ, \sigma)$,
\item w-LC-irresolute \cite{Sheik} if $\TSh^{-1}(\TSN)$ is a w-lc set in $(\TSP,\tau)$ $\forall$ w-lc set $\TSN$ in $(\TSQ, \sigma)$,
\item GLC-irresolute \cite{Balachandran} if $\TSh^{-1}(\TSN)$ is a glc set in $(\TSP,\tau)$ $\forall$ glc set $\TSN$ in $(\TSQ, \sigma)$,
\end{enumerate}
\end{dfn}

\section{Bitopological spaces}

Kelly \cite{Kelly} initiated a systematic study of the concept of bitopological spaces in 1963. Furthers various authors, like Arya and Nour \cite{Arya2}, Di Maio and Noiri \cite{Noiri2}, Fukutake \cite{Fukutake2}, Reilly \cite{Reilly}, Popa \cite{Popa1}, Maki \cite{Maki6}, Arockiarani \cite{Arockiarani}, Gnanambal \cite{Gnanambal} and Sheik Jhon \cite{Sheik} have changed their attention to put the individual concepts of topology to bitopological spaces. Here we define some of the definitions, which are used in our study. 

\begin{dfn}\label{dfn1.6.1}
Let $\TSi, \TSj\in \{1, 2\}$ be fixed integers. In a bitopological space $(\TSP, \tau_1, \tau_2)$, a subset $\clrD$ of $(\TSP, \tau_1, \tau_2)$ is said to
\begin{enumerate}[\rm (i)]
\item $(\TSi, \TSj)$-g\#-closed \cite{Veerakumar} if $\tau_\TSj- \cl(\clrD)\subset \TSM$ when $\clrD\subset\TSM$ and $\TSM\in \tau_\TSi$,
\item $(\TSi, \TSj)$-*g$\alpha$-closed \cite{Vigneshwaran} if $\tau_\TSj$-$\cl(\clrD)\subset\TSM$ when $\clrD\subset\TSM$ and $\TSM$ is regular open in $\tau_\TSi$,
\item $(\TSi, \TSj)$-g$\xi$*-closed \cite{VivekPrabu} if $\tau_\TSj$-$\mbox{pcl}(\clrD) \subset\TSM$ when $\clrD\subset\TSM$ and $\TSM$ is regular open in $\tau_i$,
\item $(\TSi, \TSj)$-$\alpha$gp-closed \cite{Navalagi2} if $\tau_\TSj$-$\cl(\tau_\TSi-\mbox{int}(\clrD))\subset\TSM$ when $\clrD\subset\TSM$ and $\TSM\in\tau_\TSi$,
\item $(\TSi, \TSj)$-$\ddot{g}$-closed \cite{key} if $\tau_\TSj-\cl(\clrD) \subset \TSM$ when $\clrD\subset\TSM$ and $\TSM$ is semiopen in $\tau_\TSi$,
\item $(\TSi, \TSj)$-rb-closed \cite{Nagaveni2} if $\tau_\TSj$-$\mbox{pcl}(\clrD) \subset\TSM$ when $\clrD\subset\TSM$ and $\TSM\in\tau_\TSi$,
\item $(\TSi, \TSj)$-gspr-closed \cite{Navalagi} if $\tau_\TSj$-$\cl(\clrD) \subset \TSM$ when $\clrD \subset \TSM$ and $\TSM\in\mbox{GO}(\TSP, \tau_\TSi)$.
\item $(\TSi, \TSj)$-gsp-closed \cite{Dontchev} if $\tau_\TSj$-$\cl(\clrD) \subset \TSM$ when $\clrD \subset \TSM$ and $\TSM\in \mbox{GO}(\TSP, \tau_\TSi)$.
\item $(\TSi, \TSj)$-rgb-closed \cite{Mariappa} if $\tau_\TSj$-$\cl(\clrD) \subset \TSM$ when $\clrD \subset \TSM$ and $\TSM\in \mbox{GO}(\TSP, \tau_i)$.
\end{enumerate}
\end{dfn}

The complements of the above mentioned closed sets are their respective open sets. 

\begin{dfn}\label{dfn1.6.2}
A map $\TSh:(\TSX, \tau_{1}, \tau_{2})\to (\TSY, \sigma_{1}, \sigma_{2})$ is called
\begin{enumerate}[(i)]
\item $\tau_{\TSj}$-$\sigma_{\TSk}$-continuous \cite{Maki6} if $f^{-1}(\TSV)\in \tau_{\TSj}$ for every $\TSV\in\sigma_{\TSk}$,

\item $\TSC(\TSi,\TSj)$-$\sigma_{\TSk}$-continuous \cite{Shaik John} if $f^{-1}(\TSV)\in \TSC(\TSi,\TSj)$ for every $\sigma_{\TSk}$-closed set in $(\TSV,\sigma_{1},\sigma_{2})$,

\item $\TSD(\TSi,\TSj)$-$\sigma_{\TSk}$-continuous \cite{Maki6} if $f^{-1}(\TSV)\in \TSD(\TSi,\TSj)$ for every $\sigma_{\TSk}$-closed set in $(\TSY,\sigma_{1},\sigma_{2})$,

\item $\TSW(\TSi,\TSj)$-$\sigma_{\TSk}$-continuous \cite{Fukutake} if $f^{-1}(\TSV)\in \TSW(\TSi,\TSj)$ for every $\sigma_{\TSk}$-closed set in $(\TSY,\sigma_{1},\sigma_{2})$,

\item $\clrD_{\TSr}(\TSi,\TSj)$-$\sigma_{\TSk}$-continuous \cite{Arockiarani} if $f^{-1}(\TSV)\in \clrD_{\TSr}(\TSi,\TSj)$ for every $\sigma_{\TSk}$-closed set in $(\TSY,\sigma_{1},\sigma_{2})$,

\item $\zeta(\TSi,\TSj)$-$\sigma_{\TSk}$-continuous \cite{Gnanambal} if $f^{-1}(\TSV)\in\zeta(\TSi,\TSj)$ for every $\sigma_{\TSk}$-closed set in $(\TSY,\sigma_{1},\sigma_{2})$.
\end{enumerate}
\end{dfn}

\begin{dfn}\label{dfn1.6.3}
A map $\TSh: (\TSP, \tau_1, \tau_2)\rightarrow(\TSQ \sigma_1, \sigma_2)$ is termed bi-continuous \cite{Maki6} if $\TSh$ is $\tau_1$-$\sigma_1$-continuous and $\tau_2$-$\sigma_2$-continuous.
\end{dfn}

\begin{dfn}\label{dfn1.6.4}
A map $\TSh: (\TSP, \tau_1, \tau_2)\rightarrow(\TSQ \sigma_1, \sigma_2)$ is termed strongly-bi-continuous \cite{Maki6} (briefly s-bi-continuous) if $\TSh$ is bi-continuous, $\tau_1$-$\sigma_2$-continuous and $\tau_2$-$\sigma_1$-{\break}continuous,
\end{dfn}

\section{Some separation axioms in topological spaces}

From the literature survey on separation axioms we observed that there is a significant work for different relatively weak form of separation axioms like $\TST_\TSk$ spaces $(\TSk=0, 1, 2)$, normal and regular axioms in particular several other neighbouring forms of them have been studied in many papers.

Maheshwari and Prasad \cite{Maheshwari} established and deliberated the new class of space called s-normal space using semi-open sets. It was further studied by Noiri \cite{Dontchev2} and Arya \cite{Arya1} introduced g-regular and g-normal spaces using g-closed sets. Noiri and Popa \cite{Popa} further investigated the concepts introduced by Sheik John \cite{Sheik1} introduced and studied the w-normal, w-regular using w-closed sets.

