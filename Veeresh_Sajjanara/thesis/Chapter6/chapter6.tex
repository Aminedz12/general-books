\chapter{Weakly semi closed separation axioms in topological spaces}
\graphicspath{{Chapter6/Chapter6Figs/EPS/}{Chapter6/Chapter6Figs/}}

\section{Introduction}\label{sec6.1}

From the literature survey on separation axioms, we observed that there is an significant work on weak forms of separation axioms like $\TST_{\TSk}$ spaces $(\TSk=0, 1, 2)$, normal and regular axioms in particular, several other neighbouring forms of them have been elaborated in many articals. In the year 1975, Maheshwari and Prasad \cite{Maheshwari} introduced a new class of spaces called s-normal space using semi open sets. Using the concept of generalized closed sets of Levine \cite{Levine}, T. Noiri and Papa \cite{Popa} introduced g-regular and g-normal spaces in topological spaces. Sheik John \cite{Sheik1} introduced and studied w-regular and w-normal spaces. Recently, Basavaraj M. Ittanagi et al \cite{Basavaraj} introduced and studied basic properties of ws-closed sets, ws-continuous functions and ws-closed maps in topological spaces. In this paper, using the notion of ws-open sets, ws-separation axioms are introduced and studied. Also the relationships with some other functions are discussed.

This chapter has four sections. In second section of this chapter, a new class of space known as ws- separation axioms namely ws-$\TST_{0}$ space, ws-$\TST_{1}$ space, ws-$\TST_{2}$ space, are introduced and studied. It is observed that each $\alpha$-$\TST_{0}$ is ws-$\TST_{0}$ space, each $\alpha$-$\TST_{1}$ is ws-$\TST_{1}$ space and each $\alpha$-$\TST_{2}$ is ws-$\TST_{2}$ space, few of its major results are obtained.

\newpage

In third section, we innovate the concepts of ws-regular spaces. It is observed that each ws-regular spaces are regular. Characterizations of ws-regular spaces in topological space are obtained.

In fourth section, we establish the concept of ws-normal spaces by deliberating its characterization with its few properties.

\section[$\ws-\TST_{\TSk}$ Space $(\TSk=0,1,2)$]{\boldmath$\ws-\TST_{\TSk}$ Space $(\TSk=0,1,2)$}\label{sec6.2}

\begin{dfn}\label{defi6.2.1}
A topological space $\TSP$ is called an $\ws$-$\TST_{0}$ if for each pair of distinct points a, b of $\TSP$, $\exists$ an ws-open sets $\TSH$ in $\TSP$ containing one of them and not the other.
\end{dfn}

\begin{exm}\label{exam6.2.1}
Let $\TSP= \{1, 2, 3, 4\}$ and $\tau =\{\TSP, \phi, \{1\},\{2\},\{1,2\},\{1,2,3\}\}$ Then $(\TSP, \tau)$ is $\ws$-$\TST_{0}$ space. Seeing that for two distinct points `$1$' and `$2$' of $(\TSP, \tau)$ there exist an ws-open set $\{1\} \ni: 1 \in \{1\}$, $2 \not\in \{1\}$
\end{exm}

\begin{thm}\label{thm6.2.1}
\begin{enumerate}[(i)]
\item $\forall \TST_{0}$-space is $\ws$-$\TST_{0}$ space
\item $\forall$ semi-$\TST_{0}$ space is $\ws$-$\TST_{0}$ space
\item $\forall$ $\alpha$-$\TST_{0}$ space is $\ws$-$\TST_{0}$ space
\item $\forall$ $\TSg\#$-$\TST_{0}$ space is $\ws$-$\TST_{0}$ space
\item $\forall$ $\ddot{g}$-$\TST_{0}$ space is $\ws$-$\TST_{0}$ space
\item $\forall$ $\TSg\xi$*-$\TST_{0}$ space is $\ws$-$\TST_{0}$ space
\item $\forall$ $\TSg\#\TSs$-$\TST_{0}$ space is $\ws$-$\TST_{0}$ space
\item $\forall$ $\rb$-$\TST_{0}$ space is $\ws$-$\TST_{0}$ space
\item $\forall$ regular-$\TST_{0}$ space is $\ws$-$\TST_{0}$ space
\end{enumerate}
\end{thm}

\begin{proof}
(i) Let $\TSP$ is a $\TST_{0}$ space. For each pair of disjoint points $\TSa$, $\TSb$ of $\TSP \exists$ open set $\TSH$ in $\TSP \ni: \TSa \in \TSH$, $\TSb \not\in \TSH$ and $\TSb \in \TSH$, $\TSa \not\in \TSH$. But every open set is $\ws$-open then $\exists$ $\ws$-open set $\TSH$ in $\TSP \ni: \TSa \in \TSH$, $\TSb \not\in \TSH$ and $\TSb\in \TSH$, $\TSa \not\in \TSH$. Henceforth $\TSP$ is a $\ws$-$\TST_{0}$ space.

Similarly the remaining (ii to ix) results can be proved

We use example \ref{exam6.2.2} to prove the inverse of theorem is untrue.
\end{proof}

\begin{exm}\label{exam6.2.2}
Let $\TSP= \{1,2,3,4\}$ and $\tau =\{\TSP, \phi, \{1\}, \{1,2\},\{1, 2 ,3\}\}$. Then $\TSg\#$-(respectively $\ddot{g}$, *$\TSg\alpha$) open $(\TSP, \tau)=\{\TSP, \phi, \{1\}, \{1,2\},\{1, 2,3\}\}$. $\alpha$-(semi-open, $\alpha\TSg\TSp$, $\TSg\#\TSs$, *$\TSg\alpha$) open $(\TSP, \tau) =\{\TSP,\phi, \{1\}, \{1,2\},\{1,3\},\{1,4\},\{1,2,3\},\{1,2,4\},\{1,3,4\}\}$, $\rb$-open=$\{\TSP, \phi, \{1\}\}$, $\ws$-open $(\TSP, \tau)=\{\TSP,\phi, \{1\},\{2\},\{3\},\{1,2\},\{1,3\},\{1,4\},\break \{2,3\},\{1, 2, 3\},\{1,2, 4\},\{1, 3, 4\}\}$. Here $(\TSP, \tau)$ is $\ws$-$\TST_{0}$ space but it is not $\TST_{0}$-space, $\TSg\#$-$\TST_{0}$ space, $\ddot{g}$-$\TST_{0}$ space, *$\TSg\alpha$-$\TST_{0}$ space, $\alpha$-$\TST_{0}$ space, $\TSg\#\TSs$-$\TST_{0}$ space, $\TSg\xi$*-$\TST_{0}$ space and $\rb$-$\TST_{0}$ space. Seeing that for two distinct points `$1$' and `$2$' there do not exist open set, $\alpha$-open set, regular-open set $\TSG$ in $\TSP$ containing one of them and not the other.
\end{exm}

\begin{thm}\label{thm6.2.2}
Let $\TSP$ is a topological space and $\TSQ$ is an $\ws$-$\TST_{0}$ space. If $\TSh: \TSP \to Q$ is injective and $\ws$-irresolute then $\TSP$ is $\ws$-$\TST_{0}$ space.
\end{thm}

\begin{proof}
Suppose $\TSa$, $\TSb \in \TSP \ni:\TSa \neq \TSb$. Seeing that $\TSh$ is injective thus $\TSf(\TSa) \neq \TSf(\TSb)$. Also $\TSQ$ is $\ws$-$\TST_{0}$ space then $\exists$ a $\ws$-open sets $\TSM$ in $\TSQ \ni: \TSf(\TSa) \in \TSM$, $\TSf(\TSb) \not\in \TSM$ or $\exists$  a $\ws$-open sets $\TSN$ in $\TSQ \ni :\TSf(\TSb) \in \TSN$, $\TSf(\TSa) \not\in \TSN$ with $\TSf(\TSa) \neq \TSf(\TSb)$. Seeing that $\TSh$ is $\ws$-irresolute thus $\TSh^{-1} (\TSM)$ is a $\ws$-open sets in $\TSP \ni: \TSa \in \TSh^{-1} (\TSM)$, $\TSb \not\in \TSh^{-1} (\TSM)$ or $\TSh^{-1} (\TSN)$ is a $\ws$-open sets in $\TSP \ni:\TSb \in \TSh^{-1} (\TSN)$, a $\not\in \TSh^{-1} (\TSN)$. Hence $\TSP$ is $\ws$-$\TST_{0}$ space.
\end{proof}

\begin{thm}\label{thm6.2.3}
$(\TSP, \tau)$ is $\ws$-$\TST_{0}$ space iff for each pair of distinct $\TSa$, $\TSb$ of $\TSP$, $\ws$-$\cl(\{1\}) \neq \ws-\cl(\{2\})$.
\end{thm}

\begin{proof}
Take $(\TSP, \tau)$ be $\ws$-$\TST_{0}$ space. Assume $\TSa$, $\TSb \in \TSP \ni:\TSa \neq \TSb$, thus $\exists$ a $\ws$-open set $\TSN$ containing one of the points but not the other, say a $\in \TSN$ and $\TSb \not\in \TSN$. Then $\TSN^{\text{C}}$ is $\ws$-closed containing $\TSb$ but not $\TSa$. But $\ws$-$\cl(\{2\})$ is the smallest $\ws$-closed set containing $\TSb$. Henceforth $\ws$-$\cl(\{2\}) \subset \TSN^{\text{c}}$ and hence $\TSa\not\in \ws$-$\cl(\{2\})$. Thus $\ws$-$\cl(\{1\}) \neq \ws$-$\cl(\{2\})$. Inversely, Pretend $\TSa$, $\TSb \in \TSP$, $\TSa\neq \TSb$ and $\ws$-$\cl(\{1\}) \neq \ws$-$\cl(\{2\})$. Assume $\TSc\in \TSP \ni: \TSc\in \ws$-$\cl(\{1\})$ but $\TSc \not\in \ws$-$\cl(\{2\})$. If a $\in \ws$-$\cl(\{2\})$ then $\ws$-$\cl(\{1\}) \subset \ws$-$\cl(\{2\})$ and hence $\TSc \in \ws$-$\cl(\{2\})$. This is a contradiction. Henceforth $\TSa\not\in \ws$-$\cl(\{2\})$. That is a $\in (\ws\text{-}\cl(\TSb))^{\text{c}}$. Henceforth ($\ws$-$\cl(\{2\}))^{\text{c}}$ is $\ws$-open set containing $\TSa$ but not $\TSb$. Hence $(\TSP, \tau)$ is $\ws$-$\TST_{0}$ space.
\end{proof}

\begin{dfn}\label{defi6.2.2}
A topological space $\TSP$ is called a $\ws$-$\TST_{1}$ if for each pair of distinct points $\TSa$, $\TSb$ of $\TSP$, $\exists$ a $\ws$-open sets $\TSH_{1}$, $\TSH_{2}$ in $\TSP \ni: \TSa \in \TSH_{1}$, $\TSb \not\in \TSH_{1}$ and $\TSb \in \TSH_{2}$, $\TSa \not\in \TSH_{2}$.
\end{dfn}

\begin{exm}\label{exam6.2.3}
Let $\TSP= \{1, 2, 3\}$ and $\tau =\{\TSP, \phi, \{1\}, \{2, 3\}\}$. Then $(\TSP, \tau)$ is $\ws$-$\TST_{1}$ space. Seeing that for two distinct points `$1$' and `$2$' of $(\TSP, \tau) \exists$ an $\ws$-open set $\{1\}, \{2\} \ni: 1 \in \{1\}$, $2 \not\in \{1\}$ and $2 \in \{2\}$, $1 \not\in \{2\}$.
\end{exm}

\begin{thm}\label{thm6.2.4}
\begin{enumerate}[(i)]
\item $\forall$ $\TST_{1}$-space is $\ws$-$\TST_{1}$ space
\item $\forall$ semi - $\TST_{1}$ space is $\ws$-$\TST_{1}$ space
\item $\forall$ $\alpha$-$\TST_{1}$ space is $\ws$-$\TST_{1}$ space
\item $\forall$ $\TSg\#$-$\TST_{1}$ space is $\ws$-$\TST_{1}$ space
\item $\forall$ $\ddot{g}$-$\TST_{1}$ space is $\ws$-$\TST_{1}$ space
\item $\forall$ $\TSg\xi$*-$\TST_{1}$ space is $\ws$-$\TST_{1}$ space
\item $\forall$ $\TSg\#\TSs$-$\TST_{1}$ space is $\ws$-$\TST_{1}$ space
\item $\forall$ $\rb$-$\TST_{1}$ space is $\ws$-$\TST_{1}$ space
\item $\forall$ regular-$\TST_{1}$ space is $\ws$-$\TST_{1}$ space
\end{enumerate}
\end{thm}

\begin{proof}
(i) Take $\TSP$ is a $\TST_{1}$ space. For each pair of disjoint points $\TSa$, $\TSb$ of $\TSP \ \exists$  open set $\TSH_{1}$, $\TSH_{2}$ in $\TSP \ni: \TSa \in \TSH_{1}$, $\TSb \not\in \TSH_{1}$ and a $\in \TSH_{2}$, $\TSb \not\in \TSH_{2}$. But each open set is $\ws$-open then $\exists$  $\ws$-open set $\TSH_{1}$, $\TSH_{2}$ in $\TSP \ni: \TSa \in \TSH_{1}$, $\TSb \not\in \TSH_{1}$ and $\TSb \in \TSH_{2}$, $\TSa \not\in \TSH_{2}$ Henceforth $\TSP$ is a $\ws$-$\TST_{1}$ space.

Similarly the remaining (ii to ix) results can be proved.
\end{proof}

We use example \ref{exam6.2.4} to prove the inverse of theorem is untrue.

\begin{exm}\label{exam6.2.4}
Let $\TSP= \{1,2,3,4\}$ and $\tau =\{\TSP, \phi, \{1\}, \{1, 2\},\{1, 2, 3\}\}$. Then $\TSg\#$-(respectively $\ddot{g}$, *$\TSg\alpha$)open $(\TSP, \tau)=\{\TSP, \phi, \{1\}, \{1,2\},\{1, 2 ,3\}\}$. $\alpha$-(semi-open, $\alpha gp$, $\TSg\#\TSs$, *$\TSg\alpha$)open $(\TSP, \tau)=\{\TSP, \phi, \{1\}, \{1, 2\},\{1, 3\},\{1, 4\},\{1, 2, 3\},\{1, 2, 4\},\{1, 3, 4\}\}$, rb-open$=\{\TSP, \phi, \{1\}\}$, ws-open $(\TSP, \tau)=\{\TSP, \phi, \{1\},\{2\},\{3\},\{1, 2\},\{1, 3\},\{1, 4\},\break \{2 ,3\},\{1, 2, 3\},\{1, 2, 4\},\{1, 3, 4\}\}$. Here $(\TSP, \tau)$ is $\ws$-$\TST_{1}$ space but it is not $\TST_{1}$-space, $\TSg\#$-$\TST_{1}$ space, $\ddot{g}$-$\TST_{1}$ space, *$\TSg\alpha$-$\TST_{1}$ space, $\alpha$-$\TST_{1}$ space, $\TSg\#\TSs$-$\TST_{1}$ space, $\TSg\xi$*-$\TST_{1}$ space and rb-$\TST_{1}$ space. Seeing that for two distinct points `$1$' and `$2$' there do not exist open set, $\TSg\#$-open, $\ddot{g}$-open, *$\TSg\alpha$-open, $\alpha$-open, $\TSg\#\TSs$-open, $\TSg\xi$*-open and $\rb$-open set $\TSH_{1}$, $\TSH_{2}$ in $\TSP \ni: 1 \in \TSH_{1}$, $2 \not\in \TSH_{1}$ and $2 \in \TSH_{2}$, $1 \not\in \TSH_{2}$.
\end{exm}

\begin{thm}\label{thm6.2.5}
Let $\TSP$ is a topological space and $\TSQ$ is a $\ws$-$\TST_{1}$ space. If $\TSh: \TSP \to \TSQ$ is injective and $\ws$-irresolute then $\TSP$ is $\ws$-$\TST_{1}$ space.
\end{thm}

\begin{proof}
Theorem \ref{thm6.2.2}.
\end{proof}

\begin{thm}\label{thm6.2.6}
If $(\TSP, \tau)$ is $\ws$-$\TST_{1}$ space then $(\TSP, \tau)$ is $\ws$-$\TST_{0}$ space.
\end{thm}

\begin{proof}
Take $(\TSP, \tau)$ be a $\ws$-$\TST_{1}$ space. For each pair of disjoint points $\TSa$, $\TSb$ of $(\TSP, \tau) \ \exists$ open sets $\TSH_{1}$, $\TSH_{2}$ in $(\TSP, \tau) \ni: \TSa \in \TSH_{1}$, $\TSb \not\in \TSH_{1}$ and $\TSb \in \TSH_{2}$, $\TSa \not\in \TSH_{2}$. Hence we have $\TSa \in \TSH_{1}$, $\TSb \not\in \TSH_{1}$. Henceforth $(\TSP, \tau)$ is a $\ws$-$\TST_{0}$ space.
\end{proof}

\begin{thm}\label{thm6.2.7}
A topological space $\TSP$ is $\ws$-$\TST_{1}$ space iff $\forall \TSx\in \TSP$ singleton $\{1\}$ is $\ws$-closed set in $\TSP$.
\end{thm}

\begin{proof}
Take $\TSP$ be $\ws$-$\TST_{1}$ space and Assume $\TSa \in \TSP$, to show that $\{1\}$ is $\ws$-closed set. We will show $\TSP-\{1\}$ is $\ws$-open set in $\TSP$. take $\TSb \in \TSP-\{1\}$, implies $\TSa \neq \TSb \in \TSP$ and seeing that $\TSP$ is $\ws$-$\TST_{1}$ space thus $\exists$ two $\ws$-open sets $\TSH_{1}$, $\TSH_{2} \ni: \TSa \not\in \TSH_{1}$, $\TSb \in \TSH_{2} \subseteq \TSP-\{1\}$.

Seeing that $\TSb\in \TSH_{2} \subseteq \TSP-\{1\}$ then $\TSP-\{1\}$ is $\ws$-open set. Hence $\{1\}$ is $\ws$-closed set. Inversely, let $\TSa \neq \TSb \in \TSP$ then $\{1\}$, $\{2\}$ are $\ws$-closed sets. That is $\TSP-\{1\}$ is $\ws$-open set. Clearly, $\TSa \not\in \TSP-\{1\}$ and $\TSb \in \TSP-\{2\}$. Similarly $\TSP-\{2\}$ is $\ws$-open set, $\TSb \not\in \TSP-\{2\}$ and $\TSa \in \TSP-\{2\}$. Hence $\TSP$ is $\ws$-$\TST_{1}$ space.
\end{proof}

\begin{dfn}\label{defi6.2.3}
A topological space $\TSP$ is called a $\ws$-$\TST_{2}$ if for each pair of distinct points $\TSa$, $\TSb$ of $\TSP$, $\exists$ a $\ws$-open sets $\TSH_{1}$, $\TSH_{2}$ in $\TSP \ni: \TSa \in \TSH_{1}$, $\TSb \in \TSH_{2}$ and $\TSH_{1} \cap \TSH_{2} = \phi$.
\end{dfn}

\begin{exm}\label{exam6.2.5}
Let $\TSP= \{1,2,3\}$ and $\tau =\{\TSP, \phi, \{1\},\{2\}, \{1,2\}\}$. Then $(\TSP, \tau)$ is $\ws$-$\TST_{2}$ space. Seeing that for a pair of distinct points `$1$' and `$2$' of $(\TSP, \tau)$ there exist an $\ws$-open set $\{1\}, \{2\} \ni: 1 \in \{1\}, 2 \in \{2\}$ and $\{1\} \cap \{2\} = \phi$.
\end{exm}

\begin{thm}\label{thm6.2.8}
\begin{enumerate}[(i)]
\itemsep=10pt
\item Each $\TST_{2}$-space is $\ws$-$\TST_{2}$ space
\item Each semi-$\TST_{2}$ space is $\ws$-$\TST_{2}$ space
\item Each $\alpha$-$\TST_{2}$ space is $\ws$-$\TST_{2}$ space
\item Each $\TSg\#$-$\TST_{2}$ space is $\ws$-$\TST_{2}$ space
\item Each $\ddot{g}$-$\TST_{2}$ space is $\ws$-$\TST_{2}$ space
\item Each $\TSg\xi$*-$\TST_{2}$ space is $\ws$-$\TST_{2}$ space
\item Each $\TSg\#\TSs$-$\TST_{2}$ space is $\ws$-$\TST_{2}$ space
\item Each $\rb$-$\TST_{2}$ space is $\ws$-$\TST_{2}$ space
\item Each regular-$\TST_{2}$ space is $\ws$-$\TST_{2}$ space
\end{enumerate}
\end{thm}

\begin{proof}
\begin{itemize}
\item[(i)] Let $\TSP$ is a $\TST_{2}$ space. $\forall$ pair of disjoint points $\TSa, \TSb$ of $\TSP \ \exists$ open set $\TSH_{1}$, $\TSH_{2}$ in $\TSP \ni: \TSa \in \TSH_{1}$, $\TSb \in \TSH_{2}$ and $\TSH_{1} \cap \TSH_{2} = \phi$. But every open set is $\ws$-open then $\exists$  $\ws$-open set $\TSH_{1}$, $\TSH_{2}$ in $\TSP\ni: \TSa \in \TSH_{1}$, $\TSb \in \TSH_{2}$ and $\TSH_{1} \cap \TSH_{2} = \phi$. Henceforth $\TSP$ is a $\ws$-$\TST_{2}$ space.

Similarly the remaining (ii) to (ix) results can be proved.
\end{itemize}
\end{proof}

\begin{thm}\label{thm6.2.9}
If $(\TSP, \tau)$ is $\ws$-$\TST_{2}$ space then $(\TSP, \tau)$ is $\ws$-$\TST_{1}$ space.
\end{thm}

\begin{proof}
Take up $(\TSP, \tau)$ be a $\ws$-$\TST_{2}$ space. $\forall$ pair of disjoint points $\TSa$, $\TSb$ of $(\TSP, \tau) \exists$ disjoint open sets $\TSK$ and $\TSL$ in $(\TSP, \tau) \ni: \TSa \in \TSK$ and $\TSb \in \TSL$. Hence we have $\TSa \in \TSK$, $\TSb \not\in \TSK$ and $\TSa \in \TSL$, $\TSb \not\in \TSL$. Henceforth $(\TSP, \tau)$ is a $\ws$-$\TST_{1}$ space.
\end{proof}

\begin{thm}\label{thm6.2.10}
The given Statements are one and the same, for space $(\TSP, \tau)$,
\begin{itemize}
\item[(i)] $(\TSP, \tau)$ is $\ws$-$\TST_{2}$ space.
\item[(ii)] If $\TSa \in \TSP$, then $\forall \TSa\neq b$, there will be $\ws$-open set $\TSK$ containing $\TSa \ni: \TSb \not\in \ws$-$\cl(\TSK)$.
\end{itemize}
\end{thm}

\begin{proof}
(i) $\Rightarrow$ (ii) Let $\TSa \in \TSP$. If $\TSb \in \TSP$ is $\ni: \TSb \neq \TSa$, $\exists$ disjoint $\ws$-open sets $\TSK$ and $\TSL \ni: \TSa\in \TSK$ and $\TSb\in \TSL$.

Then $\TSa \in \TSK \subset \TSP-\TSL$ which implies $\TSP-\TSL$ is $\ws$-open and $\TSb \not\in \TSP-\TSL$. Henceforth $\TSb \not\in \ws$-$\cl(\TSK)$.

(ii) $\Rightarrow$ (i) Take up $\TSa$, $\TSb \in \TSP$ and $\TSa \neq\TSb$. By (ii), $\exists$ a $\ws$-open $\TSK$ containing $\TSa \ni: \TSb \not\in \ws$-$\cl(\TSK)$. Henceforth $\TSb\in \TSP-(\ws\text{-}\cl(\TSK))$. $\TSP-(\ws\text{-}\cl(\TSK))$ is $\ws$-open and $\TSa \in \TSP- (\ws\text{-}\cl(\TSK))$. Also $\TSK\cap\TSP-(\ws\text{-}\cl(\TSK)) = \Phi$. Hence $(\TSP, \tau)$ is $\ws$-$\TST_{2}$ space.
\end{proof}

\begin{thm}\label{thm6.2.11}
Pretend $\TSP$ be a topological space and $\TSQ$ is $\ws$-$\TST_{2}$ space. If $\TSh: \TSP \to \TSQ$ is injective and $\ws$-irresolute then $\TSP$ is $\ws$-$\TST_{2}$ space.
\end{thm}

\begin{proof}
Pretend $\TSa$, $\TSb \in \TSP \ni: \TSa \neq \TSb$. Seeing that $\TSh$ is injective, implies $\TSh(\TSa) \neq \TSh(\TSb)$. Also $\TSQ$ is $\ws$-$\TST_{2}$ space thus there are two $\ws$-open sets $\TSM$ and $\TSN$ in $\TSQ \ni: \TSh(\TSa) \in \TSM$, $\TSh(\TSb) \in \TSN$ and $\TSM\cap \TSN=\Phi$. Seeing that $\TSh$ is $\ws$-irresolute thus $\TSh^{-1}(\TSM)$, $\TSh^{-1} (\TSN)$ are two $\ws$-open sets in $\TSP$, $\TSa \in \TSh^{-1} (\TSM)$, $\TSb \in \TSh^{-1} (\TSN)$, $\TSh^{-1} (\TSM) \cap \TSh^{-1} (\TSN) = \Phi$. Hence $\TSP$ is $\ws$-$\TST_{2}$ space.
\end{proof}

\begin{thm}\label{thm6.2.12}
Pretend $\TSP$ is a topological space and $\TSQ$ is a $\TST_{2}$ space. If $\TSh: \TSP \to \TSQ$ is injective and $\ws$-continuous then $\TSP$ is $\ws$-$\TST_{2}$ space.
\end{thm}

\begin{proof}
Pretend $\TSa$, $\TSb \in \TSP \ni: \TSa \neq \TSb$. Seeing that $\TSh$ is injective, implies $\TSf(\TSa) \neq \TSf(\TSb)$. Also $\TSQ$ is an $\TST_{2}$ space, thus there are two open set $\TSM$ and $\TSN$ in $\TSQ$, $\ni: \TSh(\TSa) \in \TSM$, $\TSh(\TSb) \in \TSN$ and $\TSM\cap\TSN =\Phi$. Seeing that $\TSh$ is $\ws$-continuous thus $\TSh^{-1} (\TSM)$, $\TSh^{-1} (\TSN)$ are two $\ws$-open sets in $\TSP$.

Then $\TSa \in \TSh^{-1} (\TSM)$, $\TSb \in \TSh^{-1} (\TSM)$, $\TSh^{-1} (\TSM) \cap \TSh^{-1} (\TSN) =\Phi$. Hence $\TSP$ is $\ws$-$\TST_{2}$ space.
\end{proof}

\section[$\ws$-Regular Space]{\boldmath$\ws$-Regular Space}\label{sec6.3}

In this section, we determine a new kind of spaces called $\ws$-regular spaces using $\ws$-closed sets and verified some of their related characterizations.

\begin{dfn}\label{defi6.3.1}
A topological space $\TSP$ is termed as $\ws$-regular if for each $\ws$-closed set $\TSF$ and a point $\TSa \not\in \TSF$, $\exists$ disjoint open sets $\TSG$ and $\TSH \ni: \TSF \subseteq \TSG$ and $\TSP \in \TSH$. We have the following interrelationship between $\ws$-regularity and regularity.
\end{dfn}

\begin{thm}\label{thm6.3.1}
Each $\ws$-regular space is regular.
\end{thm}

\begin{proof}
Take up $\TSP$ be $\ws$-regular space. Assume $\TSF$ be any closed set in $\TSP$ and a point $\TSa \in \TSP \ni: \TSa \not\in \TSF$. and $\TSF$ is $\ws$-closed and $\TSa \not\in \TSF$. Seeing that $\TSP$ is $\ws$-regular space, $\exists$ a pair of disjoint open sets $\TSG$ and $\TSH \ni: \TSF \subseteq \TSG$ and $\TSa \in \TSH$. Hence $\TSP$ is a regular space.
\end{proof}

\begin{thm}\label{thm6.3.2}
Pretend $\TSP$ is a regular space and $\TST_{\ws}$-space, then $\TSP$ is $\ws$-regular.
\end{thm}

\begin{proof}
Take up $\TSP$ be a regular space and $\TST_{\ws}$-space. Assume $\TSF$ be any $\ws$-closed set in $\TSP$ and a point $\TSa \in \TSP \ni: \TSa\not\in \TSF$. Seeing that $\TSP$ is $\TST_{\ws}$-space. $\TSF$ is closed and $\TSa \not\in \TSF$. Seeing that $\TSP$ is a regular space, $\exists$ a pair of disjoint open sets $\TSG$ and $\TSH \ni: \TSF \subseteq \TSG$ and $\TSP \in \TSH$. Hence $\TSP$ is $\ws$-regular space.
\end{proof}

\begin{thm}\label{thm6.3.3}
Each subspace of a $\ws$-regular space is $\ws$-regular.
\end{thm}

\begin{proof}
Take $\TSP$ be $\ws$-regular space. Assume $\TSQ$ be a subspace of $\TSP$. Let $\TSa \in \TSQ$ and $\TSF$ be a $\ws$-closed set in $\TSQ \ni: \TSa \in \TSF$. Thus there is a closed set and so $\ws$-closed set $\clrD$ of $\TSP$ with $\TSF = \TSQ \cap \TSD$ and $\TSP \in \TSD$. Therefore we have $\TSa \in \TSP$, $\clrD$ is $\ws$-closed in $\TSP \ni: \TSa \in \TSD$. Seeing that $\TSP$ is $\ws$-regular, $\exists$  open sets $\TSG$ and $\TSH \ni: \TSa \in \TSG$, $\TSD\subseteq \TSH$ and $\TSG \cap \TSH = \Phi$. Note that $\TSQ \cap \TSG$ and $\TSQ \cap \TSH$ are open sets in $\TSQ$. Also $\TSa \in \TSG$ and $\TSa \in \TSQ$, which implies $\TSP \in \TSQ \cap \TSG$ and $\clrD \subseteq \TSH$ implies $\TSQ \cap \clrD \subseteq \TSQ \cap \TSH$, $\TSF \subseteq \TSQ \cap \TSH$. Also $(\TSQ \cap \TSG) \cap (\TSQ \cap \TSH) = \Phi$. Hence $\TSQ$ is $\ws$-regular space. We have yet another characterization of $\ws$-regularity in the following.
\end{proof}

\begin{thm}\label{thm6.3.4}
Let $\TSP$ is a space. Pretend $\TSP$ is a $\ws$-regular and a $\TST_{1}$ space then $\TSP$ is $\ws$-$\TST_{2}$ space.
\end{thm}

\begin{proof}
Pretend $\TSa$, $\TSb \in \TSP \ni: \TSa = \TSb$. Seeing that $\TSP$ is $\TST_{1}$-space thus $\exists$ an open set $\TSM \ni:\TSa \in \TSM$, $\TSb \in \TSM$. Seeing that $\TSP$ is $\ws$-regular space and $\TSM$ is an open set which contains $\TSa$, $\exists$  an $\ws$-open set $\TSN \ni: \TSa\in N\subset \ws$-$\cl(\TSN) \subseteq \TSM$. Seeing that $\TSb \in \TSM$, hence $\TSb \in \ws$-$\cl(\TSN)$. Therefore $\TSb \in \TSP$-($\ws$-$\cl(\TSN)$). Hence there are $\ws$-open sets $\TSN$ and $\TSP$-($\ws$-$\cl(\TSN)) \ni:(\TSP\text{-}(\ws-\cl(N)))\cap \TSN = \Phi$. Hence $\TSP$ is $\ws$-$\TST_{2}$ space.
\end{proof}

\begin{thm}\label{thm6.3.5}
Let $\TSh: \TSP \to \TSQ$ be a bijective, $\ws$-closed map from a topological space $\TSP$ into a $\ws$-regular space $\TSQ$. If $\TSP$ is $\TST\ws$-space, then $\TSP$ is $\ws$-regular.
\end{thm}

\begin{proof}
Take up $\TSa \in \TSP$ and $\TSF$ be $\ws$-closed set in $\TSP$ with $\TSa \in \TSF$. Seeing that $\TSP$ is $\TST\ws$-space, $\TSF$ is closed in $\TSP$. Then $\TSh(\TSF)$ is $\ws$-closed set with $\TSh(P) \in \TSh(\TSF)$ in $\TSQ$, Seeing that $\TSh$ is $\ws$-closed. As $\TSQ$ is $\ws$-regular, $\exists$ disjoint open sets $\TSM$ and $\TSN \ni: \TSh(\TSa) \in \TSM$ and $\TSh(\TSF) \subseteq \TSN$. Henceforth $\TSa \in \TSh^{-1}(\TSM)$ and $\TSF \subseteq \TSh^{-1} (\TSN)$. Hence $\TSP$ is $\ws$-regular space.
\end{proof}

\begin{thm}\label{thm6.3.6}
The given below statements are one and the same. for space $\TSP$.
\begin{enumerate}[(i)]
\item $\TSP$ is $\ws$-regular space.
\item For each $\TSa \in \TSP$ and each $\ws$-open neighbourhood $\TSM$ of $\TSP$ $\exists$ an open neighbourhood $\TSN$ of $\TSP\ni: \cl (\TSN) \subseteq \TSM$.
\end{enumerate}
\end{thm}

\begin{proof}
(i) $\Rightarrow$ (ii): Pretend $\TSP$ is a $\ws$-regular space. Let $\TSM$ be any $\ws$-neighbourhood of $\TSP$. Then $\exists$ $\ws$-open set $\TSH \ni: \TSa \in \TSH \subseteq \TSM$. Now $\TSP-\TSH$ is $\ws$-closed set and $\TSa \in\TSP - \TSH$. Seeing that $\TSP$ is $\ws$-regular $\exists$ open sets $\TSM$ and $\TSN \ni: \TSP - \TSH \subseteq \TSM$, $\TSa\in \TSN$ and $\TSM\cap \TSN = \phi$ and so $\TSN \subseteq \TSP -\TSM$. Now $\cl(\TSN) \subseteq \cl(\TSP - \TSM) = \TSP -\TSM$ and $\TSP - \TSH \subseteq \TSM$. This implies $\TSP - \TSM \subseteq \TSH \subseteq \TSM$. Therefore $\cl(\TSN) \subseteq \TSM$.

(ii) $\Rightarrow$ (i): Pretend $\TSF$ be any $\ws$-closed set in $\TSP$ and $\TSa \not\in \TSF$ or $\TSa \in \TSP -\TSF$ and $\TSP -\TSF$ is a $\ws$-open and so $\TSP -\TSF$ is a $\ws$-neighbourhood of $\TSP$. By hypothesis, $\exists$ an open neighbourhood $\TSN$ of $\TSa \ni: \TSa \in \TSN$ and $\cl(\TSN) \subseteq \TSP -\TSF$. This implies $\TSF \subseteq \TSP -\cl(\TSN)$ is an open set containing $\TSF$ and $\TSN \cap \{(\TSP - \cl(\TSN)\} = \phi$. Hence $\TSP$ is $\ws$-regular space. We have another characterization of $\ws$-regularity in the following.
\end{proof}

\begin{thm}\label{thm6.3.7}
$\TSP$ is topological space is $\ws$-regular iff for each $\ws$-closed set $\TSF$ of $\TSP$ and each $\TSa \in \TSP -\TSF$ $\exists$ open sets $\TSG$ and $\TSH$ of $\TSP \ni: \TSa \in \TSG$, $\TSF \subseteq \TSH$ and $\cl(\TSG) \cap \cl(\TSH) = \phi$.
\end{thm}

\begin{proof}
Pretend $\TSP$ is $\ws$-regular space. Assume $\TSF$ be a $\ws$-closed set in $\TSP$ with $\TSa \in \TSF$. Then $\exists$ open sets $\TSM$ and $\TSH$ of $\TSP \ni: \TSa \in \TSM$, $\TSF \subseteq \TSH$ and $\TSM \cap\TSH = \phi$. This implies $\TSM \cap \cl(\TSH) = \phi$. As $\TSP$ is $\ws$-regular, $\exists$ open sets $\TSM$ and $\TSN \ni: \TSa \in \TSM$, $\cl(\TSH) \subseteq \TSN$ and $\TSM \cap \TSN = \Phi$, so $\cl(\TSM) \cap \TSN = \Phi$. Assume $\TSG = \TSM \cap \TSM$, then $\TSG$ and $\TSH$ are open sets of $\TSP \ni: \TSa \in \TSG$, $\TSF \subseteq \TSH$ and $\cl(\TSH) \cap \cl(\TSH) = \phi$. Inversely, if for each $\ws$-closed set $\TSF$ of $\TSP$ and each $\TSa \in \TSP- \TSF \ni:$ open sets $\TSG$ and $\TSH \ni: \TSa \in \TSG$, $\TSF \subseteq \TSH$ and $\cl(\TSH) \cap \cl(\TSH) = \Phi$. This implies $\TSa \in \TSG$, $\TSF \subseteq \TSH$ and $\TSG \cap \TSH = \phi$. Hence $\TSP$ is $\ws$-regular. Now we prove that $\ws$-regularity is a hereditary property.
\end{proof}

\begin{thm}\label{thm6.3.8}
$\TSP$ is a topological space then results given below are one and the same.
\begin{enumerate}[(i)]
\item $\TSP$ is $\ws$-regular
\item For each $\TSa \in \TSP$ and each $\ws$-open set $\TSM$ in $\TSP \ni: \TSP \in \TSM$ $\exists$ an open set $\TSV$ in $\TSP \ni: \TSP \in \TSN \subseteq \cl(\TSN) \subseteq \TSM$
\item For each point $\TSa\in \TSP$ and for each $\ws$-closed set $\TSA$ with $\TSP \in \TSD$, $\exists$ an open set $\TSN$ containing $\TSP \ni: \cl(\TSN) \cap \clrD = \Phi$.
\end{enumerate}
\end{thm}

\begin{proof}
(i) $\Rightarrow$ (ii): Follows from Theorem \ref{thm6.3.4}.

(ii) $\Rightarrow$ (iii): Suppose (ii) holds. Assume $\TSa \in \TSP$ and $\clrD$ be a $\ws$-closed set of $\TSP \ni: \TSa \in \clrD$. Then $\TSP-\TSD$ is a $\ws$-open set with $\TSa \in \TSP-\clrD$. By hypothesis, $\exists$ an open set $\TSN \ni: \TSP \in \TSN \subseteq \cl(\TSN) \subseteq \TSP -\clrD$.

That is $\TSP \in \TSN$, $\TSN \subseteq \cl (\clrD)$ and $\cl(\clrD) \subseteq \TSP -\clrD$. So $\TSP \in \TSN$ and $\cl(\TSN) \cap \clrD = \Phi$.

(iii)$\Rightarrow$ (ii): Let $\TSa \in \TSP$ and $\TSM$ is a $\ws$-open set in $\TSP \ni: \TSa \in \TSM$. Then $\TSP -\TSM$ is a $\ws$-closed set and $\TSa \in \TSP -\TSM$. Then by hypothesis, $\exists$ an open set $\TSN$ containing $\TSP \ni: \cl (\TSN) \cap (\TSP -\TSM) = \Phi$. Therefore $\TSa \in \TSN$, $\cl (\TSN) \subseteq \TSM$ so $\TSP \in \TSN \subseteq \cl(\TSN) \subseteq \TSM$. The invariance of $\ws$-regularity is given in the following.
\end{proof}

\begin{thm}\label{thm6.3.9}
Pretend $\TSh: \TSP \to \TSQ$ be a bijective, $\ws$-irresolute and open map from a $\ws$-regular space $\TSP$ ino a topological space $\TSQ$, then $\TSQ$ is $\ws$-regular.
\end{thm}

\begin{proof}
Pretend $\TSb \in \TSQ$ and $\TSF$ be a $\ws$-closed set in $\TSQ$ with $\TSb \in \TSF$. Seeing that $\TSh$ is $\ws$-irresolute, $\TSh^{-1} (\TSF)$ is $\ws$-closed set in $\TSP$. Let $\TSh(\TSa) = \TSb$ so that $\TSa = \TSh^{-1} (\TSb)$ and $\TSa \in \TSh^{-1} (\TSF)$. Again $\TSP$ is $\ws$-regular space, $\exists$ open sets $\TSM$ and $\TSN \ni: \TSa \in \TSM$ and $\TSh^{-1} (\TSF) \subseteq \TSG$, $\TSM \cap\TSN = \Phi$. Seeing that $\TSh$ is open and bijective, we have $\TSb \in \TSh(\TSM)$, $\TSF \subseteq \TSh(\TSN)$ and $\TSh(\TSM) \cap \TSh(\TSN)= \TSh(\TSM \cap \TSN) = \TSh(\Phi) = \Phi$. Hence $\TSQ$ is $\ws$-regular space.
\end{proof}

\section[$\ws$-Normal Spaces]{\boldmath$\ws$-Normal Spaces}\label{sec6.4}

In this section, we introduce the concept of $\ws$-normal spaces and study some of their characterizations.

\begin{dfn}\label{defi6.4.1}
A topological space $\TSP$ is said to be $\ws$-normal if $\forall$ pair of disjoint $\ws$-closed sets $\clrD$ and $\TSE$ in $\TSP$, $\exists$ a pair of disjoint open sets $\TSM$ and $\TSN$ in $\TSP \ni: \clrD \subseteq \TSM$ and $\TSE \subseteq \TSN$. We have the following interrelationship.
\end{dfn}

\begin{thm}\label{thm6.4.1}
Each $\ws$-normal space is normal.
\end{thm}

\begin{proof}
Take $\TSP$ be a $\ws$-normal space. Assume $\clrD$ and $\TSE$ be a pair of disjoint closed sets in $\TSP$. And $\clrD$ and $\TSE$ are $\ws$-closed sets in $\TSP$. Seeing that $\TSP$ is $\ws$-normal, $\exists$ a pair of disjoint open sets $\TSG$ and $\TSH$ in $\TSP \ni: \clrD \subseteq \TSG$ and $\TSE \subseteq \TSH$. Hence $\TSP$ is normal. 

We use the following example to prove the inverse of the theorem is untrue.
\end{proof}

\begin{exm}\label{exam6.4.1}
Let Let $\TSP=\{1,2,3,4\}$, $\tau= \{\TSP, \{1\},\{2\},\{1,2\},\{1,2,3\}\}$ Then the space $\TSP$ is normal but not $\ws$-normal, seeing that the pair of disjoint $\ws$-closed sets namely, $\clrD =\{1\}$ for which there do not exists disjoint open sets $\TSG$ and $\TSH \ni: \clrD \subseteq \TSG$ and $\TSE \subseteq \TSH$.
\end{exm}

\begin{thm}\label{thm6.4.2}
A $\ws$-closed subspace of a $\ws$-normal space is $\ws$-normal.
\end{thm}

\begin{proof}
Take up $\TSP$ be $\ws$-normal space. Assume $\TSQ$ be a $\ws$-closed subspace of $\TSP$. Pretend $\clrD$ and $\TSE$ be pair of disjoint $\ws$-closed sets in $\TSQ$. Thus $\clrD$ and $\TSE$ be pair of disjoint $\ws$-closed sets in $\TSP$. seeing that $\TSP$ is $\ws$-normal, $\exists$ disjoint open sets $\TSG$ and $\TSH$ in $\TSP \ni: \clrD \subseteq \TSG$ and $\TSE \subseteq \TSH$. 

Seeing that $\TSG$ and $\TSH$ are open in $\TSP$, $\TSQ \cap \TSG$ and $\TSQ \cap \TSH$ are open in $\TSQ$. Also we have $\clrD \subseteq \TSG$ and $\TSE \subseteq \TSH$ implies $\TSQ \cap \clrD \subseteq \TSQ \cap \TSG$, $\TSQ \cap \TSE \subseteq \TSQ \cap \TSH$. So $\clrD \subseteq \TSQ \cap \TSG$ and $\TSE \subseteq \TSQ \cap \TSH$ and $(\TSQ \cap \TSG)\cap (\TSQ \cap \TSH) = \TSQ \cap (\TSG\cap \TSH) = \phi$. Hence $\TSQ$ is $\ws$-normal.
\end{proof}

\begin{thm}\label{thm6.4.3}
If $\TSP$ is normal and $\TST\ws$-space, then $\TSP$ is $\ws$-normal.
\end{thm}

\begin{proof}
Take up $\TSP$ be a normal space. Assume $\clrD$ and $\TSE$ be a pair of disjoint $\ws$-closed sets in $\TSP$. Seeing that $\TST\ws$-space, $\clrD$ and $\TSE$ are closed sets in $\TSP$. Also $\TSP$ normal, $\exists$ a pair of disjoint open sets $\TSG$ and $\TSH$ in $\TSP \ni: \clrD \subseteq \TSG$ and $\TSE \subseteq \TSH$. Hence $\TSP$ is $\ws$-normal.
\end{proof}

\begin{thm}\label{thm6.4.4}
Each $\ws$-normal space is $\TSw$-normal.
\end{thm}

\begin{proof}
Take up $\TSP$ be a $\ws$-normal space. Assume $\clrD$ and $\TSE$ be a pair of disjoint $\TSw$-closed sets in $\TSP$. $\clrD$ and $\TSE$ are $\ws$-closed sets in $\TSP$. seeing that $\TSP$ is $\ws$-normal, $\exists$ a pair of disjoint open sets $\TSG$ and $\TSH$ in $\TSP \ni: \clrD \subseteq \TSG$ and $\TSE \subseteq \TSH$. Hence $\TSP$ is $\TSw$-normal.
\end{proof}

\begin{thm}\label{thm6.4.5} 
Map $\TSh: \TSP \to \TSQ$ is bijective, open, $\ws$-irresolute from a $\ws$-normal space $\TSP$ onto $\TSQ$ then is $\ws$-normal.
\end{thm}

\begin{proof}
Take up $\clrD$ and $\TSE$ be disjoint $\ws$-closed sets in $\TSQ$. Thus $\TSh^{-1} (\clrD)$ and $\TSh^{-1} (\TSE)$ are disjoint $\ws$-closed sets in $\TSP$ as $\TSh$ is $\ws$-irresolute. Seeing that $\TSP$ is $\ws$-normal, $\exists$ disjoint open sets $\TSG$ and $\TSH$ in $\TSP \ni: \TSh^{-1} (\clrD) \subseteq \TSG$ and $\TSh^{-1} (\TSE) \subseteq \TSH$. As $\TSh$ is bijective and open, $\TSh(\TSG)$ and $\TSh(\TSH)$ are disjoint open sets in $\TSQ \ni: \clrD \subseteq \TSh(\TSG)$ and $\TSE \subseteq \TSh(\TSH)$. Hence $\TSQ$ is $\ws$-normal.
\end{proof}

\begin{thm}\label{thm6.4.6}
The following statements for a space $\TSP$ are one and the same:
\begin{enumerate}[(1)]
\item $\TSP$ is $\ws$-normal.
\item $\forall$ $\ws$-closed set $\clrD$ and each $\ws$-open set $\TSK \ni: \clrD \subseteq \TSM$, $\exists$ an open set $\TSN \ni: \clrD \subseteq \TSL \subseteq \cl(\TSL) \subseteq \TSM$.
\item For any disjoint $\ws$-closed sets $\clrD$, $\TSE$, $\exists$ an open set $\TSL \ni: \clrD \subseteq \TSL$ and $\cl(\TSL) \cap \TSE = \Phi$.
\item For each pair $\clrD$, $\TSE$ of disjoint $\ws$-closed sets there exist open sets $\TSM$ and $\TSL \ni: \TSD \subseteq \TSM$, $\TSE \subseteq \TSL$ and $\cl(\TSM) \cap \cl(\TSL) = \Phi$.
\end{enumerate}
\end{thm}

\begin{proof}
(1) $\Rightarrow$ (2): Take $\clrD$ be a $\ws$-closed set and $\TSM$ be a $\ws$-open set $\ni: \clrD \subseteq \TSM$. 

Then $\clrD$ and $\TSP -\TSM$ are disjoint $\ws$-closed sets in $\TSP$. Seeing that $\TSP$ is $\ws$-normal, $\exists$ a pair of disjoint open sets $\TSL$ and $\TSW$ in $\TSP \ni: \clrD \subseteq \TSL$ and $\TSP -\TSM \subseteq \TSW$.

Now $\TSP -\TSW \subseteq \TSP - (\TSP -\TSM)$, so $\TSP -\TSW \subseteq \TSM$ also $\TSL \cap\TSW = \Phi$ implies $\TSL \subseteq \TSP -\TSW$, so $\cl(\TSL) \subseteq \cl(\TSP -\TSW)$ which implies $\cl(\TSL) \subseteq \TSP - \TSW$. Henceforth $\cl(\TSL) \subseteq \TSP -\TSW \subseteq \TSM$. So $\cl (\TSL) \subseteq \TSM$. Hence $\clrD \subseteq \TSL \subseteq \cl(\TSL) \subseteq \TSM$.

(2) $\Rightarrow$ (3): Let $\clrD$ and $\TSE$ be a pair of disjoint $\ws$-closed sets in $\TSP$.

Now $\clrD\cap\TSE = \Phi$, so $\TSD\subseteq \TSP -\TSE$, where $\clrD$ is $\ws$-closed and $\TSP - \TSE$ is $\ws$-open. Then by (ii) $\exists$ an open set $\TSL \ni: \clrD \subseteq \TSL \subseteq \cl(\TSL) \subseteq \TSP - \TSE$. Now $\cl (\TSL) \subseteq \TSP -\TSE$ implies $\cl(\TSL) \cap \TSE =\Phi$. Thus $\clrD \subseteq \TSL$ and $\cl(\TSL) \cap \TSE = \Phi$.

(3) $\Rightarrow$ (4): Let $\clrD$ and $\TSE$ be a pair of disjoint $\ws$-closed sets in $\TSP$. Then from (iii) $\exists$ an open set $\TSM \ni: \clrD \subseteq \TSM$ and $\cl (\TSM) \cap\TSE = \Phi$. Seeing that $\cl (\TSL)$ is closed, so $\ws$-closed set. Therefore $\cl(\TSL)$ and $\TSE$ are disjoint $\ws$-closed sets in $\TSP$. By hypothesis, $\exists$ an open set $\TSL$, $\ni: \TSE \subseteq \TSL$ and $\cl (\TSM)\cap \cl (\TSL) = \Phi$.

(4) $\Rightarrow$ (1): Let $\clrD$ and $\TSE$ be a pair of disjoint $\ws$-closed sets in $\TSP$. Then from (iv) $\exists$ an open sets $\TSM$ and $\TSL$ in $\TSP \ni: \clrD \subseteq \TSM$, $\TSE \subseteq \TSL$ and $\cl (\TSM) \cap\cl (\TSL) = \Phi$. So $\clrD \subseteq \TSE \subseteq \TSL$ and $\TSM \cap \TSL = \Phi$. Hence $\TSP$ $\ws$-normal.
\end{proof}

\begin{thm}\label{thm6.4.7}
Let $\TSP$ be a topological space. Then the following are one and the same:
\begin{enumerate}[(1)]
\item $\TSP$ is normal.
\item For any disjoint closed sets $\clrD$ and $\TSE$, $\exists$ disjoint $\ws$-open sets $\TSM$ and $\TSL \ni: \clrD \subseteq \TSM$, $\TSE \subseteq \TSL$.
\item For any closed set $\clrD$ and any open set $\TSL \ni: \clrD \subseteq \TSL$, $\exists$ an $\ws$-open set $\TSM$ of $\TSP \ni: \clrD \subseteq \TSM \subseteq \alpha$-$\cl(\TSM) \subseteq \TSL$.
\end{enumerate}
\end{thm}

\begin{proof}
(1) $\Rightarrow$ (2): Suppose $\TSP$ is normal. Seeing that each open set is $\ws$-open [ ] (ii) follows.

(2) $\Rightarrow$ (3): Suppose (ii) holds. Let $\clrD$ be a closed set and $\TSL$ be an open set containing $\clrD$. Then $\clrD$ and $\TSP -\TSL$ are disjoint closed sets. By (ii), $\exists$  disjoint $\ws$-open sets $\TSM$ and $\TSW \ni: \clrD\subseteq \TSM$ and $\TSP-\TSL \subseteq \TSW$, seeing that $\TSP -\TSL$ is closed, so $\ws$-closed. We have $\TSP -\TSL \subseteq \alpha\text{ int} (\TSW)$ and $\TSM \cap \alpha\text{ int} (\TSW) = \Phi$ and so we have $\Cl (\TSM) \cap \alpha\text{ int} (\TSW) = \Phi$. Hence $\clrD \subseteq \TSM \subseteq \alpha-\cl(\TSM) \subseteq \TSP -\alpha\text{ int}(\TSW) \subseteq \TSL$. Thus $\clrD\subseteq\TSM \subseteq \alpha-\cl (\TSM) \subseteq \TSL$.

(3) $\Rightarrow$ (1): Let $\clrD$ and $\TSE$ be a pair of disjoint closed sets of $\TSP$. Then $\TSA \subseteq \TSP-\TSE$ and $\TSP-\TSE$ is open. $\exists$ a $\ws$-open set $\TSG$ of $\TSP \ni: \clrD \subseteq \TSG \subseteq \alpha-\cl (\TSG) \subseteq \TSP-\TSE$. Seeing that $\clrD$ is closed, it is $\ws$-closed, we have $\clrD \subseteq \text{ int} (\TSG)$. Take $\TSM =\text{int} (\cl (\text{int} (\alpha\text{ int} (\TSG))))$ and $\TSL = \text{int} (\cl (\text{int}(\TSP-\alpha\cl (\TSG))))$. Then $\TSM$ and $\TSL$ are disjoint open sets of $\TSP \ni: \TSA \subseteq \TSM$ and $\TSE \subseteq \TSL$. Hence $\TSP$ is normal.
\end{proof}
