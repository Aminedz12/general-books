\chapter{Number System \& Simplification}

\section*{Introduction}
\begin{center}
{\bf Figure}
\end{center}

\begin{multicols}{2}

\subsection*{Remember}
\begin{itemize}
\item The ten symbols 0, 1, 2, 3, 4, 5, 6, 7, 8, 9 are called digits.
\item 1 is neither prime nor composite.
\item 1 is an odd integer.
\item 0 is neither positive nor negative.
\item 0 is an even integer.
\item 2 is prime \& even both.
\item All prime numbers (except 2) are odd.
\end{itemize}

\textbf{Natural Numbers:}

These are the numbers (1, 2, 3, etc.) that are used for counting. It is denoted by $N$.

There are infinite natural numbers and the smallest natural number is one (1).

\textbf{Even numbers:}

Natural numbers which are divisible by 2 are even numbers. It is denoted by $E$.

$E = 2, 4, 6, 8, \ldots$

Smallest even numbers is 2. There is no largest even number.

\textbf{Off numbers:}

Natural numbers which are not divisible by 2 are odd numbers. It is denoted by $O$.

$O = 1, 3, 5, 7, \ldots$

Smallest odd number is 1.

There is no largest odd number.

Based on divisibility, there could be two types of natural numbers: Prime and Composite.

\textbf{Prime Numbers:}

Natural numbers which have exactly two factors, i.e., 1 and the number itself called prime numbers.

The lowest prime number is 2.

2 is also the only even prime number.

\textbf{Composite Numbers:}

It is a natural number that has at least one divisor different from unity and itself.

Every composite number can be factorized into its prime factors.

\textit{For Example:} $24 = 2 \times 2 \times 2 \times 3$. Hence, 24 is a composite number.

The smallest composite number is 4.

\textbf{Twin-prime Numbers:}

Pairs of such prime numbers whose difference is 2.

\textbf{Example:} 3 and 5, 11 and 13, 17 and 19.

\textbf{How to check whether a given number is prime or not ?}

Steps: (i) Find approximate square root of the given number.

(ii) Divide the given number by every prime number less than the approximate square root.

(iii) If the given number is exactly divisible by at least one of the prime numbers, the number is a composite number otherwise a prime number.

\textbf{Example:} Is 401 a prime number?

\textbf{Sol.} Approximate square root of 401 is 20.

Prime numbers less than 20 are 2, 3, 5, 7, 11, 13, 17 and 19

401 is not divisible by 2, 3, 5, 7, 11, 13, 17 or 19.

$\therefore~$ 401 is a prime number.

(\textbf{Hint:} Next prime number after 19 and 23, which is greater than 20, so we need not check further.)

\textbf{Co-prime Numbers:} Co-prime numbers are those numbers which are prime to each other i.e., they don't have any common factor other than 1.

Since these numbers do not have any common factor, their HCF is 1 and their LCM is equal to product of the numbers.

\textbf{Note:} Co-prime numbers can be prime or composite numbers. Any two prime numbers are always co-prime numbers.

\textbf{Example 1:} 3 and 5 : Both numbers are prime numbers.

\textbf{Example 2:} 8 and 15 : Both numbers are composite numbers but they are prime to each other i.e., they don't have any common factor.

\subsection*{Face value and Place value:}

Face Value is absolute value of a digit in a number.

Place Value (or Local Value) is value of a digit in relation to its position in the number.

\textbf{Example:} Face value and Place value of 9 in 14921 is 9 and 900 respectively.

\subsection*{Whole Numbers:}

The natural numbers along with zero (0), from the system of whole numbers.

It is denoted by $W$.

There is no largest whole number and

The smallest whole number is 0.

\subsection*{Integers:}

The number system consisting of natural numbers, their negative and zero is called integers.

It is denoted by $Z$ or $I$.

The smallest and the largest integers cannot be determined.

\subsection*{The Number Line:}

The number line is a straight line between negative infinity on the left to positive infinity on the right.

\begin{center}
{\bf Figure}
\end{center}

\textbf{Rational numbers:} Any number that can be put in the form of $\dfrac{p}{q}$, where $p$ and $q$ are integers and $q \neq 0$, is called a rational number.
\begin{itemize}
\item It is denoted by $Q$.
\item Every integer is a rational number.
\item Zero (0) is also a rational number. The smallest and largest rational numbers cannot be determined. Every fraction (and decimal fraction) is a rational number.

$Q = \dfrac{p}{q}$ $\dfrac{(\rm Numerator)}{(\rm Denominator)}$
\end{itemize}

\subsection*{REMEMBER}

\begin{itemize}
\item[*] If $x$ and $y$ are two rational numbers, then $\dfrac{x + y}{2}$ is also a rational number and its value lies between the given two rational numbers $x$ and $y$.
\item[*] An infinite number of rational numbers can be determined between any two rational numbers.
\end{itemize}

\noindent
\rule{\columnwidth}{1pt}


\textbf{Example 1. Find three rational numbers between 3 and 5.}

\textbf{Sol.} 1st rational number $= \dfrac{3+5}{2} = \dfrac{8}{2} = 4$

  2nd rational number (i.e., between 3 and 4) $= \dfrac{3+4}{2} = \dfrac{7}{2}$

  3rd rational number (i.e., between 4 and 5)

  $= \dfrac{4+5}{2} = \dfrac{9}{2}$.

  \textbf{Irrational numbers:} The numbers which are not rational or which cannot be put in the form of $\dfrac{p}{q}$, where $p$ and $q$ are integers and $q \neq 0$, is called irrational number.

  It i denoted by $Q'$ or $Q^{c}$.

  $\sqrt{2}, \sqrt{3}, \sqrt{5}, 2 + \sqrt{3}, 3 -\sqrt{5}, 3\sqrt{3}$ are irrational numbers.




\end{multicols}
