\begin{multicols}{2}

\noindent
\rule{\columnwidth}{1pt}

\centerline{{\large\sf\bfseries EXERCISE}}

\noindent
\rule{\columnwidth}{1pt}

\noindent
{\sf\bfseries DIRECTIONS (1--5):} {\it You have to solve equation I and II. Give answer}

\noindent
\rule{\columnwidth}{1pt}

\begin{tabular}{l@{\qquad\quad}l}
(a) If $x > y$ & (b) If $x < y$\\ 
(c) If $x \leq y$ & (d) If $x \geq y$\\ 
\multicolumn{2}{l}{(e) If $x = y$ or cannot be established.}
\end{tabular}

\begin{enumerate}[leftmargin=*]
\item {\bf I.} $4x^2 - 19x + 12 - 0$  \qquad{\bf II.} $3y^2 + 8y + 4 = 0$

\item {\bf I.} $x^2 = 729$  \hspace{2.25cm} {\bf II.} $y - \sqrt{324} = \sqrt{81}$

\item {\bf I.} $x^2 + x - 56 = 0$  \qquad\quad~ {\bf II.} $y^2 - 17y - 72 = 0$

\item {\bf I.} $20x^2 - 17x + 3 = 0$  \qquad {\bf II.} $20y^2 - 9y + 1 = 0$

\item {\bf I.} $4x^2 - 13x - 12 = 0$  \qquad {\bf II.} $y^2 - 7y - 60 = 0$

\end{enumerate}


\noindent
\rule{\columnwidth}{1pt}

\noindent
{\sf\bfseries DIRECTIONS (Qs. 6-10):} {\it In the following questions two equations numbered I and II give. answer. If}

\noindent
\rule{\columnwidth}{1pt}

\hfill{\bf [IBPS PO PRE 2016]}


\begin{tabular}{l@{\qquad\quad}l}
(a) $x > y$ & (b) $x < y$\\ 
(c) $x \geq y$ & (d) $x \leq y$\\ 
\multicolumn{2}{l}{(e) $x = y$ or relation cannot be established}
\end{tabular}

\begin{enumerate}[leftmargin=*]
\setcounter{enumi}{5}
\item $x^2 + 30x + 221 = 0$

$y^2 - 53y + 196 = 0$

\item $2x^2 - 9x + 10 = 0$

$y^2 - 18y + 72 = 0$

\item $x(35 - x) = 124$

$y(2y + 3) = 90$

\item $1/(x - 3) + 1/(x + 5) = 1/3$

$(y + 2)(27 - y) = 210$

\item $\sqrt{36x} + \sqrt{64} = 0$

$\sqrt{81y} + (4)^{2} = 0$
\end{enumerate}

\noindent
\rule{\columnwidth}{1pt}

\noindent
{\sf\bfseries DIRECTIONS (QS. 11-15):} {\it In the following questions two equations numbered I and II are given. You have to solve both the equations and give answer}

\noindent
\rule{\columnwidth}{1pt}

\noindent
\hfill{\bf [SBI PO PRE 2016]}


\begin{enumerate}[leftmargin=*]
\setcounter{enumi}{10}
\item $(x+2)(x+1) = (x-2)(x-3)$

  $(y+3)(y+2) = (y-1)$

\begin{tabular}{l@{\qquad\quad}l}
(a) $x > y$ & (b) $x < y$\\ 
(c) $x \leq y$ & (d) $x \geq y$\\ 
\multicolumn{2}{l}{(e) $x =y0$ or relation cannot be established}
\end{tabular}

\item $12x^2 + 29x + 14 = 0$

$y^2 + 9y + 18 = 0$

\begin{tabular}{l@{\qquad\quad}l}
(a) $x > y$ & (b) $x < y$\\ 
(c) $x \leq y$ & (d) $x \geq y$\\ 
\multicolumn{2}{l}{(e) $x =y0$ or relation cannot be established}
\end{tabular}

\item $5x^2 - 18x + 9 = 0$

  $3y^2 + 5y - 2 = 0$


\begin{tabular}{l@{\qquad\quad}l}
(a) $x > y$ & (b) $x < y$\\ 
(c) $x \leq y$ & (d) $x \geq y$\\ 
\multicolumn{2}{l}{(e) $x =y0$ or relation cannot be established}
\end{tabular}
\item $17^2 + 144 \div 18 = x$

  $26^2 - 18x \times 21 = y$

\begin{tabular}{l@{\qquad\quad}l}
(a) $x > y$ & (b) $x < y$\\ 
(c) $x \leq y$ & (d) $x \geq y$\\ 
\multicolumn{2}{l}{(e) $x =y0$ or relation cannot be established}
\end{tabular}

\item $30x^2 + 11x + 1 = 0$

  $42y^2 + 13y + 1 = 0$

\begin{tabular}{l@{\qquad\quad}l}
(a) $x > y$ & (b) $x < y$\\ 
(c) $x \leq y$ & (d) $x \geq y$\\ 
\multicolumn{2}{l}{(e) $x =y0$ or relation cannot be established}
\end{tabular}
\end{enumerate}

\noindent
\rule{\columnwidth}{1pt}

\noindent
{\sf\bfseries DIRECTIONS (Qs. 16-20):} {\it In the given questions, two equations numbered I and /I are given. Solve both the equations and mark the appropriate answer.}

\noindent
\rule{\columnwidth}{1pt}

\begin{tabular}{l@{\qquad\quad}l@{\qquad\quad}l}
(a) $x > y$ & (b) $x \leq y$ & (c) $x < y$\\
\multicolumn{3}{p{7.5cm}}{(d) Relationship between x and y cannot be\newline \phantom{aai} determined}\\
(e) $x \geq y$ & & \\ 
\end{tabular}

\begin{enumerate}[leftmargin=*]
  \setcounter{enumi}{15}
\item {\bf I.} $6x^2 + 25x + 24 = 0$ \qquad {\bf II.} $12y^2 + 13y + 3 = 0$

\item {\bf I.} $12x^2 - x - 1 = 0$ \qquad~~~ {\bf II.} $20y^2 - 41y + 20 = 0$

\item {\bf I.} $10x^2 + 33x + 27 = 0$\quad~~  {\bf II.} $5y^2 + 19y + 18 = 0$

\item {\bf I.} $15x^2 - 29x - 14 = 0$\quad~~ {\bf II.} $6y^2 - 5y - 25 = 0$

\item {\bf I.} $3x^2 - 22x + 7 = 0$\quad~~ {\bf II.} $y^2 - 20y + 91 = 0$

\end{enumerate}

\noindent
\rule{\columnwidth}{1pt}

\noindent
{\sf\bfseries DIRECTIONS (Qs. 21--25):} {\it In each of these questions two equations are given. You have to solve these equations and Give answer.}

\noindent
\rule{\columnwidth}{1pt}

\begin{tabular}{l@{\qquad\quad}l@{\qquad\quad}l}
(a) if $x<y$ & (b) if $x>y$ & (c) if $x=y$\\
(d) if $x\geq y$ & (e) if $x\leq y$
\end{tabular}

\begin{enumerate}[leftmargin=*]
\setcounter{enumi}{20}
\item 
\begin{itemize}
\begin{multicols}{2}
\item[{\bf I.}] $x^{2}-6x=7$ 
\item[{\bf II.}] $2y^{2}+13y+15=0$
\end{multicols}
\end{itemize}

\item 
\begin{itemize}
\begin{multicols}{2}
\item[{\bf I.}] $3x^{2}-7x+20$
\item[{\bf II.}] $2y^{2}-11y+15=0$
\end{multicols}
\end{itemize}

\item 
\begin{itemize}
\begin{multicols}{2}
\item[{\bf I.}] $10x^{2}-7x+1=0$
\item[{\bf II.}] $35y^{2}-12y+1=0$
\end{multicols}
\end{itemize}

\item 
\begin{itemize}
\begin{multicols}{2}
\item[{\bf I.}] $4x^{2}=25$
\item[{\bf II.}] $2y^{2}-13y+21=0$
\end{multicols}
\end{itemize}

\item 
\begin{itemize}
\begin{multicols}{2}
\item[{\bf I.}] $3x^{2}+7x=6$
\item[{\bf II.}] $6(2y^{2}+1)=17y$
\end{multicols}
\end{itemize}
\end{enumerate}


\noindent
\rule{\columnwidth}{1pt}

\noindent
{\sf\bfseries DIRECTIONS (Qs. 26--30):} {\it In the following questions three equations numbered I, II and III are given. You have to solve all the equations either together or separately, or two together and one separately, or by any other method and---.}

\noindent
\rule{\columnwidth}{1pt}

\noindent
{\bf Give answer If}\hfill {\bf[SBI PO Main Exam, 2015]}

\begin{tabular}{l@{\qquad\quad}l}
(a) $x<y=z$ & (b) $x\leq y<z$\\
(c) $x<y>z$ & (d) $x=y>z$\\
\multicolumn{2}{p{7.5cm}}{(e) $x=y=z$ or if none of the above relationship\newline \phantom{aai} is established}
\end{tabular}


\begin{enumerate}[leftmargin=*]
\setcounter{enumi}{25}
\item 
\begin{itemize}
\item[{\bf I.}] $7x+6y+4z=122$\qquad {\bf II.} $4x+5y+3z=88$

\item[{\bf III.}] $9x+2y+z=78$
\end{itemize}

\item 
\begin{itemize}
\item[{\bf I.}] $7x+6y=110$\qquad\qquad {\bf II.} $4x+3y=59$

\item[{\bf III.}] $x+z=15$
\end{itemize}

\item 
\begin{itemize}
\item[{\bf I.}] $x=\sqrt{\left[(36)^{1/2}\times(1296)^{1/4}\right]}$

\item[{\bf II.}] $2y+3z=33$\qquad\qquad~{\bf III.} $6y+5z=71$
\end{itemize}

\item 
\begin{itemize}
\item[{\bf I.}] $8x+7y=135$\qquad\qquad {\bf II.} $5x+6y=99$

\item[{\bf III.}] $9y+8z=121$
\end{itemize}

\item 
\begin{itemize}
\item[{\bf I.}] $(x+y)^{3}=1331$\qquad\quad~ {\bf II.} $x-y+z=0$

\item[{\bf III.}] $xy=28$
\end{itemize}

\end{enumerate}

\noindent
\rule{\columnwidth}{1pt}

\noindent
{\sf\bfseries DIRECTIONS (Qs. 31--34):} {\it In each of the following questions two equations I and II are given. You have to solve both the equations and give answer.}

\noindent
\rule{\columnwidth}{1pt}

\begin{tabular}{l@{\qquad\quad}l@{\qquad\quad}l}
(a) if $a<b$ & (b) if $a\leq b$ & (c) if $a\geq b$\\
(d) if $a=b$ & (e) if $a>b$ 
\end{tabular}

\begin{enumerate}[leftmargin=*]
\setcounter{enumi}{30}
\item 
\begin{itemize}
\begin{multicols}{2}
\item[{\bf I.}] $a^{2}-5a+6=0$
\item[{\bf II.}] $b^{2}-3b+2=0$
\end{multicols}
\end{itemize}

\item 
\begin{itemize}
\begin{multicols}{2}
\item[{\bf I.}] $2a+3b=31$
\item[{\bf II.}] $3a=2b+1$
\end{multicols}
\end{itemize}

\item 
\begin{itemize}
\begin{multicols}{2}
\item[{\bf I.}] $2a^{2}+5a+3=0$
\item[{\bf II.}] $2b^{2}-5b-3=0$
\end{multicols}
\end{itemize}

\item 
\begin{itemize}
\begin{multicols}{2}
\item[{\bf I.}]  $4a^{2}=1$
\item[{\bf II.}] $4b^{2}-12b+5=0$
\end{multicols}
\end{itemize}
\end{enumerate}


\noindent
\rule{\columnwidth}{1pt}

\noindent
{\sf\bfseries DIRECTIONS (Qs. 35--39):} {\it In each of the following questions there are two equations. Solve them and choose the correct option.}

\noindent
\rule{\columnwidth}{1pt}

\begin{tabular}{l@{\qquad\quad}l@{\qquad\quad}l}
(a) If $P<Q$ & (b) If $P>Q$ & (c) If $P\leq Q$\\
(d) If $P\geq Q$ & (e) If $P=Q$ 
\end{tabular}

\begin{enumerate}[leftmargin=*]
\setcounter{enumi}{34}
\item 
\begin{itemize}
\begin{multicols}{2}
\item[{\bf I.}] $4P^{2}-8P+3=0$
\item[{\bf II.}] $2Q^{2}-13Q+15=0$
\end{multicols}
\end{itemize}

\item 
\begin{itemize}
\begin{multicols}{2}
\item[{\bf I.}] $P^{2}+3P-4=0$
\item[{\bf II.}] $3Q^{2}-10Q+8=0$
\end{multicols}
\end{itemize}

\item 
\begin{itemize}
\begin{multicols}{2}
\item[{\bf I.}] $3P^{2}-10P+7=0$
\item[{\bf II.}] $15Q^{2}-22Q+8=0$
\end{multicols}
\end{itemize}

\item 
\begin{itemize}
\begin{multicols}{2}
\item[{\bf I.}] $20P^{2}-17P+3=0$
\item[{\bf II.}] $20Q^{2}-9Q+1=0$
\end{multicols}
\end{itemize}

\item 
\begin{itemize}
\begin{multicols}{2}
\item[{\bf I.}] $20P^{2}+31P+12=0$
\item[{\bf II.}] $21Q^{2}+23Q+6=0$
\end{multicols}
\end{itemize}
\end{enumerate}

\noindent
\rule{\columnwidth}{1pt}

\noindent
{\sf\bfseries DIRECTIONS (Qs. 40--44):} {\it For the two given equations I and II, give answer.}

\noindent
\rule{\columnwidth}{1pt}

\begin{itemize}
\item[(a)] if $a$ is greater than $b$
\item[(b)] if $a$ is smaller than $b$
\item[(c)] if $a$ is equal to $b$
\item[(d)] if $a$ is either equal to or greater than $b$
\item[(e)] if $a$ is either equal to or smaller than $b$
\end{itemize}

\begin{enumerate}[leftmargin=*]
\setcounter{enumi}{39}
\item 
\begin{itemize}
\begin{multicols}{2}
\item[{\bf I.}]  $\sqrt{2304}=a$
\item[{\bf II.}] $b^{2}=2304$
\end{multicols}
\end{itemize}

\item 
\begin{itemize}
\begin{multicols}{2}
\item[{\bf I.}] $12a^{2}-7a+1=0$
\item[{\bf II.}] $15b^{2}-16b+4=0$
\end{multicols}
\end{itemize}

\item 
\begin{itemize}
\begin{multicols}{2}
\item[{\bf I.}] $a^{2}+9a+20=0$
\item[{\bf II.}] $2b^{2}-10b+12=0$
\end{multicols}
\end{itemize}

\item 
\begin{itemize}
\begin{multicols}{2}
\item[{\bf I.}] $3a+2b=14$
\item[{\bf II.}] $a+4b-13=0$
\end{multicols}
\end{itemize}

\item 
\begin{itemize}
\begin{multicols}{2}
\item[{\bf I.}] $a^{2}-7a+12=0$
\item[{\bf II.}] $b^{2}-9b+20=0$
\end{multicols}
\end{itemize}
\end{enumerate}

\noindent
\rule{\columnwidth}{1pt}

\noindent
{\sf\bfseries DIRECTIONS (Qs. 45--49):} {\it In each question one or more equation(s) is (are) provided. On the basis of these you have.}

\noindent
\rule{\columnwidth}{1pt}

\begin{quote}
Give answer (a) if $p=q$

Give answer (b) if $p>q$

Give answer (c) if $q>p$

Give answer (d) if $p\geq q$ and

Give answer (e) if $q\geq p$
\end{quote}

\begin{enumerate}[leftmargin=*]
\setcounter{enumi}{44}
\item 
\begin{itemize}
\item[{\bf I.}] $\dfrac{5}{28}\times \dfrac{9}{8}p=\dfrac{15}{14}\times \dfrac{13}{16}q$
\end{itemize}

\item 
\begin{itemize}
\begin{multicols}{2}
\item[{\bf I.}]  $p-7=0$
\item[{\bf II.}] $3q^{2}-10q+7=0$
\end{multicols}
\end{itemize}

\item 
\begin{itemize}
\begin{multicols}{2}
\item[{\bf I.}]  $4p^{2}=16$
\item[{\bf II.}] $q^{2}-10q+25=0$
\end{multicols}
\end{itemize}

\item 
\begin{itemize}
\begin{multicols}{2}
\item[{\bf I.}]  $4p^{2}-5p+1=0$
\item[{\bf II.}] $q^{2}-2q+1=0$
\end{multicols}
\end{itemize}

\item 
\begin{itemize}
\begin{multicols}{2}
\item[{\bf I.}]  $q^{2}-11q+30=0$
\item[{\bf II.}] $2p^{2}-7p+6=0$
\end{multicols}
\end{itemize}
\end{enumerate}

\noindent
\rule{\columnwidth}{1pt}

\noindent
{\sf\bfseries DIRECTIONS (Qs. 50--52):} {\it In each equation below one or more equation(s) is/are provided. On the basis of these, you have to find out relation between $p$ and $q$.}

\noindent
\rule{\columnwidth}{1pt}

\begin{quote}
Give answer (a) if $p=q$, Given answer

(b) if $p>q$, Give answer

(c) if $q>p$, Give answer

(d) if $p\geq q$ and Give answers

(e) if $q\geq p$
\end{quote}

\begin{enumerate}[leftmargin=*]
\setcounter{enumi}{49}
\item 
\begin{itemize}
\begin{multicols}{2}
\item[{\bf I.}]  $4q^{2}+8q=4q+8$
\item[{\bf II.}] $p^{2}+9p=2p-12$
\end{multicols}
\end{itemize}

\item 
\begin{itemize}
\begin{multicols}{2}
\item[{\bf I.}]  $2p^{2}+40=18p$
\item[{\bf II.}] $q^{2}=13q-42$
\end{multicols}
\end{itemize}

\item 
\begin{itemize}
\begin{multicols}{2}
\item[{\bf I.}]  $6q^{2}+\dfrac{1}{2}=\dfrac{7}{2}q$
\item[{\bf II.}] $12p^{2}+2=10p$
\end{multicols}
\end{itemize}
\end{enumerate}

\noindent
\rule{\columnwidth}{1pt}

\noindent
{\sf\bfseries DIRECTIONS (Qs. 53--57):} {\it In each of the following questions two equations are given. You have to solve them and give answer.}

\noindent
\rule{\columnwidth}{1pt}

\begin{tabular}{l@{\qquad\quad}l@{\qquad\quad}l}
(a) if $x>y$; & (b) $if x<y$; & (c) if $x=y$;\\
(d) if $x\geq y$; & (e) if $x\leq y$; & 
\end{tabular}

\begin{enumerate}[leftmargin=*]
\setcounter{enumi}{52}
\item 
\begin{itemize}
\begin{multicols}{2}
\item[{\bf I.}] $y^{2}-6y+9=0$
\item[{\bf II.}] $x^{2}-2x-3=0$
\end{multicols}
\end{itemize}

\item 
\begin{itemize}
\begin{multicols}{2}
\item[{\bf I.}] $x^{2}-5x+6=0$
\item[{\bf II.}] $2y^{2}+3y-5=0$
\end{multicols}
\end{itemize}

\item 
\begin{itemize}
\begin{multicols}{2}
\item[{\bf I.}] $x=\sqrt{256}$
\item[{\bf II.}] $y=(-4)^{2}$
\end{multicols}
\end{itemize}

\item 
\begin{itemize}
\begin{multicols}{2}
\item[{\bf I.}] $x^{2}-6x+5=0$
\item[{\bf II.}] $y^{2}-13y+42=0$
\end{multicols}
\end{itemize}

\item 
\begin{itemize}
\begin{multicols}{2}
\item[{\bf I.}] $x^{2}+3x+2=0$
\item[{\bf II.}] $y^{2}-4y+1=0$
\end{multicols}
\end{itemize}

\item If $3x-5y=5$ and $\dfrac{x}{x+y}=\dfrac{5}{7}$, then what is the value of $x-y$?

\begin{tabular}{l@{\qquad\quad}l@{\qquad\quad}l}
(a) 9 & (b) 6 & (c) 4\\
(d) 3 & \multicolumn{2}{@{}l}{(e) None of these}
\end{tabular}

\item $\dfrac{5}{7}$ of $\dfrac{4}{15}$ of a number is 8 more than $\dfrac{2}{5}$ of $\dfrac{4}{9}$ of the same number. What is half of that number?

\begin{tabular}{l@{\qquad\quad}l@{\qquad\quad}l}
(a) 630 & (b) 315 & (c) 210\\
(d) 105 & \multicolumn{2}{@{}l}{(e) None of these}
\end{tabular}

\item The difference between a two-digit number obtained by interchanging the positions of its digits is 36. What is the difference between the two digits of that number?

\begin{tabular}{l@{\qquad\quad}l@{\qquad\quad}l}
(a) 4 & (b) 9 & (c) 3\\
\multicolumn{2}{l}{(d) Cannot be determined} & (e) None of these
\end{tabular}

\item By the how much is two-fifth of 200 greater than three-fifths of 125?

\begin{tabular}{l@{\qquad\quad}l}
(a) 15 & (b) 3\\
(c) 5 & (d) 30\\
(e) None of these &
\end{tabular}

\item If $\dfrac{x^2 - 1}{x + 1} = 2$, then $x = ?$

\begin{tabular}{l@{\qquad\quad}l}
(a) 1 & (b) 0\\
(c) 2 & (d) Can't be determined\\
(e) None of these
\end{tabular}

\item The difference between a number and its one-third is double of its one-third. What is the number?

\begin{tabular}{l@{\qquad\quad}l}
(a) 60  & (b) 18 \\
  (c) 30  & (d) Can;t be determined \\
  (e) None of these
\end{tabular}

\item Tow pens and three pencils cost \rupee~86. Four pens and a pencil cost \rupee~112. What is the difference between the cost of a pen and that of a pencil?

\begin{tabular}{l@{\qquad\quad}l}
(a) \rupee~25  & (b) \rupee~13 \\
(c) \rupee~19  & (d) Cannot be determined\\
(e) None of these
\end{tabular}

\item The difference between a two-digit number and the number after inter changing the position of the two digits is 36. What is the difference between the two digits of the number?

\begin{tabular}{l@{\qquad\quad}l}
(a) 4  & (b) 6 \\
(c) 3  & (d) Cannot be determined \\
(e) None of these
\end{tabular}

\item If the digit in the unit's place of a two-digit number is halved and the digit in the ten's place is doubled, the number thus, obtained is equal to the number obtained by interchanging the digits. Which of the following is \textbf{definitely true}?

\begin{enumerate}
\item Digits in the unit's place and the ten's place are equal.
\item Sum of the digits is a two-digit number.
\item Digit in the unit's place is half of the digit in the ten's place.
\item Digit in the unit's place is twice the digit in the ten's place.
\item None of these
\end{enumerate}

\item If $A$ and $B$ are positive integers such that $9A^2 = 12A + 96$ and $B^2 = 2B + 3$, then which of the following is the value of $5A + 7B$?

\begin{tabular}{l@{\qquad\quad}l}
(a) 31 & (b) 41 \\
(c) 36 & (d) 43 \\
(e) 27
\end{tabular}

\item On Children's Day, sweets were to be equally distributed among 175 children in a school. Actually on the Children's Day 35 children were absent and therefore, each child got 4 sweets extra. How many sweets were available in all for distribution?

\begin{tabular}{l@{\qquad\quad}l}
(a) 2480  & (b) 2680 \\
(c) 2750  & (d) 2400 \\
(e) None of these
\end{tabular}

\item A two-digit number is seven times the sum of its digits. If each digit is increased by 2, the number thus obtained is 4 more than six times the sum of its digits. Find the number

\begin{tabular}{l@{\qquad\quad}l}
(a) 42 & (b) 24 \\
(c) 48 & (d) Data inadequate \\
(e) None of these
\end{tabular}

\item One-third of Ramani's saving in National Savings Certificate is equal to one-half of his savings in Public Provident Fund. If he has \rupee~150000 as total savings, how much he saved in Public Provident Fund?

\begin{tabular}{l@{\qquad\quad}l}
(a) \rupee~60000 & (b) \rupee~50000 \\
(c) \rupee~90000 & (d) \rupee~30000 \\
(e) None of these
\end{tabular}

\item $\dfrac{1}{5}$ of a number is equal to $\dfrac{5}{8}$ of the second number. If 35 is added to the first number then it becomes 4 times of second number. What is the value of the second number?

\begin{tabular}{l@{\qquad\quad}l}
(a) 125 & (b) 70 \\
(c) 40  & (d) 25\\
(e) None of these
\end{tabular}

\item In a two-digit number, the digit at unit place is 1 more than twice of the digit at tens place. If the digit at unit and tens place be interchanged, then the difference between the new number and original number is less than 1 to that of original number. What is the original number?

\begin{tabular}{l@{\qquad\quad}l}
(a) 52 & (b) 73 \\
(c) 25 & (d) 49 \\
(e) 37
\end{tabular}

\item Free notebooks were distributed equally among children of a class. The number of notebooks each child got was one-eighth of the number of children. Had the number of children been half, each child would have got 16 noteboks. How many notebooks were distributed in all?

\begin{tabular}{l@{\qquad\quad}l}
(a) 432 & (b) 640 \\
(c) 256 & (d)  512 \\
(e) None of these
\end{tabular}

\item Twenty times a positive integer is less than its square by 96. What is the integer?

\begin{tabular}{l@{\qquad\quad}l}
(a) 24 & (b) 20 \\
(c) 30 & (d) Cannot be determined\\
(e) None of these
\end{tabular}

\item The digit in the units place of a number is equal tot he digit in the tens place of half of that number and the digit in the tens place of that number is less than the digit in units place of half of the number by 1. If the sum of the digits of the number is seven, then what is the number?

\begin{tabular}{l@{\qquad\quad}l}
(a) 52 & (b) 16 \\
(c) 34 & (d) Data inadequate \\
(e) None of these
\end{tabular}

\item The difference between a two-digit number and the number obtained by interchanging the digit is 9. What is the difference between the two digits of the number?

\begin{tabular}{l@{\qquad\quad}l}
(a) 8 & (b) 2 \\
(c) 7 & (d) Cannot be determined \\
(e) None of these
\end{tabular}

\item The difference between a number and its three-fifths is 50. What is the number?

\begin{tabular}{l@{\qquad\quad}l}
(a) 75 & (b) 100 \\
(c) 125 & (d) Cannot be determined\\
(e) None of these
\end{tabular}

\item If the numerator of a fraction is increased by 2 and the denominator is increased by 1, the fraction becomes $\dfrac{5}{8}$ and if the numerator of the same fraction is increased by 3 and the denominator is increased by 1 the fraction becomes $\dfrac{3}{4}$. What is the original fraction?

\begin{tabular}{l@{\qquad\quad}l}
(a) Data inadequate & (b) $\dfrac{2}{7}$\\[0.3cm]
(c) $\dfrac{4}{7}$ & (d) $\dfrac{3}{7}$\\
(e) None of these
\end{tabular}

\item If $2x + 3y = 26$; $2y + z = 19$ and $x + 2z = 29$, what is the value of $x + y + z$?

\begin{tabular}{l@{\qquad\quad}l}
(a) 18 & (b) 32 \\
(c) 26 & (d) 22 \\
(e) None of these
\end{tabular}

\item If the sum of a number and its square is 182, what is the number?

\begin{tabular}{l@{\qquad\quad}l}
(a) 15 & (b) 26 \\
(c) 28 & (d) 91 \\
(e) None of these
\end{tabular}

\item A certain number of tennis balls were purchased for \rupee~450. Five more balls could have been purchased for the same amount if each ball was cheaper by \rupee~15. Find the number of balls purchased.

\begin{tabular}{l@{\qquad\quad}l}
(a) 15 & (b) 20 \\
(c) 10 & (d) 25 \\
(e) None of these
\end{tabular}

\item What will be the value of $n^4 - 10n^3 + 36n^2 - 49n + 24$, if $n = 1$?

\begin{tabular}{l@{\qquad\quad}l}
(a) 21 & (b) 2 \\
(c) 1  & (d) 22 \\
(e) None of these
\end{tabular}

\item Out of total number of students in a college $12\%$ are interested in sports. $\dfrac{3}{4}$th of the total number of students are interested in dancing. $10\%$ of the total number of students are interested in singing and the remaining 15 students are not interested in any of the activities. What is the total number of students in the college?

\begin{tabular}{l@{\qquad\quad}l}
(a) 450 & (b) 500 \\
(c) 600 & (d) Cannot be determined \\
(e) None of these
\end{tabular}

\item The sum of four numbers is 64. If you add 3 to the first number, 3 is subtracted from the second number, the third is multiplied by 3 and the fourth is divided by 3, then all the results are equal. What is the difference between the largest and the smallest of the original numbers?

\begin{tabular}{l@{\qquad\quad}l}
(a) 32 & (b) 27 \\
(c) 21 & (d) Cannot be determined \\
(e) None of these
\end{tabular}

\item A classroom has equal number of boys and girls. Eight girls left to play Kho-kho, leaving twice as many boy as girls in the classroom. What was the total number of girls and boys present initially?

\begin{tabular}{l@{\qquad\quad}l}
(a) Cannot be determined & (b) 16 \\
(c) 24 & (d) 32 \\
(e) None of these
\end{tabular}

\item The difference between the digits of a two-digit number is one-ninth of the difference between the original number and the number obtained by interchanging positions of the digits. What definitely is the sum of digits of that number?

\begin{tabular}{l@{\qquad\quad}l}
(a) 5 & (b) 14 \\
(c) 12 & (d) Data inadequate\\
(e) None of these
\end{tabular}

\item The denominator of a fraction is 2 more than thrice its numerator. If the numerator as well as denominator is increased by one, the fraction becomes 1/3. What was the original fraction?

\begin{tabular}{l@{\qquad\quad}l}
(a) $\dfrac{4}{13}$ & (b) $\dfrac{3}{11}$\\[0.3cm]
(c) $\dfrac{5}{13}$ & (d) $\dfrac{5}{11}$\\
(e) None of these
\end{tabular}

\item If $2x + y = 15$, $2y + z = 25$ and $2z + x = 26$, what is the value of z?

\begin{tabular}{l@{\qquad\quad}l}
(a) 4 & (b) 7 \\
(c) 9 & (d) 12 \\
(e) None of these
\end{tabular}

\item Which of the following values of P satisfy the inequality $P(P - 3) < 4P - 12$?
  
\begin{tabular}{l@{\qquad\quad}l}
(a)  $P > 4$ or $P - 12$? & (b) $24 \leq P < 71$ \\
(c)  $P > 13$; $P < 51$ & (d) $3 < P < 4$ \\
(e) $P = 4$, $P = +3$
\end{tabular}

\item If the ages of $P$ and $R$ are added to twice the age of $Q$, the total becomes 59. If the ages of $Q$ and $R$ are added to thrice the age of $P$, the total becomes 68. And if the age of $P$ is added to thrice the age of $Q$ and thrice the age of $R$, the total becomes 108. What is the age of $P$?

\begin{tabular}{l@{\qquad\quad}l}
(a) 15 years & (b) 19 years \\
(c) 17 years & (d) 12 years \\
(e) None of these
\end{tabular}

\item The product of the ages of Harish and Seema is 240. If twice the age of Seema is more than Harish's age by 4 years, what is Seema's age in years?

\begin{tabular}{l@{\qquad\quad}l}
(a) 12 years & (b) 20 years \\
(c) 10 years & (d) 14 years \\
(e) Data inadequate
\end{tabular}

\item What would be the maximum value of $Q$ in the following equation? $5P9 + 3R7 + 2Q8 = 1114$

\begin{tabular}{l@{\qquad\quad}l}
(a) 8 & (b) 7 \\
(c) 5 & (d) 4 \\
(e) None of above
\end{tabular}

\item Two-fifths of one-fourth of three-sevenths of a number is 15. What is half of that number?

\begin{tabular}{l@{\qquad\quad}l}
(a) 96 & (b) 196 \\
(c) 94 & (d) 188 \\
(e) None of these
\end{tabular}

\item The sum of the digits of a two-digit number is 1/11 of the sum of the number and the number obtained by interchanging the position of the digits. What is the difference between the digits of the number? 

\begin{tabular}{l@{\qquad\quad}l}
(a) 3 & (b) 2 \\
(c) 6 & (d) Data inadequate \\
(e) None of these
\end{tabular}

\item If a fraction's numerator is increased by 1 and the denominator is increased by 2 then the fraction becomes $\dfrac{2}{3}$. But when the numerator is increased by 5 and the denominator is increased by 1 then the fraction becomes $\dfrac{5}{4}$. What is the value of the original fraction?

\begin{tabular}{l@{\qquad\quad}l}
(a) $\dfrac{3}{7}$ & (b) $\dfrac{5}{8}$ \\[0.3cm]
(c) $\dfrac{5}{7}$ & (d) $\dfrac{6}{7}$ \\
(e) None of these
\end{tabular}

\item In a two-digit number the digit in the unit's place is more than the digit in the ten's place by 2. If the difference between the number and the number obtained by interchanging the digits is 18 what is the original number?

\begin{tabular}{l@{\qquad\quad}l}
(a) 46 & (b) 68 \\
(c) 24 & (d) Data inadequate\\
(e) None of these
\end{tabular}

\item If $2x + y = 17y$, $2z = 15$ and $x + z = 9$ then what is the value of $4x + 3y + z$?

\begin{tabular}{l@{\qquad\quad}l}
(a) 41 & (b) 43 \\
(c) 55 & (d) 45 \\
(e) None of these
\end{tabular}

\item If the numerator of a fraction is increased by 2 and denominator is increased by 3, the fraction becomes 7/9; and if numerator as well as denominator are decreased by 1 the fraction becomes 4/5. What is the original fraction?

\begin{tabular}{l@{\qquad\quad}l}
(a) $\dfrac{13}{16}$ & (b) $\dfrac{9}{11}$ \\[0.3cm]
(c) $\dfrac{5}{6}$ & (d) $\dfrac{17}{21}$ \\
(e) None of these
\end{tabular}

\item The inequality $3n^2 - 18n + 24 > 0$ gets satisfied for which of the following values of $n$?

\begin{tabular}{l@{\qquad\quad}l}
(a) $n < 2$ \& $n > 4$ & (b) $2 < n < 4$ \\
(c) $n > 2$ & (d) $n > 4$ \\
(e) None of these
\end{tabular}

\item A sum id divided among Rakesh, Suresh and Mohan. If the difference between the shares of Rakesh and Mohan is \rupee~7000 and between those of Suresh and Mohan is \rupee~3000, what was the sum?

\begin{tabular}{l@{\qquad\quad}l}
(a) \rupee~30,000 & (b) \rupee~13,000 \\
(c) \rupee~10,000 & (d) Cannot be determined \\
(e) None of these
\end{tabular}

\item Three-fifths of a number is 30 more than 50 per cent of that number. What is 80 per cent of that number?

\begin{tabular}{l@{\qquad\quad}l}
(a) 300 & (b) 60 \\
(c) 240 & (d) Cannot be determined \\
(e) None of these
\end{tabular}

\item The difference between a two-digit number and the number obtained by interchanging the position of the digits of the number is 27. What is the difference between the digits of that number?

\begin{tabular}{l@{\qquad\quad}l}
(a) 2 & (b) 3 \\
(c) 4 & (d) Cannot be determined \\
(e) None of these
\end{tabular}

\item The sum of the ages of a father and his son is 4 times the age of the son. If the average age of the father and the son is 28 years, what is the son's age?

\begin{tabular}{l@{\qquad\quad}l}
(a) 14 years & (b) 16 years \\
(c) 12 years & (d) Data inadequate \\
(e) None of these
\end{tabular}

\item Two-fifths of one=fourth of five-eighths of a number is 6. What is 50 per cent of that number?

\begin{tabular}{l@{\qquad\quad}l}
(a) 96 & (b) 32 \\
(c) 24 & (d) 48 \\
(e) None of these
\end{tabular}

\item The sum of the digits of a two-digit number is $\dfrac{1}{5}$ of the difference between the number and the number obtained by interchanging the positions of the digits. What definitely is the difference between the digit of that number?

\begin{tabular}{l@{\qquad\quad}l}
(a) 5 & (b) 9 \\
(c) 7 & (d) Data inadequate \\
(e) None of these
\end{tabular}

\item Ashok gave 40 per cent of the amount he had to Jayant. Jayant in turn gaver one-fourth of what he received from Ashok to Prakash. After paying \rupee~200 to the taxi-driver out of the amount he got from Jayant, Prakash now has \rupee~600 left with him. How much amount did Ashok have?

\begin{tabular}{l@{\qquad\quad}l}
(a) \rupee~1,200 & (b) \rupee~4,000 \\
(c) \rupee~8,000 & (d) Data inadequate \\
(e) None of these
\end{tabular}

\item What should be the maximum value of $Q$ in the following equation?

$5P9 - 7Q2 + 9R6 = 823$

\begin{tabular}{l@{\qquad\quad}l}
(a) 7 & (b) 5 \\
(c) 9 & (d) 6 \\
(e) None of these
\end{tabular}


\item The difference between a two-digit number and the number obtained by interchanging the position of the digits of that number is 54. What is the sum of the digits of that number?

\begin{tabular}{l@{\qquad\quad}l}
(a) 6 & (b) 9 \\
(c) 15 & (d) Data inadequate \\
(e) None of these
\end{tabular}

\item The product of two number is 192 and the sum of these two numbers is 28. What is the smaller of these two numbers?

\begin{tabular}{l@{\qquad\quad}l}
(a) 16 & (b) 14 \\
(c) 12 & (d) 18 \\
(e) None of these
\end{tabular}

\item The age of Mr. Ramesh is four times the age of his son. After ten years the age of Mr. Ramesh will be only twice the age of his son. Find the present age of Mr. Ramesh's, son.

\begin{tabular}{l@{\qquad\quad}l}
(a) 10 years & (b) 11 years \\
(c) 12 years & (d) Cannot be determined \\
(e) None of these
\end{tabular}


\item In an exercise room some discs of denominations 2kg and 5kg are kept for weightlifting. If the total number of discs is 21 and the weight of all the discs of 5kg is equal to the weight of all the discs of 2kg, find the weight of all the discs together.

\begin{tabular}{l@{\qquad\quad}l}
(a) 80kg & (b) 90kg \\
(c) 56kg & (d) Cannot be determined \\
(e) None of these
\end{tabular}

\item If the number of barrels of oil consumed doubles in a 10-year period and if $B$ barrels were consumed in the year 1940, what multiple of $B$ will be consumed in the year 2000?

\begin{tabular}{l@{\qquad\quad}l}
(a) 64 & (b) 60 \\
(c) 12 & (d) 32 \\
(e) None of these
\end{tabular}


\item The sum of three consecutive even numbers is 14 less than one-fourth of 176. What is the middle number?

\begin{tabular}{l@{\qquad\quad}l}
(a) 8 & (b) 10 \\
(c) 6 & (d) Data inadequate \\
(e) None of these
\end{tabular}

\item The price of four tables and seven chairs is \rupee~12,090. \textbf{Approximately,} what will be the price of twelve tables and twenty-one chairs?

\begin{tabular}{l@{\qquad\quad}l}
(a) \rupee~32,000 & (b) \rupee~46,000 \\
(c) \rupee~38,00 & (d) \rupee~36,000 \\
(e) \rupee~39,000
\end{tabular}

\item If the price of 253 pencils is \rupee~4263.05, what will be the \textbf{approximate} value of 39 such pencils?

\begin{tabular}{l@{\qquad\quad}l}
(a) \rupee~650 & (b) \rupee~550 \\
(c) \rupee~450& (d) \rupee~700 \\
(e) \rupee~750
\end{tabular}

\item Sundari, Kusu and Jyoti took two tests each. Sundari secured $\dfrac{24}{60}$ marks in the first test and $\dfrac{32}{40}$ marks in the second test. Kusu secured $\dfrac{35}{70}$ marks in the first test and $\dfrac{54}{60}$ marks in the second test. Jyoti secured $\dfrac{27}{90}$ marks in the first test and $\dfrac{45}{50}$ marks in the second test. Who among them did register maximum progress?

\begin{tabular}{l@{\qquad\quad}l}
(a) Only Sundari & (b) Only Kusu \\
(c) Only Jyothi & (d) Both Sundari and Kusu \\
(e) Both Kusu and Jyoti 
\end{tabular}
\end{enumerate}

\noindent
\rule{\columnwidth}{1pt}

\noindent
{\sf\bfseries DIRECTIONS (Qs. 117-121):} {\it In each of the following questions two equations are given. You have to solve them and given answer accordingly.}

\noindent
\rule{\columnwidth}{1pt}

\hfill{\bf [IBPS PO/MT 2013]}

\begin{tabular}{l@{\qquad\quad}l}
(a) If $x > y$ & (b) If $x < y$\\ 
(c) If $x = y$ & (d) If $x \geq y$\\ 
(e) If $x \leq y$
\end{tabular}

\begin{enumerate}[leftmargin=*]
\setcounter{enumi}{116}
\item 

\begin{itemize}
\item[I.] $2x^2 + 5x + 1 = x^2 + 2x - 1$
\item[II.] $2y^2 - 8y + 1 = -1$
\end{itemize}

\item
\begin{itemize}
\item[I.] $\dfrac{x^2}{2} + x - \dfrac{1}{2} = 1$
\item[II.] $3y^2 - 10y + 8 = y^2 + 2y - 10$
\end{itemize}

\item
\begin{itemize}
\item[I.] $4x^2 - 20x + 19 = 4x - 1$
\item[II.] $2y^2 = 26y + 84$
\end{itemize}

\item
\begin{itemize}
\item[I.] $y^2 + y - 1 = 4 - 2y - y^2$
\item[II.] $\dfrac{x^2}{2} - \dfrac{3}{2}x = x - 3$
\end{itemize}

\item
\begin{itemize}
\item[I.] $6x^2 + 13x = 12 - x$
\item[II.] $1 + 2y^2 = 2y + \dfrac{5y}{6}$
\end{itemize}
\end{enumerate}

\noindent
\rule{\columnwidth}{1pt}

\noindent
{\sf\bfseries DIRECTIONS (Qs. 122-126):} {\it In the following questions, two equations numbered I and II are given. You to solve both the equations and give answers.}

\noindent
\rule{\columnwidth}{1pt}

\hfill{\bf [IBPS PO/MT 2014]}

\begin{tabular}{l@{\qquad\quad}l}
(a) if $x > y$\\
(b) if $x \geq y$\\
(c) if $x < y$\\
(d) if $x \leq y$\\
(e) if $x = y$ or the relationship cannot be established
\end{tabular}

\begin{enumerate}[leftmargin=*]
\setcounter{enumi}{121}
\item
\begin{itemize}
\item[I.] $12x^2 + 11x + 12 = 10x^2 + 22x$
\item[II.] $13y^2 - 18y + 3 = 9y^2 - 10y$
\end{itemize}

\item
\begin{itemize}
\item[I.] $\dfrac{18}{x^2} + \dfrac{6}{x} - \dfrac{12}{x^2} = \dfrac{8}{x^2}$
\item[II.] $y^3 + 9.68 + 5.64 = 16.95$
\end{itemize}

\item
\begin{itemize}
\item[I.] $\sqrt{1225x} + \sqrt{4900} = 0$
\item[II.] $(81)^{1/4}y + (343)^{1/3} = 0$
\end{itemize}

\item
\begin{itemize}
\item[I.] $1. \dfrac{(2)^{5} + (11)^{3}}{6} = x^3$
\item[II.] $4y^3 = -(589 \div 4) + 5y^3$
\end{itemize}

\item
\begin{itemize}
\item[I.] $(x^{7/5} \div 9) = 169 \div x^{3/5}$
\item[II.] $y^{1/4} \times y^{1/4} \times 7 = 273 \div y^{1/2}$
\end{itemize}
\end{enumerate}


\begin{center}
{\bf ANSWER KEY}
\end{center}

\bigskip

\begin{center}
{\bf Hints \& Explanations}
\end{center}

\begin{enumerate}
\item
\begin{itemize}
\item[(a)] If\ $x > y$\\
 $(x - 4)(4x - 3) = 0$\\
 $x = 4, 3/4$\\
 $(y + 2)(3y + 2) = 0$\\
 $y = -2, -2/3$\\
 $4, 3/4, -2/3, -2$\\
$\therefore~ x > y$
\end{itemize}
\item
\begin{itemize}
\item[(d)] If $x \leq y$\\
  $x = \pm 27$\\
  $y = 9 + 18 = 27$\\
$\therefore~ x \leq y$
\end{itemize}
\item~ (b)

\item
\begin{itemize}
\item[(c)] if $x \geq y$\\
  $(4x - 1)(5x - 3) = 0$\\
  $x = 1/4, 3/5$\\
  $(4y - 1)(5y - 1) = 0$\\
  $y = 1/4, 1/5$\\
  $3/5, 1/4, 1/4, 1/5$\\
  $x \geq y$
\end{itemize}
\item
  \begin{itemize}
  \item[(e)] if $x = y$ or cannot be established\\
    $(x - 4)(4x + 3) = 0$\\
    $x = 4, -3/4$\\
    $(x + 5)(x-12) = 0$\\
    $y = 12, -5$
  \end{itemize}
\item
\begin{itemize}
\item[(b)] $(x + 13)(x + 17) = 0$\\
  $x = -13, -17$\\
  $y^2 - 53y + 196 = 0$\\
  $y = 49, 4$\\
  Hence, $x < y$
\end{itemize}
\item
\begin{itemize}
\item[(b)] $2x^2 - 9x + 10 = 0$\\
  $x = 2.5,2$\\
  $y^2 - 18y + 72 = 0$\\
  $y = 12,6$\\
  Hence, $x < y$
\end{itemize}
\item
\begin{itemize}
\item[(e)] $x(35 - x) = 124$\\
  $x = 31,4$\\
  $y(2y + 3) = 90$\\
  $y = -7.5,6$\\
\end{itemize}
\item
\begin{itemize}
\item[(b)] $1/(x - 3) + 1/(x + 5) = 1/3$\\
  $x^2 - 4x - 21 = 0$\\
  $x = 7, -3$\\
  $(y + 2)(27 - y) = 210$\\
  $y^2 - 25x + 156 = 0 = > y = 12, 13$\\
  $x < y$
\end{itemize}
\item
\begin{itemize}
\item[(a)] $\sqrt{36x} + \sqrt{64} = 0$\\
  $6x + 8 = 0$\\
  $x = -1.33$\\
  $\sqrt{81y} + (4)^{2} = 0$\\
  $9y + 16 = 0$\\
  $y = -1.77$\\
  Hence, $x > y$
\end{itemize}
\item
\begin{itemize}
\item[(a)] $(x + 2)(x + 1) = (x - 2)(x - 3)$\\
  $x = \dfrac{1}{2} = 0.5$\\[0.2cm]
  $(y + 3)(y + 2) = (y - 1)(y - 2)$\\[0.2cm]
  $y = -\dfrac{1}{2} = -0.5$
\end{itemize}
\item
\begin{itemize}
\item[(a)] $12x^2 + 29x + 14 = 0$\\
  $x = -1.75, -0.6$\\
  $y^2 + 9y + 18 = 0$\\
  $y = -6, -3$
\end{itemize}
\item
  \begin{itemize}
  \item[(a)] $5x^2 - 18x + 9 = 0$\\
    $x = 0.6, 3$\\
    $3y^2 + 5y - 2 = 0$\\
    $y = 0.33, -2$
  \end{itemize}
\item
\begin{itemize}
\item[(b)] $17^2 + 144 \div 18 = x$\\
  $x = 297$\\
  $26^2 - 18 \times 21$\\
  $y - 676 - 378 = 29$
\end{itemize}
\item
\begin{itemize}
\item[(d)] $30x^2 + 11x + 1 = 0$\\
  $30x^2 + 6x + 5x + 1 = 0$\\
  $x = -0.16, -0.19$\\
  $42y^2 + 13y + 1 = 0$\\
  $42y^2 + 6y + 7y + 1 = 0$\\
  $y = -0.14, -0.16$\\
  Put on number line\\
  $-0.19, -0.16, -0.16, -0.14$
\end{itemize}
\item
\begin{itemize}
\item[(c)]
\begin{itemize}
\item[{\bf I.}] $6x^2 + 25x + 24 = 0$\\
\begin{align*}
  D & = \sqrt{b^2 - 4ac}\\
  D & = \sqrt{625 - 4 \times 24 \times 6}\\
  & = \sqrt{49} = 7
\end{align*}
\begin{align*}
  x_1 & = \dfrac{-b + 7}{12} = \dfrac{-25 + 7}{12} = \dfrac{-18}{12} = -\dfrac{3}{2}\\[0.2cm]
  x_2 & = \dfrac{-b - 7}{12} = \dfrac{-25 - 7}{12} = \dfrac{-32}{12} = -\dfrac{8}{3}\\[0.2cm]
  x & = \dfrac{-3}{2}, \dfrac{-8}{3}
\end{align*}

\item[ {\bf II.}] $12y^2 + 13y + 3 = 0$\\
\begin{align*}
  y_1 & = \dfrac{-13 + \sqrt{169 = 144}}{24}\\
  & = \dfrac{-13 + 5}{24} = \dfrac{-8}{24} = \dfrac{-1}{3}\\
  y_2 & = \dfrac{-13 - \sqrt{169 - 144}}{24} = \dfrac{-18}{24} = \dfrac{-3}{4}\\
  y & = \dfrac{-1}{3}, \dfrac{-3}{4} \Rightarrow x < y
\end{align*}
\end{itemize}
\end{itemize}
\item
\begin{itemize}
\item[(c)]
\begin{itemize}
\item[{\bf I.}] $12x^2 - x - 1 = 0$
  \begin{align*}
    x_1 & = \dfrac{-b + \sqrt{D}}{2a} = \dfrac{1 + \sqrt{1 - 4 \times 12 \times -1}}{24}\\
    & = \dfrac{1 + 7}{24} = \dfrac{8}{24} = \dfrac{1}{3}\\
    x_2 & = \dfrac{-b - \sqrt{D}}{2a}\\
    x_2 & = \dfrac{1 - 7}{24} = \dfrac{-6}{24} = \dfrac{-1}{4}\\
    x & = \dfrac{1}{3}, -\dfrac{1}{4}
  \end{align*}
\item[{\bf II.}] $20y^2 - 41y + 20$
\begin{align*}
  y_1 & = \dfrac{41 - \sqrt{1681 - 1600}}{40}\\
  y_2 & = \dfrac{41 - \sqrt{1681 - 1600}}{40}\\
  y_1 & = \dfrac{41 + 9}{40} = \dfrac{50}{40}, y_2 = \dfrac{32}{40}\\
  y & = \dfrac{5}{4}, \dfrac{4}{5} \qquad \Rightarrow \quad x < y
\end{align*}
\end{itemize}
\end{itemize}
\item
\begin{itemize}
\item[(b)]
\begin{itemize}
\item[{\bf I.}] $10x^2 + 33x + 27 = 0$
\begin{align*}
  x_1 & = \dfrac{-33 + \sqrt{b^2 - 4ac}}{2a} = \dfrac{-33 + \sqrt{1089 = 4 \times 10 \times 27}}{20}
  x_2 & = \dfrac{-33 - \sqrt{b^2 - 4ac}}{2a}\\
  x_2 & = \dfrac{-33 - \sqrt{1089 - 1080}}{20}\\
  x_1 & = \dfrac{-33 + 3}{20},\ x_2 = \dfrac{-33 - 3}{20}\\
  x_1 & = \dfrac{-30}{20},\ x_2 = \dfrac{-36}{20} = \dfrac{-9}{5},\ x = \dfrac{-3}{2}, \dfrac{-9}{5}
\end{align*}
\item[{\bf II.}] $5y^2 + 19y + 18 = 0$
\begin{align*}
  y_1 & = \dfrac{-19 - \sqrt{361 - 4 \times 18 \times 5}}{10}\\
  y_2 & = \dfrac{-19 - \sqrt{361 - 360}}{10}\\
  y_1 & = \dfrac{-19 + 1}{10}\\
  y_2 & = \dfrac{-19 - 1}{10} = \dfrac{-18}{10} = \dfrac{-9}{5} = \dfrac{-20}{10} = -2\\
  y & = \dfrac{-9}{5}, -2 \quad \Rightarrow \quad x \geq y
\end{align*}
\end{itemize}
\end{itemize}
\item
\begin{itemize}
\item[(d)]
\begin{itemize}
\item[{\bf I.}] $15x^2 - 29x - 14 = 0$
  \begin{align*}
    x_1 & = \dfrac{29 + \sqrt{841 + 60 \times 14}}{30}\\
    & = \dfrac{29 + 41}{30} = \dfrac{70}{30}\\
    x_2 & = \dfrac{29 - \sqrt{1681}}{30}\\
    x_2 & = \dfrac{29 - 41}{30} = \dfrac{-12}{30}\\
    x & = \dfrac{7}{3} \cdot \dfrac{-2}{5}
  \end{align*}
\item[{\bf II.}] $6y^2 - 5y - 25 = 0$
\begin{align*}
 y_1 & = \dfrac{5 + \sqrt{25 - 4 \times 6 \times -25}}{12} = \dfrac{5 + \sqrt{625}}{12} = \dfrac{30}{12}\\
 y_2 & = \dfrac{5 - \sqrt{25 - 4 \times 6 \times -25}}{12} \Rightarrow y_2 = \dfrac{5 - \sqrt{625}}{12} = \dfrac{-20}{12}\\
 y & = \dfrac{5}{2}, \dfrac{-5}{3}
\end{align*}
So, relationship between $x$ and $y$ can't be determined.
\end{itemize}
\end{itemize}
\item
\begin{itemize}
\item[(b)]
\begin{itemize}
\item[{\bf I.}] $3x^2 - 22x + 7 = 0$\\
  $3x^2 - 21x - x + 7 = 0$\\
  $x(3x - 1) - 7(3x - 1) = 0$\\
  $(3x - 1)(x - 7) = 0$\\
  $x = \dfrac{1}{3}, 7$

\item[{\bf II.}] $y^2 - 20y + 91 = 0$\\
  $y^2 - 13y - 7y + 91 = 0$\\
  $y(y - 7) - 13 (y - 7) = 0$\\
  $(y - 13)(y - 7) = 0$\\
  $y = 13, 7 \quad \Rightarrow \quad y \geq x$
\end{itemize}
\end{itemize}
\item
\begin{itemize}
\item[(b)]
\begin{itemize}
\item[{\bf I.}] $x^2 - 6x = 7$\\
{\bf or,}\quad $x^2 - 6x - 7 = 0$\\
{\bf or,}\quad $(x - 7)(x + 1) = 0$\\
{\bf or,}\quad $x = 7, -1$
\item[{\bf II.}] $2y^2 + 13y + 15 = 0$\\
{\bf or,}\quad $2y^2 + 3y + 10y + 15 = 0$\\
{\bf or,}\quad $(2y + 3)(y + 5) = 0$ or, $y = \dfrac{-3}{2}, -5$\\
Hence, $x > y$
\end{itemize}
\end{itemize}
\item
\begin{itemize}
\item[(a)]
\begin{itemize}
\item[{\bf I.}] $3x^2 - 7x + 2 = 0$\\
  {\bf or,}\quad $3x^2 - 6x - x + 2 = 0$\\
  {\bf or,}\quad $(x - 2)(3x - 1) = 0$\\
  {\bf or,}\quad $x = 2, 1/2$

\item[{\bf II.}] $2y^2 - 11y + 15 = 0$\\
  {\bf or,}\quad $2y^2 - 6y - 5y + 15 = 0$\\
  {\bf or,}\quad $(2y - 5)(y - 3) = 0$\\
  {\bf or,}\quad $y = 5/2. 3$\\
  Hence, $y > x$
\end{itemize}
\end{itemize}
\item
\begin{itemize}
\item[(d)]
\begin{itemize}
\item[{\bf I.}] $10x^2 - 7x + 1 = 0$\\
  {\bf or,}\quad $10x^2 - 5x - 2x + 1 = 0$\\
  {\bf or,}\quad $(2x - 1)(5x - 1) = 0$\\
  {\bf or,}\quad $x = 1/2, 1/5$
\item[{\bf II.}] $35y^2 - 12y + 1 = 0$\\
  {\bf or,}\quad $35y^2 - 7y - 5y + 1 = 0$\\
  {\bf or,}\quad $(5y - 1)(7y - 1) = 0$\\
  {\bf or,}\quad $y = \dfrac{1}{5}, \dfrac{1}{7}$\\
  Hence, $x \geq y$
\end{itemize}
\end{itemize}
\item
\begin{itemize}
\item[(a)]
\begin{itemize}
\item[{\bf I.}] $4x^2 = 25$\\
  {\bf or,}\quad $x = \pm \dfrac{5}{2}$

\item[{\bf II.}] $2y^2 - 13y + 21 = 0$\\
  {\bf or,}\quad $2y^2 - 6y - 7y + 21 = 0$\\
  {\bf or,}\quad $(y - 3)(2y - 7) = 0$\\
  {\bf or,}\quad $y = 3, \dfrac{7}{2}$\\
  Hence, $y > x$
\end{itemize}
\end{itemize}
\item
\begin{itemize}
\item[(e)]
\begin{itemize}
\item[{\bf I.}] $3x^2 + 7x - 6 = 0$\\
  {\bf or,}\quad $3x^2 + 9x - 2x - 6 = 0$\\
  {\bf or,}\quad $(x + 3)(3x - 2) = 0$\\
  {\bf or,}\quad $x = -3, \dfrac{2}{3}$
\item[{\bf II.}] $6(2y^2 + 1) = 17y$\\
  {\bf or,}\quad $12y^2 + 6 - 17y = 0$\\
  {\bf or,}\quad $12y^2 - 9y - 8y + 6 = 0$\\
  {\bf or,}\quad $(4y - 3)(3y - 2) = 0$\\
  {\bf or,}\quad $y = \dfrac{3}{4}, \dfrac{2}{3}$\\
  Hence, $y \geq x$
\end{itemize}
\end{itemize}
\item
  \begin{itemize}
\item[(a)] $7x + 6y + 4z = 122$ \quad  ...\ (i)\\
      $4x + 5y + 3z = 88$ \quad  ...\ (ii)\\
      $9x + 2y + z  = 78$ \quad  ...\ (iii)
  
  By equation (iii) $\times 3$ - equation (ii),
\begin{equation*}
  \begin{array}{cc}
    27x + 6y + 3z & = 234\\
    4x + 5y + 3z & = 88\\
    - \quad - \quad - & \quad -\\
    \hline
    23x + y & = 146\tag*{...(iv)}
  \end{array}
\end{equation*}

  By equation (iii) $\times 4$ - equation (i),
\begin{equation*}
  \begin{array}{cc}
    36x + 8y + 4z & = 312\\
    7x + 6y + 4z & = 122\\
    - \quad - \quad - & \quad -\\
    \hline
    29x + 2y & = 196\tag*{...(v)}
  \end{array}
\end{equation*}

By equation (iv) $\times 2$ - equation (v),
\begin{equation*}
  \begin{array}{cc}
    46x + 2y & = 292\\
    29x + 2y & = 190\\
    - \quad - & \quad -\\
    \hline
    17x + & = 102\tag*{...(iv)}
  \end{array}
\end{equation*}
$\Rightarrow \quad x = 6$

From equation (iv),

$\qquad 23 \times 3 + y = 146$\\
$\Rightarrow \quad y = 146 - 138 = 8$

From equation (iii),

$\qquad 9 \times 6 + 2 \times 8 + z = 78$\\
$\Rightarrow \quad 54 + 16 - z = 78$\\
$\Rightarrow \quad z = 787 - 70 = 8$

Clearly, $x < y = z$
  \end{itemize}
\item
  \begin{itemize}
  \item[(c)] By equation (II) $\times 2$ - equation (I)

\begin{tabular}{|c|c|c|}
\hline
{\bf Number} & {\bf Floor} & {\bf Person}\\
\hline
6 & Fifth Floor & P\\
\hline
5 & Fourth Floor & T\\
\hline
4 & Third Floor & V\\
\hline
3 & Second Floor & S\\
\hline
2 & First Floor & R\\
\hline
1 & Ground Floor & Q\\
\hline
\end{tabular}

From equation (I),

$7 \times 8 + 6y = 110$\\
$\Rightarrow \quad 6y = 110 - 56 = 54$\\
$\Rightarrow \quad y = 9$

From equation (iii),

$8 + z = 15 \Rightarrow z = 7$\\
Clearly, $x < y > z$
\end{itemize}
\item
  \begin{itemize}
  \item[(b)]
    \begin{itemize}
    \item[{\bf I.}] $x = \sqrt{(36) \dfrac{1}{2} \times (1296) \dfrac{1}{4}} = \sqrt{6 \times 6} = \pm 6$

      By equation $II \times 3$ - equation I
      \begin{equation*}
        \begin{array}{cc}
          6y + 9z & = 99\\
          6y + 5z & = 71\\
          - \quad - \quad & -\\
          \hline
          4z & = 28
        \end{array}
        \qquad\Rightarrow \ z = 9
      \end{equation*}

      From equation II,

      $2y + 3 \times 7 = 33$\\
      $\Rightarrow\quad 2y = 33 - 21 = 12$\\
      $\Rightarrow\quad y = 6$\\
      $x \leq y < z$
      \end{itemize}
  \end{itemize}
\item
  \begin{itemize}
  \item[(d)] By equation $1 \times 5 - II \times 8$
    \begin{equation*}
      \begin{array}{cc}
        40x + 35y & = 675\\
        40x + 48y & = 792\\
       - \quad - \quad & -\\
        \hline
        -13y & = -117
        \end{array}
    \end{equation*}
    $\Rightarrow\quad y = 9$

    From equation I,

    $\qquad 8x + 7 \times 9 = 135$\\
    $\Rightarrow\quad 8x = 135 - 63 = 72$\\
    $\Rightarrow\quad x = 9$

    From equation III,

    $\qquad 9 \times 9 + 8z = 121$\\
    $\Rightarrow\quad 8z = 121 - 81 = 40$\\
    $\Rightarrow\quad z = 5$

    Clearly, $x = y > z$
  \end{itemize}
\item
  \begin{itemize}
  \item[(e)]
    \begin{itemize}
    \item[{\bf I.}] $(x + y)^3 = 1331$

      $\Rightarrow\quad x + y = 11$\\
      $\Rightarrow\quad y = 11 - x$

      From equation III,

      $x (11 - x) = 28$\\
      $\Rightarrow\quad 11x - x^2 = 28$\\
      $\Rightarrow\quad x^2 - 11x + 28 = 0$\\
      $\Rightarrow\quad x^2 - 7x - 4x + 28 = 0$\\
      $\Rightarrow\quad x(x - 7) -4 (x - 7) = 0$\\
      $\Rightarrow\quad (x - 7) (x - 4) = 0$\\
      $\Rightarrow\quad x = 7, 4$

      From equation I

      $y = 4, 7$

      From equation II

      $7 - 4 + z = 0 \Rightarrow z = -3$\\
      $4 - 7 + z = 0 \Rightarrow z = 3$
      \end{itemize}
  \end{itemize}
\item
  \begin{itemize}
  \item[(c)] For eqn 1, the roots
    \begin{itemize}
    \item[(a).] will be $2, 3$. As $-2 \times -3 = 6 (ac)$ and $(-2) + (-3) = -5$
    \item[(b).] Similarly, for eqn II, the roots $(b)$ will be 2, 1. 
      \end{itemize}
  \end{itemize}
\item
  \begin{itemize}
  \item[(a)] $2a + 3b = 31$\quad ...(i)\\
    $3a - 2b = 1$\quad ...(ii)

    Multiply (i) by 2 and (ii) by 3 and then adding

    (i) and (ii), we get $a = \dfrac{65}{13} = 5$. Putting the value of $`a'$ in any equation, we get $b = 7$.

    Hence, $b > a$ or $a < b$.
  \end{itemize}
\item
  \begin{itemize}
\item[(a)] $a = -3/2\ \& -1$;~ $b = \dfrac{3}{2}\ \& 1$
  \end{itemize}

\item
  \begin{itemize}
\item[(b)] $a = \pm 1/2$;~ $b = 1/2, 5/2$
  \end{itemize}
\item
  \begin{itemize}
  \item[(c)]
    \begin{itemize}
    \item[{\bf I.}] $4P^2 - 8P + 3 = 0$\\
      $4P^2 - 2P - 6P + 3 = 0$\\
      $2P(2P - 1)-3 (2P - 1) = 0$\\
      $(2P - 3)(2P - 1) = 0$\\
      $\Rightarrow\quad P = 1/2. 3/2$

    \item[{\bf II.}] $2Q^2 - 13A + 15 = 0$\\
      $2Q^2 - 10Q - 3Q + 15 = 0$\\
      $2Q(Q - 5) -3(Q - 5) = 0$\\
      $\Rightarrow\quad (2Q - 3)(Q - 5) = 0$\\
      $\Rightarrow\quad Q = 3/2, 5$\\
      $\therefore\ Q \geq P$
      \end{itemize}
  \end{itemize}
\item
  \begin{itemize}
  \item[(a)]
    \begin{itemize}
    \item[{\bf I.}] $P^2 + 3P - 4 = 0$\\
      $P^2 + 4P - P - 4 = 0$\\
      $\Rightarrow\quad P(P + 4) -1(P + 4) = 0$\\
      $\Rightarrow\quad P = 1, -4$\\

    \item[{\bf II.}] $3Q^2 - 10Q + 8 = 0$\\
      $3Q^2 - 6Q - 4Q + 8 = 0$\\
      $3Q(Q - 2) -4(Q - 2) = 0$\\
      $(3Q - 4)(Q - 2) = 0$\\
      $\Rightarrow\quad Q = 4/3, 2$\\
      $\therefore~ Q > P$
      \end{itemize}
  \end{itemize}
\item
  \begin{itemize}
  \item[(b)]
    \begin{itemize}
    \item[{\bf I.}] $3P^2 - 10P + 7 = 0$\\
      $3P^2 - 3P - 7P + 7 = 0$\\
      $3P(P - 1) -7(P - 1) = 0$\\
      $\Rightarrow\quad (3P - 7)(P - 1) = 0$\\
      $\Rightarrow\quad P = 7/3, 1$

    \item[{\bf II.}] $15Q^2 - 22A + 8 = 0$\\
      $15Q^2 - 10Q - 12Q + 8 = 0$\\
      $5Q(3Q - 2) -4(3Q - 2) = 0$\\
      $(5Q - 4)(3Q - 2) = 0$\\
      $\Rightarrow\quad Q = \dfrac{4}{5}, \dfrac{2}{3}$\\
      $\therefore~ P > Q$
    \end{itemize}
  \end{itemize}
\item
  \begin{itemize}
  \item[(d)]
    \begin{itemize}
    \item[{\bf I.}] $20P^2 - 17P + 3 = 0$\\
      $20P^2 - 12P - 5P + 3 = 0$\\
      $5P(4P - 1) -3(4P - 1) = 0$\\
      $\Rightarrow\quad P = 3/5, 1/4$

    \item[{\bf II.}] $20Q^2 - 9Q + 1 = 0$\\
      $20Q^2 - 4Q - 5Q + 1 = 0$\\
      $4Q(5Q - 1) -1(5Q - 1) = 0$\\
      $(4Q - 1)(5Q - 1) = 0$\\
      $\Rightarrow\quad Q = 1/4. 1/5$\\
      $\therefore~ P \geq Q$
      \end{itemize}
  \end{itemize}
\item
  \begin{itemize}
  \item[(a)]
    \begin{itemize}
    \item[{\bf I.}] $20P^2 + 31P + 12 = 0$\\
      $20P^2 + 16P + 15P + 12 = 0$\\
      $5P(4P + 3) + 4(4P + 3) = 0$\\
      $\therefore~ P = -4/5, \dfrac{-3}{4}$

    \item[{\bf II.}] $21Q^2 + 23Q + 6 = 0$\\
      $21Q^2 + 14Q + 9Q + 6 = 0$\\
      $7Q(3Q + 2) +3(3Q + 2) = 0$\\
      $(7Q + 3)(3Q + 2) = 0$\\
      $\Rightarrow\quad Q = -3/7, -2/3$\\
      $\therefore~ Q > P$
      \end{itemize}
  \end{itemize}
\item
  \begin{itemize}
  \item[(c)]

    \textbf{From I:}
    
    If $\sqrt{2304} = a$\\
    then $a = \pm 48$\\
    (Do not consider - 48 as value of $a$)\\
    Again,

    \textbf{From II:}

    If $b^2 = 2304$ then $b = \pm 48$\\
    Hence $a = b$.
  \end{itemize}
\item
  \begin{itemize}
  \item[(b)]
    \begin{itemize}
    \item[{\bf I.}] $12a^2 - 7a + 1 = 0$
    \item[{\bf II.}] $15b^2 - 16b + 4 = 0$

      Sum of the two values of $a$, i.e., $(a_{1}, + a_{2}) = \dfrac{-(-7)}{12} = \dfrac{7}{12}$

      Similarly,

      Sum of the two values of $b$,

      i.e., $(b_1 + b_2) = \dfrac{-(-16)}{15} = \dfrac{16}{15}$

      Since $\dfrac{7}{12} < \dfrac{16}{15}$

      Therefore, $a < b$,

      Now check the equality of root

      $(12 \times 4 - 15 \times 1)^2 = \{12 \times (-16) - 15 \times (-7)\}$\\
      $\{(-7) \times 4 -(-16) \times 1 \}$\\
      $\Rightarrow\quad 33^2 = \{-87\} \{-12\}$\\
      $\Rightarrow\quad 1089 = 1044$, which is not true.

      Therefore, out answer should be $a < b$.
    \end{itemize}
  \end{itemize}
\item
  \begin{itemize}
  \item[(b)]
    \begin{itemize}
    \item[{\bf I.}] $a^2 + 9a + 20 = 0$

      Break 9 as $F_1$ and $F_2$, so that

      $F_1 \times F_2 = 20$ and $F_1 + F_2 = 9$.

      Therefore, $F_1 = 5$, $F_2 = 4$

      Now one value of $a = \dfrac{-5}{1} = -5$

      other value of $a = \dfrac{-20}{5} = -4$

    \item[{\bf II.}] $2b^2 + 10b + 12 = 0$

      The two parts of 10, ie $F_1 = 6$ and $F_2 = 4$

      $\therefore~$ Value of $b = \dfrac{-6}{2} = -3$ and $\dfrac{-12}{6} = -2$

      Obviously $b > a$.

      If general form of quadratic equation is $ax^2 + bx + c = 0$, then split $b$ into two parts so that $b_1 + b_2 = b$ and $b_1 \times b_{2e} = a \times c$

      Now say $b_1$ as $F_1$ and $b_2$ as $F_2$. Then the values of $`x'$ will be $\dfrac{-F_1}{a}$ and $\dfrac{-C}{F_1}$ or $\dfrac{-F_2}{a}$ and $\dfrac{-C}{F_2}$
      \end{itemize}
    \end{itemize}
\item
  \begin{itemize}
  \item[(a)]
    \begin{itemize}
    \item[{\bf I.}] $3a + 2b = 14$
    \item[{\bf II.}] $a + 4b = 13$

      Substract equation I from equation II after multiplying II by 3.

      We get $3a + 12b - 3a - 2b = 39 - 14$

      $\Rightarrow\quad 10b = 2.5$\\
      $\Rightarrow\quad b = 2.5$

      Put value of $b$ in equation II. We set $a + 4 \times 2.5 = 13$.

      Therefore, $a = 3$. Thus, $a > b$
      \end{itemize}
  \end{itemize}
\item
  \begin{itemize}
  \item[(e)]
    \begin{itemize}
    \item[{\bf I.}] $a^2 - 7a + 12 = 0$

      Here, $F_1 = -4$ and $F_2 = -3$

      Now, values of $a = \dfrac{-(-4)}{1} = 4$ and $\dfrac{-12}{-4} = 3$

    \item[{\bf II.}] $b^2 - 9b + 20 = 0$

      Here $F_1 = -5$ and $F_2 = -4$

      Now, values of $a = \dfrac{-(-5)}{1} = 5$ and $\dfrac{-20}{-5} = 4$

      Thus $b \geq a$.
      \end{itemize}
  \end{itemize}
\item
  \begin{itemize}
  \item[(b)] $\dfrac{5}{28} \times \dfrac{9}{8}p = \dfrac{15}{14} \times \dfrac{13}{16}a$\\
    or, $\dfrac{45p}{224} = \dfrac{195q}{224}$\\
    or, $3p = 13q$\\
    $\therefore~ p > q$
  \end{itemize}
\item
  \begin{itemize}
  \item[(b)]
    \begin{itemize}
    \item[(i)] $p - 7 = 0$ or, $p = 7$
    \item[(ii)] $3q^2 - 10q + 7 = 0$

      or, $3q^2 - 3q - 7q + 7 = 0$\\
      or, $3q(q - 1) -7(q - 1) = 0$\\
      or, $(3q - 7)(q - 1)$ or, $q = 1$ or, $\dfrac{7}{3}$\\
      $\therefore~ p > q$
      \end{itemize}
   \end{itemize}
\item
  \begin{itemize}
  \item[(c)]
    \begin{itemize}
    \item[(i)] $4p^{2}2 = 16$; $p = \sqrt{4} = 2$
    \item[(ii)] $q^2 - 10q + 25 = 0 \Rightarrow (q - 5)(q - 5) = 0$

      or, $q = 5 \therefore~ q > p$
      \end{itemize}
    \end{itemize}
\item
\begin{itemize}
\item[(e)]
  \begin{itemize}
  \item[(i)] $4p^2 - 5p + 1 = 0$ or, $4p^2 - 4p - p + i = 0$\\
    or, $4p(p - 1) -1(p - 1) = 0$\\
    or, $(4p - 1)(p - 1) = 0$\\
    or, $p = 1$ and $p = \dfrac{1}{4}$

  \item[(ii)] $q^2 - 2q + 1 = 0$\\
    $\Rightarrow (q - 1)(q - 1) = 0$\\
    or, $q = 1$\\
    $\therefore~ q \geq p$
    \end{itemize}
\end{itemize}
\item
  \begin{itemize}
\item[(c)] $q = 5, 6$ \& $p = \dfrac{3}{2}, 2$
  \end{itemize}
\item
  \begin{itemize}
  \item[(c)]
    \begin{multicols}{2}
      \begin{itemize}
      \item[{\bf I.}]
        \begin{tabular}[t]{l}
          $4q^2 + 8q = 4q + 8$\\
          or, $q^2 + q - 2 = 0$\\
          or, $(q - 1)(q + 2) = 0$\\
          $\therefore~ q = 1$ or $- 2$\\
          Hence, $q > p$
        \end{tabular}
      \item[{\bf II.}]
        \begin{tabular}[t]{l}
          $p^2 + 9p = 2p - 12$\\
          or, $p^2 + 7p + 12 = 0$\\
          or, $(p + p)(p + (c)) = 0$\\
          $\therefore~ p = -3$ or $-4$
          \end{tabular}
       \end{itemize}
     \end{multicols}
  \end{itemize}
\item
  \begin{itemize}
  \item[(c)]
        \begin{multicols}{2}
      \begin{itemize}
      \item[{\bf I.}]
        \begin{tabular}[t]{l}
          $2p^2 + 40 = 18p$\\
          or, $p^2 - 9p + 20 = 0$\\
          or, $(p - 4)(p - 5) = 0$\\
          $\therefore~ p = 4$ or $5$\\
          Hence, $q > p$
        \end{tabular}
      \item[{\bf II.}]
        \begin{tabular}[t]{l}
          $q^2 = 13q - 42$\\
          or, $q^2 - 13q + 42 = 0$\\
          or, $(q - 7)(q - 6) = 0$\\
          $\therefore~ q = 6$ or $7$
          \end{tabular}
       \end{itemize}
     \end{multicols}
  \end{itemize}
\item
  \begin{itemize}
  \item[(d)]
        \begin{multicols}{2}
      \begin{itemize}
      \item[{\bf I.}]
        \begin{tabular}[t]{l}
          $6q^2 + \dfrac{1}{2} = \dfrac{7}{2}q$\\
          or, $12q^2 - 7q + 1 = 0$\\
          or, $\left(q - \dfrac{1}{4} \right) \left(q - \dfrac{1}{3} \right) = 0$\\
          $\therefore~ q = \dfrac{1}{4}$ or $\dfrac{1}{3}$\\
          Hence, $p \geq q$
        \end{tabular}
      \item[{\bf II.}]
        \begin{tabular}[t]{l}
          $12p^2 + 2 = 10p$\\
          or, $6p^2 - 5p + 1 = 0$\\
          or, $\left(p - \dfrac{1}{3}\right) \left(p - \dfrac{1}{2}\right) = 0$\\
          $\therefore~ p = \dfrac{1}{3}$ or $\dfrac{1}{2}$
          \end{tabular}
       \end{itemize}
     \end{multicols}
  \end{itemize}
\item
  \begin{itemize}
  \item[(b)]
    \begin{itemize}
    \item[{\bf I.}] $y^2 - 6y + 9 = 0$\\
      or, $(y - 3)^2 = 0$ or, $y = 3$
    \item[{\bf II.}] $x^2 + 2x - 3 = 0$ or, $x = 1, -3$\\
      Hence, $y > x$
      \end{itemize}
  \end{itemize}
\item
  \begin{itemize}
  \item[(a)]
    \begin{itemize}
    \item[{\bf I.}] $x^2 - 5x + 6 = 0$\\
      or, $(x - 3)(x - 2) = 0$ or, $x = 2, 3$
    \item $2y^2 - 3y - 5 = 0$\\
      or, $y = 1, -\dfrac{5}{2}$\\
      Hence, $x > y$
      \end{itemize}
    \end{itemize}
\item
  \begin{itemize}
  \item[(c)]
    \begin{itemize}
    \item[{\bf I.}] $x = \sqrt{256} = 16$
    \item[{\bf II.}] $y = (-4)^2 = 16$

      Hence, $x = y$
      \end{itemize}
  \end{itemize}
\item
  \begin{itemize}
  \item[(b)]
    \begin{itemize}
    \item[{\bf I.}] $x^2 - 6x + 5 = 0$ or, $x = 1, 5$
    \item[{\bf II.}] $y^2 - 13y + 42 = 0$\\
      or, $(y - 7)(y - 6) = 0$ or, $y = 6, 7$\\
      Hence, $y > x$
      \end{itemize}
  \end{itemize}
\item
  \begin{itemize}
  \item[(b)]
    \begin{itemize}
    \item[{\bf I.}] $x^2 + 3x + 2 = 0$\\
      or, $(x + 2)(x + 1) = 0$\\
      or, $x = -2$ or, $-1$

    \item[{\bf II.}] $y^2 - 4y + 1 = 0$ or, $y = 2 \pm \sqrt{3}$\\
      Hence, $y > x$
      \end{itemize}
  \end{itemize}
\item
  \begin{itemize}
  \item[(d)] $3x - 5y = 5 \quad\qquad ...(i)$\\
    And $\dfrac{x}{x + y} = \dfrac{5}{7} \quad\Rightarrow\quad 7x = 5x + 5y$\\
    $\Rightarrow\quad 2x = 5y \qquad\quad ...(ii)$\\
    From (i) and (ii), $x = 5$ and $y = 2$\\
    $\therefore~ x - y = 3$
  \end{itemize}
\item
  \begin{itemize}
  \item[(d)] Let the number be $x$.
    $\therefore~ \dfrac{5}{7} \times \dfrac{4}{15} \times x - \dfrac{2}{5} \times \dfrac{4}{9} \times x = 8$\\
    or, $x = \dfrac{8 \times 315}{12} = 210$\\
    $\therefore~$ Half of the number = 105
  \end{itemize}
\item
  \begin{itemize}
  \item[(a)] Let the two-digit number be $10x + y$.

    Then, $(10x + y) - (10y + x) = 36$

    or, $x - y = 4$
  \end{itemize}
\item
\begin{itemize}
  \item[(c)] Reqd no. $\dfrac{2}{5} \times 200 - \dfrac{3}{5} \times 125$\\
  $= 80 - 75 = 5$
\end{itemize}
\item
  \begin{itemize}
  \item[(e)] $(x - 1) = 2 \Rightarrow x = 3$
  \end{itemize}
\item
  \begin{itemize}
  \item[(d)] Let the no. be $x$.

    Then, $x - \dfrac{x}{3} = \dfrac{2}{3}x$

    or, $\dfrac{2}{3}x = \dfrac{2}{3}x$

    So, can't be determined is the correct choice.
  \end{itemize}
\item
  \begin{itemize}
  \item[(b)] Let the cost of a pen and a pencil be \rupee~$`x'$ and \rupee~$`y'$ respectively. We have to find $(x - y).$

    From the question,

    $2x + 3y = 86 \quad..... (i)$\\
    $4x + y = 112 \quad...... (ii)$

    Subtracting (i) from (ii), we get

    $2x - 2y = 26$ or, $x - y = 13$
  \end{itemize}
\item
  \begin{itemize}
  \item[(a)] Let the two-digit no. be $10x + y$.\\
    Then, $(10x + y) - (10y + x) = 36$\\
    or, $9(x - y) = 36$\\
    or, $x - y = 4$
  \end{itemize}
\item
  \begin{itemize}
  \item[(d)] Suppose the two-digit number is $10x + y$\\
    Then, $10y + x = 20x + y/2$\\
    or $20y + 2x = 40x + y$ or, $y = 2x$
  \end{itemize}
\item
  \begin{itemize}
  \item[(b)] $9A^2 = 12A + 96 \Rightarrow 3A^2 - 4A - 32 = 0$\\[0.1cm]
    $\therefore~ A = \dfrac{4\pm \sqrt{16 + 384}}{6} = 4, -\dfrac{8}{3}$\\[0.1cm]
    $B^2 = 2B + 3 \Rightarrow B^2 - 2B - 3 = 0$\\[0.1cm]
    $\therefore~ B = \dfrac{2 \pm \sqrt{4 + 12}}{2} = 3, -1$\\[0.1cm]
    $\therefore~ 5A + 7B = 5 \times 4 + 7 \times 3 = 20 + 21 = 41$
  \end{itemize}
\item
  \begin{itemize}
  \item[(e)] Let the original number of sweets be $x$.\\
    According to the question,

    $\dfrac{x}{140} - \dfrac{x}{175} = 4$\\
    or, $175x - 140x = 4 \times 140 \times 175$\\
    or, $x = \dfrac{4 \times 140 \times 175}{35} = 2800$
  \end{itemize}
\item
  \begin{itemize}
  \item[(a)] Let the two-digit number be $10x + y$.

    $10x + y = 7(x + y) \Rightarrow x = 2y \qquad\qquad ...(i)$\\
    $10(x + 2) + y + 2 = 6(x + y + 4) + 4$\\
    or $10x + y + 22 = 6x + 6y + 28 \Rightarrow 4x - 5y = 6 \qquad  ...(ii)$

    Solving equations (i) and (ii), we get $x = 4$ and $y = 2$
  \end{itemize}
\item
  \begin{itemize}
  \item[(a)] Ratio of Ramani's saving in NSC and PPF = $3 : 2$

    His saving in PPF = $\dfrac{2}{5} \times 150000 = 60000$
  \end{itemize}
\item
  \begin{itemize}
  \item[(c)] Let $x$ be the first number and $y$ be the second number.

    $\dfrac{1}{5}x = \dfrac{5}{8}y \qquad\qquad \therefore~ \dfrac{x}{y} = \dfrac{25}{8} \quad ....(i)$\\
    $x + 35 = 4y \qquad\qquad$ or, $\dfrac{25}{8}y + 35 = 4y$\\
    $\therefore~ y = 40$
  \end{itemize}
\item
  \begin{itemize}
  \item[(e)] Let the original number be $10 x + y$\\
    $y = 2x + 1 \quad .....(i)$\\
    and $(10y + x) - (10x + y) = 10x + y - 1$\\
    or, $9y - 9x = 10x + y - 1$\\
    or, $19x - 8y = 1 \quad ...(ii)$

    Putting the value of (i) in equation (ii) we get,
    $19x - 8 (2x + 1) = 1$\\
    or, $19x - 16x - 8 = 1$\\
    or, $3x = 9$ or, $x = 3$\\
    So, $y = 2 \times 3 + 1 = 7$\\
    $\therefore~$ original number $= 10 \times 3 + 7 = 37$
  \end{itemize}
\item
  \begin{itemize}
  \item[(d)] \textbf{In case I:} Let the no. of children = $x$.

    Hence, total no. of notebooks distributed
    \begin{equation*}
      \dfrac{1}{8} x . x \quad\text{or}\quad \dfrac{x^2}{8}\tag*{.....(i)}
    \end{equation*}

    \textbf{In case II:} no. of children = $\dfrac{x}{2}$

    Now, the total no. of notebooks
    \begin{equation*}
      = 16 \times \dfrac{x}{2}\tag*{....(ii)}
    \end{equation*}

    Comparing (i) \& (ii), we get
    
    $\dfrac{x^2}{8} = 8x$\\
    or, $x = 64$\\
    Hence, total no. of notebooks

    $= \dfrac{64 \times 64}{8} = 512$
  \end{itemize}
\item
  \begin{itemize}
  \item[(a)] Let the positive integer be $x$.

    Now, $x^2 - 20x = 96$\\
    or, $x^2 - 20x - 96 = 0$\\
    or, $x^2 - 24x + 4x - 96 = 0$\\
    or, $x(x - 24) + 4(x - 24) = 0$\\
    0r, $(x - 24)(x + 4) = 0$\\
    or, $x = 24, -4$\\
    $\qquad x \neq -4$ because $x$ is a positive integer
  \end{itemize}
\item
  \begin{itemize}
  \item[(a)] Let $1/2$ of the no. $= 10x + y$ and the no. $= 10V + W$ From the given conditions, $W = x$ and $V = y - 1$

    Thus the no. $10(y - 1)+ x \quad .....(A)$

    $\therefore~ 2(10x + y) = 10(y - 1) + x \Rightarrow 8y - 19x = 10 \quad ....(i)$

    Again, from the question,

    $V + W = 7 \Rightarrow y - 1 + x = 7$\\
    $\therefore~ x + y = 8\quad ...(ii)$

    Solving equations (i) and (ii), we get $x = 2$ and $y = 6$.

    $\therefore~$ From equation ((A), Number

    $= 10(y - 1) + x = 52$
  \end{itemize}
\item
  \begin{itemize}
  \item[(e)] Suppose the two-digit number be $10x + y.$

    Then we have been given

    $10x + y - (10y + x) = 9$\\
    $\Rightarrow~ 9x - 9y = 9$\\
    $\Rightarrow~ x - y = 1$\\
    Hence, the required difference = 1

    Note that if the difference between a two-digit number and the number obtained by interchanging the digits is D, then the difference between the two digits of the number $= \dfrac{D}{9}$
  \end{itemize}
\item
  \begin{itemize}
  \item[(c)] Suppose the number is N.

    Then $N - \dfrac{3}{5}N = 50$\\
    $\Rightarrow~ \dfrac{2N}{5} = 50, \quad\therefore\quad N = \dfrac{50 \times 5}{2} = 125$
  \end{itemize}
\item
  \begin{itemize}
  \item[(d)] Let the original fraction be $\dfrac{x}{y}$.

    Then $\dfrac{x + 2}{y + 1} = \dfrac{5}{8}$ or, $8x - 5y = -11\quad......(i)$\\
    Again, $\dfrac{x + 3}{y + 1} = \dfrac{3}{4}$ or, $4x - 3y = -9\quad.....(ii)$\\
    Solving, (i) and (ii) we get $x = 3$ and $y = 7$\\
    $\therefore~$ fraction = $\dfrac{3}{7}$
  \end{itemize}
\item
  \begin{itemize}
  \item[(d)] On solving equation we get

    $x = 7$, $y = 4$, $z = 11$
  \end{itemize}
\item
  \begin{itemize}
  \item[(e)] Let the number = $x$\\
    Then, $x^2 + x = 182$\\
    or, $x^2 + x - 182 = 0$\\
    or, $x + 14x - 13x - 182 = 0$\\
    or, $x(x + 14) - 13(x + 14) = 0$\\
    or, $(x - 13)(x + 14) = 0$\\
    or, $x = 13$ (negative value is neglected)
  \end{itemize}
\item
  \begin{itemize}
  \item[(c)] Let the no. of balls = $b$\\
    Rate = $\dfrac{450}{b}$\\
    $(b + 5) \left(\dfrac{450}{b} - 15\right) = 450$\\
    or, $450 - 15b + \dfrac{2250}{b} - 75 = 450$\\
    or, $b^2 + 5b - 150 = 0$\\
    or, $(b + 15)(b - 10) = 0$\\
    or, $b = 10$ (Neglecting negative value)
    \end{itemize}
\item
  \begin{itemize}
  \item[(b)] $n^4 - 10n^3 + 36n^2 - 49n + 24$\\
    $1 - 10 + 36 - 49 + 24 = 2$
  \end{itemize}
\item
  \begin{itemize}
  \item[(b)] Let $`x'$ be the total number of students in college
    \begin{align*}
      & x -\left[\dfrac{12x}{100} + \dfrac{3x}{4} + \dfrac{10x}{100} \right] = 15\\
      & x - \left[\dfrac{48x + 300x + 40x}{400}\right] = 15 \quad\therefore~ x = 500
      \end{align*}
  \end{itemize}
\item
  \begin{itemize}
  \item[(a)] Let the first, second, third and fourth numbers be a, b, c and d respectively.

    According to the question,
    \begin{equation*}
      a + b + c + d = 64\tag*{....(i)}
    \end{equation*}
    and $a + 3 = b - 3 = 3c = \dfrac{d}{3}$

    \begin{equation*}
      \text{i.e.,}\quad a + 3 = b - 3 \Rightarrow b = a + 6\tag*{...(ii)}
    \end{equation*}
    Also, $c = \dfrac{a + 3}{3}$ and $d = 3 (a + 3)$

    Solving the above eqns, we get

    $a = 9$, $b = 15$, $c = 4$ and $d = 36$

    $\therefore~$ Difference between the largest and the smallest numbers = $36 - 4 = 32$
  \end{itemize}
\item
  \begin{itemize}
  \item[(d)] Let the no. of boys and girls in the classroom is $x$ each. From the question,
    $2(x - 8) = x \qquad\qquad \therefore~ x = 16$

    $\therefore~$ Number of boys and girls = $16 + 16 = 32$
  \end{itemize}
\item
  \begin{itemize}
  \item[(d)] $x - y = \dfrac{1}{9} \{(10x + y) - (10y + x)\} = \dfrac{1}{9}$

    $(9x - 9y) = x - y$
  \end{itemize}
\item
  \begin{itemize}
\item[(b)] By trial and error method.
  \end{itemize}
\item
  \begin{itemize}
  \item[(e)] $2x + y = 15 \Rightarrow y = 15 - 2x ..... (i)$\\
    $2y + z = 25 \Rightarrow 2(15 - 2x) + z = 25$ [from (i)]\\
    $\Rightarrow 4x - z = 5  ..... (ii)$\\
    and $2z + x = 26  ..... (iii)$\\
    Combining equation (ii) and (iii), we get $z = 11$
  \end{itemize}
\item
  \begin{itemize}
  \item[(d)] $P (P - 3) < 4 (P - 3);$\\
    $P (P - 3) -4 (P - 3) < 0$\\
    $(P - 3)(P - 4) < 0$\\
    This means that when

    $(P - 3) > 0$ then $(P - 4) < 0 \qquad ......(i)$\\
    or, when $(P - 3) < 0$ then $(P - 4) > 0 \qquad ....(ii)$

    From (i),
    $P > 3$ and $P < 4$\\
    $\therefore~ 3 < P < 4$
  \end{itemize}
\item
  \begin{itemize}
  \item[(d)] $P + R + 2Q = 59;$\\
    $Q + R + 3P = 68$\\
    and $P + 3(Q + R) = 108$\\
    Solving the above two equations, we get $P = 12$years.
  \end{itemize}
\item
  \begin{itemize}
  \item[(a)] Let the ages of Harish and Seema be $x$ and $y$ respectively. According to the question,

    $x . y = 240 \qquad .....(i)$\\
    $2y - x = 4  \qquad .....(ii)$\\
    Solving equations (i) and (ii), we get $y = 12$ years
  \end{itemize}
\item
  \begin{itemize}
  \item[(e)] $5P9 + 3R7 + 2Q8 = 1114$

    For the maximum value of $Q$, the values of $P$ and $R$ should be the minimum, i.e. zero each.

    Now, $509 + 307 + 2Q8 = 1114$\\
    or, $816 + 2Q8 = 1114$\\
    or, $2Q8 = (1114 - 816 =) 298$\\
    So, the reqd value of $Q$ is 9.
  \end{itemize}
\item
  \begin{itemize}
  \item[(e)] $\dfrac{2}{5} \times \dfrac{1}{4} \times \dfrac{3}{7} \times x = 15$\\
    $\therefore~ \dfrac{x}{2} = \dfrac{5 \times 7 \times 2 \times 5}{2} = 25 \times 7 = 175$
  \end{itemize}
\item
  \begin{itemize}
  \item[(d)] Let, the two-digit no. be $xy$, i.e. $10x + y$ then,

    $x + y = \dfrac{1}{11} \left[(10x + y) + (10y + x)\right] = x + y$\\
    Thus we see that the difference of $x$ and $y$ can't be determined.

    Hence, the answer is data inadequate.
  \end{itemize}
\item
  \begin{itemize}
  \item[(c)] Let the fraction be $\dfrac{x}{y}$ then

    $\dfrac{x + 1}{y + 2} = \dfrac{2}{3}$ or, $3x + 3 = 2y + 4$ or, $3x = 2y + 1 ....I$\\
    Also, we have

    $\dfrac{x + 5}{y + 1} = \dfrac{5}{4}$\\
    or, $4x + 20 = 5y + 5$\\
    or, $4x = 5y - 15 .....II$

    From I and II, we get

    $\dfrac{2y + 1}{3} = \dfrac{5y - 15}{4}$\\
    or, $8y + 4 = 15y - 45$\\
    $\therefore~ y = 7$ and $x = \dfrac{2y + 1}{3} = \dfrac{2 \times 7 + 1}{3} = \dfrac{15}{3} = 5$\\
    $\therefore~$ Reqd original fraction = $\dfrac{x}{y} = \dfrac{5}{7}$
  \end{itemize}
\item
  \begin{itemize}
  \item[(d)] Let the no. be $10x + y$\\
    then $y = x + 2$ or $y - x = 2 \quad ......(i)$\\
    $(10y + x) - (10x + y) = 18$\\
    or, $9y - 9x = 18$\\
    $y - x = 2$\\
    From eq. (i) and (ii) we can't get any conclusion.
  \end{itemize}
\item
  \begin{itemize}
  \item[(d)] $2x + y = 17 \Rightarrow y = 17 - 2x \quad .....(i)$\\
    $y + 2z = 15 \Rightarrow 17 - 2x + 2z = 15$\\
    $\Rightarrow 2x - 2z = 2 \Rightarrow x - z = 1 \quad ....(ii)$\\
    and $x + z = 9 \qquad .....(iii)$

    Solving equations (i) and (ii), we get

    $x = 5$, $z = 4$\\
    $\therefore~ y = 17 - 2x = 17 - 10 = 7$\\
    $4x + 3y + z = 4 \times 5 + 3 \times 7 + 4$\\
    $= 20 + 21 + 4 = 45$
  \end{itemize}
\item
  \begin{itemize}
  \item[(c)] Let the numerator and denominator be $x$ and $y$ respectively. Then $\dfrac{x + 2}{y + 3} = \dfrac{7}{9}$\\
    or, $9(x + 2) = 7(y + 3)$\\
    or, $9x - 7y = 3 \qquad ......(i)$\\
    $\dfrac{x - 1}{y - 1} = \dfrac{4}{5}$\\
    $\Rightarrow 5x - 4y = 1 \qquad .....(ii)$

    Solving (i) and (ii), we get

    $x = 5$, $y = 6$\\
    Reqd fraction $= 5/6$
  \end{itemize}
\item
  \begin{itemize}
  \item[(a)] $3n^2 - 18n + 24 = 0$\\
    or, $n^2 - 6n + 8 = 0$ or, $(n - 4)(n - 2) = 0$\\
    $\therefore~ n = 4, 2$\\
    $\therefore~ n > 4$
  \end{itemize}
\item
  \begin{itemize}
  \item[(d)] $R - M = 7000$ and $S - M = 3000$

    Here, $S + M + R$ can be found only when one more equation in terms of $S$ and $R$ is givenb. Therefore, Can't be determined is the correct answer.
  \end{itemize}
\item
  \begin{itemize}
  \item[(c)] Let the no. be $N$.

    Now, $\dfrac{3N}{5} - \dfrac{N}{2} = 30$ or, $\dfrac{N}{10} = 30$\\
    or, $n = 300$\\
    $80\%$ of $N = 240$\\
  \end{itemize}
\item
  \begin{itemize}
  \item[(b)] Let the two-digit no. be $10x + y$.\\
    Then, $(10x + y)-(10y + x) = 27$\\
    or, $x - y = 3$
  \end{itemize}
\item
  \begin{itemize}
  \item[(a)] $F + S = 4S$\\
    or, $F = 3S \Rightarrow F : S = 3: 1$\\
    The ages of father and son = 56 years

    $\therefore~$ Son's age = $\dfrac{1}{4} \times 56 = 14$ years
  \end{itemize}
\item
  \begin{itemize}
  \item[(d)] Let the number be $x$.\\
    $\therefore~ \dfrac{2}{5} \times \dfrac{1}{4} \times \dfrac{5}{8} \times x = 6$\\
    $\therefore~ x = \dfrac{6 \times 5 \times 4 \times 8}{2 \times 1 \times 5} \times \dfrac{1}{2} = 48$
  \end{itemize}
\item
  \begin{itemize}
  \item[(a)] Let the two-digit number be $10x + y$.\\
    Then $x + y = \dfrac{1}{5}(10x + y - 10y - x)$\\
    or, $x + y = \dfrac{9}{5} (x - y)$\\
    or, $4x - 14y = 0 \Rightarrow \dfrac{x}{y} = \dfrac{7}{2}$

    Using componendo \& dividendo, we have,
    $\dfrac{x + y}{x - y} = \dfrac{7 + 2}{7 - 2} = \dfrac{9}{5}$ i.e., $x - y = 5k$

    Here $k$ has the only possible value, $k = 1$. Because the difference of two single-digit numbers will always be of a single digit.
  \end{itemize}
\item
  \begin{itemize}
  \item[(c)] $J = \dfrac{2}{5}A$, $P = \dfrac{1}{4} \times \dfrac{2}{5}A = \dfrac{1}{10}A$\\
    $\dfrac{1}{10}A - 200 = 600 \quad\therefore~ \dfrac{1}{10}A = 800$\\
    $A =$ \rupee~8,000
  \end{itemize}
\item
  \begin{itemize}
  \item[(a)] For $Q$ to be maximum. $P$ and $R$ will also maximum, i.e., $P = R = 9$.\\
    So, by putting the value of $P$ and $R$ in

    $5P9 - 7Q2 + 9R6 = 823$, we get $Q = 7$
  \end{itemize}
\item
  \begin{itemize}
  \item[(d)] Let the two-digit no. be $10x + y$.\\
    According to question,

    $(10x + y)-(10y + x) = 54$\\
    $9x - 9y = 54 \qquad x - y = 6$
  \end{itemize}
\item
  \begin{itemize}
  \item[(c)] Let the two numbers be $x$ and $y$.\\
    $\therefore~ xy = 192, x + y = 28 \quad .....(i)$\\
    $\therefore~ (x - y)^2 = (x + y)^2 - 4xy = 784 - 768 = 16$\\
    $\therefore~ x - y = 4 \quad .....(ii)$\\
    Combining (i) and (ii), $x = 16$, and $y = 12$.
  \end{itemize}
\item
  \begin{itemize}
  \item[(e)] Let the present ages of Mr. Ramesh and his son be $x$ and $y$ respectively.

    $\therefore~ x = 4y$ and $(x + 10) = 2(y + 10)$ Solving the above two equations, we get $x = 20$ years and $y = 5$ years
  \end{itemize}
\item
  \begin{itemize}
  \item[(e)] Let the total number of discs of 2 kg and 5 kg be $`a'$ and $`b'$ respectively.

    Then, $a + b = 21$ and $5b = 2a$

    Solving the above two equations, we get $a = 15$, $b = 6$

    $\therefore~$ Weight of all discs together

    $= 15 \times 2 + 6 \times 5 = 60kg$
  \end{itemize}
\item
  \begin{itemize}
  \item[(a)] No. of 10-year periods = $6$\\
    $2 \times 2 \times 2 \times 2 \times 2 \times 2 \times B = 64 B$
  \end{itemize}
\item
  \begin{itemize}
  \item[(b)] Let the middle no. $= x$\\
    $(x - 2) + x + (x + 2) = \dfrac{176}{4} - 14$ or,\\
    $3x = \dfrac{120}{4}$ or, $x = 10$
  \end{itemize}
\item
  \begin{itemize}
\item[(d)] number of tables and chairs and tripled, so price is $12,090 \times 3 = 36,000$
  \end{itemize}
\item
  \begin{itemize}
\item[(a)] Price of 39 pencils = $\dfrac{4263.05}{253} \times 30 \approx$ \rupee~650
  \end{itemize}
\item
  \begin{itemize}
  \item[(c)] Percentage of marks obtained by Sundari in first and second papers is $40\%$ and $80\%$ respectively. Percentage of marks obtained by Kusu in first and second papers is $50\%$ and $90\%$, respectively. Percentage of marks obtained by Jyoti in first and second papers is $30\%$ and $90\%$ respectively.

    From the above, we see that Jyoti's progress is maximum.
  \end{itemize}
\item
  \begin{itemize}
  \item[(b)]
    \begin{itemize}
    \item[{\bf I.}] $2x^2 + 5x + 1 = x^2 + 2x - 1$\\
      $x^2 + 3x + 2 = 0$\\
      $x^2 + 2x + x + 2 = 0$\\
      $x(x + 2) + 1 (x + 2) = 0$\\
      $(x + 2)(x + 1) = 0$\\
      $x = -2, -1$

    \item[{\bf II.}] $2y^2 - 8y + 1 = -1$\\
      $2y^2 - 8y + 2 = 0$\\
      $y^2 - 4y + 1 = 0$\\[0.2cm]
      $\dfrac{+4 \pm \sqrt{16 - 4 \times 1 \times 1}}{2 \times 1}$\\[0.2cm]
      $= 2 \pm \sqrt{12} = 2 \pm 2\sqrt{3}$\\
      Hence, $y > x$
      \end{itemize}
    \end{itemize}
\item
  \begin{itemize}
  \item[(b)]
    \begin{itemize}
    \item[{\bf I.}] $x^2 + 2x - 1 = 2$\\
      $x^2 + 2x - 3 = 0$\\
      $x + 3x - x - 3 = 0$\\
      $x(x + 3) - 1(x + 3) = 0$\\
      $(x + 3)(x - 1) = 0$\\
      $x = -3, 1$

    \item[{\bf II.}] $2y^2 - 12y + 18 = 0$\\
      $y^2 - 6y + 9 = 0$\\
      $(y - 3)^2 = 0$\\
      $y = 3, 3$\\
      Hence, $y > x$
      \end{itemize}
  \end{itemize}
\item
  \begin{itemize}
  \item[(b)]
    \begin{itemize}
    \item[{\bf I.}] $4x^2 - 24x + 20 = 0$\\
      $x^2 - 6x + 5 = 0$\\
      $x^2 - 5x - x + 5 = 0$\\
      $x(x - 5) -1 (x - 5) = 0$\\
      $(x - 5)(x - 1) = 0$\\
      $x = 5, 1$

    \item[{\bf II.}] $y^2 - 13y + 42 = 0$\\
      $y^2 - 7y - 6y + 42 = 0$\\
      $y(y - 7) -6 (y - 7) = 0$\\
      $(y - 7)(y - 6) = 0$\\
      $y = 7, 6$\\
      Hence, $y > x.$
      \end{itemize}
  \end{itemize}
\item
  \begin{itemize}
  \item[(a)]
    \begin{itemize}
    \item[{\bf I.}] $2y^2 + 3y -5 = 0$\\
      $2y^2 + 5y - 2y - 5 = 0$\\
      $y(2y + 5) -1 (2y + 5) = 0$\\
      $(2y + 5)(y - 1) = 0$\\[0.2cm]
      $y = \dfrac{-5}{2}, 1$

    \item[{\bf II.}] $x^2 - 3x = 2x - 6$\\
      $x^2 - 5x + 6 = 0$\\
      $x^2 - 3x - 2x + 6 = 0$\\
      $x(x - 3) -2(x - 3) = 0$\\
      $(x - 3)(x - 2) = 0$\\
      $x = 3, 2$\\
      Hence, $x > y$
      \end{itemize}
  \end{itemize}
\item
  \begin{itemize}
  \item[(e)]
    \begin{itemize}
    \item[{\bf I.}] $6x^2 + 14x = 12$\\
      $3x^2 + 7x - 6 = 0$\\
      $(x + 3)(3x - 2) = 0$\\[0.2cm]
      $x = -3, \dfrac{2}{3}$

    \item[{\bf II.}] $1 + 2y^2 = \dfrac{17}{6}y$\\
      $12y^2 - 17y + 6 = 0$\\
      $12y^2 - 8y - 9y + 6 = 0$\\
      $4y (3y - 2) -3 (3y - 2) = 0$\\
      $(3y - 2)(4y - 3) = 0$\\[0.2cm]
      $y = \dfrac{2}{3}$, $\dfrac{3}{4}$\\[0.2cm]
      Hence, $x \leq y$
      \end{itemize}
  \end{itemize}
\item
  \begin{itemize}
  \item[(b)]
    \begin{itemize}
    \item[{\bf I.}] $12x^2 + 11x + 12 = 10x^2 + 22x$\\
      $2x^2 - 11x + 12 = 0$\\
      $2x^2 - 8x - 3x + 12 = 0$\\
      $(x - 4)(2x - 3) = 0$\\
      $x = 4$, $x = 3/2$

    \item[{\bf II.}] $13y^2 - 18y + 3 = 9y^2 - 10y$\\
      $4y^2 - 8y + 3 = 0$\\
      $4y^2 - 6y - 2y + 3 = 0$\\
      $(2y - 3)(2y - 1) = 0$\\[0.2cm]
      $y = \dfrac{3}{2}$, $\dfrac{1}{2}$\\[0.2cm]
      $\therefore~ x \geq y$
      \end{itemize}
  \end{itemize}
\item
  \begin{itemize}
  \item[(c)]
\begin{itemize}
\item[{\bf I.}]$\dfrac{18}{x^2} + \dfrac{6}{x} - \dfrac{12}{x^2} = \dfrac{8}{x^2}$\\[0.2cm]
    $\Rightarrow~ \dfrac{18 + 6x - 12}{x^2} = \dfrac{8}{x^2} \Rightarrow 6x + 6 = 8$\\[0.2cm]
  $\therefore~ x = \dfrac{2}{6} = 0.33$

\item[{\bf II.}] $y^3 + 9.68 + 5.64 = 16.95$\\
  $\Rightarrow~ y^3 = 16.95 - 15.32$\\
  $\Rightarrow~ y^3 = 1.63 = y = \sqrt[3]{1.63}$
\end{itemize}
  \end{itemize}
\item
  \begin{itemize}
  \item[(a)]
    \begin{itemize}
    \item[{\bf I.}] $35x + 70 = 0$\\[0.2cm]
      $\therefore~ x = \dfrac{-70}{35} = -2$

    \item[{\bf II.}] $(81)^{1/4}y + (343)^{1/3} = 0$\\
      $\Rightarrow~ 3y + 7 = 0 \Rightarrow 3y = -7$\\
      $\therefore~ y = -\dfrac{7}{3} = -223 \Rightarrow x > y$
     \end{itemize}
  \end{itemize}
\item
  \begin{itemize}
  \item[(a)]
    \begin{itemize}
    \item[{\bf I.}] $\dfrac{(2)^5 + (11)^{3}}{6} = x^3$[0.2cm]
      $\Rightarrow~ \dfrac{32 + 1331}{6} = x^3 \Rightarrow \dfrac{1363}{6} = x^3$\\
      $\therefore~ x^3 = 227.167$

    \item[{\bf II.}] $4y^3 = \dfrac{-589}{4} + 5y^3 \Rightarrow \dfrac{589}{4} = y^3$\\[0.2cm]
      $\therefore~ y^3 = 147.25 \quad\therefore~ x > y$
      \end{itemize}
  \end{itemize}
\item
  \begin{itemize}
  \item[(d)]
    \begin{itemize}
    \item[{\bf I.}] $x^{7/5} \div 0 = 169^1 x^{3/5}$\\[0.2cm]
      $\dfrac{x^{7/5}}{9} = \dfrac{169}{x^{3/5}}$\\[0.2cm]
      $\Rightarrow~ x^{10/5} = 9 \times 169 \Rightarrow x^2 = 9 \times 169$\\
      $x = \pm (3 \times 13) = \pm 39$

    \item[{\bf II.}] $y^{1/4} \times y^{1/4} \times 7 = \dfrac{273}{y^{1/2}}$\\[0.2cm]
      $y = \dfrac{273}{7} = 39$[0.2cm]\\
      $x \leq y$
      \end{itemize}
    \end{itemize}
\end{enumerate}
\end{multicols}

