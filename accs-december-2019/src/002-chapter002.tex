\chapter{Who can give us a Faraday’s cage in today’s digital age to safeguard our privacy?}
\vskip -15pt


\begin{center}
{\large\uppercase{Kumar Ankit}}\footnote{Kumar Ankit is currently based at New Delhi and is employed as Head - Legal and Regulatory Affairs \& Company Secretary with Vedanta Limited (Iron Ore and Ports division). He graduated in law from Gujarat National Law University, Gandhinagar and is a member of Association of Certified Fraud Examiner, Texas, US. He is immensely involved in regulatory affairs with the government, and has been a regular speakers on data governance issues at various forum. He was also awarded a fellowship from National Internet Exchange of India for working on internet governance in year 2010-11. The views expressed here are his personal views.}, 


\end{center}

\vskip 2.3cm



\vfill
\newpage

\begin{multicols}{2}

\section{INTRODUCTION}

We all grew up learning from experiments and observations, making sure that we learn well what works optimally for us. It is also equally true, that man only learn and understand things what he or she is exposed to. Like in India, we are more familiar with cricket and football than lawn-tennis or golf, or for that matter we all learn our mother tongue by getting exposed to it. For a moment consider that there is only limited exposure to only one game throughout the childhood, would I ever be able to pick up interest in anything else. Similar principle of learning can be easily applicable on food, religion, language and other habits, which define a person in a long run. Even children pick up a language of their parents or kids with whom they socialize at schools with ease. The question still remains is how do we weave a new habit about without having an exposure? How do we know what we don’t know? 

Now imagine a situation where based on our browsing history using cookies and trackers, we were auto-directed to a selected few labels on our streaming devices only. Would you ever have discovered that there were so many choices in music yet to be explored, or would your pre-conditioned mind have accustomed itself to make a choice that only music of a popular label is good enough? 

I would like to imagine a situation where our actions, emotions and urge to know (using search strings) is not used against us by directing us to content which fills up the void of our search with content that has been tailored fit to make me feel happy. In the increasing digital market of trans-boundary content, goods and services the choices for us are often made by a super intelligent, biased algorithm, backed by immense computing power specifically designed to detect tastes of sports, food, music, living lifestyle, depending on the way we have used our smartphones, connected devices, location shares, search strings and photographs. Question which I would like to ask my readers is whether we exercise the choices we actually want to exercise or do we just go by the flow of algorithm to let it determine what we should experience? This is the question for today which has implications on how the global economy sustains and caters to its users. 

The modern discussion over privacy has been resultant with the increasing use of data posting by gullible users (70\% of them are below age of 35\footnote{\url{https://www2.deloitte.com/content/dam/Deloitte/in/Documents/technology-media-telecommunications/in-tmt-rise-of-on-demand-content.pdf}}) using smartphones\footnote{\url {https://webobjects.cdw.com/webobjects/media/pdf/Solutions/Networking/White-Paper-Cisco-The-Zettabyte-Era-Trends-and-Analysis.pdf }}, social media, digital records, and knowing individuals as a customer. Given that all readers are using devices connected to a connected network, it is estimated that by 2020, 3.4 devices per captia will be connected to the online network collectively generating and handling $\sim$2.3 ZB\footnote{1 ZB = 1billion TBs= 1trillion GBs. Each ZB is estimated at \$100 Trillion.} of data\footnote{Supra Note }. It is inevitable that the future beholds that the generation of data will keep increasing in geometric progression with better access to connectivity and convenience. 

It is also equally true that some of the world largest company like Alphabet generate around 85\% of their global revenues ($\sim$\$ 136 Billions) from advertisement using search platforms like google, content aggregators like YouTube, google maps and hardware. Similarly, Facebook revenue from advertisement has risen sharply. This has virtually created a duopoly in the digital media industry. With such vast amount of data getting generated, it is important that ethical and acceptable practices are followed by the aggregators and third party data handlers, and it allows the data generators to be acquainted with their original choices and exercise it, and regulated practices for those who are constantly monitoring the data and purpose for which the data is retained and used. As more and more activities move online, where people will research, shop and find product recommendations, the importance of privacy and data protection will increasingly be recognized especially when the current system for data protection is highly fragmented, with divergent regulatory approaches for global and national practices. 

I would like you to visit a website (\url{haveibeenpwned.com}) to understand if any of your email, social media accounts has ever been breached and if your password has been shared on is appearing on the web.

\section{EXERCISE OF FREE WILL WITH CHOICES}

Choice is an interesting metaphor for will. As human beings we are all entrusted a birth right of having a free will. We are all aware that information is power, but it also now well established that information is also money. However, the way the information is gathered and used is a matter of scrutiny. The information about choice of user is generated by web analytics and results in more online advertisement getting generated by social media optimization, or banner marketing, search engine advertisement, marketing emails, online video or local online advertisement. It is equally important that people react to the online advertisements much more than the offline or yellow classified book days. Digital videos are often embedded with advertisement, which our sub-conscious minds picks up subtly as the preferable option in case an option of choice arises. Every website is beaming with native advertisement, in-feed advertisements in such ways that it is hard to differentiate between the original content and the advertised ones. The data generated by the internet users along with their preferences and behavior is the Holy Grail for most advertisers. However the reality is that Google and Facebook have more magnitude, and more ways to analyze it. 

I would people to explore aspects of their individuality to exercise free choice. Users of wearable devices and social media networks may not conceive of themselves as having volunteered data but their activities of use and engagement result in the generation of vast amounts of data about individual lifestyles, choices and preferences. We need to our choices are future ready in view of technological advancements and social developments and do we need Privacy Enhancement Technologies (PETs) which allow online users to protect the privacy of their Personally Identifiable Information (PII) provided to and handled by Data Governors (DGs). Further, the balance between privacy and security measures is turning to be tilting towards the security measures, but at the same time should not have anything to do with the consumer behavior. 

Privacy and data protection require that information about individuals should not be automatically made available to other individuals and organisations. The dangers to privacy in an age of information can originate not only from the State but from non-State actors as well. However, what happens when you have consented to provide data to an aggregator but in turn it has been handed over to a third party processor with whom you do not have any privy of contract. In case of a data breach who is the one responsible, you, the aggregator, third party or the unknown who got access to the data due to actions or inaction by any of the above. The yardstick for 

determining the scope of data control should be that each person must be able to exercise a substantial degree of control over that data and its use. Data protection is legal safeguard to prevent misuse of information. The challenge is the way the information is getting generated. It is a known fact that penetration of smartphones has given unheard access of information to all. The pros of getting information are weighed against the misuse of information getting collated. Data mining processes together with knowledge discovery can be combined to create facts about individuals\footnote{R. Pildes, Why rights are not trumps: social meanings, expressive harms, and constitutionalism, 27(2) The Journal of Legal Studies (1998) at pp. 725-763}.

An example is worth sharing with this audience. My one and half year old kid who barely knows how to communicate using words and translates mumbled information by using facial expressions, laughs and cries is fond of popular online content streaming on his smart TV. He is quite happy when we play him a cartoon from a favourite channel. However, similar cartoon in a different channel does not make him happy. The issue is that he has been so accustomed to the similar make video being played in loop by the algorithm that he is not able to exercise an emotion to anything else which is available on the YouTube. As I stated earlier, he made a choice based on the limited amount of information which has been fed to him by the algorithm. If the kids can be tuned into liking or disliking things by mere repetition of similar contents based on what you have watched previously, it will lead a market domination of someone who can aptly pay for maintaining that dominion based on better probabilistic abilities. The exercise of choice is undermined heavily by not knowing what we don’t know.

\section{GENESIS AND ORIGIN OF\\ PRIVACY}

In 1891, the American lawyers Samuel Warren and Louis Brandeis described the right to privacy in a famous article: it is the right to be let alone\footnote{ Holvast J. (2009) History of Privacy. In: Matyáš V., Fischer-Hübner S., Cvrček D., Švenda P. (eds) The Future of Identity in the Information Society. Privacy and Identity 2008. IFIP Advances in Information and Communication Technology, vol 298. Springer, Berlin, Heidelberg available at \url{https://doi.org/10.1007/978-3-642-03315-5_2}}. Technology did not rule its subject then as now, but the fundamental questions still remains that do we have a right to be let alone. Privacy in modern times is the most valuable substance known to man and a man should do all to protect it. A brief history of the major developments in the area of privacy have been depicted for understanding the nuances of the societal ask and sovereign reaction to such asks.

\begin{itemize}

\item[{\bf 1)}] \textbf{1950 - Universal Declaration of Human Rights}

When United Nations proclaimed the draft of the Universal Declaration of Human Rights (‘UDHR’) in a general assembly on December 10, 1948 at Paris, little was known that the Article 12 of the UDHR\footnote{Amended in 1950.} would lay foundation stone for a regime on which the entire current generation would depend for their fundamental right to make and exercise choices. Article 12\footnote{Article 12 of UDHR states “No one shall be subjected to arbitrary interference with his privacy, family, home or correspondence, nor to attacks upon his honour and reputation. Everyone has the right to the protection of the law against such interference or attacks.”  \url{https://www.un.org/en/universal-declaration-human-rights/}} casted obligation / duty on the member states to guarantee freedom of privacy to its subject.

\item[{\bf 2)}] \textbf{1980 - OECD Privacy Guidelines }

Thereafter in 1980, the OECD Privacy Guidelines of 1980\footnote{ Accessed from \url{https://www.oecd.org/internet/ieconomy/oecdguidelinesontheprotectionofprivacyandtransborderflowsofpersonaldata.htm} on 18.11.19 at 14:33 hours.} (Guidelines) for the first time established an agreed set of privacy principles. These guidelines were drafted under the chairmanship of The Hon. Mr. Justice M.D. Kirby, Chairman of the Australian Law Reform Commission, and for the first time they recognized the fact that participating countries should have a common interest in protecting privacy and individual liberties; and processing and trans-border flows of personal data requires the development of compatible rules and practices which do not hinder flows or create unjustified obstacles ; and that is it inevitable to avoid the free flow trans-border transfer of data as it contributes significantly to the economic and social development of the countries. These Guidelines laid principles which prescribed that the collection of personal data should be relevant and limited to the purpose of collection and obtained only through lawful and fair means with the knowledge and consent, and that such data should be protected by reasonable security safeguards to avoid unauthorized access.

\item[{\bf 3)}] \textbf{2005 - APEC Privacy Framework \& the Cross-Border Privacy Enforcement Arrangement (CPEA) }

Asia Pacific Economic Cooperation (APEC)\footnote{\url{https://www.apec.org/Publications/2005/12/APEC-Privacy-Framework}} Privacy Framework is a set of principles and implementation guidelines that were created in order to establish effective privacy protections that avoid barriers to information flows, and ensure continued trade and economic growth. The idea was to have a regime for effective information privacy protection and the free flow of information in the Asia-Pacific region for improving consumer confidence and ensuring the growth of electronic commerce. CPEA creates a framework for the voluntary sharing of information and provision of assistance for information privacy enforcement related activities. 

\item[{\bf 4)}]\textbf{2012 - Charter of Fundamental Rights of the European Union}

The Charter of Fundamental Rights of the European Union (CFR-EU)\footnote{2012/C 326/02 accessed from \url{https://eur-lex.europa.eu/legal-content/EN/TXT/?uri=CELEX:12012P/TXT}} enlists the fundamental rights of every person living in the EU region. It was introduced to bring consistency and clarity to the rights established at different times and in different ways in individual EU Member State\footnote{\url{https://www.europarl.europa.eu/charter/pdf/text\_en.pdf }}. CFR EU grants everyone the right to respect for his or her private life and communications in its Art. 7. Article 8 provides for protection of personal data where everyone has the right to the protection of personal data concerning, and such data must be processed fairly for specified purposes and on the basis of the consent of the person concerned or some other legitimate basis laid down by law. Everyone has the right of access to data which has been collected concerning him or her, and the right to have it rectified. Compliance with these rules shall be subject to control by an independent authority. It is now understood by all that the data protection is directly related to trades in digital economy, and insufficient protection can have significant effect of reducing consumer protection. At the same time overtly stringent protection can restrict business. 

\item[{\bf 5)}] \textbf{1995-2016 General Data Protection Regulation (GDPR)}

The European Data Protection Directive is created, reflecting technological advances and introducing new terms including processing, sensitive personal data and consent, among others. In 2002, the EU adopts the Directive on Privacy and Electronic Communications. In 2014, a ruling by the Court of Justice of the EU finds that European law gives people the right to ask search engines like Google to remove results for queries that include their name. The concept becomes known as “the right to be forgotten”. The General Data Protection Regulation (GDPR) is approved by the EU parliament after 4 years of discussions.

\end{itemize}

\section{THE INDIAN APPROACH TO\\ PRIVACY}

A recent incident of me getting calls from an agency recruiting insurance agents is also worth mentioning. It is quite known that data is generated first hand by us and then sold in the open market as a commodity. You would have often visited malls, shopping places, office spaces, places of interest where your information is often required to be entered in registers. The data gatherers love this sweet spot of data mines where the control is relatively easier for compiling information. In my case, I had visited an online testing center for an examination, and voila I have received atleast 10 calls from the people asking me to join them as insurance agents. When enquired upon details regarding source of my mobile number, the callers were easy to point that they have received this number from the examination center. Data like my name, address, telephone number, profession, family, choices, habits etc. are often available at various offline places like schools, colleges, banks, directories, surveys and online on various websites, mobile applications, smart devices. Passing of such information to interested parties can lead to intrusion in privacy like incessant marketing calls. It is important to underscore that privacy is not lost or surrendered merely because the individual is in a public place.

While there may not be tangible damage incurred in getting your name on a calling list and getting calls, the data if used in aggregate with other available online and offline information, may limit your choices of choosing between options, as it creates a complete alter-ego of yourself as an artificial person. This profiling can make, break or manipulate any market condition, or provide valuable insights on election strategies, or make hidden affairs make available for sale online. I believe that privacy attaches to the person since it is an essential facet of the dignity of the human being.

The framers of the Constitution of India did not engrave a right to privacy as a part of right to ‘life’ and ‘personal liberty’ under Article 21 ‘right to life’ which was guaranteed as a fundamental right to its citizens but Supreme Court has interpreted that Right to Privacy is very integral part of the ‘right to life’.

\begin{itemize}

\item[{\bf 1)}] \textbf{Right to Choose}

Article 19(1)(a) of the Indian Constitution guarantees to its citizens a right to freedom of speech and expression. However in Sakal Papers\footnote{Sakal Papers (P) Ltd. And Others (In Petition No. 331 Of 60) And v. 2. B.N Sarpotdar And Another (In Petitions Nos. 67 And 68 Of 61) Supreme Court Of India (25 Sep, 1961) 1962 SCR 3 842}  the editors had challenged Newspaper (Price and Page) Act, 1956, and the Daily Newspaper (Price and Page) Order, 1960 stating that the state cannot direct content or determine the pricing of a newspaper. A 5 judge constitutional bench had agreed that acts designed to curtail freedom of expression are unconstitutional. I am mentioning this case here, as in the current age of technology advancements, it is not states but vested parties who have powers to curtain freedom of expression including freedom to exercise choice. Privacy recognises the autonomy of the individual and the right of every person to make essential choices which affect the course of life.

\item[{\bf 2)}] \textbf{Right to Privacy}

The Hon’ble Supreme Court of India\footnote{The decision of Supreme Court is binding on India’s lower courts, as India is a Common Law Country.} in 1994 has in one of the landmark judgment of R. Rajagopal and Anr v. State of Tamil Nadu\footnote{ (1994) 2 SCC 148} broadened the scope of Article 21 by pronouncing that the “the right to privacy is implicit in the right to life and liberty guaranteed to the citizens of this country under Article 21, however this is not an absolute right”\footnote{\url{https://indiankanoon.org/doc/501107/}}. The tort of right to privacy suggests a tort action for damages resulting from an unlawful invasion of privacy. Subsequently in 1997, in a case People’s Union of Civil Liberties v. Union of India\footnote{(1997) 1 SCC 301} popularly known as ‘phone tapping case’, the Hon’ble Supreme Court of India laid down guidelines under which the government can curtail individual privacy according to procedure established by law. 

In 2015, Supreme Court in judgment in the matter of Shreya Singhal v. Union of India\footnote{ (2015) 5 SCC 1}, has held that Section 66-A of the Information Technology Act, 2000 is unconstitutional because it violates the fundamental rights of freedom of speech and expression guaranteed by Article 19(1)(a) of the Constitution. As long as written words are within its ambit, merely because they are written on a public medium on the internet would not take such actions beyond their purview, especially in view of Section 65-B of the Evidence Act, 1872. This gives right of expression to the people to keep weaving their thoughts, expression, grief’s, likes and dislikes on internet. In 2018, a 9 judge bench of the Hon’ble Supreme Court of India was constituted to determine whether a state sponsored collection of details of its citizen for various purposes is a violation of right of privacy or not. In case of K.S. Puttaswamy (Retd) v. Union of India\footnote{(2017) 10 SCC 1} the nine Judge Bench has given an unanimous answer to the query with a conclusive, unambiguous and emphatic determination that right to privacy is a part of fundamental rights which can be traced to Articles 14, 19 and 21 of the Constitution of India. It also stated that in the privacy safeguards individual to control vital aspects of his or her life. Personal choices governing a way of life are intrinsic to privacy. Intrusive policies and deficiencies should not alter the relationship between state and its citizens. The dangers to privacy in an age of information can originate not only from the state but also from non-state actors as well. States are utilising technology in the most imaginative ways particularly in view of increasing global terrorist attacks and heightened public safety concerns. 
\end{itemize}

\section{INTERNATIONAL PRIVACY\\ ACCIDENTS CONCERNING}

\begin{itemize}

\item[{\bf 1)}] \textbf{Census Ruling – Germany}

In a decision of 1983\footnote{BVerfG, decision dated December 15, 1983, BVerfGE 65, p.1}, the German Federal Constitutional Court recognized the right to informational self-determination. Data protection is not explicitly enshrined in Basic Law as per Germany’s constitution, but it does enjoy protection by virtue of what is known as the “Census Ruling” by Germany’s highest court. The decision was in response to a census that became the subject of numerous constitutional complaints of violations of respondents’ civil rights. Following the decision, the federal government was compelled to separate personal data from the census questionnaires and ensure greater anonymity for survey-takers\footnote{Alvar Freude and Trixy Freude in Echoes of History: Understanding German Data Protection available at \url{https://www.bfna.org/research/echos-of-history-understanding-german-data-protection/}}. 

\item[{\bf 2)}] \textbf{Integrity for Online Searches - Germany}

In 2008, in its decision on online searches\footnote{BVerfG decision dated February 27, 2008, BVerfGE 120, p. 274.}, the German Federal Constitutional Court introduced a “fundamental right to the guarantee of confidentiality and integrity of information technology systems, established that the general right to privacy contained in Art. 2 para. 1, in conjunction with Art. 1 para. 1 of the German Basic Law, also encompasses the fundamental right to the confidentiality and integrity of information technology systems. This was in a judicial challenge to a law regarding online searches and government Trojan Horse software, which allowed law enforcement to monitor online communications of suspected criminals.

\item[{\bf 3)}]  \textbf{Right to be forgotten}

The courts answered the critical questions whether a search engine be held liable when it has not published the information\footnote{Google Spain SL and Google Inc. v Agencia Española de Protección de Datos (AEPD) and Mario Costeja González}? The decision required Google to delist certain internet search results when a search query was made using an individual’s name. The impact of the digital age results in information on the internet being permanent. Any endeavour to remove information from the internet does not result in its absolute obliteration. The footprints remain. It is thus, said that in the digital world preservation is the norm and forgetting a struggle\footnote{Ravi Antani, “The Resistance Of Memory : Could The European Union's Right To Be Forgotten Exist In The United States?”, 30 Berkeley Tech LJ 1173 (2015).]}. 

\item[{\bf 4)}] \textbf{Safety Concerns vs. Free Flow of Information}

The free flow of personal data between competent authorities for the purposes of the prevention, investigation, detection or prosecution of criminal offences or the execution of criminal penalties, including the safeguarding against and the prevention of threats\footnote{\url{https://www.cnet.com/news/privacy-vs-safety/}} to public security within the Union and the transfer of such personal data to third countries and international organizations, should be facilitated while ensuring a high level of protection of personal data. Those developments require the building of a strong and more coherent framework for the protection of personal data in the Union, backed by strong enforcement.  Knowledge about a person gives a power over that person. The personal data collected is capable of effecting representations, influencing decision-making processes and shaping behaviour. It can be used as a tool to exercise control over us like the “big brother” State exercised. One such technique being adopted by the States is “profiling” by both state and non-state entities. 
\end{itemize}

\section{INCIDENTS OF DATA BREACH}

110 million customer’s data including names, addresses, email addresses and telephone numbers of Target Stores was reported in January, 2014. In November 2018, Marriott International announced that cyber thieves had stolen data on approximately 500 million customers. The breach actually occurred on systems supporting Starwood hotel brands starting in 2014. The attackers remained in the system after Marriott acquired Starwood in 2016 and were not discovered until September 2018.

In 2016, Uber learnt that hackers were able to get names, email addresses, and mobile phone numbers of users and 0.6 Million Uber drivers of the Uber app. It is also reported that Uber paid the hackers \$100,000 to destroy the data with no way to verify that they did, claiming it was a “bug bounty” fee. Uber’s valuation dropped from was \$68 billion to \$48 billion during this period. Every year we have seen numerous such instances of data loss. 

In 2018, HSBC bank has confirmed that some \textbf{US customers' bank accounts were hacked in October, 2018 and the} the perpetrators may have accessed information including account numbers and balances, statement and transaction histories and payee details, as well as users' names, addresses and dates of birth impacting approximately 1\% of its total consumer base\footnote{\url{https://www.bbc.com/news/technology-46117963 }}. A data theft incident\footnote{\url{https://www.cnet.com/news/international-bank-hsbc-hit-by-bangalore-breach/}} had happened in 2006, where an bank’s staff in Bangalore caused customer data to be leaked, leading to a small number of accounts from the U.K. being compromised.

In 2019 alone, we have seen several instances across geographies where personal data have been breached either voluntarily or by hackers. A few noteworthy incidents are like when white hats have been hired by YouTube channels to increase the number of viewership\footnote{\url{https://www.news18.com/news/buzz/pewdiepie-supporters-hack-printers-again-but-this-time-with-a-serious-intention-1975529.html}}, security lapse on an Indian oil company website where details concerning millions of Aadhar numbers were breached\footnote{\url{https://www.livemint.com/companies/news/indane-leaked-millions-of-aadhaar-numbers-french-security-researcher-1550559833002.html
}}, JustDial data breach exposed the personal data of 100M users\footnote{\url{https://economictimes.indiatimes.com/tech/internet/data-breach-at-justdial-leaks-100-million-user-details/articleshow/68930607.cms
}}, First American Financial Corp leaked hundreds of millions of personal records\footnote{\url{https://www.cpomagazine.com/cyber-security/security-oversight-at-first-american-causes-data-leak-of-900-million-records/}}, 5 million tax revenue records were stolen in Bulgaria (a country of just 7 Million people)\footnote{\url{https://edition.cnn.com/2019/07/21/europe/bulgaria-hack-tax-intl/index.html
}} , personal data of millions of StockX customers was breached in a hacking attack\footnote{\url{https://techcrunch.com/2019/08/03/stockx-hacked-millions-records/
}}. Oneplus reported data loss of its customers\footnote{\url{https://www.forbes.com/sites/kateoflahertyuk/2019/11/23/oneplus-confirms-hackers-accessed-customer-details-heres-what-to-do/\#2ed5212523c6}}. These reported incidents are just tip of the iceberg of the numerous events which keep happening across the globe and a majority of them are never notified to public or government at large, as it leads to question and reduced market stature.  

The larger question is why do companies need all this data stored in their databases beyond the purpose it is required. These digital footprints and extensive data can be analysed computationally to reveal patterns, trends, and associations, especially relating to human behaviour and interactions and hence, is valuable information. This is the age of “big data”. The algorithms are more effective and the computational power has magnified exponentially. The primary purpose of retaining this information is focused advertisement. In a lifetime of a customer, fundamental details like name, parents’ name, date of birth do not change. Nowadays the phone number is linked so closely with social security, banking relationship that it has become a significant part of your identity. Added on this is our whimsical approach of updating details on social media websites. In short, I can easily presume that all information which any person will require for mismanaging an online profile or hacking into an account is available online and with smart algorithms and observation, no information can even remain personal. The incidents of snooping around using these information will also increase with social engineering followed by malware infections The Indian legislature has tried to update its cyber laws according to the demands of the society but has been lagging behind tremendously, all thanks to the ever growing technology.

\section{INDIAN LEGISLATURE}

\begin{itemize}

\item[{\bf 1)}] \textbf{Information Technology Act, 2000}

The main safeguards on privacy and data protection are enumerated under the Information Technology Act, 2000 and its related regulations namely The Information Technology (Reasonable Security Practices and Procedures and Sensitive Personal Data or Information) Rules 2011. The Information Technology Act defines various sets of data (personal, civil and criminal liability in case of breach of data protection and violation of confidentiality and privacy. The IT Act also imposes stringent penalties of imprisonment up to two years or fine up to 100,000 rupees or both, on any person who secures access to any electronic record, information etc., and who, without consent of the person concerned, discloses such record, information etc., to any other person. 

According to section 43A of the ITA, a body corporate not implementing and maintaining reasonable security practices and procedures in respect of sensitive personal data or information possessed, dealt or handled by it in a computer resource owned, controlled or operated by it, is liable to pay damages to the person so affected for wrongful loss or gain to any person. The Rules mandates the basic principle of privacy law that the body corporate needs to obtain free, prior and informed consent. 

\item[{\bf 2)}] \textbf{Consumer Protection Act, 2019}

Besides the Information Technology Act, 2000 the Consumer Protection Act, 2019 has been passed by the Indian parliament, and has expanded the scope of remedies of any consumer who provides personal data to business establishments. This Act has been gazetted in August, 2019 and as per sub-section 47 of section 2, it has been stated that data should also be protected by the establishment\footnote{\url{http://egazette.nic.in/WriteReadData/2019/210422.pdf}}.The effective date for implementation of this Act is yet to be announced by the government. From that day onwards, any misutilisation of such personal data will fall within scope of unfair trade practices, leading to severe penalty under the Consumer Protection Act, 2019. The act also highlights that it is principle duty of the establishment collecting personal information, to not disclose to other person any personal information given in confidence by the consumer unless such disclosure is made in accordance with the provisions of any law for the time being in force. India is one of the first countries to recognize data as a commodity demanding protection and safekeeping under Consumer Protection legislations.

\item[{\bf 3)}] \textbf{Internet of Things Policy of 2015}\footnote{\url{https://meity.gov.in/writereaddata/files/Revised-Draft-IoT-Policy_0.pdf
}}

The internet of things (‘IoT’) is a set-up of interconnected objects, people or systems that process and react to physical and virtual information. It achieves outcomes that aim to improve user experience or the performance of devices and systems. While India is moving to smart city concept, where there would be intelligent transport systems, smart grids, city maintenance, and similar other civil objectives which will necessary talk to each other to generate demand and work in sync to cater the Indian needs in areas of agriculture, health, water quality, and natural disasters. M2M\footnote{The stark difference between M2M and IoT arises where M2M focuses on communication among machines. IoT, on the other hand, seeks to build on this concept and connect ‘things’ with ‘systems’, ‘people’, and the likes.} refers to automated exchange of data between machines, installations, individual modules, and systems – all without human intervention. With IoT and M2M connecting more things and people to the internet, it will consequentially transform lives especially in the areas of health, home automation, retail and transport. The communication between multiple devices, and enormous data transfer between their users, would result in sharing personal information, thereby raising privacy and data protection concerns. It is imperative that such privacy issues be considered at the first stage. These devices are capable of providing an increasing and detailed view of an individual activity and a comprehensive picture of his personality, likes and dislikes.

\item[{\bf 4)}] \textbf{THE PERSONAL DATA PROTECTION BILL, 2018}

The primary objective of this bill currently pending at parliament of India is to protect personal data of (data principals) as an essential facet of the informational privacy in order to foster a free and fair digital economy, respecting the informational privacy of individuals, and ensuring empowerment, progress and innovation. The bill intends to protect the autonomy of individuals by government and private entities (data fiduciaries) in relation with their personal data, to specify where the flow and usage of personal data is appropriate, to create a relationship of trust between persons and entities processing their personal data, to specify the rights of individuals whose personal data are processed, to create a framework for implementing organisational and technical measures in processing personal data, to lay down norms for cross-border transfer of personal data, to ensure the accountability of entities processing personal data, to provide remedies for un-authorised and harmful processing, and to establish a Data Protection Authority for overseeing processing activities. It remains to be seen\footnote{\url{https://economictimes.indiatimes.com/news/politics-and-nation/data-protection-bill-likely-to-be-placed-in-parliament-in-winter-session-official/articleshow/71538572.cms?from=mdr }} the way in which this bill will be brought to action for data processing by those who are processing personal data in a fair and reasonable manner by ensuring appropriate security safeguards while processing the data.
\end{itemize}

\section{LEGISLATIONS IN\\ CONTEMPORARY COUNTRIES}

\begin{itemize}

\item[{\bf 1)}] \textbf{Argentina:}

The Federal Constitution provides an article which provides for Personal Data Protection Act of 2000 of Argentina Personal data includes any kind of information that relates to individuals, except for basic information such as name, occupation, date of birth, and address. FPIC (Free Prior Informed Consent) is required from the data giver by the data collector, which also includes option to delete data as and when required by the DS.

\item[{\bf 2)}] \textbf{Australia:}

The Privacy Act 1988 (Privacy Act) was introduced to promote and protect the privacy of individuals and to regulate how Australian Government agencies and organisations with an annual turnover of more than \$3 million, and some other organisations, handle personal information. Office of the Australian Information Commissioner is has powers to make legally binding rules and guidelines. Anyone can make a complaint to the office for free at any time, and the office will investigate as soon as possible. 

\item[{\bf 3)}] \textbf{Brazil}

The Brazilian Internet Act was passed in 2014 and followed on by the recently enacted the Brazilian General Data Protection Law (LGPD) in August 2018. LGPD enters into force from in August 2020   deals with policies on the collection, maintenance, treatment and use of personal data on the Internet. Prior consent from individuals is required for DS and in case of minors between 16-18 years, consent of guardians is required. No one can give consent if below 16 years. Data Protection Impact Assessments (DPIAs) introduced by LGPD will allow assessing whether data processing will result in high risks to the rights and freedom of natural persons. 2\% of the sales revenue of the company, group or conglomerate in Brazil, and fines of up to BRL 50 million (approximately EUR 11, 395,140) per violation. 

\item[{\bf 4)}] \textbf{Japan}

Japan’s Act on the Protection of Personal Information (“APPI”) was originally passed in 2005 and was amended in 2016. The APPI went into effect May 2017. The APPI protects the personal information of individuals in Japan by establishing rules for governments and certain business operators to protect an individual’s rights with respect to the acquiring and handling an individual’s personal information. The APPI was originally enacted approximately 10 years ago but was recently amended and the amendments came into force on 30 May 2017.  APPI protects a “principal” which means “a specific individual identifiable by personal information” and casts upon obligation on a ‘business operator’ to protect this information. The business operator in-turn may collect a reasonable fees from a principal who requests to be informed about the utilization of retained personal data.

\item[{\bf 5)}] \textbf{United States of America}

There is no central legislation governing the protection of private data, but there are multiple legislations which have limited scope, but deal with protection of private data for the subjects. For example, Children Online Privacy Protection Act, 1998 prohibits collection of any online information from children below 13, and mandates verifiable parental consents if any information is collected from the children above 13. Driver’s Privacy Protection Act govern the data collected by the government for drivers. Video Privacy Protection Act and Cable Communication Policy Act protect subscribers from privacy. Gramm Leach Billey Act protects ‘non-public personal information ‘obtained by any financial institution, whereas the Health Information Portability and Accountability Act protects information that concerns health of a subject. Some states in US, also have state specific privacy legislations eg. New York has Cybersecurity Regulation applicable to financial institutions which required minimum standards, annual risk assessment audits and filing on annual compliance reports. California Consumer Privacy Act (CCPA), enacted in 2018, creates new consumer rights relating to the access to, deletion of, and sharing of personal information that is collected by businesses.

\item[{\bf 6)}]\textbf{United Kingdom}

The Data Protection Act, 2018 (DPA) is the UK’s implementation of the General Data Protection Regulation (GDPR) and anyone who processes or possesses personal data even for profiling purpose is mandated to provide access to personal data, including modification, amendment or deletion. DPA also has extra-territorial jurisdiction and damages or penalties can go reach upto 4 times of world-wide revenue turnover.
\end{itemize}

\section{SAFEGUARDING INFORMATION}

As I mentioned earlier, that information is the most prized thing about an individual in current times. We should follow the same concept for securing information about us, like we do it with all prized things in order to safeguard it. As we do not disclose in public about our prized possessions, similarly we should reserve flow of information to third party applications, bots, chats, advertisement, cookies as much as possible which browsing internet or using smart mobiles/devices. Unless absolutely necessary, do not let the information about your personal choices be on the internet. Future developments in technology and social ordering may well reveal the ways in in which a privacy right inheres that are not at present evident to us. But for now, besides being careful about what and with whom the data is being shared, there is not much one can do, besides not utilising the services. The idea of privacy is to ensure that human personality is away from unwanted intrusions. As we are aware that every transaction over internet by an individual user leaves electronic tracks generally without their knowledge, and contain powerful information which provide knowledge of the sort of person that the user is and their interests. Remaining analog is the best thing one can do to prevent privacy breaches. Besides that the author proposes that all banks, credit cards, financial institutions should have uniform and singular KYC authentication. Similarly services should be bucketed as per the platform hosts, and singular login should be made applicable for utilising services of all such things being hosted on this platform. Third party should not have any information beyond the purpose for which it was originally collected by the data fiduciary. This transformation of approach can bring us the Faraday’s cage protection. 

\vskip 0.3cm

Digital Security Trainings promoting healthy habits to safeguard privacy and security are need for the day for government, citizens and young ones. Usage of Virtual Private Network should be definitely increased. The control over data which has already been given to the data fiduciary should be regulated/moderated at periodic intervals to ensure correctness and deletion of unnecessary data. 

\section{CONCLUSION}

Education took us from thumb impression to signature

Technology has taken us from signature to thumb impression, again\footnote{Quoted in Right to Privacy Judgment of the Supreme Court of India.}

A keystone of GDPR is the stipulation of ‘adequacy requirements’ which restrict the transfer of personal data to any third country or international organisation that does not “ensure an adequate level of protection.” In doing so, the European Commission will consider whether the legal framework prevalent in India where personal data will be sought to be transferred, affords adequate protection to data subjects in respect of privacy and protection of their data. Since this will directly impact business in India for MNCs that deal with such personal data, it is believed that India will soon pass the PDPA legislation.

Last year, we all saw an unnerving plethora of data breaches and scandals, and quite notable was the incident by Cambridge Analytica. The worrying aspect is not the extensive breach but how they remained unnoticed for such long period. The working model for increasing our efficiency should be to embrace specific code of ethics for all software titans of the industry. If the awareness level of the consumers is elevated, such that they remain vigilant with the way we interact with digital technologies, we can collectively ensure that the technology companies are forced to improve and adapt an accredited level of ethically sound practice.
\end{multicols}
