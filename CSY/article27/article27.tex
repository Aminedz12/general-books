\documentclass{beamer}

\usepackage{txfonts}
\usepackage{hyperref}
\usepackage{fancybox}
\usepackage{xfrac}
\usepackage{cancel}

\newcommand{\heart}{\ensuremath\heartsuit}

\usepackage{mathtools,amssymb}
\newcommand{\myarrow}{\scalebox{2}[2]{$\mathclap{\curvearrowleft}\mkern2.2mu
                                                 \mathclap{\curvearrowright}$}}

\DeclareMathOperator{\Bin}{\mathrm{Bin}}

\hypersetup{colorlinks=false,linkbordercolor=red,linkcolor=green,pdfborderstyle={/S/U/W 1}}

\addtobeamertemplate{navigation symbols}{}{ \hspace{1em}    \usebeamerfont{footline}%
    \insertframenumber / \inserttotalframenumber}

\geometry{papersize={15cm,12cm}}
\usepackage{lipsum}

\makeatletter
\newenvironment<>{contdproof}[1][\proofname]{%
    \par
    \def\insertproofname{#1\@addpunct{.}}%
    \usebeamertemplate{proof begin}#2}
  {\usebeamertemplate{proof end}}
\makeatother


\setbeamertemplate{theorems}[numbered]

\newtheorem*{nonumdefinition}{Definition}
\newtheorem*{nonumproblem}{Problem}
\newtheorem*{nonumtheorem}{Theorem}
\newtheorem*{nonumremark}{Remark}
\newtheorem*{answer}{Answer}
\newtheorem*{nonumremarks}{Remarks}
\newtheorem*{nonumexamples}{Examples}
\newtheorem*{nonumsolution}{Solution}
\newtheorem*{nonumexample}{Example}
\newtheorem*{nonumproposition}{Proposition}
\newtheorem{proposition}[theorem]{Proposition}


\usepackage{tikz}
\newcommand*\mycirc[1]{%
  \tikz[baseline=(C.base)]\node[draw,circle,inner sep=.7pt](C) {#1};\:
}

\newcommand\myheading[1]{%
  \par\bigskip
  {\color{blue}{\large #1}}\par\smallskip}

%\usetheme{Warsaw}
%\usetheme{Berkeley} %sample 1

\usetheme{Berlin} % sample 2
%\usetheme{AnnArbor} % sample 3

\let\otp\titlepage
\renewcommand{\titlepage}{\otp\addtocounter{framenumber}{-1}}

\title{Lecture 27 : Random Intervals and Confidence IntervalsThe confidence interval formulas for the mean in an
normal distribution when $\sigma$ is known}
\author{}
\date{}

\begin{document}
\begin{frame}[plain]
\titlepage
\end{frame}

\begin{frame}
\myheading{1. Introduction}

In this lecture we will derive the formulas for the symmetric two-sided confidence
interval and the lower-tailed confidence intervals for the mean in a normal distribution
\textit{when the variance $\sigma^2$ is known}. At the end of the lecture I assign the problem
of proving the formula for the upper-tailed confidence interval as HW 12. We will
need the following theorem from probability theory that gives the distribution of the
statistic $\overline{X}$ - the point estimator for $\mu$.

Suppose that $X_1,X_2, \ldots,X_n$ is a random sample from a normal distribution with
mean $\mu$ and variance $\sigma^2$. We assume $\mu$ is unknown but $\sigma^2$ is known. We will need
the following theorem from Probability Theory.

\begin{theorem}
$\overline{X}$ has normal distribution with mean $\mu$ and variance $\sigma^2/n$.
Hence the random variable $Z = (\overline{X} - \mu)/ \dfrac{\sigma}{\sqrt{n}}$ has standard normal distribution.
\end{theorem}
\end{frame}

\begin{frame}
\myheading{2 The two-sided confidence interval formula}

Now we can prove the theorem from statistics giving the required confidence interval
for $\mu$. Note that it is symmetric around $\overline{X}$. There are also asymmetric two-sided
confidence intervals. We will discuss them later. This is one of the basic theorems
that you have to learn how to prove.

\begin{theorem}
The random interval $\left(\overline{X} - z_{\alpha/2} \dfrac{\sigma}{\sqrt{n}}, \overline{X}+ z_{\alpha / 2} \dfrac{\sigma}{\sqrt{n}} \right)$ is a $100(1 - \alpha) \% - $ confidence interval for $\mu$. 
\end{theorem}
\end{frame}

\begin{frame}
\begin{Proof}
 We are required to prove
$$
P\left(\mu \in  \left(\overline{X} - z_{\alpha/2}  \frac{\sigma }{\sqrt{n}}, \overline{X} + z_{\alpha/2} \frac{\sigma}{\sqrt{n}} \right) \right) = 1 - \alpha.
$$
We have
\begin{align*}
\text{LHS } & = P\left(\overline{X} - z_{\alpha/2} \frac{\sigma}{\sqrt{n}} < \mu, \mu < \overline{X} + z_{\alpha /2} \frac{\sigma}{\sqrt{n}} \right)\\
& = P \left(\overline{X} - \mu < z_{\alpha /2} \frac{\sigma}{\sqrt{n}}, - z_{\alpha /2} \frac{\sigma}{\sqrt{n}} < \overline{X} - \mu \right) \\
& = P \left(\overline{X} - \mu < z_{\alpha /2} \frac{\sigma}{\sqrt{n}}, \overline{X} - \mu > -z_{\alpha /2} \frac{\sigma}{ \sqrt{n}} \right)\\
& = P \left(\left(\overline{X} - \mu \right) / \frac{\sigma}{\sqrt{n}} < z_{\alpha /2}, (\overline{X} - \mu) / \frac{\sigma}{\sqrt{n}} > - z_{\alpha/2} \right)\\
& = P \left(Z < z_{\alpha /2}, Z > -z_{\alpha/ 2} \right) = P \left(-z_{\alpha /2} < Z < z_{\alpha/2 } \right) = 1 - \alpha
\end{align*}
To prove the last equality draw a picture.
\end{Proof}
\end{frame}

\begin{frame}
Once we have an actual sample $x_1, x_2, \ldots ,x_n$ we obtain the observed value $\overline{x}$ for the
random variable $\overline{X}$ and the observed value $\left(\overline{x} - z_{\alpha /2} \dfrac{\sigma}{\sqrt{n}}, \overline{x} + z_{\alpha/z} \dfrac{\sigma}{\sqrt{n}} \right)$ for the confidence (random) interval $\left(\overline{X} - z_{\alpha /2} \dfrac{\sigma}{\sqrt{n}}, \overline{X} + z_{\alpha/2} \dfrac{\sigma}{\sqrt{n}}  \right)$. The observed value of the confidence
(random) interval is also called the two-sided $100(1 − \alpha)\%$ confidence interval for $\mu$.

\myheading{3. The lower-tailed confidence interval}

In this section we will give the formula for the lower-tailed confidence interval for $\mu$.

\begin{theorem}
The random interval $\left(-\infty, \overline{X} + z_{\alpha} \frac{\sigma}{\sqrt{n}} \right)$ is a 100 $(1 - \alpha)\%$-confidence
interval for $\mu$.
\end{theorem}
\end{frame}

\begin{frame}
\begin{proof}
We are required to prove
$$
P \left( \mu \in \left(-\infty, \overline{X} + z_{\alpha} \frac{\sigma}{\sqrt{n}} \right) \right) = 1 - \alpha. 
$$
We have
\begin{align*}
\text{LHS} & = P \left(\mu < \overline{X} + z_{\alpha} \frac{\sigma}{\sqrt{n}} \right) = P \left(-z_\alpha \frac{\sigma}{\sqrt{n}} < \overline{X} - \mu \right)\\
& = P \left(-z_{\alpha} < (\overline{X} - \mu) / \frac{\sigma}{\sqrt{n}} \right)\\
& = P \left(-z_{\alpha} < Z \right)\\
& = 1- \alpha
\end{align*}
To prove the last equality draw a picture - I want \textit{you} to draw the picture on tests
and the homework.
\end{proof}
\end{frame}

\begin{frame}
Once we have an actual sample $x_1, x_2, \ldots,x_n$ we obtain the observed value $\overline{x}$ for
the random variable $\overline{X}$ and the observed value $\left(-\infty, \overline{x} + z_{\alpha} \dfrac{\sigma}{\sqrt{n}} \right)$ for the 
confidence (random) interval $\left(-\infty, \overline{X}+ z_{\alpha} \dfrac{\sigma}{\sqrt{n}} \right)$. The observed value of the confidence (random) interval is also called the lower-tailed $100(1 - \alpha) \%$ confidence interval for $\mu$.

The number random variable $\overline{X} + z_{\alpha}  \dfrac{\sigma}{\sqrt{n}}$ or its observed value $\overline{x} + z_{\alpha} \dfrac{\sigma}{\sqrt{n}}$ is often called a confidence \textit{upper bound} for $\mu$ because
$$
P\left(\mu < \overline{X} + z_{\alpha} \frac{\sigma}{\sqrt{n}} \right) =  1 - \alpha.
$$
\end{frame}

\begin{frame}
\myheading{4. The upper-tailed confidence interval for $\mu$}

Homework 12 (to be handed in on Monday, Nov.28) is to prove the following theorem.

\begin{theorem}
The random interval $\left(\overline{X} − z_{\alpha} \dfrac{\sigma}{\sqrt{n}}, \infty \right)$ is a $100(1 - \alpha)\%$ confidence
interval for $\mu$.
\end{theorem}
\end{frame}


\end{document}


