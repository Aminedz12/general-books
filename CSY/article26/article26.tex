\documentclass{beamer}

\usepackage{txfonts}
\usepackage{hyperref}
\usepackage{fancybox}
\usepackage{xfrac}
\usepackage{cancel}

\newcommand{\heart}{\ensuremath\heartsuit}

\usepackage{mathtools,amssymb}
\newcommand{\myarrow}{\scalebox{2}[2]{$\mathclap{\curvearrowleft}\mkern2.2mu
                                                 \mathclap{\curvearrowright}$}}

\DeclareMathOperator{\Bin}{\mathrm{Bin}}

\hypersetup{colorlinks=false,linkbordercolor=red,linkcolor=green,pdfborderstyle={/S/U/W 1}}

\addtobeamertemplate{navigation symbols}{}{ \hspace{1em}    \usebeamerfont{footline}%
    \insertframenumber / \inserttotalframenumber}

\geometry{papersize={15cm,12cm}}
\usepackage{lipsum}

\makeatletter
\newenvironment<>{contdproof}[1][\proofname]{%
    \par
    \def\insertproofname{#1\@addpunct{.}}%
    \usebeamertemplate{proof begin}#2}
  {\usebeamertemplate{proof end}}
\makeatother


\setbeamertemplate{theorems}[numbered]

\newtheorem*{nonumdefinition}{Definition}
\newtheorem*{nonumproblem}{Problem}
\newtheorem*{nonumtheorem}{Theorem}
\newtheorem*{nonumremark}{Remark}
\newtheorem*{answer}{Answer}
\newtheorem*{nonumremarks}{Remarks}
\newtheorem*{nonumexamples}{Examples}
\newtheorem*{nonumsolution}{Solution}
\newtheorem*{nonumexample}{Example}
\newtheorem*{nonumproposition}{Proposition}
\newtheorem{proposition}[theorem]{Proposition}


\usepackage{tikz}
\newcommand*\mycirc[1]{%
  \tikz[baseline=(C.base)]\node[draw,circle,inner sep=.7pt](C) {#1};\:
}

\newcommand\myheading[1]{%
  \par\bigskip
  {\color{blue}{\large #1}}\par\smallskip}

%\usetheme{Warsaw}
%\usetheme{Berkeley} %sample 1

\usetheme{Berlin} % sample 2
%\usetheme{AnnArbor} % sample 3

\let\otp\titlepage
\renewcommand{\titlepage}{\otp\addtocounter{framenumber}{-1}}

\title{Lecture 26 : Random Intervals and Confidence Intervals}
\author{}
\date{}

\begin{document}
\begin{frame}[plain]
\titlepage
\end{frame}

\begin{frame}
\myheading{1. The definition of a random interval}

Let $X_1$ and $X_2$ be random variables defined on the same sample space $S$ such that
$X_1(s) < X_2(s)$ for all $s \in  S$. Then $I = (X_1, X_2)$ is called an (open) random interval.
For each $s \in S$ we obtain an ordinary interval $I(s) = (X_1(s), X_2(s))$. Thus we may
think of a random interval as an \textit{interval-valued} random variable defined on $S$. The
point of this lecture is that \textit{confidence intervals are random intervals}.

\begin{nonumexample}
Suppose $X$ is a random variable defined on $S$ and $a$ is a positive number
then $I = (X -a,X+a)$ is the random interval with random center $X$ and (deterministic)
width $2a$. More generally the random interval $I = (X -Y,X + Y)$ has random
center $X$ and random width $2Y$.
\end{nonumexample}
\end{frame}

\begin{frame}
\myheading{2. Probabilities connected with random intervals}

Now consider a random interval interval $I = (X_1,X_2)$ and a fixed number $a$. We
want to compute the probability that the random interval $I$ will contain (or cover)
the fixed number $a$. But this is just the probability that $a$ will be between $X_1$ and
$X_2$ hence we have
\begin{equation*}
P(a \in   (X_1,X_2)) = P(X_1 < a,a < X_2) = P(X_1 < a,X_2 > a). \tag{1}\label{eq-1}
\end{equation*}
The probability on the right is the probability that the random variable $X_1$ will be
less than the number $a$ and the random variable $X_2$ will be greater than the number
$a$. This is just a random variable computation of the type we have done many times
in the course already. Technically we should think of the formula \eqref{eq-1} as the \textit{definition}
of the probability that $a$ will be inside $I$ but this is a technical point - it is the only
reasonable definition.

\myheading{Warning}

The probability $P(X_1 < a,X_2 > a)$ in equation (1) is almost never equal
to the product probability $P(X_1 < a)  \cdot  P(X_2 > a)$ because $X_1$ and $X_2$ are almost
never independent. For example in the above problem $X_1 = Z -1$ and $X_2 = Z +1$ so
$X_2 = X_1+2$ so $X_1$ and $X_2$ are perfectly correlated and in particular not independent.
We will conclude with an example of how to compute such probabilities.
\end{frame}

\begin{frame}
\begin{nonumproblem}
Suppose that $Z$ has standard normal distribution. Compute $P(0 \in (Z -
1, Z + 1))$.
\end{nonumproblem}


\begin{nonumsolution}
According to the equation (1) we have
$$
P(0 \in  (Z - 1, Z + 1)) = P(Z - 1 < 0, 0 < Z + 1).
$$
But
$$
P(Z - 1 < 0, 0 < Z + 1) = P(Z < 1,-1 < Z) = P(-1 < Z < 1) = P(-1 \leq  Z  \leq 1).
$$
By the ``handy formula'' we have
$$
P(-1 \leq Z \leq 1) = 2 \Phi (1) - 1 = 2(.8413) -1 = .6826
$$
\end{nonumsolution}
\end{frame}

\begin{frame}
\myheading{3. In which we go completely random}
In the first part of the course we were given a random variable $X$ and we computed
probabilities like $P(a \leq X \leq b)$. But we have an equality of events
$$
(a \leq X \leq  b) = (X \in(a, b))
$$
so
$$
P(a \leq X  \leq b) = P(X \in (a, b)).
$$
In other words we were computing the probability that a \textit{random variable} was in an
\textit{ordinary interval}. We have just learned how to compute the probability that a fixed
\textit{real number} is in a \textit{random interval} e.g. $P(0 \in (Z - 1, Z + 1))$. It remains to ``go
completely random'' and learn how to compute the probability that a \textit{random variable}
is in a \textit{random interval}. Actually we can already do this. Let's do an example.
\end{frame}

\begin{frame}
\begin{nonumproblem}
Suppose $Z$ has standard normal distribution. Compute $P(2Z \in  (Z -1, Z + 1))$.
\end{nonumproblem}

\begin{nonumsolution}
We have
\begin{align*}
& P(2Z  \in (Z - 1, Z + 1) = P(Z - 1 \leq 2Z, 2Z \leq Z + 1) = P(-1 q \leq Z, Z  \leq 1)\\
& = P(-1\leq Z \leq 1) = 2 \Phi (1) - 1 = .6826.
\end{align*}
\end{nonumsolution}

\begin{nonumremark}
In the above we had to do a little manipulation of inequalities, namely
$Z - 1 \leq 2Z  \Leftrightarrow  -1 \leq Z$ (subtract $Z$ from each side or ``bring the $Z$ from the
left-hand side to the right-hand side'') and $2Z \leq Z + 1  \Leftrightarrow  Z  \leq 1$ (again subtract
$Z$ from each side or bring the $Z$ from the right-hand side to the left-hand side'').
\end{nonumremark}
\end{frame}

\begin{frame}
\myheading{4. The definition of a confidence (random) interval}

Suppose now that $X_1,X_2,  \ldots,X_n$ is a random sample from a population whose probability
mass function (or probability density function) depends on an unknown parameter
$\theta$.  Let $\alpha$ be a real number between 0 and 1. Then a $100(1 - \alpha) \%$ confidence
interval for the unknown parameter $\alpha$ is a \textit{random interval} $I = (W_1,W_2)$ where
$W_1 = h(X_1,X_2, \ldots , X_n)$ and $W_2 = g(X_1,X_2, \ldots ,X_n)$ are \textit{statistics} such that
\begin{equation*}
P(\theta \in  (W_1,W_2)) = 1 - \alpha. \tag{2}\label{eq-2}
\end{equation*}
If we hadn't given the definition in Equation \eqref{eq-1} we wouldn’t have been able to make
the correct definition in Equation \eqref{eq-2}. If we have an actual sample $x_1, x_2,  \ldots ,x_n$
then we plug $x_1, x_2, \ldots,x_n$ into the functions $h$ and $g$ to get numbers $w_1$ and $w_2$
and an ordinary interval $(w_1,w_2)$. This ordinary interval is the \textit{observed value} of the
confidence interval $I = (W_1,W_2)$ on the sample $x_1, x_2,  \ldots,x_n$. This actual interval is
also called a confidence interval for $\theta$. It is important to keep the difference between
the confidence random interval and its obseved value on a sample firmly in mind.
\end{frame}

\begin{frame}
The rest of the course will be concerned with finding formulas for confidence intervals
in various situations - e.g. a 90\% confidence interval for the mean in a normal distribution.
In each case we will verify that the equation \eqref{eq-2} is satisfied. It is imperative
that you all learn how to do these verifications - these will be ``good citizen'' problems.
\end{frame}


\end{document}


