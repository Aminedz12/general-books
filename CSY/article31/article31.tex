\documentclass{beamer}

\usepackage{txfonts}
\usepackage{hyperref}
\usepackage{fancybox}
\usepackage{xfrac}
\usepackage{cancel}


\newcommand{\heart}{\ensuremath\heartsuit}

\usepackage{mathtools,amssymb}
\newcommand{\myarrow}{\scalebox{2}[2]{$\mathclap{\curvearrowleft}\mkern2.2mu
                                                 \mathclap{\curvearrowright}$}}

\DeclareMathOperator{\Bin}{\mathrm{Bin}}

\hypersetup{colorlinks=false,linkbordercolor=red,linkcolor=green,pdfborderstyle={/S/U/W 1}}

\addtobeamertemplate{navigation symbols}{}{ \hspace{1em}    \usebeamerfont{footline}%
    \insertframenumber / \inserttotalframenumber}

\geometry{papersize={15cm,15cm}}
\usepackage{lipsum}

\makeatletter
\newenvironment<>{contdproof}[1][\proofname]{%
    \par
    \def\insertproofname{#1\@addpunct{.}}%
    \usebeamertemplate{proof begin}#2}
  {\usebeamertemplate{proof end}}
\makeatother


\setbeamertemplate{theorems}[numbered]

\newtheorem*{nonumdefinition}{Definition}
\newtheorem*{nonumproblem}{Problem}
\newtheorem*{nonumlemma}{Lemma}
\newtheorem*{nonumtheorem}{Theorem}
\newtheorem*{nonumproof}{Proof}
\newtheorem*{nonumremark}{Remark}
\newtheorem*{answer}{Answer}
\newtheorem*{nonumremarks}{Remarks}
\newtheorem*{nonumexamples}{Examples}
\newtheorem*{nonumsolution}{Solution}
\newtheorem*{nonumexample}{Example}
\newtheorem*{nonumproposition}{Proposition}
\newtheorem{proposition}[theorem]{Proposition}



\theoremstyle{alphtheorem}
\newtheorem{alphtheorem}{Theorem}
\renewcommand{\thealphtheorem}{\Alph{alphtheorem}}
\renewcommand{\thesection}{\arabic{section}}



\usepackage{tikz}
\newcommand*\mycirc[1]{%
  \tikz[baseline=(C.base)]\node[draw,circle,inner sep=.7pt](C) {#1};\:
}

\newcommand\myheading[1]{%
  \par\bigskip
  {\color{blue}{\large #1}}\par\smallskip}

%\usetheme{Warsaw}
%\usetheme{Berkeley} %sample 1

\usetheme{Berlin} % sample 2
%\usetheme{AnnArbor} % sample 3

\let\otp\titlepage
\renewcommand{\titlepage}{\otp\addtocounter{framenumber}{-1}}

\title{Lecture 31: The prediction interval formulas for the next observation from a normal distribution when $\sigma$ is known}
\author{}
\date{}

\begin{document}
\begin{frame}[plain]
\titlepage
\end{frame}

\begin{frame}
\myheading{1. Introduction}

In this lecture we will derive the formulas for the symmetric two-sided prediction
interval for the $n + 1-$-st observation and the upper-tailed prediction interval for the
$n +1$-st observation from a normal distribution \textit{when the variance $\sigma^2$ is known}. We
will need the following theorem from probability theory that gives the distribution of
the statistic $\overline{X} - X_{n +1}$.

Suppose that $X_1,X_2, \ldots, X_n,X_{n+1}$ is a random sample from a normal distribution
with mean $\mu$ and variance $\sigma^2$. We assume $\mu$ is unknown but $\sigma^2$ is known.

\begin{theorem}
The random variable $\overline{X} - X_{n-1}$ has normal distribution with mean zero
and variance $\dfrac{n+1}{n} \sigma^2$. Hence we find that the random variable $Z = \left(\overline{X}-X_{n+1} \right)/ \left(
\sqrt{\dfrac{n+1}{n}} \sigma \right)$ has standard normal distribution.
\end{theorem}

\myheading{2. The two-sided prediction interval formula}

Now we can prove the theorem from statistics giving the required prediction interval
for the next observation $x_{n+1}$ in terms of n observations $x_1, x_2, \ldots, x_n$. Note that it
is symmetric around $\overline{X}$. This is one of the basic theorems that you have to learn how
to prove. There are also asymmetric two-sided prediction intervals.
\end{frame}

\begin{frame}
\begin{theorem}
The random interval $\left(\overline{X} -z_{\sfrac{\alpha}{2}} \sqrt{\frac{n+1}{n}} \sigma, \; \overline{X} + z_{\sfrac{\alpha}{2}}  \sqrt{\frac{n+1}{n}} \sigma \right)$ is a $100 (1-\alpha)\%$-prediction interval for $x_{n+1}$.
\end{theorem}

\begin{proof}
We are required to prove
$$
P\left(X_{n+1} \in \left(\overline{X} - z_{\sfrac{\alpha}{2}} \sqrt{\frac{n+1}{n}}  \sigma, \; \overline{x} + x_{\sfrac{\alpha}{2}} \sqrt{\frac{n+1}{n}} \sigma \right) \right) =1-\alpha.
$$
We have
\begin{align*}
\text{LHS} & = P\left(\overline{X} - z_{\sfrac{\alpha}{2}} \sqrt{\frac{n+1}{n}} \sigma < X_{n+1}, X_{n+1} < \overline{X} + z_{\sfrac{\alpha}{2}}\sqrt{n+1}{n} \sigma \right)\\
& = P \left(\overline{X} - X_{n+1} < z_{\sfrac{\alpha}{2}} \sqrt{\frac{n+1}{n}} \sigma \right)\\
& = P \left(\overline{X} - X_{n+1} < z_{\sfrac{\alpha}{2}} \sqrt{\frac{n+1}{n}} \sigma, \overline{X} - X_{n+1} > -z_{\sfrac{\alpha}{2}} \sqrt{\frac{n+1}{n}}  \sigma  \right)\\
%& = P\left( \left( \overline{X} - X_{n+1})/ \sqrt{\frac{n+1}{n}} \sigma < z_{\sfrac{\alpha}{2}}, \left(\overline{X} - X_{n+1} \right) / \sqrt{\frac{n+1}{n}} \sigma > -z_{\sfrac{\alpha}{2}} \right)\\
& = P\left( Z < z_{\sfrac{\alpha}{2}}, Z > z_{\sfrac{\alpha}{2}}\right) = P\left( -z_{\sfrac{\alpha}{2}} < Z < z_{\sfrac{\alpha}{2}}\right)  = 1-\alpha
\end{align*}
To prove the last equality draw a picture.
\end{proof}
\end{frame}

\begin{frame}
Once we have an actual sample $x_1, x_2, \ldots ,x_n$ we obtain the observed value $\left(\overline{x} - z_{\sfrac{\alpha}{2}} \sqrt{\dfrac{n+1}{n}} \sigma, \; \overline{x} - z_{\sfrac{\alpha}{2}}  \sqrt{\dfrac{n+1}{n}} \sigma \right)$ for the prediction (random) interval $\left(\overline{X} - z_{\sfrac{\alpha}{2}} \sqrt{\dfrac{n+1}{n}} \sigma, \overline{X} + z_{\sfrac{\alpha}{2}} \sqrt{\frac{n+1}{n}} \sigma \right)$  The observed value of the prediction (random) interval is also called the two-sided $100(1 -\alpha)\%$ prediction interval for $x_{n+1}$.

\myheading{3. The upper-tailed prediction interval}

In this section we will give the formula for the upper-tailed prediction interval for the
next observation $x_{n+1}$. 

\begin{theorem}
The random interval $\left(\overline{X} - z_{\alpha} \sqrt{n+1}{n} \sigma, \infty \right)$ is a $100(1-\alpha)\%$ -prediction
interval for the next observation $x_{n+1}$.
\end{theorem}

\begin{nonumproof}
We are required to prove
$$
P(X_{n+1} \in (\overline{X} - z_{\alpha} \sqrt{\frac{n+1}{n}} \sigma, \infty)) = 1 - \alpha.
$$
\end{nonumproof}
\end{frame}

\begin{frame}
\begin{proof}[Proof (Cont.)]
We have
\begin{align*}
\text{LHS } & = P \left(\overline{X} - z_{\alpha} \sqrt{\frac{n+1}{n}}  \sigma < X_{n+1}\right)
\end{align*}

To prove the last equality draw a picture - I want you to draw the picture on tests
and the final.
\end{proof}

Once we have an actual sample $x_1, x_2, \ldots, x_n$ we obtain the observed value $\left(\overline{x} - z_{\alpha} \sqrt{\dfrac{n+1}{n}} \sigma, \infty \right)$ of the upper-tailed prediction (random) interval $\left(\overline{X} - z_{\alpha} \sqrt{\frac{n+1}{n}} \sigma, \infty \right)$ The observed value of the upper-tailed prediction (random) interval is also called the
upper-tailed $100(1 - \alpha)\%$ prediction interval for $x_{n+1}$.

The number random variable $\overline{X} -z_{\alpha} \sqrt{\dfrac{n+1}{n}} \sigma$ or its observed value $\overline{x} -z_{\alpha} \sqrt{\dfrac{n+1}{n}} \sigma$ is often called a prediction \textit{lower bound} for $x_{n+1}$ because
$$
P \left(\overline{X}- z_{\alpha} \sqrt{\dfrac{n+1}{n}} \sigma < X_{n+1} \right) = 1 - \alpha.
$$
\end{frame}
\end{document}
