\chapter*{ಅರಿಕೆ}

\phantom{a}

\vskip  -1.2cm

ನಮ್ಮ ರಾಜ್ಯದ ಜನರಲ್ಲಿ ವಿಜ್ಞಾನವನ್ನು ಪ್ರಚಾರ ಮಾಡಿ ವೈಜ್ಞಾನಿಕ ಮನೋಭಾವನೆ \linebreak ಬೆಳವಣಿಗೆಗೆ ಉತ್ತೇಜನ ನೀಡುವುದು ಕರ್ನಾಟಕ ರಾಜ್ಯ ವಿಜ್ಞಾನ ಪರಿಷತ್ತಿನ ಮುಖ್ಯ ಧ್ಯೇಯ. ರಾಜ್ಯದ ವಿವಿಧ ಸ್ಥಳಗಳಲ್ಲಿ ಸ್ವಯಂ ಪ್ರೇರಣೆಯಿಂದ ರೂಪುಗೊಂಡಿರುವ ಪರಿಷತ್ತಿನ ಘಟಕ\-ಗಳು ಹಾಗೂ ಜಿಲ್ಲಾ ಸಮಿತಿಗಳ ಮುಖೇನ ಸ್ಥಳೀಯವಾಗಿ ಈ ಧ್ಯೇಯವನ್ನು ಈಡೇರಿಸುವಲ್ಲಿ ನಿರತವಾಗಿವೆ.

ಉಪನ್ಯಾಸಗಳು, ವಿಚಾರ ಸಂಕಿರಣಗಳು, ವೈಜ್ಞಾನಿಕ ಪ್ರದರ್ಶನಗಳು ಮುಂತಾದವು\-ಗಳನ್ನು ಏರ್ಪಡಿಸುವ ಮೂಲಕ ದಿನನಿತ್ಯದ ಸಮಸ್ಯೆಗಳಿಗೆ ವೈಜ್ಞಾನಿಕ ಪರಿಹಾರಗಳನ್ನು ಹುಡುಕು\-ವಲ್ಲಿ ಜನತೆಗೆ ನೆರವು ನೀಡುವ ಮೂಲಕ ಪರಿಷತ್ತಿನ  ಧ್ಯೇಯಗಳನ್ನು ಸಫಲಗೊಳಿಸುವ ಪ್ರಯತ್ನ ನಡೆದಿದೆ. ಪುಸ್ತಕಗಳ ಪ್ರಕಟಣೆಗೆ ಕರ್ನಾಟಕ ರಾಜ್ಯ ವಿಜ್ಞಾನ ಪರಿಷತ್ತು ವಿಜ್ಞಾನದ ಎಲ್ಲಾ ಪ್ರಕಾರ\-ಗಳನ್ನು ತನ್ನ ವ್ಯಾಪ್ತಿಗೆ ತೆಗೆದುಕೊಂಡಿದ್ದು ವಿದ್ಯಾರ್ಥಿಗಳ ಮತ್ತು ಜನ\-ಸಾಮಾನ್ಯರ ದೈನಂದಿನ ಜೀವನಕ್ಕೆ ಸಂಬಂಧಿಸಿದ ವೈಜ್ಞಾನಿಕ ಅಂಶಗಳಿಗೆ ಆದ್ಯತೆ ನೀಡಿ ಪುಸ್ತಕಗಳನ್ನು \linebreak ಪ್ರಕಟಿಸುತ್ತಿದೆ.

ಈ ಹಿನ್ನೆಲೆಯಲ್ಲಿ ಹೆಸರಾಂತ ಲೇಖಕರಾದ ಶ್ರೀ ಬಿ.ಕೆ.ವಿಶ್ವನಾಥರಾವ್​ ಅವರು ಬರೆದ “ಮಾಯಾಚೌಕಗಳ ಮಾಯಾ ಪ್ರಪಂಚ” ಎಂಬ ಪುಸ್ತಕವನ್ನು ಕ.ರಾವಿ.ಪ ಹೊರತರುತ್ತಿರು\break ವುದು ಸಂತೋಷದ ವಿಷಯ.

\medskip
\medskip

\begin{flushright}
\begin{tabular}{c}
{\bf ಡಾ॥ ಎಚ್. ಎಸ್. ನಿರಂಜನ ಆರಾಧ್ಯ} \\
ಅಧ್ಯಕ್ಷರು, ಕರಾವಿಪ
\end{tabular}
\end{flushright}

\noindent
ಬೆಂಗಳೂರು\\
ಅಕ್ಟೋಬರ್ {\rm  2011}\relax
\vfill
\eject

\begin{center}
{\bf 2ನೇ ಆವೃತ್ತಿ}
\end{center}

\noindent ವೈಜ್ಞಾನಿಕ ಮನೋಭಾವನೆ ಬೆಳವಣಿಗೆಗೆ ಉತ್ತೇಜನ ನೀಡುವುದು ಕರ್ನಾಟಕ ರಾಜ್ಯ ವಿಜ್ಞಾನ ಪರಿಷತ್ತಿನ ಮುಖ್ಯ ಉದ್ದೇಶ, ಪರಿಷತ್ತು ಪ್ರಕಟಿಸುವ ನಿಯತಕಾಲಿಕೆಗಳು, ಕಿರುಹೊತ್ತಿಗೆಗಳು ಈ ಪ್ರಯತ್ನಕ್ಕೆ ಬೆಂಬಲ ನೀಡುತ್ತಲಿವೆ. ಈಗಾಗಲೇ ೪೦ ವರ್ಷಕ್ಕೆ ಕಾಲಿಟ್ಟಿರುವ “ಬಾಲವಿಜ್ಞಾನ” ಮಾಸಪತ್ರಿಕೆ ಈ ದಿಶೆಯಲ್ಲಿ ಸಾಕಷ್ಟು ಯಶಸ್ಸುಗಳಿಸಿ ಜನಪ್ರಿಯವಾಗಿದೆ. ವಿಜ್ಞಾನ ವಿಷಯ\-ಗಳನ್ನು ಕಿರುಹೊತ್ತಿಗೆಗಳ ಮುಖೇನ ಪ್ರಕಟಿಸುವ ಕಾರ್ಯ ಪರಿಷತ್ತು ಕೈಗೆತ್ತಿಕೊಂಡು ಈಗಾಗಲೇ ೨೦೦ಕ್ಕೂ ಹೆಚ್ಚು ಪುಸ್ತಕಗಳನ್ನು  ಪ್ರಕಟಿಸಿದೆ. 

ರಾಜ್ಯದ ಪ್ರೌಢಶಾಲಾ ಹಂತದ ಗಣಿತ ವಿಷಯಕ್ಕೆ ಸಂಬಂಧಿಸಿದ ಪಠ್ಯಕ್ರಮದಲ್ಲಿ ಮಾಯಾ\-ಚೌಕಗಳು ಸೇರ್ಪಡೆಯಾಗಿದ್ದು, ಶಾಲಾ ಶಿಕ್ಷಕರಿಂದ ಬೇಡಿಕೆ ಇರುವುದರಿಂದ ಈ ಎರಡನೇ ಆವೃತ್ತಿಯ ಮರುಮುದ್ರಣವನ್ನು ಹೊರತರಲಾಗುತ್ತಿದೆ, ಎಂದಿನಂತೆ ಸಹೃದಯ ಓದುಗರರು ಬೆಂಬಲ ನೀಡಿ, ಈ ಪುಸ್ತಕವು ಹೆಚ್ಚು ಹೆಚ್ಚು ಪ್ರಚುರಗೊಳಿಸಬೇಕೆಂದು ನಮ್ಮ ಮನವಿ.

\medskip

\noindent
%~ \begin{tabular}{cc}
%~ {\bf ಶ್ರೀ ಎಸ್. ಎಂ. ಕೊಟ್ರುಸ್ವಾಮಿ }&  {\bf ಶ್ರೀ ಎಸ್. ವಿ. ಸಂಕನೂರ}\\
%~ ಅಧ್ಯಕ್ಷರು, ಪುಸ್ತಕ ಪ್ರಕಟಣಾ ಸಮಿತಿ, ಕರಾವಿಪ &  ಅಧ್ಯಕ್ಷರು, ಕರಾವಿಪ
%~ \end{tabular}
%~ \begin{flushright}
%~ \begin{tabular}{c}
%~ {\bf ಶ್ರೀ ಎಸ್. ವಿ. ಸಂಕನೂರ}\\
%~ ಅಧ್ಯಕ್ಷರು, ಕರಾವಿಪ
%~ \end{tabular}
%~ \end{flushright}

\bigskip
\bigskip
\bigskip
\bigskip


{\bf ಶ್ರೀ ಎಸ್. ಎಂ. ಕೊಟ್ರುಸ್ವಾಮಿ }  \hfill {\bf ಶ್ರೀ ಎಸ್. ವಿ. ಸಂಕನೂರ}\\
ಅಧ್ಯಕ್ಷರು, ಪುಸ್ತಕ ಪ್ರಕಟಣಾ ಸಮಿತಿ, ಕರಾವಿಪ \hfill  ಅಧ್ಯಕ್ಷರು, ಕರಾವಿಪ



\newpage
