%%importing packages
\usepackage[utf8]{inputenc}
\usepackage{fontspec,graphicx}
\usepackage{fancyhdr}
\usepackage{setspace}
\usepackage{xstring}
\usepackage{polyglossia}
\usepackage{titlesec}
\usepackage{xpatch}
\usepackage{array}
\usepackage{longtable}
\usepackage{enumerate}
\usepackage{enumitem}
\usepackage{cancel}
%\usepackage{color}
%\usepackage{longfbox}
\usepackage{ifthen}
\usepackage{tfrupee}
%\usepackage{pifont}
\usepackage{float}
\usepackage{amssymb}
\usepackage{amsmath}
\usepackage{amsxtra}
\usepackage{multirow}
\usepackage{multicol}
\usepackage[belowskip=-10pt,aboveskip=0pt,justification=centering]{caption}

\usepackage{tfrupee}
%~ \usepackage{newunicodechar}
  %~ \newunicodechar{₹}{\rupee}


\floatstyle{plain}

%~ \usepackage{tikz}
%~ \newcommand*\mycirc[1]{%
  %~ \tikz[baseline=(C.base)]\node[draw,circle,inner sep=.7pt](C) {#1};\:
%~ }


\setlist[itemize]{noitemsep, topsep=0pt}
\setlist[enumerate]{noitemsep, topsep=0pt}

%%longtable settings
\setlength\LTleft{30pt}
\setlength{\LTpre}{0pt}
\setlength{\LTpost}{0pt}

%%parskip setting
\setlength{\parskip}{4pt}

\xpatchcmd{\verse}{\itemsep}{\advance\topsep-0.6em\itemsep}{}{}
\xpatchcmd{\flushright}{\itemsep}{\advance\topsep-1em\itemsep}{}{}
\xpatchcmd{\flushleft}{\itemsep}{\advance\topsep-1em\itemsep}{}{}

\setmainlanguage[numerals=Kannada]{kannada}
\setotherlanguages{english}



\usepackage[papersize={140mm,215mm},textwidth=105mm,
textheight=170mm,headheight=6mm,headsep=4mm,topmargin=17.5mm,botmargin=17.5mm,
leftmargin=17.5mm,rightmargin=17.5mm,cropmarks]{zwpagelayout}


%%defining fonts
\setmainfont[
	Script=Kannada,
	BoldFont=SHREE-KAN-OTF-0850-Bold,
	ItalicFont=SHREE-KAN-OTF-0850-Italic,
	BoldItalicFont=SHREE-KAN-OTF-0850-Bold-Italic,
	HyphenChar="200C
]{SHREE-KAN-OTF-0850}

\newfontfamily\kannadafont[
	Script=Kannada,
	BoldFont=SHREE-KAN-OTF-0850-Bold,
	ItalicFont=SHREE-KAN-OTF-0850-Italic,
	BoldItalicFont=SHREE-KAN-OTF-0850-Bold-Italic,
	HyphenChar="200C
]{SHREE-KAN-OTF-0850}

\defaultfontfeatures{Ligatures=TeX}

\newfontfamily\englishfont[
	Script=Latin,
	Ligatures=TeX,
	BoldFont=GentiumBasic-Bold,
	ItalicFont=GentiumBasic-Italic,
	BoldItalicFont=GentiumBasic-BoldItalic,
]{GentiumBasic}


%%user defined commands
\long\def\titlea#1{\centerline{{\fontsize{18}{30}\selectfont\textbf{#1}}}\smallskip}
\long\def\bookTitle#1{\centerline{{\fontsize{35}{37}\selectfont\textbf{#1}}}\smallskip}
\long\def\titleb#1{\centerline{{\fontsize{18}{30}\selectfont\textbf{#1}}}\vskip 0.75cm}
\long\def\titlec#1{\centerline{{\fontsize{18}{22}\selectfont\textbf{#1}}}\vfill}
\def\titleauthor#1{\centerline{{\LARGE\textbf{#1}}}\vfill}
\def\titletopline#1{\hfill #1\vskip 2cm}
\def\authorcaption#1{\centerline{{\large\bfseries #1}}\smallskip}
\def\copyrightpub#1{\centerline{{\bfseries #1}}}
\def\copyrightpubcaption#1{\centerline{{\bf #1}}}
\def\copyrightpubaddr#1{\centerline{#1}}
\def\pagecount#1{#1 \pageref{prelims}+\pageref{lastpage}}
\newenvironment{myquote}[1]{\medskip\par\bgroup\fontsize{10}{12}\selectfont\noindent\leftskip=10pt\rightskip=10pt#1}{\par\egroup\medskip}
\def\delimiter{\bigskip\centerline{*\quad*\quad*}\bigskip}
\def\general#1{#1}
\def\vauthor#1{{\hfill #1}}
\def\supskpt#1{$^{#1}$}

\def\enginline#1{{\fontsize{12}{14}\selectfont\eng{#1}}}
\def\publisher#1{{\fontsize{20pt}{22pt}\selectfont\bfseries #1}}
\def\place#1{{\fontsize{14pt}{16pt}\selectfont #1}}

\def\eng#1{{\englishfont\textenglish{#1}}}
\def\kan#1{{\kannadafont\textkannada{#1}}}
\def\engfoot#1{\eng{#1}}

\def\indentsecionsintoc{\addtocontents{toc}{\protect\bgroup\leftskip=1.5em\rightskip=2.3em\fontsize{11}{13}\selectfont\noindent}}
\def\addsectiontoTOC#1{\addcontentsline{toc}{section}{#1}}
\def\chapterend{%
\vskip 20pt
%\centerline{{\small\rule[3.25pt]{1cm}{0.6pt}$\;$\ding{105}$\;$\ding{105}$\;$\ding{105}$\;$\rule[3.25pt]{1cm}{0.6pt}}}
\label{chap\thechapter:end}%
}

\long\def\myframe#1{
\begin{center}
\begin{longfbox}[width=7cm,padding=6pt,rounded,border-width=0.6pt]
\large #1~\hfill\phantom{.}
\end{longfbox}
\end{center}
}


\newcommand\numinwords[1]{\expandafter\xwords\csname c@#1\endcsname}
\def\xwords#1{\ifcase#1\or
ಒಂದು\or
ಎರಡು\or
ಮೂರು\or
ನಾಲ್ಕು\or
ಐದು\or
ಆರು\or
ಏಳು\or
ಎಂಟು\or
ಒಂಬತ್ತು\or
ಹತ್ತು\or
\else
I need more words\fi}

\newcommand{\arabictokannada}[1]%
{%
\StrSubstitute{#1}{0}{೦}[\num]%
\StrSubstitute{\num}{1}{೧}[\num]%
\StrSubstitute{\num}{2}{೨}[\num]%
\StrSubstitute{\num}{3}{೩}[\num]%
\StrSubstitute{\num}{4}{೪}[\num]%
\StrSubstitute{\num}{5}{೫}[\num]%
\StrSubstitute{\num}{6}{೬}[\num]%
\StrSubstitute{\num}{7}{೭}[\num]%
\StrSubstitute{\num}{8}{೮}[\num]%
\StrSubstitute{\num}{9}{೯}[\num]%
\StrSubstitute{\num}{10}{೧೦}[\num]%
\StrSubstitute{\num}{11}{೧೧}[\num]%
\StrSubstitute{\num}{12}{೧೨}[\num]%
}

%%fancy header settings
\fancypagestyle{plain}{%
\chead[]{}
\lhead[]{}
\rhead[]{}
\cfoot[]{}
}
\lhead[\small\thepage]{\small\em ಮಾಯಾಚೌಕಗಳ ಮಾಯಾ ಪ್ರಪಂಚ}
\rhead[\small\em ಮಾಯಾಚೌಕಗಳ ಮಾಯಾ ಪ್ರಪಂಚ]{\small\thepage}
\chead[]{}
\lfoot[]{}
\rfoot[]{}
\cfoot[]{}

\renewcommand{\headrulewidth}{0pt}

\pagestyle{fancy}


%%redefining macros
\renewcommand\chaptermark[1]{\markboth{#1}{}}
%~ \def\addcontentsline#1#2#3{\addtocontents{#1}{\protect\contentsline{#2}{#3}{\protect\arabictokannada{\thepage}}}}
%~ \def\addcontentsline#1#2#3{\addtocontents{#1}{\protect\contentsline{#2}{#3}{\protect\arabictodevnag{\thepage}}}}
\makeatletter

\titleformat*{\section}{\large\bfseries}
\titlespacing*{\section}{0pt}{*1.5}{*1}

\renewcommand\labelitemi{\eng{\textbullet}}

\renewenvironment{thebibliography}[1]
     {\@mkboth{\MakeUppercase\bibname}{\MakeUppercase\bibname}%
      \list{\@biblabel{\@arabic\c@enumiv}}%
           {\settowidth\labelwidth{\@biblabel{#1}}%
            \leftmargin\labelwidth
            \advance\leftmargin\labelsep
            \@openbib@code
            \usecounter{enumiv}%
            \let\p@enumiv\@empty
            \renewcommand\theenumiv{\@arabic\c@enumiv}}%
      \sloppy
      \clubpenalty4000
      \@clubpenalty \clubpenalty
      \widowpenalty4000%
      \sfcode`\.\@m}
     {\def\@noitemerr
       {\@latex@warning{Empty `thebibliography' environment}}%
      \endlist}

\def\@dottedtocline#1#2#3#4#5{\ifnum #1>\c@tocdepth\else{{#4}\kern3pt{\normalfont\normalcolor\arabictokannada{#5};}}\fi}
  
\makeatother

\linespread{1.04}
\pretolerance=-1
\tolerance=4000
\hfuzz=1pt
%~ \hbadness=10000
%~ \tolerance=1
%~ \emergencystretch=\maxdimen
%~ \hyphenpenalty=10000
%~ \hbadness=10000

\brokenpenalty9999\relax

\newcommand{\sethyphenation}[3][]{%
  \sbox0{\begin{otherlanguage}[#1]{#2}
    \hyphenation{#3}\end{otherlanguage}}}

\makeatletter
\def\@makechapterhead#1{%
  %~ \vspace*{50\p@}%
  {\parindent \z@ \centering \normalfont
    \ifnum \c@secnumdepth >\m@ne
      \if@mainmatter
        %~ \Large\bfseries \@chapapp\space 
        \par\nobreak
        \vskip 20\p@
      \fi
    \fi
    \interlinepenalty\@M
    \LARGE \bfseries\thechapter.\space #1\par\nobreak
    \vskip 30\p@
  }}
  \makeatother
