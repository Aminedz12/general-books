\chapter{मीमांसा, भारतीयानाम् अनैतिहासिकत्वञ्च}\label{chapter4}

\Authorline{के एस् कण्णन्\footnote{pp 129--149. In Kannan, K. S. (Ed.) (2019). \textit{Swadeshi Critique of Videshi Mīmāṁsā}. Chennai: Infinity Foundation India.}}


\hfill{\sl(\url{ks.kannan.2000@gmail.com})}

\bgroup

\selectdev

\section*{सङ्ग्रहः}
\begin{quote}
{\fontsize{12}{14}\selectfont
लेखनेऽस्मिन् परिमितेतरपूर्वग्रहपरिपीडितपोल्लाकाख्यपाश्चात्त्यपण्डितोपस्थापिता\break अधिक्षेपा भारतीयसंस्कृतिविषय इतिहासविषय एवं मीमांसाशास्त्रविषये\index{purva mimamsa@Pūrva Mīmāṁsā}
 च \hbox{प्रस्तूय} पूर्वपक्षसाद्विधाय\index{purvapaksa@\textsl{pūrvapakṣa}} प्रेक्षावत्प्रदिष्टप्रत्यग्रप्रतितर्कप्रतिष्ठापनपुरस्सरं प्रतिपङ्क्ति प्रत्या\-ख्याताः~। आदावितिहासमधिकृत्य भारतीयानां पाश्चात्त्यविद्वज्जनविहितानि दूषणान्युदीर्य पश्चान्मीमां सादिशास्त्रनिकुरुम्ब दूषणैकरुचेः पोल्लाकस्य\index{Pollock, Sheldon} मतान्युपस्थाप्य विशेषतश्चारविन्दशर्माख्याधुनातनविद्वज्जनेन तथा चानन्दकुमारस्वाम्यभिधगतशताब्दकविद्वत्तल्लजेन\index{Coomaraswamy, Ananda K.}\break च पुरस्स्थापितानां सत्तर्काणामुपन्यसनेन च कृतः कण्टकोद्धार इति मत्वा कार्तार्थ्यं भावयत्ययं जनः~।}\relax
\end{quote}
\begin{center}
\begin{tabular}{l}
अज्ञोऽसि यदि नाध्येषि पाश्चात्त्य-विदुषां मतम्!~।\\ 
भ्रान्तोऽसि यदि चाध्येषि पाश्चात्त्य-विदुषां मतम्!!~॥\\[3pt]
\end{tabular}
\end{center}

\vskip -10pt

\hfill - कण्णन

\vfill\eject

\begin{center}
\begin{tabular}{l}
\enginline{“Even God cannot change the past!}\\
\enginline{But historians can!!”}\\[3pt]
\end{tabular}
\end{center}

\vskip -10pt

\hfill \enginline{- Samuel Butler}

\begin{center}
\begin{tabular}{l}
“तुरुष्कयवनादिभिर्जगति जृम्भमाणं भयम्”
\end{tabular}
\end{center}

नापरोक्षमिदम्प्रेक्षावतां यच्छतकद्वयाद्यावदिण्डालजी\enginline{(Indology)}त्यभिधानेन प्रथितं भारतीय-सर्वस्वाध्ययनं पाश्चात्त्यैर्विद्वद्भिःप्रस्थापितं सुप्रतिष्ठापितं वरीवर्तीति। परस्सहस्रं विद्वांसोऽध्ययनेऽस्मिंस्तद्दर्शितमयनमेवानुरुन्धाना अनारतं तत्र व्यापृता, यानन्वेव भारतीया अपि कतिपये\break विद्वांस उद्युक्ता उद्युञ्जानाश्च दरीदृश्यन्ते~। पाश्चात्त्यास्सर्वेऽपि दुष्टैरेवाभिसन्धिभिरतिमात्रमीरिता इति यद्यपि न सुवचं, भूयांसस्तत्र परं मलीमसमानसाः कृतमनस्कास्स्वकपरम्परैकप्रागल्भ्यप्रसाधनप्रतिपादनयोरेवञ्च भारतीयपरम्परायाःपुनर्नैच्यस्थापनैकचक्षुष्कतया संलक्ष्यमाणाः~।

\vspace{-.3cm}

\section*{भारतीयेतिहासे न्यूनताः}

दुर्विदग्धेष्वीदृग्विधेष्वेकतमः षेल्डन् पोल्लाक्\enginline{(Sheldon Pollock)}\index{Pollock, Sheldon}-नामा, यो हि नानाशास्त्रेषु कृतपरिमितेतरपरिश्रमस्सन्नपि पौरोभाग्यैकभाग्येषु प्राग्र्यः प्रथामपि पृथुलां समासाद्य विदुषो\break नैकान्प्रस्थाने विलक्षणे स्वकीये प्रस्थापयन्नमेरिकादेशस्थकोलम्बिया\enginline{(Columbia)} विश्वविद्यालये लब्धप्रतिष्ठस्सन्भारतीयसर्वकारेणापि पद्मश्र्याख्येनाग्र्येण बिरुदेन विभूषितश्च~। अलङ्कारमीमांसादिशास्त्रेषु बहुष्वाबहोः कालादध्ययनादिष्वात्मानमुद्योजयन्नेष लेखान्नैकान्ग्रन्थांश्चानेकान्विततवानस्ति~। प्रकृते मीमांसाशास्त्रमधिकृत्य\index{purva mimamsa@Pūrva Mīmāṁsā} सप्तविंशतिवर्षेभ्यः प्राग्विलिखितमेतस्यैकं लेखं परामर्ष्टुं प्रयत्नः कश्चनात्र विहितो वर्तते। “पारम्परिकभारतेतिहाससमस्या मीमांसा च” \enginline{(“Mīmāṁ\-sā and the Problem of History in Traditional India” 1989)} इत्यभिधानकस्तदीयस्स लेखः~। इतिहासशब्देन चात्र नहि महाभारतादयो\index{Mahabharata@\textsl{Mahābhārata}} ग्रन्था निर्दिश्यन्ते किंतर्हि \enginline{“history”} इत्याङ्ग्लेन शब्देन निर्दिश्यमाना ज्ञानशाखा प्राग्घटितावलिनिरूपणतद्विमर्शनकृत्यजाततात्पर्यवती~।

भारतीयाः खल्वीदृक्षेतिहासविषये लक्षितपाराङ्मुख्या इति तु विदितचरमेव~। पण्डितेन लार्सन्\enginline{(Larson)}\index{Larson, G. J.} नाम्ना प्रोक्तमेवेदं यद् भारतीयचिन्तनप्रणाल्यामितिहासाख्यं कल्पनमेव स्फुटं न स्थानं कञ्चन लभते, प्राङ्नवमदशकात्~। ऐतिहासिकं च विवरणं दक्षिणैष्याभागे तावदसमासादितपरिस्फुटास्पदम्~। लार्सनोक्तमुल्लिखतीत्थं पोल्लाकः -\index{Pollock, Sheldon}

\begin{myquote}
\eng{“History is a category which has no demonstrable place within any South Asian ‘indigenous conceptual system’ (at least prior to the middle of the nineteenth century)... South Asians themselves seldom if ever used [a historical] explanation... In a South Asian environment, historical interpretation is no interpretation. It is a zero-category”.}

~\hfill \eng{Larson\index{Larson, G. J.} (1980) cited in Pollock (1989:603)}
\end{myquote}

उक्तिमिमामनुमोदमानः पोल्लाको\index{Pollock, Sheldon} नात्रातिरेकिनीमुक्तिं काञ्चिद् विभावयति, किन्तर्हि निष्प्रतिद्वन्द्वं भाषितमिदमित्यव्यभिचरितं सत्यमित्येव वा।

इतोऽग्रे च मेक्डोनेल्लस्य\enginline{(Macdonell)}वाक्यमिदमुद्धरति पोल्लाको यत्प्राक्कालीने भारत इतिहासानुपलब्धेर्निदानं नामेदं यत्तत्रोल्लेखार्हो विषय एव नाभात्कश्चिदपीति

\begin{myquote}
\eng{“Early India wrote no history because it never made any”}

~\hfill \eng{(Macdonell\index{Macdonell, Arthur A.} (1900) cited in Pollock (1989:607))!}
\end{myquote}

सिद्धान्ततया चोपस्थापितमेतन्मेक्डोनेल्लेन! कुल्के \enginline{(Kulke)}\index{Kulke, Hermann} नामापरः पुनः पाश्चात्त्यो हेतुमत्रेत्थमूहाञ्चक्रे यद्ब्राह्मणकायस्थयोर्यो विभागस्समजनि सामाजिकस्स एवेति~। तद्यथा ब्राह्मणैर्बौद्धिकानि साधनान्यात्मसात्कृतानि, कायस्थैस्तावल्लेखभण्डारस्य \enginline{(archives)} साधनानि वशीकृतानि \enginline{(Pollock 1989:607)}~। एवमेव लेफेब्र् \enginline{(Lefebvre)}\index{Lefebvre, Henri} नाम्नोऽपरस्य मतमप्यसावुपस्थापयति यज्जगतो महत्याश्चाक्रिक्या वृत्तेर्मात्रस्य सर्वदाकलनमित्येतद्धेतुक एवेदृक्षस्य व्यतिकरस्य यदन्वेवेतिहासस्य समग्रस्यापि पौराणिककथास्वेवान्तर्भाव इत्यपि च \enginline{(Pollock 1989:607)}~। प्रचुरोऽप्ययमभिप्रायो न तावांस्तृप्तिकर – इति पुनःप्रब्रुवाणःपोल्लाकस्स्वकीयमौदार्यमप्युपस्थापयन्निव लक्ष्यते~। अपरमपीदृशमेवौदार्यमस्याधिभारतान् यन्नाम नीट्शे\enginline{(Neitszche)}\index{Nietzsche, F.} नाम्नश्चिन्तकस्य लपितस्योपन्यसनं यच्च तावज्जीवन्ति नाम पशव इतिहासपराङ्मुखाः, मनुष्य एव खलु पराकुर्यात्प्रतिक्षणमभिवर्धमानं प्राचीनकालीनं भारमिति \enginline{(Pollock 1989:603)}~। अर्थाच्चित्रमत्राक्षिप्यते यन्नात्यन्तम्भिन्ना भारतीयाश्चतुष्पाद्भ्य इति!~।

\vspace{-.3cm}

\section*{नातिभिन्ना नाम ग्रीककथा}

भारतीयानां विषय इत्थमपलापपरम्परामेव प्रभूतां परिवाहयन् पोल्लाकोऽ\index{Pollock, Sheldon}स्मदीयचिन्तनावर्तनीतो नानतिरिक्तां सृतिं दधतां ग्रीकाणां विषये तावदल्पामेवापलापिकां वाङ्मयीं झरीं वाहयतीति तु\break चित्रमेव~। लान्गिनसस्तु \enginline{(Longinus)}\index{Longinus} भेदमेव न विदध इतिहासकारनाटककारयोर्मध्य इत्युल्लिखत्यपि स्वयं पोल्लाक एव \enginline{(Pollock 1989:605)}! जनगृहीतिस्तावदयथार्था यतो हि प्राक्तने काले ग्रीकदेश ऐतिहासिकं वस्तु न तत्त्वज्ञानस्याभूद्विषयो, नापि मतचिन्तनस्य, न वा सांस्कृतिकपरिशीलनस्य~। नाप्नोति स्म तात्त्विकचिन्तनप्रसङ्गे वा साधारणजनचिन्तनप्रसङ्गे वैतिहासिकी काचिद् विचारणेति भणितिःपोल्लाकस्यैवेति वेदनीयम्\endnote{\enginline{(“contrary to accepted belief, the idea of history did not constitute in itself an important philosophical, religious or cultural question in antiquity, and that history was largely marginalised in both philosophical and popular thought” Pollock (1989:605)}}। न चेदमविदितं यद्ग्रीकरोमकानामैतिहासिकीषु कथासु स्वेषां दैवतानामपि विलसितानि नाल्पीयस्सु स्थलेषु सन्दर्भेषु च गोचरीभवन्त्येवेति~। मेकिन्टैर्\enginline{(MacIntyre)}\index{McIntyre, A.} नामक ऐतिहासिकोऽपि अरिस्टाटल्-\enginline{(Aristotle)}\index{Aristotle}-प्रभृतिषु ग्रीकचिन्तकेषु निश्चिततयैतिहासिकमिदमिति शृङ्गग्राहिकया दर्शयितुं किमपि न पार्यत इत्याहेति पोल्लाक एव सूचयति \enginline{(Pollock 1989:605)}! एवमेव बोएर्-\enginline{(Boer)}\index{Boer, W. D.}-अभिधोऽप्यनयैव भङ्ग्याऽऽह – पौरणिकैतिहासिकयोर्भिदा न स्फुटा ग्रीकलिखितेषु~। पुराणेष्वेव देवा इतिहासेष्वेव मनुष्या\break इतीदृशो विषयविभागोऽपि नाभिलक्ष्यते नामेति \enginline{(Pollock 1989:605 fn “It is not that gods appear in myth and men in history, but they both appear in time and in history”)}~। पोल्लाकः\index{Pollock, Sheldon} परमधोगतटिप्पण्यामेव निक्षिपति विषयानीदृग्विधानिति यत्तस्यजागरमभिवीक्षणीयम् ननु!

अत्रान्तरे विषयान्तरं शाखाचङ्क्रमणन्यायेन प्रविविक्षुः पोल्लाकः स्टैटेन्क्रानस्य \enginline{(Steitencron)}\index{von Steitencron, H.} सिद्धान्तमावाहयति \enginline{(Pollock 1989:606)}~। यश्चेत्थम् – सप्तमाष्टमनवमशतकेषु कैस्ताब्देषु पल्लवशिल्पेषु शिवस्य गङ्गाधरमूर्ते रूपाणि “झडिति लभ्यानि” संलक्ष्यन्ते~। तच्च कुत इत्याकाङ्क्षामुत्थापयन् स्वयमेवोत्तरयति स्टैटेन्क्रान्~।

गङ्गान् (इत्युक्ते गङ्गाभिधान् महीभृतः) पल्लवनृपा यन्निर्जितवन्तस्तत्स्मारणार्थं प्रक्रान्तं सत्, पल्लवेतिहासमेव तच्छिल्पं विलिखतीव भाति - इत्याह स्टैटेन्क्रान्~। वस्तुतस्त्वनेके कवयोऽपि पर्यायोक्तभङ्ग्या वा समासोक्तिनिरूपितकेन वा तथाविधानि स्वकालिकानि घटितानि श्लोकेषु रूपयामासुरेव~। राज्ञोऽग्निमित्रस्य वृत्तमेव स्वीये मालविकाग्निमित्रे नाटके कालिदासो निरूपितवानित्यपि किल सम्भाव्यते?

किं चातः? ऐतिहासिकं विषयं कमपि प्रकाशयितुं न पारयेयुर्भारतीया इति मन्वानस्य पोल्लाकस्य\index{Pollock, Sheldon}
 मनो यन्न प्रतीयात् तद्धि परीक्षणीयत्वेनावशिष्यत इति~।

लौकिकमपि विषयं दैविकघटनान्विततयैव भारतीयाः प्रतिपादयन्तीत्याक्षेपणीयत्वेनाभीक्ष्णं विलिखति पोल्लाकः~। अत एव च भारतीयसंस्कृतेर्निगूढ ऐतिहासिको भारोऽवतारणीयः इति च स घोषयति~।

संस्कृतग्रन्था हि भूयस्त्वेनाश्चर्यकारित्वेन च ग्रन्थकर्तुर्नामादिकं न बिभ्रति~। यद्वा कर्तृभिन्नं\break नामान्तरमपि दधति! एवं चार्थशास्त्रकामशास्त्रालङ्कारशास्त्रवेदान्तशास्त्रादिग्रन्थास्सर्वेऽप्यैतिहासिकीं\index{Arthasastra@\textsl{Arthaśāstra}}\index{Alankarasastra@Alaṅkāraśāstra}\index{Vedanta@Vedānta}\index{Kamasastra@Kāmaśāstra} परिस्थितिमननुलक्ष्यैव प्रमेयाणि स्वकीयानि प्रतिपिपादयिषन्ति~। अतो हि हेतोः परस्सहस्रं\break पुटानां पठन्तोऽपि संस्कृतग्रन्थराशावितिहाससंबद्धतया तत्तत्पुरुषाणां तत्तत्स्थलानां तत्तद्घटितानां वा परामर्शो न लोचनगोचरीभवतीति भणत्वेषः\enginline{(Pollock 1989:606)}~।


\section*{मीमांसामधि}

स्वेलेखे विषयानीदृक्षान् परिलक्ष्याधिमीमांसाशास्त्रं\index{purva mimamsa@Pūrva Mīmāṁsā} शस्त्राहतिविधाने प्रवणतामेति पोल्लाकस्य\index{Pollock, Sheldon} मानसम्~। स च ब्रूते यत्पारम्परिकसंस्कृतसंस्कृतौ नामेतिहासस्य साधारण एवाभावो दरीदृश्यते यच्चानुपमितं विस्मयावहं समस्यात्मकं चेति~। निदानं पुनरस्य सर्वस्य मीमांसाशास्त्रगतत्वेन स विभावयति~। तस्य तर्कस्तावदेवम्प्रकारकः – ब्राह्मणानां शास्त्रमिदं यन्मीमांसा नाम, सांस्कृतिकान्विधिनिषेधांश्च सैव विदधाति; यश्चेतिहासोऽस्माकमधजिगमिषाया विषयस्तस्यैव प्रत्याख्यात्री सा~। इतिहासाध्ययनमेव व्यर्थमिति वा ज्ञानविरोधीति वात्यप्रस्तुतं तच्छास्त्रज्ञानसम्पादनलिप्सोरिति वा प्रतिपादयति सा~। वाक्यार्थविचारो मीमांसाया लक्ष्यम्~। तत्रापि धर्मो हि विषयो मीमांसायाः~। धर्मश्च\index{dharma@\textsl{dharma}} पुनःप्रत्यक्षानुमित्योरविषयः~। धर्मनियमा यत्रोदितास्ते हि ग्रन्था अतीन्द्रियाधारत्वेन नाम जगदुस्तत्त्वानि~। शब्दार्थयोस्सम्बन्धनित्यत्वं\index{apauruseya@\textsl{apauruṣeya}}\index{vedas@\textsl{Veda}-s} वेदानामपौरुषेयत्वं श्रुतेरनादित्वमाम्नायानामविदितकर्तृकत्वमित्यादयस्सर्वे विचारा मीमांसकैः प्रस्तुताः~। तेषां तर्हीदृशप्रस्तावोऽपि लक्ष्यं च किञ्चिदधिकृत्यैवेति सम्भावनीयम्।

वेदेषु सन्ति हि नामान्यृषीणां विविधसूक्तैः सम्बद्धानि~। किन्तु ते मन्त्ररचयितार इति न गण्यन्ते~। किंतर्हि वेदग्रन्थपरम्परारक्षका\index{vedas@\textsl{Veda}-s} इत्येतावन्मात्रम्~। न सन्ति वेदेषूल्लेखा ऐतिहासिकानां पुरुषाणाम्~। निरुक्ताख्यश्चोपायो मीमांसकानामत्यनुकूलस्सञ्जातो\index{Mimamsaka@Mīmāṁsaka} यतो हि तत्रत्या ऐतिहासिका उल्लेखा अपि सदातनानां सत्यानामेव निर्देष्टृत्वेन व्याख्यातुं शक्यन्ते निरुक्तसाचिव्येन~।\index{Nirukta@\textsl{Nirukta}}
 निरुक्तग्रन्थेऽपि यद्धि पुनरैतिहासिकं व्याख्यानमिति निरूपितं तदपि नाममात्रेण विहितम्~। आध्यात्मिकदृशा सामासोक्तिकश्लैषिकरूपितकमेव प्राधान्यमापन्नं तत्रेति~।

इदमाकूतमस्य पोल्लाकस्य\index{Pollock, Sheldon} यद्वेदेष्वैतिहासिकोल्लेखानां रिक्तीकरणं यर्हि संसाधितं तर्हीतिहाससम्बद्धविषयतिरोधापनानुगुणमेव सत्यख्यापनं कर्तव्यतयाऽपन्नमिति~। \enginline{(Pollock 1989:609)}\break यद्यपि वास्तविकघटितकान्येवाधारीकृत्य निरुक्तस्था\index{Nirukta@\textsl{Nirukta}} ऐतिहासिका व्याख्यातुं प्रायतन्त, तथापि तेषां न कोऽपि ग्रन्थोऽवशिष्टोऽक्षिसाक्षात्क्रियते~।

यावती वै संस्कृतिस्तावती वेदमयत्वेनैव निरूपिता वर्तत इत्याह याचयावतीचत्वेन पोल्लाकः\index{Pollock, Sheldon} \enginline{(Pollock 1989:609)}~। नयेन ह्यनेन, विद्याजातं समस्तमपि भारतीयानां वेदानुगुणतयैव\index{vedas@\textsl{Veda}-s} निरूपणीयतया प्रतिपन्नम्~। मनुस्मृतौ चापि सर्वज्ञानमयो हि स (मनुस्मृतिः\index{Manusmrti@\textsl{Manusmṛti}} २.७) इति वचनेन वेदानां सर्वज्ञत्वं प्रतिपादितम्~। वेदानामनन्तत्वं चानन्ता वै वेदा इति तैत्तिरीयसंहिताया\index{Taittiriya Samhita@\textsl{Taittirīya Saṁhitā}} वचनेन \enginline{(3.10.11.4)} समाम्नातम्~। उत्तरोत्तरे च काले भवास्सर्वेऽपि ग्रन्था नानाशास्त्रका वेदराश्यन्तर्भाविततयैव विभाविता वर्तन्ते~। तच्च तैस्स्वस्यैव वेदत्वप्रख्यापनेन यद्वा वेदसंक्षेपकत्वेन यदपि वा वेदोदिततत्त्वजातनिष्पादितत्वेन~। अग्निपुराणं वा भवतु रामायण-महाभारतादिकं\index{Ramayana@\textsl{Rāmāyaṇa}}\index{Mahabharata@\textsl{Mahābhārata}} वा भवतु नाट्यशास्त्रं\index{Natyasastra@\textsl{Nāṭyaśāstra}} वापि भवतु पञ्चमवेदत्वेनैव व्यपदिदिक्षन्त्यात्मानमेते । वेदानां वेदमिति\index{vedas@\textsl{Veda}-s} छान्दोग्योपनिषदीतिहासपुराणे\index{Chandogya Upanisad@\textsl{Chāndogya Upaniṣad}} समकक्ष्यतया लक्षिते स्तो ननु (छान्दोग्योपनिषत् ७.१.२)~। न्यायसूत्रभाष्य\index{Nyayasutra bhasya@\textsl{Nyāyasūtra Bhāṣya}} (४.१.६१) इतिहासशब्देन वास्तविकघटितजातमेव यद्यपि निर्दिष्टं, तथापि यद्धि सदातनं तस्यैव ग्रन्थरूपेणाविष्करणत्वेनैव पर्यवसन्नं तत्~। मीमांसा\index{purva mimamsa@Pūrva Mīmāṁsā} हि व्यक्त्यपेक्षयाऽऽकृतिमेव ननु पुरस्करोति~। इदमपि च तत्तत्कालघटितत्वापेक्षया सना पौनःपुन्येनोक्तस्यैव तत्त्वस्य तुलनामारूढम्~। रामायणमहाभारतादीनां व्याख्यानमप्यध्यात्मपरत्वेनैव साधारण्येन विवक्ष्यते ननु? यथा नाम नीलकण्ठेन\index{Nilakantha@Nīlakaṇṭha} महाभारतव्याख्याप्रसङ्गे~। चतुर्दशविद्यास्थानानां\index{vidyasthana@\textsl{vidyāsthāna}} तात्पर्यं समग्रस्यास्य भारतकाव्यस्य च तात्पर्यं च सारतोऽभिन्न एवेत्येकाशयत्वमनयोरविप्रलपनीयम्~। महेश्वरतीर्थगोविन्दराजौ\index{Mahesvaratirtha@Maheśvaratīrtha}\index{Govindaraja@Govindarāja} श्रीवैष्णवपरम्परापरावप्यधि रामायणमित्थमेव प्रवृत्तौ लक्ष्येते \enginline{(Pollock 1989:610)}~।

अयं तर्हि पोल्लाकस्य\index{Pollock, Sheldon} सिद्धान्तो यदितिहासोऽपि नाम नात्यन्तमनवस्थितस्संस्कृतवाङ्मये निरूपिते भारते \enginline{(Pollock 1989:610)}~। किन्तर्ह्यन्यसत्यापेक्षया तिरस्कार्यत्वमापन्नो यत्रैतिहासिकस्य सत्यस्य नाम न मौलिकं किमपि ज्ञानदृष्ट्या प्रयोजनम् न वा सामाजिकं किञ्चन प्रयोजनम्~। एवञ्च तस्य सिद्धान्तो यद्भारते \enginline{system} (“व्यवस्था”) इत्यस्यैव स्थानम्, न पुनः \enginline{process}(“क्रिया”) इत्यस्य~। अर्थात् सामाजिकी या व्यवस्था तस्या एव स्थानं, न पुनर्मानवस्य सर्जनात्मिकायाः प्रवृत्तेरित्याकूत्याभिहितम्~। अस्य चानुगमश्चेत्थं यदैतिहासिक्यः परिणतयः \enginline{(transformations)} पूर्वकाले चोत्तरकाले च निराकृता भवन्तीति~।

\vspace{-.5cm}

\section*{विमर्शनमनैतिहासिकतारोपस्य}

इत्थं पोल्लाकवादजालं\index{Pollock, Sheldon} पुरो विन्यस्य तद्विमर्शनकार्य आत्मानमधुना व्यापारयामः~। ऐतिहासिकं नाम वाङ्मयं भारतीयानां न भूयस्त्वेन लभ्यमिति विषये विप्रतिपदनं विद्वद्भिर्न कैरपि साधारण्येन क्रियते~। इतिहाससम्बन्धिनीमिमां परिस्थितिं भारतस्य परिदेवयन्तो दृश्यन्तेऽपरेऽपि पण्डिताः~। तद्यथा मधुराविजयनामकैतिहासिककाव्यसम्पादनसन्दर्भीयपीठिकायां खिद्यति तिरुवेङ्कटाचारी -

\begin{myquote}
\eng{“It is an irony that the country with the most ancient civilization should have very few original histories about its past”}

~\hfill \eng{(Tiruvenkatachari\index{Tiruvenkatachari, S.} 1957:6)}
\end{myquote}

भारतीयानामहरहर्जीवनं वर्णयन् आबोयर्-नामा संशोधकश्च पुरातनकालीनानामितिहाससमुल्लिखितकानां सरणिं साधरणीमित्थमवतारयति –

\begin{myquote}
\eng{ “During the entire period of ancient history, royal and local chronicles, when they exist, repeatedly convert historical facts into myth and legend. This complicates considerably the task of the modern historian and occasionally reduces him to the expedient of basing his hypotheses upon deduction alone.” }

\vskip -8pt

~\hfill \eng{Auboyer\index{Auboyer, J.} (1961:xi)}
\end{myquote}

\vfill\eject

नौडौनामा शोधकोऽपि वैयाकरणोदीरितोदाहरणभणितकमाधारीकृत्य कदाचिदिमे भारतीयास्स्वमितिहासं निर्मित्सन्तीति सहासमाह –

\begin{myquote}
\eng{ “The historians of India (4$^{\text{th}}$ BC) are reduced to the expedient of constructing history on a foundation of grammatical examples!”}

~\hfill \eng{(Naudou\index{Naudou, J.} 1956:1454 cited in Auboyer\index{Auboyer, J.} 1961)}
\end{myquote}

नहि सर्वे भारतशास्त्रविदो \enginline{(Indologists)} भारतीयनागरिकताविषये न्यक्करणतत्पराः~। पोल्लाकगुरुरिङ्गालसोऽत्रोल्लेख्यो\index{Pollock, Sheldon} यद्वचांस्यधिक्षेपगन्धविदूरगानि –

\begin{myquote}
\eng{ “We know nothing of the personal lives of Sanskrit poets, just as they tell us nothing of the personal lives of their patron. The persons here have melted into the types of poet and king.”}

~\hfill \eng{Ingalls\index{Ingalls, Daniel H. H.} (1965:24) }
\end{myquote}

परं दोषानाविद्धं किमपि नास्त्येव पुनर्भारतीयनागरिकतायां पोल्लाकस्य दृष्टौ प्रायेण~। दक्षिणेष्याभाग एष सर्वोऽपि सर्वमनुष्यशोषणभूमिरमुष्य मते~। परन्तु शृण्वन्तु बाषाम् \enginline{(Basham)}\index{Basham, A. L.} इत्यस्य व्यतिरेरिचानं वचनमिदं यज्जगत्यन्यत्र न क्वापि प्रजासु मध्ये पारस्परिकस्सम्बन्धः प्रजानां राज्यस्य च सम्बन्धश्चैतावान् न्याय्य आसीदेतावांश्च मानुष्यभरश्च \enginline{(humane)}~। नान्यत्र नागरिकतायां दासानां सङ्ख्या तावत्यल्पा वासीत्तथाच न क्वाप्यन्यस्यामाद्यायां नागरिकतायां कस्यामपि जनानामधिकाराणां \enginline{(rights)} तादृक्षं समीचीनं संरक्षणं यथा अर्थशास्त्र\index{Arthasastra@\textsl{Arthaśāstra}} इह कौटलीय\index{Kautalya@Kauṭalya} इति~। रणाङ्गणे धर्मयुद्धप्रकारश्च\index{dharma@\textsl{dharma}} यथा मनुना घोषितस्तथा न क्वाप्यन्यत्रापीषदपीति \enginline{(Basham 1967:8-9)}~।

यत्तु पोल्लाकः\index{Pollock, Sheldon} साक्रोशमुदगिरत्सर्वं वेदसाद्विहितमत्रेति \enginline{(Pollock 1989:609)}, यच्च क्रैस्ता\break मतान्तरकरणपरायणा जगर्जुर्यद्दुर्भिक्षदूरोग(वर्णव्यवस्थाख्य)दौर्जन्यप्रभृतिभिश्शोषितास्सम-\break\-भवन्दुःखदौर्मनस्यभरिताश्च प्रजा अत्रत्या इति तस्योभयस्यापि प्रत्याख्यानं बाषमेनैव प्रत्तमस्ति \enginline{(Basham\index{Basham, A. L.} 1967:9)} यज्जनास्स्सम्यगेव नूनमनूनं सौख्यमन्वभवन्नैन्द्रियिकाणाम् अतीन्द्रियिकाणामुभयेषामपीति~।

अथ च स्वीये भारतीयार्थशास्त्रमतानाम् इतिहास\enginline{(History of Indian Political Ideas)}\break इत्यभिधे पुस्तके घोषलाख्यो जुघोष\enginline{(U N Ghoshal)}\index{Ghoshal, U. N.} यत्प्राचीनभारतीयवैलक्षण्यनिरूपकलक्ष्मत्रयमभिलक्ष्य प्रोक्तमरविन्देन महर्षिरिति व्यपदिष्टेन \enginline{(Aurobindo)}\index{Sri Aurobindo} यन्नाम प्राथम्येना\-ध्यात्मि\-कता भारतीयानाम् यच्च तेषां चित्तस्य वैशिष्ट्यस्य द्योतकम्; द्वितीयं तेषां जीवनोत्साहोऽदम्यो\break यदुत्था प्रभूता सर्जनशीलता; तृतीयमन्तिमं च दृढा मनीषिता यत्र नाम प्रागल्भ्यमार्दवे सहैव\break स्तस्सारल्याढ्यत्वे चापि सहैव स्त इत्यादिकम् \enginline{(Ghoshal\index{Ghoshal, U. N.} 1959:3)}~।

\section*{वेदराशिरितिहासश्च}
\index{vedas@\textsl{Veda}-s}

यत्तु पोल्लाकेन\index{Pollock, Sheldon} लपितं वेदस्य यत्प्रामुख्यं प्रत्तं तेन धर्मज्ञानस्यान्यद्वाराणि\index{dharma@\textsl{dharma}} निरस्तानि सन्तीति तदपि व्युदस्तं वेदेनैव~। यदाह श्रुतिस्स्मृतिः\index{sruti@\textsl{śruti}}
\index{smrti@\textsl{smṛti}} प्रत्यक्षमैतिह्यमनुमानश्चतुष्टयम् (तैत्तिरीयारण्यकम्\index{Taittiriya Aranyaka@\textsl{Taittirīya Āraṇyaka}} १.२.१) इति~। वचनेनानेन वेदोक्तमात्रस्य प्रामाण्यं प्राधान्यं वा, प्रत्यक्षादिकस्य नावकाशप्रसङ्ग इति वा वादस्तस्य पोल्लाकस्य समस्तो निरस्तो भवति~। एवं चैतिह्यस्यापि स्थानं दत्तमस्तीति हेतुना यच्छब्दोक्तं तस्यापि प्रामाण्यमूरीकृतं लक्ष्यते यदाह सायणो\index{Sayanacarya@Sāyaṇācārya} भाष्ये स्वीय ऐतिह्यं विवृण्वन्नैतिह्यं नामेतिहासपुराणमहाभारतब्राह्मणादिकमिति~।\index{Mahabharata@\textsl{Mahābhārata}} एवं ज्ञानद्वाराणां समेषां स्थानं यथोचितमुपपादितमेव लक्ष्यते~। मन्त्रस्यास्य\index{mantra@\textsl{mantra}} भावं विवृण्वान आह सायणस्तदेतत्स्मृत्यादिचतुष्टयं धर्मयाथार्थ्यावगतिकारणीभूतं प्रमाणमिति~।

मार्क्स(\enginline{Marx})वादानुयायी पोल्लाको मार्क्सवादाभिघातकमभिप्रायमेवमभिलपतीत्यपि विस्मयस्यैव विषयः~। “यद्धि नाम तार्किकं तद्धि स्वस्मिन्नेवैतिहासिकमन्तर्भावयति \enginline{(The logical\break contains within itself the historical)} इति बत मार्क्सवादिनां सूत्रम् \enginline{(Frolov\index{Frolov, I.}\break 1984:174)}~। इत्युक्ते मार्क्सवादिनामयमाग्रहो यन्मार्क्सवाद एव वस्तुतो वस्तुतत्त्वानुसारी~।\break आतश्चेतिहासस्सर्वोऽपि मार्क्स्वादिनां नयमेवानुसृत्य घटिष्यत इति~। मार्क्स्वादसिद्धान्तानुसारमेव खलु जगति सर्वं सर्वदा प्रसिद्ध्यतीति! मार्क्स्तर्कस्य महिमाऽयं यद्राज्यं समाज इत्यादिकं सर्वमपि मार्क्स्तर्कमेवानुरुणद्धीति~। पश्चात्काले तु धनिकाराधनरूपः \enginline{(capitalism)} सिद्धान्तो\break नङ्क्ष्यति समाजवादश्च विराराजत इत्यादिकं सर्वं मार्क्स्वादादेव सेत्स्यतीत्याह फ्रोलोव् \enginline{(Frolov)}\index{Frolov, I.}\break नामकः~। वस्तुतस्तु तत्सर्वं नैव जगति जघट इति पामरैरपि परिज्ञातमेव~।

इतिहासपुराणानां यदान्तरिकार्थपरिकल्पनं तात्त्विकार्थविभावनं वा (यदेव \enginline{allegorical interpretation} इति कथयन्ति) तदधिकृत्य स्वामसम्मतिं दिशति पोल्लाकः~।\index{Pollock, Sheldon}\index{vedas@\textsl{Veda}-s} वेदमन्त्राणामर्थत्रयमाहुर्वेदव्याख्यातार आधिभौतिकमाधिदैविकमाध्यात्मिकं चेति तावदास्ताम्~। नानास्तरीयव्याख्यानं तावत् क्रैस्तेष्वपि वर्तत एवेति (क्वचिच्च स्तरसप्तकात्मकमपीति च) पोल्लाको ज्ञापनीयः~।

कुतः खलु हिन्दुभिश्चैनैरिव(\enginline{Chinese})घटितलिखितिर्नैव विहितेति मर्मोद्घाटनमार्गनाख्येन विदुषैवं न्यरूपि~।

\begin{myquote}
\eng{“Hindus did not preserve records as diligently as the Chinese did, “what the Hindus felt worth preserving was the meaning of events, not a record of when events took place.””}

~\hfill \eng{Organ\index{Organ, Troy Wilson} (1970:30-31)}
\end{myquote}

कीथोऽपि कारणमेवमूहाञ्चक्रे यत्कालतत्त्वस्यैव भारतीयैस्सङ्ख्यावद्भिर्गौणस्थानपरिकल्पितकं हेतुरत्रेति~।

\begin{myquote}
\eng{“Indifference to chronology is seen everywhere in India, and must be definitely connected, in the ultimate issue, with the quite secondary character ascribed to time by philosophies.”}

~\hfill \eng{(Keith\index{Keith, A. B.} 1920:146-147)}
\end{myquote}

इतिहासविषये पाश्चात्त्यानामाग्रहविशेषो लक्ष्यते यदनुसारं च ते महाभारतमूलकथा\index{Mahabharata@\textsl{Mahābhārata}} नामाहवमात्रिकेति प्रतिपिपादयिषन्ति~। परन्तु नहि पाश्चात्त्यानां मानदण्डा एव मानार्हतयाभ्युपेयाः~। यथाह हीहसाख्यो विद्वान्~।

\vskip 2pt

\begin{myquote}
\eng{“Europe’s literary criteria were not applicable to India. Albrecht Weber’s idea that the original \textit{Mahābhārata} consisted only of the battle chapters was a case of ‘arguing from Homer’.”}

~\hfill \eng{(Heehs\index{Heehs, Peter} 2003:177--178)}
\end{myquote}

\vskip 3pt

हेगेलोक्तं(\enginline{Hegel}) सर्वज्ञप्रमाणमिति मन्यन्ते नैके विद्वांसो यच्च हाल्बफासो वा रंजनघोषो\index{Ghosh, Ranjan} वा नोररीकुर्युः~। हेगेलोक्तं पाश्चात्त्येतिहासमात्रन्वयीत्याह हाल्बफासः~।

\vskip 3pt

\begin{myquote}
\eng{“Hegel’s scheme of the history of philosophy is primarily designed to deal with the history of European thought from Thales to Kant and Hegel\index{Kant, Immanuel}\index{Hegel, G. W. F.} himself... where in this scheme does Asia, and India in particular, have its place?”}

~\hfill \eng{(Halbfass\index{Halbfass, W.} 1988:88)}
\end{myquote}

\vskip 3pt

पाश्चात्त्यानामैकदेशिक्यः प्रतीतयस्सार्वदेशिकतया न निभालनीया इति, स्वानुकूलानिर्वर्तकत्वेन परिकल्पितानां क्रमाणां नहि सर्वान्वयित्वमित्यप्याह घोषाख्यो विपश्चित्\endnote{\enginline{“To categorzie the Indian concept of history as prehistory within Hegelian principles or strategic British historiographic imperialist schemes is cutting down the richness of possibility as “historicality shrinks in scope to enable a narrowly constructed historiography to speak for all of history.”(Ghosh 2007:216)}}~। भारतीयाश्चिन्तनप्रणाल्यः परेषां प्रक्रिया अवश्यमतिशेरत इत्येव घण्टाघोषं घोषयति घोषः~।

\vskip 3pt

\begin{myquote}
\eng{“Compared to other civiliations that view history in term of thousands of years, the Indians – Buddhists, Jains and Hindus – narrated it in terms of billions of years...”}

~\hfill \eng{(Ghosh\index{Ghosh, Ranjan} 2007:213-4)}
\end{myquote}

\vskip 3pt

अपि च नैतिहासिकत्वमेव गरीयस्तत्त्वं, विशिष्टातिरेकि हि सामान्यमित्यादिकमपि तस्यैव भणितिः\endnote{\enginline{“Indian history can flaunt the luxury of achronicity and ahistoricity... The Indian mind would prefer the “general to the particular”, and meaning to chronology.”(Ghosh 2007:213)}}।

\vskip 3pt

नीट्शेप्रोक्तपशुमानवभिदायाः\index{Nietzsche, F.} प्रतीपत्वमेव प्रतीयत उक्तावस्याम्~। “नृपशुरथवा पशुपतिः” “स योगी ह्यथवा पशुः” इत्याद्युक्तिवदत्रापि बहीरूपसाम्यमवालोक्यते~। (अधुन्वन् मूर्धानं नृपशुरथवा पशुपतिरिति जगन्नाथोक्तिः; सुभाषितेन गीतेन युवतीनां च लीलया~। मनो न भिद्यते यस्य स योगी ह्यथवा पशुः~॥ इत्येवंरूपके सुभाषिते चात्रोल्लेख्ये~।)

\vskip 3pt

इतिहासप्रज्ञा हिन्दुषु किं नासीदेव? इति प्रश्नमेव स्वलेखस्य शीर्षिकात्वेन प्रतिपद्यमानोऽरविन्दशर्मा \enginline{(Arvind Sharma)}\index{Sharma, Arvind} तावद् भारतीयानां रुचिविषये सामर्थ्यविषये च नैकानि सत्यान्यधीतिहासं प्राचीकटत्~। तत्रत्याः केचनांशा अवश्योल्लेख्या अत्र~। शिलाशासनानां\index{inscriptions} प्रामुख्यं भारतेतिहासे विशदीकुर्वन्कांश्चन मुख्यानंशानेवं स द्योतयति~।

\vskip 3pt

यद्यपि सन्ति नानाप्रकारा इतिहासविलेखने, शिलाशासनानि\index{inscriptions} ताम्रपत्राणि च परं भागं गृह्णन्ति भारतीयेतिहाससन्दर्भे~। कियन्त्यासञ्छिलाशासनानि कुत्र च कुतश्चेति विवेच्यमेव~। दक्षिणभारते नवतिसहस्रं शिलाशासनानां लभ्यत इति विलिखति सर्कारः \enginline{(Sircar)}\index{Sircar, D. C.} इति तन्मतमादावाविष्करोति~।

\vskip 3pt

\begin{myquote}
\eng{“The favoured medium in which the rulers of India left behind their records are inscriptions. About 90,000 inscriptions\index{inscriptions} have so far been discovered in different parts of India...Many of these inscriptions have not yet been published. Every year new inscriptions are being discovered.”}

~\hfill \eng{(Sircar\index{Sircar, D. C.} 1977:91) }
\end{myquote}

\vskip 3pt

शकवर्षगणन उपयुज्यमानैः प्रकारैरपि भारतीयानामितिहासप्रज्ञा स्पष्टीभवति~। त्रयोदशगणनाप्रकारानुल्लिखति बाषम् \enginline{(Basham)}~\index{Basham, A. L.}। बाषमस्य पट्टिकैवम्प्रकारिका – (अल्बिरूनी(\enginline{Al Biruni})\index{Al Biruni} चापि किञ्चिद्भिन्नांश्चतुर्दशप्रकारान्निर्दिशति~।)
\begin{myquote}
\eng{“A. L. Basham lists [these eras]}
\begin{enumerate}[topsep=1pt]
\renewcommand{\labelenumi}{\eng{\theenumi}}
\itemsep=4pt
\item \eng{Era of the Kaliyuga (3102 BC);}
 
 \item \eng{Śrī Lankan Buddha Era (544 BC);}

 \item \eng{Era of Mahāvīra (528 BC);}

 \item \eng{Vikrama Era (58 BC); }

 \item \eng{Śaka Era (78 AD); }

 \item \eng{Licchavi Era (110 AD); }

 \item \eng{Kalacūrī Era (248 AD); }

 \item \eng{Gupta Era (320 AD); }

 \item \eng{Harṣa Era (606 AD); }

 \item \eng{Kollam Era of Malabār (825 AD); }

 \item \eng{Nevār Era (878 AD); }

\itemsep=2pt

 \item \eng{Era of Vikramāditya VI Cālukya (1075 AD); and }

 \item \eng{Lakṣmaṇa Era of Bengal (1119 AD).”}
\end{enumerate}

\vskip -2pt

~\hfill \enginline{(Sharma\index{Sharma, Arvind} 2003:208 fn)}
\end{myquote}

सुराज्यव्यवस्थासु निबन्धपुस्तकपत्रिकास्वारोप्य रक्षणीयानंशानक्षपटलनिरूपणप्रसङ्गे कौटल्यस्सूचयति~।\index{Kautalya@Kauṭalya}

अक्षपटलप्रक्रिया कौटलीयप्रोक्ता तावदवश्यमविगणनीया ।

“अक्षपटलम् अध्यक्षः प्राङ्मुखमुदङ्मुखं वा विभक्तोपस्थानं निबन्धपुस्तकस्थानं कारयेत्~। तत्र\break अधिकरणानां सङ्ख्यां, प्रचारसञ्जाताग्रं, क्रमान्तानां द्रव्यप्रयोगे वृद्धिक्षयव्ययप्रयामव्याजीयोगस्थानवेतनविष्टिप्रमाणं, रत्नसारफल्गुकुप्यानामर्घप्रतिवर्णकप्रतिमानमानोन्मानावमानभाण्डं, देशग्रामजातिकुलसङ्घानां धर्मव्यवहारचरित्रसंस्थानं,\index{dharma@\textsl{dharma}} राजोपजीविनां प्रग्रहप्रदेशभोगपरिहारभक्तवेतनलाभं, राज्ञश्च पत्नीपुत्राणां रत्नभूमिलाभं, निर्देशौत्पातिकप्रतीकारलाभं, मित्रामित्राणां च सन्धिविक्रमप्रदानादानं, निबन्धपुस्तकस्थं कारयेत्~। ततः सर्वाधिकरणानां करणीयं सिद्धं शेषं आयव्ययौ नीवीम् उपस्थानं प्रचारचरित्रसंस्थानं च निबन्धेन प्रयच्छेत्~॥” 

~\hfill कौटलीयार्थशास्त्रम्\index{Arthasastra@\textsl{Arthaśāstra}}\index{Kautalya@Kauṭalya} २.७.१\endnote{\enginline{“An office of very great importance, situated in the capital, is the Akṣapaṭala. It is a sort of records-cum-audit office. There is an adhyakṣa in charge, with a special building of his own with many halls and record rooms (2.7.1). The records to be maintained there pertain to}
\begin{itemize}
\itemsep=0pt
\item[(1)] \enginline{the activity of each state department,}
\item[(2)] \enginline{the working of state factories and conditions governing production in them,}
\item[(3)] \enginline{prices, samples and standards of measuring instruments for various kinds of goods,}
\item[(4)] \enginline{laws, transactions, customs, and regulations in force in different regions, villages, castes, families and corporations,}
\item[(5)] \enginline{salaries and other perquisites of state servants,}
\item[(6)] \enginline{what is made over to the king and other members of the royal family, and}
\item[(7)] \enginline{payments made to and amounts received from foreign princes, whether allies or foe (2.7.2). A more comprehensive record-house can hardly be thought of.”}
\end{itemize}

\hfill \enginline{(Kangle 1988, Vol.3:201)}}

\vskip -2pt

चोलराज्येष्वक्षपटलपद्धतिः कियत्यद्भुतासीदिति बाषमनिरूपितकमवलोकनीयम्~। स आह -

\vskip -2pt

\begin{myquote}
\eng{“To transmit the royal decrees a corps of secretaries and clerks was maintained, and remarkable precautions were taken to prevent error. Under the Coḷas, for instance, orders were first written by scribes at the king’s dictation, and the accuracy of the drafts was attested by competent witnesses. Before being sent to their recipients they were carefully transcribed, and a number of witnesses, sometimes amounting to as many as thirteen, again attested them. In the case of grants of land and previleges an important court official was generally deputed to ensure that the royal decress were put into effect. Thus records were kept with great care, and nothing was left to chance; the royal scribes themselves were often important personages.”}

~\hfill \eng{(Basham\index{Basham, A. L.} 1999:100)}
\end{myquote}

\vskip -2pt

चीनदेशीययात्रिकेन जुयन् जाङ्गेन \enginline{(Zuanzang = Hiuen Tsang)}\index{Huen Tsang} पुनः - प्रतिपुरं \enginline{(every district)} लेख्यराशेरुल्लेखः कृत - इति वदति बीलाख्य इतिहासवित्~। \enginline{(Beal\index{Beal, Samuel} 1969:78)} । परन्त्वधुना तानि लेख्यानि नैव लभ्यन्ते! साधारण्येन जुयन् जाङ्गेन दत्तेषु ह्युल्लेखेषु प्रत्ययः महत्तरः वर्ततेऽन्यदत्तोल्लेखापेक्षया~।

\newpage

वंशावलिविलेखन एव प्रधान आदरो भारतीयानामिति थापरपि निरूपयति यश्चाद्यावध्यनिरुद्ध इति च ।

\begin{myquote}
\eng{“The core of historical tradition in India was the genealogical records. These have remained constant in the Indian scene throughout the centuries and in fact up to the present day”}

~\hfill \eng{(Thapar\index{Thapar, Romila} 1978:278)}
\end{myquote}

यच्च शास्त्रेषु पूर्वाचार्यादीनां स्मरणानि कृतानि तान्यपीतिहासदृष्टिमेव भारतीयानां स्पष्टीकुर्वन्ति~। अरविन्दशर्मा उल्लिखति प्रकृष्टानाकरान्कांश्चनावलम्ब्य यन्महर्षिणा पाणिनिना\index{Panini@Pāṇini} पूर्वाचार्याणां चतुष्षष्टिरुल्लिखिता वर्तत इति । एवमेव चरकाचार्येण\index{Caraka} पञ्चाशदधिकाः पूर्वे विशेषज्ञा भरतमुनिना\index{Bharata} पुनश्शताधिकाः पूर्वाचार्यास्स्मर्यन्ते~। अर्थशास्त्रेऽप्यनेकेषां\index{Arthasastra@\textsl{Arthaśāstra}} पूर्वाचार्याणामुल्लेखः कृतः~।  \enginline{(Sharma\index{Sharma, Arvind} 2003:215)} 

न केवलं काव्येषु शास्त्रेषु चापि तु कलास्वप्यैतिहासिकांशनिरूपणमप्रतिहतमासीत्~। अत्रसन्दर्भे स्टैटेन्क्रानित्यनेन प्रतिपादिताः कलाकृतिषु लभ्याः ऐतिहासिकांशा आद्रियन्ते~। पल्लवनृपकालीनगङ्गाधरमूर्तिरूपणं समुद्रगुप्तस्यावदातकर्मणो शिल्पद्वारा निरूपणमुदयगिरौ चोल्लोखमत्रार्हतः  \enginline{(Sharma 2003:216)}

अत्र सन्दर्भे गवेषणीयोऽपरोंऽशोऽपि वर्तते~। तच्च कुतः खलु दक्षिणभारत इयन्ति शिलाशासनानि\index{inscriptions} पत्राणि लभ्यन्ते न पुनरुत्तरे भारत इति~। स्पियराख्यविदुषोऽभिप्रायं सङ्गृह्णन्नरविन्दशर्मा वक्ति -

\begin{myquote}
\eng{“[That inscriptions] relatively abound in those areas where Islamic rule took longest to penetrate, invites the proposition that they may also have suffered iconoclastic destruction, in keeping with the pattern of the relative paucity of such evidence from the Hindu period available from areas under prolonged Islamic rule”}

~\hfill \eng{(Spear (1994) cited in Sharma\index{Sharma, Arvind} 2003:13)}
\end{myquote}

तुरुष्कैः परधर्षणतत्परैर्विनाशकार्यमतितरां विततमिति सूचयति विट्जेलपि~। स आह -

\begin{myquote}
\eng{“In Nepal the temperate climate and the \textit{almost complete absence of Muslim incursions} worked together to preserve these old mss.”}

~\hfill \eng{(Witzel\index{Witzel, Michael} 1990:9) (\textit{italics ours})}
\end{myquote}

विट्ज़ेलवाक्यमनुमोदमानश्चारविन्दशर्मा तुरुष्कविहितध्वंसकार्यैकान्तिक्यं सुष्ठु स्फुटयति~।

\newpage

\begin{myquote}
\eng{ “An extreme case of the conspiracy of negative forces in relation to the manuscript tradition is provided by Kashmir, where ‘no mss. older than c. 1500 AD remain. Local Hindu and Muslim chroniclers agree in blaming the Sultans Sikander and Ali (1389-1419/20) for their wholesale destruction by burning and dumping them in the Dal Lake’.” }

\vskip -2pt

~\hfill \eng{(Sharma\index{Sharma, Arvind} 2003:212)}
\end{myquote}

दौर्जन्यैकनिलयैस्तैस्तथाविहितं तुरुष्कैरिति तुरुष्कैतिहासिकैरपि प्रतिपन्नमिति चात्र स स्पष्टं सूचयति~।

भारतीयं विज्ञानं तैरेव नाशितमिति अल्बिरूनीनामकेनैतिहासिकेन\index{Al Biruni} स्वतः प्रोक्तमिति सचौमतमुदाहरत्यरविन्दशर्मा~।

\begin{myquote}
\eng{“As a result of Maḥmūd’s devastating raids ‘Hindu sciences have retired far away from those parts of the country conquered by us, and have fled to places where our hand cannot yet reach, to Kashmir, Benares, and other places.’” }

~\hfill \eng{(Sachau\index{Sachau, E. C.} cited in Sharma 2003:212)}
\end{myquote}

यथा देवालयास्तथैव भारतीयकलाकृतयोऽपि तुरुष्कैर्नाशिताः इति विट्जेलप्यङ्गीकरोति\index{Witzel, Michael}

\begin{myquote}
\eng{“...Hindu historiography suffered serious obscuration during the period of Islamic occupation, as this period also involved the destruction of holy images and temples which were one form of material in which such history was preserved.” }

~\hfill \eng{(Sharma\index{Sharma, Arvind} 2003:220)}
\end{myquote}

अत्र श्रीवरस्य\index{Srivara@Śrīvara} राजतरङ्गिण्यां\index{Rajatarangini@\textsl{Rāja-taraṅgiṇī}} विद्यमानमेतद्वचनमुल्लेखार्हं यत्र ग्रन्थालयानामेवाग्निसात्करणं नीचैस्तुरुष्कैर्विहितमिति स्पष्टमुट्टङ्कितम् –

\vskip 2pt

\begin{verse}
सेकन्धरधरानाथो यवनैः प्रेरितः पुरा~।\\ पुस्तकानि च सर्वाणि तृणान्यग्निरिवादहत्~॥ (१.५.७५)
\end{verse}

\vskip 2pt

उपसमाप्ति लिखत्यरविन्दशर्मा स्वलेखे सम्पूर्णध्वंसकार्यस्य तुरुष्कनिष्पादितस्य स्वरूपं निरूपयन् यत्तद्द्धि पूर्णत्वं ध्वंसविजृम्भितस्य यद्ध्वंसकार्यं सञ्जातमित्यपि सूचिकाः नावशेष्यन्ते~। अर्थादीदृग्विधनिःशेषप्रमार्जनपटवस्तुरुष्का इति~।

\begin{myquote}
\eng{“The perfect genocide\index{perfect genocide@``perfect genocide''} is one which never occurred, because no one was left behind to tell the story. The point to be made is that the scale of destruction can be such as destroys the very evidence of that destruction. One then faces what might be called an evidentiary “black hole.”}

\vskip -2pt

~\hfill \eng{(Sharma\index{Sharma, Arvind} 2003:220)}
\end{myquote}

\eject

तक्षशिलानालन्दाख्यविश्वविद्यालयद्वयविध्वंसनं विहितं खलु तुलुुष्कैः~। तादृक्षाधुनिकविश्वविद्यालयस्य आक्स्फ़र्ड्-केम्ब्रिड्जाख्यस्य सग्रन्थालयस्य विध्वंसनं यद्यद्य विधीयते का तर्हि कथा\break स्यादाङ्ग्लेतिहासनिर्मितिशास्त्रस्येति प्रष्टव्यं भवति~।

\begin{myquote}
\eng{“By the end of the 12th century the two major universities of ancient India, those of Takṣaśilā\index{Taksasila@Takṣaśilā} and Nālandā\index{Nalanda@Nālandā} had disaapeared...What prospect would we hold out for British historiography in the future, if the Universities of Oxford and Cambridge\index{Oxford}\index{Cambridge} were utterly destroyed today along with all the libraries.” }

~\hfill \eng{(Sharma\index{Sharma, Arvind} 2003:222)} 
\end{myquote}

एतादृशमहत्तरग्रन्थराशिविनाशनोत्तरकालेऽप्यद्यापि कोटित्रयाधिकहस्तप्रतयः प्राधान्येन संस्कृतभाषया लभ्यन्त इत्युक्ते (गोयलप्रभृतयः \enginline{Goyal\index{Goyal, Pawan}
 \textit{et al} (2012))} कियान् पर्वताकारो ग्रन्थस्तोमः भारतैर्विरचितस्स्यादित्यूहनैकविषयः~।

स्वेतरसर्वसंस्कृतिविद्वेषिभिर्मुस्लिमैरेतादृक्कुत्सितधिक्करणीयकृत्येषु नित्यमुद्वृत्तं प्रवृत्तैः पाश्चात्त्यग्रन्थालया न पश्चात्काले नाशयिष्यन्त इति के नामाशंसीरन्प्रेक्षावन्तस्सप्रत्ययम्?

\vspace{-.35cm}

\section*{विमर्शनं मीमांसामतदूषणस्य}
\index{purva mimamsa@Pūrva Mīmāṁsā}

अथ पोल्लाकेनोत्थापितानामनेकेषां\index{Pollock, Sheldon} वेदमीमांसायाम\index{vedas@\textsl{Veda}-s}धिकृतानां प्रश्नानामुत्तमान्युत्तराण्यानन्दकुमारस्वामिनो \enginline{(Ananda Coomaraswamy\index{Coomaraswamy, Ananda K.} 1877-1947)} लेखेषु लभ्यन्ते~। वेदा\index{vedas@\textsl{Veda}-s} वा तदङ्गभूतानि शास्त्राण्यन्यानि वा तत्समकक्षानि भगवतो निःश्वसितानीति वा व्याहृतय इति वा निर्दिश्यन्ते ननु \enginline{(Coomaraswamy 1934:175)}~। ते चादावृषिभिः श्रूयन्ते~। ऋषीणामपि श्रवणं स्वमतिस्फूर्तिनिबन्धनतापेक्षयाप्यन्तस्समाहिततानिबन्धनमेवेति विषये न विशेरते विद्वन्मणयः~। वाल्मीकिरपि\index{Valmiki@Vālmīki} सर्वं रामायणं\index{Ramayana@\textsl{Rāmāyaṇa}} योगदृष्ट्या विलोकयति स्म यत्र चिरनिर्वृत्तानि वृत्तान्तान्यपि प्रत्यक्षमिव\break दर्शितानि भवन्ति। तादृक्षस्य प्रतिभानस्य मूलं चर्ग्वेदेऽप्यक्षिलक्षीभवति~। सन्दर्भेऽस्मिन् ब्लूम\-फील्डस्या\enginline{(Bloomfield)}\index{Bloomfield, Maurice}भिप्रायमानन्दकुमारस्वामी पुनरुच्चरति~। मन्त्रब्राह्मणे\index{brahmana@\textsl{brāhmaṇa}}\index{mantra@\textsl{mantra}} भिन्नकालिक\break इत्याधुनिकानामाग्रहः खलु~। तयोर्भिन्नकालिकत्वमकिञ्चित्करम्~। वस्तुतस्तु \hbox{ब्राह्म\-स्यैव}\break वाङ्मयस्य प्रकारद्वयदेश्य एव ते~। तच्च प्रकारद्वयं समकालिकतयैव विभावितमा च बहोः\break कालात्परम्परायां भारतीयायाम्~। एतच्च सर्वं तु ब्रौनाभिमतेनात्यन्तं भिन्नम् \enginline{(Norman Brown)}\index{Brown, Norman} यस्तावदाह यदृग्वेदे\index{Rg Veda@\textsl{Ṛg Veda}} वस्तुतोऽनुल्लिखितानामेव विषयाणां प्रसरणमुत्तरस्मिन्काले लोचनगोचरीभवतीति \enginline{(“The later material is so liable to follow ideals not really in the Ṛgveda”\index{Rg Veda@\textsl{Ṛg Veda}} Brown 1931:108)}~। एतत्प्रतिद्वन्द्वितयो उपनिषत्स्वपि नूतनास्सिद्धान्ता नाविष्क्रियन्त इत्याह कुमारस्वामी, किंतर्हि नूतना विभिन्ना वा शब्दावलिरेव तत्र प्रयुज्यमाना लक्ष्यत इति~। उदाहरणार्थं यं वरुणमाहुर्वेदे तमेव ब्रह्माणमुत्तरस्मिन् काले जगदुः~।

एतावता भाषिकी विभिन्नता परिष्कृतता वा नास्त्युपनिषत्सु वैदिकापेक्षयेति नोररीकरणीयापतति~। तथा त्वास्थातुं प्रेक्षावता केन नाम प्रक्रम्येत~।

\begin{myquote}
\eng{“It is not, of course, intended to deny that there is a linguistic development in the Upaniṣads,\index{Upanisads@\textsl{Upaniṣad-s}} when we compare them with Ṛgveda,\index{Rg Veda@\textsl{Ṛg Veda}} which denial would be absurd...”}

~\hfill \eng{(Coomaraswamy\index{Coomaraswamy, Ananda K.} 1935a:411)}
\end{myquote}

साहित्येतिहासोऽपरस्तत्त्वशास्त्रीयेतिहासोऽपरः~। यदेव वेदेष्वधियज्ञतयाभिहितं, तदेव ब्राह्मणेषूपनिषत्सु चाध्यात्मपरतयोपदिष्टं भवति~। न ह्यधियज्ञे\index{yajna@\textsl{yajña}} निरूप्यमाणे साहित्येऽधितत्त्वमपि तावत्यैव स्फुटतया निरूपितं भवत्वित्याग्रहो ग्रहीतव्यः~। वैदिकं नाम वाङ्मयमतिविस्तृतं सदपि नह्यान्तरिकः कोऽपि विरोधस्तत्र कुत्रचिदधिगम्यत इति नगरियसो विस्मयस्य विषयः~।

\begin{myquote}
\eng{“It is true that the material is so extensive, and so infallibly consistent with itself... it is by no means impossible to extract from the mantras\index{mantra@\textsl{mantra}} the doctrines assumed in them.”}

~\hfill \eng{(Coomaraswamy 1935a:412)} 
\end{myquote}

यथा ब्लूमफील्ड आह – वेदस्य\index{vedas@\textsl{Veda}-s} सर्वोऽपि भागः सर्वमितरं भागं सम्यगेव वेत्ति, सर्वस्सर्वेण च सुसम्बद्ध एव संलक्ष्यत इति~। अतो यज्ञेभ्यस्तत्त्वानि\index{yajna@\textsl{yajña}} तत्त्वेभ्यश्च यज्ञानवसातुं न न पारयन्ति मनीषिणः~।

\begin{myquote}
\eng{“[I have] a growing faith in the synchronism of \textit{mantra, brāhmaṇa}\index{mantra@\textsl{mantra}}\index{sutra@\textsl{sūtra}}\index{brahmana@\textsl{brāhmaṇa}} and \textit{sūtra...mantra} and \textit{brāhmaṇa} are for the least part chronological distinctions; that they represent two modes of literary activity, and two modes of literary speech, which are largely contemporaneous, the \textit{mantra} being the earliest lyric and the \textit{brāhmaṇas} the earliest epic-didactic manifestation of the same cycle of thought. Both forms existed together, for aught we know, from earliest times; only the redaction of the \textit{mantra}\index{mantra@\textsl{mantra}} collections in their present arrangement seems on the whole to have preceded the redaction of the \textit{brāhmaṇas...}”}

~\hfill \eng{(Bloomfield\index{Bloomfield, Maurice} 1893:144)}
\end{myquote}

एड्जर्टनोऽप्याद्यासूपनिषत्सु निगदितानां तत्त्वानां समेषां निदानस्याम्नायवाङ्मयानुपलम्भनीयत्वस्य निराकरणं कण्ठोक्तं प्रवक्ति \enginline{(Edgerton\index{Edgerton, Franklin} 1916:197)}।

\enginline{“The more I study the Upaniṣads,\index{Upanisads@\textsl{Upaniṣad-s}} the more I become impressed... [that] every idea contained in at least the older Upaniṣads, with almost no exceptions, is not new to the Upaniṣads, but can be found set forth, or at least \textit{very} clearly foreshadowed, in the older Vedic texts.”}

अर्थाद्ये हि नाम मन्त्रकृतो मानुषा वा अतिमानुषा वा स्वोक्तिभिरवसेयानंशान् सम्यगेव प्रत्यपद्यन्तेत्येव वक्तव्यं भवति~। \enginline{(Coomaraswamy\index{Coomaraswamy, Ananda K.} 1935a:412)} नो चेद् गणितसूत्राणि बहूनि केनचन कथञ्चिदज्ञात्वैव विलिखितानीति ब्रुवतो वचने यस्साहसस्स एव वक्तव्य आपतिष्यति~। तथापि च तस्य भाषिकी तात्त्विकी चान्तस्स्फूर्तिरभ्युपगन्तव्या भविष्यति \enginline{(Coomaraswamy 1935a:412)}~।

\enginline{“what in fact the consistency proves is that those who composed the \textit{mantras},\index{mantra@\textsl{mantra}} whether human or superhuman beings, must have been fully aware of all their implications, or if not it would be as if we had come upon a series of elegant mathematical formulae, and yet believed that they had been written down blindly, which is as much to say under verbal as well as theoretical inspiration.”}

वेदेषु ताक्षिकं ज्ञान\enginline{(knowledge of carpentry)}मभिलक्ष्यत इत्यनेन हेतुना, लोकेऽपि तादृशस्य ज्ञानस्य पश्चादेव तथा वचनं शक्यसंभवमिति हेतोश्च तत्रत्यं साहित्यं काथञ्चित्कं मानुष्यकमिति चैतिहासिकमिति चाभ्युगन्तव्यमेव~। \enginline{(Coomaraswamy\index{Coomaraswamy, Ananda K.} 1935a:412)}

\enginline{“it is impossible to suppose that the Veda in its present form could have antedated, let us say, a knowledge of carpentry, which means that the \textit{ipsissima} verba of the Veda, as distinct from their references, must be thought of as in some sense of human and temporal origin.”}

सनातनस्य धर्मस्य\index{sanatanadharma@\textsl{sanātana-dharma}} सनातनत्वं नाम नहि वेदगतानां\index{vedas@\textsl{Veda}-s} शब्दानां तथात्वेनाभिसन्धानं किंतर्हि तत्रत्यानां तत्त्वानां चिरन्तनत्वम् \enginline{(Coomaraswamy 1935a:412)}~। वैदिकसाहित्यकालसंसूचनमात्रादेव वेदानां\index{vedas@\textsl{Veda}-s} सनातनत्वं न विहन्यते~।

\enginline{“It is not with respect to the words in which it is recorded that the \textit{sanātana dharma}\index{sanatanadharma@\textsl{sanātana-dharma}} is eternal; the “eternity” of tradition has nothing to do with the possible “dating” of a given scripture as late as the first millennium BC.”}

ऐतिहासिकस्य क्रमस्य विषयेऽप्यानन्दकुमारस्वामी वक्तुमेवमभिलषति यत् तात्त्विकं नाम विभावनं पाश्चात्त्यैरादृतेन क्रमेण तावद् विवर्धमानं लक्ष्यते~। किंच साधारण्येनोच्यमाने, बहुत्र चापि, पारम्परिक्युक्तिरर्वाक्काले स्वस्या उल्लिखिततामेव सूचयति, न पुनस्स्वस्या ऐदम्प्राथम्यमाविष्कारस्य, यतस्ततोऽपि पूर्वं तस्या मौखिकः प्रचार एव वरीवर्ति स्म, येन च हेतुना तस्य चायमेव कर्तेत्यास्थातुं प्रायेण नैव शक्यं स्यात्~। विषयेऽस्मिन् रेने ग्वेनोन् \enginline{(Rene Guenon)}\index{Guenon, Rene} इत्यस्य मेधाविनोऽभिप्रायाणां साङ्गत्यं स निरूपयति \enginline{(Coomaraswamy 1947:73)}।\index{Coomaraswamy, Ananda K.}

\vskip 2pt

\enginline{“On the limitation of the historical method, Cf. Rene Guenon, \textit{Introduction to the Study of Hindu Doctrines}, 1945 pp. 18, 20, 58, 65, 237, 300. Historical method is only of limited value here, partly because metaphysical doctrines, ‘do not ‘evolve’ in the Western sense of the word,’ and partly because ‘in a general way and in most instances a traditional text is no more than a recording, at a relatively recent date, of a teaching which was originally transmitted by word of mouth, and to which an author can rarely be assigned’.”}

\vskip 2pt

नैरुक्तिकैतिहासिकमताभिवीक्षणं ब्लूमफील्डेनापि पोल्लागभिप्रेतोच्चाटकत्वेनैवेव\index{Pollock, Sheldon} विहितमिति विदाङ्कुर्वन्तु विद्वांसः \enginline{(Bloomfield\index{Bloomfield, Maurice} 1893:186)}

\vskip 2pt

\enginline{“The Indian \textit{nairuktas} and \textit{aitihāsikas}, and after them the commentators, never hesitate to urge the primary naturalistic conceptions which they have established somewhere or the other, correctly or incorrectly, through every legend which they have occasion to present. Western interpreters have... largely fallen into the error of marking pretty nearly every legendary narrative the \textit{corpus vile} of naturalistic anatomy.”}

\vskip 2pt

अन्येभ्यो देशेभ्य आगता “आर्या”\index{Aryan Invasion Theory} इति कश्चन पाश्चात्त्यो वादोऽपि वर्तते खलु! यमधिकृत्य ब्रुवन् कुमारस्वाम्याह यत्तादृशं लौकिकीकृतं व्याख्यानं (यस्य च \enginline{euhemeristic\index{Euhemerization} interpretation} इत्यभिधानं वर्तते) तावन्मात्रं स्यात्, यस्य तु वस्तुत ऐतिहासिकस्सारो न कश्चिद् वर्तत इति \enginline{(Coomaraswamy\index{Coomaraswamy, Ananda K.} 1935b:vii)}~। वैदिकोक्तकथानामनुहारिण्य ऐतिहासिक्यो घटना न जात्वसम्भवा इति नास्माकमभिप्रायो यतो ह्यैतिहासिकमपि नाम वस्तु तात्त्विकमेवानुहरेत्तत्र तत्र~। नेदमप्यतथ्यं यत्तत्त्वैकपरेऽपि हि साहित्य ऐतिहासिका अंशा दुरूहा इति~। यज्ञकार्याणि\index{yajna@\textsl{yajña}} कुर्वाणैर्मन्त्रगानं च कुर्वद्भिर्वैदिकैरश्वा वा रथा वा न विदिता इति वक्तुं न पार्यते नाम, न वा तैर्नानुभूतं नदीनां समुद्राणां वा तरणमिति, न वा कृषिस्तैरविदिता चेति~।

\vskip 2pt

कुमारस्वामिना तावदिदमास्थितं यदृग्वेदादिषु मूलग्रन्थेष्वैतिहासिका एव विषया निरूपिता इति न वक्तुं पार्यते~। किंतर्हि “अग्रे” इत्युक्तदिशा तात्त्विकेन प्रकारेण~। “अग्रे” इति तु तात्त्विकं वचनं न वास्तविकङ्कथनम्~। जीवनं हि नाम सर्वदा तरणमेव, सर्वदापि कुतश्चिदिहागमनमेव, इतश्च परमं पदं प्रति प्रस्थानमेव~। पूर्वमीमांसाया\index{purva mimamsa@Pūrva Mīmāṁsā} आशयो नामेदृश एव, परन्तु स्वतन्त्रेण प्रकारेणात्र निरूपित इति लेखं स्वं समापयति कुमारस्वामी \enginline{(Coomaraswamy\index{Coomaraswamy, Ananda K.} 1935b:25)}~।

इतीत्थं नामोत्तरकालीनानां पोल्लाकवादानामुत्तमान्युत्तराणि\index{Pollock, Sheldon} स्वतःपूर्वपक्षीकृत्येव\index{purvapaksa@\textsl{pūrvapakṣa}} पूर्वकालीनेन कुमारस्वामिनापूर्वयोत्तमया भङ्ग्या प्रत्तानीति शम्॥

\begin{center}
“निष्कारुण्यतमैस्तुरुष्कयवनैर्निष्कारणद्वेषिभिः”
\end{center}
\egroup

\section*{आकरग्रन्थाः}

\retainauthsanskrit

\begin{thebibliography}{99}
\itemsep=2pt
\bibitem{chap4-key01} \enginline{\textbf{\textit{Arthaśāstra}}. See Venkatanathacharya (1960).}

~\phantom{Arth}~\;\enginline{See Kangle (1988).}

 \bibitem{chap4-key02} \enginline{Auboyer, Jeannine. (1961). \textit{Daily Life in Ancient India}. (Trans. by Taylor, S. W. 1965$^{1}$, 2002). London: Phoenix Press.}

 \bibitem{chap4-key03} \enginline{Basham, A. L. (1967$^{2}$). \textit{The Wonder That was India}. London: Picador.}

 \bibitem{chap4-key04} \enginline{Beal, Samuel. (1969, 1884$^{1}$). \textit{Buddhist Records of the Western World\break Translated from the Chinese of Hiuen-Tsiang (AD 629)}. Delhi: Munshiram Manoharlal.}

 \bibitem{chap4-key05} \enginline{Bloomfield, Maurice. (1893). “Contributions to the Interpretation of the Veda". \textit{Journal of American Oriental Society}. Vol. 29. pp. 143--189.}

 \bibitem{chap4-key06} \enginline{Brown, Norman W. (1931). “The Sources and Nature of puruṣa in the Puruṣasūkta (Rigveda 10. 91)”. \textit{Journal of the American Oriental Society}, Vol. 51, No. 2. pp. 108--118.}

 \bibitem{chap4-key07} \enginline{\textbf{\textit{Chandogya Upaniṣad.}} See Limaye et al.}

 \bibitem{chap4-key08} \enginline{Coomaraswamy, Ananda Kentish. (1934). \textit{The Transformation of Nature in Art}. New York: Dover Publications.}

 \bibitem{chap4-key09} \enginline{---. (1935a). “Angel and Titan". \textit{Journal of American Oriental Society}. Vol. 55. No. 4. pp. 373--419.}

 \bibitem{chap4-key10} \enginline{---. (1935b). \textit{The Ṛg Veda as Land-Nama-Bok}. London: Luzac and Co.}

 \bibitem{chap4-key11} \enginline{---. (1947). \textit{Time and Eternity}. Ascona (Switzerland): Artibus Asiae.}

 \bibitem{chap4-key12} \enginline{Edgerton, Franklin. (1916). “Sources of the Philosophy of the Upaniṣads." \textit{Journal of American Oriental Society}. Vol. 36. pp. 197--204.}

 \bibitem{chap4-key13} \enginline{Frolov, I. (1984, 1980$^{1}$). \textit{Dictionary of Philosophy}. Moscow: Progress Publishers.}

 \bibitem{chap4-key14} \enginline{Ghosh, Ranjan. (2007) “India, Itihasa, and Inter-Historiographical Discourse”. \textit{History and Theory}, Vol. 46, No. 2. pp. 210--217.}

 \bibitem{chap4-key15} \enginline{Ghoshal, U. N. (1959). \textit{A History of Indian Political Ideas}. London: Oxford University Press.}

 \bibitem{chap4-key16} \enginline{Goyal, Pawan., Huet, Gérard., Kulkarni, Amba., Scharf, Peter., and Bunker, Ralph. (2012). “A Distributed Platform for Sanskrit Processing”. \textit{Proceedings of COLING 2012: Technical Papers}, COLING 2012 (2012). pp. 1011--1028.}

 \bibitem{chap4-key17} \enginline{Halbfass, W. (1988). \textit{India and Europe:} \textit{an Essay in Understanding}. Albany: State University of New York Press.}

 \bibitem{chap4-key18} \enginline{Heehs, Peter. (2003). “Shades of Orientalism: Paradoxes and Problems in Indian Historiography". \textit{History and Theory}, Vol. 42, No. 2. pp. 169--195.}

 \bibitem{chap4-key19} \enginline{Ingalls, Daniel H. H. (1965). \textit{Anthology of Sanskrit Court Poetry}. Cambridge: Harvard Oriental Series.}

 \bibitem{chap4-key20} \enginline{Joshi, D. S. (Ed.) (1922). \textit{Nyāyasūtras with Bhāṣya of Vātsyāyana and Vṛtti of Viśvanātha}. Poona: Anandashrama Mudranalaya.}

 \bibitem{chap4-key21} \enginline{Kangle, R. P. (1988, 1965$^{1}$). \textit{The Kauṭilīya Arthaśāstra}. (Three Parts.) Delhi: Motilal Banarsidass.}

 \bibitem{chap4-key22} \enginline{Kaul, Srikanth (Ed.) (1966). \textit{Rājataraṅgiṇī of Śrīvara and Śuka}. Hoshiarpur: Vishveshvaranand Institute.}

 \bibitem{chap4-key23} \enginline{Keith, A. B. (1920). \textit{A History of Sanskrit Literature}. London: Oxford University Press.}

 \bibitem{chap4-key24} \enginline{Limaye, Acharya V. P. and Vadekar, R. D. (Ed.) (1958). \textbf{\textit{Eighteen Principal Upanishads Vol. 1.}} Poona: Vaidika Samshodhana Mandala.}

\bibitem{chap4-key25} \enginline{{\it\bfseries Madhurāvijaya}. See Tiruvenkatachari (1957)}.

 \bibitem{chap4-key26} \enginline{\textbf{\textit{Manusmṛti.}} See Shastri (1983)}.

 \bibitem{chap4-key27} \enginline{\textbf{\textit{Nighaṇṭu and  Nirukta.}} See Sarup (1967).}

 \bibitem{chap4-key28} \enginline{\textbf{\textit{Nyāyasūtras and Bhāṣya.}} See Joshi (1922).}

 \bibitem{chap4-key29} \enginline{Organ, T. W. (1970). \textit{The Hindu Quest for the Perfection of Man}. Athens: Ohio University Press.}

 \bibitem{chap4-key30} \enginline{Pollock, Sheldon. (1989). “Mīmāṁsā and the Problem of History in Traditional India”. \textit{The Journal of American Oriental Society}, Vol. 109, No. 4. pp 603--610.}

 \bibitem{chap4-key31} \enginline{\textbf{\textit{Rājataraṅgiṇī}} of Śrīvara. See Kaul (1966).}

 \bibitem{chap4-key32} \enginline{Sachau, Edward C. (1914). \textit{Alberuni's India}. 2 vols. in 1. London: Kegal Paul, Trench, Trubner \& Co.}

 \bibitem{chap4-key33} \enginline{Sarup, Lakshman. (Ed.) (1967, 1927$^{1}$). \textit{Nighaṇṭu and Nirukta}. Delhi: Motilal Banarsidass. }

 \bibitem{chap4-key34} \enginline{Sathwalekar, Sripad Damodar (Ed.) (1957). \textit{Taittirīya Saṁhitā}. Paradi (Surat): Swadhyaya Mandala.}

 \bibitem{chap4-key35} \enginline{Sircar, D. C. (1977). \textit{Early Indian Numismatic and Epigraphical Studies}. Calcutta: Indian Museum.}

 \bibitem{chap4-key36} \enginline{Sharma, Aravind. (2003). ``Did the Hindus Lack a Sense of History?". \textit{Numen}, Vol. 50. No. 2. pp. 190--227.}

 \bibitem{chap4-key37} \enginline{Shastri, J. L. (Ed.) (1983). \textit{Manusmṛti of Manu with Manvarthamuktāvalī of Kullūkabhaṭṭa}. Delhi: Motilal Banarsidass.}

 \bibitem{chap4-key38} \enginline{Shastri, Vasudev Abhyankara (Ed.) (1967). \textit{Taittirīya Āraṇyaka with the commentary of Sāyaṇa}. Vol 1. Poona: Anandashrama Press. }

 \bibitem{chap4-key39} \enginline{\textbf{\textit{Taittirīya Saṁhitā}}. See Sathwalekar (1957).}

 \bibitem{chap4-key40} \enginline{\textbf{\textit{Taittirīya Āraṇyaka}}. See Shastri (1967).}

 \bibitem{chap4-key41} \enginline{Thapar, Romila. (1978). \textit{Ancient Indian Social History: Some Interpretations}. New Delhi: Orient Longman.}

 \bibitem{chap4-key42} \enginline{Tiruvenkatachari, S. (Ed.)(1957). \textit{Madhurāvijayam}. Annamalai: Annamalai University.}

 \bibitem{chap4-key43} \enginline{Venkatanathacharya, N. S. (Ed.) (1960). \textit{Kauṭalīyārthaśāstram}. Mysore: Oriental Research Institute.}

 \bibitem{chap4-key44} \enginline{Witzel, Michael. (1990). “On Indian Historical Writing: The Role of Vaṁśāvalis." \textit{Journal of the Japanese Association of South Asian Studies 2}. pp. 1--57.}

\end{thebibliography}

\enginline{\theendnotes}
