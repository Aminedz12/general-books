
\chapter*{Our Contributors \namesinorder{(in alphabetical order of last names)}}\label{contributors}

\vskip 4pt

\section*{Parthasarathy Desikan}

\vskip 4pt

Parthasarathy Desikan has four decades of experience in R\&D in\break petroleum and organic chemical industry after which he has been\break actively studying old classics in Tamil and Sanskrit, including many\break associated with Sanatana Dharma, Srivaishnavism and Vishishtadvaita. He has translated `The Light of Ramayana' by Justice Kodandaramayya from English into Tamil, `Vasantika Parinayam' from Sanskrit and\break Prakrit into English, the Tamil Epic `Silappadikaram' into English blank verse and Vedanta Desika’s Sanskrit poem `Sri Paduka Sahasram' along with its Tamil Bhashyam by Vidwan Tirumalai Thathacharya into English. He has recently published in e book format, a mini-purana in English titled, “A Spiral Fantasy.” The book aims to analyse the spiral nature of the path between a seeker and the Truth he seeks, or between a devotee and Bhagavan. The last sentence can be rewritten as ``He has contributed over one hundred blogs to \url{www.medhajournal.com}, apart from taking part in their forum discussions."

\vskip 4pt

\section*{A.V.Gopalakrishnan}

\vskip 4pt

A.V.Gopalakrishnan holds a Diploma in Mechanical Engineering and is currently practicing as a Bank Valuer. He is a researcher on Bible as well as Tamil literature and has presented a paper at Tamilnadu History Congress. He blogs on the Historicity of Bible and the Vedic nature of Sangam Tamil literature \url{@https://tamilsamayam.wordpress.com/}, \url{@http://pagadhu.blogspot.in/}, \url{@https://thamilkalanjiyam.blogspot.in/}.

\section*{Subhodeep Mukhopadhyay}

Subhodeep Mukhopadhyay is a writer and researcher in the area of Indology and Cultural studies and is the author of the book \textit{Ashoka the Ungreat}. He completed his certification in Sanskrit from Chinmaya Mission and Ramakrishna Vivekananda Center and is a student of Vedanta in the lineage of Swami Parthasarathy. He has a background in data science and finance, and consults for clients in the areas of financial inclusion, sustainable energy and civilizational studies. He writes extensively in IndiaFacts.org, Indology.in, Maitriwords.com and TheTinyMan.in. Proficient in finance as well as technology, he has prepared reports on West Bengal persecutions and the changing religious demographics of West Bengal. 

\section*{Nilesh Nilkanth Oak}

Nilesh Nilkanth Oak is an author and writes extensively on ancient Indian history at \url{www.nileshoak.wordpress.com}. He is working on multiple Indic projects (Documentaries, courses, textbooks, research publications). He conducts original research and has written three books in the past which are critically acclaimed. He published ‘When did the Mahabharata War Happen? The Mystery of Arundhati’ 2011 and the book was nominated for the Lakatos award, given annually by the London School of Economics. He published 'The Historic Rama (aka, 12209 BCE)' \& 'Bhishma Nirvana' in 2014, and in 2018, respectively. Nilesh holds an executive MBA (Emory University), an M.S. (University of Alberta) and B.S. (UDCT), both in Chemical Engineering and is Adjunct Faculty at School of Indic Studies, Institute of Advanced Sciences, Dartmouth, MA, USA. He resides in Atlanta, GA, USA.

\section*{Prakruti Prativadi}

Prakruti Prativadi is an aerospace/electrical engineer, working in the industry and on NASA projects. Prakruti is trained in classical Bharatanatyam by renowned Bharatanatyam gurus since she was a young child. She teaches Bharatanatyam at the school she founded, Kala Saurabhi Dance School, in the US and continues to actively perform in the US and in India. Also trained in Carnatic classical music and fluent in Tamil, Kannada and Sanskrit, she has spent seven years researching the Natyashastra, Abhinavabharati, the Upanishads, and other Sanskrit treatises on Indian art and aesthetics. Prakruti has written a book based on her research, titled “Rasas in Bharatanatyam”. Prakruti continues to explore the Vedas, Upanishads, and shastras and is releasing her second book this year.

\section*{K. Chitra Rao}

K. Chitra Rao is a graduate in History and has participated in many conferences including IHC, SIHC, TNHC, AIOC, BISY etc. She has contributed more than 500 letters and articles to leading newspapers such as The Hindu and The Indian Express etc. For the last thirty years she has been an active member of Bharathiya Ithihasa Sankalana Samithi and has participated in and assisted in organizing many conferences and workshops. Keenly interested in bhakthi songs, she runs a Bhajana mandali.

\section*{R. Saraswati Sainath}

R. Saraswati Sainath has a Doctorate in Sanskrit from University of Madras and is Assistant Professor, Department of Sanskrit at Dr. M.G.R. Janaki College of Arts and Science for Women, Chennai, having previously taught Sanskrit at Memorial University of Newfoundland, St. Johns as well as Sanskrit, Tamil and Bioethics at McGill University, Montreal. She has Diploma in Manuscriptology and has worked as Editorial Assistant for the New Catalogus Catalogorum Project of Madras University. She has several paper presentations and publications to her credit, spanning a variety of subjects such as “The Dance of Śiva as Portrayed by Appar in his Tevaram,” “Some Lesser Known Advaitins from Tamil Nadu,” and “Hinduism and the Ethics of Human Embryonic Stem Cell Research.” She is the author of the book “The Bhagavadgita: Path to Wisdom” and translated the story of Rama as it appears in the Mahabharata into Tamil. She is currently working on a book on the Tirumurai, translating some passages for the first time in print so far. Fluent in Tamil, English, Hindi, Sanskrit and French, she is also proficient in Carnatic Vocal Music and Veena.

\section*{B. Sankareswari}

B. Sankareswari has done her Masters in two ancient and living languages Tamil and Sanskrit, and her Doctorate on a topic that connects both the languages, namely, “Impact of Sanskrit Models in Tamil Grammar”. She has won many fellowships and merit awards in her academics and career. She has published articles in more than 17 journals, made more than 20 paper presentations in conferences, independently authored five books, edited four more and has guided over forty research students in their pursuit of M.Phils and Ph.Ds. Her current research is a Comparative Study of the linguistic framework and rules of Tholkappiyam and Ashtadhyaayi. While her prime topic of study is the comparative study of the language structure, grammar and literature of Sanskrit and Tamil, she is equally interested in topics such as ‘Influence of English in spoken Tamil’, ‘Educator’s role in teaching grammatical structure of languages’, ‘Simplistic Models in Teaching,’ to mention a few.

\section*{Dr. Shivshankar Sastry}

Dr. Shivshankar Sastry is a General Surgeon by profession, having done his MBBS and MS from JIPMER, Pondicherry and FRCS in Edinburgh. His interest in military aircraft dates from the age of 7, but this took a back seat during his medical education. His interests later enabled him to contribute to the building of the Indian military website "Bharat-Rakshak.com" and serve as forum moderator where he was able to fully expand on his interests which include, apart from military aviation, sociology, psychology, history and strategic affairs. He published an e-book titled "Pakistan - Failed Sate" in 2007. His interest in Indology has been shaped by his reading of Prof SN Balagangadhara and Sri Aurobindo and meeting with Sri Rajiv Malhotra at his Bengaluru book launches. He is currently working on a book which he hopes will lead to the fixing of an approximate date for the origins of the Sanskrit language.

\section*{Therani Nadathur Sudarshan}

Therani Nadathur Sudarshan is a computer scientist, programmer and hands-on technologist / engineer, start-up founder, entrepreneur and technology consultant. He is deeply interested in discovering the immense practicality of the Indian Knowledge Systems. His primary\break
 research interests lie in Symbol Systems for representation and\break
 intelligence-spanning man-made material systems (AI), naturally occurring systems (biological) and the Indic symbol systems. Sampradaya studies (Visistadvaita) and practices are part of his upbringing and immensely influence every activity. He is also an active participant in the vibrant temple cultural events at many Vishnu temples in Tamil Nadu.

\section*{Madhusudhan Therani}

Madhusudhan Therani has been leading technology development for more than 20 years in the machine learning,artificial intelligence\break
 space. He has a pedigree designing / building robots, hybrid controllers and the whole analog/digital space,enjoys coding/hacking and continues to publish.He has published 25+ research papers in information sciences and product design.He holds a PhD in Robotics and Automation from Carnegie Mellon University in addition to two Masters Degrees in Engineering.More available here https://www.linkedin.com/in/madhutherani\url{https://www.linkedin.com/in/madhutherani}. He has deep roots in the theories and practice of sanatana primarily through his sampradaya and feels that the need to develop a more critical understanding of the West using our own drishti is critical - not only for our own civilization - but for the future of humanity and our planet.

\section*{V. Yamuna Devi}

V. Yamuna Devi received Master of Philosophy and Doctorate Degrees in Sanskrit from Kuppuswami Sastri Research Institute, Chennai. With five years of teaching experience at Stella Maris College for women, she is currently working as Assistant Professor at KSRI. She has assisted in publications of KSRI, published research articles in reputed journals, and presented research papers in National and International conferences, the latest being 16th World Sanskrit Conference held in Bangkok, on topic “Some Astrological Charts in Sanskrit literature”. Her Thesis is published as two books - “Amarakoshodghatana of Kshiraswamin A Socio-Cultural Study” and “Aspects of Nature in Amarakoshodghatana of Kshiraswamin,” published by Karnatak Historical Research Society, Dharwad.

