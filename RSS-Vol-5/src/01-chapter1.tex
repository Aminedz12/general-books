\chapter{History in India: A Critique from the Perspective of Mīmāṁsā}\label{chapter1}

\Authorline{Shrinivas Tilak\footnote{pp 41--72. In Kannan, K. S. (Ed.) (2019). \textit{Swadeshi Critique of Videshi Mīmāṁsā}. Chennai: Infinity Foundation India.}}

\vskip -7pt

\hfill{\sl(\url{shrinivas.tilak@gmail.com})}

\section*{Introduction}

“He who controls the past controls the future. He who controls the present controls the past” famously wrote Eric Arthur Blair\index{Blair, Eric Arthur} (better known by his pen name George Orwell\index{Orwell, George}) in his novel \textit{1984}\index{1984@\textsl{1984}}.\endnote{Orwell\index{Orwell, George} was an English author and journalist who had also served as a police officer with the Indian Imperial Police in Burma (now Myanmar) from 1922--1927."} History no doubt is a powerful tool that makes available a storehouse of information about how people and societies behave. History helps us understand change and how the society one lives in came to be. Though this statement has global significance and application, cultures and societies perceive and relate to history differently.

The differing perspectives on history between the West and India therefore raise some difficult questions: How to recognize the historical sense of a society like India’s whose past is recorded in ways very different from Western conventions? Arthur A. Macdonell\index{Macdonell, Arthur A.} went so far as to declare that early India wrote no history because it never made any. He blamed the doctrine of \textit{karma}\index{karman@\textsl{karman}}, which gave Brahmins (“whose task it would naturally have been to record great deeds”) no incentive to record historical events (Macdonell\index{Macdonell, Arthur A.} 1996:10-11).

Sheldon Pollock\index{Pollock, Sheldon}, the Arvind Raghunathan Professor of Sanskrit and South Asian Studies, Columbia University, New York and General Editor of the Murty Classical Library of India, has raised similar questions and comments. With his involvement as participant in the Sanskrit Knowledge System on the Eve of Colonialism Project\index{SKSEC Project}, he has emerged, as it were, the guardian of India’s cultural, literary, and social past with considerable power to influence public policy in India and project its image in the world. His writings therefore deserve careful scrutiny.

What follows critically examines his views on the status of history in India using the traditional Indian format of public debate: \textit{Pūrvapakṣa}\index{purvapaksa@\textsl{pūrvapakṣa}} (factual presentation of opponent’s thesis), \textit{Uttarapakṣa}\index{uttarapaksa@\textsl{uttarapakṣa}} (critical examination and refutation of the thesis), and \textit{Siddhānta}\index{siddhanta@\textsl{siddhānta}} (statement on outcome of the exchange).


\section*{\textit{Pūrvapakṣa}}

Professor Pollock (hereafter Pollock) is graceful enough to acknowledge that history perhaps is not an appropriate expression to use in the context of India, because ‘history’ as a disciplinary subject is a product of Western scholarship and ideas. He frowns upon the search for instances of a sense of history by non-Western historians and intellectuals in their own traditions that typified European modernity, of its sense of skepticism, its individualism—the search for the Indian Vico\index{Vico, Giambattista}, the Chinese Descartes, the Arab Montaigne (Pollock 2007:380). Historical and historiographical awareness was not absent in India\break prior to the arrival of a European knowledge system under colonial rule. One would therefore expect a fair treatment of this topic in two very widely read and influential articles by Pollock dealing with history and historical consciousness\index{historical consciousness} in India.

The burden of his thesis may be summarized here: (1) By declaring the Veda-s as ‘authorless’\index{deemed authorless!Veda-s} and ‘timeless’, the Pūrva Mīmāṁsā\index{purva mimamsa@Pūrva Mīmāṁsā} thinkers deprived Indic intellectual, literary, and ritual texts of their temporality through a process Pollock calls ‘Vedicization’\index{vedicization@``Vedicization''} and (2) ‘Vedicization’ in turn deprived these texts of their historicality\index{lack of historicality!Veda-s} (see Pollock 1989, 1990). While such views have been received with welcome acceptance by left wing intellectuals in India and beyond; nuanced correctives have been suggested by other scholars - Dunkin Jalki\index{Jalki, Dunkin} (2013), E. M. Jan Houben\index{Houben, Jan E. M.} (2002), Roy Perrett\index{Perrett, Roy} (1999), and Romila Thapar\index{Thapar, Romila} (2002, 2013), to name a few.

After factually presenting below these findings of Pollock\index{Pollock, Sheldon} as \textit{Pūrvapakṣa}, my \textit{Uttarapakṣa} seeks to demonstrate that Pollock’s verdict on history in India\index{history in India} is at variance with how the past is understood, recorded, and contextualized as Itihāsa\index{Itihasa@Itihāsa} in India by Indians themselves.


\section*{I. Mīmāṁsā Imposed Vedicization}
\index{Mimamsa@textsl{Mīmāṁsā}, allegations on!imposed Vedicization}
\index{vedicization@``Vedicization''}


Indologists routinely invoke one or more of the six traditionally recognized \textit{darśana-}s\index{saddarsana@\textsl{ṣaḍdarśana}} in their quest to discover factors that positively or negatively influenced the ancient Indians in their attitudes towards life, their psyche, and socio-cultural ethos. Pollock\index{Pollock, Sheldon} selected the Pūrva Mīmāṁsā\index{purva mimamsa@\textsl{Pūrva Mīmāṁsā}} \textit{darśana}, which he describes as ‘a pedagogically and thus culturally normative discipline of Brahmanical learning’ as a tool to gain insights into the status of history in ancient India (Pollock 1989:607). The burden of his thesis is that the Pūrva Mīmāṁsā \textit{darśana} (\textbf{hereafter Mīmāṁsā}) successfully mediated the transformation of the ritual discourse into a discourse of social power to sustain the relations of domination constitutive of traditional Indian society, which are characterized by the systematic exclusion from property, power, and status of three-quarters of the population for more than two millennia (Pollock 1990:316).

According to Jaimini, author of the foundational text of the Mīmāṁsā (the \textit{Mīmāṁsā Sūtra}-s\index{Mimamsasutra@\textsl{Mīmāṁsā sūtra}} = \textbf{MS}), Veda-s are not the work of divine or human authors (\textit{apauruṣeya}\index{apauruseya@\textsl{apauruṣeya}}) (MS 1.1.5). Pollock interprets this claim to mean that the Veda-s were deemed to be authorless\index{deemed authorless!Veda-s} because otherwise they might be fallible, like other authored texts familiar to humanity. Since they are not a product of historical persons the Veda-s are also deemed to be timeless. He then uses this unique feature of the Veda-s to posit a timeless, uniform, and overarching system of Indic thought that was immune from the normal vicissitudes of temporality and historicality\index{lack of historicality!Veda-s} (Pollock 1989:607ff).

Pollock’s next logical step is to hold the Mīmāṁsā system responsible for emptying the Vedic canon of all historical consciousness\index{historical consciousness} as well as historical referential intention in India. The result was all other sorts of Sanskrit intellectual practices seeking to validate their truth-claims by their affinity to the Veda had perforce to conform to this new, special model of what counts as knowledge, and so to suppress or deny the evidence of their own historical existence. Such suppression took place even in the case of the discipline of Itihāsa\index{Itihasa@Itihāsa}, ‘history’ (Pollock\index{Pollock, Sheldon} 1989:609). Subsequently, virtually all Brahmanical learning in classical and medieval India came to view itself in one way or another as genetically linked to the Veda - a process, which Pollock calls ‘vedicization’\index{vedicization@Vedicization} (Pollock 1989:609).\endnote{In his 1990 article Pollock called it ‘ Vedicization’\index{vedicization@Vedicization} (Pollock\index{Pollock, Sheldon} 1990:328)}

History, concludes Pollock, is not simply absent from or unknown to Sanskritic culture; it is denied in favor of a model of truth that accorded history no epistemological value or social significance. Sanskritic culture lacks historical referentiality. There is not even a single passing reference to a historical person, place, or event. There is nothing in the ancient Sanskrit texts that, historically\index{lack of historicality!Veda-s}\index{vedas@\textsl{Veda}-s!lack of historicality} speaking matters, declares Pollock brazenly (Pollock 1989:606).\endnote{The last sentence of the abstract of the paper, however, ends with this line: ‘History, consequently, seems not so much to be unknown in Sanskritic India as to be denied' (Pollock 1989:603)."}

Standing on Pollock’s shoulders another scholar goes one up on him:

\begin{myquote}
“Mīmāṁsā scholarship [is] utterly irresponsible by any post-structuralist standards of cultural sensitivity, and could well be impeached as an epistemically violent enterprise, in that it effectively erases the worldview of the Vedic and Brahminical literature by reinscribing on it the presuppositions of classical sastric discourse. Mimamsa is not a hermeneutical enterprise, as scholars such as Othmar Gätcher [\textit{Hermeneutics\index{hermeneutics} and Language in Pūrva Mīmāṁsā: A Study in Śābara Bhāṣya\index{Sabara Bhasya@\textsl{Śābara Bhāṣya}}}. Delhi: Motilal Banarsidass, 1983] would have it… No responsible historian could claim that Kumarila\index{Kumarila@Kumārila} understood the Vedas any better than Friedrich Max Muller\index{Max Muller, Friedrich} who valorized the poetic essence of the Rg Veda\index{Rg Veda@\textsl{Ṛg Veda}} while infamously denouncing its mythological excursions as a “disease of language."

~\hfill (Fisher\index{Fisher, Elaine} 2008:8-9)
\end{myquote}


\section*{II. Mīmāṁsā Denied/Suppressed Historical Consciousness\index{historical consciousness}}

Pollock’s empiricist understanding of ‘no history’ in ancient India is based on his adherence to the scientific notion of time as a straight line that is constituted by succession of abstract ‘nows’ and distinguished by the intervals between them. He deems these two sets of relations sufficient to declare that there is no way of knowing (1) \textit{when} something of importance happened in early India, (2) what came \textit{earlier }or \textit{later} and (3) \textit{how long} this or that dynasty lasted. Under Pollock’s influence, the view that India and (especially) Hinduism have been largely devoid of historical writing and historical consciousness\index{historical consciousness} has become axiomatic in the fields of Indology and Indian history in the West as well as within India itself.

Pollock asserts that the cyclic concept of time (\textit{yuga}\index{yuga@\textsl{yuga}}) negated the difference between myth and history as well as negated the possibility of unique events (which form a precondition to historical time). For him, the \textit{yuga} theory of time is regressive (not progressive) in its teleological move from the Age of Kṛta\index{Krta@Kṛta} to the Age of Kali\index{Kali@Kali}. The notion that time, place, and causality merge with Brahmā at the end of each \textit{yuga} rejects individuality (\textit{ahaṅkāra}) as a causal factor. For Pollock the notion of \textit{mokṣa}\index{moksa@\textsl{mokṣa}} or \textit{nirvāṇa}\index{nirvana@\textsl{nirvāṇa}} transcends history, or it even denies history in a realm where the aim of life is to leave this material world. This philosophy of life-negation added to ‘anti-historical’ tendencies of ancient Indians (See Hossain\index{Hossain, Purba} 2016).

In sum, Pollock holds the Mīmāṁsā \textit{darśana} responsible for ‘vedicizing’\index{vedicization@Vedicization} the ancient Indic thought system and suppressing Itihāsa\index{Itihasa@Itihāsa} (which originally simply meant ‘what has actually taken place’) into mere textualization of eternity. Like language, which in the Mīmāṁsā view expresses the general (\textit{ākṛti}\index{akrti@\textsl{ākṛti}}) and not the particular (\textit{vyakti}\index{vyakti@\textsl{vyakti}}), Itihāsa became a reference for something that is eternally repeated. It was no longer contingent, the localized, and the individual: that is, the historical (Pollock 1989:610). What is worse, Mīmāṁsā\index{purva mimamsa@Pūrva Mīmāṁsā} ultimately furnished the Dharmaśāstra-s\index{Dharmasastra@Dharmaśāstra} a meta-legal framework for an explicit program of power to inculcate and legitimate such concrete modes of domination as caste hierarchy, untouchability, and female heteronomy (Pollock\index{Pollock, Sheldon} 1990: 336).


\section*{\textit{Uttarapakṣa}}

In his \textit{The Language of the Gods in the World of Men: Sanskrit, Culture, and Power in Premodern India} Pollock acknowledges that there is a natural tendency in social and cultural theory to generalize Western experience and familiar forms of life and experience as scientific descriptions, and as modes of understanding life tendencies across cultures (Pollock 2006:259; 19). Elsewhere in the same book he grants that one of the most serious conceptual impediments in understanding South Asian culture comes from the fact that tools deployed to understand it are shaped by ‘Western exemplars’ (Pollock\index{Pollock, Sheldon} 2006:274). A closer examination of Pollock’s various writings on history and culture in India, however, suggests that wittingly or unwittingly he reproduces in his writings some of the very same Eurocentric\index{Eurocentricity} formulations of the writing of history and modernity that he claims are not applicable to the situation in India.\endnote{For an in depth critique of Pollock on this point see Jalki 2013.}

Also implicit in Pollock’s\index{Pollock, Sheldon} philosophy of history is G. W. F. Hegel’s\index{Hegel, G. W. F.} argument that India had no history, which Hegel had predicated upon finding a necessary connection between history writing and history doing/making. History combines objective and subjective meanings (\textit{historia rerum gestorum} and \textit{res gestae}). It is not two different things that happen to have been given the same name. Rather, the two meanings are deeply connected, for the unity of history [writing] and the actual deeds and events of history make their appearance simultaneously, and they emerge together from a common source. This common source is the state, argued Hegel\index{Hegel, G. W. F.}, which supplies a context for history writing and helps produce it. History is the realm of self-conscious free choice and directional change creatively making the new and unprecedented; and the main locus of this creative making is the state (see Trautmann\index{Trautmann, Thomas} 2012:193).

As Pollock brings to bear on his accounts an outsider’s perspective, he is unable to show what ancient or pre-modern Indian readers themselves understood to be the interpretive protocols employed by the Mīmāṁsā thinkers in order to understand and present the \hbox{Veda-s} and the horizon they project. Where Pollock purports to know when a text is mythic or literary and not historical, he must warn his contemporary readers that this is his categorization that is being imposed on ancient texts that belong to an alien thought system. If not, Pollock’s assertion that some discourse is ‘mythic and not historical’ tells us more about Pollock, the interpreter, than about ‘historical consciousness\index{historical consciousness} of Indians in ancient India (based on Pollock 2007:379, see also Chekuri\index{Chekuri, Christopher} 2007).

Pollock is convinced that he can explain ahistoricity in the Vedic worldview and in the Indic tradition by reference to a generally accepted indigenous theory of cyclic time. True, the shape and accounting of time is essential to the writing of history of any nation. Like other Western Indologists he makes much of the alleged lack of history in India linking it to a cyclic concept of time\index{cyclicity of time}, which is contrasted (consciously or not) with the linear time of the Judeo-Christian tradition. A sharp dichotomy of linear and cyclic time that Pollock posits, however, is contestable because elements of each do overlap. Jan E. M. Houben\index{Houben, Jan E. M.}, for instance, acknowledges that the cyclic calendrical temporality presaged in the ritual system placed the ritual actor in a timeless reality, which did not stimulate any interest in detailed history. Yet, the narrativity and historicity of the world, he observes, is only temporarily set aside and that too in the case of the Brahmin priests and a few others directly involved in the performance of \textit{yajña}\index{yajna@\textsl{yajña}} (Houben\index{Houben, Jan E. M.} 2002:472).

Romila Thapar\index{Thapar, Romila}, on her part, is at pains to point out that linear and cyclic notions of time co-existed in ancient India, and were used according to the context. Even in cyclic time\index{cyclic time} \index{linear time} the present is not a repetition of the past, as has been maintained. Each cycle records change. The linear and cyclical views of time are not mutually exclusive — provided only a segment of the cycle is regarded or described. \textit{Sub specie aeternitatis}, of course, time was regarded as cyclical. While cyclic notion of time was included in \textit{Mahābhārata}\index{Mahabharata@\textsl{Mahābhārata}}, Dharmaśāstra-s\index{Dharmasastra@Dharmaśāstra}, and the Purāṇa-s\index{Purana@Purāṇa}; linear time was used in genealogies (\textit{vaṁśāvali}), biographies (\textit{carita}-s), and dynastic chronicles (Thapar 2002:26-45, Devy\index{Devy, G. N.} 1998:11). There is now growing recognition that cultures have their own versions of history which, however fanciful, reflect their particular perception of the past (Chakraborty\index{Chakraborty, Biswadeep} 2016).


\section*{III. Mīmāṁsā Hermeneutics Sustains Itihāsa}
\index{hermeneutics!Mimamsa@Mīmāṁsā}
\index{Itihasa@Itihāsa}

\subsection*{\textit{Mīmāṁsā hermeneutical principles}}

If mathematics is the source of all science in the West; in ancient India critical reflection (\textit{mīmāṁsā) }was the major source of hermeneutics and interpretive enterprise. The Mīmāṁsā\index{purva mimamsa@Pūrva Mīmāṁsā} \textit{darśana} arose in response to the need for an exegesis of Vedic \textit{yajña}\index{yajna@\textsl{yajña}}. Out of an analysis of the institution of \textit{yajña} came into being the Aitihāsika tradition of an indigenous school of history and historiography quite different from anything produced in the Western world (see below). Main philosophical inquiries of Mīmāṁsā have similarly developed out of Vedic exegetical themes. Its foundational text, the \textbf{\textit{Mīmāṁsā sūtra}}\index{Mimamsasutra@\textsl{Mīmāṁsā sūtra}} attributed to Jaimini\index{Jaimini} (perhaps 200 BCE), is probably the most ancient philosophical \textit{sūtra}. It has been commented on in \textit{Śābarabhāṣya}\index{Sabara Bhasya@\textsl{Śābara Bhāṣya}} (possibly 3rd to 5th century CE), which was again commented on by Kumārila\index{Kumarila@Kumārila} and Prabhākara\index{Prabhakara@Prabhākara} (700 CE?).

From the Mīmāṁsā perspective the events or stories to be found in the Veda-s serve to illustrate a specific purpose of the injunction associated with them (technically called \textit{arthavāda}\index{arthavada@\textsl{arthavāda}}). \textit{Arthavāda} is like the preamble or statement of objects in a statute. A statement of \textit{arthavāda }type has no legal force by itself, but it is not entirely useless since like a statement of objects or preamble it can help to clarify an ambiguous injunction (\textit{vidhi}\index{vidhi@\textsl{vidhi}}), or give a justifiable reason for it. Sometimes a \textit{vidhi} is also seen couched in the form of \textit{arthavāda}. A physician’s written prescription, for instance, comes with his license number entered as proof of his qualification to practice medicine. This assures the patient who then buys the prescription and gets well in course of time. A descriptive statement (\textit{arthavāda}\index{arthavada@\textsl{arthavāda}}) may be attached to a Vedic injunction for similar reasons (MS 1:2.1ff). 

A statement of \textit{arthavāda} is divided into three sub-types. First, the descriptive part of \textit{arthavāda} is called ‘\textit{guṇavāda}\index{gunavada@\textsl{guṇavāda}},’ which often seeks to enhance the meaning of a \textit{vidhi }statement with resort to metaphor. Second, ‘\textit{anuvāda}’\index{anuvada@\textsl{anuvāda}} means elaboration of something previously known, as in the phrase, ‘fire burns.’ Everybody knows fire burns—experientially. \textit{Anuvāda} implies putting this empirical truth formally as a proposition. The third is \textit{bhūtārthavāda}\index{bhutarthavada@\textsl{bhūtārthavāda}}, which in the above example would mean listing the ingredients of the medicine that is being prescribed along with details of dosage etc. In other words, the purpose of \textit{bhūtārthavāda} and \textit{guṇavāda }is to narrate a convincing account (centered on truth iterated by \textit{anuvāda}) to motivate followers to observe a prescribed rule or injunction. Consider for instance the injunction ``Do not drink liquor" whose purport (or moral) is that one must not drink. To narrate the ‘story’ that ‘a man who got drunk was ruined’ in order to expand this purport is \textit{arthavāda}\index{arthavada@\textsl{arthavāda}} (based on Subramanian 2010).


\subsection*{\textit{The Liṅga\index{linga@\textsl{liṅga}} principle}}

The Mīmāṁsā theory of interpretation is based upon an analysis of the imperative mood (\textit{liṅ}) because the core of the Veda-s is defined by commands to perform yājñic acts accompanied by recitation of sacred \textit{mantra}-s. In most cases the meaning of a Vedic injunction is clear on face value (see \textit{Śābarabhāṣya}\index{Sabara Bhasya@\textsl{Śābara Bhāṣya}} on MS 3.3.14). However, when the meaning of a word or expression is not clear on the face of it, its latent force or suggestive power is brought out with recourse to the suggestive power of some other word or expression associated with the injunction. This is called \textit{liṅga}\index{linga@\textsl{liṅga}}, which Kumārila\index{Kumarila@Kumārila} Bhaṭṭa pithily describes in one phrase — declaratory power of words (\textit{ukti-sāmarthya}). If Śruti\index{sruti@\textsl{śruti}} (= the Veda) offers the clear and obvious meaning of a word, \textit{liṅga} is recovery of a word’s obscure meaning by implication (Sarkar\index{Sarkar, Kishori Lal} 1909:126 citing Laugākṣī Bhāskara\index{Laugaksi Bhaskara@Laugākṣī Bhāskara}).

\subsection*{\textit{The Vākya principle}}

\textit{Vākya}\index{vakya@\textsl{vākya}} (a matter of syntactical arrangement) is called for when the word of a sentence needs to be read along with other words in the sentence in order to make the entire sentence meaningful. This procedure may involve (i) supplying ellipses (\textit{adhyāhāra}\index{adhyahara@\textsl{adhyāhāra}} and \textit{anuṣaṅga}\index{anusanga@\textsl{anuṣaṅga}}) and/or (ii) moving subordinate clauses in a sentence up or down (\textit{upakarṣa}\index{upakarsa@\textsl{upakarṣa}} and \textit{apakarṣa}\index{apakarsa@\textsl{apakarṣa}}) in order to provide proper context (Sarkar 1909:141).  According to the \textit{Śloka-vārttika}\index{Sloka-varttika@\textsl{Śloka-vārttika}} of Kumārila\index{Kumarila@Kumārila} Bhaṭṭa, one derives a special sense of the sentence upon examination of the structure of the sentence by using the principle of \textit{vākya}. Thus, in ‘\textit{vākya}’, the emphasis is on the inter-relationship between the words and clauses of a sentence (\textit{samabhivyāhāra}). The modern rule of \textit{noscitur a sociis} is also based on the same approach (Sarkar\index{Sarkar, Kishori Lal} 1909:109-110).\endnote{Ganganath Jha\index{Jha, Ganga Natha} calls it ‘syntactical connection’ (Jha 1964:220).}


\subsection*{\textit{The Prakaraṇa\index{prakarana@\textsl{prakaraṇa}} principle}}

Laugākṣī Bhāskara\index{Laugaksi Bhaskara@Laugākṣī Bhāskara} defines \textit{prakaraṇa} as the inter-relationship between passages (\textit{ubhayākaṅkṣā prakaraṇam}; Jha\index{Jha, Ganga Natha} 1964:220). It is based on the recovery of the latent or implicit relation of ideas, which must have been present to the mind of the author (Jha calls it the principle of context; see Sarkar 1909:106). Thus, a paragraph or clause when read by itself, does not clearly indicate its purpose, yet becomes clear when read with paragraphs belonging to another topic in the same text (or even in other texts) being discussed.


\subsection*{\textit{The Atideśa\index{atidesa@\textsl{atideśa}} principle}}

The method of performance of most \textit{yajña-}s\index{yajna@\textsl{yajña}} is given in clear and extensive details in the Brāhmaṇa texts. These are known as Prakṛti\index{Prakrti@\textsl{Prakṛti}}-yajña-s\index{yajna@\textsl{yajña}}\index{yajna@\textsl{yajña}!Prakrtiyajna@Prakṛti \textsl{yajña}}. However, there are other \textit{yajña-}s whose rules are not given anywhere, and which therefore are known as Vikṛti\index{Vikrti@\textsl{Vikṛti}}-yajña-s\index{yajna@\textsl{yajña}}\index{yajna@\textsl{yajña}!Vikrtiyajna@Vikṛti \textsl{yajña}}. The \textit{atideśa} principle was created to resolve this difficulty according to which the Vikṛti \textit{yajña} is to be performed according to the rules of the Prakṛti-yajña-s (See MS\index{Mimamsasutra@\textsl{Mīmāṁsā sūtra}} 7:1.12; Katju\index{Katju, Markandeya} 1993).


\subsection*{\textit{Application of Mīmāṁsā principles}}

Application of the \textit{liṅga}\index{linga@\textsl{liṅga}} principle (also called \textit{lakṣaṇārtha} = the suggestive power of the words or expression) can be illustrated with reference to a decision of the Supreme Court in U.P Bhoodan Yagna Samiti vs. Brij Kishore Case where the words ‘landless persons’ were held to refer to landless peasants only and not to landless businessmen (Barhi-nyāya). In Sardar Mohammad Ansar Khan vs. State of U.P.15, the occasion was as to which of the two clerks appointed on the same day in an Intermediate College would be senior, and hence entitled to promotion as Head Clerk. Controversy arose because there was no rule to cater to this situation. Recourse was therefore had to Chapter 2, Regulation 3 of the U.P. Intermediate Education Regulations, which states that where two teachers are appointed on the same day, the senior in age will be senior. Using the principle of \textit{atideśa}\index{atidesa@\textsl{atideśa}} it was held that the same principle which applies to teachers should be also applied to clerks, and hence the senior in age would be deemed senior (Katju\index{Katju, Markandeya} 1993).

\subsection*{\textit{Yuga\index{yuga@\textsl{yuga}}: the temporal principle of itihāsa\index{Itihasa@Itihāsa} (apauruṣeya\index{apauruseya@\textsl{apauruṣeya}} and pauruṣeya)}}

A \textit{sūkta} from the \textit{Ṛgveda}\index{Rg Veda@\textsl{Ṛg Veda}} affirms that the world issues forth periodically by the will of Vidhātā (Ordainer) as creation (\textit{sarga}) just as it did previously (\textit{dhātā yathāpūrvam akalpayat}; \textit{Ṛgveda} 10.190.3). The concept of ‘period’ is elaborated in the \textit{Manusmṛti}\index{Manusmrti@\textsl{Manusmṛti}} \index{smrti@\textsl{smṛti}} in terms of \textit{yuga}: worlds arise and dissolve and arise again and again through the four \hbox{\textit{yuga-s.}} \textit{Yuga} also configures the relation of time and ultimate truth and reality (\textit{brahman}\index{brahman@\textsl{brahman}}) that is deemed to be invariable (\textit{nitya}). Duration of each of the \textit{yuga-}s\index{yuga@\textsl{yuga}} is a decreasing number of human years in thousands: Kṛta\index{Krta@Kṛta} 1,728,000 human years, Tretā\index{Treta@Tretā} 1,296,000, Dvāpara\index{Dvapara@Dvāpara} 864,000, and Kali\index{Kali@Kali} 4,320,000 human years. The twelve thousand divine years (which are the total of four human ages) make one age of the gods, a \textit{mahā-yuga}\index{maha-yuga@\textsl{mahā-yuga}} (‘great age’). One thousand of these ages of the gods make a day of Brahmā the Creator, whose night is also of equal length. Known as Kalpa, this period comes to twelve million years of the gods, or 4.32 thousand million human years. Waking at the end of his day-and-night, Brahmā creates [his] mind, which brings forth creation by modifying itself, impelled by Brahma’s creative desire (\textit{Manusmṛti}\index{Manusmrti@\textsl{Manusmṛti}} \index{smrti@\textsl{smṛti}} 1:68–80).

Invariable (\textit{nitya}) truth is realized (and revealed to humanity) through the \textit{yuga-}s by \textit{devatā-}s (such as Agni and Vāyu) and by \textit{ṛṣi-}s and \hbox{\textit{muni-}s}. Their realization of truth is recorded in the form of episodes or dialogues, and preserved in a seed or root form\endnote{Commenting on \textit{Ṛgveda}\index{Rg Veda@\textsl{Ṛg Veda}} (1:1.1) Sāyaṇācārya\index{Sayanacarya@Sāyaṇācārya} states that at the end of every yuga\index{yuga@\textsl{yuga}} the great sages obtained the hidden Vedas along with itihāsa (see Singhal\index{Singhal, K. C.} and Gupta\index{Gupta, Roshan} 2003:23).} (I venture to suggest, as \textit{apauruṣeya}\index{apauruseya@\textsl{apauruṣeya}} Itihāsa\index{Itihasa@Itihāsa}\index{Itihasa@Itihāsa!apauruseya@\textsl{apauruṣeya}}). Though there are references to past events in the Veda, Veda-s are not themselves historical because these events may be repeated as such across the \textit{yuga-}s\index{yuga@\textsl{yuga}}. Humans, too, migrate through a particular \textit{yuga} in which they were born from life to life, body to body, seeking \textit{mokṣa}\index{moksa@\textsl{mokṣa}} to escape from the rounds and cycles of time. In this they receive guidance and instruction from heroic actions performed by \textit{avatāra}-s\index{avatara@\textsl{avatāra}} of Viṣṇu (Rāma and Kṛṣṇa for instance) whose deeds are recorded as \textit{pauruṣeya}\index{pauruseya@\textsl{pauruṣeya}} Itihāsa\index{Itihasa@\textsl{Itihāsa}}\index{Itihasa@Itihāsa!pauruseya@\textsl{pauruṣeya}} (please note that the categories of \textit{apauruṣeya}\index{apauruseya@\textsl{apauruṣeya}} Itihāsa and \textit{pauruṣeya} Itihāsa are my suggestion; they are not so described in the traditional texts).

In his commentaries on major Upaniṣads, Śaṅkarācārya\index{Sankaracarya@Śaṅkarācārya} analyzes and brings out salient facts about Itihāsa from the various episodes featured in them. The \textit{Kaṭha Upaniṣad}\index{Katha Upanisad@\textsl{Kaṭha Upaniṣad}}, for instance, features an episode involving Vājaśravas, and his son Naciketas who received instruction from Yama. Śaṅkarācārya explains that this episode functions as \textit{arthavāda}\index{arthavada@\textsl{arthavāda}}, that is, it pertains to an event that may actually have happened or not.  The first part of the proposition—‘an event that may actually have happened’ refers to the \textit{bhūtārthavāda}\index{bhutarthavada@\textsl{bhūtārthavāda}} component of \textit{arthavāda} (see above); it also comes closest to the Western notion of history because it occurs in time that is measurable, is connected to a probable causal factor, and is verifiable against an empirical criterion such as an inscription or a written record. The second part of the proposition is closer to what in the West is known as myth, legend, or fiction.  This episode is nevertheless instructive, insists Śaṅkarācārya, in that there is something to learn from the behavior or actions of characters involved (such as Naciketas) (based on Subramanian 2010). 

It should be noted here that Śaṅkarācārya\index{Sankaracarya@Śaṅkarācārya} does not vitiate the distinction between ‘Vedic (i.e. \textit{apauruṣeya}\index{apauruseya@\textsl{apauruṣeya}}) Itihāsa’\index{Itihasa@Itihāsa}\index{Itihasa@Itihāsa!apauruseya@\textsl{apauruṣeya}} and ‘worldly (i.e. \textit{pauruṣeya}\index{pauruseya@\textsl{pauruṣeya}}) Itihāsa\index{Itihasa@Itihāsa}\index{Itihasa@Itihāsa!pauruseya@\textsl{pauruṣeya}} .’ He distinguishes episodes from the Śruti\index{sruti@\textsl{śruti}} from those occurring in texts of worldly (\textit{laukika} or \textit{pauruṣeya)} Itihāsa such as the \textit{Rāmāyaṇa}\index{Ramayana@\textsl{Rāmāyaṇa}} or the \textit{Mahābhārata}\index{Mahabharata@\textsl{Mahābhārata}}, which are known products of recognized composers and authors like Vālmīki\index{Valmiki@Vālmīki} and Vyāsa\index{Vyasa@Vyāsa} that belong to a specific period: the Tretā\index{Treta@Tretā}-yuga\index{yuga@\textsl{yuga}} and Dvāpara\index{Dvapara@Dvāpara}-yuga respectively. He distinguishes them in the manner an injunction (\textit{vidhi}\index{vidhi@\textsl{vidhi}}) is distinguished from didactic material pertaining to the injunction (\textit{arthavāda}) in the Mīmāṁsā\index{Mimamsa@Mīmāṁsā} hermeneutics\index{hermeneutics}\index{hermeneutics!Mimamsa@Mīmāṁsā} (discussed above).

\subsection*{\textit{Itihāsa\index{Itihasa@Itihāsa}: a broad and inclusive category}}

Pollock\index{Pollock, Sheldon} seizes on the distinction Kumārila\index{Kumarila@Kumārila} Bhaṭṭa made between the transcendent disciplines that were ‘independently authoritative’ (\textit{adṛṣṭārthaka}\index{adrstarthaka@\textsl{adṛṣṭārthaka}}) from those that were not so (\textit{dṛṣṭārthaka}\index{drstarthaka@\textsl{dṛṣṭārthaka}}) to assert that Itihāsa belonged to the \textit{dṛṣṭārthaka}\index{drstarthaka@\textsl{dṛṣṭārthaka}} category and as such had no role to play in the teaching of \textit{dharma} (Pollock\index{Pollock, Sheldon} 1990:320). Accordingly, he abstracts from Itihāsa\index{Itihasa@Itihāsa} only one specific meaning that is akin to the meaning of ‘history’ he has in mind: ‘thus it was’ (\textit{iti ha āsa}). In the Indian tradition, however, Itihāsa has a far richer, wider-ranging, and comprehensive meaning and purpose as discernible from the allusion to Itihāsa in the \textit{Chāndogya Upaniṣad\index{Chandogya Upanisad@\textsl{Chāndogya Upaniṣad}}}\index{Upanisad@Upaniṣad!Chandogya@\textsl{Chāndogya}} (7.1.2). 

The \textit{Śatapatha Brāhmaṇa\index{Satapatha Brahmana@\textsl{Śatapatha Brāhmaṇa}}} includes Itihāsa as part of learning and recitation during the performance of the ten-day Pāriplava Rite\index{Pariplava rite@Pāriplava rite} that was part of the Aśvamedha Yajña. The very first ten-day period commenced on the day the horse was set free to roam unchallenged through territories of the rival kings for a period of one year (Hence, it is that there are thirty-six Pāriplava rites in one Aśvamedha \textit{yajña}\index{yajna@\textsl{yajña}}). During this ten-day period, a different Ākhyāna (narrative) was recited every day to a particular group of individuals. The logic behind the rite is that kings lording over various domains (the world of humans, ancestors, aquatic creatures, birds, etc) and their subjects must be brought to vest in the \textit{yajamāna} (the king commissioning the Aśvamedha Yajña). The eighth and ninth days are particularly interesting because the Ākhyāna pertained to Itihāsa\index{Itihasa@Itihāsa} and Purāṇa\index{Purana@Purāṇa} respectively. Here are the relevant parts of the verses from the \textit{Śatapatha Brāhmaṇa} \index{Satapatha Brahmana@\textsl{Śatapatha Brāhmaṇa}} introducing the proceedings of those two consecutive days:

\begin{myquote}
“Now on the eighth day… ‘matsya sāmmada the king (rāja)’, thus he says; ‘of him (matsya rāja), those moving within the waters (udakecarā = fish) are his people and here they are seated;’, thus [he says]. Fish and fish-killers (i.e. fishermen) have come thither: it is them he instructs; ‘the itihāsa is the veda: this it is;’ thus [saying], he says (ācakṣīta- more in the sense of ‘introduces’) some (kaṁcid) itihāsa…
\end{myquote}

\begin{myquote}
Now on the ninth day… thus he says; ‘of him, birds are his people and here they are seated;’, thus [he says]. Birds and fowlers have come thither: it is them he instructs; ‘the purāṇa is the veda: this it is;’ thus [saying], he says some purāṇa…”

~\hfill (\textit{Śatapatha Brāhmaṇa} 13.4.3.12-13; see Eggeling 1885).
\end{myquote}

In this way, by the end of the thirty-six ten-day periods, the king became vested in different categories of lordships and ‘knowledges’ over differing domains and subjects. It is important to remember that the king and his subjects received such ‘knowledges’ in different domains through Itihāsa\index{Itihasa@Itihāsa} and Purāṇa\index{Purana@Purāṇa}. Given that this happened thirty-six times the amount of information received by the audience must have been fairly extensive. Over time, the recitation of Itihāsa and Purāṇa came to be vested in the Sūta-s, a class of skilled bards, who, despite belonging to the ‘lower \textit{varṇa} [class],’ were highly respected for their knowledge. Their dissemination of Itihāsa\index{Itihasa@Itihāsa} and Purāṇa\index{Purana@Purāṇa} to the public at large has indeed remained a critical component of Hindu \textit{dharma}\index{dharma@\textsl{dharma}}; one that enables all and sundry Hindus to remain connected to their rich heritage (Srestha 2017). As a collective term Itihāsa is often mentioned as distinct from the Purāṇa and yet is also treated much the same as the Purāṇa\index{Purana@Purāṇa}. Thus the \textit{Vāyupurāṇa}\index{Vayu-purana@\textsl{Vāyu-purāṇa}}\index{Purana@Purāṇa!vayupurana@\textsl{Vāyupurāṇa}} calls itself both a Purāṇa and an Itihāsa as does the \textit{Brahmāṇḍapurāṇa}\index{Purana@Purāṇa}\index{Brahmandapurana@\textsl{Brahmāṇḍa-purāṇa}}\index{Purana@Purāṇa!Brahmandapurana@\textsl{Brahmāṇḍa-purāṇa}}. The \textit{Brahmapurāṇa}\index{Purana@Purāṇa}\index{Brahmandapurana@\textsl{Brahmāṇḍa-purāṇa}}\index{Purana@Purāṇa!Brahmapurana@\textsl{Brahmapurāṇa}}   calls itself both a Purāṇa and Ākhyāna, while the \textit{Mahābhārata}\index{Mahabharata@\textsl{Mahābhārata}} calls itself by all these terms (Pargiter 1962:35). 

In light of the above, it is understandable why, by the time of Kauṭilya\index{Kautalya@Kauṭalya}, Itihāsa acquired a far wider connotation to embrace all areas of human interest from the mundane to the spiritual. For Kauṭilya the \textit{Atharvaveda}\index{Atharvaveda@\textsl{Atharvaveda}} and the Itihāsa ‘veda’ fell within the ambit of the \hbox{Veda-s} for which reason he put both on the same footing. Elsewhere in the \textit{Arthaśāstra}\index{Arthasastra@\textsl{Arthaśāstra}}, while discussing the training an ideal king should undergo, he talks of Itihāsa and includes under its rubric the Purāṇa, Itivṛtta (past record), Ākhyāyikā (tale), Udāharaṇa (illustrative story), and even the Dharmaśāstra-s\index{Dharmasastra@Dharmaśāstra} (Shamasastri\index{Shamasastri, R.} n.d.:14). It is in this broader sense that the \textit{Rāmāyaṇa}\index{Ramayana@\textsl{Rāmāyaṇa}}, the \textit{Mahābhārata}\index{Mahabharata@\textsl{Mahābhārata}}, and the \hbox{Purāṇa-s} are included under the category of Itihāsa\index{Itihasa@Itihāsa} — record of exploits of heroes who could be king, poet, or priest according to the kind of \textit{varṇa} [class] or world a hero was born into. Their exploits were kept alive as narratives to be told to successive generations. The reading of relevant texts of Itihāsa was ordained for the kings and the administrators. Shivaji\index{Shivaji} (1630-1680), the legendary king of the Marathas, was a product of this practice (Sathe n.d.). This understanding of \textit{itihāsa} was in vogue till the end of the 18th century when Sir William Jones\index{Jones, Sir William}, a pioneer among the British scholars, advised colonial authorities to restrict study of ancient Indian history based on the Purāṇa-s\index{Purana@Purāṇa} in the schools established and operated by the East India Company.

\subsection*{\textit{Itihāsa and the Aitihāsika School of interpretation}}

The \textit{Nirukta}\index{Nirukta@\textsl{Nirukta}} of Yāska\index{Yaska@Yāska} is a commentary on the \textit{Nighaṇṭu}\index{Nighantu@\textsl{Nighaṇṭu}}, a lexicon of Vedic words and terminology, of hoary antiquity. Yāska, who lived probably during 800 BCE, refers to an Aitihāsika School of interpretation, which serves to (1) connect a given \textit{mantra}\index{mantra@\textsl{mantra}} with its deity in a comprehensible way — by acting as an anchoring story to put forth an argument for the transcendent nature of \textit{mantra}-s\index{mantra@\textsl{mantra}} and \textit{mantra} users, (2) explain obscure Saṁvāda \textit{sūkta}-s of the \textit{Ṛgveda}\index{Rg Veda@\textsl{Ṛg Veda}}  by offering a ‘key’ to the myths and dialogues alluded to therein, (3) provide extended genealogies or biographical pedigrees of \hbox{\textit{ṛṣi-}s}, (4) present or resolve conflict, (5) mediate between the \textit{laukika} (\textit{pauruṣeya}\index{apauruseya@\textsl{apauruṣeya}} = worldly) and \textit{lokottara} (\textit{apauruṣeya} = Vedic) worlds (see Patton\index{Patton, Laurie L.} 1996:211).\endnote{Pollock notes that no textbook of Aitihāsika interpretation has been preserved (Pollock 1989:608).} Yāska\index{Yaska@Yāska} thus understands the role of Itihāsa\index{Itihasa@Itihāsa} to be the teaching of the philosophy of life with supporting references to relevant traditional narratives (\textit{pāraṁparika-kathā}).

\subsection*{\textit{Role of Itihāsa: spreading dharma\index{dharma@\textsl{dharma}} through vedopabṛṁhaṇa}}

Pollock\index{Pollock, Sheldon} acknowledges that there was a growing trend among those who performed the acts of \textit{dharma} required by the Veda to also perform acts that were counted as \textit{dharma} but that were not directly based on Vedic injunctions. Mīmāṁsaka-s\index{Mimamsaka@Mīmāṁsaka}, custodians of Vedic \textit{dharma}, addressed and acted upon such expansion of the realm of dharma\index{dharma@\textsl{dharma}} beyond the limited ritual realm.\endnote{A popular collection of \textit{subhāṣita}-s includes one with this ending: “Who on earth but the Mīmāṃsakas\index{Mimamsaka@Mīmāṁsaka} respectfully guard the Veda?” (\textit{bhinnā mīmāṁsakebhyo vidadhati bhuvi ke sādaraṁ vedarakṣām. Subhāṣitaratnabhāṇḍāgāra}, p. 43; Note \# 46 Pollock 1990).} Towards that objective they assumed that the authority for these other actions was conferred not by directly perceptible Vedic texts, but by [\textit{Smṛti}\index{smrti@\textsl{smṛti}}] texts inferentially proven to exist or to have once existed.\endnote{Śabarasvāmin\index{Sabarasvamin@Śabarasvāmin} commented: it is not unreasonable to hold that the knowledge of these texts is remembered, while the texts themselves (that is, their actual wording) have been lost (\textit{Śābarabhāṣya}\index{Sabara Bhasya@\textsl{Śābara Bhāṣya}} 7.7.7-8); Pollock 1990 endnote \# 27.} The concept of \textit{puruṣārtha}\index{purusartha@\textsl{puruṣārtha}} (human need/goal) was first conceptualized within the domain of Mīmāṁsā\index{purva mimamsa@Pūrva Mīmāṁsā} to accommodate for this widening scope of \textit{dharma} (MS\index{Mimamsasutra@\textsl{Mīmāṁsā sūtra}} 4.1.1ff.; Pollock\index{Pollock, Sheldon} 1990:323).

Subsequently, Manu and the other Smṛti-s began to treat \textit{dharma} both as \textit{kratvartha}, that is, regular performance of such \textit{yajña-}s\index{yajna@\textsl{yajña}} as the Agnihotra and other rites/obligations, as well as formal study of the Veda; and as \textit{puruṣārtha}, that is, performance of the whole range of duties prescribed for the four \textit{varṇa-}s\index{varna@\textsl{varṇa}} and four \textit{āśrama-}s\index{asrama@\textsl{āśrama}} (Endnote \# 22; Pollock 1990:323). It does not occur to Pollock that he is describing here \textit{vedopabṛṁhaṇa}, a process (sponsored by the Mīmāṁsā system) of expanding the Vedic teachings by bringing Itihāsa into service to spread the \textit{puruṣārtha}\index{purusartha@\textsl{puruṣārtha}} component of \textit{dharma} among those who did not have direct access to its \textit{kratvartha}\index{kratvartha@\textsl{kratvartha}} component—women and \textit{śūdra-}s (see below). \textit{Vedopabṛṁhaṇa} thus invalidates the process of Vedicization, the cornerstone of Pollock’s thesis of the Mīmāṁsā denial and suppression of ‘history’ in ancient India! 

The foregoing suggests that in many cultures history was/is understood in the sense akin to Itihāsa\index{Itihasa@\textsl{Itihāsa}}: record of significant actions in which great heroes (male and female) are often implicated and are variously recognized as instruments of providence, justice or the spirit of times (\textit{Zeitgeist}) destined to accomplish a definite plan and purpose. In his \textit{Heroes and Hero-Worship}, Carlyle outlined one such way of conceptualizing history that is reminiscent of Itihāsa\index{Itihasa@\textsl{Itihāsa}}:

\begin{myquote}
“Universal History, the history of what man has accomplished in this world, is at the bottom the history of the Great Men who have worked here ”.

~\hfill (Hook\index{Hook, Sydney} 1965:14)
\end{myquote}

Pollock\index{Pollock, Sheldon} does not seem to have entertained Carlyle’s vision of history. If he had, he would not declare ‘India has no history’! Such a summary verdict sounds unconvincing because it fails to explain why Buddhists and Jains (who spurned Veda-s), too, did not attach great importance to ‘history’ as Pollock conceptualizes it. Moreover, Pollock attaches undue importance to Mīmāṁsā\index{purva mimamsa@Pūrva Mīmāṁsā} (which is but one of the six major \textit{darśana-}s) in selecting it as a guiding light in his search for history in India. As against the Mīmāṁsā view of the Veda-s as authorless, the Nyāya\index{Nyaya@Nyāya} and Yoga attribute the authorship of \hbox{Veda-s} to God, and the Vedantin-s consider \textit{brahman}\index{brahman@\textsl{brahman}} (not the Veda) as ultimately real and eternal. Pollock is also unperturbed by the fact that though he abides by modern, objectivist notions of history in the West, the underlying belief in the opposition of ‘factual’ (true) history and ‘fictive’ literature, on which it is based, is itself relatively new. Until nineteenth-century, history in Europe was considered a form of literature with no prejudice as to its truth value (See Perrett\index{Perrett, Roy}, 1999:315; Hossain\index{Hossain, Purba} 2016). While one may agree with Pollock that all narratives necessarily manipulate time by rearranging it to configure a meaningful pattern, it must also be remembered that there can be different modes of configuring temporality in different times and cultures (even within a single culture). These modes, again, can be linear or distinctly nonlinear (See Thapar\index{Thapar, Romila} 2002:26-45).

\section*{IV. Mīmāṁsā Fosters \textit{Dharma}\index{dharma@\textsl{dharma}} and\hfill \break \textit{Vedopabṛṁhaṇa}}

In his commentary on the \textit{Mīmāṁsā sūtra}\index{Mimamsasutra@\textsl{Mīmāṁsā sūtra}} Śabarasvāmin\index{Sabarasvamin@Śabarasvāmin} laid out a comprehensive and useful framework for studying \textit{dharma}: what is \textit{dharma}, its nature and characteristics (\textit{lakṣaṇa-}s), its sources \hbox{(\textit{sādhana-}s)}, what appear as, but are really not, its sources (\textit{sādhanābhāsāni}), and what is the ‘other’ (\textit{para}) [i.e. that to which \textit{dharma}\index{dharma@\textsl{dharma}} relates or reaches out; later identified with \textit{mokṣa}\index{moksa@\textsl{mokṣa}}]. In a practical sense, \textit{dharma}\index{dharma@\textsl{dharma}} is that which\break (1) sustains the universe, (2) supports, and (3) upholds all human efforts to live in virtue, goodness, and mutual expectancy (\textit{sāpekṣatā})\break(\textit{Śābarabhāṣya}\index{Sabara Bhasya@\textsl{Śābara Bhāṣya}} on MS\index{Mimamsasutra@\textsl{Mīmāṁsā sūtra}} 3:3.14).

The fact that Mīmāṁsā philosophical thinking emerged out of exegetical concerns means that the Mīmāṁsā is not exclusively concerned with ontology as Pollock\index{Pollock, Sheldon} presupposes. In company with most contemporary Western scholars he considers metaphysics and ontology as the first elements of philosophical thinking and accordingly transposes this model onto the Mīmāṁsā system. The fact is, such is not quite the case for Mīmāṁsā\index{purva mimamsa@Pūrva Mīmāṁsā} where the main focus is on the Brāhmaṇa portion of the Veda-s, which are primarily action-oriented, prescriptive texts. Since Mīmāṁsaka-s\index{Mimamsaka@Mīmāṁsaka} accord the Veda a specific epistemological place and role, the Veda is the source of transcendental knowledge only. In all other fields (including what Pollock calls ‘history’) ontological and empirical perspectives are accepted and encouraged.

\subsection*{\textit{Itihāsa: bridging Dharma and Mokṣa}}

Dharma with its concern for right action in the material and phenomenal world of men and women has a temporal dimension while \textit{mokṣa}, the ultimate goal of life, transcends temporality. Hindu poets and philosophers have traditionally espoused the bridge provided by Itihāsa and Purāṇa texts to make the passage from \textit{dharma} to \textit{mokṣa}. The metaphor of bridge acts as a linking function which, among other things, brings together elements that are different temporally, spatially or in other ways. The bridge is a not a stable habitat, you are not expected to stand or stay on it for long periods. It is rather, ‘being on the way’ from somewhere to somewhere.

This idea of the connectedness of \textit{dharma} (operating in the mundane, material domain) and \textit{mokṣa} (the transcendent domain) is central to Mīmāṁsā. For this reason Jaimini considers alienation from the omnipotent Supreme Being (Pradhāna) to be imperfection (\textit{doṣa}). Hence all beings are asked to be in relation to Pradhāna (\textit{abhisaṁbandhāt}; MS\index{Mimamsasutra@Mīmāṁsā sūtra} 6:3.1-3) and the act of relating to Pradhāna is part of the goal of performing \textit{yajña}\index{yajna@\textsl{yajña}} (Organ\index{Organ, Troy Wilson} 1970:243). The \hbox{Dharmaśāstra-s}\index{Dharmasastra@Dharmaśāstra} similarly claim an essential continuity between dharma and \textit{mokṣa}: performance of selfless action (\textit{karma}; initially discussed in the Vedic notion of \textit{karman}\index{karman@\textsl{karman}}) as prescribed for one's dharma\index{dharma@\textsl{dharma}} leads to \textit{mokṣa}\index{moksa@\textsl{mokṣa}}. Thus the timeless ideal of \textit{mokṣa} cannot be so easily separated from the temporal ideal of \textit{dharma}\index{dharma@\textsl{dharma}} as Pollock chooses to do (See Perrett 1999).  For this reason, the alleged lack of importance placed on history in India (and Hinduism) may be attributed to classical Indian epistemology rather than to the ‘authorless’ and ‘timeless’ quality of the Veda as alleged by Pollock.

\subsection*{\textit{Karman and Temporal Awareness}}

\vskip 2pt

Though Pollock\index{Pollock, Sheldon} holds the system of Mīmāṁsā responsible for suppressing history, he conveniently ignores the close connection the Indian tradition posits between the doctrine of karma, the Vedic concept of \textit{karman}\index{karman@\textsl{karman}} (actions involved in the performance of \textit{yajña}), and awareness of the three time frames (past, present, future). The concept of \textit{karman} is alluded to in various \textit{sūkta-}s of the \textit{Ṛgveda}\index{Rg Veda@\textsl{Ṛg Veda}} (1:22.19; 2:21.1; 3:33.7 for instance). The Mīmāṁsaka-s\index{Mimamsaka@Mīmāṁsaka} undertook the scrutiny of all actions enjoined in the Veda-s by dividing the Vedic corpus into two broad divisions: sacred formulae (\textit{mantra}-s\index{mantra@\textsl{mantra}}) and injunctions (\textit{vidhi-}s\index{vidhi@\textsl{vidhi}}). These commands also guide everyday acts, which constitute the very essence of human existence. Without action knowledge is fruitless and without action happiness (whether worldly or transcendent) is impossible. In Vedic thinking human and cosmic fullness is reached only through the performance of \textit{yajña}\index{yajna@\textsl{yajña}} which, among other things, re-enacts the primordial creative act by which the world came into being and remains extant during the current \textit{yuga}\index{yuga@\textsl{yuga}}.

\vskip 2pt

For Jaimini reality, therefore, is ordered according to the institution of \textit{yajña}, and all Mīmāṁsā categories are shaped to focus on \textit{yajña}, which is so essential that all its components are significant only insofar as they serve its performance. The words (\textit{śabda-}s\index{sabda@\textsl{śabda}}) of the Veda-s are meaningful solely as a set of injunctions for yājñic action. The Mīmāṁsā provides a framework that permits actions to express both diversity of interests and an underlying authority.  \textit{Dharma} arises from the Veda and the \textit{dharma} of any entity is a function of the way an entity is treated, acted upon, and related to, during the \textit{yajña} and in relation to \textit{yajña}\index{yajna@\textsl{yajña}} (Clooney\index{Clooney, Francis X.} 1990:124,153).  \textit{Dharma}\index{dharma@\textsl{dharma}} is formally defined as that which motivates people to do right actions that are conducive to highest goal or welfare (\textit{niḥśreyasa}) and that are indicated by commands or injunctions (\textit{codanā lakṣaṇo'rtho dharmaḥ} MS\index{Mimamsasutra@\textsl{Mīmāṁsā sūtra}} 1:1.1-2).

\vskip 2pt

In post-Vedic thought \textit{karman}\index{karman@\textsl{karman}} becomes that which remains as the subtle structure of temporal reality once the \textit{prima facie} elements have faded away or have been transformed, as that which all existing beings have in common and in which they share. For the \textit{Gītā}\index{Bhagavadgita@\textsl{Bhagavadgītā}}, \textit{karman}\index{karman@\textsl{karman}} is the constitutive element of the human being (BG 8:3) and the theme of \textit{karman} is discussed at length in chapters two and three (Panikkar\index{Panikkar, Raimundo} 1972). It is therefore surprising that Pollock\index{Pollock, Sheldon} does not take into account the possible relevance of the doctrine of \textit{karman} in his discussions of the status of history in ancient India.

\subsection*{\textit{Karman and Kāla}}

Raimundo Panikkar observes that in Indian way of thinking the locus of \textit{karman} is the temporal existence of reality, the temporal existence of this world and, above all, the human being (Panikkar 1972:35). It is in this line of thinking that the concept of historicity and historical consciousness\index{historical consciousness} finds its place.\endnote{This line of interpretation of karman\index{karman@\textsl{karman}} as source of ‘history’ and historical consciousness\index{historical consciousness} is based on Panikkar\index{Panikkar, Raimundo} 1972.}\textit{Karman} is the crystallization of actions past, as well as of the results of acts which are no longer in the past, but which emerge and are present in the contemporary situation of the bearer of that particular \textit{karman}. The forces that energize an action leave their mark on the agent as well as on the world. Within the agent these energizing forces leave a residual effect (\textit{karmāśaya}) that shape and direct future actions and carry the combined influence (\textit{saṁskāra-}s\index{samskara@\textsl{saṁskāra}}) of past actions forward into the agent's future. Some of the internal effects of actions show up directly in the habits and character of the succeeding generations (Panikkar 1975:86-87). From the Vedic perspective, an isolated [individual] being is an abstraction; an artificial and unnatural separation from the common reality of which it is part. Human being therefore is karmic, temporal, and historical (Panikkar 1972:42). The law of \textit{karman} gives expression to this fundamental human condition, yet at the same time allows for its overcoming—to \textit{mokṣa}\index{moksa@\textsl{mokṣa}}.\endnote{This line of interpretation is suggested by Panikkar 1972:41ff.} Through their commitment to this background theory of \textit{karman} and \textit{kāla}, Hindus are able to temporalize and historicize consciousness rather more comprehensively and deliberately than Pollock would care to admit (adapted from Pertt 1999:607).

In the Indic context historical consciousness\index{historical consciousness} is bound with the recognition that events produce effects and consequences (both external and internal) to the authors and agents of these events, and their community. Itihāsa as a record of meaningful and inspiring heroic actions (\textit{nārāśaṁsī}) was already spelled out in the \textit{Śatapatha Brāhmaṇa}\index{Satapatha Brahmana@\textsl{Śatapatha Brāhmaṇa}} (13:4.3.12; see Singhal\index{Singhal, K. C.} and Gupta\index{Gupta, Roshan} 2003:23), and the \textit{Mahābhārata}\index{Mahabharata@\textsl{Mahābhārata}} recommends that actual doings of great kings and seers are to be analyzed within the parameters of \textit{dharma\index{dharma@\textsl{dharma}}, artha\index{artha@\textsl{artha}}, kāma} and \textit{mokṣa}\index{purusartha@\textsl{puruṣārtha}} (Shendge\index{Shendge, Malati J.} 1996). The \textit{Mīmāṁsā Paribhāṣā}\index{Mimamsaparibhasa@\textsl{Mīmāṁsā Paribhāṣā}} of Kṛṣṇa Yajvan\index{Krsna Yajvan@Kṛṣṇa Yajvan} refers to two main categories of narratives: individual heroic action (\textit{parakṛti}) and collective heroic actions (\textit{purākalpa})(Swami Madhavananda\index{Madhavananda, Swami} 1987:70-71).\endnote{For Rājaśekhara\index{Rajasekhara@Rājaśekhara} (renowned 10th century poet and literary critic), Itihāsa is of two types: of a single hero (nāyaka = protagonist; the \textit{Rāmāyaṇa}\index{Ramayana@\textsl{Rāmāyaṇa}}) and of many heroes (the \textit{Mahābhārata}\index{Mahabharata@\textsl{Mahābhārata}}) and identifies them as Parakriyā [Parakṛti] and Purākalpa respectively. The awareness that Itihāsa is a narrative about the past as well as the past itself brings Rājaśekhara to modern historiographical concerns (Devy\index{Devy, G. N.} 1998:17).} Once performed, an individual heroic action can be detached from its remarkable performer to let it develop legacy and consequences of its own. Itihāsa invests such memorable and autonomized acts (individual or collective) with cultural and social dimensions, which succeeding generations are invited to emulate.

Pollock\index{Pollock, Sheldon} nevertheless does not regard purāṇic genealogies and historical biographies (\textit{carita-}s) as historical (even though they evince historical consciousness;\index{historical consciousness} e.g., Bāṇa’s \textit{Harṣacarita}) because they do not conform to ‘Enlightenment’ conceptions of historical consciousness. Against this assertion of Pollock, Thapar insists that historical consciousness is present in all societies, which may or may not produce direct history writing. In searching for historical consciousness in Indic literary texts, for instance, she came across embedded history (as in epics, myths and genealogy) where historical consciousness has to be prised out and externalized history such as familial, institutional, and regional chronicles or biographies (where the text makes deliberate use of the past)(Thapar\index{Thapar, Romila} 2013:59-61, 683; Hossain\index{Hossain, Purba} 2016).


\section*{V. Spreading Dharma Using Mīmāṁsā\hfill \break Hermeneutics}
\index{hermeneutics!Mimamsa@Mīmāṁsā}

In the light of the above discussion it may be posited that Dharmaśāstra-s\index{Dharmasastra@Dharmaśāstra} and Purāṇa-s\index{Purana@Purāṇa} presuppose that a proper understanding of Itihāsa\index{Itihasa@Itihāsa} is crucial for fulfilling \textit{dharma}\index{dharma@\textsl{dharma}} and the other ends of life.This is because Itihāsa acts as a storehouse of the past for what needs to be remembered, i.e., values that guide fulfillment of four \textit{puruṣārtha-}s\index{purusartha@\textsl{puruṣārtha}} through the four stages of life (\textit{āśrama-}s\index{asrama@āśrama}) using the power of language and narrative.\endnote{One popular verse puts it as follows: \textit{dharmārthakāmamokṣāṇām upadeśasamanvitam; purāvṛttam kathāyuktam itihāsam pracakṣate} (Sathe n.d.:22).} The sense of time and culture in a given tradition is conditioned by the language(s) and linguistic conventions operating in that tradition. The manner in which a tradition internalizes temporal modalities of its collective existence is determined by its language/s. Here, grammarians play a major mediating role between language and tradition.\endnote{French philosopher Paul Ricoeur, for instance, observed that it is Sanskrit grammar and the system of its verb tenses that have been decisive in India's sense of time than the Sanskrit vocabulary designating time (Ricoeur 1975 Introduction.)} Literary texts, on their part, depict the idealized \textit{āśrama}\index{asrama@āśrama} model of the four life stages, which Kālidāsa\index{Kalidasa@Kālidāsa} describes in the life history of two kings of the Raghu dynasty: Dilīpa and Raghu pointing out that the householder’s stage (\textit{gṛhastha}) is the one that sustains and makes possible the three other stages (\textit{Raghuvaṁśa}\index{Raghu-vamsa@\textsl{Raghu-vaṁśa}} 5.10).

\newpage

\subsection*{\textit{Mantra-rāmāyaṇa\index{Mantra-ramayana@\textsl{Mantra-rāmāyaṇa}} of Nīlakaṇṭha\index{Nilakantha@Nīlakaṇṭha}}}

The mediating role of Itihāsa in the fulfilment of dharma\index{dharma@\textsl{dharma}} can be illustrated with particular reference to the \textit{Mantra-rāmāyaṇa} of Nīlakaṇṭha Caturdhara, a Marathi-speaking Brahmin who flourished in the second half of the seventeenth century in a family established in Karpuragrāma (modern Kopargaon); a town on the banks of the River Godāvarī in what is now the state of Maharashtra. Nīlakaṇṭha moved to Vārāṇasī where he undertook the study of Veda and Vedāṅga, Mīmāṁsā, and Advaita Vedānta in the era when Aurangzeb was the emperor (1658-1707). Nīlakaṇṭha is better known for his commentary on the \textit{Mahābhārata}\index{Mahabharata@\textsl{Mahābhārata}} (\textit{Bhāratabhāvadīpa}\index{Bharata-bhavadipa@\textsl{Bhārata-bhāvadīpa}}), which is now recognized as a necessary companion volume to read and understand the \textit{Mahābhārata}. He also wrote two other popular works for the purpose of illuminating the hidden meaning of Vedic \textit{mantra}-s: the \textit{Mantra-rāmāyaṇa}\index{Mantra-ramayana@\textsl{Mantra-rāmāyaṇa}} (MR) and the \textit{Mantra-bhāgavata}\index{Mantra-bhagavata@\textsl{Mantra-bhāgavata}} by arranging the select \textit{mantra}-s drawn from the \textit{Ṛgveda}\index{Rg Veda@\textsl{Ṛg Veda}} in such a way that they reveal the story centered on Rāma or Kṛṣṇa and the teaching of \textit{dharma}\index{dharma@\textsl{dharma}} - the \textit{Rāmāyaṇa}\index{Ramayana@\textsl{Rāmāyaṇa}} and the \textit{Bhāgavata}\index{Bhagavata@\textsl{Bhāgavata}} respectively.\endnote{Nīlakaṇṭha insists that the Rāmakathā is as present in the Veda as is the Ūrvaśī-Purūravas saṁvāda (dialogue) in (\textit{Ṛgveda} 10.85; Dwivedi 1998:15).}

\vskip 2pt

Here, Nīlakaṇṭha’s purpose is different from that of other retellings of the \textit{Rāmāyaṇa} he is familiar with. Vedic commentarial tradition for reading Ṛgvedic \textit{sūkta-}s\textit{/saṁvāda-}s initiated by Yāska\index{Yaska@Yāska} and others were oriented toward the explanation of the proper performance of the Vedic rituals. Nīlakaṇṭha's explanation is derived from (1) semantic elucidation of Vedic \textit{mantra}-s\index{mantra@\textsl{mantra}} (\textit{nigama-nirukta})\endnote{For Nīlakaṇṭha’s explanation of \textit{‘nigamanirukta’} see Kahrs (1998).} and (2) adoption of the Mīmāṁsan injunctive perspective to shape and frame his own message and philosophy using Kumārila\index{Kumarila@Kumārila} Bhaṭṭa’s argument (perhaps following Kauṭilya) that considered the \textit{Mahābhārata} and \textit{Rāmāyaṇa} to be Dharmaśāstra-s\index{Dharmasastra@Dharmaśāstra} and as such sources of instruction in the four ends of man (\textit{puruṣārtha-}s\index{purusartha@\textsl{puruṣārtha}})(Fitzgerald\index{Fitzgerald, James} 1991).

\vskip 2pt

Nīlakaṇṭha\index{Nilakantha@Nīlakaṇṭha} is able to expand the horizon of the Veda-s and realign it with the horizon of the Vedāntic scholarly milieu in which he lived (Vārāṇasī of the 17th century) thanks to the basic fluidity of the Vedic texts, and indeed Śruti\index{sruti@\textsl{śruti}} itself. By relating the past to the contemporary situation through the process of \textit{upabṛṁhaṇa} he was able to add new material covering immediate past to the existing corpus. This is in line with Mīmāṁsan hermeneutical stance that ‘canon’ is not rigidly demarcated on the basis of particular Vedic texts (Patton 1996:425). Thus, commenting on MS\index{Mimamsasutra@\textsl{Mīmāṁsā sūtra}} 2:4.9 Śabarasvāmin\index{Sabarasvamin@Śabarasvāmin} writes, ‘all branches of the Veda and all Brāhmaṇa texts communicate to us about the same [ritual] activity’ thereby implying that the Vedic canon is not the closed and fixed entity (Patton\index{Patton, Laurie L.} 1996:425 FN \# 33).\endnote{See also Kumārila\index{Kumarila@Kumārila} Bhaṭṭa \textit{Tantravārttika} on MS 2:4.9.}

\vskip 2pt

The above stance enables Nīlakaṇṭha\index{Nilakantha@Nīlakaṇṭha}, the commentator, to consider all the texts from the compendium of the Veda as ‘one’ in order to make sense of its part (see the \textit{prakaraṇa}\index{prakarana@\textsl{prakaraṇa}} principle discussed above). The basis for this strategy also came from Kumārila’s\index{Kumarila@Kumārila} statement that “One can create one large sentence on a particular subject out of several independent sentences of the \textit{vidhi} or \textit{arthavāda}\index{arthavada@\textsl{arthavāda}} type’’ (Kumārila Bhaṭṭa \textit{Tantra-vārttika}\index{Tantra-varttika@\textsl{Tantra-vārttika}} on MS 1:4.13.24). Nīlakaṇṭha next selects \hbox{\textit{mantra}-s} from the \textit{Ṛgveda} and identifies in them elements of the Rāma story on the basis of the Mīmāṁsā principles of \textit{liṅga,\index{linga@\textsl{liṅga}} vākya,}\index{vakya@\textsl{vākya}} and \textit{prakaraṇa}. He then adds other \textit{mantra}-s which are not so explicit, but which can be relevant by context or by narrative connection, as Nīlakaṇṭha sees it (\textit{ekavākyatā, liṅgaviśeṣa;} Dwivedi\index{Dwivedi, Prabhunath} 1998; MR verse \# 22).

\vskip 2pt

The \textit{sūkta} entitled ‘Vamro Vaikhānasaḥ’ (\textit{Ṛgveda}\index{Rg Veda@\textsl{Ṛg Veda}} 10.99) is traditionally attributed to a sage named Vamra Vaikhānasa. Nīlakaṇṭha stipulates that Vamra is none other than Vālmīki. Then, by clever use of the principles of \textit{liṅga} and \textit{prakaraṇa}, he posits that the first five verses of this \textit{sūkta} are by Vamra/Ādikavi Vālmīki\index{Valmiki@Vālmīki} and that they encapsulate the Rāma story. The \textit{Mantra-rāmāyaṇa}\index{Mantra-ramayana@\textsl{Mantra-rāmāyaṇa}} accordingly begins with a reading of these five verses as a telling of the whole Rāmakathā in a seed/root form. He then offers their rereading from the \textit{ādhyātmika} perspective suggesting that the rest of the work will proceed in the like manner (See Dwivedi 1998; MR verses \# 15, 19).

\vskip 2pt

A \textit{pūrvapakṣin} (Pollock\index{Pollock, Sheldon} in the present context) might object that the use of the \textit{liṅga} or \textit{prakaraṇa} principle is used in a restricted sense in Mīmāṁsā because a fundamental tenet of that philosophical position holds that not every \textit{mantra}\index{mantra@\textsl{mantra}} or \textit{vidhi}\index{vidhi@\textsl{vidhi}} from the \textit{Ṛgveda}\index{Rg Veda@\textsl{Ṛg Veda}} can be interpreted on every level of meaning. Some are simply about ritual action. In reply, we may assert with Nīlakaṇṭha\index{Nilakantha@Nīlakaṇṭha} that the meaning of texts can be different for different readers of the texts. Attention may be drawn to Yāska's\index{Yaska@Yāska} practice of explaining the same word in a variety of meanings and commenting on the same verse in either (or both) \textit{ādhyātmika} and \textit{aitihāsika} sense (see Dwivedi 1998; MR verses \# 12, 43). To the objection that Nīlakaṇṭha follows Purāṇic and not orthodox interpretation, Nīlakaṇṭha can draw attention to the fact that Rāma's divinity was understood and expressed in different ways by different narrators (see Minkowski\index{Minkowski, Christopher Z.} 2002, FN \# 93).

\vskip 2pt

Another \textit{pūrvapakṣin }might object that since the Rāmakathā is nowhere mentioned in the Veda-s, Nīlakaṇṭha’s approach to find this wholly new meaning in the \textit{mantra}-s\index{mantra@\textsl{mantra}} from the \textit{Ṛgveda}\index{Rg Veda@\textsl{Ṛg Veda}} departs from the Mīmāṁsā’s typical hermeneutical approach to analyze and interpret the Veda-s. In response, Nīlakaṇṭha\index{Nilakantha@Nīlakaṇṭha} invokes the maxim that a post should not be blamed if a blind man walks into it: that no one has read the Rāmakathā as the primary meaning of the Vedic verses before does not mean that such an interpretation is wrong\endnote{\textit{nanu rāmāyaṇīyakathā kasyām cid api śākhāyām vṛtravadhādivanna dṛśyate'to'syāhāḥ śrutimūlatveva nāstīti cet naiṣa sthāṇoraparādho yadenamandho na paśyati iti nyāyena tvayi vedārthānabhijñe sati na rāmāyaṇamaparādhyati.} (See Dwivedi 1998: 11. The maxim of the blind man and the post is found in \textit{Nirukta}\index{Nirukta@\textsl{Nirukta}} 1.16, in exactly these words; also Minkowski Forthcoming FN \# 65).} Nīlakaṇṭha's innovation lies in the way existing techniques and repositories of knowledge are taken together in the service of the task he chose: \textit{vedopabṛṁhaṇa}. Though an Advaitin in philosophical outlook; Nīlakaṇṭha brings inputs in his thinking from the tradition of devotion to Rāma and Kṛṣṇa. His innovation lay in stating that the Veda-s also refer to Viṣṇu as the \textit{saguṇa-brahman}\index{brahman@\textsl{brahman}}, i.e., to Viṣṇu in his incarnated action as Rāma in a narrativized \textit{kāvya} form of \textit{Rāmāyaṇa} (See Minkowski: Forthcoming).  The \textit{Rāmāyaṇa}, in turn, holds in high esteem all that is found in Veda because such expressions as \textit{vedoktam} (spoken about in the Veda) and \textit{vedopabṛṁhitam} (described and discussed in the Veda) occur often in it.


\subsection*{\textit{Western insensitivity to works of Itihāsa}}

Pollock’s\index{Pollock, Sheldon} writings lack sensitivity to the peculiarity of understanding India’s past in terms of India’s own cultural context. Christopher Minkowski\index{Minkowski, Christopher Z.} (a noted Sanskritist and collaborator of Pollock in the Sanskrit Knowledge System on the Eve of Colonialism project) concludes his article on Nīlakaṇṭha’s\index{Nilakantha@Nīlakaṇṭha} \textit{Mantra-rāmāyaṇa}\index{Mantra-ramayana@\textsl{Mantra-rāmāyaṇa }} with an assessment of Nīlakaṇṭha’s creativity with all the smugness a Western Sanskritist and Vedist can summon:

\begin{myquote}
“The study of Nīlakaṇṭha's works might be useful in learning about the later destiny of Vedic literature. But the question might still be raised about his usefulness to studies of the Vedas “in themselves." Are we likely to revise our translations or interpretations of any verse of the \textit{Ṛgveda}\index{Rg Veda@\textsl{Ṛg Veda}} based on Nīlakaṇṭha's contributions? Probably not. Do his glosses preserve any precious linguistic archaeological specimens that might shed some light on Vedic language? Probably not. What then is the use of Nīlakaṇṭha 's work for those of us studying the Veda today? Theodor Aufrecht, a Vedist of note in the last century, already dismissed Nīlakaṇṭha's work, saying that it ``perverted" the Vedic verses into a reference to Rāma and Kṛṣṇa. And although we probably would not say it quite that way today, I doubt that we would take Nīlakaṇṭha 's texts any more seriously. But there is at least this second order value: a reading of Nīlakaṇṭha's Mantrarahasya works can remind us of the assumptions we make today in doing our work, the location of our own disciplinary boundaries, the distinction that we make between the Vedas' destiny and the Vedas' meaning.”

~\hfill (Minkowski Forthcoming:28)
\end{myquote}

Minkowski’s musings on the \textit{Mantra-rāmāyaṇa}\index{Mantra-ramayana@\textsl{Mantra-rāmāyaṇa}} reveal how tightly Western academics control exegesis of the Veda-s. Swadeshi interpreters of the Veda-s, on their part, need to proceed in their work keeping intact the integrity of the Vedic texts without bracketing out their ‘mythic’ or didactic portions as Pollock\index{Pollock, Sheldon} and Minkowski\index{Minkowski, Christopher Z.} would like to suggest. It is necessary to view \textit{Mantra-rāmāyaṇa} as a holistic work produced by Nīlakaṇṭha’s\index{Nilakantha@Nīlakaṇṭha} use of myth, rhetoric, and Itihāsa as discernible in the \textit{Rāmāyaṇa}\index{Ramayana@\textsl{Rāmāyaṇa}}, which thematically is connected with the heroic narratives and genealogical tendencies of the Itihāsa\index{Itihasa@Itihāsa}-Purāṇa\index{Purana@Purāṇa} tradition as explained by Rājaśekhara\index{Rajasekhara@Rājaśekhara} (see Devy\index{Devy, G. N.} 1998; endnote \#7). Nīlakaṇṭha’s work can be legitimated using Kumārila\index{Kumarila@Kumārila} Bhaṭṭa’s argument that traditional literature may be acceptable as authoritative insofar as it exhibits the property of ‘being rooted in the Veda-s’ (\textit{veda-mūlatvam})—even if that means, in some cases, inferring the reality of a no-longer-accessible Vedic text (\textit{Tantra-vārttika}\index{Tantra-varttika@\textsl{Tantra-vārttika}} 1.3.1. ff).\endnote{Smṛtyadhikaraṇa of the \textit{Mīmāṃsā\index{Mimamsasutra@\textsl{Mīmāṁsā sūtra}} sūtra} (1.3.1-2); see Minkowski 2005:240-41, where he cites relevant remarks of Nīlakaṇṭha.}


\section*{\textit{Siddhānta}}

\subsection*{\textit{Itihāsa: the fifth Veda}}

The fact that ‘history’ as is understood in the West, is subsumed in the broader, inclusive category of Itihāsa\index{Itihasa@Itihāsa} (deemed to be the ‘fifth Veda’) in the tradition of India altogether escapes Pollock’s attention.  One probable reason for this lacuna might be that a range of possible answers that can be elicited out of a given tradition/texts/action depend upon the questions that are asked of it. Any scholar inevitably determines and comes with his/her own agenda of such an inquiry (as Pollock\index{Pollock, Sheldon} indeed does). German hermeneutist Hans-Georg Gadamer\index{Gadamer, Hans Georg} formalized this phenomenon into the general notion of pre-understanding (\textit{Vorverstandnis}), which is an integral part of the interpreter’s own horizon, and which is informed by the effective history (\textit{Wirkungsgeschichte}) that emanates from the given text/action. The possibility for understanding is therefore conditioned because the interpreter must engage and negotiate with the history of the text/action he/she is studying (Bilimoria\index{Bilimoria, Purushottam} 2008:70). 

In this effort the interpreter may attempt any understanding of the text/document/action by approaching it purely from the prevalent perspective of its original authors/actors from the outside in (i.e. etically) or from the inside out (i.e. emically). Thinking with Indians, i.e. from the inside out, noted Vedicist and Sanskritist Jan Gonda\index{Gonda, Jan} observed (unlike Pollock) that Indian civilization, in the main, stands in striking contrast to Western, modern ‘\textit{mentalité}.'  Without being one-sidedly intellectual, it gives free scope to the emotional and imaginative sides of human nature against which distinctions between the subjective and the objective, reality and appearance are almost meaningless (Gonda 1975:8).

Instead of simply extracting whatever he had wanted from the selected texts from ancient India and then casting the rest aside, Pollock should have remembered that his sources relate to India’s past in various ways (an acknowledgment with which he started his quest of ‘history’ in ancient India). In this quest he ignores the fact that his sources reveal not just ‘data’ or information, but also consciousness of the understanding of the past, and what it means to think about the past. ‘Historians rarely heal themselves,’ laments Pollock; ‘they rarely historicize their own reading.’ It is therefore not surprising that there is no acknowledgment here of the role of Pollock the interpreter’s \textit{present} in his interpretation of India’s \textit{past} (see Pollock\index{Pollock, Sheldon} 2007:370). He conveniently sidesteps the ‘emic’ view on the past treating his sources as mere informants. As Thapar has observed, this move perpetrates violence against Itihāsa\index{Itihasa@Itihāsa} leaving Pollock’s central thesis ‘India is without history’ an ideological affront (Thapar\index{Thapar, Romila} 2013). India’s response should be: So what? The West is without Itihāsa! 

The Vedic tradition was more concerned to address the central paradox of human existence: on the cosmic scale the duration of human life is insignificant. This passage, albeit brief, is the source of all reflection of any significance. The disparity between the lived time and cosmic time is the source of all human anxiety and suffering. Yet, it is also the \textit{raison d'être} of the Vedic thought and quest that seeks to provide relief from pain and suffering, relief in the form of various injunctions (and the supporting explanatory material described as \textit{arthavāda}\index{arthavada@\textsl{arthavāda}}) pertaining to dharma\index{dharma@\textsl{dharma}}. They lie ‘atemporally’ and in ‘seed/root form’ (i.e. as \textit{apauruṣeya\index{apauruseya@\textsl{apauruṣeya}}}\index{Itihasa@Itihāsa!apauruseya@\textsl{apauruṣeya}}  Itihāsa\index{Itihasa@Itihāsa}) in the Vedic canon as myths, eulogies (\textit{praśasti}), heroic tales, and genealogies awaiting to externalize and sprout in the flow of time in every \textit{yuga}\index{yuga@\textsl{yuga}} through the process of \textit{upabṛṁhaṇa.} The \textit{pauruṣeya}\index{pauruseya@\textsl{pauruṣeya}}\index{Itihasa@Itihāsa!pauruseya@\textsl{pauruṣeya}} Itihāsa\index{Itihasa@Itihāsa}/Purāṇa\index{Purana@Purāṇa} tradition acts as its medium and agency in order to extend and expand Vedic teachings on \textit{dharma} for the benefit of all those who do not have direct access to the Veda-s. This gives the lie to the process of ‘Vedicization’ invented by Pollock to hold the Mīmāṁsā system responsible for suppressing ‘history’ and depriving the rights of disadvantaged masses.

Nīlakaṇṭha's\index{Nilakantha@Nīlakaṇṭha} \textit{Mantra-rāmāyaṇa}\index{Mantra-ramayana@\textsl{Mantra-rāmāyaṇa}} is a prime instance of how \textit{Vedopabṛṁ\-haṇa}, a process based on the hermeneutical principles of Mīmāṁsā, continued until the pre-modern times. His innovation lay in affirming once again that Rāma's \textit{avatāra}\index{avatara@\textsl{avatāra}} was understood from the very beginning as bringing forth the ethical and spiritual teachings of the Veda centered on dharma\index{dharma@\textsl{dharma}} to the masses. In this, Nīlakaṇṭha was emulating Vālmīki\index{Valmiki@Vālmīki} himself who introduced the Veda-s to Lava and Kuśa (the two sons of Rāma) and then elaborated on their teachings by reciting the \textit{Rāmāyaṇa}\index{Ramayana@\textsl{Rāmāyaṇa}} to them (\textit{Rāmāyaṇa} 1:4.6).


\section*{Bibliography}

\begin{thebibliography}{99}
\bibitem{chap2-key01} Acharya, Narayan Ram (Comp.) (2011). \textit{Subhāṣitaratnabhāṇḍāgāram}. Delhi: Chowkhambha Sanskrit Pratishthan.

 \bibitem{chap2-key02} \textbf{\textit{Arthaśāstra}} of Kauṭilya. See Shamasastry.

 \bibitem{chap2-key03} Bhattacharya, Sibesh. (2010). \textit{Understanding Itihasa}. Shimla: Indian Institute of Advanced Study, Shimla.

 \bibitem{chap2-key04} Bilimoria, Purushottama. (2008). Being and Text: Dialogic Fecundation of Western Hermeneutics and Hindu Mīmāṁsā in the Critical Era. In Sherma and Sharma (2008). pp.~45--80.

 \bibitem{chap2-key05} Chakraborty, Biswadeep. (2016). Did Ancient India have Historical Traditions? | \url{www.academia.edu/9936391/Did_Ancient_India_have_Historical_Traditions}; accessed on Sept 19, 2016.

 \bibitem{chap2-key06} Chekuri, Christopher. (2007). “Writing politics back into history.” \textit{History and Theory.} Vol~46, Issue 3 (Oct 2007). pp.~384--395.

 \bibitem{chap2-key07} Clooney, Francis X., S.J. (1990). \textit{Thinking Ritually: Rediscovering the Pūrva Mīimāṁsā sūtras of Jaimini}. Vienna: Publications of the de Nobili Research Library.

 \bibitem{chap2-key08} Devy, G. N. (1998). \textit{Of Many Heroes: An Indian Essay in Literary Historiography}. Hyderabad: Orient Longman.

 \bibitem{chap2-key09} Dwivedi, Prabhunath. (Ed.) (1998). \textit{Mantrarāmāyaṇa.} [of Nīlakaṇṭha Caturdhara]. Edited with Hindi translation. Lucknow: Uttar Pradesh Sanskrit Samsthanam. \url{https://archive.org/details/MantraRamayana}, Accessed on August 21, 2016.

 \bibitem{chap2-key10} Eggeling, Julius. (Ed.) (1885). \textit{Śatapatha Brāhmaṇa}. Oxford: Clarendon Press.

 \bibitem{chap2-key11} Fisher, Elaine. (2008). “A Book with No Author: Does Mimamsa Circumvent the Intentional Fallacy?” \textit{Journal of Indian Philosophy}, vol~36 (2008). pp.~533--542.

 \bibitem{chap2-key12} Fitzgerald, James. (1991). “India’s Fifth Veda: the Mahābhārata’s Presentation of Itself.” In Sharma (1991). pp.~150--171.

 \bibitem{chap2-key13} Hook, Sidney. (1965, 1943$^{1}$). \textit{The Hero in History: Study in Limitation and Possibility}. Boston: Beacon Press.

 \bibitem{chap2-key14} Hossain, Purba. (2016). “Moksha, Mimamsa and Yuga: Does philosophy account for the supposed absence of history in early India?”\break \url{https://www.academia.edu/8688401}; Accessed on Sept 23, 2016.

 \bibitem{chap2-key15} Houben, Jan E. M. (2002). “The Brahmin Intellectual: History, Ritual and ‘Time Out of Time.’” \textit{Journal of Indian Philosophy} 30.5 (2002). pp.~463--479.

 \bibitem{chap2-key16} Ikari, Y. (Ed.) (Forthcoming). \textit{Proceedings of the Second International Vedic Workshop.} Kyoto.

 \bibitem{chap2-key17} Jalki Dunkin. (2013). “Colonialism and its impact on India.” \textit{International Journal of Social Science and Humanities} Vol~II (1) June 2013. pp.~123--128.

 \bibitem{chap2-key18} Jha, Ganga Natha. (1964, 1942$^{1}$). \textit{Purva Mimamsa in its Sources}. Varanasi: Benares Hindu University.

 \bibitem{chap2-key19} —. (Tr.) (1983). \textit{Tantravārttika: a commentary on Śabara's Bhāṣya on the Pūrvamīmāṁsā sūtras of Jaimini by Kumārila Bhaṭṭa.} 2nd edition. Delhi: Sri Satguru Publications.

 \bibitem{chap2-key20} Kahrs, Elvind. (1998). \textit{Indian Semantic Analysis}. Cambridge: Cambridge University Press.

 \bibitem{chap2-key21} Kale, M. R. (Tr.) (1922). \textit{Raghuvaṁśa} (of Kālidāsa with commentary \textit{‘Samjīvanī’} of Mallinātha). Bombay: Gopal Narayan \& Co.

 \bibitem{chap2-key22} Katju, Markandeya. (1993). “The Mimamsa Principles of Interpretation.” \textit{1 SCC (Jour) 16.} \textless  \url{http://www.supremecourtcases.com/index2.php?option=com_content&itemid=1&do_pdf=1&id=561}\textgreater . Accessed on 19 Oct 2016.

 \bibitem{chap2-key23} Macdonell, Arthur A. (1996, 1900$^{1}$). \textit{A History of Sanskrit Literature}. New York: Haskell.  

 \bibitem{chap2-key24} Madhavananda, Swami (Tr.) (1987). \textit{Mīmāṁsā Paribhāṣā} [of Kṛṣṇa Yajvan]. Calcutta: Advaita Ashrama.

 \bibitem{chap2-key25} \textbf{\textit{Mantrarāmāyaṇa}} [of Nīlakaṇṭha Caturdhara]. See Dwivedi (1998).

 \bibitem{chap2-key26} \textbf{\textit{Manu-smṛti}} Sanskrit text with English translation. Accessed on 19 October 2016.

 \bibitem{chap2-key27} \textbf{\textit{Mīmāṁsā Paribhāṣā}} [of Kṛṣṇa Yajvan]. See Madhavananda (1987).

 \bibitem{chap2-key28} \textbf{\textit{Mīmāṁsā sūtra}} of Jaimini. See Sandal (1923).

 \bibitem{chap2-key29} Minkowski, Christopher  Z. (2002). “Nīlakaṇṭha Caturdhara's Mantrakāśīkhaṇḍa.” \textit{The Journal of the American Oriental Society}, April 1, 2002. pp. 329--344.

 \bibitem{chap2-key30} —. (2005). “What Makes a “Traditional?”” In Squarcini (2005). pp.~225--252.

 \bibitem{chap2-key31} —. Forthcoming. Nīlakaṇṭha Caturdhara and the Genre of Mantrarahasyaprakāśikā. In \textit{The Proceedings of the Second International Vedic Workshop}. (Ed.) Ikari, Y., Kyoto.

 \bibitem{chap2-key32} Organ, Troy Wilson. (1970). \textit{The Hindu Quest for the Perfection of Man}. Athens, Ohio: Ohio University.

 \bibitem{chap2-key33} Panikkar, Raimundo. (1972). “The Law of Karman and the Historical Dimension of Man.” \textit{Philosophy East and West}, Vol. 22, No.~1 (Jan., 1972). pp.~25--43.

 \bibitem{chap2-key34} —. (1975). \textit{Temps et Histoire dans la Tradition de l’Inde.} In Ricoeur (1975). pp.~74--89.

 \bibitem{chap2-key35} Pargiter, Frederick Eden. (1962, 1922$^{1}$). \textit{Ancient Indian Historical Tradition}. Delhi: Motilal Banarsidass.

 \bibitem{chap2-key36} Patton, Laurie L. (1996). \textit{Myth as Argument: The Bṛhaddevatā as Canonical Commentary}, Volume 41. Walter de Gruyter GmbH \& Co. KG.

 \bibitem{chap2-key37} Perrett, Roy, W. (1999). “History, Time, and Knowledge in Ancient India.” \textit{History and Theory} 38, no.~3 (October 1, 1999). pp.~307--321.

 \bibitem{chap2-key38} Pollock, Sheldon. (1989). “Mīmāṁsā and the Problem of History in Traditional India.”  \textit{Journal of the American Oriental Society}, 109(4). pp.~603--610.

 \bibitem{chap2-key39} —. (1990). “From Discourse of Ritual to Discourse of Power in Sanskrit Culture.” \textit{Journal of Ritual Studies} 4/2 (SUMMER 1990). pp.~316--345.

 \bibitem{chap2-key40} —. (2006). \textit{The Language of the Gods in the World of Men: Sanskrit, Culture, and Power in Premodern India}. Berkeley: University of California Press.

 \bibitem{chap2-key41} —. (2007). “Pretextures of Time.” \textit{History and Theory} 46, no.~3 (2007). pp.~366--383.

 \bibitem{chap2-key42} \textbf{\textit{Raghuvaṁśa}} [of Kālidāsa with commentary \textit{‘Samjīvanī’} of Mallinātha]. See Kale (1922).

 \bibitem{chap2-key43} Ricoeur, Paul. (Ed.) (1975). \textit{les Cultures et le Temps.} Paris: Payot.

 \bibitem{chap2-key44} Ridderbos, Kantika. (Ed.) (2002). \textit{Time}. Cambridge: Cambridge University Press.

 \bibitem{chap2-key45} Sandal, Mohan Lal (Tr.). (1923). \textit{Mīmāṁsā sūtra} of Jaimini. Allahabad: Panini Office. accessed on August 30, 2016.

 \bibitem{chap2-key46} Sarkar, Kishori Lal. (1909). \textit{The Mimansa Principles of Interpretation as applied to Hindu Law.} Calcutta: Thacker, Spink \& Co. 

 \bibitem{chap2-key47} \textbf{\textit{Śatapatha Brāhmaṇa}}. See Eggeling (1885).

 \bibitem{chap2-key48} Sathe, Sriram. (n.d.). \textit{Bharateeya Historiography.} Hyderabad: Bharateeya Itihasa Sankalana Samiti.

 \bibitem{chap2-key49} Shamasastry, R. (n.d.). \textit{Arthashastra} of Kautilya. (Translated into English) \url{http://www.lib.cmb.ac.lk/wp-content/uploads/2014/01/Arthashastra_of_Chanakya_-_English.pdf};\break Accessed on October 5, 2016.

 \bibitem{chap2-key50} Sharma, Arvind. (Ed.) (1991). \textit{Essays on the Mahābhārata}. Leiden: E. J. Brill.

 \bibitem{chap2-key51} Shendge, Malati J. (1996). “The Other Half of the Indo-European Postulate: Early India in a New Perspective.” \textit{Annals of the Bhandarkar Oriental Research Institute}, vol.~LXXXVII, Pts 1-4 (1996). pp.~179--193.

 \bibitem{chap2-key52} Sherma, Rita and Sharma, Arvind. (Ed.s) (2008). \textit{Hermeneutics and Hindu Thought: Toward a Fusion of Horizons.} Dordrecht: Springer.

 \bibitem{chap2-key53} Singhal, K. C. and Gupta, Roshan. (2003). \textit{The Ancient History of India, Vedic Period: A New Interpretation}. New Delhi: Atlantic Publishers.

 \bibitem{chap2-key54} Squarcini, F. (2005). \textit{Boundaries, Dynamics and Construction of Traditions in South Asia} Florence: Ferenze University Press.

 \bibitem{chap2-key55} Srestha, Angirasa. (2017). “[I]tihāso vedaḥ so’yamiti –itihāsa is the veda: this it is.” \url{http://indiafacts.org/access-ritual-knowledge-hinduism-case-veda-agama/} Accessed on March 12, 2017.

 \bibitem{chap2-key56} \textbf{\textit{Subhāṣitaratnabhāṇḍāgāra}}. See Acharya (2011).

 \bibitem{chap2-key57} Subramanian, V. (2010). “Vedic History and Worldly History” Parts One \& Two. In \url{http://www.kamakoti.org/hindudharma/part12/chap9.htm//V Subrahmanian} (\url{http://books.google.com/books?id=TCWU_ua5kxgC&lpg=PA436&ots=MerpQ3WOs5&dq=arthavada%20vedanta&pg=PA436#v=onepage&q=arthavada%20vedanta&f=true
}); Accessed on Sept 20, 2016.

 \bibitem{chap2-key58} \textbf{\textit{Tantravārttika}}. See Jha (1983).

 \bibitem{chap2-key59} Thapar, Romila. (2002). “Cyclic and Linear Time in Early India.” In Ridderbos (2002). pp.~26--45.

 \bibitem{chap2-key60} —. (2013). \textit{The Past Before Us: Historical Traditions of Early North India.} Cambridge, MA \& London: Harvard University Press and Permanent Black.

 \bibitem{chap2-key61} Trautmann, Thomas. (2012). “Does India Have History? Does History Have India?” \textit{Comparative Studies in Society and History} 2012; 54 (1). pp.~174--205.

 \bibitem{chap2-key62} \textbf{\textit{Vālmīki Rāmāyaṇa.}} (1992). (Text as constituted in its critical edition). 1st ed.\ Vadodara, India: Oriental Institute.

 \end{thebibliography}

\theendnotes

