
\chapter{The Science of the Sacred}\label{chapter9}

\Authorline{T. N. Sudarshan\footnote{pp xx--xx. In Kannan, K. S. (Ed.) (2019). \textit{Swadeshi Critique of Videshi Mīmāṁsā}. Chennai: Infinity Foundation India.}}

\begin{flushright}
\textit{\sf\em (tnsudarshan@gmail.com)}
\end{flushright}


\section*{Abstract}

The profound notion of the sacred (\textit{pavitratā}\index{pavitrata@\textsl{pavitratā }}) is critical to understand the variegated Indian knowledge systems (\textit{śāstra-}s\index{sastra@\textsl{śāstra}}) and their associated practices. Modern \textit{sanātana dharma }\index{sanatanadharma@\textsl{sanātana-dharma}} embodies these knowledge\break systems and is reified in its practices and various \textit{dharma-s}\index{dharma@\textsl{dharma}}. The fundamental and inherent limitations of Western scholarship arising mostly due to its origins, structure and evolution cannot grasp or confront the existence of such a conceptualization as a core structural and governing principle. The origins of Indological scholarship entwined with the colonial obsession of “othering” and its use as a tool to aid oppressive regimes has been well-discussed and documented. The neo-Orientalists\index{Neo-Orientalists} (Sheldon Pollock\index{Pollock, Sheldon} and others) have redefined post-colonial Indology, using creative and sophisticated applications of Western\break (combinations of Marxism, philology\index{Philology}, and postmodernism) methods and theories to Indian systems of knowledge. The roots of the obsession of neo-Orientalist desacralisation\index{desacralisation} (and of \textit{videśī} Indology in general) and its echoes in Indian secular discourse - the irreverence for \textit{sanātanic} conceptualizations of \textit{pavitratā} are explicated in this paper. The notion of the \textbf{sacred} - as defined by Western systems of knowledge - religious (Abrahamic), (Western) secular and (Western) scientific are discussed and juxtaposed with the \textit{dhārmic} notion of “sacredness”. The role of science and the associated discourse replacing the “sacred” in Western collective conscience and ideology is illustrated. The distributed and natural sense of Indian \textit{pavitratā}\index{pavitrata@\textsl{pavitratā}} and the centralized, institutionally enforced artificial sense of sacred of the West are contrasted. The flawed understanding of the sense of sacred and the obsession with Westernizing (liberating) India is established as the root cause of the neo-Orientalist obsession with the desacralisation\index{desacralisation} of Sanskrit, \textit{saṁskṛti}\index{samskrti@\textsl{saṁskṛti}} and \textit{sanātana dharma}\index{sanatanadharma@\textsl{sanātana-dharma}}.


\section*{Preliminaries}

The notion of the sacred is essential to the praxis-driven “discourse” of \textit{dhārmic} living. The sense of sacred underlies all human activity and is embedded in the consciousness of \textit{dhārmic} civilizational ethos. Every human activity, even the most mundane (from waking up to falling asleep), has deep sacred connotations. The bodies of knowledge specific to various \textit{sampradāya}-s\index{sampradaya@\textsl{sampradāya}} (ex: the \textit{āhnika grantha-s }in Śrīvaiṣṇava \textit{sampradāya}) which specify the context and performance of these activities are well known. Though very few communities practice these strictly today, the fact that the “sacred” dimension has (since millennia) had such a deep influence on every conceivable human activity vis-à-vis \textit{dhārmic} living, and has had the requisite textual and practical evidence to support it has to be understood as a prior cultural baseline. This baseline “sociological” state has to be the background before any serious discourse on the “attempts” at its desacralisation\index{desacralisation} (in a \textit{dhārmic} context) can be attempted.

\section*{The Neo-Orientalist\index{Neo-Orientalists} Discourse}

The neo-Orientalists are the latest (academic grouping) amongst multiple (five) waves of post-independence Indologists\index{Indology, five waves of} (Malhotra\index{Malhotra, Rajiv} 2016b). The others being the Marxists, post-colonialists, the subalternists and the postmodernists. Assuming one was to agree with this overall characterization, all the five (one can possibly view them as a co-existing continuum) have (relentlessly) attempted to desacralise the (\textit{dhārmic)} underpinnings of Indian society. These attempts are currently very much alive and active across various channels (electronic media, publishing, other channels). It takes various guises and primarily aims to “weaken” the unifying (\textit{dhārmic}) ethos of India (primarily in a political and cultural sense).

The contributions of the neo-Orientalists\index{Neo-Orientalists} have been significant - new theoretical methods, inference techniques and argument “frames” to aid and accelerate these collective efforts at desacralisation\index{desacralisation}. The synthesis of these methods by media channels and academic wielders can be seen proliferating via the “left liberal” discourse. A complete analysis of the use of these interpretive methods is beyond the scope of this paper. Attempts shall be made to give a broader understanding of the underlying issues and the nature of the motivations of the scholars being discussed.

The \textit{dhārmic} sense of sacred is closely tied to the Sanskrit language and the embedded cultural matrix (the \textit{saṁskṛti}\index{samskrti@\textsl{saṁskṛti}}). As outlined in (Malhotra\index{Malhotra, Rajiv} 2016a), Pollock\index{Pollock, Sheldon} has attempted to undermine the \textit{dhārmic} civilization via “scholarly” methods during an academic career of more than 30 years. These “researches” postulate various theses, primarily by “theorizing” about the role of the language of Sanskrit. A significant milestone of this “research” is the 2006 book, \textit{The Language of the Gods in the World of Men}. The aims of the book are to “explore” supposed historical re-invention (\textit{desacralisation}) of Sanskrit.

\begin{myquote}
“This book is an \textit{attempt to understand two great moments of transformation in culture and power} in pre-modern India. The first occurred around the beginning of the Common Era, when Sanskrit, \textit{long a sacred language restricted to religious practice, was reinvented as a code for literary and political expression}.... The second moment occurred around the beginning of the second millennium, when \textit{local speech forms were newly dignified as literary languages and began to challenge Sanskrit for the work of both poetry and polity}, and in the end replaced it. Concomitantly new, limited power formations came into existence.” 

\vskip -5pt

~\hfill (Pollock 2006:1) (\textit{italics ours})
\end{myquote}

The fundamental assumption behind the theorizations in this book is that the split between the sacred and the non-sacred was \textit{already} part of the \textit{Saṁskṛtic} tradition. Many of Pollock's theses depend on this assumption of natively present sacred \textit{versus} non-sacred dichotomies. This book and various other succeeding theses are based on this sleight. The \textit{pāramārthika}\index{paramarthika sat@\textsl{pāramārthika sat}} and \textit{vyāvahārika}\index{vyavaharika sat@\textsl{vyāvahārika sat}} categories (these are well-acknowledged categories (though not in the sense that Pollock portrays them) in the Advaita traditions but are not acknowledged in all Vedantic traditions) do not apply to all of Indian \textit{darśana-}s\index{darsana@\textsl{darśana}} and \textit{sampradāya-}s\index{sampradaya@\textsl{sampradāya}} and, strictly speaking, cannot be used as a basis from which to generalize and formulate divisive theses. Refer (Pollock 2006:3). The other sleight is misrepresentation and forceful separation of the categories of \textit{śāstra}\index{sastra@\textsl{śāstra}} and \textit{kāvya}\index{kavya@\textsl{kāvya}}. The origins of \textit{kāvya} are posited based on doubtful dating techniques and sweeping generalizations (\textit{kāvya}'s dichotomies with \textit{śāstra}) made on their basis. Pollock states as fact that there is broad agreement on these differences – but amongst whom? (Pollock 2006:3,4). With the aid of none-too-innovative story-telling, assuming nonexistent parallels to the experiences the West had with the Church, and ignoring the non-centralized nature of Indian society, Pollock rather lamely posits evolutionary reasons (those based on the influence of political-power and its centralization) for the sacred nature of Sanskrit. (Pollock 2006:28,29).

Pollock befuddles and with the aid of incorrect characterizations makes sweeping claims on the role of Sanskrit. The oral tradition – the backbone/basis of “\textit{Saṁskṛti}\index{samskrti@\textsl{saṁskṛti}} – as practiced” is rather conveniently, ignored. Acknowledging that the oral tradition would hinder chronology based manipulations and the creation of falsely ascribed origins and events, Pollock rather slyly generates facts to justify almost all of his theses. (Pollock\index{Pollock, Sheldon} 2006:50). The 2006 book, a winner of many awards, is filled with such dubious characterizations. The Kātantra school of grammar is used as a wedge to introduce the divisive thesis of native desacralisation\index{desacralisation} (Pollock 2006:62,70). A supposed lack of epigraphic evidence (no proof offered) is used to posit native attempts at desacralisation (Pollock 2006:170).We find similar observations about Pollock's innovative theories aimed at excavating non-existing native schisms and characterising them as native attempts at desacralisation (Malhotra\index{Malhotra, Rajiv} 2016a:224,226,249).

\section*{The Sacred Discourse}

Is the notion of sacred universal? Western anthropological and sociological approaches (the Western-universalist discourse) to this question have yielded many theses over the past few centuries. This question pre-supposes a \textit{“without-centric”}\index{without-centric@\textsl{without-centric}} nature of the sacred, in contrast to the \textit{“within-centric”}\index{without-centric@\textsl{without-centric}} formulation of the \textit{dhārmic} conceptualizations. Any modern discussion of the sacred is incomplete without discussing the work of Romanian philosopher and religious historian Mircea Eliade\index{Eliade, Mircea}. The influence of the Vedic civilization and Indian philosophy is apparent (and is also acknowledged by Eliade) in his work. Eliade proposes the term \textit{“hierophany”}\index{hierophany} to connote the manifestation of the divine, inherent to the nature of anything sacred (Eliade\index{Eliade, Mircea} 1959:11). He also notes that the modern West (circa 1957) finds this idea rather difficult to accept. The situation today (60 years on) has only exacerbated. (Eliade 1959:10). The sense of the sacred and the essentially “\textit{inner}” nature of its experience are acknowledged (Eliade 1959:11). The nature of divinity (which he identifies as “power” and “being”) is, according to Eliade, the reason (anthropologically) for man's deep desire to make the sacred a reality (Eliade 1959:12).

The essentially non-sacred nature of Western modernity and the lack of this existential dimension in modern living are also articulated. The non-modern is labeled as being societally primitive and archaic by Eliade; this is only to be expected as the originating perspective is the West. From a \textit{swadeshi} perspective, one would need to essentially redefine and reclaim these dissonant categories and reframe Eliade's articulation.

\begin{myquote}
“Religious man attempts to \textit{remain as long as possible in a sacred universe, and hence what his total experience of life proves to be in comparison with the experience of the man without religious feeling, of the man who lives, or wishes to live, in a desacralized\index{desacralisation} world.} It should be said at once \textit{that the completely profane world, the wholly desacralized cosmos, is a recent discovery in the history of the human spirit. It does not devolve upon us to show by what historical processes and as the result of what changes in spiritual attitudes and behavior modern man has desacralized his world and assumed a profane existence.} For our purpose it is enough to observe that \textit{desacralization pervades the entire experience of the nonreligious man of modern societies and that, in consequence, he finds it increasingly difficult to rediscover the existential dimensions of religious man in the archaic societies.”} 

~\hfill (Eliade 1959:13) (\textit{italics ours})
\end{myquote}

The fundamentally irreconcilable natures of the two modes of experiencing reality – sacred (non-Western, traditional) and profane (Western-modern) – are made explicit. Eliade calls it “an abyss”. The “ephemeral” nature of modernity and material identification of experience are also alluded to (Eliade 1959:14). Eliade's\index{Eliade, Mircea} categories and influential discourse has not had much impact, in the sense of leading to a deeper anthropological analysis of modernity and the West, though his ideas have been applied variously in other disciplines of Western academia. \textit{Swadeshi} scholars should attempt to formulate new critiques based on Eliade's\index{Eliade, Mircea} framework to help reverse the gaze.

Understanding S.N. Balagangadhara's\index{Balagangadhara, S. N.} becomes critical in this context, especially so when one discusses Western “religious” categories (such as sacred and profane). From a recent review of his seminal works

\begin{myquote}
“S. N. Balagangadhara argues that it is \textit{necessary to dissect how the West experiences the world in order to clear the ground before the contribution of Indian culture can be assessed.} For the \textit{last few hundred years, academic contexts have been dominated by questions Europe has asked. This way of asking questions means that it has not asked questions in other ways. Whether adopted by Western intellectuals or non-Western intellectuals, who parasitically formulate problems according to it, that way is tied to Western culture. Only by understanding this can we discover how Indians can ask different questions, and what contribution Indian culture can make.} His work establishes how little we understand Western culture. Speaking a Western language does not mean we understand what it is.” 

\vskip -5pt

~\hfill (Shah\index{Shah, Prakash} 2014) (\textit{italics ours})
\end{myquote}

There are fundamental epistemological issues relating to the current discourse on India, which is seriously “skewed”, being driven on Western assumptions and presuppositions. This is another “focus-area” for \textit{swadeshi} scholarship: control of epistemology thereby leading to control of the discourse. Many fundamentally flawed definitional notions need to be questioned and be laid to rest once for all, the notion of “Hindu religion” for one. In his book (Balagangadhara\index{Balagangadhara, S. N.} 1994), he provides the basis for his arguments for the (Western) sense of universality of religion referencing the various Western fields of study. Accordingly, religion is a Western conception. How flawed is the Western argument for religion and its universality? \textbf{Immensely} flawed. In 11 chapters, he systematically demolishes the Western narrative of religion as espoused both by the Westerner and the colonized non-Westerner. In his own words -

\begin{myquote}
“My aim is to show that a \textit{provincial experience of a small segment of humanity does not become universal by decree.} Nor does a specific group become \textit{‘the universal audience'} by merely pretending to be one.” 

~\hfill (Balagangadhara 1994:8) (\textit{italics ours})
\end{myquote}

As to the target audience of this book, from an Indian perspective, if one were to use recently coined terminology, it would be aimed at all of Macaulay's children and more so the intellectual sepoys. Balagangadhara\index{Balagangadhara, S. N.} stresses on the Western nature of the discourse, the questions posed and frameworks used. The evolution of religion in Europe and its influence on everything about the West, including Science is explicated.

He goes so far as to say that religion is a European notion: India \textbf{does not} have religions in the Western sense. Science, he implies is also a new Western form of religion sharing much of the same assumptions and proselytizing zeal as Christianity. He introduces the concept/notion of a \textbf{\textit{Root Model of Order}}\index{root model of order@\textsl{root model of order}}. The root model of order is that which enables structuring of knowledge and learning in a culture. (Balagangadhara 1994:400).

The influence of this root model of order on the evolution of a society (in this case, the “Western”) is apparent once we acknowledge the existence of a root model of order. The pursuit of knowledge, the organization of knowledge and the structures of learning are influenced by the root model of order (Balagangadhara 1994:401). Science too has been influenced by this root model of order resulting in accumulations of theoretical knowledge (Balagangadhara 1994:403). He goes so far as to say that religion was a necessary condition for science to develop the way it has in the West (Balagangadhara 1994:406). The so-called scientific attitude is only a continuation of the religious attitude (Balagangadhara 1994:407). On the effects that the religion based root model of order has on learning structures, he effectively identifies the dominant process of learning as the \textit{theoretical} (Balagangadhara 1994:410). What would a different root model of order look like? What would be the configuration of such a “different” culture? How is the “Indian” experience different when it is compared with the “Western”? Balagangadhara (1994:411) characterises it as a practical approach — \textit{Ritual}\index{ritual}.

\begin{myquote}
“We can now take the crucial step towards \textit{identifying the entity that could structure another configuration of learning. It is a structured set of generic actions; it could be described as a-intentional, agent-less, and goal-less.} Does such an entity exist? Yes. Where? In Asia. What is it? \textit{Ritual.”} 

~\hfill (Balagangadhara 1994:415) (\textit{italics ours})
\end{myquote}

According to Balagangadhara\index{Balagangadhara, S. N.}, the performative and practical nature of knowledge is unique to the \textit{dhārmic} civilisation and is fundamentally different with the Western structures of knowledge and society (Balagangadhara\index{Balagangadhara, S. N.} 1994:415). It should be noted that this scholar has been attacked and vilified by Western academia for his deep scholarship and ideas. This should provide sufficient basis as to why his constructs have been used in the current context. This, in my opinion, is possibly another focus-area for \textit{swadeshi} scholarship — a scholar whose ideas need to be engaged with more constructively.

The understanding of Western religions – of the Abrahamic variety – has been dimensionally enhanced by scholars like Malhotra\index{Malhotra, Rajiv}. In his book, \textit{Being Different} (Malhotra 2011), he introduces new juxtaposed categories “Embodied Knowing”\index{embodied knowing} vs “History-centrism”\index{history-centrism} which highlight the stark differences between \textit{dhārmic} and Abrahamic approaches to the sacred.

\begin{myquote}
“\textit{Dharma}\index{dharma@\textsl{dharma}} and Judeo-Christian traditions differ \textit{fundamentally in their approaches to knowing the divine}. The \textit{dharma} family (including Hinduism, Buddhism, Sikhism and Jainism) has developed an extensive range of inner sciences and experiential technologies called \textit{‘adhyatmavidya’} to access divinity and higher states of consciousness... \textit{Their truth must be rediscovered and directly experienced by each person.} I have coined the term \textit{embodied knowing} to refer to inner sciences and \textit{adhyatma-vidya}.” 

~\hfill (Malhotra\index{Malhotra, Rajiv} 2011:5,6) (\textit{italics ours})
\end{myquote}

The limited nature of the Abrahamic (Western) approach to religion (and hence to the sacred) and their dependence on events (actual or contrived) is explicitly characterised by Malhotra as “History-centrism”. The discussion of the nature of the sacred is deeply affected by this History-centric baggage that the West carries. Relevant as it is to our discussion, it is important to note that all scholarship and “realities” emanating from the West (culturally), including the “sacred” discourse, are compromised because of these civilizational (ideological) realities. Malhotra describes this situation thus:

\begin{myquote}
“I have coined the term \textit{history-centrism} to refer to this fixation on specific and often incompatible claims to divine truth revealed in the course of history. I regard \textit{this historical fixation as the major difference between dharmic and Judeo-Christian paths and as a problem which can breed untold psychological, religious and social conflict.”} 

~\hfill (Malhotra 2011:5,6) (\textit{italics ours})
\end{myquote}

\section*{The “Western” Discourse}

The post-enlightenment, imperialistic expansion of Europe, which\break brought along with it the Industrial Revolution, changed fundamentally, the way in which Europe engaged with its past and with the rest of the (non-European) world. The “West” was defined in a sociological sense during this period.

\begin{myquote}
“The West...is not to be found by recourse to a compass. Geographical boundaries help to locate it, but they shift from time to time. The West is, rather, a cultural term, but with a very strong diachronic dimension. It is not, however, simply an idea, it is a community. It implies both a historical structure and a structured history....\textit{The West, from this perspective, is not Greece, and Rome and Israel but the people of Western Europe turning to the Greek and Roman and Hebrew texts for inspiration, and transforming those texts in ways that would have astonished their authors.} (Berman 1983: 2-3; italics in the original.)” 

~\hfill (Balagangadhara\index{Balagangadhara, S. N.} 1994: 396) (\textit{italics ours})
\end{myquote}

The modern discourse of the West through the construct of Science, creative Western historiography, the functionalism\index{functionalism} of Durkheim\index{Durkheim, E.}, the rational sociology of Weber\index{Weber, Max} and the dialectic materialism of Marx\index{Marx, Karl} slowly but surely removed any discourse of sacrality associated with the religion – (Judeo-Christian) – influenced structures (family, marriage, worship, food, festivals etc.) in society. The core structures that religion had created remained as they were and were not dismantled. The modernity in the Western discourse replaced one universal (Christian) worldview with other universal worldviews.

Marshall Sahlins\index{Sahlins, Marshall} in his (scathing) 2008 book on the Western understanding of human nature, viz. \textit{The Western Illusion of Human Nature,} explicates the arrogance of the West in historical perspective.

\begin{myquote}
“For more than two millennia, the \textit{peoples we call “Western” have been haunted by the specter of their own inner being: an apparition of human nature so avaricious and contentious that, unless it is somehow governed, it will reduce society to anarchy.} The political science of the unruly animal has come for the most part in two contrasting and alternating forms: either hierarchy or equality, monarchical authority or republican equilibrium: \textit{either a system of domination that (ideally) restrains people's natural self-interest by an external power; or a self-organizing system of free and equal powers whose opposition (ideally) reconciles their particular interests in the common interest.... I claim it is a specifically Western metaphysics, for it supposes an opposition between nature and culture that is distinctive to the West and contrastive with the many other peoples who think beasts are basically human rather than humans are basically beasts—for them there is no “nature,” let alone one that has to be overcome.}” 

\vskip -5pt

~\hfill (Sahlins 2008:1,2) (\textit{italics ours})
\end{myquote}

\vspace{-.5cm}

\section*{The \textit{Dhārmic} nature of the Sacred}

So much has been written (and much yet to be written) on the sacred nature of the \textit{dhārmic} civilization that it is impossible to discuss all its varied perspectives. The Vedic civilization is built on the fundamental basis of the Veda-s\index{vedas@\textsl{Veda}-s}, the oral-signified chants of primordial origin that encapsulate the “vibrational” basis of cosmological existence. Currently around 12 \textit{śākhā-}s of 1131 branches ($\sim$1\%) are extant. The \textit{śāstra-}s\index{sastra@\textsl{śāstra}} (basis-knowledge) and \textit{śrauta (}practice-centric\textit{)} literature form the basis of the \textit{śruti} – and are of divine origin. The \textit{smṛti}\index{smrti@\textsl{smṛti}} genre of interpretive literature, the \textit{darśana}\index{darsana@\textsl{darśana}} texts and meta-texts also describe this immanent “sacred” in variegated dimensionality. All of the \textit{upāsanā}\index{upasana@\textsl{upāsanā}} (praxis) genres (\textit{stotra}\index{stotra@\textsl{stotra}} and \textit{mantra-}s\index{mantra@\textsl{mantra}}) of literature are experiential entry points to the sacred dimension. The \textit{darśana-}s and associated related literatures are also based on this “essential” sense of the sacred and attempt to discuss and describe this essence multi-dimensionally. This all-encompassing sacrality of the \textit{dhārmic} nature of knowledge and its vast literatures is unquestionable to anyone living in this land. The geography and history of this land are also considered to be sacred. The deeply practical culture of learning and embodied\index{embodied knowing} living that is unique to \textit{dhārmic} living has at its core - \textbf{\textit{the sacred}}. To even postulate that the sacred is an externally manifested man-made attribution (the nature of \textit{pavitratā}\index{pavitrata@\textsl{pavitratā}}) is impossible to conceive, unless of course, one has the requisite motives to do so. It is the very nature of the cosmos. Any uninhibited, motive-free human in his natural state will acknowledge its presence. To deny this essence is neither groundbreaking nor innovative. It can be at best considered a willful display of hubris and derision masquerading as scholarship.

\vspace{-.5cm}

\section*{The Sociological Dimension}

A critical discursive dimension of the desacralisation\index{desacralisation} narrative is to focus on the social ills of society and attribute them to the core tenets of the \textit{dhārmic} society. The manufacture of causation attributable to the core structures is a standard academic trope. Poverty, illness, colonization, social stratification etc. — all of these are generally attributed to the nature of \textit{dharma}\index{dharma@\textsl{dharma}}. This discourse normally entails that \textit{dharma} and its sense of the sacred make society weak. The stronger way for a society is violence and conquest – the recommended Western way. The philological methods of Sheldon Pollock\index{Pollock, Sheldon} aim at excavating (via political\index{Philology! political} philology\index{Philology}) sociological ills through creative analysis of texts. A prescriptive application (via Liberation Philology\index{Philology! Liberation}) of Western sociological constructs is presented as “solution” to these ills. This in short - is the essence of the neo-Orientalist\index{Neo-Orientalists} discourse.

Programs of “liberation” (\textit{The White Man's Burden}) have been the standard colonial socio-experimentation used by the colonial powers of Europe to justify the excesses of primitive violence and greed. During the past two millennia, similar “programs” were used to justify slavery, the crusades, native-American genocide and various other violent enterprises sponsored by the Church and the \textit{West} in its various forms – and forms of it are seen today in its (that of West) interferences across the world (in the guise of world peace, human-rights etc.). Aurobindo's\index{Sri Aurobindo} essay on Social Reform is one of the earliest and is possibly one of the more coherent responses to the Westernization discourse.

\begin{myquote}
“\textit{Reform is not an excellent thing in itself as many Europeanized intellects imagine; neither is it always safe and good to stand unmoved in the ancient paths as the orthodox obstinately believe. Reform is sometimes the first step to the abyss,} but immobility is the most perfect way to stagnate and to putrefy. Neither is moderation always the wisest counsel: the mean is not always golden. It is often a euphemism for purblindness, for a tepid indifference or for a cowardly inefficiency.” 

~\hfill (Aurobindo 1890-1910:Social Reform) (\textit{italics ours})
\end{myquote}

This (in my opinion) is to be acknowledged as Aurobindo's prescient response to the exercise of Liberation Philology\index{Philology! Liberation} - which prescribes the import of Western societal constructs as solutions to ills of \textit{dhārmic} society. Solutions need to be wrought using internal mechanisms, not imported.

\begin{myquote}
“Neither antiquity nor modernity can be the test of truth or the test of usefulness. All the Rishis do not belong to the past; the Avatars still come; revelation still continues.” 

\vskip -5pt

~\hfill (Aurobindo\index{Sri Aurobindo} 1890-1910:Social Reform)
\end{myquote}

\newpage

Manu has been the primary target of this “liberation” discourse - the principal target of the subalterns, the postmodernists and the favorite whipping boy of the dalit-studies programs. \textit{Smṛti-}s\index{smrti@\textsl{smṛti}} need to be rewritten contextually – there definitely is a need to recalibrate “details” of practice in cognizance of changes in society. The role of specific customs also need to be questioned and if possible re-contextualized without losing sight of the underlying motivation and intent. Aurobindo gives a veritable prescription to address societal ills in the Indian context via Indian sociological frameworks. Does blind following of customs constitute \textit{dharma}\index{dharma@\textsl{dharma}} or is the opposition to all of it \textit{dharma}? What is the balance? How do we seek evolutionary harmony? What then, is the direction of social reform? Aurobindo has sagely advice.

\begin{myquote}
“Men have long been troubling themselves about social reform and blameless orthodoxy, and orthodoxy has crumbled without social reform being effected. But all the time God has been going about India getting His work done in spite of the talking. \textit{Unknown to men the social revolution prepares itself, and it is not in the direction they think.”} 

~\hfill (Aurobindo 1890-1910:Social Reform) (\textit{italics ours})
\end{myquote}

\section*{Discussion}

The preceding sections help understand the nature of the sacred in a Western sense and also in the \textit{dhārmic} sense. The Western experience with Christianity has influenced almost all of its anthropological and sociological discourse. Even the supposedly objective discourse of Science is influenced at its core by the experiences of Western religion. Both Balagangadhara\index{Balagangadhara, S. N.} and C.K.Raju\index{Raju, C. K.} have reached similar conclusions using distinctly different approaches - the deeply (Christian) religious nature of modernity and science. Though not apparent in external trappings, surface structures and symbols, the root model\index{root model of order@\textsl{root model of order}} of order (to use Balagangadhara's terminology) governing Western society today is the same – \textit{religion} (Christianity) – since nearly two millennia. Even the superficially secular socio-models of Marxism have primarily been theoretical constructs with relatively shallow practical impact (the reign of the USSR notwithstanding). None of these Western models have yet to shake off their “religious” core. The structures of power, influence and (most importantly) \textbf{learning} - all derive from the same root model\index{root model of order@\textsl{root model of order}} of order.

The perspectives that are allowed by the introduction of the construct of a \textit{configuration of learning} are exceedingly illuminating. Any society governed by the book requires theorization as essential basis for any sort of knowledge. The “written” has supremacy over experience and empirical proof. Western science at its core is a religion. Most of the “fundamental” learning is theoretical; all of mathematics is theoretical and axiomatically biased (assumptions of the nature of logic and inference are peculiar to the West). The theories of science too are of similar nature. Technology, driven by the materialistic and consumptive nature of capitalism, ignores most of the “biased basics” which govern science and mathematics. Its role is of an “applied” nature, limited to manipulating in the best possible manner (profit motives of capitalism) some principles (however incorrect) derived from \textit{theoretical} science. Technology does not promise or guarantee perfection or universal correctness, but performs within well-defined limits. The relentless cycles of consumption and waste that drive capitalism also drive the ever improving (but forever imperfect) cycles of technology.

The \textit{dhārmic} nature of knowledge and learning is of a fundamentally different nature. How? It is about understanding and acknowledging the \textit{complete} nature of reality (and the \textit{limited} nature of human senses). Techniques and practices developed by Vedic masters over millennia to help understand the dimensions of reality (in the \textit{dhārmic} systems, consciousness is the fundamental reality, not materialism) require, as a result, a learning culture that is experiential having a practical — not theoretical — basis.

Traditional learning is achieved through a personal \textit{guru} \hbox{(\textit{gurukula-}s)}, wherein the teacher imparts knowledge that is contextual to the learner and is primarily on the experiential plane. As Indian learning is mostly around the planes of practice (including those activities which involve the transcending of the physically apparent dimensions), it depends on ritual\index{ritual} as its primary carrier. The notion of ritual is central to the Indian “\textit{learning}” experience. \textbf{\textit{The notion of the sacred thus becomes much more important to a practical culture than to a theoretical culture.}}

In the Western system of Religion, control is centralized and the notions of knowledge and identifications of the sacred are by \textit{“consensus \& decree”}. Similar underlying structures and phenomena can be seen in the praxis of Science (academic journals, the Nobel Prize etc.) too. The \textit{dhārmic} notion of sacred is essential for \textit{“practice”} and underlies all human action. Without it, the learning (practice) culture will not have survived. Oral tradition is one among multiple modes of knowledge transmission (textual, oral and other modes (the \textit{śaktipāt}). The sacred underlies all of these transmission modes. The configuration of learning and practice is a fundamental structural difference. Once one grasps this, it becomes all the more obvious why the notion of the sacred is essential in \textit{dhārmic} societies.

The neo-Orientalists\index{Neo-Orientalists} are (as should be apparent by now) only continuing the theoretical exercises driven by the religion-centric root model\index{root model of order@\textsl{root model of order}} of order governing the West. The desacralisation\index{desacralisation} that has happened in the West via Science has only succeeded in transferring the “theoretical sacred” notions from religion to the edifice of Science. The deeper structures – learning configuration and root models of order – are essentially the same. The need to desacralize is important in theoretical cultures, and especially so when there is any encounter with an “\textit{other religion”}. Orientalism\index{Orientalism} of the preceding centuries was precisely this reaction. Indology and its school under discussion (neo-Orientalism\index{Neo-Orientalists}) are only continuing this exercise. Academic discourse is the “intellectual” mechanism provided by Western structure to enable systematic engagement with the \textit{other}. The framework needed to assert control and co-opt (digest) the \textit{dhārmic} other into the prevalent Western universalist discourse is thus made possible. In the light of this new understanding, Indology can be seen to be a peculiar form of (structural) anthropology – \textit{to} \textit{explain a } \textbf{\textit{distributed, practice-oriented learning culture}} in terms of a \textbf{\textit{centralized theoretical learning culture.}}


\section*{Implications}

From a \textit{dhārmic} perspective, the essential nature of the human as conceived by the West is very limited. The understanding of the complete nature of reality is also limited. The structure of learning that underlies the West is theoretical in its essential nature. The peculiarity of the (Western, Christian) assumptions that underlie Western Mathematics and Science is well known. The material artifacts that signify the superiority of the Western worldview in recent centuries (mostly driven by need for conquest and plunder) is the primary reason that the dominant discourse today is Western in nature. The academic structures of the West lie at the forefront of this conquest. The role of the neo-Orientalists\index{Neo-Orientalists} is critical for continuing the Western-universalist world-view. India's core \textit{dhārmic} structure has been under theoretical onslaught since the inception of Indology. Even after centuries of European colonization, the \textit{dhārmic} structures have not succumbed to these frontal attacks. Not only does \textit{swadeshi} scholarship need to address the arguments of the West but also address those of the (West-trained) ethnically Indian (\textbf{\textit{sepoy}}) scholars. As recently conceptualized, 70 years of independent India has produced five “waves” of sepoy-assisted Western interpretation of \textit{dhārmic} systems. Should these waves of Western interpretation be allowed to interpret events and influence media and other channels (academia) unopposed? Is it not time that these theories be analyzed from a \textit{swadeshi} perspective and dealt with on our own terms? Should these Western ideas continue to influence the \textit{dhārmic} discourse? It is thus appropriate, now, that we reflect on Sahlins' succinct description of the Western understanding of human nature.

\begin{myquote}
“It's all been a huge mistake. My modest conclusion is that \textit{Western civilization has been largely constructed on a mistaken idea of “human nature.”} (Sorry, beg your pardon; it was all a mistake.) It is probably true, however, \textit{that this perverse idea of human nature endangers our existence.}” 

~\hfill (Sahlins\index{Sahlins, Marshall} 2008:112) (\textit{italics ours})
\end{myquote}


\section*{Conclusion}

The discussion of the nature of the desacralisation\index{desacralisation} attempted by neo-Orientalist scholarship required investigation of the conceptual structures that underlie Western civilization. Interesting core structures which have a basis in the Western idea of religion are revealed. The intellectual evolution of the West via the path of Science required the creation of alternative discursive structures that would transfer the Western notion of the sacred (along with many other structures) from religion to science. The theoretical process of desacralisation, via the rhetorical devices of the humanities and the (anthropological) social-sciences has been evolving over centuries. These devices have been innovatively used by Western scholarship to intellectually dismantle other civilizations historically. The Indian experience of this “scholarship” via the schools of Indology in its various \textit{avatars} is slowly being acknowledged to be a civilizational threat. 

This academic discourse and the resulting “practical” process of desacralisation\index{desacralisation} will continue, for, it is part of the proselytizing nature of the Western root model\index{root model of order@\textsl{root model of order}} of order. From a \textit{swadeshi} perspective, it is important to acknowledge this reality. Scholarship, which acknowledges these realities and provides coherent narratives, based on \textit{dhārmic} root models of order, practice-centric configurations of learning and the \textit{dhārmic} ethos, are essential. The variegated conception and perception of the “sacred” based on practice-oriented (ritual\index{ritual}) \textit{dhārmic} knowledge systems must be the \textbf{primary basis} for the\textit{ all-encompassing syncretic nature of dhārmic society}. There should be no compromise.

\section*{Bibliography}

\begin{thebibliography}{99}
\bibitem{chap9-key01} Alexander, J. C. (1982). \textit{Theoretical Logic in Sociology. Marx and Durkheim}. London: Routledge \& Kegan Paul.

 \bibitem{chap9-key02} Alston, W. P. (1991). \textit{Perceiving God: The Epistemology of Religious Experience}. Ithaca, NY: Cornell University Press.

 \bibitem{chap9-key03} Aurobindo. (1890-1910). \textit{Sri Aurobindo Birth Centenary Library. Volume 3: The Harmony of Virtue.} \textless \url{http://incarnateword.in/sabcl/03/social-reform}\textgreater . Accessed on December 09, 2016

 \bibitem{chap9-key04} —. (2006). \textit{The Complete Works of Sri Aurobindo}. Pondicherry: Sri Aurobindo Ashram Publications Department.

 \bibitem{chap9-key05} Balagangadhara, S. N. (1994). \textit{“The Heathen in His Blindness”: Asia, the West, and the Dynamic of Religion}. Leiden: E.J. Brill.

 \bibitem{chap9-key06} —. (2012). \textit{Reconceptualizing India Studies}. New Delhi: Oxford University Press.

 \bibitem{chap9-key07} —., and Jhingran, D. (2014). \textit{Do All Roads Lead to Jerusalem?: The Making of Indian Religions}. New Delhi: Manohar Publications.

 \bibitem{chap9-key08} Bhawuk, D. P. (2011). \textit{Spirituality and Indian Psychology: Lessons from the Bhagavad-Gita}. New York: Springer.

 \bibitem{chap9-key09} Cornelissen, M., Misra, G., and Varma, S. (2011). \textit{Foundations of Indian Psychology}. Delhi: Pearson.

 \bibitem{chap9-key10} Eliade, M. (1958). \textit{Patterns in Comparative Religion}. New York: Sheed \& Ward.

 \bibitem{chap9-key11} —. (1959). \textit{Cosmos and History: The Myth of the Eternal Return}. New York: Harper.

 \bibitem{chap9-key12} —. (1978). \textit{A History of Religious Ideas}. Chicago: University of Chicago Press.

 \bibitem{chap9-key13} —., and Trask, W. R. (1958). \textit{Yoga: Immortality and Freedom}. New York: Pantheon Books.

 \bibitem{chap9-key14} —, (1959). \textit{The Sacred and the Profane: The Nature of Religion}. New York: Harcourt, Brace \& World.

 \bibitem{chap9-key15} Haralambos, M., and Holborn, M. (2008). \textit{Sociology: Themes and Perspectives}. London: Collins.

 \bibitem{chap9-key16} King, R. (1999). \textit{Orientalism and Religion: Postcolonial Theory, India and 'the Mystic East',} London: Routledge.

 \bibitem{chap9-key17} Malhotra, R. (2011). \textit{Being Different.} New Delhi: HarperCollins.

 \bibitem{chap9-key18} —. (2016a). \textit{The Battle for Sanskrit: Is Sanskrit Political or Sacred, Opressive or Liberating, Dead or Alive? } New Delhi: HarperCollins.

 \bibitem{chap9-key19} —. (2016b). The Five Waves of Indology. \url{https://www.pgurus.com/rajiv-malhotras-lecture-on-swadeshi-indology-at-ignca-new-delhi-part-1/} Accessed on January 15, 2017.

 \bibitem{chap9-key20} Marx, K., and Raines, J. C. (2002). \textit{Marx on Religion}. Philadelphia: Temple University.

 \bibitem{chap9-key21} Pollock, S. I. (2006). \textit{The Language of the Gods in the World of Men: Sanskrit, Culture, and Power in Premodern India}. Berkeley: University of California Press.

 \bibitem{chap9-key22} Raju, C.K. (2011) \textit{Decolonising Math and Science Education}, \url{http://ckraju.net/papers/decolonisation-paper.pdf}. Accessed on January 15, 2016.

 \bibitem{chap9-key23} —. (2010), \textit{Nonwestern Logic}, \url{http://ckraju.net/papers/Nonwestern-logic.pdf}. Accessed on January 15, 2016.

 \bibitem{chap9-key24} Rao, K. R., and Paranjpe, A. C. (2016). \textit{Psychology in the Indian Tradition}. New Delhi: Springer.

 \bibitem{chap9-key25} Sahlins, M. (2008). \textit{The Western Illusion of Human Nature: With Reflections on the Long History of Hierarchy, Equality and the Sublimation of Anarchy in the West, and Comparative Notes on Other Conceptions of the Human Condition}. Chicago, IL: Prickly Paradigm Press.

 \bibitem{chap9-key26} Shah, Prakash. (2014). “Critiquing the Western Account of India Studies within a Comparative Science of Cultures”. \textit{International Journal of Hindu Studies}. 18 (1): pp. 67--72. doi:10.1007/s11407-014-9153-y.

 \bibitem{chap9-key27} Sugirtharajah, S. (2003). \textit{Imagining Hinduism: A Postcolonial Perspective}. London: Routledge.

 \bibitem{chap9-key28} Wynn, M. (2008). “Phenomenology of Religion.” \textit{Stanford Encyclopedia of Philosophy.} \url{http://plato.stanford.edu/entries/phenomenology-religion/} Accessed on November 14, 2016.

 \end{thebibliography}

