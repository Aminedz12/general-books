
\chapter*{Our Contributors \namesinorder{(in alphabetical order of last names)}}\label{contributors}

\vspace{-1cm}

\section*{Megh Kalyanasundaram}

Megh Kalyanasundaram is an Indian citizen with close to nine years of lived experience in China. He is an alumnus of DAV Gopalapuram, University of Madras and ISB (Indian School of Business) where his study has included quantitative and qualitative research methods as part of his post-graduate specialization in Strategy, Marketing, Leadership. His professional experience includes stints as Market Leader in the China-entity of a Fortune 40 technology firm and as Head of Business Development, Sales and Marketing of an Indian talent development multinational. He also has served a term on the Board of a non-profit (Shanghai Indian Association) as General Secretary Sponsorship and Music. His research interests broadly revolve around ancient India in global and transnational history narratives with focus on some aspects of ancient Indian chronology. Other professional pursuits have included building differentiated digital platform of Indic texts targeted at specific learning and research needs, singing and composing, compering and event planning.

\vspace{-.3cm}

\section*{K S Kannan}

Prof. K S Kannan is the Academic Director of the Swadeshi Indology Conference Series. He is a Chair Professor at IIT-Madras, a Former Director, Karnataka Sanskrit University, a former professor at the Centre for Ancient History and Culture, Jain University, Bengaluru. He has a Ph.D. in Sanskrit and has been awarded honorary D.Litt. for his contributions to the field by Rastriya Samskrit Vidya Peetha, Tirupati. His pursuits include research into Sanskrit literature and Indian Philosophy and he is the author of about 20 books, prominent among them being \textit{Theoretical Foundations of Ayurveda} published by FRLHT, Bengaluru (2007) and \textit{Virūpākṣa Vasantotsava Campū}, Annotated Edition, published by Kannada Vishvavidyalaya, Hampi, Karnataka (2001). Vibhinnate (published 2016, Bangalore) a Kannada translation of \textit{Being Different}, a seminal work of Rajiv Malhotra, is co-authored by him. He is the General Editor of the Swadeshi Indology Conference Proceedings series of which this is the fifth volume.

\vspace{-.3cm}

\section*{T. N. Madhusudan}

T.N. Madhusudan has been in the technology start-up space for the last decade working on large scale data analysis, machine learning, text analysis, and developing scalable systems with the latest technology stacks. He has a pedigree designing/building robots, writing hybrid controllers and understanding the whole analog/digital space, enjoys coding/hacking and continues to publish. He has published 25+ research papers in information integration, online advertising, e-commerce, product design. More available here \url{https://www.linkedin.com/in/madhutherani}. He has deep roots in the theories and practice of \textit{sanātana} primarily through his \textit{sampradāya} and feels that the need to develop a more critical understanding of the West using our own \textit{dṛṣṭi} is critical - not only for our own civilization - but for the future of humanity.

\vspace{-.3cm}

\section*{Alok Mishra}

Alok Mishra is a young scholar doing his M.A. (Ācārya) in Vyākaraṇa from Shringeri and has a B.A (Śāstri) from Bhopal. He has studied the Yajurveda (Śukla) from Samanvaya Ved Vidyālaya, Haridwar and Badri Bhagata Veda Vidyālaya (VHP’s wing), Delhi. Apart from this he has studied Vedānta in Mattur, Karnataka.

\vspace{-.3cm}

\section*{Manogna H Sastry}

Manogna H Sastry is an astrophysicist from the Indian Institute of Astrophysics, focussing on Inflationary Cosmology. She has served as the Chief Operations Officer and Research Associate at Centre for Fundamental Research and Creative Education for several years. Her research interests span major domains including - astrophysics, Indology, civilisational studies, sustainability and education. She is a passionate sustainability practitioner and environmentalist, working on solid waste management issues of Bengaluru.

\vspace{-.3cm}

\section*{T. N. Sudarshan}

Therani Nadathur Sudarshan is a computer scientist, programmer and hands-on technologist / engineer, start-up founder, entrepreneur and technology consultant. He is deeply interested in discovering the immense “practicality” of the Indian Knowledge Systems. His primary research interests lie in Symbol Systems for representation and intelligence - spanning man-made material systems (AI), naturally occurring systems (biological) and the Indic symbol systems. \textit{Sampradāya} studies (Viśiṣṭādvaita) and practices are part of his upbringing and immensely influence every activity. He is also an active participant in the vibrant temple culture / events at many Vishnu temples in Tamil Nadu.

\vspace{-.3cm}

\section*{Shrinivas Tilak}

Dr. Shrinivas Tilak, Ph.D. History of Religions, McGill University, Montreal, Canada - has taught courses related to India and Hinduism at Concordia University in Montreal, Canada. His publications include \textit{The Myth of Sarvodaya: A Study in Vinoba's Concept} (New Delhi: Breakthrough Communications 1984); \textit{Religion and Aging in the Indian Tradition} (Albany, N. Y.: State University of New York Press, 1989), \textit{Understanding Karma in Light of Paul Ricoeur's Philosophical Anthropology and Hermeneutics} (Charleston, SC: BookSurge, revised, paperback edition, 2007). He is also the Editor of the Proceedings of Swadeshi Indology Conference – 3.

\vspace{-.3cm}

\section*{Charu Uppal}

Charu Uppal, is a Senior Lecturer at Karlstad University in Sweden. Her research, which generally follows under the broad umbrella of Media Studies focuses on the role of media in bringing about social change, identity formation and mobilizing citizens towards cultural and political activism. Her work has appeared in journals such as \textit{Journal of Creative Communication, International Communication Gazette and Global Media and Communication.}

