\chapter{{\sl\bfseries Rāmāyaṇa} Casts its Ancient Spell?}\label{chapter3} 

\lhead[\small\thepage\quad Reclaiming~ {\sl Rāmāyaṇa}]{}
\rhead[]{\small\thechapter. {\sl Rāmāyaṇa} Casts its Ancient Spell?\quad\thepage}

\addtoendnotes{\protect\bigskip{\noindent\Large\bfseries\thechapter. {\sl\bfseries Rāmāyaṇa} Casts its Ancient Spell?}\bigskip}


On Sept 25, 1990, L K Advani\index{Advani, L. K.} started on a {\sl Rām-ratha-yātrā} from Somnath to Ayodhya\index{Ayodhya} to call for the rebuilding of the Ram Mandir in Ayodhya. The {\sl ratha} — which resembled that of Arjuna’s\index{Arjuna} in the tele-serial {\sl Mahābhārata}\index{Mahabharata@\textsl{Mahābhārata}} — was welcomed enthusiastically — with zealous cries of “Jai Śrī Rām!” Its route was strewn with flowers, women carried coconuts, incense sticks and sandalwood paste to offer {\sl ārati} to the {\sl ratha} which was then smeared with {\sl tilak} and the dust off its wheels was taken religiously. It did much to stir national unity — also, it allegedly catalyzed the destruction of the Babri Masjid\index{Babri Masjid} two years later.   

Pollock’s article, “{\sl Rāmāyaṇa} and Political Imagination in India” (1993)\endnote{This is Pollock’s only essay that operates on Plane 3 philology.\index{philology!Plane 3}}, commences with Advani’s {\sl ratha-yātrā}. In what follows, Pollock attempts to show how the {\sl Rāmāyaṇa}, “a heroic tale of love, loss, and recovery from the classical past should be invoked to empower and give substance to the politics of the present” (Pollock 1993:262). From historic sources — archeological, epigraphical, and literary — he tries to demonstrate that the “{\sl Rāmāyaṇa} imaginary”\index{imaginary} (an image of Rāma as a “destroyer” of {\sl rākṣasa}-s)\index{rakshasa@\textit{rākṣasa}} came to occupy “a public political space” from the twelfth century onwards — when India faced a cultural and political confrontation by the Muslims/Turks.\index{Muslim!political confrontation}\index{Muslim!cultural confrontation} At this convenient juncture, he says, the Hindu\index{Hindu!king} rulers used the {\sl Rāmāyaṇa} to consolidate their power — they cast themselves in the role of Rāma\index{Rama@Rāma} and the foreign invaders as {\sl rākṣasa}-s; and this trend, Pollock says, continues to this day. Furthermore, Pollock claims that Vālmīki’s\index{Valmiki@Vālmīki} {\sl Rāmāyaṇa} itself carries certain elements/instruments in its narrative that allows for an “easy deployment” of itself for “dangerous” political purposes: it is for this reason that it was employed in the twelfth century, and it is for this reason that it continues to be employed for perpetrating “political outrages” today. In his words,   

\begin{myquote}
“I suggest in what follows that the {\sl Rāmāyaṇa} came alive in the realm of public political discourse in western and central India in the eleventh to fourteenth centuries in a dramatic and unparalleled way. I believe the text offers unique imaginative instruments - in fact, two linked instruments - whereby, on the one hand, a divine political order can be conceptualized, narrated, and historically grounded, and, on the other, a fully demonized Other can be categorized, counterposed, and condemned. The makers of elite culture in medieval South Asia chose these instruments for the work of divinization and demonization at this historical moment because of the emergence of two enabling conditions. One was the peculiar salience that a far older political theology now seems to have achieved in the service of the legitimation or enhancement or perhaps just self-understanding of kingship. The other was the appearance of Others who - whether, in fact, they presented an unprecedented unassimilability or could opportunistically be represented as such - were especially vulnerable to the demonizing formulation the {\sl Rāmāyaṇa} made available.”
\hfill Pollock (1993:264)
\end{myquote}

\vskip .2cm

Pollock’s views may be summarized thus:
\begin{itemize}
\itemsep=1pt
\item[(a)] Vālmīki’s\index{Valmiki@Vālmīki} {\sl Rāmāyaṇa} allows for the “divinization” of Hindu\index{Hindu!political order}\break political order, and the “demonization” of the Other.  Accordingly, then, the {\sl Rāmāyaṇa} has — for a thousand years — served as a code in which “proto-communist relations could be activated and theocratic legitimation could be rendered” (Pollock 1993:288). 

\item[(b)] From the twelfth century onwards — i.e., simultaneous with the “appearance of the Others” — it came to be employed in equations of politics and power: Hindu\index{Hindu!king} kings could be portrayed as the protector/“divine-king” Rāma, and the “outsiders” were {\sl rākṣasa}-s.  

\item[(c)] For this reason, it continues to be deployed today for “dangerous” Hindu-Muslim\index{politics!Hindu-Muslim} politics. 
\end{itemize}

Let us examine carefully the substantiation that Pollock offers for his claims.

\newpage

\section{Archeological Evidence (Rise of Rāma Temple Cult)}\label{sec3.1}
\index{evidence!archeological}

Pollock’s first evidence is that of archeology — the “rise of Rāma\index{Rama@Rāma!temples of} temples” after the twelfth century. Rāma-worship and “Rāma-cult” commences, says Pollock, somewhere in the twelfth century, “assumes a prominent place within the context of a political theology”, and attains a “centrality” by the fourteenth century. Its growth can, he claims, be traced in virtual synchrony with “a set of particular historical events”— i.e. the “appearance of the Others”. 

In actuality, there is little work that traces the “rise of Rāma cult” in India — and it is perhaps Hans Bakker’s\index{Bakker, Hans} work that we must resort to. Bakker’s argument is similar to that of Pollock’s — he postulates the emergence of Rāma worship “in the latest period of independent Hindu\index{Hindu!rule} rule in north India” and before the “firm establishment of Muslim\index{Muslim!power, ``firm establishment'' of} power” (Bakker 1986:66). According to him, worship of the other {\sl avatāra}-s\index{avatara@\textit{avatāra}} of Viṣnu were based on “regional, popular, and not specifically Vaiṣnava traditions”— the Rāma cult had to wait for “favorable historical circumstances”. He writes,  

\begin{myquote}
“This seems to have occurred when Hindus were driven into a defensive position by Muslim power, but this factor would never have led to a cult of such dimensions, impact and importance, had not a wave of emotional devotion ({\sl bhakti}) a particular kind completely transformed the outlook and character of Hindu religion, in particular of Vaiṣnavism”.

\hfill Bakker (1986:66)
\end{myquote}

Pollock finishes what Bakker started. His “evidence”\index{evidence!inscriptional}\index{evidence!inscriptional!Ratanpur}\index{evidence!inscriptional!Ramtek}\index{evidence!inscriptional!Vijayanagar} is twelfth-century inscriptional references to 

\begin{itemize}
\item[(1)] two temples dedicated to Rāma in the kingdom of the Kalacuris of Ratanpur (in the Chhattisgarh area of Madhya Pradesh), 

\item[(2)] the Rāma complex at Ramtek, and 

\item[(3)] the Rāmacandra shrine at Hampi in the kingdom of Vijayanagara\index{Vijayanagara}  (Pollock 1993:266-9). 
\end{itemize}

Pollock talks about the construction or reinvigoration of “several major cultic centers” devoted to Rāma\index{Rama@Rāma} after the twelfth-thirteenth centuries, but he does not cite them. He is not sure himself what the situation in the Gāhaḍavāla kingdom of Uttar Pradesh was like — he refers to Bakker’s\index{Bakker, Hans} work on Ayodhya\index{Ayodhya} to point out that the “Gāhaḍavāla dynasty begins to develop Ayodhya\index{Ayodhya} as a major Vaiṣṇava center by way of a substantial temple building program” (Pollock 1993:266). He does not cite any inscriptional evidence\index{evidence!inscriptional} but nevertheless states with confirmation that a Rāma temple was constructed at Svargadvara ghat, probably by Jayacandra. 

Pollock’s substantiation for his grand claims of significant growth of building Rāma temples\index{Rama@Rāma!temples of} all over India from the Gupta period onward is decidedly thin. On the other hand, the available documentation shows that temples of Śiva\index{Siva@Śiva} and Viṣṇu\index{Visnu@Viṣṇu} were built more than other deities in the period under examination.\index{misinterpretation!techniques of!cherry-picking data} We see that although the Rāmacandra temple was located in the nucleus of the royal complex at Vijayanagara, the official epigraphic records of the Vijayanagara\index{Vijayanagara}\index{Vijayanagara kings} kings mention often their {\sl rāṣṭra-devatā} viz., Virūpākṣa.\index{Virupaksa@Virūpākṣa} Anila Verghese\index{Verghese, Anila} notes, “Pampā Virūpākṣa has undoubtedly been the principal deity at the site from before the founding of the empire onwards, and he was adopted as the guardian deity of the Vijayanagara state” (Verghese 1995:132). Another scholar, G. Michell,\index{Michell, G.} writes that the nucleus of Vijayanagara’s early sacred center at Hampi consisted of Śaivite shrines (Michell 1995:31-32). A sample of epigraphical record of the time is given below:

\begin{myquote}
“...  Oh wonder! Though (like Kṛṣṇa)\index{Krsna@Kṛṣṇa} he (King Īśvara)\index{Isvara(Siva)@Īśvara(Śiva)} was the son of the glorious Devakī, though (like Viṣṇu) he had lotus eyes, though he acquired tribute ({\sl bali}) by his valor which was able to subdue the three worlds, (just as Viṣṇu in his Vāmanāvatāra\index{Vamana@Vāmana} acquired the three worlds from Bali\index{Bali} by his three steps), and though he bore (the auspicious marks of) the conch and the discus in his hand— he became still more famous by the name of Īśvara, as he obtained prosperity ({\sl bhūti}), universal worship, and the daughter of a king, (just as the god Īśvara wears ashes ({\sl bhūti}) is universally worshipped and is the husband of the daughter of the mountain).\hfill Hultzsch (1892:367)
\end{myquote}

Compare this with the landscape of sacred centers in late thirteenth and the early fourteenth centuries Andhra. Keilhorn\index{Keilhorn} notes, 

\begin{myquote}
“From the above abstract it will appear that most of the donations recorded here were made in the favor of the god Viṣṇu,\index{Visnu@Viṣṇu} under the names of Viṣṇu-bhaṭṭāraka, Nārāyaṇa-bhaṭṭāraka, Vāmanasvāmideva and Chakravāmideva. The same divinity I understand to be denoted by the name Tribhuvanasvāmideva. But besides him, we find among the donees also Umāmaheśvara, clearly a form of the god Śiva,\index{Siva@Śiva} and Bhailasvāmideva, a name in a fragmentary inscription\index{inscriptions!Bhilda} from Bhilda, mentioned by Dr.~Hall in the {\sl Journ. Beng}. As. Soc, vol XXXI, p.112, is distinctly given as a designation of Ravi, the Sun’.”
\hfill Hultzsch  (1892:168)
\end{myquote}

Both these records were inscribed at the height of {\sl Turuṣka}\index{Turuska@\textit{Turuṣka}} penetration into the Deccan. Yet, we see no reference to Rāma\index{Rama@Rāma} or such “political imagination” in them. A cursory glance at the list of “royal cults” of the period (epigraphy and actual sites) shows, in fact, that a very striking variety of deities were worshipped in different regions\endnote{See Brajdulal Chattopadhyaya\index{Chattopadhyaya, Brajdulal} (1998).}: Bṛhadīśvara\index{Brhadisvara@Bṛhadīśvara} and Gaṅgaikoṇḍacolapuram\index{Gangaikondacolapuram@Gaṅgaikoṇḍacolapuram} were the royal “cult-centers” of Rājarāja Coḷa and Rājendra Coḷa of Tamil Nadu; the coins and epigraphs of the Kadambas consistently refer to Śrīsaptakoṭīśvara as their deity; the Śilāhāra-s always invoked Mahālakṣmī in their inscriptions;\index{inscriptions!Silahara@Śilāhāra} the Cālukya-s and Vāghela-s of Gujarat considered Somanātha\index{Somanatha(Siva)@Somanātha(Śiva)} as their most important deity. In Rajasthan, the site of Ekaliṅga\index{Ekalinga@Ekaliṅga} emerged as a major centre of royal cult in the kingdom of the Guhila-s of Mewar; in Orissa, the Bhañja-s worshipped Śiva and Stambheśvarī, and Jagannātha.

Of course these are but a few examples. Yet, it shows that the formation of royal or regional cults cannot be pinned down to a particular time-period nor can it be categorized as “reactionary”.  What Bakker and Pollock do is choose certain insignificant incidents and piece them together to form a convenient narrative. C. Talbot\index{Talbot, Cynthia} also notes, “Inscriptions from Andhra provide little support for Pollock’s thesis, as far as the {\sl Rāmāyaṇa} itself is concerned, for there are few direct references to the epic story.” (Talbot 1995:696) 

In Chattopadhyaya’s words, “the evidence adduced by Bakker and Pollock does not appear extraordinarily significant and can be explained in ways other than what have been advocated by them” (Chattopadhyaya\index{Chattopadhyaya, Brajdulal} 1998:106).

\section{Textual Evidence (Epigraphical)}\label{sec3.2}
\index{epigraphical evidence}
\index{evidence!textual}
\index{evidence!epigraphical}

Pollock’s second category of evidence is that of inscriptions. In his words, 

\begin{myquote}
“If the architectural remains associated with Rāma\index{Rama@Rāma} have yet to be systematically worked through and synthetically analyzed, this is even more the case with the inscriptional materials (beyond those associated with a temple cult) that refer to or invoke the god-king or in one way or another process {\sl Rāmāyaṇa} themes (Sircar 1980 and Diskelkar 1960 are the sole, unhelpful guides). So here, too, my findings have to be regarded as provisional, but again I would be surprised if further work would require fundamental revision of my conclusion: The {\sl Rāmāyaṇa} supplies serious material to the political imagination of pre-modern India as coded in the inscriptional record only from the later medieval period on; references in the first millennium are remarkably few but gain in frequency and complexity especially after the twelfth century.”
\hfill Pollock (1993:270)
\end{myquote}

Before the twelfth century, Pollock says, references to Rāma\index{Rama@Rāma} in inscriptions\index{inscriptions} are “static, formulaic allusions”. In this period, he writes, “Rāma and {\sl Rāmāyaṇa} mythemes function as peripheral rhetorical embellishments, inflecting and texturing a given discourse but not constituting it” (Pollock 1993:272). Very different are, according to him, the materials of the succeeding period. After the twelfth century, he says, “the political world comes to be read through - identified with, cognized by-the narrative provided by the epic tale.” (Pollock 1993:272). He furnishes two evidences in support of his stand: 
\begin{itemize}
\itemsep=0pt
\item[(a)] the Dabhoi\index{inscriptions!Dabhoi} stone inscription of the Vāghelā family of Gujarat\break (AD ({\sl sic}) 1253) 
\item[(b)] Hansi\index{inscriptions!Hansi} inscription of 1167 AD ({\sl sic}), which can be regarded as a {\sl praśasti}\index{prasasti@\textsl{praśasti}} of Cāhamāna\index{Cahamanas@Cāhamānas} Pṛthvīrāja II. 
\end{itemize}
Let us examine each in its own context. 

\smallskip
\noindent
{\bf (a) Dabhoi Inscription:}\index{Dabhoi inscription} The Dabhoi inscription is a {\sl praśasti} (panegyric) composed by Someśvaradeva, a {\sl purohita} of the Rāṇaka-s of Ḍholkā (author of {\sl Kīrti-kaumudī}),\index{Kirtikaumudi@\textsl{Kīrti-kaumudī}} and is dated 14 May 1253 ({\sl vikrama saṁvat 1811 jyeṣṭha śudi 15 vudhadine}). Vīsaladeva, who was then the ruler of Gujarat, had, at the time, ordered the restoration of a Śiva\index{Siva@Śiva} temple at Dabhoi, and in his honor was this {\sl praśasti}\index{prasasti@\textsl{praśasti}} composed/inscribed. It contains a total of 116 verses. Let us look at its details verse-by-verse (paraphrased in some places) ({\sl Trans}. Bühler; cited in Hultzsch 1892:20-23):

\begin{quote}
{\bf (Verse no.) 1-3:} A {\sl maṅgala} to Śiva-Vaidyanātha,\index{Vaidyanatha(Siva)@Vaidyanātha(Śiva)} and a fragment of it reads, “May glorious Vaidyanātha himself with his eight bodies grant their desires to the creatures.”


{\bf 4:} Description of Vīsaladeva’s\index{Visaladeva@Vīsaladeva} ancestors— “the line of the progeny of that (man), the good deeds of which (line)… (cannot be described—) even by eloquent men.”


{\bf 5:} <Lost>

\smallskip
{\bf 6:} “Won over by the eminent qualities of this conqueror of his foes, the guardian goddess (Śrī) of the Gūrjara\index{Gurjara@Gūrjara} princes became of her own choice his bride, just as (the goddess Śrī became the bride) of (Vishṇu)\index{Vishnu@Vishṇu} ({\sl sic}), the foe of Bāṇa (at the churning of the ocean).”\endnote{Bühler notes that these verses are identical to {\sl Kīrti-kaumudī}\index{Kirtikaumudi@\textsl{Kīrti-kaumudī}} II.2. Here, the lines refer to Mūlarāja, the founder of the Chalukya dynasty of Aṇhilvād, and hence, the verse in the {\sl praśasti}\index{prasasti@\textsl{praśasti}} must also refer to the same person (Hultzsch 1892:21).} 

\smallskip
{\bf 7:} [Continued description of Mūlarāja] – (the wives of his enemies tremble or fly into the jungles), “when he, an embodied stream of the sentiment of heroism, stands on the back of… with the intention of fighting.”

\smallskip
{\bf 8:} <Lost>

\smallskip
{\bf 9:} [Description of Vāghelās, beginning with Arṇorāja]\index{Arnoraja@Arṇorāja} — “By whom, even the son of Dhavala, an imitator of Kṛishṇa\index{Krishna@Kṛishṇa} ({\sl sic}), this realm of famous Gūrjara land was made free from thorns.”\endnote{Bühler notes that this is in line with {\sl Kīrti-kaumudī} 2.63, “By that good warrior who imitated Kṛṣṇa,\index{Krsna@Kṛṣṇa} even by the son of Dhavala, was begun the clearance of the kingdom from thorns.” (Hultzsch 1892:21).}

\smallskip
{\bf 10:} “[Arṇorāja] slew on the battle field Raṇasiṁha who resembled Rāvaṇa”.\index{Ravana@Rāvaṇa}

\smallskip
{\bf 11:} “Now when his son valiant Lavaṇaprasāda\index{Lavanaprasada@Lavaṇaprasāda} (was able to sustain) the load of Gūrjara land, he (Arṇorāja) offered, his heart being averse to the world, a battle-sacrifice at which he which he gave his life as fee.”

\smallskip
{\bf 12:} “[Owing to some deeds of Lavaṇaprasāda] the kingdom of the Gūrjaras\index{Gurjara@Gūrjara} was even greater than that of Rāma”.\index{Rama@Rāma} 

\smallskip
{\bf 13:} Mention of a fight near Vardhamāna (the modern town of Vaḍhvān in North-eastern Kāṭhiāvāḍ) with some unnamed powerful foes.

\smallskip
{\bf 14:} “By whom the chief of Naḍūla was deeply wounded with his mighty sword; owing to this (severe stroke), yon kings quake even today, just as the mountains at a thunder-clap”\endnote{Bühler\index{Buhler, George@Bühler, George} notes that this is identical with {\sl Kīrti-kaumudī}, 2. 69 (Hultzsch 1892:22).}

\smallskip
{\bf 15:} “How many godlike kings are there not on earth? But even all of them became troubled by the mere mention of the king of the Turushkas. When that ({\sl Turushka king}), excessively angry, approached in order to fight, ({\sl it was Lavaṇaprasāda}) who placed only…” ({\sl italics not ours})

\smallskip
{\bf 16:} “By whom ({\sl Lavaṇaprasāda}),\index{Lavanaprasada@Lavaṇaprasāda} the king of the Turushkas... who has spattered the earth with the blood flowing from the cut-off heads of numerous kings— when he came in front, with dry lips, full of doubt— was conquered at\break Stambha with his arm ({\sl strong}) like a post ({\sl stambha}) and terrible through the sword”. ({\sl italics not ours})

\smallskip
{\bf 17:} ... “if he (Lavaṇaprasāda)\index{Lavanaprasada@Lavaṇaprasāda} is a mortal, how is it that he conquered the lord of the Mlechchhas ({\sl sic})?”
\end{quote}

\smallskip
Note here that in Lavaṇaprasāda’s time, there were three Muslim\index{Muslim!attacks}\index{Muslim} attacks on Gujarat: 
\begin{itemize}
\itemsep=0pt
\item[(a)] the (unsuccessful) expedition of Shahabuddin Ghori\index{Shahabuddin Ghori}  (1178 C.E) 
\item[(b)] the first expedition of Qutbuddin\index{Qutbuddin} in 1194 C.E (during which Aṇhilvāḍ was sacked), and 
\item[(c)] the second expedition of Qutbuddin in 1196 C.E, which was unsuccessful but led to the temporary conquest of Gujarat (and the temporary occupation of Aṇhilvāḍ). 
\end{itemize}
G. Bühler\index{Buhler, George@Bühler, George} notes that this inscription perhaps does not refer to any of the three battles. He writes, “… the most probable conjecture seems to me that it [the battle that the Dabhoi\index{inscriptions!Dabhoi} inscription refers to] happened after the occupation of Aṇhilvāḍ in 1196. Sometime later the Muhammadans did suffer a defeat in Gujarat and the province shook their yoke off. The details of these events are not given either by the Muhammadan or the Hindu authors; but our passage probably contains an allusion to them, and it may be that Lavaṇaprasāda was the liberator of his country.” (Hultzsch 1892:23). 

\begin{quote}
{\bf 18:} [Lavaṇaprasāda], ``a repository of medicine like valor, cured (his country) when the crowd of the princes of Dhārā, of the Dekhan and of Maru, who resembled diseases (attacked it)”.

\smallskip
{\bf  19:} “He (Lavaṇaprasāda)\index{Lavanaprasada@Lavaṇaprasāda} who raises his race, seems to me greater than Yudhishṭhira,\index{Yudhisthira@Yudhiṣṭhira} whose relatives were all destroyed, though their power to remove a Salya is equal”. 

\smallskip
{\bf 20:} <Lost>

\smallskip
{\bf 21:} [referring to the erection of a temple in Kumāra near Vaḍhvāṇ], “Who (Lavaṇaprasāda)\index{Lavanaprasada@Lavaṇaprasāda} caused to be erected in the neighborhood of Vardhamāna, a ({\sl temple of}) Kumāra rivaling the ocean ({\sl in the possession of treasures}) and surpassing the moon ({\sl in brilliancy})”. ({\sl italics not ours})

\smallskip
{\bf 22-24:} Not clear

\smallskip
{\bf 25:} [Praise of Vīradhavala] “From him sprang a son, who was the image of Daśaratha\index{Dasaratha@Daśaratha} and Kakutstha, who swallowed like a mouthful the armies of hostile kings— Vīradhavala. When the flood of his fame spread, the cleverness of faithless women, whose minds are distressed with the longing after enjoyments— in the art of approaching ({\sl their lovers}) was destroyed”. ({\sl italics not ours})
\end{quote}

\smallskip
Bühler\index{Buhler, George@Bühler, George} notes that in “the remaining verses referring to Vīradhavala, 26-51, little more than single letters or words are legible, except in verse 45, where an unsuccessful combined attack of the lord of Dhārā and of the ruler of the Dekhan is mentioned. The portion of the {\sl praśasti}\index{prasasti@\textsl{praśasti}} which celebrates Vīsaladeva’s\index{Visaladeva@Vīsaladeva} great deeds and virtues, verses 52 – 108, is likewise in a very bad condition…” (Hultzsch 1892:23)

So the Dabhoi inscription\index{inscriptions!Dabhoi} refers to Gūrjara kingdom\index{Gurjara@Gūrjara} ruled over by Lavaṇaprasāda\index{Lavanaprasada@Lavaṇaprasāda} as “greater than {\sl Rāma-rājya}” (verse 12). It also mentions the defeat of the {\sl turuṣka} king by Lavaṇaprasāda who, the inscription asserts, could not be a mere mortal (verse 17). However, the “meaning-conjuncture” — an expression which Pollock uses to point to the identity of the king as victor over the {\sl turuṣka}-s\index{Turuska@\textit{Turuṣka}} with Rāma,\index{Rama@Rāma} the slayer of Rāvaṇa\index{Ravana@Rāvaṇa} — does not take place in this record.  

Interestingly, the record refers to Arṇorāja,\index{Arnoraja@Arṇorāja} founder of the Vāghela line, as imitating the feats of Kṛṣṇa,\index{Krsna@Kṛṣṇa} and his adversary Raṇasiṁha (not a {\sl turuṣka}), slain on battlefield, is called Rāvaṇa (verse 10). Lavaṇaprasāda,\index{Lavanaprasada@Lavaṇaprasāda} victor over the {\sl turuṣka} king, is mentioned to be more famous than Yudhiṣṭhira\index{Yudhisthira@Yudhiṣṭhira} (verse 19). His son Vīradhavala, was “the image of Daśaratha\index{Dasaratha@Daśaratha} and Kākutstha” (verse 25). We see that the composer of the record drew upon a repertoire of available motifs — and the “political mytheme” of Rāma v/s Rāvaṇa was not one of them. 

How this substantiates Pollock’s claim, only Pollock can explain. 

\smallskip
\noindent
{\bf (b) Hansi Inscription:}\index{Hansi Inscription}\index{inscriptions!Hansi} The Hansi record (1167 CE) constitutes 22 lines of writings — partly prose, partly verse. It is also a {\sl praśasti},\index{prasasti@\textit{praśasti}} and its aim was to describe Kilhaṇa’s conquest of Panchapura. Kilhaṇa\index{Kilhana@Kilhaṇa} was a maternal uncle and feudatory of the Cāhāmana\index{Cahamanas@Cāhāmanas} king, Pṛthvirāja.\index{Prthviraja@Pṛthvirāja} Kilhaṇa was put in charge of the fort of Hansi to check the progress of Hammīra, the Muslim\index{Muslim!emperor Hammira@emperor Hammīra} emperor. Let us examine its details— it reads as follows [{\sl Trans.} Bhandarkar]\endnote{(See Bhandarkar 1912:17-19). Relevant portions of the translation has been summarized here and proper diacritics have been introduced. Where the words are within quotes, the original spellings are retained and author's additions are presented in [~].}:

\begin{quote}
{\bf (Verse) 1:} Obeisance to [an unspecified] Goddess, and invokes the blessings of the god, Murāri.

\smallskip
{\bf 2:} Informs us that there was a king of the Cāhamāna\index{Cahamanas@Cāhamānas} lineage called Pṛthvīrāja,\index{Prthviraja@Pṛthvīrāja} and his maternal uncle called Kilhaṇa.\index{Kilhana@Kilhaṇa} 

\smallskip
{\bf 3:} Informs that Hammīra had become the cause of anxiety to the world, and Pṛthvirāja put Kilhaṇa in charge of the fort of Hansi.\index{inscriptions!Hansi} 

\smallskip
{\bf 4:} Kilhaṇa belonged to the race of Gūhilauta. 

\smallskip
{\bf 5:} Kilhaṇa erected a {\sl pratolī} (gateway) which with its flags set Hammīra as it were at defiance. 

\smallskip
{\bf 6:} Near the gateway were constructed two {\sl koṣṭhaka}-s (granaries).

\smallskip
{\bf Lines 9-10} (prose) speak of a letter sent to Kilhaṇa\index{Kilhana@Kilhaṇa} by Vibhīṣaṇa.\index{Vibhisana@Vibhīṣaṇa}

\smallskip
{\bf 7:} [The letter begins] ``the lord of demons (Vibhīshaṇa)\index{Vibhishana@Vibhīshaṇa} who has obtained a boon from Rāma, the crest jewel of the lineage of Raghu, respectfully speaks thus to Kilhaṇa staying in the fort (gaḍha) of Āsī[= Hansi]”.  

\smallskip
{\bf 8:} “In the work of the building the bridge, we both assisted the leaders of the monkeys and bears. And you yourself (Kilhaṇa) have written saying that the lord of Paṁchapura, a string of pearls and this city, had been given to you by the Omnipresent (Rāma).”

\smallskip
{\bf 9:} In this verse, Pṛthvīrāja is compared to Rāma\index{Rama@Rāma} and Kilhaṇa to Hanumān. 

\smallskip
{\bf 10:} Vibhīṣaṇa bestows conventional praise on Kilhaṇa. 

\smallskip
{\bf 11:} Refers to his having burnt Paṁchapura, and captured but not killed its lord.

\smallskip
{\bf 12:} Eulogy of Kilhaṇa.

\smallskip
{\bf 13:} Vibhīṣaṇa\index{Vibhisana@Vibhīṣaṇa} requests Kilhaṇa to accept the string of pearls or even Laṅkā\index{Lanka@Laṅkā} but promise safety to him.

\smallskip
{\bf Line 19-20} is prose – it informs us that this string of pearls was presented by the ocean to Rāmabhadra when he was intent upon constructing the bridge.

{\bf 14-15:} One Valha who belonged to the Ḍoḍa race and who was a subordinate of Kilhaṇa\index{Kilhana@Kilhaṇa} and that his son was Lakṣmaṇa\index{Laksmana@Lakṣmaṇa} under whose auspices the {\sl praśasti}\index{prasasti@\textsl{praśasti}} was composed.
\end{quote}

This is followed by the date: Thursday, 7th of the bright half of Māgha of the Vikrama year 1224 (1170 C.E).

We see that this {\sl praśasti}\index{prasasti@\textsl{praśasti}} draws an identification of the Cāhamāna\index{Cahamanas@Cāhamānas} king Pṛthvīrāja\index{Prthviraja@Pṛthvīrāja} with Rāma,\index{Rama@Rāma} and of Kilhaṇa,\index{Kilhana@Kilhaṇa} Pṛthvīrāja’s maternal uncle, with Hanumān. However, if the record is located within the context of other contemporary inscriptions, it yields a different picture: Usually, in most Indian texts, heroes are identified with Viṣṇu,\index{Visnu@Viṣṇu} or Mahāvarāha (who lifts the earth submerged in the ocean of {\sl turuṣka}\index{Turuska@\textit{Turuṣka}} rule), or as Agastya (who is the swallower of the ocean) (Chattopadyaya 1998:110). In the medieval epigraphs, therefore, “whether in the context of Yavana\index{Yavana} raids or outside them, the king— as a hero and a ruler— has many identities: Indra,\index{Indra} Viṣṇu, Trivikrama, Mahāvarāha, Śiva,\index{Siva@Śiva} Pṛthu,\index{Prthuvainya@Pṛthu Vainya} Agastya, Kāma, Revanta, Yudhiṣṭhira,\index{Yudhisthira@Yudhiṣṭhira} Bhīma,\index{Bhima@Bhīma} Rāma, and so on” (Chattopadyaya 1998:110). This point is better illustrated by the juxtaposition of extracts from other inscriptions of the period under examination. 

\newpage

Consider the evidence\index{evidence!inscriptional}\index{evidence!epigraphical} furnished by the following five inscriptions: 
\begin{itemize}
\itemsep=0pt
\item[{\bf 1.}] Ajaygadh rock inscription\index{Ajaygadh rock inscription}\index{inscriptions!Ajaygadh} of Candella Vīravarman (1261 C.E) 

“... After him, Pṛthvīvarman was king, similar to Pṛthu; and then Madana ruled over the kingdom, a god of love to the opponents. Then came the illustrious king Paramardin, who, as a leader, even in his youth, struck down opposing heroes… then the prince Trailokyavarman ruled the kingdom, a very creator in providing strong places. Like Viṣṇu he was, in lifting up the earth, immerged in the ocean formed by the streams of {\sl turuṣkas}\index{Turuska@\textit{Turuṣka}}. Victorious is his son Vīra, that ruler of the earth of spotless bravery who has delighted the damsels of heaven by sending them, as lovers, the hostile heroes whom he cut down on the field of battle. Victorious (and) to be worshipped by all men is he whom when he strikes down the wicked (and) disperses crowds of opponents, people gaze at— wondering whether he be Viṣnu riding on Garuḍa or Śiva\index{Siva@Śiva} about on his bull.” 
\hfill (Hultzsch 1892:329)

\item[{\bf 2.}] Bitragunta Grant\index{Bitragunta grant} of Saṅgama\index{inscriptions!Bitragunta grant - Sangama@Bitragunta grant - Saṅgama} II (1278 C.E): 

“... from him were produced five heroic sons, as, formerly, the (five) celestial trees from the milk ocean— first, king Harihara; the, the ruler of the earth, Kampa; then, the protector of the earth, Bukka; (and) afterwards, Mārapa and Muddapa. Of these, king Harihara— by whom the Sultan (Suratrāṇa), who resembled Sutrāman (Indra)\index{Indra} was defeated— ruled the earth for a long time. His younger brother, king Kampana, whose name became true to its meaning as he made the enemies tremble, ruled the earth for a long time…. Into the flames of his [Harihara II’s] valor the {\sl yavana,\index{Yavana} turushka} and Andhra hostile kings fell like moths.. (Heras 1929:120) 

\item[{\bf 3.}] Satyamangalam Plates\index{Satyamangalam plates}\index{inscriptions!Satyamangalam plates} of Devaraya II (1346 C.E): 

“Through the wind (which was produced) by the flapping of the ears of his elephants on the field of battle, the {\sl tulushka} horsemen experienced the fate of cotton (were blown away)…” (Hultzsch 1892:40)

\item[{\bf 4.}] Machchishahr Copper Plate\index{Machchishahr copper plate} Inscription\index{inscriptions!Machchishahr copper plate} (1197 C.E): 

“To him was born a king called Vijayacandra who was capable ({\sl dakṣa}) of destroying ({\sl viccheda}) the allies ({\sl pakṣa}) of (enemy) kings ({\sl bhūbhṛt}); just as Indra is capable ({\sl dakṣa}) of cutting under the wings of the (fabulous flying) mountains and who (Indra), had washed off the heat of the terrestrial world with streams of water from the clouds in the shape of the eyes of the Hammīra women, when he was indulging in the sport of subduing the world (?).”
\hfill (Prasad 1990:67)

\item[{\bf 5.}] Batesvar Chandella inscription\index{Batesvar Chandella inscription}\index{inscriptions!Batesvar Chandella} of Paramardideva\index{Paramardideva} (1195 C.E): 

“Among them appeared the lord of the earth Madanavarman, who with his flashing sword scattered (his) adversaries (and) whose vigor became known by his onslaught on hostile kings, elated with pride; (resembling) the great Indra who cut off the wings of the mountains with his thunderbolt (and) whose might became famous by his killing (the demon) Vala”.

\hfill (Hultzsch 1892:212)
\end{itemize}

The few samples suffice to show that Pollock has arbitrarily isolated Rāma from the variegated world of a host of divinities and legendary kings, just to suit his theory/purpose and has wilfully and skillfully avoided any reference to others.

\section{Textual Evidence (Literary)}\label{sec3.3}
\index{literary evidence}
\index{evidence!textual}
\index{evidence!literary}

Pollock’s last category of evidence is what he calls “historiographical”\index{evidence!textual}\index{evidence!historiographical} or literary evidence (Pollock 1993:273). Pollock takes just two literary examples to strengthen his argument: 
\begin{itemize}
\itemsep=1pt
\item[(a)] Hemacandra’s\index{Hemacandra} {\sl Dvyāśrayakāvya}\index{Dvyasrayakavya@\textsl{Dvyāśrayakāvya}} and
\item[(b)] Jayāṅka’s\index{Jayanka@Jayāṅka} {\sl Pṛthvīrāja-vijaya.}\index{Prthvirajavijaya@\textsl{Pṛthvīrāja-vijaya}}
\end{itemize}

Admittedly, the two texts consider their respective kings, Jayasiṁha Siddharāja\index{Jayasimha Siddhiraja@Jayasiṁha Siddharāja} and Pṛthvīrāja\index{Prthviraja@Pṛthvīrāja} III Cāhamāna\index{Cahamanas@Cāhamānas} as incarnations of Rāma,\index{Rama@Rāma} and the latter text, in particular, dwells at length on the depredations by the {\sl turuṣka}-s\index{Turuska@\textit{Turuṣka}} in the region of Ajayameru, Rajasthan. Yet, to say that this lends incontestable support to Pollock’s conjecture of “mythopolitical equivalence” is a stretch of imagination. In the period under examination, there were innumerous forms of comparisons, and no one form dominated the other — for example, {\sl kāvya}-s like Gaṅgādevī’s\index{Gangadevi@Gaṅgādevī} {\sl Madhurā-vijaya},\index{Madhuravijaya@\textsl{Madhurāvijaya}} Nayacandra’s {\sl Hammīra Mahākāvya},\index{Hammira Mahakavya@\textsl{Hammīra Mahākāvya}} etc are literary works that talk of {\sl turuṣka}\index{Turuska@\textit{Turuṣka}} invasions without reference to Rāma,\index{Rama@Rāma} and texts like Sandhyākaranandin’s\index{Sandhyakaranandin@Sandhyākaranandin} {\sl Rāma-carita}\index{Ramacaritam@\textsl{Rāma-caritam}} use the “{\sl Rāmāyaṇa}-mytheme” in the context of a local warfare that is not {\sl turuṣka}\index{Turuska@\textit{Turuṣka}}.

Let us look at Gaṅgādevi’s {\sl Madhurā-vijaya} to understand how other images (other than the “{\sl Rāmāyaṇa} imaginary”) were used to depict the Muslims.\index{Muslim!depiction in kavya@depiction in \textsl{kāvya}} {\sl Madhurā-vijaya} is a {\sl mahā-kāvya} written in the second half of the fourteenth century in celebration of Gaṅgādevī’s husband’s victory over the {\sl turuṣka}-s of Madurai. Its eighth {\sl sarga reads} thus: 

\vskip .2cm

\begin{myquote} 
............................................... {{\sl vyāghra-purīti sā yathārtham}} || 1 ||\\
{\sl adhiraṅgam avāpta-yoganidraṁ harim udvejayatīti jāta-bhītiḥ} |\\
{\sl patitaṁ muhur iṣṭakā-nikāyaṁ phaṇa-cakreṇa nivārayaty ahīndraḥ} || 2 ||\\
{\sl ….nughūrṇad ūrṇanābhaṁ vana-vetaṇḍa-vimardinīm avasthām} |\\
{\sl viratāny aparicchada-prapañco bhajate hanta ! gaja-pramāthi-nāthaḥ} || 3 ||\\ 
{\sl ghuṇa-jagdha-kavāṭa-sampuṭāni sphuṭa-dūrvāṅkura-sandhi-maṇḍapāni} |\\
{\sl ślatha-garbha-gṛhāṇi vīkṣya dūye bhṛśam anyāny api devatā-kulāni} || 4 ||\\
{\sl mukharāṇi purā mṛdaṅga-ghoṣair abhito deva-kulāni yāny abhūvan} |\\
{\sl tumulāni bhavanti pheravāṇāṁ ninadais tāni bhayaṅkarair idānīm} || 5 ||\\
{\sl satatādhvara-dhūma-saurabhaiḥ prāṅ-nigamodghoṣaṇavadbhir agrahāraiḥ} |\\
{\sl adhunājani visra-māṁsa-gandhair adhika-kṣība-tuluṣka-siṁha-nādaiḥ} || 6 ||\\
\hspace{5cm} ...\qquad
\end{myquote}

The verses have been translated as, 

\begin{myquote}
“O King! That city, which was called “Madhurāpurī” for its sweet beauty, has now become the city of wild animals, making true its older name “Vyāghra-purī”, the city of tigers, for humans dwell there no longer”.~(1) 

“The famed temple of Śrī Raṅgapaṭṭaṇa has fallen to decay, and its structure being reduced to rubble. So much that Viṣṇu\index{Visnu@Viṣṇu} who famously slept there in his deep {\sl yoga-nidrā}, has now literal protection only of the hood of Ādiśeṣa who has to be ever cautious from the falling bricks of the debris”. (2) 

“How do I describe the condition of the abode of the slayer of Gajāsura! In the bygone days, after slaying Gajāsura, Lord Śiva\index{Siva@Śiva} had taken its skin for his garment. And now being stripped he has gone back to being {\sl digambara}. Wild elephants have now made the Śiva-liṅga their plaything, and all but spider-webs are the decorations of his abode.” (3) 

“[when such is the state of those famous temples] how would other {\sl devasthāna}-s be any better! Moths have eaten away the once-beautiful wooden structures, the maṇḍapa-s have developed cracks in which now grass grows, and {\sl garbha-gṛha}-s of many others are dilapidated and crumbling. My Lord, my heart is crying as I describe to you the situation of our beloved {\sl devatā-kula}.” (4) 

“Those {\sl deva-mandira}-s which used to resound with the joyous and pious beats of {\sl mṛdaṅga}, today only the echo of fearful howls of jackals can be heard there.” (5) 

That Gaṅgā of South, mighty Kāverī, which used to earlier flow in proper channels curbed with dams created by our noble rulers of past – she now flows like a vagabond without discipline; like her new lords, these {\sl turuṣka}-s,\index{Turuska@\textit{Turuṣka}} her dams being breached beyond repair. (6)…” 

\hfill {\sl Madhurā-vijaya}\index{Madhuravijaya@\textsl{Madhurāvijaya}} (8.1-6) [{\sl Trans.} S.K  Tiwari]
\end{myquote}

So it continues, and further the situation is shown to escalate to cataclysmic proportions, and Gaṅgādevī’s\index{Gangadevi@Gaṅgādevī} husband takes the form of Mahā-varāha to restore peace. 

Nowhere, we can see, are motifs from the {\sl Rāmāyaṇa} used in the descriptions here.  

On the other hand, let us take the example of Sandhyākaranandin's\index{Sandhyakaranandin@Sandhyākaranandin} {\sl Rāma-carita}\index{Ramacaritam@\textsl{Rāma-caritam}} – this is a {\sl śleṣa kāvya} (paronomastic) that produces two levels of meanings at the same time: on one level, the {\sl kāvya} narrates Sītā’s\index{Sita@Sītā} deliverance after the slaying of Rāvaṇa\index{Ravana@Rāvaṇa} by Rāma.\index{Rama@Rāma}  On the second level, it narrates the account of the Pāla ruler (Sandhyākara’s patron), Rāmapāla, and his slaying of Bhīma,\index{Bhima (Kaivarta king)@Bhīma (Kaivarta king)} the Kaivarta king, (who usurped the territory of Varendrī for a short time). We can see an application of Pollock’s “{\sl Rāmāyaṇa} mytheme”, if that, in a context that does not involve Muslims. 

Moreover, {\sl Pṛthvīrāja-vijaya}’s\index{Prthvirajavijaya@\textsl{Pṛthvīrāja-vijaya}} “demonization” of the Ghurids cannot be interpreted as a reflection of a true Hindu “abhorrence” of Muslims.\index{Muslim!depiction in kavya@depiction in \textsl{kāvya}} Jayāṅka\index{Jayanka@Jayāṅka} was simply engaging in a mode of dehumanizing the Ghaznavid and Ghurid rivals of the Cāhamānas\index{Cahamanas@Cāhamānas} that was frequently employed to designate foes in literature — Muslim or otherwise. 

Note, for example, the following four references in Śyāmilaka’s\index{Syamilaka@Śyāmilaka} {\sl Pāda-tāḍitaka},\index{Padataditaka@\textsl{Pāda-tāḍitaka}} a {\sl bhāṇa} of the Gupta period: 
\begin{itemize}
\item[(1)] “Aha, this is the character of a Diṇḍi. The Diṇḍi-s are not very different from the monkeys, what does he then find lovable in the Diṇḍi-s?”

\item[(2)] “Here is a man with the face of a he-goat, whose loins are covered with a piece of cloth, and whose shoulders are full of thick hairs. (Besides) he comes biting a radish. If he is not a Dāśeraka then he must be a devil”. 

\item[(3)] “What merit has he discovered in this (slave) maid Barbarikā...? Moreover, this Barbarī, the veritable goddess of darkness with whiteness in the teeth and eyes only, appears like night with a very strip of the crescent moon. But this is not strange. For the men of Saurāṣṭra and the monkeys are all of the same class”. 

\item[(4)] “ ... who will listen to the Yavana\index{Yavana} courtesans’ words which are like the chattering of a monkey, full of shrill sounds and of indistinguishable consonants and which are interspersed with the (occasional) display of the forefingers?” 

\hfill (Chattopadhyaya 1998:81).\index{Chattopadhyaya, Brajdulal} 
\end{itemize}

Similarly, in the same Hemacandra’s\index{Hemacandra} {\sl Dvayāśraya mahākāvya}\index{Dvyasrayakavya@\textsl{Dvayāśrayakāvya}} that Pollock cites, the character of Grāharipu, ruler of Saurāṣṭra-deśa, is described as a cruel tyrant, anti-religious, killer of pilgrims, and he is one who causes calamities, plunders people and destroys forts and important places. (Chattopadhyaya\index{Chattopadhyaya, Brajdulal} 1998:81). But Pollock admits that, “True the Cālukyas could imagine the Colas as {\sl rākṣasa}-s, or the Colas could thus position the Sinhalese” (Pollock 1993:283). It is, however, not enough to simply refer to counter-evidence; it must be shown how the same can be reconciled with the notion of “the utter dichotomization of the enemy”. 

Muslims,\index{Muslim!oscillating representation} a closer scrutiny will show, were not universally reviled. Representations of Muslims generally oscillated depending on the prevailing political conditions: in times of military conflict and radically fluctuating spheres of influence, the rhetoric was often negative in tone; whereas long established Muslim rulers were conceptually assimilated into the Sanskritic political imagination. 

Cynthia Talbot\index{Talbot, Cynthia} shows in her extensive work on “Hindu-Muslim identities in pre-colonial India” (1995) that anti-Muslim polemics reflected a defensive stance of the Hindus, but “from the early fifteenth through mid-sixteenth centuries, there was little dramatic change in the power balance, and tensions subsided momentarily… in this… phase, we witness no demonization of Muslim” (Talbot 1995:706). 

\newpage

Chattopadhyaya also notes, “… the reality… could be represented in two ways. In one representation, the destroyer of the Yavana\index{Yavana} (who is a destroyer of social order) is comparable to Viṣṇu;\index{Visnu@Viṣṇu} in another, Yavana, as a benign ruler, gives succor to Viṣṇu who, leaving the burden of preservation to the ruler, retires to peaceful sleep in the ocean of milk. It cannot be argued that chronologically, one representation replaces the other” (Chattopadhyaya\index{Chattopadhyaya, Brajdulal} 1998:60). 

We see, therefore, that Pollock’s method of isolating references relating to Rāma\index{Rama@Rāma} from their larger contexts has the effect of exaggerating their importance. One must not settle for such neat narratives, and must, instead, work towards a more sophisticated theorizing. 


