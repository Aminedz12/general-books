\chapter{A B C of Bird Watching}%%%1

Bird watching as an activity could be carried out for sheer 
pleasure of observing birds in their environment. After sometime 
it may become a serious study of exploring Nature. Whether it is 
pursued as a hobby or as a scientific interest, bird watching 
needs to be done in a systematic manner. 
\begin{enumerate}
\item \underline{Before you start} : Try to get familiar with the geography of 
the area. Within 100 km you may come across different types 
of environment. It may be countryside, woodlands, forests, 
wetlands, plantations, etc.,. The birds association differ 
in different environment. They can very well be observed in 
parks, gardens and campuses. 

\item \underline{When to start} : As a beginner it is better to start when 
birds could be seen in large numbers. October to April is 
the best period. A number of migratory birds descend down to 
feed in marshes and wetlands. 

The birds are active during early morning or before 
dusk. The timings for bird watching : 
\begin{center}
\begin{tabular}{ccc}
7 a.m. & to & 10 a.m. \\
& \& & \\
4 p.m. & to & 6 p.m.
\end{tabular}
\end{center}

\item \underline{What we need} :
\begin{itemize}
\item[i)] Dress : Dull Green or Brown coloured. The dress should 
blend with surrounding. Black, white or striking coloured 
dresses should be avoided. 

\item[ii)] Rubber soled shoes. 

\item[iii)] A cap 

\item[iv)] Binocular size 8 $\ast$ 30 (8 - magnification, 30 - diameter 
of object lens in mm.) 

\item[v)] A field notebook. 

\item[vi)] A handguide on birds. There are a few other things one 
need when carrying out studies like maps, camera, tape 
recorder, etc.,. The beginner should not worry about 
them. 
\end{itemize}

\item \underline{How to observe} :
\begin{itemize}
\item[i)] Make full use of eyes and ears. They are equally important 
tools. It is better to get familiar with bird calls. 

\item[ii)] Choose a place from which one may be able to scan a 
large area, e.g. sitting on a mound, under a large 
shady tree. 

\item[iii)] Always walk zigzag, circular or alongside the bird. Do 
not approach the bird directly. 

\item[iv)] Use binocular for details. Do not observe constantly to 
avoid eye strain. 
\end{itemize}

\item \underline{What to observe} : Keep careful notes of all that you observe. 
It is better to get familiar with birds from their 
pictures or observing them in zoo or museum from time to 
time. 
\begin{itemize}
\item[i)] Note down date, time, weather, locality. 

\item[ii)] Size  : Sparrow, bulbul, myna, crow or kite 
(bigger - smaller). 

\item[iii)] Shape : Slim, stout. 

\item[iv)] Bill - Straight, pointed, curved, slender, thick, 
hooked, conical. Also note colour of beak.

\item[v)] Legs : Size, long, short, toes. 

\item[vi)] Tail : Long, short, forked. Tip : round, pointed. Also 
observe movements of tail. 

\item[vii)] Crest over the head, colour, shape. 

\item[viii)] Colour of body - Bright, sober. 

Colour of upper part and lower part, wings. Conspicuous 
marks, look at breast, spotted, streaked or stripped 
Tail. Bands at tip. Any spots, Rump. Any patch. 
In waterbirds marking on wings are important. In some 
Male and Female differ in colour and appearance. During 
breeding some birds assume breeding plumage. 

\item[ix)] Voice : Musical, metallic, harsh, soft, trilling. 

\item[x)] Behavior : How birds feed and manner of eating. Behavior 
during breeding season. Flying habit. 

\item[xi)] Where the bird was found, on tree, ground, on post, in 
bush, grass. 

\item[xii)] Details about place visited. Marsh, Garden, Grove, 
Kere, Cultivated field, Fallow land, Plantation, Forest, 
Scrub. 
\end{itemize}

\item \underline{Code of Behavior} : 
\begin{itemize}
\item[i)] Permission to enter private lands must be taken. 

\item[ii)] While walking in cultivated lands, keep to paths. 

\item[iii)] Don't throw away litter. 

\item[iv)] Be careful during dry periods. A chance match stick 
thrown on grass may start a devastating fire. 

\item[v)] Don't disturb any natural thing. Don't touch nest, eggs, 
etc.,.

\item[vi)] Be familiar with the Wildlife Protection Act. 
\end{itemize}
\end{enumerate}

Useful Books for Bird watchers. 
\begin{enumerate}
\item The Book of Indian Birds - by Salim Ali. 

\item Collins Handguide to the Birds of the Indian Sub continent - by Martin Woodcock. 

\item About Indian Birds. 

Laeeq Futehally And Salim Ali 
\end{enumerate}
