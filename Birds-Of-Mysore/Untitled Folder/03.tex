\chapter{Bird Watching in C.F.T.R.I Campus}

With natural habitats disappearing in countrysides, gardens and parks 
are becoming refuge for many woodland species of birds. The gardens and 
the parks in the city are small islands of woods. C.F.T.R.I campus is a sort of 
mix of garden and park where birds have been found in fairly large numbers. 
In such an environment one can almost come in contact with nature 
everyday and experience the joy of self - discovery. 

The mansion, built in Baroque style and looking resplendent in yellow 
ochre, stands at the highest elevation of the landscape. The ledges of the 
mansion provide roosting and nesting places for the feral Blue Rock Pigeons 
which are found hovering around it with their deep gootr-goo, gootr-goo 
notes. The roads leading to various blocks and residences loop around this 
stately building. It is surrounded by 100 acres of land, which is left 
untouched and it provides plenty of cover to the avian neighbours. The 
cultivated garden in the front together with the built area occupies 
comparatively a small portion of the total ground. Within last 25 to 30 years 
of its occupation by the premiere laboratory, a large area was left 
unexploited giving rise to natural growth of vegetation. To this varied 
landscape is added the rambling hedgerows and private gardens interspersed 
with buildings and thickets. Such growth is an open invitation to a large 
number of insect species like grasshoppers, crickets and other gramnivorous 
insects. The soil is perennially covered by the grasses which grow 
luxuriantly. The leaf fall from the deciduous trees during winter, and 
aftermath of cut and dried grasses, provide rich food for fungi and ants and 
many other insects which can be seen swarming all around. The leaf litter 
found under the trees and hedges shelter a large number of bugs and beetles. 
The moths though not butterflies, deposit their eggs among the succulent 
leaves with multicolored caterpillars among the garden plants. With such 
plethora of insect life there is very abundance of food especially for the 
passerine birds. 

Since the estate was handed over to house CFTRI in 1952, lot of 
changes have taken place; the built area has increased. A mixture of natural 
and cultivated trees has occupied the campus, that is sprawling over hundred 
acres of land. A lot of shrubbery is growing along the rows of residential 
areas and the area is parceled out into boulevards, orchards and lawns. This 
manmade mosaic invites a large number of woodland birds in and out of 
season. The whole campus acts like a supermarket for bird populations in 
the environment of Mysore city. 

\heading{Bird Life}
While entering through the main gate you may witness the antics of 
the Roller or Blue Jay, with brilliant display of blue bands in flight. It is a 
beauteous bird and chosen as the Bird of Karnataka. It feasts on insects, 
pouncing on them from a vantage point. Because of its blue throat it is 
considered as incarnation of Siva. A little further, on way to the main 
building, a group of snowwhite little Egrets with black spear like bill will be 
seen stealthily advancing to pick up insects from the well watered lawns. 
During early periods, their lacy plumes were in great demand by milliners 
for adorning women's hats in Europe and America. They were slaughtered 
ruthlessly almost to extinction. Luckily change in women's fashion has 
saved them from complete extermination. A funny Spotted Owlet may be 
seen staring at you from the hollow of the large peepal tree. The ledges of 
the building provide a sanctuary for hundreds of Blue Rock Pigeons. The 
swifts too have found a refuge there. The pigeons are seen whirling round 
the building only to return hurriedly to their perch. On way to the Southern 
gate, concealed in the leafy branches of the groves of copper pods, will be 
noticed a lonely hawk, called Shikra, lying in wait for its prey which include 
young ones of birds. 

Along the culvert, on way to school, lurks the Whitebreasted 
Waterhen. It harbours along the marshy area with its family. Among the 
thickets bordering the Nullah and a patch of woodland behind the bunglows, 
I have hardly noticed any birds. Most of the varieties of birds I have noted 
are amongst the residential area behind the main building. Here, small 
flocks of crows abound. They are not the familiar House Crows with grey 
neck, but glossy jet black Jungle Crows. This is very intriguing indeed. 
Again, they have outcompeted the common House Sparrow from the human 
dwellings. Here and there, on open grounds, Mynahs are seen loitering and 
noisly hopping. In the spring the area reverberates with kuoo-kuoo, the call 
of the male Koel. It starts with low call reaching crescendo and breaks 
abruptly; it starts all over again monotonously. It is almost enchanting music 
heard at the break of the day from spring to pre-monsoon showers. Large 
flocks of Roseringed Parakeets, using campus as stopover, zoom across 
screaming noisily, keeak, keeak, keeack. The Pond Heron wearing its 
maroon breeding coat, and found singly, is watching patiently for its 
favourite quarry --- the frogs. From nearby tall copper pods, we hear Kutroo - Kutroo, the call of the Crimson Barbet which is often mistaken for 
woodpecker. In the bushes and the hedgerows are the flowerpeckers and 
warblers. Warblers are busily searching for insect larvae, constantly cocking 
their slender tail and uttering tee tee from time to time. Around the 
residential quarters, you find many other birds such as Hoopoes flashing 
their hood, Minivets, with their beautiful scarlet colouring, and Common 
Brown Babblers rummaging through the litters on the ground searching for 
their prey. We are greeted by the joyful calls of the celebrated songster, the 
Redvented Bulbul. The Crested Sepoy Bulbul makes its frequent 
appearance. Bulbuls are fond of gardens. If you are lucky you may notice a 
pair of Golden Orioles among leafy trees. The Green Bee -eaters line 
themselves on over-head wires, often seen hacking their prey or gliding back 
to their base. The Crow Pheasant are heard making deep resonant call, 
coop-coop-coop, repeated quickly. They are closely related to cuckoos, and 
are considered auspicious if one comes across them. The Spotted Dove 
enters your residence and nests in the verandah in some nook. Its calls krookruk-
krukroo ... kroo-kroo - kroo, are often heard over long distances. 

One hears cheerful towit towit notes of the Tailorbird among the 
shrubs and bush trees. One of the best songsters is the Magpie Robbin. 
Mornings are filled with its sweet songs. During breeding season, you may 
hear its plaintive notes. ``Did-he-do-it", call of Red Wattled Lapwing is very 
common. Lapwings are seen often in pair. Perhaps it is the only bird in the 
campus breeding in the open ground. I found its four stone coloured mottled 
eggs near the Technology Block. The old banyan and peepal trees are fond 
haunts of the fruit eating Common Grey Hornbill. Small parties fly in 
glidingly from tree to tree in follow my leader fashion. I have never been 
able to see, in spite of my careful search, their unique nest in which breeding 
female is imprisoned. You may also hear the chattering song of 
Whitebreasted Kingfisher seen with its brilliant turqoise blue coloured 
feathers and with a long red beak. It is not necessarily confined to water, as 
it feeds on terrestrial insects too. 


The Purple Sunbird makes its clumsy pendulous nest right in your 
garden or inside your house. You will see pairs with distinct colouration. 
Besides you come across many other varieties of woodland birds, the larks, 
drongos, wagtails, munias, minivets, Grey Tit, and raptors. Small parties of 
White Ibis are seen occasionally mingled with Egrets. 


Birds are driven by their appetite, perching places, suitable nesting 
sites, and mixed vegetation. Correlation will be found with the type of vegetation. The vegetation in the campus is of secondary 
type and it provides less habitable sites for birds. Their density depends on 
these factors. As there are no suitable nesting places, we do not come across 
resident birds, except Blue Rock Pigeons. The birds sneak in from outside on a brief visit, 
mainly for feeding. But it is still a wonder that within 100 acres, I could 
come across 53 species of birds belonging to 32 families. 

From 1986 onwards, I have made several tours around the campus on 
weekends to study the birds. I have tried to focus attention on their 
association with the background. So far as illustrations are concerned, one 
will find excellent coloured plates in Salim Ali's ``The Book of Indian 
Birds". The real joy will be in watching them yourself; all you need is a
binocular and a guide. There are many more fascinating facts of birdlife 
awaiting your discovery. 

\vskip 2cm

\bigskip

\hfill{N.P.Dani~ \quad }

\smallskip
\hfill{Rtd. Scientist~~}

\smallskip
\hfill{CFTRI~~~~\quad }

