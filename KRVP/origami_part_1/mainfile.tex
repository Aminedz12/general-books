\documentclass[12pt,twoside,openany]{book}

\input macros
\input dictionary



\begin{document}

\fontsize{13}{15}\selectfont
%\frontmatter

%\frontmatter

\input src/titlepage

\newpage

%\input src/copyright

%\newpage

%\input src/arpane
%\input src/mununxdi
%\input src/arike1
%%%%\input src/arike2%%%
%\input src/mununxdi

%\newpage

\input src/preface

\newpage

~

\phantom{a}

\thispagestyle{empty}

\newpage



%\input src/000a-title


%\input src/000b-copyright
%\rhead[]{{\fontsize{10}{12}\selectfont ಮುನ್ನುಡಿ\quad\arabictokannada{\thepage}}}
%\input src/000c-preface



\renewcommand{\mtctitle}{ವಿಷಯ ಪರಿವಿಡಿ}

%{\contentsname}{ಭಾಗ : 3 ಬಿಲ್ಲೆಗಳು [Tiles] \newline {\LARGE ವಿಷಯ ಪರಿವಿಡಿ}}

%\dominitoc

%\tableofcontents


%\thispagestyle{empty}

 \mainmatter


%\rhead[]{{\fontsize{10}{12}\selectfont \leftmark\quad\arabictokannada{\thepage}}}
{
%\setcounter{chapter}{0}
\input src/001-chapter001.tex
}
%\minitoc
\makeatletter
\renewcommand\chapter{\if@openright\cleardoublepage\else\ifthenelse{\arabic{chapter} > 0}{\chapterend}{}\fi
                    %~ \thispagestyle{plain}%
                    \global\@topnum\z@
                    \@afterindentfalse
                    \secdef\@chapter\@schapter}
\makeatother

\chapterend
\end{document}
