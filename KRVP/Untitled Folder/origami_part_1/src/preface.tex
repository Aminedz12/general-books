{
\makeatletter
\def\@makechapterhead#1{%
  \vspace*{5\p@}%
  {\parindent \z@ \raggedright \normalfont
    \ifnum \c@secnumdepth >\m@ne
      \if@mainmatter
        \LARGE\bfseries \@chapapp\space\thechapter
        \vskip 4pt
        \par\nobreak
        \vskip 5\p@
      \fi
    \fi
    \interlinepenalty\@M
    \LARGE\bfseries #1\par\nobreak
    \vskip 15\p@
  }}
  \makeatother
 %\chapter{leVKakara binanxha}

%\lhead[{\footnotesize\fontfamily{txr}\selectfont\thepage}]{{\footnotesize\sl\bfseries leVKakara binanxha}}



\chapter*{ಲೇಖಕನ ಮಾತು}

ಗಣಿತ  ವಿಷಯವು ಪ್ರಾರ್ಥಮಿಕ ಮತ್ತು ಪ್ರೌಢ ಶಾಲಾ ಹಂತದಲ್ಲಿ "ಕಬ್ಬಿಣದ ಕಡಲೆ" ಎಂಬ ಮಾತು ಎಲ್ಲರಿಗೂ ತಿಳಿದ ಸಂಗತಿಯಾಗಿದೆ. ಆದರೆ ಈ ಮಾತಿನ ಜೊತೆಗೆ `ಕಬ್ಬಿನ ರಸ' ಎಂಬ ಮಾತು ಸಹ ಇತ್ತಿಚೆಗೆ ಸೇರ್ಪಡೆಯಾಗಿದೆ. ಕಬ್ಬಿಣ ಕಡಲೆಯ ಬದಲಾಗಿ ಕಬ್ಬಿನ ರಸ ಬರಲು ಮುಖ್ಯ ಕಾರಣವೆನೆಂದರೆ, ಈಗ ಗಣಿತವನ್ನು ಪ್ರಾಯೋಗಿಕವಾಗಿ ಕಲಿಸುತ್ತಿದ್ದಾರೆ. ಪ್ರಾಯೋಗಿಕ ಗಣಿತವು ಶಾಲೆಯಲ್ಲಿ ಸ್ಥಾಪಿತವಾದ "ಗಣಿತ ಕ್ಲಬ್" [Mathematics Club] ಮೂಲಕ ಸಾಧ್ಯವಾಗಿದೆ. ಗಣಿತ ಕ್ಲಬ್ ನಲ್ಲಿ ಮುಖ್ಯವಾಗಿ "ಓರಿಗಾಮಿ ವಿಧಾನ' ದಿಂದ ಗಣಿತವನ್ನು ಕಲಿಸುತ್ತಿದ್ದಾರೆ. 

\medskip

ಓರಿಗಾಮಿ [Origami] ಕಲೆ ಜಪಾನ ಕಲೆಯಾಗಿದ್ದು ಅದನ್ನು ಹೀಗೆ ಹೇಳುತ್ತಾರೆ. "ಕೈ ಬೆರಳುಗಳ ಕೈಚಳಕದಲ್ಲಿ ಕಾಗದದಿಂದ ಸುಂದರ ಆಕೃತಿಗಳನ್ನು ಮಾಡುವ ಕಲೆಗೆ ಓರಿಗಾಮಿ ಕಲೆ" ಎನ್ನುತ್ತಾರೆ. ಈ ಕಲೆಯಲ್ಲಿ ಕಾಗದವನ್ನು ಕತ್ತರಿಸದೇ ಕೇವಲ ಮಡಚಿ ವಿವಿಧ ಬಗೆಯ ಆಕೃತಿಗಳನ್ನು ತಯಾರಿಸುತ್ತಾರೆ. ಈ ಕಲೆಯಿಂದ ಪ್ರಾಣಿ, ಪಕ್ಷಿಗಳ, ಎಲೆ ಹೂ, ಹಣ್ಣುಗಳ, ಮನೆ ಗೊಂಬೆಗಳ ಮುಂತಾದ ಆಟಿಕೆಗಳನ್ನು ತಯಾರಿಸುತ್ತಾರೆ. 

\medskip

ಕಿೃ. ಶ. 109 ನೇ ವರ್ಗದಲ್ಲಿ ಚೀನಾ ದೇಶವು ಕಾಗದ ತಯಾರಿಕೆಯನ್ನು ಆರಂಭಿಸಿತು. ಆಗ ಓರಿಗಾಮಿ ಕಲೆ ಮೊಳಕೆ ಬಡೆಯಿತು ಎಂದು ಕರೆಯುತ್ತಾರೆ. ಕಿೃ. ಶ. 1983 ರಲ್ಲಿ ಟಿ. ಸುಂದರ ರಾವ್ ಎಂಬವರು ಭಾರತದಲ್ಲಿ ಓರಿಗಾಮಿ ಕಲೆಯನ್ನು ಪರಿಚಯಿಸಿದರು. ಇವರು ರಚಿಸಿದ "Geometric excrecise in paper folding" ಪುಸ್ತಕವು ವಿಶ್ವಮಾನ್ಯತೆ ಪಡಿದಿದೆ. 

\medskip

ಇಂತಹ ಕಲೆಯನ್ನು ಪ್ರಾರಂಭದಲ್ಲಿ ಅಲಂಕಾರ ವಸ್ತುಗಳ ತಯಾರಿಕೆಗೆ ಉಪಯೋಗಿಸುತ್ತಿದ್ದರು. ಈಗ ಓರಿಗಾಮಿ ಕಲೆಯನ್ನು ಗಣಿತದ ಕಲಿಕೆಗೆ ಮತ್ತು ಕಲಿಸುವಿಕೆಗೆ ಉಪಯೋಗಿಸುತ್ತಾರೆ. ಆದ್ದರಿಂದ ನಾನು ಶಾಲಾ ಮಕ್ಕಳಿಗೆ, ಶಿಕ್ಷಕರಿಗೆ ಹಾಗೂ ಸಾಮಾನ್ಯ ಜನರು ಸಹ ಉಪಯೋಗವಾಗಲೆಂದು ಈ ಪುಸ್ತಕವನ್ನು ರಚಿಸಿದ್ದೇನೆ. 

\medskip

ಈ ಪುಸ್ತಕದ ಪ್ರಕಟನೆಗೆ ಕರಾವಿಪದ ಅಧ್ಯಕ್ಷರು, ಸದಸ್ಯರು ಮತ್ತು ಪ್ರಕಟನಾ ಮಂಡಳಿಯ ಅಧ್ಯಕ್ಷರು, ಸದಸ್ಯರಿಗೆ ನಾನು ಚಿರಯಣಿಯಾಗಿದ್ದೆನೆ. ಈ ಪುಸ್ತಕವು ಸುಂದರವಾಗಿ ಮುದ್ರಣವಾಗಲು ಸಹಕರಿಸಿದ ಕರಾವಿಪಸಿಬ್ಬಂದಿಗೆ ಮತ್ತು ಪ್ರಕಟನೆಗೆ ಸಹಾಯಮಾಡಿದವರಿಗೆ ನಾನು ಸ್ಮರಿಸುತ್ತಿದೆ. 

\eject

ಮಕ್ಕಳು, ಶಿಕ್ಷಕರು ಹಾಗೂ ಸಾರ್ವಜನಿಕರು ಈ ಪುಸ್ತಕವನ್ನು ಮಾಡಿ ಗಣಿತ ವಿಷಯವು ಕಬ್ಬಿಣ ಕಡಲೆ ಅಲ್ಲ ಇದೊಂದು ಕಬ್ಬಿನ ರಸ ಎಂಬುದನ್ನು ತೋರಿಸಿ ಕೊಡಬೇಕೆಂದು ವಿನಂತಿ. 



\begin{flushright}
{\bf ವೈ. ಬಿ. ಗುರಣ್ಣವರ}\\
{\bf ನಿವೃತ್ತ ಮುಖ್ಯೋಪಾಧ್ಯಾಯರು, ಕುಂದಗೋಳ}\\
\end{flushright}


}\relax




