\chapter{PamaRna aMtima parxmeVya}
\vskip -20pt

jAnf oLeLxya gaNitada meVSuTxrX. vidAyxthiRgaLige kutUhalakaravAda aneVka samaseyxgaLanunx AgAgeyx heVLutitxdadxru. oMdu sala $3, 4$ matutx {\rm 5} anukarxma saMKeyxgaLu hAgU pUNaRsaMKeyxgaLu. I $3, 4, 5$ pUNaRsaMKeyxgaLa naDuve EnAdarU visheVSa saMbaMdhavideyeV? eMdaru jAnf mAsatxru.

rameVsha takaSxNa utatxrisida $3\times 3=3^2=9$, $4\times 4=4^2=16$, $5\times 5=5^2=25$  
\begin{align*}
9+16 &=25\\
3^2+4^2&=5^2
\end{align*}
ideV riVti $4, 5$ matutx {\rm 6} pUNaRsaMKeyxgaLa naDuve EnAdarU visheVSa saMbaMdha\-videyeV? eMdu punaH keVLidaru. kamala dheYyaRvAgi niMtu utatxrisidaLu.

\begin{tabular}{>{$}l<{$}>{$}l<{$}>{$}l<{$}>{$}l<{$}}
   &4\times 4=4^2=16 & 5\times 5=5^2=25 & 6\times 6=6^2=36\\  
   &16+25 =41        &                   &  
\end{tabular}\\
AdadxriMda  $4^2+5^2$ eMbudu $6^2$ ge samavalalx. hAgAdare $3, 4, 5$ pUNaRsaMKeyxgaLa naDuve iruvaMte parasapxra saMbaMdhaviruva beVre saMKAyxtarxyagaLu ideyeV? I parxshenx kirxsatx pUvaR $800$ riMda $500$ raSuTx hiMdeyeV BAratiVya gaNitajacnxranunx kADitutx. avaru adakekx kelavu utatxragaLanunx paDedidadxru. shulavx sUtarxgaLalilx kelavu muKayx saMKAyxtarxyagaLive. shulavx sUtarx eMdareVnu? sArf eMda sureVsha shulavx eMdare hagagx yajacnxyAgAdi mADuva saMdaBaRdalilx idara baLake baMditutx. pAshAcxtayxrigU muMce BAratiVyarige tiLiditutx.
$$
3, 4, 5, \quad 5, 12, 13, \quad 7, 24, 25, \quad 8, 15, 17, \quad 12, 35, 37
$$
meVle tiLisiruva oMdoMdu tarxyadalilxyU eraDu saMKeyxgaLa vagaRgaLa motatx mUraneya saMKeyxya vagaRkekx samanAgive.

kirx.~pU. sumAru {\rm 582} riMda {\rm 497} ralilxdadx girxVkf gaNitajacnx  peYthAgorasf I samaseyxyanunx tiLisidadx. elilx avana parxmeVya heVLi noVDoVNa eMdaru jAnf? puTaTxsAvxmi edudx niMtu ``yAvudeV laMbakoVna tirxBujadalilx vikaNaRda meVlina cwkada sale uLideraDu BujagaLa meVlina cwkada salegaLa motatxkekx sama'' eMda. idanenxV nAvu peYthAgorasf parxmeVya eMdu heVLuvudu.

 $3,4,5$, $5,12,13$ modalAda visheVSa pUNaRsaMKeyxgaLa guMpige peYthA\-gorasfna saMKAyxtarxyagaLu enunxtetxVve. saMKAyxtarxyada visheVSaveVnu \-heVLutitxrA? eMdAga rahiVmf edudxniMtu parxtiyoMdu saMKAyxtarxyagaLalUlx cikakx eraDu saMKeyx\-gaLa vagaRgaLa motatx doDaDx saMKeyxya vagaRkekx sama. AdadxriMda peYthAgorasf tarxvaLigaLu BujagaLAguvaMte racisida elAlx tirxBujagaLU laMbakoVna tirxBujagaLAgirutatxve eMda. jAnf mAsatxru muMduvareyutAtx, DayoVPAyxMTasf kirx.~sha. sumAru {\rm 250} ra veVLege alekAsxMDirxyAdalilxdadxvanu. aMdu parxsidadhxvAgiruva biVjagaNitada parxvataRkaralilx parxsidadhxnAgidadxvanu.

datatx dhana pUNARMkavanunx eraDu dhanapUNARMkagaLa motatxvAgi oDeyalu sAdhayxveV? sAdhayx $5=1+4$, athavA $5=2+3$. ideV riVti $11=1+10$, $11=2+9$,  $11=3+8$, $11=4+7$, $11=5+6$.~ saMKeyx doDaDxdAdare parihAragaLa saMKeyxyU hecAcxgirutatxde. idanunx $a+b=5$, $m+n=11$ eMdu aDakavAgi bareyabahudu.\break $a$\; matutx \;$b$\; eMba eraDu ajAcnxtagaLu avugaLa motatx sadA $5$ AgiruvaMte baMdhitavAgive. ideV riVti \;$m$\; matutx \;$n$\; eMba eraDu ajAcnxtagaLu avugaLa motatx sadA $11$ AgiruvaMte baMdhitavAgive.
$$
a+b=5
$$
I samiVkaraNavanunx anukarxmavAgi $a$ matutx $b$ ajAcnxtagaLa $1.4$, $2.3$, $3.2$, $4.1$ belegaLu tALepaDisutatxve. beVre yAva belegaLigU alalx. I nAlukx jote belegaLige samiVkaraNada parihAragaLeMdu hesaru. $m+n=11$ eMbudakekx hatutx parihAragaLive. idanenxV sAvaRtirxkavAgi  $x+y=z$ eMdu bareyutetxVve. idara parihAragaLu anaMta saMKeyxyalilxve. idu atayxMta saraLa rUpada DayoVPAyxMTeYsf samiVkaraNa.

DayAPAyxMTeYnf samiVkaraNa eMdareVnu? eMda BAsakxra. ``pUNARMkagaLa guNaka\-gaLiruva, eraDu athavA hecucx ajAcnxtagaLanunx oLagoMDiruva matutx sAvaRtirxkavAgi heVLuvudAdare, parxtiyoMdu ajAcnxtavanUnx tALe noVDabalalx mwlayxgaLu anaMta saMKeyxyalilxruva anidhaRraNiVya samiVkaraNagaLanunx DayoVPAyxMTeYnf samiVkaraNa enunxtetxVve".

udAharaNege: \quad $x^2+y^2=z^2$

idaralilx $x,y$ matutx $z$ eMba mUru ajAcnxtagaLive. 

$x=3$, $y=4$, $z=5$ matutx $x=5$, $y=12$, $z=13$ muMtAda peYthAgorasf tarxyagaLu I samiVkaraNakekx tALe hoMdutatxve. I riVtiya peYthAgorasf tarxyagaLu eSiTxve? 

{\rm 1665} ralilx PamAR gatisida naMtara parxkaTitavAda avana parxbaMdha saMkalanagaLalilx ``PamARna aMtima parxmeVya'' eMba hesariniMda parxpaMcada beLakanunx kaMDitu.

PamARna anaMtara matAyxrU parxyatinxsalilalxve? eMdaLu paMkajA.

PamARna anaMtara avana shirxVmaMtikeyanunx tiLididadx gaNitajacnxru Atana aMtima parxmeVyada sAdhaneya shoVdhaneyalilx niratarAdaru. Adare PamAR kaMDidadx A ``atayxMta aduBxtavAda sAdhane'' mAtarx yAra keYgU sigalilalx. pAyxrisf akAyxDami {\rm 1816} ralilx PamARna aMtima parxmeVyakekx sAdhane niVDidavarige bahumAna niVDi gwravisutetxVveMdu  GoVSisitu.

jamaRniya kAlfRvilf helfmx pirxVDarishf gwsf {\rm (1777 - 1855)} eMbuvana muMde, avana senxVhitaru I viSayavanunx taMdAga, avanu idanunx laGuvAgi sivxVkarisida.

muMde idanunx yAru muMduvarisidaru? eMda rahiVmf. jamaRniya porxPesarf pAlf vUlfPxshekxVlf I parxmeVyakekx saMpUNaR sAdhane niVDuvavarige oMdu lakaSx mAkfsxRdhana bahumAnavAgi koDuvudAgi  GoVSisida. aneVkaru parxyatinxsidaru, Adare PamARna aMtima parxmeVyada sAdhane laBayxvAgalilalx.

DeVviDf hilfbaTfR {\rm (1862 - 1943)} jamaRniya gaNita vidAvxMsa `PamaRna aMtima parxmeVyakekx' sAdhane koTiTxdedxVne eMdu yAreV heVLali adaralilx EnoV nUyxnate nusiLiruvudu KaMDita. adu EneMdu patetxhacicx, eMdu hilfbaTfR avana anuyAyigaLige heVLutitxdadxnaMte.

$n$ na yAva beleya tanaka samiVkaraNa sarihoMduvudilalx eMdu kaMDu\-hiDididadxru? eMdaLu lalita.

$n$ na bele {\rm 3} riMda $25,000$ davarege iruvAga $x^n+y^n=z^n$ DayAPAyxMTeYnf samiVkaraNavanunx sarihoMduva pUNARMkagaLu ilalxveMdu daqDhapaDisalAgitutx.

I samaseyx bage hariyiteV? eMda umeVsha. {\rm 1995} ralilx laMDaninxna obabx mahA meVdhAvi I samaseyxge parihAra niVDida eMba aMsha elalxrU saMtoVSapaDabeVkAdedx.

Iga Ita amerikAdalilx nelesirutAtxne. \textbf{adanunx tiLiyalu sAvaRtirxka sUtarxveV\-nAdarU ideyeV?} eMda rameVsha $x=2n+1$,~ $y=2n^2+2n$,~ $z=2n^2+2n+1.$ $n$ ge belekoTaTxre peYthAgorasf tarxya namage dorakutatxde aMdare $x^2+y^2=z^2$ DayoPAyxMTeYnf samiVkaraNada parihAra doreyutatxde. 
\begin{flalign*}
\text{ udAharaNege:} \quad n &=1 \quad \text{AdAga} \quad 3, 4, 5&\\
n &=2 \quad \text{AdAga} \quad 5, 12, 13
\end{flalign*}
Adare BAratada shulavx sUtarxkAraru kaMDukoMDidadx {\rm 8, 15, 17} matutx {\rm 12, 35, 37}~nunx $n$ ge yAva bele koTaTxrU I sUtarx niVDuvudilalx. shulavx sUtarxkArarigU yAvudeV sAvaRtirxka sUtarx gotitxtutx eMbudakekx AdhAravilalx. peYthAgorasfna anaMtara baMda gaNitajacnxru (ciMtakaru) peYthAgorasf tarxyagaLanunx niVDabalalx matetx yAvudAdarU sUtarxvaunx niVDidAdxreyeV eMda umeVsha. jAnf niVDidAdxre eMdu tiLisutatx, A sUtarxveV $x=2n$, $y=n^{2}-1$, matutx $z=n^2+1$, I sUtarxda parxkAra $n=4$ AdAga $8, 15, 17$ peYthAgorasf tirxvaLigaLanunx niVDutatxde.
\begin{align*}
x+y &=z\\
x^2+y^2 &=z^2
\end{align*}

eMba DayoVPAyxMTeYnf samiVkaraNagaLa parihAragaLu anaMta saMKeyxyalilxve. idanunx koDabalalx sAvaRtirxka sUtarxgaLu tiLidive eMdAga $x^3+y^3=z^3$,\break $x^4+y^4=z^4$,~ $x^5+y^5=z^5$ gaLigU I tiVmARnagaLu anavxyavAguvudeV eMdu keVLida joVsePf.

Aga jAnf mAsatxru BASaNa AraMBisidaru. PArxnisxna piyare Da PamAR\break {\rm (1601 - 1665)} $n$ na bele {\rm 2} kikxMta jAsitxyAda yAvudeV pUNARMkavAda I samiVkaraNavanunx tALe hoMduva pUNARMkagaLu ideyeV athavA ilalxveV eMbudakekx sAdhane niVDuvudu avanige savAlAgitutx.

``I sAvaRtirxka parxmeVyakekx nijakUkx atayxMta aduBxtavAda sAdhaneyanunx nAnu paDedidedxVne. Adare adanunx nirUpisalu I puTada KAli aMcina jAga sAladu eMdidadx''

avana parxkAra $2$ neya GAtakikxMta meVlina yAvudeV GAtada saMKeyxyanunx adeV GAtada eraDu saMKeyxgaLAgi oDeyalu sAdhayxvilalx. aMdare $n$ na bele {\rm 2} kikxMta jAsitx\-yAda pUNARMkavAdAga $x^n+y^n=z^n$ DayoVPAyxMTeYnf~samiVkaraNavanunx sari hoMduva pUNARMkagaLilalx.

