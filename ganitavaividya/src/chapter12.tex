\chapter{aparUpada EkakAlika samiVkaraNa}
\vskip -20pt

shirxV raMgaNaNxnavaru gaNita mAsatxru. avara taragati eMdare vidAyxthiRgaLige bahaLa iSaTx. taragatiyalilx pATha mADutAtx EnAdaroMdu hosa viSayavanunx heVLi vidAyxthiRgaLa kutUhalavanunx keraLisutitxdadxru. adu {\rm 9} neya taragati hiMdina taragatiyalilx aBAyxsa mADida aneVka viSayagaLanunx vidAyxthiRgaLiMdaleV keVLutAtx, oMdu haMtadalilx samiVkaraNa eMdareVnu? samiVkaraNadalilx nAvu gamanisabeVkAda aMshagaLeVnu hiVge elAlx keVLutAtx shirxraMgaNaNxnavaru saraLasamiVkaraNakekx baMdu udAharaNe keVLidaru. rameVsha taTaTxne edudx utatxriseVbiTaTx. kapupx halageya meVle baMdu bare eMdaru. dheYyaRvAgi bareda $x+4=10$ I samiVkaraNadalilx avayxkatx pada eSiTxde eMdaru? avayxkatx pada $x$ oMdeV alalxveV eMdu utatxrisida adara GAtaveVnu? eMdaru. $x$ na GAta oMdeV eMda. I samiVkaraNadalilx $x$ na bele Enu eMdaru? $x$ na bele {\rm 6} eMda.

eraDu avayxkatxpadagaLiruva oMdu samiVkaraNa heVLi eMdaru raMgamamx edudxniMtu $x+y=10$ eMdaLu. $x$ matutx $y$ gaLa bele Enu? eMdaru. gwri edudxniMtu $x=4$, $y=6$ eMdaLu. kAMta edudx niMtu $x=6$, $y=4$ eMda. lalita edudx niMtu \;$x$\; matutx $y$ gaLa belegaLanunx nidiRSaTxvAgi iSeTxV eMdu heVLalu sAdhayxvilalx eMdaLu. mAsatxru raMgaNaNx hAgAdare Enu mADabeVku? eMdaru. umeVsha edudx niMtu inonxMdu samiVkaraNa koTaTxre $x$ matutx $y$ gaLa bele heVLabahudu eMda. avananenxV matotxMdu samiVkaraNa koDuvaMte keVLidaru $x-y=4$ eMda. eraDu samiVkaraNagaLanUnx kapupx halageya meVle baredaru.
\begin{align*}
x+y &=10\\
x-y &=4
\end{align*}
rahiVmananunx karedu idu yAva riVtiya samiVkaraNa eMdaru.? idu EkakAlika samiVkaraNa aMda. sari $x$ matutx $y$ gaLa bele heVLu eMdaru. $+y$ matutx $-y$~nunx hoDedu $x$ matutx $x$ gaLanunx kUDi, {\rm 10} matutx {\rm 4} nunx kUDi, $2x=14$\; $\therefore$ $x=7$  $x=7$  AdameVle $y$ na bele {\rm 3} eMda. $x$ matutx $y$ gaLige matAyxvudAdarU bele ideyeV eMdaru. ilalx ideV nidiRSaTx bele eMda.

mAsatxru raMgaNaNxnavaru matotxMdu samaseyxyanunx vidAyxthiRgaLa muMdiTaTxru
A samaseyxyeV $\sqrt{x}+y=7$,~ $x+\sqrt{y}=11$ vidAyxthiRgaLu mADalu AraMBisidaru. savxlapx hotitxna naMtara namage heVge mADabeVkeMdu toVcutitxlalx eMdaru. idoMdu visheVSavAda EkakAlika samiVkaraNa. iMtaha samiVkaraNavanunx Adhunika BAratiVya gaNitajacnx shirxVnivAsa rAmAnujanf sulaBavAgi mADutitxdadxnaMte. Iga noVDi mADi toVrisutetxVne eMdu heVLi samaseyx biDisalu AraMBisidaru.
\begin{align*}
\sqrt{x}+y&=7 \tag{\rm 1}\\
x+\sqrt{y} &=11 \tag{\rm 2}\\
\sqrt{x}+y &=7 \tag{\rm 1}\\
\therefore \quad \sqrt{x} & =7-y\\
x&=(7-y)^2\\
x &=49-14y+y^2\\
x+\sqrt{y} =11 \tag{\rm 2}\\
x \quad \text{ge bele AdeVshisidare}\\
49-14y+y^2+\sqrt{y}=11\\
\sqrt{y}=t \quad \text{Agirali}\\
49-14t^2+t^4+t=11\\
t^4-14t^2+t=-38\\
t=2 \quad \text{Agirali}
\end{align*}

$$
\begin{array}{c|ccccc}
\cline{2-6}
2 &1 & 0 & -14 & 1  & 38\\
&\dfrac{0}{1}& \dfrac{2}{2} &\dfrac{4}{-10}& \dfrac{-20}{-19}&\dfrac{-38}{0}
\end{array}
$$

\hspace{1.2cm}
\begin{tabular}{>{$}r<{$}@{\;}>{$}r<{$}}
\sqrt{y} &=t\\
t&=2\\
\therefore \quad y&=4
\end{tabular}
\begin{align*}
x+\sqrt{y} &=11\\
x+\sqrt{4} &=11\\
x+2 &=11\\
x &=11-2\\
x &=9 \qquad 
\text{AdadxriMda} \quad x=9, \quad \text{matutx} \quad y=4
\end{align*}

\begin{center}
{\bf gaNitada bagegx heVLike} 
\end{center}
\begin{enumerate}[\rm 1)]
\item atayxMta savxSaTxvU Ada vAsatxva saMgatiyU yAva heVLikeyU aMtimavAgi gaNitada rUpavanunx tALabeVku.

\item adu pUNaR, idu pUNaR, pUNaRdiMda pUNaRvu horahomimxde pUNaRdiMda pUNaRvu horabaMdarU uLidiruvudu pUNaRvAgiyeV ide.

~\hfill -upaniSatf
\end{enumerate}
