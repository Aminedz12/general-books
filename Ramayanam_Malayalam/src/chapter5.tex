%%01_sundharakaandam

\chapter{സുന്ദരകാണ്ഡം}

\begin{verse}
സകലശുകകുലവിമലതിലകിതകളേബരേ!\\
സാരസ്യപീയൂഷസാരസര്‍വസ്വമേ!\\
കഥയ മമ കഥയ മമ കഥകളതിസാദരം\\
കാകുല്‍സ്ഥലീലകള്‍ കേട്ടാല്‍ മതിവരാ.\\
കിളിമകളൊടതിസരസമിതി രഘുകുലാധിപന്‍-\\
കീര്‍ത്തി കേട്ടീടുവാന്‍ ചോദിച്ചനന്തരം\\
കളമൊഴിയുമഴകിനൊടു തൊഴുതു ചൊല്ലീടിനാള്‍\\
കാരുണ്യമൂര്‍ത്തിയെച്ചിന്തിച്ചു മാനസേ.\\
ഹിമശിഖരിസുതയൊടു ചിരിച്ചു ഗംഗാധര-\\
നെങ്കിലോ കേട്ടുകൊള്‍കെന്നരുളിച്ചെയ്തു.
\end{verse}

%%02_samdralanganam

\section{സമുദ്രലംഘനം}

\begin{verse}
ലവണജലനിധി ശതകയോജനാവിസ്തൃതം\\
ലംഘിച്ചു ലംങ്കയില്‍ ചെല്ലുവാന്‍ മാരുതി\\
മനുജപരിവൃഢചരണനളിനയുഗളം മുദാ\\
മാനസേ ചിന്തിച്ചുറപ്പിച്ചു നിശ്ചലം\\
കപിവരരൊടമിതബലസഹിതമുരചെയ്തിതു:\\
“കണ്ടുകൊള്‍വിന്‍ നിങ്ങലെങ്കിലെല്ലാവരും.\\
മമ ജനകസദൃഷനഹമതിചപലമംബരേ\\
മാനേന പോകുന്നിതാശരേശാലയേ\\
അജതനയതനയശരസമമധികസാഹസാ-\\
ലദ്യൈവ പശ്യാമി രാമപത്നീമഹം.\\
അഖിലജഗദധിപനൊടു വിരവിലറിയിപ്പനി-\\
ങ്ങദ്യ കൃതാര്‍ത്ഥനായേന്‍ കൃതാര്‍ത്ഥേസ്മ്യഹം.\\
പ്രണതജനബഹുജനന മരണ ഹരനാമകം\\
പ്രാണപ്രയാണകാലേ നിരൂപിപ്പവന്‍\\
ജനിമരണജലനിധിയെ വിരവൊടു കടക്കുമ-\\
ജ്ജന്മനാ കിം പുനസ്തസ്യ ദൂതോസ്മ്യഹം\\
തദനു മമ ഹൃദി സപദി രഘുപതിരനാരതാം\\
തസ്യാംഗുലീയവുമുണ്ടു ശിരസി മേ\\
കിമപി നഹി ഭയമുദധി സപദി തരിതും നിങ്ങള്‍\\
കീശപ്രവരരേ! ഖേദിയായ്കേതുമേ.’\\
ഇതി പവനതനയനുരചെതു വാലും നിജ-\\
മേറ്റമുയര്‍ത്തിപ്പരത്തിക്കരങ്ങളും\\
അതിവിപുലഗളതലവുമാര്‍ജവമാക്കി നി-\\
ന്നാകുഞ്ചിതാംഘ്രിയായൂര്‍ദ്ധ്വനയനനായ്\\
ദശവദനപുരിയില്‍ നിജ ഹൃദയവുമുറപ്പിച്ചു\\
ദക്ഷിണദിക്കുമാലോക്യ ചാടീടിനാന്‍.
\end{verse}

%%03_maarghavignam

\section{മാര്‍ഗവിഘനം}

\begin{verse}
പതഗപതിരിവ പവനസുതനഥ വിഹായസാ\\
ഭാനുബിംബാഭയാ പോകും ദശാന്തരേ\\
അമരസമുദയമനിലതനയബലവേഗങ്ങ-\\
ളാലോക്യ ചൊന്നാര്‍ പരീക്ഷണാര്‍ത്ഥം തദാ\\
സുരസയൊടു പവനസുതസുഖഗതി മുടക്കുവാന്‍\\
തൂര്‍ണം നടന്നിതു നാഗജനനിയും.\\
ത്വരിതമനിലജമതിബലങ്ങളറിഞ്ഞതി-\\
സൂക്ഷ്മദൃശാ പവികെന്നതു കേട്ടവള്‍\\
ഗഗനപഥി പവനസുത ജവഗതി മുടക്കുവാന്‍\\
ഗര്‍വേണ ചെന്നു തല്‍സന്നിധൗ മേവിനാള്‍.\\
കഠിനതരമലറിയവളവനൊടുരചെയ്തിതു:\\
“കണ്ടീലയോ ഭാവാനെന്നെക്കപിവര!\\
ഭഹരഹിതമിതുവഴി നടക്കുന്നവര്‍കളെ\\
ഭക്ഷിപ്പതിന്നു മാം കല്പിച്ചിതീശ്വരന്‍.\\
വിധിവിഹിതമശനമിഹ നൂനമദ്യ ത്വയാ\\
വീരാ! വിശപ്പെനിക്കേറ്റമുണ്ടോര്‍ക്ക നീ\\
മമ വദനകുഹരമതില്‍ വിരവിനൊടു പൂക നീ\\
മറ്റൊന്നുമോര്‍ത്തു കാലം കളയായ്കെടോ!”\\
സരസമിതി രഭസതരമതനു സുരസാഗിരം\\
സാഹസാല്‍ കേട്ടനിലാത്മജന്‍ ചൊല്ലിനാന്‍:\\
“അഹമഖിലജഗദധിപനമരഗുരുശാസനാ-\\
ലാശു സീതാന്വേഷണത്തിന്നു പോകുന്നു.\\
അവളെ നിശിചരപുരിയില്‍ വിരവിനൊടു ചെന്നു ക-\\
ണ്ടദ്യ വാ ശ്വോ വാ വരുന്നതുമുണ്ടു ഞാന്‍.\\
ജനകനരപതി ദുഹിതൃചരിതമഖിലം ദ്രുതം\\
ചെന്നു രഘുപതിയോടറിയിച്ചു ഞാന്‍\\
തവ വദനകുഹരമതിലപഗതഭായാകുലം\\
താത്പര്യമുള്‍ക്കൊണ്ടു വന്നു പുക്കീടുവന്‍.\\
അനൃതമകതളിരിലൊരുപൊഴുതുമറിവീലഹ-\\
മാശു മാര്‍ഗം ദേഹി ദേവി! നമോസ്തു തേ.”\\
തദനു കപികുലവരനൊടവളുമുരചെയ്തിതു:\\
‘ദാഹവും ക്ഷുത്തും പൊറുക്കരുതേതുമേ.’\\
‘മനസി തവ സുദൃഢമിതി യദി സപദി സാദരം\\
വാ പിളര്‍ന്നീടെ’ന്നു മാരുതി ചൊല്ലിനാന്‍.\\
അതിവിപുലമുടലുമൊരു യോജനായാമമാ-\\
യാശുഗനന്ദനന്‍ നിന്നതു കണ്ടവള്‍\\
അതിലധികതരവദനവിവരമൊടനാകുല-\\
മത്ഭുതമായഞ്ചു യോജനാവിസ്തൃതം.\\
പവനതനയനുമതിനു ഝടിതി ദശയോജനാ-\\
പരിമിതി കലര്‍ന്നു കാണായോരനന്തരം\\
നിജമനസി ഗുരുകുതുകമൊടു സുരസയുംതദാ\\
നിന്നാളിരുപതു യോജന വായുമായ്.\\
മുഖകുഹരമതിവിപുലമിതി കരുതി മാരുതി\\
മുപ്പതു യോജന വണ്ണമായ് മേവിനാന്‍.\\
അലമലമിതയമമലനരുതു ജയമാര്‍ക്കുമെ-\\
ന്നന്‍പതു യോജന വാ പിളര്‍ന്നീടിനാള്‍.\\
അതു പൊഴുതു പവനസുതനതികൃശ ശരീരനാ-\\
യംഗുഷ്ഠതുല്യനായുള്‍പ്പുക്കരുളിനാന്‍.\\
തദനു ലഘുതരമവനുമുരുതരതപോബലാല്‍\\
തത്ര പുറത്തു പുറപ്പെട്ടു ചൊല്ലിനാന്‍:\\
‘ശൃടു സുമുഖി! സുരസുഖപരേ! സുരസേ! ശൂഭേ!\\
ശുദ്ധേ! ഭുജംഗമാതാവേ! നമോസ്തുതേ!\\
ശരണമിഹ ചരണസരസിജയുഗളമേവ തേ\\
ശാന്തേ! ശാരണ്യേ! നമസ്തേ നമോസ്തുതേ!\\
പ്ലവഗപരിവൃഢവചനനിശമനദശാന്തരേ\\
പേര്‍ത്തും ചിരിച്ചു പറഞ്ഞു സുരസയും:\\
‘വരിക തവ ജയമതി സുഖേന പോയ് ചെന്നു നീ\\
മല്ലഭാവൃത്താന്തമുള്ളവണ്ണം മുദാ\\
രഘുപതിയൊടഖിലമറിയിക്ക തല്‍കോപേന\\
രക്ഷോഗണത്തെയുമൊക്കെയൊടുക്കണം\\
അറിവതിനു തവ ബലവിവേകവേഗാദിക-\\
ളാദിതേയന്മാരയച്ചു വന്നേനഹം.’\\
നിജചരിതമഖിലമവനോടറിയിച്ചുപോയ്\\
നിര്‍ജരലോകം ഗമിച്ചാള്‍ സുരസയും.\\
പവനസുതനഥ ഗഗനപഥി ഗരുഡതുല്യനായ്\\
പാഞ്ഞു പാരാവാരമീതേ ഗമിക്കുമ്പോള്‍\\
ജലനിധിയുമചലവരനോടു ചൊല്ലീടിനാന്‍:\\
‘ചെന്നു നീ സല്‍ക്കരിക്കേണം കപീന്ദ്രനെ.\\
സഗരനരപതിതനയരെന്നെ വളര്‍ക്കയാല്‍\\
സാഗരമെന്നു ചൊല്ലുന്നിതെല്ലാവരും.\\
തദഭിജനഭവനറിക രാമന്‍ തിരുവടി\\
തസ്യ കാര്യാര്‍ഥമായ് പോകുന്നതുമിവന്‍\\
ഇടയിലൊരു പതനമവനില്ല തല്‍ക്കാരണാ-\\
ലിച്ഛയാ പൊങ്ങിത്തളര്‍ച്ച് തീര്‍ത്തീടണം.’\\
മണികനകമയനമലനായ മൈനാകവും\\
മാനുഷവേഷം ധരിച്ചു ചൊല്ലീടിനാന്‍:\\
‘ഹിമശിഖരിതനയനഹമറിക, കപിവീര! നീ-\\
യെന്മേലിരുത്തു തളര്‍ച്ചയും തീര്‍ക്കെടോ!\\
സലിലനിധി സരഭസമയയ്ക്കയാല്‍ വന്നു ഞാന്‍\\
സാദവും ദാഹവും തീര്‍ത്തു പൊയ്ക്കൊള്‍കെടോ!\\
അമൃതസമജലവുമതിമധുരമധുപൂരവു-\\
മാര്‍ദ്രപക്വങ്ങളും ഭക്ഷിച്ചുകൊള്‍ക നീ.’\\
‘അലമലമിതരുതരുതു രാമകാര്യാര്‍ഥമാ-\\
യാശു പോകുംവിധൗ പാര്‍ക്കരുതെങ്ങുമേ.\\
പെരുവഴിയിലശനശയനങ്ങള്‍ ചെയ്കെന്നതും\\
പേര്‍ത്തു മറ്റൊന്നു ഭാവിക്കയെന്നുള്ളതും\\
അനുചിതമിതറിക രഘുകുലതിലകകാര്യങ്ങ-\\
ളന്‍പോടു സാധിച്ചൊഴിഞ്ഞരുതൊന്നുമേ.\\
വിഗതഭയമിനി വിരവൊടിന്നു ഞാന്‍ പോകുന്നു\\
ബന്ധുസല്ക്കാരം പരിഗ്രഹിച്ചേനഹം.’\\
പവനസുതനിവയുമുരചെയ്തു തന്‍ കൈകളാല്‍\\
പര്‍വതാധീശ്വരനെത്തലോടീടിനാന്‍.\\
പുനരവനുമനിലസമമുഴറി നടകൊണ്ടിതു\\
പുണ്യജനേന്ദ്രപുരം പ്രതി സംഭ്രമാല്‍.\\
തദനു ജലനിധിയിലതിഗഭീരദേശാലയേ\\
സന്തതം വാണെഴും ഛായാഗ്രഹിണിയും\\
സരിദധിപനുപരി പരിചോടു പോകുന്നവന്‍-\\
തന്‍ നിഴലാശു പിടിച്ചു നിര്‍ത്തീടിനാള്‍.\\
അതുപൊഉതുമമ ഗതി മുടക്കിയതാരെന്നി-\\
തന്തരാ പാര്‍ത്തു കീഴ്പോട്ടു നോക്കീടിനാന്‍.\\
അതിവിപുലതര ഭയകരാംഗിയെക്കണ്ടള-\\
വംഘ്രിപാതേന കൊന്നീടിനാന്‍ തല്‍ക്ഷണേ.\\
നിഴലതു പിടിച്ചു നിര്‍ത്തിക്കൊന്നു തിന്നുന്ന\\
നീചയാ സിംഹികയെക്കൊന്നനന്തരം\\
ദശവദനപുരിയില്‍ വിരവോടു പോയീടുവാന്‍\\
ദക്ഷിണദിക്കുനോക്കിക്കുതിച്ചീടിനാന്‍!\\
ചരമഗിരി ശിരസി രവിയും പ്രവേശിച്ചിതു\\
ചാരുലങ്കാഗോപുരാഗ്രേ കപീന്ദ്രനും.\\
ദശവദനനഗരമതിവിമലവിപുലസ്ഥലം\\
ദക്ഷിണവാരിധി മധ്യേ മനോഹരം\\
ബഹുലഫലകുസുമ ദലയുതവിടപിസങ്കുലം\\
വല്ലീകുലാവൃതം പക്ഷിമൃഗാന്വിതം\\
മണികനകമയമമരപുരസദൃശമംബുധീ-\\
മധ്യേ ത്രികൂടാചലോപരി മാരുതി\\
കമലമകള്‍ ചരിതമറിവതിനു ചെന്നന്‍പോടു\\
കണ്ടിതു ലങ്കാനഗരം നിരുപമം.\\
കനകവിരചിതമതില്‍ കിടങ്ങും പലതരം\\
കണ്ടു കടപ്പാന്‍ പണിയെന്നു മാനസേ\\
പരവശതയൊടു ഝടിതി പലവഴി നിരൂപിച്ചു\\
പദ്മനാഭന്‍തന്നെ ധ്യാനിച്ചുമേവിനാന്‍\\
നിശി തമസി നിശിചരപുരേ കൃശരൂപനായ്\\
നിര്‍ജനദേശേ കടപ്പനെന്നോര്‍ത്തവന്‍\\
നിജമനസി നിശിചരകുലാരിയെ ധ്യാണിച്ചു\\
നിര്‍ജരവൈരിപുരം ഗമിച്ചീടിനാന്‍.
\end{verse}

%%04_lankaalakshmeemoksham

\section{ലങ്കാലക്ഷ്മീമോക്ഷം}

\begin{verse}
പ്രകൃതിചപലനുമധികചപലമചലം മഹല്‍-\\
പ്രാകാരവും മുറിച്ചാകാരവും മറ-\\
ച്ചവനിമകളടിമലരുമകതളിരിലോര്‍ത്തുകൊ-\\
ണ്ടഞ്ജനാനന്ദനനഞ്ജസാ നിര്‍ഭയം,\\
ഉടല്‍ കടുകിനൊടു സമമിടത്തുകാല്‍ മുമ്പില്‍ വ-\\
ച്ചുള്ളില്‍ കടപ്പാന്‍ തുടങ്ങും ദശാന്തരേ\\
കഠിനതരമലറിയൊരു രജനിചരിവേഷമായ്\\
കാണായിതാശു ലങ്കാശ്രീയെയും തദാ.\\
‘ഇവിടെ വരുവതിനു പറകെന്തുമൂലം ഭവാ-\\
നേകനായ് ചോരനോ ചൊല്ലുനിന്‍ വാഞ്ഛിതം\\
അസുരസുരനരപശുമൃഗാദിജന്തുക്കള്‍ മ-\\
റ്റാര്‍ക്കുമേ വന്നുകൂടാ ഞാനറിയാതെ.’\\
ഇതി പരുഷവചനമൊടണഞ്ഞു താഡിച്ചിതൊ-\\
ന്നേറെ രോഷേണ താഡിച്ചു കപീന്ദ്രനും\\
രഘുകുലജവരസചിവ വാമമുഷ്ടിപ്രഹാ-\\
രേണ പതിച്ചു വമിച്ചിതു ചോരയും.\\
കപിവരനൊടവളുമെഴുനേറ്റു ചൊല്ലീടിനാള്‍:\\
‘കണ്ടേനെടോ തവ ബാഹുബലം സഖേ!\\
വിധിവിഹിതമിതു മമ പുരൈവ ധാതാവുതാന്‍\\
വീരാ! പറഞ്ഞിതെന്നോടിതു മുന്നമേ.\\
‘സകലജഗദധിപതി സനാതനന്‍ മാധവന്‍\\
സാക്ഷാല്‍ മഹാവിഷ്ണുമൂര്‍ത്തി നാരായണന്‍\\
കമലദലനയനനവനിയിലവതരിക്കുമുള്‍-\\
ക്കാരുണ്യമോടഷ്ടവിംശതി പര്യയേ.\\
ദശരഥനൃപതിതനയനായ് മമ പ്രാര്‍ഥനാല്‍\\
ത്രേതായുഗേ ധര്‍മദേവരക്ഷാര്‍ഥമായ്\\
ജനകനൃപവരനു മകളായ് നിജ മായയും\\
ജാതയാം പംക്തിമുഖവിനാശത്തിനായ്\\
സരസിരുഹനയന നടവിയിലഥ തപസ്സിനായ്\\
സഭ്രാതൃഭാര്യനായ് വാഴും ദശാന്തരേ\\
ദശവദനനവനിമകളെയുമപഹരിച്ചുടന്‍\\
ദക്ഷിണവാരിധി പിക്കിരിക്കുന്ന നാള്‍\\
സപദി രഘുവരനൊടരുണജന്നു സാചിവ്യവും\\
സംഭവിക്കും, പുനസ്സുഗ്രീവശാസനാല്‍\\
സകല ദിശി കപികള്‍ തിരവാന്‍ നടക്കുന്നതില്‍\\
സന്നദ്ധനായ് വരുമേകന്‍ തവാന്തികേ.\\
കലഹമനോടു ഝടതി തുടരുമളവെത്രയും\\
കാതരയായ് വരും നീയെന്നു നിര്‍ണയം.\\
രണനിപുണനൊടു ഭവതി താഡനവും കൊണ്ടു\\
രാമദൂതന്നു നല്‍കേണമനുജ്ഞയും\\
ഒരു കപിയൊടൊരു ദിവസമടി ഝടിതി കൊള്‍കില്‍ നീ-\\
യോടീ വാങ്ങിക്കൊള്ളുകെ’ന്ന് വിരിഞ്ചനും\\
കരുണയൊടു ഗതകപടമായ് നിയോഗിക്കയാല്‍\\
കാത്തിരുന്നേനിവിടം പല കാലവും\\
ലഘുഗതിയൊടിനിയൊരിടരൊഴിയെ നടകൊള്‍ക നീ\\
ലങ്കയും നിന്നാല്‍ ജിതയായിതിന്നെടോ!\\
നിഖിലനിശിചരകുലപതിക്കു മരണവും\\
നിശ്ചയമേറ്റമടുത്തു ചമഞ്ഞിതു.\\
ഭഗവദനുചര! ഭവതു ഭാഗ്യം ഭവാനിനി-\\
പ്പരാതെ ചെന്ന് കണ്ടീടുക ദേവിയെ.\\
ത്രിദശകുലരിപുദശമുഖാന്തഃ പുരവരേ\\
ദിവ്യ ലീലാവനേ പാദപസങ്കുലേ.\\
നവകുസുമദലസഹിത വിടപിയുത ശിംശപാ-\\
നാമ വൃക്ഷത്തിന്‍ ചുവട്ടിലതിശുചാ\\
നിശിചരികള്‍ നടുവിലഴലൊടു മരുവിടുന്നെടോ,\\
നിര്‍മലഗാത്രിയാം ജാനകി സന്തതം.\\
ത്വരിതമവള്‍ചരിതമുടനവനോടറിയിക്ക പോ-\\
യംബുധിയും കടന്നംബരാന്തേ ഭവാന്‍\\
അഖിലജഗദധിപതി രഘൂത്തമന്‍ പാതു മാ-\\
മസ്തുതേ സ്വസ്തിരത്യുത്തമോത്തംസമേ!’\\
ലഘുവചനമധുരമിതി ചൊല്ലി മറഞ്ഞിതു\\
ലങ്കയില്‍ നിന്നു വാങ്ങീ മലര്‍മങ്കയും.
\end{verse}

%%05_seethaasandarshanam

\section{സീതാസന്ദര്‍ശനം}

\begin{verse}
ഉദകനിധിനടുവില്‍ മരുവും ത്രികൂടാദ്രിമേ-\\
ലുല്ലംഘിതേബ്ധൗ പവനാത്മജന്മാനാ\\
ജനകനരപതിവരമകള്‍ക്കും ദശാസ്യനും\\
ചെമ്മേ വിറച്ചിതു വാമഭാഗംതുലോം,\\
ജനകനരപതിദുഹിതൃവരനു ദക്ഷംഗവും:\\
ജാതനെന്നാകില്‍ വരും സുഖം ദുഃഖവും.\\
തദനു കപികുലപതി കടന്നിതു ലങ്കയില്‍\\
താനതിസൂക്ഷ്മശരീരനായ് രാത്രിയില്‍.\\
ഉദിതരവികിരണരുചിപൂണ്ടൊരു ലങ്കയി-\\
ലൊക്കെത്തിരഞ്ഞാനൊരേടമൊഴിയാതെ.\\
ദശവദന  മണിനിലയമായിരിക്കും മമ\\
ദേവിയിരിപ്പേടമെന്നോര്‍ത്തു മാരുതി\\
കനകമണിനികരവിരചിതം പുരിയിലെങ്ങുമേ\\
കാണാഞ്ഞു ലങ്കാവചനമോര്‍ത്തീടിനാന്‍.\\
ഉടമയൊടു മസുരപുരി കനിവിനൊടു ചൊല്ലിയോ-\\
രുദ്യാനദേശേ തിരഞ്ഞു തുടങ്ങിനാന്‍.\\
ഉപവനവുമമൃതസമസലിലയുത വാപിയു-\\
മുത്തുംഗസൗധങ്ങളും ഗോപുരങ്ങളും\\
സഹജസുതസചിവ ബലപതികള്‍ ഭവനങ്ങളും\\
സൗവര്‍ണ സാലദ്ധ്വജപതാകങ്ങളും\\
ദശവദനമണിഭവനശോഭ കാണും വിധൗ\\
ദിക്പാലമന്ദിരം ധിക്കൃതമായ് വരും.\\
കനകമണിരചിതഭവനങ്ങളിലെങ്ങുമേ\\
കാണാഞ്ഞു പിന്നെയും നീളെ നോക്കുംവിധൗ\\
കുസുമചയസുരഭിയൊടു പവനനതിഗൂഢമായ്\\
കൂടാത്തടഞ്ഞു കൂട്ടിക്കൊണ്ടുപോയുടന്‍\\
ഉപവനവുമുരുതര തരു പ്രവരങ്ങളു-\\
മുന്നതമായുള്ള ശിംശപാവൃക്ഷവും\\
അതിനികടമഖിലജഗദീശ്വരി തന്നെയു-\\
മാശുഗനാശു കാട്ടിക്കൊടുത്തീടിനാന്‍.\\
മലിനതരചികുര വസനം പൂണ്ടു ദീനയായ്\\
മൈഥിലിതാന്‍ കൃശഗാത്രിയായെത്രയും\\
ഭയവിവശമവനിയിലുരുണ്ടും സദാ ഹൃദി\\
ഭര്‍ത്താവുതന്നെ നിനച്ചു നിനച്ചലം\\
നയനജല മനവരതമൊഴുകിയൊഴുകിപ്പതി-\\
നാമത്തെ രാമ രാമേതി ജപിക്കയും\\
നിശിചരികള്‍ നടുവിലഴലൊടു മരുവുമീശ്വരി\\
നിത്യസ്വരൂപിണിയെക്കണ്ടു മാരുതി\\
വിടപിവരശിരസി നിബിഡച്ഛദാന്തര്‍ഗതന്‍\\
വിസ്മയം പൂണ്ടു മറഞ്ഞിരുന്നീടിനാന്‍.\\
ദിവസകരകുലപതി രഘൂത്തമന്‍തന്നുടെ\\
ദേവിയാം സീതയെക്കണ്ടു കപിവരന്‍\\
‘കമലമകളഖില ജഗദീശ്വരി തന്നുടല്‍\\
കണ്ടേന്‍ കൃതാര്‍ത്ഥോസ്മ്യഹം കൃതാര്‍ത്ഥോസ്മ്യഹം!\\
ദിവസകരകുലപതി രഘൂത്തമന്‍ കാര്യവും\\
ദീനതയെന്നിയേ സാധിച്ചിതിന്നു ഞാന്‍.’
\end{verse}

%%06_raavanantepurappadu

\section{രാവണന്റെ പുറപ്പാട്}

\begin{verse}
ഇതിപലവുമകതളിരിലോര്‍ത്തു കപിവര-\\
നിത്തിരിനേരമിരിക്കും ദശാന്തരേ\\
അസുരകുലവരനിലയനത്തിന്‍ പുറത്തുനി-\\
ന്നാശു ചില ഘോഷശബ്ദങ്ങള്‍ കേള്‍ക്കായി\\
കിമിദമിതി സപദി കിസലയചലനിലീനനായ്\\
കീടവദ്ദേഹം മറച്ചു മരുവിനാന്‍.\\
വിബുധകുലരിപു ദശമുഖന്‍ വരവെത്രയും\\
വിസ്മയത്തോടു കണ്ടു കപികുഞ്ജരന്‍.\\
അസുരസുര നിശിചരവരാംഗനാവൃന്ദവു-\\
മത്ഭുതമായുള്ള ശൃംഗാരവേഷവും\\
ദശവദനനനവരതമകതളിരിലുണ്ടു ’തന്‍-\\
ദേഹനാശം ഭവിക്കുന്നതെന്നീശ്വരാ\\
സകലജഗദധിപതി സനാതനന്‍ സന്മയന്‍\\
സാക്ഷാല്‍ മുകുന്ദനെയും കണ്ടു കണ്ടു ഞാന്‍\\
നിശിതതരശരശകലിതാംഗനായ് കേവലേ\\
നിര്‍മലമായ ഭഗവല്‍പദാംബുജേ.\\
വരദനജനമരുമമൃതാനന്ദപൂര്‍ണമാം\\
വൈകുണ്ഠരാജ്യമെനിക്കെന്നു കിട്ടുന്നു?\\
അതിനു ബത! സമയമിദമിതി മനസി കരുതി ഞാ-\\
നംഭോജപുത്രിയെക്കൊണ്ടുപോന്നീടിനേന്‍\\
അതിനുമൊരു പരിഭവമൊടുഴറിവന്നീലവ-\\
നായുര്‍വിനാശകാലം നമുക്കാഗതം.\\
ശിരസി മമ ലിഖിതമിഹ മരണസമയം ദൃഢം\\
ചിന്തിച്ചു കണ്ടാലതിനില്ല ചഞ്ചലം\\
കമലജനുമറിയരുതു കരുതുമളവാര്‍ക്കുമേ\\
കാലസ്വരൂപനാമീശ്വരന്‍ തന്മതം’\\
സതതമകതളിരിലിവകരുതി രഘുനാഥനെ\\
സ്വാത്മനാ ചിന്തിച്ചു ചിന്തിച്ചിരിക്കവേ,\\
കപികള്‍ കുലവരനവിടെയാശു ചൊല്ലും മുമ്പേ\\
കണ്ടിതു രാത്രിയില്‍ സ്വപനം ദശാനനന്‍\\
രഘുജനന തിലകവചനേന രാത്രൗ വരും\\
കശ്ചില്‍ കപിവരന്‍ കാമരൂപാന്വിതന്‍\\
കൃപയൊടൊരു കൃമിസദൃശസൂക്ഷ്മശരീരനായ്\\
കൃല്‍സ്നം പുരവരമന്വിഷ്യ നിശ്ചലം\\
തരുനികരവരശിരസി വന്നിരുന്നാദരാല്‍\\
താര്‍മകള്‍തന്നെയും കണ്ടു രാമോദന്തം\\
അഖിലമവളൊടു ബത! പറഞ്ഞടയാളവു-\\
മാശു കൊടുത്തുടനാശ്വസിപ്പിച്ചു പോം.\\
അതുപൊഴുതിലവനറിവതിന്നു ഞാന്‍ ചെന്നുക-\\
ണ്ടാധിവളര്‍ത്തുവന്‍ വാങ്മയാസ്ത്രങ്ങളാല്‍\\
രഘുപതിയൊടതുമവനശേഷമറിയിച്ചു\\
രാമനുമിങ്ങു കോപിച്ചുടനേ വരും.\\
രണശിരസി സുഖമരണമ്തിനിശിതമായുള്ള\\
രാമശരമേറ്റെനിക്കും വരും ദൃഢം.\\
പരമഗതിവരുവതിനു പരമൊരുപദേശമാം\\
പന്ഥാവിതു മമ പാര്‍ക്കയില്ലേതുമേ\\
സുരനിവഹമതബലവശാല്‍ സത്യമായ് വരും\\
സ്വപ്നം ചിലര്‍ക്കു ചിലകാലമൊക്കണം.\\
നിജമനസി പലവുമിതി വിരവൊടു നിരൂപിച്ചു\\
നിശ്ചിത്യ നിര്‍ഗമിച്ചീടിനാന്‍ രാവണന്‍.\\
കനകമണിവലയകടകാംഗദനൂപുര-\\
കാഞ്ചീമുഖാഭരണാരവമന്തികേ\\
വിവശതര ഹൃദയമൊടു കേട്ടുനോക്കും വിധൗ\\
മിസ്മയമാമ്മാറു കണ്ടു പുരോഭുവി-\\
വിബുധരിപു നിശിചരകുലാധിപന്‍ തന്‍ വര-\\
വെത്രയും ഭീതയായ് വന്നിതു സീതയും.\\
ഉരസിജവുമുരുതുടകളാല്‍ മറച്ചാധിപൂ-\\
ണ്ടുത്തമാംഗം താഴ്ത്തി വേപഥുഗാത്രിയായ്\\
നിജരമണനിരുപമശരീരം നിരാകുലം\\
നിര്‍മലം ധ്യാനിച്ചിരിക്കും ദശാന്തരേ\\
ദശവദനനയുഗശരപരവശതയാ സമം\\
ദേവീസമീപേ തൊഴുതിരുന്നീടിനാന്‍.
\end{verse}

%%07_raavananteicchaabhangam

\section{രാവണന്റെ ഇച്ഛാഭംഗം}

\begin{verse}
അനുസരണമധുരരസവചനവിഭവങ്ങളാ-\\
ലാനന്ദരൂപിണിയോടു ചൊല്ലീടിനാന്‍:\\
‘ശൃണു സുമുഖി! തവ ചരണനളിന ദാസോസ്മ്യഹം\\
ശോഭനശീലേ! പ്രസീദ പ്രസീദ മേ.\\
നിഖിലജഗദധിപമസുരേശമാലോക്യ മം\\
നിന്നിലേ നീ മറഞ്ഞെന്തിരുന്നീടുവാന്‍?\\
ത്വരിതമതികുതുകമൊടുമൊന്നു നോക്കീടു മാം\\
ത്വല്‍ഗതമാനസനെന്നറികെന്നെ നീ.\\
ഭവതി, തവ രമണമപി ദശരഥതനൂജനെ\\
പാര്‍ത്താല്‍ ചിലര്‍ക്കു കാണാം ചിലപ്പോഴെടോ.\\
പലസമയമഖിലദിശി നന്നായ്ത്തിരകിലും\\
ഭാഗ്യവതാമപി കണ്ടുകിട്ടാ പരം.\\
സുമുഖി! ദശരഥതനയനാല്‍ നിനക്കേതുമേ\\
സുന്ദരീ! കാര്യമില്ലെന്നു ധരിക്ക നീ\\
ഒരു പൊഴുതുമവനു പുനരൊന്നിലുമാശയി-\\
ല്ലോര്‍ത്താലൊരു ഗുണമില്ലവനോമലേ!\\
സുദൃഢമനവരതമുപഗൂഹനം ചെയ്കിലും\\
സുഭ്രൂ! സുചിരമരികേ വസിക്കിലും\\
തവ ഗുണസമുദയമലിവോടു ഭുജിക്കിലും\\
താല്‍പരിയം നിന്നിലില്ലവനേതുമേ.\\
ശരണമവനൊരുവരുമൊരിക്കലുമില്ലിനി-\\
ശ്ശക്തിവിഹീനന്‍ വരികയുമില്ലല്ലോ\\
കിമപി നഹി ഭവതി കരണീയം ഭവതിയാല്‍\\
കീര്‍ത്തിഹീനന്‍ കൃതഘ്നന്‍ തുലോം നിര്‍മ്മമന്‍\\
മദരഹിതനറിയരുതു കരുതുമളവാര്‍ക്കുമേ\\
മാനഹീനന്‍ പ്രിയേ! പണ്ഡിതമാനവാന്‍\\
നിഖിലവനചരനിവഹമദ്ധ്യസ്ഥിതന്‍ ഭൃശം\\
നിഷ്കിഞ്ചനപ്രിയന്‍ ഭേദഹീനാത്മകന്‍\\
ശ്വപചനുമൊരവനിസുരവരനുമവനൊക്കുമീ\\
ശ്വാക്കളും ഗോക്കളും ഭേദമില്ലേതുമേ\\
ഭവതിയെയുമൊരുശബരതരുണിയെയുമാത്മനാ\\
പാര്‍ത്തുകണ്ടാലവനില്ല ഭേദം പ്രിയേ!\\
ഭവതിയെയുമകതളിരിലവനിഹ മറന്നിതു\\
ഭര്‍ത്താവിനെപ്പാര്‍ത്തിരുന്നതിനിമതി.\\
ത്രയി വിമുഖനവനനിശമതിനു നഹി സംശയം\\
ത്വദ്ദാസദാസോഹമദ്യ ഭജസ്വം മാം\\
കരഗതമൊരമലമണിവരമുടനുപേക്ഷിച്ചു\\
കാചത്തെയെന്തു കാംക്ഷിക്കുന്നിതോമലേ!\\
സുരദനുജദിതിജഭുജഗാപ്സരോഗന്ധര്‍വ-\\
സുന്ദരീവര്‍ഗം പരിചരിക്കും മുദാ,\\
നിയതമതിഭയസഹിതമമിതബഹുമാനേന\\
നീ മല്‍പ്പരിഗ്രഹമായ് മരുവീടുകില്‍.\\
കളയരുതു സമയമിഹ ചെറുതു വെറുതേ മമ\\
കാന്തേ! കളത്രമായ് വാഴ്ക നീ സന്തതം\\
കളമൊഴികള്‍ പലരുമിഹ വിടുപണികള്‍ ചെയ്യുമ-\\
ക്കാലനും പേടിയുണ്ടെന്നെ മനോഹരേ!\\
പുരുഷഗുണമിഹ മനസി കരുതു പുരുഹൂതനാല്‍\\
പൂജ്യനാം പുണ്യപുമാനെന്നറിക മാം\\
സരസമനുസര സദയമയി തവ വശാനുഗം\\
സൗജന്യസൗഭാഗ്യസാരസര്‍വസ്വമേ!\\
സരസിരുഹമുഖി! ചരണകമലപതിതോസ്മ്യഹം\\
സന്തതം പാഹി മാം പാഹി മാം പാഹി മാം.”\\
വിവിധമിതി ദശവദനനനുസരണപൂര്‍വകം\\
വീണു തൊഴുതപേക്ഷിച്ചോരനന്തരം\\
ജനകജയുമവനൊടതിനിടയിലൊരു പുല്‍ക്കൊടി\\
ജാതരോഷം നുള്ളിയിട്ടു ചൊല്ലീടിനാള്‍:\\
സവിതൃകുലതിലകനിലതീവ ഭീത്യാ ഭവാന്‍\\
സന്ന്യാസിയായ് വന്നിരുവരും കാണാതെ\\
സഭയമതിവിനയമൊടു ശുനീവ ഹവിരധ്വരേ\\
സാഹസത്തോടു മാം കട്ടുകൊണ്ടീലയോ?\\
ദശവദന! സുദൃഢമനുചിതമിതു നിനയ്ക്ക നീ\\
തല്‍ഫലം നീതാനനുഭവിക്കും ദൃഢം.\\
ദശരഥജനിശിതശര ദലിതവപുഷാ ഭവാന്‍\\
ദേഹം വിനാ യമലോകം പ്രവേശിക്കും.\\
രഘുജനനതിലകനൊരു മനുജനിതി മാനസേ\\
രാക്ഷസരാജ! നിനക്കു തോറ്റം ബലാല്‍\\
ലവണജലനിധിയെ രഘുകുലതിലകനശ്രമം\\
ലംഘനം ചെയ്യുമതിനില്ല സംശയം\\
ലവസമയമൊടു നിശിത വിശിഖ പരിപാതേന\\
ലങ്കയും ഭസ്മമാക്കീടുമരക്ഷണാല്‍\\
സഹജസുതസചിവ ബലപതികളൊടുകൂടവേ\\
സന്നമാം നിന്നുടെ സൈന്യവും നിര്‍ണയം.\\
അവനവനനിപുണതരനവനിഭരനാശന-\\
നദ്യ ധാതാവപേക്ഷിച്ചതു കാരണം\\
അവതരണമവനിതലമതിലതി ദയാപര-\\
നാശു ചെയ്തീടിനാന്‍ നിന്നെയൊടുക്കുവാന്‍\\
ജനകനൃപവരനു മകളായ് പിറന്നേനഹം\\
ചെമ്മേയതിന്നൊരു കാരണഭൂതയായ്.\\
അറിക തവ മനസി പുനരനി വിരവിനോടു വ-\\
ന്നാശു മാം കൊണ്ടുപോം നിന്നെയും കൊന്നവന്‍.”\\
ഇതി മിഥിലനൃപതിമകള്‍ പരുഷവചനങ്ങള്‍ കേ-\\
ട്ടേറ്റവും ക്രുദ്ധനായോരു ദശാനനന്‍\\
അതി ചപല കരഭുവി കരാളം കരവാള-\\
മാശു ഭൂപുത്രിയെ കൊല്ലുവാനോങ്ങിനാന്‍\\
അതുപൊഴുതിലതികരുണയൊടു മയതനൂജയു-\\
മാത്മഭര്‍ത്താരം പിടിച്ചടക്കീടിനാള്‍:\\
ഒഴികൊഴിക ദശവദന! ശൃണു മമ വചോ ഭവാ-\\
നൊല്ലാതെ കാര്യമോരായ്ക മൂഢപ്രഭോ!\\
ത്യജ മനുജതരുണിയെയൊരുടയവരുമെന്നിയേ\\
ദീനയായ് ദുഃഖിച്ചതീവ കൃശാംഗിയായ്\\
പതിവിരഹപരവശതയൊടുമിഹ പരാലയേ\\
പാര്‍ത്തു പാതിവ്രത്യമാലംബ്യ രാഘവം\\
പകലിരവു നിശിചരികള്‍ പരുഷവചനം കേട്ടു\\
പാരം വശം കെട്ടിരിക്കുന്നിതുമിവള്‍.\\
ദുരിതമിതിലധികമിഹ നഹി നഹി സുദുര്‍മതേ!\\
ദുഷ്കീര്‍ത്തി ചേരുമോ വീരപുംസാം വിഭോ!\\
സുരദനുജദിതിജ ഭുജഗാപ്സരോഗന്ധര്‍വ-\\
സുന്ദരീവര്‍ഗം നിനക്കു വശഗതാം.”\\
ദശമുഖനുമധികജളനാശു മണ്ഡോദരീ-\\
ദാക്ഷിണ്യവാക്കുകള്‍ കേട്ടു സലജ്ജനായ്\\
നിശിചരികളൊടു സദയമവനുമുരചെയ്തിതു:\\
‘നിങ്ങള്‍ പറഞ്ഞു വശത്തു വരുത്തുവിന്‍.\\
ഭയജനകവചനമനുസരണവചനങ്ങളും\\
ഭാവവികാരങ്ങള്‍കൊണ്ടും ബഹുവിധം\\
അവനിമകളകതളിരഴിച്ചെങ്കലാക്കുവി-\\
നമ്പോടു രണ്ടുമാസം പാര്‍പ്പനിന്നിയും.’\\
ഇതി രജനിചരികളൊടു ദശവദനനും പറ-\\
ഞ്ഞീര്‍ഷ്യയോടന്തഃപുരം പുക്കു മേവിനാന്‍.\\
അതികഠിനപരുഷതരവചനശരമേല്‍ക്കയാ-\\
ലാത്മാവു ഭേദിച്ചിരുന്നിതു സീതയും\\
‘അനുചിതമിതലമലമടങ്ങുവിന്‍ നിങ്ങളെ’-\\
ന്നപ്പോള്‍ ത്രിജടയുമാശു ചൊല്ലീടിനാള്‍:\\
ശൃണു വചനമിതു മമ നിശാചരസ്ത്രീകളേ!\\
ശീലാവതിയെ നമസ്കരിച്ചീടുവിന്‍.\\
സുഖരഹിതഹൃദയമൊടുറങ്ങിനേനൊട്ടു ഞാന്‍\\
സ്വപ്നമാഹന്ത! കണ്ടേനിദാനീം ദൃഢം\\
അഖിലജഗദധിപനഭിരാമനാം രാമനു-\\
മൈരാവതോപരി ലക്ഷ്മണവീരനും\\
ശരനികരപരിപതനദഹനകണജാലേന\\
ശങ്കാവിഹീനം ദഹിപ്പിച്ചു ലങ്കയും\\
രണശിരസി ദശമുഖനെ നിഗ്രയിച്ചശ്രമം\\
രാക്ഷസരാജ്യം വൈഭീഷണനും നല്കി\\
മഹിഷിയെയുമഴകിനൊടു മടിയില്‍വെച്ചാദരാല്‍\\
മാനിച്ചുചെന്നയോധ്യാപുരം മേവിനാന്‍.’\\
കുലിശധരരിപു ദശമുഖന്‍ നഗ്നരൂപിയായ്\\
ഗോമയമയ മഹാഹ്രദം തന്നിലേ\\
തിലരസവുമുടല്‍ മുഴുവനലിവിനൊടണിഞ്ഞുടന്‍\\
ധൃത്വാ നളദമാല്യം നിജ മൂര്‍ദ്ധനി\\
നിജസഹജ സചിവസുത സൈന്യസമേടനായ്\\
നിര്‍മഗ്നനായ്ക്കണ്ടു വിസ്മയം തേടിനേന്‍.\\
രജനിചരകുലപതി വിഭീഷണന്‍ ഭക്തനായ്\\
രാമപാദാബ്ജവും സേവിച്ചു മേവിനാന്‍\\
കലുഷതകള്‍ കളവിനിഹ രാക്ഷസസ്ത്രീകളേ!\\
കണ്ടുകൊള്ളാമിതു സത്യമത്രേ ദൃഢം.\\
കരുണയൊടു വയമതിനു കതിപയദിനം മുദാ\\
കാത്തുകൊള്ളേണമിവളെ നിരാമയം.\\
രജനിചരയുവതികളിതി ത്രിജടാവചോ-\\
രീതികേട്ടത്ഭുതഭീതി പൂണ്ടീടിനാര്‍\\
മാനസി പരവശതയൊടുറങ്ങിനാരേവരും\\
മാനസേ ദുഃഖം കലര്‍ന്നു വൈദേഹിയും.
\end{verse}

%%08_hanumalseethaasamvaadam

\section{ഹനുമല്‍സീതാസംവാദം}

\begin{verse}
‘ഉഷസി നിശിചരികളിവരുടലു മമ ഭക്ഷിക്കു-\\
മുറ്റവരായിട്ടൊരുത്തരുമില്ല മേ\\
മരണമിഹ വരുവതിനുമൊരു കഴിവു കണ്ടീല\\
മാനവവീരനുമെന്നെ മറന്നിതു\\
കളവനിഹ വിരവിനൊടു ജീവനമദ്യ ഞാന്‍\\
കാകില്‍സ്ഥനും കരുണാഹീനനെത്രയും.’\\
മനസി മുഹുരിവ പലതുമോര്‍ത്തു സന്താപേന\\
മന്ദമന്ദമെഴുന്നേറ്റു നിന്നാകുലാല്‍\\
തരളതരഹൃദയമൊടു ഭര്‍ത്താരമോര്‍ത്തോര്‍ത്തു\\
താണു കിടന്നൊരു ശിംശപാശാഖയും\\
സഭയപരവശതരളമാലംബ്യ ബാഷ്പവും\\
സന്തതം വാര്‍ത്തു വിലാപം തുടങ്ങിനാള്‍.\\
പവനസുതനിവ പലവുമാലോക്യ മാനസേ\\
പാര്‍ത്തു പതുക്കെ പറഞ്ഞു തുടങ്ങിനാന്‍:\\
‘ജഗദമലനയനവരഗോത്രേ ദശരഥന്‍\\
ജാതനായാനവന്‍തന്നുടെ പുത്രരായ്\\
രതിരമണതുല്യരായ് നാലുപേരുണ്ടിതു\\
രാമഭരത സൗമിത്രി ശത്രുഘ്നന്മാര്‍.\\
രജനിചരകുലനിധനഹേതുഭൂതന്‍ പിതു-\\
രാജ്ഞയാ കാനനം തന്നില്‍ വാണീടിനാന്‍.\\
ജനകനൃപസുതയുമവരജനുമായ് സാദരം\\
ജാനകീദേവിയെ തത്ര ദശാനനന്‍\\
കപടയതിവേഷമായ് കട്ടുകൊണ്ടീടിനാന്‍\\
കാണാഞ്ഞു ദുഖിച്ചു രാമനും തമ്പിയും\\
വിപിനഭുവി വിരവൊടു തിരഞ്ഞു നടക്കുമ്പോള്‍\\
വീണു കിടക്കും ജടായുവിനെക്കണ്ടു.\\
പരമഗതി പുനരവനു നല്‍കിയമ്മാല്യവല്‍\\
പര്‍വതപാര്‍ശ്വേ നടക്കുംവിധൗ തദാ\\
തരണിസുതനൊടു സപദി സഖ്യവും ചെയ്തിതു\\
സത്വരം കൊന്നിതു ശക്രസുതനെയും;\\
തരണിതനയനുമഥ കപീന്ദ്രനായ് വന്നിതു\\
തല്‍പ്രത്യുപകാരമാശു സുഗ്രീവനും\\
കപിവരരെ വിരവിനൊടു നാലുദിക്കിങ്കലും\\
കണ്ടുവരുവാനയച്ചോരനന്തരം\\
പുനരവരിലൊരുവനഹമത്ര വന്നീടിനേന്‍\\
പുണ്യവാനായ സമ്പാതിതന്‍ വാക്കിനാല്‍\\
ജലനിധിയുമൊരു ശതകയോജനാവിസ്തൃതം\\
ചെമ്മേ കുതിച്ചു ചാടിക്കടന്നീടിനേന്‍.\\
രജനിചരപുരിയില്‍ മുഴുവന്‍ തിരഞ്ഞേനഹം\\
രാത്രിയിലത്ര താതാനുഗ്രഹവശാല്‍\\
തരുനികരവരമരിയ ശിംശപാവൃക്ഷവും\\
തന്മൂലദേശേ ഭവതിയേയും മുദാ\\
കനിവിനൊടു കണ്ടു കൃതാര്‍ഥനായേനഹം\\
കാമലാഭാല്‍ കൃതകൃത്യനായീടിനേന്‍.\\
ഭഗവദനുചരരിലഹമഗ്രേസരന്‍ മമ\\
ഭാഗ്യമഹോ! മമ ഭാഗ്യം നമോസ്തുതേ.’\\
പ്ലവഗകുലവരനിതി പറഞ്ഞടങ്ങീടിനാന്‍\\
പിന്നെയിളകാതിരുന്നാനരക്ഷണം\\
‘കിമതി രഘുകുലവരചരിത്രം ക്രമേണ മേ\\
കീര്‍ത്തിച്ചിതാകാശമാര്‍ഗേ മനോഹരം?\\
പവനനൊരുകൃപയൊടു പറഞ്ഞു കേള്‍പ്പിക്കയോ\\
പാപിയാമെന്നുടെ മാനസഭ്രാന്തിയോ?\\
സുചിരതരമൊരുപൊഴുതുറങ്ങാതെ ഞാനിഹ\\
സ്വപ്നമോ കാണ്മാനവകാശമില്ലല്ലോ.\\
സരസതരപതിചരിതമാശു കര്‍ണ്ണാമൃതം\\
സത്യമായ് വന്നിതാവൂ മമ ദൈവമേ!\\
ഒരു പുരുഷനിതു മമ പറഞ്ഞുവെന്നാകില-\\
ത്യുത്തമന്‍ മുമ്പില്‍ മേ കാണായ് വരേണമേ!’\\
ജനകനൃപദുഹിതൃവചനം കേട്ടു മാരുതി\\
ജാതമോദം മന്ദമന്ദമിറങ്ങിനാന്‍.\\
വിനയമൊടുമവനിമകള്‍ ചരണനളിനാന്തികേ\\
വീണു നമസ്കരിച്ചാന്‍ ഭക്തിപൂര്‍വകം\\
തൊഴുതു ചെറുതകലെയവനാശു നിന്നീടിനാന്‍\\
തുഷ്ട്യാ കലപിംഗതുല്യശരീരനായ്.\\
ഇവിടെ നിശിചരപതി വലീമുഖവേഷമാ-\\
യെന്നെ മോഹിപ്പിപ്പതിന്നു വരികയോ?\\
ശിവശിവ! കിമിതി കരുതി മിഥിലനൃപപുത്രിയും\\
ചേതസി ഭീതി കലര്‍ന്നു മരുവിനാള്‍.\\
കുസൃതി ദശമുഖനു പെരുതെന്നു നിരൂപിച്ചു\\
കുമ്പിട്ടിരുന്നതു കണ്ടു കപീന്ദ്രനും:\\
‘ശരണമിഹ ചരണസരസിജമഖിലനായികേ!\\
ശങ്കിക്ക വേണ്ടാ കുറഞ്ഞൊന്നുമെന്നെ നീ.\\
തവ സചിവനഹമിഹ തഥാവിധനല്ലഹോ!\\
ദാസോസ്മി കോസലേന്ദ്രസ്യ രാമസ്യ ഞാന്‍.\\
സുമുഖീ! കപികുലതിലകനായ സൂര്യാത്മജന്‍\\
സുഗ്രീവഭൃത്യന്‍ ജഗല്‍പ്രാണനന്ദനന്‍\\
കപടമൊരുവരൊടുമൊരുപോഴുതുമറിയുന്നീല\\
കര്‍മണാ വാചാ മനസാപി മാതാവേ!’\\
പവനസുതമധുരതരവചനമതു കേട്ടുടന്‍\\
പത്മലയാദേവി ചോദിച്ചിതാദരാല്‍:\\
ഋതമൃജുമൃദുസ്ഫുടവര്‍ണവാക്യം തെളി-\\
ഞ്ഞിങ്ങനെ ചൊല്ലുന്നവര്‍ കുറയും തുലോം.\\
സദയമിഹ വദ മനുജവാനരജാതികള്‍\\
തങ്ങളില്‍ സംഗതി സംഭവിച്ചീടുവാന്‍\\
കലിതരുചി ഗഹനഭൂവി കാരണമെന്തെടോ\\
കാരുണ്യവാരാന്നിധേ! കപികുഞ്ജരാ!\\
തിരുമനസി ഭവതി പെരികെ പ്രേമമുണ്ടെന്ന-\\
തെന്നോടു ചൊന്നതിന്‍ മൂലവും ചൊല്ലു നീ.’\\
“ശൃണു സുമുഖി! നിഖിലമഖിലേശവൃത്താന്തവും\\
ശ്രീരാമദേവനാണേ സത്യമോമലേ!\\
ഭവതി പതിവചനമവലംബ്യ രണ്ടംഗമാ-\\
യാശ്രയാശങ്കലുമാശ്രമത്തിങ്കലും\\
മരുവിനതു പൊഴുതിലൊരു കനകമൃഗമാലോക്യ\\
മാനിനി പിമ്പേ നടന്നു രഘുപതി.\\
നിശിതതര വിശിഖഗണചാപവുമായ് ചെന്നു\\
നീചനാം മാരീചനെക്കൊന്നു രാഘവന്‍\\
ഉടനുടലുമുലയെ മുഹുരുടജഭുവി വന്നപോ-\\
തുണ്ടായ വൃത്താന്തമോ പറയാവതോ?\\
ഉടനവിടെയവിടെയടവിയിലടയെ നോക്കിയു-\\
മൊട്ടു കരഞ്ഞുതിരിഞ്ഞുഴലും വിധൗ\\
ഗഹനഭുവി ഗഗനചരപതി ഗരുഡസന്നിഭന്‍\\
കേണു കിടക്കും ജടായുവിനെക്കണ്ടു\\
അവനുമഥ തവ ചരിതമഖിലമറിയിച്ചള-\\
വാശുകൊടുത്തിതു മുക്തി പക്ഷീന്ദ്രനും\\
പുനരടവികളിലവരജേന സാകം ദ്രുതം\\
പുക്കിതിരഞ്ഞു കബന്ധഗതി നല്കി\\
ശബരി മരുവിന മുനിവരാശ്രമേ ചെന്നുടന്‍\\
ശാന്താത്മകന്‍ മുക്തിയും കൊടുത്തീടിനാന്‍\\
അഥ ശബരിവിമലവചനേന പോന്നൃശ്യമൂ-\\
കാദ്രിപ്രവരപാര്‍ശ്വേ നടക്കും വിധൗ\\
തപനസുതനിരുവരെയുമഴകിനൊടുകണ്ടതി-\\
താല്‍പര്യമുള്‍ക്കൊണ്ടയച്ചിതെന്നെത്തദാ\\
ബത! രവികുലോത്ഭവന്മാരുടെ സന്നിധൗ\\
ബ്രഹ്മചാരീവേഷമാലംബ്യ ചെന്നു ഞാന്‍\\
നൃപതികുലവരഹൃദയമഖിലവുമറിഞ്ഞതി-\\
നിര്‍മലന്മാരെച്ചുമലിലെടുത്തുടന്‍\\
തരണിസുതനികടഭുവി കൊണ്ടുചെന്നീടിനേന്‍\\
സഖ്യം പരസ്പരം ചെയ്യിച്ചിതാശു ഞാന്‍.’\\
ദഹനനെയുമഴകിനൊടു സാക്ഷിയാക്കിക്കൊണ്ടു\\
ദണ്ഡമിരുവര്‍ക്കുമാശു തീര്‍ത്തീടുവാന്‍.\\
തപനസുതഗൃഹിണിയെ ബലാലടക്കിക്കൊണ്ട\\
താരാപതിയെ വധിച്ചു രാഘുവരന്‍\\
ദിവസകരതനയനു കൊടുത്തിതു രാജ്യവും\\
ദേവിയെയാരാഞ്ഞു കാണ്മാന്‍ കപീന്ദ്രനും\\
പ്ലവഗകുലപരിവൃഢരെ നാലുദിക്കിങ്കലും\\
പ്രത്യേകമേകൈകലക്ഷം നിയോഗിച്ചാന്‍\\
അതുപൊഴുതു രഘുപതിയുമലിവൊടരികേ വിളി-\\
ച്ചംഗുലീയം മമ കൈയില്‍ നല്കീടിനാന്‍.\\
‘ഇതു ജനകനൃപതിമകള്‍ കൈയില്‍ കൊടുക്ക നീ-\\
യെന്നുടെ നാമാക്ഷരാന്വിതം’, പിന്നെയും\\
സപദി തവ മനസി ഗുരുവിശ്വാസസിദ്ധയേ\\
സാദരം ചൊന്നാനടയാളവാക്യവും.\\
അതു ഭവതി കരതളിരിലിനി വിരവില്‍ നല്കുവ-\\
നാലോകയാലോകയാനന്ദപൂര്‍വകം.’\\
ഇതി മധുരതരമനിലതനയനുരചെയ്തുട-\\
നിന്ദിരാദേവിതന്‍ കൈയില്‍ നല്കീടിനാന്‍\\
പുനരധികവിനയമൊടു തൊഴുതുതൊഴുതാദരാല്‍\\
പിന്നോക്കില്‍ വാങ്ങി വണങ്ങി നിന്നീടിനാന്‍.\\
മിഥിലനൃപസുതയുമതു കണ്ടതിപ്രീതയായ്\\
മേന്മേലൊഴുകുമാനന്ദബാഷ്പാകുലം\\
രമണമിവ നിജശിരസി കനിവിനൊടു ചേര്‍ത്തിതു\\
രാമനാമാങ്കിതമംഗുലീയം മുദാ:\\
‘പ്ലവഗകുലപരിവൃഢ! മഹാമതിമന്‍! ഭവാന്‍\\
പ്രാണദാതാ മമ പ്രീതികാരീ ദൃഢം.\\
ഭഗവതി പരമാത്മനി ശ്രീനിധൗ രാഘവേ\\
ഭക്തനതീവ വിശ്വാസ്യന്‍ ദയാപരന്‍\\
പല ഗുണവുമുടയവരെയൊഴികെ മറ്റാരെയും\\
ഭര്‍ത്താവയയ്ക്കയുമില്ല മത്സന്നിധൗ.\\
മമ സുഖവുമനുദിനമിരിക്കും പ്രകാരവും\\
മല്‍പ്പരിതാപവും കണ്ടുവല്ലോ ഭവാന്‍\\
കമലദലനയനനകതളിരിലിനി മാം പ്രതി\\
കാരുണ്യമുണ്ടാംപരിചറിയിക്ക നീ.\\
രജനിചരവരനശനമാക്കുമെന്നെക്കൊണ്ടു\\
രണ്ടുമാസം കഴിഞ്ഞാലെന്നു നിര്‍ണയം\\
അതിനിടയില്‍ വരുവതിനു വേല ചെയ്തീടു നീ-\\
യത്രനാളും പ്രാണനെദ്ധരിച്ചീടുവന്‍.\\
ത്വരിതമിഹ ദശമുഖനെ നിഗ്രഹിച്ചെന്നുടെ\\
ദുഃഖം കളഞ്ഞു രക്ഷിക്കെന്നു ചൊല്ലു നീ,’\\
അനിലതനയനുമഖിലജനനി വചനങ്ങള്‍ കേ-\\
ട്ടാകുലം തീരുവാനാശു ചൊല്ലീടിനാന്‍:\\
‘അവനിപതിസുതനൊടടിയന്‍ ഭവദ്വാര്‍ത്തക-\\
ളങ്ങുണര്‍ത്തിച്ചുകൂടുന്നതിന്‍ മുന്നമേ\\
അവരജനുമഖിലകപികുലബലവുമായ് മുതിര്‍-\\
ന്നാശു വരുമതിലില്ലൊരു സംശയം\\
സുതസചിവസഹജസഹിതം ദശഗ്രീവനെ\\
സൂര്യാത്മജാലയത്തിന്നയയ്ക്കും ക്ഷണാല്‍.\\
ഭവതിയെയുമതികരുണമഴകിനൊടുവീണ്ടു നിന്‍-\\
ഭര്‍ത്താവയൊദ്ധ്യയ്ക്കെഴുന്നള്ളുമാദരാല്‍.’\\
ഇതി പവനസുതവചനമുടമയോടുകേട്ടപോ-\\
തിന്ദിരാദേവി ചോദിച്ചരുളീടിനാള്‍:\\
ഇഹ വിതതജലനിധിയെ നിഖിലകപിസേനയോ-\\
ടേതൊരുജാതി കടന്നുവരുന്നതും\\
മനുജപരിവൃഢ’നിതി വിചാരിച്ച നേരത്തു\\
മാരുതി മൈഥിലിയോടു ചൊല്ലീടിനാന്‍:\\
‘മനുജപരിവൃഢനെയുമവരജനെയുമമ്പോടു\\
മറ്റുള്ള വാനരസൈന്യത്തെയും ക്ഷണാല്‍\\
മമ ചുമലില്‍ വിരവിനൊടെടുത്തു കടത്തുവന്‍\\
മൈഥിലീ! കിം വിഷാദം വൃഥാ മാനസേ?\\
ലഘുതരമമിതരജനിചരകുലമശേഷേണ\\
ലങ്കയും ഭസ്മമാക്കീടുമനാകുലം.\\
ദ്രുതമതിനു സുതനു! മമ ദേഹ്യനുജ്ഞാമിനി\\
ദ്രോഹം വിനാ ഗമിച്ചീടുവനോമലേ!\\
വിരഹകലുഷിതമനസി രഘുവരനു മാം പ്രതി\\
വിശ്വാസമാശു വന്നീടുവാനായ് മുദാ\\
തരിക സരഭസമൊരടയാളവും വാക്യവും\\
താവകം ചൊല്ലുവാനായരുള്‍ചെയ്യണം.’\\
ഇതി പവനതനയവചനേന വൈദേഹിയു-\\
മിത്തിരിനേരം വിചാരിച്ചു മാനസേ\\
ചികൂരഭരമതില്‍ മരുവുമമല ചൂഡാമണി\\
ചിന്മയി മാരുതികൈയില്‍ നല്കീടിനാള്‍.\\
‘ശൃണു തനയ! പുനരൊരടയാളവാക്യം ഭവാന്‍\\
ശ്രുത്വാ ധരിച്ചു കര്‍ണേ പറഞ്ഞീടു നീ\\
സപദി പുനരതുപൊഴുതു വിശ്വാസമെന്നുടെ\\
ഭര്‍ത്താവിനുണ്ടായ് വരുമെന്നു നിര്‍ണയ്ം.\\
ചിരമമിതസുഖമൊടുരുതപസി ബഹുനിഷ്ഠയാ\\
ചിത്രകൂടാചലത്തിങ്കല്‍ വാഴുംവിധൗ\\
പലലമതു പരിചനൊടുണക്കുവാന്‍ ചിക്കി ഞാന്‍\\
പാര്‍ത്തതും കാത്തിരുന്നീടും ദശാന്തരേ\\
തിരുമുടിയുമഴകിനൊടു മടിയില്‍ മമ വെച്ചുടന്‍\\
തീര്‍ഥപാദന്‍ദ് വിരവോടുറങ്ങീടിനാന്‍\\
അതുപൊഴുതിലതിചപലനായ ശക്രാത്മാജ-\\
നാശു കാകാകൃതി പൂണ്ടു വന്നീടിനാന്‍.\\
പലപൊഴുതു പലലശകലങ്ങള്‍ കൊത്തീടിനാന്‍\\
ഭക്ഷിച്ചു കൊള്ളുവാനെന്നോര്‍ത്തു ഞാന്‍ തദാ\\
പുരുഷതരമുടനുടനെടുത്തെറിഞ്ഞീടിനേന്‍\\
പാഷാണജാലങ്ങള്‍ കൊണ്ടതുകൊണ്ടവന്‍\\
വപുഷി മമ ശിതചരണ നഖരതുണ്ഡങ്ങളാല്‍\\
വായ്പോടു കീറിനാനേറെക്കപിതനായ്\\
പരമപുരുഷനുമുടനുണര്‍ന്നു നോക്കും വിധൗ\\
പാരമൊലിക്കുന്ന ചോരകണ്ടാകുലാല്‍\\
തൃണശകലമതികുപിതനായെടുത്തശ്രമം.\\
ദിവ്യാസ്ത്രമന്ത്രം ജപിച്ചയച്ചീടിനാന്‍.\\
സഭയമവനഖിലദിശി പാഞ്ഞു നടന്നിതു\\
സങ്കടം തീര്‍ത്തു രക്ഷിച്ചുകൊണ്ടീടുവാന്‍.\\
അമരപതികമലജ ഗിരീശമുഖ്യന്മാര്‍ക്കു-\\
മാവതല്ലെന്നയച്ചോരവസ്ഥാന്തരേ\\
രഘുതിലകനടിമലരിലവശമൊടു വീണിതു\\
രക്ഷിച്ചുകൊള്ളേണമെന്നെക്കൃപാനിധേ!\\
അപരമൊരു ശരണമിഹ നഹി നഹി നമോസ്തുതേ\\
ആനന്ദമൂര്‍ത്തേ! ശരണം നമോസ്തുതേ.’\\
ഇതിസഭയമടിമലരില്‍ വീണു കേണീടിനാ-\\
നിന്ദ്രാത്മജനാം ജയന്തനുമന്നേരം\\
സവിതൃകുലതിലകനഥ സസ്മിതം ചൊല്ലിനാന്‍:\\
‘സായകം നിഷ്ഫലമാകയില്ലെന്നുമേ.\\
അതിനു തവ നയനമതിലൊന്നുപോ നിശ്ചയ-\\
മന്തരമില്ല നീ പൊയ്ക്കൊള്‍ക നിര്‍ഭയം.’\\
ഇതി സദയമനുദിവസമെന്നെ രക്ഷിച്ചവ-\\
നിന്നുപേക്ഷിച്ചതഎന്തെന്നുടെ ദുഷ്കൃതം.\\
ഒരു പിഴയുമൊരുപൊഴുതിലവനോടു ചെയ്തീല ഞാ-\\
നോര്‍ത്താലിതെന്നുടെ പാപമേ കാരണം.’\\
വിവിധമിതി ജനകനൃപദുഹിതൃവചനം കേട്ടു\\
വീരനാം മാരുതപുത്രനും ചൊലിനാന്‍:\\
‘ഭവതി പുനരിവിടെ മരുവീടുന്നതേതുമേ\\
ഭര്‍ത്താവറിയായ്കകൊണ്ടു വരാഞ്ഞതും.\\
ഝടിതി വരുമിനി നിശിചരൗഘവും ലങ്കയും\\
ശാഖാമൃഗാവലി ഭസ്മമക്കും ദൃഢം.’\\
പവനസുതവചനമിതി കേട്ടു വൈദേഹിയും\\
പാരിച്ച മോദേന ചോദിച്ചരുളിനാള്‍:\\
‘അധികകൃശതനുരിഹ ഭവാന്‍ കപിവീരരു-\\
മീവണ്ണമുള്ളവരല്ലയോ ചൊല്ലു നീ.\\
നിഖില നിശിചരരചലനിഭവിപുലമൂര്‍ത്തികള്‍\\
നിങ്ങളവരോടെതിര്‍ക്കുന്നതെങ്ങനെ?’\\
പവനജനുമവനിമകള്‍ വചനമതു കേട്ടുടന്‍\\
പര്‍വതതുല്യനായ് നിന്നാനതിദ്രുതം.\\
അഥ മിഥിലനൃപതിസുതയോടു ചൊല്ലീടിനാ-\\
നഞ്ജനാപുത്രന്‍ പ്രഭഞ്ജനനന്ദനന്‍:\\
‘ഇതു കരുതുകകമലരിലിങ്ങനെയുള്ളവ-\\
രിങ്ങിരുപത്തൊന്നുവെള്ളം പടവരും.’\\
പവനസുതമൃദുവചനമിങ്ങനെ കേട്ടുടന്‍\\
പത്മപത്രാക്ഷിയും പാര്‍ത്തുചൊല്ലീടിനാള്‍:\\
‘അതിവിമലനമിതബലനാശരവംശത്തി-\\
നന്തകന്‍ നീയതിനന്തരമില്ലെടോ!\\
രജനി വിരവൊടുകഴിയുമിനിയുഴറുകെങ്കില്‍ നീ\\
രാക്ഷസസ്ത്രീകള്‍ കാണാതെ നിരാകുലം.\\
ജലനിധിയുമതിചപലമിന്നേ കടന്നങ്ങു\\
ചെന്നു രഘുവരനെക്കാണ്‍ക നന്ദന!\\
മമ ചരിതമഖിലമറിയിച്ചു ചൂഡാരത്ന-\\
മാശു തൃക്കൈയില്‍ കൊടുക്ക വരയെ നീ.\\
വിരവിനൊടു വരിക രവിസുതനുമുരുസൈന്യവും\\
വീരപുമാന്മാരിരുവരുമായ് ഭവാന്‍.\\
വഴിയിലൊരു പിഴയുമുപരോധവുമെന്നിയേ\\
വായുസുത! പോക നല്ലവണ്ണം ധ്രുവം.’\\
വിനയഭയകുതുകഭക്തി പ്രമൊദാന്വിതം\\
വീരന്‍ നമസ്കരിച്ചീടിനാനന്തികേ\\
പ്രിയവചനസഹിതനഥ ലോകമാതാവിനെ-\\
പ്പിന്നെയും മൂന്നു വലത്തു വെച്ചീടിനാന്‍.\\
‘വിട തരിക ജനനി! വിടകൊള്‍വാനടിയനു\\
വേഗേന, ഖേദം വിനാ വാഴ്ക സന്തതം.’\\
‘ഭവതു ശുഭമയി തനയ! പഥി തവ നിരന്തരം;\\
ഭര്‍ത്താരമാശു വരുത്തീടുകത്ര നീ\\
സുഖമൊടിഹ ജഗതി സുചിരം ജീവ ജീവ നീ\\
സ്വസ്ത്യസ്തു പുത്ര! തേ സുസ്ഥിരശക്തിയും.’\\
അനിലതനയനുമഖിലജനനിയൊടു സാദര-\\
മാശീര്‍വചനമാദായ പിന്‍വാങ്ങിനാന്‍.
\end{verse}

%%09_lankaamardhanam

\section{ലങ്കാമര്‍ദ്ദനം}

\begin{verse}
ചെറുതകലെയൊരു വിടപിശിഖരവുമമര്‍ന്നവന്‍\\
ചിന്തിച്ചു കണ്ടാന്‍ മനസി ജിതശ്രമം:\\
‘പരപുരിയിലൊരു നൃപതികാര്യാര്‍ഥമായതി-\\
പാടവമുള്ളൊരു ദൂതം നിയോഗിച്ചാല്‍\\
സ്വയമതിനൊരഴിനിലയൊഴിഞ്ഞു സാധിച്ചഥ\\
സ്വസ്വാമികാര്യത്തിനന്തരമെന്നിയേ\\
നിജഹൃദയ ചതുരതയൊടപരമൊരു കാര്യവും\\
നീതിയോടേ ചെയ്തു പോമവനുത്തമന്‍.\\
അതിനു മുഹുരഹമഖിലനിശിചരകുലേശനെ-\\
യന്‍പോടു കണ്ടു പറഞ്ഞു പോയീടണം\\
അതിനു പെരുവഴിയുമിതു സുദൃഢ’മിതി ചിന്ത ചെ-\\
യ്താരാമമൊക്കെപ്പൊടിച്ചു തുടങ്ങിനാന്‍.\\
മിഥിലനൃപമകള്‍ മരുവുമതിവിമലശിംശപാ-\\
വൃക്ഷമൊഴിഞ്ഞുള്ളതൊക്കെത്തകര്‍ത്തവന്‍\\
കുസുമദലഫലസഹിതഗുല്മവല്ലീതരു-\\
ക്കൂട്ടങ്ങള്‍ പൊട്ടിയലറി വീഴും വിധൗ\\
ജനനിവഹ ഭയജനന നാദഭേദങ്ങളും\\
ജംഗമജാതികളായ പതത്രികള്‍\\
അതിഭയമൊടഖില ദിശി ദിവി ഖലു പറന്നുട-\\
നാകാശമൊക്കെപ്പരന്നൊരു ശബ്ദവും\\
രജനിചരപുരി ഝടിതി കീഴ്മേല്‍ മറിച്ചിതു\\
രാമദൂതന്‍ മഹാവീര്യപരാക്രമന്‍\\
ഭയമൊടതുപൊഴുതു നിശിചരികളുമുണര്‍ന്നിതു\\
പാര്‍ത്തനേരം കപിവീരനെക്കാണായി.\\
‘ഇവനമിതബലസഹിതനിടിനിനദമൊച്ചയു-\\
മെന്തൊരു ജന്തുവിതെന്തിനു വന്നതും?\\
സുമുഖി! തവ നികടഭുവി നിന്നു വിശേഷങ്ങള്‍\\
സുന്ദരഗാത്രി! ചൊല്ലീലയോ ചൊല്ലെടോ!\\
മനസി ഭയമധികമിവനെക്കണ്ടു ഞങ്ങള്‍ക്കു\\
മര്‍ക്കടാകാരം ധരിച്ചിരിക്കുന്നതും\\
നിശി തമസി വരുവതിനു കാരണമെന്തു ചൊല്‍\\
നീയറിഞ്ഞീലയോ ചൊല്ലിവനാരെടോ.’\\
‘രജനിചരകുലരചിതമായകളൊക്കവേ\\
രാത്രിഞ്ചരന്മാര്‍ക്കൊഴിഞ്ഞറിയാവതോ?\\
ഭയമിവനെ നികടഭൂവി കണ്ടു മന്മാനസേ\\
പാരം വളരുന്നിതെന്താവതീശ്വരാ?’\\
അവനിമകളവരൊടിതു ചൊന്ന നേരത്തവ-\\
രാശു ലങ്കേശ്വരനോടു ചൊല്ലീടിനാര്‍:\\
‘ഒരു വിപിനചരനമിതബലനചലസന്നിഭ-\\
നുദ്യാനമൊക്കെപ്പൊടിച്ചു കളഞ്ഞിതു.\\
പൊരുവതിനു കരുതിയവനപഗതഭയാകുലം\\
പൊട്ടിച്ചിതു ചൈത്യപ്രാസാദമൊക്കവേ.\\
മുസലധരനനിശമതു കാക്കുന്നവരെയും\\
മുല്പെട്ടു തച്ചുകൊന്നീടിനാനശ്രമം.\\
ഭുവനമതിലൊരുവരെയുമവനു ഭയമില്ലഹോ!\\
പോയീലവനവിടുന്നിനിയും പ്രഭോ!’\\
ദശവദനനിതി രജനിചരികള്‍ വചനം കേട്ടു\\
ദന്ദശൂകോപമക്രോധവിവശനായ്\\
‘ഇവനിവിടെ നിശി തമസി ഭയമൊഴിയെ വന്നവ-\\
നേതുമെളിയവനല്ലെന്നു നിര്‍ണയം.\\
നിശിതശരകുലിശമുസലാദ്യങ്ങള്‍ കൈക്കൊണ്ടു\\
നിങ്ങള്‍ പോകാശു നൂറായിരം വീരന്മാര്‍.’\\
നിശിചരകുലാധിപാജ്ഞാകരന്മാരതി-\\
നിര്‍ഭയം ചെല്ലുന്നതു കണ്ടു മാരുതി\\
ശിഖരികുലമൊടുമവനിമുഴുവനിളകും വണ്ണം\\
സിംഹനാദം ചെയ്തതു കേട്ടു രാക്ഷസര്‍\\
സഭയതരഹൃദയമഥ മോഹിച്ചു വീണിതു\\
സംഭ്രമത്തോടടുത്തീടിനാര്‍ പിന്നെയും.\\
ശിതവിശിഖമുഖ നിഖില ശസ്ത്രജാലങ്ങളെ\\
ശീഘ്രം പ്രയോഗിച്ചനേരം കപീന്ദ്രനും\\
മുഹുരുപരി വിരവിനൊടുയര്‍ന്നു ജിതശ്രമം\\
മുല്‍ഗ്ഗരം കൊണ്ടു താഡിച്ചൊടുക്കീടിനാന്‍.\\
നിയുതനിശിചരനിധനനിശമനദശാന്തരേ\\
നിര്‍ഭയം ക്രുദ്ധിച്ചു നക്തഞ്ചരേന്ദ്രനും\\
അഖിലബലപതിവരരിലൈവരെച്ചെല്ലുകെ-\\
ന്നത്യന്തരോഷാല്‍ നിയോഗിച്ചനന്തരം\\
പരമരണനിപുണനൊടെതിര്‍ത്തു പഞ്ചത്വവും\\
പഞ്ചസേനാധിപന്മാര്‍ക്കും ഭവിച്ചിതു.\\
തദനു ദശവദനനുദിതക്രുധാ ചൊല്ലിനാന്‍:\\
‘തല്‍ ബലമത്ഭുതം മത്ഭയോല്‍ഭൂതിദം\\
പരിഭവമൊടമിതബലസഹിതമപി ചെന്നൊരു\\
പഞ്ചസേനാധിപന്മാര്‍ മരിച്ചീടിനാര്‍.\\
ഇവനെ മമ നികടഭുവി ഝടിതി സഹ ജീവനോ-\\
ടിന്നു ബന്ധിച്ചു കൊണ്ടന്നു വെച്ചീടുവാന്‍\\
മഹിത മതി ബലസഹിതമെഴുവരൊരുമിച്ചുടന്‍\\
മന്ത്രിപുത്രന്മാര്‍ പുറപ്പെടുവിന്‍ ഭൃശം.’\\
ദശവദനവചനനിശമനബലസമന്വിതം\\
ദണ്ഡമുസലഖഡ്ഗേഷുചാപാദികള്‍\\
കഠിനതരമലറി നിജകരമതിലെടുത്തുടന്‍\\
കര്‍ബുരേന്ദ്രന്മാരെടുത്താര്‍, കപീന്ദ്രനും\\
ഭുവനതലമുലയെ മുഹുരലറിമരുവും വിധൗ,\\
ഭൂരിശസ്ത്രം പ്രയോഗിച്ചാരനുക്ഷണം.\\
അനിലജനുമവരെ വിരവോടു കൊന്നീടിനാ-\\
നാശു ലോഹസ്തംഭതാഡനത്താലഹോ!\\
നിജസചിവതനയരെഴുവരുമമിത സൈന്യവും\\
നിര്‍ജരലോകം ഗമിച്ചതു കേള്‍ക്കയാല്‍\\
മനസി ദശമുഖനുമുരുതാപവും ഭീതിയും\\
മാനവും ഖേദവുമ് നാണവും തേടിനാന്‍.\\
‘ഇനിയൊരുവനിവനൊടു ജയിപ്പതിനില്ല മ-\\
റ്റിങ്ങനെ കണ്ടീല മറ്റു ഞാനാരെയും.\\
ഇവരൊരുവരെതിരിടുകിലസുരസുരജാതിക-\\
ളെങ്ങുമേ നില്ക്കുമാറില്ല ജഗത്ത്രയേ.\\
അവര്‍ പലരുമൊരു കപിയൊടേറ്റുമരിച്ചതി-\\
ന്നയ്യോ! സുകൃതം നശിച്ചിതു മാമകം.’\\
പലവുമിതി കരുതിയൊരു പരവശത കൈക്കൊണ്ടു\\
പാരം തളര്‍ന്നൊരു താതനോടാദരാല്‍\\
വിനയമൊടു തൊഴുതിളയമകനുമുരചെയ്തിതു:\\
‘വീരപുംസാമിദം യോഗ്യമല്ലേതുമേ.\\
അലമലമിതറികിലനുചിതമഖിലഭൂഭൃതാ-\\
മാത്മഖേദം ധൈര്യശൗര്യതേജോഹരം.\\
അരിവരനെ നിമിഷമിഹ കൊണ്ടുവരുവ’നെ-\\
ന്നക്ഷകുമാരനും നിര്‍ഗമിച്ചീടിനാന്‍.\\
കപിവരനുമതുപൊഴുതു തോരണമേറിനാന്‍\\
കാണായിതക്ഷകുമാരനെ സന്നിധൗ\\
ശരനികരശകലിതശരീരനായ് വന്നിതു\\
ശാഖാമൃഗാധിപന്‍ താനുമതുനേരം.\\
മുനിവിനൊടു ഗഗനഭുവി നിന്നു താണാശുതന്‍\\
മൂര്‍ദ്ധനി മുല്‍ഗരംകൊണ്ടെറിഞ്ഞീടിനാന്‍.\\
ശമനപുരി വിരവിനൊടു ചെന്നു പുക്കീടിനാന്‍\\
ശക്തനാമക്ഷകുമാരന്‍ മനോഹരന്‍.\\
വിബുധകുലരിപു നിശിചരാധിപന്‍ രാവണന്‍\\
വൃത്താന്തമാഹന്ത! കേട്ടു ദുഃഖാര്‍ത്തനായ്\\
അമരപതിജിതമമിതബലസഹിതമാത്മജ-\\
മാത്മഖേദത്തോടണച്ചു ചൊല്ലീടിനാന്‍:\\
‘പ്രിയതനയ! ശൃണു വചനമിഹ തവ സഹോദരന്‍\\
പ്രേതാധിപാലയം പുക്കതു കേട്ടീലേ?\\
മമ സുതനെ രണശിരസി കൊന്ന കപീന്ദ്രനെ\\
മാര്‍ത്താണ്ഡജാലയത്തിന്നായച്ചീടുവാന്‍\\
ത്വരിതമഹമതുലബലമോടു പോയീടുവന്‍\\
ത്വല്‍ക്കനിഷ്ഠോദകം പിന്നെ നല്കീടുവന്‍.’\\
ഇതി ജനകവചനമലിവോടു കേട്ടാദരാ-\\
ലിന്ദ്രജിത്തും പറഞ്ഞീടിനാന്‍ തല്‍ക്ഷണേ:\\
‘ത്യജ മനസി ജനക! തവ ശോകം മഹാമതേ!\\
തീര്‍ത്തുകൊള്‍വന്‍ ഞാന്‍ പരിഭവമൊക്കവേ.\\
മരണവിരഹിതനവനതിനില്ല സംശയം\\
മറ്റൊരുത്തന്‍ ബലാലത്ര വന്നീടുമോ?\\
ഭയമിവനു മരണകൃതമില്ലെന്നു കാണ്‍കില്‍ ഞാന്‍\\
ബ്രഹ്മാസ്ത്രമെയ്തു ബന്ധിച്ചു കൊണ്ടീടുവന്‍.\\
ഭുവനതലമഖിലമരവിന്ദോത്ഭവാദിയാം\\
പൂര്‍വദേവാരികള്‍ തന്ന വരത്തിനാല്‍\\
വലമഥനമപി യുധി ജയിച്ച നമ്മോടൊരു\\
വാനരന്‍ വന്നെതിരിട്ടതുമത്ഭുതം.\\
അതു കരുതുമളവിലിഹ നാണമാമെത്രയും\\
ഹന്തുമശക്യോപി ഞാനവിളംബിതം.\\
കൃതിഭിരപി നികൃതിഭിരപി ഛത്മനാപി വാ\\
കൃച്ഛ്റേണഞാന്‍ ത്വത്സമീപേ വരുത്തുവന്‍.\\
സപദി വിപദുപഗതമിഹ പ്രമദാകൃതം\\
സമ്പദ്വിനാശകരം പരം നിര്‍ണയം\\
സസുഖമിഹ നിവസ മയി ജീവതി ത്വം വൃഥാ\\
സന്താപമുണ്ടാകരുതു കരുതുമാം.’\\
ഇതി ജനകനൊടു നയഹിതങ്ങള്‍ സൂചിച്ചുട-\\
നിന്ദ്രജിത്തും പുറപ്പെട്ടു സന്നദ്ധനായ്\\
രഥകവച വിശിഖ ധനുരാദികള്‍ കൈക്കൊണ്ടു\\
രാമദൂതം ജേതുമാശു ചെന്നീടിനാന്‍.\\
ഗരുഡനിഭനഥ ഗഗനമുല്‍പ്പതിച്ചീടിനാന്‍\\
ഗര്‍ജനപൂര്‍വകം മാരുതി വീര്യവാന്‍.\\
ബഹുമതിയുമകതളിരില്‍ വന്നു പരസ്പരം\\
ബാഹുബലവീര്യവേഗങ്ങള്‍ കാണ്‍കയാല്‍,\\
പവനസുതശിരസി ശരമഞ്ചുകൊണ്ടെയ്തിതു\\
പാകാരിജിത്തായ പഞ്ചാസ്യവിക്രമന്‍.\\
അഥ സപദി ഹൃദി വിശിഖമെട്ടുകൊണ്ടെയ്തു മ-\\
റ്റാറാറു ബാണം പദങ്ങളിലും തഥാ\\
ശിതവിശിഖമധികതരമൊന്നു വാല്‍മേലെയ്തു\\
സിംഹനാദേന പ്രപഞ്ചം കുലുക്കിനാന്‍\\
തദനു കപികുലതിലകനമ്പുകൊണ്ടാര്‍ത്തനായ്\\
സ്തംഭേന സൂതനെക്കൊന്നിതു സത്വരം.\\
തുരഗയുതരഥവുമഥ ഝടിതി പൊടിയാക്കിനാന്‍\\
ദൂരത്തു ചാടിനാന്‍ മേഘനിനാദനും\\
അപരമൊരു രഥമധികവിതതമുടനേറിവ-\\
ന്നസ്ത്രശസ്ത്രൗഘവരിഷം തുടങ്ങിനാന്‍.\\
രുഷിതമതിദശവദനതനയശരപാതേന\\
രോമങ്ങള്‍ നന്നാലു കീറി കപീന്ദ്രനും\\
അതിനുമൊരു കെടുതിയവനില്ലെന്നു കാണ്‍കയാ-\\
ലംഭോജസംഭവബാണമെയ്തീടിനാന്‍.\\
അനിലജനുമതിനെ ബഹുമതിയൊടുടനാദരി-\\
ച്ചാഹന്ത! മോഹിച്ചു വീണിതു ഭൂതലേ.\\
ദശവദനസുതനനിലതനയനെ നിബന്ധിച്ചു\\
തന്‍പിതാവിന്‍ മുമ്പില്‍വെച്ചു വണങ്ങിനാന്‍\\
പവനജനു മനസിയൊരു പീഡയുണ്ടായീല\\
പണ്ടു ദേവന്മാര്‍ കൊടുത്ത വരത്തിനാല്‍.\\
നളിനദളനേത്രനാം രാമന്‍ തിരുവടി-\\
നാമാമൃതം ജപിച്ചീടും ജനം സദാ\\
അമലഹൃദി മധുമഥന ഭക്തിവിശുദ്ധരാ-\\
യജ്ഞാനകര്‍മകൃതബന്ധനം ക്ഷണാല്‍\\
സുചിരവിരചിതമപി വിമുച്യ ഹരിപദം\\
സുസ്ഥിരം പ്രാപിക്കുമില്ലൊരു സംശയം.\\
രഘുതിലകചരണയുഗമകതളിരില്‍ വെച്ചൊരു\\
രാമദൂതന്നു ബന്ധം ഭവിച്ചീടുമോ?\\
മരണജനിമയവികൃതിബന്ധമില്ലാതോര്‍ക്കു\\
മറ്റുള്ള ബന്ധനംകൊണ്ടെന്തു സങ്കടം?\\
കപടമതികലിതകരചരണ വിവശത്വവും\\
കാട്ടിക്കിടന്നു കൊടുത്തോരനന്തരം\\
പലരുമതി കുതുകമൊടു നിശിചരണഞ്ഞുടന്‍\\
പാശഖണ്ഡേന ബന്ധിച്ചതു കാരണം\\
ബലമിയലുമമരരിപു കെട്ടിക്കിടന്നെഴും\\
ബ്രഹ്മാസ്ത്രബന്ധംനം വേര്‍പെട്ടതപ്പൊഴേ.\\
വ്യഥയുമവനകതളിരിലില്ലയെന്നാകിലും\\
ബദ്ധനെന്നുള്ള ഭാവം കളഞ്ഞീലവന്‍.\\
നിശിചരരെടുത്തുകൊണ്ടാര്‍ത്തുപോകും വിധൗ\\
നിശ്ചലനായ്ക്കിടന്നാന്‍ കാര്യഗൗരവാല്‍.
\end{verse}

%%10_hanumaanraavanasabhayil

\section{ഹനുമാന്‍ രാവണസഭയില്‍}

\begin{verse}
അനിലജനെ നിശിചരകുലാധിപന്‍ മുമ്പില്‍വെ-\\
ച്ചാദിദേയാധിപാരാതി ചൊല്ലീടിനാന്‍:\\
‘അമിത നിശിചരവരരെ രണശിരസി കൊന്നവ-\\
നാശു വിരിഞ്ചാസ്ത്രബദ്ധനായീടിനാന്‍.\\
ജനക! തവ മനസി സചിവന്മാരുമായിനി-\\
ച്ചെമ്മേ വിചാര്യ കാര്യം നീ വിധീയതാം.\\
പ്ലവഗകുലവരനറിക സാമാന്യനല്ലിവന്‍\\
പ്രത്യര്‍ത്ഥിവര്‍ഗത്തിനെല്ലാമൊരന്തകന്‍.’\\
നിജതനയവചനമിതി കേട്ടു ദശാനനന്‍\\
നില്ക്കും പ്രഹസ്തനോടോര്‍ത്തു ചൊല്ലീടിനാന്‍:\\
‘ഇവനിവിടെ വരുവതിനു കാരണമെന്തെന്നു-\\
മെങ്ങുനിന്നത്ര വരുന്നതെന്നുള്ളതും\\
ഉപവനവുമനിശമതു കാക്കുന്നവരെയു-\\
മൂക്കോടു മറ്റുള്ള നിക്തഞ്ചരരെയും\\
ത്വരിതമതിബലമൊടു തകര്‍ത്തുപൊടിച്ചതും\\
തൂമയോടാരുടെ ദൂതനെന്നുള്ളതും\\
ഇവനൊടിനി വിരവിനൊടു ചോദിക്ക നീ’യെന്നു-\\
മിന്ദ്രാരി ചൊന്നതു കേട്ടു പ്രഹസ്തനും\\
പവനസുതനൊടു വിനയനയസഹിതമാദരാല്‍\\
പപ്രച്ഛ, ’നീയാരയച്ചു വന്നൂ കപേ!\\
നൃപസദസി കഥയ മമ സത്യം മഹാമതേ!\\
നിന്നെയയച്ചു വിടുന്നുണ്ടു നിര്‍ണയം\\
ഭയമഖിലമകതളിരില്‍നിന്നു കളഞ്ഞാലും\\
ബ്രഹ്മസഭയ്ക്കൊക്കുമിസ്സഭ പാര്‍ക്ക നീ.\\
അനൃതവചനമുമലമധര്‍മകര്‍മങ്ങളു-\\
മത്ര ലങ്കേശരാജ്യത്തിങ്കലില്ലെടോ.’\\
നിഖില നിശിചരകുലബലാധിപന്‍ ചോദ്യങ്ങള്‍\\
നീതിയോടേ കേട്ടു വായുതനയനും\\
മനസി രഘുകുലവരനെ മുഹുരപി നിരൂപിച്ചു\\
മന്ദഹാസേന മന്ദേതരം ചൊല്ലിനാന്‍:\\
‘സ്ഫുടവചനമതിവിശദമിതി ശൃണു ജളപ്രഭോ!\\
പൂജ്യനാം രാമദൂതന്‍ ഞാനറിക നീ\\
ഭുവനപതി മമ പതി പുരന്ദരപൂജിതന്‍\\
പുണ്യപുരുഷന്‍ പുരുഷോത്തമന്‍ പരന്‍\\
ഭുജകുലപതിശയനനമലനഖിലേശ്വരന്‍\\
പൂര്‍വദേവാരാതി ഭുക്തിമുക്തിപ്രദന്‍\\
പുരമഥനഹൃദയമണിനിലയനനിവാസിയാം\\
ഭൂതേശസേവിതന്‍ ഭൂതപഞ്ചാത്മകന്‍\\
ഭുജഗകുലരിപുമണിരഥധ്വജന്‍ മാധവന്‍\\
ഭൂപതി ഭൂതിവിഭൂഷണസമ്മിതന്‍\\
നിജജനകവചനമതു സത്യമാക്കീടുവാന്‍\\
നിര്‍മലന്‍ കാനനത്തിന്നു പുറപ്പെട്ടു\\
ജനകജയുമവരജനുമായ് മരുവുന്ന നാള്‍\\
ചെന്നു നീ ജാനകിയെക്കട്ടുകൊണ്ടീലേ?\\
തവ മരണമിഹ വരുവതിന്നൊരു കാരണം\\
താമരസോത്ഭവകല്പിതം കേവലം.\\
തദനു ദശരഥതനയനും മതംഗാശ്രമേ\\
താപേന തമ്പിയുമായ് ഗമിച്ചീടിനാന്‍\\
തപനതനയനൊടനലസാക്ഷിയായ് സഖ്യവും\\
താല്‍പര്യമുള്‍ക്കൊണ്ടു ചെയ്തോരനന്തരം\\
അമരപതിസുതനെയൊരു ബാണേന കൊന്നുട-\\
നര്‍ക്കാത്മജന്നു കിഷ്കിന്ധയും നല്കിനാന്‍.\\
അടിമലരിലവനമനമഴകിനൊടുചെയ്തവ-\\
ന്നാധിപത്യം കൊടുത്താധിതീര്‍ത്തീടിനാന്‍.\\
അതിനവനുമവനിതനയാന്വേഷണത്തിനാ-\\
യാശകള്‍ തോറുമേകൈക നൂറായിരം\\
പ്ലവഗകുലപരിവൃഢരെ ലഘുതരമയച്ചതി-\\
ലേകനഹമിഹ വന്നു കണ്ടീടിനേന്‍.\\
മനജവിടപികളെയുടനുടനിഹ തകര്‍ത്തതും\\
വാനരവംശപ്രകൃതിശീലം വിഭോ!\\
ഇകലില്‍ നിശിചരവരരെയൊക്കെ മുടിച്ചതു-\\
മെന്നെ വധിപ്പതിനായ്വന്ന കാരണം.\\
മരണഭയമകതലിരിലില്ലയാതേ ഭുവി\\
മറ്റൊരു ജന്തുക്കളില്ലെന്നു നിര്‍ണയം.\\
ദശവദന! സമരഭുവി ദേഹരക്ഷാര്‍ത്ഥമായ്\\
ത്വല്‍ഭൃത്യവര്‍ഗത്തെ നിഗ്രഹിച്ചേനഹം\\
ദശനിയുത ശതവയസി ജീര്‍ണമെന്നാകിലും\\
ദേഹികള്‍ക്കേറ്റം പ്രിയം ദേഹമോര്‍ക്ക നീ.\\
തവ തനയകരഗളിതവിധിവിശിഖപാശേന\\
തത്ര ഞാന്‍ ബദ്ധനായേനൊരു കാല്‍ക്ഷണം\\
കമലഭവമുഖസുരവരപ്രഭാവേന മേ\\
കായത്തിനേതുമേ പീഡയുണ്ടായ്വരാ.\\
പരിഭവവുമൊരു പൊഴുതു മരണവുമകപ്പെടാ\\
ബദ്ധഭാവേന വന്നീടിനേനത്ര ഞാന്‍.\\
അതിനുമിതുപൊഴുതിലൊരു കാരണമുണ്ടുകേ-\\
ളദ്യ ഹിതം തവ വക്തുമുദ്യുക്തനായ്.\\
അകതളിരിലറിവുകുറയുന്നവര്‍ക്കേറ്റമു-\\
ള്ളജ്ഞാനമൊക്കെ നീക്കേണം ബുധജനം\\
അതു ജഗതികരുതു കരുണാത്മനാം ധര്‍മമെ-\\
ന്നാത്മോപദേശമജ്ഞാനിനാം മോക്ഷദം...\\
മനസി കരുതുക ഭുവനഗതിയെ വഴിയേ ഭവാന്‍\\
മഗ്നനായീതൊലാ മോഹമഹാംബുധൗ\\
ത്യജ മനസി ദശവദന! രാക്ഷസീംബുദ്ധിയെ\\
ദൈവീംഗതിയെസ്സമാശ്രയിച്ചീടു നീ.\\
അതു ജനനമരണഭയനാശിനി നിര്‍ണയ-\\
മന്യയായുള്ളതു സംസാരകാരിണി.\\
അമൃതഘനവിമലപരമാത്മബോധോചിത-\\
മത്യുത്തമാന്വയോത്ഭൂതനല്ലോ ഭവാന്‍.\\
കളക തവ ഹൃദി സപദി തത്ത്വബോധേന നീ\\
കാമകോപദ്വേഷലോഭമോഹാദികള്‍\\
കമലഭവസുതതനയനന്ദനനാകയാല്‍\\
കര്‍ബ്ബുരഭാവം പരിഗ്രഹിയായ്ക നീ.\\
ദനുജസുരമനുജഖഗ മൃഗ ഭുജഗഭേദേന\\
ദേഹാത്മബുദ്ധിയെ സന്ത്യജിച്ചീടു നീ\\
പ്രകൃതി ഗുണപരവശതയാ ബദ്ധനായ് വരും\\
പ്രാണദേഹങ്ങളാത്മാവല്ലറികെടോ.\\
അമൃതമയനജനമലനദ്വയനവ്യയ-\\
നാനന്ദപൂര്‍ണനേകന്‍ പരന്‍ കേവലന്‍\\
നിരുപമനമേയനവ്യക്തന്‍ നിരാകുലന്‍\\
നിര്‍ഗുണന്‍ നിഷ്കളന്‍ നിര്‍മമന്‍ നിര്‍മലന്‍\\
നിയമ പര നിലയനനന്തനാദ്യന്‍ വിഭു\\
നിത്യം നിരാകാരനാത്മാ പരബ്രഹ്മം.\\
വിധി ഹരിഹരാദികള്‍ക്കും തിരിയാതവന്‍\\
വേദാന്തവേദ്യനവേദ്യനജ്ഞാനിനാം\\
സകലജഗദിദമറിക മായാമയം വിഭോ!\\
സച്ചിന്മയം സത്യബോധം സനാതനം\\
ജഡമഖിലജഗദിദമനിത്യമറിക നീ\\
ജന്മ ജരാമരണാദി ദുഃഖാന്വിതം.\\
അറിവതിനു പണി പരമപുരുഷമറിമായങ്ങ-\\
ളാത്മാനമാത്മനാ കണ്ടു തെളിക നീ\\
പരമഗതി വരുവതിനു പരമൊരുപദേശവും\\
പാര്‍ത്തു കേട്ടീടും ചൊല്ലിത്തരുന്നുണ്ടു ഞാന്‍.\\
അനവരതമകതളിരിലമിതഹരിഭക്തികൊ-\\
ണ്ടാത്മവിശുദ്ധി വരുമെന്നു നിര്‍ണയം\\
അകമലരുമഘമകലുമളവതി വിശുദ്ധമാ-\\
യാശു തത്ത്വജ്ഞാനവുമുദിക്കും ദൃഢം.\\
വിമലതരമനസി ഭഗവത്തത്ത്വമിജ്ഞാന-\\
വിശ്വാസകേവലാനന്ദാനുഭൂതിയാല്‍\\
രജനിചരവനദഹന മന്തോക്ഷരദ്വയം\\
രാമരാമേതി സദൈവ ജപിക്കയും\\
രതിസപതി നിജഹൃദി വിഹായ നിത്യം മുദാ\\
രാമപദദ്ധ്യാനമുള്ളീലുറയ്ക്കയും\\
അറിവു ചെറുതകതളിരിലൊരു പുരുഷനുണ്ടെങ്കി-\\
ലാഹന്ത! വേണ്ടുന്ന, താകയാലാശു നീ\\
ഭജ ഭവഭയാപഹം, ഭക്തലോകപ്രിയം\\
ഭാനുകോടിപ്രഭം വിഷ്ണുപാദാംബുജം\\
മധുമഥന ചരണ സരസിജയുഗളമാശു നീ\\
മൗഢ്യം കളഞ്ഞു ഭജിച്ചു കൊണ്ടീടെടോ!\\
കുസൃതികളുമിനി മനസി കനിവൊടു കളഞ്ഞു വൈ-\\
കുണ്ഠലോകം ഗമിപ്പാന്‍ വഴിനോക്കു നീ.\\
പരധന കളത്ര മോഹേന നിത്യം വൃഥാ\\
പാപമാര്‍ജിച്ചു കീഴ്പോട്ടു വീണീടൊലാ.\\
നളിനദളനയനമഖിലേശ്വരം മാധവം\\
നാരായണം ശരണാഗതവത്സലം\\
പരമപുരുഷം പരമാത്മാനമദ്വയം\\
ഭക്തിവിശ്വാസേന സേവിക്ക സന്തതം.\\
ശരണമിഹ ചരണകമലേ പതിച്ചീടെടോ!\\
ശത്രുഭാവത്തെ ത്യജിച്ചു സന്തുഷ്ടനായ്.\\
കലുഷമനവധി ഝടിതി ചെയ്തിതെന്നാകിലും\\
കാരുണ്യമീവണ്ണമില്ല മറ്റാര്‍ക്കുമേ.\\
രഘുപതിയെ മനസി കരുതുകിലവനു ഭൂതലേ\\
രണ്ടാമതുണ്ടാകയില്ല ജന്മം സഖേ!\\
സനകമുഖമുനികള്‍ വചനങ്ങളിതോര്‍ക്കെടോ!\\
സത്യം മയോക്തം വിരിഞ്ചാദി സമ്മതം.’\\
അമൃതസമവചനമിതി പവനതനയോദിത-\\
മത്യന്തരോഷേണ കേട്ടു ദശാനനന്‍\\
നമനമിരുപതിലുമഥ കനല്‍ ചിതറുമാറുടന്‍\\
നന്നായുരുട്ടി മിഴിച്ചു ചൊല്ലീടിനാന്‍:\\
‘തിലസദൃശമിവനെയിനി വെട്ടിനുറുക്കുവിന്‍\\
ധിക്കാരമിത്ര കണ്ടീല മറ്റാര്‍ക്കുമേ.\\
മമ നികടഭുവി വടിവൊടൊപ്പമിരുന്നു മാം\\
മറ്റൊരു ജന്തുക്കളിങ്ങനെ ചൊല്ലുമോ?\\
ഭയവുമൊരു വിനയവുമിവന്നു കാണ്മാനില്ല\\
പാപിയായോരു ദുഷ്ടാത്മാ ശഠനിവന്‍\\
കഥയ മമ കഥയ മമ രാമനെന്നാരു ചൊല്‍\\
കാനനവാസി സുഗ്രീവനിന്നാരെടോ?\\
അവരെയുമനന്തരം ജാനകി തന്നെയു-\\
മത്യന്തദുഷ്ടനാം നിന്നെയും കൊല്ലുവന്‍.’\\
ദശവദനവചനമിതി കേട്ടു കോപം പൂണ്ടു\\
ദന്തം കടിച്ചു കപീന്ദ്രനും ചൊല്ലിനാന്‍:\\
‘നിനവു തവ മനസി പെരുതെത്രയും നന്നു നീ\\
നിന്നോടെതിരൊരു നൂറുനൂറായിരം\\
രജനിചരകുലപതികളായ് ഞെളിഞ്ഞുള്ളൊരു\\
രാവണന്മാരൊരുമിച്ചെതിര്‍ത്തീടിലും\\
നിയതമിതു മമ ചെറുവിരല്‍ക്കു പോരാ പിന്നെ\\
നീയെന്തു ചെയ്യുന്നതെന്നോടു കശ്മല!’\\
പവനസുതവചനമിതി കേട്ടു ദശാസ്യനും\\
പാര്‍ശ്വസ്ഥിതന്മാരൊടാശു ചൊല്ലീടിനാന്‍:\\
‘ഇവിടെ നിശിചരരൊരുവരായുധപാണിയാ-\\
യില്ലയോ കള്ളനെക്കൊല്ലുവാന്‍ ചൊല്ലുവിന്‍.’\\
അതുപൊഴുതിലൊരുവനവനോടടുത്തീടിനാ-\\
നപ്പോള്‍ വിഭീഷണന്‍ ചൊല്ലിനാന്‍ മെല്ലവേ:\\
‘അരുതരുതു ദുരിതമിതു ദൂതനെക്കൊല്ലുകെ-\\
ന്നാര്‍ക്കടുത്തൂ നൃപന്മാര്‍ക്കു ചൊല്ലീടുവിന്‍?\\
ഇവനെ വയമിവിടെ വിരവോടു കൊന്നീടിനാ-\\
ലെങ്ങനെയങ്ങറിയുന്നിതു രാഘവന്‍?\\
അതിനു പുനരിവനൊരടയാളമുണ്ടാക്കി നാ-\\
മങ്ങയയ്ക്കേണമതല്ലോ നൃപോചിതം.’\\
ഇതി സദസി ദശവദനസഹജവചനേന താ-\\
നെങ്കിലതങ്ങനെ ചെയ്കെന്നു ചൊല്ലിനാന്‍:
\end{verse}

%%11_lankaadahanam

\section{ലങ്കാദഹനം}

\begin{verse}
‘വദനമപി കരചരണവല്ല ശൗര്യാസ്പദം\\
വാനരന്മാര്‍ക്കു വാല്‍മേല്‍ ശൗര്യമാകുന്നു.\\
വയമതിനു ഝടിതി വസനേന വാല്‍ വേഷ്ടിച്ചു\\
വഹ്നികൊളുത്തിപ്പുരിത്തിലെല്ലാടവും\\
രജനിചരപരിവൃഢരെടുത്തു വാദ്യം കൊട്ടി\\
രാത്രിയില്‍ വന്നൊരു കള്ളനെന്നിങ്ങനെ\\
നിഖിലദിശി പലരുമിഹ കേള്‍ക്കുമാറുച്ചത്തില്‍\\
നീളേ വിളിച്ചു പറഞ്ഞു നടത്തുവിന്‍.\\
കുലഹതകനിവനറിക നിസ്തേജനെന്നു തല്‍\\
കൂട്ടത്തില്‍ നിന്നു നീക്കീടും കപികുലം.’\\
ദശവദനവചനമിതുകേട്ടു രക്ഷാകുലം\\
ദീനത കൈവിട്ടു വാതാത്മജന്റെ വാല്‍\\
തിലരസഘൃതാദിസംസിക്ത വസ്ത്രങ്ങളാല്‍\\
തീവ്രം തെരുതെരെച്ചുറ്റും ദശാന്തരേ\\
അതുലബലനചലതരമവിടെ മരുവീടിനാ-\\
നത്യായതസ്ഥൂലമായിതു വാല്‍ തദാ.\\
വസനഗണമഖിലവുമൊടുങ്ങിച്ചമഞ്ഞിതു\\
വാലുമതീവ ശേഷിച്ചിതു പിന്നെയും.\\
നിഖില നിലയന നിഹിത പട്ടാംബരങ്ങളും\\
നീളെത്തിരഞ്ഞു കൊണ്ട്വന്നു ചുറ്റീടിനാര്‍.\\
അതുമുടനൊടുങ്ങി വാല്‍ ശേഷിച്ചു കണ്ടള-\\
വങ്ങുമിങ്ങും ചെന്നു കൊണ്ടുവന്നീടിനാര്‍.\\
തിലജഘൃതസുസ്നേഹസംസിക്തവസ്ത്രങ്ങള്‍\\
ദിവ്യപട്ടാംശുകജാലവും ചുറ്റിനാര്‍.\\
നികൃതി പെരുതിവനു വസനങ്ങളില്ലൊന്നിനി\\
സ്നേഹമുമെല്ലാമൊടുങ്ങീതശേഷവും\\
അലമല്മിതമലനിവനെത്രയും ദിവ്യനി-\\
താര്‍ക്കു തോന്നീ വിനാശത്തിനെന്നാര്‍ ചിലര്‍.\\
‘അനലമിഹവസനമിതിനനലമിനിവാലധി-\\
യ്ക്കാശു കൊളുത്തുവിന്‍ വൈകരുതേതുമേ’\\
പുനരവരുമതു പൊഴുതു തീ കൊളുത്തീടിനാര്‍\\
പുച്ഛാഗ്രദേശേ പുരന്ദരാരാതികള്‍\\
ബലസഹിതമബലമിവ രജ്ജുഖണ്ഡംകൊണ്ടു\\
ബദ്ധ്വാ ദൃഢതരം ധൃത്വാ കപിവരം\\
കിതവമതികളുമിതൊരു കള്ളനെന്നിങ്ങനെ\\
കൃത്വാ രവമരം ഗത്വാ പുരവരം\\
പറകളെയുമുടനുടനറഞ്ഞറഞ്ഞങ്ങനെ\\
പശ്ചിമദ്വാരദേശേ ചെന്നനന്തരം\\
പവനജനുമതികൃശശരീരനായീടിനാന്‍\\
പാശവുമപ്പോള്‍ ശിഥിലമായ് വന്നിതു.\\
ബലമൊടവനതിചപലമചലനിഭഗാത്രനായ്\\
ബന്ധവും വേര്‍പ്പെട്ടു മേല്പോട്ടു പൊങ്ങിനാന്‍\\
ചരമഗിരിഗോപുരാഗ്രേ വായുവേഗേന\\
ചാടിനാന്‍ വാഹകന്മാരെയും കൊന്നവന്‍\\
ഉഡുപതിയൊടുരസുമടവുയരമിയലുന്ന ര-\\
ത്നോത്തുംഗ സൗധാഗ്രമേറി മേവീടിനാന്‍.\\
ഉദവസിതനികരമുടനുടനുപരി വേഗമോ-\\
ടുല്‍പ്ലുത്യ പിന്നെയുമുല്‍പ്ലുത്യ സത്വരം\\
കനകമണിമയനിലയമഖിലമനിലാത്മജന്‍\\
കത്തിച്ചു കത്തിച്ചു വര്‍ദ്ധിച്ചിതഗ്നിയും.\\
പ്രകൃതിചപലതയൊടവനചലമോരോമണി-\\
പ്രാസാദജാലങ്ങള്‍ ചുട്ടു തുടങ്ങിനാന്‍,\\
ഗജതുരഗരഥബലപദാതികള്‍ പംക്തിയും\\
ഗമ്യങ്ങളായുള്ള രമ്യ ഹര്‍മ്യങ്ങളും.\\
അനലശിഖകളുമനിലസുതഹൃദയവും തെളി-\\
ഞ്ഞാഹന്ത! വിഷ്ണുപദം ഗമിച്ചൂ തദാ!\\
വിഭുധപതിയൊടു നിശിചരാലയം വെന്തൊരു\\
വൃത്താന്തമെല്ലാമറിയിച്ചു കൊള്ളുവാന്‍\\
അഹമഹമികാധിയാ പാവകജ്വാലക-\\
ളംബരത്തോളമുയര്‍ന്നു ചെന്നൂ മുദാ.\\
ഭുവനതലഗത വിമല ദിവ്യരത്നങ്ങളാല്‍\\
ഭൂതിപരിപൂര്‍ണമായുള്ള ലങ്കയും\\
പുനരനിലസുതനിതി ദഹിപ്പിച്ചതെങ്കിലും\\
ഭൂതിപരിപൂര്‍ണമായ് വന്നിതത്ഭുതം!\\
ദശവദനസഹജ ഗൃഹമെന്നിയേ മറ്റുള്ള\\
ദേവാരിഗേഹങ്ങള്‍ വെന്തുകൂടി ജവം\\
രഘുകുലപതിപ്രിയഭൃത്യനാം മാരുതി\\
രക്ഷിച്ചുകൊണ്ടാന്‍ വിഭീഷണമന്ദിരം.\\
കനകമണിമയനിലയനികര മതുവെന്തോരോ\\
കാമിനീവര്‍ഗം വിലാപം തുടങ്ങിനാര്‍.\\
ചികുരഭരവസന ചരണാദികള്‍ വെന്താശു\\
ജീവനും വേര്‍പെട്ടു ഭൂമൗ പതിക്കയും\\
ഉടലുരുകിയുരുകിയുടനുഴറിയലറിപ്പാഞ്ഞു-\\
മുന്നതമായ സൗധങ്ങളിലേറിയും\\
ദഹനനുടനവിടെയുമടുത്തു ദഹിപ്പിച്ചു\\
താഴത്തുവീണുപിടഞ്ഞു മരിക്കയും\\
‘മമ തനയ! രമണ! ജനക! പ്രാണനാഥ! ഹാ!\\
മാമകം കര്‍മമയ്യോ! വിധി ദൈവമേ!\\
മരണമുടനുടലുരുകി മുറുകിവരുകെന്നതു\\
മാറ്റുവാനാരുമില്ലയ്യോ! ശിവശിവ!\\
ദുരിതമിതു രജനിചരവരവിരചിതം ദൃഢം\\
മറ്റൊരു കാരണമില്ലിതിനേതുമേ.\\
പരധനവുമമിതപരദാരങ്ങളും ബലാല്‍\\
പാപി ദശാസ്യന്‍ പരിഗ്രഹിച്ചാന്‍ തുലോം.\\
അറികിലനുചിതമതു, മദേന ചെയ്തീടായ്വി-\\
നാരുമതിന്റെ ഫലമിതു നിര്‍ണയം.\\
മനുജതരുണിയെയൊരു മഹാപാപി കാമിച്ചു\\
മറ്റുള്ളവര്‍ക്കുമാപത്തായിതിങ്ങനെ\\
സുകൃതദുരിതങ്ങളും കാര്യമകാര്യവും\\
സൂക്ഷിച്ചുചെയ്തുകൊള്ളേണം ബുധജനം.\\
മദനശരപരവശതയൊടു ചപലനായി വന്‍\\
മാഹാത്മ്യമുള്ള പതിവ്രതമാരെയും\\
കരബലമൊടനുദിനമണഞ്ഞു പിടിച്ചതി-\\
കാമി ചാരിത്രഭംഗം വരുത്തീടിനാന്‍.\\
അവര്‍മനസി മരുവിന തപോമയ പാവക-\\
നദ്യ രാജ്യേ പിടിപെട്ടിതു കേവലം.’\\
നിശിചരികള്‍ ബഹുവിധമൊരോന്നേ പറകയും\\
നില്ക്കും നിലയിലേ വെന്തു മരിക്കയും\\
ശരണമിഹ കിമിതി പലവഴിയുമുടനോടിയും\\
ശാഖികള്‍ വെന്തു മുറിഞ്ഞുടന്‍ വീഴ്കയും\\
രഘുകുലവരേഷ്ടദൂതന്‍ ത്രിയാമാചര-\\
രാജ്യമെഴുനൂറുയോജനയും ക്ഷണാല്‍\\
സരസബഹുവിഭവയുതഭോജനം നല്‍കിനാന്‍\\
സന്തുഷ്ടനായിതു പാവകദേവനും.\\
ലഘുതരമനിലതനയനമൃതനിധി തന്നിലേ\\
ലാംഗൂലവും തച്ചു തീ പൊലിച്ചീടിനാന്‍\\
പവനജനു ദഹനനപി ചുട്ടതില്ലേതുമേ\\
പാവകനിഷ്ടസഖിയാക കാരണം;\\
പതിനിരതയാകിയ ജാനകീദേവിയാല്‍\\
പ്രാര്‍ത്ഥിതനാകയാലും കരുണാവശാല്‍.\\
അവനിതനയാകൃപാവൈഭവമത്ഭുത-\\
മത്യന്തശീതളനായിതു വഹ്നിയും.\\
രജനിചര കുലവിപിന പാവകനാകിയ\\
രാമനാമസ്മൃതികൊണ്ടു മഹാജനം\\
തനയധനദാരമോഹാര്‍ത്തരെന്നാകിലും\\
താപത്രയാനലനെക്കടന്നീടുന്നു.\\
തദഭിമതകാരിയായുള്ള ദൂതന്നുസ-\\
ന്താപം പകൃതാനലേന ഭവിക്കുമോ?\\
ഭവതി യദി മനുജജനനം ഭുവി സാമ്പ്രതം\\
പങ്കജലോചനനെബ്ഭജിച്ചീടുവിന്‍.\\
ഭുവനപതി ഭുജഗപതിശയനഭജനം ഭുവി\\
ഭൂതദൈവാത്മസംഭൂതതാപാപഹം.\\
തദനു കപികുലവരനുമവനിതനയാപദം\\
താണുതൊഴുതു നമസ്കൃത്യ ചൊല്ലിനാന്‍:\\
‘അഹമിനിയുമുഴറി നടകൊള്ളുവനക്കര-\\
യ്ക്കാജ്ഞാപയാശു ഗച്ഛാമി രാമാന്തികം.\\
രഘുവരനുമവരജനുമരുണജനുമായ്ദ്രുത-\\
മാഗമിച്ചീടുമനന്തസേനാസമം\\
മനസി തവ ചെറുതു പരിതാപമുണ്ടാകൊലാ\\
മദ്ഭരം കാര്യമിനിജ്ജനകാത്മജേ!’\\
തൊഴുതമിതവിനയമിതി ചൊന്നവന്‍ തന്നോടു\\
ദുഃഖമുള്‍ക്കൊണ്ടു പറഞ്ഞിതു സീതയും.\\
‘മമ രമണചരിതമുരചെയ്ത നിന്നെക്കണ്ടു\\
മാനസതാപമകന്നിതു മാമകം\\
കഥമിനിയുമഹമിഹവസാമി ശോകേന മല്‍-\\
ക്കാന്തവൃത്താന്തശ്രവണസൗഖ്യം വിനാ?’\\
ജനകനൃപദുഹിതൃഗിരമിങ്ങനെ കേട്ടവന്‍\\
ജാതാനുകമ്പം തൊഴുതു ചൊല്ലീടിനാന്‍:\\
‘കളക ശുചമിനി വിരഹമലമതിനുടന്‍ മമ\\
സ്കന്ധമാരോഹ ക്ഷണേന ഞാന്‍ കൊണ്ടുപോയ്\\
തവ രമണസവിധമുപഗമ്യ യോജിപ്പിച്ചു\\
താപമശേഷമദ്യൈവ തീര്‍ത്തീടുവന്‍.’\\
പവനസുതവചനമിതി കേട്ടു വൈദേഹിയും\\
പാരം പ്രസാദിച്ചു പാര്‍ത്തു ചൊല്ലീടിനാള്‍\\
‘അതിനു തവ കരുതുമളവില്ലൊരു ദണ്ഡമെ-\\
ന്നാത്മനി വന്നിതു വിശ്വാസമദ്യ മേ\\
ശുഭചരിതനതിബലമൊടാശു ദിവ്യാസ്ത്രേണ\\
ശോഷണ ബന്ധനാദ്യൈരപി സാഗരം\\
കപികുലബലേന കടന്നു ജഗത്ത്രയ-\\
കണ്ടകനെക്കൊന്നു കൊണ്ടുപോകാശു മാം.\\
മറിവൊടൊരു നിശി രഹസി കൊണ്ടുപോയാലതു\\
മല്‍ പ്രാണനാഥകീര്‍ത്തിക്കു പോരാ ദൃഢം.\\
രഘുകുലജവരനിവിടെ വന്നു യുദ്ധം ചെയ്തു\\
രാവണനെക്കൊന്നു കൊണ്ടുപോയ്ക്കൊള്ളുവാന്‍\\
അതിരഭസമയി തനയ! വേല ചെയ്തീടു നീ-\\
യത്ര നാളും ധരിച്ചീടുവന്‍ ജീവനെ.’\\
ഇതി സദയമവനൊടരുള്‍ചെയ്തയച്ചീടിനാ-\\
ളിന്ദിരാദേവിയും, പിന്നെ വാതാത്മജന്‍\\
തൊഴുതഖിലജനനിയൊടു യാത്ര വഴങ്ങിച്ചു\\
തൂര്‍ണം മഹാര്‍ണവം കണ്ടു ചാടീടിനാന്‍.
\end{verse}

%%12_hanumaanteprathyaagamanam

\section{ഹനുമാന്റെ പ്രത്യാഗമനം}

\begin{verse}
ത്രിഭുവനവുമുലയെ മുഹുരൊന്നലറീടിനാന്‍\\
തീവ്രനാദം കേട്ടു വാനരസംഘവും\\
‘കരുതുവിനിതൊരു നിനദമാശു കേള്‍ക്കായതും\\
കാര്യമാഹന്ത! സാധിച്ചു വരുന്നിതു\\
പവനസുതനതിനു നഹി സംശയം മാനസേ\\
പാര്‍ത്തു കാണ്‍കൊച്ച കേട്ടാലറിയാമതും.”\\
കപിനിവഹമിതി ബഹുവിധം പറയും വിധൗ\\
കാണായിതദ്രിശിരസി വാതാത്മജം.\\
കപിനിവഹവീരരേ! കണ്ടിതു സീതയെ\\
കാകുല്‍സ്ഥവീരനനുഗ്രഹത്താലഹം.\\
നിശിചരവരാലയമാകിയ ലങ്കയും\\
നിശ്ശേഷമുദ്യാനവും ദഹിപ്പിച്ചിതു.\\
വിബുധകുലവൈരിയാകും ദശഗ്രീവനെ\\
വിസ്മയമാമ്മാറു കണ്ടു പറഞ്ഞിതു.\\
ഝടിതി ദശരഥസുതനൊടിക്കഥ ചൊല്ലുവാന്‍\\
ജാംബവദാദികളേ! നടന്നീടുവിന്‍.’\\
അതുപൊഴുതു പവനതനയനെയുമവരാദരി-\\
ച്ചാലിംഗ്യ ഗാഢമാംചുംബ്യ വാലാഞ്ചലം\\
കുതുകമൊടു കപിനിചയമനിലജനെ മുന്നിട്ടു\\
കൂട്ടമിട്ടാര്‍ത്തുവിളിച്ചു പോയീടിനാര്‍.\\
പ്ലവഗകുലപരിവൃഢരുമുഴറി നടകൊണ്ടുപോയ്\\
പ്രസ്രവണാചലം കണ്ടു മേവീടിനാര്‍.\\
കുസുമദലഫലമധുലതാതരുപൂര്‍ണമാം\\
ഗുല്മ സമാവൃതം സുഗ്രീവപാലിതം.\\
ക്ഷിധിതപരിപീഡിതരായ കപികുലം\\
ക്ഷുദ്വിനാശാര്‍ത്ഥമാര്‍ത്ത്യാ പറഞ്ഞീടിനാര്‍:\\
‘ഫലനികരസഹിതമിഹ മധുരമധുപൂരവും\\
ഭക്ഷിച്ചു ദാഹവും തീര്‍ത്തു നാമൊക്കവേ\\
തരണിസുതസവിധമുപഗമ്യ വൃത്താന്തങ്ങള്‍\\
താമസം കൈവിട്ടുണര്‍ത്തിക്ക സാദരം.\\
അതിനനുവദിച്ചരുളേണ’മെന്നാശപൂ-\\
ണ്ടംഗദനോടപേക്ഷിച്ചോരനന്തരം\\
അതിനവനുമവരൊടുടനാജ്ഞയെച്ചെയ്കയാ-\\
ലാശു മധുവനം പുക്കിതെല്ലാവരും\\
പരിചൊടതിമധുരമധുപാനവും ചെയ്തവര്‍\\
പക്വഫലങ്ങള്‍ ഭക്ഷിക്കും ദശാന്തരേ\\
ദധിമുഖനുമനിശമതു പാലനം ചെയ്വിതു\\
ദാനമാനേന സുഗ്രീവസ്യ ശാസനാല്‍\\
ദധിവദനവചനമൊടു നിയതമതു കാക്കുന്ന\\
ദണ്ഡധരന്മാരടുത്തു തടുക്കയാല്‍\\
പവനസുതമുഖ കപികള്‍ മുഷ്ടിപ്രഹാരേണ\\
പാഞ്ഞാര്‍ ഭയപ്പെട്ടവരുമതിദ്രുതം.\\
ത്വരിതമഥ ദധിമുഖനുമാശു സുഗ്രീവനെ-\\
ത്തൂര്‍ണമാലോക്യ വൃത്താന്തങ്ങള്‍ ചൊല്ലിനാന്‍:\\
‘തവ മധുവനത്തിനു ഭംഗം വരുത്തിയാര്‍\\
താരേയനാദികളായ കപിബലം.\\
സുചിരമതു തവ കരുണയാ പരിപാലിച്ചു\\
സുസ്ഥിരമാധിപത്യേന വാണേനഹം.\\
വലമഥനസുതതനയനാദികളൊക്കവേ\\
വന്നുമല്‍ ഭൃത്യജനത്തെയും വെന്നുടന്‍\\
മധുവനവുമിതുപൊഴുതഴ്ജിച്ചിതെ’ന്നിങ്ങനെ\\
മാതുലവാക്യമാകര്‍ണ്യ സുഗ്രീവനും\\
നിജമനസി മുഹുരപിവളര്‍ന്ന സന്തോഷേണ\\
നിര്‍മലാത്മാ രാമനോടു ചൊല്ലീടിനാന്‍:\\
‘പവനതനയാദികള്‍ കാര്യവും സാധിച്ചു\\
പാരം തെളിഞ്ഞു വരുന്നിതു നിര്‍ണയം.\\
മധുവനമതല്ലയെന്നാകിലെന്നെബ്ബഹു-\\
മാനിയാതേ ചെന്നു കാണ്‍കയില്ലാരുമേ.\\
അവരെ വിരവൊടു വരുവതിന്നു ചൊല്ലങ്ങുചെ-\\
ന്നാത്മനി ഖേദിക്ക വേണ്ടാ വൃഥാ ഭവാന്‍.’\\
അവനുമതുകേട്ടുഴറിച്ചെന്നു ചൊല്ലിനാ-\\
നഞ്ജനാപുത്രാദികളോടു സാദരം.\\
അനിലതനയാംഗദജാംബവദാദിക-\\
ളഞ്ജസാ സുഗ്രീവഭാഷിതം കേള്‍ക്കയാല്‍\\
പുനരവരുമതുപൊഴുതു വാച്ച അന്തോഷേണ\\
പൂര്‍ണവേഗം നടന്നാശു ചെന്നീടിനാര്‍.\\
പുകള്‍പെരിയ പുരുഷമണി രാമന്‍ തിരുവടി\\
പുണ്യപുരുഷന്‍ പുരുഷോത്തമന്‍ പരന്‍\\
പുരമഥനഹൃദി മരുവുമഖിലജഗദീശ്വരന്‍\\
പുഷ്കരനേത്രന്‍ പുരന്ദരസേവിതന്‍\\
ഭുജഗപതിശയനനമലന്‍ ത്രിജഗല്‍പരി-\\
പൂര്‍ണന്‍ പുരുഹൂതസോദരന്‍ മാധവന്‍\\
ഭുജഗനിവഹാശന വാഹനന്‍ കേശവന്‍\\
പുഷ്കരപുത്രീരമണന്‍ പുരാതനന്‍\\
ഭുജഗകുലഭൂഷണാരാധിതാംഘ്രിദ്വയന്‍\\
പുഷ്കരസംഭവപൂജിതന്‍ നിര്‍ഗുണന്‍\\
ഭുവനമതി മഖപതി സതാംപതി മല്‍പതി\\
പുഷ്കരബാന്ധവപുത്രപ്രിയസഖി\\
ബുധജനഹൃദിസ്ഥിതന്‍ പൂര്‍വദേവാരാതി\\
പുഷ്കരബാന്ധവവംശസമുത്ഭവന്‍\\
ഭുജബലവതാംവരന്‍ പുണ്യജനാന്തകന്‍\\
ഭൂപതിനന്ദനന്‍ ഭൂമിജാവല്ലഭന്‍\\
ഭുവനതലപാലകന്‍ ഭൂതപഞ്ചാത്മകന്‍\\
ഭൂരിഭൂതിപ്രദന്‍ പുണ്യജനാര്‍ച്ചിതന്‍\\
ഭുജഭവകുലാധിപന്‍ പുണ്ഡരീകാനനന്‍\\
പുഷ്പബാണോപമന്‍ ഭൂരികാരുണ്യവാന്‍\\
ദിവസകരപുത്രനും സൗമിത്രിയും മുദാ\\
ദിഷ്ടപൂര്‍ണം ഭജിച്ചന്തികേ സന്തതം\\
വിപിനഭുവി സുഖതരമിരിക്കുന്നതു കണ്ടു\\
വീണു വണങ്ങിനാര്‍ വായുപുത്രാദികള്‍.\\
പുനരഥ ഹരീശ്വരന്‍തന്നെയും വന്ദിച്ചു\\
പൂര്‍ണമോദം പറഞ്ഞാനഞ്ജനാത്മജന്‍:\\
‘കനിവിനൊടു കണ്ടേനഹം ദേവിയെത്തത്ര\\
കര്‍ബുരേന്ദ്രാലയേ സങ്കടമെന്നിയേ.\\
കുശലവുമുടന്‍ വിചാരിച്ചിതു താവകം\\
കൂടെസ്സുമിത്രാതനയനും സാദരം\\
ശിഥിലതരചികുരമൊടശോകവനികയില്‍\\
ശിംശപാമൂലദേശേ വസിച്ചീടിനാള്‍.\\
അനശനമൊടതികൃശശരീരയായന്വഹ-\\
മാശരനാരീപരിവൃതയായ് ശുചാ\\
അഴല്‍പെരുകി മറുകി ബഹുബാഷ്പവും വാര്‍ത്തുവാര്‍-\\
ത്തയ്യോ! സദാ രാമരാമേതി മന്ത്രവും\\
മുഹുരപി ജപിച്ചു ജപിച്ചു വിലാപിച്ചു\\
മുഗ്ദ്ധാംഗി മേവുന്ന നേരത്തു ഞാന്‍ തദാ\\
അതികൃശശരീരനായ് വൃക്ഷശാഖാന്തരേ\\
ആനന്ദമുള്‍ക്കൊണ്ടിരുന്നേനനാകുലം.\\
തവ ചരിതമമൃതസമമഖിലമറിയിച്ചഥ\\
തമ്പിയോടും നിന്തിരുവടിതന്നൊടും\\
ചെറുതുടജഭുവി രഹിതയായ് മരുവും വിധൗ\\
ചെന്നു ദശാനനന്‍ കൊണ്ടങ്ങുപോയതും\\
സവിതൃസുതനൊടൂ ഝടതി സഖ്യമുണ്ടായതും\\
സംക്രന്ദനാത്മജന്‍തന്നെ വധിച്ചതും\\
ക്ഷിതിദുഹിതുരന്വേഷണാര്‍ഥം കപീന്ദ്രനാല്‍\\
കീശൗഘമാശു നിയുക്തമായീടിനാര്‍.\\
അഹമവരിലൊരുവനിവിടേക്കുവന്നീടിനേ-\\
നര്‍ണവം ചാടിക്കടന്നതി വിദ്രുതം\\
രവിതനയസചിവനഹമാശുഗനന്ദനന്‍\\
രാമദൂതന്‍ ഹനുമാനെന്നു നാമവും\\
ഭവതിയെയുമിഹ ഝടതി കൊണ്ടുകൊണ്ടേനഹോ\\
ഭാഗ്യമാഹന്ത! ഭാഗ്യം! കൃതാര്‍ഥോസ്മ്യഹം.\\
ഫലിതമഖിലം മമാദ്യ പ്രയാഅം ദൃശം\\
പത്മജാലോകനം പാപവിനാശനം.\\
മമ വചനമിതി നിഖിലമാകര്‍ണ്യ ജാനകി\\
മന്ദമന്ദം വിചാരിച്ചിതു മാനസേ.\\
‘ശ്രവണയുഗളാമൃതം കേന മേ ശ്രാവിതം?\\
ശ്രീമതാമഗ്രേസരനവന്‍ നിര്‍ണയം.\\
മമ നയനയുഗളപഥമായതു പുണ്യവാന്‍\\
മാനവവീരപ്രസാദേന ദൈവമേ!’\\
വചനമിതിമിഥിലതനയോദിതം കേട്ടു ഞാന്‍\\
വാനരാകാരേണ സൂക്ഷ്മശരീരനായ്\\
വിനയമൊടു തൊഴുതടിയില്‍ വീണു വണങ്ങിനേന്‍\\
വിസ്മയത്തോടു ചോദിച്ചിതും ദേവിയും:\\
‘അറിവതിനു പറക നീയാരെന്നതെന്നോടി’-\\
ത്യാദി വൃത്താന്തം വിചാരിച്ചനന്തരം\\
കഥിതമഖിലം മയാ ദേവവൃത്താന്തങ്ങള്‍\\
കഞ്ജദളാക്ഷിയും വിശ്വസിച്ചീടിനാള്‍.\\
അതുപൊഴുതിലകതളിരിലഴല്‍ കളവതിന്നു ഞാ-\\
നംഗുലീയം കൊടുത്തീടിനേനാദരാല്‍\\
കരതളിരിലതിനെ വിരവൊടു വാങ്ങിത്തദാ\\
കണ്ണുനീര്‍കൊണ്ടു കഴുകിക്കളഞ്ഞുടന്‍\\
ശിരസി ദൃശി ഗളഭുവിയുരഃസ്ഥലത്തിങ്കലും\\
ശീഘ്രമണച്ചു വിലാപിച്ചിതേറ്റവും.\\
‘പവനസുത! കഥയ മമ ദുഃഖമെല്ലാം ഭവാന്‍\\
പദ്മാക്ഷനോടു, നീ കണ്ടിതല്ലോ സഖേ!\\
നിശിചരകളനുദിനമുപദ്രവിക്കുന്നതും\\
നീയങ്ങു ചെന്നു ചൊല്‍കെ’ന്നു ചൊല്ലീടിനാള്‍.\\
‘തവ ചരിതമഖിലമലിവോടുണര്‍ത്തിച്ചു ഞാന്‍\\
തമ്പിയോടും കപിസേനയോടും ദ്രുതം\\
വയമവനിപതിയെ വിരവോടു കൂട്ടിക്കൊണ്ടു\\
വന്നു ദശാസ്യകുലവും മുടിച്ചുടന്‍\\
സകുതുകമയോദ്ധ്യാപുരിക്കാശു കൊണ്ടുപോം\\
സന്താപമുള്ളിലുണ്ടാകരുതേതുമേ.\\
ദശരഥസുതന്നു വിശ്വാസാര്‍ത്ഥമായിനി\\
ദേഹി മേ ദേവി! ചിഹ്നം ധന്യമാദരാല്‍\\
പുനരൊരടയാളവാക്കും പറഞ്ഞീടുക\\
പുണ്യപുരുഷനു വിശ്വാസസിദ്ധയേ.’\\
അതുമവനിസുതയൊടഹമിങ്ങനെ ചൊന്നള-\\
വാശു ചൂഡാരത്നമാദരാല്‍ നല്‍കിനാള്‍.\\
കമലമുഖി കനിവിനൊടു ചിത്രകൂടാചലേ\\
കാന്തനുമായ്വസിക്കുന്നാളൊരു ദിനം\\
കഠിനതരനഖരനികരേണ പീഡിച്ചൊരു\\
കാകവൃത്താന്തവും ചൊല്‍കെന്നു ചൊല്ലിനാള്‍.\\
തദനു പലതരമിവ പറഞ്ഞും കരഞ്ഞുമുള്‍-\\
ത്താപം കലര്‍ന്നു മരുവും ദശാന്തരേ\\
ബഹുവിധ വചോവിഭവേന ദുഃഖം തീര്‍ത്തു\\
ബിംബാധരിയേയുമാശ്വസിപ്പിച്ചു ഞാന്‍\\
വിടയുമുടനഴകൊടു വഴങ്ങിച്ചു പോന്നിതു\\
വേഗേന പിന്നെ മറ്റൊന്നു ചെയ്തേനഹം.\\
അഖിലനിശിചരകുലപതിക്കഭീഷ്ടാസ്പദ-\\
മാരാമമോക്കെത്തകര്‍ത്തേനതിന്നുടന്‍\\
പരിഭവമൊടടല്‍ കരുതി വന്ന നിശാചര-\\
പാപികളെക്കൊല ചെയ്തേനസംഖ്യകം.\\
ദശവദനസുതനെ മുഹുരക്ഷകുമാരനെ\\
ദണ്ഡധരാലയത്തിന്നയച്ചീടിനേന്‍.\\
അഥ ദശമുഖാത്മജബ്രഹ്മാസ്ത്രബദ്ധനാ-\\
യാശരാധീശനെക്കണ്ടു പറഞ്ഞു ഞാന്‍.\\
ലഘുതരമശേഷം ദഹിപ്പിച്ചിതു ബത\\
ലങ്കാപുരം, പിന്നെയും ദേവിതന്‍ പദം\\
വിഗതഭയമടിയിണ വണങ്ങി വാങ്ങിപ്പോന്നു\\
വീണ്ടും സമുദ്രവും ചാടിക്കടന്നു ഞാന്‍\\
തവ ചരണനളിനമധുനൈവ വന്ദിച്ചിതു\\
ദാസന്‍ ദയാനിധേ! പാഹി മാം പാഹി മാം.’\\
ഇതി പവനസുതവചനമാഹന്ത! കേട്ടള-\\
വിന്ദരാകാന്തനും പ്രീതി പൂണ്ടീടിനാന്‍.\\
‘സുരജനസുദുഷ്കരം കാര്യം കൃതം ത്വയാ\\
സുഗ്രീവനും പസാദിച്ചിതു കേവലം.\\
സദയമുപകാരമിച്ചെയ്തതിന്നാദരാല്‍\\
സര്‍വസ്വവും മമ തന്നേന്‍ നിനക്കു ഞാന്‍.\\
പ്രണയമനസാ ഭവാനാല്‍ കൃതമായതിന്‍-\\
പ്രത്യുപകാരം ജഗത്തിങ്കലില്ലെടോ!’\\
പുനരപി രമാവരന്‍ മാരുതപുത്രനെ-\\
പ്പൂര്‍ണമോദം പുണര്‍ന്നീടിനാനാദരാല്‍.\\
ഉരസി മുഹുരപി മുഹരണച്ചു പുല്കീടിനാ-\\
നോര്‍ക്കെടോ! മാരുതപുത്രഭാഗ്യോദയം!\\
ഭുവനതലമതിലൊരുവനിങ്ങനെയില്ലഹോ!\\
പൂര്‍ണപുണ്യൗഘസൗഭാഗ്യമുണ്ടായെടോ!\\
പരമശിവനിതി രഘുകുലാധിപന്തന്നുടെ\\
പാവനമായ കഥയരുള്‍ചെയ്തതു\\
ഭഗവതി ഭവാനി പരമേശ്വരി കേട്ടു\\
ഭക്തിപരവശയായ് വണങ്ങീടിനാള്‍.\\
കിളിമകളുമതിസരസമിങ്ങനെച്ചൊന്നതു\\
കേട്ടു മഹാലോകരും തെളിയേണമേ.
\end{verse}

\begin{center}
ഇത്യദ്ധ്യാത്മ രാമയണേ ഉമാമഹേശ്വരസംവാദേ\\
സുന്ദരകാണ്ഡം സമാപ്തം
\end{center}
