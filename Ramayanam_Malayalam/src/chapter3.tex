%%01_aaranyakaandam
\chapter{ആരണ്യ കാണ്ഡം}

\begin{verse}
ബാലികേ! ശുകകുലമൗലിമാലികേ! ഗുണ-\\
ശാലിനീ! ചാരുശീലേ ചൊല്ലീടു മടിയാതെ\\
നീലനീരദനിഭന്‍ നിര്‍മലന്‍ നിരഞ്ജനന്‍\\
നീലനീരജദളലോചനന്‍ നാരായണന്‍\\
നീലലോഹിതസേവ്യന്‍ നിഷ്കളന്‍ നിത്യന്‍ പരന്‍\\
കാലദേശാനുരൂപന്‍ കാരുണ്യനിലയനന്‍\\
പാലനപരായണന്‍ പരമാത്മാവുതന്റെ\\
ലീലകള്‍ കേട്ടാല്‍ മതിയാകയില്ലൊരിക്കലും.\\
ശ്രീരാമചരിതങ്ങളതിലും വിശേഷിച്ചു\\
സാരമായൊരു മുക്തിസാധനം രസായനം.\\
ഭാരതീഗുണം തവ പരമാമൃതമല്ലോ\\
പാരാതെ പറകെന്നു കേട്ടു പൈങ്കിളി ചൊന്നാള്‍.\\
ഫാലലോചനന്‍ പരമേശ്വരന്‍ പശുപതി\\
ബാലശീതാംശുമൗലി ഭഗവാന്‍ പരാപരന്‍\\
പ്രലേയാചലമകളോടരുള്‍ചെയ്തീടിനാന്‍:\\
ബാലികേ! കേട്ടുകൊള്‍ക പാര്‍വതി ഭക്തപ്രിയേ!\\
രാമനാ പരമാത്മാവാനന്ദരൂപനാത്മാ-\\
രാമനദ്വയനേകനവ്യയനഭിരാമന്‍\\
അത്രിതാപസപ്രവരാശ്രമേ മുനിയുമാ-\\
യെത്രയും സുഖിച്ചു വാണീടിനാനൊരുദിനം.
\end{verse}

%%02_mahaaranyapravesham
\section{മഹാരണ്യപ്രവേശം}

\begin{verse}
പ്രത്യുഷസ്യുത്ഥായ തന്‍ നിത്യകര്‍മവും ചെയ്തു\\
നത്വാ താപസം മഹാപ്രസ്ഥാനമാരംഭിച്ചാന്‍.\\
‘പുണ്ഡരീകോത്ഭവേഷ്ടപുത്ര! ഞങ്ങള്‍ക്കു മുനി-\\
മണ്ഡലമണ്ഡിതമാം ദണ്ഡകാരണ്യത്തിനു\\
ദണ്ഡമെന്നിയേ പോവാനായനുഗ്രഹിക്കേണം\\
പണ്ഡിതശ്രേഷ്ഠ! കരുണാനിധേ! തപോനിധേ!\\
ഞങ്ങളെപ്പെരുവഴി കൂട്ടേണമതിനിപ്പോ-\\
ളിങ്ങുനിന്നയക്കേണം ശിഷ്യരില്‍ ചിലരെയും.’\\
ഇങ്ങനെ രാമവാക്യമത്രിമാമുനി കേട്ടു\\
തിങ്ങീടും കൗതൂഹലംപൂണ്ടുടനരുള്‍ചെയ്തു:\\
‘നേരുള്ള മാര്‍ഗം ഭവാനേവര്‍ക്കും കാട്ടീടുന്നി-\\
താരുള്ളതഹോ തവ നേര്‍വഴി കാട്ടീടുവാന്‍!\\
എങ്കുലും ജഗദനുകാരിയാം നിനക്കൊരു\\
സങ്കടം വേണ്ടാ വഴികാട്ടീടും ശിഷ്യരെല്ലാം.\\
ചെല്ലുവിന്‍ നിങ്ങള്‍ മുമ്പില്‍ നടക്കെ’ന്നവരോടു\\
ചൊല്ലി മാമുനിതാനുമൊട്ടു പിന്നാലെ ചെന്നാന്‍.\\
അന്നെരം തിരിഞ്ഞു നിന്നരുളിച്ചെയ്തു മുനി-\\
തന്നോടു രാമചന്ദ്രന്‍ വന്ദിച്ചു ഭക്തിപൂര്‍വം:\\
‘നിന്തിരുവടി കനിഞ്ഞങ്ങെഴുന്നള്ളീടണ-\\
മന്തികേ ശിഷ്യജനമുണ്ടല്ലോ വഴിക്കുമേ.’\\
എന്നുകേട്ടാശീര്‍വാദം ചെയ്തുടന്‍ മന്ദം മന്ദം\\
ചെന്നുതന്‍ പര്‍ണശാല പുക്കിരുന്നരുളിനാന്‍.\\
പിന്നെയും ക്രോശമാത്രം നടന്നാരവരപ്പോള്‍\\
മുന്നിലാമ്മാറു മഹാവാഹിനി കാണായ് വന്നു.\\
അന്നേരം ശിഷ്യര്‍കളോടരുളിച്ചെയ്തു രാമ-\\
‘നിന്നദി കടപ്പതിനെന്തുപായങ്ങളുള്ളൂ?’\\
എന്നുകേട്ടവര്‍കളും ചൊല്ലിനാ’രെന്തുദണ്ഡം\\
മന്നവ! നല്ല തോണിയുണ്ടെന്നു ധരിച്ചാലും.\\
വേഗേന ഞങ്ങള്‍ കടത്തിടുന്നതുണ്ടുതാന്‍-\\
മാകുലം വേണ്ട ഞങ്ങള്‍ക്കുണ്ടല്ലോ പരിചയം.\\
എങ്കിലോ തോണി കരേറീടാ’മെന്നവര്‍ ചൊന്നാര്‍\\
ശങ്ക കൂടാതെ ശീഘ്രം തോണിയും കടത്തിനാര്‍.\\
ശ്രീരാമന്‍ പ്രസാദിച്ചു താപസകുമാരക-\\
ന്മാരോടു’നിങ്ങള്‍ കടന്നങ്ങു പോകെ’ന്നു ചൊന്നാന്‍.\\
ചെന്നുടനത്രിപാദം വന്ദിച്ചു കുമാരന്മാ-\\
രൊന്നൊഴിയാതെ രാമവൃത്താന്തമറിയിച്ചാര്‍.\\
ശ്രീരാമസീതാസുമിത്രാത്മജന്മാരുമഥ\\
ഘോരമായുള്ള മഹാകാനനമകം പുക്കാര്‍\\
ഝില്ലിഝംകാരനാദമണ്ഡിതം സിംഹവ്യാഘ്ര-\\
ശല്യാദിമൃഗഗണാകീര്‍ണമാതപഹീനം\\
ഘോരരാക്ഷസകുലസേവിതം ഭയാനകം\\
ക്രൂരസര്‍പ്പാദിപൂര്‍ണം കണ്ടു രാഘവന്‍ ചൊന്നാന്‍:\\
‘ലക്ഷ്മണ! നന്നായ് നാലുപുറവും നോക്കിക്കൊള്‍ക\\
ഭക്ഷണാര്‍ഥികളല്ലോ രക്ഷസാം പരിഷകള്‍.\\
വില്ലിനി നന്നായ്ക്കുഴിയെകുലയ്ക്കയും വേണം\\
നല്ലൊരു ശരമൂരിപ്പിടിച്ചുകൊള്‍ക കൈയില്‍.\\
മുന്നില്‍ നീ നടക്കണം വഴിയേ വൈദേഹിയും\\
പിന്നാലെ ഞാനും നടന്നീടുവന്‍ ഗതഭയം.\\
ജീവാത്മാപരമാത്മാക്കള്‍ക്കു മദ്ധ്യസ്ഥയാകും\\
ദേവിയാം മഹാമായാശക്തിയെന്നതുപോലെ\\
ആവയോര്‍മദ്ധ്യേ നടന്നീടുകവേണം സീതാ-\\
ദേവിയുമെന്നാലൊരു ഭീതിയുമുണ്ടായ്വരാ.’\\
ഇത്തരമരുള്‍ചെയ്തു തല്‍പ്രകാരേണ പുരു-\\
ഷോത്തമന്‍ ധനുര്‍ദ്ധാനായ് നടന്നോരു ശേഷം\\
പിന്നിട്ടാനുടനൊരു യോജന വഴിയപ്പോള്‍\\
മുന്നിലാമ്മാറങ്ങൊരു പുഷകരിണിയും കണ്ടാര്‍\\
കല്ഹാരോല്‍പലകുമുദാംബുജരക്തോല്‍പല-\\
ഫുല്ലപുഷ്പേന്ദീവരശോഭിതമച്ഛജലം\\
തോയപാനവും ചെയ്തു വിശ്രാന്തന്മാരായ് വൃക്ഷ-\\
ച്ഛായാഭൂതലേ പുനരിരുന്നു യഥാസുഖം.
\end{verse}

%%03_viraadhavadham
\section{വിരാധവധം}

\begin{verse}
അന്നേരമാശു കാണായ്വന്നിതു വരുന്നത-\\
ത്യുന്നതമായ മഹാസത്വമത്യുഗ്രരവം\\
ഉദ്ധൂതവൃക്ഷം കരാളോജ്ജ്വലദംഷ്ട്രാന്വിത-\\
വക്ത്രഗഹ്വരം ഘോരാകാരമാരുണ്യനേത്രം\\
വാമാംസസ്ഥലന്യസ്തശൂലാഗ്രത്തിങ്കലുണ്ടു\\
ഭീമശാര്‍ദൂലസിംഹമഹിശവരാഹാദി\\
വാരണമൃഗവനഗോചരജന്തുക്കളും\\
പൂരുഷന്മാരും കരഞ്ഞേറ്റവും തുള്ളിത്തുള്ളി\\
പച്ചമാസങ്ങളെല്ലാം ഭക്ഷിച്ചു ഭക്ഷിച്ചും കൊ-\\
ണ്ടുച്ചത്തിലലറി വന്നീടിനാനതും നേരം.\\
ഉത്ഥാനം ചെയ്തു ചാപബാണങ്ങള്‍ കൈക്കൊണ്ടഥ\\
ലക്ഷ്മണന്‍തന്നോടരുള്‍ചെയ്തിതു രാമചന്ദ്രന്‍:\\
‘കണ്ടോ നീ ഭയങ്കരനായൊരു നിശാചര-\\
നുണ്ടു നമ്മുടെ നേരേ വരുന്നു ലഘുതരം.\\
സന്നാഹത്തോടു ബാണം തൊടുത്തു നോക്കിക്കൊണ്ടു\\
നിന്നുകൊള്ളുക ചിത്തമുറച്ചു കുമാര! നീ.\\
വല്ലഭേ! ബാലേ! സീതേ! പേടിയായ്കേതുമെടോ\\
വല്ലജാതിയും പരിപാലിച്ചു കൊള്‍വനല്ലോ.’\\
എന്നരുള്‍ചെയ്തുനിന്നാനേതുമൊന്നിളകാതെ\\
വന്നുടനടുത്തിതു രാക്ഷസപ്രവരനും\\
നിഷ്ഠുരതരമവനെട്ടാശപൊട്ടുംവണ്ണ-\\
മട്ടാഹാസം ചെയ്തിടിവെട്ടീടും നാദംപോലെ\\
ദൃഷ്ടിയില്‍നിന്നു കനല്‍ക്കട്ടകള്‍ വീഴുംവണ്ണം\\
പുഷ്ടകോപേന ലോകം ഞെട്ടുമാറുരചെയ്താന്‍:\\
‘കഷ്ടമാഹന്ത കഷ്ടം നിങ്ങളാരിരുവരും\\
ദുഷ്ടജന്തുക്കളേറ്റമുള്ള വന്‍കാട്ടിലിപ്പോള്‍\\
നില്ക്കുന്നിതസ്തഭയം ചാപതൂണീരബാണ-\\
വല്ക്കല ജടകളും ധരിച്ചു മുനിവേഷം\\
കൈക്കൊണ്ടു മനോഹരിയായൊരു നാരിയോടു-\\
മുള്‍ക്കരുത്തേറുമതിബാലന്മാരല്ലോ നിങ്ങള്‍\\
കിഞ്ചനഭയം വിനാ ഘോരമാം കൊടുങ്കാട്ടില്‍\\
സഞ്ചരിച്ചീടുന്നതുമെന്തൊരുമൂലം ചൊല്‍വിന്‍:’\\
രക്ഷോവാണികള്‍ കേട്ടു തല്‍ക്ഷണമരുള്‍ചെയ്താ-\\
നിക്ഷ്വാകുകുലനാഥന്‍ മന്ദഹാസാനന്തരം:\\
‘രാമനെന്നെനിക്കുപേരെന്നുടെ പത്നിയിവള്‍\\
വാമലോചന സീതാദേവിയെന്നല്ലോ നാമം\\
ലക്ഷ്മണനെന്നു നാമമിവനും മല്‍സോദരന്‍\\
പുക്കിതു വനാന്തരം ജനകനിയോഗത്താല്‍\\
രക്ഷോജാതികളാകുമിങ്ങനെയുള്ളവരെ\\
ശിക്ഷിച്ചു ജഗത്ത്രയം രക്ഷിപ്പാനറിക നീ.’\\
ശ്രുത്വാ രാഘവവാക്യമട്ടാഹാസവും ചെയ്തു\\
വക്ത്രവും പിളര്‍ന്നൊരു സാലവും പറിച്ചോങ്ങി\\
ക്രുദ്ധനാം നിശാചരന്‍ രാഘവനോടു ചൊന്നാന്‍:\\
‘ശ്ക്തനാം വിരാധനെന്നെന്നെ നീ കേട്ടിട്ടില്ലേ?\\
ഇത്രിലോകത്തിലെന്നെയാരറിയാതെയുള്ള-\\
തെത്രയും മൂഢന്‍ ഭവാനെന്നിഹ ധരിച്ചേന്‍ ഞാന്‍.\\
മത്ഭയം നിമിത്തമായ് താപസരെല്ലാമിപ്പോ-\\
ളിപ്രദേശത്തെ വെടിഞ്ഞൊക്കവേ ദൂരെപ്പോയാര്‍.\\
നിങ്ങള്‍ക്കു ജീവിക്കയിലാശയുണ്ടുള്ളിലെങ്കി-\\
ലംഗനാരത്നത്തെയുമായുധങ്ങളും വെടി-\\
ഞ്ഞെങ്ങാനുമോടിപ്പോവിനല്ലായ്കിലെനിക്കൊപ്പോള്‍\\
തിങ്ങീടും വിശപ്പടക്കീടുവന്‍ ഭവാന്മാരാല്‍.’\\
ഇത്തരം പറഞ്ഞവന്‍ മൈഥിലിതന്നെനോക്കി-\\
സ്സത്വരമടുത്തതു കണ്ടു രാഘവനപ്പോള്‍\\
പത്രികള്‍കൊണ്ടുതന്നെ ഹസ്തങ്ങളറുത്തപ്പോള്‍\\
ക്രുദ്ധിച്ചു രാമംപ്രതി വക്ത്രവും പിളര്‍ന്നതി-\\
സത്വരം നക്തഞ്ചരനടുത്താനതു നേര-\\
മസ്ത്രങ്ങള്‍കൊണ്ടു ഖണ്ഡിച്ചീടിനാന്‍ പാദങ്ങളും.\\
ബദ്ധരോഷത്തോടവന്‍ പിന്നെയുമടുത്തപ്പോ-\\
ളുത്തമാംഗവും മുറിച്ചീടിനാനെയ്തു രാമന്‍.\\
രക്തവും പരന്നിതു ഭൂമിയിലതു കണ്ടു\\
ചിത്തകൗതുകത്തോടു പുണര്‍ന്നു വൈദേഹിയും.\\
നൃത്തവും തുടങ്ങിനാരപ്സരസ്ത്രീകളെല്ലാ-\\
മത്യുച്ചം പ്രയോഗിച്ചു ദേവദുന്ദുഭികളും.\\
അന്നേരം വിരാധന്‍ തന്നുള്ളില്‍ നിന്നുണ്ടായൊരു\\
ധന്യരൂപനെക്കാണായ് വന്നിതാകാശമാര്‍ഗേ.\\
സ്വര്‍ണഭൂഷണം പൂണ്ടു സൂര്യസന്നിഭകാന്ത്യാ\\
സുന്ദരശരീരനായ് നിര്‍മലാംബരത്തൊടും\\
രാഘവം പ്രണതാര്‍ത്തിഹാരിണം ഘൃണാകരം\\
രാകേന്ദുമുഖം ഭവഭഞ്ജനം ഭയഹരം\\
ഇന്ദിരാരമണമിന്ദീവരദളശ്യാമ-\\
മിന്ദ്രാദിവൃന്ദാരകവൃന്ദവന്ദിതപദം\\
സുന്ദരം സുകുമാരം സുകൃതിജനമനോ-\\
മന്ദിരം രാമചന്ദ്രം ജഗതാമഭിരാമം\\
വന്ദിച്ചു ദണ്ഡനമസ്കാരവും ചെയ്തു ചിത്താ-\\
നന്ദം പൂണ്ടവന്‍ പിന്നെ സ്തുതിച്ചു തുടങ്ങിനാന്‍:\\
‘ശ്രീരാമ! രാമ! രാമ! ഞാനൊരു വിദ്യാധരന്‍\\
കാരുണ്യമൂര്‍ത്തേ! കമലാപതേ! ധരാപതേ!\\
ദുര്‍വാസാവായ മുനിതന്നുടെ ശാപത്തിനാല്‍\\
ഗര്‍വിതനായോരു രാത്രിഞ്ചരനായേനല്ലോ\\
നിന്തിരുവടിയുടെ മാഹാത്മ്യംകൊണ്ടു ശാപ-\\
ബന്ധവും തീര്‍ന്നു മോക്ഷം പ്രാപിച്ചേനിന്നു നാഥ!\\
സന്തതമിനിച്ചരണാംബുജയുഗം തവ\\
ചിന്തിക്കായ്വരേണമേ മാനസത്തിനു ഭക്ത്യാ.\\
വാണികള്‍കൊണ്ടു നാമകീര്‍ത്തനം ചെയ്യാകേണം\\
പാണികള്‍കൊണ്ടു ചരണാര്‍ച്ചനം ചെയ്യാകേണം\\
ശ്രോത്രങ്ങള്‍കൊണ്ടു കഥാശ്രവണം ചെയ്യാകേണം\\
നേത്രങ്ങള്‍കൊണ്ടു രാമലിംഗങ്ങള്‍ കാണാകേണം\\
ഉത്തമാഗേന നമസ്കരിക്കായ് വന്നീടേണ-\\
മുത്തമഭക്തന്മാര്‍ക്കു ഭൃത്യനായ് വരേണം ഞാന്‍\\
നമസ്തേ ഭഗവതേ ജ്ഞാനമൂര്‍ത്തയേ നമോ\\
നമസ്തേ രാമായാത്മാരാമായ നമോ നമഃ\\
നമ്സ്തേ രാമായ സീതാഭിരാമായ നിത്യം\\
നമസ്തേ രാമായ ലോകാഭിരാമായ നമഃ\\
ദേവലോതത്തിന്നു പോവാനനുഗ്രഹിക്കേണം\\
ദേവദേവേശ! പുനരൊന്നപേക്ഷിച്ചീടുന്നേന്‍.\\
നിന്മഹാമായാദേവിയെന്നെ മോഹിപ്പിച്ചീടാ-\\
യ്കംബുജവിലോചനാ! സന്തതം നമസ്കാരം.’\\
ഇങ്ങനെ വിജ്ഞാപിതനാകിയ രഘുനാഥ-\\
നങ്ങനെത്തന്നെയെന്നു കൊടുത്തു വരങ്ങളും.\\
‘മുക്തനെന്നിയേ കണ്ടുകിട്ടുകയില്ലയെന്നെ\\
ഭക്തിയുണ്ടായാലുടന്‍ മുക്തിയും ലഭിച്ചീടും.’\\
രാമനോടനുജ്ഞയും കൈക്കൊണ്ടു വിദ്യാധരന്‍\\
കാമലാഭേന പോയി താകലോകവും പുക്കാന്‍.\\
ഇക്കഥ ചൊല്ലി സ്തുതിച്ചീടിന പുരുഷനു\\
ദുഷ്കൃതമകന്നു മോക്ഷത്തെയും പ്രാപിച്ചീടാം.
\end{verse}

%%04_sharabhangamandirapravesham
\section{ശരഭംഗമന്ദിരപ്രവേശം}

\begin{verse}
രാമലക്ഷ്മണന്മാരും ജാനകിതാനും പിന്നെ\\
ശ്രീമയമായ ശരഭംഗമന്ദിരം പുക്കാര്‍.\\
സാക്ഷാലീശ്വരനെ മാംസേക്ഷണങ്ങളെക്കൊണ്ടു\\
വീക്ഷ്യ താപസവരന്‍ പൂജിച്ചു ഭക്തിയോടെ.\\
കന്ദപക്വാദികളാലാതിഥ്യം ചെയ്തു ചിത്താ-\\
നന്ദമുള്‍ക്കൊണ്ടു ശരഭംഗനുമരുള്‍ചെയ്തു:\\
‘ഞാനനേകം നാളുണ്ടു പാര്‍ത്തിരിക്കുന്നിതത്ര\\
ജാനകിയോടും നിന്നെക്കാണ്മതിന്നാശയാലേ\\
ആര്‍ജവബുദ്ധ്യാ ചിരം തപസാ ബഹുതര-\\
മാര്‍ജിച്ചേനല്ലോ പുണ്യമിന്നു ഞാനവയെല്ലാം\\
മര്‍ത്ത്യനായ് പിറന്നോരു നിനക്കു തന്നീടിനേ-\\
നദ്യ ഞാന്‍ മോക്ഷത്തിനായുദ്യോഗം പൂണ്ടേനല്ലോ.\\
നിന്നെയും കണ്ടു മമ പുണ്യവു നിങ്കലാക്കി-\\
യെന്നിയേ ദേഹത്യാഗം ചെയ്യരുതെന്നു തന്നെ\\
ചിന്തിച്ചു ബഹുകാലം പാര്‍ത്തു ഞാനിരുന്നിതു\\
ബന്ധവുമറ്റു കൈവല്യത്തെയും പ്രാപിക്കുന്നേന്‍.’\\
യോഗീന്ദ്രനായ ശരഭഗനാം തപോധനന്‍\\
യോഗേശനായ രാമന്‍തന്‍പാദം വണങ്ങിനാന്‍.\\
‘ചിന്തിച്ചീടുന്നേനന്തസ്സന്തതം ചരാചര-\\
ജന്തുക്കളന്തര്‍ഭാഗേ വസന്തം ജഗന്നാഥം\\
ശ്രീരാമം ദൂര്‍വാദളശ്യാമളമംഭോജാക്ഷം\\
ചീരവാസസം ജടാമകുടം ധനുര്‍ദ്ധരം\\
സൗമിത്രിസേവ്യം ജനകാത്മജാസമന്വിതം\\
സൗമുഖ്യമനോഹരം കരുണാരത്നാകരം.’\\
കുണ്ഠഭാവവും നീക്കി സീതയാ രഘുനാഥം\\
കണ്ടുകണ്ടിരിക്കവേ ദേഹവും ദഹിപ്പിച്ചു\\
ലോകേശപാദം പ്രാപിച്ചീടിനാന്‍ തപോധന-\\
ആകാശമാര്‍ഗേ വിമാനങ്ങളും നിറഞ്ഞുതേ.\\
നാകേശാദികള്‍ പുഷ്പവൃഷ്ടിയും ചെയ്തീടിനാന്ര്‍\\
പാകശാസനന്‍ പദാംഭോജവും വണങ്ങിനാന്‍.\\
മൈഥില്യാ സൗമിത്രിണാ താപസഗതികണ്ടു\\
കൗസല്യാതനയനും കൗതുകമുണ്ടായ്വന്നു\\
തത്രൈവ കിഞ്ചില്‍ക്കാലം കഴിഞ്ഞോരനന്തരം\\
വൃത്രാരിമുഖ്യന്മാരുമൊക്കെപ്പോയ് സ്വര്‍ഗം പുക്കാര്‍.
\end{verse}

%%05_munimandalasamaagamam
\section{മുനിമണ്ഡലസമാഗമം}

\begin{verse}
ദണ്ഡകാരണ്യതലവാസികളായ മുനി-\\
മണ്ഡലം ദാശരഥി വന്നതു കേട്ടുകേട്ടു\\
ചണ്ഡദീധിതികുലജാതനാം ജഗന്നാഥന്‍\\
പുണ്ഡരീകാക്ഷന്‍തന്നെക്കാണ്മാനായ് വന്നീടിനാര്‍.\\
രാമലക്ഷ്മണന്മാരും ജാനകീദേവിതാനും\\
മാമുനിമാരെ വീണു നമസ്കാരവും ചെയ്താര്‍.\\
താപസന്മാരുമാശീര്‍വാദം ചെയ്തവര്‍കളോ-\\
ടാഭോഗാനന്ദവിവശന്മാരായരുള്‍ചെയ്താര്‍:\\
“നിന്നുടെ തത്ത്വം ഞങ്ങളിങ്ങറിഞ്ഞിരിക്കുന്നു\\
പന്നഗോത്തമതല്പേ പള്ളികൊള്ളുന്ന ഭവാന്‍\\
ധാതാവര്‍ത്ഥിക്കമൂലം ഭൂഭാരം കളവാനായ്\\
ജാതനായിതു ഭുവി മാര്‍ത്താണ്ഡകുലത്തിങ്കല്‍\\
ലക്ഷ്മണനാകുന്നതു ശേഷനും, സീതാദേവി\\
ലക്ഷ്മിയാകുന്നതല്ലോ, ഭരതശത്രുഘ്നന്മാര്‍\\
ശംഖചക്രങ്ങ,ളഭിഷേകവിഘ്നാദികളും\\
സങ്കടം ഞങ്ങള്‍ക്കു തീര്‍ത്തീടുവാനെന്നു നൂനം.\\
നാനാതാപസകുലസേവിതാശ്രമസ്ഥലം\\
കാനനം കാണ്മാനാശു നീ കൂടെപ്പോന്നീടണം\\
ജാനകിയോടും സുമിത്രാത്മജനോടും കൂടി\\
മാനസേ കാരുണ്യമുണ്ടായ്വരുമല്ലോ കണ്ടാല്‍.”\\
എന്നരുള്‍ചെയ്ത മുനിശ്രേഷ്ഠന്മാരോടുകൂടി\\
ചെന്നവരോരോമുനിപര്‍ണസാലകള്‍കണ്ടാര്‍.\\
അന്നേരം തലയോടുമെല്ലുകളെല്ലാമോരോ\\
കുന്നുകള്‍പോലെ കണ്ടു രാഘവന്‍ ചോദ്യംചെയ്താന്‍:\\
‘മര്‍ത്ത്യമസ്തകങ്ങളുമസ്ഥിക്കൂട്ടവുമെല്ലാ-\\
മത്രൈവമൂലമെന്തോന്നിത്രയുണ്ടാവാനഹോ?’\\
തദ്വാക്യം കേട്ടു ചൊന്നാര്‍ താപസജനം, ’രാമ-\\
ഭദ്ര! നീ കേള്‍ക്ക മുനിസത്തമന്മാരെക്കൊന്നു\\
നിര്‍ദയം രക്ഷോഗണം ഭക്ഷിക്ക നിമിത്തമാ-\\
യിദ്ദേശമസ്ഥിവ്യാപ്തമായ്ച്ചമഞ്ഞിതു നാഥാ!’\\
ശ്രുത്വാ വൃത്താന്തമിത്ഥം കാരുണ്യപരവശ-\\
ചിത്തനായോരു പുരുഷോത്തമനരുള്‍ചെയ്തു:\\
“നിഷ്ഠുരതരമായ ദുഷ്ടരാക്ഷസകുല-\\
മൊട്ടൊഴിയാതെ കൊന്നു നഷ്ടമാക്കീടുവന്‍ ഞാന്‍.\\
ഇഷ്ടാനുരൂപം തപോനിഷ്ഠയാ വസിക്ക സ-\\
ന്തുഷ്ട്യാ താപസകുലമിഷ്ടിയും ചെയ്തു നിത്യം.”\\
സത്യവിക്രമനിതി സത്യവും ചെയ്തു തത്ര\\
നിത്യസംപൂജ്യമാനനാ വനവാസികളാല്‍\\
തത്ര തത്രൈവ മുനിസത്തമാശ്രമങ്ങളില്‍\\
പൃത്ഥ്വീനന്ദിനിയോടുമനുജനോടും കൂടി\\
സത്സംസര്‍ഗാനന്ദേന വസിക്കു കഴിഞ്ഞിതു\\
വത്സരം ത്രയോദശ, മക്കാലം കാണായ് വന്നു
\end{verse}

%%06_sutheekshnaashramapravesham
\section{സുതീക്ഷ്ണാശ്രമപ്രവേശം}

\begin{verse}
വിഖ്യാതമായ സുതീക്ഷ്ണാശ്രമം മനോഹരം\\
മുഖ്യതാപസകുലശിഷ്യസഞ്ചയപൂര്‍ണം\\
സര്‍വര്‍ത്തുഗുണഗണസമ്പന്നമനുപമം\\
സര്‍വകാലാനന്ദദാനോദയപത്യത്ഭുദം\\
സര്‍വപാദപലതാഗുല്മസംകുല സ്ഥലം\\
സര്‍വസല്‍പക്ഷിമൃഗഭുജംഗനിഷേവിതം.\\
രാഘവനവരജന്‍തന്നോടും സീതയോടു-\\
മാഗതനായിതെന്നു കേട്ടൊരു മുനിശ്രേഷ്ഠന്‍\\
കുംഭസംഭവനാകുമഗസ്ത്യശിഷ്യോത്തമന്‍\\
സംപ്രീതന്‍ രാമമന്ത്രോപാസനരതന്‍ മുനി\\
സംഭ്രമത്തോടു ചെന്നു കൂട്ടിക്കൊണ്ടിങ്ങു പോന്നു\\
സംപൂജിച്ചരുളിനാനര്‍ഘ്യപാദ്യാദികളാല്‍\\
ഭക്തിപൂണ്ടശ്രുജലനേത്രനായ് സഗദ്ഗദം\\
ഭക്തവത്സലനായ രാഘവനോടു ചൊന്നാന്‍:\\
‘നിന്തിരുവടിയുടെ നാമമന്ത്രത്തെത്തന്നെ\\
സന്തതം ജപിപ്പു ഞാന്‍ മല്‍ഗുരുനിയോഗത്താല്‍\\
ബ്രഹ്മശങ്കരമുഖവന്ദ്യമാം പാദമല്ലോ\\
നിന്മഹാമായാര്‍ണവം കടപ്പാനൊരു പോതം\\
ആദ്യന്തമില്ലാതൊരു പരമാത്മാവല്ലോ നീ\\
വേദ്യമല്ലൊരുനാളുമാരാലും ഭവത്തത്ത്വം\\
ത്വദ്ഭക്തഭൃത്യഭൃത്യഭൃത്യനായീടണം ഞാന്‍\\
ത്വല്‍പാദാംബുജം നിത്യമുള്‍ക്കാമ്പിലുദിക്കണം.\\
പുത്രഭാര്യാര്‍ഥനിലയാന്ധകൂപത്തില്‍ വീണു\\
ബദ്ധനായ്മുഴുകീടുമെന്നെ നിന്തിരുവടി\\
ഭക്തവാത്സല്യകരുണാകടാക്ഷങ്ങള്‍ തന്നാ-\\
ലുദ്ധരിച്ചീടേണമേ സത്വരം ദയാനിധേ!\\
മൂത്രമാംസാമേദ്ധ്യാന്ത്രപുല്‍ഗലപിണ്ഡമാകും\\
ഗാത്രമോര്‍ത്തോളമതികശ്മല, മതിങ്കലു-\\
ള്ളാസ്ഥയാം മഹാമോഹപാശബന്ധവും ഛേദി-\\
ച്ചാര്‍ത്തിനാശന! ഭവാന്‍ വാഴുകെന്നുള്ളില്‍ നിത്യം.\\
സര്‍വഭൂതങ്ങളുടെയുള്ളില്‍ വീണീടുന്നതും\\
സര്‍വദാ ഭവാന്‍തന്നെ കേവലമെന്നാകിലും\\
ത്വന്മന്ത്രജപരതന്മാരായ ജനങ്ങളെ\\
ത്വന്മഹാമായാദേവി ബന്ധിച്ചീടുകയില്ല.\\
ത്വന്മന്ത്രജപവിമുഖന്മാരാം ജനങ്ങളെ\\
ത്വന്മഹാമായാദേവി ബന്ധിപ്പിച്ചീടുന്നതും.\\
സേവാനുരൂപഫലദാനതല്‍പ്പരന്‍ ഭവാന്‍\\
ദേവപാദപങ്ങളെപ്പോലെ വിശ്വേശ! പോറ്റീ!\\
വിശ്വസംഹാരസൃഷ്ടിസ്ഥിതികള്‍ ചെയ്വാനായി\\
വിശ്വമോഹിനിയായ മായതന്‍ ഗുണങ്ങളാല്‍\\
രുദ്രപങ്കജഭവവിഷ്ണുരൂപങ്ങളായി\\
ചിദ്രൂപനായ ഭവാന്‍ വാഴുന്നു മോഹാത്മനാം.\\
നാനാരൂപങ്ങളായിത്തോന്നുന്നു ലോകത്തിങ്കല്‍\\
ഭാനുമാന്‍ ജലംപ്രതി വെവ്വേറെ കാണുമ്പോലെ.\\
ഇങ്ങനെയുള്ള ഭഗവല്‍സ്വരൂപത്തെ നിത്യ\\
മെങ്ങനെയറിഞ്ഞുപാസിപ്പു ഞാന്‍ ദയാനിധേ!\\
അദ്യൈവ ഭചരണാംബുജയുഗം മമ\\
പ്രത്യക്ഷമായ്വന്നിതു മല്‍ത്തപോബലവശാല്‍\\
ത്വന്മന്ത്രജപവിശുദ്ധാത്മനാം പ്രസാദിക്കും\\
നിര്‍മലനായ ഭവാന്‍ ചിന്മയനെന്നാകിലും\\
സന്മയമായി പരബ്രഹ്മമായരൂപമായ്\\
കര്‍മണാമഗോചരമായൊരു ഭവദ്രൂപം\\
ത്വന്മായാവിഡംബനരചിതം മാനുഷ്യകം\\
മന്മഥകോടികോടി സുഭഗം കമനീയം\\
കാരുണ്യപൂര്‍ണനേത്രം കാര്‍മുകബാണധരം\\
സ്മേരസുന്ദരമുഖമജിനാംബരധരം\\
സീതാസംയുതം സുമിത്രാത്മജനിഷേവിത-\\
പാദപങ്കജം നീലനീരദകളേബരം\\
കോമളമതിശാന്തമനന്തഗുണമഭി-\\
രാമമാത്മാരാമമാനന്ദസമ്പൂര്‍ണാമൃതം\\
പ്രത്യക്ഷമദ്യ മമ നേത്ര ഗോചരമായോ-\\
രിത്തിരുമേനി നിത്യം ചിത്തേ വാഴുകവേണം\\
മുറ്റിടും ഭക്ത്യാ നാമമുച്ചരിക്കായീടണം\\
മറ്റൊരു വരമപേക്ഷിക്കുന്നേനില്ല പോറ്റീ!\\
വന്ദിച്ചു കൂപ്പിസ്തുതിച്ചീടിന മിനിയോടു\\
മന്ദഹാസവുംപൂണ്ടു രാഘവനരുള്‍ചെയ്തു:\\
‘നിത്യവുമുപാസനാശുദ്ധമായിരിപ്പൊരു\\
ചുത്തണ് ഞാനറിഞ്ഞത്രേ കാണ്മാനായ് വന്നൂ മുനേ!\\
സന്തതമെന്നെത്തന്നെ ശരണം പ്രാപിച്ചു മ-\\
ന്മന്ത്രോപാസകന്മാരായ് നിരപേക്ഷന്മാരുമായ്\\
സന്തുഷ്ടന്മാരായുള്ള ഭക്തന്മാര്‍ക്കെന്നെ നിത്യം\\
ചിന്തിച്ചവണ്ണംതന്നെ കാണായ് വന്നീടുമല്ലോ.\\
ത്വല്‍കൃതമേതല്‍ സ്തോത്രം മല്‍പ്രിയം പഠിച്ചീടും\\
സല്‍കൃതിപ്രവരനാം മര്‍ത്ത്യനു വിശേഷിച്ചും\\
സദ്ഭക്തി ഭവിച്ചീടും ബ്രഹ്മജ്ഞാനവുമുണ്ടാ-\\
മല്പവുമതിനില്ല സംശയം നിരൂപിച്ചാല്‍.\\
താപസോത്തമ! ഭവാനെന്നെസ്സേവിക്കമൂലം\\
പ്രാപിക്കുമല്ലോ മമ സായുജ്യം ദേഹനാശേ.\\
ഉണ്ടൊരാഗ്രഹം തവാചാര്യനാമഗസ്ത്യനെ-\\
ക്കണ്ടു വന്ദിച്ചുകൊള്‍വാനെന്തതിനാവതിപ്പോള്‍?\\
തത്രൈവ കിഞ്ചില്‍ കാലം വസ്തുമുണ്ടത്യാഗ്രഹ-\\
മെത്രയുണ്ടടുത്തതുമഗസ്ത്യാശ്രമം മുനേ!’\\
ഇത്ഥം രാമോക്തി കേട്ടു ചൊല്ലിനാന്‍ സുതീക്ഷ്ണനു-\\
‘മസ്തു തേ ഭദ്ര,മതുതോന്നിയതതിന്നു ഞാന്‍\\
കാട്ടുവേനല്ലോ വഴി കൂടെപ്പോന്നടുത്തനാള്‍\\
വാട്ടമെന്നിയേ വസിക്കേണമിന്നിവിടെ നാം.\\
ഒട്ടുനാളുണ്ടു ഞാനും കണ്ടിട്ടെന്‍ ഗുവുവിനെ\\
പുഷ്ടമോദത്തോടൊക്കെത്തക്കപ്പോയ് കാണാമല്ലോ.’\\
ഇത്ഥമാനന്ദംപൂണ്ടു രാത്രിയും കഴിഞ്ഞപ്പോ-\\
ളുത്ഥാനം ചെതു സന്ധ്യാവന്ദനം കൃത്വാ ശീഘ്രം\\
പ്രീതനാം മുനിയോടും ജാനകീദേവിയോടും\\
സോദരനോടും മന്ദം നടന്നു മദ്ധ്യാഹ്നേ പോയ്-\\
ച്ചെന്നിതു രാമനഗസ്ത്യാനുജാശ്രമേ ജവം\\
വന്നു സല്‍കാരം ചെയ്താനഗസ്ത്യസഹജനും.\\
വന്യഭോജനവും ചെയ്തന്നവരെല്ലാവരു-\\
മന്യോന്യസല്ലാപവും ചെയ്തിരുന്നോരു ശേഷം.
\end{verse}

%%07_agasthyasandarshanam
\section{അഗസ്ത്യസന്ദര്‍ശനം}

\begin{verse}
ഭാനുമാനുദിച്ചപ്പോളര്‍ഘ്യവും നല്‍ഗി മഹാ-\\
കാനനമാര്‍ഗേ നടകൊണ്ടിതു മന്ദം മന്ദം\\
സര്‍വര്‍ത്തുഫലകുസുമാഢ്യപാദപലതാ-\\
സംവൃതം നാനാമൃഗസഞ്ചയനിഷേവിതം\\
നാനാപക്ഷികള്‍നാദംകൊണ്ടതിമനോഹരം\\
കാനനം ജാതിവൈരരഹിതജന്തുപൂര്‍ണം\\
നന്ദനസമാനമാനന്ദദാനാഢ്യം മുനി-\\
നന്ദനവേദദ്ധ്വനിമണ്ഡിതമനുപമം\\
ബ്രഹ്മര്‍ഷിപ്രവരന്മാരമരമുനികളും\\
സമ്മോദംപൂണ്ടു വാഴും മന്ദിരനികരങ്ങള്‍\\
സംഖ്യയില്ലാതോളമുണ്ടോരോരോ തരം നല്ല\\
സംഖ്യാവത്തുക്കളുമുണ്ടറ്റമില്ലാതവണ്ണം.\\
ബ്രഹ്മലോകവുമിതിനോടു നേരല്ലെന്നത്രേ\\
ബ്രഹ്മജ്ഞന്മാരായുള്ളോര്‍ ചൊല്ലുന്നു കാണുംതോറും.\\
ആശ്ചര്യമോരോന്നിവ കണ്ടുകണ്ടവരും ചെ-\\
ന്നാശ്രമത്തിന്നു പുറത്തടുത്തു ശുഭദേശേ\\
വിശ്രമിച്ചനന്തരമരുള്ടിച്ചെയ്തു രാമന്‍\\
വിശ്രുതനായ സുതീക്ഷ്ണന്‍തന്നോ’ടിനിയിപ്പോള്‍\\
വേഗേന ചെന്നു ഭവാനഗസ്ത്യമുനീന്ദ്രനോ-\\
ടാഗതനായോരെന്നെയങ്ങുണര്‍ത്തിച്ചീടണം.\\
ജാനകിയോടും ഭ്രാതാവായ ലക്ഷ്മണനോടും\\
കാനനദ്വാരേ വസിച്ചീടുന്നിതുപാശ്രമം.’\\
ശ്രുത്ര്വാ രാമോക്തം സുതീക്ഷ്ണന്മഹാപ്രസാദമി-\\
ത്യുക്ത്വാ സത്വരം ഗത്വാചാര്യമന്ദിരം മുദാ\\
നത്വാ തം ഗുരുവരമഗസ്ത്യം മുനികുല-\\
സത്തമം രഘൂത്തമഭക്തസഞ്ചയവൃതം\\
രാമമന്ത്രാര്‍ത്ഥവ്യാഖ്യാതല്‍പ്പരം ശിഷ്യന്മാര്‍ക്കാ-\\
യ്കാമദ മഗസ്ത്യ മാത്മാരാമം മുനീശ്വരം\\
ആരൂഢവിനയംകൊണ്ടാനതവക്ത്രത്തോടു-\\
മാരാല്‍ വീണുടന്‍ ദണ്ഡനമസ്കാരവും ചെയ്താന്‍:\\
‘രാമനാം ദാശരഥി സോദരനോടും നിജ-\\
ഭാമിനിയോടുമുണ്ടിങ്ങാഗതനായിട്ടിപ്പോള്‍\\
നില്ക്കുന്നു പുറത്തു ഭാഗത്തു കാരുണ്യാബ്ധേ! നിന്‍\\
തൃക്കഴലിണ കണ്ടു വന്ദിപ്പാന്‍ ഭക്തിയോടെ.’\\
മുമ്പേ തന്നകക്കാമ്പില്‍ കണ്ടറിഞ്ഞിരിക്കുന്നു\\
ഗുംഭസംഭവന്‍ പുനരെങ്കിലുമരുള്‍ചെയ്താന്‍:\\
‘ഭദ്രം തേ, രഘുനാഥമാനയ ക്ഷിപ്രം രാമ-\\
ഭദ്രം മേ ഹൃദിസ്ഥം ഭക്തവത്സലം ദേവം.\\
പാര്‍ത്തിരുന്നീടുന്നു ഞാനെത്രനാളുണ്ടു കാണ്മാന്‍\\
പ്രാര്‍ത്ഥിച്ചു സദാകാലം ധ്യാനിച്ചു രാമരൂപം\\
രാമരാമേതി രാമമന്ത്രവും ജപിച്ചതി-\\
കോമളം കാളമേഘശ്യാമളം നളിനാക്ഷം.’\\
ഇത്യുക്ത്വാ സരഭസമുത്ഥായ മുനിപ്രവ-\\
രോത്തമന്‍ മദ്ധ്യേ ചിത്തമത്യന്തഭക്ത്യാ മുനി-\\
സത്തമരോടും നിജശിഷ്യസഞ്ചയത്തോടും\\
ഗത്വാ ശ്രീരാമചന്ദ്രവക്ത്രം പാര്‍ത്തരുള്‍ചെയ്താന്‍:\\
‘ഭദ്രം തേ നിരന്തരമസ്തു സന്തതം രാമ-\\
ഭദ്രം മേ ദിഷ്ട്യാ ചിരമദ്യൈവ സമാഗമം\\
യോഗ്യനായിരിപ്പാരിഷ്ടാതിഥി ബലാല്‍ മമ\\
ഭാഗ്യപൂര്‍ണത്വേന സംപ്രാപ്തനായിതു ഭവാന്‍.\\
അദ്യ വാസരം മമ സഫലമത്രയല്ല\\
മത്തപസ്സാഫല്യവും വന്നിതു ജഗല്‍പ്പതേ!’\\
കുംഭസംഭവന്‍ തന്നെക്കണ്ടുരാഘവന്‍താനും\\
തമ്പിയും വൈദേഹിയും സംഭ്രമസമന്വിതം\\
കുമ്പിട്ടു ഭക്ത്യാ ദണ്ഡനമസ്കാരം ചെയ്തപ്പോള്‍\\
കുംഭജന്മാവുമെടുത്തെഴുന്നേല്പിച്ചു ശീഘ്രം\\
ഗാഢാശ്ലേഷവും ചെതു പരമാനന്ദത്തോടും\\
ഗൂഢപാദീശാംശജനായ ലക്ഷ്മണനേയും\\
ഗാത്രസ്പര്‍ശനപരമാഹ്ലാദജാതസ്രവ-\\
ന്നേത്രകീലാലാകുലനായ താപസവരന്‍\\
ഏകേന കരേണ സംഗൃഹ്യ രോമാഞ്ചാന്വിതം\\
രാഘവനുടെ കരപങ്കജമതിദ്രുതം\\
സ്വാശ്രമം ജഗാമ ഹൃഷ്ടാത്മനാ മുനിശ്രേഷ്ഠ-\\
നാശ്രിതജനപ്രിയനായ വിശ്വേശം രാമം\\
പാദ്യാര്‍ഘ്യാസനമധൂപര്‍ക്കമുഖ്യങ്ങളുമാ-\\
പാദ്യ സമ്പൂജ്യ സുഖമായുപവിഷ്ടം നാഥം\\
വന്യഭോജ്യങ്ങള്‍കൊണ്ടു സാദരം ഭുജിപ്പിച്ചു\\
ധന്യനാം തപോധനനേകാന്തേ ചൊല്ലിടിനാന്‍:
\end{verse}

%%08_agasthyasthuthi
\section{അഗസ്ത്യസ്തുതി}

\begin{verse}
നീ വരുന്നതും പാര്‍ത്തു ഞാനിരുന്നിതു നൂനം\\
ദേവകളോടും കമലാസനനോടും ഭവാന്‍\\
ക്ഷീരവാരിധിതീരത്തിങ്കല്‍ നിന്നരുള്‍ചെയ്തു\\
‘ഘോരരാവണന്‍തന്നെക്കൊന്നു ഞാന്‍ ഭൂമണ്ഡല-\\
ഭാരാപഹരണംചെയ്തിടുവനെ’ന്നു തന്നെ\\
സാരസാനന! സകലേശ്വര! ദയാനിധേ!\\
ഞാനന്നു തുടങ്ങി വന്നിവിടെ വാണീടിനേ-\\
നാനന്ദസ്വരൂപനാം നിന്നുടല്‍ കണ്ടുകൊള്‍വാന്‍.\\
താപസജനത്തോടും ശിഷ്യസംഘതത്തോടും\\
ശ്രീപാദാംബുജം നിത്യം ധ്യാനിച്ചു വസിച്ചു ഞാന്‍.\\
ലോകസൃഷ്ടിക്കുമുന്നമേകനായാന്ദനായ്\\
ലോകകാരണന്‍ വികല്പോപാധിവിരഹിതന്‍\\
തന്നുടെ മായ തനിക്കാശ്രയഭൂതയായി\\
തന്നുടെ ശക്തിയെന്നും പ്രകൃതിമഹാമായാ\\
നിര്‍ഗുണനായ നിന്നെയാവരണംചെയ്തിട്ടു\\
തല്‍ഗുണങ്ങളെയനുസരിപിച്ചീടുന്നതും\\
നിര്‍വ്യാജം വേദാന്തികള്‍ ചൊല്ലുന്നുനിന്നെ മുന്നം\\
ദിവ്യമാമവ്യാകൃതമെന്നുപനിഷദ്വശാല്‍.\\
മായാദേവിയെ മൂലപ്രകൃതിയെന്നും ചൊല്ലും\\
മായാതീതന്മാരെല്ലാം സംസൃതിയെന്നും ചൊല്ലും\\
വിദ്വാന്മാരവിദ്യഎന്നും പറയുന്നുവല്ലോ\\
ശക്തിയെപ്പലനാമം ചൊല്ലുന്നു പലതരം.\\
നിന്നാല്‍ സംക്ഷോഭ്യമാണയാകിയ മായതന്നില്‍-\\
നിന്നുണ്ടായ് വന്നു മഹത്തത്ത്വമെന്നല്ലോ ചൊല്‍വൂ.\\
നിന്നുടെ നിയോഗത്താല്‍ മഹത്തത്ത്വത്തിങ്കലേ\\
നിന്നുണ്ടായ് വന്നു പുനരഹങ്കാരവും പുരാ.\\
മഹത്തത്ത്വമുമഹങ്കാരവും സംസാരവും\\
മഹാദ്വേദികളേവം മൂന്നായിച്ചൊല്ലീടുന്നു.\\
സാത്വികം രാജസവും താമസമെന്നീവണ്ണം\\
വേദ്യമായ്ച്ചമഞ്ഞിതു മൂന്നുമെന്നറിഞ്ഞാലും.\\
താമസത്തിങ്കല്‍നിന്നു സൂക്ഷ്മതന്മാത്രകളും\\
ഭൂമിപൂര്‍വകസ്ഥൂല പഞ്ചഭൂതവും പിന്നെ\\
രാജസത്തിങ്കല്‍ നിന്നുണ്ടായിതിന്ദ്രിയങ്ങളും\\
തേജോരൂപങ്ങളായ ദൈവതങ്ങളും പിന്നെ\\
സാത്വികത്തിങ്ങല്‍ നിന്നു മനസ്സുമുണ്ടായ് വന്നു\\
സൂത്രരൂപകം ലിംഗമിവറ്റില്‍ നിന്നുണ്ടായി.\\
സര്‍വത്ര വ്യാപ്തസ്ഥൂലസഞ്ചയത്തിങ്കല്‍നിന്നു\\
ദിവ്യനാം വിരാള്‍പ്പ്രുമാനുണ്ടായിതെന്നു കേള്‍പ്പൂ.\\
അങ്ങനെയുള്ള വിരാള്‍പ്പുരുഷന്‍തന്നെയല്ലോ\\
തിങ്ങീടും ചരാചരലോകങ്ങളാകുന്നതും\\
ദേവമാനുഷതിര്യഗ്യോനിജാതികള്‍ ബഹു-\\
സ്ഥാവരജംഗമൗഘ പൂര്‍ണമായുണ്ടായ് വന്നു\\
ത്വന്മായാഗുണങ്ങളെ മൂന്നുമാശ്രയിച്ചല്ലോ\\
ബ്രഹ്മാവും വിഷ്ണുതാനും രുദ്രനുമുണ്ടായ് വന്നു.\\
ലോകസൃഷ്ടിക്കു രജോഗുണമാശ്രയിച്ചല്ലോ\\
ലോകേശനായ ധാതാ നാഭിയില്‍ നിന്നുണ്ടായി\\
സത്വമാം ഗുണത്തിങ്കല്‍നിന്നു രക്ഷിപ്പാന്‍ വിഷ്ണു\\
രുദ്രനും തമോഗുണം കൊണ്ടു സംഹരിപ്പാനും.\\
ബുദ്ധിജാതകളായ വൃത്തികള്‍ ഗുണത്രയം\\
നിത്യമംശിച്ചു ജാഗ്രല്‍ല്വപ്നവും സുഷുപ്തിയും\\
ഇവറ്റിന്നെല്ലാം സാക്ഷിയായ ചിന്മയന്‍ഭവാന്‍\\
നിവൃത്തന്‍ നിത്യനേകനവ്യയനല്ലോ നാഥ!\\
യാതൊരുകാലം സൃഷ്ടിച്ചെയ്വാനിച്ഛിച്ചു ഭവാന്‍\\
മോദമോടപ്പോളംഗീകരിച്ചു മായതന്നെ\\
തന്മൂലം ഗുണവാനെപ്പോലെയായിതു ഭവാന്‍\\
ത്വന്മഹാമായ രണ്ടുവിധമായ് വന്നാളല്ലോ.\\
വിദ്യയുമവിദ്യയുമെന്നുള്ള ഭേദാഖ്യയാ\\
വിദ്യയെന്നല്ലോ ചൊല്‍വൂ നിവൃത്തിനിരതന്മാര്‍.\\
അവിദ്യാവശന്മാരായ് വര്‍ത്തിച്ചീടിന ജനം\\
പ്രവൃത്തിനിരതന്മാരെന്നത്രേ ഭെദമുള്ളൂ.\\
വേദാന്തവാക്യാര്‍ത്ഥവേദികളായ് സമന്മാരായ്\\
പാദഭക്തന്മാരായുള്ളവര്‍ വിദ്യാത്മകന്മാര്‍\\
അവിദ്യാവശഗന്മാര്‍ നിത്യസംസാരികളെ-\\
ന്നവശ്യം തത്ത്വജ്ഞന്മാര്‍ ചൊല്ലുന്നു നിരന്തരം.\\
വിദ്യാഭ്യാസൈകരതന്മാരായ ജനങ്ങളെ\\
നിത്യമുക്തന്മാരെന്നു ചൊല്ലുന്നു തത്ത്വജ്ഞന്മാര്‍.\\
ത്വന്മന്ത്രോപാസകന്മാരായുള്ള ഭക്തന്മാര്‍ക്കു\\
നിര്‍മലയായ വിദ്യ താനേ സംഭവിച്ചീടും.\\
മറ്റുള്ള മൂഢന്മാര്‍ക്കു വിദ്യയുണ്ടാകെന്നതും\\
ചെറ്റില്ല നൂറായിരം ജന്മങ്ങള്‍ കഴിഞ്ഞാലും.\\
ആകയാല്‍ ത്വത്ഭക്തിസമ്പന്നന്മാരായുള്ളവ-\\
രേകാന്തമുക്തന്മാരില്ലേതും സംശയമോര്‍ത്താല്‍,\\
ത്വത്ഭക്തി സുധാഹീനന്മാരായുള്ളവര്‍ക്കെല്ലാം\\
സ്വപനത്തില്‍പോലും മോക്ഷം സംഭവിക്കയുമില്ല.\\
ശ്രീരാമ! രഘുപതേ! കേവലജ്ഞാനമൂര്‍ത്തേ!\\
ശ്രീരമണാത്മാരാമ! കാരുണ്യാമൃതസിന്ധോ!\\
എന്തിനു വളരെ ഞാനിങ്ങനെ പറയുന്നു\\
ചിന്തിക്കില്‍ സാരം കിഞ്ചില്‍ ചൊല്ലുവന്‍ ധരാപതേ!\\
സാധുസംഗതിതന്നെ മോക്ഷകാരണമെന്നു\\
വേദാന്തജ്ഞന്മാരായ വിദ്വാന്മാര്‍ ചൊല്ലീടുന്നു\\
സാധുക്കളാകുന്നതു സമചിത്തന്മാരല്ലോ\\
ബോധിപ്പിച്ചീടുമാത്മജ്ഞാനവും ഭക്തന്മാര്‍ക്കായ്\\
നിസ്പൃഹന്മാരായ് വിഗതൈഷണന്മാരായ് സദാ\\
ത്വത്ഭക്തന്മാരായ് നിവൃത്താഖിലകാമന്മാരായ്\\
ഇഷ്ടാനിഷ്ടപ്രാപ്തികള്‍ രണ്ടിലും സമന്മാരായ്\\
നഷ്ടസംഗന്മാരുമായ് സന്ന്യസ്തകര്‍മാക്കളായ്\\
തുഷ്ടമാനസന്മാരായ് ബ്രഹ്മതത്പരന്മാരായ്\\
ശിഷ്ടാചാരൈകപരായണന്മാരായി നിത്യം\\
യോഗാര്‍ത്ഥം യമനിയമാദിസമ്പന്നന്മാരാ-\\
യേകാന്തേ ശമദമസാധനയുക്തന്മാരായ്\\
സാധുക്കളവരോടു സംഗതിയുണ്ടാകുമ്പോള്‍\\
ചേതസി ഭവല്‍കഥാശ്രവണേ രതിയുണ്ടാം\\
ത്വല്‍കഥാശ്രവണേന ഭക്തിയും വര്‍ദ്ധിച്ചിടും\\
ഭക്തി വര്‍ദ്ധിച്ചീടുമ്പോള്‍ വിജ്ഞാനമുണ്ടായ്വരും\\
വിജ്ഞാനജ്ഞാനാദികള്‍കൊണ്ടു മോക്ഷവും വരും\\
വിജ്ഞാതമെന്നാല്‍ ഗുരുമുഖത്തില്‍നിന്നിതെല്ലാ\\
ആകയാല്‍ ത്വത്ഭക്തിയും നിങ്കലേ പ്രേമവായ്പും\\
രാഘവ! സദാ ഭവിക്കേണമേ ദയാനിധേ!\\
ത്വല്‍പാദാബ്ജങ്ങളിലും ത്വത്ഭക്തന്മാരിലുമെ-\\
ന്നുള്‍പ്പൂവില്‍ ഭക്തിപുനരെപ്പോഴുമുണ്ടാകണം\\
ഇന്നല്ലോ സഫലമായ് വന്നതു മമ ജന്മ-\\
മിന്നു മല്‍ക്രതുക്കളും വന്നിതു സഫലമായ്.\\
ഇന്നല്ലോ തപസ്സിനും സാഫല്യമുണ്ടായ് വന്നു\\
ഇന്നല്ലോ സഫലമായ് വന്നതു മന്നേത്രവും\\
സീതയാ സാര്‍ദ്ധം ഹൃദി വസിക്ക സദാ ഭവാന്‍\\
സീതാവല്ലഭ! ജഗന്നായക! ദാശരഥേ!\\
നടക്കുമ്പോഴുമിരിക്കുമ്പോഴുമൊരേടത്തു\\
കിടക്കുമ്പോഴും ഭുജിക്കുമ്പോഴുമെന്നു വേണ്ടാ\\
നാനാ കര്‍മങ്ങളനുഷ്ഠിക്കുമ്പോള്‍ സദാ കാലം\\
മാനസേ ഭദ്രൂപം തോന്നേണം ദയാംബുധേ!\\
കുംഭസംഭവനില്‍ സ്തുതിച്ചു ഭക്തിയോടെ\\
ജംഭാരിതന്നാല്‍ മുന്നം നിക്ഷിപ്തമായ ചാപം\\
ബാണതൂണീരത്തോടും കൊടുന്നു ഖഡ്ഗത്തോടും\\
ആനന്ദവിവശനായ് പിന്നെയുമരുള്‍ചെയ്താന്‍:\\
ഭൂഭാരഭൂതമായ രാക്ശസവംശം നിന്നാല്‍\\
ഭൂപതേ! വിനഷ്ടമായീടണം വൈകീടാതെ\\
സാക്ഷാല്‍ ശ്രീനാരായണനായ നീ മായയോടും\\
രാക്ഷസവധത്തിനായ് മര്‍ത്ത്യനായ് പിറന്നതും.\\
രണ്ടുയോജന വഴി ചെല്ലുമ്പോളിവിടെ നി-\\
ന്നുണ്ഡല്ലോ പുണ്യഭൂമിയാകിയ പഞ്ചവടി,\\
ഗൗതമീതീരേ നല്ലൊരാശ്രമം ചപച്ചതില്‍\\
സീതയാ വസിക്കപോയ് ശേഷമുള്ളൊരു കാലം\\
തത്രൈവ വസിച്ചു നീ ദേവകാര്യങ്ങളെല്ലാ\\
സ്ത്വരം ചെയ്കെ’ന്നുടനനുജ്ഞ നല്കി മുനി.\\
ശ്രുത്വൈതല്‍ സ്തോത്രസാരമഗസ്ത്യസുഭാഷിതം\\
തത്ത്വാര്‍ത്ഥസമന്വിതം രാഘവന്‍ തിരുവടി\\
ബാണചാപാദികളും തത്രൈവ നിക്ഷേപിച്ചു\\
വീണുടന്‍ നമസ്കരിച്ചഗസ്ത്യപാദാംബുജം\\
യാത്രയുമയപ്പിച്ചു സുമിത്രാത്മജനോടും\\
പ്രീത്യാ ജാനകിയോടുമെഴുന്നള്ളീടുന്നേരം
\end{verse}

%%09_jataayusangamam
\section{ജടായൂസംഗമം}

\begin{verse}
അദ്രിശൃംഗാഭം തത്ര പദ്ധതിമദ്ധ്യേ കണ്ടു\\
പത്രിസത്തമനാകും വൃദ്ധനാം ജടായുഷം\\
എത്രയും വളര്‍ന്നൊരു വിസ്മയം പൂണ്ടു രാമന്‍\\
ബദ്ധരോഷേണ സുമിത്രാത്മജനോടു ചൊന്നാന്‍:\\
‘രക്ഷസാംപ്രവരനിക്കിടക്കുന്നതു മുനി-\\
ഭക്ഷക്നിവനെ നീ കണ്ടതില്ലയോ സഖേ?\\
വില്ലിങ്ങു തന്നീടു നീ ഭീതിയുമുണ്ടാഒല്ലാ\\
കൊല്ലുവനിവനെ ഞാന്‍ വൈകാതെയിനിയിപ്പോള്‍.’\\
ലക്ഷ്മണന്‍ തന്നോടിത്ഥം രാമന്‍ ചൊന്നതു കേട്ടു\\
പക്ഷിശ്രേഷ്ഠനും ഭയപീഡിതനായിച്ചൊന്നാന്‍:\\
‘വദ്ധ്യനല്ലഹം തവ താതനു ചെറുപ്പത്തി-\\
ലെത്രയുമിഷ്ടനായ വയസ്യനറിഞ്ഞാലും\\
നിന്തിരുവടിക്കും ഞാനിശ്ടത്തെച്ചെയ്തീടുവന്‍\\
ഹന്തവ്യനല്ല ഭവത്ഭക്തനാം ജടായു ഞാന്‍.’\\
എന്നവ കേട്ടു ബഹു സ്നേഹമുള്‍ക്കൊണ്ടു നാഥന്‍\\
നന്നായാശ്ലേഷംചെയ്തു നല്കിനാനനുഗ്രഹം:\\
‘എങ്കില്‍ ഞാനിരിപ്പതിനടുത്തു വസിക്ക നീ\\
സങ്കടമിനിയൊന്നുകൊണ്ടുമേ നിനക്കില്ല.\\
ശങ്കിച്ചേനല്ലോ നിന്നെ ഞാനതു കഷ്ടം! കഷ്ടം!\\
കിങ്കരപ്രവരനായ് വാഴുക മേലില്‍ ഭവാന്‍.’
\end{verse}

%%10_panchavateepravesham
\section{പഞ്ചവടീപ്രവേശം}

\begin{verse}
എന്നരുള്‍ചെയ്തു ചെന്നു പുക്കിതു പഞ്ചവടി-\\
തന്നിലാമ്മാറു സീതാലക്ഷ്മണസമേതനായ്.\\
പര്‍ണശാലയും തീര്‍ത്തു ലക്ഷ്മണന്‍ മനോജ്ഞമായ്\\
പര്‍ണപുഷ്പങ്ങള്‍കൊണ്ടു തല്പവുമുണ്ടാക്കിനാന്‍.\\
ഉത്തമഗംഗാനദിക്കുത്തര തീരേ പുരു-\\
ഷോത്തമന്‍ വസിക്കിതു ജാനകീദേവിയോടും.\\
കദളീപനസാമ്രാദ്യഖിലഫലവൃക്ഷ-\\
വൃത കാനനേ ജനസംബാധവിവര്‍ജിതേ\\
നീരുജസ്ഥലേ വിനോദിപ്പിച്ചു ദേവിതന്നെ\\
ശ്രീരാമനയോദ്ധ്യയില്‍ വാണതുപോലേ വാണാന്‍.\\
ഫലമൂലാദികളും ലക്ഷ്മണനനുദിനം\\
പലവും കൊണ്ടുവന്നു കൊടുക്കും പ്രീതിയോടെ\\
രാത്രിയിലുറങ്ങാതെ ചാപബാണവും ധരി-\\
ച്ചാസ്ഥയാ രക്ഷാര്‍ത്ഥമായ് നിന്നീടും ഭക്തിയോടെ.\\
സീതയെ മദ്ധ്യേയാക്കി മൂവരും പ്രാതഃകാലേ\\
ഗൗതമിതന്നില്‍ കുളിച്ചര്‍ഘ്യവും കഴിച്ചുടന്‍\\
പോരുമ്പോള്‍ സൗമിത്രി പാനീയവും കൊണ്ടുപോരും\\
വാരം വാരം പ്രീതി പൂണ്ടിങ്ങനെ വാഴും കാലം
\end{verse}

%%11_lakshmanopadhesham
\section{ലക്ഷ്മണോപദേശം}

\begin{verse}
ലക്ഷ്മണനൊരുദിനമേകാന്തേ രാമദേവന്‍-\\
തൃക്കഴല്‍ കൂപ്പി വിനയാന്വിതനായിച്ചൊന്നാന്‍:\\
‘മുക്തിമാര്‍ഗത്തെയരുള്‍ചെയ്യണം ഭഗവാനേ!\\
ഭക്തനാമടിയനോടജ്ഞാനം നീങ്ങുംവണ്ണം.\\
ജ്ഞാനവിജ്ഞാനഭക്തിവൈരാഗ്യ ചിഹ്മമെല്ലാം\\
മാനസാനന്ദം വരുമാറരുള്‍ചെയ്തീടണം.\\
ആരും നിന്തിരുവടിയൊഴിഞ്ഞില്ലിവയെല്ലാം\\
നേരോടെയുപദേശിച്ചീടുവാന്‍ ഭൂമണ്ഡലേ.’\\
ശ്രീരാമനതുകേട്ടു ലക്ഷ്മണന്‍തന്നോടപ്പോ-\\
ളാരൂഢാനന്ദമരുള്‍ചെയ്തിതു വഴിപോലെ:\\
‘കേട്ടാലുമെങ്കിലതിഗുഹ്യമാമുപദേശം\\
കേട്ടോളം തീര്‍ന്നുകൂടും വികല്പഭ്രമമെല്ലാം.\\
മുമ്പിനാല്‍ മായാസ്വരൂപത്തെ ഞാന്‍ ചൊല്ലീടുവ-\\
നമ്പോടു പിന്നെ ജ്ഞാനസാധനം ചൊല്ലാമല്ലോ.\\
വിജ്ഞാനസഹിതമാം ജ്ഞാനവും ചൊല്‍വന്‍ പിന്നെ\\
വിജ്ഞേയമാത്മസ്വരൂപത്തെയും ചൊല്ലാമെടോ!\\
ജ്ഞേയമായുള്ള പരമാത്മാനമറിയുമ്പോള്‍\\
മായാസംബന്ധഭയമൊക്കെ നീങ്ങീടുമല്ലോ.\\
ആത്മാവല്ലാതെയുള്ള ദേഹാദിവസ്തുക്കളി-\\
ലാത്മാവെന്നുള്ള ബോധം യാതൊന്നു ജഗത്ത്രയേ\\
മായയാകുന്നതു നിര്‍ണയമതിനാലേ\\
കായസംബന്ധമകും സംസാരം ഭവിക്കുന്നു.\\
ഉണ്ടല്ലോ പിന്നെ വിക്ഷേപാവരണങ്ങളെന്നു\\
രണ്ടു രൂപം മായയ്ക്കെന്നറിക സൗമിത്രേ നീ.\\
എന്നതില്‍ മുന്നേതല്ലോ ലോകത്തെക്കല്പിക്കുന്ന-\\
തെന്നറികതിസ്ഥൂലസൂക്ഷ്മഭെദങ്ങളോടും\\
ലിംഗാദിബ്രഹ്മാന്തമാമവിദ്യാരൂപമതും\\
സംഗാദിദോഷങ്ങളെസ്സംഭവിപ്പിക്കുന്നതും.\\
ജ്ഞാനരൂപിണിയാകും വിദ്യയായതു മറ്റേ-\\
താനന്ദപ്രാപ്തിഹേതുഭൂതയെന്നറിഞ്ഞാലും.\\
മായാകല്പിതം പരമാത്മനി വിശ്വമെടോ!\\
മായകൊണ്ടല്ലോ വിശ്വമുണ്ടെന്നു തോന്നിക്കുന്നു.\\
രജ്ജുഖണ്ഡത്തിങ്കലെപ്പന്നഗബുദ്ധിപോലെ\\
നിശ്ചയം വിചാരിക്കിലേതുമൊന്നില്ലയല്ലോ.\\
മാനവന്മാരാല്‍ കാണപ്പെട്ടതും കേള്‍ക്കായതും\\
മാനസത്തിങ്കല്‍ സ്മരിക്കപ്പെടുന്നതുമെല്ലാം\\
സ്വപ്നസന്നിഭം വിചാരിക്കിലില്ലാതൊന്നല്ലോ\\
വിഭ്രമം കളഞ്ഞാലും വികല്പമുണ്ടാകേണ്ട.\\
ജന്മസംസാരവൃക്ഷമൂലമായതു ദേഹം\\
തന്മൂലം പുത്രകളത്രാദിസംബന്ധമെല്ലാം\\
ദേഹമായതു പഞ്ചഭൂതസഞ്ചയമയം\\
ദേഹസംബന്ധം മായാവൈഭവം വിചാരിച്ചാല്‍.\\
ഇന്ദ്രിയദശകവുമഹങ്കാരവും ബുദ്ധി\\
മനസ്സും ചിത്തമൂലപ്രകൃതിയെന്നിതെല്ലാം\\
ഓര്‍ത്തുകണ്ടാലുമൊരുമിച്ചിരിക്കുന്നതല്ലോ\\
ക്ഷേത്രമായതു ദേഹമെന്നുമുണ്ടല്ലോ നാമം.\\
എന്നിവറ്റിങ്കല്‍നിന്നു വേറൊന്നു ജീവനതും\\
നിര്‍ണയം പരമാത്മാ നിശ്ചലന്‍ നിരാമയന്‍.\\
ജീവാത്മസ്വരൂപത്തെയറിഞ്ഞു കൊള്‍വാനുള്ള\\
സാധനങ്ങളെക്കേട്ടുകൊള്ളുക സൗമിത്രേ! നീ\\
ജീവാത്മാവെന്നും പരമാത്മാവെന്നതുമോര്‍ക്കില്‍\\
കേവലം പര്യായശബ്ദങ്ങളെന്നറിഞ്ഞാലും.\\
ഭേദമേതുമേയില്ല രണ്ടുമൊന്നത്രേ നൂനം\\
ഭേദമുണ്ടെന്നു പറയുന്നതജ്ഞന്മാരല്ലോ.\\
മാനവും ഡംഭം ഹിംസാവക്രത്വം കാമം ക്രോധം\\
മാനസേ വെടുഞ്ഞു സന്തുഷ്ടനായ് സദാകാലം\\
അന്യാക്ഷേപാദികളും സഹിചു സമബുദ്ധ്യാ\\
മന്യുഭാവവുമകലെക്കളഞ്ഞനുദിനം\\
ഭക്തികൈക്കൊണ്ടു ഗുരുസേവയും ചെയ്തു നിജ-\\
ചിത്തശുദ്ധിയും ദേഹശുദ്ധിയും ചെയ്തുകൊണ്ടു\\
നിത്യവും സല്‍ക്കര്‍മങ്ങള്‍ക്കിളക്കം വരുത്താതെ\\
സത്യത്തെസ്സമാശ്രയിച്ചാനന്ദസ്വരൂപനായ്.\\
മാനസവചനദേഹങ്ങളെയടക്കിത്ത-\\
ന്മാനസേ വിഷയ്സൗഖ്യങ്ങളെച്ചിന്തിയാതെ\\
ജനന ജരാമരണങ്ങളെച്ചിന്തിച്ചുള്ളി-\\
ലനഹങ്കാരത്വേന സമഭാവനയോടും\\
സര്‍വാത്മാവാകുമെങ്കലുറച്ച മനസ്സോടും\\
സര്‍വദാ രാമരാമേത്യമിത ജപത്തൊടും\\
പുത്രദാരാര്‍ഥാദിഷു നിസ്നേഹത്വവും ചെയ്തു\\
സക്തിയുമൊന്നിങ്കലും കൂടാതെ നിരന്തരം\\
ഇഷ്ടാനിഷ്ടപ്രാപ്തിക്കു തുല്യഭാവത്തോടു സ-\\
ന്തുഷ്ടനായ് വിവിക്തശുദ്ധസ്ഥലേ വസിക്കേണം.\\
പ്രാകൃതജനങ്ങളുമായ് വസിക്കരുതൊട്ടു-\\
മേകാന്തേ പരമാത്മജ്ഞാനതല്‍പരനായി\\
വേദാന്തവാക്യാര്‍ഥങ്ങളവലോകനംചെയ്തു\\
വൈദികകര്‍മങ്ങളുമാത്മനി സമര്‍പ്പിച്ചാല്‍\\
ജ്ഞാനവുമകതാരിലുറച്ചു ചമഞ്ഞീടും\\
മനസേ വികല്പങ്ങളേതുമേയുണ്ടാകൊല്ലാ.\\
ആത്മാവാകുന്നതെന്തെന്നുണ്ടോ കേളതുമെങ്കി-\\
ലാത്മാവല്ലല്ലോ ദേഹപ്രാണബുദ്ധ്യഹങ്കാരം\\
മാനസാദികളൊന്നുമിവറ്റില്‍നിന്നു മേലേ\\
മാനമില്ലാത പരമാത്മാവുതാനേ വേറെ\\
നില്പിതു ചിദാത്മാവു ശുദ്ധമവ്യക്തം ബുദ്ധം\\
തല്‍പദാത്മാ ഞാനിഹ ത്വല്‍പദാര്‍ഥവുമായി\\
ജ്ഞാനംകൊണ്ടെന്നെ വഴിപോലെ കണ്ടറിഞ്ഞീടാം\\
ജ്ഞാനമാകുന്നതെന്നെക്കാട്ടുന്ന വസ്തുതന്നെ.\\
ജ്ഞാനമുണ്ടാകുന്നതു വിജ്ഞാനംകൊണ്ടുതന്നെ\\
ഞാനിതെന്നറിവിനു സാധനമാകയാലേ.\\
സര്‍വത്ര പരിപൂര്‍ണനാത്മാവു ചിദാനന്ദന്‍\\
സര്‍വസത്വാന്തര്‍ഗതനപരിച്ഛേദ്യനല്ലോ.\\
ഏകനദ്വയന്‍ പരനവ്യയന്‍ ജഗന്മയന്‍\\
യോഗേശനജനഖിലാധാരന്‍ നിരാധാരന്‍\\
നിത്യസത്യജ്ഞാനാദിലക്ഷണന്‍ ബ്രഹ്മാത്മകന്‍\\
ബുദ്ധ്യുപാധികളില്‍ വേറിട്ടവന്‍ മായാമയന്‍\\
ജ്ഞാനംകൊണ്ടുപഗമ്യന്‍ യോഗിനാമേകാന്മനാം\\
ജ്ഞാനമാചാര്യശാസ്ത്രൗഘോപദേശൈക്യജ്ഞാനം\\
ആത്മനോരേവം ജീവപരയോര്‍മൂലവിദ്യാ\\
ആത്മനി കാര്യകാരണങ്ങളും കൂടിച്ചേര്‍ന്നു\\
ലയിച്ചീടുമ്പോളുള്ളോരവസ്ഥയല്ലോ മുക്തി\\
ലയത്തോടാശു വേറിട്ടിരിപ്പതാത്മാവൊന്നേ.\\
ജ്ഞാനവിജ്ഞാനവൈരാഗ്യത്തോടു സഹിതമാ-\\
നന്ദമായിട്ടുള്ള കൈവല്യസ്വരൂപമി-\\
തുള്ളവണ്ണമേ പറവാനുമിതറിവാനു-\\
മുള്ളം നല്ലുണര്‍വുള്ളോരില്ലാരും ജഗത്തിങ്കല്‍.\\
മത്ഭക്തിയില്ലാതവര്‍ക്കെത്രയും ദുര്‍ല്ലഭം കേള്‍\\
മത്ഭക്തികൊണ്ടുതന്നെ കൈവല്യം വരും താനും.\\
നേത്രമുണ്ടെന്നാകിലും കാണ്മതിന്നുണ്ടു പണി\\
രാത്രിയില്‍ തന്റെ പദം ദീപമുണ്ടെന്നാകിലേ\\
നേരുള്ള വഴിയറിഞ്ഞീടാവിതവ്വണ്ണമേ\\
ശ്രീരാമഭക്തിയുണ്ടെന്നാകിലേ കാണായ് വരൂ.\\
ഭക്തനു നന്നായ് പ്രകാശിക്കുമാത്മാവു നൂനം\\
ഭക്തിക്കു കാരണവുമെന്തെന്നു കേട്ടാലും നീ.\\
മത്ഭക്തന്മാരോട്ളുള്ള നിത്യസംഗമമതും\\
മത്ഭക്തന്മാരെക്കനിവോടു സേമിക്കുന്നതും\\
ഏകാദശ്യാദി വ്രതാനുഷ്ഠാനങ്ങളും പുന-\\
രാകുലമെന്നിയേ സാധിച്ചു കൊള്‍കയുമഥ\\
പൂജനം വന്ദനവും ഭാവനം ദാസ്യം നല്ല\\
ഭോജനമഗ്നിവിപ്രാണാം കൊടുക്കയുമഥ\\
മല്‍ക്കഥാപാഠശ്രവണങ്ങള്‍ ചെയ്കയും മുദാ\\
മല്‍ഗുണനാമങ്ങളെക്കീര്‍ത്തിച്ചു കൊള്ളുകയും\\
സന്തതമിത്ഥമെങ്കല്‍ വര്‍ത്തിക്കും ജനങ്ങള്‍ക്കൊ-\\
രന്തരം വരാതൊരു ഭക്തിയുമുണ്ടായ് വരും.\\
ഭക്തിവര്‍ദ്ധിച്ചാല്‍പ്പിന്നെ മറ്റൊന്നും വരേണ്ടതി-\\
ല്ലുത്തമോത്തമന്മാരായുള്ളവരവരല്ലോ\\
ഭക്തിയുക്തനു വിജ്ഞാനജ്ഞാന വൈരാഗ്യങ്ങള്‍\\
സദ്യഃസംഭവിച്ചീടുമെന്നാല്‍ മുക്തിയും വരും.\\
മുക്തിമാര്‍ഗം താവകപ്രശ്നാനുസാരവശാ-\\
ലുക്തമായതു നിനക്കെന്നാലെ ധരിക്ക നീ\\
മക്തവ്യമല്ല നൂനമെത്രയും ഗുഹ്യം മമ\\
ഭക്തന്മാര്‍ക്കൊഴിഞ്ഞുപദേശിച്ചീടാരുതല്ലോ.\\
ഭക്തനെന്നാകിലവന്‍ ചോദിച്ചീലെന്നാകിലും\\
വ്യക്തവ്യമവനോടു വിശ്വാസം വരികയാല്‍\\
ഭക്തിവിശ്വാസശ്രദ്ധായുക്തനാം മര്‍ത്ത്യനിതു\\
നിത്യമായ് പാഠംചെയ്കിലജ്ഞാനമകന്നുപോം.\\
ഭക്തിസംയുക്തന്മാരാം യോഗീന്ദ്രന്മാര്‍ക്കു നൂനം\\
ഹസ്തസംസ്ഥിതയല്ലോ മുക്തിയെന്നറിഞ്ഞാലും.’
\end{verse}

%%12_shoorpanakhaagamanam
\section{ശൂര്‍പ്പണഖാഗമനം}

\begin{verse}
ഇത്തരം സൗമിത്രിയോടരുളിച്ചെയ്തു പുന-\\
രിത്തിരിനേരമിരുന്നീടിനോരനന്തരം\\
ഗൗതമീതീരേ മഹാകാനനേ പഞ്ചവടീ-\\
ഭൂതലേ മനോഹരേ സഞ്ചരിച്ചീടുന്നൊരു\\
യാമിനീചരി ജനസ്ഥാനവാസിനിയായ\\
കാമരൂപിണി കണ്ടാള്‍ കാമിനി വിമോഹിനി\\
പങ്കജധ്വജകുലിശാങ്കുശാങ്കിതങ്ങളായ്\\
ഭംഗിതേടീടും പാദപാതങ്ങളതുനേരം\\
പാദസൗന്ദര്യം കണ്ടു മോഹിതയാകയാലെ\\
കൗതുകമുള്‍ക്കൊണ്ടു രാമാശ്രമമകം പുക്കാള്‍.\\
ഭാനുമണ്ഡലസഹസ്രോജ്ജ്വലം രാമനാഥം\\
ഭനുഗോത്രജം ഭവഭയനാശനം പരം\\
മാനവവീരം മനോമോഹനം മായാമയം\\
മാസസഭവസമം മാധവം മധുഹരം\\
ജാനകിയോടുംകൂടെ വാണീടുന്നതുകണ്ടു\\
മീനകേതനബാണപീഡിതയായാളേറ്റം\\
സുന്ദരവേഷത്തോടും മന്ദഹാസവും പൊഴി-\\
ഞ്ഞിന്ദിരാവരനോടു മന്ദമായുരചെയ്താള്‍:\\
ആരെടോ ഭവാന്‍? ചൊല്ലീടാരുടെ പുത്രനെന്നും\\
നേരോടെന്തിവിടേക്കു വരുവാന്‍ മൂലമെന്നും\\
എന്തൊരുമൂലം ജടാവല്ക്കലാദികളെല്ലം\\
എന്തിനു ധരിച്ചിതു താപസവേഷമെന്നും\\
എന്നുടെ പരമാര്‍ഥം മുന്നേ ഞാന്‍ പറഞ്ഞീടാം\\
നിന്നോടു നീയെന്നോടു പിന്നെച്ചോദിക്കുമല്ലോ.\\
രാക്ഷസേശ്വരനായ രാവണഭഗിനി ഞാ-\\
നാഖ്യയാ ശൂര്‍പ്പണഖ കാമരൂപിണിയല്ലോ.\\
ഖരദൂഷണത്രിശിരാക്കളാം ഭ്രാതാക്കന്മാര്‍-\\
ക്കരികേ ജനസ്ഥാനേ ഞാനിരിപ്പതു സദാ.\\
നിന്നെ ഞാനാരെന്നതു മറിഞ്ഞീലതും പുന-\\
രെന്നോടു പരമാര്‍ത്ഥം ചൊല്ലണം ദയാനിധേ!\\
‘സുന്ദരീ! കേട്ടുകൊള്‍ക ഞാനയോദ്ധ്യാധിപതി-\\
നന്ദനന്‍ ദാശരഥി രാംഅനെന്നല്ലോ നാമം.\\
എന്നുടെ ഭാര്യയിവള്‍ ജനകാത്മജാ സീത\\
ധന്യേ! മല്‍ ഭ്രാതാവായ ലക്ഷ്മണനിവനെടോ.\\
എന്നാലെന്തൊരു കാര്യം നിനക്കു മനോഹരേ!\\
നിന്നുടെ മനോഗതം ചൊല്ലുക മടിയാതെ.’\\
എന്നതു കേട്ടനേരം ചൊല്ലിനാള്‍ നിശാചരി-\\
‘യെന്നോടു കൂടെപ്പോന്നു രമിച്ചു കൊള്ളേണം നീ\\
നിന്നെയും പിരിഞ്ഞു പോവാന്‍ മമ ശക്തി പോരാ\\
എന്നെ നീ പരിഗ്രഹിച്ചീടണം മടിയാതെ.’\\
ജാനകിഅതന്നെക്കടാക്ഷിച്ചു പുഞ്ചിരിപൂണ്ടു\\
മാനവവീരനവളോടരുള്‍ചെയ്തീടിനാന്‍:\\
‘ഞാനിഹ തപോധനവേഷവും ധരിച്ചോരോ\\
കാനനം തോറും നടന്നീടുന്നു സദാകാലം.\\
ജാനകിയാകുമിവളെന്നുടെ പത്നിയല്ലോ\\
മാനസേ പാര്‍ത്താല്‍ വെടിഞ്ഞീടരുതൊന്നുകൊണ്ടും.\\
സാപത്ന്യോത്ഭവദുഃഖമെത്രയും കഷ്ടം! കഷ്ടം!\\
താപത്തെസ്സഹിപ്പതിനാളല്ല നീയുമെടോ!\\
ലക്ഷ്മണന്‍ മമ ഭ്രാതാ സുന്ദരന്‍ മനോഹരന്‍\\
ലക്ഷ്മീദേവിക്കുതന്നെയൊക്കും നീയെല്ലാം കൊണ്ടും\\
നിങ്ങളില്‍ച്ചേരുമേറെ നിര്‍ണയം മനോഹരേ!\\
സംഗവും നിന്നിലേറ്റം വര്‍ദ്ധിക്കുമവനെടോ!\\
മംഗലശീലനനുരൂപനെത്രയും നിന-\\
ക്കങ്ങു നീ ചെന്നു പറഞ്ഞീടുക വൈകീടാതെ.’\\
എന്നതു കേട്ടനേരം സൗമിത്രിസമീപേ പോയ്-\\
ച്ചെന്നവളപേക്ഷിച്ചാള്‍ ഭര്‍ത്താവാകെന്നു തന്നെ.\\
ചൊന്നവളോട് ചിരിച്ചവനുമൗരചെയ്താ-\\
‘നെന്നുടെ പരമാര്‍ത്ഥം നിന്നോടു പറഞ്ഞീടാം\\
മന്നവനായ രാമന്‍ തന്നുടെ ദാസന്‍ ഞാനോ-\\
ധന്യേ! നീ ദാസിയാവാന്‍ തക്കവളല്ലയല്ലോ.\\
ചെന്നു നീ ചൊല്ലീടഖലേശ്വരനായ രാമന്‍-\\
തന്നോടു തവ കുലശീലാചാരങ്ങളെല്ലാം\\
എന്നാലന്നേരം തന്നെ കൈകൊള്ളുമല്ലോ രാമന്‍\\
നിന്നെ’യെന്നതു കേട്ടു രാവണസഹോദരി\\
പിന്നെയും രഘുകുലനായകനോടു ചൊന്നാ-\\
‘ളെന്നെ നീ പരിഗ്രഹിച്ചീടുക നല്ലൂ നിന-\\
ക്കൊന്നുകൊണ്ടുമേയൊരു സങ്കടമുണ്ടായ് വരാ.\\
മന്നവാ! ഗിരിവന ഗ്രാമദേശങ്ങള്‍ തോറു-\\
മെന്നോടുകൂടെനടന്നോരോരോഭോഗമെല്ലാ-\\
മന്യോന്യം ചേര്‍ന്നു ഭുജിക്കായ്വരുമനാരതം.’\\
ഇത്തരമവളുരചെയ്തതു കേട്ടനേര-\\
മുത്തരമരുള്‍ചെയ്തു രാഘവന്‍ തിരുവടി:\\
‘ഒരുത്തനായാലവനരികേ ശൂശ്രൂഷിപ്പാ-\\
നൊരുത്തിവേണമതിനിവളുണ്ടെനിക്കിപ്പോള്‍\\
ഒരുത്തിവേണമവനതിനാരെന്നു തിര-\\
ഞ്ഞിരിക്കുന്നേരമിപ്പോള്‍ നിന്നെയും കണ്ടുകിട്ടി\\
വരുത്തും ദൈവമൊന്നുകൊതിച്ചാലിനി നിന്നെ\\
വരിച്ചു കൊള്ളുമവനില്ല സംശയമേതും.\\
തെരിക്കെന്നിനിക്കാലം കളഞ്ഞീടാതെ ചെല്ക\\
കരത്തെഗ്രഹിച്ചീടും കടുക്കെന്നവനെടോ!’\\
രാഘവവാക്യം കേട്ടു രാവണസഹോദരി\\
വ്യാകുലചേതസ്സോടും ലക്ഷ്മണാന്തികേ വേഗാല്‍\\
ചെന്നു നിന്നപേക്ഷിച്ചനേരത്തു കുമാരനു-\\
‘മെന്നോടിത്തരം പറഞ്ഞീടൊല്ല വെറുതേ നീ\\
നിന്നിലില്ലേതുമൊരു കാംക്ഷയെന്നറിക നീ\\
മന്നവനായ രാമന്‍ തന്നോടു പറഞ്ഞാലും.’\\
പിന്നയുമതു കേട്ടു രാഘവസമീപേപോയ്-\\
ച്ചെന്നുനിന്നപേക്ഷിച്ചാളാശയാ പലതരം.\\
കാമവുമാശാഭംഗംകൊണ്ടു കോപവുമതി-\\
പ്രേമവുമാലസ്യവും പൂണ്ടു രാക്ഷസിയപ്പോള്‍\\
മായാരൂപവും വേര്‍പെട്ടഞ്ജനശൈലംപോലെ\\
കായാകാരവും ഘോരദംഷ്ട്രയും കൈക്കൊണ്ടേറ്റം\\
കമ്പമുള്‍ക്കൊണ്ടു സീതാദേവിയോടടുത്തപ്പോള്‍\\
സംഭ്രമത്തോടു രാമന്‍ തടുത്തു നിര്‍ത്തുന്നേരം\\
ബാലകന്‍ കണ്ടു ശീഘ്രം കുതിച്ചു ചാടിവന്നു\\
വാളുറയൂരിക്കാതും മുലയും മൂക്കുമെല്ലാം\\
ഛേദിച്ചനേരമവളലറി മുറയിട്ട\\
നാദത്തെക്കൊണ്ടു ലോകമൊക്കെ മാറ്റൊലിക്കൊണ്ടു.\\
നീലപര്വതത്തിന്റെ മുകളില്‍നിന്നു ചാടി\\
നാലഞ്ചുവഴി വരുമരുവിയാറുപോലെ\\
ചോരയുമൊലിപ്പിച്ചു കാളരാത്രിയെപ്പോലെ\\
ഘോരയാം നിശാചരി വേഗത്തില്‍ നടകൊണ്ടാള്‍.\\
രാവണന്‍ തന്റെ വരവുണ്ടിനിയിപ്പോളെന്നു\\
ദേവദേവനുമരുള്‍ചെയ്തിരുന്നരുളിനാന്‍.
\end{verse}

%%13_kharavadham
\section{ഖരവധം}

\begin{verse}
രാക്ഷസപ്രവരനായീടിന ഖരന്‍ മുന്‍പില്‍\\
പക്ഷമറ്റവനിയില്‍ പര്‍വതം വീണപോലെ\\
രോദനംചെയ്തു മുമ്പില്‍പ്പതനം ചെയ്ത നിജ-\\
സോദരിതന്നെ നോക്കിച്ചൊല്ലിനാനാശു ഖരന്‍:\\
‘മൃത്യുതന്‍ വക്ത്രത്തിങ്കല്‍ സത്വരം പ്രവേശിച്ച-\\
തത്രചൊല്ലാരെന്നെന്നോടെത്രയും വിരയേ നീ.’\\
വീര്‍ത്തുവീര്‍ത്തേറ്റം വിറച്ചലറിസ്സഗദ്ഗദ-\\
മാര്‍ത്തിപൂണ്ടോര്‍ത്തു ഭീത്യാ ചൊല്ലിനാളവളപ്പോള്‍:\\
‘മര്‍ത്ത്യന്‍മാര്‍ ദശരഥപുത്രന്മാരിരുവരു-\\
ണ്ടുത്തമഗുണവാന്മാരെത്രയും പ്രസിദ്ധന്മാര്‍\\
രാമലക്ഷ്മണന്മാരെന്നവര്‍ക്കു നാമമൊരു\\
കാമിനിയുണ്ടു കൂടെ സീതയെന്നവള്‍ക്കു പേര്‍.\\
അഗ്രജന്‍ നിയോഗത്താലുഗ്രനാമവരജന്‍\\
ഖഡ്ഗേന ഛേദിച്ചിതു മല്‍ക്കുചാദികളെല്ലാം\\
ശൂരനായീടും നീയിന്നവരെക്കൊലചെയ്തു\\
ചോരനല്കുക ദാഹം തീരുമാറെനിക്കിപ്പോള്‍.\\
പച്ചമാസവും തിന്നു രക്തവും പാനം ചെയ്കി-\\
ലിച്ഛവന്നീടും മമ നിശ്ചയമറിഞ്ഞാലും.’\\
എന്നിവ കേട്ടു ഖരന്‍ കോപത്തോടുരചെതാന്‍:\\
“ദുര്‍ന്നയമേറെയുള്ള മാനുഷാധമന്മാരെ\\
കൊന്നു മല്‍ഭഗിനിക്കു ഭക്ഷിപ്പാന്‍ കൊടുക്കണ-\\
മെന്നതിനാശു പതിന്നാലുപേര്‍ പോക നിങ്ങള്‍\\
നീ കൂടെച്ചെന്നു കാട്ടിക്കൊടുത്തിടെന്നാലിവ-\\
രാകൂതം വരുത്തീടും നിനക്കു മടിയാതെ.”\\
എന്നവളോടു പറഞ്ഞയച്ചാന്‍ ഖരനേറ്റ-\\
മുന്നതന്മാരാം പതിന്നാലു രാക്ഷസരെയും\\
ശൂലമുദ്ഗരമുസലാസിചാപേഷു ഭിണ്ഡി-\\
പാലാദി പലവിധമായുധങ്ങളുമായി\\
ക്രുദ്ധന്മാരാര്‍ത്തുവിളിച്ചുദ്ധതന്മാരായ്ച്ചെന്നു\\
യുദ്ധസന്നദ്ധന്മാരായടുത്താരതുനേരം.\\
ബദ്ധവൈരേണ പതിന്നാല്‍വരുമൊരുമിച്ചു\\
ശസ്ത്രൗഘം പ്രയോഗിച്ചാര്‍ ചുറ്റും നിന്നൊരിക്കലേ\\
മിത്രഗോത്രോല്‍ഭൂതനാമുത്തമോത്തമന്‍ രാമന്‍\\
ശത്രുക്കളയച്ചോരു ശസ്ത്രൗഘം വരുന്നേരം\\
പ്രത്യേകമോരോ ശരംകൊണ്ടവ ഖണ്ഡിച്ചുടന്‍\\
പ്രത്രര്‍ഥിജനത്തെയും വധിച്ചാനോരോന്നിനാല്‍.\\
ശൂര്‍പ്പണഖയുമതു കണ്ടുപേടിച്ചു മണ്ടി-\\
ബ്ബാഷ്പവും തൂകി ഖരന്‍ മുമ്പില്‍ വീണലറിനാള്‍.\\
‘എങ്ങുപൊയ്ക്കളഞ്ഞിതു നിന്നോടു കൂടെപ്പറ-\\
ഞ്ഞിങ്ങു നിന്നയച്ചവര്‍ പതിന്നാല്‍വരും ചൊല്‍ നീ’\\
‘അങ്ങു ചെന്നേറ്റനേരം രാമസായകങ്ങള്‍കൊ-\\
ണ്ടിങ്ങിനി വരാതവണ്ണം പോയാര്‍ തെക്കോട്ടവര്‍’\\
എന്നു ശൂര്‍പ്പണഖയും ചൊല്ലിനാളതു കേട്ടു\\
വന്ന കോപത്താല്‍ ഖരന്‍ ചൊല്ലിനാനതുനേരം:\\
‘പോരിക നിശാചരര്‍ പതിന്നാലായിരവും\\
പോരിനു ദൂഷണനുമനുജന്‍ ത്രിശിരസ്സും’\\
ഘോരനാം ഖരനേവം ചൊന്നതുകേട്ടനേരം\\
ശൂരനാം ത്രിശിരസ്സും പടയും പുറപ്പെട്ടു.\\
വീരനാം ദൂഷണനും ഖരനും നടകൊണ്ടു\\
ധീരതയോടു യുദ്ധം ചെയ്വതിനുഴറ്റോടെ.\\
രാക്ഷസപ്പടയുടെ രൂക്ഷമാം കോലാഹലം\\
കേള്‍ക്കായനേരം രാമന്‍ ലക്ഷ്മണനോടു ചൊന്നാന്‍:\\
‘ബ്രഹ്മാംണ്ഡം നടുങ്ങുമാരെന്തൊരു ഘോഷമിതു?\\
നമ്മോടു യുദ്ധത്തിനു വരുന്നു രക്ഷോബലം\\
ഘോരമായിരിപ്പൊരു യുദ്ധവുമുണ്ടാമിപ്പോള്‍\\
ധീരതയോടുമത്ര നീയൊരു കാര്യം വേണം.\\
മൈഥിലിതന്നെയൊരു ഗുഹയിലാക്കിക്കൊണ്ടു\\
ഭീതികൂടാതെ പരിപാലിക്കവേണം ഭവാന്‍.\\
ഞാനൊരുത്തനേ പോരുമിവരെയൊക്കെക്കൊല്‍വാന്‍\\
മാനസേ നിനക്കു സന്ദേഹമുണ്ടായീടൊലാ.\\
മറ്റൊന്നും ചൊല്ലുന്നില്ലെന്നെന്നെയാണയുമിട്ടു\\
കറ്റവാര്‍കുഴലിയെ രക്ഷിച്ചുകൊള്ളേണം നീ’\\
ലക്ഷ്മീദേവിയേയുംകൊണ്ടങ്ങനെ തന്നെയെന്നു\\
ലക്ഷ്മണന്‍ തൊഴുതുപോയ് ഗഹ്വരമകം പുക്കാന്‍\\
ചാപബാണങ്ങളേയുമെടുത്തു പരികര-\\
മാഭോഗാനന്ദമുറപ്പിച്ചു സന്നദ്ധനായി\\
നില്ക്കുന്ന നേരമാര്‍ത്തു വിളിച്ചു നക്തഞ്ചര-\\
രൊക്കെ വന്നൊരുമിച്ചു ശസ്ത്രൗഘം പ്രയോഗിച്ചാര്‍\\
വൃക്ഷങ്ങള്‍ പാഷാണങ്ങളെന്നിവകൊണ്ടുമേറ്റം\\
പ്രക്ഷേപിച്ചിതു വേഗാല്‍ പുഷ്കരനേത്രന്‍മെയ്മേല്‍.\\
തല്‍ക്ഷണമവയെല്ലാമെയ്തു ഖണ്ഡിച്ചു രാമന്‍\\
രാക്ഷോവീരന്മാരെയും സായകാവലി തൂകി\\
നിഗ്രഹിച്ചിതു നിശിതാഗ്രബാണങ്ങള്‍ തന്നാ-\\
ലഗ്രേ വന്നടുത്തോരു രാക്ഷസപ്പടയെല്ലാം.\\
ഉഗ്രനാം സേനാപതി ദൂഷണനതുനേര-\\
മുഗ്രസന്നിഭനായ രാമനോടടുത്തിതു\\
തൂകിനാന്‍ ബാണഗണ, മവറ്റെ രഘുവരന്‍\\
വേഗേന ശരങ്ങളാലെണ്മണിപ്രായമാക്കി\\
നാലുബാണങ്ങളെയ്തു തുരഗം നാലിനെയും\\
കാലവേശ്മനി ചേര്‍ത്തു സാരഥിയോടും കൂടെ.\\
ചാപവും മുറിച്ചു തല്‍കേതുവും കളഞ്ഞപ്പോള്‍\\
കോപേന തേരില്‍നിന്നു ഭൂമിയില്‍ ചാടിവീണാന്‍\\
പില്‍പ്പാടു ശതഭാരായസനിര്‍മിതമായ\\
കേല്‍പ്പേറും പരിഘവു ധരിച്ചു വന്നാനവന്‍.\\
തല്‍ബാഹുതന്നെ ഛേദിച്ചീടിനാന്‍ ദാശരഥി\\
തല്‍പരിഘത്താല്‍ പ്രഹരിച്ചിതു സീതാപതി.\\
മസ്തകം പിളര്‍ന്നവനുര്‍വിയില്‍ വീണു സമാ-\\
വര്‍ത്തിപത്തനം പ്രവേശിച്ചിതു ദൂഷണനും.\\
ദൂഷണന്‍ വീണ നേരം വീരനാം ത്രിശിരസ്സും\\
രോഷേണ മൂന്നു ശരംകൊണ്ടു രാമനെയെയ്താന്‍\\
മൂന്നും ഖണ്ഡിച്ചു രാമന്‍ മൂന്നു ബാണങ്ങളെയ്താന്‍\\
മൂന്നുമെയ്തുടന്‍ മുറിച്ചീടിനാന്‍ ത്രിശിരസ്സും\\
നൂറു ബാണങ്ങളെയ്താനന്നേരം ദാശരഥി\\
നൂറും ഖണ്ഡിച്ചു പുനരായിരം ബാണമെയ്താന്‍.\\
അവയും മുറിച്ചവനയുതം ബാണമെയ്താ-\\
നവനീപതി വീരനവയും നുറുക്കിനാന്‍.\\
അര്‍ധചന്ദ്രാകാരമായിരിപ്പോരമ്പുതന്നാ-\\
ലുത്തമാംഗങ്ങള്‍ മൂന്നും മുറിച്ചു പന്താടിനാന്‍\\
അന്നേരം ഖരനാദിത്യാഭതേടീടും രഥം\\
തന്നിലാമ്മാറു കരയേറി ഞാണൊലിയിട്ടു\\
വന്നു രാഘവനോടു ബാണങ്ങള്‍ തൂകീടിനാ-\\
നൊന്നിനൊന്നെയ്തു മുറിച്ചീടിനാനവയെല്ലാം.\\
രാമബാണങ്ങള്‍കൊണ്ടും ഖരബാണങ്ങള്‍ കൊണ്ടും\\
ഭൂമിയുമാകാശവും കാണരുതാതെയായി.\\
നിഷ്ഠുരതരമായ രാഘവശരാസനം\\
പിട്ടിച്ചാന്‍ മുഷ്ടിദേശേ ബാണമെയ്താശു ഖരന്‍\\
ചട്ടയും നുറുക്കിനാന്‍ ദേഹവും ശരങ്ങള്‍ കൊ-\\
ണ്ടൊട്ടൊഴിയാതെ പിളര്‍ന്നീടിനാനതുനേരം.\\
താപസദേവാദികളായുള്ള സാധുക്കളും\\
താപമോടയ്യോ കഷ്ടം കഷ്ടമെന്നുര ചെയ്താര്‍.\\
ജയിപ്പൂതാക രാമന്‍ ജയിപ്പൂതാകയെന്നു\\
ഭയത്തോടമരരും താപസന്മാരും ചൊന്നാര്‍.\\
തല്കാലേ കുംഭോത്ഭവന്‍ തന്നുടെ കൈയില്‍ മുന്നം\\
ശക്രനാല്‍ നിക്ഷിപ്തമായിരുന്ന ശരാസനം\\
തൃക്കൈയില്‍ കാണായ്വന്നിതെത്രയും ചിത്രം ചിത്രം!\\
മുഖ്യവൈഷ്ണവചാപം കൈക്കൊണ്ടുനില്ക്കുന്നേരം\\
ദിക്കുകളൊക്കെ നിറഞ്ഞോരു വൈഷ്ണവതേജ-\\
സ്സുള്‍ക്കൊണ്ടു കാണായ്വന്നു രാമചന്ദ്രനെയപ്പോള്‍.\\
ഖണ്ഡിച്ചാന്‍ ഖരനുടെ ചാപവും കവചവും\\
കുണ്ഡല ഹാരകിരീടങ്ങളുമരക്ഷണാല്‍\\
സൂതനെക്കൊന്നു തുരഗങ്ങളും തേരും പൊടി-\\
ച്ചാതിനായകനടുത്തീടിന നേരത്തിങ്കല്‍\\
മറ്റൊരു തേരില്‍ കരയേറിനാനാശു ഖരന്‍\\
തെറ്റെന്നു പൊടിച്ചിതു രാഘവനതുമപ്പോള്‍.\\
പിന്നെയും ഗദയുമായടുത്താനാശു ഖരന്‍\\
ഭിന്നമാക്കിനാന്‍ വിശിഖങ്ങളാലതും രാമന്‍.\\
ഏറിയ കോപത്തോടെ പിന്നെ മറ്റൊരു തേരി-\\
ലേറിവന്നസ്ത്രപ്രയോഗം തുടങ്ങിനാന്‍ ഖരന്‍.\\
ഘോരമാമാഗ്നേയാസ്ത്രമെയ്തതു രഘുവരന്‍\\
വാരുണാസ്ത്രേണ തടുത്തീടിനാന്‍ ജിതശ്രമം\\
പിന്നെക്കൗബേരമസ്ത്രമെയ്തതൈന്ദ്രാസ്ത്രം കൊണ്ടു\\
മന്നവന്‍ തടുത്തതു കണ്ടു രാക്ഷസവീരന്‍\\
നൈര്യതമസ്ത്രം പ്രയോഗിച്ചിതു യാമ്യാസ്ത്രേണ\\
വീരനാം രഘുപതി തടുത്തു കളഞ്ഞപ്പോള്‍\\
വായവ്യമയച്ചതുമൈന്ദ്രാസ്ത്രം കൊണ്ടു ജഗ-\\
ന്നായകന്‍ തടുത്തതു കണ്ടു രാക്ഷസവീരന്‍\\
ഗാന്ധര്‍വമയച്ചതു ഗൗഹ്യകമസ്ത്രം കൊണ്ടു\\
ശാന്മമായതു കണ്ടു ഖരനും കോപത്തോടെ\\
ആസുരമസ്ത്രം പ്രയോഗിച്ചതു കണ്ടു രാമന്‍\\
ഭാസുരമായ ദൈവാസ്ത്രം കൊണ്ടു തടുക്കയാല്‍\\
തീക്ഷ്ണമാമൈഷീകാസ്ത്രമെയ്തതു രഘുപതി\\
വൈഷ്ണവാസ്ത്രേണ കളഞ്ഞാശു മൂന്നമ്പു തന്നാല്‍\\
സാരഥിതന്നെക്കൊന്നു തുരഗങ്ങളെക്കൊന്നു\\
തേരുമെപ്പേരും പൊടിപെടുത്തു കളഞ്ഞപ്പോള്‍\\
യാതുധാനാധിപതി ശൂലവും കൈക്കൊണ്ടതി-\\
ക്രോധേന രഘുവരനോടടുത്തീടന്നേരം\\
ഇന്ദ്രദൈവതമസ്ത്രമയച്ചോരളവു ചെ-\\
ന്നിന്ദ്രാരിതലയറുത്തീടിനാന്‍ ജഗന്നാഥന്‍.\\
വീണിതു ലങ്കാനഗരോത്തരദ്വാരേ തല\\
തൂണി പുക്കിതു വന്നു ബാണവുമതുനേരം.\\
കണ്ടു രാക്ഷസരെല്ലാമാരുടെ തലെയെന്നു\\
കുണ്ഠഭാവേന നിന്നു സംശയം തുടങ്ങിനാര്‍.\\
ഖരദൂഷണ ത്രിശിരാക്കളാം നിശാചര-\\
വരരും പതിന്നാലായിരവും മരിച്ചിതു\\
നാഴിക മൂന്നേമുക്കാല്‍കൊണ്ടു രാഘവന്‍ തന്നാ-\\
ലൂഴിയില്‍ വീണാളല്ലോ രാവണഭഗിനിയും.\\
മരിച്ച നിശാചരര്‍ പതിന്നാലായിരവും\\
ധരിച്ചാരല്ലോ ദിവ്യവിഗ്രഹമതുനേരം\\
ജ്ഞാനവും ലഭിച്ചിതു രാഘവന്‍പക്കല്‍ നിന്നു\\
മാനസേ പുനരവരേവരുമതുനേരം\\
രാമനെ പ്രദക്ഷിണംചെയ്തുടന്‍ നമസ്കരി-\\
ച്ചാമോദം പൂണ്ടുകൂപ്പിസ്തുതിച്ചാര്‍ പലതരം:\\
‘നമസ്തേ പാദാംബുജം രാമ! ലോകാഭിരാമ!\\
സമസ്തപാപഹരം സേവകാഭീഷ്ടപ്രദം\\
സമസ്തേശ്വര! ദയാവാരിധേ! രഘുപതേ!\\
രമിച്ചീടണം ചിത്തം ഭവതി രമാപതേ!\\
ത്വല്‍പാദാംബുജം നിത്യം ധ്യാനിച്ചുമുനിജന-\\
മുത്ഭവമരണദുഃഖങ്ങളെക്കളയുന്നു.\\
മുല്പാടു മഹേശനെത്തപസ്സുചെയ്തു സന്തോ-\\
ഷിപ്പിച്ചു ഞങ്ങള്‍ മുമ്പില്‍ പ്രത്യക്ഷനായ നേരം,\\
‘ഭേദവിഭ്രമം തീര്‍ത്തു സംസാരവൃക്ഷമൂല-\\
ച്ഛേദന കുഠാരമായ് ഭവിക്ക ഭവാ’നിതി\\
പ്രാര്‍ഥിച്ചു ഞങ്ങള്‍ മഹാദേവനോടതുമൂല-\\
മോര്‍ത്തരുള്‍ചെയ്തു പരമേശ്വരനതു നേരം:\\
യാവിനീചരന്മാരായ് ജനിക്ക നിങ്ങളിനി\\
രാമനായവതരിച്ചീടുവന്‍ ഞാനും ഭൂമൗ.\\
രാക്ഷസദേഹന്മാരാം നിങ്ങളെ ഛേദിച്ചന്നു\\
മോക്ഷവും തന്നീടുവനില്ല സംശയമേതും.’\\
എന്നരുള്‍ചെയ്തു പരമേശ്വരനതുമൂലം\\
നിര്‍ണയം മഹാദേവനായതും രഘുപതി.\\
ജ്ഞാനോപദേശം ചെയ്തു മോക്ഷവു തന്നീടണ-\\
മാനന്ദസ്വരൂപനാം നിന്തിരുവടി നാഥാ!’\\
എന്നവരപേക്ഷിച്ച നേരത്തു രഘുവരന്‍\\
മന്ദഹാസവും പൂണ്ടു സാനന്ദമരുള്‍ചെയ്തു:\\
‘വിഗ്രഹേന്ദ്രിയ മനഃപ്രാണാഹങ്കാരാദികള്‍-\\
ക്കൊക്കവേ സാക്ഷിഭൂതനായതു പരമാത്മാ\\
ജാഗ്രത്സ്വപ്നാഖ്യാദ്യവസ്ഥാഭേദങ്ങള്‍ക്കും മീതേ\\
സാക്ഷിയാം പരബ്രഹ്മം സച്ചിദാനന്ദമേകം.\\
ബാല്യകൗമാരാദികളാഗമാപായികളാം\\
കാല്യാദിഭേദങ്ങള്‍ക്കും സാക്ഷിയായ് മീതേ നില്ക്കും\\
പരമാത്മാവു പരബ്രഹ്മമാനന്ദാത്മകം\\
പരമം ധ്യാനിക്കുമ്പോള്‍ കൈവല്യം വന്നുകൂടും.’\\
ഈവണ്ണമുപദേശം ചെയ്തു മോക്ഷവും നല്കി\\
ദേവദേവേശന്‍ ജഗല്‍കാരണന്‍ ദാശരഥി\\
രാഘവന്‍ മൂന്നേമുക്കാല്‍ നാഴികകൊണ്ടു കൊന്നാന്‍\\
വേഗേന പതിന്നാലു സഹസ്രം രക്ഷോബലം.\\
സൗമിത്രി സീതാദേവിതന്നോടു കൂടെ വന്നു\\
രാമചന്ദ്രനെ വീടു നമസ്കാരവും ചെയ്താന്‍.\\
ശസ്തൗഘനികൃത്തമാം ഭര്‍ത്തൃവിഗ്രഹം കണ്ടു\\
മുക്തബാഷ്പോദം വിദേഹാത്മജ മന്ദം മന്ദം\\
തൃക്കൈകള്‍ കൊണ്ടു തലോടിപ്പൊറുപ്പിച്ചീടിനാ-\\
ളൊക്കവേ പുണ്ണുമതിന്‍ വടുവും മാച്ചീടിനാള്‍.\\
രക്ഷോവീരന്മാര്‍ വീണുകിടക്കുന്നതു കണ്ടു\\
ലക്ഷ്മണന്‍ നിജഹൃദി വിസ്മയം തേടീടിനാന്‍.\\
‘രാവണന്‍തന്റെ വരവുണ്ടിനിയിപ്പോ’ളെന്നു\\
ദേവദേവനുമരുള്‍ചെയ്തിരുന്നരുളിനാന്‍.\\
പിന്നെ ലക്ഷ്മണന്‍തന്നെ വൈകാതെ നിയോഗിച്ചാന്‍:\\
‘ചെന്നു നീ മുനിവരന്മാരോടു ചൊല്ലീടണം\\
യുദ്ധം ചെയ്തതും ഖരദൂഷണത്രിശിരാക്കള്‍\\
സിദ്ധിയെ പ്രാപിച്ചതും പതിന്നാലായിരവും\\
താപസന്മാരോടറിയിച്ചു നീ വരികെ’ന്നു\\
പാപനാശനനരുള്‍ചെയ്തയച്ചോരു ശേഷം\\
സുമിത്രാപുത്രന്‍ തപോധനന്മാരോടു ചൊന്നാ-\\
നമിത്രാന്തകന്‍ ഖരന്‍ മരിച്ചവൃത്താന്തങ്ങള്‍.\\
ക്രമത്താലിനിക്കാലം വൈകാതെയൊടുങ്ങീടു-\\
മമര്‍ത്ത്യവൈരികളെന്നുറച്ചു മിനിജനം\\
പലരുംകൂടി നിരൂപിച്ചു നിര്‍മിച്ചീടിനാര്‍\\
പലലാശികള്‍മായ തട്ടായ്വാന്‍ മൂന്നുപേര്‍ക്കും\\
അംഗുലീയവും ചൂഡാരത്നവും കവചവു-\\
മംഗേ ചേര്‍ത്തീടുവാനായ്ക്കൊടുത്തു വിട്ടീടിനാര്‍.\\
ലക്ഷ്മണനവ മൂന്നും കൊണ്ടുവന്നാശു രാമന്‍-\\
തൃക്കാല്ക്കല്‍വെച്ചു തൊഴുതീടിനാന്‍ ഭക്തിയോടെ.\\
അംഗുലീയകമെടുത്തംബുജവിലോചന-\\
നംഗുലത്തിന്മേലിട്ടു ചൂഡാരത്നവും പിന്നെ\\
മൈഥിലിതനിക്കു നല്കീടിനാന്‍, കവചവും\\
ഭ്രാതാവുതനിക്കണിഞ്ഞീടുവാനരുളിനാന്‍.
\end{verse}

%%14_shoorpanakhaavilaapam
\section{ശൂര്‍പ്പണഖാവിലാപം}

\begin{verse}
രാവണഭഗിനിയും രോദനംചെയ്തു പിന്നെ\\
രാവണനോടു പറഞ്ഞീടുവാന്‍ നടകൊണ്ടാള്‍.\\
സാക്ഷാലഞ്ജനശൈലം പോലെ ശൂര്‍പ്പണഖയും\\
രാക്ഷസരാജന്‍ മുമ്പില്‍ വീണുടന്‍ മുറയിട്ടാള്‍.\\
മുലയും മൂക്കും കാതും കൂടാതെ ചോരയുമാ-\\
യലറും ഭഗിനിയോടവനുമുരചെയ്താന്‍:\\
‘എന്തിതു വത്സേ! ചൊല്ലീടെന്നോടു പരമാര്‍ത്ഥം\\
ബന്ധമുണ്ടായതെന്തു വൈരൂപ്യം വന്നീടുവാന്‍?’\\
ശക്രനോ കൃതാന്തനോ പാശിയോ കുബേരനോ\\
ദുഷ്കൃതംചെയ്തതവന്‍തന്നെ ഞാനൊടുക്കുവന്‍.\\
സത്യംചൊല്ലെ’ന്ന നേരമവളുമുരചെയ്താ-\\
‘ളെത്രയും മൂഢന്‍ ഭവാന്‍ പ്രമത്തന്‍ പാനസക്തന്‍\\
സ്ത്രീജിതനതിശഠനെന്തറിഞ്ഞിരിക്കുന്നു?\\
രാജാവെന്നന്തുകൊണ്ടു ചൊല്ലുന്നു നിന്നെ വൃഥാ?\\
ചാരചക്ഷുസ്സം വിചാരവുമില്ലേതും നിത്യം\\
നാരീസേവയും ചെയ്തു കിടന്നീടെല്ലായ്പോഴും\\
കേട്ടതില്ലയോ ഖരദൂഷണത്രിശിരാക്കള്‍\\
കൂട്ടമേ പതിന്നാലായിരവും മുടിഞ്ഞതും?\\
പ്രഹരാര്‍ദ്ധേന രാമന്‍ വേഗേന ബാണഗണം\\
പ്രഹരിച്ചൊടുക്കിനാനെന്തൊരു കഷ്ടമോര്‍ത്താല്‍!\\
എന്നതു കേട്ടു ചോദിച്ചീടിനാന്‍ ദശാനനന്‍\\
‘എന്നോടു ചൊല്ലീടേവന്‍ രാമനാകുന്നതെന്നും\\
എന്തൊരു മൂലമവന്‍ കൊല്ലുവാനെന്നുമെന്നാ-\\
ലന്തകന്‍തനിക്കുനല്കീടുവനവനെ ഞാന്‍.’\\
സോദരി ചൊന്നാളതു കേട്ടു രാവണനോടു:\\
‘യാതുധാനാധിപതേ! കേട്ടാലും പരമാര്‍ഥം.\\
ഞാനൊരു ദിനം ജനസ്താനദേശത്തിങ്കല്‍ നി-\\
ന്നാനന്ദംപൂണ്ടു താനേ സഞ്ചരിച്ചീടും കാലം\\
കാനനത്തൂടേചെനു ഗൗതമീതടം പുക്കേന്‍;\\
സാനന്ദം പഞ്ചവടി കണ്ടു ഞാന്‍ നില്ക്കുന്നേരം\\
ആശ്രമത്തിങ്കല്‍ തത്ര രാമനെക്കണ്ടേന്‍ ജഗ-\\
ദാശ്രയഭൂതന്‍ ജടാവല്‍ക്കലങ്ങളും പൂണ്ടു\\
ചാപബാണങ്ങളോടുമെത്രയും തേജസ്സോടും\\
താപസവേഷത്തോടും ധര്‍മദാരങ്ങളോടും\\
സോദരനായീടുന്ന ലക്ഷ്മണനോടും കൂടി-\\
സ്സാദരമിരിക്കുമ്പോളടുത്തുചെന്നു ഞാനും.\\
ശ്രീരാമോത്സംഗേ വാഴും ഭാമിനിതന്നെക്കണ്ടാല്‍\\
നാരികളവ്വണ്ണം മറ്റില്ലല്ലോ ലോകത്തിങ്കല്‍\\
ദേവഗന്ധര്‍വ നാഗമാനുഷനാരിമാരി-\\
ലേവം കാണ്മാനുമില്ല കേള്‍പ്പാനുമില്ല നൂനം\\
ഇന്ദിരാദേവിതാനും ഗൗരിയും വാണിമാതു-\\
മിന്ദ്രാണിതാനും മറ്റുള്ളപ്സരസ്ത്രീവര്‍ഗവും\\
നാണംപൂണ്ടൊളിച്ചീടുമവളെ വഴിപോലെ\\
കാണുമ്പോളനംഗനും ദേവതയവളല്ലോ.\\
തല്‍പതിയാകും പുരുഷന്‍ ജഗല്‍പതിയെന്നു\\
കല്പിക്കാം വികല്പമില്ലല്പവുമിതിനിപ്പോള്‍.\\
ത്വല്‍പത്നിയാക്കീടുവാന്‍ തക്കവളവളെന്നു\\
കല്പിച്ചുകൊണ്ടിങ്ങു പോന്നീടുവാനൊരുമ്പെട്ടേന്‍.\\
മല്‍കുചനാസാകര്‍ണച്ഛേദനം ചെയ്താനപ്പോള്‍\\
ലക്ഷ്മണന്‍ കോപത്തോടെ രാഘവനിയോഗത്താല്‍.\\
വൃത്താന്തം ഖരനോടു ചെന്നു ഞാനറിയിച്ചേന്‍\\
യുദ്ധാര്‍ത്ഥം നക്തഞ്ചരാനീകിനിയോടുമവന്‍\\
രോഷവേഗേന ചെന്നു രാമനോടേറ്റ നേരം\\
നാഴികമൂന്നെമുക്കാല്‍കൊണ്ടവനൊടുക്കിനാന്‍.\\
ഭസ്മമാക്കീടും പിണങ്ങീടുകില്‍ വിശ്വം ക്ഷണാല്‍\\
വിസ്മയം രാമനുടെ വിക്രമം വിചാരിച്ചാല്‍!\\
കന്നല്‍നേര്‍മിഴിയാളാം ജാനകീദേവിയിപ്പോള്‍\\
നിന്നുടെ ഭാര്യയാകില്‍ ജന്മസാഫല്യം വരും.\\
ത്വത്സകാശത്തിങ്കലാക്കീടുവാന്‍തക്കവണ്ണ-\\
മുത്സാഹം ചെയ്തീടുകിലെത്രയും നന്നു ഭവാന്‍.\\
തത്സാമര്‍ത്ഥ്യങ്ങളെല്ലാം പത്മാക്ഷിയാകുമവ-\\
ളുത്സംഗേ വസിക്കകൊണ്ടാകുന്നു ദേവാരാതേ!\\
രാമനോടേറ്റാല്‍ നില്പാന്‍ നിനക്കു ശക്തിപോരാ\\
കാമവൈരിക്കും നേരേ നില്ക്കരുതെതിര്‍ക്കുമ്പോള്‍\\
മോഹിപ്പിച്ചൊരുജാതി മായയാ ബാലന്മാരെ,\\
മോഹനഗാത്രിതന്നെക്കൊണ്ടുപോരികേയുള്ളൂ.’\\
സോദരീവചനങ്ങളിങ്ങനെ കേട്ടശേഷം\\
സാദരവാക്യങ്ങളാലാശ്വസിപ്പിച്ചു തൂര്‍ണം\\
തന്നുടെ മണിയറതന്നിലങ്ങകം പുക്കാന്‍\\
വന്നതില്ലേതും നിദ്ര ചിന്തയുണ്ടാകമൂലം.\\
‘എത്രയും ചിത്രം ചിത്രമോര്‍ത്തോളമിദമൊരു\\
മര്‍ത്ത്യനാല്‍ മൂന്നേമുക്കാല്‍ നാഴികനേരം കൊണ്ടു\\
ശക്തനാം നക്തഞ്ചരപ്രവരന്‍ ഖരന്‍താനും\\
യ്ദ്ധവൈദ്ഗ്ദ്ധ്യമേറും സോദരരിരുവരും\\
പത്തികള്‍ പതിന്നാലായിരവും മുടിഞ്ഞുപോല്‍!\\
വ്യക്തം മാനുഷനല്ല രാമനെന്നതു നൂനം.\\
ഭക്തവത്സലനായ ഭഗവാന്‍ പത്മേക്ഷണന്‍\\
മുക്തിദാനൈകമൂര്‍ത്തി മുകുന്ദന്‍ ഭക്തപ്രിയന്‍\\
ധാതാവു മുന്നം പ്രാര്‍ഥിച്ചോരു കാരണമിന്നു\\
ഭൂതലേ രഘുകുലേ മര്‍ത്ത്യനായ് പിറന്നിപ്പോള്‍\\
എന്നെക്കൊല്ലുവാനൊരുമ്പെട്ടുവന്നാനെങ്കിലോ\\
ചെന്നു വൈകുണ്ഠരാജ്യം പരിപാലിക്കാമല്ലോ.\\
അല്ലെങ്കിലെന്നും വാഴാം രാക്ഷസരാജ്യമെന്നാ-\\
ലല്ലലില്ലൊന്നുകൊണ്ടും മനസി നിരൂപിച്ചാല്‍,\\
കല്യാണപ്രദനായ രാമനോടേല്ക്കുന്നതി-\\
നെല്ലാജാതിയും മടിക്കേണ്ട ഞാനൊന്നുകൊണ്ടും.’\\
ഇത്ഥമാത്മനി ചിന്തിച്ചുറച്ചു രക്ഷോനാഥന്‍\\
തത്ത്വജ്ഞാനത്തോടുകൂടത്യാനന്ദവും പൂണ്ടാന്‍.\\
സാക്ഷാല്‍ ശ്രീനാരായണന്‍ രാമനെന്നറിഞ്ഞഥ\\
രാക്ഷസപ്രവരനും പൂര്‍വവൃത്താന്തമോര്‍ത്താന്‍:\\
‘വിദ്വേഷബുദ്ധ്യാ രാമന്‍ തന്നെ പ്രാപിക്കേയുള്ളൂ\\
ഭക്തികഒണ്ടെന്നില്‍ പ്രസാകിക്കയില്ലഖിലേശന്‍.’
\end{verse}

%%15_raavanamaareechasamvaadam
\section{രാവണമാരീചസംവാദം}

\begin{verse}
ഇത്തരം നിരൂപിച്ചു രാത്രിയും കഴിഞ്ഞിതു\\
ചിത്രഭാനുവുമുദയാദ്രിമൂര്‍ദ്ധനി വന്നു\\
തേരതിലേറീടിനാന്‍ ദേവസഞ്ചയവൈരി\\
പാരാതെ പാരാവാരപാരമാം തീരം തത്ര\\
മാരീചാശ്രമം പ്രാപിച്ചീടിനാനതിദ്രുതം\\
ഘോരനാം ദശാനനന്‍ കാര്യഗൗരവത്തോടും.\\
മൗനവും പൂണ്ടു ജടാവല്ക്കലാദിയും ധരി-\\
ച്ചാനന്ദാത്മകനായ രാമനെ ധ്യാനിച്ചുള്ളില്‍\\
രാമരാമേതി ജപിച്ചുറച്ചു സമാധിപൂ-\\
ണ്ടാമോദത്തോടു മരുവീടിന മാരീചനും\\
ലൗകികാത്മനാ ഗൃഹത്തിങ്കലാഗതനായ\\
ലോകോപദ്രവകാരിയായ രാവണന്‍തന്നെ\\
കാണ്ടു സംഭ്രമത്തോടുമുത്ഥാനംചെയ്തു പൂണ്ടു-\\
കൊണ്ടു തന്മാറിലണച്ചാനന്ദാശ്രുക്കളോടും\\
പൂജിച്ചു യഥാവിധി മാനിച്ചു ദശകണ്ഠന്‍\\
യോജിച്ചു ചിത്തമപ്പോള്‍ ചോദിച്ചു മാരീചനും:\\
‘എന്തൊരാഗമനമിതേകനായ്ത്തന്നെയൊരു\\
ചിന്തയുണ്ടെന്നപോലെ തോന്നുന്നു ഭാവത്തിങ്കല്‍\\
ചൊല്ലുക രഹസ്യമല്ലെങ്കിലോ ഞാനും തവ\\
നല്ലതു വരുത്തുവാനുള്ളോരില്‍ മുമ്പനല്ലോ.\\
ന്യായമായ് നിഷ്കല്മഷമായിരിക്കുന്ന കാര്യം\\
മായമെന്നിയേ ചെയ്വാന്‍ മടിയില്ലെനിക്കേതും.’\\
മാരീചവാക്യമേവം കേട്ടു രാവണന്‍ ചൊന്നാ-\\
‘നാരുമില്ലെനിക്കു നിന്നെപ്പോലെ മുട്ടുന്നേരം\\
സാങ്കേതാധിപനായ രാജാവു ദശരഥന്‍\\
ലോകൈകാധിപനുടെ പുത്രന്മാരായുണ്ടുപോല്‍\\
രാമലക്ഷ്മണന്മാരെന്നിരുവരിതുകാലം\\
കോമളഗാത്രിയായോരംഗനാരത്നത്തോടും\\
ദണ്ഡകാരണ്യേ വന്നു വാഴുന്നിതവര്‍ ബലാ-\\
ലെന്നുടെ ഭഗിനിതന്‍ നാസികാകുചങ്ങളും\\
കര്‍ണവും ഛേദിച്ചതു കേട്ടുടന്‍ ഖരാദികള്‍\\
ചെന്നിതു പതിന്നാലായിരവുമവരേയും\\
നിന്നുതാനേകനായിട്ടെതിര്‍ത്തു രണത്തിങ്കല്‍\\
കൊന്നിതു മൂന്നേമുക്കാല്‍ നാഴിക കൊണ്ടു രാമന്‍.\\
തല്‍പ്രാണേശ്വരിയായ ജാനകിതന്നെ ഞാനു-\\
മിപ്പോഴേ കൊണ്ടിങ്ങു പോന്നീടുവനതിന്നു നീ\\
ഹേമവര്‍ണംപൂണ്ടൊരു മാനായ്ച്ചെന്നടവിയില്‍\\
കാമിനിയായ സീതതന്നെ മോഹിപ്പിക്കണം.\\
രാമലക്ഷ്മണന്മാരെയകറ്റി ദൂരത്താക്കൂ\\
വാമഗാത്രിയെയപ്പോള്‍ കൊണ്ടു ഞാന്‍ പോന്നീടുവന്‍\\
നീ മമ സഹായമായിരിക്കില്‍ മനോരഥം\\
മാമകം സാധിച്ചീടുമില്ല സംശയമേതും.’\\
പംക്തികന്ധരവാക്യം കേട്ടു മാരീചനുള്ളില്‍\\
ചിന്തിച്ചു ഭയത്തോടുമീവണ്ണമുരചെയ്താന്‍:\\
‘ആരുപദേശിച്ചിതു മൂലനാശനമായ\\
കാരിയം നിന്നോടവന്‍ നിന്നുടെ ശത്രുവല്ലോ\\
നിന്നുടെ നാശം വരുത്തീടുവാനവസരം-\\
തന്നെ പാര്‍ത്തിരിപ്പോരു ശത്രുവാകുന്നതവന്‍.\\
നല്ലതു നിനക്കു ഞാന്‍ ചൊല്ലുവന്‍ കേള്‍ക്കുന്നാകില്‍\\
നല്ലതല്ലേതും നിനക്കിത്തൊഴിലറിക നീ.\\
രാമചന്ദ്രനിലുള്ള ഭീതികൊണ്ടകതാരില്‍\\
മാമകേ രാജരത്നരമണിരഥാദികള്‍\\
കേള്‍ക്കുമ്പോളതിഭീതനായുള്ളൂ ഞാനോ നിത്യം;\\
രാക്ഷസവംശം പരിപാലിച്ചുകൊള്‍ക നീയും.\\
ശ്രീനാരായണന്‍ പരമാത്മാവുതന്നെ രാമന്‍\\
ഞാനതിന്‍പരമാര്‍ഥമറിഞ്ഞേന്‍ കേള്‍ക്ക നീയും.\\
നാരദാദികള്‍ മുനിശ്രേഷ്ഠന്മാര്‍ പറഞ്ഞു പ-\\
ണ്ടോരോരോ വൃത്താന്തങ്ങള്‍ കേട്ടേന്‍ പൗലസ്ത്യ പ്രഭോ!\\
പത്മസംഭവന്‍ മുന്നം പ്രാര്‍ഥിച്ച കാലം നാഥന്‍\\
പത്മലോചനനരുള്‍ചെയ്തിതു വാത്സല്യത്താല്‍\\
‘എന്തു ഞാന്‍ വേണ്ടുന്നതു ചൊല്ലുകെ’ന്നതു കേട്ടു\\
ചിന്തിച്ചു വിധാതാവുമര്‍ആഥിച്ചു ’ദയാനിധേ!\\
നിന്തിരുവടിതന്നെ മാനുഷവേഷംപൂണ്ടു\\
പംക്തികന്ധരന്‍തന്നെക്കൊല്ലണം മടിയാതെ.’\\
അങ്ങനെതന്നെയെന്നു സമയം ചെയ്തു നാഥന്‍\\
മംഗലം വരുത്തുവാന്‍ ദേവതാപസര്‍ക്കെല്ലാം.\\
മാനുഷനല്ല രാമന്‍ സാക്ഷാല്‍ ശ്രീനാരായണന്‍-\\
താനേന്നു ധരിച്ചു സേവിച്ചുകൊള്ളുക ഭക്ത്യാ.\\
പോയാലും പുരംപുക്കു സുഖിച്ചു വസിക്ക നീ\\
മായാമാനുഷന്‍തന്നെസ്സേവിച്ചുകൊള്‍ക നിത്യം.\\
എത്രയും പരമകാരുണികന്‍ ജഗന്നാഥന്‍\\
ഭക്തവത്സലന്‍ ഭജനീയനീശ്വരന്‍ നാഥന്‍.’\\
മാരീചന്‍ പറഞ്ഞതുകേട്ടു രാവണന്‍ ചൊന്നാന്‍:\\
‘നേരത്രേ പറഞ്ഞതു നിര്‍മലനല്ലോ ഭവാന്‍.\\
ശ്രീനാരായണസ്വാമി പരമന്‍ പരമാത്മാ-\\
താനരവിന്ദോത്ഭവന്‍തന്നോടു സത്യം ചെയ്തു\\
മര്‍ത്ത്യനായ് പിറന്നെന്നെക്കൊല്ലുവാന്‍ ഭാവിച്ചതു\\
സത്യസങ്കല്പനായ ഭഗവാന്‍ താനെങ്കിലോ\\
പിന്നെയവ്വണ്ണമല്ലെന്നാക്കുവാനാളാരെടോ?\\
നിന്നു നിന്നജ്ഞാനം ഞാനിങ്ങനെയോര്‍ത്തീലൊട്ടും.\\
ഒന്നുകൊണ്ടും ഞാനടങ്ങീടുകയില്ല നൂനം\\
ചെന്നു മൈഥിലിതന്നെ കൊണ്ടുപോരികവേണം.\\
ഉത്തിഷ്ഠമഹാഭാഗ! പൊന്മാനായ് ചമഞ്ഞുചെ-\\
ന്നെത്രയുമകറ്റുക രാമലക്ഷ്മണന്മാരെ.\\
അന്നേരം തേരിലേറ്റിക്കൊണ്ടിങ്ങു പോന്നീടുവന്‍\\
പിന്നെ നീ യഥാസുഖം വാഴുക മുന്നേപ്പോലെ.\\
ഒന്നിനി മറത്തു നീയുരചെന്നുന്നതാകി-\\
ലെന്നുടെ വാള്‍ക്കൂണാക്കീടുന്നതുണ്ടിന്നുതന്നെ.’\\
എന്നതു കേട്ടു വിചാരിച്ചിതു മാരീച്ചനും:\\
‘നന്നല്ല ദുഷ്ടായുധമേറ്റു നിര്യാണം മന്നാല്‍\\
ചെന്നുടന്‍ നരകത്തില്‍ വീണുടന്‍ കിടക്കണം\\
പുണ്യസഞ്ചയം കൊണ്ടു മുക്തനായ് വരുമല്ലോ\\
രാമസായകമേറ്റുമരിച്ചാ’ലെന്നു ചിന്തി-\\
ച്ചാമോദം പൂണ്ടു പുറപ്പെട്ടലുമെന്നു ചൊന്നാന്‍:\\
‘രാക്ഷസരാജ! ഭവാനാജ്ഞാപിച്ചാലുമെങ്കില്‍\\
സാക്ഷാല്‍ ശ്രീരാമന്‍ പരിപാലിച്ചുകൊള്‍ക പോറ്റീ!’\\
എന്നുരചെയ്തു വിചിത്രാകൃതി കലര്‍ന്നൊരു\\
പൊന്‍ നിറമായുള്ളൊരു മൃഗവേഷവും പൂണ്ടാന്‍.\\
പംക്തികന്ധരന്‍ തേരിലാമ്മാറു കരേറിനാന്‍\\
ചെന്താര്‍ബാണനും തേരിലേറിനാനതു നേരം.\\
ചെന്താര്‍മാനിനിയായ ജാനകിതന്നെയുള്ളില്‍\\
ചിന്തിച്ചു ദശാസ്യനുമന്ധനായ് ചമഞ്ഞിതു.\\
മാരീചന്‍ മനോഹരമായൊരു പൊന്മാനായി\\
ചാരുപുള്ളികള്‍ വെള്ളികൊണ്ടു നേത്രങ്ങള്‍ രണ്ടും\\
നീലക്കല്‍കൊണ്ടു ചേര്‍ത്തു മുഗ്ദ്ധഭാവത്തോടോരോ\\
ലീലകള്‍ കാട്ടിക്കാട്ടിക്കാടിലുള്‍പ്പുക്കും പിന്നെ\\
വേഗേന പുറപ്പെട്ടും തുള്ളിച്ചാടിയുമനു-\\
രാഗഭാവേന ദൂരെപ്പോയ്നിന്നു കടാക്ഷിച്ചും\\
രാഘവാശ്രമസ്ഥലോപാന്തേ സഞ്ചരിക്കുമ്പോള്‍\\
രാകേന്ദുമുഖി സീത കണ്ടു വിസ്മയം പൂണ്ടാള്‍\\
രാവണവിചേഷ്ടിതമറിഞ്ഞു രഘുനാഥന്‍\\
ദേവിയോടരുള്‍ചെയ്താനേകാന്തേ,’കാന്തേ! കേള്‍ നീ\\
രക്ഷോനായകന്‍ നിന്നെക്കൊണ്ടു പോവതിനിപ്പോള്‍\\
ഭിക്ഷുരൂപേണ വരുമന്തികേ ജനകജേ!\\
നീയൊരു കാര്യം വേണമതിനു മടിയാതെ\\
മായാസീതയെപ്പര്‍ണശാലയില്‍ നിര്‍ത്തീടണം.\\
വഹ്നിമണ്ഡലത്തിങ്കല്‍ മറഞ്ഞുവസിക്ക നീ\\
ധന്യേ! രാവണവധം കഴിഞ്ഞു കൂടുവോളം.\\
ആശ്രയാശങ്കലോരാണ്ടിരുന്നീടണം ജഗ-\\
ദാശ്രുയഭൂതേ! സീതേ! ധര്‍മരക്ഷാര്‍ത്ഥം പ്രിയേ!’\\
രാമചന്ദ്രോക്തികേട്ടു ജാനകീദേവിതാനും\\
കോമളഗാത്രിയായ മായാസീതയെത്തത്ര\\
പര്‍ണശാലയിലാക്കി വഹ്നിമണ്ഡലത്തിങ്കല്‍\\
ചെന്നിരുന്നിതു മഹാവിഷ്ണുമായയുമപ്പോള്‍.
\end{verse}

%%16_maareechanigraham
\section{മാരീചനിഗ്രഹം}

\begin{verse}
മായാനിര്‍മിതമായ കനകമൃഗം കണ്ടു\\
മായാസീതയും രാമചന്ദ്രനോടുരചെയ്താള്‍:\\
‘ഭര്‍ത്താവേ! കണ്ടീലയോ കനകമയമൃഗ-\\
മെത്രയും ചിത്രം ചിത്രം! രത്നഭൂഷിതമിദം.\\
പേടിയില്ലിതിനേതുമെത്രയുമടുത്തുവ-\\
ന്നീടുന്നു മെരുക്കമുണ്ടെത്രയുമെന്നു തോന്നും\\
കളിപ്പാനതിസുഖമുണ്ടിതു നമുക്കിങ്ങു\\
വിളിച്ചീടുക വരുമെന്നു തോന്നുന്നൂ നൂനം.\\
പിടിച്ചുകൊണ്ടിങ്ങു പോന്നീടുക വൈകീടാതെ\\
മടിച്ചീടരുതേതും ഭര്‍ത്താവേ! ജഗല്‍പതേ!’\\
മൈഥിലീ വാക്യം കേട്ടു രാഘവനരുള്‍ചെയ്തു\\
സോദരന്‍തന്നോടു: ’നീ കാത്തുകൊള്ളുകവേണം\\
സീതയെയവള്‍ക്കൊരു ഭയവുമുണ്ടാകാതെ\\
യാതുധാനന്മാരുണ്ടു കാനനംതന്നിലെങ്ങും.’\\
എന്നരുള്‍ചെയ്തു ധനുര്‍ബാണങ്ങളെടുത്തുടന്‍\\
ചെന്നിതു മൃഗത്തെക്കൈക്കൊള്ളുവാന്‍ ജഗന്നാഥന്‍.\\
അടുത്തു ചെല്ലുന്നേരം വേഗത്തിലോടിക്കള-\\
ഞ്ഞടുത്തുകൂടായെന്നു തോന്നുമ്പോള്‍ മന്ദം മന്ദം\\
അടുത്തു വരു,മപ്പോള്‍ പിടിപ്പാന്‍ ഭാവിച്ചീടും\\
പടുത്വമോടു ദൂരെക്കുതിച്ചു ചാടുമപ്പോള്‍\\
ഇങ്ങനെതന്നെയൊട്ടു ദൂരത്തായതു നേര-\\
മെങ്ങനെ പിടിക്കുന്നു വേഗമുണ്ടിതിനേറ്റം\\
എന്നുറച്ചാശവിട്ടു രാഘവനൊരു ശരം\\
നന്നായിത്തൊടുത്തുടന്‍ വലിച്ചു വിട്ടീടിനാന്‍.\\
പൊന്മാനുമതുകൊണ്ടു ഭൂമിയില്‍ വീണനേരം\\
വന്മലപോലെയൊരു രാക്ഷസവേഷംപൂണ്ടാന്‍.\\
മാരീചന്‍തന്നെയിതു ലക്ഷ്മണന്‍ പറഞ്ഞതു\\
നേരത്രേയെന്നു രഘുനാഥനും നിരൂപിച്ചു.\\
ബാണമേറ്റവനിയില്‍ വീണപ്പോള്‍ മാരീചനും\\
പ്രാണവേദനയോടു കരഞ്ഞാനയ്യോ പാപം!\\
‘ഹാഹാ! ലക്ഷ്മണ! മമ ഭ്രാതാവേ! സഹോദരാ!\\
ഹാഹാ! മേ വിധിബലം പാഹി മാം ദയാനിധേ!’\\
ആതുരനാദം കേട്ടു ലക്ഷ്മണനോടു ചൊന്നാള്‍\\
സീതയും: ’സൗമിത്രേ! നീ ചെല്ലുക വൈകീടാതെ\\
അഗ്രജനുടെവിലാപങ്ങള്‍ കേട്ടീലേ ഭവാ-\\
നുഗ്രന്മാരായ നിശാചരന്മാര്‍ കൊല്ലും മുമ്പേ\\
രക്ഷിച്ചുകൊള്‍ക ചെന്നു ലക്ഷ്മണ! മടിയാതെ\\
രക്ഷോവീരന്മാരിപ്പോള്‍ കൊല്ലുമല്ലെങ്കിലയ്യോ!’\\
ലക്ഷ്മണനതു കേട്ടു ജാനകിയോടു ചൊന്നാന്‍:\\
‘ദുഃഖിയായ്കാര്യേ! ദേവീ! കേള്‍ക്കണം മമ വാക്യം.\\
മാരീചന്‍തന്നെ പൊന്മാനായ്വന്നതവന്‍ നല്ല\\
ചോരനെത്രയുമേവം കരഞ്ഞതവന്‍തന്നെ\\
അന്ധനായ് ഞാനുമിതു കേട്ടു പോയകലുമ്പോള്‍\\
നിന്തിരുവടിയേയും കൊണ്ടുപോയീടാമല്ലോ\\
പംക്തികന്ധരന്‍ തനിക്കതിനുള്ളുപായമി-\\
തെന്തറിയാതെയരുള്‍ചെയ്യുന്നിതത്രയല്ല\\
ലോകവാസികള്‍ക്കാര്‍ക്കും ജയിച്ചുകൂടായല്ലോ\\
രാഘവന്‍ തിരുവടിതന്നെയെന്നറിയണം.\\
ആര്‍ത്തനാദവും മമ ജ്യേഷ്ഠനുണ്ടാകയില്ല\\
രാത്രിചാരികളുടെ മായയിതറിഞ്ഞാലും\\
വിശ്വനായകന്‍ കോപിച്ചീടുകിലരക്ഷണാല്‍\\
വിശ്വസംഹാരം ചെയ്വാന്‍ പോരുമെന്നറിഞ്ഞാലും\\
അങ്ങനെയുള്ള രാമന്‍തന്‍മുഖാംബുജത്തില്‍നി-\\
ന്നെങ്ങനെ ദൈന്യനാദം ഭവിച്ചീടുന്നു നാഥേ!’\\
ജാനകിയതുകേട്ടു കണ്ണുനീര്‍ തൂകിത്തൂകി\\
മാനസേ വളര്‍ന്നൊരു ഖേദകോപങ്ങളോടും\\
ലക്ഷ്മണന്‍തന്നെ നോക്കിച്ചൊല്ലിനാളതുനേരം:\\
‘രക്ഷോജാതിയിലത്രേ നീയുമുണ്ടായി നൂനം\\
ഭ്രാതൃനാശത്തിനത്രേ കാംക്ഷയാകുന്നു തവ\\
ചേതസി ദുഷ്ടാത്മാവേ! ഞാനിതോര്‍ത്തീലയല്ലോ.\\
രാമനാശാകാംക്ഷിതനാകിയ ഭരതന്റെ\\
കാമസിദ്ധ്യര്‍ഥമവന്‍തന്നുടെ നിയോഗത്താല്‍\\
കൂടെപ്പോന്നിതു നീയും രാമനുനാശം വന്നാല്‍\\
ഗൂഢമായെന്നെയും കൊണ്ടങ്ങു ചെല്ലുവാന്‍ നൂനം.\\
എന്നുമേ നിനക്കെന്നെക്കിട്ടുകയില്ലതാനു-\\
മിന്നു മല്‍പ്രാണത്യാഗം ചെയ്വന്‍ ഞാനറിഞ്ഞാലും.\\
ചേതസി ഭാര്യാഹരണോദ്യതനായ നിന്നെ\\
സോദരബുദ്ധ്യാ ധരിച്ചീല രാഘവനേതും\\
രാമനെയൊഴിഞ്ഞു ഞാന്‍ മറ്റൊരു പുരുഷനെ\\
രാമപാദങ്ങളാണെ തീണ്ടുകയില്ലയല്ലോ.’\\
ഇത്തരം വാക്കുകേട്ടു സൗമിത്രി ചെവി രണ്ടും\\
സത്വരം പൊത്തിപ്പുനരവളോടുരചെയ്താന്‍:\\
‘നിനക്കു നാശമടുത്തിരിക്കുന്നിതു പാര-\\
മെനിക്കു നിരൂപിച്ചാല്‍ തടുത്തുകൂടാതാനും.\\
ഇത്തരം ചൊല്ലീടുവാന്‍ തോന്നിയതെന്തേ ചണ്ഡീ!\\
ധിക്ധിഗത്യന്തം ക്രൂരചിത്തം നാരികള്‍ക്കെല്ലാം\\
വനദേവതമാരെ! പരിപാലിച്ചു കൊള്‍വിന്‍\\
മനുവംശാധീശ്വരപത്നിയെ വഴിപോലെ.’\\
ദേവിയെദ്ദേവകളെബ്ഭരമേല്പിച്ചു മന്ദം\\
പൂര്‍വജന്‍തന്നെക്കാന്മാന്‍ നടന്നൂ സൗമിത്രിയും.
\end{verse}

%%17_seethaapaharanam
\section{സീതാപഹരണം}

\begin{verse}
അന്തരം കണ്ടു ദശകന്ധരന്‍ മദനബാ-\\
ണാന്ധനായവതരിച്ചീടിനാനവതിയില്‍.\\
ജടയും വല്ക്കലവും ധരിച്ചു സന്ന്യാസിയാ-\\
യുടജാങ്കണേ വന്നു നിന്നിതു ദശാസ്യനും.\\
ഭിക്ഷുവേഷത്തെപ്പൂണ്ടരക്ഷോനാഥനെക്കണ്ടു\\
തല്‍ക്ഷണം മായാസീതാദേവിയും വിനീതയായ്\\
നത്വാ സംപൂജ്യഭക്ത്യാ ഫലമൂലാദികളും\\
ദത്വാ സ്വാഗതവാക്യമുക്ത്വാ പിന്നെയും ചൊന്നാള്‍:\\
‘അത്രൈവ ഫലമൂലാദികളും ഭുജിച്ചുകൊ-\\
ണ്ടിത്തിരിനേരമിരുന്നീടുക തപോനിധേ!\\
ഭര്‍ത്താവു വരുമിപ്പോള്‍ ത്വല്‍പ്രിയമെല്ലാം ചെയ്യും\\
ക്ഷുത്തൃഡാദിയും തീര്‍ത്തു വിശ്രമിച്ചാലും ഭവാന്‍.’\\
ഇത്തരം മായാദേവീമുഗ്ദ്ധാലാപങ്ങള്‍ കേട്ടു\\
സത്വരം ഭിക്ഷുരൂപി സസ്മിതം ചോദ്യംചെയ്താന്‍:\\
‘കമലവിലോചനേ! കമനീയാംഗി! നീയാ-\\
രമലേ! ചൊല്ലീടു നിന്‍ കമിതാവാരെന്നതും.\\
നിഷ്ഠുരജാതികളാം രാക്ഷസരാദിയായ\\
ദുഷ്ടജന്തുക്കളുള്ള കാനനഭൂമിതന്നില്‍\\
നീയൊരു നാരീമണി താനേ വാഴുന്നതെന്തോ-\\
രായുധപാണികളുമില്ലല്ലോ സഹായമായ്?\\
നിന്നുടെ പരമാര്‍ത്ഥമൊക്കവേ പറഞ്ഞാല്‍ ഞാ-\\
നെന്നുടെ പരമാര്‍ത്ഥം പറയുന്നുണ്ടുതാനും.’\\
മേദിനീസുതയതു കേട്ടുരലെയ്തീടിനാള്‍:\\
‘മേദിനീപതിവരനാമയോദ്ധ്യാധിപതി\\
വാട്ടമില്ലാത ദശരഥനാം നൃപാധിപ\\
ജ്യേഷ്ഠനന്ദനനായ രാമനത്ഭുതവീര്യന്‍-\\
തന്നുടെ ധര്‍മപത്നിജനകാത്മജ ഞാനോ\\
ധന്യനാമനുജനും ലക്ഷ്മണനെന്നു നാമം.\\
ഞങ്ങള്‍ മൂവരും പിതുരാജ്ഞയാ തപസ്സിനാ-\\
യിങ്ങു വന്നിരിക്കുന്നു ദണ്ഡകവനം തന്നില്‍.\\
പതിന്നാലാണ്ടു കഴിവോളവും വേണം താനു-\\
മതിനു പാര്‍ത്തീടുന്നു സത്യമെന്നറുഞ്ഞാലും.\\
നിന്തിരുവടിയെ ഞാനറിഞ്ഞീലേതും പുന-\\
രെന്തിനായെഴുന്നള്ളീ ചൊല്ലേണം പരമാര്‍ത്ഥം.’\\
‘എങ്കിലോ കേട്ടാലും നീ മംഗലശീലേ! ബാലേ!\\
പങ്കജവിലോചനേ൧ പഞ്ചബാണാധിവാസേ!\\
പൗലസ്ത്യതനയനാം രാക്ഷസരാജാവു ഞാന്‍\\
ത്രൈലോകത്തിങ്കലെന്നെയാരറിയാതെയുള്ളൂ?\\
നിര്‍മലേ! കാമപരിതപ്തനായ്ച്ചമഞ്ഞു ഞാന്‍\\
നിന്മൂലമതിന്നു നീ പോരേണം മയാ സാകം.\\
ലങ്കയാം രാജ്യം വാനോര്‍നാട്ടിലും മനോഹരം\\
കിങ്കരനായേന്‍ തവ ലോകസുന്ദരീ! നാഥേ!\\
താപസവേഷം പൂണ്ട രാമനാലെന്തുഫലം?\\
താപമുള്‍ക്കൊണ്ടു കാട്ടിലിങ്ങനെ വസിക്കേണ്ട\\
ശരണാഗതനായോരെന്നെ നീ ഭജിച്ചാലു-\\
മരുണാധരീ! മഹാഭോഗങ്ങള്‍ ഭുജിച്ചാലും.’\\
രാവണവാക്യമേവം കേട്ടതിഭയത്തോടും\\
ഭാവവൈവര്‍ണ്യം പൂണ്ടു ജാനകിചൊന്നാള്‍ മന്ദം:\\
‘കേവലമടുത്തിതു മരണം നിനക്കിപ്പോ-\\
ളേവം നീ ചൊല്ലുന്നാകില്‍ ശ്രീരാമദേവന്‍തന്നാല്‍\\
സോദരനോടും കൂടി വേഗത്തില്‍ വരുമിപ്പോള്‍\\
മേദിനീപതി മമ ഭര്‍ത്താ ശ്രീരാമചന്ദ്രന്‍.\\
തൊട്ടുകൂടുമോ ഹരിപത്നിയെ ശശത്തിനു?\\
കഷ്ടമായുള്ള വാക്കു ചൊല്ലാതെ ദുരാത്മാവേ!\\
രാമബാണങ്ങള്‍കൊണ്ടു മാറിടം പിളര്‍ന്നു നീ\\
ഭൂമിയൊല്‍ വീഴാനുള്ള കാരണമിതു നൂനം.’\\
ഇങ്ങനെ സീതാവാക്യം കേട്ടു രാവണനേറ്റം\\
തിങ്ങീടും ക്രോധം പൂണ്ടു മൂര്‍ച്ഛിതനായന്നേരം\\
തന്നുടെ രൂപം നേരേ കാട്ടിനാന്‍ മഹാഗിരി\\
സന്നിഭം ദശാനനം വിശതി മഹാഭുജം\\
അഞ്ജനശൈലാകാരം കാണായ നേരമുള്ളി-\\
ലഞ്ജസാ ഭയപ്പെട്ടു വനദേവതമാരും.\\
രാഘവപത്നിയേയും തേരതിലെടുത്തു വെ-\\
ച്ചാകാശമാര്‍ഗേ ശീഘ്രം പീയിതു ദശാസ്യനും.\\
‘ഹാഹാ! രാഘവ! രാമ! സൗമിത്രേ! കാരുണ്യാബ്ധേ!\\
ഹാഹാ! മല്‍ പ്രാണേശ്വരാ! പാഹി മാം ഭയാതുരാം.’\\
ഇത്തരം സീതാവിലാപം കേട്ടു പക്ഷീന്ദ്രനും\\
സത്വരമുത്ഥാനം ചെയ്തെത്തിനാന്‍ ജടായുവും.\\
‘തിഷ്ഠ തിഷ്ഠാഗ്രേ മമ സ്വാമിതന്‍പത്നിയേയും\\
കട്ടുകൊണ്ടെവിടേക്കു പോകുന്നു മൂഢാത്മാവേ!\\
അധ്വരത്തിങ്കല്‍ ചെന്നു ശുനകന്‍, മന്ത്രം കൊണ്ടു\\
ശുദ്ധമാം പുരോഡാശം കൊണ്ടു പോകുന്നപോലെ.’\\
പദ്ധതിമദ്ധ്യേ പരമോദ്ധതബുദ്ധിയോടും\\
ഗൃദ്ധ്രരാജനുമൊരു പത്രവാനായുല്ലൊരു\\
കുധ്രരാജനെപ്പോലെ ബദ്ധവൈരത്തോടതി-\\
ക്രുദ്ധനായഗ്രേ ചെന്നു യുദ്ധവും തുടങ്ങിനാന്‍.\\
അബ്ധിയും പത്രാനിലക്ഷുബ്ധമായ്ച്ചമയുന്നി-\\
തദ്രികളിളകുന്നു വിദ്രുതമതുനേരം.\\
കാല്‍നഖങ്ങളെക്കൊണ്ടു ചാപങ്ങള്‍ പൊടിപെടു-\\
ത്താനനങ്ങളും കീറിമുറിഞ്ഞു വശംകെട്ടു\\
തീക്ഷ്ണതുണ്ഡാഗ്രംകൊണ്ടു തേര്‍ത്തടം തകര്‍ത്തിതു\\
കാല്‍ക്ഷണംകൊണ്ടു കൊന്നുവീഴ്ത്തിനാനശ്വങ്ങളെ.\\
രൂക്ഷത പെരുകിയ പക്ഷപാതങ്ങളേറ്റു\\
രാക്ഷസപ്രവരനും ചഞ്ചലമുണ്ടായ്വന്നു.\\
യാത്രയും മുടങ്ങി മല്‍ക്കീര്‍ത്തിയുമൊടുങ്ങീതെ-\\
ന്നാര്‍ത്തിപൂണ്ടുഴന്നൊരു രാത്രിചാരീന്ദ്രനപ്പോള്‍\\
ധാത്രീപുത്രിയെത്തത്ര ധാത്രിയില്‍ നിര്‍ത്തിപ്പുന-\\
രോര്‍ത്തുതന്‍ ചംന്ദ്രഹാസമിളക്കി ലഘുതരം\\
പക്ഷിനായകനുടെ പക്ഷങ്ങള്‍ ഛേദിച്ചപ്പോ-\\
ളക്ഷിതിതന്നില്‍ വീണാനക്ഷമനായിട്ടവന്‍.\\
രക്ഷോനായകന്‍ പിന്നെ ലക്ഷ്മീദേവിയേയും കൊ-\\
ണ്ടക്ഷതചിത്തത്തോടും ദക്ഷിണദിക്കുനോക്കി\\
മറ്റൊരു തേരിലേറിത്തെറ്റെന്നു നടകൊണ്ടാന്‍\\
മറ്റാരും പാലിപ്പാനില്ലുറ്റവരായിട്ടെന്നോര്‍-\\
ത്തിറ്റിറ്റു വീണീടുന്ന കണ്ണുനീരോടുമപ്പോള്‍\\
കറ്റവാര്‍കുഴലിയാ ജാനകീദേവിതാനും\\
‘ഭര്‍ത്താവുതന്നെക്കണ്ടു വൃത്താന്തം പറഞ്ഞൊഴി-\\
ഞ്ഞുത്തമനായ നിന്റെ ജീവനും പോകായ്കെ’ന്നു\\
പൃത്ഥ്വീപുത്രിയും വരം പത്രിരാജനു നല്കി\\
പൃത്ഥ്വീമണ്ഡലമകന്നാശു മേല്പോട്ടു പോയാള്‍.’\\
‘അയ്യോ! രാഘവ! ജഗന്നായക! ദയാനിധേ!\\
നീയെന്നെയുപേക്ഷിച്ചതെന്തു ഭര്‍ത്താവേ! നാഥ!\\
രക്ഷോനായകനെന്നെക്കൊണ്ടിതാ പോയീടുന്നു\\
രക്ഷിതാവായിട്ടാരുമില്ലെനിക്കയ്യോ! പാപം!\\
ലക്ഷ്മണ! നിന്നോടു ഞാന്‍ പരുഷം ചൊന്നേനല്ലോ\\
രക്ഷിച്ചുകൊള്ളേണമേ ദേവരാ! ദയാനിധേ!\\
രാമ! രാമാത്മാരാമ! ലോകാഭിരാമ! രാമ!\\
ഭൂമിദേവിയുമെന്നെ വെടിഞ്ഞാളിതുകാലം.\\
പ്രാണവല്ലഭ! പരിത്രാഹി മാം ജഗല്‍പ്പതേ!\\
കൗണപാധിപനെന്നെക്കൊന്നു ഭക്ഷിക്കും മുമ്പേ\\
സത്വരം വന്നു പരിപാലിച്ചുകൊള്ളേണമേ\\
സത്വചേതസാ മഹാസത്വവാരിധേ! നാഥാ!’\\
ഇത്തരം വിലപിക്കും നേരത്തു ശീഘ്രം രാമ-\\
ഭദ്രനിങ്ങെത്തുമെന്ന ശങ്കയാ നക്തഞ്ചരന്‍\\
ചിത്തവേഗേന നടന്നീടിനാനതുനേരം\\
പൃത്ഥ്വീപുത്രിയും കീഴ്പോട്ടാശു നോക്കുന്ന നേരം\\
അദ്രിനാഥാഗ്രേ കണ്ടു പഞ്ചവാനരന്മാരെ\\
വിദ്രുതം വിഭൂഷണസഞ്ചയമഴിച്ചുത-\\
ന്നുത്തരീയാര്‍ദ്ധഖണ്ഡംകൊണ്ടു ബന്ധിച്ചു രാമ-\\
ഭദ്രനു കാണ്മാന്‍ യോഗം വരികെന്നകതാരില്‍\\
സ്മൃത്വാ കീഴ്പോട്ടു നിക്ഷേപിച്ചിതു സീതാദേവി\\
മത്തനാം നക്തഞ്ചരനറിഞ്ഞീലതുമപ്പോള്‍.\\
അബ്ധിയുമുത്തീര്യ തന്‍പത്തനം ഗത്വാ തൂര്‍ണം\\
ശുദ്ധാന്തമദ്ധ്യേ മഹാശോകകാനനദേശേ\\
ശുദ്ധഭൂതലേ മഹാശിംശപാതരുമൂലേ\\
ഹൃദ്യമാരായ നിജരക്ഷോനാരികളെയും\\
നിത്യവും പാലിച്ചുകൊള്‍കെന്നുറപ്പിച്ചു തന്റെ\\
വസ്ത്യമുള്‍പ്പുക്കു വസിച്ചീടിനാന്‍ ദശാനനന്‍.\\
ഉത്തമോത്തമയായ ജാനകീദേവി പാതി-\\
വ്രത്യമാശ്രിത്യ വസിച്ചീടിനാളതുകാലം.\\
വസ്ത്രകേശാദികളുമെത്രയും മലിനമായ്\\
വക്ത്രവും കുമ്പിട്ടു സന്തപ്തമാം ചിത്തത്തോടും\\
രാമരാമേതി ജപധ്യാനനിഷ്ഠയാ ബഹു-\\
യാമിനീചരകുലനാരികളുടെ മദ്ധ്യേ\\
നീഹാരശീതാതപവാതപീഡയും സഹി-\\
ച്ചാഹാരാദികളേതും കൂടാതെ ദിവാരാത്രം\\
ലങ്കയില്‍ വസിച്ചിതാതങ്കമുള്‍ക്കൊണ്ടു മായാ\\
സങ്കടം മനുഷ്യജന്മത്തിങ്കലാര്‍ക്കില്ലാത്തൂ?
\end{verse}

%%18_seethaanveshanam
\section{സീതാന്വേഷണം}

\begin{verse}
രാമനും മായാമൃഗവേഷത്തെക്കൈക്കൊണ്ടൊരു\\
കാമരൂപിണം മാരീചാസുരമെയ്തുകൊന്നു\\
വേഗേന നടകൊണ്ടാനാശ്രമം നോക്കിപ്പുന-\\
രാഗമക്കാതലായ രാഘവന്‍ തിരുവടി.\\
നാലഞ്ചു ശരപ്പാടു നടന്നോരനന്തരം\\
ബാലകന്‍ വരവീഷദ്ദൂരവേ കാണായ് വന്നു\\
ലക്ഷ്മണന്‍ വരുന്നതു കണ്ടു രാഘവന്‍ താനു-\\
മുള്‍ക്കാമ്പില്‍ നിരൂപിച്ചു കല്പിച്ചു കരണീയം.\\
ലക്ഷ്മണനേതുമറിഞ്ഞീലല്ലോ പരമാര്‍ത്ഥ-\\
മിക്കാലമിവനേയും വഞ്ചിക്കെന്നതേ വരൂ.\\
രാക്ഷോനായകന്‍ കൊണ്ടുപോയതു മായാസീതാ\\
ലക്ഷ്മീദേവിയെയുണ്ടോ മറ്റാര്‍ക്കും ലഭിക്കുന്നു?\\
അഗ്നിമണ്ഡലത്തിങ്കല്‍ വാഴുന്ന സീതതന്നെ\\
ലക്ഷ്മണനറിഞ്ഞാലിക്കാര്യവും വന്നുകൂടാ.\\
ദുഃഖിച്ചുകൊള്ളൂ ഞാനും പ്രകൃതനെന്നപോലെ\\
മൈക്കണ്ണീതന്നെത്തിരഞ്ഞാശുപോയ് ചെല്ലാമല്ലോ\\
രക്ഷോനായകനുടെ രാജ്യത്തി,ലെന്നാല്‍ പിന്നെ-\\
ത്തല്‍ക്കുലത്തോടും കൂടി രാവണന്‍ തന്നെക്കൊന്നാല്‍\\
അഗ്നിമണ്ഡലേ വാഴും സീതയെസ്സത്യവ്യാജാല്‍\\
കൈക്കൊണ്ടു പോകാമയോദ്ധ്യയ്ക്കു വൈകാതെ, പിന്നെ\\
അക്ഷയധര്‍മമോടു രാജ്യത്തെ വഴിപോലെ\\
രക്ഷിച്ചു കിഞ്ചില്‍കാലം ഭൂമിയില്‍ വസിച്ചീടാം.\\
പുഷ്കരോത്ഭാവനിത്ഥം പ്രാര്‍ത്ഥിക്ക നിമിത്തമാ-\\
യര്‍ക്കവംശത്തിങ്കല്‍ ഞാന്‍ മര്‍ത്ത്യനായ് പിറന്നതും\\
മായാമാനുഷനാകുമെന്നുടെ ചരിതവും\\
മായാവൈഭവങ്ങളും കേള്‍ക്കയും ചൊല്ലുകയും\\
ഭക്തിമാര്‍ഗേണചെയ്യും ഭക്തനപ്രയാസേന\\
മുക്തിയും സിദ്ധിച്ചീടുമില്ല സംശയമേതും.\\
ആകയാലിവനെയും വഞ്ചിച്ചു ദുഃഖിപ്പൂ ഞാന്‍\\
പ്രാകൃതപുരുഷനെപ്പോലെ’യെന്നകതാരില്‍\\
നിര്‍ണയിച്ചവരജനോടരുള്‍ചെയ്തീടിനാന്‍:\\
‘പര്‍ണശാലയില്‍ സീതയ്ക്കാരൊരു തുണയുള്ളൂ?\\
എന്തിനിങ്ങോട്ടു പോന്നു ജാനകിതന്നെബ്ബലാ-\\
ലെന്തിനു വെടിഞ്ഞു നീ, രാക്ഷസരവളേയും\\
കൊണ്ടുപോകയോ കൊന്നു ഭക്ഷിച്ചുകളകയോ\\
കണ്ടകജാതികള്‍ക്കെന്തോന്നരുതാത്തതോര്‍ത്താല്‍?’\\
അഗ്രജവാക്യമേവം കേട്ടു ലക്ഷ്മണന്‍താനു-\\
മഗ്രേ നിന്നുടനുടന്‍ തൊഴുതു വിവശനായ്\\
ഗദ്ഗദാക്ഷരമുരചെയ്തിതു ദേവിയുടെ\\
ദുര്‍ഗ്രഹവചനങ്ങള്‍ ബാഷ്പവും തൂകിത്തൂകി:\\
‘ഹാ ഹാ ലക്ഷ്മണ! പരിത്രാഹി സൗമിത്രേ! ശീഘ്രം\\
ഹാ ഹാ രാക്ഷസനെന്നെ നിഗ്രഹിച്ചീടുമിപ്പോള്‍’\\
ഇത്തരം നക്തഞ്ചരന്‍തന്‍ വിലാപങ്ങള്‍ കേട്ടു\\
മുഗ്ദ്ധഗാത്രിയും തവ നാദമെന്നുറയ്ക്കയാല്‍\\
അത്യര്‍ത്ഥം പരിതാപം കൈക്കൊണ്ടു വിലപിച്ചു\\
സത്വരംചെന്നു രക്ഷിക്കെന്നെന്നോടരുള്‍ ചെയ്തു.\\
‘ഇത്തരം നാദം മമ ഭ്രാതാവിനുണ്ടായ്വരാ\\
ചിത്തമോഹവും വേണ്ട സത്യമെന്നറിഞ്ഞാലും\\
രാക്ഷസനുടെ മായാഭാഷിതമിതു നൂനം\\
കാല്‍ക്ഷണം പൊറുക്കെന്നു ഞാന്‍ പലവുരു ചൊന്നേന്‍.\\
എന്നതു കേട്ടു ദേവി പിന്നെയുമുരചെയ്താ-\\
ളെന്നോടു പലതരമിന്നവയെല്ലാമിപ്പോള്‍\\
നിന്തിരുമുമ്പില്‍ നിന്നു ചൊല്ലുവാന്‍ പണിയെന്നാല്‍\\
സന്താപത്തോടു ഞാനും കര്‍ണങ്ങള്‍ പൊത്തിക്കൊണ്ടു\\
ചിന്തിച്ചു ദേവകളെ പ്രാര്‍ത്ഥിച്ചു രക്ഷാര്‍ത്ഥമായ്\\
നിന്തിരുമലരടി വന്ദിപ്പാന്‍ വിടകൊണ്ടേന്‍.’\\
‘എങ്കിലും പിഴച്ചിതു പോന്നതു സൗമിത്രേ നീ\\
ശങ്കയുണ്ടായീടാമോ ദുര്‍വചനങ്ങള്‍ കേട്ടാല്‍?\\
യോഷമാരുടെ വാക്കു സത്യമെന്നോര്‍ക്കുന്നവന്‍\\
ഭോഷനെത്രയുമെന്നു നീയറിയുന്നതില്ലേ?\\
രാക്ഷസാം പരിഷകള്‍ കൊണ്ടു പൊയ്കളകയോ\\
ഭക്ഷിച്ചു കളകയോ ചെയ്തതെന്നറിഞ്ഞീല.’\\
ഇങ്ങനെ നിനച്ചുടജാന്തര്‍ഭാഗത്തിങ്കല്‍ ചെ-\\
ന്നെങ്ങുമേ നോക്കിക്കാണാഞ്ഞാകുലപ്പെട്ടു രാമന്‍\\
ദുഃഖഭാവവും കൈക്കൊണ്ടെത്രയും വിലപിച്ചാന്‍\\
നിഷ്കളനാത്മാരാമന്‍ നിര്‍ഗുണനാത്മാനന്ദന്‍:\\
‘ഹാ ഹാ വല്ലഭേ! സീതേ! ഹാഹാ മൈഥിലീ! നാഥേ!\\
ഹാ ഹാ ജാനകീദേവീ! ഹാ ഹാ മല്‍പ്രാണേശ്വരീ!\\
എന്നെ മോഹിപ്പിപ്പതിന്നായ് മറഞ്ഞിരിക്കയോ?\\
ധന്യേ! നീ വെളിച്ചത്തു വന്നീടു മടിയാതെ.’\\
ഇത്തരം പറകയും കാനനംതോറും നട-\\
ന്നത്തല്‍പൂണ്ടന്വേഷിച്ചും കാണാഞ്ഞു വിവശനായ്\\
‘മനദേവതമാരേ! നിങ്ങളുമുണ്ടോ കണ്ടൂ\\
വനജേക്ഷണയായ സീതയെ സത്യം ചൊല്‍വിന്‍\\
മൃഗസഞ്ചയങ്ങളേ! നിങ്ങളുമുണ്ടോ കണ്ടു\\
മൃഗലോചനയായ ജനകപുത്രിതന്നെ?\\
പക്ഷിസഞ്ചയങ്ങളേ! നിങ്ങളുമുണ്ടോ കണ്ടൂ\\
പക്ഷ്മളാക്ഷിയെ മമ ചൊല്ലുവിന്‍ പരമാര്‍ത്ഥം.\\
വൃക്ഷവൃന്ദമേ! പറഞ്ഞീടുവിന്‍ പരമാര്‍ത്ഥം\\
പുഷ്കരാക്ഷിയെ നിങ്ങളെങ്ങാനുമുണ്ടോ കണ്ടൂ?’\\
ഇത്ഥമോരോന്നേ പറഞ്ഞത്രയും ദുഃഖം പൂണ്ടു\\
സത്വരം നീളെത്തിരഞ്ഞെങ്ങുമേ കണ്ടീലല്ലോ.\\
സര്‍വദൃക് സര്‍വേശ്വരന്‍ സര്‍വജ്ഞന്‍ സര്‍വാത്മാവാം\\
സര്‍വകാരണനേകനചലന്‍ പരിപൂര്‍ണന്‍\\
നിര്‍മലന്‍ നിരാകാരന്‍ നിരഹങ്കാരന്‍ നിത്യന്‍\\
ചിന്മയനഖണ്ഡാനന്ദാത്മകന്‍ ജഗന്മയന്‍\\
മായയാ മനുഷ്യഭാവേന ദുഃഖിച്ചീടിനാന്‍\\
കാര്യമാനുഷന്‍ മൂഢാത്മാക്കളെയൊപ്പിപ്പാനായ്.\\
തത്ത്വജ്ഞന്മാര്‍ക്കു സുഖദുഃഖ ഭേദങ്ങളൊന്നും\\
ചെത്തേ തോന്നുകയുമില്ലജ്ഞാനമില്ലായ്കയാല്‍.
\end{verse}

%%19_jataayugathi
\section{ജടായുഗതി}

\begin{verse}
ശ്രീരാമദേവനേവം തിരഞ്ഞു നടക്കുമ്പോള്‍\\
തേരഴിഞ്ഞുടഞ്ഞുവീണാകുലമടവിയില്‍\\
ശസ്ത്രചാപങ്ങളോടും കൂടവേ കിടക്കുന്ന-\\
തെത്രയുമടുത്തു കാണായിതു മദ്ധ്യേ മാര്‍ഗം.\\
അന്നേരം സൗമിത്രിയോടരുളിച്ചെയ്തു രാമന്‍:\\
‘ഭിന്നമായോരു രഥം കാണെടോ കുമാര! നീ.\\
തന്വംഗിതന്നെയൊരു രാക്ഷസന്‍ കൊണ്ടുപോമ്പോ-\\
ളന്യരാക്ഷസനവനോടു പോര്‍ചെയ്തീടിനാന്‍.\\
അന്നേരമഴിഞ്ഞ തേര്‍ക്കോപ്പിതാ കിടക്കുന്നു\\
എന്നു വന്നീടാമവര്‍ കൊന്നാരോ ഭക്ഷിച്ചാരോ?\\
ശ്രീരാമനേവം പറഞ്ഞിത്തിരി നടക്കുമ്പോള്‍\\
ഘോരമായൊരു രൂപം കാണായി ഭയാനകം\\
‘ജാനകിതന്നെത്തിന്നു തൃപ്തനായൊരു യാതു-\\
ധാനനക്കിടക്കുന്നതത്ര നീ കണ്ടീലയോ?\\
കൊല്ലുവനിവനെ ഞാന്‍ വൈകാതെ ബാണങ്ങളും\\
വില്ലുമിങ്ങാശു തന്നീടെ’ന്നതു കേട്ടനേരം\\
വിത്രസ്തഹൃദയനായ് പക്ഷിരാജനും ചൊന്നാന്‍:\\
‘വദ്ധ്യനല്ലഹം തവ ഭക്തനായൊരു ദാസന്‍\\
മിത്രമെത്രയും തവ താതനു വിശേഷിച്ചും\\
സ്നിഗ്ദ്ധനായിരിപ്പൊരു പക്ഷിയാം ജടായു ഞാന്‍.\\
ദുഷ്ടനാം ദശമുഖന്‍ നിന്നുടെ പത്നിതന്നെ-\\
ക്കട്ടുകൊണ്ടാകാശേ പോകുന്നേരമറിഞ്ഞു ഞാന്‍\\
പെട്ടെന്നു ചെന്നു തടുത്തവനെ യുദ്ധം ചെയ്തു\\
മുട്ടിച്ചു തേരും വില്ലും പൊട്ടിച്ചു കളഞ്ഞപ്പോള്‍\\
വെട്ടിനാന്‍ ചന്ദ്രഹാസംകൊണ്ടവന്‍, ഞാനുമപ്പോള്‍\\
പുഷ്ടവേദനയോടും ഭൂമിയില്‍ വീണേനല്ലോ.\\
നിന്തിരുവടിയെക്കണ്ടൊഴിഞ്ഞു മരിയായ്കെ-\\
ന്നിന്ദിരാദേവിയോടു വരവും വാങ്ങിക്കൊണ്ടേന്‍\\
തൃക്കണ്‍പാര്‍ക്കണമെന്നെക്കൃപയാ കൃപാനിധേ!\\
തൃക്കഴലിണ നിത്യമുള്‍ക്കാമ്പില്‍ വസിക്കണം.’\\
ഇത്തരം ജടായുതന്‍ വാക്കുകള്‍ കേട്ടു നാഥന്‍\\
ചിത്തകാരുണ്യം പൂണ്ടു ചെന്നടുത്തിരുന്നു തന്‍-\\
തൃക്കൈകള്‍കൊണ്ടു തലോടീടിനാനവനുടല്‍\\
ദുഃഖാശ്രുപ്ലുതനയനത്തോടും രാമചന്ദ്രന്‍.\\
‘ചൊല്ലുചൊല്ലഹോ മമ വല്ലഭാവൃത്താന്തം നീ-\\
യെല്ലാ’മെന്നതു കേട്ടു ചൊല്ലിനാന്‍ ജടായുവും:\\
‘രക്ഷോനായകനായ രാവണന്‍ ദേവിതന്നെ-\\
ദ്ദക്ഷിണദിശി കൊണ്ടുപോയാനെന്നറിഞ്ഞാലും.\\
ചൊല്ലുവാനില്ല ശക്തി മരണപീഡയാലേ\\
നല്ലതു വരുവതിനായനുഗ്രഹിക്കേണം.\\
‘നിന്തിരുവടിതന്നെക്കണ്ടുകണ്ടിരിക്കവേ\\
ബന്ധമറ്റീടുംവണ്ണം മരിപ്പാനവകാശം\\
വന്നതു ഭവല്‍കൃപാപാത്രമാകയാലഹം\\
പുണ്യപുരുഷ! പുരുഷോത്തമ! ദയാനിധേ!\\
നിന്തിരുവടി സാക്ഷാല്‍ ശ്രീമഹാവിഷ്ണു പരാ-\\
നന്ദാത്മാ പരമാത്മാ മായാമാനുഷരൂപി\\
സന്തതമന്തര്‍ഭാഗേ വസിച്ചീടുക വേണം\\
നിന്തിരുമേനി ഘനശ്യാമളമഭിരാമം\\
അന്ത്യകാലത്തിങ്കലീവണ്ണം കാണായമൂലം\\
ബന്ധവുമറ്റു മുക്തനായേന്‍ ഞാനെന്നു നൂനം.\\
ബന്ധുഭാവേന ദാസനാകിയോരടിയനെ-\\
ബ്ബന്ധൂകസുമസമതൃക്കരതലം തന്നാല്‍\\
ബന്ധുവത്സല! മന്ദം തൊട്ടരുളേണമെന്നാല്‍\\
നിന്തിരുമലരടിയോടു ചേര്‍ന്നീടാമല്ലോ.’\\
ഇന്ദിരാപതിയതു കേട്ടുടന്‍ തലോടിനാന്‍\\
മന്ദമന്ദം പൂര്‍ണാത്മാനന്ദം വന്നീടും വണ്ണം\\
അന്നേരം പ്രാണങ്ങളെ ത്യജിച്ചു ജടായുവും\\
മന്നിടം തന്നില്‍ വീണനേരത്തു രഘുവരന്‍\\
കണ്ണുനീര്‍ വാര്‍ത്തു ഭക്തവാത്സല്യപരവശാ-\\
ലര്‍ണോജനേത്രന്‍ പിതൃമിത്രമാം പക്ഷീന്ദ്രന്റെ\\
ഉത്തമാംഗത്തെയെടുത്തുത്സംഗസീമ്നി ചേര്‍ത്തി-\\
ട്ടുത്തരകാര്യാര്‍ഥമായ് സോദരനോടു ചൊന്നാന്‍:\\
‘കാഷ്ഠങ്ങള്‍കൊണ്ടുവന്നു നല്ലൊരു ചിത തീര്‍ത്തു\\
കൂട്ടണമഗ്നി സംസ്കാരത്തിനു വൈകീടാതെ.’\\
ലക്ഷ്മണനതു കേട്ടു ചിതയും തീര്‍ത്തീടിനാന്‍\\
തല്‍ക്ഷണം കുളിച്ചു സംസ്കാരവും ചെയ്തു പിന്നെ\\
സ്നാനവും കഴിച്ചുദകക്രിയാദിയും ചെയ്തു\\
കാനനേ തത്ര മൃഗം വധിച്ചു മാംസഖണ്ഡം\\
പുല്ലിന്മേല്‍ വെച്ചു ജലാദികളും നല്കീടിനാന്‍\\
നല്ലൊരു ഗതിയവനുണ്ടാവാന്‍ പിത്രര്‍ത്ഥമായ്.\\
പക്ഷികളിവയെല്ലാം ഭക്ഷിച്ചു സുഖിച്ചാലും\\
പക്ഷീന്ദ്രനതുകൊണ്ടു തൃപ്തനായ് ഭവിച്ചാലും\\
കാരുണ്യപൂര്‍ത്തി കമലേക്ഷണന്‍ മധുവൈരി-\\
സാരൂപ്യം ഭവിക്കെന്നു സാദരമരുള്‍ചെയ്തു.\\
അന്നേരം വിമാനമാരുഹ്യ ഭാസുരം ഭാനു-\\
സന്നിഭം ദിവ്യരൂപം പൂണ്ടൊരു ജടായുവും\\
ശംഖാരിഗദാപത്മമകുടപീതാംബരാ-\\
ദ്യങ്കിതരൂപം പൂണ്ട വിഷ്ണുപാര്‍ഷദന്മാരാല്‍\\
പൂജിതനായി സ്തുതിക്കപ്പെട്ടു മുനികളാല്‍\\
തേജസാ സകല് ദിഗ്വ്യാപ്തനായ്ക്കാണായ്വന്നു\\
സന്നതഗാത്രത്തോടുമുയരെക്കൂപ്പിത്തൊഴു-\\
തുന്നതഭക്തിയോടേ രാമനെ സ്തുതി ചെയ്താന്‍:
\end{verse}

%%20_jataayusthuthi
\section{ജടായുസ്തുതി}

\begin{verse}
അഗണ്യഗുണമാദ്യമവ്യയമപ്രമേയ-\\
മഖില ജഗല്‍സൃഷ്ടിസ്ഥിതി സംഹാരമൂലം\\
പരമം പരാപരമാനന്ദം പരാത്മാനം\\
വരദമഹം പ്രണതോസ്മി സന്തതം രാമം.\\
മഹിത കടാക്ഷവിക്ഷേപിതാമരശുചം\\
രഹിതാവധിസുഖമിന്ദിരാമനോഹരം\\
ശ്യാമളം ജടാമകുടോജ്ജ്വലം ചാപശര-\\
കോമളകരാംബുജം പ്രണതോസ്മ്യഹം രാമം.\\
ഭുവനകമനീയരൂപമീഡിതം ശത-\\
രവിഭാസുരമഭീഷ്ടപ്രദം ശരണദം\\
സുരപാദപമൂലരചിതനിലയനം\\
സുരസഞ്ചയസേവ്യം പ്രണതോസ്മ്യഹം രാമം.\\
ഭവകാനനദവദഹന്നാമധേയം\\
ഭവപങ്കജഭവമുഖ ദൈവതം ദേവം\\
ദനുജപതി കോടി സഹസ്രവിനാശനം\\
മനുജാകാരം ഹരിം പ്രണതോസ്മ്യഹം രാമം\\
ഭവഭാവനാഹരം ഭഗവല്‍സ്വരൂപിണം\\
ഭവഭീവിരഹിതം മുനിസേവിതം പരം\\
ഭവസാഗരതരണാംഘ്രിപോതകം നിത്യം\\
ഭവനാശായാനിശം പ്രണതോസ്മ്യഹം രാമം.\\
ഗിരിശഗിരിസുതാഹൃദയാംബുജാവാസം\\
ഗിരിനായകധരം ഗിരിപക്ഷാരിസേവ്യം\\
സുരസഞ്ചയദനുജേന്ദ്രസേവിതപാദം\\
സുരപമണിനിഭം പ്രണതോസ്മ്യഹം രാമം.\\
വിധിമാധവശംഭുരൂപഭേദേന ഗുണ-\\
ത്രിതയവിരാജിതം കേവലം വിരാജന്തം\\
ത്രിദശമുനിജനദ്തുതമവ്യക്തമജം\\
ക്ഷിതിജാമനോഹരം പ്രണതോസ്മ്യഹം രാമം.\\
മന്മഥശതകോടി സുന്ദരകളേബരം\\
ജന്മനാശാദിഹീനം ചിന്മയം ജഗന്മയം\\
നിര്‍മലം ധര്‍മകര്‍മാധാരമപ്യനാധാരം\\
നിര്‍മമമാത്മാരാമം പ്രണതോസ്മ്യഹം രാമം.’\\
ഇസ്തുതികേട്ടു രാമചന്ദ്രനും പ്രസന്നനായ്\\
പത്രീന്ദ്രന്‍ തന്നോടരുളിച്ചെയ്തു മധുരമായ്:\\
‘അസ്തു തേ ഭദ്രം ഗച്ഛ പദം മേ വിഷ്ണോഃ പരം\\
ഇസ്തോത്രമെഴുതിയും പഠിച്ചും കേട്ടുകൊണ്ടാല്‍\\
ഭക്തനായുള്ളവനു വന്നീടും മത്സാരൂപ്യം\\
പക്ഷീന്ദ്രാ! നിന്നെപ്പോലെ മല്‍പരായണനായാല്‍.’\\
ഇങ്ങനെ രാമവാക്യം കേട്ടൊരു പക്ഷിശ്രേഷ്ഠ-\\
നങ്ങനെ തന്നെ വിഷ്ണു സാരൂപ്യം പ്രാപിച്ചുപോയ്\\
ബ്രഹ്മപൂജിതമായ പദവും പ്രാപിച്ചുതേ\\
നിര്‍മലരാമനാമം ചൊല്ലുന്ന ജനം പോലെ.
\end{verse}

%%21_kabandhagathi
\section{കബന്ധഗതി}

\begin{verse}
പിന്നെ ശ്രീരാമന്‍ സുമിത്രാത്മജനോടും കൂടി\\
ഖിന്നനായ് വനാന്തരം പ്രാപിച്ചു ദുഃഖത്തോടും\\
അന്വേഷിച്ചോരോ ദിശി സീതയെക്കാണായ്കയാല്‍\\
സന്നധൈര്യേണ വനമാര്‍ഗേ സഞ്ചരിക്കുമ്പോള്‍\\
രക്ഷോരൂപത്തോടൊരു സത്വത്തെക്കാണായ്വന്നു\\
തല്‍ക്ഷണമേവം രാമചന്ദ്രനുമരുള്‍ചെയ്താന്‍:\\
‘വക്ഷസി വദനവും യോജന ബാഹുക്കളും\\
ചക്ഷുരാദികളുമില്ലെന്തൊരു സത്വമിദം?\\
ലക്ഷ്മണ! കണ്ടായോ നീ കണ്ടോളം ഭയമുണ്ടാം\\
ഭക്ഷിക്കുമിപ്പോളിവന്‍ നമ്മെയെന്നറിഞ്ഞാലും.\\
പക്ഷിയും മൃഗവുമല്ലെത്രയും ചിത്രം ചിത്രം!\\
വക്ഷ്സി വക്ത്രം കാലും തലയുമില്ല താനും\\
രക്ഷസ്സു പിടിച്ചുടന്‍ ഭക്ഷിക്കും മുമ്പേ നമ്മെ\\
രക്ഷിക്കുംപ്രകാരവും കാണ്ടീല നിരൂപിച്ചാല്‍\\
തത്ഭുജമദ്ധ്യസ്ഥന്മാരായിതു കുമാര! നാം\\
കല്പിതം ധാതാവിനാഎന്തെന്നാലതു വരും.’\\
രാഘവനേവം പറഞ്ഞീടിനോരനന്തര-\\
മാകുലമകന്നൊരു ലക്ഷ്മണനുരചെയ്താന്‍:\\
‘പോരും വ്യാകുലഭാവമെന്തിനി വിചാരിപ്പാ-\\
നോരോരോ കരം ഛേദിക്കേണം നാമിരുവരും.’\\
തല്‍ക്ഷണം ഛേദിച്ചിതു ദക്ഷിണഭുജം രാമന്‍\\
ലക്ഷ്മണന്‍ വാമകരം ഛേദിച്ചാനതു നേരം\\
രാക്ഷോവീരനുമതി വിസ്മയം പൂണ്ടു രാമ-\\
ലക്ഷ്മണന്മാരെക്കണ്ടു ചോദിച്ചാന്‍ ഭയത്തോടെ:\\
‘മത്ഭുജങ്ങളെ ഛേദിച്ചീടുവാന്‍ ശക്തന്മാരാ-\\
യിബ്ഭുവനത്തിലാരുമുണ്ടായീലിതിന്‍ കീഴില്‍\\
അത്ഭുതാകാരന്മാരാം നിങ്ങളാരിരുവരും\\
സല്‍പുരുഷന്മാരെന്നു കല്പിച്ചീടുന്നേന്‍ ഞാനും\\
ഘോര കാനനപ്രദേശത്തിങ്കല്‍ വരുവാനും\\
കാരണമെന്തു നിങ്ങള്‍ സത്യം ചൊല്ലുകവേണം.’\\
ഇത്തരം കബന്ധവാക്യങ്ങള്‍ കേട്ടൊരു പുരു-\\
ഷോത്തമന്‍ ചിരിച്ചുടനുത്തരമരുള്‍ചെയ്തു:\\
‘കേട്ടാലും ദശരഥനാമയോദ്ധ്യാധിപതി-\\
ജ്യേഷ്ഠനന്ദനനഹം രാമനെന്നല്ലോ നാമം.\\
സോദരനിവന്‍ മമ ലക്ഷ്മണനെന്നു നാമം\\
സീതയെന്നുണ്ടു മമ ഭാര്യയായൊരു നാരി.\\
പോയിതു ഞങ്ങള്‍ നായാട്ടിന്നതുനേരമതി-\\
മായാവി നിശാചരന്‍ കട്ടുകൊണ്ടങ്ങുപോയാന്‍\\
കാനനംതോറും ഞങ്ങള്‍ തിരഞ്ഞുനടക്കുമ്പോള്‍\\
കാണായി നിന്നെയതിഭീഷണവേഗത്തൊടും\\
പാണികള്‍ക്കൊണ്ടു തവ വേഷ്ടിതന്മാരാകയാല്‍\\
പ്രാണരക്ഷാര്‍ഥം ഛേദിച്ചീടിനേന്‍ കരങ്ങളും\\
ആരെടോ വികൃതരൂപം ധരിച്ചോരു ഭവാന്‍?\\
നേരോടെ പറകെ’ന്നു രാഘവന്‍ ചോദിച്ചപ്പോള്‍\\
സന്തുഷ്ടാത്മനാ പറഞ്ഞീടിനാന്‍ കബന്ധനും:\\
‘നിന്തിരുവടിതന്നെ ശ്രാരാമദേവനെങ്കില്‍\\
ധന്യനായ് വന്നേനഹം, നിന്തിരുവടിതന്നെ\\
മുന്നിലാമ്മാറു കാണായ്വന്നൊരു നിമിത്തമായ്’\\
ദിവ്യനായിരിപ്പൊരു ഗന്ധര്‍വനഹം രൂപ-\\
യൗവനദര്‍പ്പിതനായ് സഞ്ചരിച്ചീടുംകാലം\\
സുന്ദരീജനമനോധൈര്യവും ഹരിച്ചതി-\\
സുന്ദരനായോരു ഞാന‍ ക്രീഡിച്ചു നടക്കുമ്പോള്‍\\
അഷ്ടാവക്രനെക്കണ്ടു ഞാനപഹസിച്ചിതു\\
രുഷ്ടനായ് മഹാമുനി ശാപവും നല്കീടിനാന്‍\\
ദുഷ്ടനായുള്ളോരു നീ രാക്ഷസനായ് പോകെന്നാന്‍\\
തുഷ്ടനായ്പ്പിന്നെശ്ശാപാനുഗ്രഹം നല്കീടിനാന്‍.\\
സാക്ഷാല്‍ ശ്രീനാരായണന്‍ തന്തിരുവടി തന്നെ\\
മോക്ഷദന്‍ ദശരഥപുത്രനായ് ത്രേതായുഗേ\\
വന്നവതരിച്ചു നിന്‍ ബാഹുക്കളറുക്കുന്നാള്‍\\
വന്നീടുമല്ലോ ശാപമോക്ഷവും നിനക്കെടോ!\\
താപസശാപംകൊണ്ടു രാക്ഷസനായോരു ഞാന്‍\\
താപേന നടന്നീടും കാലമങ്ങൊരു ദിനം\\
ശതമന്യുവിനെപ്പാഞ്ഞടുത്തേനതിരുഷാ\\
ശതകോടിയാല്‍ തലയറുത്തു ശതമഖന്‍\\
വജ്രമേറ്റിട്ടും മമ വന്നീല മരണമ-\\
തബ്ജസംഭവന്‍ മമ തന്നൊരു വരത്തിനാല്‍.\\
വധ്യനല്ലായ്കമൂലം വൃത്തിക്കു മഹേന്ദ്രനു-\\
മുത്തമാംഗത്തെ മമ കുക്ഷിയിലാക്കീടിനാന്‍.\\
വക്ത്രപാദങ്ങള്‍ മമ കുക്ഷിയിലായ ശേഷം\\
ഹസ്തയുഗ്മവുമൊരു യോജനായതങ്ങളായ്\\
വര്‍ത്തിച്ചീടുന്നേനത്ര വൃത്തിക്കു ശക്രാജ്ഞയാ\\
സത്വസഞ്ചയം മമ ഹസ്തമ്ദ്ധ്യസ്ഥമായാല്‍\\
വക്ത്രേണ ഭാക്ഷിച്ചു ഞാന്‍ വര്‍ത്തിച്ചേനിത്രനാളു-\\
മുത്തമോത്തമ! രഘുനായക! ദയാനിധേ!\\
വഹ്നിയും ജ്വലിപ്പിച്ചു ദേഹവും ദഹിപ്പിച്ചാല്‍\\
പിന്നെ ഞാന്‍ ഭാര്യാമാര്‍ഗ്ഗമൊക്കെവേ ചൊല്ലീടുവന്‍.’\\
മേദിനി കുഴിച്ചതിലിന്ധനങ്ങളുമിട്ട്\\
വീതിഹോത്രനെ ജ്വലിപ്പിച്ചിതു സൗമിത്രിയും\\
തത്രൈവ കബന്ധദേഹം ദഹിപ്പിച്ച നേരം\\
തദ്ദേഹത്തിങ്കല്‍നിന്നങ്ങുത്ഥിതനായ്ക്കാണായി\\
ദിവ്യവിഗ്രഹത്തോടും മന്മഥസമാനനായ്\\
വര്‍വഭൂഷണപരിഭൂഷിതനായന്നേരം\\
രാമദേവനെ പ്രദക്ഷിണവും ചെയ്തു ഭക്ത്യാ\\
ഭൂമിയില്‍ സാഷ്ടാംഗമായ് വീണുടന്‍ നമസ്കാരം\\
മൂന്നുരുചെയ്തു കൂപ്പിത്തൊഴുതുനിന്നു പിന്നെ\\
മാന്യനാം ഗന്ധര്‍വനുമാനന്ദവിവശനായ്\\
കോള്‍മയിര്‍ക്കൊണ്ടു ഗദ്ഗദാക്ഷരപാണികളാം\\
കോമളപദങ്ങളാല്‍ സ്തുതിച്ചു തുടങ്ങിനാന്‍:
\end{verse}

%%22_kabandhasthuthi
\section{കബന്ധസ്തുതി}

\begin{verse}
നിന്തിരുവടിയുടെ തത്ത്വമിതൊരുവര്‍ക്കും\\
ചിന്തിച്ചാലറിഞ്ഞുകൂടാവതല്ലെന്നാകിലും\\
നിന്തിരുവടിതന്നെ സ്തുതിപ്പാന്‍ തോന്നീടുന്നു\\
സന്തതമന്ധത്വംകൊണ്ടെന്തൊരു മഹാമോഹം!\\
അന്തവുമാദിയുമില്ലാതൊരു പരബ്രഹ്മം\\
അന്തരാത്മനി തെളിഞ്ഞുണര്‍ന്നു വസിക്കണം.\\
അന്ധകാരങ്ങളകന്നാനന്ദമുദിക്കണം\\
ബന്ധവുമറ്റു മോക്ഷപ്രാപ്തിയുമരുളണം.\\
അവ്യക്തമതിസൂക്ഷ്മമായൊരു ഭവദ്രൂപം\\
സുവ്യക്തഭാവേന ദേഹദ്വയവിലക്ഷണം\\
ദൃഗ്രൂപമേക, മന്യത്സകലം ദൃശ്യം ജഡം\\
ദുര്‍ഗ്രാഹ്യമനാത്മകമാകയാലജ്ഞാനികള്‍\\
എങ്ങനെയറിയുന്നു മാനസവ്യതിരിക്തം\\
മങ്ങീടാതൊരു പരമാത്മാനം ബ്രഹ്മാനന്ദം.\\
ബുദ്ധ്യാത്മാഭാസങ്ങള്‍ക്കുള്ളൈക്യമായതു ജീവന്‍\\
ബുദ്ധ്യാദി സാക്ഷിഭൂതം ബ്രഹ്മമെന്നതും നൂനം\\
നിര്‍വികാരബ്രഹ്മണി നിഖിലാത്മനി നിത്യേ\\
നിര്‍വിഷയാഖ്യേ ലോകമജ്ഞാനമോഹവശാല്‍\\
ആരോപിക്കപ്പെട്ടൊരു തൈജസം സൂക്ഷ്മദേഹം\\
ഹൈരണ്യമതു വിരാട്പുരുഷനതിസ്ഥൂലം\\
ഭാവനാവിഷയമായൊന്നതു യോഗീന്ദ്രാണാം\\
കേവലം തത്ര കാണായീടുന്നു ജഗത്തെല്ലാം.\\
ഭൂതമായതും ഭവ്യമായതും ഭവിഷ്യത്തും\\
ഹേതുനാ മഹത്തത്ത്വാദ്യാവൃത സ്ഥൂലദേഹേ\\
ബ്രഹ്മാണ്ഡകോശേ വിരാള്‍പുരുഷേ കാണാകുന്നു\\
സന്മയമെന്നപോലെ ലോകങ്ങള്‍ പതിന്നാലും.\\
തുംഗനാം വിരാട്പുമാനാകിയ ഭഗവാന്‍ ത-\\
ന്നംഗങ്ങളല്ലോ പതിന്നാലു ലോകവും നൂനം,\\
പാതാളം പാദമൂലം പാര്‍ഷ്ണികള്‍ മഹാതലം\\
നാഥ! തേ ഗുല്‍ഫം രസാതലവും തലാതലം\\
ചാരുജാനുക്കളല്ലോ സുതലം രഘുപതേ!\\
ഊരുകാണ്ഡങ്ങള്‍ തവ വിമലമതലവും\\
ജഘനം മഹീതലം നാഭി തേ നഭസ്തലം\\
രഘുനാഥോരസ്ഥലമായതു സുരലോകം\\
കണ്ഠദേശം തേ മഹര്‍ല്ലോകമെന്നറിയേണം\\
തുണ്ഡമായതു ജനലോകമെന്നതു നൂനം\\
ശംഖദേശം തേ തപോലോകമിങ്ങതിന്‍മീതേ\\
പങ്കജയോനിവാസമാകിയ സത്യലോകം\\
ഉത്തമാംഗം തേ പുരുഷോത്തമ! ജഗല്‍പ്രഭോ!\\
സത്താമാത്രക! മേഘജാലങ്ങള്‍ കേശങ്ങളും.\\
ശക്രാദിലോകപാലന്മാരെല്ലാം ഭുജങ്ങള്‍ തേ\\
ദിക്കുകള്‍ കര്‍ണങ്ങളുമശ്വികള്‍ നാസികയും.\\
വക്ത്രമായതു വഹ്നി നേത്രമാദിത്യന്‍ തന്നെ\\
ചിത്രമെത്രയും മനസ്സായതു ചന്ദ്രനല്ലോ.\\
ഭ്രൂഭംഗമല്ലോ കാലം ബുദ്ധി വാക്പതിയല്ലോ\\
കോപകാരണമഹങ്കാരമായതു രുദ്രന്‍.\\
വാക്കെല്ലാം ഛന്ദസ്സുകള്‍ ദംഷ്ട്രകള്‍ യമനല്ലോ\\
നക്ഷത്രപംക്തിയെല്ലാം ദ്വിജപംക്തികളല്ലോ\\
ഹാസമായതു മോഹകാരിണിമഹാമായ\\
വാസനാസൃഷ്ടിസ്തവാപാംഗമോക്ഷണമല്ലോ.\\
ധര്‍മം നിന്‍ പുരോഭാഗമധര്‍മം പൃഷ്ഠഭാഗം\\
ഉന്മേഷനിമേഷങ്ങള്‍ ദിനരാത്രികളല്ലോ\\
സപ്തസാഗരങ്ങള്‍ നിന്‍ കുക്ഷിദേശങ്ങളല്ലോ\\
സപ്തമാരുതന്മാരും നിശ്വാസഗണമല്ലോ.\\
നദികളെല്ലാം തവ നാഡികളാകുന്നതും\\
പൃഥിവീധരങ്ങള്‍ പോലസ്ഥികളാകുന്നതും\\
വൃക്ഷാദ്യൗഷധങ്ങള്‍ തേ രോമങ്ങളാകുന്നതും\\
ത്ര്യക്ഷനാം ദേവന്‍തന്നെ ഹൃദയമാകുന്നതും\\
വൃഷ്ടിയായതും തവ രേതസ്സെന്നറിയണം\\
പുഷ്ടമാം മഹീപതേ! കേവലജ്ഞാനശക്തി\\
സ്ഥൂലമായുള്ള വിരാട്പുര്ഷരൂപം തവ\\
കാലേ നിത്യവും ധ്യാനിക്കുന്നവനുണ്ടാം മുക്തി;\\
നിന്തിരുവടിയൊഴിഞ്ഞില്ല കിഞ്ചന വസ്തു\\
സന്തതമീദൃഗ്രൂപം ചിന്തിച്ചു വണങ്ങുന്നേന്‍\\
ഇക്കാലമിതില്‍ക്കാളും മുഖ്യമായിരിപ്പോന്നി-\\
തിക്കാണാകിയ രൂപമെപ്പോഴും തോന്നീടണം\\
താപസവേഷം ധരാവല്ലഭം ശാന്താകാരം\\
ചാപേഷുകരം ജടാവല്ക്കലവിഭൂഷണം\\
കാനനേ വിചിന്വനം ജാനകീം സലക്ഷ്മണം\\
മാനവശ്രേഷ്ഠം മനോജ്ഞം മനോഭവസമം\\
മാനസേ വസിപ്പതിന്നാലയം ചിന്തിക്കുന്നേന്‍\\
ഭാനുവംശോല്‍ഭൂതനാം ഭഗവന്‍! നമോ നമഃ\\
സര്‍വജ്ഞന്‍ മഹേശ്വരനീശ്വരന്‍ മഹാദേവന്‍\\
ശര്‍വനവ്യയന്‍ പരമേഷ്വരിയോടും കൂടി\\
നിന്തിരുവടിയേയും ധ്യാണിച്ചുകൊണ്ടു കാശ്യാം\\
സന്തതമിരുന്നരുളീടുന്നു മുക്ത്യര്‍ത്ഥമായ്.\\
തത്രൈവ മുമുക്ഷുക്കളായുള്ള ജനങ്ങള്‍ക്കു\\
തത്ത്വബോധാര്‍ത്ഥം നിത്യം താരകബ്രഹ്മവാക്യം\\
രാമരാമേതി കനിഞ്ഞുപദേശവും നല്കി-\\
സ്സോമനാം നാഥന്‍ വസിച്ചീടുന്നു സദാകാലം\\
പരമാത്മാവു പരബ്രഹ്മം നിന്തിരുവടി\\
പരമേശ്വരനായതറിഞ്ഞു വഴിപോലെ\\
മൂഢന്മാര്‍ ഭവത്തത്ത്വമെങ്ങനെയറിയുന്നു\\
മൂടിപ്പോകയാല്‍ മഹാമായാമോഹാന്ധകാരേ?\\
രാമഭദ്രായ പരമാത്മനേ നമോ നമഃ\\
രാമചന്ദ്രായ ജഗത്സാക്ഷിണേ നമോ നമഃ\\
പാഹിമാം ജഗന്നാഥ! പരമാനന്ദരൂപ!\\
പാഹി സൗമിത്രിസേവ്യ! പാഹിമാം ദയാനിധേ!\\
നിന്മഹാമായാദേവിയെന്നെ മോഹിപ്പിച്ചീടാ-\\
യ്കംബുജവിലോചന! സന്തതം നമസ്കാരം.’\\
ഇത്ഥമര്‍ത്ഥിച്ചു ഭക്ത്യാ സ്തുതിച്ച ഗന്ധര്‍വനോ-\\
ടുത്തമ പുരുഷനാം ദേവനുമരുള്‍ചെയ്തു:\\
‘സന്തുഷ്ടനായേന്‍ തവ സ്തുത്യാ നിശ്ചലഭക്ത്യാ\\
ഗന്ധര്‍വശ്രേഷ്ഠ! ഭവാന്‍ മല്‍പദം പ്രാപിച്ചാലും\\
സ്ഥാനം മേ സനാതനം യോഗീന്ദ്രഗമ്യം പര-\\
മാനന്ദം പ്രാപിക്ക നീ മല്‍ പ്രസാദത്താലെടോ!\\
അത്രയുമല്ല പുനരോന്നനുഗ്രഹിപ്പന്‍ ഞാ-\\
നിസ്തോത്രം ഭക്ത്യാ ജപിച്ചീടുന്ന ജനങ്ങള്‍ക്കും\\
മുക്തി സംഭവിച്ചീടുമില്ല സംശയമേതും;\\
ഭക്തനാം നിനക്കധഃപതനമിനി വരാ.’\\
ഇങ്ങനെ വരം വാങ്ങിക്കൊണ്ടു ഗന്ധര്‍വശ്രേഷ്ഠന്‍\\
മംഗലം വരുവാനായ്ത്തൊഴുതു ചൊല്ലീടിനാന്‍:\\
‘മുമ്പിലാമ്മാറു കാണാം മതംഗാശ്രമം തത്ര\\
സമ്പ്രതി വസിക്കുന്നു ശബരീതപസ്വിനി\\
ത്വല്‍പാദാംബുജഭക്തികൊണ്ടേറ്റം പവിത്രയാ-\\
യെപ്പോഴും ഭവാനെയും ധ്യാനിച്ചു വിമുക്തയായ്\\
അവളെച്ചെന്നു കണ്ടാല്‍ വൃത്താന്തം ചൊല്ലുമവ-\\
ളവനീസുതതന്നെ ലഭിക്കും നിങ്ങള്‍ക്കെന്നാല്‍.’
\end{verse}

%%23_shabaryaashramapravesham
\section{ശബര്യാശ്രമപ്രവേശം}

\begin{verse}
ഗന്ധര്‍വനേവം ചൊല്ലി മറഞ്ഞോരനന്തരം\\
സന്തുഷ്ടന്മാരായൊരു രാമലക്ഷ്മണന്മാരും\\
ഘോരമാം വനത്തൂടെ മന്ദമന്ദം പോയ്ചെന്നു\\
ചാരുതചേര്‍ത്ത ശബര്യാശ്രമമകം പുക്കാര്‍.\\
സംഭ്രമത്തോടും പ്രത്യുത്ഥായ താപസി ഭക്ത്യാ\\
സമ്പതിച്ചിതു പാദാംഭോരുഹയുഗത്തിങ്കല്‍.\\
സന്തോഷപൂര്‍ണാശ്രുനേത്രങ്ങളോടവളുമാ-\\
മന്ദമുള്‍ക്കൊണ്ടു പാദ്യാര്‍ഘ്യാസനാദികളാലേ\\
പൂജിച്ചു തല്‍പാദതീര്‍ഥാഭിഷേകവും ചെയ്തു\\
ഭോജനത്തിനു ഫലമൂലങ്ങള്‍ നല്കീടിനാള്‍.\\
പൂജയും പരിഗ്രഹിച്ചാനന്ദിച്ചിരുന്നിതു\\
രാജീവനേത്രന്മാരാം രാജനന്ദനന്മാരും\\
അന്നേരം ഭക്തിപൂണ്ടു തൊഴുതു ചൊന്നാളവള്‍:\\
‘ധന്യയായ്വന്നേനഹമിന്നുപുണ്യാതിരേകാല്‍.\\
എന്നുടെ ഗുരുഭൂതന്മാരായ മുനിജനം\\
നിന്നെയും പൂജിച്ചനേകായിരത്താണ്ടു വാണാര്‍\\
അന്നു ഞാനവരെയും ശുശ്രൂഷിച്ചിരുന്നിതു\\
പിന്നെപ്പോയ് ബ്രഹ്മപദം പ്രാപിച്ചാരവര്‍കളും.\\
എന്നോടു ചൊന്നാരവരേതുമേ ഖേദിയാതെ\\
ധന്യേ! നീ വസിച്ചാലുമിവിടെത്തന്നെ നിത്യം.\\
പന്നഗശായി പരന്‍ പുരുഷന്‍ പരമാത്മാ\\
വന്നവതരിച്ചിതു രാക്ഷസവധാര്‍ത്ഥമായ്\\
നമ്മെയും ധര്‍മത്തെയും രക്ഷിച്ചുകൊള്‍വാനിപ്പോള്‍\\
നിര്‍മലന്‍ ചിത്രകൂടത്തിങ്കല്‍ വന്നിരിക്കുന്നു.\\
വന്നിടുമിവിടേക്കു രാഘവനെന്നാലവന്‍-\\
തന്നെയും കണ്ടു ദേഹത്യാഗവും ചെയ്താലും നീ\\
വന്നീടുമെന്നാല്‍ മോക്ഷം നിനക്കുമെന്നു നൂനം’\\
വന്നിതവ്വണ്ണം ഗുരുഭാഷിതം സത്യമല്ലോ.\\
നിന്തിരുവടിയുടെ വരവും പാര്‍ത്തു പാര്‍ത്തു\\
നിന്തിരുവടിയേയു ധ്യാനിച്ചു വസിച്ചു ഞാന്‍\\
ശ്രീപാദം കണ്ടുകൊള്‍വാന്‍ മല്‍ഗുരുഭൂതന്മാരാം\\
താപസന്മാര്‍ക്കുപോലും യോഗം വന്നീലയല്ലോ.\\
ജ്ഞാനമില്ലാതെ ഹീനജാതിയിലുള്ള മൂഢ\\
ഞാനിതിനൊട്ടുമധികാരിണിയല്ലയല്ലോ.\\
വാങ്മനോവിഷയമല്ലാതൊരു ഭവദ്രൂപം\\
കാണ്മാനുമവകാശം വന്നതു മഹാഭാഗ്യം\\
തൃക്കഴലിനെ കൂപ്പിസ്തുതിച്ചുകൊള്‍വാനുമി\\
ങ്ങുള്‍ക്കമലത്തിലറിയപ്പോകാ ദയാനിധേ!’\\
രാഘവനതുകേട്ടു ശബരിയോടു ചൊന്നാ-\\
‘നാകുലം കൂടാതെ ഞാന്‍ പറയുന്നതു കേള്‍ നീ.\\
പൂരുഷസ്ത്രീജാതീനാമാശ്രമാദികളല്ല\\
കാരണം മമ ഭജനത്തിനു ജഗത്ത്രയേ.\\
ഭക്തിയൊന്നൊഴിഞ്ഞു മറ്റില്ല കാരണമേതും\\
മുക്തിവന്നീടുവാനുമില്ല മറ്റേതുമൊന്നും.\\
തീര്‍ഥസ്നാനാദി തപോദാനവേദാധ്യയന-\\
ക്ഷേത്രോപവാസയാഗാദ്യഖില കര്‍മങ്ങളാല്‍\\
ഒന്നിനാലൊരുത്തനും കണ്ടുകിട്ടുകയില്ല-\\
യെന്നെ മല്‍ഭക്തിയൊഴിഞ്ഞൊന്നുകൊണ്ടൊരുനാളും.\\
ഭക്തിസാധനം സംക്ഷേപിച്ചു ഞാന്‍ ചൊല്ലീടുവ-\\
നുത്തമേ! കേട്ടുകൊള്‍ക മുക്തിവന്നീടുവാനായ്\\
മുഖ്യസാധനമല്ലോ സജ്ജനസംഗം, പിന്നെ\\
മല്‍ക്കഥാലാപം രണ്ടാം സാധനം, മൂത്താമതും\\
മല്‍ഗുണേരണം, പിന്നെ മദ്വചോവ്യാഖ്യാതൃത്വം\\
മല്‍ക്കലാജാതാചാര്യോപാസന മഞ്ചാമതും,\\
പുണ്യശീലത്വം യമനിയമാദികളോടു-\\
മെന്നെ മുട്ടാതെ പൂജിക്കെന്നുള്ളതാറാമതും,\\
മന്മന്ത്രോപാസകത്വമേഴാമ, തെട്ടാമതും\\
മംഗലശീലേ! കേട്ടു ധരിച്ചുകൊള്ളേണം നീ\\
സര്‍വഭൂതങ്ങളിലും മന്മതിയുണ്ടാകയും\\
സര്‍വദാ മല്‍ഭക്തന്മാരില്‍ പാരമാസ്തിക്യവും\\
സര്‍വബാഹ്യാര്‍ഥങ്ങളില്‍ വൈരാഗ്യം ഭവിക്കയും\\
സര്‍വലോകാത്മാ ഞാനെന്നെപ്പോഴു മുറയ്ക്കയും,\\
മത്തത്ത്വവിചാരം കേളൊമ്പതാമതു ഭദ്രേ!\\
ചിത്തശുദ്ധിക്കു മൂലമാദിസാധനം നൂനം.\\
ഉക്തമായിതു ഭക്തിസാധനം നവവിധ-\\
മുത്തമേ! ഭക്തി നിത്യമാര്‍ക്കുള്ളു വിചാരിച്ചാല്‍.\\
തിര്യഗ്യോനിജങ്ങള്‍ക്കെന്നാകിലും മൂഢമാരാം\\
നാരികള്‍ക്കെന്നാകിലും പൂരുഷനെന്നാകിലും\\
പ്രേമലക്ഷണയായ ഭക്തിസംഭവിക്കുമ്പോള്‍\\
വാമലോചനേ! മമ തത്ത്വാനുഭൂതിയുണ്ടാം.\\
തത്ത്വാനുഭവസിദ്ധനായാല്‍ മുക്തിയും വരും\\
തത്ര ജന്മനി മര്‍ത്ത്യന്നുത്തമതപോധനേ!\\
ആകയാല്‍ മോക്ഷത്തിനു കാരണം ഭക്തിതന്നെ\\
ഭാഗവതാഢ്യേ! ഭഗവല്‍പ്രിയേ! മുനിപ്രിയേ!\\
ഭക്തിയുണ്ടാകകൊണ്ടു കാണായ്വന്നിതു തവ\\
മുക്തിയുമടുത്തിതു നിനക്കു തപോധനേ!\\
ജാനകീമാര്‍ഗമറിഞ്ഞീടില്‍ നീ പറയണം\\
കേന വാ നീതാ സീതാ മല്‍പ്രിയാ മനോഹരി?\\
രാഘവവാക്യമേവം കേട്ടൊരു ശബരിയു-\\
മാകുലമകലുമാറാദരാലുരചെയ്താള്‍:\\
‘സര്‍വവുമറിഞ്ഞിരിക്കുന്ന നിന്തിരുവടി\\
സര്‍വജ്ഞനെന്നകിലും ലോകാനുസരണാര്‍ഥം\\
ചോദിച്ചമൂലം പറഞ്ഞീടുവന്‍ സീതാദേവി\\
ഖേദിച്ചു ലങ്കാപുരിതന്നില്‍ വാഴുന്നൂ നൂനം\\
കൊണ്ടുപോയതു ദശകണ്ഠനെന്നറിഞ്ഞാലും\\
കണ്ടിതു ദിവ്യദൃശാ തണ്ടലര്‍മകളെ ഞാന്‍\\
മുമ്പിലാമ്മാറു കുറഞ്ഞൊന്നു തെക്കോട്ടു ചെന്നാല്‍\\
പമ്പയാം സരസ്സിനെക്കാണാം, തല്‍ പുരോഭാഗേ\\
പശ്യ പര്‍വതവരമൃശ്യമൂകാഖ്യം, തത്ര\\
വിസ്വസിച്ചിരിക്കുന്നു സുഗ്രീവന്‍ കപിശ്രേഷ്ഠന്‍\\
നാലുമന്ത്രികളോടും കൂടെ മാര്‍ത്താണ്ഡാത്മജന്‍;\\
ബാലിയെപ്പേടിച്ചു സങ്കേതമായനുദിനം;\\
ബാലിക്കുമുനിശാപം പേടിച്ചു ചെന്നുകൂടാ\\
പാലനം ചെയ്ക ഭവാനവനെ വഴിപോലെ.\\
സഖ്യവും ചെതുകൊള്‍ക സുഗ്രീവന്‍ തന്നോടെന്നാല്‍\\
ദുഃഖങ്ങളെല്ലാം തീര്‍ന്നുകാര്യവും സാധിച്ചീടും.\\
എങ്കില്‍ ഞാനഗ്നിപ്രവേശം ചെയ്തു ഭവല്‍പാദ-\\
പങ്കജത്തോടു ചേര്‍ന്നുകൊള്ളുവാന്‍ തുടങ്ങുന്നു.\\
പാര്‍ക്കേണം മുഹൂര്‍ത്തമാത്രം ഭവാനത്രൈവ മേ\\
തീര്‍ക്കേണം മയാകൃതബന്ധനം ദയാനിധേ!’\\
ഭക്തിപൂണ്ടിത്ഥമുക്ത്വാ ദേഹത്യാഗവും ചെയ്തു\\
മുക്തിയും സിദ്ധിച്ചിതു ശബരിക്കതുകാലം.\\
ഭക്തവത്സലന്‍ പസാദിക്കിലിന്നവര്‍ക്കെന്നി-\\
ല്ലെത്തീടും മുക്തി നീചജാതികള്‍ക്കെന്നാകിലും.\\
പുഷ്കരനേത്രന്‍ പ്രസാദിക്കിലോ ജന്തുക്കള്‍ക്ക്‍\\
ദുഷ്കരമായിട്ടൊന്നുമില്ലെന്നു ധരിക്കണം.\\
ശ്രീരാമഭക്തിതന്നെ മുക്തിയെസ്സിദ്ധിപ്പിക്കും\\
ശ്രീരാമപാദാംഭോജം സേവിച്ചു കൊള്‍ക നിത്യം\\
ഓരോരോ മന്ത്രതന്ത്ര ധ്യാനകര്‍മാദികളും\\
ദൂരെസ്സന്ത്യജിച്ചു തന്‍ ഗുരുനാഥോപദേശാല്‍\\
ശ്രീരാമചന്ദ്രന്‍ തന്നെ ധ്യാനിച്ചുകൊള്‍ക നിത്യം\\
ശ്രീരാമമന്ത്രം ജപിച്ചീടുക സദാകാലം\\
ശ്രീരാമചന്ദ്രകഥ കേള്‍ക്കയും ചൊല്ലുകയും\\
ശ്രീരാമഭക്തന്മാരെപ്പൂജിച്ചു കൊള്ളുകയും\\
ശ്രീരാമമയം ജഗത്സര്‍വമെന്നുറയ്ക്കുമ്പോള്‍\\
ശ്രീരാമചന്ദ്രന്‍ തന്നോടൈക്യവും പ്രാപിച്ചീടാം.\\
രാമരാമേതി ജപിച്ചീടുക സദാകാലം\\
ഭാമിനി! ഭദ്രേ! പരമേശ്വരി! പത്മേക്ഷണേ!\\
ഇത്ഥമീശ്വരന്‍ പരമേശ്വരിയോടു രാമ-\\
ഭദ്രവൃത്താന്തമരുള്‍ചെയ്തതു കേട്ടനേരം\\
ഭക്തികൊണ്ടേറ്റം പരവശയായ് ശ്രീരാമങ്കല്‍\\
ചിത്തവുമുറപ്പിച്ചു ലയിച്ചുരുദ്രാണിയും,\\
പൈങ്കിളിപ്പൈതല്‍താനും പരമാനന്ദം പൂണ്ടു\\
ശങ്കര! ജയിച്ചരുളെന്നിരുന്നരുളിനാള്‍.
\end{verse}

\begin{center}
ഇത്യദ്ധ്യാത്മരാമായണേ ഉമാമഹേശ്വരസംവാദേ\\
ആരണ്യകാണ്ഡം സമാപ്തം
\end{center}
