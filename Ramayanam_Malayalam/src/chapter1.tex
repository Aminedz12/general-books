\chapter{ബാലകാണ്ഡം}

\section{ഹരിഃ ശ്രീഗണപതയേ നമഃ അവിഘ്നമസ്തു}

\begin{verse}
ശ്രീരാമ! രാമ! രാമ! ശ്രീരാമചന്ദ്ര! ജയ\\
ശ്രീരാമ! രാമ! രാമ! ശ്രീരാമഭദ്ര! ജയ\\
ശ്രീരാമ! രാമ! രാമ! സീതാഭിരാമ! ജയ\\
ശ്രീരാമ! രാമ! രാമ! ലോകാഭിരാമ! ജയ\\
ശ്രീരാമ! രാമ! രാമ! രാവണാന്തക! രാമ!\\
ശ്രീരാമ! മമ ഹൃദി രമതാം രാമ! രാമ!\\
ശ്രീരാഘവാത്മാ രാമ! ശ്രീരാമ! രമാപതേ!\\
ശ്രീരാമ! രമണീയവിഗ്രഹ! നമോസ്തുതേ.\\
നാരായണായ നമോ നാരായണായ നമോ\\
നാരായണായ നമോ നാരായണായ നമഃ
\end{verse}

\begin{verse}
ശ്രീരാമനാമം പാടി വന്ന പൈങ്കിളിപ്പെണ്ണേ!\\
ശ്രീരാമചരിതം നീ ചൊല്ലീടു മടിയാതെ.\\
ശാരികപ്പൈത‌‌‌ല്‍താനും വന്ദിച്ചു വന്ദ്യന്മാരെ\\
ശ്രീരാമസ്മൃതിയോടെ പറഞ്ഞു തുടങ്ങിനാള്‍.\\
കാരണനായ ഗണനായക‍ന്‍ ബ്രഹ്മാത്മകന്‍\\
കാരുണ്യമൂര്‍ത്തി ശിവശക്തിസംഭവന്‍ ദേവന്‍\\
വാരണമുഖന്‍ മമ പ്രാരബ്ധ വിഘ്നങ്ങളെ\\
വാരണം ചെയ്തീടുവാനാവോളം വന്ദിക്കുന്നേന്‍\\
വാണീടുകനാരതമെന്നുടെ നാവുതന്മേല്‍\\
വാണിമാതാവേ! വര്‍ണവിഗ്രഹേ! വേദാത്മികേ!\\
നാണമെന്നിയേ മുദാ നാവിന്മേല്‍ നടനം ചെയ്-\\
കേണാങ്കാനനേ! യഥാ കാനനേ ദിഗംബരന്‍\\
വാരിജോത്ഭവമുഖവാരിജവാസേ! ബാലേ!\\
വാരിധിതന്നില്‍ തിരമാലകളെന്നപോലെ\\
ഭാരതീ! പദാവലി തോന്നേണം കാലേ കാലേ\\
പാരാതെ സലക്ഷണം മേന്മേല്‍ മംഗലശീലേ!\\
വൃഷ്ണിവംശത്തില്‍വന്നു കൃഷ്ണനായ് പിറന്നൊരു\\
വിഷ്ണു വിശ്വാത്മാ വിശേഷിച്ചനുഗ്രഹിക്കേണം\\
വിഷ്ണുജോത്ഭവസുതനന്ദനപുത്രന്‍ വ്യാസന്‍\\
വിഷ്ണു താന്‍തന്നെ വന്നു പിറന്ന തപോധനന്‍\\
വിഷ്ണുതന്‍മായാഗുണചരിത്രമെല്ലാം കണ്ട\\
കൃഷ്ണനാം പുരാണകര്‍ത്താവിനെ വണങ്ങുന്നേന്‍.\\
നാന്മാറനേരായ രാമായണം ചമയ്ക്കയാല്‍\\
നാന്മുഖനുള്ളില്‍ ബഹുമാനത്തെ വളര്‍ത്തൊരു\\
വാല്മീകി കവിശ്രേഷ്ഠനാകിയ മഹാമുനി-\\
താന്‍ മമ വരം തരികെപ്പൊഴും വന്ദിക്കുന്നേന്‍\\
രാമനാമത്തെസ്സദാകാലവും ജപിച്ചീടും\\
കാമനാശനനുമാവല്ലഭന്‍ മഹേശ്വരന്‍\\
ശ്രീമഹാദേവന്‍ പരമേശ്വരന്‍ സര്‍വേശ്വരന്‍\\
മാമകേ മനസി വാണീടുവാന്‍ വന്ദിക്കുന്നേന്‍\\
വാരിജോത്ഭവനാദിയാകിയ ദേവന്മാരും\\
നാരദപ്രമുഖന്മാരാകിയ മുനികളും\\
വാരിജശരാരാതി പ്രാണനാഥയും മമ\\
വാരിജാമകളായ ദേവിയും തുണയ്ക്കേണം.\\
കാരണഭൂതന്മാരാം ബ്രാഹ്മണരുടെ ചര-\\
ണാരുണാംബുജലീനപംസുസഞ്ചയം മമ\\
ചേതോദര്‍പ്പണത്തിന്റെ മാലിന്യമെല്ലാം തീര്‍ത്തു\\
ശോധന ചെയ്തീടുവാനാവോളം വന്ദിക്കുന്നേന്‍.\\
ആധാരം നാനാജഗന്മയനാം ഭഗവാനും\\
വേദമെന്നല്ലോ ഗുരുനാഥന്‍ താനരുള്‍ചെയ്തു;\\
വേദത്തിനാധാരഭൂതന്മാരിക്കാണായൊരു\\
ഭൂദേവ പ്രവരന്മാര്‍ തദ്വരശാപാദികള്‍\\
ധാതൃശങ്കരവിഷ്ണുപ്രമുഖന്മാര്‍ക്കും മതം\\
വേദജ്ഞോത്തമന്മാര്‍ മാഹാത്മ്യങ്ങളാര്‍ക്കു ചൊല്ലാം?\\
പാദസേവകനായ ഭക്തനാം ദാസന്‍ ബ്രഹ്മ-\\
പാദജനജ്ഞാനിനാമാദ്യനായുള്ളോരു ഞാന്‍\\
വേദസമ്മിതമായ് മുമ്പുള്ള ശ്രീരാമായണം\\
ബോധഹീനന്‍മാര്‍ക്കറിയാംവണ്ണം ചൊല്ലീടുന്നേന്‍.\\
വേദവേദാംഗവേദാന്താദി വിദ്യകളെല്ലാം\\
ചേതസി തെളിഞ്ഞുണര്‍ന്നാവോളം തുണയ്ക്കേണം.\\
സുരസംഹതിപതി തദനു സ്വാഹാപതി\\
വരദന്‍ പിതൃപതി നിര്യതി ജലപതി\\
തരസാ സദാഗതി സദയം നിധിപതി\\
കരുണാനിധി പശുപതി നക്ഷത്രപതി\\
സുരവാഹിനീപതിതനയന്‍ ഗണപതി\\
സുരവാഹിനീപതി പ്രമഥഭൂതപതി\\
ശ്രുതിവാക്യാത്മാ ദിനപതി ഖേടാനാം പതി\\
ജഗതി ചരാചരജാതികളായുള്ളോരും\\
അഗതിയായോരടിയന്നനുഗ്രഹിക്കേണ-\\
മകമേ സുഖമേ ഞാനനിശം വന്ദിക്കുന്നേന്‍.\\
അഗ്രജന്‍ മമ സതാം വിദുഷാമഗ്രേസരന്‍\\
മല്‍ഗുരുനാഥനനേകാന്തേവാസികളോടും\\
ഉള്‍ക്കുരുന്നിങ്കല്‍ വാഴ്ക, രാമനാമാചാര്യനും\\
മുഖ്യന്മാരായ ഗുരുഭൂതന്മാര്‍ മറ്റുള്ളോരും.\\
ശ്രീരാമായണം പുരാ വിരിഞ്ചവിരചിതം\\
നൂറുകോടി ഗ്രന്ഥമുണ്ടി,ല്ലതു ഭൂമിതന്നില്‍\\
രാമനാമത്തെജ്ജപിച്ചോരു കാട്ടാളന്‍ മുന്നം\\
മാമുനിപ്രവരനായ് വന്നതു കണ്ടു ധാതാ\\
ഭൂമിയിലുള്ള ജന്തുക്കള്‍ക്കു മോക്ഷാര്‍ഥമിനി\\
ശ്രീമഹാരാമായണം ചമയ്ക്കെന്നരുള്‍ചെയ്തു\\
വീണാപാണിയുമുപദേശിച്ചു രാമായണം\\
വാണിയും വാല്മീകിതന്‍ നാവിന്മേല്‍ വാണീടിനാള്‍\\
വാണീടുകവ്വണ്ണമെന്‍ നാവിന്മേലേവം ചൊല്‍വാന്‍\\
നാണമാകുന്നുതാനുമതിനെന്താവതിപ്പോള്‍?\\
വേദശാസ്ത്രങ്ങള്‍ക്കധികാരിയല്ലെന്നതോര്‍ത്തു\\
ചേതസി സര്‍വം ക്ഷമിച്ചീടുവിന്‍ കൃപയാലെ.\\
അധ്യാത്മപ്രദീപകമത്യന്തം രഹസ്യമി-\\
തധ്യാത്മരാമായണം മൃത്യുശാസനപ്രോക്തം\\
അധ്യയനം ചെയ്തീടും മര്‍ത്യജന്മികള്‍ക്കെല്ലാം\\
മുക്തിസിദ്ധിക്കുമസന്ദിഗ്ദ്ധമിജ്ജന്മംകൊണ്ടേ...\\
ഭക്തികൈക്കൊണ്ടു കേട്ടുകൊള്ളുവിന്‍ ചൊല്ലീടുവ-\\
നെത്രയും ചുരുക്കി ഞാന്‍ രാമമാഹാത്മ്യമെല്ലാം\\
ബുദ്ധിമത്തുക്കളായോരിക്കഥ കേള്‍ക്കുന്നാകില്‍\\
ബദ്ധരാകിലുമുടന്‍ മുക്തരായ് വന്നുകൂടും.\\
ധാത്രീഭാരത്തെത്തീര്‍പ്പാന്‍ ബ്രഹ്മാദിദേവഗണം\\
പ്രാര്‍ഥിച്ചു ഭക്തിപൂര്‍വം സ്തോത്രം ചെയ്തതുമൂലം\\
ദുഗ്ദ്ധാബ്ധിമധ്യേ ഭോഗിസത്തമനായീടുന്ന\\
മെത്തമേല്‍ യോഗനിദ്ര ചെയ്തീടും നാരായണന്‍\\
ധാത്രീമണ്ഡലംതന്നില്‍ മാര്‍ത്താണ്ഡകുലത്തിങ്കല്‍\\
ധാത്രീന്ദ്രവീരന്‍ ദശരഥനു തനയനായ്\\
രാത്രിചാരികളായ രാവണാദികള്‍തമ്മെ\\
മാര്‍ത്താണ്ഡാത്മജപുരം പ്രാപിപ്പിച്ചോരുശേഷം\\
ആദ്യമാം ബ്രഹ്മത്വം പ്രാപിച്ച വേദാന്തവാക്യ-\\
വേദ്യനാം സീതാപതി ശ്രീപാദം വന്ദിക്കുന്നേന്‍
\end{verse}

\section{ഉമാമഹേശ്വരസംവാദം}

\begin{verse}
കൈലാസാചലേ സൂര്യകോടിശോഭിതേ വിമ-\\
ലാലയേ രത്നപീഠേ സംവിഷ്ടം ധ്യാനനിഷ്ഠം\\
ഫാലലോചനം മുനിസിദ്ധദേവാദിസേവ്യം\\
നീലലോഹിതം നിജഭര്‍ത്താരം വിശ്വേശ്വരം
\end{verse}

\begin{verse}
വന്ദിച്ചു വാമോത്സംഗേ വാഴുന്ന ഭഗവതി\\
സുന്ദരി ഹൈമവതി ചോദിച്ചു ഭക്തിയോടെ.\\
സര്‍വാത്മാവായ നാഥ! പരമേശ്വര! പോറ്റീ!\\
സര്‍വലോകാവാസ! സര്‍വേശ്വര! മഹേശ്വരാ!\\
ശര്‍വ! ശങ്കര! ശരണാഗതജനപ്രിയ!\\
സര്‍വദേവേശ! ജഗന്നായക! കാരുണ്യാബ്ധേ!\\
അത്യന്തം രഹസ്യമാം വസ്തുവെന്നിരിക്കിലു-\\
മെത്രയും മഹാനുഭാവന്മാരായുള്ള ജനം\\
ഭക്തിവിശ്വാസശുശ്രൂഷാദിഗള്‍ കാണുന്തോറും\\
ഭാക്തന്മാര്‍ക്കുപദേശം ചെയ്തീടുമെന്നു കേള്‍പ്പൂ.\\
ആകയാല്‍ ഞാനുണ്ടൊന്നു നിന്തിരുവടിതന്നോ-\\
ടാകാംക്ഷാപരവശചേതസാ ചോദിക്കുന്നു.\\
കാരുണ്യമെന്നെക്കുറിച്ചുണ്ടെങ്കിലെനിക്കിപ്പോള്‍\\
ശ്രീരാമദേവതത്ത്വമുപദേശിച്ചീടണം.\\
തത്ത്വഭേദങ്ങള്‍ വിജ്ഞാനജ്ഞാനവൈരാഗ്യാദി\\
ഭക്തിലക്ഷണം സാംഖ്യയോഗഭേദാദികളും\\
ക്ഷേത്രോപവാസഫലം യാഗാദികര്‍മഫലം\\
തീര്‍ഥസ്നാനാദിഫലം ദാനധര്‍മാദി ഫലം\\
വര്‍ണധര്‍മങ്ങള്‍ പുനരാശ്രമധര്‍മങ്ങളു-\\
മെന്നിവയെല്ലാമെന്നോടൊന്നൊഴിയാതവണ്ണം\\
നിന്തിരുവടിയരുള്‍ചെയ്തു കേട്ടതുമൂലം\\
സന്തോഷമകതാരിലേറ്റവുമുണ്ടായ് വന്നു\\
ഭന്ധമോക്ഷങ്ങളുടെ കാരണം കേള്‍ക്കമൂല-\\
മന്ധത്വം തീര്‍ന്നുകൂടി ചേതസി ജഗല്‍പതേ!\\
“ശ്രീരാമദേവന്‍തന്റെ മാഹാത്മ്യം കേള്‍പ്പാനുള്ളില്‍\\
പാരമാഗ്രഹമുണ്ടു ഞാനതിന്‍ പാത്രമെങ്കില്‍\\
കാരുണ്യാംബുധേ! കനിഞ്ഞരുളിച്ചെയ്തീടണ-\\
മാരും നിന്തിരുവടിയൊഴിഞ്ഞില്ലതു ചൊല്‍വാന്‍.”\\
ഈശ്വരി കാര്‍ത്ത്യായനി പാര്‍വതി ഭഗവതി\\
ശാശ്വതനായ പരമേശ്വരനോടീവണ്ണം\\
ചോദ്യം ചെയ്തതു കേട്ടു തെളിഞ്ഞു ദേവന്‍ ജഗ-\\
ദാദ്യനീശ്വരന്‍ മന്ദഹാസം പൂണ്ടരുള്‍ ചെയ്തു.\\
“ധന്യേ! വല്ലഭേ! ഗിരികന്യേ! പാര്‍വതീ! ഭദ്രേ!\\
നിന്നോളമാര്‍ക്കുമില്ല ഭഗബദ്ഭക്തി നാഥേ!\\
ശ്രീരാമദേവതത്ത്വം കേള്‍ക്കേണമെന്നു മന-\\
താരിലാകാംക്ഷയുണ്ടായ്വന്നതു മഹാഭാഗ്യം\\
മുന്നമെന്നോടിതാരും ചോദ്യം ചെയ്തീല, ഞാനും\\
നിന്നാണെ കേള്‍പ്പിച്ചതില്ലാരെയും ജീവനാഥേ!\\
അത്യന്തം രഹസ്യമായുള്ളൊരു പരമാത്മ-\\
തത്ത്വാര്‍ഥമറികയിലാഗ്രഹമുണ്ടായതും\\
ഭക്ത്യതിശയം പുരുഷോത്തമന്‍തങ്കലേറ്റം\\
നിത്യവും ചിത്തകാമ്പില്‍ വര്‍ധിക്കതന്നെ മൂലം.\\
ശ്രീരാമപാദാംബുജം വന്ദിച്ചു സംക്ഷേപിച്ചു\\
സാരമായുള്ള തത്ത്വം ചൊല്ലുവാന്‍ കേട്ടാലും നീ.\\
ശ്രീരാമന്‍ പരമാത്മാ പരമാനന്ദമൂര്‍ത്തി\\
പുരുഷന്‍ പ്രകൃതിതന്‍ കാരണനേകന്‍ പരന്‍\\
പുരുഷോത്തമന്‍ ദേവ്നനന്തനാദിനാഥന്‍\\
ഗുരുകാരുണ്യമൂര്‍ത്തി പരമന്‍ പരബ്രഹ്മം\\
ജഗദുത്ഭവസ്ഥിതിപ്രളയകര്‍ത്താവായ\\
ഭഗവാന്‍ വിരിഞ്ചനാരായണശിവാത്മകന്‍\\
അദ്വയനാദ്യനജനവ്യയനാത്മാരാമന്‍\\
തത്ത്വാത്മാ സച്ചിന്മയന്‍ സകളാത്മകനീശന്‍\\
മാനുഷനെന്നു കല്പിച്ചീടുവോരജ്ഞാനികള്‍\\
മാനസം മായാതമസ്സംവൃതമാകമൂലം.\\
സീതാരാഘവമരുല്‍സൂനുസംവാദം മോക്ഷ-\\
സാധനം ചൊല്‍വന്‍ നാഥേ! കേട്ടാലും തെളിഞ്ഞു നീ.\\
എങ്കിലോ മുന്നം ജഗന്നായകന്‍ രാമദേവന്‍\\
പങ്കജവിലോചനന്‍ പരമാനന്ദമൂര്‍ത്തി\\
ദേവഗണ്ടകനായ പംക്തികണ്ഠനെക്കൊന്നു\\
ദേവിയുമനുജനും വാനരപ്പടയുമായ്\\
സത്വരമയോധ്യപുക്കഭിഷേകവും ചെയ്തു\\
സത്താമാത്രാത്മാ സകലേശനവ്യയന്‍ നാഥന്‍\\
മിത്രമുത്രാദികളാം മിത്രവര്‍ഗത്താലുമ-\\
ത്യുത്തമന്മാരാം സഹോദരവീരന്മ്ലാരാലും\\
കീകസാത്മജാസുതതാം വിഭീഷണനാലും\\
ലോകേശാത്മജനായ വസിഷ്ഠാദികളാലും\\
സേവ്യനായ് സൂര്യകോടിതുല്യതേജസാ ജഗ-\\
ച്ഛ്രാവ്യമാം ചതിതവും കേട്ടുകേട്ടാനന്ദിച്ചു\\
നിര്‍മലമണിലസല്‍കാഞ്ചനസിംഹാസനേ\\
തന്മായാദേവിയായ ജാനകിയോടും കൂടി\\
സാനന്ദമിരുന്നരുളീടുന്നനേരം പര-\\
മാനന്ദമൂര്‍ത്തി തിരുമുമ്പിലാമ്മാറുഭക്ത്യാ\\
വന്ദിച്ചു നില്‍ക്കുന്നൊരു ഭക്തനാം ജഗല്‍പ്രാണ-\\
നന്ദനന്‍തന്നെത്തൃക്കണ്‍പാര്‍ത്തു കാരുണ്യമൂര്‍ത്തി\\
മന്ദഹാസവും പൂണ്ടു സീതയോടരുള്‍ചെയ്തു.\\
“സുന്ദരരൂപേ! ഹനുമാനെ നീ കണ്ടായല്ലീ?\\
നിന്നിലുമെന്നിലുമുണ്ടെല്ലാനേരവുമിവന്‍-\\
തന്നുള്ളീലഭേദയായുള്ളോരു ഭക്തി നാഥേ!\\
ധന്യേ! സന്തതം പരമാത്മാജ്ഞാനത്തെയൊഴി-\\
ച്ചൊന്നൊലുമൊരുനേരമാശയുമില്ലയല്ലോ.\\
നിര്‍മ്മലനാത്മജ്ഞാനത്തിന്നിവന്‍ പാത്രമത്രേ\\
നിര്‍മ്മമന്‍ നിത്യബ്രഹ്മചാരികള്‍മുമ്പനല്ലോ.\\
കല്മഷമിവനേതുമില്ലെന്നു ധരിച്ചാലും\\
തന്മനോരഥത്തെ നീ നല്കണം മടിയാതെ\\
നമ്മുടെ തത്ത്വമിവന്നറിയിക്കേണമിപ്പോള്‍\\
ചിന്മയേ! ജഗന്മയേ! സന്മയേ! മായാമയേ!\\
ബ്രഹ്മോപദേശത്തിനു ദുര്‍ലഭം പാത്രമിവന്‍\\
ബ്രഹ്മജ്ഞാനാര്‍തികളിമുത്തമോത്തമനെടോ!\\
ശ്രീരാമദേവനേവമരുളിച്ചെയ്ത നേരം\\
മാരുതിതന്നെ വിളിച്ചരുളിച്ചെയ്തു ദേവി:\\
“വീനന്മാര്‍ ചൂടും മകുടത്തിന്‍ നായകക്കല്ലേ!\\
ശ്രീദ്രാമപാദഭക്തപ്രവര! കേട്ടാലും നീ\\
സച്ചിദാനന്ദമേകമദ്വയം പരബ്രഹ്മം\\
നിശ്ചലം സര്‍വോപാധിനിര്‍മുക്തം സത്താമാത്രം\\
നിശ്ചയിച്ചറിഞ്ഞുകൂടാതൊരു വസ്തുവെന്നു\\
നിശ്ചയിച്ചാലുമുള്ളില്‍ ശ്രീരാമദേവനെ നീ.\\
നിര്‍മലം നിരഞ്ജനം നിര്‍ഗുണം നിര്‍വികാരം\\
സന്മയം ശാന്തം പരമാത്മാനം സദാനന്ദം\\
ജ്ന്മനാശാദിഗളില്ലാതൊരു വസ്തു പര-\\
ബ്രഹ്മമീ ശ്രീരാമനെന്നറിഞ്ഞുകൊണ്ടാലും നീ.\\
സര്‍വകാരണം സര്‍വവ്യാപിനം സര്‍വാത്മാനം\\
സര്‍വജ്ഞം സര്‍വേശ്വരം സര്‍വസാക്ഷിണം നിത്യം\\
സര്‍വദം സര്‍വാധാരം സര്‍വദേവതാമയം\\
നിര്‍വികാരാത്മാ രാമദേവനെന്നറിഞ്ഞാലും.\\
എന്നുടെ തത്ത്വമിനിച്ചൊല്ലിടാമുള്ളവണ്ണം\\
നിന്നോടു, ഞാന്‍താന്‍ മൂലപ്രകൃതിയായതെടോ\\
എന്നുടെ പതിയായ പരമാത്മാവുതന്റെ\\
സന്നിധിമാത്രംകൊണ്ടു ഞാനിവ സൃഷ്ടിക്കുന്നു.\\
തത്സാന്നിധ്യംകൊണ്ടെന്നാല്‍ സൃഷ്ടമാമവയെല്ലാം\\
തത്സ്വരൂപത്തിങ്കലാക്കീടുന്നു മൂഢജനം\\
തത്സ്വരൂപത്തിനുണ്ടോ ജനനാദികളെന്നു\\
തത്സ്വരൂപത്തെയറുഞ്ഞവനേയറിയാവൂ.\\
ഭൂമിയില്‍ ദിനകരവംശത്തിലയോധ്യയില്‍\\
രാമനായ് സര്‍വേശ്വരന്‍താന്‍ വന്നു പിറ്ന്നതും\\
ആമിഷഭോജികളെ വധിപ്പാനായ്ക്കൊണ്ടു വി-\\
ശ്വാമിത്രനോടും കൂടെയെഴുന്നള്ളിയ കാലം\\
ക്രുദ്ധയായടുത്തോരു ദുഷ്ടയാം താടകയെ\\
പദ്ധതിമധ്യേ കൊന്നു സത്വരം സിദ്ധാശ്രമം\\
ബദ്ധമോദേന പുക്കു യാഗരക്ഷയും ചെയ്തു\\
സിദ്ധസങ്കല്പനായ കൗശികമുനിയോടും\\
മൈഥിലരാജ്യത്തിനായ്ക്കൊണ്ടു പോകുന്ന നേരം\\
ഗൗതമപത്നിയായോരഹല്യാശാപം തീര്‍ത്തു.\\
പാദപങ്കജം തൊഴുതവളെയനുഗ്രഹി-\\
ച്ചാദരപൂര്‍വം മിഥിലാപുരമകംപുക്കു\\
മുപ്പുരവൈരിയുടെ ചാപവും മുറിച്ചുടന്‍\\
മല്‍പാണിഗ്രഹണമുംചെയ്തു പോരുന്നനേരം\\
മുല്‍പുക്കു തടുത്തൊരു ഭാര്‍ഗവരാമന്‍തന്റെ\\
ദര്‍പ്പവുമടക്കി വന്‍പോടയോധ്യയും പുക്കു\\
ദ്വാദശസംവത്സരമിരുത്തു സുഖത്തോടെ്\\
താതനുമഭിഷേകത്തിന്നാരംഭിച്ചാനതു\\
മാതാവു കൈകേകിയും മുടക്കിയതുമൂലം\\
ഭ്രാതാവാകിയ സുമിത്രാത്മജനോടും കൂടെ\\
ചിത്രകൂടം പ്രാപിച്ചു വസിച്ചകാലം താതന്‍\\
വൃത്രാരിപുരം പുക്ക വൃത്താന്തം കേട്ടശേഷം\\
ചിന്താശോകത്തോടുദകക്രിയാദികള്‍ ചെയ്തു\\
ഭക്തനാം ഭരതനെയയച്ചു രാജ്യത്തിനായ്\\
ദണ്ഡകാരണ്യം പുക്കകാലത്തു വിരാധനെ\\
ഖണ്ഡിച്ചു കുംഭോത്മഭവനാമഗസ്ത്യനെക്കണ്ടു\\
പണ്ഡിതന്മാരാം മുനിമാരോടു സത്യംചെയ്തു\\
ദണ്ഡമെന്നിയേ രക്ഷോവംള്ശത്തെയൊടുക്കുവാന്‍\\
പുക്കിതു പഞ്ചവടി തത്ര വാണീടും കാലം\\
പുഷ്കരശരപരവശയായ് വന്നാളല്ലോ\\
രക്ഷോനായകനുടെ സോദരി ശൂര്‍പ്പണഖാ;\\
ലക്ഷ്മണനവളുടെ നാസികാച്ഛേദം ചെയ്തു.\\
ഉന്നതനായ ഖരന്‍ കോപിച്ചു യുദ്ധത്തിനായ്\\
വന്നിതു പതിന്നാലു സഹസ്രം പടയോടും\\
കൊന്നിതു മൂന്നേമുക്കാല്‍നാഴികകൊണ്ടുതന്നെ;\\
പിന്നെശ്ശൂര്‍പ്പണഖ് പോയ് രാവണനോടു ചൊന്നാള്‍\\
മായയാ പൊന്മാനായി വന്ന മാരീചന്‍തന്നെ-\\
സ്സായകം പ്രയോഗിച്ചു സല്‍ഗതി കൊടുത്തപ്പോള്‍\\
മായാസീതയെക്കൊണ്ടു രാവണന്‍ പോയശേഷം\\
മായാമാനുഷന്‍ ജടായുസ്സിനു മോക്ഷം നല്‍ഗി\\
രാക്ഷസവേഷം പൂണ്ട കബന്ധന്‍തന്നെക്കൊന്നു\\
മോക്ഷവും കൊടുത്തു പോയ് ശബരിതന്നെക്കണ്ടു\\
മോക്ഷദനവളുടെ പൂജയും കൈക്കൊണ്ടഥ\\
മോക്ഷദാവനും ചെയ്തു പുക്കിതു പമ്പാതീരം.\\
തത്ര കണ്ടിതു നിന്നെപ്പിന്നെ നിന്നോടുംകൂടി\\ 
പിത്രനന്ദനനായ സുഗ്രീവന്‍തന്നെ ക്കണ്ടു\\
മിത്രമായിരിപ്പൂതെന്നന്യോന്യം സംഖ്യം ചെയ്തു\\
വൃത്രാരിപുത്രനായ ബാലിയെ വധംചെയ്തു.\\
സീതാന്വേഷണം ചെയ്തു ദക്ഷിണജലധിയില്‍\\
സേതുബന്ധനം ലങ്കാവര്‍ദ്ദനം പിന്നെശ്ശേഷം\\
പുത്രമിത്രാമാത്യഭൃത്യാദികളോടും കൂടി\\
യുദ്ധസന്നദ്ധനായ ശത്രുവാം ദശാസ്യനെ\\
ശസ്ത്രേണ വധംചെയ്തു രക്ഷിച്ചു ലോകത്രയം\\
ഭക്തനാം വിഭീഷണന്നഭിഷേകവും ചെയ്തു.\\
പാവകന്‍തങ്കല്‍ മറഞ്ഞിരുന്നോരെന്നെപ്പിന്നെ\\
പാവനയെന്നു ലോകസമ്മതമാക്കിക്കൊണ്ടു\\
പാവകനോടു വാങ്ങി പുഷ്പകം കരയേറി\\
ദേവകളോടുമനുവാദം കൊണ്ടയോദ്ധ്യയാം\\
രാജ്യത്തിന്നഭിഷേകം ചെയ്തു ദേവാദികളാല്‍\\
പൂജ്യനായിരുന്നരുളീടിനാന്‍ ജഗന്നാഥന്‍.\\
യാജ്യനാം നാരായണന്‍ ഭക്തിയുള്ളവര്‍ക്കു സാ-\\
യുജ്യമാം മോക്ഷത്തെ നല്കീടിനാന്‍ നിരഞ്ജനന്‍.\\
ഏവമാദിഗളായ കര്‍മങ്ങള്‍ തന്റെ മായാ-\\
ദേവിയാമെന്നെക്കൊണ്ടു ചെയ്യിപ്പിക്കുന്നു നൂനം\\
രാമനാം ജഗല്‍ഗുരു നിര്‍ഗുണന്‍ ജഗദഭി-\\
രാമനവ്യയനേകനാനന്ദാത്മകനാത്മാ-\\
രാമനദ്വയന്‍ പരന്‍ നിഷ്കളന്‍ വിദ്വദ്ഭൃംഗാ-\\
രാമനച്യുതന്‍ വിഷ്ണുഭഗവാന്‍ നാരായണന്‍\\
ഗമിക്കെന്നതുംപുനരിരിക്കെന്നതും കിഞ്ചില്‍\\
ഭ്രമിക്കെന്നതും തഥാ ദുഃഖിക്കെന്നതുമില്ല.\\
നിര്‍വികാരാത്മാ തേജോമയനായ് നിറഞ്ഞൊരു\\
നിര്‍വൃതനൊരുവസ്തു ചെയ്കയില്ലൊരു നാളും\\
നിര്‍മലന്‍ പരിണാമഹീനനാനന്ദ മൂര്‍ത്തി\\
ചിന്മയന്‍ മായാമയന്‍ തന്നുടെ മായാദേവി\\
കര്‍മങ്ങള്‍ ചെയ്യുന്നതു താനെന്നു തോന്നിക്കുന്നു\\
തന്മായാഗുണങ്ങളെത്താനനുസരിക്കയാല്‍.”\\
അഞ്ജനാതനയനോടിങ്ങനെ സീതാദേവി\\
കഞ്ജലോചനതത്ത്വമുപദേശിച്ച ശേഷം\\
അഞ്ജസാ രാമദേവന്‍ മന്ദഹാസവും ചെയ്തു\\
മഞ്ജുളവാചാ പുനരവനോടുരചെയ്തു.\\
“പരമാത്മാവാകുന്ന ബിംബത്തിന്‍ പ്രതിബിംബം\\
പരിചില്‍ കാണുന്നതു ജീവാത്മാവറികെടോ!\\
തേജോരൂപിണിയാകുമെന്നുടെ മായതങ്കല്‍\\
വ്യാജമെന്നിയേ നിഴലിക്കുന്നു കപിവര!\\
ഓരോരോ ജലാശയേ കേവലം മഹാകാശം\\
നേരേ നീ കാണ്മീലയോ, കണ്ടാലുമതുപോലെ\\
സാക്ഷാലുള്ളൊരു പരബ്രഹ്മമാം പരമാത്മാ\\
സാക്ഷിയായുള്ള ബിംബം നിശ്ചലമതു സഖേ!\\
തത്ത്വമസ്യാദി മഹാവാക്യാര്‍ഥംകൊണ്ടു മമ\\
തത്ത്വത്തെയറിഞ്ഞിടാമാചാര്യകാരുണ്യത്താല്‍\\
മദ്ഭക്തനായുള്ളവനിപ്പദമറിയുമ്പോള്‍\\
മദ്ഭാവം പ്രാപിച്ചീടുമില്ല സംശയമേതും.\\
മത്ഭാക്തിവിമുഖന്മാര്‍ ശാസ്ത്രഗര്‍ത്തങ്ങള്‍തോറും\\
സത്ഭാവംകൊണ്ടു ചാടിവീണു മോഹിച്ചീടുന്നു\\
ഭക്തിഹീനന്മാര്‍ക്കു നൂറായിരം ജന്മംകൊണ്ടും\\
സിദ്ധിക്കയില്ല തത്ത്വജ്ഞാനവും കൈവല്യവും\\
പരമാത്മാവാം മമ ഹൃദയം രഹസ്യമി-\\
തൊരുനാളും മത്ഭക്തിഹീനന്മാരായ്മേവീടും\\
നരന്മാരോടു പറഞ്ഞറിയിക്കരുതെടോ!\\
പരമമുപദേശമില്ലിതിന്‍മീതെയൊന്നും.”\\
ശ്രീമഹാദേവന്‍ മഹാദേവിയോടരുള്‍ ചെയ്തു:\\
രാമമാഹാത്മ്യമിദം പവിത്രം ഗുഹ്യതമം\\
സാക്ഷാല്‍ ശ്രീരാമപ്രോക്തം വായുപുത്രനായ്ക്കൊണ്ടു\\
മോക്ഷദം പാപഹരം ഹൃദ്യമാനന്ദോദയം\\
സര്‍വവേദാന്തസാരസംഗ്രഹം രാമതത്ത്വം\\
ദിവ്യനാം ഹനൂമാനോടുപദേശിച്ചതെല്ലാം\\
ഭാക്തിപൂണനാരതം പഠിച്ചീടുന്ന പുമാന്‍\\
മുക്തനായ് വരുമൊരു സംശയമില്ല നാഥേ!\\
ബ്രഹ്മഹത്യാദിദുരിതങ്ങളും ബഹുവിധം\\
ജന്മങ്ങള്‍ തോറുമാര്‍ജിച്ചുള്ളവയെന്നാകിലും\\
ഒക്കവേ നശിച്ചുപോനെന്നരുള്‍ചെയ്തു രാമന്‍\\
മര്‍ക്കടപ്രവരനോടെന്നതു സത്യമല്ലോ.\\
ജാതിനിന്ദിതന്‍, പരസ്ത്രീധനഹാരി, പാപി\\
മാതൃഘാതകന്‍, പിതൃഘാതകന്‍, ബ്രഹ്മഹന്താ,\\
യോഗിവൃന്ദാപകാരി, സുവര്‍ണസ്തേയി, ദുഷ്ടന്‍\\
ലോകനിന്ദിതനേറ്റമെങ്കിലുമവന്‍ ഭക്ത്യാ\\
രാമനാമത്തെജ്ജപിച്ചീടുകില്‍ ദേവകളാ-\\
ലാമോദപൂര്‍വം പൂജ്യുനായ് വരുമത്രയല്ല\\
യോഗീന്ദ്രന്മാരാല്‍പ്പോലുമലഭ്യമായ വിഷ്ണു-\\
ലോകത്തെ പ്രാപിച്ചീടുമില്ല സംശയമേതും.”\\
ഇങ്ങനെ മഹാദേവനരുള്‍ചെയ്തതു കേട്ടു\\
തിങ്ങീടും ഭക്തിപൂര്‍വമരുള്‍ചെയ്തിതു ദേവി:\\
“മംഗലാത്മാവേ! മമ ഭര്‍ത്താവേ! ജഗല്‍പതേ!\\
ഗംഗാകാമുക! പരമേശ്വര! ദയാനിധേ!\\
പന്നഗവിഭൂഷണ! ഞാനനുഗൃഹീതയായ്\\
ധന്യയായ് കൃതാര്‍ഥയായ് സ്വസ്ഥയായ് വന്നേനല്ലോ.\\
ഛിന്നമായ് വന്നു മമമ് സന്ദേഹമെല്ലാമിപ്പോള്‍\\
സന്നാമായിതു മോഹമൊക്കെ നിന്നനുഗ്രഹാല്‍.\\
നിര്‍മലം രാമതത്ത്വാമൃതമാം രസായനം\\
ത്വന്മുഖോദ്ഗളിതമാവോളം പാനം ചെയ്താലും\\
എന്നുള്ളില്‍ തൃപ്തിവരികെന്നുള്ളതില്ലയല്ലോ\\
നിര്‍ണയമതുമൂലമൊന്നുണ്ടു ചൊല്ലുന്നു ഞ്ഞാന്‍\\
സംക്ഷേപിച്ചരുള്‍ചെയ്തതേതുമേ മതിയല്ല\\
സാക്ഷാല്‍ ശ്രീ നാരായണന്‍തന്മാഹാത്മങ്ങളെല്ലാം.\\
കിംക്ഷണന്മാര്‍ക്കു വിദ്യയുണ്ടാകയില്ലയല്ലോ\\
കിംകണന്മാരായുള്ളോര്‍ക്കര്‍ഥവുമുണ്ടായ്വരാ\\
കിമൃണന്മാര്‍ക്കു നിത്യസൗഖ്യവുമുണ്ടായ്വരാ\\
കിംദേവന്മാര്‍ക്കു ഗതിയും പുനരതുപോലെ.\\
ഉത്തമമായ രാമചരിതം മനോഹരം\\
വിസ്തരിച്ചരുളിച്ചെയ്തീടണം മടിയാതെ.”\\
ഈശ്വരന്‍ ദേവന്‍ പരമേശ്വരന്‍ മഹേശ്വര-\\
നീശ്വരിയുടെ ചോദ്യമിങ്ങനെ കേട്ടനേരം\\
മന്ദഹാസവും ചെയ്തു ചന്ദ്രശേഖരന്‍ പരന്‍\\
സുന്ദരഗാത്രീ! കേട്ടുകൊള്ളുകെന്നരുള്‍ചെയ്തു.\\
വേധാവു ശതകോടി ഗ്രന്ഥവിസ്തരം പുരാ-\\
വേദസമ്മിതമരുള്‍ചെയ്തിതു രാമായണം\\
വാല്മീകി പുനരിരുപത്തിനാലായിരമായി\\
നാന്മുഖന്‍നിയൊഗത്താല്‍ മാനുഷമുക്ത്യര്‍ഥമായ്\\
ചമച്ചാനതിലിതു ചുരുക്കി രാമദേവന്‍\\
നമുക്കുപദേശിച്ചീടിനാനേവം പുരാ.\\
അധ്യാത്മരാമായണമെന്ന പേരിതിന്നിദ-\\
മധ്യയനം ചെയ്യുന്നോര്‍ക്കധ്യാത്മജ്ഞാനമുണ്ടാം.\\
പുത്രസന്തതി ധനസമൃദ്ധി ദീര്‍ഘായുസ്സും\\
മിത്രസമ്പത്തി കീര്‍ത്തി രോഗശാന്തിയുമുണ്ടാം.\\
ഭക്തിയും വര്‍ധിച്ചീടും മുക്തിയും സിദ്ധിച്ചീടു-\\
മെത്രയും രഹസ്യമിതെങ്കിലോ കേട്ടാലും നീ.
\end{verse}

%%unit 2). shivan katha aakyaanam cheyyunnu
\section{ശിവന്‍ കഥ ആഖ്യാനം ചെയ്യുന്നു}

\begin{verse}
പങ് ക്തികന്ധരമുഖരാക്ഷസവീരന്മാരാല്‍\\
സന്തദം ഭാരണ സന്തപ്തയാം ഭൂമിദേവി\\
ഗോരൂപം പൂണ്ടു ദേവതാപസഗണത്തോടും\\
സാരസാസനലോകം പ്രാപിച്ചു കരഞ്ഞേറ്റം\\
വേദനയെല്ലാം വിധാതാവിനോടറിയിച്ചാള്‍\\
വേധാവും മുഹൂര്‍ത്തമാത്രം വിചാരിച്ചശേഷം\\
‘വേദനായകനായ നാഥനോടിവ ചെന്നു\\
വേദനം ചെയ്കയെന്യേ മറ്റൊരു കഴിവില്ല.’\\
സാരസോദ്ഭവനേവം ചിന്തിച്ചു ദേവന്മാരോ-\\
ടാരൂഢഖേദം തമ്മെക്കൂട്ടിക്കൊണ്ടങ്ങുപോയി,\\
ക്ഷീരസാഗരതീരം പ്രപിച്ചു ദേവമുനി-\\
മാരോടുകൂടി സ്തുതിച്ചീടിനാന്‍ ഭക്തിയോടെ.\\
ഭാവനയോടുംകൂടി പുരുഷസൂക്തംകൊണ്ടു\\
ദേവനെസ്സേവിച്ചിരുന്നീടിനാന്‍ വഴിപോലെ.\\
അന്നെരമൊരു പതിനായിരമാദിത്യന്മാ-\\
രൊന്നിച്ചു കിഴക്കുദിച്ചുയരുന്നതുപോലെ\\
പദ്മസംഭവന്‍തനിക്കന്‍പോടു കാണായ്വന്നു\\
പദ്മലോചനനായ പദ്മനാഭനെ മോദാല്‍.\\
മുഗ്ദ്ധന്മാരായുള്ളൊരു സിദ്ധയോഗികളാലും\\
ദുര്‍ദ്ദര്‍ശമായ ഭഗവദ്രൂപം മനോഹരം\\
ചന്ദ്രികാമന്ദസ്മിതസുന്ദരാനനപൂര്‍ണ-\\
ചന്ദ്രമണ്ഡലമരവിന്ദലോചനം ദേവം\\
ഇന്ദ്രനീലാഭം പരമിന്ദിരാമനോഹര-\\
മന്ദിരവക്ഷഃസ്ഥലം വന്ദ്യമാനന്ദോദയം\\
വത്സലാഞ്ഛനവത്സം പാദപങ്കജഭക്ത-\\
വത്സലം സമസ്തലോകോത്സവം സത്സേവിതം\\
മേരുസന്നിഭകിരീടോദ്യല്‍കുണ്ഡലമുക്താ-\\
ഹാരകേയൂരാംഗദ കടക കടിസൂത്ര-\\
വലയാംഗുലീയകാദ്യഖിലവിഭൂഷണ-\\
കലിത കളേബരം കമലാമനോഹരം\\
കരുണാകരം കണ്ടു പരമാനന്ദം പൂണ്ടു\\
സരസീരുഹഭവന്‍ മധുരസ്ഫുടാക്ഷരം\\
സരസപദങ്ങളാല്‍ സ്തുതിച്ചുതുടങ്ങിനാന്‍:\\
“പരമാനന്ദമൂര്‍ത്തേ! ഭഗവന്‍! ജയജയ!\\
മോക്ഷകാമികളായ സിദ്ധയോഗീന്ദ്രന്മാര്‍ക്കും\\
സാക്ഷാല്‍ കാണ്മതിന്നരുതാതൊരു പാദാംബുംജം\\
നിത്യവും നമോസ്തു തേ സകലജഗല്‍പതേ!\\
നിത്യനിര്‍മലമൂര്‍ത്തേ! നിത്യവും നമോസ്തുതേ.\\
സത്യജ്ഞാനാനന്താനന്ദമൃതാദ്വയമേകം\\
നിത്യവും നമോസ്തുതേ കരുണാജലനിധേ\\
വിശ്വത്തെ സൃഷ്ടിച്ചു രക്ഷിച്ചു സംഹരിച്ചീടും\\
വിശ്വനായക! പോറ്റീ! നിത്യവും നമോസ്തുതേ.\\
സ്വാദ്ധ്യായതപോദാനയജ്ഞാദികര്‍മങ്ങളാല്‍\\
സാധ്യമല്ലൊരുവനും കൈവല്യമൊരുനാളും\\
മുക്തിയെസ്സിദ്ധിക്കേണമെങ്കിലോ ഭവല്‍പാദ-\\
ഭക്തികൊണ്ടൊഴിഞ്ഞു മറ്റൊന്നിനുമാവതില്ല.\\
നിന്തിരുവടിയുടെ ശ്രീ പാദാംബുജദ്വന്ദ്വ-\\
മന്തികേ കാണായ്വന്നിതെനിക്കു ഭാഗ്യവശാല്‍.\\
സത്വചിത്തന്മാരായ താപസശ്രേഷ്ര്ഠന്മാരാല്‍\\
നിത്യവും ഭക്ത്യാ ബുദ്ധ്യാ ധരിക്കപ്പെട്ടൊരു നിന്‍\\
പാദപങ്കജങ്ങളില്‍ ഭക്തി സംഭവിക്കണം\\
ചേതസി സദാകാലം ഭക്തവത്സല! പോറ്റീ!\\
സംസാരാമയപരിതപ്തമാനസന്മാരാം\\
പുംസാം ത്വദ്ഭക്തിയൊഴിഞ്ഞില്ല ഭേഷജമേതും\\
മരണമോര്‍ത്തു മമ മനസി പരിതാപം\\
കരുണാമൃതനിധേ! പെരികെ വളരുന്നു.\\
മരണകാലേ തവ തരുണാരുണസമ-\\
ചരണസരസിജസ്മരണമുണ്ടാവാനായ്\\
തരിക വരം നാഥ! കരുണാകര! പോറ്റീ!\\
ശരണം ദേവ! രമാരമണ! ധരാപതേ!\\
പരമാനന്ദമൂര്‍ത്തേ! ഭഗവന്‍ ജയ ജയ!\\
പരമ! പരമാത്മന്‍! പരബ്രഹ്മാഖ്യ ജയ!\\
പരചിന്മയ! പരാപര! പത്മാക്ഷ! ജയ!\\
വരദ! നാരായണ! വൈകുണ്ഠ! ജയ ജയ!\\
ചതുരാനനനിതി സ്തുതിചെയ്തോരുനേരം\\
മധുരതരമതിവിശദസ്മിതപൂര്‍വം\\
അരുളിച്ചെയ്തു നാഥ“നെന്തിപ്പോളെല്ലാവരു-\\
മൊരുമിച്ചെന്നെക്കാണ്മാനിവിടേക്കുഴറ്റോടെ\\
വരുവാന്‍ മൂല, മതു ചൊല്ലുകെ“ന്നതു കേട്ടു\\
സരസീരുഹഭവനീവണ്ണമുണര്‍ത്തിച്ചു:\\
“നിന്തിരുവടിതിരുവുള്ളത്തിലേറാതെ ക-\\
ണ്ടെന്തൊരു വസ്തു ലോകത്തിങ്കലുള്ളതു പോറ്റീ!\\
എങ്കിലുമുണര്‍ത്തിക്കാം മൂന്നു ലോകത്തിങ്കലും\\
സംങ്കടം മുഴുത്തിരിക്കുന്നിതിക്കാലം നാഥ!\\
പൗലസ്ത്യതനയനാം രാവണന്‍തന്നാലിപ്പോള്‍\\
ത്രൈലോക്യം നശിച്ചിതു മിക്കതും ജഗല്‍പതേ!\\
മദ്ദത്തവരബലദര്‍പ്പിതനായിട്ടതി-\\
നിര്‍ദ്ദയം മുടിക്കുന്നു വിശ്വത്തെയെല്ലാമയ്യോ!\\
ലോകപാലന്മാരെയും തച്ചാട്ടിക്കളഞ്ഞവ-\\
നേകശാസനമാക്കിച്ചമച്ചു ലോകമെല്ലാം\\
പാകശാസനനേയും സമരേ കെട്ടിക്കൊണ്ടു\\
നാകശാസനവും ചെയ്തീടിനാന്‍ ദശാനനന്‍\\
യാഗാദികര്‍മങ്ങളും മുടങ്ങിയത്രയല്ല\\
യോഗീന്ദ്രന്മാരാം മുനിമാരെയും ഭക്ഷിക്കുന്നു\\
ധര്‍മപത്നികളെയും പിടിച്ചുകൊണ്ടുപോയാന്‍\\
ധര്‍മവും മറഞ്ഞിതു മുടിഞ്ഞു മര്യാദയും.\\
മര്‍ത്ത്യനാലൊഴിഞ്ഞവനില്ല മറ്റാരാലുമേ\\
മൃത്യുവെന്നതും മുന്നേ കല്പിതം ജഗല്‍പതേ!\\
നിന്തിരുവടി തന്നെ മര്‍ത്ത്യനായ് പിറന്നിനി\\
പങ് ക്തികന്ധരന്‍തന്നെക്കൊല്ലണം ദയാനിധേ!\\
സന്തതം നമസ്കാരമതിനു മധുരിപോ!\\
ചെന്തളിരടിയിണ ചിന്തിക്കായ് വരേണമേ!”\\
പത്മസംഭവനിത്ഥമുണര്‍ത്തിച്ചതുനേരം\\
പത്മലോചനന്‍ ചിരിച്ചരുളിച്ചെയ്താനേവം:\\
“ചിത്തശുദ്ധിയോടെന്നെസ്സേവിച്ചു ചിരകാലം\\
പുത്രലാഭാര്‍ഥം പുരാ കശ്യപപ്രജാപതി\\
ദത്തമായിതു വരം സുപ്രസന്നേന മയാ\\
തദ്വചസ്സത്യം കര്‍ത്തുമുദ്യോഗമദ്യൈവ മേ.”\\
കശ്യപന്‍ ദശരഥനാമ്നാ രാജന്യേന്ദ്രനായ്\\
കാശ്യപീതലേ തിഷ്ഠത്യധുനാ വിധാതാവേ!\\
തസ്യ വല്ലഭയാകുമദിതി കൗസല്യയും\\
തസ്യാമാത്മജനായി വന്നു ഞാന്‍ ജനിച്ചീടും.\\
മത്സഹോദരന്മാരായ് മൂന്നുപേരുണ്ടായ്വരും\\
ചിത്സ്വരൂപിണി മമ ശക്തിയാം വിശ്വേശ്വരി\\
യോഗമായാദേവിയും ജനകാലയേ വന്നു\\
കീകസാത്മജകുലനാശികാരിണിയായി\\
മേദിനിതന്നിലയോനിജയായുണ്ടായ്വരു-\\
മാദിതേയന്മാര്‍ കപിവീരരായ് പിറക്കേണം\\
മേദിനീദേവിക്കതിഭാരംകൊണ്ടുണ്ടായൊരു\\
വേദന തീര്‍പ്പനെന്നാ“ലെന്നരുള്‍ചെയ്തു നാഥന്‍\\
വേദനായകനെയുമയച്ചു മറഞ്ഞപ്പോള്‍\\
വേധാവും നമസ്കരിച്ചീടിനാന്‍ ഭക്തിയോടെ.\\
ആദിതേയന്മാരെല്ലാമാധിതീര്‍ന്നതുനേര-\\
മാദിനായകന്‍ മറഞ്ഞീടിനോരാശ നോക്കി\\
ഖേദവുമകന്നുള്ളില്‍ പ്രീതിപൂണ്ടുടനുടന്‍\\
മേദിനിതന്നില്‍ വീണു നമസ്കാരവും ചെയ്താര്‍.\\
മേദിനീദേവിയേയുമാശ്വസിപ്പിച്ചശേഷം\\
വേധാവും ദേവകളോടരുളിച്ചെയ്താനേവം:\\
“ദാനവാരാതി കരുണാനിധി ലക്ഷ്മീപതി\\
മാനവപ്രവരനായ് വന്നവതരിച്ചീടും\\
വാസരാധീശാന്വയേ സാദരമയോധ്യയില്‍\\
വാസവാദികളായ നിങ്ങളുമൊന്നു വേണം\\
വാസുദേവനെപ്പരിചരിച്ചുകൊള്‍വാനായി-\\
ദ്ദാസഭാവേന ഭൂമിമണ്ഡലേ പിറക്കേണം\\
മാനിയാം ദശാനനഭൃത്യന്മാരാകും യാതു-\\
ധാനവീരന്മാരോടു യുദ്ധം ചെയ്വതിന്നോരോ\\
കാനനഗിരിഗുഹാദ്വാരവൃക്ഷങ്ങള്‍തോറും\\
വാനരപ്രവരന്മാരായേതും വൈകീടാതെ.”\\
സുത്രാമാദികളോടു പത്മസംഭവന്‍ നിജ\\
ഭര്‍തൃശാസനമരുള്‍ചെയ്തുടന്‍ കൃതാര്‍ഥനായ്\\
സത്യലോകവും പുക്കു സത്വരം ധരിത്രിയു-\\
മസ്തസന്താപമതിസ്വസ്ഥയായ് മരുവിനാള്‍.\\
തല്‍കാലെ ഹരിപ്രമുഖന്മാരാം വിബുധന്മാ-\\
രൊക്കവേ ഹരിരൂപധാരികളായാരല്ലോ.\\
മാനുഷഹരി സഹായാര്‍ഥമായ് തതസ്തതോ\\
മാനുഷഹരിസമവേഗവിക്രമത്തോടെ\\
പര്‍വതവൃക്ഷോപലയോധികളായുന്നത-\\
പര്‍വത തുല്യശരീരന്മാരായനാരതം\\
ഈശ്വരം പ്രതീക്ഷ്യമാണന്മാരായ് പ്ലവഗവൃ-\\
ന്ദേശ്വരന്മാരും ഭുവി സുഖിച്ചു വാണാരല്ലോ.
\end{verse}

%%unit 3). puthralaabhaalochana
\section{പുത്രലാഭാലോചന}

\begin{verse}
അമിതഗുണവാനാം നൃപതി ദശരഥ-\\
നമലനയോധ്യാപതി ധര്‍മാത്മാ വീരന്‍\\
അമരകുലവരതുല്യനാം സത്യപരാ-\\
ക്രമനംഗജസമന്‍ കരുണാരത്നാകരന്‍\\
കൗസല്യാദേവിയോടും ഭര്‍തൃശുശ്രൂഷയ്ക്കേറ്റം\\
കൗശല്യമേറീടും കൈകേയിയും സുമിത്രയും\\
ഭാര്യമാരിവരോടും ചേര്‍ന്നു മന്ത്രികളുമായ്\\
കാര്യാകാര്യങ്ങള്‍ വിചാരിച്ചു ഭൂതലമെല്ലാം\\
പരിപാലിക്കുംകാലമനപത്യത്വംകൊണ്ടു\\
പരിതാപേന ഗുരുചരണാംബുജദ്വയം\\
വന്ദനം ചെയ്തു ചോദിച്ചീടിനാ“നെന്തു നല്ലൂ\\
നന്ദനന്മാരുണ്ടാവാനെന്നരുള്‍ചെയ്തീടണം.\\
പുത്രന്മാരില്ലായ്കയാലെനിക്കു രാജ്യാദി സ-\\
മ്പത്തുസര്‍വവും ദുഃഖപ്രദമെന്നറിഞ്ഞാലും.”\\
വരിഷ്ഠതപോധനന്‍ വസിഷ്ഠനതു കേട്ടു\\
ചിരിച്ചു ദശരഥനൃപനോടരുള്‍ചെയ്തു:\\
“നിനക്കു നാലു പുത്രന്മാരുണ്ടായ്വരുമതു\\
നിനച്ചു ഖേദിക്കേണ്ട മനസി നരമപതേ!\\
വൈകാതെ വരുത്തണമൃശ്യശൃംഗനെയിപ്പോള്‍\\
ചെയ്ക നീ ഗുണനിധേ! പുത്രകാമേഷ്ടികര്‍മം.“
\end{verse}

%unit 4). puthrakaameshti
\section{പുത്രകാമേഷ്ടി}

\begin{verse}
തന്നുടെ ഗുരുവായ വസിഷ്ഠനിയോഗത്താല്‍\\
മന്നവന്‍ വൈഭണ്ഡകന്‍തന്നെയും വരുത്തിനാന്‍.\\
ശാലയും പണിചെയ്തു സരയൂതീരത്തിങ്കല്‍\\
ഭൂലോകപതി യാഗം ദീക്ഷിച്ചാനതുകാലം.\\
അശ്വമേധാനന്തരം താപസന്മാരുമായി\\
വിശ്വനായകസമനാകിയ ദശരഥന്‍\\
വിശ്വനായകനവതാരം ചെയ്വതിനായി\\
വിശ്വാസഭക്തിയോടും പുത്രകാമേഷ്ടികര്‍മം\\
ഋശ്യശൃംഗനാല്‍ ചെയ്യപ്പെട്ടൊരാഹുതിയാലേ\\
വിശ്വദേവതാഗണം തൃപ്തമായദുനേരം\\
ഹേമപാത്രസ്ഥമായ പായസത്തോടും കൂടി\\
ഹോമകുണ്ഡത്തില്‍നിന്നു പൊങ്ങിനാന്‍ വഹ്നിദേവന്‍.\\
‘താവകം പുത്രീയമിപ്പായസം കൈക്കൊള്‍കനീ\\
ദേവനിര്‍മിത’മെന്നു പറഞ്ഞു പാവകനും്\\
ഭൂപതിപ്രവരനു കൊള്ടുത്തു മറഞ്ഞിതു\\
താപസാജ്ഞയാ പരിഗ്രഹിച്ചു നൃപതിയും.\\
ദക്ഷിണ്ചെയ്തു നമസ്കരിച്ചു ഭക്തിപൂര്‍വം\\
ദക്ഷനാം ദശരഥന്‍ തല്‍ക്ഷണം പ്രീതിയോടെ.\\
കൗസല്യാദേവിക്കര്‍ദ്ധം കൊടുത്തു നൃപവരന്‍\\
ശൈഥില്യാത്മനാ പാതി നല്കിനാന്‍ കൈകേയിക്കും\\
അന്നേരം സുമിത്രയ്ക്കു കൗസല്യാദേവിതാനും\\
തന്നുടെ പാതികൊടുത്തീടിനാള്‍ മടിയാതെ.\\
എന്നതു കണ്ടു പാതി കൊടുത്തു കൈകേയിയും\\
മന്നവനതു കണ്ടു സന്തോഷം പൂണ്ടാനേറ്റം.\\
തന്‍ പ്രജകള്‍ക്കു പരമാനന്ദം വരുമാറു\\
ഗര്‍ഭവും ധരിച്ചിതു മൂവരുമതുകാലം.\\
അപ്പോഴേ തുടങ്ങി ക്ഷോണീന്ദ്രനാം ദശരഥന്‍\\
വിപ്രേന്ദ്രന്മാരെയൊക്കെ വരുത്തിത്തുടങ്ങിനാന്‍.\\
ഗര്‍ഭരക്ഷാര്‍ഥം ജപഹോമാദികര്‍മങ്ങളു-\\
മുല്പലാക്ഷികള്‍ക്കനുവാസരം ക്രമത്താലെ.\\
ഗര്‍ഭചിഹ്നങ്ങളെല്ലാം വര്‍ധിച്ചുവരുന്തോറു-\\
മുള്‍പ്രേമം കൂടെക്കൂടെ വര്‍ധിച്ചു നൃപേന്ദ്രനും.\\
തല്‍പ്രണയിനിമാര്‍ക്കുള്ളാഭരണങ്ങള്‍ പോലെ\\
വിപ്രാദി പ്രജകള്‍ക്കും ഭൂമിക്കും ദേവകള്‍ക്കും\\
അല്പമായ് ചമഞ്ഞിതു സന്താപം ദിനന്തോറു-\\
മല്പഭാഷിണിമാര്‍ക്കും വര്‍ധിച്ചു തേജസ്സേറ്റം.\\
സീമന്തപുംസവനാദിക്രിയകളും ചെയ്തു\\
കാമാന്തം ദാനങ്ങളും ചെയ്തിതു നരവരന്‍.\\
ഗര്‍ഭവും പരിപൂര്‍ണമായ് ചമഞ്ഞതുകാല-\\
മര്‍ഭകന്മാരും നാല്‍വര്‍ പിറന്നാരുടനുടന്‍.\\
ഉച്ചത്തില്‍ പഞ്ചഗ്രഹം നില്‍ക്കുന്ന കാലത്തിങ്ക-\\
ലച്യുതനയോധ്യയില്‍ കൗസല്യാത്മജനായാന്‍.\\
നക്ഷത്രം പുനര്‍വസു നവമിയല്ലോ തിഥി\\
നക്ഷത്രാധിപനോടുകൂടവേ ബൃഹസ്പതി\\
കര്‍ക്കടകത്തിലത്യുച്ചസ്ഥിതനാതിട്ടല്ലോ\\
അര്‍ക്കനുമത്യുച്ചസ്ഥനുദയം കര്‍ക്കടകം.\\
അര്‍ക്കജന്‍ തുലാത്തിലും ഭാര്‍ഗവന്‍ മീനത്തിലും\\
വക്രനുമുച്ചസ്ഥനായ് മകരംരാശിതന്നില്‍\\
നില്ക്കുമ്പോളവതരിച്ചീടിനാന്‍ ജഗന്നാഥന്‍\\
ദിക്കുകളൊക്കെ പ്രസാദിച്ചിതു ദേവകളും.\\
പെറ്റിതു കൈകേയിയും പുഷ്യനക്ഷത്രം കൊണ്ടേ\\
പിറ്റേന്നാള്‍ സുമിത്രയും പെറ്റിതു പുത്രദ്വയം\\
ഭഗവാന്‍ പരമാത്മാ മുകുന്ദന്‍ നാരായണന്‍\\
ജഗദീശ്വരന്‍ ജന്മരഹിതന്‍ പദ്മേക്ഷണന്‍\\
ഭുവനേശ്വരന്‍ വിഷ്ണുതന്നുടെ ചിഹ്നത്തോടു-\\
മവതാരം ചെയ്തപ്പോള്‍ കാണായി കൗസല്യയ്ക്കും\\
സഹസ്രകിരണന്മാരൊരുമിച്ചൊരുനേരം\\
സഹസ്രായുതമുദിച്ചുയരുന്നതുപോലെ\\
സഹസ്രപത്രോത്ഭവനാരദസനകാദി\\
സഹസ്രനേത്രമുഖവിബുധേന്ദ്രന്മാരാലും\\
വന്ദ്യമായിരിപ്പൊരു നിര്‍മലമകുടവും\\
സുന്ദരചികുരവുമളകസുഷമയും\\
കാരുണ്യാമൃതരസസമ്പൂര്‍നനയനവു-\\
മാരുണ്യാംബര പരിശോഭിതജഘനവും\\
ശംഖചക്രാബ്ജഗദാശോഭിതഭുജങ്ങളും\\
ശംഖസന്നിഭഗളരാജിത കൗസ്തുഭവും\\
ഭക്തവാത്സല്യം ഭാക്തന്മാര്‍ക്കു കണ്ടറിവാനായ്\\
വ്യക്തമായിരിപ്പൊരു പാവനശ്രീവത്സവും\\
കുണ്ഡലമുക്താഹാരകാഞ്ചീനൂപുരമുഖ-\\
മണ്ഡനങ്ങളുമിന്ദുമണ്ഡലവദനവും\\
പണ്ടു ലോകങ്ങളെല്ലാമളന്ന പാദാബ്ജവും\\
കണ്ടുകണ്ടുണ്ടായൊരു പരമാനന്ദത്തൊടും\\
മോക്ഷദനായ ജഗല്‍സാക്ഷിയാം പരമാത്മാ\\
സാക്ഷാല്‍ ശ്രീനാരായണന്‍ താനിതെന്നറിഞ്ഞപ്പോള്‍\\
സുന്ദരഗാത്രിയായ കൗസല്യാദേവിതാനും\\
വന്ദിച്ചു തെരുതെരെ സ്തുതിച്ചുതുടങ്ങിനാള്‍:\\
“നമസ്തേ ദേവദേവ! ശംഖചക്രാബ്ജധര!\\
നമസ്തേ വാസുദേവ! മധുസൂദന! ഹരേ!\\
നമസ്തേ നാരായണ! നമസ്തേ നരകാരേ!\\
സമസ്തേശ്വര! ശൗരേ! നമസ്തേ ജഗല്‍പതേ!\\
നിന്തിരുവടി മായാദേവിയെക്കൊണ്ടു വിശ്വം\\
സന്തതം സൃഷ്ടിച്ചു രക്ഷിച്ചു സംഹരിക്കുന്നു.\\
സത്വാദിഗുണത്രയമാശ്രയിച്ചെന്തിന്നിതെ-\\
ന്നുത്തമന്മാര്‍ക്കുപോലുമറിവാന്‍ വേലയത്രേ.\\
പരമന്‍ പരാപരന്‍ പരബ്രഹ്മാഖ്യന്‍ പരന്‍\\
പരമാത്മാവു പരന്‍ പുരുഷന്‍ പരിപൂര്‍ണന്‍\\
അച്യുതനനന്തനവ്യക്തനവ്യയനേകന്‍\\
നിശ്ചലന്‍ നിരുപമന്‍ നിര്‍വാണപ്രദന്‍ നിത്യന്‍\\
നിര്‍മലന്‍ നിരാമയന്‍ നിര്‍വികാരാത്മാ ദേവന്‍\\
നിര്‍മമന്‍ നിരാകുലന്‍ നിരഹങ്കാരമൂര്‍ത്തി\\
നിഷ്കളന്‍ നിഗമാന്തവാക്യാര്‍ഥവേദ്യന്‍ നാഥന്‍\\
നിഷ്ക്രിയന്‍ നിരാകാരന്‍ നിര്‍ജരനിഷേവിതന്‍\\
നിഷ്കാമന്‍ നിയമിതാം ഹൃദയനിലയനന്‍\\
അദ്വയനജനമൃതാനന്ദന്‍ നാരായണന്‍\\
വിദ്വന്മാനസപദ്മമധുപന്‍ മധുവൈരി\\
സത്യജ്ഞാനാത്മാ സമസ്തേശ്വരന്‍ സനാതനന്‍\\
സത്വസഞ്ചയജീവന്‍ സനകാദിഭിസ്സേവ്യന്‍\\
തത്ത്വാര്‍ഥബോധരൂപന്‍ സകലജഗന്മയന്‍\\
സത്താമാത്രകനല്ലോ നിന്തിരുവടി നൂനം.\\
നിന്തിരുവടിയുടെ ജഠരത്തിങ്കല്‍ നിത്യ-\\
മന്തമില്ലാതോളം ബ്രഹ്മാണ്ഡങ്ങള്‍ കിടക്കുന്നു.\\
അങ്ങനെയുള്ള ഭാവാനെന്നുടെ ജഠരത്തി-\\
ലിങ്ങനെ വസിപ്പതിനെന്തു കാരണം പോറ്റീ!\\
ഭക്തന്മാര്‍വിഷയമായുള്ളൊരു പാരവശ്യം\\
വ്യക്തമായ് കാണായ്വന്നു മുഗ്ദ്ധയാമെനിക്കിപ്പോള്‍.\\
ഭര്‍തൃപുത്രാര്‍ഥാകുലസംസാരദുഃഖാംബുധൗ\\
നിത്യവും നിമഗ്നയായത്യര്‍ഥം ഭ്രമിക്കുന്നേന്‍\\
നിന്നുടെ മഹാമായതന്നുടെ ബലത്തിനാ-\\
ലിന്നു നിന്‍പാദാംഭോജം കാണാനും യോഗം വന്നു.\\
ത്വല്‍ക്കാരുണ്യത്താല്‍ നിത്യമുള്‍ക്കാമ്പില്‍ വസിക്കേണ-\\
മിക്കാണാകിയ രൂപം ദുഷ്കൃതമൊടുങ്ങുവാന്‍.\\
വിശ്വമോഹിനിയായ നിന്നുടെ മഹാമായ\\
വിശ്വേശ! മോഹിപ്പിച്ചീടായ്ക മാം ലക്ഷ്മീപതേ!\\
ദേവലമലൗകികം വൈഷ്ണവമായ രൂപം\\
ദേവേശ! മറയ്ക്കേണം മറ്റുള്ളോര്‍ കാണും മുന്‍പേ.\\
ലാളനാശ്ശേഷാദ്യനുരൂപമായിരിപ്പൊരു\\
ബാലഭാവത്തെ മമ കാട്ടേണം ദയാനിധേ!\\
പുത്രവാത്സല്യവ്യാജമായൊരു പരിചര-\\
ണത്താലേ കടക്കേണം ദുഃഖസംസാരാര്‍ണവം.”\\
ഭക്തിപൂണ്ടിത്ഥം വീണുവണങ്ങി സ്തുതിച്ചപ്പോള്‍\\
ഭക്തവത്സലന്‍ പുരുഷോത്തമനരുള്‍ചെയ്തു:\\
“മാതാവേ! ഭവതിക്കെന്തിഷ്ടമാകുന്നതെന്നാ-\\
ലേതുമന്തരമില്ല ചിന്തിച്ചവണ്ണം വരും.\\
ദുര്‍മദം വളര്‍ന്നൊരു രാവണന്‍ തന്നെക്കൊന്നു\\
സമ്മോദം ലോകങ്ങള്‍ക്കു വരുത്തിക്കൊള്‍വാന്‍ മുന്നം\\
ബ്രഹ്മശങ്കരപ്രമുഖാമരപ്രവന്മാര്‍\\
നിര്‍മലപദങ്ങളാല്‍ സ്തുതിച്ചുസേവിക്കയാല്‍\\
മാനവവംശത്തിങ്കല്‍ നിങ്ങള്‍ക്കു തനയനായ്\\
മാനുഷവേഷം പൂണ്ടു ഭൂമിയില്‍ പിറന്നു ഞാന്‍.\\
പുത്രനായ് പിറക്കണം ഞാന്‍തന്നെ നിങ്ങള്‍ക്കെന്നു ചിത്തത്തില്‍ നിരൂപിച്ചു സേവിച്ചു ചിരകാലം\\
പൂര്‍വജന്മനി പുനരതുകാരണമിപ്പോ-\\
ളേവംഭൂതകമായ വേഷത്തെക്കാട്ടിത്തന്നു.\\
ദുര്‍ലഭം മദ്ദര്‍ശനം മോക്ഷത്തിനായിട്ടുള്ളോ-\\
ന്നില്ലല്ലോ പിന്നെയൊരു ജന്മസംസാരദുഃഖം\\
എന്നുടെ രൂപമിദം നിത്യവും ധ്യാനിച്ചുകൊള്‍-\\
കെന്നാല്‍ വന്നീടും മോക്ഷമില്ല സംശയമേതും.\\
യാതൊരു മര്‍ത്യനിഹ നമ്മിലേ സംവാദമി\\
താദരാല്‍ പഠിക്കതാന്‍ കേള്‍ക്കതാന്‍ ചെയ്യുന്നതും\\
സാധിക്കുമവനു സാരൂപ്യമെന്നറിഞ്ഞാലും\\
ചേതസി മരിക്കുമ്പോള്‍ മല്‍സ്മരണയുമുണ്ടാം.”\\
ഇത്തരമരുള്‍ചെയ്തു ബാലഭാവത്തെപ്പൂണ്ടു\\
സത്വരം കാലും കൈയും കുടഞ്ഞു കരയുന്നോന്‍\\
ഇന്ദ്രനീലാഭപൂണ്ട സുന്ദരരൂപനര-\\
വിന്ദലോചനന്‍ മുകുന്ദന്‍ പരമാനന്ദാത്മാ\\
ചന്ദ്രചൂഡാരവിന്ദമന്ദിരവൃന്ദാരക-\\
വൃന്ദവന്ദിതന്‍ ഭുവി വന്നവതാരം ചെയ്താന്‍.\\
നന്ദനനുണ്ടായിതെന്നാശു കേട്ടൊരു പങ് ക്തി-\\
സ്യന്ദനനഥപരമാനന്ദാകുലനായാന്‍\\
പുത്രജന്മത്തെച്ചൊന്ന ഭൃത്യവര്‍ഗ്ഗത്തിനെല്ലാം\\
വസ്ത്രഭൂഷണാദ്യഖിലാര്‍ത്ഥദാനങ്ങള്‍ ചെയ്താന്‍\\
പുത്രവക്ത്രാബ്ജം കണ്ടു തുഷ്ടനായ് പുറപ്പെട്ടു\\
ശുദ്ധമായ് സ്നാനംചെയ്തു ഗുരുവിന്‍ നിയോഗത്താല്‍\\
ജാതകര്‍മവും ചെയ്തു ദാനവും ചെയ്തു; പിന്നെ-\\
ജ്ജാതനായിതു കൈകേയീസുതന്‍ പിറ്റേന്നാളും.\\
സുമിത്രാപുത്രന്മാരായുണ്ടായിതിവുവരു-\\
മമിത്രാന്തകന്‍ ദശരഥനും യഥാവിധി\\
ചെയ്തിതു ജാതകര്‍മം ബാലന്മാര്‍ക്കെല്ലാവര്‍ക്കും\\
പെയ്തിതു സന്തോഷംകൊണ്ടശ്രുക്കള്‍ ജനങ്ങള്‍ക്കും.\\
സ്വര്‍ണരത്നൗഘവസ്ത്രഗ്രാമാദി പദാര്‍ഥങ്ങ-\\
ളെണ്ണമില്ലാതോളം ദാനംചെയ്തു ഭൂദേവാനാം\\
വിണ്ണവര്‍നാട്ടിലുമുണ്ടായിതു മഹോത്സവം\\
കണ്ണുകളായിരവും തെളിഞ്ഞു മഹേന്ദ്രനും\\
സമസ്തലോകങ്ങളുമാത്മാവാമിവങ്കലേ\\
രമിച്ചീടുന്നു നിത്യമെന്നോര്‍ത്തു വസിഷ്ഠനും\\
ശ്യാമളനിറംപൂണ്ട കോമളകുമാരനു\\
രാമനെന്നൊരു തിരുനാമവുമിട്ടാനല്ലോ\\
ഭരണനിപുണനാം കൈകേയീതനയനു\\
ഭരതനെന്നു നാമമരുളിച്ചെയ്തു മുനി,\\
ലക്ഷണാന്വിതനായ സുമിത്രാതനയനു\\
ലക്ഷ്മണനെന്നു തന്നെ നാമവുമരുള്‍ചെയ്തു;\\
ശത്രുവൃന്ദത്തെ ഹനിച്ചീടുക നിമിത്തമായ്\\
ശത്രുഘ്നനെന്നു സുമിത്രാത്മജാവരജനും\\
നാമധേയവും നാലു പുത്രര്‍ക്കും വിധിച്ചേവം\\
ഭൂമിപാലനും ഭാര്യമാരുമായാനന്ദിച്ചാന്‍.\\
സാമോദം ബാലക്രീഡാതല്‍പരന്മാരാം കാലം\\
രാമലക്ഷ്മണന്മാരും തമ്മിലൊന്നിച്ചു വാഴും\\
ഭരതശത്രുഘ്നന്മാരൊരുമിച്ചെല്ലാനാളും\\
മരുവീടുന്നു പായസാംശാനുസാരവശാല്‍\\
കോമളന്മാരായൊരു സോദരന്മാരുമായി\\
ശ്യാമളനിറം പൂണ്ട ലോകാഭിരാമദേവന്‍\\
കാരുണ്യാമൃതപൂര്‍ണാപാംഗവീക്ഷണം കൊണ്ടും\\
സാരസ്യാവ്യക്തവര്‍ണാലാപപീയൂഷം കൊണ്ടും\\
വിശ്വമോഹനമായ രൂപസൗന്ദര്യം കൊണ്ടും\\
നിശ്ശേഷാനന്ദപ്രദദേഹമാര്‍ദവം കൊണ്ടും\\
ബന്ധൂകദന്താംബരചുംബനരസം കൊണ്ടും\\
ബന്ധുരദന്താങ്കുരസ്പഷ്ടഹാസാഭ കൊണ്ടും\\
ഭൂതലസ്ഥിതപാദാബ്ജദ്വയയാനം കൊണ്ടും\\
ചേതോമോഹനങ്ങളാം ചേഷ്ടിതങ്ങളെക്കൊണ്ടും\\
താതനുമമ്മമാര്‍ക്കും നഗരവാസികള്‍ക്കും\\
പ്രീതിനല്‍കിനാന്‍ സമസ്തേന്ദ്രിയങ്ങള്‍ക്കുമെല്ലാം.\\
ഫാലദേശാന്തേ സ്വര്‍ണാശ്വത്ഥമ്പര്‍ണാകാരമായ്\\
മാലേയമണിഞ്ഞതില്‍ പറ്റീടും കുരളവും\\
അഞ്ജനമണിഞ്ഞതിമഞ്ജുളതരമായ\\
കഞ്ജനേത്രവും കടാക്ഷാവലോകനങ്ങളും\\
കര്‍ണാലങ്കാരമണികുണ്ഡലം മിന്നീടുന്ന\\
സ്വര്‍ണദര്‍പ്പണസമഗണ്ഡമണ്ഡലങ്ങളും\\
ശാര്‍ദൂലനഖങ്ങളും വിദ്രുമമണികളും\\
ചേര്‍ത്തുടന്‍ കാര്‍ത്തസ്വരമണികള്‍ മധ്യേമധ്യേ\\
കോര്‍ത്തു ചാര്‍ത്തീടുന്നൊരു കണ്ഠകാണ്ഡോദ്യോതവും\\
മുത്തുമാലകള്‍ വനമാലകളോടും പൂണ്ട\\
വിസ്തൃതോരസി ചാര്‍ത്തും തുളസീമാല്യങ്ങളും\\
നിസ്തുലപ്രഭവത്സലാഞ്ഛനവിലാസവും\\
അംഗദങ്ങളും വലയങ്ങള്‍ കങ്കണങ്ങളും\\
അംഗുലീയങ്ങള്‍കൊണ്ടു ശോഭിച്ച കരങ്ങളും\\
കാഞ്ചനസദൃശപീതാംബരോപരി ചാര്‍ത്തും\\
കാഞ്ചികള്‍ നൂപുരങ്ങളെന്നിവ പലതരം\\
അലങ്കാരങ്ങള്‍ പൂണ്ടു സോദരന്മാരോടുമൊ-\\
രലങ്കാരത്തെച്ചേര്‍ത്താന്‍ ഭൂമിദേവിക്കു നാഥന്‍.\\
ഭര്‍ത്താവിന്നധിവാസമുണ്ടായോരയോധ്യയില്‍\\
പൊല്‍ത്താര്‍മാനിനിതാനും കളിച്ചു വിളങ്ങിനാള്‍.\\
ഭൂതലത്തിങ്കലെല്ലാമന്നുതൊട്ടനുദിനം\\
ഭൂതിയും വര്‍ധിച്ചിതു ലോകവുമാനന്ദിച്ചു.\\
ദമ്പതിമാരെ ബാല്യംകൊണ്ടേവം രഞ്ജിപ്പിച്ചു\\
സമ്പ്രതി കൗമാരവും സമ്പ്രാപിച്ചിതു മെല്ലെ.\\
വിധിനന്ദനനായ വസിഷ്ഠമഹാമുനി\\
വിധിപൂര്‍വകമുപനിച്ചിതു ബാലന്മാരെ.\\
ശ്രുതികളോടു പുനരംഗങ്ങളുപാംഗങ്ങള്‍\\
സ്മൃതികളുപസ്മൃതികളുമശ്രമമെല്ലാം\\
പാഠമായതു പാര്‍ത്താലെന്തൊരത്ഭുത, മവ\\
പാടവമേറും നിജശ്വാസങ്ങള്‍തന്നെയല്ലോ.\\
സകല ചരാചരഗുരുവായ് മരുവീടും\\
ഭഗമാന്‍തനിക്കൊരു ഗുരുവായ് ചമഞ്ഞീടും\\
സഹസ്രപത്രോത്ഭവപുത്രനാം വസിഷ്ഠന്റെ\\
മഹത്വമേറും ഭാഗ്യമെന്തു ചൊല്ലാവതോര്‍ത്താല്‍.\\
ധനുര്‍വേദാംഭോനിധിപാരഗന്മാരായ് വന്നു\\
തനയന്മാരെന്നതു കണ്ടോരു ദശരഥന്‍\\
മനസി വളര്‍ന്നൊരു പരമാനന്ദം പൂണ്ടു\\
മുനിനായകനേയുമാനന്ദിപ്പിച്ചു നന്നായ്.\\
ആമോദം വളര്‍ന്നുള്ളില്‍ സേവ്യസേവകഭാവം\\
രാമലക്ഷ്മണന്മാരും കൈക്കൊണ്ടാ, രതുപോലെ\\
കോമണന്മാരായ്മേവും ഭരതശത്രുഘ്നന്മാര്‍\\
സ്വാമിഭൃത്യകഭാവം കൈക്കൊണ്ടാരനുദിനം.\\
രാഘവനതുകാലമേകദാ കുതൂഹലാല്‍\\
വേഗമേറീടുന്നൊരു തുരഗരത്നമേറി\\
പ്രാണസമ്മിതനായ ലക്ഷമണനോടും ചേര്‍ന്നു\\
ബാണതൂണീരഖഡ്ഗാദ്യായുധങ്ങളും പൂണ്ടു\\
കാനനദേശേ നടന്നീടിനാന്‍ നായാട്ടിനായ്-\\
ക്കാണായ ദുഷ്ടമൃഗസഞ്ചയം കൊലചെയ്താന്‍.\\
ഹരിണഹരികരികരടിഗിരികിരി-\\
ഹരിശാര്‍ദൂലാദികളമിതവന്യമൃഗം\\
വധിച്ചു കൊണ്ടുവന്നു ജനകന്‍ കാല്ക്കല്‍ വെച്ചു\\
വിധിച്ചവണ്നം നമസ്കരിച്ചു വണങ്ങിനാന്‍.\\
നിത്യവുമുഷസ്യൂഷസ്യുത്ഥായ കുളിച്ചൂത്തു\\
ഭക്തികൈക്കൊണ്ടു സന്ധ്യാവന്ദനം ചെയ്തശേഷം\\
ജനകജനനിമാര്‍ചരണാംബുജം വന്ദി-\\
ച്ചനുജനോടു ചേര്‍ന്നു പൗരകാര്യങ്ങളെല്ലാം\\
ചിന്തിച്ചു ദണ്ഡനീതി നീങ്ങാതെ ലോകം തങ്കല്‍\\
സന്തതം രഞ്ജിപ്പിച്ചു ധര്‍മ്മപാലനം ചെയ്തു\\
ബന്ധുക്കളോടും ഗുരുഭൂതന്മാരോടും ചേര്‍ന്നു\\
സന്തുഷ്ടാത്മനാ മൃഷ്ടഭോജനം കഴിച്ചഥ\\
ധര്‍മശാസ്ത്രാദിപുരാണേതിഹാസങ്ങള്‍ കേട്ടു\\
നിര്‍മല ബ്രഹ്മാനന്ദലീനചേതസാ നിത്യം\\
പരമന്‍ പരാപരന്‍ പരബ്രഹ്മാഖ്യന്‍ പരന്‍\\
പുരുഷന്‍ പരമാത്മാ പരമാനന്ദമൂര്‍ത്തി\\
ഭൂമിയില്‍ മനുഷ്യനായവതാരംചെയ്തേവം\\
ഭൂമിപാലകവൃത്തി കൈക്കൊണ്ടു വാണീടിനാന്‍.\\
ചേതസാ വിചാരിച്ചു കാണ്‍കിലോ പരമാര്‍ഥ-\\
മേതുമേ ചെയ്യുന്നോന,ല്ലില്ലല്ലോ വികാരവും\\
ചിന്തിക്കില്‍ പരിണാമമില്ലാതൊരാത്മാനന്ദ-\\
മെന്തൊരു മഹാമായാവൈഭവം ചിത്രം! ചിത്രം!
\end{verse}

%unit 5). vishvamithrante yaagaraksha

\section{വിശ്വാമിത്രന്റെ യാഗരക്ഷ}

\begin{verse}
അക്കാലം വിശ്വാമിത്രനാകിയ മുനികുല-\\
മുഖ്യനുമയോധ്യയ്ക്കാമ്മാറെഴുന്നള്ളീടിനാന്‍\\
രാമനായവനിയില്‍ മായയാ ജനിച്ചൊരു\\
കോമളമായ രൂപം പൂണ്ടൊരു പരാത്മാനം\\
സത്യജ്ഞനാനന്താനന്ദാമൃതം കണ്ടുകൊള്‍വാന്‍\\
ചിത്തത്തില്‍ നിറഞ്ഞാശു വഴിഞ്ഞ ഭക്തിയോടെ.\\
കൗശികന്‍തന്നെ ക്കണ്ടു ഭൂപതി ദശരഥ-\\
നാശു സംഭ്രമത്തോടും പ്രത്യുത്ഥാനവും ചെയ്തു\\
വിധിനന്ദനനോടും ചെന്നെതിരേറ്റു യഥാ-\\
വിധി പൂജയും ചെയ്തു വന്ദിച്ചുനിന്നു ഭാക്ത്യാ.\\
സസ്മിതം മുനിവരന്‍തന്നോടു ചൊല്ലീടിനാന്‍:\\
“അസ്മജ്ജന്മവുമിന്നു വന്നിതു സഫലമായ്\\
നിന്തിരുവടിയെഴുന്നള്ളിയമൂലം കൃതാര്‍-\\
ത്ഥാന്തരാത്മാവായിതു ഞാനിഹ തപോനിധേ!\\
ഇങ്ങനെയുള്ള് നിങ്ങളെഴുന്നള്ളീടും ദേശം\\
മംഗലമായ്വന്നാശു സമ്പത്തും താനേ വരും,\\
എന്തൊന്നു ചിന്തിച്ചെഴുന്നള്ളിയതെന്നുമിപ്പോള്‍\\
നിന്തിരുവടിയരുള്‍ചെയ്യണം ദയാനിധേ!\\
എന്നാലാകുന്നതെല്ലാം ചെയ്വന്‍ ഞാന്‍ മടിയാതെ\\
ചൊന്നാലും പരമാര്‍ഥം താപസകുലപതേ!”\\
വിശ്വാമിത്രനും പ്രീതനായരുള്‍ചെയ്തീടിനാന്‍\\
വിശ്വാസത്തോടു ദശരഥനോടതുനേരം:\\
“ഞാനമവാസ്യതോറും പിതൃദേവാദികളെ\\
ധ്യാനിച്ചു ചെയ്തീടുന്ന ഹോമത്തെ മുടക്കുന്നോര്‍\\
മാരീചസുബാഹുമുഖ്യന്മാരാം നക്തഞ്ചര-\\
ന്മാരിരുവരുമനുചരന്മാരായുള്ളോരും\\
അവരെ നിഗ്രഹിച്ചു യാഗത്തെ രക്ഷിപ്പാനാ-\\
യവനീപതേ! രാമദേവനെയയയ്ക്കേണം\\
പുഷ്കരോത്ഭവപുത്രന്‍തന്നോടു നിരൂപിച്ചു\\
ലക്ഷ്മണനേയും കൂടെ നല്‍കണം മടിയാതെ.\\
നല്ലതു വന്നീടുക നിനക്കു മഹീപതേ!\\
കല്യാണമതേ! കരുണാനിധേ! നരപതേ!”\\
ചിന്താചഞ്ചലനായ പങ് ക്തിസ്യന്ദനനൃപന്‍\\
മന്ത്രിച്ചു ഗുരുവിനോടേകാന്തേ ചൊല്ലീടിനാന്‍:\\
“എന്തു ചൊല്‍വതു ഗുരോ! നന്ദനന്‍തന്നെ മമ\\
സന്ത്യജിച്ചീടുവതിനില്ലല്ലോ ശക്തിയോട്ടും\\
എത്രയും കൊതിച്ച കാലത്തിങ്കല്‍ ദൈവവശാല്‍\\
സിദ്ധിച്ച തനയനാം രാമനെപ്പിരിയുമ്പോള്‍\\
നിര്‍ണയം മരിക്കും ഞാന്‍, രാമനെ നല്കീടാഞ്ഞാ-\\
ലന്വയനാശം കൂടെ വരുത്തും വിശ്വാമിത്രന്‍.\\
എന്തെന്നു നല്ലതിപ്പോളെന്നു നിന്തിരുവടി\\
ചിന്തിച്ചു തിരിച്ചരുളിച്ചെയ്തീടുകവേണം.”\\
“എങ്കിലോ ദേവഗുഹ്യം കേട്ടാലുമതിഗോപ്യം\\
സങ്കടമുണ്ടാകേണ്ട സന്തതം ധരാപതേ!\\
മാനുഷനല്ല രാമന്‍ മാനവശിഖാമണേ!\\
മാനമില്ലാത പരമാത്മാവു സദാനന്ദന്‍\\
പത്മസംഭവന്‍ മുന്നം പ്രാര്‍ഥിക്കമൂലമായി\\
പത്മലോചനന്‍ ഭൂമിഭാരത്തെക്കളവാനായ്\\
നിന്നുടെ തനയനായ് കൗസല്യാദേവിതന്നില്‍\\
വന്നവതരിച്ചിതു വൈകുണ്ഠന്‍ നാരായണന്‍”\\
നിന്നുടെ പൂര്‍വജന്മം ചൊല്ലുവന്‍ ദശരഥ!\\
മുന്നം നീ ബ്രഹ്മാത്മജന്‍ കശ്യപ പ്രജാപതി\\
നിന്നുടെ പത്നിയാകുമദിതി കൗസല്യ കേ-\\
ളന്നിരുവരും കൂടിസ്സന്തതിയുണ്ടാവാനായ്\\
ബഹുവത്സരമുഗ്രം തപസ്സു ചെയ്തു നിങ്ങള്‍\\
മുഹുരാത്മനി വിഷ്ണു പൂജാധ്യാനാദിയോടും\\
ഭക്തവത്സലന്‍ ദേവന്‍ വരദന്‍ ഭഗവാനും\\
പ്രത്യക്ഷീഭവിച്ചു നീ ’വാങ്ങിക്കൊള്‍ വര’മെന്നാന്‍.\\
‘പുത്രനായ് പിറക്കേണമെനുക്കു ഭവാനെന്നു\\
സത്വരമപേക്ഷിച്ച കാരണമിന്നു നാഥന്‍\\
പുത്രനായ് പിറന്നിതു രാമനെന്നറിഞ്ഞാലും\\
പൃഥ്വീന്ദ്ര! ശേഷന്‍ തന്നെ ലക്ഷ്മണനാകുന്നതും\\
ശംഖചക്രങ്ങളല്ലോ ഭാരതശത്രുഘ്നന്മാര്‍\\
ശങ്കക്കൈവിട്ടു കേട്ടുകൊണ്ടാലുംമിനിയും നീ.\\
“യോഗമായാദേവിയും സീതയായ് മിഥിലയില്‍\\
യാഗവേലായാമയോനിജയായുണ്ടായ് വന്നു.\\
ആഗതനായാന്‍ വിശ്വാമിത്രനുമവര്‍തമ്മില്‍\\
യോഗം കൂട്ടീടുവതിനെന്നറിഞ്ഞീടണം നീ\\
എത്രയും ഗുഹ്യമിതു വക്തവ്യമല്ലതാനും\\
പുത്രനെക്കൂടെയയച്ചീടുക മടിയാതെ.”\\
സന്തുഷ്ടനായ ദശരഥനും കൗശികനെ\\
വന്ദിച്ചു യഥാവിധി പൂജിച്ചു ഭക്തിപൂര്‍വം\\
‘രാമലക്ഷ്മണന്മാരെക്കൊണ്ടുപൊയ്ക്കൊണ്ടാലു’മെ-\\
ന്നാമോദം പൂണ്ടു നല്കി ഭൂപതി പുത്രന്മാരെ\\
‘വരിക രാമ! രാമ! ലക്ഷ്മണ! വരിക’യെ-\\
ന്നരികേ ചേര്‍ത്തു മാറിലണച്ചു ഗാഢം ഗാഢം\\
പുണര്‍ന്നു പുണര്‍ന്നുടന്‍ നുകര്‍ന്നു ശിരസ്സിങ്കല്‍\\
‘ഗുണങ്ങള്‍ വരുവാനായ് പോവി’നെന്നുരചെയ്താന്‍\\
ജനകജനനിമാര്‍ ചരണാംബുജം കൂപ്പി\\
മുനിനായകന്‍ ഗുവുപാദവും വന്ദിച്ചുടന്‍\\
വിശ്വാമിത്രനെച്ചെന്നു വന്ദിച്ചു കുമാരന്മാര്‍\\
വിശ്വരക്ഷാര്‍ഥം പരിഗ്രഹിച്ചു മുനീന്ദ്രനും.\\
ചാപതൂണീരബാണഖഡ്ഗപാണികളായ\\
ഭൂപതികുമാരന്മാരോടും കൗശികമുനി യാത്രയുമയപ്പിച്ചാശീര്‍വാദങ്ങളും ചൊല്ലി\\
തീര്‍ഥപാദന്മാരോടും നടന്നു വിശ്വാമിത്രന്‍.\\
മന്ദംപോയ് ചില ദേശം കടന്നൊരനന്തരം\\
മന്ദഹാസവുംചെയ്തിട്ടരുളിച്ചെയ്തു മുനി:\\
‘രാമ! രാഘവ! രാമ! ലക്ഷ്മണകുമാര! കേള്‍\\
കോമളന്മാരായുള്ള ബാലന്മാരല്ലോ നിങ്ങള്‍\\
ദാഹമെന്തെന്നും വിശപ്പെന്തെന്നുമറിയാത\\
ദേഹങ്ങളല്ലോ മുന്നം നിങ്ങള്‍ക്കെന്നതുമൂലം\\
ദാഹവും വിശപ്പുമുണ്ടാകാതെയിരിപ്പാനായ്\\
മാഹാത്മ്യമേറുന്നൊരു വിദ്യകളിവ രണ്ടും\\
ബാലകന്മാരേ! നിങ്ങള്‍ പഠിച്ചു ജപിച്ചാലും\\
ബലയും പുനരതിബലയും മടിയാതെ.\\
ദേവനിര്‍മ്മിതകളീവിദ്യക’ളെന്നു രാമ-\\
ദേവനുമനുജനുമുപദേശിച്ചു മുനി\\
ക്ഷുല്‍പിപാസാദികളും തീര്‍ന്ന ബാലന്മാരുമാ-\\
യപ്പോഴേ ഗംഗ കടന്നീടിനാന്‍ വിശ്വാമിത്രന്‍
\end{verse}

% unit 6). thaadaka vadham 

\section{താടകാവധം}

\begin{verse}
താടകാവനം പ്രാപിച്ചീടിനോരനന്തരം\\
ഗൂഢസ്മേരവും പൂണ്ടു പറഞ്ഞു വിശ്വാമിത്രന്‍:\\
“രാഘവ! സത്യപരാക്രമവാരിധേ! രാമ!\\
പോകുമാറില്ലീവഴിയാരുമേയിതുകാലം\\
കാടിതു കണ്ടായോ നീ? കാമരൂപിണിയായ\\
താടക ഭയങ്കരി വാണീടും ദേശമല്ലോ.\\
അവളെപ്പേടിച്ചാരും നേര്‍വഴി നടപ്പീല\\
ഭുവനവാസിജനം ഭൂവനേശ്വര! പോറ്റീ!\\
കൊല്ലണമവളെ നീ വല്ലജാതിയുമതി-\\
നില്ലൊരു ദോഷ“മെന്നു മാമുനി പറഞ്ഞപ്പോള്‍\\
മെല്ലവേയൊന്നു ചെറുഞാണൊലി ചെയ്തു രാമ-\\
നെല്ലാലോകവുമൊന്നു വിറച്ചിതതുനേരം\\
ചെറുഞാണൊലി കേട്ടു കോപിച്ചു നിശാചരി\\
പെരികെ വേഗത്തോതുമടുത്തു ഭക്ഷിപ്പാനായ്\\
അന്നേരമൊരു ശരമയച്ചു രാഘവനും\\
ചെന്നു താടകാമാറില്‍ കൊണ്ടിതു രാമബാണം\\
പാരതില്‍ മല ചിറ്കറ്റു വീണതുപോലെ\\
ഘോരരൂപിണിയായ താടക വീണാളല്ലോ.\\
സ്വര്‍ണരത്നാഭരണഭൂഷിതഗാത്രിയായി\\
സുന്ദരിയായ യക്ഷിതന്നെയും കാണായ്വന്നു.\\
ശാപത്താല്‍ നക്തഞ്ചരിയായൊരു യക്ഷിതാനും\\
പ്രാപിച്ചു ദേവലോകം രാമദേവാനുജ്ഞയാ\\
കൗശികമുനീന്ദ്രനും ദിവ്യാസ്ത്രങ്ങളെയെല്ലാ-\\
മാശു രാഘവനുപദേശിച്ചു സലക്ഷ്മണം\\
നിര്‍മലന്മാരാം കുമാരന്മാരും മുനീന്ദ്രനും\\
രമ്യകാനനേ തത്ര വസിച്ചു കാമാശ്രമേ\\
രാത്രിയും പിന്നിട്ടവര്‍ സന്ധ്യാവന്ദനം ചെയ്തു\\
യാത്രയും തുടങ്ങിനാരാസ്ഥയാ പുലര്‍കാലേ.\\
പുക്കിതു സിദ്ധാശ്രമം വിശ്വാമിത്രനും മുനി-\\
മുഖ്യന്മാരെതിരേറ്റു വന്ദിച്ചാരതുനേരം\\
രാമലക്ഷ്മണന്മാരും വന്ദിച്ചു മുനികളെ\\
പ്രേമമുള്‍ക്കൊണ്ടു മിനിമാരും സല്‍ക്കാരം ചെയ്താര്‍.\\
വിശ്രമിച്ചനന്തരം രാഘവന്‍ തിരുവടി\\
വിശ്വാമിത്രനെ നോക്കി പ്രീതിപൂണ്ടരുള്‍ചെയ്തു\\
“താപസോത്തമ! ഭവാന്‍ ദീക്ഷിക്ക യാഗമിനി\\
താപംകൂടാതെ രക്ഷിച്ചീടുവനേതു ചെയ്തും.\\
ദുഷ്ടരാം നിശാചരേന്ദ്രന്മാരെക്കാട്ടിത്തന്നാല്‍\\
നഷ്ടമാക്കുവന്‍ ബാണംകൊണ്ടു ഞാന്‍ തപോനിധേ!”\\
യാഗവും ദീക്ഷിച്ചിതു കൗശികനതുകാല-\\
മാഗമിച്ചിതു നക്തഞ്ചരന്മാര്‍ പടയോടും\\
മധ്യാഹ്നകാല്ളേ മേല്‍ഭാഗത്തിങ്കല്‍നിന്നു തത്ര\\
രക്തവൃഷ്ടിയും തുടങ്~ഈടിനാരതുനേരം.\\
പാരാതെ രണ്ടു ശരം തൊടുത്തു രാമദേവന്‍\\
മാരീചസുബാഹുവീരന്മാരെ പ്രയോഗിച്ചാന്‍.\\
കൊന്നിതു സുബാഹുവാമവനെയൊരു ശര-\\
മന്നേരം മാരീചനും ഭീതിപൂണ്ടോടീടിനാന്‍.\\
ചെന്നിതു രാമബാണം പിന്നാലെ കൂടെക്കൂടെ\\
ഖിന്നനായേറിയൊരു യോജന പാഞ്ഞാനവന്‍\\
അര്‍ണ്ണവം തന്നില്‍ ചെന്നു വീണിതു മാരീചനു-\\
മന്നേരമവിടെയും ചെന്നിതു ദഹിപ്പാനായ്.\\
പിന്നെ മറ്റെങ്ങുമൊരു ശരണമില്ലാഞ്ഞവ-\\
‘നെന്നെ രക്ഷിക്കേണ’മെന്നഭയം പുക്കീടിനാന്‍.\\
ഭക്തവത്സലനഭയം കൊടുത്തതുമൂലം\\
ഭക്തനായ് വന്നാനന്നു തുടങ്ങി മാരീചനും\\
പറ്റലര്‍കുലകാലനാകിയ സൗമിത്രിയും\\
മറ്റുള്ള പടയെല്ലാം കൊന്നിതു ശരങ്ങളാല്‍.\\
ദേവകള്‍ പുഷ്പവൃഷ്ടി ചെയ്തിതു സന്തോഷത്താല്‍\\
ദേവദുന്ദുഭികളും ഘോഷിച്ചിതതുനേരം\\
യക്ഷകിന്നരസിദ്ധചാരണഗന്ധര്‍വന്മാര്‍\\
തല്‍ക്ഷണേ കൂപ്പിസ്തുതിച്ചേറ്റവുമാനന്ദിച്ചാര്‍.\\
വിശ്വാവിത്രനും പരമാനന്ദം പൂണ്ടു പുണര്‍-\\
ന്നശ്രുപൂര്‍ണാര്‍ദ്രാകുലനേത്രപത്മങ്ങളോടും\\
ഉത്സംഗേ ചേര്‍ത്തു പരമാശീര്‍വാദവും ചെയ്തു\\
വത്സന്മാരെയും ഭുജിപ്പിച്ചിതു വാത്സല്യത്താല്‍.\\
ഇരുന്നു മൂന്നു ദിനമോരോരോ പുരാണങ്ങള്‍\\
പറഞ്ഞു രസിപ്പിച്ചു കൗശികനവരുമായ്\\
അരുള്‍ചെയ്തിതു നാലാംദിവസം പിന്നെ മുനി:\\
‘അരുതു വൃഥാകാലം കളകെന്നുള്ളതേതും.\\
ജനകമഹീപതിതന്നുടെ മഹായജ്ഞ-\\
മിനി വൈകാതെ കാന്മാന്‍പോക നാം വത്സന്മാരേ!\\
ചൊല്ലെഴും ത്രയംബകമാകിന മാഹേശ്വര-\\
വില്ലുണ്ടു വിദേഹരാജ്യത്തിങ്കലിരിക്കുന്നു.\\
ശ്രീമഹാദേവന്‍തന്നെ വെച്ചിരിക്കുന്നു പുരാ-\\
ഭൂമിപാലേന്ദ്രന്മാരാലര്‍ച്ചിതമനുദിനം,\\
ക്ഷോണീപാലേന്ദ്രകുലജാതനാകിയ ഭവാന്‍\\
കാണണം മഹാസത്വമാകിയ ധനൂരത്നം.’\\
താപസേന്ദ്രന്മാരോടുമീവണ്ണമരുള്‍ചെയ്തു\\
ഭുപതിബാലന്മാരും കൂടെപ്പോയ് വിശ്വാമിത്രന്‍\\
പ്രാപിച്ചു ഗംഗാതീരം ഗൗതമാശ്രമം തത്ര\\
ശോഭപൂണ്ടൊരു പുണ്യദേശമാനന്ദപ്രദം\\
ദിവ്യപാദപലതാകുസുമഫലങ്ങളാല്‍\\
സര്‍വമോഹനകരം ജന്തുസഞ്ചയഹീനം\\
കണ്ടു കൗതുകംപൂണ്ടു വിശ്വാമിത്രനെ നോക്കി-\\
പ്പുണ്ഡരീകേക്ഷണനുമീവണ്ണമരുള്‍ചെയ്തു:\\
“ആശ്രമപദമിദമാര്‍ക്കുള്ളു മനോഹര-\\
മാശ്രയയോഗ്യം നാനാജന്തു വര്‍ജിതം താനും\\
എത്രയുമാഹ്ലാദമുണ്ടായിതു മനസി മേ\\
തത്ത്വമെന്തെന്നതരുള്‍ചെയ്യേണം തപോനിധേ!“
\end{verse}

%unit 7). ahalyaamoksham

\section{അഹല്യാമോക്ഷം}

\begin{verse}
എന്നതുകേട്ടു വിശ്വാമിത്രനുമുരചെയ്തു\\
പന്നഗശായി പരന്‍തന്നോടു പരമാര്‍ഥം\\
കേട്ടാലും പുരാവൃത്തമെങ്കിലോ കുമാര! നീ\\
വാട്ടമില്ലാത തപസ്സുള്ള ഗൗതമമുനി\\
ഗംഗാരോധസി നല്ലോരാശ്രമത്തിങ്കലത്ര\\
മംഗലം വര്‍ധിച്ചീടും തപസാ വാഴുംകാലം\\
ലോകേശന്‍ നിജസുതയായുള്ളോരഹല്യയാം\\
ലോകസുന്ദരൈയായ ദിവ്യകന്യകാരത്നം\\
ഗൗതമമുനീന്ദ്രനു കൊടുത്തു വിധാതാവും\\
കൗതുകം പൂണ്ടു ഭാര്യാഭര്‍ത്താക്കന്മാരായവര്‍;\\
ഭര്‍തൃശുശ്രൂഷാബ്രഹ്മചര്യാദിഗുണങ്ങള്‍ ക-\\
ണ്ടെത്രയും പ്രസാദിച്ചു ഗൗതമമുനീന്ദ്രനും\\
തന്നുടെ പത്നിയായോരഹല്യയോടും ചേര്‍ന്നു\\
പര്‍ണശാലയിലത്ര വസിച്ചു ചിരകാലം.\\
വിശ്വമോഹിനിയായോരഹല്യാരൂപം കണ്ടു\\
ദുശ്ച്യവനനും കുസുമായുധവശനായാന്‍\\
ചെന്തൊണ്ടിവായ്മലരും പന്തൊക്കും മുലകളും\\
ചന്തമേറീടും തുടക്കാമ്പുമാസ്വദിപ്പതി-\\
നെന്തൊരു കഴിവെന്നു ചിന്തിച്ചു ശതമഖന്‍\\
ചെന്താര്‍ബാണാര്‍ത്തികൊണ്ടു സന്താപം മുഴുക്കയാല്‍\\
സന്തതം മനക്കാമ്പില്‍ സുന്ദരഗാത്രീരൂപം\\
ചിന്തിച്ചു ചിന്തിച്ചനംഗാന്ധനായ് വന്നാനല്ലോ.\\
അന്തരാത്മനി വിബുധേന്ദ്രനുമതിനിപ്പോ-\\
ളന്തരം വരാതെയൊരന്തരമെന്തെന്നോര്‍ത്തു.\\
ലോകേശാത്മജസുതനന്ദനനുടെ രൂപം\\
നാകനായകന്‍ കൈക്കൊണ്ടന്ത്യയാമാദിയിങ്കല്‍\\
സന്ധ്യാവന്ദനത്തിനു ഗൗതമന്‍ പോയനേര-\\
മന്തരാ പുക്കാനുടജാന്തരേ പരവശാല്‍.\\
സുത്രാമാവഹല്യയെ പ്രാപിച്ചു സസംഭ്രമം\\
സത്വരം പുറപ്പെട്ട നേരത്തു ഗൗതമനും\\
മിത്രന്‍ തന്നുദയമൊട്ടടുത്തീലെന്നു കണ്ടു\\
ബദ്ധസന്ദേഹം ചെന്ന നേരത്തു കാണായ്വന്നു\\
വൃത്രാരാതിക്കു മിനിശ്രേഷ്ഠനെ ബലാലപ്പോള്‍\\
വിത്രസ്തനായെത്രയും വേപഥു പുണ്ടു നിന്നാന്‍.\\
തന്നുടെ രൂപം പരിഗ്രഹിച്ചു വരുന്നവന്‍-\\
തന്നെക്കണ്ടതികോപം കൈക്കൊണ്ടു മുനീന്ദ്രനും\\
‘നില്ലുനില്ലാരാകുന്നതെന്തിതു ദുഷ്ടാത്മാവേ!\\
ചൊല്ലു ചൊല്ലെന്നോടു നീയെല്ലാമേ പരമാര്‍ഥം\\
വല്ലാതെ മമ രൂപം കൈക്കൊള്‍വാനെന്തു മൂലം\\
നിര്‍ലജ്ജനായ ഭവാനേതൊരു മഹാപാപി?\\
സത്യമെന്നോടു ചൊല്ലീടറിഞ്ഞേനല്ലോ തവ\\
വൃത്താന്തം പറയായ്കില്‍ ഭസ്മമാക്കുവനിപ്പോള്‍’\\
ചൊല്ലിനാനതുനേരം താപസേന്ദ്രനെ നോക്കി:\\
‘സ്വര്‍ല്ലോകാധിപനായ കാമകിങ്കരനഹം\\
വല്ലായ്മയെല്ലാമകപ്പെട്ടിതു മൂഢത്വംകൊ-\\
ണ്ടെല്ലാം നിന്തിരുവടി പൊറുത്തുകൊള്ളേണമേ.’\\
“സഹസ്രഭഗനായി ബ്ഭവിക്ക ഭവാനിനി-\\
സ്സഹിച്ചീടുക ചെയ്ത് ദുഷ്കര്‍മഫലമെല്ലാം.”\\
തപസ്വീശ്വരനായ ഗൗതമന്‍ ദേവേന്ദ്രനെ-\\
ശ്ശപിച്ചാശ്രമമകം പുക്കപ്പോളഹല്യയും\\
വേപഥുപൂണ്ടു നില്കുന്നതുകണ്ടരുള്‍ചെയ്തു\\
താപസോത്തമനായ ഗൗതമന്‍ കോപത്തോടെ:\\
‘കഷ്ടമെത്രയും തവ ദുര്‍വൃത്തം ദുരാചാരേ!\\
ദുഷ്ടമാനസേ! തവ സാമര്‍ഥ്യം നന്നു പാരം.\\
ദുഷ്കൃതമൊടുങ്ങുവാനിതിന്നു ചൊല്ലീടുവന്‍\\
നിഷ്കൃതിയായുള്ളൊരു ദുസ്തരമഹാവ്രതം.\\
കാമകിങ്കരേ! ശിലാരൂപവുംകൈക്കൊണ്ടു നീ\\
രാമപാദാബ്ജം ഭജിച്ചിവിടേ വസിക്കണം\\
നീഹാരാതപവായുവര്‍ഷാദികളും സഹി-\\
ച്ചാഹാരാദികളേതും കൂടാതെ ദിവാരാത്രം\\
നാനാജന്തുക്കളൊന്നുമിവിടെയുണ്ടയ്വരാ\\
കാനനദേശേ മദീയാശ്രമേ മനോഹരേ\\
ഇങ്ങനെ പല ദിവ്യവത്സരം കഴിയുമ്പോ-\\
ളിങ്ങെഴുന്നള്ളും രാമദേവനുമനുജനും\\
ശ്രീരാമപാദാംഭോജസ്പര്‍ശമുണ്ടായീടുന്നാള്‍\\
തീരും നിന്‍ ദുരിതങ്ങളെല്ലാമെന്നറിഞ്ഞാലും\\
പിന്നെ നീ ഭക്തിയോടെ പൂജിച്ചു കുമ്പിട്ടു കൂപ്പി\\
നാഥനെ സ്തുതിക്കുമ്പോള്‍ ശാപമോക്ഷവും വന്നു\\
പൂതമാനസയായാലെന്നെയും ശുശ്രൂഷിക്കാം.\\
എന്നരുള്‍ചെയ്തു മുനി ഹിമവല്‍പാര്‍ശ്വം പുക്കാ-\\
നന്നുതൊട്ടിവിടെ വാണീടിനാളഹല്യയും\\
“നിന്തിരുമലരടിച്ചെന്തളിര്‍പ്പൊടിയേല്പാ-\\
നെന്തൊരു കഴിവെന്നു ചിന്തിച്ചു ചിന്തിച്ചുള്ളില്‍\\
സന്താപം പൂണ്ടുകൊണ്ടു സന്തതം വസിക്കുന്നു\\
സന്തോഷസന്താനസന്താനമേ! ചിന്താമണേ!\\
ആരാലും കണ്ടുകൂടാതൊരു പാഷാണാംഗിയായ്\\
ഘോരമാം തപസ്സോടുമിവിടെ വസിക്കുന്ന\\
ബ്രഹ്മനന്ദനയായ ഗൗതമപത്നിയുടെ\\
കല്മഷമശേഷവും നിന്നുടെ പാദങ്ങളാല്‍\\
ഉന്മൂലനാശം വരുത്തീടണമിന്നുതന്നെ\\
നിര്‍മലയായ് വന്നീടുമഹല്യാദേവിയെന്നാല്‍.”\\
ഗാഥിനന്ദനന്‍ ദാശരഥിയോടേവം പറ്-\\
ഞ്ഞാശു തൃക്കയ്യും പിടിച്ചുടജാങ്കണം പുക്കാന്‍\\
ഉഗ്രമാം തപസ്സോടുമിരിക്കും ശിലാരൂപ-\\
മഗ്രേ കാണ്‍കെന്നു കാട്ടിക്കൊടുത്തു മുനിവരന്‍.\\
ശ്രീപാദാംബുജം മെല്ലെ വെച്ചിതു രാമദേവന്‍\\
ശ്രീപതി രഘുപതി സല്‍പ്പതി ജഗല്‍പ്പതി\\
രാമോഹമെന്നു പറഞ്ഞാമോദം പൂണ്ടു നാഥന്‍\\
കോമളരൂപന്‍ മുനിപത്നിയെ വണങ്ങിനാന്‍.\\
അന്നെരം നാഥന്‍തന്നെ കാണായിതഹല്യയ്ക്കും\\
വന്നൊരാനന്ദമേതും ചൊല്ലാവതല്ലയല്ലോ.\\
താപസശ്രേഷ്ഠനായ കൗശികമിനിയോടും\\
താപസഞ്ചയം നീങ്ങുമാറു സോദരനോടും\\
താപനാശനകരനായൊരു ദേവന്‍ തന്നെ-\\
ച്ചാപബ്ബാണങ്ങളോടും പീതമാം വസ്ത്രത്തോടും\\
ശ്രീവത്സവക്ഷസ്സോടും സുസ്മിത വക്ത്രത്തോടും\\
ശ്രീവാസാംബുജദലസന്നിഭനേത്രത്തോടും\\
വാസവനീലമണിസങ്കാശഗാത്രത്തോടും\\
വാസവാദ്യമരൗഘവന്ദിതപാദത്തോടും\\
പത്തുദിക്കിലുമൊക്കെ നിറഞ്ഞ കാന്തിയോടും\\
ഭക്തവത്സലന്‍തന്നെക്കാണായിതഹല്യയ്ക്കും\\
തന്നുടെ ഭര്‍ത്താവായ ഗൗതമതപോധനന്‍-\\
തന്നോടു മുന്നമുരചെയ്തതുമോര്‍ത്താളപ്പോള്‍.\\
നിര്‍ണയം നാരായണന്‍താനിതു ജഗന്നാഥ-\\
നര്‍ണോജവിലോചനന്‍ പത്മജാമനോഹരന്‍\\
ഇത്ഥമാത്മനി ചിന്തിച്ചുത്ഥാനംചെയ്തു ഭക്ത്യാ\\
സത്വരമര്‍ഘ്യാദികള്‍കൊണ്ടു പൂജിച്ചീടിനാള്‍.\\
സന്തോഷാശ്രുക്കളൊഴുകീടും നേത്രങ്ങളോടും\\
സന്താപം തീര്‍ന്നു ദണ്ഡനമസ്കാരവും ചെയ്താള്‍.\\
ചിത്തകാമ്പിങ്കലേറ്റം വര്‍ധിച്ച ഭക്തിയോടു-\\
മുത്ഥാനം ചെയ്തു മുഹുരഞ്ജലിബന്ധത്തോടും\\
വ്യക്തമായൊരു പുളകാഞ്ചിതദേഹത്തോടും\\
വ്യക്തമല്ലാതവന്ന ഗദ്ഗദവര്‍ണത്തോടും\\
അദ്വയനായൊരനാദ്യസ്വരൂപനെക്കണ്ടു\\
സദ്യോജാതാനന്ദാബ്ധിമഗ്നയായ് സ്തുതിചെയ്താള്‍.
\end{verse}

%%unit 8). ahalyaasthudhi

\section{അഹല്യാസ്തുതി}

\begin{verse}
ഞാനഹോ കൃതാര്‍ഥയായേന്‍ ജഗന്നാഥ! നിന്നെ-\\
“ക്കാണായ്വന്നതുമൂലമത്രയുമല്ല ചൊല്ലാം:\\
പത്മജരുദ്രാദികളാലപേക്ഷിതം പാദ-\\
പത്മസംലഗ്നപാംസുലേശമിന്നെനിക്കല്ലോ\\
സിദ്ധിച്ചു ഭവല്‍പ്രസാദാതിരേകത്താലതി-\\
ന്നെത്തുമോ ബഹുകല്പകാലമാരാധിച്ചാലും?\\
ചിത്രമെത്രയും തവ ചേഷ്ടിതം ജഗല്‍പതേ!\\
മര്‍ത്യഭാവേന വിമോഹിപ്പിച്ചീടുന്നിതേവം.\\
ആനന്ദമയനായോരതിമായികന്‍ പൂര്‍ണന്‍\\
ന്യൂനാതിരേകശൂന്യനചലനല്ലോ ഭവാന്‍.\\
ത്വല്‍പ്പാദാംബുജപാംസുപവിത്രാ ഭാഗീരഥി\\
സര്‍പ്പഭൂഷണവിരിഞ്ചാതികളെല്ലാരെയും\\
ശുദ്ധമാക്കീടുന്നതും ത്വല്‍പ്രഭാവത്താലല്ലോ\\
സിദ്ധിച്ചേനല്ലോ ഞാനും ത്വല്‍പ്പാദസ്പര്‍ശമിപ്പോള്‍.\\
പണ്ടു ഞാന്‍ ചെയ്ത പുണ്യമെന്തു വര്‍ണിപ്പതു വൈ-\\
കുണ്ഠ! തല്‍ക്കുണ്ഠാത്മനാം ദുര്‍ലഭമൂര്‍ത്തേ! വിഷ്ണോ!\\
മര്‍ത്ത്യനായവതരിച്ചോരു പൂരുഷം ദേവം\\
ചിത്തമോഹനം രമണീയദേഹിനം രാമം\\
ശുദ്ധമത്ഭുതവീര്യം സുന്ദരം ധനുര്‍ദ്ധരം\\
തത്ത്വമദ്വയം ഭജിച്ചീടുന്നേനിനി നിത്യം\\
ഭക്ത്യൈവ മറ്റാരെയും ഭജിച്ചീടുന്നേനില്ല.\\
യാതൊരു പാദാംബുജമാരായുന്നിതു വേദം\\
യാതൊരു നാഭിതന്നിലുണ്ടായി വിരിഞ്ചനും\\
യാതൊരു നാമം ജപിക്കുന്നിതു മഹാദേവെന്‍\\
ചേതസാ തത്സ്വാമിയെ ഞാന്‍ നിത്യം വണങ്ങുന്നേന്‍\\
നാരദമുനീന്ദ്രനും ചന്ദ്രശേഖരന്‍താനും\\
ഭാരതീ രമണനും ഭാരതീദേവിതാനും\\
ബ്രഹ്മലോകത്തിങ്കല്‍നിന്നന്വഹം കീര്‍ത്തിക്കുന്നു\\
കല്മഷഹരം രാമചരിതാം രസായനം\\
കാമരാഗദികള്‍ തീര്‍ന്നാനന്ദം വരുവാനായ്\\
രാമദേവനെ ഞാനും ശരണം പ്രാപിക്കുന്നേന്‍.\\
ആദ്യനദ്വയനേകനവ്യക്തനനാകുലന്‍\\
വേദ്യനല്ലാരാലുമെന്നാലും വേദാന്തവേദ്യന്‍\\
പരമന്‍ പരാപരന്‍ പരമാത്മാവു പരന്‍\\
പരബ്രഹ്മാഖ്യന്‍ പരമാനന്ദമൂര്‍ത്തി നാഥന്‍\\
പുരുഷന്‍ പുരാതനന്‍ കേവലസ്വയംജ്യോതി-\\
സ്സകലചരാചരഗുരു കാരുണ്യമൂര്‍ത്തി\\
ഭുവനമനോഹരമായൊരു രൂപം പൂണ്ടു\\
ഭുവനത്തിങ്കലനുഗ്രഹത്തെ വരുത്തുവാന്‍.\\
അങ്ങനെയുള്ള രാമചന്ദ്രനെസ്സദാകാലം\\
തിങ്ങിന ഭക്ത്യാ ഭജിച്ചീടുന്നേന്‍ മനസി ഞാന്‍\\
സ്വതന്ത്രന്‍ പരിപൂര്‍ണനാനന്ദനാത്മാരാമ-\\
നതന്ദ്രന്‍ നിജമായാഗുണബിംബിതനായി\\
ജഗദുത്ഭവസ്ഥിതിസംഹാരാദികള്‍ ചെയ്വാ-\\
നഖണ്ഡന്‍ ബ്രഹ്മവിഷ്ണുരുദ്രനാമങ്ങള്‍ പൂണ്ടു\\
ഭേദരൂപങ്ങള്‍ കൈക്കൊണ്ടൊരു നിര്‍ഗുണമൂര്‍ത്തി\\
വേദാന്തവേദ്യന്‍ മമ ചേതസി വസിക്കണം.\\
രാമ! രാഘവ! പാദപങ്കജം നമോസ്തു തേ\\
ശ്രീമയം ശ്രീദേവീപാണിദ്വയപത്മാര്‍ച്ചിതം.\\
മാനഹീനന്മാരാം ദിവ്യന്മാരാലനുധ്യേയം\\
മാനാര്‍ഥം മൂന്നിലകമാക്രാന്ത ജഗത്രയം.\\
ബ്രഹ്മാവിന്‍ കരങ്ങളാല്‍ ക്ഷളിതം പത്മോപമം\\
നിര്‍മലം ശംഖചക്രകുലിശ മത്സാങ്കിതം\\
മന്മനോനികേതനം കല്മഷവിനാശനം\\
നിര്‍മലാത്മാനാം പരമാസ്പദം നമോസ്തു തേ\\
ജഗദാശ്രയം ഭവാന്‍ ജഗത്തായതും ഭവാന്‍\\
ജഗതാമാദിഭൂതനായതും ഭവാനല്ലോ.\\
സര്‍വഭൂതങ്ങളിലുമസക്തനല്ലോ ഭവാന്‍\\
നിര്‍വികാരാത്മാ സാക്ഷിഭൂതനായതും ഭവാന്‍\\
അജനവ്യയന്‍ ഭവാനജിതന്‍ നിരഞ്ജനന്‍\\
വചസാം വിഷയമല്ലാതൊരാനന്ദമല്ലോ.\\
വാച്യവാചകോഭയഭേദേന ജഗന്മയന്‍\\
വാച്യനായ് വരേണമേ വാക്കിനു സദാ മമ.\\
കാര്യകാരണകര്‍ത്തൃഫലസാധനഭെദം\\
മായയാ ബഹുവിധരൂപയാ തോന്നിക്കുന്നു.\\
കേവലമെന്നാകിലും നിന്തിരുവടിയതു\\
സേവകന്മാര്‍ക്കുപോലുമറിവാനരുതല്ലോ.\\
ത്വന്മായാവിമോഹിതചേതസാമജ്ഞാനിനാം\\
ത്വന്മാഹാത്മ്യങ്ങള്‍ നേരേയറിഞ്ഞുകൂടായല്ലോ:\\
മാനസേ ചിശ്വാത്മാവാം നിന്തിരുവടിതന്നെ\\
മാനുഷനെന്നു കല്പിച്ചീടുവോരജ്ഞാനികള്‍.\\
പുറത്തുമകത്തുമെല്ലാടവുമൊക്കെ നിറ-\\
ഞ്ഞിരിക്കുന്നതു നിത്യം നിന്തിരുവടിയല്ലോ.\\
ശുദ്ധനദ്വയന്‍ സമന്‍ നിത്യന്‍ നിര്‍മലനേകന്‍\\
ബുദ്ധനവ്യക്തന്‍ ശാന്തനസംഗന്‍ നിരാകാരന്‍\\
സത്വാദിഗുണത്രയയുക്തയാം ശക്തിയുക്തന്‍\\
സത്വങ്ങളുള്ളില്‍ വാഴും ജീവാത്മാവായ നാഥന്‍\\
ഭക്താനാം മുക്തിപ്രദന്‍ യുക്താനാം യോഗപ്രദന്‍\\
സക്താനാം ഭുക്തിപ്രദന്‍ സിദ്ധാനാം സിദ്ധിപ്രദന്‍\\
തത്ത്വാധാരാത്മാദേവന്‍ സകലജഗന്മയന്‍\\
തത്ത്വജ്ഞന്‍ നിരുപമന്‍ നിഷ്കളന്‍ നിരഞ്ജനന്‍\\
നിര്‍ഗുണന്‍ നിശ്ചഞ്ചലന്‍ നിര്‍മലന്‍ നിരാധാരന്‍\\
നിഷ്ക്രിയന്‍ നിഷ്കാരണന്‍ നിരഹങ്കാരന്‍ നിത്യന്‍\\
സത്യജ്ഞാനാനന്താനന്ദാമൃതാത്മകന്‍ പരന്‍\\
സത്താമാത്രാത്മാ പരമാത്മാ സര്‍വാത്മാ വിഭു\\
സച്ചിദ്ബ്രഹ്മാത്മാ സമസ്തേശ്വരന്‍ മഹേശ്വര-\\
നച്യുതനാദിനാഥന്‍ സര്‍വദേവതാമന്‍\\
നിന്തിരുവടിയായതെത്രയും മൂഢാത്മാവാ-\\
യന്ധയായുള്ളോരു ഞാനെങ്ങനെയറിയുന്നു\\
നിന്തിരുവടിയുടെ തത്ത്വമെന്നാലും ഞാനോ\\
സന്തതം ഭൂയോ ഭൂയോ നമസ്തേ നമോ നമഃ\\
യത്രകുത്രാപി വസിച്ചീടിലുമെല്ലാനാളും\\
പൊല്‍ത്തളിരടികളിലിളക്കം വരാതൊരു\\
ഭക്തിയുണ്ടാകവേണമെന്നൊഴിഞ്ഞപരം ഞാ-\\
നര്‍ഥിച്ചീടുന്നേനില്ലനമസ്തേ നമോ നമഃ\\
നമസ്തേ രാമ! രാമ! പുരുഷാധ്യക്ഷ! വിഷ്ണോ!\\
നമസ്തേ രാമ! രാമ! ഭക്തവത്സല! രാമ!\\
നമസ്തേ ഹൃഷീകേശ! രാമ! രാഘവ! രാമ!\\
നമസ്തേ നാരായണ! സന്തതം നമോസ്തു തേ!\\
സമസ്തകര്‍മാര്‍പ്പണം ഭവതി കരോമി ഞാന്‍\\
സമസ്തമപരാധം ക്ഷമസ്വ ജഗല്‍പ്പതേ!\\
ജനനമരണദുഃഖാപഹം ജഗന്നാഥം\\
ദിനനായകകോടിസദൃശപ്രഭം രാമം\\
കരസാരസയുഗസുധൃതശരചാപം\\
കരുണാകരം കാളജലദഭാസം രാമം\\
കനകരുചിരദിവ്യാംബരം രമാവരം\\
കനകോജ്ജ്വലരത്നകുണ്ഡലാഞ്ചിതഗണ്ഡം\\
കമലദളലോലവിമലവിലോചനം\\
കമലോദ്ഭവനതം മനസാ രാമമീഡേ\\
പുരതഃസ്ഥിതം സാക്ഷാദീശ്വരം രഘുനാഥം\\
പുരുഷോത്തമം കൂപ്പി സ്തുതിച്ചാള്‍ ഭക്തിയോടെ\\
ജോകേശാത്മജയാകുമഹല്യതാനും പിന്നെ\\
ലോകേശ്വരാനുജ്ഞയാ പോയിതു പവിത്രയായ്.\\
ഗൗതമനായ തന്റെ പതിയെ പ്രാപിച്ചുട-\\
നാധിയും തീര്‍ത്തു വസിച്ചീടിനാളഹല്യയും.\\
ഇസ്തുതി ഭക്തിയോടേ ജപിച്ചീടുന്ന പുമാന്‍\\
ശുദ്ധനായഖിലപാപങ്ങളും നശിച്ചുടന്‍\\
പരമം ബ്രഹ്മാനന്ദം പ്രാപിക്കുമത്രയല്ല\\
വരുമൈഹികസൗഖ്യം പുരുഷന്മാര്‍ക്കു നൂനം.\\
ഭക്ത്യാ നാഥനെ ഹൃദി സന്നിധാനംചെയ്തുകൊ-\\
ണ്ടിസ്തുതി ജപിച്ചീടില്‍ സാധിക്കും സകലവും.\\
പുത്രാര്‍ഥി ജപിക്കില്ലോ നല്ല പുത്രന്മാരുണ്ടാ-\\
മര്‍ഥാര്‍ഥി ജപിച്ചീടിലര്‍ഥവുമേറ്റമുണ്ടാം.\\
ഗുരുതല്പഗന്‍ കനകസ്തേയി സുരാപായി\\
ധരണീസുരഹന്താ പിതൃമാതൃഹാ ഭോഗി\\
പുരുഷാധമനേറ്റമെങ്കിലുമവന്‍ നിത്യം\\
പുരുഷോത്തമം ഭക്തവത്സലം നാരായണം\\
ചേതസി രാമചന്ദ്രം ധ്യാനിച്ചു ഭക്ത്യാ ജപി-\\
ച്ചാദരാല്‍ വണങ്ങുകില്‍ സാധിക്കുമല്ലോ മോക്ഷം\\
സദ്വൃത്തനെന്നായീടില്‍ പറയേണമോ മോക്ഷം\\
സദ്യസ്സംഭവിച്ചീടും സന്ദേഹമില്ലയേതും.
\end{verse}

%unit 9). seethaasvayamvaram

\section{സീതാസ്വയംവരം}

\begin{verse}
വിശ്വാമിത്രനുംപരമാനന്ദം പ്രാപിച്ചപ്പോള്‍\\
വിശ്വനായകന്‍ തന്നോടീവണ്ണമരുള്‍ചെയ്താന്‍;\\
“ബാലകന്മാരേ! പോക മിഥിലാപുരിക്കു നാം\\
കാലവും വൃഥാ കളഞ്ഞീടുകയരുതല്ലോ.\\
യാഗവും മഹാദേവചാപവും കണ്ടു പിന്നെ\\
വേഗമോടയോദ്ധ്യയും പുക്കു താതനെക്കാണാം.”\\
ഇത്തരമരുള്‍ചെയ്തു ഗംഗയും കടന്നവര്‍\\
സത്വരം ചെന്നു മിഥിലാപുരമകം പുക്കു.\\
മുനിനായകനായ കൗശികന്‍ വിശ്വാമിത്രന്‍\\
മിനിവാടം പ്രാപിച്ചിതെന്നതു കേട്ടനേരം\\
മനസി നിറഞ്ഞൊരു പരമാനന്ദത്തോടും\\
ജനകമഹീപതി സംഭ്രമസമന്വിതം\\
പൂജാസാധനങ്ങളുമെടുത്തു ഭക്തിയോടു-\\
മാചാര്യനോടുമൃഷിവാടം പ്രാപിച്ചനേരം\\
ആമോദപൂര്‍വം പൂജിച്ചാചാരംപൂണ്ടു നിന്ന\\
രാമലക്ഷ്മണന്മാരെക്കാണായി നൃപേന്ദ്രനും\\
ചന്ദ്രസൂര്യന്മാരെന്നപോലെ ഭൂപാലേശ്വര-\\
നന്ദനന്മാരെക്കണ്ടു ചോദിച്ചു നൃപേന്ദ്രനും:\\
“കന്ദര്‍പ്പന്‍ കണ്ടു വന്ദിച്ചീടിന ജഗദേക-\\
സുന്ദരന്മാരാമിവരാരെന്നു കേള്‍പ്പിക്കേണം\\
നരനാരായണന്മാരാകിയ മൂര്‍ത്തികളോ\\
നരവീരാകാരം കൈക്കൊണ്ടു കാണായതിപ്പോള്‍?\\
വിശ്വാമിത്രനുമതുകേട്ടരുള്‍ചെയ്തീടിനാന്‍:\\
“വിശ്വസിച്ചാലും മമ വാക്യം നീ നരപതേ!\\
വീരനാം ദശരഥന്‍തന്നുടെ പുത്രന്മാരില്‍\\
ശ്രീരാമന്‍ ജ്യേഷ്ഠനിവന്‍ ലക്ഷ്മണന്‍ മൂന്നാമവന്‍\\
എന്നുടെ യാഗം രക്ഷിച്ചീടുവാനിവരെ ഞാന്‍\\
ചെന്നു കൂട്ടിക്കൊണ്ടുപോന്നീടിനേനിതുകാലം\\
കാടകം പുക്കനേരം വന്നൊരു നിശാചരി\\
താടകതന്നെയൊരു ബാണംകൊണ്ടെയ്തുകൊന്നാന്‍\\
പേടിയും തീര്‍ന്നു സിദ്ധാശ്രമവും പുക്കു യാഗ-\\
മാടല്‍ക്കൂടാതെ രക്ഷിച്ചീടിനാന്‍ വഴിപോലെ.\\
ശ്രീപാദാംബുജരജഃസ്പൃഷ്ടികൊണ്ടഹല്യതന്‍\\
പാപവും നശിപ്പിച്ചു പാവനയാക്കീടിനാന്‍\\
പാരമേശ്വരമായ ചാപത്തെക്കാണ്മാനുള്ളില്‍\\
പാരമാഗ്രഹമുണ്ടു നീയതു കാട്ടീടേണം.”\\
ഇത്തരം വിശ്വാമിത്രന്‍ തന്നുടെ വാക്യം കേട്ടു\\
സത്വരം ജനകനും പൂജിച്ചു വഴിപോലെ\\
സല്‍ക്കാരയോഗ്യന്മാരാം രാജപുത്രന്മാരെക്ക\\
ണ്ടുള്‍ക്കുരുന്നിങ്കല്‍ പ്രീതി വര്‍ധിച്ച ജനകനും\\
തന്നുടെ സചിവനെ നിളിച്ചു നിയോഗിച്ചു;\\
‘ചെന്നു നീ വരുത്തേണമീശ്വരനുടെ ചാപം’\\
എന്നതുകേട്ടു മന്ത്രിപ്രവന്‍ നടകൊണ്ടാ-\\
നന്നേരം ജനകനും കൗശികനോടു ചൊന്നാന്‍:\\
രാജനന്ദനനായ ബാലകന്‍ രഘുവരന്‍\\
രാജീവവിലോചനന്‍ സുന്ദരന്‍ ദാശരഥി\\
വില്ലിതു കുലച്ചുടന്‍ വലിച്ചുമുറിച്ചീടില്‍\\
വല്ലഭവനിവന്‍ മമ നന്ദനയ്ക്കെന്നു നൂനം’\\
‘എല്ലാമീശ്വരനെന്നേ ചൊല്ലാവിതെനിക്കിപ്പോള്‍\\
വില്ലിഹ വരുത്തീടു’കെന്നരുള്‍ചെയ്തു മുനി.\\
കിങ്കരന്മാരെ നിയോഗിച്ചിതു മന്ത്രീന്ദ്രനും\\
ഹുങ്കാരത്തോടെ വന്നു ചാപവാഹകന്മാരും\\
സത്വരമയ്യായിരം കിങ്കരന്മാരും കൂടി\\
മൃത്യുശാസനചാപമെടുത്തുകൊണ്ടുവന്നാര്‍.\\
‘ഘണ്ടാസഹസ്രമണി വസ്ത്രാദി വിഭൂഷിതം\\
കണ്ടാലും ത്രൈയംബക’മെന്നിതു മന്ത്രീന്ദ്രനും\\
ചന്ദ്രശേഖരനുടെ പള്ളീവില്‍ കണ്ടു രാമ-\\
ചന്ദ്രനുമാനന്ദമുള്‍ക്കൊണ്ടു വന്ദിച്ചീടിനാന്‍\\
‘വില്ലെടുക്കാമോ? കുലച്ചീടാമോ? വലിക്കാമോ?\\
ചൊല്ലുകെ’ന്നതു കേട്ടു ചൊല്ലിനാന്‍ വിശ്വാമിത്രന്‍.\\
എല്ലാമാ,മാകുന്നതു ചെയ്താലും മടിക്കേണ്ട\\
കല്യാണമിതുമൂലം വന്നുകൂടീടുമല്ലോ.’\\
മന്ദഹാസവും പൂണ്ടു രാഘാവനതു കേട്ടു\\
മന്ദംമന്ദംപോയ് ചെന്നുനിന്നു കണ്ടിതു ചാപം\\
ജ്വലിച്ച തേജസ്സോടുമെടുത്തു വേഗത്തോടെ\\
കുലച്ചുവലിച്ചുടന്‍ മുറിച്ചു ജിതശ്രമം\\
നിന്നരുളുന്നനേരമീരേഴു ലോകങ്ങളു-\\
മൊന്നു മാറ്റൊലിക്കൊണ്ടു, വിസ്മയപ്പെട്ടു ജനം.\\
പാട്ടുമാട്ടവും കൂത്തും പുഷ്പവൃഷ്ടിയുമോരോ\\
കൂട്ടമേ വാദ്യങ്ങളും മംഗലസ്തുതികളും\\
ദേവകളൊക്കെപ്പരമാനന്ദംപൂണ്ടു ദേവ-\\
ദേവനെസ്സേവിക്കയുമപ്സരസ്ത്രീകളെല്ലാം\\
ഉത്സാഹം കൈക്കൊണ്ടു വിശ്വേശ്വരനുടെ വിവാ-\\
ഹോത്സവാരംഭഘോഷം കണ്ടു കൗതുകം പൂണ്ടാര്‍.\\
ജനകന്‍ ജഗത്സ്വാമിയാകിയ ഭഗവാനെ-\\
ജ്ജനസംസദി ഗാഢാശ്ലേഷവും ചെയ്താനല്ലോ...\\
ഇടിവെട്ടീടുംവണ്ണം വില്‍മുറിഞ്ഞൊച്ച കേട്ടു\\
നടുങ്ങി രാജാക്കന്മാരുരഗങ്ങളെപ്പോലെ\\
മൈഥിലി മയില്‍പ്പേടപോലെ സന്തോഷം പൂണ്ടാള്‍\\
കൗതുകമുണ്ടായ്വന്നു ചേതസി കൗശികനും\\
മൈഥിലിതന്നെപ്പരിചാരികമാരും നിജ-\\
മാതാക്കന്മാരുംകൂടി നന്നായിച്ചമയിച്ചാര്‍.\\
സ്വര്‍ണവര്‍ണത്തെപ്പൂണ്ട മൈഥിലി മനോഹരി\\
സ്വര്‍ണഭൂഷണങ്ങളുമണിഞ്ഞു ശോഭയോടെ\\
സ്വര്‍ണമാലയും ധരിച്ചാദരാല്‍ മന്ദംമന്ദ-\\
മര്‍ണോജനേത്രന്‍ മുമ്പില്‍ സത്രപം വിനീതയായ്\\
വന്നുടന്‍ നേത്രോല്പലമാലയുമിട്ടാള്‍ മുന്നേ\\
പിന്നാലെ വരണാര്‍ഥമാലയുമിട്ടീടിനാള്‍\\
മാലയും ധരിച്ചു നീലോല്പലകാന്തി തേടും\\
ബാലകന്‍ ശ്രീരാമനുമേറ്റവും വിളങ്ങിനാന്‍.\\
ഭുമിനന്ദനയ്ക്കനുരൂപനായ് ശോഭിച്ചീടും\\
ഭൂമിപാലകബാലന്‍തന്നെക്കണ്ടവര്‍കളും\\
ആനന്ദാംബുധിതന്നില്‍ വീണുടന്‍ മുഴുകിനാര്‍-\\
മാനവവീരന്‍ വാഴ്കെന്നാശിയും ചൊല്ലീടിനാര്‍.\\
അന്നേരം വിശ്വാമിത്രന്‍തന്നോടു ജനകനും\\
വന്ദിച്ചുചൊന്നാ’നിനിക്കാലത്തെക്കളയാതെ\\
പത്രവും കൊടുത്തയച്ചീടേണം ദൂതന്മാരെ-\\
സ്സത്വരം ദശരഥഭൂപനെ വരുത്തുവാന്‍.’\\
വിശ്വാമിത്രനും മിഥിലാധിപന്‍താനും കൂടി\\
വിശ്വാസം ദശരഥന്‍തനിക്കു വരുംവണ്ണം\\
നിശ്ശേഷവൃത്താന്തങ്ങളെഴുതിയയച്ചിതു\\
വിശ്രമത്തോടു നടകൊണ്ടിതു ദൂതന്മാരും\\
സാകേതപുരിപുക്കു ഭൂപാലന്‍തന്നെക്കണ്ടു\\
ലോകൈകാധിപന്‍ കൈയില്‍ കൊടുത്തു പത്രമതും\\
സന്ദേശം കണ്ടു പങ് ക്തിസ്യന്ദനന്‍താനുമിനി-\\
സ്സന്ദേഹമില്ല പുറപ്പെടുകെന്നുരചെയ്തു.\\
അഗ്നിമാനുപാധ്യായനാകിയ വസിഷ്ഠനും\\
പത്നിയാമരുന്ധതിതാനുമായ് പുറപ്പെട്ടു.\\
കൗതുകംപൂണ്ടു ചതുരംഗവാഹിനിയോടും\\
കൗസല്യാദികളായ ഭാര്യമാരോടും കൂടി\\
ഭരതശത്രുഘ്നന്മാരാകിയ പുത്രന്മാരും\\
പരമോത്സവയോഗ്യവാദ്യഘോഷങ്ങളോടും\\
മിഥിലാപുരമകംപുക്കിതു ദശരഥന്‍\\
മിഥിലാധിപന്‍താനും ചെന്നെതിരേറ്റുകൊണ്ടാന്‍.\\
വന്ദിച്ചു ശതാനന്ദന്‍തന്നോടും കൂടെച്ചെന്നു\\
വന്ദ്യനാം വസിഷ്ഠനെത്തദനുപത്നിയേയും\\
അര്‍ഘ്യപാദ്യാദികളാലര്‍ച്ചിച്ചു യഥാവിധി\\
സല്‍കരിച്ചിതു യഥായോഗ്യമുര്‍വീന്ദ്രന്‍താനും.\\
രാമലക്ഷ്മണന്മാരും വന്ദിച്ചു പിതാവിനെ-\\
സാമോദം വസിഷ്ഠനാമാചാര്യപാദാബ്ജവും\\
തൊഴുതു മാത്രുജനങ്ങളെയും യഥാക്രമം\\
തൊഴുതു ശ്രീരാമപാദാംഭോജമനുജന്മാര്‍\\
തൊഴുതു ഭരതനെ ലക്ഷ്മണകുമാരനും\\
തൊഴുതു ശത്രുഘ്നനും ലക്ഷ്മണപാദാംഭോജം.\\
വക്ഷസി ചേര്‍ത്തു താതന്‍ രാമനെപ്പുണര്‍ന്നിട്ടു\\
ലക്ഷ്മണനെയും ഗാഢാശ്ലേഷവും ചെയ്തീടിനാന്‍.\\
ജനകന്‍ ദശരഥന്‍തന്നുടെ കൈയും പിടി-\\
ച്ചനുമോദത്തോടുരചെയ്തിതു മധുരമായ്:\\
‘നാലു കന്യകമാരുണ്ടെനിക്കു കൊടുപ്പാനായ്\\
നാലു പുത്രന്മാര്‍ ഭാവാന്‍തനിക്കുണ്ടല്ലോതാനും\\
ആകയാല്‍ നാലു കുമാരന്മാര്‍ക്കും വിവാഹം ചെ-\\
യ്താകിലോ നിരൂപിച്ചാലേതുമേ മടിക്കേണ്ട.’\\
വസിഷ്ഠന്‍താനും ശതാനന്ദനും കൗശികനും\\
വിധിച്ചു മുഹൂര്‍ത്തവും നാല്വര്‍ക്കും യഥാക്രമം\\
ചിത്രമായിരിപ്പൊരു മണ്ഡപമതും തീര്‍ത്തു\\
മുത്തുമാലകള്‍ പുഷ്പഫലങ്ങള്‍ തൂക്കി നാനാ-\\
രത്നമണ്ഡിതസ്തംഭതോരണങ്ങളും നാട്ടി\\
രത്നകമണ്ഡിതസ്വര്‍ണപീഠവും വെച്ചു ഭക്ത്യാ,\\
ശ്രീരാമപാദാംബോജം കഴുകിച്ചനന്തരം\\
ഭേരിദുന്ദുഭിമുഖ്യവാദ്യഘോഷങ്ങളോടും\\
ഹോമവും കഴിച്ചു തന്‍പുത്രിയാം വൈദേഹിയെ\\
രാമനു നല്‍കീടിനാന്‍ ജനകമഹീന്ദ്രനും.\\
തല്‍പാദതീര്‍ഥം നിജശിരസി ധരിച്ചുട-\\
നുള്‍പ്പുളകാംഗത്തോടെ നിന്നിതു ജനകനും\\
യാതൊരു പാദതീര്‍ഥം ശിരസി ധരിക്കുന്നു\\
ഭൂതേശവിധിമുനീന്ദ്രാദികള്‍ ഭക്തിയോടെ.\\
ഊര്‍മിളതന്നെ വേട്ടു ലക്ഷ്മണകുമാരനും\\
കാമ്യാംഗിമാരാം ശ്രുതകീര്‍ത്തിയും മാണ്ഡവിയും\\
ഭരതശത്രുഘ്നന്മാര്‍തമ്മുടെ പത്നിമാരായ്\\
പരമാനന്ദം പൂണ്ടു വസിച്ചാരെല്ലാവരും.\\
കുശികാത്മജനോടും വസിഷ്ഠനോടും കൂടി\\
വിശദസ്മിതപൂര്‍വം പറഞ്ഞു ജനകനും:\\
‘മുന്നം നാരദനരുള്‍ചെയ്തു കേട്ടിരിപ്പു ഞാ-\\
നെന്നുടെ മകളായ സീതാവൃത്താന്തമെല്ലാം\\
യാഗഭൂദേശം വിശുദ്ധ്യര്‍ഥമായുഴുതപ്പോ-\\
ളേകദാ സീതാമധ്യേ കാണായി കന്യാരത്നം.\\
ജാതയായൊരു ദിവ്യകന്യകതനിക്കു ഞാന്‍\\
സീതയെന്നൊരു നാമം വിളിച്ചേനതുമൂലം.\\
പുത്രിയായ് വളര്‍ത്തു ഞാനിരിക്കും കാലത്തിങ്ക-\\
ലത്ര നാരദനെഴുന്നള്ളിനാനൊരു ദിനം.\\
എന്നോടു മഹാമുനിതാനരുള്‍ചെയ്താനപ്പോള്‍:\\
‘നിന്നുടെ മകളായ സീതാവൃത്താന്തം കേള്‍ നീ.\\
പരമാനന്ദമൂര്‍ത്തി ഭഗവാന്‍ നാരായണന്‍\\
പരമാത്മാവാമജന്‍ ഭക്തവത്സലന്‍ നാഥന്‍\\
ദേവകാര്യാര്‍ഥം പങ്തികണ്ഠനിഗ്രഹത്തിനായ്\\
ദേവേന്ദ്രവിരിഞ്ചരുദ്രാദികളര്‍ഥിക്കയാല്‍\\
ഭൂമിയില്‍ സൂര്യാന്വയേ വന്നവതരിച്ചിതു\\
രാമനായ് മായാമര്‍ത്ത്യവേഷംപൂണ്ടറിഞ്ഞാലും.\\
യോഗേശന്‍ മനുഷ്യനായിടുമ്പോളിതുകാലം\\
യോഗമായാദേവിയും മാനുഷവേഷത്തോടെ\\
ജാതയായിതു തവ വേശ്മനി തല്‍ക്കാരണാല്‍\\
സാദരം ശ്രീരാമനു കൊടുക്ക മടിയാതെ.’\\
ഇത്ഥം നാരദനരുളിച്ചെയ്തു മറഞ്ഞിതു\\
പുത്രിയായ് വളര്‍ത്തിതു ഭക്തികൈക്കൊണ്ടു ഞാനും.\\
സീതയെ ശ്രീരാഘവനെങ്ങനെ കൊടുക്കാവൂ!\\
ചേതസി നിരൂപിച്ചാലെങ്ങനെയറിയുന്നു?\\
എന്നതോര്‍ത്തിരിക്കുമ്പോളൊന്നു മാനസേ തോന്നി\\
പന്നഗവിഭൂഷണന്‍ തന്നനുഗ്രഹശക്ത്യാ.\\
മൃത്യുശാസനചാപം മുറിച്ചീടുന്ന പുമാന്‍\\
ഭര്‍ത്താവാകുന്നതു മല്‍പുത്രിക്കെന്നൊരു പണം\\
ചിത്തത്തില്‍ നിരൂപിച്ചു വരുത്തി നൃപന്മാരെ\\
ശക്തിയില്ലിതിനെന്നു പൃഥ്വീപാലകന്മാരും\\
ഉദ്ധതഭാവമെല്ലാമകലെക്കളഞ്ഞുടന്‍\\
ബുദ്ധിയും കെട്ടു പോയങ്ങടങ്ങിക്കൊണ്ടാരല്ലോ.\\
അത്ഭുതപുരുഷനാമുല്പലനേത്രന്‍ തന്നെ\\
ത്വല്‍പ്രസാദത്താലിന്നുസിദ്ധിച്ചേന്‍ ഭാഗ്യവശാല്‍\\
ദര്‍പ്പകസമനായ ചില്‍പ്പുരുഷനെ നോക്കി\\
പില്പാടു തെളിഞ്ഞുരചെയ്തിതു ജനകനും:\\
‘അദ്യ മേ സഫലമായ്വന്നു മാനുഷജന്മം\\
ഖദ്യോതായതസഹസ്രോദ്ദ്യോതരൂപത്തൊടും\\
ഖദ്യോതാന്വയേ പിറന്നൊരു നിന്തിരുവടി\\
വിദ്യുത്സംയുതമായ ജീമൂതമെന്നപോലെ\\
ശക്തിയാം ദേവിയോടും യുക്തനായ് കാണ്‍കമൂലം\\
ഭക്തവത്സല! മമ സിദ്ധിച്ചു മനോരഥം.’\\
രക്തപങ്കജചരണാഗ്രേ സന്തതം മമ\\
ഭക്തിസംഭവിക്കേണം മുക്തിയും ലഭിക്കേണം\\
ത്വല്‍പ്പാദാംബുജഗളിതാംബുബിന്ദുക്കള്‍ ധരി-\\
ച്ചുല്പലോത്ഭവന്‍ ജഗത്തൊക്കവേ സൃഷ്ടിക്കുന്നു;\\
ത്വല്‍പ്പാദാംബുജഗളിതാംബുധാരണംകൊണ്ടു\\
സര്‍പ്പഭൂഷണന്‍ ജഗത്തൊക്കെസ്സംഹരിക്കുന്നു;\\
ത്വല്‍പ്പാദാംബുജഗളിതാംബുധാരണംകൊണ്ടു\\
സല്‍പ്പുമാന്‍ മഹാബലി സിദ്ധിച്ചാനൈന്ദ്രം പദം.\\
ത്വല്‍പ്പാദാംബുജരജഃസ്പൃഷ്ടികൊണ്ടഹല്യയും\\
കില്ബിഷത്തോടു വേറുപെട്ടു നിര്‍മലയായാള്‍.\\
‘നിന്തിരുവടിയുടെ നാമകീര്‍ത്തനംകൊണ്ടു\\
ബന്ധവുമകന്നു മോക്ഷത്തെയും പ്രാപിക്കുന്നു\\
സന്തതം യോഗസ്ഥന്മാരാകിയ മുനീന്ദ്രന്മാര്‍;\\
ചിന്തിക്കായ്വരേണമേ പാദപങ്കജദ്വയം.’\\
ഇത്ഥമോരോന്നേ ചൊല്ലി സ്തുതിച്ചു ജനകനും\\
ഭക്തികൈക്കൊണ്ടു കൊടുത്തീടിനാന്‍ മഹാധനം.\\
കരികളറുനൂറും പതിനായിരം തേരും\\
തുരഗങ്ങളെയും നല്കീടിനാന്‍ നൂറായിരം\\
പത്തിയുമൊരു ലക്ഷം മുന്നൂറു ദാസികളും\\
വസ്ത്രങ്ങള്‍ ദിവ്യങ്ങളായുള്ളതും ബഹുവിധം.\\
മുത്തുമാലകള്‍ ദിവ്യരത്നങ്ങള്‍ പലതരം\\
പ്രത്യേഗം നൂറുകോടിക്കാഞ്ചനഭാരങ്ങളും\\
സീതാദേവിക്കു കൊടുത്തീടിനാന്‍ ജനകനും\\
പ്രീതി കൈക്കൊണ്ടു പരിഗ്രഹിച്ചു രാഘവനും.\\
വിധിനന്ദനപ്രമുഖന്മാരാം മുനികളെ\\
വിധിപൂര്‍വകം ഭക്ത്യാ പൂജിച്ചു വണങ്ങിനാന്‍.\\
സമ്മാനിച്ചിതു സുമന്ത്രാദിമന്ത്രികളെയും\\
സമ്മോദംപൂണ്ടു ദശരഥനും പുറപ്പെട്ടു.\\
കല്മഷമകന്നൊരു ജനകനൃപേന്ദ്രനും\\
തന്മകളായ സീതതന്നെയുമാശ്ലേഷിച്ചു\\
നിര്‍മ്മലഗാത്രിയായ പുത്രിക്കു പതിവ്രതാ-\\
ധര്‍മങ്ങളെല്ലാമുപദേശിച്ചു വഴിപോലെ.\\
ചിന്മയന്‍ മായാമയനായ രാഘവന്‍ നിജ-\\
ധര്‍മ്മദാരങ്ങളോടും കൂടവേ പുറപ്പെട്ടു.\\
മൃദംഗാനകഭേരീതൂര്യഘോഷങ്ങളോടും\\
മൃദുനാദങ്ങള്‍ തേടും വീണയും കുഴലുകള്‍\\
ശ്രൃംഗകാഹളങ്ങളും വദ്ദളമിടയ്ക്കകള്‍\\
ശ്രൃംഗാരരസപരിപൂര്‍ണവേഷങ്ങളോടും\\
ആന തേര്‍ കുതിര കാലാളായ പടയോടു-\\
മാനന്ദമോടും പിതൃമാതൃഭ്രാതാക്കളോടും\\
കൗശികവസിഷ്ഠാദി താപസേന്ദ്രന്മാരായ\\
ദേശികന്മാരോടും ഭൃത്യാമാത്യാദികളോടും\\
വേഗമോടയോധ്യയ്ക്കാമ്മാറങ്ങു തിരിച്ചപ്പോ-\\
ളാകാശദേശേ വിമാനങ്ങളും നിറഞ്ഞുതേ.\\
സന്നാഹത്തോടു നടന്നീടുമ്പോള്‍ ജനകനും\\
പിന്നാലെ ചെന്നു യാത്രയയച്ചോരനന്തരം\\
വെണ്‍കൊറ്റക്കുട തഴവെണ്‍ചാമരങ്ങളോടും\\
തിങ്കള്‍മണ്ഡലംതൊഴുമാലവട്ടങ്ങളോടും\\
ചെങ്കൊടിക്കൂറകള്‍കൊണ്ടങ്കിതധ്വജങ്ങളും\\
കുങ്കുമമലയജകസ്തൂരീഗന്ധത്തോടും\\
നടന്നു വിരവോടു മൂന്നു യോജന വഴി\\
കടന്നനേരം കണ്ടു ദുര്‍ന്നിമിത്തങ്ങളെല്ലാം.
\end{verse}

%%unit - 10). bhaargavadarpashamanam

\section{ഭാര്‍ഗവദര്‍പ്പശമനം}

\begin{verse}
അന്നേരം വസിഷ്ഠനെ വന്ദിച്ച് ദശരഥന്‍\\
‘ദുര്‍ന്നിമിത്തങ്ങാളുടെ കാരണം ചൊല്ലുകെ’ന്നാന്‍\\
‘മന്നവ! കുറഞ്ഞോരു ഭീതിയുണ്ടാകുമിപ്പോള്‍\\
പിന്നേടമഭയവുമുണ്ടാമെന്നറിഞ്ഞാലും.\\
ഏതുമേ പേടിക്കേണ്ട നല്ലതേ വന്നുകൂടൂ\\
ഖേദവുമുണ്ടാദേണ്ട കീര്‍ത്തിയും വര്‍ധിച്ചീടും.’\\
ഇത്തരം വിധിസുതനരുളിച്ചെയ്യുന്നേരം\\
പദ്ധതിമധ്യേ കാണായ്വന്നു ഭാര്‍ഗവനെയും.\\
നീലനീരദനിഭാനിര്‍മലവര്‍ണത്തോടും\\
നീല ലോഹിതശിഷ്യന്‍ ബഡവാനലസമന്‍\\
ക്രുദ്ധനായ് പരശുബാണാസനങ്ങളും പൂണ്ടു\\
പദ്ധതിമധ്യേ വന്നു നിന്നപ്പോള്‍ ദശരഥന്‍\\
ബദ്ധസാധ്വസം വീണു നമസ്കാരവും ചെയ്താന്‍\\
ബുദ്ധിയും കെട്ടു നിന്നു മറ്റുള്ള ജനങ്ങളും.\\
ആര്‍ത്തനായ് പങ്ക്തിരഥന്‍ ഭാര്‍ഗവരാമന്‍തന്നെ-\\
പ്പേര്‍ത്തു വന്ദിച്ചു ഭക്ത്യാ കീര്‍ത്തിച്ചാന്‍ പലതരം:\\
‘കാര്‍ത്തവീര്യാരേ! പരിത്രാഹി കാരുണ്യാംബുധേ!\\
മാര്‍ത്താണ്ഡകുലം പരിത്രാഹി മാം ജമദഗ്നി-\\
പുത്ര! മാം പരിത്രാഹി രേണുകാത്മജ! വിഭോ!\\
പരശുപാണേ! പരിമാലയ കുലം മമ\\
പരമേശ്വരപ്രിയ! പരിപാലയ നിത്യം.\\
പാര്‍ത്ഥിവസമുദായരക്തതീര്‍ത്ഥ്ത്തില്‍ കുളി-\\
ച്ചാസ്ഥായാ പിതൃഗണതര്‍പ്പണം ചെയ്ത നാഥ!\\
കാന്നുകൊള്ളുക തപോവാരിധേ! ഭൃഗുപതേ!\\
കാല്‍ത്തളിരിണ തവ ശരണം മമ വിഭോ!’\\
ഇത്തരം ദശരഥന്‍ ചൊന്നതാദരിയാതെ\\
ബദ്ധരോഷേണ വഹ്നിജ്വാല പൊങ്ങീടും വണ്ണം\\
വക്ത്രവും മധ്യാഹ്നാര്‍ക്കമണ്ഡലംപോലെ ദീപ്ത്യാ\\
സത്വരം ശ്രീരാമനോടരുളിച്ചെയ്തീടിനാന്‍:\\
‘ഞാനൊഴിഞ്ഞുണ്ടോ രാമനിത്രഭുവനത്തിങ്കല്‍?\\
മാനവനായ ഭവാന്‍ ക്ഷത്രിയനെന്നാകിലോ\\
നില്ലുനില്ലരക്ഷണമെന്നോടു യുദ്ധം ചെയ്വാന്‍\\
വില്ലിങ്കല്‍ നിനക്കേറ്റം വല്ലഭാമുണ്ടല്ലോ കേള്‍\\
നീയല്ലോ ബലാല്‍ ശൈവചാപം ഖാണ്ഡിച്ചതെന്റെ\\
കൈയിലുണ്ടൊരു ചാപം വൈഷ്ണവം മഹാസാരം.\\
ക്ഷത്രിയകുലജാതനാകില്‍ നീയിതുകൊണ്ടു\\
സത്വരം പ്രയോഗിക്കില്‍ നിന്നോടു യുദ്ധം ചെയ്വന്‍.\\
അല്ലായ്കില്‍ കൂട്ടാത്തോടെ സംഹരിച്ചീടുന്നതു-\\
ണ്ടില്ല സന്ദേഹമെനിക്കെന്നതു ധരിച്ചാലും.\\
ക്ഷത്രിയകുലാന്തകന്‍ ഞാനെന്നതറിഞ്ഞീലേ?\\
ശത്രുത്വം നമ്മില്‍ പണ്ടുപണ്ടേയുണ്ടെന്നോര്‍ക്ക നീ.’\\
രേണുകാത്മജനേവം പറഞ്ഞോരനന്തരം\\
ക്ഷോണിയും പാരമൊന്നു വിറച്ചു ഗിരിഗളും\\
അന്ധകാരംകൊണ്ടൊക്കെ മറഞ്ഞു ദിക്കുകളും\\
സിന്ധുവാരിയുമൊന്നു കലങ്ങി മറിഞ്ഞിതു.\\
എന്തോന്നു വരുന്നിതെന്നോര്‍ത്തു ദേവാദികളും\\
ചിന്ത പൂണ്ടുഴന്നിതു താപസവരന്മാരും.\\
പങ്ക്തിസ്യന്ദനന്‍ ഭീതികൊണ്ടു വേപഥു പൂണ്ടു.\\
സന്താപമുണ്ടായ്വന്നു വിരിഞ്ചതനയനും\\
മുഗ്ദ്ധഭാവവും പൂണ്ടു രാമനാം കുമാരനും\\
ക്രുദ്ധനാം പരശുരാമന്‍തന്നോടരുള്‍ചെയ്തു:\\
“ചൊല്ലെഴും മഹാനുഭാവന്മാരാം പ്രൗഢാത്മാക്കള്‍\\
വല്ലാതെ ബാലന്മാരോടിങ്ങനെ തുടങ്ങിയാല്‍\\
ആശ്രയമരവ്‍ക്കെന്തോന്നുള്ളതു തപോനിധേ!\\
സ്വാശ്രമകുലധര്‍മമെങ്ങനെ പാലിക്കുന്നു?\\
നിന്തിരുവടി തിരുവുള്ളത്തിലേറുന്നതി-\\
ന്നന്തരമുണ്ടോ പിന്നെ വരുന്നു നിരൂപിച്ചാല്‍?\\
അന്ധനായിരിപ്പൊരു ബാലകനുണ്ടോ ഗുണ-\\
ബന്ധനം ഭവിക്കുന്നു സന്തതം ചിന്തിച്ചാലും.\\
ക്ഷത്രിയകുലത്തിങ്കലുത്ഭവിക്കയും ചെയ്തേന്‍\\
ശസ്ത്രാസ്ത്രപ്രയോഗസാമര്‍ത്ഥ്യമില്ലല്ലോതാനും.\\
ശത്രുമിത്രോദാസീനഭേദവുമെനിക്കില്ല\\
ശത്രുസംഹാരംചെയ്വാന്‍ ശക്തിയുമില്ലയല്ലോ.\\
അന്തകാന്തകന്‍പോലും ലംഘിച്ചീടുന്നതല്ല\\
നിന്തിരുവടിയുടെ ചിന്തിത, മതുമൂലം\\
വില്ലിങ്ങു തന്നാലും ഞാനാകിലോ കുലച്ചീടാ-\\
മല്ലെങ്കില്‍ തിരുവുള്ളക്കേടുമുണ്ടാകവേണ്ട”\\
സുന്ദരന്‍ സുകുമാരനിന്ദിരാപതി രാമന്‍\\
കന്ദര്‍പ്പകളേബരന്‍ കഞ്ജലോചനന്‍ പരന്‍\\
ചന്ദ്രചൂഡാരവിന്ദമന്ദിരമഹേന്ദ്രാദി-\\
വൃന്ദാരകേന്ദ്രമുനിവൃന്ദവന്ദിതന്‍ ദേവന്‍\\
മന്ദഹാസവും പൂണ്ടു വന്ദിച്ചു മന്ദേതരം\\
നന്ദിച്ചു ദശരഥനന്ദനന്‍ വില്ലും വാങ്ങി\\
നിന്നരുളുന്നനേരമീരേഴു ലോകങ്ങളുമ്-\\
മൊന്നിച്ചു നിറഞ്ഞൊരു തേജസ്സു കാണായ്വന്നു.\\
കുലച്ചു ബാണമേകമെടുത്തു തൊടുത്താശു\\
വലിച്ചു നിറച്ചുടന്‍ നിന്നിതു ജതശ്രമം\\
ചോദിച്ചു ഭൃഗുപതിതന്നോടു രഘുപതി:\\
‘മോദത്തോഅരുളിച്ചെയ്തീടേണം ദയാനിധേ!\\
മാര്‍ഗണം നിഷ്ഫലമായ്വരികയില്ല മമ\\
ഭാര്‍ഗവരാമ! ലക്ഷ്യം കാട്ടിത്തന്നീടവേണം’\\
ശ്രീരാമവചനം കേട്ടന്നേരം ഭാര്‍ഗവനു-\\
മാരൂഢാനന്ദമതിനുത്തരമരുള്‍ചെയ്തു:\\
“ശ്രീരാമ! രാമ! മഹാബാഹോ! ജാനകീപതേ!\\
ശ്രീരമണാത്മാരാമ! ലോകാഭിരാമ! രാമ!\\
ശ്രീരാമ! സീതാഭിരാമാനന്ദാത്മക! വിഷ്ണോ!\\
ശ്രീരാമ! രാമ! രമാരമണ! രഘുപതേ!\\
ശ്രീരാമ! രാമ! പുരുഷോത്തമ! ദയാനിധേ!\\
ശ്രീരാമ! സൃഷ്ടിസ്ഥിതിപ്രളയഹേതുമൂര്‍ത്തേ!\\
ശ്രീരാമ! ദശരഥനന്ദന! ഹൃഷീകേശ!\\
ശ്രീരാമ! രാമ! രാമ! കൗസല്യാത്മജ! ഹരേ!\\
എങ്കിലോ പുരാവൃത്തം കേട്ടുകൊണ്ടാലും മമ\\
പങ്കജവിലോചന! കാരുണ്യവാരാന്നിധേ!\\
ചക്രതീര്‍ത്ഥത്തിങ്കല്‍ച്ചെന്നെത്രയും ബാല്യകാലേ\\
ചക്രപാണിയെത്തന്നെ തപസ്സുചെയ്തേന്‍ ചിരം\\
ഉഗ്രമാം തപസ്സുകൊണ്ടിന്ദ്രിയങ്ങളെയെല്ലാം\\
നിഗ്രഹിച്ചനുദിനം സേവിച്ചേന്‍ ഭഗവാനെ.\\
വിഷ്ണു കൈവല്യമൂര്‍ത്തി ഭഗവാന്‍ നാരായണന്‍\\
ജിഷ്ണുസേവിതന്‍ ഭജനീയനീശ്വരന്‍ നാഥന്‍\\
മാധവന്‍ പ്രസാദിച്ചു മല്‍പുരോഭാഗേ വന്നു\\
സാദരം പ്രത്യക്ഷനായരുളിച്ചെയ്തീടിനാന്‍!\\
‘ഉത്തിഷ്ഠോത്തിഷ്ഠ ബ്രഹ്മന്‍! തുഷ്ടോഹം തപസാ തേ\\
സിദ്ധിച്ചു സേവാഫലം നിനക്കെന്നറിഞ്ഞാലും\\
മത്തേജോഹുക്തന്‍ ഭവാനെന്നതുമറിഞ്ഞാലും\\
കര്‍ത്തവ്യം പലതുണ്ടു ഭവതാ ഭൃഗുപതേ!\\
കൊല്ലണം പിതൃഹന്താവാകിയ ഹേഹയനെ\\
ചൊല്ലെഴും കാര്‍ത്തവീര്യാര്‍ജുനനാം നൃപേന്ദ്രനെ\\
വല്ലജാതിയു,മവന്‍ മല്‍ക്കലാംശജനല്ലോ\\
വല്ലഭം ധനുര്‍വേദത്തിന്നവനേറുമല്ലോ.\\
ക്ഷത്രിയവംശമിരുത്തൊന്നു പരിവൃത്തി\\
യുദ്ധേ നിഗ്രഹിച്ചു കശ്യപനു ദാനം ചെയ്ക\\
പൃഥ്വീമണ്ഡലമൊക്കെ, പ്പിന്നെശ്ശാന്തിയെ പ്രാപി-\\
ച്ചുത്തമമായ തപോനിഷ്ഠയാ വസിച്ചാലും.\\
പിന്നെ ഞാന്‍ ത്രേതായുഗേ ഭൂമിയില്‍ ദശരഥന്‍-\\
തന്നുടെ തനയനായ് വന്നവതരിച്ചീടും.\\
അന്നു കണ്ടീടാം തമ്മിലെന്നാലെന്നുടെ തേജ-\\
സ്സന്യൂനം ദാശരഥി തന്നിലാക്കീടുക നീ.\\
പിന്നെയും തപസ്സു ചെയ്താപ്രഹ്മപ്രളയാന്ത-\\
മെന്നെ സ്സേവിച്ചു വസിച്ചീടുക മഹാമുനേ!’\\
എന്നരുള്‍ചെയ്തു മറഞ്ഞീടിനാന്‍ നാരായണന്‍\\
തന്നിയോഗങ്ങളെല്ലാം ചെയ്തിതു ഞാനും നാഥ!\\
തന്തിരുവടിതന്നെ വന്നവതരിച്ചൊരു\\
പങ്ക്തിസ്യന്ദനസുതനല്ലോ നീ ജഗല്‍പതേ!\\
എങ്കലുള്ളൊരു മഹാവൈഷ്ണവതേജസ്സെല്ലാം\\
നിങ്കലാക്കീടുവാനായ് തന്നിതു ശരാസനം.\\
ബ്രഹ്മാദിദേവകളാല്‍ പ്രാര്‍ഥിക്കപ്പെട്ടുള്ളൊരു\\
ധര്‍മങ്ങള്‍ മായാബലംകൊണ്ടു സാധിപ്പിക്ക നീ.\\
സാക്ഷാല്‍ ശ്രീനാരായണന്‍ താനല്ലോ ഭവാന്‍ ജഗത്-\\
സാക്ഷിയായീടും വിഷ്ണുഭഗവാന്‍ ജഗന്മയന്‍.\\
ഇന്നിപ്പോള്‍ സഫലമായ് വന്നിതു മമ ജന്മം\\
മുന്നംചെയ്തൊരു തപസ്സാഫല്യമെല്ലാ വന്നു.\\
ബ്രഹ്മമുഖ്യന്മാരാലും കണ്ടുകിട്ടീടാതൊരു\\
നിര്‍മലമായ രൂപം കാണായ്വന്നതുമൂലം\\
ധന്യനായ് കൃതാര്‍ഥനായ് സ്വസ്ഥനായ് വന്നേനല്ലോ;\\
നിന്നുടെ രൂപമുള്ളില്‍ സന്തതം വസിക്കേണം.\\
അജ്ഞാനോത്ഭവങ്ങളാം ജന്മാദിഷഡ്ഭാവങ്ങള്‍\\
സുജ്ഞാനസ്വരൂപനാം നിങ്കലില്ലല്ലോ പോറ്റീ!\\
നിര്‍വികാരാത്മാ പരിപൂര്‍ണനായിരിപ്പൊരു\\
നിര്‍വാണപ്രദനല്ലോ നിന്തിരുവടി പാര്‍ത്താല്‍.\\
വഹ്നിയില്‍ ധൂമംപോലെ വാരിയില്‍ നുരപോലെ\\
നിന്നുടെ മഹാമായാവൈഭവം ചിത്രം ചിത്രം!\\
യാവല്‍പര്യന്തം മായാസംവൃതംലോകമോര്‍ത്താല്‍\\
താവല്‍പര്യന്തമറിയാവല്ല ഭവത്തത്ത്വം.\\
സത്സംഗംകൊണ്ടു ലഭിച്ചീടിന ഭക്തിയോടും\\
ത്വത്സേവാരതന്മാരാം മാനുഷര്‍ മെല്ലെമെല്ലെ\\
ത്വന്മായാരചിതമാം സംസാരപാരാവാരം\\
തന്മറുകരയേറീടുന്നിതു കാലംകൊണ്ടേ\\
ത്വല്‍ജ്ഞാനപരന്മാരാം മാനുഷജനങ്ങള്‍ക്കു-\\
ള്ളജ്ഞാനം നീക്കുവൊരു സദ്ഗുരു ലഭിച്ചീടും.\\
സദ്ഗുരുവരങ്കല്‍നിന്നന്‍പോടു വാക്യജ്ഞാന-\\
മുള്ക്കാമ്പിലുദിച്ചിടും ത്വല്‍പ്രസാദത്താലപ്പോള്‍.\\
കര്‍മബന്ധത്തിങ്കല്‍ നിന്നാശു വേര്‍പെട്ടു ഭവ-\\
ച്ചിന്മയപദത്തിങ്കലാഹന്ത! ലയിച്ചീടും.\\
ത്വല്‍ഭക്തിവിഹീനന്മാരായുള്ള ജനങ്ങള്‍ക്കു\\
കല്പകോടികള്‍കൊണ്ടും സിദ്ധിക്കയില്ലയല്ലോ\\
വിജ്ഞാനജ്ഞാനസുഖം മോക്ഷമെന്നറിഞ്ഞാലും;\\
അജ്ഞാനം നീക്കി ത്വല്‍ബോധം മമ സിദ്ധിക്കേണം.\\
ആകയാല്‍ ത്വല്‍പാദപത്മങ്ങളില്‍ സദാകാല-\\
മാകുലംകൂടാതൊരു ഭക്തി സംഭവിക്കേണം.\\
നമസ്തേ ജഗല്‍പതേ! നമസ്തേ രമാപതേ!\\
നമസ്തേ ദാശരഥേ! നമസ്തേ സതാംപതേ!\\
നമസ്തേ വേദപതേ! നമസ്തേ ദേവപതേ!\\
നമസ്തേ മഖപതേ നമസ്തേ ധരാപതേ!\\
നമസ്തേ ധര്‍മപതേ! നമസ്തേ സീതാപതേ!\\
നമസ്തേ കാരുണ്യാബ്ധേ! നമസ്തേ ചാരുമൂര്‍ത്തേ!\\
നമസ്തേ രാമരാമ! നമസ്തേ രാമചന്ദ്ര!\\
നമസ്തേ രാമരാമ! നമസ്തേ രാമഭദ്ര!\\
സന്തതം നമോസ്തുതേ ഭഗവന്‍! നമോസ്തുതേ\\
ചിന്തയേ ഭവച്ചരണാംബുജം നമോസ്തുതേ\\
സ്വര്‍ഗതിക്കായിട്ടെന്നാല്‍ സഞ്ചിതമായ പുണ്യ-\\
മൊക്കെ നിന്‍ ബാണത്തിന്നു ലക്ഷ്യമായ് ഭവിക്കേണം!\\
എന്നതു കേട്ടു തെളിഞ്ഞന്നേരം ജഗന്നാഥന്‍\\
മന്ദഹാസവും ചെയ്തു ഭര്‍ഗവനോടു ചൊന്നാന്‍:\\
‘സന്തോഷം പ്രാപിച്ചേന്‍ ഞാന്‍ നിന്തിരുവടിയുള്ളി-\\
ലെന്തൊന്നു ചിന്തിച്ചതെന്നാലവയെല്ലാം തന്നേന്‍.’\\
പ്രീതി കൈക്കൊണ്ടു ജമദഗ്നിപുത്രനുമപ്പോള്‍\\
സാദരം ദശരഥപുത്രനോടരുള്‍ചെയ്തു:\\
‘ഏതാനുമനുഗ്രഹമുണ്ടെന്നെക്കുറിച്ചെങ്കില്‍\\
പാദഭക്തന്മാരിലും പാദപത്മങ്ങളിലും\\
ചേതസി സദാകാലം ഭക്തി സംഭവിക്കേണം\\
മാധവാ! രഘുപതേ! രാമ! കാരുണ്യാംബുധേ!\\
‘ഇസ്തോത്രം മയാ കൃതം ജപിച്ചീടുന്ന പുമാന്‍\\
ഭക്തനായ് തത്ത്വജ്ഞനായീടേണം, വിശേഷിച്ചും\\
മൃത്യുവന്നടുക്കുമ്പോള്‍ ത്വല്‍പാദാംബുജഗതി\\
ചിത്തേ സംഭവിപ്പതിന്നായനുഗ്രഹിക്കേണം.’\\
അങ്ങനെത്തന്നെയെന്നു രാഘവന്‍ നിയോഗത്താല്‍\\
തിങ്ങിന ഭക്തിപൂണ്ടു രേണുകാതനയനും\\
സാദരം പ്രദക്ഷിണം ചെയ്തു കുമ്പിട്ടുക്കൂപ്പി\\
പ്രീതനായ് ചെന്നു മഹേന്ദ്രാചലം പുക്കീടിനാന്‍.\\
ഭൂപതി ദശരഥന്‍ താനതിസന്തുഷ്ടനായ്\\
താപവുമകന്നുതന്‍ പുത്രനാം രാമന്‍തന്നെ\\
ഗാഢമായാശ്ലേഷംചെയ്താനന്ദാശ്രുക്കളോടും\\
പ്രൗഢാത്മാവായ വിധിനന്ദനനോടും കൂടി\\
പുത്രന്മാരോടും പടയോടും ചെന്നയോധ്യയില്‍\\
സ്വസ്ഥമാനസനായ് വാണീടിനാന്‍ കീര്‍ത്തിയോടെ.\\
ശ്രീരാമാദികള്‍ നിജഭാര്യമാരോടുംകൂടി\\
സ്വൈരമായ് രമിച്ചു വാണീടിനാരെല്ലാവരും\\
വൈകുണ്ഠപുരിതന്നില്‍ ശ്രീഭഗവതിയോടും\\
വൈകുണ്ഠന്‍വാഴുമ്പോലെ രാഘവന്‍ സീതയോടും\\
ആനന്ദമൂര്‍ത്തി മായാമാനുഷവേഷം കൈക്കൊ-\\
ണ്ടാനന്ദംപൂണ്ടു വസിച്ചീടിനാനനുദിനം.\\
കേകയനരാധിപനാകിയ യുധാജിത്തും\\
കൈകേയീതനയനെക്കൂട്ടിക്കൊണ്ടങ്ങു ചെല്‍വാന്‍\\
ദൂതനെയയച്ചതു കണ്ടൊരു ദശരഥന്‍\\
സോദരനായ് മേവീടും ശത്രുഘ്നനോടുംകൂടി\\
സാദരം ഭരതനെപ്പോവാനായ് നിയോഗിച്ചാ-\\
നാദരവോടു നടന്നീടിനാരവര്‍കളും.\\
മാതുലന്‍തന്നെക്കണ്ടു ഭരതശത്രുഘ്നന്മാര്‍\\
മോദമുള്‍ക്കൊണ്ടു വസിച്ചീടിനാരതുകാലം.\\
മൈഥിലിയോടും നിജനന്ദനനോടും ചേര്‍ത്തു\\
കൗസല്യാദേവിതാനും പരമാനന്ദം പൂണ്ടാള്‍.\\
രാമലക്ഷ്മണന്മാരാം പുത്രന്മാരോടും നിജ-\\
ഭാമിനിമാരോടുമാനന്ദിച്ചു ദശരഥന്‍\\
സാകേതപുരിതന്നില്‍ സുഖിച്ചു വാണീടിനാന്‍\\
പാകശാസനനമരാലയേ വാഴുമ്പോലെ.\\
നിര്‍വികാരാന്മാവായ പരമാനന്ദമൂര്‍ത്തി\\
സര്‍വലോകാനന്ദാര്‍ഥം മനുഷ്യാകൃതിപൂണ്ടു\\
തന്നുടെ മായാദേവിയാകിയ സീതയോടു-\\
മൊന്നിച്ചു വാണാനയോധ്യാപുരി തന്നിലന്നേ.
\end{verse}

\begin{center}
ഇത്യദ്ധ്യാത്മരാമായണേ ഉമാമഹേശ്വരസംവാദേ\\
ബാലകാണ്ഡം സമാപ്തം
\end{center}
		
