%%1). ayodhyakaandam
\chapter{അയോദ്ധ്യാകാണ്ഡം}

\begin{verse}
താര്‍മകള്‍ക്കന്‍പുള്ള തത്തേ! വരികെടോ\\
താമസശീലമകറ്റേണമാശു നീ\\
രാമദേവന്‍ ചരിതാമൃതമിന്നിയു-\\
മാമോദമുള്‍ക്കൊണ്ടു ചൊല്ലൂ സരസമായ്.\\
എങ്കിലോ കേള്‍പ്പിന്‍ ചുരുക്കി ഞാന്‍ ചൊല്ലുഅന്‍\\
പങ്കമെല്ലാമകലും പലജാതിയും.\\
സങ്കടമേതും വരികയുമില്ലല്ലോ\\
പങ്കജനേത്രന്‍ കഥകള്‍ കേട്ടീടിനാല്‍.\\
ഭാര്‍ഗവിയാകിയ ജാനകിതന്നുടെ\\
ഭാഗ്യജലനിധിയാകിയ രാഘവന്‍\\
ഭാര്‍ഗവന്‍ തന്നുടെ ദര്‍പ്പ ശമിപ്പിച്ചു\\
മാര്‍ഗവും പിന്നിട്ടയോധ്യാപുരിപുക്കു\\
താതനോടും നിജമാതൃജനത്തോടും\\
ധാതൃസുതനാം ഗുരുവരന്‍തന്നോടും\\
ഭ്രാതാക്കളോടും പടയോടുമൊന്നിച്ചു\\
മേദിനീപുത്രിയാം ഭാമിനിതന്നോടും\\
വന്നെതിരേറ്റൊരു പൗരജനത്തോടും\\
ചെന്നു മഹാരാജധാനിയകംപുക്കു.\\
വന്നിതു സൗഖ്യം ജഗത്തിനു രാഘവന്‍-\\
തന്നുടെ നാനാഗുണഗണം കാണ്‍കയാല്‍.\\
രുദ്രന്‍ പരമേശ്വരന്‍ ജഗദീശ്വരന്‍\\
കദ്രുസുതഗണഭൂഷണആഭൂഷിതന്‍\\
ചിദ്രൂപനദ്വയന്‍ മൃത്യുഞ്ജയന്‍ പരന്‍\\
ഭദ്രപ്രദന്‍ ഭഗവാന്‍ ഭവഭഞ്ജനന്‍\\
രുദ്രാണിയാകിയ ദേവിക്കുടന്‍ രാമ-\\
ഭദ്രകഥാമൃതസാരം കൊടുത്തപ്പോള്‍\\
വിദ്രുമതുല്യാധരിയായ ഗൗരിയാ-\\
മദ്രിസുതയുമാനന്ദവിവശയായ്,\\
ഭര്‍ത്തൃപാദപ്രണാമംചെയ്തു സമ്പൂര്‍ണ-\\
ഭക്തിയോടും പുനരേവമരുള്‍ ചെയ്തു:\\
‘നാരായണന്‍ നളിനായതലോചനന്‍\\
നാരീജനമനോമോഹനന്‍ മാധവന്‍\\
നാരദസേവ്യന്‍ നളിനാസനപ്രിയന്‍\\
നാരകാരാതി നളിനശരഗുരു\\
നാഥന്‍ നരസഖന്‍ നാനാജഗന്മയന്‍\\
നാദവിദ്യാത്മകന്‍ നാമസഹസ്രവാന്‍\\
നാളീകരമ്യവദനന്‍ നരകാരി\\
നാളീകബാന്ധവവംശസമുദ്ഭവന്‍\\
ശ്രീരാമദേവന്‍ പരന്‍ പുരുഷോത്തമന്‍\\
കാരുണ്യവാരിധി കാമഫലപ്രദന്‍\\
രാക്ഷസവംശവിനാശനകാരണന്‍\\
സാക്ഷാല്‍ മുകുന്ദനാനന്ദപ്രദന്‍ പുമാന്‍\\
ഭക്തജനോത്തമഭുക്തിമുക്തിപ്രദന്‍\\
സക്തിവിമുക്തന്‍ വിമുക്തഹൃദിസ്ഥിതന്‍\\
വ്യക്തനവ്യക്തനനന്തനനാമയന്‍\\
ശക്തിയുക്തന്‍ ശരണാഗതവത്സലന്‍\\
നക്തഞ്ചരേശ്വരനായ ദശാസ്യനു\\
മുക്തികൊടുത്തവന്‍ തന്റെ ചരിത്രങ്ങള്‍\\
നക്തന്ദിവം ജീവിതാവധി കേള്‍ക്കിലും\\
തൃപ്തിവരാ മമ വേണ്ടീല മുക്തിയും.’\\
ഇത്ഥം ഭഗവതി ഗൗരി മഹേശ്വരി\\
ഭക്ത്യാ പരമേശ്വരനോടു ചൊന്നപ്പോള്‍\\
മന്ദസ്മിതം ചെയ്തു മന്മഥനാശനന്‍\\
സുന്ദരി! കേട്ടുകൊള്‍കെന്നരുളിച്ചെയ്തു.
\end{verse}

%%2). naaradaraaghavasanmvadam

\section{നാരദരാഘവസംവാദം}

\begin{verse}
എങ്കിലൊരു ദിനം ദാശരഥി രാമന്‍\\
പങ്കജലോചനന്‍ ഭക്തപരായണന്‍\\
മംഗലദേവതാകാമുകന്‍ രാഘവ-\\
നംഗജനാശനവന്ദിതന്‍ കേശവന്‍\\
അംഗജലീലപൂണ്ടന്തഃപുരത്തിങ്കല്‍\\
മംഗലഗാത്രിയാം ജാനകി തന്നൊടും\\
നീലോല്പലദളശ്യാമളവിഗ്രഹന്‍\\
നീലോല്പലദളലോലവിലോചനന്‍\\
നീലോപലാഭന്‍ നിരുപമന്‍ നിര്‍മലന്‍\\
നീലഗളപ്രിയന്‍ നിത്യന്‍ നിരാമയന്‍\\
രത്നാഭരണവിഭൂഷിതദേഹനായ്\\
രത്നസിംഹാസനം തന്മേലനാകുലം\\
രത്നദണ്ഡം പൂണ്ട വെണ്‍ചാമരംകൊണ്ടു\\
പത്നിയാല്‍ വീജിതനായതികോമളന്‍\\
ബാലനിശാകരഫാലദേശേ ലസന്‍\\
മാലേയപങ്കമലങ്കരിച്ചങ്ങനെ\\
ബാലാര്‍ക്കസന്നിഭകൗസ്തുഭകന്ധരന്‍\\
പ്രാലേയഭാനുസമാനനയാ സമം\\
ലീലയാ താംബൂലചര്‍വണാദ്യൈരനു-\\
വേലം വിനോദിച്ചിരുന്നരുളുന്നേരം\\
ആലോകനാര്‍ഥം മഹാമുനി നാരദന്‍\\
ഭൂലോകമപ്പോളലങ്കരിച്ചീടിനാന്‍.\\
മുഗ്ദ്ധശരച്ചന്ദ്രതുല്യതേജസ്സോടും\\
ശുദ്ധസ്ഫടികസങ്കാശശരീരനായ്\\
സത്വരമംബരത്തിങ്കല്‍ നിന്നാദരാല്‍\\
തത്രൈവ വേഗാലവതരിച്ചീടിനാന്‍.\\
ശ്രീരാമദേവനും സംഭ്രമം കൈക്കൊണ്ടു\\
നാരദനെക്കണ്ടെഴുന്നേറ്റു സാദരം\\
നാരീമണിയായ ജാനകിതന്നൊടും\\
പാരില്‍ വീണാശു നമസ്കരിച്ചീടിനാന്‍\\
പാദ്യാസനാചമനീയാര്‍ഘ്യപൂര്‍വക-\\
മാദ്യേന’പൂജിത’നായൊരു നാരദന്‍\\
തന്നിയോഗത്താലിരുന്നൊരു രാഘവന്‍\\
മന്ദസ്മിതംപൂണ്ടു നന്ദിച്ചു സാദരം\\
മന്ദം മുനിവരന്‍ തന്നോടരുള്‍ചെയ്തു:\\
‘വന്ദേ പദം കരുണാനിധേ! സാമ്പ്രതം.\\
നാനാവിഷയസംഗംപൂണ്ടു മേവിന\\
മാനസത്തോടു സംസാരികളായുള്ള\\
മാനവന്മാരായ ഞങ്ങള്‍ക്കു ചിന്തിച്ചാല്‍\\
ജ്ഞാനിയാകും തവ പാദപങ്കേരുഹം\\
കണ്ടുകൊള്‍വാനതി ദുര്‍ലഭം നിര്‍ണയം;\\
പണ്ടു ഞാന്‍ ചെയ്തൊരു പുണ്യഫലോദയം-\\
കൊണ്ടു കാണ്മാനവകാശവും വന്നിതു\\
പുണ്ഡരീകോത്ഭവപുത്ര! മഹാമുനേ!\\
എന്നുടെ വംശവും ജന്മവും രാജ്യവു-\\
മിന്നു വിശുദ്ധമായ് വന്നു തപോനിധേ!\\
‘എന്നാലിനിയെന്തു കാര്യമെന്നും പുന-\\
രെന്നോടരുള്‍ചെയ്കവേണം ദയാനിധേ!\\
എന്തൊരു കാര്യം നിരൂപിച്ചെഴുന്നള്ളി?\\
സന്തോഷമുള്‍ക്കൊണ്ടരുള്‍ചെയ്കയും വേണം.\\
മന്ദനെന്നാകിലും കാരുണ്യമുണ്ടെങ്കില്‍\\
സന്ദേഹമില്ല സാധിപ്പിപ്പനെല്ലാമേ\\
എത്ഥമാകരാ‍ണ്യ രഘുവരന്‍ തന്നോടു\\
മുഗ്ദ്ധഹാസേന മുനിവരനാകിയ\\
നാരദനും ഭക്തവത്സലനാം മനു-\\
വീരനെ നോക്കിസ്സരസമരുള്‍ചെയ്തു:\\
‘എന്തിനിന്നെന്നെ മോഹിപ്പിപ്പതിന്നു നീ\\
സന്തതം ലോകാനുകാരികളായതി-\\
ചാതുര്യമുള്ളൊരു വാക്കുകളേറ്റവും\\
മാധുര്യമോടു ചൊല്ലീടുന്നതിങ്ങനെ?\\
മുഗ്ദ്ധങ്ങളായുള്ള വാക്യങ്ങളെക്കൊണ്ടു\\
ചിത്തമോഹം വളര്‍ക്കേണ്ട രഘുപതേ!\\
ലൗകികമായുള്ള വാക്യങ്ങളെന്നാലും\\
ലോകോത്തമന്മാര്‍ക്കു വേണ്ടിവരുമല്ലോ.\\
യോഗേശനായ നീ സംസാരി ഞാനെന്നു\\
ലോകേശ! ചൊന്നതു സത്യമത്രേ ദൃഢം.\\
സര്‍വജഗത്തിനും കാരണആഭൂതയായ്\\
സര്‍വമാതാവായ മായാഭഗവതി\\
സര്‍വജഗത്പിതാവാകിയ നിന്നുടെ\\
ദിവ്യഗൃഹിണിയാകുന്നതു നിര്‍ണയം\\
ഈരേഴു ലോകവും നിന്റെ ഗൃഹമപ്പോള്‍\\
ചേരും ഗൃഹസ്ഥനാകുന്നതെന്നുള്ളതും\\
നിന്നുടെ സന്നിധിമാത്രേണ മായയില്‍-\\
നിന്നു ജനിക്കുന്നു നാനാ പ്രജകളും.\\
അര്‍ണോജസംഭവനാദി തൃണാന്തമാ-\\
യൊന്നൊഴിയാതെ ചരാചരജന്തുക്കള്‍\\
ഒക്കവേ നിന്നപത്യം പുനരാകയാ-\\
ലൊക്കും പറഞ്ഞതു സംസാരിയെന്നതും.\\
ഇക്കണ്ട ലോകജന്തുക്കള്‍ക്കു സര്‍വദാ\\
മുഖ്യനാകും പിതാവായതും നീയല്ലോ.\\
ശുക്ലരക്താസിതവര്‍ണഭേദം പൂണ്ടു\\
സത്വരജസ്തമോ നാമ ഗുണത്രയ-\\
യുക്തയായീടിന വിഷ്ണുമഹാമായാ-\\
ശക്തിയല്ലോ തവ പത്നിയാകുന്നതും.\\
സത്വങ്ങളെ ജ്ജനിപ്പിക്കുന്നതുമവള്‍\\
സത്യം ത്വയോക്തമതിനില്ല സംശയം\\
പുത്രമിത്രാര്‍ഥകളത്രവസ്തുക്കളില്‍\\
സക്തനായുള്ള ഗൃഹസ്ഥന്‍ മഹാമതേ!\\
ലോകത്രയമഹാഗേഹത്തിനു ഭവാ-\\
നേകനായോരു ഗൃഹസ്ഥനാകുന്നതും.\\
നാരായണന്‍ നീ രമാദേവി ജാനകി\\
മാരാരിയും നീയുമാദേവി ജാനകി\\
സാരസസംഭവനായതും നീ തവ\\
ഭാരതീദേവിയാകുന്നതും ജാനകി\\
ആദിത്യനല്ലോ ഭവാന്‍ പ്രഭാ ജാനകി\\
ശീതകിരണന്‍ നീ രോഹിണി ജാനകി\\
ആദിതേയാധിപന്‍ നീ ശചി ജാനകി\\
ജാതവേദസ്സു നീ സ്വാഹാ മഹീസുതാ\\
അര്‍ക്കജന്‍ നീ ദണ്ഡനീതിയും ജാനകി\\
രക്ഷോവരന്‍ ഭവാന്‍ താമസി ജാനകി\\
പുഷ്കരാക്ഷന്‍ ഭവാന്‍ ഭാര്‍ഗവി ജാനകി\\
ശക്രദൂതന്‍ നീ സദാഗതി ജാനകി\\
രാജരാജന്‍ ഭവാന്‍ സമ്പല്‍ക്കരി സീതാ\\
രാജരാജന്‍ നീ വസുന്ധര ജാനകി\\
രാജപ്രവര കുമാര! രഘുപതേ!\\
രാജീവലോചന! രാമ! ദയാനിധേ\\
രുദ്രനല്ലോ ഭവാന്‍ രുദ്രാണി ജാനകി\\
സ്വര്‍ദ്രുമം നീ ലതാരൂപിണി ജാനകി\\
വിസ്തരിച്ചെന്തിനേറെപ്പറഞ്ഞീടുന്നു?\\
സത്യപരാക്രമ! സല്‍ഗുണവാരിധേ!\\
യാതൊന്നു യാതൊന്നു പുല്ലിംഗവാചകം\\
വേദാന്തവേദ്യ! തല്‍ സര്‍വവുമേവ നീ\\
ചേതോവിമോഹന! സ്ത്രീലിംഗവാചകം\\
യാതൊന്നതൊക്കവേ ജാനകീദേവിയും\\
നിങ്ങളിരുവരുമെന്നിയേ മറ്റൊന്നു-\\
മെങ്ങുമേ കണ്ടീല കേള്‍പ്പാനുമില്ലല്ലോ\\
അങ്ങനെയുള്ളോരു നിന്നെത്തിരഞ്ഞറി-\\
ഞ്ഞെങ്ങനെ സേവിച്ചു കൊള്‍വൂ ജഗല്‍പതേ!\\
മായയാ മൂടി മറഞ്ഞിരിക്കുന്നൊരു\\
നീയല്ലോ നൂനമവ്യാകൃതമായതും\\
പിന്നെയതിങ്കല്‍നിന്നുള്ളൂ മഹത്തത്ത്വ-\\
മെന്നതതിങ്കല്‍ നിന്നുണ്ടായി സൂത്രവും.\\
സര്‍വാത്മകമായ ലിംഗമതിങ്കല്‍ നി-\\
ന്നുര്‍വീപതേ! പുനരുണ്ടായ്ച്ചമഞ്ഞതും\\
എന്നതഹങ്കാരബുദ്ധിപഞ്ചപ്രാണ-\\
നിന്ദ്രിയജാലസംയുക്തമായൊന്നല്ലോ;\\
ജന്മമൃതിസുഖദുഃഖാദികളുണ്ടു\\
നിര്‍മലന്മാര്‍ ജീവനെന്നു ചൊല്ലുന്നതും.\\
ചൊല്ലാവതല്ലാതനാദ്യവിദ്യാഖ്യയെ\\
ചൊല്ലുന്നു കാരണോപാധിയെന്നും ചില‍ര്‍.\\
സ്ഥൂലവും സൂക്ഷ്മവും കാരണമെന്നതും\\
മൂലമാം ചിത്തിനുള്ളോരുപാധിത്രയം.\\
എന്നിവറ്റാല്‍ വിശിഷടം ജീവനായതു-\\
മന്യൂനനാം പരന്‍ തദ്വിയുക്തന്‍ വിഭോ!\\
സര്‍വപ്രപഞ്ചത്തിനും ബിംബഭൂതനായ്\\
സര്‍വോപരിസ്ഥിതനായ് സര്‍വസാക്ഷിയായ്\\
തേജോമയനാം പരന്‍ പരമാത്മാവു\\
രാജീവലോചനനാകുന്ന നീയല്ലോ.\\
നിങ്കല്‍നിന്നുണ്ടായ്വരുന്നിതു ലോകങ്ങള്‍\\
നിങ്കലത്രേ ലയിക്കുന്നിതൊക്കെയോര്‍ക്കുംവിധൗ\\
കാരണമെല്ലാറ്റിനും ഭവാന്‍ നിര്‍ണയം\\
നാരയണ! നരകാരേ! നരാധിപ!\\
ജീവനും രജ്ജൂവിങ്കല്‍സര്‍പ്പമെന്നുള്ള\\
ഭാവനകൊണ്ടു ഭയത്തെ വഹിക്കുന്നു.\\
നേരേ പരമാത്മാ ഞാനെന്നറിയുമ്പോള്‍\\
തീരും ഭവഭയമൃത്യുദുഃഖാദികള്‍.\\
ത്വല്‍ക്കഥാനാമശ്രവണാദികൊണ്ടുടന്‍\\
ഉള്‍ക്കാമ്പിലുണ്ടായ്വരും ക്രമാല്‍ ഭക്തിയും.\\
ത്വല്‍പാദപങ്കജഭക്തി മുഴുക്കുമ്പോള്‍\\
ത്വല്‍ബോധവും മനഃകാമ്പിലുദിച്ചിടും\\
ഭക്തി മുഴുത്തു തത്ത്വജ്ഞാനമുണ്ടായാല്‍\\
മുക്തിയും വന്നീടുമില്ലൊരു സംശയം\\
ത്വത്ഭക്തഭൃത്യഭൃത്യന്മാരിലേകനെ-\\
ന്നല്പജ്ഞനാമെന്നെയും കരുതേണമേ!\\
മായയാലെന്നെ മോഹിപ്പിയാതേ ജഗ-\\
ന്നായക! നിത്യമനുഗ്രഹിക്കേണമേ\\
ത്വന്നാഭിപങ്കജത്തിങ്കല്‍നിന്നേകദാ\\
മുന്നമുണ്ടായി ചതുര്‍മുഖന്‍ മല്‍പിതാ.\\
നിന്നുടെ പൗത്രനായ് ഭക്തനായ് മേവിനോ-\\
രെന്നെയനുഗ്രഹിക്കേണം വിശേഷിച്ചും\\
പിന്നെയും പിന്നെയും വീളു നമസ്കരി-\\
ച്ചെന്നിവണ്ണം പറഞ്ഞീടിനാന്‍ നാരദന്‍\\
ആനന്ദബാഷ്പപരിപ്ലുതനേത്രനായ്\\
വീണാധരന്‍ മുനി പിന്നെയുംചൊല്ലിനാന്‍:\\
“ഇപ്പോളിവിടേക്കു ഞാന്‍ വന്ന കാരണ-\\
മുല്പലസംഭാവന്‍ തന്റെ നിയോഗത്താല്‍\\
രാവണനെക്കൊന്നു ലോകങ്ങള്‍ പാലിപ്പാന്‍\\
ചേവകളോടരുള്‍ചെയ്തതു കാരണം\\
മര്‍ത്ത്യനായ്വന്നു ജനിച്ചി ദശരഥ-\\
പുത്രനായെന്നതോ നിശ്ചയമെങ്കിലും\\
പൂജ്യനായൊരു ഭാവാനെ ദശരഥന്‍\\
രാജ്യരക്ഷാര്‍ത്ഥമഭിഷേകമിക്കാലം\\
ചെന്നുമാറെന്നൊരുമ്പട്ടിരിക്കുന്നിതു\\
നീയുമതിന്നനുകൂലമായ് വന്നിടും\\
പിന്നെദ്ദശമുഖനെക്കൊന്നുകൊള്ളുവാ-\\
നെന്നുമവകാശമുണ്ടായ്വരായല്ലോ.\\
സത്യത്തെ രക്ഷിച്ചുകൊള്ളുകെന്നെന്നോടു\\
സത്വരംചെന്നു പറകെന്നരുള്‍ചെയ്തു\\
സത്യസന്ധന്‍ ഭവാനെന്ങ്കിലും മാനസേ\\
മര്‍ത്ത്യജന്മംകൊണ്ടു വിസ്മൃതനായ് വരും.’\\
ഇത്തരം നാരദന്‍ ചൊന്നതു കേട്ടതി-\\
നുത്തരമായരുള്‍ചെയ്തിതു രാഘവന്‍:\\
‘സത്യത്തെ ലംഘിക്കയില്ലൊരുനാളും ഞാന്‍\\
ചിത്തേ വിഷാദമുണ്ടാകായ്കതുമൂലം.\\
കാലവിളംബനമെന്തിനെന്നല്ലല്ലീ\\
മൂലമതിനുണ്ടതും പറഞ്ഞീടുവന്‍\\
കാലാവലോകന്ം കാര്യസാധ്യം നൃണം\\
കാലസ്വരൂപനല്ലോ പരമേശ്വരന്‍,\\
പ്രാരാബ്ധകര്‍മഫലൗഘക്ഷയം വരു-\\
ന്നേരത്തൊഴിഞ്ഞു മറ്റാവതില്ലാര്‍ക്കുമേ.\\
കാരണമാത്രം പുരുഷപ്രയാസമെ-\\
ന്നാരുമറിയാതിരിക്കയുമല്ലല്ലോ.\\
നാളെ വനത്തിനു പോകുന്നതുണ്ടു ഞാന്‍\\
നാളീകലോചനന്‍ പാദങ്ങള്‍ തന്നാണേ.\\
പിന്നെച്ചതുര്‍ദശസംവത്സരം വനം-\\
തന്നില്‍ മുനിവേഷമോടു വാണീടുവന്‍.\\
എന്നാല്‍ നിശാചരവംശവും രാവണന്‍-\\
തന്നെയും കൊന്നുമുടിക്കുന്നതുണ്ടല്ലോ.\\
സീതയെക്കാരണഭൂതയാക്കിക്കൊണ്ടു\\
യാതുധാനാന്വയനാശം വരുത്തുവന്‍.\\
സത്യമിതെ’ന്നരുള്‍ചെയ്തു രഘുപതി,\\
ചിത്തപ്രമോദേന നാരദനന്നേരം\\
രാഘവന്‍തന്നെ പ്രദക്ഷിണവും ചെയ്തു\\
വേഗേന ദണ്ഡനമസ്കാരവും ചെയ്തു\\
ദേവമുനീന്ദ്രനനുജ്ഞയും കൈക്കൊണ്ടു\\
ദേവലോകം ഗമിച്ചീടിനാനാദരാല്‍.\\
നാരദരാഘവസംവാദമിങ്ങനെ\\
നേരേ പഠിക്കതാന്‍ കേള്‍ക്കതാനോര്‍ക്കതാന്‍\\
ഭാക്തിക്കൈക്കൊണ്ടുചെന്നുന്ന മനുഷ്യനു\\
മുക്തിലഭിക്കുമതിനില്ല സംശയം.\\
ശേഷമിന്നും കഥ കേള്‍ക്കണമെങ്കിലോ\\
ദോഷമകലുവാന്‍ ചൊല്ലുന്നതുണ്ടു ഞാന്‍.
\end{verse}

%%3). Sreeraamaabhishekaarambam

\section{ശ്രീരാമാഭിഷേകാരംഭം}

\begin{verse}
എങ്കിലോ രാജാ ദശരഥനേകദാ\\
സങ്കലിതാനന്ദമാമ്മാറിരിക്കുമ്പോള്‍\\
പങ്കജസംഭവപുത്രന്‍ വസിഷ്ഠനാം\\
തന്‍കുലാചാര്യനെ വന്ദിച്ചു ചൊല്ലിനാന്‍:\\
‘പൗരജനങ്ങളും മന്ത്രിമുഖ്യന്മാരും\\
ശ്രീരാമനെ പ്രശംസിക്കുന്നിതെപ്പൊഴും\\
ഓരോ ഗുണഗണം കണ്ടവര്‍ക്കുണ്ടക-\\
താരിലാനന്ദമതിനില്ല സംശയം.\\
വൃദ്ധനായ് വന്നിതു ഞാനുമൊട്ടാകയാല്‍\\
പുത്രരില്‍ ജ്യേഷ്ഠനാം രാമകുമാരനെ\\
പൃഥ്വീപരിപാലനാര്‍ഥമഭിഷേക-\\
മെത്രയും വൈകാതെ ചെയ്യണമെന്നു ഞാന്‍\\
കല്പിച്ചതിപ്പോഴതങ്ങനെയെങ്കില-\\
തുള്‍പ്പൂവിലോര്‍ത്തു നിയോഗിക്കയും വേണം.\\
ഇപ്രജകള്‍ക്കനുരാഗമവങ്കലു-\\
ണ്ടെപ്പോഴുമേറ്റമതോര്‍ത്തുകണ്ടീലയോ?\\
വന്നീല മാതുലനെക്കാണ്മതിന്നേറെ\\
മുന്നമേ പോയ ഭരതശത്രുഘ്നന്മാര്‍.\\
വന്നു മുഹൂര്‍ത്തമടുത്തദിനം തന്നെ\\
പുണ്യമതീവ പുഷ്യം നല്ല നക്ഷത്രം.\\
എന്നാലവര്‍ വരുവാന്‍ പാര്‍ക്കയില്ലിനി-\\
യൊന്നുകൊണ്ടുമതു നിര്‍ണയം മാനസേ.\\
എന്നാലതിനു വേണ്ടുന്ന സംഭാരങ്ങ-\\
ളിന്നുതന്നേ ബത സംഭരിച്ചീടണം\\
രാമനോടും നിന്തിരുവടി വൈകാതെ\\
സാമോദമിപ്പോഴേ ചെന്നറിയിക്കണം.\\
തോരണപങ്ക്തികളെല്ലാമുയര്‍ത്തുക\\
ചാരുപതാകകളോടുമത്യുന്നതം.\\
ഘോരമായുള്ള പെരുമ്പറനാദവും\\
പൂരിക്ക ദിക്കുകളൊക്കെ മുഴങ്ങവേ.’\\
മന്നവനായ ദശരഥനാദരാല്‍\\
പിന്നെസ്സുമന്ത്രരെ നോക്കിയരുള്‍ചെയ്തു:’\\
എല്ലാം വസിഷ്ഠനരുളിച്ചെയ്യുംവണ്ണം\\
കല്യാണമുള്‍ക്കൊണ്ടൊരുക്കിക്കൊടുക്ക നീ\\
നാളെ വേണമഭിഷേകമിളമയായ്\\
നാളീകനേരനാം രാമനു നിര്‍ണയം.’\\
നന്ദിതനായ സുമന്ത്രരുമന്നേരം\\
വന്ദിച്ചു ചൊന്നാന്‍ വസിഷ്ഠനോടാദരാല്‍:\\
‘എന്തോന്നു വേണ്ടുന്നതെന്നരുള്‍ചെയ്താലു-\\
മന്തരമെന്നിയേ സംഭരിച്ചീടുവന്‍.’\\
ചിത്തേ നിരൂപിച്ചുകണ്ടു സുമന്ത്രരോ-\\
ടിത്ഥം വസിഷ്ഠമുനിയുമരുള്‍ചെയ്തു:\\
‘കേള്‍ക്ക നാളെപ്പുലര്‍കാലേ ചമയിച്ചു\\
ചേല്ക്കണ്ണിമാരായ കന്യകമാരെല്ലാം\\
മധ്യകക്ഷ്യേ പതിനാറു പേര്‍ നില്ക്കണം\\
മത്തഗജങ്ങളെ പൊന്നണിയിക്കണം.\\
ഐരാവതകുലജാതനാം നാല്‍ക്കൊമ്പ-\\
നാരാല്‍ വരേണമലങ്കരിച്ചങ്കണേ.\\
ദിവ്യനാനാതീര്‍ഥവാരിപൂര്‍ണങ്ങളായ്\\
ദിവ്യരത്നങ്ങളമഴ്ത്തി വിചിത്രമായ്\\
സ്വര്‍ണകലശസഹസ്രം മലയജ-\\
പര്‍ണങ്ങള്‍കൊണ്ടു വായ്കെട്ടിവെച്ചീടണം.\\
പുത്തന്‍ പുലിത്തോല്‍ വരുത്തുക മൂന്നിഹ\\
ഛത്രം സുവര്‍ണദണ്ഡം മണിശോഭിതം\\
മുക്താമണിമാല്യരാജിത നിര്‍മല-\\
വസ്ത്രങ്ങള്‍ മാല്യങ്ങളാഭരണങ്ങളും\\
സല്‍കൃതന്മാരാം മുനിജനം വന്നിഹ\\
നില്ക്കകുശപാണികളായ് സഭാന്തികേ.\\
നര്‍ത്തകിമാരോടു വാരവധൂജനം\\
നര്‍ത്തക ഗായക വൈണികവര്‍ഗവും\\
ദിവ്യവാദ്യങ്ങളെല്ലാം പരയോഗിക്കണ-\\
മുര്‍വീശ്വരാങ്കണേനിന്നു മനോഹരം\\
ഹസ്ത്യശ്വപത്തിരഥാദി മഹാബലം\\
വസ്ത്രാദ്യലങ്കാരമോടു വന്നീടണം\\
ദേവാലയങ്ങള്‍തോറും ബലിപൂജയും\\
ദീപാവലികളും വേണം മഹോത്സവം\\
ഭൂപാലരേയും വരുവാന്‍ നിയോഗിക്ക\\
ശോഭയോടെ രാഘവാഭിഷേകാര്‍ഥമായ്\\
ഇത്ഥം സുമന്ത്രരേയും നിയോഗിച്ചതി-\\
സത്വരം തേരില്‍ക്കരേറി വസിഷ്ഠനും\\
ദാശരഥിഗൃഹമെത്രയും ഭാസ്വര-\\
മാശു സന്തോഷേണ സമ്പ്രാപ്യ സാദരം\\
നിന്നതുനേരമറിഞ്ഞു രഘുവരന്‍\\
ചെന്നുടന്‍ ദണ്ഡനമസ്കാരവും ചെയ്താന്‍.\\
രത്നാസനവും കൊടുത്തിരുത്തീ തദാ\\
പത്നിയോടുമതിഭക്ത്യാ രഘൂത്തമന്‍\\
പോല്ക്കലശസ്ഥിതനിര്‍മലവാരിണാ\\
തൃക്കാല്‍ കഴുകിച്ചു പാദാബ്ജതീര്‍ഥവും\\
ഉത്തമാംഗേന ധരിച്ചു വിശുദ്ധനായ്\\
ചിത്തമോദേന ചിരിച്ചരുളിച്ചെയ്തു:\\
‘പുണ്യവാനായേതടിയനതീവ കേ-\\
ളിന്നു പാദോദകതീര്‍ഥം ധരിക്കയാല്‍.’\\
എന്നിങ്ങനെ രാമചന്ദ്രവാക്യം കേട്ടു\\
നന്നായ്ച്ചിരിച്ചു വസിഷ്ഠനരുള്‍ചെയ്തു:\\
‘നന്നുനന്നെത്രയും നിന്നുടെ വാക്കുക-\\
ളൊന്നുണ്ടുചൊല്ലുന്നിതിപ്പോള്‍ നൃപാത്മജ!\\
ത്വല്‍പാദപങ്കജതീര്‍ഥം ധരിക്കയാല്‍\\
ദര്‍പ്പകവൈരിയും ധന്യനായീടിനാന്‍.\\
ത്വല്‍പ്പാദതീര്‍ഥവിശുദ്ധനായ് വന്നിതു\\
മല്‍പ്പിതാവായ വിരിഞ്ചനും ഭൂപതേ!\\
ഇപ്പോള്‍ മഹാജനങ്ങള്‍ക്കുപദേശാര്‍ഥ-\\
മത്ഭുതവിക്രമ! ചൊന്നതു നീയെടോ!\\
നന്നായറിഞ്ഞിരിക്കുന്നിതു നിന്നെ ഞാ-\\
നിന്നവനാകുന്നതെന്നതുമിന്നെടോ!\\
സാക്ഷാല്‍ പരബ്രഹ്മമാം പരമാത്മാവു\\
മോക്ഷദന്‍ നാനാജഗന്മയനീശ്വരന്‍\\
ലക്ഷ്മീഭഗവതിയോടും ധരണിയി-\\
ലിക്കാലമത്ര ജനിച്ചിതു നിശ്ചയം.\\
ദേവകാര്യാര്‍ഥസിദ്ധ്യര്‍ഥം കരുണയാ\\
രാവണനെക്കൊന്നു താപം കെടുപ്പാനും\\
ഭക്തജനത്തിന്നു മുക്തിസിദ്ധിപ്പാനു-\\
മിത്ഥമവതരിച്ചീടിന ശ്രീപതേ!\\
ദേവകാര്യാര്‍ഥമതീവ ഗുഹ്യംപുന-\\
രേവം വെളിച്ചത്തിടാഞ്ഞിതു ഞാനിദം\\
കാര്യങ്ങളെല്ലാമനുഷ്ഠിച്ചു സാധിക്ക\\
മായയാ മായാമനുഷ്യനായ് ശ്രീനിധേ!\\
ശിഷ്യനല്ലോ ഭവാനാചാര്യനേഷ ഞാന്‍\\
ശിക്ഷിക്കവേണം ജഗദ്ധിതാര്‍ഥം പ്രഭോ!\\
സാക്ഷാല്‍ ചരാചരാചാര്യനല്ലോ ഭവാ-\\
നോര്‍ക്കില്‍ പിതൃണാം പിതാമഹനും ഭവാന്‍.\\
സര്‍വേഷ്വഗോചരനായന്തര്യാമിയായ്\\
സര്‍വജഗദ്യന്ത്രവാഹകനായ നീ\\
ശുദ്ധതത്ത്വാത്മകമായൊരു വിഗ്രഹം\\
ധൃത്വാ നിജാധീനസംഭവനായുടന്‍\\
മര്‍ത്ത്യവേഷേണ ദശരഥപുത്രനായ്\\
പൃഥ്വീതലേ യോഗമായയാ ജാതനാം.\\
എന്നതു മുന്നേ ധരിച്ചിരിക്കുന്നു ഞാ-\\
നെന്നോടു ധാതാവു താനരുള്‍ചെയ്കയാല്‍.\\
എന്നതറിഞ്ഞത്രേ സൂര്യാന്വയത്തിനു\\
മുന്നേ പുരോഹിതനായിരുന്നു മുദാ.\\
ഞാനും ഭവാനോടു സംബന്ധകാംക്ഷയാ\\
നൂനം പുരോഹിതകര്‍മമനുഷ്ഠിച്ചു.\\
നിന്ദ്യമായുള്ളതു ചെയ്താലൊടുക്കത്തു\\
നന്നായ് വരികിലതും പിഴയല്ലല്ലോ.\\
ഇന്നു സഫലമായ്വന്നു മനോരഥ-\\
മൊന്നപേക്ഷിക്കുന്നതുണ്ടു ഞാനിന്നിയും.\\
യോഗേശ! തേ മഹാമായാഭഗവതി\\
ലോകൈകമോഹിനി മോഹിപ്പിയായ്ക മാം\\
ആചാര്യനിഷ്കൃതികാമന്‍ ഭവാനെങ്കി-\\
ലാശയം മായയാ മോഹിപ്പിയായ്ക മേ.\\
ത്വല്‍ പ്രസംഗാല്‍ സര്‍വമുക്തമിപ്പോളിദ-\\
മപ്രവക്തവ്യം മയാ രാമ! കുത്രചില്‍.\\
രാജാ ദശരഥന്‍ ചൊന്നതു കാരണം\\
രാജീവനേത്ര! വന്നേനിവിടേക്കു ഞാന്‍.\\
ഉണ്ടഭിഷേകമടുത്ത നാളെന്നതു\\
കണ്ടു ചൊല്‍വാനായുഴറിവന്നേനഹം.\\
വൈദേഹിയോടുമുപവാസവും ചെയ്തു\\
മേദിനിതന്നില്‍ ശയനവും ചെയ്യണം.\\
ബ്രഹ്മചര്യത്തോടിരിക്ക, ഞാനോരോരോ\\
കര്‍മങ്ങള്‍ ചെന്നങ്ങൊരുക്കുവന്‍ വൈകാതെ\\
വന്നീടുഷസ്സിനു നീ“യെന്നരുള്‍ചെയ്തു\\
ചെന്നു തേരില്‍ കരേറി മുനിശ്രേഷ്ഠനും.\\
പിന്നെ ശ്രീരാമനും ലക്ഷ്മണന്‍തന്നോടു\\
നന്നേ ചിരിച്ചരുള്‍ ചെയ്തു രഹസ്യമായ്:\\
“താതനെനിക്കഭിഷേകമിളമയായ്\\
മോദേന ചെയ്യുമടുത്തനാള്‍ നിര്‍ണയം\\
തത്ര നിമിത്തമാത്രം ഞാനതിന്നൊരു\\
കര്‍ത്താവു നീ രാജ്യഭോക്താവും നീയത്രേ\\
വത്സ! മമ ത്വം ബഹിഃ പ്രാണനാകയാ-\\
ലുത്സവത്തിന്നു കോപ്പിട്ടു കൊണ്ടാലും നീ\\
മത്സമനാകുന്നതും ഭവാന്‍ നിശ്ചയം\\
മത്സരിപ്പാനില്ലിതിന്നു നമ്മോടാരും.”\\
ഇത്തരമോരോന്നരുള്‍ചെയ്തിരിക്കുമ്പോള്‍\\
പൃത്ഥ്വീന്ദ്രഗേഹം പ്രവിശ്യ വസിഷ്ഠനും\\
വൃത്താന്തമെല്ലാം ദശരഥന്‍തന്നോടു\\
ചിത്തമോദാലറിയിച്ചു സമസ്തവും.\\
രാജീവസംഭവനന്ദനന്‍തന്നോടു\\
രാജാ ദശരഥനാനന്ദപൂര്‍വകം\\
രാജീവനേത്രാഭിഷേകവൃത്താന്തങ്ങള്‍\\
പൂജാവിധാനേന ചൊന്നതു കേള്‍ക്കയാല്‍\\
കൗസല്യയോടും സുമിത്രയോടും ചെന്നു\\
കൗതുകമോടറിയിച്ചാനൊരു പുമാന്‍\\
സമ്മോദമുള്‍ക്കൊണ്ടതുകേട്ട നേരത്തു\\
നിര്‍മലമായൊരു മാല്യവും നല്കിനാള്‍.\\
കൗസല്യയും തനയാഭ്യുദയാര്‍ഥമായ്\\
കൗതുകമോടു പൂജിച്ചിതു ലക്ഷ്മിയെ.\\
‘നാഥേ! മഹാദേവീ! നീയേ തുണ’ യെന്നു\\
ചേതസി ഭക്ത്യാ വണങ്ങി വാണീടിനാള്‍.
\end{verse}

%%4). abhishekavighnam
\section{അഭിഷേകവിഘ്നം}

\begin{verse}
‘സത്യസന്ധന്‍ നൃപവീരന്‍ ദശരഥന്‍\\
പുത്രാഭിഷേകം കഴിച്ചീടുമെന്നുമേ\\
കേകയപുത്രീവശഗതനാകയാ-\\
ലാകുലമുള്ളില്‍ വളരുന്നിതേറ്റവും\\
ദുര്‍ഗേ! ഭഗവതി! ദുര്‍ഷ്കൃതനാശിനി!\\
ദുര്‍ഗതി നീക്കിത്തുണച്ചീടുകംബികേ!\\
കാമുകനല്ലോ നൃപതി ദശരഥന്‍\\
കാമിനി കൈകേയീചിത്തമെന്തീശ്വര!\\
നല്ലവണ്ണം വരുത്തേണ“മെന്നിങ്ങനെ\\
ചൊല്ലി വിഷാദിച്ചിരിക്കുന്നതുനേരം\\
വാനവരെല്ലാരുമൊത്തു നിരൂപിച്ചു\\
വാണീഭഗവതി തന്നോടപേക്ഷിച്ചു:\\
‘ലോകമാതാവേ! സരസ്വതീ! ഭാരതീ!\\
വേഗാലയോധ്യക്കെഴുന്നള്ളുകവേണം.\\
രാമാഭിഷേകവിഘ്നം വരുത്തീടുവാ-\\
നാമവരാരും മറ്റില്ല നിരൂപിച്ചാല്‍\\
ചെന്നുടന്‍ മന്ഥരതന്നുടെ നാവിന്മേല്‍-\\
ത്തന്നെ വസിക്കവളെക്കൊണ്ടു ചൊല്ലിച്ചു\\
പിന്നെ വിരവോടു കൈകേയിയെക്കൊണ്ടു-\\
തന്നെ പറയിച്ചുകൊണ്ടു മുടക്കണം.\\
പിന്നെയിങ്ങോട്ടെഴുന്നള്ളാം മടിക്കരു-\\
തെ“ന്നമരന്മാര്‍ പറഞ്ഞോരനന്തരം\\
വാണിയും മന്ഥരതന്‍ വദനാന്തരേ\\
വാണീടിനാള്‍ ചെന്നു ദേവകാര്യാര്‍ഥമായ്.\\
അപ്പോള്‍ ത്രിവക്രയാം കുബ്ജയും മാനസേ\\
കല്പിച്ചുറച്ചുടന്‍ പ്രാസാദമേറിനാള്‍\\
വേഗേന ചെന്നൊരു മന്ഥരയെക്കണ്ടു\\
കൈകേയിതാനുമവളോടു ചൊല്ലിനാള്‍:\\
‘മന്ഥരേ! ചൊല്ലു നീ രാജ്യമെല്ലാടവു-\\
മെന്തൊരുമൂലമലങ്കരിച്ചീടുവാന്‍?’\\
‘നാളീകലോചനനാകിയ രാമനു\\
നാളെയഭിഷേകമുണ്ടെന്നു നിര്‍ണയം.\\
ദുര്‍ഭഗേ! മൂഢേ! മഹാഗര്‍വിതേ! കിട-\\
ന്നെപ്പോഴും നീയുറങ്ങീടൊന്നറിയാതെ\\
ഏറിയൊരാപത്തു വന്നടുത്തൂ നിന-\\
ക്കാരുമൊരു മന്ധുവില്ലെന്നു, നിര്‍ണയം.\\
രാമാഭിഷേകമടുത്തനാളുണ്ടെടോ!\\
കാമിനിമാര്‍കുലമൗലിമാണിക്യമേ!’\\
ഇത്ഥമവള്‍ ചെന്നതു കേട്ടു സംഭ്രമി-\\
ച്ചുത്ഥാനവും ചെയ്തു കേകയപുത്രിയും\\
ചിത്രമായോരു ചാമീകരനൂപുരം\\
ചിത്തമോദേന നല്‍കീടിനാളാദരാല്‍:\\
“സ്ന്തോഷമാര്‍ന്നിരിക്കുന്നകാലത്തിങ്ക\\
ലെന്തൊരു താപമുപാഗതമെന്നു നീ\\
ചൊല്ലുവാന്‍ കാരണം ഞാനറിഞ്ഞീലതി-\\
നില്ലൊരവകാശമേതും നിരൂപിച്ചാല്‍.\\
എന്നുടെ രാമകുമാരനോടം പ്രിയ-\\
മെന്നുള്ളിലാരെയുമില്ല മറ്റോര്‍ക്ക നീ.\\
അത്രയുമല്ല ഭരതനേക്കാള്‍ മമ\\
പുത്രനാം രാമനെ സ്നേഹമെനിക്കെറും\\
രാമനും കൗസല്യാദേവിയെക്കാളെന്നെ\\
പ്രേമമേറും നൂനമില്ലൊരു സംശയം.\\
ഭക്തിയും വിശ്വാസവും ബഹുമാനവു-\\
മിത്ര മാറ്റാരെയുമില്ലെന്നറിക നീ.\\
നല്ല വസ്തുക്കളെനിക്കുതന്നേ മറ്റു-\\
വല്ലവര്‍ക്കും കൊടുപ്പൂ മമ നന്ദനന്‍.\\
ഇഷ്ടമില്ലാതൊരു വാക്കു പറകയി-\\
ല്ലൊട്ടുമേ ഭെദമവനില്ലൊരിക്കലും.\\
അശ്രാന്തമെന്നെയത്രേ മടികൂടാതെ\\
ശുശ്രൂഷചെയ്തു ഞായം പ്രീതിപൂര്‍വകം.\\
മൂഢേ നിനക്കെന്തു രാമങ്കല്‍ നിന്നൊരു\\
പേടിയുണ്ടാവാനവകാശമായതും?\\
സര്‍വജനപ്രിയനല്ലോ മമാത്മജന്‍\\
നിര്‍വൈരമാനസന്‍ വാക്കുകള്‍ കേട്ടള-\\
വാകുലാചേതസാ പിന്നെയും ചൊല്ലിനാള്‍:\\
“പാപേ! മഹാഭയകാരണം കേള്‍ക്ക നീ\\
ഭൂപതി നിന്നെ വഞ്ചിച്ചതറിഞ്ഞീലേ?\\
ത്വല്‍പ്രിയനായ ഭരതനേയും ബലാല്‍\\
തല്‍പ്രിയനായ ശത്രുഘ്നനേയും നൃപന്‍\\
മാതുലനെക്കാണ്മതിന്നായയച്ചതും\\
ചേതസി കല്പിച്ചുകൊണ്ടുതന്നേയിതും.\\
രാജ്യാഭിഷേകം കൃതം രാമനെങ്കിലോ\\
രാജ്യാനുഭൂതി സൗമിത്രിക്കു നിര്‍ണയം\\
ഭാഗ്യമത്രേ സുമിത്രയ്ക്ക,തും കണ്ടു നിര്‍-\\
ഭാഗ്യയായോരു നീ ദാസിയായ് നിത്യവും\\
കൗസല്യതന്നെ പരിചരിച്ചീടുക\\
കൗസല്യാനന്ദനന്‍തന്നെ ബ്ഭരതനും\\
സേവിച്ചുകൊണ്ടു പൊറുക്കെന്നതും വരും\\
ഭാവിക്കയും വേണ്ട രാജത്വമേതുമേ.\\
നാട്ടില്‍നിന്നാട്ടിക്കളകിലുമാ,മൊരു\\
വാട്ടം വരാതെ വധിച്ചീടുകിലുമാം.\\
സാപത്ന്യജാതപരാഭവം കൊണ്ടുള്ള\\
താപവും പൂണ്ടു ധരണിയില്‍ വാഴ്കയില്‍\\
നല്ലൂ മരണമതിനില്ല സംശയം\\
ചൊല്ലുവന്‍ ഞാന്‍ തവ നല്ലതു കേള്‍ക്ക നീ.\\
ഉത്സാഹമുണ്ടു നിനക്കെങ്കിലിക്കാലം\\
ത്വത്സുതന്‍തന്നെ വാഴിക്കും നരവരന്‍\\
രാമനീരേഴാണ്ടു കാനനവാസവും\\
ഭൂമിപാലാജ്ഞയാ ചെയ്യുമാറാകണം.\\
നാടടക്കം ഭരതന്നു വരുമതി-\\
പ്രൗഢകീര്‍ത്ത്യാ നിനക്കും വസിക്കാം ചിരം.\\
വേണമെന്നാകിലതിന്നൊരുപായവും\\
പ്രാണസമേ! തവ ചൊല്ലിത്തരുവന്‍ ഞാന്‍.\\
മുന്നം സുരാസുരയുദ്ധേ ദശരഥന്‍-\\
തന്നെ മിത്രാര്‍ഥം മഹേന്ദ്രനര്‍ഥിക്കയാല്‍\\
മന്നവന്‍ ചാപബാണങ്ങളും കൈക്കൊണ്ടു,\\
തന്നുടെ സൈന്യൈസ്സമം തേരിലേറിനാന്‍.\\
നിന്നോടു കൂടവേ വിണ്ണിലകംപുക്കു\\
സന്നദ്ധനായ് ചെന്നസുരരോടേറ്റപ്പോള്‍\\
ഛിന്നമായ് വന്നു രഥാക്ഷകീലം പോരി-\\
ലെന്നതറിഞ്ഞതുമില്ല, ദശരഥന്‍.\\
സത്വരം കീലരന്ധ്രത്തിങ്കല്‍ നിന്നുടെ\\
ഹസ്തദണ്ഡം സമാവാവേശ്യ ധൈര്യേണ നീ\\
ചിത്രമത്രേ പതിപ്രാണരക്ഷാര്‍ഥമായ്\\
യുദ്ധം കഴിവോളമങ്ങനെ നിന്നതും\\
ശത്രുക്കളെ വധംചെയ്തു പൃത്ഥ്വീന്ദ്രനും\\
യുദ്ധനിവൃത്തനായോരു ദശാന്തരേ\\
നിന്‍ തൊഴില്‍ കണ്ടതിസന്തോഷമുള്‍ക്കൊണ്ടു\\
ചെന്തളിര്‍മേനി പുണര്‍ന്നു പുണര്‍ന്നുടന്‍\\
പുഞ്ചിരി പൂണ്ടു പറഞ്ഞിതു ഭൂപനും\\
‘നിന്‍ ചരിതം നന്നു നന്നു നിരൂപിച്ചാല്‍\\
രണ്ടു വരം തരാം നീയെന്നെ രക്ഷിച്ചു-\\
കൈണ്ടതുമൂലം വരിച്ചുകൊണ്ടാലും നീ.’\\
ഭര്‍ത്തൃവാക്യം കേട്ടു നീയുമന്നേരത്തു\\
ചിത്തസമ്മോദം കല‍ര്‍ന്നു ചൊല്ലീടിനാള്‍:\\
ദത്തമായോരു വരദ്വയം സാദരം\\
ന്യസ്തം ഭവതി മയാ നൃപതീശ്വരാ!\\
ഞാനൊരവസരത്തിങ്കലപേക്ഷിച്ചാ-\\
ലൂനം വരാതെ തരികെന്നതേ വേണ്ടു.’\\
എന്നു പറഞ്ഞിരിക്കുന്ന വരദ്വയ-\\
മിന്നപേക്ഷിച്ചുകൊള്ളേണം മടിയാതെ:\\
ഞാനും മറന്നു കിടന്നിതു മുന്നമേ\\
മാനസേ തോന്നീ ബലാലീശ്വരാജ്ഞയാ\\
ധീരതയോടിനി ക്ഷിപ്രമിപ്പോള്‍ ക്രോധാ-\\
ഗാരം പ്രവിശ്യ കോപേന, കിടക്ക നീ.\\
ആഭരണങ്ങളും പൊട്ടിച്ചെറിഞ്ഞതി-\\
ശോഭ പൂണ്ടോരു കാര്‍കൂന്തലഴിച്ചിട്ടു\\
പൂമേനിയും പൊടികൊണ്ടങ്ങണിഞ്ഞിഹ\\
ഭൂമിയില്‍ത്തന്നെ മലിനാംബരത്തോടും\\
കണ്ണുനീരാലേ മുഖവും മുലകളും\\
നന്നായ് നനച്ചു, കരഞ്ഞുഉ കരഞ്ഞുകൊ-\\
ണ്ടര്‍ഥിച്ചുകൊള്‍ക വരദ്വയം ഭൂപതി\\
സത്യം പറഞ്ഞാലുറപിച്ചു മാനസം”\\
മഥര ചൊന്നപോലേയതിനേതുമൊ-\\
രന്തരം കൂടാതെ ചെന്നു കൈകേയിയും\\
പത്ഥ്യമിതൊക്കെത്തനിക്കെന്നു കല്പിച്ചു\\
ചിത്തമോഹേന കോപാലയം മേവിനാള്‍.\\
കൈകേയി മന്ഥരയോടു ചൊന്നാ“ളിനി\\
രാഘവന്‍ കാനനത്തിന്നു പോവോളവും\\
ഞാനിവിടെ കിടന്നീടുഅവനല്ലായ്കില്‍\\
പ്രാണനേയും കളഞ്ഞീടുവന്‍ നിര്‍ണയം\\
ഭൂപരിത്രാണാര്‍ത്ഥമിന്നു ഭരതനു\\
ഭൂപതി ചെയ്താനഭിഷേകമെങ്കില്‍ ഞാന്‍\\
വേറെ നിനക്കു ഭോഗാര്‍ത്ഥമായ് നല്‍കുവന്‍\\
നൂറു ദേശങ്ങളതിനില്ല സംശയം.”\\
‘ഏതുമിതിനൊരിളക്കം വരായ്കില്‍ നീ\\
ചേതസി ചിന്തിച്ച കാര്യം വരും ദൃഢം’\\
എന്നു പറഞ്ഞു പോയീടിനാള്‍ മന്ഥര\\
പിന്നെയവ്വണ്ണമനുഷ്ഠിച്ചു രാജ്ഞിയും.\\
ധീരനായേറ്റം ദയാന്വിതനായ് ഗുണാ-\\
ചാരസംയുക്തനായ് നീതിജ്ഞനായ് നിജ-\\
ദേശികവാക്യസ്ഥിതനായ് സുശീലനാ-\\
യാശയശുദ്ധനായ് വിദ്യാനിരതനായ്\\
ശിഷ്ടനായുള്ളവനെന്നങ്ങിരിക്കിലും\\
ദുഷ്ടസംഗംകൊണ്ടു കാലാന്തരത്തിനാല്‍\\
സജ്ജനനിന്ദ്യനായ് വന്നുകൂടും ദൃഢം.\\
ദുര്‍ജനസംസര്‍ഗമേറ്റമകലവേ\\
വര്‍ജിക്കവേണം പ്രയത്നേന സത്പുമാന്‍\\
കജ്ജളം പറ്റിയാല്‍ സ്വര്‍ണവും നിഷ്പ്രഭം.\\
എങ്കിലോ രാജാ ദശരഥനാദരാല്‍\\
പങ്കജനേത്രാഭ്യുദയം നിമിത്തമായ്\\
മന്ത്രിപ്രഭൃതികളോടും പറഞ്ഞുകൊ-\\
ണ്ടന്തഃപുരമകം പുക്കരുളീടിനാന്‍\\
അന്നേരമാത്മപ്രിയതമയാകിയ\\
തന്നുടെ പത്നിയെക്കാണായ്ക കാരണം\\
എത്രയും വിഹ്വലനായൊരു ഭൂപനും\\
ചിത്തതാരിങ്കല്‍ നിരൂപിച്ചിതീദൃശം:\\
‘മന്ദസ്മിതം ചെയ്തരികേ വരുംപുരാ\\
സുന്ദരിയാമവളിന്നെങ്ങു പോയിനാള്‍?\\
മന്ദമാകുന്നിതുന്മേഷമെന്മാനസേ.”\\
“ചൊല്ലുവിന്‍ ദാസികളേ! ഭവത്സ്വാമിനി\\
കല്യാണഗാത്രി മറ്റെങ്ങുപോയീടിനാള്‍?\\
ഏവം നരപതി ചോദിച്ച നേരത്തു\\
ദേവിതന്നാളികളും പറഞ്ഞീടിനാര്‍:\\
“ക്രോധാലയം പ്രവേശിച്ചിതതിന്മൂല-\\
മേതുമറിഞ്ഞീല ഞങ്ങളോ മന്നവ!\\
തത്ര ഗത്വാ നിന്തിരുവടി ദേവിതന്‍-\\
ചിത്തമനുസരിച്ചീടുക വൈകാതെ.\\
എന്നതു കേട്ടു ഭയേന മഹീപതി\\
ചെന്നങ്ങരികത്തിരുന്നു, സസംഭ്രമം\\
മന്ദമന്ദം തലോടിത്തലോടി, ’പ്രിയേ\\
സുന്ദരീ! ചൊല്ലുചൊല്ലെന്തിതു വല്ലഭേ!\\
നാഥേ! വെറും നിലത്തുള്ള പൊടിയണി-\\
ഞ്ഞാതങ്കമോടു കിടക്കുന്നതെന്തു നീ!\\
ചേതോവിമോഹനരൂപേ! ഗുണശീലേ?\\
ഖേദമുണ്ടായതെന്തെന്നോടു ചൊല്കെടോ.\\
മല്‍പ്രജാവൃന്ദമായുള്ളവരാരുമേ\\
വിപ്രിയം ചെയ്കയുമില്ല നിനക്കെടോ\\
നാരികളോ നരന്മാരോ ഭവതിയോ-\\
ടാരൊരു വിപ്രിയം ചെയ്തതു വല്ലഭേ!\\
ദണ്ഡ്യനെന്നാകിലും വധ്യനെന്നാകിലും\\
ദണ്ഡമെനിക്കതിനില്ല നിരൂപിച്ചാല്‍\\
നിര്‍ദ്ധനനെത്രയുമിഷ്ടന്‍ നിനക്കെങ്കി-\\
ലര്‍ത്ഥപതിയാക്കിവെപ്പനവനെ ഞാന്‍\\
അര്‍ഥവാനേറ്റമനിഷ്ടന്‍ നിനക്കെങ്കില്‍\\
നിര്‍ദ്ധനനാക്കുവേനെന്നുമവനെ ഞാന്‍\\
വധ്യനെ നൂനമവധ്യനാക്കീടുവന്‍\\
വധ്യനാക്കിടാമവധ്യനെ വേണ്ടുകില്‍\\
നൂനം നിനക്കധീനം മമ ജീവനം\\
മാനിനീ! ഖേദിപ്പതിനെന്തു കാരണം?\\
‘മല്‍പ്രാണനെക്കാള്‍ പ്രിയതമനാകുന്ന-\\
തിപ്പോളെനിക്കു മല്‍പുത്രനാം രാഘവന്‍\\
അങ്ങനെയുള്ള രാമന്‍ മമ നന്ദനന്‍\\
മംഗലശീലനാം ശ്രീരാമനാണേ ഞാന്‍\\
അംഗനാരത്നമേ! ചെയ്വന്‍ തവ ഹിത-\\
മിങ്ങനെ ഖേദിപ്പിയായ്ക മാം വല്ലഭേ!\\
ഇത്ഥമ് ദശരഥന്‍ കൈകേയിതന്നോടു\\
സത്യം പറഞ്ഞതു കേട്ടു തെളിഞ്ഞവള്‍\\
കണ്ണുനീരും തുടച്ചുത്ഥാനവും ചെയ്തു\\
മന്നവന്‍ തന്നോടു മന്ദമുരചെയ്താള്‍:\\
‘സത്യപ്രതിജ്ഞനായുള്ള ഭവാന്‍ മമ\\
സത്യം പറഞ്ഞതു നേരെങ്കിലെന്നുടെ\\
പത്ഥ്യമായുള്ളതിനെപ്പറഞ്ഞീടുവന്‍\\
വ്യര്‍ത്ഥമാക്കീടായ്ക സത്യത്തെ മന്നവ!\\
‘എങ്കിലോ പണ്ടു സുരാസുരായോധനേ\\
സങ്കടം തീര്‍ത്തു രക്ഷിച്ചേന്‍ ഭാവാനെ ഞാന്‍\\
സന്തുഷ്ട ചിത്തനായന്നു ഭവാന്‍ മമ\\
ചിന്തിച്ചു രണ്ടു വരങ്ങള്‍ നല്‍കീലയോ?\\
വേണ്ടുന്ന നാളപേക്ഷിക്കുന്നതുണ്ടെന്നു\\
വേണ്ടും വരങ്ങള്‍ തരികെന്നു ചൊല്ലി ഞാന്‍\\
വെച്ചിരിക്കുന്നു ഭവാങ്കലതു രണ്ടു-\\
മിച്ഛയുണ്ടിന്നു വാങ്ങീടുവാന്‍ ഭൂപതേ!\\
എന്നതിലൊന്നു രാമ്യാഭിഷേകം ഭവാ-\\
നിന്നു ഭരതനു ചെയ്യേണമെന്നതും\\
പിന്നെ മറ്റേതു രാമന്‍ വനവാസത്തി-\\
നിന്നുതന്നെ ഗമിക്കേണമെന്നുള്ളതും.\\
ഭൂപതിവീരന്‍ ജടാവല്ക്കലം പൂണ്ടു\\
താപസവേഷം ധരിച്ചു വനാന്തരേ\\
കാലം പതിന്നാലുമ്വത്സരം വാഴണം\\
മൂലഫലങ്ങള്‍ ഭുജിച്ചു മഹീപതേ!\\
ഭൂമിപാലിപ്പാന്‍ ഭരതനെയാക്കണം\\
രാമനുഷസി വനത്തിനു പോകണം\\
എന്നിവ രണ്ടു വരങ്ങളും നല്‍കുകി-\\
ലിന്നു മരണമെനിക്കില്ല നിര്‍ണയം\\
എന്നു കൈകേയി പറഞ്ഞോരനന്തരം\\
മന്നവന്‍ മോഹിച്ചു വീണാനവനിയില്‍\\
വജ്രമേറ്റദ്രി പതിച്ചപോലെ ഭുവി\\
സജ്വരചേതസാ വീണീതു ഭൂപനും.\\
പിന്നെ മുഹൂര്‍ത്തമാത്രം ചെന്ന നേരത്തു\\
കണ്ണുനീര്‍ വാര്‍ത്തു വിറച്ചു നൃപാധിപന്‍.\\
‘ദുസ്സഹവാക്കുകള്‍ കേള്‍ക്കായതെന്തയ്യോ!\\
ദുസ്സ്വപ്നമാഹന്ത! കാണ്‍കയോ ഞാനിഹ!\\
ചിത്തഭ്രമം ബലാലുണ്ടാകയോ മമ\\
മൃത്യുസമയമുപസ്ഥിതമാകയോ?\\
കിം കിമേതല്‍ കൃതം ശങ്കര! ദൈവമേ!\\
പങ്കജലോചന ഹാ, പരബ്രഹ്മമേ!’\\
വ്യാഘ്രിയെപ്പോലെ സമീപേവസിക്കുന്ന\\
മൂര്‍ഖമതിയായ കൈകേയിതന്മൂഖം\\
നോക്കി നോക്കി ബ്ഭയം പൂണ്ടു ദശരഥന്‍\\
ദീര്‍ഘമായ് വീര്‍ത്തുവീര്‍ത്തേവമുരചെയ്തു.\\
‘എന്തിവണ്ണം പറയുന്നിതു ഭദ്രേ! നീ-\\
യെന്തു നിന്നോടു പിഴച്ചിതു രാഘവന്‍?\\
മല്‍പ്രാണഹാനികരമായ വാക്കു നീ-\\
യിപ്പോളുരചെയ്വതിനെന്തു കാരണം?\\
എന്നോടു രാമഗുണങ്ങളെ വര്‍ണിച്ചു\\
മുന്നമെല്ലാ നീ പറഞ്ഞു കേള്‍പ്പുണ്ടു ഞാന്‍.\\
‘എന്നെയും കൗസല്യാദേവിയെയുമവന്‍-\\
തന്നുള്ളിലില്ലൊരു ഭേദമൊരിക്കലും’\\
എന്നല്ലോ മുന്നം പറഞ്ഞിരുന്നു, നിന-\\
ക്കിന്നിതു തോന്നുവാനെന്തൊരു കാരണം?\\
‘നിന്നുടെ പുത്രനു രാജ്യം തരാമല്ലോ\\
ധന്യശീലേ രാമന്‍ പോകണമെന്നുണ്ടോ?\\
രാമനാലേതും ഭയം നിനക്കുണ്ടാകാ\\
ഭൂമിപതിയായ് ഭരതനിരുന്നാലും’\\
എന്നു പറഞ്ഞു കരഞ്ഞു കരഞ്ഞു പോയ്\\
ചെന്നുടന്‍ കാല്‍ക്കല്‍ വീണു മഹീപാലനും\\
നേത്രങ്ങളും ചുവപ്പിച്ചു കൈകേയിയും\\
ധാത്രീപതീശ്വരനോടു ചൊല്ലീടിനാള്‍:\\
‘ഭ്രാന്തനെന്നാകയോ ഭൂമീപതേ! ഭവാന്‍\\
ഭ്രാന്തിവാക്യങ്ങള്‍ ചൊല്ലുന്നതെന്തിങ്ങനെ?\\
ഘോരങ്ങളായ നരകങ്ങളില്‍ച്ചെന്നു-\\
ചേരുമസത്യവാക്യങ്ങള്‍ ചൊല്ലീടിനാല്‍.\\
പങ്കജനേത്രനാം രാമനുഷസ്സിനു\\
ശങ്കവിഹീനം വനത്തിനു പോകായ്കില്‍\\
എന്നുടെ ജീവനെ ഞാന്‍ കളഞ്ഞീതുവന്‍\\
മന്നവന്‍ മുമ്പില്‍നിന്നില്ലൊരു സംശയം.\\
സത്യസന്ധന്‍ ഭുവി രാജാ ദശരഥ-\\
നെത്രയുമെന്നുള്ള കീര്‍ത്തി രക്ഷിക്കണം\\
സാധുമാര്‍ഗത്തെ വെടിഞ്ഞതു കാരണം\\
യാതനാദുഃഖാനുഭൂതിയുണ്ടാക്കേണ്ട.\\
രാംഓപരി ഭവാന്‍ ചെയ്ത ശപഥവും\\
ഭൂമിപതേ വൃഥാ മിഥ്യയാക്കിടൊലാ.\\
കൈകേയിതന്നുടെ നിര്‍ബന്ധവാക്യവും\\
രാഘവനോടു വിയോഗം വരുന്നതും\\
ചിന്തിച്ചു ദുഃഖസമുദ്രേ നിമഗ്നനായ്\\
സന്താപമോടു മോഹിച്ചു വീണിടിനാന്‍\\
പിന്നെയുണര്‍ന്നിരുന്നും കിടന്നു മകന്‍-\\
തന്നെയോര്‍ത്തും കരഞ്ഞും പറഞ്ഞും സദാ\\
രാമരാമേതി രാമേതി പ്രലാപേന\\
യാമിനി പോയിതു വത്സരതുല്യയായ്.\\
ചെന്നാരരുണോദയത്തിനു സാദരം\\
വന്ദികള്‍ ഗായകന്മാരെന്നിവരെല്ലാം\\
മംഗലവാദ്യസ്തുതിജയശബ്ദേന\\
സംഗീതഭേദങ്ങളെന്നിവറ്റെക്കൊണ്ടും\\
പള്ളിക്കുറുപ്പുണര്‍ത്തീടിനാരന്നേര-\\
മുള്ളിലുണ്ടായ കോപേന കൈകേയിയും\\
ക്ഷിപ്രമവരെ നിവാരണവും ചെയ്താള്‍,\\
വിഭ്രമം കൈക്കൊണ്ടു നിന്നാരവര്‍കളും.\\
അപ്പോളഭിഷേകകോലാഹലാര്‍ഥമായ്\\
തല്‍പുരമൊക്കെ നിറഞ്ഞു ജനങ്ങളാല്‍\\
ഭൂമിദേവന്മാരും ഭൂമിപാലന്മാരും\\
ഭൂമിസ്പൃശോ വൃഷലാദിജനങ്ങളും\\
താപസവര്‍ഗവും കന്യകാവൃന്ദവും\\
ശോഭതേടുന്ന വെണ്‍കൊറ്റക്കുട തഴ\\
ചാമരം താലവൃന്തം കൊടിതോരണം\\
ചാമീകരാഭരണാദ്യലങ്കാരവും\\
വാരണവാജി രഥങ്ങള്‍ പദാദിയും\\
വാരനാരീജനം പൗരജനങ്ങളും\\
ഹേമരത്നോജ്ജ്വല ദിവ്യസിംഹാസനം\\
ഹേമകുംഭങ്ങളും ശാര്‍ദ്ദൂലചര്‍മവും\\
മറ്റും വസിഷ്ഠന്‍ നിയോഗിച്ചതൊക്കവേ\\
കുറ്റമൊഴിഞ്ഞാശു സംഭരിച്ചീടിനാര്‍.\\
സ്ത്രീബാലവൃദ്ധാവധിപുരവാസിക-\\
ളാബദ്ധകൗതൂഹലാബ്ധി നിമഗ്നരായ്\\
രാത്രിയില്‍ നിദ്രയും കൈവിട്ടു മാനസേ\\
ചീര്‍ത്ത പരമാനന്ദത്തോടു മേവിനാര്‍.\\
നമ്മുടെ ജീവനാം രാമകുമാരനെ\\
നിര്‍മലരത്നകിരീടമണിഞ്ഞതി-\\
രമ്യമകരായിതമണികുണ്ഡല-\\
സമുഗ്ദ്ധശോഭിതഗണ്ഡസ്ഥലങ്ങളും\\
പുണ്ഡരീകച്ഛദലോചനഭംഗിയും\\
പുണ്ഡരീകാരാതിമണ്ഡലതുണ്ഡവും\\
ചന്ദ്രികാസുന്ദരമന്ദസ്മിതാഭയും\\
കുന്ദമുകുളസമാനദന്തങ്ങളും\\
ബന്ധുകസൂനസമാനാധരാഭയും\\
കന്ധരരാജിതകൗസ്തുഭരത്നവും\\
ബന്ധുരാഭം തിരുമാറുമുദരവും\\
സന്ധ്യാഭ്രസന്നിഭപീതാംബരാഭയും\\
പൂഞ്ചോലമീതേ വിളങ്ങി മിന്നീടുന്ന\\
കാഞ്ചനകാഞ്ചികളും തനുമധ്യവും\\
കുംഭീകുലോത്തമന്‍ തുമ്പിക്കരം കണ്ടു\\
കുമ്പിട്ടു കൂപ്പീടു മൂരുകാണ്ഡങ്ങളും\\
കുംഭീന്ദ്രമസ്തകസന്നിഭജാനുവു-\\
മംഭോജബാണനിഷംഗാഭജംഘയും\\
കമ്പം കലര്‍ന്നു കമഠപ്രവരനും\\
കുമ്പിടുന്നോരു പുറവടി ശോഭയും\\
അംഭോജതുല്യമാമംഘ്രിതലങ്ങളും\\
ജംഭാരിരത്നം തൊഴും തിരുമേനിയും\\
ഹാരകടകവലയാംഗുലീയാദി\\
ചാരുതരാഭരണാവലിയും പൂണ്ടു\\
വാരണവീരന്‍ കഴുത്തില്‍ തിറത്തോടു\\
ഗൗരാതപത്രം ധരിച്ചരികേ നിജ\\
ലക്ഷ്മണനാകിയ സോദരന്‍തന്നോടും\\
ലക്ഷ്മീനിവാസനാം രാമചന്ദ്രന്‍ മുദാ\\
കാണായ്വരുന്നു നമുക്കിനിയെന്നിദം\\
മാനസതാരില്‍ കൊതിച്ച നമുക്കെല്ലാം\\
ക്ഷോണീപതിസുതനാകിയ രാമനെ-\\
ക്കാണായ്വരും പ്രഭാതേ ബത നിര്‍ണയം.\\
രാത്രിയാം രാക്ഷസി പോകുന്നതില്ലെന്നു\\
ചീര്‍ത്ത വിഷാദമോടൗത്സുക്യമുള്‍ക്കൊണ്ടു\\
മാര്‍ത്താണ്ഡദേവനെക്കാണാഞ്ഞു നോക്കിയും\\
പാര്‍ത്തുപാര്‍ത്താനന്ദപൂര്‍ണാമൃതാബ്ധിയില്‍\\
വീണു മുഴുകിയും പിന്നെയും പൊങ്ങിയും\\
വാണീടിനാര്‍ പുരവാസികളാദരാല്‍.
\end{verse}

%%5). vicchinnaabhishekam

\section{വിച്ചിന്നാഭിഷേകം}

\begin{verse}
അന്നേരമാദിത്യനുമുദിച്ചീടിനാന്‍\\
മന്നവന്‍ പള്ളിക്കുറുപ്പുണര്‍ന്നീലിന്നും\\
എന്തൊരു മൂലമതിനെന്നു മാനസേ\\
ചിന്തിച്ചു ചിന്തിച്ചു മന്ദമന്ദം തദാ\\
മന്ത്രിപ്രവരനാകുന്ന സുമന്ത്രരു-\\
മന്തഃപുരമകം പുക്കാനതിദ്രുതം.\\
‘രാജീവമിത്രഗോത്രോദ്ഭൂത! ഭൂപതേ!\\
രാജരാജേന്ദ്രപ്രവര! ജയ ജയ!\\
ഇത്ഥം നൃപനെ സ്തുതിച്ചു നമസ്കരി-\\
ച്ചുത്ഥാനവുംചെയ്തു വന്ദിച്ചുനിന്നപ്പോള്‍\\
എത്രയും ഖിന്നനായ് കണ്ണുനീരും വാര്‍ത്തു\\
പൃത്ത്വിയില്‍ത്തന്നെ കിടക്കും നരേന്ദ്രനെ\\
ചിത്താകുലതയാ കണ്ടു സുമന്ത്രരും\\
സത്വരം കൈകേയിതന്നോടു ചോദിച്ചാന്‍:\\
‘ദേവനാരീസമേ! രാജപ്രിയതമേ!\\
ദേവി കൈകേയി! ജയ ജയ സന്തതം\\
ഭൂലോകപാലന്‍ പ്രകൃതി പകരുവാന്‍\\
മൂലമെന്തൊന്നു മഹാരാജവല്ലഭേ!\\
ചൊല്ലൂകെന്നോടെ’ന്നു കേട്ടു കൈകേയിയും\\
ചൊല്ലിനാളാശു സുമന്ത്രരോടന്നേരം:\\
‘ധാത്രീപതീന്ദ്രനു നിദ്രയുണ്ടായീല\\
രാത്രിയിലെന്നതു കാരണമാകയാല്‍\\
സ്വസ്ഥനല്ലാതെ ചമഞ്ഞിതു തന്നുടെ\\
ചിത്തത്തിനസ്വതന്ത്രത്വം ഭവിക്കയാല്‍\\
രാമരാമേതി രാമേതി ജപിക്കയും\\
രാമനെത്തന്നെ മനസി ചിന്തിക്കയും\\
ഉദ്യല്‍പ്രജാഗരസേവയും ചെയ്കയാ-\\
ലത്യന്തമാകുലനായിതു മന്നവന്‍.\\
രാമനെക്കാണാഞ്ഞു ദുഃഖം നൃപേന്ദ്രനു\\
രാമനെച്ചെന്നു വരുത്തുക വൈകാതെ.\\
എന്നതുകെട്ടു സുമന്ത്രരും ചൊല്ലിനാന്‍:\\
‘ചെന്നു കുമാരനെക്കൊണ്ടുവരാമല്ലോ\\
രാജവചനമനാകര്‍ണ്യ ഞാനിഹ\\
രാജീവലോചനേ! പോകുന്നതെങ്ങനെ?’\\
എന്നതു കേട്ടു ഭൂപാലനും ചൊല്ലിനാന്‍:\\
‘ചെന്നു നീ തന്നെ വരുത്തുക രാമനെ\\
സുന്ദരനായൊരു രാമകുമാരനാം\\
നന്ദനന്‍ തന്മുഖം വൈകാതെ കാണണം’\\
എന്നതു കേട്ടു സുമന്ത്രരുഴറിപ്പോയ്-\\
ച്ചെന്നു കൗസല്യാസുതനോടു ചൂല്ലിനാന്‍:\\
‘താതന്‍ ഭവാനെകുണ്ടല്ലോ വിളിക്കുന്നു\\
സാദരം വൈകാതെഴുന്നള്ളുക വേണം’\\
മന്ത്രിപ്രവരവാക്യം കേട്ടു രാഘവന്‍\\
മന്ദേതരമവന്‍തന്നോടു കൂടവേ\\
സൗമിത്രിയോടും കരേറി രാഥോപരി\\
പ്രേമവിവശനാം താതന്‍ മരുവീടും\\
മന്ദിരേ ചെന്നു പിതാവിന്‍ പദദ്വയം\\
വന്ദിച്ചു വീണു നമസ്കരിച്ചീടിനാന്‍.\\
രാമനെച്ചെന്നെടുത്താലിംഗനം ചെയ്വാന്‍\\
ഭൂമിപനാശു സമുത്ഥായ സംഭ്രമാല്‍\\
ബാഹുക്കള്‍ നീട്ടിയ നേരത്തു ദുഃഖേന\\
മോഹിച്ചു ഭൂമിയില്‍ വീണിതു ഭൂപനും\\
രാമരാമേതി പറഞ്ഞു മോഹിച്ചൊരു\\
ഭൂമിപനെക്കണ്ടു വേഗേന രാഘവന്‍\\
താതനെച്ചെന്നെടുത്താശ്ലേഷവും ചെയ്തു\\
സാദരം തന്റെ മടിയില്‍ കിടത്തിനാന്‍.\\
നാരീജനങ്ങളതു കണ്ടനന്തര-\\
മാരൂഢശോകം വിലാപം തുടങ്ങിനാര്‍\\
രോദനം കേട്ടു വസിഷ്ഠമുനീന്ദ്രനും\\
ഖേദന മന്ദിരം പുക്കിതു സത്വരം\\
ശ്രീരാമദേവനും ചോദിച്ചിതന്നേരം\\
‘കാരണമെന്തൊന്നു താതദുഃഖത്തിനു\\
നേരെപറവിനറിഞ്ഞവരെ’ന്നതു-\\
നേരം പറഞ്ഞിതു കേകയപുത്രിയും:\\
‘കാരണം താതദുഃഖത്തിനു നീതന്നെ\\
പാരില്‍ സുഖം ദുഃഖമൂലമല്ലോ നൃണം.\\
ചേതസി നീ നിരൂപിക്കിലെളുതിനി-\\
ത്താതനു ദുഃഖനിവൃത്തി വരുത്തുവാന്‍\\
ഭര്‍ത്തൃദുഃഖോപശാന്തിക്കു കിഞ്ചില്‍ ത്വയാ\\
കര്‍ത്തവ്യമായൊരു കര്‍മമെന്നായ് വരും\\
സത്യവാദിശ്രേഷ്ഠനായ പിതാവിനെ\\
സത്യപ്രതിജ്ഞനാക്കീടുക നീയതു\\
ചിത്തഹിതം നൃപതീന്ദ്രനു നിര്‍ണയം\\
പുത്രരില്‍ ജ്യേഷ്ഠനാകുന്നതു നീയല്ലോ\\
രണ്ടു വരം മമ ദത്തമായിട്ടുണ്ടു\\
പണ്ടു നിന്‍ താതനാല്‍ സന്തുഷ്ടചേതസാ\\
നിന്നാലെ സാധ്യമായുള്ളോന്നതു രണ്ടു-\\
മിന്നു താരേണമെന്നര്‍ത്ഥിക്കയും ചെയ്തേന്‍\\
നിന്നോടതു പറഞ്ഞീടുവാന്‍ നാണിച്ചു\\
ഖിന്നനായ് വന്നിതു താതനറിക നീ.\\
സത്യപാശേന സംബദ്ധനാം താതനെ\\
സത്വരം രക്ഷിപ്പതിനു യോഗ്യന്‍ ഭവാന്‍\\
പുന്നാമമാകും നരഗത്തില്‍ നിന്നുടന്‍\\
തന്നുടെ താതനെ ത്രാണനംചെയ്കയാല്‍\\
പുത്രനെന്നുള്ള ശബ്ദം വിധിച്ചു ശത-\\
പത്രസമുത്ഭവനെന്നതറിക നീ.’\\
മാതൃവചനശൂലാഭിഹതനായ\\
മേദിനീപാലകുമാരനാം രാമനും\\
എത്രയുമേറ്റം വ്യഥിതനായ് ചൊല്ലിനാന്‍\\
‘ഇത്രയെല്ലാം പറയേണമോ മാതാവേ!\\
താതാര്‍ത്ഥമായിട്ടു ജീവനെത്തന്നെയും\\
മാതാവുതന്നെയും സീതയെത്തന്നെയും\\
ഞാനുപേക്ഷിപ്പനതിനില്ല സംശയം\\
മാനസേ ഖേദമതിനില്ലെനിക്കേതും\\
രാജ്യമെന്നാകിലും താതന്‍ നിയോഗിക്കില്‍\\
ത്യാജ്യമെന്നാലെന്നറിക നീ മാതാവേ!\\
ലക്ഷ്മണന്‍തന്നെ ത്യജിക്കെന്നു ചൊല്കിലും\\
തല്‍ക്ഷണം ഞാനുപേക്ഷിപ്പനറിക നീ\\
പാവകന്‍ തങ്കല്‍പ്പതിക്കേണമെങ്കിലു-\\
മേവം വിഷം കുടിക്കേണമെന്നാകിലും\\
താതന്‍ നിയോഗിക്കിലേതുമേ സംശയം\\
ചേതസി ചെറ്റില്ലെനിക്കെന്നറിക നീ\\
താതകാര്യമനാജ്ഞപ്തമെന്നാകിലും\\
മോദേന ചെന്നുന്ന നന്ദനനുത്തമന്‍\\
പിത്രാ നിയുക്തനായിട്ടു ചെയ്യുന്നവന്‍\\
മധ്യമനായുള്ള പുത്രനറിഞ്ഞാലും\\
ഉക്തമെന്നാകിലുമിക്കാര്യമെന്നാലെ\\
കര്‍ത്തവ്യമല്ലെന്നു വെച്ചടങ്ങുന്നവന്‍\\
പിത്രോര്‍മലമെന്നു ചൊല്ലുന്നു സജ്ജന-\\
മിത്ഥമെല്ലാം പരിജ്ഞാതം മയാധുനാ.\\
ആകയാല്‍ താതനിയോഗമനുഷ്ഠിപ്പാ-\\
നാകുലമേതുമെനിക്കില്ല നിര്‍ണയം\\
സത്യം കരോമ്യഹം സത്യം കരോമ്യഹം\\
സത്യം മയോക്തം മറിച്ചു രണ്ടായ്വരാ!\\
രാമപ്രതിജ്ഞ കേട്ടോരു കൈകേയിയും\\
രാമനോടാശു ചൊല്ലീടിനാളാദരാല്‍:\\
‘താതന്‍ നിനക്കഭിഷേകാര്‍ഥമായുട-\\
നാദരാല്‍ സംഭരിച്ചോരു സംഭാരങ്ങള്‍-\\
കൊണ്ടഭിഷേകം ഭരതനു ചെയ്യണം;\\
രണ്ടാംവരം പിന്നെയൊന്നുണ്ടു വേണ്ടുന്നു\\
നീ പതിന്നാലു സംവത്സരം കാനനേ\\
താപസവേഷേണ വാഴുകയും വേണം.\\
നിന്നോടതു നിയോഗിപ്പാന്‍ മടിയുണ്ടു\\
മന്നവനിന്നതു ദുഃഖമാകുന്നതും.”\\
എന്നതു കേട്ടു ശ്രീരാമനും ചൊല്ലിനാന്‍:\\
“ഇന്നതിനെന്തൊരു വൈഷമ്യമായതു”\\
ചെകഭിഷേകം ഭരതനു ഞാനിനി\\
വൈകാതെ പോവന്‍ വനത്തിനു മാതാവേ!\\
എന്തതെന്നോടു ചൊല്ലാഞ്ഞു പിതാവതു\\
ചിന്തിച്ചു, ദുഃഖിപ്പതിനെന്തു കാരണം?\\
രാജ്യത്തെ രക്ഷിപ്പതിനു മതിയവന്‍\\
രാജ്യമുപേക്ഷിപ്പതിനു ഞാനും മതി.\\
ദണ്ഡമത്രേ രാജ്യഭാരം വഹിപ്പതു\\
ദണ്ഡകവാസത്തിനേറ്റമെളുഅതല്ലോ.\\
സ്നേഹമെന്നെക്കുറിച്ചേറുമമ്മയ്ക്കുമ-\\
ദ്ദേഹംമാത്രം ഭരിക്കെന്നു വിധിക്കയാല്‍.\\
ആകാശഗംഗയെപ്പാതാളലോകത്തു\\
വേഗേന കൊണ്ടുചെന്നാക്കി ഭഗീരഥന്‍\\
തൃപ്തി വരുത്തി പിതൃക്കള്‍ക്കു, പൂരുവും\\
തൃപ്തനാക്കീടിനാന്‍ താതനു തന്നുടെ\\
യൗവനംനല്‍ക്ജ്ജരാനരയും വാങ്ങി\\
ദിവ്യന്മാരായാര്‍, പിതൃപ്രസാദത്തിനാല്‍.\\
അല്പമായുള്ളോരു കാര്യം നിരൂപിച്ചു\\
മല്‍പ്പിതാ ദുഃഖിപ്പതിനില്ലവകാശം.’\\
രാഘവവാക്യമേവം കേട്ടു ഭൂപതി\\
ശോകേന നന്ദനന്‍ തന്നോടു ചൊല്ലിനാന്‍:\\
സ്ത്രീജിതനായതികാമുകനായൊരു\\
രാജാധമനാകുമെന്നെയും വൈകാതെ\\
പാശേന ബന്ധിച്ചു രാജ്യം ഗ്രഹിക്ക നീ\\
ദോഷം നിനക്കതിനേതുമകപ്പെടാ.\\
അല്ലായ്കിലെന്നോടു സത്യദോഷം പറ്റു-\\
മല്ലോ കുമാര! ഗുണാംബുധേ! രാഘവ!\\
പൃത്ഥ്വീപതീന്ദ്രന്‍ ദശരഥനും പുന-\\
രിത്ഥം പറഞ്ഞു കരഞ്ഞുതുടങ്ങിനാന്‍:\\
“ഹാ രാമ! ഹാ ജഗന്നാഥ! ഹാ ഹാ രാമ!\\
ഹാ രാമ! ഹാ ഹാ മമ പ്രാണവല്ലഭ!\\
നിന്നെപ്പിരിഞ്ഞു പൊറുക്കുന്നതെങ്ങനെ?\\
എന്നെപ്പിരിഞ്ഞു നീ ഘോരമഹാവനം-\\
തന്നില്‍ ഗമിക്കുന്നതെങ്ങനെ നന്ദന?\\
എന്നിത്തരം പലജാതി പറകയും\\
കണ്ണുനീരോലോലവാര്‍ത്തു കരകയും\\
നന്നായ് മുറുകെ മുറുകെത്തഴുകയും\\
പിന്നെച്ചുടുചുടെ ദീര്‍ഘമായ് വീര്‍ക്കയും\\
ഖിന്നനായോരു പിതാവിനെക്കണ്ടുടന്‍\\
തന്നുടെ കൈയാല്‍ കുളിര്‍ത്ത ജലംകൊണ്ടു\\
കണ്ണും മുഖവും തുടച്ചു രഘൂത്തമന്‍\\
ആശ്ലേഷ നീതിവാഗ്വൈഭവാദ്യങ്ങളാ-\\
ലാശ്വസിപ്പിച്ചാന്‍ നയകോവിദന്‍ തദാ.\\
“എന്തിനെന്‍ താതന്‍ വൃഥൈവ ദുഃഖിക്കുന്ന-\\
തെന്തൊരു ദണ്ഡമിതിന്നു മഹീപതേ!\\
സത്യത്തെ രക്ഷിച്ചുകൊള്ളുവാന്‍ ഞങ്ങള്‍ക്കു\\
ശക്തിപോരായ്കയുമില്ലിതു രണ്ടിനും.\\
സോദരന്‍ നാടുഭരിച്ചിരുന്നീടുക\\
സാദരം ഞാനരണ്യത്തിലും വാഴുവന്‍.\\
ഓര്‍ക്കിലീ രാജ്യഭാരം വഹിക്കുന്നതില്‍-\\
സൗഖ്യമേറും വനത്തിങ്കല്‍ വാണീടുവാന്‍.\\
ഏതുമേ ദണ്ഡമില്ലാതെ കര്‍മം മമ\\
മാതാവെനിക്കു വിധിച്ചതു നന്നല്ലോ.\\
മാതാവു കൗസല്യതന്നെയും വന്ദിച്ചു\\
മൈഥിലിയോടും പറഞ്ഞിനി വൈകാതെ\\
പോവതിന്നായ് വരുന്നേ’നെന്നരുള്‍ചെയ്തു\\
ദേവനും മാതൃഗേഹം പുക്കതു നേരം.\\
ധാര്‍മികയാകിയ മാതാ സുസമ്മത-\\
ബ്രാഹ്മണരെക്കൊണ്ടു ഹോമപൂജാദികള്‍\\
പുത്രാഭ്യുദയത്തിനായ്ക്കൊണ്ടു ചെയ്യിച്ചു.\\
വിത്തമതീവ ദാനങ്ങള്‍ ചെതാദരാല്‍\\
ഭക്തികൈക്കൊണ്ടു ഭഗവല്‍പ്പദാംബുജം\\
ചിത്തത്തില്‍ നന്നായുറപ്പിച്ചിളകാതെ\\
നന്നായ് സമാധിയുറച്ചിരിക്കുന്നേരം\\
ചെന്നൊരു പുത്രനെയും കണ്ടതില്ലല്ലോ.\\
അന്തികേ ചെന്നു കൗസല്യയോടന്നേരം\\
സന്തോഷമോടു സുമിത്ര ചൊല്ലീടിനാള്‍:\\
“രാമനുപഗതനായതു കണ്ടീലേ\\
ഭൂമിപാലപ്രിയേ! നോക്കീടുകെ,ന്നപ്പോള്‍\\
വന്ദിച്ചു നില്ക്കുന്ന രാമകുമാരനെ\\
മന്ദേതരം മുറുകപ്പുണര്‍ന്നീടിനാള്‍.\\
പിന്നെ മടിയിലിരുത്തി നെറുകയില്‍\\
നന്നായ് മുകര്‍ന്നു മുകര്‍ന്നു കുതൂഹലാല്‍\\
ഇന്ദീവരദളശ്യാമകളേബരം\\
മന്ദമന്ദ തലോടിപ്പറഞ്ഞീടിനാള്‍:\\
“എനെന്‍ മഗനേ! മുഖാംബുജം വാടുവാന്‍\\
ബന്ധമുണ്ടായതും, പാരം വിശക്കയോ?\\
വന്നിരുന്നീടു ഭുജിപ്പതിനാശു നീ’-\\
യെന്നു മാതാവു പറഞ്ഞോരന്തരം\\
വന്നശോകത്തെയടക്കി രഘുവരന്‍\\
തന്നുടെ മാതാവിനോടരുളിച്ചെയ്തു\\
‘ഇപ്പോള്‍ ഭുജിപ്പാനവസരമില്ലമ്മേ!\\
ക്ഷിപ്രമരണ്യവാസത്തിനു പോകണം.\\
മുല്പാടു കേകയപുത്രിയാമമ്മയ്ക്കു\\
മല്‍പിതാ രണ്ടു വരം കൊടുത്തീടിനാന്‍\\
ഒന്നു ഭരതനെ വാഴിക്കയെന്നതു-\\
മെന്നെ വനത്തിന്നയയ്ക്കെന്നു മറ്റേതും\\
തത്ര പതിന്നാലു സംവത്സരം വസി-\\
ച്ചത്ര വന്നീടുവന്‍ പിന്നെ ഞാന്‍ വൈകാതെ.\\
സന്താപമേതും മനസ്സിലുണ്ടാകാതെ,\\
സന്തുഷ്ടയായ് വസിച്ചീടുക മാതാവും.’\\
ശ്രീ രാമവാക്യവേവം കേട്ടു കൗസല്യ\\
പാരില്‍ മോഹിച്ചു വീണീടിനാളാകുലാല്‍.\\
പിന്നെ മോഹം തീര്‍ന്നിരുന്നു ദുഃഖാര്‍ണവം-\\
തന്നില്‍ മുഴുകിക്കരഞ്ഞു കരഞ്ഞുടന്‍\\
തന്നുടെ നന്ദനന്‍ തന്നോടു ചൊല്ലിനാ-\\
‘ളിന്നു നീ കാനനത്തിന്നു പോയീടുകില്‍\\
എന്നെയും കൊണ്ടുപോകേണം മടിയാതെ\\
നിന്നെപിരിഞ്ഞാല്‍ ക്ഷണാര്‍ദ്ധം പൊറുക്കുമോ?’\\
ദണ്ഡകാരണ്യത്തിനാശുന്നുനീ പോകില്‍ ഞാന്‍\\
ദണ്ഡധരാലയത്തിന്നു പോയീടുവന്‍.\\
പൈതലേ വേര്‍വിട്ടുപോയ പശുവിനു-\\
ള്ളാധി പറഞ്ഞറിയിച്ചീടരുതല്ലോ\\
നാടുവാഴേണം ഭരതനെന്നാകില്‍ നീ\\
കാടുവാഴേണമെന്നുണ്ടോ വിധിമതം?\\
എന്തു പിഴച്ചിതു കൈകേയിയോടു നീ\\
ചിന്തിക്ക ഭൂപനോടും കുമാരാ! ബലാല്‍\\
താതനും ഞാനുമൊക്കും ഗുരുത്വം കൊണ്ടു\\
ഭേദം നിനക്കു ചെറ്റില്ലെന്നു നിര്‍ണയം\\
പോകണമെന്നു താതന്‍ നിയോഗിക്കില്‍, ഞാന്‍\\
പോകരുതെന്നു ചെറുക്കുന്നതുണ്ടല്ലോ\\
എന്നുടെ വാക്യത്തെ ലംഘിച്ചു ഭൂപതി-\\
തന്നുടെ വാചാ ഗമിക്കുന്നതാകിലോ,\\
ഞാനുമെന്‍ പ്രാണങ്ങളെ ത്യജിച്ചീടുവന്‍\\
മാനവവംശവും പിന്നെ മുടിഞ്ഞുപോം.’\\
തത്ര കൗലസ്യാവചനങ്ങിളിങ്ങനെ\\
ചിത്തതാപേന കേട്ടോരു സൗമിത്രിയും\\
ശോകരോഷങ്ങള്‍ നിറഞ്ഞ നേത്രാഗ്നിനാ\\
ലോകങ്ങളെല്ലാം ദഹിച്ചുപോകുംവണ്ണം\\
രാഘവന്‍തന്നെ നോക്കിപ്പറഞ്ഞീടിനാന്‍:\\
‘ആകുലമെന്തിതു കാരണമുണ്ടാവാന്‍?\\
ഭ്രാന്തചിത്തം ജഡം വൃദ്ധം വധുജിതം\\
ശാന്തേതരം ത്രപാഹീനം ശഠപ്രിയം\\
ബന്ധിച്ചുതാതനേയും പിന്നെ ഞാന്‍ പരി-\\
പന്ഥികളായുള്ളവരേയുമൊക്കവേ\\
അന്തകന്‍വീട്ടിന്നയച്ചഭിഷേകമൊ-\\
രന്തരംകൂടാതെ സാധിച്ചുകൊള്ളുവന്‍\\
ബന്ധമില്ലേതുമിതിന്നു ശോകിപ്പതി-\\
നന്തര്‍മൃദാ വസിച്ചീടുക മാതാവേ!\\
ആര്യപുത്രാഭിഷേകം കഴിച്ചീടുവന്‍\\
ശൗര്യമെനിക്കതിനുണ്ടെന്നു നിര്‍ണയം\\
കാര്യമെല്ലാത്തതുചെയ്യുന്നതാകിലോ-\\
ചാര്യനും ശാസനം ചെയ്കെന്നതേ വരൂ.\\
ഇത്ഥം പറഞ്ഞു ലോകത്രയം തദ്രുഷാ\\
ദഗ്ദ്ധമാമ്മാറു സൗമിത്രി നില്ക്കുന്നേരം\\
മന്ദഹാസം ചെയ്തു മന്ദേതരം ചെന്നു\\
നന്ദിച്ചു ഗാഢമായാലിംഗനംചെയ്തു\\
സുന്ദരനിന്ദിരാമന്ദിരവത്സനാ-\\
നന്ദസ്വരൂപനിന്ദിന്ദിരവിഗ്രഹന്‍\\
ഇന്ദീവരാക്ഷനിന്ദ്രാദിവൃന്ദാരക-\\
വൃന്ദവന്ദ്യാംഘ്രിയുഗ്മാരവിന്ദന്‍ പൂര്‍ണ-\\
ചന്ദ്രബിംബാനനനിന്ദുചൂഡപ്രിയന്‍\\
വന്ദാരുവൃന്ദമന്ദാരദാരൂപമന്‍.
\end{verse}

%%6). lakshmanopadesham

\section{ലക്ഷമണോപദേശം}

\begin{verse}
വത്സ! സൗമിത്രേ! കുമാര! നീ കേള്‍ക്കണം\\
മത്സരാദ്യം വെടിഞ്ഞെന്നുടെ വാക്കുകള്‍\\
നിന്നുടെ തത്ത്വമറിഞ്ഞിരിക്കുന്നിതു\\
മുന്നമേ ഞാനെടോ, നിനുള്ളിലെപ്പോഴും\\
എന്നെക്കുറിച്ചുള്ള വാത്സല്യപൂരവും\\
നിന്നോളമില്ല മറ്റാര്‍ക്കുമെന്നുള്ളതും.\\
നിന്നാലസാധ്യമായില്ലൊരു കര്‍മവും\\
നിര്‍ണയമെങ്കിലുമൊന്നിതു കേള്‍ക്ക നീ.\\
ദൃശ്യമായുള്ളൊരു രാജ്യദേഹാദിയും\\
വിശ്വവും നിശ്ശേഷധാന്യധനാദിയും\\
സത്യമെന്നാകിലേ തല്‍പ്രയാസം തവ\\
യുക്ത, മതല്ലായ്കിലെന്തതിനാല്‍ ഫലം?\\
ഭോഗങ്ങളെല്ലാം ക്ഷണപ്രഭാചഞ്ചലം\\
വേഗേന നഷ്ടമാമായുസ്സുമോര്‍ക്ക നീ.\\
വഹ്നിസന്തപ്താലോഹസ്ഥാംബുബിന്ദുനാ\\
സന്നിഭം മര്‍ത്ത്യജന്മം ക്ഷണഭംഗുരം\\
ചക്ഷുഃശ്രവണഗളസ്ഥമാം ദര്‍ദ്ദുരം\\
ഭക്ഷണത്തിന്നപേക്ഷിക്കുന്നതുപോലെ\\
കാലാഹിനാ പരിഗ്രസ്തമാം ലോകവു-\\
മാലോലചേതസാ ഭോഗങ്ങള്‍ തേടുന്നു.\\
പുത്രമിത്രാര്‍ഥകളത്രാദിസംഗമ-\\
മെത്രയുമല്പകാലസ്ഥിതമോര്‍ക്ക നീ\\
പാന്ഥര്‍ പെരുവഴിയമ്പലം തന്നിലേ\\
താന്തരായ് കൂടി വിയോഗം വരുമ്പോലെ\\
നദ്യാമൊഴുകുന്ന കാഷ്ഠങ്ങള്‍പോലെയു-\\
മെത്രയും ചഞ്ചലമാലയസംഗമം\\
ലക്ഷ്മിയുമസ്ഥിരയല്ലോ മനുഷ്യര്‍ക്കു\\
നില്ക്കുമോ യൗവനവും പുനരധ്രുവം?\\
സ്വപ്നസമാനം കളത്രസുഖം നൃണാ-\\
മല്പമായുസ്സും നിരൂപിക്ക ലക്ഷ്മണ!\\
രാഗാദിസങ്കുലമായുള്ള സംസാര-\\
മാകെ നിരൂപിക്കില്‍ സ്വപ്നതുല്യം സഖേ!\\
ഓര്‍ക്ക ഗന്ധര്‍വനഗരസമമതില്‍\\
മൂര്‍ഖന്മാര്‍ നിത്യമനുവര്‍ത്തിച്ചീടുന്നു\\
ആദുത്യദേവനുദിച്ചിതു വേഗേന\\
യാദഃപതിയില്‍ മറഞ്ഞിതു സത്വരം\\
നിദ്രയും വന്നിതുദയശൈലോപരി\\
വിദ്രുതം വന്നിതു പിന്നെയും ഭാസ്കരന്‍.\\
ഇത്ഥം മതിഭ്രമമുളോരു ജന്തുക്കള്‍\\
ചിത്തേ വിചാരിപ്പതില്ല കാലാന്തരം\\
ആയുസ്സു പോകുന്നതേതുമറിവീല\\
മായാസമുദ്രത്തില്‍ മുങ്ങിക്കിടക്കയാല്‍.\\
വാര്‍ധക്യമോടും ജരാനരയുംപൂണ്ടു\\
ചീര്‍ത്ത മോഹേന മരിക്കുന്നിതു ചിലര്‍\\
നേത്രേന്ദ്രിയംകൊണ്ടു കണ്ടിരിക്കെ പുന-\\
രോര്‍ത്തറിയുന്നീല മായതന്‍ വൈഭവം.\\
ഇപ്പോളിതു പകല്‍ പില്പാടു രാത്രിയും\\
പില്പാടു പിന്നെപ്പകലുമുണ്ടായ്വരും\\
ഇപ്രകാരം നിരൂപിച്ചു മൂഢാത്മാക്കള്‍\\
ചില്‍പ്പുരുഷന്‍ ഗതിയേതുമറിയാതെ\\
കാലസ്വരൂപനാമീശ്വരന്‍തന്നുടെ\\
ലീലാവിശേഷങ്ങളൊന്നു മോരായകയാല്‍\\
ആമകുംഭാബുസമാനമായുസ്സുടന്‍\\
പോമതേതും ധരിക്കുന്നതില്ലാരുമേ.\\
രോഗങ്ങളായുള്ള ശത്രുക്കളും വന്നു\\
ദേഹം നശിപ്പിക്കുമേവനും നിര്‍ണയം.\\
വ്യാഘ്രയെപ്പോലെ ജരയുമടുത്തുവ-\\
ന്നാക്രമിച്ചീടും ശരീരത്തെ നിര്‍ണയം.\\
മൃത്യുവുംകൂടൊരുനേരം പിരിയാതെ\\
ഛിദ്രവും പാര്‍ത്തുപാര്‍ത്തുള്ളീലിരിക്കുന്നു.\\
ദേഹം നിമിത്തമഹംബുദ്ധി കൈക്കൊണ്ടു\\
മോഹം കലര്‍ന്നു ജന്തുക്കള്‍ നിരൂപിക്കും\\
ബ്രാഹ്മണോഹം നരേന്ദ്രോഹമാഢ്യോഹമെ-\\
ന്നാമ്രേഡിതം കലര്‍ന്നീടും ദശാന്തരേ\\
ജന്തുക്കള്‍ ഭക്ഷിച്ചു കാഷ്ഠിച്ചു പോകിലാം\\
വെന്തു വെണ്ണീറായ് ചമഞ്ഞുപോയീടിലാം\\
മണ്ണീനു കീഴായ് കൃമികളായ് പോകിലാം\\
നന്നല്ല ദേഹം നിമിത്തം മഹാമോഹം.\\
ത്വങ്മാംസരക്താസ്ഥിവിണ്‍മൂത്രരേതസാം\\
സമ്മേളനം പഞ്ചഭൂതകനിര്‍മിതം\\
മായാമയമായ് പരിണാമിയായോരു\\
കായം വികാരിയായുള്ളോന്നിതധ്രുവം.\\
ദേഹാഭിമാനം നിമിത്തമായുണ്ടായ\\
മോഹേന ലോകം ദഹിപ്പിപ്പതിന്നു നീ\\
മാനസതാരില്‍ നിരൂപിച്ചതും തവ\\
ജ്ഞാനമില്ലായ്കെന്നറിക നീ ലക്ഷ്മണ!\\
ദോഷങ്ങളൊക്കവേ ദേഹാഭിമാനിനാം\\
രോഷേണ വന്നു ഭവിക്കുന്നിതോര്‍ക്ക നീ\\
ദേഹോഹമെന്നുള്ള ബുദ്ധി മനുഷ്യര്‍ക്ക്\\
മോഹമാതാവാമവിദ്യയാകുന്നതും\\
ദേഹമല്ലോര്‍ക്കില്‍ ഞാനായതാത്മാവെന്നു\\
മോഹൈകഹന്ത്രിയായുള്ളതു വിദ്യ കേള്‍\\
സംസാരകാരിണിയായതവിദ്യയും\\
സംസാരനാശിനിയായതു വിദ്യയും.\\
ആകയാല്‍ മോക്ഷാര്‍ഥിയാകില്‍ വിദ്യാഭ്യാസ-\\
മേകാന്തചേതസാ ചെയ്ക വേണ്ടുന്നതും\\
തത്ര കാമക്രോധലോഭമോഹാദികള്‍\\
ശത്രുക്കളാകുന്നതെന്നുമറിക നീ\\
മുക്തിക്കു വിഘ്നം വരുത്തുവാനെത്രയും\\
ശ്ക്തിയുള്ളോന്നതില്‍ ക്രോധമറികെടോ\\
മാതാപിതൃഭ്രാതൃവിത്രസഖികളെ\\
ക്രോധം നിമിത്തം ഹനിക്കുന്നതു പുമാന്‍.\\
ക്രോധമൂലം മനസ്താപമുണ്ടായ് വരും\\
ക്രോധമൂലം നൃണാം സംസാരബന്ധനം\\
ക്രോധമല്ലോ നിജ ധര്‍മക്ഷയകരം\\
ക്രോധം പരിത്യജിക്കേണം ബുധജനം.\\
ക്രോധമല്ലോ യമനായതു നിര്‍ണയം\\
വൈതരണ്യാഖ്യയാകുന്നതു തൃഷണയും\\
സന്തോഷമാകുന്നതു നന്ദനം വനം\\
സന്തതം ശാന്തിയേ കാമസുരഭി കേള്‍\\
ചിന്തിച്ചു ശാന്തിയെത്തന്നെ ഭജിക്ക നീ\\
സന്താപമെന്നാലൊരുജാതിയും വരാ.\\
ദേഹേന്ദ്രിയപ്രാണവുദ്ധ്യാദികള്‍ക്കെല്ലാ-\\
മഹന്ത മേലേ വസിപ്പതാത്മാവു കേള്‍.\\
ശുദ്ധസ്വയംജ്യോതിരാനന്ദപൂര്‍ണമായ്\\
തത്ത്വാര്‍ഥമായ് നിരാകാരമായ് നിത്യമായ്\\
നിര്‍വികല്പം പരം നിര്‍വികാരം ഘനം\\
സര്‍വൈകകാരണം സര്‍വജഗന്മയം\\
സര്‍വൈകസാക്ഷിണം സര്‍വജ്ഞാമീശ്വരം\\
സര്‍വദാ ചേതസി ഭാവിച്ചു കേള്‍ക നീ.\\
സാരജ്ഞനായ നീ കേള്‍ സുഖദുഃഖദം\\
പ്രാരബ്ധമെല്ലാമനുഭവിച്ചീടണം.\\
കര്‍മേന്ദ്രിയങ്ങളാല്‍ കര്‍ത്തവ്യമൊക്കവേ\\
നിര്‍മായമാചരിച്ചീടുകെന്നേ വരൂ.\\
കര്‍മങ്ങള്‍ സംഗങ്ങളൊന്നിലും കൂടാതെ\\
കര്‍മഫലങ്ങളില്‍ കാംക്ഷയും കൂടാതെ\\
കര്‍മങ്ങളെല്ലാം വിധിച്ചവണ്ണം പര-\\
ബ്രഹ്മണി നിത്യേ സമര്‍പ്പിച്ചു ചെയ്യണം.\\
നിര്‍മലമായുള്ളൊരാത്മാവുതന്നോടു\\
കര്‍മങ്ങളൊന്നുമേ പറ്റുകയില്ലെന്നാല്‍\\
ഞാനിപ്പറഞ്ഞതെല്ലാമേ ധരിച്ചു തല്‍-\\
ജ്ഞാനസ്വരൂപം വിചാരിച്ചു സന്തതം\\
മാനത്തെയൊക്കെ ത്യജിച്ചു നിത്യം പര-\\
മാനന്ദമുള്‍ക്കൊണ്ടു മായാവിമോഹങ്ങള്‍\\
മാനസത്തിങ്കല്‍നിന്നാശു കളക നീ\\
മാനമല്ലോ പരമാപദാമാസ്പദം.”\\
സൗമിത്രിതന്നോടിവണ്ണമരുള്‍ചെയ്തു\\
സൗമുഖ്യമോടു മാതാവോടു ചൊല്ലിനാന്‍:\\
“കേള്‍ക്കണമമ്മേ! തെളിഞ്ഞു നീയെന്നുടെ\\
വാക്കുകളേതും വിഷാദമുണ്ടാകൊലാ.\\
ആത്മാവിനേതുമേ പീഡയുണ്ടാക്കരു-\\
താത്മാവിനെയറിയാതവരെപ്പോലെ.\\
സര്‍വലോകങ്ങളിലും വസിച്ചീടുന്ന\\
സര്‍വജനങ്ങളും തങ്ങളില്‍ തങ്ങളില്‍\\
സര്‍വദാ കൂടിവാഴ്കെന്നുള്ളതില്ലല്ലോ\\
സര്‍വജ്ഞയല്ലോ ജനനി! നീ കേവലം.\\
ആശു പതിന്നാലു സംവത്സരം വന-\\
ദേശേ വസിക്കു വരുന്നതുമുണ്ടു ഞാന്‍.\\
ദുഃഖങ്ങളെല്ലാമകലെക്കളഞ്ഞുട-\\
നുള്‍ക്കനിവോടുമനുക്രഹിച്ചീടണം.\\
അച്ഛനെന്തുള്ളിലൊന്നിച്ഛയെന്നാലതി-\\
ങ്ങിച്ഛയെന്നങ്ങുറച്ചീടണമമ്മയും.\\
ഭര്‍ത്തൃകര്‍മാനുകരണമത്രേ പാതി-\\
വ്രതനിഷ്ഠാ വധൂനാമെന്നു നിര്‍ണയം.\\
മാതാവു മോദാലനുവദിച്ചീടുകി-\\
ലേതുമേ ദുഃഖമെനിക്കില്ല കേവലം.\\
കനനവാസം സുഖമായ് വരും തവ\\
മാനസേ ഖേദം കുറച്ചു വാണീടുകില്‍”\\
എന്നു പറഞ്ഞു നമസ്കരിച്ചീടിനാന്‍\\
പിന്നെയും പിന്നെയും മാതൃപാദാന്തികേ.\\
പ്രീതി കൈക്കൊണ്ടെടുത്തുത്സംഗസീമ്നി ചേര്‍-\\
ത്താദരാല്‍ മൂര്‍ദ്ധ്നി ബാഷ്പാഭിഷേകം ചെയ്തു\\
ചൊല്ലിനാളാശീര്‍വചനങ്ങളാശു കൗ-\\
സല്യയും ദേവകളോടിരന്നീടിനാള്‍:\\
“സൃഷ്ടികര്‍ത്താവേ! വിരിഞ്ച! പത്മാസന!\\
പുഷ്ടദയാബ്ധേ! പുരുഷോത്തമ! ഹരേ!\\
മൃത്യുഞ്ജയ! മഹാദേവ! ഗൗരീപതേ!\\
വൃത്രാരിമുമ്പായ ദിക്പാലകന്മാരേ!\\
ദുര്‍ഗേ! ഭഗവതീ! ദുഃഖവിനാശിനീ!\\
സര്‍ഗസ്ഥിതിലയകാരിണീ! ചണ്ഡികേ!\\
എന്മകനാശു നടക്കുന്ന നേരവും\\
കല്മഷം തീര്‍ന്നിരുന്നീടുന്ന നേരവും\\
തന്മതി കെട്ടുറങ്ങീടുന്ന നേരവും\\
സമ്മോദമാര്‍ന്നു രക്ഷിച്ചീടുവിന്‍ നിങ്ങാള്‍.”\\
ഇത്ഥമര്‍ത്ഥിച്ചു തന്‍ പുത്രനാം രാമനെ-\\
ബ്ബദ്ധബാഷ്പം ഗാഢഗാഢം പുണര്‍ന്നുടന്‍,\\
‘ഈരേഴുസംവത്സരം കാനനേ വസി-\\
ച്ചാരാല്‍ വരികെ’ന്നനുവദിച്ചീടിനാള്‍.\\
തല്‍ക്ഷണേ രാഘവം നത്വാ സഗദ്ഗദം\\
ലക്ഷ്മണന്‍ താനും പറഞ്ഞാനനാകുലം:\\
‘എന്നുള്ളിലുണ്ടായിരുന്നൊരു സംശയം\\
നിന്നരുളപ്പാടു കേട്ടു തീര്‍ന്നൂ തുലോം.\\
ത്വല്‍പ്പാദസേവാര്‍ത്ഥമായിന്നടിയനു-\\
മിപ്പോള്‍ വഴിയേ വിടകൊള്‍വനെന്നുമേ.\\
മോദാലതിന്നായനുവദിച്ചീടണം\\
സീതാപതേ! രാമചന്ദ്ര! ദയാനിധേ!\\
പ്രാണങ്ങളെക്കളഞ്ഞീടുവനല്ലായ്കി-\\
ലേണാങ്കതുല്യവദന! രഘുപതേ!”\\
“എങ്കില്‍ നീ പോന്നുകൊണ്ടാലും’മെന്നാദരാല്‍\\
പങ്കജലോചനന്‍താനുമരുള്‍ചെയ്തു.\\
വൈദേയിതന്നോടു യാത്ര ചൊല്ലീടുവാന്‍\\
മോദേന സീതാഗൃഹം പുക്കരുളിനാന്‍.\\
ആഗതനായ ഭര്‍ത്താവിനെക്കണ്ടവള്‍\\
വേഗേന സസ്മിതമുത്ഥാനവുംചെയ്തു\\
കാഞ്ചനപാത്രസ്ഥമായ തോയംകൊണ്ടു\\
വാഞ്ഛയാ തൃക്കാല്‍ കഴുകിച്ചു സാദരം\\
മന്ദാക്ഷമുള്‍ക്കൊണ്ടു മന്ദസ്മിതം ചെയ്തു\\
സുന്ദരി മന്ദമന്ദം പറഞ്ഞീടിനാള്‍:\\
“ആരുമകമ്പടികൂടാതെ ശ്രീപാദ-\\
ചാരേണ വന്നതുമെന്തു കൃപാനിധേ!\\
വാരണവീരനെങ്ങു മമ വല്ലഭ!\\
ഗൗരാതപത്രവും താലവൃന്താദിയും\\
ചാമരദ്വന്ദ്വവും വാദ്യഘോഷങ്ങളും\\
ചാമീകരാഭരണാദ്യലങ്കാരവും\\
സാമന്തഭൂപാലരേയും പിരിഞ്ഞതി-\\
രോമാഞ്ചമോടെഴുന്നളിയതെന്തയ്യോ!”\\
ഇത്ഥം വിദേഹാത്മജാവചനം കേട്ടു\\
പൃത്ഥ്വീപതിസുതന്‍താനുമരുള്‍ചെയ്തു:\\
“തന്നിതു ദണ്ഡകാരണ്യരാജം മമ\\
പുണ്യം വരുത്തുവാന്‍ താതനറികെടോ.\\
ഞാനതു പാലിപ്പതിന്നാശു പോകുന്നു\\
മാനസേ ഖേദമിളച്ചു വാണീടുക.\\
മാതാവു കൗസല്യതന്നെയും ശുശ്രൂഷ\\
ചെയ്തു സുഖേന വസിക്ക നീ വല്ലഭേ!”\\
ഭര്‍ത്തൃവാക്യം കേട്ടു ജാനകിയും രാമ-\\
ഭദ്രനോടിത്ഥമാഹന്ത ചൊല്ലീടിനാള്‍:\\
“രാത്രിയില്‍ കൂടെപ്പിരിഞ്ഞാല്‍ പൊറാതോള-\\
മാസ്ഥയുണ്ടല്ലോ ഭവാനെപ്പിതാവിനും\\
എന്നിരിക്കെ വനരാജ്യം തരുവതി-\\
നിന്നു തോന്നീടുവാനെന്തൊരു കാരണം?\\
മന്നവന്‍ താനല്ലയോ കൗതുകത്തോടു-\\
മിന്നലെ രാജ്യാഭിഷേകമാരംഭിച്ചു\\
സത്യമോ ചൊല്ലു ഭര്‍ത്താവേ! വിരവോടു\\
വൃത്താന്തമെത്രയും ചിത്രമോര്‍ത്താലിദം.”\\
എന്നതു കേട്ടരുള്‍ചെയ്തു രഘുവരന്‍:\\
“തന്വീകുലമൗലിമാലികേ! കേള്‍ക്ക നീ\\
മന്നവന്‍ കേകയപുത്രിയാമമ്മയ്ക്കു\\
മുന്നമേ രണ്ടു വരം കൊടുത്തീടിനാന്‍\\
വിണ്ണവര്‍നാട്ടില്‍ സുരാസുരയുദ്ധത്തി-\\
നന്യൂനവിക്രമം കൈക്കൊണ്ടുപോയ നാള്‍.\\
ഒന്നു ഭരതനെ വാഴിക്കയെന്നതു-\\
മെന്നെ വനത്തിന്നയയ്ക്കെന്നു മറ്റേതും.\\
സത്യവിരോധം വരുമെന്നു തന്നുടെ\\
ചിത്തേ നിരൂപിച്ചു പേടിച്ചു താതനും\\
മാതാവിനാശു വരവും കൊടുത്തിതു\\
താത,നതുകൊണ്ടു ഞാനിന്നു പോകുന്നു.\\
ദണ്ഡകാരണ്യേ പതിന്നാലു വത്സരം\\
ദണ്ഡമൊഴിഞ്ഞു വസിച്ചു വരുവന്‍ ഞാന്‍.\\
നീയതിന്നേതും മുടക്കം പറകൊല്ല\\
മയ്യല്‍ കളഞ്ഞു മാതാവുമായ് വാഴ്ക നീ.\\
രാഘവനിത്ഥം പറഞ്ഞതു കേട്ടൊരു\\
രാകാശശിമുഖിതാനുമരുള്‍ചെയ്തു:\\
“മുന്നില്‍ നടപ്പന്‍ വനത്തിനു ഞാന്‍ മമ\\
പിന്നാലെ വേണമെഴുന്നള്ളുവാന്‍ ഭവാന്‍.\\
എന്നെപ്പിരിഞ്ഞു പോകുന്നതുചിതമ\\
ല്ലൊന്നുകൊണ്ടും ഭവാനെന്നു ധരിക്കണം.’\\
കാകുല്‍സ്ഥനും പ്രിയവാദിനിയാകിയ\\
നാഗേന്ദ്രഗാമിനിയോടു ചൊല്ലീടിനാന്‍:\\
‘എങ്ങനെ നിന്നെ ഞാന്‍ കൊണ്ടുപോകുന്നതു\\
തിങ്ങിമരങ്ങള്‍ നിറഞ്ഞ വനങ്ങളില്‍?\\
ഘോരസിംഹവ്യാഘ്രസൂകരസൈരിഭ-\\
വാരണവ്യാളഭല്ലൂകവൃകാദികള്‍\\
മാനുഷഭോജികളായുള്ല രാക്ഷസര്‍\\
കാനനംതന്നില്‍ മറ്റും ദുഷ്ടജന്തുക്കള്‍\\
സംഖ്യയില്ലാതോളമുണ്ടവറ്റെക്കണ്ടാല്‍\\
സങ്കടം പൂണ്ടു ഭയമാം നമുക്കെല്ലാം\\
നാരീജനത്തിനെല്ലാം വിശേഷിച്ചുമൊ-\\
ട്ടേറെയുണ്ടാം ഭയമെന്നറിഞ്ഞീടെടോ!\\
മൂലഫലങ്ങള്‍ കട്വമ്ലകഷായങ്ങള്‍\\
ബാലേ! ഭുജിപ്പതിനാകുന്നതും തത്ര\\
നിര്‍മലവ്യഞ്ജനാപൂപാന്നപാനാദി\\
സന്മധുക്ഷീരങ്ങളില്ലൊരു നേരവും.\\
നിമ്നോന്നതഗുഹാഗഹ്വരശര്‍ക്കരാ-\\
ദുര്‍മാര്‍ഗമെത്രയും കണ്ടകവൃന്ദവും\\
നേരേ പെരുവഴിയുമറിയാവത-\\
ല്ലാരെയും കാണ്മാനുമില്ലറിഞ്ഞീടുവാന്‍.\\
ശീതവാതാതപപീഡയു0 പാരമാം\\
പാദചാരേണ വേണം നടന്നീടുവാന്‍.\\
ദുഷ്ടരായുള്ളൊരു രാക്ഷസരെക്കണ്ടാ-\\
ലൊട്ടും പൊറുക്കയില്ലാര്‍ക്കുമറികെടോ!\\
എന്നുടെ ചൊല്ലിനാല്‍ മാതാവു തന്നെയും\\
നന്നായ് പരിചരിച്ചിങ്ങിരുന്നീടുക.\\
വന്നീടുവന്‍ പതിന്നാലു സംവത്സരം\\
ചെന്നാലതിനുടനില്ലൊരു സംശയം”\\
ശ്രീരാമവാക്കു കേട്ടോരു വൈദേഹിയു-\\
മാരൂഢതാപേന പിന്നെയും ചൊല്ലിനാള്‍:\\
നാഥ! പതിര്വതയാം ധര്‍മപത്നി ഞാ-\\
നാധരവുമില്ല മറ്റെനിക്കാരുമേ,\\
ഏതുമേ ദോഷവുമില്ല ദയാനിധേ!\\
പാദശുശ്രൂഷാവ്രതം മുടക്കായ്ക മേ.\\
നിന്നുടെ സന്നിധൗ സന്തതം വാണീടു-\\
മെന്നെ മറ്റാര്‍ക്കാനും പീഡിച്ചു കൂടുമോ?\\
വല്ലതും മൂലഫലജലാഹാരങ്ങള്‍\\
വല്ലഭോച്ഛിഷ്ടമെനിക്കമൃതോപമം.\\
ഭര്‍ത്താവുതന്നോടുകൂടെ നടക്കുമ്പോ-\\
ളെത്രയും കൂര്‍ത്തുമൂര്‍ത്തുള്ള കല്ലും മുള്ളും\\
പുഷ്പാസ്തരണതുല്യങ്ങളെനിക്കതും\\
പുഷ്പബാണോപമ! നീ വെടിഞ്ഞീടൊലാ.\\
ഏതുമേ പീഡയുണ്ടാകയില്ലെന്മൂലം\\
ഭീതിയുമേതുമെനിക്കില്ല ഭര്‍ത്താവേ!\\
കശ്ചില്‍ ദ്വിജന്‍ ജ്യോതിശ്ശാസ്ത്രവിശാരദന്‍\\
നിശ്ചയിച്ചെന്നോടു പണ്ടരുളിച്ചെയ്തു\\
ഭര്‍ത്താവിനോടും വനത്തില്‍ വസിപ്പതി-\\
നെത്തും ഭവതിക്കു സങ്കടമില്ലേതും.\\
ഇത്ഥം പുരൈവ ഞാന്‍ കേട്ടിരിക്കുന്നതു\\
സത്യമതിന്നിയുമൊന്നു ചൊല്ലീടുവന്‍\\
രാമായണങ്ങള്‍ പലതും കവിവര-\\
രാമോദമോടു പറഞ്ഞു കേള്‍പ്പുണ്ടു ഞാന്‍\\
ജാനകിയോടു കൂടാതെ രഘുവരന്‍\\
കാനനവാസത്തിനെന്നു പോയിട്ടുള്ളൂ?\\
ഉണ്ടോ പുരുഷന്‍ പ്രകൃതിയെ വേറിട്ടു?\\
രണ്ടുമൊന്നത്രേ വിചാരിച്ചു കാണ്‍കിലോ?\\
പാണിഗ്രഹണമന്ത്രാര്‍ത്ഥവു, മോര്‍ക്കണം\\
പ്രാണാവസാനകാലത്തും പിരിയുമോ?\\
എന്നിരിക്കെ പുനരഎന്നെയുപേക്ഷിച്ചു\\
തന്നേ വനത്തിനായ്ക്കൊണ്ടെഴുന്നള്ളുകില്‍\\
എന്നുമെന്‍ പ്രാണപരിത്യാഗവും ചെയ്വ-\\
നിന്നുതന്നെ നിന്തിരുവടി തന്നാണേ.”\\
എന്നിങ്ങനെ ദേവി ചൊന്നതു കേട്ടൊരു\\
മന്നവന്‍ മന്ദസ്മിതം പൂണ്ടരുള്‍ചെയ്തു:\\
‘എങ്കിലോ വല്ലഭേ! പോരികെ വൈകാതെ\\
സങ്കടമിന്നിതു ചൊല്ലിയുണ്ടാകേണ്ടാ.\\
ദാനമരുന്ധതിക്കായ്ക്കൊണ്ടു ചെയ്ക നീ\\
ജാനകീ! ഹാരാദിഭൂഷണമൊക്കവേ.’\\
ഇത്ഥമരുള്‍ചെയ്തു ലക്ഷ്മണന്‍തന്നോടു\\
പൃത്ഥ്വീസുരോത്തമന്മാരെ വരുത്തുകെ-\\
ന്നത്യാദരമരുള്‍ചെയ്തനേരം ദ്വിജേ-\\
ന്ദ്രോത്തമന്മാരെ വരുത്തി കുമാരനും.\\
വസ്ത്രങ്ങളാഭരണങ്ങള്‍ പശുക്കളു-\\
മര്‍ത്ഥമവധിയില്ലാതോളമാദരാല്‍\\
സദ്വൃത്തരായ്ക്കുലശീലഗുണങ്ങളാ-\\
ലുത്തമന്മാരായ്ക്കുടുംബികളാകിയ\\
വേദവിജ്ഞാനികളാം ദ്വിജേന്ദ്രന്മാര്‍ക്കു\\
സാദരം ദാനങ്ങള്‍ ചെയ്തു ബഹുവിധം.\\
മാതാവുതന്നുടെ സേവകന്മാരായ\\
ഭൂദേവസത്തമന്മാര്‍ക്കും കൊടുത്തിതു\\
പിന്നെ നിജാന്തഃപുരവാസികള്‍ക്കും മ-\\
റ്റന്യരാം സേവകന്മാര്‍ക്കും വഹുവിധം.\\
ദാനങ്ങള്‍ ചെയ്കയാലാനന്ദമഗ്നരായ്\\
മാനവനായകനാശീര്‍വചനവും\\
ചെയ്തിതു താപസന്മാരും ദ്വിജന്മാരും\\
പെയ്തുപെയ്തീടുന്നിതശ്രുജലങ്ങളും.\\
ജാനകീദേവിയുമന്‍പോടരുന്ധതി-\\
ക്കാനന്ദമുള്‍ക്കൊണ്ടു ദാനങ്ങള്‍ നല്‍കിനാള്‍.\\
ലക്ഷ്മണവീരന്‍ സുമിത്രയാമമ്മയെ\\
തല്‍ക്ഷണേ കൗസല്യകൈയില്‍ സമര്‍പ്പിച്ചു\\
വന്ദിച്ചനേരം സുമിത്രയും പുത്രനെ\\
നന്ദിച്ചെടുത്തു സമാശ്ലേഷവും ചെയ്തു\\
നന്നായനുഗ്രഹംചെയ്തു തനയനു\\
പിന്നെയുപദേശവാക്കുമരുള്‍ചെയ്താള്‍:\\
‘അഗ്രജന്‍ തന്നെപ്പരിചരിച്ചെപ്പോഴു-\\
മഗ്രേനടന്നുകൊള്ളേണം പിരിയാതെ.\\
മഗ്രേനടന്നുകൊള്ളേണം പിരിയാതെ.\\
രാമനെ നിത്യം ദശരഥനെന്നുള്ളി-\\
ലാമോദമോടുനിരൂപിച്ചുകൊള്ളണം.\\
എന്നെ ജനകാത്മജയെന്നുറച്ചുകൊള്‍\\
പിന്നെയയോദ്ധ്യയെന്നോര്‍ത്തീടടവിയെ.\\
മായാവിഹീനമീവണ്ണമുറപ്പിച്ചു\\
പോയാലുമെങ്കില്‍ സുഖമായ് വരിക തേ.’\\
മാതൃവചനം ശിരസി ധരിച്ചുകൊ-\\
ണ്ടാദരവോടു തൊഴുതു സൗമിത്രിയും\\
തന്നുടെ ചാപശരാദികള്‍ കൈക്കൊണ്ടു\\
ചെന്നു രാമാന്തികേ നിന്നു വണങ്ങിനാന്‍.\\
തല്‍ക്ഷണേ രാഘവന്‍ ജാനകിതന്നൊടും\\
ലക്ഷ്മണനോടും ജനകനെ വന്ദിപ്പാന്‍\\
പോകുന്നനേരത്തു പൗരജനങ്ങളെ\\
രാഗമോടെ കടാക്ഷിച്ചു കുതൂഹലാല്‍.\\
കോമളനായ കുമാരന്‍ മനോഹരന്‍\\
ശ്യാമളരമ്യകളേബരന്‍ രാഘവന്‍\\
കാമദേവോപമന്‍ കാമദന്‍ സുന്ദരന്‍\\
രാമന്‍തിരുവടി നാനാജഗദഭി-\\
രാമനാത്മാരാമനംബുജലോചനന്‍\\
കാമാരി സേവിതന്‍ നാനാജഗന്മയന്‍\\
താതാലയംപ്രതി പോകുന്ന നേരത്തു\\
സാദം കലര്‍ന്നൊരു പൗരജനങ്ങളും\\
പാദചാരേണ നടക്കുന്നതു കണ്ടു\\
ഖേദംകലര്‍ന്നു പരസ്പരം ചൊല്ലിനാര്‍:\\
‘കഷ്ടമാഹന്ത! കഷ്ടം! പശ്യ പശ്യ ഹാ!\\
കഷ്ടമെന്തിങ്ങനെ വന്നിതു ദൈവമേ!\\
സോദരനോടും പ്രണയിനിതന്നോടും\\
പാദചാരേണ സഹായവുംകൂടാതെ\\
ശര്‍ക്കരാകണ്ടകനിമ്നോന്നതയുത-\\
ദുര്‍ഘടമായുള്ള് ദുര്‍ഗമാര്‍ഗങ്ങളില്‍\\
രക്തപത്മത്തിനു കാഠിന്യമേകുന്ന\\
മുഗ്ദ്ധമൃദുതരസ്നിഗ്ദ്ധപാദങ്ങളാല്‍\\
നിത്യം വനാന്തേ നടക്കെന്നു കല്പിച്ച\\
പൃഥ്വീശചിത്തം കഠോരമത്രേ തുലോം.\\
പുത്രവാത്സല്യം ദശരഥന്‍തന്നോളം\\
മര്‍ത്ത്യരിലാര്‍ക്കുമില്ലിന്നലെയോളവും\\
ഇന്നിതു തോന്നുവാനെന്തൊരു കാരണ’-\\
മെന്നതുകേട്ടുടന്‍ ചൊല്ലിനാനന്യനും:\\
‘കേകയപുത്രിക്കു രണ്ടു വരം നൃപ-\\
നേകിനാന്‍പോലതു കാരണം രാഘവന്‍\\
പോകുന്നിതു വനത്തിന്നു, ഭരതനും\\
വാഴ്കെന്നു വന്നുകൂടും ധരാമണ്ഡലം.\\
പോകനാമെങ്കില്‍ വനത്തിന്നു കൂടവേ\\
രാഘവന്‍തന്നെപ്പിരിഞ്ഞാല്‍ പൊറുക്കുമോ?’\\
ഇപ്രകാരം പുരവാസികളായുള്ള\\
വിപ്രാദികള്‍വാക്കു കേട്ടോരനന്തരം\\
വാമദേവന്‍ പുരവാസികള്‍തമ്മോടു\\
സാമോദമേവമരുള്‍ചെയ്തിതന്നേരം.
\end{verse}

%%7). raamaseethaathathvam

\section{രാമസീതാതത്ത്വം}

\begin{verse}
രാമനെ ച്ചിന്തിച്ചു ദുഃഖിയായ്കാരുമേ\\
കോമളഗാത്രിയാം ജാനകിമൂലവും.\\
തത്ത്വമായുള്ളതു ചൊല്ലുന്നതുണ്ടു ഞാന്‍\\
ചിത്തം തെളിഞ്ഞു കേട്ടീടുവിനേവരും.\\
രാമനാകുന്നതു സാക്ഷാല്‍ മഹാവിഷ്ണു\\
താമരസാക്ഷനാമാദിനാരായണന്‍.\\
ലക്ഷ്മണനായതനന്തന്‍ ജനകജാ\\
ലക്ഷ്മീഭഗവതി ലോകമായാ പരാ.\\
മായാഗുണങ്ങളെത്താനവലംബിച്ചു\\
കായഭേദം ധരിക്കുന്നിതാത്മാ പരന്‍\\
രാജസമായ ഗുണത്തോടു കൂടവേ\\
രാജീവസംഭവനായ് പ്രപഞ്ചദ്വയം\\
വ്യക്തമായ് സൃഷ്ടിച്ചു സത്വപ്രധാനനായ്\\
ഭക്തപരായണന്‍ വിഷ്ണുരൂപം പൂണ്ടു\\
നിത്യവും രക്ഷിച്ചുകൊള്ളുന്നതീശ്വര-\\
നാദ്യനജന്‍ പരമാത്മാവു സാദരം.\\
രുദ്രവേഷത്താല്‍ തമോഗുണയുക്തനാ-\\
യദ്രിജാവല്ലഭന്‍ സംഹരിക്കുന്നതും.\\
വൈവസ്വതന്‍ മനു ഭക്തിപ്രസന്നനായ്\\
ദെവന്‍ മകരാവതാരമനുഷ്ഠിച്ചു\\
വേദങ്ങളെല്ല്ളാം ഹയഗ്രീവനെക്കൊന്നു\\
വേധാവിനാക്കിക്കൊടുത്തതീ രാഘവന്‍.\\
പാഥോനിധിമഥനേ പണ്ടു മന്ദരം\\
പാതാളലോകം പ്രവേശിച്ചതുനേരം\\
നിഷ്ഠുരമായൊരു കൂര്‍മാകൃതി പൂണ്ടു\\
പൃഷ്ഠേ ഗിരീന്ദ്രം ധരിച്ചതീ രാഘവന്‍\\
ദുഷ്ടനായൊരു ഹിരണ്യാക്ഷനെക്കൊന്നു\\
ഘൃഷ്ടിയായ് തേറ്റമേല്‍ ക്ഷോണിയെപ്പൊങ്ങിച്ചു\\
കാരണവാരിധിതന്നില്‍ കളിച്ചതും\\
കാരണപൂരുഷനാകുമീ രാഘവന്‍.\\
നിര്‍ഹ്രാദമോടു നരസിംഹരൂപമായ്\\
പ്രഹ്ളാദനെപ്പരിപാലിച്ചുകൊള്ളുവാന്‍\\
ക്രൂരങ്ങളായ നഖരങ്ങളൊക്കൊണ്ടു\\
ഘോരനായോരു ഹിരണ്യകശിപുതന്‍\\
വക്ഷഃപ്രദേശം പ്രപാടനം ചെയ്തതും\\
രക്ഷാചതുരനാം ലക്ഷ്മീവരനിവന്‍\\
പുത്രലാഭാര്‍ഥമദിതിയും ഭക്തിപൂ-\\
ണ്ടര്‍ത്ഥിച്ചു സാദരമര്‍ച്ചിക്ക കാരണം\\
എത്രയും കാരുണ്യമോടവള്‍തന്നുടെ\\
പുത്രനായിന്ദ്രാനുജനായ് പിറന്നതി\\
ഭക്തനായോരു മഹാബലിയോടു ചെ-\\
ന്നര്‍ത്ഥിച്ചു മൂന്നടിയാക്കി ജഗത്രയം\\
സത്വരം വാങ്ങി മരുത്വാനു നല്‍കിയ\\
ഭക്തപ്രിയനാം ത്രിവിക്രമനുമിവന്‍.\\
ധാത്രീസുരദ്വേഷികളായ് ജനിച്ചൊരു\\
ധാത്രീപതികുലനാശം വരുത്തുവാന്‍\\
ധാത്രിയില്‍ ഭാര്‍ഗവനായിപ്പിറന്നതും\\
ധാത്രീവരനായ രാഘവനാമിവന്‍.\\
ധാത്രിയിലിപ്പോള്‍ ദശരഥപുത്രനായ്\\
ധാത്രീസുതാവരനായ് പിറന്നീടിനാന്‍\\
രാത്രിഞ്ചരകുലമൊക്കെ നശിപ്പിച്ചു\\
ധാത്രീഭാരം തീര്‍ത്തു ധര്‍മത്തെ രക്ഷിപ്പാന്‍\\
ആദ്യനജന്‍ പരമാത്മാ പരാപരന്‍\\
വേദ്യനല്ലാത വേദാന്തവേദ്യന്‍ പരന്‍\\
നാരായണന്‍ പുരുഷോത്തമനവ്യയന്‍\\
കാരണമാനുഷന്‍ രാമന്‍ മനോഹരന്‍\\
രാവണനിഗ്രഹാര്‍ഥം വിപിനത്തിനു\\
ദേവഹിതാര്‍ഥം ഗമിക്കുന്നതിന്നതിന്‍\\
കാരണം മന്ഥരയല്ല കൈകേയിയ-\\
ല്ലാരും ഭരമിക്കായ്ക രാജാവുമല്ലല്ലോ,\\
വിഷ്ണുഭഗവാന്‍ ജഗന്മയന്‍ മാധവന്‍\\
വിഷ്ണു മഹാമായാദേവി ജനകജാ\\
സൃഷ്ടിസ്ഥിതിലയകാരിണിതന്നൊടും\\
പുഷ്ടപ്രമോദം പുറപ്പെട്ടതിന്നിപ്പോള്‍.\\
ഇന്നലെ നാരദന്‍ വന്നു ചൊന്നാനവന്‍-\\
തന്നോടു രാഘവന്‍താനുമരുള്‍ചെയ്തു:\\
നക്തഞ്ചരാന്വയനിഗ്രഹത്തിന്നു ഞാന്‍\\
വ്യക്തം വനത്തിന്നു നാളെ പുറപ്പെടും.’\\
എന്നതുമൂലം ഗമിക്കുന്നു രാഘവ-\\
നിന്നു വിഷാദം കളവിനെല്ലാവരും\\
രാമനെച്ചിന്തിച്ചു ദുഃഖിയായ്കാരുമേ\\
രാമരാമേതി ജപിപ്പിനെല്ലാവരും.\\
നിത്യവും രാമരാമേതി ജപിക്കുന്ന\\
മര്‍ത്ത്യനു മൃത്യുഭയാദികളൊന്നുമേ\\
സിദ്ധിക്കയി,ല്ലതേയല്ല കൈവല്യവും\\
സിദ്ധിക്കുമേവനുമെന്നതു നിര്‍ണയം\\
ദുഃഖേസൗഖ്യാദി വികല്പങ്ങളില്ലാത\\
നിഷ്കളന്‍ നിര്‍ഗുണനാത്മാ രഘൂത്തമന്‍\\
ന്യൂനാതിരേകവിഹീനന്‍ നിരഞ്ജന-\\
നാനാന്ദപൂര്‍ണനനന്തനനാകുലന്‍\\
അങ്ങനെയുള്ള ഭഗവത്സ്വരൂപത്തി-\\
ലെങ്ങനെ ദുഃഖാദി സംഭവിച്ചീടുന്നു?\\
ഭക്തജനാനാം ഭജനാര്‍ഥമായ് വന്നു\\
ഭക്തപ്രിയന്‍ പിറന്നീടിനാന്‍ ഭൂതലേ\\
പങ്ക്തിരഥാഭീഷ്ടസിദ്ധ്യര്‍ഥമായ് വന്നു\\
പങ്ക്തികണ്ഠന്‍തന്നെക്കൊന്നു ജഗത്ത്രയം\\
പാലിപ്പതിന്നായവതരിച്ചീടിനാന്‍\\
ബാലിശന്മാരേ! മനുഷ്യനായീശ്വരന്‍.’\\
രാമവിഷയമീവണ്ണമരുള്‍ചെയ്തു\\
വാമദേവന്‍ വിരമിച്ചോരനന്തരം\\
വാമദേവവചനാമൃതം സേവിച്ചു\\
രാമനെ നാരായണനെന്നറിഞ്ഞുടന്‍\\
പൗരജനം പരമാനന്ദമായൊരു\\
വാരാന്നിധിയില്‍ മുഴുകിനാരേവരും.\\
രാമസീതാരഹസ്യം മുഹുരീദൃശ-\\
മാമോദപൂര്‍വകം ധ്യാനിപ്പവര്‍ക്കെല്ലാം\\
രാമദേവങ്കലുറച്ചൊരു ഭക്തിയു-\\
മാമയനാശവും സിദ്ധിക്കുമേവനും.\\
ഗോപനീയം രഹസ്യം പരമീദൃശം\\
പാപവിനാശനം ചൊന്നതിന്‍ കാരണം\\
രാമപ്രിയന്മാര്‍ ഭവാന്മാരെന്നോര്‍ത്തു ഞാന്‍\\
രാമതത്ത്വം പരമോപദേശം ചെയ്തു.\\
താപവും തീര്‍ന്നിതു പൗരജനങ്ങള്‍ക്കു\\
താപസശ്രേഷ്ഠനും മോദാലെഴുന്നള്ളി.
\end{verse}

%%8). vanayaathra

\section{വനയാത്ര}

\begin{verse}
രാഘവന്‍ താതഗേഹം പ്രവേശിച്ചുടന്‍\\
വ്യാകുലഹീനം വണങ്ങിഹരുള്‍ചെയ്തു.\\
കൈകേയിയാകിയ മാതാവുതന്നോടു:\\
‘ശോകം കളഞ്ഞാലുമമ്മേ! മനസി തേ.\\
സൗമിത്രിയും ജനകാത്മജയും ഞാനും\\
സൗമുഖ്യമാര്‍ന്നു പോവാനായ് പുറപ്പെട്ടു\\
ഖേദമകലെക്കളഞ്ഞിനി ഞങ്ങളെ\\
താതനാജ്ഞാപിക്ക വേണ്ടതു വൈകാതെ.\\
ഇഷ്ടവാക്യം കേട്ടു കൈകേയി സാദരം\\
പെട്ടെന്നെഴുന്നേറ്റിരുന്നു സസംഭ്രമം\\
ശ്രീരാമനും മൈഥിലിക്കുമനുജനും\\
ചീരങ്ങള്‍ വെവ്വേറെ നല്കിനാളമ്മയും.\\
ധന്യവസ്ത്രങ്ങളുപേക്ഷിച്ചു രാഘവന്‍\\
വന്യചീരങ്ങള്‍ പരിഗ്രഹിച്ചീടിനാന്‍\\
പുഷ്കരലോചനാനുജ്ഞയാ വല്ക്കലം\\
ലക്ഷ്മണന്‍ താനുമുടുത്താനതുനേരം\\
ലക്ഷ്മീഭഗവതിയാകിയ ജാനകി\\
വല്ക്കലം കൈയില്‍ പിടിച്ചുകൊണ്ടാകുലാല്‍\\
പക്ഷമെന്തുള്ളിലെന്നുള്ളതറിവാനായ്\\
തല്‍ക്ഷണേ ലജ്ജയാ ഭര്‍ത്തൃമുഖാംബുജം\\
ഗൂഢമായ് നോക്കിനാളെങ്ങനെ ഞാനിതു\\
ഗാഢമുടുക്കുന്നതെന്നുള്ള ചിന്തയാ.\\
മംഗലദേവതാവല്ലഭന്‍ രാഘവ-\\
നിംഗിതജ്ഞന്‍ തദാ വാങ്ങിപ്പരുഷമാം\\
വല്ക്കലം ദിവ്യാംബരോപരി വേഷ്ടിച്ചു\\
സല്‍ക്കാരമാനം കലര്‍ന്നു നിന്നീടിനാന്‍.\\
എന്നതു കണ്ടൊരു രാജദാരങ്ങളു-\\
മന്യരായുള്ള ജനങ്ങളുമൊക്കവേ\\
വന്ന ദുഃഖത്താല്‍ കരയുന്നതു കേട്ടു-\\
നിന്നരുളീടും വസിഷ്ഠമഹാമുനി\\
കോപേന ഭര്‍ത്സിച്ചു കൈകേയിതന്നോടു\\
താപേന ചൊല്ലിനാ’നെന്തിതു തോന്നുവാന്‍?’\\
ദുഷ്ടേ! നിശാചരി! ദുര്‍വൃത്തമാനസേ!\\
കഷ്ടമോര്‍ത്തോളം കഠോരശീലേ! ഖലേ!\\
രാമന്‍ വനത്തിനു പോകേണമെന്നല്ലോ\\
താമസശീലേ! വരത്തെ വരിച്ചു നീ.\\
ജാനകീദേവിക്കു വല്ക്കലം നല്‍കുവാന്‍\\
മാനസേ തോന്നിയതെന്തൊരു കാരണം?\\
ഭക്ത്യാ പത്രിവ്രതയാകിയ ജാനകി\\
ഭര്‍ത്താവിനോടുകൂടെ പ്രയാണം ചെയ്കില്‍\\
സര്‍വാഭരണവിഭൂഷിതഗാത്രിയായ്\\
ദിവ്യാംബരം പൂണ്ടനുഗമിച്ചീടുക.\\
കാനനദുഃഖനിവാരണാര്‍ഥം പതി-\\
മാനസവും രമിപ്പിച്ചു സദാകാലം\\
ഭര്‍ത്തൃശുശ്രൂഷയും ചെയ്തു പിരിയാതെ\\
ചിത്തശുദ്ധ്യാ ചരിച്ചീടുകെന്നേ വരൂ.”\\
ഇത്ഥം വസിഷ്ഠോക്തി കേട്ടു ദശരഥന്‍\\
നത്വാ സുമന്ത്രരോടേവമരുള്‍ചെയ്തു:\\
‘രാജയോഗ്യം രഥമാശു വരുത്തുക\\
രാജീവനേത്രപ്രയാണായ സത്വരം.\\
ഇത്ഥമുക്ത്വാ രാമവക്ത്രാംബുജം പാര്‍ത്തു\\
‘പുത്ര! ഹാ രാമ! സൗമിത്രേ! ജനകജേ!\\
രാമ! രാമ! ത്രിലോകാഭിരാമാംഗ! ഹാ!\\
ഹാ! മമ പ്രാണസമാന! മനോഹര!’\\
ദുഃഖിച്ചു ഭൂമിയില്‍ വീണു ദശരഥ-\\
നുള്ക്കാമ്പഴിഞ്ഞു കരയുന്നതുനേരം\\
തേരുമൊരുമിച്ചു നിര്‍ത്തി സുമന്ത്രരും\\
ശ്രീരാമദേവനുമപ്പോളരുള്‍ചെയ്തു:\\
‘തേരില്‍ കരേറുക സീതേ വിരവില്‍ നീ\\
നേരമിനിക്കളഞ്ഞീടരുതേതുമേ.’\\
സുന്ദരി വന്ദിച്ചു തേരില്‍ കരേറിനാ-\\
ളിന്ദിരാവല്ലഭനാകിയ രാമനും\\
മാനസേ ഖേദം കളഞ്ഞു ജനകനെ\\
വീണുവണങ്ങി പ്രദഷിണവും ചെയ്തു\\
താണുതൊഴുതുടന്‍ തേരില്‍ കരേറിനാന്‍:\\
ബാണചാപാസിതൂണീരാദികളെല്ലാം\\
കൈക്കൊണ്ടു വന്ദിച്ചു താനും കരേറിനാന്‍\\
ലക്ഷ്മണനപ്പോള്‍, സുമന്ത്രരുമാകുലാല്‍\\
ദുഃഖേന തേര്‍ നടത്തീടിനാന്‍, ഭൂപനും\\
നില്‍ക്ക നില്‍ക്കെന്നു ചൊന്നാന്‍, രഘുനാഥനും\\
ഗച്ഛ ഗച്ഛേതി വേഗാലരുള്‍ചെയ്തിതു;\\
നിശ്ചലമായിതു ലോകവുമന്നേരം.\\
രാജീവലോചനന്‍ ദൂരെ മറഞ്ഞപ്പോള്‍\\
രാജാവു മോഹിച്ചു വീണിതു ഭൂതലേ.\\
സ്ത്രീബാലവൃദ്ധാവധി പുരവാസികള്‍\\
താപം മുഴുത്തു വിലപിച്ചു പിന്നാലെ\\
‘തിഷ്ഠ തിഷ്ഠ പ്രഭോ! രാമ! ദയാനിധേ!\\
ദൃഷ്ടിക്കമൃതമായോരു തിരുമേനി\\
കാണായ്കിലെങ്ങനെ ഞങ്ങള്‍ പൊറുക്കുന്നു?\\
പ്രാണനോ പോയിതല്ലോ വിധി ദൈവമേ!’\\
ഇത്തരം ചൊല്ലി പ്രലപിച്ചു സര്‍വരും\\
സത്വരം തേരിന്‍ പിറകേ നടകൊണ്ടാര്‍.\\
മന്നവന്‍താനും ചിരം പ്രലപിച്ചഥ\\
ചൊന്നാന്‍ പരിചാരകന്മാരൊടാകുലാല്‍:\\
‘എന്നെയെടുത്തിനിക്കൊണ്ടുപോയ് ശ്രീരാമന്‍-\\
തന്നുടെ മാതൃഗേഹത്തിങ്കലാക്കുവിന്‍.\\
രാമനെ വേറിട്ടു ജീവിച്ചു ഞാനിനി\\
ഭൂമിയില്‍ വാഴ്കെന്നതില്ലെന്നു നിര്‍ണയം.’\\
എന്നതു കേട്ടൊരു ഭൃത്യജനങ്ങളും\\
മന്നവന്‍തന്നെയെടുത്തു കൗസല്യതന്‍\\
മന്ദിരത്തിങ്കലാക്കീടിനാരന്നേരം\\
വന്നൊരു ദുഃഖേന മോഹിച്ചു വീണിതു.\\
പിന്നെയുണര്‍ന്നു കരഞ്ഞു തുടങ്ങിനാന്‍\\
ഖിന്നയായ് മേവുന്ന കൗസല്യ തന്നൊടും.\\
ശ്രീരാമനും തമസാനദിതന്നുടെ\\
തീരം ഗമിച്ചു വസിച്ചു നിശാമുഖേ\\
പാനീയമാത്രമുപജീവനം ചെയ്തു\\
ജാനകിയോടും നിരാഹാരനായൊരു\\
വൃക്ഷമൂലേ ശയനം ചെയ്തുറങ്ങിനാന്‍\\
ലക്ഷ്മണന്‍ വില്ലുമമ്പും ധരിച്ചന്തികേ\\
രക്ഷിച്ചു നിന്നു സുമന്ത്രരുമായോരോ\\
ദുഃഖവൃത്താന്തങ്ങളും പറഞ്ഞാകുലാല്‍\\
പൗരജനങ്ങളും ചെന്നരികെ പുക്കു\\
ശ്രീരാമനെയങ്ങു കൊണ്ടുപോയ്ക്കൂടായ്കില്‍\\
കാനനവാസം നമുക്കുമെന്നേവരും\\
മാനസത്തിങ്കലുറച്ചു മരുവിനാര്‍\\
പൗരജനത്തിന്‍ പരിദേവനം കണ്ടു\\
ശ്രീരാമദേവനുമുള്ളില്‍ നിരൂപിച്ചു:\\
‘ശൂര്യനുദിച്ചാലയയ്ക്കയുമില്ലിവര്‍\\
കാര്യത്തിനുംവരും വിഘ്നമെന്നാലിവര്‍\\
ഖേദം കലര്‍ന്നു തളര്‍ന്നുറങ്ങുന്നിതു,\\
ബോധമില്ലിപ്പോളിനിയുണരും മുമ്പേ\\
പോകനാമിപ്പൊഴേ കൂട്ടുക തേരെ’ന്നു\\
രാഘവന്‍ വാക്കുകള്‍ കേട്ടു സുമന്ത്രരും\\
വേഗേന തേരുമൊരുമിച്ചിതന്നേരം\\
രാഘവന്മാരും ജനകതനൂജയും\\
തേരിലേറീടിനാരേതു മറിഞ്ഞീല\\
പൗരജനങ്ങ,ളന്നേരം സുമന്ത്രരും\\
ചെറ്റയോദ്ധ്യാഭിമുഖം ഗമിച്ചിട്ടഥ\\
തെറ്റെന്നു തെക്കോട്ടുതന്നെ നടകൊണ്ടു.\\
ചുറ്റും കിടന്ന പുരവാസികളെല്ലാം\\
പിറ്റേന്നാള്‍ തങ്ങളുണര്‍ന്നു നോക്കുന്നേരം\\
കണ്ടീല രാമനെയെന്നു കരഞ്ഞതി-\\
കുണ്ഠിതന്മാരായ് പുരിപുക്കു വേനിനാര്‍.\\
സീതാസമേതനാം രാമനെസ്സന്തതം\\
ചേതസി ചിന്തിച്ചു ചിന്തിച്ചനുദിനം\\
പുത്രമിത്രാദികളോടുമിടചേര്‍ന്നു\\
ചിത്തശുദ്ധ്യാ വസിച്ചീടിനാരേവരും.\\
മംഗലദേവതാവല്ലഭന്‍ രാഘവന്‍\\
ഗംഗാതടം പുക്കു ജാനകിതന്നൊടും\\
മംഗലസ്നാനവും ചെതു സഹാനുജം\\
ശൃംഗിവേരാവിദൂരേ മരുവീടിനാന്‍.\\
ദാശരഥിയും വിദേഹതനൂജനും\\
ശിംശപാമൂലേള് സുഖേന വാണീടിനാര്‍.
\end{verse}

%%9). guhasangamam

\section{ഗുഹസംഗമം}

\begin{verse}
രാമാഗമനമഹോത്സവമെത്രയു-\\
മാമോദമുള്‍ക്കൊണ്ടു കേട്ടു ഗുഹന്‍ തദാ\\
സ്വാമിയായിഷ്ടവയസ്യനായുള്ളൊരു\\
രാമന്‍തിരുവടിയെക്കണ്ടു വന്ദിപ്പാന്‍\\
പക്വമനസ്സോടു ഭക്ത്യൈവ സത്വരം\\
പക്വഫലമധുപുഷ്പാദികളെല്ലാം\\
കൈക്കൊണ്ടു ചെന്നു രാമാഗ്രേ വിനിക്ഷിപ്യ\\
ഭക്ത്യൈവ ദണ്ഡനമസ്കാരവും ചെയ്തു.\\
പെട്ടെന്നെടുത്തെഴുന്നേല്‍പിച്ചു വക്ഷസി\\
തുഷ്ട്യാ ദൃഢമണച്ചാശ്ലേഷവും ചെയ്തു\\
മന്ദഹാസം പൂണ്ടു മാധുര്യപൂര്‍വകം\\
മന്ദേതരം കുശലപ്രശ്നവും ചെയ്തു.\\
കഞ്ജവിലോചനന്‍തന്‍ തിരുമേനിക-\\
ണ്ടഞ്ജലിപൂണ്ടു ഗുഹനുമുരചെയ്തു:\\
‘ധന്യനായേനടിയനിന്നു കേവലം\\
നിര്‍ണയം നൈഷാദജന്മവും പാവനം\\
നൈഷാദമായുള്ള രാജ്യമിതുമൊരു\\
ദൂഷണഹീനമധീനമല്ലോ തവ,\\
കിങ്കരനാമടിയനേയും രാജ്യവും\\
സങ്കടം കൂടാതെ രക്ഷിച്ചുകൊള്ളുക.\\
സന്തോഷമുള്‍ക്കൊണ്ടിനി നിന്തിരുവടി\\
സന്തതമത്ര വസിച്ചരുളീടണം\\
അന്തഃപുരം മമ ശുദ്ധമാക്കീടണ-\\
മന്തര്‍മുദാ പാദപത്മരേണുക്കളാല്‍.\\
മൂലഫലങ്ങള്‍ പരിഗ്രഹിക്കേണമേ\\
കാലേ കനിവോടനുഗ്രഹിക്കേണമേ!’\\
ഇത്തരം പ്രാര്‍ഥിച്ചുനില്ക്കും ഗുഹനോടു\\
മുഗ്ദ്ധഹാസംപൂണ്ടരുള്‍ചെയ്തു രാഘവന്‍:\\
“കേള്‍ക്ക നീ വാക്യം മദീയം മമ സഖേ!\\
സൗഖ്യമിതില്‍പ്പരമില്ലെനിക്കേതുമേ.\\
സംവത്സരം പതിന്നാലു കഴിയണം\\
സംവസിച്ചീടുവാന്‍ ഗ്രാമലയങ്ങളില്‍\\
അനൃദത്തം ഭുജിക്കെന്നതുമില്ലെന്നു\\
മന്യേ വനവാസകാലം കഴിവോളം.\\
“രാജ്യം മമൈതല്‍ ഭവാന്‍ മത്സഖിയല്ലോ\\
പൂജ്യനാം നീ പരിപാലിക്ക സന്തതം.\\
കുണ്ഠഭാവം ചെറുതുണ്ടാകയും വേണ്ട,\\
കൊണ്ടുവരിക വടക്ഷീരമാശുനീ.”\\
തല്‍ക്ഷണം കൊണ്ടുവന്നൂ വടക്ഷീരവും\\
ലക്ഷ്മണനോടും കലര്‍ന്നു രഘൂത്തമന്‍\\
ശുദ്ധവടക്ഷീരഭൂതികളെക്കൊണ്ടു\\
ബദ്ധമായോരു ജടാമകുടത്തൊടും\\
സോദരന്‍തന്നാല്‍ കുശദലാദ്യങ്ങളാല്‍\\
സാദരമാസ്തൃതമായ തല്പസ്ഥലേ\\
പാനീയമാത്രമശിച്ചു വൈദേഹിയും\\
താനുമായ് പള്ളിക്കുറുപ്പുകൊണ്ടീടിനാന്‍\\
പ്രാസാദമൂര്‍ദ്ധനി പര്യങ്കേ യഥാ പുരാ\\
വാസവും ചെയ്തുറങ്ങീടുന്നതുപോലെ.\\
ലക്ഷ്മണന്‍ വില്ലുമമ്പും ധരിച്ചന്തികേ\\
രക്ഷിച്ചുനിന്നു ഗുഹനോടു കൂടവേ\\
ലക്ഷ്മീപതിയായ രാഘവസ്വാമിയും\\
ലക്ഷ്മീഭഗവതിയാകിയ സീതയും\\
വൃക്ഷമൂലേ കിടക്കുന്നതു കണ്ടതി-\\
ദുഃഖം കലര്‍ന്നു ബാഷ്പാകുലനായ് ഗുഹന്‍\\
ലക്ഷ്മണനോടു പറഞ്ഞുതുടങ്ങിനാന്‍:\\
‘പുഷ്കരനേത്രനെക്കണ്ടീലയോ സഖേ!’\\
‘പര്‍ണതല്പേ ഭുവി ദാരുമൂലേ കിട-\\
ന്നര്‍ണോജനേത്രനുറങ്ങുമാറായിതു\\
സ്വര്‍ണതല്പേ ഭവനോത്തമേ സല്‍പ്പുരേ\\
പുണ്യപുരുഷന്‍ ജനകാത്മജയോടും\\
പള്ളിക്കുറുപ്പുകൊള്ളും മുന്നമിന്നിഹ\\
പല്ലവപല്യങ്കസീമ്നി വനാന്തരേ.\\
ശ്രീരാമദേവനു ദുഃഖമുണ്ടാകുവാന്‍\\
കാരണഭൂതയായ് വന്നിതു കൈകേയി\\
മന്ഥരാചിത്തമാസ്ഥായ കൈകേയിതാന്‍\\
ഹന്ത! മഹാപാപമാചരിച്ചാളല്ലോ’\\
ശ്രുത്വാ ഗുഹോക്തികളിത്ഥമാഹന്ത സൗ-\\
മിത്രിയും സത്വരമുത്തരം ചൊല്ലിനാന്‍:\\
‘ഭദ്രമതേ! ശൃണു! മദ്വചനം രാമ-\\
ഭദ്രനാമം ജപിച്ചീടുക സന്തതം.\\
കസ്യ ദുഃഖസ്യ കോ ഹേതുര്‍ജഗത്ത്രയേ\\
കസ്യ സുഖസ്യ വാ കോ ഹി ഹേതുസ്സഖേ!\\
പൂര്‍വജന്മാര്‍ജിതകര്‍മമത്രേ ഭുവി\\
സര്‍വലോകര്‍ക്കും സുഖദുഃഖകാരണം\\
ദുഃഖസുഖങ്ങള്‍ ദാനം ചെയ്വതിന്നാരു-\\
മുള്‍ക്കാമ്പിലോര്‍ത്തു കണ്ടാലില്ല നിര്‍ണയം.\\
ഏകന്‍ മമ സുഖദാതാ ജഗതി മ്-\\
റ്റേകന്‍ മമ ദുഃഖദാതാവിതി വൃഥാ\\
തോന്നുന്നിതജ്ഞാനബുദ്ധികള്‍ക്കെപ്പോഴും\\
തോന്നുകയില്ല ബുധന്മാര്‍ക്കതേതുമേ.\\
ഞാനിതിനിന്നു കര്‍ത്താവെന്നു തോന്നുന്നു\\
മാനസതാരില്‍ വൃഥാഭിമാനേന കേള്‍\\
ലോകം നിജകര്‍മസൂത്രബദ്ധം സഖേ!\\
ഭോഗങ്ങളും നിജകര്‍മാനുസാരികള്‍.\\
മിത്രാര്യുദാസീനബാന്ധവദ്വേഷ്യമ-\\
ദ്ധ്യസ്ഥസുഹൃജ്ജനഭേദബുദ്ധിഭ്രമം\\
ചിത്രമത്രേ നിരൂപിച്ചാല്‍ സ്വകര്‍മങ്ങള്‍\\
യത്ര വിഭവ്യതേ തത്ര യഥാ തഥാ.\\
ദുഃഖംസുഖം നിജകര്‍മവശാഗത-\\
മൊക്കെയെന്നുള്‍ക്കാമ്പുകൊണ്ടു നിനച്ചതില്‍\\
യദ്യദ്യദാഗതം തത്ര കാലാന്തരേ\\
തത്തദ്ഭുജിച്ചതിസ്വസ്ഥനായ് വാഴേണം.\\
ഭോഗത്തിനായ്ക്കൊണ്ടു കാമിക്കയും വേണ്ട\\
ഭോഗം വിധികൃതം വര്‍ജിക്കയും വേണ്ട\\
വ്യര്‍ത്ഥമോര്‍ത്തോളം വിഷാദപ്രഹര്‍ഷങ്ങള്‍\\
ചിത്തേ ശുഭാശുഭകര്‍മഫലോദയേ\\
മര്‍ത്ത്യദേഹം പുണ്യപാപങ്ങളെക്കണ്ടു\\
നിത്യമുല്‍പ്പന്നം വിധിവിഹിതം സഖേ!\\
സൗഖ്യദുഃഖങ്ങള്‍ സഹജമേവര്‍ക്കുമേ\\
നീക്കാവതല്ല സുരാസുരന്മാരാലും\\
ലോകേ സുഖാനന്തരം ദുഃഖമായ് വരു-\\
മാകുലമില്ല ദുഃഖാനന്തരം സുഖം\\
നൂനം ദിനരാത്രിപോലെ ഗതാഗതം\\
മാനസേ ചിന്തിക്കിലത്രയുമല്ലെടോ\\
ദുഃഖമധ്യേ സുഖമായും വരും പിന്നെ\\
ദുഃഖം സുഖമധ്യസംസ്ഥമായും വരും\\
രണ്ടുമന്യോന്യസംയുക്തമായേവനു-\\
മുണ്ടു ജലപങ്കമെന്നപോലെ സഖേ!\\
ആകയാല്‍ ധൈര്യേണ വിദ്വജ്ജനം	ഹൃദി\\
ശോകഹര്‍ഷങ്ങള്‍ കൂടാതെ വസിക്കുന്നു\\
ഇഷ്ടമായുള്ളതുതന്നെ വരുമ്പോഴു-\\
മിഷ്ടമല്ലാത്തതുതന്നെ വരുമ്പോഴും\\
തുഷ്ടാത്മനാ മരുവുന്നു ബുധജനം\\
ദൃഷ്ടമെല്ലാം മഹാമായേതി ഭാവനാല്‍.’\\
ഇത്ഥം ഗുഹനും സുമിത്രാത്മജനുമായ്\\
വൃത്താന്തഭേദം പറഞ്ഞു നില്‍ക്കുന്നേരം\\
മിത്രനുദിച്ചിതു സത്വരം രാഘവന്‍\\
നിത്യകര്‍മങ്ങളും ചെയ്തരുളിച്ചെയ്തു\\
‘തോണി വരുത്തുകെ’ന്നപ്പോള്‍ ഗുഹന്‍ നല്ല\\
തോണിയും കൊണ്ടുവന്നാശു വണങ്ങിനാന്‍.\\
‘സ്വാമിന്നിയം ദ്രോണികാ സമാരുഹ്യതാം\\
സൗവിത്രിണാ ജനകാത്മജയാ സമം\\
തോണി തുഴയുന്നതുമടിയന്‍തന്നെ\\
മാനവവീര! മമ പ്രാണവല്ലഭ.’\\
ശൃംഗിവേരാധിപന്‍വാക്കു കേട്ടന്നേരം\\
മംഗലദേവതയാകിയ സീതയെ\\
കൈയും പിടിച്ചു കരേറ്റിഗ്ഗുഹനുടെ\\
കൈയും പിടിച്ചു താനും കരേറിടിനാന്‍.\\
ആയുധമെല്ലാമെടുത്തു സൗമിത്രിയു-\\
മായതമായൊരു തോണി കരേറിനാന്‍.\\
ജ്ഞാതിവര്‍ഗത്തോടുകൂടെഗ്ഗുഹന്‍ പര-\\
മാദരവോടു വഹിച്ചിതു തോണിയും.\\
മംഗലാപാംഗിയാം ജാനകീദേവിയും\\
ഗംഗയെ പ്രാര്‍ത്ഥിച്ചു നന്നായ് വണങ്ങിനാള്‍:\\
‘ഗംഗേ! ഭഗവതി! ദേവി! നമോസ്തു തേ.\\
സംഗേന ശംഭുതന്‍ മൗലിയില്‍ വാഴുന്ന\\
സുന്ദരി! ഹൈമവതി! നമസ്തേ നമോ\\
മന്ദാകിനീ! ദേവീ ഗംഗേ! നമോസ്തു തേ.\\
ഞങ്ങള്‍ വനവാസവും കഴിഞ്ഞാദരാ-\\
ലിങ്ങു വന്നാല്‍ ബലിപൂജകള്‍ നല്കുവന്‍\\
രക്ഷിച്ചുകൊള്‍ഗ നീയാപത്തു കൂടാതെ\\
ദക്ഷാരിവല്ലഭേ ഗംഗേ! നമോസ്തുതേ’\\
ഇത്തരം പ്രാര്‍ത്ഥിച്ചു വന്ദിച്ചിരിക്കവേ\\
സത്വരം പാരകൂലം ഗമിച്ചീടിനാര്‍.\\
തോണിയില്‍നിന്നു താഴത്തിറങ്ങീ ഗുഹന്‍\\
താണു തൊഴുതപേക്ഷിച്ചാന്‍ മനോഗതം:\\
‘കൂടെ വിടകൊള്‍വതിന്നടിയന്നുമൊ-\\
രാടല്‍കൂടാതെയനുജ്ഞ നല്‍കീടണം.\\
പ്രാണങ്ങളെ കളഞ്ഞീടുവനല്ലായ്കി-\\
ലേണാങ്കബിംബാനന! ജഗതീപതേ!’\\
നൈഷാദവാക്യങ്ങള്‍ കേട്ടു മനസി സ-\\
ന്തോഷേണ രാഘവനേവമരുള്‍ചെയ്തു:\\
‘സത്യം പതിന്നാലു സംവത്സരം വിപി-\\
നത്തില്‍ വസിച്ചു വരുവന്‍ വിരവില്‍ ഞാന്‍\\
ചിത്തവിഷാദമൊഴിഞ്ഞു വാണീടു നീ\\
സത്യവിരോധം വരാ രാമഭാഷിതം.’\\
ഇത്തരമോരോവിധമരുളിച്ചെയ്തു\\
ചിത്തമോദേന ഗാഢാശ്ലേഷവും ചെയ്തു\\
ഭക്തനെപ്പോകെന്നയച്ചു രഘൂത്തമന്‍\\
ഭക്ത്യാ നമസ്കരിച്ചഞ്ജലിയും ചെയ്തു\\
മന്ദമന്ദം തോണിമേലേ ഗുഹന്‍ വീണ്ടും\\
മന്ദിരം പുക്കു ചിന്തിച്ചു മരുവിനാന്‍.
\end{verse}

%%10). bharadvaajaashramapravesham

\section{ഭരദ്വാജാശ്രമപ്രവേശം}

\begin{verse}
വൈദേഹിതന്നോടു കൂടവേ രാഘവന്‍\\
സോദരനോടുമൊരു മൃഗത്തെക്കൊന്നു\\
സാദരം ഭുക്ത്വാ സുഖേന വസിച്ചിതു\\
പാദാപമൂലേ ദലാഢ്യതല്‍പസ്ഥലേ\\
മാര്‍ത്താണ്ഡദേവനുദിച്ചോരനന്തരം\\
പാര്‍ഥിവനര്‍ഘ്യാദി നിത്യകര്‍മം ചെയ്തു\\
ചെന്നു ഭരദ്വാജനായ തപോധനന്‍-\\
തന്നാശ്രമപദത്തിന്നടുത്താദരാല്‍\\
ചിത്തമോദത്തോടിരുന്നോരു നേരത്തു\\
തത്ര കാണായിതൊരു വടുതന്നെയും\\
അപ്പോളവനോടരുള്‍ചെയ്തു രാഘവന്‍:\\
‘ഇപ്പോഴേ നീ മുനിയോടുണര്‍ത്തിക്കണം\\
രാമന്‍ ദശരഥനന്ദനനുണ്ടു തന്‍\\
ഭാമിനിയോടുമനുജനോടും വന്നു\\
പാര്‍ത്തിരിക്കുന്നിതുടജാന്തികേയെന്ന\\
വാര്‍ത്ത വൈകാതെയുണര്‍ത്തിക്ക’യെന്നപ്പോള്‍\\
താപസശ്രേഷ്ഠനോടപ്രഹ്മചാരി ചെ-\\
ന്നാഭോഗസന്തോഷമോടു ചൊല്ലീടിനാന്‍:\\
‘ആശ്രമോപാന്തേ ദശരഥപുത്രനു-\\
ണ്ടാശ്രിതവത്സല! പാര്‍ത്തിരുന്നീടുന്നു.’\\
ശ്രുത്വാ ഭരദ്വാജനിത്ഥം സമുത്ഥായ\\
ഹസ്തേ സമാദായ സാര്‍ഘ്യപാദ്യാദിയും\\
ഗത്വാ രഘൂത്തമസന്നിധൗ സത്വരം\\
ഭക്ത്യൈവ പൂജയിത്വാ സഹലക്ഷ്മണം\\
ദൃഷ്ട്വാ രമാവരം രാമം ദയാപരം\\
തുഷ്ട്യാ പരമാനന്ദാബ്ധൗ മുഴുഗിനാന്‍\\
ദാശരഥിയും ഭരദ്വാജപാദങ്ങ-\\
ളാശു വണങ്ങിനാന്‍ ഭാര്യാനുജാന്വിതം.\\
ആശീര്‍വചനപൂര്‍വം മുനിപുംഗവ-\\
നാശയാനന്ദമിയന്നരുളിച്ചെയ്തു:\\
‘പാദരജസാ പവിത്രമാക്കീടു നീ\\
വേദാത്മക! മമ പര്‍ണശാലാമിമാം.’\\
ഇത്ഥമുക്ത്വോടജമാനീയ സീതയാ\\
സത്യസ്വരൂപം സഹാനുജം സാദരം\\
പൂജാവിധാനേന പൂജിച്ചുടന്‍ ഭര-\\
ദ്വാജതപോധനശ്രേഷ്ഠനരുള്‍ചെയ്തു.\\
‘നിന്നോടു സംഗമമുണ്ടാക കാരണ-\\
മിന്നു വന്നു തപസ്സാഫല്യമൊക്കവേ.\\
ജ്ഞാതം മയാ തവോദന്തം രഘുപതേ!\\
ഭൂതമാഗാമികം വാ കരുണാനിധേ!\\
ഞാനറിഞ്ഞേന്‍ പരമാത്മാ ഭവാന്‍ കാര്യ-\\
മാനുഷനായിതു മായയാ ഭൂതലേ.\\
ബ്രഹ്മണാ പണ്ടു സംപ്രാര്‍ത്ഥിതനാകയാല്‍\\
ജന്മമുണ്ടായതു യാതൊന്നിനെന്നതും\\
കാനനവാസാവകാശമുണ്ടായതും\\
ഞാനറിഞ്ഞീടിനേനിന്നതിനെന്നെടോ!\\
ജ്ഞാനദൃഷ്ട്യാ തവ ധ്യാനൈകജാതയാ\\
ജ്ഞാനമൂര്‍ത്തേ! സകലത്തെയും കണ്ടു ഞാന്‍.\\
എന്തിനു ഞാന്‍ വളരെപ്പറഞ്ഞീടുന്നു?\\
സന്തുഷ്ടബുദ്ധ്യാ കൃതാര്‍ത്ഥനായേനഹം”\\
ശ്രീപതി രാഘവന്‍ വന്ദിച്ചു സാദരം\\
താപസശ്രേഷ്ഠനോടേവമരുള്‍ചെയ്തു:\\
‘ക്ഷത്രബന്ധുക്കളായുള്ളോരു ഞങ്ങളെ-\\
ച്ചിത്തമോദത്തോടനുഗ്രഹിക്കേണമേ.’\\
ഇത്ഥമന്യോന്യമാഭാഷണവും ചെയ്തു\\
തത്ര കഴിഞ്ഞിതു രാത്രി മുനിയുമായ്
\end{verse}

%%11). vaalmeekyaashramapravesham

\section{വാല്മീക്യാശ്രമപ്രവേശം}

\begin{verse}
ഉത്ഥാനവും ചെയ്തുഷസി മുനിവര-\\
പുത്രരായുള്ള കുമാരകന്മാരുമായ്\\
ഉത്തമയായ കാളിന്ദീനദിയേയു-\\
മുത്തീര്യ താപസാദിഷ്ടമാര്‍ഗേണ പോയ്\\
ചിത്രകൂടാദ്രിയെ പ്രാപിച്ചിതു ജവാല്‍\\
തത്ര വാല്മീകിതന്നാശ്രമം നിര്‍മലം\\
നാനാമുനികുലസങ്കുലം കേവലം\\
നാനാമൃഗദ്വിജാകീര്‍ണം മനോഹരം\\
ഉത്തമവൃക്ഷലതാപരിശോഭിതം\\
നിത്യകുസുമഫലദലസംയുതം\\
തത്ര ഗത്വാ സമാസീനം മുനികുല-\\
സത്തമം ദൃഷ്ട്വാ നമസ്കരിച്ചീടിനാന്‍.\\
രാമം രമാവരം വീരം മനോഹരം\\
കോമളം ശ്യാമളം കാമദം മോഹനം\\
കന്ദ്രര്‍പ്പസുന്ദരമിന്ദീവരേക്ഷണ-\\
മിന്ദ്രാദിവൃന്ദാരകൈരഭിവന്ദിതം\\
ബാണതൂണീരധനുര്‍ദ്ധരം വിഷ്ടപ-\\
ത്രാണനിപുണം ജടാമകുടോജ്ജ്വലം\\
ജാനകീലക്ഷ്മണോപേതം രഘൂത്തമം\\
മാനവേന്ദ്രം കണ്ടു വാല്മീകിയും തദാ\\
സന്തോഷബാഷ്പാകുലാക്ഷനായ് രാഘവന്‍-\\
തന്‍ തിരുമേനി ഗാഢം പുണര്‍ന്നീടിനാന്‍.\\
നാരായണം പരമാനന്ദവിഗ്രഹം\\
കാരുണ്യപീയൂഷസാഗരം മാനുഷം\\
പൂജയിത്വാ ജഗല്‍പൂജ്യം ജഗന്മയം\\
രാജീവലോചനം രാജേന്ദ്രശേഖരം\\
ഭക്തിപൂണ്ടര്‍ഘ്യപാദ്യാദികള്‍ക്കൊണ്ടഥ\\
മുക്തിപ്രദനായ നാഥനു സാദരം\\
പക്വമധുരമധുഫലമൂലങ്ങ-\\
ളൊക്കെ നിവേദിച്ചു ഭോജനാര്‍ഥം മുദാ.\\
ഭുക്ത്വാ പരിശ്രമം തീരുത്തു രഘുവരന്‍\\
നത്വാ മുനിവരന്‍തന്നോടരുള്‍ചെയ്തു:\\
“താതാജ്ഞയാ വനത്തിന്നു പുറപ്പെട്ടു\\
സോദരനോടും ജനകാത്മജയോടും.\\
ഹേതുവോ ഞാന്‍ പറയേണമെന്നില്ലല്ലോ\\
വേദാന്തിനാം ഭവതാമറിയാമല്ലോ.\\
യാതൊരേടത്തു സുഖേന വസിക്കാവൂ\\
സീതയോടുംകൂടിയെന്നരുള്‍ചെയ്യണം\\
ഇദ്ദിക്കിലൊട്ടുകാലം വസിച്ചീടുവാന്‍\\
ചിത്തേ പെരികയുണ്ടാശ മഹാമുനേ!\\
ഇങ്ങനെയുള്ള ദിവ്യന്മാരിരിക്കുന്ന\\
മംഗലദേശങ്ങള്‍ മുഖ്യവാസോചിതം\\
എന്നതു കേട്ടു വാല്മീകിമഹാമുനി\\
മന്ദസ്മിതം ചെയ്തിവണ്ണമരുള്‍ചെയ്തു:\\
‘സര്‍വലോകങ്ങളും നിങ്കല്‍ വസിക്കുന്നു\\
സര്‍വലോകേഷു നീയും വസിച്ചീടുന്നു\\
ഇങ്ങനെ സാധാരണം നിവാസസ്ഥല-\\
മങ്ങനെയാകയാലെന്തു ചൊല്ലാവതും.\\
സീതാഹനിതനായ വാഴുവാനിന്നൊരു\\
ദേശം വിശേഷേണ ചോദിക്ക കാരണം\\
സൗഖ്യേന തേ വസിപ്പാനുള്ള മന്ദിര-\\
മാഖ്യാവിശേഷേണ ചൊല്ലുന്നതുണ്ടു ഞാന്‍.\\
സന്തുഷ്ടരായ് സമദൃഷ്ടികളായ് ബഹു-\\
ജന്തുക്കളില്‍ ദ്വേഷഹീനമതികളായ്\\
ശാന്തരായ് നിന്നെബ്ഭജിപ്പവര്‍തമ്മുടെ\\
സ്വന്തം നിനക്കു സുഖവാസമന്ദിരം\\
നിത്യധര്‍മ്മാധര്‍മമെല്ലാമുപേക്ഷിച്ചു\\
ഭക്ത്യാ ഭവാനെബ്ഭജിക്കുന്നവരുടെ\\
ചിത്തസരോജം ഭവാനിരുന്നീടുവാ-\\
നുത്തമമായ് വിളങ്ങീടുന്ന മന്ദിരം\\
നിത്യവും നിന്നെശ്ശരണമായ് പ്രാപിച്ചു\\
നിര്‍ദ്വന്ദ്വരായ് നിസ്പൃഹരായ് നിരീഹരായ്\\
ത്വന്മന്ത്രജാപകരായുള്ള മാനുഷര്‍-\\
തന്മനഃപങ്കജം തേ സുഖമന്ദിരം.\\
ശാന്തന്മാരായ് നിരഹങ്കാരികളുമായ്\\
ശാന്തരാഗദ്വേഷമാനസന്മാരുമായ്\\
ലോഷ്ടാശ്മകാഞ്ചനതുല്യമതികളാം\\
ശ്രേഷ്ഠമതികള്‍ മനസ്തവ മന്ദിരം\\
നിങ്കല്‍ സമസ്തകര്‍മങ്ങള്‍ സമര്‍പ്പിച്ചു\\
നിങ്കലേ ദത്തമായോരു മനസ്സോടും\\
സന്തുഷ്ടരായ് മരുവുന്നവര്‍മാനസം\\
സന്തതം തേ സുഖവാസായ മന്ദിരം.\\
ഇഷ്ടം ലഭിച്ചിട്ടു സന്തോഷമില്ലൊട്ടു-\\
മിഷ്ടേതരാപ്തിക്കനുതാപവുമില്ല\\
സര്‍വവും മായേതി നിശ്ചിത്യ വാഴുന്ന\\
ദിവ്യമനസ്തവ വാസായ മന്ദിരം\\
ഷഡ്ഭാവഭേദവികാരങ്ങളൊക്കെയു-\\
മിള്‍പ്പൂവിലോര്‍ക്കിലോ ദേഹത്തിനേയുള്ളൂ,\\
ക്ഷുത്തൃഡ്ഭയസുഖദുഃഖാദി സര്‍വവും\\
ചിത്തേ വിചാരിക്കിലാത്മാവിനില്ലേതും\\
ഇത്ഥമുറച്ചു ഭജിക്കുന്നവരുടെ\\
ചിത്തം തവ സുഖവാസായ മന്ദിരം.\\
യാതൊരുത്തന്‍ ഭവന്തം പരം ചില്‍ഘനം\\
വേദസ്വരൂപമനന്തമേകം സതാം\\
വേദാന്തവേദ്യമാദ്യം ജഗല്‍ക്കാരണം\\
നാദാന്തരൂപം പരബ്രഹ്മമച്യുതം\\
സര്‍വഗുഹാശയസ്ഥം സമസ്താധാരം\\
സര്‍വഗതം പരാത്മാനമലേപകം\\
വാസുദേവം വരദം വരേണ്യം ജഗ-\\
ദ്വാസിനാമാത്മനാ കാണുന്നതും സദാ\\
തസ്യചിത്തേ ജനകാത്മജയാ സമം\\
നിസ്സംശയം വസിക്കീടുക ശ്രീപതേ!\\
സന്തതാഭ്യാസദൃഢീകൃതചേതസാം\\
സന്തതം ത്വല്‍പ്പാദസേവാരതാത്മനാം\\
സന്തതം ത്വന്നാമമന്ത്രജപശുചി\\
സന്തോഷചേതസാം ഭക്തിദ്രവാത്മനാം\\
അന്തര്‍ഗതനായ് വസിക്ക നീ സീതയാ\\
ചിന്തിതചിന്താമണേ! ദയാവാരിധേ!
\end{verse}

%%12). vaalmeekiyudeaathmakathaa

\section{വാല്മീകിയുടെ ആത്മകഥ}

\begin{verse}
കര്‍ണാമൃതം തവ നാമമാഹാത്മ്യമോ\\
വര്‍ണിപ്പതിന്നാര്‍ക്കുമാവതുമല്ലല്ലോ\\
ചിന്മയനായ നിന്‍ നാമമഹിമയാല്‍\\
ബ്രഹ്മമുനിയായ് ചമഞ്ഞിതു ഞാനെടോ.\\
ദുര്‍മതി ഞാന്‍ കിരാതന്മാരുമായ് പുരാ\\
നിര്‍മരിയാദങ്ങള്‍ ചെയ്തേന്‍ പലതരം\\
ജന്മമാത്രദ്വിജത്വം മുന്നമുള്ളതും\\
ബ്രഹ്മകര്‍മങ്ങളുമൊക്കെ വെടിഞ്ഞു ഞാന്‍\\
ശൂദ്രസമാചാരതല്‍പരനായൊരു\\
ശൂദ്രതരുണിയുമായ് വസിച്ചേന്‍ ചിരം.\\
പുത്രരേയും വളരെജ്ജനിപ്പിച്ചിതു\\
നിസ്ത്രപം ചോരന്മാരോടുകൂടെച്ചേര്‍ന്നു\\
നിത്യവും ചോരനായ് വില്ലുമമ്പും ധരി-\\
ച്ചെത്ര ജന്തുക്കളെക്കൊന്നേന്‍ ചതിച്ചു ഞാന്‍!\\
എത്ര വസ്തു പറിച്ചേന്‍ ദ്വിജന്മാരോടു-\\
മത്ര മുനീന്ദ്രവനത്തില്‍ നിന്നേകദാ\\
സപ്തമുനികള്‍ വരുന്നതു കണ്ടു ഞാന്‍\\
തത്ര വേഗേന ചെന്നേന്‍ മുനിമാരുടെ\\
വസ്ത്രാദികള്‍ പറിച്ചീടുവാന്‍ മൂഢനായ്\\
മദ്ധ്യാഹ്നമാര്‍ത്താണ്ഡതേജസ്സ്വരൂപികള്‍\\
നിര്‍ദയം പ്രാപ്തനാം ദുഷ്ടനാമെന്നെയും\\
വിദ്രുതം നിര്‍ജനേ ഘോരമഹാവനേ\\
ദൃഷ്ട്വാ സസംഭ്രമമെന്നോടരുള്‍ചെയ്തു:\\
തിഷ്ഠ തിഷ്ഠ ത്വയാ കര്‍ത്തവ്യമ്ത്ര കിം?\\
ദുഷ്ടമതേ പരമാര്‍ഥം പറകെ’ന്നു\\
തുഷ്ട്യാ മുനിവരന്മാരരുള്ചെയ്തപ്പോള്‍\\
നിഷ്ഠുരാത്മാവായ ഞാനുമവര്‍കളോ-\\
ടിഷ്ടം മദീയം പറഞ്ഞേന്‍ നൃപാത്മജ!\\
‘പുത്രദാരാദികളുണ്ടെനിക്കെത്രയും\\
ക്ഷുത്തൃഡ് പ്രപീഡിതന്മാരായിരിക്കുന്നു\\
വൃത്തികഴിപ്പാന്‍ വഴിപോക്കരോടു ഞാന്‍\\
നിത്യം പിടിച്ചുപറിക്കുമാറാകുന്നു\\
നിങ്ങളോടും ഗ്രഹിച്ചീടണമേതാനു-\\
മിങ്ങനെ ചിന്തിച്ചു വേഗേന വന്നു ഞാന്.‍’\\
ചൊന്നാര്‍ മുനിവരന്മാരതു കേട്ടുട-\\
നെന്നോടു മന്ദസ്മിതം ചെയ്തു സാദരം:\\
‘എങ്കില്‍ നീ ഞ്ങ്ങള്‍ ചൊല്ലുന്നതു കേള്‍ക്കണം\\
നിന്‍കുടുംബത്തോടു ചെന്നു ചോദിക്ക നീ\\
നിങ്ങളെച്ചൊല്ലി ഞാന്‍ ചെയ്യുന്ന പാപങ്ങള്‍\\
നിങ്ങള്‍ക്കൂടെപ്പകുത്തൊട്ടു വാങ്ങീടുമോ?\\
എന്നു നീ ചെന്നു ചോദിച്ചു വരുവോളം\\
നിന്നീടുമെത്രൈവ ഞങ്ങള്‍ നിസ്സംശയം.’\\
ഇത്ഥമാക‍ര്‍ണ്യ ഞാന്‍ വീണ്ടു പോയ്ച്ചെന്നു മല്‍-\\
പുത്രദാരാദികളോടു ചോദ്യം ചെയ്തേന്‍:\\
‘ദുഷ്കര്‍മസഞ്ചയം ചെയ്തു ഞാന്‍ നിങ്ങളെ-\\
യൊക്കെബ്ഭരിച്ചുകൊള്ളുന്നു ദിനം പ്രതി\\
തല്‍ഫലമൊട്ടൊട്ടു നിങ്ങള്‍ വാങ്ങീടുമോ?\\
മല്‍പാപമൊക്കെ ഞാന്‍തന്നേ ഭുജിക്കെന്നോ?\\
സത്യം പറയേണ’മെന്നു ഞാന്‍ ചൊന്നതി-\\
നുത്തരമായവരെന്നോടു ചൊല്ലിനാര്‍:\\
നിത്യവും ചെയ്യുന്ന കര്‍മഗണഫലം\\
കര്‍ത്താവൊഴിഞ്ഞു മറ്റന്യന്‍ ഭുജിക്കുമോ?\\
താന്താന്‍ നിരന്തരം ചെയ്യുന്ന കര്‍മങ്ങള്‍\\
താന്താനനുഭവിച്ചീടുകെന്നേ വരൂ.’\\
ഞാനുമതുകേട്ടു ജാതനിര്‍വേദനായ്\\
മാനസേ ചിന്തിച്ചു ചിന്തിച്ചൊരോതരം\\
താപസന്മാര്‍ നിന്നരുളുന്ന ദിക്കിനു\\
താപേന ചെന്നു നമസ്കരിച്ചീടിനേന്‍\\
നിത്യതപോധനസംഗമഹേതുനാ\\
ശുദ്ധമായ് വന്നിതെന്നന്തഃകരണവും.\\
തൃക്ത്വാ ധനുശ്ശരാദ്യങ്ങളും ദൂരെ ഞാന്‍\\
ഭക്ത്യാ നമസ്കരിച്ചേന്‍ പാദസന്നിധൗ:\\
‘ദുര്‍ഗതിസാഗരേ മഗ്നനായ് വീഴുവാന്‍\\
നിര്‍ഗമിച്ചീടുമെന്നെക്കരുണാത്മനാ\\
രക്ഷിച്ചുകൊള്ളേണമേ ശരണാഗത-\\
രക്ഷണം ഭൂഷണമല്ലോ മഹാത്മനാം.’\\
സ്പ്ഷ്ടമിത്യുക്ത്വാ പതിതം പദാന്തികേ\\
ദൃഷ്ട്വാ മുനിവരന്മാരുമരുള്‍ചെയ്തു:\\
‘ഉത്തിഷ്ഠ ഭദ്രമുത്തിഷ്ഠ തേ സന്തതം\\
സ്വസ്ത്യസ്തു ചിത്തശുദ്ധിസ്സദൈവാസ്തു തേ\\
സദ്യഃഫലം വരും സജ്ജനസംഗമാ-\\
ദ്വിദ്വജ്ജനാനാം മഹത്ത്വമേതാദൃശം\\
ഇന്നുതന്നെ തരുന്നുണ്ടൊരുപദേശ-\\
മെന്നാല്‍ നിനക്കതിനാലേ ഗതിവരും\\
അന്യോന്യമാലോകനം ചെയ്തു മാനസേ\\
ധന്യതപോധനന്മാരും വിചാരിച്ചു:\\
‘ദുര്‍വൃത്തനേറ്റം ദ്വിജാധമനാമിവന്‍\\
ദിവ്യജനത്താലുപേക്ഷ്യനെന്നാകിലും\\
രക്ഷ രക്ഷേതി ശരണം ഗമിച്ചവന്‍\\
രക്ഷണീയന്‍ പ്രയത്നേന ദുഷ്ടോപി വാ\\
രോക്ഷമാര്‍ഗോപദേശേന രക്ഷിക്കണം\\
സാക്ഷാല്‍ പരബ്രഹ്മബോധപ്രദാനേന.’\\
ഇത്ഥമുക്ത്വാ രാമനാമവര്‍ണദ്വയം\\
വ്യത്യസ്തവര്‍ണരൂപേണ ചൊല്ലിത്തന്നാര്‍.\\
‘നിത്യം മരാമരേത്യേവം ജപിക്ക നീ\\
ചിത്തമേകാഗ്രമാക്കിക്കൊണ്ടനാരതം\\
ഞങ്ങളിങ്ങോട്ടു വരുവോളവും പുന-\\
രിങ്ങനെതന്നെ ജപിച്ചിരുന്നീടു നീ.’\\
ഇത്ഥമനുഗ്രഹം ദത്വാ മുനീന്ദ്രന്മാര്‍\\
സത്വരം ദിവ്യപഥാ ഗമിച്ചീടിനാര്‍.\\
നത്വാ മരേതി ജപിച്ചിരുന്നേനഹം\\
ഭക്ത്യാ സഹസ്രയുഗം കഴിവോളവും\\
പുറ്റുകൊണ്ടെന്നുടല്‍ മൂടിച്ചമഞ്ഞിതു\\
മുറ്റും മറഞ്ഞു ചമഞ്ഞിതു ബാഹ്യവും\\
താപസേന്ദ്രന്മാരുമന്നെഴുന്നള്ളിനാര്‍\\
ഗോപതിമാരുദയം ചെയ്തതുപോലെ\\
നിഷ്ക്രമിച്ചീടെന്നു ചൊന്നതു കേട്ടു ഞാന്‍\\
നിര്‍ഗമിച്ചീടിനേനാശു നാകൂദരാല്‍.\\
വല്മീകമദ്ധ്യതോ നിന്നു ജനിക്കയാ-\\
ലമ്മുനീന്ദ്രന്മാരഭിധാനവും ചെയ്താര്‍;\\
‘വാല്മീകിയാം മുനിശ്രേഷ്ഠന്‍ ഭവാന്‍ ബഹു-\\
ലാമ്നായവേദിയായ് ബ്രഹ്മജ്ഞാനാക നീ.’\\
എന്നരുള്‍ചെയ്തെഴുന്നള്ളി മുനികളു-\\
മന്നു തുടങ്ങൈ ഞാനിങ്ങനെ വന്നതും.\\
രാമനാമത്തിന്‍ പ്രഭാവം നിമിത്തമായ്\\
രാമ! ഞാനിങ്ങനെയായ് ചമഞ്ഞീടിനേന്‍.\\
ഇന്നു സീതാസുമിത്രാത്മജന്മാരൊടും\\
നിന്നെ മുദാകാണ്മതിന്നവകാശവും\\
വന്നിതെനിക്കു മുന്നം ചെയ്ത പുണ്യവും\\
നന്നായ് ഫലിച്ചു കരുണാജലനിധേ!\\
രാജീവലോചനം രാമം ദയാപരം\\
രാജേന്ദ്രശേഖരം രാഘവം ചക്ഷുഷാ\\
കാണായമൂലം വിമുക്തനായേനഹം\\
ത്രാണനിപുണ! ത്രിദശകുലപതേ!\\
‘സീതയാ സാര്‍ദ്ധം വസിപ്പതിനായൊരു\\
മോദകരസ്ഥലം കാട്ടിത്തരുവന്‍ ഞാന്‍.\\
പോന്നാലു’മെന്നെഴുന്നള്ളിനാനന്തികേ\\
ചേര്‍ന്നുള്ള ശിഷ്യപരിവൃതനാം മുനി.\\
ചിത്രകൂടാചലഗംഗയോരന്തരാ\\
ചിത്രമായോരുടജം തീര്‍ത്തു മാമുനി\\
തെക്കും വടക്കും കിഴക്കും പടിഞ്ഞാറു-\\
മക്ഷിവിമോഹനമായ് രണ്ടു ശാലയും\\
നിര്‍മിച്ചിവിടെയിരിക്കെന്നരുള്‍ചെയ്തു\\
മന്മഥതുല്യന്‍ ജനകജതന്നോടും\\
നിര്‍മലനാകിയ ലക്ഷ്മണന്‍ തന്നോടും\\
ബ്രഹ്മാത്മനാ മരുവീടിനാന്‍ രാമനും.\\
വാല്മീകിയാല്‍ നിത്യപൂജിതനായ് സദാ\\
കാമ്യാംഗിയായുള്ള ജാനകിതന്നോടും\\
സോദരനാകിയ ലക്ഷ്മണന്‍ തന്നോടും\\
സാദരമാനന്ദമുള്‍ക്കൊണ്ടു മേവിനാന്‍.\\
ദേവമുനിവരസേവിതനാകിയ\\
ദേവരാജന്‍ ദിവി വാഴുന്നതു പോലെ.
\end{verse}

%%13). dasharathante charamagati

\section{ദശരഥന്റെ ചരമഗതി}

\begin{verse}
മന്ത്രിവരനാം സുമന്ത്രരുമേറിയോ-\\
രന്തശ്ശുചാ ചെന്നയോദ്ധ്യ പുക്കീടിനാന്‍.\\
വസ്ത്രേണ വക്ത്രവുമാച്ഛാദ്യ കണ്ണുനീ-\\
രത്യര്‍ത്ഥമിറ്റിറ്റു വീണും തുടച്ചു മ-\\
ത്തേരും പുറത്തുഭാഗത്തു നിര്‍ത്തിച്ചെന്നു\\
ധീരതയോടു നൃപനെ വണങ്ങിനാന്‍.\\
‘ധാത്രീപതേ! ജയ വീരമൗലേ! ജയ\\
ശാസ്ത്രമതേ! ജയ ശൗര്യാംബുധേ! ജയ\\
കീര്‍ത്തിനിധേ! ജയ സ്വാമിന്‍! ജയ ജയ\\
മാര്‍ത്താണ്ഡഗോത്രജാതോത്തംസമേ! ജയ.’\\
ഇത്തരം ചൊല്ലി സ്തുതിച്ചു വണങ്ങിയ\\
ഭൃത്യനോടാശു ചോദിച്ചു നൃപോത്തമന്‍.\\
‘സോദരനോടും ജനകാന്മജയോടു-\\
മേതൊരു ദിക്കിലിരിക്കുന്നു രാഘവന്‍?\\
നിര്‍ലജ്ജനായതി പാപിയാമെന്നോടു\\
ചൊല്ലുവാനെന്തോന്നു ചൊല്ലിയതെന്നുടെ\\
ലക്ഷ്മണ,നെന്തു പറഞ്ഞു വിശേഷിച്ചു\\
ലക്ഷ്മീസമയായ ജാനകീദേവിയും?\\
ഹാ രാമ! ഹാ ഗുണവാരിധേ! ലക്ഷ്മണ!\\
വാരിജലോചനേ! ബാലേ മിഥിലജേ!\\
ദുഃഖം മുഴുത്തു മരിപ്പാന്‍ തുടങ്ങന്ന\\
ദുഷ്കൃതിയാമെന്നരികത്തിരിപ്പാനും\\
മക്കളെയും കണ്ടെനിക്കുമരിപ്പാനു-\\
മിക്കാലമില്ലാതെ വന്നു സുകൃതവും.’\\
ഇത്ഥം പറഞ്ഞു കേഴുന്ന നൃപേന്ദ്രനോ-\\
ടുള്‍ത്താപമോടുരചെയ്തു സുമന്ത്രരും:\\
‘ശ്രീരാമസീതാസുമിത്രാത്മജന്മാരെ-\\
ത്തേരിലേറ്റിക്കൊണ്ടുപോയേന്‍ തവാജ്ഞയാ.\\
ശൃംഗിവേരാഖ്യപുരസവിധേ ചെന്നു\\
ഗംഗാതടേ വസിച്ചീടും ദശാന്തരേ\\
കണ്ടു തൊഴുതിതു ശൃംഗിവേരാധിപന്‍\\
കൊണ്ടുവന്നൂ ഗുഹന്‍ മൂലഫലാദികള്‍.\\
തൃക്കൈകള്‍ക്കൊണ്ടതു തൊട്ടു പരിഗ്രഹി-\\
ച്ചക്കുമാരന്മാര്‍ ജടയുംധരിച്ചിതു.\\
പിന്നെ രഘൂത്തമനെന്നോടു ചൊല്ലിനാ-\\
‘നെന്നെ നിരൂപിച്ചു ദുഃഖിയായ്കാരുമേ.\\
ചൊല്ലേണമെന്നുടെ താതനോടും ബലാ-\\
ലല്ലലുള്ളത്തിലുണ്ടാകാതിരിക്കണം.\\
സൗഖ്യമയോദ്ധ്യയിലേറും വനങ്ങളില്‍\\
മോക്ഷസിദ്ധിക്കും പെരുവഴിയായ്വരും.\\
മാതാവിനും നമസ്കാരം, വിശേഷിച്ചു\\
ഖേദമെന്നെക്കുറിച്ചുണ്ടാകരുതേതും.\\
പിന്നെയും പിന്നെയും ചൊല്ക പിതാവതി-\\
ഖിന്നനായ് വാര്‍ധക്യപീഡിതനാകയാല്‍\\
എന്നെപ്പിരിഞ്ഞുള്ള ദുഃഖമശേഷവും\\
ധന്യവാക്യാമൃതം കൊണ്ടടക്കീടണം.’\\
ജാനകിയും തൊഴുതെന്നോടു ചൊല്ലിനാ-\\
ളാനനപത്മവും താഴിത്തി മന്ദംമന്ദം\\
അശ്രുകണങ്ങളും വാര്‍ത്തു സഗദ്ഗദം:\\
‘ശ്വശ്രുപാദേഷു സാഷ്ടാംഗം നമസ്കാരം.’\\
തോണി കരേറി ഗുഹനോടു കൂടവേ\\
പ്രാണവിയോഗേന നിന്നേനടിയനും\\
അക്കരെച്ചെന്നിറങ്ങിപ്പോയ് മറവോള-\\
മിക്കരെ നിന്നു ശവശരീരം പോലെ.\\
നാലഞ്ചു നാഴിക ചെന്നവാറേ ധൈര്യ-\\
മാലംബ്യ മന്ദം നിവൃത്തനായീടിനാന്‍.\\
തത്ര കൗസല്യ കരഞ്ഞു തുടങ്ങിനാള്‍:\\
“ദത്തമല്ലോ പണ്ടുപണ്ടേ വരദ്വയം\\
ഇഷ്ടയായോരു കൈകേയിക്കു രാജ്യമോ\\
തുഷ്ടനായ് നല്‍കിയാല്‍ പോരായിരുന്നിതോ?\\
മല്‍പ്പുത്രനെ കാനനാന്തേ കളവതി-\\
നിപ്പാപിയെന്തു പിഴച്ചിതു ദൈവമേ!\\
ഏവമെല്ലാം വരുത്തിത്തനിയേ പരി-\\
ദേവനം ചെയ്വതിനെന്തൊരു കാരണം?”\\
ഭുപതി കൗസല്യ ചൊന്നോരു വക്കുകള്‍\\
താപേന കേട്ടു മന്ദം പറഞ്ഞീടിനാന്‍:\\
‘പുണ്ണിലൊരു കൊള്ളീവെക്കുന്നതുപോലെ\\
പുണ്യമില്ലാത മാം ഖേദിപ്പിയായ്ക നീ.\\
ദുഃഖമുള്‍ക്കൊണ്ടു മരിപ്പാന്‍ തുടങ്ങുമെ-\\
ന്നുള്‍ക്കാമ്പുരുക്കിച്ചമയ്ക്കായ്ക വല്ലഭേ!\\
പ്രാണപയാണമടുത്തു, തപോധനന്‍\\
പ്രാണവിയോഗേ ശപിച്ചതു കാരണം.\\
കേള്‍ക്ക നീ ശാപവൃത്താന്തം മനോഹരേ!\\
സാക്ഷാല്‍ തപസ്വികളീശ്വരന്മാരല്ലോ.\\
അര്‍ദ്ധരാത്രൗ ശരജാലവും ചാപവും\\
ഹസ്തേ ധരിച്ചു മൃഗയാവിവശനായ്\\
വാഹിനീതീരെ വനാന്തരേ മാനസേ\\
മോഹേന നില്ക്കുന്ന നേരമൊരു മുനി\\
ദാഹേന മാതാപിതാക്കള്‍നിയോഗന്നാല്‍\\
സാഹസത്തോടിരുട്ടത്തു പുറപ്പെട്ടു\\
കുംഭവും കൊണ്ടു നീര്‍ കോരുവാന്‍ വന്നവന്‍\\
കുംഭേന വെള്ളമന്‍പോടു മുക്കും വിധൗ\\
കുംഭത്തില്‍ നീരകം പുക്ക ശബ്ദം കേട്ടു\\
കുംഭി തുമ്പിക്കയ്യിലംഭോഗതമിതി\\
ചിന്തിച്ചുടന്‍ നാദഭേദിനം സായകം\\
സന്ധായ ചാപേ ദൃഢമയച്ചീടിനേന്‍.\\
‘ഹാ! ഹാ! ഹതോസ്മ്യഹം ഹാ! ഹാ! ഹതോസ്മ്യഹം\\
ഹാ!’ ഹേതി കേട്ടിതു മാനുഷവാക്യവും.”\\
ഞാനൊരു ദോഷമാരോടുമേ ചെയ്തീല\\
കേന വാ ഹന്ത! ഹതോഹം വിധേ! വൃഥാ.\\
പാര്‍ത്തിരിക്കുന്നതു മാതാപിതാക്കന്മാ-\\
രാര്‍ത്തികൈക്കൊണ്ടു തണ്ണീര്‍ക്കു ദാഹിക്കയാല്‍.’\\
ഇത്തരം മര്‍ത്ത്യനാദം കേട്ടു ഞാനതി-\\
ത്രസ്തനാന്‍ തത്ര ചൊന്നത്തലോടും തദാ\\
താപസബാലകന്‍പാദങ്ങളില്‍ വീണു\\
താപേന ചൊന്നേന്‍ മുനിസുതനോടു ഞാന്‍.\\
സ്വാമിന്‍ ദശരഥനായ രാജാവു ഞാന്‍\\
മാമപരാധിനം രക്ഷിക്കവേണമേ!\\
ഞാനറിയാതെ മൃഗയാവിവശനാ-\\
യാന തണ്ണീര്‍ കുടിക്കും നാദമെന്നോര്‍ത്തു\\
ബാണമെയ്തേനതി പാപിയായോരു ഞാന്‍\\
പ്രാണന്‍ കളയുന്നതുണ്ടിനി വൈകാതെ.’\\
പാദങ്ങളില്‍ വീണു കേണീടുമെന്നോടു\\
ഖേദം കലര്‍ന്നു ചൊന്നാന്‍ മുനിബാലകന്‍:\\
‘കര്‍മമത്രേ തടുക്കാവതല്ലാര്‍ക്കുമേ\\
ബ്രഹ്മഹത്യാപാപമുണ്ടാകയില്ല തേ\\
വൈഷ്യനത്രേ ഞാന്‍ മമ പിതാക്കന്മാരെ-\\
യാശ്വസിപ്പിക്ക നീയേതുമേ വൈകാതെ.\\
വാര്‍ദ്ധക്യമേറി ജരാനരയും പൂണ്ടു\\
നേത്രവും കാണാതെ പാര്‍ത്തിരുന്നീടുന്നു\\
ദാഹേന ഞാന്‍ ജലംകൊണ്ടങ്ങു ചെല്ലുവാന്‍\\
ദാഹം കെടുക്ക നീ തണ്ണീര്‍ കൊടുത്തിനി\\
വൃത്താന്തമെല്ലാമവരോടറിയിക്ക\\
സത്യമെന്നാലവര്‍ നിന്നെയും രക്ഷിക്കും\\
എന്നുടെ താതനും കോപമുണ്ടാകിലോ\\
നിന്നെയും ഭസ്മമാക്കീടുമറിക നീ\\
പ്രാണങ്ങള്‍ പോകാഞ്ഞു പീഡയുണ്ടേറ്റവും\\
ബാണം പറിക്ക നീ വൈകരുതേതുമേ?\\
എന്നതു കേട്ടു ശല്യോദ്ധരണം ചെയ്തു\\
പിന്നെസ്സജലം കലശവും കൈക്കൊണ്ടു\\
ദമ്പതിമാരിരിക്കുന്നവിടേക്കതി-\\
സംഭ്രമത്തോടു ഞാന്‍ ചെല്ലും ദശാന്തരേ\\
‘വൃദ്ധതയോടു നേത്രങ്ങളും വേറുപെ-\\
ട്ടര്‍ദ്ധരാത്രിക്കു വിശന്നു ദാഹിച്ചഹോ\\
വര്‍ത്തിക്കുമെങ്ങള്‍ക്കു തണ്ണീര്‍ക്കു പോയൊരു\\
പുത്രനുമിന്നു  മറന്നു കളഞ്ഞിതോ?\\
മറ്റില്ലൊരാശ്രയം ഞങ്ങള്‍ക്കൊരു നാളും\\
മുറ്റും ഭവാനൊഴിഞ്ഞെന്തു വൈകീടുവാന്‍?\\
‘ഭക്തിമാനേറ്റവും മുന്നമെല്ലാമതി-\\
സ്വസ്ഥനായ് വന്നിതാ നീ കുമാരാ! ബാലാല്‍?’\\
ഇപ്രകാരം നിരൂപിച്ചിരിക്കുംവിധൗ\\
മല്‍പാദവിന്യാസജധ്വനി കേള്‍ക്കായി\\
കാല്‍പെരുമാറ്റം മദീയം തദാ കേട്ടു\\
താത്പര്യമോടു പറഞ്ഞു ജനകനും:\\
‘വൈകുവാനെന്തുമൂലം മമ നന്ദന!\\
വേഗേന തണ്ണീര്‍ തരിക നീ സാദരം.’\\
ഇത്ഥമാകര്‍ണ്യ ഞാന്‍ ദമ്പതിമാര്‍പദം\\
ഭക്ത്യാ നമസ്കരിച്ചെത്രയും ഭീതനായ്\\
വൃത്താന്തമെല്ലാമറിയിച്ചിതന്നേരം\\
‘പുത്രനല്ലല്ലയോദ്ധ്യാധിപനാകിയ\\
പൃത്ഥ്വീവരന്‍ ഞാന്‍ ദശരഥനെന്നു പേര്‍\\
രാത്രൗ വനാന്തേ മൃഗയാവിവശനായ്\\
ശാര്‍ദൂലമുഖ്യമൃഗങ്ങളെയും കൊന്നു\\
പാര്‍ത്തിരുന്നേന്‍ നദീതീരേ മൃഗാശയാ.\\
കുംഭത്തില്‍ നീരകം പുക്ക ശബ്ദം കേട്ടു\\
കുംഭീവരന്‍ നിജ തുമ്പിക്കരംതന്നില്‍\\
അംഭസ്സുകൊള്ളുന്ന ശബ്ദമെന്നോര്‍ക്കയാ-\\
ലമ്പയച്ചെരറിയാതെയതും ബലാല്‍\\
പുത്രനു കൊണ്ടനേരത്തു കരച്ചില്‍ കേ-\\
ട്ടെത്രയും ഭീതനായ് തത്ര ചെന്നീടിനേന്‍.\\
ബാലനക്കണ്ടു നമസ്കരിച്ചേനതു-\\
മൂലമവനുമെന്നോടു ചൊല്ലീടിനാന്‍:\\
‘കര്‍മമത്രേ മമ വന്നതിതു തവ\\
ബ്രഹ്മഹത്യാപാപമുണ്ടാകയില്ല തേ.\\
കണ്ണും പൊടിഞ്ഞു വയസ്സുമേറെപ്പുക്കു\\
പര്‍ണശാലാന്തേ വിശന്നു ദാഹത്തോടും\\
എന്നെയും പാര്‍ത്തിരിക്കും പിതാക്കന്മാര്‍ക്കു\\
തണ്ണീര്‍കൊടുക്ക’യെന്നെന്നോടു ചൊല്ലിനാന്‍.\\
‘ഞാനതു കേട്ടുഴറ്റോടു വന്നേനിനി\\
ജ്ഞാനികളാം നിങ്ങളൊക്കെ ക്ഷമിക്കണം.\\
ശ്രീപാദപങ്കജമെന്നിയേ മറ്റില്ല\\
പാപിയായോരടിയന്നവലംബനം\\
ജന്തുവിഷയ കൃപാവരന്മാരല്ലോ\\
സന്തതം താപസപുംഗവന്മാര്‍ നിങ്ങള്‍.’\\
ഇത്ഥമാകര്‍ണ്യ കരഞ്ഞുകരഞ്ഞവ-\\
രെത്രയും ദുഃഖം കലര്‍ന്നു ചൊല്ലീടിനാര്‍:\\
‘പുത്രനെവിടെക്കിടക്കുന്നിതു ഭവാന്‍\\
തത്രൈവ ഞങ്ങളെക്കൊണ്ടുപോയീടണം.’\\
ഞാനതു കേട്ടവര്‍തമ്മെയെടുത്തതി-\\
ദീനതയോടു മകനുടല്‍ കാട്ടിനേന്‍\\
കഷ്ടമാഹന്ത! കഷ്ടം! കര്‍മമെന്നവര്‍\\
തൊട്ടു തലോടി തനയശരീരവും\\
പിന്നെപ്പലതരം ചൊല്ലി വിലപിച്ചു\\
ഖിന്നതയോടവരെന്നോടു ചൊല്ലിനാര്‍:\\
‘നീയിനി നല്ല ചിത ചമച്ചീടണം\\
തീയുമേറ്റം ജ്വലിപ്പിച്ചു വൈകീടാതെ.’\\
തത്ര ഞാനും ചിതക്കൂട്ടിനേനന്നേരം\\
പുത്രേണ സാകം പ്രവേശിച്ചവര്‍കളും\\
ദഗ്ദ്ധദേഹന്മാരുമായ് ചെന്നു മൂവരും\\
വൃത്രാരിലോകം ഗമിച്ചു വാണീടിനാര്‍.\\
വൃദ്ധതപോധനനന്നേരമെന്നോടു\\
പുത്രശോകത്താല്‍ മരിക്കെന്നു ചൊല്ലിനാന്‍.\\
‘ശാപകാലം നമുക്കാഗതമായിതു\\
താപസവാക്യമസത്യമായും വരാ.’\\
മന്നവനേവം പറഞ്ഞു വിലപിച്ചു\\
പിന്നെയും പിന്നെയും കേണു തുടങ്ങിനാന്‍:\\
‘ഹാ രാമ! പുത്ര! ഹാ സീതേ! ജനകജേ!\\
ഹാ രാമ! ലക്ഷ്മണ! ഹാഹാ ഗുണാംബുധേ!\\
നിങ്ങളെയും പിരിഞ്ഞെന്മരണം പുന-\\
രിങ്ങനെ വന്നതു കൈകേയിസംഭവം.’\\
രാജീവനേത്രരെച്ചിന്തിച്ചു ച്ചിന്തിച്ചു\\
രാജാ ദശരഥന്‍ പുക്കു സുരാലയം.
\end{verse}

%%14). bharathaagamanam

\section{ഭരതാഗമനം}

\begin{verse}
ദുഃഖിച്ചു രാജനാരീജനവും പുന-\\
രൊക്കെ വാവിട്ടു കരഞ്ഞുതുടങ്ങിനാര്‍.\\
വക്ഷസി താഡിച്ചു കേഴുന്ന ഘോഷങ്ങള്‍\\
തല്‍ക്ഷണം കേട്ടു വസിഷ്ഠമുനീന്ദ്രനും\\
മന്ത്രികളോടുമുഴറി സസംഭ്രമ-\\
മന്തഃപുരമകം പുക്കരുളിച്ചെയ്തു:\\
‘തൈലമയദ്രോണിതന്നിലാക്കൂ ധരാ-\\
പാലകന്‍ തന്നുടല്‍ കേടുവന്നീടായ്വാന്‍.’\\
എന്നരുള്‍ചെയ്തു ദൂതന്മാരെയും വിളി-\\
‘ച്ചിന്നുതന്നെ നിങ്ങള്‍ വേഗേന പോകണം.\\
വേഗമേറീടും കുതിരയേറിച്ചെന്നു\\
കേകയരാജ്യമകം പുക്കു ചൊല്ലുക\\
മാതുലനായ യുധാജിത്തിനോടിനി\\
ഏതുമേ കാലം കളയാതയയ്ക്കണം\\
ശത്രുഘ്നനോടും ഭരതനെയെന്നതി\\
വിദ്രുതം ചെന്നുചൊല്കെ’ന്നയച്ചീടിനാന്‍.\\
സത്വരം കേകയരാജ്യമകം പുക്കു\\
നത്വാ യുധാജിത്തിനോടു ചൊല്ലീടിനാര്‍:\\
‘കേള്‍ക്ക നൃപേന്ദ്ര! വസിഷ്‍ഠനരുള്‍ചെയ്ത\\
വാക്കുകള്‍, ശത്രുഘ്നനോടും ഭരതനെ\\
ഏതുമേ വൈകാതയോധ്യയ്ക്കയയ്ക്കെ’ന്നു\\
ദൂതവാക്യം കേട്ടനേരം നരാധിപന്‍\\
ബാലകന്മാരോടു പോകെന്നു ചൊല്ലിനാന്‍\\
കാലേ പുറപ്പെട്ടിതു കുമാരന്മാരും.\\
ഏതാനുമങ്ങഒരാപത്തകപ്പെട്ടിതു\\
താതനെന്നാകിലും ഭ്രാതാവിനാകിലും\\
എന്തക്കപ്പെട്ടിതെന്നുള്ളില്‍ പലതരം\\
ചിന്തിച്ചു ചിന്തിച്ചു മാര്‍ഗേ ഭരതനും\\
സന്താപമോടുമയോധ്യാപുരി പുക്കു\\
സന്തോഷവര്‍ജിതം ശബ്ദഹീനം തഥാ\\
ഭ്രഷ്ടലക്ഷ്മീകംജനോല്‍ബാധവര്‍ജിതം\\
ദൃഷ്ട്വാ വിഗതോത്സവം രാജ്യമെന്തിദം\\
തേജോവിഹീനമകംപുക്കിതു, ചെന്നു\\
രാജഗേഹം രാമലക്ഷ്മണവര്‍ജിതം\\
തത്ര കൈകേയിയെക്കണ്ടു കുമാരന്മാര്‍\\
ഭക്ത്യാ നമസ്കരിച്ചീടിനാരന്തികേ\\
പുത്രനെക്കണ്ടു സന്തോഷേണ മാതാവു-\\
മുത്ഥായ ഗാഢമാലിംഗ്യ മടിയില്‍ വെ-\\
ച്ചുത്തമാഗേ മുകര്‍ന്നാശു ചോദിച്ചിതു:\\
“ഭദ്രമല്ലീ മല്‍കുലത്തിങ്കലൊക്കവേ?\\
മാതാവിനും പിതൃഭാതൃജനങ്ങള്‍ക്കു-\\
മേതുമേ ദുഃഖമില്ലല്ലീ പറക നീ.’\\
ഇത്തരം കൈകേയി ചൊന്നനേരത്തതി-\\
നുത്തരമാശു ഭരതനും ചൊല്ലിനാന്‍:\\
‘ഖേദമുണ്ടച്ഛനെക്കാണാഞ്ഞെനിക്കുള്ളീല്‍\\
താതനെവിടെ വസിക്കുന്നു മാതാവേ?\\
മാതാവിനോടു പിരിഞ്ഞു രഹസി ഞാന്‍\\
താതനെപ്പണ്ടു കണ്ടിലൊരുനാളുമേ\\
ഇപ്പോള്‍ ഭവതി താനേ വസിക്കുന്നതെ-\\
ന്തുള്‍പ്പൂവിലുണ്ടു മേ താപവും ഭീതിയും.\\
മല്‍പ്പിതാവെങ്ങു? പറകെ’ന്നതുകേട്ടു\\
തല്‍പ്രിയമാശു കൈകേയിയും ചൊല്ലിനാള്‍:\\
‘എന്മകനെന്തു ദുഃഖിപ്പാനവകാശം\\
നന്മനോവാഞ്ഛിതമൊക്കെ വരുത്തി ഞാന്‍.\\
അശ്വമേധാദിയാഗങ്ങളെല്ലാം ചെയ്തു\\
വിശ്വമെല്ലാടവു കീര്‍ത്തി പരത്തിയ\\
സല്‍പ്പുരുഷന്മാര്‍ ഗതി ലഭിച്ചീടിനാന്‍\\
ത്വല്‍പിതാ’വെന്നു കേട്ടോരു ഭരതനും\\
ക്ഷോണീതലേ ദുഃഖവിഹ്വലചിത്തനായ്\\
വീണു വിലാപം തുടങ്ങിനാനെത്രയും.
\end{verse}

%%15). bharathantevilaapam

\section{ഭരതന്റെ വിലാപം}

\begin{verse}
‘ഹാ താത! ദുഃഖസമുദ്രേ നിമജ്യ മാ-\\
മേതൊരു ദിക്കിനു പോയിതു ഭൂപതേ!\\
എന്നെയും രാജ്യഭാരത്തെയു രാഘവന്‍-\\
തന്നുടെ കൈയില്‍ സമര്‍പ്പിയാതെ പിരി-\\
ഞ്ഞെങ്ങു പൊയ്ക്കൊണ്ടു പിതാവേ! ഗുണനിധേ!\\
ഞങ്ങള്‍ക്കുമാരുടയോരിനി ദൈവമേ!\\
പുത്രനീവണ്ണം കരയുന്നതു നേര-\\
മുത്ഥാപ്യ കൈകേയി കണ്ണുനീരും തുട-\\
‘ച്ചാശ്വസിച്ചീടുക ദുഃഖേന കിം ഫല-\\
മീശ്വര കല്പിതമെല്ലാമറിക നീ,\\
അഭ്യുദയം വരുത്തീടിനേന്‍ ഞാന്‍ തവ\\
ലഭ്യമെല്ലാമേ ലഭിച്ചതറിക നീ’\\
മാതൃവാക്യം സമാകര്‍ണ്യ ഭരതനും\\
ഖേദപരവശചേതസാ ചോദിച്ചു:\\
‘ഏതാനുമൊന്നു പറഞ്ഞതില്ലേ മമ\\
താതന്‍ മരിക്കുന്ന നേരത്തു മാതാവേ!’\\
‘ഹാ രാമ രാമ! കുമാര! സീതേ മമ\\
ശ്രീരാമ! ലക്ഷ്മണ! രാമ! രാമ! രാമ!\\
സീതേ! ജനകസുതേതി പുനഃപുന-\\
രാതുരനായ് വിലപിച്ചി മരിച്ചിതു\\
താത’നതു കേട്ടനേരം ഭരതനും\\
മാതാവിനോടു ചോദിച്ചാ’നതെന്തയ്യോ!\\
താതന്‍ മരിക്കുന്ന നേരത്തു രാമവും\\
സീതയും സൗമിത്രിയുമരികത്തില്ലേ?\\
എന്നതു കേട്ടു കൈകേയിയും ചൊല്ലിനാള്‍:\\
മന്നവ‍ന്‍ രാമനഭിഷേകമാരഭ്യ\\
സന്നദ്ധനായതു കണ്ട നേരത്തു ഞാ-\\
നെന്നുടെ നന്ദനന്‍ തന്നെ വാഴിക്കേണം\\
എന്നു പറഞ്ഞഭിഷേകം മുടക്കിയേന്‍\\
നിന്നോടതിന്‍ പ്രകാരം പറയാമല്ലോ,\\
രണ്ടു വരം മമ തന്നു തവ പിതാ\\
പണ്ടതിലൊന്നിനാല്‍ നിന്നെ വാഴിക്കെന്നും\\
രാമന്‍ വനത്തിനു പോകെന്നു മറ്റേതും\\
ഭൂമിപന്‍ തന്നോടിതുകാലമര്‍ഥിച്ചേന്‍\\
സത്യപരായണനായ നരപതി\\
പൃത്ഥ്വീതലം നിനക്കും തന്നു രാമനെ\\
കാനനവാസത്തിനായയച്ചീടിനാന്‍.\\
ജാനകീദേവി പാതിവ്രത്യമാലംബ്യ\\
ഭര്‍ത്ത്രാ സമം ഗമിച്ചീടിനാളാശു സൗ-\\
മിത്രിയും ഭ്രാതാവിനോടു കൂടെപ്പോയാന്‍.\\
താതനവരെ നിനച്ചു വിലാപിച്ചു\\
ഖേദേന രാമരാമേതി ദേവാലയം\\
പുക്കാനറി’കെന്നു മാതൃവാക്യം കേട്ടു\\
ദുഃഖിച്ചു ഭൂമിയില്‍ വീണു ഭരതനും\\
മോഹം കലര്‍ന്നനേരത്തു കൈകേയിയു-\\
‘മാഹന്ത ശോകത്തിനെന്തൊരു കാരണം?\\
രാജ്യം നിനക്കു സമ്പ്രാപ്തമായ് വന്നിതു\\
പൂജ്യനായ് വാഴ്ക ചാപല്യം കളഞ്ഞു നീ.’\\
എന്നു കൈകേയി പറഞ്ഞതു കേട്ടുട-\\
നൊന്നു കോപിച്ചു നോക്കീടിനാന്‍ മാതരം\\
ക്രോധാഗ്നിതന്നില്‍ ദഹിച്ചുപോമമ്മയെ-\\
ന്നാധി പൂണ്ടീടിനാര്‍ കണ്ടുനിന്നോര്‍കളും.\\
‘ഭര്‍ത്താവിനെക്കൊന്ന പാപേ! മഹാഘോരേ!\\
നിസ്ത്രപേ! നിര്‍ദ്ദയേ! ദുഷ്ടേ! നിശാചരീ!\\
നിന്നുടെ ഗര്‍ഭത്തിലുത്ഭവിച്ചേനൊരു\\
പുണ്യമില്ലാത മഹാപാപി ഞാനഹോ\\
നിന്നോടുരിയാടരുതിനി ഞാന്‍ ചെന്നു\\
വഹ്നിയില്‍ വീണു മരിപ്പനല്ലായ്കിലോ\\
കാളകൂടം കുടിച്ചീടുവനല്ലായ്കില്‍\\
വാളെടുത്താശു കഴുത്തറുത്തീടുവന്‍\\
വല്ല കണക്കിലും ഞാന്‍ മരിച്ചീടുവ-\\
നില്ലൊരു സംശയം ദുഷ്ടേ! ഭയങ്കരീ!\\
ഘോരമായുള്ള കുംഭീപാകമാകിയ\\
നാരകം തന്നില്‍ വസിക്കുമിതുമൂലം.’\\
ഇത്തരം മാതരം ഭര്‍ത്സിച്ചു ദുഃഖിച്ചു\\
സത്വരം ചെന്നു കൗസല്യാഗൃഹം പുക്കാന്‍.\\
പാദേ നമസ്കരിച്ചോരു ഭരതനെ\\
മാതാവു കൗസല്യയും പുണര്‍ന്നീടിനാള്‍\\
കണ്ണുനീരോടും മെലിഞ്ഞതി ദീനയായ്\\
ഖിന്നയായോരു കൗസല്യ ചൊല്ലീടിനാള്‍:\\
‘കര്‍മദോഷങ്ങളിതെല്ലാമകപ്പെട്ടി-\\
തെന്മകന്‍ ദൂരത്തകപ്പെട്ട കാരണം\\
ശ്രീരാമനുമനുജാതനും സീതയും\\
ചീരാംബരജടാധാരികളായ് വനം\\
പ്രാപിച്ചിതെന്നെയും ദുഃഖാംബുരാശിയില്‍\\
താപേന മഗ്നയാക്കീടിനാര്‍ നിര്‍ഭയം.\\
ഹാ രാമ! രാമ! രഘുവംശനായക!\\
നാരായണ! പരമാത്മന്‍! ജഗല്‍പ്പതേ!\\
നാഥ! ഭവാന്‍ മമ നന്ദനനായ് വന്നു\\
ജാതനായീടിനാന്‍ കേവലമെങ്കിലും\\
ദുഃഖമെന്നെപ്പിരിയുന്നീലൊരിക്കലു-\\
മുള്‍ക്കാമ്പിലോര്‍ത്താല്‍ വിധിബലമാം തുലോം.’\\
ഇത്ഥം കരയുന്ന മാതാവുതന്നെയും\\
നത്വാ ഭരതനും ദുഃഖേന ചൊല്ലിനാന്‍:\\
‘ആതുരമാനസയാകായ്കിതുകൊണ്ടു\\
മാതാവു ഞാന്‍ പറയുന്നതു കേള്‍ക്കണം.\\
രാഘവരാജ്യാഭിഷേകം മുടക്കിനാള്‍\\
കൈകേയിയാകിയ മാതാവു മാതാവേ!\\
ഞാനറിഞ്ഞിട്ടില്ല രാഘവന്‍തന്നാണെ\\
ഞാനറിഞ്ഞത്രേയതെങ്കിലോ മാതാവേ!\\
ബ്രഹ്മഹത്യാശതജാതമാം പാപവു-\\
മമ്മേ! ഭുജിക്കുന്നതുണ്ടു ഞാന്‍ നിര്‍ണയം.\\
ബ്രഹ്മാത്മജനാം വസിഷ്ഠമുനിയെയും\\
ധര്‍മദാരങ്ങളരുന്ധതിതന്നെയും\\
ഖഡ്ഗേന നിഗ്രഹിച്ചാലുള്ള പാപവു-\\
മൊക്കെയനുഭവിച്ചീടുന്നതുണ്ടു ഞാന്‍.’\\
ഇങ്ങനെ നാനാശപഥങ്ങളും ചെയ്തു\\
തിങ്ങിന ദുഃഖം കലര്‍ന്നു ഭരതനും\\
കേഴുന്നനേരം ജനനിയും ചൊല്ലിനാള്‍:\\
‘ദോഷം നിനക്കേതുമില്ലെന്നറിഞ്ഞു ഞാന്‍.’\\
ഇത്ഥം പറഞ്ഞു പുണര്‍ന്നു ഗാഢം ഗാഢം-\\
മുത്തമാഗേ മുകര്‍ന്നാളതു കണ്ടവ-\\
രൊക്കെ വാവിട്ടു കരഞ്ഞു തുടങ്ങിനാ-\\
രക്കഥ കേട്ടു വസിഷ്ഠമുനീന്ദ്രനും\\
മന്ത്രിജനത്തോടുമന്‍പോടെഴുന്നള്ളി\\
സന്താപമോടു തൊഴുതു ഭരതനും\\
രോദനംകണ്ടരുള്‍ചെയ്തു വസിഷ്ഠനും:\\
‘ഖേദം മതിമതി കേളിതു കേവലം\\
വൃദ്ധന്‍ ദശരഥനായ രാജാധിപന്‍\\
സത്യപരാക്രമന്‍ വിജ്ഞാനവീര്യവാന്‍\\
മര്‍ത്ത്യസുഖങ്ങളാം രാജഭോഗങ്ങളും\\
ഭുക്ത്വാ യഥാവിധി യജ്ഞങ്ങളും ബഹു\\
കൃത്വാ ബഹുധനദക്ഷിണയും മുദാ\\
ദത്വാ ത്രിവിഷ്ടപം ഗത്വാ യഥാസുഖം\\
ലബ്ധ്വാ പുരന്ദരാര്‍ദ്ധാസനം ദുര്‍ലഭം\\
വൃത്രാരിമുഖ്യത്രിദശൗഘവന്ദ്യനാ-\\
യാനന്ദമോടിരിക്കുന്നതിനെന്തു നീ-\\
യാനനം താഴ്ത്തി നേത്രാംബു തൂകീടുന്നു?\\
ശുദ്ധനാത്മാ ജന്മനാശാദിവര്‍ജിതന്‍\\
നിത്യന്‍ നിരുപമനവ്യയനദ്വയന്‍\\
സത്യസ്വരൂപന്‍ സകലജഗന്മയന്‍\\
മൃത്യുജന്മാദിഹീനന്‍ ജഗല്‍കാരണന്‍\\
ദേഹമത്യര്‍ഥം ജഡം ക്ഷണഭംഗുരം\\
മോഹൈകകാരണം മുക്തിവിരോധംകം\\
ശുദ്ധിവിഹീനം പവിത്രമല്ലൊട്ടുമേ\\
ചിത്തേ വിചാരിച്ചു കണ്ടാലൊരിക്കലും\\
ദുഃഖിപ്പതിനവകാശമില്ലേതുമേ\\
ദുഃഖേന കിം ഫലം മൃത്യുവശാത്മനാം?\\
താതനെന്നാകിലും പുത്രനെന്നാകിലും\\
പ്രേതരായാലതിമൂഢരായുള്ളവര്‍\\
മാറത്തലച്ചു തൊഴിച്ചു മുറവിളി-\\
ച്ചേറെത്തളര്‍ന്നു മോഹിച്ചു വീണിടുവോര്‍\\
നിസ്സാരമെത്രയും സംസാരമോര്‍ക്കിലോ\\
സത്സംഗമൊന്നേ ശുഭകരമായുള്ളൂ.\\
തത്ര സൗഖ്യം വരുത്തീടുവാന്‍ നല്ലതു\\
നിത്യമായുള്ളൊരു ശാന്തിയറിക നീ.\\
ജന്മമുണ്ടാകില്‍ മരണവും നിശ്ചയം\\
ജന്മം മരിച്ചവര്‍ക്കും വരും നിര്‍ണയം\\
ആര്‍ക്കും തടുക്കരുതാതൊരവസ്ഥയെ-\\
ന്നോര്‍ക്കണമെല്ലാം സ്വകര്‍മവശാഗതം.\\
തത്ത്വമറിഞ്ഞുള്ള വിദ്വാനൊരിക്കലും\\
പുത്രമിത്രാര്‍ഥകളത്രാദി വസ്തുനാ\\
വേര്‍പെടുന്നേരവും ദുഃഖമില്ലേതുമേ:\\
സ്വോപേതമെന്നാല്‍ സുഖവുമൊല്ലേതുമേ.\\
ബ്രഹ്മണാ സൃഷ്ടങ്ങളായതും പാര്‍ക്കിലോ\\
സംഖ്യയില്ലാതോളമുണ്ടിതെന്നാല്‍ ക്ഷണ-\\
ഭംഗുരമായുള്ള ജീവിതകാലത്തി-\\
ലെന്തൊരാസ്ഥാ മഹാജ്ഞാനിനാമുള്ളതും?\\
ബന്ധമെന്തീ ദേഹദേഹികള്‍ക്കെന്നതും\\
ചിന്തിച്ചു മായാഗുണവൈഭവങ്ങളു-\\
മന്തര്‍മുദാ കണ്ടവര്‍ക്കെന്തു സംഭ്രമം?\\
കമ്പിതപത്രാഗ്രലഗ്നാംബുബിന്ദുവല്‍\\
സമ്പതിച്ചീടുമായുസ്സതിനശ്വരം\\
പ്രാക്തനദേഹസ്ഥകര്‍മണാ പിന്നെയും\\
പ്രാപ്തമാം ദേഹിക്കു ദേഹം പുനരപി.\\
ജീര്‍ണവസ്ത്രങ്ങളുപേക്ഷിച്ചു ദേഹികള്‍\\
പൂര്‍ണശോഭം നവവസ്ത്രങ്ങള്‍ കൊള്ളുന്നു\\
ജീര്‍ണദേഹങ്ങളവ്വണ്ണമുപേക്ഷിച്ചു\\
പൂര്‍ണശോഭം നവദേഹങ്ങള്‍ കൊള്ളുന്നു\\
കാലചക്രത്തിന്‍ ഭ്രമണവേഗത്തിനു\\
മൂലമിക്കര്‍മഭേദങ്ങളറിക നീ.\\
ദുഃഖത്തിനെന്തൊരു കാരണം ചൊല്ലു നീ\\
മുഖ്യജനമതം കേള്‍ക്ക ഞാന്‍ ചൊല്ലുവന്‍\\
ആത്മാവിനില്ല ജനനം മരണവു-\\
മാത്മനി ചിന്തിക്ക ഷഡ്ഭാവവുമില്ല.\\
നിത്യനാനന്ദസ്വരൂപന്‍ നിരാകുലന്‍\\
സത്യരൂപന്‍ സകല്ശ്വരന്‍ ശാശ്വതന്‍\\
ബുദ്ധ്യാദിസാക്ഷി സര്‍വാത്മാ സനാതനന്‍\\
അദ്വയനേകന്‍ പരന്‍ പരമന്‍ ശിവന്‍\\
ഇത്ഥമനാരതം ചിന്തിച്ചു ചിന്തിച്ചു\\
ചിത്തേ ദൃഢമായറിഞ്ഞു ദുഃഖങ്ങളും\\
ത്യക്ത്വാ തുടങ്ങുക കര്‍മസമൂഹവും\\
സത്വരമേതും വിഷാദമുണ്ടാകൊലാ.’
\end{verse}

%%16). samskaarakarmam

\section{സമ്സ്കാരകര്‍മ്മം}

\begin{verse}
ശ്രുത്വാ ഗുരുവചനം നൃപനന്ദനന്‍\\
കൃത്വാ യഥാവിധി സംസ്കാരകര്‍മവും\\
മിത്രഭൃത്യാമാത്യസോദരോപാധ്യായ-\\
യുക്തനായോരു ഭരതകുമാരനും\\
താതശരീരമെണ്ണത്തോണിതന്നില്‍നി-\\
ന്നാദരപൂര്‍വമെടുത്തു നീരാടിച്ചു\\
ദിവ്യാംബരാഭരണാലേപനങ്ങളാല്‍\\
സര്‍വാംഗമെല്ലാമലങ്കരിച്ചീടിനാന്‍\\
അഗ്നിഹോത്രാഗ്നിതന്നാലഗ്നിഹോത്രിയെ\\
സംസ്കരിക്കുംവണ്ണമാചാര്യസംയുതം\\
ദത്വാ തിലോദകം ദ്വാദശവാസരേ\\
ഭക്ത്യാ കഴിച്ചിതു പിണ്ഡവുമാദരാല്‍\\
വേദപരായണന്മാരാം ദ്വിജാവലി-\\
ക്കോദനഗോധനഗ്രാമരത്നാംബരം\\
ഭൂഷണലേപനതാംബൂലപൂഗങ്ങള്‍\\
ഘോഷേണ ദാനവും ചെയ്തു സസോദരം\\
വീണു നമസ്കരിച്ചാശീര്‍വചനമാ-\\
ദാനവും ചെതു വിശുദ്ധനായ് മേവിനാന്‍.\\
ജാനകീലക്ഷ്മണസംയുക്തനായുടന്‍\\
കാനനം പ്രാപിച്ച രാമകുമാരനെ\\
മാനസേ ചിന്തിച്ചു ചിന്തിച്ചനുദിനം\\
മാനവവീരനായോരു ഭരതനും\\
സാനുജനായ് വസിച്ചീടിനാനദ്ദിനം\\
നാനാസുഹൃജ്ജനത്തോടുമനാകുലം.\\
തത്ര വസിഷ്ഠമുനീന്ദ്രന്‍ മുനികുല-\\
സത്തമന്മാരുമായ് വന്നു സഭാന്തികേ\\
അര്‍ണോരുഹാസനസന്നിഭനാം മുനി\\
സ്വര്‍ണാസനേ മരുവീടിനാനാദരാല്‍.\\
ശത്രുഘ്നസംയുക്തനായ ഭരതനെ-\\
ത്തത്ര വരുത്തിയ നേരമവര്‍കളും\\
മന്ത്രികളോടും പുരവാസികളോടു-\\
മന്തരാനന്ദം വളര്‍ന്നു മരുവിനാര്‍\\
കുമ്പിട്ടുനിന്ന ഭരതകുമാരനോ-\\
ടംഭോജസംഭവനന്ദനന്‍ ചൊല്ലിനാന്‍;\\
‘ദേശകാലോചിതമായുള്ള വാക്കുകള്‍\\
ദേശികനായ ഞാനാശു ചൊല്ലീടുവന്‍.\\
സത്യസന്ധന്‍ തവ താതന്‍ ദശരഥന്‍\\
പൃത്ഥ്വീതലം നിനക്കദ്യ നല്‍കീടിനാന്‍\\
പുത്രാഭ്യുദയാര്‍ഥമേഷ കൈകേയിക്കു\\
ദത്തമായോരു വരദ്വയം കാരണം\\
മന്ത്രപൂര്‍വമഭിഷേകം നിനക്കു ഞാന്‍\\
മന്ത്രികളോടുമന്‍പോടു ചെയ്തീടുവന്‍\\
രാജ്യമരാജകമാം ഭവാനാലിനി\\
ത്യാജ്യമല്ലെന്നു ധരിക്ക കുമാര! നീ\\
താതനിയോഗമനുഷ്ഠിക്കയും വേണം\\
പാതകമുണ്ടാമതല്ലായ്കിലേവനും\\
ഒന്നൊഴിയാതെ ഗുണങ്ങള്‍ നരന്മാര്‍ക്കു\\
വന്നുകൂടുന്നു ഗുരുപ്രസാദത്തിനാല്‍.\\
എന്നരുള്‍ചെയ്ത വസിഷ്ഠമുനിയോടു\\
നന്നായ് തൊഴുതുണര്‍ത്തിച്ചു ഭരതനും:\\
‘ഇന്നടിയന്നു രാജ്യംകൊണ്ടു കിം ഫലം?\\
മന്നവനാകുന്നതും മമ പൂര്‍വജന്‍\\
ഞങ്ങളവനുടെ കിങ്കരന്മാരത്രേ\\
നിങ്ങളിതെല്ലാമറിഞ്ഞല്ലോ മേവുന്നു.\\
നാളെപ്പുലര്‍കാലേ പോകുന്നതുണ്ടു ഞാന്‍\\
നാളീകനേത്രനെക്കൊണ്ടിങ്ങു പോരുവാന്‍\\
ഞാനും ഭവാനുമരുന്ധതീദേവിയും\\
നാനാപുരവാസികളുമമാത്യരും\\
ആന തേര്‍ കാലാള്‍ കുതിരപ്പടയോടു-\\
മാനകശംഖപടഹവാദ്യത്തൊടും\\
സോദരഭൂസുരതാപസസാമന്ത-\\
മേദിനീപാലകവൈശ്യശൂദ്രാദിയും\\
സാദരമാശുകൈകേയിയൊഴിഞ്ഞുള്ള\\
മാതൃജനങ്ങളുമായിട്ടുപോകണം.\\
രാമനിങ്ങാഗമിച്ചീടുവോളം ഞങ്ങള്‍\\
ഭൂമിയില്‍ത്തന്നെ ശയിക്കുന്നതേയുള്ളൂ.\\
മൂലഫലങ്ങള്‍ ഭുജിച്ചു ഭസിതവു-\\
മാലേപനം ചെയ്തു വല്ക്കലവും പൂണ്ടു\\
താപസവേഷം ധരിച്ചു ജടപൂണ്ടു\\
താപം കലര്‍ന്നു വസിക്കുന്നതേയുള്ളൂ.’\\
ഇത്ഥം ഭരതന്‍ പറഞ്ഞതു കേട്ടവ-\\
രെത്രയും നന്നുനന്നെന്നു ചൊല്ലീടിനാര്‍.
\end{verse}

%%17). bharathante vanayaathra

\section{ഭരതന്റെ വനയാത്ര}

\begin{verse}
ചിത്തേ നിനക്കിതു തോന്നിയതത്ഭുത-\\
മുത്തമന്മാരിലത്യുത്തമനല്ലോനീ’\\
സാധിക്കളേവം പുകഴ്ത്തുന്നതുനേര-\\
മാദിത്യദേവനുദിച്ചു, ഭരതനും\\
ശത്രുഘ്നനോടുകൂടെപ്പുറപ്പെട്ടിതു\\
തത്ര സുമന്ത്രനിയോഗേന സൈന്യവും\\
സത്വരം രാമനെക്കാണ്മാന്‍ നടന്നിതു\\
ചിത്തേ നിറഞ്ഞു വഴിഞ്ഞ മോദത്തോടും.\\
രാജദാരങ്ങള്‍ കൗസല്യാദികള്‍ തദാ\\
രാജീവനേത്രനെക്കാണാന്‍ നടന്നിതു\\
താപസശ്രേഷ്ഠന്‍ വസിഷ്ഠനും പത്നിയും\\
താപസവൃന്ദേന സാകം പുറപ്പെട്ടു\\
ഭൂമി കിളര്‍ന്നു പൊങ്ങീടും പൊടികളും\\
വ്യോമനി ചെന്നു പരന്നു ചമഞ്ഞിതു.\\
രാഘവാലോകനാനന്ദവിവശരാം\\
ലോകരറിഞ്ഞില്ല മാര്‍ഗഖേദങ്ങളും.\\
ശൃംഗിവേരാഖ്യപുരം ഗമിച്ചിട്ടുടന്‍\\
ഗംഗാതടേ ചെന്നുനിന്നുപെരുമ്പട.\\
കേകയപുത്രീസുതന്‍ പടയോടുമി-\\
ങ്ങാഗതനായതു കേട്ടു ഗുഹന്‍ തദാ\\
ശങ്കിതമാനസനായ് വന്നു തന്നുടെ\\
കിങ്കരന്മാരോടു ചൊന്നാനതു നേരം\\
‘ബാണചാപാദിശസ്ത്രങ്ങളും കൈക്കൊണ്ടു\\
തോണികളൊക്കെ ബന്ധിച്ചു സന്നദ്ധരായ്\\
നില്പിനെല്ലാവരും ഞാനങ്ങു ചെന്നുക-\\
ണ്ടിപ്പോള്‍ വരുന്നതുമുണ്ടു വൈകീടാതെ.\\
അന്തികേ ചെന്നു വന്ദിച്ചാലവനുടെ-\\
യന്തര്‍ഗതമറിഞ്ഞീടുന്നതുണ്ടല്ലോ.\\
രാഘവനോടു വിരോധത്തിനെങ്കിലോ\\
പോകരുതാരുമിവരിനി നിര്‍ണയം\\
ശുദ്ധരെന്നാകില്‍ കടത്തുകയും വേണം\\
പദ്ധതിക്കേതും വിഷാദവും കൂടാതെ.’\\
ഇത്ഥം വിചാരിച്ചുറച്ചു ഗുഹന്‍ ചെന്നു\\
സത്വരം കാല്ക്കല്‍ നമസ്കരിച്ചീടിനാന്‍.\\
നാനാവിധോപായനങ്ങളും കാഴ്ചവെ-\\
ച്ചാനന്ദപൂര്‍വം തൊഴുതു നിന്നീടിനാന്‍\\
ചീരാംബരം ഘനശ്യാമം ജടാധരം\\
ശ്രീരാമമന്ത്രം ജപന്തമനാരതം\\
ധീരം കുമാരം കുമാരോപമം മഹാ-\\
വീരം രഘുവരസോദരം സാനുജം\\
മാരസമാനശരീരം മനോഹരം\\
കാരുണ്യസാഗരം കണ്ടു ഗുഹന്‍ തദാ\\
ഭൂമിയില്‍ വീണു ഗുഹോഹമിത്യുക്ത്വാ പ്ര-\\
ണാമവും ചെയ്തു ഭരതനുമന്നേരം\\
ഉത്ഥാപ്യ ഗാഢമാലിംഗ്യ രഘുനാഥ-\\
ഭക്തം വയസ്യ, മനാമയവാക്യവും\\
ഉക്ത്വാ ഗുഹനോടു പിന്നെയും ചൊല്ലിനാന്‍:\\
‘ഉത്തമപൂരുഷോത്തംസരത്നം ഭവാന്‍\\
ആലിംഗനം ചെയ്തുവല്ലോ ഭവാനേ ലോ-\\
കാലംബനഭൂതനാകിയ രാഘവന്‍.\\
ലക്ഷ്മീഭഗവതീദേവിക്കൊഴിഞ്ഞു സി-\\
ദ്ധിക്കുമോ മറ്റൊരുവര്‍ക്കുമതോര്‍ക്ക നീ.\\
ധന്യനാകുന്നിതു നീ ഭുവനത്തിങ്ക-\\
ലിന്നതിനില്ലൊരു സംശയം മത്സഖേ!\\
സോദരനോടും ജനകാത്മജയോടു-\\
മേതൊരിടത്തു നിന്നന്‍പോടു കണ്ടിതു\\
രാമനെ നീ, യവനെന്തു പറഞ്ഞതും\\
നീ മുദാ രാമനോടെന്തോന്നു ചൊന്നതും\\
യാതൊരിടത്തുറങ്ങീ രഘുനായകന്‍\\
സീതയോടും കൂടി, നീയവിടം മുദാ\\
കാട്ടിത്തരികെ’ന്നു കേട്ടു ഗുഹന്‍ തദാ\\
വാട്ടമില്ലാത്തൊരു സന്തോഷചേതസാ\\
ഭക്തന്‍ ഭരതനത്യുത്തമനെന്നു തന്‍-\\
ചിത്തേ നിരൂപിച്ചുടന്‍ നടന്നീടിനാന്‍.\\
യത്ര സുപ്തോ നിശി രാഘവന്‍ സീതയാ\\
തത്ര ഗത്വാ ഗുഹന്‍ സത്വരം ചൊല്ലിനാന്‍:\\
‘കണ്ടാലുമെങ്കില്‍ കുശാസ്തൃതം സീതയാ\\
കൊണ്ടല്‍വര്‍ണന്‍തന്‍ മഹാശയനസ്ഥലം.’\\
കണ്ടു ഭരതനും മുക്തബാഷ്പോദകം\\
തൊണ്ട വിറച്ചു സഗദ്ഗദം ചൊല്ലിനാന്‍:\\
‘ഹാ സുകുമാരീ! മനോഹരീ! ജാനകീ!\\
പ്രാസാദമൂര്‍ദ്ധ്നി സുവര്‍ണതല്പസ്ഥലേ\\
കോമളസ്നിഗ്ദ്ധധവളാംബരാസ്തൃതേ\\
രാമേണ ശേതേ മഹാസുഖം സാ കഥം\\
ശേതേ കുശമയവിഷ്ടരേ നിഷ്ഠുരേ\\
ഖേദേന സീതാ മദീയാഗ്രജന്മനാ.\\
മദ്ദോഷകാരണാലെന്നതു ചിന്തിച്ചു-\\
മിദ്ദേഹമാശു പരിത്യജിച്ചീടുവന്‍\\
കില്ബിഷകാരിണിയായ കൈകേയിതന്‍\\
ഗര്‍ഭത്തില്‍നിന്നു ജനിച്ചതു കാരണം\\
ദുഷ്കൃതിയായതി പാപിയാമെന്നെയും\\
ധിക്കരിച്ചീടിനേന്‍ പിന്നെയും പിന്നെയും\\
ജന്മസാഫല്യവും വന്നിതനുജനു\\
നിര്‍മലമാനസന്‍ ഭാഗ്യവാനെത്രയും\\
അഗ്രജന്‍തന്നെപ്പരിചരിച്ചെപ്പോഴും\\
വ്യഗ്രം വനത്തിനു പോയതവനല്ലോ.\\
ശ്രീരാമദാസദാസന്മാര്‍ക്കു ദാസനാ-\\
യാരൂഢഭക്തിപൂണ്ടേഷ ഞാനും സദാ\\
നിത്യവും സേവിച്ചുകൊള്‍വനെന്നാല്‍ വരും\\
മര്‍ത്ത്യജന്മത്തിന്‍ ഫലമെന്നു നിര്‍ണയം.\\
ചൊല്ലു നീയെന്നോടെവിടെ വസതി കൗ-\\
സല്യാതനയനവിടേക്കു വൈകാതെ.\\
ചെന്നു ഞാനിങ്ങു കൂട്ടിക്കൊണ്ടുപോരുവ-\\
നെന്നതു കേട്ടു ഗുഹനുമുരചെയ്താന്‍:\\
മംഗലദേവതാവല്ലഭന്‍ തങ്കലി-\\
ന്നിങ്ങനെയുള്ളൊരു ഭക്തിയുണ്ടാകയാല്‍\\
പുണ്യവാന്മാരില്‍വെച്ചഗ്രേസരന്‍ ഭവാന്‍\\
നിര്‍ണയമെങ്കിലോ കേള്‍ക്ക മഹാമതേ!\\
ഗംഗാനദി കടന്നാലടുത്തെത്രയും\\
മംഗലമായുള്ള ചിത്രകൂടാചലം\\
തന്നികടേ വസിക്കുന്നിതു സീതയാ\\
തന്നുടെ സോദരനോടും യഥാസുഖം.’\\
ഇത്ഥം ഗുഹോക്തികള്‍ കേട്ടു ഭരതനും\\
“തത്ര ഗച്ഛാമഹേ ശീഘ്രം പ്രിയസഖേ!\\
തര്‍ത്തുമമര്‍ത്ത്യതടിനിയെ സത്രരം\\
കര്‍ത്തുമുദ്യോഗം സമര്‍ഥോ ഭവാദ്യ നീ.’\\
ശ്രുത്വാ ഭരതവാക്യം ഗുഹന്‍ സാദരം\\
ഗത്വാ വിബുധനദിയെക്കടത്തുവാന്‍\\
ഭൃത്യജനത്തോടു കൂടെസ്സസംഭരമം\\
വിസ്താരയുക്തം മഹാക്ഷേപണീയുതം\\
അഞ്ജസാ കൂലദേശം നിറച്ചീടിനാ-\\
നഞ്ഞൂറു തോണി വരുത്തി നിരത്തിനാന്‍.\\
ഊറ്റമായോരു തുഴയുമെടുത്തതി-\\
ല്ലേറ്റം വലിയൊരു തോണിയില്‍ താന്‍ മുദാ\\
ശത്രുഘ്നനോടു ഭരതനേയും മുനി-\\
സത്തമനായ വസിഷ്ഠനെയും തഥാ\\
രാമമാതാവായ കൗസല്യതന്നെയും\\
വാമശീലാംഗിയാം കൈകേയിതന്നെയും\\
ഉത്തമയാം സുമിത്രാദേവിതന്നെയും\\
പൃത്ഥ്വീശപത്നിമാര്‍ മറ്റുള്ളവരെയും\\
ഭക്ത്യാ തൊഴുതു കരേറ്റി മന്ദം തുഴ-\\
ഞ്ഞസ്തഭീത്യാ കടത്തീടിനാനാദരാല്‍\\
ഉമ്പര്‍തടിനിയെക്കുമ്പിട്ടനാകുലം\\
മുമ്പേ കടന്നിതു വമ്പടയും തദാ.\\
ശീഘ്രം ഭരദ്വാജതാപസേന്ദ്രാശ്രമം\\
വ്യാഘ്രഗോവൃന്ദപൂര്‍ണം വിരോധം വിനാ\\
സംപ്രാപ്യ സംപ്രീതനായ ഭരതനും\\
വന്‍പടയൊക്കവേ ദൂരെ നിര്‍ത്തീടിനാന്‍.\\
താനുമനുജനുമായുടജാങ്കണേ\\
സാനന്ദമാവിശ്യ നിന്നോരനന്തരം\\
ഉജ്ജ്വലന്തം മഹാതേജസാ താപസം\\
വിജ്വരാത്മാനമാസീനം വിധിസമം\\
ദൃഷ്ട്വാ നനാമ സാഷ്ടാംഗം സസോദരം\\
പുഷ്ടഭക്ത്യാ ഭരദ്വാജമുനീശ്വരം\\
ജ്ഞാത്വാ ദശരഥനന്ദനം ബാലകം\\
പ്രീത്യൈവ പൂജയാമാസ മുനീന്ദ്രനും.\\
ഹൃഷ്ടവാചാ കുശലപ്രശ്നവും ചെയ്തു\\
ദൃഷ്ട്വാ തദാ ജടാവല്കലധാരിണം\\
തുഷ്ടികലര്‍ന്നരുള്‍ ചെയ്താ’നിതെന്തെടോ\\
കഷ്ടമിക്കൊപ്പുപപന്നമല്ലൊട്ടുമേ.\\
രാജ്യവും പാലിച്ചു നാനാജനങ്ങളാല്‍\\
പൂജ്യനായോരു നീയെന്തിനായിങ്ങനെ\\
വല്കലവും ജടയും പൂണ്ടു താപസ-\\
മുഖ്യവേഷത്തെപ്പരിഗ്രഹിച്ചീടുവാന്‍?\\
എന്തൊരു കാരണം വന്‍പടയോടുമാ-\\
ഹന്ത! വനാന്തരേ വന്നതും ചൊല്ലു നീ.’\\
ശ്രുത്വാ ഭരദ്വാജവാക്യം ഭരതനു-\\
മിത്ഥം മുനിവരന്‍തന്നോടു ചൊല്ലിനാന്‍;\\
‘നിന്തിരുവുള്ളത്തിലേറാതെ ലോകത്തി-\\
ലെന്തൊരു വൃത്താന്തമുള്ളൂ മഹാമുനേ!\\
എങ്കിലും വാസ്തവം ഞാനുണര്‍ത്തിപ്പനി-\\
സ്സംങ്കടം പോവാനനുഗ്രഹിക്കേണമേ!\\
രാമാഭിഷേകവിഘ്നത്തിനു കാരണം\\
രാമപാദാബ്ജങ്ങളാണേ തപോനിധേ!\\
ഞാനേതുമേയൊന്നറിഞ്ഞീല രാഘവന്‍\\
കാനനത്തിന്നെഴുന്നള്ളൂവാന്‍ മൂലവും\\
കേകയപുത്രിയാമമ്മതന്‍ വാക്കായ\\
കാകോളവേഗമേ മൂലമതിന്നുള്ളൂ.\\
ഇപ്പോളശുദ്ധനോ ശുദ്ധനോ ഞാനതി-\\
നിപ്പാദപത്മം പ്രമാണം ദയാനിധേ!\\
ശ്രീരാമചന്ദ്രനു ഭൃത്യനായ് നല്‍പ്പാദ-\\
വാരിജയുഗ്മം ഭജിക്കെന്നിയേ മമ\\
മറ്റുള്ള ഭോഗങ്ങളാലെന്തൊരു ഫലം?\\
മുറ്റുമതിനൊഴിഞ്ഞില്ലൊരുകാംക്ഷിതം\\
ശ്രീരാഘവന്‍ ചരണാന്തികേ വീണു സം-\\
ഭാരങ്ങളെല്ലാമവിടെസ്സമര്‍പ്പിച്ചു\\
പൗരവസിഷ്ഠാദികളോടു കൂടവേ\\
ശ്രീരാമചന്ദ്തനഭിഷേകവും ചെയ്തു\\
രാജ്യത്തിനാശു കൂട്ടിക്കൊണ്ടു പോയിട്ടു\\
പൂജ്യനാം ജ്യേഷ്ഠനെസ്സേവിച്ചു കൊള്ളൂവന്‍.’\\
ഇങ്ങനെ കേട്ടു ഭരതവാക്യം മുനി\\
മംഗലാത്മാനമേനം പുണര്‍ന്നീടിനാന്‍.\\
ചുംബിച്ചു മൂര്‍ദ്ധ്നി സന്തോഷിച്ചരുളിനാന്‍:\\
‘കിം ബഹുനാ വത്സാ! വൃത്താന്തമൊക്കെ ഞാന്‍\\
ജ്ഞാനദൃശാ കണ്ടറിഞ്ഞിരിക്കുന്നിതു\\
മാനസേ ശോകമുണ്ടാകൊലാ കേള്‍ക്ക നീ.\\
ലക്ഷ്മണനേക്കാള്‍ നിനക്കേറുമേ ഭക്തി\\
ലക്ഷ്മീപതിയായ രാമങ്കല്‍ നിര്‍ണയം.\\
ഇന്നിനിസ്സല്‍ക്കരിച്ചീടുവന്‍ നിന്നെ ഞാന്‍\\
വന്ന പടയോടുമില്ലൊരു സംശയം.\\
ഊണും കഴിഞ്ഞിങ്ങുറങ്ങിപ്പുലര്‍കാലേ\\
വേണം രഘുനാഥനെചെന്നു കൂപ്പുവാന്‍.’\\
എല്ലാമരുള്‍ചെയ്തവണ്ണമെനിക്കതി-\\
നില്ലൊരു വൈമുഖ്യമെന്നു ഭരതനും\\
കാല്‍കഴുകിസ്സമാചമ്യ മുനീന്ദ്രനു-\\
മേകാഗ്രമാനസനായതി വിദ്രുതം.\\
ഹോമഗേഹസ്ഥനായ് ധ്യാനവും ചെയ്തിതു\\
കാമസുരഭിയെ,ത്തല്‍ക്ഷണേ കാനനം\\
ദേവേന്ദ്രലോകസമാനമായ് വന്നിതു,\\
ദേവകളായിച്ചമഞ്ഞു തരുക്കളും\\
ദേവവനിതമാരായി ലതകളും\\
ഭാവനാവൈഭവമെത്രയുമദ്ഭുതം!\\
ഭക്തഭക്ഷ്യാദിപേയങ്ങള്‍ ഭോജ്യങ്ങളും\\
ഭുക്തിപ്രസാധനം മറ്റും ബഹുവിധം\\
ഭോജനശാലഗള്‍ സേനാഗൃഹങ്ങളും\\
രാജഗേഹങ്ങളുമെത്ര മനോഹരം!\\
സ്വര്‍ണരത്നവ്രാതനിര്‍മിതമൊക്കവേ\\
വര്‍ണിപ്പതിന്നു പണിയുണ്ടനന്തനും.\\
കര്‍മണാ ശാസ്ത്രദൃഷ്ടേന വസിഷ്ഠനെ-\\
സ്സമ്മോദമോടു പൂജിച്ചിതു മുമ്പിനാല്‍\\
പശ്ചാത് സസൈന്യം ഭരതം സസോദര-\\
മിച്ഛാനുരൂപേണ പൂജിച്ചനന്തരം\\
തൃപ്തരായ് തത്ര ഭരദ്വാജമന്ദിരേ\\
സുപ്തരായാരമരാവതീസന്നിഭേ.\\
ഉത്ഥാനവും ചെയ്തുഷസി നിയമങ്ങള്‍\\
കൃത്വാ ഭരദ്വാജപാദങ്ങള്‍ കൂപ്പിനാര്‍.\\
താപസന്‍ തന്നോടനുജ്ഞയു കൈക്കൊണ്ടു\\
ഭുപതിനന്ദനന്മാരും പുറപ്പെട്ടു\\
ചിത്രകൂടാചലം പ്രാപ്യ മഹാബലം\\
തത്ര പാര്‍പ്പിച്ചു ദൂരേ കിഞ്ചിദന്തികേ\\
മിത്രമായോരു ഗുഹനും സുമന്ത്രരും\\
ശത്രുഘ്നനും താനുമായി ഭരതനും\\
ശ്രീരാമസന്ദര്‍ശനാകാംക്ഷയാ മന്ദ-\\
മാരാഞ്ഞു നാനാതപോധനമണ്ഡലേ\\
കാണാഞ്ഞൊരോരോ മിനിവരന്മാരോടു\\
താണു തൊഴുതു ചോദിച്ചുമത്യാദരം\\
‘കുത്ര വാഴുന്നു രഘൂത്തമനത്ര സൗ-\\
മിത്രിയോടും മഹീപുത്രിയോടും മുദാ?’\\
ഉത്തമനായ ഭരതകുമാരനോ-\\
ടുത്തരം താപസന്മാരുമരുള്‍ചെയ്തു:\\
‘ഉത്തരതീരേ സുരസരിതഃ സ്ഥലേ\\
ചിത്രകൂടാദ്രിതന്‍പാര്‍ശ്വേ മഹാശ്രമേ\\
ഉത്തമപൂരുഷന്‍ വാഴുന്നി’തെന്നുകേ-\\
ട്ടേത്രയും കൗതുകത്തോടെ ഭരതനും\\
തത്രൈവ ചെന്ന നേരത്തു കാണായ്വന്നി-\\
തത്യത്ഭുതമായ രാമചന്ദ്രാശ്രമം.\\
പുഷ്പഫലദലപൂര്‍ണവല്ലീതരു-\\
ശഷ്പ രമണീയ കനനമണ്ഡലേ\\
ആമ്രകദളീബകുളപനസങ്ങ-\\
ളാമ്രാതകാര്‍ജുനനാഗപുന്നാഗങ്ങള്‍\\
കേരപൂഗങ്ങളും കോവിദാരങ്ങളു-\\
മേരണ്ഡചമ്പകാശോകതാലങ്ങളും\\
മാലതീജാതി പ്രമുഖലതാവലീ-\\
ശാലികളായ തമാലസാലങ്ങളും\\
ഭൃംഗാദി നാനാ വിഹംഗനാദങ്ങളും\\
തുംഗമാതംഗഭുജംഗപ്ലവംഗ കു-\\
രംഗാദിനാനാമൃഗ വ്രാതലീലയും\\
ഭംഗ്യാ സമാലോക്യ ദൂരേ ഭരതനും\\
വൃക്ഷാഗ്രസംലഗ്നവല്കലാലങ്കൃതം\\
പുഷ്കരാക്ഷാശ്രമം ഭക്ത്യാ വണങ്ങിനാന്‍.\\
ഭാഗ്യവാനായ ഭരതനതുനേരം\\
വാര്‍ഗരജസി പതിഞ്ഞുകാണായ്വന്നു\\
സീതാ രഘുനാഥ പാദാരവിന്ദങ്ങള്‍\\
നൂതനമായതി ശോഭനം പാവനം\\
അങ്കുശാബ്ജദ്ധ്വജ വജ്രമത്സ്യാദികൊ-\\
ണ്ടങ്കിതം മംഗലമാനന്ദമഗ്നനായ്\\
വീണുരുണ്ടും പണിഞ്ഞും കരഞ്ഞും തദാ\\
രേണുതന്‍ മൗലിയില്‍ കോരിയിട്ടീടിനാന്‍\\
‘ധന്യോഹമിന്നഹോ ധന്യോഹമിന്നഹോ\\
മുന്നം മയാ കൃതം പുണ്യപൂരം പരം\\
ശ്രീരാമപാദപത്മാഞ്ചിതം ഭൂതല-\\
മാരാലെനിക്കു കാണ്മാനവകാശവും\\
വന്നിതല്ലോ മുഹുരിപ്പാദപാംസുക്ക-\\
ളന്വേഷണം ചെയ്തുഴലുന്നിതേറ്റവും\\
വേധാവുമീശനും ദേവകദംബവും\\
വേദങ്ങളും നാരദാദി മുനികളും.’\\
ഇത്ഥമോര്‍ത്തത്ഭുതപ്രേമരസാപ്ലുത-\\
ചിത്തനായാനന്ദബഷ്പാകുലാക്ഷനായ്\\
മന്ദമന്ദം പരമാശ്രമസന്നിധൗ\\
ചെന്നുനിന്നോരു നേരത്തു കാണായിതു.\\
സുന്ദരം രാമചന്ദ്രം പരമാനന്ദ-\\
മന്ദിരമിന്ദ്രാദിവൃന്ദാരകവൃന്ദ-\\
വന്ദിതമിന്ദിരാമന്ദിരോരസ്ഥല-\\
മിന്ദ്രാവരജമിന്ദീവരലോചനം\\
ദൂര്‍വാദളനിഭശ്യാമളം കോമളം\\
പൂര്‍വജം നീലനളിനദളേക്ഷണം\\
രാമം ജടമകുടം വല്ക്കലാംബരം\\
സോമബിംബാഭപ്രസന്നവക്ത്രാംബുജം\\
ഉദ്യത്തരുണാരുണായുതശോഭിതം\\
വിദ്യുത്സമാംഗിയാം ജാനകിയായൊരു\\
വിദ്യയുമായ് വിനോദിച്ചിരിക്കുന്നൊരു\\
വിദ്യോതമാനമാത്മാനമവ്യാകുലം\\
വക്ഷസി ശ്രീവത്സലക്ഷണമവ്യയം\\
ലക്ഷ്മീനിവാസം ജഗന്മയമച്യുതം\\
ലക്ഷ്മണസേവിതപാദപങ്കരുഹം\\
ലക്ഷ്മണലക്ഷ്യസ്വരൂപം പുരാതനം\\
ദക്ഷാരിസേവിതം പക്ഷീന്ദ്രവാഹനം\\
രക്ഷോവിനാശനം രക്ഷാവിചക്ഷണം\\
ചക്ഷുഃശ്രവണപ്രവരപല്യങ്കഗം\\
കുക്ഷിസ്ഥിതാനേകപത്മജാണ്ഡം പരം\\
കാരുണ്യപൂര്‍ണം ദശരഥനന്ദന-\\
മാരണ്യവാസരസികം മനോഹരം\\
ദുഃഖവും പ്രീതിയും ഭക്തിയുമുള്‍ക്കൊണ്ടു\\
തൃക്കാല്ക്കല്‍ വീണു നമസ്കരിച്ചീടിനാന്‍.\\
രാമനവനെയും ശത്രുഘ്നനേയുമാ-\\
മോദാലെടുത്തു നിവര്‍ത്തിസ്സസംഭ്രമം\\
ദീര്‍ഘബാഹുക്കളാലാലിംഗനംചെദ്യ്തു\\
ദീര്‍ഘനിശ്വാസവുമന്യോന്യമുള്‍ക്കൊണ്ടു\\
ദീര്‍ഘനേത്രങ്ങളില്‍ നിന്നു ബാഷ്പോദകം\\
ദീര്‍ഘകാലം വാര്‍ത്തു സോദരന്മാരെയും\\
ഉത്സംഗസീമനി ചേര്‍ത്തു പുനരപി\\
വത്സങ്ങളുമണച്ചാനന്ദപൂര്‍വകം\\
സത്സംഗമേറെയുള്ളോരു സൗമിത്രിയും\\
തത്സമയേ ഭരതാംഘ്രികള്‍ കൂപ്പിനാന്‍.\\
ശത്രുഘ്നനുമതിഭക്തികലര്‍ന്നു സൗ-\\
മിത്രിതന്‍ പാദാംബുജങ്ങള്‍ കൂപ്പീടിനാന്‍.\\
ഉഗ്രതൃഷാര്‍ത്തമാരായ പശുകുല-\\
മഗ്രേ ജലാശയം കണ്ടപോലേ തദാ\\
വേഗേന സന്നിധൗ ചെന്നാശു കണ്ടിതു\\
രാഘവന്‍ തന്‍ തിരുമേനി മനോധൃതം.\\
രോദനംചെയ്യുന്ന മാതാവിനെക്കണ്ടു\\
പാദങ്ങളില്‍ നമിച്ചാന്‍ രഘുനാഥനും\\
എത്രയുമാര്‍ത്തികൈക്കൊണ്ടു കൗസല്യയും\\
പുത്രനു ബാഷ്പധാരാഭിഷേകംചെയ്തു\\
ഗാഢമാശ്ലിഷ്യ ശിരസി മുകര്‍ന്നുട-\\
നൂഢമോദം മുലയും ചുരന്നു തദാ.\\
അന്യരായുള്ളൊരു മാതൃജനത്തെയും\\
പിന്നെ നമസ്കരിച്ചീടിനാനാദരാല്‍.\\
ലക്ഷ്മണന്‍താനുമവ്വണ്ണം വണങ്ങിനാന്‍\\
ലക്ഷ്മീസമയായ ജാനകീദേവിയും.\\
ഗാഢമാശ്ലിഷ്യ കൗസല്യാദികള്‍ സമാ-\\
രൂഢഖേദം തുടച്ചീടിനാര്‍ കണ്ണുനീര്‍.\\
തത്ര സമാഗതം ദൃഷ്ട്വാ ഗുരുവരം\\
ഭക്ത്യാ വസിഷ്ഠം സാഷ്ടാംഗമാമ്മാറുടന്‍\\
നത്വാ രഘൂത്തമനാശു ചൊല്ലീടിനാ-\\
‘നെത്രയും ഭാഗ്യവാന്‍ ഞാനെന്നു നിര്‍ണയം.\\
താതനു സൗഖ്യമല്ലീ നിജമാനസേ\\
ഖേദമുണ്ടോ പുനരെന്നെപ്പിരികയാല്‍?\\
എന്തോന്നു ചൊന്നതെന്നോടു ചൊല്ലീടുവാ-\\
നെന്തു സൗമിത്രിയെക്കൊണ്ടു പറഞ്ഞതും.’\\
രാമവാക്യം കേട്ടു ചൊന്നാന്‍ വസിഷ്ഠനും:\\
‘ധീമതാം ശ്രേഷ്ഠ! താതോദന്തമാശു കേള്‍.\\
നിന്നെപ്പിരിഞ്ഞതു തന്നെ നിരൂപിച്ചു\\
മന്നവന്‍ പിന്നെയും പിന്നെയും ദുഃഖിച്ചു\\
രാമരാമേതി സീതേതി കുമാരേതി\\
രാമേതി ലക്ഷ്മണേതി പ്രലാപം ചെയ്തു\\
ദേവലോകംചെന്നു പുക്കനറിക നീ\\
ദേവഭോഗേന സുഖിച്ചു സന്തുഷ്ടനായ്.’\\
കര്‍ണശൂലാഭം ഗുരുവചനം സമാ-\\
കര്‍ണ്യ രഘുവരന്‍ വീണിതുഭൂമിയില്‍.\\
തല്‍ക്ഷണമുച്ചൈര്‍ വിലപിച്ചിതേറ്റവും\\
ലക്ഷ്മണനോടും ജനനീജനങ്ങളും.\\
ദുഃഖമാലോക്യ മറ്റുള്ള ജനങ്ങളു-\\
മൊക്കെ വാവിട്ടു കരഞ്ഞുതുടങ്ങിനാര്‍:\\
‘ഹാ! താത! മാം പരിത്യജ്യ വിധിവശാ-\\
ലേതൊരു ദിക്കിനു പോയിതയ്യോ ഭവാന്‍!\\
ഹാ ഹാ ഹതോഹതനാഥോസ്മി മാമിനി\\
സ്നേയേന ലാളിപ്പതാരനുവാസരം?\\
ദേഹമിനി ത്യജിച്ചീടുന്നതുണ്ടു ഞാന്‍\\
മോഹമെനിക്കിനിയില്ല ജീവിക്കയില്‍.’\\
സീതയും സൗമിത്രിതാനുമവ്വണ്ണമേ\\
രോദനം ചെയ്തു വീണീടിനാര്‍ ഭൂതലേ.\\
തദ്ദശായാം വസിഷ്ഠോക്തികള്‍ കേട്ടവ-\\
രുള്‍ത്താപമൊട്ടു ചിരുക്കി മരുവിനാര്‍.\\
മന്ദാകിനിയിലിറങ്ങിക്കുളിച്ചവര്‍\\
മന്ദേതരമുദകക്രിയയും ചെയ്താര്‍.\\
പിണ്ഡം മധുസഹിതേംഗുദീസല്‍ഫല-\\
പിണ്യാകനിര്‍മിതാന്നംകൊണ്ടു വെച്ചിതു\\
യാതൊരന്നം താന്‍ ഭുജിക്കുന്നതുമതു\\
സാദരം നല്ക പിതൃക്കള്‍ക്കുമെന്നാല്ലോ\\
വേദസ്മൃതികള്‍ വിധിച്ചതെന്നോര്‍ത്തതി-\\
ഖേദേന പിണ്ഡദാനാനന്തരം തദാ\\
സ്നാനം കഴിച്ചു പുണ്യാഹവും ചെയ്തഥ\\
സ്നാനാദനന്തരം പ്രാപിച്ചിതാശ്രമം\\
അന്നുപവാസവും ചെയ്തിതെല്ലാവരും\\
വന്നുദിച്ചീടിനാനാദിത്യദേവനും\\
മന്ദാകിനിയില്‍ കുളിച്ചൂത്തു സന്ധ്യയും\\
വന്ദിച്ചു പോന്നാശ്രമേ വസിച്ചീടിനാര്‍.
\end{verse}

%%18). bharatharaaghavasamvaadam

\section{ഭരതരാഘവസംവാദം}

\begin{verse}
അന്നേരമാശു ഭരതനും രാമനെ-\\
ച്ചെന്നു തൊഴുതു പറഞ്ഞു തുടങ്ങിനാന്‍:\\
‘രാമരാമ പ്രഭോ! രാമ! മഹാഭാഗ!\\
മാമകവാക്യം ചെവിതന്നു കേള്‍ക്കണം.\\
ഉണ്ടടിയനഭിഷേകസംഭാരങ്ങള്‍\\
കൊണ്ടുവന്നിട്ട,തുകൊണ്ടിനി വൈകാതെ\\
ചെയ്കവേണമഭിഷേകവും പാലനം\\
ചെയ്ക രാജ്യം തവ പൈത്ര്യം യഥോചിതം.\\
ജ്യേഷ്ഠനല്ലോ ഭവാന്‍ ക്ഷത്രിയാണാമതി\\
ശ്രേഷ്ഠമാം ധര്‍മം പ്രജാപരിപാലനം.\\
അശ്വമേധാദിയും ചെയുതു കീര്‍ത്ത്യാ ചിരം\\
വിശ്വമെല്ലാം പരത്തിക്കുലതന്തവേ\\
പുത്രരേയും ജനിപ്പിച്ചു രാജ്യം നിജ\\
പുത്രങ്കലാക്കി വനത്തിനു പോകണം.\\
ഇപ്പോളനുചിതമത്രേ വനവാസ-\\
മത്ഭുതവിക്രമ! നാഥ! പ്രസീദ മേ.\\
മാതാവുതന്നുടെ ദുഷ്കൃതം താവക-\\
ചേതസി ചിന്തിക്കരുതു ദയാനിധേ!’\\
ഭ്രാതാവുതന്നുടെ പാദാംബുജം ശിര-\\
സ്യാദായ ഭക്തിപൂണ്ടിത്ഥമുരചെയ്തു\\
ദണ്ഡനമസ്കാരവും ചെയ്തു നിന്നിതു\\
പണ്ഡിതനായ ഭരതകുമാരനും\\
ഉത്ഥാപ്യ രാഘവനുത്സംഗമാരോപ്യ\\
ചിത്തമോദേന പുണര്‍ന്നു ചൊല്ലീടിനാന്‍:\\
‘മദ്വാക്യമത്ര കേട്ടാലും കുമാര! നീ\\
യത്ത്വയോക്തം മയാ തത്തഥൈവ ശ്രുതം.\\
താതനെന്നെപ്പതിന്നാലു സംവത്സരം\\
പ്രീതനായ് കാനനം വാഴ്കെന്നു ചൊല്ലിനാന്‍.\\
പ്ത്രാ നിനക്കു രാജ്യം മാതൃസമ്മതം\\
ദത്തമായീ പുനരെന്നതു കാരണം\\
ചേതസാ പാര്‍ക്കില്‍ നമുക്കിരുവര്‍ക്കുമി-\\
ത്താതനിയോഗമനുഷ്ഠിക്കയും വേണം.\\
യാതൊരുത്തന്‍ പിതൃവാക്യത്തെ ലംഘിച്ചു\\
നീതിഹീനം വസിക്കുന്നിതു ഭൂതലേ\\
ജീവന്മൃതനവന്‍ പിന്നെ നരകത്തില്‍\\
മേവും വരിച്ചാലുമില്ലൊരു സംശയം.\\
ആകയാല്‍ നീ പരിപാലിക്ക രാജ്യവും\\
പോക, ഞാന്‍ ദണ്ഡകംതന്നില്‍ വാണീടുവന്‍.’\\
രാമവാക്യം കേട്ടു ചൊന്നാന്‍ ഭരതനും:\\
‘കാമുകനായ താതന്‍ മൂഢമാനസന്‍\\
സ്ത്രീജിതന്‍ ഭ്രാന്തനുന്മത്തന്‍ വയോധികന്‍\\
രാജഭാവംകൊണ്ടു രാജസമാനസന്‍\\
ചൊന്നവാക്യം ഗ്രാഹ്യമല്ല മഹാമതേ!\\
മന്നവനായ് ഭവാന്‍ വാഴ്ക മടിയാതെ.’\\
എന്നു ഭരതവാക്യം കേട്ടു രാഘവന്‍\\
പിന്നെയും വന്ദസ്മിതം ചെയ്തു ചൊല്ലിനാന്‍:\\
‘ഭൂമിഭര്‍ത്താ പിതാ നാരീജിതനല്ല\\
കാമിയുമല്ല മൂഢാത്മാവുമല്ല കേള്‍\\
താതനസത്യഭയംകൊണ്ട് ചെയ്തതി-\\
നേതുമേ ദോഷം പറയരുതോര്‍ക്ക നീ.\\
സാധുജനങ്ങള്‍ നരകത്തിലുമതി-\\
ഭീടിപൂണ്ടീടുമസത്യത്തില്‍ മാനസേ.’\\
‘എങ്കില്‍ ഞാന്‍ വാഴ്വന്‍ വനേ നിന്തിരുവടി\\
സങ്കടമെന്നിയേ രാജ്യവും വാഴുക.’\\
സോദരനിത്ഥം പറഞ്ഞതു കേട്ടതി-\\
സാദരം രാഘവന്‍ പിന്നെയും ചൊല്ലിനാന്‍:\\
രാജ്യം നിനക്കുമെനിക്കു വിപിനവും\\
പൂജ്യനാം താതന്‍ വിധിച്ചതു മുന്നമേ.\\
വ്യത്യയമായനുഷ്ഠിച്ചാല്‍ നമുക്കതു\\
സത്യവിരോധം വരുമെന്നു നിര്‍ണയം.’\\
‘എങ്കില്‍ ഞാനും നിന്തിരുവടി പിന്നാലെ\\
കിങ്കരനായ് സുമിത്രാത്മജനെപ്പോലെ\\
പോരുവന്‍ കാനനത്തിന്നതരുതെങ്കില്‍\\
ചേരുവന്‍ ചെന്നു പരലോകമാശു ഞാന്‍\\
നിത്യോപവാസേന ദേഹമുപേക്ഷിപ്പ’-\\
നിത്യേവമാത്മനി നിശ്ചയിച്ചന്തികേ\\
ദര്‍ഭവിരിച്ചു കിഴക്കു തിരിഞ്ഞു നി-\\
ന്നപ്പോള്‍ വെയിലത്തു പുക്കു ഭരതനും.\\
നിര്‍ബന്ധബുദ്ധി കണ്ടപ്പോള്‍ രഘുവരന്‍\\
തല്‍ബോധനാര്‍ത്ഥം നയനാന്തസംജ്ഞയാ\\
ചൊന്നാന്‍ ഗുരുവിനോടപ്പോള്‍ വസിഷ്ഠനും\\
ചെന്നു കൈകേയീസുതനോടു ചൊല്ലിനാന്‍:’\\
മൂഢനായീടൊലാ കേള്‍ക്ക നീയെങ്കിലോ\\
ഗൂഢമായോരു വൃത്താന്തം നൃപാത്മജ!\\
രാമനാകുന്നതു നാരായണന്‍ പരന്‍\\
താമരസോത്ഭവനര്‍ത്ഥിക്ക കാരണം\\
ഭൂമിയില്‍ സൂര്യകുലത്തിലയോദ്ധ്യയില്‍\\
ഭൂമിപാലാത്മജനായിപ്പിറന്നിതു.\\
രാവണനെക്കൊന്നു ധര്‍മത്തെ രക്ഷിച്ചു\\
ദേവകളേപ്പരിപാലിച്ചു കൊള്ളുവാന്‍.\\
യോഗമായാദേവിയായതു ജാനകി\\
ഭോഗിപ്രവരനാകുന്നതു ലക്ഷ്മണന്‍\\
ലോകമാതാവുംപിതാവും ജനകജാ-\\
രാഘവന്മാരെന്നറിക വഴിപോലെ.\\
രാവണനെക്കൊല്‍വതിന്നു വനത്തിനു\\
ദേവകാര്യാര്‍ത്ഥം പുറപ്പെട്ടു രാഘവന്‍.\\
മന്ഥരാവാക്യവും കൈകേയിചിത്തനിര്‍-\\
ബന്ധവുംദേവകൃതമെന്നറിക നീ.\\
ശ്രീരാമദേവനിവര്‍ത്തനത്തിങ്കലു-\\
ള്ളാഗ്രഹം നീയും പരിത്യജിച്ചീടുക\\
കാരണപൂരുഷാനുജ്ഞയാ സത്വരം\\
നീ രാജധാനിക്കു പോക മടിയാതെ\\
മന്ത്രികളോടും ജനനീജനത്തോടു-\\
മന്തമില്ലാതെ പടയോടുമിപ്പോഴേ\\
ചെന്നയോദ്ധ്യാപുരി പുക്കു വസിക്ക നീ\\
വന്നീടുമഗ്രജന്൬ താനുമനുജനും\\
ദേവിയുമീരേഴു സംവത്സരാവധൗ\\
രാവണന്‍ തന്നെ വധിച്ചു സപുത്രകം’\\
ഇത്ഥം ഗുരൂക്തികള്‍ കേട്ടു ഭരതനും\\
ചിത്തേ വളര്‍ന്നൊരു വിസ്മയം കൈക്കൊണ്ടു\\
ഭക്ത്യാ രഘൂത്തമസന്നിധൗ സാദരം\\
ഗത്വാ മുഹുര്‍ന്നമസ്കൃത്വാ സസോദരം:\\
‘പാദുകാം ദേഹി രാജേന്ദ്ര! രാജ്യായ തേ\\
പാദബുദ്ധ്യാ മമ സേവിച്ചുകൊള്ളുവാന്‍.\\
യാവത്തവാഗമനം ദേവദേവ! മേ\\
താവദേവാനാരതം ഭജിച്ചീടുവന്‍.’\\
ഇത്ഥം ഭരതോക്തി കേട്ടു രഘൂത്തമന്‍\\
പൊല്‍ത്താരടികളില്‍ ചേര്‍ത്ത മെതിയടി\\
ഭക്തിമാനായ ഭരതനു നല്കിനാന്‍;\\
നത്വാ പരിഗ്രഹിച്ചീടിനാന്‍ തമ്പിയും.\\
ഉത്തമരത്നവിഭൂഷിത പാദുകാ-\\
മുത്തമാംഗേ ചേര്‍ത്തു രാമനരേന്ദ്രനെ\\
ഭക്ത്യാ പ്രദക്ഷിണം കൃത്വാ നമസ്കരി-\\
ച്ചുത്ഥായ വന്ദിച്ചു ചൊന്നാന്‍ സഗദ്ഗദം:\\
‘മനബ്ദപൂര്‍ണേ പ്രഥമദിനേ ഭവാന്‍\\
മന്നതില്ലെന്നു വന്നീടുകില്‍ പിന്നെ ഞാന്‍\\
അന്യദിവസമുഷസി വലിപ്പിച്ചു\\
വഹ്നിയില്‍ ചാടി മരിക്കുന്നതുണ്ടല്ലോ.\\
എന്നതു കേട്ടു രഘുപതിയും നിജ\\
കണ്ണുനീരും തുടച്ചന്‍പോടു ചൊല്ലിനാന്‍:\\
‘അങ്ങനെ തന്നെയൊരന്തരമില്ലതി-\\
നങ്ങു ഞാനന്നുതന്നെ വരും നിര്‍ണയം.’\\
എന്നരുള്‍ചെയ്തു വിടയും കൊടുത്തിതു\\
ധന്യന്‍ ഭരതന്‍ നമസ്കരിച്ചീടിനാന്‍.\\
പിന്നെ പ്രദക്ഷിണവും ചെയ്തു വന്ദിച്ചു\\
മന്ദേതരം പുറപ്പെട്ടു ഭരതനും\\
മാതൃജനങ്ങളും മന്ത്രിവരന്മാരും\\
ഭ്രാതാവുമാചാര്യനും മഹാസേനയും\\
ശ്രീരാമദേവനെച്ചേതസി ചേര്‍ത്തുകൊ-\\
ണ്ടാരൂഢമോദേന കൊണ്ടുപോയീടിനാര്‍.\\
ശൃംഗിവേരാധിപനായ ഗുഹനെയും\\
മംഗലവാചാ പറഞ്ഞയച്ചീടിനാന്‍.\\
മുമ്പില്‍ നടന്നു ഗുഹന്‍ വഴി കാട്ടുവാന്‍\\
പിമ്പേ പെരുമ്പടയും നടകൊണ്ടിതു.\\
കൈകേയിതാനും സുതാനുവാദംകൊണ്ടു\\
ശോകമകന്നു നടന്നു മകനുമായ്\\
ഗംഗ കടന്നു ഗുഹാനുവാദേന നാ-\\
ലംഗപ്പടയോട്ളുകൂടെ കുമാരന്മാര്‍\\
ചെന്നയോദ്ധ്യാപുരിപുക്കു രഘുവരന്‍-\\
തന്നെയും ചിന്തിച്ചു ചിന്തിച്ചനുദിനം\\
ഭക്ത്യാ വിശുദ്ധബുദ്ധ്യാ പുരവാസികള്‍\\
നിത്യസുഖേന വസിച്ചിതെല്ലാവരും\\
താപസവേഷം ധരിച്ചു ഭരതനും\\
താപേന ശത്രുഘ്നനും വ്രതത്തോടുടന്‍\\
ചെന്നു നന്ദിഗ്രാമമന്‍പോടു പുക്കിതു\\
വന്നിതാനന്ദം ജഗദ്വാസികള്‍ക്കെല്ലാം.\\
പാദുകാം വെച്ചു സിംഹാസനേ രാഘവ-\\
പാദങ്ങളെന്നു സങ്കല്പിച്ചു സാദരം\\
ഗന്ധപുഷ്പാദ്യങ്ങള്‍ കൊണ്ടു പൂജിച്ചു കൊ-\\
ണ്ടന്തികേ സേവിച്ചു നിന്നാരിരുവരും.\\
നാനാമുനിജനസേവിതനായൊരു\\
മാനവവീരന്‍ മനോഹരന്‍ രാഘവന്‍\\
ജാനകിയോടുമനുജനോടും മുദാ\\
മാനസാനന്ദം കലര്‍ന്നു ചില ദിനം\\
ചിത്രകൂടാചലേ വാണോരനന്തരം\\
ചിത്തേ, നിരൂപിച്ചുകണ്ടു രഘുവരന്‍.\\
‘മിത്രവര്‍ഗങ്ങളയോദ്ധ്യയില്‍നിന്നു വ-\\
ന്നെത്തുമിവിടെയുരുന്നാലിനിയുടന്‍\\
സത്വരം ദണ്ഡകാരണ്യത്തിനായ്ക്കൊണ്ടു\\
ബദ്ധമോദം ഗമിച്ചീടുക വേണ്ടതും’\\
ഇത്ഥം വിചാര്യ ധരിത്രീസുതയുമ-\\
ത്യുത്തമനായ സൗമിത്രിയുമായ്ത്തദാ\\
തത്യാജ ചിത്രകൂടാചലം രാഘവന്‍\\
സത്യസന്ധന്‍ നടകൊണ്ടാന്‍ വനാന്തരേ.
\end{verse}

%%19). athryaashramapravesham

\section{അത്ര്യാശ്രമപ്രവേശം}

\begin{verse}
അത്രി തന്നാശ്രമം പൂക്കു മുനീന്ദ്രനെ\\
ഭക്ത്യാ നമസ്കരിച്ചൂ രഘുനാഥനും\\
‘രാമോഹമദ്യ ധന്യോസ്മി മഹാമുനേ!\\
ശ്രീമല്‍പ്പദം തവ കാണായ കാരണം.’\\
സാക്ഷാല്‍ മഹാവിഷ്ണു നാരായണന്‍ പരന്‍\\
മോക്ഷദനെന്നതറിഞ്ഞു മുനീന്ദ്രനും\\
പൂജിച്ചിതര്‍ഘ്യപാദ്യാദികള്‍കൊണ്ടു തം\\
രാജീവലോചനം ഭ്രാതൃഭാര്യാന്വിതം\\
ചൊല്ലിനാന്‍ ഭൂപാലനന്ദനന്മാരോടു:\\
‘ചൊല്ലെഴുമെന്നുടെ പത്നിയുണ്ടത്ര കേള്‍.\\
എത്രയും വൃദ്ധ, തപസ്വിനിമാരില്‍ വെ-\\
ച്ചുത്തമയായ ധര്‍മജ്ഞാ തപോധനാ\\
പര്‍ണശാലാന്തര്‍ഗൃഹേ വസിക്കുന്നിതു\\
ചെന്നു കണ്ടാലും ജനകനൃപാത്മജേ!’\\
എന്നതു കേട്ടു രാമാജ്ഞയാ ജാനകി\\
ചെന്നനസൂയാപദങ്ങള്‍ വണങ്ങിനാള്‍.\\
‘വത്സേ! വരികരികേ ജനകാത്മജേ!\\
സത്സംഗമം ജന്മസാഫല്യമോര്‍ക്ക നീ.’\\
വത്സേ പിടിച്ചുചേര്‍ത്താലിംഗനം ചെയ്തു\\
തത്സ്വഭാവം തെളിഞ്ഞൂ മുനിപത്നിയും\\
വിശ്വകര്‍മാവിനാല്‍ നിര്‍മിതമായൊരു\\
വിശ്വവിമോഹനമായ ദുകൂലവും\\
കുണ്ഡലവുമംഗരാഗവുമെന്നിവ\\
മണ്ഡനാര്‍ത്ഥമനസൂയ നല്കീടിനാള്‍.\\
‘നന്നു പാത്രിവ്രത്യമാശ്രിത്യ രാഘവന്‍-\\
തന്നോടുകൂടെ നീ പോന്നതുമുത്തമം\\
കാന്തി നിനക്കു കുറയായ്കൊരിക്കലും\\
ശാന്തനാകും തവ വല്ലഭന്‍ തന്നൊടും\\
ചെന്നു മഹാരാജധാനിയകം പുക്കു\\
നന്നായ് സുഖിച്ചു സുചിരം വസിക്ക നീ.’\\
ഇത്ഥമനുഗ്രഹവും കൊടുത്താദരാല്‍\\
ഭര്‍ത്തുരഗ്രേ ഗമിക്കെന്നയച്ചീടിനാള്‍.\\
മൃഷ്ടമായ് മൂവരേയും ഭുജിപ്പിച്ചഥ\\
തുഷ്ടികലര്‍ന്നു തപോധനനത്രിയും\\
ശ്രീരാമനോടരുള്‍ചെയ്തു, ’ഭവാനഹോ\\
നാരായണനായതെന്നറിഞ്ഞേനഹം.\\
നിന്മഹാമായ ജഗത്രയവാസിനാം\\
സമ്മോഹകാരിണിയായതു നിര്‍ണയം.’\\
ഇത്തരമത്രിമുനീന്ദ്രവാക്യം കേട്ടു\\
തത്ര രാത്രൗ വസിച്ചു രഘുനാഥനും.\\
ദേവനുമാദേവിയോടരുളിച്ചെയ്തി-\\
തേവമെന്നാള്‍ കിളിപ്പൈതലക്കാലമേ.
\end{verse}

\begin{center}
ഇത്യദ്ധ്യാത്മരാമായണേ ഉമാമഹേശ്വരസംവാദേ\\
അയോദ്ധ്യാകാണ്ഡം സമാപ്തം
\end{center}


