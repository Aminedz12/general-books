%%01_yuddhakaandam

\chapter{യുദ്ധകാണ്ഡം}

\begin{verse}
നാരായണ! ഹരേ! നാരായണ! ഹരേ!\\
നാരായണ! ഹരേ! നാരായണ! ഹരേ!\\
നാരായണ! രാമ! നാരായണ! രാമ!\\
നാരായണ! രാമ! നാരായണ! ഹരേ!\\
രാമ! രമാരമണ! ത്രിലോകീപതേ!\\
രാമ! സീതാഭിരാമ! ത്രിദശപ്രഭോ!\\
രാമ! ലോകാഭിരാമ! പ്രണവാത്മക!\\
രാമ! നാരായണാത്മാരാമ! ഭൂപതേ!\\
രാമകഥാമൃതപാനപൂര്‍ണാനന്ദ-\\
സാരാനുഭൂതിക്കു സാമ്യമില്ലേതുമേ.\\
ശാരികപ്പൈതലേ! ചൊല്ലുചൊല്ലിന്നിയും\\
ചാരു രാമായണയുദ്ധം മനോഹരം.\\
ഇത്ഥമാകര്‍ണ്യ കിളിമകള്‍ ചൊല്ലിനാള്‍\\
ചിത്തം തെളിഞ്ഞു കേട്ടീടുവിനെങ്കിലോ.\\
ചന്ദ്രചൂഡന്‍ പരമേശ്വരനീശ്വരന്‍\\
ചന്ദ്രികാമന്ദസ്മിതം പൂണ്ടരുളിനാന്‍.\\
ചന്ദ്രാനനേ! ചെവിതന്നു മുദാ രാമ-\\
ചന്ദ്രചരിതം പവിത്രം ശൃണു പ്രിയേ!\\
ശ്രീരാമചന്ദ്രന്‍ ഭുവനൈകനായകന്‍\\
താരകബ്രഹ്മാത്മകന്‍ കരുണാകരന്‍\\
മാരുതി വന്നു പറഞ്ഞതു കേട്ടുള്ളി-\\
ലാരൂഢമോദാലരുള്‍ചെയ്തിതാദരാല്‍.
\end{verse}

%%02_shreeraamaadhikaludenishchayam

\section{ശ്രീരാമാദികളുടെ നിശ്ചയം}

\begin{verse}
‘ദേവകളാലുമസാധ്യമായുള്ളോന്നു\\
കേവലം മാരുതി ചെയ്തതോര്‍ക്കും വിധൗ\\
ചിത്തേ നിരൂപിക്കപോലുമശക്യമാ-\\
മബ്ധി ശതയോജനായതമശ്രമം\\
ലംഘിച്ചു രാക്ഷസവീരരേയും കൊന്നു\\
ലങ്കയും ചുട്ടുപൊട്ടിച്ചിതു വിസ്മയം.\\
ഇങ്ങനെയുള്ള ഭൃത്യന്മാരൊരുത്തനു-\\
മെങ്ങുമൊരുനാളുമില്ലെന്നു നിര്‍ണയം.\\
എന്നെയും ഭാനുവംശത്തെയും ലക്ഷ്മണന്‍-\\
തന്നെയും മിത്രാത്മജനെയും കേവലം\\
മൈഥിലിയെക്കണ്ടു വന്നതു കാരണം\\
വാതാത്മജന്‍ പരിപാലിച്ചിതു ദൃഢം\\
അങ്ങനെയായതെല്ലാ,മിനിയുമുട-\\
നെങ്ങനെ വാരിധിയെക്കടന്നീടുന്നു?\\
നക്ര മകരചക്രാദി പരിപൂര്‍ണ-\\
മുഗ്രമായുള്ള സമുദ്രം കടന്നുപോയ്\\
രാവണനെപ്പടയോടുമൊടുക്കി ഞാന്‍\\
ദേവിയെയെന്നു കാണുന്നിതു ദൈവമേ!’\\
രാമവാക്യം കേട്ടു സുഗ്രീവനും പുന-\\
രാമയം തീരുമാറാശു ചൊല്ലീടിനാന്‍:\\
‘ലംഘനം ചെയ്തു സമുദ്രത്തെയും ബത\\
ലങ്കയും ഭസ്മീകരിച്ചവിളംബിതം\\
രാവണന്‍തന്നെസ്സകുലം കൊലചെയ്തു\\
ദേവിയേയും കൊണ്ടു പോരുന്നതുണ്ടു ഞാന്‍.\\
ചിന്തയുണ്ടാകരുതേതുമേ മാനസേ\\
ചിന്തയാകുന്നതു കാര്യവിനാശിനി.\\
ആരാലുമോര്‍ത്താല്‍ ജയിച്ചുകൂടാതൊരു\\
ശുരരിക്കാണായ വാനരസഞ്ചയം \\
വഹ്നിയില്‍ ചാടേണമെന്നു ചൊല്ലീടിലും\\
പിന്നെയാമെന്നു ചൊല്ലുന്നവരല്ലിവര്‍.\\
വാരിധിയെക്കടപ്പാനുപായം പാര്‍ക്ക\\
നേരമിനിക്കളയാതെ രഘുപതേ!\\
ലങ്കയില്‍ ചെന്നു നാം പുക്കിതെന്നാകിലോ\\
ലങ്കേശനും മരിച്ചാനെന്നു നിര്‍ണയം.\\
ലോകത്രയത്തിങ്കലാരെതിര്‍ക്കുന്നിതു\\
രാഘവാ! നിന്‍ തിരുമുമ്പില്‍ മഹാരണേ\\
അസ്ത്രേണ ശോഷണം ചെയ്ക ജലധിയെ\\
സത്വരം സേതു ബന്ധിക്കിലുമാം ദൃഢം\\
വല്ല കണക്കിലുമുണ്ടാം ജയം തവ\\
നല്ല നിമിത്തങ്ങള്‍ കാണ്‍ക രഘുപതേ!’\\
ഭക്തിശക്ത്യന്വിതമിത്രപുത്രോക്തിക-\\
ളിത്ഥമാകര്‍ണ്യ കാകുല്‍സ്ഥനും തല്‍ക്ഷണേ\\
മുമ്പിലാമ്മാറു തൊഴുതുനില്കും വായു-\\
സംഭവനോടു ചോദിച്ചരുളീടിനാന്‍:
\end{verse}

%%03_lankaavivaranam

\section{ലങ്കാവിവരണം}

\begin{verse}
ലങ്കാപുരത്തിങ്കലുള്ള വൃത്താന്തങ്ങള്‍\\
ശങ്കാവിഹീനമെന്നോടറിയിക്ക നീ\\
കോട്ടമതില്‍ കിടങ്ങെന്നിവയൊക്കവേ\\
കാട്ടിത്തരികവേണം വചസാ ഭവാന്‍.’\\
എന്നതു കേട്ടു തൊഴുതു വാതാത്മജന്‍\\
നന്നായ് തെളീഞ്ഞുണര്‍ത്തിച്ചരുളീടിനാന്‍:\\
‘മധ്യേ സമുദ്രം ത്രികൂടാചലം വളര്‍-\\
ന്നത്യുന്നതമതിന്‍ മൂര്‍ദ്ധ്നി ലങ്കാപുരം\\
പ്രാണഭയമില്ലയാത  ജനങ്ങള്‍ക്കു\\
കാണാം കനകവിമാനസമാനമായ്\\
വിസ്താരമുണ്ടങ്ങെഴുനൂറുയോജന\\
പുത്തന്‍ കനകമതിലതിന്‍ ചുറ്റുമേ\\
ഗോപുരം നാലുദിക്കിങ്കലുമുണ്ടതി-\\
ശോഭിതമായതിനേഴു നിലകളും\\
അങ്ങനെതന്നെയതിനുള്ളിലുള്ളിലായ്\\
പൊങ്ങും മതിലുകളേഴുണ്ടൊരുപോലെ\\
ഏഴിനും നന്നാലു ഗോപുരപംക്തിയും\\
ചൂഴവുമായിരുപത്തെട്ടു ഗോപുരം\\
എല്ലാറ്റിനും കിടങ്ങുണ്ടത്യഗാധമായ്\\
ചൊല്ലുവാന്‍ വേല യന്ത്രപ്പാലപംക്തിയും\\
അണ്ടര്‍കോന്‍ദിക്കിലെഗ്ഗോപുരം കാപ്പതി-\\
നുണ്ടു നിശാചരന്മാര്‍ പതിനായിരം.\\
ദക്ഷിണഗോപുരം രക്ഷിച്ചു നില്ക്കുന്ന\\
രക്ഷോവരരുണ്ടു നൂറായിരം സദാ\\
ശക്തരായ് പശ്ചിമഗോപുരം കാക്കുന്ന\\
നക്തഞ്ചരരുണ്ടു പത്തുനൂറായിരം\\
ഉത്തരഗോപുരം കാത്തുനില്പാനതി-\\
ശക്തരായുണ്ടൊരുകോടി നിശാചരര്‍.\\
ദിക്കുകള്‍ നാലിലുമുള്ളതിലര്‍ദ്ധമു-\\
ണ്ടുഗ്രതയോടു നടുവു കാത്തീടുവാന്‍\\
അന്തഃപുരം കാപ്പതിന്നുമുണ്ടത്ര പേര്‍\\
മന്ത്രശാലയ്ക്കുണ്ടതിലിരട്ടിജ്ജനം.\\
ഹാടക നിര്‍മിത ഭോജനശാലയും\\
നാടകശാല നടപ്പന്തല്‍ പിന്നെയും\\
മജ്ജനശാലയും മദ്യപാനത്തിനു\\
നിര്‍ജനമായുള്ള നിര്‍മലശാലയും\\
ലങ്കാവിരചിതാലങ്കാരഭേദമാ-\\
തങ്കാപഹം പറയാവല്ലനന്തനും\\
തല്‍പുരംതന്നില്‍ നീളെത്തിരഞ്ഞേനഹം\\
മല്‍പിതാവിന്‍ നിയോഗേന ചെന്നേന്‍ ബലാല്‍\\
പുഷ്പിതോദ്യാനദേശേ മനോമോഹനേ\\
പത്മജാദേവിയേയും കണ്ടു കൂപ്പിനേന്‍.\\
അംഗുലീയം കൊടുത്താശു ചൂഡാരത്ന-\\
മിങ്ങു വാങ്ങിക്കൊണ്ടടയാളവാക്യവും\\
കേട്ടു വിടവഴങ്ങിച്ചു പുറപ്പെട്ടു\\
കാട്ടിയേന്‍ പിന്നെക്കുറഞ്ഞോരവിവേകം.\\
ആരാമമെല്ലാം തകര്‍ത്തതു കാക്കുന്ന\\
വീരരെയൊക്കെ ക്ഷണേന കൊന്നീടിനേന്‍.\\
രക്ഷോവരാത്മജനാകിയ ബാലക-\\
നക്ഷകുമാരനവനെയും കൊന്നു ഞാന്‍\\
എന്നുവേണ്ടാ ചുരുക്കിപ്പറഞ്ഞീടുവന്‍\\
മന്നവ! ലങ്കാപുരത്തിങ്കലുള്ളതില്‍\\
നാലൊന്നു സൈന്യമൊടുക്കി വേഗേന പോയ്\\
കാലേ ദശമുഖനെക്കണ്ടു ചൊല്ലിനേന്‍.\\
നല്ലതെല്ലാം പിന്നെ, രാവണന്‍ കോപേന\\
ചൊല്ലിനാന്‍ തന്നുടെ ഭൃത്യരോ’ടിപ്പോഴേ\\
കൊല്ലുക വൈകാതിവനെ’യെന്നന്നേരം\\
കൊല്ലുവാന്‍ വന്നവരോടു വിഭീഷണന്‍\\
ചൊല്ലിനാനഗ്രജന്‍ തന്നോടുമാദരാല്‍:\\
‘കൊല്ലുമാറില്ല ദൂതന്മാരെയാരുമേ\\
ചൊല്ലുള്ള രാജധര്‍മങ്ങളറിഞ്ഞവര്‍\\
കൊല്ലാതയയ്ക്കടയാളപ്പെടുത്തതു\\
നല്ലതാകുന്നതെ’ന്നപ്പോള്‍ ദശാനനന്‍\\
ചൊല്ലിനാന്‍ വാലധിക്കഗ്നി കൊളുത്തുവാന്‍\\
സസ്നേഹവാസസാ പുച്ഛം പൊതിഞ്ഞവ-\\
രഗ്നികൊളുത്തിനാരപ്പോളടിയനും\\
ചുട്ടുപൊട്ടിച്ചേനെഴുന്നൂറു യോജന\\
വട്ടമായുള്ള ലങ്കാപുരം സത്വരം\\
മന്നവ! ലങ്കയിലുള്ള പടയില്‍ നാ-\\
ലൊന്നുമൊടുക്കിനേന്‍ ത്വല്‍ പ്രസാദത്തിനാല്‍.\\
ഒന്നുകൊണ്ടുമിനിക്കാലവിളംബനം\\
നന്നല്ല പോക പുറപ്പെടുകാശു നാം.\\
യുദ്ധസന്നദ്ധരായ് ബദ്ധരോഷം മഹാ-\\
പ്രസ്ഥാനമാശു കുരു ഗുരുവിക്രമം\\
സംഖ്യയില്ലാതോളമുള്ള മഹാകപി-\\
സംഘേന ലങ്കാപുരിക്കു ശങ്കാപഹം\\
ലംഘനം ചെയ്തു നക്തഞ്ചരനായക-\\
കിങ്കരന്മാരെ ക്ഷണേന പിതൃപതി-\\
കിങ്കരന്മാര്‍ക്കു കൊടുത്തു, ദശാനന-\\
ഹുങ്കൃതിയും തീര്‍ത്തു സംഗരാന്തേ ബലാല്‍\\
പങ്കജനേത്രയെക്കൊണ്ടു പോരാം വിഭോ!\\
പങ്കജനേത്ര! പരംപുരുഷ! പ്രഭോ!
\end{verse}

%%04_yuddhayaathra

\section{യുദ്ധയാത്ര}

\begin{verse}
അഞ്ജനാനന്ദനന്‍ വാക്കുകള്‍ കേട്ടഥ\\
സഞ്ജാതകൗതുകം സംഭാവ്യ സാദരം\\
അഞ്ജസാ സുഗ്രീവനോടരുള്‍ചെയ്തിതു\\
കഞ്ജവിലോചനനാകിയ രാഘാവന്‍:\\
‘ഇപ്പോള്‍ വിജയമുഹൂര്‍ത്തകാലം പട-\\
യ്ക്കുല്‍പന്നമോദം പുറപ്പെടുകേവരും.\\
നക്ഷതമുത്രമതും വിജയപ്രദം\\
രക്ഷോജനര്‍ക്ഷമാം മൂലം ഹതിപ്രദം.\\
ദക്ഷിണനേത്രസ്ഫുരണവുമുണ്ടു മേ\\
ലക്ഷണമെല്ലാം നമുക്കു ജയപ്രദം\\
സൈന്യമെല്ലാം പരിപാലിച്ചുകൊള്ളണം\\
സൈന്യാധിപനായ നീലന്‍ മഹാബലന്‍\\
മുമ്പും നടുഭാഗവുമിരുഭാഗവും\\
പിന്‍പടയും പരിപാലിച്ചുകൊള്ളുവാന്‍\\
വമ്പരാം വാനരന്മാരെ നിയോഗിക്ക\\
രംഭപ്രമാഥിപ്രമുഖരായുള്ളവര്‍.\\
മുമ്പില്‍ ഞാന്‍ മാരുതികണ്ഠവുമേറി മല്‍-\\
പിമ്പേ സുമിത്രാത്മജനംഗദോപരി\\
സുഗ്രീവനെന്നെപ്പിരിയാതരികവേ\\
നിര്‍ഗമിച്ചീടുക മറ്റുള്ള വീരരും.\\
നീലന്‍ ഗജന്‍ ഗവയന്‍ ഗവാക്ഷന്‍ ബലി\\
ശൂലി സമാനനാം മൈന്ദന്‍ വിവിദനും\\
പങ്കജസംഭവസൂനു സൂഷേണനും\\
തുംഗന്‍ നളനും ശതബലി താരനും\\
ചൊല്ലുള്ള വാനരനായകന്മാരോടു\\
ചൊല്ലുവാനാവതല്ലാതൊരു സൈന്യവും\\
കൂടിപ്പുറപ്പെടുകേതുമേ വൈകരു-\\
താടലുണ്ടാകരുതാര്‍ക്കും വഴിക്കെടോ!’\\
ഇത്ഥമരുള്‍ചെയ്തു മര്‍ക്കടസൈനിക-\\
മദ്ധ്യേ സഹോദരനോടും രഘുപതി\\
നക്ഷത്രമണ്ഡലമദ്ധ്യേ വിളങ്ങുന്ന\\
നക്ഷത്രനാഥനും ഭാസ്കരദേവനും\\
ആകാശമാര്‍ഗേ വിളങ്ങുന്നതുപോലെ\\
ലോകനാഥന്മാര്‍ തെളിഞ്ഞു വിളങ്ങിനാര്‍.\\
ആര്‍ത്തുവിളിച്ചു കളിച്ചു പുളച്ചു ലോ-\\
കാര്‍ത്തി തീര്‍ത്തീടുവാന്‍ മര്‍ക്കടസഞ്ചയം\\
രാത്രിഞ്ചരേശ്വരരാജ്യംപ്രതി പര-\\
മാസ്ഥയാ വേഗാല്‍ നടന്നു തുടങ്ങിനാര്‍.\\
രാത്രിയിലൊക്കെ നിറഞ്ഞു പരന്നൊരു\\
വാര്‍ദ്ധി നടന്നങ്ങടുക്കുന്നതു പോലെ\\
ചാടിയുമോടിയുമോരോ വനങ്ങളില്‍\\
തേടിയും പക്വഫലങ്ങള്‍ ഭുജിക്കയും\\
ശൈലവനനദീജാലങ്ങള്‍ പിന്നിട്ടു\\
ശൈലശരീരികളായ കപികുലം\\
ദക്ഷിണസിന്ധു തന്നുത്തരതീരവും\\
പുക്കു മഹേന്ദ്രാചലാന്തികേ മേവിനാര്‍.\\
മാരുതിതന്നുടെ കണ്ഠദേശേ നിന്നു\\
പാരിലിറങ്ങീ രഘുകുലനാഥനും\\
താരേയകണ്ഠമമര്‍ന്ന സൗമിത്രിയും\\
പാരിലിഴിഞ്ഞു വണങ്ങിനാനാഗ്രജം.\\
ശ്രീരാമലക്ഷ്മണന്മാരും കപീന്ദ്രരും\\
വാരിധിതീരം പ്രവേശിച്ചനന്തരം\\
സൂര്യനും വാരിധിതന്നുടെ പശ്ചിമ-\\
തീരം പ്രവേശിച്ചിതപ്പോള്‍ നൃപാധിപന്‍\\
സൂര്യാത്മജനോടരുള്‍ചെയ്തിതാശു ’നാം\\
വാരിയുമൂത്തു സന്ധ്യാവന്ദനം ചെയ്തു\\
വാരാന്നിധിയെക്കടപ്പാനുപായവും\\
ധീരരായുള്ളവരൊന്നിച്ചു മന്ത്രിച്ചു\\
പാരാതെ കല്പിക്കവേണമിനിയുടന്‍\\
വാനരസൈന്യത്തെ രക്ഷിച്ചുകൊള്ളണം\\
സേനാധിപന്മാര്‍ കൃശാനുപുത്രാദികള്‍\\
രാത്രിയില്‍ മായാവിശാരദന്മാരായ\\
രാത്രിഞ്ചരന്മാരുപദ്രവിച്ചീടുവോര്‍’\\
ഏവമരുള്‍ചെയ്തു സന്ധ്യയും വന്ദിച്ചു\\
മേവിനാന്‍ പര്‍വതാഗ്രേ രഘുനാഥനും.\\
വാനരവൃന്ദം മകരാലയം കണ്ടു\\
മാനസേ ഭീതികലര്‍ന്നു മരുവിനാര്‍\\
നക്രചക്രൗഘഭയങ്കരമെത്രയു-\\
മുഗ്രം വരുണാലയം ഭീമനിസ്വനം\\
അത്യുന്നതതരംഗാഢ്യമഗാധമി-\\
തുത്തരണം ചെയ്വതിന്നരുതാര്‍ക്കുമേ.\\
ഇങ്ങനെയുള്ള സമുദ്രം കടന്നുചെ-\\
ന്നെങ്ങനെ രാവണന്‍തന്നെ വധിക്കുന്നു?\\
ചിന്താപരവശന്മാരായ് കപികളു-\\
മന്ധബുദ്ധ്യാ രാമപാര്‍ശ്വേ മരുവിനാര്‍;\\
ചന്ദ്രനുമപ്പോഴുദിച്ചു പൊങ്ങീടിനാന്‍\\
ചന്ദ്രമുഖിയെ നിരൂപിച്ചു രാമനും\\
ദുഃഖം കലര്‍ന്നു വിലാപം തുടങ്ങിനാ-\\
നൊക്കെ ലോകത്തെയനുകരിച്ചീടുവാന്‍.\\
ദുഃഖഹര്‍ഷഭയക്രോധലോഭാദികള്‍\\
സൗഖ്യമദമോഹകാമജന്മാദികള്‍\\
അജ്ഞാനലിംഗത്തിനുള്ളവയെങ്ങനെ\\
സുജ്ഞാനരൂപനായുള്ള ചിദാത്മനി\\
സംഭവിക്കുന്നു വിചാരിച്ചു കാണ്‍കിലോ\\
സംഭവിക്കുന്നിതു ദേഹാഭിമാനിനാം\\
കിം പരമാത്മനി സൗഖ്യദുഃഖാദികള്‍\\
സമ്പ്രസാദത്തിങ്കലില്ല രണ്ടേതുമേ.\\
സമ്പ്രതി നിത്യമാനന്ദമാത്രം പരം\\
ദുഃഖാദി സര്‍വവും ബുദ്ധിസംബൂതങ്ങള്‍\\
മുഖ്യനാം രാമന്‍ പരാത്മാ പരംപുമാന്‍\\
മായാഗുണങ്ങളില്‍ സംഗതനാകയാല്‍\\
മായാവിമോഹിതന്മാര്‍ക്കു തോന്നും വൃഥാ\\
ദുഃഖിയെന്നും സുഖിയെന്നുമെല്ലാമതു-\\
മൊക്കെയോര്‍ത്താലബുധന്മാരുടെ മതം.
\end{verse}

%%05_raavanaadikaludeaalochana

\section{രാവണാദികളുടെ ആലോചന}

\begin{verse}
അക്കഥനില്ക്ക, ദശരഥപുത്രരു-\\
മര്‍ക്കാത്മജാദികളായ കപികളും\\
വാരാന്നിധിക്കു വടക്കേക്കര വന്നു\\
വാരിധിപോലെ പരന്നോരനന്തരം\\
ശങ്കാവിഹീനം ജയിച്ചു ജഗത്ത്രയം\\
ലങ്കയില്‍ വാഴുന്ന ലങ്കേശ്വരന്‍ തദാ\\
മന്തികള്‍തമ്മെ വരുത്തി വിരവോടു\\
മന്ത്രനികേതനം പുക്കിരുന്നീടിനാന്‍.\\
ആദിതേയാസുരേന്ദ്രാദികള്‍ക്കു മരു-\\
താത്തൊരു കര്‍മങ്ങള്‍ മാരുതി ചെയ്തതും\\
ചിന്തിച്ചു ചിന്തിച്ചു നാണിച്ചു രാവണന്‍\\
മന്ത്രികളോടു കേള്‍പ്പിച്ചാനവസ്ഥകള്‍:\\
‘മാരുതി വന്നിവിടെച്ചെയ്ത കര്‍മങ്ങ-\\
ളാരുമറിയാതിരിക്കയുമില്ലല്ലോ\\
ആര്‍ക്കും കടക്കരുതാതൊരു ലങ്കയി-\\
ലൂക്കോടുവന്നകം പുക്കൊരു വാനരന്‍\\
ജാനകിതന്നെയും കണ്ടു പറഞ്ഞൊരു\\
ദീനതകൂടാതഴിച്ചാനുപവനം.\\
നക്തഞ്ചരന്മാരെയും വധിച്ചെന്നുടെ\\
പുത്രനാമക്ഷകുമാരനെയും കൊന്നു\\
ലങ്കയും ചുട്ടുപൊട്ടിച്ചു സമുദ്രവും\\
ലംഘനംചെയ്തൊരു സങ്കടമെന്നിയേ\\
സ്വസ്ഥനായ് പോയതോര്‍ത്തോളം നമുക്കുള്ളി-\\
ലെത്രയും നാണമാമില്ലൊരു സംശയം.\\
ഇപ്പോള്‍ കപികുലസേനയും രാമനു-\\
മബ്ധിതന്നുത്തരതീരേ മരുവുന്നോര്‍.\\
കര്‍ത്തവ്യമെന്തു നമ്മാലിനിയെന്നതും\\
ചിത്തേ നിരൂപിച്ചു കല്പിക്ക നിങ്ങളും.\\
മന്ത്രവിശാരദന്മാര്‍ നിങ്ങളെന്നുടെ\\
മന്ത്രികള്‍ ചൊന്നതു കേട്ടതുമൂലമായ്\\
വന്നീലൊരാപത്തിനിയും മമ ഹിതം\\
നന്നായ് വിചാരിച്ചു ചൊല്ലുവിന്‍ വൈകാതെ.\\
എന്നുടെ കണ്ണുകളാകുന്നതും നിങ്ങ-\\
ളെന്നിലേ സ്നേഹവും നിങ്ങള്‍ക്കചഞ്ചലം.\\
ഉത്തമം മദ്ധ്യമം പിന്നേതധമവു-\\
മിത്ഥം ത്രിവിധമായുള്ള വിചാരവും\\
സാദ്ധ്യമിദ, മിദം ദുസ്സാദ്ധ്യമാ, മിദം\\
സാദ്ധ്യമല്ലെന്നുള്ള മൂന്നു പക്ഷങ്ങളും\\
കേട്ടാല്‍ പലര്‍ക്കുമൊരുപോലെ മാനസേ\\
വാട്ടമൊഴിഞ്ഞു തോന്നീടുന്നതും മുദാ\\
തമ്മിലന്യോന്യം പറയുന്ന നേരത്തു\\
സമ്മതം മാമകം നന്നുനന്നീദൃശം.’\\
എന്നുറച്ചൊന്നിച്ചു കല്പിച്ചതുത്തമം\\
പിന്നെ രണ്ടാമതു മധ്യമം ചൊല്ലുവന്‍\\
ഓരോ തരം പറഞ്ഞൂനങ്ങളുള്ളതു\\
തീരുവാനായ് പ്രതിപാദിച്ചനന്തരം\\
നല്ലതിതെന്നൈകമത്യമായേവനു-\\
മുള്ളിലുറച്ചു കല്പിച്ചു പിരിവതു\\
മദ്ധ്യമമായുള്ള മന്ത്രമതെന്നിയേ\\
ചിത്താഭിമാനേന താന്‍താന്‍ പറഞ്ഞതു\\
സാധിപ്പതിന്നു ദുസ്തര്‍ക്കം പറഞ്ഞതു\\
ബാധിച്ചു മറ്റേവനും പറഞ്ഞീര്‍ഷ്യയാ\\
കാലുഷ്യചേതസാ കല്പിച്ചി കൂടാതെ\\
കാലവും ദീര്‍ഘമായീടും പരസ്പരം\\
നിന്ദയും പൂണ്ടു പിരിയുന്ന മന്ത്രമോ\\
നിന്ദ്യമായുള്ളോന്നധമമതെത്രയും.\\
എന്നാലിവിടെ നമുക്കെന്തു നല്ലതെ-\\
ന്നൊന്നിച്ചു നിങ്ങള്‍ വിചാരിച്ചു ചൊല്ലുവിന്‍.’\\
ഇങ്ങനെ രാവണന്‍ ചൊന്നതുകേട്ടള-\\
വിംഗിതജ്ഞന്മാര്‍ നിശാചരര്‍ ചൊല്ലിനാര്‍:\\
‘നന്നുനന്നെത്രയുമോര്‍ത്തോളമുള്ളീലി-\\
തിന്നൊരു കാര്യവിചാരമുണ്ടായതും\\
ലോകങ്ങളെല്ലാം ജയിച്ച ഭവാനിന്നൊ-\\
രാകുലമെന്തു ഭവിച്ചതു മാനസേ?\\
മര്‍ത്ത്യനാം രാമങ്കല്‍ നിന്നു ഭയം തവ\\
ചിത്തേ ഭവിച്ചതുമെത്രയുമത്ഭുതം!\\
വൃത്രാരിയെപ്പുരാ യുദ്ധേ ജയിച്ചുടന്‍\\
ബദ്ധ്വാ വിനിക്ഷിപ്യ പത്തനേ സത്വരം\\
വിശ്രുതയായൊരു കീര്‍ത്തി വളര്‍ത്തതും\\
പുത്രനാം മേഘനിനാദനതോര്‍ക്ക നീ.\\
വിത്തേശനെപ്പുരാ യുദ്ധമദ്ധ്യേ ഭവാന്‍\\
ജിത്വാ ജിതശ്രമം പോരും ദശാന്തരേ\\
പുഷ്പകമായ വിമാനം ഗ്രഹിച്ചതു-\\
മത്ഭുതമെത്രയുമോര്‍ത്തുകണ്ടോളവും.\\
കാലനെപോരില്‍ ജയിച്ച ഭവാനുണ്ടോ\\
കാലദണ്ഡത്താലൊരു ഭയമുണ്ടാവൂ?\\
ഹുങ്കാരമാത്രേണതന്നെ വരുണനെ\\
സംഗരത്തിങ്കല്‍ ജയിച്ചീലയോ ഭവാന്‍?\\
മറ്റുള്ള ദേവകളെപ്പറയേണമോ\\
പറ്റലരാരു മറ്റുള്ളതു ചൊല്ലു നീ.\\
പിന്നെ മയനാം മഹാസുരന്‍ പേടിച്ചു\\
കന്യകാരത്നത്തെ നല്കീലയോ തവ?\\
ദാനവന്മാര്‍ കരം തന്നു പൊറുക്കുന്നു\\
മാനവന്മാരെക്കൊണ്ടെന്തു ചൊല്ലേണമോ?\\
കൈലാസശൈലമിളക്കിയെടുത്തുട-\\
നാലോലമമ്മാനമാടിയ കാരണം\\
കാലാരി ചന്ദ്രഹാസത്തെ നല്കീലയോ\\
മൂലമുണ്ടോ വിഷാദിപ്പാന്‍ മനസി തേ?\\
ത്രൈലോക്യവാസികളെല്ലാം ഭവല്‍ബല-\\
മാലോക്യ ഭീതികലര്‍ന്നു മരുവുന്നു\\
മാരുതിവന്നിവിടെച്ചെയ്ത കര്‍മങ്ങള്‍\\
വീരരായുള്ള നമുക്കോര്‍ക്കില്‍ നാണമാം\\
നാമൊന്നുപേക്ഷിക്ക കാരണാലേതുമൊ-\\
രാമയമെന്നിയേ പൊയ്ക്കൊണ്ടതുമവന്‍\\
ഞങ്ങളാരാനുമറിഞ്ഞാകിലെന്നുമേ-\\
യങ്ങവന്‍ ജീവനോടെ പോകയില്ലല്ലോ.\\
ഇത്ഥം ദശമുഖനോടറിയിച്ചുടന്‍\\
പ്രത്യേഗമോരോ പ്രതിജ്ഞയും ചൊല്ലിനാര്‍.\\
‘മാനമോടിന്നിനി ഞങ്ങളിലേകനെ\\
മാനസേ കല്പിച്ചയയ്ക്കുന്നതാകിലോ\\
മാനുഷജാതികളില്ല ലോകത്തിങ്കല്‍\\
വാനരജാതിയുമില്ലെന്നതും വരും\\
ഇന്നൊരു കാര്യവിചാരമാക്കിപ്പല-\\
രൊന്നിച്ചുകൂടി നിരൂപിക്കയെന്നതും\\
എത്രയും പാരമിളപ്പം നമുക്കതു-\\
മുള്‍ത്താരിലോര്‍ത്തരുളേണം ജഗല്‍പ്രഭോ!’\\
നക്തഞ്ചരവരരിത്ഥം പറഞ്ഞള-\\
വുള്‍ത്താപമൊട്ടു കുറഞ്ഞു ദശാസ്യനും.
\end{verse}

%%06_raavanakumbhakarnasambhaashanam

\section{രാവണകുംഭകര്‍ണസംഭാഷണം}

\begin{verse}
നിദ്രയും കൈവിട്ടു കുംഭകര്‍ണന്‍ തദാ\\
വിദ്രുതമഗ്രജന്‍തന്നെ വണങ്ങിനാന്‍.\\
ഗാഢഗാഢം പുണര്‍ന്നൂഢമോദം നിജ\\
പീഠമതിന്മേലിരുത്തി ദശാസ്യനും\\
വൃത്താന്മമെല്ലാമവരജന്‍ തന്നോടു\\
ചിത്താനുരാഗേണ കേള്‍പ്പിച്ചനന്തരം\\
ഉള്‍ത്താരിലുണ്ടയ ഭീതിയോടുമവന്‍\\
നക്തഞ്ചരാധീശ്വരനോടു ചൊല്ലിനാന്‍:\\
"ജീവിച്ചു ഭൂമിയില്‍ വാഴ്കെന്നതില്‍ മമ\\
ദേവത്വമാശു കിട്ടുന്നതു നല്ലതും\\
ഇപ്പോള്‍ ഭവാന്‍ ചെയ്ത കര്‍മങ്ങളൊക്കെയും\\
ത്വല്‍ പ്രാണഹാനിക്കുതന്നെ ധരിക്ക നീ.\\
രാമന്‍ ഭവാനെ ക്ഷണം കണ്ടുകിട്ടുകില്‍\\
ഭൂമിയില്‍ വാഴ്വാനയയ്ക്കയില്ലെന്നുമേ.\\
ജീവിച്ചിരിക്കയിലാഗ്രഹമുണ്ടെങ്കില്‍\\
സേവിച്ചുകൊള്ളുക രാമനെ നിത്യമായ്.\\
രാമന്‍ മനുഷ്യനല്ലേകസ്വരൂപനാം\\
ശ്രീമാന്‍ മഹാവിഷ്ണു നാരായണന്‍ പരന്‍.\\
സീതയാകുന്നതു ലക്ഷ്മീഭഗവതി\\
ജാതയായാള്‍ തവ നാശം വരുത്തുവാന്‍.\\
മോഹേന നാദഭേദം കേട്ടു ചെന്നുടന്‍\\
ദേഹനാശം മൃഗങ്ങള്‍ക്കു വരുന്നിതു\\
മീനങ്ങളെല്ലാം രസത്തിങ്കല്‍ മോഹിച്ചു\\
താനേ ബളിശം വിഴുങ്ങി മരിക്കുന്നു\\
അഗ്നിയെക്കണ്ടു മോഹിച്ചു ശലഭങ്ങള്‍\\
മഗ്നമായ് മൃത്യുഭവിക്കുന്നിതവ്വണ്ണം\\
ജാനകിയെക്കണ്ടു മോഹിക്ക കാരണം\\
പ്രാണവിനാശം ഭവാനുമകപ്പെടും.\\
നല്ലതല്ലേതുമെനിക്കിതെന്നുള്ളതും\\
ഉള്ളിലറിഞ്ഞിരിക്കുന്നിതെന്നാകിലും\\
ചെല്ലുമതിങ്കല്‍ മനസ്സതിന്‍ കാരണം\\
ചൊല്ലുവന്‍ മുന്നം കഴിഞ്ഞ ജന്മത്തിലെ\\
വാസനകൊണ്ടതു നീക്കരുതാര്‍ക്കുമേ\\
ശാസനയാലുമടങ്ങുകയില്ലതു\\
വിജ്ഞാനമുള്ള ദിവ്യന്മാര്‍ക്കുപോലും, മ-\\
റ്റജ്ഞാനികള്‍ക്കോ പറയേണ്ടതില്ലല്ലോ!\\
കാട്ടിയതെല്ലാമപനയം നീയതു\\
നാട്ടിലുള്ളോര്‍ക്കുമാപത്തിനായ് നിര്‍ണയം.\\
ഞാനിതിനിന്നിനി രാമനേയും മറ്റു\\
വാനരന്മാരെയുമൊക്കെയൊടുക്കുവന്‍\\
ജാനകിതന്നെയനുഭവിച്ചീടു നീ\\
മാനസേ ഖേദമൂണ്ടാകരുതേതുമേ.\\
ദേഹത്തിനന്തരം വന്നുപോം മുന്നമേ\\
മോഹിച്ചതാഹന്ത! സാധിച്ചുകൊള്‍ക നീ\\
ഇന്ദ്രിയങ്ങള്‍ക്കു വശനാം പുരുഷനു\\
വന്നീടുമാപത്തു നിര്‍ണയമോര്‍ത്തു കാണ്‍.\\
ഇന്ദ്രിയനിഗ്രഹമുള്ള പുരുഷനു\\
വന്നുകൂടും നിജ സൗഖ്യങ്ങളൊക്കവേ.’\\
ഇന്ദ്രാരിയാം കുംഭകര്‍ണോക്തികേട്ടള-\\
വിന്ദ്രജിത്തും പറഞ്ഞീടിനാനാദരാല്‍:\\
‘മാനുഷനാകിയ രാമനെയും മറ്റു\\
വാനരന്മാരെയുമൊക്കെയൊടുക്കി ഞാന്‍\\
ആശു വരുവനനുജ്ഞയെച്ചെയ്കിലെ’-\\
ന്നാശരാധീശ്വരനോടു ചൊല്ലീടിനാന്‍.
\end{verse}

%%07_raavanavibheeshanasambhaashanam

\section{രാവണവിഭീഷണസംഭാഷണം}

\begin{verse}
അന്നേരമാഗതനായ വിഭീഷണന്‍\\
ധന്യന്‍ നിജാഗ്രജന്‍തന്നെ വണങ്ങിനാന്‍.\\
തന്നരികത്തങ്ങിരുത്തിദ്ദശാനനന്‍\\
ചൊന്നാനവനോടു പഥ്യം വിഭീഷണന്‍:\\
‘രാക്ഷസാധീശ്വര! വീര! ദശാനന!\\
കേള്‍ക്കണമെന്നുടെ വാക്കുകളിന്നു നീ.\\
നല്ലതു ചൊല്ലേണമെല്ലാവരും തനി-\\
ക്കുള്ളവരോടു ചൊല്ലുള്ള ബുധജനം\\
കല്യാണമെന്തു കുലത്തിനെന്നുള്ളതു-\\
മെല്ലാവരുമൊരുമിച്ചു ചിന്തിക്കണം.\\
യുദ്ധത്തിനാരുള്ളതോര്‍ക്ക നീ രാമനോ-\\
ടിത്രിലോകത്തിങ്കല്‍ നക്തഞ്ചരാധിപ?\\
മത്തനുന്മത്തന്‍ പ്രഹസ്തന്‍ വികടനും\\
സുപ്തഘ്നയജ്ഞാന്തകാദികളും തഥാ\\
കുംഭകര്‍ണന്‍ ജംബുമാലി പ്രജംഘനും\\
കുംഭന്‍ നികുംഭനകമ്പനന്‍ കമ്പനന്‍\\
വമ്പന്‍ മഹോദരനും മഹാപാര്‍ശ്വനും\\
കുംഭഹനും ത്രിശിരസ്സതികായനും\\
ദേവാന്തകനും നരാന്തകനും മറ്റു\\
ദേവാരികള്‍ വജ്രദംഷ്ട്രാദി വീരരും\\
യൂപാക്ഷനും ശോണിതാക്ഷനും പിന്നെ വി-\\
രൂപാക്ഷധൂമ്രാക്ഷനും മകരാക്ഷനും\\
ഇന്ദ്രനെസ്സംഗരേ ബന്ധിച്ച വീരനാ-\\
മിന്ദ്രജിത്തിന്നുമാമല്ലവനോടെടോ!\\
നേരേ പൊരുതു ജയിപ്പതിനാരുമേ\\
ശ്രീരാമനോടു കരുതായ്ക മാനസേ.\\
ശ്രീരാമനായതു മാനുഷനല്ല കേ-\\
ളാരെന്നറിവാനുമാമല്ലൊരുവനും.\\
ദേവേന്ദ്രനുമല്ല വഹ്നിയുമല്ലവന്‍\\
വൈവസ്വതനും നിര്യതിയുമല്ല കേള്‍.\\
പാശിയുമല്ല ജഗല്‍പ്രാണനല്ല വി-\\
ത്തേശനുമല്ലവനീശാനനുമല്ല\\
വേധാവുമല്ല ഭുജംഗാധിപനുമ-\\
ല്ലാദിത്യരുദ്രവസുക്കളുമല്ലവന്‍.\\
സാക്ഷാല്‍ മഹാവിഷ്ണു നാരായണന്‍ പരന്‍\\
മോക്ഷദന്‍ സൃഷ്ടിസ്ഥിതിലയകാരണന്‍.\\
മുന്നം ഹിരണ്യാക്ഷനെക്കൊല ചെയ്തവന്‍\\
പന്നിയായ് മന്നിടം പാലിച്ചുകൊള്ളുവാന്‍.\\
പിന്നെ നരസിംഹരൂപം ധരിച്ചിട്ടു\\
കൊന്നു ഹിരണ്യകശിപുവാം വീരനെ.\\
ലോകൈകനായകന്‍ വാമനമൂര്‍ത്തിയായ്\\
ലോകത്രയം ബലിയോടു വാങ്ങീടിനാന്‍.\\
കൊന്നാനിരുപത്തൊരുതുട രാമനായ്\\
മന്നവന്മാരെയസുരാംശമാകയാല്‍\\
അന്നന്നസുരരെയൊക്കെയൊടുക്കുവാന്‍\\
മന്നിലവതരിച്ചീടും ജഗന്മയന്‍.\\
ഇന്നു ദശരഥപുത്രനായ് വന്നിതു\\
നിന്നെയൊടുക്കുവാനെന്നറിഞ്ഞീടു നീ.\\
സത്യസങ്കല്പനാമീശ്വരന്‍ തന്മതം\\
മിഥ്യയായ് വന്നു കൂടായെന്നു നിര്‍ണയം.\\
എങ്കിലെന്തിന്നു പറയുന്നതെന്നൊരു\\
ശങ്കയുണ്ടാകിലതിന്നു ചൊല്ലീടുവന്‍.\\
സേവിപ്പവര്‍ക്കഭയത്തെക്കൊടുപ്പോരു\\
ദേവനവന്‍ കരുണാകരന്‍ കേവലന്‍.\\
ഭക്തപ്രിയന്‍ പരമന്‍ പരമേശ്വരന്‍\\
ഭുക്തിയും മുക്തിയും നല്കും ജനാര്‍ദനന്‍\\
ആശ്രിതവത്സലനംബുജലോചന-\\
നീശ്വരനിന്ദിരാവല്ലഭന്‍ കേശവന്‍.\\
ഭക്തിയോടും തന്‍തിരുവടി തന്‍ പദം\\
നിത്യമായ് സേവിച്ചു കൊള്‍ക മടിയാതെ.\\
മൈഥിലീദേവിയെക്കൊണ്ടെക്കൊടുത്തു തല്‍-\\
പാദാംബുജത്തില്‍ നമസ്കരിച്ചീടുക.\\
കൈതൊഴുതാശു രക്ഷിക്കെന്നു ചൊല്ലിയാല്‍\\
ചെയ്തപരാധങ്ങളെല്ലാം ക്ഷമിച്ചവന്‍\\
തന്‍പദം നല്കീടുമേവനും നമ്മുടെ\\
തമ്പുരാനോളം കൃപയില്ല മറ്റാര്‍ക്കും.\\
കാടകം പുക്കനേരത്തതിബാലകന്‍\\
താടകയെക്കൊലചെയ്താനൊരമ്പിനാല്‍.\\
കൗശികന്‍ തന്നുടെ യാഗരക്ഷാര്‍ഥമായ്\\
നാശം സുബാഹുമുഖ്യന്മാര്‍ക്കു നല്കിനാന്‍.\\
തൃക്കാലടിവെച്ചു കല്ലാമഹല്യയ്ക്കു\\
ദുഷ്കൃതമെല്ലാമൊടുക്കിയതോര്‍ക്ക നീ.\\
ത്രൈയംബകം വില്ലു ഖണ്ഡിച്ചു സീതയാം\\
മയ്യല്‍മിഴിയാളെയും കൊണ്ടുപോകുമ്പോള്‍\\
മാര്‍ഗമധ്യേ കുഠാരായുധനാകിയ\\
ഭാര്‍ഗവന്‍തന്നെജ്ജയിച്ചതുമത്ഭുതം.\\
പിന്നെ വിരാധനെക്കൊന്നുകളഞ്ഞതും\\
ചെന്ന ഖരാദികളെക്കൊലചെയ്തതും\\
ഉന്നതനാകിയ ബാലിയെക്കൊന്നതും\\
മന്നവനാകിയ രാഘവനല്ലയോ?\\
അര്‍ണവം ചാടിക്കടന്നിവിടേക്കുവ-\\
ന്നര്‍ണോജനേത്രയെക്കണ്ടു പറഞ്ഞുടന്‍\\
വഹ്നിക്കു ലങ്കാപുരത്തെസ്സമര്‍പ്പിച്ചു\\
സന്നദ്ധനായ്പ്പോയ മാരുതി ചെയ്തതും\\
ഒന്നൊഴിയാതെയറിഞ്ഞിരിക്കെ തവ\\
നന്നുനന്നാഹന്ത! തോന്നുന്നതിങ്ങനെ!\\
നന്നല്ല സജ്ജനത്തോടു വൈരം വൃഥാ\\
തന്വംഗി തന്നെക്കൊടുക്ക മടിയാതെ.\\
നഷ്ടമതികളായീടുമമാത്യന്മാ-\\
രിഷ്ടം പറഞ്ഞു കൊല്ലിക്കുമതോര്‍ക്ക നീ\\
കാലപുരം ഗമിയാതിരിക്കേണ്ടുകില്‍\\
കാലം കളയാതെ നല്ക വൈദേഹിയെ\\
ദുര്‍ബലനായുള്ളവന്‍ പ്രബലന്‍ തന്നോ-\\
ടുള്‍പ്പൂവില്‍ മത്സരം വെച്ചു തുടങ്ങിയാല്‍\\
പില്പാടു നാടും നഗരവും സേനയും\\
തല്‍പ്രാണനും നശിച്ചീടുമരക്ഷണാല്‍.\\
ഇഷ്ടം പറയുന്ന ബന്ധുക്കളാരുമേ\\
കഷ്ടകാലത്തിങ്കലില്ലെന്നു നിര്‍ണയം.\\
തന്നുടെ ദുര്‍ന്നയംകൊണ്ടു വരുന്നതി-\\
നിന്നു നാമാളല്ല പോകെന്നു വേര്‍പെട്ടു\\
ചെന്നു സേവിക്കും പ്രബലനെ ബന്ധുക്ക-\\
ളന്നേരമോര്‍ത്താല്‍ ഫലമില്ല മന്നവ!\\
രാമശരമേറ്റു മൃത്യു വരുന്നേര-\\
മാമയമുള്ളിലെനിക്കുണ്ടതുകൊണ്ടു\\
നേരെ പറഞ്ഞുതരുന്നതു ഞാ, നിനി\\
താരാര്‍മകളെക്കൊടുക്ക വൈകീടാതെ.\\
യുദ്ധമേറ്റുള്ള പടയും നശിച്ചുട-\\
നര്‍ഥവുമെല്ലാമൊടുങ്ങിയാല്‍ മാനസേ\\
മാനിനിയെക്കൊടുക്കാമെന്നു തോന്നിയാല്‍\\
സ്ഥാനവുമില്ല കൊടുപ്പതിനോര്‍ക്ക നീ.\\
മുമ്പിലേയുള്ളില്‍ വിചാരിച്ചുകൊള്ളണം\\
വമ്പനോടേറ്റാല്‍വരും ഫലമേവനും.\\
ശ്രീരാമനോടു കലഹം തുടങ്ങിയാ-\\
ലാരും ശരണമില്ലെന്നതറിയണം.\\
പങ്കജനേത്രനെസ്സേവിച്ചു വാഴുന്നു\\
ശങ്കരനാദികളെന്നതുമോര്‍ക്ക നീ\\
രാക്ഷസരാജ! ജയിക്ക ജയിക്ക നീ\\
സാക്ഷാല്‍ മഹേശ്വരനോടു പിണങ്ങൊലാ.\\
കൊണ്ടല്‍നേര്‍വര്‍ണനു ജാനകീദേവിയെ\\
കൊണ്ടെക്കൊടുത്തു സുഖിച്ചു വസിക്ക നീ.\\
സംശയമെന്നിയേ നല്കുക ദേവിയെ\\
വംശം മുടിച്ചുകളയായ്ക വേണമേ.’\\
ഇത്ഥം വിഭീഷണന്‍ പിന്നെയും പിന്നെയും\\
പത്ഥ്യമായുള്ളതു ചൊന്നതു കേട്ടൊരു\\
നക്തഞ്ചരാധിപനായ ദശാസ്യനും\\
ക്രുദ്ധനായ് സോദരനോടു ചൊല്ലീടിനാന്‍:\\
‘ശത്രുക്കളല്ല ശത്രുക്കളാകുന്നതു\\
മിത്രഭാവത്തോടരികേ മരുവിന\\
ശത്രുക്കള്‍ ശത്രുക്കളാകുന്നതേവനും\\
മൃത്യുവരുത്തുമവരെന്നു നിര്‍ണയം.\\
ഇത്തരമെന്നോടു ചൊല്ലുകിലാശു നീ\\
വധ്യനാമെന്നാലതിനില്ല സംശയം.’\\
രാത്രിഞ്ചരാധിപനിത്തരം ചൊന്നള-\\
വോര്‍ത്താന്‍ വിഭീഷണന്‍ ഭാഗവതോത്തമന്‍:\\
‘മൃത്യുവശഗതനായ പുരുഷനു\\
സിദ്ധൗഷധങ്ങളുമേല്ക്കയില്ലേതുമേ,\\
പോരുമിവനോടിനി ഞാന്‍ പറഞ്ഞതു\\
പൗരുഷംകൊണ്ടു നീക്കാമോ വിധിമതം?\\
ശ്രീരാമദേവപാദാംഭോജമെന്നി മ-\\
റ്റാരും ശരണമെനിക്കില്ല കേവലം.\\
ചെന്നു തൃക്കാല്ക്കല്‍ വീണന്തികേ സന്തതം\\
നിന്നു സേവിച്ചുകൊള്‍വന്‍ ജന്മമുള്ള നാള്‍.’\\
സത്വരം നാലമാത്യന്മാരുമായവ-\\
നിത്ഥം നിരൂപിച്ചുറച്ചു പുറപ്പെട്ടു.\\
ദാരധനാലയമിത്രഭൃത്യൗഘവും\\
ദൂരെ പരിത്യജ്യ രാമപാദാംബുജം\\
മാനസത്തിങ്കലുറപ്പിച്ചു തുഷ്ടനായ്\\
വീണുവണങ്ങിനാനഗ്രജന്‍ തന്‍പദം.\\
കോപിച്ചു രാവണന്‍ ചൊല്ലിനാനന്നേര-\\
‘മാപത്തെനിക്കു വരുത്തുന്നതും ഭവാന്‍;\\
രാമനെച്ചെന്നു സേവിച്ചു കൊണ്ടാലുമൊ-\\
രാമയമിങ്ങതിനില്ലെന്നു നിര്‍ണയം.\\
പോകായ്കിലോ, മമ ചന്ദ്രഹാസത്തിനി-\\
ന്നേകാന്തഭോജനമായ് വരും നീയെടോ!’\\
എന്നതു കേട്ടു വിഭീഷണന്‍ ചൊല്ലിനാ-\\
‘നെന്നുടെ താതനു തുല്യനല്ലോ ഭവാന്‍\\
താവകമായ നിയോഗമനുഷ്ഠിപ്പ-\\
നാവതെല്ലാമതു സൗഖ്യമല്ലോ മമ\\
സങ്കടം ഞാന്‍മൂലമുണ്ടാകരുതേതു-\\
മെങ്കിലോ ഞാനിതാ വേഗേന പോകുന്നു.\\
പുത്രമിത്രാര്‍ത്ഥകളത്രാദികളോടു-\\
മത്ര സുഖിച്ചു സുചിരം വസിക്ക നീ\\
മൂലവിനാശം നിനക്കു വരുത്തുവാന്‍\\
കാലന്‍ ദശരഥമന്ദിരേ രാമനായ്\\
ജാതനായാന്‍ ജനകാലയേ കാലിയും\\
സീതാഭിധാനേന ജാതയായീടീനാള്‍\\
ഭൂമിഭാരം കളഞ്ഞീടുവാനായ് മുതിര്‍-\\
ന്നാമോദമോടിങ്ങു വന്നാരിരുവരും.\\
എങ്ങനെ പിന്നെ ഞാന്‍ ചൊന്ന ഹിതോക്തിക-\\
ളങ്ങു ഭവാനുള്ളിലേല്ക്കുന്നതു പ്രഭോ!\\
രാവണന്‍തന്നെ വധിപ്പാനവനിയില്‍\\
ദേവന്‍ വിധാതാവപേക്ഷിച്ച കാരണം\\
വന്നു പിറന്നിതു രാമനായ് നിര്‍ണയം\\
പിന്നെയതിന്നന്യഥാത്വം ഭവിക്കുമോ?\\
ആശരവംശവിനാശം വരും മുമ്പേ\\
ദാശരഥിയെ ശരണം ഗതോസ്മി ഞാന്‍.’
\end{verse}

%%08_vibheeshananshreeraamasannidhiyil

\section{വിഭീഷണന്‍ ശ്രീരാമസന്നിധിയില്‍}

\begin{verse}
രാവണന്‍തന്‍ നിയോഗേന വിഭീഷണന്‍\\
ദേവദേവേശപാദാബ്ജസേവാര്‍ത്ഥമായ്\\
ശോകം വിനാ നാലമാത്യരുമായുട-\\
നാകാശമാര്‍ഗേ ഗമിച്ചാനതിദ്രുതം.\\
ശ്രീരാമദേവനിരുന്നരുളുന്നതിന്‍\\
നേരേ മുകളില്‍നിന്നുച്ചൈസ്തരമവന്‍\\
വ്യക്തവര്‍ണേന ചൊല്ലീടിനാനെത്രയും\\
ഭക്തിവിനയവിശുദ്ധമതിസ്ഫുടം:\\
‘രാമ! രമാരമണ! ത്രിലോകീപതേ!\\
സ്വാമിന്‍! ജയജയ! നാഥ! ജയ ജയ!\\
രാജീവനേത്ര! മുകുന്ദ! ജയ ജയ!\\
രാജശിഖാമണേ! സീതാപതേ! ജയ!\\
രാവണന്‍തന്നുടെ സോദരന്‍ ഞാന്‍ തവ\\
സേവാര്‍ത്ഥമായ് വിടകൊണ്ടേന്‍ ദയാനിധേ!\\
ആമ്നായമൂര്‍ത്തേ! രഘുപതേ! ശ്രീപതേ!\\
നാമ്നാ വിഭീഷണന്‍ ത്വല്‍ഭക്തസേവകന്‍.\\
‘ദേവിയെക്കട്ടതനുചിതം നീ’യെന്നു\\
രാവണനോടു ഞാന്‍ നല്ലതു ചൊല്ലിയേന്‍.\\
ദേവിയെ ശ്രീരാമനായ്ക്കൊണ്ടു നല്കുകെ-\\
ന്നാവോളമേറ്റം പറഞ്ഞേന്‍ പലതരം\\
വിജ്ഞാനമാര്‍ഗമെല്ലാമുപദേശിച്ച-\\
തജ്ഞാനിയാകയാലേറ്റതില്ലേതുമേ.\\
പഥ്യമായ് വന്നിതവന്നു വിധിവശാല്‍.\\
വാളുമായെന്നെ വധിപ്പാനടുത്തിതു\\
കാളഭുജംഗവേഗേന ലങ്കേശ്വരന്‍\\
മൃത്യുഭയത്താലടിയനുമെത്രയും\\
ചിത്താകുലതയാ പാഞ്ഞു പാഞ്ഞിങ്ങിഹ\\
നാലമാത്യന്മാരുമായ് വിടകൊണ്ടേനൊ-\\
രാലംബനം മറ്റെനിക്കില്ല ദൈവമേ!\\
ജന്മമരണമോക്ഷാര്‍ത്ഥം ഭവച്ചര-\\
ണാംബുജം മേ ശരണം കരുണാംബുധേ!’\\
ഇത്ഥം വിഭീഷണവാക്യങ്ങള്‍ കേട്ടള-\\
വുത്ഥായ സുഗ്രീവനും പറഞ്ഞീടിനാന്‍:\\
‘വിശ്വേശ! രാക്ഷസന്‍ മായാവിയെത്രയും\\
വിശ്വാസയോഗ്യനല്ലെന്നതു നിര്‍ണയം.\\
പിന്നെ, വിശേഷിച്ചു രാവണരാക്ഷസന്‍\\
തന്നുടെ സോദരന്‍ വിക്രമമുള്ളവന്‍\\
ആയുധപാണിയായ് വന്നാനമാത്യരും\\
മായാവിശാരദന്മാരെന്നു നിര്‍ണയം.\\
ഛിദ്രം കുറഞ്ഞൊന്നു കാണ്‍കിലും നമ്മുടെ\\
നിദ്രയിലെങ്കിലും നിഗ്രഹിച്ചീടുമേ.\\
ചിന്തിച്ചുടന്‍ നിയോഗിക്ക കപികളെ\\
ഹന്തവ്യനിന്നിവനില്ലൊരു സംശയം.;\\
ശത്രുപക്ഷത്തിങ്കലുള്ള ജനങ്ങളെ\\
മിത്രമെന്നോര്‍ത്തുടന്‍ വിശ്വസിക്കുന്നതില്‍\\
ശത്രുക്കളെത്തന്നെ വിശ്വസിച്ചീടുന്ന-\\
തുത്തമമാകുന്നതെന്നതോര്‍ക്കേണമേ.\\
ചിന്തിച്ചുകണ്ടിനി നിന്തിരുവുള്ളത്തി-\\
ലെന്തെന്നഭിമതമെന്നരുള്‍ചെയ്യണം.\\
മറ്റുള്ള വാനരവീരരും ചിന്തിച്ചു\\
കുറ്റം വരായ്വാന്‍ പറഞ്ഞാര്‍ പലതരം.\\
അന്നേരമുത്ഥായ വന്ദിച്ചു മാരുതി\\
ചൊന്നാന്‍, ‘വിഭീഷണനുത്തമനെത്രയും\\
വന്നു ശരണം ഗമിച്ചവന്‍തന്നെ നാം\\
നന്നു രക്ഷിക്കുന്നതെന്നെന്നുടെ മതം\\
നക്തഞ്ചരാന്വയത്തിങ്കല്‍ ജനിച്ചവര്‍\\
ശത്രുക്കളേവരുമെന്നു വന്നീടുമോ?\\
നല്ലവരുണ്ടാമവരിലുമെന്നുള്ള-\\
തെല്ലാവരും നിരൂപിച്ചുകൊള്ളേണമേ.\\
ജാതിനാമാദികള്‍ക്കല്ല ഗുണഗണ-\\
ഭേദമെന്നത്രേ ബുധന്മാരുടെ മതം\\
ശാശ്വതമായുള്ള ധര്‍മം നൃപതികള്‍-\\
ക്കാശ്രിതരക്ഷണമെന്നു ശാസ്ത്രോക്തിയും.’\\
ഇത്ഥം പലരും പലവിധം ചൊന്നവ\\
ചിത്തേ ധരിച്ചരുള്‍ചെയ്തു രഘുപതി:\\
‘മാരുതി ചൊന്നതുപപന്നമെത്രയും\\
വീര! വിഭാകരപുത്ര! വരികെടോ!\\
ഞാന്‍ പറയുന്നതു കേള്‍പ്പിനെല്ലാവരും\\
ജാംബവദാദി നീതിജ്ഞവരന്മാരേ!\\
ഉര്‍വീശനായാലവനാശ്രിതന്മാരെ\\
സര്‍വശോരക്ഷേച്ഛുനശ്വപചാനപി.\\
രക്ഷിയാഞ്ഞാലവന്‍ ബ്രഹ്മഹാ കേവലം\\
രക്ഷിതാവശ്വമേധം ചെയ്ത പുണ്യവാന്‍\\
എന്നു ചൊല്ലുന്നതു വേദശാസ്ത്രങ്ങളില്‍\\
പുണ്യപാപങ്ങളറിയരുതേതുമേ.\\
മുന്നമൊരു കപോതം നിജപേടയോ-\\
ടൊന്നിച്ചൊരു വനം തന്നില്‍ മേവീടിനാന്‍\\
ഉന്നതമായൊരു പാദപാഗ്രേ തദാ\\
ചെന്നൊരു കാട്ടാളനെയ്തു കൊന്നീടാന്‍\\
തന്നുടെ പക്ഷിണിയെസ്സുരതാന്തരേ,\\
വന്നൊരു ദുഃഖം പൊറാഞ്ഞു കരഞ്ഞവന്‍\\
തന്നെ മറന്നിരുന്നീടും ദശാന്തരേ\\
വന്നിതു കാറ്റും മഴയും ദിനേശനും\\
ചെന്നു ചരമാബ്ധിതന്നില്‍ മറഞ്ഞിതു;\\
ഖിന്നനായ് വന്നു വിശന്നു കിരാതനും\\
താനിരിക്കുന്ന വൃക്ഷത്തിന്‍ മുരടതില്‍\\
ദീനതയോടു നില്ക്കുന്ന കാട്ടാളനെ-\\
ക്കണ്ടു കരുണ കലര്‍ന്നു കപോതവും\\
കൊണ്ടുവന്നാശു കൊടുത്തിതു വഹ്നിയും\\
തന്നുടെ കൈയിലിരുന്ന കപോതിയെ\\
വഹ്നിയിലിട്ടു ചുട്ടാശു തിന്നീടിനാന്‍\\
എന്നതുകൊണ്ടു വിശപ്പടങ്ങീടാഞ്ഞു\\
പിന്നെയും പീഡിച്ചിരിക്കും കിരാതനു\\
തന്നുടെ ദേഹവും നല്കിനാനമ്പോടു\\
വഹ്നിയില്‍ വീണു കിരാതാശനാര്‍ത്ഥമായ്.\\
അത്രപോലും വേണമാശ്രിതരക്ഷണം\\
മര്‍ത്ത്യനെന്നാലോ പറയേണ്ടതില്ലല്ലോ.\\
എന്നെശ്ശരണമെന്നോര്‍ത്തിങ്ങുവന്നവ-\\
നെന്നുമഭയം കൊടുക്കുമതേയുള്ളൂ.\\
പിന്നെ വിശേഷിച്ചുമൊന്നു കേട്ടീടുവി-\\
നെന്നെച്ചതിപ്പതിനാരുമില്ലെങ്ങുമേ.\\
ലോകപാലന്മാരെയും മറ്റു കാണായ\\
ലോകങ്ങളെയും നിമിഷമാത്രംകൊണ്ടു\\
സൃഷ്ടിച്ചു രക്ഷിച്ചു സംഹരിച്ചീടുവാ-\\
നൊട്ടുമേ ദണ്ഡമെനിക്കില്ല നിശ്ചയം,\\
പിന്നെ ഞാനാരെബ്ഭയപ്പെടുന്നു, മുദാ\\
വന്നീടുവാന്‍ ചൊല്ലവനെ മടിയാതെ,\\
വ്യഗ്രിയായ്കേതുമിതുചൊല്ലി മാനസേ\\
സുഗ്രീവ! നീ ചെന്നവനെ വരുത്തുക.\\
എന്നെശ്ശരണം ഗമിക്കുന്നവര്‍ക്കു ഞാ-\\
നെന്നുമഭയം കൊടുക്കുമതിദ്രുതം.\\
പിന്നെയവര്‍ക്കൊരു സംസാരദുഃഖവും\\
വന്നുകൂടാ നൂനമെന്നു മറിക നീ.’\\
ശ്രീരാമവാക്യാമൃതം കേട്ടു വാനര-\\
വീരന്‍ വിഭീഷണന്‍ തന്നെ വരുത്തിനാന്‍\\
ശ്രീരാമപാദാന്തികേ വീണു സാഷ്ടാംഗ-\\
മാരൂഢമോദം നമസ്കരിച്ചീടിനാന്‍.\\
രാമം വിശാലാക്ഷമിന്ദീവരദള-\\
ശ്യാമളം കോമളം ബാണധനുര്‍ദ്ധരം\\
സോമബിംബാഭ പ്രസന്നമുഖാംബുജം\\
കാമദം കാമോപമം കമലാവരം\\
കാന്തം കരുണാകരം കമലേക്ഷണം\\
ശാന്തം ശരണ്യം വരേണ്യം വരപ്രദം\\
ലക്ഷ്മണസംയുതം സുഗ്രീവമാരുതി-\\
മുഖ്യകപികുലസേവിതം രാഘവം\\
കണ്ടു കൂപ്പിത്തൊഴുതേറ്റം വിനീതനാ-\\
യുണ്ടായ സന്തോഷമോടും വിഭീഷണന്‍\\
ഭക്തപ്രിയനായ ലോകൈകനാഥനെ\\
ഭക്തി പരവശനായ് സ്തുതിച്ചീടിനാന്‍:\\
‘ശ്രീരാമ! സീതാമനോഹര! രാഘവ!\\
ശ്രീരാമ! രാജേന്ദ്ര! രാജീവലോചന!\\
ശ്രീരാമ! രാക്ഷസവംശവിനാശന!\\
ശ്രീരാമപാദാംബുജം നമസ്തേ സദാ.\\
ചണ്ഡാംശുഗോത്രോത്ഭവായ നമോ നമ-\\
ശ്ചണ്ഡകോദണ്ഡധരായ നമോ നമഃ\\
പണ്ഡിതഹൃല്‍പുണ്ഡരീകചണ്ഡാംശവേ\\
ഖണ്ഡപരശുപ്രിയായ നമോ നമഃ\\
രാമായ സുഗ്രീവമിത്രായ കാന്തായ\\
രാമായ നിത്യമനന്തായ ശാന്തായ\\
രാമായ വേദാന്തവേദ്യായ ലോകാഭി-\\
രാമായ രാമഭദ്രായ നമോ നമഃ\\
വിശ്വോത്ഭവസ്ഥിതിസംഹാരഹേതവേ\\
വിശ്വായ വിശ്വരൂപായ നമോ നമഃ\\
നിത്യമനാദിഗൃഹസ്ഥായ തേ നമോ\\
നിത്യായ സത്യായ ശുദ്ധായ തേ നമഃ\\
ഭക്തപ്രിയായ ഭഗവതേ രാമായ\\
മുക്തി പ്രദായ മുകുന്ദായ തേ നമഃ\\
വിശ്വേശനാം നിന്തിരുവടിതാനല്ലോ\\
വിശ്വോത്ഭവസ്ഥിതി സംഹാര കാരണം\\
സന്തതം ജംഗമാജംഗമഭൂതങ്ങ-\\
ളന്തര്‍ബഹിര്‍വ്യാപ്തനാകുന്നതും ഭവാന്‍.\\
നിന്മഹാമായയാ മൂടിക്കിടക്കുമാ\\
നിര്‍മലമാം പരബ്രഹ്മമജ്ഞാനിനാം\\
തന്മൂലമായുള്ള പുണ്യപാപങ്ങളാല്‍\\
ജന്മമരണങ്ങളുണ്ടായ് വരുന്നിതും\\
അത്രനാളേക്കും ജഗത്തൊക്കവേ ബലാല്‍\\
സത്യമായ് തോന്നുമതിനില്ല സംശയം.\\
എത്ര നാളേക്കറിയാതെയിരിക്കുന്നി-\\
തദ്വയമാം പരബ്രഹ്മം സനാതനം\\
പുത്രദാരാദി വിഷയങ്ങളിലതി-\\
സക്തി കലര്‍ന്നു രമിക്കുന്നിതന്വഹം\\
ആത്മാവിനെയറിയായ്കയാല്‍ നിര്‍ണയ-\\
മാത്മനി കാണേണമാത്മാനമാത്മനാ.\\
ദുഃഖപ്രദം വിഷയേന്ദ്രിയസംയോഗ-\\
മോക്കെയുമോര്‍ത്താലൊടുക്കമനാത്മനാ\\
ആദികാലേ സുഖമെന്നു തോന്നിക്കുമ-\\
തേതും വിവേകമില്ലാതവര്‍ മാനസേ\\
ഇന്ദ്രാഗ്നിധര്‍മരക്ഷോവരുണാനില-\\
ചന്ദ്രരുദ്രാജാഹിപാദികളൊക്കെയും\\
ചിന്തിക്കിലോ നിന്തിരുവടി നിര്‍ണയ-\\
മന്തവുമാദിയുമില്ലാതെ ദൈവമേ!\\
കാലസ്വരൂപനായീടുന്നതും ഭവാന്‍\\
സ്ഥൂലങ്ങളില്‍ വെച്ചതിസ്ഥൂലനും ഭവാന്‍\\
നൂനമണുവിങ്കല്‍ നിന്നണീയാന്‍ ഭവാന്‍\\
മാനമില്ലാത മഹത്തത്ത്വവും ഭവാന്‍.\\
സര്‍വലോകാനാം പിതാവായതും ഭവാന്‍\\
സര്‍വലോകേശ! മാതാവായതും ഭവാന്‍\\
സര്‍വദാ സര്‍വധാതാവായതും ഭവാന്‍.\\
ദര്‍വീകരേന്ദ്രശയന! ദയാനിധേ!\\
ആദിമദ്ധ്യാന്തവിഹീനന്‍ പരിപൂര്‍ണ-\\
നാധാരഭൂതന്‍ പ്രപഞ്ചത്തിനീശ്വരന്‍\\
അച്യുതനവ്യയനവ്യക്തനദ്വയന്‍\\
സച്ചില്‍പ്പുരുഷന്‍ പുരുഷോത്തമന്‍ പരന്‍\\
നിശ്ചലന്‍ നിര്‍മമന്‍ നിഷ്കളന്‍ നിര്‍ഗുണന്‍\\
നിശ്ചയിച്ചാര്‍ക്കുമറിഞ്ഞുകൂടാതവന്‍.\\
നിര്‍വികാരന്‍ നിരാകാരന്‍ നിരീശ്വരന്‍\\
നിര്‍വികല്പന്‍ നിരൂപാശ്രയന്‍ ശാശ്വതന്‍\\
ഷഡ്ഭാവഹീനന്‍ പ്രകൃതിപരന്‍ പുമാന്‍\\
സദ്ഭാവയുക്തന്‍ സനാതനന്‍ സര്‍വഗന്‍\\
മായാമനുഷ്യന്‍ മനോഹരന്‍ മാധവന്‍\\
മായാവിഹീനന്‍ മധുകൈടഭാന്തകന്‍\\
ഞാനിഹ ത്വല്‍പാദഭക്തി നിശ്രേണിയെ-\\
സ്സാനന്ദമാശു സമ്പ്രാപ്യ രഘുപതേ!\\
ജ്ഞാനയോഗാഖ്യസൗധം കരേറീടുവാന്‍\\
മാനസേ കാമിച്ചു വന്നേന്‍ ജഗല്‍പ്പതേ!\\
സീതാപതേ! രാമ! കാരുണികോത്തമ!\\
യാതുധാനാന്തക! രാവണാരേ! ഹരേ!\\
പാദാംബുജം നമസ്തേ ഭവസാഗര-\\
ഭീതനാമെന്നെ രക്ഷിച്ചുകൊള്ളേണമേ!’\\
ഭക്തിപരവശനായ് സ്തുതിച്ചീടിന\\
ഭക്തനെക്കണ്ടു തെളിഞ്ഞു രഘൂത്തമന്‍\\
ഭക്തപ്രിയന്‍ പരമാനന്ദമുള്‍ക്കൊണ്ടു\\
മുഗ്ദ്ധസ്മിതപൂര്‍വമേവമരുള്‍ചെയ്തു:\\
‘ഇഷ്ടമായുള്ള വരത്തെ വരിക്ക സ-\\
ന്തുഷ്ടനാം ഞാന്‍ വരദാനൈകതല്‍പരന്‍.\\
ഒട്ടുമേ താപമൊരുത്തനെന്നെക്കണ്ടു\\
കിട്ടിയാല്‍ പിന്നെയുണ്ടാകയില്ലോര്‍ക്ക നീ.’\\
രാമവാക്യാമൃതം കേട്ടു വിഭീഷണ-\\
നാമോദമുള്‍ക്കൊണ്ടുണര്‍ത്തിച്ചരുളിനാന്‍:\\
‘ധന്യനായേന്‍ കൃതകൃത്യനായേനഹം\\
ധന്യാകൃതേ കൃതകാമനായേനഹം.\\
ത്വല്‍പാദപത്മാവലോകനംകൊണ്ടു ഞാ-\\
നിപ്പോള്‍ വിമുക്തനായേനില്ല സംശയം.\\
മത്സമനായൊരു ധന്യനില്ലൂഴിയില്‍\\
മത്സമാനായൊരു ശുദ്ധനുമില്ലഹോ!\\
മത്സമനായ് മറ്റൊരുവനുമില്ലിഹ\\
ത്വത്സ്വരൂപം മമ കാണായ കാരണല്‍.\\
കര്‍മബന്ധങ്ങള്‍ നശിപ്പതിനായിനി\\
നിര്‍മലമാം ഭവദ്ജ്ഞാനവും ഭക്തിയും\\
ത്വദ്ധ്യാനസൂക്ഷ്മവും ദേഹി മേ രാഘവ!\\
ചിത്തേ വിഷയസുഖാശയില്ലേതുമേ.\\
ത്വല്‍പ്പാദപങ്കജഭക്തിരേവാസ്തു മേ\\
നിത്യമിളക്കമൊഴിഞ്ഞു കൃപാനിധേ!’\\
ഇത്ഥമാകര്‍ണ്യ സമ്പ്രീതനാം രാഘവന്‍\\
നക്തഞ്ചരാധിപന്‍തന്നോടരുള്‍ചെയ്തു:\\
‘നിത്യം വിഷയവിരക്തരായ് ശാന്തരായ്\\
ഭക്തി വളര്‍ന്നതിശുദ്ധമതികളായ്\\
ജ്ഞാനികളായുള്ള യോഗികള്‍ മാനസേ\\
ഞാനിരിപ്പൂ മമ സീതയുമായ് മുദാ.\\
ആകയാലെന്നെയും ധ്യാനിച്ചു സന്തതം\\
വാഴ്ക നീയെന്നാല്‍ നിനക്കു മോക്ഷം വരും.\\
അത്രയുമല്ല നിന്നാല്‍ കൃതമായൊരു\\
ഭക്തികരസ്തോത്രമത്യന്തശുദ്ധനായ്\\
നിത്യവും ചൊല്കയും കേള്‍ക്കയും ചെയ്കിലും\\
മുക്തിവരുമതിനില്ലൊരു സംശയം.’\\
ഇത്ഥമരുള്‍ചെയ്തു ലക്ഷ്മണന്‍ തന്നോടു\\
ഭക്തപ്രിയനരുള്‍ചെയ്തിതു സാദരം:\\
‘എന്നെക്കനിവോടു കണ്ടതിന്റെ ഫല-\\
മിന്നുതന്നേ വരുത്തേണമതിന്നു നീ\\
ലങ്കാധിപനിവനെന്നഭിഷേകവും\\
ശങ്കാവിഹീനമന്‍പോടു ചെയ്തീടുക.\\
സാഗരവാരിയും കൊണ്ടുവന്നീടുക\\
ശാഖാമൃഗാധിപന്മാരുമായ് സത്വരം\\
അര്‍ക്കചന്ദ്രന്മാരുമാകാശഭൂമിയും\\
മല്‍ക്കഥയും ജഗത്തിങ്കലുള്ളന്നിവന്‍\\
വാഴ്ക ലങ്കാരാജ്യമേവം മമാജ്ഞയാ\\
ഭാഗവതോത്തമനായ വിഭീഷണന്‍.’\\
പങ്കജനേത്രവാക്യം കേട്ടു ലക്ഷ്മണന്‍\\
ലങ്കാപുരാധിപത്യാര്‍ത്ഥമഭിഷേക-\\
മന്‍പോടു വാദ്യഘോഷേണ ചെയ്തീടിനാന്‍.\\
വമ്പരാം വാനരാധീശ്വരന്മാരുമായ്\\
സാധുവാദേന മുഴങ്ങി ജഗത്ത്രയം\\
സാധുജനങ്ങളും പ്രീതി പൂണ്ടീടിനാര്‍.\\
ആദിതേയോത്തമന്മാര്‍ പുഷ്പവൃഷ്ടിയു-\\
മാധി വേറിട്ടു ചെയ്തീടിനാരാദരാല്‍.\\
അപ്സരസ്ത്രീകളും നൃത്തഗീതങ്ങളാ-\\
ലപ്പുരുഷോത്തമനെബ്ഭജിച്ചീടിനാര്‍.\\
ഗന്ധര്‍വകിന്നര കിംപുരുഷന്മാരു-\\
മന്തര്‍മുദാ സിദ്ധവിദ്യാധരാദിയും\\
ശ്രീരാമചന്ദ്രനെ വാഴിത്തി സ്തുതിച്ചിതു\\
ഭേരീനിനാദം മുഴക്കിനാരുമ്പരും.\\
പുണ്യജനേശ്വരനായ വിഭീഷണന്‍-\\
തന്നെപ്പുണര്‍ന്നു സുഗ്രീവനും ചൊല്ലിനാന്‍:\\
‘പാരേഴുരണ്ടിനും നാഥനായ് വാഴുമീ\\
ശ്രീരാമകിങ്കരന്മാരില്‍ മുഖ്യന്‍ ഭവാന്‍.\\
രാവണനിഗ്രഹത്തിന്നു സഹായവു-\\
മാവോളമാശു ചെയ്യേണം ഭവാനിനി\\
കേവലം ഞങ്ങളും മുന്‍ നടക്കുന്നുണ്ടു\\
സേവയാ സിദ്ധിക്കുമേറ്റമനുഗ്രഹം.’\\
സുഗ്രീവവാക്യമാകര്‍ണ്യ വിഭീഷണ-\\
നഗ്രേ ചിരിച്ചവനോടു ചൊല്ലീടിനാന്‍:\\
‘സാക്ഷാല്‍ ജഗന്മയനാമഖിലേശ്വരന്‍\\
സാക്ഷിഭൂതന്‍ സകലത്തിന്നുമാകയാല്‍\\
എന്തു സഹായേന കാര്യമവിടേക്കു\\
ബന്ധു ശത്രുക്കളെന്നുള്ളതുമല്ല കേള്‍.\\
ഗൂഢസ്ഥനാനന്ദപൂര്‍ണനേകാത്മകന്‍\\
കൂടസ്ഥനാശ്രയം മറ്റാരുമില്ലെടോ!\\
മൂഢത്വമത്രേ നമുക്കു തോന്നുന്നതു\\
ഗൂഢത്രിഗുണഭവേന മായാബലാല്‍\\
തദ്വശന്മാരൊക്കെ നാമെന്നറിഞ്ഞുകൊ-\\
ണ്ടദ്വയഭാവേന സേവിച്ചുകൊള്‍ക നാം.’\\
നക്തഞ്ചരപ്രവരോക്തികള്‍ കേട്ടൊരു\\
ഭക്തനാം ഭാനുജനും തെളിഞ്ഞീടിനാന്‍.
\end{verse}

%%09_shukabandhanam

\section{ശുകബന്ധനം}

\begin{verse}
രക്ഷോവരനായ രാവണന്‍ ചൊല്കയാല്‍\\
തല്‍ക്ഷണേ വന്നു ശുകനും നിശാചരന്‍\\
പുഷ്കരേ നിന്നു വിളിച്ചു ചൊല്ലീടിനാന്‍\\
മര്‍ക്കടരാജനാം സുഗ്രീവനോടിദം:\\
‘രാക്ഷസാധീശ്വരന്‍ വാക്കുകള്‍ കേള്‍ക്ക നീ\\
ഭാസ്കരസൂനോ! പരാക്രമവാരിധേ!\\
ഭാനുതനയനാം ഭാഗധേയാംബുധേ!\\
വാനരരാജമഹാകുല സംഭവ!\\
ആദിതേയേന്ദ്രസുതാനുജനാകയാല്‍\\
ഭ്രാതൃസമാനന്‍ ഭവാന്‍ മമ നിര്‍ണയം.\\
നിന്നോടു വൈരമെനിക്കേതുമില്ല മ-\\
റ്റെന്നില്‍ വിരോധം നിനക്കുമില്ലേതുമേ.\\
രാജകുമാരനാം രാമഭാര്യാമഹം\\
വ്യാജേന കൊണ്ടുപോന്നേനതിനെന്തു തേ?\\
മര്‍ക്കടസേനയോടുമതി വിദ്രുതം\\
കിഷ്കിന്ധയാം നഗരിക്കു പോയ്ക്കൊള്‍ക നീ.\\
ദേവാദികളാലുമപ്രാപ്യമായോന്നു\\
കേവലമെന്നുടെ ലങ്കാപുരമെടോ!\\
അല്പസാരന്മാര്‍ മനുഷ്യരുമെത്രയും\\
ദുര്‍ബലന്മാരായ വാനരയൂഥവും\\
എന്തോന്നു കാട്ടുന്നിതെന്നോടിവിടെ വ-\\
ന്നന്ധകാരം നിനച്ചീടായ്ക നീ വൃഥാ.’\\
ഇത്ഥം ശുകോക്തികള്‍ കേട്ടു കപികുല-\\
മുത്ഥായ ചാടിപ്പിടിച്ചാരതിദ്രുതം.\\
മുഷ്ടിപ്രഹരങ്ങളേറ്റു ശുകനതി-\\
ക്ലിഷ്ടനായേറ്റം കരഞ്ഞു തുടങ്ങിനാന്‍:\\
"രാമ രാമ! പ്രഭോ! കാരുണ്യ വാരിധേ!\\
രാമ! നാഥ! പരിത്രാഹി രഘുപതേ!\\
ദൂതരെക്കൊല്ലുമാറില്ല പണ്ടാരുമേ\\
നാഥ! ധര്‍മത്തെ രക്ഷിച്ചുകൊള്ളേണമേ.\\
വാനരന്മാരെ നിവാരണം ചെയ്താശു\\
മാനവ വീര! ഹതോഹം പ്രപാഹി മാം.’\\
ഇത്ഥം ശുകപരിദേവനം കേട്ടൊരു\\
ഭക്തപ്രിയന്‍ വരദന്‍ പുരുഷോത്തമന്‍\\
വാനരന്മാരെ വിലക്കിനാനന്നേര-\\
മാനന്ദമുള്‍ക്കൊണ്ടുയര്‍ന്നു ശുകന്‍ തദാ\\
ചൊല്ലിനാന്‍ സുഗ്രീവനോടു: ‘ഞാനെന്തോന്നു\\
ചൊല്ലേണ്ടതങ്ങു ദശഗ്രീവനോടതു\\
ചൊല്ലീടുകെ’ന്നതു കേട്ടു സുഗ്രീവനും\\
ചൊല്ലിനാനാശു ശുകനോടു സത്വരം:\\
‘ചൊല്ലുള്ള ബാലിയെപ്പോലെ ഭവാനെയും\\
കൊല്ലണമാശു സപുത്രബലാന്വിതം.\\
ശ്രീരാമപത്നിയെ കട്ടുകൊണ്ടീടിന\\
ചോരനെയും കൊന്നു ജാനകിതന്നെയും\\
കൊണ്ടുപോകേണമെനിക്കു കിഷ്ക്കിന്ധയ്ക്ക്\\
രണ്ടില്ലതിനെന്നു ചെന്നു ചൊല്ലീടു നീ".\\
അര്‍ക്കാത്മജോക്തികള്‍ കേട്ടു തെളിഞ്ഞള-\\
വര്‍ക്കാന്വയോത്ഭവന്‍ താനുമരുള്‍ചെയ്തു:\\
‘വാനരന്മാരേ! ശുകനെ ബന്ധിച്ചകൊ-\\
ണ്ടൂനമൊഴിഞ്ഞിങ്ങു കാത്തുകൊണ്ടീടുവിന്‍\\
ഞാനുരചെയ്തേയയയ്ക്കാവിതെ’ന്നതു-\\
മാനന്ദമോടരുള്‍ചെയ്തു രഘുവരന്‍\\
വാനരന്മാരും പിടിച്ചുകെട്ടിക്കൊണ്ടു\\
ദീനത കൈവിട്ടുകാത്തുകൊണ്ടീടിനാര്‍\\
ശാര്‍ദൂലവിക്രമം പൂണ്ട കപിബലം\\
ശാര്‍ദൂലനായ നിശാചരന്‍ വന്നുക-\\
ണ്ടാര്‍ത്തനായ് രാവണനോടുചൊല്ലീടിനാന്‍.\\
വാര്‍ത്തകളുള്ളവണ്ണമതുകേട്ടോരു\\
രാത്രിഞ്ചരേശ്വരനാകിയ രാവണന്‍\\
ആര്‍ത്തിപൂണ്ടേറ്റവും ദീര്‍ഗ്ഘചിന്താന്വിതം\\
ചീര്‍ത്ത ഖേദത്തോടു ദീര്‍ഗ്ഘമായേറ്റവും\\
വീര്‍ത്തുപായങ്ങള്‍ കാണാഞ്ഞിരുന്നീടിനാന്‍.
\end{verse}

%%10_sethubandhanam

\section{സേതുബന്ധനം}

\begin{verse}
തല്‍ക്കാലമര്‍ക്കകുലോത്ഭവന്‍ രാഘവ-\\
നര്‍ക്കാത്മജാദി കപിവരന്മാരൊടും\\
രക്ഷോവരനാം വിഭീഷണന്‍ തന്നൊടും\\
ലക്ഷ്മണനോടും വിചാരം തുടങ്ങിനാന്‍:\\
‘എന്തുപായം സമുദ്രം കടപ്പാനെന്നു\\
ചിന്തിച്ചു കല്പിക്ക നിങ്ങളെല്ലാരുമായ്’\\
എന്നരുള്‍ചെയ്തതു കേട്ടവരേവരു-\\
മൊന്നിച്ചുകൂടി നിരൂപിച്ചു ചൊല്ലിനാര്‍:\\
‘ദേവപ്രവരനായോരു വരുണനെ-\\
സ്സേവിക്കവേണമെന്നാല്‍ വഴിയും തരും’\\
എന്നതു കേട്ടരുള്‍ചെയ്തു രഘുവരന്‍:\\
‘നന്നതു തോന്നിയതങ്ങനെ തന്നെ’യെ-\\
ന്നര്‍ണവതീരേ കിഴക്കുനോക്കിത്തൊഴു-\\
തര്‍ണോജലോചനനാകിയ രാഘവന്‍\\
ദര്‍ഭ വിരിച്ചു നമസ്കരിച്ചീടിനാ-\\
നത്ഭുതവിക്രമന്‍ ഭക്തിപൂണ്ടെത്രയും\\
മൂന്നഹോരാത്രമുപാസിച്ചതങ്ങനെ\\
മൂന്നുലോകത്തിനും നാഥനാമീശ്വരന്‍.\\
ഏതുമിളകീല വാരിധിയുമതി-\\
ക്രോധേന രക്താന്തനേത്രനാം നാഥനും\\
‘കൊണ്ടുവാ ചാപബാണങ്ങള്‍ നീ ലക്ഷ്മണ!\\
കണ്ടുകൊണ്ടാലും മമ ശരവിക്രമം.\\
ഇന്നു പെരുവഴി മീളുന്നതല്ലെങ്കി-\\
ലര്‍ണവം ഭസ്മമമാക്കിച്ചമച്ചീടുവന്‍.\\
മുന്നം മദീയപൂര്‍വന്മാര്‍ വളര്‍ത്തതു-\\
മിന്നു ഞാനില്ലാതെയാക്കുവന്‍ നിര്‍ണയം.\\
സാഗരമെന്നുള്ള പേരും മറന്നുള്ളി-\\
ലാകുലമെന്നിയേ വാഴുകിലെന്നുമേ\\
നഷ്ടമാക്കീടുവന്‍ വെള്ളം, കപികുലം\\
പുഷ്ടമോദം പാദചാരേണ പോകണം.’\\
എന്നരുള്‍ചെയ്തു വില്ലും കുഴിയെക്കുല-\\
ച്ചര്‍ണവത്തോടരുള്‍ചെയ്തു രഘുവരന്‍:\\
‘സര്‍വഭൂതങ്ങളും കണ്ടുകൊള്ളേണമെന്‍\\
ദുര്‍വാരമായ ശിലീമുഖവിക്രമം\\
ഭസ്മമാകീടുവന്‍ വാരാന്നിധിയെ ഞാന്‍\\
വിസ്മയമെല്ലാവരും കണ്ടു നില്ക്കണം’\\
ഇത്ഥം രഘുവരന്‍ വാക്കു കേട്ടന്നേരം\\
പൃത്ഥ്വീരുഹങ്ങളും കാനനജാലവും\\
പൃത്ഥ്വിയും കൂടെ വിറച്ചു ചമഞ്ഞിതു,\\
മിത്രനും മങ്ങി; നിറഞ്ഞു തിമിരവു-\\
മബ്ധിയും ക്ഷോഭിച്ചു, മിട്ടാല്‍ കവിഞ്ഞുവ-\\
ന്നുത്തുംഗമായ തരംഗാവലിയൊടും\\
ത്രസ്തങ്ങളായ് പരിതപ്തങ്ങളായ് വന്നി-\\
തത്യുഗ്രനക്രതിമിഝഷാദ്യങ്ങളും.\\
അപ്പോള്‍ ഭയപ്പെട്ടു ദിവ്യരൂപത്തോടു-\\
മപ്പതി ദിവ്യാഭരണസമ്പന്നനായ്\\
പത്തുദിക്കും നിറഞ്ഞോരു കാന്ത്യാ നിജ-\\
ഹസ്തങ്ങളില്‍ പരിഗൃഹ്യ രത്നങ്ങളും\\
വിത്രസ്തനായ് രാമപാദാന്തികേ വെച്ചു\\
സത്രപം ദണ്ഡനമസ്കാരവും ചെയ്തു\\
രക്താന്തലോചനനാകിയ രാമനെ\\
ഭക്ത്യാ വണങ്ങി സ്തുതിച്ചാന്‍ പലതരം.\\
ത്രാഹി മാം ത്രാഹി മാം ത്രൈലോക്യപാലക!\\
ത്രാഹി മാം ത്രാഹി മാം വിഷ്ണോ! ജഗല്‍പ്പതേ!\\
ത്രാഹി മാം ത്രാഹി മാം പൗലസ്ത്യനാശന!\\
ത്രാഹി മാം ത്രാഹി മാം രാമ! രമാപതേ!\\
ആദികാലേ തവ മായാഗുണവശാല്‍\\
ഭൂതങ്ങളെബ്ഭവാന്‍ സൃഷ്ടിച്ചതുനേരം\\
സ്ഥൂലങ്ങളായുള്ള പഞ്ചഭൂതങ്ങളെ-\\
ക്കാലസ്വരൂപനാകും നിന്തിരുവടി\\
സൃഷ്ടിച്ചിതേറ്റം ജഡസ്വഭാവങ്ങളാ-\\
യ്ക്കഷ്ടമതാര്‍ക്കു നീക്കാവു തവ മതം?\\
പിന്നെ വിശേഷിച്ചതിലും ജഡത്വമായ്-\\
ത്തന്നെ ഭവാന്‍ പുനരെന്നെ നിര്‍മിച്ചതും\\
മുന്നേ ഭവന്നിയോഗസ്വഭാവത്തെയി-\\
ന്നന്യഥാ കര്‍ത്തുമാരുള്ളതു ശക്തരായ്?\\
താമസോത്ഭൂതങ്ങളായുള്ള ഭൂതങ്ങള്‍\\
താമസശീലമായ്തന്നേ വരൂ വിഭോ!\\
താമസമല്ലോ ജഡത്വമാകുന്നതും\\
കാമലോഭാദികളും താമസഗുണം.\\
മായാരഹിതനായ് നിര്‍ഗുണനായ നീ\\
മായാഗുണങ്ങളെയംഗീകരിച്ചപ്പോള്‍\\
വൈരാജനാമവാനായ് ചമഞ്ഞൂ ഭാവാന്‍\\
കാരണപുരുഷനായ് ഗുണാത്മാവുമായ്.\\
അപ്പോള്‍ വിരാട്ടിങ്കല്‍നിന്നു ഗുണങ്ങളാ-\\
ലുല്‍പ്പന്നരായിതു ദേവാദികള്‍ തദാ.\\
തത്ര സത്വത്തിങ്കല്‍നിന്നല്ലോ ദേവകള്‍\\
തദ്രജോഭൂതങ്ങളായ് പ്രജേശാദികള്‍\\
തത്തമോഭൂതനായ് ഭൂതപതിതാനു-\\
മുത്തമപുരുഷ! രാമ! ദയാനിധേ!\\
മായയാ ഛന്നനായ് ലീലാമനുഷ്യനായ്\\
മായാഗുണങ്ങളെക്കൈക്കൊണ്ടനാരതം\\
നിര്‍ഗുണനായ് സദാ ചിദ്ഘനനായൊരു\\
നിഷ്കളനായ് നിരാകാരനായിങ്ങനെ\\
മോക്ഷദനാം നിന്തിരുവടിതന്നെയും\\
മൂര്‍ഖനാം ഞാനെങ്ങനെയറിഞ്ഞീടുന്നു?\\
മൂര്‍ഖജനങ്ങള്‍ക്കു സന്മാര്‍ഗപ്രാപക-\\
മോര്‍ക്കില്‍ പ്രഭൂണാം ഹിതം ദണ്ഡമായതും\\
ദുഷ്ടപശൂനാം യഥാ ലകുടം തഥാ\\
ദുഷ്ടാനുശാസനം ധര്‍മം ഭവാദൃശാം.\\
ശ്രീരാമദേവം പരം ഭക്തവത്സലം\\
കാരണപുരുഷം കാരുണ്യസാഗരം\\
നാരായണം ശരണ്യം പുരുഷോത്തമം\\
ശ്രീരാമമീശം ശരണം ഗതോസ്മി ഞാന്‍\\
രാമചന്ദ്രാഭയം ദേഹി മേ സന്തതം\\
രാമ! ലങ്കാമാര്‍ഗമാശു ദദാമി തേ.’\\
ഇത്ഥം വണങ്ങി സ്തുതിച്ച വരുണനോ-\\
ടുത്തമപുരുഷന്‍ താനുമരുള്‍ചെയ്തു:\\
‘ബാണം മദീയമമോഘമതിന്നിഹ\\
വേണമൊരു ലക്ഷ്യമെന്തതിന്നുള്ളതും?\\
വാട്ടമില്ലാതൊരു ലക്ഷ്യമതിന്നു നീ\\
കാട്ടിത്തരേണമെനിക്കു വാരാന്നിധേ!’\\
അര്‍ണവനാഥനും ചൊല്ലിനാനന്നേര-\\
‘മന്യൂനകാരുണ്യസിന്ധോ! ജഗല്‍പ്പതേ!\\
ഉത്തരസ്യാം ദിശി മത്തീരഭൂതലേ\\
ചിത്രദ്രുമകുല്യദേശം സുഭിക്ഷദം\\
തത്ര പാപാത്മാക്കളുണ്ടു നിശാചര-\\
രെത്രയും പാരമുപദ്രവിച്ചീടുവോര്‍.\\
വേഗാലവിടേക്കയയ്ക്ക ബാണം തവ\\
ലോകോപകാരകമാമതു നിര്‍ണയം.’\\
രാമനും ബാണമയച്ചാനതുനേര-\\
മാമയം തേടീടുമാഭീരമണ്ഡലം\\
എല്ലാമൊടുക്കി വേഗേന ബാണം പോന്നു\\
മെല്ലവേ തൂണീരവും പുക്കിതാദരാല്‍\\
ആഭീരമണ്ഡലമൊക്കെ നശിക്കയാല്‍\\
ശോഭനമായ് വന്നു തല്‍പ്രദേശം തദാ\\
തല്‍ക്കുലദേശവുമന്നുതൊട്ടെത്രയും\\
മുഖ്യജനപദമായ് വന്നിതെപ്പൊഴും.\\
സാഗരം ചൊല്ലിനാന്‍ സാദരമന്നേര-\\
‘മാകുലമെന്നിയേ മജ്ജലേ സത്വരം\\
സേതു ബന്ധിക്ക നളനാം കപിവര-\\
നേതുമവനൊരു ദണ്ഡമുണ്ടായ് വരാ.\\
വിശ്വകര്‍മാവിന്‍മകനവനാകയാല്‍\\
വിശ്വശില്പക്രിയാതല്‍പ്പരനെത്രയും\\
വിശ്വദുരിതാപഹാരിണിയായ് തവ\\
വിശ്വമെല്ലാം നിറഞ്ഞീടുന്ന കീര്‍ത്തിയും\\
വര്‍ദ്ധിക്കു’മെന്നു പറഞ്ഞു തൊഴുതുട-\\
നബ്ധിയും മെല്ലെ മറഞ്ഞരുളീടിനാന്‍.\\
സന്തുഷ്ടനായൊരു രാമചന്ദ്രന്‍ തദാ\\
ചിന്തിച്ചു സുഗ്രീവലക്ഷ്മണന്മാരൊടും\\
പ്രാജ്ഞനായീടും നളനെ വിളിച്ചുട-\\
നാജ്ഞയെച്ചെയ്തിതു സേതുസംബന്ധേന.\\
തല്‍ക്ഷണേ മര്‍ക്കടമുഖ്യനാകും നളന്‍\\
പുഷ്കരനേത്രനെ വന്ദിച്ചു സത്വരം\\
പര്‍വതതുല്യശരീരികളാകിയ\\
ദുര്‍വാരവീര്യമിയന്ന കപികളും\\
സര്‍വദിക്കിങ്കലും നിന്നു സരഭസം\\
പര്‍വതപാഷാണപാദപജാലങ്ങള്‍\\
കൊണ്ടുവരുന്നവ വാങ്ങിത്തെരുതെരെ\\
കുണ്ഠവിഹീനം പടുത്തു തുടങ്ങിനാന്‍.\\
നേരേ ശാതയോജനായതമായുട-\\
നീരഞ്ചുയോജന വിസ്താരമാംവണ്ണം\\
ഇത്ഥം പടുത്തു തുടങ്ങുംവിധൗ രാമ-\\
ഭദ്രനാം ദാശരഥി ജഗദീശ്വരന്‍\\
വ്യോമകേശം പരമേശ്വരം ശങ്കരം\\
രാമേശ്വരമെന്ന നാമമരുള്‍ചെയ്തു\\
ശോഭനമായ മുഹൂര്‍ത്തേന സംസ്ഥാപ്യ\\
പാപഹരായ ത്രിലോകഹിതാര്‍ഥമായ്\\
പൂജിച്ചു വന്ദിച്ചു ഭക്ത്യാ നമസ്കൃത്യ\\
രാജീവലോചനനേവമരുള്‍ ചെയ്തു:\\
‘യാതൊരു മര്‍ത്ത്യനിവിടെ വന്നാദരാല്‍\\
സേതുബന്ധം കണ്ടു രാമേശ്വരനെയും\\
ഭക്ത്യാ ഭജിക്കുന്നിതപ്പോളവന്‍ ബ്രഹ്മ-\\
ഹാത്യാദി പാപങ്ങളോടു വേര്‍പെട്ടാതി-\\
ശുദ്ധനായ് വന്നുകൂടും മമാനുഗ്രഹാല്‍;\\
മുക്തിയും വന്നീടുമില്ലൊരു സംശയം.\\
സേതുബന്ധത്തിങ്കല്‍ മജ്ജനവും ചെയ്തു\\
ഭൂതേശനാകിയ രാമേശ്വരനെയും\\
കണ്ടു വണങ്ങിപ്പുറപ്പെട്ടു ശുദ്ധനായ്\\
കുണ്ഠത കൈവിട്ടു വാരാണസി പുക്കു\\
ഗംഗയില്‍ സ്നാനവും ചെയ്തു ജിതശ്രമം\\
ഗംഗാസലിലവും കൊണ്ടുവന്നാദരാല്‍\\
രാമേശ്വരന്നഭിഷേകവും ചെയ്തഥ\\
ശ്രീമല്‍ സമുദ്രേ കളഞ്ഞു തല്‍ഭാരവും\\
മജ്ജനംചെയ്യുന്ന മര്‍ത്ത്യനെന്നോടു സാ-\\
യുജ്യം വരുമതിനില്ലൊരു സംശയം.’\\
എന്നരുള്‍ചെയ്തിതു രാമന്‍ തിരുവടി\\
നന്നായ് തൊഴുതു സേവിച്ചിതെല്ലാവരും\\
വിശ്വകര്‍മാത്മജനാം നളനും പിന്നെ\\
വിശ്വാസമോടു പടുത്തു തുടങ്ങിനാന്‍\\
വിദ്രുതമദ്രിപാഷാണതരുക്കളാ-\\
ലദ്ദിനേ തീര്‍ന്നു പതിന്നാലു യോജന.\\
തീര്‍ന്നിതിരുപതു യോജന പിറ്റേന്നാള്‍\\
മൂന്നാം ദിനമിരുപത്തൊന്നു യോജന\\
നാലാം ദിനമിരുപത്തിരണ്ടായതു-\\
പോലെയിരുപത്തുമൂന്നുമഞ്ചാം ദിനം\\
അഞ്ചുനാള്‍കൊണ്ടു ശതയോജനായതം\\
ചഞ്ചലമെന്നിയേ തീര്‍ന്നോരനന്തരം\\
സേതുവിന്മേലേ നടന്നു കപികളു-\\
മാതങ്കഹീനം കടന്നുതുടങ്ങിനാര്‍.\\
മാരുതികണ്ഠേ കരേറി രഘൂത്തമന്‍\\
താരേയകണ്ഠേ സുമിത്രാതനയനും\\
ആരുഹ്യ ചെന്നു സുബേലാചലമുക-\\
ളേറിനാര്‍ വാനരസേനയോടും ദ്രുതം.\\
ലങ്കാപുരാലോകനാശയാ രാഘവന്‍\\
ശങ്കാവിഹീനം സുബേലാചലോപരി\\
സംപ്രാപ്യ നോക്കിയനേരത്തു കണ്ടിതു\\
ജംഭാരിതന്‍പുരിക്കൊത്ത ലങ്കാപുരം.\\
സ്വര്‍ണമയദ്ധ്വജപ്രാകാരതോരണ-\\
പൂര്‍ണമനോഹരം പ്രാസാദസങ്കുലം\\
കൈലാസശൈലേന്ദ്രസന്നിഭഗോപുര-\\
ജാലപരിഘശതഘ്നീസമന്വിതം\\
പ്രാസാദമൂര്‍ദ്ധ്നി വിസ്തീര്‍ണദേശേ മുദാ\\
വാസവതുല്യപ്രഭാവേന രാവണന്‍\\
രത്നസിംഹാസനേ മന്ത്രിഭിസ്സംകുലേ\\
രത്നദണ്ഡാതപത്രൈരുപശോഭിതേ\\
ആലവട്ടങ്ങളും വെഞ്ചാമരങ്ങളും\\
ബാലത്തരുണിമാരെക്കൊണ്ടു വീയിച്ചു\\
നീലശൈലാഭം ദശകിരീടോജ്ജ്വലം\\
നീലമേഘോപമം കണ്ടു രഘൂത്തമന്‍\\
വിസ്മയം കൈക്കൊണ്ടു മാനിച്ചു മാനസേ\\
സസ്മിതം വാനരന്മാരോടു ചൊല്ലിനാന്‍:\\
‘മുന്നേ നിബദ്ധനായോരു ശുകാസുരന്‍-\\
തന്നെ വിരവോടയയ്ക്ക മടിയാതെ\\
ചെന്നു ദശഗ്രീവനോടു വൃത്താന്തങ്ങ-\\
ളൊന്നൊഴിയാതെയറിയിക്ക വൈകാതെ.’\\
എന്നരുള്‍ ചെയ്തതു കേട്ടു തൊഴുതവന്‍\\
ചെന്നു ദശാനനന്‍ തന്നെ വണങ്ങിനാന്‍.
\end{verse}

%%11_raavanashukasamvaadam

\section{രാവണശുകസംവാദം}

\begin{verse}
പംക്തിമുഖനുമവനോടു ചോദിച്ചാ-\\
‘നെന്തു നീ വൈകുവാന്‍ കാരണം ചൊല്കെടോ!\\
വാനരേന്ദ്രന്മാരറിഞ്ഞു പിടിച്ചഭി-\\
മാനവിരോധം വരുത്തിയാരോ? തവ\\
ക്ഷീണഭാവം കലര്‍ന്നീടുവാന്‍ കാരാണം\\
മാനസേ ഖേദം കളഞ്ഞു ചൊല്ലീടെടോ!’\\
രാത്രിഞ്ചരേന്ദ്രോക്തി കേട്ടു ശുകന്‍ പര-\\
മാര്‍ത്ഥം ദശാനനനോടു ചൊല്ലീടിനാന്‍:\\
‘രാക്ഷസരാജപ്രവര! ജയ! ജയ!\\
മോക്ഷോപദേശമാര്‍ഗേണ ചൊല്ലീടുവന്‍.\\
സിന്ധുതന്നുത്തരതീരോപരി ചെന്നൊ-\\
രന്തരമെന്നിയേ ഞാന്‍ തവ വാക്യങ്ങള്‍\\
ചൊന്നനേര്രത്തവരെന്നെപ്പിടിച്ചുടന്‍\\
കൊന്നുകളവാന്‍ തുടങ്ങുംദശാന്തരേ\\
‘രാമ രാമപ്രഭോ! പാഹി പാഹീ’ തി ഞാ-\\
നാമയം പൂണ്ടു കരഞ്ഞ നാദം കേട്ടു\\
ദൂതനവദ്ധ്യനയപ്പിനയപ്പിനെ-\\
ന്നാദരവോടരുള്‍ചെയ്തു ദയാപരന്‍.\\
വാനരന്മാരുമയച്ചാരതുകൊണ്ടു\\
ഞാനും ഭയം തീര്‍ന്നു നീളേ നടന്നുടന്‍\\
വാനരസൈന്യമെല്ലാം കണ്ടുപോന്നിതു\\
മാനവവീരനനുജ്ഞയാ സാദരം.\\
പിന്നെ രഘൂത്തമനെന്നോടു ചൊല്ലിനാന്‍:\\
‘ചെന്നു നീ രാവണന്‍തന്നൊടു ചൊല്ലുക\\
സീതയെ നല്കീടുകൊന്നുകി, ലല്ലായ്കി-\\
ലേതുമേ വൈകാതെ യുദ്ധം തുടങ്ങുക.\\
രണ്ടിലുമൊന്നുഴറിച്ചെയ്തു കൊള്ളണം\\
രണ്ടും കണക്കെനിക്കെന്നു പറയണം.\\
എന്തു ബലംകൊണ്ടു സീതയെക്കട്ടുകൊ-\\
ണ്ടന്ധനായ്പ്പോന്നങ്ങിരുന്നുകൊണ്ടു ഭവാന്‍\\
പോരുമതിന്നു ബലമെങ്കിലെന്നോടു\\
പോരിനായ്ക്കൊണ്ടു പുറപ്പെടുകാശു നീ.\\
ലങ്കാപുരവും നിശാചരസേനയും\\
ശങ്കാവിഹീനം ശരങ്ങളെക്കൊണ്ടു ഞാന്‍\\
ഒക്കെപ്പൊടിപെടുത്തെന്നുള്ളില്‍ വന്നിങ്ങു\\
പുക്കൊരു രോഷവുമാശു തീര്‍ത്തീടുവന്‍\\
നക്തഞ്ചരകുലശ്രേഷ്ഠന്‍ ഭവാനൊരു\\
ശക്തനെന്നാകില്‍ പുറപ്പെടുകാശു നീ.’\\
എന്നരുളിച്ചെയ്തിരുന്നരുളീടിനാന്‍\\
നിനുടെ സോദരന്‍ തന്നോടുകൂടവേ\\
സുഗ്രീവലക്ഷ്മണന്മാരോടുമൊന്നിച്ചു\\
നിഗ്രഹിപ്പാനായ് ഭവന്തം രണാങ്കണേ.\\
കണ്ടുകൊണ്ടാലുമസംഖ്യം ബലം ദശ-\\
കണ്ഠപ്രഭോ! കപിപുംഗവപാലിതം\\
പര്‍വതസന്നിഭന്മാരായ വാനര-\\
രുര്‍വി കുലുങ്ങവേ ഗര്‍ജനവും ചെയ്തു\\
സര്‍വലോകങ്ങളും ഭസ്മമാക്കീടുവാന്‍\\
ഗര്‍വംകലര്‍ന്നു നില്ക്കുന്നിതു നിര്‍ഭയം\\
സംഖ്യയുമാര്‍ക്കും ഗണിക്കാവതല്ലിഹ\\
സംഖ്യാവതാംവരനായ കുമാരനും\\
ഹുങ്കാരമേറിയ വാനരസേനയില്‍\\
സംഘപ്രധാനന്മാരെ കേട്ടുകൊള്ളുക.\\
ലങ്കാപുരത്തെയും നോക്കി നോക്കി ദ്രുതം\\
ശങ്കാവിഹീനമലറിനില്ക്കുന്നവന്‍\\
നൂറായിരം പടയോടും രിപുക്കളെ\\
നീറാക്കുവാനുഴറ്റോടെ വാല്‍ പൊങ്ങിച്ചു\\
കാലനും പേടിച്ചു മണ്ടുമവനോടു\\
നീലനാം സേനാപതി വഹ്നിനന്ദനന്‍\\
അംഗദനാകുമിളയരാജാവതി-\\
നങ്ങേതുപത്മകിഞ്ജല്ക്കസമപ്രഭന്‍\\
വാല്‍കൊണ്ടു ഭൂമിയില്‍ തച്ചുതച്ചങ്ങനെ\\
ബാലിതന്‍ നന്ദനനദ്രിശൃംഗോപമന്‍\\
തല്‍പാര്‍ശ്വസീമ്നി നില്ക്കുന്നതു വാതജന്‍\\
ത്വല്‍പുത്രഘാതകന്‍ രാമചന്ദ്രപ്രിയന്‍\\
സുഗ്രീവനോടു പറഞ്ഞു നില്ക്കുന്നവ-\\
നുഗ്രനാം ശ്വേതന്‍ രജതസമപ്രഭന്‍\\
രംഭനങ്ങേതവന്‍ മുമ്പില്‍ നില്ക്കുന്നവന്‍\\
വമ്പനായുള്ള ശരഭന്‍ മഹാബലന്‍.\\
മൈന്ദനങ്ങേതവന്‍ തമ്പി വിവിദനും\\
വൃന്ദാരകവൈദ്യനന്ദനന്മാരല്ലോ\\
സേതുകര്‍ത്താവാം നളനതിനങ്ങേതു\\
ബോധമേറും വിശ്വകര്‍മാവുതന്‍ മകന്‍\\
താരന്‍ പനസന്‍ കുമുദന്‍ വിനതനും\\
വീരന്‍ വൃഷഭന്‍ വികടന്‍ വിശാലനും\\
മാരുതിതന്‍പിതാവാകിയ കേസരി\\
ശൂരനായീടും പ്രമാഥി ശതബലി\\
സാരനാം ജാംബവാനും വേഗദര്‍ശിയും\\
വീരന്‍ ഗജനും ഗവയന്‍ ഗവാക്ഷനും\\
ശൂരന്‍ ദധിമുഖന്‍ ജ്യോതിര്‍മുഖനതി-\\
ഘോരന്‍ സുമുഖനും ദുര്‍മുഖന്‍ ഗോമുഖന്‍\\
ഇത്യാദി വാനരനായകന്മാരെ ഞാന്‍\\
പ്രത്യേകമെങ്ങനെ ചൊല്ലുന്നതും പ്രഭോ!\\
ഇത്തരം വാനരനായകന്മാരറു-\\
പത്തേഴുകോടിയുണ്ടുള്ളതറിഞ്ഞാലും\\
ഉള്ളം തെളിഞ്ഞു പോര്‍ക്കായിരുപത്തൊന്നു\\
വെള്ളം പടയുമുണ്ടുള്ളതവര്‍ക്കെല്ലാം\\
ദേവാരികളെയൊടുക്കുവാനായ് വന്ന\\
ദേവാംശസംഭവന്മാരിവരേവരും\\
ശ്രീരാമദേവനും മാനുഷനല്ലാദി-\\
നാരായണനാം പരന്‍ പുരുഷോത്തമന്‍.\\
സീതയാകുന്നതു യോഗമായാദേവി\\
സോദരന്‍ ലക്ഷ്മണനായതനന്തനും\\
ലോകമാതാവും പിതാവും ജനകജാ-\\
രാഘവന്മാരെന്നറിക വഴിപോലെ.\\
വൈരമവരോടു സംഭവിച്ചീടുവാന്‍\\
കാരണമെന്തെന്നതോര്‍ക്ക നീ മാനസേ.\\
പഞ്ചഭൂതാത്മകമായ ശരീരവും\\
പഞ്ചത്വമാശു ഭവിക്കുമെല്ലാവനും\\
പഞ്ചപഞ്ചാത്മക തത്ത്വങ്ങളെക്കൊണ്ടു\\
സഞ്ചിതം പുണ്യപാപങ്ങളാല്‍ ബദ്ധമായ്\\
ത്വങ്മാംസമേദോസ്ഥിമൂത്രമലങ്ങളാല്‍\\
സമ്മേളിതമതിദുര്‍ഗന്ധമെത്രയും\\
ഞാനെന്ന ഭാവമതിങ്കലുണ്ടായ് വരും\\
ജ്ഞാനമില്ലാതെ ജനങ്ങള്‍ക്കതോര്‍ക്ക നീ.\\
ഹന്ത! ജഡാത്മകമായ കായത്തിങ്ക-\\
ലെന്തൊരാസ്ഥാ ഭവിക്കുന്നതും ധീമതാം\\
യാതൊന്നു മൂലമായ് ബ്രഹ്മഹത്യാദിയാം\\
പാതകൗഘങ്ങള്‍ കൃതങ്ങളാകുന്നതും\\
ഭോഗഭോക്താവായ ദേഹം ക്ഷണം കൊണ്ടു\\
രോഗാദിമൂലമായ് സമ്പതിക്കും ദൃഢം\\
പുണ്യപാപങ്ങളോടും ചേര്‍ന്നു ജീവനും\\
വന്നു കൂടുന്നുസുഖദുഃഖബന്ധനം.\\
ദേഹത്തെ ഞാനെന്നു കല്പിച്ചു കര്‍മങ്ങള്‍\\
മോഹത്തിനാലവശത്വേന ചെയ്യുന്നു\\
ജന്മമരണങ്ങളുമതുമൂലമായ്\\
സമ്മോഹിതന്മാര്‍ക്കു വന്നു ഭവിക്കുന്നു.\\
ശോകജരാമരണാദികള്‍ നീക്കുവാ-\\
നാകയാല്‍ ദേഹാഭിമാനം കളക നീ.\\
ആത്മാവു നിര്‍മലനവ്യയനദ്വയ-\\
നാത്മാനമാത്മനാ കണ്ടു തെളിക നീ.\\
ആത്മാവിനെ സ്മരിച്ചീടുക സന്തത-\\
മാത്മനി തന്നെ ലയിക്ക നീ കേവലം.\\
പുത്രദാരാര്‍ത്ഥഗൃഹാദി വസ്തുക്കളില്‍\\
സക്തി കളഞ്ഞു വിരക്തനായ് വാഴുക.\\
സൂകരാശ്വാദി ദേഹങ്ങളിലാകിലും\\
ഭോഗം നരകാദികളിലുമുണ്ടല്ലോ.\\
ദേഹം വിവേകാഢ്യമായതും പ്രാപിച്ചി-\\
താഹന്ത! പിന്നെ ദ്വിജത്വവും വന്നിതു.\\
കര്‍മഭൂവാമത്ര ഭാരതഖണ്ഡത്തില്‍\\
നിര്‍മലം ബ്രഹ്മജന്മം ഭവിച്ചീടിനാല്‍\\
പിന്നെയുണ്ടാകുമോ ഭോഗത്തിലാഗ്രഹം\\
ധന്യനായുള്ളവനോര്‍ക്ക മഹാമതേ!\\
പൗലസ്ത്യപുത്രനാം ബ്രാഹ്മണാഢ്യന്‍ ഭവാന്‍\\
ത്രൈലോക്യസമ്മതന്‍ ഘോരതപോധനന്‍\\
എന്നിരിക്കെപ്പുനരജ്ഞാനിയെപ്പോലെ\\
പിന്നെയും ഭോഗാഭിലാഷമെന്തിങ്ങനെ?\\
ഇന്നു തുടങ്ങി സമസ്തസംഗങ്ങളും\\
നന്നായ് പരിത്യജിച്ചീടുക മാനസേ.\\
രാമനെത്തന്നെ സമാശ്രയിച്ചീടുക\\
രാമനാകുന്നതാത്മാ പരനദ്വയന്‍.\\
സീതയെ രാമനു കൊണ്ടക്കൊടുത്തു തല്‍-\\
പാദപത്മാനുചരനായ് ഭവിക്ക നീ.\\
സര്‍വപാപങ്ങളില്‍നിന്നു വിമുക്തനായ്\\
ദിവ്യമാം വിഷ്ണുലോകം ഗമിക്കായ് വരും.\\
അല്ലായ്കിലാശു കീഴ്പോട്ടു കീഴ്പോട്ടുപോയ്-\\
ച്ചെല്ലും നരകത്തിലില്ലൊരു സംശയം.\\
നല്ലതത്രേ ഞാന്‍ നിനക്കു പറഞ്ഞതു\\
നല്ല ജനത്തോടു ചോദിച്ചു കൊള്‍കെടോ!\\
‘രാമരാമേതി രാമേതി ജപിച്ചുകൊ-\\
ണ്ടാമയം വേറിട്ടു സാധിക്ക മോക്ഷവും\\
സത്സംഗമത്തോടു രാമചന്ദ്രം ഭക്ത-\\
വത്സലം ലോകശരണ്യം ശരണദം\\
ദേവം മരതകകാന്തികാന്തം രമാ-\\
സേവിതം ചാപബാണായുധം രാഘവം\\
സുഗ്രീവസേവിതം ലക്ഷ്മണസംയുതം\\
രക്ഷാനിപുണം വിഭീഷണസേവിതം\\
ഭക്ത്യാ നിരന്തരം ധ്യാനിച്ചു കൊള്‍കിലോ\\
മുക്തി വന്നീടുമതിനില്ല സംശയം.’\\
ഇത്ഥം ശുകവാക്യമജ്ഞാനനാശനം\\
ശ്രുത്വാ ദശാസ്യനും ക്രോധതാമ്രാക്ഷനായ്\\
ദഗ്ദ്ധനായ്പ്പോകും ശുകനെന്നു തോന്നുമാ-\\
റത്യന്തരോഷേണ നോക്കിയുരചെയ്താന്‍:\\
‘ഭൃത്യനായുള്ള നീയാചാര്യനെപ്പോലെ\\
നിസ്ത്രപം ശിക്ഷ ചൊല്‍വാനെന്തു കാരണം?\\
പണ്ടു നീ ചെയ്തോരുപകാരമോര്‍ക്കയാ-\\
ലുണ്ടു കാരുണ്യമെനിക്കതു കൊണ്ടു ഞാന്‍\\
ഇന്നു കൊല്ലുന്നതില്ലെന്നു കല്പിച്ചിതെന്‍\\
മുന്നില്‍ നിന്നാശു മറയത്തു പോക നീ\\
കേട്ടാല്‍ പൊറുക്കരുതാതൊരു വാക്കുകള്‍\\
കേട്ടു പൊറുപ്പാന്‍ ക്ഷമയുമെനിക്കില്ല.\\
എന്നുടെ മുമ്പില്‍ നീ കാല്‍ക്ഷണം നില്‍ക്കിലോ\\
വന്നുകൂടും മരണം നിനക്കിന്നുമേ.’\\
എന്നതു കേട്ടു പേടിച്ചു വിറച്ചവന്‍\\
ചെന്നു തന്മന്ദിരം പുക്കിരുന്നീടിനാന്‍.
\end{verse}

%%12_shukantepoorvavrutthaantham

\section{ശുകന്റെ പൂര്‍വവൃത്താന്തം}

\begin{verse}
ബ്രാഹ്മണശ്രേഷ്ഠന്‍ പുരാ ശുകന്‍ നിര്‍മലന്‍\\
ബ്രാഹ്മണ്യവും പരിപാലിച്ചു സന്തതം\\
കാനനത്തിങ്കല്‍ വാനപ്രസ്ഥനായ് മഹാ-\\
ജ്ഞാനികളില്‍ പ്രധാനിത്വവും കൈക്കൊണ്ടു\\
ദേവകള്‍ക്കഭ്യുദയാര്‍ത്ഥമായ് നിത്യവും\\
ദേവാരികള്‍ക്കു വിനാശത്തിനായ്ക്കൊണ്ടും\\
യാഗാദികര്‍മങ്ങള്‍ചെയ്തു മേവീടിനാന്‍\\
യോഗം ധരിച്ചു പരബ്രഹ്മനിഷ്ഠയാ.\\
വൃന്ദാരകാഭ്യുദയാര്‍ഥിയായ് രാക്ഷസ-\\
നിന്ദാപരനായ് മരുവും ദശാന്തരേ\\
നിര്‍ജരവൈരികുലശ്രേഷ്ഠനാകിയ\\
വജ്രദംഷ്ട്രന്‍ മഹാദുഷ്ടനിശാചരന്‍\\
എന്തൊന്നു നല്ലൂ, ശുകാപകാരത്തിനെ\\
ന്നന്തരവും പാര്‍ത്തു പാര്‍ത്തിരിക്കും വിധൗ\\
കുംഭോത്ഭവനാമഗസ്ത്യന്‍ ശുകാശ്രമേ\\
സമ്പ്രാപ്തനായാനൊരു ദിവസം ബലാല്‍.\\
സംപൂജിതനാമഗസ്ത്യതപോധനന്‍\\
സംഭോജനാര്‍ത്ഥം നിമന്ത്രിതനാകയാല്‍\\
സ്നാതും ഗതേ മുനൗ കുംഭോത്ഭവേ തദാ\\
യാതുധാനാധിപന്‍ വജ്രദംഷ്ട്രാസുരന്‍\\
ചെന്നാനഗസ്ത്യരൂപം ധരിച്ചന്തരാ\\
ചൊന്നാന്‍ ശുകനോടു മന്ദഹാസാന്വിതം:\\
ഒട്ടുനാളുണ്ടു മാംസം കൂട്ടിയുണ്ടിട്ടു\\
മൃഷ്ടമായുണ്ണേണമിന്നു നമുക്കെടോ?\\
ഛാഗമാംസം വേണമല്ലോ, കറി മമ\\
ത്യാഗിയല്ലോ ഭവാന്‍ ബ്രാഹ്മണസത്തമന്‍.\\
എന്നളവേ ശുകന്‍പത്നിയോടും തഥാ\\
ചൊന്നാനതങ്ങനെയെന്നവളും ചൊന്നാല്‍\\
മദ്ധ്യേ ശുകപത്നിവേഷം ധരിച്ചവന്‍\\
ചിത്തമോഹം വളര്‍ത്തീടിനാന്‍ മായയാ.\\
മര്‍ത്ത്യമാംസം വിളമ്പിക്കൊടുത്തമ്പോടു\\
തത്രൈവ വജ്രദംഷ്ട്രന്‍ മറഞ്ഞീടിനാന്‍.\\
മര്‍ത്ത്യമാംസം കണ്ടു മൈത്രാവരുണിയും\\
ക്രുദ്ധനായ് ക്ഷിപ്രം ശുകനെശ്ശപിച്ചിതു:\\
‘മര്‍ത്ത്യരെബ്ഭക്ഷിച്ചു രാക്ഷസനായിനി\\
പൃത്ഥ്വിയില്‍ വാഴുക മത്തപോവൈഭവാല്‍.’\\
ഇത്ഥം ശപിച്ചതു കേട്ടു ശുകന്‍ താനു-\\
‘മെത്രയും ചിത്രമിതെന്തൊരു കാരണം;\\
മാംസോത്തരം ഭുജിക്കേണമെനിക്കെന്നു\\
ശാസന ചെയ്തതും മറ്റാരുമല്ലല്ലോ\\
പിന്നെയതിനു കോപിച്ചു ശപിച്ചതു-\\
മെന്നുടെ ദുഷ്കര്‍മമെന്നേ പറയാവൂ.’\\
‘ചൊല്ലു ചൊല്ലെന്തു പറഞ്ഞതു നീ സഖേ?\\
നല്ല വൃത്താന്തമിതെന്നോടു ചൊല്ലണം.’\\
എന്നതു കേട്ടു ശുകനുമഗസ്ത്യനോ-\\
ടന്നേരമാശു സത്യം പറഞ്ഞീടിനാന്‍:\\
‘മജ്ജനത്തിന്നെഴുന്നള്ളിയ ശേഷമി-\\
തിജ്ജനത്തോടു വീണ്ടും വന്നരുള്‍ചെയ്തു\\
വ്യഞ്ജനം മാംസസമന്വിതം വേണമെ-\\
ന്നഞ്ജസാ ഞാനതു കേട്ടിതു ചെയ്തതും.’\\
ഇത്ഥം ശുകോക്തികള്‍ കേട്ടൊരഗസ്ത്യനും\\
ചിത്തേ മുഹൂര്‍ത്തം വിചാരിച്ചരുളിനാന്‍\\
വൃത്താന്തമുള്‍ക്കാമ്പുകൊണ്ടു കണ്ടൊരള-\\
വുള്‍ത്താപമോടരുള്‍ചെയ്താനഗസ്ത്യനും:\\
‘വഞ്ചിതന്മാരായ് വയം ബത! യാമിനീ-\\
സഞ്ചാരികളിതു ചെയ്തതു നിര്‍ണയം.\\
ഞാനും മതിമൂഢനായ്ച്ചമഞ്ഞേന്‍ ബലാ-\\
ലൂനം വരാ വിധിതന്‍ മതമെന്നുമേ.\\
മിഥ്യയായ് വന്നുകൂടാ മമ ഭാഷിതം\\
സത്യപ്രധാനനല്ലോ നീയുമാകയാല്‍\\
നല്ലതു വന്നുകൂടും മേലില്‍ നിര്‍ണയം\\
കല്യാണമായ് ശാപമോക്ഷവു നല്കുവന്‍.\\
ശ്രീരാമപത്നിയെ രാവണന്‍ കൊണ്ടുപോ-\\
യാരാമസീമനി വെച്ചുകൊള്ളൂം ദൃഢം.\\
രാവണഭൃത്യനായ് നീയും വരും ചിരം\\
കേവലം നീയവനിഷ്ടനായും വരും.\\
രാഘവന്‍ വാനരസേനയുമായ് ചെന്നൊ-\\
രാകുലമെന്നിയേ ലങ്കാപുരാന്തികേ\\
നാലു പുറവും വളഞ്ഞിരിക്കുന്നൊരു\\
കാലമവസ്ഥയറിഞ്ഞു വന്നീടുവാന്‍\\
നിന്നെയയയ്ക്കും ദശാനനനന്നു നീ\\
ചെന്നു വണങ്ങുക രാമനെസ്സാദരം\\
പിന്നെ വിശേഷങ്ങളൊന്നൊഴിയാതെ പോയ്-\\
ച്ചെന്നു ദശമുഖന്‍ തന്നോടു ചൊല്ലുക.\\
രാവണനാത്മതത്ത്വോപദേശം ചെയ്തു\\
ദേവപ്രിയനായ് വരും പുനരാശു നീ.\\
രാക്ഷസഭാവമശേഷമുപേക്ഷിച്ചു\\
സാക്ഷാല്‍ ദ്വിജത്വവും വന്നുകൂടും ദൃഢം.’\\
ഇത്ഥമനുഗ്രഹിച്ചു കലശോത്ഭവന്‍\\
സത്യം തപോധനവാക്യം മനോഹരം.
\end{verse}

%%13_maalyavaantevaakyam

\section{മാല്യവാന്റെ വാക്യം}

\begin{verse}
സാരനായോരു ശുകന്‍ പോയനന്തരം\\
ഘോരനാം രാവണന്‍ വാഴുന്ന മന്ദിരേ\\
വന്നിതു രാവണമാതാവുതന്‍ പിതാ\\
ഖിന്നനായ് രാവണനെക്കണ്ടു ചൊല്ലുവാന്‍.\\
സല്‍ക്കാരവും കുശലപ്രശ്നവും ചെയ്തു\\
രക്ഷോവരനുമിരുത്തി യഥോചിതം\\
കൈകസീതാതന്‍ മതിമാന്‍ വിനീതിമാന്‍\\
കൈകസീനന്ദനന്‍തന്നോടു ചൊല്ലിനാന്‍:\\
‘ചൊല്ലുവന്‍ ഞാന്‍ തവ നല്ലതു, പിന്നെ നീ-\\
യെല്ലാം നിനക്കൊത്തവണ്ണമനുഷ്ഠിക്ക.\\
ദുര്‍ന്നിമിത്തങ്ങളീ ജാനകി ലങ്കയില്‍\\
വന്നതില്‍പ്പിന്നെപ്പലതുണ്ടു കാണുന്നു.\\
കണ്ടീലയോ നാശഹേതുക്കളായ് ദശ-\\
കണ്ഠപ്രഭോ! നീ നിരൂപിക്ക മാനസേ.\\
ദാരുണമായിടിവെട്ടുന്നിതന്വഹം\\
ചോരയും പെയ്യുന്നിതുഷ്ണമായെത്രയും\\
ദേവലിംഗങ്ങളിളകി വിയര്‍ക്കുന്നു\\
ദേവിയാം കാളിയും ഘോരദംഷ്ട്രാന്വിതം\\
നോക്കുന്ന ദിക്കില്‍ ചിരിച്ചുകാണാകുന്നി,\\
ഗോക്കളില്‍നിന്ന് ഖരങ്ങള്‍ ജനിക്കുന്നു\\
മൂഷികന്‍ മാര്‍ജാരനോടു പിണങ്ങുന്നു\\
രോഷാല്‍ നകുലങ്ങളോടുമവ്വണ്ണമേ.\\
പന്നഗജാലം ഗരുഡനോടും തഥാ\\
നിന്നെതിര്‍ത്തീടാന്‍ തുടങ്ങുന്നു നിശ്ചയം.\\
മുണ്ഡനായേറ്റം കരാളവികടനായ്\\
വര്‍ണവും പിംഗലകൃഷ്ണമായ് സന്തതം\\
കാലനെയുണ്ടു കാണുന്നിതെല്ലാടവും\\
കാലമാപത്തിനുള്ളോന്നിതു നിര്‍ണയം.\\
ഇത്തരം ദുര്‍ന്നിമിത്തങ്ങളുണ്ടായതി-\\
നത്രൈവ ശാന്തിയെച്ചെയ്തുകൊള്ളേണമേ.\\
വംശത്തെ രക്ഷിച്ചുകൊള്ളുവാനേതുമേ\\
സംശയമെന്നിയേ സീതയെക്കൊണ്ടുപോയ്\\
രാമപാദേ വെച്ചു വന്ദിക്ക വൈകാതെ\\
രാമനാകുന്നതു വിഷ്ണു നാരായണന്‍\\
വിദ്വേഷമെല്ലാം ത്യജിച്ചു ഭുജിച്ചുകൊള്‍-\\
കദ്വയനാം പരമാത്മാനമവ്യയം\\
ശ്രീരാമപാദപോതംകൊണ്ടുസംസാര-\\
വാരിധിയെക്കടക്കുന്നിതു യോഗികള്‍.\\
ഭക്തികൊണ്ടന്തഃകരണവും ശുദ്ധമായ്\\
മുക്തിയെ ജ്ഞാനികള്‍ സിദ്ധിച്ചുകൊള്ളുന്നു.\\
ദുഷ്ടനാം നീയും വിശുദ്ധനാം ഭക്തികൊ-\\
ണ്ടൊട്ടുമേ കാലം കളയാതെ കണ്ടു നീ\\
രാക്ഷസവംശാത്തെ രക്ഷിച്ചു കൊള്ളുക\\
സാക്ഷാല്‍ മുകുന്ദനെസ്സേവിച്ചുകൊള്ളുക.\\
സത്യമത്രേ ഞാന്‍ പറഞ്ഞതു കേവലം\\
പഥ്യം നിനക്കിതു ചിന്തിക്ക മാനസേ.’\\
സാന്ത്വനപൂര്‍വം ദശമുഖന്‍ തന്നോടു\\
ശാന്തനാം മാല്യവാന്‍ വംശരക്ഷാര്‍ഥമായ്\\
ചൊന്നതു കേട്ടു പൊറാഞ്ഞു ദശമുഖന്‍\\
പിന്നെയും മാല്യവാന്‍ തന്നോടു ചൊല്ലിനാന്‍:\\
‘മാനവനായ കൃപണനാം രാമനെ\\
മാനസേ മാനിപ്പതിനെന്തു കാരണം?\\
മര്‍ക്കടാലംബനം നല്ല സാമര്‍ഥ്യമെ-\\
ന്നുള്‍ക്കാമ്പിലോര്‍ക്കുന്നവന്‍ ജളനെത്രയും\\
രാമന്‍ നിയോഗിക്കയാല്‍ വന്നിതെന്നോടു\\
സാമപൂര്‍വം പറഞ്ഞൂ ഭവാന്‍ നിര്‍ണയം.\\
നേരത്തുപോയാലുമിന്നി, വേണ്ടുന്ന നാള്‍\\
ചാരത്തു ചൊല്ലി വിടുന്നുണ്ടു നിര്‍ണയം.\\
വൃദ്ധന്‍ ഭാവാനതിസ്നിഗ്ദ്ധനാം മിത്രമി-\\
ത്യുക്തികള്‍ കേട്ടാല്‍ പൊറുത്തുകൂടാ ദൃഢം.’\\
ഇത്ഥം പറഞ്ഞമാത്യന്മാരുമായ് ദശ-\\
വക്ത്രനും പ്രാസാദമൂര്‍ദ്ധ്നി കരേറിനാന്‍.
\end{verse}

%%14_yuddhaarambham

\section{യുദ്ധാരംഭം}

\begin{verse}
വാനരസേനയും കണ്ടകമേ ബഹു-\\
മാനവും കൈക്കൊണ്ടിരിക്കും ദശാന്തരേ\\
യുദ്ധത്തിനായ് രജനീചരവീരരെ-\\
സ്സത്വരം തത്ര വരുത്തി വാഴും വിധൗ\\
രാവണനെക്കണ്ടു കോപിച്ചു രാഘവ-\\
ദേവനും സൗമിത്രിയോടു വില്‍ വാങ്ങിനാന്‍.\\
പത്തു കിരീടവും കൈകളിരുപതും\\
വൃത്രനോടൊത്ത ശരീരവും ശൗര്യവും\\
പത്തു കിരീടങ്ങളും കുടയും നിമി-\\
ഷാര്‍ദ്ധേന ഖണ്ഡിച്ച നേരത്തു രാവണന്‍\\
നാണിച്ചു താഴത്തിറങ്ങി ഭയംകൊണ്ടു\\
ബാണത്തെ നോക്കിനോക്കിച്ചരിച്ചീടിനാന്‍.\\
മുഖ്യപ്രഹസ്തപ്രമുഖപ്രവരന്മാ-\\
രൊക്കവേ വന്നു തൊഴുതോരനന്തരം\\
‘യുദ്ധമേറ്റീടുവിന്‍ കോട്ടയില്‍ പുക്കട-\\
ച്ചത്യന്തഭീത്യാ വസിക്കയില്ലത്ര നാം.’\\
ഭേരീമൃദംഗഢക്കാപണവാനക-\\
ദാരുണഗോമുഖാദ്യങ്ങള്‍ വാദ്യങ്ങളും\\
വാരണാശ്വോഷ്ട്രഖര ഹരി ശാര്‍ദൂല-\\
സൈരിഭസ്യന്ദനമുഖ്യയാനങ്ങളില്‍\\
ഖഡ്ഗശൂലേഷു ചാപപ്രാസതോമര\\
മുല്‍ഗരയഷ്ടിശക്തിച്ഛുരികാദികള്‍\\
ഹസ്തേ ധരിച്ചുകൊണ്ടസ്തഭീത്യാ ജവം\\
യുദ്ധസന്നദ്ധരായുദ്ധതബുദ്ധിയോ-\\
ടബ്ധികളദ്രികളുര്‍വിയും തല്‍ക്ഷണ-\\
മുദ്ധൂതമായിതു സത്യലോകത്തോളം.\\
വജ്രഹസ്താശയില്‍ പുക്കാന്‍ പ്രഹസ്തനും\\
വജ്രദംഷ്ട്രന്‍ തഥാ ദക്ഷിണദിക്കിലും\\
ദുശ്ച്യവനാരിയാം മേഘനാദന്‍ തദാ\\
പശ്ചിമഗോപുരദ്വാരി പുക്കീടിനാന്‍.\\
മിത്രവര്‍ഗാമാത്യഭൃത്യജനത്തൊടു-\\
മുത്തരദ്വാരി പുക്കാന്‍ ദശവക്ത്രനും.\\
നീലനും സേനയും പൂര്‍വദിഗ്ഗോപുരേ\\
ബാലിതനയനും ദക്ഷിണഗോപുരേ\\
വായുതനയനും പശ്ചിമഗോപുരേ\\
മായാമനുഷ്യനാമാദിനാരായണന്‍\\
മിത്രതനയസൗമിത്രി വിഭീഷണ-\\
മിത്രസംയുക്തനായുത്തരദിക്കിലും\\
ഇത്ഥമുറപ്പിച്ചു രാഘവരാവണ-\\
യുദ്ധം പ്രവൃത്തമായ് വന്നു വിചിത്രമായ്.\\
ആയിരം കോടി മഹാകോടികളോടു-\\
മായിരമര്‍ബുദമായിരം ശംഖങ്ങള്‍\\
ആയിരം പുഷ്പങ്ങളായിരം കല്പങ്ങ-\\
ളായിരം ലക്ഷങ്ങളായിരം ദണ്ഡങ്ങള്‍\\
ആയിരം ധൂളികളായിരമായിരം\\
തോയാകരപ്രളയങ്ങളെന്നിങ്ങനെ\\
സംഖ്യകളോടു കലര്‍ന്ന കപിബലം\\
ലങ്കാപുരത്തെ വളഞ്ഞാരതിദ്രുതം.\\
പൊട്ടിച്ചടര്‍ത്ത പാഷാണങ്ങളെക്കൊണ്ടും\\
മുഷ്ടികള്‍കൊണ്ടും മുസലങ്ങളെക്കൊണ്ടും\\
ഉര്‍വീരുഹംകൊണ്ടുമുര്‍വീധരം കൊണ്ടും\\
സര്‍വതോ ലങ്കാപുരം തകര്‍ത്തീടിനാര്‍.\\
കോട്ടമതിലും കിടങ്ങും തകര്‍ത്തുടന്‍\\
കൂട്ടമിട്ടാര്‍ത്തു വിളിച്ചടുക്കുന്നേരം\\
വൃഷ്ടിപോലെ ശരജാലം പൊഴിക്കയും\\
വെട്ടുകൊണ്ടറ്റു പിളര്‍ന്നു കിടക്കയും\\
അസ്ത്രങ്ങള്‍ ശസ്ത്രങ്ങള്‍ ചക്രങ്ങള്‍ ശക്തിക-\\
ളര്‍ദ്ധചന്ദ്രാകാരമായുള്ള പത്രികള്‍\\
ഖഡ്ഗങ്ങള്‍ ശൂലങ്ങള്‍ കുന്തങ്ങളീട്ടികള്‍\\
മുല്‍ഗരപംക്തികള്‍ ഭിണ്ഡിപാലങ്ങളും\\
തോമരദണ്ഡം മുസലങ്ങള്‍ മുഷ്ടികള്‍\\
ചാമീകരപ്രഭപൂണ്ട ശാതഘ്നികള്‍\\
ഉഗ്രങ്ങളായ വജ്രങ്ങളിവകൊണ്ടു\\
നിഗ്രഹിച്ചീടിനാര്‍ നക്തഞ്ചരേന്ദ്രരും.\\
ആര്‍ത്തിമുഴുത്തു ദശാസ്യനവസ്ഥകള്‍\\
പേര്‍ത്തുമറിവതിനായയച്ചീടിനാന്‍\\
ശാര്‍ദ്ദൂലനാദിയാം രാത്രിഞ്ചരന്മാരെ\\
രാത്രിയില്‍ച്ചെന്നാരവരും കപികളായ്\\
മര്‍ക്കടേന്ദ്രന്മാരറിഞ്ഞു പിടിച്ചടി-\\
ച്ചുല്‍ക്കടരോഷേണ കൊല്‍വാന്‍ തുടങ്ങുമ്പോള്‍\\
ആര്‍ത്തനാദം കേട്ടു രാഘവനും കരു-\\
ണാര്‍ദ്രബുദ്ധ്യാ കൊടുത്താനഭയം ദ്രുതം.\\
ചെന്നവരും ശുകസാരണരെപ്പോലെ\\
ചൊന്നതു കേട്ടു വിഷാദേന രാവണന്‍\\
മന്ത്രിച്ചുടന്‍ വിദ്യുജ്ജിഹ്വനുമായ് ദശ-\\
കന്ധരന്‍ മൈഥിലി വാഴുമിടം പുക്കാന്‍.\\
രാമശിരസ്സും ധനുസ്സുമിതെന്നുടന്‍\\
വാമാക്ഷിമുന്നിലാമ്മാറു വെച്ചീടിനാന്‍.\\
ആയോധനേ കൊന്നു കൊണ്ടുപോന്നേനെന്നു\\
മായയാ നിര്‍മിച്ചുവെച്ചതു കണ്ടപ്പോള്‍\\
സത്യമെന്നോര്‍ത്തു വിലാപിച്ചു മോഹിച്ചു\\
മുഗ്ദ്ധാംഗി വീണുകിടക്കും ദശാന്തരേ\\
വന്നൊരു ദൂതന്‍ വിരവോടു രാവണന്‍-\\
തന്നെയും കൊണ്ടുപോന്നീടിനാനന്നേരം.\\
വൈദേഹി തന്നോടു ചൊന്നാള്‍ സരമയും:\\
‘ഖേദമശേഷമകലെക്കളക നീ\\
എല്ലാം ചതിയെന്നു തേറീടിതൊക്കവേ\\
നല്ലവണ്ണം വരും നാലു നാളുള്ളിലി-\\
ങ്ങില്ലൊരു സംശയം കല്യാണദേവതേ?\\
വല്ലഭന്‍ കൊല്ലും ദശാസ്യനെ നിര്‍ണയം.’\\
ഇത്ഥം സരമാസരസവാക്യം കേട്ടു\\
ചിത്തം തെളിഞ്ഞിരുന്നീടിനാള്‍ സീതയും.\\
മംഗലദേവതാവല്ലഭാജ്ഞാവശാ-\\
ലംഗദന്‍ രാവണന്‍ തന്നോടു ചൊല്ലിനാന്‍:\\
‘ഒന്നുകില്‍ സീതയെ കൊണ്ടുവന്നെന്നുടെ\\
മുന്നിലാമ്മാറു വെച്ചീടുക വൈകാതെ.\\
യുദ്ധത്തിനാശു പുറപ്പെടുകല്ലായ്കി-\\
ലത്തല്‍ പൂണ്ടുള്ളിലടച്ചങ്ങിരിക്കിലും\\
രാക്ഷസസേനയും ലങ്കാനഗരവും\\
രാക്ഷസരാജനാം നിന്നോടുകൂടവേ\\
സംഹരിച്ചീടുവന്‍ ബാണമെയ്തെന്നുള്ള\\
സിംഹനാദം കേട്ടതില്ലയോ രാവണ?\\
ജ്യാനാദഘോഷവും കേട്ടതില്ലേ ഭവാന്‍\\
നാണം നിനക്കേതുമില്ലയോ മാനസേ?’\\
ഇത്ഥമധിക്ഷേപവാക്കുകള്‍ കേട്ടതി-\\
ക്രുദ്ധനായോരു രാത്രിഞ്ചരവീരനും\\
വൃത്രാരിപുത്രതനയനെക്കൊല്‍കെന്നു\\
നക്തഞ്ചരാധിപന്മാരോടു ചൊല്ലിനാന്‍.\\
ചെന്നു പിടിച്ചാര്‍ നിശാചരവീരരും\\
കൊന്നു ചുഴറ്റിയെറിഞ്ഞാന്‍ കപീന്ദ്രനും.\\
പിന്നെയപ്രാസാദവും തകര്‍ത്തീടിനാ-\\
നൊന്നു കുതിച്ചങ്ങുയര്‍ന്നു വേഗേന പോയ്.\\
മന്നവന്‍ തന്നെത്തൊഴുതു വൃത്താന്തങ്ങ-\\
ളൊന്നൊഴിയാതെയുണര്‍ത്തിനാനംഗദന്‍.\\
പിന്നെസ്സുഷേണന്‍ കുമുദന്‍ നളന്‍ ഗജന്‍\\
ധന്യന്‍ ഗവയന്‍ ഗവാക്ഷന്‍ മരുല്‍സുതന്‍\\
എന്നിവരാദിയാം വാനരവീരന്മാര്‍\\
ചെന്നു ചുഴന്നു കിടങ്ങും നിരത്തിനാര്‍.\\
കല്ലും മലയും മരവും ധരിച്ചാശു\\
നില്ലുനില്ലെന്നു പറഞ്ഞടുക്കുന്നേരം\\
ബാണചാപങ്ങളും വാളും പരിചയും\\
പ്രാണഭയം വരും വെണ്മഴു കുന്തവും\\
ദണ്ഡങ്ങളും മുസലങ്ങള്‍ ഗദകളും\\
ഭിണ്ഡിപാലങ്ങളും മുല്‍ഗരജാലവും\\
ചക്രങ്ങളും പരിഘങ്ങളുമീട്ടികള്‍\\
സുക്രകചങ്ങളും മറ്റുമിത്യാദികള്‍\\
ആയുധമെല്ലാമെടുത്തു പിടിച്ചുകൊ-\\
ണ്ടായോധനത്തിന്നടുത്താരരക്കരും.\\
വാരണനാദവും വാജികള്‍ നാദവും\\
തേരുരുള്‍നാദവും ഞാണൊലിനാദവും\\
രാക്ഷസരാര്‍ക്കയും സംഹനാദങ്ങളും\\
രൂക്ഷതയേറും കപികള്‍ നിനാദവും\\
തിങ്ങിമുഴങ്ങിപ്പുഴങ്ങി പ്രപഞ്ചവു-\\
മെങ്ങുമിടതൂര്‍ന്നു മാറ്റൊലിക്കൊണ്ടുതേ.\\
ജംഭാരിമുമ്പാം നിലമ്പരും കിന്നര-\\
കിംപുരുഷോരഗഗുഹ്യകസംഘവും\\
ഗന്ധര്‍വസിദ്ധവിദ്യാധരചാരണാ-\\
ദ്യന്തരീക്ഷാന്തരേ സഞ്ചരിക്കും ജനം\\
നാരദനാദികളായ മുനികളും\\
ഘോരമായുള്ള യുദ്ധം കണ്ടുകൊള്ളുവാന്‍\\
നാരികളോടും വിമാനയാനങ്ങളി-\\
ലാരുഹ്യ പുഷ്കരാന്തേ നിറഞ്ഞീടിനാര്‍.\\
തുംഗനാമിന്ദ്രജിത്തേറ്റാനതുനേര-\\
മംഗദന്‍തന്നോടതിന്നു കപീന്ദ്രനും\\
സൂതനെക്കൊന്നു തേരും തകര്‍ത്താന്‍ മേഘ-\\
നാദനും മറ്റൊരു തേരിലേറീടിനാന്‍.\\
മാരുതിതന്നെ വേല്‍കൊണ്ടു ചാട്ടീടിനാന്‍\\
ധീരനാകും ജംബുമാലി നിശാചരന്‍\\
സാരഥിതന്നോടു കൂടവേ മാരുതി\\
തേരും തകര്‍ത്തവനെക്കൊന്നലറിനാന്‍.\\
മിത്രതനയന്‍ പ്രഹസ്തനോടേറ്റിതു\\
മിത്രാരിയോടു വിഭീഷണവീരനും\\
നീലന്‍ നികുംഭനോടേറ്റാന്‍ തപനനെ-\\
ക്കാലപുരത്തിനയച്ചാന്‍ മഹാഗജന്‍.\\
ലക്ഷ്മണനേറ്റാന്‍ വിരൂപാക്ഷനോടഥ\\
ലക്ഷ്മീപതിയാം രഘൂത്തമന്‍തന്നോടു\\
രക്ഷധ്വജാഗ്നിധ്വജാദികള്‍ പത്തുപേര്‍\\
തല്‍ക്ഷണേ പോര്‍ചെയ്തു പുക്കാര്‍ സുരാലയം.\\
വാനരന്മാര്‍ക്കു ജയം വന്നിതന്നേരം\\
ഭാനുവും വാരിധിതന്നില്‍ വീണീടിനാന്‍.\\
ഇന്ദ്രാത്മജാത്മജനോടേറ്റു തോറ്റുപോ-\\
യിന്ദ്രജിത്തംബരാന്തേ മറഞ്ഞീടിനാന്‍.\\
നാഗാസ്ത്രമെയ്തു മോഹിപ്പിച്ചിതു ബത\\
രാഘവന്മാരെയും വാനരന്മാരെയും\\
വന്നകപികളെയും നരന്മാരെയു-\\
മൊന്നൊഴിയാതെ ജയിച്ചേനിതെന്നവന്‍\\
വെന്നിപ്പെരുമ്പറ കൊട്ടിച്ചു മേളിച്ചു\\
ചെന്നു ലങ്കാപുരം തന്നില്‍ മേവീടിനാന്‍.\\
താപസവൃന്ദവും ദേവസമൂഹവും\\
താപം കലര്‍ന്നു വിഭീഷണവീരനും\\
ഹാ! ഹാ! വിഷാദേന ദുഃഖവിഷണ്ണരായ്\\
മോഹിതന്മാരായ് മരുവും ദശാന്തരേ\\
സപ്തദ്വീപങ്ങളും സപ്താര്‍ണവങ്ങളും\\
സപ്താചലങ്ങളുമുള്‍ക്ഷോഭമാംവണ്ണം\\
സപ്താശ്വകോടിതേജോമയനായ് സുവര്‍-\\
ണാദ്രിപോലേ പവനാശനനാശനന്‍\\
അബ്ധിതോയം ദ്വിധാ ഭിത്വാ സ്വപക്ഷയു-\\
ഗ്മോദ്ധൂതലോകത്രയത്തോടതിദ്രുതം\\
നാഗാരി രാമപാദം വണങ്ങീടിനാന്‍\\
നാഗാസ്ത്രബന്ധനം തീര്‍ന്നിതു തല്‍ക്ഷണേ.\\
ശാഖാമൃഗങ്ങളുമസ്ത്രനിര്‍മുക്തരായ്\\
ശോകവും തീര്‍ന്നു തെളിഞ്ഞു വിളങ്ങിനാര്‍\\
ഭക്തപ്രിയന്‍ മുദാ പക്ഷിപ്രവരനു\\
ബദ്ധസമ്മോദമനുഗ്രഹം നല്കിനാന്‍.\\
കൂപ്പിത്തൊഴുതനുവാദവും കൈക്കൊണ്ടു\\
മേല്പോട്ടുപോയ് മറഞ്ഞീടിനാന്‍ താര്‍ക്ഷ്യനും\\
മുന്നേതിലും ബലവീര്യവേഗങ്ങള്‍ പൂ-\\
ണ്ടുന്നതന്മാരാം കപിവരന്മാരെല്ലാം\\
മന്നവന്‍തന്‍ നിയോഗേന മരങ്ങളും\\
കുന്നും ശിലയുമെടുത്തെറിഞ്ഞീടിനാര്‍\\
‘വന്ന ശത്രുക്കളെക്കൊന്നു മമാത്മജന്‍\\
മന്ദിരം പുക്കിരിക്കുന്നതിന്‍ മുന്നമേ\\
വന്നാരവരുമിങ്ങെന്തൊരു വിസ്മയം\\
നന്നുനന്നെത്രയുമെന്നേ പറയാവൂ.\\
ചെന്നറിഞ്ഞീടുവിനെന്തൊരു ഘോഷമി’\\
തെന്നു ദശാനനന്‍ ചൊന്നോരനന്തരം\\
ചെന്നുദൂതന്മാരറിഞ്ഞു ദശാനനന്‍\\
തന്നോടു ചൊല്ലിനാര്‍ വൃത്താന്തമൊക്കവേ\\
‘വീര്യബലവേഗവിക്രമം കൈക്കൊണ്ടു\\
സൂര്യാത്മജാദികളായ കപികുലം\\
ഹസ്തങ്ങള്‍ തോറുമലാതവും കൈക്കൊണ്ടു\\
ഭിത്തിതന്നുത്തമാംഗത്തിന്മേല്‍ നില്ക്കുന്നോര്‍\\
നാണമുണ്ടെങ്കില്‍ പുറത്തു പുറപ്പെടു-\\
കാണുങ്ങളെങ്കിലെന്നാര്‍ത്തു പറകയും\\
കേട്ടതില്ലേ ഭവാ’നെന്നവര്‍ ചൊന്നതു\\
കേട്ടു ദശാസ്യനും കോപേന ചൊല്ലിനാന്‍:\\
‘മാനവന്മാരെയുമേറെ മദമുള്ള\\
വാനരന്മാരെയും കൊന്നൊടുക്കീടുവാന്‍\\
പോക ധൂമ്രാക്ഷന്‍ പടയോടുകൂടവേ\\
വേഗേന യുദ്ധം ജയിച്ചു വരിക നീ.’\\
ഇത്ഥമനുഗ്രഹം ചെയ്തയച്ചാനതി-\\
ക്രുദ്ധനാം ധൂമ്രാക്ഷനും നടന്നീടിനാന്‍.\\
ഉച്ചൈസ്തരമായ വാദ്യഘോഷത്തൊടും\\
പശ്ചിമഗോപുരത്തൂടെ പുറപ്പെട്ടാന്‍\\
മാരുതിയോടെതിര്‍ത്താനവനും ചെന്നു\\
ദാരുണമായിതു യുദ്ധവുമെത്രയും.\\
വേലസി വെണ്മഴു കുന്തം ശരാസനം\\
ശൂലം മുസലം പരിഘഗദാദികള്‍\\
കൈക്കൊണ്ടു വാരണവാജിരഥങ്ങളി-\\
ലുള്‍ക്കരുത്തോടേറി രാക്ഷസവീരരും.\\
കല്ലും മരവും മലയുമായ് പര്‍വത-\\
തുല്യശരീരികളായ കപികളും\\
തങ്ങളിലേറ്റു പൊരുതുമരിച്ചിതൊ-\\
ട്ടങ്ങുമിങ്ങും മഹാവീരരായുള്ളവര്‍.\\
ചോരയുമാറായൊഴുകീ പല വഴി\\
ശൂരപ്രവരനാം മാരുതി തല്‍ക്ഷണേ\\
ഉന്നതമായൊരു കുന്നിന്‍കൊടുമുടി\\
തന്നെയടര്‍ത്തെടുത്തൊന്നെറിഞ്ഞീടിനാന്‍.\\
തേരില്‍നിന്നാശു ഗദയുമെടുത്തുടന്‍\\
പാരിലാമ്മാറു ധൂമ്രാക്ഷനും ചാടിനാന്‍.\\
തേരും കുതിരകളും പൊടിയായിതു\\
മാരുതിക്കുള്ളില്‍ വര്‍ധിച്ചിതു കോപവും\\
രാത്രിഞ്ചരരെയൊടുക്കിത്തുടങ്ങിനാ-\\
നാര്‍ത്തി മുഴുത്തതു കണ്ടു ധൂമ്രാക്ഷനും\\
മാരുതിയെഗ്ഗദകൊണ്ടടിച്ചീടിനാന്‍\\
ധീരതയോടതിനാകുലമെന്നിയേ\\
പാരം വളര്‍ന്നൊരു കോപവിവശനായ്\\
മാരുതി രണ്ടാമതൊന്നെറിഞ്ഞീടിനാന്‍.\\
ധൂമ്രാക്ഷനേറുകൊണ്ടുമ്പര്‍പുരത്തിങ്ക-\\
ലാമ്മാറു ചെന്നു സുഖിച്ചു വാണീടിനാന്‍.\\
ശേഷിച്ച രാക്ഷസര്‍ കോട്ടയില്‍ പുക്കിതു\\
ഘോഷിച്ചിതംഗനമാര്‍ വിലാപങ്ങളും.\\
വൃത്താന്തമാഹന്ത! കേട്ടു ദശാസ്യനും\\
ചിത്തതാപത്തോടു പിന്നെയും ചൊല്ലിനാന്‍:\\
‘വജ്രഹസ്താരിപ്രവരന്‍ മഹാബലന്‍\\
വജ്രദംഷ്ട്രന്‍തന്നെ പോക യുദ്ധത്തിനായ്\\
മാനുഷവാനരന്മാരെജ്ജയിച്ചഭി-\\
മാനകീര്‍ത്ത്യാ വരികെ’ന്നയച്ചീടിനാന്‍.\\
ദക്ഷിണഗോപുരത്തൂടേ പുറപ്പെട്ടു\\
ശക്രാത്മജാത്മജനോടെതിര്‍ത്തീടിനാന്‍.\\
ദുര്‍ന്നിമിത്തങ്ങളുണ്ടായതനാദൃത്യ\\
ചെന്നു കപികളോടേറ്റു മഹാബലന്‍\\
വൃക്ഷശിലാശൈലവൃഷ്ടികൊണ്ടേറ്റവും\\
രക്ഷോവരന്മാര്‍ മരിച്ചു മഹാരണേ.\\
ഖഡ്ഗശസ്ത്രാസ്ത്രശക്ത്യാദികളേറ്റേറ്റു\\
മര്‍ക്കടന്മാരും മരിച്ചാരസംഖ്യമായ്.\\
പത്തംഗയുക്തമായുള്ള പെരുമ്പട\\
നക്തഞ്ചരന്മാര്‍ക്കു നഷ്ടമായ് വന്നിതു\\
രക്തനദികളൊലിച്ചു പലവഴി\\
നൃത്തം തുടങ്ങീ കബന്ധങ്ങളും ബലാല്‍.\\
താരേയനും വജ്രദംഷ്ട്രനും തങ്ങളില്‍\\
ഘോരമായേറ്റം പിണങ്ങിനില്ക്കും വിധൗ\\
വാളും പറിച്ചുടന്‍ വജ്രദംഷ്ട്രന്‍ ഗള-\\
നാളം മുറിച്ചെറിഞ്ഞീടിനാനംഗദന്‍.\\
അക്കഥ കേട്ടാശു നക്തഞ്ചരാധിപന്‍\\
ഉള്‍ക്കരുത്തേറുമകമ്പനന്‍തന്നെയും\\
വന്‍പടയോടുമയച്ചാനതുനേരം\\
കമ്പമുണ്ടായിതു മേദിനിക്കന്നേരം.\\
ദുശ്ച്യവനാരിപ്രവരനകമ്പനന്‍\\
പശ്ചിമഗോപുരത്തൂടെ പുറപ്പെട്ടാന്‍\\
വായുതനയനോടേറ്റവനും നിജ-\\
കായം വെടിഞ്ഞു കാലാലയം മേവിനാന്‍.\\
മാരുതിയെ സ്തുതിച്ചു മഹാലോകരും\\
പാരം ഭയം പെരുത്തൂ ദശകണ്ഠനും\\
സഞ്ചരിച്ചാന്‍ നിജരാക്ഷസസേനയില്‍\\
പഞ്ചദ്വയാസ്യനും കണ്ടാനതുനേരം\\
രാമേശ്വരത്തോടു സേതുവിന്‍മേലുമാ-\\
രാമദേശാന്തം സുബേലാചലോപരി\\
വാനരസേന പരന്നതും കോട്ടക-\\
ളൂനമായ് വന്നതും കണ്ടോരനന്തരം\\
‘ക്ഷിപ്രം പ്രഹസ്തനെക്കൊണ്ടുവരികെ’ന്നു\\
കല്പിച്ചനേരമവന്‍ വന്നു കൂപ്പിനാന്‍.\\
‘നീയറിഞ്ഞീലയോ വൃത്താന്തമൊക്കവേ\\
നായകന്മാര്‍ പടയ്ക്കാരുമില്ലായ്കയോ?\\
ചെല്ലുന്ന ചെല്ലുന്ന രാക്ഷസവീരരെ-\\
കൊല്ലുന്നതും കണ്ടിരിക്കയില്ലിങ്ങു നാം.\\
ഞാനോ ഭവാനോ കനിഷ്ഠനോ പോര്‍ചെയ്തു\\
മാനുഷവാനരന്മാരെയൊടുക്കുവാന്‍\\
പോകുന്നതാരെന്നു ചൊല്‍’കെന്നു കേട്ടവന്‍\\
‘പോകുന്നതിന്നു ഞാ’നെന്നു കൈകൂപ്പിനാന്‍.\\
തന്നുടെ മന്ത്രികള്‍ നാലുപേരുള്ളവര്‍\\
ചെന്നു നാലംഗപ്പടയും വരുത്തിനാര്‍.\\
നാലൊന്നു ലങ്കയിലുള്ള പടയ്ക്കെല്ലാ-\\
മാലംബനമാം പ്രഹസ്തന്‍ മഹാരഥന്‍.\\
കുംഭഹനും മഹാനാദനും ദുര്‍മുഖന്‍\\
ജംഭാരിവൈരിയും വീരന്‍ സമുന്നതന്‍\\
ഇങ്ങനെയുള്ളൊരു മന്ത്രികള്‍ നാല്‍വരും\\
തിങ്ങിന വന്‍പടയോടും നടന്നിതു.\\
ദുര്‍ന്നിമിത്തങ്ങളുണ്ടായതു കണ്ടവന്‍\\
തന്നകതാരിലുറച്ചു സന്നദ്ധനായ്\\
പൂര്‍വപുരദ്വാരദേശേ പുറപ്പെട്ടു\\
പാവകപുത്രനോടേറ്റോരനന്തരം\\
മര്‍ക്കടന്മാര്‍ ശിലാവൃക്ഷാചലംകൊണ്ടു\\
രക്ഷോഗണത്തെയൊടുക്കിത്തുടങ്ങിനാര്‍.\\
ചക്രഖഡ്ഗപ്രാസശക്തി ശസ്ത്രാസ്ത്രങ്ങള്‍\\
മര്‍ക്കടന്മാര്‍ക്കുമേറ്റൊക്കെ മരിക്കുന്നു.\\
ഹസ്തിവരന്മാരുമശ്വങ്ങളും ചത്തു\\
രക്തം നദികളായൊക്കെയൊലിക്കുന്നു\\
അംഭോജസംഭവനന്ദനന്‍ ജാംബവാന്‍\\
കുംഭഹനുവിനേയും ദുര്‍മുഖനേയും\\
കൊന്നു മഹാനാദനേയും സമുന്നതന്‍\\
തന്നെയും പിന്നെ പ്രഹസ്തന്‍ മഹാരഥന്‍\\
നീലനോടേറ്റുടന്‍ ദ്വന്ദ്വയുദ്ധംചെയ്തു\\
കാലപുരിപുക്കിരുന്നരുളീടിനാന്‍.\\
സേനാപതിയും പടയും മരിച്ചതു\\
മാനിയാം രാവണന്‍ കേട്ടു കോപാന്ധനായ്.
\end{verse}

%%15_ravanantepadapurappaadu

\section{രാവണന്റെ പടപ്പുറപ്പാട്}

\begin{verse}
‘ആരെയും പോരിന്നയയ്ക്കുന്നതില്ലിനി\\
നേരേ പൊരുതു ജയിക്കുന്നതുണ്ടല്ലോ.\\
നമ്മോടുകൂടെയുള്ളോര്‍കള്‍ പോന്നീടുക\\
നമ്മുടെ തേരും വരുത്തുകെ’ന്നാനവന്‍\\
വെണ്മതിപോലെ കുടയും പിടിപ്പിച്ചു\\
പൊന്മയമായൊരു തേരില്‍ക്കരേറിനാന്‍.\\
ആലവട്ടങ്ങളും വെഞ്ചാമരങ്ങളും\\
നീലത്തഴകളും മുത്തുക്കുടകളും\\
ആയിരം വാജികളെക്കൊണ്ടു പൂട്ടിയ\\
വായുവേഗം പൂണ്ട തേരില്‍ കരയേറി\\
മേരുശിഖരങ്ങള്‍പോലെ കിരീടങ്ങള്‍\\
ഹാരങ്ങളാദിയാമാഭരണങ്ങളും\\
പത്തു മുഖവുമിരുപതു കൈകളും\\
ഹസ്തങ്ങളില്‍ ചാപബാണായുധങ്ങളും\\
നീലാദ്രിപോലെ നിശാചരനായകന്‍\\
കോലാഹലത്തോടുകൂടെപ്പുറപ്പെട്ടാന്‍.\\
ലങ്കയിലുള്ള മഹാരഥന്മാരെല്ലാം\\
ശങ്കാരഹിതം പുറപ്പെട്ടാരന്നേരം.\\
മക്കളും മന്ത്രികള്‍ തമ്പിമാരും മരു-\\
മക്കളും ബന്ധുക്കളും സൈന്യപാലരും\\
തിക്കിത്തിരക്കി വടക്കുഭാഗത്തുള്ള\\
മുഖ്യമാം ഗോപുരത്തൂടെ തെരുതെരെ\\
വിക്രമമേറിയ നക്തഞ്ചരന്മാരെ-\\
യൊക്കെപ്പുരോഭുവി കണ്ടു രഘുവരന്‍\\
മന്ദസ്മിതം ചെയ്തു നേത്രാന്തസംജ്ഞയാ\\
മന്ദം വിഭീഷണന്‍ തന്നോടരുള്‍ചെയ്തു:\\
"നല്ല വീരന്മാര്‍ വരുന്നതു കാണെടോ!\\
ചൊല്ലേണമെന്നോടിവരെ യഥാഗുണം."\\
എന്നതു കേട്ടു വിഭീഷണന്‍ രാഘവന്‍-\\
തന്നോടു മന്ദസ്മിതംചെയ്തു ചൊല്ലിനാന്‍:\\
‘ബാണചാപത്തോടു ബാലാര്‍ക്കകാന്തിപൂ-\\
ണ്ടാനക്കഴുത്തില്‍ വരുന്നതകമ്പനന്‍.\\
സിംഹധ്വജംപൂണ്ട തേരില്‍ കരയേറി\\
സിംഹപരാക്രമന്‍ ബാണചാപത്തൊടും\\
വന്നവനിന്ദ്രജിത്താകിയ രാവണ-\\
നന്ദനന്‍ നമ്മെ മുന്നം ജയിച്ചാനവന്‍.\\
ആയോധനത്തിന്നു ബാണചാപങ്ങള്‍ പൂ-\\
ണ്ടായതമായൊരു തേരില്‍ കരയേറി\\
കായം വളര്‍ന്നു വിഭൂഷണം പൂണ്ടതി-\\
കായന്‍ വരുന്നതു രാവണന്തന്മകന്‍.\\
പൊന്നണിഞ്ഞാനക്കഴുത്തില്‍ വരുന്നവ-\\
നുന്നതനേറ്റം മഹോദരന്‍ മന്നവ!\\
വാജിമേലേറിപ്പരിഘം തിരിപ്പവ-\\
നാജിശൂരേന്ദ്രന്‍ വിശാലന്‍ നരാന്തകന്‍.\\
വെള്ളെരുതിന്‍ മുകളേറി ത്രിശൂലവും\\
തുള്ളിച്ചിരിക്കുന്നവന്‍ ത്രിശിരസ്സല്ലോ\\
രാവണന്‍തന്മകന്‍ മറ്റേതിനങ്ങേതു\\
ദേവാന്തകന്‍ തേരില്‍ വന്നതു മന്നവ!\\
കുംഭകര്‍ണാത്മജന്‍ കുംഭനങ്ങേതവന്‍\\
തമ്പി നികുംഭന്‍ പിരിഘായുധനല്ലോ.\\
ദേവകുലാന്തകനാകിയ രാവണ-\\
നേവരോടും നമ്മെ വെല്‍വാന്‍ പുറപ്പെട്ടു.’\\
ഇത്ഥം വിഭീഷണന്‍ ചൊന്നതു കേട്ടതി-\\
നുത്തരം രാഘവന്‍ താനുമരുള്‍ചെയ്തു:\\
‘യുദ്ധേ ദശമുഖനെക്കൊലചെയ്തുടന്‍\\
ചിത്തകോപം കളഞ്ഞീടുവനിന്നു ഞാന്‍’\\
എന്നരുളിച്ചെയ്തു നിന്നരുളുന്നേരം\\
വന്ന പടയോടു ചൊന്നാന്‍ ദശാസ്യനും:\\
‘എല്ലാവരും നാമൊഴിച്ചു പോന്നാലവര്‍\\
ചെല്ലുമകത്തുകടന്നൊരു ഭാഗമേ\\
പാര്‍ത്തു ശത്രുക്കള്‍ കടന്നുകൊള്ളും മുന്നേ\\
കാത്തുകൊള്‍വിന്‍ നിങ്ങള്‍ ചെന്നു ലങ്കാപുരം.\\
യുദ്ധത്തിനിന്നു ഞാന്‍ പോരുമിവരോടു\\
ശക്തിയില്ലായ്കയുമില്ലിതിനേതുമേ.’\\
ഏവം നിയോഗിച്ചനേരം നിശാചര-\\
രേവരും ചെന്നു ലങ്കാപുരം മേവിനാര്‍.\\
വൃന്ദാരകാരാതി രാവണന്‍ വാനര-\\
വൃന്ദത്തെയെയ്തെയ്തു തള്ളിവിട്ടീടിനാന്‍.\\
വാനരേന്ദ്രന്മാരഭയം തരികെന്നു\\
മാനവേന്ദ്രന്‍ കാല്‍ക്കല്‍ വീണിരന്നീടിനാര്‍.\\
വില്ലും ശരങ്ങളുമാശു കൈക്കൊണ്ടു കൗ-\\
സല്യാതനയനും പോരിന്നൊരുമിച്ചാന്‍.\\
‘വമ്പനായുള്ളോരിവനോടു പോരിനു\\
മുമ്പിലടിയനനുഗ്രഹം നല്കണം.’\\
എന്നു സൗമിത്രിയും ചെന്നിരന്നീടിനാന്‍\\
മന്നവന്‍ താനുമരുള്‍ചെയ്തിതന്നേരം:\\
‘വൃത്രാരിയും പോരില്‍ വിത്രസ്തനായ് വരും\\
നക്തഞ്ചരേന്ദ്രനോടേറ്റാലറിക നീ\\
മായയുമുണ്ടു നിശാചരര്‍ക്കേറ്റവും\\
ന്യായവുമില്ലിവര്‍ക്കാര്‍ക്കുമൊരിക്കലും\\
ചന്ദ്രചൂഡപ്രിയനാകയുമുണ്ടവന്‍\\
ചന്ദ്രഹാസാഖ്യമാം വാളുമുണ്ടായുധം\\
എല്ലാം നിരൂപിച്ചു ചിത്തമുറപ്പിച്ചു\\
ചെല്ലേണമല്ലോ കലഹത്തിനെ’ന്നെല്ലാം.\\
ശിക്ഷിച്ചരുള്‍ചെയ്തയച്ചോരനന്തരം\\
ലക്ഷ്മണനും തൊഴുതാശു പിന്‍വാങ്ങിനാന്‍.\\
ജാനകീചോരനെക്കണ്ടൊരു നേരത്തു\\
വാനരനായകനാകിയ മാരുതി\\
തേര്‍ത്തടം തന്നില്‍ കുതിച്ചു വീണീടിനാ-\\
നാര്‍ത്തനായ് വന്നു നിശാചരനാഥനും.\\
ദക്ഷിണഹസ്തവുമോങ്ങിപ്പറഞ്ഞിതു\\
രക്ഷോവരനോടു മാരുതപുത്രനും:\\
‘നിര്‍ജരന്മാരെയും താപസന്മാരെയും\\
സജ്ജനമായ മറ്റുള്ള ജനത്തെയും\\
നിത്യമുപദ്രവിക്കുന്ന നിനക്കു വ-\\
ന്നെത്തുമാപത്തു കപികുലത്താലെടോ!\\
നിന്നെയടിച്ചു കൊല്‍വാന്‍വന്നു നില്ക്കുന്നൊ-\\
രെന്നെയൊഴിച്ചുകൊള്‍ വീരനെന്നാകില്‍ നീ\\
വിക്രമമേറിയ നിന്നുടെ പുത്രനാ-\\
മക്ഷകുമാരനെക്കൊന്നതു ഞാനെടോ!’\\
എന്നു പറഞ്ഞൊന്നടിച്ചാന്‍ കപീന്ദ്രനും\\
നന്നായ് വിറച്ചു വീണാന്‍ ദശകണ്ഠനും.\\
പിന്നെയുണര്‍ന്നു ചൊന്നാ’നിവിടേക്കിന്നു\\
വന്ന കപികളില്‍ നല്ലനല്ലോ ഭവാന്‍.’\\
‘നന്മയെന്തായതെനിക്കിന്നിതുകൊണ്ടു\\
നമ്മുടെ തല്ലുകൊണ്ടാല്‍ മറ്റൊരുവരും\\
മൃത്യുവരാതെ ജീവിപ്പവരില്ലല്ലോ\\
മൃത്യുവന്നീല നിനക്കതുകൊണ്ടു ഞാന്‍\\
എത്രയും ദുര്‍ബലനെന്നുവന്നൂ നമ്മി-\\
ലിത്തിരി നേരമിന്നും പൊരുതീടണം.’\\
എന്ന നേരത്തൊന്നടിച്ചാന്‍ ദശാനനന്‍\\
പിന്നെ മോഹിച്ചു വീണാന്‍ കപിശ്രേഷ്ഠനും.\\
നീലനന്നേരം കുതികൊണ്ടു രാവണ-\\
ന്മേലെ കരേറിക്കിരീടങ്ങള്‍ പത്തിലും\\
വില്ലുതന്മേലും കൊടിമരത്തിന്മേലു-\\
മുല്ലാസമോടു മകുടങ്ങള്‍ പത്തിലും\\
ചാടി ക്രമേണ നൃത്തം തുടങ്ങീടീനാന്‍;\\
പാടിത്തുടങ്ങിനാന്‍ നാരദനും തദാ.\\
പാവകാസ്ത്രംകൊണ്ടു പാവകപുത്രനെ\\
രാവണനെയ്തുടന്‍ തള്ളി വിട്ടീടിനാന്‍.\\
തല്‍ക്ഷണേ കോപിച്ചു ലക്ഷ്മണന്‍ വേഗേന\\
രക്ഷോവരനെ ചെറുത്താനതു നേരം.\\
ബാണഗണത്തെ വര്‍ഷിച്ചാരിരുവരും\\
കാണരുതാതെ ചമഞ്ഞിതു പോര്‍ക്കളം.\\
വില്ലു മുറിച്ചുകളഞ്ഞിതു ലക്ഷ്മണ-\\
നല്ലല്‍ മുഴുത്തുനിന്നൂ ദശകണ്ഠനും.\\
പിന്നെ മയന്‍ കൊടുത്തോരു വേല്‍ സൗമിത്രി-\\
തന്നുടെ മാറിലാമ്മാറു ചാട്ടീടിനാന്‍.\\
അസ്ത്രങ്ങള്‍കൊണ്ടു തടുക്കരുതാഞ്ഞു സൗ-\\
മിത്രിയും ശാക്തിയേറ്റാശു വീണീടിനാന്‍.\\
ആടലായ് വീണ കുമാരനെച്ചെന്നെടു-\\
ത്തീടുവാനാശു ഭാവിച്ചു ദശാനനന്‍.\\
കൈലാസശൈലമെടുത്ത ദശാസ്യനു\\
ബാലശരീരമിളക്കരുതാഞ്ഞിതു.\\
രാഘവന്‍തന്നുടെ ഗൗരവമോര്‍ത്തതി-\\
ലാഘവം പൂണ്ടിതു രാവണവീരനും.\\
കണ്ടുനില്ക്കുന്നൊരു മാരുതപുത്രനും\\
മണ്ടിയണഞ്ഞൊന്നടിച്ചാന്‍ ദശാസ്യനെ.\\
ചോരയും ഛര്‍ദിച്ചു തേരില്‍ വീണാനവന്‍.\\
മാരുതിതാനും കുമാരനെ തല്‍ക്ഷണേ\\
പുഷ്പസമാനമെടുത്തുകൊണ്ടാദരാല്‍\\
ചില്‍പ്പുരുഷന്‍മുമ്പില്‍ വെച്ചുവണങ്ങിനാന്‍.\\
മാറും പിരിഞ്ഞു ദശമുഖന്‍ കയ്യിലാ-\\
മ്മാറുപുക്കു മയദത്തമാം ശക്തിയും.\\
ത്രൈലോക്യനായകനാകിയ രാമനും\\
പൗലസ്ത്യനോടു യുദ്ധം തുടങ്ങീടിനാന്‍.\\
ഗന്ധവാഹാത്മജന്‍ വന്ദിച്ചു ചൊല്ലിനാന്‍:\\
‘പംക്തിമുഖനോടു യുദ്ധത്തിനെന്നുടെ\\
കണ്ഠമേറിക്കൊണ്ടു നിന്നരുളിക്കൊള്‍ക\\
കുണ്ഠതയെന്നിയേ കൊല്ക ദശാസ്യനെ.’\\
മാരുതി ചൊന്നതു കേട്ടു രഘുത്തമ-\\
നാരുഹ്യ തല്‍കണ്ഠദേശേ വിളങ്ങിനാന്‍.\\
ചൊന്നാന്‍ ദശാനനന്‍തന്നോടു രാഘവന്‍:\\
‘നിന്നെയടുത്തു കാണ്മാന്‍ കൊതിച്ചേന്‍ തുലോം.\\
ഇന്നതിന്നാശു യോഗം വന്നിതാകയാല്‍\\
നിന്നെയും നിന്നോടുകൂടെ വന്നോരെയും\\
കൊന്നു ജഗത്ത്രയം പാലിച്ചുകൊള്ളുവ-\\
നെന്നുടെ മുമ്പിലരക്ഷണം നില്ലു നീ.’\\
എന്നരുള്‍ചെയ്തു ശസ്ത്രാസ്ത്രങ്ങള്‍ തൂകിനാ-\\
നൊന്നിനൊന്നൊപ്പമെയ്താന്‍ ദശവക്ത്രനും\\
ഘോരമായ് വന്നിതു പോരുമന്നേരത്തു\\
വാരാന്നിധിയുമിളകി മറിയുന്നു.\\
മാരുതിതന്നെയുമെയ്തു മുറിച്ചിതു\\
ശൂരനായോരു നിശാചരനായകന്‍\\
ശ്രീരാമദേവനും കോപം മുഴുത്തതി-\\
ധീരതകൈക്കൊണ്ടെടുത്തൊരു സായകം\\
രക്ഷോവരനുടെ വക്ഷപ്രദേശത്തെ\\
ലക്ഷ്യമാക്കി പ്രയോഗിച്ചാനതിദ്രുതം\\
ആലസ്യമായിതു ബാണമേറ്റന്നേരം\\
പൗലസ്ത്യചാപവും വീണിതു ഭൂതലേ.\\
നക്തഞ്ചരാധിപനായ ദശാസ്യനു\\
ശക്തിക്ഷയം കണ്ടു സത്വരം രാഘവന്‍\\
തേരും കൊടിയും കുടയും കുതിരയും\\
ചാരുകിരീടങ്ങളും കളഞ്ഞീടിനാന്‍\\
സാരഥിതന്നെയും കൊന്നുകളഞ്ഞള-\\
വാരൂഢതാപേന നിന്നു ദശാസ്യനും.\\
രാമനും രാവണന്‍തന്നോടരുള്‍ചെയ്താ-\\
‘നാമയം പാരം നിനക്കുണ്ടു മാനസേ\\
പോയാലുമിന്നു ഭയപ്പെടായ്കേതുമേ\\
നീയിനി ലങ്കയില്‍ച്ചെന്നങ്ങിരുന്നാലും\\
ആയുധവാഹനത്തോടൊരുമ്പെട്ടുകൊ-\\
ണ്ടായോധനത്തിനു നാളെ വരേണം നീ.’\\
കാകുല്‍സ്ഥവാക്കുകള്‍ കേട്ടു ഭയപ്പെട്ടു\\
വേഗത്തിലങ്ങു നടന്നു ദശാനനന്‍.\\
രാഘവാസ്ത്രം തുടരെത്തുടര്‍ന്നുണ്ടെന്നൊ-\\
രാകുലം പൂണ്ടു തിരിഞ്ഞുനോക്കിത്തുലോം\\
വേപഥുഗാത്രനായ് മന്ദിരം പ്രാപിച്ചു\\
താപമുണ്ടായതു ചിന്തിച്ചു മേവിനാന്‍.
\end{verse}

%%16_kumbhakarnanteneethivaakyam

\section{കുംഭകര്‍ണന്റെ നീതിവാക്യം}

\begin{verse}
മാനവേന്ദ്രന്‍ പിന്നെ ലക്ഷ്മണന്‍ തന്നെയും\\
വാനരരാജനാമര്‍ക്കാത്മജനെയും\\
രാവണബാണവിദാരിതന്മാരായ\\
പാവകപുത്രാദി വാനരന്മാരെയും\\
സിദ്ധൗഷധം കൊണ്ടു രക്ഷിച്ചു തന്നുടെ\\
സിദ്ധാന്തമെല്ലാമരുള്‍ചെയ്തു മേവിനാന്‍.\\
രാത്രിഞ്ചരേന്ദ്രനും ഭൃത്യജനത്തോടു\\
പേര്‍ത്തും നിജാര്‍ത്തികളോര്‍ത്തു ചൊല്ലീടിനാന്‍:\\
‘നമ്മുടെ വീര്യബലങ്ങളും കീര്‍ത്തിയും\\
നന്മയുമര്‍ത്ഥപുരുഷകാരാദിയും\\
നഷ്ടമായ് വന്നിതൊടുങ്ങീ സുകൃതവും\\
കഷ്ടകാലം നമുക്കാഗതം നിശ്ചയം.\\
വേധാവു താനുമനാരണ്യഭൂപനും\\
വേദവതിയും മഹാനന്ദികേശനും\\
രംഭയും പിന്നെ നളകൂബരാദികളും\\
ജംഭാരിമുമ്പാം നിലിമ്പവരന്മാരും\\
കുംഭോത്ഭവാദികളായ മുനികളും\\
ശംഭുപ്രണയിനിയാകിയ ദേവിയും\\
പുഷ്ടതപോബലം പൂണ്ടു പാതിവ്രത്യ-\\
നിഷ്ഠയോടെ മരുവുന്ന സതികളും\\
സത്യമായ് ചൊല്ലിയ ശാപവചസ്സുകള്‍\\
മിഥ്യയായ് വന്നുകൂടായെന്നു നിര്‍ണയം.\\
ചിന്തിച്ചു കാണ്മിന്‍ നമുക്കിനിയും പുന-\\
രെന്തോന്നു നല്ലൂ ജയിച്ചുകൊള്‍വാനഹോ!\\
കാലാരിതുല്യനാകും കുംഭകര്‍ണനെ-\\
ക്കാലംകളയാതുണര്‍ത്തുക നിങ്ങള്‍പോയ്.\\
ആറുമാസം കഴിഞ്ഞെന്നിയുണര്‍ത്തീടു-\\
മാറില്ലുറങ്ങിത്തുടങ്ങീട്ടവനുമി-\\
ന്നൊന്‍പതു നാളേ കഴിഞ്ഞതുള്ളൂ നിങ്ങ-\\
ളന്‍പോടുണര്‍ത്തുവിന്‍ വല്ല പ്രകാരവും.’\\
രാക്ഷസരാജനിയോഗേന ചെന്നോരോ\\
രാക്ഷസരെല്ലാമൊരുമ്പെട്ടുണര്‍ത്തുവാന്‍.\\
ആനകദുന്ദുഭിമുഖ്യവാദ്യങ്ങളു-\\
മാന തേര്‍ കാലാള്‍ കുതിരപ്പടകളും\\
കുംഭകര്‍ണോരസി പാഞ്ഞുമാര്‍ത്തും ജഗല്‍-\\
ക്കമ്പം വരുത്തിനാരെന്തൊരു വിസ്മയം!\\
കുംഭകര്‍ണശ്രവണാന്തരേ പിന്നെയും\\
കുംഭിവരന്മാരെക്കൊണ്ടു നാസാരന്ധ്ര-\\
സംഭൂതരോമം പിടിച്ചു വലിപ്പിച്ചും\\
തുമ്പിക്കരമറ്റലറിയുമാനകള്‍\\
ജംഭാരിവൈരിക്കു കമ്പമില്ലേതുമേ.\\
ജൃംഭാസമാരംഭമോടുമുണര്‍ന്നിതു\\
സംഭ്രമിച്ചോടിനാരാശരവീരരും.\\
കുംഭസഹസ്രം നിറച്ചുള്ള മദ്യവും\\
കുംഭസഹസ്രം നിറച്ചുള്ള രക്തവും\\
സംഭോജ്യമന്നവും കുന്നുപോലേ കണ്ടൊ-\\
രിമ്പം കലര്‍ന്നെഴുന്നേറ്റിരുന്നീടിനാന്‍.\\
ക്രവ്യങ്ങളാദിയായ് മറ്റുപജീവന-\\
ദ്രവ്യമെല്ലാം ഭുജിച്ചാനന്ദചിത്തനായ്\\
ശുദ്ധാചമനവും ചെയ്തിരിക്കും വിധൗ\\
ഭൃത്യജനങ്ങളും വന്നു വണങ്ങിനാര്‍.\\
കാര്യങ്ങളെല്ലാമറിയിച്ചുണര്‍ത്തിയ\\
കാരണവും കേട്ടു പംക്തികണ്ഠാനുജന്‍\\
‘എങ്കിലോ, വൈരികളെക്കൊലചെയ്തു ഞാന്‍\\
സങ്കടം തീര്‍ത്തുവരുവ’നെന്നിങ്ങനെ\\
ചൊല്ലിപ്പുറപ്പെട്ട നേരം മഹോദരന്‍\\
മെല്ലെത്തൊഴുതു പറഞ്ഞാനതുനേരം:\\
‘ജ്യേഷ്ഠനെക്കണ്ടു തൊഴുതു വിടവാങ്ങി\\
വാട്ടം വരാതെ പൊയ്ക്കൊള്ളുക നല്ലതും.’\\
ഏവം മഹോദരന്‍ ചൊന്നതു കേട്ടവന്‍\\
രാവണന്‍ തന്നെയും ചെന്നു വണങ്ങിനാന്‍.\\
ഗാഢമായാലിംഗനം ചെയ്തിരുത്തിനാ-\\
നൂഢമോദം നിജസോദരന്‍തന്നെയും.\\
‘ചിത്തേ ധരിച്ചതില്ലോര്‍ക്ക നീ കാര്യങ്ങള്‍\\
വൃത്താന്തമെങ്കിലോ കേട്ടാലുമിന്നെടോ!\\
സോദരിതന്നുടെ നാസാകുചങ്ങളെ-\\
ച്ഛേദിച്ചതിന്നു ഞാന്‍ ജാനകീദേവിയെ\\
ശ്രീരാമലക്ഷ്മണന്മാരറിയാതെ ക-\\
ണ്ടാരാമസീമ്നി കൊണ്ടന്നു വെച്ചീടിനേന്‍.\\
വാരിധിയില്‍ ചിറകെട്ടിക്കടന്നവന്‍\\
പോരിന്നു വാനരസേനയുമായ് വന്നു\\
കൊന്നാന്‍ പ്രഹസ്താദികളെപ്പലരെയു-\\
മെന്നെയുമെയ്തു മുറിച്ചാന്‍ ജിതശ്രമം\\
കൊല്ലാതെ കൊന്നയച്ചാനതുകാരണ-\\
മല്ലല്‍ മുഴുത്തു ഞാന്‍ നിന്നെയുണര്‍ത്തിനേന്‍.\\
മാനവന്മാരെയും വാനരന്മാരെയും\\
കൊന്നു നീയെന്നെ രക്ഷിച്ചു കൊള്ളേണമേ!’\\
എന്നതും കേട്ടു ചൊന്നാന്‍ കുംഭകര്‍ണനും.:\\
നന്നു നന്നെത്രയും നല്ലതേ നല്ലുകേള്‍.\\
നല്ലതും തിയ്യതും താനറിയാതവന്‍\\
നല്ലതറിഞ്ഞു ചൊല്ലുന്നവര്‍ ചൊല്ലുകള്‍\\
നല്ലവണ്ണം കേട്ടുകൊള്ളുകിലും നന്ന-\\
തല്ലാതവര്‍ക്കുണ്ടോ നല്ലതുണ്ടാകുന്നു?\\
സീതയെ രാമനു നല്‍കുകെ’ന്നിങ്ങനെ\\
സോദരന്‍ ചൊന്നാനതിന്നു കോപിച്ചു നീ\\
ആട്ടിക്കളഞ്ഞതു നന്നു നന്നോര്‍ത്തുകാണ്‍,\\
നാട്ടില്‍ നിന്നാശു വാങ്ങി ഗുണമൊക്കവേ.\\
നല്ലവണ്ണം വരുംകാലമല്ലെന്നതും\\
ചൊല്ലാമതുകൊണ്ടതും കുറ്റമല്ലെടോ!\\
നല്ലതൊരുത്തരാലും വരുത്താവത-\\
ല്ലല്ലല്‍ വരുത്തുമാപത്തണയുന്ന നാള്‍.\\
കാലദേശാവസ്ഥകളും നയങ്ങളും\\
മൂലവും വൈരികള്‍ കാലവും വീര്യവും\\
ശത്രുമിത്രങ്ങളും മധ്യസ്ഥപക്ഷവു-\\
മര്‍ത്ഥപുരുഷകാരാദി ഭേദങ്ങളും\\
നാലുപായങ്ങളുമാറു നയങ്ങളും\\
മേലില്‍ വരുന്നതുമൊക്കെ നിരൂപിച്ചു\\
പത്ഥ്യം പറയുമമാത്യനുണ്ടെങ്കിലോ\\
ഭര്‍ത്തൃസൗഖ്യം വരും കീര്‍ത്തിയും വര്‍ധിക്കും.\\
ഇങ്ങനെയുള്ളോരമാത്യധര്‍മം വെടി-\\
ഞ്ഞെങ്ങനെ രാജാവിനിഷ്ടമെന്നാലതു\\
കര്‍ണസുഖം വരുമാറു പറഞ്ഞുകൊ-\\
ണ്ടന്വഹമാത്മാഭിമാനവും ഭാവിച്ചു\\
മൂലവിനാശം വരുമാറു നിത്യവും\\
മൂഢരായുള്ളോരമാത്യജനങ്ങളില്‍\\
നല്ലതു കാകോളമെന്നതു ചൊല്ലുവോ-\\
രല്ലല്‍ വിഷമുണ്ടവര്‍ക്കെന്നിയില്ലല്ലോ.\\
മൂഢരാം മന്ത്രികള്‍ ചൊല്ലു കേട്ടീടുകില്‍\\
നാടുമായുസ്സും കുലവും നശിച്ചുപോം.\\
നാദഭേദം കേട്ടു മോഹിച്ചുചെന്നുചേര്‍-\\
ന്നാധിമുഴുത്തു മരിക്കും മൃഗകുലം.\\
അഗ്നിയെക്കണ്ടു മോഹിച്ചു ശലഭങ്ങള്‍\\
മഗ്നരായഗ്നിയില്‍ വീണു മരിക്കുന്നു.\\
മത്സ്യങ്ങളും രസത്തിങ്കല്‍ മോഹിച്ചു ചെ-\\
ന്നത്തല്‍പ്പെടുന്നു ബളിശം ഗ്രസിക്കയാല്‍.\\
ആഗ്രഹമൊന്നിങ്കലേറിയാലാപത്തു\\
പോക്കുവാനാവതല്ലാതവണ്ണം വരും.\\
നമ്മുടെ വംശത്തിനും നല്ല നാട്ടിനു-\\
മുന്മൂലനാശം വരുത്തുവാനായല്ലോ\\
ജാനകിതന്നിലൊരാശയുണ്ടായതും\\
ഞാനറിഞ്ഞേനതു രാത്രിഞ്ചരാധിപ!\\
ഇന്ദ്രിയങ്ങള്‍ക്കു വശനായിരിപ്പവ-\\
നെന്നുമാപത്തൊഴിഞ്ഞില്ലെന്നു നിര്‍ണയം.\\
ഇന്ദ്രിയഗ്രാമം ജയിച്ചിരിക്കുന്നവ-\\
നൊന്നുകൊണ്ടും വരാ നൂനമാപത്തുകള്‍.\\
നല്ലതല്ലെന്നങ്ങറിഞ്ഞിരിക്കെബ്ബലാല്‍\\
ചെല്ലുമൊന്നിങ്കലൊരുത്തനഭിരുചി.\\
പൂര്‍വജന്മാര്‍ജിതവാസനയാലതി-\\
നാവതല്ലേതുമതിന്‍ വശനായ് വരും.\\
എന്നാലതിങ്കല്‍ നിന്നാശു മനസ്സിനെ-\\
ത്തന്നുടെ ശാസ്ത്രവിവേകോപദേശങ്ങള്‍-\\
കൊണ്ടു വിധേയമാക്കിക്കൊണ്ടിരിപ്പവ-\\
നുണ്ടോ ജഗത്തിങ്കലാരാനുമോര്‍ക്ക നീ.\\
മുന്നം വിചാരകാലേ ഞാന്‍ ഭവാനോടു-\\
തന്നെ പറഞ്ഞതില്ലേ ഭവിഷ്യല്‍ഫലം?\\
ഇപ്പോളുപഗതമായ് വന്നിതീശ്വര-\\
കല്പിതമാര്‍ക്കും തടുക്കാവതല്ലല്ലോ.\\
മാനുഷനല്ല രാമന്‍ പുരുഷോത്തമന്‍\\
നാനാജഗന്മയന്‍ നാരായണന്‍ പരന്‍\\
സീതയാകുന്നതു യോഗമായാദേവി\\
ചേതസി നീ ധരിച്ചീടുകെന്നിങ്ങനെ\\
നിന്നോടുതന്നെ പറഞ്ഞുതന്നീലയോ\\
മന്നവ! മുന്നമേയെന്തതോരാഞ്ഞതും?\\
ഞാനൊരു നാള്‍ വിശാലായാം യഥാസുഖം\\
കാനനാന്തേ നരനാരായണാശ്രമേ\\
വാഴുന്ന നേരത്തു നാരദനെപ്പരി-\\
തോഷേണ കണ്ടു നമസ്കരിച്ചീടിനേന്‍.\\
‘ഏതൊരു ദിക്കില്‍ നിന്നാഗതനായിതെ-\\
ന്നാദരവോടരുള്‍ ചെയ്ക മഹാമുനേ!\\
എന്തൊരു വൃത്താന്തമുള്ളൂ ജഗത്തിങ്ക-\\
ലന്തരം കൂടാതരുള്‍ചെയ്ക’യെന്നെല്ലാം\\
ചോദിച്ചനേരത്തു നാരദനെന്നോടു\\
സാദരം ചൊന്നാനുദന്തങ്ങളൊക്കവേ:\\
‘രാവണപീഡിതന്മാരായ് ചമഞ്ഞൊരു\\
ദേവകളും മുനിമാരുമൊരുമിച്ചു\\
ദേവദേവേശനാം വിഷ്ണുഭഗവാനെ-\\
സ്സേവിച്ചുണര്‍ത്തിച്ചു സങ്കടമൊക്കവേ.\\
ത്രൈലോക്യകണ്ടകനാകിയ രാവണന്‍\\
പൗലസ്ത്യപുത്രനതീവ ദുഷ്ടന്‍ ഖലന്‍\\
ഞങ്ങളെയെല്ലാമുപദ്രവിച്ചീടുന്നി-\\
തെങ്ങുമിരിക്കരുതാതെ ചമഞ്ഞിതു.\\
മര്‍ത്ത്യനാലെന്നിയേ മൃത്യുവില്ലെന്നതു-\\
മുക്തം വിരിഞ്ചനാല്‍ മുന്നമേ കല്പിതം.\\
മര്‍ത്ത്യനായ്ത്തന്നെ പിറന്നു ഭവാനിനി-\\
സ്സത്യധര്‍മങ്ങളെ രക്ഷിക്കവേണമേ!’\\
ഇത്ഥമുണര്‍ത്തിച്ച നേരം മുകുന്ദനും\\
ചിത്തകാരുണ്യം കലര്‍ന്നരുളിച്ചെയ്തു:\\
‘പൃഥ്വിയില്‍ ഞാനയോദ്ധ്യായാം ദശരഥ-\\
പുത്രനായ് വന്നു പിറന്നിനിസ്സത്വരം\\
നക്തഞ്ചരാധിപന്‍തന്നെയും നിഗ്രഹി-\\
ച്ചത്തല്‍ തീര്‍ത്തീടുവനിത്രിലോകത്തിങ്കല്‍.\\
സത്യസങ്കല്പനാമീശ്വരന്‍ തന്നുടെ\\
ശക്തിയോടും കൂടി രാമനായ് വന്നതും\\
നിങ്ങളെയെല്ലാമൊടുക്കുമവനിനി\\
മംഗലം വന്നുകൂടും ജഗത്തിങ്കലും.’\\
എന്നരുള്‍ചെയ്തു മറഞ്ഞു മഹാമുനി\\
നന്നായ് നിരൂപിച്ചുകൊള്‍ക നീ മാനസേ.\\
രാമന്‍ പരബ്രഹ്മമായ സനാതനന്‍\\
കോമളനിന്ദീവരദളശ്യാമളന്‍.\\
മായാമനുഷ്യവേഷം പൂണ്ട രാമനെ-\\
ക്കായേന വാചാ മനസാ ഭജിക്ക നീ.\\
ഭക്തി കണ്ടാല്‍ പ്രസാദിക്കും രഘൂത്തമന്‍\\
ഭക്തിയല്ലോ മഹാജ്ഞാനമാതാവെടോ!\\
ഭക്തിയല്ലോ സതാം മോക്ഷപ്രദായിനി\\
ഭക്തിഹീനന്മാര്‍ക്കു കര്‍മവും നിഷ്ഫലം.\\
സംഖ്യയില്ലാതോളമുണ്ടവതാരങ്ങള്‍\\
പങ്കജനേത്രനാം വിഷ്ണുവിനെങ്കിലും\\
സംഖ്യാവതാം മതം ചൊല്ലുവന്‍ നിന്നുടെ\\
ശങ്കയെല്ലാമകലെക്കളഞ്ഞീടുവാന്‍.\\
രാമാവതാരസമമല്ലതൊന്നുമേ\\
നാമജപത്തിനാലേ വരും മോക്ഷവും\\
ജ്ഞാനസ്വരൂപനാകുന്ന ശിവന്‍ പരന്‍\\
മാനുഷാകാരനാം രാമനാകുന്നതും\\
താരകബ്രഹ്മമെന്നത്രെ ചൊല്ലുന്നതും\\
ശ്രീരാമദേവനെത്തന്നെ ഭജിക്ക നീ.\\
രാമനെത്തന്നെ ഭജിച്ചു വിദ്വജ്ജന-\\
മാമയം നല്കുന്ന സംസാരസാഗരം.\\
ലംഘിച്ചു രാമപാദത്തെയും പ്രാപിച്ചു\\
സങ്കടം തീര്‍ത്തുകൊള്ളുന്നിതു സന്തതം.\\
‘ബുദ്ധതത്ത്വന്മാര്‍ നിരന്തരം രാമനെ-\\
ച്ചിത്താംബുജത്തിങ്കല്‍ നിത്യവും ധ്യാനിച്ചു\\
തച്ചരിത്രങ്ങളും ചൊല്ലി നാമങ്ങളു-\\
മുച്ചരിച്ചാത്മാനമാത്മനാ കണ്ടുക-\\
ണ്ടച്യുതനോടു സായുജ്യവും പ്രാപിച്ചു\\
നിശ്ചലാനന്ദേ ലയിക്കുന്നിതന്വഹം.\\
മായാവിമോഹങ്ങളെല്ലാം കളഞ്ഞുടന്‍\\
നീയും ഭജിച്ചുകൊള്‍കാനന്ദമൂര്‍ത്തിയെ.’
\end{verse}

%%17_kumbhakarnavadham

\section{കുംഭകര്‍ണവധം}

\begin{verse}
സോദരനേവം പറഞ്ഞതു കേട്ടതി-\\
ക്രോധം മുഴുത്ത ദശാസ്യനും ചൊല്ലിനാന്‍:\\
ജ്ഞാനോപദേശമെനിക്കു ചെയ്വാനല്ല\\
ഞാനിന്നുണര്‍ത്തിവരുത്തി, യഥാസുഖം\\
നിദ്രയെസ്സേവിച്ചുകൊള്‍ക, നീയെത്രയും\\
ബുദ്ധിമാനെന്നതുമിന്നറിഞ്ഞേനഹം.\\
വേദശാസ്ത്രങ്ങളും കേട്ടുകൊള്ളാമിനി\\
ഖേദമകന്നു സുഖിച്ചു വാഴുന്ന നാള്‍.\\
ആമെങ്കിലാശു ചെന്നായോധനം ചെയ്തു\\
രാമാദികളേ വധിച്ചു വരിക നീ.’\\
അഗ്രജന്‍ വാക്കുകളിത്തരം കേട്ടള-\\
വുഗ്രനാം കുംഭകര്‍ണന്‍ നടന്നീടിനാന്‍.\\
വ്യഗ്രവും കൈവിട്ടു യുദ്ധേ രഘൂത്തമന്‍\\
നിഗ്രഹിച്ചാല്‍ വരും മോക്ഷമെന്നോര്‍ത്തവന്‍\\
പ്രാകാരവും കടന്നുത്തുംഗ ശൈലരാ-\\
ജാകാരമോടലറിക്കൊണ്ടതിദ്രുതം.\\
ആയിരം ഭാരമിരുമ്പുകൊണ്ടുള്ള ത-\\
ന്നായുധമായുള്ള ശൂലവും കൈക്കൊണ്ടു\\
വാനരസേനയില്‍ പുക്കോരു നേരത്തു\\
വാനരവീരരെല്ലാവരുമോടിനാര്‍.\\
കുംഭകര്‍ണന്‍തന്‍ വരവുകണ്ടാകുലാല്‍\\
സംഭ്രമം പൂണ്ടു വിഭീഷണന്‍ തന്നോടു\\
‘വന്‍പുള്ള രാക്ഷസനേവനിവന്‍ പറ-\\
കംബരത്തോളമുയരമുണ്ടത്ഭുതം!’\\
ഇത്ഥം രഘൂത്തമന്‍ ചോദിച്ചളവതി-\\
നുത്തരമാശു വിഭീഷണന്‍ ചൊല്ലിനാന്‍:\\
‘രാവണസോദരന്‍ കുംഭകര്‍ണന്‍ മമ\\
പൂര്‍വജനെത്രയും ശക്തിമാന്‍ ബുദ്ധിമാന്‍\\
ദേവകുലാന്തകന്‍ നിദ്രാവശനിവ-\\
നാവതില്ലാര്‍ക്കുമേറ്റാല്‍ ജയിച്ചീടുവാന്‍.’\\
തച്ചരിത്രങ്ങളെല്ലാമറിയിച്ചു ചെ-\\
ന്നിച്ഛയാ പൂര്‍വജന്‍ കാല്ക്കല്‍ വീണീടിനാന്‍.\\
‘ഭ്രാതാ വിഭീഷണന്‍ ഞാന്‍ ഭവദ്ഭക്തിമാന്‍\\
പ്രീതിപൂണ്ടെന്നെയനുഗ്രഹിക്കേണമേ!\\
സീതയെ നല്കുക രാഘവനെന്നു ഞാ-\\
നാദരപൂര്‍വമാവോളമപേക്ഷിച്ചേന്‍.\\
ഖഡ്ഗവും കൈക്കൊണ്ടു നിഗ്രഹിച്ചീടുവാ-\\
നുഗ്രതയോടുമടുത്തതു കണ്ടു ഞാന്‍\\
ഭീതനായ് നാലമാത്യന്മാരുമായ് പോന്നു\\
സീതാപതിയെശ്ശരണമായ് പ്രാപിച്ചേന്‍.’\\
ഇത്ഥം വിഭീഷണവാക്കുകള്‍ കേട്ടവന്‍\\
ചിത്തം കുളുര്‍ത്തു പുണര്‍ന്നാനനുജനെ.\\
പിന്നെപ്പുറത്തു തലോടിപ്പറഞ്ഞിതു:\\
‘ധന്യനല്ലോ ഭവാനില്ല കില്ലേതുമേ.\\
ജീവിച്ചിരിക്ക പലകാലമൂഴിയില്‍\\
സേവിച്ചുകൊള്ളുക രാമപാദാംബുജം.\\
നമ്മുടെ വംശത്തെ രക്ഷിപ്പതിന്നു നീ\\
നിര്‍മലന്‍ ഭാഗവതോത്തമനെത്രയും.\\
നാരായണപ്രിയനെത്രയും നീയെന്നു\\
നാരദന്‍തന്നെ പറഞ്ഞുകേട്ടേനഹം.\\
മായാമയമിപ്രപഞ്ചമെല്ലാ ,മിനി-\\
പ്പോയാലുമെങ്കില്‍ നീ രാമപാദാന്തികേ.’\\
എന്നതു കേട്ടഭിവാദ്യവും ചെയ്തതി-\\
ഖിന്നനായ് ഭാഷ്പവും വാര്‍ത്തു വാങ്ങീടിനാന്‍.\\
രാമപാര്‍ശ്വം പ്രാപ്യ ചിന്താവിവശനായ്\\
ശ്രീമാന്‍ വിഭീഷണന്‍ നില്ക്കും ദശാന്തരേ\\
ഹസ്തപാദങ്ങളാല്‍ മര്‍ക്കടവീരരെ\\
ക്രുദ്ധനായൊക്കെ മുടിച്ചു തുടങ്ങിനാന്‍.\\
പേടിച്ചടുത്തുകൂടാഞ്ഞു കപികളു-\\
മോടിത്തുടങ്ങിനാര്‍ നാനാദിഗന്തരേ.\\
മത്തഹസ്തീന്ദ്രനെപ്പോലെ കപികളെ-\\
പ്പത്തുനൂറായിരം കൊന്നാനരക്ഷണാല്‍.\\
മര്‍ക്കടരാജനതുകണ്ടൊരു മല\\
കൈക്കൊണ്ടെറിഞ്ഞിതു മാറില്‍ തടുത്തവന്‍\\
കുത്തിനാന്‍ ശൂലമെടുത്തതുകൊണ്ടതി-\\
വിത്രസ്തനായ് വീണു മോഹിച്ചിതര്‍ക്കജന്‍.\\
അപ്പോളവനെയുമൂക്കോടെടുത്തുകൊ-\\
ണ്ടുത്പന്നമോദം നടന്നു നിശാചരന്‍.\\
യുദ്ധേ ജയിച്ചു സുഗ്രീവനെയുംകൊണ്ടു\\
നക്തഞ്ചരേശ്വരന്‍ ചെല്ലുന്ന നേരത്തു\\
നാരീജനം മഹാപ്രാസാദമേറിനി-\\
ന്നാരൂഢമോദം പനിനീരില്‍ മുക്കിയ\\
മാല്യങ്ങളും കളഭങ്ങളും തൂകിനാ-\\
രാലസ്യമാശുതീര്‍ന്നീടുവാനാദരാല്‍.\\
മര്‍ക്കടരാജനതേറ്റു മോഹം വെടി-\\
ഞ്ഞുല്‍ക്കടരോഷേണ മൂക്കും ചെവികളും\\
ദന്തനഖങ്ങളെക്കൊണ്ടു മുറിച്ചുകൊ-\\
ണ്ടന്തരീക്ഷേ പാഞ്ഞു പോന്നാനതിദ്രുതം.\\
ക്രോധവുമേറ്റമഭിമാനഹാനിയും\\
ഭീതിയുമുള്‍ക്കൊണ്ടു രക്താഭിഷിക്തനായ്\\
പിന്നെയും വീണ്ടും വരുന്നതുകണ്ടതി-\\
സന്നദ്ധനായടുത്തു സുമിത്രാത്മജന്‍.\\
പര്‍വതത്തിന്മേല്‍ മഴ പൊഴിയും വണ്ണം\\
ദുര്‍വാരബാണഗണം പൊഴിച്ചീടിനാന്‍.\\
പത്തുനൂറായിരം വാനരന്മാരെയും\\
വക്ത്രത്തിലാക്കിയടയ്ക്കുമവനുടന്‍\\
കര്‍ണനാസാവിലത്തൂടെ പുറപ്പെടും\\
പിന്നെയും വാരി വിഴുങ്ങുമവന്‍ തദാ.\\
രക്ഷോവരനുമന്നേരം നിരൂപിച്ചു\\
ലക്ഷ്മണന്‍തന്നെയുപേക്ഷിച്ചു സത്വരം\\
രാഘവന്‍തന്നോടടുത്താനതു കണ്ടു\\
വേഗേന ബാണം പൊഴിച്ചു രഘൂത്തമന്‍.\\
ദക്ഷിണഹസ്തവും ശൂലവും രാഘവന്‍\\
തല്‍ക്ഷണേ ബാണമെയ്താശു ഖണ്ഡിക്കയാല്‍\\
യുദ്ധാങ്കണേ വീണു, വാനരവൃന്ദവും\\
നക്തഞ്ചരന്മാരുമൊട്ടുമരിച്ചിതു.\\
വാമഹസ്തേ മഹാസാലവും കൈക്കൊണ്ടു\\
രാമനോടേറ്റമടുത്തു നിശാചരന്‍.\\
ഇന്ദ്രാസ്ത്രമെയ്തു ഖണ്ഡിച്ചാനതു വീണു-\\
മിന്ദ്രാരികള്‍ പലരും മരിച്ചീടിനാര്‍.\\
ബദ്ധകോപത്തോടലറിയടുത്തിതു\\
നക്തഞ്ചരാധിപന്‍ പിന്നെയുമന്നേരം\\
അര്‍ധചന്ദ്രാകാരമായ രണ്ടമ്പുകൊ-\\
ണ്ടുത്തുംഗപാദങ്ങളും മുറിച്ചീടിനാന്‍.\\
വക്ത്രവുമേറ്റം പിളര്‍ന്നു വിഴുങ്ങുവാന്‍\\
നക്തഞ്ചരേന്ദ്രന്‍ കുതിച്ചടുക്കുന്നേരം\\
പത്രികള്‍ വായില്‍ നിറച്ചു രഘൂത്തമന്‍\\
വൃത്രാരിദൈവതമായ് വിളങ്ങീടിനോ-\\
രസ്ത്രമെയ്തുത്തമാംഗത്തെയും ഖണ്ഡിച്ചു\\
വൃത്രാരിതാനും തെളിഞ്ഞാനതുനേരം.\\
ഉത്തമാംഗം പുരദ്വാരി വീണു മുറി-\\
ഞ്ഞബ്ധിയില്‍ വീണിതു ദേഹവുമന്നേരം.
\end{verse}

%%18_naaradasthuthi

\section{നാരദസ്തുതി}

\begin{verse}
സിദ്ധഗന്ധര്‍വവിദ്യാധര ഗുഹ്യക-\\
യക്ഷ ഭുജംഗഖഗാപ്സരോവൃന്ദവും\\
കിന്നരചാരണ കിംപുരുഷന്മാരും\\
പന്നഗതാപസദേവസമൂഹവും\\
പുഷ്പവര്‍ഷം ചെയ്തു ഭക്ത്യാ പുകഴ്ത്തിനാര്‍\\
ചില്‍പുരുഷം പുരുഷോത്തമമദ്വയം.\\
ദേവമുനീശ്വരന്‍ നാരദനും തദാ\\
സേവാര്‍ത്ഥമമ്പോടവതരിച്ചീടിനാന്‍.\\
രാമം ദശരഥനന്ദനമുല്‍പ്പല-\\
ശ്യാമളം കോമളം ബാണധനുര്‍ദ്ധരം\\
പൂര്‍ണചന്ദ്രാനനം കാരുണ്യപീയൂഷ-\\
പൂര്‍ണസമുദ്രം മുകുന്ദം സദാശിവം\\
രാമം ജഗദഭിരാമമാത്മാരാമ-\\
മാമോദമാര്‍ന്നു പുകഴ്ന്നു തുടങ്ങിനാന്‍:\\
‘സീതാപതേ! രാമ! രാജേന്ദ്ര! രാഘവ!\\
ശ്രീധര! ശ്രീനിധേ! ശ്രീപുരുഷോത്തമ!\\
ശ്രീരാമ! ദേവദേവേശ! ജഗന്നാഥ!\\
നാരായണാ! നിരാധാരാ! നമോസ്തുതേ\\
വിശ്വസാക്ഷിന്‍! പരമാത്മന്‍! സനാതന!\\
വിശ്വമൂര്‍ത്തേ! പരബ്രഹ്മമേ! ദൈവമേ!\\
ദുഃഖസുഖാദികളെല്ലാമനുദിനം\\
കൈക്കൊണ്ടു മായയാ മാനുഷാകാരനായ്\\
ശുദ്ധതത്ത്വജ്ഞനായ് ജ്ഞാനസ്വരൂപനായ്\\
സത്യസ്വരൂപനായ് സര്‍വലോകേശനായ്\\
സത്വങ്ങളുള്ളിലെ ജീവസ്വരൂപനായ്\\
സത്വപ്രധാനഗുണപ്രിയനായ് സദാ\\
വ്യക്തനായവ്യക്തനായതിസ്വസ്ഥനായ്\\
നിഷ്കളനായ് നിരാകാരനായിങ്ങനെ\\
നിര്‍ഗുണനായ് നിഗമാന്തവാക്യാര്‍ത്ഥമായ്\\
ചിദ്ഘനാത്മാവായ് ശിവനായ് നിരീഹനായ്\\
ചക്ഷുരുന്മീലനകാലത്തു സൃഷ്ടിയും\\
ചക്ഷുര്‍ന്നിമീലനംകൊണ്ടു സംഹാരവും\\
രക്ഷയും നാനാവിധാവതാരങ്ങളാല്‍\\
ശിക്ഷിച്ചു ധര്‍മത്തെയും പരിപാലിച്ചു\\
നിത്യം പുരുഷ പ്രകൃതികാലാഖ്യനായ്\\
ഭക്തപ്രിയനാം പരമാത്മനേ നമഃ\\
യാതൊരാത്മാവിനെക്കാണുന്നിതെപ്പോഴും\\
ചേതസി താപസേന്ദ്രന്മാര്‍ നിരാശയാ\\
തല്‍സ്വരൂപത്തിനായ്ക്കൊണ്ടു നമസ്കാരം\\
ചില്‍സ്വരൂപ! പ്രഭോ! നിത്യം നമോസ്തുതേ.\\
നിര്‍വികാരം വിശുദ്ധജ്ഞാനരൂപിണം\\
സര്‍വലോകാധാരമാദ്യം നമോ നമഃ\\
ത്വല്‍പ്രസാദംകൊണ്ടൊഴിഞ്ഞു മറ്റൊന്നിനാല്‍\\
ത്വദ്ബോധമുണ്ടായ് വരികയുമില്ലല്ലോ.\\
ത്വല്‍പ്പാദപത്മങ്ങള്‍ കണ്ടു സേവിപ്പതി-\\
നിപ്പോളെനിക്കവകാശമുണ്ടായതും.\\
ചില്‍പ്പുരുഷ! പ്രഭോ! നിന്‍കൃപാവൈഭവ-\\
മെപ്പോഴുമെന്നുള്ളില്‍ വാഴ്ക ജഗല്‍പ്പതേ!\\
കോപകാമദ്വേഷമത്സരകാര്‍പ്പണ്യ-\\
ലോഭമോഹാദി ശത്രുക്കളുണ്ടാകയാല്‍\\
മുക്തിമാര്‍ഗങ്ങളില്‍ സഞ്ചരിച്ചീടുവാന്‍\\
ശക്തിയുമില്ല നിന്മായാബലവശാല്‍.\\
ത്വല്‍ക്കഥാപീയൂഷപാനവും ചെയ്തുകൊ-\\
ണ്ടുള്‍ക്കാമ്പില്‍ നിന്നെയും ധ്യാനിച്ചനാരതം\\
ത്വല്‍പ്പൂജയും ചെയ്തു നാമങ്ങളുച്ചരി-\\
ച്ചിപ്രപഞ്ചത്തിങ്കലൊക്കെ നിരന്തരം\\
നിന്‍ചരിതങ്ങളും പാടി വിശുദ്ധനായ്\\
സഞ്ചരിപ്പാനായനുഗ്രഹിക്കേണമേ!\\
രാജരാജേന്ദ്ര! രഘുകുലനായക!\\
രാജീവലോചന! രാമ! രമാപതേ!\\
പാതിയും പോയിതു ഭൂഭാരമിന്നു നീ\\
ബാധിച്ച കുംഭകര്‍ണന്‍തന്നെക്കൊല്‍കയാല്‍.\\
ഭോഗീന്ദ്രനാകിയ സൗമിത്രിയും നാളെ\\
മേഘനിനാദനെക്കൊല്ലുമായോധനേ.\\
പിന്നെ മറ്റന്നാള്‍ ദശഗ്രീവനെബ്ഭവാന്‍\\
കൊന്നു ജഗത്ത്രയം രക്ഷിച്ചുകൊള്ളുക.\\
ഞാനിനി ബ്രഹ്മലോകത്തിനു പോകുന്നു\\
മാനവ വീര! ജയിക്ക ജയിക്ക നീ.’\\
ഇത്ഥം പറഞ്ഞു വണങ്ങി സ്തുതിച്ചതി-\\
ഭക്തിമാനാകിയ നാരദനും തദാ\\
രാഘവനോടനുവാദവും കൈക്കൊണ്ടു\\
വേഗേന പോയ് മറഞ്ഞീടിനാനന്നേരം.
\end{verse}

%%19_athikaayavadham

\section{അതികായവധം}

\begin{verse}
കുംഭകര്‍ണന്‍ മരിച്ചോരു വൃത്താന്തവും\\
കമ്പം വരുമാറു കേട്ടു ദശാനനന്‍\\
മോഹിച്ചു ഭൂമിയില്‍ വീണു പുനരുടന്‍\\
മോഹവും തീര്‍ന്നു മുഹൂര്‍ത്തമാത്രംകൊണ്ടു\\
പിന്നെപ്പലതരം ചൊല്ലി വിലാപിച്ചു\\
ഖിന്നനായോരു ദശഗ്രീവനെത്തദാ\\
ചെന്നു തൊഴുതു പറഞ്ഞു ത്രിശിരസ്സു-\\
മുന്നതനായോരതികായവീരനും\\
ദേവാന്തകനും നരാന്തകനും മുഹു-\\
രേവം മഹോദരനും മഹാപാര്‍ശ്വനും\\
മത്തനുമുന്മത്തനുമൊരുമിച്ചതി-\\
ശക്തിയേറീടും നിശാചരവീരന്മാര്‍\\
എട്ടുപേരും സമരത്തിനൊരുമ്പെട്ടു\\
ദുഷ്ടനാം രാവണന്‍തന്നോടു ചൊല്ലിനാര്‍:\\
‘ദുഃഖിപ്പതിനെന്തു കാരണം? ഞങ്ങള്‍ ചെ-\\
ന്നൊക്കെ രിപുക്കളെക്കൊന്നു വരാമല്ലോ.\\
യുദ്ധത്തിനായയച്ചീടുകില്‍ ഞങ്ങളെ-\\
ശ്ശത്രുക്കളാലൊരു പീഡയുണ്ടായ് വരാ.’\\
‘എങ്കിലോ നിങ്ങള്‍പ്പോയ്ച്ചെന്നു യുദ്ധം ചെയ്തു\\
സങ്ക‍ടം തീര്‍ക്കെ’ന്നു ചൊന്നാന്‍ ദശാനനന്‍.\\
‘കണ്ടുകൂടാതോളമുള്ള പെരുമ്പട-\\
യുണ്ടതുംകൊണ്ടു പൊയ്ക്കൊള്‍വിനെല്ലാവരും.’\\
ആയുധവാഹനഭൂഷണജാലവു-\\
മാവോളവും കൊടുത്താന്‍ ദശകന്ധരന്‍.\\
വെള്ളം കണക്കെപ്പരന്ന പെരുംപട-\\
യ്ക്കുള്ളില്‍ മഹാരഥന്മാരിവരെണ്‍വരും\\
പോര്‍ക്കു പുറപ്പെട്ടു ചെന്നതു കണ്ടള-\\
വൂക്കോടടുത്തു കപിപ്രവരന്മാരും.\\
സംഖ്യയില്ലാതോളമുള്ള പെരുംപട\\
വന്‍കടല്‍പോലെ പരന്നതു കണ്ടള-\\
വന്തകന്‍ വീട്ടിലാക്കീടിനാന്‍ സത്വര-\\
മെന്തൊരു വിസ്മയം ചൊല്ലാവതല്ലേതും.\\
കല്ലും മലയും മരങ്ങളും കൈക്കൊണ്ടു\\
ചെല്ലുന്ന വീരരോടേറ്റു നിശാചരര്‍\\
കൊല്ലുന്നിതാശു കപിവരന്മാരെയും\\
നല്ല ശസ്ത്രാസ്ത്രങ്ങള്‍ തൂകി ക്ഷണാന്തരേ.\\
വാരണവാജി രഥങ്ങളും കാലാളും\\
ദാരുണന്മാരായ രാക്ഷസവീരരും\\
വീണു മരിച്ചുള്ള ചോരപ്പുഴകളും\\
കാണായിതു പലതായൊലിക്കുന്നതും.\\
അന്തമില്ലാതെ കബന്ധങ്ങളും പല-\\
തന്തികേ നൃത്തമാടിത്തുടങ്ങീ ബലാല്‍.\\
രാക്ഷസരൊക്കെ മരിച്ചതു കണ്ടതി-\\
രൂക്ഷതയോടുമടുത്താന്‍ നരാന്തകന്‍.\\
കുന്തവുമേന്തിക്കുതിരപ്പുറമേറി-\\
യന്തകനെപ്പോലെ വേഗാലടുത്തപ്പോള്‍\\
അംഗദന്‍ മുഷ്ടികള്‍കൊണ്ടവന്‍തന്നുടല്‍\\
ഭംഗം വരുത്തി യമപുരത്താക്കിനാന്‍.\\
ദേവാന്തകനും പരിഘവുമായ് വന്നു\\
ദേവേന്ദ്രപുത്രതനയനോടേറ്റിതു.\\
വാരണമേറി മഹോദരവീരനും\\
തേരിലേറി ത്രിശിരസ്സുമണഞ്ഞിതു.\\
മൂവരോടും പൊരുതീടിനാനംഗദന്‍\\
ദേവാദികളും പുകഴ്ത്തിനാരന്നേരം.\\
കണ്ടുനില്ക്കും വായുപുത്രനും നീലനും\\
മണ്ടിവന്നാശു തുണച്ചാരതുനേരം.\\
മാരുതികൊന്നിതു ദേവാന്തകനെയും\\
വീരനാം നീലന്‍ മഹോദരന്‍ തന്നെയും\\
ശുരനാകും ത്രിശിരസ്സിന്‍തലകളെ\\
മാരുതി വെട്ടിക്കളഞ്ഞു കൊന്നീടിനാന്‍.\\
വന്നു പൊരുതാന്‍ മഹാപാര്‍ശ്വനന്നേരം\\
കൊന്നുകളഞ്ഞാനൃഷഭന്‍ മഹാബലന്‍.\\
മത്തനുമുന്മത്തനും മരിച്ചാര്‍ കപി-\\
സത്തമന്മാരോടെതിര്‍ത്തതിസത്വരം.\\
വിശ്വൈകവീരനതികായനന്നേര-\\
മശ്വങ്ങളായിരം പൂട്ടിയ തേരതില്‍\\
ശസ്ത്രാസ്ത്രജാലം നിറച്ചു വില്ലും ധരി-\\
ച്ചസ്ത്രജ്ഞനത്യര്‍ത്ഥമുദ്ധതചിത്തനായ്\\
യുദ്ധത്തിനായ് ചെറുഞാണൊലിയുമിട്ടു\\
നക്തഞ്ചരശ്രേഷ്ഠപുത്രനടുത്തപ്പോള്‍\\
നില്ക്കരുതാഞ്ഞു ഭയപ്പെട്ടു വാനര-\\
രൊക്കെ വാല്‍പൊങ്ങിച്ചു മണ്ടിത്തുടങ്ങിനാര്‍.\\
സാമര്‍ഥ്യമേറെയുള്ളോരതികായനെ-\\
സ്സൗമിത്രി ചെന്നു ചെറുത്താനതുനേരം.\\
ലക്ഷ്മണബാണങ്ങള്‍ ചെന്നടുക്കും വിധൗ\\
തല്‍ക്ഷണേ പ്രത്യങ്മുഖങ്ങളായ് വീണുപോം.\\
ചിന്ത മുഴുത്തേതുമാവതല്ലാഞ്ഞേറ്റ-\\
മന്ധനായ് സൗമിത്രി നില്ക്കുന്നതുനേരം\\
മാരുതദേവനും മാനുഷനായ് വന്നു\\
സാരനാം സൗമിത്രിയോടു ചൊല്ലീടിനാന്‍:\\
‘പണ്ടു വിരിഞ്ചന്‍ കൊടുത്തൊരു കഞ്ചുക-\\
മുണ്ടതുകൊണ്ടിവനേല്ക്കയില്ലായുധം.\\
ധര്‍മത്തെ രക്ഷിച്ചുകൊള്ളുവാനിന്നിനി\\
ബ്രഹ്മാസ്ത്രമെയ്തിവന്‍ തന്നെ വധിക്ക നീ.\\
പിന്നെ നിന്നാല്‍ വധിക്കപ്പെടുമിന്ദ്രജി-\\
ത്തുന്നതനായ ദശാനനന്‍ തന്നെയും\\
കൊന്നു പാലിക്കും ജഗത്ത്രയം രാഘവ-’\\
നെന്നു പ്രറഞ്ഞു മറഞ്ഞു സമീരണന്‍.\\
ലക്ഷ്മണനും നിജപൂര്‍വജന്‍ തന്‍പദ-\\
മുള്‍ക്കാമ്പില്‍ നന്നായുറപ്പിച്ചു വന്ദിച്ചു\\
പുഷ്കരസംഭവബാണം പ്രയോഗിച്ചു\\
തല്‍ക്ഷണേ കണ്ഠം മുറിച്ചാനതുനേരം.\\
ഭൂമൗ പതിച്ചോരതികായമസ്തക-\\
മാമോദമുള്‍ക്കൊണ്ടെടുത്തു കപികുലം\\
രാമാന്തികേ വെച്ചു കൈതൊഴുതീടിനാ-\\
രാമയം പൂണ്ടു ശേഷിച്ച രക്ഷോഗണം\\
രാവണനോടറിയിച്ചാരവസ്ഥകള്‍;\\
ഹാ! വിധിയെന്നലറീ ദശകണ്ഠനും.
\end{verse}

%%20_indrajitthintevijayam

\section{ഇന്ദ്രജിത്തിന്റെ വിജയം}

\begin{verse}
‘മക്കളും തമ്പിമാരും മരുമക്കളു-\\
മുള്‍ക്കരുത്തേറും പടനായകന്മാരും\\
മന്തികളും മരിച്ചീടിനാരേറ്റവ-\\
രെന്തിനിനല്ലതു ശങ്കര! ദൈവമേ!’\\
ഇത്ഥം വിലപിച്ചനേരത്തു ചെന്നിന്ദ്ര-\\
ജിത്തും നമസ്കരിച്ചീടിനാന്‍ താതനെ.\\
‘ഖേദമുണ്ടാകരുതേതുമേ മാനസേ\\
താതനു ഞാനിഹ ജീവിച്ചിരിക്കവേ.\\
ശത്രുക്കളെക്കൊലചെയ്തു വരുന്നതു-\\
ണ്ടത്തലും തീര്‍ത്തിങ്ങിരിന്നരുളേണമേ!\\
സ്വസ്ഥനായ് വാഴുക ചിന്തയും കൈവിട്ടു\\
യുദ്ധേ ജയിപ്പാനനുഗ്രക്കേണമേ!’\\
എന്നതു കേട്ടു തനയനേയും പുണര്‍-\\
‘ന്നെന്നേ സുഖമേ ജയിച്ചു വരിക നീ.’\\
വമ്പനാം പുത്രനും കുമ്പിട്ടു താതനെ-\\
ത്തന്‍ പടയോടും നടന്നു തുടങ്ങിനാന്‍.\\
ശംഭുപ്രസാദം വരുത്തുവാനായ്ച്ചെന്നു\\
ജംഭാരിജിത്തും നികുംഭില പുക്കിതു\\
സംഭാരജാലവും സമ്പാദ്യ സാദരം\\
സംഭാവ്യ ഹോംമമാരംഭിച്ചിതന്നേരം\\
രക്തമാല്യാംബരഗന്ധാനുലേപന-\\
യുക്തനായ്ത്തത്ര ഗൂരൂപദേശാന്വിതം\\
ഭക്തിപൂണ്ടുജ്ജ്വലിപ്പിച്ചഗ്നിദേവനെ\\
ശക്തി തനിക്കു വര്‍ധിച്ചുവരുവാനായ്\\
നക്തഞ്ചരാധിപപുത്രനുമെത്രയും\\
വ്യക്തവര്‍ണസ്വരമന്ത്രപുരസ്കൃതം\\
കര്‍ത്തവ്യമായുള്ള കര്‍മം കഴിച്ചഥ\\
ചിത്രഭാനു പ്രസാദത്താലതിദ്രുതം\\
ശസ്ത്രാസ്ത്രചാപരഥാദികളോടുമ-\\
ന്തര്‍ദ്ധാനവിദ്യയും ലബ്ധ്വാ നിരാകുലം\\
ഹൊമസമാപ്തിവരുത്തിപ്പുറപ്പെട്ടു\\
രാമാദികളോടു പോരിനായാശരന്‍\\
പോര്‍ക്കളം പുക്കോരുനേരം കപികളും\\
രാക്ഷസരെച്ചെറുത്താര്‍ത്തടുത്തീടിനാര്‍.\\
മേഘജാലം വരിഷിക്കുന്നതുപോലെ\\
മേഘനാദന്‍ കണ തൂകിത്തുടങ്ങിനാന്‍.\\
പാഷാണപര്‍വതവൃക്ഷാദികള്‍കൊണ്ടു\\
ഭീഷണന്മാരായ വാനരവീരരും\\
ദാരുണമായ് പ്രഹരിച്ചുതുടങ്ങിനാര്‍.\\
വാരണവാജി പദാതിരഥികളും\\
അന്തകന്‍തന്‍ പുരിയില്‍ ചെന്നുപുക്കവര്‍-\\
ക്കന്തം വരുന്നതു കണ്ടൊരു രാവണി\\
സന്താപമോടുമന്തര്‍ദ്ധാനവും ചെയ്തു\\
സന്തതം തൂകിനാന്‍ ബ്രഹ്മാസ്ത്രസഞ്ചയം.\\
വൃക്ഷങ്ങള്‍ വെന്തുമുറിഞ്ഞുവീഴുംവണ്ണ-\\
മൃക്ഷപ്രവരന്മാര്‍ വീണു തുടങ്ങിനാര്‍.\\
വമ്പരാം മര്‍ക്കടന്മാരുടെ മെയ്യില്‍വ-\\
ന്നമ്പതും നൂറുമിരുനൂറുമഞ്ഞൂറും\\
അമ്പുകള്‍കൊണ്ടു പിളര്‍ന്നു തെരുതെരെ-\\
ക്കമ്പം കലര്‍ന്നു മോഹിച്ചു വീണീടിനാര്‍.\\
അമ്പതുബാണം വിവിദനേറ്റൂ പുന-\\
രൊമ്പതു മൈന്ദനു മഞ്ചുഗജന്മേലും\\
തൊണ്ണൂറുബാണം നളനുംതറച്ചിത-\\
വ്വണ്ണമേറ്റു ഗന്ധമാദനന്‍ മെയ്യിലും\\
ഈരൊമ്പതേറ്റിതു നീലനും മുപ്പതു-\\
മീരഞ്ചുബാണങ്ങള്‍ ജാംബവാന്‍ മെയ്യിലും\\
ആറു പനസനു, മേഴുവിനത, നീ-\\
രാറു സുഷേണനുമെട്ടു കുമുദനും.\\
ആറഞ്ചുബാണമൃഷഭനും കേസരി-\\
ക്കാറുമൊരമ്പതും കൂടെ വന്നേറ്റിതു.\\
പത്തു ശതബലിക്കൊമ്പതു ധൂമ്രനും,\\
പത്തുമൊരെട്ടും പ്രമാഥിക്കുമേറ്റിതു.\\
പത്തും പുനരിരുപത്തഞ്ചുമേറ്റിതു\\
ശക്തിയേറും വേഗദര്‍ശിക്കതുപോലെ.\\
നാല്പതുകൊണ്ടു ദധിമുഖന്‍ മെയ്യിലും\\
നാല്പത്തിരണ്ടു ഗവാക്ഷനു മേറ്റിതു.\\
മൂന്നു ഗവയനുമഞ്ചു ശരഭനും\\
മൂന്നുമൊരു നാലുമേറ്റൂ സുമുഖനും\\
ദുര്‍മുഖനേറ്റിതിരുപത്തിനാലമ്പു,\\
സമ്മാനമായറുപത്തഞ്ചു താരനും.\\
ജ്യോതിര്‍മുഖനുമറുപതേറ്റു പുന-\\
രാതങ്കമോടെമ്പതഗ്നിവദനനും\\
അംഗദന്‍മേലെഴുപത്തഞ്ചു കൊണ്ടിതു\\
തുംഗനാം സുഗ്രീവനേറ്റു ശരശതം.\\
ഇത്ഥം കപികുലനായകന്മാരറു-\\
പത്തേഴു കോടിയും വീണിതു ഭൂതലേ.\\
മര്‍ക്കടന്മാരിരുപത്തൊന്നു വെള്ളവു-\\
മര്‍ക്കതനയനും വീണോരനന്തരം\\
ആവതില്ലേതുമിതിന്നു നമുക്കെന്നു\\
ദേവദേവന്മാരുമന്യോന്യമന്നേരം\\
വ്യാകുലം പൂണ്ടു പറഞ്ഞു നില്ക്കേ,രുഷാ\\
രാഘവന്മാരെയുമെയ്തു വീഴ്ത്തീടിനാന്‍\\
മേഘനാദന്‍ മഹാവീര്യവ്രതധരന്‍.\\
ശോകവിഷണ്ണമായ് നിശ്ചലമായിതു\\
ലോകവും, കൗണപാധീശജയത്തിനാ-\\
ലാഖണ്ഡലാരിയും ശംഖനാദം ചെയ്തു\\
വേഗേന ലങ്കയില്‍ പുക്കിരുന്നീടിനാന്‍;\\
ലേഖസമൂഹവും മാഴ്കീ ഗതാശയാ.
\end{verse}

%%21_aushadhaharanayaathra

\section{ഔഷധാഹരണയാത്ര}

\begin{verse}
കൈകസീനന്ദനനായ വിഭീഷണന്‍\\
ഭാഗവതോത്തമന്‍ ഭക്തപരായണന്‍\\
പോക്കുവന്‍ മേലിലാപത്തു ഞാനെന്നോര്‍ത്തു\\
പോര്‍ക്കളം കൈവിട്ടു വാങ്ങിനിന്നീടിനാന്‍.\\
കൊള്ളിയും മിന്നിക്കിടക്കുന്നതില്‍ പ്രാണ-\\
നുള്ളവരാരെന്നറിയേണമെന്നോര്‍ത്തു\\
നോക്കി നൊക്കിസ്സഞ്ചരിച്ചു തുടങ്ങിനാ-\\
നാക്കമേറും വായുപുത്രനുമന്നേരം\\
ആരിനിയുള്ളതൊരു സഹായത്തിനെ-\\
ന്നാരായ്കവേണമെന്നോര്‍ത്തവനും തദാ\\
ശാഖാമൃഗങ്ങള്‍ കിടക്കുന്നവര്‍കളില്‍\\
ചാകാതവരിതിലാരെന്നു നോക്കുവാന്‍\\
ഏകാകിയായ് നടക്കുന്ന നേരം തത്ര\\
രാഘവഭക്തന്‍ വിഭീഷണനെക്കണ്ടു.\\
തമ്മിലന്യോന്യമറിഞ്ഞു ദുഃഖം പൂണ്ടു\\
നിര്‍മലന്മാര്‍ നടന്നീടിനാര്‍ പിന്നെയും\\
പാഥോജസംഭവനന്ദനന്‍ ജാംബവാന്‍\\
താതനനുഗ്രഹംകൊണ്ടു മോഹം തീര്‍ന്നു\\
കണ്ണൂ മിഴിപ്പാനരുതാഞ്ഞിരിക്കുമ്പോള്‍\\
ചെന്നു വിഭീഷണന്‍ ചോദിച്ചിതാദരാല്‍:\\
‘നിന്നുടെ ജീവനുണ്ടോ കപിപുംഗവ?\\
നന്നായിതെങ്കില്‍; നീയെന്നെയറിഞ്ഞിതോ?\\
‘കണ്ണു മിഴിച്ചുകൂടാ രുധിരംകൊണ്ടു\\
നിന്നുടെ വാക്കുകേട്ടുള്ളില്‍ വിഭാതി മേ\\
രാക്ഷസരാജന്‍ വിഭീഷണനെന്നതു;\\
സാക്ഷാല്‍ പരമാര്‍ഥമെന്നോടു ചൊല്ലുക.’\\
‘സത്യം വിഭീഷണനായതു ഞാനെടോ!\\
സത്യമതേ!’ പുനരെന്നതു കേട്ടവന്‍\\
ചോദിച്ചിതാശരാധീശ്വരന്‍ തന്നോടു\\
‘ബോധമുണ്ടല്ലോ ഭവാനേറ്റമാകയാല്‍\\
മേഘനാദാസ്ത്രങ്ങളേറ്റു മരിച്ചൊരു\\
ശാഖാമൃഗങ്ങളില്‍ നമ്മുടെ മാരുതി\\
ജീവനോടേ പുനരെങ്ങാനുമുണ്ടെങ്കി-\\
ലാവതെല്ലാം തിരയേണമിനിയെടോ!’\\
ചോദിച്ചിതാശു വിഭീഷണ,‘നെന്തെടോ\\
വാതാത്മജനില്‍ വാത്സല്യമുണ്ടായതും?\\
രാമസൗമിത്രി സുഗ്രീവാംഗദാദിക-\\
ളാമവരേവരിലും വിശേഷിച്ചു നീ\\
ചോദിച്ചതെന്തു സമീരണപുത്രനെ?\\
മോദിച്ചതെന്തവനെക്കുറിച്ചേറ്റവും?\\
‘എങ്കിലോ കേള്‍ക്ക നീ മാരുതിയുണ്ടെങ്കില്‍\\
സങ്കടമില്ല മറ്റാര്‍ക്കുമറിഞ്ഞാലും.\\
മാരുതപുത്രന്‍ മരിച്ചിതെന്നാകില്‍ മ-\\
റ്റാരുമില്ലൊക്കെ മരിച്ചതിനൊക്കുമേ.’\\
സാരസസംഭവപുത്രവാക്യം കേട്ടു\\
മാരുതിയും ബഹുമാനിച്ചു സാദരം\\
‘ഞാനിതല്ലോ മരിച്ചീലെന്നവന്‍ കാല്ക്ക-\\
ലാമോദമുള്‍ക്കൊണ്ടു വീണു വണങ്ങിനാന്‍.\\
ഗാഢമായാശ്ലേഷവും ചെയ്തു ജാംബവാന്‍\\
കൂടെത്തലയില്‍ മുകര്‍ന്നു ചൊല്ലീടിനാന്‍:\\
‘മേഘനാദാസ്ത്രങ്ങളേറ്റു മരിച്ചൊരു\\
ശാഖാമൃഗങ്ങളെയും പിന്നെ നമ്മുടെ\\
രാഘവന്മാരെയും ജീവിച്ചിരുത്തുവാ-\\
നാകുന്നവരാരുമില്ല നീയെന്നിയേ.\\
പോകവേണം നീ ഹിമവാനെയും കട-\\
ന്നാകുലമറ്റു കൈലാസശൈലത്തോളം.\\
കൈലാസസന്നിധിയിങ്കലൃഷഭാദ്രി-\\
മേലുണ്ടു ദിവ്യൗഷധങ്ങളവറ്റിനു\\
നാലിനും നാമങ്ങളും കേട്ടുകൊള്ളുക.\\
മുമ്പില്‍ വിശല്യകരണിയെന്നൊന്നെടോ!\\
പിമ്പു സന്ധാനകരണി, മൂന്നാമതും\\
നല്ല സുവര്‍ണകരണി, നാലാമതും\\
ചൊല്ലുവന്‍ ഞാന്‍ മൃതസഞ്ജീവനി സഖേ!\\
രണ്ടു ശൃംഗങ്ങളുയര്‍ന്നു കാണാമവ-\\
രണ്ടിനും മദ്ധ്യേ മരുന്നുകള്‍ നില്പതും.\\
ആദിത്യനോളം പ്രഭയുണ്ടു നാലിനും\\
വേദസ്വരൂപങ്ങളെന്നുമറിക നീ.\\
വാരാന്നിധിയും വനങ്ങള്‍ ശൈലങ്ങളും\\
ചാരുനദികളും രാജ്യങ്ങളും കട-\\
ന്നാരാല്‍ വരിക മരുന്നുകളുംകൊണ്ടു\\
മാരുതനന്ദന! പോക നീ വൈകാതെ.’\\
ഇത്ഥം വിധിസുതന്‍ വാക്കുകള്‍ കേട്ടവന്‍\\
ഭക്ത്യാ തൊഴുതു മഹേന്ദ്രമേറീടിനാന്‍.\\
മേരുവിനോളം വളര്‍ന്നു ചമഞ്ഞവന്‍\\
വാരാന്നിധിയും കുലപര്‍വതങ്ങളും\\
ലങ്കയും രാക്ഷസരും വിറയ്ക്കും വണ്ണം\\
ശങ്കാരഹിതം കരുത്തോടലറിനാന്‍.\\
വായുവേഗേന കുതിച്ചുയര്‍ന്നംബരേ\\
പോയവന്‍ നീഹാരശൈലവും പിന്നിട്ടു\\
വൈരിഞ്ചമണ്ഡവും ശങ്കരശൈലവും\\
നേരേ ധരാനദിയുമളകാപുരം\\
മേരുഗിരിയുമൃഷഭാദ്രിയും കണ്ടു\\
മാരുതി വിസ്മയപ്പെട്ടു നോക്കീടിനാന്‍.
\end{verse}

%%22_kaalanemiyudepurappadu

\section{കാലനേമിയുടെ പുറപ്പാട്}

\begin{verse}
മാരുതനന്ദനനൗഷധത്തിന്നങ്ങു\\
മാരുതവേഗേന പോയതറിഞ്ഞൊരു\\
ചാരവരന്മാര്‍ നിശാചരാധീശനോ-\\
ടാരുമറിയാതെ ചെന്നു ചൊല്ലീടിനാര്‍.\\
ചാരവാക്യം കേട്ടു രാത്രിഞ്ചരാധിപന്‍\\
പാരം വിചാരം കലര്‍ന്നു മരുവിനാന്‍\\
ചിന്താവശനായ് മുഹൂര്‍ത്തമിരുന്നള-\\
വന്തര്‍ഗൃഹത്തിങ്കല്‍ നിന്നു പുറപ്പെട്ടു\\
രാത്രിയിലാരും സഹായവും കൂടാതെ\\
രാത്രിഞ്ചരാധിപന്‍ കാലനേമീഗൃഹം\\
പ്രാപിച്ചളവതിവിസ്മയം പൂണ്ടവ-\\
നാപൂര്‍ണമോദം തൊഴുതു സന്ത്രസ്തനായ്\\
അര്‍ഘ്യാദികള്‍കൊണ്ടു പൂജിച്ചു ചോദിച്ചാ-\\
‘നര്‍ക്കോദയം വരുംമുമ്പേ ലഘുതരം\\
ഇങ്ങെഴുന്നള്ളുവാനെന്തൊരു കാരണ-\\
മിങ്ങനെ മറ്റുള്ളകമ്പടി കൂടാതെ?’\\
ദുഃഖനിപീഡിതനാകിയ രാവണ-\\
നക്കാലനേമിതന്നോടു ചൊല്ലീടിനാന്‍:\\
‘ഇക്കാലവൈഭവമെന്തുചൊല്ലാവതു-\\
മൊക്കെ നിന്നോടു ചൊല്‍വാനത്ര വന്നതും\\
ശക്തിമാനാകിയ ലക്ഷ്മണനെന്നുടെ\\
ശക്തിയേറ്റാശു വീണീടിനാന്‍ ഭൂതലേ.\\
പിന്നെ വിരിഞ്ചാസ്ത്രമെയ്തു മമാത്മജന്‍\\
മന്നവന്മാരെയും വാനരന്മാരെയും\\
കൊന്നു രണാങ്കണംതന്നില്‍ വീഴ്ത്തീടിനാന്‍\\
വെന്നിപ്പറയുമടിപ്പിച്ചിതാത്മജന്‍.\\
ഇന്നു ജീവിപ്പിച്ചു കൊള്ളുവാന്‍ മാരുത-\\
നന്ദനനൗഷധത്തിന്നു പോയീടിനാന്‍.\\
ചെന്നു വിഘ്നം വരുത്തേണമതിന്നു നീ\\
നിന്നോടുപായവും ചൊല്ലാമതിന്നെടോ!\\
താപസനായ് ചെന്നു മാര്‍ഗമദ്ധ്യേ പുക്കു\\
പാപവിനാശനമായുള്ള വാക്കുകള്‍\\
ചൊല്ലി മോഹിപ്പിച്ചു കാലവിളംബനം\\
വല്ല കണക്കിലും നീ വരുത്തീടണം.’\\
താമസവാക്കുകല്‍ കേട്ട നേരം കാല-\\
നേമിയും രാവണന്‍തന്നോടു ചൊല്ലിനാന്‍:\\
‘സാമവേദജ്ഞ! സര്‍വജ്ഞ! ലങ്കേശ്വര!\\
സാമമാമെന്നുടെ വാക്കു കേള്‍ക്കേണമേ!\\
നിന്നെക്കുറിച്ചു മരിപ്പതിനിക്കാല-\\
മെന്നുള്ളിലേതും മടിയില്ല നിശ്ചയം.\\
മാരീചനെക്കണക്കെ മരിപ്പാന്‍ മന-\\
താരിലെനിക്കേതുമില്ലൊരു ചഞ്ചലം.\\
മക്കളും തമ്പിമാരും മരുമക്കളും\\
മക്കളുടെ നല്ല മക്കളും ഭൃത്യരും\\
ഒക്കെ മരിച്ചു നീ ജീവിച്ചിരുന്നിട്ടു\\
ദുഃഖമൊഴിഞ്ഞെന്തൊരു ഫലമുള്ളതും?\\
എന്തു രാജ്യംകൊണ്ടു പിന്നെയൊരു ഫലം?\\
എന്തു ഫലം തവ ജാനകിയെക്കൊണ്ടും?\\
ഹന്ത! ജഡാത്മകമായ ദേഹംകൊണ്ടു-\\
മെന്തു ഫലം തവ ചിന്തിച്ചു കാണ്‍കെടോ!\\
സീതയെ രാമനു കൊണ്ടക്കൊടുത്തു നീ\\
സോദരനായ്ക്കൊണ്ടു രാജ്യവും നല്കുക.\\
കാനനംതന്നില്‍ മുനിവേഷവും പൂണ്ടു\\
മാനസശുദ്ധിയോടും കൂടി നിത്യവും\\
പ്രത്യുഷസ്യുത്ഥായ ശുദ്ധതോയേ കുളി-\\
ച്ചത്യന്തഭക്തിയോടര്‍ക്കോദയം കണ്ടു\\
സന്ധ്യാനമസ്കാരവും ചെയ്തു ശീഘ്രമേ-\\
കാന്തേ സുഖാസനം പ്രാപിച്ചു തുഷ്ടനായ്\\
സര്‍വവിഷയസംഗങ്ങളും കൈവിട്ടു\\
സര്‍വേന്ദ്രിയങ്ങളും പ്രത്യാഹരിച്ചുടന്‍\\
ആത്മനി കണ്ടുകണ്ടാത്മാനമാത്മനാ\\
സ്വാത്മോദയംകൊണ്ടു സര്‍വലോകങ്ങളും\\
സ്ഥാവരജംഗമജാതികളായുള്ള\\
ദേവതിര്യങ്മനുഷ്യാദി ജന്തുക്കളും\\
ദേഹബുദ്ധീന്ദ്രിയാദ്യങ്ങളും നിത്യനാം\\
ദേഹി സര്‍വത്തിനുമാധാരമെന്നതും\\
ആബ്രഹ്മസ്തംബപര്യന്തമായെന്തോന്നു\\
താത്പര്യമുള്‍ക്കൊണ്ടു കണ്ടതും കേട്ടതും\\
ഒക്കെ പ്രകൃതിയെന്നത്രേ ചൊല്ലപ്പെടും\\
സദ്ഗുരുമായയെന്നും പറഞ്ഞീടുന്നു.\\
ഇക്കണ്ട ലോകവൃക്ഷത്തിന്നനേകധാ\\
സര്‍ഗസ്ഥിതിവിനാശങ്ങള്‍ക്കു കാരണം\\
ലോഹിതശ്വേതകൃഷ്ണാദിമയങ്ങളാം\\
ദേഹങ്ങളെ ജനിപ്പിക്കുന്നതും മായാ.\\
പുത്രഗണം കാമക്രോധാദികളെല്ലാം\\
പുത്രികളും തൃഷ്ണാഹിംസാദികളെടോ!\\
തന്റെ ഗുണങ്ങളെക്കൊണ്ടു മോഹിപ്പിച്ചു\\
തന്റെ വശത്താക്കുമാത്മാവിനെയവള്‍.\\
കര്‍ത്തൃത്വഭോക്തൃത്വമുഖ്യഗുണങ്ങളെ\\
നിത്യമാത്മാവാകുമീശ്വരന്‍ തങ്കലേ\\
ആരോപണംചെയ്തു തന്റെ വശത്താക്കി\\
നേരേ നിരന്തരം ക്രീഡിച്ചുകൊള്ളുന്നു.\\
ശുദ്ധനാത്മാപരനേകനവളോടു\\
യുക്തനായ് വന്നു പുറത്തു കാണുന്നിതു\\
തന്നുടെയാത്മാവിനെത്താന്‍ മറക്കുന്നി-\\
തന്വഹം മായാഗുണവിമോഹത്തിനാല്‍.\\
ബോധസ്വരൂപനായോരു ഗുരുവിനാല്‍\\
ബോധിതനായാല്‍ നിവൃത്തേന്ദ്രിയനുമായ്\\
കാണുന്നിതാത്മാവിനെ സ്പഷ്ടമായ് സദാ\\
വേണുന്നതെല്ലാമവനു വന്നു തദാ.\\
ദൃഷ്ട്വാ പ്രകൃതിഗുണങ്ങളോടാശു വേര്‍-\\
പെട്ടു ജീവന്മുക്തനായ് വരും ദേഹിയും.\\
നീയുമേവം സദാത്മാനം വിചാരിച്ചു\\
മായാഗുണങ്ങളില്‍നിന്നു വിമുക്തനായ്\\
ആദ്യപ്രകൃതിവിമുക്തനാത്മാവിതി\\
ജ്ഞാത്വാ നിരസ്താശയാ ജിതകാമനായ്\\
ധ്യാനനിരതനായ് വാഴുകെന്നാല്‍ വരു-\\
മാനന്ദമേതും വികല്പമില്ലോര്‍ക്ക നീ.\\
ധ്യാനിപ്പതിന്നു സമര്‍ത്ഥനല്ലെങ്കിലോ\\
മാനസേ പാവനേ ഭക്തിപരവശേ\\
നിത്യം സഗുണനാം ദേവനെയാശ്രയി-\\
ച്ചത്യന്തശുദ്ധ്യാ സ്വബുദ്ധ്യാ നിരന്തരം\\
ഹൃല്‍പത്മകര്‍ണികാമദ്ധ്യേ സുവര്‍ണ പീ-\\
ഠോല്പലേ രത്നഗണാഞ്ചിതേ നിര്‍മലേ\\
ശ്ലേക്ഷ്ണേ മൃദുതരേ സീതയാ സംസ്ഥിതം\\
ലക്ഷ്മണസേവിതം ബാണധനുര്‍ദ്ധരം\\
വീരാസനസ്ഥം വിശാലവിലോചന-\\
മൈരാവതീതുല്യപീതാംബരധരം\\
ഹാരകിരീടകേയൂരാംഗദാംഗുലീ-\\
യോരുരത്നാഞ്ചിതകുണ്ഡലനൂപുര-\\
ചാരു കടക കടിസൂത്ര കൗസ്തുഭ-\\
സാരസമാല്യവനമാലികാധരം\\
ശ്രീവത്സവക്ഷസം രാമം രമാവരം\\
ശ്രീവാസുദേവം മുകുന്ദം ജനാര്‍ദനം\\
സര്‍വഹൃതിസ്ഥിതം സര്‍വേശ്വരം പരം\\
സര്‍വവന്ദ്യം ശരണാഗതവത്സലം\\
ഭക്ത്യാ പരബ്രഹ്മയുക്തനായ് ധ്യാനിക്കില്‍\\
മുക്തനായ് വന്നുകൂടും ഭവാന്‍ നിര്‍ണയം.\\
തച്ചരിത്രം കേട്ടുകൊള്‍കയും ചൊല്കയു-\\
മുച്ചരിച്ചും രാമരാമേതി സന്തതം\\
ഇങ്ങനെ കാലം കഴിച്ചുകൊള്ളുന്നാകി-\\
ലെങ്ങനെ ജന്മങ്ങള്‍ പിന്നെയുണ്ടാകുന്നു?\\
ജന്മജന്മാന്തരത്തിങ്കലുമുള്ളോരു\\
കല്മഷമൊക്കെ നശിച്ചുപോം നിശ്ചയം.\\
വൈരം വെടിഞ്ഞതിഭക്തിസംയുക്തനായ്\\
ശ്രീരാമദേവനെത്തന്നെ ഭജിക്ക നീ.\\
ദേവം പരിപൂര്‍ണമേകം സദാ ഹൃദി\\
ഭാവിതം ഭാവരൂപം പുരുഷം പരം\\
നാമരൂപാദിഹീനം പുരാണം ശിവം\\
രാമദേവം ഭജിച്ചീടു നീ സന്തതം.’\\
രാക്ഷസേന്ദ്രന്‍ കാലനേമി പറഞ്ഞൊരു\\
വാക്കുകള്‍ പീയൂഷതുല്യങ്ങള്‍ കേള്‍ക്കയാല്‍\\
ക്രോധതാമ്രാക്ഷനായ് വാളുമായ് തദ്ഗളം\\
ഛേദിപ്പതിന്നൊരുമ്പെട്ടു ചൊല്ലീടിനാന്‍:\\
‘നിന്നെ വെട്ടിക്കളഞ്ഞിട്ടിനിക്കാര്യങ്ങള്‍\\
പിന്നെയെല്ലാം വിചാരിച്ചുകൊള്ളാമെടോ!’\\
കാലനേമി ക്ഷണദാചരനന്നേരം\\
മൂലമെല്ലാം വിചാരിച്ചു ചൊല്ലീടിനാന്‍:\\
‘രാക്ഷസരാജ! ദുഷ്ടാത്മന്‍! മതി മതി\\
രൂക്ഷതാഭാവമിതുകൊണ്ടു കിം ഫലം?\\
നിന്നുടെ ശാസനം ഞാനനുഷ്ഠിപ്പന-\\
തെന്നുടെ സദ്ഗതിക്കെന്നു ധരിക്ക നീ.\\
സത്യസ്വരൂപത്തെ വഞ്ചിപ്പതിന്നു ഞാ-\\
നദ്യ സമുദ്യുക്തനായേന്‍ മടിയാതെ.’\\
എന്നു പറഞ്ഞു ഹിമാദ്രിപാര്‍ശ്വേ ഭൃശം\\
ചെന്നിരുന്നാന്‍ മുനിവേഷമായ് തല്‍ക്ഷണേ.\\
കാണായിതാശ്രമം മായാവിരചിതം\\
നാനാമുനിജനസേവിതമായതും.\\
ശിഷ്യജനപരിചാരകസംയുത-\\
മൃഷ്യാശ്രമം കണ്ടു വായുതനയനും\\
ചിന്തിച്ചു നിന്നാ’നിവിടെയൊരാശ്രമ-\\
മെന്തുമൂലം? പണ്ടു കണ്ടിട്ടുമില്ല ഞാന്‍.\\
മാര്‍ഗവിഭ്രംശം വരികയോ? കേവല-\\
മോര്‍ക്കണമെന്‍മനോവിഭ്രമമല്ലല്ലീ?\\
നാനാപ്രകാരവും താപസനെക്കണ്ടു\\
പാനീയപാനവും ചെയ്തു ദാഹം തീര്‍ത്തു\\
കാണാം മഹൗഷധം നില്ക്കുമത്യുന്നതം\\
ദ്രോണാചലം രഘുപുംഗവാനുഗ്രഹാല്‍.’\\
ഇത്ഥം നിരൂപിച്ചൊരു യോജനായതം\\
വിസ്താരമാണ്ട മായാശ്രമമശ്രമം\\
രംഭാപനസഖര്‍ജൂരകേരാമ്രാദി\\
സമ്പൂര്‍ണമത്യച്ഛതോയവാപീയുതം\\
കാലനേമിത്രിയാമാചരനും തത്ര\\
ശാലയിലൃത്വിക്സദസ്യാദികളൊടും\\
ഇന്ദ്രയാഗം ദൃഢമാമ്മാറനുഷ്ഠിച്ചു\\
ചന്ദ്രചൂഡപസാദം വരുത്തീടുവാന്‍\\
ഭക്ത്യാ ശിവപൂജയും ചെയ്തു വാഴുന്ന\\
നക്തഞ്ചരേന്ദ്രനാം താപസശ്രേഷ്ഠനെ\\
വീണു നമസ്കാരവും ചെയ്തുടന്‍ ജഗല്‍-\\
പ്രാണതനയനുമിങ്ങനെ ചൊല്ലിനാന്‍:\\
‘രാമദൂതോഹം ഹനുമാനിതി മമ\\
നാമം, പവനജനഞ്ജനാനന്ദനന്‍\\
രാമകാര്യാര്‍ത്ഥമായ് ക്ഷീരാംബുരാശിക്കു\\
സാമോദമിന്നു പോകുന്നു തപോനിധേ!\\
ദേഹരക്ഷാര്‍ത്ഥമിവിടേക്കു വന്നിതു\\
ദാഹം പൊറാഞ്ഞു തണ്ണീര്‍കുടിച്ചീടുവാന്‍.\\
എങ്ങു ജലസ്ഥലമെന്നരുള്‍ചെയ്യണ-\\
മെങ്ങുമേ പാര്‍ക്കരുതെന്നെന്‍ മനോഗതം.’\\
മാരുതി ചൊന്നതു കേട്ടു നിശാചരന്‍\\
കാരുണ്യഭാവം നടിച്ചു ചൊല്ലീടിനാന്‍:\\
‘മാമകമായ കമണ്ഡലുസ്ഥം ജല-\\
മാമയം തീരുവോളം കുടിച്ചീടുക.\\
പക്വഫലങ്ങളും ഭക്ഷിച്ചനന്തരം\\
ദുഃഖം കളഞ്ഞു കുറഞ്ഞൊന്നുറങ്ങുക.\\
ഏതും പരിഭ്രമിക്കേണ്ട ഭവാനിനി-\\
ബ്ഭൂതവും ഭവ്യവും മേലില്‍ ഭവിപ്പതും\\
ദിവ്യദൃശാ കണ്ടറിഞ്ഞിരിക്കുന്നിതു\\
സുവ്യക്തമായതുകൊണ്ടു ചൊല്ലീടുവന്‍.\\
വാനരന്മാരും സുമിത്രാതനയനും\\
മാനവവീരനിരീക്ഷിതരാകയാല്‍\\
മോഹവും തീര്‍ന്നെഴുന്നേറ്റിതെല്ലാവരു-\\
മാഹവത്തിന്നൊരുമിച്ചു നിന്നീടിനാര്‍.’\\
ഇത്ഥമാകര്‍ണ്യ ചൊന്നാന്‍ കപിപുംഗവ-\\
‘നെത്രയും കാരുണ്യശാലിയല്ലോ ഭവാന്‍.\\
പാരം പെരുതു മേ ദാഹമതുകൊണ്ടു\\
പോരാ കമണ്ഡലുസംസ്ഥിതമാം ജലം.’\\
വായുതനയനേവം ചൊന്ന നേരത്തു\\
മായാവിരചിതനായ വടുവിനെ\\
തോയാകരം ചെന്നു കാട്ടിക്കൊടുക്കെന്നു\\
ഭൂയോ മുദാ കാലനേമിയും ചൊല്ലിനാന്‍.\\
‘നേത്രനിമീലനം ചെയ്തു പാനീയവും\\
പീത്വാ മമാന്തികം പ്രാപിക്ക സത്വരം\\
എന്നാല്‍ നിനക്കൗഷധം കണ്ടു കിട്ടുവാ-\\
നിന്നു നല്ലോരു മന്ത്രോപദേശം ചെയ്വന്‍.’\\
എന്നതു കേട്ടു വിശ്വാസേന മാരുതി\\
ചെന്നാനയച്ച വടുവിനോടും മുദാ\\
കണ്ണുമടച്ചു വാപീതടം പ്രാപിച്ചു\\
തണ്ണീര്‍ കുടിപ്പാന്‍ തുടങ്ങും ദശാന്തരേ\\
വന്നു ഭയങ്കരിയായ മകരിയു-\\
മുന്നതനായ മഹാകപിവീരനെ\\
തിന്നു കളവാനൊരുമ്പെട്ട നേരത്തു\\
കണ്ണു മിഴിച്ചു കപീന്ദ്രനും നോക്കിനാന്‍.\\
വക്ത്രം പിളര്‍ന്നു കണ്ടോരു മകരിയെ\\
ഹസ്തങ്ങള്‍ കൊണ്ടു പിളര്‍ന്നാന്‍ കപിവരന്‍\\
ദേഹമുപേക്ഷിച്ചു മേല്പോട്ടു പോയിതു\\
ദേഹിയും മിന്നല്‍പോലെ തദത്യത്ഭുതം.\\
ദിവ്യവിമാനദേശേ കണ്ടിതന്നേരം\\
ദിവ്യരൂപത്തൊടു നാരീമണിയെയും\\
ചേതോഹരാംഗിയാമപ്സരസ്ത്രീമണി\\
വാതാത്മജനോടു ചൊന്നാളതുനേരം:\\
‘നിന്നുടെ കാരുണ്യമുണ്ടാകയാലെനി-\\
ക്കിന്നു വന്നൂ ശാപമോക്ഷം കപിവര!\\
മുന്നമൊരപ്സരസ്ത്രീ ഞാ, നൊരു മുനി-\\
തന്നുടെ ശാപേന രാക്ഷസിയായതും\\
ധന്യമാലീതി മേ നാമം മഹാമതേ!\\
മാന്യനാം നീയിനിയൊന്നു ധരിക്കണം\\
അത്ര പുണ്യാശ്രമേ നീ കണ്ട താപസന്‍\\
നക്തഞ്ചരന്‍ കാലനേമി മഹാഖലന്‍.\\
രാവണപ്രേരിതനായ് വന്നിരുന്നവന്‍\\
താവകമാര്‍ഗവിഘ്നം വരുത്തീടുവാന്‍.\\
താപസവേഷം ധരിച്ചിരിക്കുന്നിതു\\
താപസദേവഭൂദേവാദി ഹിംസകന്‍\\
ദുഷ്ടനെ വേഗം വധിച്ചു കളഞ്ഞിനി-\\
പ്പുഷ്ടമോദം ദ്രോണപര്‍വതം പ്രാപിച്ചു\\
ദിവ്യൗഷധങ്ങളും കൊണ്ടങ്ങു ചെന്നിനി\\
ക്രവ്യാദവംശമശേഷമൊടുക്കുക.\\
ഞാനിനി ബ്രഹ്മലോകത്തിനു പോകുന്നു\\
വാനരവീര! കുശലം ഭവിക്ക തേ!’\\
പോയാളിവണ്ണം പറഞ്ഞവള്‍, മാരുതി\\
മായാവിയാം കാലനേമിതന്നന്തികേ\\
ചെന്നാ, നവനോടു ചൊന്നാനസുരനും:\\
‘വന്നീടുവാനിത്ര വൈകിയതെന്തെടോ?\\
കാലമിനിക്കളയാതെ വരിക നീ\\
മൂലമന്ത്രോപദേശം ചെയ്വനാശു ഞാന്‍\\
ദക്ഷിണയും തന്നഭിവാദ്യവും ചെയ്ക\\
ദക്ഷനായ് വന്നുകൂടും ഭവാന്‍ നിര്‍ണയം.’\\
തല്‍ക്ഷണേ മുഷ്ടിയും ബദ്ധ്വാ ദൃഢതരം\\
രക്ഷഃ പ്രവരോത്തമാംഗേ കപിവരന്‍\\
ഒന്നടിച്ചാനതുകൊണ്ടവനും തദാ\\
ചെന്നു പുക്കീടിനാന്‍ ദര്‍മരാജാലയം.
\end{verse}

%%23_divyavshadhafalam

\section{ദിവ്യൗഷധഫലം}

\begin{verse}
ക്ഷീരാര്‍ണവത്തെയും ദ്രോണാചലത്തെയും\\
മാരുതി കണ്ടു വണങ്ങി നോക്കും വിധൗ\\
ഔഷധാവാസമൃഷഭാദ്രിയും കണ്ടി-\\
തൗഷധമൊന്നുമേ കണ്ടതുമില്ലല്ലോ.\\
കാണാഞ്ഞു കോപിച്ചു പര്‍വതത്തെപ്പറി-\\
ച്ചേണാങ്കബിംബം കണക്കെപ്പിടിച്ചവന്‍\\
കൊണ്ടുവന്നന്‍പോടു രാഘവന്‍ മുമ്പില്‍ വെ-\\
ച്ചിണ്ടല്‍ തീര്‍ത്തീടിനാന്‍ വന്‍പടയ്ക്കന്നേരം.\\
കൊണ്ടല്‍നേര്‍വര്‍ണനും പ്രീതി പൂണ്ടാന്‍\\
നീലകണ്ഠനുമാനന്ദമായ് വന്നിതേറ്റവും.\\
ഔഷധത്തിന്‍കാറ്റു തട്ടിയ നേരത്തു\\
ദോഷമകന്നെഴുനേറ്റിതെല്ലാവരും.\\
‘മുന്നമിരുന്നവണ്ണം തന്നെയാക്കണ-\\
മിന്നുതന്നെ ശൈലമില്ലൊരു സംശയം\\
അല്ലായ്കിലെങ്ങനെ രാത്രിഞ്ചരബലം\\
കൊല്ലുന്നി’തെന്നരുള്‍ ചെയ്തോരനന്തരം\\
കുന്നുമെടുത്തുയര്‍ന്നാന്‍ കപിപുംഗവന്‍;\\
വന്നാനരനിമിഷംകൊണ്ടു പിന്നെയും.\\
യുദ്ധേ മരിച്ച നിശാചരന്മാരുടല്‍\\
നക്തഞ്ചരേന്ദ്രനിയോഗേന രാക്ഷസര്‍\\
വാരാന്നിധിയിലിട്ടീടിനാരെന്നതു-\\
കാരണം ജീവിച്ചതില്ല രക്ഷോഗണം.
\end{verse}
\newpage

%%24_meghanaadavadham

\section{മേഘനാദവധം}

\begin{verse}
രാഘവന്മാരും മഹാകപിവീരരും\\
ശോകമകന്നു തെളിഞ്ഞു വാഴുംവിധൗ\\
മര്‍ക്കടനായകന്മാരോടു ചൊല്ലിനാ-\\
നര്‍ക്കതനയനുമംഗദനും തദാ:\\
‘നില്ക്കരുതാരും പുറത്തിനി വാനര-\\
രൊക്കെക്കടക്ക മുറിക്ക മതിലുകള്‍\\
വയ്ക്ക ഗൃഹങ്ങളിലൊക്കവേ കൊള്ളിയും\\
വൃക്ഷങ്ങളൊക്കെ മുറിക്ക തെരുതെരെ\\
കൂപതടാകങ്ങള്‍ തൂര്‍ക്ക, കിടങ്ങുകള്‍\\
ഗോപുരദ്വാരാവധി നിരത്തീടുക\\
മിക്കതുമൊക്കെയൊടുങ്ങി നിശാചര-\\
രുള്‍ക്കരുത്തുള്ളവരിന്നുമുണ്ടെങിലോ\\
വെന്തു പൊറാഞ്ഞാല്‍ പുറത്തു പുറപ്പെടു-\\
മന്തകന്‍ വീട്ടിന്നയയ്ക്കാമനുക്ഷണം.’\\
എന്നതു കേട്ടവര്‍ കൊള്ളിയും കൈക്കൊണ്ടു\\
ചെന്നു തെരുതെരെ വെച്ചു തുടങ്ങിനാര്‍\\
പ്രാസാദ ഗോപുര ഹര്‍മ്യ ഗേഹങ്ങളും\\
കാസീസകാഞ്ചന രൂപ്യതാമ്രങ്ങളും\\
ആയുധശാലകളാഭരണങ്ങളു-\\
മായതനങ്ങളും മജ്ജനശാലയും\\
വാരണവൃന്ദവും വാജിസമൂഹവും\\
തേരുകളും വെന്തു വെന്തു വീണിടുന്നു.\\
സ്വര്‍ഗലോകത്തോളമെത്തീ ദഹനനും\\
ശക്രനോടങ്ങറിയിപ്പാനനാകുലം\\
മാരുതി ചുട്ടതിലേറെ നന്നായ് ചമ-\\
ച്ചോരു ലങ്കാപുരം ഭൂതിയായ് വന്നിതു.\\
രാത്രിഞ്ചരസ്തീകള്‍ വെന്തലറിപ്പാഞ്ഞു-\\
മാര്‍ത്തിമുഴുത്തു തെരുതെരെച്ചാകയും\\
മാര്‍ത്താണ്ഡഗോത്രജനാകിയ രാഘവന്‍\\
കൂര്‍ത്തുമൂര്‍ത്തുള്ള ശരങ്ങള്‍ പൊഴിക്കയും\\
ഗോത്രാരിജിത്തും ജയിച്ചതുമെത്രയും\\
പാര്‍ത്തോളമത്ഭുതമെന്നു പറകയും\\
രാത്രിഞ്ചരന്മാര്‍നിലവിളിഘോഷവും\\
രാത്രിഞ്ചരസ്ത്രീകള്‍ കേഴുന്ന ഘോഷവും\\
മാനവേന്ദ്രന്‍ ധനുര്‍ജ്ജ്യാനാദഘോഷവും\\
ആനകള്‍ വെന്തലറീടുന്ന ഘോഷവും\\
വാനരന്മാര്‍ നിന്നലുറന്ന ഘോഷവും\\
ദീനതപൂണ്ട തുരഗങ്ങള്‍ നാദവും\\
സന്തതം തിങ്ങി മുഴങ്ങിച്ചമഞ്ഞിതു\\
ചിന്ത മുഴുത്തു ദശാനനവീരനും\\
കുംഭകര്‍ണാത്മജന്മാരില്‍ മുമ്പുള്ളൊരു\\
കുംഭനോടാശു നീ പൊകെന്നു ചൊല്ലിനാന്‍.\\
തമ്പിയായുള്ള നികുംഭനുമന്നേരം\\
മുമ്പില്‍ ഞാനെന്നു മുതിര്‍ന്നു പുറപ്പെട്ടാന്‍.\\
കുംഭനും താനും പ്രജംഘനുമെത്രയും\\
വമ്പുള്ള യൂപാക്ഷനും ശോണിതാക്ഷനും\\
വമ്പടയോടും പുറപ്പെട്ടു ചെന്നള-\\
വിമ്പം കലര്‍ന്നടുത്താര്‍ കപിവീരരും.\\
രാത്രിയിലാര്‍ത്തങ്ങടുത്തു പൊരുതൊരു\\
രാത്രിഞ്ചരന്മാര്‍ തെരുതെരെച്ചാകയും\\
കൂര്‍ത്ത ശസ്ത്രാസ്ത്രങ്ങള്‍കൊണ്ടു കപികളും\\
ഗാത്രങ്ങള്‍ ഭേദിച്ചു ധാത്രിയില്‍ വീഴ്കയും\\
ഏറ്റുപിടിച്ചുമടിച്ചുമിടിച്ചുമ-\\
ങ്ങേറ്റം കടിച്ചും പൊടിച്ചും പരസ്പരം\\
ചീറ്റം മുഴുത്തു പറിച്ചും മരാമരം\\
തോറ്റുപോകായ്കെന്നു ചൊല്ലിയടുക്കയും\\
വാനര രാക്ഷസന്മാര്‍ പൊരുതാരഭി-\\
മാനം നടിച്ചും ത്യജിച്ചും കളേബരം.\\
നാലഞ്ചു നാഴിക നേരം പൊരുതപ്പോള്‍\\
കാലപുരിപുക്കിതേറ്റ രക്ഷോഗണം.\\
കമ്പനന്‍ വന്‍പോടടുത്താനതുനേര-\\
മമ്പുകൊണ്ടേറ്റമകന്നു കപികളും.\\
കമ്പം കലര്‍ന്നൊഴിച്ചാരതു കണ്ടഥ\\
ജംഭാരിനന്ദനപുത്രനും കോപിച്ചു\\
കമ്പനന്‍തന്നെ വധിച്ചോരനന്തരം\\
പിമ്പേ തുടര്‍ന്നങ്ങടുത്താന്‍ പ്രജംഘനും\\
യൂപാക്ഷനും തഥാ ശോണിതനേത്രനും.\\
കോപിച്ചടുത്താരതു നേരമംഗദന്‍\\
കൗണപന്മാര്‍ മൂവരോടും പൊരുതതി-\\
ക്ഷീണനായ് വന്നിതു ബാലിതനയനും.\\
മൈന്ദനുമാശു വിവിദനുമായ്ത്തത്ര\\
മന്ദേതരം വന്നടുത്താരതുനേരം\\
കൊന്നാന്‍ പ്രജംഘനെത്താരേയനുമഥ\\
പിന്നെയവ്വണ്ണം വിവിദന്‍ മഹാബലന്‍\\
കൊന്നിതു ശോണിതനേത്രനെയുമഥ\\
മൈന്ദനും യൂപാക്ഷനെക്കൊന്നു വീഴ്ത്തിനാന്‍.\\
നക്തഞ്ചരവരന്മാരവര്‍ നാല്‍വരും\\
മൃത്യുപുരം പ്രവേശിച്ചോരനന്തരം\\
കുംഭനണഞ്ഞു ശരം പൊഴിച്ചീടിനാന്‍\\
വമ്പരാം വാനരന്മാരൊക്കെ മണ്ടിനാര്‍.\\
സുഗ്രീവനും തേരിലാമ്മാറു ചാടിവീ-\\
ണുഗ്രതയോടവന്‍ വില്‍ കളഞ്ഞീടിനാന്‍.\\
മുഷ്ടിയുദ്ധം ചെയ്തനേരത്തു കുംഭനെ-\\
പ്പെട്ടെന്നെടുത്തെറിഞ്ഞീടിനാനബ്ധിയില്‍\\
വാരാന്നിധിയും കലക്കി മറിച്ചതി-\\
ഘോരനാം കുംഭന്‍ കരേറി വന്നീടിനാന്‍\\
സൂര്യാത്മജനുമതുകണ്ടു കോപിച്ചു\\
സൂര്യാത്മജാലയത്തിന്നയച്ചീടിനാന്‍.\\
സുഗ്രീവനഗ്രജനെക്കൊന്നനേരമ-\\
ത്യുഗ്രന്‍ നികുംഭന്‍ പരിഘവുമായുടന്‍\\
സംഹാരരുദ്രനെപ്പോലെ രണാജിരേ\\
സിംഹനാദംചെയ്തടുത്താനതു നേരം,\\
സുഗ്രീവനെപ്പിന്നിലിട്ടു വാതാത്മജ-\\
നഗ്രേ ചെറുത്താന്‍ നികുംഭനെത്തല്‍ക്ഷണേ.\\
മാരുതിമാറിലടിച്ചാന്‍ നികുംഭനും\\
പാരില്‍ നുറുങ്ങിവീണൂ തല്‍പരിഘവും\\
ഉത്തമാംഗത്തെപ്പറിച്ചെറിഞ്ഞാനതി-\\
ക്രുദ്ധനായോരു ജഗല്‍പ്രാണപുത്രനും.\\
പേടിച്ചുമണ്ടിനാര്‍ ശേഷിച്ച രാക്ഷസര്‍\\
കൂടെത്തുടര്‍ന്നടുത്താര്‍ കപിവീരരും\\
ലങ്കയില്‍ പുക്കടച്ചാരവരും ചെന്നു\\
ലങ്കേശനോടറിയിച്ചാരവസ്ഥകള്‍.\\
കുംഭാദികള്‍ മരിച്ചോരുദന്തം കേട്ടു\\
ജംഭാരിവൈരിയും ഭീതിപൂണ്ടീടിനാന്‍.\\
പിന്നെ ഖരാത്മജനാം മകരാക്ഷനോ-\\
ടന്യൂനകോപേന ചൊന്നാന്‍ ദശാനനന്‍:\\
"ചെന്നു നീ രാമാദികളെജ്ജയിച്ചിങ്ങു\\
വന്നീടു"കെന്ന നേരം മകരാക്ഷനും\\
തന്നുടെ സൈന്യസമേതം പുറപ്പെട്ടു\\
സന്നാഹമോടുമടുത്തു രണാങ്കണേ.\\
പന്നഗതുല്യങ്ങളായ ശരങ്ങളെ\\
വഹ്നികീലാകാരമായ് ചൊരിഞ്ഞീടിനാന്‍.\\
നിന്നുകൂടാഞ്ഞു ഭയപ്പെട്ടു വാനരര്‍\\
ചെന്നഭയം തരികെന്നു രാമാന്തികേ\\
നിന്നുപറഞ്ഞതു കേട്ടളവേ രാമ-\\
ചന്ദ്രനും വില്ലു കുഴിയെക്കുലച്ചുടന്‍\\
വില്ലാളികളില്‍ മുമ്പുള്ളവന്‍ തന്നോടു\\
നില്ലെന്നണഞ്ഞു ബാണങ്ങള്‍ തൂകീടിനാന്‍.\\
ഒന്നിനൊന്നൊപ്പമെയ്താന്‍ മകരാക്ഷനും\\
ഭിന്നമായീ ശരീരം കമലാക്ഷനും\\
അന്യോന്യമൊപ്പം പൊരുതുനില്ക്കുന്നേര-\\
മൊന്നു തളര്‍ന്നു ചമഞ്ഞൂ ഖരാത്മജന്‍.\\
അപ്പോള്‍ കൊടിയും കുടയും കുതിരയും\\
തല്‍പാണിതന്നിലിരുന്നൊരു ചാപവും\\
തേരും പൊടിപെടുത്താനെയ്തു രാഘവന്‍\\
സാരഥിതന്നെയും കൊന്നാനതുനേരം.\\
പാരിലാമ്മാറുചാടിശ്ശൂലവുംകൊണ്ടു\\
പാരമടുത്ത മകരാക്ഷനെത്തദാ\\
പാവകാസ്ത്രംകൊണ്ടു കണ്ഠവും ഛേദിച്ചു\\
ദേവകള്‍ക്കാപത്തുമൊട്ടു തീര്‍ത്തീടിനാന്‍.\\
രാവണിതാനതറിഞ്ഞു കോപിച്ചു വ-\\
ന്നേവരേയും പൊരുതാശുപുറത്താക്കി\\
രാവണനോടറിയിച്ചാനതു കേട്ടു\\
ദേവകുലാന്തകനാകിയ രാവണന്‍\\
ഈരേഴുലോകം നടുങ്ങുംപടി പരി-\\
ചാരകന്മാരോടു കൂടിപ്പുറപ്പെട്ടാന്‍.\\
അപ്പോളതു കണ്ടു മേഘനിനാദനും\\
തല്‍പ്പാദയുഗ്മം പണിഞ്ഞു ചൊല്ലീടിനാന്‍:\\
‘ഇപ്പോളടിയനരികളെ നിഗ്രഹി-\\
ച്ചുള്‍പ്പൂവിലുണ്ടായ സങ്കടം പോക്കുവന്‍\\
അന്തഃപുരം പുക്കിരുന്നരുളീടുക\\
സന്താപമുണ്ടാകരുതിതു കാരണം.’\\
ഇത്ഥം പറഞ്ഞു പിതാവിനെ വന്ദിച്ചു\\
വൃത്രാരിജിത്തും പുറപ്പെട്ടു പോരിനായ്.\\
യുദ്ധോദ്യമം കണ്ടു സൗമിത്രി ചെന്നു കാ-\\
കുല്‍സ്ഥനോടിത്ഥമുണര്‍ത്തിച്ചരുളിനാന്‍:\\
‘നിത്യം മറഞ്ഞുനിന്നിങ്ങനെ രാവണ-\\
പുത്രന്‍ കപിവരന്മാരെയും നമ്മെയും\\
അസ്ത്രങ്ങളെയ്തുടനന്തം വരുത്തുന്ന-\\
തെത്രനാളേക്കു പൊറുക്കണമിങ്ങനെ?\\
ബ്രഹ്മാസ്ത്രമെയ്തു നിശാചരന്മാര്‍ കുല-\\
മുന്മൂലനാശം വരുത്തുക സത്വരം.’\\
സൗമിത്രി ചൊന്നവാക്കിങ്ങനെ കേട്ടഥ\\
രാമഭദ്രസ്വാമിതാനുമരുള്‍ചെയ്തു:\\
‘ആയോധനത്തിങ്കലോടുന്നവരോടു-\\
മായുധം പോയവരോടും വിശേഷിച്ചു\\
നേരേ വരാതവരോടും, ഭയംപൂണ്ടു\\
പാദാന്തികേവന്നുവീഴുന്നവരോടും\\
പൈതാമഹാസ്ത്രം പ്രയോഗിക്കരുതെടോ!\\
പാതകമുണ്ടാമതല്ലായ്കിലേവനും\\
ഞാനിവനോടു പോര്‍ചെയ്വനെല്ലാവരും\\
ദീനതയെന്നിയേ കണ്ടുനിന്നീടുവിന്‍.’\\
എന്നരുള്‍ചെയ്തു വില്ലും കുലച്ചന്തികേ\\
സന്നദ്ധനായതു കണ്ടൊരു രാവണി\\
തല്‍ക്ഷണേ ചിന്തിച്ചു കല്പിച്ചു ലങ്കയില്‍\\
പുക്കു മായാസീതയെത്തേരില്‍വെച്ചുടന്‍\\
പശ്ചിമഗോപുരത്തൂടേ പുറപ്പെട്ടു\\
നിശ്ചലനായ് നിന്നനേരം കപികളും\\
തേരില്‍ മായാസീതയെക്കണ്ടു ദുഃഖിച്ചു\\
മാരുതിതാനും പരവശനായിതു.\\
വാനരവീരരെല്ലാവരും കാണവേ\\
ജാനകീദേവിയെ വെട്ടിനാന്‍ നിര്‍ദയം\\
‘അയ്യോ! വിഭോ! രാമരാമേ’തി വാവിട്ടു\\
മയ്യല്‍മിഴിയാള്‍ മുറവിളിച്ചീടിനാള്‍\\
ചോരയും പാരില്‍ പരന്നിതതുകണ്ടു\\
മാരുതി ജാനകിയെന്നു തേറീടിനാന്‍.\\
ശോഭയില്ലേതും നമുക്കിനി യുദ്ധത്തി-\\
നാപത്തിതില്‍പരമെന്തുള്ളതീശ്വരാ!\\
നാമിനി വാങ്ങുക, സീതാവധം മമ\\
സ്വാമി തന്നോടുണര്‍ത്തിപ്പാന്‍ കപികളേ!’\\
ശാഖാമൃഗാധിപന്മാരെയും വാങ്ങിച്ചു\\
ശോകാതുരനായ മാരുതനന്ദനന്‍\\
ചെല്ലുന്നതു കണ്ടു രാഘവനും തദാ\\
ചൊല്ലിനാന്‍ ജാംബവാന്‍തന്നോടു സാകുലം\\
‘മാരുതിയെന്തുകൊണ്ടിങ്ങോട്ടു പോന്നിതു?\\
പോരില്‍ പുറംതിരിഞ്ഞീടുമാറില്ലവന്‍\\
നീ കൂടെയങ്ങുചെന്നീടുക സത്വരം\\
ലോകേശനന്ദന! പാര്‍ക്കരുതേതുമേ!’\\
ഇത്ഥമാകര്‍ണ്യ വിധിസുതനുംകപി-\\
സത്തമന്മാരുമായ് ചെന്നു ലഘുതരം\\
‘എന്തുകൊണ്ടിങ്ങു വാങ്ങിപ്പോന്നിതു ഭവാന്‍?\\
ബന്ധമെന്തങ്ങോട്ടു തന്നെ നടക്ക നീ.’\\
എന്നനേരം മാരുതാത്മജന്‍ ചൊല്ലിനാ-\\
‘നിന്നു പേടിച്ചു വാങ്ങീടുകയല്ല ഞാന്‍.\\
ഉണ്ടൊരവസ്ഥയുണ്ടായിട്ടതിപ്പൊഴേ\\
ചെന്നു ജഗല്‍സ്വാമിയോടുണര്‍ത്തിക്കണം\\
പോരിക നീയുമിങ്ങോട്ടിനി’ യെന്നുടന്‍\\
മാരുതി ചൊന്നതു കേട്ട,വന്‍ താനുമായ്\\
ചെന്നു തൊഴുതുണര്‍ത്തിച്ചിതു മൈഥിലി\\
തന്നുടെ നാശവൃത്താന്തമെപ്പേരുമേ.\\
ഭൂമിയില്‍ വീണു മോഹിച്ചു രഘൂത്തമന്‍\\
സൗമിത്രിതാനുമന്നേരം തിരുമുടി\\
ചെന്നു മടിയിലെടുത്തുചേര്‍ത്തീടിനാന്‍\\
മന്നവന്‍ തന്‍ പദമഞ്ജനാപുത്രനും\\
ഉത്സംഗസീമനി ചേര്‍ത്താനതു കണ്ടു\\
നിസ്സംജ്ഞരായൊക്കെ നിന്നു കപികളും\\
ദുഃഖം കെടുപ്പതിനായുള്ള വാക്കുക-\\
ളൊക്കെപ്പറഞ്ഞു തുടങ്ങീ കുമാരനും\\
എന്തൊരു ഘോഷമുണ്ടായതെന്നാത്മനി\\
ചിന്തിച്ചവിടേക്കു വന്നു വിഭീഷണന്‍.\\
ചോദിച്ചനേരം കുമാരന്‍ പറഞ്ഞിതു\\
മാതരിശ്വാത്മജന്‍ ചൊന്ന വൃത്താന്തങ്ങള്‍.\\
കയ്യിണകൊട്ടിച്ചിരിച്ചു വിഭീഷണ-\\
‘നയ്യോ! കുരങ്ങന്മാരെന്തറിഞ്ഞു വിഭോ!\\
ലോകേശ്വരിയായ ദേവിയെക്കൊല്ലുവാന്‍\\
ലോകത്രയത്തിങ്കലാരുമുണ്ടായ് വരാ.\\
മായാനിപുണനാം മേഘനിനാദനി-\\
ക്കാര്യമനുഷ്ഠിച്ചതെന്തിനെന്നാശു കേള്‍.\\
മര്‍ക്കടന്മാര്‍ ചെന്നുപദ്രവിച്ചീടാതെ\\
തക്കത്തിലാശു നികുംഭിലയില്‍ ചെന്നു\\
പുക്കുടന്‍ തന്നുടെ ഹോമം കഴിപ്പതി-\\
നായ്ക്കൊണ്ടു കണ്ടോരുപായമത്യത്ഭുതം.\\
ചെന്നിനി ഹോമം മുടക്കേണമല്ലായ്കി-\\
ലെന്നുമവനെ വധിക്കരുതാര്‍ക്കുമേ.\\
രാഘവ! സ്വാമിന്‍! ജയജയ മാനസ-\\
വ്യാകുലം തീര്‍ന്നെഴുന്നേല്ക്ക ദയാനിധേ!\\
ലക്ഷ്മണനുമടിയനുംകപികുല-\\
മുഖ്യപ്രവരരുമായിട്ടു പോകണം\\
ഓര്‍ത്തു കാലം കളഞ്ഞീടരുതേതുമേ\\
യാത്രയയ്ക്കേണ’മെന്നു വിഭീഷണന്‍\\
ചൊന്നതു കേട്ടളവാലസ്യവും തീര്‍ന്നു\\
മന്നവന്‍ പോവാനനുജ്ഞ നല്കീടിനാന്‍.\\
വസ്തുവൃത്താന്തങ്ങളെല്ലാം ധരിച്ച നേ-\\
രത്തു കൃതാര്‍ത്ഥനായ് ശ്രീരാമഭദ്രനും.\\
സോദരന്‍ തന്നെയും രാക്ഷസപുംഗവ-\\
സോദരന്‍തന്നെയും വാനരന്മാരെയും\\
ചെന്നു ദശഗ്രീവനന്ദനന്‍തന്നെയും\\
കൊന്നു വരികെന്നനുഗ്രഹം നല്കിനാന്‍.\\
ലക്ഷമണനോടു മഹാകപിസേനയും\\
രക്ഷോവരനും നടന്നാനതുനേരം\\
മൈന്ദന്‍ വിവിദന്‍ സുഷേണന്‍ നളന്‍ നീല-\\
നിന്ദ്രാത്മജാത്മജന്‍ കേസരി താരനും\\
ശൂരന്‍ വൃഷഭന്‍ ശരഭന്‍ വിനതനും\\
വീരന്‍ പനസന്‍ കുമുദന്‍ വികടനും\\
വാതാത്മജന്‍ വേഗദര്‍ശി വിശാലനും\\
ജ്യോതിര്‍മുഖന്‍ സുമുഖന്‍ ബലിപുംഗവന്‍\\
ശ്വേതന്‍ ദധിമുഖനഗ്നിമുഖന്‍ ഗജന്‍\\
മേദുരന്‍ ധൂമ്രന്‍ ഗവയന്‍ ഗവാക്ഷനും\\
മറ്റുമിത്യാദി ചൊല്ലുള്ള കപികളും\\
മുറ്റും നടന്നിതു ലക്ഷ്മണന്‍തന്നൊടും.\\
മുന്നില്‍ നടന്നു വിഭീഷണന്‍താനുമായ്\\
ചെന്നു നികുംഭില പുക്കു നിറഞ്ഞിതു.\\
നക്തഞ്ചരവരന്മാരെച്ചുഴലവേ\\
നിര്‍ത്തി ഹോമം തുടങ്ങീടിനാന്‍ രാവണി.\\
കല്ലും മലയും മരവുമെടുത്തുകൊ-\\
ണ്ടെല്ലാവരുമായടുത്തു കപികളും\\
എറ്റുമേറുംകൊണ്ടു വീണുതുടങ്ങിനാ-\\
രറ്റമില്ലാതോരോ രാക്ഷസവീരരും.\\
മുറ്റുകയില്ല ഹോമം നമുക്കിങ്ങിനി\\
പ്പറ്റലരെച്ചെറ്റകറ്റിയൊഴിഞ്ഞെന്നു\\
കല്പിച്ചു രാവണി വില്ലും ശരങ്ങളും\\
കെല്പോടെടുത്തു പോരിന്നടുത്തീടിനാന്‍.\\
മുമ്പില്‍ വേഗംപൂണ്ടടുക്കുന്ന മാരുത-\\
സംഭവന്‍തന്നെത്തടുത്തു നിര്‍ത്തീടിനാന്‍\\
വന്നു നികുംഭിലയാല്‍ത്തറമേലേറി\\
നിന്നു ദശാനനപുത്രനുമന്നേരം\\
കണ്ടു വിഭീഷണന്‍ സൗമിത്രിതന്നോടു\\
കുണ്ഠതതീര്‍ത്തു മറഞ്ഞു തുടങ്ങിനാന്‍:\\
‘വീര! കഴിഞ്ഞീല ഹോമമിവനെങ്കില്‍\\
നേരേ വെളിച്ചത്തു കണ്ടുകൂടാ ദൃഢം.\\
മാരുതനന്ദനന്‍ തന്നോടു കോപിച്ചു\\
നേരിട്ടു വന്നതു കണ്ടതില്ലേ ഭവാന്‍?\\
മൃത്യുസമയമടുത്തിതിവന്നിനി\\
യുദ്ധം തുടങ്ങുക വൈകരുതേതുമേ.’\\
ഇത്ഥം വിഭീഷണന്‍ ചൊന്നനേരത്തു സൗ-\\
മിത്രിയുമസ്ത്രശസ്ത്രങ്ങള്‍ തൂകീടിനാന്‍\\
പ്രത്യസ്ത്രശസ്ത്രങ്ങള്‍കൊണ്ടു തടുത്തിന്ദ്ര-\\
ജിത്തുമത്യര്‍ത്ഥമസ്ത്രങ്ങളെയ്തീടിനാന്‍.\\
അപ്പോള്‍ കഴുത്തിലെടുത്തു മരുല്‍സുത-\\
നുത്പന്നമോദം കുമാരനെസ്സാദരം.\\
ലക്ഷ്മണപാര്‍ശ്വേ വിഭീഷണനെക്കണ്ടു\\
തല്‍ക്ഷണം ചൊന്നാന്‍ ദശാനനപുത്രനും:\\
‘രാക്ഷസജാതിയില്‍ വന്നു പിറന്ന നീ\\
സാക്ഷാല്‍ പിതൃവ്യനല്ലോ മമ കേവലം\\
പുത്രമിത്രാദിവര്‍ഗത്തെയൊടുക്കുവാന്‍\\
ശത്രുജനത്തിനു ഭൃത്യനായിങ്ങനെ\\
നിത്യവും വേലചെയ്യുന്നതോര്‍ത്തീടിനാ-\\
ലെത്രയും നന്നുനന്നെന്നതേ ചൊല്ലാവൂ.\\
ഗോത്രവിനാശം വരുത്തും ജനങ്ങള്‍ക്കു\\
പാര്‍ത്തുകണ്ടോളം ഗതിയില്ല നിര്‍ണയം\\
ഊര്‍ദ്ധ്വലോകപ്രാപ്തി സന്തതികൊണ്ടത്രേ\\
സാദ്ധ്യമാകുന്നതെന്നല്ലോ ബുധമതം.\\
ശാസ്ത്രജ്ഞനാം നീ കുലത്തെയൊടുക്കുവാ-\\
നാസ്ഥയാ വേലചെയ്യുന്നതുമത്ഭുതം.’\\
എന്നതു കേട്ടു വിഭീഷണന്‍ ചൊല്ലിനാന്‍:\\
‘നന്നു നീയും നിന്‍ പിതാവുമറിക നീ.\\
വംശം മുടിക്കുന്നതിന്നു നീയേതുമേ\\
സംശയമില്ല വിചാരിക്ക മാനസേ\\
വംശത്തെ രക്ഷിച്ചുകൊള്ളുവനിന്നു ഞാ-\\
നംശുമാലീകുലനായകാനുഗ്രഹാല്‍.’\\
ഇങ്ങനെ തമ്മില്‍ പറഞ്ഞു നില്ക്കുന്നേരം\\
മങ്ങാതെ ബാണങ്ങള്‍ തൂകീ കുമാരനും\\
എല്ലാമതെയ്തുമുറിച്ചു കളഞ്ഞഥ\\
ചൊല്ലിനാനാശു സൗമിത്രിതന്നോടവന്‍:\\
‘രണ്ടു ദിനം മമ ബാഹുപരാക്രമം\\
കണ്ടതില്ലേ നീ കുമാര! വിശേഷിച്ചും?\\
കണ്ടുകൊള്‍കല്ലായ്കിലിന്നു ഞാന്‍ നിന്നുടല്‍-\\
കൊണ്ടു ജന്തുക്കള്‍ക്കു ഭക്ഷണമേകുവന്‍.’\\
ഇത്ഥം പറഞ്ഞേഴു ബാണങ്ങള്‍കൊണ്ടു സൗ-\\
മിത്രിയുടെയുടല്‍ കീറിനാനേറ്റവും.\\
പത്തു ബാണം വായുപുത്രനെയേല്പിച്ചു\\
സത്വരം പിന്നെ വിഭീഷണന്‍ തന്നെയും.\\
നൂറു ശരമെയ്തു വാനരവീരരു-\\
മേറെ മുറിഞ്ഞു വശംകെട്ടു വാങ്ങിനാര്‍.\\
തല്‍ക്ഷണേ ബാണം മഴപൊഴിയുംവണ്ണം\\
ലക്ഷ്മണന്‍ തൂകിനാന്‍ ശക്രാരിമേനിമേല്‍.\\
വൃത്രാരിജിത്തും ശരസഹസ്രേണ സൗ-\\
മിത്രികവചം നുറുക്കിയിട്ടീടിനാന്‍.\\
രക്താഭിഷിക്തശരീരികളായിതു\\
നക്തഞ്ചരനും സുമിത്രാതനയനും.\\
പാരമടുത്തഞ്ചുബാണം പ്രയോഗിച്ചു\\
തേരും പൊടിച്ചു കുതിരകളെക്കൊന്നു\\
സാരഥിതന്റെ തലയും മുറിച്ചതി-\\
സാരമായോരു വില്ലും മുറിച്ചീടിനാന്‍.\\
മറ്റൊരു ചാപമെടുത്തു കുലച്ചവ-\\
നറ്റമില്ലാതോളം ബാണങ്ങള്‍ തൂകിനാന്‍.\\
പിന്നെ മൂന്നമ്പെയ്തതും മുറിച്ചീടിനാന്‍\\
മന്നവന്‍ പംക്തികണ്ഠാത്മജനന്നേരം\\
ഊറ്റമായോരു വില്ലും കുഴിയെക്കുല-\\
ച്ചേറ്റമടുത്തു ബാണങ്ങള്‍ തൂകീടിനാന്‍.\\
സത്വരം ലങ്കയില്‍ പുക്കു തേരും പൂട്ടി\\
വിദ്രുതം വന്നിതു രാവണപുത്രനും.\\
ആരുമറിഞ്ഞീല പോയതും വന്നതും\\
നാരദന്‍ താനും പ്രശംസിച്ചിതന്നേരം.\\
ഘോരമായുണ്ടായ സംഗരം കണ്ടൊരു\\
സാരസസംഭവനാദികള്‍ ചൊല്ലിനാര്‍:\\
‘പണ്ടു ലോകത്തിങ്കലിങ്ങനെയുള്ള പോ-\\
രുണ്ടായതില്ലിനിയുണ്ടാകയുമില്ല.\\
കണ്ടാലുമീദൃഷം വീരപുരുഷന്മാ-\\
രുണ്ടോ ജഗത്തിങ്കല്‍ മറ്റിവരെപ്പോലെ?\\
ഇത്ഥം പലരും പ്രശംസിച്ചു നില്പതിന്‍-\\
മദ്ധ്യേ ദിവസത്രയം കഴിഞ്ഞൂ ഭൃശം.\\
വാസരം മൂന്നു കഴിഞ്ഞോരനന്തരം\\
വാസവദൈവതമസ്ത്രം കുമാരനും\\
ലാഘവം ചേര്‍ന്നു കരേണ ബന്ധിപ്പിച്ചു\\
രാഘവന്‍ തന്‍ പദാംഭോരുഹം മാനസേ\\
ചിന്തിച്ചുറപ്പിച്ചയച്ചാനതു ചെന്നു\\
പംക്തികണ്ഠാത്മജന്‍ കണ്ഠവും ഛേദിച്ചു\\
സിന്ധുജലത്തില്‍ മുഴുകി വിശുദ്ധമാ-\\
യന്തരാ തൂണിയില്‍ വന്നു പുക്കു ശരം\\
ഭൂമിയില്‍ വീണിതു രാവണിതന്നുടെ-\\
ലാമയം തീര്‍ന്നിതു ലോകത്രയത്തിനും\\
സന്തുഷ്ടമാനസന്മാരായ ദേവകള്‍\\
സന്തതം സൗമിത്രിയെ സ്തുതിച്ചീടിനാര്‍.\\
പുഷ്പങ്ങളും വരിഷിച്ചാരുടനുട-\\
നപ്സരസ്ത്രീകളും നൃത്തം തുടങ്ങിനാര്‍.\\
നേത്രങ്ങളായിരവും വിളങ്ങീ തദാ\\
ഗോത്രാരിതാനും പ്രസാദിച്ചിതേറ്റവും\\
താപമകന്നു പുകഴ്ന്നു തുടങ്ങിനാര്‍\\
താപസന്മാരും ഗഗനചരന്മാരും\\
ദുന്ദുഭിനാദവും ഘോഷിച്ചിതേറ്റമാ-\\
നന്ദിച്ചിതാശു വിരഞ്ചനുമന്നേരം.\\
ശങ്കാവിഹീനം ചെറുഞാണൊലിയിട്ടു\\
ശംഖും വിളിച്ചുടന്‍ സിംഹനാദം ചെയ്തു\\
വാനരന്മാരുമായ് വേഗേന സൗമിത്രി\\
മാനവേന്ദ്രന്‍ ചരണാംബുജം കൂപ്പിനാന്‍.\\
ഗാഢമായാലിംഗനംചെയ്തു രാഘവ-\\
നൂഢമോദം മുകര്‍ന്നീടിനാന്‍ മൂര്‍ദ്ധനി.\\
ലക്ഷ്മണനോടു ചിരിച്ചരുളിച്ചെയ്തു:\\
‘ദുഷ്കരമെത്രയും നീ ചെയ്ത കാരിയം\\
രാവണി യുദ്ധേ മരിച്ചതു കാരണം\\
രാവണന്‍ താനും മരിച്ചാനറിക നീ\\
ക്രുദ്ധനായ് നമ്മോടു യുദ്ധത്തിനായ് വരും\\
പുത്രശോകത്താലിനി ദശഗ്രീവനും.
\end{verse}

%%25_raavanavilaapam

\section{രാവണവിലാപം}

\begin{verse}
ഇത്ഥമന്യോന്യം പറഞ്ഞിരിക്കുന്നേരം\\
പുത്രന്‍ മരിച്ചതു കേട്ടൊരു രാവണന്‍\\
വീണിതു ഭൂമിയില്‍ മോഹം കലര്‍ന്നതി-\\
ക്ഷീണനായ് പിന്നെ വിലാപം തുടങ്ങിനാന്‍:\\
‘ഹാ ഹാ കുമാര! മണ്ഡോദരീനന്ദന!\\
ഹാ ഹാ സുകുമാര! വീര! മനോഹര!\\
മല്‍ക്കര്‍മദോഷങ്ങളെന്തു ചൊല്ലാവതു\\
ദുഃഖമിതെന്നു മറക്കുന്നതുള്ളില്‍ ഞാന്‍!\\
വിണ്ണവര്‍ക്കും ദ്വിജന്മാര്‍ക്കും മുനിമാര്‍ക്കു-\\
മിന്നു നന്നായുറങ്ങീടുമാറായിതു.\\
നമ്മെയും പേടിയില്ലാര്‍ക്കുമിനി മമ\\
ജന്മവും നിഷ്ഫലമായ് വന്നതീശ്വരാ!’\\
പുത്രഗുണങ്ങള്‍ പറഞ്ഞും നിരൂപിച്ചു-\\
മത്തല്‍ മുഴുത്തു കരഞ്ഞു തുടങ്ങിനാന്‍:\\
‘എന്നുടെ പുത്രന്‍ മരിച്ചതു ജാനകി-\\
തന്നുടെ കാരണമെന്നതുകൊണ്ടു ഞാന്‍\\
കൊന്നവള്‍തന്നുടെ ചോര കുടിച്ചൊഴി-\\
ഞ്ഞെന്നുമേ ദുഃഖമടങ്ങുകയില്ല മേ.’\\
ഖഡ്ഗവുമോങ്ങിച്ചിരിച്ചലറിത്തത്ര\\
നിര്‍ഗമിച്ചീടിനാന്‍ ക്രുദ്ധനാം രാവണന്‍.\\
സീതയും ദുഷ്ടനാം രാവണനെക്കണ്ടു\\
ഭീതയായെത്രയും വേപഥുഗാത്രിയായ്\\
ഹാ! രാമ! രാമ! രാമേതി ജപത്തൊടു-\\
മാരാമദേശേ വസിക്കും ദശാന്തരേ\\
ബുദ്ധിമാനായ സുപാര്‍ശ്വന്‍ നയജ്ഞന-\\
ത്യുത്തമന്‍ കര്‍ബുരസത്തമന്‍ വൃത്തവാന്‍\\
രാവണന്‍ തന്നെത്തടുത്തു നിര്‍ത്തിപ്പറ-\\
യാവതെല്ലാം പറഞ്ഞീടിനാന്‍ നീതികള്‍:\\
‘ബ്രഹ്മകുലത്തില്‍ ജനിച്ച ഭവാനിഹ\\
നിര്‍മലനെന്നു ജഗത്ത്രയസമ്മതം.\\
താവകമായ ഗുണങ്ങള്‍ വര്‍ണിപ്പതി-\\
നാവതല്ലോര്‍ക്കില്‍ ഗുഹനുമനന്തനും.\\
ദേവദേവേശ്വരനായ പുരവൈരി-\\
സേവകന്മാരില്‍ പ്രധാനനല്ലോ ഭവാന്‍.\\
പൗലസ്ത്യനായ കുബേരസഹോദരന്‍\\
ത്രൈലോക്യവന്ദ്യനാം പുണ്യജനാധിപന്‍\\
സാമവേദജ്ഞന്‍ സമസ്തവിദ്യാലയന്‍\\
വാമദേവാധിവാസാത്മാ ജിതേന്ദ്രിയന്‍\\
വേദവിദ്യാവ്രതസ്നാനപരായണന്‍\\
ബോധവാന്‍ ഭാര്‍ഗവശിഷ്യന്‍ വിനയവാന്‍.\\
എന്നിരിക്കെബ്ഭവാനിന്നു യുദ്ധാന്തരേ\\
നന്നുനന്നെത്രയുമോര്‍ത്തു കല്പിച്ചതും\\
സ്ത്രീവധമാകിയ കര്‍മത്തിനാശു നീ\\
ഭാവിച്ചതുംതവ ദുഷ്കീര്‍ത്തിവര്‍ദ്ധനം.\\
രാത്രിഞ്ചരേന്ദ്രപ്രവര! പ്രഭോ! മയാ-\\
സാര്‍ദ്ധം വിരവോടു പോരിക പോരിനായ്.\\
മാനവന്മാരെയും വാനരന്മാരെയും\\
മാനേന പോര്‍ചെയ്തു കൊന്നുകളഞ്ഞു നീ\\
ജാനകീദേവിയെ പ്രാപിച്ചുകൊള്ളുക\\
മാനസതാപവും ദൂരെ നീക്കീടുക.’\\
നീതിമാനായ സുപാര്‍ശ്വന്‍ പറഞ്ഞതു\\
യാതുധാനാധിപന്‍ കേട്ടു സന്തുഷ്ടനായ്\\
ആസ്ഥാനമണ്ഡപേ ചെന്നിരുന്നെത്രയു-\\
മാസ്ഥയാ മന്ത്രികളോടും നിരൂപിച്ചു\\
ശിഷ്ടരായുള്ള നിശാചരന്മാരുമായ്\\
പുഷ്ടരോഷം പുറപ്പെട്ടിതു പോരിനായ്\\
ചെന്നു രക്ഷോബലം രാമനോടേറ്റള-\\
വൊന്നൊഴിയാതെയൊടുക്കിനാന്‍ രാമനും.\\
മന്നവന്‍ തന്നോടെതിര്‍ത്തിതു രാവണന്‍\\
നിന്നു പോര്‍ചെയ്താനഭേദമായ് നിര്‍ഭയം.\\
പിന്നെ രഘൂത്തമന്‍ ബാണങ്ങളെയ്തെയ്തു\\
ഭിന്നമാക്കീടിനാന്‍ രാവണദേഹവും.\\
പാരം മുറിഞ്ഞു തളര്‍ന്നു വശംകെട്ടു\\
ധീരതയും വിട്ടു വാങ്ങീ ദശാനനന്‍.\\
പോരുമിനി മമ പോരുമെന്നോര്‍ത്തതി-\\
ഭീരുവായ് ലങ്കാപുരം പുക്കനന്തരം.
\end{verse}

%%26_raavanantehomavignam

\section{രാവണന്റെ ഹോമവിഘ്നം}

\begin{verse}
ശുക്രനെച്ചെന്നു നമസ്കരിച്ചെത്രയും\\
ശുഷ്കവദനനായ് നിന്നു ചൊല്ലീടിനാന്‍:\\
‘അര്‍ക്കാത്മജാദിയാം മര്‍ക്കടവീരരു-\\
മര്‍ക്കാന്വയോദ്ഭൂതനാകിയ രാമനും\\
ഒക്കെയൊരുമിച്ചു വാരിധിയും കട-\\
ന്നിക്കരവന്നു ലങ്കാപുരം പ്രാപിച്ചു\\
ശക്രാരിമുഖ്യനിശാചരന്മാരെയു-\\
മൊക്കെയൊടുക്കി ഞാനേകാകിയായിതു.\\
ദുഃഖവുമുള്‍ക്കൊണ്ടിരിക്കുമാറായിതു\\
സദ്ഗുരോ! ഞാന്‍ തവ ശിഷ്യനല്ലോ വിഭോ!\\
വിജ്ഞാനിയാകിയ രാവണനാലിതി-\\
വിജ്ഞാപിതനായ ശുക്രമഹാമുനി\\
രാവണനോടുപദേശിച്ചി‘തെങ്കില്‍ നീ\\
ദേവതമാരെ പ്രസാദം വരുത്തുക.\\
ശീഘ്രമൊരു ഗുഹയും തീര്‍ത്തു ശത്രുക്കള്‍-\\
തോല്ക്കും പ്രകാരമതിരഹസ്യസ്ഥലേ\\
ചെന്നിരുന്നാശു നീ ഹോമം തുടങ്ങുക;\\
വന്നുകൂടും ജയമെന്നാല്‍ നിനക്കെടോ!\\
വിഘ്നംവരാതെ കഴിഞ്ഞുകൂടുന്നാകി-\\
ലഗ്നികുണ്ഡത്തിങ്ങല്‍ നിന്നു പുറപ്പെടും\\
ബാണതൂണീരചാപാശ്വരഥാദികള്‍\\
വാനവരാലുമജയ്യനാം പിന്നെ നീ.\\
മന്ത്രം ഗ്രഹിച്ചുകൊള്‍കെന്നോടു സാദര-\\
മന്തരമെന്നിയേ ഹോമം കഴിക്ക നീ.’\\
ശുക്രമുനിയോടു മൂലമന്ത്രംകേട്ടു\\
രക്ഷോഗണാധിപനാകിയ രാവണന്‍\\
പന്നഗലോകസമാനമായ് തീര്‍ത്തിതു\\
തന്നുടെ മന്ദിരം തന്നില്‍ ഗുഹാതലം.\\
ദിവ്യമാം ഹവ്യഗവ്യാദി ഹോമായ സ-\\
ദ്രവ്യങ്ങള്‍ തത്ര സമ്പാദിച്ചുകൊണ്ടവന്‍\\
ലങ്കാപുരദ്വാരമൊക്കെ ബന്ധിച്ചതില്‍\\
ശങ്കാവിഹീനമകംപുക്കു ശുദ്ധനായ്\\
ധ്യാനമുറപ്പിച്ചു തല്‍ഫലം പ്രാര്‍ഥിച്ചു\\
മൗനവും ദീക്ഷിച്ചു ഹോമം തുടങ്ങിനാന്‍.\\
വ്യോമമാര്‍ഗത്തോളമുത്ഥിതമായൊരു\\
ഹോമധൂമം കണ്ടു രാവണസോദരന്‍\\
രാമചന്ദ്രന്നു കാടിക്കൊടുത്തീടിനാന്‍.\\
‘ഹോമം തുടങ്ങി ദശാനനന്‍ മന്നവ!\\
ഹോമം കഴിഞ്ഞു കൂടീടുകിലെന്നുമേ\\
നാമവനോടു തോറ്റീടും മഹാരണേ.\\
ഹോമം മുടക്കുവാനായയച്ചീടുക\\
സാമോദമാശു കപികുലവീരരെ.’\\
ശ്രീരാമസുഗ്രീവശാസനം കൈക്കൊണ്ടു\\
മാരുതപുത്രാംഗദാദികളൊക്കവേ\\
നൂറുകോടിപ്പടയോടും മഹാമതി-\\
ലേറിക്കടന്നങ്ങു രാവണമന്ദിരം\\
പുക്കു പുരപാലകന്മാരെയുംകൊന്നു\\
മര്‍ക്കടവീരരൊരുമിച്ചനാകുലം\\
വാരണവാജിരഥങ്ങളെയും പൊടി-\\
ച്ചാരാഞ്ഞു തത്ര ദശാസ്യഹോമസ്ഥലം.\\
വ്യാജാല്‍ സരമ നിജകരസംജ്ഞയാ\\
സൂചിച്ചിതു ദശഗ്രീവഹോമസ്ഥലം.\\
ഹോമഗുഹാദ്വാരബന്ധനപാഷാണ\\
മാമയഹീനം പൊടിപെടുത്തംഗദന്‍\\
തത്ര ഗുഹയിലകം പുക്ക നേരത്തു\\
നക്തഞ്ചരേന്ദ്രനെക്കാണായിതന്തികേ.\\
മറ്റുള്ളവര്‍കളുമംഗദാനുജ്ഞയാ\\
തെറ്റെന്നു ചെന്നു ഗുഹയിലിറങ്ങിനാര്‍.\\
കണ്ണുമടച്ചുടന്‍ ധ്യാനിച്ചിരിക്കുമ-\\
പ്പുണ്യജനാധിപനെക്കണ്ടു വാനരര്‍\\
താഡിച്ചു താഡിച്ചു ഭൃത്യജനങ്ങളെ-\\
പ്പീഡിച്ചു കൊല്കയും സംഭാരസഞ്ചയം\\
കുണ്ഡത്തിലൊക്കെയൊരിക്കലേ ഹോമിച്ചു\\
ഖണ്ഡിച്ചിതു ലഘുമേഖലാജാലവും.\\
രാവണന്‍ കൈയിലിരുന്ന മഹാസ്രുവം\\
പാവനി ശീഘ്രം പിടിച്ചു പറിച്ചുടന്‍\\
താഡനംചെയ്താനതുകോണ്ടു സത്വരം\\
ക്രീഡയാ വാനരശ്രേഷ്ഠന്‍ മഹാബലന്‍.\\
ദന്തങ്ങള്‍കൊണ്ടും നഖങ്ങള്‍കൊണ്ടും ദശ-\\
കന്ധരവിഗ്രഹം കീറിനാനേറ്റവും.\\
ധ്യാനത്തിനേതുമിളക്കമുണ്ടായീല\\
മാനസേ രാവണനും ജയകാംക്ഷയാ.\\
മണ്ഡോദരിയെപ്പിടിച്ചു വലിച്ചു ത-\\
ന്മണ്ഡനമെല്ലാം നുറുക്കിയിട്ടീടിനാന്‍.\\
വിസ്രസ്തനീവിയായ് കഞ്ചുകഹീനയായ്\\
വിത്രസ്തയായ് വിലാപം തുടങ്ങീടിനാള്‍:\\
‘വാനരന്മാരുടെ തല്ലുകൊണ്ടീടുവാന്‍\\
ഞാനെന്തു ദുഷ്കൃതം ചെയ്തതു ദൈവമേ!\\
നാണം നിനക്കില്ലയോ രാക്ഷസേശ്വര?\\
മാനം ഭവാനോളമില്ല മറ്റാര്‍ക്കുമേ.\\
നിന്നുടെ മുമ്പിലിട്ടാശു കപിവര-\\
രെന്നെത്തലമുടി ചുറ്റിപ്പിടിപെട്ടു\\
പാരിലിഴയ്ക്കുന്നതും കണ്ടിരിപ്പതു-\\
പോരേ പരിഭവമോര്‍ക്കില്‍ ജളമതേ!\\
എന്തിനായ്ക്കൊണ്ടു നിന്‍ ധ്യാനവും ഹോമവു-\\
മന്തര്‍ഗതമിനിയെന്തോന്നു ദുര്‍മതേ!\\
ജീവിതാശാ തേ ബലീയസീ മാനസേ\\
ഹാ! വിധിവൈഭവമെത്രയുമത്ഭുതം.\\
അര്‍ദ്ധം പുരുഷനു ഭാര്യയല്ലോ ഭുവി\\
ശത്രുക്കള്‍ വന്നവളെപ്പിടിച്ചെത്രയും\\
ബദ്ധപ്പെടുത്തുന്നതും കണ്ടിരിക്കയില്‍\\
മൃത്യു ഭവിക്കുന്നതുത്തമമേവനും\\
നാണവും പത്നിയും വേണ്ടീലിവന്നു തന്‍-\\
പ്രാണഭയംകൊണ്ടു മൂഢന്‍ മഹാഖലന്‍.’\\
ഭാര്യാവിലാപങ്ങള്‍ കേട്ടു ദശാനനന്‍\\
ധൈര്യമകന്നുതന്‍ വാളുമായ് സത്വരം\\
അംഗദന്‍ തന്നോടടുത്താനതു കണ്ടു\\
തുംഗശരീരികളായ കപികളും\\
രാത്രിഞ്ചരേശ്വരപത്നിയേയുമയ-\\
ച്ചാര്‍ത്തുവിളിച്ചു പുറത്തുപോന്നീടിനാര്‍.\\
ഹോമമശേഷം മുടക്കി വയമെന്നു\\
രാമാന്തികേ ചെന്നു കൈതൊഴുതീടിനാര്‍.\\
മണ്ഡോദരിയോടനുസരിച്ചന്നേരം\\
പണ്ഡിതനായ ദശാസ്യനും ചൊല്ലിനാന്‍:\\
‘നാഥേ! ധരിക്ക ദൈവാധീനമൊക്കെയും\\
ജാതനായാല്‍ മരിക്കുന്നതിന്‍ മുന്നമേ\\
കല്പിച്ചതെല്ലാമനുഭവിച്ചീടണ-\\
മിപ്പോളനുഭവമിത്തരം മാമകം.\\
ജ്ഞാനമാശ്രിത്യ ശോകം കളഞ്ഞീടു നീ\\
ജ്ഞാനവിനാശനം ശോകമറിക നീ\\
അജ്ഞാനസംഭവം ശോകമാകുന്നതു-\\
മജ്ഞാനജാതമഹങ്കാരമായതും.\\
നശ്വരമായ ശരീരാദികളിലേ\\
വിശ്വാസവും പുനരജ്ഞാനസംഭവം\\
ദേഹമൂലം പുത്രദാരാദിബന്ധവും\\
ദേഹിക്കു സംസാരവുമതുകാരണം.\\
ശോകഭയക്രോധലോഭമോഹസ്പൃഹാ-\\
രാഗഹര്‍ഷാദി ജരാമൃത്യു ജന്മങ്ങള്‍\\
അജ്ഞാനജങ്ങളഖിലജന്തുക്കള്‍ക്കു-\\
മജ്ഞാനമെല്ലാമകലെക്കളക നീ.\\
ജ്ഞാനസ്വരൂപനാത്മാപരനദ്വയ\\
നാനന്ദപൂര്‍ണസ്വരൂപനലേപകന്‍\\
ഒന്നിനോടില്ല സംയോഗമതിന്നു മ-\\
റ്റൊന്നിനോടില്ല വിയോഗമൊരിക്കലും.\\
ആത്മാനമിങ്ങനെ കണ്ടു തെളിഞ്ഞുട-\\
നാത്മനി ശോകം കളക നീ വല്ലഭേ!\\
ഞാനിനി ശ്രീരാമലക്ഷ്മണന്മാരെയും\\
വാനരന്മാരെയും കൊന്നു വന്നീടുവന്‍.\\
അല്ലായ്കിലോ രാമസായകമേറ്റു കൈ-\\
വല്യവും പ്രാപിപ്പനില്ലൊരു സംശയം\\
എന്നെ രാമന്‍ കൊലചെയ്യുകില്‍ സീതയെ-\\
ക്കൊന്നുകളഞ്ഞുടനെന്നോടു കൂടവേ\\
പാവകന്‍ തങ്കല്‍പതിച്ചു മരിക്ക നീ\\
ഭാവനയോടുമെന്നാല്‍ ഗതിയും വരും.’\\
വ്യഗ്രിച്ചതു കേട്ടു മണ്ഡോദരിയും ദ-\\
ശഗ്രീവനോടു പറഞ്ഞാളതുനേരം:\\
‘രാഘവനെജ്ജയിപ്പാനരുതാര്‍ക്കുമേ\\
ലോകത്രയത്തിങ്കലെന്നു ധരിക്ക നീ\\
സാക്ഷാല്‍ പ്രധാന പുരുഷോത്തമനായ\\
മോക്ഷദന്‍ നാരായണന്‍ രാമനായതും.\\
ദേവന്‍ മകരാവതാരമനുഷ്ഠിച്ചു\\
വൈവസ്വതമനുതന്നെ രക്ഷിച്ചതും\\
രാജീവലോചനന്‍ മുന്നമൊരു ലക്ഷ-\\
യോജന വിസ്തൃതമായൊരു കൂര്‍മമായ്\\
ക്ഷീരസമുദ്രമഥനകാലേ പുരാ\\
ഘോരമാം മന്ദരം പൃഷ്ഠേ ധരിചതും\\
പന്നിയായ് മുന്നം ഹിരണ്യാക്ഷനെക്കൊന്നു\\
മന്നിടം തേറ്റമേല്‍ വെച്ചു പൊങ്ങിച്ചതും\\
ഘോരനായോരു ഹിരണ്യകശിപുതന്‍\\
മാറിടം കൈനഖം കൊണ്ടു പിളര്‍ന്നതും\\
മൂന്നടി മണ്ണു ബലിയോടും യാചിച്ചു\\
മൂലോകവും മൂന്നടിയായളന്നതും\\
ക്ഷത്രിയരായ് പിറന്നോരസുരന്മാരെ\\
യുദ്ധേ വധിപ്പതിന്നായ് ജമദഗ്നിതന്‍\\
പുത്രനായ് രാമനാമത്തെ ധരിച്ചതും\\
പൃത്ഥ്വീപതിയായ രാമനിവന്‍ തന്നെ\\
മാര്‍ത്താണ്ഡവംശേ ദശരഥപുത്രനായ്\\
ധാത്രീസുതാവരനാകിയ രാഘവന്‍\\
നിന്നെ വധിപ്പാന്‍ മനുഷ്യനായ് ഭൂതലേ\\
വന്നു പിറന്നതുമെന്നു ധരിക്ക നീ\\
പുത്രവിനാശം വരുത്തുവാനും തവ\\
മൃത്യുഭവിപ്പാനുമായ് നീയവനുടെ\\
വല്ലഭയെക്കട്ടുകൊണ്ടുപോന്നു വൃഥാ\\
നിര്‍ലജ്ജനാകയാല്‍ മൂഢ! ജളപ്രഭോ!\\
വൈദേഹിയെക്കൊടുത്തീടുക രാമനു\\
സോദരനായ്ക്കൊണ്ടു രാജ്യവും നല്കുക.\\
രാമന്‍ കരുണാകരന്‍ പുനരെത്രയും\\
നാമിനിക്കാനനം വാഴ്ക തപസ്സിനായ്.’\\
മണ്ഡോദരീവാക്കു കേട്ടൊരു രാവണന്‍\\
ചണ്ഡപരാക്രമന്‍ ചൊന്നാനതുനേരം:\\
‘പുത്രമിത്രാമാത്യസോദരന്മാരെയും\\
മൃത്യുവരുത്തി ഞാനേകനായ് കാനനേ\\
ജീവിച്ചിരിക്കുന്നതും ഭംഗിയല്ലെടോ\\
ഭാവിച്ചവണ്ണം ഭവിക്കയില്ലൊന്നുമേ\\
രാഘവന്‍ തന്നോടെതിര്‍ത്തു യുദ്ധം ചെയ്തു\\
വൈകുണ്ഠരാജ്യമനുഭവിച്ചീടുവന്‍.’
\end{verse}

%%27_raamaraavanayudham

\section{രാമരാവണയുദ്ധം}

\begin{verse}
ഇത്ഥം പറഞ്ഞു യുദ്ധത്തിനൊരുമ്പെട്ടു\\
ബദ്ധമോദം പുറപ്പെട്ടിതു രാവണന്‍\\
മൂലബലാദികള്‍ സംഗരത്തിന്നു തല്‍-\\
കാലേ പുറപ്പെട്ടുവന്നിതു ഭൂതലേ.\\
ലങ്കാധിപന്നു സഹായമായ വേഗേന\\
സംഖ്യയില്ലാത ചതുരംഗസേനയും\\
പത്തു പടനായകന്മാരുമൊന്നിച്ചു\\
പത്തു കഴുത്തനെ കൂപ്പിപ്പുറപ്പേട്ടാര്‍.\\
വാരിധിപോലെ പരന്നു വരുന്നതു\\
മാരുതിമുമ്പാം കപികള്‍ കണ്ടെത്രയും\\
ഭീതി മുഴുത്തു വാങ്ങീടുന്നതു കണ്ടു\\
നീതിമാനാകിയ രാമനും ചൊല്ലിനാന്‍:\\
‘വാനരവീരരേ! നിങ്ങളിവരോടു\\
മാനം നടിച്ചുചെന്നേല്ക്കരുതാരുമേ\\
ഞാനിവരോടു പോര്‍ചെയ്തൊടുക്കീടുവ-\\
നാനന്ദമുള്‍ക്കൊണ്ടു കണ്ടുകൊള്‍കേവരും.’\\
എന്നരുള്‍ചെയ്തു നിശാചരസേനയില്‍\\
ചെന്നു ചാടീടിനാനേകനാമീശ്വരന്‍\\
ചാപബാണങ്ങളും കൈക്കൊണ്ടു രാഘവന്‍\\
കോപേന ബാണജാലങ്ങള്‍ തൂകീടിനാന്‍.\\
എത്ര നിശാചരരുണ്ടു വന്നേറ്റതി-\\
ങ്ങത്ര രാമന്മാരുമുണ്ടെന്നതുപോലെ\\
രാമമയമായ് ചമഞ്ഞിതു സംഗ്രാമ-\\
ഭൂമിയുമെന്തൊരു വൈഭവമന്നേരം!\\
‘എന്നോടുതന്നേ പൊരുന്നിതു രാഘവ’-\\
നെന്നു തോന്നീ രജനീചരര്‍ക്കൊക്കവേ\\
ദ്വാദശനാഴികനേരമൊരുപോലെ\\
യാതുധാനാവലിയോടു രഘൂത്തമന്‍\\
അസ്ത്രം വരിഷിച്ച നേരമാര്‍ക്കും തത്ര\\
ചിത്തേ തിരിച്ചറിയായതില്ലേതുമേ.\\
വാസരരാത്രി നിശാചരവാനര\\
മേദിനിവാരിധി ശൈലവനങ്ങളും\\
ഭേദമില്ലാതെ ശരങ്ങള്‍ നിറഞ്ഞിതു;\\
മേദുരന്മാരായ രാക്ഷസവീരരും\\
ആനയും തേരും കുതിരയും കാലാളും\\
വീണു മരിച്ചു നിറഞ്ഞിതു പോര്‍ക്കളം.\\
കാളിയും കൂളികളും കബന്ധങ്ങളും\\
കാളനിശീഥിനിയും പിശാചങ്ങളും\\
നായും നരിയും കഴുകുകള്‍ കാകങ്ങള്‍\\
പേയും പെരുത്തു ഭയങ്കരമാംവണ്ണം\\
രാമചാപത്തിന്‍ മണിതന്‍ നിനാദവും\\
വ്യോമമാര്‍ഗേ തുടരെത്തുടരെക്കേട്ടു\\
ദേവഗന്ധര്‍വയക്ഷാപ്സരോവൃന്ദവും\\
ദേവമുനീന്ദ്രനാം നാരദനും തദാ\\
രാഘവന്‍തന്നെ സ്തുതിച്ചു തുടങ്ങിനാ-\\
രാകാശചാരികളാനന്ദപൂര്‍വകം.\\
ദ്വാദശനാഴികകൊണ്ടു നിശാചരര്‍\\
മേദിനിതന്നില്‍ വീണീടിനാരൊക്കവേ.\\
മേഘത്തിനുള്ളില്‍നിന്നര്‍ക്കബിംബം പേലെ\\
രാഘവന്‍ തന്നെയും കാണായിതന്നേരം.\\
ലക്ഷ്മണന്‍ താനും വിഭീഷണനും പുന-\\
രര്‍ക്കതനയനും മാരുതപുത്രനും\\
മറ്റുള്ള വാനരവീരരും വന്ദിച്ചു\\
ചുറ്റും നിറഞ്ഞിതു, രാഘവനന്നേരം.\\
മര്‍ക്കടനായകന്മാരോടരുള്‍ചെയ്തി-\\
‘തിക്കണക്കേ യുദ്ധമാശു ചെയ്തീടുവാന്‍\\
നാരായണനും പരമേശനുമൊഴി-\\
ഞ്ഞാരുമില്ലെന്നു കേള്‍പ്പുണ്ടു ഞാന്‍ മുന്നമേ.’\\
രാക്ഷസരാജ്യം മുഴുവനതുനേരം\\
രാക്ഷസസ്ത്രീകള്‍ മുറവിളി കൂട്ടിനാര്‍.\\
‘താത! സഹോദര! നന്ദന! വല്ലഭ!\\
നാഥ! നമുക്കവലംബനമാരയ്യോ!\\
വൃദ്ധയായേറ്റം വിരൂപയായുള്ളൊരു\\
നക്തഞ്ചരാധിപസോദരി രാമനെ\\
ശ്രദ്ധിച്ചകാരണമാപത്തിതൊക്കവേ\\
വര്‍ദ്ധിച്ചു വന്നതു മറ്റില്ല കാരണം.\\
ശൂര്‍പ്പണഖയ്ക്കെന്തു കുറ്റമതില്‍പ്പരം\\
പേപ്പെരുമാളല്ലയോ ദശകന്ധരന്‍!\\
ജാനകിയെക്കൊതിച്ചാശു കുലംമുടി-\\
ച്ചാനൊരു മൂഢന്‍ മഹാപാപി രാവണന്‍\\
അര്‍ദ്ധപ്രഹരമാത്രേണ ഖരാദിയെ\\
യുദ്ധേ വധിച്ചതും വൃത്രാരിപുത്രനെ\\
മൃത്യുവരുത്തി വാഴിച്ചു സുഗ്രീവനെ\\
സത്വരം വാനരന്മാരെയയച്ചതും\\
മാരുതി വന്നിവിടെച്ചെയ്ത കര്‍മവും\\
വാരിധിയില്‍ ചിറകെട്ടിക്കടന്നതും\\
കണ്ടിരിക്കെ നന്നുതോന്നുന്നതെത്രയു-\\
മുണ്ടോ വിചാരമാപത്തിങ്കലുണ്ടാവൂ?\\
സിദ്ധമല്ലായ്കില്‍ വിഭീഷണന്‍ ചൊല്ലിനാന്‍\\
മത്തനായന്നതും ധിക്കരിച്ചീടിനാന്‍.\\
ഉത്തമന്‍ നല്ല വിവേകി വിഭീഷണന്‍\\
സത്യവ്രതന്‍ മേലില്‍ നന്നായ് വരുമവന്‍.\\
നീചനിവന്‍ കുലമൊക്കെ മുടിപ്പതി-\\
നാചരിച്ചാനിതു തന്മരണത്തിനും.\\
നല്ല സുതന്മാരെയും തമ്പിമാരെയും\\
കൊല്ലിച്ചു മറ്റുള്ളമാത്യജനത്തെയും\\
എല്ലാമനുഭവിച്ചീടുവാന്‍ പണ്ടു താന്‍\\
വല്ലായ്മചെയ്തതുമെല്ലാം മറന്നിതോ?\\
ബ്രഹ്മസ്വമായതും ദേവസ്വമായതും\\
നിര്‍മരിയാദമടക്കിനാനേറ്റവും.\\
നാട്ടിലിരിക്കും പ്രജകളെ പീഡിച്ചു\\
കാട്ടിലാക്കിച്ചമച്ചീടിനാന്‍ കശ്മലന്‍\\
അര്‍ത്ഥമന്യായേന നിത്യമാര്‍ജിക്കയും\\
മിത്രജനത്തെ വെറുത്തു ചമയ്ക്കയും\\
ബ്രാഹ്മണരെക്കൊലചെയ്കയും മറ്റുള്ള\\
ധാര്‍മികന്മാര്‍ മുതലൊക്കെയടക്കയും\\
പാരം ഗുരുജനദോഷവുമുണ്ടിവ-\\
നാരെയുമില്ല കൃപയുമൊരിക്കലും.\\
ഇമ്മഹാപാപി ചെയ്തൊരു കര്‍മത്തിനാല്‍\\
നമ്മെയും ദുഃഖിക്കുമാറാക്കിനാനിവന്‍.’\\
ഇത്ഥം പുരസ്ത്രീജനത്തിന്‍ വിലാപങ്ങള്‍\\
നക്തഞ്ചരാധിപന്‍ കേട്ടു ദുഃഖാര്‍ത്തനായ്\\
‘ശത്രുക്കളെക്കൊന്നൊടുക്കുവാനിന്നിനി\\
യുദ്ധത്തിനാശു പുറപ്പെടുകെങ്കില്‍ നാം.’\\
എന്നതു കേട്ടു വിരൂപാക്ഷനുമതിന്‍-\\
മുന്നേ മഹോദരനും മഹാപാര്‍ശ്വനും\\
ഉത്തരഗോപുരത്തൂടെ പുറപ്പെട്ടു\\
ശസ്ത്രങ്ങള്‍ തൂകിത്തുടങ്ങിനാരേറ്റവും.\\
ദുര്‍ന്നിമിത്തങ്ങളുണ്ടായതനാദരി-\\
ച്ചുന്നതനായ നിശാചരനായകന്‍\\
ഗോപുരവാതില്‍ പുറപ്പെട്ടു നിന്നിതു\\
ചാപലമെന്നിയേ വാനരവീരരും\\
രാക്ഷസരോടെതിര്‍ത്താരതു കണ്ടേറ്റ-\\
മൂക്കോടടുത്തു നിശാചരവീരരും.\\
സുഗ്രീവനും വിരൂപാക്ഷനും തങ്ങളി-\\
ലുഗ്രമാംവണ്ണം പൊരുതാരതു നേരം.\\
വാഹനമാകിയ വാരണവീരനെ-\\
സ്സാഹസം കൈക്കൊണ്ടു വാനരരാജനും\\
കൊന്നതു കണ്ടു വിരൂപവിലോചനന്‍\\
ചെന്നിതു വാളും പരിചയും കൈക്കൊണ്ടു.\\
കുന്നുകൊണ്ടൊന്നെറിഞ്ഞാന്‍ കപിരാജനും\\
നന്നായിതെന്നു വിരൂപാക്ഷനുമഥ\\
വെട്ടിനാന്‍ വാനരനായകവക്ഷസി\\
പുഷ്ടകോപത്തോടു മര്‍ക്കടരാജനും\\
നെറ്റിമേലൊന്നടിച്ചാനതു കൊണ്ടവന്‍\\
തെറ്റെന്നു കാലപുരംപുക്കു മേവിനാന്‍.\\
തേരിലേറിക്കൊണ്ടടുത്താന്‍ മഹോദരന്‍\\
തേരും തകര്‍ത്തു സുഗ്രീവനവനെയും\\
മൃത്യുപുരത്തിനയച്ചതു കണ്ടതി-\\
ക്രുദ്ധനായ് വന്നടുത്താന്‍ മഹാപാര്‍ശ്വനും\\
അംഗദന്‍ കൊന്നാനവനെയുമന്നേരം\\
പൊങ്ങും മിഴികളോടാശരാധീശനും\\
പോര്‍മദത്തോടുമടുത്തു കപികളെ-\\
ത്താമസാസ്ത്രംകൊണ്ടു വീഴ്ത്തിനാനൂഴിയില്‍.\\
രാമനുമൈന്ദ്രാസ്ത്രമെയ്തു തടുത്തിതു\\
താമസാസ്ത്രത്തെയുമപ്പോള്‍ ദശാനനന്‍\\
ആസുരമസ്ത്രമെയ്താനതു വന്നള-\\
വാതുരന്മാരായിതാശു കപികളും\\
വാരണസൂകര കുക്കുട ക്രോഷ്ടുക-\\
സാരമേയോരഗ സൈരിഭ വായസ-\\
വാനരസിംഹരുരു വൃക കാക ഗൃ-\\
ദ്ധ്രാനനമായ് വരുമാസുരാസ്ത്രാത്മകം\\
മുല്‍ഗരപട്ടസ ശക്തി പരശ്വധ-\\
ഖഡ്ഗ ശൂല പ്രാസ ബാണായുധങ്ങളും\\
രൂക്ഷമായ് വന്നു പരന്നതു കണ്ടള-\\
വാഗ്നേയമസ്ത്രമെയ്താന്‍ മനുവീരനും.\\
ചെങ്കനല്‍ക്കൊള്ളികള്‍ മിന്നല്‍ നക്ഷത്രങ്ങള്‍\\
തിങ്കളുമാദിത്യനഗ്നിയെന്നിത്തരം\\
ജ്യോതിര്‍മ്മയങ്ങളായ് ചെന്നു നിറഞ്ഞള-\\
വാസുരമസ്ത്രവും പോയ് മറഞ്ഞൂ ബലാല്‍\\
അപ്പോള്‍ മയന്‍ കൊടുത്തോരു ദിവ്യാസ്ത്രമെ-\\
യ്തല്പേതരായുധം കാണായിതന്തികേ.\\
ഗാന്ധര്‍വമസ്ത്രം പ്രയോഗിച്ചതിനെയും\\
ശാന്തമാക്കീടിനാന്‍ മാനവവീരനും.\\
സൗര്യാസ്ത്രമെയ്താന്‍ ദശാനനനന്നേരം\\
ധൈര്യേണ രാഘവന്‍ പ്രത്യസ്ത്രമെയ്തതും\\
ഖണ്ഡിച്ചനേരമാഖണ്ഡലവൈരിയും\\
ചണ്ഡകരാംശു സമങ്ങളാം ബാണങ്ങള്‍\\
പത്തുകൊണ്ടെയ്തു മര്‍മങ്ങള്‍ ഭേദിച്ചള-\\
വുത്തമപൂരുഷനാകിയ രാഘവന്‍\\
നൂറുശരങ്ങളെയ്താനതു കൊണ്ടുടല്‍-\\
കീറി മുറിഞ്ഞിതു നക്തഞ്ചരേന്ദ്രനും\\
ലക്ഷ്മണനേഴുശരങ്ങളാലൂക്കോടു\\
തല്‍ക്ഷണേ കേതു ഖണ്ഡിച്ചു വീഴ്ത്തീടിനാന്‍.\\
അഞ്ചുശരമെയ്തു സൂതനെയും കൊന്നു\\
ചഞ്ചലഹീനം മുറിച്ചിതു ചാപവും\\
അശ്വങ്ങളെഗ്ഗദകൊണ്ടു വിഭീഷണന്‍\\
തച്ചുകൊന്നാ,നതുനേരം ദശാനനന്‍\\
ഭൂതലേ ചാടിവീണാശു വേല്‍ക്കൊണ്ടതി-\\
ക്രോധാല്‍ വിഭീഷണനെ പ്രയോഗിച്ചിതു.\\
ബാണങ്ങള്‍ മൂന്നുകൊണ്ടെയ്തു മുറിച്ചിതു\\
വീണിതു മൂന്നും നുറുങ്ങി മഹീതലേ.\\
അപ്പോള്‍ വിഭീഷണനെക്കൊല്ലുമാറവന്‍\\
കല്പിച്ചുമുന്നം മയന്‍കൊടുത്തോരു വേല്‍\\
കൈക്കൊണ്ടു ചാട്ടുവാനോങ്ങിയ നേരത്തു\\
ലക്ഷ്മണന്‍ മുല്പുക്കു ബാണങ്ങളെയ്തിതു.\\
നക്തഞ്ചരാധിപന്‍ തന്നുടലൊക്കവേ\\
രക്തമണിഞ്ഞു മുറിഞ്ഞു വലഞ്ഞുടന്‍\\
നില്ക്കും ദശാനനന്‍ കോപിച്ചു ചൊല്ലിനാന്‍\\
ലക്ഷ്മണന്‍തന്നോടു, ‘നന്നു നീയെത്രയും\\
രക്ഷിച്ചവാറു വിഭീഷണനെത്തദാ.\\
രക്ഷിക്കില്‍ നന്നു നിന്നെപ്പുന, രെന്നുടെ\\
ശക്തി വരുന്നതു കണ്ടാലു, മിന്നൊരു\\
ശക്തനാകില്‍ ഭവാന്‍ ഖണ്ഡിക്ക വേലിതും.’\\
എന്നു പറഞ്ഞു വേഗേന ചാട്ടീടിനാന്‍\\
ചെന്നു തറച്ചിതു മാറത്തു ശക്തിയും.\\
അസ്ത്രങ്ങള്‍കൊണ്ടു തടുക്കരുതാഞ്ഞുടന്‍\\
വിത്രസ്തനായ് തത്ര വീണു കുമാരനും.\\
വേല്‍കൊണ്ടു ലക്ഷ്മണന്‍ വീണതു കണ്ടുള്ളില്‍\\
മാല്‍കൊണ്ടു രാമനും നിന്നു വിഷണ്ണനായ്.\\
ശക്തി പറിപ്പതിന്നാര്‍ക്കും കപികള്‍ക്കു\\
ശക്തി പോരാഞ്ഞു രഘുകുലനായകന്‍\\
തൃക്കൈകള്‍കൊണ്ടു പിടിച്ചു പറിച്ചുട-\\
നുള്‍ക്കോപമോടു മുറിച്ചെറിഞ്ഞീടിനാന്‍.\\
മിത്രതനയ സുഷേണ ജഗല്‍പ്രാണ-\\
പുത്രാദികളോടരുള്‍ ചെയ്തിതാദരാല്‍:\\
‘ലക്ഷ്മണന്‍ തന്നുടെ ചുറ്റുമിരുന്നിനി\\
രക്ഷിച്ചുകൊള്‍വിന്‍ വിഷാദിക്കരുതേതും.\\
ദുഃഖസമയമല്ലിപ്പോളുഴറ്റോടു\\
രക്ഷോവരനെ വധിക്കുന്നതുണ്ടു ഞാന്‍.\\
കല്യാണമുള്‍ക്കൊണ്ടു കണ്ടുകൊള്‍വിന്‍ നിങ്ങ-\\
ളെല്ലാവരുമിന്നു മല്‍ക്കരകൗശലം.\\
ശക്രാത്മജനെ വധിച്ചതും വേഗത്തി-\\
ലര്‍ക്കാത്മജാദികളോടുമൊരുമിച്ചു\\
വാരിധിയില്‍ ചിറകെട്ടിക്കടന്നതും\\
പോരില്‍ നിശാചരന്മാരെ വധിച്ചതും\\
രാവണനിഗ്രഹസാദ്ധ്യമായിട്ടിവന്‍\\
കേവലമിപ്പോളഭിമുഖനായിതു.\\
രാവണനും ബത! രാഘവനും കൂടി\\
മേവുക ഭൂമിയിലെന്നുള്ളതില്ലിനി.\\
രാത്രിഞ്ചരേന്ദ്രനെക്കൊല്ലുവന്‍ നിര്‍ണയം\\
മാര്‍ത്താണ്ഡവംശത്തിലുള്ളവനാകില്‍ ഞാന്‍.\\
സപ്തദ്വീപങ്ങളും സപ്താംബുധികളും\\
സപ്താചലങ്ങളും സൂര്യചന്ദ്രന്മാരും\\
ആകാശഭൂമികളെന്നിവയുള്ള നാള്‍\\
പോകാത കീര്‍ത്തി വര്‍ദ്ധിക്കും പരിചു ഞാന്‍\\
ആയോധനേ ദശകണ്ഠനെക്കൊല്‍വനൊ-\\
രായുധപാണിയെന്നാകില്‍ നിസ്സംശയം.\\
ദേവാസുരോരഗചാരണ താപസ-\\
രേവരും കണ്ടറിയേണം മമ ബലം.’\\
ഇത്ഥമരുള്‍ചെയ്തു നക്തഞ്ചരേന്ദ്രനോ-\\
ടസ്ത്രങ്ങളെയ്തു യുദ്ധം തുടങ്ങീടിനാന്‍.\\
തത്സമം ബാണം നിശാചരാധീശനു-\\
മുത്സാഹമുള്‍ക്കൊണ്ടു തൂകിത്തുടങ്ങിനാന്‍.\\
രാഘവരാവണന്മാര്‍ തമ്മിലിങ്ങനെ\\
മേഘങ്ങള്‍ മാരിചൊരിയുന്നതുപോലെ\\
ബാണഗണംപൊഴിച്ചീടുന്നതുനേരം\\
ഞാണൊലികൊണ്ടു മുഴങ്ങി ജഗത്ത്രയം.\\
സോദരന്‍ വീണു കിടക്കുന്നതോര്‍ത്തുള്ളി-\\
ലാധി മുഴുത്തു രഘുകുലനായകന്‍\\
താരേയതാതനോടേവമരുള്‍ചെയ്തു:\\
‘ധീരതയില്ല യുദ്ധത്തിനേതും മമ.\\
ഭൂതലേ വാഴ്കയില്‍ നല്ലതെനിക്കിനി\\
ഭ്രാതാവുതന്നോടുകൂടെ മരിപ്പതും.\\
വില്പിടിയും മുറുകുന്നതില്ലേതുമേ\\
കെല്പുമില്ലാതെ ചമഞ്ഞു നമുക്കിഹ\\
നില്പാനുമേതുമരുതു, മനസ്സിനും\\
വിഭ്രമമേറി വരുന്നിതു മേല്ക്കുമേല്‍.\\
ദുഷ്ടനെക്കൊല്‍വാനുപായവും കണ്ടീല\\
നഷ്ടമായ് വന്നിതു മാനവും മാനസേ.’\\
ഏവമരുള്‍ചെയ്തനേരം സുഷേണനും\\
ദേവദേവന്‍തന്നൊടാശു ചൊല്ലീടിനാന്‍:\\
‘ദേഹത്തിനേതും നിറം പകര്‍ന്നീലൊരു\\
മോഹമത്രേ കുമാരന്നെന്നു നിര്‍ണയം.\\
വക്ത്രനേത്രങ്ങള്‍ക്കുമേതും വികാരമി-\\
ല്ലത്തല്‍ തീര്‍ന്നിപ്പോളുണരുമവരജന്‍.’\\
എന്നുണര്‍ത്തിച്ചനിലാത്മജന്‍ തന്നോടു\\
പിന്നെ നിരൂപിച്ചു ചൊന്നാന്‍ സുഷേണനും:\\
‘മുന്നെക്കണക്കേ വിശല്യകരണിയാ-\\
കുന്ന മരുന്നിന്നു കൊണ്ടുവന്നീടുക.’\\
എന്നളവേ ഹനുമാനും വിരവോടു\\
ചെന്നു മരുന്നതും കൊണ്ടുവന്നീടിനാന്‍.\\
നസ്യവുംചെയ്തു സുഷേണന്‍, കുമാരനാ-\\
ലസ്യവും തീര്‍ന്നു തെളിഞ്ഞു വിളങ്ങിനാന്‍.\\
പിന്നെയുമൗഷധശൈലംകപിവരന്‍\\
മുന്നമിരുന്നവണ്ണംതന്നെയാക്കിനാന്‍\\
മന്നവന്‍തന്നെ വണങ്ങിനാന്‍ തമ്പിയും\\
നന്നായ്മുറുകെപ്പുണര്‍ന്നിതു രാമനും.\\
‘നിന്നുടെ പാരവശ്യം കാണ്‍ക കാരണ-\\
മെന്നുടെ ധൈര്യവും പോയിതു മാനസേ.’\\
എന്നതു കേട്ടുരചെയ്തു കുമാരനു-\\
‘മൊന്നു തിരുമനസ്സിങ്കലുണ്ടാകണം.\\
സത്യം തപോധനന്മാരോടു ചെയ്തതും\\
മിഥ്യയായ് വന്നുകൂടായെന്നു നിര്‍ണയം.\\
ത്രൈലോക്യകണ്ടകനാമിവനെക്കൊന്നു\\
പാലിച്ചുകൊള്‍ക ജഗത്ത്രയം വൈകാതെ.’\\
ലക്ഷ്മണന്‍ ചൊന്നതു കേട്ടു രഘൂത്തമന്‍\\
രക്ഷോവരനോടെതിര്‍ത്താനതിദ്രുതം\\
തേരുമൊരുമിച്ചു വന്നു ദശാസ്യനും\\
പോരിനു രാഘവനോടെതിര്‍ത്തീടിനാന്‍.\\
പാരില്‍നിന്നിക്ഷ്വാകുവംശതിലകനും\\
തേരില്‍നിന്നാശരവംശതിലകനും\\
പോരതിഘോരമായ് ചെയ്തൊരു നേരത്തു\\
പാരമിളപ്പം രഘൂത്തമനുണ്ടെന്നു\\
നാരദനാദികള്‍ ചൊന്നതു കേള്‍ക്കയാല്‍\\
പാരം വളര്‍ന്നൊരു സംഭ്രമത്തോടുടന്‍\\
ഇന്ദ്രനും മാതലിയോടു ചൊന്നാന്‍,‘മമ\\
സ്യന്ദനം കൊണ്ടക്കൊടുക്ക നീ വൈകാതെ\\
ശ്രീരാഘവന്നു ഹിതം വരുമാറു നീ\\
തേരും തെളിച്ചു കൊടുക്ക മടിയാതെ.’\\
മാതലി താനതു കേട്ടുടന്‍ തേരുമായ്\\
ഭൂതലം തന്നിലിഴിഞ്ഞു ചൊല്ലീടിനാന്‍:\\
‘രാവണനോടു സമരത്തിനിന്നു ഞാന്‍\\
ദേവേന്ദ്രശാസനയാ വിടകൊണ്ടിതു\\
തേരതിലാശു കരേറുക പോരിനായ്\\
മാരുതതുല്യവേഗേന നടത്തുവന്‍.\\
എന്നതുകേട്ടു രഥത്തെയും വന്ദിച്ചു\\
മന്നവന്‍ തേരിലാമ്മാറു കരേറിനാന്‍.\\
തന്നോടു തുല്യനായ് രാഘവനെക്കണ്ടു\\
വിണ്ണിലാമ്മാറൊന്നു നോക്കി ദശാനനന്‍.\\
പേമഴപോലെ ശരങ്ങള്‍ തൂകീടിനാന്‍\\
രാമനും ഗാന്ധര്‍വമസ്ത്രമെയ്തീടിനാന്‍.\\
രാക്ഷസമസ്ത്രം പ്രയോഗിച്ചതുനേരം\\
രാക്ഷസരാജനും രൂക്ഷമായെത്രയും\\
ക്രൂരനാഗങ്ങളാമസ്ത്രത്തെ മാറ്റുവാന്‍\\
ഗാരുഡമസ്ത്രമെയ്തു രഘുനാഥനും.\\
മാതലിമേലും ദശാനനന്‍ ബാണങ്ങ-\\
ളെയ്തു കൊടിയും മുറിച്ചു കളഞ്ഞിതു.\\
വാജികല്‍ക്കും ശരമേറ്റമേറ്റു പുന-\\
രാജിയും ഘോരമായ് വന്നു, രഘുവരന്‍\\
കൈകാല്‍ തളര്‍ന്നു തേര്‍ത്തട്ടില്‍ നില്ക്കും വിധൗ\\
കൈകസീനന്ദനനായ വിഭീഷണന്‍\\
ശോകാതിരേകം കലര്‍ന്നു നിന്നീടിനാന്‍\\
ലോകരുമേറ്റം വിഷാദം കലര്‍ന്നിതു.\\
കാലപുരത്തിനയപ്പേനിനിയെന്നു\\
ശൂലം പ്രയോഗിച്ചിതാശരാധീശനും.\\
അസ്ത്രങ്ങള്‍ കൊണ്ടു തടപൊറാഞ്ഞോര്‍ത്തുടന്‍\\
വൃത്രാരി തന്നുടെ തേരിലിരുന്നൊരു\\
ശക്തിയെടുത്തയച്ചു രഘുനാഥനും\\
പത്തുനുറുങ്ങി വീണു തത്ര ശൂലവും.\\
നക്തഞ്ചരേന്ദ്രനുടെ തുരഗങ്ങളെ-\\
ശ്ശസ്ത്രങ്ങള്‍കൊണ്ടു മുറിച്ചിതു രാഘവന്‍\\
സാരഥി തേരും തിരിച്ചടിച്ചാര്‍ത്തനായ്\\
പോരിലൊഴിച്ചു നിര്‍ത്തീടിനാനന്നേരം.\\
ആലസ്യമൊട്ടകന്നോരു നേരം തത്ര-\\
പൗലസ്ത്യനും സൂതനോടു ചൊല്ലീടിനാന്‍:\\
‘എന്തിനായ്ക്കൊണ്ടു നീ പിന്തിരിഞ്ഞു ബലാ-\\
ലന്ധനായ് ഞാനത്ര ദുര്‍ബലനാകയോ?\\
കൂടലരോടെതിര്‍ത്താല്‍ ഞാനൊരുത്തനോ-\\
ടോടിയൊളിച്ചവാറെന്നു കണ്ടു ഭവാന്‍?\\
നീയല്ല സൂതനെനിക്കിനി രാമനു\\
നീയതിബാന്ധവനെന്നറിഞ്ഞേനഹം.’\\
ഇത്ഥം നിശാചരാധീശന്‍ പറഞ്ഞതി-\\
നുത്തരം സാരഥി സത്വരം ചൊല്ലിനാന്‍:\\
‘രാമനെ സ്നേഹമുണ്ടായിട്ടുമല്ല മ-\\
ത്സ്വാമിയെ ദ്വേഷമുണ്ടായിട്ടുമല്ല മേ\\
രാമനോടേറ്റു പൊരുതു നില്ക്കുന്നേര-\\
മാമയം പൂണ്ടു തളര്‍ന്നതു കണ്ടു ഞാന്‍.\\
സ്നേഹം ഭവാനെക്കുറിച്ചേറ്റമാകയാല്‍\\
മോഹമകലുവോളം പോര്‍ക്കളം വിട്ടു\\
ദൂരെനിന്നാലസ്യമെല്ലാം കളഞ്ഞിനി-\\
പ്പോരിന്നടുക്കേണമെന്നു കല്പിച്ചത്രെ.\\
സാരഥിതാനറിയേണം മഹാരഥ-\\
ന്മാരുടെ സാദവും വാജികള്‍ സാദവും\\
വൈരികള്‍ക്കുള്ള ജയാജയകാലവും\\
പോരില്‍ നിമ്നോന്നതദേശവിശേഷവും.\\
എല്ലാമറിഞ്ഞു രഥം നടത്തുന്നവ-\\
നല്ലോ നിപുണനായുള്ള സൂതന്‍ പ്രഭോ!’\\
എന്നതു കേട്ടു തെളിഞ്ഞഥ രാവണ-\\
നൊന്നു പുണര്‍ന്നൊരു കൈവളയും കൊടു-\\
‘ത്തിന്നിനിത്തേരടുത്താശു കൂട്ടീടുക\\
പിന്നോക്കമില്ലിനിയൊന്നുകൊണ്ടുമെടോ!\\
ഇന്നോടു നാളെയോടൊന്നു തിരിഞ്ഞിടും\\
മന്നവനോടുള്ള പോരെന്നറിക നീ.’\\
സൂതനും തേരതിവേഗേന പൂട്ടിനാന്‍\\
ക്രോധം മുഴുത്തങ്ങടുത്തിതു രാമനും.\\
തങ്ങളിലേറ്റമണഞ്ഞു പൊരുന്നള-\\
വങ്ങുമിങ്ങും നിറയുന്നു ശരങ്ങളാല്‍.
\end{verse}

%%28_agasthyaagamanavumaadithyasthuthiyum

\section{അഗസ്ത്യാഗമനവും ആദിത്യസ്തുതിയും}

\begin{verse}
അങ്ങനെയുള്ള പോര്‍ കണ്ടുനില്ക്കുന്നേര-\\
മെങ്ങനെയെന്നറിഞ്ഞീലഗസ്ത്യന്‍ തദാ\\
രാഘവന്‍ തേരിലിറങ്ങി നിന്നീടിനാ-\\
നാകാശദേശാല്‍ പ്രഭാകരസന്നിഭന്‍.\\
വന്ദിച്ചുനിന്നു രഘുകുലനാഥനാ-\\
നന്ദമിയന്നരുള്‍ചെയ്താനഗസ്ത്യനും:\\
അഭ്യുദയം നിനക്കാശുവരുത്തുവാ\\
നിപ്പോഴിവിടേക്കു വന്നിതു ഞാനെടോ!\\
താപത്രയവും വിഷാദവും തീര്‍ന്നുപോ-\\
മാപത്തു മറ്റുള്ളവയുമകന്നുപോം.\\
ശത്രുനാശം വരും രോഗവിനാശനം\\
വര്‍ദ്ധിക്കുമായുസ്സു സല്‍ക്കീര്‍ത്തിവര്‍ദ്ധനം\\
നിത്യമാദിത്യഹൃദയമാം മന്ത്രമി-\\
തുത്തമമെത്രയും ഭക്ത്യാ ജപിക്കെടോ!\\
ദേവാസുരോരഗചാരണകിന്നര\\
താപസഗുഹ്യകയക്ഷരക്ഷോഭൂത\\
കിംപുരുഷാപ്സരോ മാനുഷാദ്യന്മാരും\\
സമ്പ്രതി സൂര്യനെത്തന്നെ ഭജിപ്പതും.\\
ദേവകളാകുന്നതാദിത്യനാകിയ\\
ദേവനത്രേ പതിന്നാലു ലോകങ്ങളും\\
രക്ഷിപ്പതും നിജരശ്മികള്‍കൊണ്ടവന്‍\\
ഭക്ഷിപ്പതുമവന്‍ കല്പകാലാന്തരേ.\\
ബ്രഹ്മനും വിഷ്ണുവും ശ്രീമഹാദേവനും\\
ഷണ്മുഖന്‍ താനും പ്രജാപതിവൃന്ദവും\\
ശക്രനും വൈശ്വാനരനും കൃതാന്തനും\\
രക്ഷോവരനും വരുണനും വായുവും\\
യക്ഷാധിപനുമീശാനനും ചന്ദ്രനും\\
നക്ഷത്രജാലവും ദിക്കരിവൃന്ദവും\\
വാരണവക്ത്രനുമാര്യനും മാരനും\\
താരാഗണങ്ങളും നാനാഗ്രഹങ്ങളും\\
അശ്വിനീപുത്രരുമഷ്ടവസുക്കളും\\
വിശ്വദേവന്മാരും സിദ്ധരും സാദ്ധ്യരും\\
നാനാപിതൃക്കളും പിന്നെ മനുക്കളും\\
ദാനവന്മാരുമുരഗസമൂഹവും\\
വാരമാസര്‍ത്തു സംവത്സര കല്പാദി\\
കാരകനായതും സൂര്യനിവന്‍ തന്നെ.\\
വേദാന്തവേദ്യനാം വേദാത്മകനിവന്‍\\
വേദാര്‍ത്ഥവിഗ്രഹന്‍ വേദജ്ഞസേവിതന്‍\\
പൂഷാ വിഭാകരന്‍ മിത്രന്‍ പ്രഭാകരന്‍\\
ദോഷാകരാത്മകന്‍ ത്വഷ്ടാ ദിനകരന്‍\\
ഭാസ്കരന്‍ നിത്യനഹസ്കരനീശ്വരന്‍\\
സാക്ഷി സവിതാ സമസ്തലോകേക്ഷണന്‍\\
ഭാസ്വാന്‍ വിവസ്വാന്‍ നഭസ്വാന്‍ ഗഭസ്തിമാന്‍\\
ശാശ്വതന്‍ ശംഭു ശരണ്യന്‍ ശരണദന്‍\\
ലോകശിശിരാരി ഘോരതിമിരാരി\\
ശോകാപഹാരി ലോകാലോകവിഗ്രഹന്‍\\
ഭാനു ഹിരണ്യഗര്‍ഭന്‍ ഹിരണ്യേന്ദ്രിയന്‍\\
ദാനപ്രിയന്‍ സഹസ്രാംശു സനാതനന്‍\\
സപ്താശ്വനര്‍ജുനാശ്വന്‍ സകലേശ്വരന്‍\\
സുപ്തജനാവബോധപ്രദന്‍ മംഗലന്‍\\
ആദിത്യനര്‍ക്കനരുണനനന്തഗന്‍\\
ജ്യോതിര്‍മയന്‍ തപനന്‍ സവിതാ രവി\\
വിഷ്ണു വികര്‍ത്തനന്‍ മാര്‍ത്താണ്ഡനംശുമാ-\\
നുഷ്ണകിരണന്‍ മിഹിരന്‍ വിരോചനന്‍\\
പ്രദ്യോതനന്‍ പരന്‍ ഖദ്യോതനുദ്യോത-\\
നദ്വയന്‍ വിദ്യാവിനോദന്‍ വിഭാവസു\\
വിശ്വസൃഷ്ടിസ്ഥിതി സംഹാരകാരണന്‍\\
വിശ്വവന്ദ്യന്‍ മഹാവിശ്വരൂപന്‍ വിഭു\\
വിശ്വവിഭാവനന്‍ വിശ്വൈകനായകന്‍\\
വിശ്വാസഭക്തിയുക്താനാം ഗതിപ്രദന്‍\\
ചണ്ഡകിരണന്‍ തരണി ദിനമണി\\
പുണ്ഡരീക പ്രബോധപ്രദനര്യമാ\\
ദ്വാദശാത്മാ പരമാത്മാ പരാപര-\\
നാദിതേയന്‍ ജഗദാദിഭൂതന്‍ ശിവന്‍\\
ഖേദവിനാശനന്‍ കേവലാത്മാവിന്ദു-\\
നാദാത്മകന്‍ നാരദാദി നിഷേവിതന്‍\\
ജ്ഞാനസ്വരൂപനജ്ഞാനവിനാശനന്‍\\
ധ്യാനിച്ചുകൊള്‍ക നീ നിത്യമിദ്ദേവനെ\\
സന്തതം ഭക്ത്യാ നമസ്കരിച്ചീടുക\\
സന്താപനാശകരായ നമോ നമഃ\\
അന്ധകാരാന്തകരായ നമോ നമഃ\\
ചിന്താമണേ ചിദാനന്ദായതേ നമഃ\\
നീഹാരനാശകരായ നമോ നമഃ\\
മോഹവിനാശകരായ നമോ നമഃ\\
ശാന്തായ രൗദ്രായ സൗമ്യായ ഘോരായ\\
കാന്തിമതാം കാന്തിരൂപായ തേ നമഃ\\
സ്ഥാവരജംഗമാചാര്യായ തേ നമഃ\\
ദേവായ വിശ്വൈകസാക്ഷിണേ തേ നമഃ\\
സത്വപ്രധാനായ തത്ത്വായ തേ നമഃ\\
സത്യസ്വരൂപായ നിത്യം നമോ നമഃ\\
ഇത്ഥമാദിത്യഹൃദയം ജപിച്ചു നീ\\
ശത്രുക്ഷയം വരുത്തീടുക സത്വരം.’\\
ചിത്തം തെളിഞ്ഞഗസ്ത്യോക്തി കേട്ടെത്രയും\\
ഭക്തിവര്‍ധിച്ചു കാകുല്‍സ്ഥനും കൂപ്പിനാന്‍\\
പിന്നെ വിമാനവുമേറി മഹാമുനി\\
ചെന്നു വീണാധരോപാന്തേ മരുവിനാന്‍.
\end{verse}

%%29_raavanavadham

\section{രാവണവധം}

\begin{verse}
രാഘവന്‍ മാതലിയോടരുളിച്ചെയ്തി-\\
‘താകുലമെന്നിയേ തേര്‍ നടത്തീടു നീ.’\\
മാതലി തേരതിവേഗേന കൂട്ടിനാ-\\
നേതുമേ ചഞ്ചലമില്ല ദശാസ്യനും.\\
മൂടീ പൊടികൊണ്ടു ദിക്കുമുടനിട-\\
കൂടീ ശരങ്ങളുമെന്തൊരു വിസ്മയം!\\
രാത്രിഞ്ചരന്റെ കൊടിമരം ഖണ്ഡിച്ചു\\
ധാത്രിയിലിട്ടു ദശരഥപുത്രനും.\\
യാതുധാനാധിപന്‍ വാജികള്‍തമ്മെയും\\
മാതലിതന്നെയുമേറെയെയ്തീടിനാന്‍.\\
ശൂലം മുസലഗദാദികളും മേല്‍ക്കു-\\
മേലേ പൊഴിച്ചിതു രാക്ഷസരാജനും.\\
സായകജാലം പൊഴിച്ചവയും മുറി-\\
ച്ചായോധനത്തിന്നടുത്തിതു രാമനും.\\
‘ഏറ്റമണഞ്ഞുമകന്നും വലംവച്ചു-\\
മേറ്റുമിടംവെച്ചുമൊട്ടു പിന്‍വാങ്ങിയും\\
സാരഥിമാരുടെ സൗത്യകൗശല്യവും\\
പോരാളികളുടെ യുദ്ധ കൗശല്യവും\\
പണ്ടു കീഴില്‍ കണ്ടതില്ല നാമീവണ്ണ-\\
മുണ്ടാകയുമില്ലിവണ്ണമിനിമേലില്‍.’\\
എന്നു ദേവാദികളും പുകഴ്ത്തീടിനാര്‍\\
നന്നുനന്നെന്നു തെളിഞ്ഞിതു നാരദന്‍.\\
പൗലസ്ത്യരാഘവന്മാര്‍‍തൊഴില്‍ കാണ്‍കയാല്‍\\
ത്രൈലോക്യവാസികള്‍ ഭീതിപൂണ്ടീടിനാര്‍.\\
വാതമടങ്ങി മറഞ്ഞിതു സൂര്യനും\\
മേദിനിതാനും വിറച്ചിതു പാരമായ്.\\
പാഥോനിധിയുമിളകി മറിഞ്ഞിതു\\
പാതാളവാസികളും നടുങ്ങീടിനാര്‍.\\
‘അംബുധി അംബുധിയോടൊന്നെതിര്‍ക്കിലു-\\
മംബരമംബരത്തോടെതിര്‍ത്തീടിലും\\
രാഘവരാവണയുദ്ധത്തിനു സമം\\
രാഘവരാവണയുദ്ധമൊഴിഞ്ഞില്ല.’\\
കേവലമിങ്ങനെ നിന്നു പുകഴ്ത്തിനാര്‍\\
ദേവാദികളുമന്നേരത്തു രാഘവന്‍\\
രാത്രിഞ്ചരന്റെ തലയൊന്നറുത്തുടന്‍\\
ധാത്രിയിലിട്ടാനതുനേരമപ്പൊഴേ\\
കൂടെ മുളച്ചു കാണായിതവന്‍ തല\\
കൂടെ മുറിച്ചു കളഞ്ഞു രണ്ടാമതും.\\
ഉണ്ടായിതപ്പോളതും പിന്നെ രാഘവന്‍\\
ഖണ്ഡിച്ചു ഭൂമിയിലിട്ടാനരക്ഷണാല്‍\\
ഇത്ഥം മുറിച്ചു നൂറ്റൊന്നു തലകളെ\\
പൃത്ഥ്വിയിലിട്ടു രഘുകുലസത്തമന്‍.\\
പിന്നെയും പത്തു തലയ്ക്കൊരു വാട്ടമി-\\
ല്ലെന്നേ! വിചിത്രമേ! നന്നുനന്നെത്രയും\\
ഇങ്ങനെ നൂറായിരം തല പോകിലു-\\
മെങ്ങും കുറവില്ലവന്‍ തല പത്തിനും.\\
രാത്രിഞ്ചരാധിപന്‍ തന്റെ തപോബലം\\
ചിത്രം വിചിത്രം വിചിത്രമത്രേ തുലോം.\\
കുംഭകര്‍ണന്‍ മകരാക്ഷന്‍ ഖരന്‍ ബാലി\\
വമ്പനാം മാരീചനെന്നിവരാദിയാം\\
ദുഷ്ടരെക്കൊന്ന ബാണത്തിനിന്നെന്തതി-\\
നിഷ്ഠുരനാമിവനെക്കൊല്ലുവാന്‍ മടി-\\
യുണ്ടായതിദ്ദശകണ്ഠനെക്കൊല്ലുവാന്‍\\
കണ്ടീലുപായവുമേതുമൊന്നീശ്വരാ!\\
ചിന്തിച്ചു രാഘവന്‍ പിന്നെയുമദ്ദശ-\\
കന്ധരന്‍ മെയ്യില്‍ ബാണങ്ങള്‍ തൂകീടിനാന്‍.\\
രാവണനും പൊഴിച്ചീടിനാന്‍ ബാണങ്ങള്‍\\
ദേവദേവന്‍ തിരുമേനിമേലാവോളം.\\
കൊണ്ടശരങ്ങളെക്കൊണ്ടു രഘുവര-\\
നുണ്ടായിതുള്ളിലൊരു നിനവന്നേരം.\\
പുഷ്പസമങ്ങളായ് വന്നു ശരങ്ങളും\\
കെല്പു കുറഞ്ഞു ദശാസ്യനു നിര്‍ണയം.\\
ഏഴു ദിവസം മുഴുവനീവണ്ണമേ\\
രോഷേണ നിന്നു പൊരുതോരനന്തരം\\
മാതലിതാനും തൊഴുതു ചൊല്ലീടിനാ-\\
‘നേതും വിഷാദമുണ്ടാകായ്ക മാനസേ\\
മുന്നമഗസ്ത്യതപോധനനാദരാല്‍\\
തന്ന ബാണംകൊണ്ടു കൊല്ലാം ജഗല്‍പ്രഭോ!\\
പൈതാമഹാസ്ത്രമതായതെ’ന്നിങ്ങനെ\\
മാതലി ചൊന്നതു കേട്ടു  രഘുവരന്‍:\\
‘നന്നു പറഞ്ഞതു നീയതെന്നോടിനി-\\
ക്കൊന്നീടുവേന്‍ ദശകണ്ഠനെ നിര്‍ണയം.’\\
എന്നരുളിച്ചെയ്തു വൈരിഞ്ചമസ്ത്രത്തെ\\
നന്നായെടുത്തു തൊടുത്തിതു രാഘവന്‍.\\
സൂര്യാനലന്മാരതിന്നു തരം, തൂവല്‍\\
വായുവും മന്ദരമേരുക്കള്‍ മധ്യമായ്\\
വിശ്വമെല്ലാം പ്രകാശിച്ചൊരു സായകം\\
വിശ്വാസഭക്ത്യാ ജപിച്ചയച്ചീടീനാന്‍.\\
രാവണന്‍ തന്റെ ഹൃദയം പിളര്‍ന്നു ഭൂ-\\
ദേവിയും ഭേദിച്ചു വാരിധിയില്‍ പുക്കു\\
ചോരകഴുകി മുഴുകി വിരവോടു\\
മാരുതവേഗേന രാഘവന്‍തന്നുടെ\\
തൂണിയില്‍ വന്നിങ്ങു വീണു തെളിവോടു\\
ബാണവു,മെന്തൊരു വിസ്മയ,മന്നേരം\\
തേരില്‍നിന്നാശു മറിഞ്ഞു വീണീടിനാന്‍\\
പാരില്‍ മരാമരം വീണപോലെ തദാ.\\
കല്പകവൃക്ഷപ്പുതുമലര്‍ തൂകിനാ-\\
രുത്പന്നമോദേന വാനവരേവരും\\
അര്‍ക്കകുലോത്ഭവന്‍ മൂര്‍ദ്ധനി മേല്ക്കുമേല്‍\\
ശക്രനും നേത്രങ്ങളൊക്കെ തെളിഞ്ഞിതു.\\
പുഷ്കരസംഭവനും തെളിഞ്ഞീടിനാ-\\
നര്‍ക്കനും നേരേയുദിച്ചാനതുനേരം.\\
മന്ദമായ് വീശിത്തുടങ്ങി പവനനും\\
നന്നായ് വിളങ്ങീ ചതുര്‍ദശലോകവും.\\
താപസന്മാരും ജയജയശബ്ദേന\\
താപമകന്നു പുകഴ്ന്നു തുടങ്ങിനാര്‍.\\
ശേഷിച്ച രാക്ഷസരോടിയകം പുക്കു\\
കേഴത്തുടങ്ങിനാരൊക്കെ ലങ്കാപുരേ.\\
അര്‍ക്കജന്‍ മാരുതി നീലാംഗദാദിയാം\\
മര്‍ക്കടവീരരുമാര്‍ത്തു പുകഴ്ത്തിനാര്‍.\\
അഗ്രജന്‍ വീണതു കണ്ടു വിഭീഷണന്‍\\
വ്യഗ്രിച്ചരികത്തു ചെന്നിരുന്നാകുലാല്‍\\
ദുഃഖം കലര്‍ന്നു വിലാപം തുടങ്ങിനാ-\\
‘നൊക്കെ വിധിബലമല്ലോ വരുന്നതും.\\
ഞാനിതൊക്കെപ്പറഞ്ഞീടിനേന്‍ മുന്നമേ\\
മാനം നടിച്ചെന്നെയും വെടിഞ്ഞീടിന\\
വീര! മഹാശയനോചിതനായ നീ\\
പാരിലീവണ്ണം കിടക്കുമാറായതും\\
കണ്ടിതെല്ലാം ഞാനനുഭവിക്കേണമെ-\\
ന്നുണ്ടു ദൈവത്തിനതാര്‍ക്കൊഴിക്കാവതും?’\\
ഏവം കരയും വിഭീഷണന്‍ തന്നോടു\\
ദേവദേവേശനരുള്‍ചെയ്തിതാദരാല്‍:\\
‘എന്നോടഭിമുഖനായ് നിന്നുപോര്‍ചെയ്തു\\
നന്നായ് മരിച്ച മഹാശൂരനാമിവന്‍\\
തന്നെക്കുറിച്ചു കരയരുതേതുമേ\\
നന്നല്ലതു പരലോകത്തിനു സഖേ!\\
വീരരായുള്ള രാജാക്കള്‍ധര്‍മം നല്ല\\
പോരില്‍ മരിക്കുന്നതെന്നറിയേണമേ!\\
പോരില്‍ മരിച്ചു വീരസ്വര്‍ഗസിദ്ധിക്കു\\
പാരം സുകൃതികള്‍ക്കെന്നി യോഗം വരാ.\\
ദോഷങ്ങളെല്ലാമൊടുങ്ങി നീ വന്നിനി-\\
ശ്ശേഷക്രിയയ്ക്കു തുടങ്ങുക വൈകാതെ.’\\
ഇത്ഥമരുള്‍ചെയ്തു നിന്നരുളുന്നേരം\\
തത്ര മണ്ഡോദരി കേണു വന്നീടിനാള്‍.\\
ലങ്കാധിപന്‍മാറില്‍ വീണു കരഞ്ഞുമാ-\\
തങ്കമുള്‍ക്കൊണ്ടു മോഹിച്ചു പുനരുടന്‍\\
ഓരോതരം പറഞ്ഞും പിന്നെ മറ്റുള്ള\\
നാരീജനങ്ങളും കേണു തുടങ്ങിനാര്‍.\\
പംക്തിരഥാത്മജനപ്പോളരുള്‍ചെയ്തു\\
പംക്തിമുഖാനുജന്‍തന്നോടു സാദരം:\\
‘രാവണന്‍ തന്നുടല്‍ സംസ്കരിച്ചീടുക\\
പാവകനെജ്ജ്വലിപ്പിച്ചിനിസ്സത്വരം.’\\
തത്ര വിഭീഷണന്‍ ചൊന്നാ‘നിവനോള-\\
മിത്ര പാപം ചെയ്തവരില്ല ഭൂതലേ.\\
യോഗ്യമല്ലേതുമടിയനിവനുടല്‍\\
സംസ്കരിച്ചീടുവാ’നെന്നു കേട്ടേറ്റവും\\
വന്ന ബഹുമാനമോടെ രഘൂത്തമന്‍\\
പിന്നെയും ചൊന്നാന്‍ വിഭീഷണന്‍തന്നോടു:\\
‘മദ്ബാണമേറ്റു രണാന്തേ മരിച്ചൊരു\\
കര്‍ബുരാധീശ്വരനറ്റിതു പാപങ്ങള്‍\\
വൈരവുമാമരണാന്തമെന്നാകുന്നി-\\
തേറിയ സദ്ഗതിയുണ്ടാവതിന്നു നീ\\
ശേഷക്രിയകള്‍ വഴിയേ കഴിക്കൊരു\\
ദോഷം നിനക്കതിനേതുമകപ്പെടാ.’\\
ചന്ദനഗന്ധാദികൊണ്ടു ചിതയുമാ-\\
നന്ദേന കൂട്ടി മുനിവരന്മാരുമായ്\\
വസ്ത്രാഭരണമാല്യങ്ങള്‍കൊണ്ടും തദാ-\\
നക്തഞ്ചരാധിപദേഹമലങ്കരി-\\
ച്ചാര്‍ത്തുവാദ്യങ്ങളും ഘോഷിച്ചുകൊണ്ടഗ്നി-\\
ഹോത്രികളെസ്സംസ്കരിക്കുന്നവണ്ണമേ\\
രാവണദേഹം ദഹിപ്പിച്ചു തന്നുടെ\\
പൂര്‍വജനായുദകക്രിയയും ചെയ്തു.\\
നാരികള്‍ ദുഃഖം പറഞ്ഞുപോക്കിച്ചെന്നു\\
ശ്രീരാമപാദം നമസ്കരിച്ചീടിനാര്‍.\\
മാതലിയും രഘുനാഥനെ വന്ദിച്ചു\\
ജാതമോദംപോയ് സുരാലയം മേവിനാന്‍.\\
ചെന്നു നിജനിജമന്ദിരം പുക്കിതു\\
ജന്യാവലോകനം ചെയ്തു നിന്നോര്‍കളും.
\end{verse}

%%30_vibheeshanaraajyaabhishekam

\section{വിഭീഷണരാജ്യാഭിഷേകം}

\begin{verse}
ലക്ഷ്മണനോടരുള്‍ചെയ്തിതു രാമനും\\
‘രക്ഷോവരനാം വിഭീഷണനായ് മയാ\\
ദത്തമായോരു ലങ്കാരാജ്യമുള്‍പ്പുക്കു\\
ചിത്തമോദാലഭിഷേകം കഴിക്ക നീ.’\\
എന്നതു കേട്ടു കപിവരന്മാരോടും\\
ചെന്നു ശേഷിച്ച നിശാചരന്മാരുമായ്\\
അര്‍ണവതോയാദി തീര്‍ത്ഥജലങ്ങളാല്‍\\
സ്വര്‍ണകലശങ്ങള്‍ പൂജിച്ചു ഘോഷിച്ചു\\
വാദ്യഘോഷത്തോടു താപസന്മാരുമാ-\\
യാര്‍ത്തുവിളിച്ചഭിഷേകവും ചെയ്തിതു.\\
ഭുമിയും ചന്ദ്രദിവാകരരാദിയും\\
രാമകഥയുമുള്ളന്നു വിഭീഷണന്‍\\
ലങ്കേശനായ് വാഴുകെന്നു കിരീടാദ്യ-\\
ലങ്കാരവും ചെയ്തു ദാനപുരസ്കൃതം\\
പൂജ്യനായോരു വിഭീഷണനായ്ക്കൊണ്ടു\\
രാജ്യനിവാസികള്‍ കാഴ്ചയും വെച്ചിതു.\\
വാച്ച കുതൂഹലംപൂണ്ടു വിഭീഷണന്‍\\
കാഴ്ചയുമെല്ലാമെടുപ്പിച്ചുകൊണ്ടുവ-\\
ന്നാസ്ഥയാ രാഘവന്‍ തൃക്കാല്ക്കല്‍ വെച്ചഭി-\\
വാദ്യവുംചെയ്തു വിഭീഷണനാദരാല്‍.\\
നക്തഞ്ചരേന്ദ്ര പ്രസാദത്തിനായ് രാമ-\\
ഭദ്രനതെല്ലാം പരിഗ്രഗിച്ചീടിനാന്‍.\\
‘ഇപ്പോള്‍ കൃതകൃത്യനായേനഹ’മെന്നു\\
ചില്‍പുരുഷന്‍ പ്രസാദിച്ചരുളീടിനാന്‍.\\
അഗ്രേ വിനീതനായ് വന്ദിച്ചു നില്ക്കുന്ന\\
സുഗ്രീവനെപ്പുനരാലിംഗനം ചെയ്തു\\
സന്തുഷ്ടനായരുള്‍ചെയ്തിതു രാഘവന്‍:\\
‘ചിന്തിച്ചതെല്ലാം ലഭിച്ചു നമുക്കെടോ!\\
ത്വല്‍സഹായത്വേന രാവണന്‍തന്നെ ഞാ-\\
നുത്സാഹമോടു വധിച്ചതു നിശ്ചയം\\
ലങ്കേശ്വരനായ് വിഭീഷണന്‍തന്നെയും\\
ശങ്കാവിഹീനമഭിഷേകവും ചെയ്തു.’
\end{verse}

%%31_seethaasveekaaram

\section{സീതാസ്വീകാരം}

\begin{verse}
പിന്നെ ഹനുമാനെ നോക്കിയരുള്‍ചെയ്തു\\
മന്നവന്‍: ’നീ പോയ് വിഭീഷണാനുജ്ഞയാ\\
ചെന്നു ലങ്കാപുരം പുക്കറിയിക്കണം\\
തന്വംഗിയാകിയ ജാനകിയോടിദം\\
നക്തഞ്ചരാധിപനിഗ്രഹമാദിയാം\\
വൃത്താന്തമെല്ലാം പറഞ്ഞു കേള്‍പ്പിക്കണം.\\
എന്നാലവളുടെ ഭാവവും വാക്കുമി-\\
ങ്ങെന്നോടു വന്നു പറക നീ സത്വരം.’\\
എന്നതുകേട്ടു പവനതനയനും\\
ചെന്നു ലങ്കാപുരം പ്രാപിച്ചനന്തരം\\
വന്നു നിശാചരര്‍ സല്‍ക്കരിച്ചീടിനാര്‍\\
നന്ദിതനായൊരു മാരുതപുത്രനും\\
രാമപാദാബ്ജവും ധ്യാനിച്ചിരിക്കുന്ന\\
ഭൂമീസുതയെ നമസ്കരിച്ചീടിനാന്‍.\\
വക്ത്രപ്രസാദമാലോക്യ കപിവരന്‍\\
വൃത്താന്തമെല്ലാം പറഞ്ഞു തുടങ്ങിനാന്‍:\\
‘ലക്ഷ്മണനോടും വിഭീഷണന്‍തന്നൊടും\\
സുഗ്രീവനാദിയാം വാനരന്മാരൊടും\\
രക്ഷോവരനാം ദശഗ്രീവനെക്കൊന്നു\\
ദുഃഖമകന്നു തെളിഞ്ഞു മേവീടിനാന്‍\\
ഇത്ഥം ഭവതിയോടൊക്കെപ്പറകെന്നു\\
ചിത്തം തെളിഞ്ഞരുള്‍ചെയ്തിതറിഞ്ഞാലും.’\\
സന്തോഷമെത്രയുണ്ടായിതു സീതയ്ക്കെ-\\
ന്നെന്തു ചൊല്ലാവതു! ജാനകീദേവിയും\\
ഗദ്ഗദവര്‍ണേന ചൊല്ലിനാ‘ളെന്തു ഞാന്‍\\
മര്‍ക്കടശ്രേഷ്ഠ! ചൊല്ലേണ്ടതു ചൊല്ലു നീ.\\
ഭര്‍ത്താവിനെക്കണ്ടുകൊള്‍വാനുപായമെ-\\
ന്തെത്ര പാര്‍ക്കേണമിനിയും ശുചൈവ ഞാന്‍\\
നേരത്തതിന്നു യോഗം വരുത്തീടു നീ\\
ധീരത്വമില്ലിനിയും പൊറുത്തീടുവാന്‍.’\\
വാതാത്മജനും രഘുവരന്‍തന്നോടു\\
മൈഥിലീഭാഷിതം ചെന്നു ചൊല്ലീടിനാന്‍.\\
ചിന്തിച്ചു രാമന്‍ വിഭീഷണന്‍തന്നോടു\\
സന്തുഷ്ടനായരുള്‍ചെയ്താന്‍, ‘വിരയെ നീ\\
ജാനകീദേവിയെച്ചെന്നു വരുത്തുക\\
ദീനതയുണ്ടുപോല്‍ കാണായ്കകൊണ്ടു മാം.\\
സ്നാനം കഴിപ്പിച്ചു ദിവ്യാംബരാഭര-\\
ണാനുലേപാദ്യലങ്കാരമണിയിച്ചു\\
ശില്പമായോരു ശിബികമേലാരോപ്യ\\
മല്‍പുരോഭാഗേ വരുത്തുക സത്വരം.’\\
മാരുതിതന്നോടുകൂടെ വിഭീഷണ-\\
നാരാമദേശം പ്രവേശിച്ചു സാദരം.\\
വൃദ്ധമാരായ നാരീജനത്തെക്കൊണ്ടു\\
മുഗ്ദ്ധാംഗിയെകുളിപ്പിച്ചു ചമയിച്ചു\\
തണ്ടിലെടുപ്പിച്ചുകൊണ്ടു ചെല്ലുന്നേര-\\
മുണ്ടായ് ചമഞ്ഞിതൊരു ഘോഷനിസ്വനം\\
വാനരവീരരും തിക്കിത്തിരക്കിയ-\\
ജ്ജാനകീദേവിയെക്കണ്ടു കൊണ്ടീടുവാന്‍\\
കൂട്ടമിട്ടങ്ങണയുന്നതു കണ്ടൊരു\\
യാഷ്ടികന്മാരണഞ്ഞാട്ടിയകറ്റിനാര്‍.\\
കോലാഹലം കേട്ടു രാഘവന്‍ കാരുണ്യ-\\
ശാലി വിഭീഷണന്‍ തന്നോടരുള്‍ചെയ്തു:\\
‘വാനരന്മാരെയുപദ്രവിപ്പാനുണ്ടോ\\
ഞാനുരചെയ്തിതു നിന്നോടിതെന്തെടോ?\\
ജാനകീദേവിയെക്കണ്ടാലതിനൊരു\\
ഹാനിയെന്തുള്ളതതു പറഞ്ഞീടു നീ.\\
മാതാവിനെച്ചെന്നു കാണുന്നതുപോലെ\\
മൈഥിലിയെച്ചെന്നു കണ്ടാലുമേവരും.\\
പാദചാരേണ വരേണമെന്നന്തികേ\\
മേദിനീനന്ദിനി: കിം തത്ര ദൂഷണം?’\\
കാര്യാര്‍ത്ഥമായ് പുരാ നിര്‍മിതമായൊരു\\
മായാജനകജാരൂപം മനോഹരം\\
കണ്ടു കോപംപൂണ്ടു വാച്യവാദങ്ങളെ-\\
പ്പുണ്ടരീകാക്ഷന്‍ ബഹുവിധം ചൊല്ലിനാന്‍.\\
ലക്ഷ്മണനോടു മായാസീതയും ശുചാ\\
തല്‍ക്ഷണേ ചൊല്ലിനാ’ളേതുമേ വൈകാതെ\\
വിശ്വാസമാശു മല്‍ഭര്‍ത്താവിനും മറ്റു\\
വിശ്വത്തില്‍ വാഴുന്നവര്‍ക്കും വരുത്തുവാന്‍\\
കുണ്ഡത്തിലഗ്നിയെ നന്നായ് ജ്വലിപ്പിക്ക\\
ദണ്ഡമില്ലേതുമെനിക്കതില്‍ ചാടുവാന്‍.’\\
സൗമിത്രിയുമതു കേട്ടു രാഘൂത്തമ-\\
സൗമുഖ്യഭാവമാലോക്യ സസംഭ്രമം\\
സാമര്‍ത്ഥ്യമേറുന്നവാനരന്മാരുമായ്\\
ഹോമകുണ്ഡം തീര്‍ത്തു തീയും ജ്വലിപ്പിച്ചു\\
രാമപാര്‍ശ്വം പ്രവേശിച്ചു നിന്നീടിനാന്‍.\\
ഭൂമീസുതയുമന്നേരം പ്രസന്നയായ്\\
ഭര്‍ത്താരമാലോക്യ ഭക്ത്യാ പ്രദക്ഷിണം\\
കൃത്വാ മുഹുസ്സ്വയം ബദ്ധാഞ്ജലിയൊടും\\
ദേവദ്വിജേന്ദ്രതപോധനന്മാരെയും\\
പാവകന്‍തന്നെയും വന്ദിച്ചു ചൊല്ലിനാള്‍:\\
‘ഭര്‍ത്താവിനെയൊഴിഞ്ഞന്യനെ ഞാന്‍ മമ\\
ചിത്തേ നിരൂപിച്ചതെങ്കിലതിന്നു നീ\\
സാക്ഷിയല്ലോ സകലത്തിനുമാകയാല്‍\\
സാക്ഷാല്‍ പരമാര്‍ത്ഥമിന്നറിയിക്ക നീ.’ എന്നു\\
പറഞ്ഞുടന്‍ മൂന്നു വലം വെച്ചു\\
വഹ്നിയില്‍ ചാടിനാള്‍ കിഞ്ചില്‍ ഭയം വിനാ.\\
ദുശ്ച്യവനാദികള്‍ വിസ്മയപ്പെട്ടിതു\\
നിശ്ചലമായിതു ലോകവുമന്നേരം.\\
ഇന്ദ്രനും കാലനും പാശിയും വായുവും\\
വൃന്ദാരകാധിപന്മാരും കുബേരനും\\
മന്ദാകിനീധരന്‍ താനും വിരിഞ്ചനും\\
സുന്ദ്രരിമാരാകുമപ്സരസ്ത്രീകളും\\
ഗന്ധര്‍വ കിന്നര കിംപുരുഷന്മാരും\\
ദന്ദശൂകന്മാര്‍ പിതൃക്കള്‍ മുനികളും\\
ചാരണ ഗുഹ്യക സിദ്ധസാദ്ധ്യന്മാരും\\
നാരദതുംബുരു മുഖ്യജനങ്ങളും\\
മറ്റും വിമാനാഗ്രചാരികളൊക്കവേ\\
ചുറ്റും നിറഞ്ഞിതു രാമന്‍ തിരുവടി\\
നിന്നരുളും പ്രദേശത്തിങ്കലന്നേരം\\
വന്ദിച്ചിതെല്ലാവരേയും നരേന്ദ്രനും.\\
രാമചന്ദ്രം പരമാത്മാനമന്നേരം\\
പ്രേമമുള്‍ക്കൊണ്ടു പുകഴ്ന്നു തുടങ്ങിനാര്‍.\\
‘സര്‍വലോകത്തിനും കര്‍ത്താ ഭവാനല്ലോ\\
സര്‍വത്തിനും സാക്ഷിയാകുന്നതും ഭവാന്‍\\
അജ്ഞാനവിഗ്രഹനാകുന്നതും ഭവാന്‍\\
അജ്ഞാനനാശകനാകുന്നതും ഭവാന്‍\\
സൃഷ്ടികര്‍ത്താവാം വിരിഞ്ചനാകുന്നതു-\\
മഷ്ടവസുക്കളിലഷ്ടമനായതും\\
ലോകത്തിനാദിയും മദ്ധ്യവുമന്തവു-\\
മേകനാം നിത്യസ്വരൂപന്‍ ഭവാനല്ലോ.\\
കര്‍ണങ്ങളായതുമശ്വിനീദേവകള്‍\\
കണ്ണുകളായതുമാദിത്യചന്ദ്രന്മാര്‍.\\
ശുദ്ധനായ് നിത്യനായദ്വയനായൊരു\\
മുക്തനാകുന്നതും നിത്യം ഭാവാനല്ലോ.\\
നിന്നുടെ മായയാ മൂടിക്കിടപ്പവര്‍\\
നിന്നെ മനുഷ്യനെന്നുള്ളിലോര്‍ത്തീടുവോര്‍.\\
നിന്നുടെ നാമസ്മരണമുള്ളോരുള്ളില്‍\\
നന്നായ് പ്രകാശിക്കുമാത്മപ്രബോധവും.\\
ദുഷ്ടനാം രാവണന്‍ ഞങ്ങളുടെ പദ-\\
മൊട്ടൊഴിയാതെയടക്കിനാന്‍ നിര്‍ദയം.\\
നഷ്ടനായാനവനിന്നു നിന്നാലിനി-\\
പ്പുഷ്ടസൗഖ്യം വസിക്കാം ത്വല്‍ക്കരുണയാ.’\\
ദേവകളിത്ഥം പുകഴ്ത്തും ദശാന്തരേ.\\
ദേവന്‍ വിരിഞ്ചനും വന്ദിച്ചു വാഴ്ത്തിനാന്‍:\\
‘വന്ദേ പദ പരമാനന്ദമദ്വയം\\
വന്ദേ പദമശേഷസ്തുതികാരണം\\
അദ്ധ്യാത്മജ്ഞാനികളാല്‍ പരിസേവിതം\\
ചിത്തസത്താമാത്രമവ്യയമീശ്വരം\\
സര്‍വഹൃദിസ്ഥിതം സര്‍വജഗന്മയം\\
സര്‍വലോകപ്രിയം സര്‍വജ്ഞമത്ഭുതം.\\
രത്നകിരീടം രവിപ്രഭം കാരുണ്യ-\\
രത്നാകരം രഘുനാഥം രമാവരം\\
രാജരാജേന്ദ്രം രജനീചരാന്തകം\\
രാജീവലോചനം രാവണനാശനം\\
മായാപരമജം മായാമയം മനു-\\
നായകം മായാവിഹീനം മധുദ്വിഷം\\
മാനവും മാനഹീനം മനുജോത്തമം\\
മാധുര്യസാരം മനോഹരം മാധവം\\
യോഗിചിന്ത്യം സദാ യോഗിഗമ്യം മഹാ-\\
യോഗവിധാനം പരിപൂര്‍ണമച്യുതം\\
രാമം രമണീയരൂപം ജഗദഭി-\\
രാമം സദൈവ സീതാഭിരാമം ഭജേ!’\\
ഇത്ഥം വിധാതൃസ്തുതികേട്ടു രാഘവന്‍\\
ചിത്തമാനന്ദിച്ചിരുന്നരുളുന്നേരം\\
ആശ്രയാശന്‍ ജഗദാശ്രയഭൂതയാ-\\
മാശ്രിതവത്സലയായ വൈദേഹിയെ\\
കാഴ്ചയായ്കൊണ്ടുവന്നാശുവണങ്ങിനാ-\\
നാശ്ചര്യമുള്‍ക്കൊണ്ടു നിന്നിതെല്ലാവരും.\\
‘ലങ്കേശനിഗ്രഹാര്‍ത്ഥം വിപിനത്തില്‍നി-\\
ന്നെങ്കലാരോപിതയാകിയ ദേവിയെ\\
ശങ്കാവിഹീനം പരിഗ്രഹിച്ചീടുക\\
സങ്കടം തീര്‍ന്നു ജഗത്ത്രയത്തിങ്കലും.’\\
പാവകനെ പ്രതിപൂജിച്ചു രാഘവന്‍\\
ദേവിയെ മോദാല്‍ പരിഗ്രഹിച്ചീടിനാന്‍.\\
പങ്കേരുഹാക്ഷനും ജാനകീദേവിയെ-\\
സ്വാങ്കേ സമാവേശ്യ ശോഭിച്ചിതേറ്റവും.
\end{verse}

%%32_devendrasthuthi

\section{ദേവേന്ദ്രസ്തുതി}

\begin{verse}
സംക്രന്ദനന്‍ തദാ രാമനെ നിര്‍ജര-\\
സംഘേന സാര്‍ധം വണങ്ങി സ്തുതിച്ചിതു:\\
‘രാമചന്ദ്ര! പ്രഭോ! പാഹി മാം പാഹി മാം\\
രാമഭദ്ര! പ്രഭോ! പാഹി മാം പാഹി മാം.\\
ഞങ്ങളെ രക്ഷിപ്പതിന്നു മറ്റാരുള്ള-\\
തിങ്ങനെ കാരുണ്യപീയൂഷവാരിധേ!\\
നിന്തിരുനാമാമൃതം ജപിച്ചീടുവാന്‍\\
സന്തതം തോന്നേണമെന്‍ പോറ്റി! മാനസേ.\\
നിന്‍ ചരിതാമൃതം ചൊല്‍വാനുമെപ്പോഴു-\\
മെന്‍ ചെവികൊണ്ടു കേള്‍പ്പാനുമനുദിനം\\
യോഗം വരുവാനനുഗ്രഹിച്ചീടണം\\
യോഗമൂര്‍ത്തേ! ജനകാത്മജാവല്ലഭ!\\
ശ്രീമഹാദേവനും നിന്തിരുനാമങ്ങള്‍\\
രാമരാമേതി ജപിക്കുന്നിതന്വഹം.\\
ത്വല്‍പാദതീര്‍ത്ഥം ശിരസി വഹിക്കുന്നി-\\
തെപ്പോഴുമാത്മശുദ്ധിക്കുമാവല്ലഭന്‍.’\\
ഏവം പലതരം ചൊല്ലി സ്തുതിച്ചോരു\\
ദേവേന്ദ്രനോടരുള്‍ചെയ്തിതു രാഘവന്‍:\\
‘മൃത്യുഭവിച്ച കപികുലവീരരെ-\\
യത്തല്‍കളഞ്ഞു ജീവിപ്പിക്കയും വേണം.\\
പക്വഫലങ്ങള്‍ കപികള്‍ ഭക്ഷിക്കുമ്പോ-\\
ളൊക്കെ മധുരമാക്കിച്ചമച്ചീടുക.\\
വാനരന്മാര്‍ക്കു കുടിപ്പാന്‍ നദികളും\\
തേനായൊഴുകേണ’മെന്നു കേട്ടിന്ദ്രനും\\
എല്ലാമരുള്‍ചെയ്തവണ്ണം വരികെന്നു\\
കല്യാണമുള്‍ക്കൊണ്ടനുഗ്രഹിച്ചീടിനാന്‍.\\
നന്നായുറങ്ങിയുണര്‍ന്നവരെപ്പോലെ\\
മന്നവന്‍ തന്നെത്തൊഴുതാരവര്‍കളും.\\
ചന്ദ്രചൂഡന്‍ പരമേശ്വരനും രാമ-\\
ചന്ദ്രനെ നോക്കിയരുള്‍ചെയ്തിതന്നേരം:\\
‘നിന്നുടെ താതന്‍ ദശരഥന്‍ വന്നിതാ\\
നിന്നു വിമാനമമര്‍ന്നു നിന്നെക്കാണ്മാന്‍.\\
ചെന്നുവണങ്ങുകെ’ന്നന്‍പോടു കേട്ടഥ\\
മന്നവന്‍ സംഭ്രമംപൂണ്ടു വണങ്ങിനാന്‍.\\
വൈദേഹിതാനും സുമിത്രാതനയനു-\\
മാദരവോടു വന്ദിച്ചു ജനകനെ.\\
ഗാഢം പുണര്‍ന്നു നിറുകയില്‍ ചുംബിച്ചു\\
ഗൂഢനായോരു പരമപുരുഷനെ\\
സൗമിത്രിതന്നെയും മൈഥിലിതന്നെയും\\
പ്രേമപൂര്‍വം പുണര്‍ന്നാനന്ദമഗ്നനായ്\\
ചിന്മയനോടു പറഞ്ഞു ദശരഥ-\\
‘നെന്മകനായിപ്പിറന്ന ഭവാനെ ഞാന്‍.\\
നിര്‍മലമൂര്‍ത്തേ! ധരിച്ചതിന്നാകയാല്‍\\
ജന്മമരണാദി ദുഃഖങ്ങള്‍ തീര്‍ന്നിതു.\\
നിന്മഹാമായ മോഹിപ്പിയായ്കെന്നെയും\\
കല്മഷനാശന! കാരുണ്യവാരിധേ!’\\
താതവാക്യം കേട്ടു രാമചന്ദ്രന്‍ തദാ\\
മോദേന പോവാനനുവദിച്ചീടിനാന്‍.\\
ഇന്ദ്രാദി ദേവകളോടും ദശരഥന്‍\\
ചെന്നമരാവതി പുക്കു മരുവിനാന്‍.\\
സത്യസന്ധന്‍തന്നെ വന്ദിച്ചനുജ്ഞായാ\\
സത്യലോകം ചെന്നു പുക്കുവിരിഞ്ചനും.\\
കാത്യായനീദേവിയോടും മഹേശ്വരന്‍\\
പ്രീത്യാ വൃഷാരൂഢനായെഴുന്നള്ളിനാന്‍.\\
ശ്രീരാമചന്ദ്രനിയോഗേന പോയിതു\\
നാരദനാദി മഹാമുനിവൃന്ദവും\\
പുഷ്കരനേത്രനെ വാഴ്ത്തി നിരാകുലം\\
പുഷ്കരചാരികളും നടന്നീടിനാര്‍.
\end{verse}

%%33_ayodhyayilekyaathra
\newpage

\section{അയോദ്ധ്യയിലേക്ക് യാത്ര}

\begin{verse}
മന്നവന്‍തന്നെ വന്ദിച്ചപേക്ഷിച്ചിതു\\
പിന്നെ വിഭീഷണനായ ഭക്തന്‍ മുദാ:\\
‘ദാസനാമെന്നെക്കുറിച്ചു വാത്സല്യമു-\\
ണ്ടേതാനുമെങ്കിലത്രൈവ സന്തുഷ്ടനായ്\\
മംഗലദേവതയാകിയ സീതയാ\\
മംഗളസ്നാനവുമാചരിച്ചീടണം.\\
മേളമായിന്നു വിരുന്നും കഴിഞ്ഞിങ്ങു\\
നാളെയങ്ങോട്ടെഴുന്നള്ളീടുകയുമാം.’\\
എന്നു വിഭീഷണന്‍ ചൊന്നതു കേട്ടുടന്‍\\
മന്നവര്‍ മന്നവന്‍ താനുമരുള്‍ചെയ്തു:\\
‘സോദരനായ ഭരതനയോദ്ധ്യയി-\\
ലാധിയും പൂണ്ടു സഹോദരന്‍തന്നൊടും\\
എന്നെയും പാര്‍ത്തിരിക്കുന്നിതു ഞാനവന്‍-\\
തന്നോടുകൂടിയൊഴിഞ്ഞലങ്കാരങ്ങള്‍\\
ഒന്നുമനുഷ്ഠിക്കയെന്നുള്ളതില്ലെടോ!\\
ചെന്നൊരു രാജ്യത്തില്‍ വാഴ്കയെന്നുള്ളതും.\\
സ്നാനാശനാദികളാചരിക്കെന്നതും\\
നൂനമവനോടു കൂടിയേയാവിതു.\\
എന്നു പതിന്നാലു സംവത്സരം തിക-\\
യുന്നതെന്നുള്ളതുംപാര്‍ത്തവന്‍ വാഴുന്നു.\\
ചെന്നീല ഞാനന്നുതന്നെയെന്നാലവന്‍\\
വഹ്നിയില്‍ ചാടി മരിക്കുമേ പിറ്റേന്നാള്‍.\\
എന്നതുകൊണ്ടുഴറുന്നിതു ഞാനിഹ\\
വന്നു സമയവുമേറ്റമടുത്തങ്ങു\\
ചെന്നുകൊള്‍വാന്‍ പണിയുണ്ടതിന്‍ മുന്നമേ\\
നിന്നില്‍ വാത്സല്യമില്ലായ്കയുമല്ല മേ\\
സല്‍ക്കരിച്ചീടു നീ സത്വരമെന്നുടെ\\
മര്‍ക്കടവീരരെയൊക്കവേ സാദരം.\\
പ്രീതിയവര്‍ക്കു വന്നാലെനിക്കും വരും\\
പ്രീതി, യതിന്നൊരു ചഞ്ചലമില്ല കേള്‍.\\
എന്നെക്കനിവോടു പൂജിച്ചതിന്‍ ഫലം\\
വന്നുകൂടും കപിവീരരെപ്പൂജിച്ചാല്‍.’\\
പാനാശന സ്വര്‍ണരത്നാംബരങ്ങളാല്‍\\
വാനരന്മാര്‍ക്കലംഭാവം വരുംവണ്ണം\\
പൂജയുംചെയ്തു കപികളുമായ് ചെന്നു\\
രാജീവനേത്രനെക്കൂപ്പി വിഭീഷണന്‍.\\
‘ക്ഷിപ്രമയോദ്ധ്യയ്ക്കെഴുന്നള്ളുവാനിഹ\\
പുഷ്പകമായ വിമാനവുമുണ്ടല്ലോ.’\\
രാത്രിഞ്ചരാധിപ നിത്ഥമുണര്‍ത്തിച്ച\\
വാര്‍ത്തകേട്ടാസ്ഥയോടും പുരുഷോത്തമന്‍\\
കാലത്തു നീ വരുത്തീടുകെന്നാനഥ\\
പൗലസ്ത്യയാനവും വന്നു വന്ദിച്ചിതു.\\
ജാനകിയോടുമനുജനോടും ചെന്നു\\
മാനവവീരന്‍ വിമാനവുമേറിനാന്‍.\\
അര്‍ക്കാത്മജാദി കപിവരന്മാരൊടും\\
നക്തഞ്ചരാധിപനോടും രഘൂത്തമന്‍\\
മന്ദസ്മിതം പൂണ്ടരുള്‍ചെയ്തിതാദരാല്‍:\\
‘മന്ദേതരം ഞാനയോദ്ധ്യയ്ക്കു പോകുന്നു.\\
മിത്രകാര്യം കൃതമായിതു നിങ്ങളാല്‍\\
ശത്രുഭയമിനി നിങ്ങള്‍ക്കകപ്പെടാ.\\
മര്‍ക്കടരാജ! സുഗ്രീവ! മഹാമതേ!\\
കിഷ്കിന്ധയില്‍ ചെന്നു വാഴ്ക നീ സൗഖ്യമായ്.\\
ആശരാധീശ! വിഭീഷണ! ലങ്കയി-\\
ലാശുപോയ് വാഴ്ക നീയും ബന്ധുവര്‍ഗവും.’\\
കാകുല്‍സ്ഥനിത്ഥമരുള്‍ചെയ്തനേരത്തു\\
വേഗത്തില്‍ വന്ദിച്ചവര്‍കളും ചൊല്ലിനാര്‍:\\
‘ഞങ്ങളും കൂടെ വിടകൊണ്ടയോധ്യയി-\\
ലങ്ങു കൗസല്യാദികളെയും വന്ദിച്ചു\\
മംഗലമാമ്മാറഭിഷേകവും കണ്ടു\\
തങ്ങള്‍ തങ്ങള്‍ക്കുള്ളവിടെ വാണീടുവാന്‍\\
ഉണ്ടാകവേണം തിരുമനസ്സെങ്കിലേ\\
കുണ്ഠത ഞങ്ങള്‍ക്കു തീരൂ ജഗല്‍പ്രഭോ!’\\
‘അങ്ങനെത്തന്നെ നമുക്കുമഭിമതം\\
നിങ്ങള്‍ക്കുമങ്ങനെ തോന്നിയതത്ഭുതം.\\
എങ്കിലോ വന്നു വിമാനമേറീടുവിന്‍\\
സങ്കടമെന്നിയേ മിത്രവിയോഗജം.’\\
സേനയാ സാര്‍ദ്ധം നിശാചരരാജനും\\
വാനരന്മാരും വിമാനമേറീടിനാര്‍.\\
സംസാരനാശനാനുജ്ഞയാ പുഷ്പകം\\
ഹംസസമാനം സമുല്‍പ്പതിച്ചു തദാ.\\
നക്തഞ്ചരേന്ദ്ര സുഗ്രീവാനുജപ്രിയാ\\
യുക്തനാം രാമനെക്കൊണ്ടു വിമാനവും\\
എത്രയും ശോഭിച്ചിതംബരാന്തേ തദാ\\
മിത്രബിംബം കണക്കേ ധനദാസനം\\
ഉത്സംഗസീമ്നി വിന്യസ്യ സീതാം ഭക്ത-\\
വത്സലന്‍ നാലുദിക്കും പുനരാലോക്യ\\
‘വത്സേ! ജനകാത്മജേ! ശൃണു വല്ലഭേ!\\
സത്സേവിതേ! സരസീരുഹലോചനേ!\\
പശ്യ ത്രികൂടാചലോത്തമാംഗസ്ഥിതം\\
വിശ്വവിമോഹനമായ ലങ്കാപുരം.\\
യുദ്ധാങ്കണം കാണ്‍കതിലിങ്ങു ശോണിത-\\
കര്‍ദമമാംസാസ്ഥിപൂര്‍ണം ഭയങ്കരം.\\
അത്രൈവ വാനര രാക്ഷസന്മാര്‍ തമ്മി-\\
ലെത്രയും ഘോരമായുണ്ടായി സംഗരം.\\
അത്രൈവ രാവണന്‍ വീണുമരിച്ചിതെ-\\
ന്നസ്ത്രമേറ്റുത്തമേ! നിന്നുടെ കാരണം.\\
കുംഭകര്‍ണന്‍ മകരാക്ഷനുമെന്നുടെ-\\
യമ്പുകൊണ്ടത്ര മരിച്ചിതു വല്ലഭേ!\\
വൃത്രാരിജിത്തുമതികായനും പുന-\\
രത്ര സൗമിത്രിതന്നസ്ത്രമേറ്റുത്തമേ!\\
വീണു മരിച്ചിതു പിന്നെയും മറ്റുള്ള\\
കൗണപന്മാരെ കപികള്‍ കൊന്നീടിനാര്‍.\\
സേതു ബന്ധിച്ചതും കാണേടോ! സാഗരേ\\
ഹേതു ബന്ധിച്ചതതിന്നു നീയല്ലയോ?\\
സേതുബന്ധം മഹാതീര്‍ത്ഥം പ്രിയേ! പഞ്ച-\\
പാതകനാശനം ത്രൈലോക്യപൂജിതം\\
കണ്ടാലുമുണ്ടാം ദുരിതവിനാശനം\\
കണ്ടാലുമങ്ങതിന്നത്രേ രാമേശ്വരം.\\
എന്നാല്‍ പ്രതിഷ്ഠിതനായ മഹേശ്വരന്‍\\
പന്നഗഭൂഷണന്‍ തന്നെ വണങ്ങു നീ.\\
അത്ര വന്നെന്നെശ്ശരണമായ് പ്രാപിച്ചി-\\
തുത്തമനായ വിഭീഷണന്‍ വല്ലഭേ!\\
പുഷ്കരനേത്രേ! പുരോഭുവി കാണെടോ!\\
കിഷ്കിന്ധയാകും കപീന്ദ്രപുരീമിമാം.’\\
ശ്രുത്വാ മനോഹരം ഭര്‍ത്തൃവാക്യം മുദാ\\
പൃത്ഥ്വീസുതയുമപേക്ഷിച്ചിതന്നേരം:\\
‘താരാദിയായുള്ള വാനരസുന്ദരി-\\
മാരെയും കണ്ടങ്ങു കൊണ്ടുപോയീടണം\\
കൗതൂഹലമയോദ്ധ്യാപുരിവാസിനാം\\
ചേതസി പാരമുണ്ടായ് വരും നിര്‍ണയം.\\
വാനരവീരരുമൊട്ടുനാളുണ്ടല്ലോ\\
മാനിനിമാരെപ്പിരിഞ്ഞിരുന്നീടുന്നു!\\
ഭര്‍ത്തൃവിയോഗജദുഃഖമിന്നെന്നോള-\\
മിത്രിലോകത്തിങ്കലാരറിഞ്ഞിട്ടുള്ളൂ?\\
എന്നാലിവരുടെ വല്ലഭമാരെയു-\\
മിന്നുതന്നേ കൂട്ടിക്കൊണ്ടുപോയീടണം.’\\
രാഘവന്‍ ത്രൈലോക്യനായകന്‍ തന്നിലു-\\
ള്ളാകൂതമപ്പോളറിഞ്ഞു വിമാനവും\\
ക്ഷോണീതലം നോക്കി മന്ദമന്ദം തദാ\\
താണതു കണ്ടരുള്‍ചെയ്തു രഘൂത്തമന്‍:\\
‘വാനര വീരരേ! നിങ്ങള്‍ നിജനിജ-\\
മാനിനിമാരെ വരുത്തുവിനേവരും.’\\
മര്‍ക്കടവീരരതുകേട്ടു മോദേന\\
കിഷ്കിന്ധ പുക്കു നിജാംഗനമാരെയും\\
പോകെന്നു ചൊല്ലി വിമാനം കരേറ്റിനാര്‍\\
ശാഖാമൃഗാധിപന്മാരും കരേറിനാര്‍.\\
താരാര്‍മകളായ ജാനകീദേവിയും\\
താരാരുമാദികളോടു മോദാന്വിതം\\
ആലോകനാലാപമന്ദഹാസാദി ഗാ-\\
ഢാലിംഗനഭ്രൂചലനാദികള്‍കൊണ്ടു\\
സംഭാവന ചെയ്തവരുമായ് വേഗേന\\
സംപ്രീതിപൂണ്ടു തിരിച്ചു വിമാനവും.\\
വിശ്വൈകനായകന്‍ ജാനകിയോടരു-\\
ളിച്ചെയ്തിതു പരമാനന്ദസംയുതം:\\
‘പശ്യ മനോഹരേ! ദേവി! വിചിത്രമാ-\\
മൃശ്യമൂകാചലമുത്തുംഗമെത്രയും\\
അത്രൈവ വൃത്രാരിപുത്രനെക്കൊന്നതും\\
മുഗ്ദ്ധാംഗി! പഞ്ചവടി നാമിരുന്നേടം\\
വന്ദിച്ചുകൊള്‍കഗസ്ത്യാശ്രമം ഭക്തിപൂ-\\
ണ്ടിന്ദീവരാക്ഷി! സുതീക്ഷ്ണാശ്രമത്തെയും.\\
ചിത്രകൂടാചലം പണ്ടു നാം വാണേട-\\
മത്രൈവ കണ്ടൂ ഭരതനെ നാമെടോ\\
ഭദ്രേ! മുദാ ഭരദ്വാജാശ്രമം കാണ്‍ക\\
ശുദ്ധികരം യമുനാതടശോഭിതം.\\
ഗംഗാനദിയതിന്നങ്ങേതതിന്നങ്ങു\\
ശൃംഗിവേരന്‍ ഗുനന്‍ വാഴുന്ന നാടെടോ!\\
പിന്നെസ്സരയൂനദിയതിന്നങ്ങേതു\\
ധന്യമയോദ്ധ്യാനഗരം മനോഹരേ!’\\
ഇത്ഥമരുള്‍ചെയ്തനേരത്തു രാഘവന്‍-\\
ചിത്തമറിഞ്ഞാശു താണു വിമാനവും\\
വന്ദിച്ചിതു ഭരദ്വാജമുനീന്ദ്രനെ\\
നന്ദിച്ചനുഗ്രഹം ചെയ്തു മുനീന്ദ്രനും.\\
രാമനും ചോദിച്ചിതപ്പോ‘ളയോദ്ധ്യയി-\\
ലാമയമേതുമൊന്നില്ലയല്ലീ മുനേ?\\
മാതൃജനത്തിനും സൗഖ്യമല്ലീ മമ\\
സോദരന്മാര്‍ക്കു മാചാര്യജനത്തിനും?’\\
താപസശ്രേഷ്ഠനരുള്‍ചെയ്തിതന്നേരം\\
‘താപമൊരുവര്‍ക്കുമില്ലയോദ്ധ്യാപുരേ.\\
നിത്യം ഭരതശത്രുഘ്നകുമാരന്മാര്‍\\
ശുദ്ധമാകും ഫലമൂലവും ഭക്ഷിച്ചു\\
ഭക്ത്യാ ജടാവല്ക്കലാദികളും പൂണ്ടു\\
സത്യസ്വരൂപനാം നിന്നെയും പാര്‍ത്തുപാര്‍-\\
ത്താഹന്ത! സിംഹാസനേ പാദുകം വെച്ചു\\
മോഹം ത്യജിച്ചു പുഷ്പാഞ്ജലിയും ചെയ്തു\\
കര്‍മങ്ങളെല്ലാമതിങ്കല്‍ സമര്‍പ്പിച്ചു\\
സമ്മതന്മാരായിരിക്കുന്നിതെപ്പോഴും.\\
ത്വല്‍പ്രസാദത്താലറിഞ്ഞിരിക്കുന്നിതു\\
ചില്‍പുരുഷ പ്രഭോ! വൃത്താന്തമൊക്കെ ഞാന്‍.\\
സീതാഹരണവും സുഗ്രീവസഖ്യവും\\
യാതുധാനന്മാരെയൊക്കെ വധിച്ചതും\\
യുദ്ധപ്രകാരവും മാരുതിതന്നുടെ\\
യുദ്ധപരാക്രമവും കണ്ടിതൊക്കവേ.\\
ആദിമദ്ധ്യാന്തമില്ലാതെ പരബ്രഹ്മ-\\
മേതും തിരിക്കരുതാതൊരു വസ്തു നീ.\\
സാക്ഷാല്‍ മഹാവിഷ്ണു നാരായണനായ\\
മോക്ഷപ്രദന്‍ നിന്തിരുവടി നിര്‍ണയം\\
ലക്ഷ്മീഭഗവതി സീതയാകുന്നതും\\
ലക്ഷ്മണനായതനന്തന്‍ ജഗല്‍പ്രഭോ!\\
ഇന്നു നീ ശുദ്ധമാക്കേണം മമാശ്രമം\\
ചെന്നയോദ്ധ്യാപുരം പുക്കീടടുത്തനാള്‍.’\\
കര്‍ണാമൃതമാം മുനിവാക്കു കേട്ടുപോയ്\\
പര്‍ണശാലാമകംപുക്കിതു രാഘവന്‍.\\
പൂജിതനായ് ഭ്രാതൃഭാര്യാസമന്വിതം\\
രാജീവനേത്രനും പ്രീതിപൂണ്ടീടിനാന്‍.
\end{verse}

%%34_hanumalbharathasamvaadam

\section{ഹനൂമല്‍ഭരതസംവാദം}

\begin{verse}
പിന്നെ മുഹൂര്‍ത്തമാത്രം നിരൂപിച്ചഥ\\
ചൊന്നാനനിലാത്മജനോടു രാഘവന്‍:\\
‘ചെന്നയോദ്ധ്യാപുരം പ്രാപിച്ചു സോദരന്‍-\\
തന്നെയും കണ്ടു വിശേഷമറിഞ്ഞു നീ\\
വന്നീടുകെന്നുടെ വൃത്താന്തവും പുന-\\
രൊന്നൊഴിയാതെയവനോടു ചൊല്ലണം.\\
പോകുന്ന നേരം ഗുഹനെയും ചെന്നുക-\\
ണ്ടേകാന്തമായറിയിച്ചീടവസ്ഥകള്‍.’\\
മാരുതി മാനുഷവേഷം ധരിച്ചു പോയ്\\
ശ്രീരാമവൃത്തം ഗുഹനെയും കേള്‍പ്പിച്ചു\\
സത്വരം ചെന്നു നന്ദിഗ്രാമമുള്‍പ്പുക്കു\\
ഭക്തനായീടും ഭരതനെ കൂപ്പിനാന്‍.\\
പാദുകവും വെച്ചു പൂജിച്ചനാരതം\\
ചേതസാ രാമനെ ധ്യാനിച്ചു ശുദ്ധനായ്\\
സോദരനോടുമമാത്യജനത്തോടു-\\
മാദരപൂര്‍വം ജടാവല്ക്കലം പൂണ്ടു\\
മൂലഫലവും ഭുജിച്ചു കൃശാംഗനായ്\\
ബാലനോടുംകൂടെ വാഴുന്നതുകണ്ടു\\
മാരുതിയും ബഹുമാനിച്ചിതേറ്റവു-\\
മാരുമില്ലിത്ര ഭക്തന്മാരവനിയില്‍\\
എന്നു കല്പിച്ചു വണങ്ങി വിനീതനായ്\\
നിന്നു മധുരമാമ്മാറു ചൊല്ലീടിനാന്‍:\\
‘അഗ്രജന്‍തന്നെ മുഹൂര്‍ത്തമാത്രേണ നി-\\
ന്നഗ്രേ നിരാമയം കാണാം ഗുണനിധേ!\\
സീതയോടും സുമിത്രാത്മജന്‍ തന്നോടു-\\
മാദരവേറും പ്ലവഗബലത്തൊടും\\
സുഗ്രീവനോടും വിഭീഷണന്‍ തന്നോടു-\\
മുഗ്രമായുള്ള രക്ഷോബലം തന്നൊടും\\
പുഷ്പകമാം വിമാനത്തിന്മേലേറിവ-\\
ന്നിപ്പോളിവിടെയിറങ്ങും ദയാപരന്‍.\\
രാവണനെക്കൊന്നു ദേവിയേയും വീണ്ടു\\
ദേവകളാലഭിവന്ദിതനാകിയ\\
രാഘവനെക്കണ്ടു വന്ദിച്ചു മാനസേ\\
ശോകവും തീര്‍ന്നു വസിക്കാമിനിച്ചിരം.’\\
ഇത്ഥമാകര്‍ണ്യ ഭരതകുമാരനും\\
ബദ്ധസമ്മോദം വിമൂര്‍ച്ഛിതനായ് വീണു.\\
സത്വരമാശ്വസ്തനായനേരം പുന-\\
രുത്ഥായ ഗാഢമായാലിംഗനം ചെയ്തു.\\
വാനരവീര ശിരസി മുദാ പര-\\
മാനന്ദബാഷ്പാഭിഷേകവും ചെയ്തിതു.\\
‘ദേവോത്തമനോ നരോത്തമനോ ഭവാ-\\
നേവമെന്നെക്കുറിച്ചിത്ര കൃപയോടും\\
ഇഷ്ടവാക്യം ചൊന്നതിന്നനുരൂപമായ്\\
തുഷ്ട്യാ തരുവതിനില്ല മറ്റേതുമേ.\\
ശോകം മദീയ കളഞ്ഞ ഭവാനു ഞാന്‍\\
ലോകം മഹാമേരുസാകം തരികിലും\\
തുല്യമായ് വന്നുകൂടാ പുനരെങ്കിലും\\
ചൊല്ലീടെടോ രാമകീര്‍ത്തനം സൗഖ്യദം.\\
മാനവനാഥനു വാനരന്മാരോടു\\
കാനനേ സംഗമമുണ്ടായതെങ്ങനെ?\\
വൈദേഹിയെക്കട്ടുകൊണ്ടവാറെങ്ങനെ\\
യാതുധാനാധിപനാകിയ രാവണന്‍?’\\
ഇത്തരം ചോചിച്ച രാജകുമാരനോ-\\
ടുത്തരം മാരുതപുത്രനും ചൊല്ലിനാന്‍:\\
‘എങ്കിലോ നിങ്ങളച്ചിത്രകൂടാചല-\\
ത്തിങ്കല്‍നിന്നാധികലര്‍ന്നു പിരിഞ്ഞനാള്‍\\
ആദിയായിന്നോളമുള്ളോരവസ്ഥക-\\
ളാദരമുള്‍ക്കൊണ്ടു ചൊല്ലുന്നതുണ്ടു ഞാന്‍.\\
ഒന്നൊഴിയാതെ തെളിഞ്ഞു കേട്ടീടുക\\
വന്നുപോം ദുഃഖവിനാശം തപോനിധേ!’\\
എന്നു പറഞ്ഞറിയിച്ചാനഖിലവും\\
മന്നവന്‍തന്‍ ചരിത്രം പവിത്രം പരം.\\
ശത്രുഘ്നമിത്രഭൃത്യാമാത്യവര്‍ഗവും\\
ചിത്രം വിചിത്രമെന്നോര്‍ത്തുകൊണ്ടീടിനാര്‍.
\end{verse}

%%35_ayodhyaapravesham

\section{അയോദ്ധ്യാപ്രവേശം}

\begin{verse}
ശത്രുഘ്നനോടു ഭരതകുമാരനു-\\
മത്യാദരം നിയോഗിച്ചാനനന്തരം;\\
‘പൂജ്യനാം നാഥനെഴുന്നള്ളുന്നേരത്തു\\
രാജ്യമലങ്കരിക്കേണമെല്ലാടവും.\\
ക്ഷേത്രങ്ങള്‍ തോറും ബലിപൂജയോടുമ-\\
ത്യാസ്ഥയാ ദീപാവലിയുമുണ്ടാക്കണം.\\
സൂതവൈതാളിക വന്ദിസ്തുതിപാഠ\\
കാദി ജനങ്ങളുമൊക്കെ വന്നീടണം.\\
വാദ്യങ്ങളെല്ലാം പ്രയോഗിക്കയും വേണം\\
പാദ്യാദികളുമൊരുക്കണമേവരും.\\
രാജദാരങ്ങളമാത്യജനങ്ങളും\\
വാജിഗജരഥപംക്തി സൈന്യങ്ങളും\\
വാരനാരീജനത്തോടുമലങ്കരി-\\
ച്ചാരൂഢമോദം വരണമെല്ലാവരും.\\
ചേര്‍ക്ക കൊടിക്കൂറകള്‍ കൊടിക്കൊക്കവേ\\
മാര്‍ഗമടിച്ചു തളിപ്പിക്കയും വേണം.\\
പൂര്‍ണകുംഭങ്ങളും ധൂപദീപങ്ങളും\\
തൂര്‍ണം പുരദ്വാരി ചേര്‍ക്ക സമസ്തരും.\\
താപസവൃന്ദവും ഭൂസുരവര്‍ഗവും\\
ഭൂപതി വീരരുമൊക്കെ വന്നീടണം.\\
പൗരജനങ്ങളാബാലവൃദ്ധാവധി\\
ശ്രീരാമനെക്കാണ്മതിന്നു വരുത്തണം.’\\
ശത്രുഘ്നനും ഭരതാജ്ഞയാ തല്‍പ്പുരം\\
ചിത്രമാമ്മാറങ്ങലങ്കരിച്ചീടിനാന്‍;\\
ശ്രീരാമദേവനെ കാണ്മതിന്നായ് വന്നു\\
പൗരജനങ്ങള്‍ നിറഞ്ഞിതയോദ്ധ്യയില്‍.\\
വാരണേന്ദ്രന്മാരൊരു പതിനായിരം\\
തേരുമവ്വണ്ണം പതിനായിരമുണ്ടു.\\
നൂറായിരം തുരഗങ്ങളുമുണ്ടഞ്ചു-\\
നൂറായിരമുണ്ടു കാലാള്‍പടകളും.\\
രാജനാരീജനം തണ്ടിലേറിക്കൊണ്ടു\\
രാജകുമാരനെക്കാണ്മാനുഴറിനാര്‍.\\
പാദുകാം മൂര്‍ദ്ധനിവെച്ചു ഭരതനും\\
പാദചാരേണ നടന്നു തുടങ്ങിനാന്‍.\\
ആദരവുള്‍ക്കൊണ്ടു ശത്രുഘ്നനാകിയ\\
സോദരന്‍താനും നടന്നാനതുനേരം.\\
പൂര്‍ണചന്ദ്രാഭമാം പുഷ്പകമന്നേരം\\
കാണായ് ചമഞ്ഞിതു ദൂരേ മനോഹരം.\\
പൗരജനാദികളോടു കുതൂഹലാല്‍\\
മാരുതപുത്രന്‍ പറഞ്ഞാനതുനേരം:\\
‘ബ്രഹ്മണാ നിര്‍മിതമാകിയ പുഷ്പകം\\
തന്മേലരവിന്ദനേത്രനും സീതയും\\
ലക്ഷ്മണസുഗ്രീവനക്തഞ്ചരാധിപ-\\
മുഖ്യമായുള്ളൊരു സൈന്യസമന്വിതം\\
കണ്ടുകൊള്‍വിന്‍ പരമാനന്ദവിഗ്രഹം\\
പുണ്ഡരീകാക്ഷം പുരുഷോത്തമം പരം.’\\
അപ്പോള്‍ ജനപ്രീതിജാതശബ്ദം ഘന-\\
മഭ്രദേശത്തോളമുല്‍പ്പതിച്ചൂബലാല്‍.\\
ബാലവൃദ്ധസ്ത്രീ തരുണവര്‍ഗാരവ\\
കോലാഹലം പറയാവതല്ലേതുമേ.\\
വാരണവാജിരഥങ്ങളില്‍ നിന്നവര്‍\\
പാരിലിറങ്ങി വണങ്ങിനാരേവരും.\\
ചാരുവിമാനാഗ്രസംസ്ഥിതനാം ജഗല്‍-\\
ക്കാരണഭൂതനെക്കണ്ടു ഭരതനും\\
മേരുമഹാഗിരിമൂര്‍ദ്ധനി ശോഭയാ\\
സൂര്യനെക്കണ്ടപോലെ വണങ്ങീടിനാന്‍.\\
ചില്‍പ്പുരുഷാജ്ഞയാ താണിതു മെല്ലവേ\\
പുഷ്പകമായ വിമാനവുമന്നേരം.\\
ആനന്ദബാഷ്പം കലര്‍ന്നു ഭരതനും\\
സാനുജനായ് വിമാനം കരേറീടിനാന്‍.\\
വീണു നമസ്കരിച്ചോരനുജന്മാരെ\\
ക്ഷോണീന്ദ്രനുത്സംഗസീമ്നി ചേര്‍ത്തീടിനാന്‍.\\
കാലമനേകം കഴിഞ്ഞു കണ്ടീടിന\\
ബാലകന്മാരെ മുറുകെത്തഴുകിനാന്‍.\\
ഹര്‍ഷാശ്രുധാരയാ സോദരമൂര്‍ദ്ധനി\\
വര്‍ഷിച്ചു വര്‍ഷിച്ചു വാത്സല്യപൂരവും\\
വര്‍ദ്ധിച്ചു വര്‍ദ്ധിച്ചു വാഴുന്ന നേരത്തു\\
ശത്രുഘ്നപൂര്‍വജനും ഭരതന്‍പദം\\
ഭക്ത്യാ വണങ്ങിനാനാശു സൗമിത്രിയെ\\
ശത്രുഘ്നനും വണങ്ങീടിനാനാദരാല്‍.\\
സോദരനോടും ഭരതകുമാരനും\\
വൈദേഹിതന്‍ പദം വീണു വണങ്ങിനാന്‍.\\
സുഗ്രീവനംഗദന്‍ ജാംബവാന്‍ നീലനു-\\
മുഗ്രനാം മൈന്ദന്‍ വിവിദന്‍ സുഷേണനും\\
താരന്‍ ഗജന്‍ ഗവയന്‍ ഗവാക്ഷന്‍ നളന്‍\\
വീരന്‍ വൃഷഭന്‍ ശരഭന്‍ പനസനും\\
ശൂരന്‍ വിനതന്‍ വികടന്‍ ദധിമുഖന്‍\\
ക്രൂരന്‍ കുമുദന്‍ ശതബലി ദുര്‍മുഖന്‍\\
സാരനാകും വേഗദര്‍ശി സുമുഖനും\\
ധീരനാകുംഗന്ധമാദനന്‍ കേസരി\\
മറ്റുമേവം കപിനായകന്മാരെയും\\
മുറ്റുമാനന്ദേന ഗാഢം പുണര്‍ന്നിതു\\
മാരുതിവാചാ ഭരതകുമാരനും\\
പുരുഷവേഷം ധരിച്ചാര്‍ കപികളും.\\
പ്രീതിപൂര്‍വം കുശലം വിചാരിച്ചതി-\\
മോദം കലര്‍ന്നു വസിച്ചാരവര്‍കളും\\
സുഗ്രീവനെക്കനിവോടു പുണര്‍ന്നഥ\\
ഗദ്ഗദവാചാ പറഞ്ഞു ഭരതനും:\\
‘നൂനം ഭവല്‍സഹായേന രഘുവരന്‍\\
മാനിയാം രാവണന്‍തന്നെ വധിച്ചതും\\
നാലു സുതന്മാര്‍ ദശരഥഭൂപനി-\\
ക്കാലമഞ്ചാമനായിച്ചമഞ്ഞു ഭവാന്‍.\\
പഞ്ചമഭ്രാതാ ഭവാനിനി ഞങ്ങള്‍ക്കു\\
കിഞ്ചനസംശയമില്ലെന്നറികെടോ!’\\
ശോകാതുരയായ കൗസല്യതന്‍ പദം\\
രാഘവന്‍ ഭക്ത്യാ നമസ്കരിച്ചീടിനാന്‍.\\
കാലേ കനിഞ്ഞു പുണര്‍ന്നാളുടന്‍ മുല-\\
പ്പാലും ചുരന്നിതു മാതാവിനന്നേരം.\\
കൈകേയിയാകിയ മാതൃപദത്തെയും\\
കാകുല്‍സ്ഥനാശു സുമിത്രാപദാബ്ജവും\\
വന്ദിച്ചു മറ്റുള്ള മാതൃജനത്തെയും\\
നന്ദിച്ചവരുമണച്ചു തഴുകിനാര്‍.\\
ലക്ഷ്മണനും മാതൃപാദങ്ങള്‍ കൂപ്പിനാന്‍\\
ഉള്‍ക്കാമ്പഴിഞ്ഞു പുണര്‍ന്നാരവര്‍കളും.\\
സീതയും മാതൃജനങ്ങളെ വന്ദിച്ചു\\
മോദമുള്‍ക്കൊണ്ടു പുണര്‍ന്നാരവര്‍കളും.\\
സുഗ്രീവനാദികളും തൊഴുതീടിനാ-\\
രഗ്രേ വിനീതയായ് നിന്നിതു താരയും.\\
ഭക്തിപരവശനായ ഭരതനും\\
ചിത്തമഴിഞ്ഞു തല്‍പ്പാദുകാദ്വന്ദ്വവും\\
ശ്രീരാമപാദാരവിന്ദങ്ങളില്‍ ചേര്‍ത്തു\\
പാരില്‍ വീണാശു നമസ്കരിച്ചീടിനാന്‍.\\
‘രാജ്യം ത്വയാ ദത്തമെങ്കല്‍ പുരാദ്യ ഞാന്‍\\
പൂജ്യനാം നിങ്കല്‍ സമര്‍പ്പിച്ചിതാദരാല്‍.\\
ഇന്നു മജ്ജന്മം സഫലമായ് വന്നിതു\\
ധന്യനായേനടിയനിന്നു നിര്‍ണയം.\\
വന്നു മനോരഥമെല്ലാം സഫലമായ്\\
വന്നിതു മല്‍ക്കര്‍മസാഫല്യവും പ്രഭോ!\\
പണ്ടേതിലിന്നു പതിന്മടങ്ങായുട-\\
നുണ്ടിഹ രാജഭണ്ഡാരവും ഭൂപതേ!\\
ആനയും തേരും കുതിരയും പാര്‍ത്തുകാ-\\
ണൂനമില്ലാതെ പതിന്മടങ്ങുണ്ടല്ലോ.\\
നിന്നുടെ കാരുണ്യമുണ്ടാകകൊണ്ടു ഞാ-\\
നിന്നയോളം രാജ്യമത്ര രക്ഷിച്ചതും\\
ത്യാജ്യമല്ലൊട്ടും ഭവാനാലിനിത്തവ\\
രാജ്യവും ഞങ്ങളേയും ഭുവനത്തെയും\\
പാലനം ചെയ്ക ഭവാനിനി മറ്റേതു-\\
മാലംബനമില്ല കാരുണ്യവാരിധേ!’
\end{verse}

%%36_raajyaabhishekam

\section{രാജ്യാഭിഷേകം}

\begin{verse}
ഇത്ഥം പറഞ്ഞ ഭരതനെക്കണ്ടവ-\\
രെത്രയും പാരം പ്രശംസിച്ചു വാഴ്ത്തിനാര്‍.\\
സന്തുഷ്ടനായ രഘുകുലനാഥനു-\\
മന്തര്‍മുദാ വിമാനേന മാനേനപോയ്\\
നന്ദിഗ്രാമേ ഭരതാശ്രമേ ചെന്നഥ\\
മന്ദം മഹീതലം തന്നിലിറങ്ങിനാന്‍.\\
പുഷ്പകമായ വിമാനത്തെ മാനിച്ചു\\
ചില്‍പുരുഷനരുള്‍ചെയ്താനനന്തരം:\\
‘ചെന്നു വഹിക്ക നീ വൈഷ്രവണന്‍ തന്നെ\\
മുന്നേക്കണക്കേ വിശേഷിച്ചു നീ മുദാ.\\
വന്നീടു ഞാന്‍ നിരൂപിക്കുന്ന നേരത്തു\\
നിന്നെ വിരോധിക്കയുമില്ലൊരുത്തനും.’\\
എന്നരുള്‍ചെയ്തതു കേട്ടു വന്ദിച്ചുപോയ്\\
ചെന്നളകാപുരി പുക്കു വിമാനവും.\\
സോദരനോടും വസിഷ്ഠനാമാചാര്യ-\\
പാദം നമസ്കരിച്ചു രഘുനായകന്‍.\\
ആശീര്‍വചനവും ചെയ്തു മഹാസന-\\
മാശു കൊടുത്തു വസിഷ്ഠമുനീന്ദ്രനും\\
ദേശികാനുജ്ഞയാ ഭദ്രാസനേ ഭുവി\\
ദാശരഥിയുമിരുന്നരുളീടിനാന്‍.\\
അപ്പോള്‍ ഭരതനും കേകയപുത്രിയു-\\
മുല്പലസംഭവപുത്രന്‍ വസിഷ്ഠനും\\
വാമദേവാദി മഹാമുനിവര്‍ഗവും\\
ഭൂമിദേവോത്തമന്മാരുമമാത്യരും\\
രക്ഷിക്ക ഭൂതലമെന്നപേക്ഷിച്ചിതു\\
ലക്ഷ്മീപതിയായ രാമനോടന്നേരം.\\
‘ബ്രഹ്മസ്വരൂപനാത്മാരാമനീശ്വരന്‍\\
ജന്മനാശാദികളില്ലാത മംഗലന്‍\\
നിര്‍മലന്‍ നിത്യന്‍ നിരുപമനദ്വയന്‍\\
നിര്‍മമന്‍ നിഷ്കളന്‍ നിര്‍ഗുണനവ്യയന്‍\\
ചിന്മയന്‍ ജംഗമാജംഗമാന്തര്‍ഗതന്‍\\
സന്മയന്‍ സത്യസ്വരൂപന്‍ സനാതനന്‍\\
തന്മഹാമായയാ സര്‍വലോകങ്ങളും\\
നിര്‍മിച്ചു രക്ഷിച്ചു സംഹരിക്കുന്നവന്‍.’\\
ഇങ്ങനെയങ്ങവര്‍ ചൊന്നതു കേട്ടള-\\
വിംഗിതജ്ഞന്‍ മന്ദഹാസപുരസ്സരം:\\
‘മാനസേ ഖേദമുണ്ടാകരുതാര്‍ക്കുമേ\\
ഞാനയോദ്ധ്യാധിപനായ് വസിക്കാമല്ലോ.\\
എങ്കിലതിന്നൊരുക്കീടുകെല്ലാ’മെന്നു\\
പങ്കജലോചനാനുജ്ഞയാ സംഭ്രമാല്‍\\
അശ്രുപൂര്‍ണാക്ഷനായ് ശത്രുഘ്നനും തദാ\\
ശ്മശ്രുനികൃന്തകന്മാരെ വരുത്തിനാന്‍.\\
സംഭാരവുമഭിഷേകാര്‍ഥമേവരും\\
സംഭരിച്ചീടിനാരാനന്ദചേതസാ.\\
ലക്ഷ്മണന്‍താനും ഭരതകുമാരനും\\
രക്ഷോവരനും ദിവാകരപുത്രനും\\
മുമ്പേ ജടാഭാരശോധനയും ചെയ്തു\\
സമ്പൂര്‍ണമോദം കുളിച്ചു ദിവ്യാംബരം-\\
പൂണ്ടു മാല്യാനുലേപാദ്യലങ്കാരങ്ങ-\\
ളാണ്ടു കുതൂഹലം കൈക്കൊണ്ടനാരതം\\
ശ്രീരാമദേവനും ലക്ഷ്മണനും പുന-\\
രാരൂഢമോദമലങ്കരിച്ചീടിനാര്‍\\
ശോഭയോടേ ഭരതന്‍ കുണ്ഡലാദിക-\\
ളാഭരണങ്ങളെല്ലാമനുരൂപമായ്.\\
ജാനകീദേവിയെ രാജനാരീജനം\\
മാനിച്ചലങ്കരിപ്പിച്ചാരതിമുദാ.\\
വാനരനാരീജനത്തിനും കൗസല്യ-\\
താനാദരാലലങ്കാരങ്ങള്‍ നല്കിനാള്‍.\\
അന്നേരമത്ര സുമന്ത്രര്‍ മഹാരഥം\\
നന്നായ് ചമച്ചു യോജിപ്പിച്ചു നിര്‍ത്തിനാന്‍.\\
രാജരാജന്‍ മനുവീരന്‍ ദയാപരന്‍\\
രാജയോഗ്യം മഹാസ്യന്ദനമേറിനാന്‍.\\
സൂര്യതനയനുമംഗദവീരനും\\
മാരുതിതാനും വിഭീഷണനും തദാ\\
ദിവ്യാംബരാഭരണാദ്യലങ്കാരേണ\\
ദിവ്യഗജാശ്വരഥങ്ങളിലാമ്മാറു\\
നാഥന്നകമ്പടിയായ് നടന്നീടിനാര്‍.\\
സീതയും സുഗ്രീവപത്നികളാദിയാം\\
വാനരനാരിമാരും വാഹനങ്ങളില്‍\\
സേനാപരിവൃതമാരായനാരതം\\
പിമ്പേ നടന്നിതു ശംഖനാദത്തൊടും\\
ഗംഭീരവാദ്യഘോഷങ്ങളോടും തദാ\\
സാരഥ്യവേല കൈക്കൊണ്ടാന്‍ ഭരതനും\\
ചാരുവെഞ്ചാമരം നക്തഞ്ചരേന്ദ്രനും\\
ശ്വേതാതപത്രം പിടിച്ചു ശത്രുഘ്നനും\\
സോദരന്‍ ദിവ്യവ്യജനവും വീയിനാന്‍.\\
മാനുഷവേഷം ധരിച്ചു ചമഞ്ഞുള്ള\\
വാനരേന്ദ്രന്മാര്‍ പതിനായിരമുണ്ടു\\
വാരണേന്ദ്രന്മാര്‍ കഴുത്തിലേറിപ്പരി-\\
വാരജനങ്ങളുമായ് നടന്നീടിനാര്‍.\\
രാമനീവണ്ണമെഴുന്നള്ളുന്നേരത്തു\\
രാമമാരും ചെന്നു ഹര്‍മ്യങ്ങളേറിനാര്‍.\\
കണ്ണിനാനന്ദപൂരം പുരുഷം പരം\\
പുണ്യപുരുഷമാലോക്യ നാരീജനം\\
ഗേഹധര്‍മങ്ങളുമൊക്കെ മറന്നുള്ളില്‍\\
മോഹപരവശമാരായ് മരുവിനാര്‍.\\
മന്ദമന്ദം ചെന്നു രാഘവന്‍ വാസവ-\\
മന്ദിരതുല്യമാം താതാലയം കണ്ടു\\
വന്ദിച്ചകംപുക്കു മാതാവുതന്‍ പദം\\
വന്ദിച്ചിതന്യപിതൃപ്രിയമാരെയും.\\
പ്രീത്യാ ഭരതകുമാരനോടന്നേര-\\
മാസ്ഥയാ ചൊന്നാനവിളംബിതം, ‘ഭവാന്‍\\
ഭാനുതനയനും നക്തഞ്ചരേന്ദ്രനും\\
വാനരനായകന്മാര്‍ക്കും യഥോചിതം\\
സൗഖ്യേന വാഴ്വതിന്നോരോ ഗൃഹങ്ങളി-\\
ലാക്കുക വേണമവരെ വിരയെ നീ.’\\
എന്നതു കേട്ടതു ചെയ്താന്‍ ഭരതനും\\
ചെന്നവരോരോ ഗൃഹങ്ങളില്‍ മേവിനാര്‍.\\
സുഗ്രീവനോടു പറഞ്ഞു ഭരതനു-\\
‘മഗ്രജനിപ്പോളഭിഷേകകര്‍മവും\\
മംഗലമാമ്മാറു നീ കഴിച്ചീടണ-\\
മംഗദനാദികളോടും യഥാവിധി.\\
നാലു സമുദ്രത്തിലും ചെന്നു തീര്‍ത്ഥവും\\
കാലേ വരുത്തുഗ മുമ്പിനാല്‍ വേണ്ടതും.\\
എങ്കിലോ ജാംബവാനും മരുല്‍പുത്രനു-\\
മംഗദന്‍താനും സുഷേണനും വൈകാതെ\\
സ്വര്‍ണകലശങ്ങള്‍ തന്നില്‍ മലയജ-\\
പര്‍ണേന വായ്ക്കെട്ടി വാരിയും പൂരിച്ചു\\
കൈണ്ടുവരികെ’ ന്നയച്ചോരളവവര്‍\\
കൊണ്ടുവന്നീടിനാരങ്ങനെ സത്വരം.\\
പുണ്യനദീജലം പുഷ്കരമാദിയാ-\\
മന്യതീര്‍ത്ഥങ്ങളിലുള്ള സലിലവു-\\
മൊക്കെ വരുത്തി മറ്റുള്ള പദാര്‍ത്ഥങ്ങള്‍\\
മര്‍ക്കടവൃന്ദം വരുത്തിനാര്‍ തല്‍ക്ഷണേ.\\
ശത്രുഘ്നനുമമാത്യൗഘവുമായ് മറ്റു\\
ശുദ്ധപദാര്‍ത്ഥങ്ങള്‍ സംഭരിച്ചീടിനാര്‍.\\
രത്നസിംഹാസനേ രാമനേയും ചേര്‍ത്തു\\
പത്നിയേയും വാമഭാഗേ വിനിവേശ്യ\\
വാമദേവന്‍ മുനിജാബാലി ഗൗതമന്‍\\
വാല്മീകിയെന്നവരോടും വസിഷ്ഠനാം\\
ദേശികന്‍ ബ്രാഹ്മണശ്രേഷ്ഠരോടും കൂടി\\
ദാശരഥിക്കഭിഷേകവും ചെയ്തിതു.\\
പൊന്നിന്‍കലശങ്ങളായിരത്തെട്ടുമ-\\
ങ്ങന്യൂനശോഭം ജപിച്ചാര്‍ മറകളും.\\
നക്തഞ്ചരേന്ദ്രനും വാനരവീരനും\\
രത്നദണ്ഡംപൂണ്ട ചാമരം വീയിനാര്‍.\\
ശത്രുഘ്നവീരന്‍ കുടപിടിച്ചീടിനാന്‍\\
ക്ഷത്രിയവീരരുപചരിച്ചീടിനാര്‍.\\
ലോകപാലന്മാരുപദേവതമാരു-\\
മാകാശമാര്‍ഗേ പുകഴ്ന്നു നിന്നീടിനാര്‍.\\
മാരുതന്‍ കൈയില്‍ കൊടുത്തയച്ചാന്‍ ദിവ്യ-\\
ഹാരം മഹേന്ദ്രന്‍ മനുകുലനാഥനു\\
സര്‍വരത്നോജ്ജ്വലമായ ഹാരം പുന-\\
രുര്‍വീശ്വരനുമലങ്കരിച്ചീടിനാന്‍.\\
ദേവഗന്ധര്‍വ യക്ഷാപ്സരോവൃന്ദവും\\
ദേവദേവേശ്വരനെ ബ്ഭജിച്ചീടിനാര്‍.\\
പൂര്‍ണഭക്ത്യാ പുഷ്പവൃഷ്ടിയും ചെയ്തുകാ-\\
രുണ്യനിധിയെ ബ്ഭജിച്ചിതെല്ലാവരും.\\
സ്നിഗ്ദ്ധ ദൂര്‍വാദളശ്യാമളം കോമളം\\
പത്മപത്രേക്ഷണം സൂര്യകോടിപ്രഭം\\
ഹാരകിരീടവിരാജിതം രാഘവം\\
മാരസമാനലാവണ്യം മനോഹരം\\
പീതാംബരപരിശോഭിതം ഭൂധരം\\
സീതയാ വാമാങ്കസംസ്ഥയാ രാജിതം\\
രാജരാജേന്ദ്രം രഘുകുലനായകം\\
രാജീവബാന്ധവവംശസമുത്ഭവം\\
രാവണനാശനം രാമം ദയാപരം\\
സേവകാഭീഷ്ടദം സേവ്യമനാമയം.\\
ഭക്തികൈക്കൊണ്ടുമാദേവിയോടും വന്നു\\
ഭര്‍ഗനുമപ്പോള്‍ സ്തുതിച്ചുതുടങ്ങിനാന്‍:\\
‘രാമായ ശക്തിയുക്തായ നമോ നമഃ\\
ശ്യാമളകോമളരൂപായ തേ നമഃ\\
കുണ്ഡലിനാഥതല്പായ നമോ നമഃ\\
കുണ്ഡലമണ്ഡിതഗണ്ഡായ തേ നമഃ\\
ശ്രീരാമദേവായ സിംഹാസനസ്ഥായ\\
ഹാരകിരീടധരായ നമോ നമഃ\\
ആദിമദ്ധ്യാന്തഹീനായ നമോ നമോ\\
വേദസ്വരൂപായ രാമായ തേ നമഃ\\
വേദാന്തവേദ്യായ വിഷ്ണവേ തേ നമോ\\
വേദജ്ഞവന്ദ്യായ നിത്യായ തേ നമഃ\\
ചന്ദ്രചൂഡന്‍ പുകഴ്ന്നോരുനേരം വിബു-\\
ധേന്ദ്രനും ഭക്ത്യാ പുകഴ്ത്തിത്തുടങ്ങിനാന്‍:\\
‘ബ്രഹ്മവരംകൊണ്ടഹംകൃതനായോരു\\
ദുര്‍മദമേറിയ രാവണരാക്ഷസന്‍\\
മല്‍പ്പദമെല്ലാമടക്കിനാന്‍ കശ്മലന്‍\\
തല്‍പ്പുത്രനെന്നെ ബന്ധിച്ചു മഹാരണേ.\\
ത്വല്‍പ്രസാദത്താലവന്‍ മൃതനാകയാ-\\
ലിപ്പോളെനിക്കു ലഭിച്ചിതു സൗഖ്യവും.\\
അന്നന്നിവണ്ണമോരോതരമാപത്തു\\
വന്നാലതും തീര്‍ത്തു രക്ഷിച്ചുകൊള്ളുവാന്‍\\
ഇത്ര കാരുണ്യമൊരുത്തര്‍ക്കുമില്ലെന്ന-\\
തുത്തമപൂരുഷ! ഞാന്‍ പറയേണമോ?\\
എല്ലാം ഭവല്‍ക്കരുണാബലമെന്നി മ-\\
റ്റില്ലൊരാലംബനം നാഥ! നമോസ്തുതേ.’\\
ആദിത്യരുദ്രവസുപ്രമുഖന്മാരു-\\
മാദിതേയോത്തമന്മാരുമതുനേരം\\
ആശരവംശവിനാശനനാകിയ\\
ദാശരഥിയെ വെവ്വേറെ പുകഴ്ത്തിനാര്‍:\\
‘യജ്ഞഭാഗങ്ങളെല്ലാമടക്കിക്കൊണ്ടാ-\\
നജ്ഞാനിയാകിയ രാവണരാക്ഷസന്‍.\\
ത്വല്‍ക്കടാക്ഷത്താലതൊക്കെ ലഭിച്ചിതു\\
ദുഃഖവും തീര്‍ന്നിതു ഞങ്ങള്‍ക്കു ദൈവമേ!\\
ത്വല്‍പ്പാദപദ്മം ഭജിപ്പതിന്നെപ്പോഴും\\
ചില്‍പുരുഷപ്രഭോ! നല്കീടനുഗ്രഹം\\
രാമായ രാജീവനേത്രായ ലോകാഭി-\\
രാമായ സീതാഭിരാമായ തേ നമഃ’\\
ഭക്ത്യാ പിതൃക്കളും ശ്രീരാമഭദ്രനെ-\\
ച്ചിത്തമഴിഞ്ഞു പുകഴ്ന്നുതുടങ്ങിനാര്‍:\\
‘ദുഷ്ടനാം രാവണന്‍ നഷ്ടനായാനിന്നു\\
തുഷ്ടരായ് വന്നിതു ഞങ്ങളും ദൈവമേ!\\
പുഷ്ടിയും വാച്ചിതു ലോകത്രയത്തിങ്ക-\\
ലിഷ്ടിയുമുണ്ടായിതിഷ്ടലാഭത്തിനാല്‍.\\
പിണ്ഡോദകങ്ങളുദിക്കായകാരണം\\
ദണ്ഡവും തീര്‍ന്നിതു ഞങ്ങള്‍ക്കു ദൈവമേ!’\\
യക്ഷന്മാരൊക്കെ സ്തുതിച്ചാരനന്തരം\\
രക്ഷോവിനാശനനാകിയ രാമനെ:\\
‘രക്ഷിതന്മാരായ് ചമഞ്ഞിതു ഞങ്ങളും\\
രക്ഷോവരനെ വധിച്ചമൂലം ഭവാന്‍.\\
പക്ഷീന്ദ്രവാഹന! പാപവിനാശന!\\
രക്ഷ രക്ഷ പ്രഭോ! നിത്യം നമോസ്തു തേ.’\\
ഗന്ധര്‍വസംഘവുമൊക്കെ സ്തുതിച്ചിതു\\
പങ് ക്തികണ്ഠാന്തകന്‍തന്നെ നിരാമയം:\\
‘അന്ധനാം രാവണന്‍ തന്നെ ബ്ഭയപ്പെട്ടു\\
സന്തതം ഞങ്ങളൊളിച്ചു കിടന്നതും.\\
ഇന്നു തുടങ്ങി തവ ചരിത്രങ്ങളും\\
നന്നായ് സ്തുതിച്ചു പാടിക്കൊണ്ടനാരതം\\
സഞ്ചരിക്കാമിനിക്കാരുണ്യവാരിധേ!\\
നിന്‍ ചരണാംബുജം നിത്യം നമോ നമഃ’\\
കിന്നരന്മാരും പുകഴ്ന്നു തുടങ്ങിനാര്‍\\
മന്നവന്‍തന്നെ മനോഹരമാംവണ്ണം:\\
‘ദുര്‍ന്നയമേറിയ രാക്ഷസരാജനെ-\\
ക്കൊന്നു കളഞ്ഞുടന്‍ ഞങ്ങളെ രക്ഷിച്ച\\
നിന്നെ ബ്ഭജിപ്പാനവകാശമുണ്ടായി-\\
വന്നതും നിന്നുടെ കാരുണ്യവൈഭവം.\\
പന്നഗതല്പേ വസിക്കും ഭവല്‍പ്പദം\\
വന്ദാമഹേ വയം വന്ദാമഹേ വയം.’\\
കിംപുരുഷന്മാര്‍ പരംപുരുഷന്‍ പദം\\
സംഭാവ്യ ഭക്ത്യാ പുകഴ്ന്നാരതിദ്രുതം:\\
‘കമ്പിതന്മാരായ് വയം ഭയം പൂണ്ടൊളി-\\
ച്ചേന്‍പോറ്റി! രാവണനെന്നു കേള്‍ക്കുന്നേരം\\
അംബരമാര്‍ഗേ നടക്കുമാറില്ലിനി\\
നിന്‍ പാദപത്മം ഭജിക്കായ് വരേണമേ.’\\
സിദ്ധ സമൂഹവുമപ്പോള്‍ മനോരഥം\\
സിദ്ധിച്ചമൂലം പുകഴ്ത്തിത്തുടങ്ങിനാര്‍:\\
‘യുദ്ധേ ദശഗ്രീവനെക്കൊന്നു ഞങ്ങള്‍ക്കു\\
ചിത്തഭയം തീര്‍ത്ത കാരുണ്യവാരിധേ!\\
രക്താരവിന്ദാഭപൂണ്ട ഭവല്‍പ്പദം\\
നിത്യം നമോ നമോ നിത്യം നമോ നമഃ.’\\
വിദ്യാധരന്മാരുമത്യാദരം പൂണ്ടു\\
ഗദ്യപദ്യാദികള്‍കൊണ്ടു പുകഴ്ത്തിനാര്‍:\\
‘വിദ്വജ്ജനങ്ങള്‍ക്കുമുള്ളില്‍ തിരിയാതെ\\
തത്ത്വാത്മനേ, പരമാത്മനേ, തേ നമഃ.’\\
ചാരുരൂപം തേടുമപ്സരസാം ഗണം\\
ചാരണന്മാരുരഗന്മാര്‍ മരുത്തുകള്‍\\
തുംബുരുനാരദഗുഹ്യകവൃന്ദവു-\\
മംബരചാരികള്‍ മറ്റുള്ളവര്‍കളും.\\
സ്പഷ്ടവര്‍ണോദ്യന്മധുരപദങ്ങളാല്‍\\
തുഷ്ട്യാ കനക്കെ സ്തുതിച്ചോരനന്തരം\\
രാമചന്ദ്രാനുഗ്രഹേണ സമസ്തരും\\
കാമലാഭേന നിജനിജ മന്ദിരം\\
പ്രാപിച്ചു താരകബ്രഹ്മവും ധ്യാനിച്ചു\\
താപത്രയവുമകന്നു വാണീടിനാര്‍.\\
സച്ചില്‍പരബ്രഹ്മപൂര്‍ണമാത്മാനന്ദ-\\
മച്യുതമദ്വയമേകമനാമയം\\
ഭാവനയാ ഭഗവല്‍പ്പദാംഭോജവും\\
സേവിച്ചിരുന്നാര്‍ ജഗത്ത്രയവാസികള്‍.\\
സിംഹാസനോപരി സീതയാ സംയുതം\\
സിംഹപരാക്രമം സൂര്യകോടിപ്രഭം\\
സോദരവാനരതാപസരാക്ഷസ-\\
ഭൂദേവവൃന്ദനിഷേവ്യമാണം പരം\\
രാമമഭിഷേകതീര്‍ത്ഥാര്‍ദ്രവിഗ്രഹം\\
ശ്യാമളം കോമളം ചാമീകരപ്രഭം\\
ചന്ദ്രബിംബാനനം ചാര്‍വായതഭുജം\\
ചന്ദ്രികാമന്ദഹാസോജ്ജ്വലം രാഘവം\\
ധ്യാനിപ്പവര്‍ക്കഭീഷ്ടാസ്പദം കണ്ടുക-\\
ണ്ടാനന്ദമുള്‍ക്കൊണ്ടിരുന്നിതെല്ലാവരും.
\end{verse}

%%37_vaanaraadikalkanugraham

\section{വാനരാദികള്‍ക്ക് അനുഗ്രഹം}

\begin{verse}
വിശ്വംഭരാപരിപാലനവും ചെയ്തു\\
വിശ്വനാഥന്‍ വസിച്ചീടും ദശാന്തരേ\\
സസ്യസമ്പൂര്‍ണമായ് വന്നിതവനിയും\\
ഉത്സവയുക്തങ്ങളായി ഗൃഹങ്ങളും\\
വൃക്ഷങ്ങളെല്ലാമതിസ്വാദു സംയുത-\\
പക്വങ്ങളോടു കലര്‍ന്നു നിന്നീടുന്നു.\\
ദുര്‍ഗന്ധപുഷ്പങ്ങളക്കാലമൂഴിയില്‍\\
സദ്ഗന്ധയുക്തങ്ങളായ് വന്നിതൊക്കവേ.\\
നൂറായിരം തുരഗങ്ങള്‍ പശുക്കളും\\
നൂറുനൂറായിരത്തില്‍പ്പുറം പിന്നെയും\\
മുപ്പതുകോടി സുവര്‍ണഭാരങ്ങളും\\
സുബ്രാഹ്മണര്‍ക്കു കൊടുത്തു രഘൂത്തമന്‍.\\
വസ്ത്രാഭരണമാല്യങ്ങളസംഖ്യമായ്\\
പൃത്ഥ്വീസുരോത്തമന്മാര്‍ക്കു നല്കീടിനാന്‍.\\
സ്വര്‍ണരത്നോജ്ജ്വലം മാല്യം മഹാപ്രഭം\\
വര്‍ണവൈചിത്ര്യമനഘമനുപമം\\
ആദിത്യപുത്രനു നല്കിനാനാദരാ-\\
ലാദിതേയാധിപപുത്രതനയനും\\
അംഗദദ്വന്ദ്വം കൊടുത്തോരനന്തരം\\
മംഗലാപാംഗിയാം സീതയ്ക്കു നല്കിനാന്‍\\
മേരുവും ലോകത്രയവും കൊടുക്കിലും\\
പോരാ വിലയതിനങ്ങനെയുള്ളൊരു\\
ഹാരം കൊടുത്തതു കണ്ടു വൈദേഹിയും\\
പാരം പ്രസാദിച്ചു മന്ദസ്മിതാന്വിതം\\
കണ്ഠദേശത്തിങ്കല്‍ നിന്നങ്ങെടുത്തിട്ടു\\
രണ്ടുകൈകൊണ്ടും പിടിച്ചു നോക്കീടിനാള്‍\\
ഭര്‍ത്തൃമുഖാബ്ജവും മാരുതി വക്ത്രവും\\
മദ്ധ്യേമണിമയമാകിയ ഹാരവും.\\
ഇംഗിതജ്ഞന്‍ പുരുഷോത്തമനന്നേരം.\\
മംഗലദേവതയോടു ചൊല്ലീടിനാന്‍:\\
‘ഇക്കണ്ടവര്‍കളിലിഷ്ടനാകുന്നതാ-\\
രുള്‍ക്കമലത്തില്‍ നിനക്കു മനോഹരേ!\\
നല്കീടവന്നു നീ മറ്റാരുമില്ല നി-\\
ന്നാകൂതഭംഗം വരുത്തുവാനോമലേ!’\\
എന്നതു കേട്ടു ചിരിച്ചു വൈദേഹിയും\\
മന്ദം വിളിച്ചു ഹനുമാനു നല്കിനാള്‍.\\
ഹാരവും പൂണ്ടു വിളങ്ങിനാനേറ്റവും\\
മാരുതിയും പരമാനന്ദസംയുതം.\\
അഞ്ജലിയോടും തിരുമുമ്പില്‍ നിന്നീടു-\\
മഞ്ജനാപുത്രനെക്കണ്ടു രഘുവരന്‍\\
മന്ദമരികേ വിളിച്ചരുള്‍ചെയ്തിതാ-\\
നന്ദപരവശനായ് മധുരാക്ഷരം:\\
‘മാരുതനന്ദന! വേണ്ടും വരത്തെ നീ\\
വീര! വരിച്ചുകൊള്‍കേതും മടിയാതെ.’\\
എന്നതു കേട്ടു വന്ദിച്ചു കപീന്ദ്രനും\\
മന്നവന്‍ തന്നോടപേക്ഷിച്ചരുളിനാന്‍:\\
‘സ്വാമിന്‍! പ്രഭോ! നിന്തിരുവടിതന്നുടെ\\
നാമവും ചാരുചരിത്രവുമുള്ളനാള്‍\\
ഭൂമിയില്‍ വാഴ്വാനനുഗ്രഹിച്ചീടണം\\
രാമനാമം കേട്ടുകൊള്‍വാനനാരതം.\\
രാമജപസ്മരണ ശ്രവണങ്ങളില്‍\\
മാമകമാനസേ തൃപ്തിവരാ വിഭോ!\\
മറ്റു വരം മമ വേണ്ടാ ധയാനിധേ!\\
മുറ്റുമിളക്കമില്ലാതൊരു ഭക്തിയു-\\
മുണ്ടായിരിക്കേണ’മെന്നതു കേട്ടൊരു\\
പുണ്ഡരീകാക്ഷനനുഗ്രഹം നല്കിനാന്‍:\\
‘മല്‍ക്കഥയുള്ളനാള്‍ മുക്തനായ് വാഴ്ക നീ\\
ഭക്തികൊണ്ടേ വരൂ ബ്രഹ്മത്വവും സഖേ!’\\
ജാനകീദേവിയും ഭോഗാനുഭൂതികള്‍\\
താനേ വരികെന്നനുഗ്രഹിച്ചീടിനാള്‍.\\
ആനന്ദബാഷ്പപരീതാക്ഷനായവന്‍\\
വീണു നമസ്കൃത്യ പിന്നെയും പിന്നെയും\\
രാമസീതാജ്ഞയാ പാരം പണിപ്പെട്ടു\\
രാമപാദാബ്ജവും ചിന്തിച്ചു ചിന്തിച്ചു\\
ചെന്നു ഹിമാചലം പുക്കു തപസ്സിനായ്;\\
പിന്നെഗ്ഗുഹനെ വിളിച്ചു മനുവരന്‍:\\
‘ഗച്ഛ സഖേ! പുരം ശൃംഗിവേരം ഭവാന്‍\\
മച്ചരിത്രങ്ങളും ചിന്തിച്ചു വാഴ്ക നീ.\\
ഭോഗങ്ങളെല്ലാം ഭുജിച്ചു ചിരം പുന-\\
രേകഭാവം ഭജിച്ചീടുകെന്നോടു നീ.’\\
ദിവ്യാംബരാഭരണങ്ങളെല്ലാം കൊടു-\\
ത്തവ്യാജഭക്തനു യാത്ര വഴങ്ങിനാന്‍.\\
പ്രേമഭാരേണ വിയോഗദുഃഖംകൊണ്ടു\\
രാമനാലാശ്ലിഷ്ടനായ ഗുഹന്‍ തദാ\\
ഗംഗാനദീപരിശോഭിതമായൊരു\\
ശൃംഗിവേരം പ്രവേശിച്ചു മരുവിനാന്‍.\\
മൂല്യമില്ലാത വസ്ത്രാഭരണങ്ങളും\\
മാല്യകളഭഹരിചന്ദനാദിയും\\
പിന്നെയും പിന്നെയും വേണ്ടുവോളം നല്കി\\
മന്നവന്‍ ഗാഢഗാഢം പുണര്‍ന്നാദരാല്‍\\
മര്‍ക്കടനായകന്മാര്‍ക്കും കൊടുത്തുപോയ്-\\
കിഷ്കിന്ധപൂകെന്നയച്ചരുളീടിനാന്‍\\
സുഗ്രീവനും വിയോഗേന ദുഃഖംകൊണ്ടു\\
കിഷ്കിന്ധപുക്കു സുഖിച്ചു മരുവിനാന്‍.\\
സീതാജനകനായീടും ജനകനെ\\
പ്രീതിയോടേ പറഞ്ഞാശ്ലേഷവും ചെയ്തു\\
സീതയെക്കൊണ്ടു കൊടുപ്പിച്ചോരോതരം\\
നൂതനപട്ടാംബരാഭരണാദിയും\\
നല്കി വിദേഹരാജ്യത്തിന്നു പോകെന്നു\\
പുല്കിക്കനിവോടു യാത്ര വഴങ്ങിനാന്‍.\\
കാശിരാജാവിനും വസ്ത്രാഭരണങ്ങ-\\
ളാശയാനന്ദം വരുമാറു നല്കിനാന്‍.\\
പിന്നെ മറ്റുള്ള നൃപന്മാര്‍ക്കുമൊക്കവേ\\
മന്നവന്‍ നിര്‍മലഭൂഷണാദ്യങ്ങളും\\
സമ്മാനപൂര്‍വം കൊടുത്തയച്ചീടിനാന്‍\\
സമ്മോദമുള്‍ക്കൊണ്ടുപോയാരവര്‍കളും.\\
നക്തഞ്ചരേന്ദ്രന്‍ വിഭീഷണനന്നേരം\\
ഭക്ത്യാ നമസ്കരിച്ചാന്‍ ചരണാംബുജം.\\
‘മിത്രമായ് നീ തുണച്ചോരുമൂലം മമ\\
ശത്രുക്കളെജ്ജയിച്ചേനൊരുജാതി ഞാന്‍.\\
ആചന്ദ്രതാരകം ലങ്കയില്‍ വാഴ്ക നീ\\
നാശമരികളാലുണ്ടാകയില്ല തേ.\\
എന്നെ മറന്നുപോകാതെ നിരൂപിച്ചു\\
പുണ്യജനാധിപനായ് വസിച്ചീടെടോ!\\
വിഷ്ണുലിംഗത്തെയും പൂജിച്ചു നിത്യവും\\
വിഷ്ണുപരായണനായ് വിശുദ്ധാത്മനാ\\
മുക്തനായ് വാണീടുകെ’ന്നു നിയോഗിച്ചു\\
മുക്താഫഫലമണിസ്വര്‍ണഭാരങ്ങളും\\
ആവോളവും കൊടുത്താശു പോവാനയ-\\
ച്ചാവിര്‍മുദാ പുണര്‍ന്നീടിനാന്‍ പിന്നെയും.\\
ചിത്തേ വിയോഗദുഃഖംകൊണ്ടു കണ്ണുനീ-\\
രത്യര്‍ത്ഥമിറ്റിറ്റു വീണും വണങ്ങിയും\\
ഗദ്ഗദവര്‍ണേന യാത്രയും ചൊല്ലിനാന്‍\\
നിര്‍ഗമിച്ചാനൊരുജാതി വിഭീഷണന്‍.\\
ലങ്കയില്‍ച്ചെന്നു സുഹൃജ്ജനത്തോടുമാ-\\
തങ്കമൊഴിഞ്ഞു സുഖിച്ചു വാണീടിനാന്‍.
\end{verse}

%%38_shreeraamanteraajyabhaarafalam

\section{ശ്രീരാമന്റെ രാജ്യഭാരഫലം}

\begin{verse}
ജാനകീദേവിയോടുംകൂടി രാഘവ-\\
നാനന്ദമുള്‍ക്കൊണ്ടു രാജഭോഗാന്വിതം\\
അശ്വമേധാദിയാം യാഗങ്ങളും ചെയ്തു\\
വിശ്വപവിത്രയാം കീര്‍ത്തിയും പൊങ്ങിച്ചു\\
നിശ്ശേഷസൗഖ്യം വരുത്തി പ്രജകള്‍ക്കു\\
വിശ്വമെല്ലാം പരിപാലിച്ചരുളിനാന്‍.\\
വൈധവ്യദുഃഖം വനിതമാര്‍ക്കില്ലൊരു\\
വ്യാധിഭയവുമൊരുത്തര്‍ക്കുമില്ലല്ലോ.\\
സസ്യപരിപൂര്‍ണയല്ലോ ധരിത്രിയും\\
ദസ്യുഭയവുമൊരുത്തര്‍ക്കുമില്ലല്ലോ.\\
ബാലമരണമകപ്പെടുമാറില്ല\\
കാലേ വരിഷിക്കുമല്ലോ ഘനങ്ങളും\\
രാമപൂജാപരന്മാര്‍ നരന്മാര്‍ ഭുവി\\
രാമനെ ധ്യാനിക്കുമേവരും സന്തതം\\
വര്‍ണാശ്രമങ്ങള്‍ തനിക്കുതനിക്കുള്ള-\\
തൊന്നുമിളക്കം വരുത്തുകില്ലാരുമേ.\\
എല്ലാവനുമുണ്ടനുകമ്പ മാനസേ\\
നല്ലതൊഴിഞ്ഞൊരു ചിന്തയില്ലാര്‍ക്കുമേ.\\
നോക്കുമാറില്ലാരുമേ പരദാരങ്ങ-\\
ളോര്‍ക്കയുമില്ല പരദ്രവ്യമാരുമേ.\\
ഇന്ദ്രിയനിഗ്രഹമെല്ലാവനുമുണ്ടു\\
നിന്ദയുമില്ല പരസ്പരമാര്‍ക്കുമേ.\\
നന്ദനന്മാരെപ്പിതാവു രക്ഷിക്കുന്ന-\\
വണ്ണം പ്രജകളെ രക്ഷിച്ചു രാഘവന്‍.\\
സാകേതവാസികളായ ജനങ്ങള്‍ക്കു\\
ലോകാന്തരസുഖമെന്തോന്നിതില്‍പ്പരം?\\
വൈകുണ്ഠലോകഭോഗത്തിനു തുല്യമായ്\\
ശോകമോഹങ്ങളകന്നു മേവീടിനാര്‍.
\end{verse}

%%39_raamaayanathintefalasruthi

\section{രാമായണത്തിന്റെ ഫലശ്രുതി}

\begin{verse}
അധ്യാത്മരാമായണമിദമെത്രയു-\\
മത്യുത്തമോത്തമം മൃത്യുഞ്ജയപ്രോക്തം\\
അധ്യയനംചെയ്കില്‍ മര്‍ത്ത്യനജ്ജന്മനാ\\
മുക്തി സിദ്ധിക്കുമതിനില്ല സംശയം.\\
മൈത്രീകരം ധനധാന്യവൃദ്ധിപ്രദം\\
ശത്രുവിനാശനമാരോഗ്യവര്‍ധനം\\
ദീര്‍ഘായുരര്‍ഥപ്രദം പവിത്രം പരം\\
സൗഖ്യപ്രദം സകലാഭീഷ്ടസാധകം\\
ഭക്ത്യാ പഠിക്കിലും ചൊല്കിലും തല്‍ക്ഷണേ\\
മുക്തനായീടും മഹാപാതകങ്ങളാല്‍\\
അര്‍ഥാഭിലാഷി ലഭിക്കും മഹാധനം\\
പുത്രാഭിലാഷി സുപുത്രനേയും തഥാ.\\
സിദ്ധിക്കുമാര്യജനങ്ങളാല്‍ സമ്മതം\\
വിദ്യാഭിലാഷി മഹാബുധനായ് വരും.\\
വന്ധ്യയുവതി കേട്ടീടുകില്‍ നല്ലൊരു\\
സന്തതിയുണ്ടാമവള്‍ക്കെന്നു നിര്‍ണയം.\\
ബദ്ധനായുള്ളവന്‍ മുക്തനായ് വന്നീടു-\\
മര്‍ഥി കേട്ടീടുകിലര്‍ഥവാനായ് വരും\\
ദുര്‍ഗങ്ങളെല്ലാം ജയിക്കായ് വരുമതി-\\
ദുഃഖിതന്‍ കേള്‍ക്കില്‍ സുഖിയായ് വരുമവന്‍.\\
ഭീതനിതു കേള്‍ക്കില്‍ നിര്‍ഭയനായ് വരും\\
വ്യാധിതന്‍ കേള്‍ക്കിലനാതുരനായ് വരും.\\
ഭൂതദൈവാത്മോത്ഥമായുടനുണ്ടാകു-\\
മാധികളെല്ലാമകന്നുപോം നിര്‍ണയം.\\
ദേവപിതൃഗണതാപസമുഖ്യന്മാ-\\
രേവരുമേറ്റം പ്രസാദിക്കുമത്യരം.\\
കല്മഷമെല്ലാമകലുമതെയല്ല\\
ധര്‍മാര്‍ത്ഥകാമമോക്ഷങ്ങള്‍ സാധിച്ചിടും.\\
അദ്ധ്യാത്മരാമായണം പരമേശ്വര-\\
നദ്രിസുതയ്ക്കുപദേശിച്ചിതാദരാല്‍\\
നിത്യവും ശുദ്ധബുദ്ധ്യാ ഗുരുഭക്തിപൂ-\\
ണ്ടദ്ധ്യയനം ചെയ്കിലുംമുദാ കേള്‍ക്കിലും\\
സിദ്ധിക്കുമെല്ലാമഭീഷ്ടമെന്നിങ്ങനെ\\
ബദ്ധമോദം പരമാര്‍ഥമിതൊക്കവേ\\
ഭക്ത്യാ പറഞ്ഞടങ്ങീ കിളിപ്പൈതലും\\
ചിത്തം തെളിഞ്ഞു കേട്ടു മഹാലോകരും.
\end{verse}

\begin{center}
ഇത്യദ്ധ്യാത്മരാമായണേ ഉമാമഹേശ്വരസംവാദേ\\
യുദ്ധകാണ്ഡം സമാപ്തം
\end{center}



