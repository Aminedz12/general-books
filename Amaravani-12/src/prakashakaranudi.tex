
\chapter{ಪ್ರಕಾಶಕರ ನುಡಿ}
ಶ್ರೀ ಕೀಲಕ ಸಂವತ್ಸರದ ವೈಶಾಖ ಶುಕ್ಲ ಏಕಾದಶೀ ಬುಧವಾರ(೮-೫-೧೯೬೮) ಶಿಷ್ಯರ ಅತ್ಯಂತ ಭಕ್ತಿಪೂರ್ವಕವಾದ ಪ್ರಾರ್ಥನೆಯನ್ನು ಮನ್ನಿಸಿ ವಿದ್ಯಾಪೀಠಗಳನ್ನು  ಶ್ರೀಗುರು- ದಂಪತಿಗಳು ಆರೋಹಣಮಾಡಿದ ಮಹಾಪರ್ವ. ಆದರ ನೆನಪನ್ನು ಹಸಿರಾಗಿಸುವ ವೈಶಾಖ ಶುಲ್ಕ ಏಕಾದಶೀ ನಮ್ಮ ಶ್ರೀಮಂದಿರದಲ್ಲಿ ಚಿರಸ್ಮರಣೀಯವಾದ ಪರ್ವ. ವಿದ್ಯಾಪುರುಷನಾದ ಮಹಾಗುರುವಿಗೆ ಕಾಣಿಕೆಯಾಗಿ ಅವರ ಪ್ರವಚನಗಳ ಸಂಕಲನವಾದ ಅಮರವಾಣೀ ಗ್ರಂಥಮಾಲೆಯ ೧೨ನೇ ಪುಷ್ಪವನ್ನು ಇಂದು ಆದ್ದರಿಂದ ಅವನ ಆಡಿದಾವರೆಗಳಲ್ಲಿ ಸಮರ್ಪಿಸುತ್ತಿದ್ದೇವೆ.

ಆಡು ಮುಟ್ಟದ ಸೊಪ್ಪಿಲ್ಲವೆಂಬ ನಾಣ್ನುಡಿಯುಂಟು. ಅಂತೆಯೇ ಶ್ರೀರಂಗ ಮಹಾಗುರುವಿನ ಗಮನಹರಿಯದ ಯಾವುದಾದರೂ ಜೀವನರಂಗವುಂಟೇ? ಎಂದೆನಿಸುತ್ತದೆ.

ಆರ್ಷವಿದ್ಯಾಸಂಪ್ರದಾಯವು ವೇದ-ಉಪವೇದ-ವೇದಾಂಗ-ಇತಿಹಾಸ ಪುರಾಣ ಇತ್ಯಾದಿಯಾಗಿ ಬೆಳದು ಬಂದಿರುವುದು ವಿದಿತವಾದ ವಿಷಯ. ಶ್ರೀರಂಗ ಮಹಾಗುರುವು ಆಗಿಂದಾಗ್ಗೆ ಈ ಆರ್ಷವಿದ್ಯೆಗಳಾ ಬಗ್ಗೆ ಅನುಗ್ರಹಿಸುತ್ತಿದ್ದ ಪ್ರವಚನಗಳನ್ನು, ವಚನಾಮೃತಗಳನ್ನು, ಆರ್ಷವಿದ್ಯಾಸಂಪ್ರದಾಯದ ಚೌಕಟ್ಟಿನಲ್ಲಿ, ಅಳವಡಿಸಿ ಮುದ್ರಣಮಾಡಾಲು ಯತ್ನಿಸಿದೆ. ಈ ಸಂಪುಟವು ವೇದಾಂಗ, ದರ್ಶನ, ಇತಿಹಾಸ, ಪುರಾಣಗಳ ಬಗ್ಗೆ ಕೆಲವು ಪ್ರವಚನಗಳನ್ನು ಒಳಗೊಂಡಿದೆ. ಛಂದಸ್ಸಿನ ವಿಷಯ ಅಮರವಾಣೀ ೮ನೆಯ ಸಂಪುಟದಲ್ಲಿ ಬಂದಿದೆ. ನಿರುಕ್ತ, ಕಲ್ಪ, ಜ್ಯೌತಿಷಗಳ ಬಗ್ಗೆ ಹಲವು ಮಾತಗಳು ೪ ಹಾಗೂ ೫ನೆಯ ಸಂಪುಟಗಳಲ್ಲಿವೆ. ಇವುಗಳ ಬಗ್ಗೆ ಸುದೀರ್ಘವಾದ ಪ್ರವಚನಗಳು ಉಪಲಬ್ಧವಾಗಿಲ್ಲ. ಇತಿಹಾಸ ಪುರಾಣಪ್ರಕರಣದಲ್ಲಿ ರಾಮಾಯಣ, ಭಾಗವತಗಳ ಬಗ್ಗೆ ಪ್ರವಚನಗಳಿವೆ.

ಈ ಸಂಪುಟಕೇ ಭೂಮಿಕೆಯನ್ನು ಬರೆದಿರುವವರು ನಮ್ಮ ಶ್ರೀಮಂದಿರದ ಹಿಂದಿನ ಕಾರ್ಯದರ್ಶಿಗಳಾದ ವಿದ್ವಾನ್ ಶೇಷಾಚಲಶರ್ಮರು. ಈ ಸಂಪುಟದಲ್ಲಿ ಪ್ರಕಟವಾಗಿರುವ ಪ್ರವಚನಗಳ ಮೂಲಬರಹವನ್ನು ಬಹುತೇಕವಾಗಿ ಪರಿಷ್ಕರಿಸಿದವರು ಹಾಗೂ ಸಂಸ್ಕೃತ ಶ್ಲೋಕಗಳಿಗೆ ಕನ್ನಡ ಅನುವಾದ ಮಾಡಿರುವವರು ಹಿರಿಯ ಸೋದರರಾದ ವಿದ್ವಾನ್ ಎನ್. ಎಸ್. ರಾಮಭದ್ರಾಚಾರ್ಯರು. ಉಪಶೀರ್ಷಿಕೆಗಳ ರಚನೆ, ಮುದ್ರಣಾದ ಕರಡಚ್ಚಿನ ಪರಿಶಿಲಿನೆ ಇತ್ಯಾದಿ ಕಾರ್ಯಗಳನ್ನು ನಿರ್ವಹಿಸಿದವರು ಡಾ|| ಕೆ.ಎಲ್. ಶಂಕರನಾರಾಯಣ ಜೋಯಿಸರು ಹಾಗೂ ಅವರ ಸಂಗಡಿಗರು. ಇವರೆಲ್ಲರ ಸೇವೆಯನ್ನು ಕೃತಜ್ಞತೆಯಿಂದ ಸ್ಮರಿಸುತ್ತೇನೆ.
ಪ್ರವಚನಗಳಾ ಸಮಕಾಲದಲ್ಲಿ ಅವುಗಳನ್ನು ಬರೆದಿಟ್ಟು ಈ ವಿಷಯಗಳು ನಮ್ಮವರೆಗೂ ತಲಪುವಂತೆ ಮಾಡಿರುವ ಸೋದರರಿಗೆಲ್ಲಾ ಧನ್ಯವಾದಗಳು. ಶ್ರೀ ಜಿ.ಕೆ. ಶ್ರೀನಿವಾದ ಮೂರ್ತಿ ಮೊದಲಾದ ಕೆಲವು ಸೋದರರು ನೀಡಿರುವ ಅರ್ಥಿಕ ನೆರವಿನಿಂದ ಈ ಸಂಪುಟದ ಬೆಲೆಯನ್ನು ಕಡಿಮೆಯಾಗಿಯೇ ಇಡಲು ಸಾಧ್ಯವಾಗಿದೆ. ಇವಹ ಹಾಗೂ ಪ್ರಕಾಶನ ಕಾರ್ಯದಲ್ಲಿ ಸಹಾಯಕರಾಗಿರುವ ಶ್ರೀಮಂದಿರದ ಸೋದರ ಸೋದರಿಯರಿಗೆಲ್ಲಾ ವಂದನೆಗಳು.

ಅಂದವಾಗಿ ಮುದ್ರಣ ಮಾಡಿಕೊಟ್ಟಿರುವ ಚೆಂಚು ಮುದ್ರಣಯದ ಶ್ರೀ. ಆರ್. ನರಸಿಂಹರವರು ನಮ್ಮ ಕೃತಜ್ಞತೆಗೆ ಪಾತ್ರರು.

ನಮಗೆ ಸ್ಫೂರ್ತಿಯ ಚಿಲುಮೆಯಾಗಿ, ನಮ್ಮನ್ನು ಹರಸಿ ಪ್ರೋತ್ಸಾಹಿಸುತ್ತಿರುವ ಪರಮಪೂಜ್ಯ ಶ್ರೀ ವಿಜಯಲಕ್ಷ್ಮೀ ಶ್ರೀಮಾತೆಯವರಿಗೆ ಪ್ರಾನಪ್ರಣಾಮಗಳು.

ವೈಶಾಖ ಶುಕ್ಲ ಏಕಾದಶೀ \hfill  ಶಂಕರನಾರಾಯಣದಾಸ

ಬಹುಧಾನ್ಯ ಸಮ್ವತ್ಸರ \hfill (ಶ್ರೀಕಂಠ)

೭-೫-೧೯೯೮  \hfill ಕಾರ್ಯದರ್ಶಿ
