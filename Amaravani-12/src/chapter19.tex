\chapter{ಶ್ರೀಮದ್ರಾಮಾಯಣ ಪಾರಾಯಣದ ಪರಿಸಮಾಪ್ತಿಯಲ್ಲಿ ಪುರುಷಸೂಕ್ತ ಪಠನದ ಔಚಿತ್ಯ}

(ಶ್ರೀ ರಾಮಾಯಣ ಪಾರಾಯಣದ ಪರಿಮಾಪ್ತಿಯಲ್ಲಿ, ಪಟ್ಟಾಭಿಷೇಕ ಸಂದರ್ಭದಲ್ಲಿ, ಅಂಜಲಿಸಮರ್ಪಣೆ ಮಾಡುವಾಗ ``ಪುರುಷಸೂಕ್ತ" ಪಠನದ ಔಚಿತ್ಯ ಕುರಿತು ಶಿಷ್ಯರೋಬ್ಬರ ಪ್ರಶ್ನೆಗೆ ಶ್ರೀ ಗುರುವು ಕೊಟ್ಟ ಸಮಾಧಾನ)

\section*{ಆಧಿವ್ಯಾಧಿಹರವಾದ ಸಿದ್ಧೌಷದವಾಗಬಹುದು ಕಾವ್ಯ}

ಜೀವನದಲ್ಲಿ ಅನೇಕ ರಸಭರಿತ ಕಾವ್ಯಗಳ ಆನಂದವನ್ನು ಮನುಷ್ಯನು ಅನುಭವಿಸುತ್ತಾನೆ. ಅವುಗಳಲ್ಲಿ ಶುದ್ಧರಸವಿರಬಹುದು, ಸಂಕರ ರಸವಿರಬಹುದು. ಅಡುಗೆಯಲ್ಲೂ ಉಪ್ಪು-ಹುಳಿ-ಖಾರ ಮುಂತಾದ ಷಡ್ರಸಗಳನ್ನು ಅನುಭವಿಸಿರುವುದುಂಟು. ಆದರೆ ಯಾವುಯಾವುದೋ ರಸಕೊಡುವುದಕ್ಕಿಂತ ಸರಿಯಾದ ರಸ ಕೊಟ್ಟು ಜೀವನ ಪಾವನ ಮಾಡಿಕೊಳ್ಳುವುದು ಉತ್ತಮ. ಮೊದಲು ಜೀವನದಲ್ಲಿ ರಸವೆಂದರೇನು? ಎಂಬುದನ್ನು ತಿಳಿಯಬೇಕು. ಕಾಯಿಲೆ ಏನು? ಅದರ ನಿವಾರಣೆ ಹೇಗೆ? ಎಂಬುದನ್ನರಿತು ಆ ನಿವಾರಣೆಗನುಗುಣವಾದ ಔಷಧವನ್ನು ಕೊಟ್ಟರೆ ಅದು ಸಿದ್ಧೌಷಧವಾಗುತ್ತದೆ. ಹಾಗೆಯೇ ಆಧಿ-ವ್ಯಾಧಿಹರವಾದ ಸಿದ್ಧೌಷಧವಾಗಬಹುದು, ಕಾವ್ಯ. ತನ್ನ ಒಳ-ಹೊರಗಣ ಸೌಂದರ್ಯದಿಂದ ಅದು ಆಧಿವ್ಯಾಧಿಗಳನ್ನು ನಿವಾರಣೆಮಾಡುತ್ತದೆ. 

\section*{ಕಾವ್ಯರಚನೆಗೆ ಆದಕವಿಯಲ್ಲಿದ ಸಜ್ಜು}

ಹೀಗಿರುವಾಗ ವಾಲ್ಮೀಕಿ ಮಹರ್ಷಿಗಳು ತತ್ತ್ವಜ್ಞರಾಗಿ, ತಪಸ್ಸ್ವಾಧ್ಯಾಯ ನಿರತರಾಗಿ, ವಾಗ್ವಿದಾಂವರರಾಗಿ, ಮುನಿಪುಂಗವರಾದ ನಾರದರ ಅನುಗ್ರಹದಿಂದ ಬಂದ ಉಪದೇಶನುಸಾರ ಒಂದು ಸತ್ಕಾವ್ಯವನ್ನು ರಚಿಸಿದರು. ಅವರೂ ತಪಸ್ವಿಗಳು. ಅವರ ಆ ತಪಸ್ಸಿಗೆ ಫಲವೇನು? ಆ ಸ್ವಾಧ್ಯಾಯಕ್ಕೆ ಫಲವೇನು? ಅವುಗಳಲ್ಲಿಯೇ ನಿರತನಾಗಿರುವ ಮನುಷ್ಯನ ವೈಶಿಷ್ಟ್ಯವೇನು? ತಮ್ಮ ಆ ಮನಸ್ಸಿಗೆ ಬಂದ ಅಭಿಪ್ರಾಯವನ್ನು, ಆತ್ಮದಲ್ಲಿರುವುದನ್ನು ಅಂತಃಕರಣದಲ್ಲಿರುವುದನ್ನು ವಾಕ್ಕಿನಲ್ಲಿ ಹೇಳುವ ಯೋಗ್ಯತೆಯೂ ಉಂಟು ಆ ಮಹಾಕವಿಗೆ. ಲೋಕದಲ್ಲಿ ಕೆಲವರಿಗೆ ತಮ್ಮ ಮನಸ್ಸಿನಲ್ಲಿರುವುದನ್ನು ಹಾಗೆಯೇ ಹೊರಗೆ ಹೇಳಲಾಗುವುದಿಲ್ಲ. ಕೆಲವರಿಗೂ ತಮ್ಮಲ್ಲಿರುವ ಬೇನೆಯನ್ನೂ ಹೇಳಿಕೊಳ್ಳಲು ಸಾಧ್ಯವಿಲ್ಲ. ಕೆಲವರಿಗೆ ಅದು ಸುಲಭಸಾಧ್ಯ.

\section*{ನಾರದರ ಹಿನ್ನೆಲೆ}

ಮಹರ್ಷಿ ನಾರದರನ್ನು ಪ್ರಶ್ನಿಸಿ, ಅವರು ಕೊಟ್ಟ ಉತ್ತರವನ್ನು ಮನಸ್ಸಿನಲ್ಲಿ ಧರಿಸಿಕೊಂಡು ಮನನಶೀಲರಾದ ತಪಸ್ವಿ  ವಾಲ್ಮೀಕಿಗಳು  ಏನು ಕಂಡುಕೊಂಡರು? ಎಂದರೆ ಎರಡನ್ನು. ಈ ಶರೀರವೆಂಬ ಪಟ್ಟಣದಲ್ಲಿ  ಹತ್ತು ಮುಖಗಳವೆ. ಮುಖವೆಂದರೆ ಮುಂದಕ್ಕೆ  ಪ್ರವೇಶ ಮಾಡುವುದು. ನದೀಮುಖ; ಸೇನಾಮುಖ ಎಂಬಂತೆ. ಪುರುಷನು ಕಣ್ಣಿನ ಮೂಲಕವೂ ಪ್ರವೇಶ ಮಾಡುತ್ತಾನೆ. ಮೂಗಿನ ಮೂಲಕವೂ ಪ್ರವೇಶ ಮಾಡುತ್ತಾನೆ. ಹೀಗೆ ಪುರುಷನು ದಶಮುಖನಾಗಿರುತ್ತಾನೆ. ತನ್ನ ರಾಜ್ಯಕ್ಕೆ ಇಂದ್ರಿಯಗಳನ್ನೆಲ್ಲ ಒಂದೊಂದು ಕಾರ್ಯನಿರ್ವಹಣೆಯಲ್ಲಿ ತೆಗೆದುಕೊಂಡಿರುವ ಯಾವ ಬ್ರಹ್ಮವುಂಟೋ, ಅದನ್ನು ಈ ಬ್ರಹ್ಮಪುರದಲ್ಲಿ ಜ್ಞಾನಿಯಾದವನು,
\begin{shloka}
ನ ಶಕ್ಯಃ ಚಕ್ಷುಷಾ ದ್ರಷ್ಟುಂ ದೇಹೇ ಸೂಕ್ಷ್ಮತಮೋ ವಿಭುಃ|\label{246}\\
ದೃಶ್ಯತೇ ಜ್ಞಾನಚಕ್ಷುರ್ಭಿಃ ತಪಶ್ಚಕ್ಷುರ್ಭಿರೇವ ಚ|| 
\end{shloka}
ಎಂಬಂತೆ ಆರಿತುಕೊಳ್ಳುತ್ತಾನೆ. ಜ್ಞಾನ-ತಪಸ್ಸುಗಳಿಂದ ತಿಳಿಯಬೇಕಾದ ಸೂಕ್ಷ್ಮತಮನಾದ ಯಾವ ವಿಭುವುಂಟೋ, ಅದನ್ನು ಮನನಶೀಲರಾದ ನಾರದಮುನಿಗಳು ಅರಿತು ಕೊಂಡಿರಲು; ಅಂತಹವರ ಮುಂದೆ ಕುಳಿತು ಇಷ್ಟು ಲಕ್ಷಣವುಳ್ಳವನು ಯಾವನಿದ್ದಾನೆ ಎನ್ನುತ್ತಾರೆ ವಾಲ್ಮೀಕಿಗಳು. ಆ‌ ಎಲ್ಲಾ ಗುಣಗಳನ್ನು ನೋಡಿದರೆ ಸಾಕ್ಷಾತ್ ಆತ್ಮನಲ್ಲಿರುವ ಗುಣಗಳಾವುವುಂಟೋ ಅವುಗಳಿಂದ ಕೊಡಿಕೊಂಡಿದ್ದಾನೆ ಎಂಬುದು ಗೊತ್ತಾಗುತ್ತೆ. `ಅದನ್ನು ತೆಗೆದುಕೊಂಡು ಬನ್ನಿ ಬಹಳ ಚೆನ್ನಾಗಿದೆ' ಅಂದರೆ, ಆ ಪದಾರ್ಥ ಮತ್ತು ಅದು ಇರುವ ಜಾಗ ಇವೆರಡನ್ನೂ ತಿಳಿದಿರಬೇಕು. ಹೇಳಿದವರ ಯೋಗ್ಯತೆಯೇನು? ಅವರು (ನಾರದರು) ತಪಸ್ಸ್ವಾಧ್ಯಾಯ ನಿರತರು, ವಾಗ್ವಿದಾಂವರರು.

\section*{ಕಥನಾಯಕನ ಒಳಹೊರ ಸೌಂದರ್ಯ}

ಅವರನ್ನು ಕುರಿತು ವಾಲ್ಮೀಕಿಗಳ ಪ್ರಾರ್ಥನೆ -

\begin{shloka}
ಮಹರ್ಷೇ ತ್ವಂ ಸಮರ್ಥೋಽಸಿ ಜ್ಞಾತುಮೇವಂವಿಧಂ ನರಮ್||\label{246a}\\
ಶ್ರುತ್ವಾ ಚೈತತ್ ತ್ರಿಲೋಕಜ್ಞೋ ವಾಲ್ಮೀಕೇರ್ನಾರದೋ ವಚಃ||\label{246b}\\
ಶ್ರೂಯತಾಂ ಇತಿ ಚಾಮಂತ್ರ್ಯ ಪ್ರಹೃಷ್ಟೋ ವಾಕ್ಯಮಬ್ರವೀತ್||\\
ಬಹವೋ ದುರ್ಲಭಾಶ್ವೈವ ಯೇ ಕೀರ್ತಿತಾ ಗುಣಾಃ|\label{247c}\\
ಮುನೇ ವಕ್ಷ್ಯಾಮ್ಯಹಂ ಬುದ್ಧ್ವಾ ತೈರ್ಯುಕ್ತಃ ಶ್ರೂಯತಾಂ ನರಃ||\\
\end{shloka}

ಅವರೂ(ನಾರದರೂ) ``ಬುದ್ಧ್ವಾ" ಎಂದು ಹೇಳುತ್ತಾರೆ. ತಿಳಿದು ಹೇಳಬೇಕು. ಕೇಳಿ ಹೇಳುವುದರಲ್ಲಿ, ವಿಚಾರಮಾಡಿ ಹೇಳುವುದರಲ್ಲಿ ಬುದ್ಧ್ವಾ ಇಲ್ಲ. ರಾಮನ ಬಗ್ಗೆ ಹೇಳುವಾಗ ಅವನಿಗೆದೇವರ ಹೆಸರನ್ನು ಎಲ್ಲೂ ಹೇಳಿಲ್ಲ; ಉದ್ದಕ್ಕೂ ಮನುಷ್ಯನ ಹೆಸರೆ ಅವನಾದರೋ ಹೇಗಿದ್ದಾನೆ? 

\begin{shloka}
ನಿಯತಾತ್ಮಾ ಮಹಾವೀರ್ಯೋ ದ್ಯುತಿಮಾನ್ ಧೃತಿಮಾ ವಶೀ|\label{247b}\\
ಪ್ರಜಾಪತಿಸಮಃ ಶ್ರೀ ಮಾನ್ 
\end{shloka}

ಒಳಗಿನ ಸ್ವರೂಪವನ್ನು ವರ್ಣನೆ ಮಾಡಿಯಾಯಿತು; ಇನ್ನು ಹೊರಗಡೆಯ ಭೌತಿಕವಾದ ಆಕಾರದ ವಿಷಯವನ್ನು ಹೇಳುತ್ತಾರೆ-

\begin{shloka}
ಆಜಾನುಬಾಹುಃ ಸುಶಿರಾಃ ಸುಲಲಾಟಃ ಸುವಿಕ್ರಮಃ|\label{247}\\
ಸಮಃ ಸಮವಿಭಕ್ತಾಂಗಃ ಸ್ನಿಗ್ಧವರ್ಣಃ ಪ್ರತಾಪವಾನ್||\\
ಪೀನವಕ್ಷಾಃ ವಿಶಾಲಕ್ಷಃ ಲಕ್ಷ್ಮೀವಾನ್ ಶುಭ ಲಕ್ಷಣಃ||
\end{shloka}

ಎಂದು ತಿರುಮೇನಿಯ (ಸಂಪದ್ಯುಕ್ತ ಶರೀರದ) ವರ್ಣನೆ. `ಅವರುಡೈಯುಳ್ಳಂ ಎನ್ನುಂಡೋ' ಅದಾಯಿತು ಅವರ ಹೊರಗಿನ ರೂಪವೂ ಅದಕ್ಕೆ ಅನುಗುಣವಾಗಿದೆ.

\begin{shloka}
ಸರ್ವದಾಭಿಗತಃ ಸದ್ಭಿಃ ಸಮುದ್ರ ಇವ ಸಿಂಧುಭಿಃ||\label{247e}
\end{shloka}

ಎಂದು ನಾರದರಿಂದ ಹೇಳಲ್ಪಟ್ಟ ಗುಣಶಾಲಿ ರಾಮನೊಬ್ಬ. ಆತ ನರಾವತಾರ ಪಡೆದ ಜನ್ಮತಾಳಿದ್ದರೂ ದೇವನೇ. 

\begin{shloka}
ದೇವಾ ಮಾನುಷ ರೂಪೇಣ ಚರಂತ್ಯತ್ರ ಮಹೀತಲೇ|\label{247a}
\end{shloka}

ದೇವರ ಭಾವವನ್ನು - ಧರ್ಮವನ್ನು ಹೊತ್ತು ಬಂದಿದ್ದಾನೆ.

\begin{shloka}
ರಾಮೋ ವಿಗ್ರಹವಾನ್ ಧರ್ಮಃ|\label{247d}
\end{shloka}

ಎಂದು ಹೇಳಿರುವಂತೆ ಮೂರರ್ತಿವೆತ್ತ ಧರ್ಮವಾಗಿದ್ದಾನೆ.

ಆತ್ಮನೇ ಮೂರ್ತಿವೆತ್ತು ಬಂದಾಗ ಆ ಪ್ರಕೃತಿಯಲ್ಲೂ ಕೋಪವು ಬರುವಂತೆಯೇ ಅವನು ಪ್ರಕೃತಿಯಲ್ಲಿ ಪ್ರಕಾಶಿಸುವಾಗ ತನ್ನಂತೆಯೇ ಪ್ರಕಾಶಿಸುತ್ತಾನೆ. ನರನೇ ಆದರೂ ದೇವಗುಣಗಳು ಕಂಡುಬರುತ್ತವೆ. ಲೋಕಕಂಟಕರನ್ನು ನಿವಾರಿಸಿ ಲೋಕರಕ್ಷಣೆ ಮಾಡನಬೇಕೆಂಬ ಬ್ರಹ್ಮಾದಿಗಳ ಪ್ರಾರ್ಥನೆಯನ್ನನುಸರಿಸಿ ಇವನ ಜನನ. ಪಾಯಸ ಹಂಚುವಾಗ ವಿಷ್ಣುವಿನ ಅರ್ಧಾಂಶದಿಂದ ಇವನು ಹುಟ್ಟಿದವನು ಎಂದು ಹೇಳಿದೆ. ಹಿಂದೆ ಒಂದು ಸಂಕಲ್ಪವು ಪೂರ್ವಭಾವಿಯಾಗಿತ್ತು. ಅವ್ಯಕ್ತವಾಗಿದ್ದ ಸತ್ಯಸಂಕಲ್ಪವು ವ್ಯಕ್ತವಾದಾಗ, ಆ ಸಂದರ್ಭದಲ್ಲಿ ಬರೆದ ಕಾವ್ಯವಿದು. ರಾಮಾಯಣ ಗ್ರಂಥದ ಆಧಾರದ ಮೇಲೆಯೇ ಹೇಳುತ್ತಿದ್ದೇನೆ, ಫಸ್ಟ್ ಹ್ಯಾಂಡ್ ವಿಷಯ. ನಿಗಮಾಂತ ದೇಶಿಕರು ಹೀಗೆ ಹೇಳಿದ್ದಾರೆ, ಶತದೂಷಣೀಯಲ್ಲಿ ಹೀಗೆ ಹೇಳಿದೆ, ಎಂದಲ್ಲ. ಎಂದರೆ ಅವು ಪರಿಹರಣೀಯ ಎಂದರ್ಥವಲ್ಲ. ಮೊದಲನೇ ವಿಷಯವನ್ನು ತೆಗೆದುಕೊಂಡು ಅದಕ್ಕನುಗುಣವಾದ್ದರೆ ಅವುಗಳನ್ನೂ ತೆಗೆದುಕೊಳ್ಳೋಣ. ಯಾರು ನರನಂತೆಯೇ ಇದ್ದು ತನ್ನ ಆತ್ಮಗುಣಗಳನ್ನು ಬಿಡದೆ, ತನ್ನ ಲೋಕದಲ್ಲಿದ್ದರೂ ಲೋಕದ ನಡೆಯನ್ನು ತೋರಿಸಿದನೋ, ಅವರಾರು? ಸುಗ್ರೀವನಿಗೆ ಪಟ್ಟಾಭಿಷೇಕ ಮಾಡುವಾಗಲೂ

`ಸಲಿಲೇನ ಸಹಸ್ರಾಕ್ಷಂ ವಸವೋ ವಾಸವಂ ಯಥಾ' ಎಂದು ಬರುತ್ತೆ. ಇಲ್ಲಿ ಶ್ರೀರಾಮನ ಪಟ್ಟಾಭಿಷೇಕದಲ್ಲೂ-

\begin{shloka}
ಅಭ್ಯಷಿಂಚನ್ ನರವ್ಯಾಘ್ರಂ ಪ್ರಸನ್ನೇನ ಸುಗಂಧಿನಾ||\label{248a}\\
ಸಲಿಲೇನ ಸಹಸ್ರಾಕ್ಷಂ ವಸವೋ ವಾಸವಂ ಯಥಾ||\label{248b}
\end{shloka}

ಬರುತ್ತದೆ.

\section*{ಪುರುಷಸೂಕ್ತದ ಬಳಕೆಯ ಕುರಿತು}

ಹೀಗೆ ಅಂಜಲಿ ಸಮರ್ಪಿಸುವುದಕ್ಕೆ ಪೂರ್ವಭಾವಿಯಾಗಿ ಪುರುಷಸೂಕ್ತವನ್ನು ಹೇಳುವ ಸಂಪ್ರದಾಯವಿದೆ. ಅದಿಲ್ಲದೆಯೇ ಹೇಳುವ ಸಂಪ್ರದಾಯವೂ ಇದೆ. ಎರಡರಲ್ಲಿ ಯಾವುದರಲ್ಲಿ ಔಚಿತ್ಯವಿದೆ ನೋಡೋಣ. ಶ್ರೀರಾಮನನ್ನು ಮನುಷ್ಯನೆಂದೇ ಉದ್ದಕ್ಕೂ ಹೇಳಿದೆ. ಮಾತಲಿ ಸಾರಥ್ಯವಹಿಸುವಾಗ ದೇವತೆಗಳಿಂದ ಸ್ತುತಿ ಎಂಬ ಜಾಗವನ್ನು ಬಿಟ್ಟರೆ ಉಳಿದೆಡೆಯಲ್ಲೆಲ್ಲಾ ರಾಮನಿಗೆ ಮನುಷ್ಯಭಾವವನ್ನೇ ಹೇಳಿದೆ. (ಕೃಷ್ಣಾವತಾರದಲ್ಲಾದರೋ ಹೆಜ್ಜೆ ಹೆಜ್ಜೆಗೂ ದೇವತ್ವವ್ಯಾಪಾರಗಳನ್ನು ಹೇಳಿದೆ.) ಇಲ್ಲಿಮನುಷ್ಯದೇಹದಿಂದಲೇ ಕೂಡಿದ್ದರೂ ಆತ್ಮಗುಣವನ್ನು ಬಿಡದೆ ನಡೆದ ಪರಮಪುರುಷನ ವಿಷಯವಾದ್ದರಿಂದ ಪುರುಷ ಸೂಕ್ತವನ್ನು ಹೇಳುವುದರಲ್ಲಿ ಔಚಿತ್ಯವಿದೆ!

\begin{shloka}
ತತಃ ಪಶ್ಯತಿ ಧರ್ಮಾತ್ಮಾ ತತ್ಸರ್ವಂ ಯೋಗಮಾಸ್ಥಿತಃ\label{248}
\end{shloka}

ಎಂಬ ಮಾತನ್ನು ಗಮನಿಸಿ. ಇಲ್ಲಿ ಯೋಗವೆಂದರೆ ಭೂತ-ಭವಿಷ್ಯದ್-ವರ್ತಮಾನಗಳನ್ನು ತಿಳಿದು ಹೇಳಿದರೆಂದೇ? ಅಥವಾ ಆತ್ಮನನ್ನು ನೋಡಿ, ಆತ್ಮನನ್ನು ಭೂಮಿಗೆ ಅವತಾರ ಮಾಡಿಸಿ ಆತ್ಮಗುಣದಿಂದಲೇ ಪ್ರಕಾಶಿಸುತ್ತಿರುವ ಕಾವ್ಯವೊಂದನ್ನು ತರಲು ಬೇಕಾದ ಧರ್ಮದಿಂದ ಕೂಡಿಕೊಂಡರೆಂದೇ? ಎಂದರೆ, ಇದು ಎರಡನೇ ತರಹದ್ದಾಗಿದೆ. ತತ್ತ್ವಜ್ಞರ ಕಡೆಯಿಂದ ಬಂದ ಕಥೆ ಇನ್ನು ಪುರುಷಸೂಕ್ತದ ನವಿನಿಯೋಗವನ್ನು ಗಮನಿಸಿದರೆ ಆತ್ಮನು ದೇಹವನ್ನು ಬಿಟ್ಟು ಹೋದಮೇಲೂ ಪುರುಷಸೂಕ್ತದಿಂದ ಔರ್ಧ್ವದೈಹಿಕ ಕರ್ಮದಲ್ಲಿ ಅದಕ್ಕೆ ಸ್ನಾನಮಾಡಿಸುತ್ತಾರೆ ಪುರುಷನಿದ್ದ ದೇಹ ಎಂದು ``ಅದೈಪ್ಪತ್ತಿ" ಎಂದು ಪಾರಮ್ಯ ಹೇಳಿ ಕೊಂಡಾಟ ನಡೆಸುತ್ತಾರೆ. 

\section*{ಪುರುಷಸೂಕ್ತದ ವಿಷಯವೇನು?}

ಪುರುಷಸೂಕ್ತದ ವಿಷಯವೇನು? ಲೋಕಸೃಷ್ಟಿಕರ್ತನಾಗಿ ಪುರುಷರೂಪವಾಗಿಯೇ ಯಾವನು ಇರುವನೋ ಅವನ ವಿಷಯ. 

\begin{shloka}
ಸಹಸ್ರಶೀರ್ಷಾ ಪುರುಷಃ| ಸಹಸ್ರಾಕ್ಷಸ್ಸಹಸ್ರಪಾತ್|\label{249d}\\
ಸ ಭೂಮಿಂ ವಿಶ್ವತೋ ವೃತ್ವಾ| ಅತ್ಯತಿಷ್ಠದ್ದಶಾಂಗುಲಮ್||
\end{shloka}

ಈ ಹತ್ತು ಅಂಗುಲ ಯಾವ ಮನೆಯ ಗೋಡೆಯನ್ನು ಬೆರಳಿನಲ್ಲಿ ಲೆಕ್ಕಹಾಕಿಕೊಂಡರು? ಎಂದರೆ ಹೃದಯವಾಸಿ ಪುರುಷನ ಲೆಕ್ಕ. `ಏತಾವಾನಸ್ಯ ಮಾಹಿಮಾ'\label{249f} ಎಂದು ವರ್ಣಿತ ಮಹಿನಾದ ಇವನಿಗೆ ವಿಷಯ ವಾವುದಪ್ಪಾ ಎಂದರೆ `ಅತ್ಯತಿಷ್ಠದ್ದಶಾಂಗುಲಮ್' ಅದು ಹೃದಯ-

\begin{shloka}
ಹೃದಯಂ ತದ್ವಿಜಾನೀಯಾದ್ವಿಶ್ವಸ್ಯಾಯತನಂ ಮಹತ್|\label{249e}\\
ಸ ಬ್ರಹ್ಮ ಸ ಶಿವಃ ಸ ಹರಿಃ ಸೇಂದ್ರಃ ಸೋಽಕ್ಷರಃ ಪರಮಃ ಸ್ವರಾಟ್||\label{249c}
\end{shloka}

ಆ ಪುರುಷನನ್ನು ಎಲ್ಲೆಲೋ, ಇಂಗ್ಲೆಂಡಿನಲ್ಲೋ, ಪಾತಾಳಗರಡಿ ಹಾಕಿ ಬಾವಿಯಲ್ಲೋ ಹುಡುಕಿ ಹೇಳಲಿಲ್ಲ. ಅವನನ್ನು ಹೃದಯದಲ್ಲಿ ಕಂಡರು. ಹೃದಯದಲ್ಲಿದ್ದೆ ಆಪಾದಮಸ್ತಕವಾಗಿ ಶಾಖಕೊಡುತ್ತಿದ್ದಾನೆ ಇದೆಲ್ಲ ಲಕ್ಷಣಗಳು ಪುರುಷನಲ್ಲೇ ಸಿಕ್ಕುವುದು. ಗೋಡೆಯಲ್ಲಿ ಸಿಕ್ಕುವುದಿಲ್ಲ. 

ಸ ಬ್ರಹ್ಮ ಸ ಶಿವಃ ಸ ಹರಿಃ ಸೇಂದ್ರಃ ಸೋಕ್ಷರಃ ಪರಮಃಸ್ವರಾಟ್, ಎಂಬ ಪುರುಷನಾವನುಂಟೋ ಅವನು ಈ ಪುರಿಯಲ್ಲಿ ವಾಸಮಾಡುತ್ತಿದ್ದಾನೆ.

\begin{shloka}
`ದ್ಯುತಿಮಾ ಧೃತಿಮಾನ್ ವಶೀ; ಎಂಬಂತೆ \label{249}
\end{shloka}

ಮೊದಲೇ ನಿತ್ಯನಾಗಿ ಸರ್ವಗತನಾಗಿ ವೇದವೇದ್ಯನಾದ ಪರಮಾತ್ಮತತ್ತ್ವವಾವುದುಂಟೋ ಅದೇ ಇದು. ಈ ವಿಷಯವನ್ನು ಶ್ರೀವೈಷ್ಣವಸಿದ್ಧಾಂತ, ಶಾಂಕರಸಿದ್ಧಾಂತ, ಮಧ್ವಸಿದ್ಧಾಂತದ ಸರಣಿಗಳಿಂದ ಹೇಳುತ್ತಿಲ್ಲ. ಇಲ್ಲಿ ದೇಹವಿದೆ, ಇದರಲ್ಲಿ ಲೈಫ್ ಇದೆ. 

\section*{ರಾಮಾಯಣದ ಆವಿರ್ಭಾವ}

\begin{shloka}
ಅದನ್ನು ನೋಡಲು - ``ಯೋಗಮಾಸ್ಥಿತಃ"\label{249b} ಬಂದಿದೆ.
\end{shloka}

ಅದರಲ್ಲು ಒಂದು ಘಟನೆ ಇದೆ. ಅದು ಪುರುಷಪಕ್ಷಿಯನ್ನು ಕೊಂದ ಘಟನೆ. ಆ ಒಂದು ಘಟನೆ ನಡೆದು ರಸ ಉಕ್ಕಿ ಬರುತ್ತದೆ. ಹಾಗೆ ರಸ ಉಕ್ಕಿದಾಗ ಒಂದು ಶ್ಲೋಕ ರೂಪತಾಳುತ್ತದೆ. ಅದನ್ನು ಬ್ರಹ್ಮನು -

\begin{shloka}
ಮಚ್ಛಂದಾದೇವ ತೇ ಬ್ರಹ್ಮನ್  ಪ್ರವೃತ್ತೇಯಂ ಸರಸ್ವತೀ||\label{250c}
\end{shloka}

ಎಂದುಹೇಳಿ, ನಾರದರು ಹೇಳಿದ್ದನ್ನನುಸರಿಸಿ ಇದೇ ರಸದಿಂದ ಕವಿತೆಯನ್ನು ಮಾಡಲು ಆಜ್ಞಾಪಿಸುತ್ತಾನೆ. ವಿಷಾದದಿಂದ ಬರುವ ರಸವೂ ಇದೆ. ಇಲ್ಲಿ ಶ್ಲೋಕದ ರಸ ಕರುಣೆಯಿಂದ ಉಂಟಾಯಿತು. ಕರುಣೆ ಏಕೆ ಹುಟ್ಟಿತು? 

\begin{shloka}
`ಯಸ್ತಾದೃಶಂ ಚಾರುರವಂ ಕ್ರೌಂಚಂ ಹನ್ಯಾದಕಾರಣಾತ್|'\label{250d}
\end{shloka}

ಎಂಬುದು ಕಾರಣ, ಅಲ್ಲೊಂದು ಚಾರುರವ ಅಲ್ಲದೆ, 

\begin{shloka}
`ಭ್ರಾತರೌ ಸ್ವರಸಂಪನ್ನೌ' `ಸ್ವಂಚಿತಾಯತನಿಃಸ್ವನಮ್'||\label{250b}
\end{shloka}

ಎಲ್ಲೆಡೆಯಲ್ಲೂ ಅದೇ `ಚಾರುನಿಃಸ್ವನಮ್'\label{250} `ತಂತ್ರೀಲಯ ಸಮನ್ವಿತಮ್'.

\section*{ವೇದಪುರುಷನೇ ರಾಮನಾಗಿದ್ದಾನೆ}

ರಾಮಾಯಣವು ವೇದ ಪುರುಷನನ್ನೇ ವಿಷಯವಾಗಿ ಹೊಂದಿರುವ ಕಾವ್ಯ. ಅದನ್ನೇ ಹೊರಗಡೆ ಕಥಾರೂಪದಲ್ಲಿ ಹೇಳುವುದು. ವೇದದಲ್ಲೂ ಮುಖ್ಯ ವಿಷಯ ಪುರುಷ ಅಂದರೆ ಶುದ್ಧಾತ್ಮನ ಸ್ಥಿತಿ. ವೇದಪುರುಷನನ್ನು ಕಾವ್ಯದಲ್ಲಿ ತಂದ ವಾಲ್ಮೀಕಿಯ ವಿಷಯವಾದರೂ ಏನು? ಅವರು ಕುಳಿತಾಗಲೂ ಪ್ರಾಚೇತಸರಾಗಿಯೇ ಕುಳಿತಿದ್ದಾರೆ; ಬಹಿಶ್ಚೇತಸರಲ್ಲ. ಒಳಗಿನ ವಿಶಿಷ್ಟವಾದ ಪ್ರಕೃಷ್ಟವಾದ ತೇಜಸ್ಸಿನಿಂದ ಕೂಡಿದ್ದಾರೆ. ಹೀಗೆ-

\begin{shloka}
ಧರ್ಮಾತ್ಮಾ ಸತ್ಯಸಂಧಶ್ಚ ರಾಮೋ ದಾಶರಥಿರ್ಯದಿ|\label{250a}\\
ಪೌರುಷೇ ಚಾಪ್ರತಿದ್ವಂದ್ವಃ ಶರೈನಂ ಜಹಿ ರಾವಣಿಮ್||
\end{shloka}

ಎಂದು ಹೇಳಿರುವಂತಹ ಗುಣವುಳ್ಳ ರಾಮನ ಕಥೆಯನ್ನು ಬರೆಯುತ್ತಾರೆ. ಧರ್ಮಾತ್ಮಾ, ಸತ್ಯಸಂಧಶ್ಚ ಎರಡೂ ಆತನಿಗೆ ಸಲ್ಲುವ ವಿಷಯ. ಸತ್, ಸತ್ಯಸ್ವರೂಪ, ಸತ್ಯ ಎಂದು ಕರೆಯಲ್ಪಟ್ಟಿರುವುದು ಆತ್ಮವೇ. ಇನ್ನು `ಧರ್ಮಾತ್ಮಾ'. ಸತ್ಯಕ್ಕೆ ಕವಚವು ಧರ್ಮ.  ಆ ಸತ್ಯ ಧರ್ಮಗಳೆರಡೂ ಮೂರ್ತಿವೆತ್ತಿದರೆ ರಾಮ. ಹಾಗೆ ಬಂದವನು ರಾಮ. ಅವನು ಬರುವುದಕ್ಕೆ ಯೋನಿ ಅಥವಾ ದ್ವಾರವು ದಶರಥ. ಅದಕ್ಕೇ `ದಾಶರಥಿರ್ಯದಿ' ಎಂದು ಪದವಿನ್ಯಾಸ. 

ನರನ ದೃಷ್ಟಿಯಿಂದ `ನೀರಿಮಾನ್' `ಶತ್ರುನಿಬರ್ಹಣಃ'\label{250f} ಹೊರಗೆ ಇದೆಲ್ಲ ಕಂಡರೂ ಒಳಗೆ ನೋಟಕ್ಕೆ ಹೇಗೆ ಕಾಣುತ್ತಾನೆ ಎಂಬುದನ್ನು ಕಂಡುಕೊಂಡು ಪುರುಷಸೂಕ್ತವನ್ನು ಹೇಳಿ ನಂತರ, 

\begin{shloka}
`ವಸವೋ ವಾಸವಂ ಯಥಾ'\label{250e} ಎಂದು ಅವನನ್ನು ಸ್ತುತಿಸುವುದಾಗಿದೆ.
\end{shloka}

ರಾಮನು ಇಂದ್ರಸಮಾನನಾಗಿದ್ದಾನೆ; ರಾಜೇಂದ್ರನಾಗಿರಯೂ ಕುಳಿತಿದ್ದಾನೆ; ವಿಷ್ಣುವಿನ ಅರ್ಧಾಂಶವಾಗಿಯೂ ಇದ್ದಾನೆ; ಎಂಬುದನ್ನು ಕಾಣುತ್ತದೆ ವಾಲ್ಮೀಕಿದೃಷ್ಟಿ. ತಾತ್ತ್ವಿಕ ದೃಷ್ಟಿಗೂ ಆತನು ಪುರುಷನೇ. ಸಾಮಾನ್ಯ ಪುರುಷನಲ್ಲ; ಅವನು ಉತ್ತಮ ಪುರುಷ. 

`ತಸ್ಯ ಧೀರಾಃ ಪರಿಜಾನಂತಿ ಯೋನಿಮ್|' -ಧೀರರು ಅವನನ್ನು ಅರಿಯುತ್ತಾರೆ; ಅವನು ``ತಮಸಸ್ತು ಪಾರೇ" ``ತಮಸಃ ಪರಸ್ತಾತ್"\label{251} ಆಗಿದ್ದಾನೆ. ಇತ್ಯಾದಿ ವಾಕ್ಯಗಳೂ ಇವೆ. ಇಹಲೋಕದ ದೃಷ್ಟಿಯಿಂದ ಮಾತ್ರವಲ್ಲ; ಪಾರಮಾರ್ಥಿಕ ದೃಷ್ಟಿಯಿಂದಲೂ ಉತ್ತಮ. ಹೀಗೆ ಸತ್ಯಪ್ರಧಾನವಾಗಿ, ತತ್ತ್ವಪ್ರಧಾನವಾಗಿ, ಧರ್ಮಪ್ರಧಾನವಾಗಿ ಹರಿದುಬಂದಿರುವುದರಿಂದ ಆ ರಾಮಾಯಣ ಸಾಮಾನ್ಯ ಗ್ರಂಥರಾಶಿಗೆ ಸೇರಿಸಲ್ಪಡ ತಕ್ಕದ್ದಲ್ಲ. ಹಾಲಿಗೆ ಕಸ್ತೂರಿ ಹಾಕಿದ್ದರೆ, ಅಲ್ಲಿ ಹಾಲಿದ್ದರೂ ಕಸ್ತೂರಿಯು ಪ್ರಧಾನವಾಗಿ ಕಾಣುತ್ತೆ. ಇಂಗು ಹಾಕಿದ್ದರೆ, ಇಂಗು ಪ್ರಧಾನವಾಗಿ ಕಾಣುತ್ತೆ. ರಾಮಾಯಣವು ಧರ್ಮ ಹಾಗೂ ಆತ್ಮಗುನ ಪ್ರಾಧಾನ್ಯ ಪೂರ್ವಕವಾದ ವ್ಯವಹಾರವನ್ನು ಹೇಳುತ್ತದೆ. ಈ ಬಾಹ್ಯವಾದ ಸ್ವರಾಜ್ಯದಲ್ಲಿ ಅಭಿಷೇಕ ಮಾಡುವಾಗ ಅವನೊಬ್ಬ ಆದರ್ಶನಾದ ಪುರುಷ, ಆತ್ಮರಾಜ್ಯದಲ್ಲಿ ಅಭಿಷೇಕ ಮಾಡುವಾಗ ಉತ್ತಮ ಪುರುಷ. ಆತನಿಗೆ ಕೇವಲ ಭೂಮಿಯಲ್ಲಿ ಮಾತ್ರ ನಿಂರತು ಅಂಜಲಿ ಸಮರ್ಪಿಸುತ್ತಿಲ್ಲ. ದೇವತೆಗಳಿಂದಲೂ ಪುಷ್ಪವೃಷ್ಟಿ .

\begin{shloka}
ಶ್ರುತ್ವಾ ಚೈತತ್ ತ್ರಿಲೋಕಜ್ಞಃ\label{251b}
\end{shloka}
ಎಂದು ಹೇಳಲ್ಪಟ್ಟಂತೆ ತ್ರಿಲೋಕದರ್ಶಿಯಾದ ನಾರದನೂ ``ಬುದ್ಧ್ವಾ" ಎಂದರೆ ತಿಳಿದು ಹೇಳಬೇಕಾದವನ ವಿಷಯ ಹೇಳಲು ತ್ರಿಲೊಕಗಳಲ್ಲೂ ಪ್ರವೇಶಿಸಬೇಕು. ನರನ ಸನ್ನಿಧಿಗೆ ಮಾತ್ರವಲ್ಲ, ಪರಮ ವ್ಯೋಮಭಾಸ್ಕರನಾದ ನಾರಾಯಣನ ಸನ್ನಿಧಿಗೆ ಹೋಗಿ ತಲುಪುವವರೆಗೂ ವ್ಯಾಪ್ತಿ ಇದೆ. ಸಂಪ್ರದಾಯವಿರಲಿ; ಇಲ್ಲದಿರಲಿ; ಅದರ ವಿಷಯ ಹಾಗೆ ಹೋಗುತ್ತೆ. ರಾಮನು ಉತ್ತಮ ಪುರುಷ ಎಂದು ಗುಟ್ಟು ಸಿಗುವುದನ್ನು ಋಷಿಗಳು ತಿಳಿದುಕೊಂಡುಬಿಟ್ಟರು.

\begin{shloka}
ಅಹಂ ವೇದ್ಮಿ ಮಹಾತ್ಮಾನಂ ರಾಮಂ ಸತ್ಯಪರಾಕ್ರಮಂ|\label{251a}\\
ವಸಿಷ್ಠೋಽಪಿ ಮಹಾತೇಜಾಃ ಯೇ ಚೇಮೇ ತಪಸಿ ಸ್ಥಿತಾಃ||
\end{shloka}

ಗುಟ್ಟು ಬಿಟ್ಟುಕೊಡುವಂತಿಲ್ಲ. ದೇವರಹಸ್ಯ, ಹೊರಗೆ ಬಿಡುವಂತಿಲ್ಲ. ಆ ರಾಮನನ್ನು ತಿಳಿದ ಜ್ಞಾನಿಯೂ ಹಾಗೆಯೇ ಎಲ್ಲರೊಡನೆ ಬೆರೆತಿರಬೇಕು. ಮನುಷ್ಯನಂತೆಯೇ ಮಾತನಾಡಬೇಕು, ಅಲ್ಲೇನೋ ಒಂದು ಧ್ವನಿ ಇದೆ, ಅದನ್ನು ಪತ್ತೆಹಚ್ಚಿಕೊಳ್ಳಬೇಕು. ಅದೇ ಉಸಿರು ಕಾವ್ಯಾಕ್ಕೆ. 

\section*{ಸತ್ಯಪರಾಕ್ರಮ ಶ್ರೀರಾಮ}

\begin{shloka}
ಅಹಂ ವೇದ್ಮಿ ಮಹಾತ್ಮಾನಂ ರಾಮಂ ಸತ್ಯಪರಾಕ್ರಮಂ|\\
ವಸಿಷ್ಠೋಽಪಿ ಮಹಾತೇಜಾಃ ಯೇ ಚೇಮೇ ತಪಸಿ ಸ್ಥಿತಾಃ||
\end{shloka}

ಸತ್ಯವೇ ಸತ್ಯಪರಾಕ್ರಮವಾಗಿ ಬಂದಿದೆ. ಅದಕ್ಕನುಗುಣವಾಗಿ ತನ್ನ ಪರಾಕ್ರಮವನ್ನು ತೋರಿಸುವ ರಾಮನನ್ನು ಕೇಳಬೇಕಾದರೆ, ಸತ್ಯದೂರರಾದವರು ಅದರ ಜಾಡನ್ನು ಹಿಡಿಯಲಾರರು. ಸತ್ಯದೊಡನೆ ಅದನ್ನು ಹಿಂಬಾಲಿಸುವವರೇ ಹಿಡಿಯ ಬಲ್ಲರು. ನಿನ್ನ ಪುರೋಹಿತ ಗುರು ವಸುಷ್ಠರಿಗೆ ಗೊತ್ತು. ಜ್ಞಾನವೃದ್ಧರೂ, ತಪೋವೃದ್ಧರೂ ಆಗಿ ಇಂದ್ರಿಯಗಳನ್ನು ಸ್ವಾಧೀನದಲ್ಲಿಟ್ಟು ಕೊಂಡಿರುವ ಆ ವಸಿಷ್ಠರನ್ನು ಕೇಳಿ ನೋಡು ಎಂದಾಗ, ಕಳ್ಳರಿರುವಾಗ ಅವರೆದುರಿಗೆ ದುಡ್ಡಿನ ಸಮಾಚಾರವನ್ನು ನೇರವಾಗಿ ಹೇಳದೆ ಅದು ಜೋಪಾನವಾಗಿದೆ ತಾನೇ? ಎಂದು ಗುಟ್ಟುಬಿಟ್ಟುಕೊಡದೆ ಅಂತರಂಗದಲ್ಲಿ ಮಾತನಾಡಿಕೊಳ್ಳುವಂತೆ ಇದೆ. ಅದು. 

ಪರಮಪುರುಷನನ್ನು ಭೂಮಿಗೆ ಇಳಿಸಿದರೆ ಅಂದು ಭೂಮಿ ಸ್ವರ್ಗವಾಗುತ್ತದೆ. 

\begin{shloka}
ಅನ್ವಗಾದಿವ ಹಿ ಸ್ವರ್ಗಃ ಗಾಂಗತಂ ಪುರುಷೋತ್ತಮಮ್||
\end{shloka}

ಅವನನ್ನು ಮತ್ತೆ ವೈಕುಂಠಕ್ಕೆ ಕೊಂಡೊಯ್ಯುವವರೆಗೂ ಕಾವ್ಯ ಬೆಳೆದಿದೆ. ದಿವಿಯ ಆ ಮಹಾಶಕ್ತಿಯನ್ನು ಭುವಿಗೆ ಇಳಿಸಿ, ನಾಲ್ಕು ಭಾಗಮಾಡಿ ಲೋಕಕಾರ್ಯಾನಂತರ ಮತ್ತೆ ಅವನ ಲೋಕಕ್ಕೆ ಸೇರಿಸಿದ್ದಾರೆ. ಈ ಹಿನ್ನೆಲೆ ಸಿಕ್ಕಿದರೆ ಯಾವುದಕ್ಕೂ ಜಾಡು ಸಿಕ್ಕಿದಂತೆ ಗುಪ್ತವಾದ ಮಾರ್ಗವಿದೆ ಇದರಲ್ಲಿ. 

\begin{shloka}
ಕೌಸಲ್ಯಾಽಜನಯದ್ರಾಮ‌ ಸರ್ವಲಕ್ಷಣಸಂಯುತಮ್|\label{252}\\
ಪ್ರೋದ್ಯಮಾನೇ ಜಗನ್ನಾಥಂ ಸರ್ವಲೋಕನಮಸ್ಕೃತಮ್|
\end{shloka}

ಸರ್ವಲೋಕ ನಮಸ್ಕೃತನಾಗಿ ಜಗತ್ತಿಗೇ ನಾಥನಾಗಿ ಸರ್ವಲಕ್ಷಣ ಸಂಪನ್ನನೂ ಆಗಿರುವ ರಾಮ ಬಹಳ ಗಂಭೀರವಾಗಿದ್ದಾನೆ. ಅವನು ಒಳಗಿದ್ದು ಎಷ್ಟು ಹೊರಹೊಮ್ಮಿ ಬರುತ್ತಾನೋ ಅಷ್ಟೇ. 

\section*{ರಾಮರಾಜ್ಯ}

\begin{shloka}
`ಚತ್ವಾರಿ ವಾಕ್ಪರಿಮಿತಾ ಪದಾನಿ ತಾನಿ ವಿದುರ್ಬಾಹ್ಮಣಾ ಯೇ ಮನೀಷಿಣಃ|\\
ಗುಹಾ ತ್ರೀಣಿ ನಿಹಿತಾ ನೇಂಗಯಂತಿ ತುರೀಯಂ ವಾಚೋ ಮನುಷ್ಯಾ ವದಂತಿ||\\
ತ್ರಿಪಾದೂರ್ಧ್ವ ಉದೈತ್ಪುರುಷಃ'\label{252a}
\end{shloka}

ಎಂಬಂತೆ ಆ ತ್ರಿಪಾತ್ ಗುಟ್ಟಾಗಿರುತ್ತದೆ. ನಮ್ಮಲ್ಲಿರುವ - ಭುವಿಯಲ್ಲಿರುವ ವಾಲ್ಮೀಕಿಯು ಆತ್ಮವಾನ್, ವೇದ ವೇದಾಂಗ ತತ್ತ್ವಜ್ಞನಾಗಿ, ತ್ರಿಪಾದೂರ್ಧ್ವದಲ್ಲಿರುವ ಪುರುಷನನ್ನು ಕಂಡುಕೊಂಡು, ಅವನನ್ನು ವರ್ಣಿಸಿದ್ದರಿಂದಲೇ ರಾಮಾಯಣವು ವೇದ. ನರನ ಕಾವ್ಯ ನಾರಾಯಣದ ಕಾವ್ಯ. ನಮ್ಮ ಜೀವನವನ್ನು ಹೇಳುತ್ತದೆ. ಆತ್ಮ ಜೀವನವನ್ನೂ ಹೇಳುತ್ತದೆ. ಮಧ್ಯೆ ಎಚ್ಚರಿಕೆ ಕೊಡುತ್ತದೆ. 

`ಜಾನಕಿಯೇ, ನಿನ್ನನ್ನು ಬಿಟ್ಟರೂ, ಲಕ್ಷ್ಮಣನನ್ನು ಬಿಟ್ಟರೂ, ಋಷಿಗಳಲ್ಲಿ ಮಾಡಿರುವ ಸತ್ಯ ಪ್ರತಿಜ್ಞೆಯನ್ನು  ಬಡುವುದಿಲ್ಲ' ಎನ್ನುತ್ತಾನೆ ರಾಮನು. ಆ ಸತ್ಯವು ಅಂಗಡಿಯಲ್ಲಿ ಸಿಕ್ಕುವ ಪದಾರ್ಥವಲ್ಲ. ಅಸತ್ಯದಲ್ಲಿ ಸತ್ಯವು ಸಿಕ್ಕುವುದಿಲ್ಲ. ಸತ್ಯದಲ್ಲಿಯೇ ಅದು ಸಿಕ್ಕುವುದು. ಪುರುಷನೂ ಅವನೇ. ಆದರಿಂದ ಅವನ ಬಗ್ಗೆ ಇರುವ ಸೂಕ್ತವನ್ನು ಹೇಳಿ ಪುಷ್ಪಾಂಜಲಿ. ನಮ್ಮ ಪರಮವ್ಯೋಮದಲ್ಲಿ ಬೆಳಗುವಂತೆ ಭೂಮಿಯಲ್ಲೂ ನಮ್ಮನ್ನು ರಾಮರೂಪದಿಂದ ಕಾಪಾಡಲಿ. ವ್ಯಾಸರು ಭಾರತದಲ್ಲಿ ಹೇಳಿರುವಂತೆ.

\begin{shloka}
ಪ್ರಾಣಾಪಾನೌ ಸಮಾವಾಸ್ತಾಂ ರಾಮೇ ರಾಜ್ಯಂ ಪ್ರಶಾಸತಿ|\label{253}
\end{shloka}

ಆತ್ಮನು ತನ್ನ ಹತೋಟಿಯನ್ನು ಮೀರದಂತೆ ತನ್ನ ಸ್ಥಾನದಲ್ಲಿರುವಂತೆ ಮಾಡಿಕಟ್ಟಿದ ರಾಜ್ಯ ರಾಮರಾಜ್ಯ. ಎಲ್ಲರೂ ಆತ್ಮಾರಾಮರಾಗಿರುವೆಡೆಯಲ್ಲಿಯೇ ಪ್ರಾಣಾಪಾನಗಳ ಸಾಮರಸ್ಯ. 

\section*{ರಾಮಾಯಣ ಮತ್ತು ಪುರುಷಸೂಕ್ತಗಳ ಸಂಬಂಧ}

ಪರಮ ಪುರುಷನನ್ನು ಕಂಡು ಅವನನ್ನು ಒಂದು ಚರಿತ್ರೆಯ ರೂಪದಲ್ಲಿ, ಲೋಕದಲ್ಲಿ ಕಾವ್ಯ ರೂಪದಲ್ಲಿ ತಂದಿದ್ದಾರೆ. ಸ್ಥೂಲದೃಷ್ಟಿಗೆ ರಾಮನು ನರ, ಸೂಕ್ಷ್ಮದೃಷ್ಟಿಗೆ ದೇವತೆ,ಪರದೃಷಿಗೆ ಸತ್ಯರೂಪವೇ ಆಗಿದ್ದಾನೆ. ಅಂತಹವನಿಗೆ ವೇದದ ಪುರುಷಸೂಕ್ತ. ಭೂಲೋಕದಲ್ಲಿದ್ದು ಕೊನೆಗೆ ಬ್ರಹ್ಮಲೋಕದವರೆಗೂ ವ್ಯಾಪಿಸುವ ರಾಮನ ಕಥೆ ರಾಮಾಯಣ . 

\begin{shloka}
ರಾಮೋ ರಾಜ್ಯಮುಪಾಸಿತ್ವಾ ಬ್ರಹ್ಮಲೋಕಂ ಪ್ರಯಾಸ್ಯತಿ|\label{253a}
\end{shloka}

ಭೂರ್ಭುವಸ್ಸುವರ್ಲೋಕಗಳನ್ನು ವ್ಯಾಪಿಸಿರುವ ಗಾಯತ್ರ್ಯಾತ್ಮಕವಾದ ಗ್ರಂಥ. ಇದು ಪುರುಷಸೂಕ್ತವನ್ನು ಏಕೆ ಹೇಳಬೇಕು ಎಂಬ ಪ್ರಶ್ನೆಗೆ ಉತ್ತರ. ಪುರುಷ ಸೂಕ್ತದಲ್ಲಿ ಸೃಷ್ಟಿಯಜ್ಞ, ಆ ಯಜ್ಞದಲ್ಲಿ ದರ್ವಿ, ಆಜ್ಯ ಎಲ್ಲ ಬರುತ್ತದೆ. ಪುರುಷನ ಬಗೆಗೆ ಇರುವ ಸೂಕ್ತವೂ ಇದಾಗಿದೆ, ನಮ್ಮ ಜೀವನದ ಕಥೆಯೂ ಆಗಿದೆ, ಎರಡನ್ನೂ ಹರಿಗೆಡಹಿಲ್ಲದೆ ಜೋಡಿಸಿರುವ ಕಥೆ ರಾಮಾಯಣ. ವಿಷಯವನ್ನು ತಿಳಿಯದೆ ಸುಳ್ಳು ಸುಳ್ಳಾದ ಪ್ರಶಂಸೆ ಸರಿಯಲ್ಲ. ಯಾರೋ ಒಬ್ಬರು ಘಂಟೆ ಹೊಡೆಯುತ್ತಾ ದೇವರ ಪೂಜೆಮಾಡಿ ಕೊನೆಯ ಘಂಟೆಯಾದ ಮೇಲೆ ನಿತ್ಯವೂ ಕುದುರೆಗೆ ಒಂದು ಹಿಡಿ ಗುಗ್ಗುರಿಯನ್ನು ಹಾಕುತ್ತಿದ್ದರಂತೆ. ಆ ಸಂಕೇತದಂತೆ ಕುದುರೆಯು ನಿತ್ಯವೂ ಕೊನೆಯ ಘಂಟೆಯಾದೊಡನೆಯೇ ಒಮ್ಮೆ ಹೇಷಾರವ ಮಾಡಿ ಬಂದು ನಿಲ್ಲುತ್ತಿತ್ತಂತೆ, ಅದನ್ನು ನೋಡಿ ಇನ್ನೊಬ್ಬರು ಅವರ ದೇವರ ಪೂಜೆಯ ಮಹಿಮೆಯನ್ನು ನೋಡಲು ಮನುಷ್ಯ ಮಾತ್ರವಲ್ಲದೆ, ಪ್ರಾಣಿಗಳು ಕೂಡ ದೈವಭಕ್ತಿಯಿಂದ ಅಲ್ಲಿಬಂದು ಸೇರುತ್ತವೆ, ಎಂದರಂತೆ. ಇಲ್ಲಿ ಕುದುರೆ ಬಂದಿರುವುದು ಕಡಲೆಗುಗ್ಗುರಿಯಲ್ಲಿರುವ ಭಕ್ತಿಯಿಂದ. ದೇವರ ನಿವೇದನಕ್ಕೆ ಮುರುಕು, ಮುತ್ಸೋರೆ ಮುಂತಾದ ತಿಂಡಿಗಳನ್ನು ಇಟ್ಟಿರುವುದು ತಿಳಿದರೆ ಮಕ್ಕಳು ಎಷ್ಟು ಭಕ್ತಿಯಿಂದ ಕೊನೆಯವರೆಗೂ ಕುಳಿತಿರುತ್ತವೆ? ಆದರೆ ಆ ವಿನಿಯೋಗ ಮುಗಿದ ತಕ್ಷಣವೇ ಒಂದು ನಿಮುಷವೂ ಅವು ಇರುವುದಿಲ್ಲ. ಇವತ್ತು ಕ್ಷೀರಾನ್ನ ದೇವಾರಿಗೆ ಎಂದರೆ ಹಯಗ್ರೀವರಲ್ಲಿ ಎಂದೂ ಇಲ್ಲದ ಭಕ್ತಿ ಬರುತ್ತದೆ. ಅದಕ್ಕೆ ಹಿಂದೆ ಹಯಗ್ರೀವರೇ ಇರಲಿಲ್ಲವೇ? ಇಂತಹ ಪ್ರಶಂಸೆಯನ್ನು ಹೇಳುತ್ತಿಲ್ಲ. 

\section*{ಯಾವ ನೋಟದೊಂದಿಗೆ ರಾಮಾಯಣವನ್ನಾಶ್ರಯಿಸ ಬೇಕು?}

ರಾಮಾಯಣದಲ್ಲಿ ಏನಿದೆ? ಒಂದು ಸಾಧಾರಣ ಕಥೆ. ವಿಂಡ್ಸರ್ ತನ್ನ ಹೆಂಡತಿಯ ಸಲುವಾಗಿ ರಾಜ್ಯ ಬಿಡಲಿಲ್ಲವೇ? ಅದರಂತೆಯೇ ರಾಮನೂ ತನ್ನ ತಂದೆಯ ಮಾತಿಗಾಗಿ ``ಸತ್ಯ, ಬದುಕಿದೆಯಾ'' ಎಂದು ರಾಜ್ಯ ಬಿಟ್ಟು ಹೋದ. ಇನ್ನು ಹೆಂಡತಿ ಅಗಲಿ ಹೋದರೆ ಯಾರು ತಾನೇ ಅಳುವುದಿಲ್ಲ? - ಎಂದು ವ್ಯಾಖ್ಯಾನಿಸುವವರಿದ್ದಾರೆ.

ರಾಮಾಯಣದ ಬಲೆಗೆ ನೀವು ಬೀಳಬೇಕಾದರೆ ಏನು ಕಾರಣ? ಬುಡಭದ್ರವಾಗಿರಬೇಕು ನಿಮ್ಮ ವಿಚಾರ. ರಾಮಾಯಣದಲ್ಲಿ ರಾಮನು ರಾವಣನಿಗೆ ಒಂದು ಬಾಣ ಬಿಟ್ಟರೆ ಕುಯ್ಯೋ-ಮುರ್ರೋ ಅನ್ನಬೇಕು, ಹಾಗೆಯೇ ಎನ್ನಬೇಕು. ಹಾಗೆಯೇ ಆಗಬೇಕು ರಾಮಣನಿಗೆ, ಎಂದು ರಸಾಸ್ವಾದನೆ ಮಾಡವವರು ಕೆಲವರು. ಏನೋ ಕಪಿಯ ಬಾಲಕ್ಕೆ ಬೆಂಕಿ ಹಚ್ಚಿದ; ಆದರೆ ಆ ಕಪಿ ಸರಿಯಾಗಿ ಕೈ ಕೊಟ್ಟಿತು. ರಾವಣನಿಗೆ ಕಪಿಬುದ್ಧಿಯನ್ನು ಸರಿಯಾಗಿ ತೋರಿಸಿತು. ಅದು ತಾನೇ ರಾಮಾಯಣದಲ್ಲಿ ಬಹು ಸ್ವಾರಸ್ಯವಾದ ಜಾಗ, ಎನ್ನುವರು ಕೆಲವರು. ರಾಮಾಯಣ ಕಥೆ ಕೇಳಲು ಹೋಗಿ ಇನ್ನೊಬ್ಬ ನಿದ್ರೆಮಾಡಿ, ತನ್ನ ಮೇಲೆ ನಿಂತಿದ್ದ ಮನುಷ್ಯನ ಭಾರವನ್ನೇ ರಾಮಾಯಣವೆಂದು ತಿಳಿದು ರಾಮಾಣ್ಯ ಎಂದರೆ ಸಾಮಾನ್ಯದೇ? ಆಳಿಗಾಳು, ಹೊರೆಗೆ ಹೊರೆ ಎಂದನಂತೆ, ಹರಿಕಥೆ ಕೇಳಲು ಹೋಗಿ ಮೈದಾನದಲ್ಲಿ ನಿದ್ರಿಸಿ ಬಾಯಿ ಬಿಟ್ಟು  ಕೊಂಡಿರುವಾಗ ಯಾವುದೋ ನಾಯಿಯು ಬಾಯಲ್ಲಿ ಮೂತ್ರ ವಿಸರ್ಜನೆ ಮಾಡುತ್ತಿರುವಾಗ ಎಬ್ಬಿಸಲ್ಪಟ್ಟವನೊಬ್ಬ ಹರಿಕಥೆ ಹೇಗಿತ್ತು? ಎಂದರೆ ಒಗಚಾಗಿತ್ತು ಎಂದನಂತೆ. ಹರಿಕಥೆಯಲ್ಲಿ ಎಲ್ಲರೂ ಎದ್ದು ಹೋಗಿ ಬಿಡಲು ಬೆಳಗಿನವರೆಗೂ ಕುಳಿತಿದ್ದ ಶಾಸ್ತ್ರಿಗಳೊಬ್ಬರನ್ನು ನೋಡಿ ಹರಿಕಥಾಭಾಗವತರು ತಮಗೊಬ್ಬರಿಗೆ ತಾನೇ ಈ ಹರಿಕಥೆಯಲ್ಲಿ ಶ್ರದ್ಧೆ-ಎಂದಾಗ ಶಾಸ್ತ್ರಿಗಳು ಅಲ್ಲಾ ಸ್ವಾಮಿ, ನಾನು ಕುಳಿತದ್ದು ಈ ಜನಖಾನಕ್ಕಾಗಿ ನೀವು ಕಥೆ ಮುಗಿಸುವವರೆಗೂ ಅದನ್ನು ತೆಗೆದುಕೊಳ್ಳುವಂತಿಲ್ಲ, ಎಂದರಂತೆ. ಹೀಗೇ ಸಂಪ್ರದಾಯಕ್ಕೋ, ಚಟಕ್ಕೋ ಸಿಕ್ಕಿದ ಮನಸ್ಸೇ? ಅಥವಾ ವಿಷಯದಲ್ಲಿ ರುಚಿ ಹುಟ್ಟಿ ಬಂದಿದ್ದೇ? ಎಂಬುದನ್ನು ನೋಡಬೇಕು. ರಾಮನನ್ನು ಇಬ್ಬರು ಅಣ್ಣತಂಮಂದಿರು ಹಂಚಿಕೊಂಡರಂತೆ. `ರಾ' ಎಂದರೆ ಬಾ ಎಂದು ಒಬ್ಬನು ಹೇಳಿದಾಗ, `ಮಾ' ಎಂದರೆ ಬೇಡ, ಎಂದ ಇನ್ನೊಬ್ಬ. ರಾಮ ಆಯಿತಲ್ಲಾ; ಎರಡನ್ನೂ ಸೇರಿಸಿದರೆ. ಇವೆರಡೂ ಸೇರಿದ ರಾಮ, ಎಷ್ಟರ ಮಟ್ಟಗೆ ರಾಮರ ಕಲ್ಯಾಣ ಗುಣವನ್ನು ಬೋಧಿಸಬಹುದು. ಪರಮಾರ್ಥದಲ್ಲಿ ರುಚಿ ಹುಟ್ಟಿ, ಅದರಿಂದ ಬೆಳೆಯುವ ವಿಶ್ವಾಸ ತಾನೇ ಗಟ್ಟಿ! ಬುಡಭದ್ರವಾಗಿರುತ್ತದೆ!

\section*{ಆಧಿವ್ಯಾಧಿನಿವಾರಣೆಗೆ ಸಿದ್ಧೌಷಧವಾಗಿದೆ ರಾಮಾಯಣ}

ಆತ್ಮಜೀವನವು ಕೆಡದಂತೆ ಲೋಕ ಜೀವನವನ್ನು ನಡೆಸಬೇಕು ಎನ್ನುವವನು, ರಾಮಾಯಣದಲ್ಲಿ, ಲೋಕದಲ್ಲಿ ಭೋಗವನನ್ನನುಭವಿಸುವಾಗ ಲಭಿಸುವ ಇಂದ್ರಿಯ ತೃಪ್ತಿ, ಶೃಂಗಾರಾದಿ ರಸಗಳು ಇತ್ಯಾದಿಗಳನ್ನು ನೋಡಲಿ, ಇದಲ್ಲದ `ಶುಚಿರ್ವಶ್ಯಃ ಸಮಾಧಿಮಾನ್,\label{255} ಆದ ರಾಮನ ಸ್ವರೂಪವನ್ನು ಅನುಭವಿಸಿದರೆ, ಆಗ ಆತ್ಮತೃಪ್ತಿ. ರಾಮಾಯಣದಲ್ಲಿ ಭೌತಿಕ-ದೈವಿಕ-ಆಧ್ಯಾತ್ಮಿಕ ಈ ಮೂರು ಅಂಶಗಳೂ ಇವೆ. ಭೋಗ-ಯೋಗ-ಸಮಾಧಿಸ್ಥಿತರಾಗಿ ಆತ್ಮಾರಾಮರಾಗಿ ಮೂರರ ರಸಾಯನವಾದ ರಾಮಾಯಣವನ್ನು ರಚಿಸಿದರು (ವಾಲ್ಮೀಕಿಗಳು). ಅದರಲ್ಲಿ ಆತ್ಮನಿಗೆ ಬೇಕಾದ ಸಾಪಾಡೂ ಇದೆ. ಇಂದ್ರಿಯಗಳಿಗೆ ಬೇಕಾದ ಸಾಪಾಡೂ ಇದೆ. ಅಂದರೆ, ಅದು ಆಧಿವ್ಯಾಧಿ ನಿವಾರಣೆಗೆ ಸಿದ್ಧೌಷಧವಾಗಿದೆ.  
