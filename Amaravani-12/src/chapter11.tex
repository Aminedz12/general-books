\chapter{ರಾಮಾಯಣ ಪಾರಾಯಣಕ್ಕೆ ಭೂಮಿಕೆ} 

(ದಿನಾಂಕ ೧೧-೧೦-೧೯೫೯ ವಿಜಯದಶಮೀ ಪರ್ವದಂದು `ರಾಮಾಯಣ ಪಾರಾಯಣ ಮಾಡಬೇಕೆಂದಿದ್ದೇನೆ, ಅದನ್ನು ಮಾಡುವುದು ಹೇಗೆ' ಎನ್ನುವ ಶ್ರೀ ಶ್ರೀನಿವಾಸನ್‍ರ ಪ್ರಶ್ನೆಗೆ ಉತ್ತರರೂಪವಾಗಿ ಬಂದ ಮಾತುಗಳು. 

ಈ ವಿಜಯದಶಮಿಯ ದಿನ ಸುಕೃತಿಗಳಾದ ಹನ್ನೊಂದು ಮಂದಿ ಚೇತನರಿಗೆ ಜ್ಞಾನದೀಕ್ಷೆಯನ್ನು ಕರುಣಿಸಿದ ನಂತರ ಎಲ್ಲರೂ ವಿಜ್ಞಾನಮಂದಿರದ ಸಭೆಯಲ್ಲಿ ನೆರೆದಿರಲು ದೀಪಹಸ್ತಳಾದ ಜಗನ್ಮಾತೆಯೊಡನೆ ಭಗವಂತನು ಬಿಜಯ ಮಾಡಿಸುತ್ತಾನೆ. 

ಭಗವಂತನು ತನ್ನ ಸನ್ನಿಧಿಗೆ ಬರಮಾಡಿಕೊಂಡ ನಿಮ್ಮ ಸೋದರರ ಕಾರ್ಯವು ಭಗವಂತನ ಸಂಕಲ್ಪದಿಂದ ಚೆನ್ನಾಗಿ ನೆರವೇರಿತು. ಎಲ್ಲಾ ಸೋದರರೂಕೂಡ ಭಗವಂತನ ಧ್ಯಾನವನ್ನು ಸ್ವಲ್ಪ ಹೊತ್ತು ನೆಮ್ಮದಿಯಾಗಿ ಮಾಡೋಣ ಎಂದು ಎಲ್ಲರೂ ಕಿಂಚಿತ್‍ ಕಾಲ ಧ್ಯಾನ ಮಾಡಿದ ನಂತರ ಹೊರಬಂದ ಭಗವಂತನ ಅಮೃತವಾಣಿ.) 

\section*{ಚೇತನರಿಗೆ ಭಗವಂತನ ಕೃಪಾವಾಣಿ} 

ಯಾವ ತೇಜೋನಿಧಿಯೂ ನಾದಲೋಲನೂ ಆದ ವನಮಾಲಾಧಾರಿಯ ವಕ್ಷ: ಸ್ಥಲದಲ್ಲಿ, ಸುಚೇತಸ್ಸುಮನಸಗಳಿಂದ ಹೆಣೆಯಲ್ಪಟ್ಟ ಹಾರರೂಪರಾದ ಚೇತನರು ಅವನ ವನಮಾಲೆಯ ಜೊತೆಯಲ್ಲಿ, ಅವನ ಹೃದಯದಲ್ಲಿ ವಿಹರಿಸುತ್ತಿದ್ದಾರೋ ಅಂತಹ ಅವನ ಚೇತನರಿಗೆಲ್ಲಾ ಅವನ ಆತ್ಮಗಂಧವು ಲೇಪಿತವಾಗಿ ಅದರಿಂದ ಎಲ್ಲಾ ಚೇತನಗಳೂ ಬೆಳಗಲಿ. ಭಗವಂತನ ಕುಟುಂಬಭರಣಕ್ಕೆ ಆತನ ಆತ್ಮಲಕ್ಷ್ಮಿಯೂ ಬೇಕಾಗಿದ್ದಾಳೆ. ಅಂತಹ ಆತ್ಮಲಕ್ಷ್ಮಿಯ ಕಡೆಯ ಅನುಗ್ರಹವೂ ಬೆಂಬಲವೂ ಈ ಕಾರ್ಯದಲ್ಲಿ ಒದಗಿರುವುದರಿಂದ ಅವರ ಕಡೆಯ ಅಮೃತಹಸ್ತವು ನಿಮ್ಮ ಶಿರಸ್ಸಿನಲ್ಲಿರಲಿ. 

(ತಂಬೂರವನ್ನು ಮೇಲಕ್ಕೆ ಬಲಗಡೆ ನಿಲ್ಲಿಸಿಕೊಂಡು ಈ ಸ್ತೋತ್ರಗಳನ್ನು ಗಾನಮಾಡಿದರು. (೧) ಕರಕಮಲ (೨) ಅನನ್ಯಾಶ್ಚಿಂತಯಂತೋ ಮಾಂ (೩) ಸರ್ವಧರ್ಮಾನ್ಪರಿತ್ಯಜ್ಯ (೪) ಉನ್‍ತಿರುವಡಿ ಶರಣಮೆನ್ರು (೫) ಪಾಹಿ ರಾಮಚಂದ್ರ (೬) ಸಾಕ್ಷಾತ್ಕಾರನೀಸದ್ಭಕ್ತಿ (೭) ನಗುಮೋಮು (೮) ಶೃಂಗಾರಂ ಕ್ಷಿತಿನಂದನಾ ವಿಹರಣೇ (೯) ಬದ್ಧೇನಾಂಜಲಿನಾ ಇಷ್ಟು ಗಾನವಾದ ಮೇಲೆ ಪುಷ್ಪಪ್ರಸಾದ ವಿನಿಯೋಗ.) 

\section*{ವಿಜಯದಶಮಿ-ರಾಮಮಹಿಮಾಗಾನಕ್ಕೆ ಶುಭದಿನ} 

ಎಲ್ಲರೂ ಆರಾಮವಾಗಿ ಕುಳಿತಿದ್ದ ಪಕ್ಷದಲ್ಲಿ, ಆತ್ಮಾರಾಮರಾಗಿ ಅಂತರ್ದೃಷ್ಟಿಯುಳ್ಳವರಾಗಿ ಮತ್ತು ಲೋಕದೃಷ್ಟಿಯುಳ್ಳವರಾಗಿ ಧರ್ಮವೃಕ್ಷವನ್ನು ಬೆಳೆಸುವುದಕ್ಕೋಸ್ಕರ ತಪೋಮಗ್ನರಾಗಿ, ಯೋಗಮಗ್ನರಾಗಿ ಪ್ರಾಚೀನಾಗ್ರದರ್ಭದಲ್ಲಿ ಕುಳಿತು ಯಾವ ರಾಮದೇವರ ರಾಮಾಯಣವನ್ನು ವಾಲ್ಮೀಕಿಗಳು ರಚಿಸಿದರೋ ಅದರ ಪಾರಾಯಣವನ್ನು ಮಾಡಬೇಕೆಂದಿದ್ದೇನೆ. ಅದನ್ನು ಮಾಡುವುದು ಹೇಗೆ? ಎಂದು ಶ್ರೀಮಾನ್‍ ಶ್ರೀನಿವಾಸನ್‍ರಿಂದ ಪ್ರಶ್ನೆ ೨-೩ ವರ್ಷಗಳ ಹಿಂದೆ ಬಂದಿತ್ತು. `ಆಗಲಪ್ಪಾ ಹೇಳುತ್ತೇನೆ, ಸಕಾಲದಲ್ಲಿ' ಎಂದು ಹೇಳಿದ್ದೆ. 

ರಾಜಾಧಿರಾಜರೆಲ್ಲಾ ಧರ್ಮವಿಜಯಕ್ಕೋಸ್ಕರ ಹೊರಡುವ ವಿಜಯದಶಮಿಯಿದಾಗಿದೆ. ಮೇಲೂ ಅಂದು ಆತ್ಮಕಲ್ಯಾಣ, ಲೋಕಕಲ್ಯಾಣಕರವಾದ ತನ್ನ ವ್ಯಾಪಾರಗಳಿಂದ ಮುನಿಗಳ ಪ್ರಾಣೇಂದ್ರಿಯಾದಿಗಳು ತಣಿಯುವಂತೆ ಯಾವ ಧರ್ಮಮೂರ್ತಿಯಾದ ಶ್ರೀರಾಮಚಂದ್ರ ಪರಂಧಾಮನು ಭೂಲೋಕದಲ್ಲಿಳಿದು, ಭೂಲೋಕವನ್ನು ಪರಂಧಾಮವನ್ನಾಗಿ ಮಾಡಿ, ಸುಖ ಶಾಂತಿಗಳನ್ನು ಕೊಟ್ಟು, ಪ್ರಾಣಾಪಾನಗಳು ಸಮವಾಗುವಂತೆ ರಾಜ್ಯವ್ಯವಸ್ಥೆಮಾಡಿದಾಗ ಮುನಿಗಳು ಧ್ಯಾನಮಗ್ನರಾಗಿ ಕುಳಿತರೋ ಅಂತಹ ರಾಮನನ್ನು ಗಾನಮಾಡುವುದಕ್ಕೆ ಶುಭದಿನವಾಗಿದೆ. 

\section*{ರಾಮರಾಜ್ಯದಲ್ಲಿ ಪ್ರಾಣಾಪಾನಗಳ ಸಮಸ್ಥಿತಿ} 

\begin{shloka} 
`ಪ್ರಾಣಾಪಾನೌ ಸಮಾವಾಸ್ತಾಂ ರಾಮೇ ರಾಜ್ಯಂ ಪ್ರಶಾಸತಿ'|\label{161}
\end{shloka} 

ಎಂದು ವ್ಯಾಸರು ಮಹಾಭಾರತದಲ್ಲಿ ಹೇಳಿರುವಂತೆ ರಾಮದೇವರು ರಾಜ್ಯಭಾರ ಮಾಡಿದಾಗ ಪ್ರಾಣಾಪಾನಗಳು ವೈಷಮ್ಯವನ್ನು ತೊರೆದು ಸಾಮ್ಯವನ್ನು ತಾಳಬೇಕಾದರೆ, ಆ ಮಾತಿನ ಗಾಂಭೀರ್ಯವೇನು? ಲೋಕದಲ್ಲಿ ತುಂಬಾ ಸುಖಶಾಂತಿಗಳು ದೊರೆತಾಗಲೇ ಸ್ವಲ್ಪ ಉಸಿರು ಕಟ್ಟಿದಂತಾಗುತ್ತದೆ. ಇನ್ನು ಆತ್ಮಾರಾಮರಾಗಿಯೇ ಕುಳಿತುಬಿಟ್ಟಾಗ ಕೇಳಬೇಕೆ? ಒಳಗೆ ಹೋಗೋ ಹಾಗಿಲ್ಲ, ಹೊರಗೆ ಬರುವ ಹಾಗಿಲ್ಲ ಎಂಬ ಸ್ಥಿತಿ ಬಂದುಬಿಟ್ಟಾಗ, `ಎತ್ತು ಏರಿಗೆಳೆದರೆ ಕೋಣ ನೀರಿಗೆಳೆಯಿತು' ಎಂಬಂತೆ ಒಂದು ಮೇಲಕ್ಕೆ ಒಂದು ಕೆಳಕ್ಕೆ ಕೆಲಸ ಮಾಡುತ್ತಿರುವ ಪ್ರಾಣಾಪಾನಗಳ ವ್ಯಾಪಾರವು ಒಂದೇ ನೆಲೆಯಲ್ಲಿ ನಿಂತಿದೆ. ಇಂದ್ರಿಯಗಳ ಚೆಲ್ಲಾಟವಿಲ್ಲ. ಇಂತಹ ಯಾವ ಸ್ಥಿತಿಯುಂಟೋ ಅದರ ವರ್ಣನೆ ಇದು. ಇಂತಹ ಸ್ಥಿತಿಯಲ್ಲಿ ಯಾವ ಮಹರ್ಷಿಗಳು ರಾಮನಿಗಾಗಿ ಸಂತತಧಾರೆಯಿಂದ ಅವನ ಚಿನ್ಮಯಮೂರ್ತಿಯನ್ನು ಧ್ಯಾನ ಮಾಡುತ್ತಿದ್ದರೋ ಅಂತಹ ಸ್ಥಿತಿಯು ಹೃದಯದಲ್ಲಿರುವ ಕಾಲವೇ ರಾಮರಾಜ್ಯ. 

\section*{ಮಹರ್ಷಿಹೃದಯಕ್ಕೇ ಹೋಗಿ ನಿಲ್ಲುವಂತೆ ಪಾರಾಯಣ ಮಾಡಲು ಭೂಮಿಕೆ} 

ಆ ರಾಮರಾಜ್ಯವು ಆ ರಾಮದೇವರ ಕಾಲದಲ್ಲಿ ಮಾತ್ರವಲ್ಲ. (ತಮ್ಮ ಹೃದಯಸ್ಥಾನ ತೋರಿಸುತ್ತಾ) ಈ ರಾಮದೇವರ ಕಾಲದಲ್ಲೂ ಉಂಟು. ಈ ರಾಮದೇವರ ಕಾಲದಲ್ಲಿಯೂ ನೀವು ತತ್ಪರಾಯಣರಾಗಿ ಇಂತಹ ಪಾರಾಯಣವನ್ನೇ ನಿಮ್ಮ ಮನಸ್ಸಿಗೆ ಸುಖಶಾಂತಿಗಳುಂಟಾಗುವ ರೀತಿಯಲ್ಲಿ ಮಾಡಲು ಒಂದು ಭೂಮಿಕೆ. ಈ ಪುಸ್ತಕವನ್ನು ಭೂಮಿಯು ಹೊತ್ತಿರುವುದರಿಂದ ಭೂಮಿಕೆಯಿದೆ ನಿಜ. ಆದರೆ ನಿಮ್ಮ ಹೃದಯದಲ್ಲಿ ಅದನ್ನು ಹೊತ್ತುಕೊಂಡು ಅವನಿಗನುಗುಣವಾಗಿ ಜೀವನ ನಡೆಸಲು ಒಂದುಭೂಮಿಕೆ ಬೇಕು. ರಾಮದೇವರು ಕಾಡಿಗೆ ಹೊರಟಾಗ ಅವರನ್ನು ಅಯೋಧ್ಯೆಯ ಜನರು ಹಿಂಬಾಲಿಸಲು ರಾಮರು ಅವರನ್ನು ಕುರಿತು ``ಏನಪ್ಪ! ನಿಮ್ಮ ಗೃಹ್ಯಾಗ್ನಿಗಳನ್ನು ಬಿಟ್ಟು ಹೇಗೆ ಬರುತ್ತೀರಿ? ಅದನ್ನು ಪೂಜಿಸುವವರಾರು?" ಎಂದು ಕೇಳಲು- 

\begin{shloka} 
`ಹೃದಯೇಷ್ವೇವ ತಿಷ್ಠಂತಿ ವೇದಾ ಯೇ ನಃ ಪರಂ ಧನಂ !\label{162b}
\end{shloka} 

ಎಂದು ಉತ್ತರ ಕೊಡುತ್ತಾರೆ. ವೇದರೂಪದಲ್ಲಿ ಅವುಗಳನ್ನು ಹೃದಯದಲ್ಲಿ ಸಮಾರೋಪಣೆ ಮಾಡಿಕೊಂಡಿದ್ದಾರೆ `ಸ ಕ್ಷಯ ಏಹಿ'\label{162a} ಎಂದು ತಮ್ಮ ತಮ್ಮ ಗೃಹಿಣಿಯರೊಡಗೂಡಿ ಹೊರಗಡೆ ನಿತ್ಯ ಕರ್ಮವಾದ ಅಗ್ನಿಪೂಜೆಯನ್ನು ಮಾಡಲಾಗದಿದ್ದರೂ ಆ ಅಗ್ನಿಯು ಅವರಿಗೆ ಹೃದಯದಲ್ಲೇ ಇದೆ. ಅದನ್ನವರು ಎಂದಿಗೂ ಬಿಟ್ಟಿಲ್ಲ. ಅಂತೆಯೇ ರಾಮಾಯಣವನ್ನು ಹೃದಯದಲ್ಲಿ ಧರಿಸಬೇಕು. ವಾಲ್ಮೀಕಿಯ ಹೃದಯದಲ್ಲಿತ್ತು ರಾಮಾಯಣ. ಆ ಭೂಮಿಯಿಂದ ಹೊರಗೆ ಬಂದ ರಾಮಾಯಣ ತನ್ನ ಆ ಹೃದಯದಲ್ಲೇ ಹೋಗಿ ನಿಲ್ಲುವಂತೆ ಪಾರಾಯಣ ಮಾಡಲು ಒಂದು ಭೂಮಿಕೆಯನ್ನು ಕೊಟ್ಟನಂತರ, ಪಾರಾಯಣವನ್ನು ಶ್ರೀನಿವಾಸನ್‍ ಸ್ವಲ್ಪ ನಡೆಸಲಿ. ಕಾಲ ಚಕ್ರವು (ಗಡಿಯಾರದಕಡೆ ನೋಡಿ) ಮುಂದಕ್ಕೆ ಓಡುತ್ತಿದೆ. ಬಹಳ ಕ್ಷುದ್ಬಾಧೆಯಿಲ್ಲದಿದ್ದರೆ ಕೊಂಚಕಾಲದಲ್ಲಿ ಅದನ್ನು ಮುಂದಿಡೋಣ. (ಕನ್ನಡಕವನ್ನು ತೆಗೆದು ಹಾಕಿಕೊಂಡು) ರಾಮಾಯಣವನ್ನು ಓದಬೇಕಾದರೆ ಎರಡು ಕಣ್ಣು ಸಾಲದು. ಇನ್ನೆರಡು ಕಣ್ಣು ಬೇಕಾಗಿದೆ. ಬನ್ನೀಪ್ಪಾ! ಶ್ರೀನಿವಾಸನ್‍ ಮುಂದೆ ಕುಳಿತುಕೊಳ್ಳಿ. 

\section*{ಮಹಾತ್ಮರ ಮನೋಧರ್ಮದ ಚೌಕಟ್ಟಿನೊಳಗೆ ಪಾರಾಯಣ} 

ರಾಮಾಯಣವನ್ನು ವಾಲ್ಮೀಕಿಯು ಮೊಟ್ಟಮೊದಲು ನಾರದರ ಮುಖದಿಂದ ಪ್ರಶ್ನೋತ್ತರದ ಮೂಲಕ ಸಂಗ್ರಹವಾಗಿ ತೆಗೆದುಕೊಂಡ ವಿಷಯ ಕೇಳಿರಬಹುದು. ನಾರದರ ವರ್ಣನೆಯು ವಾಲ್ಮೀಕಿರಾಮಾಯಣದಲ್ಲಿ ಈ ರೀತಿ ಇದೆ- 

\begin{shloka} 
`ತಪಸ್ಸ್ವಾಧ್ಯಾಯನಿರತಂ ತಪಸ್ವೀ ವಾಗ್ವಿದಾಂ ವರಮ್‍|\label{162}\\ 
ನಾರದಂ ಪರಿಪಪ್ರಚ್ಛ ವಾಲ್ಮೀಕಿರ್ಮುನಿಪುಂಗವಮ್‍||
\end{shloka} 

ಇಂತಹ ವಿಶಿಷ್ಟಗುಣವುಳ್ಳ ನಾರದಮುನಿಯನ್ನು ವಾಲ್ಮೀಕಿಮುನಿಯು ಪ್ರಶ್ನೆ ಮಾಡುತ್ತಾರೆ. ಲೋಕದಲ್ಲಿ ಸಾಮಾನ್ಯಜನರನ್ನು ಕೇಳಿದರೆ ರಾಮಾಯಣ ಗೀಮಾಯಣ ಎಂದು ತರಹಾವರಿ ವ್ಯಾಖ್ಯಾನ ಹೇಳುತ್ತಾರೆ. ಆದರೆ ಇಷ್ಟು ಗುಣವುಳ್ಳ ಮಹಿಮಾನ್ವಿತರನ್ನು ಕೇಳಿದರೆ, ಅವರ ಮನೋಧರ್ಮ ಎಲ್ಲಿಯವರೆಗೆ ಹೋಗಬಹುದು ಎಂಬುದನ್ನು ತಿಳಿದು ಅಷ್ಟಕ್ಕೆ ಒಂದು ಚೌಕಟ್ಟನ್ನು ಹಾಕಿಕೊಂಡು ಅದರೊಳಗೆ ರಾಮಾಯಣವನ್ನು ಪಾರಾಯಣ ಮಾಡಬೆಕು. 

\section*{ವಿಹಾಯಸವನ್ನೂ ಏರಬಲ್ಲ ನಾರದರಿಂದ ರಾಮಗುಣವರ್ಣನೆ} 

ವಾಲ್ಮೀಕಿಗಳ ಕುತೂಹಲ ಹೇಗಿತ್ತಪ್ಪಾ ಈ ವಿಷಯದಲ್ಲಿ? ಎಂದರೆ- 

\begin{shloka} 
`ಪರಂ ಕೌತೂಹಲಂ ಹಿ ಮೇ\label{163}\\ 
ಮಹರ್ಷೇ ತ್ವಂ ಸಮರ್ಥೋಽಸಿ ಜ್ಞಾತುಮೇವಂ ವಿಧಂ ನರಃ'|| 
\end{shloka} 

ಅದಕ್ಕೆ ನಾರದರ ಉತ್ತರ ಈ ರೀತಿ ಇದೆ- 

\begin{shloka}
`ಶ್ರುತ್ವಾ ಚೈತತ್‍ ತ್ರಿಲೋಕಜ್ಞಃ ವಾಲ್ಮೀಕೇರ್ನಾರದೋ ವಚಃ|\label{163d}\\ 
ಶ್ರೂಯತಾಮಿತಿ ಚಾಮಂತ್ರ್ಯ ಪ್ರಹೃಷ್ಟೋ ವಾಕ್ಯಮಬ್ರವೀತ್‍||\\ 
ಬಹವೋ ದುರ್ಲಭಾಶ್ಚೈವ ಯೇ ತ್ವಯಾ ಕೀರ್ತಿತಾ ಗುಣಾಃ\label{163a}\\ 
ಮುನೇ ವಕ್ಷ್ಯಾಮ್ಯಹಂ ಬುದ್ಧ್ವಾ ತೈರ್ಯುಕ್ತಃ ಶ್ರೂಯತಾಂ ನರಃ||\\ 
ಇಕ್ಷ್ವಾಕು ವಂಶಪ್ರಭವೋ ರಾಮೋ ನಾಮ ಜನೈಃ ಶ್ರುತಃ|\label{163aa}\\ 
ನಿಯತಾತ್ಮಾ ಮಹಾವೀರ್ಯೋ ದ್ಯುತಿಮಾನ್‍ ಧೃತಿಮಾನ್‍ ವಶೀ||\\ 
ಬುದ್ಧಿಮಾನ್ನೀತಿಮಾನ್ವಾಗ್ಮೀ ಶ್ರೀಮಾಂಚ್ಛತ್ರುನಿಬರ್ಹಣಃ|\\ 
ಸಮಸ್ಸಮವಿಭಕ್ತಾಂಗಃ ಸ್ನಿಗ್ಧವರ್ಣಃ ಪ್ರತಾಪವಾನ್‍|\\ 
ಪೀನವಕ್ಷಾ ವಿಶಾಲಾಕ್ಷೋ ಲಕ್ಷ್ಮೀವಾನ್‍ ಶುಭಲಕ್ಷಣಃ||\\ 
ಯಶಸ್ವೀ ಜ್ಞಾನಸಂಪನ್ನಃ ಶುಚಿರ್ವಶ್ಯಃ ಸಮಾಧಿಮಾನ್‍|\\ 
ಆರ್ಯಸ್ಸರ್ವಸಮಶ್ಚೈವ ಸದೈಕಪ್ರಿಯದರ್ಶನಃ||
\end{shloka} 

ಇಂತಹ ಗುಣಶೀಲಗಳಿಂದ ಕೂಡಿದ ಒಬ್ಬ ಮಹಾಪುರುಷನ ಸಚ್ಚರಿತ್ರವನ್ನು ಕೇಳಬೇಕು. ಇಂತಹ ಆತ್ಮಗುಣಗಳಿಂದ ಕೂಡಿದ ಮಹಾಪುರುಷನ ಸಚ್ಚರಿತ್ರವನ್ನು ನಾರದರು ತಾನೇ ಹೇಳಬೇಕು? `ಸಜಗಾಮ ವಿಹಾಯಸಂ'\label{163c} ಭೂಮಿಯಲ್ಲೇ ನಿಂತವರಲ್ಲ. ಆಕಾಶಕ್ಕೆ ಹಾರುವವರು. ಅಂಥವರಾದಾಗಲೇ ಅವರು ಹಾಗೆ ಹೇಳುತ್ತಾರೆ. `ಬುದ್ದ್ವಾ'-ತಿಳಿದು ಹೇಳುತ್ತಾರೆ. 

\section*{ರಾಮಾಯಣರಚನೆಗೆ ಜೀವಾತುವಾದ ಘಟನೆ} 

ನಾರದರು ಅಂತರ್ದೃಷ್ಟಿಯಲ್ಲಿ , ದಿವ್ಯಚಕ್ಷುಸ್ಸಿನಿಂದ ನೋಡಿ ಹೇಳಿದ ಬಳಿಕ ವಾಲ್ಮೀಕಿಗಳು, ಹೇಳಿದ್ದನ್ನು ಹಾಗೆಯೇ ಮನಸ್ಸಿಗೆ ತೆಗೆದುಕೊಂಡು ಸ್ನಾನಾರ್ಥವಾಗಿ ತಮಸಾತೀರಕ್ಕೆ ಹೋಗುತ್ತಾರೆ. ಆ ನೀರು- 

\begin{shloka} 
`ಅಕರ್ದಮಮಿದಂ ತೀರ್ಥಂ ಭರದ್ವಾಜ ನಿಶಾಮಯ|\label{164}\\ 
ರಮಣೀಯಂ ಪ್ರಸನ್ನಾಂಬು ಸನ್ಮನುಷ್ಯಮನೋ ಯಥಾ'||
\end{shloka} 

ಮಹಾಪುರುಷನ ಸಚ್ಚರಿತ್ರವನ್ನು ಚಿಂತಿಸುತ್ತಿರುವ ಅವರಿಗೆ ಆ ತಿಳಿಯಾದ ನೀರು ಸಂತರ ತಿಳಿಯಾದ ಮನಸ್ಸಿನಂತೆ ಕಾಣುತ್ತದೆ. ಇದಾದ ಬಳಿಕ ಬಂದು ವೈಶಿಷ್ಟ್ಯಪೂರ್ಣವಾದ ದೃಶ್ಯವನ್ನು ನೋಡುತ್ತಾರೆ. ಅದೇ ನಮ್ಮ ರಾಮಾಯಣಕ್ಕೆ ಒಂದು ಜೀವನಾಡಿಯಾಗಿದೆ. ಸೊಂಪಾದ ತರುಲತೆಗಳಿಂದ ಕೂಡಿದ ಕಾಡು. ರಮಣೀಯವಾದ ಸನ್ನಿವೇಶ. ಕ್ರೌಂಚಪಕ್ಷಿಗಳೆರಡು. ಅವುಗಳ ಚಾರು ನಿಃಸ್ವನ. `ಅನಪಾಯಿನಂ, ಚಾರುನಿ:ಸ್ವನಂ'. ಕರುಣಾಳುವಾದ ಮಹರ್ಷಿಯ ಮುಂದೆ ಕಾವ್ಯರಚನೆಗೆ ಜೀವಾತುವಾದ ಒಂದು ಘಟನೆ ನಡೆದಿದೆ. ಬೇಡನೊಬ್ಬನು ಅವುಗಳಲ್ಲಿ ಒಂದು(ಗಂಡು) ಪಕ್ಷಿಯನ್ನು ಹೊಡೆದು ಬೀಳಿಸುತ್ತಾನೆ. ಆತನ ವಿಷಯದಲ್ಲಿ ಹೇಳಿರುವ `ಪಾಪನಿಶ್ಚಯಃ, ವೈರನಿಲಯಃ' ಎಂಬ ಎರಡು ಪದಗಳನ್ನು ಜ್ಞಾಪಕದಲ್ಲಿಡಬೇಕು. ಋಷಿಯಲ್ಲಿ ಕರುಣಾರಸ ಉಕ್ಕಿಬರುತ್ತದೆ. 

\section*{ರಾಮಾಯಣಗಾನದ ಕುರಿತು}

ಕರುಣಾರಸವನ್ನು ಗಾನಮಾಡುವುದಕ್ಕೆ ಗಾಂಧಾರವನ್ನೋ ನಿಷಾದವನ್ನೋ ಇಟ್ಟುಕೊಳ್ಳಬೇಕು. ಆ ಕಂಡೀಷನ್‍ ಬಂದಾಗ ಸ್ವಾಭಾವಿಕವಾಗಿ ಆ ಗ್ರಾಮಕ್ಕೆ ಮನಸ್ಸು ಹೋಗುವುದೆಂದು ಸ್ವರಶಾಸ್ತ್ರಜ್ಞರು ಹೇಳುತ್ತಾರೆ. ಕರುಣೆಯುಂಟಾಗಲು ಕಾರಣ ಧರ್ಮನಾಶವಾಯಿತಲ್ಲಾ ಎಂಬುದು. ಆಗ ಸ್ವಭಾವವಾಗಿ ಉಂಟಾದ ಗಾಂಧಾರನಿಷಾದಗಳಲ್ಲೇ ಹಾಡಬೇಕು. ಸುಮ್ಮನೆ ತಮ್ಮ ತಮ್ಮ ಮನಸ್ಸಿಗೆ ತಕ್ಕಂತೆ ಗಾನಮಾಡಲಾಗುವುದಿಲ್ಲ. `ಪಾಠ್ಯೇ ಗೇಯೇ ಚ ಮಧುರಂ'\label{164a} 

\section*{ಕಾವ್ಯದ ರೂಪರೇಖೆ ಕೆಡದಂತೆ ಹೊರಗಿಡಬಲ್ಲವರ ಬಗ್ಗೆ ಮಹರ್ಷಿಯ ಚಿಂತೆ} 

ಋಷಿಯೇನೋ ಖುಷಿಯಾಗಿ ಒಳಪ್ರಪಂಚಕ್ಕೆ ಹೋದಾಗ ಏನೇನೋ ಕಂಡಿತು ಅಂಗೈನೆಲ್ಲಿಯಾಗಿ, ಅದನ್ನು ಕಾವ್ಯರೂಪದಲ್ಲಿ ರಚಿಸಿಬಿಟ್ಟರು. ಆದರೆ ಅಷ್ಟು ಭಾವವನ್ನು ಹೊತ್ತು ಅದರ ರೂಪರೇಖೆ ಕೆಡದಂತೆ (ಮುಡಿಗುಸುಮ ಬಾಡದಂತೆ) ಲೋಕದಲ್ಲಿಡಬಲ್ಲವರು ಯಾರಪ್ಪಾ ಎಂಬ ಚಿಂತೆ ಬಂದಿತು- 

\begin{shloka} 
ಕೃತ್ವಾಪಿ ತನ್ಮಹಾಪ್ರಾಜ್ಞಃ ಸಭವಿಷ್ಯಂ ಸಹೋತ್ತರಂ|\label{165}\\ 
ಚಿಂತಯಾಮಾಸ ಕೋನ್ವೇತತ್ಪ್ರಯುಂಜೀಯಾದಿತಿ ಪ್ರಭುಃ||
\end{shloka} 

\section*{ರಾಮಾಯಣದ ಉಗಮ} 

ಋಷಿಯು ಕರುಣಾರಸದಲ್ಲಿ ಕಾವ್ಯರಚನೆ ಮಾಡಲು ಒಂದು ನಿಮಿತ್ತ ಒದಗಿತ್ತು. ನಿಮಿತ್ತ ಒಳ್ಳೆಯದಕ್ಕಾಗಿ ಬರಬಹುದು. ಕೆಟ್ಟದ್ದಕ್ಕಾಗಿ ಬರಬಹುದು. ಈ ಉಪಘಟನೆಯ ಪರಿಣಾಮವೇನಾಯಿತು? 

\begin{shloka} 
ಋಷೇರ್ಧರ್ಮಾತ್ಮನಸ್ತಸ್ಯ ಕಾರುಣ್ಯಂ ಸಮಪದ್ಯತ|\label{165d}\\ 
ತತಃ ಕರುಣವೇದಿತ್ವಾ ಅಧರ್ಮೋಽಯಮಿತಿ ದ್ವಿಜಃ||
\end{shloka} 

ರಾಮಾಯಣದ ಉಗಮಕ್ಕೆ ಈ ಘಟನೆಯು ಕಾರಣವಾಗಿದೆ. ಋಷಿಯು ಧರ್ಮಾತ್ಮರು. ಧರ್ಮವನ್ನೇ, ಧರ್ಮಪುರುಷನ ಗುಣಗಳನ್ನೇ ಚಿಂತಿಸುತ್ತಿದ್ದಾರೆ. ಹಿಂದಿನ ಸಂಸ್ಕಾರ `ಧರ್ಮಾತ್ಮನೋ ಗುಣವತಃ'\label{165a} ಧರ್ಮಾತ್ಮಾ ಸತ್ಯಸಂಧಶ್ಚ'.\label{165b} ಇಂತಹ ಗುಣಗಳನ್ನು ಕೇಳಿದ ಬಳಿಕ ಇಂತಹದನ್ನು ಮನಸ್ಸು ಮೆಲಕು ಹಾಕದೇ ಇರುವುದಿಲ್ಲ. ಅದಕ್ಕೇ ನೀರನ್ನು ನೋಡಿದಾಗ- 

\begin{shloka} 
`ರಮಣೀಯಂ ಪ್ರಸನ್ನಾಂಬು ಸನ್ಮನುಷ್ಯಮನೋ ಯಥಾ'|\label{165c}
\end{shloka} 

ಆ ನೀರೂ ಸನ್ಮನುಷ್ಯನ ಮನಸ್ಸಿನಂತೆ ತಿಳಿಯಾಗಿ ಕಾಣುತ್ತದೆ ಅವರಿಗೆ. ಮಹರ್ಷಿಯ ಮನಕರಗಿ ರಸವೊಸರಿ ಭಾವವೀಣೆಯು ಮಿಡಿದಾಗ ಬಂದ ಕಾವ್ಯ ರಾಮಾಯಣ. 

\section*{ವಾಲ್ಮೀಕಿಗಿರಿಯಿಂದ ಹರಿದು ಬಂದಿದೆ ರಾಮಾಯಣ ಮಹಾನದೀ} 

ಒಂದು ಸರಸ್ವತೀನದಿಯು ಅಂತರ್ವಾಹಿನಿಯಾಗಿ ವಾಲ್ಮೀಕಿಯ ಹೃದಯಾಂತರದಲ್ಲಿ ಹರಿಯುತ್ತಿದೆ. ಅಂತರ್ವಾಹಿನಿಯಾದ ಇದು, ಬಹಿರ್ವಾಹಿನಿಯಾದ ತಮಸಾನದಿ, ಈ ಎರಡೂ ಸೇರಿ ರಾಮಾಯಣ ಮಹಾನದಿಯು ಹೊರಡುತ್ತದೆ. ಉನ್ನತವಾದ ವಾಲ್ಮೀಕಿಗಿರಿಯಿಂದ ಹೊರಟುಬರುತ್ತದೆ ಆ ಮಹಾನದಿ. ಭೂಲೋಕದಲ್ಲಿ ಒಂದಿಷ್ಟು, ಭುವದಲ್ಲಿಷ್ಟು, ಸುವದಲ್ಲಿಷ್ಟು. ಮೂರರಲ್ಲೂ ಹರಿದು ಬಂದ ನದಿಯಿದು. ವಾಲ್ಮೀಕಿಯ ಶಿಷ್ಯರೊಬ್ಬರು ಹಾಡಿರುವಂತೆ- 

\begin{shloka} 
`ವಾಲ್ಮೀಕಿಗಿರಿಸಂಭೂತಾ ರಾಮಸಾಗರಗಾಮಿನೀ|\label{166b}\\ 
ಪುನಾತಿ ಭುವನಂ ಪುಣ್ಯಾ ರಾಮಾಯಣಮಹಾನದೀ'||
\end{shloka} 

\section*{ಶಾಂತಿಸಮೃದ್ಧವಾದ ನೆಲೆಯಿಂದ ಬಂದ ಕಾವ್ಯವನ್ನು ಅಲ್ಲಿಗೇ ಹೋಗಿ ನಿಲ್ಲುವಂತೆ ಗಾನ ಮಾಡಬೇಕು.} 

\begin{shloka} 
`ಮಾ ನಿಷಾದ ಪ್ರತಿಷ್ಠಾಂ ತ್ವಮಗಮಃ ಶಾಶ್ವತೀಃ ಸಮಾಃ|\label{166}\\ 
ಯತ್ಕ್ರೌಂಚಮಿಥುನಾದೇಕಮವಧೀಃ ಕಾಮಮೋಹಿತಮ್‍'|| 
\end{shloka} 

ಇದನ್ನು ಹೇಳುವಾಗ ಋಷಿಯ ದೃಷ್ಟಿ ಮೇಲೆ, ಆ ಪಕ್ಷಿಗಳು ಸುಖವಾಗಿ ಚಾರುನಿಃಸ್ವನ ಮಾಡುತ್ತಿದ್ದ ಕಡೆಗೆ, ಕೆಳಗೆ ಬಂದು ಬಿದ್ದಕಡೆಗೆ ಎರಡರಲ್ಲೂ ದೃಷ್ಟಿಯಿದೆ. ಇದನ್ನು ವಾಲ್ಮೀಕಿಗಳು ಹೇಗೆ ಹಾಡಿದರು? ನೀವೇನು ವಾಲ್ಮೀಕಿಯನ್ನು ನೋಡಿದ್ದೀರಾ? ಎಂದರೆ ಆ ಮನೋಧರ್ಮ ಬಂದರೆ ಹಾಗೇ ಹಾಡಿಸುತ್ತದೆ. ಆ ವಾಲ್ಮೀಕಿಗಳ ಹೃದಯದ ಮಟ್ಟಕ್ಕೆ ಹತ್ತಿ ಅಲ್ಲೇ ನಿಲ್ಲಿಸಬೇಕು. ಯಾವುದು ಯಾವ ಲೋಕದಲ್ಲಿ ಹುಟ್ಟಿಬಂತೋ ಅದಕ್ಕೆ ಅಲ್ಲೇ ನೆಲೆ. ಶಾಂತಿಸಮೃದ್ಧವಾದ ನೆಲೆಯಿಂದ ಬಂದಿದೆ. ಶಾಂತಾರಾಮರಾಗಿರುವ ಹೃದಯದಲ್ಲೇ ಅದು ಹೋಗಿ ನಿಲ್ಲಬೇಕು. ಅದು ಸತ್ಯಲೋಕ. 

\section*{ಮಹಾಪುರುಷಸೂಕ್ತವಾದ ಈ ಕಾವ್ಯವನ್ನು ರಚಿಸಿದ ಕವಿ ಪರಮಾತ್ಮಸ್ವರೂಪನೇ ಆಗಿದ್ದಾನೆ.} 

ರಾಮಾಯಣ ಚಿರಂಜೀವಿಯಾಗಿರಲು ಬ್ರಹ್ಮನ ವರ- 

\begin{shloka} 
`ಯಾವತ್‍ ಸ್ಥಾಸ್ಯಂತಿ ಗಿರಯಃ ಸರಿತಶ್ಚ ಮಹೀತಲೇ|\label{166a}\\ 
ತಾವದ್ರಾಮಾಯಣಕಥಾ ಲೋಕೇಷು ಪ್ರಚರಿಷ್ಯತಿ'||\\ 
`ಯಾವದ್ರಾಮಾಯಣಕಥಾ ತ್ವತ್ಕೃತಾ ಪ್ರಚರಿಷ್ಯತಿ|\\ 
ತಾವದೂರ್ಧ್ವಮಧಶ್ಚತ್ವಂ ಮಲ್ಲೋಕೇಷು ನಿವತ್ಸಸಿ'||
\end{shloka} 

ಇಂತಪ್ಪ ಕಾವ್ಯವನ್ನು ರಚಿಸುವ ಕವಿ ಎಂತಹವನಾಗಿರಬೇಕು? ಕವಿಯೆಂದರೆ ಪರಮಾತ್ಮ. ಅಸ್ಖಲಿತ ಜ್ಞಾನಸಂಪನ್ನನಾಗಿ ಅವನೇ ಪರಮಾತ್ಮಸ್ವರೂಪನಾಗಿರುತ್ತಾನೆ. ಈ ಕಾವ್ಯನಾಯಕನು ಪರಿಪೂರ್ಣಗುಣೋಪೇತನಾಗಿದ್ದಾನೆ. ನಾಲ್ಕು ಪುರುಷಾರ್ಥಗಳನ್ನು ನೀಡುವ ಮಹಾಪುರುಷಸೂಕ್ತವಾಗಿದೆ ರಾಮಾಯಣ. ಅದಕ್ಕೇ ಪುರುಷಸೂಕ್ತದಿಂದ ಅದಕ್ಕೆ ಪುಷ್ಪಾಂಜಲಿಯ ಸಮರ್ಪಣೆ. 

\section*{ಋಷಿಹೃದಯವನ್ನೇರಿ, ಭಾವ ಗ್ರಹಿಸಿ, ಅದನ್ನು ಹೊರಪಡಿಸಬಲ್ಲ ಕುಶಲವರು} 

ರಾಮಾಯಣವನ್ನು ರಚಿಸಿದೊಡನೆಯೇ ಯಾರಪ್ಪಾ ಇದನ್ನು ಸರಿಯಾಗಿ ಇಡಬಲ್ಲರು? ಎಂಬ ಚಿಂತೆ ಋಷಿಗೆ- 

\begin{shloka} 
`ಕೃತ್ವಾಪಿ ತನ್ಮಹಾಪ್ರಾಜ್ಞಃ ಸಭವಿಷ್ಯಂ ಸಹೋತ್ತರಂ|\label{167}\\ 
ಚಿಂತಯಾಮಾಸ ಕೋನ್ವೇತತ್ಪ್ರಯುಂಜೀಯಾದಿತಿ ಪ್ರಭುಃ'||
\end{shloka} 

ಆಗ ಲವಕುಶರು ಸಿಕ್ಕುತ್ತಾರೆ. ಋಷಿ ಹೃದಯವನ್ನು ಹತ್ತಿ, ಭಾವ ಗ್ರಹಿಸಲು ಅವರು ಯೋಗ್ಯರು. ಏಕೆಂದರೆ ``ಕುಶೀಲವೌ ಚೈವ ಮಹಾತಪಸ್ವಿನೌ"\label{167a} `ತಪಸ್ವಿನೌ ಮಹಾಭಾಗೌ' `ಮುನಿವೇಷೌ ಕುಶೀಲವೌ' `ಮಹಾತ್ಮಾನೌ ಮಹಾಭಾಗೌ'. ರಾಮನ ಗುಣವನ್ನು ಗಾನಮಾಡಲು, ಧರ್ಮಾತ್ಮನಾದ ಕ್ಷತ್ರಿಯನ ಗುಣಗಾನ ಮಾಡಲು ಅಂತಹ ರಕ್ತದವರೇ ಬೇಕು. ಇಲ್ಲದಿದ್ದರೆ ಆ ರಸ ಹೇಗೆ? ಇವರಾದರೋ ಸ್ವಯಂ ರಾಮನ ಮಕ್ಕಳೇ ಆಗಿದ್ದಾರೆ. ಎಲ್ಲದರಲ್ಲೂ ಅವನನ್ನೇ ಹೋಲುತ್ತಾರೆ. 

\begin{shloka} 
`ಬಿಂಬಾದಿವೋದ್ಧೃತೌ ಬಿಂಬೌ ರಾಮದೇಹಾತ್ತಥಾಪರೌ'|\label{167b}
\end{shloka} 

ಇಬ್ಬರದೂ ಒಂದೇ ರೂಪ, ಒಂದೇ ಧ್ವನಿ, ಒಬ್ಬರದು ಸುಂದರ. ಮತ್ತೊಬ್ಬರದು ವಕ್ರವಲ್ಲ. ಯಮಳರ ಧ್ವನಿ ಹೇಗಿದೆ? ಒಂದೇ ಶ್ರುತಿಯಲ್ಲಿರುವ ಎರಡು ಗಂಟೆಗಳನ್ನು ಬಾರಿಸಿದರೆ ಒಂದರಲ್ಲೊಂದು ಲೀನವಾಗುವಂತಿದೆ ಇವರ ಧ್ವನಿ. 

\section*{ರಾಮನನ್ನೂ ಋಷಿಗಳನ್ನೂ ಮೆಚ್ಚಿಸಿದ ಕುಶಲವರ ಗಾನ} 

ಗಾಂಧರ್ವತತ್ತ್ವಜ್ಞರೂ ಮೂರ್ಛನಾಸ್ಥಾನಕೋವಿದರೂ ಆಗಿ ಮಾರ್ಗವಿಧಾನ ಸಂಪತ್ತಿನಿಂದ ಗಾನ ಮಾಡಿದರು. ದೇಶೀವಿಧಾನದಲ್ಲಿ ತಮ್ಮ ತಮ್ಮ ಬುದ್ಧಿದೇಶದಲ್ಲಿದ್ದ ಹಾಗೆಯೇ ಹಾಡಿದರೆ ದೇಶಿ. ಇವರದು ಮಾರ್ಗ. ವಾಲ್ಮೀಕಿಮಹರ್ಷಿಗಳಿಂದ ಯಾವ ಮಾರ್ಗದಲ್ಲಿ ಬಂದಿತೋ ಆ ಮಾರ್ಗದಲ್ಲೇ ಹಾಡಿದರು. ವಿರಿಂಚಿಯ ಸೃಷ್ಟಿಯಲ್ಲಿ ಹೇಗೆ ಮಾರ್ಗವುಂಟೋ ಹಾಗೇಹಾಡಿದರು. ಆ ಪದ ಮೇಲೆ ಹೇಳಿದ ಮಾತು. ಮಾರ್ಗ ಎಂಬುದಕ್ಕೆ ಪಾರಿಭಾಷಿಕವಾದ ಅರ್ಥವೂ ಇದೆ. ಲವಕುಶರು ಹಾಡಲಾಗಿ ಅದನ್ನು ಮೊಟ್ಟಮೊದಲಾಗಿ ಒರೆಗಲ್ಲಾಗಿ ಯಾವುದನ್ನು ಮಾಡಿಕೊಂಡರು? ಮೊದಲು ಅಪ್ರಿಷಿಯೇಟ್‍ ({\eng Appreciate}) ಮಾಡಿದವರಾರು? ಮಹರ್ಷಿಗಳು. ಇಂದ್ರಿಯಗಳನ್ನು ಜಯಿಸಿದವರು. ಅವರಿಂದ ಮೆಚ್ಚಿಗೆ ಪಡೆದವರು ಲವಕುಶರು. ಸಂಗೀತಗಾರರಿಗೆ ಬೇರೆ ಸಂಗೀತಗಾರರ ಸಂಗೀತವನ್ನು ಕೇಳಿದರೆ, ಹೇಗಿತ್ತು? ಎಂದರೆ ಅಸೂಯೆಯಿಂದ `ಇತ್ತು' ಎನ್ನುತ್ತಾರೆ. ಆದರೆ ಋಷಿಗಳು ಹಾಗಲ್ಲ, ಮುಕ್ತಕಂಠದಿಂದ ಹೊಗಳುತ್ತಾರೆ. ಋಜುಬುದ್ಧಿಯಿಂದ- 

\begin{shloka} 
`ಸಾಧು ಸಾಧ್ವಿತಿ ತಾವೂಚುಃ ಪರಂ ವಿಸ್ಮಯಮಾಗತಾಃ|'\label{168f}
\end{shloka} 

ಸಾಧಾರಣವಾದ ವಿಸ್ಮಯವಲ್ಲ- `ಪರಂ ವಿಸ್ಮಯಂ' 

\begin{shloka} 
`ಪ್ರಶಶಂಸುಃ ಪ್ರಶಸ್ತವ್ಯೌ ಗಾಯಮಾನೌ ಕುಶೀಲವೌ|\label{168}
\end{shloka} 

ಇಂದ್ರಿಯಗಳಲ್ಲೇ ಯಾರು ಓಡಾಡುತ್ತಾರೋ ಅಂತಹವರಿಂದ ಹೊಗಳಿಕೆ ಪಡೆಯುವುದು ಸುಲಭ, ಆದರೆ ಅತೀಂದ್ರಿಯವಾದ ಪರಮಾನಂದವನ್ನನುಭವಿಸಿದವರಿಗೂ ಪರಂ ವಿಸ್ಮಯವನ್ನುಂಟುಮಾಡಬೇಕಾಗಿದ್ದರೆ, ಹೇಗಿರಬೇಕು ಗಾನ?- 

\begin{shloka} 
`ಅಹೋ ಗೀತಸ್ಯ ಮಾಧುರ್ಯಂ ಶ್ಲೋಕಾನಾಂ ಚ ವಿಶೇಷತಃ|\\ 
ಚಿರನಿರ್ವತ್ತಮಪ್ಯೇತತ್‍ ಪ್ರತ್ಯಕ್ಷಮಿವದರ್ಶಿತಮ್‍'||
\end{shloka} 

ಮೆಚ್ಚಿಗೆಯನ್ನು ಹೇಗೆ ತೋರಿಸುತ್ತಾರೆ? `ಆಯುಷ್ಯಮಪರೇ ಪ್ರೋಚುಃ'.\label{168g} ಕಥಾನಾಯಕನಾದ ಶ್ರೀರಾಮದೇವರೇ ಇದನ್ನು ಕೇಳುತ್ತಾರೆ. ಲಕ್ಷ್ಮಣ, ಭರತ, ಶತ್ರುಘ್ನರಿಗೂ, `ಕೇಳಿ!' ಎಂದು ಹೇಳುತ್ತಾರೆ. ಏಕೆಂದರೆ- 

\begin{shloka} 
`ಮಮಾಪಿ ತದ್ಭೂತಿಕರಂ ಪ್ರಚಕ್ಷತೇ'|\label{168b}
\end{shloka} 

ಅವನಿಗೇ ಭೂತಿಕರ ಹೇಗೆ? ಎಂದರೆ ಅವನ ಒಂದು, ಧರ್ಮಚರಿತ್ರದ ವಿಷಯವೆಲ್ಲಾ ಅದರಲ್ಲಿರುವುದರಿಂದ ಅವನಿಗೂ ಅದು ಭೂತಿಕರ. ಅದಕ್ಕೇ ಅದನ್ನು ಕೇಳಿ ಆತನೂ ಮೈಮರೆಯುತ್ತಾನೆ- 

\begin{shloka} 
`ಸ ಚಾಪಿ ರಾಮಃ ಪರಿಷದ್ಗತಃ ಶನೈರ್ಬುಭೂಷಯಾಸಕ್ತಮನಾ ಬಭೂವ ಹ'|\label{168cc}
\end{shloka} 

\section*{ನರರನ್ನು ನಾರಾಯಣಪದಕ್ಕೇರಿಸುವ ಧರ್ಮಸೇತು- ರಾಮಾಯಣ} 

ನಾಲ್ಕು ಪುರುಷಾರ್ಥಗಳನ್ನೂ ಕರುಣಿಸುವ ಮಹಾಕಾವ್ಯ, ವೇದಕ್ಕೆ ಸಮಾನವಾದುದು ರಾಮಾಯಣ. 

\begin{shloka} 
`ಯಃ ಕರ್ಣಾಂಜಲಿಸಂಪುಟೈರಹರಹಃ ಸಮ್ಯಕ್ಪಿಬತ್ಯಾದರಾತ್‍\label{168a}\\ 
ವಾಲ್ಮೀಕೇರ್ವದನಾರವಿಂದಗಲಿತಂ ರಾಮಾಯಣಾಖ್ಯಂ ಮಧು|\\ 
ಜನ್ಮವ್ಯಾಧಿಜರಾವಿಪತ್ತಿಮರಣೈರತ್ಯಂತಸೋಪದ್ರವಂ\\ 
ಸಂಸಾರಂ ಸ ವಿಹಾಯ ಗಚ್ಛತಿ ಪುಮಾನ್ವಿಷ್ಣೋಃ ಪದಂ ಶಾಶ್ವತಮ್‍||'\\ 
`ವೇದವೇದ್ಯೇ ಪರೇ ಪುಂಸಿ ಜಾತೇ ದಶರಥಾತ್ಮಜೇ|\label{168c}\\ 
ವೇದಃ ಪ್ರಾಚೇತಸಾದಾಸೀತ್‍ ಸಾಕ್ಷಾದ್ರಾಮಾಯಣಾತ್ಮನಾ'||
\end{shloka} 

ಹೃದಯದ ಆತ್ಮಶ್ರೀಯ ಬೆಳಕಿನಲ್ಲಿ ಕಂಡ ರಾಮಾಯಣಕ್ಕೆ ಲೋಕದ ವಿಷಯವನ್ನೂ ಹೆಣೆದು ಅವ್ಯಕ್ತದೊಡನೆ ವ್ಯಕ್ತವನ್ನೂ ಸೇರಿಸಿ ವ್ಯಕ್ತದಲ್ಲಿರುವವರನ್ನು ಅವ್ಯಕ್ತದಲ್ಲಿ ತೆಗೆದುಕೊಂಡು ಹೋಗಿ ನಿಲ್ಲಿಸಿದ್ದಾರೆ. ನಾರಾಯಣನನ್ನು ನರನನ್ನಾಗಿಳಿಸಿ, ನರನನ್ನು ನಾರಾಯಣನ ಪದದಲ್ಲಿ ಹತ್ತಿಸಿ ನಿಲ್ಲಿಸಲು ಧರ್ಮಸೇತುವನ್ನಾಗಿ ಮಾಡಿದ್ದಾರೆ ಈ ರಾಮಾಯಣವನ್ನು. ಇದರಿಂದ ಬರುವ ಪ್ರಯೋಜನ ಬರೀ ತೆಂಗಿನಕಾಯಿ, ಬಾಳೇಹಣ್ಣು, ಕೊಬ್ಬರಿಸಕ್ಕರೆ, ದಕ್ಷಿಣೆ ಮಾತ್ರವಲ್ಲ, ನಾಲ್ಕು ಪುರುಷಾರ್ಥಗಳೂ. ಏಕೆಂದರೆ ಪರಲೋಕವನ್ನು ಕಂಡ, ಇಹಲೋಕದಲ್ಲಿ ಆ ದಿವ್ಯ ಸಂಸ್ಕೃತಿಯನ್ನು ನಡೆಸುವ ಒಂದು ವ್ಯವಸ್ಥೆಯನ್ನು ಮಾಡುವ, ದಂಡನೀತಿಯುಳ್ಳ ರಾಮರಾಜ್ಯದ ಕಥೆಯಾಗಿದೆ ಅದು. ಇಹ ಪರ ಲೋಕಗಳೆಂಬ ಎರಡು ಬಾಳಾಟಗಳಲ್ಲೂ ಒಂದು ನೆಮ್ಮದಿ. ಅದರಿಂದ ಇದನ್ನು ತೆಗೆದುಕೊಂಡರೆ ಒಂದೆರೆಡು ಶ್ಲೋಕ ಮಾತ್ರ ಹಾಡಿ ಇಂದು ಪಾರಾಯಣ ಆರಂಭಿಸೀಪ್ಪಾ. ಕೈಯಲ್ಲಿಟ್ಟು ಹಾಗೆಯೇ ಮನಸ್ಸಿನಲ್ಲೂ ಅದನ್ನು ಇಡಲು ಭೂಮಿಕೆ. ವಸಂತ ನವರಾತ್ರಿಯಲ್ಲೂ ಹಾಗೇ ಶರತ್ಕಾಲದ ನವರಾತ್ರಿಯಲ್ಲೂ ಇದರ ಪಾರಾಯಣ ಮಾಡುವರು. ವಸಂತಕಾಲದಲ್ಲಿ ಕೋಗಿಲೆಗಳು ಪಂಚಮದಲ್ಲಿ ಅವ್ಯಕ್ತ ಮಧುರವಾಗಿ ಆನಂದದಿಂದ ಗಾನಮಾಡುತ್ತವೆ. ಹಾಗೆ `ರಾಮ ರಾಮ' ಎಂದು ಮಧುರವಾಗಿ ಗಾನಮಾಡುತ್ತಿರುವ ವಾಲ್ಮೀಕಿಕೋಕಿಲವನ್ನು ಸ್ತೋತ್ರ ಮಾಡುತ್ತೇನಪ್ಪಾ. (ಅನಂತರ ಒಂದೆರೆಡು ಶ್ಲೋಕ ಓದಿ- 

\begin{shloka} 
`ಕೂಜಂತಂ ರಾಮರಾಮೇತಿ ಮಧುರಂ ಮಧುರಾಕ್ಷರಮ್‍|\label{169}\\ 
ಆರುಹ್ಯ ಕವಿತಾಶಾಖಾಂ ವಂದೇ ವಾಲ್ಮೀಕಿಕೋಕಿಲಮ್‍'||
\end{shloka} 

ಎಂದು ಸ್ತುತಿಗಾನವನ್ನು ಗುರುಭಗವಂತನು ಮಾಡಿದ ನಂತರ ಶ್ರೀ ಶ್ರೀನಿವಾಸರು `ರಾಮಸ್ತ್ವಂ ಹೃದಯೇ' ಇತ್ಯಾದಿ ಗುರುಸ್ತುತಿಯಿಂದಾರಂಭಿಸಿ ಒಂದೆರೆಡು ಶ್ಲೋಕಗಳ ಪಾರಾಯಣ ಮಾಡಿದರು. ಮಂಗಳ. ಗುರುಭಗವಂತನೂ ಮಂಗಳಾಶಾಸನ ಮಾಡಿ `ರಾಜಾಧಿರಾಜರಾಜಾಯ', `ಅಭ್ಯಷಿಂಚನ್‍ ನರವ್ಯಾಘ್ರಂ'- ವೇಣುನಾದದೊಡನೆ ಮಂಗಳ.) 

`ಹೊಟ್ಟೆಗೆ ತಿಂಡಿ ಕೊಡುವಂತೆಯೇ ಮನಸ್ಸು, ಬುದ್ಧಿ ಆತ್ಮಗಳೂ ಕೆಲವು ಆಹಾರವನ್ನು ಕೇಳುತ್ತೆ. ಅದನ್ನು ಕೊಡುವುದಕ್ಕೆ ಇಲ್ಲಿ ಕುಳಿತಿದ್ದಾಗಿದೆ. ಮತ್ತೆ ಯಾವಯಾವುದು ಆಹಾರ ಕೇಳುತ್ತದೆಯೋ ಅದಕ್ಕೆಲ್ಲಾ ಭಗವಂತನ ಸಂಕಲ್ಪದಂತೆ ಆಹಾರ ಕೊಡಬೇಕು.' 

(ಶ್ರೀ ಶ್ರೀನಿವಾಸರು ಭವತಾರಿಣಿಯಾದ ಶ್ರೀಮಾತೆಯವರಿಗೆ ನಮಸ್ಕರಿಸಿ ಅವರ ಆಶೀರ್ವಾದವನ್ನು ಬೇಡಿದಾಗ `ಬರೀರಾಮರ ಕಥೆಯಲ್ಲ ಇದು. ಸೀತೆಯ ಕಥೆಯೂ ಆಗಿದೆ. ಸೀತಾರಾಮರ ಕಥೆಯೂ ಆಗಿದೆ. ನನ್ನ ಹೃದಯದ ಜ್ಯೋತಿಗೆ ತಗುಲಿಸಿ ಅದನ್ನು ಕೊಡುತ್ತೇನೆ.' ಎಂದು ಅನುಗ್ರಹಿಸಿದರು. ಭರತನಿಗೆ ದಯಪಾಲಿಸಿದಂತೆ ಪಾದುಕೆಗಳೊಡನೆ ದಯಪಾಲಿಸಬೇಕೆಂದು ಭಕ್ತಿಗದ್ಗದರಾಗಿ ಶ್ರೀ ಶ್ರೀನಿವಾಸರು ಬೇಡಲು `ಆಗಲಿಪ್ಪಾ ನಿನ್ನಿಚ್ಛೆಯಂತೆಯೇ ಆಗಲಿ' ಎಂದು ಭಗವಂತನು ಅನುಗ್ರಹಿಸಿದನು.) 
