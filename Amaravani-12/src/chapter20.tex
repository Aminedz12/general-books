\chapter{ಭಾಗವತ ಓದಲು ಹಿನ್ನೆಲೆ}

(ದಿನಾಂಕ ೩-೭-೧೯೬೧ ರಂದು ಶ್ರೀ ಗುರುಭಗವಂತನ ಸುಪುತ್ರರಾದ ಚಿ|| ನಾರಾಯಣ ಅವರು `ಭಾಗವತವನ್ನು ಓದುಬೇಕೆಂದಿದ್ದೇನೆ. ಅದರ ಮೊದಲನೆಯ ಪುಸ್ತಕವನ್ನು ಕೊಡಿ' ಎಂದು ಗುರುಭಗವಂತನನ್ನು ಕೇಳಿದರು. ಆ‌ ಸಂದರ್ಭದಲ್ಲಿ ನಡೆದ ಪಾಠವಿದು.)

\section*{ಋಷಿಸಾಹಿತ್ಯವನ್ನು ಅರ್ಥಮಾಡಿಕೊಳ್ಳಲು ಭೂಮಿಕೆಯ ಅವಶ್ಯಕತೆ}

ಭಾಗವತವನ್ನು ಓದುಬೇಕು ಎನ್ನುವುದು ಸರಿ. ಆದರೆ ಅದಕ್ಕೆ ಒಂದು ಹಿನ್ನೆಲೆ ಬೇಕು. ಅದನ್ನು ಯಾವ ದೃಷ್ಟಿಯಿಂದ ಗ್ರಹಿಸಬೇಕು? ಎನ್ನುವುದನ್ನರಿತು ಉಪಯೋಗಿಸಿದಾಗ ಅದರ ಪ್ರಯೋಜನ ಪಡೆಯಬಹುದು. ಈ ದೃಷ್ಟಿಯಿಂದ ಭಾಗವತದ ಬಗ್ಗೆ ಒಂದು ಭೂಮಿಕೆಯನ್ನು ಕೊಡುತ್ತೇನೆ.

ಯಾವುದೇ ಒಂದು ವಸ್ತುವನ್ನು ಗ್ರಹಿಸಬೇಕಾದರೂ ಒಂದು ದೃಷ್ಟಿಕೋಣಬೇಕು. ನಮ್ಮ ದೃಷ್ಟಿಯಿಂದ ನಾವು ಎಲ್ಲವನ್ನೂ ಗ್ರಹಿಸಲಾಗುವುದಿಲ್ಲ. ಉದಾಹರಣೆಗೆ ಒಂದು ಕೋಳಿ ಆಹಾರವನ್ನು  ತೆಗೆದುಕೊಳ್ಳುವಾಗ, ಆದು ಕಲ್ಲನ್ನು ತೆಗೆದುಕೊಳ್ಳುವುದೇ? ಕಾಳನ್ನು  ತೆಗೆದುಕೊಳ್ಳುವುದೇ? ಎನ್ನುವುದು ಎಷ್ಟು ಸೂಕ್ಷ್ಮವಾಗಿ ಗಮನಿಸಿದರೂ ನಮಗೆ ತಿಳಿಯವುದಿಲ್ಲ. ಅದು ಆಹಾರವಾದ ಕಾಳನ್ನು ಮಾತ್ರ ತೆಗೆದುಕೊಳ್ಳುತ್ತದೆ ಎಂದು ಊಹೆಯಿಂದ ಹೇಳುವುದಾದರೆ, ಹಾಗಿಲ್ಲ. ಕೋಳಿ ಪಾರಿವಾಳ ಮೊದಲಾದ ಹಕ್ಕಿಗಳು ಕಲ್ಲನ್ನೂ ಆಹಾರವಾಗಿ ತೆಗೆದುಕೊಳ್ಳುವುದುಂಟು. ಕೋಳಿಯ ದೃಷ್ಟಿಯು ಬಹು ಸೂಕ್ಷ್ಮ. ಅದು ಆಹಾರವನ್ನು ತೆಗೆದುಕೊಳ್ಳುವ ಕೊಕ್ಕಿನ ಹಿಂಬದಿಯಲ್ಲಿಯೇ ಅದರ ಕಣ್ಣು. ಆದ್ದರಿಂದ ಅದು ಬಹಳ ಸೂಕ್ಷ್ಮವಾಗಿ ಗ್ರಹಿಸಿ ಆಹಾರವನ್ನು ತೆಗೆದುಕೊಳ್ಳಬಹುದು. ನಮ್ಮ ಕಣ್ಣು ಅಷ್ಟು ಸೂಕ್ಷ್ಮವಾಗಿಲ್ಲದಿರುವುದರಿಂದ ಅದನ್ನು ಗ್ರಹಿಸುವುದು ನಮಗೆ ಸಾಧ್ಯವಿಲ್ಲ. 

ಹೀಗೆ ನಮ್ಮ ಕಣ್ಣಿಗೆ ಕಾಣುವ ಅಂಶಗಳ ನೋಟದಿಂದಲೇ ಯಾವುದೇ ನಿಶ್ಚಯಕ್ಕೆ ಬರಲು ಸಾಧ್ಯವಿಲ್ಲ. ಹೀಗಿರುವಾದ ಯಾವುದೋ ಭೂಮಿಕೆಯಲ್ಲಿ ಬರೆದ ಒಂದು ಗ್ರಂಥವನ್ನು-ಭಾಗವತವನ್ನು ಯಾವ ಹಿನ್ನೆಲೆಯೂ ಇಲ್ಲದೆ ನಮ್ಮ ಕೈಗೆ ಕೊಟ್ಟರೆ ಅದರ ಅಕ್ಷರವನ್ನು ನಾವು ಓದಬಹುದಾದರೂ ಅದನ್ನು ಅರ್ಥಮಾಡಿಕೊಳ್ಳುವುದು ಸಾಧ್ಯವಿಲ್ಲ. ಆದ್ದರಿಂದ ಅದು ಅರ್ಥವಾಗಬೇಕಾದರೆ ಅದಕ್ಕೆ ಬೇಕಾದ ಒಂದು ಭೂಮಿಕೆಯ ತಯಾರಾಬೇಕು. 


\section*{ವಸ್ತುವಿನ ಪೂರ್ಣಪರಿಚವಿಲ್ಲದಿದ್ದಾಗ ಆಗುವ ಅನರ್ಥ}

ಸುಮ್ಮನೆ ಒಂದು ವಸ್ತುವನ್ನು ಕೊಟ್ಟು ಅದರ ಉಪಯೋಗವನ್ನೂ, ಉಪಯೋಗ ಪಡೆಯುವ ವಿಧಾನವನ್ನೂ ತಿಳಿಸದಿದ್ದರೆ ಆ ವಸ್ತುವಿನ ಪ್ರಯೋಜನ ಪಡೆಯುವುದು ಸಾಧ್ಯವಿಲ್ಲ. ಉದಾಹರಾಣೆಗೆ ಒಂದು ಹಲಸಿನ ಹಣ್ಣನ್ನು ಯಾವ ಹಿನ್ನೆಲೆಯೂ ಇಲ್ಲದೆ ಕೊಟ್ಟರೆ, ಅದರ ಮೇಲುಗಡೆ ನೆಕ್ಕಿನಲ್ಲಿ ರಸ ಸಿಕ್ಕುತ್ತದೆಯೇ? ಇದು ಪರಸ್ಥಳದ ಪಾದ್ರುಯು ಪನಸ (ಹಲಸಿನ ಹಣ್ಣು) ತಿಂದ ಕಥೆಯಂತಾಗುತ್ತದೆ. ಕಾಶಿಯಲ್ಲೊಬ್ಬ ಪಾದ್ರಿ ಇದ್ದ. ಅವನು ಹಲಸಿನ ಹಣ್ಣನ್ನು ನೋಡಿ, ಅದು ತಿನ್ನುವ ಪದಾರ್ಥವೆಂದು ತಿಳಿದು, ಅದನ್ನು ಕೊಂಡುಕೊಂಡು ಬಂದ. ಹಲಸಿನ ಹಣ್ಣು ತಿನ್ನುವ ಪದಾರ್ಥವೆಂದು ಗೊತ್ತಿತ್ತೇ ವಿನಹ, ಅದನ್ನು ಹೇಗೆ ಉಪಯೋಗಿಸಬೇಕು? ಅದರಲ್ಲಿ ತಿನ್ನಬೇಕಾದದ್ದು ಯಾವುದು? ಎನ್ನುವುದು ಅವನಿಗೆ ತಿಳಿಯದು. ನಂತರ ಮನೆಗೆ ಬಂದು ಅದನ್ನು ಕೊಯ್ದ. ಕೊಯ್ಯುವಾಗ ಅದರಲ್ಲಿ ಬರುವ ಹಾಲನ್ನು ನೋಡಿ, ಇದೇ ಹಣ್ಣಿನ ಸಾರವಾದ ಪದಾರ್ಥ ಎಂದುಕೊಂಡು, ಆ ಹಾಲನ್ನು ಸ್ವಲ್ಪವೂ ವ್ಯರ್ಥವಾಗದಂತೆ ಉಪಾಯವಾಗಿ ಕೊಯ್ದು . ಹಲಸಿನ ಹಣ್ಣು ಸ್ವಲ್ಪ ಕಾಯಾಗಿದ್ದುದರಿಂದ ಸ್ವಲ್ಪ ಹೆಚ್ಚಾಗಿಯೇ ಹಾಲು ಇತ್ತು. ಆ ಹಾಲನ್ನು ಹಾಗೆಯೇ ಕುಡಿದು ಬಿಡುತ್ತೇನೆ ಎಂದು ಹಣ್ಣನ್ನು ಮೇಲಕ್ಕೆತ್ತಿದ. ಅದರ ರದವು ಗಡ್ಡಕ್ಕೆಲ್ಲಾ ಅಂಟಿಕೊಂಡು ಬಿಟ್ಟಿತು. ನಾಲಿಗೆಯೆಲ್ಲಾ ಅಂಟಾಯಿತು. ನಂತರ ಅದನ್ನು ತೊಳುಯುವ ವಿಧಾನ ತಿಳಿಯದೆ ತಣ್ಣೀರಿನಲ್ಲಿ ತೊಳೆಯಲು ಶುರು ಮಾಡಿದ. ಗಡ್ಡವೆಲ್ಲಾ ಜಿಡ್ಡಾಗಿ ಮತ್ತಷ್ಟು ಅಂಟಿಕೊಂಡಿತು. ಉಪಾಯವಿಲ್ಲದೆ ಗಡ್ಡವನ್ನೆಲ್ಲಾ ಬೋಳಿಸಿಬಿಟ್ಟ. ದಾರಿಯಲ್ಲಿ ಹೋಗುತ್ತಿರುವಾಗ ಕೆಲವರು ಕ್ಷೌರ ಮಾಡಿಸಿಕೊಂಡು ಗಂಗೆಯಲ್ಲಿ ಸ್ನಾನ ಮಾಡಿ ಬರುತ್ತಿದ್ದರು. ಆಗ ಆ ಪಾದ್ರಿಯ `ನೀವೂ ಹಲಸಿನ ಹಣ್ಣು ತಿಂದಿರಾ?' ಎಂದು ಎಲ್ಲರನ್ನೂ ಕೇಳಿದ. ಕೆಲವರು ಹಲಸಿನ ಹಣ್ಣನ್ನು ತಿಂದಿದ್ದವರಿದ್ದು ಇವನಿಗೆ ಹೇಗೆ ಗೊತ್ತಾಯಿತು ನಾವು ಹಲಸಿನ ಹಣ್ಣು ತಿಂದದ್ದು? ಇರಲಿ, ನಮಗೇನು? ಎಂದು ಕೊಂಡು `ಹೂಂ' ಎಂದು ಬಿಟ್ಟರು. ಹಲಸಿನ ಹಣ್ಣನ್ನು ತಿನ್ನದೇ ಇದ್ದ ಕೆಲವರು `ಇದೇನು ನೀವು ಕೇಳುತ್ತಿರುವುದು?' ಎಂದರು. `ಕ್ಷೌರ ಮಾಡಿಸಿಕೊಂಡಿದ್ದೀರಲ್ಲಾ, ಅದಕ್ಕೆ ಹಲಸಿನ ಹಣ್ಣು ತಿಂದಿರಾ? ಎಂದು ಕೇಳಿದೆ' ಎಂದ. ಅವರಿಗೆ ಮತ್ತೂ ಆಶ್ಚರ್ಯ ವಾಯಿತು - ಕ್ಷೌರ ಮಾಡಿಸಿ ಕೊಳ್ಳುವುದಕ್ಕೂ ಹಲಸಿನ ಹಣ್ಣು ತಿನ್ನುವುದಕ್ಕೂ ಏನು ಸಂಬಂಧ? ಎಂದು. ಅವರಿಗೆ ಹೇಗೆ ತಿಳಿಯಬೇಕು ಇವನ ಕಥೆ? ಅನಂತರ ವಿಚಾರಿಸಲಾಗಿ ತಾನು ಹಲಸಿನ ಹಣ್ಣು ತಿಂದ ರೀತಿ ತಿಳಿಸಿದ. ಆ ಮೇಲೆ ಅವರು ಹಲಸಿನ ಹಣ್ಣನ್ನು ತಿನ್ನುವುದು ಹೇಗೆ ಎಂಬುದನ್ನು ಅವನಿಗೆ ತಿಳಿಯಪಡಿಸಿದರು. ವಸ್ತುವಿನ ಪರಿಚಯವಿಲ್ಲದೇ ಉಪಯೋಗಿಸಲು ಹೊರಟಾಗ ಎಲ್ಲರಿಗೂ ಆಗುವುದು ಹೇಗೆಯೇ.


\section*{ವಸ್ತುವಿನ ಸ್ವರೂಪ, ಉಪಯೋಗಗಳ ಅರಿವಿದ್ದಾಗ ಮಾತ್ರ  ಪ್ರಯೋಜನಲಾಭ}

ಆದ್ದರಿಂದ ಯಾವುದೇ ಒಂದು ವಸ್ತುವನ್ನು ಕೊಡಬೇಕಾದರೂ, ಅದರ ಉಪಯೋಗವನ್ನು ಉಪಯೋಗಿಸುವುದಕ್ಕೆ ಬೇಕಾದ ಹಿನ್ನೆಲೆಯನ್ನೂ ತಿಳಿಸಿಕೊಡಬೇಕು. ಹಲಸಿನ ಹಣ್ಣಿನಲ್ಲಿ ಸಿಪ್ಪೆ, ಅಂಟು ಎಲ್ಲವನ್ನೂ ತೆಗೆದು ತೊಳೆಯನ್ನು ತಾನೇ ಆಸ್ವಾದಿಸಬೇಕು. ಹೀಗೆಯೇ ಭಾಗವತದ ಆಸ್ವಾದನೆ ಹೇಗೆ? ಅದನ್ನು ಯಾರು ಆಸ್ವಾದಿಸಬೇಕು? ಆಸ್ವಾದನೆಗೆ ಬೇಕಾದ ಹಿನ್ನೆಲೆಯೇನು? ಎನ್ನುವ ವಿಚಾರ ಬರುತ್ತದೆ. 

ಭಾಗವತ ಆಸ್ವಾದನೆಯು ಹೇಗೆ? ಎಂದರಿಯಬೇಕಾದರೆ ಅದರ ಸ್ವರೂಪವೇನು? ಎನ್ನುವುದನ್ನು ಗಮನಿಸಬೇಕಾಗುವುದು. ಅದರ ಸ್ವರೂಪ, ಉಪಯೋಗಗಳನ್ನು ಕೊಡದೇ ವಸ್ತುವನ್ನು ಮಾತ್ರ ಕೊಟ್ಟರೆ ಇವನ ಮನಸ್ಸಿಗನುಗುಣವಾಗಿ ವಸ್ತುವನ್ನು ಉಪಯೋಗಿಸಿಬಿಡುತ್ತಾನೆ. ರೇಡಿಯೋ ಒಂದನ್ನು ಉಪಯೋಗಿಸೌವ ಕ್ರಮವನ್ನು ತಿಳಿಸದೇ ಒಬ್ಬನಿಗೆ ಕೊಟ್ಟರೆ ಆದರಲ್ಲಿರುವ ಬ್ಯಾಂಡನ್ನು ಹೇಗೆ ಹೇಗೋ ತಿರುಗಿಸಿ ಯಂತ್ರವನ್ನು ಉಪಯೋಗವಿಲ್ಲದಂತೆ ಮಾಡಿಬಿಡುತ್ತಾನೆ. ಆದ್ದರಿಂದ ರೇಡಿಯೋ ಕೊಡುವಾಗ ಅದರ ಉಪಯೋಗಿಸುವ ವಿಧಾನವನ್ನೂ ತಿಳಿಸಿಕೊಡಬೇಕು. 

ಹೀಗೆಯೇ ಒಂದು ದೊಡ್ಡ ವಿಮಾನವನ್ನು ಕೊಟ್ಟುಬಿಟ್ಟರೆ, ಅದನ್ನು ಹೇಗೆ ಉಪಯೋಗಿಸುವುದು? ಎಂಬ ಪ್ರಶ್ನೆಯು ಮೂಡುತ್ತದೆ. ಆದ್ದರಿಂದ ಅದನ್ನು ಕೊಡುವಾಗ ಅದನ್ನು ಉಪಯೋಗಿಸುವ ಅಥವಾ ನಡೆಸುವ ರೀತಿಯನ್ನು ಹೇಳಿಕೊಡಬೇಕು. ಹಾಗಾದರೆ ತಾನೇ ಅದರಿಂದ ಉಪಯೋಗ ಪಡೆಯಲು ಸಾಧ್ಯ. ನಡೆಸುವುದರ ಜೊತೆಗೆ ಅದರ ಮೆಕ್ಯಾನಿಸಂ ಅನ್ನೂ ತಿಳಿಸಿದರೆ ಇನ್ನೂ ವಿಶೇಷ. ಆಗ ಅದು ಕೆಟ್ಟುಹೋಗದಂತೆ ನೋಡಿಕೊಂದು, ಆಕಸ್ಮಿಕವಾಗಿ ಕೆಟ್ಟರೆ ಅದನ್ನು ಸರಿಪಡಿಸಿಕೊಳ್ಳುವ ಯೋಗ್ಯತೆಯೂ ಬರುತ್ತದೆ. ಹೀಗೆ ಯಾವುದೇ ವಸ್ತುವನ್ನು ಕೊಡಬೇಕಾದರೂ ಅದರ ಸ್ವರೂಪ, ಉಪಯೋಗಗಳನ್ನು ತಿಳಿಸಿಕೊಟ್ಟಾಗಲೇ ಅದರಿಂದ ಸರಿಯಾದ ಪ್ರಯೋಜವನ್ನು ಪಡೆಯುಬಹುದು. 

\section*{ಋಷಿಸಾಹಿತ್ಯವನ್ನು ಮರ್ಮಜ್ಞರ ಉಪದೇಶದಿಂದ ಪಡೆದುಕೊಳ್ಳಬೇಕು}

ಹೀಗೆ ಮಹರ್ಷಿಗಳು ತಂದ ಶಾಸ್ತ್ರ, ಕಾವ್ಯ, ರಾಜಕೀಯ, ಭಗವದ್ಭಾಗವತರ ಗ್ರಂಥಗಳನ್ನು ಸುಮ್ಮನೆ ಹೊರಗಿಟ್ಟರೆ ಅಥವಾ ಕೊಟ್ಟರೆ ತಮ್ಮ ತಮ್ಮ ಮನಸ್ಸಿಗನುಗುಣವಾಗಿಯೇ ತೆಗೆದುಕೊಳ್ಳುತ್ತಾರೆ. ಮಹರ್ಷಿಗಳ ಮನಸ್ಸು ಅವರಿಗೆ ತಿಳಿಯಲಾರದು. ಆ ಗ್ರಂಥದ ಪ್ರಯೋಜನವೂ ದೊರೆಯುವುದಿಲ್ಲ. ಆದ್ದರಿಂದಲೇ ಉತ್ತಮ ಮನಸ್ಸಿನಿಂದ ಹೊರಟ ಗ್ರಂಥಗಳನ್ನು ಸುಮ್ಮನೆ ಓದಲು ಕೊಡದೆ ಸರಿಯಾದ ಭೂಮಿಕೆಯನ್ನು ಕೊಡುವುದಕ್ಕಾಗಿ ಉಪದೇಶ ಪರಂಪರೆಯಿಂದ ತಂದರು. ಈ ಉಪದೇಶವನ್ನು ಅದರ ಗುಟ್ಟು ತಿಳಿದವರಿಂದ ಪಡೆಯಬೇಕು. ಇಲ್ಲದಿದ್ದರೆ ಅದರ ಮರ್ಮವು ತಿಳಿಯುವುದಿಲ್ಲ. ಮಹರ್ಷಿಗಳ ತಮ್ಮ ಮನಸ್ಸಿನ ಮಟ್ಟಕ್ಕೆ ಏರಿದಾಗ ಕಂಡ ಸತ್ಯವನ್ನು ಸಾಹಿತ್ಯರೂಪದಲ್ಲಿ ನೀಡಿದ್ದಾರೆ. ಅದನ್ನು ಆ ಮಟ್ಟಕ್ಕೇರದೇ ಸಾಮಾನ್ಯ ಜನಗಳು ತಮ್ಮ ಮನಸ್ಸಿನ ಮಟ್ಟದಿಂದ ನೋಡಿದಾಗ ಅದರ ಆಸ್ವಾದನೆಯು ದೊರೆಯುವುದೆ ಹೇಗೆ? ಕೇವಲ ಅಕ್ಷರರಾಶಿಗಳನ್ನು ನೋಡಿ ಅದರ ಮರ್ಮ ತಿಳಿಯದಿದ್ದರೆ ಗ್ರಂಥವು ಉಪಯೋಗವಿಲ್ಲ ಎಂದು ಬಿಡುವುದೂ ಉಂಟು. ಆದರೆ ಅದರ ಮರ್ಮಜ್ಞರಿಂದ ಉಪದೇಶ ಪಡೆದಲ್ಲಿ ಗ್ರಂಥದ ಪ್ರಯೋಜನಕ್ಕೂ, ಆಸ್ವಾದನೆಗೂ ನಾವು ಯೋಗ್ಯರಾಗುತ್ತೇವೆ. 

ಭೌತಿಕವಾಗಿಯೇ ಹೇಳುವುದಾದರೂ ಯಾವುದಾದರೂ ಯಂತ್ರವನ್ನು ರಚಿಸಿದಾಗ ಅಥವಾ ಯಂತ್ರದಿಂದ ಪ್ರಯೋಜನ ಪಡೆಯಬೇಕಾದಾಗ ಅದರ ಮರ್ಮಜ್ಞದಿಂದ ಅದರ ಉಪಯೋಗ ಮತ್ತು ಉಪಯೋಗಿಸುವ ವಿಧಾನಗಳನ್ನು ತಿಳಿದುಕೊಳ್ಳಬೇಕಾಗುವುದು. ಇಲ್ಲದಿದ್ದರೆ ಯಂತ್ರದ ಉಪಯೋಗ ಪಡೆಯದೆ ಯಂತ್ರವನ್ನು ಕೆಡಿಸುವುದುದಾಗುತ್ತದೆ. ಹೀಗೆ ಯಾವುದನ್ನೇ ಉಪಯೋಗಿಸಬೇಕಾದರೂ ಅದರದರ ಮರ್ಮಜ್ಞರಿಂದ ಉಪದೇಶ ಅಥವಾ ಯೋಗ್ಯ ತಿಳುವಳಿಕೆ ಅತ್ಯಗತ್ಯ ಎನ್ನುವುದು ಗಮನಿಸಬೇಕಾದ ಅಂಶ. 

\section*{ಮಹರ್ಷಿಗಳು ತಂದ ಸಾಹಿತ್ಯದ ಔನ್ನತ್ಯಕ್ಕೇರಲು ಯೋಗ್ಯ ತರಬೇತಿ ಬೇಕು}

ಮಹರ್ಷಿಗಳು ಹೊರತಂದ ಶಾಸ್ತ್ರ, ಸಾಹಿತ್ಯ, ರಾಜಕೀಯ ಮೊದಲಾದ ಎಲ್ಲಾ ವಿಚಾರಗಳೂ ದೈವೀಪ್ರಪಂಚಕ್ಕೆ, ಆಧ್ಯಾತ್ಮಪ್ರಪಂಚಕ್ಕೆ ನಮ್ಮನ್ನು ಕೊಂಡೊಯ್ಯುವ ಸಾಧನಗಳಾಗಿವೆ. ಇಂತಹ ಒಂದು ಮಹತ್ತರ ಸಾಧನವನ್ನು ಉಪಯೋಗಿಸಿಕೊಳ್ಳಲು ಯೋಗ್ಯ ತರಬೇತಿ ಬೇಕು. ರಾಕೆಟ್ಟಿನಿಂದ ಬಾಹ್ಯಾಂತರಿಕ್ಷಕ್ಕೆ ಹೋಗುವವನು, ಬಾಹ್ಯಾಂತರಿಕ್ಷದ ಸನ್ನಿವೇಶಗಳನ್ನು ಗಮನಿಸಿ ಅಲಿನ ಬದಲಾವಣೆಗಳಿಂದ ತೊಂದರೆಯಾಗದಂತೆ ಮೊದಲೇ ಆಮ್ಲಜನಕ ಹಾಗೂ ಇತರ ಉಪಕರಣಗಳನ್ನು ತೆಗೆದುಕೊಂಡು ಪೂರ್ವಸಿದ್ಧತೆಯೊಡನೆ ಹೋಗಬೇಕು. ಹಾಗಲ್ಲದೆ ರಾಕೆಟ್ ಇದೆ, ಅದು ಹಾರುತ್ತದೆ, ಎಂದು ಅದರಲ್ಲಿ ಕುಳಿತು ಹಾರಿಬಿಟ್ಟರೆ ಮುಗಿಯಿತು ಅವನ ಕಥೆ. ನಿರ್ದಿಷ್ಟ ಗುರಿಯನ್ನು ತಲುಪುವ ಯೋಚನೆಯಂತೂ ಇಲ್ಲವೇ ಇಲ್ಲ. ಹೇಗೆ ಇಲ್ಲಿ ಮೇಲಕ್ಕೆ ಹಾರಲು ಬೇಕಾದ ರಾಕೆಟ್ಟಿನ ಸಾಧನವಿದ್ದರೂ ಉಪಯೋಗದ ಬೆಗ್ಗೆ ತರಬೇತಿ ಪಡೆಯದೆ ಅದನನ್ನು ಉಪಯೋಗಿಸುವುದು ಸಾಧ್ಯವಿಲ್ಲವೋ ಹಾಗೆಯೇ ನಮಗೆ ಮಹರ್ಷಿಗಳ ಸಾಹಿತ್ಯವು ಸಿಕ್ಕಿದರೂ, ಅದರ ಅಕ್ಷರಗಳನ್ನು ನಾವು ಓದುಬಹುದಾದರೂ ಅದನ್ನು ಗಮನಿಸಲು ಬೇಕಾದ ತರಬೇತಿಯಿಲ್ಲದೆ ಉಪಯೋಗವನ್ನು ಪಡೆಯಲು ಸಾಧ್ಯವಿಲ್ಲ. ಹೇಗೆ ರಾಕೆಟ್ಟಿನಲ್ಲಿ ಸರಿಯಾದ ಸಿದ್ಧತೆಯಿದ್ದಾಗ ನಮ್ಮ ಶರೀರವು ಬಾಹ್ಯಾಂತರಿಕ್ಷವನ್ನು ಹತ್ತಬಲ್ಲದೋ, ಹಾಗೆಯೇ ಗ್ರಂಥದ ಹಿರಿವೆಯು ಸರಿಯಾಗಿ ಮನಸ್ಸಿಗೆ ಬಂದಿದ್ದು, ಸರಿಯಾದ ಹಿನ್ನೆಲೆಯೊಡನೆ ಓದಲು ಆರಂಭಿಸಿದಾಗ ನಮ್ಮ ಮನಸ್ಸು ಮತ್ತು ಬುದ್ಧಿಗಳು ಆ ಗ್ರಂಥದ ಔನ್ನತ್ಯಕ್ಕೆ ಏರತೊಡಗುತ್ತವೆ. ಮನಸ್ಸು ಮತ್ತು ಬುದ್ದಿಗಳನ್ನು ತನ್ನ ಮಟ್ಟಕ್ಕೆಳೆದುಕೊಂಡು ನಮ್ಮನ್ನು  ಪಾವನ ಮಡುವ ಗ್ರಂಥವಿದು ಎಂದು ಗೊತ್ತಾಗುವುದು ಸರಿಯಾದ ತರಬೇತಿ ಪಡೆದಾಗಲೇ. 

\section*{ಆಧ್ಯಾತ್ಮಿಕ ವಿಷಯಗಳನ್ನು ಬುದ್ಧಿ ನೋಡುವಂತಾಗಲು ಮರ್ಮಜ್ಞರ ಉಪದೇಶ ಬೇಕು}

ನಮ್ಮ ಬುದ್ಧಿಗೆ ತರಬೇತಿ ಕೊಡುವುದು ಹೇಗೆ? ಎಂದರೆ ಹೊರಗಡೆಯಲ್ಲಿ ನಮ್ಮ ದೃಷ್ಟಿ ಪ್ರಸಾರವಾಗದಿರುವ ಅಥವಾ ನಮ್ಮ ದೃಷ್ಟಿಯಿಂದ ಗ್ರಹಿಸಲಾಗದ ಸೂಕ್ಷ್ಮಾಂಶಗಳನ್ನು ಮೈಕ್ರೋಸ್ಕೋಪಿನ ಸಹಾಯದಿಂದ ನೋಡಬಹುದು. ನಮ್ಮ ದೃಷ್ಟಿಯನ್ನೆಲ್ಲವನ್ನೂ ಮೈಕ್ರೋಸ್ಕೋಪಿನ ಲೆನ್ಸಿನೊಡನೆ ಫೋಕಸ್ ಮಾಡಿಕೊಂಡು ನೋಡಿದಾಗ ತಾನೇ ನಮಗೆ ವಸ್ತುವು ಕಾಣುತ್ತದೆ. ಹೊರವಸ್ತುವನ್ನು ನೋಡಲು ಒಂದು ಲೆನ್ಸ್ ಇರುವುಂತೆಯೇ ಒಳಗಡೆಯೂ ಒಂದು ಲೆನ್ಸ್ - ಬುದ್ಧಿಯೆಂಬುದು ಇದೆ. ಹೊರಗಡೆಯ ವಿಚಾರಗಳನ್ನು ಇಂದ್ರಿಯಗಳ ಮೂಲಕ ನೋಡುವಂತೆ, ಒಳಗಿನ ಆಧ್ಯಾತ್ಮಿಕ ವಿಷಯಗಳನ್ನು ಬುದ್ಧಿಯಿಂದ ನೋಡಬಹುದು. ಆದರೆ ಇಂದು ನಮ್ಮ ಬುದ್ಧಿಯಿಂದ ಆಧ್ಯಾತ್ಮಿಕ ವಿಚಾರಗಳನ್ನು ಗ್ರಹಿಸಲು ಸಾಧ್ಯವಾಗುತ್ತಿಲ್ಲ. ಅದರ ಕಾರಣ ಇಂದು ನಮ್ಮ ಬುದ್ದಿಯು ಕೇವಲ ಗ್ಲಾಸ್ ಆಗಿದೆ. ಗ್ಲಾಸಿನಿಂದ ಸೂಕ್ಷ್ಮ ದರ್ಶನ ಮಾಡುವುದು ಸಾಧ್ಯವಾಗುವುದಿಲ್ಲ. ಆ ಬುದ್ಧಿಯನ್ನು ಲೆನ್ಸಾನ್ನಾಗಿ ಮಾಡಿಕೊಳ್ಳಬೇಕು, ಹಾಗೂ ಅದರ ಮೂಲಕ ನೋಡುವುದಕ್ಕೆ ಪ್ರಯತ್ನ ಪಡಬೇಕು. ಹಾಗೆ ಅದಕ್ಕೆ ಪವರ್ (ಸೂಕ್ಷ್ಮವಾದ ಆಧ್ಯಾತ್ಮಿಕ ವಿಷಯಗಳನ್ನು ಗ್ರಹಿಸುವಂತಹ ಶಕ್ತಿ) ಕೊಡಬೇಕಾದರೆ ಮರ್ಮಜ್ಞರ ಉಪದೇಶ ಬೇಕು. 

\section*{ಉಪದೇಶದಿಂದ ಸಂಸ್ಕಾರಗೊಂಡ ಬುದ್ಧಿಯಿಂದ ಮಾತ್ರ ಗ್ರಂಥದ ಆಸ್ವಾದನೆ}

ಉಪದೇಶಕರು ಶಿಷ್ಯನ ಬುದ್ಧಿಯನ್ನನುಸರಿಸಿ, ಪರೀಕ್ಷಿಸಿ, ಬೌದ್ಧಿಕವಾದ ದೋಷವನ್ನು ಹೋಗಲಾಡಿಸಿ ನಿರ್ಮಲವನ್ನಾಗಿ ಮಾಡಿ ಒಳ್ಳೆಯ ಹಿನ್ನೆಲೆಯನ್ನು ಕೊಡುತ್ತಾರೆ. ಆ ಹಿನ್ನೆಲೆಯನ್ನು ಸರಿಯಾಗಿ ಗ್ರಹಿಸಿದಾಗ ಬುದ್ಧಿಯಲ್ಲಿ ಚೈತನ್ಯವು ತುಂಬಿಕೊಳ್ಳುತ್ತದೆ. ಅಂತಹ ಚೈತನ್ಯ ಸಂಪನ್ನವಾದ ಬುದ್ಧಿಯು ಗ್ರಂಥವನ್ನು ಸರಿಯಾಗಿ ನೋಡಲು ಹಾಗೂ ಅರ್ಥಮಾಡಿಕೊಳ್ಳಲು ಶಕ್ತವಾಗುತ್ತದೆ. 

ಬುದ್ಧಿಗೆ ಸಂಸ್ಕಾರವಿಲ್ಲದೆ ಗ್ರಂಥವನ್ನು ಕೊಟ್ಟಲ್ಲಿ ಹಲಸಿನ ಹಣ್ಣನ್ನೂ ಒಂದು ಮೊಂಡುಚಾಕುವನ್ನೂ ಕೊಟ್ಟಂತಾಗುತ್ತದೆ. ಆ ಚಾಕುವು ಹಣ್ಣನ್ನು ಕೊಯ್ಯುವಂತಿಲ್ಲ ಇವನು ಹಣ್ಣನ್ನು ತಿನ್ನುವಂತಿಲ್ಲ. ಕೊಟ್ಟವರು ಮಾತ್ರ ಹಣ್ಣು-ಚಾಕು ಎರಡನ್ನೂ ಕೊಟ್ಟಂತಾಗುತ್ತದೆ ಅಷ್ಟೇ. ಅದೇ ಚಾಕುವು ತೀಕ್ಷ್ಣವಾಗಿದ್ದರೆ ಕೊಯ್ದು ಆಸ್ವಾದಿಸಬಹುದು. ಹೀಗೆಯೇ ಬುದ್ಧಿಯು ಮೊಂಡಾಗಿರುವಾಗ ಅದಕ್ಕೆ ಇಂತಹ ಗ್ರಂಥಗಳನ್ನು ಕೊಟ್ಟರೆ ಅದರ ಒಳಗೆ ಪ್ರವೇಶಿಸಲಾರದೇ ಆಸ್ವಾದಿಸಲಾರದೇ ಎಸೆದು ಬಿಡುತ್ತದೆ. ಆದರೆ ಅದೇ ಬುದ್ಧಿಯನ್ನು ಉಪದೇಶಿದಿಂದ ಹರಿತಮಾಡಿಕೊಟ್ಟಾಗ ಸುಖವಾಗಿ ಆ ಗ್ರಂಥದ ಹಿರಿಮೆಯನ್ನು ಆಸ್ವಾದಿಸಲು ಯೋಗ್ಯವಾಗುತ್ತದೆ. ಈ ಆಸ್ವಾದನೆಯ ಯೋಗ್ಯತೆಯನ್ನು ತರಲೋಸುಗವೇ ಭೂಮಿಕೆಯು ಅವಶ್ಯವಾಗುತ್ತದೆ.

\section*{ಸ್ಥಾನಭೇದದಿಂದ ಒಂದೇ ವಸ್ತು ವಿವಿಧವಾಗಿ ಕಾಣುವುದುಂಟು}

ಲೋಕದಲ್ಲಿ ವಿವಿಧ ದೃಷ್ಟಿಗಳುಂಟು. ಒಂದೇ ವಸ್ತುವನ್ನು ವಿವಿಧ ದೃಷ್ಟಿಗಳಿಂದ ನೋಡುವುದೂ ಉಂಟು. ಹೀಗೆಯೇ ಒಂದೇ ವಸ್ತುವು ಸ್ಥಾನಭೇದದಿಂದ ವಿವಿಧ ರೀತಿಯಲ್ಲಿ ಕಾಣುವುದೂ ಉಂಟು.ಉದಾಹರಣೆಗೆ ತೆಗೆದುಕೊಳ್ಳೊಣ-

(ಒಂದು ಬೆರಳನ್ನು ತೋರಿಸಿ) ಇದೆಷ್ಟು? (ಒಂದು ಎಂದು ಉತ್ತರ) (ಆ ಬೆರಳನ್ನು ಮೂಗಿನ ನೇರಕ್ಕೆ ಹಿಡಿದು) ಈಗ ನೋಡಿ, ಇದೆಷ್ಟು? (ಎರಡು ಎಂದು ಉತ್ತರ)ಇಲ್ಲಿ ದೃಷ್ಟಿಯು ವಸ್ತುವನ್ನು ಎರಡಾಗಿ ಕಣುವಂತೆ ಮಾಡುತ್ತದೆ ಮಧ್ಯದಲ್ಲಿ ಮೂಗಿನ ನೇರಕ್ಕೆ ಬೆರಳಿಟ್ಟರೆ ಎರಟು ತಿಳಿಯುತ್ತೆ. ಆದರೆ ಬಲ ಅಥವಾ ಎಡ ಭಾಗದಲ್ಲಿ ಬೆರಳಿಟ್ಟಾಗ ಎರಡು ದೃಷ್ಟಿಗಳೂ ಒಂದೇ ಕಡೆಗೆ ವಾಲಿ ಬೆರಳು ಒಂದು ಎಂದು ತಿಳಿಯುತ್ತೆದೆ. ಹೀಗೆಯೇ ಒಂದೇ ಬೆರಳನ್ನು ವೇಗವಾದ ಅಲ್ಲಾಡಿಸಿದಾಗಲೂ ಅನೇಕ ಬೆರಳುಗಳಿದ್ದಂತೆ ತೋರಿಸುತ್ತದೆ. ಇಲ್ಲಿ ವೇಗವು ದೃಷ್ಟಿಯ ಪೃಥಕ್ಕರಣವನ್ನು ಮಾಡಿ ಒಂದು ವಸ್ತುವನ್ನು ಆನೇಕ ವಸ್ತುವಾಗಿ ತೋರುವಂತೆ ಮಾಡುತ್ತದೆ. ಹೀಗೆ ವೇಗ ಹಾಗೂ ವಿಶೇಷವಿನ್ಯಾಸಗಳು ವಸ್ತುವು ಅನೇಕವಾಗಿ ಕಾಣಲು ಕಾರಣವಾಗುತ್ತದೆ. 

\section*{ಸಾಹಿತ್ಯದ ಗಮ್ಯವನ್ನು ತಲುಪಲು ಮಹರ್ಷಿದೃಷ್ಟಿ ಬೇಕು}


ಹೀಗಿರಲು ಮಹರ್ಷಿಗಳ ಸಾಹಿತ್ಯವನ್ನು ನಮ್ಮ ದೃಷ್ಟಿಯಿಂದ ನೋಡಿದಾಗ ನಮಗೆ ಅರ್ಥವಾಗುದು ಹೇಗೆ? ಆದ್ದರಿಂದ ಮಹರ್ಷಿಗಳ ದೃಷ್ಟಿಯೊಂದೆ ನಮ್ಮನ್ನೆ ಮಹರ್ಷಿಗಳ ಪಥಕ್ಕೆ ಒಯ್ಯುತ್ತದೆ. ಸಂಪೂರ್ಣ ಸಿದ್ಧತೆಗಳೊಡನೆ ಒಂದು ರಾಕೆಟ್ಟು ಮೇಲೇರಿದಾಗ ತನ್ನ ರೂಟಿಗೆ ಅಥವಾ ಪಥೆಕ್ಕೆ ಕೊಂಡೊಯಯ್ದು ಗಮ್ಯಸ್ಥಾನವನ್ನು ತಲುಪಿಸಿ ಅಲ್ಲಿನ ಸನ್ನಿವೇಶವನ್ನು ತೋರಿಸುವಂತೆಯೇ ಇಲ್ಲಿಯೂ ಮಹರ್ಷಿಗಳ ಪಥಕ್ಕೆ ನಮ್ಮನ್ನು ಒಯ್ದು ಆ ಸಾಹಿತ್ಯದ ಮೂಲವೇನಿದೆಯೋ ಅಥವಾ ಅದರ ಗಮ್ಯಸ್ಥಾನವೇನುಂಟೋ ಅಲ್ಲಿಗೆ ತಲುಪಿಸಬಲ್ಲುದು. 

\section*{ಬುದ್ಧಿಯು ಯೋಗ್ಯಭೂಮಿಯಾದಾಗಲೇ ವಿಷಯ್ದ ಅರಿವು}

ಹೀಗೆ ಮಹರ್ಷಿಗಳ ದೃಷ್ಟಿಯೊಡನೆ ಫೋಕಸ್ ಮಾಡಿಕೊಳ್ಳುವುದಕ್ಕಾಗಿ ತಕ್ಕ ಭೂಮಿಕೆ ಬೇಕು. ನಾವು ನಿಲ್ಲಬೇಕಾದ ಒಂದು ಭೂಮಿ ಬೇಕು. ಯಾವುದೇ ಒಂದು ವಸ್ತು ನಿಲ್ಲಬೇಕಾದರೂ ಭೂಮಿ ಬೇಕು. ಇಲ್ಲಿ ಕನ್ನಡಕವಿದೆ. ಇದನ್ನು ಮೇಲೆತ್ತಿ ಆಕಾಶದಲ್ಲಿಡುತ್ತೇನೆ ಎಂದರೆ ಅಲ್ಲಿ ನಿಲ್ಲುವುದಿಲ್ಲ. ಕೈಬಿಟ್ಟ ತಕ್ಷಣದಲ್ಲೇ ಕೆಳಕ್ಕೆ ಬೀಳುತ್ತದೆ, ಏಕೆ? ಅದಕ್ಕೊಂದು ಭೂಮಿ ಇಲ್ಲದಿರುವುದಿಂದ. ಒಂದು ಬಿಲ್ಡಿಂಗ್ ಕಟ್ಟಬೇಕಾದರೂ ಯೋಗ್ಯ ತಳಹದಿ ಬೇಕು. ಅಡಿಪಾಯ ಭದ್ರವಾಗಿದ್ದಾಗ ಮೇಲೆ ಬಿಲ್ಡಿಂಗ್ ಕಟ್ಟುವುದು ಸುಲಭ. ಅಡಿಪಾಯವೇ ಇಲ್ಲದೆ ಬಿಲ್ಡಿಂಗ್ ಕಟ್ಟಲಾಗುವುದಿಲ್ಲ. ಅಂತೆಯೇ ಬುದ್ಧಿಯಲ್ಲಿ ಒಂದು ವಿಷಯ ನಿಲ್ಲಬೇಕಾಡರೆ ಬುದ್ಧಿಯು ಆ ವಿಷಯಕ್ಕೆ ಯೋಗ್ಯವಾದ ಭೂಮಿಯಾಗಬೇಕು. ಹಾಗಾದಾಗ ತನೇ ಆ ವಿಷಯದ ಹಿರಿಮೆಯು ನಮಗೆ ಅರಿವಾಗುತ್ತದೆ. 

\section*{ಉಪದೇಶದೊಂದಿಗೆ ಸಾಧನೆಯೂ ಆವಶ್ಯಕ}

ಸರ್ಕಸ್ಸಿನವನು ತಂತಿಯ ಮೇಲೆ ನಡೆಯುವುದನ್ನೂ ದೊಡ್ದಕಲ್ಲನ್ನು ಎತ್ತಿ ಬಿಡುವುದನ್ನೂ ನೋಡುತ್ತೇವೆ. ನಾವೂ ಒಂದು ತಂತಿಯನ್ನು ಕಟ್ಟಿ, ಅದರ ಮೇಲೆ ನಡೆಯಲು ಹೊರಟರೆ ಕೆಳಕ್ಕೆ ಬಿದ್ದು ಹಲ್ಲು ಮುರಿದುಕೊಳ್ಳಬೇಕಾಗುತ್ತದೆ. ಅವನಾದರೋ ಅದಕ್ಕೋಸ್ಕರವಾಗಿ ಶ್ರಮಿಸಿ, ಸಾಧನೆಯ ಮೂಲಕ ಆ ಶಕ್ತಿಯನ್ನು ಗಳಿಸಿಕೊಂಡಾದಾನೆ. ಅಂತೆಯೇ ನಾವೂ ಭಾಗವತವನ್ನು ಗ್ರಹಿಸಲು ಯೋಗ್ಯಭೂಮಿಯಾಗುವುದಿಲ್ಲ. ಆ ಭೂಮಿಕೆಯನ್ನು ಸರಿಯಾಗಿ ಗ್ರಹಿಸಿ ತತ್ಪರಾಯಣರಾಗಿ ಪಾರಾಯಣ ಮಾಡಿದಾಗ ತಾನೇ ವಿಷಯಕ್ಕೆ ನಮ್ಮ ಬುದ್ಧಿಯು ಭೂಮಿಕೆಯಾಗುವುದು! ಸಾಧನೆಯಿಲ್ಲದಿದ್ದರೆ ಕೇವಲ ಉಪದೇಶದಿಂದ ಕೆಲಸ ಸಾಧ್ಯವಾಗುವುದಿಲ್ಲ.

\section*{ಸಾಧನದ ಉಪಯೋಗವನ್ನು ಪರಿಶ್ರಮದಿಂದ ಪಡೆಯಬೇಕು}

ಈಜುವುದನ್ನು ನೋಡಿದ ಒಬ್ಬ ಇದೇನು ಮಹಾ! ನೀರಿನಲ್ಲಿ ಮಲಗಿ ಕೈಕಾಲು ಬಡಿದರೆ ಸರಿ, ಆಯಿತು ಎಂದುಕೊಂಡನಂತೆ. ಹಾಗೆ ನೀರಿನಲ್ಲಿ ಬಿದ್ದು ಸುಮ್ಮನೆ ಕೈಕಾಲು ಬಡಿಯ ಹೊರಟರೆ, ಕೊನಗೆ ಬಾಯಿಬಿಡಿದುಕೊಳ್ಳಬೇಕಾದ ಪರಿಸ್ಥಿತಿ ಬಂದೀತು. ಅದಕ್ಕೋಸ್ಕರವಾಗಿ ವಿಶೇಷವಾದ ಶ್ರಮದಿಂದ ಈಜುವುದನ್ನು ಕಲಿಯಬೇಕು, ಕೈಕಾಲುಬಡಿಯುವುದರಲ್ಲಿಯೂ ಒಂದು ವಿಧಾನವಿದೆ. ಹೇಗೆ ಕೈಕಾಲು ಬಡಿದರೆ ನೀರಿನಲ್ಲಿ ಮುಳುಗುವುದಿಲ್ಲವೋ ಆ ರೀತಿ ಬಿಡಿದಾಗ ತಾನೇ ಈಜಬಹುದು. ಹೀಗೆಯೇ ಸಂಸಾರ ಸಾಗರವನ್ನು ದಾಟುವುದಕ್ಕೆ ಒಂದು ಸಾಧನವಿದ್ದರೆ ಅದರ ಉಪಯೋಗವನ್ನು ಅದಕ್ಕೆ ಬೇಕಾದ ಶ್ರಮದಿಂದಲಲೇ ಪಡೆಯಬೇಕು. ಸುಮ್ಮನೆ ಸಾಧನವಿದೆ ಎಂದು ಉಪಯೋಗಿಸಲು ಹೊರಟರೆ ತಾರಕವಾಗುವ ಬದಲು ಮಾರಕವಾದೀತು.

(ಹಿಂದಿನ ಒಂದು ಘಟನೆಯನ್ನು ಗುರುಭಗವಂತನು ತಿಳಿಸಿದನು.)

ಹಿಂದೊಮ್ಮೆ ಮೈಸೂರಿನ ಕುಕ್ಕರಹಳ್ಳಿಕೆರೆಯಲ್ಲಿ ಜೊತೆಗಾರರೊಡನೆ ಈಜುವುದಕ್ಕೆ ಹೋಗಿದ್ದೆ, ಆಗ ಕಾಲೇಜ್ ನ ಹುಡುಗರಿಗೆ ಈಜುವುದು ಒಂದು ಸಂತೋಷದ ವಿಷಯ. ಆಗ ಒಂದು ದಿವನ ನಾನು ಒಬ್ಬ ಮನುಷ್ಯನ್ನು ಎದೆಯಮೇಲೆ ಕುಳ್ಳಿರಿಸಿಕೊಂಡು ಈಜಿಕೊಂಡು ಕೆರೆಯ ಆಚೆ ದಡಕ್ಕೆ ಹೋದೆ. ಇದನ್ನು ನೋಡಿದ ಇನ್ನೊಬ್ಬ ತಕ್ಷಣವೇ ಹುರುಪಿನಿಂದ ಅದನ್ನು ಅನುಸರಿಸಲು ಹೋಗಿ ನೀರಿನಲ್ಲಿ ಮುಳುಗಬೇಕಾಯಿತು. ಆತನು ಎದೆಯ ಮೇಲೆ ಕುಳಿತುಕೊಂಡುದನ್ನೂ ಆಚೆಯುದಡಕ್ಕೆ ಒಯ್ದದ್ದನ್ನೂ ಗಮನಿಸಿದ್ದನೇ ಹೊರತು, ಕುಳಿತುಕೊಳ್ಳುವವರನ್ನು ತಮ್ಮ ಈಜುವಿಕೆಗೆ ಅಪಾಯವಿಲ್ಲದಂತೆ ಹೇಗೆ ಕೂರಿಸಿಕೊಂಡಿದ್ದರು? ಅದಕ್ಕೋಸ್ಕರ ಎಷ್ಟು ಶ್ರಮಿಸಿದ್ದರು? ಎಂಬುದಾವುದನ್ನೂ ಗಮನಿಸಿರಲಿಲ್ಲ. 

ಹೀಗೆಯೇ ಸರ್ಕಸ್ಸಿನಲ್ಲಿ ಕೈಯಮೇಲೆ ಒಬ್ಬ ಅಳನ್ನು ನಿಲ್ಲಿಸಿಕೊಂಡು ಮೇಲೆತ್ತಿ ಕೇವಲ ಅಂಗೈಯಿಂದ ಒಬ್ಬ ವಯಸ್ಕನನ್ನೆತ್ತುವ ಸಾಹಸಪ್ರದರ್ಶನ ಮಾಡುವುದುಂಟು. ಅಲ್ಲೊಂದು ಸೂಕ್ಷ್ಮವಿದೆ, ಅಳನ್ನು ಕೈಮೇಲೆತ್ತಬೇಕಾದಾಗ ಕೈಮೇಲೇರುವವನಿಗೆ ಕಾಲನ್ನುಕೆರೆದು ಒಂದು ಸೂಚನೆಕೊಡುತ್ತಾನೆ. ಆಗ ಅವನು ಉಸಿರನ್ನು ಮೇಲಕ್ಕೆಳೆದು ಜಂಪ್ ಮಾಡುವ ಸಿದ್ಧತೆ ಮಾಡಿಕೊಳ್ಳುತ್ತಾನೆ, ಆ ಸಮಯಕ್ಕೆ ಸರಿಯಾಗಿ ಈತ ಕೈ ಮೇಲೆತ್ತುತ್ತಾನೆ, ಉಸಿರು ಕಟ್ಟಿದಾಗ ಮೈ ಹಗುರವಾಗುದಿಲ್ಲವ? ಆದರೆ ಇದನ್ನು ಕೇವಲ ಹೊರ ನೋಟದಿಂದ ಗ್ರಹಿಸಲಾಗುವುದಿಲ್ಲ. ಹಾಗೆಯೇ ಒಂದು ಅತ್ಯುತ್ತಮವಾದ ಸಾಹಿತ್ಯವಿದ್ದರೂ, ನಮಗೆ ಅದನ್ನು ಓದುವುದಕ್ಕೆ ಬರುವುದಾದರೂ ಅದನನ್ನು ಗ್ರಹಿಸುವುದಕ್ಕೆ ತಕ್ಕಂತಹ ಶ್ರಮಪಡದಿದ್ದಾಗ ಸಾಹಿತ್ಯವು ಅರ್ಥವಾಗುವುದಿಲ್ಲ. 

\section*{ಭಾಗವತಾದಗಳಲ್ಲಿ ಭೌತಿಕಾಂಶವನ್ನು ಮಾತ್ರ ಗ್ರಹಿಸದಿರಲು ಯೋಗ್ಯ ಹಿನ್ನೆಲೆ ಆವಶ್ಯಕ}

ಸಾಹಿತ್ಯವನ್ನು ನಮ್ಮ ಮಟ್ಟದಲ್ಲಿ ನಾವು ಗ್ರಹಿಸಲು ಹೋರಾಡುತ್ತೇವೆ, ಹಾಗೆ ಗ್ರಹಿಸಹೊರಟು ನಮ್ಮ ಮನಸ್ಸಿಗೆ ಪರಿಚಿತವಾಗಿರುವ ಭೌತಿಕ ಅಥವಾ ಸಾಂಸಾರಿಕ ವಿಚಾರಗಳನ್ನು ಗ್ರಹಿಸಿ, ಉಳಿದುದನ್ನು ಏನೋ ತತ್ತ್ವ ಎಂದು ಬಿಟ್ಟು ಬಿಡುವುದಾಗುತ್ತದೆ. ನಮ್ಮ ಬುದ್ಧಿಗೆ ತೋರಿದಷ್ಟು ವಿಚಾರವನ್ನು ತೆಗೆದುಕೊಂಡು ಭಾಗವ್ತದಲ್ಲೇನಿದೆ ನಾವು ಕಾಣದ ವಿಚಾರ? ಕೃಷ್ಣ ಒಬ್ಬ ದೊಡ್ಡವನು ಎನ್ನುತ್ತಾರೆ. ಬೀದಿಯಲ್ಲಿ ಹೋಗುವ ಹೆಣ್ಣು ಮಕ್ಕಳನ್ನು ಕೆಣಕಿ, ಕದ್ದು ಲೂಟಿ ಮಾಡಿದ ವ್ಯಕ್ತಿಯನ್ನು ಹೊಗಳುತ್ತಾರೆ. ಇದನ್ನು ಓದಿದರೆ ಪುಣ್ಯ ಬೇರೆ ಬರುತ್ತಂತೆ, ಇದು ದೊಡ್ಡ ಕಥೆಯಂತೆ, ಆದರ್ಶವಂತೆ. ಏನು ಆದರ್ಶವೋ? ಎನ್ನಿಸುತ್ತದೆ, ಅಥವಾ ಇನ್ನಾರಾದರೂ ಈ ರೀತಿ ಕೆದಕಿದಾಗ ನಮ್ಮ ಬುದ್ದಿಗೆ ಸರಿಯಾದ ಹಿನ್ನೆಲೆಯಿಲ್ಲದಿದ್ದರೆ ಅವರು ಹೇಳುವುದು ವಾಸ್ತವವೆನಿಸುತ್ತದೆ.

ಶ್ರೀಕೃಷ್ಣ ಬೀದಿಯ ಹೆಣ್ಣುಹುಡುಗರನ್ನು ಕೆಣಕುತ್ತಿದ್ದ. ಈಗಲೂ ಯಾವನೋ ಒಬ್ಬ ಬೀದಿಯಲ್ಲಿ ಹೋಗುವ ಹೆಂಗಸನ್ನು ಕೆಣಕಿದ. ಇದನ್ನೂ ಓದಿ ಪುಣ್ಯ ಬರುತ್ತೆ ಎಂದರೆ ಛೇ, ಈಗಿನವರ ಚರಿತ್ರೆಯಿಂದ ಪುಣ್ಯ ಬರುವುದಿಲ್ಲ ಎನ್ನಬೇಕಾಗುತ್ತದೆ. ಈ ವಿಚಾರವೆಲ್ಲವೂ ಸಾಮಾನ್ಯ ಮಟ್ಟದ ಬುದ್ಧಿಗೆ ಸರಿಯೆಂದೇ ಅನ್ನಿಸುತ್ತದೆ. ಹೀಗೆ ಅಂತಹ ಗ್ರಂಥದ ಬಗ್ಗೆ ಈ ರೀತಿಯಾದ ಒಂದು ವಿರುದ್ದವಾದ ಅಭಿಪ್ರಾಯವು ಬಂದ ಅನಂತರ ಸರಿಯಾದ ಪರಿಚಯ ಕೊಡಲು ಹೋದರೂ, ನೀವು ಶುದ್ಧ ಅರ್ಥೋಡಾಕ್ಸ್. ಏನು ನಿಮ್ಮ ಪವಿತ್ರ ಗ್ರಂಥವೋ? ಎಂದು ಹಾಸ್ಯ ಮಾಡಿ ಬಿಡುತ್ತಾರೆ. ಆಗ ಅದರ ಆಸ್ವಾದನೆಯಿಂದ ಎಂದೆಂದಿಗೂ ದೂರವಿರಬೇಕಾಗುತ್ತದೆ. ಅದ್ದರಿಂದ ಮೊದಲಿಗೆ ಇಂತಹ ಇಂಪ್ರೆಷನ್ ಬೀಳುವ ಮೊದಲೇ ಸರಿಯಾದ ಹಿನ್ನೆಲೆಯು ಬೇಕು.

ಇತರ ದೇಶದವರು ರಾಮಾಯಣ, ಭಾರತ, ಭಾಗವತಾದಿ ಗ್ರಂಥಗಳನ್ನೂ ಓದುತ್ತಾರೆ. ಅವರಿಗೆ ಯಾವ ಹಿನ್ನೆಲೆಯೂ ಇರುವುದಿಲ್ಲ. ಅವರ ಸಾಹಿತ್ಯ ಕೇವಲ ಭೌತಿಕವಾದುದು. ಭೌತಿಕವಾದ ವಿಷಯವನ್ನು ಗ್ರಹಿಸಿ ಅಭ್ಯಾಸವಾದ ಬುದ್ಧಿಯು ನಮ್ಮ ಗ್ರಂಥಗಳಲ್ಲಿಯೂ ಆ ಭೌತಿಕವಾದ ಅಂಶಗಳನ್ನೇ ತೆಗೆದುಕೊಳ್ಳುತ್ತದೆ. ಅಲ್ಲಿ ಕಾಣುವ ವಿರೋಧಾಂಶಗಳನ್ನು ಕೇವಲ ಮೌಢ್ಯವೆಂದು ನಿರಾಕರಿಸಿಬಿಡುತ್ತದೆ. ನಮ್ಮ ದೇಶದ ಸಾಹಿತ್ಯಕ್ಕೆ ಬೇಕಾದ ಒಂದು ಸರಿಯಾದ ಭೂಮಿಕೆಯನ್ನೇ ಪಡೆಯದಿರುವ ಇತರ ದೇಶದವರು ನಮ್ಮ ಸಾಹಿತ್ಯವನ್ನು ಪರಸ್ಥಳದ ಪಾದ್ರಿಯು ಪಸನ (ಹಲಸು) ತಿಂದಂತೆಯೇ ಗ್ರಹಿಸುವುದು.

\section*{ಆಧ್ಯಾತ್ಮಿಕವಾದ ಹಸಿವಿದ್ದು ಓದಿದಾಗಲೇ ಭಾಗವತಾದಿಗಳ ಅರಿವು}

ರಾಮಾಯಣ-ಭಾರತ-ಭಾಗವತ ಮೊದಲಾದ ಗ್ರಂಥಗಳು ಕೇವಲ ಭೌತಿಕ ವಿಚಾರಕ್ಕೆ ಪ್ರಾಧಾನ್ಯವನ್ನು ಕೊಟ್ಟ ಗ್ರಂಥಗಳೇ ಅಲ್ಲ. ಅವು ಆಧ್ಯಾತ್ಮಿಕವಾದ ಹಸಿವನ್ನು ನೀಗಿಸಲು ಹೊರಟ ಗ್ರಂಥಗಳು. ಜೀವನದಲ್ಲಿ ಅನೇಕ ವಿಧವಾದ ಹಸಿವುಗಳು ಉಂಟು. ಹೊಟ್ಟೆಯ ಹಸಿವು - ಅದಕ್ಕೆ ಆಹಾರ ಕೊಟ್ಟರೆ ಹೋಗುತ್ತದೆ. ಕಣ್ಣಿನ ಹಸಿವು- ಒಳ್ಳೆಯ ಬಣ್ಣದ ದರ್ಶನದಿಂದ ಹೋಗುತ್ತದೆ. ಮೂಗಿನ ಹಸಿವು - ಸುಗಂಧ ಪದಾರ್ಥಗಳ ಗ್ರಹಣದಿಂದ ಹೋಗುತ್ತದೆ. ಹೀಗೆ ಒಂದೊಂದಕ್ಕೂ ಒಂದೊಂದು ಹಸಿವುಂಟು. ಅದದಕ್ಕೆ ಆಹಾರವನ್ನು ಕೊಟ್ಟರೆ ಅವುಗಳ ಹಸಿವು ಹೋಗುವುದೂ ನಮಗೆ ಗೊತ್ತಿರುವ ವಿಷಯ. 

ಮನುಷ್ಯನ ದೇಹದಲ್ಲಿರುವುದು ಕೇವಲ ಇಂದ್ರಿಯಗಳೇ ಅಲ್ಲ, ಮನಸ್ಸು-ಬುದ್ಧಿ-ಆತ್ಮ ಇವುಗಳೂ ಇವೆ. ಇವುಗಳಿಗೂ ಒಂದೊಂದು ಹಸಿವಿರುವುದಿಲ್ಲವೆ? ಮನಸ್ಸು-ಬುದ್ಧಿ-ಆತ್ಮಗಳ ಹಸಿವು ಎಂತಹುದು? ಅವುಗಳಿಗೆ ಬೇಕಾದುದೇನು? ಎನ್ನುವ ವಿಷಯ ಬಂದರೆ, ಆದಕ್ಕೆ  ನಮ್ಮಲ್ಲಿ ಉತ್ತರವಿಲ್ಲ. ಏಕೆಂದರೆ ಕೇವಲ ಇಂದ್ರಿಯಗಳ ಹಸಿವು ಹಾಗೂ ಅವುಗಳ ತೀರಿಕೆಯಲ್ಲೇ ಆಸಕ್ತರಾಗಿರುವ ನಮಗೆ ಮನಸ್ಸು-ಬುದ್ಧಿ-ಆತ್ಮ ಇವುಗಳ ಹಸಿವೇ ಹುಟ್ಟಿಲ್ಲ. ಈ ಹಸಿವು ಹುಟ್ಟದೇ ಹಸಿವು ತೀರಿಸುವ ಆಹಾರವೇಕೆ? `ಅಜೀರ್ಣೇ ಭೋಜನಂ ವಿಷಂ'\label{265} ಎಂಬಂತೆ ಆತ್ಮದ ಹಸಿವಿಲ್ಲದೆ ಆಧ್ಯಾತ್ಮಿಕವಾದ ಹಸಿವನ್ನು ತೀರಿಸುವ ಸಾಹಿತ್ಯವನ್ನು ತೆಗೆದುಕೊಂಡಾಗ ವಿಷದಂತೆ ಬೇಸರವಾಗುತ್ತದೆ. ಆದ್ದರಿಂದ ಆಧ್ಯಾತ್ಮಿಕವಾದ ಹಸಿವಿಗೆ ಆಹಾರವಾಗಿ ಹೊರಟಿರುವ ಗ್ರಂಥಗಳನ್ನು ಓದಬೇಕಾದರೆ ಮೊದಲು ಆ ಹಸಿವು ಉಂಟಾಗಬೇಕು. ಈಗ ನಮಗೆ ಅಜೀರ್ಣವಾಗಿದೆ. ಬುದ್ಧಿಯಲ್ಲಿ ಕಸದಂತಹ ಅನೇಕ ವಿಷಯಗಳು ಸೇರಿಸಿಕೊಂಡಿವೆ. ಅವುಗಳೆಲ್ಲವನ್ನೂ ಯೋಗ್ಯವಾದ ಭೂಮಿಕೆ ಅಥವಾ ಉಪದೇಶದಿಂದ ತೊಳೆದುಕೊಂಡು ಹಸಿವನ್ನು ಹುಟ್ಟಿಸಿಕೊಳ್ಳಬೇಕು. ನಂತರ ಅದರ ತೀರಿಕೆಗಾಗಿ ಗ್ರಂಥವನ್ನು ಓದಿದರೆ ಆಗ ಗ್ರಂಥದ ಹಿರಿಮೆಯ ಅರಿವಾಗುತ್ತದೆ.

\section*{ಭಾಗವತಾದಿಗಳು ಭೌತಿಕ ವಿಷಯವನ್ನು ಮಾತ್ರ ಚಿತ್ರಿಸುತ್ತವೆಯೇ?}

ಭಾರತ-ಭಾಗವತ ಮೊದಲಾದ ಗ್ರಂಥಗಳಲ್ಲಿ ಭೌತಿಕ ಹಾಗೂ ಸಾಂಸಾರಿಕವಾದ ವಿಷಯಗಳು ಹಲವರು ಇರಬಹುದು. ಅದರ ಜೊತೆಗೆ ಅವುಗಳು ಮೇಲೆ ಆಧ್ಯಾತ್ಮಿಕ ಸಾಧನವಾಗುವ ಹಿರಿಯ ಜವಾಬ್ದಾರಿಯೂ ಇದೆ. ರಾಕೆಟ್ ನಮ್ಮನ್ನು ಬಾಹ್ಯಾಂತರಿಕ್ಷಕ್ಕೆ ಒಯ್ಯುವ ಸಾಧನ. ಆದನ್ನು ನೋಡಿ ಇದೇನು ಮಹಾ; ಇದೂ ರೈಲಿನಂತೆಯೇ, ರೈಲಿಗೂ ಎಂಜಿನ್ನಿದೆ, ಇದಕ್ಕೂ ಇದೆ. ರೈಲಿಗೂ ಕಲ್ಲಿದ್ದಲು ಬೇಕು. ರೈಲನ್ನೂ ಲೋಹದಿಂದ ತಯಾರಿಸಿದ್ದಾರೆ, ಇದನ್ನೂ ಲೋಹದಿಂದ ತಯಾರಿಸಿದ್ದಾರೆ. ರೈಲು ಕಂಬಿಯ ಮೇಲೆ ಹೋಗುತ್ತದೆ, ಇದು ಕಂಬಿಯಿಲ್ಲದೆ ಹೋಗುತ್ತದೆ. ಇಷ್ಟೇ ತಾನೇ ವ್ಯತ್ಯಾಸ ಎಂದು ಕಂಬಿಯಿಲ್ಲದೆ ರೈಲು ಬಿಟ್ಟರೆ (ಬಾಯಿಗೆ ಬಂದಂತೆ ಹೇಳಿದರೆ) ರಾಕೆಟ್ಟಿನ ಹಿರಿಮೆಯ ಅರಿವಾಗುವುದೇ?

ಹೀಗೆಯೇ ಯಾವುದೋ ಒಂದಂಶದಲ್ಲಿ ಭೌತಿಕ ಜೀವನದ ಸಾಮ್ಯ ಕಂಡು ಬಂದರೆ, ಇದು ಯಾವ ಮಹಾಗ್ರಂಥ? ನಮ್ಮ ಜೀವನದಲ್ಲಿರುವುದನ್ನೇ ಇಲ್ಲಿಯೂ ಹೇಳಿದೆ ಎಂದುಕೊಂಡುಬಿಟ್ಟರೆ  ಎಂದೆಂದೂ ಗ್ರಂಥದ ಪ್ರಯೋಜನ ಪಡೆಯುವಂತಿಲ್ಲ. ಭಾಗವತವನ್ನು ನೋಡಿ ಮೊದಲು ಹೇಳಿದಂತೆ ಇದು ಶ್ರೀ ಕೃಷ್ಣನ ಕಥೆ ತಾನೇ? ಅದೇ ಬೆಣ್ಣೆ ಕದ್ದ, ಬೀದಿಯಲ್ಲಿ ಹೋಗುವ ಹೆಣ್ಣು ಹುಡುಗರನ್ನು ಕೆಣಕಿದ, ಸೀರೆ ಕದ್ದ, ಈ ವಿಷಯಗಳೆಲ್ಲಾ ಇದೆಯಲ್ಲಾ, ಅದೇ ತಾನೇ? ಏನು ಹೊಗಳುತ್ತೀರೋ? ಇವೆಲ್ಲಾ ಮಹತ್ಕಾರ್ಯವೇ? ಉದ್ದಕ್ಕೂ ಈ ವಿಷಯವೇ, ಇದನ್ನೋದಿದರೆ ಏನು ಪುಣ್ಯ ಬರುತ್ತದೆಯೋ? ಎಂದು ಪ್ರಶ್ನೆಗಳು ಏಳುವುದು ಸಹಜ, ಭೌತಿಕ ಜೀವನದಲ್ಲಿಯೇ ಇದ್ದು ಅಷ್ಟಂಶಗಳನ್ನು ಮಾತ್ರ ಗ್ರಹಿಸಿದಾಗ. ಆದರೆ ಇದನ್ನು ಬರೆದವರು ಯಾರು? ಅವರೇನು ಹೇಳಿದ್ದಾರೆ? ಅವರ ಮನಸ್ಸೇನು? ಏನನ್ನು ಚಿತ್ರಿಸುವುದಕ್ಕೋಸ್ಕರ ಅವರು ಈ ರೀತಿ ಬರೆದ್ದಾರೆ? ಕೇವಲ ಭೌತಿಕವಾದ ವಿಷಯಗಳನ್ನು ಚಿತ್ರಿಸುವುದೇ ಅವರ ಉದ್ದೇಶವೇ? ಎಂದು ತೆಗೆದುಕೊಂಡಾಗ ಹಾಗಲ್ಲ, ಭೌತಿಕ ಪ್ರಪಂಚದಲ್ಲೇ ಇರುವ ನಮಗೋಸ್ಕರ ಇನ್ನಾವುಇದೋ ಮನಸ್ಸನ್ನು ಹರತಂದು ಕಥಾರೂಪದಲ್ಲಿಟ್ಟಿದ್ದಾರೆ.

ಈ ಹಿನ್ನಲೆಯೊಡನೆ ಓದಿದಾಗ ನಮಗೆ ಕಥೆಗಳಲ್ಲಿರುವ ತತ್ತ್ವವು ತಿಳಿಯದಿದ್ದರೂ ಈ ಹಿನ್ನೆಲೆಯಿಂದ ಗ್ರಂಥ ಬರೆಯುವವರು ಕೇವಲ ಭೌತಿಕ ವಿಷಯವನ್ನು ಚಿತ್ರಿಸಲಾರರು. ಅದರ ಹಿಂದೆ ಇನ್ನೋನೋ ಇರಬೇಕು. ನಮ್ಮ ಬುದ್ಧಿಗೆ ಅರ್ಥವಾದಷ್ಟೇ ಗ್ರಂಥದಲ್ಲಿರುವುದಲ್ಲ. ಆದ್ದರಿಂದ ಮರ್ಮಜ್ಞರಿಂದ ವಿಷಯ ತಿಳಿಯೋಣ ಎಂದು ಕೊಂಡು ವಿಷಯವನ್ನು ಸರಿಯಾಗಿ ಗ್ರಹಿಸಲು ಬುದ್ಧಿಯು ಪ್ರಯತ್ನಿಸುತ್ತದೆ. ತನಗೆ ತಿಳಿದಿದ್ದನ್ನು ಮಾತ್ರ ಅಂದುಕೊಂಡು ಸುಮ್ಮನಾಗಿಬುಡುವುದಿಲ್ಲ. ಹೀಗೆ ಹುಡುಕಲು ಹೊರಟಾಗ ತಾನೇ ವಿಷಯವನ್ನು ಸರಿಯಾಗಿ ತಿಳಿಯಲು ಸಾಧ್ಯ.

\section*{ಹಿನ್ನೆಲೆಯಿಲ್ಲದೆ ಓದಿದವರ ಅಪಪ್ರಚಾರ}

ಭಾಗವತಾದಿಗ್ರಂಥಗಳನ್ನು ಸರಿಯಾದ ಹಿನ್ನೆಲೆಯಿಲ್ಲದೆ ಓದಿದ ಇತರರು ಅದನ್ನು ತಪ್ಪಾಗಿ ಪ್ರಚಾರ ಮಾಡುವುದು ಉಂಟು. ಭಾರತೀಯರೆಲ್ಲರೂ ಮೂಢರು, ವಿಷಯವೇ ಇಲ್ಲದೆ ವ್ಯವಹಾರಗಳು ಇವರದು ಎಂಬ ಅವರ ಅಪಪ್ರಚಾರಕ್ಕೆ ಮರುಳಾಗುವ ಜನರೂ ಉಂಟು.

\section*{ಮಹಾತ್ಮರು ತಂದ ಗ್ರಂಥವನ್ನರಿಯಲು ಅವರಿಗೂ ನಮಗೂ ಇರುವ ವೈಲಕ್ಷಣ್ಯಜ್ಞಾನ ಬೇಕು}

ಲೋಕದಲ್ಲಿ ಇಂದ್ರಿಯಗಳನ್ನವಲಂಬಿಸಿ ಕೇವಲ ಭೌತಿಕವಾದ ಜೀವನವನ್ನು ನಡೆಸುತ್ತಿರುವ ಪ್ರಾಣಿಗಳು ಹೇಗುಂಟೋ, ಅಂತೆಯೇ ಸೃಷ್ಟಿ ಹೇಗಿದೆ? ಆ ಸೃಷ್ಟಿಗನುಗುಣವಾದ ಸ್ಥಿತಿ ಯಾವುದು? ಆ ಸ್ಥಿತಿಯು ಕೆಡದಿರಬೇಕಾದರೆ ಹೇಗಿರಬೇಕು? ಆ ಸ್ಥಿತಿಯು ಒಂದು ಕ್ರಮವಾಗಿ ವಿಕಾಸಗೊಂಡಾಗ ಎಲ್ಲಿ ಲೀನವಾಗುತ್ತದೆ? ಲಯ ಹೇಗೆ ಆಗುತ್ತದೆ? ಎಂಬುದೆಲ್ಲವನ್ನೂ ಜ್ಞಾನದೃಷ್ಟಿಯಿಂದರಿತು, ತಮ್ಮ ಬಾಳನ್ನು ಸೃಷ್ಟಿನಿಯಮಕ್ಕೆ ವಿರೋಧವಿಲ್ಲದಂತೆ ನಡೆಸಿ ಪಾವನರಾದ ಮಹಾತ್ಮರೂ ಉಂಟು. ಇಂತಹ ಮಹಾತ್ಮರ ಕಡೆಯಿಂದ ಬಂದ ಗ್ರಂಥವಾಗಿದೆ ಭಾಗವತ. ಇಂತಹವರ ಕಡೆಯಿಂದ ಬಂದ ಗ್ರಂಥವನ್ನು ಹೇಗೆ ಗ್ರಹಿಸಬೇಕು? ಎನ್ನುವುದನ್ನು ನಾವರಿಯಬೇಕಾದರೆ, ಅವರ ಜೀವನದ ಆಳವೆಷ್ಟು? ನಮ್ಮ ಜೀವನದ ಆಳವೆಷ್ಟು? ನಮಗೂ ಅವರಿಗೂ ಇರುವ ವೈಲಕ್ಷಣ್ಯವೇನು? ಎನ್ನುವುದನ್ನು ಅರಿಯಬೇಕು. ಹೀಗೆ ಅರಿತಾಗ ತಾನೇ ನಮ್ಮ ಜೀವನದ ಮಟ್ಟದಲ್ಲಿಯೇ ಗ್ರಂಥವನ್ನು ನೋಡದೆ ಅವರ ಜೀವನದ ಆಳಕ್ಕೆ ನಮ್ಮ ಮನಸ್ಸು ಇಳಿಯಲು ಸಹಾಯವಾಗುತ್ತದೆ. 

\section*{ಮಹರ್ಷಿಗಳ ಮತ್ತು ನಮ್ಮ ಜೀವನಗಳಲ್ಲಿರುವ ವೈಲಕ್ಷಣ್ಯ }

ನಮ್ಮ ಜೀವನವನ್ನು ನಾವು ನೋಡಿಕೊಂಡಾಗ ನಾವು ಮಾಡುವ ಕೆಲಸವೆಷ್ಟು? ಎನ್ನುವುದು ಗಮನಕ್ಕೆ  ಬರುತ್ತದೆ. ಇಂದ್ರಿಯಗಳ ಸಹಾಯದಿಂದ ವಿಷಯಗಳನ್ನು ತಿಳಿಯುವುದು, ಶರೀರದ ಜ್ಞಾನೇಂದ್ರಿಯ ಅಥವಾ ಕರ್ಮೇಂದ್ರಿಯಗಳು ಅಪೇಕ್ಷಿಸುವ ವಿಷಯಗಳನ್ನು ಕೊಡುವುದು, ನಂತರ ಅವುಗಳ ವ್ಯಾಪಾರ ಮುಗಿದರೆ ನಿದ್ರಿಸುವುದು, ಇಷ್ಟು ಬಿಟ್ಟರೆ ಜೀವನದಲ್ಲಿ ಇನ್ನೇನನ್ನೂ ಕಾಣೆವು. ಅಂದರೆ ನಾವು ಮಾಡುವ ಕೆಲಸವೆಲ್ಲವೂ ಆತ್ಮಾವಲಂಬಿಯಾದುದು. ನಾವು ಇಂದ್ರಿಯಗಳಿಂದ ಇಂದ್ರಿಯಗಳಿಗೆ ಬೇಕಾದ ವಿಷಯಗಳನ್ನು ಹೇಗೆ ಗ್ರಹಿಸುತ್ತೇವೆಯೋ ಹಾಗೆಯೇ ಅವರು ಆತ್ಮನ ಮೂಲಕ ಜೀವನಕ್ಕೆ ಬೇಕಾದ ಎಲ್ಲಾ ವಿಷಯಗಳನ್ನೂ ಅರಿಯಬಲ್ಲವರು. ನಮಗೆ ಇಹದ ಏಕದೇಶ ಮಾತ್ರ ಗೊತ್ತು, ಅವರಿಗೆ ಇಹದ ಜೊತೆಗೆ ಪರವೂ ಗೊತ್ತು. 

\section*{ಪೂರ್ಣವನ್ನರಿಯುವಲ್ಲಿ ಇಂದ್ರಿಯಗಳ ಅಸಮರ್ಥತೆ}

ಜೀವನಕ್ಕೆ ಒಂದು ಅವಲಂಬನೆ ಬೇಕು. ನಾವು ಇಂದ್ರಿಯಗಳನ್ನವಲಂಬಿಸಿದೆವು, ಆವರು ಆತ್ಮವನ್ನವಲಂಬಿಸಿದರು. ಇಬ್ಬರಿಗೂ ಒಂದು ಅವಲಂಬನವಿದೆಯೆಲ್ಲಾ ಎಂದರೆ, ಸರಿ, ಇಂದ್ರಿಯಗಳು ಗ್ರಹಿಸುವುದಾದರೂ ಏನನ್ನು? ಅವುಗಳ ಯೋಗ್ಯತೆಯಾದರೂ ಏನು? ಅವುಗಳನ್ನೇ ಅವಲಂಬಿಸಿರುವುದರಿಂದ ಆಗುವ ನಷ್ಟವೇನು? ಎನ್ನುವುದನ್ನು ಅರಿಯಬೇಕಾಗುವುದು.

ಇಂದ್ರಿಯಗಳ ಗ್ರಹಣ ಸಾಮರ್ಥ್ಯವನ್ನಾದರೂ ನೋಡೋಣ, ಒಂದು ಪದಾರ್ಥವನ್ನು ತಂದರೆ ಅದರಲ್ಲಿರುವ ಎಲ್ಲಾ ಅಂಶಗಳನ್ನೂ ಯಾವ ಇಂದ್ರಿಯವೂ ಗ್ರಹಿಸಲಾರದು. ಕಣ್ಣು ವಸ್ತುವಿನ ಬಣ್ಣವನ್ನು ಮಾತ್ರ ಗ್ರಹಿಸಬಲ್ಲದು, ಅದರ ವಾಸನೆಯನ್ನಾಗಲೀ ರುಚಿಯನ್ನಾಗಲೀ ಗ್ರಹಿಸಲಾರದು. ಹೀಗೆಯೇ ಮೂಗು ವಾಸನೆಯನ್ನೂ, ಚರ್ಮವು ಸ್ಪರ್ಶವನ್ನೂ, ರಸನೆಯು ರಸವನ್ನೂ, ಕಿವಿಯು ಶಬ್ಥವನ್ನೂ (ಅಂದರೆ ಅದರದರ ವಿಷಯವನ್ನು ಮಾತ್ರ) ಗ್ರಹಿಸಬಲ್ಲದು. ಈ ಗ್ರಹಣವು ಸಹ ಎಲ್ಲಾ ಕಾಲದಲ್ಲಿಯೂ ಆಗುವುದಿಲ್ಲ. ಇಂದ್ರಿಯಗಳು ಕೆಡದೇ ಇದ್ದು ಅಂದರೆ ತಮಗೆ ಯಾವುದೇ ವ್ಯಾಧಿಯೂ ಇಲ್ಲದೇ ಇದ್ದಾಗ, ಒಂದು ವಸ್ತುವನ್ನು ಕೊಟ್ಟರೆ ಅದರಲ್ಲಿ ತಮತಮಗೆ ಸಂಬಂಧಪಟ್ಟ ಅಂಶಗಳನ್ನು ಮಾತ್ರ ತೆಗೆದುಕೊಳ್ಳುತ್ತವೆ. ಪೂರ್ಣವನ್ನು ಗ್ರಹಿಸಲು ಸಾಧ್ಯವಿಲ್ಲ. ಅಪೂರ್ಣವಾದ ಇಂದ್ರಿಯಗಳು ಪೂರ್ಣವನ್ನು ಹೇಗೆ ತಾನೇ ಗ್ರಹಿಸಿಯಾವು? ಪೂರ್ಣವನ್ನು ಗ್ರಹಿಸಬೇಕಾದರೆ ಪೂರ್ಣದಿಂದಲೇ ಸಾಧ್ಯ. `ಪೂರ್ಣಸ್ಯ ಪುರ್ಣಮಾದಾಯ ಪೂರ್ಣಮೇವಾವಶಿಷ್ಯತೇ ೧/೧೬ ಎನ್ನುವಾಗ ಒಂದನ್ನು ಹದಿನಾರು ಭಾಗ ಮಾಡಿ ಒಂದನ್ನು ಗ್ರಹಿಸಿದೆ ಎಂದರ್ಥ. `೧/೬೪ ಎನ್ನುವಾಗಲೂ ಒಂದನ್ನು ಅರವತ್ತನಾಲ್ಕು ಭಾಗ ಮಾಡಿ ಅದರಲ್ಲಿ ಒಂದನ್ನು ಗ್ರಹಿಸಿದೆ ಎಂದರ್ಥ. ಇವು ಪೂರ್ಣ ಸಂಖ್ಯೆಗಳಲ್ಲ, ಕೇವಲ ಅಂಶಗಳು.

\section*{ಇಂದ್ರಿಯಾವಲಂಬಿಯ ಬಾಳು ಅಧೋಗತಿಗೆ}

ಹೀಗೆ ಇಂದ್ರಿಯಗಳನ್ನು ಅವಲಂಬಿಸಿ ನಡೆಸುವ ಜೀವನವು ಅಂಶ ಜೀವನವಾಗುತ್ತದೆಯೇ ಹೊರತು ಪೂರ್ಣಜೀವನವಾಗುವುದಿಲ್ಲ. ಇಂದ್ರಿಯಗಳು ಹುಬ್ಬಿನಿಂದ ಕೆಳಗಿರುವ ವಸ್ತುಗಳನ್ನು ಮಾತ್ರ ಗ್ರಹಿಸಬಲ್ಲವು. ಹುಬ್ಬುಗಳ ಮೇಲಕ್ಕೆ ಅವುಗಳ ಪ್ರಸಾರವಿಲ್ಲ. ಆದ್ದರಿಂದ ಇಂದ್ರಿಯಗಳು ಕೇವಲ ಅಧೋಮುಖವಾದ ಗತಿಯುಳ್ಳವು. ಊರ್ಧ್ವಗತಿಯಿಲ್ಲ. ಇಂತಹ ಇಂದ್ರಿಯಗಳನ್ನು ಅವಲಂಬಿಸಿ ಬಾಳುವವನ ಗತಿಯೂ ಅಧೋಗತಿಯೇ.

\section*{ಇಂದ್ರಿಯಾವಲಂಬಿಗಳಾದ ನಮ್ಮ ಬುದ್ಧಿ ಪರಿಮಿತ ವಿಷಯವನ್ನು ಮಾತ್ರ ಗ್ರಹಿಸಬಲ್ಲದು}

ಆಂಶಿಕವಾಗಿಯೇ ಆಗಲಿ ವಿಷಯಗಳನ್ನು ಗ್ರಹಿಸಿದುದಾದರೂ ಕೇವಲ ಇಂದ್ರಿಯಗಳೇ? ಎಂದರೆ, ಅದೂ ಇಲ್ಲ. ಇಂದ್ರಿಯಗಳೇ ಗ್ರಹಿಸುವುದಾದರೆ ಮೃತ ಶರೀರಕ್ಕೆ ಏತಕ್ಕಾಗಿ ಏನೂ ಗೊತ್ತಾಗುವುದಿಲ್ಲ? ಆದ್ದರಿಂದ ಆತ್ಮವು ಕಣ್ಣು, ಕಿವಿ ಇತ್ಯಾದಿಯಾದ ಆಯಾ ಇಂದ್ರಿಯಗಳ ದ್ವಾರಾ ಗ್ರಹಿಸಿದ ವಿಷಯಗಳನ್ನು ಮಾತ್ರ ನಾವು ಬಲ್ಲೆವು. ಕಣ್ಣು, ಕಿವಿ, ಇವುಗಳಿಗೆ ಹೇಗೆ ಒಂದು ನಿಯತವಾದ ವಿಷಯವುಂಟೋ ಅಂತೆಯೇ ಮನಸ್ಸಿಗೂ ಒಂದು ವಿಷಯವಿರುವುದಾದರೆ, ಆ ಮನಸ್ಸಿಗೆ ಬೇಕಾದ ವಿಷಯವಾವುದು? ಬುದ್ಧಿಯು ತನಗಾಗಿ ಕೇಳುವ ವಿಷಯವಾವುದು? ಜೀವನು ಏನು ಕೇಳುತ್ತಾನೆ? ಸೃಷ್ಟಿಯಲ್ಲಿ ಮನಸ್ಸು, ಬುದ್ದಿ ಆತ್ಮಗಳೂ ಇರುವಾಗ ಅವುಗಳಿಗೂ ವಿಷಯ ಬೇಡವೇ? ಎನ್ನುವುದನ್ನು ಗಮನಿಸಬೇಕಲ್ಲವೇ? ಕೇವಲ ಇಂದ್ರಿಯಗಳ ಮೂಲಕ ಗ್ರಹಿಸಬಹುದಾದಷ್ಟು  ಮಾತ್ರ ಗ್ರಹಿಸಿರುವ ನಮಗೆ ಮನಸ್ಸು-ಬುದ್ಧಿ-ಆತ್ಮ ಇವುಗಳಿಗೆ ಬೇಕಾದ ವಿಷಯವೂ ಒಂದಿದೆ, ಇಂದ್ರಿಯಗಳಂತೆಯೇ ಅವುಗಳಿಗೂ ತಮ್ಮದೇ ಆದ ಅಪೇಕ್ಷೆ ಇದೆ ಎನ್ನುವುದೇ ಹೊಸ ವಿಷಯವಾಗಿರುವಾಗ ಜೀವವು ಏನನ್ನಪೇಕ್ಷಿಸುತ್ತದೆ? ಎನ್ನುವುದು ಅರಿವಾಗುವುದೆಂತು? ಒಂದೇ ಕಾಲದಲ್ಲಿ ಒಂದು ಭಾಗದಲ್ಲಿ ಎರಡು ವಸ್ತುಗಳನ್ನು ಇಡಲಾಗುವುದೇ? ಹೀಗೆ ಇಂದ್ರಿಯವ್ಯಾಪಾರವನ್ನು ಮಾತ್ರ ಬಲ್ಲ ಇಂದ್ರಿಯವ್ಯಾಪಾರವನ್ನೇ ಸರ್ವದಾ ಅವಲಂಬಿಸಿರುವ ನಮಗೆ ಅತೀಂದ್ರಿಯ ವಿಷಯಗಳು ತಿಳಿಯುವುದಾದರೂ ಹೇಗೆ ಇಂದ್ರಿಯಪ್ರಪಂಚದಲ್ಲಿ ವಿಷಯವನ್ನು ಗ್ರಹಿಸಿ ಕೇವಲ ಪರಿಮಿತ ಕ್ಷೇತ್ರದಲ್ಲಿರುವ ವಿಷಯವನ್ನು ಮಾತ್ರ ಗ್ರಹಿಸಬಲ್ಲದು ನಮ್ಮ ಬುದ್ದಿ.

\section*{ಆತೀಂದ್ರಿಯವನ್ನು ಅರಿತವರು ತಮ್ಮ ಅನುಭವಗಳನ್ನು ಗ್ರಂಥಗಳ ಮೂಲಕ ಹರಿಸಿದ್ದಾರೆ.}

ಆತ್ಮಾವಲಂಬಿಗಳಾದರೋ, ಜ್ಞಾನಿಗಳಾದರೋ ಇಂದ್ರಿಯಗಳನ್ನೂ ಬಲ್ಲರು, ಆದರೆ ಅಷ್ಟರಲ್ಲಿಯೇ ಅವರು ನಿಲ್ಲಲಿಲ್ಲ. ಸೃಷ್ಟಿಯಲ್ಲಿರುವ ಮನಸ್ಸು-ಬುದ್ಧಿ-ಆತ್ಮ ಇವುಗಳು ಕೇಳುವ ವಿಷಯವೇನು? ಎನ್ನುವುದನ್ನೂ ಹುಡುಕಿ, ಯಾವುದು ಸ್ಥಿರ? ಯಾವುದು ಅಸ್ಥಿರ? ಎಂಬ ಬಗ್ಗೆ ವಿವೇಚನೆ ನಡೆಸಿ ಆ ವಿಷಯವನ್ನೂ ತಿಳಿದವರು. ಹೀಗೆ ಹುಡುಕಿ ತಮ್ಮ ಬುದ್ಧಿಯನ್ನು ಒಳ್ಳೆಯ ಸಂಸ್ಕಾರಗಳ ಮೂಲಕ ದುರ್ಬೀನನ್ನಾಗಿ ಮಾಡಿಕೊಂಡು ಅತೀಂದ್ರಿಯ ಕ್ಷೇತ್ರವನ್ನೂ ಅರಿತವರು. ಹೇಗೆ ಇಂದ್ರಿಯ ಪರಿಮಿತಿಯನ್ನು ದಾಟಿ, ಅತೀಂದ್ರಿಯ ಕ್ಷೇತ್ರವನ್ನೂ ಆತ್ಮವನ್ನೂ ಕಂಡು, ಅಂತಹ ಹಿಂದಿನ ಪ್ರಪಂಚಕ್ಕಿಂತ ರಮ್ಯವಾದುದು ಬೇರಾವುದೂ ಇಲ್ಲವೆಂದರಿತವರು. ಹಾಗೆಯೇ ಹಿಂದಿನ ಪ್ರಪಂಚಕ್ಕೆ ಹೋದಾಗ ಅಲ್ಲಿ ಕಂಡ ವಿಷಯಗಳನನ್ನೂ, ತಾವು ಅನುಭವಿಸಿದ ಆನಂದವನ್ನೂ ಇತರರಿಗೆ ಅರಿವಾಗುವಂತೆ ಮಾಡಲು, ತಮ್ಮ ಅನುಭವವನ್ನೂ ಸುಖವನ್ನೂ ಎಲ್ಲರಿಗೂ ಹರಿಸಲು ಗ್ರಂಥಗಳನ್ನು ತಂದರು.

\section*{ಭೌತಿಕದೊಂದಿಗೆ ಆಧ್ಯಾತ್ಮಿಕವನ್ನೂ ಸೇರಿಸಿ ಜೀವಿ ಮೇಲಕ್ಕೇರುವಂತೆ ಮಾಡಿದ ಪ್ರಯತ್ನವೇ ಭಾಗವತಾದಿಗ್ರಂಥಗಳು}

ಒಬ್ಬನಿಗೆ ಹೊಟ್ಟೆನೋವಾಗುತ್ತಿದ್ದರೆ, ಒಳಗೆ ಆಗುತ್ತಿರುವ ನೋವನ್ನು ಹೊರಗೆ ಹೇಗೆ ತೋರಿಸುವುದು? ಆ ನೋವನ್ನು ಸೂಚಿಸಲು ಆ ನೋವಿನಿಂದಲೇ ಒಂದು ಧ್ವನಿ ಹೊರಡುತ್ತದೆ. ಅಹಹಹಹ ಎಂದು ನೋವನ್ನು ಸೂಚಿಸಿದರೆ,`ಇದೇನು ಹೀಗೆನ್ನುತ್ತೀಯೆ' ಎಂದು ಕೇಳಿದಲ್ಲಿ ನಿನಗೇನು ಗೊತ್ತಯ್ಯ ಒಳಗಿನ ನೋವು?' ಎನ್ನುವುದೂ ಉಂಟು. ಹೀಗೆ ಒಳಗಿನ ನೋವನ್ನು ಹೊರಕ್ಕೆ ಧ್ವನಿಯ ಮೂಲಕ, ಅಥವಾ ನರಳಿಕೆಯ ಮೂಲಕ ತರಬೇಕು. ಹೀಗೆ ಹೊಟ್ಟೆ ನೋವನ್ನು ಸೂಚಿಸಲು ಹೊರಗೆ ಹೇಗೆ ಧ್ವನಿ ಅವಲಂಬನವೋ, ಅಂತೆಯೇ ತಾನು ಕಂಡ ಆತ್ಮ, ಅದನ್ನು ಕಂಡಾಗ ತನಗಾಡ ಸಂತೋಷ ಇವುಗಳನ್ನು ವರ್ಣಿಸಲು ಅದಕ್ಕೆ ಬೇಕಾದ ಒಂದು ಸಾಹಿತ್ಯವನ್ನು ರಚಿಸಿದರು. ಇಲ್ಲದಿದ್ದರೆ ತಮಗಾಗುತ್ತಿರುವ ಸಂತೋಷವನ್ನು ಹೊರಪಡಿಸುವುದು ಹೇಗೆ? ಅದನ್ನು ತೆಗೆದುಕೊಳ್ಳುವವರೋ ಕೇವಲ ಇಂದ್ರಿಯಾವ ಲಂಬಿಗಳು. ಕೇವಲ ಆತ್ಮನ ವಿಷಯವನ್ನೋ ಅದರ ಸಂತೋಷವನ್ನೋ ಹೇಳಿದರೆ, ಅದು ತಮ್ಮದಲ್ಲವೆಂದು ಬಿಡುತ್ತಾರೆ. ಆದ್ದರಿಂದ ಬುದ್ದಿವಂತಿಕೆಯಿಂದ ಇಂದ್ರಿಯಕ್ಕೆ ಬೇಕಾದ ವಿಷಯವಿಟ್ಟಂತೆ ತೋರಿಸಿ, ಇಂದ್ರಿಯಗಳಿಂದಲೇ ತಮ್ಮ ಕಡೆಗೆ ಬರುವಂತೆ ಉಪಾಯ ಮಾಡಿದರು.

ಒಬ್ಬ ಮನುಷ್ಯನಿಗೆ ತಲೆ ತಿರುಗು ಬಂದರೆ ಅದನ್ನು ಹೇಗೆ ತಿಳಿಸುವುದು? ಸುಮ್ಮನೆ ತಲೆ ತಿರುಗುತ್ತಿದೆ  `ಎಂದರೆ,' ಅಯ್ಯಾ ನೀನು ಹೇಳುವುದು ಶುದ್ಧ ಸುಳ್ಳು, ನಿಂತೇ ಇದೆ ನಿನ್ನ ತಲೆ, ತಿರುಗುತ್ತಿದೆ ಎನ್ನುತ್ತೀಯಲ್ಲಾ?' ಎಂದು ಕೇಳಬಹುದು. ಇಲ್ಲಿ ಒಳಗೆ ತಿರುಗುವ ಅನುಭವ ಮಾತ್ರ ಆಗುತ್ತಿದೆ, ಅದು ಹೊರಗೆ ಕಾಣುವುದು ಹೇಗೆ? ಎಂದು ಯೋಚಿಸಿ, ಅದನ್ನು (ಆ ಅನುಭವವನ್ನು) ಹೊರಗಿಡಲು ಒಂದು ಸಾಧನವನ್ನು ಹುಡುಕಿದ ಬುಗುರಿಯಂತೆ ತಲೆ ತಿರುಗುತ್ತಿದೆ ಎಂದು ಬುಗುರಿಯನ್ನು ಆಡಿಸಿದರೆ ಮಧ್ಯ ಕಾಲದಲ್ಲಿ ಬಹಳ ವೇಗವಾಗಿ ತಿರುಗುತ್ತಿರುವಾಗ ನಿಶ್ಚಲವಾಗಿರುವಂತೆ ಕಾಣುತ್ತದೆ. ಅದನ್ನು ನೋಡಿ, `ಅಯ್ಯಾ ನೋಡು ಬುಗುರಿಯು ನಿಶ್ಚಲವಾಗಿ ಕಂಡರೂ ತಿರುಗುತ್ತಿರುವುದು ಹೇಗೆ ನಿಜವೇ ಹಾಗೆಯೇ ನನ್ನ ತಲೆಯೂ' ಎಂದು ತನ್ನ ತಲೆ ತಿರುಗಿವಿಕೆಯನ್ನು ಬುಗುರಿಯ ಮೇಲಿಟ್ಟು ಹೊರಪಡಿಸಿದ, ಅಂದರೆ ಒಳಗಿನ ತಿರುಗುವಿಕೆಯನ್ನು ಬಿಂಬಿಸುವ ಹೊರಸಾಧನವಾಗಿ ಬುಗುರಿಯನ್ನು ಬಳಸಿಕೊಂಡ. ಅಂದರೆ ಉಪಮಾನದಿಂದ ಉಪಮೇಯದ ಕಡೆಗೆ ಉಪಾಯ ಮಾಡಿ ಕರೆದು ಕೊಂಡಂತಾಯಿತು. 

ಹೀಗೆಯೇ ಜ್ಞಾನಿಗಳೂ ಒಳಪ್ರಪಂಚದ ವಿಷಯವನ್ನು ತಾವು ಅನುಭವಿಸಿ ಆನಂದಿಸಿದರು. ಒಳ ಪ್ರಪಂಚದ ಅರಿವಿಲ್ಲದ ಜನಗಳಿಗೆ ಆ ಆನಂದದ ಪರಿಚಯ ಮಾಡಿಕೊಡುವುದಕೋಸ್ಕರವಾಗಿ ಹೊರಪ್ರಪಂಚದಲ್ಲಿ ಜನಗಳು ಕಂಡಿರುವ ವಿಷಯಗಳನ್ನು ಉಪಯೋಗಿಸಿಕೊಂಡರು. ಭೌತಿಕ ಜೀವಿಗಳಿಗೋಸ್ಕರವಾಗಿ ಭೌತಿಕ ಮಾನವರ ನಿದರ್ಶನ. ಅಲಂಕಾರ, ಉಡಿಗೆ-ತೊಡಿಗೆಗಳು ಇವುಗಳ ಮೂಲಕ ಮಹರ್ಷಿಗಳು ಒಳಪ್ರಪಂಚದ ವಿಷಯಗಳನ್ನು ಹೊರಪಡಿಸಿದರು. ಆದರೆ ಅವರ ಲಕ್ಷ್ಯ ಮಾತ್ರ ಉಪಮೇಯವಾದ ಒಳಪ್ರಪಂಚವೇ ಆಗಿತ್ತು. ಭೌತಿಕ-ಆಧ್ಯಾತ್ಮಿಕ ಎರಡೂ ಸೇರಿರುವ ಕಥೆಗಳನ್ನು ಕಟ್ಟಿದರು. ಇದರಿಂದ ಭೌತಿಕವಿದೆಯೆಂದು ಹೋಗಿ ಆಧ್ಯಾತ್ಮಿಕಕ್ಕೇರುವಂತೆ ಮಾಡಿದರು `ಚಾಮುಂಡೀಬೆಟ್ಟ ಅಣ್ಣಾಸ್ವಾಮಿಗಳ ಮನೆಯ ಹತ್ತಿರ (ಅವರ ಬಂಧುಗಳ ಮನೆ) ಬಂದಿದೆಯಂತೆ' ಎಂದರೆ `ಇದೇನು?' ಎಂದು ತಕ್ಷಣ ಜಾಗಬಿಟ್ಟು ಅಣ್ಣಾಸ್ವಾಮಿಗಳು ಮನೆಯ ಹತ್ತಿರ ಹೋಗುವುದಿಲ್ಲವೇ? ಹಾಗೆ ಇರುವ ಜಾಗವನ್ನು ಬಿಡಿಸಿ, `ಆಮೇಲೆ ಎಲ್ಲಯ್ಯಾ ಚಾಮುಂಡಿಬೆಟ್ಟ?' ಎಂದರೆ, `ಅಲ್ಲಯ್ಯಾ, ಚಾಮುಂಡಿ ಎನ್ನುವವಳು ಬಂದಿದ್ದಳು, ಅವಳು ಒಳ್ಳೇ ಬೆಟ್ಟದ ಹಾಗಿದ್ದಳು' ಎಂದು ಹೇಳೋಣ. ಹೀಗೆ ಭೌತಿಕ ಪ್ರಪಂಚದ ವಿಷಯವನ್ನು ತೋರಿಸಿ ತಮ್ಮ ಕಡೆಗೆಳೆಯುವ ಪ್ರಯತ್ನ ಮಾಡಿದರು. ಒಟ್ಟಿನಲ್ಲಿ ಭೌತಿಕ ಪ್ರಪಂಚದಿಂದ ತಮ್ಮ ಕಡೆಗೆಳೆದು, ತಮ್ಮ ಆನಂದದ ಸವಿಯನ್ನುಣಿಸುವುದು ಅವರ ಮಹಾಪ್ರಯತ್ನದ ಉದ್ದೇಶ. ಆ ಉದ್ದೇಶದ ಫಲವೇ ಭಾಗವತಾದಿಗ್ರಂಥಗಳು.

\section*{ಅಸ್ಥಿರವಾದ ಇಂದ್ರಿಯಾವಲಂಬನವನ್ನು ತಪ್ಪಿಸಿ ಸ್ಥಿರವಾದ ಭಗವದವಲಂಬನವನ್ನು ನೀಡುತ್ತವೆ ಭಾಗವತಾದಿಗ್ರಂಥಗಳು}

ಜೀವವು ಈಗ ಇಂದ್ರಿಯಗಳನ್ನು ಅವಲಂಬಿಸಿರುವುದೇನೋ ಸರಿಯೇ,‌ಆದರೆ ಈ ಅವಲಂಬನ ತಪ್ಪಿದರೆ ಜೀವ ಎಲ್ಲಿರಬೇಕು? ಮತ್ತೆ ಯಾವುದಾದರೂ ಇಂದ್ರಿಯದ ಗೂಡನ್ನೇ ಹುಡುಕಬೇಕು. ಹೀಗೆ ಅಸ್ಥಿರವಾದ ಇಂದ್ರಿಯದ ಗೂಡುಗಳಲ್ಲಿ ಎಲ್ಲಿಯವರೆವಿಗೆ ಅಲೆಯುತ್ತಿರುವುದು? ಇದು ಅಸ್ಥಿರವಾದ ಅವಲಂಬನ, ಇದೊಂದು ಬಾಡಿಗೆಯ ಮನೆ, ಸ್ವಂತವಲ್ಲ, ಇದು ತಪ್ಪಿ ಹೋದರೆ ಬೇರೆ ಮನೆ ಮಾಡಬೇಕು, ಇಲ್ಲದಿದ್ದರೆ ಬೀದಿಯಲ್ಲೆಲ್ಲಾದರೂ ಬಿಕಾರಿಯಂತಿರಬೇಕು. ಬುದ್ಧಿವಂತನಾದರೆ ಬಾಡಿಗೆಯ ಮನೆಯಲ್ಲಿರುವಾಗಲೇ ಸ್ವಂತ ಮನೆಯೊಂದನ್ನು ನೋಡಿಕೊಳ್ಳಬೇಕು. ಇಂದ್ರಿಯಾವಲಂಬನವು ಅಸ್ಥಿರವೆಂದು ಮನಗಂಡು ಮಹರ್ಷಿಗಳು ಸ್ಥಿರವಾದ ಅವಲಂಬನವನ್ನ್ನು ಹುಡಿಕಿ ಪಡೆದರು. ತಾವು ಅವಲಂಬನವನ್ನುಇ ಇತರರಿಗೂ ದೊರಕಿಸಿಕೊಡಲು ಸಾಹಿತ್ಯವನ್ನು ತಂದರು.

ಜೀವಕ್ಕೆ ಸ್ಥಿರವಾದ ನೆಲೆಯೆಂದರೆ ಭಗವಂತೆನೇ. ಇಂದ್ರಿಯದ ಕಡೆ ಅಂಟಿರುವ ಜೀವವನ್ನು  ಮೆಲ್ಲಗೆ ಭಗವಂತನ ಕಡೆ ತಿರುಗಿಸಲು ಇಂದ್ರಿಯಕ್ಕೆ ಬೇಕಾದ ವಿಷಯಗಳನ್ನು ಭಗವಂತನಲ್ಲಿ ತೋರಿಸಿದರು. ಹೀಗೆ ಇಂದ್ರಿಯಾವಲಂಬನವನ್ನು  ಅವುಗಳ ಆಕರ್ಷಣೆಯಿಂದ ಮೆಲ್ಲಗೆ ತಪ್ಪಿಸಿ, ಆತ್ಮಾವಲಂಬಿಯಾದ ಜೀವನವನ್ನುಂಟು ಮಾಡಲು ಇಂದ್ರಿಯಗಳಲ್ಲಿ ಮೆಲ್ಲಿಗೆ ಆತ್ಮಭಾವವನ್ನು ತುಂಬಿದರು. ಮುಕುಂದ ಮಾಲೆಯ ಈ ಶ್ಲೋಕವನ್ನು ನೋಡೋಣ -

\begin{shloka}
ಜಿಹ್ವೇ ಕೀರ್ತಯ ಕೇಶವಂ, ಮುರರಿಪುಂ ಚೇತೋ ಭಜ, ಶ್ರೀಧರಂ\\
ಪಾಣಿದ್ವಂದ್ವ ಸಮರ್ಚಯ, ಅಚ್ಯತಕಥಾಃ ಶ್ರೋತ್ರದ್ವಯ ತ್ವಂ ಶೃಣು|\\
ಕೃಷ್ಣಂ ಲೋಕಯ ಲೋಚನದ್ವಯ, ಹರೇರ್ಗಚ್ಛಾಂಘ್ರಿಯುಗ್ಮಾಲಯಂ,\\
ಜಿಘ್ರಘ್ರಾಣ ಮುಕುಂದಪಾದತುಲಸೀಂ ಮೂರ್ಧನ್ನಮಾಧೋಕ್ಷಜಮ್||
\end{shloka}

ಇಲ್ಲಿ ಎಲ್ಲ ಇಂದ್ರಿಯಗಳನ್ನೂ ಭಗವಂತನ ಕಡೆಗೆ ಸೆಳೆಯಲು ಅವುಗಳಿಗೆ ಒಂದೊಂದು ಕೆಲಸವನ್ನಂಟಿಸಿದರು. ಕೊಡುವ ಆಹಾರದಲ್ಲಿ, ನೋಡುವ ವಸ್ತುವಿನಲ್ಲಿ, ಆಘ್ರಾಣಿಸುವ ವಾಸನೆಯಲ್ಲಿ ಭಗವಂತನತನವನ್ನು ತೋರಿಸಿದರು. ಹಾಗೆ ಭಗವತ್ಸಂಬಂಧವಾದ ಅನುಭವದಿಂದ ಹೊರಟ ಗ್ರಂಥವಾದ್ದರಿಂದಲ್ಲೆ ಭಾಗವತವಾಯಿತು. ಓದುಗರಿಗೆ ಭಗವಂತನ ಸಂಬಂಧವನ್ನುಂಟುಮಾಡುವುದರಿಂದಲೂ ಭಾಗವತವೆನ್ನಿಸಿಕೊಂಡಿತು.

\section*{ಭಾಗವತಾದಿ ಗ್ರಂಥಗಳನ್ನು ಉಪದೇಶಿಸುವವನಲ್ಲಿರಬೇಕಾದ ಯೋಗ್ಯತೆ}

ಇಂತಹ ಮಹಾಗ್ರಂಥವನ್ನು ಉಪದೇಶಿಸಲು ಯೋಗ್ಯನಾದ ಉಪದೇಶಕ ಅವಶ್ಯಕ. ಇಂದು ಉಪದೇಶ ಕೊಡುತ್ತೇನೆ ಎಂದು ಗ್ರಂಥದ ಮೊದಲನೆಯ ಶ್ಲೋಕ ಹೇಳಿಕೊಡುವ ಸಂಪ್ರದಾಯ ಉಳಿದಿದೆ. ಆದರೆ ಆ ಶ್ಲೋಕವನ್ನು ಇವನು ನೋಡುವ ಬದಲು ಅವರು ಹೇಳಿದಂತಾಯಿತು ಅಷ್ಟೆ. ಅದರಲ್ಲಿ ಇವನಿಗೆ ಹೊಸದೇನೂ ಲಭಿಸುವುದಿಲ್ಲ. ಆದ್ದರಿಂದ ಉಪದೇಶಕನೂ ಅದರ ಮರ್ಮಜ್ಞನಾಗಿದ್ದು, ಮಹರ್ಷಿಗಳ ಸಾಹಿತ್ಯವನ್ನು ಸರಿಯಾದ ಹಿನ್ನೆಲೆಯೊಡನೆ ಗ್ರಹಿಸಿ ಅಂತೆಯೇ ಲೋಕದ ಮುಂದಿಡುವ ವ್ಯಕ್ತಿಯಾಗಬೇಕು.
