\chapter{ರಾಮಾಯಣ - ೨}

(ದಿನಾಂಕ ೨೬-೪-೧೯೫೮ ರಂದು ಶ್ರೀಗುರುವು ಶಿಕ್ಷೆಯ ಬಗ್ಗೆ ಮಾತನಾಡುತ್ತಾ ಪ್ರಾಸಂಗಿಕವಾಗಿ ರಾಮಾಯಣದ ಬಗ್ಗೆ ಆಡಿದ ಮಾತುಗಳು.)

\section*{ರಾಮನಿಂದ ರಾವಣವಧೆ ಎನ್ನುವುದೇ ದೈವೀಸಂಕಲ್ಪ}

ದಶರಥನೇ ಏಕೆ ತನ್ನ ರಾಜ್ಯದ ಋಷಿಗಳಿಗೆ ಕಂಟಕನಾಗಿದ್ದ ರಾವಣನ ಮೇಲೆ ಕೈ ಮಾಡಲಿಲ್ಲ? ದೌರ್ಬಲ್ಯದಿಂದಲೇ? ಯಾವುದರಿಂದ ಬೇಕಾದರೂ ಆಗಿರಲಿ. ಆ ಕೆಲಸವನ್ನು  ಶ್ರೀರಾಮನೇ ಮಾಡಬೇಕೆಂಬುದು ಸಂಕಲ್ಪ. ಬ್ರಹ್ಮಾದಿದೇವತೆಗಳು ಭಗವಂತನಲ್ಲಿ ಹೋಗಿ ಬೇಡಿದ್ದೇನು? ಇಷ್ಟೆಲ್ಲಾ ಹಿಂದಿನ ಕಥೆಯಿದೆ. ಮನೆಯನ್ನೆಲ್ಲಾ ಕಟ್ಟುವುದಕ್ಕೆ ಏರ್ಪಾಟಾಗಿ ಪಾಯಾಹಾಕಿ ಎಲ್ಲಾ ಆಗಿದೆ. ನೂರಾರು ಜನ ನಿಂತಿದ್ದಾರೆ. ಆ ತಳಹದಿ ಕಲ್ಲುಹಾಕುವ ಕೆಲಸವನ್ನು ರಾಜೇಂದ್ರಪ್ರಸಾದರೇ ಏಕೆ ಮಾಡಬೇಕು? ಅಷ್ಟು ಜನರ ಪೈಕಿ ಬೇರೆ ಯಾರಿಗೂ ಆ ಕಲ್ಲನ್ನು ಎತ್ತಿಹಾಕುವ ಶಕ್ತಿಯಿಲ್ಲವೇ? ಎಂದರೆ ಅಧ್ಯಕ್ಷರ ಅಮೃತಹಸ್ತದಿಂದಲೇ ಆ ಕೆಲಸ ನೆರವೇರಬೇಕಾಗಿರುವಾಗ ಬೇರೆ ಯಾರೂ ಮಾಡಕೂಡದು. ಹಾಗೆಯೇ ರಾಮನೇ ಆ ರಾವಣಸಂಹಾರಕಾರ್ಯವನ್ನು ಮಾಡಬೇಕೆಂದು ಸಂಕಲ್ಪ.

\section*{ಸಂಕಲ್ಪದಂತೆ ಭಗವತ್ಕಾರ್ಯಕ್ಕೆ ಸಾಧನ}

ವಾಲ್ಮೀಕಿಗಳು ಪ್ರಾಚೀನಾಗ್ರದರ್ಭೆಯಲ್ಲಿ ಕುಳಿತು ತತ್ತ್ವದೃಷ್ಟಿಯಿಂದ ನೋಡಿ ಮಾಡಿದ ರಚನೆ ರಾಮಾಯಣ. ತತ್ತ್ವ ದೃಷ್ಟಿಯಿಂದ ಅದರಲ್ಲೇನಾದರೂ ದೋಷವಿದೆಯೇ? ಹೇಳಿ! `ಇದೆ, ಇಲ್ಲ' ಎಂದರೆ ತತ್ತ್ವವೆಂದರೇನು? ಎಂದು ಕೇಳುತ್ತೇನೆ. ಹಾಗೆ ತತ್ತ್ವಕ್ಕೂ ಸರಿಪಡುವಂತೆ ಬೇರೆ ತರಹಾ ಯೋಜನೆ ಮಾಡಿರುವ ಜಾಗವೂ ಉಂಟು. ಉದಾಹರಣೆಗೆ ಮಹಾಭಾರತದ ಕೃಷ್ಣನ ಪಾತ್ರ. ಕೃಷ್ಣನು ತಾನೇ ದುರ್ಯೋಧನಾದಿಗಳನ್ನು ಸಂಹಾರಮಾಡದೆ `ನಿಮಿತ್ತಮಾತ್ರಂ ಭವ ಸವ್ಯಸಾಚಿನ್\label{214a}' ಎಂದು ಅರ್ಜುನನ್ನು ಗೋಗರೆಯುವುದು ಏಕೆ? ಎಂದರೆ ಹಾಗೆ ಮಾಡಬೇಕಾಗುತ್ತೆ. ಪ್ರಣವನಾದ ಮಾಡಲು ಆತ್ಮನು ನಾಲಿಗೆಯನ್ನೇ ಗೋಗರೆಯಬೇಕಾಗುತ್ತದೆ. `ಜಿಹ್ವೇ ಕೀರ್ತಯ ಕೇಶವಂ\label{214} ಎಂದು ಹೇ ಕಿತ್ತು ಹಾಕುತ್ತೇನೆ' ಎಂದು ಇತರರನ್ನು ಹೆದರಿಸಬಹುದು. ಆದರೆ ಕೇಶವಕೀರ್ತನೆ ಮಾಡಲು ತನ್ನ ನಾಲಿಗೆಗೆ ಸಲಾಮು ಹಾಕಲೇಬೇಕಾಗುತ್ತೆ. 

\section*{ಆತ್ಮ ಮೂಲವಾಗಿ ಹೊರಟ ಗ್ರಂಥದ ಮೇಲೆ ಇಂದ್ರಿಯಮೂಲವಾದ ಪ್ರಶ್ನೆಗಳು ಸಲ್ಲವು}

`ರಾಮನೇನೋ ತಪ್ಪು ಮಾಡಿದ. ಅದಕ್ಕಾಗಿ ತನ್ನ ಮಗನೇ ಆದರೂ ಆತನನ್ನು ದಶರಥ ಕಾಡಿಗಟ್ಟಿದ. ಅಲ್ಲಿ ಪಶ್ಚಾತ್ತಾಪ ಪಟ್ಟು ಹಿಂದಿರುಗಿದ ಮೇಲೆ ಮನೆಗೆ ಸೇರಿಸಿಕೊಂಡ' ಎಂದು ಕಥೆ ಕಟ್ಟುವಿರಾ? ಅದು ತತ್ತ್ವಕ್ಕೆ ಸರಿಹೊಂದುತ್ತದೆಯೇ?

ಊರ್ಮಿಳೆಗೆ ಸ್ಥಾನವನ್ನು ವಾಲ್ಮೀಕಿ ಕೊಡಲಿಲ್ಲವೆಂದು ತೆಗಳಬೇಡಿ. ಅದು ತತ್ತ್ವದ ಕಥೆ. ಆ ತತ್ತ್ವ ಯೋಜನೆಯಲ್ಲಿ ಭೂಮಿಕೆಯು ಇರಬೇಕೋ ಬೇಡವೋ? ಅದಕ್ಕಿಂತ ಇನ್ನು ಯಾವುದಕ್ಕೆ ಪ್ರಾಮುಖ್ಯ ಕೊಡಬೇಕೋ ನಿಮಗೆ ಗೊತ್ತೇ? ಇಂದ್ರಿಯ ಮೂಲವಾಗಿ ಬಗೆಬಗೆಯ ಪ್ರಶ್ನೆಗಳು ಬರಬಹುದು. ಆದರೆ ಆತ್ಮಮೂಲವಾಗಿ ಹೊರಟ ಗ್ರಂಥ ವಾಲ್ಮೀಕಿಯದು.

\section*{ರಾವಣ ಶರಣಾಗಿದ್ದರೆ ಕ್ಷಮಿಸಬಹುದಿತ್ತೇ?}

`ರಾವಣನು ಈಗ ಶರಣಾಗತನಾದರೆ ಬಿಟ್ಟು ಬಿಡುತ್ತೇನೆ' ಎಂದು ರಾಮನು ಹೇಳಿದ್ದು  ಸರಿಯೇ? ಸಾವಿರಾರು ಜನ ಋಷಿಗಳ ಕೊಲೆಗೆ ಆತ ಕಾರಣನಾಗಿದ್ದನಲ್ಲಾ? ಎಂದರೆ, ಒಂದು ಪಕ್ಷ ಆತನಿಗೆ ಪಶ್ಚಾತ್ತಾಪ ಬಂದು ಬಿಟ್ಟಿದ್ದರೆ ನಿಜವಾಗಿಯೂ ಆತನು ವಿವೇಕಿಯಾಗಿಬಿಟ್ಟಿದ್ದರೆ ಅಲ್ಲಿಂದ ಮುಂದೆ ಶಿಕ್ಷೆ ಏಕೆ? ಯಾರ ಮನಸ್ಸು ಹೇಗಿರುತ್ತದೆಯೋ? ವಿಭೀಷಣನು ವಿವೇಕ ಹೇಳಿದ್ದಾನೆ. ಅವನಿಗೆ ಧರ್ಮವೂ ಗೊತ್ತು. ಒಂದು ಪಕ್ಷ ರಾವಣನು ಸಾತ್ವಿಕನಾಗಿ ಬಿಟ್ಟರೆ ಸತ್ವಗುಣಕ್ಕೆ ಶಿಕ್ಷೆಯೆಲ್ಲಿ? ಅದರ ಮೇಲೂ ರಾಮನು ಶಿಕ್ಷೆ ಮಾಡಿದರೆ ಅದು ಸರಿಯಾಗಲಾರದು.

\section*{ರಾಮ-ರಾವಣರ ತಾರತಮ್ಯಚಿಂತನೆ}

ರಾಮನ ಆತ್ಮಗುಣ ಕ್ಷಮೆ. ಸೀತೆಯೂ ಹಾಗೆಯೇ. ರಾಮ-ರಾವಣರ ತಾರತಮ್ಯ ಚಿಂತನೆಯೂ ಇರಬೇಕು. ಶತ್ರುವೆಂದು ರಾವಣನ ಗುಣವರ್ಣನೆಯಲ್ಲಿ ಜಿಗುಪ್ಸೆಯಿಲ್ಲ. ಅದನ್ನು ಹೇಳಿ-

\begin{shloka}
`ಯದ್ಯಧರ್ಮೋ ನ ಬಲವಾನ್ ಸ್ಯಾದಯಂ ರಾಕ್ಷಸೇಶ್ವರಃ|\label{215}\\
ಸ್ಯಾದಯಂ ಸುರಲೋಕಸ್ಯ ಸಶಕ್ರಸ್ಯಾಪಿ ರಕ್ಷಿತಾ'||
\end{shloka}

ರಾಮನಲ್ಲಾದರೆ `ಧರ್ಮಾತ್ಮಾ ಸತ್ಯಸಂಧಶ್ಚ' ಇನ್ನು `ದುರಾತ್ಮಾ ಅಸತ್ಯಸಂಧ' ರಾವಣ. ಆತ್ಮಗುಣಗಳೇ ರಾಮನಲ್ಲಿರುವುದು. ಅವತಾರ ಮಾಡಿರುವ ಆತ್ಮವೇ ರಾಮ. ಆತನು ದಯಾಳು, ಕ್ಷಮಾವಂತ.
