\chapter{ಶ್ರೀರಾಮಪಟ್ಟಾಭಿಷೇಕಕ್ಕೆ ಹಿನ್ನೆಲೆ}

(ದಿನಾಂಕ ೭-೧೦-೬೦ ನೇ ಶುಕ್ರವಾರ. ಅಂದು ಮಹಾಭರಣೀ ಪರ್ವ. ಆ ಪರ್ವದಂದು ಗುರುಭಗವಂತನು ಅನುಗ್ರಹಿಸಿದ ರಾಮಾಯಣಪಾಠ.)

\section*{ಶ್ರೀರಾಮಪಟ್ಟಾಭಿಷೇಕ ನಮ್ಮ ಹೃದಯಸಿಂಹಾಸನದಲ್ಲಾಗಬೇಕು}

ರಾಮದೇವರು ಇಲ್ಲಿ ಬಿಜಯಮಾಡಿಸಿದ್ದಾರೆಂಬ ಭಾವನೆಯಿಂದ ನಾವೆಲ್ಲರೂ ಸೇರಿ ಅವನ ಪಟ್ಟಾಭಿಷೇಕದಲ್ಲಿ ಭಾಗಿಗಳಾಗಿದ್ದೇವೆಂಬ ಮನಸ್ಸುಳ್ಳವರಾಗಿದ್ದೇವೆಂದು ತಿಳಿದುಕೊಂಡರೆ ತಾನೇ ಚೆನ್ನಾಗಿರುತ್ತೆ. ರಾಮದೇವರ ವಿಷಯವನ್ನು ಲೋಕದಲ್ಲಿ, ನಮ್ಮ ದೇಶದಲ್ಲಿ ಎಲ್ಲರೂ ಹೇಳುತ್ತಿರುತ್ತಾರೆ. `ನವರಾತ್ರಿ ಬಂತು, ಶ್ರೀರಾಮನವಮಿ ಬಂತು ಶ್ರೀರಾಮಾಯಣ ಪಾರಾಯಣ, ಪಟ್ಟಾಭಿಷೇಕ ಮಾಡಯ್ಯಾ ಎಂದು ಪ್ರತಿಯೊಬ್ಬರ ಮನೆಯಲ್ಲಿಯೂ ರಾಮಾಯಣವನ್ನು ತಂದಿಟ್ಟುಕೊಂಡು ಪಾರಾಯಣ ಮಾಡುವುದು, ಇಷ್ಟಜನ, ಬಂಧು ಮಿತ್ರರು, ಪಾಮರ ಪಂಡಿತರನ್ನು ಸಭೆ ಸೇರಿಸಿ ಪಟ್ಟಾಭಿಷೇಕ ಮಾಡುವ ಪದ್ಧತಿ, ನಮ್ಮಲ್ಲಿ ಸಂಪ್ರದಾಯವಾಗಿ ಬಂದು ನಿಂತಿದೆ. ಇವೇನೋ ಸಾಂಪ್ರದಾಯಿಕವಾಗಿ ಬಂದಿದ್ದರೂ ಬೇರೂರಿ ನಿಲ್ಲಬೇಕಾದರೆ ಪೂರ್ವಭಾವಿಯಾಗಿ ಹೇಗೆ ಸಿಂಹಾಸನ ಸಿದ್ಧತೆ, ಕಿರೀಟ, ಹಾರ, ಹಣ್ಣು, ಅಲಂಕಾರ ಸಿದ್ಧತೆಗಳನ್ನು ಮಾಡಿಕೊಂಡು ಪಟ್ಟಾಭಿಷೇಕವನ್ನು ಮಾಡುತ್ತೇವೆಯೋ ಹಾಗೆ ನಮ್ಮ ಹೃದಯದಲ್ಲಿ, ಹೃದಯಸಿಂಹಾಸನದಲ್ಲಿ ಅವನನ್ನು ಕೂರಿಸಿ, ಇಂದ್ರಿಯ ಸಿದ್ಧತೆ ಮಾಡಿಕೊಳ್ಳಬೇಕು. ಹೇಗೆ ಪುಸ್ತಕಕ್ಕೆ ದೀಪಕ್ಕೆ ಪೀಠಬೇಕೋ ಹಾಗೆ ಬೆಳಗುವ ರಾಮನನ್ನು ಹೃದಯದಲ್ಲಿ ಕೂರಿಸಲು ಹೃದಯಪೀಠವನ್ನು ಸ್ಥಿರವಾಗಿ ಸಿದ್ಧಮಾಡಬೇಕು.

\section*{ರಾಮಾಯಣ ನಮ್ಮ ಅಂತರಂಗಕ್ಕೂ ವಿಷಯವಾಗಬೇಕು}

ಇಲ್ಲಿ ಹುಡುಗರೂ ಹೆಂಗಸರೂ ವೃದ್ಧರೂ ಮಕ್ಕಳೂ ಎಲ್ಲರೂ ಅವನ ಪಟ್ಟಾಭಿಷೇಕಕ್ಕೆ ನೆರೆದಿದ್ದಾರೆ. ಹೊರಗಣ್ಣೆಗೆ ಕಾಣುವುದು ಪುಷ್ಪ, ಫಲ, ಸಕ್ಕರೆ, ರಸಬಾಳೆಹಣ್ಣು ಮೊದಲಾದವುಗಳು. ಇವೆಲ್ಲವೂ ಅಂತರಂಗದ ಪ್ರತೀಕವಾಗಿದೆ. ಹೊರಗಣ್ಣಿಗೆ ರಾಮದೇವರು ಪುಸ್ತಕವಾಗಿ ಕಂಡರೂ ನಮ್ಮ ಬುದ್ಧಿ , ಆತ್ಮಕ್ಕೂ ವಿಷಯವಾಗಿರಬೇಕು. ಶ್ರೀರಾಮನ ಮಹಿಮೆ ಅಷ್ಟರವರೆವಿಗೂ ಇರುತ್ತೆ. ವಾಲ್ಮೀಕಿಗಳು ತಾವು ಬರೆಯುವಾಗ ಪುಸ್ತಕರೂಪವಾಗಿ ಕಾಣಲಿಲ್ಲ. ವಾಲ್ಮೀಕಿಗಳು ಮೊದಲು ಪುಸ್ತಕ ಪ್ರಾರಂಭಿಸಿದಾಗ ಅವರ ಹೃದಯದಲ್ಲಿ ರಾಮನನ್ನು ಕಂಡರು. ಆಗ ಅವರ ಹೃದಯದಿಂದ ಉಕ್ಕಿ ಬಂದ ವಾಣಿಯು ನಮ್ಮ ಅಸ್ತಿಯಾಗಿದೆ.

\section*{ಸಂಪ್ರದಾಯದಲ್ಲಿ ನ್ಯೂನತೆ}

ಇದನ್ನು ಗುರು ಶಿಷ್ಯ ಪರಂಪರೆಯಿಂದ ಉಪದೇಶವಾಗಿ ತಂದರು. ಗ್ರಂಥ, ಕರ್ತೃ, ಅದರ ರೂಪರೇಖೆ ಕೆಡದಂತೆ ತೆಗೆದುಕೊಳ್ಳುವುದೇ ಉಪದೇಶ. ಸಂಪ್ರದಾಯವೆಂದರೆ ಗುರುವಿನಿಂದ ಶಿಷ್ಯನಿಗೆ, ಅವನು ಅವನ ಶಿಷ್ಯನಿಗೆ ಹೀಗೆ ಹರಿಸದು ಬಂದ ವಿಚಾರವೆಂದರ್ಥ. ಹೀಗೆ ಕೊಟ್ಟ ಕ್ರಮ ಶುದ್ಧವಾಗಿ ಬಂದಿದ್ದರೆ ಸಾಕ್ಷಾತ್ ರಾಮನನ್ನು ಅನುಭವಿಸಲು ಅವಕಾಶವಿರುತ್ತಿತ್ತು.

\section*{ರಾಮಾಯನಪಾರಾಯಣ ರಾಮನ ಅನುಭವವನ್ನುಂಟುಮಾಡುವಂತಿರಬೇಕು}

ಹೇಗೆ ವಿವಾಹವಯಸ್ಕಳಾದ ಹುಡುಗಿಗೆ ಯಾವನನ್ನು ಮದುವೆ ಮಾಡಿಕೊಳ್ಳುವಳೋ ಅವನ ಫೋಟೋವನ್ನು ಕೊಟ್ಟಾಗ ಆ‌ ಫೋಟೋ ವಾಸ್ತವವಾಗಿ ಅವನಲ್ಲದಿದ್ದರೂ ಅವರನ್ನು ಮದುವೆ ಮಾಡಿಕೊಳ್ಳಬೇಕು, ಅವರೇ ದೈವ ಎನ್ನುವಂತೆ ಕೈಲಿ ಬರೀ ಕಾಗದ ಹಿಡಿದಿದ್ದರೂ ಹೇಗೆ ಅವಳನ್ನು ಆಡಿಸುವ ಮನಸ್ಸು ಎಲ್ಲೋ ಇದೆಯೋ ಹಾಗೆಯೇ ರಾಮಯಣ ಪಾರಾಯಣದಲ್ಲಿ ಅಕ್ಷರಗಳನ್ನು ಓದುತ್ತಿದ್ದರೂ ವಾಲ್ಮೀಕಿಯ ಹೃದಯದ ವರೆಗೆ ಹೋಗಬೇಕು. ನಮ್ಮ ಮನಸ್ಸನ್ನು ವಾಲ್ಮೀಕಿಯ ಹೃದಯದಲ್ಲಿ ಇಟ್ಟು, ಬೇರೆಡೆಗೆ ಹೋಗದಂತೆ ಅವನಿಗೇ ಸಲ್ಲಿಸತಕ್ಕದ್ದಾಗಿದೆ. ಲೋಕದಲ್ಲಿ ಉದಾಹರಣೆಯಿಂದ ನೋಡಿದಾಗಲೂ ಸಂಸಾರದ ಕೆಲಸಗಳಲ್ಲಿ ತೊಡಗಿದ್ದರೂ ತಾಯಿಗೆ ಮಗು ಎಲ್ಲಿ ಹೋಗಿದೆಯೋ ಎಂದು ನಿಗಾ ಇರುತ್ತೆ. ಹೀಗೆ ಲೋಕದಲ್ಲಿದ್ದರೂ ರಾಮನ ಕಡೆ ಗಮನವಿರಬೇಕು. ರಾಮನ ಅನುಭವವೂ ಬೆರೆತಿದ್ದರೆ ಆ ಅನುಭವ ನಿಲ್ಲಬಹುದು. 

\section*{ರಾಮಾಯಣ ಕಲ್ಪವೃಕ್ಷವಾಗಿದೆ}

ಎಲ್ಲ ವಸ್ತುಗಳಿಗೂ ಆಧಾರವಾದುದು ಭೂಮಿ. ಆಕಾಶದಲ್ಲಿ ಯಾವುದೂ ನಿಲ್ಲುವುದಿಲ್ಲ. ಇಲ್ಲಿನ ಫಲ, ಪುಷ್ಪ, ಎಲ್ಲವು ಅಷ್ಟೇ. ಹಾಗೆ ನಾವು ರಾಮನ ಕಡೆಗೆ ಹೋಗಲೂ ದಾರಿಬೇಕು. ಅಯನವೆಂದರೆ ಮಾರ್ಗ, ಚರಿತ್ರೆ. ಹಾಗೆ ರಾಮಾಯಣ ಪಾರಾಯಣ ಮಾಡುವಾಗ ನಾವೂ ಒಂದು ದಾರಿ ಮಾಡಿಕೊಳ್ಳಬೇಕು. ಆದರಲ್ಲಿ ಗಿಡ, ಮುಳ್ಳು ಇತ್ಯಾದಿಗಳಿದ್ದರೆ ಹೋಗಲಾಗುವುದಿಲ್ಲ. ಸುಮ್ಮನೆ ಪಾರಾಯಣ ಮಾಡಿ ಸಕ್ಕರೆ ಹಂಚಿದರೆ ಬಾಹ್ಯಕ್ಕೆ ಆಯಿತು. ರಾಮಾಯಣದಲ್ಲಿ ಹೊರಗಡೆ ಪಂಚೇಂದ್ರಿಯಕ್ಕೂ ವಿಷಯವಿದೆ. ಒಳಗೆ ಆತ್ಮಾರಾಮನಾಗಿ ಯೋಗಿಯಾಗಲೂ ಅವಕಾಶವಿದೆ. ಹೀಗೆ ಇಂದ್ರಿಯಗಳಿಗೆ, ಮನಸ್ಸಿಗೆ, ಬುದ್ಧಿಗೆ, ಆತ್ಮಕ್ಕೆ  ಎಲ್ಲಕ್ಕೂ ಕಲ್ಪವೃಕ್ಷವಾಗಿದೆ ರಾಮಾಯಣ. 

\section*{ವಾಲ್ಮೀಕಿಗಳ ಹೃದಯದಲ್ಲಿ ಬೆಳಗುತ್ತಿದ್ದ ಹಿರಣ್ಯಗರ್ಭನನ್ನು ಭೌತಿಕವಾಗಿಯೂ ತೋರಿಸಿಕೊಟ್ಟರು ನಾರದರು}

ಹೇಗೆ `ಮಗು ಹಾಗಿದ್ದರೆ ಚೆನ್ನು' ಎಂದು ತಮ್ಮ ಹೃದಯಕ್ಕೆ  ತಂದುಕೊಂಡು ತಮ್ಮ ಬಸುರಿನಲ್ಲಿ ತಂದುಕೊಳ್ಳುವರೋ ಹಾಗೆಯೇ ವಾಲ್ಮೀಕಿಗಳು ಹೃದಯಗರ್ಭದಲ್ಲಿ, ಹಿರಣ್ಯಗರ್ಭನಾಗಿ ಬೆಳಗುವ ಶ್ರೀರಾಮಚಂದ್ರನನ್ನು ಹೆತ್ತು, ಹೊರತಂದರು. ಇಂತಹವರು ಲೋಕ ದಲ್ಲಿವ್ದಾರೆಯೇ? ಎಂಬುದಾಗಿ ನಾರದರನ್ನು ಕೇಳಿದಾಗ `ಅಂಥ ಪುರುಷರು ಅಪರೂಪ, ನಿರ್ದೋಷನಾದ ಪುರುಷನನ್ನು ಭೌತಿಕವಾಗಿಯೇ ಹುಡುಕುವುದು ಕಷ್ಟ, ಇಂಥವರು ವಿರಳ, ಆದರೆ ಇಂತಹವನೊಬ್ಬನನ್ನು ಕೇಳಿದ್ದೇನೆ' ಎಂದು ಶ್ರೀರಾಮನ ಕಥೆಯನ್ನು ಅವರು ವಾಲ್ಮೀಕಿಗಳಿಗೆ ತಿಳಿಸುತ್ತಾರೆ. 

\section*{ಒಳ-ಹೊರ ಜೀವನಗಳನ್ನು ಬೆಸೆದು ಒಂದಕ್ಕೊಂದು ವಿರೋಧಬಾರದಂತೆ ಬದುಕುವುದಕ್ಕಾಗಿ ರಾಮಾಯನ ಪಾರಾಯಣ}

ಎಂದೋ ಹುಟ್ಟಿ ಅಳಿದ ರಾಮನ ಕಥೆಯನ್ನು ಕೇಳಿ ಏನು ಪ್ರಯೋಜನ? ಈ ರಾಮಾಯಣವನ್ನು ಓದಿ ನಮಗಾಗಬೇಕಾದ್ದೇನು? ಎಂದರೆ ಹೊರಭೋಗಕ್ಕೂ ಆತ್ಮಯೋಗಕ್ಕೂ ರಾಮಾಯಣವು ಬೇಕು. ಮನುಷ್ಯನಿಗೆ ಐಹಿಕ-ಪಾರಮಾರ್ಥಿಕವೆಂಬ ಎರಡು ಜೀವನವುಂಟು. ಇವನ್ನು ಬೆಸೆದು ಒಂದಕ್ಕೊಂದಕ್ಕೆ ವಿರೋಧಬಾರದಂತಹ ಆದರ್ಶವು ಅದರಲ್ಲಿರುವುದರಿಂದ ಆ‌ ವಿಚಾರವನ್ನು ಇಟ್ಟಿದ್ದೇವೆ. ಉದಾಹರಣೆಗೆ ಹೇಗೆ ಉಸಿರು ಬಿಡುವುದು, ನಂತರವೇ ಉಸಿರು ತೆಗೆದುಕೊಳ್ಳುವುದು ಎರಡೂ ಇರುವುದೋ ಹಾಗೆ ಒಳ-ಹೊರ ಜೀವನಕ್ಕೆ ವಿರೋಧವಾಗದಂತೆ ರಾಮದೇವರು ರಾಜ್ಯಭಾರ ಮಾಡುತ್ತಿದ್ದರು.

\begin{shloka}
`ಪ್ರಾಣಾಪಾನೌ ಸಮಾವಾಸ್ತಾಂ ರಾಮೇ ರಾಜ್ಯಂ ಪ್ರಶಾಸತಿ|'
\end{shloka}

\section*{ಶ್ರೀರಾಮಾವತಾರ ಕಾರಣ}

ಮುನಿಗಳು `ಕೊರ್ಮೋಂಗಾನೀವ ಸರ್ವಶಃ' ಎಂಬುದಕ್ಕನುಗುಣವಾಗಿ ತಮ್ಮ ಇಂದ್ರಿಯಗಳನ್ನೆಲ್ಲಾ ಹೊರಮುಖ ಮತ್ತು ಒಳಮುಖ ಎರಡೂ ಮಾಡಿಕೊಂಡು ಹೊರಗೆ ಅವನ ಲೋಕದ ಭೋಗ, ಒಳಗೆ ಆತ್ಮಾರಾಮನಾದ ಅವನ ಯೋಗ, ಹೊರಗೆ ಬಂದಾಗ ಅವನು ಯಾವುದನ್ನು ಸೃಷ್ಟಿಸಿದ್ದಾನೋ ಅದನ್ನು ಸನಾತನವಸ್ತುವೇ ತಂದುಕೊಟ್ಟಿದೆ ಎಂದು ಯೋಗದೊಡನೇ ಭೋಗಿಸುವುದು, ಹೀಗಿದ್ದರು. ಹೀಗೆ ಆತ್ಮಾರಾಮರಾಗಿದ್ದು, ಲೋಕಕಲ್ಯಾಣಕ್ಕಾಗಿ ಇದ್ದಂತಹ ಮಹರ್ಷಿಗಳಿಗೆ ಎದುರು ಬಿದ್ದ ಆಸುರೀಶಕ್ತಿಗಳನ್ನು ಸಂಹಾರ ಮಾಡುವುದಕ್ಕಾಗಿ ದೇವತೆಗಳು ಭಗವಂತನನ್ನು ಮೊರೆಯಿಟ್ಟರು. ಈ ಶಕ್ತಿಗಳನ್ನು ದೇವತೆಗಳು ಏಕದೇಶದಲ್ಲಿ ಮಾತ್ರ ತಡೆಯಬಲ್ಲವರಾಗಿದ್ದರು. ಪೂರ್ಣವಾಗಿ ಅಲ್ಲ. ಅದಕ್ಕಾಗಿ ಮೊರೆ `ಪರಿಪೂರ್ಣಸ್ವರೂಪನಾದ ನೀನೇ ಅವತರಸಿ ಕೆಳಗೆ ಬಾರಪ್ಪಾ, ಲೋಕದ ಮನುಷ್ಯರು ಮಾಡಿದರೆ ಸಂಕುಚಿತವಾಗಿ ಏಕದೇಶಕ್ಕೆ ಮಾತ್ರ ವ್ಯಾಪಿಸುತ್ತೆ' ಎಂದು.

\section*{ಗುಣಸಂಪದ್ವಿಭೂಷಣನಾಗಿದ್ದಾನೆ ಶ್ರೀರಾಮ}

ದಿವಿಗೂ ಭುವಿಗೂ ಸೇತುವೆ ಕಟ್ಟಿ ಎಲ್ಲರೂ ಆತ್ಮಾರಾಮರಾಗಿರಲಿ ಎಂಬುದೇ ರಾಮದೇವರ ಅಭಿಪ್ರಾಯ. ಅದನ್ನರಿತವರು ವಾಲ್ಮೀಕಿಗಳು. ಉದಾತ್ತನಾಯಕನಾದ ರಾಮನ ಗುಣಗಳನ್ನು ವಾಲ್ಮೀಕಿಮುನಿಗಳು ಹೀಗೆ ಹೇಳಿದ್ದಾರೆ.

\begin{shloka}
ಧರ್ಮಜ್ಞಃ ಸತ್ಯಸಂಧಶ್ಚ ಪ್ರಜಾನಾಂ ಚ ಹಿತೇ ರತಃ|\\
ಯಶಸ್ವೀ ಜ್ಞಾನಸಂಪನ್ನಃ ಶುಚಿರ್ವಶ್ಯಸ್ಸಮಾಧಿಮಾನ್||\\
ಪ್ರಜಾಪತಿಸಮಃ ಶ್ರೀಮಾನ್ ಧಾತಾ ರಿಪುನಿಷೂದನಃ|\\
ರಕ್ಷಿತಾ ಸ್ವಸ್ಯ ಧರ್ಮಸ್ಯ ಸ್ವಜನಸ್ಯ ಚ ರಕ್ಷಿತಾ||\\
ವೇದವೇದಾಂಗತತ್ತ್ವಜ್ಞಃ ಧನುರ್ವೇದೇ ಚ ನಿಷ್ಠಿತಃ|\\
ಸರ್ವಶಾಸ್ತ್ರಾರ್ಥತತ್ತ್ವಜ್ಞಃಸ್ಮೃತಿಮಾನ್ ಪ್ರತಿಭಾನವಾನ್||\\
ಸರ್ವಲೋಕಪ್ರಿಯಸ್ಸಾಧುರದೀನಾತ್ಮಾ ವಿಚಕ್ಷಣಃ|\\
ಸರ್ವದಾಭುಗತಸ್ಸದ್ಭಿಃ ಸಮುದ್ರ ಇವ ಸಿಂಧುಭಿಃ||\\
ಆರ್ಯಸ್ಸರ್ವಸಮಶ್ಚೈವ ಸದೈಕಪ್ರಿಯದರ್ಶನಃ|
\end{shloka}

ಇಂಥಾ ಗುಣಗಳಿಂದ ಕೂಡಿದ ಶ್ರೀರಾಮಚಂದ್ರನನ್ನು ಅವನ ಮನೋಧರ್ಮ ಸಹಿತವಾಗಿ ನಮ್ಮ ಮುಂದಿಟ್ಟದ್ದಾರೆ. ವಾಕ್ಕಿಗೂ, ಅರ್ಥಕ್ಕೂ ಹೊಂದಿಕೆಯಿದ್ದಾಗ ವ್ಯವಹಾರ 

\begin{shloka}
ವಾಗರ್ಥಾವಿವ ಸಂಪೃಕ್ತೌ ವಾಗರ್ಥಪ್ರತಿಪತ್ತಯೇ|
\end{shloka}
ಅರ್ಥಬಿಟ್ಟು ವಾಕ್ಕಿದ್ದರೆ, ಹಾವನ್ನು ಬಿಟ್ಟು ಸೋಲೆ ತೆಗೆದುಕೊಂಡಂತೆ. ಹಾಗೆ ಶ್ರೀರಾಮಚಂದ್ರನ ಸಾಹಿತ್ಯವನ್ನು ವಾಗರ್ಥದೊಡನೆ ತಂದಿಟ್ಟಿದ್ದಾರೆ, ಅದನ್ನು ನೋಡಿ ಮತ್ತೆ ಮತ್ತೆ ಸವಿಯುವಂತೆ.

ಆಂಜನೇಯರು ಜಗನ್ಮಾತೆಯಾದ ಸೀತಾದೇವಿಯ ಮುಂದೆ -

\begin{shloka}
`ರಾಮಃ ಕಮಲಪತ್ರಾಕ್ಷಃ ಸರ್ವಸತ್ವಮನೋಹರಃ|\\
ರೂಪದಾಕ್ಷಿಣ್ಯಸಂಪನ್ನಃ ಪ್ರಸೂತೋ ಜನಕಾತ್ನಜೇ||\\
ತೇಜಸಾದಿತಯಸಂಕಾಶಃ ಕ್ಷಮಯಾ ಪೃಥಿವೀಸಮಃ|\\
ಬೃಹಸ್ಪತಿಮೋ ಬುದ್ಧ್ಯಾ ಯಶಸಾ ವಾಸವೋಪಮಃ||\\
ರಾಮೋ ಭಾಮಿನಿ ಲೋಕಸ್ಯ ಚಾತುರ್ವರ್ಣ್ಯಸ್ಯ ರಕ್ಷಿತಾ||\\
ಮರ್ಯಾದಾನಾಂ ಚ ಲೋಕಸ್ಯ ಕರ್ತಾಕಾರಯಿತಾ ಚ ಸಃ|\\
ಆರ್ಚಿಷ್ಮಾನರ್ಚಿತೋಽತ್ಯರ್ಥಂ ಬ್ರಹ್ಮಚರ್ಯವ್ರತೇ ಸ್ಥಿತಃ||\\
ಸಾಧೂನಾಮುಪಕಾರಜ್ಞಃ ಪ್ರಚಾರಜ್ಞಶ್ಚ ಕರ್ಮಣಾಂ|\\
ರಾಜವಿದ್ಯಾವಿನೀತಶ್ಚ ಬ್ರಾಹ್ಮಣಾನಾಮುಪಾಸಕಃ||\\
ಶ್ರುತವಾನ್ ಶೀಲಸಂಪನ್ನೋ ವಿನೀತಶ್ಚ ಪರಂತಪಃ|\\
ಧನುರ್ವೇದೇ ಚ ವೇದೇಷು ವೇದಾಂಗೇಷು ಚ ನಿಷ್ಠಿತಃ||\\
ವಿಪುಲಾಂಸೋ ಮಹಾಬಾಹುಃ ಕಂಬುಗ್ರೀವಶ್ಮುಭಾನನಃ|\\
ಗೂಢಜತ್ರುಸ್ಸುತಾಮ್ರಾಕ್ಷೋ ರಾಮೋ ದೇವಿ ಜನೈಶ್ಶ್ರುತಃ||\\
ದುಂದುಭಿಸ್ವನನಿರ್ಘೋಷಃ ಸ್ನಿಗ್ಧವರ್ಣ ಪ್ರತಾಪವಾನ್|\\
ಸಮಸ್ಸಮವಿಭಕ್ತಾಂಗೋ ವರ್ಣಂ ಶ್ಯಾಮಂ ಸಮಾಶ್ರಿತಃ||
\end{shloka}
ಈ ರೀತಿ ವರ್ಣನೆ ಮಾಡಿರುತ್ತಾರೆ. ಕಾಥೆಯ ಉದಾತ್ತನಾಯಕ ಶ್ರೀರಾಮಚಂದ್ರ .

\section*{ಆಸುರೀಸಂಪತ್ತಿನ ನಾಶಕ್ಕಾಗಿ ರಾಮಾಯಣ}

ತಾನೂ ದೈವೀಸಂಪತ್ತಿಗೆ ಹೋಗದೆ, ಉಳಿದವರಿಗೂ ಅವಕಾಶಕೊಡದೇ ತ್ರ್ಯೆಲೋಕ್ಯದಲ್ಲೇ ಕಂಟಕನಾಗಿದ್ದಾನೆ ದಶಮುಖ-ರಾವಣ. ನೀನೇ ಉದ್ಧಾರ ಮಾಡಬೇಕೆಂದು ದೇವತೆಗಳು ಭಗವಂತನಲ್ಲಿ ಮೊರೆಯಿಟ್ಟಾಗ ದೈವೀಸಂಪತ್ತಿನೊಡನೆ ಲೋಕದಲ್ಲಿ ಅವತರಿಸಿ ಆಸುರೀಸಂಪತ್ತನ್ನು ಮೆಟ್ಟಿ ಲೋಕವನ್ನು ಉದ್ಧರಿಸಿದನು. ಹೀಗೆಯೇ, ಮನಸ್ಸಿನ ಹತ್ತುಮುಖಗಳು ಆತ್ಮನನ್ನು ದುರುಪಯೋಗಪಡಿಸಿಕೊಳ್ಳುತ್ತಿರುವಾಗ. ಈ ನಮ್ಮ ಆಸುರೀ ಸಂಪತ್ತನ್ನು ಮೆಟ್ಟಿ ನಮ್ಮ ಆತ್ಮನಿಗೆ ಸಹಜಸಿದ್ಧವಾದ ಸ್ವಾತಂತ್ರ್ಯವನ್ನು ಗಳಿಸಲೋಸುಗವೇ ಈ ರಾಮಾಯಣ.

\section*{ಆತ್ಮವಂತನಾದ ರಾಜನಿದ್ದಾಗಲೇ ಧರ್ಮರಕ್ಷಣೆ}

ಇದನ್ನು ಮರೆಯಬಾರದೆಂದೇ ಪರಂಪರೆಯಾಗಿ ವಿಶಿಷ್ಟಧರ್ಮದೊಡನೆ ರಾಜ್ಯಪರಿಪಾಲನೆ ಮಾಡಿದರು. ಕಾಲದೋಷದಿಂದ ಇಂದು ರಾಜರಿಲ್ಲ. ರಾಜರಲ್ಲಿ ದೋಷವಿದ್ದರೆ ತೆಗೆಯುವುದು ಸರಿ. ಆದರೆ ರಾಜ ಸರಿಯಾಗಿರುವವನು ಬೇಕು. ಏಕೆಂದರೆ-

\begin{shloka}
`ದೇವಾ ಮಾನುಷರೂಪೇಣ ಚರಂತ್ಯತ್ರ ಮಹೀತಲೇ'|
\end{shloka}

ಬಾಲಬುದ್ಧಿಯಿಂದ ರಾಜನನ್ನು ಮನುಷ್ಯನೆಂದು ತಿಳಿಯಬಾರದು. `ಆತ್ಮವಾನ್ ರಾಜಾ' ಎಂಬುದೇ ಮೊದಲಾಗಿ ರಾಜನ ಗುಣಗಳನ್ನು ಹೇಳಿದೆ. `ಯಥಾ ರಾಜಾ ತಥಾ ಪ್ರಜಾಃ' ಎಂಬಂತೆ ಪ್ರಜೆಗಳೈಗೂ ಈ ಗುಣ ಬಂದಾಗ ಇನ್ನೇನು ಹೇಳೋಣ, `ಧರ್ಮೋ ರಕ್ಷತಿ ರಕ್ಷಿತಃ'|

\section*{ರಾಮಾಯಣವನ್ನು ಯೋಗ್ಯಸ್ಥಾನದಲ್ಲಿಡಲು ಭೂಮಿಕೆ ಆವಶ್ಯಕ}

ಅಂಥಾ ರಾಮಾಯನವನ್ನು ನಮ್ಮ ಪಾರಾಯಣಕ್ಕಿಟ್ಟುಕೊಂಡಿದ್ದೇವೆ. ಯಾವ ಯಾವ ವಿಷಯ ಹೇಗಿರಬೇಕೋ ಹಾಗಿದ್ದರೆ ಚೆನ್ನು. ಉದಾಹರಣೆಗೆ ಸಕ್ಕರೆಯನ್ನು ನಾಲಿಗೆಯ ಮಧ್ಯದಲ್ಲಿಟ್ಟರೆ ರುಚಿ ಗೊತ್ತಾಗುವುದಿಲ್ಲ. ಆದ್ದರಿಂದ ಸಕ್ಕರೆಯ ರುಚಿ ಗೊತ್ತಾಗುವ ಕೇಂದ್ರದಲ್ಲಿ ಅದನ್ನಿಡಬೇಕು. ರಾಮಾಯಣವನ್ನು ಎಲ್ಲಿಟ್ಟರೆ ಚೆನ್ನಾಗಿರುತ್ತೆ ಎಂಬ ಬಗ್ಗೆ ಒಂದು ಭೂಮಿಕೆ ಬೇಕು.

\section*{ದುರ್ಜನವಧೆ ಹಿಂಸೆಯಾಗಲಾರದೇ?}

ಉಸಿರೆಳೆದುಕೊಂಡರೆ ಬಿಡಬಾರದು, ಬಿಟ್ಟರೆ ತೆಗೆದುಕೊಳ್ಳಬಾರದು ಎಂಬ ನಿಯಮವಿದ್ದರೆ ಎಷ್ಟು ಕಷ್ಟ ಎರಡೂ ಇದ್ದರೆ ತಾನೇ ವ್ಯವಹಾರ. ಹಾಗೆಯೇ, ಮಹರ್ಷಿಗಳು ಅಂತರಂಗ ಜೀವನದಲ್ಲಿಯೇ ತೃಪ್ತರಾಗಿ ಬಿಟ್ಟು ಬಹಿರಂಗ ಜೀವನವನ್ನೇನೂ ಮರೆತವರಾಗಿರಲಿಲ್ಲ.

\begin{shloka}
ಕೃಷಿಗೋಕ್ಷವಾಣಿಜ್ಯಂ ಲೋಕಾನಾಮಿಹ ಜೀವನಂ|\\
ಊರ್ಧ್ವಂ ಚೈವ ತ್ರಯೀ ವಿದ್ಯಾ ಸಾ‌ ಭೂಆನ್ಭಾವಯತ್ಯುತ ||\\
ತಸ್ಯಾಂ ಪ್ರವರ್ತಮಾನಾಯಾಂ ಯೇಸ್ಯುಸ್ತತ್ಪರಿಪಂಥಿನಃ|\\
ದಸ್ಯವಸ್ತದ್ವಧಾರ್ಥಾಯ ಬ್ರಹ್ಮಾ ಕ್ಷತ್ರಮಥಾಸೃಜತ್||
\end{shloka}
ಧರ್ಮವಿರೋಧಿಗಳಾಗಿರುವವರು ದಸ್ಯುಗಳು. ಅವರನ್ನು ಕೊಲ್ಲಬೇಕು. `ಕೊಲ್ಲುವವರು ಯಾರು? ಏಕೆ ಕೊಲ್ಲಬೇಕು? ಕೊಲ್ಲುವುದು ಪ್ರಾಣಿಹಿಂಸೆಯಲ್ಲವೇ?, ಎಂದರೆ `ರೋಗಹುಟ್ಟಿಕೊಂಡಾಗ ರೋಗವನ್ನೇಕೆ ಕೊಲ್ಲಬೇಕು? ಆರೋಗ್ಯವೂ ಒಂದು ಕಡೆ ಇರಲಿ, ಅನಾರೋಗ್ಯವೂ ಒಂದು ಕಡೆ ಇರಲಿ, ಎರಡೂ, ಇರಲಿ, ಎರಡೂ ಸೃಷ್ಟಿಯಲ್ಲಿರುವುದು ತಾನೇ? ಇರಲಿ, ನಾವು ನೂಟ್ರಲ್ (ಮಧ್ಯಸ್ಥರು)' ಎಂದರೆ ಇದು ತೀರ್ಮಾನವಾಗದ ವಾದ. ಒಬ್ಬ ಸ್ತೀಗೆ ಮಗು, ಹೊಟ್ಟೆನೋವು ಎರಡೂ ಇದೆ. ಮಗುವಿನಂತೆಯೇ ಹೊಟ್ಟೇನೋವೂ ಇವಳ ಹೊಟಟ್ಟೆಯಲ್ಲೇ ಹುಟ್ಟಿದೆ. ಆದೂ ದೈವಸೃಷ್ಟಿ ಯಲ್ಲಿ ಸಹಜವಾಗಿದೆ. ಅವಳು ಅದನ್ನು ರಕ್ಷಿಸಬೇಕು ತಾನೇ? ರಕ್ಷಿಸದಿದ್ದರೆ ಹಿಂಸೆಯಾಗುವುದಿಲ್ಲವೇ? ಹೇಗೆ ಹೊಟ್ಟೇನೋವನ್ನು ರಕ್ಷಿಸಿದರೆ ಮಗುಹಾಲಿಲ್ಲದೆ ಸತ್ತು ಹೋಗುತ್ತದೆ. ಇದೊಂದು ಸಮಸ್ಯೆಯಾದೀತು. ಮಗುವಿನ ರಕ್ಷಣೆಗಾಗಿ ಹೊಟ್ಟೆನೋವನ್ನು ಓಡಿಸಲೇಬೇಕು.

ಸನಾತನಿಗಳಾದ ಆರ್ಯಮಹರ್ಷಿಗಳ ಕಟ್ಟುಪಾಡು ಮನಸೋ ಇಚ್ಛೆಯಿಂದಲ್ಲ. ಅದು ದೈವದ ಕಟ್ಟುಪಾಡು. ದೈವದ ಅಪ್ಪಣೆಗೆ ಅನುಕೂಲವಾದದ್ದೇ ಧರ್ಮ. ವಿರೋಧವಾದದ್ದೇ ಅಧರ್ಮ. ಅದನ್ನು ಮೇಲೆ ಹಾಕಿಕೊಳ್ಳಬೇಡಿ. ಅಂಥಾ ಧರ್ಮವು ಮೂರ್ತಿಮತ್ತಾಗಿ `ಯದಾ ಯದಾ ಹಿ ಧರ್ಮಸ್ಯ' ಎಂದು ಯಾವನು ಹೇಳಿದನೋ ಆ ಮೂರ್ತಿಯೇ ಅರ್ಜುನನನ್ನು ಕುರಿತು `ಕೇವಲ ರಾಜ್ಯವನ್ನು ಕಟ್ಟಬೇಡಪ್ಪಾ. ಅದರೊಡನೆ ಆತ್ಮನನ್ನೂ ಸೇರಿಸಿ ಕಟ್ಟು. ಎಂದು ಹೇಳಿ ಆನ್ವೀಕ್ಷಕೀ ಎಂಬ ಸಾಂಖ್ಯಯೋಗ ಶಾಸ್ತ್ರವನ್ನೂ ಹೇಳುತ್ತಾನೆ.

ರಾಜನಿಗೆ ಎರಡು ಬಗೆಯ ರಾಜ್ಯವುಂಟು. ಆ ರಾಜ್ಯದಲ್ಲಿ ಅಮರರೆಲ್ಲಾ ಬೇಕು. ಪ್ರಜೆಗಳ ರಾಷ್ಟ್ರವೆಲ್ಲಾ ಪರಮವ್ಯೋಮದಲ್ಲಿದೆ. ಹೀಗಾಗಿ ಪ್ರಜೆಗಳು ತೋರಿಂದತೆ ಹೋಗದ ಹಾಗೆ ದಂಡ. ಹಾಗೆ ಕೃಷ್ಣನ ಕೈಯಲ್ಲಿ ದಂಡ (ಚಾವಟಿ). ಮತ್ತೊಂದಿ ಕೈಯಲ್ಲಿ ಜ್ಞಾನಮುದ್ರೆ -

\begin{shloka}
ಕರಕಮಲನಿದರ್ಶಿತಾತ್ಮಮುದ್ರಃ\\
ಪರಿರಲಿತೋನ್ನತ ಬಱಿಬಱಚೂಡಃ|\\
ಇತರಕರಗೃಹೀತ ವೇತ್ರತೋತ್ರಃ\\
ಮಮ ಹೃದಿ ಸನ್ನಿಧಿಮಾತನೋತು ಶೌರಿಃ||
\end{shloka}

ಅಲ್ಲಿಯೂ ಕೂಡ ಈ ದುಷ್ಟರ ಬಾಧೆ ಜಾಸ್ತಿಯಾಗಿ `ನಾಶಮಾಡಯ್ಯಾ ಎಂದರೆ ಅರ್ಜುನ ಆಸ್ಥಾನದಲ್ಲಿ, ಆಸಮಯದಲ್ಲಿ ಕರುಣೆ ತೋರಿಸುತ್ತಾನೆ ದುರ್ಯೋಧನನಲ್ಲಿ. ಹಾಗಾದರೆ ಯಾವವಸ್ತು ನಾಶಮಾಡಲು ಯೋಗ್ಯವಾಗಿದೆಯೊ ಅದನ್ನು ನಾಶಮಾಡಬೇಕು. ೧೩ ದೋಷಗಳು ದುರ್ಯೋಧನನಲ್ಲಿವೆ. ಆದ್ದರಿಂದ ಅವನು ಲೋಕಕಂಟಕ. `ನೀನು ಮಾಡುವೆಯೋ ನಾನು ಮಾಡಿಸಲೋ' ಎಂದು ತನ್ನ ಸುದರ್ಶನವನ್ನು ಹಿಡಿದು ನಿಲ್ಲುತ್ತಾನೆ. ಅಂಥ ಪುರುಷನಿಗೆ ಅಗ್ರಪೂಜೆ ಸಲ್ಲಿಸುವಾಗ ವಿರೋಧಿಸಿ ಹತನಾದ ಶಿಶುಪಾಲ. 

\section*{ವಿಶ್ವಚಕ್ರದಲ್ಲಿಯೇ ವರ್ತಿಸಬಲ್ಲವನು ಭಗವಂತ}

`ರಾಷ್ಟ್ರನಾಯಕರಿಗೂ ರಾಜರಿಗೂ ಏನು ವ್ಯತ್ಯಾಸ?' ರಾಜನು ಭೂಪತಿ ಪೃಥಿವೀಪತಿ ಏನಯ್ಯಾ ವ್ಯತ್ಯಾಸ? ಎಂದರೆ ಪರ್ಯಾಯ ಆದರೆ ಅವನು ಭೂಪತಿ ಮಾತ್ರನಲ್ಲ, ಜಗತ್ಪತಿ, ಆದ್ದರಿಂದಲೇ ರಾಮನಾಗಲೀ ಕೃಷ್ಣನಾಗಲೀ, ಪಾರಾಯಣ ಮಾಡುವಾಗ ಅದರಬಗ್ಗೆ ಆದರಬೇಕು. ಆದ್ದರಿಂದಲೇ ಉತ್ತಮ ನಾಮಾವಳಿ. ವಿರಾಟ್, ಸಂರಾಟ್, ರಾಜಾ, ಏಕರಾಟ್, ಚಕ್ರವರ್ತಿ ಎಂದಾಗ ಈ ಭೂಮಿ ಮಾತ್ರವೇ ಅಲ್ಲ, ಇಡೀ ವಿಶ್ವಚಕ್ರದಲ್ಲಿಯೇ ವರ್ತಿಸಬಲ್ಲವನು ಎಂದು ತಿಳಿಯುತ್ತೆ. 

\section*{ವಾಲ್ಮ್ಮೀಕಿ ಕೋಗಿಲೆಯ ಗಾನಾಮೃತವರ್ಷದಿಂದ ತಾಪಶಾಂತಿ}

ಪಕ್ಷಿಗಳು ನಮ್ಮನ್ನು ದ್ವಿಜರೆಂದು ಹೇಳಿರೆಕ್ಕೆ ಅಂಟಿಸಿ `ಹಾರುವಯ್ಯಾ ಹಾರು' ಎಂದರೆ ಹೇಗೆ ಹಾರುವಿರಿ? ಇಲ್ಲಿಯೂ ವಾಲ್ಮೀಕಿ ದ್ವಿಜ. ಆದರೆ ರೆಕ್ಕೆ ಪುಕ್ಕ ಅಂಟಿಸಲಿಲ್ಲ. ವಾಲ್ಮ್ಮೀಕಿ ಕೋಗಿಲೆ ಎಂದಾಗ ರೆಕ್ಕೆ, ಪುಕ್ಕ, ಅಷ್ಟನ್ನು ತೆಗೆದು ಕೊಳ್ಳುವುದಲ್ಲ. ಸ್ವರವನ್ನು, ಮಧುರವಾದ ವಾಲ್ಮೀಕಿಕೋಕಿಲೆಯ ನಾದವನ್ನು ನಾವು ಗ್ರಹಿಸಬೇಕು. ರಾವಣ ರಾಕ್ಷಸಪತಿ. ರಾಮ ಲೋಕಪತಿ. ರಾಕ್ಷಸಪತಿಯಲ್ಲಿಲ್ಲದ ಗುಣಗಳು ಲೋಕಪತಿಯಲ್ಲುಂಟು. ಆದ್ದರಿಂದ ಆದರ್ಶನಾಯಕನನ್ನು ನಾವಿಟ್ಟು ಕೊಳ್ಳಬೇಕು. ಅಂಥಾ‌ ನಾಯಕನನ್ನು ಒಳಗೂ ಹೊರಗೂ ಇಟ್ಟುಕೊಂಡು ವಾಲ್ಮ್ಮೀಕಿಕೋಗಿಲೆ ಹಾಡಿತು. ಅಜ್ಞಾನದ ಬಿರುಬೇಗೆಯಲ್ಲಿ ಸಿಕ್ಕಿ ನರಳುವವರಿಗೆ ವಸಂತಮಾಧವ ಬಂದಾಗ ವಾಲ್ಮೀಕಿಕೋಗಿಲೆ ಕೂಗಿ ಆನಂದರ್ಷದಿಂದ ತಾಪಶಾಂತಿಯಾಯಿತಿ.

\section*{ರಾಷ್ಟ್ರದೇಶ ಜನಪದಗಳನ್ನು ಕುರಿತು}

ಋಷಿಗಳು ರಾಷ್ಟ್ರ, ದೇಶ, ಜನಪದ ಇವುಗಳ ಕನ್ನೆಪ್ಷನ್ ({\eng Conception}) ಹೇಗಿಟ್ಟಿದ್ದಾರೆ?

\begin{shloka}
`ವಶುಧಾನ್ಯಹಿರಣ್ಯಸಂಪದಾ ಶೋಭತೇ-ರಾಜತೇ ಇತಿ ರಾಷ್ಟ್ರಂ|\\
ಭರ್ತುರ್ದಂಡಕೋಶವೃದ್ಧಿಂ ದಿಶತಿ - ದದಾತೀತಿ ದೇಶಃ|\\
ಜವಸ್ಯ ವರ್ಣಾಶ್ರಮಲಕ್ಷಣಸ್ಯ, ದ್ರವ್ಯೋತ್ಪತ್ತೇರ್ವಾ ಪದಂ -\\
ಸ್ಥಾನಮಿತಿ ಜನಪದಃ|
\end{shloka}

\section*{ರಾಮಾಯಣವು ವೇದವೇ ಆಗಿದೆ}

ತಮ್ಮ ಹೃದಯಗರ್ಭದಲ್ಲಿ ವಾಲ್ಮಿಕಿಗಳು ಅಂತಹ ಋಷಿಸಮೇತನಾದ ಶ್ರೀರಾಮಚಂದ್ರನಿಗೆ ಅಮೃತಾಭಿಷೇಚನ ಮಾಡಿದ ನಂತರ ಆ ಅಮೃತರಸವನ್ನು ಸವಿದ ನಂತರವೇ ಸರಸ್ವತಿಯು ಬ್ರಹ್ಮನ ಇಚ್ಛೆಗೆ ವಿರೋಧವಿಲ್ಲದೆ ಹೊರಪಟ್ಟಿದ್ದರಿಂದಲೇ -

\begin{shloka}
`ವೇದಃ ಪ್ರಾಚೇತಸಾದಾಸೀತ್ಸಾಕ್ಷಾದ್ರಾಮಾಯಣಾತ್ಮನಾ'|
\end{shloka}
ಎಂದು ವಾಲ್ಮೀಕಿಗಳಿಂದ ರಾಮಚರಿತವಾಗಿ ಬಂದಿದೆ. ವೇದ ಬೇರೆಯಲ್ಲ ಇದು ಬೇರೆಯಲ್ಲ. ವೇದವೆಂದರೆ ಜ್ಞಾನ ಅದು ಅವ್ಯಕ್ತವಾಗಿದ್ದಾಗ ಹೇಗೆ? ವ್ಯಕ್ತವಾಗಿದ್ದಾಗ ಹೇಗೆ? ಗಮನಿಸಬೇಕು ಒಳಹೊಕ್ಕು ನೋಡಿದಾಗ ವೇದದಿಂದ ದೊರೆಯುವ ವಿಷಯಗಳೆ. ವಾಕ್ಯ, ಪದ ಬೇರೆಯಿರಬಹುದು. ಅರ್ಥಗಳನ್ನು ಬಿಟ್ಟು ಶಬ್ದರಾಶಿ ಹೇಳು' ಎಂದಾಗ ಬೇರಾಗುತ್ತದೆ. ಉದಾಹರಣೆಗೆ `ಅಯ್ಯಾ ಬಾರಯ್ಯ ಇಲ್ಲಿ'ಎಂಬುದರ ಉಚ್ಚಾರಣೆ ಅರ್ಥವನ್ನು ಬಿಟ್ಟಾಗ ಕೇವಲ ಶಬ್ಬವನ್ನು ಹೇಳಿದಾಗ ಹೋಗು ಎನ್ನುವ ಅರ್ಥವನ್ನು ಕೊಡಬಹುದು. ಹಾಗಾಗಬಾರದು.

\section*{ರಾಮಚರಿತಗಾನಕ್ಕೆ ರಾಮಪುತ್ರರೇ ಯೋಗ್ಯರು}

ರಾಮನ ಒಂದು ಚರಿತವನ್ನು `ಆತ್ಮಾ ವೈ ಪುತ್ರಮಾಮಾಸಿ' ಎಂಬಂತೆ ಹೇಳಲು ಲವಕುಶರೇ ತಕ್ಕವರು.

\begin{shloka}
`ಬಿಂಬಾದಿವೋದ್ಧೃತೌ ಬಿಂಬಬೌ ರಾಮದೇಹಾತ್ತಥಾಪರೌ'|
\end{shloka}
ಎಂಬಂತೆ ರಾಮನ ಪ್ರತಿರೂಪಗಳಂತಿರುವವರು. ತಂದೆಯ ಒಂದು ಗುಣಾವಳಿ, ಮೇಲೆ ಮುನಿಗಳಲ್ಲಿ ಶಿಕ್ಷಣವನ್ನು ಪಡೆದವರು (ಚಿ||ಗುಂಡಣ್ಣನನ್ನು ನಿರ್ದೇಶಿಸಿ) ದಾಶರಥಿ ಇಲ್ಲಿಯೂ ಕುಳಿದ್ದಾನೆ. ಇವನ ಮಕ್ಕಳು ಲವಕುಶರಾಗೊಲ್ಲ, ರಾಮಾಯಣ ಹಾಡಲಾರರು. (ಎಂದು ತಮಾಷೆ ಮಾಡಿದರು) ಇನ್ನು ಅವನವನ ಮನಸ್ಸು, ಬುದ್ಧಿಗನುಗುಣವಾಗಿ ಅವನವನ ಜೀವನ. ಅಷ್ಟನ್ನು ಜ್ಞಾಪಿಸುತ್ತೇನೆ. ಇನ್ನು ವಾಲ್ಮೀಕಿಕೋಕಿಲೆಯನ್ನು ಗಾನಮಾಡಲು ಅನುಮತಿ ಕೇಳುತ್ತೇನೆ. (ಸ್ತೋತ್ರಗಾನ ಮಾಡಿದರು)

\section*{(ಇನ್ನೊಂದು ಪ್ರಸಂಗದಲ್ಲಿ ರಾಮಾಯಣಪಾರಾಯಣದ ಕುರಿತು ಅನುಗ್ರಹಿಸಿದ ವಿಚಾರಗಳು)}

\section*{ರಾಮಾಯಣವನ್ನು ಕುರಿತು ಕೆಲವು ಮುಖ್ಯಾಂಶಗಳು}

೧) ಪ್ರಜಾಪ್ರಭುತ್ವವೆಂದು ಹೇಳುತ್ತೀರಿ. ಪ್ರಭುವನ್ನು ಏಕೆ ತೆಗೆದು ಹಾಕುತ್ತೀರಿ? ಪ್ರಭು ಇಲ್ಲದೆ ಪ್ರಜಾಪ್ರಭುತ್ವ ಎಂತಹುದು? ಪ್ರಜೆ, ಪ್ರಭುತ್ವ ಎರಡೂ ಸೇರಿದರೆ ತಾನೇ ಪ್ರಜಾಪ್ರಭುತ್ವವಾಗುವುದು. ಪ್ರಭುತ್ವವನ್ನು ತೆಗೆದರೆ ಪ್ರಜಾಪ್ರಭುತ್ವವೆಲ್ಲಿ ಬಂತು? ದೇಶದೋಷಗಳಾವುವು? ಜನಪದಗುಣಗಳಾವುವು? ಎಂಬ ಬಗ್ಗೆ ನಮಗೆ ತಾರತಮ್ಯಜ್ಞಾನವಿಲ್ಲ. 

೨)ರಾಮಾಯಣ ಒಳ್ಳೆಯದಾಗಿಬಹುದು. ಅದನ್ನು ನಾವು ಇಡುವ ಸ್ಥಳದಲ್ಲಿ ಇಡದಿದ್ದರೆ ಕಿವಿಯ ಆಭರಣವನ್ನು ಗಡ್ಡಕ್ಕೆ ಕಪ್ಪನ್ನು ಕಿವಿಗೆ ಇಟ್ಟರೆ ಏನು ಲಾಭವಾದೀತು?  

೩)ಬಾಹುಪರಾಕ್ರಮವನ್ನು ಅವನು ಎಂದೂ ತೋರಿದವನಲ್ಲ. ಸತ್ಯ ಪರಾಕ್ರಮವನ್ನು ಮಾತ್ರ ತೋರಿದವನು. ಅದನ್ನು ಉದ್ದಕ್ಕೂ ಹೇಳುತ್ತಾರೆ.

೪)`ಗೋಬ್ರಾಹ್ಮಣೇಭ್ಯಃ' ಎಂಬುದನ್ನು `ಗೋಸಜ್ಜನೇಭ್ಯಃ' ಎಂದು ಪಾಠಾಂತರ ಮಾಡಿದರು. ನಿಜವರಿಯದ ವರ್ತನೆಯಿದು.

೫)ರಾಮಾಯಣದ ಪುಸ್ತಕಕ್ಕಿಂತ ವ್ಯಾಸಪಿಠಿಕೆಯೇ ದೊಡ್ಡಬಾಯಿತು, ಎಂಬಂತೆ ಆಗಿದೆ. 

೬)ನಮ್ಮ ಅನಾತ್ಮಶ್ರೀಯನ್ನು ತೊಲಗಿಸಿ ಆತ್ಮಶ್ರೀಯನ್ನು ಹೆಚ್ಚಿಸಿಕೊಳ್ಳಲು (ಬೆಳೆಸಿಕೊಳ್ಳಲು) ಪಾರಾಯಣ. 
