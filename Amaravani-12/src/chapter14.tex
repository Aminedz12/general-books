\chapter{ರಾಮಾಯಣ - ೧}

(ಶಿವರಾತ್ರಿಯ ಶುಭಪರ್ವದಂದು ಶ್ರೀಗುರುವು ಅನುಗ್ರಹಿಸಿದ ಪಾಠ. ಉಪಸ್ಥಿತರಿದ್ದವರು ಶ್ರೀ ವರದದೇಶಿಕಾಚಾರ್ಯ ರಂಗಪ್ರಿಯರು ಹಾಗು ಎಚ್. ವೆಂಕಟೇಶಯ್ಯನವರು.)

\section*{ರಾಮಾಯಣದಲ್ಲಿ  ಚೇತನನಿಗೂ ಇಂದ್ರಿಯಗಳಿಗೂ ಬೇಕಾದ ರಸವಿದೆ}

ನಾರಾಯಣನನ್ನು ನರನ್ನಾಗಿ ಅವತಾರ ಮಾಡಿಸಿ ನರನನ್ನು  ನಾರಾಯಣನ ಪದಕ್ಕೆ ಕೊಂಡೊಯ್ಯುವ ಧರ್ಮಸೇತುವಾಗಿದೆ  - ಈ ಮಹಾಕಾವ್ಯ. ಇದರಲ್ಲಿ ಅಂತರ್ದೃಷ್ಟಿಗೂ ವಿಷಯವಿದೆ, ಬಾಹ್ಯದೃಷ್ಟಿಗೂ ವಿಷಯವಿದೆ. ಇದು ಮಹಾಪುರುಷ ಸೂಕವಾಗಿದೆ. ಇಂದ್ರಿಯಗಳೆಂಬ ಕುದುರೆಗಳು ಚೇತನನನ್ನು ಹೊತು ತರುತ್ತವೆ. ಅವೆರಡಕ್ಕೂ (ಇಂದ್ರಿಯಾಶ್ವ- ಚೇತನ) ಆಹಾರ ಕೊಡುತ್ತಾರೆ ವಾಲ್ಮೀಕಿಗಳು. ಇಂದ್ರಿಯಗಳಿಗೆ ಬೇಕಾದ ರಸವೂ ಇದೆ. ದೇವತೆಗಳಿಗೆ ಬೇಕಾದ ಆಜ್ಯಾಹುತಿಯೂ ಇದೆ. ಆತ್ಮನಿಗೆ ಬೇಕಾದ ಶಾಂತರಸವೂ ಇದೆ. ನೀನು ಗಾಡಿಯಲ್ಲಿ ಪ್ರಯಾಣ ಮಾಡಿರುವುದುಂಟಲ್ಲವೆ? ಮೋಟಾರಿನ ಆಹಾರ ಪೆಟ್ರೋಲ್ ಪ್ರಯಾಣಿಕನ ಆಹಾರ ಹಣ್ಣು. ನೀವು ಅದರಲ್ಲಿ ಪ್ರಯಾಣ ಮಾಡಿ ಬಂದಾಗ  ಯಜಮಾನನು ಇಬ್ಬರಿಗೂ ಉಪಚಾರ ಮಾಡುವನಲ್ಲವೇ? ಹಾಗೆಯೇ ವಾಲ್ಮೀಕಿ ಭಗವಂತನೂ. 

\section*{ನಿತ್ಯಸತ್ಯವಾಗಿದೆ ರಾಮಾಯಣ}

ಲೋಕಸೃಷ್ಟಿಗೆ ಯಾವನು ಕಾರಣನೋ ಈ ಆದಿಕಾವ್ಯದ ಸೃಷ್ಟಿಗೂ ಆತನೇ ಪ್ರೇರಕ.

ರಾಮಾಯಣಪಾರಾಯಣ, ರಾಮಪಟ್ಟಾಭಿಷೇಕ, ಸಾಮ್ರಾಜ್ಯಪಟ್ಟಾಭಿಷೇಕ ಇದೆಲ್ಲಾ ಸಂಭ್ರಮದ ದಿನ. ಆದರೆ ನಡೆಯುವುದೇನು? ರಾಮಾಯಣ ಪುಸ್ತಕದ ಬಗ್ಗೆ ಆಸ್ತಿಯಾಗಿ ಪರಿಗಣನೆ. ಹಂಚಾಟಕ್ಕಾಗಿ ಕಚ್ಚಾಟ. ಬಾಲಕಾಂಡ ಒಬ್ಬರಿಗೆ, ಅಯೋಧ್ಯಾಕಾಂಡ ಒಬ್ಬರಿಗೆ, ಯುದ್ಧಕಾಂಡ ಒಬ್ಬರಿಗೆ ಹಂಚಾಣಿಕೆಯಲ್ಲಿ ಮತ್ತೊಂದು ಯುದ್ಧಕಾಂಡ. 

ಪುಸ್ತಕಾಧಾರದ ಮೇಲೆ ನಿಂತಿದೆ. ನಿತ್ಯವೂ ಸತ್ಯವೂ ಶುದ್ಧವೂ ಆದ ವಿಷಯದ ಆಧಾರದ ಮೇಲೆ ನಿಲ್ಲಬೇಕು. ನಿಮ್ಮ ಜೀವನದ ಮೇಲೆ, ಪರಮಾತ್ಮನ ಮೇಲೆ ನಿಲ್ಲಬೇಕು, ಪುಸ್ತಕದ ಮೇಲಲ್ಲ. ಪುಸ್ತಕ ಹೋದರೆ ರಾಮಾಯಣವೇ ಇಲ್ಲವೆಂದಾಗುವುದಿಲ್ಲ. ಪಂಚಾಂಗ ಹೋದರೆ ನಕ್ಷತ್ರ ಹೋಯಿತೇ? ಸತ್ಯ, ಶಾಶ್ವತ. 

\begin{shloka}
`ನ ತೇ ವಾಗನೃತಾ ಕಾವ್ಯೇ ಕಾಚಿದತ್ರ ಭವಿಷ್ಯತಿ'|\label{209}\\
`ಯಾವತ್ಸ್ಥಾ ಸ್ಯಂತಿ ಗಿರಿಯಃ ಸರಿತಶ್ವ ಮಹೀತಲೇ|\label{210b}\\
ತಾವದ್ರಾಮಾಯಣ ಕಥಾ ತ್ವಕ್ಕೃತಾ ಪ್ರಚರಿಷ್ಯತಿ'||
\end{shloka}

ಇದು ಬ್ರಹ್ಮನ ಮಾತು.

\section*{ಆಧಿವ್ಯಾಧಿನಿವಾರಣೆಗಾಗಿ ರಾಮಾಯಣ ರಚನೆ}

ಇಂತಹ ರಾಮಾಯಣವನ್ನು ಅವರ ಅಶಯವನ್ನು ಮನಸ್ಸಿಗೆ ತಂದುಕೊಂಡು ಸರಿಯಾಗಿ ಓದಬೇಕು. ಹೆಸರು ವೈದೇಹಿ. ಅದನ್ನು ವೈ-ದೇಹಿ ಎಂದರೆ ಅರ್ಥ ಕೊಡುತ್ತದೆಯೇ? ವ್ಯಾಧಿಯ ನಿವಾರಣೆಗಾಗಿ ಮತ್ತು ನಾಲಿಗೆಯ ರುಚಿಗಾಗಿ ವೈದ್ಯನು ಒಂದು ತರಹ ಯೋಗದಿಂದ ಹರೀತಕೀ ರಸಾಯನ, ಜೀರಕಾದಿ ಲೇಹ್ಯ ಮುಂತಾದ ಔಷಧಿಗಳನ್ನು ಮಾಡಿಕೊಡುವಂತೆ, ಪೂರ್ಣಮನೋಧರ್ಮವುಳ್ಳ ವಾಲ್ಮೀಕಿ ಭಗವಂತನೂ ಒಂದು ತರಹದ ಯೋಗದಿಂದ ರಾಮಾಯಣವನ್ನು ಆಧಿ ನಿವಾರಣೆ ಮತ್ತು ಚತುರ್ವಿಧ ಪುರುಷಾರ್ಥಸಿದ್ಧಿಗಾಗಿ ರಚಿಸಿದರು. ಒಂದೇ ಅಕ್ಕಿಯಿಂದ ಪೊಂಗಲ್, ಪುಳಿಯೋಗರೆ ಮುಂತಾದ ಆಹಾರಗಳನ್ನು ಉಪ್ಪು, ಹುಳಿ, ಖಾರ, ಮುಂತಾದ ರಸಗಳ ಸೇರುವೆಯಿಂದ ರುಚಿಕರವಾಗಿರುವಂತೆಯೂ, ಅನಾಮಯವಾಗಿರುವಂತೆಯೂ ಗೃಹಿಣಿ ತಯಾರಿಸುವಂತೆ, ನವರಸಗಳ ಮೇಳನದಿಂದ ವಾಲ್ಮೀಕಿಗಳು ಚೇತರನ್ನು ಅನಾಮಯರನ್ನಾಗಿಮಾಡಿ ಆಧಿವ್ಯಾಧಿರಹಿತರಾಗಿ ಇಡುವ ರಾಮಾಯಣವನ್ನು ರಚಿಸಿದ್ದಾರೆ. ಒಬ್ಬೊಬ್ಬರಿಗೆ ಒಂದೊಂದು ತರಹದ ಆಹಾರ ರುಚಿಸುವಂತೆ ಒಬ್ಬೊಬ್ಬರಿಗೆ ಒಂದೊಂದು ತರಹದ ರಸ ರುಚಿಸುತ್ತದೆ.

\section*{ರಾಮಾಯಣದ ಉಗಮ ಯೋಗಭೂಮಿಯಲ್ಲಿ}

`ಏನೋ ಕಣ್ಮುಚ್ಚಿಕೊಂಡು ಬರೆದ ವಾಲ್ಮೀಕಿ. ನೀವೇಕೆ ಅದನ್ನು ನಂಬುತ್ತೀರಿ?' ಎಂದರೆ ಹೌದು,  ಕಣ್ಮುಚ್ಚಿಕೊಂಡರೇನೇ ರಾಮಾಯಣ ತಿಳಿಯುವುದು. ಇದಕ್ಕೆ  ಬೇಕಾದ ವಿಷಯವನ್ನು ಯೋಗಭೂಮಿ, ತಪೋಭೂಮಿ, ಧ್ಯಾನಭೂಮಿಯಲ್ಲಿ ಕಂಡುಕೊಂಡರು ವಾಲ್ಮೀಕಿಗಳು. 

\begin{shloka}
`ತತಃ ಪಶ್ಯತಿ ಧರ್ಮಾತ್ಮಾ ತತ್ಸರ್ವಂ ಯೋಗಮಾಸ್ಥಿತಃ|\label{210a}\\
`ತತ್ಸರ್ವಂ ಧರ್ಮವಿರ್ಯೇಣ ಯಥಾವತ್ಸಂಪ್ರಪಶ್ಯತಿ|\label{210c}
\end{shloka}
ಎಂದು ರಾಮಾಯಣದ ಉಗಮವನ್ನು ಹೇಳಿದ್ದಾರೆ.

\section*{ರಾಮಾಯಣದ ಫಲ}

ರಾಮಾಯಣದ ಫಲ-

\begin{shloka}
`ಆಯುಷ್ಯಂ ಪುಷ್ಟಿಜನಕಂ'\label{210}\\
`ಇದಂ ಪವಿತ್ರಂ ಪಾಪಘ್ನಂ ಪುಣ್ಯಂ ವೇದೈಶ್ವ ಸಮ್ಮತಮ್'|\label{211h}\\
`ಯಃ ಕಣಾಂಜಲಿಸಂಪುಟೈರಹರಹಃ ಸಮ್ಯಕ್ಪಿಬತ್ಯಾದರಾತ್\label{211e}\\
ವಾಲ್ಮೀಕೆರ್ವದನಾರವಿಂದಗಳಿತಂ ರಾಮಾಯಣಾಖ್ಯಂ ಮಧು|\\
ಜನ್ಮವ್ಯಾಧಿಜರಾವಿಪತ್ತಿ ಮರಣೈರತ್ಯಂತ ಸೋಪದ್ರವಂ\\
ಸಂಸಾರಂ ಸ ವಿಹಾಯ ಗಚ್ಛತ್ತಿಪುಮಾನ್ವಿಷ್ಣೋಃ ಪದಂ ಶಾಶ್ವತಮ್'||
\end{shloka}
ಎಂದು. ಜನ್ಮ, ವ್ಯಾಧಿ, ಜರಾ, ವಿಪತ್ತಿ, ನಿಧನಗಳನ್ನು ಹೇಳುವ ಕಥೆಯನ್ನೇ ಕೇಳಿದರೆ ನಮಗೆ ಇವುಗಳಿಂದ ಬಿಡುಗಡೆಯಾಗಿ ಶಾಶ್ವತ ವಿಷ್ಟುಪದ ಪ್ರಾಪ್ತಿ ಹೇಗೆ?

ಸ್ವಯಂ ಕಥಾನಾಯಕನನ್ನೆ ಮೈಮರೆಸಿ, ಅವನಿಗೇ ಭೂತಿಕರವಾದ ಕಾವ್ಯ-

\begin{shloka}
`ಮಮಾಪಿ ತದ್ಭೂತಿಕರಂ ಪ್ರಚಕ್ಷತೇ\label{211d}\\
ಮಹಾನುಭಾವಂ ಪರಿತಂ ನಿಬೋಧತ'|\\
`ಬುಭೂಷಯಾಸಕ್ತಮನಾ ಬಭೂವಹ'|\label{211c}
\end{shloka}
ಜ್ಞಾನಿಗಳ ಕಾವ್ಯವನ್ನು ಸರಿಯಾಗಿ ಮನನ ಮಾಡುತ್ತಿದ್ದರೆ ನಮ್ಮ ಮನಸ್ಸು ಜ್ಞಾನಾಕಾರಾವನ್ನು ತಾಳುತ್ತದೆ. ತಾದಾತ್ಮ್ಯ ಹೊಂದುತ್ತದೆ.

\section*{ರಾಮಾಯಣ ಮಹಾತ್ಮರಿಗೂ ಪರಮಾನಂದದಾಯಕ}

ರಾಮಾಯಣದ ಬಗ್ಗೆ ಯಾರಿಂದ ಎಲ್ಲಿ ಪ್ರಶಂಸೆ ನಡೆಯಿತು? ಎಂದರೆ, ಇಂದ್ರಿಯಗಳಲ್ಲಿ ಒಡಾಡುವ ಪಾಮರರಿಂದ ಮೆಚ್ಚಿಗೆ ಪಡೆಯುವುದು ಸುಲಭ. ಆದರೆ `ತಪಃ ಶ್ಲಾಘ್ಯೈಮಹಾತ್ಮಭಿಃ'\label{211} ಪ್ರಶಂಸೆ ಪಡೆಯುವುದು ಕಷ್ಟ `ತಪಸ್ವಿನಾಂ, ದ್ವಿಜಾತೀನಾಂ, ಸಾಧೂನಾಂ ಚ ಸಮಾಗಮೇ' ಅವರಿಗೆ ಪರಂಭಾವವನ್ನುಂಟು ಮಾಡಿದ್ದರಿಂದಲೇ `ಪರಂ ಮಿಸ್ಮಯಮಾಗತಾಃ'\label{211a} ಅಂತಹವರಿಗೂ ಆಶ್ಚರ್ಯ! ಪರಮಾನಂದ! ಆದ್ದರಿಂದಲೇ `ಬಾಷ್ಪಪರ್ಯಾಕುಲೇಕ್ಷಣಾಃ'\label{211b} ಸಾಧುಸಾಧ್ವಿತಿ ತಾವೂಚುಃ'\label{211g}

\section*{ರಾಮಾಯಣ ರಚನೆಗೆ ಹಿನ್ನೆಲೆಯಾದ ಘಟನೆಗಳು}

ನಾರದರಿಂದ ಕೇಳಿದ ಸನಾತನಾದ ರಾಮನ ಗುಣವನ್ನೇ ಧ್ಯಾನಿಸುತ್ತಿದ್ದುದರಿಂದ-

`ರಮಣೀಯಂ ಪ್ರಸನ್ನಾಂಬು\label{211f} ಸನ್ಮನುಷ್ಯಮನೋ ಯಥಾ'| ಸೊಂಪಾದ ರಮಣೀಯ ವನರಾಜಿ. ತಿಳಿನೀರಿನ ತಮಸಾನದಿ. ಕ್ರೌಂಚಗಳ ಚಾರನಿಃಸ್ವನ. ಆಗ ಬೇಡನಿಂದನಡೆದ ಕ್ರೌಂಚವಧೆ, ವಾಲ್ಮೀಕಿಯ ಶಾಪಕ್ಕೆ ಕಾರಣ. `ಮಾನಿಷಾದ ಪ್ರತಿಷ್ಠಾಂತ್ವಂ'. ಅಧರ್ಮಾಸಹನೆ. ಶೋಕವೊತ್ತರಿಸಿ ಬಂದಿತು -

\begin{shloka}
``ತತಃ ಕರುಣವೇದಿತ್ತ್ವಾತ್ ಅಧರ್ಮೋಽಯಮಿತಿ ದ್ವಿಜಃ '|\label{211i}
\end{shloka}

ಆಗ ಹೊರಬಂದ `ಮಾನಿಷಾದ' ಕಾವ್ಯದ ಉಸಿರು. ಕರುಣಾರಸಪ್ರಧಾನವಾಗಿ ಕರುಣಾಕಾಕುತ್ಸ್ಥ  -ನನ್ನೇ ಕಥಾನಾಯಕನನ್ನಾಗಿ ಉಳ್ಳ ಗ್ರಂಥ. ಅದರಲ್ಲಿ ಸೀತಾರಾಮ ವಿಯೋಗದ ಧ್ವನಿಯಿದೆ.

\section*{ಕುಶಲವರಲ್ಲಿದ್ದ ರಾಮಾಯಣಗಾನಕ್ಕೆ ಬೇಕಾದ ಯೋಗ್ಯತೆ}

ಕಾವ್ಯ ರಚನೆ ಮಾಡಿದರಲ್ಲಾ, ಗ್ರಂಥವನ್ನು ಪಬ್ಲಿಷ್ ({\eng Publish}) ಮಾಡಿಬಿಡಬಾರದೇ? ಅಂದರೆ ಯೋಗ್ಯನಾದ ಅಧಿಕಾರಿಗಾಗಿ ನಿರೀಕ್ಷಣೆ. ದೊರಕದಿದ್ದರೆ ರಹಸ್ಯವಾಗಿಯೇ ಇರಲಿ. ಅನಧಿಕಾರಿಗೆ ಕೊಡುವುದು ಬೇಡ ಎಂಬ ನಿಲುವಿದೆ. 

ಕಾವ್ಯವನ್ನು ಹೇಳಿದವರ ಯೋಗ್ಯತೆ - `ಮಹಾತಪಸ್ವಿನೌ, ಮಹಾತ್ಮಾನೌ, ಮಹಾಭಾಗೌ, ಸರ್ವಲಕ್ಷಣಲಕ್ಷಿತೌ, ಗಾಂಧರ್ವತತ್ತ್ವಜ್ಞೌ, ವೇದೇಷು ಪರಿನಿಷ್ಠತೌ,ವೇದೋಪಬೃಂಹಣಾರ್ಥಾಯ, ತೌ ಮುನೀ, ಮೂರ್ಛನಾಸ್ಥಾನಕೋವಿದೌ, ಅಶ್ವಿನಾವಿವರೂಪಿಣೌ ಎಂಬುದಾಗಿದೆ, ಚಿನ್ಮಯಯಮಳರು, ಬುದ್ಧಿನಿಕಷದಲ್ಲಿ ರಾಮಾಯಣ ಸ್ವರ್ಣವನ್ನು ಒರೆಹಚ್ಚಿ ನೋಡಿದವರು, ರೂಪವೋ -ಬಿಂಬಾದಿ ವೋದ್ಧೃತೌ, ಬಿಂಬೌ, ದೇವವರ್ಚಸ್ಸಿನವರು, ರಾಮನ ವೀರ್ಯಕ್ಕೆ ಹುಟ್ಟಿದವರು, ರಾಜಾಧಿರಾಜನ ಗುಣವನ್ನು  ರಸರೂಪು ಕೆಡದಂತೆ ಸರಿಯಾಗಿ ಹೇಳಬಲ್ಲವರು.' ವಿದ್ಯೆ, ಶಾರೀರ, ರೂಪ, ಏಲ್ಲಾ ಸುಯೋಗ್ಯ, ಇದರ ಜೊತೆಗೆ ಉಪದೇಶಕರ ಯೋಗ್ಯತೆ, ನಾರದರ ಯೋಗ್ಯತೆ ಇವೆಲ್ಲವನ್ನೂ ಗಮನಿಸಬೇಕು. 

\section*{ರಾಮಾಯಣಗಾನ ಹೇಗೆ?}

ಲವಕುಶರು ರಾಮಾಯಣವನ್ನು ಹೇಗೆ ಗಾನಮಾಡಿದರು?

\begin{shloka}
`ಕಾಮಾರ್ಥಗುಣಸಂಯುಕ್ತಂ ಧರ್ಮಾರ್ಥಗುಣವಿಸ್ತರಮ್'|\label{212}\\
`ಸಮುದ್ರಮಿವರತ್ನಾಢ್ಯಂ ಸರ್ವಶ್ರುತಿಮನೋಹರಮ್'|\\
`ತದುಪಗತಸಮಾಸಸಂಧಿಯೋಗಂ ಸಮಮಧುರೋಪನತಾರ್ಥವಾಕ್ಯಬದ್ಧಮ್|\\
ರಘವರಚರಿತಂ ಮುನಿಪ್ರಣೀತಂ ದಶಶಿರಸಶ್ವ ವಧಂ ನಿಶಾಮಯಧ್ವಮ್'||\\
`ಜಾತಿಭಿಸ್ಸಪ್ತಭಿರ್ಬದ್ಧಂ ತಂತ್ರೀಲಯಸಮನ್ವಿತಮ್'|
\end{shloka}

ಯಾವ ಜಾತಿ? ಹಳ್ಳಿ ಹದಿನೆಂಟು ಜಾತಿಗಳೇ? ನಾಲ್ಕು ಜಾತಿಗಳೇ? ಗಾನದಲ್ಲಿ ಶುದ್ಧಜಾತಿ, ಮಿಶ್ರಜಾತಿ ಎಂತು ಎರಡುವಿಧ. ಮೊದಲನೆಯದು ಏಳು ಜಾತಿ. ಎರಡನೆಯದು ಹನ್ನೊಂದು ಜಾತಿ. ಅವರು ಹಾಡಿದ್ದು ಶುದ್ಧಜಾತಿಯಲ್ಲಿ. `ಲಯಸಮನ್ವಿತಂ' ದ್ರುತ, ಮಧ್ಯ, ವಿಲಂಬಿತಗಳೆಂಬ ವಿರಾಮಕಾಲ. `ಮೂರ್ಛನಾಸ್ಥಾ ನಕೋವಿದೌ' ಏನು ಹುಡುಗರಿಗೆ ಮೂರ್ಛೆಗೀರ್ಛೆ ಬರುತ್ತಿತ್ತೆ? 

\begin{shloka}
``ಪಾದಬದ್ಧೋ ಕ್ಷರಸಮಃ ತಂತ್ರೀಲಯಸಮನ್ವಿತಃ|\label{213}\\
ಶೋಕಾರ್ತಸ್ಯ ಪ್ರವೃತ್ತೋ ಮೇ ಶ್ಲೋಕೋ ಭವತು ನಾನ್ಯಥಾ||''
\end{shloka}

ಎಂದು ಹೇಳಿದಂತೆ ಕಾವ್ಯವು ಬರುವಾಗ ಲಯ-ಮೂರ್ಛನ, ಇವೆಲ್ಲವನ್ನೂ ಹೊತ್ತೇ ಬಂದಿದೆ. ಅದನ್ನು ಗಾನಮಾಡುವವರಿಗೆ  ಅದರ ಅರಿವಿಬೇಕು. (ನಂತರ ಕಲ್ಯಾಣೀ ರಾಗದಲ್ಲಿ ಒಂದು ಶ್ಲೋಕವನ್ನು ಹಾಡಿದರು.) ಇದನ್ನು ಹತ್ತು ಸಲ ಹೇಳಿ! ಹಗೆಯೇ ಕುಳಿತುಕೊಳ್ಳತ್ತೆ. `ವಾಲ್ಮೀಕಿಯು `ಕಲ್ಯಾಣಿರಾಗದಲ್ಲಿ ಹಾಡಿ!' ಎಂದು ನಿಮಗೆ ಹೇಳಿದ್ದರೇ? ನೀವು ಹಾಡಿದಿರಿ ಇದರಲ್ಲಿ?' ಎಂದರೆ ಅವರ ಆ ರಸ ಭಾವಕ್ಕೆ ಅನುಗುಣವಾಗಿದ್ದರೆ ಸರಿ. (`ಕೂಜಂತಂ ರಾಮರಾಮೇತಿ'ಯನ್ನು ಆನಂದಭೈರವಿಯಲ್ಲಿ ಹಾಡಿ ನಗಿಸಿದರು.) ಇದನ್ನೇ ಕೇಳಿದರೆ ಸಂತೋಷವಾಗುವುದಿಲ್ಲವೇ? ಏಕೆ ನಕ್ಕಿರಿ? ಒಂದು ಪಕ್ಷ ಇದನ್ನೇ ಮೊದಲು ಕೇಳಿದ್ದರೆ ಅದಕ್ಕೂ ಸಂತೋಷಪಡುತ್ತಿದ್ದಿರಿ. ತೋಟ್ಟಿಲಾಡಿಸುವಾಗ ಜೋಗುಳಕ್ಕೆ  ಬದಲು `ಚಲ್ ಚಲರೇ ನೌಜವಾನ್,' ಇದನ್ನು ಹಾಡುತ್ತಾ ಹೋದರೆ ಹಾಗೆಯೇ ಪದ್ಧತಿ ಬೆಳೆದುಬಿಡುತ್ತೆ. 

\section*{ರಾಮನಗುಣಗಳ ನಿತ್ಯಪಾರಾಯಣವಾದಾಗ ಪೂರ್ಣತೆ}

ಕಥಾನಾಯಕ-ಲೋಕಾಭಿರಾಮ, ಆತ್ಮಾರಾಮ. ಇದೆಲ್ಲಾ ನಿತ್ಯಪಾರಾಯಣವಾಗಬೇಕು. ಹಾಗೆ ಮಾಡಿದರೆ ಜೀವನ ರಸಮಯವಾಗುತ್ತೆ, ಭಾವಮಯವಾಗುತ್ತೆ, ಆನಂದಮಯವಾಗುತ್ತೆ, ಒಂದುತರಹ ತೃಪ್ತಿಯುಂಟಾಗುತ್ತೆ, ಭಾವಮಯವಾಗುತ್ತೆ, ಆನಂದಮಯವಾಗುತ್ತೆ, ಒಂದುತರಹ ತೃಪ್ತಿಯುಂಟಾಗುತ್ತೆ. ಪೂರ್ಣಜೀವಿಯಾಗುತ್ತಾನೆ. 

\section*{ಸತ್ಯದ ಅರಿವಾದಾಗಲೇ ನೈಜ ಅನುರಕ್ತಿ}

ಸತ್ಯಾಂಶ ತಿಳಿಯುವುದಕ್ಕೆ ಮೊದಲು ಇರುವ ಭಯವಾಗಲೀ, ಭಕ್ತಿಯಾಗಲೀ, ಅನುರಕ್ತಿಯಾಗಲೀ, ಬಹುಕಾಲ ಉಳಿಯುವುದಿಲ್ಲ. ಹಗ್ಗವನ್ನು ಹಾವೆಂದು ಭ್ರಮಿಸಿ ಭಯಪಡುವುದು, ಹಾವನ್ನು ಹಗ್ಗವೆಂದು ಭ್ರಮಿಸಿ ಭಯಪಡದಿರುವುದು ಎರಡೂ ತಪು. ಮೊದಲನೆಯದರಲ್ಲಿ ಕೊನೆಗೆ ಔದಾಸೀನ್ಯ, ತಾತ್ಸಾರ, ನಗೆಪಾಡು. ಎರಡನೆಯದರಲ್ಲಿ ಅಪಾಯ.

ಪಟ್ಟಾಭಿಷೇಕ-ಕುಟ್ಟಾಭಿಷೇಕ  ರೊಂಬ ಆಡಂಬ್ರ ತೋನ್ರದ್. ಇಂದು  ಶಿವರಾತ್ರಿ. ದೊರೆಸ್ವಾಮಿಗಳಿಗೆ ಶಿವಪಾರಾಯಣವಾಗಬೇಕಿತ್ತು. ಆದರೆ ಶಿವನು ಯಾವುದನ್ನು ಪಾರಾಯಣ ಮಾಡುತ್ತಿದ್ದಾನೆಯೋ ಅದರ ಪಾರಾಯಣವಾಯಿತು. ಕಾಲ ಅಲ್ಪವಾಗಿದೆ. ಅಷ್ಟರಲ್ಲೇ ವಿಷಯವಿಡಬೇಕಾಗಿದೆ. ಆದ್ದರಿಂದ ಶಿವರಾತ್ರಿಯನ್ನೂ  ನೋಡುವುದಿಲ್ಲ, ನವರಾತ್ರಿಯನ್ನೂ ನೋಡುವುದಿಲ್ಲ,ಯಾವಾಗ ವಿರಾಮ ಸಿಕ್ಕಿದರೆ ಆವಾಗ, ಎಷ್ಟು ದೊರಕಿದರೆ ಅಷ್ಟು. 


