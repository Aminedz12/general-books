\chapter{ವ್ಯಾಕರಣಶಾಸ್ತ್ರ}

(ವ್ಯಾಕರಣಶಾಸ್ತ್ರಕ್ಕೆ ಸಂಬಂಧಿಸಿದ ಈ ಪಾಠವು  ಹೆಡತಲೆಯಲ್ಲಿ ನಡೆದುದು. ವ್ಯಾಕರಣಶಾಸ್ತ್ರದ ಹೊರವಿವರಕ್ಕಿಂತಲೂ ಅದಕ್ಕೆ ಹಿನ್ನೆಯೆಯಾಗಿ ದೇವತಾತ್ಮಕವೂ, ತಾತ್ತ್ವಿಕವೂ ಆಗಿರುವ ಅಂಶವನ್ನು ಸರಳವೂ ನೇರವೂ ಆಗಿ ನಿರೂಪಿಸುತ್ತದೆ ಈ ಪಾಠ. ಇದನ್ನು ಸಂಗ್ರಹಿಸಿ ಬರಹರೂಪಕ್ಕೆ ತಂದವರು ಶ್ರೀ ಹೆಚ್.ಎಸ್. ವರದದೇಶಿಕಾಚಾರ್ಯ ರಂಗಪ್ರಿಯರು)

\section*{`ವ್ಯಾಕರಣ' ಶಬ್ದಾರ್ಥದ ವಿವರಣೆ}

`ವ್ಯಾಕರಣ ಎಂಬ ಶಬ್ದಕ್ಕೆ ಸಮ್ಯ್ಕ್ ಕೃತಂ (ಚೆನ್ನಾಗಿ ಮಾಡಲ್ಪಟ್ಟುದು) ಎಂಬ ಅರ್ಥವು ಮಹಾಭಾರತದ ಒಂದು ಶ್ಲೋಕದಲ್ಲಿ ಹೇಳಲ್ಪಟ್ಟಿದೆ. ವ್ಯಾಕರಣವು  ವೇದಾಂಗವಾದ ಶಾಸ್ತ್ರವಾದುದರಿಂಡ ಇತರ ಶಾಸ್ತ್ರಗಳಂತೆಯೇ ಇದೂ. ಕೂಡ ಪ್ರಕೃತಿ ತತ್ತ್ವಗಳನ್ನು ಚೆನ್ನಾಗಿ ವಿವೇcಅನೆ ಮಾಡಿರುವ ಶಸನವೆಂದು ನಿಸ್ಸಂದೇಹವಾಗಿ ಹೇಳಬಹುದು. ಮೇಲ್ಕಂಡ ಕಾರಣದಿಂದ ನಾವು ವ್ಯಾಕರಣಕ್ಕೆ ಪ್ರಕೃತಿ-ತತ್ತ್ವ  ರಹಸ್ಯಗಳನ್ನೆಲ್ಲಾ ಚೆನ್ನಾಗಿ ವಿಮರ್ಶಿಸಿ ಸಮ್ಯಕ್ ಕೃತವಾದ ಶಾಸ್ತ್ರವೆಂದು ಅರ್ಥಮಾಡಬಹುದು.

\section*{ವ್ಯಾಕರಣವಿದ್ಯೆಯ ಅಧಿದೇವತೆಯನ್ನು ಕುರಿತು}

ಈ ಶಾಸ್ತ್ರವ್ ತತ್ತ್ವಮಯವಾಗಿರುವುದು ಹೇಗೆ? ಇದರ ನಿರ್ಮಾಣವಾದುದೆಂತು? ಈ ವಿದ್ಯೆಗೆ ಅಧಿಪತಿ  ಯಾರು? ಈ ಅಂಶಗಳನ್ನೂ ನಾವು ವಿಚಾರಮಾಡಬೇಕಾಗುತ್ತದೆ.

ಕೆಲವರು ಹಯಗ್ರೀವರನ್ನು ವಾಗೀಶರೆಂದೂ, ವಿದ್ಯಾಧಿಪತಿಯೆಂದೂ ಪೂಜಿಸುವರು. ಮತ್ತೆ ಕೆಲವರು ವಾಗ್ದೇವತೆಯಾದ ಸರಸ್ವತೀದೇವಿಯೇ ವಿದ್ಯಾಧಿದೇವತೆಯೆಂದು ಅವರಳನ್ನು ಭಜಿಸುವ ರೂಢಿಯಿದೆ. ಇದಲ್ಲದೆ ಕವಿವರೇಣ್ಯನಾದ ಕಾಳಿದಾಸನು ತನ್ನ ರಘುವಂಶಮಹಾಕಾವ್ಯದ ಮಂಗಳಶ್ಲೋಕದಲ್ಲಿ ಪಾರ್ವತೀಪರಮೇಶ್ವರರನ್ನೇ ವಾಗರ್ಥಗಳಿಗೆ ನಾಯಕರೆಂದು ಸ್ತುತಿಸಿದ್ದಾನೆ. ಹೀಗಿರುವಾಗ ವಿದ್ಯಾಧೀಶರೆಂದು ಕರೆಸಿಕೊಳ್ಳುವ ಈ ಮೂವರ ಸ್ವರೂಪವನ್ನು ತಿಳಿದುಕೊಂಡು ಮೂವರೂ ಕೂಡ ಹೇಗೆ ವಿದ್ಯಾಧಿಪತಿಗಳಾಗಿದ್ದಾರೆಂಬುದನ್ನು ನೋಡೋಣ.

\section*{ವ್ಯಾಕರಣಕ್ಕೆ ಮೂಲವಾದ ಚತುರ್ದಶಸೂತ್ರಗಳು}

ಎಲ್ಲಕ್ಕಿಂತಲೂ ಮೊದಲು ವ್ಯಾಕರಣಶಾಸ್ತ್ರಕ್ಕೆ  ಮೂಲಭೂತವಾಗಿರುವ ಚತುರ್ದಶ ಸೂತ್ರಗಳನ್ನು ನೋಡೋಣ.

೧) ಉಇಉಣ್  ೨) ಋಲೃಕ್ ೩) ಎಓಜ್ ೪)ಐಜೌಚ್ ೫)ಹಯವರಟ್ ೬)ಲಣ್  ೭) 


ಸಂಸ್ಕೃತಭಾಷೆಯಲ್ಲಿರುವ ಸಕಲವರ್ಣಮಾಲೆಯೂ ಈ ಸೂತ್ರಗಳಲ್ಲಿ ಅಡಕವಾಗಿರುವುದಲ್ಲವೇ? ಈ ವರ್ಣಗಳ ವಿವಿಧಸಂಯೋಜನೆಯಿಂದಲೇ ಸಮಸ್ತ ವೈಯಾಕ್ರಣಶಬ್ದಜಾಲವೂ ಉತ್ಪನ್ನವಾಗಿದೆ. ಆದ್ದರಿಂದಲೇ ಈ ಸೂತ್ರಗಳನ್ನು ವ್ಯಾಕರಣಶಾಸ್ತಕ್ಕೆ ಮೂಲವೆಂದು ಹೇಳಿರುವುದು.

\section*{ತತ್ತ್ವಮಯವಾದ ವರ್ಣಾಗಳಿಂದ ಕೂಡಿದೆ ಶಬ್ದಬ್ರಹ್ಮ}

ಈ ವರ್ಣಗಳು ತತ್ತ್ವಮಯವಾದುವೆಂದೂ, ಶಬ್ದಬ್ರಹ್ಮನು ವರ್ಣಾತ್ಮಕನೆಂದೂ ಹೆಳಿರುವ ದೇವೀಭಾಗವತದ ಈ ಶ್ಲೋಕವನ್ನು ನೋಡಿ-
\begin{shloka}
ವಾಸಾಂತೇ ಬಾಲಮಧ್ಯೇ ಡಫಕಠಸಹಿತೇ ಕಂಠದೇಶೇ ಸ್ವರಾಣಾಂ\\
ಹಂಕ್ಷಂ ತತ್ತ್ವಾರ್ಥಯುಕ್ತಂ ಸಕಲದಲಗತಂ ವರ್ಣರೂಪಂ ನಮಾಮಿ||
\end{shloka}

ವಾ-ಸ ಎಂದರೆ ವ-ಶ-ಷ-ಸ ಎಂಬ ನಾಲ್ಕು ವರ್ಣಗಳು. ನಾಲ್ಕು ವರ್ಣಗಳುಳ್ಳ ನಾಲ್ಕು ದಳಗಳನ್ನೊಗೊಂಡ ಪದ್ಮವು ಮೂಲಾಧಾರಪದ್ಮ. ಇದು ಪೃಥಿವೀ ಬೀಜವಾದ ಲಕಾರಸಂಜ್ಞಕ. ಅಂತೆಯೇ ಬ್ರಹ್ಮಾತ್ಮಕವಾದುದು ಈ ಪದ್ಮ.

ಬಾ-ಲ ಎಂದರೆ ಬ-ಭ-ಮ-ಯ-ರ-ಲ ಎಂಬ ಆರು ಆಕ್ಷರಗಳುಳ್ಳ ಷಡ್ದಳ ಪದ್ಮ. ಜಲಬೀಜವಾದ ವಕಾರಸಂಜ್ಞಕ. ಇದು ವಿಷ್ಣ್ವಾತ್ಮಕವಾದುದು. ಇದು ಸ್ವಾಧಿಷ್ಠಾನಪದ್ಮವೆಂಬ ಹೆಸರನ್ನು ಪಡೆದಿದೆ.

ಡ-ಫ ಎಂದರೆ ಡ-ಢ-ಣ-ತ-ಥ-ದ-ಧ-ನ-ಪ-ಫ ಎಂಬ ಹತ್ತು ಅಕ್ಷರಗಳುಳ್ಳ ದಶದಲ ಪದ್ಮ. ಅಗ್ನಿ ಬೀಜವಾದ ರ (ರೇಫ) ಸಂಜ್ಞಕ. ಅಗ್ನಿ ತತ್ತ್ವಾತ್ಮಕ. ರುದ್ರಾತ್ಮಕವಾದುದು, ಇದೇ ಮಣಿಪೂರ ಪದ್ಮ.

ಕ-ಠ ಎಂದರೆ ಕ-ಖ-ಗ-ಘ-ಙ-ಚ-ಛ-ಜ-ಝ-ಞ-ಟ-ಠ ಎಂಬ ಹನ್ನೆರಡು ವರ್ಣಗಳುಳ್ಳ ದ್ವಾದಶದಳವುಳ್ಳ ಪದ್ಮ. ಇದು ವಾಯುಬೀಜವಾದ ಯಕಾರ ಸಂಜ್ಞಾತ್ಮಕವಾದುದು. ದಿ ವಾಯುತತ್ತ್ವಾತ್ಮಕ, ಅನಾಹತಪದ್ಮ. 

ಸಕಲಸ್ವರವರ್ಣಸ್ಥಾನವಾದುದು ಷೋಡಶದಳ ಪದ್ಮ. ಇದು ಹಕಾರಸಂಜ್ಞಕವಾದುದು. ಆಕಾಶತತ್ತ್ವಾತ್ಮಕ. ಇದೇ ವಿಶುದ್ಧಿ ಪದ್ಮ. ಕಂಠದೇಶದಲ್ಲಿರುವ ಪದ್ಮವಿದು.

ಈ ಎಲ್ಲಕ್ಕಿಂತಲೂ ಮೇಲೆ ಭ್ರೂಯುಗದ ನೇರದಲ್ಲಿ ಹಂ-ಕ್ಷಂ ಎಂಬ  ಎರಡು ದಳಗಳುಳ್ಳ ಆಜ್ಞಾಪದ್ಮ.

ಈ ಎಲ್ಲ ಪದ್ಮ (ಚಕ್ರ) ಗಳಲ್ಲೂ, ಅಂದರೆ ಎಲ್ಲಾ ತತ್ತ್ವಗಳನ್ನೂ ಒಳಗೊಂಡ ಎಲ್ಲಾ ಪದ್ಮಗಳ ಸಕಲದಳಗಳಲ್ಲಿಯೂ ಪರಿಪೂರ್ಣವಾಗಿ ವ್ಯಾಪಿಸಿರುವುದರಿಂದಲೇ, ಸುಕೃತಿಮಾತ್ರಗಮ್ಯನಾದ ಆ ಅಂತರ್ಗತಚೈತನ್ಯರೂಪವನ್ನು ಇಲ್ಲಿ ತತ್ತ್ವಾರ್ಥಯುಕ್ತನೆಂದೂ ಸಕಲದಳಗತನೆಂದೂ ಕರೆದಿದ್ದಾರೆ.

\section*{ವ್ಯಾಕರಣದ ಆವಿರ್ಭಾವ}

ಈ ಶಾಸ್ತ್ರವು (ವ್ಯಾಕರಣವು) ಹೊರಗೆ ಬಂದ ಬಗೆಯನ್ನು ಕೆಳಗಿನ ಶ್ಲೋಕವು ವಿವರಿಸುತ್ತದೆ.
\begin{shloka}
ನೃತ್ತಾವಸಾನೇ ನಟರಾಜರಾಜಃ\\
ನನಾದ ಢಕ್ಕಾಂ ನವ-ಪಂಚವಾರಮ್|\\
ಉದ್ಧರ್ತುಕಾಮಃ ಸನಕಾದಿಸಿದ್ಧಾನ್\\
ಏತದ್ವಿಮರ್ಶೇ ಶಿವಸೂತ್ರಜಾಲಮ್||
\end{shloka}

(ನಟರಾಜನು ತನ್ನ ನೃತ್ತ್ಯದ ಕೊನೆಯಲ್ಲಿ, ನಸಕಾದಿಸಿದ್ಧರನ್ನು ಉದ್ಧರಿಸುವ ಸಲುವಾಗಿ ತನ್ನ ಢಕ್ಕೆಯನ್ನು ಒಂಭತ್ತು ಮತ್ತು ಐದು ಬಾರಿ ಬಾರಿಸಿದನು. ಇದನ್ನು ವಿಚಾರಿಸಿ ನೋಡಲಾಗಿ ಇದೇ ಶಿವಸೂತ್ರಜಾಲ-ಮಹೇಶ್ವರ ಸೂತ್ರಗಳು.) ಈ ಶಾಸ್ತ್ರವು ಕೇವಲ ತತ್ತ್ವಜಿಜ್ಞಾಸುಗಳಿಗೆ ಮೊದಲು ಉಪದಿಷ್ಟವಾಯಿತೆಂದು ``ಸಿದ್ಧಾನ್" ಎಂಬ ಪದವು ನಿರೂಪಿಸುತ್ತದೆ. ಭಗವಾನ್ೞ್ ಮಹೇಶ್ವರನು ಉಪಯೋಗಿಸಿದ `ಢಕ್ಕಾ' ಎಂಬುದು ವಾದ್ಯವಿಶೇಷ. ಶಾಸ್ತ್ರೀಯವಾಗಿ ಮನೀಷಿಗಳಾದ ಬ್ರಾಹ್ಮಣ (ಬ್ರಹ್ಮಜ್ಞಾನಿ) ರಿಂದ ಮಾತ್ರ ಜ್ಞೇಯವಾದ ಅಂತರ್ನಾದವನ್ನು ಬಾಹ್ಯಪ್ರಪಂಚಕ್ಕೆ ಅನುಕರಣಮಾಡಿ ತೋರಿಸುವುದಕ್ಕಾಗಿಯೇ ಈ ವಾದ್ಯದ ಉಪಯೋಗ. ಹೀಗೆ ಹೊರಗಿನ ಜನರಿಗೆ ಕೇಳಿಸುವಂತಹ ಧ್ವನಿಗೆ `ವೈಖರೀ' ಎಂದು ಹೆಸರು.

\section*{ವಿದ್ಯಾಧಿದೇವತೆಗಳನ್ನು ಕುರಿತಾದ ಆರ್ಷಸಾಹಿತ್ಯ}

ಚಂದ್ರಮಂಡಲಾಂತರ್ಗತನೂ, ಮಹತ್-ತತ್ತ್ವಾಧಿಪತಿಯೂ ಆದ ಪರಮೇಶ್ವರನೇ ಶಬ್ಧಬ್ರಹ್ಮನಾದ ಅಂತರ್ಗತಚೈತನ್ಯನೆಂಬುದನ್ನು ಶಾರದಾತಿಲಕದ ಕೆಳಗಿನ ಶ್ಲೋಕವು ವಿವರಿಸುತ್ತದೆ.

\begin{shloka}
ನಿತ್ಯಾನಂದವಪುರ್ನಿರಂತರಗಲತ್ಪಂಚಾಶದರ್ಣೈಃ ಕ್ರಮಾತ್\\
ವ್ಯಾಪ್ತಂ ಯೇನ ಚರಾಚರಾತ್ಮಕಮಿದಂ ಶಬ್ದಾರ್ಥರೂಪಂ ಜಗತ್|
\end{shloka}
\begin{shloka}
ಶಬ್ದಬ್ರಹ್ಮ ಯದೂಚಿರೇ ಸುಕೃತಿನಶ್ಚೈತನ್ಯಮಂತರ್ಗತಂ\\
ತದ್ವೋಽವ್ಯಾದನಿಶಂ ಶಶಾಂಕಸದನಂ ವಾಚಾಮಧೀಶಂ ಮಹಃ||
\end{shloka}

(ಯಾವನು ನಿತ್ಯಾನಂಡ ಶರೀರನೋ, ನಿರಂತರವಾಗಿ (ತಡೆಯಿಲ್ಲದೇ) ಹರಿಯುತ್ತಿರುವ ಐವತ್ತು ವರ್ಣರೂಪವಾದ ಶಕ್ತಿಗಳಿಂದ ಚರಾಚರಾತ್ಮಕವೂ ಶಬ್ದಾರ್ಥ ರೂಪವೂ ಆದ ಜಗತ್ತನ್ನು ತುಂಬಿರುವನೋ, ಯಾವನನ್ನು ಸುಕೃತಿಗಳಾದ ಜ್ಞಾನಿಗಳು ಅಂತರ್ಗತವಾದ ಚೈತನ್ಯವೆಂದೂ ಶಬ್ದಬ್ರಹ್ಮನೆಂದೂ ಹೇಳುವರೋ, ಅಂತಹ ಶಶಾಂಕಸದನನೂ ವಾಗಧೀಶನೂ ಆದ ತೇಜೋರೂಪಿಯು ನಿಮ್ಮನ್ನು ಯಾವಾಗಲೂ ಕಾಪಾಡಲಿ) ಶಶಾಂಕಸದನನಾದ ಮಹೇಶ್ವರನೇ ವಾಗಧೀಶನೆಂದು ಈ ಶ್ಲೋಕದಿಂದ ತಿಳಿದುಬರುತ್ತದೆ.
\begin{shloka}
ವಾಗರ್ಥಾವಿವ ಸಂಪೃಕ್ತೌ ವಾಗರ್ಥಪ್ರತಿಪತ್ತಯೇ|\\
ಜಗತಃ ಪಿತರೌ ವಂದೇ ಪಾರ್ವತೀಪರಮೇಶ್ವರೌ||
\end{shloka}

(ಮಾತು ಮತ್ತು ಅರ್ಥಗಳಂತೆ ಸೇರಿರುವ, ಪ್ರಪಂಚದ ತಾಯಿ ತಂದೆಗಳಾದ ಪಾರ್ವತೀ-ಪರಮೇಶ್ವರರನ್ನು, ಮಾತು ಮತ್ತು ಅರ್ಥಗಳ ಲಾಭಕ್ಕಾಗಿ ನಮಸ್ಕರಿಸುತ್ತೇನೆ.) ಈ ಶ್ಲೋಕದಲ್ಲೂ ಕೂಡ ಪಾರ್ವತೀ-ಪರಮೇಶ್ವರರನ್ನು ವಗರ್ಥಪ್ರತಿಪತ್ತಿದಾಯಕರೆಂದು ಹೇಳಿದೆ.

\begin{shloka}
ಅಂಕೋನ್ಮ್ತುಕ್ತಶಶಾಂಕಕೋಟಿಸದೃಶೀಂ ಆಪೀನತುಂಗಸ್ತನೀಂ|\\
ಚಂದ್ರಾರ್ಧಾಂಕಿತಮಸ್ತಕಾಂ ಮಧುಮದಾದಾಲೋಲನೇತ್ರತ್ರಯಾಂ||
\end{shloka}

\begin{shloka}
ಬಿಭ್ರಾಣಾಮನಿಶಂ ವರಂ ಜಪವಟೀಂ ವಿಧ್ಯಾಂ ಕಪಾಲಂ ಕರೈಃ|\\
ಆಧ್ಯಾಂ ಯೌವನಗರ್ವಿತಾಂ ಲಿಪಿತನುಂ ವಾಗೀಶ್ವರೀಮಾಶ್ರಯೇ||
\end{shloka}

(ಕಳಂಕರಹಿತವಾದ ಕೋಟಿಚಂದ್ರರ ಕಾಂತಿಯುಳ್ಳವಳೂ, ವೃದ್ಧಿಗೊಂಡ ಉನ್ನತ ಸ್ತನಗಳುಳ್ಳವಳೂ, ಅರ್ಧಚಂದ್ರನನ್ನು ಶಿರಸ್ಸಿನಲ್ಲಿ ಧರಿಸಿದವಳೂ, ಮಧುಮದದಿಂದ ಚಂಚಲವಾದ ನೇತ್ರವುಳ್ಳವಳೂ, ವರ, ಜಪಮಾಲೆ, ವಿದ್ಯೆ, ಕಪಾಲಗಳನ್ನು ತನ್ನ ನಾಲ್ಕು  ಕೈಗಳಿಂದ ಧರಿಸಿರುವವಳೂ, ಮೊದಲಿಗಳೂ, ಯೌವನಗರ್ವಿತೆಯೂ, ಲಿಪಿಶರೀರಿಣಿಯೂ ಆದ ವಾಗೀಶ್ವರಿಯನ್ನು ಆಶ್ರಯಿಸುತ್ತೇನೆ.) ಈ ಶ್ಲೋಕದಲ್ಲಿ ಲಿಪಿತನುವಾದ ವಾಗೀಶ್ವರಿಯನ್ನು ಸ್ಸ್ತುತಿಸಿದೆ. ಇವಳೂ ಚಂದ್ರಾಂಕಿತಮಸ್ತಕಳು. ಇವಳ ಸ್ಥಾನವು ಕಂಠವು. `ಕಂಠೇಽಸ್ಸಿ ಭಾರತೀಸ್ಥಾನಂ' (ಅಧ್ಯಾತ್ಮವಿವೇಕ)

\begin{shloka}
ಶಿವಶಕ್ತಿಮಯಂ ಸಾಕ್ಷಾತ್ ಛಾಯಶ್ರಿತಜಗತ್ತ್ರಯಂ||
\end{shloka}

(ಲಿಪಿತರುವು ಸಾಕ್ಷಾತ್ ಶಿವಶಕ್ತಿಮಯವಾದುದು. ಇದರ ನೆರಳನ್ನು ಮೂರು ಲೋಕಗಳೂ ಆಶ್ರಯಿಸಿವೆ) ಎಂದು ಲಿಪಿತರುವನ್ನು ಸ್ತೋತ್ರ  ಮಾಡಿರುವ ಎಡೆಯೂ ಇದೆ.

ಈ ಕೆಳಗಿನ ಶ್ಲೋಕಗಳು ವಿದ್ಯಾಧಿಪನಾದ ಶ್ರೀ ಹಯಗ್ರೀವನನ್ನು ಸ್ತುತಿಸುತ್ತವೆ.

\begin{itemize}
\item[೧)] 

\begin{shloka}
ವಂದೇ ಪೂರಿತcಅಂದ್ರಮಂಡಲಗತಂ ಶ್ವೇತಾರವಿಂದಾಸನಂ \\
ಮಂದಾಕಿನ್ಯಮೃತಾಬ್ಜಕುಂದಕುಮುದಕ್ಷೀರೇಂದುಹಾಸಂ ಹರಿಮ್||\\

\medskip
ಮುದ್ರಾ-ಪುಸ್ತಕ-ಶಂಖ೦-ಚಕ್ರವಿಧೃತಶ್ರೀಮುದ್ಭುಜಾಮಂಡಲಂ\\
ನಿರ್ಯನ್ನಿರ್ಮಲ-ಭಾರತೀ-ಪರಿಮಲಂ ವಿಶ್ವೇಶಮಶ್ವಾನನಮ್||
\end{shloka}

\item[೨)] 

\begin{shloka}
ಅಂಕೇನೋದೂಹ್ಯ ವಾಗ್ದೇವೀಂ ಆಚಾರ್ಯಕಮುಪಾಶ್ರಿತಃ||
\end{shloka}
\end{itemize}

\begin{itemize}
\item[1)] (ಪೂರ್ಣಚಂದ್ರಮಂಡಲದಲ್ಲಿರುವವನೂ, ಬೆಳಾವರೆಯನ್ನೇ ಆಸನವಾಗಿ ಉಳ್ಳವನೂ, ಮಂದಾಕಿನೀ-ಅಮೃತಮಯವಾದ ಕಮಲ-ಮೊಲ್ಲೆ-ಬೆಳೈದಿಲೆ-ಹಾಲು-ಚಂದ್ರರಂತೆ ಶುಭ್ರವಾದ ನಗೆಯುಳ್ಳವನೂ, ಮಹಾವಿಷ್ಣುವೂ, ಜ್ಞಾನ (ವ್ಯಾಖ್ಯಾ) ಮುದ್ರೆ-ಪುಸ್ತಕ-ಶಂಖ-ಚಕ್ರಗಳನ್ನು ಧರಿಸಿರುವ ಕಾಮ್ತಿಯುತವಾದ ಭುಜಗಳುಳ್ಳವನೂ, ನಿರ್ಮಲವಾದ ಭಾರತೀಪರಿಮಲವನ್ನು ಹೊಮ್ಮಿಸುವವನೂ ಆದ ವಿಶ್ವೇಶ್ವರನಾದ ಹಯಗ್ರೀವನನ್ನು ನಮಸ್ಕರಿಸುತ್ತೇನೆ).
 
\item[2)] (ಹಯಗ್ರೀವನು ವಾಗ್ದೇವಿಯನ್ನು ತನ್ನ ತೊಡೆಯಲ್ಲಿಟ್ಟುಕೊಂಡು ಅವಳಿಗೆ ಆಚಾರ್ಯನಾಗಿದ್ದಾನೆ.)

ಮೇಲಿನ ಶ್ಲ್ಲೋಕಗಳು ಹಯಗ್ರೀವನೇ ವಾಗೀಶನೆಂದು ಹೇಳುತ್ತವೆ.
\end{itemize}

\section*{ಹಯಗ್ರೀವಮೂರ್ತಿಯ ವಿವರಣೆ}

ಭಾವಚಿತ್ರದಲ್ಲಿ ಹಯಗ್ರೀವಸ್ಥಾನವನ್ನು ಸರಿಯಾಗಿ ಗಮನಿಸಿದರೆ, ಅಲ್ಲಿ ಅವರು ಪರಿಪೂರ್ಣ ಚಮ್ದ್ರಮಂದಲಮಧ್ಯವರ್ತಿಯಾಗಿರುವುದನ್ನು ಕಾಣಬಹುದು. ಸಹಸ್ರಾರವೆಂಬ ಕಮಲದ ಮೇಲಿರುವುದರಿಂದ ``ಶ್ವೇತಾರವಿಂದಾಸನಂ" ಎಂದು ಹೇಳಿರುವುದು ಹೊಂದುತ್ತದೆ. ಹಯಗ್ರೀವರ ಹಾಸಕ್ಕೆ (ನಗುವಿಗೆ ) ಉಪಮಾನವಾಗಿರುವ ಮಂದಕಿನಿಯು ಗಂಗಾ ಎಂಬ ಮತ್ತೊಂದು ಹೆಸರುಳ್ಳ `ಇಡಾ' ನಾಡಿಯೇ ಆಗಿದೆ. ಅಮೃತ-ಕ್ಷೀರ ಎಂಬ ಪದಗಳು ಯೋಗಿಮಾತ್ರ ಪೇಯಗಳಾದ ಸುಷುಮ್ನಾಮಾರ್ಗದಿಂದ ಮೇಲೆ ಹೋದಾಗ ಸಹಸ್ರಾರದ ಬಳಿ ಇರುವ ವಸ್ತುಗಳಾಅಗಿವೆ. ಹಗ್ರೀವರು ಹಸ್ತದಲ್ಲಿ ಧರಿಸಿದ ಶಂಖ, ಚಕ್ರ ಮತ್ತು ಪುಸ್ತಕಗಳು ಕ್ರಮವಾಗಿ ಅಹಂಕಾರತತ್ವ, ಮನಸ್ತತ್ತ್ವ ಮತ್ತು ಮಾತೃಕಾಸೃಷ್ಟಿ ತತ್ತ್ವ ರಹಸ್ಯಗಳ ಸೂಚಕಗಳಾಗಿವೆ.

\section*{ವಿದ್ಯಾಧಿದೇವತೆಗಳಾ ಸ್ಥಾನಮಾನಗಳ ಕುರಿತು}

ಹಯಗ್ರೀವರು, ಅವರ ಕೆಳಗೆ ಕಂಠದೇಶದಲ್ಲಿ ಭಾರತೀದೇವೀ, ಅಲ್ಲಿಂದ ಕೆಳಗೆ ಶಕ್ತಿಯುತನಾದ ಶಿವ ಈ ಮೂವರೂ ವಿದ್ಯಾಧಿಪತಿಗಳು ಮೂವರೂ ಚಂದ್ರಮಂಡಲ ಮಧ್ಯಗರು. ಹಯಗ್ರೀವರನ್ನು ಶ್ರವಣನಕ್ಷತ್ರದಲ್ಲಿ ಪೂಜಿಸಲು ಕಾರಣವೇನೆಂದರೆ ಹಗ್ರೀವರು ಉತ್ತರಾಯನಕ್ಷತ್ರವಾದ ಶ್ರವಣದ ಕಡೆಯೂ, ಸರಸ್ವತಿಯು ದಕ್ಷಿಣಾಯನ ನಕ್ಷತ್ರವಾದ ಮೂಲದ ಕಡೆಯೂ ಇರುವುದು.

\section*{ಕಾಳಿದಸನು ಪಾರ್ವತೀಪರಮೇಶ್ವರರನ್ನೇ ವಾಗರ್ಥದೇವತೆಗಳಾಗಿ ಸ್ತೋತ್ರಮಾಡಲು ಕಾರಣ}

ಹಯಗ್ರೀವರು ಮತ್ತು ವಾಗ್ದೇವಿಯರಿದ್ದರೂ ಅವರನ್ನು ಬಿಟ್ಟು, ಕಾಳಿದಾಸ ಕವಿಯು ಪಾರ್ವತೀ ಪರಮೇಶ್ವರರನ್ನು ಸ್ತುತಿಸಲು ಕಾರಣ ಮೊದಲಿನ ಈರ್ವರೂ ವಾಗೀಶರಾಗಿದ್ದರೂ ಶಬ್ದಾರ್ಥಗಳು ಹೊರಗೆ (ಬಾಹ್ಯದಲ್ಲಿ) ವೈಖರಿಯ ರೂಪದಲ್ಲಿ ಬಂದು ಕವಿಗೆ ಸಹಾಯಕವಾಗಬೇಕಾದರೆ ಅದಕ್ಕೆ ಪಾರ್ವತೀಪರಮೇಶ್ವರರ ಅನುಗ್ರಹವು ಬೇಕೇ ಬೇಕೇ. ಪಾರ್ವತೀಪರಮೇಶ್ವರೌ ಎಂಬುದರ ಬದಲು ಕೆಲವರು `ಜಾನಕೀರಘುನಾಯಕೌ' ಎಂದು ಸೇರಿಸುವುದು ಕೇವಲ ಅಂಧಜ್ಞಾನದಿಂದ. ಸೀತಾರಾಮರನ್ನು ಸ್ತುತಿಸಬೇಕೆಮ್ದು ಅವರಿಗೆ ಇಷ್ಟವಿದ್ದರೆ ಅದಕ್ಕೆ ಈ ಶ್ಲ್ಲೋಕವು ಸ್ಥಾನವಲ್ಲ. ಇಲ್ಲಿ ಹಾಗೆ ಸೇರಿಸಿದರೆ ಅದು ಅಪ್ರಕೃತ, ಅವೈಜ್ಞಾನಿಕ ಮತ್ತು ಅಸಂಗತವಾಗುತ್ತದೆ. ಸೀತಾರಾಮರನ್ನು ಸ್ತ್ತುತಿಮಾಡುವ ಮಹರ್ಷಿ ವಾಲ್ಮೀಕಿಗಳ ಗ್ರಂಥದಲಿ ಅವರನ್ನು  ಸ್ತುತಿಮಾಡಲಿ.

\section*{ವಿದ್ಯಾಧಿದೇವತೆಗಳಲ್ಲಿರುವ ಸಾಮರಸ್ಯ}

ಮತ್ಸರವಿಲ್ಲದೆ ಈ ರೀತಿ ವೈಜ್ಞಾನಿಕದೃಷ್ಟಿಯಿಂದ ನೋಡಿದಲ್ಲಿ ಹಯಗ್ರೀವ ಸರಸ್ವತೀ ಪರಮೇಶ್ವರ ಮೂವರೂ ವಿದ್ಯಾಧಿದೇವತೆಗಳೆಂದು ಸಮಂಜಸವಾದ ಉತ್ತರ ಬರುತ್ತದೆ. ಆದರೆ ಈ ಮೂವರ ಸ್ಥನಬೇರೆ. ಇವರು ಕೂಡುವ ಫಲಗಳು ಬೇರೆ ಬೇರೆ.
 
\section*{ವ್ಯಾಕರಣವನ್ನು ಕುರಿತು ಮಹಾಭಾರತದ ಪ್ರಮಾಣಚನ}

ಮಹಾಭಾರತದ ಉದ್ಯೋಗಪರ್ವದಲ್ಲಿ ಸನತ್ಸುಜಾತರು ಧೃತರಾಷ್ಟ್ರನಿಗೆ ಮಾಡುವ ಉಪದೇಶದಲ್ಲಿ ವ್ಯಾಕರಣಶಾಸ್ತ್ರದ ಬಗೆಗೆ ಈ ಮಾತುಗಳಿರುವುದನ್ನು ಗಮನಿಸಬಹುದು.

\begin{shloka}
ಅವಿಚಿನ್ವನ್ನಿಮಂ ವೇದೇ ತಪಸಾ ಪಶ್ಯತಿ ಪ್ರಭುಮ್||
\end{shloka}

\begin{shloka}
ತೂಷ್ಣೀಂಭೂತ ಉಪಾಸೀತ ನ ಚೇಷ್ಟೇನ್ಮನಸಾಪಿ ಚ|\\
ಉಪಾವರ್ತಸ್ವ ತದ್ಬ್ರಹ್ಮ ಅಮ್ತರಾತ್ಮನಿ ವಿಶ್ರುತಮ್||
\end{shloka}

\begin{shloka}
ಮೌನಾನ್ನ ಸ ಮುನಿರ್ಭವತಿ ನಾರಣ್ಯವಸನಾನ್ಮುನಿಃ|\\
ಸ್ವಲಕ್ಷಣಂ ತು ಯೋ ವೇದ ಸ ಮುನಿಃ ಶ್ರೇಷ್ಠ ಉಚ್ಯತೇ||
\end{shloka}

\begin{shloka}
ಸರ್ವಾರ್ಥಾನಾಂ ವ್ಯಾಕರಣಾತ್ ವೈಯಾಕರಣ ಉಚ್ಯತೇ|\\
ತನ್ಮೂಲತೋ ವ್ಯಾಕರಣಂ ವ್ಯಾಕರೋತೀತಿ ತತ್ತಥಾ||
\end{shloka}

\begin{shloka}
ಪ್ರತ್ಯಕ್ಷದರ್ಶೀ ಲೋಕಾನಾಂ ಸರ್ವದರ್ಶೀ ಭವೆನ್ನರಃ|\\
ಸತ್ಯೇ ವೈ ಬ್ರಾಹ್ಮಣಸ್ತಿಷ್ಟನ್ ತದ್ವಿದ್ವಾನ್ ಸರ್ವವಿದ್ಭವೇತ್||
\end{shloka}

ಯೋಗಿಯಾದವನು ವೇದಗಳಲ್ಲಿ ಅವನನ್ನು ಹುಡುಕದೆ, ತಪಸ್ಸಿನ ಮೂಲಕವಾಗಿ ನೋಡಬಲ್ಲನು. ಯೋಗಿಯು ಇದ್ದಕಡೆಯಿಂದ ಕದಲದೇ, ಬಾಯಿಂದ ಮಾತಾಡದೇ, ಮಾನಸಿಕ ಚೇಷ್ಟೆಯು ಇಲ್ಲದೇ ಅವನನ್ನು ಕಾಣಬಲ್ಲನು ಮಹಾರಾಜ! ಪ್ರಸಿದ್ಧವಾದ ಆ ಬ್ರಹ್ಮನನ್ನು ನೀನು ನಿನ್ನ್ನೊಳಗೆ ಉಪಾಸಿಸು. ಬರಿಯ ಮೌನದಿಂದಾಗಲೀ ಕೇವಲ ವನವಾಸದಿಂದಾಗಲೀ ಒಬ್ಬನು ಮುನಿಯಾಗುವುದಿಲ್ಲ. ತನ್ನೊಳಗೆ ಇರುವ ಆ ಪರಮಾತ್ಮನನ್ನು ಯಾವನು ತಿಳಿದಿರುವನೋ ಅವನೇ ಶ್ರೇಷ್ಠನಾದ ಮುನಿಯೆಂದು ಹೇಳಲ್ಪಡುವನು. ಈ ಎಲ್ಲಾ ಅರ್ಥವನ್ನೂ ವಿವರಿಸಬಲ್ಲವನಾಅ ಕಾರಣಾದಿಂದಲೇ, ಆ ಮುನಿಮೂಲದಿಂದಲೇ ವ್ಯಾಕರಣವು ಬಂದಿತು. ಆ ಬ್ರಹ್ಮನನ್ನು ಮೂಲವಾಗಿ ಉಳ್ಳ ಪ್ರಪಮ್ತತ್ತ್ವವನ್ನು ವಿವರಿಸುವ ಶಾಸ್ತ್ರವೇ ವ್ಯಾಕರಣವೆನಿಸುವುದು. ಆಂತಹ ವೈಯಾಕ್ರಣನೇ ಲೋಕಗಳಾನ್ನೆಲ್ಲಾ ಸಾಕ್ಷತ್ಕರಿಸಿಕೊಂಡು ಎಲ್ಲವನ್ನೂ ನೋಡಬಲ್ಲವನಾಗುವನು. ಸತ್ಯದಲ್ಲಿ (ಬ್ರಹ್ಮದಲ್ಲಿ) ನೆಲೆಗೊಂಡವನೇ ಸರ್ವಜ್ಞನೆನಿಸುವನು.
