\chapter{ರಾಮಾಯಣ ಮತ್ತು ತಿರುವಾಯ್ಮೊಳಿ}

(ರಾಮಾಯಣ ಮತ್ತು ನಮ್ಮಾಳ್ವಾರರ ಪ್ರಬಂಧದ ಬಗೆಗೆ ಶ್ರೀರಂಗ ಮಹಾಗುರುವು ಆಡಿದ ಮಾತುಗಳು.)

\section*{ಜ್ಞಾನಿಗಳ ಹೃದಯದಿಂದ ಬಂದ ಧ್ವನಿಯನ್ನು ಗ್ರಹಿಸಲು ಸಮಾನಹೃದಯಬೇಕು}

ವಾಲ್ಮೀಕಿಗಳಾಗಲಿ, ನಮ್ಮಾಳ್ವಾರಾಗಲಿ ತಮ್ಮ ರಾಮಾಯಣ ಮತ್ತು ಪ್ರಬಂಧಗಳನ್ನು ರಚಿಸುವಾಗ ಯಾವುದನ್ನು ಆಧಾರವಾಗಿಟ್ಟುಕೊಂಡು ಬರೆದರು? ಅವರಿಗೆ ಯಾವ ರಾಮಾಯಣ-ಪ್ರಬಂಧಗಳು ಸಿಕ್ಕಿದುವು? ರಾಮಾಯಣ ಗ್ರಂಥವನ್ನು ಓದಿ ಪಟ್ಟಾಭಿಷೇಕ ಮಾಡುವುದಿದೆ. ಆ ಪಟ್ಟಾಭಿಷೇಕದಲ್ಲಿ ಸ್ವಲ್ಪ ವ್ಯತ್ಯಾಸವಾದರೆ ಕುಟ್ಟಾಭಿಷೇಕ, ಮಟ್ಟಾಭಿಷೇಕ ಎಲ್ಲಾ ಬರುತ್ತೆ. ರಾಮಾಯನ ಹೇಗೆ ಬಂತು? ಜ್ಞಾನಿಗಳ ಹೃದಯದಿಂದ ಬಂದ ರಾಮಾಯಣ ಒಂದು ಸಮಾಚಾರ ಕೊಡುತ್ತೆ. ಉದಾಹರಣೆಗೆ ಭೌತಿಕವಾಗಿಯೂ, ಮಗುವಿಗೆ ಹಸಿವಾದರೆ ಅದು ಒಂದು ಧ್ವನಿ ಕೊಡುತ್ತೆ. ಆ ಧ್ವನಿ ತಂದೆತಾಯಿಗಳಿಗೆ ಒಂದು ಕಾಲಿಂಗ್ ಬೆಲ್ ಆಗಿರುತ್ತೆ. ಹಸಿವನ್ನು ಸೂಚಿಸುವ ಧ್ವನಿ ತಂದೆತಾಯಿಗಳಿಗೆ ಅರ್ಥವಾಗುತ್ತೆ. ಹೀಗೆಯೇ ಜ್ಞಾನಿಗಳ ಹೃದಯದಿಂದ ಬಂದ ಧ್ವನಿಗೂ ಒಂದು ಅರ್ಥವಿದೆ. ಅದನ್ನು ಗ್ರಹಿಸಬೇಕಾದರೆ ಒಂದು ಹೃದಯ ಬೇಕು.

\section*{ಸ್ಥಾನಭೇದದಿಂದ ಅರ್ಥಭೇದ}

ನಿಮಗೆ ಹೇಗೆ ಗೊತ್ತು, ಅದು ಆ ಹೃದಯದಿಂದ ಬಂತು ಎನ್ನುವುದು? ಎಲ್ಲರ ಹೃದಯವೂ ಒಂದೇ ತಾನೇ? ಅಂಥ ಭೇದವೇನು ಅದರಲ್ಲಿ? ಎಂದರೆ ಸ್ಥಾನದಿಂದ ಅರ್ಥಭೇದ. ಈ ವ್ಯವಹಾರ ಲೋಕದಲ್ಲೂ ಇದೆ. ಆಫೀಸಿನಲ್ಲಿ ಕಾಲಿಂಗ್ಬೆಲ್ ಸದ್ದು ಕೇಳಿದರೆ ದೇವಸ್ಥಾನದ ಪೆರಮಣಿ ಎಂದು ದೇವಸ್ಥಾನಕ್ಕೆ ಹೋಗುವುದಿಲ್ಲ. ಹೀಗೆಯೇ ಚರ್ಚಿನ ಘಂಟೆಗೆ ಬೇರೆ ಅರ್ಥ. ಹೀಗೆ ಘಂಟೆ ಒಂದೇ ಆದರೂ ಅದು ಬರುವ ಸ್ಥಾನದಿಂದ ಅದಕ್ಕೆ ಬೇರೆ ಬೇರೆ ಅರ್ಥಗಳು ಇವೆ ಎನ್ನುವುದು ಅನುಭವ ಸಿದ್ಧವಾದ ವಿಷಯವೇ ಆಗಿದೆ. ಹೀಗೆಯೇ ಪಾಮರನೂ ಲೋಕದಲ್ಲಿ ಮಾತನಾಡುವುದುಂಟು, ಪಂಡಿತನೂ ಮಾತನಾಡುವುದುಂಟು. ಆಯಾ‌ ಮಾತನ್ನಾಡಿದ ವ್ಯಕ್ತಿಗನುಗುಣವಾಗಿ ಆಯಾ ಮಾತಿನ ಅರ್ಥವನ್ನು ತೆಗೆದುಕೊಳ್ಳುವುದು ಲೋಕರೂಢಿಯಲ್ಲಿ ಬೆಳೆದು ಬಂದದಿದೆ. ಹೀಗೆ ವಾಲ್ಮೀಕಿಗಳ ಹಾಗೂ ನಮ್ಮಾಳ್ಳಾರರ ಗ್ರಂಥದ ಅರ್ಥ ತಿಳಿಯಬೇಕಾದರೆ, ರಾಮಾಯಣ ಮತ್ತು ಪ್ರಬಂಧಗಳ ಹಿನ್ನೆಲೆಯ ಅರ್ಥವೇನು? ಎನ್ನುವುದನ್ನು ತಿಳಿಯಬೇಕು. 


\section*{ಶಬ್ದದ ಹಿಂದಿರುವ ಅರ್ಥವನ್ನು ಗ್ರಹಿಸಿದಾಗ ಮಾತ್ರ ಮೂಲತಲುಪಲು ಸಾಧ್ಯ}

ಒಂದು ಅರ್ಥವನ್ನು ಹಿಂದಿಟ್ಟುಕೊಂಡೇ ಒಂದು ಶದ್ಬ ಹೊರಡುತ್ತೆ. ಆ ಅರ್ಥವನ್ನು ಗಮನಿಸಿದಾಗಲೇ ಅಥವಾ ತನ್ನ ಹುಟ್ಟಿಗೆ ಕಾರಣವಾದ ಅರ್ಥಕ್ಕೆ  ಕೊಂಡೊಯ್ದಾಗಲೇ ಅದು ಸಾರ್ಥಕ. ಉದಾಹರಣೆಗೆ - ನಾನು ಹಾಡಿರುವುದನ್ನು ಬೆಂಗಳೂರಿನಲ್ಲಿ ಹೋಗಿ ಹಾಡಬೇಕಾದರೆ ನನ್ನ ಹೃದಯ ಬೇಕು. ಅವರು ಹಾಡಿದುದೇನು? ಹೇಗೆ ಹಾಡಿದರು? ಅವರ ಹೃದಯದ ಹಿಂದಿದ್ದ ಭಾವವೇನು? ಎಂದು ಗಮನಿಸಿದಾಗ ತಾನೇ, ನಾನು ಹಾಡಿದ ಹಾಡು ಬರಬಹುದು. ಇಲ್ಲದಿದ್ದರೆ ನನ್ನ ಹಾಡನ್ನು ಅನುಕರಿಸಲಾಗುವುದಿಲ್ಲ. ಅಂದರೆ ನನ್ನ ಮತಿಯನ್ನನುಸರಿಸಿ ಹಾಡಬೇಕು.

\section*{ಕೃತಿ ನಷ್ಟವಾದ ಮಾತ್ರಕ್ಕೆ ಮತಿ ನಷ್ಟವಾಗುವುದಿಲ್ಲ}

ಈಗ ನಮ್ಮಾಳ್ವಾರರ ಕೃತಿ ನಷ್ಟವಾಯಿತು, ಎಲ್ಲಾ ಹೋಯಿತು, ಎಂದು ಹೇಳುವುದುಂಟು. ಅವರ ಕೃತಿ ನಷ್ಟವಾದರೂ ಆ ಕೃತಿಗೆ ಹಿನ್ನೆಲೆಯಾಗಿ ಇರುವ ಮತಿಯೇನೂ ನಷ್ಟವಾಗಿಲ್ಲ. ಕೃತಿ ಹೋದರೆ ಮತಿಯೇ ಹೋಗಿಬಿಟ್ಟಿತೇ? ಅವರ ಮತಿಯನ್ನು ಹುಡುಕಲು ಅವರ ಉಳ್ಳಕ್ಕೆ-ಮನಸ್ಸಿಗೆ ಹೋಗಬೇಕು. ಅವರ ಉಳ್ಳಕ್ಕೆ-ಅವರ ಮತಿಯ ಜಾಗಕ್ಕೆ ಎಂದರೆ, ಅವರಿ ಎಷ್ಟು ಆಳದಿಂದ ಕೃತಿಯನ್ನು ತಂದರೋ ಅಲ್ಲಿಗೇ ಹೋದರೆ ಅಲ್ಲಿ ಕೃತಿ ಸಿಕ್ಕಬಹುದೋ ಇಲ್ಲವೋ? ಅಲ್ಲಿ ಸಿಕ್ಕುತ್ತದೆ ಅವರ ಕೃತಿ. ಅಲ್ಲಿ ಸಿಕ್ಕಿದ ಅವರ ಮತಿನುಗುಣವಾದ ಕೃತಿ ತಾನೇ ಸಜೀವ ಕೃತಿಯಾದೀತು.

\section*{ರಾಮಾಯಣಾದಿಗಳು ಆತ್ಮನವರೆಗೂ ಹೋಗಿ, ನೋಡಿ ಹೊರತಂದವುಗಳು}

ರಾಮಾಯಣ-ಭಾರತ-ಭಾಗವತಗಳು ಕೇವಲ ಲೈಬ್ರರಿಯ ರೆಫರೆನ್ಸೆ ಪುಸ್ತಕಗಳಾಗಿವೆ. ಗೀತೆ-ಉಪನಿಷತ್ತುಗಳು ಕೊಟೇಷನ್ ({\eng Quotation}) ಗೋಸ್ಕರವಾಗಿ ಉಳಿದುಕೊಂಡಿವೆ. ಆದರೆ ಅವು ಜ್ಞಾನಿಗಳ ಹೃದಯದಲ್ಲಿ ಅನುಭವವೇದ್ಯವಾದ ವಿಷಯವನ್ನು ತಿಳಿಸಲು ನಾದ-ಸ್ವರ-ಅಕ್ಷರ ರೂಪವಾಗಿ ಹೊರಟಿದ್ದ ಧ್ವನಿಯಾಗಿವೆ. ಧ್ವನಿಗಳೇ ಪದೆ, ವಾಕ್ಯಗಳಾಗಿವೆ. ಹೀಗೆ ರಾಮಾಯಣದ ಹುಟ್ಟು ವೈಜ್ಞಾನಿಕವಾಗಿಯೇ ಇದೆ. ಇಂತಹ ರಾಮಾಯಣ ಹುಟ್ಟಿದ್ದೆಲ್ಲಿ? ಹೇಗೆ ಹುಟ್ಟಿತು? ಎಂಬುದನ್ನು ರಾಮಾಯಣದಿಂದಲೇ ತಿಳಿಯಬಹುದಾಗಿದೆ. `ಪ್ರಾಚೀನಾಗ್ರೇಷು ದರ್ಭೇಷು'\label{231} ಅಂತಃ ಬಹಿಃ ಪ್ರಾಚೀನಾಗ್ರವಾದ ದರ್ಭೆಯಲ್ಲಿ ಧ್ಯಾನೂರೂಢನಾಗಿ ಕುಳಿತು, ಅಂಗೈನೆಲ್ಲಿಯಂತೆ `ಪಾಣಾವಾಮಲಕಂ ಯಥಾ'\label{232} ಒಳಹೊರ ರೂಪಗಳನ್ನು ತತ್ತ್ವತಃ ನೋಡಿ ರಚಿಸಿದ ಕಾವ್ಯವಾಗಿದೆ ರಾಮಾಯಣ. ವಾಲ್ಮೀಕಿಗಳು ಅಂತಃ ಬಹಿಃ ನೋಡುವುದು ಸರಿ; ನಮಗೆಲ್ಲಿ ಸಾಧ್ಯ ಅದು? ಎಂದರೆ ಅದು ವಾಲ್ಮೀಕಿಗಳಿಗೆ ಮಾತ್ರವೇನೂ ಅಲ್ಲ; ನಮಗೂ ಉಂಟು. `ಮಸಸ್ಸಿನಲ್ಲಿ ಹೇಗೆ ನೋಡೋಣ.' (ವಿಚಾರಿಸಿ, ಸನ್ನಿವೇಶ ನೋಡಿ ಹೇಳಬೇಕು) ಎನ್ನುವ ವ್ಯವಹಾರವಿದೆ. ಹೀಗೆ ಅಂತರಂಗವಾಗಿ ಮನಸ್ಸಿನಲ್ಲಿ ಬಂದ ವಿಚಾರ ಹೊರ ಬೀಳುವುದಕ್ಕೆ ಸನ್ನಿವೇಶ ನೋಡುವುದು ಕಂಡು ಬಂದಿದೆ. ನಾವು ಮನಸ್ಸಿನವರೆಗೆ ಮಾತ್ರ ಒಳಗೆ ನೋಡುತ್ತೇವೆ. ಆದರೆ ವಾಲ್ಮೀಕಿಗಳು ಇನ್ನೂ ಆಳವಾಗಿ ಆತ್ಮನವರೆಗೂ ಹೋಗಿ ಅಲ್ಲಿಂದ ನೋಡಿ ಬರೆದ ಕಾವ್ಯ; ರಾಮಾಯಣ. 

\section*{ಸಂಸ್ಕೃತ-ತತ್ತ್ವ ಭೂಮಿಯಲ್ಲಿ ಹುಟ್ಟಿದ ಭಾಷೆ}

ಲೋಕದಲ್ಲಿ ಕನ್ನಡ-ಇಂಗ್ಲಿಷ್ ಮೊದಲಾದ ಭಾಷೆಗಳಿರುವಂತೆ ಸಂಸ್ಕೃತ ಮತ್ತು ತಮಿಳು. ವಾಲ್ಮೀಕಿಗಳು ರಾಮಾಯಣವನ್ನು ಸಂಸ್ಕೃತದಲ್ಲಿ ಬರೆದರೆ, ಆಳ್ವಾರರು ಪ್ರಬಂಧವನ್ನು ತಮಿಳಿನಲ್ಲಿ ಬರೆದಿದ್ದಾರೆ. ಐತಿಹ್ಯವೊಂದು ರಾಮಾಯಣವು ಸಾಕ್ಷಾದ್ವೇದವೇ ಎಂದು ಹೇಳುತ್ತೆ. `ವೇದಃ ಪ್ರಾಚೇತಸಾದಾಸೀತ್\label{232c} ಸಾಕ್ಷಾತ್ ರಾಮಾಯಣಾತ್ಮನಾ.' ಆದರೆ ರಾಮಾಯಣದ ಆರಂಭ `ತಪಸ್ಸ್ವಾಧ್ಯಾಯ ನಿರತಂ'\label{232} ಎಂದು ಆರಂಭವಾಗಿದೆ. `ಅಗ್ನಿಮೀಳೇ' ಎಂದಾಗಲೀ, `ಇಷೇ ತ್ವೋರ್ಜೇತ್ವಾ' ಎಂದಾಗಲೀ `ಅಗ್ನ ಆಯಾಹಿ' ಎಂದಾಗಲೀ ಆರಂಭವಾಗಿಲ್ಲ. ಅಥವಾ ಮೊದಲು ಬೇಡ; ಮೊದಲಿನಿಂದ ಕೊನೆಯವರೆಗೆ ಹುಡುಕಿದರೂ ಅಶೀತಿದ್ವಯದಲ್ಲಿನ (ಕೃಷ್ಣಯಜುರ್ವೇದ) ಇಲ್ಲವೇ ಯಾವು ವೇದದಲ್ಲಿನ ಮಂತ್ರವೂ ಕಾಣುವುದಿಲ್ಲ. ನಮ್ಮಾಳ್ವಾರರ ಪ್ರಬಂಧದಲ್ಲಿಯಾಗಲೀ ಯಾವುದಾದರೂ ಮಾತಿದೆಯೇ? ತಮಿಳುವೇದ ಕನ್ನಡ ವೇದ ಎಂಬ ವೇದವಿದೆಯೇ? ಬಂಗಾಳಿ, ಐಸಿಷ್, ಸ್ವೀಡಿಷ್ ಮೊದಲಾಗಿ ಭಾಷೆಗಳೆಲ್ಲವೂ ಆಯಾ‌ ದೇಶದ ಹೆಸರಿನಲ್ಲಿ ಹೊರಟಿದೆ. ಹೀಗಿರುವಾಗ ಸಂಸ್ಕೃತ ಭಾಷೆ ಯಾವ ದೇಶದ್ದು? ಸಂಸ್ಕೃತ ದೇಶವಿದೆಯೇ? `ಜಿಹ್ವೇ ಕೀರ್ತಯ ಕೇಶವಂ'\label{232} ಇದು ಸಂಸ್ಕೃತ. ಅದೇ `ಎಲೈ ನಾಲಗೆಯೇ ಕೇಶವನನ್ನು ಕಿರ್ತಿಸು' ಎಂದರೆ ಕನ್ನಡ. ಭಾಷೆ ಯಾವ ಮೂಲದಾಗಿ ಬಂತು? ರಾಮಾಯಣ ಪೂರ್ತಿ ಸಂಸ್ಕೃತಿ ಭಾಷೆಯದು ಸಂಸ್ಕೃತನ್ನಾಶಯಿಸಿದನಂತೆ. `ಇಲ್ವಲಃ ಸಂಸ್ಕೃತಂ ವದನ್'\label{232} ಹೀಗೆಯೇ ಆಂತನೇಯರು ಆಶೋಕವನದಲ್ಲಿ ಸೀತೆಯೊಡನೆ  ಸಂಸ್ಕೃತವೂ ಮಾನುಷವೂ ಆದ ಭಾಷೆಯನ್ನಾಡಿದರಂತೆ-

\begin{shloka}
ವಾಚಂ ಚೋದಾಹರಿಷ್ಯಾಮಿ ಮಾನುಷೀಮಿಹ ಸಂಸ್ಕೃತಾಮ್|\label{232b}\\
ಯದಿ ವಾಚಂ ಪ್ರದಾಸ್ಯಾಮಿ ದ್ವಿಜಾತಿರುವ ಸಂಸ್ಕೃತಾಮ್|\\
ರಾವಣಂ ಮನ್ಯಮಾನಾ‌ ಸಾ ಸೀತಾ ಭೀತಾ ಭವಿಷ್ಯತಿ||
\end{shloka}

`ಬ್ರಾಹ್ಮಣರಂತೆ ಸಂಸ್ಕೃತವನ್ನು ಮಾತನಾಡಿದರೆ ರಾವಣನೆಂದು ತಿಳಿದು ಸೀತೆ ಭೀತಳಾಗುತ್ತಾಳೆ. ಆದ್ದರಿಂದ ಸಂಸ್ಕಾರದಿಂದ ಕೂಡಿದ ಮನುಷ್ಯರ ಭಾಷೆಯನ್ನೇ ಆಡುತ್ತೇನೆ' ಎಂದು ನಿಶ್ಚಯಿಸಿಕೊಂಡನು ಎಂದಿದೆ. ಹೀಗಿರುವಾಗ ಯಾವುದು ಸಂಸ್ಕೃತ? ನಮ್ಮದು ತಮಿಳ್ವೇದ, ಆದರ ಮಹಿಮೆಯೇ ಬೇರೆ ಅಂದರೆ ರಾಮಾಯಣದಲ್ಲಿ ತಮಿಳು ವೇದವೂ ಇಲ್ಲ. ಆದ್ದರಿಂದ ತಮಿಳು ಕನ್ನಡ ಮೊದಲಾದ ಭಾಷೆಗಳು ದೇಶದ ಮೇಲೆ ನಿಂತಿದೆ. ಆದರೆ ಯಾವುದೇ ದೇಶದಲ್ಲಿ ಹುಟ್ಟದಿದ್ದರೂ ತತ್ತ್ವಭೂಮಿಯಲ್ಲಿ ಹುಟ್ಟಿದ ಭಾಷೆ ಸಂಸ್ಕೃತ. 

\section*{ಒಂದೇ ಶಬ್ದವು ಧ್ವನಿವ್ಯತ್ಯಾಸದಿಂದ ವಿಭಿನ್ನ ಅರ್ಥಗಳನ್ನು ಕೊಡುತ್ತದೆ}

(ಈ ನಡುವೆ ಒಂದು ಸಂಭಾಷಣೆ)

ಶ್ರೀರಙ್ಗರವರು - ತಮಿಳಿನಲ್ಲಿ ಆತ್ಮ ಎನ್ನುವದಕ್ಕೆ ಏನು ಹೇಳುತ್ತೀರಿ? \\
ಶಿಷ್ಯರೊಬ್ಬರು - ಆವಿ. \\
ಶ್ರೀರಙ್ಗರವರು - ಆವಿ ಎಂದರೆ ಹಬೆ ಎಂದೂ ಅರ್ಥವಿದೆ. ಈ ಆವಿಗಳೆರಡಕ್ಕೂ ವ್ಯತ್ಯಾಸವೇನು? \\
ಶಿಷ್ಯರೊಬ್ಬರು - ಸಂದರ್ಭಾನುಸಾರ ಅರ್ಥಮಾಡಿಕೊಳ್ಳಬೇಕಾಗುತ್ತೆ.\\
ಶ್ರೀರಙ್ಗರವರು - ಆದರೂ ಒಂದೇ ಶಬ್ದ ಬೇರೆ ಬೇರೆ ಅರ್ಥಕೊಡಬೇಕಾದರೆ  ಅದರ ಉಚ್ಚಾರಣೆ ಬೇರೆ ಬೇರೆ ರೀತಿಯಾಗುತ್ತದೆ. ಅಂದರೆ ಟೋನ್ ({\eng Tone}) ಬದಲಾಯಿಸಬೇಕಾಗುತ್ತದೆ.

ಶಿಷ್ಯರೊಬ್ಬರು - ಆವಿಯಲ್ಲಿ ಸುಡುವ ಧರ್ಮವಿದೆ, ಹಾಗೆಯೇ ಆತ್ಮವಸ್ತುವೂ ತೇಜಸ್ಸಾದುದರಿಂದ ಅದರ ಸಾಮ್ಯದ ಮೇಲೆ ಆವಿ ಎನ್ನುವುದು ಬಂದಿರಬಹುದು. ಇದು ಒಂದು ಯುಕ್ತಿ. 

ಶ್ರೀರಙ್ಗರವರು - ಆದರೆ ಆತ್ಮ ಎನ್ನುವ ಪದ ವಿಭಿನ್ನವಾದ ಅರ್ಥದಲ್ಲಿ ವ್ಯವಹಾರದಲ್ಲಿದೆ. ಉದಾಹರಣೆಗಳಲ್ಲಿ ನೋಡಿ. ದೇಹ-ಜೀವಾತ್ಮ-ಪರಮಾತ್ಮ ಈ ಎಲ್ಲ ಅರ್ಥಗಳಲ್ಲಿಯೂ ಅದರ ವ್ಯವಹಾರವಿದೆ. ರಾಮಾಯಣ ಮತ್ತು ಪ್ರಬಂಧದ ಮಧ್ಯೆ ಧ್ವನಿಯ ಬಗ್ಗೆ ಏಕೆ ಚರ್ಚೆ ಎಂದರೆ, ರಾಮಾಯಣವಾಗಲಿ, ಪ್ರಬಂಧವಾಗಲಿ ಮಹರ್ಷಿಗಳ ಯೋಗಭೂಮಿಕೆಯಿಂದ ಬಂದು ತನ್ನದೇ ಆದ ಅರ್ಥವನ್ನು ಕೊಡುತ್ತವೆ - ಎನ್ನುವುದನ್ನು ತಿಳಿಯಲು ಇದು ಅವಶ್ಯಕವಾಗಿದೆ. ಆರುಸಾವಿರ ಪಡಿ, ಹನ್ನೆರಡು ಸಾವಿರ ಪಡಿ, ಇಪ್ಪತ್ನಾಲ್ಕುಸಾವಿರ ಪಡಿ, ಮೂವತ್ತೆರೆಡು ಸಾವಿರಪಡಿ, ನಲವತ್ತೆಂಟುಸಾವಿರ ಪಡಿ, ಎಪ್ಪತ್ತೆರಡುಸಾವಿರ ಪಡಿ ಮೊದಲಾಗಿ, ಆಳ್ವಾರರ ಪ್ರಬಂಧಕ್ಕೆ ವಿವಿಧ ವ್ಯಾಖ್ಯಾನಗಳುಂಟು. ಈ ವ್ಯಾಖ್ಯಾನಗಳಲ್ಲಿ ಯಾವುದು ಸರಿ? ಎಲ್ಲರೂ ಒಂದೊಂದು ಅರ್ಥವನ್ನು ಹೇಳುತ್ತಾರೆ. ಒಂದು ಶಬ್ಧಕ್ಕೆ ನಿರ್ದಿಷ್ಟವಾದ ಅರ್ಥವೇ ಇಲ್ಲವೇ? ಎಂದರೆ ಒಂದು ಶಬ್ದಕ್ಕೆ ಒಂದು ನಿರ್ದಿಷ್ಟವಾದ ಅರ್ಥವಿದೆ. ಒಂದೇ ಅಕ್ಷರಗಳುಳ್ಳ ಶಬ್ದ ವಿವಿಧಾರ್ಥಗಳನ್ನುಳ್ಳದ್ದೆಂದು ತೋರಬಹುದಾದರೂ ಆ ವಿವಿಧಾರ್ಥಗಳನ್ನು ಕೊಡಲು ಅದರ ಉಚ್ಚಾರಣೆಯಲ್ಲೂ ವೈವಿಧ್ಯವಿರುತ್ತದೆ. ಆದ್ದರಿಂದ ಧ್ವನಿಯ ಮೇಲೆ ಅರ್ಥ. ಆವಿ ಎಂದು ನೀವು ಉಚ್ಚರಿಸಿದ ಧ್ವನಿಯಲ್ಲಿ ಆತ್ಮ ಆಗೋಲ್ಲ. ಧ್ವನಿವ್ಯತ್ಯಾಸವಾದರೆ ಅರ್ಥವ್ಯತ್ಯಾಸವುಂಟಾವುದಕ್ಕೆ ಅನೇಕ ಉದಾಹರಣೆಗಳನ್ನು ಲೋಕದಲ್ಲಿಯೂ ಕಾಣಬಹುದು. `ಅಂಗೇ ಸ್ವಾಮಿ' ಎನ್ನುವಲ್ಲಿ ಸಂಸ್ಕೃತದ `ಅಂಗೇ' ತೆಗೆದುಕೊಂಡರೆ ಅಂಗದಲ್ಲಿ ಎಂದಾಗಿಬಿಡುತ್ತದೆ. ಅದನ್ನು ಬೇರೆ ರೀತಿಯಲ್ಲಿ ಹೇಳಿದರೆ ಅರ್ಥವಿಲ್ಲ. ಪುಸ್ತಕ ಹುಡುವಾಗ `ಎಲ್ಲಿ' ಎಂದು ಕೇಳಿದರೆ `ಅಲ್ಲೇ' ಎನ್ನುವುದು ಮೊದಲನೆಯ ಬಾರಿಯ ಉತ್ತರ. ಮತ್ತೆ ಎರಡನೆಯ ಬಾರಿ ಕೇಳಿದಾಗಲೂ ಸ್ವಲ್ಪ ಆಗ್ರಹದೊಡನೆ `ಅಲ್ಲೇ' ಬರುತ್ತದೆ. ಮತ್ತೆ `ಎಲ್ಲಿ?' ಎಂದು ಕೇಳಿದರೆ `ಅಲ್ಲೇ' ಎಂದು ವ್ಯಂಗ್ಯಪೂರ್ವಕ ಆಗ್ರಹ. ಇಲ್ಲಿ ಮೂರನೆಯ `ಅಲ್ಲೇ' ಗೆ ಆಗ್ರಹವೇ ಅರ್ಥ. ಆಗ್ರಹದಿಂದ, `ಸ್ವಾಮಿನೇ ಮಮ ಮಙ್ಗಲಮ್' ಎಂದರೆ ಅಲ್ಲಿ ಆಗ್ರಹವೇ ಧ್ವನಿಯ ಅರ್ಥ. `ಸ್ವಾಮಿ, ಇಪ್ಪಡಿಯಾ ನೀ ಪಣ್ಣರದ್,' ಎನ್ನುವಾಗ ಏಯ್; ಹೀಗೆ ನೀವು ಮಾಡುವುದು ಎನ್ನುವಂತೆ `ಸ್ವಾಮೀ' ಒಂದು ರೀತಿ ತಿರಸ್ಕಾರಾರ್ಥಕೊಡುವ ಸಂಬೋಧನೆ. ಆವಿ ಅಡಪ್ಪಾವಿ ಎನ್ನುವಾಗ ಎರಡು ಕಡೆಯೂ ಆವಿ ಇದೆ. ಆದರೆ ಅರ್ಥವ್ಯತ್ಯಾಸವುಂಟು. ಲೋಕದಲ್ಲಿ ಸಂದರ್ಭಾನುಸಾರ ಹಿಂದೆ ಮುಂದೆ ನೋಡಿಕೊಂಡು ಅರ್ಥಕೊಡಬೇಕು. `ಎಲಾ ಇವನಾ' ತಕ್ಷಣ ಗೊತ್ತು; ಮುಂದೆ ಏನೋ ಆಪತ್ತ್ತು ಬಂದಿದೆ ಎಂದು. `ಸ್ವಾಮಿ, ವಂದ್ಕೊಂಡಿರ್ಕಾರ್', ಎಂದಾಗ ಲೊಚ್ಚೆಹಕಿದರೆ ಅಲ್ಲಿ ರಸಾಸ್ವಾದನೆ ಅರ್ಥವಲ್ಲ. ಅವರ ಆಗಮನ ಬಗ್ಗೆ ಇರುವ ಅನುತ್ಸಹ ಇದರ ಅರ್ಥ. ಶಬ್ದಕ್ಕೆ ಲಿಪಿಯಿಂದ ಅರ್ಥವನ್ನು ಹೇಳುವುದಕ್ಕಾಗುವುದಿಲ್ಲ. ಭಗವನ್ತನನ್ನು ನೋಡಿ `ಅಹಹಹಹ,' ಎನ್ನುವುದುಂಟು. `ಸೌಂದರ್ಯ ಅಲ್ಲಿ ಅರ್ಥ, ಆದರೆ ಅದೇ ಮುಳ್ಳು  ಚುಚ್ಚಿದಾಗ `ಅಹಹಹ,' ಎಂದರೆ ಅಲ್ಲಿ `ನೋವು' ಅರ್ಥ. ನಗುವಿನ ಸಂದರ್ಭದಲ್ಲಿ `ಅಹಹಹಹ' ಎಂದರೆ ಅಲ್ಲಿ ಸಂತೋಷವೇ ಅರ್ಥ. ಖಾರವಾದಗಲೂ `ಅಹಹಹಹ' ಎನ್ನುವುದುಂಟು. ಲಿಪಿಯಲ್ಲಿ ಎಲ್ಲ ಒಂದೇ. ಆದರೆ ಧ್ವನಿಯ ಮೇಲೆ ಅರ್ಥವ್ಯತ್ಯಾಸ ಪುಸ್ತಕದಲ್ಲಿ ಮಾತ್ರ ಒಂದು ಶಬ್ದಕ್ಕೆ ಅನೇಕಾರ್ಥ ಎಂದಿಲ್ಲ. ಸಹಜವಾಗಿಯೇ ಲೋಕದಲ್ಲಿಯೂ ಹಾಗಿದೆ. 

\section*{ಶಬ್ದವು ಕ್ರಿಯೆಗೆ ಸಹಾಯಕ ಎನ್ನುವ ಬಗ್ಗೆ ಗುರುವಿನ ಪ್ರಯೋಗಾತ್ಮಕ ವಿವರಣೆ}

(ಶಿಷ್ಯರೊಬ್ಬರನ್ನು ಕುರಿತು)\\
ಶ್ರೀಗುರುಗಳು - ನೀವು ಸ್ವಲ್ಪ ತಮಿಳಿನಲ್ಲಿ ಅಳುವಿರಾ?\\
ಶಿಷ್ಯರು - ಅದೆಪ್ಪಡಿ ಮುಡಿಯುಂ (ಅದು ಹೇಗೆ ಸಾಧ್ಯ?)\\

ಶ್ರೀಗುರುಗಳು - ಏಕೆ ಆಗುವುದಿಲ್ಲ? ಎಂದರೆ ಅಳು ಎಲ್ಲರಿಗೂ ಒಂದೆ. ಆದ್ದರಿಂದ ಕ್ರಿಯೆ ಬಂದಾಗ ತನ್ನಿಂದ ತಾನೇ ಧ್ವನಿ ಹುಟ್ಟಿಕೊಳ್ಳುತ್ತದೆ. ಆ ಧ್ವನಿಗೆ ಆ ಧ್ವನಿಯೇ ಅರ್ಥ. ಆಯಿತು, ಆಕಳಿಕೆ ಎನ್ನುವುದು ಯಾವ ಭಾಷೆ?

ಶಿಷ್ಯರೊಬ್ಬರು - ಕನ್ನಡ\\
ಮತ್ತೊಬ್ಬರು - ತಮಿಳಿನಲ್ಲಿ ಅದನ್ನು `ಕೊಟ್ಟಾವಿ' ಎನ್ನುತ್ತಾರೆ.\\
ಶ್ರೀಗುರುಗಳು - ಹಾಗಾದರೆ ಕೊಟ್ಟಾವಿ ಎಂದು ಹೇಳಿಕೊಂಡು ಕೊಟ್ಟಾವಿ ಬಿಡಿ. (೫-೬ ಬಾರಿ ಪ್ರಯತ್ನಿಸಿದರೂ ಕೊಟ್ಟಾವಿ ಬರಲಿಲ್ಲ. ಕೊಟ್ಟಾವಿ ಎಂದಾಗ ಆಕಳಿಕೆ ನಿಲ್ಲುತ್ತಿತ್ತು) ಅದೇ ಆಕಳಿಕೆ ಎಂದುಕೊಂಡು ಆಕಳಿಸಲು ಸೌಲಭ್ಯವಿದೆ.

(ಇದನ್ನು ಪ್ರಯೋಗಾತ್ಮಕವಾಗಿ ತೋರಿಸಿಕೊಟ್ಟರು)

ಒಂದು ಶಬ್ದ ಹೊರಟರೆ ಅದು ನಡೆಯುವ ಕ್ರಿಯೆಗೆ ಅವಿರೋಧವಾಗಿ ಬರಬೇಕು. ಸೀತ್ಕಾರ, ಫೂತ್ಕಾರ ಇವು ಯಾವ ಭಾಷೆಯ ಶಬ್ದ? ಸಂಸ್ಕೃತ  ಎನ್ನುವುದಾದರೆ ತಮಿಳಿನಲೇನು?

ಶಿಷ್ಯರೊಬ್ಬರು - ಉರುವಿ.

ಶ್ರೀಗುರುಗಳು:- ಹಾಗಾದರೆ ತಮಿಳುದೇಶದವರು ಪಾಯಸವನ್ನು `ಪೂ' ಎಂದೇನೂ ಸವಿಯುವುದಿಲ್ಲವಲ್ಲ! `ಸೀತ್' ಎನ್ನುವ ಶಬ್ದತಾನೇ ಅವರು ಸವಿಯುವಾಗಲೂ ಬರುತ್ತದೆ. ಉರುವಿಯೂ ಇಲ್ಲ. ಹೀಗೆಯೇ ಫೂತ್ ಎನ್ನುವಾಗ ಉಗುಳುವಾಗ ಫ್ ಫ್ ಎಂದು ಒಳಕ್ಕೇನೂ ಉಗುಳನ್ನೆಳೆದುಕೊಳ್ಳುವುದಿಲ್ಲವಲ್ಲ! ಹಾಗೆ ಸೀತ್ಕಾರ, ಫೂತ್ಕಾರ ಮೊದಲಾದ ಶಬ್ದಗಳು ಸಂಸ್ಕಾರದ ಮೇಲೆ ಬರುವುದರಿಂದ ಸಂಸ್ಕೃತ ಶಬ್ದಗಳೇ. 

\section*{ವಸ್ತುವಿನ ಆಂಶಿಕ ಕ್ರಿಯೆ - ಧರ್ಮಗಳಿಂದ ಪರ್ಯಾಯಪದಗಳ ಹುಟ್ಟು }

ಸಂಸ್ಕೃತದಲ್ಲಿ ಜೃಂಭಣ ಎನ್ನುವ ಶಬ್ದವಿದೆಯಲ್ಲ; ಅದೂ ಕ್ರಿಯೆಗೆ ಸಹಾಯಕವಾಗುವುದಿಲ್ಲ. ಆದರೆ ಆ ಶಬ್ದ ಹೇಗೆ ಬಂದಿದೆ? ಅದು ಕ್ರಿಯೆಯಲ್ಲಿ ನಡೆಯುವ ಯಾವುದೋ ಒಂದಂಶವನ್ನು ಗಮನಿಸಿ ಹೇಳುತ್ತದೆ. ಆದ್ದರಿಂದ ಅದು ಪರ್ಯಾಯಪದ. ಉದಾಹರಣೆಗೆ (ಶಿಷ್ಯರೊಬ್ಬರನ್ನು ಕುರಿತು) ನಿಮ್ಮ ಹೆಸರೇನು? 

ಶಿಷ್ಯರು - ಶ್ರೀನಿವಾಸ 

(ಅವರ ಅಣ್ಣನನ್ನು ಕುರಿತು) ಶ್ರೀಗುರುಗಳು - ಮನೆಯಲ್ಲಿ ಅವರನ್ನು ಇನ್ನೇನಾದರೂ ಕೂಗುವುದುಂಟೋ?

ಆ ಶಿಷ್ಯರು - ಪಾಪಚ್ಚಿ.

ಶ್ರೀಗುರುಗಳು - ಹಾಗಾದರೆ ಶ್ರೀನಿವಾಸ, ಪಾಪಚ್ಚಿ ಪರ್ಯಾಯಪದಗಳಾದವು. ಶ್ರೀನಿವಾಸ ಎಂಬ ಪದ ಭಗವಂತನಲ್ಲೆ ಅನ್ವರ್ಥವಾಗಿದ್ದು ಇವರಲ್ಲಿ ರೂಢವಾಗಿದೆ. ಹಾಗೆ ಇಂತಹವರ ತಮ್ಮ ಎನ್ನಬಹುದು. ಇವರ ತಂದೆಯ ಕಡೆಯವರು ಇಂತಹವರ ಮಗ ಎನ್ನಬಹುದು. ಇವರ ಹೆಂಡತಿಯ ಕಡೆಯವರು ಇಂತಹವರ ಯಜಮಾನರು ಎನ್ನಬಹುದು. ಹಾಗೆಯೇ ಇವರ ಸ್ನೇಹಿತರ ಕಡೆಯಿಂದ ಇಂತಹವರ ಸ್ನೇಹಿತ ಎನ್ನಬಹುದು. ಹೀಗೆ ಅವರವರು ಗಮನಿಸಿರುವ ಸಂಬಂಧಕ್ಕನುಸಾರವಾಗಿ ಇವರಿಗೆ ಪರ್ಯಾಯಪದಗಳು ಹುಟ್ಟಿಕೊಳ್ಳುತ್ತವೆ. ಈ ಎಲ್ಲ ಪದಗಳೂ ಇವರಿಗೆ ಅನ್ವಯಿಸುವುದಾದರೂ ಎಲ್ಲಕ್ಕೂ ಒಂದೇ ಅರ್ಥವಲ್ಲ ಕೆಲವು ಇವರನ್ನು ನೇರವಾಗಿ ತಿಳಿಸಿದರೆ, ಇನ್ನು ಕೆಲವು ಇವರ ಸಂಬಂಧವನ್ನು ಇತರರ ಮೂಲಕ ಹರಿಯಿಸಿ ಇವರನ್ನು ತಿಳಿಸುತ್ತವೆ. ಹೀಗೆ ಒಂದೇ ಪದಾರ್ಥವನ್ನು ತಿಳಿಸಲು ಅನೇಕ ಪದಗಳಿದ್ದರೂ ಎಲ್ಲ ಪದಗಳಿಗೂ ಒಂದೇ ಅರ್ಥವಲ್ಲ. ಅದು ಯಾವ ಧರ್ಮವನ್ನು ಪದಾರ್ಥದಲ್ಲಿ ತಿಳಿಸುತ್ತದೆ ಎಂದು ನೋಡಬೇಕು. ಉದಾಹರಣೆಗೆ ನಳಿನಿ ಎನ್ನುವ ಪದ ಕಮಲಕ್ಕುಂಟು. ನೀರಜ, ಪಂಕಜ, ಕ್ಷೀರಜ, ಕಂಜ, ಪದ್ಮ. ಶತಪತ್ರ, ಕುಶೇಶಯ ಹೀಗೆ ಅನೇಕ ಪದಗಳಿದ್ದರೂ ನಾಳ-ನೀರು-ದಳ ಮೊದಲಾದ ಭಿನ್ನಭಿನ್ನ ಧರ್ಮಗಳ ಮೇಲೆ ಶಬ್ದ ಹೊರಟಿದೆ ಎನ್ನುವುದನ್ನು ಗಮನಿಸಿದಾಗ ಅದರ ಅರ್ಥಸಿಗುತ್ತದೆ. ಒಂದು ಪದದ ಅರ್ಥದಲ್ಲಿರುವ ಪೂರ್ಣಧರ್ಮವನ್ನು ಗಮನಿಸಬೇಕು. ಅನಂತರ ಯಾವ ಧರ್ಮವನ್ನು ಪುರಸ್ಕರಿಸಿ ಪದ ಹೊರಟಿದೆ ಎನ್ನುವುದನ್ನು ನೋಡಬೇಕು. ಆಗ ಆ ಪದದ ಅರ್ಥತಿಳಿಯುತ್ತದೆ.

(ಅನಂತರ ಶಿಷ್ಯರೊಬ್ಬರನ್ನು ಕುರಿತು)

ಶ್ರೀಗುರುಗಳು - ತಮಿಳಿನಲ್ಲಿ ಬಿಕ್ಕಳಿಕೆ ಎನ್ನುವುದಕ್ಕೆ ಏನು ಹೇಳುತ್ತಾರೆ?

ಶಿಷ್ಯರು - ವಿಕ್ಕಲ್ ಎನ್ನುತ್ತಾರೆ. 

(ಶ್ರೀಗುರುಗಳು ತಮ್ಮ ಮಾತನ್ನು ಮುಂದುವರಿಸುತ್ತಾ) ನೋಡಿ, ಅಲ್ಲಿ ಪದ ಕ್ರಿಯೆಗೆ ಅನುಸಾರವಾಗಿ ಹುಟ್ಟಿಕೊಂಡಿದೆ. ಕನ್ನಡದಲ್ಲಿ ಬಿಕ್ ಎಂದು ದೇಶಭೇದದ ಮೇಲೆ ಒತ್ತಿ ಬಿಕ್ಕಳಿಸುವುದರಿಂದ ಬಿಕ್ಕಳಿಕೆಯಾಗಿದೆ. ಇದನ್ನು ಸಂಸ್ಕೃತದಲ್ಲಿ ಹಿಕ್ಕಾ‌ ಎನ್ನುತ್ತಾರೆ. ಹೀಗೆ ಪದ ಅರ್ಥದತ್ತ ತಲುಪಿಸಲು ಸಹಾಯಕವಾಗಿದೆ. 

\section*{ಏಕದೇಶಗ್ರಹಣದಿಂದ ಅರ್ಥ ಮಾಡಿಹೊರಟರೆ ಅನರ್ಥ}

ಒಬ್ಬ ಮನುಷ್ಯ ಬೇಲಿ ಹತ್ತಿರ ಹಾವು ನೋಡಿದ. ಘಟಸರ್ಪ ಭುಸ್ ಎಂದು ಕಚ್ಚುವುದಕ್ಕೆ ಬಂತು. ಅನಂತರ ಓಡಿಬಂದು ಗಾಬರಿಯಿಂದ ಕೈಯನ್ನು ಹೆಡೆಯಾಕಾರಕ್ಕೆ ಹಿಡಿದು ಬಾಯಲ್ಲಿ ಭುಸ್ ಎನ್ನುವ ಧ್ವನಿ ಮಾಡಿದ. ಇಲ್ಲಿ ಈ ಅಭಿನಯದಿಂದ ಹೃದಯಜ್ಞನಾದವನು, ಅವನ ಗಾಬರಿ ಮೊದಲಾದುವನ್ನು ಗಮನಿಸಿ ಸರ್ಪಭೀತಿಯನ್ನು ಹಿಡಿಯಬಲ್ಲ. ಆದರೆ ಕೇವಲ ಅಭಿನಯ ಮತ್ತು ಶಬ್ದಗಳನ್ನು ಗಮನಿಸಿದರೆ, ಓಡಿ ಬಂದು ಸುಸ್ತಾಯಿತು ಉಸ್ ಎಂದ `ಸ್ವಲ್ಪ ತಡೆದು ಹೇಳುತ್ತೇನೆ' ಎಂದು ಕೈ ಅಲ್ಲಾಡಿಸಿದ ಎಂದೂ, ಹೀಗೆಯೇ ಅಭಯಮುದ್ರೆಯನ್ನು ತೋರಿಸುತ್ತಿದ್ದಾನೆಯೋ ಏನೋ ಎಂದೂ ವಿವಿಧ ವ್ಯಾಖ್ಯಾನಗಳು ಹುಟ್ಟಿಕೊಳ್ಳುತ್ತವೆ. ಹೀಗೆ ಏಕದೇಶವನ್ನು ನೋಡಿ ಕೇವಲ ಅದರಿಂದಲ್ಲೆ ಅರ್ಥ ಮಾಡುಹೊರಟಾಗ ಆಗುವುದು. 

\section*{ಪದದ ಉಚ್ಚಾರಣೆಯು ಮೂಲಾರ್ಥಕ್ಕೆ ಕರೆದೊಯ್ಯುವಮ್ತಿರಬೇಕು}

ಜ್ಞಾನಿಗಳ ಹೃದಯದಿಂದ ಬಂದ ಒಂದು ಧ್ವನಿಯನ್ನು ಅದರ ಅಭಿರಪ್ರಾಯ ಕೆಡದಂತೆ ಹೇಗೆ ಇಡಬೇಕು ಎಂಬುದನ್ನು ಅರಿಯಬೇಕು. ಒಬ್ಬನ ಮಗುವಿಗೆ ಜ್ವರ ಬಂದು `ಉಂ ಉಂ' ಎಂದು ನರಳುತತ್ತಿತ್ತು. ಅವನು ಡಾಕ್ಟರರ ಹತ್ತಿರ ಹೋಗಿ `ಸ್ವಾಮಿ, ಮಗು ಉ ಹು ಹುಂ ಎನ್ನುತ್ತಿದೆ' ಎಂದರೆ ಡಾಕ್ಟರ್ ಏನು ಅರ್ಥ ಮಾಡಿಕೊಳ್ಳಬೇಕು.? ಇದು ಮಗುವಿನ ಕಾಯಿಲೆಯನ್ನು ಆಭಿವ್ಯಕ ಪಡಿಸುವ ಧ್ವಾನಿಯಾಗಲಿಲ್ಲ. ಹೀಗೆ ಧ್ವನಿ ಬದಲಾಯಿಸಿದರೆ ಮೂಲದ ಅರ್ಥವನ್ನು ಹಿಡಿಯಲಾಗುವುದಿಲ್ಲ. ಪದದ ಧ್ವನಿ ಅಥವಾ ಉಚ್ಚಾರಣೆಯು, ಪದ ಯಾವ ಅರ್ಥವನ್ನು ಹೊತ್ತು ಹೊರಟಿತೋ ಆ ಅರ್ಥಕ್ಕೆ ಸ್ವಲ್ಪವೂ ಕುಂಡಿಲ್ಲದಂತೆ ಕರೆದೊಯ್ಯುವಂತಿರಬೇಕು. (ಆವಿ ಎನ್ನುವುದನ್ನು  ಆರ್ಥಕ್ಕನುಗುಣವಾಗಿ ಉಚ್ಚರಿಸುವುದಾದರೆ, ಹೀಗೆ ಉಚ್ಚರಿಸಬೇಕು; ಎಂದು ಆತ್ಮಾರ್ಥಕವಾದ ಉಚ್ಚಾರಣೆಯನ್ನು ತೋರಿಸಿದರು.)

\section*{ಮಾತಿನ ಹಿಂದಿನ ಭಾವವನ್ನಸರಿಸಿ ಅರ್ಥಮಾಡಬೇಕು}

ಕೇವಲ ಪದ ಹಿಡಿದುಕೊಂಡು ಅರ್ಥಮಾಡಿಬಿಟ್ಟರೆ, ರಾಮಾಯಣವಾಗಲಿ ಆಳ್ವಾರರ ಕೃತಿಯಾಗಲಿ ನಿರರ್ಥಕವಾಗುವುವು. ವ್ಯವಹಾರದಿಂದ ಅರ್ಥವ್ಯತ್ಯಾಸವಾಗಿಬಿಡುತ್ತದೆ. `ನಿಧಾನವಯ್ಯಾ' ಎಂದು ಜೋರಾಗಿ ಉಚ್ಚರಿಸಿದರೆ ಅದಕ್ಕೆ ವೇಗವೆಂದೇ ಅರ್ಥ. ವಾಲ್ಮೀಕಿಮಹರ್ಷಿಗಳು ಜ್ಞಾನಾರೂಢರಾಗಿ ಯೋಗಭೂಮಿಯಲ್ಲಿ ತತ್ತ್ವತಃ ನೋಡಿ, ಸನ್ನಿವೇಶಾನುಗುಣವಾಗಿ ಅಂತರ್ಧರ್ಮವನ್ನೂ ಬಹಿರ್ಧರ್ಮವನ್ನೂ ಭಾಷಾ ರೂಪವಾಗಿ ಹೊರಬಿಟ್ಟಿದ್ದಾರೆ. ಹಾವನ್ನು ಸೂಚಿಸಲು ಹೊರಟಾಗಲೂ ಒಂದು ಹೊರಗಣ ಧ್ವನಿಯ ಅವಲಂಬನ ಬೇಕೇ ಬೇಕು. ರಾಮಾಯಣದ ಅರ್ಥವನ್ನು ವಾಲ್ಮೀಕಿಯ ಹೃದಯದಲ್ಲಿ ಹೋಗಿ ಹುಡುಕಬೇಕು.

ಗ್ರಂಥವೊಂದೇ ಅರ್ಥಕೊಡುವುದಿಲ್ಲ. ಗ್ರಂಥದ ಹಿಂದಿನ ಭಾವವೂ ಬೇಕು. ದೊಡ್ಡಮರ ಎನ್ನುವಾಗ ಅದರ ಎತ್ತರವನ್ನು ಅಡಿಗಳಲ್ಲಿ ಅಳೆಯಬಹುದು. ಪ್ರಾಣವನ್ನು ಬಹಳ ಉನ್ನತಸ್ಥಿತಿಗೆ ಕೊಂಡೊಯ್ದು ಎಂದರೆ ಏನು? ಹತ್ತಡಿ ಎತ್ತರಕ್ಕೆ? ಶರೀರ ಕುಳಿತಾಗ ಎರಡೂವರೆ ಅಡಿ ಮೊತ್ತ ಕಾಣುತ್ತದಲ್ಲ! ಎಂದರೆ ಇಲ್ಲಿ ಔನ್ನತ್ಯ ಶರೀರದ ಆಳತೆಯ ಮೇಲಲ್ಲ. 

\section*{ಜ್ಞಾನಿಗಳ ಹೃದಯದೊಡನೆ ಅವರ ಕೃತಿಗಳನ್ನು ಹೊರಗಿಡಬೇಕು}

ವಾಲ್ಮೀಕಿಮಹರ್ಷಿಗಳು ಯೋಗಭೂಮಿಯಿಂದ ಕಾವ್ಯವನ್ನು ಹೊರತಂದರು. ಅನಂತರ ತಂದದ್ದೇನೋ ಆಯಿತು, ಇದನ್ನು ಹೊರಗಿಡವುದು ಹೇಗೆ? ಕಾವ್ಯತನ್ನ ಹೃದಯದಿಂದ ಯಾವ ಧರ್ಮದೊಡನೆ ಹೊರಟಿತೋ, ಆ ಧರ್ಮವನ್ನು ಸ್ವಲ್ಪವೂ ಮಾಸದೇ ಗ್ರಹಿಸಿ ಹಾಗೆಯೇ ಹೊರಗಿಡುವವರು ಯಾರು? ಎನ್ನುವ ಯೋಚನೆ. ಹಾವು ಒಂದು ಪೊರೆಯನ್ನು ಬಿಟ್ಟುಹೋದರೆ, ಎಂದೂ ಹಾವು ನೋಡದ ಮನುಷ್ಯನಿಗೆ ಅದರಿಂದ ಭಯವಾಗುವುದಿಲ್ಲ. ಎಂದಾದರೂ ಹಾವು ನೋಡಿದ್ದರೆ ಅದರಿಂದ ಭಯವಾಗಬಹುದು. ಹೀಗೆಯೇ ಮಹರ್ಷಿಗಳ ಹೃದಯವನ್ನೇ ಅರಿಯದವರಿಗೆ ಕಾವ್ಯಸಿಕ್ಕಿದಾಗ ಕೇವಲ ಪೊರೆಯಂತೆ ಅದು ಭಾವಗ್ರಾಹಿಯಾಲಾರದು. ವಾಕ್ ಎಂದರೆ ಮಾತು ಅರ್ಥವಿಲ್ಲದಿದ್ದರೆ ಕೇವಲ ಶೋಲೆ. ಆದ್ದರಿಂದಲೇ ವಾಗರ್ಥಾವಿವ ಸಂಪೃಕ್ತೌ ವಾಗರ್ಥಗಳಂತೆ\label{238} ಸೇರಿಕೊಂಡು ಇರುವವರು ಪ್ರಕೃತಿ ಪುರುಷರು. ಆದ್ದರಿಂದಲೇ ಪದ=ಪ್ರಕೃತಿ+ಪ್ರತ್ಯಯ ಎಂದು ವ್ಯವಹಾರ. ಹಾವೇ ನೋಡದೆ ಪೊರೆಯನ್ನು ನೋಡಿದರೆ ಏಕದೇಶ. ಹೀಗೆ ಜ್ಞಾನಿಗಳ ಹೃದಯವಿಲ್ಲದೆ ಕೇವಲ ಮಾತಿದ್ದರೆ ಕೇವಲ ವಾಕ್. ಆದ್ದರಿಂದ ವಾಲ್ಮ್ಮೀಕಿ ಮಹರ್ಷಿಗಳು ಕಾವ್ಯರಚಿಸಿಯಾಯಿತು; ಅನಂತರ ಮಾಡಿದ್ದೇನು? 

\begin{shloka}
ಕೃತ್ವಾಽಪಿ ತನ್ಮಹಾಪ್ರಾಜ್ಞಃ ಸಭವಿಷ್ಯಂ ಸಹೋತ್ತರಮ್|\label{238}\\
ಚಿಂತಯಾಮಾಸ ಕೋಽನ್ವೇತತ್ಪ್ರಯುಂಜೀಯಾದಿತಿ ಪ್ರಭುಃ||\\
ತಸ್ಯ ಚಿಂತಯಮಾನಸ್ಯ ಮಹರ್ಷೇರ್ಭಾವಿತಾತ್ಮನಃ|\\
ಅಗೃಹ್ಣೀತಾಂ ತತಃಪಾದೌ ಮುನಿವೇಷೌ ಕುಶೀಲವೌ||\\
ಕುಶೀಲವೌ ತು ಧರ್ಮಜ್ಞೌ ರಾಜಪುತ್ರೌ ಯಶಸ್ವಿನೌ|\\
ಭ್ರಾತರೌ ಸ್ವರಸಂಪನ್ನೌ ದದರ್ಶಾಶ್ರಮವಾಸಿನೌ||\\
ಸತು ಮೇಧಾವಿನೌ ದೃಷ್ಟಾ ವೇದೇಷು ಪರಿನಿಷ್ಠಿತೌ|\\
ವೇದೋಪಬೃಂಹಣಾರ್ಥಾಯ ತಾವಗ್ರಾಹಯತ ಪ್ರಭಃ||
\end{shloka}

ವಿಷಯವನ್ನು ಪ್ರಚಾರ ಪಡಿಸುವುದು ಗುರುತರ ಜವಾಬ್ದಾರಿ. ವಿಚಾರವನ್ನು ಹೃದಯದೊಡನೆ ಹೇಳುವವರಿಲ್ಲವಲ್ಲ! ತಂದೆ ಹುಡುಗನ ಕೈಯಲ್ಲಿ ಪಕ್ಕದ ಮನೆಯವರಿಗೆ `ಪೆರಮಾಳಂಶೇಕ್ ಇಂಗೇ ಏಳಣಂ' ಎಂದು ಕಳುಹಿಸಿದ. ಹುಡುಗನ ಗಂಟಲು ಕಟ್ಟಿತ್ತು. ಅವನು ಅಲ್ಲಿ ಹೋಗಿ ವಿಕಾರವಾದ ಧ್ವನಿಯಲ್ಲಿ `ತಾತ; ಪ್ರಕೃತಿಯಲ್ಲಿ ಪ್ರತಿಬಂಧಕವಿದ್ದು ಅದು ವ್ಯಂಗ್ಯವಾಗಿತ್ತು. ಸಹಜತೆಗೆ ವಿರೋಧವಾಗಿ ಬಂದರೆ ಅದು ವ್ಯಂಗ್ಯ ತಾನೇ? ಗಾಂರ್ಭಿರ್ಯವಿದ್ದರೆ ಅದು ಭಾವದೊಡನೆ ಬರಬೇಕು, ವ್ಯಂಗ್ಯವಾಗಬಾರದು.

\section*{ಜ್ಞಾನಿಗಳ ಹೃದಯದ ಅರಿವಾದಾಗ ಅದೇ ಒಂದು ಧ್ವನಿಯನ್ನು  ಹೊರಡಿಸುತ್ತದೆ}

ನಮ್ಮಾಳ್ಮಾರರ ಕೃತಿಯಲ್ಲಿ ಹಾಗೆಯೇ. (ಈ ಸಂದರ್ಭದಲ್ಲಿ ನಮ್ಮಾಳ್ವಾರರ ಹಾಡನ್ನು ಸಾಂಪ್ರದಾಯಿಕವಾಗಿ ಹಾಡಿ ಎಂದು ಸಭೆಯಲ್ಲಿ ಶಿಷ್ಯರೊಬ್ಬರನ್ನು ಕೇಳಿದರು. ಅವರು ಆ ಧಾಟಿ ತಮಗೆ ಬರುವುದಿಲ್ಲವೆಂದು ಹೇಳಿದರು. ತಮಗೆ ಬರುವ ಧಾಟಿಯಲ್ಲಿ ಅವರು ಗುರುವಿನ ಅನುಮತಿಯಂತೆ ಹಾಡಿದರು) ಆಳ್ವಾರರ ಕೃತಿಗಳಲ್ಲಿ ಇಶೈ-ಇಯಲ್-ನಾಟಕ ಎಂದು ಮೂರು ರೀತಿಗಳಿವೆ. ಆಳ್ವಾರರ ಪಾಡಲ್ಗಳನ್ನು ಸರಿಯಾಗಿ ಹೇಳಬೇಕಾದರೆ ಇಶೈನಲ್ಲೇ ಹೇಳಬೇಕು. ಅದಾಗದೆ ರಾಗಪರಿಚಯ ವಿಲ್ಲದಿದ್ದರೆ, ಅರ್ಥ ಜ್ಞಾನದೊಡನೆ ಹೇಳುವುದಾದರೆ ಇಯಲ್ ನಲ್ಲಿ ಹೇಳಬಹುದು. ಇಯಲ್ ನಲ್ಲಿ ಹೇಳಿದರೂ ಪುರುಳ್ ವ್ಯತ್ಯಾಸವಾಗಬಾರದು ಎಂದು ಕೆಲವರು ಹೇಳುವುದುಂಟು. ಆದರೆ ಅರ್ಥಜ್ಞಾನ ಎದ್ದು ತೋರಬೇಕಾದರೆ ಅದು ಇಶೈನಲ್ಲೇ ಬರಬೇಕು. ಅದು ವ್ಯತ್ಯಾಸವಾದರೆ ಆಳ್ವಾರರ ಹೃದಯ ಬರುವುದಿಲ್ಲ. ಭಾಷಾ ಜ್ಞಾನವಿದ್ದರೆ  ಇಯಲ್ ನಲ್ಲಿ ಹೇಳಬಹುದಲ್ಲ ಎಂದರೆ, ಸರಿಯಾಗಿ ಅರ್ಥಜ್ಞಾನವಿದ್ದರೆ ಆಳ್ವಾರರ ಕೃತಿ ಯಾವ ಅರ್ಥದಿಂದ ಹೊರಟಿತೋ ಆ ಅರ್ಥ ತಿಳಿದಿದ್ದರೆ, ಆ ಅರ್ಥದಿಂದಲೇ ಒಂದು ಧ್ವನಿ ಹೊರಡುತ್ತದೆ. ಆ ಧ್ವನಿಯೇ ಅರ್ಥದ ಕಡೆಗೆ ಒಯ್ಯಬಹುದು. ಅಂತಹ ಧ್ವನಿ ಇಶೈಯೂ ಆಗಿರುತ್ತದೆ. ಇಯಲ್ಲೂ ಆಗಿರುತ್ತದೆ. ಆ ಧ್ವನಿ ಹೊರಬಂದಾಗ ಅದಕ್ಕೆ ಬೇಕಾದ ಅಭಿನಯವನ್ನೂ ಆ ಧ್ವನಿಯೇ ಉಂಟು ಮಾಡುತ್ತದೆ. ಆದ್ದರಿಂದ ಆಳ್ವಾರರ ಹೃದಯದ ಆಳವನ್ನು ಮುಟ್ಟಿಬಂದ ಪಾಡಲ್ ಮಾತ್ರ ಪಾಡಲ್ಲಾಗಬಹುದು. ಅದೇ ಹೃದಯವಿಲ್ಲದಿದ್ದರೆ, ರಾಗಜ್ಞಾನದಿಂದ ಎಷ್ಟೇ  ಹಾಡಿದರೂ ಪ್ರಯೋಜನವಿಲ್ಲ. (ಭಾವ ಸಹಿತವಾದ ಗೀತದ ಉದಾಹರಣೆಗೆ `ಕ್ಷೀರ ಸಾಗರ ಶಯನ ' ಎನ್ನುವ ಕೀರ್ತನೆಯನ್ನು ಹಾಡಿ ತೋರಿಸಿದರು. ಇಲ್ಲಿ ಕ್ಷೀರಸಾಗರ ಎನ್ನುವಾಗ ಸಾಗರದ ಗತಿ-ಅದರ ಪ್ರಶಾಂತತೆ-ಅದರ ಗಾಂಭಿರ್ಯ ಇವುಗಳ ಒಂದು ಚಿತ್ರ, ಹಾಗೂ ಶಯನ ಎನ್ನುವಾಗ ಶಯನಕ್ಕನುಕೂಲವಾದ ಉಚ್ಚಾರಣೆ ಇವು ಹೇಗಿರಬೇಕು ಎನ್ನುವುದು ವಿವರಿಸಿ ತಿಳಿಯಪಡಿಸಿದರು.)

\section*{ರಾಮಾಯಣವನ್ನು ಗಾನಮಾಡಲು ಕುಶಲವರಿಗಿದ್ದ ಯೋಗ್ಯತೆ} 

ಹೀಗೆ ಮಹರ್ಷಿ ಹೃದಯದಿಂದ ಬಂದ ಕಾವ್ಯವಾಗಿದೆ ರಾಮಾಯಣ. ಆದರೆ ಇಂದು ಅವರ ಹೃದಯದ ಭಾವ ಹಾರಿಸಹೋಗಿದೆ. ಉದಾಹರಣೆಗೆ ಒಬ್ಬರ ಮನೆಯಲ್ಲಿ ಮೆಣಸಿನಕಾಯಿ ಇರುತ್ತೆ. ಅದು ಖಾರವಾಗಿದ್ದು ಅದರ ಖಾರವನ್ನು ಅನುಭವಿಸಿದವರು `ಖಾರ ಖಾರ' ಎನ್ನುತ್ತಿದ್ದರು. ಆ ಮೆಣಸಿನಕಾಯಿ ಒಂದು ನೂರು ವರ್ಷದ ಮೇಲೆ ತನ್ನ ಖಾರವನ್ನು ಕಳೆದುಕೊಂಡುಬಿಟ್ಟಿತು. ಅನಂತರ ಆ ಮನೆಯ ಮೊಮ್ಮಕ್ಕಳು ಅದನ್ನು `ಖಾರ ಖಾರ' ಎನ್ನಬೇಕು ಎನ್ನ್ವ ಪರಂಪರೆಯಿಂದ ಖಾರ ಎನ್ನುತ್ತಿದ್ದರು. ಅವರಿಗೆ ಖಾರದ ಪರಿಚಯವಿರಲಿಲ್ಲ. ಆದರೂ ಕೇವಲ ಪರಂಪರೆಯಿಂದ (ಪರಂಪರೆಯಾಗಿ ಹೇಳಿಕೊಳ್ಳುತ್ತ ಬಂದಿದ್ದುದರಿಂದ) ಖಾರ ಎಂದು ಹೇಳುತ್ತಿದ್ದರು. ಇಲ್ಲಿ ಆ ಮೊಮ್ಮಕ್ಕಳ ಬಾಯಿಂದ ಬರುವ ಖಾರ ಹೇಗೆ ಹೃದಯ ಹೀನವೋ (ಅರ್ಥ ಶೂನ್ಯವೋ) ಹಾಗೆ ಕಾವ್ಯವೂ ಆಗುತ್ತೆ. ವಾಲ್ಮೀಕಿಗಳು ಕಾವ್ಯ ಮಾಡುದರು (ರಚಿಸಿದರು). ಇಂತಹ ಕಾವ್ಯವನ್ನು ತನ್ನ ಹೃದಯದೊಡನೆ ಲೋಕದಲ್ಲಿಡಲು ಯಾರು? ಹೃದಯ ಬಂತು; ಅನುಭವವೂ ಬಂತು; ಅದಕ್ಕನುಗುಣವಾದ ಧ್ವನಿಯೂ ಕಾವ್ಯರೂಪದಲ್ಲಿ ಬಂತು; ಆದರೂ ಅದರ ರೂಪ ರೇಖೆ ಕೆಡದಂತೆ ಹೊರಗಿಡುವವರು ಯಾರು? ಇದೇ ಮಹತ್ತಾದ ಚಿಂತೆ. ಹೀಗೆ  ಚಿಂತಿಸುವ ವಾಲ್ಮ್ಮೀಕಿಗಳಿಗೆ ಸರಿಯಾದ ಇಬ್ಬರು ಶಿಷ್ಯರು ದೊರಕಿದರು. ಅವರೆಂತಹವರು-

\begin{shloka}
ತಸ್ಯ ಚಿಂತಯಮಾನಸ್ಯ ಮಹರ್ಷೇರ್ಭಾವಿತಾತ್ಮನಃ|\\
ಅಗೃಹ್ಣೀತಾಂ ತತಃ ಪಾದೌ ಮುನಿವೇಷೌ ಕುಶೀಲವೌ||\\
ಕುಶೀಲವೌ ತು ಧರ್ಮಜ್ಞೌ ರಾಜಪುತ್ರೌ ಯಶಸ್ವಿನೌ|\label{240}\\
ಭ್ರೌತರೌ ಸ್ವರಸಂಪನ್ನೌ ದದರ್ಶಾಶ್ರಮವಾಸಿನೌ||\\
ಸ ತು ಮೇಧಾವಿನೌ ದೃಷ್ವಾ ವೇದೇಷು ಪರಿನಿಷ್ಠಿತೌ|\\
ವೇದೋಪಬೃಂಹಣಾರ್ಥಾಯ ತಾವಗ್ರಾಹಯತ ಪ್ರಭುಃ||
\end{shloka}

ಇಲ್ಲಿ ಬಳಸಿರುವ ವಿಶೇಷಣಗಳು ಆತಿ ರಮ್ಯ ಹಾಗೂ ಅರ್ಥವತ್ತಾಗಿವೆ. ಅವರು ಮುನಿವೇಷವನ್ನು ಧರಿಸಿದವರು. ಧರ್ಮವನ್ನು ತಿಳಿದವರು, ಹಾಗೆಯೇ ರಾಜನ ಕಥೆಯನ್ನು ವಿಸ್ತರಿಸಲು ತಕ್ಕಂತೆ ರಾಜಪುತ್ರರೂ ಆಗಿದ್ದಾರೆ. ಹಾಗೆಯೇ ಯಶಸ್ವಿಗಳೂ ಆಗಿದ್ದಾರೆ. `ಎತ್ತು ಏರಿಗೆಳೆಯಿತು ಎಂದರೆ ಕೋಣ ನೀರಿಗೆಳೆಯಿತು' ಎಂಬಂತೆ ಇಬ್ಬರೂ ಬೇರೆಯೇ? ಎಂದರೆ, ಅಲ್ಲ; ಇಬ್ಬರೂ ಸಹೋದರರೇ. ಇಷ್ಟರ ಮೇಲೆ ಅಶ್ರಮವಾಸಿಗಳು. ಹಾಗೆಯೇ ಧಾರಣಾಶಕ್ತಿಸಂಪನ್ನರು, ವೇದಗಳಲ್ಲಿ ಪರಿನಿಷ್ಠಿತರು. ಇಲ್ಲಿ ಮುನಿವೇಷೌ- ಆಶ್ರಮವಾನಿನೌ-ಧರ್ಮಜ್ಞೌ-ವೇದೇಷು ಪರಿನಿಷ್ಠತೌ ಎನ್ನುವ ವಿಶೇಷಣಗಳು ಅವರು ಮಹರ್ಷಿಹೃದಯವನ್ನು ರೂಪುರೇಖೆ ಕೆಡದಂತೆ ಗ್ರಹಿಸಿ ರಕ್ಷಿಸಬಲ್ಲವರು ಎನ್ನುವುದನ್ನು ಚೆನ್ನಾಗಿ ಸಮರ್ಥಿಸುತ್ತವೆ. ಹಾಗೆಯೇ ಭ್ರಾತರೌ-ಸ್ವರಸಂಪನ್ನೌ- ಮೇಧಾವಿನೌ- ರಾಜಪುತ್ರೌ ಎನ್ನುವ ವಿಶೇಷಣಗಳು, ಎಲ್ಲವಿಧಗಳಲ್ಲಿಯೂ ಅವರಿಬ್ಬರಲ್ಲಿರುವ ಆನುರೂಪ್ಯ, ಹಾಗೆಯೇ ಗಾನಕ್ಕೆ ಬೇಕಾದ ಸ್ವರಸಂಪನ್ನತೆ, ಅದರ ಜೊತೆಗೆ ಧಾರಣಾ ಶಕ್ತಿ, ಇಷ್ಟಲ್ಲದೆ ರಾಜನ ಕಥೆಯನ್ನು ಸರಿಯಾಗಿ ಗ್ರಹಿಸಲು ರಾಜಪುತ್ರತ್ವ ಬೇರೆ ಇರುವುದನ್ನು ಸಮರ್ಥಿಸಿ ಹೇಳುತ್ತವೆ. ಇಂತಹ ಶಿಷ್ಯರನ್ನು ಕಂಡು ವಾಲ್ಮ್ಮೀಕಿಗಳು ಅತ್ಯಂತ ತುಷ್ಟರಾದರು. ಇದು ರಾಮಾಯಣವನ್ನು  ವಾಲ್ಮ್ಮೀಕಿಗಳು ಎಲ್ಲಿಟ್ಟರು ಎನ್ನುವುದನ್ನು ಹೇಗೆ ಸೂಚಿಸುತ್ತದೆಯೋ ಅಂತೆಯೇ ರಾಮಾಯಣವನ್ನು ಗ್ರಹಿಸಲು ಎಂತಹ ಭೂಮಿಕೆ ಬೇಕು ಎನ್ನುವುದನ್ನೂ ತಿಳಿಸುತ್ತದೆ. ಅಂತಹ ಹೃದಯ ಇಲ್ಲದೆ ಅಂತಹ ಕಾವ್ಯ ಹೇಗೆ ಸೊಗಸೀತು. 

\section*{ರಾಗವು ಭಾವಕ್ಕೆ ಪೋಷಕವಾಗವಂತಿರಬೇಕು}

ಹೀಗೆಯೇ ಆಳ್ವಾರರ ಪ್ರಬಂಧವೂ. ಹೇಳುವಾGಅ ವ್ಯಂಗ್ಯ ಬರಬಾರದು. ಅದರ ಸಹಜತೆಗೆ ಕುಂದು ಬಂದರೆ ಹೇಗೆ ಹೇಳಿದರೂ ಅದು ವ್ಯಂಗ್ಯವೇ!

ತಿರುವಾಯ್ಮೊಳಿ ಎಂದರೆ ಶ್ರೇಷ್ಠವಾದ ಬಾಯಿಂದ ಬಂದ ಭಾಷೆ ಎಂದರ್ಥ. ಅದಕ್ಕೆ ಉಚಿತವಾದ ರೀತಿಯಲ್ಲಿದ್ದರೆ ಅದು ತಿರುವಾಯ್ಮೊಳಿ. ನಾವು ಹೇಳಿದುದೆಂದು ಬೇಡ. ಆಳ್ವಾರರೇ ಹೇಳಿದ್ದಾರೆ ತಮ್ಮ ಹಾಡಿನಲ್ಲಿ, ಹೇಗೆ ಅದರ ಗಾನ ಎಂಬುದನ್ನು. `ಪೊಲಿಕ ಪೊಲಿಕ-ಆಡಿ ಪಾಡಿ ಯುಳಿದರಕ್ಕಂಡೋಂ' ಎಂಬಲ್ಲಿ ಇದನ್ನು ಸಂಪ್ರದಾಯದ ರೀತಿ ಹೇಳಿದರೆ ಕೆಟ್ಟು ಹೋಗುತ್ತದೆ. ಅವರು ಅನುಭವದೊಡನೆ ಆಡಿ ಪಾಡಿ ಆನಂದದಿಂದ ಹೇಳಿದ್ದಾರೆ. ಆದ್ದರಿಂದ ಇದನ್ನು ಆನಂದದಿಂದಲೇ ಹೇಳಬೇಕು. `ಯಾಕೋ ನನಗೆ ಆನಂದ ಉಂಟಾಗುತ್ತದೆಯಲ್ಲಾ' ಎಂಬ ಮಾತನ್ನು ದುಃಖದ ಧ್ವನಿಯಲ್ಲಿ (ಹೇಳಿ ತೋರಿಸಿದರು) ಹೀಗೆ ಹೇಳಿದರೆ ಭಾವ ಹೋಯಿತು (`ಉನ್ನದು ಎನ್ನದಾವಿಯುಂ' ಎಂಬ ಹಾಡನ್ನು ಹಾಡಿ ತೋರಿಸಿದರು) ವಾಲ್ಮೀಕಿಗಳ ಹೃದಯಕ್ಕನುಸಾರವಾಗಿ ರಾಮಾಯಣದ ಗಾನ ಹೇಗಿರಬೇಕು? ರಾಗಬದ್ಧವಾಗಿರಬೇಕು, ತಾಳಬದ್ಧವಾಗಿರಬೇಕು. ರಾಗದಲ್ಲಿ ಹೇಳಿಬಿಡುವುದು ಎಂದರೆ ಭಾವ ಹಾರಿಹೋದರೆ ಎಂತಹ ಹಾಡಬೇಕು? ಈಗ ಮಾ ನಿಷಾದ ಎನ್ನುವುದನ್ನು ತೆಗೆದುಕೊಳ್ಳೋಣ. ಎರಡು ಹಕ್ಕಿಗಳು ತಮ್ಮಷ್ಟಕ್ಕೆ ತಾವು ಆನಂದವಾಗಿವೆ. ಪ್ರಶಾಂತವಾದ ತಮಸಾನದಿಯ ತೀರದಲ್ಲಿ ಆನಂದವಾಗಿ ವೃಕ್ಷದ ಮೇಲೆ ವಿಹರಿಸುತ್ತಿರುವ ಆ ಪಕ್ಷಿಗಳಲ್ಲಿ ಗಂಡು ಪಕ್ಷಿಯನ್ನು ವ್ಯಾಧ ತನ್ನ ಬಾಣದಿಂದ ಹೊಡೆದುಕೊಂದ. ಇದನ್ನು ನೋಡಿದರು ವಾಲ್ಮೀಕಿಗಳು. ಅವರ ಮನಸ್ಸಿನಲ್ಲಿ ಕರುಣೆ ಉಕ್ಕಿ ಬಂತು, ಅನಂತರ ದುಃಖ. ಇಲ್ಲಿ ದುಃಖವನ್ನು ಸೂಚಿಸಬೇಕಾದರೂ ಋಷಿಯ ಮನಸ್ಸು, ಅವರ ಮನಸ್ಸಿನ ದಯೆ, ದುಃಖ ಇವುಗಳು, ಅದಕ್ಕನುಗುಣವಾಗಿಯೇ ಇರಬೇಕು. ಈಗ ಆ ಭಾವ ಬರುತ್ತದೆಯೇ ಗಮನಿಸಿ (ಎಂದು ಹೇಳಿ ಅಠಾಣರಾಗದ ಅಲಾಪನೆ ಮಾಡಿದರು. ಆಗ ದುಃಖಕ್ಕಿಂತ ಅದು ಉತ್ಸಾಹಕ್ಕೆ ಪೋಷಕವಾಗಿದ್ದುದು ಆಲಾಪನೆಯ ರೀತಿಯಲ್ಲಿಯೇ ಸ್ಪಷ್ಟವಾಗಿ ತಿಳಿದುಬರುತ್ತಿತ್ತು. ರಾಗಜ್ಞಾನವೇ ಇಲ್ಲದ ಉಳಿದವರಿಗೂ ಇದು ಹೊಂದದಿರುವ ರಾಗ ಎನಿಸಿತು. ಎರಡನೆಯ ಬಾರಿ ಸ್ವಲ್ಪ ಬದಲಾಯಿಸಿ ಹೇಳಿದರು. ಆಗಲೂ ಹೊಂದಿಕೊಳ್ಳುತ್ತಿರಲಿಲ್ಲ. ಅನಂತರ ಮೂರನೆಯ ಬಾರಿ ಧನ್ಯಾಸಿ ರಾಗದಲ್ಲಿ ಹೇಳಿದಾಗ ಆಲಾಪನೆಯಲ್ಲಿ ಕೃತಿ ಪೆಟ್ಟೆಗೆಯಲ್ಲಿಟಟ್ಟಂತೆ ಬಹು ಅಚ್ಚುಕಟ್ಟಾಗಿ ಭಾವ ಪೋಷಕವಾಗಿದ್ದಿತು. ವಾಲ್ಮೀಕಿಗಳೇ ತಲೆದೂಗುವಂತಿತ್ತು. ಅನಂತರ `ಪಾದಬದ್ಧೋಽಕ್ಷರಸಮಃ'\label{242} ಎಂಬುದನ್ನು ಕಲ್ಯಾಣೀ ರಾಗದಲ್ಲಿ ಹಾಡಿದರು ಅದು ಬಹು ರಮಣೀಯವಾಗಿತ್ತು.)


\section*{ರಾಮಾಯಣಗಾನ ಮೊದಲು ಋಷಿಸದಸ್ಸಿನಲ್ಲಿಯೇ ಏಕೆ?}

ವಾಲ್ಮೀಕಿಗಳ ಹೃದಯವನ್ನು ಹೊತ್ತು ಬಂದ ಕುಶಲವರು ಕಾವ್ಯವಾಹಿನಿಯನ್ನು ಋಷಿಸದಸ್ಸಿನಲ್ಲಿ ಗಾನರೂಪವಾಗಿ ಹರಿಸಿದರು. ಋಷಿಗಳಾದ ವಾಲ್ಮ್ಮೀಕಿಗಳ ಕಾವ್ಯವನ್ನು ಮೊದಲು ಹಾಡಿದುದು ಋಷಿಗಳ ಸದಸ್ಸಿನಲ್ಲಿ. ಮೊಟ್ಟ ಮೊದಲು ಕಾವ್ಯದ ಪ್ರಯೋಗವನ್ನು ನಡೆಸಿ, ಅದನ್ನು ಒರೆಹಚ್ಚಿದುದು ಋಷಿಸದನದಲ್ಲಿ, ಏಕೆ? ಹಾವಿನ ಹೆಜ್ಜೆ ಹಾವೇ ಬಲ್ಲದು, ಅಂತೆಯೇ ಕಳ್ಳನ ಗೊತ್ತು. ಹಾಗೆಯೇ  ಋಷಿ ಹೃದಯಜ್ಞರಾದವರು ಅಥವಾ ಋಷಿಗಳಾದವರು ತಾನೇ ಋಷಿವಿರಚಿತವಾದ ಕಾವ್ಯದ ಗುಟ್ಟನ್ನು ತಿಳಿಯಬಲ್ಲರು. 

\section*{ಭಾವ-ರಸ-ಸಂಗೀತ ಸಮ್ಮೀಲಿತವಾದ ರಾಮಾಯಣದ ಬಗ್ಗೆ ಋಷಿಸದಸ್ಸಿನ ಪ್ರಶಂಸೆ}

ಅವರ ಅಭಿಪ್ರಾಯವಾದರೂ ಏನು?

\begin{shloka}
ಸಾಧು ಸಾಧ್ವಿತಿ ತಾವೂಚಃ ಪರಂ ವಿಸ್ಮಯಮಾಗತಾಃ|\\
ತೇ ಪ್ರೀತಮನಸಃ ಸರ್ವೇ ಮುನಯೋ ಧರ್ಮವತ್ಸಲಾಃ||\\
ಪ್ರಶಶಂಸುಃ ಪ್ರಶಸ್ತವ್ಯೌ ಗಾಯಮಾನೌ ಕುಶೀಲವೌ|\label{243a}\\
ಅಹೋ ಗೀತಸ್ಯ ಮಾಧುರ್ಯಂ ಶ್ಲೋಕಾನಾಂ ತು ವಿಶೇಷತಃ||\\
ಚಿರನಿರ್ವೃತ್ತಮಪ್ಯೇತತ್ ಪ್ರತ್ಯಕ್ಷಮಿವ ದರ್ಶಿತಮ್|\label{243}\\
\end{shloka}

ಒಂದೊಂದು ವಿಶೇಷಣವೂ ತೂಗಿ-ತೂಕಹಾಕಿ ಹಾಕಲ್ಪಟ್ಟಿದೆ. ಪ್ರೀತ ಮನಸಃ, ಧರ್ಮವತ್ಸಲಾಃ ಧರ್ಮವತ್ಸಲರಾದುದರಿಂದ ಅವರು ಪ್ರೀತಿಮನಸ್ಕರು. ಗೀತವೂ ಮಧುರ; ಕಾವ್ಯವು ಅದಕ್ಕಿಂತಲೂ ಮಧುರ. ಗೀತದ ಮಾಧುರ\char"200D\char"0CCDಯ,  ಕಾವ್ಯದ ಮಾಧುರ್ಯಗಳೆರೆಡೂ ಸೇರಿ, ಮಹರ್ಷಿಗಳ ಹೃದಯವೂ ಸೇರಿದಾಗ ಚಿರನಿರ್ವೃತ್ತವಾದ ಕಾವ್ಯವೂ ಎದುರಿಗೇ ನೋಡುತ್ತಿರುವಂತೆ ಕಾಣುವುದರಲ್ಲಿ ಆಶ್ಚರ್ಯವೇನು? ಹೀಗೆ ರಾಮಾಯಣ್ದ ಗಾನವು ಭಾವ, ರಾಸ, ಸಂಗೀತ ಎಲ್ಲ ದೃಷ್ಟಿಗಳಿಂದಲೂ ಸರಿಯಾಗಿರಬೇಕು. 

\section*{ಕಥೆಯ ಮೂಲಕ ತತ್ತ್ವವನ್ನು ಮುಂದಿಡಲು ರಚಿತವಾಗಿದೆ ರಾಮಾಯಣ}

ರಾಮಾಯಣ ಕೇವಲ ಚಾರಿತ್ರಿಕ ಕಥೆಯಲ್ಲ. ಅದರಲ್ಲಿ ಚಾರಿತ್ರಿಕಾಂಶ ಇಲ್ಲವೇ ಇಲ್ಲ ಎಂದಲ್ಲ. ಅದು ಇರಲಿ ಬಿಡಲಿ, ಕಥೆ ಮುಖ್ಯವಲ್ಲ. ಕಥೆಯನ್ನು ಆಧಾರವಾಗಿಟ್ಟುಕೊಂಡು ತತ್ತ್ವವನ್ನು ಮುಂದಿಡುವುದೇ ಮಹರ್ಷಿಗಳ ಅಭಿಪ್ರಾಯ ಹಾಗಾದರೆ ಕಥೆಯ ಆಧಾರವೇಕೆ? ನೇರವಾಗಿ ಹೇಳಬಹುದಿತ್ತಲ್ಲ ಎಂದರೆ, ನೇರವಾಗಿ ಹೇಳುವುದಾದರೆ ಹೇಗೆ ಹೇಳಬೇಕು? ಅತ್ಮ ಸ್ವಪ್ರಕಾಶನಾಗಿ ದೇದೀಪ್ಯಮಾನವಾಗಿದ್ದಾನೆ ಎಂಬುದನ್ನು ಕೇವಲ ಭೌತಿಕಪ್ರಪಂಚದಲ್ಲಿ ಇರುವವರಿಗೆ ಹೇಗೆ ತಿಳಿಸುವುದು? ಅದಕ್ಕೋಸ್ಕರ ಅದರಲ್ಲಿ ಭೌತಿಕವಾದ ವಿಷಯಗಳ ಬೆರಕೆ. ಬಳೆ ಹೇಗಿತ್ತು? ಎಂದರೆ ಕೋಡುಬಳೆ ಹಾಗಿತ್ತು ಎನ್ನುವುದಿಲ್ಲವೇ? ಆದರೆ ಬಳೆಯನ್ನು ಕೋಡುಬಳೆಯಂತೆ ತಿನ್ನಲಾಗುವುದೇ? ಬಳೆಯ ಪರಿಚಯ ಕೊಡಲು ಇವನಿಗೆ ಪರಿಚಯವಿರುವ ಒಂದು ವಸ್ತುವಿನ ಮೂಲಕ ಸಾಮ್ಯವನ್ನಿಟ್ಟ. ಚಂದ್ರಮುಖಿಯೆಂದರೆ, ಗ್ರಹಣ, ವೃದ್ಧಿ, ಕ್ಷಯ ಈ ಯಾವುದನ್ನೂ ತೆಗೆದುಕೊಳ್ಳಬೇಕಾಗಿಲ್ಲ. ಆಹ್ಲಾದವುಂಟುಮಾಡುವುದನ್ನು ಮಾತ್ರ ತೆಗೆದುಕೊಳ್ಳುತ್ತೇವೆ. ಉಪಮೇಯದ ಧರ್ಮ ಕೆಡದಂತೆ ಉಪಮಾನದ ಸಾದೃಶ್ಯವನ್ನು ತೆಗೆದುಕೊಳ್ಳಬೇಕು. 

\section*{ಜ್ಞಾನಿಗಳ ದೃಷ್ಟಿಯನ್ನು ನಮ್ಮ ದೃಷ್ಟಿಯೊಂದಿಗೆ ಸೇರಿಸಿಕೊಂಡಾಗ ಅನುಭವಲಾಭ}

ರಾಮಾಯನ ಕೇವಲ ಲೌಕಿಕ ಕಥೆಯೇ ಆದರೆ, ಆದರಿಂದ ನಮಹಗೆ ಏನು ಪ್ರಯೋಜನ? `ಅವಳೇನು? ಮಂಥರೆ' `ಅವನೊಬ್ಬ ಮಾರೀಚ ಕಣಯ್ಯ' `ಒಳ್ಳೆ  ಕುಂಭಕರ್ಣ.' ಹೀಗೆ ರಾಮಾಯಣದ ಮಂಥರೆ, ಕುಂಭಕರ್ಣ, ಮಾರೀಚರು ಲೋಕದಲ್ಲೂ ಉಂಟು. ರಾಮನ ಹೆಂಡತಿಯೊಬ್ಬಳನ್ನು ರಾವಣ ಹೊತ್ತುಕೊಂಡು ಹೋದನು. ಕಾಶ್ಮೀರದ ಗಡಿಯಲ್ಲಿ ನೂರಾರು ಸ್ತ್ರೀಯರನ್ನು ಪಾಕಿಸ್ತಾನದವರು ಹೊತ್ತುಕೊಂಡು ಹೋದರು. ಆದ್ದರಿಂದ ಅನೇಕ ರಾಮಾಯಣಗಳಿಗೆ ಅವಕಾಶವಿದೆ ಎಂದರೆ, ನಮ್ಮಲ್ಲಿ ನಡೆಯುತ್ತಿರುವ ಘಟನೆಗೆ ಫಲ ದಃಖವೇ ಆದರೆ ರಾಮಾಯಣದ ಫಲಶ್ರುತಿಯಲ್ಲಿ ಹೇಳಿರುವುದೇನು?

\begin{shloka}
ಯಃಕರ್ಣಾಂಜಲಿಸಂಪುಟೈರಸರಹಃ ಸಮ್ಯಕ್ ಪಬತ್ಯಾದರಾತ್\label{244a}\\
ವಾಲ್ಮೀಕೇರವ್ದನಾರವಿಂದಗಲಿತಂ ರಾಮಾಯಣಾಖ್ಯಂ ಮಧು|\\
ಜನ್ಮವ್ಯಾಧಿ ಜರಾವಿಪತ್ತಿಮರಣೈಃ ಅತ್ಯಂತಸೋಪದ್ರವಂ\\
ಸಂಸಾರಂ ಸ ವಿಹಾಯ ಗಚ್ಛತಿ ಪುಮಾನ್ ವಿಷ್ಣೋಃಪದಂ ಶಾಶ್ವತಮ್||
\end{shloka}

ರಾಮಾಯಣವನ್ನು ಓದಿದರೆ ವಿಷ್ಣುವಿನ ಶಾಶ್ವತವಾದ ಪದಲಾಭ. ಆದ್ದರಿಂದ ಯೋಗಭೂಮಿಯೇ ರಾಮಾಯಣದ ಜನ್ಮಭೂಮಿ. ಆದರೆ ಇಂದು ಎಲ್ಲರೂ ಸಂಸಾರದಿಂದ ಬಿಡುಗಡೆ ಹೊಂದಿ ವಿಷ್ಣುವಿನ ಪದವನ್ನು ಹೊಂದುತ್ತಿದ್ದಾರೆಯೇ? ಎಂದರೆ, ಸರಿ, ಎಳನೀರು ಮಧುರ, ಬಹುದಾಹಶಾಮಕ ಎಂದು (ಕುಡಿಯಲು) ಕೊಟ್ಟರೆ ಅದರ ಮಟ್ಟೆಯನ್ನು (ಹೊರ ಭಾಗವನ್ನು) ಕಚ್ಚಿದರೆ ಮಾಧುರ್ಯ ಸಿಕ್ಕುತ್ತದೆಯೇ? ಹನುಮನು ಬೆಂಕಿಯಿಂದ ಲಂಕೆಯನ್ನು ಸುಟ್ಟರು ಪುಸ್ತಕ ಸುಡುವುದಿಲ್ಲ. ಅದು ಹೇಗೆ ಹೊರಟಿತೋ ಹಾಗೆ ತೆಗೆದುಕೊಂಡಾಗ ತಾನೇ ಅದರ ಆನಂದ. ಹಾಗೆ ತೆಗೆದು ಕೊಂಡಾಗ ಆದು ಜೀವನದ ಆದರ್ಶ. ಆ ಕೃತಿ ಇಂದು ಉಸಿರಿಲ್ಲದೆ ನಿರ್ಜೀವ ಕೃತಿಯಾಗಿದೆ. ಅದಕ್ಕೆ ನಮ್ಮ ಉಸಿರನ್ನು ಕೂಡಿಸಿ ಚೈತನ್ಯ ತುಂಬಿ ಅನುಭವಿಸಿ ನೋಡಬೇಕು. ಆಳ್ವಾರರ ಲಕ್ಷ್ಯವೆಲ್ಲಿತ್ತು? 

\begin{shloka}
ಅಂತರ್ಲಕ್ಷ್ಯಂ ಬಹಿರ್ದೃಷ್ಟಿಃ ನಿಮೇಷೋನ್ಮೇಷವರ್ಜಿತಾ |\label{244}\\
ಏಷಾ ಸಾ‌ ವೈಷ್ಣವೀ (ಶಾಂಭವೀ) ಮುದ್ರಾ ವೇದಶಾಸ್ತ್ರೇಷು ಗೋಪಿತಾ||
\end{shloka}

ಇಂತಹ ಅಕ್ಷ್ಯದಿಂದ ಹೊರಟ ಪ್ರಬಂಧವನ್ನು ತಿಳಿಯಬೇಕಾದರೆ ಅಂತರ್ದೃಷ್ಟಿ  ಬೇಕು. ಆ ದೃಷ್ಟಿಯೊಡನೆ ನಮ್ಮ ದೃಷ್ಟಿಯನ್ನು ಬೆಸೆದುಕೊಳ್ಳಬೇಕು. ವಾಲ್ಮೀಕಿಗಳ ರಾಮಾಯಣವನ್ನು ಅನುಭವಿಸಬೇಕಾದರೆ ವಾಲ್ಮೀಕಿಗಳ ದೃಷ್ಟಿಯನ್ನು ನಮ್ಮ ದೃಷ್ಟಿಯಲ್ಲಿ ಬೆಸೆದುಕೊಳ್ಳಬೇಕು. ಅಂದು ತಿಳಿಯುತ್ತದೆ ಫಲಶ್ರುತಿಯ ವಾಸ್ತವಿಕತೆ.

\begin{center}
ಇತಿ ಶಮ್ 
\end{center}
