\chapter{ಶ್ರೀಮದ್ರಾಮಾಯಣ-ಪಾರಾಯಣ} 

(ದಿನಾಂಕ ೨೭-೭-೧೯೫೮ ರಂದು ಪ್ರಾತಃಕಾಲ ರಾಮಾಯಣದ ಬಗೆಗೆ ಶಿಷ್ಯರೊಬ್ಬರ ಪ್ರಶ್ನೆಗೆ ಉತ್ತರವಾಗಿ ಶ್ರೀರಂಗಮಹಾಗುರುವು ಆಡಿದ ಮಾತುಗಳು. ಬರಹ ರೂಪ ಶ್ರೀ ಎಚ್‍. ಎಸ್‍. ವರದದೇಶಿಕಾಚಾರ್ಯ ರಂಗಪ್ರಿಯರವರಿಂದ) 

{\bf ಇಂದು ರಾಮಾಯಣ ಪಾರಾಯಣ ಮಾಡುವವರಲ್ಲಿ ಕಂಡುಬರುವ ವಿವಿಧವಾದ ಹಿನ್ನೆಲೆ} 


ಯಾವ ರೀತಿ ರಾಮಾಯಣಪಾರಾಯಣ ಮಾಡಬೇಕು? ಎಂದು ಪ್ರಶ್ನೆ ಬಂದಿದೆ. ಲೋಕದಲ್ಲಿ ನೋಡಿದರೆ ಎಲ್ಲರೂ ರಾಮಾಯಣಪಾರಯಣ ಮಾಡುತ್ತಾರೆ. ಒಂದು ರಾಮಾಯಣದ ಪುಸ್ತಕವಿದ್ದರೆ, ಅದನ್ನು ಅಕ್ಷರಜ್ಞಾನ, ಭಾಷಾಜ್ಞಾನ, ಕವಿಹೃದಯ, ಉಪದೇಶದ ಮರ್ಮ, ಸಂಪ್ರದಾಯದ ಶಿಕ್ಷಣ ಇವೆಲ್ಲದರಿಂದಲೂ ಗಮನಿಸಿ ಓದುವ ಕ್ರಮವೂ ಇದೆ. ಕೇವಲ ವಾಡಿಕೆಯಿಂದ ಪಾರಾಯಣ ಮಾಡಿ ಅಷ್ಟರಿಂದಲೇ ತೃಪ್ತಿಪಡುವುದೂ ಉಂಟು. ಪಾಮರರಾಗಿರುವವರೂ ರಾಮಾಯಣ ಪಾರಾಯಣದಲ್ಲಿ ತೊಡಗಿದ್ದು, ಆ ಮಧ್ಯೆ ಲೋಕದಲ್ಲಿ ಏನಾದರೂ ಅನುಕೂಲ ಒದಗಿದರೆ ಅದು ರಾಮಾಯಣ ಪಾರಾಯಣದ ಮಹಿಮೆ ಎಂದೂ ಹೇಳುವುದುಂಟು. ಇವನಿಗೆ ಉಂಟಾದ ಅನುಕೂಲ ರಾಮಾಯಣದಿಂದ ಉಂಟಾದದ್ದೇ? ಎನ್ನುವುದನ್ನು ನೋಡಬೇಕು. 


ಇದಾವುದೂ ಇಲ್ಲದೆಯೇ ಪಾಶ್ಚಾತ್ಯದೇಶದಲ್ಲಿ ವೈನ್‍ ಮಾರಾಟ ಮಾಡುವವನೂ ಬೇಕಾದಷ್ಟು ಅನುಕೂಲ ಪಡೆದಿರುವುದುಂಟು. ಅವನು ರಾಮಾಯಣ ಪಾರಾಯಣದ ಬದಲು ವೈನ್‍ ಮಾರಾಟ ಮಾಡಿದರೆ ಹೆಚ್ಚು ಹಣ ಎಂದು ಹೇಳಬಹುದು. ವೈನ್‍ ವ್ಯಾಪಾರಿಗಾದರೆ ಈ ಜಾಗದಲ್ಲಿ ಎರಡು ಕಾಸು ಮೂರು ಕಾಸು ಹೀಗೆ ಲಾಭಗಳು ಒದಗಿ ಕೊನೆಗೆ ಇಷ್ಟು ಹಣ ಬರುತ್ತೆ ಎಂದು ಹಣದ ಅಂದಾಜಾದರೂ ಇರುತ್ತೆ. ಇವನಿಗೆ ಆ ನಿಬಂಧನೆಯೂ ಇಲ್ಲ. 


{\bf ರಾಮಾಯಣ ಪಾರಾಯಣದ ಪ್ರಯೋಜನ ಕುರಿತು ವಿಮರ್ಶೆ} 


ನಾವು ಯಾವ ಧ್ಯೇಯವನ್ನಿಟ್ಟುಕೊಂಡು ರಾಮಾಯಣದ ಪಾರಾಯಣ ಮಾಡಬೇಕು? ಎಂದು ವಿಚಾರಮಾಡಿದರೆ ಅದರ ಬಗೆಗೆ ಇಷ್ಟು ಪ್ರಶ್ನೆಗಳು ಏಳುತ್ತವೆ. ನಮಗೆ ಬೇಕಾದುದೇನು? ಆ ಪ್ರಯೋಜನವನ್ನು ರಾಮಾಯಣವು ಕೊಡುತ್ತದೆಯೇ? ರಾಮಾಯಣದಿಂದ ಎಷ್ಟು ತರಹದ ಪ್ರಯೋಜನವನ್ನು ಪಡೆಯಬಹುದು? ಒಬ್ಬನ ಬಳಿಗೆ ಹೋಗಿ ನಮಗೆ ಬೇಕಾದ ಕಾಫಿಯೋ, ನಶ್ಯವೋ, ರಂಜಕದ ಕಡ್ಡಿಯೋ ನಿಮ್ಮಲ್ಲಿ ಇದೆಯೇ? ಎಂದು ಕೇಳುತ್ತೇವೆ. ಅವನು `ಅದು ಇಲ್ಲ' ಎಂದು ಹೇಳಿದರೆ, ಹಾಗಾದರೆ ನಿನ್ನಲ್ಲಿ ಏನು ಎದೆ? ಎಂದು ಕೇಳುತ್ತೇವೆ. ಆಗ ಅವನು `ನನ್ನಲ್ಲಿ ತಿರುವಡಿ ಜೋಡು, ನಾಮದ ಪೆಟ್ಟಿಗೆ, ಶಂಖ ಚಕ್ರಗಳು ಇವೆ' ಎನ್ನುತ್ತಾನೆ. ಅವುಗಳಿಂದ ಆಗುವ ಪ್ರಯೋಜನ ನಮಗೂ ಏನಾದರೂ ಇದ್ದರೆ, ಆಗ ನಾವು ಅದಕ್ಕಾಗಿ ಆಸೆ ಪಡುತ್ತೇವೆ. ರಾಮಾಯಣ ಬರೆದವರು ಎಷ್ಟು ವಿಧವಾದ ಪ್ರಯೋಜನಕ್ಕಾಗಿ 

ಅದನ್ನು ರಚಿಸಿದರು? ಮಕ್ಕಳನ್ನು ಕೊಡುವುದು ಒಂದೇ ರಾಮಾಯಣ ಪಾರಾಯಣಕ್ಕೆ ಪ್ರಯೋಜನವೇ? ಈ ಮೇಲೆ ಹೇಳಿದ ಪ್ರಯೋಜನಗಳೇ ಅಲ್ಲದೆ, ಇನ್ನೂ ಬೇರೆ ತರಹ ಪ್ರಯೋಜನ ಪಡೆಯುವವರೂ ಇದ್ದಾರೆ. ಭಾರತ ದೇಶದಲ್ಲಿ ಸುಮ್ಮನೆ ರಾಮನ ಕಥೆಯನ್ನು ಹೇಳುವವರೂ ಕೇಳುವವರೂ ಅನೇಕರು ಇರುವುದರಿಂದ ರಾಮಾಯಣ ಬರದಿದ್ದರೂ ಕೆಲವರು ಜನಾರ್ದನನ ಕಟ್ಟೆ (ಹರಟೆ ಚಾವಡಿ)ಯಲ್ಲಿ ಕುಳಿತು, ರಾಮ ಹಾಗಿದ್ದ, ಆಂಜನೇಯ ಹೀಗೆ ಮಾಡಿದ ಎಂದು ಹೇಳುತ್ತಿದ್ದರೂ, ಅವರಿಗೆ ಶಾಕಾಯಲವಣಾಯ ಗಿಟ್ಟುತ್ತಿರುತ್ತದೆ. ಏನೋ, ರಾಮ ಮೂರು ಕಾಸು ಕೊಡುತ್ತಾನೆ ಎಂದು ಅವರು ಹೇಳುತ್ತಾರೆ. ಅಚ್ಚೇರು ಅವಲಕ್ಕಿ, ಒಂದೂವರೆ ಸೇರು ಪಡಿ ಅಕ್ಕಿ, ಒಂದು ಬೇಲದಹಣ್ಣು, ಎರಡು ಚೂರು ಅಡಕೆ, ಒಂದು ಸೇರು ರಾಗಿ ಇದೆಲ್ಲ ಫಲ ಸಿಗುತ್ತದೆ ಎಂದು ರಾಮಾಯಣದಲ್ಲಿ ಹೇಳಿದೆಯೇ? ಅದೇ ತುಳಸೀದಾಸ ರಾಮಾಯಣವನ್ನು ಹೇಳುವವನಿಗೆ ರಾಗಿ ಜೋಳ ಯಾವುದೂ ಸಿಕ್ಕುವುದಿಲ್ಲ. ನಮ್ಮ ದಕ್ಷಿಣ ದೇಶದಲ್ಲಿ ರಾಮಾಯಣ ಪಾರಾಯಣ ಮಾಡುವವನಿಗೆ-ರಾಮಾಯಣ ಹೇಳುವವನಿಗೆ (ವಾಲ್ಮೀಕಿ ರಾಮಾಯಣ) ಗೋಧಿ ಸಿಕ್ಕುವುದಿಲ್ಲ. 


ರಾಮಾಯಣದ ಘೋಷಣೆಯಿಂದ ಯಾವ ರೀತಿ ತಮ್ಮ ತಮ್ಮ ಇಷ್ಟವನ್ನು ಈಡೇರಿಸಿಕೊಳ್ಳಬಹುದು? `ರಾಮನೇ ಹಾಗೆ ತ್ಯಾಗ ಮಾಡಿದ; ನೀವೂ ಭೂದಾನ ಯಜ್ಞ ಮಾಡಿ' ಎಂದು ಹೇಳುತ್ತಾ ತಮ್ಮ ಕಾರ್ಯಸಾಧನೆ ಮಾಡಿಕೊಳ್ಳುವುದೂ ಉಂಟು. 


`ಯಜ್ಞೋ ದಾನಂ ತಪಶ್ಚೈವ ಪಾವನಾನಿ ಮನೀಷಿಣಾಮ್‍' ಎಂದು ಗೀತೆಯಲ್ಲಿ ಭಗವಂತ ಹೇಳಿದ್ದಾನೆ, ಆದುದರಿಂದ ಭೂದಾನ ಯಜ್ಞಕ್ಕೆ ನಾಲ್ಕೆಕರೆ ಜಮೀನು ಕೊಟ್ಟು ಬಿಟ್ಟೆ ಎಂದು ಹೇಳಿಕೊಳ್ಳಬಹುದು. ಆದರೆ ನಮ್ಮ ಆ ಧ್ಯೇಯ ಸರಿಯಾಗಿದೆಯೇ? ಎಂಬುದನ್ನು ಒರೆ ಹಚ್ಚಿ ನೋಡಿಕೊಳ್ಳಬೇಕು. ಕೋರ್ಟಿನಲ್ಲಿ ವ್ಯಾಜ್ಯ ಹಾಕಿದ್ದೆ, ರಾಮಾಯಣ ಪಾರಾಯಣ ಮಾಡಿದೆ, ಕೋರ್ಟಿನಲ್ಲಿ ಜಯವಾಯಿತು, ತಮ್ಮಂದಿರಿಗೆ ಆಸ್ತಿ ಪಾಲನ್ನು ಕೊಡದೇ ಎಲ್ಲವನ್ನೂ ನಾನೇ ದಕ್ಕಿಸಿಕೊಂಡೆ. ನೋಡಿ, ರಾಮಾಯಣದ ಫಲ ಎಂದು ಹೇಳಬಹುದು. ಆದರೆ ``ರಾಜ್ಯವು ತನ್ನ ತಮ್ಮನಿಗೇ ಇರಲಿ" ಎಂದು ತ್ಯಾಗ ಮಾಡಿ ಹೋದ ಧರ್ಮಾತ್ಮನಾದ ರಾಮನು ಕಳ್ಳವ್ಯಾಜ್ಯಕ್ಕೆ ಸಹಾಯ ಮಾಡಿದ ಎಂದರೆ ಶೋಚನೀಯ. ಇವನ ವ್ಯಾಜ್ಯದ ಗೆಲುವಿಗೆ ರಾಮಾಯಣ ಹೊಣೆಯಾಯಿತೇ? ಎಂಬುದನ್ನೂ ನೋಡಬೇಕು. 


ಸ್ವಯಂ ರಾಮಾಯಣದ ಪಾರಾಯಣಕ್ಕೆ ಪ್ರವರ್ತಕರಾದವರೇ ಅದರ ಪಾರಾಯಣದ ಫಲವನ್ನು ಏನು ಹೇಳಿದ್ದಾರೆ? ನೋಡಿ. ``ಜನ್ಮ-ವ್ಯಾಧಿ-ಜರಾ-ವಿಪತ್ತಿ-ಮರಣೈಃ ಅತ್ಯಂತ ಸೋಪದ್ರವಂ; ಸಂಸಾರಂ ಸ ವಿಹಾಯ ಗಚ್ಛತಿ ಪುಮಾನ್‍ವಿಷ್ಣೋಃ ಪದಂ ಶಾಶ್ವತಮ್‍" ಎಂದು. 


ಇದರ ಮಧ್ಯೆ ವಾಲ್ಮೀಕಿಯೇ ನೇರವಾಗಿ ಹೇಳಿದ್ದು ಯಾವುದು? ವಾಲ್ಮೀಕಿಯು ಬರೆದದ್ದು ಒಂದೇ ಕಾಂಡ; ಉಳಿದದ್ದೆಲ್ಲಾ ಪ್ರಕ್ಷಿಪ್ತ ಎಂದು ಮೊದಲಾಗಿ ಚರಿತ್ರಕಾರರು ಬೇರೆ ಗಲಾಟೆ ಮಾಡುತ್ತಾರೆ. ಆ ಸಂದೇಹ ನಿವಾರಣೆಯ ಕೆಲಸ ಬೇರೆ ಇದೆ. 


{\bf ಪ್ರಯೋಜನಪ್ರಾಪ್ತಿಗನುಗುಣವಾದ ವ್ಯವಹಾರವಿರಬೇಕು} 


ನಮ್ಮ ಪ್ರಯೋಜನ ಪ್ರಾಪ್ತಿಗೆ ಅನುಗುಣವಾಗಿ ಇರುವ ಪಾರಾಯಣ ವಿಧಾನವಾದರೂ ಏನು? ಔಷಧಿಯಿಂದ ಗುಣ ಪಡೆಯಬೇಕಾದರೆ `ನೀನು ನಾಲ್ಕು ದಿವಸ ಖಾರ ತೆಗೆದುಕೊಳ್ಳಕೂಡದು, ಪೂರ್ತಾ ರೆಸ್ಟ್‍ ತೆಗೆದುಕೊಳ್ಳಬೇಕು, ಈ ಸಿಸ್ಟಮ್‍ನಲ್ಲಿದ್ದರೆ ಈ ಮೆಡಿಸಿನ್‍ ನಿನ್ನ ಖಾಯಿಲೆ ವಾಸಿ ಮಾಡುತ್ತೆ' ಎಂದು ಡಾಕ್ಟರ್‍ ಹೇಳುವ ವಿಧಾನ ಅನುಸರಿಸುವಂತೆ, ನಿದ್ರೆ ಬರಬೇಕಾದರೆ ಸೊಳ್ಳೆ, 

ತಿಗಣೆ ಯಾವುದರಿಂದಲೂ ಕಡಿಸಿಕೊಳ್ಳಬಾರದು, ಮನಸ್ಸಿನಲ್ಲಿ ಯಾವ ಯೋಚನೆಯೂ ಇರಬಾರದು, ಬೆಚ್ಚಗೆ ಹಾಸಿಗೆ ಇರಬೇಕು, ಬೇಸಿಗೆಯ ಕಾಲವಾದರೆ ತೆಳ್ಳನೆಯ ಹಾಸಿಗೆ, ತಂಗಾಳಿ ಇರಬೇಕು. ಹಾಗಿದ್ದರೆ ನಿದ್ರೆ ಬರುತ್ತೆ, ಎಂದು ನಿಯಮ ವಿಧಾನ. `ಈಗ ಅದಕ್ಕೆಲ್ಲಾ ವಿರಾಮವಿಲ್ಲಾ ಡಾಕ್ಟರ್‍, ನಮಗೆ ಆಫೀಸರ್‍ ಗಲಾಟೆ, ಇಪ್ಪತ್ನಾಲ್ಕು ಗಂಟೆಯೂ ನಿದ್ರೆಯೂ ಆಗುತ್ತಿರಬೇಕು, ಕೆಲಸವೂ 

ಆಗುತ್ತಿರಬೇಕು, ಅದಕ್ಕೆ ಮೆಡಿಸಿನ್‍ ಕೊಡಿ ಎಂದರೆ ಹಾಗೆ ಮಾಡಲು ಸಾಧ್ಯವೇ? ಅದರ ಬದಲು ನಿದ್ರೆ ಹೋಗಲು ಔಷಧಿ ಕೊಡಿ ಎಂದು ಕೇಳಲಿ. ಎರಡೂ ಏಕ ಕಾಲದಲ್ಲಿ ಸಾಧ್ಯವಿಲ್ಲ. ನಿದ್ರೆಗೆ ಎಷ್ಟುಬೇಕೋ ಅಷ್ಟು ಅವಕಾಶ ಕೊಟ್ಟು ಉಳಿದಕಾಲದಲ್ಲಿ ಲೋಕವ್ಯಾಪಾರ ಮಾಡುವ ವ್ಯವಸ್ಥೆ ಇಟ್ಟುಕೊಂಡರೆ ಚೆನ್ನಾಗಿರುತ್ತೆ. 


{\bf ಅವರವರ ಸಂಸ್ಕಾರಾನುಗುಣವಾಗಿ ಬಗೆಬಗೆಯ ಪಾರಾಯಣಗಳು} 


ರಾಮಾಯಣವನ್ನು ಅಂಶತಃ ಓದಿ ಫಲಪಡೆಯುವವರು, ಪೂರ್ತಿ ಓದಿ ಫಲ ಪಡೆಯುವವರು, ಪೂರ್ತಿ ಮಾಡುವುದಕ್ಕಾಗದೇ ಸಪ್ತಸರ್ಗಿ ಕ್ರಮವನ್ನು ರೂಢಿಯಲ್ಲಿರಿಸಿಕೊಂಡಿರುವವರು, ಹೀಗೆಲ್ಲ ಉಂಟು. ಮಾವಿನ ಹಣ್ಣು ಎಂದ ಮಾತ್ರಕ್ಕೇ ಎಲ್ಲಾ ಒಂದೇ ಅಲ್ಲ. ಕವಿಯ ಕರ್ಮ ಕಾವ್ಯ ಎಂದರೆ ಕವಿಯ ವ್ಯಕ್ತಿತ್ವವೇನು? ಕಾವ್ಯದ ಸ್ವರೂಪವೇನು? ಎಂಬುದೂ ತಿಳಿದಿರಬೇಕು. ಮಾವಿನ ಹಣ್ಣಿನಲ್ಲೂ ರಸಪುರಿಯ ಹಣ್ಣು ಇದೆ, ಬಾದಾಮಿ ಹಣ್ಣು ಇದೆ, ತೋತಾಪುರಿ ಹಣ್ಣು ಇದೆ. ಆದರೆ ಎಲ್ಲ ಒಂದೇ ಅಲ್ಲ. ಆಯಾಯಾ ಜಾತಿಯ ಮರದ ವಾಸನೆ, ಕ್ಷೇತ್ರ, ಅದಕ್ಕಾಗಿರುವ ಸಂಸ್ಕಾರ ಮುಂತಾದ ಧರ್ಮವನ್ನೆಲ್ಲಾ ಅದು ಹೊತ್ತು ತರುತ್ತದೆ. ಹಾಗೆಯೇ ಬಾಳೆಯ ಗಿಡದಲ್ಲೂ ಯಾವ ತರಹದ ಭೂಮಿಯಲ್ಲಿ ಯಾವ ತರಹದ ಸಂಸ್ಕಾರಕ್ಕೆ ಒಳಪಟ್ಟು ಆಗಿರುವ ಹಣ್ಣು ಎಂಬುದನ್ನೂ ತಿಳಿಯಬೇಕು. ರಾಮ ಎನ್ನುವ ಪದ ಎಷ್ಟೋ ಜನರ ಬಾಯಿಂದ ಬರಬಹುದು. `ಅಯ್ಯೋ ರಾಮಾ! ಹೀಗಾಯಿತೇ?' ಎಂದು ಹೇಳುವುದುಂಟು `ನೀನು ಶುದ್ಧ ಸಾಪಾಡ್‍ ರಾಮ' `ನಾಳೆ ಎಂಟು ಗಂಟೆಗೆ ಷೇವ್‍ ಮಾಡಿಸಿಕೊಳ್ಳಬೇಕು ಸಿದ್ಧವಾಗಿರೋ ರಾಮ' ಎಂದು ಮುಂತಾಗಿ 

ಹೇಳುವುದುಂಟು. `ರಾಮ ನನ್ನನ್ನು ಕಾಪಾಡಪ್ಪ' ಎಂದೂ ಒಂದು ಧ್ವನಿಯಿಂದ ಜ್ಞಾನಿಗಳು ಹೇಳುವುದೂ ಉಂಟು. ಈ ಎಲ್ಲ ರಾಮರಿಗೂ ಒಂದೇ ಅರ್ಥ ಅಲ್ಲ. ಆ ಶಬ್ದಕ್ಕೆ ಅರ್ಥ, ಅದರ ಹಿನ್ನೆಲೆಯಿಂದ ಕಾವ್ಯವನ್ನು ತಂದವರ ವ್ಯಕ್ತಿತ್ವದ ಮೇಲೆ. ಹಣ್ಣು ಯಾವ ಮರದಿಂದ ಯಾವ ಕ್ಷೇತ್ರದಿಂದ ಬಂದಿದೆಯೋ, ಅದರದರ ಗುಣವನ್ನು ಹೊತ್ತು ತರುತ್ತದೆ ; ಸಂಸ್ಕಾರದ 

ಗುಣವನ್ನೂ ಹೊತ್ತು ತರುತ್ತದೆ. ಭಗವಂತ ದಯಾಮಯ. ಸಮುದ್ರದಿಂದ ಮೇಲೆ ಹೋದ ಉಪ್ಪುನೀರು ಮಳೆಯ ರೂಪದಲ್ಲಿ ಬರುವಾಗ ಉಪ್ಪನ್ನು ಕಳೆದುಕೊಂಡು 

ಶುದ್ಧವಾಗಿ ಸಿಹಿನೀರಾಗಿ ಬರುತ್ತದೆ. ಪ್ರಾಕೃತಿಕವಾದ ಬದಲಾವಣೆಯಾದರೂ, ಆ ಶುದ್ಧವಾದ ವರ್ಷಾಜಲವೇ, ನಮ್ಮ ಊರಿಗೆ ಬರುವಾಗ ಒಗರು - ಸಪ್ಪೆ - ಸಿಹಿ - ಉಪ್ಪು - ಕೆಂಪುಬಣ್ಣ - ಬಗ್ಗಡ ಮುಂತಾದ ಬೇರೆ ಬೇರೆ ಗುಣ ವರ್ಣಗಳೊಡನೆ ಕೂಡಿರುವುದಕ್ಕೆ ಕಾರಣವೇನು? ಕ್ಷೇತ್ರದ ಗುಣ. ಹಾಗೆಯೇ, ಜ್ಞಾನದ ಅಮೃತವರ್ಷವು ಜ್ಞಾನಿಯಿಂದ ಹಾಗೆಯೇ ಬಂದರೂ, ಅದು ನಮ್ಮ ಕ್ಷೇತ್ರದಲ್ಲಿ ಹರಿದು ಬಂದು, ಅಲ್ಲಿಂದ ಇನ್ನೊಬ್ಬರ ಕ್ಷೇತ್ರದಲ್ಲಿ ಹರಿದು ಹೋಗುವ ವೇಳೆಗೆ ಬೇಕಾದಷ್ಟು ರೂಪಾಂತರ ತಾಳಿ ಬಿಡುತ್ತದೆ. ಬಗೆ ಬಗೆಯ ಪಾರಾಯಣಗಳು ಅದಕ್ಕೆ ಲೋಕನಿದರ್ಶನ. 


{\bf ಚೇತನರ ಸಂಸ್ಕಾರಾನುಗುಣವಾಗಿ ರೂಪಾಂತರಗಳು} 


ಹಾಗೆಯೇ- 


\begin{center} 

{\bf ವಾಲ್ಮೀಕಿಗಿರಿಸಂಭೂತಾ ರಾಮಸಾಗರಗಾಮಿನೀ|\\ 

ಪುನಾತಿ ಭುವನಂ ಪುಣ್ಯಾ ರಾಮಾಯಣಮಹಾನದೀ||} 

\end{center} 


ಹಿಮಾಲಯದಿಂದ ಹೊರಟು ಉತ್ತರದೇಶದಲ್ಲಿ ಹರಿಯುವ ಶುದ್ಧ ಗಂಗಾ ನದಿಯಲ್ಲೇ, ಛಾನಲ್‍ ಮಾಡಿ, ನಮ್ಮ ಮನೆಗೊಬ್ಬರಕ್ಕೂ, ಬಚ್ಚಲಿಗೂ ಹರಿಸಿದರೆ ಅದಕ್ಕೆ ಕೊಳಕಿನ ರೂಪವೇ ಉಂಟಾಗುತ್ತದೆ. ಗಂಗಾದೇವಿಯು ದುಷ್ಟಳೇ? ಅಂದರೆ, ನಮ್ಮ ಕ್ಷೇತ್ರದ ಮಹಿಮೆ. ಹೇಗೆ ಜ್ಞಾನವೂ, ಅದನ್ನು ಹೊತ್ತು ಕೊಳ್ಳುವ ಚೇತನರ ಸಂಸ್ಕಾರಾನುಗುಣವಾಗಿ ರೂಪಾಂತರವನ್ನು ತಾಳಿ ಬಿಡುತ್ತದೆಯೋ ಹಾಗೆಯೇ ಶುದ್ಧವೂ ನಿಶ್ಚಿತಜ್ಞಾನವುಳ್ಳದ್ದೂ ಆದ ರಾಮಾಯಣವೂ, ನಮ್ಮ ಬುದ್ಧಿಯಲ್ಲಿ ದೋಷಗಳಿದ್ದರೆ, ಬೇಕಾದಷ್ಟು ಟೀಕಾ ಟೋಕಗಳನ್ನು ಹೊತ್ತುಕೊಳ್ಳಬೇಕಾಗುತ್ತದೆ. 


{\bf ಹೇಗಿದ್ದಾಗ ರಾಮಾಯಣವು ಮೂಲಕ್ಕೆ ಕೊಂಡೊಯ್ಯುತ್ತದೆ?} 


ಒಂದು ಕಡೆ ಚಪಲತೆಯಿರುವೆಡೆಯಲ್ಲಿ ಕಪಿಲಾಜಲದಂತೆ ಬಗ್ಗಡವಾಗಿದ್ದರೂ, ಮತ್ತೊಂದು ಕಡೆ, 


\begin{center} 

{\bf ಅಕರ್ದಮಮಿದಂ ತೀರ್ಥಂ ಭರದ್ವಾಜ ನಿಶಾಮಯ|\\ 

ರಮಣೀಯಂ ಪ್ರಸನ್ನಾಂಬು ಸನ್ಮನುಷ್ಯಮನೋ ಯಥಾ||} 

\end{center} 


ಎಂದು ಹೇಳಿರುವಂತೆ ಸಂತರ ಮನಸ್ಸಿನಂತೆ ಶಾಂತವಾದ ತಿಳಿಯಾಗಿರುವ ನೀರೂ ಉಂಟು. ನೀರು ಸ್ವಚ್ಛವಾಗಿದ್ದರೆ, ಪಕ್ಕದ ವಸ್ತುಗಳು ಸರಿಯಾಗಿ ಪ್ರತಿಬಿಂಬಿಸುವಂತೆ ಮನಸ್ಸೂ ಶುದ್ಧವಾಗಿದ್ದರೆ ಅದರಲ್ಲಿ `ಆತ್ಮರತ್ನ' ವೂ ಕಾಣುತ್ತದೆ. ಸ್ವಚ್ಛವಾದ ದರ್ಪಣದಲ್ಲಿ ಮುಖ ಸರಿಯಾಗಿ ಕಾಣುವುದಲ್ಲವೇ? ಹಾಗೆಯೇ ರಾಮಾಯಣವನ್ನು ತಂದವರ ವ್ಯಕ್ತಿತ್ವ, ಅವರಿಗೆ ಉಪದೇಶ ಮಾಡಿದವರ ವ್ಯಕ್ತಿತ್ವ, ಏನು? ಇಂದಿಗೂ ಉಪದೇಶ ನಡೆಯುತ್ತಿದೆಯಲ್ಲಾ, ಅಲ್ಲೂ ಅದೇ ತರಹದ ವ್ಯಕ್ತಿತ್ವವಿದೆಯೇ? ತಪಃ ಸ್ವಾಧ್ಯಾಯ ನಿರತರಾದ ಮುನಿಗಳಿಂದ ಉಪದೇಶ ಬರುತ್ತಿದೆಯೇ? ವ್ಯಾಖ್ಯಾತೃವು ಆ ಗ್ರಂಥಕ್ಕೆ ವ್ಯಪದೇಶ ಬರದಂತೆ ಮಾಡುತ್ತಾನೆಯೇ? ಆ ಪ್ರದೇಶಕ್ಕೆ ಕೊಂಡೊಯ್ಯುತ್ತಾನೆಯೇ? ಎಂಬುದನ್ನು ನೋಡಬೇಕು. ಇದಾದ ಮೇಲೆ ಪಟ್ಟ! 


{\bf ಪದಾರ್ಥಜ್ಞಾನವಿದ್ದರೆ ಅದನ್ನು ಬಳಸುವುದೂ ರಕ್ಷಿವುದೂ ಸಾಧ್ಯ} 


ಅಷ್ಟು ಉದಾತ್ತವಾದ ಗ್ರಂಥವನ್ನು ನಮ್ಮ ಯೋಗ್ಯತೆಗನುಗಣವಾಗಿ ಹೇಗೆ ಉಪಯೋಗಿಸಿಕೊಳ್ಳಬೇಕು? ಒಂದು ದೇಶದಲ್ಲಿ ಮಹಾಕಾವ್ಯವಿರಬಹುದು. ಮಹಾಕಾಯವುಳ್ಳ ಪುರುಷನೂ ಇರಬಹುದು. ಭೀಮನ ಗದೆಯೂ, ಭೀಮನ ಅಂಗಿಯೂ ನಮಗೆ ಈಗ ಸಿಕ್ಕಿದರೆ ಅದನ್ನು ಹೇಗೆ ಬಳಸಿಕೊಳ್ಳಬಹುದು? ಎತ್ತಲಾರದ ಗದೆ, ಹಾಕಿಕೊಳ್ಳಲಾರದ ಅಂಗಿ. ಭೀಮನ ಅಂಗಿ ಎಂದು ಅದನ್ನು ಕತ್ತರಿಸಲೂ ಇಷ್ಟವಿಲ್ಲ. ಅವನ ಒಂದಂಗಿಯನ್ನು ಕತ್ತರಿಸಿದರೆ ಎಷ್ಟು ಥಾನು ಬಟ್ಟೆ ಇದೆ, ಅದರಲ್ಲಿ ಭಾರತದೇಶದಲ್ಲಿರುವವರಿಗೆಲ್ಲಾ ಅಂಗಿ ಆಗುವುದಯ್ಯ ಎಂದರೆ ಉಪಯೋಗಕ್ಕೆ ತಕ್ಕಂತೆ ಏಕೆ ಕತ್ತರಿಸಿಕೊಳ


ಕೂಡದು? ಅವನ ಕಾಯದಷ್ಟು ದೊಡ್ಡಕಾಯ ನಮಗಿಲ್ಲವಲ್ಲ! ನಮ್ಮ ಷರಟನ್ನು ಕತ್ತರಿಸಿ ಸೂತಿಕಾಗೃಹದಲ್ಲಿರುವ ಮಗುವಿಗೆ ಐದು ಅಂಗಿಗಳನ್ನು ಹೊಲಿಸಬಹುದಲ್ಲವೇ? ಮಹಾಕಾಯವುಳ್ಳವರ ಉಡುಪು ಸಿಕ್ಕಿದರೂ ನಮಗೆ ಉಪಯೋಗವಾಗುವ ರೀತಿಯಲ್ಲಿ ಅದನ್ನು ಮಾಡಿಕೊಳ್ಳಬೇಕು. ನಾವಿಂದು ಅದನ್ನು ಎಷ್ಟರ ಮಟ್ಟಿಗೆ ಧರಿಸಬಹುದು! ಅವರ ದೇಹ ನಮಗಿದೆಯೇ? ದೇಹವಿದ್ದರೂ ಆ ಬಟ್ಟೆಯನ್ನು ಎಷ್ಟರ ಮಟ್ಟಿಗೆ ಚೊಕ್ಕಟವಾಗಿಟ್ಟುಕೊಳ್ಳಬಹುದು. ಮಕ್ಕಳಿಗೆ ಒಳ್ಳೆಯ ಸಿಲ್ಕ್‍ ಬಟ್ಟೆ ಹಾಕಿದರೆ ಎರಡು ಘಂಟೆಯೊಳಗೆ ಮಲ ಮೂತ್ರ ವಿಸರ್ಜನೆ ಮಾಡಿ ಅದನ್ನು ಹಾಳುಮಾಡಿಬಿಡುತ್ತವೆ. ನಿರ್ಮಲತೆಯು ಕೆಡದಂತೆ ಅದರ ರಕ್ಷಣೆ ಮಾಡಬೇಕು. `ಅವನ ಪ್ರಾಣ ಹೊರಟು ಹೋದೀತು, ಅದು ಹೊರಟು ಹೋಗದಂತೆ ಈ ಮನೆಗೆ ಪಹರೆ ಇರಿ' ಎನ್ನುತ್ತಾನೆ, ಜೀವದ ಬಗ್ಗೆ ಕಲ್ಪನೆಯೇ ಇಲ್ಲದ ಮೂರ್ಖ. ಜೀವಕ್ಕೂ ನಮ್ಮಂತೆಯೇ ಒಂದು ಆಕಾರವಿದೆ ಎಂದು ಆತನ ಮೂಢನಂಬಿಕೆ. ಅದು ಸೂಕ್ಷ್ಮರೂಪಿಯೂ, ತೇಜೋರೂಪಿಯೂ ಆಗಿದೆ ಎಂಬುದನ್ನು ಅರಿತು ಅದರ ರಕ್ಷಣೆ ಹೇಗೆ ಮಾಡಿದರೆ ಸರಿಯಾಗುತ್ತದೆ? ಎನ್ನುವುದನ್ನು ಮೊದಲು ಮನದಟ್ಟು ಮಾಡಿಕೊಂಡು ಆಮೇಲೆ ಅದಕ್ಕೆ ಹೋಗಬೇಕು. `ಈ ಹಾಲನ್ನು ಒಳಗಿಟ್ಟಿರೀಪ್ಪಾ' ಎಂದು ಕೊಟ್ಟರೆ, ಅದನ್ನು ಒಂದು ತಿಂಗಳ ಕಾಲ ಹಾಗೆಯೇ ಕಬ್ಬಿಣದ ಪೆಟ್ಟಿಗೆಯಲ್ಲಿ ಭದ್ರವಾಗಿಟ್ಟು ಆಮೇಲೆ `ಜೋಪಾನವಾಗಿಟ್ಟಿದ್ದೇನೆ, ನಿನ್ನ ಪದಾರ್ಥವನ್ನು ತೆಗೆದುಕೋ' ಎಂದರೆ ಆ ವೇಳೆಗೆ ಹಾಲು ಹಾಳಾಗಿರುತ್ತೆ. ಹೂವನ್ನು `ಸರಿಯಾಗಿಟ್ಟಿರಪ್ಪಾ' ಎಂದು ಕೊಟ್ಟರೆ ಅದಕ್ಕೆ ನೀರು ಕುಡಿಸಿ ತೊಂಬೆಯೊಳಗಿಟ್ಟು ಮೆತ್ತೆ ಹಾಕಿ `ಅರ್ಧದಳವನ್ನೂ ಪೋಲು ಮಾಡಿಲ್ಲ; ಎಲ್ಲವನ್ನೂ ತೊಂಬೆಯೊಳಗಿಟ್ಟು ಸೀಲ್‍ ಮಾಡಿಟ್ಟಿದ್ದೆನು' ಎಂದರೆ ಕೊಟ್ಟವನಿಗೆ ಎಷ್ಟು ಸಂಕಟವಾಗಬಹುದು! `ಮಗುವನ್ನು ಭದ್ರವಾಗಿ ನೋಡಿಕೊಳ್ಳಪ್ಪಾ' ಎಂದರೆ ಕಬ್ಬಿಣದ ಪೆಟ್ಟಿಗೆಯಲ್ಲಿ ಬೀಗ ಹಾಕಿ ಇಟ್ಟಿದ್ದು ಕೊಟ್ಟನಂತೆ! ಪದಾರ್ಥದ ಸ್ವರೂಪವನ್ನರಿತು ಅದು ಕೆಡದಂತೆ ಅದಕ್ಕೆ ಬೇಕಾದ ವಾತಾವರಣದಲ್ಲಿಟ್ಟರೆ ಆಗ ಅದರ ರಕ್ಷಣೆಯಾಗುತ್ತದೆ. ``ಹಿಟ್ಟು ಕೇಳಬೇಡ, ಬಟ್ಟೆ ಕೇಳಬೇಡ, ಗಿಣಿಯಂತೆ ಸಾಕುತ್ತೇನೆ" ಎಂದನಂತೆ. ಜೀವನಕ್ಕೆ ಬೇಕಾದ್ದೇ ಹಿಟ್ಟು ಬಟ್ಟೆ. ಅವೆರಡೂ ಇಲ್ಲದೆ ಪಂಜರದಲ್ಲಿ ಹಾಕಿಡುತ್ತೇನೆ ಎಂದರ್ಥವೇ? 


{\bf ರಾಮಾಯಣವನ್ನು ಅರ್ಥಮಾಡಿಕೊಳ್ಳಲು ಒಂದು ಅಧಿಕಾರಸಂಪತ್ತಿರಬೇಕು} 


ಸಾಕ್ಷಾತ್‍ ಜ್ಞಾನಗಂಗೆಯೇ ಮಹಾತ್ಮರ ಜ್ಞಾನಭೂಮಿಯಿಂದ ರಾಮಾಯಣದ ರೂಪದಲ್ಲಿ ಹರಿದು ಬಂತು, ನಿಜ. ರಾಮಾಯಣವು ವೇದಸ್ವರೂಪ. ಆದರೆ ರಾಮಾಯಣದಲ್ಲಿ ``ಭದ್ರಂ ಕರ್ಣೇಭಿಃ" ಆಗಲಿ ``ಇಷೇತ್ವೋರ್ಜೇತ್ವಾ" ಆಗಲಿ ಅಶೀತಿದ್ವಯವಾಗಲಿ ಎಲ್ಲೂ ಇಲ್ಲ ಎಂದರೆ, ಹಾಗೆಯೇ ತಿರುವಾಯ್‍ ಮೊಳಿಯನ್ನು ತಮಿಳ್‍ ವೇದಂ ಶೈದವಾರ್‍ ಶಠಗೋರ್ಪ ಎನ್ನುತ್ತಾರೆ, ``ಉಯರ್‍ವರ ಉಯರ್‍ ವಲಂ" ಎಂಬುದರಲ್ಲಿ ವೇದದ ವಾಕ್ಯವಾವುದೂ ಇಲ್ಲವಲ್ಲಾ; ಎಂದರೆ ವೇದವೆಂದರೇನು? ಭಗವಂತನ ಸೃಷ್ಟಿ ವಿದ್ಯೆ. ರಾಮಾಯಣ, ತಿರುವಾಯ್‍ ಮೊಳಿಗಳಲ್ಲಿ ವೇದವು ಪಂಕ್ತಿ ಪಂಕ್ತಿಯಾಗಿಲ್ಲದಿದ್ದರೂ, ಅದರ ಪ್ರತಿಪಾದ್ಯ ವಿಷಯವು ಅಲ್ಲೂ ಇದೆ. ಕಾವ್ಯದ, ಕವಿಯ ಸ್ವರೂಪವೇನು? ಅವರಲ್ಲಿ ಯಾವರೀತಿಯ ಮಾತು ಬಂದೀತು? ಅದನ್ನು ನಾವು ಯಾವ ರೀತಿಯಲ್ಲಿ ತೆಗೆದುಕೊಳ್ಳಬೇಕು? ಹೇಗೆ ತೆಗೆದುಕೊಂಡರೆ ಸರಿಯಾಗುತ್ತದೆ? ಎಂಬುದನ್ನು ನೋಡಬೇಕು. `ಸುಂದರವಾದ ವಿಗ್ರಹ ನೋಡಪ್ಪಾ!' ಎಂದು ಕುರುಡನ ಮುಂದೆ ವಿಗ್ರಹ ಹಿಡಿದರೆ ಆತನು ಮೂಸಿ ನೋಡಿದರೆ ತಿಳಿಯುವುದೇ? `ಎಂತಹ ಸುಶ್ರಾವ್ಯವಾದ ಕಂಠವಪ್ಪಾ!' ಎಂದು ಕಿವುಡನ ಮುಂದೆ ಹೇಳಿದರೆ ಕಣ್ಣಿನಿಂದ ಅವನಿಗೆ ಅದರ ಮಾಧುರ್ಯದ ಅರಿವಾಗುವುದೆ? ಜೇನುತುಪ್ಪವು ಮಧುರವೆಂದು ಅದನ್ನು ಕಿವಿಗೆ ಬಿಟ್ಟು ಕೊಂಡರೆ ಸಿಹಿ ತಿಳಿಯುವುದೇ? ಇಂದ್ರಿಯಗಳ ವಿಷಯವೇ ಆದರೂ ಅದರ ಸ್ಥಾನಪಲ್ಲಟ ಮಾಡಿದರೆ ರಸಗ್ರಹಣವಿಲ್ಲ. ಬುದ್ಧಿ ಪಲ್ಲಟವಾಗಿರುವಾಗ ಸದ್ವಿಷಯವನ್ನು ಕೊಟ್ಟರೂ ಅದು ಗ್ರಹಿಸುವುದಿಲ್ಲ. ಸದ್ಭುದ್ಧಿ ದುರ್ಬುದ್ಧಿ ಎರಡಕ್ಕೂ ವಿಷಯ ಬೇರೆ ಬೇರೆ. ಪ್ರಸಾದಬುದ್ಧಿ ಇಲ್ಲದವರಿಗೆ ದೇವರ ಪ್ರಸಾದ ಕೊಟ್ಟರೂ ನಿರ್ಮಾಲ್ಯವಾಗುತ್ತದೆ. ಸನ್ಮನುಷ್ಯ-ಯೋಗಾತಿಶಯವುಳ್ಳವನಿಗೆ ಸದ್ವಿಷಯ ಕೊಡಬೇಕು. ವಿಷಯದ ಸ್ವರೂಪ, ಅಧಿಕಾರಿಗಳನ್ನು ನೋಡಿ ಕೊಡಬೇಕು. ಲವಕುಶರಿಗಾದರೆ ರಾಮಾಯಣವನ್ನು ಹೇಳಲು ಕಥಾನಾಯಕನ ವಂಶದಲ್ಲಿಯೇ, ಆತನ ವೀರ್ಯದಿಂದಲೇ ಮೂಡಿ ಬಂದಿರುವ ಯೋಗ್ಯತೆ ಇದೆ. ಆದರೆ ನಮನಮಗಿರುವ ಯೋಗ್ಯತೆಯಂತೆ ನಾವು ಅದನ್ನು ಉಪಯೋಗಿಸಿಕೊಳ್ಳಬೇಕು. ನೀವೂ ವರದಾಚಾರ್ಯರೂ ಇಷ್ಟು ಬೆಳಕಿನ ಸೂರ್ಯಕಿರಣಗಳನ್ನು ಬಯಸುತ್ತೀರಿ. ಅಂತೆಯೇ ಸಣ್ಣ ಮಗುವಿಗೂ ಸೂರ್ಯನು ಬೇಕು. ಕಣ್ಣು ನೋವುಳ್ಳವನಿಗೂ ಬೇಕು, ಆದರೆ ಕಣ್ಣಿಗೆ ಶಾಖವನ್ನು ಕೊಡಲು ಅಲ್ಲ, ದೇಹಕ್ಕೆ. ಕತ್ತಲೆಯಲ್ಲಿ ಯಾವುದನ್ನಾದರೂ ತಡಕುತ್ತಿದ್ದರೆ ಆಗ ಸೂರ್ಯ ಬೇಕು. ಹಾಗೆಯೇ ಜೀವನವೇ ಅಂಧಕಾರಮಯವಾಗಿದೆ, ದಡ ಕಾಣಲು ಕಣ್ಣು ಕಾಣುತ್ತಿಲ್ಲ, ಎಲ್ಲಪ್ಪಾ ಜ್ಞಾನಸೂರ್ಯ! ಎಂದು ಮಹಾತ್ಮರು ತಮ್ಮ ಕಣ್ಣಿಗೆ ವಿಷಯವಾಗಿಸಿಕೊಳ್ಳುತ್ತಾರೆ. ತಮ್ಮ ಕಣ್ಣಿಗೆ ಎಷ್ಟು ತಡೆಯುತ್ತದೆಯೋ ಅಷ್ಟು ಮಟ್ಟಿಗೆ ಕಾವ್ಯದ ಪ್ರಕಾಶವನ್ನು ನಾವು ಅವರ ಉಪದೇಶದಿಂದ ಪಡೆಯಬೇಕು. ಯೋಗ್ಯತೆಯನ್ನು ನೋಡಿ ಉಪದೇಶ. ಮಗುವಿನ ಕೈಯಲ್ಲಿ ಪೆನ್‍ ಕೊಡುವುದಿಲ್ಲ, ಅದು ಏನಾದರೂ ನಿಬ್ಬನ್ನು ಕುಟ್ಟಿ ನೋಡಿದರೆ! ಆದುದರಿಂದ ಅದನ್ನು ಹೊರುವ ಸಾಮರ್ಥ


ವಿದ್ದರೂ ಅದಕ್ಕೆ ಜವಾಬ್ದಾರಿಯಿಲ್ಲದಿರುವುದರಿಂದ ನಾವೇ ತೆಗೆದುಕೊಂಡು ಹೋಗಿ ಇಡುತ್ತೇವೆ. ದೊಡ್ಡ ಬಂಡೆಯನ್ನು ಮಗುವಿನ ಮೇಲೆ ಹೊರಿಸಲಾಗುವುದಿಲ್ಲ. ಹಿಮಾಲಯ ಪರ್ವತದ ಬಳಿ ಕರೆದುಕೊಂಡು ಹೋಗಿ `ನಮ್ಮದೇನೂ ಅಬ್ಜಕ್ಷನ್‍ {\rm (Objection)} ಇಲ್ಲ ನೀವು ಬೇಕಾದರೆ ನಿಮ್ಮ ಊರಿಗೆ ತೆಗೆದುಕೊಂಡು ಹೋಗಬಹುದು,' ಎಂದು ಉದಾರತೆಯಿಂದ ಒಬ್ಬರು ಹೇಳಿದರಂತೆ. ``ನಾನೇನೋ ಹೊತ್ತುಕೊಂಡು ಹೋಗುತ್ತೇನೆ, ಆದರೆ ಯಾರಾದರೂ ಎತ್ತಿಡಬೇಕು; ಅದಕ್ಕೇನು! ಅವಶ್ಯವಾಗಿ ಹೊತ್ತುಕೊಂಡು ಹೋಗುತ್ತೇನೆ" ಎಂದು ಉತ್ತರ. ಇಲ್ಲಿ ಎರಡೂ ಅಸಂಭವ. ಅದುದರಿಂದ ನಮ್ಮ ಯೋಗ್ಯತೆ ನೋಡಿ ಗ್ರಹಿಸಬೇಕು. ಅಷ್ಟು ಅನ್ನವನ್ನು ಮೊಟ್ಟ ಮೊದಲು ಮಗುವಾಗಿರುವಾಗ ತಿನ್ನಲಾಗುವುದಿಲ್ಲ. ಇಷ್ಟಿಷ್ಟನ್ನು ಕೊಡುತ್ತಾ ಬನ್ನಿ. ಮೊದಲು ಸ್ತನ್ಯ, ಆಮೇಲೆ ಸ್ವಲ್ಪ ಸ್ವಲ್ಪ ತುತ್ತುಗಳು. ಕಾಲಕ್ರಮದಲ್ಲಿ ಭಕ್ಷ್ಯ, ಲೇಹ್ಯ, ಚೋಷ್ಯ, ಪೇಯಗಳೆಂಬ ನಾಲ್ಕು ಬಗೆಯ ಪಾಕಗಳನ್ನು ತಿನ್ನುವ ಯೋಗ್ಯತೆ ಬರುತ್ತದೆ. ಹಾಗೆಯೇ ನಾಲ್ಕು ವಿಧವಾದ ಪುರುಷಾರ್ಥಗಳ ವಿಷಯದಲ್ಲಿಯೂ. ಇದು ಸಾಹಿತ್ಯದಿಂದ ದೊರಕುವ ಆಹಾರ, ರಸ. 


{\bf ಉಪದೇಶ ಪದಾರ್ಥದ ವಿವರಣೆ} 


ರಾಮಾಯಣದ ವಿಷಯಕ್ಕೆ ಸಂಪೂರ್ಣ ಅಧಿಕಾರಿ ಯಾರು? ಎಂದರೆ ಆ ಕಾವ್ಯ ರಚನೆಯ ವಿಷಯ ಒಂದನ್ನು ಬಿಟ್ಟು ಉಳಿದ ಎಲ್ಲ ವಿಷಯಗಳಲ್ಲಿಯೂ ವಾಲ್ಮೀಕಿಯಷ್ಟೇ ಯೋಗ್ಯತೆ ಇರಬೇಕು. ಮೊದಲು ಹುಟ್ಟಿದ ವಿಷಯದಲ್ಲಿ ಅಣ್ಣನಲ್ಲದಿದ್ದರೂ ಅವನಷ್ಟೇ ಉಳಿದ ಯೋಗ್ಯತೆಗಳೂ ಕುಟುಂಬನಿರ್ವಹಣೆ ಮಾಡಲು ತಮ್ಮನಿಗೂ ಇರಬೇಕು. ವಾಲ್ಮೀಕಿರಾಮಾಯಣದ ಪಾರಾಯಣಗಳು ಎಲ್ಲರ ಮನೆಯಲ್ಲಿಯೂ ಶರನ್ನವರಾತ್ರಿ, ರಾಮನವರಾತ್ರಿ, ರಾಮಾಯಣಸಪ್ತಾಹ ಮುಂತಾದವುಗಳಲ್ಲಿ ನಡೆಯುತ್ತಿದ್ದರೂ, ಯೋಗ್ಯತೆಯನ್ನು ಪಡೆದರೆ ತಾನೇ ಅಧಿಕಾರಯುತವಾಗಿ ಅದನ್ನು ಕೊಡಲು ಬಿಡಲೂ ಶಕ್ತಿ ಬರುತ್ತದೆ. ರಾಮಾಯಣವು ಉಪದೇಶರೂಪವಾಗಿಯೇ ಬಂದಿದೆ. ``ನಿಮಗಾಗಿದೆಯೋ ಉಪದೇಶ?" (ಹೌದು, ಎಂದು ಉತ್ತರ ಬಂತು.) ಅವರು ಅಡಿಗೆ ಮಾಡಿಕೊಟ್ಟರಲ್ಲಾ - ಉಪದೇಶಿಸಿದರಲ್ಲಾ ಅದು ನಿಮಗೆ ಅಂದು ಎಷ್ಟು ರುಚಿಯಾಗಿತ್ತು? ಕೇವಲ ಉಪದೇಶ ಮಾಡಿದರೆಂಬ ರಸವೇ? ಅಥವಾ ಇನ್ನೇನಾದರೂ ರಸ ತುಂಬಿದರೋ ಉಪದೇಶಕರು? ಅವರಿಂದ ಒಂದು ಶ್ಲೋಕ ಬಂದಿತು ಅಷ್ಟೇ. ಆದರೆ ಬಹಳ ದೂರದಲ್ಲಿರುವ ಒಂದು ವಿಷಯವನ್ನು ಕುರಿತು ಯಾರು ಅಲ್ಲಿಗೆ ಹೋಗಬೇಕೆಂದು ಅಪೇಕ್ಷಿಸುತ್ತಾರೆಯೋ ಅವರಿಗೂ ನಮಗೂ ಇರುವ ಅಂತರ ತಮಗೂ ಆ ವಿಷಯಕ್ಕೂ ಇರುವ ಅಂತರ, ಇವು ಏನು? ಎಷ್ಟು? ಎಂದು ತಿಳಿದು ಉಪದೇಶಿಸಿದರೋ ಉಪದೇಶಕರು? ಉದಾಹರಣೆಗೆ; `ಮದ್ರಾಸ್‍ ಮೈಸೂರುಗಳಿಗಿರುವ ಅಂತರ ಇಷ್ಟು, ಅಡ್ರೆಸ್‍ ತೆಗೆದುಕೊಳ್ಖಿ, ಸಂಪಿಗೆ ರೋಡಿನಲ್ಲಿ ಹೀಗೆ ಬಂದರೆ ಇಲ್ಲೊಂದು ಕ್ರಾಸ್‍ ಸಿಗುತ್ತದೆ. ಅಲ್ಲಿ ನಾಲ್ಕನೆಯ ಕ್ರಾಸಿಗೆ ಬಂದರೆ ಅಲ್ಲೊಂದು ದೊಡ್ಡ ಬಾವಿ, ಅಲ್ಲಿ ನಮ್ಮ ಮನೆ ಎಂಬ ಮಾತನ್ನು ಹಿಡಿದುಕೊಂಡರೆ ಗೈಡೆನ್ಸ್‍ಸಿಗುತ್ತದೆ. ಅದನ್ನು ನೊಡದೆ ಸುಮ್ಮನೆ ಹೀಗೆ ಉಂಟು, ಹಾಗೆ ಇದೆ, ಇಲ್ಲಿಂದಲ್ಲಿಗೆ ಬನ್ನಿ ಎಂದು ಏನೇನೋ ಹೇಳಿದರೆ ಅದು ನಿಮ್ಮ ದೇಶಕ್ಕೆ ತಲುಪಿಸುವುದಿಲ್ಲ. ಒಂದು ನಿದರ್ಶನ ಕೊಡುತ್ತೇನೆ, ಒಂದೆರೆಡು ನಿಮಿಷ ಬನ್ನಿ. (ವಿಜ್ಞಾನ ಮಂದಿರದ ಮುದ್ರೆಯುಳ್ಳ ಪಟವನ್ನು ತೋರಿಸುತ್ತಾ) ಇದರಲ್ಲಿ ಏನೇನು ಕಾಣುತ್ತದೆ ಹೇಳಿ, (ಅವರು ಕೆಲವು ವಿಷಯಗಳನ್ನು ಮಾತ್ರ ಹೇಳಿದರು. ಗೋ-ಹಯಮುಖ, ಊರ್ಧ್ವಮುಖವಾದ ಹಂಸ. ಊರ್ಧ್ವಮುಖವಾದ ಎರಡು ತಲೆಗಳು ಇವುಗಳನ್ನು ಹೇಳಲಿಲ್ಲ. ಅವುಗಳನ್ನು ಅವರಿಗೆ ತಾವೇ ತೋರಿಸಿ) ನೋಡಿ. ಇವುಗಳೆಲ್ಲಾ ಇಲ್ಲೇ ಇದ್ದವಲ್ಲಾ ನಿಮಗೇಕೆ ಮೊದಲು ಕಾಣಿಸಲಿಲ್ಲ? ಅಂದರೆ ದೃಷ್ಟಿ ಇರಲಿಲ್ಲ. ಇದರಲ್ಲಿ ಮೇಲಕ್ಕೆ ಹಾರಿಹೋಗುತ್ತಿರುವ ಪರಮಹಂಸವನ್ನು ಬಿಟ್ಟರೂ ನ್ಯೂನತೆ, ಆ ಹಂಸದ ಕಡೆಗೆ ಊರ್ಧ್ವಮುಖವಾದ ದೃಷ್ಟಿಯಿಟ್ಟು ಲಕ್ಷಿಸುತ್ತಿಸುವ ಎರಡು ಚೇತನರನ್ನು ನೋಡದಿದ್ದರೂ ನ್ಯೂನತೆ, ಹೇಳದಿದ್ದರೂ ನ್ನೂನತೆ. ಅದನ್ನು ಸರಿಯಾಗಿ ಉಪದೇಶ ಮಾಡದಿದ್ದುದರಿಂದ ಮೊದಲು ತಿಳಿಯಲಿಲ್ಲ; ಈಗ ತಿಳಿಯಲು ಕಾರಣ, ಸರಿಯಾಗಿ ಉಪದೇಶವಾಗಿರುವುದು. ಉಪದೇಶ ಎಂದರೆ ನಿಮ್ಮ ಬುದ್ಧಿ ದೇಶಕ್ಕೆ ಅಲ್ಲಿರುವುದನ್ನು ತೆಗೆದುಕೊಳ್ಳುವಂತೆ ಮಾಡಿ ವಿಷಯದ ಬಳಿಗೆ ವಿಷಯದೇಶಕ್ಕೆ ಕೊಂಡೊಯ್ಯುವುದು. ಹಾಗೆಯೇ ವಾಲ್ಮೀಕಿಯ ಹೃದಯವನ್ನು ತಿಳಿದು ಅವರ ಹೃದಯದೇಶಕ್ಕೆ ನಿಮ್ಮನ್ನು ಒಯ್ಯಲು ಉಪದೇಶ ಬೇಕು. 


{\bf ಉಪದೇಶಪರಂಪರೆಯ ಕುರಿತು ವಿಮರ್ಶೆ} 


ಜನಜಂಗುಳಿಯಲ್ಲಿ ಸಿಕ್ಕಿ ಸಾವಿರಾರು ಜನರ ಮನಸ್ಸು ಇಂದ್ರಿಯಗಳ ಕೈವಾಡದಿಂದ ಅದು ಮಲಿನವಾಗಿಬಿಟ್ಟಿದೆ. ಕಾಲವನ್ನು ಸೂಚಿಸುತ್ತಿರುವ ನಿಮ್ಮ ಗಡಿಯಾರವನ್ನು ತೆಗೆದುಕೊಂಡು, ಅದನ್ನು ಕ್ಲೀನಾಗಿ ಇಟ್ಟುಕೊಳ್ಳದೆ ಅದರ ಮೇಲೆ ಕೊಳೆ, ತೊಪ್ಪೆ, ಎಲ್ಲಾ ಬೀಳಿಸಿ ಮುಚ್ಚಿಬಿಟ್ಟಾಗ ಅದು ವಿಷಯವನ್ನು ತೋರಿಸುವುದಿಲ್ಲ. ಹಾಗೆಯೇ ಭಗವಂತನ ಸುಂದರವಾದ ಪೋಟೋವನ್ನು ನೋಡಿ ನೂರಾರು ಜನರು ಮುಟ್ಟಿ ಮುಟ್ಟಿ ನೋಡಿ ಅನೇಕರ ಹೆಬ್ಬೆಟ್ಟಿನ ಮುದ್ರೆಗಳು ಅದರ ಮೇಲೆ ಆಗಿಬಿಟ್ಟಾಗ ಭಗವಂತನ ಗುರುತು- ಚಿತ್ರ ಕಾಣಿಸುವ ಬದಲು ಜನರ ಬೆಟ್ಟಿನ ಚಿತ್ರ ಮಾತ್ರ ಕಾಣಿಸುತ್ತದೆ. ಅದು ಭಗವಂತನನ್ನು ಮರೆಸಿ ಬಿಡುತ್ತದೆ. ಅದಕ್ಕಾಗಿಯೇ ``ಪೋಟೋವನ್ನು ಬಾರ್ಡರಿನಲ್ಲಿ ಹಿಡಿದುಕೊಳ್ಳಬೇಕು" ಎಂದು ಎಚ್ಚರಿಕೆ ಕೊಡುತ್ತಾರೆ. ಹೀಗೆ ರಾಮಾಯಣದ ವಿಷಯದಲ್ಲಿ ಉಪದೇಶಕರು ನಿಮಗೇನಾದರು ಜವಾಬ್ದಾರಿಯನ್ನು ಕೊಟ್ಟರೇ? ಅವರಿಗೇ - ಉಪದೇಶಕರಿಗೇ ಒಂದು ಐಡಿಯಾ ಇಲ್ಲದಿದ್ದರೆ ಕೇವಲ ಅಕ್ಷರಗಳನ್ನು ಹೇಳುವುದರಿಂದೇನು? ಮನೆಯಲ್ಲಿ ಗರ್ಭಿಣಿಯಾದ ಹೆಂಗಸಿದ್ದಾಳೆ. ಅವಳ ವೇದನೆ ಹೇಗಿದೆ ಎಂಬುದು ಅದರ ಅನುಭವವಿಲ್ಲದವರಿಗೆ ತಿಳಿಯುವುದೇ? ಅನುಭವಿ - ಅನನುಭವಿ ಇಬ್ಬರೂ ಅದನ್ನು ವರ್ಣಿಸಿದ್ದಾರೆ, ಇಬ್ಬರ ಭಾಷೆಯೂ ಬೇರೆ ಬೇರೆ. ಸಂತೋಷವಾಗಿದೆ ಎಂಬುದನ್ನು ದುಃಖದಲ್ಲೇ ಕೊರಗುತ್ತಿರುವವನು ಹೇಳಿದರೆ ರಸವಿರುತ್ತದೆಯೇ? ಕಷ್ಟಪಟ್ಟುಕೊಂಡು ಆ ಮಾತನ್ನು ಹೇಳಿದರೆ ಸಂತೋಷ ಬರುತ್ತದೆಯೇ? ಎರಡರಲ್ಲೂ ಧ್ವನಿ ಬೇರೆ ಬೇರೆ. ಐಡಿಯಾ ಇಲ್ಲದವನಿಗೆ ಧ್ವನಿಯೇ ಇರುವುದಿಲ್ಲ. ತನ್ನಂತೆ ತಾನು ಹೇಳುತ್ತಿರುತ್ತಾನೆ. ಉಪದೇಶ ಮಾಡುವವರಿಗೆ ಪೂರ್ಣಯೋಗ್ಯತೆ ಇರಬೇಕು. ಹೇಳುವವರಿಗೆ ಎಷ್ಟು ಯೋಗ್ಯತೆ ಇರಬೇಕೋ ತೆಗೆದುಕೊಳ್ಳುವವರಿಗೆ ಅಷ್ಟೇ ಯೋಗ್ಯತೆ ಇದೆಯೇ? ಆ ಮಹಾತ್ಮರು ಕೊಡುವ ಅಮೃತವನ್ನು ಬಿಂದುಮಾತ್ರವಾದರೂ ನಾವು ತೆಗೆದುಕೊಳ್ಳಬಲ್ಲೆವೇ? ಗಮನಿಸಬೇಕು. ಪೂರ್ತಿ ಆ ಔಷಧ ಕುಡಿಯುವ ಯೋಗ್ಯತೆ ಬರಲಿಲ್ಲವಾದರೆ ಸ್ವಲ್ಪಸ್ವಲ್ಪವಾಗಿ ಒಂದೊಂದು ತೊಟ್ಟು ಹಾಕಿ ಕೊಟ್ಟರೆ, ಕಾಲಕ್ರಮದಲ್ಲಿ ಪೂರ್ತಿ ಔಷಧಿ ಕುಡಿಯುವ ಯೋಗ್ಯತೆ ಉಂಟಾಗುತ್ತದೆ. ಜೀವನದಲ್ಲಿ ಸಂಸಾರ ಹಾಲಾಹಲವನ್ನುಂಡು ಜೀವಿ ತಳಮಳಿಸುತ್ತಿದ್ದಾನೆ. ಅಂತಹ ಸಂದರ್ಭದಲ್ಲಿ- 


\begin{center} 

{\bf ವಂದೇ ಗುರೂಣಾಂ ಚರಣಾರವಿಂದೇ\\ 

ಸಂದರ್ಶಿತಸ್ವಾತ್ಮಸುಖಾವಬೋಧೇ|\\ 

ಜನಸ್ಯ ಯೇ ಜಾಂಗಲಿಕಾಯಮಾನೇ\\ 

ಸಂಸಾರಹಾಲಾಹಲಮೋಹಶಾಂತ್ಯೈ||} 

\end{center} 


ಎಂದು ಹೇಳಿರುವಂತೆ ಮೋಹಶಾಂತಿ ಉಂಟುಮಾಡುವುದಕ್ಕಾಗಿ, ಮಹರ್ಷಿಮುನಿ ಪ್ರಣೀತವಾದ ಚರಿತ್ರವನ್ನು ನಾವು ಎಷ್ಟು ಪ್ರಮಾಣದಲ್ಲಿ ಸವಿಯಬಲ್ಲೆವು? ಅದರ ಸೊಬಗು ಸೌಂದರ್ಯಗಳೇನು? ಬಾಹ್ಯದೃಷ್ಟಿ ಅಂತರ್ದೃಷ್ಟಿಗಳಿಗೆ ವಿಷಯವಾಗಿ ಪರಮಾನಂದ ಪರಮಪವಿತ್ರತೆಯನ್ನು ಕೊಡುವ ಕಾವ್ಯವನ್ನು ತಕ್ಕ ಪರಮಗುರುಗಳು ಉಪದೇಶಿಸಿದರು. ಆ ಪರಮಾನಂದ, ಪರಮಶಾಂತಿ ತಿಳಿಯಬೇಕಾದರೆ, ಅದರ ಜೀವನಾಡಿ ತಿಳಿಯದಿದ್ದರೆ ಬಹಳ ಕಠಿನ ಸಮಸ್ಯೆಯಾಗುತ್ತೇಪ್ಪಾ! 


\large{{\bf ರಾಮಾಯಣಾನುಸಂಧಾನದ ಫಲ}} 


ವಾಲ್ಮೀಕಿಗಳು ಅಥವಾ ಅವರ ಅನುಯಾಯಿಗಳು ಈ ರೀತಿ ಫಲ ಹೇಳುತ್ತಾರೆ. 


\begin{center} 

ನ ಪುತ್ರಮರಣ ಕಿಂಚಿತ್‍\\ 

{\bf ನ ಚಾಗ್ನಿಜಂ ಭಯಂ ಕಿಂಚಿತ್‍ ನಾಪಿ ಜ್ವರಕೃತಂ ತಥಾ|\\ 

ನಗರಾಣಿ ಚ ರಾಷ್ಟ್ರಾಣಿ ಧನಧಾನ್ಯಯುತಾನಿ ಚ||\\ 

ನಿತ್ಯಂ ಪ್ರಮುದಿತಾಃ ಸರ್ವೆ

ಯಥಾ ಕೃತಯುಗೇ ತಥಾ|\\ 

ಇದಂ ಪವಿತ್ರಂ ಪಾಪಘ್ನಂ ಪುಣ್ಯಂ ವೇದೈಶ್ಚ ಸಮ್ಮಿತಮ್‍|\\ 

ಯಃ ಪಠೇದ್ರಾಮಚರಿತಂ ಸರ್ವಪಾಪೈಃ ಪ್ರಮುಚ್ಯತೇ||} 

\end{center} 


ಮನುಷ್ಯ ವರ್ತನಕ್ಕೆ ಅಂದರೆ ಜೀವನಕ್ಕೆ ಬೇಕಾದ ಫಲವೂ ಇದೆ. 


\begin{center} 

{\bf ಯಃ ಕರ್ಣಾಂಜಲಿಸಂಪುಟೈಃ ಅಹರಹಃ ಸಮ್ಯಕ್‍ ಪಿಬತ್ಯಾದರಾತ್‍\\ 

ವಾಲ್ಮೀಕೇರ್ವದನಾರವಿಂದಗಲಿತಂ ರಾಮಾಯಣಾಖ್ಯಂ ಮಧು|\\ 

ಜನ್ಮವ್ಯಾಧಿಜರಾವಿಪತ್ತಿಮರಣೈಃ ಅತ್ಯಂತಸೋಪದ್ರವಂ\\ 

ಸಂಸಾರಂ ಸ ವಿಹಾಯ ಗಚ್ಛತಿ ಪುರ್ಮಾ ವಿಷ್ಣೋಃಪದಂ ಶಾಶ್ವತಮ್‍||} 

\end{center} 


ಎಂಬಂತೆ ಪರಜೀವನಕ್ಕೆ ಬೇಕಾದ ಫಲವೂ ಇದೆ. 


{\bf ರಾಮಾಯಣವು ಎಂದೆಂದೂ ಉಳಿಯುವ ಕಾವ್ಯ} 


ಉಪದೇಶಕರಾದ ನಾರದರ ವಾಕ್ಯವಿದೆ; ಬ್ರಹ್ಮನಕಡೆಯಿಂದ ಬಂದ ಉಪದೇಶವಿದೆ. ಬ್ರಹ್ಮರೂ ಉಪದೇಶಕರೇ. 


ಎಲ್ಲಿಯೋ ಬೇಡ ಪಕ್ಷಿಯನ್ನು ಕೊಂದುಹಾಕಿದ ಸನ್ನಿವೇಶ. ಇದನ್ನು ಬಳಸಿಕೊಂಡು ಬ್ರಹ್ಮನೂ, ನಾರದರೂ ಎನ್‍ಕರೇಜ್‍ಮೆಂಟ್‍ {\rm (Encouragement)} ಕೊಟ್ಟು ಕಾವ್ಯರಚನೆ ಮಾಡಿಸಿ ಪ್ರಸಿದ್ಧಿ ಪಡಿಸಲು ಕಾರಣವೇನು? ಎಂದರೆ ಅವರ ಕಡೆಯಿಂದ ಬಂದ ಶ್ಲೋಕ 


\begin{center} 

{\bf ಶ್ಲೋಕ ಏವ ತ್ವಯಾ ಬದ್ಧಃ ನಾತ್ರ ಕಾರ್ಯಾ ವಿಚಾರಣಾ|\\ 

ಮಚ್ಛಂದಾದೇವ ತೇ ಬ್ರಹ್ಮನ್‍ ಪ್ರವೃತ್ತೇಯಂ ಸರಸ್ವತೀ||\\ 

ರಾಮಸ್ಯ ಚರಿತಂ ಸರ್ವಂ ಕುರು ತ್ವಮೃಷಿಸತ್ತಮ|\\ 

ಧರ್ಮಾತ್ಮನೋ ಗುಣವತಃ ಲೋಕೇ ರಾಮಸ್ಯ ಧೀಮತಃ||\\ 

ವೃತ್ತಂ ಕಥಯ ವೀರಸ್ಯ ಯಥಾ ತೇ ನಾರದಾಚ್ಘುತಮ್‍||} 

\end{center} 


ಎಂಬುದಾಗಿದೆ. ನಾರದನ ಒಂದು ಅಭಿಪ್ರಾಯವನ್ನು ತೆಗೆದುಕೊಂಡು ಧರ್ಮ ಪ್ರಭುವಾದ ಆ ರಾಮನ ಚರಿತೆಯನ್ನು ರಚನೆ ಮಾಡು; `ಮಚ್ಛಂದಾದೇವ ತೇ ಬ್ರಹ್ಮನ್‍ ಪ್ರವೃತ್ತೇಯಂ ಸರಸ್ವತೀ'- ಎಂದು ತನ್ನದನ್ನೂ ತೆಗೆದುಕೊಳ್ಳುವಂತೆ ಹೇಳಿ, `ತತ್ರೈವ ಅಂತರಧೀಯತ'- ಅಲ್ಲಿಯೇ ಅಂತರ್ಧಾನರಾದರು. `ಭೂಮಿಯಲ್ಲಿ ಬೆಟ್ಟ-ನದಿ ಇವೆಲ್ಲ ಇರುವವರೆಗೂ ನಿನ್ನ ಕಾವ್ಯ ಇರುತ್ತದೆ' ಎಂದು ಬ್ರಹ್ಮ ವರ ಕೊಡುತ್ತಾನೆ. ಒಂದು ಪಕ್ಷ ಅಂಬೇಡ್ಕರ್‍ ಅವರು ಮನುಸ್ಮೃತಿಯನ್ನು ಸುಡಿಸಿದಂತೆ - ದ್ರಾವಿಡ ಕಳಗಂದವರು - ಗೀಮಾಯಣದವರು ರಾಮಾಯಣವನ್ನು ಸುಡಿಸಿಬಿಟ್ಟರೆ ಆ ವರ ವಿಫಲವಾಗುತ್ತದೆಯಲ್ಲವೇ? ಎಂದರೆ ಪಂಚಾಂಗ ಹೋದರೆ ನಕ್ಷತ್ರ ಹೋಯಿತೇ. ಧರ್ಮವು ನಿಶ್ಚಲವೂ, ನಿತ್ಯವೂ, ಅಮರವೂ, ಅವಿನಾಶಿಯೂ ಆಗಿರುವುದರಿಂದ, ಜಗತ್ತಿನ ಸೃಷ್ಟಿ ಎಲ್ಲಿದೆಯೋ, ಅಲ್ಲಿ ``ಧರ್ಮೋ ವಿಶ್ವಸ್ಯ ಜಗತಃ ಪ್ರತಿಷ್ಠಾ," ಎಂದು ಹೇಳಲ್ಪಟ್ಟಿರುವ ಧರ್ಮವಿರುತ್ತೆ. ಧರ್ಮಾತ್ಮನ ಅವತಾರದಲ್ಲಿ ಆ ಧರ್ಮ ರಹಸ್ಯವಾಗಿ ಅಡಗಿರುತ್ತೆ, ದಾರು(ಮರ)ವಿನಲ್ಲಿ ಅಗ್ನಿ ಅಡಗಿರುವಂತೆ. 


{\bf ನವರಸನಾಯಕ ಶ್ರೀರಾಮ} 


ಇಂತಹ ಅಮರ ಕಾವ್ಯವನ್ನು ನೋಡಿ ರಸಾಸ್ವಾದನೆ ಮಾಡುವವರೂ ಉಂಟು. ಸಾಹಿತ್ಯದಲ್ಲಿ ನವರಸಗಳೂ ಉಂಟು. ಅವುಗಳೆಲ್ಲವೂ ಶಾಂತದಲ್ಲಿ ತೆಗೆದುಕೊಂಡು ಹೋಗಿ ನಿಲ್ಲಿಸಬೇಕು. ಷಡ್ರಸಗಳಲ್ಲಿರುವ ಸವಿಯನ್ನು ಸವಿಯುವಂತೆಯೇ ಸಾಹಿತ್ಯದಲ್ಲಿ ಶೃಂಗಾರ, ಕರುಣ, ವೀರ, ಹಾಸ್ಯ ಮುಂತಾದ ರಸಗಳನ್ನು ಸವಿಯಬಹುದು. ಶಾಂತರಸಕ್ಕನುಗುಣವಾಗಿರಬೇಕು ಹಾಸ್ಯ. ಕಾಲೇಜಿನ ಹುಡುಗರು ``ಏ ಕೋಳಿಕಳ್ಳ" ಎಂದು ಯಾರನ್ನಾದರೂ ಗೇಲಿಮಾಡುವ ಹಾಸ್ಯವಲ್ಲ. ಕಡವಡ್ಡನ (ಬುಡು-ಬುಡುಕೆಯವನ) ಮೇಲೆ ಪೆಟ್ಲು ಕೊಳವೆಯಿಂದ ಬೀಜವನ್ನು ಹೊಡೆದು, ಅವನ ನೋವನ್ನು ನೋಡುವ ಹುಡುಗರ ಶೌರ್ಯಪ್ರಕಟಣೆ ವೀರರಸವಲ್ಲ. ಮಹಾತ್ಮನ ಪವಿತ್ರವಾದ ಚರಿತ್ರೆಯಲ್ಲಿ - ಜೀವನದಲ್ಲಿ ಇರುವ ರಸಗಳು ಹೇಗಿರುತ್ತವೆ? ಎಂಬುದನ್ನು ಅನುಭವಿಸಬೇಕು. ಅವನ ಚರಿತ್ರೆ ಚಿಂತನೆಗಳನ್ನು ನಾವು ಮಾಡುತ್ತಿದ್ದರೆ, ಆ ಆಕಾರದಲ್ಲಿ ನಾವು ತಾದಾತ್ಮ್ಯ ಹೊಂದುತ್ತೇವೆ. ಅಳುವವನನ್ನು ನೋಡಿದರೆ ನಮಗೂ ಅಳು ಬರುತ್ತದೆ. ಭಗವಂತನು ರಸವನ್ನು ಸೃಷ್ಟಿಮಾಡಿ ಅದಕ್ಕನುಗುಣವಾದ ನಾಯಕನೂ ತಾನೇ ಆದ ಪಕ್ಷದಲ್ಲಿ ಹೇಗಿರುತ್ತದೆ? ಆ ರಸಕ್ಕೆ ನಾವು ಹತ್ತಿ ಅದನ್ನು ತೆಗೆದುಕೊಳ್ಳಬೇಕಾದ ಪಕ್ಷದಲ್ಲಿ (ತಂಬೂರಿಯನ್ನು ತೆಗೆದುಕೊಂಡು) ಹೇಗೆ? 


\begin{center} 

{\bf ಶೃಂಗಾರಂ ಕ್ಷಿತಿನಂದನಾವಿಹರಣೇ ವೀರಂ ಧನುರ್ಭಂಜನೇ\\ 

ಕಾರುಣ್ಯಂ ಬಲಿಭುಙ್ಮುಖೇ ಅದ್ಭುತರಸಂ ಸಿಂಧೌ ಗಿರಿಸ್ಥಾಪನೇ|\\ 

ಹಾಸ್ಯಂ ಶೂರ್ಪಣಖಾಮುಖೇ ಭಯಮಘೇ ಬೀಭತ್ಸಮಾಜೇರ್ಮುಖೇ\\ 

ರೌದ್ರಂ ರಾವಣಭಂಜನೇ ಮುನಿಜನೇ ಶಾನ್ತಂ ವಪುಃ ಪಾತು ನಃ||} 

\end{center} 


(ಹಾಡಿದರು) ಏಕೆ ಹೇಳಿದೆ? ನವರಸಗಳು ಎಲ್ಲ ಮನುಷ್ಯರಲ್ಲೂ ಇವೆ. ಆದರೆ ಅವು ಸಾತ್ತ್ವಿಕ-ರಾಜಸ-ತಾಮಸ ಪ್ರಕೃತಿಭೇದೇನ ಬೇರೆ ಬೇರೆಯಾಗಿರುತ್ತವೆ. ಆದರೆ ಸಾತ್ತ್ವಿಕರಸದಿಂದ ಕೂಡಿದ ರಾಮನಲ್ಲಿ ಏನು ರಸವಿತ್ತು? ಆ ರಾವಣನಲ್ಲಿ ಏನು ರಸವಿತ್ತು? ಸಂಚಾರೀ ವ್ಯಭಿಚಾರಿಗಳೇನಿತ್ತು? ಒಂಭತ್ತು ರಸಗಳೂ ಅವುಗಳ ಸ್ವರೂಪಕ್ಕನುಗುಣವಾದ ಧ್ವನಿಯನ್ನು ಕೊಟ್ಟು ಕೇಳುವುದಕ್ಕೆ ಮೊದಲು ಮನಸ್ಸಿಗೆ 

ಹೇಗಿದ್ದವು? ಈಗ ಹೇಗಿದೆ? ಈ ಪರಿವರ್ತನೆಯು ಒಳ್ಳೆಯದೇ, ಕೆಟ್ಟದ್ದೇ? ನೋಡಿಕೊಳ್ಳೋಣ. 


{\bf ಸಂಸ್ಕಾರವಿದ್ದರೆ ಆಸ್ವಾದನೆ ಸಾಧ್ಯ} 


ರಾಮಾಯಣದ ಫಲ, ಭೌತಿಕ-ದೈವಿಕ-ಆಧ್ಯಾತ್ಮಿಕ-ಸೌಂದರ್ಯಾನುಭವ. ರಾಮಾಯಣವು ರಸಪ್ರಧಾನವಾದ ಕಾವ್ಯ; ಒಳ್ಳೆಯ ರಸಗಳು ಇವೆ ಇದರಲ್ಲಿ. ಅದನ್ನನುಭವಿಸಲು ನಾವೂ ರಸಿಕರಾಗಬೇಕು. ತಾವು ಸಕ್ಕರೆಯನ್ನು ತಿಂದಿದ್ದೀರಿ; ಕಬ್ಬಿನ ಹಾಲನ್ನು ಕುಡಿದಿದ್ದೀರಿ. ಎಲ್ಲೆಲ್ಲಿ ಹೇಗಿರುತ್ತದೆಯೋ ಯಾವರೀತಿ ಸಿಹಿಯಾಗಿರುತ್ತದೆಯೋ ಇದನ್ನು ಹೇಳುವಾಗ ಆ ಸಂಸ್ಕಾರ ಜಿನುಗುತ್ತದೆ. ಕಬ್ಬಿನ ಹಾಲನ್ನೇ ಕುಡಿಯದವನಿಗೆ ಆ ಸಂಸ್ಕಾರ ಏಳುತ್ತದೆಯೇ? ನೀವು ಟೂರ್‍ಗೆ ಹೋಗಿರುವಾಗ ಕುಡಿದಿರುವುದರಿಂದ ಆ ರಸ ಜಿನುಗಿತು. ಕಬ್ಬಿನ ಹಾಲೇ ಕುಡಿಯದೇ-ನೋಡದೇ ಇರುವವನಿಗೆ ಕಬ್ಬಿನ ಹಾಲೂ ಹಸುವಿನ ಹಾಲಿನಂತೆಯೇ ಇದೆ ಎಂಬ ಭಾವನೆ ಬರಬಹುದು. ಆ ಸಂಸ್ಕಾರವಾದರೂ ಇದ್ದರೆ ರಸ ಜಿನುಗಲು ಸಹಾಯವಾಗುತ್ತದೆ. ಜೀವಿಗಳ ಹೃದಯದಲ್ಲಿ ಅನಾದಿಯಾದ ಆತ್ಮನಲ್ಲಿ ಆ ಸಂಸ್ಕಾರಗಳಿವೆ. ಆ ಸಂಸ್ಕಾರವನ್ನು ಧ್ವನಿ ಏಳಿಸುತ್ತೆ. 

ಅಸ್ಥಾನದಲ್ಲಿಟ್ಟ ಪದಾರ್ಥವು ರಸವನ್ನು ಉಂಟುಮಾಡುವುದಿಲ್ಲ. ಸಕ್ಕರೆಯನ್ನು ನಾಲಗೆಯ ಮೇಲೆ ಹಾಕಿಕೊಂಡರೆ ರಸ ಉಂಟಾಗುತ್ತದೆ, ಅದೂ ತುದಿಯಲ್ಲಿ. ಅದೇ ನಾಲಗೆಯ ಅಡಿಯಲ್ಲಿ ಕೂರಿಸಿದರೆ ರಸವಿಲ್ಲ. ನಾಲಗೆಗೇ ವಿಷಯ ಆದರೂ ಕೆಳಗಡೆ ಇದರ ರಸಜ್ಞಾನವಿಲ್ಲ. ಮೇಲಿನ ಪಾರ್ಟಿನವರೇ ತೆಗೆದುಕೊಳ್ಳಬೇಕು. ಅದನ್ನು ಕೆಳಗಿನ ಪಾರ್ಟಿನವರಿಗೆ ಕೊಟ್ಟರೆ ಆ ಕೆಲಸವಿಲ್ಲ. ಅದನ್ನು ದಿಂಡಾಗಿ ಹೊರುವ ಯೋಗ್ಯತೆ ಇದೆ ಅದಕ್ಕೆ ಅಷ್ಟೇ. ಅದೇನಾದರೂ (ಮೇಲ್ಗಡೆಯದು) ತಿಂದರೆ ಶೇಷ ಪ್ರಸಾದವನ್ನು ತಿನ್ನಲು ಕೆಳಗಿನವರಿಗೆ ಯೋಗ್ಯತೆ ಅಷ್ಟೇ. 


{\bf ರಾಮಾಯಣಪಠನಕ್ಕೆ ಒಂದು ಅಧಿಕಾರಸಂಪತ್ತಿರಬೇಕು} 


ಹಾಗೆಯೇ, ಲವಕುಶರು ರಾಮಾಯಣವನ್ನು ಗಾನಮಾಡಲು ಯೋಗ್ಯಾಧಿಕಾರಿಗಳು. ರಾಮನ ವೀರ್ಯವು ಕೆಡದಂತೆ ಸಂರಕ್ಷಿಸಿದ ಹೆತ್ತ ಮಾತೆ ಸೀತೆ ಇದ್ದಾಳೆ, ಅವಳ ಮಕ್ಕಳು ಅದನ್ನು ಹೇಳುವ ಪ್ರಕೃತಿಯನ್ನು ಹೊಂದಿದ್ದಾರೆ. ಗುರುಕುಲದಲ್ಲಿ ಶಿಕ್ಷಣವಾಗಿದೆ. ಕ್ಷತ್ರವೀರ್ಯವೂ ಇದೆ. ಬ್ರಹ್ಮವೀರ್ಯವೂ ಇದೆ. ಸರ್ವಸಂಪನ್ನರೂ ಆಗಿದ್ದಾರೆ. ರಾಮನ ಪ್ರತಿಬಿಂಬವಾಗಿದ್ದಾರೆ. ಪ್ರತಿನಿಧಿಗಳಾಗಿದ್ದಾರೆ. ಅಂತಹವರ ಕೈಗೆ ಬಿದ್ದ ಗ್ರಂಥ. ಅವರ ಕಂಠ ಹೇಗಿದೆ? ಆ ಗಮಕ ಕಲೆ ಹೇಗಿರಬೇಕು? ರಾಮಾಯಣದ ಶ್ಲೋಕವನ್ನೇ ಇಂಗ್ಲಿಷ್‍ ಅಕ್ಷರದಲ್ಲಿ ಬರೆದುಬಿಡಬೇಕು, ಆಗ 


\begin{center} 

{\bf ಕೂಜಂತಂ ರಾಮರಾಮೇತಿ ಮಧುರಂ ಮಧುರಾಕ್ಷರಮ್‍|\\ 

ಆರುಹ್ಯ ಕವಿತಾಶಾಖಾಂ ವಂದೇ ವಾಲ್ಮೀಕಿಕೋಕಿಲಮ್‍||} 

\end{center} 


ಎಂಬುದನ್ನು ಅಕ್ಷರಜ್ಞಾನವಿದ್ದವರೆಲ್ಲಾ ಓದಿ ರಸಾಸ್ವಾದನೆ ಮಾಡಬಲ್ಲರೇ? ಸರಿಯಾಗಿ ಓದಬಲ್ಲರೇ? ಇಂಗ್ಲಿಷರ ಉಚ್ಚಾರಣೆ ಒಂದು ತರಹ, ಹಿಂದೀಜನರ ಉಚ್ಚಾರಣೆ ಒಂದು ತರಹ; ನಮ್ಮ ಈ ಪ್ರದೇಶದವರು ಓದುವುದು ಒಂದು ತರಹ. ಭ್ರಷ್ಟ ಎಂದರೆ ಬ್ರಷ್ಟ ಎನ್ನುತ್ತಾರೆ. ಭದ್ರವಾಗಿ ಎಂದರೆ ದಬ್ರವಾಗಿ ಎನ್ನುತ್ತಾರೆ. ಗ್ರಾಮೀಣರು ತಮ್ಮ ನಾಲಿಗೆ ಹೊರಡುವುದಕ್ಕೆ ಅನುಗುಣವಾಗಿ ಆ ಸಾಹಿತ್ಯವನ್ನು ತಿರುಗಿಸಿಕೊಂಡು ಬಿಡುತ್ತಾರೆ. ಇನ್ನಾವುದೋ ಚಿಂತೆ, ಯೋಚನೆಗಳು ಮನಸ್ಸನ್ನು ಬಾಧಿಸುತ್ತಿರುವಾಗ `ಕೂಜಂತಂ ರಾಮರಾಮೇತಿ' ಎಂಬುದೂ ಅವಸರವಸರವಾಗಿ ಬರುತ್ತದೆ. ಅವೆಲ್ಲ ಸ್ಟ್ಯಾಂಡರ್ಡ್‍ ಆಗುವುದಿಲ್ಲ. ಪರಸ್ಸಹಸ್ರರೀತಿಯಲ್ಲಿ ರಾಮಾಯಣ ಓದುತ್ತಾರೆ. ಆದರೆ ಸರಿಯಾಗಿ ಓದುವುದು ಹೇಗೆ? ಕೋಕಿಲ ಧರ್ಮವೂ ಅದರಲ್ಲಿರಬೇಕು ವಾಲ್ಮೀಕಿ ಧರ್ಮವೂ ಅದರಲ್ಲಿರಬೇಕು. ವಾಲ್ಮೀಕಿಕೋಕಿಲ ಧರ್ಮವೂ ಇರಬೇಕು. 


{\bf ``ವಾಲ್ಮೀಕಿಗಿರಿಸಂಭೂತಾ" ಶ್ಲೋಕದ ಹಿನ್ನೆಲೆ} 


ಕೋಗಿಲೆಯನ್ನು ಮುಂದೆ ಬಿಟ್ಟು ಹಿಂದೆ ವಾಲ್ಮೀಕಿ ಇದ್ದಾರೆ ಕೋಗಿಲೆಯ ವಿಷಯವನ್ನು ಭಾವಿಸುವಾಗ ವಾಲ್ಮೀಕಿಯ ವಿಷಯವನ್ನು ಮರೆಯಬಾರದು. ವಾಲ್ಮೀಕಿಯನ್ನು ಭಾವಿಸಿ ಕೋಗಿಲೆಯನ್ನು ಮರೆಯಬಾರದು. (ಕೂಜಂತಂ ರಾಮರಾಮೇತಿ ಎಂಬುದನ್ನು ಒಮ್ಮೆ ಹಾಡಿ ತೋರಿಸಿದರು) ಎಲ್ಲಿ? ವಾಲ್ಮೀಕಿಗಳು ಪ್ರಾಚೀನಾಗ್ರದರ್ಭಗಳನ್ನು ಹರಡಿ ಯೋಗಮಾಸ್ಥಿತರಾಗಿದ್ದಾರೆ. ಅಂತರ್ಲಕ್ಷ್ಯ ಬಹಿರ್ದೃಷ್ಟಿಯುಳ್ಳವರಾಗಿದ್ದಾರೆ. ಅವರ ದೃಷ್ಟಿ ನಿಮೇಷೋನ್ಮೇಷವರ್ಜಿತವಾಗಿದೆ. ಹೊರಗಡೆ ಕಣ್ಣನ್ನು ತೆರೆದು ತಮ್ಮನ್ನು ನೋಡುತ್ತಿದ್ದಾರೆಂದು ಶಿಷ್ಯರು ತಿಳಿದಿದ್ದರೂ ಅವರು ಅಂಗೈನೆಲ್ಲಿಯಂತೆ ತಮ್ಮೊಳಗೆ ನೋಡುತ್ತಿರುವುದು ಆ ದಿವ್ಯ ದಂಪತಿಗಳನ್ನು, ಅವರ ವೀರಕಥೆ ರಾಮಾಯಣ, ಮುಂದೆ ಕೋಗಿಲೆಯು ಫಲಭರಿತ, ಪುಷ್ಪಭರಿತ ಆಮ್ರವೃಕ್ಷದ ಮೇಲೆ ಕುಳಿತಿದೆ. ವಸಂತ ಕಾಲದ ಶೀತಲ-ಮಂದಸುರಭಿತ ಗಂಧವಹ ಬೀಸುತ್ತಿದೆ. ಅಂತಹ ಮಧುಮಾಸ ಬಂದದ್ದರಿಂದಲೇ ಹಿಂದೆ ಗಂಟಲು ಕಟ್ಟಿದ್ದ ಕೋಗಿಲೆ ಈಗ ನವೋತ್ಸಾಹದಿಂದ ಹಾಡುತ್ತಿದೆ. ಹಿಂದೆ ವಾಕ್ಸಂಯಮಮಾಡಿದ್ದ ಗಂಟಲು. 

ತಪೋನಿಯಮದಿಂದ ಮೌನವಾಗಿದ್ದ ಮುನಿ ಕೋಗಿಲೆ ಕೂಗುತ್ತಿದೆ. ಕುಹೂ ಎಂದು ಸದ್ದು ಮಾಡುತ್ತಿದೆ. ಆ ಪಿಚ್ಚು ಇರುತ್ತೆ ಅದರಲ್ಲಿ. ಆ ಪಿಚ್ಚು ಬೇಕು. `ಕೂ' ಎಂಬುದು ಕೋಗಿಲೆಯ ಕಡೆಗೆ ತೋರಿಸುತ್ತೆ. ಈ ಕೋಗಿಲೆಯೂ ರಾಮನ ವಿಷಯ ಬಂದಾಗ ಆ ಮಹಾತಾರಕನಾಮವನ್ನು, ಆ ರಾಮನ ಮಹಾಮಂತ್ರವನ್ನು ಜಪಿಸಲು ಅದಕ್ಕೊಂದು ಸ್ಥಿತಿಬೇಕು. ತಾರಕಸ್ಥಾನದಿಂದ ಗಾಂಧಾರ ಗ್ರಾಮದಲ್ಲಿ ಇಂದ್ರಿಯಗಳ ಲಯಮಾಡಿಕೊಂಡು ಬ್ರಹ್ಮರಂಧ್ರದಲ್ಲಿ ಮನಸ್ಸಿಟ್ಟು ``ರಾಮ ರಾಮ" ಎಂದು ಜಪಿಸುತ್ತದೆ. `ಕೂ' ಎಂದು ಕೂಗುವುದಕ್ಕೆ ಎಷ್ಟು ಮಾತ್ರಾಕಾಲವೋ ಅಷ್ಟೇ ಮಾತ್ರಾ ಕಾಲ. `ಕೂಜ' ಅವ್ಯಕ್ತೇ ಶಬ್ದೇ. ಆ ಮೆಲ್ಲುಲಿ ಮಧುರವಾದ ಧ್ವನಿ ಮಹಾಯೋಗಸ್ಥಿತಿಯಲ್ಲಿ ಆ ಮಟ್ಟದಲ್ಲಿರುವುದರಿಂದ. ಆ ಒಂದು ಪದವನ್ನು ಉಚ್ಚರಿಸಬೇಕಾದರೂ ಬಹಳ ಜವಾಬ್ದಾರಿ ಬೇಕು. ನಾಲಗೆಗೆ ಮಾಧುರ್ಯವಿರಬೇಕಾದರೆ ಹೀಗೆ ಬರುತ್ತೆ ಧ್ವನಿ. (ಉಚ್ಚರಿಸಿ ತೋರಿಸಿದರು) `ಮಧುರ' ಹೇಳಬೇಕಾದರೆ ಮೇಲಕ್ಕೆ ಹೀಗೆ ಹೋಗುತ್ತದೆ. ಜೀವನದ ಮಹಾವೃಕ್ಷದಲ್ಲಿ ಆ ವಾಲ್ಮೀಕಿ ಕೋಗಿಲೆ ಅನೇಕ ಸೋಪಾನಗಳನ್ನು ಹತ್ತಿ ಯೋಗಿಯ ಸಪ್ತಚಕ್ರಗಳನ್ನು ದಾಟಿ ಸಮಾಧಿಯಲ್ಲಿ ಕುಳಿತಿರುವಂತೆ ಜೀವನವೃಕ್ಷದ ಇಪ್ಪತ್ತೈದು ಮೆಟ್ಟಿಲುಗಳನ್ನು ದಾಟಿ ಷಡ್ವಿಂಶಃಪರಮಾತ್ಮಾ ಎಂಬ ಇಪ್ಪತ್ತಾರನೆಯ ಪರಮಾತ್ಮನನ್ನು ಸವಿಯುತ್ತದೆ. 


{\bf ತತ್ಪರಾಯಣರಾಗಿ ಪಾರಾಯಣ ಮಾಡಬೇಕು} 


ಕೋಗಿಲೆಯು ಹೇಗೆ ಮಹಾಶಾಖೆಯನ್ನು ಹತ್ತಿ ಗಿರಿಯಿಂದ ಗಿರಿಗೆ ಪ್ರತಿಧ್ವನಿತವಾಗುವಂತೆ ಕೂಗುವುದೋ, ಇಲ್ಲಿ ನಾನು ಕೂಗಿದಾಗ ಆ ಧ್ವನಿಯು ತಂಬೂರಿಯಲ್ಲಿ ಹೇಗೆ ಪ್ರತಿಧ್ವನಿತವಾಗುತ್ತದೆಯೋ ಅಂತಹ ಧ್ವನಿಯಲ್ಲಿ ಅವರು ಕೂಗಿದರು. ಎಲ್ಲರ ಹೃದಯದಲ್ಲಿಯೂ ಪ್ರತಿಧ್ವನಿತವಾಗುವಂತೆ ಕೂಗಿದರು. ಅದರ ಸವಿಯನ್ನುಣಿಸುವವರು ಎಲ್ಲಪ್ಪಾ ಇದ್ದಾರೆ? ಎಷ್ಟು ಅಧಿಕಾರ ಸಂಪನ್ನರಾಗಿದ್ದರು ಕುಶಲವರು? ಸ್ವರಲಕ್ಷಣ ಸಂಪನ್ನರು, ರೂಪ ಸಂಪನ್ನರು, ವೇದವೇದಾಂಗತತ್ತ್ವಜ್ಞರು ಇತ್ಯಾದಿ ಈ ಯೋಗ್ಯತೆ ನಿಮಗೇನಿದೆ? ಫಾರಂ ಭರ್ತಿ ಮಾಡಿಕೊಡಿ ಎಂದರೆ ನಮ್ಮದು ಎಲ್ಲ ಕಾಲಮ್ಮಿನಲ್ಲಿಯೂ ಬ್ಲಾಂಕ್‍. ಆದರೆ ಒಂದನ್ನು ಮಾತ್ರ ಭರ್ತಿಮಾಡಬಹುದು. ಅದನ್ನು ಓದಬೇಕೆಂಬ ಆಸೆಯೆಂದು ಮಾತ್ರ ಆ ಪಟ್ಟಿಯಲ್ಲಿ ಫಿಲಪ್‍ ಮಾಡಲು ಯೋಗ್ಯವಾಗಿದೆ. ನಮ್ಮಲ್ಲಿ ಕೋಗಿಲೆಯಂತೆ ಕಂಠ ಬೇಕು. ಸುಮ್ಮನೆ ಕೋಗಿಲೆಯಂತೆ ಕೂಗಿಬಿಟ್ಟರೆ ಆಗದು; ಜ್ಞಾನಬೇಕು. ಎಲ್ಲಿ ಎತ್ತರಿಸಬೇಕು? ಎಲ್ಲಿ ತಗ್ಗಿಸಬೇಕು? ಗರ್ತದಲ್ಲಿರುವುದನ್ನು ಹೇಳಬೇಕಾದರೆ, ತುಂಬಾ ಹಳ್ಳದಲ್ಲಿ ಗುಳಿ ಎಂಬ ಶಬ್ದವನ್ನು ಅದರ ಧ್ವನಿಯಲ್ಲಿ ಹೇಳಬೇಕು. (`ತುಂಬಾ ಗುಳಿ ಇದೆ' ಎಂಬುದನ್ನು ಉಚ್ಚರಿಸಿ ತೋರಿಸಿದರು) ``ವಾಲ್ಮೀಕಿ ಗಿರಿಸಂಭೂತಾ" ಈ ಮಾತನ್ನು ಅದಕ್ಕೆ ಅನುಗುಣವಾದ ಧ್ವನಿಯಲ್ಲಿ ಉಚ್ಚರಿಸಬೇಕು. ಮೇಲಕ್ಕೆ ಎತ್ತರಿಸಿ ಹೇಳಬೇಕು. ಕೆಳಕ್ಕೆ ಒತ್ತಿ ಹೇಳಿದರೆ- `ಗುಳಿ' ಎಂಬುದನ್ನು ಹೇಳುವಂತೆ ಹೇಳಿದರೆ (ಗಿರಿ ಎಂಬುದನ್ನು ಕೆಳಕ್ಕೆ ಒತ್ತಿ ಹೇಳಿ ತೋರಿಸಿದರು) `ಗುಳಿ' ಎಂದೇ ಅರ್ಥ. ಬೆಟ್ಟದಿಂದ ಹುಟ್ಟಿ ಬರಬೇಕಾದರೆ ನದಿಯು ಎಷ್ಟು ಎತ್ತರದಿಂದ ಹೇಗೆ ಜಳಜಳನೆ ಹರಿದು ಬಂದು ಸಾಗರವನ್ನು ಸೇರುತ್ತದೆಯೋ ಹಾಗೆಯೇ ಅದರ ಜೀವನದಿಯೂ `ರಾಮಸಾಗರ ಗಾಮಿನೀ'- ಬರಬೇಕು. ಅದರ ಭಾವನದಿಯೂ ಜೊತೆಯಲ್ಲಿ ಹರಿದು ಬಂದರೆ ಅದರಲ್ಲಿ ಜೀವಪ್ರಭಾವ ಎದ್ದು ಕಾಣುತ್ತದೆ. ಇಲ್ಲದೆ ಭಾವಶೂನ್ಯವಾಗಿದ್ದರೆ? ವಾಲ್ಮೀಕಿಗಿರಿ ಎಂದರೆ ಏನು ಭಾವ? ಎರಡು ಗುಡ್ಡಗಳ ನಡುವೆ ನಿಂತು ಹಾಡುವುದೇ? ಪರಮೋನ್ನತ ಸ್ಥಾನದಿಂದ-ಆತ್ಮಮೂಲದಿಂದ ಹರಿದು ಬಂದ ರಾಮಾಯಣ ನದಿಯು ಮನುಷ್ಯನನ್ನು ಉನ್ನತವಾದ ಮಟ್ಟಕ್ಕೆ ಕೊಂಡೊಯ್ಯುವುದು-ಮೇಲಕ್ಕೆ ಕೊಂಡೊಯ್ಯುವುದು. ಉನ್ನತ ಮಟ್ಟದಿಂದ ಹರಿದು ಬರುವುದು ನದಿ. ಒಂದು ಮನೋಧರ್ಮಕ್ಕೆ ಸಂಬಂಧಿಸಿದರೆ ಮತ್ತೊಂದು ದೇಹಧರ್ಮಕ್ಕೆ. ಒಂದು ಅಕ್ಷರವನ್ನು ಉಚ್ಚರಿಸಬೇಕಾದರೂ ಪ್ರಾಣವನ್ನು ಕೈಯಲ್ಲಿಟ್ಟು ಕೊಂಡಿರಬೇಕು. ವಾಕ್ಸಂಯಮ, ಇಂದ್ರಿಯ ಸಂಯಮ, ಮನಸ್ಸಂಯಮ, ಆತ್ಮಜಯ, ಎಲ್ಲ ಇಟ್ಟುಕೊಂಡು ಪ್ರತಿಯೊಂದು ಪದಕ್ಕೂ ಉಸಿರುಕೊಟ್ಟು ಓದುವ ಜವಾಬ್ದಾರಿ. ತತ್‍ಪರಾಯಣರಾಗಿ ಪಾರಾಯಣ ಮಾಡಬೇಕು. 


{\bf ರಾಮಾಯಣಮಹಾನದಿಯ ಮೂಲ} 


ತತ್‍ ಎಂದರೆ-`ತತ್ಸವಿತುರ್ವರೇಣ್ಯಂ ಭರ್ಗೊ

ದೇವಸ್ಯ ಧೀಮಹಿ|ಧಿಯೋ ಯೋ ನಃ ಪ್ರಚೋದಯಾತ್‍||' ಎಂದು ಪರಬ್ರಹ್ಮತತ್ತ್ವವನ್ನು ವಿಸ್ತಾರಗೊಳಿಸುವ ಯಾವ ಮಹತ್ತತ್ತ್ವ ಉಂಟೋ ಆ ಗಾಯತ್ರ್ಯಾತ್ಮಕವಾದದ್ದು ರಾಮಯಾಣದ ಪ್ರತಿಪಾದ್ಯ ವಿಷಯ. ವೇದಕ್ಕೆ ವಿರೋಧವಿಲ್ಲದಂತೆ, ಜೀವನದಲ್ಲಿ ವಿರೋಧವಿಲ್ಲದಂತೆ ಅದನ್ನು ಸಂಪಾದಿಸುವ ಯೋಗ್ಯತೆ ಬೇಕು. ವಾಲ್ಮೀಕಿಗಳು ಊರ್ಧ್ವಮೂಲವೂ ಕೆಳಗೊಂಬೆಯೂ ಆದ ಜೀವನ ವೃಕ್ಷವನ್ನು ನೋಡಿದರು. ಆ ವೃಕ್ಷವು ಆತ್ಮಮೂಲವಾಗಿ ಬಂದಿದೆ. ಇಪ್ಪತ್ನಾಲ್ಕು ಪರ್ವ-ಸೋಪಾನಗಳನ್ನು ಹೊಂದಿದೆ, ಎಂದು ಅವುಗಳ ಮೇಲೆ ಹತ್ತಿ ಆ ಜಾಗದವರೆಗೂ ಹೋಗಿ ಜ್ಞಾನವನ್ನು ಪಡೆದು ಆ ಜ್ಞಾನದಿಂದ ಲೋಕಕ್ಕೆ ರಸಭರಿತವಾದ ಒಂದು ಕಾವ್ಯ ರಚನೆ ಮಾಡಿಕೊಟ್ಟರು. ಆ ಕಾವ್ಯದಲ್ಲಿ ಶೃಂಗಾರವೇ ಮೊದಲಾದ ರಸಗಳಿದ್ದರೂ ಅವುಗಳಲ್ಲಿ ಒಂದೊಂದರಿಂದಲೂ ಆತ್ಮನನ್ನೇ ಮುಟ್ಟುತ್ತಾನೆ, ಕಣ್ಣನ್ನು ಮುಟ್ಟಿದರೂ ನನ್ನನ್ನೇ ಮುಟ್ಟುತ್ತಾನೆ ಎಂದು ಫೀಲ್‍ ಮಾಡುವಂತೆ ಜೀವನವನ್ನು ಪಾವನ ಮಾಡಲು ವಾಲ್ಮೀಕಿಗಳು ಆ ರಸವನ್ನು ತುಂಬಿ ಇಟ್ಟಿದ್ದಾರೆ-ಉಂಟು ತಾನೇಪ್ಪಾ? ಸದ್ಯದಲ್ಲಿ ಅದನ್ನು ಹೇಗೆ ಉಚ್ಚರಿಸಬೇಕು ಎಂಬ ಬಗ್ಗೆ ಒಂದು ಜವಾಬ್ದಾರಿ. ``ಈ ರೀತಿ ಹಿಡಿದು ಹೀಗಿರೀಪ್ಪಾ, ಅಲ್ಲಿಂದ ಮುಂದಕ್ಕೆ ಈ ತರಹದಲ್ಲಿ ಮಾಡಿಕೊಳ್ಳೋಣ" ಎನ್ನುವ ರೀತಿಯಲ್ಲಿ ನಮ್ಮ ನಿರ್ಣಯ. ಬಹಳ ಜವಾಬ್ದಾರಿ ಬೇಕು. ಸರ್ಜನ್‍ ಚಿಕಿತ್ಸೆ ನಡೆಸಬೇಕಾದರೆ ಶರೀರಜ್ಞಾನವನ್ನು ತಿಳಿದು ಇಲ್ಲಿ ಏಟು ಹಾಕಿದರೆ ಸೆರಿಬ್ರಂಗೆ ಪೆಟ್ಟಾಗುವುದು, ಇಲ್ಲಿ ಹಾಕಿದರೆ ಆಪ್ಟಿಕ್‍ನರ್ವಿಗೆ ತೊಂದರೆಯಾಗುತ್ತದೆ, ಎಂದು ಅನಾಟಮಿಯ ವಿಜ್ಞಾನದೊಡನೆ, ಅದಕ್ಕೆ ತಕ್ಕಂತೆ ಹಿಡಿತದಿಂದ ಶಸ್ತ್ರವನ್ನು ಪ್ರಯೋಗಿಸುತ್ತಾನೆ. ಅದೇ ಕಟುಕನು ಹಾಗೆ ಮಾಡುವುದಿಲ್ಲ. ಸುಮ್ಮನೆ ತಲೆ ಕತ್ತರಿಸಿದರೆ ಸಾಕು ಅವನಿಗೆ. ಶರೀರ ಶಾಸ್ತ್ರವನ್ನುನುಸರಿಸಿ ಪ್ರಯೋಗಿಸುವುದಾವುದೋ ಅದೇ ಶಸ್ತ್ರ. ಹಾಗೆಯೇ ಮಹಾ ಕವಿಯಾದ ವಾಲ್ಮೀಕಿಗಳ ಕಾವ್ಯಶರೀರದ ರಚನೆ ಹೇಗಿದೆ? ಸೃಷ್ಟಿರಹಸ್ಯವೇನು? ಪ್ರಾಣ ಎಂದರೇನು? ಶಾರೀರನ ಸ್ವರೂಪವೇನು? ಹೇಗೆ ಬೆಳೆದುಬಂದಿದೆ ಇತ್ಯಾದಿ ಜೀವನ ರಹಸ್ಯವನ್ನರಿತು ಕಾವ್ಯವನ್ನು ಪ್ರಯೋಗಿಸುತ್ತಾರೆ. ಅಲ್ಲಿ ಶಸ್ತ್ರ; ಇಲ್ಲಿ ಶಾಸ್ತ್ರ. 


{\bf ಇಂದು ನಡೆಯುತ್ತಿರುವ ರಾಮೋತ್ಸವದ ಪರಿಸ್ಥಿತಿ} 


ಶ್ರೀರಾಮನ ರಹಸ್ಯವನ್ನರಿಯದವನು ಬೇಜವಾಬ್ದಾರಿಯಿಂದ ``ಪಿಬ ರೇ ರಾಮರಸಂ" ಎಂದು ಹೇಳುವಾಗ ತಾನು ಬಾಟಲಿನಲ್ಲಿ ವೈನ್‍ ಕುಡಿದ ರಸವನ್ನು ನೆನೆದು ಅದೇ ಭಾವವನ್ನು ತಾನೇ ಹೇಳುತ್ತಾನೆ ಇಲ್ಲಿ ರಾಮರಸಂ ಎಂದಾಗ ಒಂದು ಜೋಕ್‍ನಿಂದ ಹೇಳುತ್ತಾನೆ. ರಾಮನನ್ನು ಅನುಭವಿಸಿದ ರಸದಿಂದಲ್ಲ. ಹಾಡುಗಾರಿಕೆಯಲ್ಲಿ ಆತನ ಹಾಡು ಇಂಪಾಗಿರಬಹುದು. ಆದರೆ ವಿಷಯದ ದೃಷ್ಟಿಯಿಂದ ಅನೇಕ ದೋಷಗಳಿವೆ. ವಿಷಯ ಸ್ಥಾನಬದ್ಧವಾಗಿ ರಸಬದ್ಧವಾಗಿ ಇರಬೇಕು; ಆಗ ಗಾನ. ಈಗ ರಾಮನವಮಿಯಲ್ಲಿ ಅನೇಕವಾಗಿ ರಾಮಾಯಣ ಪಾರಾಯಣಗಳು, ರಾಮ ಕೀರ್ತನೆಗಳು ನಡೆಯುತ್ತಿರುತ್ತವೆ. ಆದರೆ ಉತ್ಸವದ ಸಂಭ್ರಮವೆಲ್ಲ ರಾಮಾಯಣ ಹಾಗು ಕಲೆಯ ಕೊಲೆಯ ಬಗ್ಗೆ ನಡೆಯುವ ಸಂಭ್ರಮವೆಂದು ಅನ್ನಿಸುತ್ತದೆ. ಹಾಗೆ ಹೇಳಿದರೆ ಅದಕ್ಕೆ ಏರ್ಪಾಟು ಮಾಡಿದವರ ಮನಸ್ಸಿಗೆ ನೋವಾಗಬಹುದಾದರೂ, ವಿಷಯದ ದೃಷ್ಟಿಯಿಂದ ಆ ಬಗ್ಗೆ ಮಾತನಾಡುವಾಗ ನಾವು ಹಾಗೆ ತಾನೇ ಹೇಳಬೇಕು. ವ್ಯಕ್ತಿಯ ಬಗ್ಗೆ ಅಲ್ಲ. ಬ್ಯಾಕ್‍ ಗ್ರೌಂಡ್‍ ಇಲ್ಲದೇ ಇದ್ದರೆ ಬಹಳ ತೊಂದರೆ. ಹೀಗೆ ಮಿಡುಕಾಟವಾಗಬಾರದು. ಎಲ್ಲದರ ಮೇಲೂ ಒಂದು ಸಿಂಹಾವಲೋಕನ ಮಾಡಿ ಅವರ (ವಾಲ್ಮೀಕಿಗಳ-ಕುಶಲವರ) ಯೋಗ್ಯತೆ ಏನು? ಅವರ ಟೈಮ್‍ ಅಂಡ್‍ ಸ್ಪೇಸ್‍ ಯಾವುದು? ನಮ್ಮ ಯೋಗ್ಯತೆ ಏನು? ನಾವು ಇರುವ ಟೈಮ್‍ ಅಂಡ್‍ ಸ್ಪೇಸ್‍ ಯಾವುದು? ಎಂಬುದನ್ನು ತಿಳಿದು, ಅದನ್ನು ಇಲ್ಲಿಗೆ ತರಬೇಕಾದರೆ ಎಷ್ಟು ಜವಾಬ್ದಾರಿ ಇದೆ? ಎಂಬುದನ್ನು ತಿಳಿಯಬೇಕು. ಇಲ್ಲದಿದ್ದರೆ ಗಲಿಬಿಲಿಯಾಗುತ್ತದೆ. ಅದಕ್ಕೆ ಇಷ್ಟು ವಿಷಯವಪ್ಪಾ! 


{\bf ರಾಮಾಯಣ ಪಠನಕ್ಕೆ ಅಧಿಕಾರಿಗಳಾದ ಲವಕುಶರು} 


\begin{center} 

(ಸಾಯಂಕಾಲ ೫ ಘಂಟೆ ವೇಳೆಯಲ್ಲಿ ಮುಂದುವರಿಯಿತು) 

\end{center} 


ಬೆಳಗಿನ ಪಾಠದಲ್ಲಿ ರಾಮಾಯಣದಲ್ಲಿ ಹೊರಗಿನ ಸೌಂದರ್ಯ, ಒಳಗಿನ ಸೌಂದರ್ಯ ಎರಡನ್ನೂ ಸೇರಿಸಿ ಅನುಭವಿಸಬೇಕು, ಕಾವ್ಯವನ್ನು ಗಾನ ಮಾಡಿದವರ ಯೋಗ್ಯತೆಯೇನು? ಆ ಲಕ್ಷಣಗಳೇನು? ಎಂಬುದನ್ನು ತಿಳಿದು ಪಾರಾಯಣ ಮಾಡಬೇಕು ಎಂದು ಹೇಳಿದೆ. ಅವರು ಅಂತಹ ಯೋಗ್ಯತೆಯುಳ್ಳವರು ಎಂಬುದಕ್ಕೆ ಆಧಾರ ಶ್ಲೋಕಗಳು ರಾಮಾಯಣದಲ್ಲಿಯೇ ಇವೆ. 


\begin{center} 

{\bf ಪ್ರಾಪ್ತರಾಜ್ಯಸ್ಯ ರಾಮಸ್ಯ ವಾಲ್ಮೀಕಿರ್ಭಗವಾನೃಷಿಃ|\\ 

ಚಕಾರ ಚರಿತಂ ಕೃತ್ಸ್ನಂ ವಿಚಿತ್ರಪದಮಾತ್ಮವಾನ್‍||\\ 

ಕೃತ್ವಾಪಿ ತನ್ಮಹಾಪ್ರಾಜ್ಞಃ ಸಭವಿಷ್ಯಂ ಸಹೋತ್ತರಮ್‍|\\ 

ಚಿಂತಯಾಮಾಸ ಕೋನ್ವೇತತ್‍ ಪ್ರಯುಂಜೀಯಾದಿತಿ ಪ್ರಭುಃ||\\ 

ತಸ್ಯ ಚಿಂತಯಮಾನಸ್ಯ ಮಹರ್ಷೆ

ರ್ಭಾವಿತಾತ್ಮನಃ|\\ 

ಅಗೃಹ್ಣೀತಾಂ ತತಃ ಪಾದೌ ಮುನಿವೇಷೌ ಕುಶೀಲವೌ||} 

\end{center} 


`ಭಗವಾನ್‍' ಋಷಿಃ `ಭಾವಿತಾತ್ಮಾ, ಮಹಾಪ್ರಾಜ್ಞಃ' ಎಂಬುವು ಆ ಆದಿಕವಿಯ ಯೋಗ್ಯತೆಯನ್ನು ಕುರಿತು ಹೇಳುತ್ತವೆ. ಸ್ವರಸಂಪನ್ನೌ, ಭ್ರಾತರೌ, ಅಶ್ವಿನಾವಿವರೂಪಿಣೌ, ಮೇಧಾವಿನೌ, ವೇದೇಷು ಪರಿನಿಷ್ಠಿತೌ, ಬಿಂಬಾದಿವೋತ್ಥಿತೌ ಬಿಂಬೌ ರಾಮದೇಹಾತ್ತಥಾಪರೌ, ಮಹಾತಪಸ್ವಿನೌ, ಮುನಿವೇಷೌ, ಪಾರ್ಥಿವಲಕ್ಷಣಾನ್ವಿತೌ ಎಂಬಿವು ಕುಶಲವರ ಯೋಗ್ಯತೆಯನ್ನು ಹೇಳುತ್ತವೆ: ಇನ್ನು ಗಾನ `ಪಾಠ್ಯೇ ಗೇಯೇ ಚ ಮಧುರಂ, ಪ್ರಮಾಣೈಸ್ತ್ರಿಭಿಃ ಅನ್ವಿತಂ' ಮಂದ್ರ ಮಧ್ಯಮ ತಾರಗಳೆಂಬ ಮೂರು ಪ್ರಮಾಣಗಳಿಂದ ಕೂಡಿದ್ದು. {\bf ಜಾತಿಭಿಃ ಸಪ್ತಭಿಃ ಬದ್ಧಂ, ತಂತ್ರೀಲಯಸಮನ್ವಿತಮ್‍}. ಇಲ್ಲೂ ತಂತ್ರೀಲಯಸಮನ್ವಿತಮ್‍. ಹಿಂದೆ ಭರದ್ವಾಜರಿಗೆ ಹೇಳುವಾಗಲೂ `ಪಾದಬದ್ಧಃ ಅಕ್ಷರಸಮಃ ತಂತ್ರೀಲಯ ಸಮನ್ವಿತಃ' ತಂತ್ರೀಲಯ ತಾಳಗತಿಗಳಿಗೆ ಹೊಂದಿಕೊಳ್ಳುತ್ತದೆ. `ಅಗಾಯತಾಂ ಮಾರ್ಗವಿಧಾನಸಂಪದಾ' `ದೇಶೀ, ಮಾರ್ಗ' ಎಂದು ಗಾನದಲ್ಲಿ ಎರಡು ವಿಧ. ಜನರ ಮನೋರಂಜನೆಗೆ ಮಾತ್ರ ಅನುಗುಣವಾಗಿರುವುದು ದೇಶೀ. ಆತ್ಮಭಂಜನವಿಲ್ಲದೆ, ಆತ್ಮರಂಜನೆ ಮತ್ತು ಲೋಕ ರಂಜನೆಗಳಿಗೆ ಅನುಗುಣವಾಗಿರುವುದು ಮಾರ್ಗ. ಇದು ದೇವತೆಗಳಿಗೂ ಗಂಧರ್ವರಿಗೂ ಯೋಗ್ಯವಾದುದು. ``ಸ್ವರ್ಗವಾಸಿಗೆ ಇದು" ಎಂದು ಭರತನಾಟ್ಯ ಶಾಸ್ತ್ರ ಹೇಳುತ್ತದೆ. ಲವಕುಶರು ರಾಗ-ತಾಳ-ಲಯ-ಯತಿ-ಮಂದ್ರ-ಮಧ್ಯ-ತಾರ ಈ ಲಕ್ಷಣಗಳೊಡನೆಯೇ ಕಾವ್ಯಗಾನ ಮಾಡಿದರು- 


\begin{center} 

{\bf ರಸೈಃ ಶೃಂಗಾರ-ಕಾರುಣ್ಯ-ಹಾಸ್ಯ-ವೀರ ಭಯಾನಕೈಃ|\\ 

ರೌದ್ರಾದಿಭಿಶ್ಚ ಸಂಯುಕ್ತಂ ಕಾವ್ಯಮೇತದಗಾಯತಾಮ್‍||} 

\end{center} 


ನವರಸಭರಿತವಾಗಿ ಸಪ್ತಸ್ವರಾತ್ಮಕವಾಗಿ ಆ ಕಾವ್ಯವನ್ನು ಗಾಂಧರ್ವತತ್ತ್ವಜ್ಞರೂ ಮೂರ್ಛನಾಸ್ಥಾನಕೋವಿದರೂ ಆಗಿ ಅವರು ಹಾಡಿದರು. 


\begin{center} 

{\bf ಸಹಿತೌ ಮಧುರಂ ರಕ್ತಂ ಕಾವ್ಯಮೇತದಗಾಯತಾಮ್‍|} 

\end{center} 


ಒಟ್ಟಿಗೆ ಒಂದೇ ಧ್ವನಿಯಿಂದ ಮಧುರವಾಗಿ ಹಾಡಿದರು. ರಾಮದೇಹದಲ್ಲೇ ಆಡುವ ಇನ್ನೊಂದು ಆತ್ಮದಂತಿದ್ದಾರೆ ಆ ಯಮಳರು. ಹಾವಿನ ಹೆಜ್ಜೆ ಹಾವು ಬಲ್ಲದು; ಕಳ್ಳನ ಹೆಜ್ಜೆ ಕಳ್ಳನು ಬಲ್ಲ; ಸಾತ್ತ್ವಿಕನ ಹೆಜ್ಜೆ ಸಾತ್ತ್ವಿಕನು ಬಲ್ಲ ಎಂದು ಹೇಳುವಂತೆ ರಾಮನ ಹೆಜ್ಜೆಯನ್ನು ಆತನ ವೀರ್ಯಕ್ಕೆ ಹುಟ್ಟಿದ ಆ ಮಹಾತಪಸ್ವಿ ಕ್ಷತ್ರಿಯರೇ ಬಲ್ಲರು. 


``ವಾಚೋ ವಿಧೇಯಂ ತತ್ಸರ್ವಂ" ಎಲ್ಲಿ, ಪುಸ್ತಕ ನೋಡೋಣ ಒಂದು ಸ್ವಲ್ಪ ಎಂದು ನೋಡುತ್ತಿರುವಾಗ ಹೇಗೋ ಹಾಗೆ ತಾಳ ಬಿಟ್ಟು ಹೋಗುವ ಭಯವಿಲ್ಲ. ವಾಚೋ ವಿಧೇಯವಾಗಿದೆ ಕಾವ್ಯ. ಮರ್ಮವನ್ನೂ ಅರಿತವರು. ಎಲ್ಲಿ ಮಾಡಿದರಪ್ಪಾ ಗಾನವನ್ನು? ಆಫ್ರಿಕಾಕ್ಕೋ, ಅಮೆರಿಕಾಕ್ಕೋ ಕಛೇರಿಗೆ ಹೋಗಲಿಲ್ಲವಲ್ಲ. ಕೀ ಹಿಡಿದು ರಸತೆಗೆದುಕೊಳ್ಳುವ ಯೋಗ್ಯಕ್ಷೇತ್ರವಿರುವ ಜಾಗದಲ್ಲಿ ಹಾಡಿದರು. 


\begin{center} 

{\bf ತೌ ಕದಾಚಿತ್ಸಮೇತಾನಾಮ್‍ ಋಷೀಣಾಂ ಭಾವಿತಾತ್ಮನಾಮ್‍|} 

\end{center} 


ವಾಲ್ಮೀಕಿಯೂ ಭಾವಿತಾತ್ಮರು. ಸಹೃದಯರಲ್ಲಿ ಇಟ್ಟಿದ್ದಾಗಿದೆ. ಉಪಸ್ಥಿತರಿದ್ದ ಋಷಿಗಳು ಏನೇನಿತ್ತೋ ಅದನ್ನೇ ಪ್ರೆಸೆಂಟ್‍ ಮಾಡುತ್ತಾರೆ- ಸಮಿತ್ತು, ಕೃಷ್ಣಾಜಿನ, ಯಜ್ಞಸೂತ್ರ ಇತ್ಯಾದಿ- 


\begin{center} 

{\bf ಹ್ಲಾದಯತ್ಸರ್ವಗಾತ್ರಾಣಿ ಮನಾಂಸಿ ಹೃದಯಾನಿ ಚ|\\ 

ಶ್ರೋತ್ರಾಶ್ರಯಸುಖಂ ಗೇಯಂ ತದ್ಬಭೌ ರಾಜಸಂಸದಿ||\\ 

ಕುಶೀಲವೌ ಚೈವ ಮಹಾತಪಸ್ವಿನೌ|\\ 

ಮಮಾಪಿ ತದ್ಭೂತಿಕರಂ ಪ್ರಚಕ್ಷತೇ||} 

\end{center} 


ಕಥಾನಾಯಕನಿಗೇ ಭೂತಿಕರವಾದ ಕಾವ್ಯ. 


\large{{\bf ರಾಮಾಯಣವನ್ನೋದಲು ಉಪದೇಶ}} 


ಇಂತಹ ವಿಶಿಷ್ಟ ಗುಣ ಸಂಪನ್ನವಾದ ಕಾವ್ಯವನ್ನು ನಾವು ಓದಬೇಕಾದರೆ ನಮಗೆ ಒಂದು ಉಪದೇಶ ಬೇಕು. `ಆ ಕಾಲಕ್ಕೆ ಅವರು ಹಾಗೆ ಉಪದೇಶ ಪಡೆದು ಗಾನಮಾಡಿದರು. ನಮಗೆ ಅದೆಲ್ಲ ಸಾಧ್ಯವೇ? ನಮ್ಮ ಕಾಲಕ್ಕೆ ಏನೋ ಯಥಾಶಕ್ತಿ ಪಾರಾಯಣ ಮಾಡುತ್ತೇವೆ' ಎಂದರೆ ಆ ಯಥಾಶಕ್ತಿಗೆ ಕೊನೆ ಮೊದಲೇ ಇಲ್ಲ. ``ಕೂಜನ್ತಂ" ಪೂರ್ತಿ ಹೇಳದೆ ಯಥಾ ಶಕ್ತಿಯಾಗಿ `ಕೂ' ಎಂದಷ್ಟು ಮಾತ್ರ ಹೇಳಿದರೆ ಮೂವತ್ತೆರಡು ಅಕ್ಷರದ ಶ್ಲೋಕ ಪೂರ್ಣವಾಗುತ್ತದೆಯೆ? ಆದರೆ ಇನ್ನೊಂದರ್ಥದಲ್ಲಿ ಯಥಾಸಕ್ತಿ ಹೇಳಿದರೆ ಸರಿ. ಮೂರು ಮೈಲಿ ನಡೆಯಲಾರದಿದ್ದಾಗ ಒಂದೊಂದು ಮೈಲಿ, ಸ್ವಲ್ಪ ಸ್ವಲ್ಪವಾಗಿ ನಡೆಯುವುದು; ಹೀಗೆ ನಡೆದ ಒಂದೊಂದು ಮೈಲಿ, ಪೂರ್ಣವಾದ ದೂರದಲ್ಲಿ ಒಂದು ಭಾಗವೇ ಆಗಿರುತ್ತದೆ. ಒಂದೆರಡು ಶ್ಲೋಕ ಮಾತ್ರ ಪರಿಚಯ ಮಾಡಿಕೊಟ್ಟರೆ ಉಪದೇಶವಾಗಿ ಬಿಟ್ಟೀತೇ? ಅದರ ಗಾಂಭೀರ್ಯ, ರಸ ಅದರ ತತ್ತ್ವ ಬಂದಂತಾಯಿತೇ? ಎಂದರೆ, ಮಹಾತ್ಮರಿಂದ ಉಪದೇಶವಾಗಿ ಬಂದರೆ ಏಕಾಗಬಾರದು? ಮಹಾತ್ಮರೆಂದರೆ ಯಾರು? ನಿಮಗಿಂತಲೂ ಮೊದಲು ಶ್ಲೋಕವನ್ನು ಉರುಹೊಡೆದಿದ್ದವರು, ಆಮೇಲೆ ನೀವು ಉರು ಹೊಡೆಯಲು ಅದನ್ನು ಹೇಳಿ ಕೊಟ್ಟವರು ಎನ್ನೋಣವೇ?. ಇಷ್ಟೇ ಮಹಾತ್ಮರ ಕಾವ್ಯದ ಸ್ವಭಾವವೇ? ಇಲ್ಲಿ ``ಕೂಜನ್ತಂ" ಎಂಬುದರ ವಿವರಣೆಯನ್ನು ಕೊಟ್ಟಮೇಲೆ ಆ ಶ್ಲೋಕವನ್ನು ಕೊಟ್ಟರೆ ಅಷ್ಟು ವಿವರಣೆಯಿಂದಲೂ ಕೂಡಿಕೊಂಡಿರುತ್ತೆ. `ಉಪದೇಶ' ಎಂದರೇನು? ರಾಮಾಯಣತತ್ತ್ವವು ಸಮೀಪದಲ್ಲಿ ಇರುವಂತೆ ಮಾಡುವುದು. ಇದರಲ್ಲಿ ಅವರ ಹೃದಯದೇಶಕ್ಕೆ, ಮನೋಭೂಮಿಕೆಗೆ, ಆತ್ಮಭೂಮಿಕೆಗೆ ಕೊಂಡೊಯ್ಯುವುದಕ್ಕೆ ಅನುಗುಣವಾದ ಯಾವ ದಾರಿಯುಂಟೋ ಅದೇ ಉಪದೇಶ, ಬ್ರಹ್ಮೋಪದೇಶ. ರಾಮಾಯಣೋಪದೇಶ ಬ್ರಹ್ಮದ ಬಳಿಗೆ-ರಾಮನ ಬಳಿಗೆ ಒಯ್ಯುವುದು. 


\large{{\bf ರಸವಿದ್ದರೆ ಜೀವವಿರುತ್ತದೆ}} 


ನಿಮ್ಮಲ್ಲಿ ನಿನ್ನೆಗಿಂತ ಇಂದು ಬದಲಾವಣೆಯಾಗಿದೆ. ಇನ್ನೂ ಹತ್ತಾರು ಸಲ ಕೇಳಿದಾಗ ಒಂದು ರಸ ಉಂಟಾಗುತ್ತದೆ. ಆ ರಸದಲ್ಲಿ ಒಂದು ಜೀವವಿರುತ್ತದೆ. ನಗುವಿನ ರಸವಿಲ್ಲದಿರುವಾಗ, ಬಾಯಿತೆರೆದು ಹಲ್ಲುಗಳನ್ನು ಬಿಡಿಸಿದರೆ ನಗು ಆಗುತ್ತದೆಯೇ? ಮನಸ್ಸಿಗೆ ಒಂದು ರಸ ಬರಬೇಕು. ಆ ಧರ್ಮ ತೆರೆದಾಗ ಬಾಯಿ ತೆರೆದು ಬರುವ ನಗುವಿನಲ್ಲಿ ಅರ್ಥವಿದೆ. ನಗು ಎಂದರೆ ಸುಮ್ಮನೆ ಬಾಯಿ ಬಿಡಬೇಕು; ಹಲ್ಲು ಕಿರಿಯಬೇಕು ಅಷ್ಟೇ ಅಲ್ಲ. ಹೃದಯ ಭೂಮಿಕೆಯಲ್ಲಿ ರಸಹುಟ್ಟುತ್ತದೆ, ಅದಕ್ಕೆ ಬಾಯಿನ ಚೇಷ್ಟೆ ಪೋಷಕವಾಗಿರುತ್ತದೆ. ಇಲ್ಲದಿದ್ದರೆ ಶೋಷಕ. 


\large{{\bf ಮೂಲದ ಮನೋಧರ್ಮವನ್ನು ಹೊತ್ತು ಪಾರಾಯಣ ಮಾಡಬೇಕು}} 


ನಮಗೆ ಲವಕುಶರ ಹಾಗೆ ಕಂಠವಿಲ್ಲ, ಅವರಂತೆ ನಾವು ತತ್ತ್ವಜ್ಞರಲ್ಲ, `ವೇದೇಷು ಪರಿನಿಷ್ಠಿತೌ' ಆಗಿಲ್ಲ. ಹೀಗಿರುವಾಗ ಜೀವನಕ್ಕೆ ಹೊತ್ತಿಗೆಯಾಗಿ ಹೊತ್ತುಕೊಂಡು ರಾಮಾಯಣ ಪಾರಾಯಣ ಮಾಡುವುದು ಹೇಗೆ? `ನೀವು ಎಂಟು ಹತ್ತು ಶ್ಲೋಕಗಳನ್ನು ವೈಜ್ಞಾನಿಕವಾಗಿ ಕಲೆಯೊಡನೆ ಹಾಡಿ ತೋರಿಸಿದಿರಿ; ಹೀಗೆಯೇ ಹಾಡುತ್ತಿದ್ದರೆ ಒಂಭತ್ತು ದಿನಗಳಲ್ಲಿ ಮುಗಿಯುವುದೆಂತು?, ಎಂದರೆ ಅಷ್ಟು ದಿನದಲ್ಲಿ ಮಾಡಲಾಗದಿದ್ದರೆ ಅರಮನೆಯಿಂದ ಒಂದು ಜೊತೆ ಪಂಚೆ ಬರುವುದಿಲ್ಲವಲ್ಲ, ಕೆಲಸ ಹೋಗುತ್ತಲ್ಲ ಎಂದರೆ ಹೆದರಿಕೆಯ ಮಾತಾಯಿತು. ಅದು ವಿಷಯದ ದೃಷ್ಟಿಯಿಂದ ಆಡಿದ ಮಾತಾಗಲಿಲ್ಲ. ಆದರು ಹಾಗೆಯೇ ಹೇಳಿ ಹೇಳಿ ಪಾಠವಾಗಿ ಬಿಟ್ಟರೆ, ಅಷ್ಟೇ ದಿನದಲ್ಲಿ ಮುಗಿಸಲೂ ಸಾಧ್ಯವಾಗುತ್ತದೆ. ಒಂದೆರೆಡು ದಿನ ಹಿಂದು ಮುಂದಾಗ ಬಹುದಷ್ಟೇ. ಹಿಂದೆ ಒಂದು ತರಹ ಓದುತ್ತಿದ್ದಿರಿ, ಈಗ ಓದಿ; ನೋಡೋಣ. ಈಗ ಸಂಸ್ಕಾರ ಬಂದಿದೆ. ನಮ್ಮಂತೆ ಹೇಳಲಾಗದಿದ್ದರೂ, ಒಳಗಡೆಯ ಸೌಂದರ್ಯ ನಿಮ್ಮ ಮನಸ್ಸಿಗೆ ಬಂದಿದೆ. ರಾಗ ಅನುಕರಿಸಲಾಗದಿದ್ದರೂ ಆ ಮನೋಧರ್ಮ ಬಂದಿದೆ. ಮನಸ್ಸು ಆ ಸಂಸ್ಕಾರ ತಾಳುತ್ತೆ. ಆ ರೂಪವನ್ನು ತಾಳುತ್ತೆ. ತತ್ಪರಾಯಣನಾಗಿ ಪಾರಾಯಣ ಮಾಡಲು ಒಂದು ಮನೋಧರ್ಮ ಉಂಟಾಗಿದೆ, (`ವಾಲ್ಮೀಕಿಗಿರಿಸಂಭೂತಾ' ಹೇಳಿಸಿದರು) ನೀವು ರಾಗದಲ್ಲಿ ಎತ್ತರಿಸಲಾಗದಿದ್ದರೂ `ಗಿರಿ' ಎಂಬಲ್ಲಿ ಮಾತ್ರ ಮನಸ್ಸು ಆ ಪಿಚ್ಚಿಗೆ ಹೋಗಿರುತ್ತದೆ. ಈಗಲೂ ನಿಮ್ಮ ಮನಸ್ಸು ಆ ಸ್ಥಾಯಿಯಲ್ಲಿರುತ್ತೆ. ಕಂಠದಲ್ಲಿ ಅದು ಬರಲಾರದೆ ತಡೆದಿದೆ ಅಷ್ಟೇ. ಆ ಮನೋಧರ್ಮದಲ್ಲಿ ಆರಂಭ ಮಾಡಿಬಿಡೀಪ್ಪಾ; ಅದು ಸರಿಹೋಗುತ್ತೆ ಕಾಲಕ್ರಮದಲ್ಲಿ. 


\large{{\bf ಮೂಲದ ಧ್ವನಿಯನ್ನು ಯಥಾವತ್ತಾಗಿ ಹೊರಪಡಿಸುವಂತಿರಬೇಕು ಪಾರಾಯಣ}} 


\begin{center} 

{\bf ನ ರಾಮಸ್ತಪಸಾ ದೇವಿ ನ ಧನೇನ ನ ವಿಕ್ರಮೈಃ|} 

\end{center} 


(ಹಾಡಿ ತೋರಿಸಿದರು) ಸೀತಾಮಾತೆಯ ವಿಷಯದಲ್ಲಿ ಹಾಗೆ ಹೇಳಬಹುದೇ? ತಪ್ಪು. ಆದರೆ ನಾವು ರಾವಣನು ಹೇಳಿದ್ದನ್ನು ಅನುವಾದ ಮಾಡುತ್ತಿದ್ದೇವೆ, ಅಷ್ಟೇ. ಅದನ್ನು ಆ ರಸದಲ್ಲೇ ಓದಬೇಕು. ನಮಗೆ ಇಷ್ಟವಿಲ್ಲವೆಂದು ಪಾರಾಯಣದ ಮಧ್ಯೇ ಆ ಶ್ಲೋಕವನ್ನು ಬಿಟ್ಟುಬಿಡಲಾಗುವುದಿಲ್ಲವಲ್ಲ. ಇದಕ್ಕೆ ಯಾವ ಧ್ಯೇಯವಿರಬೇಕು? ಯಾವ ಪಿಚ್ಚಿರಬೇಕು? ಯಾವ ಮನೋಧರ್ಮ ಇರಬೇಕು? ಎಂದರೆ ರಾವಣನು ಮೂರು ಲೋಕಗಳನ್ನು ವ್ಯಾಪಿಸಿದ ಕೀರ್ತಿಯುಳ್ಳವನು. ``ಇಂದ್ರಲೋಕವನ್ನು ಜಯಿಸಿದೆ, ಯಮಲೋಕದ ದಿಗ್ವಿಜಯ ಮಾಡಿದೆ" ಇತ್ಯಾದಿ ಕೀರ್ತಿ ಉಳ್ಳವನು. ಆದ್ದರಿಂದ `ಯಶಸಾ' ಎಂದು ಹೇಳುವಾಗ ಅಷ್ಟು ಎತ್ತರಿಸಿದ ಧ್ವನಿ. `ನಿಮ್ಮ ಕೀರ್ತಿ ಬದನವಾಳ್‍ವರೆಗೆ ತಾನೇ?' ಎಂದರೆ `ಇಲ್ಲವಯ್ಯ, ಮದ್ರಾಸು, ಬೆಂಗಳೂರು ಎಲ್ಲಾದರೂ ಹೋಗಿ ಕೇಳಿಕೋ, ನಾನು ಯಾರು ಎಂದು' ಎಂದು ಹೇಳುವಾಗ ಅದಕ್ಕೆ ಒಂದು ಧ್ವನಿ ಬರುತ್ತೆ. ಹೆಚ್ಚಾದ ಆ ಕೀರ್ತಿಗನುಗುಣವಾಗಿ ಹೇಳಬೇಕು. ರಾವಣನೂ ತಪಸ್ವಿ; ಹೌದು. ಆದರೆ ಅವನ ತಪಸ್ಸು ಯಾವ ತರಹದ್ದು ಎಂದು ನೋಡಬೇಕಾಗಿದೆ. ತಪಸ್ಸು, ಕೀರ್ತಿ, ಪರಾಕ್ರಮ, ತೇಜಸ್ಸು ಎಲ್ಲಾ ಇದ್ದರೂ ಒಂದು ತರಹ ನ್ಯೂನತೆ ಅವನಲ್ಲಿದೆ, ಅಷ್ಟೇ. ರಾವಣನೇನೂ ಒಣಜಂಭ ಕೊಚ್ಚಿಕೊಂಡಿಲ್ಲ. ಆಂಜನೇಯನು ರಾಮಭಕ್ತನೇ ಹೌದು. ಆದರೆ ರಾವಣನನ್ನು ನೋಡಿ ದಂಗುಬಡಿದವನಾಗುತ್ತಾನೆ ಸ್ವಲ್ಪ ಹೊತ್ತು. 


\begin{center} 

{\bf ಮನಸಾ ಚಿಂತಯಾಮಾಸ ತೇಜಸಾ ತಸ್ಯ ಮೋಹಿತಃ|\\ 

ಅಹೋ ರಾಕ್ಷಸರಾಜಸ್ಯ ಸರ್ವಲಕ್ಷಣಯುಕ್ತತಾ|\\ 

ಅಹೋ ರೂಪಮಹೋ ಧೈರ್ಯಮಹೋ ಸತ್ವಮಹೋ ದ್ಯುತಿಃ|} 

\end{center} 


ಆ ಪರಾಕ್ರಮವನ್ನು ವರ್ಣಿಸುವಾಗ ಅದೇ ಪಿಚ್ಚಿನಲ್ಲೇ ಗಾನ ಮಾಡಬೇಕು. (ಗಾನ ಮಾಡಿ ತೋರಿಸಿದರು) ಹೀಗೆಯೇ ಮೋಹನದಲ್ಲಿ ತಾಳಹಾಕಿಕೊಂಡು ಆಂಜನೇಯರು ಹಾಡಿದರೇ? ಎಂದರೆ ಹಾಗೆಯೇ ಹೇಳಿಸುತ್ತದೆ. ಕುಳಿತು ಚಳಿ ಬಂದಾಗ ಮೈ ಅಲ್ಲಾಡಿಸುತ್ತಾ ಹೇಳಿಸುತ್ತದೆ. ಅವಸರ ಬಂದಾಗ ಒಂದು ತರಹ ಹೇಳಿಸುತ್ತೆ. ಹಾಗೆಯೇ ರಾಕ್ಷಸರಾಜನ ಆ ಮಹಾಪರಾಕ್ರಮದ ರಸಾನುಭವ ಬಂದಾಗ ಹಾಗೆಯೇ ಹೇಳಿಸುತ್ತದೆ. ರಾವಣನ ಒಂದು ನ್ಯೂನತೆ- 


\begin{center} 

{\bf ಯದ್ಯಧರ್ಮೊ

ನ ಬಲವಾನ್‍ ಸ್ಯಾದಯಂ ರಾಕ್ಷಸೇಶ್ವರಃ|\\ 

ಅಪಿ ತ್ರೈಲೋಕ್ಯರಾಜ್ಯಸ್ಯ ಸಶಕ್ರಸ್ಯಾಪಿ ರಕ್ಷಿತಾ||} 

\end{center} 


ಅಧರ್ಮದ ಚಿತ್ರ ಬಂದಾಗ ಮನಸ್ಸಿಗೆ ಜಿಗುಪ್ಸೆ ಉಂಟಾಗುತ್ತೆ. ಆಗಿನ ಧ್ವನಿ, ರಾಗವೇ ಬೇರೆ. 


\begin{center} 

{\bf ಅಸ್ಯ ಕ್ರೂರೈಃ ನೃಶಂಸೈಶ್ಚ ಕರ್ಮಭಿರ್ಲೊ

ಕಗರ್ಹಿತೈಃ|\\ 

ತೇನ ಬಿಭ್ಯತಿ ಖಲ್ವಸ್ಮಾತ್‍ ಲೋಕಾಸ್ಸಾಮರದಾನವಾಃ||\\ 

ಅಯಂ ಹ್ಯುತ್ಸಹತೇ ಕ್ರುದ್ಧಃ ಕರ್ತುಮೇಕಾರ್ಣವಂ ಜಗತ್‍|} 

\end{center} 


(ಹಾಡಿ ತೋರಿಸಿದರು) ಈಗ ನೀವು ಈ ಶೋಕ್ಲಗಳನ್ನು ಓದಿ ಹಿಂದಿನ ಧಾಟಿಯಲ್ಲಿ ಹೇಳುವುದಕ್ಕೆ ಹೋದರೆ ಚೆಕ್‍ ಮಾಡುತ್ತದೆ. ನನ್ನಂತೆಯೇ ಹೇಳಲಾಗದಿದ್ದರೂ, ಈಗ ಮನಸ್ಸು ಅದಕ್ಕೆ ಪ್ರಿಪೇರ್‍ ಆಗಿದೆ. 


\large{\bfಉತ್ತಮಸಂಸ್ಕಾರ ಹೊಂದಿ ಪಾರಾಯಣ ಮಾಡಿದಾಗಲೇ ಫಲಸಿದ್ಧಿ} 


ಕಡೆಗೆ ದುಷ್ಟಶಕ್ತಿಗಳೆಲ್ಲ ನಿವಾರಣೆಯಾಗಿ ಪರಮಶಾಂತಿ ಉಂಟಾಗಿ, `ಲೋಕಾಸ್ಸಮಸ್ತಾಸ್ಸುಖಿನೋ ಭವನ್ತು' ಎನ್ನುವವರೆಗೆ ಅದಕ್ಕೆ ಸಂಸ್ಕಾರ ಬಂದರೆ ಮನಸ್ಸಿಗೆ ಒಂದು ಪಾಕ ಬರುತ್ತೆ. ಹಾಗೆ ಮಾಡುತ್ತಿದ್ದರೆ ರಾಮಾಯಣದಲ್ಲಿ ಮಧ್ಯೆ ಫಲ ಸಿಗುತ್ತದೆ. ಪಾರಾಯಣಕ್ಕೆ ಒಂದು ಹಿನ್ನೆಲೆ ಬೇಕು. ಹಿಂದಿನ ಸಂಸ್ಕಾರಕ್ಕೂ ಇಂದಿನ ಸಂಸ್ಕಾರಕ್ಕೂ ಎಷ್ಟು ವ್ಯತ್ಯಾಸವಿದೆ; ಮುಂದೆ ಹೇಗೆ ಬೆಳವಣಿಗೆಯಾಗುತ್ತೆ ಎಂಬುದನ್ನು ನೋಡಿ ಜವಾಬ್ದಾರಿ ಕೊಟ್ಟಾಯಿತು. ಅದರ ರಸದಲ್ಲಿ ಮುಳುಗಿ ಪಾರಾಯಣ ಮಾಡಬೇಕು. 


\begin{center} 

{\bf ಹ್ಲಾದಯತ್ಸರ್ವಗಾತ್ರಾಣಿ ಮನಾಂಸಿ ಹೃದಯಾನಿ ಚ|} 

\end{center} 


ಎಂದು ಹೇಳಿರುವಂತೆ ಸರ್ವತೃಪ್ತಿಯಾಗಬೇಕು. `ಹ್ಲಾದಯತ್ಸರ್ವಗಾತ್ರಾಣಿ' ಇಂದ್ರಿಯ ತೃಪ್ತಿಯಾಯಿತು. `ಮನಾಂಸಿ' ಮನಸ್ಸಿಗೆ ತೃಪ್ತಿಯಾಯಿತು. ಇನ್ನು ಹೃದಯ; ಎಲ್ಲಕ್ಕೂ ಪೂರ್ಣತೃಪ್ತಿ. ಆದುದರಿಂದಲೇ `ಮಮಾಪಿ ತದ್ಭೂತಿಕರಂ ಪ್ರಚಕ್ಷತೇ.' ಇಂತಹ ರಾಮಾಯಣಪಾರಾಯಣ ನಡೆಯುವಲ್ಲಿ ರಾಮಭಕ್ತಿ-ಪರಾಯಣರಾದ ಭಕ್ತರು ನೆರೆದು- 


\begin{center} 

{\bf ಲೋಕಾಸ್ಸಮಸ್ತಾಸ್ಸುಖಿನೋ ಭವನ್ತು|} 

\end{center} 


ಎಂದು ಆಶಿಸಿದರೆ, ಏಕೆ ನೆರವೇರುವುದಿಲ್ಲ? ಕೇಳಿ ಕೇಳಿ ಸಂಸ್ಕಾರ ಬಂದು, ಹಾಗೆ ಮಾಡಿಸುತ್ತದೆ. 


\begin{center} 

{\bf ಚಕ್ರವರ್ತಿತನೂಜಾಯ ಸಾರ್ವಭೌಮಾಯ ಮಙ್ಗಳಮ್‍||} 

\end{center} 
