\chapter{ಶ್ರೀರಾಮನವಮೀ} 

(ಹೆಡತಲೆಯ ವಿಜ್ಞಾನಮಂದಿರದ ಆವರಣದಲ್ಲಿ ಬಂದ ಪ್ರಸಂಗವಿದು. ತಾ|| ೭-೪-೧೯೬೮ರಂದು ಚೈತ್ರಶುದ್ಧನವಮೀ ಶ್ರೀರಾಮನವಮಿಯ ದಿನ, ಹಲವಾರು ಮಂದಿ ಸೋದರರು ಶ್ರೀ ಗುರುಸಂದರ್ಶನಕ್ಕಾಗಿ ಬಂದಿದ್ದರು. ಈ ಸಂದರ್ಭದಲ್ಲಿ ಶ್ರೀ ಗುರುವಿನಿಂದ ಹೊಮ್ಮಿದ ಮಾತುಗಳನ್ನು ಇಲ್ಲಿ ಸಂಗ್ರಹಿಸಿಕೊಡಲಾಗಿದೆ. ಆ ಮಾತುಗಳನ್ನು ಆ ಕಾಲದಲ್ಲಿಯೇ ಸಂಗ್ರಹಿಸಿ ಬರಹರೂಪಕ್ಕೆ ತಂದವರು ಶ್ರೀ ಛಾಯಾಪತಿಗಳು.) 

(ಶ್ರೀಗುರುವಿನ ಆಗಮನಕ್ಕಾಗಿ ಕಾತರದಿಂದ ನಿರೀಕ್ಷಿಸುತ್ತಿದ್ದ ಸಭೆಗೆ ಶ್ರೀಗುರುವು ದಯಮಾಡಿಸಿ ಸಜ್ಜಾಗಿದ್ದ ಆಸನದಲ್ಲಿ ಮಂಡಿಸಿ ನಂತರ ಅಂತರ್ಭಾವಮಗ್ನವಾದ ಮನಸ್ಸಿನಿಂದ ಒಂದೆರಡು ನಿಮಿಷ ಕುಳಿತಿದ್ದು ನಂತರ ಅರಳುಗಣ್ಗಳ ನೋಟದಿಂದ, ಸಭೆಯನ್ನು ಅವಲೋಕಿಸಿ ಗಂಭೀರಭಾವದಿಂದ ಆಡಿದ ಮಾತುಗಳು.) 

\section*{ಪರಮಪುರುಷನೇ ರಾಮನಾಗಿ ಅವತರಿಸಿದ}

ಶ್ರೀರಾಮನವಮಿಯ ಈ ಶುಭಸಂದರ್ಭದಲ್ಲಿ ಆತ್ಮಾರಾಮನಾದ ಅವನನ್ನು ಧ್ಯಾನದಿಂದ ಪೂಜಿಸುವ ಮೊದಲು, ರಾಮನ ಬಗೆಗೆ ಅವನಾಗಿ ಕೊಟ್ಟ ಸ್ಫೂರ್ತಿ ಏನುಂಟೋ, ಅದನ್ನು ನಿಮ್ಮ ಮುಂದೆ ಇಟ್ಟು, ನಂತರ ಆತ್ಮಾರಾಮರಾಗಿರಲು ಜಾಗ ಮಾಡಿಕೊಳ್ಳೋಣ. 

`ರಾಮ'-ಎನ್ನುವ ಶಬ್ದವೇ ಯೋಗಿಹೃದಯದಲ್ಲಿ, ಜ್ಞಾನಿಗಳ ಹೃದಯದಲ್ಲಿ ರಮಿಸಲು ಯೋಗ್ಯನಾದವನು ಎಂಬ ಅರ್ಥವನ್ನು ಕೊಡುತ್ತದೆ. `ರಾಮಾ' ಎಂಬ ಹೆಸರನ್ನು ಪಡೆದ ಸ್ತ್ರೀಯು ಪತಿಯೊಡನೆ ಕ್ರೀಡಿಸುವಂತೆ, ಜೀವವು ತನ್ನ ದೇವನೊಡನೆ ಒಂದಾಗಿ ಸೇರಿ, ಪರಮಾತ್ಮನಲ್ಲಿ-ಪರಮಪುರುಷನಲ್ಲಿ ರಮಿಸುವುದಾದರೆ, ಹಾಗೆ ರಮಿಸಲು ಯೋಗ್ಯನಾದ ದೇವನೇ ರಾಮ. ಪರಮಪುರುಷನೇ `ರಾಮ' ನಪ್ಪಾ. 

`ಆತ್ಮಕ್ರೀಡಃ ಆತ್ಮರತಿಃ\label{143} ಕ್ರಿಯಾವಾನೇಷ ಬ್ರಹ್ಮವಿದ್ಯಾಂ ವರಿಷ್ಠಃ'-ಎಂಬಂತೆ ಆತ್ಮನಲ್ಲಿ ಕ್ರೀಡಿಸುವವನಾಗಬೇಕು. ಪರಮವೈಕುಂಠದಲ್ಲಿ ಶಾಂತಿಧಾಮದಲ್ಲಿದ್ದ ಪರಮಪುರುಷನೇ ದಿವಿಯಲ್ಲಿದ್ದವನು, ಲೋಕಕಲ್ಯಾಣಕ್ಕಾಗಿ ಭೂಮಿಗೆ ಅವತರಿಸಿದನಪ್ಪಾ. 

\section*{ಧರ್ಮವನ್ನು ನೆಲೆಗೊಳಿಸಲು ಸೀತಾರಾಮರ ಅವತಾರ}

`ಅವತರಿಸುವುದು'-ಎಂದರೆ ಇಳಿಯುವುದು. ಮೇಲೆ ಅಟ್ಟ ಅಥವಾ ಮಹಡಿಯಲ್ಲಿ ಸುಖವಾಗಿದ್ದವನು ಹೇಗೆ ಕೆಳಗಿದ್ದವರಿಗೋಸ್ಕರವಾಗಿ ಏಣಿ ಅಥವಾ ಮೆಟ್ಟಲು ಮೂಲಕ ಇಳಿದು ಬರುವುದುಂಟೋ, ಹಾಗೆ ಇಳಿದು ಬರುವುದೇ ಅವತಾರವೆನಿಸುವುದು. ಏಕಪ್ಪಾ ಈ ಅವತಾರ? ಎಂದರೆ ಭಗವಂತನು ತನ್ನ ಪಾಡಿಗೆ ತಾನು ಸುಖವಾಗಿಯೇ ಇದ್ದ. ಆದರೆ ಲೋಕವನ್ನು ದುಃಖದಿಂದ ಪಾರುಮಾಡಲು, ಈ ಲೋಕವನ್ನು ತನ್ನ ಸುಖದ ನೆಲೆಯಲ್ಲಿಯೇ ನಿಲ್ಲಿಸಲು ಅವತರಿಸಿದ. 

ಸಂಗೀತವನ್ನು ತಾರಶ್ರುತಿಯಲ್ಲಿ ಹೇಳಲು ಆಗದಿದ್ದಾಗ, ಅದನ್ನೇ ಮಂದ್ರಸ್ಥಾಯಿಗೆ ತಂದು ಹೇಳುವುದೂ ಉಂಟು. ಮೇಲಿನ ಷಡ್ಜವನ್ನು ಕೆಳಕ್ಕಿಳಿಸಿದಾಗ, ಷಡ್ಜದ ಅವತಾರವಾಯಿತು ಎನ್ನಬಹುದು. ಏತಕ್ಕೆ ಇಳಿಯಿತು ಅದು? ಷಡ್ಜದ ಧರ್ಮವನ್ನು ಪ್ರಕಾಶಪಡಿಸಲು, ಹಾಗೆ ಕೆಳಕ್ಕೆ ಇಳಿದರೂ, ಷಡ್ಜಸ್ವರವು ತನ್ನ ಗುಣಧರ್ಮಗಳನ್ನು ಹಾಗೆಯೇ ಇಟ್ಟುಕೊಂಡಿದೆ. 

ಅಂತೆಯೇ ಭಗವಂತನು ಅವತಾರವೆತ್ತಿದ ಎಂದರೆ ಇಳಿದರೂ ತನ್ನ ಧರ್ಮವನ್ನು ಮರೆಯದೇ ಅದಕ್ಕೆ ಧಕ್ಕೆಯಿಲ್ಲದೇ ಇಳಿದು ಬಂದರೆ ಅವತಾರವಾಗುತ್ತೆ. `ಏಕಪ್ಪಾ ಅವತಾರಮಾಡಿದ?' ಪರಮವೈಕುಂಠದಲ್ಲಿರುವ ಭಗವಂತನು ಇಳಿದುದು ಭೂಮಿಯನ್ನು ವೈಕುಂಠವನ್ನಾಗಿಮಾಡಲು. ಏನು ಕುಂಠಿತವಾಗಿತ್ತು? ಆಸುರೀಪ್ರವೃತ್ತಿಯಿಂದ ಧರ್ಮವು ಕುಂಠಿತವಾಗಿತ್ತು. ಆಸುರೀಪ್ರವೃತ್ತಿಯನ್ನು ಕಳೆದು ಧರ್ಮವನ್ನು ನೆಲೆಗೊಳಿಸಿ ಭೂವೈಕುಂಠವನ್ನಾಗಿಮಾಡಲು ಲಕ್ಷ್ಮೀನಾರಾಯಣರು ಸೀತಾರಾಮರಾಗಿ ಅವತರಿಸಿದರಪ್ಪಾ. ಅಂತಹ ಶ್ರೀರಾಮನ ಜನ್ಮದಿನ. ಅವತಾರದ ದಿನವಿದಾಗಿದೆಯಪ್ಪಾ. 

\section*{ಲೋಕವೆಲ್ಲ ಭಗವನ್ಮಯವಾಗಿ ಬಾಳಲೆಂದು ಮಹರ್ಷಿಗಳು ಹಬ್ಬಹರಿದಿನಗಳನ್ನು ತಂದರು.} 

ಜೀವನವು ಕ್ಷಣಚಿತ್ತ-ಕ್ಷಣಪಿತ್ತ ಎನ್ನುವಂತಿದೆ. ಕ್ಷಣಕಾಲ ಸಮಾಧಾನವಿದ್ದರೆ, ಇನ್ನುಕ್ಷಣಕಾಲದಲ್ಲಿ ಆಗಲೇ ಪಿತ್ತವೇರಿರುತ್ತದೆ. ಜೀವನವು ಹೀಗಿರುವುದರಿಂದ ಮಹರ್ಷಿಗಳು ಇಂತಹ ಜೀವನವನ್ನು ಭಗವಂತನಿಗೆ ಅಭಿಮುಖವನ್ನಾಗಿಮಾಡಿ ಭಗವನ್ಮಯವಾಗಿ ಬಾಳಲನುಗುಣವಾಗಿ ಹಬ್ಬ-ಹರಿದಿನಗಳನ್ನು ತಂದು ಕೊಟ್ಟರಪ್ಪಾ. ಪ್ರತಿ-ತಿಥಿಯಲ್ಲಿಯೂ ಒಂದೊಂದು ಹಬ್ಬ. ಯುಗಾದಿ ಬಂದರೆ ಯುಗಾದಿಪುರುಷನ ನೆನಪು. ನಂತರ ಶ್ರೀರಾಮನವಮೀ. ಹೀಗೆಯೇ ಅಕ್ಷಯತದಿಗೆ, ವಿನಾಯಕನ ಚೌತಿ, ಗರುಡಪಂಚಮೀ, ಸುಬ್ರಹ್ಮಣ್ಯಷಷ್ಠೀ, ರಥಸಪ್ತಮೀ, ಗೋಕುಲಾಷ್ಟಮೀ, ವಿಜಯದಶಮೀ, ಪರಮಪದೈಕಾದಶೀ, ಹೀಗೆ ಯಾವ ದಿನ ನೋಡಿದರೂ ಅಲ್ಲಿ ಭಗವಂತನ ಗುರುತುಗಳನ್ನಿಟ್ಟು ಭಗವನ್ಮಯವಾಗಿ ಬಾಳಲನುಗುಣವಾಗಿ ಬಾಳಾಟವನ್ನು ರೂಪಿಸಿಕೊಟ್ಟರಪ್ಪಾ. ಜೀವನದ ಹಿಂದಿರುವ ಒಳ್ಳೆಯ ಜೀವನವನ್ನು ಜ್ಞಾಪಿಸುವುದಕ್ಕೋಸ್ಕರವಾಗಿ ಮಾಡಿರುವ ವಿಷಯವುಂಟು. ಇದರ ಜೊತೆಗೆ ಇಂದಿನ ದಿನ ಹರಿಪದಪುಣ್ಯಕಾಲ, ಇಂದು ವಿಷ್ಣುಪದಪುಣ್ಯಕಾಲ, ಹೀಗೆ ಪುಣ್ಯಕಾಲಗಳನ್ನೂ ಕೊಟ್ಟು ಇಡೀ ವರ್ಷವನ್ನು ಭಗವನ್ಮಯವಾಗಿ ಭಗವಂತನ ಸವಿನೆನಪಿನೊಡನೆ ಬಾಳಲು ದಾರಿಮಾಡಿಕೊಟ್ಟರು. 

\section*{ಋಷಿಸಂಸ್ಕೃತಿಯಲ್ಲಿ ಎಲ್ಲೆಲ್ಲೂ ಭಗವಂತನ ಮುದ್ರೆಗಳು}

ಇಡೀ ಜೀವನದ ಎಲ್ಲ ವ್ಯವಹಾರಗಳ ಮೇಲೂ ಭಗವಂತನ ಮುದ್ರೆಯನ್ನು ಒತ್ತಿದ್ದಾರೆ. ಒಂದು ಮಕ್ಕಳು ಮರಿಗಳನ್ನು ಎತ್ತಿ ಆಡಿಸಲಿ, ಅವಕ್ಕೆ ಹೆಸರಿಡಲಿ, ಅವುಗಳನ್ನು ತೊಟ್ಟಲಿಗೆ ಹಾಕಿ ತೂಗಲಿ, ಅಲ್ಲಿ ಭಗವಂತನನ್ನೇ ಆಡಿಸುತ್ತಾ ಅವನ ಪದವನ್ನು ಜೀವನದುದ್ದಕ್ಕೂ ಹರಿಸಿರುವುದುಂಟು. ತಾಯಿಯು ಮಗುವನ್ನು ತೂಗುವಾಗ ಹೇಳುವ ಹಾಡಾದರೂ ಏನು? ``ಜಗದೋದ್ಧಾರನ, ಪರಮಪುರುಷನ, ಅಪ್ರಮೇಯನ ಮಗನೆಂದು ತಿಳಿಯುತ ಆಡಿಸಿದಳೆಶೋದಾ" ಎಂದು. 

\section*{ಲೋಕದಲ್ಲಿ ಭಾವನೆಗನುಸಾರವಾದ ಸಿದ್ಧಿಯುಂಟು}

ತೊಟ್ಟಿಲಲ್ಲಿಕ್ಕುವಾಗ ಮೊದಲು ಕಲ್ಲುಗುಂಡನ್ನಿಟ್ಟು, ಅದನ್ನು ಭಾವಿಸಿ, ಸಜೀವವಾದ ಮಗುವಿನ ದರ್ಜೆಯಲ್ಲಿಟ್ಟು ಅದಕ್ಕೆ ಶೀತವಾಗುತ್ತೆ, ಉಷ್ಣವಾಗುತ್ತೆ, ಎಂದು ಅದನ್ನು ಎಚ್ಚರಿಕೆಯಿಂದ ಜೋಪಾನಮಾಡುವುದುಂಟು. ಅದರಲ್ಲಿ (ಗುಂಡುಕಲ್ಲಿನಲ್ಲಿ) ಮಗುವನ್ನು ಭಾವಿಸುವ ಯೋಗ್ಯತೆಯಿದೆಯೇ ಎಂದು ರಿಹರ್ಸಲ್‍ ({\eng Rehearsal}) ಮಾಡಿ ನಂತರ ನಿಜವಾದ ಮಗುವನ್ನು ಆಡಿಸುವುದುಂಟು. 

ಮಗುವೊಂದು ತಾನು ಆಡುವಾಗ ಕಲ್ಲುಚೂರನ್ನೋ, ಹೆಂಚಿನ ಬಕರೆಯ ಚೂರನ್ನೋ ಇಟ್ಟುಕೊಂಡು ಅದರಲ್ಲೇ ತಾಯ್ತನದ ಪ್ರೀತಿಯನ್ನೆಲ್ಲಾ ತೋರಿಸಿ ಬಿಡುತ್ತದೆ. `ಕಲ್ಲುಚೂರು ತಂದುಹಾಕುತ್ತೀಯಾ? ಕಸ' ಎಂದು ಎಸೆದರೆ ಹೃದಯವೇ ಕಿತ್ತಂತೆ ಅಳುತ್ತವೆ. ನಮಗೆ ಕಸ, ಮಕ್ಕಳಿಗೆ ರಸವಾಗಿದೆ ಅದು. ತನ್ನ ಹೃದಯದ ಭಾವವನ್ನು ಕಲ್ಲಿನಲ್ಲಿ ತುಂಬಿದೆ ಆ ಮಗು. ಹೀಗೆ ಕಲ್ಲನ್ನೇ ಅಷ್ಟು ನೋಯಿಸದೇ ಕಾಪಾಡುವ ಹೃದಯವಿದ್ದರೆ, ಇನ್ನು ನಿಜವಾಗಿ ಮಕ್ಕಳನ್ನು ಎಷ್ಟು ಚೆನ್ನಾಗಿ ಭಾವಿಸಬಲ್ಲದು. ಭಾವನೆಯಿಂದ ಅದೇ ರೂಪವನ್ನು ತಾಳುತ್ತದೆ. ಭಾವನೆಗನುಸಾರವಾದ ಸಿದ್ಧಿಯೂ ಉಂಟು. 

\section*{ನರರೂಪಿಯಾದ ನಾರಾಯಣನನ್ನು ಅರಿತು ಅನುಸರಿಸಿದರೆ ಎಲ್ಲವೂ ಭಗವನ್ಮಯ} 

ನರನ ರೂಪದಲ್ಲಿ ಬಂದರೂ ನಾರಾಯಣನು ತನ್ನತನದೊಡನೆಯೇ ಬರುತ್ತಾನೆ. ಆದ್ದರಿಂದ ನರರೂಪದಲ್ಲಿದ್ದರೂ ಅವನು ನಾರಾಯಣನೇ. 

\begin{shloka}
ದೇವಾ ಮಾನುಷರೂಪೇಣ ಚರಂತ್ಯೇತೇ ಮಹೀತಲೇ|\label{146b}
\end{shloka} 

(ದೇವತೆಗಳೇ ಭೂಮಿಯಲ್ಲಿ ಮನುಷ್ಯರೂಪದಿಂದ ಸಂಚರಿಸುತ್ತಾರೆ) ಎಂಬಂತೆ, ಆಸುರೀಸಂಪತ್ತನ್ನು ಮೆಟ್ಟಲು ಅವನು ಹಾಗೆ ಬರಲೇ ಬೇಕಾಗುತ್ತದೆ. ಈ ಗುಟ್ಟನ್ನರಿತು ಜೀವನವನ್ನಳವಡಿಸಿಕೊಂಡಾಗ, ನರನ ಚೇಷ್ಟೆಯು ನಾರಯಣನಿಗಭಿಮುಖವಾಗಿಯೇ ಇರುತ್ತದೆ. 

\section*{ಭಗವನ್ಮಯವಾಗಿ ಬದುಕಲು ಋಷಿಗಳು ತಂದ ಜೀವನವ್ಯವಸ್ಥೆ}

ಲೋಕದಲ್ಲಿ ಮದುವೆಯಲ್ಲಾಗಲೀ ಇತರ ಆರತಕ್ಷತೆಗಳಲ್ಲಾಗಲೀ ಭಗವಂತನ ಹೆಸರಿನಲ್ಲಿಯೇ ಎಲ್ಲವೂ ಜರುಗುತ್ತದೆ. ಭಗವಂತನ ಉತ್ಸವವನ್ನಾಚರಿಸುವ ರೀತಿಯಿರುತ್ತದೆ. ಮತ್ತೇನೂ ಬೇಡ. ಉದಯದಿಂದ ಅಸ್ತಮಾನದವರೆಗೆ ನಡೆಯುವ ವ್ಯವಹಾರಗಳನ್ನು ಗಮನಿಸಿದರೂ ಸಾಕು. ಬೆಳಿಗ್ಗೆ ಎದ್ದು- 

\begin{shloka} 
ಉತ್ತಿಷ್ಠೋತ್ತಿಷ್ಠ ಗೋವಿಂದ ಉತ್ತಿಷ್ಠ ಗರುಡಧ್ವಜ|\label{146}\\ 
ಉತ್ತಿಷ್ಠ ಕಮಲಾಕಾಂತ ತ್ರೈಲೋಕ್ಯಂ ಮಂಗಲಂ ಕುರು||
\end{shloka} 

(ಗೋವಿಂದನೇ ಏಳು, ಏಳು, ಗರುಡಧ್ವಜನೇ ಮೇಲೇಳು. ಕಮಲೆಯ ಕಾಂತನೇ ಏಳು, ಮೂರುಲೋಕಗಳಿಗೂ ಮಂಗಲವನ್ನು ಮಾಡು) ಎಂದು ಮೂರು ಲೋಕಗಳಿಗೂ ಅವನ ಮಂಗಲಸ್ಮರಣೆಯೊಡಗೂಡಿದ ಮಾಂಗಲ್ಯವನ್ನು ಕೋರುವಿಕೆ ಎಷ್ಟು ಹೃದಯಂಗಮವಾಗಿದೆ! 

\begin{shloka} 
ಪ್ರಾತಃ ಸ್ಮರಾಮಿ ಹೃದಿ ಸಂಸ್ಫುರದಾತ್ಮತತ್ತ್ವಂ\label{146a}\\ 
ಸಚ್ಚಿತ್ಸುಖಂ ಪರಮಹಂಸಗತಿಂ ತುರೀಯಮ್‍|\\ 
ಯತ್ಸ್ವಪ್ನಜಾಗರಸುಷುಪ್ತಿಮವೈತಿ ನಿತ್ಯಂ\\ 
ತದ್ಬ್ರಹ್ಮ ನಿಷ್ಕಲಮಹಂ ನ ಚ ಭೂತಸಂಘಃ|
\end{shloka} 

(ಸಚ್ಚಿತ್ಸುಖರೂಪವೂ, ಪರಮಹಂಸಗತಿಯುಳ್ಳದ್ದೂ, ತುರೀಯವೂ ಆದ, ಹೃದಯದಲ್ಲಿ ಸ್ಫುರಿಸುತ್ತಿರುವ ಆತ್ಮತತ್ತ್ವವನ್ನು ಬೆಳಗಿನಲ್ಲಿ ನಾನು ಸ್ಮರಿಸಿಕೊಳ್ಳುತ್ತೇನೆ. ಯಾವುದು ಪ್ರತಿದಿನವೂ ಎಚ್ಚರ, ಕನಸು, ನಿದ್ರೆಗಳೆಂಬ ಮೂರು ಅವಸ್ಥೆಗಳನ್ನೂ ಹೊಂದುವುದೋ, ಅಂತಹ ನಿಷ್ಕಲನಾದ ಬ್ರಹ್ಮವಸ್ತುವೇ ನಾನಾಗಿದ್ದೇನೆ. ನಾನು ಕೇವಲ ಪಂಚಭೂತಗಳ ಸಮುದಾಯವಲ್ಲ) ಎಂದು ಏಳುವಾಗಲೇ ಎಲ್ಲವನ್ನೂ ತುಂಬಿರುವ ಅವನ ರಸವನ್ನೇ ತುಂಬಿಕೊಂಡು, ಸಂಧ್ಯಾಕಾರ್ಯ, ದೇವಪೂಜೆ ಮೊದಲಾದ ಆಹ್ನಿಕಗಳ ಮೂಲಕ ಅವನ ಪೂಜೆಯನ್ನು ಮಾಡಿ, ಎಲ್ಲ ವ್ಯವಹಾರಗಳೂ ಮುಗಿದು ಮತ್ತೆ ರಾತ್ರಿ ಮಲಗುವ ಹೊತ್ತಿಗೆ ಯಾವ ನೆನಪಿನೊಡನೆ ಮಲಗುತ್ತಾರೆ ನೋಡೀಪ್ಪಾ. 

\begin{shloka} 
ಕ್ಷೀರಸಾಗರತರಂಗಶೀಕರಾಸಾರತಾರಕಿತಚಾರುಮೂರ್ತಯೇ|\label{147}\\
ಭೋಗಿಭೋಗಶಯನೀಯಶಾಯಿನೇ ಮಾಧವಾಯ ಮಧುವಿದ್ವಿಷೇ ನಮಃ||
\end{shloka} 

(ಕ್ಷೀರಸಾಗರದ ಅಲೆಗಳ ತುಂತುರು ಹನಿಗಳಿಂದ ಅಲಂಕೃತವಾದ ಮಂಗಳ ಮೂರ್ತಿಯೇ, ಆದಿಶೇಷನ ದಿವ್ಯವಿಗ್ರಹವೆಂಬ ಹಾಸುಗೆಯ ಮೇಲೆ ಶಯನಿಸಿದ ಮಧುಸೂದನನೇ ನಿನಗೆ ನಮಸ್ಕಾರ) ಎಂದು ಜೀವವು ಹೋಗಿ ದೇವನೊಡನೆ ಸೇರಿ, ಅವನು ಒಪ್ಪಿಕೊಳ್ಳುವಂತೆ ಅಪ್ಪಿ ಒಂದಾಗಿ ಸೇರಿ, ಅವನೊಡನೆಯೇ ಅವನ ಯೋಗನಿದ್ರೆಯಲ್ಲಿ, ಮನೋಹರವಾದ ಮೂರ್ತಿಯಲ್ಲಿ ಒಂದಾಗಿ ಸೇರಿಹೋಗುವ ವಿಷಯವುಂಟಪ್ಪಾ. ಮೂರ್ತವಾದ (ಆಕೃತಿಯನ್ನು ಪಡೆದ) ಜಗತ್ತಿನಲ್ಲಿಯೇ ಇದ್ದರೂ ಅಮೂರ್ತವಾದ, ಅವ್ಯಕ್ತವಾದ ತೇಜಸ್ಸಿನಲ್ಲಿ ಒಂದಾಗಿದ್ದು ಮತ್ತೆ ಏಳುವಾಗ ಭಗವಂತನೊಡನೆ ಜೀವನಮಾಡುವ ನಡೆಯಿದೇಪ್ಪಾ. ಹೀಗೆ ಭಗವಂತನ ಸ್ಮರಣೆಯನ್ನು ಹೆಜ್ಜೆ ಹೆಜ್ಜೆಗೂ ತಂದುಕೊಡುವ ರೀತಿಯಲ್ಲಿ ಹಬ್ಬಹರಿದಿನಗಳ ವ್ಯವಸ್ಥೆಯಿದೆ. 

\section*{ಶ್ರೀರಾಮನವಮಿಯ ಆಚರಣೆಯ ಕುರಿತು} 

ಇಂದು ಶ್ರೀರಾಮನವಮೀ. ರಾಮನವಮಿಯನ್ನು ಸೌರಮಾನ-ಚಾಂದ್ರಮಾನಗಳೆಂಬ ಎರಡುಸಂಪ್ರದಾಯಗಳಲ್ಲಿಯೂ ಆಚರಿಸುವುದುಂಟು. ಆದರೂ ರಾಮಾಯಣದಲ್ಲಿಯೇ `ಚೈತ್ರೇ ನಾವಮಿಕೇ ತಿಥೌ' ಎಂಬ ಮಾತು ಬರುವುದರಿಂದ ಚಾಂದ್ರಮಾನ ತಿಥಿಯನ್ನನುಸರಿಸಿ ಮಾಡುವ ರೂಢಿಯೇ ಹೆಚ್ಚಾಗಿ ಬೆಳೆದುಬಂದಿದೆ. 

\section*{ಶ್ರೀರಾಮಾವತಾರದ ಪೂರ್ವಸಂದರ್ಭ}

ರಾಮಾವತಾರದ ಸಂದರ್ಭವಾದರೂ ಏನಾಗಿದೆ? ಒಂದೆಡೆ ಮಕ್ಕಳಿಲ್ಲದ ದಶರಥನು ಮಕ್ಕಳಿಗಾಗಿ ಹಂಬಲಿಸುತ್ತಾ ಯಜ್ಞಶೀಲನಾಗಿ ಯಜ್ಞಪ್ರಿಯನಾದ ವಿಷ್ಣುವನ್ನು ಆರಾಧಿಸುತ್ತಾನೆ. ಅವನ ಪ್ರಾರ್ಥನೆ ಒಂದೆಡೆ. ಲೋಕದಲ್ಲಿ ಅಧರ್ಮವು ತಲೆಯೆತ್ತಿದ ಬಗೆಗೆ, ಆಸುರೀ ಪ್ರವೃತ್ತಿಯು ಅಧಿಕವಾದ ಬಗೆಗೆ ಋಷಿಗಳ, ದೇವತೆಗಳ ಪ್ರಾರ್ಥನೆಯು ಒಂದು ಕಡೆ. ಇಂತಹ ಸನ್ನಿವೇಶದಲ್ಲಿ ಆಸುರೀವೃತ್ತಿಯನ್ನು ಮೆಟ್ಟಲು, `ರಾಮ-ಲಕ್ಷ್ಮಣ-ಭರತ-ಶತ್ರುಘ್ನ-ಹನುಮತ್ಸಮೇತ-ಸೀತಾರಾಮರಾಗಿ' ಬಂದು ದೈವೀ ಸಂಪತ್ತನ್ನು ಭುವಿಯಲ್ಲಿ ತುಂಬಿ ಭುವಿ-ದಿವಿಗಳಿಗೆ ಸೇತುವೆ ಕಟ್ಟಿದ ಮಹಾಪುರುಷನ ಜನ್ಮದಿನವಿದಾಗಿದೆ. 

\section*{ರಘುವಂಶದ ಉದಾತ್ತತೆ} 

ಮಹಾಕವಿಯಾದ ಕಾಳಿದಾಸನು ರಘು ಅಥವಾ ಸೂರ್ಯವಂಶವನ್ನು ಕೊಂಡಾಡುವಾಗ ಬಾಯಿತುಂಬಾ ಕೊಂಡಾಡಿದ್ದಾನೆ- 

\begin{shloka} 
ಸೋಽಹಮಾಜನ್ಮಶುದ್ಧಾನಾಮ್‍ ಆಫಲೋದಯಕರ್ಮಣಾಮ್‍|\label{148b}\\ 
ಆಸಮುದ್ರಕ್ಷಿತೀಶಾನಾಮ್‍ ಆನಾಕರಥವರ್ತ್ಮನಾಮ್‍||
\end{shloka} 

(ಹುಟ್ಟಿನಿಂದಲೇ ಶುದ್ಧರಾದ, ಫಲಬರುವವರೆಗೂ ಕಾರ್ಯಶೀಲರಾದ, ಸಮುದ್ರಪರ್ಯಂತ ಭೂಮಿಗೆ ಒಡೆಯರಾದ, ಸ್ವರ್ಗಪರ್ಯಂತ ಚಲಿಸುವ ರಥದ ಹಾದಿಯುಳ್ಳ ರಘುರಾಜರನ್ನು ವರ್ಣಿಸುತ್ತೇನೆ) ಭುವಿ-ದಿವಿಗಳಲ್ಲಿ ಅಪ್ರತಿಹತವಾದ ಸಂಚಾರವುಳ್ಳ ರಾಜರುಗಳು ಅವರು. 

\begin{shloka} 
ಮಮ ದ್ವಿತಾ ರಾಷ್ಟ್ರಂ ಕ್ಷತ್ರಿಯಸ್ಯ\label{148a}\\ 
ವಿಶ್ವಾಯೋರ್ವಿಶ್ವೇ ಅಮೃತಾ ಯಥಾ ನಃ|
\end{shloka} 

(ಕ್ಷತ್ರಿಯನಾದ ನನಗೆ ರಾಷ್ಟ್ರವು ಎರಡು ಬಗೆಯಾದುದು. ದೇವತೆಗಳೂ ನಮ್ಮವರೇ) ಎಂಬಂತೆ ಐಹಿಕ-ಪಾರಮಾರ್ಥಿಕಗಳೆರಡನ್ನೂ ಧರಿಸುವವರು. ಭುವಿ-ದಿವಿಗಳ ಸಂಗಮವನ್ನು ಪಾಲಿಸುವವರು. ಅಂತಹ ಮಹಾಶ್ರೇಷ್ಠವಾದ ಸೂರ್ಯವಂಶದಲ್ಲಿ ಹುಟ್ಟಿದವನು ರಾಮ. 

\section*{ವಾಲ್ಮೀಕಿ ಕೋಗಿಲೆಯ ರಾಮಾಯಣಗಾನ} 

ಬಿಸಿಲಿನ ಬಿರುಬೇಗೆಯಲ್ಲಿ ಬೆಂದು ಸಂತಪ್ತವಾಗಿರುವ ಲೋಕಕ್ಕೆ, ವೃಕ್ಷಾಗ್ರದಲ್ಲಿ ಕುಳಿತು ಇಂಪಾಗಿ ಧ್ವನಿ ಮಾಡುವ ಕೋಗಿಲೆಯು ನವನವನಾದ ಹರ್ಷವನ್ನು ತರುವಂತೆ, ಜೀವನ ವೃಕ್ಷದ ಮೇಲೆ ಕುಳಿತು, ಭವದ ತಾಪದಿಂದ ಬಳಲಿ ಬೆಂಡಾಗಿರುವ ಜೀವಕ್ಕೆ, ತನ್ನ ಇಂಪಾದ ಸ್ವರದಿಂದ ನವಚೈತನ್ಯವನ್ನು ತುಂಬಿ, ಜೀವನಕ್ಕೆ ಇಂಪು ತಂಪುಗಳನ್ನು ಕೊಡುವಂತೆ ಹಾಡಿದ ಕೋಗಿಲೆ ವಾಲ್ಮೀಕಿಕೋಗಿಲೆಯಾಗಿದೆಯಪ್ಪಾ! 

\begin{shloka} 
ಕೂಜಂತಂ ರಾಮರಾಮೇತಿ ಮಧುರಂ ಮಧುರಾಕ್ಷರಮ್‍|\label{148}\\ 
ಆರುಹ್ಯ ಕವಿತಾಶಾಖಾಂ ವಂದೇ ವಾಲ್ಮೀಕಿಕೋಕಿಲಮ್‍||
\end{shloka} 

(ಕವಿತಾಶಾಖೆಯನ್ನೇರಿ, `ರಾಮ, ರಾಮ' ಎಂದು ಮಧುರವಾಗಿ, ಮಧುರವಾದ ಅಕ್ಷರಗಳಿಂದ ಇಂಪಾದ ಧ್ವನಿಮಾಡುತ್ತಿರುವ ವಾಲ್ಮೀಕಿ ಎಂಬ ಕೋಗಿಲೆಯನ್ನು ನಮಸ್ಕರಿಸುತ್ತೇನೆ) ಯೋಗದ ಸ್ಥಿತಿಯಲ್ಲಿದ್ದು ಎಲ್ಲವನ್ನೂ ಹೃದಯದಲ್ಲಿನೋಡಿ- 

\begin{shloka} 
ತತ್ಸರ್ವಂ ತತ್ತ್ವತೋ ದೃಷ್ಟ್ವಾ ಪಾಣಾವಾಮಲಕಂ ಯಥಾ|\label{149}
\end{shloka}

ಅಂಗೈ ನೆಲ್ಲಿಯಂತೆ, - ಅಂಗೈಯಲ್ಲಿರುವ ನೆಲ್ಲಿಕಾಯಿಯನ್ನು ತಿರುಗಿಸಿ, ಅದರ ಸಮಗ್ರವಾದ ದರ್ಶನವನ್ನು ನಮಗಿಷ್ಟಬಂದಂತೆ ಹೇಗೆ ನೋಡಬಹುದೋ ಹಾಗೆ ವಿಶ್ವಗೋಳವನ್ನೇ ವ್ಯಾಪಿಸಿರುವ ನಾರಾಯಣನ ಚರಿತ್ರೆಯನ್ನು ಮನನಮಾಡುತ್ತಾ, ಮುನಿಯಾಗಿ ತಪೋಭೂಮಿಕೆಯಿಂದ ಭಾರತಭೂಮಿಗೆ ತಂದುಕೊಟ್ಟ ವಿಷಯವುಂಟು. ಅದರ ಒಟ್ಟು ಸಾರಾಂಶವೇನಪ್ಪಾ ಎಂದರೆ ಇಷ್ಟೇ- 

\section*{ರಾಮಾಯಣದ ಸಾರ} 

\begin{shloka} 
ಕಾಲೇ ವರ್ಷತು ಪರ್ಜನ್ಯಃ ಪೃಥಿವೀ ಸಸ್ಯಶಾಲಿನೀ|\label{149a}\\ 
ದೇಶೋಽಯಂ ಕ್ಷೋಭರಹಿತಃ ಬ್ರಾಹ್ಮಣಾಸ್ಸಂತು ನಿರ್ಭಯಾಃ|
\end{shloka} 

ದಿವಿಯ ಸಹಾಯದಿಂದ ಭುವಿಯ ಬೆಳವಣಿಗೆ, ಹಾಗೆ ದಿವಿಯಿಂದ ವರ್ಷಿಸಿದಾಗಲೇ ಭೂಮಿಯು ಸಸ್ಯಶಾಲಿನಿಯಾಗಿರಬಹುದು. ಹಾಗಾಗಿ ಭುವಿ-ದಿವಿಗಳ ಸಂಗಮವು ಹಿತಕರವಾಗಿದ್ದು ದೇಶವು ಕ್ಷೋಭರಹಿತವಾಗಿ-ಶೋಭೆಯಿಂದ ಕೂಡಿರಬೇಕು. ಬ್ರಹ್ಮಜ್ಞಾನಸಂಪನ್ನನಾದವನು ಆಳಬೇಕು. ಹಾಗೆ ಬ್ರಹ್ಮಜ್ಞಾನ ಸಂಪನ್ನನಾದವನು ನಿರ್ಭಯನಾಗಿ, ಆ ಜ್ಞಾನಿಯ ಆಳ್ವಿಕೆಗೆ ಲೋಕವು ಒಳಪಟ್ಟಾಗ ನರಲೋಕವನ್ನು ನಾರಾಯಣನ ಲೋಕವನ್ನಾಗಿ ಮಾರ್ಪಡಿಸಬಹುದು. 

\begin{shloka} 
ಸ್ವಸ್ತಿಪ್ರಜಾಭ್ಯಃ ಪರಿಪಾಲಯಂತಾಂ\label{149b}\\ 
ನ್ಯಾಯ್ಯೇನ ಮಾರ್ಗೆಣ ಮಹೀಂ ಮಹೀಶಾಃ|\\ 
ಗೋಬ್ರಾಹ್ಮಣೇಭ್ಯಃ ಶುಭಮಸ್ತು ನಿತ್ಯಂ\\ 
ಲೋಕಾಸ್ಸಮಸ್ತಾಸ್ಸುಖಿನೋ ಭವಂತು||
\end{shloka} 

ರಾಮಾಯಣದ ಪ್ರಯೋಜನವಿದ್ದರೆ ಇದಕ್ಕಾಗಿ. ಜೀವನವನ್ನರಳಿಸಲು ಬೇಕಾದ ಹೊತ್ತಿಗೆಯನ್ನು ಕೊಟ್ಟಿದ್ದಾರೆ. ಜೀವನವು ಏನನ್ನು ಹೊತ್ತಿದೆಯೋ ಆ ಹೊತ್ತಿಗೆ ಇದಾಗಿದೆ. ಆದ್ದರಿಂದಲೇ ಆ ಬಗೆಗೆ ಮತ್ತೆ ಮತ್ತೆ ಜ್ಞಾಪಿಸಲು ವಸಂತನವರಾತ್ರಶರನ್ನವರಾತ್ರದ ಸಂದರ್ಭಗಳಲ್ಲಿ ರಾಮಾಯಣಪಾರಾಯಣ ಮಾಡುವ ರೂಢಿ ಬೆಳೆದುಬಂದಿದೆ. ರಾಮರಾಜ್ಯದ ಸವಿನೆನಪನ್ನು ಆಗಾಗ್ಗೆ ತಂದುಕೊಳ್ಳಲನುಗುಣವಾಗಿದೆ ಈ ನಡೆ. 

\section*{ದಿವಿಭುವಿಗಳ ಸೇತು ಶ್ರೀರಾಮಾವತಾರ}

ನರರೂಪಿಯಾಗಿ ನಾರಾಯಣನು ಬಂದಿರುವ ಗುಟ್ಟನ್ನರಿತ ವಾಲ್ಮೀಕಿಯು ಉದ್ದಕ್ಕೂ ಆ ಧ್ವನಿಯನ್ನು ಬಿಟ್ಟಿಲ್ಲ. ರಾಮನಿಗೆ ಆದ ಪಟ್ಟಾಭಿಷೇಕವನ್ನು ವರ್ಣಿಸುವಾಗ ಹೇಗೆ ವರ್ಣಿಸಿದ್ದಾರೆ? 

\begin{shloka} 
ಅಭ್ಯಷಿಂಚನ್ ನರವ್ಯಾಘ್ರಂ ಪ್ರಸನ್ನೇನ ಸುಗಂಧಿನಾ|\label{150}\\
ಸಲಿಲೇನ ಸಹಸ್ರಾಕ್ಷಂ ವಸವೋ ವಾಸವಂ ಯಥಾ||
\end{shloka} 

(ವಸುಗಳು ಸಹಸ್ರಾಕ್ಷನನ್ನು ಹೇಗೆ ಅಭಿಷೇಕಿಸಿದರೋ ಹಾಗೆ, ಪ್ರಸನ್ನವೂ ಸುಗಂಧಿಯೂ ಆದ ಸಲಿಲದಿಂದ ನರವ್ಯಾಘ್ರನಾದ ಶ್ರೀರಾಮಚಂದ್ರನನ್ನು ಅಭಿಷೇಕಿಸಿದರು.) ಭುವಿಯಲ್ಲಿ ನಡೆಯುತ್ತಿರುವ ಅಭಿಷೇಕವನ್ನು ವರ್ಣಿಸುತ್ತಿದ್ದರೂ ಅದಕ್ಕೆ ದಿವಿಯ ಉಪಮಾನ, ಸಹಸ್ರಾಕ್ಷ. 

\begin{shloka} 
ಸಹಸ್ರಶೀರ್ಷಾ ಪುರುಷಃ ಸಹಸ್ರಾಕ್ಷಃ ಸಹಸ್ರಪಾತ್‍|\label{150c}\\ 
ಸ ಭೂಮಿಂ ವಿಶ್ವತೋ ವೃತ್ವಾ ಅತ್ಯತಿಷ್ಠದ್ದಶಾಂಗುಲಮ್‍||
\end{shloka} 

ಆ ಪರಮಪುರುಷನ ಸವಿನೆನಪನ್ನು ನೆನಪಿಸುತ್ತಲೇ ಇದ್ದಾರೆ. `ತಸ್ಯ ಧೀರಾಃ ಪರಿಜಾನಂತಿ ಯೋನಿಂ'| (ಅವನ ಹುಟ್ಟನ್ನು ಜ್ಞಾನಿಗಳು ಬಲ್ಲರು), `ಅಜಾಯ ಮಾನೋ ಬಹುಧಾ ವಿಜಾಯತೇ'|\label{150a} (ಹುಟ್ಟಿಲ್ಲದವನಾದರೂ ಬಹುವಾಗಿ ಹುಟ್ಟುತ್ತಾನೆ). ಯಾವನಿಗೆ ಜನ್ಮವೇ ಇಲ್ಲವೋ, ಆ ಪ್ರಭುವೇ ತನ್ನವರ ಹಿತಕ್ಕಾಗಿ, ಲೋಕಕಲ್ಯಾಣಕ್ಕಾಗಿ, ವರಾಹನಾಗಿಯೋ, ನರಸಿಂಹನಾಗಿಯೋ, ಮತ್ಸ್ಯನಾಗಿಯೋ, ಕೂರ್ಮನಾಗಿಯೋ ಬಹುವಾಗಿ ಅವತರಿಸುತ್ತಾನಪ್ಪಾ. ಹೃದಯದಲ್ಲಿ ವಿರಾಜಮಾನನಾಗಿ, ರಾಜನಾಗಿ ನರರೂಪಿಯಾಗಿ ಬಂದರೂ ನಾರಾಯಣನೇ ಆಗಿದ್ದಾನಪ್ಪಾ. ಭಗವಂತನೇ ತನ್ನ ಸತ್ಯಸಂಕಲ್ಪದಿಂದ ಬರುವುದುಂಟಪ್ಪಾ. 

\begin{shloka} 
ಯದಾ ಧರ್ಮಗ್ಲಾನಿರ್ಭವತಿ ಜಗತಾಂ ಕ್ಷೋಭಕರಣೀ\label{150b}\\ 
ತದಾ ಲೋಕಸ್ವಾಮೀ ಪ್ರಕಟಿತವಪುಃ ಸೇತುಧೃದಜಃ|\\ 
ಸತಾಂ ಧಾತಾ ಸ್ವಚ್ಛೋ ನಿಗಮಗಣಗೀತೋ ವ್ರಜಪತಿಃ\\ 
ಶರಣ್ಯೋ ಲೋಕೇಶೋ ಮಮ ಭವತು ಕೃಷ್ಣೋಽಕ್ಷಿವಿಷಯಃ||
\end{shloka} 

(ಜಗತ್ತಿಗೆ ಕ್ಷೋಭೆಯನ್ನುಂಟುಮಾಡುವ ಧರ್ಮಗ್ಲಾನಿಯು ಯಾವಾಗ ಉಂಟಾಗುವುದೋ, ಆಗ ಲೋಕಸ್ವಾಮಿಯು, ಶರೀರದೊಡನೆ ಪ್ರಕಾಶಕ್ಕೆ ಬಂದು ಧರ್ಮಸೇತುವನ್ನು ಧರಿಸಿ, ಸತ್ಪುರುಷರನ್ನು ಕಾಪಾಡುವವನಾಗುತ್ತಾನೆ. ಸ್ವಚ್ಛನಾದ ವೇದಗಳಿಂದ ಗಾನಮಾಡಲ್ಪಟ್ಟ, ವ್ರಜಪತಿಯಾದ, ಎಲ್ಲರಿಗೂ ಶರಣಪ್ರದನಾದ, ಲೋಕೇಶನಾದ ಕೃಷ್ಣನು ನನ್ನ ದೃಷ್ಟಿ ಗೋಚರನಾಗಲಿ.) 

ಶೋಭನವಾದ ಜೀವನವನ್ನು ಕೊಡಲು ಇಳೆಗೆ ಬರುತ್ತಾನೆ. ರಾಮನಾಗಲೀ ಕೃಷ್ಣನಾಗಲೀ ತಂದದ್ದು ತನ್ನ ದೈವೀಸಂಪತ್ತನ್ನು. ಅಂತಹ ದೈವೀಸಂಪತ್ತನ್ನು ನೆಲೆಗೊಳಿಸಿ ಭುವಿ ದಿವಿಗಳ ಸೇತುವನ್ನು ಕಟ್ಟಿದುದೇ ಅವರ ಕಾರ್ಯ. 

\section*{ದೈವೀಸಂಪತ್ತನ್ನು ನೆಲೆಗೊಳಿಸಲು ಅವತರಿಸಿದ ಮಹಾಪುರುಷನ ಚರಿತ ರಾಮಾಯಣ} 

\begin{shloka} 
ರಾಮರಾಜ್ಯದಲ್ಲಿ ಪ್ರಜೆಗಳ ಸ್ಥಿತಿ ಹೇಗಿತ್ತು? ಎಂದರೆ-\\ 
ಪ್ರಾಣಾಪಾನೌ ಸಮಾವಾಸ್ತಾಂ ರಾಮೇ ರಾಜ್ಯಂ ಪ್ರಶಾಸತಿ||\label{151}
\end{shloka} 

(ರಾಮನು ರಾಜ್ಯವಾಳುವಾಗ ಪ್ರಾಣಾಪಾನಗಳು ಸಮವಾಗಿದ್ದವು.) ಐಹಿಕ ಪಾರಮಾರ್ಥಿಕ ಎರಡು ಜೀವನಗಳಿಗೂ ಘರ್ಷಣೆಯಿಲ್ಲದೇ ಜೀವನವು ನಿಶ್ಚಲವಾಗಲು ಬೇಕಾದ ವ್ಯವಸ್ಥೆಯಿತ್ತು. ಪ್ರಾಣವು ಮೇಲಕ್ಕೆಳೆದರೆ ಅಪಾನವು ಕೆಳಗೆಳೆಯುತ್ತದೆ. ಇವೆರಡರ ಘರ್ಷಣೆ ತಪ್ಪಬೇಕು. ಸಮಸ್ಥಿತಿಯಲ್ಲಿ ನಿಲ್ಲಬೇಕು. ಅದಕ್ಕಾಗಿ ಹೋರಾಡಿದ ಮಹಾಪ್ರಭು ಶ್ರೀರಾಮ. 

ರಾಮನು ಕೇವಲ ಪಟ್ಟಾಭಿಷಿಕ್ತನಾಗಿ ಕುಳಿತುಬಿಡಲಿಲ್ಲ. ಅವತಾರಕ್ಕೆ ಪ್ರತಿಜ್ಞೆಯಿದೆ. ಸಂಕಲ್ಪವಿದೆ. ರಾಜ್ಯವನ್ನು ಬಿಟ್ಟು ಅರಣ್ಯಕ್ಕೆ ತೆರಳಿದ. ಜಟಾವಲ್ಕಲ ಧಾರಿಯಾಗಿ ಋಷಿಜೀವನ ಮಾಡಿದ. `ವಿದ್ಧಿ ಮಾಂ ಋಷಿಭಿಸ್ತುಲ್ಯಂ'\label{151b} ಎಂಬುದು ಅವನ ಹೇಳಿಕೆ. ಆಸುರೀ ಸಂಪತ್ತನ್ನು ಮೆಟ್ಟಬೇಕು. ದೈವೀಸಂಪತ್ತನ್ನು ಎತ್ತಿ ಹಿಡಿಯಬೇಕು. ಅಂತಹ ಮಹಾಕಾರ್ಯಕ್ಕಾಗಿ, ಧರ್ಮಕ್ಕಾಗಿ ಅವತರಿಸಿದ ಮಹಾಪುರುಷ ಶ್ರೀರಾಮ. ಅಂತಹ ಮಹಾಪುರುಷನ ಮಹಾಚರಿತ್ರವನ್ನು-ರಾಮನ ಅಯನವನ್ನು ಹರಿಸಿದವರು ವಾಲ್ಮೀಕಿಗಳು. 

\section*{ಲೋಕಕಲ್ಯಾಣಕ್ಕಾಗಿ ಹರಿದು ಬಂದ ಮಹಾನದೀ ರಾಮಾಯಣ.} 

ಉತ್ತುಂಗಶಿಖರದಿಂದ ಹರಿದ ಗಂಗೆಯು ಧಿಮಿ ಧಿಮಿ ತಾಳದಿಂದ ನಲಿದು ಸಪ್ತಶಾಖೆಯಾಗಿ ಹರಿದು, ಲೋಕಕ್ಕೆ ತಂಪನ್ನುಂಟುಮಾಡುತ್ತಾ, ದಾರಿಯುದ್ಧಕ್ಕೂ ಬೆಳವಣಿಗೆಗೆ ಬೇಕಾದ, ಕೃಷಿಗೆ ನೀರೊದಗಿಸುತ್ತಾ, ಸಮುದ್ರವನ್ನು ಸೇರುವಂತೆ, ವಾಲ್ಮೀಕಿಮಹಾಗಿರಿಯಿಂದ ಜೀವಲೋಕಕ್ಕೆ ಜ್ಞಾನಧಾರೆಯನ್ನು ಹರಿಸಿ ಋಷಿಗಳನ್ನಾಗಿ ಮಾಡಲು ರಾಮಾಯಣಮಹಾನದಿ ಹರಿದು ಬಂದಿತು. ಲೋಕಕಲ್ಯಾಣಕ್ಕಾಗಿ ಕಲ್ಯಾಣಿಯಾಗಿ (ಕಲ್ಯಾಣಿರಾಗದಿಂದ) ಹರಿದು ಬಂದಿತು. 

\begin{shloka} 
ವಾಲ್ಮೀಕಿಗಿರಿಸಂಭೂತಾ ರಾಮಸಾಗರಗಾಮಿನೀ|\label{151a}\\ 
ಪುನಾತಿ ಭುವನಂ ಪುಣ್ಯಾ ರಾಮಾಯಣಮಹಾನದೀ||\\ 
ಶ್ಲೋಕಸಾರಸಮಾಕೀರ್ಣಂ ಸರ್ಗಕಲ್ಲೋಲಸಂಕುಲಂ|\label{152b}\\ 
ಕಾಂಡಗ್ರಾಹಮಹಾಮೀನಂ ವಂದೇ ರಾಮಾಯಣಾರ್ಣವಮ್‍||
\end{shloka} 

(ರಾಮ-ಎಂಬ ಸಮುದ್ರವನ್ನು ಸೇರುವ ನದಿ, ಪ್ರಪಂಚವನ್ನು ಇಂದ್ರಿಯ ಜೀವನವನ್ನು ತುಂಬಿಕೊಂಡು, ಇಂದ್ರಿಯಗಳನ್ನು ರಾಮನಿಗೆ ಅಭಿಮುಖವಾಗಿ ಎಳೆದು, ಜೀವನವನ್ನು, ವಿಶ್ವವನ್ನು ಪಾವನಗೊಳಿಸುವ ಮಹಾನದಿ. ವಾಲ್ಮೀಕಿಯೆಂಬ ಮೇರುಪರ್ವತದಿಂದ ಹರಿದು ಬಂದ ಮಹಾಗಂಗೆ. ರಾಮನೆಂಬ ಸಮುದ್ರವನ್ನು ಸೇರಿ ಅಮುದ್ರವಾದ ಜೀವನಕ್ಕೆ ಮುದ್ರೆಯೊತ್ತುವುದಾಗಿದೆ. 

ಶ್ಲೋಕಸಾರವೆಂಬ ಜಲರಾಶಿಯಿಂದ ತುಂಬಿ, ಸರ್ಗಗಳೆಂಬ ಅಲೆಗಳಿಂದ ಆರ್ಭಟಿಸುತ್ತಾ, ಕಾಂಡಗಳೆಂಬ ಮಹಾಮೀನ, ಮಕರ ಮೊದಲಾದ ಜಲಚರಗಳಿಂದ ತುಂಬಿಕೊಂಡಿರುವ ರಾಮಾಯಣವೆಂಬ ಅರ್ಣವಕ್ಕೆ ನಮಸ್ಕಾರ.) 

\begin{shloka} 
ತದುಪಗತಸಮಾಸಸಂಧಿಯೋಗಂ\label{152a}\\ 
ಸಮಮಧುರೋಪನತಾರ್ಥವಾಕ್ಯಬದ್ಧಮ್‍|\\ 
ರಘುವರಚರಿತಂ ಮುನಿಪ್ರಣೀತಂ\\ 
ದಶಶಿರಸಶ್ಚ ವಧಂ ನಿಶಾಮಯಧ್ವಮ್‍||
\end{shloka} 

(ಸಮಾಸ ಸಂಧಿಗಳಿಂದೊಡಗೂಡಿದ, ಸರಳವೂ ಮಾಧುರ್ಯಭರಿತವೂ ಆದ ಮಾತುಗಳಿಂದ ತುಂಬಿರುವ. ವಾಲ್ಮೀಕಿ ಮುನಿಯಿಂದ ರಚನೆಗೊಂಡ ಶ್ರೀರಾಮಚರಿತೆ ಮತ್ತು ರಾವಣವಧೆಯನ್ನು ಕೇಳಿರಿ.) 

\section*{ರಾಮಾಯಣ ಶ್ರವಣದ ಫಲ} 

ಇಂತಹ ರಾಮಾಯಣವನ್ನು ಕೇಳಿದ್ದರಿಂದ ಆಗುವುದೇನು?- 

\begin{shloka} 
ಯಃ ಕರ್ಣಾಂಜಲಸಂಪುಟೈರಹರಹಸ್ಸಮ್ಯಕ್ಪಿಬತ್ಯಾದರಾತ್‍\label{152}\\ 
ವಾಲ್ಮೀಕೇರ್ವದನಾರವಿಂದಗಳಿತಂ ರಾಮಾಯಣಾಖ್ಯಂ ಮಧು|\\ 
ಜನ್ಮವ್ಯಾಧಿಜರಾವಿಪತ್ತಿಮರಣೈಃ ಅತ್ಯಂತ ಸೋಪದ್ರವಂ|\\ 
ಸಂಸಾರಂ ಸ ವಿಹಾಯ ಗಚ್ಛತಿ ಪುರ್ಮಾ ವಿಷ್ಣೋಃ ಪದಃ ಶಾಶ್ವತಮ್‍||
\end{shloka} 

(ಕರ್ಣಗಳನ್ನೇ ಅಂಜಲಿ (ಬೊಗಸೆ)ಯಾಗಿ ಒಡ್ಡಿ, ವಾಲ್ಮೀಕಿಯ ಮುಖಕಮಲದಿಂದ ಹರಿದ ರಾಮಾಯಣವೆಂಬ ಮಧು (ಜೇನು)ವನ್ನು ಕುಡಿದರೆ, ಜನ್ಮ-ವ್ಯಾಧಿ-ಮುಪ್ಪು-ವಿಪತ್ತು-ಸಾವು ಎಂಬ ಸಂಕಟಗಳಿಂದ ತುಂಬಿದ ಅತಿಕ್ಲೇಶಕರವಾದ ಸಂಸಾರವನ್ನು ಬಿಟ್ಟು ಮನುಷ್ಯನು ಶಾಶ್ವತವಾದ ವಿಷ್ಣುಪದವನ್ನು ಸೇರುತ್ತಾನೆ.) ಅಲ್ಲಿ ನಿಲ್ಲಿಸುವುದೇ ಕಾವ್ಯದ ಗುರಿ. ರಾಮಾಯಣದ ಶ್ರವಣವು ವ್ಯಾಧಿಪರಿಹಾರ ಮಾಡುತ್ತದೆ ಎಂಬುದನ್ನೂ ಪ್ರಾಯೋಗಿಕವಾಗಿಯೇ ತೋರಿಸಬಹುದು. 

\section*{ರಾಮಾಯಣರಚನೆಯ ಹಿನ್ನೆಲೆಯಲ್ಲಿರುವ ವಾಲ್ಮೀಕಿ ಹಾಗು ನಾರದರು} 

ರಾಮಾಯಣಕಾವ್ಯದ ರಚನೆಯಾದರೂ ಹೇಗಾಯಿತು? ವಾಲ್ಮೀಕಿಗಳ ಮನಸ್ಸಿನಲ್ಲಿ ಒಂದು ಪ್ರಶ್ನೆ ಹುಟ್ಟಿಕೊಳ್ಳುತ್ತದೆ. ಆ ವೇಳೆಗೆ ನಾರದರು ಅಲ್ಲಿಗೆ ಬರಲು ಅವರನ್ನೇ ಪ್ರಶ್ನಿಸುತ್ತಾರೆ- 

\begin{shloka} 
ತಪಸ್ಸ್ವಾಧ್ಯಾಯನಿರತಂ ತಪಸ್ವೀ ವಾಗ್ವಿದಾಂ ವರಮ್‍\label{153}\\ 
ನಾರದಂ ಪರಿಪಪ್ರಚ್ಛ ವಾಲ್ಮೀಕಿರ್ಮುನಿಪುಂಗವಮ್‍||
\end{shloka} 

ತಪಸ್ಸು ಸ್ವಾಧ್ಯಾಯಗಳಲ್ಲಿ ನಿರತನೂ, ಮಾತನ್ನು ತಿಳಿದವರಲ್ಲಿ ಶ್ರೇಷ್ಠನೂ, ಮನನಶೀಲರಾದ ಮುನಿಗಳಲ್ಲಿ ಶ್ರೇಷ್ಠನೂ ಆದ ನಾರದನನ್ನು, ತಪಸ್ವಿಯಾದ ವಾಲ್ಮೀಕಿಯು ಪ್ರಶ್ನಿಸಿದ. `ವಾಗ್ವಿದಾಂ ವರಮ್‍' ಮಾತನ್ನು ಚೆನ್ನಾಗಿ ತಿಳಿದವನು. ಕೇವಲ ಮಾತನ್ನು ಜಾಣ್ಮೆಯಿಂದಾಡುವವನು ಎಂದಷ್ಟೇ ಅಲ್ಲ. ಮಾತಿನ ನೆಲೆಯರಿತವನು. 

\begin{shloka} 
ಚತ್ವಾರಿ ವಾಕ್ಪರಿಮಿತಾ ಪದಾನಿ| ತಾನಿ ವಿದುರ್ಬ್ರಾಹ್ಮಣಾ ಯೇ ಮನೀಷಿಣಃ|\label{153b}\\ 
ಗುಹಾ ತ್ರೀಣಿ ನಿಹಿತಾ ನೇಂಗಯಂತಿ! ತುರೀಯಂ ವಾಚೋ ಮನುಷ್ಯಾ  ವದಂತಿ||
\end{shloka} 

ನಾಲ್ಕನೆಯದಾದ ವೈಖರಿಯನ್ನು ಮಾತ್ರವೇ ಅಲ್ಲದೇ, ಅದರ ಒಳರೂಪಗಳಾದ, ಜ್ಞಾನಿಗಳು ಮಾತ್ರ ಅರಿಯಬಲ್ಲ ಪರಾ, ಪಶ್ಯಂತೀ, ಮಧ್ಯಮಾ ಎಂಬ ರೂಪಗಳನ್ನೂ ಬಲ್ಲವನು ನಾರದ. ಆದ್ದರಿಂದ ಪರಾರೂಪದಲ್ಲಿರುವುದನ್ನು ಅಂದಗೆಡಿಸದೇ ಹೊರತರಬಲ್ಲ ವಾಗ್ಮಿ ನಾರದ. ಅಂತಹ ನಾರದರಲ್ಲಿ ವಾಲ್ಮೀಕಿಯು ಸಕಲಕಲ್ಯಾಣಗುಣಪರಿಪೂರ್ಣನಾದ ನರನುಂಟೇ? ಎಂಬ ಪ್ರಶ್ನೆಯನ್ನಿಡುತ್ತಾನೆ. ವಾಲ್ಮೀಕಿಯ ಕುತೂಹಲ ಹೀಗೆ ವ್ಯಕ್ತವಾಗಿದೆ- 

\begin{shloka} 
ಏತದಿಚ್ಛಾಮ್ಯಹಂ ಶ್ರೋತುಂ ಪರಂ ಕೌತೂಹಲಂ ಹಿ ಮೇ|\label{153a}\\ 
ಮಹರ್ಷೇ ತ್ವಂ ಸಮರ್ಥೊಽಸಿ ಜ್ಞಾತುಮೇವಂ ವಿಧಂ ನರಮ್‍||
\end{shloka} 

ಇಲ್ಲಿ `ನರಂ'- ಎಂಬಂಶ ಗಮನಾರ್ಹ. ಇಂತಹ ಗುಣವುಳ್ಳ ಮನುಷ್ಯರುಂಟೇ? ಎಂಬುದು ಪ್ರಶ್ನೆ. ಆದ್ದರಿಂದ ಮನುಷ್ಯನಾಗಿಯೇ ರಾಮನು ಅವತರಿಸಿದನು. ಅದನ್ನು ತಿಳಿಯಲು ಸಮರ್ಥನಾದವನು ನಾರದ. ಏಕೆಂದರೆ ಅವನು ಮೂರು ಲೋಕಗಳನ್ನೂ ಬಲ್ಲವನು. ಈ ಪ್ರಶ್ನೆಗೆ ನಾರದನ ಉತ್ತರವಿದು- 

\begin{shloka} 
ಶ್ರುತ್ವಾ ಚೈತತ್ತ್ರಿಲೋಕಜ್ಞಃ ವಾಲ್ಮೀಕೆರ್ನಾರದೋ ವಚಃ|\label{153c}\\ 
ಬಹವೋ ದುರ್ಲಭಾಶ್ಚೈವ ಯೇ ತ್ವಯಾ ಕೀರ್ತಿತಾ ಗುಣಾಃ||\\ 
ಮುನೇ ವಕ್ವಮ್ಯಹಂ ಬುದ್ದ್ವಾ ತೈರ್ಯುಕ್ತಃ ಶ್ರೂಯತಾಂ ನರಃ|
\end{shloka} 

(ನೀನು ಹೇಳಿದ ಗುಣವುಳ್ಳವರು ಬಹಳ ಜನ ದುರ್ಲಭರು, ಅದನ್ನು ತಿಳಿದು ನಾನು ಹೇಳುವೆನು.) 

\section*{ನರನಾಗಿ ಬಂದ ನಾರಾಯಣನಲ್ಲಿ ಆತ್ಮಧರ್ಮವನ್ನು ಹೊರತರಲು ಬೇಕಾದ ಶುಭಲಕ್ಷಣಗಳು} 

ಆಸುರೀಸಂಪತ್ತನ್ನು ಮೆಟ್ಟಲು, ತನ್ನ ಸಂಕಲ್ಪದಂತೆ ನರನಾಗಿ ಬಂದ ನಾರಾಯಣನ ಗುಣಗಳನ್ನು ಮನಸ್ಸಿನಲ್ಲಿಟ್ಟು ನಾರದನು ಹೇಳುತ್ತಾನೆ- 

\begin{shloka} 
ಇಕ್ಷ್ವಾಕುವಂಶಪ್ರಭವೋ ರಾಮೋ ನಾಮ ಜನೈಃಶ್ಯ್ರುತಃ|\label{154}\\ 
ನಿಯತಾತ್ಮಾ ಮಹಾವೀರ್ಯೋ ದ್ಯುತಿಮಾನ್ ಧೃತಿಮಾನ್ ವಶೀ||\\ 
ಬುದ್ಧಿಮಾನ್ ನೀತಿಮಾನ್ ವಾಗ್ಮೀ ಶ್ರೀಮಾನ್ ಶತ್ರುನಿಬರ್ಹಣಃ|\\ 
ವಿಪುಲಾಂಸೋ ಮಹಾಬಾಹುಃ ಕಂಬುಗ್ರೀವೋ ಮಹಾಹನುಃ||\\ 
ಮಹೋರಸ್ಕೋ ಮಹೇಷ್ವಾಸೋ ಗೂಢಜತ್ರುರರಿಂದಮಃ|\\ 
ಆಜಾನುಬಾಹುಃ ಸುಶಿರಾಃ ಸುಲಲಾಟಃ ಸುವಿಕ್ರಮಃ||\\ 
ಸಮಃ ಸಮವಿಭಕ್ತಾಂಗಃ ಸ್ನಿಗ್ಧವರ್ಣಃ ಪ್ರತಾಪವಾನ್|\\ 
ಪೀನದಕ್ಷಾ ವಿಶಾಲಾಕ್ಷೋ ಲಕ್ಷ್ಮೀವಾನ್ ಶುಭಲಕ್ಷಣಃ||
\end{shloka} 

(ಇಕ್ವಾಕುವಂಶದಲ್ಲಿ ಹುಟ್ಟಿದ ರಾಮನೆಂಬುವನು ಜನಗಳಲ್ಲಿ ಖ್ಯಾತನಾದವನು. ಅದನು ಸಂಯಮಿ, ಮಹಾವೀರ್ಯವುಳ್ಳವನು, ಕಾಂತಿಮಂತನು, ಧೃತಿಯುಳ್ಳವನು, ಬುದ್ದಿವಂತ, ನೀತಿವಂತ, ಮಾತುಗಾರ, ಶ್ರೀಮಂತ, ಶತ್ರುನಾಶಕ, ಉಬ್ಬಿದ ವಿಶಾಲವಾದ ಹೆಗಲುಳ್ಳವನು, ಮಹಾಬಾಹು, ಶಂಖದಂತೆ ಕತ್ತುಳ್ಳವನು, ಮಹತ್ತಾದ ಹನುವುಳ್ಳವನು, ವಿಶಾಲವಕ್ಷಸ್ಥಲವುಳ್ಳವನು, ದೊಡ್ಡಬಿಲ್ಲುಳ್ಳವನು, ಮಹಾಪರಾಕ್ರಮಿ, ನೀಳವಾದ ತೋಳುಳ್ಳವನು. ಒಳ್ಳೆಯ ಶಿರಸ್ಸು, ಹಣೆಯುಳ್ಳವನು, ಸಮವಾದವನು, ಸುಶೋಭಿಯಾದ ಸುವಿಭಕ್ತವಾದ ಅವಯವಸಂಪತ್ತುಳ್ಳವನು, ಸ್ನಿಗ್ದವರ್ಣನು ಪ್ರತಾಪಶಾಲಿ, ಉಬ್ಬಿದ ಎದೆ, ವಿಶಾಲವಾದ ಕಣ್ಣುಗಳುಳ್ಳವನು, ಮಹತ್ತಾದ ಕಾಂತಿಯಿಂದಲೂ ಶುಭಲಕ್ಷಣದಿಂದಲೂ ಕೂಡಿದವನು.) 

ಆತ್ಮಧರ್ಮವನ್ನು ಹೊರತರಲು ಬೇಕಾದ ಅವನ ಶುಭಲಕ್ಷಣಗಳನ್ನು ಹೇಳಿ ಮುಂದೆ ಅವನ ಆತ್ಮಗುಣಗಳನ್ನು ಕೊಂಡಾಡುತ್ತಾನೆ- 

\begin{shloka} 
ಧರ್ಮಜ್ಞಸ್ಸತ್ಯಸಂಧಶ್ಚ ಪ್ರಜಾನಾಂ ಚ ಹಿತೇ ರತಃ|\label{155}\\ 
ಯಶಸ್ವೀ ಜ್ಞಾನಸಂಪನ್ನಃ ಶುಚಿರ್ವಶ್ಯಃ ಸಮಾಧಿಮಾನ್||\\ 
ಪ್ರಜಾಪತಿಸಮಃ ಶ್ರೀಮಾನ್ ಧಾತಾ ರಿಪುನಿಷೂದನಃ||\\ 
ರಕ್ಷಿತಾ ಜೀವಲೋಕಸ್ಯ ಧರ್ಮಸ್ಯ ಪರಿರಕ್ಷಿತಾ||\\ 
ರಕ್ಷಿತಾ ಸ್ವಸ್ಯಧರ್ಮಸ್ಯ ಸ್ವಜನಸ್ಯ ಚ ರಕ್ಷಿತಾ|\\ 
ವೇದವೇದಾಂಗತತ್ತ್ವಜ್ಞೋ ಧನುರ್ವೆದೇ ಚ ನಿಷ್ಠಿತಃ||\\ 
ಸರ್ವಶಾಸ್ತ್ರಾರ್ಥತತ್ತ್ವಜ್ಞಃ ಸ್ಪೃತಿಮಾನ್ ಪ್ರತಿಭಾನವಾನ್|\\ 
ಸರ್ವಲೋಕಪ್ರಿಯಸ್ಸಾಧುರದೀನಾತ್ಮಾ ವಿಚಕ್ಷಣಃ||\\ 
ಸರ್ವದಾಭಿಗತಃ ಸದ್ಭಿಃ ಸಮುದ್ರ ಇವ ಸಿಂಧುಭಿಃ|\\ 
ಆರ್ಯಸ್ಸರ್ವಸಮಶ್ಚೈವ ಸದೈಕಪ್ರಿಯದರ್ಶನಃ||\\ 
ಸ ಚ ಸರ್ವಗುಣೋಪೇತಃ ಕೌಸಲ್ಯಾನಂದವರ್ಧನಃ|\\ 
ಸಮುದ್ರ ಇವ ಗಾಂಭೀರ್ಯೇ ಧೈರ್ಯೇಣ ಹಿಮವಾನಿವ||\\ 
ವಿಷ್ಣುನಾ ಸದೃಶೋ ವೀರ್ಯೇ ಸೋಮವತ್ಪ್ರಿಯದರ್ಶನಃ|\\ 
ಕಾಲಾಗ್ನಿಸದೃಶಃ ಕ್ರೋಧೇ ಕ್ಷಮಯಾ ಪೃಥಿವೀಸಮಃ||\\ 
ಧನದೇನ ಸಮಸ್ತ್ಯಾಗೇ ಸತ್ಯೇ ಧರ್ಮ ಇವಾಪರಃ|
\end{shloka} 

(ಆ ಶ್ರೀರಾಮನಾದರೋ ಧರ್ಮಜ್ಞ, ಸತ್ಯಸಂಧ, ಪ್ರಜೆಗಳ ಹಿತದಲ್ಲಿ ಆಸಕ್ತಿಯುಳ್ಳವನು, ಕೀರ್ತಿಮಂತ, ಜ್ಞಾನಸಂಪನ್ನ, ಶುಚಿ, ಜಿತೇಂದ್ರಿಯ, ಸಮಾಧಿ ಯೋಗವನ್ನು ಬಲ್ಲವನು, ಬ್ರಹ್ಮನಿಗೆ ಸಮನಾದವನು, ಶ್ರೀಸಂಪನ್ನ, ಕಾಪಾಡತಕ್ಕವನು, ಶತ್ರುನಾಶಕ, ಜೀವಲೋಕ-ಧರ್ಮಗಳೆರಡನ್ನೂ ಕಾಪಾಡತಕ್ಕವನು, ಸ್ವಧರ್ಮ ಸ್ವಜನಗಳ ರಕ್ಷಕ, ವೇದ ವೇದಾಂಗಗಳ ತತ್ತ್ವವನ್ನು ಬಲ್ಲವನು, ಧನುರ್ವೆದಪಾರಂಗತ, ಸರ್ವಶಾಸ್ತ್ರಗಳ ತತ್ತ್ವವನ್ನು ಬಲ್ಲವನು, ನೆನಪಿನ ಶಕ್ತಿಯುಳ್ಳವನು, ಪ್ರತಿಭಾಸಂಪನ್ನ, ಸರ್ವಲೋಕಗಳಿಗೂ ಪ್ರಿಯ, ಸಾಧುವಾದ ನಡೆಯುಳ್ಳವನು, ದೈನ್ಯರಹಿತ, ಸಮರ್ಥ, ಸಮುದ್ರವು ನದಿಗಳಿಂದ ಸುತ್ತುವರಿದಿರುವಂತೆ ಯಾವಾಗಲೂ ಸಜ್ಜನರೊಡಗೂಡಿದವನು, ಆರ್ಯನು, ಎಲ್ಲರಿಗೂ ಸಮವಾದ ನಡೆಯುಳ್ಳವನು, ಯಾವಾಗಲೂ ಪ್ರೀತಿಕರವಾದ ದರ್ಶನವುಳ್ಳವನು, ಕೌಸಲ್ಯೆಯ ಆನಂದವನ್ನು ವೃದ್ಧಿಗೊಳಿಸುವ ಆತನು ಎಲ್ಲಗುಣಗಳಿಂದಲೂ ತುಂಬಿದವನು, ಗಾಂಭೀರ್ಯದಲ್ಲಿ ಸಮುದ್ರಕ್ಕೂ, ಧೈರ್ಯದಲ್ಲಿ ಹಿಮವಂತನಿಗೂ, ವೀರ್ಯದಲ್ಲಿ ವಿಷ್ಣುವಿಗೂ ಸಮನಾದವನು, ಚಂದ್ರನಂತೆ ನೋಟಕ್ಕೆ ಪ್ರೀತಿಕರ, ಕೋಪದಲ್ಲಿ ಕಾಲಾಗ್ನಿಗೂ, ಕ್ಷಮೆಯಲ್ಲಿ ಭೂಮಿತಾಯಿಗೂ ಸಮ, ತ್ಯಾಗದಲ್ಲಿ ಕುಬೇರನಿಗೆ ಸಮನು, ಸತ್ಯದಲ್ಲಿ ಮತ್ತೊಬ್ಬ ಮೂರ್ತಿವೆತ್ತ ಧರ್ಮದಂತಿರತಕ್ಕವನು.) 

\section*{ಆದಿಕವಿಯ ಉಪಮಾನದ ಔಚಿತ್ಯ} 

ಇಂತಹ ಮಹಾಪುರುಷನ ವಿಷಯವನ್ನು ಹೇಳಿ ನಾರದನು ಹೊರಟುಹೋದನು. ನಂತರ ವಾಲ್ಮೀಕಿಗಳು ಅದೇ ಮಹಾಪುರುಷನ ರಮಣೀಯವಾದ ಗುಣವನ್ನು ಚಿಂತಿಸುತ್ತಾ ತಮಸಾನದಿಯ ತೀರಕ್ಕೆ ಹೋದರು. ಅಲ್ಲಿರುವ ತಮಸಾನದಿ ಅವರಿಗೆ ಹೇಗೆ ಕಾಣುತ್ತಿದೆ?- 

\begin{shloka} 
ಅಕರ್ದಮಮಿದಂ ತೀರ್ಥಂ ಭರದ್ವಾಜ ನಿಶಾಮಯ|\label{156}\\ 
ರಮಣೀಯಂ ಪ್ರಸನ್ನಾಂಬು ಸನ್ಮನುಷ್ಯಮನೋ ಯಥಾ||
\end{shloka} 

(ಭರದ್ವಾಜ ನೋಡು! ಕಲುಷರಹಿತವಾದ ಈ ತೀರ್ಥವು ಸತ್ಪುರುಷನ ಮನಸ್ಸಿನಂತೆ ರಮಣೀಯವೂ, ಪ್ರಸನ್ನವಾದ ನೀರುಳ್ಳದ್ದೂ ಆಗಿದೆ.) 

ಮಹರ್ಷಿಗಳ ಮನಸ್ಸೂ ಪ್ರಸನ್ನವಾಗಿದೆ. ಅಕಲುಷವಾಗಿದೆ. ಆದ್ದರಿಂದಲೇ ಸಂದರ್ಭೊಚಿತವಾಗಿ `ಸನ್ಮನುಷ್ಯಮನೋ ಯಥಾ'-ಎಂಬ ಉಪಮಾನ ಹೊರಬಂದಿದೆ. 

\section*{ಸಹಜವಾಗಿ ಹೊರಹೊಮ್ಮಿದ ಕಾವ್ಯನದಿಯ ನಾಂದೀಶ್ಲೋಕ} 

ಆ ಸಂದರ್ಭದಲ್ಲಿ ವೈಶಿಷ್ಟ್ಯ ಪೂರ್ಣವಾದ ದೃಶ್ಯವೊಂದನ್ನು ವಾಲ್ಮೀಕಿಗಳು ಕಂಡರು- 

\begin{shloka} 
ತಸ್ಯಾಭ್ಯಾಶೇ ತು ಮಿಥುನಂ ಚರಂತಮನಪಾಯಿನಮ್‍|\label{156a}\\ 
ದದರ್ಶ ಭಗವಾನ್ ತತ್ರ ಕ್ರೌಂಚಯೋಶ್ಚಾರುನಿಸ್ವನಮ್‍||
\end{shloka} 

(ಆ ತಮಸಾತೀರದಲ್ಲಿ ಪರಸ್ಪರ ಅಗಲದಿರುವ, ಇಂಪಾಗಿ ಧ್ವನಿಮಾಡುತ್ತಿರುವ, ಕ್ರೌಂಚಪಕ್ಷಿಗಳ ಜೋಡಿಯನ್ನು ವಾಲ್ಮೀಕಿಗಳು ನೋಡಿದರು.) ಅವರ ಮನದಲ್ಲಿ ತುಂಬಿರುವ ರಮಣೀಯವಾದ ಗುಣಗಳುಳ್ಳ ಪುರುಷನ ಮಹಾಚರಿತ್ರೆಗೆ, ಪ್ರಕೃತಿಯ ರಮಣೀಯತೆಯೂ ಸೇರಿದೆ. ನಂತರ ಅವುಗಳಲ್ಲಿ ಗಂಡು ಹಕ್ಕಿಯನ್ನು ಪಾಪನಿಶ್ಚಯನೂ ವೈರನಿಲಯನೂ ಆದ ಬೇಡನು ಹೊಡೆದ. ಅದನ್ನು ಅಧರ್ಮವೆಂದು ತಿಳಿದ, ಅಂತಹ ಕರುಣಾಪೂರ್ಣನಾದ ಮಹರ್ಷಿಯಿಂದ ವಾಣಿಯೊಂದು ಹೊರಟಿತು. ಮಹರ್ಷಿಯ ಮನಸ್ಸುಕರಗಿ, ರಸವೊಸರಿ, ನಾರದರಿಂದ ಉಪದಿಷ್ಟವಾದ ಸೀತಾರಾಮಕಥಾನದಿಯು, ಈ ರಸದೊಡನೆ ಬೆರೆತು ಭಾವವೀಣೆಯು ಮಿಡಿದು ನಾದ-ಸ್ವರ-ಅಕ್ಷರರೂಪವಾದ ಕಾವ್ಯನದಿಗೆ ನಾಂದೀರೂಪವಾದ ಶ್ಲೋಕವು ಉಪಶ್ಲೋಕತ್ತ್ವವನ್ನು ಹೊಂದಿತು, ಅಂತರ್ವಾಹಿನಿಯು ಬಹಿರ್ವಾಹಿನಿಯೊಡನೆ ಸಂಗಮಿಸಿತು. 

\begin{shloka} 
ಮಾ ನಿಷಾದ ಪ್ರತಿಷ್ಠಾಂ ತ್ವಮ್‍ ಅಗಮಶ್ಯಾಶ್ವತೀಃ ಸಮಾಃ|\label{157b}\\ 
ಯತ್ಕ್ರೌಂಚಮಿಥುನಾದೇಕಮ್‍ ಅವಧೀಃ ಕಾಮಮೋಹಿತಮ್‍||
\end{shloka} 

(ಕ್ರೌಂಚಪಕ್ಷಿಗಳ ಜೋಡಿಯಿಂದ ಕಾಮಮೋಯಿತವಾದ ಒಂದನ್ನು ಯಾವಕಾರಣದಿಂದ ಹೊಡೆದು ಕೊಂದೆಯೋ, ಆ ಕಾರಣದಿಂದ ಎಲೈ ಬೇಡನೇ! ನೀನು ಬಹುಕಾಲ ಇರುವಿಕೆಯನ್ನು ಹೊಂದದಿರು.) ಅನಿರೀಕ್ಷಿತವಾಗಿಯೂ, ಅಪ್ರಯತ್ನವಾಗಿಯೂ ಅವರಿಗೇ ತಿಳಿಯದಂತೆ ಹೊರಬಿದ್ದ ಈ ಶ್ಲೋಕದ ಮೇಲೆ ತಮ್ಮ ವಿಚಾರವನ್ನು ಹರಿಸುತ್ತಾರೆ, `ಕಿಮಿದಂ ವ್ಯಾಹೃತಂ ಮಯಾ'\label{157} ನಾನು ಏನು ಹೇಳಿದೇ? ಮತ್ತೆ ಮತ್ತೆ ಅದನ್ನು ಮೆಲುಕುಹಾಕುತ್ತಾ ಹೊರಹೊಮ್ಮಿದ ವಾಣಿಯು ಶ್ಲೋಕರೂಪದಲ್ಲಿರುವುದನ್ನು ತಿಳಿದು ಅದರ ಸ್ವರೂಪದ ಬಗೆಗೆ ಆಶ್ಚರ್ಯಪಡುತ್ತಾ ಹೀಗೆ ಹೇಳುತ್ತಾರೆ. 

\begin{shloka} 
ಪಾದಬದ್ಧೋಽಕ್ಷರಸಮಃ ತಂತ್ರೀಲಯಸಮನ್ವಿತಃ|\label{157a}\\ 
ಶೋಕಾರ್ತಸ್ಯ ಪ್ರವೃತ್ತೋ ಮೇ ಶ್ಲೋಕೋ ಭವತು ನಾನ್ಯಥಾ||
\end{shloka}

(ಪಾದಬಂಧವುಳ್ಳದ್ದೂ, ಸಮಾಕ್ಷರವುಳ್ಳದ್ದೂ, ತಂತ್ರೀಲಯಗಳಿಗೆ ಹೊಂದುವ ನಡೆಯುಳ್ಳದ್ದೂ ಆಗಿ, ಶೋಕಾರ್ತನಾದ ನನ್ನಿಂದ ಹೊರಬಿದ್ದ ಈ ವಾಣಿಯು ಶ್ಲೋಕವಾಗಲಿ. ಬೇರೆಯಾಗದಿರಲಿ.) ಎನ್ನುತ್ತಾರೆ. 

\section*{ಆದಿಕಾವ್ಯದ ಆವಿರ್ಭಾವಕ್ಕೆ ಸೃಷ್ಟೀಶನ ಪ್ರೇರಣೆ ಮತ್ತು ಅನುಗ್ರಹ} 

ದಾರಿಯುದ್ದಕ್ಕೂ ಆ ಶ್ಲೋಕವನ್ನೇ ನೆನೆಸಿಕೊಂಡು ಅದನ್ನೇ ಮೆಲುಕುಹಾಕುತ್ತಾ ಆಶ್ರಮಕ್ಕೆ ಬಂದರು. ಬ್ರಹ್ಮನೂ ವಾಲ್ಮೀಕಿಗಳ ಆಶ್ರಮಕ್ಕೆ ಬಂದನು. ವಾಲ್ಮೀಕಿ ಮಹರ್ಷಿಗಳು ಅವನಿಗೆ ಸತ್ಕಾರಮಾಡಿ ಆಶ್ಚರ್ಯದಿಂದ ಸ್ತಬ್ಧರಾಗಿದ್ದಾರೆ. ಬ್ರಹ್ಮನೇ ಎದುರಿಗೆ ಬಂದು ಕುಳಿತಿದ್ದರೂ ತನ್ನಿಂದ ಹೊರಬಂದ ಶ್ಲೋಕದ ಬಗೆಗೆ ವಾಲ್ಮೀಕಿಗಳು ಚಿಂತಿಸುತ್ತಿದ್ದಾರೆ. ಇವರ ಮನದ ಅಳಲನ್ನು ತಿಳಿದು ಆ ಬಗೆಗೆ ಬ್ರಹ್ಮದೇವರ ಉತ್ತರ ಹೀಗಿದೆ- 

\begin{shloka} 
ಶ್ಲೋಕ ಏವ ತ್ವಯಾ ಬದ್ಧೋ ನಾತ್ರ ಕಾರ್ಯಾ ವಿಚಾರಣಾ|\label{157c}\\ 
ಮಚ್ಛಂದಾದೇವ ತೇ ಬ್ರರ್ಹ್ಮ ಪ್ರವೃತ್ತೇಯಂ ಸರಸ್ವತೀ||
\end{shloka} 

(ನಿನ್ನಿಂದ ಶ್ಲೋಕವೇ ರಚನೆಗೊಂಡಿದೆ. ಈ ಬಗೆಗೆ ವಿಚಾರ ಬೇಡ. ಈ ಶ್ಲೋಕವು ನನ್ನ ಛಂದದಿಂದ-ನನ್ನ ಸಂಕಲ್ಪದಿಂದಲೇ ಹೊರಬಂದಿದೆ. ಇದರಲ್ಲಿ ನಿನ್ನದೇನೂ ಇಲ್ಲ. ನೀನು ಇಂತಹ ಶ್ಲೋಕಗಳಿಂದಲೇ ದಿವ್ಯವಾದ ರಾಮಚರಿತೆಯನ್ನು ರಚಿಸು)- ಎಂಬುದಾಗಿ ಅನುಗ್ರಹಿಸುತ್ತಾನೆ. ಕಾವ್ಯಕ್ಕೆ ಸೃಷ್ಟೀಶನ ಪ್ರೇರಣೆ ಮತ್ತು ಅನುಗ್ರಹ ದೊರಕಿದೆ. ರಾಮಾಯಣವು ಚಿರಸ್ಥಾಯಿಯಾಗಿರುವಂತೆ ಬ್ರಹ್ಮನ ಆಶೀರ್ವಾದವೂ ದೊರಕಿದೆ. 

\begin{shloka} 
ಯಾವತ್‍ಸ್ಥಾಸ್ಯಂತಿ ಗಿರಯಃ ಸರಿತಶ್ಚ ಮಹೀತಲೇ|\label{158b}\\ 
ತಾವದ್ರಾಮಾಯಣಕಥಾ ತ್ವತ್ಕೃತಾ ಪ್ರಚರಿಷ್ಯತಿ||
\end{shloka} 

(ಗಿರಿಗಳೂ, ನದಿಗಳೂ ಎಲ್ಲಿನವರೆಗೆ ಭೂಮಿಯಲ್ಲಿರುವುವೋ, ಅಲ್ಲಿಯವರೆಗೆ ನೀನು ರಚಿಸಿದ ರಾಮಾಯಣಕಥೆಯು ಪ್ರಚಾರದಲ್ಲಿರುತ್ತದೆ.) ಇದು ಬ್ರಹ್ಮನ ವರ. 

\section*{ಆತ್ಮಾರಾಮನ ಮೆಚ್ಚಿಗೆಗೆ ಪಾತ್ರವಾದ ರಾಮಾಯಣ} 

ಇಂತಹ ವರಪಡೆದ ವಾಲ್ಮೀಕಿಮಹರ್ಷಿಯು ಯೋಗದಶೆಯಲ್ಲಿದ್ದು, ಅಂತರ್ದೃಷ್ಟಿಯುಳ್ಳವನಾಗಿ, ಕವಿಸ್ವರೂಪನಾಗಿ, ಅತೀಂದ್ರಿಯಜ್ಞಾನದಿಂದ ಅವ್ಯಕ್ತವಾದ ತತ್ತ್ವವನ್ನು, ವ್ಯಕ್ತವಾದ ತತ್ತ್ವದೊಡನೆ ಮೇಳೈಸಿ ಅಚತುರ್ವದನನಾಗಿ, ಆದಿಕಾವ್ಯಶ್ರೀಯನ್ನು ರಚಿಸಿದನು. ಹೀಗೆ ರಚಿಸಿದ ಕಾವ್ಯಶ್ರೀಯನ್ನು ಹೊರಗೆ ಇಡುವುದು ಹೇಗೆ? ತಾನು ರಚಿಸಿದ ಮಹಾಕಾವ್ಯವನ್ನು ಪ್ರಯೋಗಬದ್ಧವಾಗಿ ಹಾಡುವ ಪಾಠಕರ ಕೈಯಲ್ಲಿ ಕೊಟ್ಟಾಗ ತಾನೇ ಕಾವ್ಯದ ಯೋಗಕ್ಷೇಮ ನಿರ್ವಹಣೆಯಾಗುತ್ತದೆ- ಎಂಬ ಚಿಂತೆ ಮಹರ್ಷಿಯನ್ನಾವರಿಸಿತ್ತು- 

\begin{shloka} 
ಚಿಂತಯಾಮಾಸ ಕೋನ್ವೇತತ್ಪ್ರಯುಂಜೀಯಾದಿತಿ ಪ್ರಭುಃ||\label{158a}
\end{shloka} 

(ಇದನ್ನು ಯಾರು ಪ್ರಯೋಗಕ್ಕೆ ತರಬಲ್ಲರು ಎಂದು ಮಹರ್ಷಿಯು ಚಿಂತಿಸಿದರು.) ಆ ವೇಳೆಗೆ ಸರಿಯಾಗಿ ಅವರಿಂದಲೇ ಶಿಕ್ಷಣ ಪಡೆದ ಯಮಲರಾದ, ರಾಮನ ಮಕ್ಕಳೇ ಆದ, ಗಾನವಿಶಾರದರೂ-ಸ್ವರಸಂಪನ್ನರೂ ಆದ ಕುಮಾರರು ಬಂದುದನ್ನು ಕಂಡು, ಮಹರ್ಷಿಗಳು ಅವರಿಗೆ ಕಾವ್ಯವನ್ನು ಉಪದೇಶಿಸಿದರು. 

ಸಂಗೀತಕ್ಕೆ ಸಾಹಿತ್ಯವನ್ನು ಕೂಡಿಸಿದವರು ವಾಗ್ಗೇಯಕಾರರಾದ ವಾಲ್ಮೀಕಿಗಳು. ಸಾಹಿತ್ಯಕ್ಕೆ ಸಂಗೀತವನ್ನು ಕೂಡಿಸಿದವರು ಗಮಕಿಗಳಾದ ಕುಶಲವರು. ಋಷಿ ಸದಸ್ಸಿನಲ್ಲಿ ಅಧಿಕೃತವಾಗಿ ಕುಶಲವರಿಂದ ಆದಿಕಾವ್ಯಗಾನವು ಮಾರ್ಗಪದ್ಧತಿಯಿಂದ ನಡೆಯಿತು. ಕಾವ್ಯಗಾನದಿಂದ ಆತ್ಮಾರಾಮರಾದ ಮಹರ್ಷಿಗಳು ತೃಪ್ತರಾಗಿ ಪ್ರಶಂಸಾರೂಪವಾದ ಧನ್ಯೋದ್ಗಾರಗಳನ್ನೂ ಅಭಿನಂದನೆಗಳನ್ನೂ ಸೂಚಿಸಿದರು. 

\begin{shloka} 
ತಚ್ಘುತ್ವಾ ಮುನಯಃ ಸರ್ವೇ ಬಾಷ್ಪಪರ್ಯಾಕುಲೇಕ್ಷಣಾಃ|\label{158}\\ 
ಸಾಧು ಸಾಧ್ವಿತಿ ತಾವೂಚುಃ ಪರಂ ವಿಸ್ಮಯಮಾಗತಾಃ|\\ 
ತೇ ಪ್ರೀತಮನಸಃ ಸರ್ವೇ ಮುನಯೋ ಧರ್ಮವತ್ಸಲಾಃ|
\end{shloka} 

ಆ ಕಾವ್ಯವನ್ನು ಕೇಳಿದ ಮುನಿಗಳೆಲ್ಲರೂ ಆನಂದಬಾಷ್ಪದಿಂದ ತುಂಬಿದ ಕಣ್ಣುಳ್ಳವರಾಗಿ, ಅತ್ಯಾಶ್ಚರ್ಯಗೊಂಡವರಾಗಿ, ಪ್ರೀತಿ ಪೂರ್ಣವಾದ ಮನಸ್ಸುಳ್ಳವರಾಗಿ, ಧರ್ಮವತ್ಸಲರಾದ ಮಹರ್ಷಿಗಳು ಸಾಧು ಸಾಧು ಎಂದು ಮುಕ್ತ ಕಂಠದಿಂದ ಕಾವ್ಯವನ್ನು ಪ್ರಶಂಸಿಸಿದರು. 

\begin{shloka} 
ಅಹೋ ಗೀತಸ್ಯ ಮಾಧುರ್ಯಂ ಶ್ಲೋಕಾನಾಂ ತು ವಿಶೇಷತಃ|\label{159}\\ 
ಚಿರನಿರ್ವೃತ್ತಮಪ್ಯೇತತ್ಪ್ರತ್ಯಕ್ಷಮಿವ ದರ್ಶಿತಂ||
\end{shloka} 

ಗಾನದ ಮಾಧುರ್ಯ, ಶ್ಲೋಕದ ಮಾಧುರ್ಯ, ಎರಡೂ ಸೇರಿ ಶ್ರವ್ಯಕಾವ್ಯವನ್ನು ಆತ್ಮಾರಾಮರಾದ ಮಹರ್ಷಿಗಳಿಗೆ ದೃಶ್ಯಕಾವ್ಯವನ್ನಾಗಿ ಮಾಡಿಬಿಟ್ಟವು. 

ಆತ್ಮಾರಾಮನಾಗಿರುವ ಮಹರ್ಷಿಯಿಂದ ಬಂದ ವೇದರೂಪವಾಗಿರುವ ರಾಮಾಯಣೇತಿಹಾಸವನ್ನು, ಸಾಕ್ಷಾತ್‍ ರಾಮನ ಸನ್ನಿಧಿಯಲ್ಲಿ ಲವಕುಶರು ಗಾನ ಮಾಡಿದರು. ಅದನ್ನು ಕೇಳಿದ ಶ್ರೀರಾಮನೇ ಸ್ವಯಂ ತುಷ್ಟನಾಗಿ ಹೇಳಿದನು- 

`ಮಮಾಪಿ ತದ್ಭೂತಿಕರಂ ಪ್ರಚಕ್ಷತೇ'-\label{159a} ಶ್ರೀರಾಮನಿಗೂ ಅದು ಶ್ರೇಯಸ್ಕರ. ಆತ್ಮಗುಣೋಪೇತವಾದ ಕಾವ್ಯ. ಆತ್ಮಭಾವವನ್ನು ಎಬ್ಬಿಸುವ ವಿಷಯ ಅಲ್ಲಿದೆ. 

ಇಂತಹ ಆರ್ಷ-ಆರ್ಯಕಾವ್ಯವನ್ನು, ಅನಾರ್ಯತೆಯನ್ನು ಕಳೆದು ಆರ್ಯರನ್ನಾಗಿಸುವ ಕಾವ್ಯವನ್ನು ವಾಲ್ಮೀಕಿ ಮಹರ್ಷಿಗಳು ಹೊರತಂದರು. ರಾಮರಾಜ್ಯ-ಧರ್ಮರಾಜ್ಯ ಉಳಿಯಲಿ ಎಂದು ರಚಿಸಿದರು. ಅಂತಹ ಜೀವನವನ್ನು ಬೆಳೆಸಿಕೊಳ್ಳಲು ಕಾವ್ಯವು ಬೇಕು. ಇಡೀ ಜೀವನಕ್ಕೆ ಒಂದು ಕೈಪಿಡಿಯಾಗಿದೇಪ್ಪಾ ರಾಮಾಯಣ. ಅಂತಹ ಆತ್ಮಾರಾಮನ ಸೊಬಗರಿತು ಬಾಳುವ ಬಾಳಾಟವನ್ನು ಭಗವಂತನು ನಿಮಗೆಲ್ಲರಿಗೂ ಕೊಡಲಿ. ಆತ್ಮಾರಾಮನಾದ ಆ ರಾಮನ ಕಡೆಗೆ ಧ್ಯಾನದಿಂದ ಅಭಿಮುಖರಾಗಿರೋಣ. 
