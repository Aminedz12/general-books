\chapter{ಸಂಪ್ರದಾಯವು ವಿಚ್ಛಿನ್ನವಾಗಲು ಕಾರಣ}

(`ಆರ್ಯಮಹರ್ಷಿಗಳು ನೇರವಾಗಿ ವಿಷಯವನ್ನು ಮುಂದುವರಿಸಿಕೊಂಡು ಬರಲು ಸಂಪ್ರದಾಯಗಳನ್ನು ವ್ಯವಸ್ಥೆಗೊಳಿಸಿದರೂ, ಅದು ಮಧ್ಯೆ  ವಿಚ್ಛಿನ್ನವಾದುದು ಹೇಗೆ?' ಎಂದು ಶ್ರೀ ಶೇಷಾಚಲಶರ್ಮರು ಕೇಳಲು ಶ್ರೀ ಗುರುವು ನುಡಿದ ಮಾತುಗಳು)

\section*{ಇಂದ್ರಿಯಗಳ ಸ್ವಭಾವ}

ಇಂದ್ರಿಯಗಳು ತಮಗೆ ಬೇಕಾದ ವಿಷಯವನ್ನು ಮಾತ್ರ ಅಪೇಕ್ಷಿಸುತ್ತವೆ. ಅದಲ್ಲದೆ ಬೇರೆ ವಿಷಯವು ಬಂದಲ್ಲಿ ಅದನ್ನು ತಳ್ಳಿ ಬಿಡುತ್ತ್ತವೆ. ಅದು ಅವುಗಳಿಗೆ ಹೊರೆಯಾಗಿಬಿಡುತ್ತದೆ. ಇಂತಹ ಇಂದ್ರಿಯಗಳ ಸ್ವಭಾವವನ್ನರಿತ ಜ್ಞಾನಿಗಳು ಅವುಗಳಿಗೆ ಬೇಕಾದ ವಿಷಯಗಳನ್ನು ಆತ್ಮಧರ್ಮಕ್ಕೆ ಅವಿರೋಧವಾಗಿ ಕೊಡುವುದಕ್ಕೊಸ್ಕರ ಅನೇಕ ಉಪಾಯಗಳನ್ನು ಬಳಸಿದ್ದಾರೆ.

\section*{ಲೋಕದ ದೃಷ್ಟಿಯಲ್ಲಿ `ಚೆನ್ನು'}

ಆದರೆ ಇಂದು ಲೋಕದಲ್ಲಿ ಚೆನ್ನು, ಸುಖ, ಹಿತ ಈ ಪದಗಳೆಲ್ಲಾ ಬಳಕೆಯಲ್ಲಿದ್ದರೂ ಅವು ಸಜೀವವಾಗಿಲ್ಲ. ಉದಾಹರಣೆಗೆ ಒಂದು ಸಾಮಾನ್ಯ ಶಿಲ್ಪವನ್ನು ನೋಡಿ ಅದನ್ನು ಚೆನ್ನು ಎನ್ನ್ಬಹುದು. ಕೆತ್ತನೆಯು ಇನ್ನೂ ಸೂಕ್ಷ್ಮವಾಗಿದ್ದರೆ ಅದನ್ನು ಇನ್ನೂ ಚೆನ್ನು ಎನ್ನಬಹುದು. ಹೀಗೆ ಒಂದೊಂದು ಹೊಸತನ ಕಂಡಂತೆಲ್ಲಾ ಅದು ಚೆನ್ನಾಗಿದೆ ಎನ್ನುವರು. ಅದೇ ಇನ್ನೊಂದು ಹೊಸತನ ಬಂದಾಗ ಹಿಂದಿನದರಲ್ಲಿ ಅರುಚಿಯು ಹುಟ್ಟಿ  ಮುಂದಿನದು ಚೆನ್ನು ಎನ್ನಿಸಿಕೊಳ್ಳುತ್ತದೆ. ಚೆನ್ನನ್ನು ಅಳೆಯುವ ಸರಿಯಾದ ಸ್ಟ್ಯಾಂಡರ್ಡ್ ಇಲ್ಲ. ನಾವು ಉಪಯೋಗಿಸುವ ಒಂದು ಬಟ್ಟೆಯನ್ನೇ  ತೆಗೆದುಕೊಳ್ಳಿ - ನಾರು ಮಡಿಯಿಂದ ಹಿಡಿದು ಟೆರ್‌ಲಿನ್ ಮೊದಲಾದ ವಿವಿಧ ರೀತಿಗಳು ಬರುತ್ತಲೇ ಇವೆ. ದಿನಕ್ಕೊಂದು ಹೊಸ ಫ್ಯಾಷನ್ ಬಂದು ಸೇರುತ್ತಲೇ ಇದೆ. ಬಟ್ಟೆಯ ವಿಷಯದಲ್ಲಿ ಇದಕ್ಕಿಂತಲೂ ಇದು ಚೆನ್ನು ಎನ್ನುವುದನ್ನು ಹೇಗೆ ಅಳೆಯುವುದು? ಬಣ್ಣದಿಂದಲೇ? ಅದರ ಬೆಲೆಯಿಂದಲೇ? ಅದು ಗಟ್ಟಿಯಾಗಿದೆಯೆಂದೇ? ಹೇಗೆ? ಎಂದರೆ ತೀರ್ಮಾನವಿಲ್ಲ. ಜನಗಳ ಚೇತೋಹಾರಿಯಾಗುವಂತಹುದು ಚೆನ್ನು ಎನ್ನಿಸಿಕೊಳ್ಳುತ್ತದೆಯಲ್ಲವೆ. ಹೀಗೆ ಕರ್ಮ-ಜ್ಞಾನೇಂದ್ರಿಯಗಳಿಗೆ ಬೇಕಾದ ವಿಷಯಗಳನ್ನು ಆಕರ್ಷಣೀಯವಾಗಿ ಮಾಡಿಕೊಡಬೇಕೆನ್ನುವ ಪ್ರಯತ್ನವು ಲೋಕದಲ್ಲಿ ನಡೆಯುತ್ತಲೇ ಇದೆ. ಈ ಪ್ರಯತ್ನದಲ್ಲಿ ಕರ್ಮ-ಜ್ಞಾನೇಂದ್ರಿಯಗಳಿಗೂ ಹಿಂದಿರುವ ಪರಮಾತ್ಮನ ಲೋಕವು ಮರೆತು ಮುಂದೆ ಸಾಗುತ್ತಿದೆ.

\section*{ಪ್ರಕೃತಿಯ ಬಗ್ಗಡವನ್ನು ಶುದ್ಧಗೊಳಿಸಿ ಜ್ಞಾನದ ಸವಿಯನ್ನು ಸವಿಯಬೇಕು}

ಅಂತಹ ವಿಷಯವನ್ನು ಲೋಕವು ಮರೆತರೂ ಅದು (ಇಂದ್ರಿಯಾತೀತವಾದ ವಿಷಯವು) ಮಾತ್ರ ತನ್ನಪಾಡಿಗೆ ತಾನು ಹರಿಯುತ್ತಲೇ ಇದೆ. ಆದರೆ ನಿಜವಾದ ಸವಿಯನ್ನು ನೋಡಿ ತಿಳಿಯಬೇಕು. ಗಂಗಾನದಿಯು ಹಿಮವತ್ಪರ್ವತದ ಉನ್ನತ ಶಿಖರದಿಂದ ಭುವಿಯ ಸಂಪರ್ಕವಿಲ್ಲದೆ ಧಿಮಿ-ಧಿಮಿ ತಾಳದೊಡನೆ ಶುದ್ಧವಾಗಿ ಹರಿದು ಬರುತ್ತದೆ. ಅದೇ ನೀರು ಕಾಲುವೆಯಿಂದ ನಮ್ಮೂರಿಗೆ ಹರಿದುಬಂದರೆ ರೂಪಾಂತರ ಹೊಂದಿ ಉಪ್ಪುಪ್ಪಾಗಿಬಿಡುತ್ತದೆ. ಆಕಾಶದಲ್ಲಿನ ಮೇಘಗಳಿಂದ ಒಂದೇ ರೀತಿಯಾದ (ಶುದ್ಧವಾದ) ನೀರು ಬಿದ್ದರೂ ಕಪಿಲಾನದಿಯ ನೀರು ಕಪ್ಪು, ಕಾವೇರಿನದಿಯ ನೀರು ಬಿಳುಪು, ತುಂಗಭದ್ರೆಯ ನೀರು ಬಗ್ಗಡ ಅಂದರೆ ಅದು ಆಯಾ ಕ್ಷೇತ್ರದ ಫಲ. ಆದರೆ ಆ ನದಿಯ ನೀರನ್ನು ಕಾಯಿಸಿ ಬಗ್ಗಡವನ್ನು ಪೃಥಕ್ಕರಿಸಿದಾಗ ಶುದ್ಧವಾದ ನೀರು ಬರುತ್ತದೆ. ಹೀಗೆ ಪೃಥಕ್ಕರಿಸುವ ಯೋಗ್ಯತೆಯು ಬೇಕು.

ಹಾಗೆಯೇ ಚೇತನ ಪ್ರಕೃತಿಯ ವಿವಿಧ ಕ್ಷೇತ್ರಗಳಲ್ಲಿ ಹರಿದು ಬರುವಾಗ ಬಗ್ಗಡವಾಗಿದೆ. ಅಲ್ಲಿಗೆ ಜ್ಞಾನಸಂಬಂಧವಾದ ಮಾತಿನ ರೂಪವಾದ ಶೋಧಕವನ್ನು ಹಾಕಿದಾಗ ಸ್ವಲ್ಪಕಾಲ ಕಳೆದರೆ ತಿಳಿದುಕೊಳ್ಳುತ್ತದೆ. (ತಿಳಿಯಾಗುತ್ತದೆ). ಪ್ರಕೃತಿಯು ಬಗ್ಗಡವಾಗಿದ್ದರೂ ಅದು ಸಹಜ ಶುದ್ಧವಾಗಿಯೇ ಹರಿಯುತ್ತದೆ. ಉಪಾಯಗಳಿಂದ ಪ್ರಕೃತಿಯ ಬಗ್ಗಡವನ್ನು ಶಾಂತಗೊಳಿಸಿ ಶುದ್ಧತೆಯನ್ನರಿಯಬೇಕು.

\section*{ಜ್ಞಾನಸಂಬಂಧವಾದ ವಿಷಯವನ್ನು ಯುಕ್ತಿಕಲ್ಪಿತವಾದ ಪ್ರಯೋಗಗಳಿಂದ ಕೊಡಬೇಕು}

ಆದ್ದರಿಂದ ಇಂದು ಲೋಕಕ್ಕೆ ಜ್ಞಾನಸಂಬಂಧವಾದ ವಿಷಯವನ್ನು `ಪ್ರಯೋಗೈರ್ಯುಕ್ತಿಕಲ್ಪಿತೈಃ' ಎಂಬಂತೆ ಯುಕ್ತಿಕಲ್ಪಿತವಾದ ಪ್ರಯೋಗಗಳಿಂದ ಕೊಡಬೇಕು. ಹಿಂದಿನಂತೆ `ಕೂಜನ್ತಂ ರಾಮರಾಮೇತಿ' ಎಂದುಬಿಟ್ಟರೆ ಕೇಳೋಲ್ಲ. (ವಿಷಯವನ್ನು ಕೊಡುವ ರೀತಿ ಸರಿಯಾಗಿರಬೇಕು ಎಂಬ ಆಶಯದಿಂದ). ಆದ್ದರಿಂದ ಪ್ರಕೃತಿಮಾತೆಯ ಸಹಾಯವು ಬೇಕು. ಅಲ್ಲಿ ಆತ್ಮನ ಪ್ರಕಾಶ ಚೆಲ್ಲಿದಾಗ ಅದು ಬೆಳಗಬಹುದು. ಆ ಸ್ವಾಮಿಯ ಪ್ರಕಾಶ ಬೀಳಲು ತಾಯಿ! ನಿನ್ನ ಅನುರಾಗವೂ ಬೇಕು ಅದನ್ನು ಅನುಸರಿಸಿ ಇಡಬೇಕು. ಮಕ್ಕಳಿಗೆ ಸಕ್ಕರೆ ಎಂದರೆ ಪ್ರೀತಿ. ಆದರೆ ಔಷಧಿ ಬೇಡ. ಆದ್ದರಿಂದ ಷುಗರ್ ಕೋಟೆಡ್ ಔಷಧಿಯನ್ನು ಮಾಡಿಟ್ಟಿದ್ದಾರೆ. ಕಥೆಯರೂಪದಲ್ಲಿಯೋ, ಇನ್ನಾವುದಾದರೂ ರೂಪದಲ್ಲಿಯೋ ಜನಗಳಿಗೆ ಬೇಕಾದ ವಿಷಯದ ಹಿಂದುಗಡೆ ತತ್ತ್ವವನ್ನಿಟ್ಟೇ ಮನಸ್ಸನ್ನು ತಮ್ಮೆಡೆಗೆ ಎಳೆಯಲು ಪ್ರಯತ್ನಿಸಿದ್ದಾರೆ.

ಒಬ್ಬ ಮನುಷ್ಯನು ಪರಮಾತ್ಮನೇ ಆದರೂ ನರನಾಗಿ ಬಂದು ನಿಂತರೆ ಜನಗಳ ಕಣ್ಣೀಗೆ ಏನು ಕಾಣುತ್ತದೆ? ಅವರ ಕಣ್ಣಿಗೆ ಕಾಣುವುದು ಕೃತಿ. ಅದರ ಹಿಂದೆ ಹೋಗಿ ನೋಡಿದಾಗಲೇ ಮತಿ ಸಿಕ್ಕುತ್ತದೆ. ಆದರೆ ಜನಗಳ ದೃಷ್ಟಿ ಹಾಗೆ ಹರಿಯುವುದಿಲ್ಲ. ಆದ್ದರಿಂದಲೇ ಯುಕ್ತಿಕಲ್ಪಿತವಾದ ಪ್ರಯೋಗ ಉದಾಹರಣೆಗೆ ಕಾಫಿಯ ಮೇಲೆ ಪ್ರೀತಿ ತುಂಬ ಬೆಳೆದಿದ್ದಾಗ ಅದರ ಜೊತೆಗೆ ಸ್ವಲ್ಪ ವಿಭೂತಿ ಹಾಕಿಕೊಡಿ, ಆಗ ಮೈಗೆ ಸೇರುತ್ತೆ. ನೇರವಾಗಿ ವಿಭೂತಿ ಎಂದರೆ ಕೇಳುವುದಿಲ್ಲ. ಹಾಗೆಯೇ ಭಗವಂತನ ವಿಭೂತಿಯನ್ನು (ಜ್ಞಾನೈಶ್ವರ್ಯವನ್ನು) ಇಂದ್ರಿಯಗಳ ವಿಷಯಗಳ ಜೊತೆಗೇ ಸೇರಿಸಿಕೊಟ್ಟು ಬಿಡಿ. ಆಗ ಫಲಿಸುತ್ತೆ. ಹೀಗೆ ತನ್ನ ಆಸ್ತಿಯನ್ನು ಕೊಡಲು ಭಗವಂತನು ಸಮಯ ಕಾಯುತ್ತಾನೆ.

ನಾಥಮುನಿಗಳು ತಮ್ಮ ಆಸ್ತಿಯನ್ನು ಮುಂದುವರಿಸಲು ಇಚ್ಛೆಪಟ್ಟರೂ ಮಗನಿಗೆ ಆ ಸಂಸ್ಕಾರವಿಲ್ಲ. ಆದ್ದರಿಂದ ಮೊಮ್ಮಗನಿಗೆ ಕೊಡಲು ರಾಮಮಿಶ್ರರ ಮೂಲಕ ವಿಷಯವನ್ನು ಮುಂದುವರಿಸಲು ಯತ್ನಿಸಿದರು. ಶಿಷ್ಯರ ಯೋಗಭೂಮಿಕೆಯಲ್ಲಿಟ್ಟು ಅದನ್ನು ಮುಂದುವರಿಸಲು ಆಜ್ಞಾಪಿಸಿದರು. ಶಿಷ್ಯರೂ ಭಗವಂತನನ್ನು ಅನುಭವಿಸಿ ಕೊಂಡಿದ್ದವರು. ಅವರಿಗೆ ಆ ಆಸ್ತಿಯನ್ನು ಕೊಟ್ಟು ಮೊಮ್ಮಗನಿಗೆ ಕೊಟ್ಟುಬಿಡಿ ಎಂದು ಅಪ್ಪಣೆಮಾಡಿದ. ಶಿಷ್ಯರೂ ಕಾದರು, ಧನವೂ ಇತ್ತು, ಸಂಸ್ಕಾರವೂ ಇತ್ತು, ಆದರೆ ಅವರ ಮೊಮ್ಮ್ಗನಾದರೋ ರಾಜನಾಗಿ ರಾಜಾಸ್ಥಾನದಲ್ಲಿದ್ದಾನೆ. ಒಳಗೆ ಪ್ರವೇಶವೇ ಇಲ್ಲ. ಹೀಗೆ ಇಂದ್ರಿಯಪ್ರಪಂಚದಲ್ಲೇ ಇರುವ ಅವರನ್ನು ವಶಮಾಡಿಕೊಳ್ಳಲು ಬಾಯಿಗೆ ರುಚಿಯನ್ನುಂಟುಮಾಡುವ ಕೊಡುವಳಾ ಕೀರೆ (ಹಬ್ಬುಸುಂಡೆ ಕೀರೆ)ಯನ್ನು ಪ್ರತಿದಿನವೂ ಕೊಡುತ್ತಾ ಬಂದರು. ಅದಕ್ಕೆ ಪ್ರತಿಯಾಗಿ ಏನನ್ನೂ ತೆಗೆದುಕೊಳ್ಳುತ್ತಿರಲಿಲ್ಲ. ರಾಜನಿಗೂ ಈ ಸೊಪ್ಪು ಬಹುರುಚಿಕರವಾಗಿತ್ತು. ಮಧ್ಯೇ ಅವರೊಂದುದಿನ ಸೊಪ್ಪನ್ನು ತಂದುಕೊಡಲಿಲ್ಲ. ರಾಜನು ಅದೇಕೆ ಇಂದು ಆ ಸೊಪ್ಪಿನ ಅಡಿಗೆ ಮಾಡಲಿಲ್ಲ? ಎಂದು ವಿಚಾರಿಸಿದನು. ಸ್ವಾಮಿ, ಒಬ್ಬರು ಇದುವರೆಗೂ ಯಾವ ಪ್ರತಿಧನವನ್ನೂ ತೆಗೆದುಕೊಳ್ಳದೇ ಸೊಪ್ಪನ್ನು ತಂದುಕೊಡುತ್ತಿದ್ದರು. ಈ ದಿನ ಅವರು ಬರಲಿಲ್ಲ ಎಂಬುದಾಗಿ ಹೇಳಲು ಮಾರನೆಯ ದಿನ ಅವರು ಬಂದರೆ ತನ್ನಲ್ಲಿಗೆ ಕರೆದು ತರುವಂತೆ ರಾಜಾಜ್ಞೆಯಾಯಿತು. ಮರುದಿವಸ ಅದೇರೀತಿ ಬಂದಾಗ ಅವರು ರಾಜನನ್ನು ಬೇಟಿಮಾಡುವಂತಾಯಿತು. ರಾಜ್ಯಕ್ಕೆ  ಯಾವುದೋ ಧನದ ಅವಶ್ಯಕತೆಯೂ ಇತ್ತು, ತನ್ನ ಜೊತೆ ಬಂದರೆ ಉತ್ತಮವಾದ ನಿಧಿ ತೋರಿಸುತ್ತೆನೆಂದು ಹೇಳಿ ಅವರನ್ನು ಕರೆದೊಯ್ದು ಅವರ ತಾತನವರು ಕೊಟ್ಟಿದ್ದ ಧನವನ್ನು ಒಂದು ರೂಪದಲ್ಲಿ ಕೊಟ್ಟರು. ಈ ರೀತಿ ಶಿಷ್ಯರು ಗುರುಗಳ ಮಾತನ್ನು ಪಾಲಿಸಿದರು.

\section*{ಸ್ವರೂಪವು ಕೆಡದಂತೆ ಜೀವನವನ್ನಿಟ್ಟುಕೊಳ್ಳಬೇಕು}

ನಮ್ಮಾಳ್ವಾರರು `ಅತ್ತೈಪಾರ್ತ್ ಅತ್ತೆಯೇ ಶಾಪ್ಪಿಟ್ಟುಕ್ಕೊಂಡಿರುಕ್ಕಿರೇನ್' (ಅದನ್ನು ನೋಡುತ್ತಾ ಅದನ್ನೇ ಅನುಭವಿದಿಕೊಂಡಿದ್ದೇನೆ) ಎಂದು ಇಂದ್ರಿಯಾತೀತವಾದ ಚೇತನವಿರುವುದರಿಂದ ಅದನ್ನವಲಂಬಿಸಿ ಇರುವ ವಿಷಯವನ್ನು ಸಾರಿದ್ದಾರೆ. ಅಂತಹ ಸ್ಥಿತಿಯನ್ನು ನಾವು ಕಂಡೇವೇ? ಎಂದರೆ, ಅದಕ್ಕಾಗಿ ಯತ್ನಿಸಬೇಕು.

\begin{shloka}
ಯಾ ನಿಶಾ ಸರ್ವಭೂತಾನಾಂ ತಸ್ಯಾಂ ಜಾಗರ್ತಿ ಸಂಯಮೀ |\\
ಯಸ್ಯಾಂ ಜಾಗ್ರತಿ ಭೂತಾನಿ ಸಾ ನಿಶಾ ಪಶ್ಯತೋ ಮುನೇಃ ||
\end{shloka}

ಅವರ (ಜ್ಞಾನಿಗಳ) ಮಾತು ಇವರಿಗೆ (ಅಜ್ಞಾನಿಗಳಿಗೆ) ಕತ್ತಲೆ, ಇವರ ಮಾತು ಅವರಿಗೆ ಕತ್ತಲೆ, ಏನು ಲುಪ್ತವಾಗಿದೆ? ಎಲ್ಲಿ ಲುಪ್ತವಾಗಿದೆ? ಎಂಬುದನ್ನು ಪತ್ತೆಮಾಡಬೇಕು.

ಮೂಗಿನಿಂದ ಆತ್ಮನ ವಿಷಯವನ್ನು ಹಿಡಿಯಬಹುದು, ಆದರೆ ಮೂಗಿಗೆ ಆತ್ಮ ವಿಷಯವಲ್ಲ. ಹಿಂದಿನ ಪದಕ್ಕೆ (ಸ್ಥಿತಿಗೆ) ಹೋಗಿ ನೋಡಬೇಕು, ಅದಕ್ಕೆ ಸ್ಥಿತಿಬೇಕು, ಸ್ಥಿತಿಯು ಕೆಟ್ಟಾಗ ಅರ್ಥವಾಗುವುದಿಲ್ಲ. ನಾಲಿಗೆಯು ಒಂದು ಸ್ಥಿತಿಯಲ್ಲಿದ್ದಾಗ ರಸಗ್ರಹಣವುಂಟು. ಹಾಗೆಯೇ ಕಣ್ಣಿಗೂ ಒಂದು ಸ್ಥಿತಿಯುಂಟು. ಆದರೆ ಅದರ ಸ್ಥಿತಿಯು ಕೆಟ್ಟಾಗ ಶುದ್ಧವಾದ (ಬಿಳುಪಾದ) ಶಂಖವನ್ನೇ ತಂದಿಟ್ಟರೂ ಪೀತಶ್ಯಂಖಃ ಎಂದು ಬಿಡುತ್ತಾನೆ. ಆದ್ದರಿಂದ ಈ ಸ್ವರೂಪ ಕೆಡಿಸಿಕೊಳ್ಳದೇ ಇರಬೇಕು. ಹೀಗೆ ಸ್ವರೂಪವನ್ನು ಕೆಡಿಸಿಕೊಳ್ಳದೇ ಇರುವವರೇ ಮಹಾತ್ಮರು.

\section*{ಮಹಾತ್ಮರ ಅಂತರಂಗ-ಬಹಿರಂಗಗಳಲ್ಲಿ ಏಕರೂಪತೆ}

ಕಾಳಿದಾಸ ಮಹಾಕವಿಯು ಮಹಾತ್ಮರ ವರ್ಣನೆಯ ಪ್ರಸಂಗದಲ್ಲಿ (ದಿಲೀಪನನ್ನು ವರ್ಣಿಸುವಾಗ) `ಆಕಾರಸ್ದೃಶಪ್ರಜ್ಞಃ' ಎಂಬುದಾಗಿ ವರ್ಣಿಸಿದ್ದಾನೆ. ವರ್ಣನೆಯು ಗಂಭೀರವಾಗಿದೆ. ನಕ್ಕರೆ ಒಳಗಿನಿಂದ ಹೊರಗಿನವರೆಗೂ ಏಕರೂಪವಾಗಿರಬೇಕು. ಹಾಗೆ ಏಕರೂಪವಾಗಿದ್ದರೆ ತಾನೇ ನೆಮ್ಮದಿ. ಹೊರಗಿನ ನಗುವಿನ ಆಕಾರ ಒಳಗಿನ ಬುದ್ಧಿಯ ಪ್ರಸನ್ನತೆಯನ್ನು ಸೂಚಿಸುವಂತಿರಬೇಕು. ಹೃದಯವೇ ಮೂರ್ತಿವೆತ್ತುಬಂದಾಗ ಯಾವ ಒಂದು ಆಕಾರವಿರುತ್ತದೋ ಹಾಗೆಯೇ ಇರಬೇಕು. ಪ್ರಜ್ಞೆಗೂ ಆಕಾರಕ್ಕೂ ವಿರೋಧಭಾಸವಿರಲಿಲ್ಲ, ಆದ್ದರಿಂದಲೇ-

\begin{shloka}
ಧರ್ಮೋ ವಿಗ್ರಹವಾನಧರ್ಮ ವಿರತಿಂ ಧನ್ವೀ ಸ ತನ್ವೀತ ನಃ |
\end{shloka}
ಎಂಬಂತೆ ರಾಮನನ್ನು ಧರ್ಮದ ವಿಗ್ರಹವಾಗಿ ವರ್ಣಿಸಿದ್ದಾರೆ.

\section*{ಗ್ರಹಿಸುವವರಲ್ಲಿನ ವಿಕಾರದಿಂದಾಗಿ ವಸ್ತುವಿನ ಸ್ವರೂಪಗ್ರಹಣವಿಲ್ಲ}

ಹೀಗೆ ರಾಮನೋ ಕೃಷ್ಣನೋ ಅವತರಿಸಿ ಬಂದರೂ, `ಆಕಾರಸದೃಶಪ್ರಜ್ಞಃ' ಎಂಬಂತಿದ್ದರೂ ಕೆಲವರ ಕಣ್ಣಿನಲ್ಲಿ ವಿಕಾರವೇರ್ಪಟ್ಟಾಗ ವಸ್ತುವಿನ ಸ್ವರೂಪವನ್ನು ಗ್ರಹಿಸಿಕೊಳ್ಳಲು ಆಗುವುದಿಲ್ಲ. ರಾಮಕೃಷ್ಣಾದಿಗಳನ್ನು ಎಲ್ಲರೂ ಏಕರೂಪವಾಗಿ ಗ್ರಹಿಸದಿರಲು ವಿಕಾರವೇರ್ಪಡುವುದೇ ಕಾರಣವಾಗಿದೆ, ಇದನ್ನೇ ಭಗವಂತನು-

\begin{shloka}
ಅವಜಾನಂತಿ ಮಾಂ ಮೂಢಾಃ ಮಾನುಷೀಂ ತನುಮಾಶ್ರಿತಮ್ |\\
ಪರಂಭಾವಮಜಾನನ್ತಃ ಮಮಾವ್ಯಯಮನುತ್ತಮಮ್ ||
\end{shloka}

ಎಂಬುದಾಗಿ ಸಾರಿದ್ದಾನೆ. ಉತ್ತಮವಾದ ವಿಷಯವು ಎಷ್ಟೋ ಕಾಲದಿಂದ ಬಂದರೂ ಸಂಪ್ರದಾಯವು ವಿಚ್ಛೇದವಾದುದು ಹೇಗೆ? ಎಂಬ ಪ್ರಶ್ನೆಗೆ ಉತ್ತರವಾಗಿ ಹೇಳಿದ ಸಂದರ್ಭದಲ್ಲಿ ಮಾತು ಬಂತು.

\section*{ಸ್ವರೂಪವು ಕೆಡದಂತೆ ಇಟ್ಟುಕೊಂಡಾಗಲೇ ಆತ್ಮದರ್ಶನ}

ಪದಾರ್ಥಗಳನ್ನು ಅದರ ಸ್ವರೂಪವರಿತು ರಕ್ಷಿಸುವ ಕ್ರಮ ಬೇಕು. ಭಗವಂತನ ಆರಾಧನೆಗೆ ತುಳಸಿ, ಹೂವುಗಳನ್ನು ಕೊಡು ಎಂದರೆ ಜಟ್ಟಿ ಹಿಡಿಯುವಂತೆ ಅವುಗಳನ್ನು ಹಿಡಿದರೆ ಬಾಡಿ ಹೋಗುತ್ತದೆ. ಜಟ್ಟಿಯಂತಿದ್ದರೆ ಆಗುವುದಿಲ್ಲ. ಜಟ್ಟಿತನ ಅಲ್ಲಿ ಬೇಡ ಎನ್ನಬೇಕು. ಇದೇರೀತಿ ಇತರರ, ಪ್ರಕೃತಿಯ ಉಪಟಳಕ್ಕೆ ಸಿಕ್ಕಿದಾಗ ವಿಷಯವು ಬಾಡುತ್ತದೆ - ಕೆಡುತ್ತದೆ. ಆದ್ದರಿಂದ ಪ್ರಕೃತಿಯನ್ನು ಶುದ್ಧವಾಗಿಟ್ಟುಕೊಳ್ಳಳು ಆಚಾರ ಆಚಾರವೆಂದರೇನು? ಎನ್ನುವುದನ್ನು ಒಂದು ನಿರ್ವಚನವು ಹೀಗೆ ಹೇಳುತ್ತದೆ-

\begin{shloka}
ಪಂಚೇಂದ್ರಿಯಸ್ಯ ದೇಹಸ್ಯ ಬುದ್ಧೇಶ್ಚ ಮನಸಸ್ತಥಾ |\\
ದ್ರವ್ಯ-ದೇಶ-ಕ್ರಿಯಾಣಾಂ ಚ ಶುದ್ಧಿರಾಚಾರ ಇಷ್ಯತೇ ||
\end{shloka}

ಈ ಎಲ್ಲವು ಶುದ್ಧವಾಗಿಟ್ಟು ಪ್ರಕೃತಿ ರೂಪವಾದ ಈ ಕನ್ನಡಿಯನ್ನು ಚೆನ್ನಾಗಿಟ್ಟರೆ ಪ್ರತಿಬಿಂಬವು ಚೆನ್ನಾಗಿ ಬೀಳುತ್ತದೆ. ಆತ್ಮನ ಪ್ರತಿಬಿಂಬ-ಪ್ರಕಾಶಗಳು ಸ್ಫುಟವಾಗಿ ಕಾಣುತ್ತವೆ.

