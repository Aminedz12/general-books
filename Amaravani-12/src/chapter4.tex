\chapter{ಶ್ರೀ ಶಂಕರರು ಮತ್ತು ಶ್ರೀ ರಾಮಾನುಜರು}

(ದಿನಾಂಕ ೮-೫-೧೯೬೨ ರಂದು ಹೆಡತಲೆಯಲ್ಲಿ ನಡೆದ ಪಾಠ. ಅಂದು ಶಂಕರ ಜಯಂತಿ ಮತ್ತು ರಾಮಾನುಜಾಚಾರ್ಯರ ತಿರುನಕ್ಷತ್ರವೆರಡೂ ಸೇರಿದ ದಿನ. ಆ ದೃಷ್ಟಿಯಿಂದ ಶಂಕರ-ರಾಮಾನುಜರ ಜೀವನ ಮತ್ತು ದರ್ಶನಗಳ ನಿಷಯವಾಗಿ ಸಾಂಪ್ರದಾಯಿಕವಾದ ಸರಣಿ ಹೇಗೆ ಹರಿದಿದೆ? ಎಂಬ ಬಗ್ಗೆ  ಶ್ರೀಯುತರಾದ, ಕೆ.ಎಸ್. ವರದಾಚಾರ್ಯರು, ಛಾಯಾಪತಿಗಳು, ರಾಮಭದ್ರಾಚಾರ್ಯರು ಮತ್ತು ಶೇಷಾಚಲ ಶರ್ಮರು ಮಾತನಾಡಿದ ನಂತ ಈ ಬಗ್ಗೆ   ಮಂದಿರದ ದೃಷ್ಟಿಕೋನವೇನು? ಎಂಬುದನ್ನು ಕುರಿತು ಗುರುಭಗವಂತನು ಪ್ರಯೋಗಬದ್ಧವಾದ ವಾಣಿಯಿಂದ ಅಪ್ಪಣೆ ಕೊಡಿಸಿದ ವಿಷಯಗಳು.)

\section*{ವಿಷಯವೊಂದರ ಬಗ್ಗೆ ನಿರ್ಣಯಿಸಲು ಇರಬೇಕಾದ ಜವಬ್ದಾರಿ}

(ಡೆಸ್ಕಿನ ಮೇಲಿದ ಗಡಿಯಾರವನ್ನು ನೋಡಿ) ಕಾಲಚಕ್ರವು ಎದುರಿಗಿದ್ದರೆ ಮಾತನಾಡುವುದಕ್ಕಾಗುವುದಿಲ್ಲ. (ಎಂದು ಗರಿಯಾರವನ್ನು ಬೇರೆಡೆ ತೆಗೆದಿಟ್ಟು, ತಂಬೂರಿಯನ್ನು ತರಲು ಹೇಳಿ) ನಮ್ಮ ಸಂಪ್ರದಾಯದಲ್ಲಿ `ಅಡಿಯೇನ್, ದಾಸನ್'-ನಾನು ನಿಮ್ಮ ದಾಸ ಎಂದು ಹೇಳುವುದುಂಟು. ದಾಸಯ್ಯ  ಇದ್ದಲ್ಲಿ ತಂಬೂರಿ ಇರಬೇಕು. ತುಂಬುರರುನಾರದರು ಎಲ್ಲರೂ ಹತ್ತಿರದಲ್ಲಿರಬೇಕು. ಎರಡು ಕಣ್ಣಿದ್ದ ಮನುಷ್ಯ ನಾಲ್ಕು ಕಣ್ಣು ಹಾಕಿಕೊಂಡು ಬಂದರೂ ಮಾತನಾಡುವುದಕ್ಕಾಗುವುದಿಲ್ಲ (ಎಂದು ವಿನೋದವಾಗಿ ಹೇಳಿ) ಶ್ರೀ ಸೀತಾರಾಮುಗಳು `ಶಂಕರರ ಮತ್ತು ರಾಮಾನುಜರ ವಿಷಯವಾಗಿ ಮಾತನಾಡಬೇಕು' ಎಂದು ಕೇಳಿಕೊಂಡುದರ ಮೇಲೆ ಇಲ್ಲಿ ಬಂದು ಕುಳಿತದ್ದಾಗಿದೆ. ಶಂಕರರ ಅಥವಾ ರಾಮಾನುಜರ ಪೈಕಿ ಯಾವುದಾದರೂ ಒಂದು ಪಕ್ಷದ ಬಗ್ಗೆ ಮಾತನಾಡು ಎಂದು ಹೇಳಿದ್ದರೆ ಸುಲಭವಗಿತ್ತು. ಇಬ್ಬರ ಜೀವನವನ್ನೂ ಒಬ್ಬರೇ ಒಂದೇ ಟೈಂನಲ್ಲಿ ಇಡುವುದು ಕಷ್ಟಸಾಧ್ಯವಾದ ಅಂಶವಾಗಿದೆ. ಇಂದು ಶಂಕರ ಜಯಂತಿ ಹಾಗೂ ರಾಮಾನುಜರ ತಿರುನಕ್ಷತ್ರ. ಈ ದೃಷ್ಟಿಯಿಂದ ಎರಡೂ ಒಂದು ಕಡೆ ಸೇರಿದಾಗ ಇಬ್ಬರೂ ವಾದಿಪ್ರತಿವಾದಿಗಳಂತಿದ್ದಾರೆ. ಒಬ್ಬರು ಹೌದು ಎಂದದ್ದನ್ನು ಇನ್ನೊಬ್ಬರು ಇಲ್ಲ ಎನ್ನುತ್ತಾರೆ. ಸರಿಯಾದ ಜಡ್ಜುಗಳಾದವರು ಈ ವಾದಿ-ಪ್ರತಿವಾದಿಗಳಿಬ್ಬರ ಅಹವಾಲುಗಳನ್ನು ತೆಗೆದುಕೊಂಡು ಸತ್ಯವನ್ನು ಪರಿಶೀಲಿಸಿ ತೀರ್ಮಾನವನ್ನು ಕೊಡಬೇಕಾಗಿದೆ.

ಹಾಗೆ ತೀರ್ಮಾನ ಕೊಡುವಾಗ ಸಾಕ್ಷಿಗಳ ಕಡೆಯಿಂದಲೂ ವಿಷಯವನ್ನು ತೆಗೆದುಕೊಂಡು ಸತ್ಯವನ್ನು ನಿರ್ಣಯಿಸಬೇಕಾಗುತ್ತದೆ. ಹಾಗೆ ತನ್ನ ನಿರ್ಣಯಕ್ಕೆ ಸಿಗದೇ ಇದ್ದ ಪಕ್ಷದಲ್ಲಿ ವಾದಿ ಪ್ರತಿವಾದಿಗಳ ಮುದೆ ವಿಷಯವನ್ನಿಟ್ಟು `ಈ ವಿಷಯದಲ್ಲಿ ಸ್ತ್ಯವನ್ನೇ ಹೇಳ್ತೀಯೇನಯ್ಯಾ?' ಎಲ್ಲವೂ ಆದ ಮೇಲೆ ಕೊನೆಗೆ `ಸತ್ಯ' ಎಂದರೆ ಸಾಯಬಾರದು. ಸತ್ಯವನ್ನು ಬಿಟ್ಟು ಹೇಳಬಾರದು. ನಿನ್ನ ಅಂತಃಕರಣವೇ ಪ್ರಮಾಣ, ಸತ್ಯವೊಂದೇ ಮೂರು ಲೋಕಕ್ಕೂ ಪ್ರಮಾಣವಾಗಿದೆ. ಅದರ ಮೇಲೆ ಯಾವ ಶಾಸನವೂ ಇಲ್ಲ. ಆಳುವ ಶಕ್ತಿ ಅದಾಗಿದೆ. ಅದನ್ನು ನೆನೆಸಿಕೊಂಡು ಅದಕ್ಕೆ ಧಕ್ಕೆ ತರದಂತೆ ಹೇಳಬೇಕು. ಅದಕ್ಕೆ ಧಕ್ಕೆ ತಂದರೆ ತನಗೇ ಚ್ಯುತಿ. ತನಗಾಗಿ ಭಯಪಟ್ಟು ಸತ್ಯಕ್ಕೋಸ್ಕರ ಹೇಳು, ಎಂದು ಕೇಳಿ ಸತ್ಯವನ್ನೇ ತೆಗೆದುಕೊಳ್ಳಾಬೇಕಾಗುವುದು. ಆದ್ದರಿಂಡ ಹೇಳಿಕೆಯೆಲ್ಲವೂ ಮುಗಿದ ಮೇಲೆ, `ಸತ್ಯವಾಗಿ ಹೇಳಿದ್ದಿಯಾ?' ಎಂಬ ತನಿಖೆ. ಜಡ್ಜುಗಳೇ ಕಣ್ಣಿನಿಂದ ವಿಷಯ ನೋಡಿಬಿಟ್ಟರೆ ಪಂಚಾಯಿತಿಯೇ ಇಲ್ಲ. ಜಡ್ಜುಗಳೇ ಕಣ್ಣಿನಿಂದ ವಿಷಯ ನೋಡಿಬಿಟ್ಟರೆ ಪಂಚಾಯಿತಿಯೇ ಇಲ್ಲ. ಜಡ್ಜು ನೋಡದೇ ಇದ್ದಾಗ ಸಾಕ್ಷಿಗಳ ಅವಲಂಬನೆ. `ವಿಷಯವೊಂದನ್ನು ಕುರಿತು ಸತ್ಯವಾಗಿ ಹೇಳ್ತೀರಾ? ಶ್ರೀಕಂಠಯ್ಯ' ಎಂದರೆ, ತಮ್ಮ ಕಣ್ಣಿಂದಲೇ ನೋಡದೇ ಇದ್ದಾಗ ಅಣ್ಣನವರ (ಸಾಕ್ಷಿಗಳ) ಅವಲಂಬನ ಬೇಕಾಗುತ್ತದೆ.

\section*{ವಿಷಯವನ್ನು ಸತ್ಯವೇ ನಿರ್ಣಯಿಸಬೇಕು}

ಹಾಗೆಯೇ, ಶಂಕರರ ವಿಷಯದಲ್ಲಾಗಲೀ ರಾಮಾನುಜರ ವಿಷಯಗಲ್ಲಾಗಲೀ ಸರಿಯಾದ ಸಾಕ್ಷಿ ಇಲ್ಲ. ಅವರವರ ಪರಂಪರೆಯಲ್ಲಿ ಒಬ್ಬರು ಹೇಳಿದಂತೆ ಮತ್ತೊಬ್ಬರು; ಅಷ್ಟೆ, ಹೀಗಿದೆ ಸನ್ನಿವೇಶ. ಹೀಗಿರುವಾಗ ಒಂದು ನಿರ್ಣಯಕ್ಕೆ ಬರಬೇಕಾದರೆ ವಾಗಿಪ್ರತಿವಾದಿಗಳಿಬ್ಬರೂ, ಸರ್ವೋಚ್ಚಸ್ಥಾನದಲ್ಲಿದ್ದು ಎಲ್ಲರನ್ನೂ ಆಳುತ್ತಿರುವ ಸತ್ಯಕ್ಕೆ ಕಟ್ಟುಬೀಳಬೇಕು. ತಮ್ಮ ತಮ್ಮ ಹಠವನ್ನು ಬಿಟ್ಟು ಆ ಕಡೆಯಿಂದ ಏನು ತೀರ್ಮಾನ ಬರುತ್ತದೋ ಅದಕ್ಕೇ ಬಿಟ್ಟುಕೊಡಬೇಕು. ಇಲ್ಲದಿದ್ದರೆ ಸುಮ್ಮನೆ ಗುದ್ದಾಟ. ಶಂಕರ-ರಾಮಾನುಜರಿಬ್ಬರೂ `ಸತ್ಯಮೇವೋದ್ಧರಾಮ್ಯಹಮ್' ಎಂದು ಸತ್ಯದ ಉದ್ಧಾರಕ್ಕಾಗಿಯೇ ಪಣತೊಟ್ಟು ನಮ್ಮನ್ನು ಸತ್ಯದ ಕಡೆಗೆ ಒಯ್ಯುವವರಾಗುವ ಪಕ್ಷದಲ್ಲಿ ನಮಗೂ ಪ್ರಯೋಜನವುಂಟು.

\section*{ವಿಷಯ ನಿರ್ಣಯಿಸಲು ವಿಷಯದೊಡನೆ ಒಂದು ಸಂಬಂಧವಿರಬೇಕು}

ಯಾವುದೇ ಒಂದು ವಿಷಯವಾಗಿ ಮಾತನಾಡಬೇಕಾದರೂ ಆ ವಿಷಯಕ್ಕ್ಕೂ ನಮಗೂ ಸಂಬಂಧವಿರಬೇಕು. ಲೋಕದಲ್ಲಿ ವಿವಿಧ ರೀತಿಯಲ್ಲಿ ಸಂಬಂಧ ಉಂಟಾಗಿ ನಾವು ಮಾತನಾಡುವುದುಂಟು-ಕಂಡ ಗುರುತಿನ ಮೇಲೆ ಮಾತನಾಡಿಸುವುದು, ಸ್ನೇಹದ ಮೇಲೆ, ಬಂಧುತ್ವದ ಮೇಲೆ, ನಮಗೆ ಆಪ್ತರು ಎನ್ನುವುದರ ಮೇಲೆ ಇತ್ಯಾದಿ. ಪ್ರಕೃತ ನಾವು ಶಂಕರ-ರಾಮಾನುಜರ ವಿಷಯವಾಗಿ ಮಾತನಾಡಬೇಕಾದರೆ , ನಮಗೂ ಶಂಕರಭಗತ್ವಾದರಿಗೂ ಇರುವ ಸಂಬಂಧವೆಂತಹುದು? ನಮಗೂ ಭಗವದ್ರಾಮಾನುಜರಿಗೂ ಇರುವ ಸಂಬಂಧವೆಂತಹುದು? ಎನ್ನುವುದನ್ನು ಗಮನಿಸಬೇಕು. ಜ್ಞಾನ ಸಂಬಂಧವೇ? ಶುವಡಿ (ಪುಸ್ತಕ) ಸಂಬಂಧವೇ? ದೇಹಸಂಬಂಧವೇ ಅಥವಾ ರಕ್ತಸಂಬಂಧವೇ? ರಕ್ತಸಂಬಂಧದಲ್ಲಿಯೂ ದೂರದ ಸಂಬಂಧ, ಹತ್ತಿರದ ಸಂಬಂಧ ಎಂದೂ ಹೇಳಿಕೊಳ್ಳುವ ರೂಢಿಯಿದೆ. ನಮ್ಮಣ್ಣನ ಮಗನೇ ಆದರೂ ತಿರುಪೆ ಬೇಡಿ ಜೀವಿಸುತ್ತಿದ್ದರೆ, `ನಮಗೂ ಅವನಿಗೂ ಅಷ್ಟೇನೂ ಸಂಬಂಧವಿಲ್ಲ' ಎಂದು ಬಿದುವುದು, ಅದೇ ದೂರದ ಸಂಬಂಧವಾದರೂ ಪ್ರಸಿದ್ಧಿ ಪಡೆದಿದ್ದರೆ, `ಅಯ್ಯೋ, ಅವನು ನಮ್ಮ ಒಡಹುಟ್ಟಿದ ತಮ್ಮನೇ' ಎಂದೂ ಹೇಳುವುದು ಲೋಕದ ರೂಢಿಯಲ್ಲಿದೆ.

\section*{ಬುದ್ಧಿ ಕೌಶಲ್ಯ ವಾಗ್ಬಲಗಳಿಂದಲೇ ವಿಷಯ ನಿರ್ಣಯವಾಗುವುದಿಲ್ಲ}

ಗೆದ್ದೆತ್ತಿನ ಬಾಲ ಹಿಡಿಯುತ್ತೆ ಲೋಕ. ಲೋಕದಲ್ಲಿ ರಾಮಾನುಜರು ಗೆದ್ದುಬಿಟ್ಟರೆ `ಹಾಕಿ ಸ್ಮಾರ್ತರಿಗೆಲ್ಲರಿಗೂ ತಿರುಮಣ್ಣನ್ನು-ನಾಮವನ್ನು' ಎನ್ನುವುದು, ಅದೇ ಶಂಕರರು ಗೆದ್ದುಬಿಟ್ಟರೆ `ಶ್ರೀವೈಷ್ಣವರಿಗೆಲ್ಲಾ ವಿಭೂತಿ, ರುದ್ರಾಕ್ಷಿ ಹಾಕಿ' ಎನ್ನುವುದು. ಹೀಗೆ ಕೇವಲ ಬುದ್ಧಿಕೌಶಲ್ಯದಿಂದ, ವಾಗ್ಬಲದಿಂದ ವಿಷಯವನ್ನು  ತೀರ್ಮಾನಮಾಡುತ್ತಿದ್ದಾರೆ. ಇದರಲ್ಲಿ ವಿಷಯಕ್ಕೆ-ತತ್ತ್ವಕ್ಕೆ ಸ್ಥಾನವಿಲ್ಲ.

ಆದರೆ ಯಾವುದೇ ಒಂದು ವಿಷಯವನ್ನೂ ಕೇವಲ ವಾಗ್ಬಲದ ಆಧಾರದ ಮೇಲೆ ನಿರ್ಣಯಿಸುವುದು ಸಾಧ್ಯವಿಲ್ಲ. ತತ್ತ್ವಭೂಮಿಕೆಯಲ್ಲಿ ನೋಡಬೇಕಾದ ಅಂಶಗಳೂ ಇರುತ್ತವೆ. ನ್ಯಾಯಸ್ಥಾನದಲ್ಲಿ ಹೋದರೂ ಅಪರಾಧಿಯು ಎಲ್ಲರೂ ನಂಬುವಂತೆ ವಾದಿಸಿ ತನ್ನಲ್ಲಿ ನಿರಪರಾಧಿತ್ವವನ್ನು ಸಾಧಿಸಿ ಗೆದ್ದುಕೊಂಡು ಬಂದ ಪ್ರಕರಣಗಳು ಹಲವು ಉಂಟು. ಇಂತಹ ಸಂದರ್ಭಗಳಲ್ಲಿ ಅಪರಾಧಿಯೇ `ಜಡ್ಜಿಗೇನೂ ಗೊತ್ತಾಗಲಿಲ್ಲ. ಶುದ್ಧ ಕಕವ, (ಪೆದ್ದ) ನಾಲ್ಕು ಸವಾಲು ಹಾಕಿದ್ದಕ್ಕೆ ಥಂಡಿಯಾಗಿ ಹೋದ' ಎಂದು ಹೇಳಿ ದಬಾಯಿಸಿ ಬಿಡಬಹುದು.

ಸತ್ಯಕ್ಕೇ ಒಳಪಟ್ಟು ಹೇಳುವುದಾದರೆ, ಅಪರಾಧಿಯು `ಈಗ ಬಂದಿರುವುದು ಒಂದು ಕೇಸಾದರೂ, ನಾನು ಏಳೆಂಟು ಮರ್ಡರ್ ಮಾಡಿಬಿಟ್ಟಿದ್ದೇನೆ' ಎಂದು ತನ್ನ ಆತ್ಮಕ್ಕೇ ಭಯಪಟ್ಟುಕೊಂಡು ಅಂತಃಕರಣಪ್ರಮಾಣವಾಗಿ ವಿಷಯವನ್ನಿಡಬೇಕಾಗುವುದು. ಆವಾಗ ಮೊದಲಿನ ವಾಕ್ಕೌಶಲ್ಯ, ವಾಕ್ಸಾಮರ್ಥ್ಯಗಳು ಸತ್ಯ ವಿರೋಧಿಯಾಗಿ ನಿಲ್ಲಬಹುದು, ಮುಖದಲ್ಲಿ ಸಿಡುಬು ತುರಿಕಜ್ಜಿ ಇತ್ಯಾದಿಗಳಿಂದ ವಿಕಾರವಾಗಿದ್ದರೆ ಅದನ್ನು ಮೇಕಪ್ ನಿಂದ ಮುಚ್ಚಿಬಿಡಬಹುದ್. ಇಗ ಮುಖ ನೋಡಿ ಹೆಣ್ಣುಕೊಟ್ಟರೆ ಟೋಪಿ ಬೀಳುತ್ತಾರೆ. ಏಕೆಂದರೆ ಎಂದಾದರೂ ಒಂದುದಿನ ಅದನ್ನು ಅಳಿಸಿದಾಗ ನಿಜ ಹೊರಬೀಳುತ್ತೆ. ಮಾತಿನಿಂದಲೇ ಎಲ್ಲವೂ ಒಪ್ಪಿಸುವ ವಿಷಯವಲ್ಲ. ಪ್ರಯೋಗವೂ ಜೊತೆಗೆ ಬೇಕು. ಕೇವಲ ವಾಕ್ಕಿಗೇ ಬೆಲೆ ಕೊಟ್ಟರೆ ಗೆಲ್ಲುವುದು ಬುದ್ಧಿಯ ಕಸರತ್ತಿನ ಮೇಲೆ ಹೋಗುತ್ತದೆ.

\section*{ಶಂಕರ - ರಾಮಾನುಜರ ಬಗ್ಗೆ ಮಾತನಾಡಲು ನಮಗೆ ಅವರೊಡನೆ ಸಂಬಂಧವೇನು?}

ಒಬ್ಬರ ಬಗ್ಗೆ ಮಾತನಾಡಬೇಕಾದರೆ ಅವರಿಗೂ ನಮಗೂ ಸಂಬಂಧವೇನು? ನಮ್ಮದಾದ ಬದುಕು ಅವರಲ್ಲೇನಿದೆ? ಅವರದಾದ ಬದುಕು ನಮ್ಮಲ್ಲೇನಿದೆ? ಒಬ್ಬರ ಬದುಕು ಒಬ್ಬರಲ್ಲಿದ್ದರೆ ತಾನೇ ಆ ಬದುಕಿನ ಮೇಲೆ ನಮಗೂ ಅವರಿಗೂ ಸಂಬಂಧ ಉಳಿಯುತ್ತೆ. ಹಾಗಲ್ಲದೆ ಇರುವಲ್ಲಿ ನಮ್ಮ ಮನೆಗೆ ಅವರು ಬರುವುದು, ಅವರ ಮನೆಗೆ ನಾವು ಹೋಗುವುದು ಮುಂತಾದ ವ್ಯವಹಾರ ಹುಟ್ಟುವುದೂ ಇಲ್ಲ ಉಳಿಯುವುದೂ ಇಲ್ಲ.

ಈ ಸದಸ್ಸಿನಲ್ಲಿ ಶಾಂಕರರು ರಾಮಾನುಜೀಯರು ಎಲ್ಲರೂ ಇದ್ದಾರೆ. ಸೆನ್ಸೆಸ್ ನವನು ಸೆನ್ಸೆಸ್ ತೆಗೆದುಕೊಳ್ಳುವಾಗ ನಾಮ ವಿಭೂತಿಗಳನ್ನು ನೋಡಿ `ವಿಶಿಷ್ಟಾದ್ವೈತಿ' `ಅದ್ವೈತಿ', ಎಂದು ಬರೆದು ಬಿಡುತ್ತಾನೆ. `ಇದೇನಯ್ಯಾ, ಯಾರು ಹಾಗೆ ಹೇಳಿದವರೂ? ಸೆನ್ಸ್ ಲೆಸ್ ಆಗಿ ಬರೆದೆಯಲ್ಲಾ!' ಎಂದರೆ, ಯಾರು ಹೇಳಬೇಕುಸ್ವಾಮಿ? ನಾಮ, ವಿಭೂತಿ ಇವೇ ಇವೆಯಲ್ಲಾ, ಅದಕ್ಕಿಂತ ಇನ್ನೇನು ಬೇಕು ಪ್ರಮಾಣ?' ಎನ್ನುವುದರಲ್ಲಿ ಉಳಿದಿದೆ, ಅವರ ನಮ್ಮ ಸಂಬಂಧ, ನೀವು ಅವರನ್ನು ಕೊಂಡಾಡುವುದಕ್ಕೆ  ಕಾರಣವೇನು? ಅಂದರೆ ದಾಸ-ಗುರುಗಳ ಸಂಬಂಧವೂ ಅಲ್ಲಿದೆ. ಈ ಸಂಬಂಧ. ಇರುವಾಗ ಪರಸ್ಪರ ವ್ಯವಹಾರಬರುತ್ತದೆ. ಇದರಲ್ಲಿ ಮುಂದೆಮುಂದಕ್ಕೆ ಬೆಳೆಯುವ ಸಂಬಂಧವೂ ಉಂಟು. ಮುಂದೆ-ಮುಂದೆ ಹೋಗುತ್ತಾ-ಹೋಗುತ್ತಾ ಕಡಿಮೆಯಾಗುವ ಸಂಬಂಧವೂ ಉಂಟು.

ಮಗ ಮೊಮ್ಮಗ ಮರಿಮಗ ಹೀಗೆ ಮುಂದೆ ಮುಂದೆ ವಂಶವು ಬೆಳೆದಂತೆಲ್ಲಾ ಹಲವು ತಲೆಮಾರುಗಳಲ್ಲಿ ಸಂಬಂಧವು ಹ್ರಾಸವಾಗುತ್ತದೆ. ಯಾವ ಸಂಬಂಧವಿಲ್ಲದಿದ್ದರೂ ವಿದ್ಯಾಸಂಬಂಧವೊಂದು ಬೆಳೆದರೆ ದಿನೇ ದಿನೇ ವೃದ್ಧಿಯಾಗುವುದೂ ಉಂಟು. ಈಗ ನಾವು ಮಾತನಾಡಲು ಹೊರಟಿರುವುದು ಇಂದಿಗೆ ಸಾವಿರಾರು ವರ್ಷಗಳ ಹಿಂದೆ ಇದ್ದವರ ಸಮಾಚಾರ. ಹೀಗಿರುವಾಗ ನಮಗೂ ಅವರಿಗೂ ಇರುವ ಸಂಬಂಧವೇನು? ತಿರುಮೇನಿಯ (ದೈಹಿಕ) ಸಂಬಂಧವೇ? ತಿರುವಡಿ ಸಂಬಂಧವೇ? (ಗುರು ಶಿಷ್ಯ ಭಾವವೇ) ಅಥವಾ ಶುವಡಿ ಗ್ರಂಥಸಂಬಂಧವೇ? ಮಧ್ಯೇ ಎಷ್ಟೋ ಕಾಲ ಬಿಟ್ಟುಹೋಗಿದೆ. ಮಧ್ಯದಲ್ಲಿ ಬಾಯಿಂದ ಬಾಯಿಗೆ ಹತ್ತಾರು ತರಹದಲ್ಲಿ ವಿಷಯವು ಮಾರ್ಪಾಡಾಗಿ ಬಿಡಬಹುದು. ಹಾಗೆ ಪರಂಪರೆಯಿಂದ, ಮನೆಮನೆಯಿಂದ ಅವರ ಬಗ್ಗೆ ಹರಕುಮುರಕು ಮಾತುಗಳನ್ನು  ಕೇಳುತ್ತಿದ್ದೇವೆ. ಮತ್ತು ಅವರದೆಂದು ಹೇಳಲಾಗುತ್ತಿರುವ ಹಲವು ಗ್ರಂಥಗಳು ಸಿಗುತ್ತಿವೆ.

\section*{ಶಂಕರ-ರಾಮಾನುಜರ ಬಗ್ಗೆ  ಯಾವ ಹಿನ್ನೆಲೆಯಿಂದ ಮಾತನಾಡಬೇಕು}

ಹೀಗೆ ನಮಗೂ ಅವರಿಗೂ ಸಂಬಂಧದಲ್ಲಿ ಇಷ್ಟು ಅಂತರವಿರುವಾಗ ಅವರ ಬಗ್ಗೆ ಅಧಿಕೃತವಾಗಿ ಮಾತನಾಡುವುದಾದರೂ ಹೇಗೆ? ನಾವು ನಮ್ಮ ಸಮಕಾಲೀನರ ವಿಷಯವಾಗಿಯೇ ಮಾತನಾಡಲು ಹೊರಟರೂ ಅಧಿಕೃತವಾಗಿ ಮಾತನಾಡಬೇಕಾದರೆ ತುಂಬ ಒಳ ಹೊಕ್ಕು ಬಳಸಬೇಕು. ತಮಗೆ ಅವರು ಸಹಾಧ್ಯಾಯಿಗಳೇ? ಅಥವಾ ತಾವು ಅವರ ಗುರುಗಳೇ? ಅಥವಾ ಅವರು ನಿಮ್ಮ ಗುರುಗಳೇ? ಬಂಧುಗಳೇ? ಸ್ನೇಹಿತರೇ? ಮೊದಲಾದ ಅನೇಕ್ಲ ಪ್ರಶ್ನೆಗಳು ಬರುತ್ತವೆ. ಇಂತಹ ಯಾವ ಸಂಬಂಧವೂ ಇಲ್ಲದಿದ್ದರೆ ನಮ್ಮ ಸಮಕಾಲೀನರ ಬಗ್ಗೆಯೇ ಆಡುವುದು ಇಷ್ಟು  ಕಷ್ಟವಾಗಿರುವಾಗ ಇನ್ನು ಸಾವಿರಾರು ವರ್ಷಗಳ ಹಿಂದಿನವರ ಬಗ್ಗೆ ನಾವು ಆಡಬಹುದಾದುದು ಏನು? `ಮುರಾರೆಸ್ತೃತೀಯಃ ಪಂಥಾಃ' ಎಂಬಂತೆ `ನನಗೆ ತೋರಿದ್ದು ಹೇಳುತ್ತೇನೆ.' ಎಂದು ಮುರಾರಿಯ ಮೂರನೇ ಮಾರ್ಗಬೇಡ. ಶಂಕರರ ರಾಮಾನುಜರ ವಿಷಯ ಹೇಳುವಾಗ ಅವರ ಪರಂಪರೆಯಿಂದ ವಿಷಯವನ್ನವಲಂಬಿಸಿ ಹೇಳುವುದು ಕಷ್ಟ. ಆರಊ ಅವರ ವಿಗ್ರಹಗಳು, ಗ್ರಂಥಗಳು ಇವುಗಳಿಂಡ ನಾವು ತೆಗೆದುಕೊಳ್ಳಬಹುದಾದ ವಿಷಯವೇನಾದರೂ ಇದೆಯೇ? ಎಂದು ನೋಡೋಣ.

\section*{ಶ್ರೀ ರಾಮಾನುಜರ ವಿಗ್ರಹದ ಕುರಿತು}

ಗುಡಿಯಲ್ಲಿ ರಾಮಾನುಜರ ವಿಗ್ರಹವನ್ನು ಎಲ್ಲರೂ ನೋಡಿರಬಹುದು. ಅವರು ಪದ್ಮಾಸನದಲ್ಲಿದ್ದಾರೆ. ಕೈಯನ್ನು ಹೃದಯದ ನೇರದಲ್ಲಿ, ಎಡಭಾಗ ಬಲಭಾಗವೆರಡನ್ನೂ ಕೂಡಿಸುವ ಮಧ್ಯಮಾರ್ಗದಲ್ಲಿಟ್ಟಿದ್ದಾರೆ. ಅವರ್ ಬಗ್ಗೆ ಹೇಳುವಾಗ, ಬಲಭಾಗ-ಎಡಭಾಗ ಎರಡನ್ನೂ ಕೂಡಿಸುವ ಮಧ್ಯಮಾರ್ಗದಲ್ಲಿ ನಿಂತು ಹೇಳುತ್ತೇನೆ. ಹೀಗೆ ನಿಂತರೆ (ತೋರಿಸುತ್ತಾ) `ಭಕ್ತಿಯಿಂದ ನಿಂತಿದ್ದಾರೆ' ಎನ್ನುತ್ತಾರೆ. ಭಕ್ತಿ ಎಂದರೇನು? ಸೇರುವೆ-ಅವಿಭಕ್ತತೆ. ಯಾವುದರ ಜೊತೆಯಲ್ಲಿ ಅವಿಭಕ್ತವಾಗಿ ಸೇರಬೇಕು? ಎಂದರೆ ಭಗವಂತನ ಜೊತೆಯಲ್ಲಿ ಜೀವನವನ್ನು ಸೇರಿಸಿ ಕೊಳ್ಳಬೇಕು. ಅವರ ಕೈಯಲ್ಲಿ ಒಂದು ತ್ರಿದಂಡವಿರುತ್ತದೆ. ಅದಕ್ಕೆ ಮೂರು ಕಟ್ಟು ಇರುತ್ತದೆ. ಆ ತ್ರಿದಂಡವು ಕಾಯ-ವಾಕ್-ಮನಸ್ಸುಗಳ ದಂಡನವಾಗಿರುತ್ತದೆ. ಸೂತ್ರಪ್ರಾಯವಾಗಿ ಹೇಳುವುದಾದರೆ ಹಾಗೆ ಭಗವಂತನಲ್ಲಿ ಸೇರಿ ಕಾಯ-ವಾಕ್-ಮನಸ್ಸುಗಳನ್ನು ಕಟ್ಟಿಕುಳಿತುಕೊಳ್ಳಬೇಕು. `ಮೌನವಾಗಿ ಹಾಗೆ ಕುಳಿತಿದ್ದಾರಲ್ಲಾ, ಅದೇನು?' ಎಂದಾಗ ಅದರ ವಿಷಯವಾಗಿ ಹೊರಗೆ ವಿವರವನ್ನು ಕೊಡಬೇಕಾಗುತ್ತದೆ. ಇಷ್ಟು ರಾಮಾನುಜರ ಬಗ್ಗೆ ಆಡುವುದಾದರೆ ಇನ್ನು ಶಂಕರರ ಬಗ್ಗೆ ` ಒಬ್ಬರನ್ನು ಹೊಗಳಿಂಡ ಮೇಲೆ ಅವರ ಪ್ರತಿಸ್ವರ್ಧಿಯನ್ನೂ ಹೊಗಳಿ' ಎಂದರೆ ಕಷ್ಟ. ಪ್ರಕೃತದಲ್ಲಿ ಇಬ್ಬರನ್ನೂ ಒಟ್ಟಿಗೆ ಹೊಗಳಬೇಕಾಗಿದೆ. ಅವರ ಕೃಪೆ ಇದ್ದ ಪಕ್ಷೇ ಅದೇನೂ ಕಷ್ಟವಲ್ಲ.

\section*{ಶ್ರೀ ಶಂಕರರ ವಿಗ್ರಹದ ಕುರಿತು}

ಶಂಕರರ ಒಂದು ವಿಗ್ರಹವನ್ನು ಶೃಂಗೇರಿಯಲ್ಲೋ ಕಾಲಟಿಯಲ್ಲೋ ನೋಡಿದಾಗ, ಎಡಗೈಯಲ್ಲಿ ಪುಸ್ತಕವೂ ಬಲಗೈಯಲ್ಲಿ ಜ್ಞಾನಮುದ್ರೆಯೂ ಕಂಡು ಬರುತ್ತದೆ. ಅವರ ಕೈಯಲ್ಲಿರುವ ಪುಸ್ತಕವಾವುದು? ಗೀತಾಭಾಷ್ಯಾವೇ? ಸೂತ್ರಭಾಷಯವೇ? ಅಥವಾ ಅದ್ವೈತಸಿದ್ಧಿಯೇ? ಯಾವುದು? ಇನ್ನೊಂದು ಕೈಯಲ್ಲಿ ನಶ್ಯ ಹಾಕುತ್ತಿದ್ದಾರೆಯೇ? ಅಥವಾ ಚಿಟಿಕೆ ಹಾಕುತ್ತಿದ್ದಾರೆಯೇ? ಅದರ ವಿಷಯವೇನು? ಎಂದು ತೆಗೆದುಕೊಂಡಾಗ, ಒಂದು ಕೈಯಲ್ಲಿ  ಜ್ಞಾನಮುದ್ರೆಯಿಂದ, ಜೀವನದಲ್ಲಿ ಪಡೆಯ ಬೇಕಾದ ಯಾವ ಜ್ಞಾನವುಂಟೋ, ಅದರಿಂದುಂಟಾಗುವ ಮುದವೇನುಂಟೋ ಅದನ್ನು ಸೂಚಿಸುತ್ತಾ, ಅದರ ವಿಸ್ತಾರವು ಪ್ರಕೃತಿಯಲ್ಲಿ ಹೇಗೆ ಆಗಿದೆ ಎಂಬುದನ್ನು ವಿವರಿಸುವ ಪುಸ್ತಕವನ್ನು  ಇನ್ನೊಂದು ಕೈಯ್ಯಲ್ಲಿ ಹಿಡಿದಿದ್ದಾರೆನ್ನಬೇಕಾಗುವುದು.

\section*{ಯಾವ ಶಂಕರ-ರಾಮನುಜರ ಬಗ್ಗೆ ವಿಚಾರ?}

ರಾಮಾನುಜರ ವಿಗ್ರಹ ಶಂಕರರ ವಿಗ್ರಹ ಮೇಲೆ ತಿಳಿಸಿದಂತೆ ಇವೆ. ಇದರ ಬಗ್ಗೆ ವಿವರಣೆ ಕೊಡಬೇಕಾದರೆ ಇನ್ನೊಂದು ಪುಸ್ತಕ ಬರೆಯಿಸಿ ಕೊಡಬೇಕಾಗುತ್ತೆ. ಇದನ್ನೊಂದು ವಿಗ್ರಹವಾಗಿ ಅಂದರೆ ಬಿಡಿಸಿ ಹೇಳಬೇಕಾಗಿದೆ. ಆವಿಗ್ರಹಗಳನ್ನು ನಾವು ಇದೇ ರೀತಿ ನೋಡಬೇಕಾಗಿದೆ. (ದಾಶರಥಿಯ\footnote{ದಾಶರಧಿಯು ಶ್ರೀ ಗುರುಭಗವಂತರ ತಂಗಿಯವರ ಮಾಗ.} ಕಡೆಗೆ ತಿರುಗಿ ರಾಮಾನುಜರು ಯಾರು ಗೊತ್ತೇ ದಾಶರಥಿ! ರಾಮಾನುಜರು ಎನ್ನುವವರನ್ನು ಕೇಳಿದ್ದೀಯಾ? (ಹುಂ ಎಂದು ಉತ್ತರ) ಎಷ್ಟೋ ರಾಮಾನುಜರಿದ್ದಾರೆ, ಯಾವ ರಾಮಾನುಜರ ಬಗ್ಗೆ ಆಡಬೇಕು? ಇತರರನ್ನು  ಪೃಥಕ್ಕರಿಸಬೇಕಾದರೆ ಒಂದು ಇನ್ಷಿಯಲ್ ಬೇಕು. `ಭಗವದ್ರಾಮಾನುಜರ' ಮತ್ತು `ಭಗವಚ್ಛಂಕರರ' ವಿಷಯ ವನ್ನಿಡಬೇಕಾಗಿದೆ. ಇನ್ನು ಬಾಗಿ ರಾಮಾನುಜರು, ಶಂಕರರು, ಎಂದು ಹೆಸರನ್ನ್ಬಿಟ್ಟು ಕೊಳ್ಳುವವರಲ್ಲಿ ಭಗವಾನ್ ಎಂಬ ವಿಶೇಷಣವಿಲ್ಲ.

\section*{ಕಾಲದೇಶ ಅಧಿಕಾರಿಗಳನ್ನವಲಂಬಿಸಿದೆ ವಿಷಯದ ಬೆಳವಣಿಗೆ}

ಲೋಕದಲ್ಲಿ ಒಂದು ವ್ಯವಹಾರ ಬೆಳೆದು ಬಹಳ ಕಾಲವಾಗಿ ಹೋದರೆ ವಿಷಯ ವ್ಯತ್ಯಾಸವಾಗಿ ಹೋಗುತ್ತದೆ. ಉದಾಹರಣೆಗೆ ತೆಗೆದುಕೊಳ್ಳೋಣ. (ಎಂದು ಹೇಳಿ ದಾಶರಥಿಯನ್ನು ಕುರಿತು ಈಗ ಹೇಳಿರುವ ವಿಷಯವೆಲ್ಲಾ ಗೊತ್ತಾಯಿತೇ? ಹೇಳು ಎಂದು ಕೇಳಿದಾಗ ಕೆಲಕೆಲ ವಿಷಯಗಳನ್ನು ಮಾತ್ರ ಹೇಳಿದ.)

ಈಗ ವಿಷಯ ಹೇಳಿ ಎದುರಿಗೆ ಕೇಳಿದಾಗಲೇ ಇಷ್ಟು  ವ್ಯತ್ಯಾಸ ಬರುತ್ತದೆ. ಆಯಾ ಟೈಮ್ ಸ್ಪೇಸಿನಲ್ಲೇ ಇರುವಾಗಲೂ ವ್ಯವಹಾರದಲ್ಲಿ ಎಷ್ಟೋ ವ್ಯತ್ಯಾಸ ಬರುತ್ತದೆ. ಹೀಗಿರುವಾಗ ಸಾವಿರಾರು ವರ್ಷಗಳ ಕಾಲ ಉರುಳಿರುವ ಬದುಕಿನ ಮೇಲೆ, ಉರುಳಿರುವ ವಿಷಯದ ಮೇಲೆ, ಭಗ್ನಾವಶೇಷದ ಮೇಲೆ ಮಾತನಾಡುವುದು ಬಹುಜವಾಬ್ದಾರಿಯಾದ ವಿಷಯ. ಈ ವಿಷಯದಲ್ಲಿ ಮಧ್ಯವರ್ತಿಗಳ ಕೈವಾಡ ಬೇರೆ.

ಉದಾಹರಣೆಗೆ-ಒಬ್ಬ ವೈದ್ಯ ಸಂಗೀತ ಕಲಿಯಲು ಹೊರಟರೆ `ವಾತಾಪಿ ಗಣಪತಿಂ ಭಜೇಽಹಂ' ಎಂದು ಹೇಳಿದಾಗ, `ವಾತ' ಕೇಳಿದ ಮೇಲೆ `ವಾತ' ಸರಿಹೋಯ್ತು, ಆಮೇಲೆ  `ಪಿಗ' ಇದೆಯಲ್ಲಾ `ಪಿತ್ತ' ಎಂದಿರಬೇಕು ಎಂನ ನಿಗಾದಲ್ಲಿ ಸ್ವರ ಮರೆತು ಹೋಯಿತು, ತಾಳ ತಪ್ಪಿಹೋಯ್ತು. `ಏನಯ್ಯಾ, ಹೀಗಾದರೆ ಸ್ವರ, ತಾಳ ಎಲ್ಲಾ ತಪ್ಪಿ ಹೋಯಿತಲ್ಲಯ್ಯಾ? ಎಂದರೆ `ಸ್ವಲ್ಪ ತಾಳೀ ಸ್ವಾಮಿ, ಇಲ್ಲಿ ಮಾಡಬೇಕಾದ ವಿಚಾರ ಸ್ವಲ್ಪವಿದೆ' ಎನ್ನುತ್ತಾನೆ. ಅವನ ಸಂಸ್ಕಾರಕ್ಕೆ ತಕ್ಕಂತೆ ವಿಷಯ ನೋಡುತ್ತಾನೆಯೇ ಹೊರತು ವಿಷಯಕ್ಕನುಗುಣವಾಗಿ ಅವನ ಬುದ್ಧಿಯನ್ನಳವಡಿಸಿಕೊಳ್ಳುವುದಿಲ್ಲ. ಆದ್ದರಿಂದಲ್ಲೆ ಒಂದು ವಿಷಯವನ್ನು ಮುಂದುವರಿಸಬೇಕಾದರೆ ಅಧಿಕಾರಿಯನ್ನು ನೋಡಿ ಮುಂದುವರಿಸಬೇಕೆಂದು ನಿಶ್ಚಯಿಸಿ, ಅದನ್ನು ಸಂಪ್ರದಾಯ ಉಪದೇಶ ರೂಪವಾಗಿ ತಂದರು. ಆದರೆ ಅಲ್ಲಿಯೂ ಕ್ರಮ ಕೆಟ್ಟುಹೋಗಿ ವಿಷಯ ಸರಿಯಾಗಿ ಬರುವಂತಾಗಲಿಲ್ಲ. 

\section*{ಸಂಪ್ರದಾಯವನ್ನು ಕುರಿತು}
 
ಸಂಪ್ರದಾಯವೆಂದರೇನು `ಶಿಷ್ಯೋಪಾಧ್ಯಾಯಿಕಯಾ ವೃತ್ಯಾ  ಅವಿಚ್ಛೇದೇನ ಗ್ರಂಥತಃ ಅರ್ಥತಶ್ಜ ಶಾಸ್ತ್ರಪ್ರಾಪ್ತಿಃ ಸಂಪ್ರದಾಯಃ' ಎನ್ನುವಂತೆ ಮಧ್ಯೆ ವಿಚ್ಛೇದ ವಿರಬಾರದು. ಶಿಷ್ಯನ ಕೈಗೆ ಉಪಾಧ್ಯಾಯ ಗ್ರಂಥವನ್ನು ಕೊಡುವಾಗ ಅದರ ಮರ್ಯಾದೆಯೊಡನೆ ಕೊಡಬೇಕು. ಅರ್ಥ ಎನಯ್ಯಾ? ಎಂದು ಕೆಳಿದರೆ, ಶಿಷ್ಯ-ಗುರು ಎಲ್ಲರೂ ಸಮ್ಮತಿಸುವ ರೀತಿಯಲ್ಲಿರಬೇಕು. ಒರಿಜಿನಲ್ಲಾಗಿ ವಿಷಯವನ್ನು ಮುಂದುವರಿಸಬೇಕು. ಲೋಕದಲ್ಲಿ ತಂದೆತಾಯಿಗಳು ತಮ್ಮಂತೆಯೇ ಮಕ್ಕಳು ಬೆಳೆಯಬೇಕು ಎನ್ನುವುದು ಸಹಜವಾಗಿದೆ. ಅಂತೆಯೇ ಗುರುವಾದವನು ತನ್ನ ಆತ್ಮನಂತೆಯೇ ತನ್ನ ಶಿಷ್ಯರು ಬೆಳೆಯಬೇಕು ಎಂದು ಬೆಳೆಸಿದರೆ ಸಂಪ್ರದಾಯವು ಹರಿಗೆಡಹಿಲ್ಲದೆ ಮುಂದುವರಿಯುತ್ತದೆ. 

\section*{ಸಂಪ್ರದಾಯದಲ್ಲಿ ಹುಟ್ಟಿದ ನ್ಯೂನತೆ}

ಕಾರ್ತಿಕದೀಪ ಹಚ್ಚಿಕೊಂಡು ಬಂದರು, ಆದರೆ ಆರಿಹೋಯಿತು, ಅದರ ಬಗ್ಗೆ ನಿಗಾ ಇಲ್ಲ. ಎರಡೂ ವರಸೆ ದೀಪ ಹಚ್ಚಿಕೊಟ್ಟರು, ಆರಿಹೋಯಿತು, ಮಧ್ಯೆ ಗಮನಿಸಲಿಲ್ಲ. ಸುಮ್ಮನೆ ಕಡ್ಡಿಯನ್ನು ಒಂದರಿಂಡ ಒಂದಕ್ಕೆ ಹಿಡಿಯುತ್ತಾ ಬಂದರು. ವ್ಯರ್ಥ ಶ್ರಮ. ಏನು ಮಾರ್ಪಟ್ಟಿದೆ? ಎಂಬ ಗಮನವಿಲ್ಲ. ಮಾರ್ಪಟ್ಟಿದೆ ಎನ್ನುವ ವಿಷಯವನ್ನೇ ತೆಗೆದುಕೊಳ್ಳುವುದಿಲ್ಲ. ಒಂದು ಕೈಯಿಂದ ಇನ್ನೊಂದು ಕೈಗೆ ಬದಲಾಯಿಸಿದರೆ ವಿಷಯವು `ಸೆಕೆಂಡ್ ಹ್ಯಾಂಡ್' ಆಗಿ ಬಿಡುತ್ತದೆ. ಅಂಗಡಿಯಲ್ಲಿ ಟೆಕ್ಸ್ಟ್  ಬುಕ್ ಬದಲಾಗಿದೆಯೆಂದು ಅಂಗಡಿಗೆ ಕೊಂಡು ಹೋದರೆ ಅರ್ಧಬೆಲೆಗೆ ಕೊಡಿ ಎನ್ನುತ್ತಾನೆ. ಆಗಲೇ ಅದು ಸೆಕೆಂಡ್ ಹ್ಯಾಂಡ್ ಆಗಿಬಿಡುತ್ತದೆ. ಅಲ್ಲದೇ ಈ ವಿಷಯಗಳನ್ನು ಟೆಕ್ಸ್ಟ್ ಬುಕ್ಕಿನಂತೆ ಖಚಿತವಾಗಿ ಅಭ್ಯಾಸಮಾಡುವುದಿಲ್ಲ. ಅಲ್ಲದೇ ಚಿಲ್ಲರೆ-ಚಿಲ್ಲರೆಯಾಗಿ ಮಾತು ಬೆಳೆದುಬಂದಿದೆ. ಎಲ್ಲಾ ಸೆಕೆಂಡ್ ಹ್ಯಾಂಡ್ ಆಗಿದೆ.

ಇನ್ನು ಈ ಅದ್ವೈತಸಿದ್ಧಾಂತ ಸಂಪ್ರದಾಯಗಳನ್ನಾಗಲೀ ವಿಶಿಷ್ಟಾದ್ವೈತ ಸಿದ್ಧಾಂತ ಸಂಪ್ರದಾಯಗಳನ್ನಾಗಲೀ ಬೆಳೆಸುವವರೆಂದುಕೊಂಡಿರುವ ವಿದ್ವಾಂಸರಿದ್ದಾರೆ. ಅವರೂ ಸಹ ಈ ಬಗ್ಗೆ ಅನುಭವಖಚಿತವಾದ ಮತ್ತು ಸ್ವಾಭಾವಿಕವಾದ ಹಿನ್ನೆಲೆಯ ಮೇಲೆ ಇಂದು ವಿಷಯವನ್ನು ಬೆಳೆಸುತ್ತಿರುವ ಲಕ್ಷಣಗಳು ಕಾಣಿತ್ತಿಲ್ಲ. ಇದನ್ನು ವಿಷಾದದಿಂದ ಹೇಳಬೇಕಾಗಿದೆ.

\section*{ಪ್ರಸನ್ನತೆಗಾಗಿ ಸ್ತೋತ್ರಗಾನ}

ಶಂಕರ ರಾಮಾನುಜರ ಪ್ರಸಾದವು ನಮ್ಮನ್ನೆಲಾ ವಿಷಾದದಿಂದ ದೂರವಿಡಲಿ ಎಂದು ಸ್ತೋತ್ರವನ್ನು ಹೇಳುತ್ತೇನೆ. (ಎಂದು ಹೇಳಿ ಗುರುಭಗವಂತರು ಶ್ರೀ ರಾಮಾನುಜರ ವಿಷಯಕವಾದ `ಭಜ ಯತಿರಾಜಂ' ಹಾಗೂ ಶ್ರೀ ಶಂಕರರಿಂದ ರಚಿತವಾದ `ಭಜಗೋವಿಂದಂ' ಸ್ತೋತ್ರಗಳಲ್ಲಿ ಈ ಕೆಳಕಂಡ ಕೆಲವು ಶ್ಲೋಕಗಳನ್ನು ಗಾನಮಾಡಿದರು)

\begin{itemize}

 `
\begin{shloka}
\item[(1)] `ಭಜ ಯತಿರಾಜಂ ಭಜ ಯತಿರಾಜಂ ಭಜ ಯತಿರಾಜಂ ಭವಭೀರೋ|\\
ಕಾಂತಿಮತೀಸುಕುಮಾರಕುಮಾರಂ ಕೇಶವಸಿಂಹಕಿಶೋರಮುದಾರಮ್'||
\end{shloka}

\item[] `
\begin{shloka}
ರಾಮಾನುಜಂ ಅಹಿರಾಡವತಾರಂ, ಮೂಕಾಂಧಾನಪಿ ಮೋಕ್ಷಯಿತಾರಂ|| ಭಜ.\\
ಗುಣಗುಣಿನೋರ್ಭೇದಃ ಕಿಲ ನಿತ್ಯಃ ಚಿದಚಿದ್ವಯಪರಭೇದಃಸತ್ಯಃ || 
\end{shloka}

\item[] `
\begin{shloka}
ತದ್ದ್ವಯದೆಹೋ ಹರಿರಿತಿ ಸತ್ಯಂ ಪಶ್ಯತ್ತ್ವಂವಿಶಿಷ್ಟಾದ್ವೈತಂ ತತ್ತ್ವಂ|\\
ಪ್ರವಜನಶಕ್ತಃ, ಪ್ರಜ್ಞಾಯುಕ್ತಃ ಪರಹಿತಸಕ್ತಃ ಪರಮವಿರಕ್ತಃ ಭಜ||
\end{shloka}

 
\begin{shloka}
\item[(2)] `ಭಜಗೋವಿಂದಂ, ಮೂಢ ಜಹೀಹಿ, ನಲಿನೀದಲಗತ, ಬಾಲಸ್ತಾವತ್\\
ಯೋಗರತೋ ವಾ, ರಥ್ಯಾಕರ್ಪಟ, ಭಗವದ್ಗೀತಾ, ಗುರುಚರಣಾಂಬುಜ'
\end{shloka}

\end{itemize}

\section*{ಜ್ಞಾನಮೂಲದಿಂದ ತಂದು ಜವಾಬ್ದಾರಿಯಿಂದ ವಿಷಯವಿಡಬೇಕಾಗಿದೆ}

ಅವ್ಯಕ್ತವಾದ ಜ್ಞಾನವು ತನ್ನ ಬೆಳಕನ್ನು ಎಲ್ಲಾ ದೇಶದಲ್ಲಿಯೂ ಎಲ್ಲಾ ಕಾಲದಲ್ಲಿಯೂ ಗುಪ್ತವಾಗಿ ಪ್ರಚಾರ ಮಾಡುವ ಉದ್ದೇಶದಿಂದ ಒಂದು ಪ್ರಕೃತಿಯನ್ನು ಬಳಾಸುವುದುಂಟು. ಆ ಪ್ರಕೃತಿಯಲ್ಲಿ ಅಂತೆಯೇ ವಿಷಯವನ್ನು ತಂದು, ಅದರ ವಿಕಾಸವನ್ನು ಇಲ್ಲಿ ಕುಳಿತಿರುವ ಗೃಹಿಣೀ ಗೃಹಸ್ಥರ, ದೇಹಸ್ಥರೂ ಆತ್ಮಸ್ಥರೂ ಆಗಿರುವವರ ಮುಂದೆ ಶಂಕರರಾಗಿ ರಾಮಾನುಜರಾಗಿ ಆಡುವುದಾದರೆ ಸರಿಯಾಗುತ್ತೆ. ಯಾವ ಜ್ಞಾನಮಾತಾಪಿತೃಗಳುಂಟೋ ಅವರಿಂದ ತೊದಲು ನುಡಿಯನ್ನು ಕಲಿತು `ಅಮ್ಮ-ಅಪ್ಪ' ಎಂಬ ಮಾತಿನಿಂದಾರಂಭಿಸಿ, ಅವರ ಶಿಕ್ಷಣದಿಂದ ಅವರ ಅಭಿಪ್ರಾಯವನ್ನು ಮುಂದುವರಿಸುವಂತೆ, ಜ್ಞಾನವೃದ್ಧರ, ವಯೋವೃದ್ಧರ ಸೇವೆಮಾಡಿ ಅವರ ಪೋಷಣೆಯಿಂದ ಬೆಳೆದ ಗಿಣಿಯಾಗಿದ್ದರೆ ಸುಲಭವಾಗಿತ್ತು. ಆದರೆ ಬೇಡನ ಕೈಗೆ ಸಿಕ್ಕಿ ಬೇಡವಾದ ವಿಷಯ ಕಲಿತರೆ ಕಷ್ಟ. ಹೇಳಿದ್ದೆಲ್ಲಾ ವಿಪರೀತವಾಗುತ್ತೆ. `ವರೆಯಾಡ'? (ಬ???????686868 ಎಂಬರ್ಥದ ಪದ,) `ವಡೆಯಾರ' ಆಗ ಬಹುದು `ವರದಾಚಾರ್' ಹೋಗಿ `ವದರಾಚಾರ್' ಆಗಬಹುದು. ಹೀಗೆ ಬುದ್ಧಿಪಲ್ಲಟವಾಗುವುದು. ಅವರದೇ ಆದ ಬುದ್ಧಿ ಏನುಂಟೋ, ಅದರ ಜಾಗದಿಂದ, ಅವರ ಜ್ಞಾನದಿಂದ ಸ್ವಲ್ಪಜಾರಿದರೂ ಬುದ್ಧಿಪಲ್ಲಟವೇ. ವಿಷಯದ ಕಡೆಗೆ ದೃಷ್ಟಿಯಿಟ್ಟು, ನಿರ್ದೋಷವಾದ ರೀತಿಯಲ್ಲಿ ಜಾಗರೂಕತೆಯಿಂದ ವಿಷಯವನ್ನಿಡಬೇಕಾಗಿದೆ.

\section*{ಅಂಶವನ್ನೇ ಪೂರ್ಣವೆಂದುಕೊಳ್ಳುಬಾರದು}

ಇಂದು ರಾಮಾನುಜರ ತಿರುನಕ್ಷತ್ರ ಶಂಕರಜಯಂತಿಗಳು ಜ್ಞಾಪಕಕ್ಕೆ ಬರುವುದು ಪುಳಿಯೋಗರೇ ಸಕ್ಕರಪಾನಕಗಳಿಂದ. ಅವರ ನೆನಪುಕೊಡುವುದು ಅವೆರಡೇ ಆಗಿವೆ. ಇಷ್ಟರಲ್ಲೇ ನಿಂತಿದೆ ಜಯಂತಿಯ ಆಚರಣೆ. `ಪರುಪ್ಪಿಲ್ಲಾದ ಕಲ್ಯಾಣಮಾ?' (ತೊವ್ವೆಯಿಲ್ಲದ ಮದುವೆಯೇ) ಎನ್ನುವುದುಂಟು. ಆದರೆ ಮದುವೆಗೆ ಅಷ್ಟೇ ಇದ್ದರೆ ಆಗುವುದಿಲ್ಲ. ಪರುಪೊಂದಿದ್ದರೇ ಮದುವೇಯಾಗುತ್ತೆಯೇ? ಆದ್ದರಿಂಡ ಅಂಶತಃ ಈ ವಿಷಯವು ಬಂದಿದೆ. ಅಂಶಗಳನ್ನು  ಪೂರ್ಣದ ಜೊತೆಯಲ್ಲಿ ಸೇರಿಸಿ ಹೇಳಿದ್ದರೆ ಚೆನ್ನಾಗಿತ್ತು. ಅಂಶಗಳೆಲ್ಲವೂ ಪೂರ್ಣವಾಗಿ ನಿಂತರೆ ಸಾರ್ಥಕ. `ಪೂರ್ಣಸ್ಯ ಪೂರ್ಣಮಾದಾಯ ಪೂರ್ಣಮೇವಾವಶಿಷ್ಯತೇ' ಆದರೆ ಬರೀ ಅಂಶ ತೆಗೆದುಕೊಂಡು ಪೂರ್ಣವೆಂದರೆ ನಂಬಲಾಗುವುದಿಲ್ಲ. ಪೂರ್ಣವನ್ನು ತೆಗೆದುಕೊಂಡೇ ಹೇಳಬೇಕು. ಹದಿನಾರನೆಯ ಒಂದು (1/16)ಎನ್ನುವಾಗ ಹದಿನಾರು ಅಂಶಗಳು ಒಂದರಲ್ಲಿ ಸೇರಿದರೆ ಪೂರ್ಣವಾಗುತ್ತೆ. ಹಾಗಿಲ್ಲದೇ ಬಿಡಿಬಿಡಿಯಾಗಿದ್ದಾಗ ಅಂಶಗಳೇ ಆಗಿರುತ್ತವೆ.

\section*{ಮೂಲದ ಸ್ಮರಣೆಗಾಗಿ ಮಂಗಳಗಾನ}

ಇಷ್ಟು ಮಾತನಾಡಿ ವೇಣುಗಾನ ಲೋಲನ ಪದತಲದಲ್ಲಿ ಕುಳಿತು (ಗುರು ಭಗವಂತನು ಹಜಾರದಲ್ಲಿ ವೇಣುಗಾನ ಲೋಲನ ಭಾವಚಿತ್ರದ ಕೆಳಗೆ ಕುಳಿತಿದ್ದನು) ಧ್ಯಾನ ಶ್ಲೋಕವನ್ನು ಹಾಡುತ್ತೇನೆ.

\begin{shloka}
ಕರಕಮಲನಿದರ್ಶಿತಾತ್ಮಮುದ್ರಃ ಪರಿಕಲಿತೋನ್ನತ  ?????????????????????????????? ಚೂಡಃ |\\%%%%%%%%%%%%%%69
ಇತರಕರಗೃಹಿತ ವೇತ್ರತೋತ್ರಃಮಮ ಹೃದಿ ಸನ್ನಿಧಿಮಾತನೋತು ಶೌರಿಃ ||
\end{shloka}

ಯಾವ ಶಂಕರರು, ಯಾವ ಭಗವತ್ಪಾದರು ಯಾವ ಆತ್ಮಶ್ರೀನಿವಾಸನನ್ನು ಕೊಂಡಾಡಿದ್ದಾರೋ, ಯಾವ ಭಗವದ್ರಮಾನುಜರು ಜ್ಞಾನಾಚಾರ್ಯನಾದ ಕೃಷ್ಣನನ್ನು ಕೊಂಡಾಡಿದ್ದಾರೋ ಆ ಸ್ವಾಮಿಯನ್ನು ಧ್ಯಾನಿಸುತ್ತೇನೆ. ಅವನ್ನೆ ಗುರುಗಳಿಗೆಲ್ಲಾ ಆದಿಗುರುವಾಗಿದ್ದಾನೆ. ಆದ್ದರಿಂದಲೇ ಗುರುಪರಂಪರೆಯನ್ನು ಹೇಳುವಾಗ-

\begin{shloka}
ಲಕ್ಷ್ಮೀನಾಥಸಮಾರಂಭಾಂ ನಾಥಯಾಮುನಮಧ್ಯಮಾಮ್|\\
ಅಸ್ಮದಾಚಾರ್ಯ ಪರ್ಯಂತಾಂ ವಂದೇ ಗುರುಪರಂಪರಾಮ್ ||
\end{shloka}

\begin{shloka}
ಸದಾಶಿವ ಸಮಾರಂಭಾಂ ಶಂಕರಾಚಾರ್ಯಮಧ್ಯಮಾಂ|\\
ಅಸ್ಮದಾಚಾರ್ಯ ಪರ್ಯಂತಾಂ ವಂದೇ ಗುರುಪರಂಪರಾಮ್ |
\end{shloka}
ಎಂದು ಲಕ್ಷ್ಮೀನಾಥನಿಂದ ಸದಾಶಿವನಿಂದ ಆರಂಭವಾಗಿದೆ. ಅವನೊಡನೆ ಸೇರಿಕೊಂಡಿದ್ದರೆ ಬೆಲೆ.

\section*{ಆಚಾರ್ಯರ ಅವತಾರದ ಹಿನ್ನೆಲೆ}

ಯಾವ ವಸ್ತುವಾದರೂ ಅದರ ಸ್ಥಾನದಲ್ಲಿದ್ದರೆ ಬೆಲೆ. ಸ್ಥಾನಭ್ರಷ್ಟವಾದರೆ ಶೋಭೆಯಿಲ್ಲ. ದಾಶರಥಿಗೆ ಹಲ್ಲಿದೆ, ನನಗೂ ಇದೆ. ಆ ಹಲ್ಲು ಬಾಯೊಳಗೆ ಅದರ ಸ್ಥಾನದಲ್ಲಿದ್ದರೆ ಬೆಲೆ. ಸ್ಥಾನಬಿಟ್ಟು ಬಿದ್ದು ಹೋದರೆ `ನನ್ನ ಹಲ್ಲು' ಎಂದು ಹಾಗೆಯೇ ಇಡುವುದಿಲ್ಲ. ಎಸೆದು ಬಿಡುತ್ತಾರೆ.

\begin{shloka}
`ಸ್ಥಾನಭ್ರಷ್ಟಾನ ನ ಶೋಭಂತೇ ದಂತಾಃ ಕೇಶಾಃ ನಖಾಸ್ತಥಾ|\\
ಸ್ಥಾನಭ್ರಷ್ಟಾ ಹಿ ಶೋಭಂತೇ ದಂತಾಃ ಕೇಶಾಃ ನಖಾಸ್ತಥಾ||'
\end{shloka}

ತುಂಬಾ ಉದ್ದವಾದ ಕೂದಲು, ಮೃದುವಾಗಿದೆ ಎಂದರೂ ತಲೆಯಲ್ಲಿರುವಾಗ ಮಾತ್ರ. ಅಲ್ಲಿಂದ ತೆಗೆದರೆ ಅದನ್ನು ಎಸೆಯಬೇಕು. ಅಂತೆಯೇ ಉಗುರು.

ಹೀಗೆ ನಾವು ಯಾರ ವಿಷಯವನ್ನು ಕುರಿತು ಮಾತನಾಡುತ್ತೇವೋ ಅವರಿಗೂ ನಮಗೂ ಇರುವ ಸಂಬಂಧವೇನು? ಅವರ ಟೈಂಸ್ಪೇಸನ್ನರಿತು ಮಾತನಾಡಬೇಕು. ಸ್ಥಾನಭ್ರಷ್ಟ ಮಾಡಬಾರದು.

ಆದರೆ ಕೆಲವೆಡೆಗಳಲ್ಲಿ ಕೆಲವು ಸ್ಥಾನಭ್ರಷ್ಟವಾದರೂ ಶೋಭೆಯುಂಟು. ಚಮರೀಮೃಗದ ಕೂದಲು, ಹುಲಿಯುಗುರು, ಆನೆಯ ದಂತ ಇವು ಆಭರಣವಾಗುತ್ತವೆ. ಹೀಗೆ ಭವಂತ ತನ್ನ ಸ್ಥಾನ ಬಿಟ್ಟು ಬಂದರೂ ಅವನ ಶೋಭೆಗೆ ಚ್ಯುತಿಯಿಲ್ಲ. ಭಗವಂತ ತನ್ನ ಸ್ಥಾನದಲ್ಲೇ ಇದ್ದು ಬಿಟ್ಟರೆ ನಮಗೆ ತಿಳಿಯುವುದಿಲ್ಲ. ಅವನು ನಮಗೆ ತಿಳಿಯುವುದಕ್ಕಾಗಿ ನಮಗಾಗಿ ಇಳಿದು ಬರಬೇಕಾಗಿದೆ. ಹೀಗೆ ಇಳಿದು ಬರುವುದಕ್ಕೆ  ಅವತಾರವೆಂದು ಹೆಸರು. ರಾಜನ್ ಮಗುವೇ ಆದರೂ ಕೆಸರಿನಲ್ಲಿ ಬಿದ್ದರೆ, ಅದನ್ನು ಎಳೆದುಕೊಳ್ಳಬೇಕಾದಲ್ಲಿ ಕೆಸರಿಗೇ ಬರಬೇಕು. ಹಾಗಿ ಭಗವಂತನೂ ತನ್ನನ್ನು ಬಿಟ್ಟುಬಂದಿರುವ ಕಂದಗಳನ್ನು ತನ್ನಲ್ಲಿಗೊಯ್ಯಲು ಬರುವುದುಂಟು. ಹೀಗೆ ಭಗವದಾ-ಜ್ಞಾನುವರ್ತಿಗಳಾಗಿ ಹಲವು ಆಚಾರ್ಯರುಗಳು ಲೋಕಹಿತಕ್ಕೋಸ್ಕರ ಕೆಳಗಿಳಿದು ಬಂದಿದ್ದಾರೆ. 

\section*{ಮಹಾತ್ಮರೇ ಆದರೂ ಲೋಕದ ಜನರನ್ನು ಮೇಲೆತ್ತಲು ಉಪಾಯ ಅಗತ್ಯ}

ಶಂಕರರು, ರಾಮಾನುಜರು ಲೋಕಹಿತಕ್ಕಾಗಿ ಬಂದು ಅವರ ಭಾಷೆಯಲ್ಲೇ ಆಡಿಬಿಟ್ಟಿದ್ದರೆ ಯಾರಿಗೂ ಅರ್ಥವಾಗುತ್ತಿರಲಿಲ್ಲ. ಒಬ್ಬ ವ್ಯಾಕರಣವಿದ್ವಾಂಸನು ಮಕ್ಕಳಿಗೆ ಪಾಠ ಹೇಳುವಾಗ ತನ್ನ ಪಾಂಡಿತ್ಯದ ಮಟ್ಟದಲ್ಲಿ ಹೇಳಿಬಿಟ್ಟರೆ ಅರ್ಥವಾಗುವುದಿಲ್ಲ. ಸ್ವಲ್ಪಸ್ವಲ್ಪವಾಗಿ ಮಕ್ಕಳಿಗೆ ಪರಿಚಯಕೊಡಬೇಕು. ಮೊದ-ಮೊದಲು ಅದಕ್ಕೆ ಬೇಕಾದ ತಿಂಡಿಸಾಮಾನನ್ನು ಕೊಟ್ಟು `ಪಾಠಕ್ಕೆ ಹೋದರೆ ತಿಂಡಿ ಸಿಕ್ಕುತ್ತೆ' ಎಂದು ಮಾಡಿ ಕ್ರಮವಾಗಿ ಪಾಠ ಕಲಿಸಬೇಕು. ಶಂಕರರಿಗಾಗಲೀ ರಾಮಾನುಜರಿಗಾಗಲೀ ಇದರಿಂದ ಅವರ ಸ್ವಂತಲಾಭವೇನೂ ಇಲ್ಲ. ತನ್ನ ಲಾಭಕ್ಕೇ ಆದರೆ ಅವರು ಇತರರಲ್ಲಿಗೆ ಬರಲೇಬೇಕಾಗಿಲ್ಲ. ಮಹಾಜ್ಞಾನವನ್ನೇ ಇಟ್ಟುಕೊಂಡಿರುವ ಅವರಿಗೆ ಲೋಕಕ್ಕಾಗಿ ವಿಷಯಕೊಡಬೇಕಾದಾಗ ಅದನ್ನೇ ನೇರವಾಗಿ ಬಿಚ್ಚಲಾಗಲಿಲ್ಲ. ಲೋಕಕ್ಕೆ ಬೇಕಾದ ಸಾಮಾನನ್ನು ಅವರು ತರಬೇಕು. ಮಕ್ಕಳಿಗೆ ಬೇಕಾದಂತಹ ಬಿಸ್ಕತ್ತೋ ಮತ್ತೊಂದೋ ತರುವಂತೆ. ಇಲ್ಲದಿದ್ದರೆ ಎಷ್ಟು  ದೊಡ್ಡ ಬದುಕು ತಂದರೂ ಅವರ ಹತ್ತಿರಕ್ಕೇ ಯಾರೂ ಸುಳಿಯುವುದಿಲ್ಲ. ಆದ್ದರಿಂದ ಲೋಕದಲ್ಲಿರುವವರಿಗೆ ಅವರಿಗೆ ಬೇಕಾದ್ದನ್ನು ಕೊಟ್ಟು ನಂತರ ತಮ್ಮದನ್ನು ಬಿಚ್ಚಬೇಕು.


\section*{ಯಾಮುನರ ಉದಾಹರಣೆ}

ಇದಕ್ಕೊಂದು ಕಥೆ ಇದೆ-ಹಿಂದೆ ನಾಥಮುನಿಗಳ ಕಾಲದಲ್ಲಿ ಅವರ ಶಿಷ್ಯರ ಪೈಕಿ `ಮಣಕ್ಕಾಲ್ ನಂಬಿ' ಎನ್ನುವವರಿದ್ದರು. ಅವರ ಕಾಲದಲ್ಲಿ ಅವರ ಮೊಮ್ಮಕ್ಕಳಾದ ಯಾಮುನರು ಇನ್ನೂ ಚಿಕ್ಕವರಾಗಿದ್ದರು. ಆಗ ಅವರು ತಮ್ಮ ವಿಷಯವನ್ನು  ಮುಂದುವರಿಸುವ  ಜವಾಬ್ದಾರಿಯನ್ನು `ಮಣಕ್ಕಾಲ್ ನಂಬಿ' ಎಂಬ ಶಿಷ್ಯನಿಗೆ ವಹಿಸಿದರು. ಯಾಮುನರು ವಾದದಲ್ಲಿ ಗೆದ್ದು ರಾಜ್ಯಾಧಿಪತಿಗಳಾಗಿದ್ದರು. ರಾಜ್ಯಾಧಿಪತಿಗಳಾದ ಅವರಲ್ಲಿಗೆ `ಮಣಕ್ಕಾಲ್ ನಂಬಿ' ಗಳು ಹೋಗುವುದಾದರೂ ಹೇಗೆ? ಆದರೆ ಗುರುಗಳು ತಮ್ಮ ಜ್ಞಾನಸಂಪತ್ತನ್ನು ಅವರಿಗೆ ಕೊಡುವಂತೆ ವಹಿಸಿದ ಜವಬ್ದಾರಿಯಿದೆ. ಅದನ್ನು ಹೇಗೆ ನಿರ್ವಹಿಸುವುದು? ಕೊನೆಗೆ ಪ್ರತಿದಿನವೂ ಒಂದುತರಹದ ಸೊಪ್ಪನ್ನು  (ತೂದುವಳೈಕ್ಕೀರೈ) ರಾಜನಿಗೆ ಅಡಿಗೆ ಮಾಡಿ ಬಡಿಸಲು ತರಕಾರಿಯಾಗಿ ಕೊಂಡೊಯ್ಯುತ್ತಿದ್ದರು. ರಾಜನಿಗೆ ಅದೊಂದು ಹೊರ ತಿಂಡಿಯಾಗಿದ್ದಿತು. ಹೀಗೆ ಪ್ರತಿದಿನವೂ ತಪ್ಪದೆ ಈ ಸೊಪ್ಪಿನ ಪದಾರ್ಥ ರಾಜನಿಗೆ ತಲುಪಿ ಅದು ಚೆನ್ನಾಗಿ ಹಿಡಿಸಿತು. ಒಂದು ದಿನ ಅವರು ಈ ಸೊಪ್ಪನ್ನು ತ್ರಲಿಲ್ಲ. ಆಗ ರಾಜನು `ಆ ಸೊಪ್ಪು  ಏಕೆ ಇಲ್ಲ?" ಎಂದು ಕೇಳಿದ. `ಯಾರೋ ಒಬ್ಬರು ಯಾವ ಪ್ರತಿಧನವನ್ನೂ ತೆಗೆದುಕೊಳ್ಳದೆ ತಂದುಕೊಡುತ್ತಿದ್ದಾರೆ ಇಂದು ಅವರು ಬರಲಿಲ್ಲ.' ಎಂದು ಅಡೆಗೆಯವರು ಹೇಳಿದರು. ನಂತರ ಮತ್ತೊಮ್ಮೆ ಅವರು ಬಂದಾಗ ಅವರನ್ನು ಅಸ್ಥಾನಕ್ಕೆ ಕರೆಸಿದರು. ಆ ಸಮಯಕ್ಕೆ ಸರಿಯಾಗಿ ರಾಜ್ಯಕ್ಕೆ ಧನದ ಅಗತ್ಯವಿತ್ತು. ಆ ಸಂದರ್ಭಕ್ಕೆ ಸರಿಯಾಗಿ ಹೋಗಿ ಇವರು `ನಿಮ್ಮ ತಾತನವರು ನನ್ನಲ್ಲಿ ಯಾವುದೋ ಒಂಡು ದೊಡ್ಡ ನಿಧಿಯನ್ನು ಇಟ್ಟಿದ್ದಾರೆ, ನನ್ನ ಜೊತೆ ಬಂಡರೆ ಅದನ್ನು ಕೊಡುತ್ತೇನೆ' ಎಂದರು. ಆಗ ರಾಜನನ್ನು ಶ್ರೀರಂಗಕ್ಕೆ  ಜೊತೆಯಲ್ಲಿ ಕರೆದೊಯ್ದು, ಶ್ರೀರಂಗನಾಥನನ್ನ ತೋರಿಸಿ, `ಇದೇ ಆ ನಿಧಿ' ಯೆಂದು ಹೇಳಿ ಅದರ ಮರ್ಮದೊಡನೆ ಇತ್ತರು. ಅವರ ತಾತನ ಅಸ್ತಿಯನ್ನು ಅವರಿಗೇ ಕೊಟ್ಟರು. ಅವರ ಜ್ಞಾನ ಭಂಡಾರವನ್ನು ಅವರಿಗೇ ತುಂಬಿಕೊಟ್ಟರು. ಹೀಗೆ ದೊಡ್ಡ ವಿಷಯವನ್ನು ಕೊಡಬೇಕಾದರೆ ಯಾವುದಾದರೊಂದು ಉಪಾಯವನ್ನು ಅವಲಂಬಿಸಬೇಕಾಗುತ್ತದೆ.

\section*{ಲೋದಲ್ಲಿ ಆಚಾರ್ಯನ ಪಾತ್ರ}

ಜ್ಞಾನಿಗಳಾದವರು ಲೋಕಕ್ಕೆ ತಮ್ಮ ಜ್ಞಾನವನ್ನು ಕೊಡಬೇಕಾದರೆ ಲೋಕಕ್ಕೆ ಬೇಕಾದುದನ್ನು ತರಬೇಕು. ಪೇಪ್ಪರ್ ಮೆಂಟ್ ಕೊಟ್ಟರೆ ಮಕ್ಕಳು ಒಡನೆಯೇ ಮಾಮಾ ಎನ್ನುತ್ತವೆ. ಇಲ್ಲದಿದ್ದರೆ ಮಾ, ಮಾ(ಬೇಡ ಬೇಡ) ಎಂದು ನಿರಾಕರಣೆಯೇ ಗತಿ. ಅಂತೆಯೇ ಜ್ಞಾನಿಗಳಾದವರು ತಮ್ಮ ಅಭಿಪ್ರಾಯವನ್ನು ತಿಳಿಸಬೇಕಾದರೆ ಅವರವರ ಮಟ್ಟಕ್ಕಿಲಿಯಬೇಕು. `ನೀರಿಳಿಯದ ಗಂಟಲಿನಲ್ಲಿ ಕಡುಬನ್ನು ತುರುಕುವುದಕ್ಕಾಗುವುದಿಲ್ಲ.' ನಮ್ಮ ಹುಡುಗ ತಪ್ಪಿಸಿಕೊಂಡು ಹೋಗಿದ್ದರೆ, ಅವನಿರುವ ಜಾಗಕ್ಕೇ ಹೋಗಿ ಅವನನ್ನು ಕರೆತರಬೇಕು. ಅಜ್ಞಾನದಲ್ಲಿ ತೊಳಲುವ ಪಾಮರರನ್ನು ಉದ್ದಾರಮಾಡುವುದಕ್ಕೋಸ್ಕರ ಬಂದವರು ಅವರು. ಹೀಗೆ ಅವರು ನಮ್ಮವರೆಗೆ ಬಂದು ಜ್ಞಾನಾ ಚೈತನ್ಯವನ್ನು ಕೊಟ್ಟು ಬೆಳೆಸುವುದಕ್ಕೆ ಬಂದಿದ್ದಾರೆ. ಲೋಕದಲ್ಲಿ ಇಂದ್ರಿಯಗಳಿಗೆ ಬೇಕಾದ ವಿಷಯವನ್ನು ಕೊಟ್ಟು ಬಾಳಾಟ ನಡೆಸುವುದನ್ನು ನೋಡಿದ್ದೇವೆ. ಅಂದರೆ ಕಣ್ಣಿಗೆ ರೂಪವನ್ನು ಕೊಡಬಹುದು, ಮೂಗಿಗೆ ವಾಸನೆಯನ್ನು ಕೊಡಬಹುದು. ಹಾಗೆಯೇ ಆಯಾ ಇಂದ್ರಿಯಗಳಿಗೆ ಆಯಾ ವಿಷಯವನ್ನು  ಕೊಟ್ಟಾಯಿತು. ಬರೀ ಇಂದ್ರಿಯಗಳು ಮಾತ್ರ ನಮ್ಮ ಜೀವನದಲ್ಲಿವೆಯೇ? ಮನಸ್ಸು, ಬುದ್ಧಿ, ಆತ್ಮ ಮೊದಲದವುಗಳೂ ಇವೆಯಲ್ಲಾ! ಇದಕ್ಕೆಲ್ಲಾ ಏನು ವಿಷಯ? ಮನಸ್ಸಿಗೆ ಒಂದು ಆಲೊಚನೆ, ಬುದ್ಧಿಗೆ ಒಂದು ನಿರ್ಣಯ, ಜೀವಕ್ಕೆ ಆನಂದ. ಅದನ್ನು ಕೊಡಬೇಕಾದರೆ ಒಬ್ಬ ಆಚಾರ್ಯ ಬೇಕು. ಅದರ ಪೈಕಿ ಶಂಕರರು, ರಾಮಾನುಜರು ಕೊಟ್ಟದ್ದೇನು? ಎನ್ನುವುದನ್ನು ವಿಚಾರಿಸಿ ನೋಡಬೇಕು. 

\section*{ಜ್ಞಾನವನ್ನು ಲೋಕದಲ್ಲಿ ಹೇಗೆ ಬೆಳಸಬೇಕು?}

ಪ್ರಪಂಚಕ್ಕೆ ಯಾವುದಾದರೂ ಒಂದು ವಿಷಯ ಕೊಡಾಬೇಕಾದರೆ, ವಿಷಯವುಳ್ಳವರು ಈ ಪ್ರಪಂಚಕ್ಕೆ ಬರಬೇಕು. ಈ ಜಾಗದಿಂದಲೇ ಮೇಲಕ್ಕೊಯ್ಯಬೇಕು. ಅವರಿಲ್ಲಿ ಬಂದು ತೋರಿಸಿದರೆ `ನಮ್ಮ ಮನೆಗೂ ಬನ್ನೀಂದ್ರೆ' ಎಂದು ಯಾರಾದರೂ ಕರೆಯಬಹುದು. ಅವರು ಲೋಕದಂತೆ ವರ್ತಿಸಿ, ಅವರವರಿಗೆ ಬೇಕಾದುದನ್ನು ತೆಗೆದುಕೊಂಡು, ನಾಲ್ಕಾರು ಜನಗಳ ಮನೆಗೆ ಹೋದರೆ ಮತ್ತೆ ಮತ್ತೆ ಆಹ್ವಾನ ಬರುತ್ತದೆ. ಆಗ `ನಮ್ಮ ಮನೆಗೂ ಬನ್ನಿ' ಎಂದು ಅವರೂ ತಮ್ಮಲ್ಲಿಗೆ ಕರೆಯಬೇಕು. ಮತ್ತು ನಮ್ಮ ಮನೆಗೆ ಬರುವಾಗ ಅವರ ವಿಷಯವೂ ಇರಬೇಕು. ಕೇವಲ `ಪಾನಕ, ಮಾಂಗಾಯ್ ಓಗರೆ' ಇಷ್ಟೇ ಆದರೆ ಬರೀ ಇಂದ್ರಿಯಗಳಿಗಾಯಿತು. ಇಂದ್ರಿಯಗಳ ವಿಷಯವನ್ನು ತೆಗೆದುಕೊಂಡು, ಇದರ ಜೊತೆಗೆ ಆತ್ಮಕ್ಕೆ ಕೊಟ್ಟ ವಿಷಯವೇನುಂಟೋ ಅದನ್ನೂ ಆಸ್ವದನೆ ಮಾಡಬೇಕಿತ್ತು. ಆದರೆ ಆ ಆಸ್ವಾದನೆ ಮರೆಯಾಗಿದೆ.

\section*{ವಿಪಶ್ಚಿತಚೇತರಾಗಿ ಆತ್ಮನನ್ನು ನೋಡಬೇಕು}

ಆ ಆತ್ಮವನ್ನು ಹೇಗೆ ನೋಡಿ ಹೇಳಿದರು? ಲೋಕದಲ್ಲಿ ಅನೇಕ ತರಹ ನೋಟಗಳಿವೆ. ಮಕ್ಕಳ ನೋಟವೇ ಒಂಡು ತರಹ, ಚಾಳೇಶ\footnote{ಚಾಲೀಸಾ ಅಂದರೆ ನಲವತ್ತುವರ್ಷಕ್ಕೆ ಬರುವ ಕಣ್ಮಂಜು. ಅದನ್ನು ಚಾಳೇಶವೆನ್ನುವರೂಢಿಬಂದಿದೆ.} ಬಂದವರ ನೋಟವೇ ಒಂದು ತರಹ. ಕಣ್ಣಲ್ಲಿ ಪೊರೆ ಬೆಳೆದರೆ, ಅವರದೇ ಒಂದು ತರಹ ನೋಟ, ಆದ್ದರಿಂದ ಲೋಕದಲ್ಲಿ ಒಂದೇ ತರಹ ದರ್ಶನವಿಲ್ಲ. ಅವರವರ ಕಣ್ಣಿದ್ದಂತೆ ನೋಡುವ ಒಂದು ಸನ್ನಿವೇಶವಿರುತ್ತದೆ. ಹೀಗಿರುವಾಗ ವಿಷಯವನ್ನು ವಾಸ್ತವವಾಗಿ ತೆಗೆದುಕೊಂಡು ನೋಡುವವರು ಯಾರು? ಉದಾಹರಣೆಗೆ ಎರಡು ರೈಲುಗಳಿವೆ-ಒಂದರ ಪಕ್ಕದಲ್ಲಿ ಒಂದು ನಿಂತಿದೆ. ನಾವು ರೈಲಿನಲ್ಲಿ ಕುಳಿತಿದ್ದರೆ, ಪಕ್ಕದ ರೈಲು ಚಲಿಸಿದಾಗ ನಮ್ಮ ರೈಲು ಚಲಿಸಿದಂತಾಗುತ್ತದೆ. ನಮ್ಮ ದರ್ಶನವನ್ನೇ ಪ್ರಮಾಣವೆಂದುಕೊಂಡಾಗ `ನಮ್ಮ ರೈಲು ಚಲಿಸುತ್ತದೆ' ಎಂದುತಾನೇ ಹೇಳಬೇಕಾಗುತ್ತದೆ. ಆದ್ದರಿಂದ ಪ್ರತ್ಯಕ್ಷವಾಗಿ ನೋಡಿದ್ದೇನೋ ಸರಿ. ಆದರೂ ಅದನ್ನೂ ಪ್ರಮಾಣಿಸಿ ನೋಡಬೇಕು. `ಪ್ರತ್ಯಕ್ಷಿಸಿ ನೋಡಿದರೂ ಪ್ರಮಾಣಿಸಿ ನೋಡಬೇಕುಂಬುದು ಗಾದೆ.' ಹಿಂತಿರುಗಿ ಸ್ಟೇಷನ್ ಕಡೆಗೆ ನೋಡಿದರೆ ಮಾತ್ರ ಯಾವ ರೈಲು ಚಲಿಸುತ್ತದೆ ಎಂಬುದು ತಿಳಿಯುತ್ತದೆ. ಆದ್ದರಿಂಡ ಮುಂದಿನದನ್ನು ಮಾತ್ರ ನೋಡದೆ `ವಿಪಶ್ಚಿತಚೆತ' ರಾಗಿ ಹಿಂದಿನದನ್ನೂ ನೋಡಬೇಕು. ಹಾಗೆ ಹಿನ್ನೋಟವಿಲ್ಲದಿದ್ದರೆ ಸವಿರಾರು ರೂಪಾಯಿ ಸಂಬಳ ತೆಗೆದುಕೊಂಡರೂ `ತಾನು ಕುಳಿತಿರುವ ರೈಲೇ ಹೋಗುತ್ತದೆ' ಎಂದುಕೊಂಡು ಹೆಡ್ಡನಾಗುತ್ತಾನೆ.

\section*{ಭ್ರಮೆ ನಿವಾರಣೆಗೆ ಜ್ಞಾನಿಗಳ ಮಾರ್ಗದರ್ಶನವಿರಬೇಕು}

ಈ ತರಹ ಭ್ರಮೆ ಬರುವ ಜಾಗಗಳು ಬಹಳ ಹೆಚ್ಚು. ಬಹಳ ಬಿಸಿಲಾಗಿದ್ದಾಗ ಮರುಭೂಮಿಯಲ್ಲಿ ನೀರಿರುವಂತೆ ತೋರುತ್ತದೆ. ಆದರೆ ಅಲ್ಲಿ ಹೋದಾಗ ನೀರಿರುವುದಿಲ್ಲ. ಹೀಗೆ ತೋರುವುದಕ್ಕೆ ಬಿಸಿಲ್ಗುದುರೆ ಎನ್ನುತ್ತಾರೆ. ಒಂದೆಡೆ ಹೋದರೆ ಮತ್ತೊಂದೆಡೆ ನೀರಿರುವಂತೆ ತೋರುತ್ತದೆ. ತಲೆ ಸುತ್ತಿದರೆ ಮನೆ ಸುತ್ತಿದಂತೆ ಕಾಣುತ್ತದೆ. `ಹೇಗೆ ಮನೆ ಸುತ್ತುತ್ತಿದೆ ! ನೋಡಿ !' ಎಂದು ಬಿಡುತ್ತಾನೆ. `ತಲೆ ಸುತ್ತುತ್ತಿದೆ' ಎಂದರೆ, ತಲೆಯನ್ನು ಭದ್ರವಾಗಿ ಹಿಡಿದುಕೊಂಡು, `ಏನೂ ಸುತ್ತುತ್ತಲೆ ಇಲ್ಲವಲ್ಲಾ? ಎನ್ನಬಹುದು. ಆದರೆ ಇವನಲ್ಲುಂಟಾದ ವಿಕಾರ ಹಾಗೆ ಮಾಡುತ್ತದೆ. ನಾಲಿಗೆಯಲ್ಲಿ ಪಿತ್ತವೇರಿದ್ದರೆ ಆಹಾರವನ್ನು ಕಹಿ ಎಂದು ಬಿಡುತ್ತಾನೆ. ಕಣ್ಣಿನಲ್ಲಿ ಪಿತ್ತ ವ್ಯಾಪಿಸಿದ್ದರೆ, ಬೆಳ್ಳಗಿರುವುದನ್ನೂ ಹಳದಿ ಎನ್ನುತ್ತಾನೆ. ಆಯಾ ಟೈಂಸ್ಪೇಸಿನಲ್ಲಿ ಕಣ್ಣಿಗೆ ಕಾಣುವುದು ಅಷ್ಟೇ. ಹೀಗೆ ಪ್ರಕೃತಿಯಲ್ಲಿ  ಭ್ರಮೆಹೋಗುವ ಜನಗಳು ಬಹಳ. ಕಣ್ಣು  ಎಷ್ಟು ನೋಡಬಲ್ಲದೊ ಅಷ್ಟನ್ನೂ ನೋಡುತ್ತಾನೆ. ಒಂದು ಪಾವು ನೀರು ಹಿಡಿಸುವ ಪಾತ್ರಗೆ ಅಷ್ಟೇ ಹಿಡಿಸುತ್ತದೆ. ಹತ್ತು ಪಾವು ಹಿಡಿಸುವುದಾದರೆ ಅಷ್ಟನ್ನೂ ತಡೆದುಕೊಳ್ಳುತ್ತದೆ. ಆದ್ದರಿಂದ ಜ್ಞಾನಿಗಳ ಡೈರೆಕ್ಷನ್ನನ್ನು ಅನುಸರಿಸಬೇಕಾಗುತ್ತದೆ.

\section*{ಜ್ಞಾನಿಗಳು ಬಳಸುವ ಉಪಾಯವನ್ನು ಜಾಗರೂಕತೆಯಿಂಡ ಗ್ರಹಿಸಬೇಕು}

ಕೆಲವು ವೇಳೆ ಜ್ಞಾನಿಗಳು ಉಪಾಯದಿಂದ ವಿಷಯವನ್ನು ಕೊಡಬೇಕಾಗುತ್ತದೆ. ಮಲೇರಿಯಾ ಬಂದಿದೆ. ಆಗ ನೇರವಾಗಿ `ಕ್ವಿನೈನ್ ತೆಗೆದುಕೊಳ್ಳಯ್ಯ' ಎಂದರೆ ತೆಗೆದುಕೊಳ್ಳುವುದಿಲ್ಲ. ಆದರೆ ಅದಕ್ಕೆ ಷುಗರ್ ಕೋಟಿಂಗ್ ಕೊಟ್ಟು `ಸಕ್ಕರೆ' ಎಂದರೆ ತೆಗೆದುಕೊಳ್ಳುತ್ತಾನೆ. ಹೀಗೆ ಯಾವುದಾದರೂ ಉಪಾಯದಿಂದ ವಿಷಯ ಕೊಡುವುದುಂಟು. ಬುದ್ಧಿಶಾಲಿಯಾದವನು ಜ್ಞಾನಿಗಳ ಮಾರ್ಗವನ್ನು ಅನುಸರಿಸಿ ಹೋಗಬೇಕು. ಜ್ಞಾನಿಗಳಾಗಿರುವವರು ವಿಷಯವನ್ನು ಕೊಡುವಾಗ, ನಲೆಯನ್ನು ತೋರಿಸುವಾಗ ಗ್ರಹಿಸುವವನು ಜಾಗರೂಕತೆಯಿಂದ ನೋಡಬೇಕು.

\section*{ಕಾಲದೇಶಗಳ ಭೇದದಿಂದ ದರ್ಶನ ಭಿನ್ನವಾಗುತ್ತದೆ}

ಒಂದೇ ವಿಷಯದಲ್ಲಿ ವಿಭಿನ್ನವಾದ ಟೈಂಸ್ಪೇಸಿನ ದರ್ಶನಗಳು ಬೇರೆ ಬೇರೆ  ರೀತಿಯಾಗಿರಬಹುದು. ಆಯಾ ಟೈಂಸ್ಪೇಸಿನಲ್ಲಿ ನೋಡಿದಾಗ ಆಯಾ ದರ್ಶನಗಳು ಸತ್ಯವಾಗಿಯೇ ಇರುತ್ತವೆ. ಒಂದೇ ಬೆಟ್ಟನ್ನು `ಎರಡು' ಎಂದು ಹೇಳಿದರೆ, ಅದು ನಿಜವೇ? ಸುಳ್ಳೇ? ಎಂದು ಕೇಳಿದರೆ ತಕ್ಷಣ ಸುಳ್ಳು ಎನ್ನಿಸುತ್ತೆ, ಆದರೆ ಆಯಾ ಟೈಂಸ್ಪೇಸಿನಲ್ಲಿ ನೋಡಿದಾಗ ಎಲ್ಲಾ ನಿಜವೇ. ಹೇಗೆಂದರೆ ಒಂದು ಕಣ್ಣನ್ನು ಮುಚ್ಚಿ ಅಥವಾ ಎರಡೂ ದೃಷ್ಟಿಯು ಸಂಧಿಸುವ ಜಾಗದಿಂಡ ಹಿಂದಕ್ಕೆ ಬೆರಳನ್ನೊಯ್ಯು ನೊಡಿದಾಗ ಬೇರೆ ಬೇರೆಯಾಗಿ ಗ್ರಹಣವಾಗುತ್ತೆ. ಹೇಗೆಂದರೆ  ಒಂದೇ ದೀಪವನ್ನು ಇಬ್ಬರು ಪ್ರತ್ಯೇಕವಾಗಿ ಒಂದು, ಒಂದು ಎಂದರೆ ಎರಡು ಎಂದಾಗುವುದೋ ಹಾಗೆ. `ಎರಡೂ ನಿಜ' ಎನ್ನುವುದನ್ನು ಬೇರೆ ಬೇರೆ ಟೈಂಸ್ಪೇಸಿನಲ್ಲಿ ತೆಗೆದುಕೊಳ್ಳಬೇಕು.

\section*{ಶಂಕರ-ರಾಮಾನುಜರ ದರ್ಶನದ ಮರ್ಮ}

ಪ್ರಕೃತಿಯ ಪ್ರವಾಹದಲ್ಲಿ ಸಿಕ್ಕಿಕೊಂಡ ಜೀವಿಗಳು ತಮ್ಮ ಸ್ವರೂಪ ಕಳೆದುಕೊಂಡಾಗ-ಮರೆತಾಗ ನೆಲೆಯನ್ನರಿತವರು ರಹಸ್ಯವಾಅ ತತ್ತ್ವದರ್ಶನವನ್ನು ಕೊಟ್ಟರು. ಹೀಗೆ ಕೊಟ್ಟವರ ಪೈಕಿ ಶಂಕರ-ರಾಮಾನುಜರು ಸೇರಿಕೊಳ್ಳುತ್ತಾರೆ. ಈಗ ಶಂಕರ-ರಾಮಾನುಜರ ತಿರುನಕ್ಷತ್ರಗಳೆರಡೂ ಒಟ್ಟಿಗೆ ಬಂದಿವೆ. ವಿಶಿಷ್ಟಾದ್ವೈತ ಮತ್ತು ಅದ್ವೈತ ಎರಡೂ ಒಂದಾಗಿ ಬಿಟ್ಟಿದೆ.

ರಾಮಾನುಜರಿಗೆ `ತ್ರಿವಿಧ ಬ್ರಹ್ಮವಾದಿಗಳು' ಎಂದು ಹೆಸರು. ಮೂರು ಕಾಲಿನ ಮಣೆಯನ್ನು ಹಾಕಿ ಅದರ ಮೇಲೆ ಜ್ಞಾನದೀಪವನ್ನು ಹಚ್ಚಿಟ್ಟವರು ರಾಮಾನುಜರು. ಒಂದೇ ಕಾಲಿನ ರೌಂಡ್ ಟೇಬಲ್ ಇಟ್ಟು ಅದರ ಮೇಲೆ ಜ್ಞಾನದೀಪವನ್ನು ಹಚ್ಚಿಟ್ಟವರು ಶಂಕರರು. ಎರಡು ಕಾಲು ಹಾಕಿಸಿ ನಿಲ್ಲಿಸುವ ಮಣೆಯೂ ಉಂಟು. ತಮಗೆ ಬೇಕಾದ ಅಳತೆಗೆ ಸರಿಯಾಗಿ ಆಯ ನೋಡಿ ಕಾಲನ್ನಿಡಬಹುದು. ಹಾಗೆ ಮಾಡಿದ ಮೇಲೆ ಒಂದಿಂಚು ವ್ಯಾತ್ಯಾಸವಾದರೂ ಕಷ್ಟ. ನಾಲ್ಕು ಕಾಲಿನ ಮಂಚದಲ್ಲಿ ಒಂದುಕಾಲು ತೆಗೆದರೆ ಅದು ಸರಿಯಾಗಿ ನಿಲ್ಲುವುದಿಲ್ಲ. ಆದರೆ ಅದೇ ಮೂರು ಕಾಲಿನಿಂದಲೇ ನಿಲ್ಲುವಂತೆ ಕಾಲುಗಳನ್ನಿಟ್ಟಾಗ ಭದ್ರವಾಗಿಯೇ ನಿಲ್ಲುತ್ತದೆ. ಹೀಗೆ ಮೂರು ಕಾಲಿನ ಮಣೆಯೊಂದರ ಮೇಲೆ ಜ್ಞಾನದೀಪವನ್ನು ಹಚ್ಚಿಟ್ಟವರು ರಾಮಾನುಜರು. ಒಂದೇ ಕಾಲನ್ನಿಟ್ಟರೂ ಭದ್ರವಾಗಿ ನಿಲ್ಲುವ ಮಣೆಯೂ ಉಂಟು. ಅಂತಹ ಮಣೆಯೊಂದರ ಮೇಲೆ ಜ್ಞಾನದೀಪವನ್ನು ಹಚ್ಚಿಟ್ಟವರು ಶಂಕರರು. ಒಂದು ಕಾಲಿನ ಮಣೆಯ ಮೇಲೆ ದೀಪ ಹಚ್ಚಿಟ್ಟರೆ ಪ್ರಕಾಶಿಸುವುದಿಲ್ಲ ಎಂದರೆ ಅದು ಮೂರ್ಖತನವಾಗುತ್ತದೆ. `ಒಂದೇ ಕಾಲಿನ ಮಣೆ' ಎಂದರೆ ತಮ್ಮ ಜ್ಞಾನವನ್ನು ಒಂದೇ ಆಧಾರದಿಂದ ನಿಲ್ಲಿಸಿದ್ದಾರೆ. ಆದರೆ ರಾಮಾನುಜರು ಮೂರು ಆಧಾರ ಕೊಟ್ಟು ನಿಲ್ಲಿಸಿದ್ದಾರೆ. ಹೀಗೆ ರಾಮಾನುಜರು ಪರಮ ಪದಕ್ಕೆ ಸೋಪಾನ ಕಟ್ಟಿದರು. ಏಣಿ ಇಲ್ಲದೆ ಅಟ್ಟಕ್ಕೆ ಹತ್ತುವವರೂ ಉಂಟು. ಆದರೆ ಅದು ಎಲ್ಲರಿಗೂ ಸಾಧ್ಯಾವಿಲ್ಲ. ಪ್ರಕೃತಿ ಮಂಡಲದಲ್ಲಿರುವವರನ್ನು ಒಯ್ಯಲು ಏಣಿ ಬೇಕು. ಕ್ರಮಕ್ರಮವಾಗಿ ಹತ್ತಿಸಬೇಕು. ಅಂತಹವರಿಗೆ ಚಿತ್ತು-ಅಚಿತ್ತು-ಈಶ್ವರ ಎಂಬ ಒಂದು ಸಂಬಂಧ ಕೊಟ್ಟು ಕ್ರಮವಾಗಿ ಹತ್ತಿಸಿದವರು ರಾಮಾನುಜರು.

\section*{ಅದ್ವೈತ ದರ್ಶನ}

ಶಂಕರರು `ಅದ್ವೈತ' ಎಂಬುದಾಗಿ ಹೇಳಿ, ಆತ್ಮ ಒಂದೇ, ಅದನ್ನು ಬಿಟ್ಟು ಯಾವುದೂ ಇಲ್ಲ, `ನೇತಿ-ನೇತಿ' ಎಂಬುದಾಗಿ ಹೇಳಿದರು. ಅವರಿಗೆ ಅದನ್ನು ಬಿಟ್ಟು ಬಾಕಿ ವಸ್ತು ಕಾಣಲಿಲ್ಲ. ಬೇರೆ ಕಾಣುವುದಿಲ್ಲವೇ? ಎಂದರೆ ಲೌಕಿಕವಾಗಿಯೇ ತೆಗೆದುಕೊಂಡರೂ, ಮನೆಗೆ ದೊಡ್ಡ ಮನುಷ್ಯರು ಬಂದರೆ ಇತರರ ಮೇಲೆ ನಿಗಾ ಇರುವುದಿಲ್ಲ. ಅವರಿಗೆ ಆ ಕ್ಷಣದಲ್ಲಿ ಆ ರೀತಿ ತೋರಿತು. ಆತ್ಮವನ್ನು ಅನುಭವಿಸುವಾಗ ಅದೊಂದೇ ಎಂದು ತೋರಿತು. ಹಾಗೆ ಬಂದಾಗ `ಬ್ರಹ್ಮ ಸತ್ಯಂ ಜಗನ್ಮಿಥ್ಯಾ' ಎಂಬುದಾಗಿ ಹೇಳಿದರು. ಅದನ್ನು ನೋಡಿ ತಮಾಷೆ ಕಥೆಗಳೇನೋ ಬಂದಿವೆ. ಉದಾಹರಣೆಗೆ-ಅದ್ವೈತ ನಿಷ್ಣಾತರೊಬ್ಬರು ತಮ್ಮ ಶಿಷ್ಯರೊಡನೆ ರಸ್ತೆಯಲ್ಲಿ ಹೋಗುತ್ತಿದ್ದಾಗ ಆನೆಯೊಂದು ಓಡಿಸಿಕೊಂಡು ಬಂತು. ಆಗ ಅದ್ವೈತಿಗಳು ಓಡಿದರು. ಶಿಷ್ಯನು ಕೇಳಿದನಂತೆ `ಏನು ಸ್ವಾಮಿ, ಬ್ರಹ್ಮವನ್ನು ಬಿಟ್ಟು ಎಲ್ಲಾ ಮಿಥ್ಯ' ಎಂದಮೇಲೆ, ಆನೆಬಂದಾಗ ಏಕೆ ಓಡಿದಿರಿ? ಆನೆಯ ಮಿಥ್ಯೆ ತಾನೇ? ಎಂದು. ಅದಕ್ಕೆ ಅವರು ಗಜೋಪಿ ಮಿಥ್ಯಾ ತಂ ಪ್ರತಿ ಪಲಾಯನಮಪಿ ಮಿಥ್ಯಾ-ಆನೆಯು ಮಿಥ್ಯವೇ ನಾನು ಓಡಿದುದು ಮಿಥ್ಯವೇ ಎಂದರಂತೆ. ಇದು ವ್ಯಂಗ್ಯದ ವಿಷಯವಾಯಿತು.

\section*{ಆಯಾ ಟೈಮ್ ಸ್ಪೇಸ್ ನಲ್ಲಿ ದರ್ಶನಗಳು ಯಥಾರ್ಥವೇ ಆಗಿವೆ}

ಸ್ವಪ್ನವು ನಿಜವೇ? ಸುಳ್ಳೇ? ಎಂದರೆ ರಾಮಾನುಜ ಸಿದ್ದಾಂತ ನಿಜ ಎನ್ನುತ್ತದೆ. ಅದು ಯಾವ ರೀತಿ ನಿಜ? ಯಾವ ರೀತಿ ಸುಳ್ಳು? ಎಂಬುದನ್ನು ಪರಿಗಣಿಸಬೇಕು. ಸ್ವಪ್ನದಲ್ಲಿ ನಾವು ಹೊರಗಡೆ ಅನುಭವಿಸಿದಂತೆ ಅನುಭವಗಳುಂಟು. ಸ್ವಪ್ನದಲ್ಲಿ ನಾವು ಹೊರಗಡೆ ಅನುಭಸಿದಮ್ತೆ ಅನುಭವತಳುಂಟು. ಸ್ವಪ್ನದಲ್ಲಿ ಯಾರೋ ನಾಲ್ಕು ಏಟು ಹೊಡೆದಂತಾದರೆ ಅದರ ಭಯವಿರುತ್ತದೆ. ಎದ್ದಾಗಲೂ ಎದೆ ಬಡಿತ ಮೈ ಬೆವರಿರುವುದು. ಇದನ್ನು ನೋಡಿದರೆ ಸ್ವಪ್ನವು ಪರಿಣಾಮದಿಂದ ನಿಜವೆನಿಸುತ್ತದೆ. ಆದರೆ ಕಾರಣವಿಲ್ಲದೇ ಇದು ನಿಜವಾಗುವುದುಂಟೇ? ಆದ್ದರಿಂಡ ಸ್ವಪ್ನದ ಅನುಭವಗಳು ನಿಜವಾಗಿಯೂ ಒಂದು ರೀತಿ ಪರಿಣಾಮವನ್ನುಂಟುಮಾಡುತ್ತವೆ. ಆದರೆ ಸ್ವಪ್ನದಲ್ಲಿ ಕಂಡದ್ದು ಆ ಟೈಮ್-ಸ್ಪೇಸಿನಲ್ಲಿ ಮಾತ್ರ ಸತ್ಯ. ಬೇರೆ ಟೈಮ್ ಸ್ಪೇಸಿನಲ್ಲಿ ಅದರ ನಿಜಸಿಕ್ಕುವುದಿಲ್ಲ. ಸ್ವಪ್ನದಲ್ಲಿ ಮಾತ್ರ ಎಲ್ಲಾ ವ್ಯವಹಾರವೂ ನಡೆದಿರುವುದರಿಂದ ಅದೆಲ್ಲಾ ನಿಜ. ಸ್ವಪ್ನರಾಜ್ಯದಲ್ಲಿ ಚಲಾವಣೆ ಯಾಗಿರುವ ನಾಣ್ಯಗಳು ಜಾಗ್ರತ್ ಪ್ರಪಂಚದಲ್ಲಿಲ್ಲ. ಲೋಕ ವ್ಯವಹಾರದಲ್ಲೂ ಈ ರೀತಿ ಉಂಟು. ನಮ್ಮ ನಯಾ ಪೈಸಾ ವ್ಯವಹಾರ ಫ್ರಾನ್ಸಿನಲ್ಲಿಲ್ಲ. ಆದ್ದರಿಂಡ ` ಸ್ವಪ್ನ ಸತ್ಯ' ಎಂದರೆ, ಆ ಟೈಮ್ ಸ್ಪೇಸಿಗನುಗುಣವಾಗಿ ತೆಗೆದುಕೊಳ್ಳಬೇಕು.

`ಜಗನ್ಮಿಥ್ಯಾ' ಎಂದು ಶಂಕರರು ಹೇಳಿದರೂ, ಅದನ್ನು ಎಲ್ಲಾ ಟೈಮ್ ಸ್ಪೇಸಿಗೂ ಅನ್ವಯಿಸಿಕೊಳ್ಳಬಾರದು. ಅದೇ ಶಂಕರರು ಒಬ್ಬರನ್ನು ಕುರಿತು `ಮೂಢ ಜಹೀಹಿ ಧನಾಗಮತೃಷ್ಣಾಂ' ಎಂಬುದಾಗಿ ಹೇಳಿದರು. `ಧನವು ಮಿಥ್ಯೆ' ಎಂದು ತೆಗೆದು ಕೊಳ್ಳುವುದಾದರೆ ಈ ಉಪದೇಶಕ್ಕೆ ವಿಷಯವೇ ಇಲ್ಲದಂತಾಗುತ್ತದೆ. ಅವರಿಗೆ ಲೋಕದ ನಡೆಯು ಗೊತ್ತು. ಅದಕ್ಕನುಗುಣವಾದ ನೋಟವನ್ನು ನೋಡಿಯೂ ಆಡಿದ್ದಾರೆ. ಅವರೇ `ಅರ್ಥಮನರ್ಥಂ ಭಾವಯ ನಿತ್ಯಂ' ಎಂಬುದಾಗಿಯೂ ಹೇಳಿದ್ದಾರೆ. ಅಂದರೆ ಅರ್ಥವೇ ಬೇಡವೆಂದಲ್ಲ, ಅದನ್ನು ಒಂದು ಹತೋಟಿಯಲ್ಲಿಮಾತ್ರ ತೆಗೆದುಕೊಳ್ಳಬೇಕು.

\section*{ಪೂರ್ಣವನ್ನು ಮರೆತು ಅಂಶವನ್ನು ಮಾತ್ರ ಬೆಳೆಸುತ್ತಿದೆ ಲೋಕ}

ಲೋಕದಲ್ಲಿ ತಾತ್ತ್ವಿಕವಾದ ಒಂದು ವಿಷಯವನ್ನು ತೆಗೆದುಕೊಂಡು ರಾಜಕೀಯವಾಗಿ ದಂಡ ನೀತಿಯಿಂದ ತತ್ತ್ವ ಸಿಂಹಾಸಹನದಲ್ಲಿ ಭಗವಂತನ ಪ್ರತಿಷ್ಟೆಯನ್ನು ಮಾಡಿದವರೂ ಉಂಟು. ಅದನ್ನು ಮರೆತು ನಮ್ಮಷ್ಟಕ್ಕೆ ನಾವು ಅಂಶತಃ ಬಳಸಿಕೊಳ್ಳುತ್ತಿರುವುದೂ ಉಂಟು.ಉದಾಹರಣೆಗೆ ಗೋಕುಲಾಷ್ಟಮಿಯ ವಿಷಯವನ್ನೇ ತೆಗೆದುಕೊಂಡರೂ, `ಅವರ ಮನೆ ಗೋಕಲಾಷ್ಟಮಿನೇ! ಬಿಡು' ಅಂದರೆ ಆ ಮಾತು ಮಾಡುವ ಹಲವು ತಿಂಡಿಗಳ ಮೇಲೆ ನಿಂತಿದೆ. ಅದೂ ಇರಲಿ, ಅದರ ಜೊತೆಗೆ ಇನ್ನೇನು ಬೇಕು? ಎಂದರೆ ಅದನ್ನು ತಿಳಿಸಲು ಪರಿಪೂರ್ಣವಾದ ಜೀವನದ ಒಂದು ಸರಣಿಯನ್ನು ತೆಗೆದುಕೊಳ್ಳಬೇಕಾಗುತ್ತೆ. ಇಲ್ಲದಿದ್ದರೆ ಅರ್ಧಂಬರ್ಧ.

\section*{ನಾಸ್ತಿಕ ಮತ್ತು ಆಸ್ತಿಕ ದರ್ಶನ}

`ನಾಸ್ತಿಕ ದರ್ಶನ' ಎನ್ನುವುದು ಒಂದು ಪಂಥವಲ್ಲ. `ಅಸ್ತಿ' ಎನ್ನುವ ವಿಷಯ ಮನಸ್ಸಿಗೆ ಬರದಿದ್ದಾಗ `ನಾಸ್ತಿ' ಎಂದು ತಾನೇ ಹೇಳಾಬೇಕಾಗುತ್ತದೆ. ನಮ್ಮ ಕಣ್ಣಿನಿಂದ ಹುಬ್ಬಿನ ಮೇಲಿನ ವಿಷಯವನ್ನು ನಾವು ನೋಡಾಲಾರೆವು. ಅದನ್ನು ನೋಡುವ ಒಂಡು ಒಳನೋಟ ಪಡೆದಾಗ ತಾನಾಗಿಯೇ `ಅಸ್ತಿ' ಎನ್ನುವ ಬುದ್ಧಿಯು ಅದಕ್ಕೆ ತಕ್ಕ ಶಬ್ದವೂ ಬರುತ್ತದೆ. ಆಗ ಆಸ್ತಿಕ ದರ್ಶನ ವಾಗುತ್ತದೆ.

\section*{ಶಂಕರ-ರಾಮಾನುಜರು ಲೋಕಕ್ಕೆ ಅನುಗ್ರಹಿಸಿದ ಮಾರ್ಗ}

ರಾಮಾನುಜರು ತ್ರಿಭಂಗಿಯಲ್ಲಿ ತಮ್ಮ ಪ್ರಮೆಯವಾದ ವಿಷಯವನ್ನು ನಿಲ್ಲಿಸಿದ್ದಾರೆ. ಜಅ ತತ್ತ್ವವಿದೆ, ಜೀವವೂ ಇದೆ. ಇವುಗಳ ಸಹಾಯದಿಂದಲೇ ಈಶ್ವರನನ್ನು ನೋಡಲು ಸೋಪಾನ ಕಟ್ಟಬೇಕು. ಅಂದರೆ ಭುವಿಯಲ್ಲಿರುವ ಜೀವಿಯನ್ನು ದಿವಿಯ ಮಟ್ಟಕ್ಕೇರಿಸಲು ಸೋಪಾನ ಕಟ್ಟಿಕೊಟ್ಟರು ರಾಮಾನುಜರು. ಆದರೆ ಸೋಪಾನವಿಲ್ಲದೆ ರೆಕ್ಕೆ  ಕಟ್ಟಿಕೊಟ್ಟು ಭುವಿಯಿಂದ ದಿವಿಗೆ ಹಾರಿಸಿದವರು ಶಂಕರರು.

\section*{ಅಸ್ತಿ ನಾಸ್ತಿ ಎಂಬ ವ್ಯವಹಾರಕ್ಕೆ ಮೂಲ}

ಆದರೆ ಇದೊಂದೂ ತಿಳಿಯದೇ ವಾದಿಸಿದರೆ ಉಪಯೋಗವಿಲ್ಲ. ಬೀಜದಲ್ಲಿ ಏನಿದೆ? ಎಂದರೆ ಒಂದು ಗಿಡವೇ ಸಮಗ್ರವಾಗಿ ಇದೆ ಎನ್ನಬಹುದು. ಆದರೆ ಅದು ಕಾಣುವುದಿಲ್ಲ. ಕಾಣುವುದಿಲ್ಲವೆಂದರೆ ತಪ್ಪೇನಿಲ್ಲ. ಬೀಜದಲ್ಲಿ ಗಿಡ ಹೇಗಿದೆ? ಎನ್ನುವುದನ್ನು ಅರಿಯದೆ, `ಬೀಜದಲ್ಲಿ ಗಿಡವಿದೆ' ಎಂದು ಒಪ್ಪಿಕೊಳ್ಳುವುದು ಮೂರ್ಖತನವೇ. ಆದ್ದರಿಂದ ನಸ್ತಿಕನಾದರೆ ಅಪಾಯವಿಲ್ಲ. ನಾಸ್ತಿಕನಾಗಿದ್ದು ಆಸ್ತಿಕನಂತೆ ನಟಿಸಿದರೆ ಪೂರಾ ಅಪಾಯ. ಆತ್ಮನನ್ನು ನೀವೆಲ್ಲಾದರೂ ಕಂಡಿರಾ? ಎಂದರೆ ಕಾಣದಿರುವುದನ್ನು `ಕಂಡಿಲ್ಲ' ಎನ್ನಲು ಏನಡ್ದಿ? ಹೀಗೆ ನಾಸ್ತಿಕ ಪರಂಪರೆಯಲ್ಲೇ ಒಂದು ವಿಧವಾಅ ವ್ಯವಹಾರವೂ, ಆಸ್ತಿಕಪರಂಪರೆಯಲ್ಲೇ ಒಂದುವಿಧವಾದ ವ್ಯವಹಾರವೂ ಬೆಳೆದು ಬರುತ್ತದೆ. ಅವರವರ ಸಂಸ್ಕಾರಕ್ಕನುಗುಣವಾಗಿ ವ್ಯವಹಾರ ಬೆಳೆಯುತ್ತದೆ.

\section*{ಆತ್ಮವನ್ನರಿತವರು ತಂದ ವ್ಯವಹಾರಗಳು}

`ಸತ್ತು ಹೋದ' ಎನ್ನುವ ಜಾಗದಲ್ಲಿ ಬೆಳೆದು ಬಂದಿರುವ ವ್ಯವಹಾರವೇನು? `ಕಂತೆ ಒಗೆದ' `ಹಳೇ ರಾಗಿ ತರಲು ಹೋದ', `ಪಿಟೀಲು ಬಾರಿಸಿದ', ಬಾಲಂಗಚ್ಚಿ ಕಿತ್ತುಹೋಯಿತು', `ಮಣ್ಣುಪಾಲಾದ' `ನಿಶ್ಚೇಷ್ಟನಾದ' `ಗೋತ ಒಗೆದ'. ಇವೆಲ್ಲಾ ಪರ್ಯಾಯ ಶಬ್ದಗಳು - ಹೊರಗೆ ಕಾಣುವಷ್ಟನ್ನೇ ನೋಡಿ ಹೇಳುವ ಮಾತುಗಳು. ಇದರ ಜೊತೆಗೆ `ಪರಮಪದಿಸಿದ', `ದಿವಂಗತನಾದ', `ಸತ್ತುಹೋದ', `ಪ್ರಾಣಹೋಯಿತು, ಹೋದ', ಎನ್ನುವುದನ್ನು ಯಾರು ಹೇಳಬೇಕು? ಹೋದದ್ದನ್ನು ಕಂಡವರು ತಾನೇ. ಅದನ್ನು ಪತ್ತೆ ಹಚ್ಚಬಲ್ಲವರು ಇಲ್ಲದಿದ್ದರೆ, ಎಲ್ಲಿಗೆ ಹೋದ ? ಹೇಗೆ ಹೋದ? ಎಂದು ತಿಳಿಯುವುದಾದರೂ ಹೇಗೆ?

ಆದರೆ ಹೀಗೆ ಪತ್ತೆಹಚ್ಚಬಲ್ಲವಳಾದ ಸಾವಿತ್ರಿಯು ತನ್ನ ಗಂಡನ ಜೀವವನ್ನು ಅನುಸರಿಸಿ ಮತ್ತೆ ಪಡೆದಳು. `ಕರು ಹೋಯಿತು' ಎಂದರೆ ಕಾಣಬಹುದು. ಆದರೆ `ಜೀವ ಹೋಯಿತು' ಎಂದು ಹೇಗೆ ಹೇಳುವುದು? `ನೀರಿನಲ್ಲಿ ಹುಳುವಿದೆ' ಎನ್ನುವುದಾದರೆ, ಅದು `ಮೈಕ್ರೋ ಸ್ಕೋಪಿಕ್ ವರ್ಲ್ಡ್' ನ ವಿಷಯ. ಅಂತೆಯೇ ಜೀವದ ಗತಾಗತಿಯೆನ್ನುವುದು ತಪಸ್ಸೆಯಿಂದ ಅರಿಯಬೇಕಾದ ವಿಷಯ. `ಹೀಗೆ ಅರಿಯಬಲ್ಲವನೇ ಆಸ್ತಿಕ'. ನೀರಿನಲ್ಲಿ ಬ್ಯಾಕ್ಟೀರಿಯಾ ಇದೆ ಎಂದು ಸೈನ್ಟಿಸ್ಟ್ ಆದವನು ಅರಿತು ಹೇಳುತ್ತಾನೆ. ಅಂತೆಯೇ ಆತ್ಮವು ಇದೆ ಎನ್ನುವವರು ಆತ್ಮವನ್ನು ಕಂಡು ವ್ಯವಹಾರ ಮಾಡಿದ್ದುಂಟು. 

\section*{ಕಾಲಕಾಲಕ್ಕೆ ಪುನರ್ದರ್ಶನದಿಂದ ಮಹರ್ಷಿಗಳು ತಂದ ವ್ಯವಹಾರವನ್ನು ಬೆಳೆಸಬೇಕು}

ಅದನ್ನರಿತವರಾದ ಮಹರ್ಷಿಗಳು ತಮ್ಮ ಒಂದು ಪೀಳಿಗೆಯಲ್ಲಿ ಆ ವ್ಯವಹಾರವನ್ನು ತಂದರು. ಅವರ ಒಂಡು ಗ್ರಹಿಕೆಗೆ ಅನುಗುಣವಾಗಿ ವ್ಯವಹಾರ ಬೆಳೆಸಬೇಕು. ಅದನ್ನು ಲೋಕದಲ್ಲಿ ಮುಂದುವರಿಸಲು ಗುಡಿಗೋಪುರಗಳನ್ನು ಕಟ್ಟಿಸಿದರು ರಾಮಾನುಜರು. ಆ ಒಂದು ವ್ಯವಹಾರವನ್ನು ಬೆಳೆಸಲು ತಮ್ಮ ಶಿಷ್ಯರಲ್ಲಿ ಎಪ್ಪತ್ತನಾಲ್ಕು  ಜನರನ್ನು ಸಿಂಹಾಸನಾಧೀಶ್ವರರನ್ನಾಗಿ ಮಾಡಿದರು.

ಕಾಮಧೇನುವು ಹಾಲನ್ನು ಕೊಡುವುದೇನೋ ನಿಜ. ಆದರೆ ಎಂದೋ ಕರೆದ ಹಾಲು ಇಂದು ಬಳಕೆಗೆ ಯೋಗ್ಯವಲ್ಲ. `ಫ್ರೆಷ್ಯಾಗಿ' ಕರೆಯಬೇಕು. ಹಾಗೆ ಕರೆದರೆ ಸವಿ ತಿಳಿಯುತ್ತೆ. ಆಗಿಂದಾಗ್ಗೆ ಪುನರ್ದರ್ಶನಮಾಡಿ `ಅರಿಯತಕ್ಕ ವಿಷಯವೇನು?' ಎನ್ನುವುದನ್ನು ತೆಗೆದುಕೊಂಡು ಅದನ್ನು ತಂದವರೂ ಉಂಟು. ಹಾಗಿಲ್ಲದಿದ್ದರೆ ವಿಷಯ ಬೆಳೆಯುವುದಿಲ್ಲ. ಅಡಿಗೆ ಯಾರೋ ಮಾಡಿಟ್ಟರು ಎಂದು ಅದನ್ನೆ ಎಲ್ಲಾ ಕಾಲಕ್ಕೂ ಬಳಸಲಾಗುವುದಿಲ್ಲ. ಆದ್ದರಿಂದ ಅಡಿಗೆ ಮಾಡುವ ವಿಧಾನವರಿತು ಮಾಡುವವರು ಬೇಕು. ವಿಧಾನವರಿಯದಿದ್ದರೆ ಉಪಯೋಗವಿಲ್ಲ.

\section*{ಸಾಹಿತ್ಯವು ಸಜೀವವಾಗಿರಬೇಕು}

ಆಚಾರ್ಯರು ಬಂದರು, ತಮ್ಮ ತಂತ್ರವನ್ನೂ \footnote{ತಂತ್ರ-ಪರಮಸಿದ್ದಾಂತ -ಅನ್ವೇಷಣೆಯಿಂದ ಲಭಿಸಿದ ವಿಷಯ-ಸಿದ್ದಾಂತ} ಕೊಟ್ಟರು. ಆದರೆ ಇಂದು ಉಳಿದಿರುವ ಅವರ ಹರಕು ಮುರುಕು ಗ್ರಂಥದಿಂದ ನಮಗೇನೂ ಸಿಕ್ಕುವುದಿಲ್ಲ, ಪ್ರಸ್ಥಾನತ್ರಯಭಾಷ್ಯ `ಶ್ರೀ ಭಾಷ್ಯ' ಇವನ್ನೇ ನೆಚ್ಚಿದರೆ ಎಲ್ಲಾ ಆಗುವುದಿಲ್ಲ. ಲೈಫ್ ಬೇಕು. ಲೈಪ್ ಹೋದ ಮೇಲೆ, ಇದು ಉತ್ತಮ ಇದು ಅಧಮ ಎಂದಿರುವುದಿಲ್ಲ. ರಾಜನ ಶರೀರವೇ ಆದರೂ ಜೀವಹೋದ ಮೇಲೆ ಉತ್ತಮ ಎಂದುಕೊಂಡು ಹಾಗೆಯೇ ಇಡುವುದಿಲ್ಲ. ಜೀವಭೂತವಾದ ವಿಷಯವೇನು? ಎಂದು ಅರ್ಥಮಾಡಿಕೊಂಡು, ವಿಷಯವನ್ನು ಬೆಳೆಸಿಕೊಂಡು ಹೋಗಬೇಕು. ಒಂದು, ಎರಡು, ಮೂರು, ನಾಲ್ಕು, ಐದು, ಹತ್ತು ಈ ಸಂಖ್ಯೆಗಳಿಂದ (1,2,3,4,5,10) ಮನುಷ್ಯನ ತಲೆಯನ್ನು ಬರೆಯಬಹುದು. ಆದರೆ ಈ ಮಾತು ಮಾತ್ರ ಉಳಿದರೆ ವಿಷಯವು ಅರ್ಥವಾಗುವುದಿಲ್ಲ. ವಿಷಯವು ಅರ್ಥ ದೊಡನೆಯೇ ಮುಂದುವರಿಯಬೇಕು. ಅಲ್ಲಿ ಸಂಖ್ಯೆಯ ತನ್ನ ಸಂಖ್ಯಾತನವನ್ನು ಕಳೆದುಕೊಂಡು ತಲೆಯ ಅಂಗವಾಗಿ ಬಿಡುತ್ತದೆ. ಆದ್ದರಿಂದ ಆಶಯವೇನು? ಎಂಬುದನ್ನು ತೆಗೆದುಕೊಂಡು ಸಾಹಿತ್ಯವನ್ನು ಬೆಳೆಸಬೇಕು. ಹನ್ನೊಂದು (11) ಎಂಬ ಸಂಖ್ಯೆಯನ್ನು ಉಪಯೋಗಿಸಿಕೊಂಡು ಆಡನ್ನು ಮಾಡಾಬಹುದು. ಅದನ್ನೂ ಅರ್ಥದೊಡನೆ ತೆಗೆದುಕೊಂಡಾಗ ಚೆನಾಗಿರುತ್ತದೆ. ಇಲ್ಲದಿದ್ದರೆ ನಮಗೆ ಅರ್ಥವಾಗದಂತೆ ಆಗಿ ಬಿಡುತ್ತದೆ. ಅದರ ವಿನ್ಯಾಸದ ಅರಿವು ನಮಗಿರಬೇಕು.

\section*{ಸಮಯಕ್ಕೆ  ತಕ್ಕಂತೆ ವ್ಯವಹಾರವಲ್ಲ}

ಚಿದಚಿದ್ವಿಶಿಷ್ಟನಾದ ಪುರುಷ ಎಂದರೆ `ಹೂಂ ಸರಿ' ಎನ್ನುವುದು, ಪುರುಷನೊಬ್ಬನೇ ಎಂದರೆ `ಅದೂ ಸರಿ' ಎನ್ನುವುದು, ಹೀಗೆ `ಕಾರ್ಯವಾಸೀ ಕತ್ತೆ ಕಾಲ್ಕಟ್ಟು' ಎಂಬಂತಾಗಬಾರದು.

\section*{ಮೂಲ ಆಶಯ ಕಳಚಿದ ವ್ಯವಹಾರಗಳು}

ಗುಡಿಗೋಪುರಗಳನ್ನು ಕಟ್ಟಿದರೂ ವಿಷಯ ಮಾತ್ರ ಮುಂದುವರಿಯಲಿಲ್ಲ. ಪದಾರ್ಥದ ಮೇಲೆ ಪದದ ಬಳಕೆ ಬರಲಿಲ್ಲ. ಪ್ರಕೃತಿಯ ಮೇಲೆ ಭಗವಂತನೆಡೆಗೆ ಸೋಪಾನ ಕಟ್ಟಿದವರು ಭಗದ್ರಾಮಾನುಜರು. ಅದರ ಮರ್ಮವರಿಯಲಿಲ್ಲ ಮುಂದಿನವರು.

\section*{ವ್ಯವಹಾರವು ತನ್ನ ಮೂಲದೊಡನೆ ಬೆಳೆದರೆ ರಸ್ಯವಾಗುತ್ತದೆ}

ಹುಡುಗರು ಬಕರೆ ಆಡುವುದನ್ನು ನೋಡಿರಬಹುದು. ಹಂಚಿಬಕರೆಯನ್ನೇ ಮಗುವಾಗಿಟ್ಟು, ಶೀತವಾಯಿತು ಉಷ್ಣವಾಯಿತು ಎಂದುಬಿಡುತ್ತವೆ. ಇದು ಮಕ್ಕಳ ಆಟ. ಅಂತೆಯೇ ದೊಡ್ಡವರೂ ಸಹ ಮಕ್ಕಳು ಹುಟ್ಟಿದಾಗ ಮೊದಲು ಗುಂಡಪ್ಪನನ್ನು ತೊಟ್ಟಿಲಲ್ಲಿಟ್ಟು ತೂಗುವುದುಂಟು. ಕಲ್ಲಿಗೆ ಜೀವವಿಲ್ಲ ಶೀತವಿಲ್ಲ ಎನ್ನುವುದು ಗೊತ್ತು. ಆದರೂ ಮಗುವಿನ ಭಾವವನ್ನು ಅಲ್ಲಿಟ್ಟಾಗ ಆ ರೀತಿಯಾದ ವ್ಯವಹಾರ ಬೆಳೆಯುತ್ತದೆ. ಅದನ್ನು ವ್ಯಂಗ್ಯವಾಗಿ ತೆಗೆದುಕೊಳ್ಳಬಾರದು. ಮುಂದೆ ಬೆಳೆಯುವ ವ್ಯವಹಾರದ ಮೇಲೆ ಶಾಸ್ತ್ರ ಬೆಳೆಸಬೇಕು. ಆ ಮಕ್ಕಳು ಕಣ್ಣಲ್ಲಿ ಬಕರೆಯು ಸಜೀವವಾದ ಒಂದು ಶಿಶುವೇ ಆಗಿದೆ. ತನ್ನ ಹೃದಯದ ಶಿಶುಭಾವವನ್ನು ಅದರ ಮೇಲಿಟ್ಟಿದೆ. ಅಲ್ಲಿ ಆವಾಹನೆಮಾಡಿದೆ ಅದನ್ನು ಹೋರಕ್ಕೆಸೆದೆರೆ ಮಗು ಹೋಯಿತು ಎಂದು ಅಳುತ್ತದೆ. ಯಾರಾದರೂ ಮೇಲಿದ್ದ ನಮ್ಮ ಪಂಚೆ ಕಿತ್ತೆಸೆದರೆ ಕೋಪಬರುವುದಿಲ್ಲವೇ? ಏಕೆ? ನಿಮಗೂ ಅದಕ್ಕೂ ಸಂಬಂಧವೇನು? ಎಂದರೆ ನಿಮ್ಮ ಸಂಬಂಧವನ್ನು ಅದರ ಮೇಲಿಟ್ಟಿದ್ದೀರಿ. ಆದ್ದರಿಂದ ಅದು ಹೋದರೆ ತಕ್ಷಣವೇ ತವಕಪಡುತ್ತೀರಿ. ಅಂತೆಯೇ ಜೀವನದಲ್ಲಿ ಮಕ್ಕಳಾಡುವ ಆಟವನ್ನು ಒಂದು ಭಾವದೊಡನೆ ನೋಡಿದಾಗ ರಸ್ಯವಾಗಿಯೇ ಇರುತ್ತದೆ.

\section*{ಆಚಾರ್ಯರು ತಂದ ಆದರ್ಶಮರೆಯಾಗಿದೆ}

ಅಂತೆಯೇ ಆತ್ಮಜೀವನವನ್ನು ಕಲ್ಲಿನ ಮೇಲಿಟ್ಟು ಗುಡಿಗೋಪುರಗಳನ್ನು ಕಟ್ಟಿದರು. ಅದರ ಹಿನ್ನೆಲೆಯಲ್ಲಿರುವ ಸಹಜತೆಯ ಪರಿಚಯ ಬರಲಿ ಎಂಬುದಾಗಿ ಮಾಡಿದರು. ಅದನ್ನು ಅಲ್ಲಿ ತೆಗೆದುಕೊಳ್ಳಬೇಕು. ಭಗವದ್ರಾಮಾನುಜರ ಕೈ ಮುಗಿಯುವಿಕೆ ಹೇಗಿದೆ? ಎಂಬುದನ್ನು ಮನಸ್ಸಿಗೆ ತಂದುಕೊಳ್ಳಬೇಕು. ಅಂಶವನ್ನು ಪೂರ್ಣದಲ್ಲಿ ಸೇರಿಸಬೇಕು. ಅದಕ್ಕನುಗುಣವಾಗಿ ನಾವು ಬಾಳಾಬೇಕು. ಅದನ್ನು ಬಿಟ್ಟು ನಮ್ಮ ಜೀವನದ ಸಾಮಗ್ರಿಯನ್ನು ಅವರ ಮೇಲೆ ಹಾಕಿ ಕಾಯುತ್ತಾ ಇದ್ದೇವೆ. ಎಷ್ಟು ಚಿನಾಗಿದೆ ಕಲಾಪತ್ತಿನ ಪಟ್ಟು! ಎಂದು ಪಟ್ಟಿಗಾಗಿ(ಮಗುಟ) ಅಲ್ಲಿ ನೋಡುತ್ತೇವೆ. 

\section*{ಮುಖ್ಯವಾದದ್ದನ್ನು ಮರೆತು ನಡೆದುಬರುತ್ತಿರುವ ಜೀವನನಿಧಾನ}

ಒಂದು ಕಥೆ ಈ ಸಂದರ್ಭದಲ್ಲಿ ನೆನಪಿಗೆ ಬರುತ್ತೆ. ಹಿಂಡೆ ಒಬ್ಬ ಹರಿಕಥೆ ದಾಸರು ಬಂದಿದ್ದರು. ಜನರೆಲ್ಲಾ ಹರಿಕಥೆಗಾಗಿ ಬಂದು ಸೇರಿದರು. ಜನರಿಗೆ ಹರಿಕಥೆ ಅಷ್ಟು  ಸ್ವಾರಸ್ಯವಾಗಿರಲಿಲ್ಲ. ಕಥೆಯ ಪ್ರವಾಹ ಹರಿಯುತ್ತಾ ಹರಿಯುತ್ತಾ ಜನಸ್ತೋಮವು ಹರಿದು ಹೋಯಿತು. ಕೊನೆಗೆ ಹರಿಕಥೆ ಮುಗಿಯುವ ವೇಳೆಗೆ ಒಬ್ಬ ಮಾತ್ರ ಉಳಿದಿದ್ದ. ಹರಿಕಥೆ ದಾಸರು ಅವನನ್ನು ನೋಡಿ, `ಸ್ವಾಮಿ, ನೀವೊಬ್ಬರೇ ಹರಿಕಥೆಯ ರಸಾಸ್ವಾದನೆ ಮಾಡಿದವರು ಎಂದು ತಮ್ಮ ಕೃತಜ್ಞತೆಯನ್ನು ಅರ್ಪಿಸಿದರು. ಅದಕ್ಕೆ ಅವನು, `ಇಲ್ಲಾ ಸ್ವಾಮಿ, ಹರಿಕಥೆಯ ರಸಾಸ್ವಾದನೆಗೆ ನಾನಿಲ್ಲಿ ಕುಳಿತಿಲ್ಲ. ತಮ್ಮ ಹರಿಕಥೆಗಾಗಿ ಹಾಗಿರುವ ಜಮಖಾನ ನನ್ನದು. ಆ ಜಮಖಾನಕ್ಕಾಗಿ ಕಥೆ ಮುಗಿಯುವವರೆಗೂ ಕಾಯುತ್ತಾ ಕುಳಿತಿದ್ದೇನೆಯೇ ಹೊರತು ಬೇರೇನೂ ಅಲ್ಲ' ಎಂದ.

ಹೀಗೆ ನಮ್ಮ ಬದುಕನ್ನು ಗುಡಿಗೋಪುರಗಳಿಗೆಲ್ಲಾ ಹೇರಿ ಅದನ್ನು ಕಾಯುತ್ತಿದ್ದೇವೆಯೇ ಹೊರತು ಅವುಗಳಿಂದ ನಮ್ಮ ಬದುಕಿಗೆ ಬರುವ ವಿಷಯವನ್ನು ನೋಡಿಯೇ ಇಲ್ಲ. ಗುಡಿಗೋಪುರಗಳಲ್ಲಿ ಉತ್ಸ್ವ ನಡೆದಾಗ `ಪುಳಿಯೋಗರೆ' ಸರಿಯಾಗಿದ್ದರೆ ಉತ್ಸಾಹ ಜಾಸ್ತಿ. ಅದಿಲ್ಲಾಂದರೆ ರಾಮಾನುಜರ ಪಾಡಿಗೆ ರಾಮಾನುಜರು, ನಮ್ಮ ಪಾಡಿಗೆ ನಾವು, ಹೀಗಾಗಿ ಬಿಡುತ್ತೆ. ಆ ರೀತಿ ಆಗಬಾರದು. ವ್ಯಕ್ತಿಗತವಾದದೂಷಣೆಗಲ್ಲ, ಅವರಿಂದ ನಮ್ಮ ಜೀವನಕ್ಕೆ ಬರುವುದೇನು? ಎನ್ನುವುದನ್ನು ನೋಡಬೇಕು.

\section*{ಗುಡಿಗೋಪುರಗಳು ಪ್ರತಿನಿಧಿಸುವ ವಿಷಯವೇನು?}

ಒಬ್ಬ ಮನುಷ್ಯನು ಒಂದು ನಾಟಕದಲ್ಲಿ ಪಾರ್ಟ್ ವಹಿಸುವಂತೆ, ರಾಮಾನುಜರ ಬಗ್ಗೆ ಮಾತನಾಡುವ ಪಾರ್ತ್ ವಹಿಸಿ ಮಾತನಾಡಿದ್ದಾಯಿತು. ಅಂತೆಯೇ ಜೀವನಕ್ಕೆ ಸಂಬಂಧಪಟ್ಟ ಒಂದು ನಾಟಕದಲ್ಲಿ ಗುಡಿಗೋಪುರಗಳಲ್ಲಿ ಪ್ರತಿನಿಧಿಸುವ ವಿಷಯವಾದರೂ ಏನು? ಎನ್ನುವುದನ್ನು ನೋಡಬೇಕು. ಇದೆಲ್ಲಾ ಒಂದು ಭಕ್ತಿಭಾವವನ್ನು  ಕೊಡುವುದಾಗಿರಬೇಕು. ಭಕ್ತಿ ಎಂದರೇನು? `ಸಾ ಪರಾನುರಕ್ತಿರೀಶ್ವರೇ' ಈಶ್ವರನಲ್ಲಿ ಶ್ರೇಷ್ಠವಾದ ಪ್ರೀತಿ, ಸೇರುವೆ, ಪರಭಕ್ತಿ, ಪರಜ್ಞಾನ, ಪರವೈರಾಗ್ಯಗಳ ವಿಷಯವನ್ನು ಅರ್ಥಮಾಡಿಕೊಡಬೇಕು.

\section*{ವಿಷಯ ತಂದದರ ಅಶಯವರಿತು ಮುಂದುವರಿಸಬೇಕು}

`ಜಗತ್ತೇ ಮಿಥ್ಯಾ' ಎಂದು ಹೇಳಿದರೆ, ಜಗತ್ತಿನಲ್ಲಿ ವಿಷಯವಿರುವಂತೆಯೇ ಇಲ್ಲ ಆದ್ದರಿಂದ ಶಂಕರರ ಮಾತನ್ನು ಅವರ ಒರಿಜಿನಲ್ ಸ್ಪರಿಟ್ಟಿಗೆ ಅನುಗುಣವಾಗಿ ತೆಗೆದುಕೊಳ್ಳಬೇಕು. ಅವರ ಕಣ್ಣಿಗೆ ಹಾಗೆ ಕಾಣುತ್ತೆ? ಜನ, ಧನ ಎಲ್ಲಾ ಮಿಥ್ಯೆ ಯೇ ಆದ ಮೇಲೆ `ಮಾಕುರು ಧನಜನ್ಯೌವನಗರ್ವಂ' ಎನ್ನುವುದಕ್ಕೆ ವಿಷಯವೇನು? ಹಾಗಾದರೆ `ಜಗನ್ಮಿಥ್ಯೆ' ಎನ್ನುವುದನ್ನು ಯಾವ ಅರ್ಥದಲ್ಲಿ ತೆಗೆದುಕೊಳ್ಳಬೇಕು? ಎಂದರೆ ಬ್ರಹ್ಮ ಒಂದನ್ನೇ ನೋಡಿ ಅವರಂತೆಯೇ ನೋಡಿ ಹೇಳುವ ಪಕ್ಷದಲ್ಲಿ ಎಲ್ಲರೂ ಹಾಗೆಯೇ ಹೇಳಬೇಕಾಗುವುದು. ಆ ಸ್ಪಿರಿಟ್ಟನ್ನು ಬಿಟ್ಟರೆ ಪ್ರಯೋಜನವಿಲ್ಲ.

ಒಬ್ಬ ಎಣ್ಣೆಯವನು ಹೋಗುತ್ತಿದ್ದ. ನೆತ್ತಿಯಮೇಲೆ ಎಣ್ಣೆಯ ಗಡಿಗೆಯಿತ್ತು. `ಅಯ್ಯಾ, ಇದನ್ನು ಎತ್ತು ಬಿಡು' ಎಂದ ಒಬ್ಬ ಅದೇ ರೀತಿ ಎತ್ತಿ ಬಿಟ್ಟುಬಿಟ್ಟ. ಗಡಿಗೆ ಒಡೆದು ಎಣ್ಣೆಯೆಲಾ ಚೆಲ್ಲಿತು. `ಏನಯ್ಯಾ ಹೀಗೆ ಮಾಡಿದೆ?' ಎಂದು ಕೇಳಿದ್ದಕ್ಕೆ ' `ನೀನು ಹೇಳಿದಂತೆ ಮಾಡಿದೆ" ಎಂದ. ಹಾಗೆಯೇ ಹಿಂದೆಯೇ ಎತ್ತನ್ನು ಅಟ್ಟಿಸಿಕೊಂಡು ಬರುತ್ತಿದ್ದವನು ತನ್ನ ಎತ್ತನ್ನು ಬಿಟ್ಟು ಬಿಟ್ಟ. ಹೀಗೆ ಬರೀ ಮಾತನ್ನು ತೆಗೆದುಕೊಂಡರೆ ಅನರ್ಥಕ್ಕೆ ಎಡೆಯಾಗುತ್ತದೆ. ಪದ ಹಿಡಿದು ಜುಂಗಾಡಿದರೆ ಹೀಗೊ ಆಗಬಹುದು ಹಾಗೂ ಆಗಬಹುದು ಎಂದು ವಿಕಲ್ಪಗಳಿಗೆ ಶುರುವಾಗುತ್ತೆ. ಒಮ್ದು ನಿರ್ಣಯವಿಲ್ಲ. ಅದರ ಅಂತರಂಗವರಿತು ಜೀವನದ ನೆಲೆಯನು ಕಾಣಬೇಕು. ಭಕ್ತಿಯಿಂದ ಒಲಿಸಿಕೊಳ್ಳಬೇಕು ವಿವೇಕದಿಂದೊಡಗೂಡಿದ ವಿಚಾರದಿಂದ ಅರ್ಥಮಾಡಿಕೊಳ್ಳಬೇಕು. ಆ ಸ್ಪಿರಿಟ್ಟನ್ನು ಅರ್ಥಮಾಡಿಕೊಳ್ಳಬೇಕು.

\section*{ತತ್ತ್ವನಿರೂಪಣೆಯಲ್ಲಿ ವೈವಿಧ್ಯವೆಕೆ?}

ಕಾಲಭೇದೇನ ವಸ್ತುವು ಬೇರೆ ಬೇರೆಯಾಗಿ ಕಾಣುತ್ತದೆ. ಆಕಾಶವನ್ನು ನೋಡಿದಾಗ ನಕ್ಷತ್ರದಲ್ಲಿ ಕಿರಣವಿರುವಂತೆ ಕಾಣುತ್ತದೆ. ಆದರೆ ನಿಜವಾಗಿ ಕಿರಣವಿಲ್ಲ. ಕಿರಣ ಕಾಣುವುದು ತಮ್ಮ ಕಣ್ಣಿನ ರೆಪ್ಪೆಯ ಕೂದಲಿನಿಂದ. ಈ ಉಪಾಧಿಯಿಂದ ಹಾಗೆ ಕಾಣುತ್ತದೆ. ತಾತ್ತ್ವಿಕವಾದ ಸರಣಿಯಲ್ಲಿ ಹೊರಟವರು ಅವರ ಸರಣಿಯಲ್ಲಿ ವಿಷಯವನ್ನಿಟ್ಟರೆ ಆ ದೃಷ್ಟಿಕೋಣದಿಂದ ನೋಡಿದಾಗ ಅದೇ ವಸ್ತು ಕಾಣುತ್ತದೆ. ಕಾಲ ಭೇದೇನ ಒಂದೇ ವಿಷಯವನ್ನು ಬೇರೆ ರೀತಿ ಇಡಬೇಕಾಗುತ್ತೆ. ಹಾಗೆಂದ ಮಾತ್ರಕ್ಕೆ ತತ್ತ್ವವೇನೂ ಬದಲಾಯಿಸಿ ಬಿಡುವುದಿಲ್ಲ.

\section*{ಅದ್ವೈತ ವಿಶಿಷ್ಟಾದ್ವೈತಗಳೆರಡೂ ನೆಲೆ ತಲುಪಿಸುವ ವಿಷಯಗಳೇ}

ಒಂದು ವೇಳೆ ರಾಮಾನುಜರು ಶಂಕರರು ಇಬ್ಬರೂ ಒಟ್ಟಿಗೇ ಇದ್ದಿದ್ದರೆ ಒಂದು ಒಪ್ಪಂದಕ್ಕೆ ಬರುತ್ತಿದ್ದರು. `ನೀ ಏಕೆ ಹೀಗೆ ಮಾಡಿದೆ? ನಾನು ಇದಕ್ಕಾಗಿ ಹೀಗೆ ಮಾಡಿದೆ', `ನೀ ಏಕೆ ಹಾಗೆ ಮಾಡಿದೆ? ನಾನೂ ಅದಕ್ಕಾಗಿಯೇ' ಎಂದು ಹೇಳಿ ಇಬ್ಬರೂ ಆಪ್ತಸ್ನೇಹಿತರಾಗಿಬಿಡುತ್ತಿದ್ದರು. ಆದರೆ ಈಗ ಶಿಷ್ಯ ಕೈಗೆ ಸಿಕ್ಕಿ ನರಳಾಡುತ್ತಿದ್ದರೆ. ಯಾವ ಸ್ಪಿರಿಟ್ಟಿನಲ್ಲಿ ವಿಷಯ ಬಂತೋ ಅದೇ ಸ್ಪಿರಿಟ್ಟಿನಲ್ಲಿ ವಿಷಯವನ್ನು ಬೆಳೆಸಬೇಕು. ಕೇವಲ ಅದ್ವೈತವನ್ನೇ ಸರಿಯಾಗಿ ಹಿಡಿದರೂ ಬದುಕಬಹುದು. ಕೇವಲ ವಿಶಿಷ್ಟಾದ್ವೈತವನ್ನೇ ಅವಲಂಬಿಸಿದರೂ ಬದುಕಬಹುದು. ಎರಡನ್ನೂ ಬಿಟ್ಟು ವಾದ ಹಿಡಿದರೆ ವಿಷಯವಿಲ್ಲ.

\section*{ದೇಶದಲ್ಲಿ ಶಂಕರ-ರಾಮಾನುಜರು ತಂದ ವಿಷಯದ ಪ್ರಸ್ತುತಸ್ಥಿತಿ}

ದೇಶದಲ್ಲಿ ಅದ್ವೈತಿ ಮತ್ತು ವಿಶಿಷ್ಟಾದ್ದ್ವೈತಿಯೆಂಬ ವ್ಯವಹಾರಗಳು ಆಯಾ ಮತಾಭಿಮಾನವನ್ನವಲಂಬಿಸಿರುವ ಹಾಗು ಆಯಾ ವಿಷಯದಲ್ಲಿ ವಾಗ್ ವೈಖರಿಯುಳ್ಳ ಜನರಿಗೆ ಸಲ್ಲುತ್ತಿದೆಯೇ ಹೊರತು ಅಂತಹ ತತ್ತ್ವದರ್ಶನದ ಮೇಲೆ ಉಳಿದಿಲ್ಲ. ಸಂಪ್ರದಾಯವೆಂಬುದಕ್ಕೆ ಈ ಹಿಂದೆ ವಿವರಣೆಕೊಟ್ಟಂತೆ ಮೂಲಸ್ವರೂಪಕ್ಕೆ ಸ್ವಲ್ಪವೂ ಚ್ಯುತಿಬಾರದಹಾಗೆ ಒಬ್ಬರಿಂದ ಮತ್ತೊಬ್ಬರಿಗೆ ವಿಷಯ ಹರಿದುಬರುತ್ತಿರುವಂತೆ ಕಾಣುತ್ತಿಲ್ಲ. ಆಯಾ ಸಾಂಪ್ರದಾಯಿಕ ಕ್ಷೇತ್ರಗಳಲ್ಲಿ ಪ್ರವಾಸಕೈಗೊಂಡಿದ್ದಾಗ ಅವರವರು ಕೊಟ್ಟ ವಿವರಣೆಗಳಿಂದ ಅದು ಸ್ಪಷ್ಟವಾಗುತ್ತೆ. ಈ ಮಾತನ್ನು ನೋವಿನಿಂದ ಹೇಳಬೇಕಾಗಿದೆ.

\section*{ಇಂದೂ ತುರೀಯಾವಸ್ಥೆಯ ಅನುಭವ ಅಸಾಧ್ಯವೇನಲ್ಲ}

ಆಧ್ಯಾತ್ಮಿಕವಾದ ದೈವಿಕವಾದ ಜೀವನ ಅಂದು ಮಾತ್ರ, ಇಂದು ಏನೂ ಇಲ್ಲ ಎಂದಲ್ಲ. ಅದಕ್ಕಾಗಿ ದುಡಿದರೆ ಇಂದೂ ಉಂಟು. ಋಷಿಗಳ ಕಾಲದಲ್ಲೇ ತುರೀಯಾವಸ್ಥೆ ಹೊರಟುಹೋಯಿತು ಎಂದೇನೂ ಇಲ್ಲ. ಕಳೆದು ಹೋಗಿರುವುದೇನು? ಎಂದು ಶೋಧಿಸಿ ಪಡೆಯಲು ದುಡಿಯಬೇಕು.

\section*{ಶಾರದಾಪೀಠದ ಪರಂಪರೆಯ ಹಿನ್ನೆಲೆ}

ಶಂಕರರು ಮತ್ತು ರಾಮಾನುಜರು ತಮ್ಮ ಒಂದು ಸಂಪ್ರದಾಯದಲ್ಲಿ ದ್ಯೈಪಾಯನ-ನಾಥಮುನಿಗಳ ಅಭಿಪ್ರಾಯವನ್ನು ತೆಗೆದುಕೊಂಡು ಬಂದು ತಮ್ಮ ಒಂದು ಅಭಿಪ್ರಾಯವನ್ನು ಬೆಳಸುವುದಕ್ಕೆ, ಪೀಠವನ್ನು ಸ್ಥಾಪಿಸಿದರು. ಶೃಂಗೇರಿಯಲ್ಲಾದರೋ ವ್ಯಾಖ್ಯಾನಸಿಂಹಾಸನ, ಬೆಳ್ಳಿಯ ಪೀಠ, ಅದರಲ್ಲಿ ಶಾರದಮ್ಮನವರನ್ನು ಕೂರಿಸಿದ್ದಾರೆ, ಪಟ್ಟಾಭಿಷೇಕ ಮಾಡುವಾಗ ಶಾಅದೆಯಭಾಗದಲ್ಲಿ ಸ್ವಮಿಗಳನ್ನು ಕೂರಿಸಿ ಪಟ್ಟಾಭಿಷೇಕ ಮಾಡುತ್ತಾರೆ. ಅದೇತಕ್ಕೆ? ಅದರ ಹಿನ್ನೆಲೆಯಲ್ಲಿ ಒಂದು ಗಂಭೀರ ಮನೋಧರ್ಮವಿದೆ. ಲೋಕಕ್ಕೆ  ಜ್ಞಾನಧಾರೆಯನ್ನೆರೆಯುವ ತಾಯಿಯಾದ ಶಾರದೆಯು ತನ್ನ ಆಶಯವನ್ನು ಬಿಂಬಿಸಬಲ್ಲ ಕಂದನಿಗೆ ತನ್ನ ಜಾಗವನ್ನೇ ಬಿಟ್ಟುಕೊಡುವ ಗಂಭೀರ ಹಿನ್ನೆಲೆಯಿಂದ ಬೆಳೆದು ಬಂದ ಪರಂಪರೆಯಾಗಿದೆ. ತಾಯಿಯ ಹೃದಯವಾದರೂ ಎಂತಹ ಕೋಮಲವಾದುದು! ತನ್ನ ಆಶಯವನ್ನು ಮುಂದುವರಿಸುವ ತನ್ನ ಕಂದನಿಗೆ ತನ್ನ ಪೀಠವನ್ನೇ ಬಿಡುವ ತಾಯಿಯ ಔದಾರ್ಯವಾದರೂ ಏನು? ಲೋಕಮಾತೆಯ ಆಶಯವನ್ನರಿತು ಮುಂದುವರಿಸುವ ಜವಾಬ್ದಾರಿ ಸಾಮಾನ್ಯವೇ?

ಗೀತೆಯಲ್ಲಿ ` ವಿಷಾದಯೋಗದ `ನಂತರ' ಪ್ರಸಾದಯೋಗ,' ಆದರೆ ಪ್ರಸಾದವಿದ್ದು ನಂತರ ವಿಷಾದ ವಾದರೆ ಕಷ್ಟ. ತುಂಗಾನದೀತೀರ, ತಂಪಾದ ಹಿತಕರವಾದ ವಾತಾವರಣ, ಎಲ್ಲದರ ಮೇಲೆಯೂ ತಮ್ಮ ಒಂದು ಭಾವವನ್ನಿಟ್ಟು ತಂದಿದ್ದಾರೆ. ಮನಸ್ಸಿನಲ್ಲಿ ಹರ್ಷವಾದರೆ ಅದನ್ನು ಚಪ್ಪಾಳೆಯಿಂದ ಸೂಚಿಸುತ್ತಾನೆ. ಅವರು ಶಿಲೆಯ ಮೇಲೆ ಎಂತಹ ಮನಸ್ಸನ್ನಿಟ್ಟಿದ್ದಾರೆ! ಆಸ್ತಿಭಾಅ ಹೇಗೆ ಹಾಕಿದ್ದಾರೆ! ಎಷ್ಟು ಕಷ್ಟಪಟ್ಟರಪ್ಪಾ ! ಎನ್ನಿಸುತ್ತೆ. ಮುಂದಿನ ಮಕ್ಕಳ ಆಸ್ತಿಯಾಗಿಟ್ಟಿದ್ದಾರೆ. ಅದನ್ನರಿಯದೆ ಜನ ಕಂಗಾಲಾಗಿದ್ದಾರೆ. ಆದರ್ಶ ಸರಿಯಾಗಿದ್ದರೆ ಮುಂದೆಲ್ಲಾ ಸರಿಮಾಡಬಹುದು. ಮಾಡಲ್ ಸರಿಯಾಗಿದ್ದರೆ ವೈಕುಂಠವಾಗುತ್ತೆ ಇಲ್ಲದಿದ್ದರೆ ಕೈಕುಂಠವಾಗುತ್ತೆ. ಅವರು ಒಂದು ಭಿತ್ತಿಕಲ್ಪನೆಮಾಡಿಕೊಟ್ಟಿದ್ದಾರೆ. ಒಂದು ಮಾಡಲ್ ಹಾಕಿಕೊಟ್ಟಿದ್ದಾರೆ. ಜ್ಞಾನಯೋಗ, ಕರ್ಮಯೋಗ ಇವೆಲ್ಲಾ ಆತ್ಮಯೋಗದಲ್ಲಿ ನಿಲ್ಲಬೇಕು. ಶಂಕರರ ಒಳ್ಳೆಯ ಮಾಡಲ್ಲನ್ನು ಅನುಸರಿಸಿದರೆ ಅವರ ಕಿಂಕರರೂ ಬಿಡುಗಡೆ ಹೊಂದಬಹುದು.

\section*{ಒಂದೇ ವಿಷಯವನ್ನೇ ಸನ್ನಿವೇಶವರಿತು ಬೇರೆಬೇರೆ ಉಪಾಯ ಬಳಸಿ ಇಟ್ಟಿರುವುದಾಗಿದೆ.}

ಒಂದು ವಸ್ತುವಿನ ನೋಟವೇ ಬಂದಾಗ ಒಬ್ಬೊಬ್ಬರು ಒಂದೊಂದು ತರಹ ಹೇಳುವುದಿಲ್ಲ. ಆದರೆ ಆಯಾ ಸನ್ನಿವೇಶಕ್ಕನುಗುಣವಾದ ಆದರ್ಶವನ್ನು ಕೊಡಲು ಉಪಾಯಮಾಡಬಹುದು. ಹಾಗೆಮಾಡಿದ ಮಹಾಪುರುಷರಲ್ಲಿ ಶಂಕರರೂ ರಾಮಾನುಜರೂ ಗಣ್ಯರಾಗಿದ್ದಾರೆ.



\begin{shloka}

\end{shloka}
\begin{shloka}

\end{shloka}
\begin{shloka}

\end{shloka}
\begin{shloka}

\end{shloka}
\begin{shloka}

\end{shloka}


\section*{}
\section*{}
\section*{}
\section*{}
\section*{}
\section*{}
