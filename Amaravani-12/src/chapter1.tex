\chapter{ಶಿಕ್ಷಾಶಾಸ್ತ್ರ}

(ಶಿಕ್ಷಾಶಾಸ್ತ್ರದ ಬಗೆಗೆ ತನ್ನ ಶಿಷ್ಯರೊಬ್ಬರಾದ ಶ್ರೀಯುತ ಎ.ಎಸ್. ವೆಂಕಟನಾಥನ್ ಪ್ರಶ್ನೆಗೆ ಉತ್ತರವಾಗಿ  ೨೫-೯-೧೯೬೩ರಂದು ಬಂದ ಪ್ರಸಂಗವಿದು. ಇದನ್ನು  ಸಂಗ್ರಹಿಸಿ ಬರಹರೂಪಕ್ಕೆ  ತಂದವರು ಶ್ರೀ ಛಾಯಾಪತಿಗಳು.)

\section*{ವಿಷಯದ ಮಂಡನೆಗೆ ಪೀಠಿಕೆ}

ಹಿಂದೊಮ್ಮೆ  ನನ್ನನ್ನು  ಶಿಕ್ಷಾಶಾಸ್ತ್ರದ ಬಗೆಗೆ ಪ್ರಶ್ನೆಯನ್ಯ ನೀವು ಕೇಳಿದುದುಂಟು. ಪ್ರಶ್ನೆ ಮಾಡುತ್ತಿರುವ ನೀವೇ ವೇದವಿದ್ವಾಂಸರು. ಇದರ ಬಗೆಗೆ ವ್ಯಾಸಂಗ ಮಾಡುತ್ತಿರುವವರು. ಸ್ವಲ್ಪವೂ ಶಿಕ್ಷಾಶಾಸ್ತ್ರದ ಬಗೆಗೆ ಹೊರಗೆಲ್ಲೂ   ಶಿಕ್ಷ ಪಡೆಯದಿರುವ ನನ್ನಲ್ಲಿ  ವಿಷಯವನ್ನು  ತೆಗೆದುಕೊಳ್ಳುವ ಔಚಿತ್ಯವೇನು? ಎಂಬ ಮಾತನ್ನು  ಆ ಸಂದರ್ಭದಲ್ಲಿ  ನಾನು ಆಡಿದ ನೆನಪಿರಬಹುದು. ` ಈ ವಿಷಯ ಇಲ್ಲಿ ಬಿಟ್ಟರೆ ಬೇರೆಲ್ಲಿಯೂಸಿಗುವುದಿಲ್ಲವಾದ್ದರಿಂದ ಕೇಳಿದೆ'  ಎಂಬ ಉತ್ತರ ನಿಮ್ಮದಾಗಿತ್ತು. ಈಗ ಮತ್ತೆ ಆಪ್ರಸಂಗ ಬಂದಿರುವುದರಿಂದ   `ಈ ಬಗ್ಗೆ ನೀವು ನಡೆಸಿರುವ ವ್ಯಾಸಂಗವೇನು' ಎಂದು ತಿಳಿಯಲಿಚ್ಛಿಸುತ್ತೇನೆ.  ಇತರ ಮೂಲಕ ನಿಮ್ಮ ವ್ಯಾಸಂಗದ ಪರಿಚಯ ನನಗೆ ಬಂದಿರಬಹುದಾದರೂ ಅದರ ಬಗೆಗೆ ನಿಮ್ಮಿಂದಲೇ ಕೇಳಿ ತಿಳಿಯಬಯಸುತ್ತೇನೆ. 

(ಸಲಕ್ಷಣಘನಾಂತ, ಮೀಮಾಂಸಾ, ತರ್ಕಶಾಸ್ತ್ರಗಳಲ್ಲಿ ನನ್ನ ವ್ಯಾಸಂಗವಾಗಿದೆಯೆಂದು ವಿದ್ವಾಂಸರು - ಎ.ಎಸ್. ವೆಂಕಟನಾಥನ್ ತಿಳಿಸಿದರು.)

ಇಷ್ಟೊಂದು ಭಾಗ ವ್ಯಾಸಂಗವಿರುವಾಗ ನಾನು, ನೀವು ಓದಿದ ಶಾಸ್ತ್ರದ ಬಗೆಗೆ ಹೇಳಬೇಕಾಗಿಲ್ಲ. ಆದರೆ ವೇದಾಂಗವಾದ ಶಿಕ್ಷಾಶಾಸ್ತ್ರದ ಬಗೆಗೆ ಕೇಳಿದ್ದರಿಂದ ಆ ಬಗೆಗೆ ನನ್ನ ಬುದ್ಧಿಗೆ ಏನು ಬಂದಿದೆಯೋ, ಪ್ರಾಮಾಣಿಕವಾಗಿ ಏನನ್ನು  ತೆಗೆದುಕೊಂಡಿದ್ದೇನೆಯೋ ಅಷ್ಟನ್ನು ಇಡುತ್ತೇನಪ್ಪಾ. ಅದನ್ನು ಕಲಿಯಬೇಕು ಎನ್ನುವ ಆಶಯವುಳ್ಳವರೇ ಅದಕ್ಕೆ ಪಾತ್ರರು. ವಿಷಯವೂ ಇರಬಹುದು, ಪಾತ್ರವೂ  ಇರಬಹುದು, ಅದನ್ನಿಡುವ ರಸವೂ ಇರಬಹುದು. ಆದರೂ ನಮ್ಮ ನಿಮ್ಮ ಸನ್ನಿವೇಶ ಎರಡೂ ಕೂಡಿ ಬಂದಾಗ ವಿರಾಮವನ್ನು ನೋಡಿಕೊಂಡು ಮುಂದುವರಿಸೋಣಪ್ಪ.

\section*{ವಿಷಯವೇ ತನ್ನ ಸ್ವರೂಪವನ್ನು  ಸಾರುತ್ತದೆ}

ವಯಃಕಾಲದಲ್ಲಿ  ಬಹಳವಾಗಿ ಶ್ರಮಿಸಿ ಜೀವಿತದ ಬಹು ಕಾಲವನ್ನು  ಶಿಕ್ಷಾವ್ಯಾಸಂಗಕ್ಕಾಗಿ ಮುಡಿಪಿಟ್ಟು  ದುಡಿದವರ ಮುಂದೆ ಅದರ ಬಗೆಗೆ ಗ್ರಂಥವ್ಯಾಸಂಗವನ್ನೇ ಮಾಡವನು ವಿಷಯವಿಡುತ್ತಾನೆಂಬುದು ವ್ಯಂಗ್ಯಕ್ಕೆ  ಪಾತ್ರವಾಗಬಹುದಪ್ಪಾ. ಆದರೆ ವಿಷಯ ಯಾವ ವ್ಯಕ್ತಿಯ ಮೇಲೂ ನಿಲ್ಲದೆ ತನ್ನ ಬಲದ ಮೇಲೆಯೇನಿಂತಿದೆ. ನನ್ನದೂ ಏನಿಲ್ಲ. ಅವರದೂ ಏನಿಲ್ಲ. ವಿಷಯವೇನು? ಪ್ರಯೋಗವೇನು? ಎಂದು ನೋಡಿದಾಗ ವಿಷಯವೇನುಂಟೋ ಅದು ತನ್ನನ್ನು ತಾನೇ ಬಿಚ್ಚಿಕೊಳ್ಳುವ ಪಕ್ಷೇ ನನ್ನ ಪಾತ್ರವಾಗಲೀ ಅವರ ಪಾತ್ರವಾಗಲೀ ಏನೂ ಇಲ್ಲ. ವಿಷಯದ ಕಡೆಗೆ ನೋಡುತ್ತಿರುವದಷ್ಟೇ ಕೆಲಸ.

\section*{ವಿಷಯವನ್ನಿಡುಲು ಇರಬೇಕಾದ ಜವಾಬ್ದಾರಿ}

ಯಾವ  ಒಂದು ಶಿಕ್ಷಾಶಾಸ್ತ್ರವು ನಾದ - ಸ್ವರ - ಅಕ್ಷರರೂಪವಾಗಿ ಬಂದ ವೇದ ವಾಕ್ಯಗಳ ಬಗೆಗೆ ವೇದಾಂಗವಾಗಿದ್ದುಕೊಂಡು ಶಿಕ್ಷೆ  ಎನ್ನುವ ಹೆಸರನ್ನು  ಪಡೆದು, ತನ್ನದೇ ಆದ ಸ್ವರೂಪದಿಂದ ತನ್ನನ್ನು ವ್ಯಕ್ತಪಡಿಸಿಕೊಳ್ಳುತಾ, ತನ್ನದೇ ಆದ ವಿಷಯವಿಟ್ಟುಕೊಂಡು ಬಂದಿದೆಯೋ ಅಂತಹ ಶಿಕ್ಷಾಶಾಸ್ತ್ರದ ಅಧ್ಯಯನಮಾಡುವವರು ಯಾವ ರೀತಿ ತರಬೇತಿ ಪಡೆಯಬೇಕು?  ಎನ್ನುವ ವಿಷಯವನ್ನು ಮುಂದಿಡುವವನಾಗಿದ್ದೇನೆ. ಹಾಗೆ ಇಡುವ ಮೊದಲು ಇದರ ಬಗೆಗೆ ಅವರೇನು (ಪ್ರಶ್ನಿಸಿದವರು) ತೆಗೆದುಕೊಂಡಿದ್ದಾರೆ ಎಂಬುದನ್ನು  ಮನಸ್ಸಿಗೆ ತೆಗೆದುಕೊಂಡು ನಂತರ ವಿಷಯವನ್ನು ಇಡುವವನಾಗಿದ್ದೇನೆ. ಲೋಕದಲ್ಲಿ  ಸಾಮಾನ್ಯವಾಗಿ ಇಬ್ಬರು ಸಂಗೀತಗಾರರು ಸೇರಿದರೆ ಛಟ್ಟನೆ ಮೊದಲು ವಿಷಯವನ್ನಿಡುವುದಿಲ್ಲ. ಇನ್ನೊಬ್ಬರಿಗೆ ಮೊದಲು ಅವಕಾಶ ಕೊಟ್ಟು ಅವರು ತಮ್ಮ ಸರಕನ್ನಿಟ್ಟ ಬಳಿಕ, ಅವರಿಗಿಂತ ತಮ್ಮ ವೈಶಿಷ್ಟ್ಯವನ್ನು  ತೋರಿಸಲು ಮತ್ತೊಬ್ಬರು ಪ್ರಯತ್ನಿಸುವುದುಂಟು. ಆದರೆ ನಾನು ಅವರನ್ನು  ಕೇಳಿದುದು ಆ ಆಶಯದಿಂದಲ್ಲಪ್ಪ. ಅವರು ತೆಗೆದುಕೊಂಡಿರುವುದೇನು? ಎಂಬುದು ತಿಳಿದರೆ ಅವರ ಮುಂದೆ ಇಡಬೇಕಾದುದೆನು ಎಂಬುದನ್ನು ನಿರ್ಧರಿಸಿ ಇಡಲು ಸುಲಭವಾಗುತ್ತದೆ. ಅದಕ್ಕಾಗಿ ಕೇಳಿದೆನೇ ಹೊರತು ಪೈಪೋಟಿಗಾಗಿಯಲ್ಲಪ್ಪಾ, ಭಗವಂತ ಕೊಟ್ಟಿದ್ದನ್ನು  ಹೇಗಿಡಬೇಕು? ಎನ್ನುವುದನ್ನು  ತಿಳಿಸುವುದಕ್ಕಾಗಿಯಾದರೂ ನಿಮಗೆ ಬಂದಿರುವ ವಿಷಯವನ್ನು  ತಿಳಿಯಬೇಕಪ್ಪಾ. ದೃಷ್ಟಿಯಿಂದ ಕೇಳಿದೆ.

\section*{ಪ್ರಯೋಗವಿಲ್ಲದ ಗ್ರಂಥ ನಿಷ್ಪ್ರಯೋಜಕ}

`ನೀವು  ಯಾವ ಗ್ರಂಥದ ಆಧಾರದ ಮೇಲೆ ವಿಷಯವನ್ನಿಡುತ್ತೀರಿ' ಎಂದು ನನ್ನನ್ನೇ ಕೇಳಿದರೆ ಬರೀ ಕಾಗದದ ಆಧಾರದ ಮೇಲೆ ವಿಷಯವನ್ನಿಡುತ್ತಿಲ್ಲಪ್ಪಾ. ನಮ್ಮ  ಕಡೆಯಿಂದ ಬರುವ ಶಾಸ್ತ್ರಾವು  ನಿತ್ಯನಿದ್ಧವೂ, ಆತ್ಮಸಿದ್ಧವೂ ಆಗಿರುವ ವಿಷಯದ ಪಾಠವಾಗಿರುವುದರಿಂದ ಕಾಗದದ ಆಧಾರವೇನೂ ಇದಕ್ಕೆ  ಬೇಕಾಗಿಲ್ಲ. ಒಂದು ನೋಟೇ ಆದರೂ ಚಲಾವಣೆಯಲ್ಲಿರುವವರೆಗೆ ಅದರ ಬೆಲೆ. ಚಲಾವಣೆ ರದ್ಧಾದಾಗ ಅದು ಕೇವಲ ಕಾಗದದ ಚೂರಾಗಿ ಬಿಡುತ್ತದೆ. ಅಂತೆಯೇ ವಿಷಯವು  ಪ್ರಯೋಗ ದೊಡನಿದ್ದು  ಅದಕ್ಕೆ  ಹೊಂದಿಕೊಂಡಂತಿರುವ ಗ್ರಂಥ  ಬಳಕೆಯಲ್ಲಿದ್ದರೆ ಅದಕ್ಕೆ  ಬೆಲೆ.  ಆದರೆ ಆ ವಿಷಯ ಪ್ರಯೋಗದಲ್ಲಿಲ್ಲದೆ ಬರೀ ಗ್ರಂಥವಾದರೆ  ಚಲಾವಣೆಯಲ್ಲಿಲ್ಲದ ನೋಟಿನಂತೆ  ನಿಷ್ಪ್ರಯೋಜನ.

\section*{ಶಾಸ್ತ್ರವು  ಆತ್ಮಮೂಲವಾಗಿದೆ}

ನೀವು  ಇಡುವ ವಿಷಯಕ್ಕೆ  ಆಧಾರವೇನು? ಎಂದರೆ  ಸ್ವಾತ್ಮಾಧಾರವೇ ಆಗಿದೆ.  ನಿತ್ಯಸಿದ್ಧವಾದ ಆ ಶಾಸ್ತ್ರಕ್ಕೆ  ಯಾವ  ಆಡಚಣೆಯೂ ಇಲ್ಲಪ್ಪಾ. ನಮ್ಮ  ಪುಸ್ತಕ  ಪ್ರಕೃತಿಯ ಮಡಿಲಿನಲ್ಲಿ  ಹುದುಗಿರುವ ಪುಸ್ತಕ. ಅದರ ಗರ್ಭದಲ್ಲಿದ್ದು  ಹೊರಸೂಸಿದಾಗ ತಾನೇ  ಪ್ರಾಕಾಶವಾಗುತ್ತದೆ. ಒಂದು  ಆತ್ಮವಿರುವೆಡೆಯಲ್ಲಿ ಅವನಿಗೆ ಬೇಕಾದ ಶಾಸ್ತ್ರಗಳೂ ಸ್ವತಃಸಿದ್ಧವಾಗಿಯೇ ಇವೆ. ಎಲ್ಲಿ ಆತ್ಮಗಳು ಓಡಾಡುತ್ತಿವೆಯೋ ಅಲ್ಲಿ ಶಾಸ್ತ್ರಗಳಿವೆ. ವಿಷಯವು ಮೊದಲು ಮನಸ್ಸಿನಲ್ಲಿ  ಮುದ್ರಣವಾಗುತ್ತದೆ. ಎರಡನೆಯದಾಗಿ ಮುಖದಲ್ಲಿ  ಮುದ್ರಿತವಾಗುತ್ತದೆ. ಆದರೆ ಆತ್ಮನ ಮುದ್ರೆಬೀಳದಿದ್ದರೆ ಯಾವ ಶಾಸ್ತ್ರಕ್ಕೂ  ಬೆಳಕಿಲ್ಲ.  ಅಂದಮಾತ್ರಕ್ಕೆ  ಗ್ರಂಥವಾಸಂಗವು  ವ್ಯರ್ಥವೆಂದು  ಭಾವಿಸಬಾರದು. ಆ ವ್ಯಾಸಂಗವು  ಮೂಲಭೂತವಾದ ವಿಷಯದ ಕಡೆಗೆ ದೃಷ್ಟಿ ಹೊರಳಿಸುವಂತೆ  ಮಾಡಬೇಕೆಂಬುದಷ್ಟೇ ನನ್ನ ಆಶಯ. ವಾಕ್ ಸಿಕ್ಕಿದೆ, ಆದರೆ ಅರ್ಥವು  ಸಿಕ್ಕಿಲ್ಲ ಎಂಬಂತಾಗಬಾರದು. ಅರ್ಥವು ಸಿಕ್ಕಿದರೆ  ವಾಕ್ ಅರ್ಥಾನುಗುಣವಾಗಿ ಸೇರಿಕೊಂಡುಬರುತ್ತದೆ. 

\section*{ಶಿಕ್ಷಾಶಾಸ್ತ್ರವು  ನಿಸರ್ಗಸಿದ್ಧವಾಗಿರುವ ಬಗ್ಗೆ  ಉದಾಹರಣೆ}

ವಿಷಾಯವು ಹೇಗೆ ಆತ್ಮ ಸಿದ್ಧವೂ, ಹಾಗೆಯೇ ಸಹಜಸಿದ್ಧವೂ  ಆಗಿದೆ ಎಂಬುದನ್ನು  ಮನಸ್ಸಿಗೆ ತೆಗೆದುಕೊಳ್ಳಲು ಒಂದು ಉದಾಹರಣೆಯನ್ನು  ಇಡಬಯಸುತ್ತೇನೆ. `ಆಕಳಿಕೆ' ಎನ್ನುವ ಪದವನ್ನು  ತೆಗೆದುಕೊಳ್ಳಿ.  ಆಕಳಿಸುವ ಕ್ರಿಯೆಯ ಜೊತೆಯಲ್ಲಿಯೇ ಆ ಶಬ್ದವನ್ನೂ  ಉಚ್ಚರಿಸಬಹುದು. ಹಾಗೆ ಉಚ್ಚರಿಸಿದರೂ ಕ್ರಿಯೆಗೆ ಒಳ್ಳೆ  ಶಯ್ಯೆಯಾಗಿದೆ  ಆ ಪದ. `ಆಕಳಿಕೆ' ಎನ್ನುವುದು ಯಾವ ಭಾಷೆಯ ಪದ? `ಕನ್ನಡ ಭಾಷೆಯದು'  ಎನ್ನುವುದಾದರೆ `ಸಂಸ್ಕೃತದಲ್ಲಿ  ಆಕಳಿಸಿ' ಎಂದರೆ ಹೇಗೆ? ಆಕಳಿಸುವ  ಸಂಸ್ಕಾರವನ್ನು  ಹೊತ್ತುಕೊಂಡೇ ಆ ಪದ ಬಂದಿರುವುದರಿಂದ ಅದೂ `ಸಂಸ್ಕೃತವೇ. ಸೀತ್ಕಾರ ಪೂತ್ಕಾರ ಎನ್ನುವ ಪದಗಳಿವೆ. ಅವು ಸಂಸ್ಕೃತ  ಭಾಷೆಯ ಪದಗಳು. ಅದೇ ಇಂಗ್ಲೀಷ್  ದೇಶದವನೊಬ್ಬ ಪಾಯಸವನ್ನು  ಕುಡಿಯುವಾಗ ` ಸೀತ್, ಸೀತ್, ಎಂದೇನೂ ಕುಡಿಯುವುದಿಲ್ಲವಲ್ಲಾ? ಸೀತ್ ಎನ್ನುವ ಶಬ್ದವೇ ಅಲ್ಲಿಯೂ ಬರುತ್ತದೆ. ಆದ್ದರಿಂದ ಅದು ಕ್ರಿಯಾನುಗುಣವಾಗಿ ಬಂದ ಪದ. ಅದು ಬಿಟ್ಟು, ಆಕಳಿಕೆ  ಎಂಬುದು ದಕ್ಷಿಣದೇಶದ ಪದ' ಎಂದು ಕೊಂಡು  ಉತ್ತರ ದೇಶದವರು `ಊಕಳಿಕೆ' ಎಂದರೆ ಬರುವ ಆಕಳಿಕೆ  ನಿಂತು ಹೋಗುತ್ತದೆ. ಅಂತೆಯೇ ಗಿಳಿಗೆ  `ಕನ್ನಡವಕ್ಕಿ' ಎಂಬ ಹೆಸರಿದೆ. ಅದು ಕನ್ನಡ ದೇಶದ ಹಕ್ಕಿ ಎಂದೇನೂ ಅಲ್ಲ. ತಮಿಳು  ದೇಶಕ್ಕೆ ಹೋದರೆ ಅದು ತಮಿಳುವಕ್ಕಿಯಾಗುದಿಲ್ಲವಲ್ಲ! ಅಲ್ಲಿಯೂ ಕನ್ನಡವಕ್ಕಿಯೇ. ಏಕೆಂದರೆ ಇಲ್ಲಿ  `ಕನ್ನಡ' ಎನ್ನುವುದು ದೇಶವಚಿಯಲ್ಲ. ಕನ್ನಡಿಸು ಎಂದರೆ ಪ್ರತಿಫಲಿಸು-ಪ್ರತಿಬಿಂಬಿಸು ಎಂಬ ಅರ್ಥ. ಗಿಳಿಯು ವಾಕ್ಕನ್ನು  ಪ್ರತಿಫಲಿಸುವುದರಿಂದ ಅದು ಕನ್ನಡವಕ್ಕಿ. ಆದ್ದರಿಂದ ಅದು ಎಲ್ಲಿ ಹೋದರೂ ಕನ್ನಡವಕ್ಕಿಯೇ. ಸಾಹಿತ್ಯವನ್ನು ಉಪಯೋಗಿಸುವಾಗ `ಇಷ್ಟು ಮಾತ್ರೆ,  ಇಷ್ಟು ಸ್ವರ, ಇಷ್ಟು  ವರ್ಣ ಇರುವಂತೆ ಉಚ್ಚರಿಸು' ಎಂದು ನಿಯಮವುಂಟು. ವಿಷಯಕ್ಕೆ ಹೊಂದದಂತೆ ಅಡ್ಡಾದಿಡ್ಡಿ ಹೇಳಿದರೆ ವಿಷಯ ನನಿಲ್ಲುವುದಿಲ್ಲ.  `ಆಕಳಿಕೆ ' ಎಂಬ ಪದವನ್ನೇ ಆದರೂ ಆಕಳಿಕೆ  ಬಂದಾಗ ಮಾತ್ರೆ, ಸ್ವರ, ವರ್ಣ, ಬಲಗಳನ್ನು  ತೆಗೆದುಕೊಂಡು ಹಾಗೆಯೇ ಹೇಳಿದರೆ ಅದು ಆಕಳಿಕೆಗೆ ಒಪ್ಪುದ  ಶಿಕ್ಷಾಶಾಸ್ತ್ರವಾಗುತ್ತದೆ. ಅದು ಬಿಟ್ಟು  ``ಆಕಳಿಕೆ  ಮಾಡಿದೆಯೋ' ಎಂದು  ಗಡಸು ಧ್ವನಿಯಲ್ಲಿ  ಎಲ್ಲ  ಅಕ್ಷರಗಳನ್ನು  ಒತ್ತಿ  ಉಚ್ಚರಿಸುತ್ತಾ ಕೇಳಿದರೆ  ಆಕಳಿಕೆಯ ಕ್ರಿಯೆಯಲ್ಲಿನ  ಧರ್ಮಗಳು ಮರೆಯಾಗಿಬಿಡುತ್ತವೆ. 

ನಿಸರ್ಗಸಿದ್ಧವೂ ಸ್ವತಸ್ಸಿದ್ಧವೂ  ಆದ ಉದಾಹರಣೆಯಾದ್ದರಿಂದ ಆಕಳಿಕೆಯ ಉದಾಹರಣೆಯನ್ನು  ತೆಗೆದುಕೊಂಡೆ. ಆಂಧ್ರರ ಆಕಳಿಕೆ  ಒಂದು, ಕರ್ನಾಟಕದವರ ಆಕಳಿಕೆ ಒಂದು ಅದರಲ್ಲೂ ಬೇರೆ ಬೇರೆ ಜನಾಂಗದವರ  ಆಕಳಿಕೆಯೇ ಬೇರೆ ಬೇರೆ ಎಂದಿಲ್ಲ. `ಆಕಳಿಕೆ' ಎನ್ನುವಾಗ `ಆ' ತ್ರಿಮಾತ್ರದಲ್ಲಿ  ಹೇಳಬೇಡ `ಅ'  ಅಷ್ಟೇ ಸಾಕು ನಂತರ ಕಳಿಕೆ  ಎಂದುಬಿಡು  ಎಂದರೆ  ಅತೃಪ್ತಿಯಾಗುತ್ತದೆ. ತೃಪ್ತಿಯಾಗಿ ಆಕಳಿಸಿದಂತಾಗುವುದಿಲ್ಲ. ಆಕಳಿಸುವಾಗ ವಾಯುವು ಹೊರಬೀಳುವುದಕ್ಕೆ  ಎಷ್ಟು  ಮಾತ್ರಾ ಕಾಲವು  ಬೇಕೋ ಅಷ್ಟು  ಕಾಲ ಉಚ್ಚಾರಣೆಗೂ ಬೇಕು. ಅದಕ್ಕೆ   ಸರಿಯಾಗಿ  ಸ್ವರವೂ ಹುಟ್ಟಿಕೊಳ್ಳುತ್ತದೆ. ಹಾಗಾದಾಗ ಮಾತ್ರ  ಆಕಳಿಕೆ  ಪೂರ್ಣವಾಗುತ್ತದೆ. ಅದರಿಂದ ತೃಪ್ತಿಯಾಗುತ್ತದೆ. ಸಹಜವಾಗಿ ಬಂದಾಗ ಒಂದೊಂದಕ್ಕೆ  ಟೈಮ್-ಸ್ಪೇಸ್ ಡಿಸ್ಟೆನ್ಸ್  (ದೂರ) ಸರಿಯಾಗಿ ಬಿಡಬೇಕು. ಆ-ಉದಾತ್ತವಾಗಿ ಹೇಳಿ `ಕಳಿಕೆ' ಹೇಳೋಣವೇ?  `ಆ' ಇದನ್ನು  ಅನುದಾತ್ತವಾಗಿ ಹೇಳೋಣವೇ? ಅಂದರೆ ಅದಾವುದೂ ಬೇಡ. `ಆಕಳಿಕೆ' ಕ್ರಿಯೆಯು ನಡೆಯುವಾಗ ಏನು ಬರುತ್ತದೆಯೋ ಅದನ್ನು  ತಿಳಿದು ಅಂತೆಯೇ ಹೇಳಬೇಕೇ ಹೊರತು ಬೇರೇನೂ ಅಲ್ಲ. ಆಕಳಿಕೆಯನ್ನು  ಅನುಕರಣ ಮಾಡಬೇಕಾದರೆ ಅದು ಬರುವ ಜಾಗದಲ್ಲೇ ನೋಡಬೇಕು.

\section*{ತಿಳಿದ ವಿಷಯವನ್ನವಲಂಬಿಸಿ ತಿಳಿಯದ ವಿಷಯದ ಬಗ್ಗೆ  ಶಿಕ್ಷಣವಿರಬೇಕು}

`ಶಿಕ್ಷಾಶಾಸ್ತ್ರವನ್ನು  ಹೇಳುವುದಕ್ಕೆ  ಹೊರಟು ಮೊದಲೇ ಆಕಳಿಕೆ  ಶುರು ಮಾಡಿದಿರಲ್ಲಾ' ಎಂದರೆ ಜೀವನಕ್ಕೆ  ಅಂಟಿಕೊಂಡ ವ್ಯಾಪಾರವಾದ್ದರಿಂದ ಅದರ ಉದಾಹರಣೆ. ಆಕಳಿಕೆ, ತೂಕಡಿಕೆಯ ಪ್ರಪಂಚದಲ್ಲಿರುವವರನ್ನು  ತಕ್ಷಣ  ಬೇರೆ  ಪ್ರಪಂಚಕ್ಕೆ   ಒಯ್ದರೆ ಅರ್ಥವಾಗುವಿದಿಲ್ಲ. ಆದ್ದರಿಂದ ಅದರ ಮೂಲಕವೇ ಅವರ ಜೀವನದಲ್ಲಿ ಪರಿಚಿತವಾದ ವಿಷಯದ ಮೂಲಕವೇ ಒಂದು ಶಿಕ್ಷಣಕೊಡಬೇಕು. ಕೊಂಬೆ ತೋರಿಸಿ ಅದರ ನೇರದಲ್ಲಿ ಬೆರಳಿಟ್ಟು  ತೋರಿಸುತ್ತಾ, `ನೋಡಲ್ಲಿ ಆ ನೇರದಲ್ಲಿ  ಕೊಂಬೆಯ ಮೇಲೆ ಚಂದ್ರ' ಎಂದರೆ  ಚಂದ್ರನತ್ತ  ದೃಷ್ಟಿ  ಹರಿಯಿಸುವಂತೆ  ಮಾಡಲು ಕೊಂಬೆ ಒಂದು ಅವಲಂಬನ ಮಾತ್ರ..  ಇಲ್ಲದಿದ್ದರೆ ಕೊಂಬೆಗೂ ಚಂದ್ರನಿಗೂ ಲಕ್ಷಾಂತರ ಮೈಲಿ ದೂರವಿರುವಾಗ `ಕೊಂಬೆ ಮೇಲೆ ಚಂದ್ರ' ಎನ್ನುವುದು ಹೇಗೆ? (ಇದನ್ನೇ ಶಾಖಾcಅಂದ್ರಮಸನ್ಯಾಯ  ಎನ್ನುವರು) ಯಾವ ವಿಷಯವನ್ನು ತಿಳಿಸುತ್ತೇವೆಯೋ, ಅದಕ್ಕಿಂತ  ಭಿನ್ನವಾದ ವಿಷಯದತ್ತ  ಬುದ್ಧಿಯು ಪ್ರವೇಶಿಸದಿರುವುದಕ್ಕಾಗಿ ಉದಾಹರಣೆಯ ಅವಲಂಬನೆ. 

\section*{ವೇದಾಧ್ಯಾಯಿಯು ಹೇಗಿರಬೇಕಾಗುತ್ತದೆ?}

ಆಕಳಿಕೆ ಬಂದಾಗ ಅಂಗೋಪಾಂಗಗಳ ಇಡುವಿಕೆ  ಒಂದು ರೀತಿ ಇರಬೇಕು. ಮೇಲೆ ಕತ್ತೆತ್ತಿದರೆ ಆಕಳಿಕೆ ಬರುವುದಿಲ್ಲ. ಆದ್ದರಿಂದ `ಆಕಳಿಸಬೇಕಾದರೆ ಈ ರೀತಿ ಕುಳಿತುಕೊಳ್ಳಬೇಕು' ಎಂದು ವಿಧಿ. ಅಂತೆಯೇ ವೇದಾಧ್ಯಯನದಲ್ಲಿಯೂ ಹೀಗೆಯೇ ಕುಳಿತು ಹೇಳಬೇಕು ಎಂಬ ವಿಧಿಯೂ ಉಂಟು.

\begin{shloka}
ಗೀತೀ ಶೀಘ್ರ್ರೀ ಶಿರಃ ಕಂಪೀ ಯಥಾಲಿಖಿತಪಾಠಕಃ|\\
ಅನರ್ಥಜ್ಞೋಽಲ್ಪಕಂಠಶ್ಚ ಷಡೇತೇ ಪಾಠಕಾಧಮಾಃ||
\end{shloka}

(ವೇದ ಪಾಠಕರಲ್ಲಿ ಅಧಮರನ್ನು  ಈ ಶ್ಲೋಕವು ತಿಳಿಸುತ್ತದೆ. ವೇದ ಭಾಗಗಳನ್ನು  ಸಲ್ಲದೆಡೆಗಳಲ್ಲಿ  ಎಳೆದೆಳೆದು ಹೇಳುವವನು, ಅತಿವೇಗವಾಗಿ ಹೇಳುವವನು, ತಲೆಯನ್ನು  ತೂಗುತ್ತಾ ಹೇಳುವವನು, ಬರೆದುದನ್ನೇ ಆಧರಿಸಿ ಓದುವವನು, ಅರ್ಥತಿಳಿಯದೇ ಹೇಳುವವನು, ಅತ್ಯಲ್ಪವಾದ ಕಂಠ ಬಲವುಳ್ಳವನು, ಈ ಆರುಜನರೂ ಪಾಠಕಾಧಮರು)

ವಿಷಯವನ್ನೇ ಗಮನಿಸದೇ ತಲೆಯಲಾಡದಿರುವುದೇ ನಿಯಮವೆಂದು ಭಾವಿಸಿ ತಲೆಯಲ್ಲಾಡಿಸದಿದ್ದರೆ, ಒಳ್ಳೆಯ ವೇದಾಧ್ಯಾಯೇ ಎನ್ನಲಾಗುವುದಿಲ್ಲ. ಆದರೆ ವಿಷಯದ ರಸದಿಂದ ಅವನ ತಲೆ ತಾನಾಗಿಯೇ ಅಲಾಡದಿರಬೇಕು. ಅಂತಹವನು ಒಳ್ಳೆಯ ವೇದಾಧ್ಯಾಯೀ.

\section*{ಅಂಗವು ಅಂಗಿಯ ಕಡೆಗೆ ಕರೆದೊಯ್ಯುವಂತಿರಬೇಕು}

ವೇದಕ್ಕೆ  ಅಂಗವಾಗಿದೆ ಶಿಕ್ಷಾಶಾಸ್ತ್ರ. ಒಂದು ಕಡೆಗೆ ಒಯ್ಯುವುದಕ್ಕನುಗುಣವಾಗಿ ಒಂದು ಚೇಷ್ಟೆ(ಕ್ರಿಯೆ) ಹುಟ್ಟಿಕೊಳ್ಳುತ್ತದೆ. (ಅಲ್ಲಿದ್ದವರೊಬ್ಬರ ಕಡೆಗೆ ನೋಡಿ `ಇಲ್ಲಿ ಬನ್ನಿ' ಎಂದು ಕೈ ತೋರಿಸಿದರು. ಆಗ ಅವರು ಬಂದು ಅಲ್ಲಿ ಕುಳಿತರು. ಅದರ ಕಡೆ ಎಲ್ಲರ ಗಮನವನ್ನೂ  ಸೆಳೆಯುತ್ತಾ ) ನೋಡಿ ಮನಸ್ಸಿನ ಆಶಯಕ್ಕನುಗುಣವಾಗಿ ಕೂಗಿದರೆ ಆ ಕಾರ್ಯ ಜರುಗುತ್ತದೆ. ಅಂಗದಿಂದ ನಡೆದ ಕಾರ್ಯ ಅಂಗಿಯ ಕಡೆಒಯ್ಯುತ್ತದೆ. ಅದು ಬಿಟ್ಟು  `ಬನ್ನಿ' ಎಂದು  ಆಕಾಶಕ್ಕೆ  ಕೈ ತೋರಿಸಿ  `ಕುಳಿತುಕೊಳ್ಳಿ ' ಎಂದರೆ  ಏನು ಮಾಡಬೇಕೋ ತೋರುವುದಿಲ್ಲ. ಹಾಗೆ ತೋರಿಸಬಾರದು. ಅಂಗವನ್ನು  ನೋಡಿದಾಗ ಅಂಗಿಯ ಕಡೆಗೆ ಒಯ್ಯುವಂತಿರಬೇಕು. ಭಾಷೆಯನ್ನೇ ಉಪಯೋಗಿಸದೇ ಇದ್ದರೂ ಮುಖ-ಹಸ್ತವಿನ್ಯಾಸಗಳಿಂದಲೂ  ನೀವು ಬರುವಂತೆ ಮಾಡಬಹುದು. ಆದರೆ  ಹಾಗೆ ನೀವು ಬರಬೇಕಾದರೆ `ಈ ರೀತಿಯೇ ಕೈ ಅಲ್ಲಾಡಿಸಬೇಕು. ಮುಖದ ಸ್ಥಿತಿ  ಹೀಗೆಯೇ  ಇರಬೇಕು' ಎಂದು ಇದಕ್ಕೊಂದು ವಿಧಿಯುಂಟು. ಈ ನಿಯಮಕ್ಕೆ  ಅಂಗಗಳು ಒಳಪಟ್ಟರೆ ಅಂಗಿಯ ಆಶಯ ವ್ಯಕ್ತವಾಗುತ್ತದೆ, ಇಲ್ಲದಿದ್ದರೆ ಕೆಡುತ್ತದೆ.

\section*{ಅಂಗವು ಅಂಗಿಯ ಆಶಯವನ್ನ್ನೇ ವ್ಯಕ್ತಪಡಿಸುತ್ತದೆ}

`ಶಿಕ್ಷೆಯು ವೇದಾಂಗವಾಗಿದೆ' ಎಂದರೆ ವೇದದ ಅಭಿಪ್ರಾಯವನ್ನು  ಹೊರಗಿಡುವ ಒಂದು ದ್ವಾರವಾಗಿದೆ. ವೇದವೆಂದರೆ  ಜ್ಞಾನ. ಅದು ತನ್ನ ಅಭಿಪ್ರಾಯವನ್ನು  ಸ್ಪಷ್ಟಪಡಿದಲು ಹಲವು ಅಂಗಗಳನ್ನು ಮಾಡಿಕೊಂಡಿದೆ. ಹೊರಗಿನಿಂದ ಒಳಕ್ಕೆ  ಒಯ್ಯಲೂ, ಒಳ ಆಶಯವನ್ನು  ಹೊರಬಿಂಬಿಸಲೂ ಹಲವು ಅಂಗಗಳಿವೆ. ಒಂದೊಂದು  ಅಂಗಕ್ಕೂ ಅದರದೇ ಆದ ಉಪಯೋಗವಿದೆ. ಮೂಗು cಏತನನ ಅಂಗ ಸರಿ. ಆದರೂ `ಈ ಪೋಟೋ ಎಷ್ಟು ಲಕ್ಷಣವಾಗಿದೆ ನೋಡು ಮೂಗೇ' ಎಂದರೆ, ಮೂಗು ನೋಡುವುದಿಲ್ಲ. ಅದರ ಕೆಲಸ ವಾಸನೆ ಗ್ರಹಿಸುವುದು ಮಾತ್ರ . ಅಂತೆಯೇ ಕಿವಿಯ ಕೆಲಸ ಶಬ್ದವನ್ನು ಗ್ರಹಿಸುವುದು. ಅದು ಬಿಟ್ಟು ಸುವಾಸನೆಯಾದ ಹೂವನು ಅಲ್ಲಿಗೆ ಒಯ್ದರೆ ವಿಷಯಗ್ರಹಣವಿಲ್ಲ. ಯಾವ ವಿಷಯವನ್ನು  ಯಾವ ಇಂದ್ರಿಯಕ್ಕೆ  ಕೊಟ್ಟರೆ ಅಂಗಿಯವರೆಗೂ ತಲುಪುತ್ತದೆಯೋ ಆ ರೀತಿ ವಿಷಯವನ್ನು  ಆ ಅಂಗಗಳಿಗೇ ಕೊಡಬೇಕು. ಅಂತೆಯೇ ಶಿಕ್ಷಾಶಾಸ್ತ್ರವು ವೇದಕ್ಕೆ  ಮೊದಲನೆಯ ಅಂಗವಾಗಿದೆ. ವ್ಯಾಕರಣ, ಛಂದಸ್ಸು, ನಿರುಕ್ತ, ಜ್ಯೌತಿಷ, ಕಲ್ಪ ಇವುಗಳೂ ವೇದಾಂಗಗಳಾಗಿವೆ. ಅವುಗಳು ತಮ್ಮದೇ ಆದ ವಿಧಾನದಲ್ಲಿ ವೇದದ ಅಭಿಪ್ರಾಯವನ್ನು  ವ್ಯಕ್ತಪಡಿಸುತ್ತವೆ. ಪುರುಷನ ಒಂದು ಅಭಿಪ್ರಾಯವನ್ನು ಪ್ರತಿಬಿಂಬಿಸಲು-ವ್ಯಕ್ತಪಡಿಸಲು ಅಂಗರಚನೆಯಿದೆ. ಮನಸ್ಸೂ  ಅದಕ್ಕೆ  ಅಧೀನವಾಗಿ ಇತರ ಅಂಗಗಳೂ ಕೆಲಸ ಮಾಡುತ್ತವೆ. ಮನಸ್ಸಿನ ಧಾರೆಯನ್ನು  ಅಂಗದ ಮೇಲೆ ಹರಿಸಿದಾಗ ಸಂತೋಷ-ಸುಖ ಇವುಗಳು ಅಂಗದಲ್ಲಿ ವ್ಯಕ್ತವಾಗುತ್ತವೆ. 

\section*{ಅಂಗವು ಸ್ಥಾನಭ್ರಷ್ಟವಾಗಿರಬಾರದು}

ಅಂಗವು ಅದಕ್ಕೆ  ಒಂದು ಕಂಡೀಷನ್ನಲ್ಲಿರಬೇಕು. ಇಲ್ಲದಿದ್ದರೆ ಕೆಲಸ ಸಾಗಿದರೂ ಜನ ನಂಬುವುದಿಲ್ಲ. ಉದಾಹರಣೆಗೆ- ಮದುವೆಯಾಗಬೇಕಾದ ಹುಡುಗಿಯೊಬ್ಬಳಿಗೆ ಮಾಲಗಣ್ಣಿರುತ್ತದೆ. ಈ ಹುಡುಗಿಗೆ ಗೊತ್ತಾದ ವರ ಚೆನ್ನಾಗಿ ಅಲಂಕಾರ ಮಾಡಿಕೊಂಡು ಬಂದಿರುತ್ತಾನೆ. ಮಾಲಗಣ್ಣಿನವರಿಗೆ ಒಂದು ಸೌಲಭ್ಯವಿರುತ್ತದೆ. ಯಾರನ್ನು  ನೋಡುತಾರೆಯೋ, ಅವರಿಗೆ ತಿಳಿಯದಂತೆಯೇ ಬೇರೆಡೆಗೆ ನೋಡುತ್ತಿರುವಂತೆ ತಮಗೆ ಬೇಕಾದವರನ್ನು  ನೋಡಬಹುದು. ಹುಡುಗನ ಕಡೆಗೇ ನೋಡುತ್ತಿದ್ದರೂ ಹುಡುಗನಿಗೆ  ಮಾತ್ರ ಹುಡುಗಿ ಬೇರೆಡೆಗೆ ನೋಡುತ್ತಿರುವಂತೆ ಕಾಣುತ್ತದೆ. ಅದರಂತೆ ಅವನಿಗೆ ಹುಡುಗಿ ತನ್ನೆಡೆಗೆ ನೋಡದೆ ಬೇರೆಡೆಗೆ ನೋಡುತ್ತಿರುವಂತೆ ಕಾಣುತ್ತದೆ. ಅದರಂತೆ ಅವನಿಗೆ ಹುಡುಗಿ ತನ್ನೆಡೆಗೆ ನೋಡದೆ ಬೇರೆಡೆಗೆ ನೋಡುತ್ತಿದ್ದಾಳಲ್ಲ ಎಂದು ಅಸಮಾಧಾನವಾಗಬಹುದು. ಆದರೆ ಹುಡುಗಿ ಮಾತ್ರ ಅವನನ್ನೇ ನೋಡುತ್ತಿರುತ್ತಾಳೆ. `ನಾನು ನಿಮ್ಮನ್ನೇ ನೋಡಿದುದು' ಎಂದು ಹುಡುಗಿಯೇ ಹೇಳಿದರೂ ಹುಡುಗ  ನಂಬುವುದಿಲ್ಲ. ಸ್ವಲ್ಪ ಸ್ಥಾನದಿಂದ ಜಾರಿದಾಗ ಅಂಗವು ತನ್ನ ಕೆಲಸವನ್ನು  ತಾನು ಮಾಡಿದರೂ ಅದನ್ನು  ಜನ ನಂಬುವುದಿಲ್ಲ. ಆದ್ದರಿಂದ ಅಂಗವು ಸರಿಯಾದ ಕಂಡೀಷನ್ನಲ್ಲಿದ್ದು ಅಂಗಿಯ ಕಡೆಗೆ ಎಲ್ಲ ವಿಧದಿಂದಲೂ ನೋಡುವಂತಾಗಬೇಕು. ಸ್ಥಾನಭ್ರಷ್ಟತೆ ಇಲ್ಲದೇ ಇರಬೇಕು.

\section*{ವೇದವು ಸ್ವರವಿಧಿಯನ್ನು  ಹೊಂದಿಯೇ ಹೊರಬಂದಿದೆ}

ವೇದದಲ್ಲಿರುವ ಛಂದೋಮಯವಾದ, ಮಂತ್ರಮಯವಾದ ವಾಣಿಯು ಋಷಿಗಳಿಂದ ಹೊರಟಾಗ ಯಾವ ಸ್ವರವಿಧಿಗಳೂ ಅದಕ್ಕಿರಲಿಲ್ಲ. ಅದು ತಾನೇ ತಾನಾಗಿ ಹೊರಬಿತ್ತು. ಕೆಮ್ಮಿದಾಗ ಇಷ್ಟೇ ಮಾತ್ರಾಕಾಲದಲ್ಲಿ  ಕ್ಕೆಮ್ಮಬೇಕು ಎಂದು ಕೆಮ್ಮುವವನು ವಿಧಿಯ ಪ್ರಕಾರ ಕೆಮ್ಮಲಾಗುವುದಿಲ್ಲ. ಅವನ ಕೆಮ್ಮನ್ನು ನೋಡಿ ಇತರರು ಅದರಲ್ಲಿರು ಮಾತ್ರೆಗಳನ್ನು  ಗಣಿಸಿ ಮಾತ್ರಾಕಾಲವನ್ನು  ಗುರುತಿಸಿಕೊಳ್ಳಬಹುದು ಅಷ್ಟೇ. ಋಷಿಗಳಿಗೆ ಸತ್ಯಸಾಕ್ಷಾತ್ಕಾರವಾದಾಗ, ಅದರ ಬಗೆಗೆ ಆಡಿಕೊಳ್ಳಬೇಕು  ಎಂದು ಅವರಿಗೆ ಅನ್ನಿಸಿತು. ಹಾಗೆ ಆ ವಿಷಯವನ್ನು ಕುರಿತು ಆಡಿದಾಗ ತನ್ನದೇ ಆದ ನಿಯಮಾವಳಿಯಿಂದ ಅದು ಹೊರಬೀಳುತ್ತದೆ. ಅದರ ಒಡನೆಯೇ ಸ್ವರ ಮತ್ತಿತರ ವಿಧಿಗಳೆಲ್ಲಾ ಬರುತ್ತವೆ.

\section*{ಮೂಲದ ಜೊತೆಗೆ ಹೊಂದಿಕೊಂಡ ಅನುಕರಣೆ ಮೂಲಕ್ಕೆ  ಕರೆದೊಯ್ಯಬಲ್ಲದು}

ಒಬ್ಬ ಮನುಷ್ಯನನ್ನು ಜಿಗುಟಿದರೆ ಅಲ್ಲಿ ವೇದನೆಗೆ ತಕ್ಕಂತೆ ಧ್ವನಿಯು ಬರುತ್ತದೆ. ಅದನ್ನು ಹೆಳಬೇಕಾದರೆ ಆಧರ್ಮವನ್ನು ಗಮನಿಸಿ ಅನುಕರಿಸಿದರೆ ಆಸಂದರ್ಭಕ್ಕೆ  ಒಪ್ಪುತ್ತದೆ. ಕುದುರೆ ಕೆನೆಯುತ್ತದೆ ಎನ್ನುವುದು ತಿಳಿದಿದೆ. ಕುದುರೆ ಕೆನೆಯುವುದನ್ನು ಅನುಕರಿಸಿ ಒಬ್ಬ ವ್ಯಕ್ತಿಯೇ ಕುದುರೆಯಂಟೆ ಕೂಗಿದರೂ ಕುದುರೆಯ ಹತ್ತಿರಕ್ಕೆ ಮನಸ್ಸು  ಹೋಗುತ್ತದೆ. ನಮಗೆ ಪರಿಚತರಾದ ವ್ಯಕ್ತಿಯೊಬ್ಬರಿರುತ್ತಾರಾದರೆ ಅವರ ನಡೆ-ನುಡಿಗಳನ್ನು ಅನುಕರಿಸಿದರೆ ಮನಸ್ಸು ಅವರನ್ನು ಒಮ್ಮೆ  ನೋಡಿದವರಿಗೂ, ಅಲ್ಲಿಯೇ ಹೋಗಿ ಕುಳಿತುಕೊಳ್ಳುತ್ತದೆ. ಸತ್ಯಸಾಕ್ಷಾತ್ಕಾರ ಮಾಡಿಕೊಂಡಾಗ ತನಗೆ ತಾನೇ ಆನಂದಿಸಿದ್ದೋ ನೆಮ್ಮದಿಪಟ್ಟಿದ್ದೋ, ಮೊದಲಾದ ಅನುಭವವನ್ನು  ಅದರಂತೆಯೇ ಅನುಕರಿಸಿ ತೋರಿಸಿದರೆ ಅದರ ತಂಪು ತಟ್ಟುತ್ತದೆ. ಅದರ ಸುಳಿವು ಸಿಕ್ಕುತ್ತದೆ

\section*{ಶಿಕ್ಷೆಯ ಆವಿರ್ಭಾವ}

ರಾಮನು ಕಾಡಿಗೆ ಹೋದಾಗ  ಕೌಸಲ್ಯೆ  ಹೇಗೆ ಅತ್ತಳು ಎಂಬುದನ್ನು ನಾಟಕದಲ್ಲಿ ಅಭಿನಯಿಸಿ ತೋರಿಸಬೇಕಾದರೂ ಅಗಲಿಕೆಯ ದುಃಖವನ್ನು ಅನುಭವಿಸಿದ ಸಂಸ್ಕಾರ ಬೇಕು. `ರಾಮನು ಕಾಡಿಗೆ ಹೋದಾಗ ತುಂಬ ದುಃಖಪಟ್ಟಳು ಕೌಸಲ್ಯೆ' ಎಂಬುದನ್ನು  ಹರ್ಷವನ್ನು ಸೂಚಿಸುವ ಧ್ವನಿಯಲ್ಲಿ ಹೇಳಿಬಿಟ್ಟರೆ  ಆ ಧರ್ಮವು ತಲೆಯೆತ್ತುವುದೇ ಇಲ್ಲ. `ಗಂಡಸಿನ ಮುಂದೆ ಗೌರೀದುಃಖವೇ' ಎಂಬ ಆಡಿಕೆಯ ಮಾತೊಂದುಂಟು. ಗೌರಿಯ ದುಃಖವನ್ನು ಅರ್ಥಮಾಡಿಕೊಳ್ಳಲು ಒಂದು ಸಮಾನ ಮನೋಧರ್ಮವು ಬೇಕು. ಆ ಧರ್ಮವಿಲ್ಲದಿದ್ದರೆ ಅನುಕರಣೆ ಕೃತಕವಾಗುತ್ತದೆ. ತಮಾಷೆಯಾಗಿದ್ದರೂ ನಿಜವಾಗಿಯೂ ಅಳುವವನಂತೆ ಅತ್ತು  ಸರಿಯಾದ ಅಳುವನ್ನು  ತೋರಿಸಿದರೆ, ಆ ಧರ್ಮ ಮತ್ತು ಆ ರಸರಿಂಡ ಕೂಡಿದ್ದರೆ ನಿಜವಾದ ಅಳುವಿನ ಪರಿಣಾಮವನ್ನೇ ಅದು ಬೀರುತ್ತದೆ. ನಾಟಕದಲ್ಲಿ ರಸವುಂಟಾಗಬೇಕಾದರೆ ನಿಜವಾದ ವಿಷಯವು ಹೊರಟಾಗ ಯಾವ ರಸವಿರುತ್ತದೆಯೋ ಆರಸವು ಈ ದ್ವಾರಾ (ಅಭಿನಯಿಸುವವನ ಮೂಲಕ) ಅಭಿವ್ಯಕ್ತವಾಗುವಂತೆ ಬರಬೇಕು. ಅದದರದೇ ಆದ ಧರ್ಮಹೊತ್ತು ಬರದಿದ್ದರೆ ನಟನೆಯು ತಾನು ಮಾಡುವ ಕೆಲಸವನ್ನು  ಮಾಡುವುದಿಲ್ಲ. ಆದ್ದರಿಂದ ಆ ಅಂಗವು ಅಂಗಿಯ ಕಡೆಗೊಯ್ಯಬೇಕಾದರೆ ಅದಕ್ಕೆ ತಕ್ಕ ಹೊಂದಾಣಿಕೆ ಬೇಕು. 

ಸತ್ಯಸಾಕ್ಷಾತ್ಕಾರದ ಅನುಭವವನ್ನು  ಹೊತ್ತ ವಾಣಿಯು ಯಾವ ಧರ್ಮದಿಂದ ಕೂಡಿದೆಯೋ ಆ ಧರ್ಮವನ್ನು ಹಾಗೆಯೇ ಹೊತ್ತು ತರುವುದಕ್ಕೆ ತರುವುದಕ್ಕೆ ಬೇಕಾದ ವರ್ಣ-ಸ್ವರಗಳೇ ಅಲ್ಲಿಯು ಶಿಕ್ಷೆ. ಅದುಬಿಟ್ಟು `ಇಂದ ಶಿಕ್ಷೆರ್ಯ ಪ್ರಕಾರಂ ಇಪ್ಪಡಿ ಶೊಲ್ಲಣುಂ' (ಈ ಶಿಕ್ಷೆಯ ಪ್ರಕಾರ ಹೀಗೆ  ಹೇಳಬೇಕು) ಅಂದರೆ ವಿಷಯಕ್ಕೆ  ಹೊಂದಿಕೊಳ್ಳದಿದ್ದಾಗ ಅದು ನಿಜವಾದ ವಿಷಯಕ್ಕೆ ಶಿಕ್ಷೆಯಾಗುತ್ತದೆ. ರಕ್ಷೆಯಾಗುವುದಿಲ್ಲ
 
 ಋಷಿಗಳು ಸತ್ಯಸಾಕ್ಷಾತ್ಕಾರ ಮಾಡಿಕೊಂಡಾಗ ಯಾವ ಒಂದು ವೇದವಾಣಿಯು ಹೊರಬಿತ್ತೋ ಅಲ್ಲಿ ಈ ಶಿಕ್ಷೆಯ ಪ್ರಕಾರ ಹೀಗೆ ಹೇಳಬೇಕೆಂಬ (ಹೊರ) ನಿಯವವೇನೂ ಇಲ್ಲ ಅದು ಯಾವ ಧರ್ಮದಿಂದ ಹೊರಟಿತೋ ಅದನ್ನು ಗಮನಿಸಿ ಅದರ ಮನೋಧರ್ಮವು ಬಾಡದಂತೆ ಅದಕ್ಕನುಗುಣವಾದ ಸ್ವರ, ವರ್ಣ, ಮಾತ್ರಾ, ಬಲಗಳನ್ನು  ಗುರುತಿಸಿದರೆ ಅದು ಅಲ್ಲಿಯ ಶಿಕ್ಷೆ.
 
\section*{ಕಾಲಕಾಲಕ್ಕೆ  ಶಿಕ್ಷೆಯಲ್ಲಿ ಸ್ಥಿತಿಗತಿ}

ಆದರೆ ಈಗ ವಿಷಯವು ಅನೇಕ ಕೈದಾಟಿ ಬೆಳೆದು ಬಂದಿದೆ. ಋಷಿಗಳಿಂದ ಋಷೀಕ, ಋಷಿಬ್ರುವ (ತಾನು ಋಷಿಯಲ್ಲದಿದ್ದರೂ ಋಷಿಯೆಂದು ಹೇಳಿಕೊಳ್ಳುವವನು) ಮೊದಲಾದವರ ಕೈದಾಟಿ ಬಂದಿದೆ. ಮಧ್ಯೇ ಹೀಗಾಗಿರುವುದನ್ನು  ವಿಷಯ ತಿಳಿದವರು ಶಿಕ್ಷೆಯ ಸಹಜತೆಯನ್ನರಿತು, ಅದನ್ನು  ಅಂತೆಯೇ ಇಡಲು ಪ್ರಯತ್ತಿಸಿದರೆ ಅದರಂತೆಯೂ ಒಂದು ಪುಸ್ತಕವು ಬರಬಹುದು. ಅದಿದ್ದರೂ  ವಿಷಯದಿಂಡ (ಸಹಜವಾದ ಶಿಕ್ಷೆಯ ಕ್ರಮದಿಂದ) ದೂರ ಸರಿದಿರಬಾರದು. ಆ ಧರ್ಮವರಿಯದೇ ಇದ್ದರೆ ವಿಷಯವು ಕೆಡುತ್ತದೆ.

  
\section*{ಸ್ವರವ್ಯತ್ಯಾಸದಿಂದ  ಅರ್ಥವ್ಯತ್ಯಾಸ}

`ಆಕಳಿಕೆ' ಇದನ್ನ್ನು ಆ ಧರ್ಮದೊಡನೆ ಆಡದಿದ್ದರೆ ಕ್ರಿಯೆಯ ಕಡೆಗೆ ಒಯ್ಯುವುದಿಲ್ಲ. ನಿದ್ರೆ ಬರುವಾಗ `ತೂಕಡಿಕೆ ಬರುತ್ತದೆ' ಎಂದು ಜೋರಾಗಿ ಹೇಳಿದರೆ ವಿಷಯವು ವ್ಯತ್ಯಾಸವಾಗಿಬಿಡುತ್ತದೆ. ಬರುವ ನಿದ್ರೆಯೂ ಹೋಗಿಬಿಡುತ್ತದೆ. ಫುಟ್ ಬಾಲ್ - ಚೆಂಡು ಗೋಡೆಗೆ ಬಿದ್ದಾಗ  ಹಿಂದಿರುಗುವಂತೆ ವಿಷಯ ವ್ಯತ್ಯಾಸವು ಹೇಳಿದವನಿಗೇ ತಿರುಗುತ್ತದೆ. ವೇದದ ಉದಾಹರಣೆಯನ್ನೇ ತೆಗೆದುಕೊಂಡರೆ `ಇಂದ್ರ ಶತ್ರುವರ್ಧಸ್ವ' ಎಂಬ ಜಾಗದಲ್ಲಿ `ಇಂದ್ರನಿಂದ ಕೊಲ್ಲಲ್ಪಡುವ ಮಗ' ಎನ್ನುವ ಅರ್ಥಬರುವಂತೆ ಉಚ್ಚಾರಣೆ ನೆಡೆದಿದೆ. ಇದು ಸ್ವರವ್ಯತ್ಯಾಸದಿಂದ ಆದ ದೋಷ. ಸ್ವರವ್ಯತ್ಯಾಸವೊಂದರಿಂದ ಹಾಗಾಗುವುದೇ ಎಂದರೆ ಆಗುವುದರಲ್ಲಿ ಅನುಮಾನವಿಲ್ಲ. `ನಿದ್ರೆ ಮಾಡುತ್ತಾ ಇದ್ದೀಯೇನೋ?' ಎಂದು ಕೇಳಿದಾಗ `ನಿದ್ರೆ ಮಾಡುತ್ತಾ ಇದ್ದೇನೆ' ಎಂಡು ಜೋರಾಗಿ ಹೇಳಿದರೆ ಬರುವ ನಿದ್ರೆಯೂ ಹೋಗಿ ಬಿಡುತ್ತದೆ. ಇಲ್ಲಿಯೂ ಸ್ವರವ್ಯತ್ಯಾಸದಿಂದಲೇ ದೋಷವಾದುದು. ಭೌತಿಕವಾಗಿಯೇ ಸ್ವರವ್ಯತ್ಯಾಸವಾದರೆ ಕೆಲಸ ಕೆಡುತ್ತದೆ. ಇನ್ನು  ಋಷಿಗಳ ಕಡೆಯಿಂದ ಆಧ್ಯಾತ್ಮಿಕದ ವಿಷಯವಾಗಿ ಬಂದ ಮಂತ್ರದ ಬಗೆಗೆ ಹೇಳಬೇಕಾದುದೇನು?

\section*{ಮಂತ್ರವನ್ನು ಕುರಿತು}

\begin{shloka}
	`ಮನನಾತ್ ತ್ರಾಣನಾಚ್ಚೈವ ಮಂತ್ರ ಇತ್ಯಭಿಧೀಯತೇ |
\end{shloka}

(ಮನನ ಮತ್ತು ತ್ರಾಣನ - ರಕ್ಷಣದಿಂದ ಮಂತ್ರವೆಂದು ಹೇಳಲ್ಪಡುತ್ತದೆ.)

ಮನನ ಮತ್ತು ತ್ರಾಣನ ಎರಡೂ ಇದ್ದಾಗ ಮಂತ್ರ. ಲೌಕಿಕವಾಗಿಯೇ ಆದರೂ ಅನುಕರಣೆಯನ್ನು ಸರಿಯಾಗಿ ಮಾಡಿ ಇತರತ ದುಃಕವನ್ನು ಸೂಚಿಸಿದರೆ ಸಂಸ್ಕಾರವುಳ್ಳವನಿಗೆ ಬಿಕ್ಕಿ ಬಿಕ್ಕಿ ಅಳುವ ಬರುತ್ತದೆ. ಅಂತೆಯೇ ಮಂತ್ರವೂ ಕೂಡ ಮನನ ಮತ್ತು ತ್ರಾಣನಕ್ಕೆ ಅನುಗುಣವಾದ ವ್ಯವಸ್ಥೆ. ಅಲ್ಲಿ ಒಂದು ಶಿಕ್ಷೆ.

\begin{shloka}
ಶೀಕ್ಷಾಂ ವ್ಯಾಖ್ಯಾಸ್ಯಾಮಃ | ವರ್ಣಃ ಸ್ವರಃ | ಮಾತ್ರಾ ಬಲಂ |\\
ಸಾಮ ಸಂತಾನಃ|
\end{shloka}

(`ಆಕಾರದಿವರ್ಣಗಳು, ಉದಾತ್ತಾದಿಸ್ವರಗಳು, ಉಚ್ಚಾರಣಕಾಲದಮಿತಿ, ಪ್ರಯತ್ನಗಳು, ಉಚ್ಚಾರಣೆಯಲ್ಲಿ ಸಾಮ್ಯ, ಪೂರ್ವೋತ್ತರವರ್ಣಗಳಸಂಹಿತೆ, ಇವಿಷ್ಟೂ ಶಿಕ್ಷಾ ಶಾಸ್ತ್ರದ ವಿಷಯವಾಗಿದೆ)

ವೇದಾಂಗಗಳಲ್ಲಿ ಶಿಕ್ಷೆಯು ವರ್ಣ, ಸ್ವರ, ಮಾತ್ರೆಗಳ ನಿಯಮ ಹಾಗೂ ಮರ್ಮವನ್ನು ತಿಳಿಸುತ್ತದೆ.

\section*{ವೇದಾಂಗಗಳು ವೇದಪುರುಷನ ಆಶಯವ್ಯಕ್ತತೆಗೆ ದ್ವಾರಗಳು}

ಪ್ರತಿಯೊಂದು ಅಂಗವೂ ಅದದರದೇ ಆದ ಕೆಲಸದಿಂದ ಚೇತನನ ಅಶಯಾಭಿವ್ಯಕ್ತಿಗೆ ದ್ವಾರವಾಗಿದೆ. ಕಣ್ಣು `ಮೆಣಸಿನ ಪುಡಿ ಹಾಕುತ್ತೇನೆ, ಯಜಮಾನನಿಗೆ ತಲುಪಿಸು' ಎಂದರೆ ಒಪ್ಪುವುದಿಲ್ಲ. ಅಂತೆಯೇ ವೇದಾಂಗಗಳೂ ತಮ್ಮ ತಮ್ಮದೇ ಆದ ವಿಷಯಗಳಿಂದ ವೇದಪುರುಷನಾದ ಭಗವಂತನನ್ನು ಸುತ್ತಿಕೊಂಡಿವೆ. ಆರು ಅಂಗಗಳು-ವೇದಾಂಗಗಳು ಒಂದೊಂದು ರೀತಿ ಭಗವಂತನ ಹತ್ತಿರಕ್ಕೂಯ್ಯತ್ತದೆ. ಮತ್ತೊಂದು (ವ್ಯಾಕರಣವು) ಪದ, ಸಂಧಿ ಮೊದಲಾದ ವಿಷಯಗಳಿಂದ ಚೇತನನನ್ನು ಸುತ್ತಿಕೊಂಡಿದೆ. ಶಿಕ್ಷೆ, ವ್ಯಾಕರಣ, ಸಂಗೀತ ಇವೆಲ್ಲಾ ನಾದ ಮೂಲವಾದ ಶಾಸ್ತ್ರಗಳು, ನಾದ -ಸ್ವರ- ಅಕ್ಷರವಾಗಿ ವಿಕಾಸವಾಗಿ ಬಂದಿವೆ. 

(ಅಲ್ಲಿದ್ದ ಒಬ್ಬರನ್ನು ಕುರಿತು ನೋಡಿ ಕೆಳತುಟಿಯನ್ನು ಭದ್ರವಾಗಿ ಹಿಡಿದು ``ನಮ್ಮ ತಂದೆಗೆ ಪತ್ರವನ್ನು ಬರೆಯುತ್ತೇನೆ" ಎಂದು ಹೇಳಿ ಎಂದು ಸೂಚಿಸಿದರು. ಅವರು ಬಹಳವಾಗಿ ಪ್ರಯತ್ನಿಸಿ ಆಗುವುದಿಲ್ಲವೆಂದು ತಿಳಿಸಿದರು) ಏಕೆ? ಮನಸ್ಸಿನಲ್ಲಿ ಆಶಯವಿರಬಹುದು, ಅದನ್ನು ವಿಸ್ತರಿಸಲು ಅಂಗವೂ ಸಿದ್ಧವಾಗಿರಬೇಕು. ವರ್ಣವೆಂದರೆ ವಿಸ್ತಾರ. ನಂತರ ಪದ ಸ್ವರವೆನ್ನುವ ವ್ಯವಹಾರ ವ್ಯಾಕರಣದಲ್ಲಿಯೂ ಇದೆ, ಶಿಕ್ಷೆಯಲ್ಲಿಯೂ ಇದೆ. ಆದರೆ ಅವೆರಡಕ್ಕೂ ವ್ಯವ್ಯಾಸವಿದೆ. ಇಂದ್ರಿಯಗಳು ಅದದರದೇ ಆದ ವಿಷಯವನ್ನು ಆತ್ಮನಿಗೆ ತಲುಪಿಸುವಂತೆ ಶಿಕ್ಷಾದಿಗಳೂ ತಮ್ಮ ವಿಷಯದ ಮೂಲಕ ಭಗವಂತನೆಡೆಗೆ  ಒಯ್ಯುತ್ತವೆ.

\section*{ಸಂಧಿಯ ಮರ್ಮ}

ಶಿಕ್ಷಾಶಾಸ್ತ್ರವು ಅರ್ಥವಾಗಲು ಯಾವರೀತಿಯಾದ ವಿಷಯದ ಪರಿಚಯವಿರಬೇಕು? ಜೀವವೇನು? ಇದಕ್ಕೋಸ್ಕರ ನಾವು ತಿಳಿಯಬೇಕಾದ ವಿಷಯವೇನು? ಎನ್ನುವುದನ್ನೂ ತಿಳಿದುಕೊಳ್ಳಬೇಕು. ಶಿಕ್ಷಾವ್ಯಾಸಂಗಿಗೆ ಸ್ವರಜ್ಞಾನ ಚೆನ್ನಾಗಿರಬೇಕು. ವೀಣೆಯ ಅಭ್ಯಾಸವು ಬೇಕು. ನಾದ-ಸ್ವರಗಳನ್ನು ಒಂದಕ್ಕೊಂದು ಸೇರಿಸಿ ಹೇಳುವ ಕ್ರಮದ ಅರಿವಿರಬೇಕು. ಇಲ್ಲದಿದ್ದರೆ ತಪ್ಪಾಗುತ್ತದೆ. ಉದಾಹರಣೆಗೆ ದೇವ + ಇಂದ್ರ  ಏನಾಗುತ್ತದೆ ಎಂದರೆ ದೇವೇಂದ್ರ, ಗುಣಸಂಧಿಯಾಗುತ್ತದೆ ಎನ್ನುತ್ತ್ತೀರಿ. ಆದರೆ ದೇವ ಎನ್ನುವುದನ್ನು ಮಂದ್ರಸ್ಥಾಯಿಯಲ್ಲೂ ಇಂದ್ರ ಎಂಬುದನ್ನು ಮಧ್ಯಮದಲ್ಲೂ  ಹೇಳಿದರೆ ಆಗ ದೇವೇಂದ್ರ ಎಂದು ಸಂಧಿಯಾಗುವುದಿಲ್ಲ. ಸಂಧಿಯಾಗಬೇಕಾದರೆ ಸ್ವರಗಳು ಒಂದರೊಡನೊಂದು ವಿಲೀನವಾಗಬೇಕು. ಬೇರೆ ಬೇರೆ ಸ್ಥಾಯಿಯಲ್ಲಿದ್ದಾಗ ಹಾಗೆ ವಿಲೀನವಾಗುವುದಿಲ್ಲ. ಅದೇ ಒಂದೇ ಸ್ಥಾಯಿಯ ಮಂದ್ರ ಮತ್ತು ತಾರಗಳಿಗೆ ಹೊಂದಾಣಿಕೆಯಿದೆ. ಅಲ್ಲಿ ಸ್ವರ ಒಂದರೊಡನೊಂದು ವಿಲೀನವಾಗುತ್ತದೆ. ಆಗ ಸಂಧಿಯಾಗಬಹುದು. ಸ್ವರವೆಂದರೆ `ಸ್ವತೊ ರಂಜಯತೀತಿ ಸ್ವರಃ' ತಾನಾಗಿಯೇ ಪ್ರಕಾಶಕ್ಕೆ ಬರುವುದು. ಹಿಂದಿನ ಶ್ರುತಿ ಮುಂದಿನ ಶ್ರುತಿಗೆ ಸೇರುವಂತೆ ಸ್ವರದ ಉಚ್ಚಾರಣೆಯಿರಬೇಕು. ಹಾಗಾದರೆ ಹೊಂದಾಣಿಕೆ ಇರುತ್ತದೆ. ಒಬ್ಬ ಚಿತ್ರಗಾರನೇ ಆದರೂ ಹಿಂದಿನ ಕಲತ್ {(\eng colour)} ಗೆ ಮ್ಯಾಚ್ {(\eng Match)} ಆಗುವಂತೆ ಮುಂದಿನ ಕಲರ್ ಇಡಬೇಕು. `ಗಂಗಾ + ಉದಕ' ಎನ್ನುವಾಗ ಗಂಗಾ ಎನ್ನುವ ಆಕಾರಕ್ಕೆ  ಉಕಾರವು ಸೇರಿದ ಮಾತ್ರಕ್ಕೆ  ಓಕಾರ ವಾಗುವುದಿಲ್ಲ. ಆಕಾರ ಉಕಾರಗಳು ಸೇರಿ ಓಕಾರವಾಗುವುದು ಯಾವಾಗ ಎನ್ನುವುದನ್ನು ಪರಿಶೀಲಿಸಿ ನೋಡಬೇಕು. ಸ್ವರಗಳನ್ನು ಉದಾತ್ತ ಅನುದಾತ್ತ ಸ್ವರಿತಗಳೆಂದೂ ವಿಭಾಗಿಸುವುದುಂಟು. ಹ್ರಸ್ವ, ದೀರ್ಘ , ಪ್ಲುತಗಳೆಂಬ ವಿಭಾಗವೂ ಉಂಟು. ಮಂದ್ರ, ಮಧ್ಯ, ತಾರಗಳೆಂದೂ ಸ್ವರಗಳನ್ನು  ವಿಭಾಗಿಸುವುದುಂಟು. ಈ ಎಲ್ಲದರ ಪರಿಚಯವೂ ಇದ್ದಾಗ ಯಾವ ಸ್ವರವು ಯಾವ ಸ್ವರದೊಡನೆ ಸಂಧಿಯಾಗುತ್ತದೆ- ಸೇರಿಹೋಗುತ್ತದೆ ಎಂಬ ಪರಿಚಯದ್ದಾಗ ಸಂಧಿ. ಸಂಧಿ ಎಂಬ ಪದದ ಅರ್ಥವೇ ಸೇರುವೆ. ಸ್ವರ ಮೇಳನವಾಗಿ ಒಂದೇ ಆಗಿಬಿಡಬೇಕು. ವೈದ್ಯದಲ್ಲಿಯೂ ಎಲ್ಲರಸಗಳೂ ಸೇರಿ  ಏಕರಸವಾಗುವಂತೆ ಯೋಗವಿರಬೇಕು. ಅದುಬಿಟ್ಟು ಬರೀ ದ್ರವ್ಯಗಳನ್ನು ಕೂಡಿಸಿದರೆ ಯೋಗವಾಗುವುದಿಲ್ಲ .  ದ್ರವ್ಯರಸಗಳು ಒಂದರೊಳಗೊಂದು ಸೇರಿದಾಗಲೇ -ಒಂದೇ ಆದಾಗಲೇ ಯೋಗವೇರ್ಪಡುವುದು. ಅಂತೆಯೇ ಸ್ವರ ಒಂದರಲ್ಲಿ ಒಂದು ಲೀನವಾಗಬೇಕು. ಒಂದರಲ್ಲೊಂದು ಸೇರಿದರೆ ಅಲ್ಲಿ ಸಂಧಿ-ಸಂಹಿತೆಯಾಗುತ್ತದೆ. ವ್ಯಾಕರಣದಲ್ಲಿ ಸಂಧಿಯ ವಿಷಯವನ್ನು ಕುರಿತು ಹೇಳಬೇಕಾದರೂ ಜೀವವನ್ನು ಕೈಯಲ್ಲಿ ಹಿಡಿದು ಹೇಳಬೇಕು. ಸಾಕಷ್ಟು ಎಚ್ಚರಿಕೆ ಬೇಕು. ಪ್ರಯೋಗವು ಒಪ್ಪುವಂತೆ ಉಚ್ಚಾರಣೆ ಇರಬೇಕು.

ಆ ಮತು ಇ ಎರಡೂ ಸೇರಿ ಸಂಧಿಯಾಗಬೇಕಾದರೆ ಯಾವ ತೂಕದಲ್ಲಿರ ಬೇಕು? ಯಾವ ಗಾತ್ರಕ್ಕೆ  ಬಂದಾಗ ಒಂದರೊಳಗೊಂದು ಸೇರಿ ಒಂದೇ ಆಗಿ ಬಿಡುತ್ತವೆಯೋ ಅಲ್ಲಿ ಸಂಧಿಯಾಗುತ್ತದೆ. ಸ್ವಲ್ಪ  ವ್ಯತ್ಯಾಸವಾದರೂ ಸೇರುವುದಿಲ್ಲ. ಮಣಿಯ ತೂತಿನಲ್ಲಿ  ತಂತಿಯನ್ನು  ಸೇರಿಸಿದರೆ ಬಿಗಿಯಾಗಿ ಸೇರಬೇಕು. ಕಿವಿಯ ಕಡುಕಿನ (ಕಿವಿಗಿಡುವ ಆಭರಣ) ತಿರುಪು ಅದಕ್ಕೆ  ಹೊಂದಿಕೊಂಡಿರಬೇಕು-ಸರಿಯಾಗಿ ಸೇರಬೇಕು. ಹಾಗೆಯೇ `ದೇವೇಂದ್ರ' ಎನ್ನುವಲ್ಲಿ ದೇವ+ಇಂದ್ರ ಇದನ್ನು  ಹೊಂದಿಕೊಳ್ಳುವ ಶ್ರುತಿಯಲ್ಲಿಟ್ಟು ಸರಿಯಾಗಿ ಸೇರಿಸಿ ಹೇಳಿದರೆ ಚೆನ್ನಾಗಿ ಸೇರುತ್ತದೆ. ಹಾಗೆ ಸರಿಯಾಗಿ ಹೇಳಿದಾಗ ಸೇರಿಸಿದರೂ ಚೆನ್ನಾಗಿರುತ್ತದೆ, ಬಿಡಿಸಿ ಹೇಳಿದರೂ ಚಿನ್ನಾಗಿರುತ್ತದೆ. ಅದರ ಉಚ್ಚಾರಣೇ ಸಂಧಿಯಾಗುವುದಕ್ಕೆ ಬೇಕಾದ ಕಂಡೀಷನ್ ನಲ್ಲಿರಬೇಕು. ಅದಕ್ಕೇ ಶಿಕ್ಷಾಶಾಸ್ತ್ರಕ್ಕೆ ವೀಣೆಯ ಅಭ್ಯಾಸಬೇಕು ಎಂದು ಹೇಳಿದುದು.

\section*{ಶಿಕ್ಷಾವ್ಯಾಕರಣಗಳಿಗೂ ವೀಣೆಗೂ ಏನುಸಂಬಂಧ?}

ವ್ಯಾಕರಣ ಶಾಸ್ತ್ರದ ಅಧಿದೇವತೆಯ ಕೈಯಲ್ಲಿ ವೀಣೆಯಿದೆಯೆಂದು ವ್ಯಾಕರಣಾದ ಧ್ಯಾನಶ್ಲೋಕದಲ್ಲಿ ಬರುತ್ತದೆ. ವ್ಯಾಕರಣಕ್ಕೂ ವೀಣೆಗೂ ಏನು ಸಂಬಂಧ ಎಂದು ಇತ್ತೀಚಿನವರು ತಳ್ಳಿಬಿಡಬಹುದು. ಆದರೆ ಹಾಗೆ ಕೊಟ್ಟಿರುವ ಐಡಿಯ {(\eng Idea)} ಬಹು ಸುಂದಾವಾಗಿದೆ. ಶಿಕ್ಷಾಶಾಸ್ತ್ರದಲ್ಲಿಯೂ ವರ್ಣ, ಸ್ವರ, ಮಾತ್ರೆಗಳ ವಿವೇಚನೆಗೆ ವೀಣೆಬೇಕು. ಉದಾತ್ತಾನುದಾತ್ತ ಸ್ವರಿತ ಈ ಮೂರು ಸ್ವರಗಳಲ್ಲಿ ಸಂಗೀತದ ಸಪ್ತ್ವರಗಳು ಅಡಗುತ್ತವೆ.  ಷಡ್ಜ, ಪಂಚಮ, ಮಧ್ಯಮ ಮೂರು ಸೇರಿ ಸ್ವರಿತವಾಗುತ್ತದೆ. ನಿಷಾದ ಗಾಂಧಾರ, ಸೇರಿ ಉದಾತ್ತವಾಗುತ್ತದೆ. ಋಷಭ, ಧೈವತಗಳು ಸೇರಿ ಅನುದಾತ್ತವಾಗುತ್ತದೆ.

\begin{shloka}
ಉದಾತ್ತೇ ನಿಷಾದಗಾಂಧಾರೌ ಅನುದಾತ್ತ ಋಷಭ ಧೈವತೌ |\\
ಸ್ವರಿತ ಪ್ರಭವಾಹ್ಯೇತೇ ಷಡ್ಜಮಧ್ಯಮ ಪಂಚಮಾಃ||(ಪಾ-ಶಿ-೩-೨)
\end{shloka}

\section*{ಪರಮಾತ್ಮನಿರುವ ಜಾಗದಲ್ಲಿರುವುದೇ ವೇದ}

ಸಂಗೀತದಲ್ಲಿ ವಾದಿಸ್ವರ, ಸಂವಾದಿಸ್ವರಗಳೆಂದುಂಟು. ಅವುಗಳು ಪರಸ್ಪರ ಮಾತನಾಡಿಕೊಳ್ಳುತ್ತವೆ. ಅಂದರೆ ಒಂದನ್ನು ಧ್ವನಿಮಾಡಿದರೆ ಮತ್ತೊಂದು ಪ್ರತಿ ಧ್ವನಿಮಾಡುತ್ತದೆ. ನಮ್ಮ ಸ್ವರಶಾಸ್ತ್ರ ಆತ್ಮವಿದ್ದಲ್ಲೆಲ್ಲಾ ಇದೆ. ಜ್ಞಾನವಿದ್ದಲ್ಲಿ ಇದೆ. ಅಂದರೆ ವೇದವಿದ್ದಲ್ಲಿ ಸ್ವರವಿದೆ. ಆ ವೇದವೆನ್ನುವುದು ಕೇವಲ ಸಾಹಿತ್ಯವಷ್ಟೇ ಅಲ್ಲ. ಆ ಬಗೆಗೆ ವಿವರ ನೀಡುವ ಒಂದು ಮಾತಿದೆ -

\begin{shloka}
ನ ವೇದಂ ವೇದಮಿತ್ಯಾಹುಃ ವೇದೇ ವೇದೋ ನ ವಿದ್ಯತೇ |\\
ಪರಾತ್ಮಾ ವಿಂದತೇ ಯೇನ ಸ ವೇದೋ ವೇದ ಉಚ್ಯತೇ ||
\end{shloka}

(ವೇದವೆಂದು ಹೆಸರು ಪಡೆದಿರುವ ಸಾಹಿತ್ಯರಾಶಿಮಾತ್ರವನ್ನೇ ಜ್ಞಾನಿಗಳು ವೇದವೆಂದು ಕರೆಯುವುದಿಲ್ಲ. ಕೇವಲ ಸಾಹಿತ್ಯದಲ್ಲಲ್ಲ  ವೇದವಿರುವುದು. ಯಾವುದರಿಂದ ಪರಮಾತ್ಮನ ಅನುಭವವು  ಉಂಟಾಗುವುದೋ ಅದನ್ನೇ ವೇದವೆಂದು ಹೇಳುತ್ತಾರೆ.)

ಪರಮಾತ್ಮನು ಇರುವ ಜಾಗದಲ್ಲಿರುವುದೇ ವೇದ. ಜ್ಞಾನದ ಜಾಗದಲ್ಲಿರುವುದೇ ವೇದ. ಬರೀ ಆನುಪೂರ್ವಿ (ವರ್ಣಗಳ ಕ್ರಮವಾದ ಜೋಡಣೆ)ಯೇ ವೇದವಲ್ಲ. ಆದ್ದರಿಂದ ಶಾಸ್ತ್ರವಚನಗಳೇ ಈ ಅಂಶವನ್ನು  ಘಂಟಾಘೋಷವಾಗಿ ಸಾರಿವೆ-

\begin{shloka}
ಊರ್ಧ್ವಮೂಲಮಧಶ್ಯಾಖಂ ಅಶ್ವತ್ಥಂ ಪ್ರಾಹುರವ್ಯಯಂ|\\
ಛಂದಾಂಸಿ ಯಸ್ಯ ಪರ್ಣಾನಿ ಯಸ್ತಂ ವೇದ ಸ ವೇದವಿತ್||
\end{shloka}

\begin{shloka}
ಜ್ಞಾನಿನಾಮೂರ್ಧ್ವಗೋ ಭೂಯತ್ ಅಜ್ಞಾನಿನಾಮಧೋಮೂಖಃ|\\
ಏವಂ ವೈ ಪ್ರಣವಸ್ತಿಷ್ಠೇತ್ ಯಸ್ತಂ ವೇದ ಸ ವೇದವಿತ್ ||
\end{shloka}

(ಮೇಲ್ಬೇರಾಗಿಯೂ, ಕೆಳಕೊಂಬೆಯಾಗಿಯೂ ಇರುವ ನಾಶರಹಿತವಾದ ಅಶ್ವತ್ಥ  ವೃಕ್ಷವೊಂದನ್ನು ಜ್ಞಾನಿಗಳು ಅರಿತು ಹೇಳುವರು. ಯಾವುದರ ಎಲೆಗಳೇ ವೇದಗಳಾಗಿವೆಯೋ ಅಂತಹ ವೃಕ್ಷವನ್ನು ಬಲ್ಲವನೇ `ವೇದವಿತ್' -ವೇದವನ್ನು  ಬಲ್ಲವನು.

ಜ್ಞಾನಿಗಳಲ್ಲಿ ಮೇಲ್ಮುಖವಾದ ಸಂಚಾರವುಳ್ಳದ್ದಾಗಿಯೂ, ಅಜ್ಞಾನಿಗಳಲ್ಲಿ ಕೆಳಮುಖವಾಗಿಯೂ, ಪ್ರಣವವು ಇರುವುದು. ಅಂತಹ ಪ್ರಣವವನ್ನು ಯಾರು ಬಲ್ಲವನೋ ಅವನು `ವೇದವಿತ್' - ವೇದವನ್ನು ಬಲ್ಲವನು.)

ಈ ಮಾತುಗಳೆಲ್ಲಾ ಜ್ಞಾನದ ಸ್ವರೂಪವನ್ನು ಬಲ್ಲವನನ್ನು `ವೇದವಿತ್' ಎಂದು ಸಾರುತ್ತವೆ. ಇಂದಿನ ಪರಿಸ್ಥಿತಿಯಲ್ಲಿ ನಿಜವಾದ ವೇದ (ಜ್ಞಾನ)ವನ್ನು `ವಿತ್ ಬಿಟ್ಟು' - ಮಾರಿಬಿಟ್ಟು (ವಿತ್ತ್ -ಎಂದರೆ ತಮಿಳಿನಲ್ಲಿ ಮಾರಿಬಿಟ್ಟು ಎಂದರ್ಥ) ವೇದವಿತ್ ಗಳೆಂದುಕೊಳ್ಳುತ್ತಾರೆ. ಕೇವಲ ಅಕ್ಷರರಾಶಿಯೇ ವೇದವಾಗಲಾರದು. ಒಳಜ್ಞಾನದ ಅರಿವಿನೊಡಗೂಡಿದಾಗ ಮಾತ್ರ  ಸಾಹಿತ್ಯವೂ ವೇದವಾಗಬಹುದು.

ಗೀತೆಯಲಿ ವೇದವನ್ನು ತಿಳಿದವನ ಬಗೆಗೆ ಹೇಳುವಾಗ ಈ ಮಾತನ್ನು ಹೇಳಿದೆ.
\begin{shloka}
ಯಾವಾನರ್ಥ ಉದಪಾನೇ ಸರ್ವತಃ ಸಂಪ್ಲುತೋದಕೇ|\\
ತಾವಾನ್ ಸರ್ವೇಷು ವೇದೇಷು ಬ್ರಹ್ಮಣಸ್ಯ ವಿಜಾನತಃ||
\end{shloka}

(ನೀರುತುಂಬಿದ ಕೆರೆಯೊಂದಕ್ಕೆ  ಹೋದರೂ ನಮಗೆ ಎಷ್ಟು ನೀರು ಬೇಕೋ ಅಷ್ಟನ್ನೇ ನಾವು ಪಡೆಯುವಂತೆ ಜ್ಞಾನಾಭಿಲಾಷಿಯಾದ ಬ್ರಾಹ್ಮಣನಿಗೆ ಎಲ್ಲಾವೇದಗಳೂ ಸೇರಿಯು ಅಷ್ಟೆ ಪ್ರಯೋಜನ.) ನಿಜವರಿತರೆ ವೇದದಲ್ಲಿ ಹೇಳಿರುವುದೂ ಆ ನಿಜವನ್ನೇ ಎಂಬ ಅಂಶ ಅರಿವಿಗೆ ಬರುತ್ತದೆ. ವೇದದಲ್ಲಿರುವುದು ಸ್ವತಸ್ಸಿದ್ಧವಾದ ವಿಷಯ. ನಿತ್ಯಸಿದ್ಧವಾದ ವಿಷಯವೇನಿದೆಯೋ, ಅದನ್ನು ಸಾಕ್ಷಾತ್ಕಾರ ಮಾಡಿಕೊಂಡು ಹೊರಟ ಮಾತಾಗಿದೆ ವೇದ. ಹಾಗೆ ಹೊರಟ ವಾಕ್ ತನಗೆ ಬೇಕಾದ ಅಂಗಗಳನ್ನೆಲ್ಲಾ ತನ್ನ ಜೊತೆಯಲ್ಲಿ ಹೊತ್ತುಕೊಂಡೇ ಬಂದಿದೆ. ಅದನ್ನು ಬಿಡಿಸಿ ನೋಡಿದಾಗ ಅದು ಅಲ್ಲೇ ಇರುವುದು ಕಂಡುಬರುತ್ತದೆ.

\section*{ವೇದದೊಡನೆಯೇ ಶಿಕ್ಷಾಶಾಸ್ತ್ರದ ಆವಿರ್ಭಾವ}

ಮನುಷ್ಯನು ಹುಟ್ಟುವಾಗಲೇ ಅವನ ಜೀವನನಿರ್ವಹಣೆಗೆ ಬೇಕಾದ ಅವಯವಗಳೆಲ್ಲಾ ಅವನಲ್ಲೇ ಇರುತ್ತವೆ. ಆದರೆ ಹಲವು ಗೂಢವಾಗಿರುತ್ತವೆ. ಹಲವು ಪ್ರಕಾಶವಾಗಿರುತ್ತವೆ. ಹುಟ್ಟುವ ಮಗುವಿಗೆ ಹಲ್ಲು ಬರುವುದು ಸಹಜವಾದರೂ ಅದು ಪ್ರಕಾಶಕ್ಕೆ ಬರುವುದು ಸುಮಾರು ಎಂಟು ಹತ್ತು ತಿಂಗಳ ಬಳಿಕವೇ. ಹುಟ್ಟುವಾಗಲೇ ಹಲ್ಲು ಬೆಳೆಯಲು ಬೇಕಾದ ಪ್ಲಾನ್ {(\eng Plan)} ಇರುತ್ತದೆ. ಅದರ ಯೋಜನೆ ಗೂಢವಾಗಿದೆ. ಲಂಗ್ಸ್  {(\eng Lungs)} ಹಾರ್ಟ್ {(\eng Heart)} ಇವುಗಳೆಲ್ಲಾ ಮೊದಲೇ ಬೆಳೆದಿರುತ್ತವೆ. ಶ್ವಾಸಕೋಶ, ಹೃದಯ, ಮೆದುಳು, ಬೆನ್ನುಹರಿ {(\eng Spinal cord)} ಇವುಗಳನ್ನು  ಬಿಡಿಸಿ ನೋಡಿದರೆ ಅರಿಯಬಹುದು. ಸಹಜವಾಗಿಯೇ ಬೆಳೆದಿರುವ ಈ ಅಂಗಗಳ ಕ್ರಿಯೆಗಳನ್ನು ಪರಿಶೀಲಿಸಿ ಅರಿತು `ಅನಾಟಮಿ' {(\eng Anatomy)} ಯನ್ನು ಬರೆಯಬಹುದು. ಆದರೆ ಸಹಜ ಕ್ರಿಯೆಯು ಅನಾಟಮಿ ಬರೆಯುವುದಕ್ಕೆ  ಮೊದಲೂ ಇತ್ತು, ನಂತರವೂ ಇರುತ್ತದೆ. ಬರೆಯದಿದ್ದರೂ ಕ್ರಿಯೆಗಳು ನಡೆಯುತ್ತವೆ. ಅನಾಟಮಿ ಬರೆದಂದಿನಿಂದ ಕ್ರಿಯೆಗಳು ಆರಂಭವಾದವು ಎಂದೇನೂ ಇಲ್ಲ. ಅದರ ಬಗೆಗೆ ರೆಕಾರ್ಡ್ 
{\eng(Record)} ಆಗಿದ್ದು ಇತ್ತೀಚೆಗೆ ಎನ್ನಬಹುದು ಅಷ್ಟೇ. ಅಂತೆಯೇ ವೇದ ಮುಂಚೆ ಹುಟ್ಟಿತು. ಶಿಕ್ಷಾಶಾಸ್ತ್ರ ಇತ್ತೀಚಿನದು ಎಂಬ ಮಾತಿಗೆ ಅರ್ಥವಿಲ್ಲ. ವೇದದ ಜೊತೆಯಲ್ಲಿಯೇ ಶಿಕ್ಷಾಶಾಸ್ತ್ರವೂ ಹುಟ್ಟಿಕೊಂಡಿದೆ. ಆದರೆ ಅದನ್ನು ಅರಿತು ಬೇರೆಡೆಯಲ್ಲಿ ಗುರುತುಮಾಡಿದ್ದು ಇತ್ತೀಚಿನ ಕಾಲದಲ್ಲಿರಬಹುದು.

ಆದ್ದರಿಂಡ ಶಿಕ್ಷಾಶಾಸ್ತ್ರವು ವೇದದ ಜೊತೆಯಲ್ಲೇ ಹುಟ್ಟಿದುದು. ವೇದ ಪ್ರಕಾಶಕ್ಕೆ  ಬರುವಾಗಲೇ ಅದಕ್ಕೆ ಬೇಕಾದ ಸ್ವರ-ವರ್ಣ-ಮಾತ್ರೆ-ಬಲಗಳೊಡನೆಯೇ ಪ್ರಕಾಶಕ್ಕೆ ಬರುತ್ತದೆ. ಹೊಟ್ಟೆನೋವು ಬಂದರೆ ನೋವಿಗನುಗಣವಾಗಿ ಧ್ವನಿ ಮಾಡುತ್ತಾನೆ. ಆ ಧ್ವನಿಯಲ್ಲಿ ಹೊಟ್ಟೆನೋವಿನ ಹಿಸ್ಟರಿ {(\eng History)} ಇರುತ್ತದೆ.

\section*{ಜ್ಞಾನಕ್ಕೆ ಅನುಗುಣವಾದ ಸ್ವರವಿರಬೇಕು}

ವೇದದಲ್ಲಿಯೇ ಆದರೂ ಕಾರ್ಯಂ, ವೀರ್ಯಂ, ಎಂಬವುಗಳನ್ನು ಒತ್ತಿ ಹೇಳಬೇಕು ಎನ್ನುವ ಮಾತು ಇಂದೂ ಉಳಿದಿದೆ. ಏಕೆ ಹಾಗೆ? ಎಂದರೆ ಅದು ವಿಷಯಕ್ಕನುಗುಣವಾಗಿ ಹಾಗೆ ಬೆಳೆದಿದೆ. ಅದರ ಅರ್ಥ ಸ್ಫುಟವಾಗಲು ವೀರ್ಯವತ್ತರವಾಗಲು `ವೀರ್ಯಂ' ಎನ್ನುವುದನ್ನು  ಒತ್ತಿಯೇ ಹೇಳಬೇಕು. `ಅವನು ತುಂಬಾ ಧೀರ' ಎನ್ನುವಾಗ `ಧೀರ' ಎನ್ನುವುದನ್ನು ಒತ್ತಿ ಮಹಾಪ್ರಾಣಾಕ್ಷರವನ್ನೇ ಹೇಳಬೇಕು. ಆಗ ಧೈರ್ಯ ಅರಿವಿಗೆ ಬರುತ್ತದೆ. ಧ್ವನಿ ದಪ್ಪವಾಗಿಯೇ ಇರಬೇಕು. ಅದು ಬಿಟ್ಟು ತಮಿಳಿನಲ್ಲಿ ಮಹಾಪ್ರಾಣವಿಲ್ಲ ಎಂದುಬಿಟ್ಟು `ದೀರ' ಎಂದು ಬಿಟ್ಟರೆ ವಿಷಯವು ಕೆಡುತ್ತದೆ.

ಹೀಗೆಯೇ ಮಂತ್ರಪ್ರಯೋಗದಲ್ಲಿಯೂ ಒತ್ತಿ ಹೇಳಬೇಕಾದ ಜಾಗಗಳಿವೆ. ಹೊಟ್ಟೆನೋವು ಬಂದರೆ ಅದಕ್ಕನುಗುಣವಾಗಿ ಧ್ವನಿ, ನವೆ ಬಂದರೆ ಅದಕ್ಕೆ ತಕ್ಕ ಕೆರೆತ, ಕಜ್ಜಿ  ಬಂದು ಅದನ್ನು ಉಜ್ಜುವಾಗಲೂ ಅದಕ್ಕೆ ತಕ್ಕ ಸ್ವರ - ಬಲ- ಮಾತ್ರೆ ಎಲ್ಲವೂ ಉಂಟು. ವೇಗವು ಸ್ವಲ್ಪ ಕಡಿಮೆಯಾದರೂ ಕಜ್ಜಿಯೂ ಒಪ್ಪುವುದಿಲ್ಲ. ಅಂತೆಯೇ ಜ್ಞಾನದ ವಿಷಯದಲ್ಲಿಯೂ ಸ್ವರವು ಅದಕ್ಕನುಗುಣವಾಗಿರಬೇಕು.

\section*{ಓಂ ಎಂದೆಡೆಯಲ್ಲೆಲ್ಲಾ ವೇದದ ಓಂಕಾರವಿಲ್ಲ}

ವೇದದಲ್ಲಿ ಓಂಕಾರವಿದೆ. ಹಾಗೆಯೇ ` ಓಂ ಎಂದರೆ ಟೋಂ ಎನ್ನುವುದಕ್ಕೂ ಬರುವುದಿಲ್ಲ', ಬರಬರುತ್ತಾ ಓಂಕಾರ ಹಿಡಿದುಬಿಡ್ತು, ಓಮವಾಟರ್, ಇಲ್ಲೆಲ್ಲಾ ಓಂ ಎಂಬುದಿದೆ. ಈ ಎಲ್ಲಾ ಓಂಕಾರಗಳೂ, ವೇದದ ಓಂಕಾರವೂ ಒಂದೆಯೇ? ವೇದ ಓಂಕಾರಕ್ಕೂ ಇವಕ್ಕೂ ಜೋಡಿ ಬರುವುದಿಲ್ಲ. ತೇಗು ಬರುವಾಗ ಬರುವ ಓಂಕಾರವೇ ಬೇರೆ. ಇಲ್ಲಿ ಆ ಸ್ವರಲಕ್ಷಣವಿಲ್ಲ.

\section*{ಸಮಗ್ರ ಅಂಗಗಳೊಡನೆಯೇ ಶಿಕ್ಷೆಯನ್ನು ಅರಿಯಬೇಕು}

ಶಿಕ್ಷೆಗೆ ಐದು ಅಂಗಗಳುಂಟು. ಸ್ವರ, ಕಾಲ, ಸ್ಥಾನ, ಪ್ರಯತ್ನ, ಅನುಪ್ರದಾನ ಎಂಬುದಾಗಿ. ಈ ಐದೂ ಅಂಶಗಳೂ ಬೇಕು. ವೇದಾಂಗವಾದ ಶಿಕ್ಷೆಯನ್ನು ತಿಳಿಯಲಿಚ್ಛಿಸುವವನಿಗೆ ಇಷ್ಟೂ ತರಬೇತಿ ಬೇಕು. ಶಿಕ್ಷೆ-ಎಂದರೆ `ಶಿಕ್ಷ ವಿದ್ಯೋಪಾದಾನೇ' ಎಂಬಂತೆ ವಿದ್ಯೆಯ ಸ್ವೀಕಾರ ಎಂದರ್ಥ. ಹಾಗೆ ವಿಷಯವನ್ನು ಸ್ವೀಕರಿಸದಿದ್ದರೆ ವಿಷಯಕ್ಕೆ ಶಿಕ್ಷೆಯಾಗುತ್ತದೆ.

\section*{ಯಾವುದು ನಿಜವಾದ ಶಾಸ್ತ್ರ?}

ಒಂದು ಆತ್ಮವಿರುವೆಡೆಯಲ್ಲಿ ಅವನಿಗೆ ಬೇಕಾದ ಶಾಸ್ತ್ರಗಳಿವೆ. ಆ ಶಾಸ್ತ್ರಗಳು ಸ್ವತಸ್ಸಿದ್ಧವಾಗಿಯೇ ಇವೆ. ಎಲ್ಲಿ ಆತ್ಮಗಳು ಓಡಾಡುತ್ತಿವೆಯೋ ಅಲ್ಲಿಯೇ ಶಾಸ್ತ್ರಗಳಿವೆ. ಕೇವಲ ಪುಸ್ತಕಗಳೇ ಶಾಸ್ತ್ರವೆಂದೇನೂ ಅಲ್ಲ, ಕಾಗದದ ಮೇಲೆ ಬರದಿದ್ದರೆ ಶಾಸ್ತ್ರವಲ್ಲವೆಂದೇನೂ ಅಲ್ಲ. ಶಾಸ್ತ್ರದ ಮುದ್ರಣ ಮೊದಲು ಆಗುವುದು ಮನಸ್ಸಿನಲ್ಲಿ. ಆ ಮನಸ್ಸಿನಲ್ಲಿ ಫಸ್ಟ್  ಎಡಿಷನ್ {(\eng First edition)}-ಮೊದಲ ಮುದ್ರಣವಾಗುತ್ತದೆ. ನಂತರ ಮುಖ ಮೊದಲಾದ ಅಂಗಗಳ ಮೇಲೆ ಸೆಕೆಂಡ್ ಎಡಿಷನ್ {(\eng Second edition)}-ದ್ವಿತಿಯ ಮುದ್ರಣವಾಗುತ್ತದೆ. ಅಲ್ಲಿಂದ ಮುಂದೆ ಬಂಡರೆ ತೃತೀಯ ಮುದ್ರಣ. ಈ ಎಲ್ಲಕ್ಕೂ ಆತ್ಮನ ಶನಕ್ಕನುಗುಣವಾದ ಮುದ್ರೆ ಬಿದ್ದಾಗ ಮಾತ್ರ ಅವು ಶಾಸ್ತ್ರಗಳಾಗಿ ಉಳಿಯುತ್ತವೆ. ಕೈ ದಾಟಿದಂತೆ ಬೇರೆ ವಿಷಯಗಳು ಸೇರಿದರೆ ಅವು ಶಾಸ್ತ್ರಗಳಾಗಿ ಉಳಿಯುವುದು ಕಷ್ಟ. ಈಗ ನಮ್ಮ ಕೈಗೆ ಬರುವಾಗ ಎಷ್ಟೋ ಕೈದಾಟಿ ಬಂದಿವೆ. ಆದ್ದರಿಂದ ಒರಿಜಿನಲ್ ಆದ ಅಂಶವೇನು ಎನ್ನುವುದನ್ನು ಸ್ವತಂತ್ರವಾಗಿ ವಿಷಯವಿರುವೆಡೆಯಿಂದಲೇ ತೆಗೆದುಕೊಂಡು ಶಾಸ್ತ್ರಗಳಲ್ಲೇನು ಉಳಿದಿದೆ ಎಂಬತ್ತ ದೃಷ್ಟಿ ಹಾಯಿಸಬೇಕಾಗಿದೆ.

\section*{ಸ್ವರದ ಸಹಾಯದಿಂದ ವ್ಯಂಜನೋಚ್ಚಾರಣೆಯೆಂದರೇನು?}

ಪ್ರಣವವನ್ನೇ ತೆಗೆದುಕೊಂಡರೂ ಪ್ರಣವದಲ್ಲಿ ಸ್ವರವಿದೆಯೋ ವ್ಯಂಜನವಿದೆಯೋ ಎಂದರೆ ಎರಡೂ ಇದೆ. ವ್ಯಂಜನವೆಂಡರೇನು? ಎಂದರೆ ಸ್ವರದ ಸಹಾಯದಿಂಡ ಉಚ್ಚರಿಸಬಹುದಾದ ಅಕ್ಷರ ಎಂಬ ವಿವರಣೆ ನೀಡುತ್ತಾರೆ. ಸ್ವರದ ಸಹಾಯದಿಂಡ ಎಂದರೇನು? ಎಂದರೆ ಕ ಎಂಬಲ್ಲಿ ಕ್ + ಅ ಎಂಬಲ್ಲಿ ಅ ಎಂಬಸ್ವರದ ಸಹಾಯವಿಲ್ಲದ ಅದನ್ನು ಉಚ್ಚರಿಸಲಾಗುವುದಿಲ್ಲ ಎಂದೂ ವಿವರಣೆ ನೀಡುವುದು ರೂಢಿಯಲ್ಲಿದೆ. ಆದರೆ ವ್ಯಂಜಕ್ಕೆ  ಬೇಕಾದ ಸ್ವರದ ಸಹಾಯ ಎಂತಹುದು? ಕ್ , ಖ್ , ಗ್, ಘ್, ಙ್, ಎಂದೇ ಹೇಳಿ ರೂಢಿ ಮಾಡಿಬಿಟ್ಟರೆ ಅದೂ ಉಚ್ಚಾರಣೇಗೆ ಸುಲಭವೇ ಆಗುತ್ತದೆಯಲ್ಲವೇ? ಆದ್ದರಿಂದ ಸ್ವರದ ಸಹಾಯವೆಂದರೆ ನಾವು ಭಾವಿಸುವ ರೀತಿ ಮುಂದೆ ಸ್ವರದ ಸೇರ್ಪಡೆಯಷ್ಟೇ ಅಲ್ಲ. ವ್ಯಂಜನದ ಹುಟ್ಟಿಗೂ ಮೂಲವಾದ ಪ್ರೇರಕಶಕ್ತಿಯೇ ಸ್ವರ. ಸ್ವರ ಸ್ಥಾನದಿಂದಲೇ - ಆ ಶಕ್ತಿಯ ಸಹಾಯದಿಂದಲೇ ವ್ಯಂಜನಗಳು ಹುಟ್ಟಿಕೊಳ್ಳುತ್ತವೆ. ಸ್ವರವೆಂದರೆ `ಸ್ವತೋರಾಜತೇ ರಮ್ಜಯತೀತಿಚಸ್ವರಃ' ಎಂಬಂತೆ ತಾನಾಗಿಯೇ ಪ್ರಕಾಶಗೊಳ್ಳುವುದು ಸ್ವರ. ಜೊತೆಗೆ ಇತರ ವರ್ಣಗಳನ್ನೂ ಪ್ರಕಾಶಗೊಳಿಸುವ ಪ್ರೇರಕಶಕ್ತಿಯೂ ಸ್ವರವಾಗಿದೆ. ಕ್, ಖ್, ಗ್, ಇವುಗಳೂ ಕೂಡ ಸ್ವರಸ್ಥಾನದಲ್ಲಿ ಸ್ವರಗಳ ಸಹಾಯದಿಂದಲೇ ಹುಟ್ಟಿ ಪ್ರಕಾಶಗೊಳ್ಳಬೇಕು. ಸ್ವರದಿಂದ ಇತರ ವರ್ಣಗಳಿಗೆ ಶಕ್ತಿ ಹರಿಯದಿದ್ದರೆ ಮಾತೆ ಇಲ್ಲ.

ವರ್ಣೋಚ್ಚಾರಣೆಯ ಸಂವಿಧಾನವೇ ಹಾಗಿದೆ.  ಉಚ್ಚಾರಣೆಯ ಮೂಲದಿಂದ ಶಕ್ತಿಯ ಹರಿವೆ ನಿಂತರೆ ಯಾವ ಅಷರವನ್ನೂ ಉಚ್ಚರಸಲಾಗುವುದಿಲ್ಲ್. ವಿಷಯವನ್ನು ಪ್ರಯೋಗ ರೂಪದಲ್ಲಿ ಗಮನಿಸಬಹುದು. (ಅಲ್ಲಿರುವ ಒಬ್ಬರನ್ನು ತಮ್ಮೆಡೆಗೆ ಬೆನ್ನುಮಾಡಿ ಕುಳಿತುಕೊಳ್ಳುವಂತೆ ಕುಳ್ಳಿರಿಸಿಕೊಂಡರು. ಅವರ ಕೊಬ್ಬೊಟ್ಟೆಯ ಪ್ರದೆಶದಲ್ಲಿ (ಮೂಲಾಧಾರದ ಸಮೀಪದಲ್ಲಿ) ಅದುಮಿ ಹಿಡಿದರು. ` ಈಗ ಮಾತನಾಡಿ' ಎಂದು ಅವರಿಗೆ ಸೂಚಿಸಿದರು. ಅವರು ಎಷ್ಟೇ ಪ್ರಯತ್ನಿಸಿದರೂ ಅವರಿಂದ ಮಾತೇ ಹೊರಡಲಿಲ್ಲ).  ನೋಡಿ! ಮಾತು ಎಲ್ಲಿಂದ ವಿಸ್ತಾರಗೊಳ್ಳುತ್ತದೆಯೊ ಆ ಮೂಲದಲ್ಲಿಯೇ ತಡೆಯೊಡ್ಡಿದರೆ ಒಂದು ಧ್ವನಿಯೂ ಬರದಂತೆ ಮಾಡಬಹುದು. ಮೂಲದಿಂದ ಬರುವ ಶಕ್ತಿ ಬರದಿದ್ದರೆ ಮುಂದೆ ಮಾಡುವುದೇನು? ಹೀಗೆಯೇ ಸ್ವರವೂ ವ್ಯಂಜನದ ಉತ್ಪತ್ತಿಗೇ ಪ್ರೇರಣೆ ನೀಡುವ ಶಕ್ತಿ. ಸ್ವರದ ಸಹಾಯವಿಲ್ಲದೇ ವ್ಯಂಜನಕ್ಕೆ  ಪ್ರಕಾಶವೇ ಇಲ. ಕ್-ಎಂದಾಗ ಬಳಿಕೆ `ಅ' ಎನ್ನುವುದನ್ನು ಸೇರಿಸಿಕೊಳ್ಳುವುದಲ್ಲ. ಕ್ ಎಂದು ವ್ಯಂಜನವು ರೂಪುಗೊಳ್ಳುವುದಕ್ಕೆ  ಪ್ರೇರಕವಾದ ಶಕ್ತಿಯೇ ಸ್ವರ.

\section*{ನಾದೋತ್ಪತ್ತಿ}

ಈ ಸ್ವರವೂ ನಾದದಿಂದ  ವಿಸ್ತಾರಗೊಳ್ಳುತ್ತದೆ. ನಾದ ಕೇವಲ ` ಸೌಂಡ್' {(\eng Sound)} ಅಷ್ಟೇ ಅಲ್ಲ.-

\begin{shloka}
ನಕಾರಂ ಪ್ರಾಣನಾಮಾನಮ್ ದಕಾರಮನಲಂ ವಿದುಃ |\\
ಜಾತಃ ಪ್ರಾಣಾಗ್ನಿಸಂಯೋಗಾತ್ ತೇನ ನಾದೋಽಭಿಧೀಯತೇ ||
\end{shloka}

(`ನಾದ ಎಂಬ ಎರಡು ಅಕ್ಷರಗಳಲ್ಲಿ  `ನ' ಕಾರವನ್ನು  ಪ್ರಾಣಶಕ್ತಿಯನ್ನಾಗಿಯೂ ದಕಾರವನ್ನು ಅಪಾನ (ಅನಲ) ಶಕ್ತಿಯನ್ನಾಗಿಯೂ ತಿಳಿಯುತ್ತಾರೆ. ಪ್ರಾಣ ಮತ್ತು ಅಪಾನಗಳ ಸೇರುವೆಯಿಂದ ಹುಟ್ಟಿದ ಕಾರಣದಿಂದಲೇ ನಾದ ಎಂಬುದಾಗಿ ಹೇಳಲ್ಪಡುತ್ತದೆ.) ಪ್ರಾಣಾಪಾನಗಳ ಸಂಯೋಗದಿಂದಲೇ ನಾದೋತ್ಪತ್ತಿ. ಅಪಾನ ಶಕ್ತಿಯು ಪ್ರಾಣಶಕ್ತಿಯತ್ತ ಹರಿಯದಂತೆ ತಡೆದುಬಿಟ್ಟರೆ ಪ್ರಾಣಾಪಾನಗಳ ಸಂಗಮವಾದರೂ ಎಲ್ಲಿ? ಹಾಗೆ ಸೇರದಿದ್ದಾಗ ಮುಂದೆ ಮಾತಾದರೂ ಎಲ್ಲಿ?

\section*{ಆತ್ಮಸಂಬಂಧವಿದ್ದಾಗಲೇ ಸ್ವರೋತ್ಪತ್ತಿ}

ಮಾತು ಹೇಗೆ ಹುಟ್ಟುತ್ತದೆ ಎಂಬುದನ್ನೂ  ಶಿಕ್ಷಾಶಾಸ್ತ್ರದಲ್ಲಿ ಬಹುಮನೋಹರವಾಗಿ ವಿವರಿಸಿದ್ದಾರೆ.

\begin{shloka}
ಆತ್ಮಾ ಬುದ್ಧ್ಯಾ ಸಮೇತ್ಯಾರ್ಥಾನ್ ಮನೋ ಯುಙ್ತೇ ವಿವಕ್ಷಯಾ |\\
ಮನಃ ಕಾಯಾಗಿನಮಾಹಂತಿ ಸ ಪ್ರೇರಯತಿ ಮಾರುತಮ್ |\\
ಮಾರುತಸ್ತೂರಸಿ ಚರನ್ ಮಂದ್ರಂ ಜನಯತಿ ಸ್ವರಮ್||
\end{shloka}

(ಆತ್ಮನು ಬುದ್ಧಿಯೊಡಗೂಡಿ ಮಾತನಾಡುವ ಇಚ್ಛೆಯಿಂದ ತನ್ನ ಅಭಿಪ್ರಾಯವನ್ನು ಪ್ರಕಟಿಸಲು ಮನಸ್ಸನ್ನು  ಪ್ರೇರೇಪಿಸುತ್ತಾನೆ. ಮನಸ್ಸು ಶರೀರದಲ್ಲಿ ಕೆಲಸ ಮಾಡುವ ಅಗ್ನಿಯನ್ನು ಪ್ರೇರಿಸುತ್ತದೆ.  ಆ ಅಗ್ಗಿಯಾದರೋ ಪ್ರಾಣವಾಯುವನ್ನು ಪ್ರೇರಿಸುತ್ತದೆ. ಆ ಅಗ್ನಿಯಾದರೋ ಪ್ರಾಣವಾಯುವನ್ನು ಪ್ರೇರಿಸುತ್ತದೆ. ಆ ವಾಯುವು ಹೃದಯ ದೇಶದಲ್ಲಿ ಸಮ್ರಿಸುತ್ತಾ `ಮಂದ್ರಸ್ವರ' ವನ್ನುಂಟು ಮಾಡುತ್ತದೆ) ಒಂದು ಸಣ್ಣ ಧ್ವನಿಯು ಉಚ್ಚಾರಗೊಳ್ಳಬೇಕಾದರೂ ಆತ್ಮ ಮೂಲದಿಂದಲೇ ಶಕ್ತಿ ಹರಿದು ಬರಬೇಕು. `ಇದನ್ನು ಪ್ರಾಕ್ಟಿಕಲ್ {(\eng Practical)} ಆಗಿ ತೋರಿಸುವಿರಾ?' ಎಂದರೆ ತೋರಿಸಬಹುದು. ಆತ್ಮಸಂಬಮ್ಧವಿದ್ದಾಗ ಮಾತ್ರ ಉಳಿದೆಲ್ಲಕ್ಕೂ ಕೆಲಸ. ಆತ್ಮನು ಇರುವಾಗ ಅವನ ಜೊತೆಯಲ್ಲಿಯೇ ಸ್ವತಸ್ಸಿದ್ಧವಾ ಗಿರುವುದನ್ನು  ಹೇಳಿದ್ದೇನೆ. ವಿಷಯವು ಪ್ರಯೋಗದೊಡನೆ ಬೆಳೆದಾಗ ಜೀವಂತವಾಗುತ್ತದೆ.

\section*{ಸತ್ಯದ ಸ್ವಾಮಿಯ ಬಳಿ ಶಾಸ್ತ್ರಗಳು ಬೆಳಗುತ್ತಲೇ ಇರುತ್ತವೆ.}

ಮಗುವು ಹುಟ್ಟಿದಾಗ ತಾಯಿಯ ಎದೆಯಿಂಡ ಸ್ತನ್ಯವು ತಾನೇ ತಾನಾಗಿ ಸ್ರಮಿಸುತ್ತದೆ. ಆ ಅಮೃತಧಾರೆಯಿಂದಲೇ ಮಗುವಿನ ಮುಂದಿನ ಜೀವನ. ಹಾಗೆ ನನ್ನ ಸ್ವಾಮಿಯ ಬಳಿಯಲ್ಲಿ ಶಾಸ್ತ್ರಗಳೆಲ್ಲವೂ ತಾನೇ ತಾನಾಗಿ ಬೆಳಗುತ್ತಿವೆ. ಆ ಮಾತೃಸ್ಥಾನದಲ್ಲಿ ನಿಂತು ಸ್ವಾಮಿಯು ಲೋಕದಲ್ಲಿರುವ ತನ್ನ ಕಂದಗಳಿಗೆ ಅಮೃತ ಧಾರೆಯನ್ನು ಹರಿಯಿಸುತ್ತಲೇ ಇದ್ದಾನಪ್ಪಾ. ಶಾಸ್ತ್ರಗಳು ಎಂದೆಂದಿಗೂ ಅಮರವಾಗಿವೆ. ಅದನ್ನು  ಯಾರೂ ಅಳಿಸಲಾರರು. ಯಾವ ಆಟಮ್ ಬಾಂಬ್ ಗೂ ಭಾರತಿಯ ಮಹರ್ಷಿಗಳ ಶಾಸ್ತ್ರವು ಜಗ್ಗುವುದಿಲ್ಲ. ಪುಸ್ತಕಗಳನ್ನು ಹರಿದು ಹಾಳು ಮಾಡಿದರೆ ಶಾಸ್ತ್ರವು ಹೋಗುವುದಿಲ್ಲ.  ಎಂಬ್ರಿಯಾಲಜಿ {(\eng Embryology)} ಹರಿದರೆ ಸೃಷ್ಟಿಯಲ್ಲಿರುವ ಗರ್ಭಕೋಶವೇ ಹೋಗಿಬಿಡುವುದಿಲ್ಲ. ಗ್ಲೋಬನ್ನು {(\eng Globe)} ಭೂಮಿಯಾಕಾರದ ಗೋಳವನ್ನು ಒಡೆದರೆ ಭೂಮಿಯೇ ಹಾಳಾಗುವುದಿಲ್ಲ. ಭೂಮಿಯ ಬಗೆಗೆ ನಾವರಿಯಲು ಇದ್ದ ಸಾಧನವನ್ನು ನಾವು ಹಾಳುಮಾಡಿಕೊಂಡಂತಾಯಿತು ಅಷ್ಟೇ. ಅಂತೆಯೇ ಶಾಸ್ತ್ರ ಪುಸ್ತಕಗಳನ್ನು  ಹರಿದರೆ ನಮ್ಮ ಅರಿವಿಗೆ ಹಾನಿತಂದುಕೊಂಡಂತಾಗುತ್ತದೆ ಅಷ್ಟೇ. ನಿಜವಾದ ವಿಷಯಗಳು ಸಹಜವಾಗಿಯೇ ಇರುತ್ತವೆ.  ನಿಜವಾಸ ಶಾಸ್ತ್ರಗಳು ಸತ್ಯದ ಸ್ವಾಮಿಯ ಬಳಿ ಎಂದೆಂದಿಗೂ ಬೆಳಗುತ್ತಲೇ ಇರುತ್ತವೆ.ನನ್ನ ಸ್ವಾಮಿಯ ಬಳಿ ನಿತ್ಯವಾಗಿ ಶಾಸ್ತ್ರಗಳು ಬೆಳಗುತ್ತಲೇ ಇವೆ. ಅದನ್ನು ತನ್ನ ಜ್ಞಾನಪುತರಿಗೋಸ್ಕಾ ಭವಂತನು ಆಗಾಗ್ಗೆ ಹರಿಸುತ್ತಾನೆ. ಅವನಿಗಾಗಿ ಹೊರಟ ವೇದ-ಉಪವೇದ- ಕಲೆಗಳು ಎಲ್ಲವೂ ಲೋಕರಕ್ಷಣೆಗಾಗಿ ಇವೆ. `ಇಲ್ಲಿಯಲ್ಲದೆ ಬೇರೆಲ್ಲಿಯೂ ಈ ಪ್ರಪಂಚದಲ್ಲಿ ವಿಷಯ ದೊರೆಯುವುದಿಲ್ಲ' ಎಂದು ನೀವು ಹೇಳಿದ ಮಾತನು ಈ ಅರ್ಥದಲ್ಲಿ ತೆಗೆದುಕೊಂಡಿದ್ದೇನೆ. ಅದು ಕೇವಲ ಅಭಿಮಾನದಿಂದ ತೀರ್ಮಾನವಾಗುವ ವಿಷಯವಲ್ಲ. ವಿಷಯವು ಯಾವ ಸ್ವಾಮಿಯಲ್ಲಿದೆಯೋ ಅಲ್ಲಿಂದಲೇ ಆ ವಿಷಯವು ಬೆಳೆದು ಪ್ರಪಂಚಿತವಾಗಬೇಕಾಗಿದೆ. ವಿಸ್ತಾರಗೊಳ್ಳಬೇಕಾಗಿದೆ. ಅವನೇ ಕೊಡದಿದ್ದರೆ ಯಾರಿಗೂ ಇಲ್ಲ. ಮರ್ಮ್ವನ್ನು ಬಿಟ್ಟುಕೊಟ್ಟು ನಿನ್ನಂತರಂಗವೇನು? ಎನ್ನುವುದನ್ನು ಬಹಿರಂಗಪಡಿಸಿಕೊಳ್ಳುವಂತೆ ಅವನನ್ನೇ ಪ್ರಾರ್ಥಿಸಬೇಕಾಗಿದೆ. ವಿಷಯವು ಎಲ್ಲಿಂದ ಬೆಳೆದಿದೆಯೋ ಅಲ್ಲಿಂದ ಬೆಳೆದು ಅಲ್ಲಿಯೇ ನಿಲ್ಲುವಂತಾದರೆ ಅದು ಮೌಲಿಕವಾಗಿ ಉಳಿಯುತ್ತದೆ. ಹಾಗೆ ಬೆಳೆದಾಗ ತಾನೇ ಶಾಸ್ತ್ರವು ಸಫಲವಾಗುತ್ತದೆ. ಇದು ಶಿಕ್ಷಾಶಾಸ್ತ್ರಕ್ಕೆ ಮಾತ್ರವಲ್ಲ, ಎಲ್ಲ ಶಾಸ್ತ್ರಗಳಿಗೂ ಇದೇ ಮಾತು. 

\section*{ಆತ್ಮಮೂಲವಾದ ಶಾಸ್ತ್ರಗಳಿಗೆ ಎಂದಿಗೂ ಚ್ಯುತಿಯಿಲ್ಲ}

ಹಿಮರಾಶಿಗಳು ಸ್ರವಿಸಿ ತನಗೆ ತಾನೇ ಬಂದ ಧಾರೆಯಿಂದ ಗಂಗಾ, ಯಮುನಾ, ಬ್ರಹ್ಮಪುತ್ರಾ, ಸರಸ್ವತೀ, ಶತದ್ರು ಸರ್ಯೂ ಇತ್ಯಾದಿ ಹೆಸರುಗಳಿಂದ ನದಿಗಳು ತಮ್ಮಲ್ಲೇ ನೀರು ಇಟ್ಟುಕೊಂಡ್ ಹರಿಯುತ್ತವೆ. ಮಳೆಯಿಲ್ಲದಿದ್ದರೂ ಹರಿಯುತ್ತವೆ. ಮಳೆಯಿದ್ದರೆ ಮಳೆಯ ನೀರಿನಿಂಡ ತುಂಬಿ ಹರಿಯುತ್ತವೆ. ಹಾಗಿಲ್ಲದಿದ್ದಾಗ ನೀರನ್ನು ಘನೀಭೂತವಾಗಿಟ್ಟುಕೊಂಡು ಕಾಲ ಬಂದಾಗ ಕರಗಿ ಹರಿಯುತ್ತೆವೆ. ಹೀಗೆ ಆತ್ಮದಲ್ಲಿಯೇ ವೇದ-ವೇದ್ಯ-ವಿಧಿ ಎಲ್ಲವೂ ಘನೀಭೂತವಾಗಿದೆಯಪ್ಪಾ.,ಅದು ಕರಗಿ ಹರಿದಾಗ ಮಹಾನ್ದಿಯಾಗುತ್ತದೆ. ಆತ್ಮನ ಕಡೆಯಿಂದ ಹರಿದು ಬರುವಾಗ ಆ ಶಾಸ್ತ್ರಜಲಕ್ಕೆ ಎಂದು ಚ್ಯುತಿಯಿಲ್ಲ.

\section*{`ಅಂತರ್ವಾಣಿ' ಪದದ ವಿವರಣೆ}

ಶಾಸ್ತ್ರ ಬಲ್ಲವನಿಗೆ `ಅಂತರ್ವಾಣಿ' ಎಮ್ಬ ಹೆಸರೂ ಉಂಟು. ಅಂತರ್ವಾಹಿನಿಯಾಗಿ ತಾನೇ ಹರಿದು ಒಳಗೊಳಗೋ ಉಕ್ಕುವ ಆ ಭಾವವನ್ನು ಹೊರಹೊಮ್ಮಿಸಲು ವಾಣಿಯನ್ನು ಉಪಯೋಗಿಸುವವನು ಅಂತರ್ವಾಣಿ. ದೈವೀಹಿನ್ನೆಲೆಯಲ್ಲಿದ್ದ ಧರ್ಮವು ಧರೆಗೆ ಯರಿಯುವುದಾದರೆ ಅಧು ಧರೆಯನ್ನೂ ಪಾವನಮಾಡುತ್ತದೆ. ಹಾಗೆ ದೈವೀ ಹಿನ್ನೆಲೆಯಿಂದಲೇ ಆದಿಕಾವ್ಯವಾಹಿನಿ ಹರಿದಿರುವುದನ್ನೂ ನಾವು ಗಮನಿಸಬಹುದು.

\section*{ದೈವೀಹಿನ್ನೆಲೆಯಲ್ಲಿ ಅಂತರ್ವಾಹಿನಿಯಾಗಿ ಹರಿದಿದೆ ಆದಿಕಾವ್ಯ}

ಆದಿಕಾವ್ಯವಾದ ರಾಮಾಯಣವು ಹೊರಹೊಮ್ಮಿದುದು ದೈವೀ ಹಿನ್ನೆಲೆಯಿಂದಲೇ ಎಂಬುದನ್ನು ಮರೆಯಬಾರದು. ತಮಸಾತೀರದಲ್ಲಿ, ನಾರದರು ಹೇಳಿದ ರಾಮಕಥೆಯಿಂದ  ಆರ್ದ್ರವಾದ ಮನಸ್ಸಿನಿಂದ ಕೂಡಿ ವಾಲ್ಮೀಕಿಗಳು ಸಂಚರಿಸುತ್ತಿದ್ದರು. ಕ್ರೌಂಚಪಕ್ಷಿಗಳ ಜೋಡಿಯಿಂದ ಬೇಡನು ಹೊಡೆದ ಗಂಡುಪಕ್ಷಿಯು ಉರುಳಿದುದೂ, ಹೆಣ್ಣುಪಕ್ಷಿಯ ಕರುಣಾರ್ದ್ರವಾಗಿ ಧ್ವನಿಮಡಿದುದೂ ಮಹರ್ಷಿಯ ಮನವನ್ನು ಕಲಕಿತು. ಜೊತೆಗೂಡಿದ ಹಕ್ಕಿಗಳಲ್ಲಿ ಒಂದನ್ನು ಹೊಡೆದ ಬೇಡನ ಅಕೃತ್ಯದಿಂದ ಧರ್ಮಭಾವಭರಿತವಾದ ಮಹರ್ಷಿಯ ಮನಸ್ಸು ಕಲಕಿ ಅವರಿಗೆ ಅರಿವಿಲ್ಲದಂತೆಯೇ ಒಂದು ವಾಣಿ ಅವರಿಂದ ಹೊಮ್ಮಿತು-
\begin{shloka}
ಮಾ ನಿಷಾದ ಪ್ರತಿಷ್ಠಾಂ ತ್ವಮಗಮಶ್ಶಾಶ್ವತೀಸ್ಸಮಾಃ |\\
ಯತ್ಕ್ರೌಂಚಮಿಥುನಾದೇಕಮವಧೀಃ ಕಾಮಮೋಹಿತಮ್||
\end{shloka}

ಎಲೈ ಬೇಡನೇ, (ಕಾಮಮೋಹಿತವಾದ ಕ್ರೌಂಚಪಕ್ಷಿಗಳ ಜೋಡಿಯಿಂದ ಒಂದು ಹಕ್ಕಿಯನ್ನು ಹೊಡೆದು ಕೊಂದದ್ದರಿಂದ ನೀನು ಬಹುಕಾಲ ಇರುವಿಕೆಯನ್ನು ಹೊಂದಬೇಡ.) ಈ ಮಾತು ಅವರ ಬುದ್ಧಿಪೂರ್ವಕವಾಗಿ ಹೊರಟದ್ದಲ್ಲ. ತಾನಾಗಿಯೇ ಹೊರಹೊಮ್ಮಿದುದು. ಅವರ ಪ್ರಯತ್ನಕ್ಕಿಂತಲೂ ಬೇರೆಯಾದ ಆಳವಾದ ಪ್ರೇರಣೇ ಇಲ್ಲಿದೆ. ಅವರ ಬುದ್ಧಿಗೋಚರವಾಗಿ ಆ ಮಾತು ಬರದಿದ್ದುದರಿಂದಲೇ ಅವರು ಪದೇ ಪದೇ ``ಕಿಮಿದಂ ವ್ಯಾಹೃತಂ ಮಯಾ" ನಾನು ಇದೇನು ಮಾತನ್ನು  ಹೇಳಿದೆ? ಎಂದುಕೊಳ್ಳುತ್ತಾರೆ. ಏಕೆಂದರೆ ಅವರು ಶ್ಲೋಕರಚನೆಗಾಗಿ ಯಾವ ಸಿದ್ಧತೆಯನ್ನೂ ಮಾಡಿಕೊಂಡಿರಲಿಲ್ಲ. ಹಾಗಿದಪಕ್ಷೇ ``ಕಿಮಿದಂ ವ್ಯಾಹೃತಮ್ ಮಯಾ" ಎಮ್ದುಕೊಳ್ಳುತ್ತಿರಲಿಲ್ಲ. ತಾನಾಗಿಯೇ ಮೂಡಿಬಂದಿದೆ ಈ ಶ್ಲೋಕ. ನಂತರ ವಿಚಾರ ಹೊರಟಿತು. ಇದೇನು ಹೀಗೆ ಬಂತಲ್ಲಾ? ಅಲ್ಲಿ ರಗಣ ಬಂದಿತೇ, ಭಗಣ ಬಂದಿತೇ, ಲಘು  ಬಂದಿತೇ, ಗುರು ಬಂದಿತೇ ಯಾವ ಯೋಚನೆಯು ಇಲ್ಲ. ತಾನಾಗಿಯೇ ಹೊರಹರಿದ ವಾಣಿಯನ್ನು ವಿಚಾರಿಸಿ ನೋಡಿದರೆ ಅದು ಛಂದೋಬದ್ದವಾಗಿಯೇ ಹರಿದಿದೆ. ಅದನ್ನೂ ಹೀಗೆ ಹೇಳುತ್ತಾರೆ.-

\begin{shloka}
ಪಾದಬದ್ದೋಽಕ್ಷರಸಮಃ ತಂತ್ರೀಲಯಸಮನ್ವಿತಃ |\\
ಶೋಕಾರ್ತಸ್ಯ ಪ್ರವೃತ್ತೋ ಮೇ ಶ್ಲೋಕೋ ಭವತು ನಾನ್ಯಥಾ||
\end{shloka}

(ಹೊರಹರಿದ ವಾಣಿಯನ್ನು ನೋಡಿದರೆ ಪಾದಬದ್ಧವಾಗಿದೆ. ಸಮವಾದ ಅಕ್ಷರ ವಿನ್ಯಾಸವಿದೆ. ತಂತ್ರೀಲಯಗಳಿಗೂ ಹೊಂದಿಕೊಳ್ಳುವ ನೆಡೆಯಿದೆ. ಶೋಕಾರ್ತನಾದ ನನ್ನಿಂದ ಹೊರಟ ಈ ಮಾತು ಶ್ಲೋಕವೇ ಆಗಲಿ, ಬೇರೆಯಾಗದಿರಲಿ) ಇದು ವಾಣಿಯನ್ನು  ಪರಿಶೀಲಿಸಿದ ಮೇಲೆ ಹೇಳಿದ ಮಾತಾತಿದೆಯೇ ಹೊರತು, ಮೊದಲು ಈ ಯಾವ ಯೋಚನೆಯೂ ಇರಲಿಲ್ಲ. ಅದು ಅವರ ಮನಸ್ಸಿನಿಂದ, ಬುದ್ಧಿಯಿಂದ ಹೊರಬಂದ ವಾಣಿಯಲ್ಲ. ಆತ್ಮಮೂಲದಿಂದ, ಆತ್ಮಚ್ಛಂದದಿಂದ ಹೊಮ್ಮಿದ ವಾಣಿ. ಆಶ್ರಮಕ್ಕೆ ಹಿಂದಿರುಗಿದಾಗ ಬ್ರಹ್ಮನೇ ಬೇರವಾಗಿ ಬಂದರೂ ಅವನ ಸತ್ಕಾರದ ಜೊತೆಗೆ ಈ ಮಾತಿನ ಮಥನವೂ ನಡೆದಿದೆ. ತನ್ನ ಅರಿವಿಲ್ಲದೇ ವಾಣಿ ಹೊರಬಮ್ದದ್ದು `ಎಲ್ಲಿಂದ? ಹೇಗೆ?' ಎಂಬ ಸಮಸ್ಸೆ ಬಗೆಹರಿಯಲಿಲ್ಲ. ಕೊನೆಗೆ ಬ್ರಹ್ಮನೇ ಈ ಗಂಟು ಬಿಚ್ಚಬೇಕಾಯಿತು.
\begin{shloka}
`ಮಚ್ಛಂದಾದೇವ ತೇ ಬ್ರಹ್ಮನ್ ಪ್ರವೃತ್ತೇಯಮ್ ಸರಸ್ವತೀ'|
\end{shloka}

(ನನ್ನ ಛಂದ (ಸಂಕಲ್ಪ-ಪ್ರೇರಣೆ) ದಿಂದಲೇ ಈ ವಾಣಿಯು ನಿನ್ನಿಂದ ಹೊರ ಹೊಮ್ಮಿತು. ನಿನ್ನ ಮಾತು ಶ್ಲೋಕವಾಗಿಯೇ ಬಂದಿದೆ. ಇದರಲ್ಲಿ ವಿಚಾರಿಸಬೇಕಾದ ಅಂಶವೇನೂ ಇಲ್ಲ) ಎಂದು ಬ್ರಹ್ಮದೇವನೇ ಆಶ್ವಾಸನೆ ಕೊಡುತ್ತಾನೆ. ಆದ್ದರಿಂದ ಈ ಮಾತು ಬಂದುದು ಬ್ರಹ್ಮಚ್ಛಂದದಿಂದಲೇ. `ಹಿನ್ನೆಲೆಯಲ್ಲಿರುವ ಚೇಷ್ಟೆ  ನ್ನದಾಗಿದೆ' ಎಂದು ಬ್ರಹ್ಮನೇ ಒಪ್ಪಿಕೊಳ್ಳುತ್ತಾನೆ. ಹೀಗೆ ಕಾವ್ಯ ಇಲ್ಲಿ ಹರಿದುದು ಅಂತರ್ವಾಹಿನಿಯಾಗಿ. ಕಾವ್ಯಕ್ಕೆ ಹಿನ್ನೆಲೆಯಾಗಿ ಜೀವ ಕೂಡಿಕೊಂಡಾಗ ಜೀವಧಮನಿಯಲ್ಲಿ- ಜೀವನಾಡಿಯಲ್ಲಿ ಕಾವ್ಯ ಹರಿಯಿತು. ಆದ್ದರಿಂದಲೇ ಅದು ಅಂತರ್ವಾಹಿನಿಯಾಗಿ ಹೊಮಿದ ಆದಿ ಕಾವ್ಯ. ಬ್ರಹ್ಮನ ಪ್ರೇರಣೆ, ಸಂಕಲ್ಪ, ವರಗಳನ್ನು ಹೊತ್ತು ಹರಿದ ಕಾವ್ಯ. 

\section*{ವಾಗರ್ಥಗಳ ಸಂಬಂಧ}

ಬೇರಿನ ಮೂಲದಿಂದ ವೃಕ್ಷಕ್ಕೆ ಆಹರ ಸಪ್ಲೈ {(\eng Supply)} ಆಗುವಂತೆ ಶಾಸ್ತ್ರಾವು ಸ್ವಾಮಿಯಿಂದ ಹೊರಬಿದ್ದರೇನೇ ನಿತ್ಯವೂ ಆತ್ಮಸಿದ್ದವೂ ಸಹಜಸಿದ್ಧವೂ ಆಗುತ್ತದೆ. ಆ ಅರ್ಥವನ್ನು ಹಿಂದೆ ಇಟ್ಟುಕೊಂಡು ವಾಕ್ಕು ಹೊರಟಾಗ ಅದು ಅರ್ಥದ ಕಡೆಗೇ ಕೈದೋರಿ ಅಲ್ಲಿಯೇ ನಿಲ್ಲಿಸುತ್ತದೆ-ಮಾವಿನಹಣ್ಣು ಎಂಬ ಮಾತು ಮಾವಿನ ಹಣ್ಣಿನ ಕಡೆಗೇ ಕೈದೋರುವಂತೆ. ಇದರ ಗುಟ್ಟನ್ನರಿತ ಕಾಳಿದಾಸನು ತನ್ನ ಕಾವ್ಯದ ಆದಿಯಲ್ಲಿ ಹೀಗೆ ಹೇಳುತ್ತಾನೆ.

\begin{shloka}
ವಾಗರ್ಥಾವಿವ ಸಂಪೃಕ್ತೌ ವಾಗರ್ಥಪ್ರತಿಪತ್ತಯೇ |\\
ಜಗತಃ ಪಿತರೌ ವಂದೇ ಪಾರ್ವತೀಪರಮೇಶ್ವರೌ||
\end{shloka}

(ವಾಕ್ ಮತ್ತು ಅರ್ಥಗಳಂತೆ ಒಂದಾಗಿ ಸೇರಿರುವ ಜಗತ್ತಿನ ತಾಯಿ ತಂದೆಗಳಾದ ಪಾರ್ವತೀಪರಮೇಶ್ವರರನ್ನು ವಾಗರ್ಥಗಳನ್ನು ಪಡೆಯುವುದಕ್ಕಾಗಿ ವಂದಿಸುತ್ತೇನೆ.) ಎಂಬುದಾಗಿ ಜಗನ್ಮಾತಾ-ಪಿತೃಗಳನ್ನು ಒಂದೆಡೆಯಲ್ಲಿಟ್ಟು, ಅರ್ಥ-ವಾಕ್ಕುಗಳಂತೆ `ದ್ಯೌಃಪಿತಾ ಪೃಥಿವೀ ಮಾತಾ' (ದ್ಯುಲೋಕವೇ ತಂದೆ, ಭೂಮಿಯೇ ತಾಯಿ) ಎಂಬಂತೆ ದ್ಯಾವಾ-ಪೃಥಿವಿಗಳನ್ನು ಕವಿ ಹೇಳಿದ್ದಾನೆ.

ಭೂಮಿಯನ್ನು ದಿವಿಯ ಮಟ್ಟಕ್ಕೆ  ಏರಿಸುವ, ಅಂತೆಯೇ ದಿವಿಯನ್ನು ಭುವಿಗೆ ಸೇರಿಸುವ ಸುಂದರ ಮೇಳನವನ್ನು ಕವಿ ಹೇಲಿದ್ದಾನೆ. ಉಪಮಾನವನ್ನು ಕೊಡುವಾಗ ಕವಿಯ ನೋಟ ದಿವಿಭುವಿಗಳರೆಡರತ್ತಲೂ ಹರಿದು ಅವನ ಹೋಲಿಕೆಗಳು ದಿವಿಯಿಮ್ದ ಭುವಿಯತ್ತ, ಭುವಿಯಿಂಡ ದಿವಿಯತ್ತ ಓಡಾಡಿಸುವಂತಿದೆ. ದಿವಿಭುವಿಗಳೆರಡಕ್ಕೂ ಹೊಂದುವ ಉಪಮಾನಗಳಿಂಡ ಭುವಿ-ದಿವಿಗಳಿಗೆ ಸೋಪಾನವನ್ನು  ನಿರ್ಮಿಸಿ ಓಡಾಟವನ್ನು ಸುಗಮಗೊಲಿಸಿದ್ದಾನೆ.

\section*{ವೈಖರಿಯು ಪರಾ, ಪಶ್ಯಂತೀ, ಮಧ್ಯಮಾಗಳೊಡನೆ ಸೇರಿದರೆ ಪೂರ್ಣವಾಗುತ್ತದೆ}

ವಾಕ್ಕಿನ ಪ್ರಪಂಚ ನಾವು ಭಾವಿಸುವಮ್ತೆ ಕೇವಲ ಭೌತಿಕವ್ಯವಹಾರಕ್ಕಷ್ಟೇ ಸೀಮಿತವಾದುದಲ್ಲ. ಆಳವಾಗಿ ಪರಿಶೀಲಿಸಿ ನೋಡಬೇಕಾದ ಅಂಶವಿದೆ. ತಪಸ್ಯೆಯಿಂದ ನೋಡಿ ಪರಿಶೀಲಿಸಿದಾಗ ಅದು ಬಹುವಿಸ್ತಾರವಾದುದು ಎಂಬಂಶವು ಅರಿವಿಗೆ ಬರುತ್ತದೆ. ಮಾತಿನ ಬಗೆಗೆ ವಿವರಿಸುವ ಮಂತ್ರವೊಂದು ಹೀಗಿದೆ.-

\begin{shloka}
ಚತ್ವಾರಿ ವಾಕ್ ಪರಿಮಿತಾ ಪದಾನಿ|\\
ತಾನಿ ವಿದುರ್ಬ್ರಾಹ್ಮಣಾ ಯೇ ಮನೀಷಿಣಃ||
\end{shloka}

\begin{shloka}
ಗುಹಾ ತ್ರೀಣಿ ನಿಹಿತಾ ನೇಂಗಯಂತಿ |\\
ತುರೀಯಂ ವಾಚೋ ಮನುಷ್ಯಾ ವದಂತಿ||
\end{shloka}

(ಮಾತು ನಾಲ್ಕು ಪಾದಗಳನ್ನುಳ್ಳದ್ದಾಗಿದೆ. ಆ ನಾಲ್ಕು  ಪಾದಗಳನ್ನೂ, ಮನಸ್ಸನ್ನು ಆಳಬಲ್ಲ ಮನೀಸಿಗಳಾದ ಬ್ರಾಹ್ಮಣರು (ಜ್ಞಾನಿಗಳು) ಅರಿತಿರುವರು. ಆ ಪೈಕಿ ಮೂರುಪಾದಗಳು ಹೊರಗೆ ಪ್ರಕಟವಾಗುವುದಿಲ್ಲ. ಅವು ಆಳವಾದ ತಪಸ್ಸೆಯಿಂದ ಹೃದಯಗುಹೆಯಲ್ಲಿ ಸಾಕ್ಷಾತ್ಕರಿಸಿಕೊಳ್ಳಬೇಕಾದ ಸತ್ಯಗಳಾಗಿವೆ. ಪರಾ, ಪಶ್ಯಂತೀ, ಮಧ್ಯಮಾ, ವೈಖರೀ ಎಂಬ ಅದರ ನಾಲ್ಕು  ಪಾದಗಳಲ್ಲಿ ಮೊದಲ ಮೂರೂ ತಪಸ್ಯಾ ಗೋಚರಗಳೇ ಹೊರತು ಹೊರರೂಪದಲ್ಲಿ ಪ್ರಕಟಗೊಳ್ಳುವುದಿಲ್ಲ. ನಾಲ್ಕನೆಯದಾದ ವೈಖರೀ ಎಂಬುದನ್ನೇ ಮನುಷ್ಯರು ಆಡುತ್ತಾರೆ.)

ಒಂದು ಗಿಡದಲ್ಲಿ ತಾಯಿಬೇರು, ಪಾಸೆಬೇರು, ಬುಡ, ಅದರಿಂದ ಮುಂದೆ ಹೊರಗೆ ಕಾಣುವಂತೆ ಕಾಂಡ ಎಲೆ ಇತ್ಯಾದಿಗಳನ್ನು ನೋಡುತ್ತೇವೆ. ಹೊರಗಿನದನ್ನು ಮಾತ್ರ ಉಳಿಸಿಕೊಂಡು, ಒಳಗಿನದನ್ನು ಕತ್ತರಿಸಿದರೆ ಹೊರ ಬೆಳವಣಿಗೆಗೂ ವಿಷಯವಿಲ್ಲ ದಂತಾಗಿಬಿಡುತ್ತದೆ. ಒಳಗೆ ಇರುವ ಬೇರಿನ ಬಲದ ಮೇಲೆ ಹೊರಗಿನದೂ ನಿಂತಿದೆ. ಹಿಂದಿನ ಮೂರುಪಾದಗಳಿಂದ ಶಕ್ತಿ ಹರಿಯುವುದು ನಿಂತರೆ ಮುಂದಿನ ಪಾದಕ್ಕೂ  ವಿಷಯವಿಲ್ಲ.

ಹದಿನಾರನೆಯ ಒಂದು {(\eng 1/16)} ಎನ್ನುವಾಗ, ಒಂದು ಪೂರ್ಣ ಸಂಖ್ಯೆ ಮೇಲೆ ಇದೆ. ಕೆಳಗೆ ಅದನ್ನು ಎಷ್ಟು ಭಾಗವಾಗಿ ಮಾಡಿದೆ ಎಂಬ ಚಿಲ್ಲರೆ ಇದೆ. ಈ ಚಿಕ್ಕರೆ ಎಲ್ಲವೂ ಆ ಒಂದರೊಡನೆಯೇ ಸೇರಿಬಿಟ್ಟರೆ, ಆ ಚಿಲ್ಲರೆಗಳೂ ಪೂರ್ಣವಾಗಿ ಬಿಡುತ್ತವೆ. ಹಾಗೆ ಸೇರದೇ ಬಿಡಿಯಾಗಿದ್ದರೆ ಚಿಲ್ಲರೆ-ಚಿಲ್ಲರೆಯಾಗಿಯೆ ಉಳಿದು ಬಿಡುತ್ತದೆ, ಪೂರ್ಣಗೊಳ್ಳುವುದಿಲ್ಲ. ಒಂದು ಇದ್ದರೆ ನಾಲ್ಕನೇ ಒಂದು ಆಗುವುದು ಸಾಧ್ಯ. ನಾಲ್ಕು ಅಂಶಗಳೂ ಒಂದರಲ್ಲಿ ಸೇರಿದಾಗ ಪೂರ್ಣವಾಗುತ್ತದೆ. ಹೀಗೆಯೇ ವೈಖರಿಯು ಪರಾ, ಪಶ್ಯಂತೀ, ಮಧ್ಯಮಾಗಳೊಡನೆ ನೆಲೆ ಸೇರಿದರೆ ಪೂರ್ಣವಾಗುತ್ತದೆ.
\begin{shloka}
``ಪೂರ್ಣಸ್ಯ ಪೂರ್ಣಮಾದಯ ಪೂರ್ಣಮೇವಾವಶಿಷ್ಯತೇ"
\end{shloka}

ಎಂಬಂತೆ ಅಂಶವು ಅಂಶಿಯಲ್ಲಿ ಸೇರಿದರೆ ಪೂರ್ಣವಾಗುತ್ತದೆ. ಹಾಗೆ ಪೂರ್ಣದೊಂದಿಗೆ ಸೇರಿದಾಗಲೇ ಅಂಶಕ್ಕೆ ಬೆಲೆ. ಹಾಗೆ ಸೇರಿದಾಗ ಅಲ್ಲಿ ಬದಲಾವಣೆಯಿಲ್ಲ, ಚ್ಯುತಿಯಿಲ್ಲ. ಆದ್ದರಿಂದ  ಮೂಲದೊಡನೆ ತಡೆಯಿಲ್ಲದ ಸಂಬಂಧವಿಟ್ಟುಕೊಂಡು ಬೆಳೆದಾಗ ಶಾಸ್ತ್ರವು ಸಜೀವವಾಗಿರುತ್ತದೆ. ಅಂತೆಯೇ ವೇದಾಂಗವಾದ ಶಿಕ್ಷಾಶಾಸ್ತ್ರವೂ ವೇದಪುರುಷನ-ವೇದಮೂಲಜ್ಯೋತಿಯ ನಿತ್ಯಸಂಬಂಧದೊಡ ಗೂಡಿದ್ದಾಗ  ಅದಕ್ಕೆ  ಚ್ಯುತಿಯಿಲ್ಲ.

\section*{ಶಿಕ್ಷೆಯ ಆತ್ಮಮೂಲದಿಂದ ವಿಕಾಸವಾಗಿದೆಯೆಂಬ ಬಗ್ಗೆ ಶ್ರೀಗುರುವಿನ ಪ್ರಯೋಗ}

(ಶಿಕ್ಷೆ ಆತ್ಮ ಮೂಲದಿಂದ ವಿಕಾರವಾಗಿದೆ ಎಂಬಂಶವನ್ನು ಕೈಗರಿಯಾರವೊಂದರ ಉದಾಹರಣೆಯಿಂದ ಮನಮುಟ್ಟುವಂತೆ ಶ್ರೀ ಗುರುವು ವಿವರಿಸುತ್ತಾರೆ. ಅಲ್ಲಿದ್ದ ಶ್ರೀ ಸತ್ಯಧರ್ಮಾಚಾರ್ಯರು  ಎಂಬವರಿಂದ ಅವರ ಕೈಗಡಿಯಾರವನ್ನು ಕೇಳಿ ಪಡೆದು `ಇದಕ್ಕೆ ಶಿಕ್ಷೆ ಹಾಗಿದ್ದೀರಾ' ಎಂದು ಮುಗುಳ್ನಗೆಯಿಂದ ಕೇಳಿದರು.)

ಸತ್ಯಧರ್ಮಾಚಾರ್ಯರು `ವೈಂಡ್ ' {(\eng wind)} ಮಾಡಿದ್ದೇನೆ ಎಂದರು.

ಶ್ರೀಗುರುವು (ಗಡಿಯಾರವನ್ನು ಕಿವಿಯ ಸಮೀಪಕ್ಕಿಟ್ಟು ಸದ್ದನ್ನು  ಆಲಿಸಿ) `ಟೈಮ್ ಸರಿಯಾಗಿ ಹೋಗುತ್ತಿದೆಯೇ?' ಎಂದು ಪುನಃ ಪ್ರಶ್ನಿಸಿದರು. ಅದಕ್ಕೆ  ` ಹೂಂ' ಎಂದು ಉತ್ತರ ಬರಲು ಶ್ರೀಗುರುವು ಮತ್ತೆ ಕಿವಿಗಿಟ್ಟುಕೊಂಡು ಪರಿಶೀಲಿಸಿ ವಿವರಣೆಯನ್ನಾರಂಭಿಸುತ್ತಾರೆ) ನಿಮ್ಮ ಕಾಲಚಕ್ರದಲ್ಲಿ ಒಂದು ತರಹ ಸೌಂಡು {(\eng Sound)} ಇದೆ. ಹಾಗೆ ಇದೆಯೆಂದು ಹೇಳುವುದು ಹೇಗೆ? ಕಿವಿಗಿಟ್ಟು ಗಮನಿಸಿ ಹೇಳಬೇಕು. ಹಾಗೆ ಇಟ್ಟುಕೊಳ್ಳದಿದ್ದರೆ ಅರಿವಿಗೆ ಬರುವುದಿಲ್ಲ. ಗಡಿಯಾರ ಸದ್ದು ಮಾಡುತ್ತಿದ್ದರೂ ಕಿವುಡರಾದರೆ ಕಿವಿಗಿಟ್ಟುಕೊಂಡರೂ ತಿಳಿಯುವುದಿಲ್ಲ. ಸತ್ಯ, ಧರ್ಮ ಇರುವ ಜಾಗದಲ್ಲಿ ಕಾಲಚಕ್ರ ತಿರುಗುತ್ತಿದ್ದರೆ, ಅದು ಒಂದು `ಸೌಂಡ್' ಮಾಡುತ್ತ್ತಿದ್ದರೆ, ಕಿವಿಗೊಟ್ಟು ಮನಗೊಟ್ಟು ವಿಷಯದ ಸ್ವರೂಪಕ್ಕನುಗುಣವಾಗಿ ತೆಗೆದುಕೊಳ್ಳಬೇಕು. ಸತ್ಯ ಎನ್ನುವುದನ್ನು ವೇದ (ಜ್ಞಾನ) ಸ್ಥಾನದಲ್ಲಿಟ್ಟು, ಧರ್ಮ ಎನ್ನುವುದನ್ನು ಆ ಜ್ಞಾನವನ್ನು ಬೆಳಿಸುವ ಅದರ ಸಂಕಲ್ಪದಂತೆ ವಿಸ್ತಾರಪಡಿಸುವ ಗುಣ (ತ್ರಿಗುಣಗಳ) ಸ್ಥಾನದಲ್ಲಿಟ್ಟು ನೋಡಿದರೆ ಯಾವುದೋ ಒಂದು ಸದ್ದು ಮಾಡುತ್ತಾ ಮೊಳಗುತ್ತಾ ತನ್ನ ಒಂದು ಟೈಮ್-ಸ್ಪೇಸಿನಲ್ಲಿ ಅದಿರುವುದು ತಿಳಿಯುತ್ತದೆ. ಅದನ್ನು ನೋಡಿದಾಗ ಅದರ ನಿಜ ಅರಿವಿಗೆ ಬರುತ್ತದೆ. ಆಗ ಅವನು `ಅಸ್ತಿ' -ಎಂದು ಒಪ್ಪುತ್ತಾನೆ.

\begin{shloka}
``ಅಸ್ತಿ ಬ್ರಹ್ಮೇತಿ ಚೇದ್ವೇದ | ಸಂತಮೇನಂ ತತೋ ವಿದುರಿತಿ"
\end{shloka}

(ಬ್ರಹ್ಮನ ಇರುವಿಕೆಯನ್ನು ಕಂಡು ಅರಿತವನನ್ನೇ ಸಂತನೆಂದು ಜ್ಞಾನಿಗಳು ತಿಳಿಯುತ್ತಾರೆ.) ಅದನ್ನು ನೋಡುಲಾಗದೇ, ಇಲ್ಲವೆಂದರೆ ``ಅಸನ್ನೇವ ಸ ಭವತಿ" ಅರಿವಿನಿಂದ ದೂರವಾಗಿ ಅಸತ್ ಆಗುತ್ತಾನೆ.

`ಗಡಿಯಾರದಲ್ಲಿ ಸೌಂಡ್ ಇದೆಯೇ' ಎಂದರೆ ಕಿವಿಗೊಟ್ಟು ನೋಡಬೇಕು, ಸುಮ್ಮನೆ ಇಲ್ಲವೆನ್ನಬಾರದು. ಸಾಮೀಪ್ಯವನ್ನು ಹೊಂದಿ ಅದರ ಸೌಂಡನ್ನು ನೋಡಬೇಕು. ನೋಡಿದರೂ ತಿಳಿಯದಿದ್ದರೆ ಕಾರಣವೇನೆಂದು ವಿಚಾರಿಸಬೇಕು. ಕಿವಿಯ ಕೇಳ್ಮೆಯಲ್ಲಿಯೇ ದೋಷವಿದ್ದರೆ ಗಡಿಯಾರದಲ್ಲಿ ಸೌಂಡಿಲ್ಲ ಎಂದು ನಿರ್ಧರಿಸಲಾಗುವುದಿಲ್ಲ. ಒಂದು ವಿಷಯವನ್ನು ಸರಿಯಾಗಿ ತಿಳಿಯಬೇಕಾದರೆ ಪ್ರಾಮಾತಾ (ತಿಳಿಯುವವನು) ಪ್ರಮಾಣ (ತಿಳಿಯಲು ಸಾಧನವಾದ ಉಪಕರಣ) ಪ್ರಮೇಯ (ತಿಳಿಯಬೇಕಾದ ವಿಷಯ) ಮೂರೂ ಸರಿಯಾಗಿರಬೇಕು. ಇಲ್ಲಿ ಗಡಿಯಾರದ ಸದ್ದನ್ನು ಪರಿಶೀಲಿಸಿ ತಿಳಿಯ ಹೊರಡುವವ ಪ್ರಮಾತಾ. ಸದ್ದು ಕೇಳಲು ಅಳತೆಗೋಲಾದ ಕಿವೀಯೆ ಪ್ರಮಾಣ. ಗಡಿಯಾರದ ಸದ್ದು ಪ್ರಮೇಯ. ಈ ಮೂರೂ ಸುಸ್ಥಿಯಲ್ಲಿದ್ದರೆ ವಿಷಯ ಅರಿವಿಗೆ ಬರುತ್ತದೆ. ಪ್ರಮೆಯೋದಗುತ್ತದೆ.

ಶಿಕ್ಷಾಶಾಸ್ತ್ರವನ್ನು ಹೇಳಬೇಕಾದರೆ, ಆತ್ಮಮೂಲವಾಗಿರುವ ವಿಷಯವನ್ನು ಅಲ್ಲಿಂದಲೇ ಅರಿತು ಹೇಳಬೇಕಾಗುತ್ತದೆ. ಆತ್ಮ ಮೂಲವಾಗಿ ಮಂತ್ರರೂಪವಾಗಿ ಹೊರಟ ವಿಷಯ. ಅದರ ಮೂಲವಾಗಿದೆ ನಾದ. ಮೊಟ್ಟ ಮೊದಲು ಸೂಕ್ಷ್ಮನಾದ. ನಂತರ ವಿಸ್ತಾರ. ಗಡಿಯಾರವನ್ನೇ ತೆಗೆದುಕೊಂಡರೂ ಮೂಲರೂಪದಲ್ಲಿ ಒಂಡು ಸೌಂಡು ಕೇಳುತ್ತೆ. ನಂಟರ ಅದರ ಬಲದಿಂದ ಟಕ್-ಟಕ್-ಟಕ್-ಟಕ್ ಎಂದಾಡುತ್ತದೆ. ಗಮನಿಸಿದರೆ ಇಲ್ಲಿಯೂ ಮಾತ್ರಾದಿಗಳಿವೆ. ಅಂದರೆ ಅದರಲ್ಲಿ ಒಂದು ನಡೆಯಿದೆ. ಅದನ್ನು  ಹೇಳಲು ಹೊರಟಾಗ ಅದೇ ಟೈಮ್ ಸ್ಪೇಸಿನಲ್ಲಿ ಹೇಳಬೇಕು. ಅಲ್ಲಿ ಉದಾತ್ತಾನುದಾತ್ತಗಳಿವೆ. `ಟ' ಉದಾತ್ತವಾದರೆ `ಕ್' ಅನುದಾತ್ತವಾಗುತ್ತದೆ. ಸದ್ದು ಹಾಗೆಯೇ ಇದ್ದರೆ ಗಡಿಯಾರ ಸುಸ್ಥಿತಿಯಲ್ಲಿರುತ್ತದೆ. ಅದು ಬಿಟ್ಟು ಒಂದು ಮಾತ್ರೆ ಹೆಚ್ಚು ಕಡಿಮೆಯಾದರೂ ಗಡಿಯಾರ ಕೆಡುತ್ತದೆ. ನಂತರ ರಿಪೇರಿಗೆ ಓಡಾಡಬೇಕಾಗುತ್ತದೆ. ಸೌಂಡ್ ನಿಂದಲೂ ಗಡಿಯಾರದ ಕಂಡೀಷನ್ {(\eng Condition)} ತಿಳಿಯಬಹುದು.

\section*{ಸತ್ಯಧರ್ಮಗಳು ತಮ್ಮದೇ ಆದ ವಿಧಾನದಿಂದ ಮೊಳಗುತ್ತವೆ}

 ಸತ್ಯಸ್ವರೂಪವಾದ  ವೇದ ಧರ್ಮವು ಮುಂದುವರಿಯುತ್ತಿದೆಯೇ ಎಂದರೆ ಅದಕ್ಕೇ ಆದ್ ಸೌಂಡ್ ಬೇಕು. ಆ ಧರ್ಮವನ್ನು ಕೆಡದಂತೆ ಒರಿಜಿನಲ್ ಆಗಿ ಕಾಪಾಡಿಕೊಳ್ಳ್ಯುವುದಕ್ಕಾಗಿಯೇ ಶಿಕ್ಷಾಶಾಸ್ತ್ರ. ವೇದದ ನಡೆಯನ್ನು ಉಳಿಸಿಕೊಂಡು ಮಂತ್ರರೂಪವಗಿ ರಕ್ಷಿಸಿಕೊಳ್ಳಲು ಶಿಕ್ಷೆ ಬೇಕು. ಗಡಿಯಾರದ ಸೌಂಡನ್ನು ನೋಡಿ, ಅದರ ಮೂಲವರಿತವರು ಅದರ ಸುಸ್ಥಿತಿಯನ್ನರಿಯುವಂತೆ, ಶಿಕ್ಷಾಶಾಸ್ತ್ರದ ನಡೆಯನ್ನು ನೋಡಿಯೂ ಅದು ಮೂಲಕ್ಕನುಗುಣವಾಗಿದೆಯೇ? ಎಂದು ಬಲ್ಲವರು ಅರಿಯಬಹುದು. ಸತ್ಯ ಧರ್ಮಗಳು ತಮ್ಮದೇ ಆದ ವಿಧಾನದಿಂದ ಮೊಳಗುತ್ತವೆ. ಅದನ್ನರಿತು ಅದರ ಶಿಕ್ಷೆಯನ್ನು ಬೆಳೆಸಿದರೆ ಅದರಲ್ಲಿ ಜೀವ ತುಂಬಿಕೊಳ್ಳುತ್ತದೆ. ಆದ್ದರಿಂದ ಮೂಲವರಿತು, ಅಲ್ಲಿಂದ ಅಂತರ್ವಾಹಿನಿಯಾಗಿ ಹರಿಯುವ ಶಾಸ್ತ್ರದ ಜಾಡನ್ನೂ ಅರಿತು ಅದನ್ನು ಬೆಳೆಸಬೇಕಾಗಿದೆಯಪ್ಪಾ. ಅದರ ಮಧುರವಾದ ನಡೆಯರಿತು ಬೆಳೆಸಿದಾಗ ಶಾಸ್ತ್ರವು ಸ್ವರೂಪದಲ್ಲಿ ಉಳಿಯುತ್ತದೆ.

\section*{ಒಳಸ್ಥಿತಿಯ ಅನುಭವದೊಂದಿಗೆ ಬಂದಾಗಲೇ ವೇದ}

ಅಕ್ಷರ, ಪದ ಭ್ರಷ್ಟವಾದರೆ-ಜಾರಿದರೆ, ಮಾತ್ರಾಹೀನವಾದರೆ ಅದರ ನಡೆ ಕೆಡುತ್ತದೆ. `ಹಾಽವು ಹಾಽವು ಹಾಽವು' ಎಂದು ಹೇಳುತ್ತಿರುವಾಗ ಹತ್ತಿರದಲ್ಲಿ ಹಾವು ಹೋಗುತ್ತಿದ್ದರೆ ಅದರತ್ತ ಹರಿಸಿ ಅಲ್ಲಿಗೇ ಮುಗಿಸುತ್ತಾನೆ. ಈ ಹಾವನ್ನು ಕುರಿತು ಬಂದ ಮಾತಲ್ಲ ಅದು. ಆದ್ದರಿಂದ ಆ ಮಂತ್ರ ತನ್ನ ಪದದಿಂದ-ಸ್ಥಾನದಿಂದ ಜಾರಿತು.

ವೇದ ಹೇಳುವಾಗ ತಲೆಯ ಮೇಲೆ ನಿಂಬೆಯ ಹಣ್ಣನ್ನಿಟ್ಟರೂ ಜಾರದಂತೆ ಹೇಳಬೇಕು ಎಂಬ ಮಾತಿದೆ. ಆ ಮಾತು ಹೊರಗೆ ತಲೆಯಲ್ಲಾಡಿಸದಿರಬೇಕು ಎಂಬ ಅರ್ಥದಲ್ಲಿ ಇಳಿದಿದೆ. ಆದರೆ ಅದರ ವಿಷಯವೇ ಬೇರೆಯಾಗಿದೆ. ವೇದವು ಕೇವಲ ಹೊರಗೆ ಉಚಚ್ರಿಸುವ ಮಾತಲ್ಲ. ಒಳಗೆ ಒಂದು ನಿಶ್ಚಲಸ್ಥಿತಿ ಏರ್ಪಟ್ಟಾಗ ಅಲ್ಲಿಂದ ಹೊರಟುದು ವೇದವಾಣಿ. ಆ ಸ್ಥಿತಿಗೆ ಏರಿದಾಗ, ತಲೆಯ ಮೇಲೆ ನಿಂಬೆಯ ಹಣ್ಣಿಟ್ಟರೂ ಬೀಳುವುದಿಲ್ಲ. ಅಂತಹ ಒಳ ಸ್ಥಿತಿಯನ್ನು ಅನುಭವಿಸುತ್ತಾ ಆ ಧರ್ಮದೊಡನೆ ವೇದ ಹೇಳಬೇಕು. ಅದು ಬಿಟ್ಟು ತಲೆಯ ಮೇಲೆ ನಿಂಬೆಹಣ್ಣಿಟ್ಟುಕೊಂಡು ವೇದ ಹೇಳಬೇಕೆಂದರೆ ಅದರ ನೆಲೆ ಕೆಡುತ್ತದೆ. ಆ ರೀತಿ ಹೇಳಿದರೆ ಎಲ್ಲಾ ಪದ ಅಕ್ಷರಗಳಿರಬಹುದಾದರೂ ಪದಕ್ಕೂ ಅರ್ಥಕ್ಕೂ ಸಂಬಂಧವೇ ಇರುವುದಿಲ್ಲ. ತಲೆಯ ಮೇಲೆ ನಿಂಬೆಹಣ್ಣಿಟ್ಟು ಅಳ್ಳಾಡಿಸದಿರುವುದೇ ಹೆಚ್ಚುಗಾರಿಕೆಯಾದರೆ ಹೆಂಗಸರು ದೂರದ ಹೊಳೆಕೆರೆಗಳಿಂಡ ತಲೆಯ ಮೇಲೆ ನೀರುಹೊತ್ತು ನಿರಾಯಾಸವಾಗಿ ಬರುವುದಿಲ್ಲವೇ? ನಿಂಬೆಹಣ್ಣು ಬೀಳಿಸದೆ ವೇದ ಹೇಳುವವರಿಗಿಂತಲೂ ಅವರೇ ಶ್ರೇಷ್ಠರು ಎನ್ನಬೇಕಾಗುತ್ತದೆ. ಶಿಕ್ಷಾಶಾಸ್ತ್ರ ಹೊರಟಿರುವುದು ಮಂತ್ರದ ಸ್ವರೂಪವನ್ನು ರಕ್ಷಿಸುವುದಕ್ಕಾಗಿಯೇ. ಅದು ಸ್ವಲ್ಪ ವ್ಯತ್ಯಾಸವಾದರೂ ಮಂತ್ರ ಸಿದ್ಧಿಸುವುದಿಲ್ಲ.

\section*{ಮಂತ್ರಸಿದ್ಧಿ ಎಂಬುದರ ಬಗೆಗೆ ಒಂದು ಪ್ರಯೋಗ}

(ಮಂತ್ರ ಮತ್ತು ಮಂತ್ರಸಿದ್ಧಿ ಎಂಬುದರ ಬಗೆಗೆ ಒಂದು ಪ್ರಯೋಗವನ್ನು  ಶ್ರೀಗುರುವು ನಡೆಸಿದರು.) ನಾನು ನಿಮಗೆ ಒಂದು ಮಂತ್ರವನ್ನು ಉಪದೇಶಿಸುತ್ತೇನೆ. ಮಂತ್ರದ ಫಲವೇನೆಂಬುದನ್ನು ಮೊದಲೇ ಒಂದು ಚೀಟಿಯಲ್ಲಿ ಬರೆದು ಮಡಿಸಿ ನಿಮ್ಮ ಕೈಗೆ ಕೊಡುತ್ತೇನೆ. ಆದರೆ ಮಂತ್ರವನ್ನು ಉಚ್ಚರಿಸಿದಾಗ ಕೆಲವರಿಗೆ ಒಂದೇ ಬಾರಿಗೇ ಫಲಸಿದ್ಧಿಯಾಗಬಹುದು. ಅವರವರ ಪ್ರಕೃತಿಭೇದೇನ ಸ್ವಲ್ಪ ಹೆಚ್ಚು ಕಡಿಮೆಯಾದರೂ ಫಲವುಂಟೇ ಉಂಟು. ಉಚ್ಚಾರಣೆ ಸ್ವಲ್ಪ ವ್ಯತ್ಯಾಸವಾದರೂ ಫಲ ಬರುವುದಿಲ್ಲ.

(ಹೀಗೆ ಹೇಳಿ ಒಂದು ಚೀಟಿಯನ್ನು ಬರೆದು ಮಡಿಸಿ ಅಲ್ಲಿದ್ದ ಒಬ್ಬರಿಗೆ ಕೊಟ್ಟರು; ನಂತರ `ಷ' ವರ್ಣವನ್ನು ನಾಲಿಗೆಯನ್ನು ಮೇಲಕ್ಕೆ  ಒತ್ತಿ ಉಚ್ಚರಿಸಿ! ಎಂದು ನಾಲ್ಕಾರು ಬಾರಿ ಉಚ್ಚರಿಸಿ ತೋರಿಸಿದರು. ಅಲ್ಲಿದ್ದ ನಾಲ್ಕೈದು ಜನಗಳೂ ಅದನ್ನು ಹಾಗೆಯೇ ಅನುಕರಿಸಿದರು. ಒಬ್ಬೊಬ್ಬರಿಗಾಗಿ `ಆಕಳಿಕೆ' ಬರಲು ಶುರುವಾಯಿತು.' ಎಂದುಹೇಳಿ ಚೀಟಿಯನ್ನು ಬಿಚ್ಚಿ ಓದುವಂತೆ ಹೇಳಿದರು. ಚೀಟಿಯಲ್ಲಿ `ಆಕಳಿಕೆ' ಎಂದೇ ಬರೆದಿತ್ತು)

ನೋಡಿ ಈ ಮಂತ್ರವನ್ನು ಸರಿಯಾಗಿ ಉಚ್ಚರಿಸಿದರೆ ಆಕಳಿಗೆ ಬರದಿದ್ದವರಿಗೂ ಬರುತ್ತದೆ. ಅದಕ್ಕೆ ಸಹಜವಾದ ಸ್ವರ, ಮಾತ್ರೆ, ಬಲಗಳೊಡನೆ ಉಚ್ಚರಿಸಿದರೆ ಫಲ ಸಿದ್ಧಿಯಾಗುತ್ತದೆ. ಅವು ವ್ಯತ್ಯಾಸವಾದರೆ ಫಲವು ಬರುವುದಿಲ್ಲ.

\section*{ವೈದಿಕವಾದ ಉದಾಹರಣೆಗಳೊಂದಿಗೆ ಶಿಕ್ಷಾಶಾಸ್ತ್ರದ ಮಾರ್ಮಿಕ ವಿವರಣೆ}

ಶಿಕ್ಷೆಗೆ ಆಕಳಿಕೆಯ ಉದಾಹರಣೆ ಲೌಕಿಕವಾಯಿತು, ವೈದಿಕವಾಗಿಯೂ ಉದಾಹರಣೆ ಕೊಡಬಹುದು. ` ಪದ್ಭ್ಯಾಮುದರೇಣ ಶಿಶ್ನಾ'- ಎಂಬ ಮಂತ್ರವನ್ನು ಉಚ್ಚರಿಸಿದಾಗ ಆಯಾ ಪದ, ಅದರ ಧರ್ಮ, ಸ್ವರ, ಕಾಲ, ಪ್ರಯತ್ನಗಳು ಕೂಡಿರುವ ಹಾಗೆ ಉಚ್ಚರಿಸಿದರೆ ಅದು ಸಾರ್ಥಕವಾಗುತ್ತದೆ. `ಪದ್ಭ್ಯಾಂ'-ಹೇಳುವಾಗ ಕಾಲುಗಳ ಎಡೆಗೆ ಧ್ವನಿ ನಮ್ಮನ್ನು ಒಯ್ಯುವಂತಿರಬೇಕು. ಪದ ಪದಾರ್ಥವನ್ನು ಮುಟ್ಟಿ ಬರಬೇಕು. ಅಂತಯೇ `ಅಂತರೇಣಾ ತಾಲುಕೇ ಯ ಏಷ ಸ್ತನ ಇವಾವಲಂಬತೇ, ಸೇಂದ್ರ ಯೋನಿಃ'-ಎಂಬ ಮಂತ್ರವನ್ನು ಹೇಳುವಾಗ ಅದನ್ನು-ತಾಲುವಿನ ನಡುವೆ ಇರುವ ವಸ್ಸ್ತುವನ್ನು ಮುಟ್ಟಿ ಬರುವಂತೆ ಧ್ವನಿಯಿರಬೇಕು. `ಹೃದಯಂ ತದ್ವಿಜಾನೀಯಾತ್' ಎಂದು ಹೊರಗೆ ತೋರಿಸುತ್ತಾ ಹೃದಯವೇ ಇಲ್ಲದೇ ಹೇಳಬಾರದು, ಹೃದಯದೆಡೆಯಿಂದಲೇ ಧ್ವನಿ ಹುಟ್ಟಬೇಕು. `ತಸ್ಯಾಂತೇ ಸುಷಿರಗ್ಂ ಸೂಕ್ಷ್ಮಂ'-ಎನ್ನುವಲ್ಲಿ ಒಳರಂಧ್ರದತ್ತ-ಹೃದಯ ಬಿಲದತ್ತ ಸೂಕ್ಷ್ಮವಾಗಿ ನುಗ್ಗುವ ಧ್ವನಿಯಿರಬೇಕು. ಹಾಗೆ ಹರಿಯಲು ಒಳಧರ್ಮದ ಪರಿಚಯವಿರಬೇಕು. ಹಾಗೆ ಒಳಧರ್ಮವನ್ನು ಹೊತ್ತುಹರಿದಾಗ ಮಾತ್ರ ಅದು ಮಂತ್ರವಾಗುತದೆ. ಆದ್ದರಿಂದಲೇ ಆತ್ಮಮೂಲದಿಂದ ಅಂತರ್ವಾಹಿನಿಯಾಗಿ ಶಾಸ್ತ್ರ ಹರಿದು ಹೊಮ್ಮಬೇಕಾಪ್ಪಾ. ಅಂತರಂಗದ ಸ್ವಾಮಿಯ ಕಡೆಯಿಂದ ಚಿಮ್ಮಿದಾಗ ಶಿಕ್ಷಾಶಾಸ್ತ್ರ ಸಜೀವವಾಗಿ ಬೆಳೆಯುತ್ತದೆ. ಹಾಗೆ ಅದರ ಅಭ್ಯಾಸ ಬೆಳೆಯಬೇಕು.



