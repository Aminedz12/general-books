\chapter{ರಾಮಾಯಣದ ರಚನೆಗೆ ಹಿನ್ನೆಲೆ} 

(`ಕ್ರೌಂಚಪಕ್ಷಿಗಳ ಘಟನೆ ದೈವ ನಿರ್ಮಿತವೇ? ಆಕಸ್ಮಿಕವೇ? ರಾಮನೇ ಬೇಡ ಎನ್ನುತ್ತಾರಲ್ಲ, ಇದು ಸರಿಯೇ?' ಎಂಬ ಹಿರಿಯರೊಬ್ಬರ ಈ ಪ್ರಶ್ನೆಗೆ ಉತ್ತರವಾಗಿ ತಾ. ೧೦-೮-೬೧ ರಂದು ಶ್ರೀರಂಗ ಮಹಾಗುರುವು ಕೊಟ್ಟ ವಿವರಣೆ) 

\section*{ಆತ್ಮಗುಣೋಪೇತನಾದ ನರನ ಕುರಿತು ನಾರದರಲ್ಲಿ ವಾಲ್ಮೀಕಿಗಳ ಪ್ರಶ್ನೆ} 

ಘಟನೆಯ ಬಗೆಗೆ ಹಾಗೆ ಅರ್ಥಮಾಡಬಹುದೇ? ಎನ್ನುವುದನ್ನು ಹಿಂದು-ಮುಂದಣ ಸಂದರ್ಭದಿಂದ ಪರಿಶೀಲಿಸೋಣ. ನಾವು ಒಂದು ಅಭಿಪ್ರಾಯವನ್ನು ಶ್ಲೋಕಕ್ಕಿಡುವುದಾದರೆ, ಅದು ವಾಲ್ಮೀಕಿಗಳ ಹೃದಯಕ್ಕೆ ವಿರೋಧವಾಗಿರಬಾರದು. ಪ್ರಕೃತ ಈ ಘಟನೆ ಯಾವ ಹಿನ್ನೆಲೆಯಲ್ಲಿ ನಡೆದಿದೆ ಎನ್ನುವುದನ್ನು ನೋಡೋಣ- 

\begin{shloka}
ತಪಸ್ಸ್ವಾಧ್ಯಾಯನಿರತಂ ತಪಸ್ವೀ ವಾಗ್ವಿದಾಂ ವರಮ್‍|\\ 
ನಾರದಂ ಪರಿಪಪ್ರಚ್ಛ ವಾಲ್ಮೀಕಿರ್ಮುನಿಪುಂಗವಮ್‍||
\end{shloka}


ತಪಸ್ಸ್ವಾಧ್ಯಾಯನಿರತರಾದ ನಾರದರನ್ನು ತಪಸ್ವಿಗಳಾದ ವಾಲ್ಮೀಕಿಗಳು ಪ್ರಶ್ನಿಸಿದ್ದಾರೆ. ವಾಲ್ಮೀಕಿಗಳು ಗುಣಧರ್ಮಾದಿಗಳಿಂದ ಕೂಡಿದ ನರನ ವಿಷಯವನ್ನು ತಿಳಿಯಲು ಅತ್ಯಂತ ಕುತೂಹಲಿಗಳಾಗಿದ್ದಾರೆ. ಹಾಗೆಯೇ ಅಂತಹ ಮಹರ್ಷಿಗಳ ಕುತೂಹಲವನ್ನು ನಿವಾರಿಸಲು ನಾರದರಿಗೆ ಸಾಮರ್ಥ್ಯವೂ ಇದೆ- 

\begin{shloka} 
ಏತದಿಚ್ಛಾಮ್ಯಹಂ ಶ್ರೋತುಂ ಪರಂ ಕೌತೂಹಲಂ ಹಿ ಮೇ|\\ 
ಮಹರ್ಷೆ ತ್ವಂ ಸಮರ್ಥೋ\char'263ಸಿ ಜ್ಞಾತುಮೇವಂ ವಿಧಂ ನರಮ್‍||
\end{shloka}

ಏಕೆಂದರೆ ಅವರು ತ್ರಿಲೋಕಜ್ಞರು (ಶ್ರುತ್ವಾ ಚೈತತ್‍ ತ್ರಿಲೋಕಜ್ಞಃ) ನಾರದರ ಉತ್ತರ- 


\begin{shloka} 
ಬಹವೋ ದುರ್ಲಭಾಶ್ಚೈವ ಯೇ ತ್ವಯಾ ಕೀರ್ತಿತಾ ಗುಣಾಃ|\\ 
ಮುನೇ ವಕ್ಷಾಮ್ಯಹಂ ಬುದ್ಧ್ವಾ ತೈರ್ಯುಕ್ತಃ ಶ್ರೂಯತಾಂ ನರಃ|| 
\end{shloka}


`ನೀವು ಕೇಳಿದ ಗುಣವುಳ್ಳವರು ಬಹು ವಿರಳ. ಆದರೂ ನಾನು ತಿಳಿದು ಅಂತಹ ಗುಣಗಳಿಂದ ಕೂಡಿದ ನರನ ಬಗೆಗೆ ಹೇಳುತ್ತೇನೆ ಕೇಳಿರಿ.' ವಾಲ್ಮೀಕಿಗಳು ಹಾಕಿರುವ ಪ್ರಶ್ನೆಯು ಸುಲಭವಾದುದೇನೂ ಅಲ್ಲ, ಜಟಿಲವಾದುದು. 

\section*{ನರನಾಗಿಯೇ ರಾಮನನ್ನು ಗ್ರಹಿಸಬೇಕು} 


ಅದಲ್ಲದೆ ಅಂತಹ ಗುಣವಿಶಿಷ್ಟನಾದ ನರನ ವಿಷಯವನ್ನೇ ವಾಲ್ಮೀಕಿಗಳು ಕೇಳಿದ್ದಾರೆ. ಇದು ಗಮನಿಸಬೇಕಾದ ಅಂಶ. `ನರ' ಲೋಕದಲ್ಲಿ ನರನೇ ಆಗಿದ್ದು ಇಷ್ಟು ಗುಣಗಳಿಂದ ಕೂಡಿದವರಿದ್ದರೆ ತಾನೇ, ವಾಲ್ಮೀಕಿಗಳಿಗೆ ನರರನ್ನು ಮೇಲೇರಿಸಲು ಉದ್ಧರಿಸಲು ಸಾಧ್ಯವಾದೀತು. ಆದ್ದರಿಂದಲೇ ಅಂತಹ ಗುಣವಿಶಿಷ್ಟನಾದ ನಾರಾಯಣನನ್ನು ಕೇಳಲಿಲ್ಲ, ನರನನ್ನೇ ಕೇಳಿದ್ದಾರೆ. ನಾರದರೂ ನಾರಾಯಣನ ವಿಚಾರವನ್ನು ಹೇಳುತ್ತೇನೆಂದೇನೂ ಹೇಳಲಿಲ್ಲ. `ತೈರ್ಯುಕ್ತಃ ಶ್ರೂಯತಾಂ ನರಃ' ನರನ ವಿಚಾರವನ್ನೇ ಹೇಳಲಪೇಕ್ಷಿಸಿದ್ದಾರೆ. ಇಲ್ಲಿ ಮಹರ್ಷಿಗಳು ರಾಮನನ್ನು ನರನನ್ನಾಗಿಯೇ ಲೋಕದಲ್ಲಿಡಲು ಅಪೇಕ್ಷಿಸಿದ್ದಾರೆ. ಅವನು ನಾರಾಯಣನಾಗಿರಬಹುದು ಆದರೆ ಆ ಅಂಶವನ್ನು ಬಿಚ್ಚಿಡಲು ಅಪೇಕ್ಷಿಸುತ್ತಿಲ್ಲ, ಬಚ್ಚಿಡಲು ಅಪೇಕ್ಷಿಸುತ್ತಿದ್ದಾರೆ. ಆದ್ದರಿಂದ ಆರಂಭದಲ್ಲಿಯೇ ರಾಮನನ್ನು ನಾರಾಯಣನೆಂದು ಭಾವಿಸಿ ರಾಮಾಯಣದ ವ್ಯಾಖ್ಯಾನಕ್ಕೆ ತೊಡಗುವುದು ವಾಲ್ಮೀಕಿಗಳ ಅಭಿಪ್ರಾಯವಲ್ಲ. ಹೀಗೆಯೇ ಅವರು ಉದ್ದಕ್ಕೂ ರಾಮನನ್ನು ಬಚ್ಚಿಡಲು ಪ್ರಯತ್ನಿಸಿದ್ದಾರೆ. ಹಾಗೆಯೇ ಕಾವ್ಯ ಸಾಗಿದೆ. ಮುಂದೆ ವಿಶ್ವಾಮಿತ್ರರೂ ರಾಜಸಭೆಯಲ್ಲಿ- 

\begin{shloka} 
ಅಹಂ ವೇದ್ಮಿ ಮಹಾತ್ಮಾನಂ ರಾಮಂ ಸತ್ಯಪರಾಕ್ರಮಮ್‍|\\ 
ವಸಿಷ್ಠೋ\char'263ಪಿ ಮಹಾತೇಜಾಃ ಯೇ ಚೇಮೇ ತಪಸಿ ಸ್ಥಿತಾಃ||
\end{shloka}
ಎಂದು ರಾಮನ ವಿಚಾರವನ್ನು ಗಂಭೀರವಾಗಿ ಗುಟ್ಟೊಡೆಯದಂತೆ ಇಟ್ಟಿದ್ದಾರೆ. ವಾಲ್ಮೀಕಿಗಳಿಗೆ ರಾಮನನ್ನು ನರನಂತೆಯೇ ಇಡಬೇಕೆಂದು ಅಭಿಪ್ರಾಯವಾಗಿದೆ. ಆದ್ದರಿಂದ ಆ ಗುಟ್ಟನ್ನು ಬಚ್ಚಿಟ್ಟೇ ರಾಮಾಯಣದ ವ್ಯಾಖ್ಯಾನ ಮಾಡಬೇಕು. `ಗರ್ಭವು ಮುಚ್ಚಿ ಬೆಳೆದರೆ ಚೆನ್ನು' ಎಂದು ಪ್ರಕೃತಿಯೇ ಮುಚ್ಚಿ ಬೆಳೆಯುತ್ತಿರುವಾಗ ಗರ್ಭದಲ್ಲಿರುವಾಗಲೇ ಅದರ ಸ್ವಾರಸ್ಯ. ಗರ್ಭದಲ್ಲಿರುವಾಗ ಪ್ರಕೃತಿಯ ಆಶಯಕ್ಕೆ ಅಥವಾ ಸೃಷ್ಟಿಯ ಆಶಯಕ್ಕೆ ವಿರೋಧವಾಗಿ ರಹಸ್ಯವನ್ನು ಹೊರಗೆಡಹುವುದು ಬೇಡ. ಹಾಗೆಯೇ ನರನಂತೆಯೇ ರಾಮನನ್ನು ನೋಡಿ. ಅದು ಎಲ್ಲಿಗೆ ಒಯ್ಯುತ್ತದೆ ಎಂಬುದನ್ನು ಕೊನೆಯಲ್ಲಿ ನೋಡಿ. ಹಾಗೆ ನೋಡಿದಾಗ ತಾನೇ ಅದರ ಗುಟ್ಟು ತಿಳಿದೀತು! ರಸ ಒಸರೀತು! ಅಲ್ಲಿಯ ವರೆಗೆ ವಾಲ್ಮೀಕಿಹೃದಯವನ್ನು ಬಿಚ್ಚಿಡಬಾರದು. ಇದರಿಂದ ವಾಸ್ತವಿಕತೆಗೆ ಏನೂ ಧಕ್ಕೆ ಇಲ್ಲ. ವಾಲ್ಮೀಕಿಗಳೇ ರಾಮನನ್ನು ನರನನ್ನಾಗಿ ಚಿತ್ರಿಸಿದ್ದಾರೆ. ಆದ್ದರಿಂದ ನರನಾಗಿಯೇ ರಾಮನನ್ನು ಗ್ರಹಿಸಬೇಕು. ಇಲ್ಲದಿದ್ದರೆ ರಸಾಭಾಸ. ಆ ರಸಾಭಾಸ ತಲೆಕೆಳಗಾಗುವಂತೆ ಇಡಬೇಕು, ಅಂದರೆ ಸಭಾಸಾರವಾಗುವಂತಿಡಬೇಕು (ರಸಾಭಾಸ-ಸಭಾಸಾರ) 

\section*{ನಾರದರಿಂದ ಆತ್ಮಗುಣಾದಿ ಸಹಿತನಾದ ರಾಮನ ವರ್ಣನೆ} 

ಆದ್ದರಿಂದ ವಾಲ್ಮೀಕಿಗಳ ಹೃದಯವನ್ನು, ಅವರ ಆಶಯವನ್ನು ಬಿಟ್ಟು ಕೊಡಬಾರದು. ಅಂತೆಯೇ ನಾರದರೂ ಅದನ್ನು ಅನುಸರಿಸಿದ್ದಾರೆ. ಆತ್ಮ ಗುಣೋಪೇತನಾದ, ಸರ್ವಲಕ್ಷಣ ಸಂಪನ್ನನಾದ ರಾಮನ ಕಥೆಯನ್ನು ಬಿತ್ತರಿಸಿದ್ದಾರೆ. ಮೊದಲು ಅವನ ಆತ್ಮಗುಣ, ಅನಂತರ ಶರೀರಸೌಂದರ್ಯ, ಅನಂತರ ಮತ್ತೆ ಆತ್ಮಗುಣಗಳು. 

\begin{shloka} 
ಇಕ್ಷ್ವಾಕುವಂಶಪ್ರಭವೋ ರಾಮೋ ನಾಮ ಜನೈಃ ಶ್ರುತಃ|\\ 
ನಿಯತಾತ್ಮಾ ಮಹಾವೀರ್ಯೋ ದ್ಯುತಿಮಾನ್‍ ಧೃತಿಮಾನ್‍ ವಶೀ||\\ 
ಬುದ್ಧಿಮಾನ್‍ ನೀತಿಮಾನ್‍ ವಾಗ್ಮೀ ಶ್ರೀಮಾನ್‍ ಶತ್ರುನಿಬರ್ಹಣಃ|\\ 
ವಿಪುಲಾಂಸೋ ಮಹಾಬಾಹುಃ ಕಂಬುಗ್ರೀವೋ ಮಹಾಹನುಃ||\\ 
ಮಹೋರಸ್ಕೋ ಮಹೇಷ್ವಾಸೋ ಗೂಢಜತ್ರುರರಿಂದಮಃ|\\ 
ಆಜಾನುಬಾಹುಸ್ಸುಶಿರಾಃ ಸುಲಲಾಟಃ ಸುವಿಕ್ರಮಃ||\\ 
ಸಮಸ್ಸಮವಿಭಕ್ತಾಂಗಃ ಸ್ನಿಗ್ಧವರ್ಣಃ ಪ್ರತಾಪವಾನ್‍|\\ 
ಪೀನವಕ್ಷಾ ವಿಶಾಲಾಕ್ಷೋ ಲಕ್ಷ್ಮೀವಾನ್‍ ಶುಭಲಕ್ಷಣಃ||\\ 
ಧರ್ಮಜ್ಞಃ ಸತ್ಯಸಂಧಶ್ಚ ಪ್ರಜಾನಾಂ ಚ ಹಿತೇ ರತಃ|\\ 
ಯಶಸ್ವೀ ಜ್ಞಾನಸಂಪನ್ನಃ ಶುಚಿರ್ವಶ್ಯಃ ಸಮಾಧಿಮಾನ್‍||\\ 
ಪ್ರಜಾಪತಿಸಮಃ ಶ್ರೀಮಾನ್‍ ಧಾತಾ ರಿಪುನಿಷೂದನಃ|\\ 
ರಕ್ಷಿತಾ ಜೀವಲೋಕಸ್ಯ ಧರ್ಮಸ್ಯ ಪರಿರಕ್ಷಿತಾ||\\ 
ವೇದವೇದಾಂಗತತ್ತ್ವಜ್ಞಃ ಸ್ಮೃತಿಮಾನ್‍ ಪ್ರತಿಭಾನವಾನ್‍|\\ 
ಸರ್ವಲೋಕಪ್ರಿಯಸ್ಸಾಧುರದೀನಾತ್ಮಾ ವಿಚಕ್ಷಣಃ||\\ 
ಸರ್ವದಾಭಿಗತಸ್ಸದ್ಭಿಃ ಸಮುದ್ರ ಇವ ಸಿಂಧುಭಿಃ|\\ 
ಆರ್ಯಃ ಸರ್ವಸಮಶ್ಚೈವ ಸದೈಕಪ್ರಿಯದರ್ಶನಃ||\\ 
ಸ ಚ ಸರ್ವಗುಣೋಪೇತಃ ಕೌಸಲ್ಯಾನಂದವರ್ಧನಃ|\\ 
ಸಮುದ್ರ ಇವ ಗಾಂಭೀರ್ಯೇ ಧೈರ್ಯೇಣ ಹಿಮವಾನಿವ||\\ 
ವಿಷ್ಣುನಾ ಸದೃಶೋ ವೀರ್ಯೇ ಸೋಮವತ್‍ ಪ್ರಿಯದರ್ಶನಃ|| 
\end{shloka}

ಇಲ್ಲಿಯೂ ಒಡೆದಿಲ್ಲ ರಾಮನ ವಿಷ್ಣುತ್ವವನ್ನು, ಹೇಳಿರುವುದು ಅವನ ಸಾದೃಶ್ಯವನ್ನು ಮಾತ್ರ. 

\section*{ರಾಮಕಥೆಯನ್ನು ಹೃದಯದಲ್ಲಿ ಹೊತ್ತ ಮಹರ್ಷಿಯ ತಮಸಾಗಮನ} 

ಹೀಗೆ ರಾಮನ ವೃತ್ತಾಂತವನ್ನು ತಿಳಿಸಿ ವಾಲ್ಮೀಕಿಮಹರ್ಷಿಯಿಂದ ಬೀಳ್ಕೊಡುಗೆಯನ್ನು ಪಡೆದು ನಾರದರು ವಿಹಾಯಸವನ್ನು ಕುರಿತು ಪ್ರಯಾಣ ಬೆಳೆಸಿದರು- 

\begin{shloka}
ಯಥಾವತ್ಪೂಜಿತಸ್ತೇನ ದೇವರ್ಷಿರ್ನಾರದಸ್ತಥಾ|\\ 
ಆಪೃಚ್ಛ್ವೈವಾಭ್ಯನುಜ್ಞಾತಃಸಜಗಾಮ ವಿಹಾಯಸಮ್‍||\\ 
ಸ ಮುಹೂರ್ತಂ ಗತೇ ತಸ್ಮಿನ್‍ ದೇವಲೋಕಂ ಮುನಿಸ್ತದಾ|\\ 
ಜಗಾಮ ತಮಸಾತೀರಂ ವಾಲ್ಮೀಕಿರ್ಭಗವಾನೃಷಿಃ||
\end{shloka}

ಅನಂತರ ವಾಲ್ಮೀಕಿ ಮಹರ್ಷಿಗಳು ಸ್ನಾನಾರ್ಥವಾಗಿ `ತಮಸಾ' ನದಿಯನ್ನೈದಿದರು. 

ನಡೆದಿರುವ ಘಟನೆ ಇಷ್ಟು-ವಾಲ್ಮೀಕಿಗಳ ಆಶ್ರಮಕ್ಕೆ ನಾರದರು ಬಂದರು. ತಮ್ಮ ಕುತೂಹಲವನ್ನು ನಾರದರಲ್ಲಿಟ್ಟು ರಾಮಕಥಾ ರೂಪವಾದ ಉತ್ತರವನ್ನು ವಾಲ್ಮೀಕಿಗಳು ಪಡೆದರು. ಇದರಿಂದ ತಮಸಾತೀರಕ್ಕೆ ಹೋಗುವ ಮೊದಲೇ ರಾಮಕಥೆ ಅವರ ಮನಸ್ಸಿಗೆ ಬಂದಿದೆ. ಚೆನ್ನಾಗಿ ತಿಳಿದಿದೆ. ತಮ್ಮಲ್ಲಿ ಹುಟ್ಟಿದ ಕುತೂಹಲವನ್ನು ನಿವಾರಿಸಿಕೊಂಡಿದ್ದಾರೆ ಅಷ್ಟೇ ಎಂಬ ಅಂಶ ಗೊತ್ತಾಗುತ್ತದೆ. ಅದನ್ನು ಕಾವ್ಯವಾಗಿಡಬೇಕೆಂಬ ಸಂಕಲ್ಪವೇನೂ ಅವರಿಗಿರುವುದು ಸ್ಪಷ್ಟವಾಗಿಲ್ಲ. ಕಾವ್ಯರಚನೆಯ ಬಗೆಗೆ ಅವರ ಮನಸ್ಸು ಹರಿದಿರುವ ಸುಳಿವು ಇಲ್ಲೆಲ್ಲಿಯೂ ಇಲ್ಲ. ತಮ್ಮ ಕುತೂಹಲವನ್ನು ನಿವಾರಿಸಿಕೊಂಡಿರುವ ಘಟನೆ ಮಾತ್ರ ಇಲ್ಲಿದೆ. ಕುತೂಹಲನಿವಾರಣೆ ರಾಮಕಥೆಯಿಂದ ಆಗಿದೆ. ಒಟ್ಟಿನಲ್ಲಿ ತಮಸಾತೀರಕ್ಕೆ ಹೋಗುವ ಮುನ್ನ ವಾಲ್ಮೀಕಿಗಳ ಹೃದಯದಲ್ಲಿ ರಾಮಕಥೆ ನೆಲೆಸಿತ್ತು. ಅದನ್ನು ಮೆಲುಕುಹಾಕುತ್ತಾ ಅವರು ತಮಸಾತೀರಕ್ಕೆ ಹೊರಟರು. 

\section*{ಮಹರ್ಷಿಯ ಮೇಲೆ ರಾಮಕಥೆ ಬೀರಿದ ಪರಿಣಾಮ} 

ತಮಸಾತೀರವನ್ನು ತಲುಪಿ ತಮಸಾನದಿಯನ್ನು ನೋಡಿದುದು ಅವರ ಮೇಲೆ ರಾಮಕಥೆ ಬೀರಿರುವ ಪರಿಣಾಮವನ್ನು ಸೂಚಿಸುತ್ತದೆ. ದೈವೀ ಗುಣಗಳಿಂದ ಕೂಡಿದ ನರನ ಗುಣವೇ ಹೆಚ್ಚಾಗಿ ಮಹರ್ಷಿಯ ಮನಸ್ಸನ್ನು ಆಕ್ರಮಿಸಿದೆ. ಆದ್ದರಿಂದ ಆ ತೀರ್ಥವೂ 

\begin{shloka} 
ಸನ್ಮನುಷ್ಯಮನೋ ಯಥಾ|
\end{shloka}

ಎಂದು ಹೇಳಿರುವಂತೆ ಸನ್ಮನುಷ್ಯರ ಮನಸ್ಸಿನಂತೆಯೇ ಕಾಣುತ್ತದೆ. ಇಲ್ಲಿ ಆ ರಾಮಗುಣಶ್ರವಣದ ಪರಿಣಾಮ ಕಾಣುತ್ತದೆ. ಅಂದರೆ ಮಹರ್ಷಿಗಳ ಮನಸ್ಸು ರಾಮಕಥಾ ಶ್ರವಣದಿಂದ ಪ್ರಸನ್ನವಾಗಿದೆ, ಪ್ರಶಾಂತವಾಗಿದೆ. ಅಂತೆಯೇ ನಿರ್ಮಲವಾದ ಸರಸ್ಸನ್ನೂ ಅವರು ನೋಡಿದ್ದಾರೆ. ಇಲ್ಲಿ ಮಹರ್ಷಿಗಳ ಮನಸ್ಸು ಪ್ರಶಾಂತವಾಗಿ ರಾಮನ ಗುಣಜ್ಞತೆ ಧರ್ಮಜ್ಞತೆಗಳಲ್ಲಿ ಆಸಕ್ತವಾಗಿದೆ ಎನ್ನುವುದು ಬಹುವಾಗಿ ಗಮನಿಸಬೇಕಾದ ಅಂಶ. 

\section*{ಕ್ರೌಂಚವಧೆಯ ಪ್ರಕರಣ} 

ಅನಂತರ- 

\begin{shloka} 
ನ್ಯಸ್ಯತಾಂ ಕಲಶಸ್ತಾತ ದೀಯತಾಂ ವಲ್ಕಲಂ ಮಮ|\\ 
ಇದಮೇವಾವಗಾಹಿಷ್ಯೇ ತಮಸಾತೀರ್ಥಮುತ್ತಮಮ್‍|| 
\end{shloka}
ಎಂದು ಹೇಳುತ್ತಾ ಶಿಷ್ಯಹಸ್ತದಿಂದ ವಲ್ಕಲವನ್ನು ತೆಗೆದುಕೊಂಡು ಸ್ನಾನಕ್ಕೋಸ್ಕರ ತಮಸಾತೀರ್ಥವನ್ನು ಪ್ರವೇಶಿಸಲು ನಡೆದಿದ್ದಾರೆ. ಹೀಗೆ ವಲ್ಕಲವನ್ನು ತೆಗೆದುಕೊಂಡು ವನದ ರಾಮಣೀಯಕತೆಯನ್ನು ಸವಿಯುತ್ತಾ ನಡೆಯುತ್ತಿದ್ದಾರೆ ವಾಲ್ಮೀಕಿಗಳು- 

\begin{shloka} 
ಸ ಶಿಷ್ಯ ಹಸ್ತಾದಾದಾಯ ವಲ್ಕಲಂ ನಿಯತೇಂದ್ರಿಯಃ|\\ 
ವಿಚಚಾರ ಹ ಪಶ್ಯಂಸ್ತತ್ಸರ್ವತೋ ವಿಪುಲಂ ವನಮ್‍||
\end{shloka}

ಹೀಗೆ ಸಂಚರಿಸುತ್ತಿರುವಾಗ ಒಂದು ದೃಶ್ಯ ಕಣ್ಣಿಗೆ ಬಿದ್ದಿದೆ- 

\begin{shloka} 
ತಸ್ಯಾಭ್ಯಾಶೇ ತು ಮಿಥುನಂ ಚರಂತಮನಪಾಯಿನಮ್‍|\\ 
ದದರ್ಶ ಭಗವಾಂಸ್ತತ್ರ ಕ್ರೌಂಚಯೋಶ್ಚಾರುನಿಃಸ್ವನಮ್‍||
\end{shloka}

ಅಪಾಯಕಾರಕಸ್ವಭಾವವಿಲ್ಲದವೂ ಚಾರುನಿಃಸ್ವನ ಉಳ್ಳವೂ ಆದ ಕ್ರೌಂಚ ಪಕ್ಷಿಗಳ ಮಿಥುನ ಕಣ್ಣಿಗೆ ಬಿದ್ದಿದೆ. ಇಲ್ಲಿ `ಚಾರುನಿಃಸ್ವನಂ' ಗಣನೀಯಾಂಶ. 

\begin{shloka}
ತಸ್ಮಾತ್ತು ಮಿಥುನಾದೇಕಂ ಪುಮಾಂಸಂ ಪಾಪನಿಶ್ಚಯಃ|\\ 
ಜಘಾನ ವೈರನಿಲಯಃ ನಿಷಾದಸ್ತಸ್ಯ ಪಶ್ಯತಃ||
\end{shloka}

ಅನಂತರ ಆ ಮಿಥುನದಿಂದ ಪುಮಾಂಸಂ-ಗಂಡುಹಕ್ಕಿಯನ್ನು ಪಾಪ ನಿಶ್ಚಯನೂ ವೈರನಿಲಯನೂ ಆದ ಬೇಡನು ಹೊಡೆದಿದ್ದಾನೆ. ಆಗ ಸಹಚರನ ಹತ್ಯೆಯನ್ನು ನೋಡಿ ಸಹಚರಿ ಕರುಣಾಜನಕವಾದ ಧ್ವನಿಯಿಂದ ಕೂಗಿದೆ- 

\begin{shloka}
ತಂ ಶೋಣಿತಪರೀತಾಂಗಂ ಚೇಷ್ಟಮಾನಂ ಮಹೀತಲೇ|\\ 
ಭಾರ್ಯಾ ತು ನಿಹತಂ ದೃಷ್ಟ್ವಾರುರಾವ ಕರುಣಾಂ ಗಿರಮ್‍||
\end{shloka}

ಇದನ್ನು ವಾಲ್ಮೀಕಿಗಳು ಕೇಳಿ ನೋಡಿದ್ದಾರೆ. 

\section*{ಕರುಣಾವೇದಿಯಾದ ಮಹರ್ಷಿಯ ಶೋಕ ಕಾವ್ಯಕ್ಕೆ ನಾಂದಿಯಾದ ಬಗೆ} 

\begin{shloka} 
ತದಾ ತು ತಂ ದ್ವಿಜಂ ದೃಷ್ಟ್ವಾ ನಿಷಾದೇನ ನಿಪಾತಿತಮ್‍|\\ 
ಋಷೇರ್ಧರ್ಮಾತ್ಮನಸ್ತಸ್ಯ ಕಾರುಣ್ಯಂ ಸಮಪದ್ಯತ|| 
\end{shloka}

ಧರ್ಮಾತ್ಮರಾದ ಆ ಋಷಿಗೆ ಕಾರುಣ್ಯ ಮೂಡಿಬಂದಿದೆ. 

\begin{shloka} 
ತತಃ ಕರುಣವೇದಿತ್ವಾತ್‍ ಅಧರ್ಮೊಽಯಮಿತಿ ದ್ವಿಜಃ| 
\end{shloka}

ಕರುಣವೇದಿಯಾದ ಋಷಿ ಬೇಡನ ವರ್ತನೆಯನ್ನು ಅಧರ್ಮವೆಂದು ನಿಶ್ಚಯಿಸಿದ್ದಾರೆ. ಕರುಣೆ ಉಂಟಾಗುವುದು ಎಂತಹ ಸಂದರ್ಭಗಳಲ್ಲಿ? 

\begin{shloka}
ಧರ್ಮನಾಶೋ\char'263ರ್ಥನಾಶತ್ಚ ಬಾಂಧವೇಷ್ಟಧನಕ್ಷಯಃ|
\end{shloka}
ಎಂಬಂತೆ ಧರ್ಮನಾಶವೇ ಮೊದಲಾದವುಗಳಲ್ಲಿ. 

ಇಲ್ಲಿ ಮಹರ್ಷಿಗೆ ಯಾವುದರಿಂದ ಕರುಣೆ ಉಂಟಾಗಿದೆ? ಎಂದರೆ ಧರ್ಮನಾಶದಿಂದ ಕರುಣೆ ಉಂಟಾಗಿದೆ, ಧರ್ಮಾತ್ಮನಿಗೆ-ಮಹರ್ಷಿವಾಲ್ಮೀಕಿಗೆ ಕರುಣೆ ಬಂದಿತು ಎನ್ನುವಲ್ಲಿ `ಧರ್ಮಾತ್ಮಾ' ಎಂಬ ವಿಶೇಷಣ ಗಣನೀಯ. 

ಮತ್ತೆ ನಡೆದಿರುವ ಘಟನೆಯನ್ನು ಜ್ಞಾಪಿಸಿಕೊಳ್ಳಬೇಕು. ಚಾರುನಿಃಸ್ವನವುಳ್ಳ ಪಕ್ಷಿಗಳ ಮಿಥುನದಲ್ಲಿ ಒಂದನ್ನು ಬೇಡ ಹೊಡೆದುಕೊಂದ. ಇಲ್ಲಿ ಬೇಡನ ಮೇಲೆ ದ್ವೇಷವಾಗಲೀ, ಪಕ್ಷಿಗಳ ಮೇಲೆ ಪ್ರೀತಿಯಾಗಲೀ ವಾಲ್ಮೀಕಿಗಳಿಗೆ ಇಲ್ಲ. ಧರ್ಮಾತ್ಮರಾದ ಅವರಿಗೆ ಧರ್ಮನಾಶವಾಯಿತಲ್ಲ; ಎಂಬುದೇ ಚಿಂತೆ. ಅದೇ ಕರುಣೆಗೆ ಕಾರಣ. ಧರ್ಮನಾಶವಾಯಿತೆಂಬುದಾಗಿ ಮನಕರಗಿ ಹರಿಯುತ್ತಿರುವ ಕರುಣೆ ಒಂದು. ಅದರ ಜೊತೆಗೆ ಕ್ರೌಂಚಿಯ ಕರುಣಾರವದಿಂದ ಹೊರಗಿನಿಂದ ಕಣ್ಣಿಗೆ ಬಿದ್ದ ಕರುಣಾಜನಕದೃಶ್ಯ ಒಂದು. ಇಲ್ಲಿ ನಡೆದುದೇನು? 

ಮಹರ್ಷಿಯ ಮನಕರಗಿ, ರಸ ಒಸರಿ, ನಾರದರಿಂದ ಉಪದಿಷ್ಟವಾದ ``ಸೀತಾರಾಮ" ನದಿಯು ಈ ರಸದೊಡನೆ ಬೆರೆತು, ಭಾವವೀಣೆಯು ಮಿಡಿದು, ನಾದ-ಸ್ವರ-ಅಕ್ಷರ ರೂಪವಾದ ಕಾವ್ಯನದಿಗೆ ನಾಂದೀರೂಪವಾದ ಶೋಕವು ಉಪಶ್ಲೋಕತ್ವವನ್ನು ಹೊಂದಿತು. (ಸುವರ್ಣಾಕ್ಷರಗಳಲ್ಲಿ ಬರೆದಿಡಬೇಕು ಇದನ್ನು) 

\section*{ಅಂತರ್ವಾಹಿನಿಗೆ ಬಹಿರ್ವಾಹಿನಿಯ ಸಂಗಮವಾದ ಎರಡು ಘಟನೆಗಳು} 

ಇಲ್ಲಿನ ಸನ್ನಿವೇಶವನ್ನು ಶಬ್ದಗಳಿಂದ ಹೇಗೆ ವಿವರಿಸಬೇಕು ಎನ್ನುವುದೇ ಸಮಸ್ಯೆ. 

ಋಷಿಗೆ ಮೊದಲೇ ರಾಮನ ಕಥೆ ಮನಸ್ಸಿಗೆ ಬಂದಿದೆ. ಅದನ್ನೇ ಮೆಲುಕು ಹಾಕುತ್ತಾ ತಮಸಾತೀರಕ್ಕೆ ಬಂದಿದ್ದಾರೆ. ಅವರ ಮನಸ್ಸು ನಿರ್ಮಲವಾದ ತಮಸಾ ನದಿಯನ್ನೂ, ಅಲ್ಲಿನ ಪ್ರಕೃತಿ ಸೌಂದರ್ಯವನ್ನೂ ನೋಡಿ ಪ್ರಶಾಂತವಾಗಿದೆ. ಇದೊಂದು ಘಟನೆ. 


ಅನಂತರ ಕ್ರೌಂಚಪಕ್ಷಿಗಳ ದೃಶ್ಯ ಕಣ್ಣಿಗೆ ಬಿದ್ದಿದೆ. ಅವುಗಳ ಚಾರುನಿಃಸ್ವನ ಇವರ ಮನಸ್ಸಿಗೆ ನಾಟಿದೆ. ಅಂತಹ ಚಾರುನಿಃಸ್ವನದಿಂದ ಕೂಡಿ, ಕ್ರೀಡಾರತಿಯಲ್ಲಿ ತೊಡಗಿದ್ದ ಆ ಪಕ್ಷಿಗಳಲ್ಲಿ ಒಂದನ್ನು ಅಕಾರಣವಾಗಿ, ಪಾಪನಿಶ್ಚಯನೂ ವೈರನಿಲಯನೂ ಆದ ಬೇಡ ಕೊಂದಿದ್ದಾನೆ. ಇದನ್ನು ನೋಡಿ ಧರ್ಮಾತ್ಮನಾದ ಮಹರ್ಷಿಗೆ ಧರ್ಮನಾಶವಾಯಿತಲ್ಲಾ! ಎಂದು ಕರುಣೆಯುಂಟಾಗಿದೆ. ಇದಕ್ಕೆ ಪೋಷಕವಾಗಿ ಹೊರಗಡೆ ಕ್ರೌಂಚಿಯ ಕರುಣಾಜನಕವಾದ ಆಕ್ರಂದನದ ಶಬ್ದ ಬೇರೆ ಕೇಳುತ್ತಿದೆ. ಇದು ಎರಡನೆಯ ಘಟನೆ, ಹಾಗೂ ಕಾವ್ಯದ ಜೀವಾತು ಎನಿಸುವ ಘಟನೆ. 


ಅನಂತರ ಮಹರ್ಷಿಗಳ ಹೃದಯ ಕರುಣೆಯಿಂದ ತಾನೇ ತಾನಾಗಿ ಕರಗಿದೆ. ಕರುಣಾರಸ ಹರಿಯುತ್ತಿದೆ ಮನಸ್ಸಿನಲ್ಲಿ. ಆ ಕರುಣೆಯೇ ವಾಲ್ಮೀಕಿಯ ಹೃದಯವೀಣೆಯನ್ನು ಮಿಡಿದು, ನಾದ-ಸ್ವರ-ಅಕ್ಷರ ರೂಪವಾದ ಧ್ವನಿಯನ್ನು ಹೊರಡಿಸಿದೆ. ಆದರೆ ಇಲ್ಲಿ ಅಪ್ರಯತ್ನವಾಗಿ (ಪ್ರಯತ್ನವಿಲ್ಲದೆಯೇ) ಸೀತಾರಾಮರ ಕಥೆಯೂ ತಾನೆ ಸೇರಿಹೋಗಿಬಿಟ್ಟಿದೆ. ಅಂತರ್ವಾಹಿನಿಗೆ ಬಹಿರ್ವಾಹಿನಿಯ ಸಂಗಮ. ಒಂದು ದೃಷ್ಟಿಯಿಂದ ಅಂತರಂಗದಲ್ಲಿ ಹರಿಯುತ್ತಿದ್ದ ರಾಮಕಥಾವಾಹಿನಿಗೆ ಕರುಣಾವಾಹಿನಿ ಸಂಗಮ. ಇನ್ನೊಂದು ದೃಷ್ಟಿಯಿಂದ ಮಹರ್ಷಿಗಳ ಹೃದಯದಲ್ಲಿ ಧರ್ಮನಾಶದಿಂದ ಹರಿಯುತ್ತಿದ್ದ ಕರುಣಾವಾಹಿನಿಗೆ ಕ್ರೌಂಚಿಯ ಕರುಣಾಜನಕ ದೃಶ್ಯರೂಪವಾದ ಅಶ್ರುವಾಹಿನಿಯ ಸಂಗಮ. 

\section*{ಮಹರ್ಷಿಯ ಮೈಮರೆಸಿ ಹೊರಹರಿದಿದೆ ಶ್ಲೋಕ} 

ಈ ಸಂಗಮಬಿಂದುವಿನಿಂದಲೇ ಅಪ್ರಯತ್ನವಾಗಿ, ಅಪ್ರೇರಿತವಾಗಿ ನಾದ-ಸ್ವರ-ಅಕ್ಷರ ರೂಪವಾದ ಧ್ವನಿ ಕಾವ್ಯನದಿಗೆ ನಾಂದೀ ರೂಪವಾದ ವಾಹಿನಿ ಧ್ವನಿರೂಪವಾಗಿ ಹೊರಬಿದ್ದಿದೆ. ವಾಲ್ಮೀಕಿಗಳಿಗೂ ತಿಳಿಯದು. ಈ ಅಂಶ ತಾನೇ ತಾನಾಗಿ ಹೊರಬಿದ್ದು ಉಪಶ್ಲೋಕತ್ವವನ್ನು ಹೊಂದಿದ ಶ್ಲೋಕ ಹೀಗಿದೆ.- 

\begin{shloka}
ಮಾ ನಿಷಾದ ಪ್ರತಿಷ್ಠಾಂ ತ್ವಮಗಮಃ ಶಾಶ್ವತೀಃ ಸಮಾಃ|\\ 
ಯತ್‍ ಕ್ರೌಂಚಮಿಥುನಾದೇಕಮವಧೀಃ ಕಾಮಮೋಹಿತಮ್‍||
\end{shloka}

ಅನಿರೀಕ್ಷಿತವಾಗಿ ಅಪ್ರಯತ್ನಸಾಧ್ಯವಾಗಿ ಹೊರಬಿದ್ದ ಈ ಶ್ಲೋಕದ ಮೇಲೆ ವಾಲ್ಮೀಕಿಗಳ ವಿಚಾರಧಾರೆ ಮುಂದೆ ಸಾಗಿದೆ. `ಈ ಪಕ್ಷಿಯ ಶೋಕದಿಂದ ಮನಕರಗಿದ ನನ್ನಿಂದ ಇದೇನು ಹೇಳಲ್ಪಟ್ಟಿತು?' 

\begin{shloka} 
ತಸ್ಯೈವಂ ಬ್ರುವತಶ್ಚಿಂತಾ ಬಭೂವ ಹೃದಿ ವೀಕ್ಷತಃ|\\ 
ಶೋಕಾರ್ತೆನಾಸ್ಯ ಶಕುನೇಃ ಕಿಮಿದಂ ವ್ಯಾಹೃತಂ ಮಯಾ||
\end{shloka}

ಅಂದರೆ, ವಾಲ್ಮೀಕಿಗಳ ಕೈವಾಡವೇನೂ ಶ್ಲೋಕರಚನೆಯಲ್ಲಿಲ್ಲ ಎನ್ನುವುದು ಸ್ಪಷ್ಟವಾಗಿದೆ. ವಾಲ್ಮೀಕಿಗಳನ್ನೇ ಮೈಮರೆಸಿ ಹರಿದಿದೆ ಎನ್ನುವುದಕ್ಕೆ ಇದಕ್ಕಿಂತಲೂ ಸ್ಪಷ್ಟ ನಿದರ್ಶನ ಬೇರೆ ಬೇಕೆ? 

\section*{ಅಪ್ರಯತ್ನವಾಗಿ ಹೊರಬಿದ್ದ ಶ್ಲೋಕದ ಬಗ್ಗೆ ಮಹರ್ಷಿಯ ಚಿಂತೆ} 

ಅನಂತರ ಮತ್ತೆ ಅದೇ ಚಿಂತೆ. ಇದರ ಬಗ್ಗೆ ಆಶ್ಚರ್ಯ! ಮತ್ತೆ, ಶೋಕವು ಶ್ಲೋಕರೂಪವಾಗಿ ಹೊರಬಂದಿರುವುದನ್ನು ನೋಡಿ ಮತ್ತೂ ಆಶ್ಚರ್ಯವಾಗಿದೆ. ಅದನ್ನೇ ಶಿಷ್ಯನೊಡನೆ ಹೇಳಿಕೊಂಡಿದ್ದಾರೆ. 

\begin{shloka} 
ಪಾದಬದ್ಧೋ\char'263ಕ್ಷರಸಮಃ ತಂತ್ರೀಲಯಸಮನ್ವಿತಃ|\\ 
ಶೋಕಾರ್ತಸ್ಯ ಪ್ರವೃತ್ತೋ ಮೇ ಶ್ಲೋಕೋ ಭವತು ನಾನ್ಯಥಾ||
\end{shloka}


ವಾಲ್ಮೀಕಿಗಳ ಯಾವ ಪ್ರಯತ್ನವೂ ಇಲ್ಲದೇ ಹೊರಬಿದ್ದ ಧ್ವನಿಯು, ಪಾದಬದ್ಧವಾಗಿ ಅಕ್ಷರಸಮವಾಗಿ, ತಂತ್ರೀಲಯಗಳಿಂದ ಕೂಡಿ ತಾನೇ ತಾನಾಗಿ ಹೊರಟಿದೆ. ಇದರ ಬಗ್ಗೆ ವಾಲ್ಮೀಕಿಗಳಿಗೇ ಆಶ್ಚರ್ಯವಾಗಿದೆ. 

ವಾಲ್ಮೀಕಿಗಳು ಅವನಿಗೆ ಶಾಪಕೊಡಲು ಈ ಶ್ಲೋಕ ಹೇಳಬೇಕು ಎಂದು ಶ್ಲೋಕವನ್ನು ತಯಾರು ಮಾಡಿಕೊಂಡೇನೂ ಆಶ್ರಮದಿಂದ ಬಂದಿರಲಿಲ್ಲ. ಅಥವಾ ಆಶ್ರಮದಿಂದ ಬರುವಾಗ ಅವರಿಗೆ ಶೋಕವೇನೂ ಇರಲಿಲ್ಲ. ಪ್ರಶಾಂತರಾಗಿಯೇ ಇದ್ದರು. ಈ ಘಟನೆಯೇ ಅವರ ಶ್ಲೋಕಕ್ಕೆ ಕಾರಣವಾಗಿದೆ. ಆದ್ದರಿಂದಲೇ ವಾಲ್ಮೀಕಿಗಳಿಗೆ ಆಶ್ಚರ್ಯ. ಅವರು ಪ್ರಯತ್ನಪಟ್ಟು ರಚಿಸಿದ್ದರೆ `ಕಿಮಿದಂ ವ್ಯಾಹೃತಂಮಯಾ' ಎಂಬುದಕ್ಕೆ ಅವಕಾಶವಿಲ್ಲ. ಅನಂತರ ಈ ವಿಸ್ಮಯ ಮುಂದುವರಿದಿದೆ. ಇದೇ ಸಮಸ್ಯೆಯಾಗಿದೆ. ಇದೇನು ವಿಚಾರ? ಇದ್ದಕ್ಕಿದ್ದಂತೆಯೇ ಶೋಕವು ಹಾಗೆ ಹೊರಬೀಳುವುದೆಂದರೇನು? ಬಿದ್ದುದಕ್ಕೆ ಕಾರಣವೇನು? ಎಂದು ಚಿಂತಿಸುತ್ತಲೇ ಇದ್ದಾರೆ. ಈ ಘಟನೆಯಲ್ಲಿ ಅವರಿನ್ನೂ ಸ್ನಾನವನ್ನು ಮುಗಿಸಿಲ್ಲ. ಅನಂತರ ಈ ಘಟನೆಯನ್ನು `ಧ್ಯಾನಮಾಸ್ಥಿತಃ' ಚಿಂತಿಸುತ್ತಲೇ ಸ್ನಾನವನ್ನು ಮುಗಿಸಿದ್ದಾರೆ. ಅದನ್ನೇ ಮೆಲುಕು ಹಾಕುತ್ತಾ ಆಶ್ರಮಕ್ಕೆ ಹಿಂತಿರುಗಿದ್ದಾರೆ.- 

\begin{shloka}
ಸೋ\char'263ಭಿಷೇಕಂ ತತಃಕೃತ್ವಾ ತೀರ್ಥೇ ತರ್ಸ್ಮಿ ಯಥಾವಿಧಿ|\\ 
ತಮೇವ ಚಿಂತಯನ್ನರ್ಥಂ ಉಪಾವರ್ತತ ವೈ ಮುನಿಃ||\\ 
ಸ ಪ್ರವಿಶ್ಯಾಶ್ರಮಪದಂ ಶಿಷ್ಯೇಣ ಸಹ ಧರ್ಮವಿತ್‍|\\ 
ಉಪವಿಷ್ಟಃ ಕಥಾಶ್ಚಾನ್ಯಾಶ್ಚಕಾರ ಧ್ಯಾನಮಾಸ್ಥಿತಃ||
\end{shloka}


`ಧರ್ಮವಿತ್‍', `ಧ್ಯಾನಮಾಸ್ಥಿತಃ' ಎಂಬ ಎರಡು ವಿಶೇಷಣಗಳನ್ನು ಈ ಸಂದರ್ಭದಲ್ಲಿ ಬಳಸಿರುವುದು ಬಹು ಉಚಿತವಾಗಿದೆ. 


ಇಲ್ಲಿ ಗಮನಿಸಬೇಕಾದ ಅಂಶವಿದು. ವಾಲ್ಮೀಕಿಗಳಿಗೇ ತಿಳಿಯದಂತೆ ಶೋಕ ಶ್ಲೋಕವಾಗಿ ಬಂದಿದೆ. ರಾಮಾಯಣದ ನಾಂದೀಪದ್ಯವನ್ನು ರಚಿಸಬೇಕೆಂದಾಗಲಿ, ರಾಮಾಯಣವನ್ನು ರಚಿಸಬೇಕೆಂದಾಗಲಿ, ವಾಲ್ಮೀಕಿಗಳು ರಚಿಸಿದ ಶ್ಲೋಕವೇನೂ ಇದಾಗಿಲ್ಲ. ರಾಮಾಯಣದ ನಾಂದೀಶ್ಲೋಕ ಇದು ಎನ್ನುವುದು ವಾಲ್ಮೀಕಿಗಳ ಅಭಿಪ್ರಾಯವಾಗಿದ್ದರೆ ``ಕಿಮಿದಂ ವ್ಯಾಹೃತಂ ಮಹಾ" ಎನ್ನುವುದಕ್ಕೆ ಅವಕಾಶವಿಲ್ಲ. ಮತ್ತು ಈ ಶ್ಲೋಕಬರುವ ಸಂದರ್ಭದಲ್ಲಿ ರಾಮಾಯಣದ ರಚನೆಯ ಬಗ್ಗೆ ಯಾವುದೇ ಸಂಕಲ್ಪವಾಗಲಿ, ಪ್ರಯತ್ನವಾಗಲಿ ವಾಲ್ಮೀಕಿಗಳಿಗೆ ಇತ್ತೆನ್ನುವುದಕ್ಕೆ ಯಾವ ಸೂಚನೆಯೂ ಇಲ್ಲ. ಇದು ಅಲ್ಲಿರುವ ಘಟನೆಗಳಿಂದಲೇ ಸ್ಪಷ್ಟಪಡುವ ವಿಚಾರ. ವಾಲ್ಮೀಕಿಗಳ ಆಶಯವಿಷ್ಟೆ. ಕರುಣಾಜನಕವಾದ ದೃಶ್ಯದಿಂದ ಧರ್ಮಾತ್ಮನಾದ ಮಹರ್ಷಿಗೆ ಶೋಕ ಉಕ್ಕಿಬಂತು. ಶೋಕದಿಂದ ತಾನೇ ತಾನಾಗಿ ಹರಿದು ಬಂದ ಧ್ವನಿ; ಇದೇನು ಇಷ್ಟು ಸುವ್ಯವಸ್ಥಿತವಾಗಿದೆಯಲ್ಲಾ! ಇದು ಯಾರಿಂದಾದುದು? ಹೇಗೆ ಆಯಿತು? ಎನ್ನುವ ಚಿಂತೆ ಸಹಜವೇ ಆಗಿದೆ. ಇದರಲ್ಲಿ ವಾಲ್ಮೀಕಿಗಳ ಕೈವಾಡವೇನೂ ಇಲ್ಲ ಎನ್ನುವುದು ಅವರ ಚಿಂತೆಯಿಂದಲೇ ವ್ಯಕ್ತವಾಗುತ್ತದೆ. ಘಟನೆಯ ಸನ್ನಿವೇಶದಿಂದ ಶೋಕ ಉಕ್ಕಿ ಬಂದು ಬೇಡನಿಗೆ ಶಾಪರೂಪವಾದ ಧ್ವನಿ ಎಂಬ ಉದಂತವನ್ನು-ಘಟನೆಯನ್ನು ಮಾತ್ರ ದೃಷ್ಟಿಯಲ್ಲಿ ಇಟ್ಟುಕೊಂಡು ಶ್ಲೋಕಕ್ಕೆ ವಿವರ ಹೇಳಬೇಕಾಗಿದೆ. ಇದು ಆ ಘಟನೆಗಳಿಂದ ಸಿದ್ಧವಾಗುವ ಅರ್ಥ. 

\section*{ಶ್ಲೋಕ ಹೊರಹೊಮ್ಮಿದ ವೈಚಿತ್ರ್ಯದ ಬಗ್ಗೆ ಬ್ರಹ್ಮನಲ್ಲಿ ಮಹರ್ಷಿಯ ಪ್ರಶ್ನೆ} 

ಇದಿಷ್ಟು ಹೊರಗೆ ಕಾಣುವ ಘಟನೆಗಳಿಂದ ತೆಗೆದುಕೊಳ್ಳಬಹುದಾದ ಆಶಯ. ಅನಂತರ ಇದರ ಹಿನ್ನೆಲೆಯನ್ನು ಗಮನಿಸಲು ಮುಂದಿನ ಘಟನೆಯನ್ನು ಗಮನಿಸೋಣ 


ವಾಲ್ಮೀಕಿಗಳು ಘಟನೆಯನ್ನು ಮೆಲುಕುಹಾಕುತ್ತಾ ಆಶ್ರಮಕ್ಕೆ ಬಂದು ಶಿಷ್ಯರೊಡನೆ ಕುಳಿತಿರುವ ಸಮಯಕ್ಕೆ ಸರಿಯಾಗಿ ಲೋಕಕರ್ತೃವೂ ಸ್ವಯಂಭೂವೂ ಆದ ಬ್ರಹ್ಮನ ಆಗಮನ. 


\begin{shloka}
ಆಜಗಾಮ ತತೋ ಬ್ರಹ್ಮಾ ಲೋಕಕರ್ತಾ ಸ್ವಯಂಪ್ರಭುಃ|\\ 
ಚತುರ್ಮುಖೋ ಮಹಾತೇಜಾಃ ದ್ರಷ್ಟುಂ ತಂ ಮುನಿಪುಂಗವಮ್‍||
\end{shloka}
ಇದೂ ಕಾವ್ಯರಚನಾಂಗವಾದ ಘಟನೆ. 

\begin{shloka} 
ವಾಲ್ಮೀಕಿರಥ ತಂ ದೃಷ್ಟ್ವಾ ಸಹಸೋತ್ಥಾಯ ವಾಗ್ಯತಃ|\\ 
ಪ್ರಾಂಜಲಿಃ ಪ್ರಯತೋ ಭೂತ್ವಾ ತಸ್ಥೌ ಪರಮವಿಸ್ಮಿತಃ||
\end{shloka}

ಅನಂತರ ವಾಲ್ಮೀಕಿಯ ಪ್ರತ್ಯುತ್ಥಾನ, ಪ್ರಣಾಮ, ಬ್ರಹ್ಮನ ಆಗಮನದಿಂದ ಪರಮವಿಸ್ಮಯ, ಅನಂತರ ಬ್ರಹ್ಮನಿಗೆ ಯಥೋಚಿತವಾದ ಪೂಜೆ. ಬ್ರಹ್ಮನು ಯೋಗ್ಯವಾದ ಆಸನದಲ್ಲಿ ಕುಳಿತ ನಂತರ ಅವನ ಅನುಜ್ಞೆಯಂತೆ ಆಸನಸ್ವೀಕಾರ. ಬ್ರಹ್ಮನ ಅನುಜ್ಞೆಯಂತೆ ಕುಳಿತರೂ- 

\begin{shloka} 
ಉಪವಿಷ್ಟೇ ತದಾ ತರ್ಸ್ಮಿ ಸಾಕ್ಷಾಲ್ಲೋಕಪಿತಾಮಹೇ|\\ 
ತದ್ಗತೇನೈವ ಮನಸಾ ವಾಲ್ಮೀಕಿರ್ಧ್ಯಾನಮಾಸ್ಥಿತಃ||
\end{shloka}
ಮತ್ತೆ ಅದೇ ಚಿಂತನೆ. ಮನಸ್ಸು ಮತ್ತೆ ಮತ್ತೆ ಅದನ್ನೇ ಮೆಲುಕು ಹಾಕುತ್ತಿದೆ- 

\begin{shloka}
ಪಾಪಾತ್ಮನಾ ಕೃತಂ ಕಷ್ಟಂ ವೈರಗ್ರಹಣಬುದ್ಧಿನಾ|\\ 
ಯಸ್ತಾದೃಶಂ ಚಾರುರವಂ ಕ್ರೌಂಚಂ ಹನ್ಯಾದಕಾರಣಾತ್‍||
\end{shloka}

ಪಾಪಾತ್ಮನೂ, ವೈರಗ್ರಹಣಬುದ್ಧಿಯೂ ಆದ ಆತನಿಂದ ನಡೆದುದೇನು? ಚಾರುರವವುಳ್ಳ ಕ್ರೌಂಚವನ್ನು ಅಕಾರಣವಾಗಿ ಕೊಲ್ಲುವುದೇ? 

\begin{shloka}
ಶೋಚನ್ನೇವ ಮುಹುಃ ಕ್ರೌಂಚೀಂ ಉಪಶ್ಲೋಕಮಿಮಂ ಪುನಃ|\\ 
ಜಗೌ ಅಂತರ್ಗತಮನಾಃ ಭೂತ್ವಾ ಶೋಕಪರಾಯಣಃ||
\end{shloka}

ಈ ಘಟನೆಯಿಂದ ನನಗೆ ಶೋಕವುಂಟಾಗಿ ಶ್ಲೋಕರೂಪವಾದ ಧ್ವನಿ ಬೇರೆ ನನ್ನಿಂದ ಹರಿದಿದೆ; ಇದೇನು ವಿಚಿತ್ರ! ಎಂದು ವೈಚಿತ್ರ್ಯದ ಬಗ್ಗೆ ಬ್ರಹ್ಮನೊಡನೆ ಪ್ರಶ್ನೆ. ಇಲ್ಲಿಯೂ ಹರಿವು ಮಾತ್ರ ಅಂತೆಯೇ ಇದೆ. ಮಹರ್ಷಿಗೆ ತನ್ನಿಂದ ತನಗೇ ಅರಿಯದಂತೆ ನಡೆದ ಘಟನೆಯ ಬಗ್ಗೆ ಪಶ್ಚಾತ್ತಾಪವಿದೆ. 

\section*{ಬ್ರಹ್ಮಚ್ಛಂದದಿಂದ ಹೊರಹೊಮ್ಮಿದೆ ಈ ವಾಣಿ} 

ಬ್ರಹ್ಮದೇವನ ಉತ್ತರ- 

\begin{shloka}
ತಮುವಾಚ ತತೋ ಬ್ರಹ್ಮಾ ಪ್ರಹಸನ್ಮುನಿಪುಂಗವಮ್‍|\\ 
ಶ್ಲೋಕ ಏವ ತ್ವಯಾ ಬದ್ಧಃ ನಾತ್ರ ಕಾರ್ಯಾ ವಿಚಾರಣಾ||
\end{shloka}

ಋಷಿಯೇ, ನೀನು ಮಾಡಿರುವುದು ಶ್ಲೋಕವೇ ಆಗಿದೆ. ಅದರಲ್ಲಿ ವಿಚಾರ ಮಾಡಬೇಕಾಗಿಲ್ಲ. ಈ ಶ್ಲೋಕ ನಿನ್ನ ಪ್ರಯತ್ನವಿಲ್ಲದೆ ಹೇಗೆ ಹರಿಯಿತು? ಎನ್ನುವ ಬಗೆಗೆ ಸಂಶಯವಿರಬಹುದು. 

\begin{shloka}
ಮಚ್ಛಂದಾದೇವ ತೇ ಬ್ರರ್ಹ್ಮ ಪ್ರವೃತ್ತೇಯಂ ಸರಸ್ವತೀ|
\end{shloka}

ಇದು ನನ್ನ ಛಂದದಿಂದ ಹೊರಟ ವಾಣಿ. ನಿನ್ನ ಛಂದದಿಂದ ಅಲ್ಲ. ಆದ್ದರಿಂದ ಈ ವಾಣಿ ಬ್ರಹ್ಮಚ್ಛಂದವಾಗಿದೆ. (ಛಂದಶ್ಯಾಸ್ತ್ರದ ಪ್ರಕಾರ ಧಾತೃಛಂದಸ್‍ ಎಂಬ ಛಂದಸ್ಸಿಗೆ ಸೇರುತ್ತದೆ) ಇದರಿಂದ ವಾಲ್ಮೀಕಿಗಳ ಹೃದಯದ ಅಳಲು ಆರಿತು. ತನ್ನ ಪ್ರಯತ್ನದಿಂದ ಬಂದ ವಾಣಿಯಲ್ಲವಾದ್ದರಿಂದ ಪಶ್ಚಾತ್ತಾಪಕ್ಕೆ ಕಾರಣವಿಲ್ಲ ತನಗೇ ಅರಿಯದಂತೆ ಬ್ರಹ್ಮ ತನ್ನನ್ನಾಡಿಸಿದ್ದಾನೆ. ಆಡಿಸಿದ ಬ್ರಹ್ಮನೇ ನಾನು ಆಡಿಸಿದವನು ಎಂದು ಹೇಳುತ್ತಿದ್ದಾನೆ. ಇನ್ನು ಪಶ್ಚಾತ್ತಾಪವೇಕೆ? 

\section*{ಬ್ರಹ್ಮನ ಆದೇಶದಂತೆ ಧರ್ಮದಿಂದ ಗತಿಯನ್ನು ಅನ್ವೇಷಿಸಿ ಕಾವ್ಯವನ್ನು ಹೊರತಂದಿದ್ದಾರೆ} 

`ಏತಕ್ಕೆ ಈ ರೀತಿ ನುಡಿಸಿದೆ?' ಎಂದರೆ ನೀನು ರಾಮನ ಚರಿತ್ರೆಯನ್ನು ನಾರದರಿಂದ ಕೇಳಿದಂತೆ ಬರಿ. 

\begin{shloka} 
ರಾಮಸ್ಯ ಚರಿತಂ ಸರ್ವಂ ಕುರು ತ್ವಮೃಷಿಸತ್ತಮ|\\ 
ಧರ್ಮಾತ್ಮನೋ ಗುಣಮತಃ ಲೋಕೇ ರಾಮಸ್ಯ ಧೀಮತಃ||\\ 
ವೃತ್ತಂ ಕಥಯ ವೀರಸ್ಯ ಯಥಾ ತೇ ನಾರದಾಚ್ಛುತಮ್‍|
\end{shloka}

ನಿನ್ನ ರಾಮಾಯಣವು ಚಿರಸ್ಥಾಯಿಯಾಗಿರುವುದು ಎಂದು ಕಾವ್ಯರಚನೆಗೆ ಆದೇಶವನ್ನೂ ಸಂದೇಶವನ್ನೂ ನೀಡಿದ್ದಾರೆ. ಕಾವ್ಯರಚನೆಗೆ ಹಿನ್ನೆಲೆ ಇಲ್ಲಿ ಹುಟ್ಟಿಕೊಂಡಿದೆ. ವಾಲ್ಮೀಕಿಯು ಕಾವ್ಯ ರಚಿಸಬೇಕೆಂದು ಬ್ರಹ್ಮನ ಆಶಯವಾಗಿದೆ. ಆದುದರಿಂದಲೇ ಬ್ರಹ್ಮನ ಆಶಯದಿಂದ ಶೋಕವು ಶ್ಲೋಕವಾಗಿ ಹರಿದಿದೆ. ಕಾವ್ಯದ ನಾಂದೀ(ಮಂಗಳ) ಶ್ಲೋಕವು ಬ್ರಹ್ಮಚ್ಛಂದದಿಂದ ಬಂದಿದೆ; ವಾಲ್ಮೀಕಿಯ ಛಂದದಿಂದಲ್ಲ ಎಂಬುದನ್ನು ಗಮನಿಸಿದ ಮೇಲೆ, ಇನ್ನು ಬ್ರಹ್ಮನ ಆಶಯವೇನು? ಅದು ರಾಮಾಯಣದ ಮಂಗಳಶ್ಲೋಕ ಹೇಗೆ ಆಗುತ್ತದೆ? ಎನ್ನುವ ಪ್ರಶ್ನೆ ಉಳಿಯುತ್ತದೆ. ಅನಂತರ ವಾಲ್ಮೀಕಿಗಳು ಬ್ರಹ್ಮನ ಅಭಿಪ್ರಾಯವನ್ನು ತೆಗೆದುಕೊಂಡು ಯೋಗಸ್ಥಿತಿಯಲ್ಲಿ ರಾಮಾಯಣವನ್ನು ನೋಡಿದ್ದಾರೆ- 

\begin{shloka}
ತತಃ ಪಶ್ಯತಿ ಧರ್ಮಾತ್ಮಾ ತತ್ಸರ್ವಂ ಯೋಗಮಾಸ್ಥಿತಃ|\\ 
ಪುರಾ ಯತ್ತತ್ರ ನಿರ್ವೃತ್ತಂ ಪಾಣಾವಾಮಲಕಂ ಯಥಾ|| 
\end{shloka}
ಹಿಂದೆ ನಡೆದುಹೋದುದೂ ಹಾಗೆಯೇ ಆದ್ದರಿಂದಲೇ ಋಷಿಸದಸ್ಸಿನ ಪ್ರಶಂಸೆ- 

\begin{shloka}
ಚಿರನಿರ್ವೃತ್ತಮಪ್ಯೇತತ್‍ ಪ್ರತ್ಯಕ್ಷಮಿವ ದರ್ಶಿತಮ್‍||
\end{shloka}
ಎಂದು ಎರಡಕ್ಕೂ ಅವಿರೋಧ. ಧರ್ಮದಿಂದ ಗತಿಯನ್ನು ಅನ್ವೇಷಿಸಿ ರಾಮಾಯಣವನ್ನು ಹೊರತಂದಿದ್ದಾರೆ- 

\begin{shloka}
ಪ್ರಾಚೀನಾಗ್ರೇಷು ದರ್ಭೆಷು ಧರ್ಮೆಣಾನ್ವೀಕ್ಷತೇ ಗತಿಮ್‍||
\end{shloka}

\section*{ರಾಮಾಯಣ ರಚನೆಗೆ ಬ್ರಹ್ಮಸಂಕಲ್ಪವೇ ಕಾರಣ} 

ಮತ್ತೆ ಬ್ರಹ್ಮನ ಅಭಿಪ್ರಾಯವನ್ನು ಹುಡುಕಲು ಘಟನೆಗಳನ್ನು ಜ್ಞಾಪಿಸಿಕೊಳ್ಳಬೇಕು. ವಾಲ್ಮೀಕಿಗಳ ಆಶ್ರಮಕ್ಕೆ ನಾರದರು ಬಂದರು. ವಾಲ್ಮೀಕಿಗಳಿಗೆ ಅತಿಶಯವಾದ ಕುತೂಹಲವುಂಟು. ಅದಕ್ಕೆ ಅನುಗುಣವಾಗಿ ಪ್ರಶ್ನಿಸಿ ನಾರದರಿಂದ ರಾಮಕಥಾ ರೂಪವಾದ ಉತ್ತರ ಪಡೆದರು. ಇದು ಒಂದನೆಯ ಘಟ್ಟ. ಎರಡನೆಯದು ತಮಸಾತೀರದಲ್ಲಿ ನಡೆದ ಘಟನೆ, ಕಾವ್ಯಕ್ಕೆ ಜೀವಾತುವಾದ ಘಟನೆ `ಕ್ರೌಂಚವಧ'. ಆಗ ಉಂಟಾದ ಕರುಣೆ, ಅಪ್ರಯತ್ನವಾಗಿ ವಾಲ್ಮೀಕಿಗಳ ಶೋಕವು ಶ್ಲೋಕರೂಪವಾಗಿ ಹರಿದುದು, ಅನಂತರದ ಚಿಂತೆ, ಇದು ಎರಡನೆಯ ಘಟ್ಟ. ಬ್ರಹ್ಮನ ಆಗಮನ ಹಾಗೂ ಶೋಕವು ಶ್ಲೋಕವಾದುದರ ರಹಸ್ಯ ಪ್ರಕಾಶ, ಕಾವ್ಯ ರಚನೆಗೆ ಆದೇಶ, ಇದು ಮೂರನೆಯ ಘಟ್ಟ. ವಾಲ್ಮೀಕಿಗಳು ಅನಂತರ ಯೋಗಾರೂಢರಾಗಿ ಧರ್ಮದಿಂದ ರಾಮನ ಚರಿತ್ರೆಯನ್ನು ಅನ್ವೇಷಿಸಿ ರಾಮಾಯಣವನ್ನು ರಚಿಸಿದರು; ಇದು ನಾಲ್ಕನೆಯ ಘಟ್ಟ. ಈ ನಾಲ್ಕು ಘಟನೆಗಳನ್ನು ಗಮನಿಸಿ ಇವುಗಳಿಗೆ ಅವಿರೋಧವಾಗಿ ಶ್ಲೋಕವನ್ನು ತೆಗೆದುಕೊಳ್ಳಬೇಕಾಗಿದೆ. 


ಬಾಹ್ಯದೃಷ್ಟಿಯಿಂದ ನೋಡಿದಾಗ ಈ ನಾಲ್ಕು ಘಟನೆಗಳಿಗೂ ಒಂದಕ್ಕೊಂದಕ್ಕೆ ಸಂಬಂಧವಿಲ್ಲದಂತೆ ತೋರಿದರೂ ಅಂತರಂಗವಾಗಿ ಒಂದು ಶಕ್ತಿಯ ಕೈವಾಡ ಎಲ್ಲ ಕಡೆಯಲ್ಲಿಯೂ ಅನುಸ್ಯೂತವಾಗಿ ಹರಿಯುವುದು ಕಾಣುತ್ತದೆ. ಬ್ರಹ್ಮನ ಆಗಮನಕ್ಕೆ ಮುಂಚೆ ರಾಮಾಯಣದ ರಚನೆಯ ಬಗ್ಗೆಯಾಗಲಿ ವಾಲ್ಮೀಕಿಗಳೇ ಏನನ್ನೂ ಅರಿಯರು. `ತಾನೊಂದೆಣಿಸಿದರೆ ದೈವ ಒಂದೆಣಿಸಿತು' ಎಂಬಂತೆ, ತಾನು ಪಕ್ಷಿಯನ್ನು ನೋಡಿ ಕನಿಕರ ಪಟ್ಟರೆ, (ಕರುಣೆ ತೋರಿದರೆ) ಆ ಕರುಣೆಯಿಂದಲೇ ಕಾವ್ಯಗಾನವನ್ನು ಹಾಡಿಸಿದೆ ದೈವ. ಆದ್ದರಿಂದ ಬ್ರಹ್ಮಸಂಕಲ್ಪವೇ ವಾಲ್ಮೀಕಿಯ ಕಾವ್ಯರಚನೆಗೆ ಕಾರಣ ಎನ್ನುವುದು ಸುಸ್ಪಷ್ಟ. ಲೋಕಕರ್ತನಾದ ಸ್ವಯಂ ಪ್ರಭುವಿನ ಆಶಯದಿಂದ ಹೊರಹೊಮ್ಮಿದೆ ನಾಂದಿ. ಆತನ ಆಶಯಕ್ಕೆ ಅವಿರೋಧವಾಗಿ ಅದನ್ನು ತೆಗೆದುಕೊಳ್ಳಬೇಕಾಗಿದೆ. 

\section*{ವ್ಯಾಧಶಾಪರೂಪವಾದ ಅಮಂಗಲದಿಂದ ಕಾವ್ಯಾರಂಭ ಉಚಿತವೇ?} 

ಕಾವ್ಯ ರಚಿಸುವಾಗ ಅದರಲ್ಲಿ ಆದಿಮಂಗಳ-ಮಧ್ಯಮಂಗಳ-(ಅಂತ್ಯ) ಸಮಾಪ್ತಿ ಮಂಗಳ ಮಾಡಿತಾನೇ ರಚಿಸಬೇಕು. ಹೀಗಿರುವಾಗ, ಕಾವ್ಯದ ಆದಿಯಲ್ಲೇ ವ್ಯಾಧಶಾಪರೂಪವಾದ ಅಮಂಗಲದಿಂದ ಕಾವ್ಯ ಆರಂಭವಾಗಬಹುದೇ? ಎಂಬ ಪ್ರಶ್ನೆ ಏಳಬಹುದು. ಹಾಗೆಯೇ ರಾಮಾಯಣದ ಆರಂಭವನ್ನು ಮಂಗಲದಿಂದ ಬೇರೆ ರೀತಿ ಆರಂಭವಾಗುವಂತೆ ಮಾಡಬಹುದಿತ್ತಲ್ಲ! ಎಂದರೆ ವಾಲ್ಮೀಕಿಗಳು ಹೇಗೋ ಪ್ರಶಾಂತಚಿತ್ತರಾಗಿದ್ದರು. ಅದಕ್ಕನುಗುಣವಾಗಿ ಯಾರಾದರೂ ``ಕಲ್ಯಾಣೋಲ್ಲಾಸ ಸೀಮಾ" ಹಾಡಿಬಿಟ್ಟರೆ, ಅದನ್ನೇ ಪ್ರೇರೇಪಕವಾಗಿ ದೈವ ಸೂಚಿಸಬಹುದಿತ್ತಲ್ಲ! ಎಂದರೆ ಸರಿ, ಕ್ರೌಂಚವಧವೇ ನಡೆಯದಿದ್ದರೆ ವಾಲ್ಮೀಕಿಗಳಿಗೆ ದೈವಸೂಚನೆ ಹೇಗೆ ಬರುತ್ತಿತ್ತೋ? ಆಗಲಿ, ದೈವ ತನ್ನ ಒಂದು ಈ ನಿಮಿತ್ತದ ಮೂಲಕ ಹೊರಪಡಿಸಿದೆ. ಪ್ರಸಾದಮುಖವಾಗಿ ಆಗಬಹುದಿತ್ತಲ್ಲ! ಏಕೆ ಇಂತಹ ಘಟನೆ ಎಂದರೆ ವಿಷಾದದಿಂದ ಆರಂಭಿಸಿ ಪ್ರಸಾದದಲ್ಲಿ ನಿಲ್ಲಬೇಕು; ವಿಷಾದವು ಪ್ರಸಾದದಲ್ಲಿ ಪರ್ಯವಸಾನಗೊಳ್ಳಬೇಕು. 

\section*{``ಮಾ ನಿಷಾದ" ವನ್ನು ನಾರಾಯಣಪರವಾಗಿ ಅರ್ಥಮಾಡಬಹುದೇ?} 

ಆರಂಭವಾದುದು ಸರಿ. ಈ ಆರಂಭಕ್ಕೂ ರಾಮಾಯಣಕ್ಕೂ ಏನು ಸಂಬಂಧ? ಎನ್ನುವ ಪ್ರಶ್ನೆ ಉಳಿದಿದೆ. ಧರ್ಮಾತ್ಮನಾದ ವಾಲ್ಮೀಕಿಮಹರ್ಷಿಯು ಧರ್ಮನಾಶಕ್ಕೋಸ್ಕರ ಶೋಕಪಡಬೇಕಾಯಿತು. ಈ ಶೋಕಸಂದರ್ಭದಲ್ಲಿ ಬ್ರಹ್ಮಚ್ಛಂದದಿಂದ ಶೋಕವು ತಂತ್ರೀಲಯ ಸಮನ್ವಿತವಾಗಿ, ಪಾದಬದ್ಧವಾಗಿ, ಅಕ್ಷರಸಮನಾದ ಶ್ಲೋಕವಾಗಿ ಹೊರಟಿದೆ. ಸಂದರ್ಭಕ್ಕೆ ಸರಿಯಾಗಿ ದೈವ ಪ್ರವೇಶಿಸಿದೆ. ಇಲ್ಲಿ ಛಂದಸ್ಸು, ಪಾದಬದ್ಧತೆ, ಅಕ್ಷರಸಮತೆ, ತಂತ್ರೀಲಯಗಳ ಸಮನ್ವಯ ಇವುಗಳು ಬ್ರಹ್ಮನ ಛಂದದಿಂದ ಹೇಗೆ ಹೊರಟಿತೋ, ಹಾಗೆಯೇ ಮಹರ್ಷಿಯ ಆಶಯದ ಜೊತೆಗೆ ಬ್ರಹ್ಮನ ಆಶಯವೂ ಕೂಡಿಕೊಂಡಿದೆ. ಬ್ರಹ್ಮನ ಆಶಯ ಕೂಡಿಕೊಂಡಿದೆ ಎಂದ ಮೇಲೆ `ಮಾ ನಿಷಾದ......' ಎನ್ನುವುದಕ್ಕೆ ನಾರಾಯಣ ಪರವಾಗಿ ಅರ್ಥ ಮಾಡಬಹುದಲ್ಲ; ಎಂದರೆ, ಮತ್ತೆ ಪಕ್ಷಿಯ ಘಟನೆಗೆ ವಿರೋಧ ಬರುತ್ತದೆ. ನಾರಾಯಣನ ಪರವಾಗಿ ಅರ್ಥಮಾಡಿದರೆ ಬೇಡನನ್ನು ನಾರಾಯಣನೆನ್ನಬೇಕು. ಆಗ ಪಾಪನಿಶ್ಚಯ, ವೈರನಿಲಯ ಎನ್ನುವುದು ಹೊಂದುವುದಿಲ್ಲ. ಅಂತೆಯೇ ನಡೆದ ಘಟನೆಯನ್ನು ಅಧರ್ಮವೆಂದು ನಿಶ್ಚಯಿಸಿ ವಾಲ್ಮೀಕಿಗಳು ದುಃಖ ಪಡುವಂತೆಯೂ ಇಲ್ಲ. ನಾರಾಯಣನನ್ನು ಸ್ತುತಿಸಿದರೆ ವಾಲ್ಮೀಕಿಗಳು ದುಃಖಪಡುತ್ತಾರೆಯೇ? ಮತ್ತು `ಮಾ ನಿಷಾದ' ಎನ್ನುವಾಗ, ಲಕ್ಷ್ಮೀ ಎನ್ನುವ `ಮಾ' ಉಚ್ಚಾರಣೆಗೂ, `ಬೇಡ' ಎನ್ನುವ ನಿಷೇಧಾರ್ಥಕ ಪದದ ಉಚ್ಚಾರಣೆಗೂ ತುಂಬಾ ವ್ಯತ್ಯಾಸವಿದೆ. ಈ ದೃಷ್ಟಿಯಿಂದಲೂ ಆ ವ್ಯಾಖ್ಯಾನ ಸರಿಹೊಂದುವುದಿಲ್ಲ. 

\section*{``ಮಾ ನಿಷಾದ" ವು ಕಾವ್ಯದ ಮಂಗಲಶ್ಲೋಕವಾದುದರ ಔಚಿತ್ಯ} 

ಹಾಗಾದರೆ ಇದು ರಾಮಾಯಣಕ್ಕೆ ಮಂಗಲಶ್ಲೋಕವಾಗುವುದು ಹೇಗೆ? ಎಂದರೆ ಘಟನೆಯನ್ನು ಮತ್ತೆ ಜ್ಞಾಪಿಸಿಕೊಳ್ಳೋಣ. ಧರ್ಮಸಂಕಟ ಒದಗಿತ್ತು. ಅಧರ್ಮವನ್ನು ನೋಡಿ, ಅದರ ನಾಶಕ್ಕೋಸ್ಕರ ವಾಲ್ಮೀಕಿಯ ಬಾಯಿಂದ ಗಾನ ಹೊರಟಿತು. ಅಧರ್ಮನಾಶವೇ ಅಭೀಪ್ಸಿತ. ಧರ್ಮ ಉಳಿಯಬೇಕಾದರೆ ಅಧರ್ಮ ನಾಶವಾಗಬೇಕು. ಅಧರ್ಮನಾಶವು ಸೃಷ್ಟಿಕರ್ತನಿಗೂ ಅಭಿಮತ. ಸೃಷ್ಟಿಕರ್ತನ ಸೃಷ್ಟಿಗೆ ಒಂದು ಸ್ಥಿತಿಬೇಕಾದರೆ ಅಧರ್ಮನಾಶವಾಗಬೇಕು. ಈ ಅಧರ್ಮನಾಶವಾಗಬೇಕು ಎನ್ನುವುದು ಬ್ರಹ್ಮನ-ಸೃಷ್ಟಿಕರ್ತನ ಆಶಯವಾಗಿದೆ. ಅಂತೆಯೇ ರಾಮನ ಅವತಾರವೂ ಕೂಡ ಧರ್ಮವನ್ನು ಸಂಕಟಕ್ಕೀಡುಮಾಡಿದ ರಾವಣನ ನಾಶಕ್ಕೋಸ್ಕರವೇ ಆಗಿದೆ. ಅಧರ್ಮವನ್ನು ನಾಶಮಾಡಿ, ಧರ್ಮವನ್ನು ಉಳಿಸಲೋಸುಗವೇ ರಾಮಾವತಾರ. ಅಂತಹ ರಾಮಚರಿತೆಯನ್ನು ವಿವರಿಸುವ ರಾಮಾಯಣದ ಆರಂಭವನ್ನು ಅಧರ್ಮನಾಶ ಸೂಚಕವಾದ ನಾಂದಿಯಿಂದಲೇ ಆರಂಭಮಾಡಿರುವುದರಲ್ಲಿ ಔಚಿತ್ಯವಿದೆ. ಆದ್ದರಿಂದಲೇ ನಿಷಾದನಿಗೆ ಪ್ರತಿಷ್ಠೆ ಹೇಗಿಲ್ಲವೋ ಅಂತೆಯೇ ಅಧರ್ಮಕ್ಕೂ ಪ್ರತಿಷ್ಠೆ ಇರಬಾರದು. ಆ ಅಧರ್ಮದ ಅಪ್ರತಿಷ್ಠೆಯನ್ನು ಹೇಳುವ ಶ್ಲೋಕವು ಎಲ್ಲ ದೃಷ್ಟಿಯಿಂದಲೂ ರಾಮಾಯಣದ ನಾಂದೀಶ್ಲೋಕವಾಗಲು ಉಚಿತವಾಗಿದೆ. ಆದ್ದರಿಂದಲೇ ರಾಮಾಯಣದ ಹೆಸರು ಪೌಲಸ್ತವಧ. 

\begin{shloka} 
`ಪೌಲಸ್ತವಧಮಿತ್ಯೇವ ಚಕಾರ ಚರಿತವ್ರತಃ|\\ 
ದಶಶಿರಸಶ್ಚ ವಧಂ ನಿಶಾಮಯಧ್ವಮ್‍|'
\end{shloka}

ಅಧರ್ಮನಾಶವೇ ರಾಮಾಯಣದ ಸಾರಸರ್ವಸ್ವ, ಮಹರ್ಷಿಯ ಮನೋ ಮೂಲದಿಂದ ಅಧರ್ಮನಾಶಕ್ಕೋಸ್ಕರವಾಗಿ ಶೋಕ ಹೇಗೆ ಹರಿಯಿತೋ, ಅಂತೆಯೇ, ಅದು ಲೋಕಕರ್ತನಾದ ಪ್ರಭುವಿನ ಛಂದವನ್ನೂ ಹೊತ್ತು ಶ್ಲೋಕವಾಗಿ ಪೌಲಸ್ತವಧದ ಮೂಲವೂ ಆಗಿದೆ. ಅಧರ್ಮನಾಶವಾದರೆ ಧರ್ಮದಸ್ಥಿತಿ. ಅನಂತರ ``ಧರ್ಮೊ

ವಿಶ್ವಸ್ಯ ಜಗತಃ ಪ್ರತಿಷ್ಠಾ" ಧರ್ಮವು ಅಬಾಧಿತವಾಗಿ ನಿಲ್ಲಬಲ್ಲದು. ಆದ್ದರಿಂದ ಅಧರ್ಮನಾಶವೂ ಮಂಗಲವೇ. ಈ ದೃಷ್ಟಿಯಿಂದಲೂ ಇದು ನಾಂದೀಶ್ಲೋಕವಾಗಲು ಅರ್ಹವಾಗಿದೆ. 

\section*{ಅಧರ್ಮನಾಶ, ಧರ್ಮರಕ್ಷಣೆ-ರಾಮಾಯಣದ ಗುರಿ} 

ಬೇಡ ಬಾಣವನ್ನು ಬಿಟ್ಟಾಗ ಹಕ್ಕಿಯು ಮರಣಹೊಂದಿತು. ಹೀಗೆಯೇ ಲಕ್ಷ್ಮಣ ಬಾಣಬಿಟ್ಟಿದ್ದರೂ ಇಂದ್ರಜಿತ್‍ ಸಾಯುತ್ತಿದ್ದ. ಆದರೂ- 


\begin{shloka}
ಧರ್ಮಾತ್ಮಾ ಸತ್ಯಸಂಧಶ್ಚ ರಾಮೋ ದಾಶರಥಿರ್ಯದಿ|\\ 
ಪೌರುಷೇ ಚಾಪ್ರತಿದ್ವಂದ್ವಃ ಶರೈನಂ ಜ{ಹಿ} ರಾವಣಿಮ್‍||
\end{shloka}

ಎಂಬುದಾಗಿ ಬಾಣಬಿಡುವಾಗ ಆದೇಶ ಬೇರೆ ಏಕೆ? ಧರ್ಮರಕ್ಷಣೆಯೇ ರಾಮಾಯಣದ ಗುರಿ. ಇದು ಎಲ್ಲೆಲ್ಲೂ ಹರಿದಿದೆ. ಅಧರ್ಮನಾಶ, ಧರ್ಮರಕ್ಷಣೆ ಇದು ಮಹರ್ಷಿಗೂ ಅಭೀಪ್ಸಿತ, ಅಂತೆಯೇ ಸೃಷ್ಟಿಕರ್ತನಿಗೂ ಅಭೀಪ್ಸಿತ. ಆದುದರಿಂದಲೇ ಸೃಷ್ಟಿಕರ್ತನ ಛಂದದಿಂದ ಹೊರಟ ರಾಮಾಯಣ ಸೃಷ್ಟಿಕರ್ತನಿಗೂ ಸಮ್ಮತವಾಗಿದೆ. 

\section*{ರಾಮಾಯಣವನ್ನು ಅರಿಯಲು ಮಹರ್ಷಿಹೃದಯ ಬೇಕು} 

ಆದುದರಿಂದಲೇ ರಾಮಾಯಣ ಲೋಕವು ಇರುವವರೆಗೂ ಇರಬೇಕು- 

\begin{shloka} 
ಯಾವತ್‍ ಸ್ಥಾಸ್ಯಂತಿ ಗಿರಯಃ ಸರಿತಶ್ಚ ಮಹೀತಲೇ|\\ 
ತಾವದ್ರಾಮಾಯಣಕಥಾ ತ್ವತ್ಕೃತಾ ಪ್ರಚರಿಷ್ಯತಿ||
\end{shloka}

ಗಿರಿ ನದಿಗಳು ಲೋಕವಿರುವವರೆಗೂ ಇದ್ದೇ ಇರುತ್ತವೆ. ರಾಮಾಯಣವೂ ಲೋಕವಿರುವವರೆಗೂ ಇದ್ದೇ ಇರುತ್ತದೆ. ಅಧರ್ಮನಾಶವಾಗಿ ಧರ್ಮವು ನೆಲೆ ನಿಲ್ಲುವುದೇ ಮಹರ್ಷಿಗಳ ಆಶಯವಾಗಿರುವುದರಿಂದ ಋಷಿಸದಸ್ಸಿನಲ್ಲಿಯೂ ಅದಕ್ಕೆ ಸಾದರಪುರಸ್ಕಾರ. ಹೀಗೆ ರಾಮಾಯಣ ಭೌತಿಕ-ದೈವಿಕ-ಆಧ್ಯಾತ್ಮಿಕವಾದ ಹಿನ್ನೆಲೆಗಳಿಂದ ಹೊರಟು, ಧರ್ಮದ ರಕ್ಷಣೆಯನ್ನು ಸಾರಿರುವುದನ್ನು ತಿಳಿಯಲು ಮಹರ್ಷಿಗಳ ಹೃದಯವೇ ಬೇಕು. ಇಂತಹ ರಾಮಾಯಣ ಶಾಶ್ವತವಾದ ವಿಷ್ಣುವಿನ ಪರಮಪದವನ್ನು ಕೊಡುವುದರಲ್ಲಿ ಏನು ಆಶ್ಚರ್ಯ! 

\begin{shloka} 
ಯಃ ಕರ್ಣಾಂಜಲಿಸಂಪುಟೈರಹರಹಃ ಸಮ್ಯಕ್ಪಿಬತ್ಯಾದರಾತ್‍\\ 
ವಾಲ್ಮೀಕೇರ್ವದನಾರವಿಂದಗಲಿತಂ ರಾಮಾಯಣಾಖ್ಯಂ ಮಧು|\\ 
ಜನ್ಮವ್ಯಾಧಿಜರಾವಿಪತ್ತಿಮರಣೈರತ್ಯಂತಸೋಪದ್ರವಂ\\ 
ಸಂಸಾರಂ ಸ ವಿಹಾಯ ಗಚ್ಛತಿ ಪುರ್ಮಾ ವಿಷ್ಣೋಃ ಪದಂ ಶಾಶ್ವತಮ್‍||\\ 
ಶೃಂಗಾರಂ ಕ್ಷಿತಿನಂದನಾವಿಹರಣೇ ವೀರಂ ಧನುರ್ಭಂಜನೇ\\ 
ಕಾರುಣ್ಯಂ ಬಲಿಭುಙ್ಮುಖೇ\char'263ದ್ಭುತರಸಂ ಸಿಂಧೌ ಗಿರಿಸ್ಥಾಪನೇ|\\ 
ಹಾಸ್ಯಂ ಶೂರ್ಪಣಖಾಮುಖೇ ಭಯಮಘೇ ಬೀಭತ್ಸಮಾಜೇರ್ಮುಖೇ\\ 
ರೌದ್ರಂ ರಾವಣಭಂಜನೇ ಮುನಿಜನೇ ಶಾಂತಂ ವಪುಃ ಪಾತು ವಃ||\\ 
ಅಭ್ಯಷಿಂರ್ಚ ನರವ್ಯಾಘ್ರಂ ಪ್ರಸನ್ನೇನ ಸುಗಂಧಿನಾ|\label{208}\\ 
ಸಲಿಲೇನ ಸಹಸ್ರಾಕ್ಷಂ ವಸವೋ ವಾಸವಂ ಯಥಾ|| 
\end{shloka}

\begin{center} 
(ಈ ಶ್ಲೋಕಗಳನ್ನು ಕೊನೆಯಲ್ಲಿ ಗಾನಮಾಡಿದರು.) 
\end{center} 
