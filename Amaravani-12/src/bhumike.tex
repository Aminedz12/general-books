\newpage

\begin{center}
{\Large\bf ಭೂಮಿಕೆ}
\end{center}
 
 ಅನ್ವರ್ಥನಾಮಧೇಯದಿಂದ ಬೆಳಗುತ್ತಿರುವ `ಅಮರವಾಣೀ' ಗ್ರಂಥದ ಹನ್ನೆರಡನೆಯ ಸಂಪುಟವು ಈ ಮೂಲಕ ಪ್ರಕಾಶಿತವಾಗುತ್ತಿದೆ. ಇಲ್ಲಿ ವೇದಾಂಗಗಳು-ದರ್ಶನ-ಇತಿಹಾಸ-ಪುರಾಣ ಇತ್ಯಾದಿ ವಿಷಯಗಳಿಗೆ ಸಂಬಂಧಪಟ್ಟಂತೆ ಮೌಲಿಕವಾದ ದೃಷ್ಟಿಯನ್ನು ತೆಗೆಯುವಂತಹ ಶ್ರೀರಂಗಮಹಾಗುರುಗಳ ಜ್ಞಾನ-ವಿಜ್ಞಾನಮಯವಾದ ಪ್ರವಚನಗಳ ಸಂಕಲನವಿದೆ. ಶ್ರೀರಂಗ ಮಹಾಗುರುಗಳು ತಮ್ಮ ಶಿಷ್ಯರನ್ನು ಕುರಿತು ಬೇರೆ ಬೇರೆ ಸಂಧರ್ಭಗಳಲ್ಲಿ ನೀಡಿದ ಪ್ರವಚನಗಳು ಇವು. ಮಾತುಗಳು ಒಡಲಾದರೆ ಮಾತುಗಳ ಹಿಂದೆ ತುಂಬಿಹರಿಯುವ ಭಾವವೇ ಉಸಿರು. ವಾಕ್ಕಿಗೆ ಪರಾ, ಪಶ್ಯಂತೀ, ಮಧ್ಯಮಾ ಮತ್ತು ವೈಖರೀ ಎಂಬ ನಾಲ್ಕು ಅವಸ್ಥೆಗಳಿರುತ್ತವೆ. ನಾಲ್ಕನೆಯ ಅವಸ್ಥೆಯಾದ ವೈಖರೀಕ್ಷೇತ್ರವು ಮಾತ್ರ ಸಾಮಾನ್ಯ ಮನುಷರಿಗೆ ಗೋಚರವಾಗುತ್ತದೆ. ಉಳಿದ ಮೂರು ಅವಸ್ಥೆಗಳ ಅನುಭವಪೂರ್ಣವಾದ ಅರಿವು ಜ್ಞಾನವಿಜ್ಞಾನ ಸಂಪನ್ನರಾದ ಯೋಗಿಗಳಿಗೆ ಮಾತ್ರ ಲಭ್ಯವಾದುದು. ಸೃಷ್ಟಿಮೂಲವಾದ ಪರಾತ್ಪರತತ್ತ್ವವೇ ಮಾತಿನ ಮೂಲವಾಗಿರುತ್ತದೆ. ಆ ಮೂಲತತ್ತ್ವದ ಪೂರ್ಣ ಅನುಭವವನ್ನು ತಮ್ಮ ಅಂತರಂಗದಲ್ಲಿ ತುಂಬಿಕೊಂಡು ಜ್ಞಾನ-ವಿಜ್ಞಾನ ತೃಪ್ತರಾದ ಮಹಾಪುರುಷರ ಮಾತುಗಳು ಬ್ರಹ್ಮಚೈತನ್ಯ ಪೂರ್ಣವಾಗಿರುತ್ತವೆ. ಅಂತಹ ಜ್ಞಾನ-ವಿಜ್ಞಾನ ತೃಪ್ತಾತ್ಮರಾಗಿ ಯೋಗೇಶ್ವರರೂ ಯೋಗೀಶ್ವರರೂ ಆದ ಶ್ರೀರಂಗಮಹಾಗುರುಗಳ ಪ್ರವಚನಗಳು ಸೃಷಿಟರಹಸ್ಯಗಳನ್ನು ತೆರೆದು ಕೊಡುತ್ತಾ ಪ್ರತಿಯೊಮ್ದು ವಿಷಯದ ಶಾಶ್ವತ ಸತ್ಯವಾದ ಮೂಲತತ್ತ್ವದ ಕಡೆಗೆ ದಿಗ್ದರ್ಶನ ಮಾಡುತ್ತವೆ. ಶ್ರೀರಂಗ ಮಹಾಗುರುಗಳು ಅತ್ಯಂತ ಸರಳವಾದ, ಆದರೆ, ಅಷ್ಟೇ ಗಂಭೀರವಾದ ಶೈಲಿಯಿಂದ ವಿಷಯಗಳನ್ನು ಮನಮುಟ್ಟುವಂತೆ ಪ್ರವಚನಮಾಡಿ ತಿಳಿಸುತ್ತಿದ್ದರು. ಪ್ರತ್ಯಕ್ಷವಾಗಿ ಅವರ ಪ್ರವಚನಗಳನ್ನು ಕೇಳಿದ ಸುಕೃತಿಗಳಾದ ಶಿಷ್ಯರಿಗೆ ಆ ಮಾತುಗಳನ್ನು ಸ್ಮರಿಸಿದರೆ ಇಂದಿಗೂ ಮೌಲಿಕವಾದ ಸ್ಪೂರ್ತಿಯು ತುಂಬಿ ಬರುತ್ತದೆ. ಲೌಕಿಕವಾದ ದೃಷ್ಟಾಂತಗಳನ್ನೇ ನಿರೂಪಿಸುತ್ತಾ ಶ್ರೀಗುರುಗಳು ಮೌಲಿಕವಾದ ವಿಷಯದ ಪರಿಚಯವು ಆಬಾಲವೃದ್ಧರಿಗೂ ಸುವೇದ್ಯವಾಗುವಂತೆ ಪ್ರತಿಪಾದಿಸುತ್ತಿದ್ದರು. ಅವರು ಪ್ರವಚನ ಮಾಡುತ್ತಿರುವಾಗ ಪ್ರವಚನದ ವಿಷಯವು ಮೂರ್ತಿವೆತ್ತ್ ಗೋಚರ ವಾಗುತ್ತಿರುವಂಟೆ ನಮಗೆ ಭಾಸವಾಗುತ್ತಿತ್ತು. ಮಹರ್ಷಿಗಳ ಮಾತುಗಳ ನೆಲೆಯು ಭೂಃ  ಭುವಃ ಸುವಃ ಎಂಬ ಮೂರು ಧಾಮಗಳು. ಶ್ರೀರಂಗಮಹಾಗುರುಗಳ ಪ್ರವಚನಗಂಗೆಯೂ ಈ ಮೂರು ಧಾಮಗಳನ್ನು ವ್ಯಾಪಿಸಿ ತುಂಬಿಹರಿಯುತ್ತಿತ್ತು. ಆತ್ಮಸಂಸ್ಕಾರ ಸಂಪನ್ನರಾದ ಸುಕೃತಿಗಳಿಗೆ ಈ ಮಾತುಗಳು ಈ ಮೂರುಧಾಮಗಳನ್ನು ಹೊಂದುವಂತೆ ಮಾಡಿದರೆ ಸತ್ಯಾರ್ಥವು ತನ್ನನ್ನೇ ಪ್ರಕಾಶ ಪಡಿಸಿಕೊಂಡಂತೆಯೇ ಸರಿ.

ನಮ್ಮ ಸನಾತನ ಧರ್ಮ ಮತ್ತು ಸಂಸ್ಕೃತಿಗಳಿಗೆ ಸಂಬಂಧಪಟ್ಟ ಆರ್ಷಗ್ರಂಥಗಳ ಅವಲೋಕನವನ್ನು ಶ್ರೀಗುರುಗಳು ಅತ್ಯಂತ ಆಳವಾಗಿ ಮಾಡಿದ್ದರು. ಅದರೆ ಬರೀಗ್ರಂಥವನ್ನೇ ಅವಲಂಬಿಸಿ ಅವರು ವಿಷಯಗಳನ್ನು ನಿರೂಪಿಸುತ್ತಿರಲಿಲ್ಲ. ನಿಸರ್ಗಸಿದ್ಧವಾದ ಸೃಷ್ಟಿಯೇ ಅವರಿಗೆ ಮೌಲಿಕವಾದ ಪುಸ್ತಕವಾಗಿತ್ತು. ವಿಷಯ, ಪ್ರಯೋಗ ಮತ್ತು ಅನುಭವ-ಇವುಗಳೇ ಅವರ ಪ್ರವಚನಗಳ ಆಧಾರ. ಅನೇಕ ಸಂದರ್ಭಗಳಲ್ಲಿ ಅತ್ಯಂತ ಸರಳವಾದ ಪ್ರಯೋಗವಿಧಾನದಿಂದ ಅತ್ಯಂತ ಗಹನವಾದ ವಿಷಯಗಳನ್ನು ಬಾಲಕರಿಗೂ ಮನಮುಟ್ಟುವಂತೆ ನಿರೂಪಿಸುತ್ತಿದ್ದರು. ಪ್ರಯೋಗಪಾಠಕ್ಕೆ  ಅವರು ಅಳವಡಿಸುತ್ತಿದ್ದ ಸರಳವಾದ ಉಪಕರಣಗಳು ನಮ್ಮನ್ನು ಆಶ್ಚರ್ಯಚಕಿತರನ್ನಾಗಿ ಮಾಡುತ್ತಿದ್ದವು. ಪ್ರಣವದ ತತ್ತ್ವ, ನಾದ, ಸ್ವರ, ಅಕ್ಷರಗಳ ವಿಕಾಸ, ವ್ಯಾಕರಣಶಾಸ್ತ್ರ, ಸಂಗೀತಶಾಸ್ತ್ರ ಮುಂತಾದವುಗಳಿಗೆ ಸಂಬಂಧಪಟ್ಟ ವಿಷಯಗಳನ್ನು ನಿರೂಪಿಸಲು ತಂಬೂರಿ, ವೀಣೆ, ಘಂಟೆ ಮುಂತಾದ ವಾದ್ಯಗಳನ್ನು ಇಟ್ಟುಕೊಂಡು ಅತ್ಯಂತ ಪ್ರಯೋಗಾತ್ಮಕವಾದ ವಿಧಾನದಿಂದ ಜ್ಞಾನವಿಜ್ಞಾನಗಳ ರಹಸ್ಯಗಳನ್ನು  ನಿರೂಪಿಸುತ್ತಿದ್ದರು. ಮಹರ್ಷಿಗಳಿಂದ ಶಾಸ್ತ್ರಗಳ ರೂಪವಾಗಿ ಹೊರಬಿದ್ದ ವಿಷಯಗಳು ಮೌಲಿಕವಾಗಿ ಹೇಗೆ ಆತ್ಮಸ್ಥವಾದವುಗಳಾಗಿವೆ ಎಂಬುದನ್ನು ನಿರೂಪಿಸುವಾಗ ಶ್ರೀಮಹಾಗುರುಗಳ ಯೋಗೈಶ್ವರ್ಯವೇ ಪ್ರಕಟವಾಗುತ್ತಿದ್ದಂತೆ ತೋರಿಬಂದು ನಾವುಗಳು ರೋಮಾಂಚಿತರಾಗುತ್ತಿದ್ದೆವು. ಶ್ರೀಗುರುಗಳ ನಾಡೀವಿಜ್ಞಾನಸಂಪತ್ತು ಅತ್ಯದ್ಭುತವಾದುದಾಗಿತ್ತು. ನಾಡೀವಿಜ್ಞಾನದ ಬಲದಿಂದಲೇ ಜಾಗ್ರತ್, ಸ್ವಪ್ನ, ಸುಷುಪ್ತಿ ತುರೀಯ (ಸಮಾಧಿ) ಸ್ಥಿತಿಗಳ ತತ್ತ್ವಾರ್ಥಗಳನ್ನು ಪ್ರಯೋಅಗಬದ್ದವಾಗಿ ನಿರೂಪಿಸುತ್ತಿದ್ದರು. ಅವರ ದೇವಗಾನದ ಸಂಪತ್ತಿನ ವೈಭವವನ್ನು ವರ್ಣಿಸಲು ನಮ್ಮ ಮಾತುಗಳು ಅಸಮರ್ಥವಾದವು. ಅದು ಅನುಭವೈಕವೇದ್ಯವಾದ ವಿಷಯ. ಶ್ರೀಕೃಷ್ಣನ ವೇನುಗಾನವು ಪಶುಪಕ್ಷಿಮೃಗಾದಿ ಸಕಲ ಜೀವಗಳನ್ನೂ ತನ್ಮಯಗೊಳಿಸುತ್ತಿತ್ತು ಎಂಬುದನ್ನು  ಭಾಗವತ ಮುಂತಾದ ಗ್ರಂಥಗಳಲ್ಲಿ ನಾವು ಓದಿ ತಿಳಿಯುತ್ತೇವೆ. ಆದರೆ ಶ್ರೀಗುರುಗಳು ದಿವ್ಯಭಾವಪರಿಪೂರ್ಣರಾಗಿ ಮಾಡೂತ್ತಿದ್ದ ಗಾನವು ಅಧ್ಯಾತ್ಮಮಾರ್ಗಸಾಧಕರಾದ ಶಿಷ್ಯರನ್ನು ಅಂತರಂಗದ ತತ್ತ್ವಭೂಮಿಗಳಿಗೆ ಏರಿಸುತ್ತಿದ್ದ ಸನ್ನಿವೇಶವನ್ನು ಪ್ರತ್ಯಕ್ಷವಾಗಿ ಕಂಡ ಭಾಗ್ಯಶಾಲಿಗಳು ನಾವು. ಶ್ರೀರಂಗಮಹಾಗುರುಗಳ ಪ್ರವಚನಗಳು ಯಥಾರ್ಥದರ್ಶಿಗಳಾದ ಪ್ರಾಚೀನ ಮಹರ್ಷಿಗಳು ಕಂಡ ಜ್ಞಾನ-ವಿಜ್ಞಾನಮಯ ಸತ್ಯಾರ್ಥಗಳ ಮಾರ್ಗಗಳನ್ನು ತೆರೆದು ಕೊಡುವ ಅದ್ಭುತಶಕ್ತಿಯಿಂದ ಕೂಡಿವೆಯೆಂದು ಹೇಳಬಯಸುತ್ತೇವೆ. 

ಶಿಕ್ಷಾಶಾಸ್ತ್ರಕ್ಕೆ  ಸಂಬಂಧಪಟ್ಟ ಪ್ರವಚನವು ವೇದಾಂಗವಾದ ಶಿಕ್ಷೆಯ ಬಗ್ಗೆ ಮೌಲಿಕವಾದ ದೃಷ್ಟಿಯನ್ನು ತೆರೆದು ಕೊಡುತ್ತದೆ. ಆತ್ಮವಿರುವೆಡೆಯಲ್ಲಿ ಅವನಿಗೆ ಬೇಕಾದ ಶಾಸ್ತ್ರಗಳೂ ಸ್ವತಃಸಿದ್ಧವಾಗಿಯೇ ಇವೆ. ಆತ್ಮನ ಮುದ್ರೆ ಬೀಳದಿದ್ದರೆ ಯಾವಶಾಸ್ತ್ರಕ್ಕೂ ಬೆಳಕಿಲ್ಲ. ಆಕಳಿಕೆಯ ದೃಷ್ಟಾಂತವು ಈ ಸಂದರರ್ಭದಲ್ಲಿ ಸಹಜವಾಗಿ ಮೂಡಿಬಂದು ಮೂಲವಿಷಯದ ಕಡೆಗೆ ನಮ್ಮ ಗಮನವನ್ನು ಹರಿಸುತ್ತದೆ. ಶಿಕ್ಷೆಯು ವೇದಾಂಗವಾಗಿದೆ ವೇದವು ಸತ್ಯಸಾಕ್ಷತ್ಕಾರವನ್ನು ಮಾಡಿದ ಮಹರ್ಷಿಗಳಿಮ್ದ ಸಹಜವಾಗಿ ಹೊರ ಹೊಮ್ಮಿದ ವಾಙ್ಮಯ . ಅದು ತನ್ನದೇ ಆದ ನಿಯಮಾವಳಿಯಿಂದ ಹೊರಬಂದಿದೆ. ಸ್ವರ, ಮತ್ತಿತರ ವಿಧಿಗಳೆಲ್ಲ ಅದರ ಒಡನೆಯೇ ಬಂದಿವೆ. ``ಮನನಾತ್ ತ್ರಾಣನಾಚ್ಚೈವ ಮಂತ್ರ ಇತ್ಯಭಿಧೀಯತ್ತೆ" ಎಂಬಂತೆ ಮನನ ಮತ್ತು ತ್ರಾಣನ (ರಕ್ಷಣೆ) ಎರಡೂ ಇದ್ದಾಗ ಮಂತ್ರ. ಅದಕ್ಕೆ ಅನುಗುಣವಾದ ವ್ಯವಸ್ಥೆಯೇ ಶಿಕ್ಷೆ. ವೇದಾಂಗಗಳು ತಮ್ಮ  ತಮ್ಮದೇ ಆದ ವಿಷಯಗಳಿಮ್ದ ವೇದಪುರುಷನಾದ ಭಗವಂತನನ್ನು ಸುತ್ತಿಕೊಂಡಿವೆ. ಶಿಕ್ಷಾಶಾಸ್ತ್ರವು ವರ್ಣ, ಸ್ವರ, ಮಾತ್ರಾ, ಬಲಗಳಿಂದ ಭಗವಂತನ ಹತ್ತಿರಕ್ಕೆ ಒಯ್ಯುತ್ತದೆ. ನಾದೋತ್ಪತ್ತಿಯ ರಹಸ್ಯವು ಪ್ರಾಣಪಾನಗಳ ಸಂಯೋಗದಲ್ಲಿ ಅಡಗಿದೆ. ಆತ್ಮಮೂಲದಿಂದ ನಾದ, ಸ್ವರ, ಅಕ್ಷರ ಮುಂತಾದವುಗಳ ವಿಕಾಸವಾಗುತ್ತದೆ. ಇವೆಲ್ಲದರಲ್ಲೂ ಪ್ರಣವವು ವ್ಯಾಪಿಸಿದೆ. ಶಾಸ್ತ್ರಬಲ್ಲವನನ್ನು ಅಂತರ್ವಾಣಿಯೆಂದು ಕರೆಯುತ್ತಾರೆ. ಒಳಗಿನ ದೈವಭಾವವನ್ನು ಹೊರಹೊಮ್ಮಿಸಲು ವಾಣಿಯನ್ನು ಉಪಯೋಗಿಸುವವನು ಅಂತರ್ವಾಣಿ. ಈ ಸಮಂದರ್ಭದಲ್ಲಿ ದೈವಭಾವದ ಹಿನ್ನೆಲೆಯಿಂದ ಹೊರಹೊಮ್ಮಿ ಬಂದ ಆದಿಕಾವ್ಯವಾದ ರಾಮಾಯಣದ ದೃಷ್ಟಾಂತವು ಓದುಗಾ ಹೃದಯವನ್ನು ದಿವ್ಯಭಾವದಿಂದ ತುಂಬುತ್ತದೆ. ಸತ್ಯಸ್ವರೂಪವಾದ ವೇದಧರ್ಮವು ಮುಂದುವರಿಯಲು ಅದಕ್ಕೇ ಆದ ಸೌಂಡ್ (ನಾದ, ಸ್ವರ, ಮಾತ್ರಾ, ಬಲ- ಇತ್ಯಾದಿ) ಬೇಕು. ಆ ಧರ್ಮವನ್ನು ಕೆಡದಂತೆ ಒರಿಜಿನಲ್ಲಾಗಿ ಕಾಪಾಡಿಕೊಳ್ಳವುದಕ್ಕಾಗಿಯೇ ಶಿಕ್ಷಾಶಾಸ್ತ್ರ. ವೇದದ ನಡೆಯನ್ನು ಉಳಿಸಿಕೊಂಡು ಮಂತ್ರರೂಪವಾಗಿ ರಕ್ಷಿಸಿಕೊಳ್ಳಲು ಶಿಕ್ಷೆಬೇಕು. ಈ ಸಂದರ್ಭದ ಪ್ರಯೋಗಪಾಠ, ವೇದಮಂತ್ರಗಳ ಉಚ್ಚಾರಣೆಗೆ ಬೇಕಾದ ಒಳಧರ್ಮದ ಪರಿಚಯ ಮುಂತಾದ ಪ್ರಸಂಗಗಳು ಅತ್ಯಂತ ಹೃದ್ಯವಾಗಿವೆ. `ಆತ್ಮಮೂಲದಿಂದ ಅಂತರ್ವಾಹಿನಿಯಾಗಿ ಶಾಸ್ತ್ರವು ಹರಿದು ಬರಬೇಕು.' `ಅಂತರಂಗದ ಸ್ವಾಮಿಯ ಕಡೆಯಿಂದ ಚಿಮ್ಮಿದಾಗ ಶಿಕ್ಷಾಶಾಸ್ತ್ರವು ಸಜೀವವಾಗಿ ಬೆಳೆಯುತ್ತದೆ.' ಮಹಾಗುರುಗಳ ಈ ಮಾತುಗಳು ಸತ್ಯಾರ್ಥದ ಕಡೆಗೆ ನಮ್ಮನ್ನು ಒಯ್ಯುತ್ತಾ ದೈವಿಸ್ಛೂರ್ತಿಯನ್ನು ತುಂಬಿಕೊಡುತ್ತವೆ.

ವೇದಾಂಗವಾದ ವ್ಯಾಕರಣವು ವೇದಪುರುಷನ ಮುಖ ಎಂದು ವರ್ಣಿತವಾಗಿದೆ. ವ್ಯಾಕರಣಶಾಸ್ತ್ರಕ್ಕೆ ಮೂಲವಾದ್ ಹದಿನಾಲ್ಕು ಶಿವಸೂತ್ರಗಳ ತಾತ್ತ್ವಿಕ ನಿರೂಪಣೆಯು ನಮ್ಮನ್ನು ವಾಕ್ಕಿನ ಮೂಲಕ್ಕೆ ಕೊಂಡೊಯ್ಯುತ್ತದೆ. ಢಕ್ಕಾ ಎಂಬ ವಾದ್ಯವು ಪ್ರಾಣ ಮತ್ತು ಅಪಾನತತ್ತ್ವಗಳ ಆಧಾರದ ಮೇಲೆ ರಚಿತವಾದ ಒಂದು ಅದ್ಬುತವಾದ ವಾದ್ಯ. ಶಿವನ ಢಕ್ಕಾ ವಾದ್ಯವು ಮೂಲತಃ ತತ್ತ್ವಮಯವಾದುದು. ಬಹಳ ಹಿಂದೆ ಶ್ರೀಗುರುಗಳ ಒಂದು ಢಕ್ಕಾವಾದ್ಯವನ್ನು ಸಂಗ್ರಹಿಸಿ ಅದರ ಮೂಲಕ ಶಿವಸೂತ್ರಗಳು ಹೇಗೆ ವ್ಯಕ್ತವಾಗುತ್ತವೆ ಎಂಬುದನ್ನು ಪ್ರಯೋಗತ್ಮಕವಾಗಿ ತೋರಿಸುತ್ತಿದ್ದುದು ನಮಗೆ  ಇಂದಿಗೂ ಸ್ಮೃತಿಪಥದಲ್ಲಿದ್ದು ಸ್ಫೂರ್ತಿಯನ್ನು ನೀಡುತ್ತದೆ. ಹಯಗ್ರೀವ, ಭಾರತೀ, ಈಶ್ವರ- ಈ ಮೂರು ದೇವತೆಗಳೂ ವಿದ್ಯೆಗೆ ಅಧಿಪತಿಗಳು ಎಂಬ ವಿಷಯದ ನಿರೂಪಣೆಯು ನಮ್ಮ ಹೃದಯವನ್ನು ದಿವ್ಯಭಾವದಿಂದ ತುಂಬುತ್ತದೆ.

ದರ್ಶನಕ್ಕೆ  ಸಂಬಂಧಪಟ್ಟ ಪಾಠದಲ್ಲಿ ದರ್ಶನ ಎಂಬ ಶಬ್ದದ ಅರ್ಥಕ್ಕೆ ಎಷ್ಟು ವ್ಯಾಪ್ತಿಯಿದೆ? ಎಂಬ ನಿರೂಪಣೆಯು ಅತ್ಯಂತ ಮನನೀಯವಾಗಿದೆ, ಶಂಕರ ಭಗವತ್ಪದರಿಗೂ ನಮಗೂ, ಅಂತೆಯೇ ಭಗವದ್ರಾಮಾನುಜರಿಗೂ ನಮಗೂ ಇರುವ ಯಥಾರ್ಥವಾದ ಸಂಬಂಧದ ಬಗ್ಗೆ ಶ್ರೀರಂಗ ಮಹಾಗುರುಗಳು ನೀಡಿರುವ ಪ್ರವಚನವು ತತ್ತ್ವ ಜಿಜ್ಞಾಸುಗಳಿಗೆ ಅತ್ಯಂತ ಸೂಫರ್ತಿಯನ್ನು ನೀಡುತ್ತದೆ. ಅದ್ವೈತ ಮತ್ತು ವಿಶಿಷ್ಟಾದ್ವೈತ ತತ್ತ್ವಗಳ ನಿರೂಪಣೆಯು ನಮ್ಮನ್ನು ತತ್ತ್ವಭೂಮಿಗೆ ಕೊಂಡೊಯ್ಯುವ ರಾಜಮಾರ್ಗವಾಗಿದೆ. ಶ್ರೀರಾಮಾನುಜರು ಮೂರುಕಾಲಿನ ಮಣೆಯಮೇಲೆ ಜ್ಞಾನದೀಪವನ್ನು ಹಚ್ಚಿಟ್ಟರು. ಶ್ರೀಶಂಕರರು ಒಂದೇಕಾಲಿನ ಮಣೆಯ ಮೇಲೆ ಜ್ಞಾನದೀಪವನ್ನು ಬೆಳಗಿಸಿದರು ಎಂಬ ದೃಷ್ಟಾಮ್ತವಂತೂ ನಮ್ಮನ್ನು ಸತ್ಯಜ್ಯೋತಿಯ ನೆಲೆಗೇ ಕೊಂಡೊಯ್ದು ಮುಟ್ಟಿಸುತ್ತದೆ.

`ಜೀವವಿಜ್ಞಾನ' ಪಾಠದಲ್ಲಿ ಈ ಶಬ್ದವು ಆಧುನಿಕ ವಿಜ್ಞಾನ ಕ್ಷೇತ್ರದಲ್ಲಿ ಬಳಕೆಗೆ ಬಂದಿದ್ದರೂ ಜೀವಸ್ವರೂಪವನ್ನು ಯಥಾರ್ಥವಾಗಿ ಕಂಡ ಮಹರ್ಷಿಗಳ ದೃಷ್ಟಿಯಲ್ಲಿ ಜೀವವಿಜ್ಞಾನವೆಂದರೇನು? ಎಂಬ ವಿವರಣೆಯು ಅತ್ಯಂತ ಮೌಲಿಕವಾದುದು. `ಅಧುನಿಕ ವಿಜ್ಞಾನಿಯು ಅಂಶವನ್ನು ನೋಡುತ್ತಾನೆ. ಜ್ಞಾನಿಯು ಅಂಶಿಯನ್ನು  ನೋಡುತ್ತಾನೆ' ಎಂಬ ವಾಣಿಯು ಸತ್ಯದರ್ಶಿಯಾದ ಶ್ರೀಗುರುವಿನ ಅಮರವಾಣಿಯಾಗಿದೆ. `ಜೀವವಿಕಾಸವು ಪ್ರಣವಮೂಲದಿಂದ ಆಗುತ್ತದೆ. ಜ್ಞಾನಿ-ಯಾದವನು ಸೃಷ್ಟಿಯ ವಿಕಾಸಕ್ರಮವನ್ನು ತಿಳಿದುಕೊಂಡಾಗ ವಿಜ್ಞಾನಿ ಎನ್ನಿಸಿಕೊಳ್ಳುತ್ತಾನೆ' ಈ ಮಾತುಗಳು ಜ್ಞಾನ-ವಿಜ್ಞಾನ ಶಬ್ದಗಳು ಸತ್ಯಾರ್ಥವನ್ನು ಪ್ರಕಾಶಪಡಿಸುತ್ತವೆ.

``ಸ್ಮೃತಿ ಮತ್ತು ವಿಸ್ಮ್ತಿ" ಪಾಠದಲ್ಲಿ ಈ ಶಬ್ದಗಳ ಅರ್ಥದ ಬಗ್ಗೆ  ನಮಗೆ ಹೊದದೃಷ್ಟಿಯು ತೆಗೆಯುತ್ತದೆ. ಈ ಸಂದರ್ಭದಲ್ಲಿ ಗರ್ಭಸ್ಥ ಶಿಶುವಿನ ದೃಷ್ಟಾಂತವು ಅತ್ಯಂತ  ಸ್ಫೂರ್ತಿದಾಯಕವಾದುದು. ಸೂತಿಕಗೃಹದ ದೀಪದ ಮಹಿಮೆ ಆರ್ಷದೃಷ್ಟಿಯನ್ನು ತೆರೆದುಕೊಡುವ ಅಂಜನವಾಗಿದೆ. ಧರ್ಮಶಾಸ್ತ್ರಗಳಿಗೆ ಸ್ಮೃತಿ ಎಂಬ ಹೆಸರು ಹೇಗೆ? ಎಂಬ ವಿವರಣೆಯು ಅತ್ಯಂತ ಮೌಲಿಕವಾದುದು.

``ಪುಸ್ತಕಗಳಿಂದ ಆತ್ಮಸಂಸ್ಕರಣ" ಎಂಬ ಪಾಠವು ಆರ್ಷಗ್ರಂಥಗಳ ಪಾಠಕರಿಗೆ ಹೊಸದೃಷ್ಟಿಯನ್ನು ತೆಗೆಯುವಂತಿದೆ. `ವಿಷಯವು ಪುಸ್ತಕಕ್ಕೆ  ಬರಬೇಕಾದರೆ ಮೊದಲು ಆತ್ಮಸ್ಥವಾಗಿರಬೇಕು' ಎಂಬ ಮಾತು ಈ ಪ್ರವಚನದ ಮೂಲಸೂತ್ರವಾಗಿದೆ. `ಪುಸ್ತಕವು ಸತ್ಯಕ್ಕೆ ಲೀಡ್ ಮಾಡುವಂತಿರಬೇಕು. ನಾಟ್ಯಸರಸ್ವತಿಯ ಕೈಯಲ್ಲಿರುವ ವಿದ್ಯಾಮಯವಾದ ಪುಸ್ತಕವೇ ನಿಜವಾದ ಪುಸ್ತಕ' ಎಂಬ ಶ್ರೀಗುರುವಿನ ಮಾತು ಮನನೀಯವಾದ ಮಹಾಮಂತ್ರವಾಗಿದೆ.

ರಾಮಾಯನದ ಬಗ್ಗೆ  ಶ್ರೀಗುರುಗಳ ಪ್ರವಚನವು ನಮ್ಮನ್ನು ಯೋಗಿಹೃದಯ ವೇದ್ಯವಾದ ಶ್ರೀರಾಮಧಾಮಕ್ಕೇ ಕೊಂಡೊಯುತ್ತದೆ. ಇದರಿಂದ ಓದುಗರಿಗೆ ಮಹಾಕವಿ ದರ್ಶನ ಭಾಗ್ಯ ಪ್ರಾಪ್ತವಾಗುತ್ತದೆ. ರಾಮಾಯಣವು ವೇದಸಮಾನವಾದ. ಆದಿಕಾವ್ಯ. ವೇದವೇದ್ಯನಾದ ಪರಮಪುರುಷನೇ ಈ ಆದಿಕಾವ್ಯದಲ್ಲಿ ಪ್ರತಿಪಾದ್ಯನಾಗಿದ್ದಾನೆ. ಅದುದರಿಂದಲೇ ಪುರುಷಸೂಕ್ತ  ಪಠನದಿಂದ ಶ್ರೀರಾಮಪಟ್ಟಾಭಿಷೇಕದ ಆಚರಣೆಯು ಅತ್ಯಂತ ಮಂಗಳಮಯವಾಗಿದೆ. ರಾಮಾಯಣವು ಪುರುಷಾರ್ಥಮಯವಾದ ಜೀವನವನ್ನು ಯಥಾರ್ಥವಾಗಿ ನಿರೂಪಿಸುವ ಮಹಾಕಾವ್ಯ. ಧರ್ಮವಿಗ್ರಹನೂ ಸತ್ಯಪರಾಕ್ರಮಿಯೂ ಆದ ಶ್ರೀರಾಮನು ಇದರ ನಾಯಕ. ಶ್ರೀರಂಗಮಹಾಗುರುಗಳು ರಾಮಾಯಣದ ಶ್ಲೋಕಗಳಿಗೆ ಅತ್ಯಂತ ಸಹವಾದ ರಾಗಸಂವಿಧಾನವನ್ನು ಮಾಡಿ ದಿವ್ಯಗಾನ ವಿಧಾನದಿಂದ ಹಾಡುತ್ತಿದ್ದರು. ಆ ಸಂದರ್ಭದಲ್ಲಿ ಸ್ವಯಂ ಆತ್ಮಾರಾಮರಾದ ಅವರ ಯೋಗೈಶ್ವರ್ಯವು ವ್ಯಕ್ತವಾಗುತ್ತಿದ್ದುದನ್ನು ಪ್ರತ್ಯಕ್ಷವಾಗಿ ಕಂಡ ಭಾಗ್ಯಶಾಲಿಗಳು ನಾವು. ಅವರ ಹೃದಯ ಸಿಂಹಾಸನದಲ್ಲಿ ಶ್ರೀರಾಮಯೋಅಗ ವೈಭವವೇ ಬೆಳಗುತ್ತಿತ್ತು. ಜ್ಞಾನಮಯವೂ, ತತ್ತ್ವಮಯವೂ ಆದ ಕವಿತಾಶಾಖೆಯ ತುದಿಗೆ ಏರಿದ ಯೋಗಿ ಕವಿಕೋಕಿಲದ ಪ್ರಣವಮಯವಾದ ತಾರಕನಾಮಕೂಜನದ ಶಾಂತವಾದ ಸುಮಧುರ ಪಂಚಮಸ್ವರಘೋಷವು ಮೊಳಗುತ್ತಿತ್ತು. ಶ್ರೀರಾಮಪಟ್ಟಾಭಿಷೇಕ -ಪ್ರಕರಣದ ಶ್ಲೋಕಗಳನ್ನು ದೇವಗಾನವಿಧಾನದಿಂದ ಹಾಡುತ್ತಿರುವಾಗ ಮಹಾಗುರುಗಳ ಶ್ರೀಮುಖದಿಂದ ಏಕಕಾಲದಲ್ಲಿ ವೀಣಾ, ಶಂಖ, ವೇಣು, ದುಂದುಭಿ ಮುಂತಾದ ದಿವ್ಯವಾದ್ಯಗಳ ಮಹಾನಾದಗಳು ಮೊಳಗುತ್ತಿದ್ದವು. ಮಹರ್ಷಿಗಳು ಆಚರಿಸಿದ ಯಥಾರ್ಥವಾದ ಶ್ರೀರಾಮಪಟ್ಟಾಭಿಷೇಕದ ಮಹಾದರ್ಶನವು ಅದಾಗಿತ್ತು. ಹೀಗೆ ಶ್ರೀರಾಮಾಯಣದ ಬಗ್ಗೆ ಶ್ರೀರಂಗ ಮಹಾಗುರುಗಳಾ ಈ ಪ್ರವಚನ ಮಾಲೆಯು ಶ್ರೀಮದ್ರಾಮಾಯಣದ ಮಹಾದರ್ಶನವೇ ಆಗಿದೆ. ನಮ್ಮ ಹೃದಯಸಿಂಹಾಸನದಲ್ಲಿ ಮಹಾಪುರುಷನಾದ ಶ್ರೀರಾಮನನ್ನು ಬೆಳಗಿಸಿದ ಮಹಾಗುರುಗಳ ಪಾದಾರವಿಂದಗಳಿಗೆ ಪ್ರಾಣಪ್ರಣಾಮಗಳು.

`ಭಾಗವತ ಓದಲು ಹಿನ್ನೆಲೆ,' ಋಷಿಸಂಪ್ರದಾಯದಲ್ಲಿ ವಿಚ್ಛಿನ್ನತೆ ಬರಲು ಕಾರಣ, `ಹಿಂದಿನದು ಏಕೆ ಉಳಿಯಬೇಕು?' ಎಂಬ ಪ್ರವಚನಗಳೂ ಸಹ ಸಂಕ್ಷಿಪ್ತವಾಗಿದ್ದರೂ ಮಹರ್ಷಿಗಳ ಹೃದಯವನ್ನು ತೆರೆದು ತೋರಿಸುವ ಮಾರ್ಗದರ್ಶಿಗಳಾಗಿವೆ. ಶ್ರೀ ರಂಗ ಮಹಾಗುರುಗಳ ಇತರ ಪ್ರವಚನಗಳನ್ನು ಕೇಳಿದರವರಿಗೆ ಹಾಗೂ ಓದಿದವರಿಗೆ ಈ ಮಾತುಗಳ ಹಿಂದಿರುವ ಹೃದಯಭಾವವನ್ನು ಗ್ರಹಿಸುವುದು ಸುಲಭಸಾಧ್ಯವಾಗುತ್ತದೆ. ರಾಮಾಯಣ, ಭಾರತ , ಭಾಗವತ ಮೊದಲಾದ ಗ್ರಂಥಗಳು ಕೇವಲ ಭೌತಿಕವಿಚಾರಕ್ಕೆ ಪ್ರಾಧಾನ್ಯವನ್ನು  ಕೊಟ್ಟ ಗ್ರಂಥಗಳೇ ಅಲ್ಲ. ಅವು ಆಧ್ಯಾತ್ಮಿಕವಾದ ಹಸಿವನ್ನು ನೀಗಿಸಲು ಹೊರಟ ಆರ್ಷಗ್ರಂಥಗಳು. ದೇಹ, ಇಂದ್ರಿಯ, ಮನಸ್ಸು, ಬುದ್ಧಿಗಳಿಗೆ ಹಸಿವು ಉಂಟಾದರೆ ಅದನ್ನು  ನೀಗಿಸುವ ಆಹಾರವು ಬೇರೆ. ಅಂತೆಯೇ ಆತ್ಮನಿಗೆ ಉಂಟಾಗುವ ಹಸಿವನ್ನು ನೀಗಿಸುವ ಆಹಾರವು ಬೇರೆ. ಅಂತೆಯೇ ಆತ್ಮನಿಗೆ ಉಂಟಾಗುವ ಹಸಿವನ್ನು ನೀಗಿಸುವ ಆಹಾರವು ಯಾವುದು? ಎಂಬುದನ್ನು ತಿಳಿಯಬೇಕು. ಅಂತಹ ಹಸಿವನ್ನು ತೀರಿಸಲು ಬೇಕಾದ ಆಹಾರವನ್ನು ಪಡೆಯಲು ಭಾಗವತವನ್ನು ಓದಿದರೆ ಆಗ ಗ್ರಂಥದ ಹಿರಿಮೆ ಅರ್ಥವಾಗುತ್ತದೆ. ಈ ಸಂದರ್ಭದಲ್ಲಿ ಶ್ರೀಗುರುಗಳು ``ಜಿಹ್ವೇ ಕೀರ್ತಯ--" ಎಂಬ ಸ್ತೋತ್ರವನ್ನು ಭಗವದ್ವೈ ಭವದೊಡನೆ ಹಾಡಿ ತೋರಿಸಿದ್ದಾರೆ.

ಶ್ರೀರಂಗ ಮಹಾಗುರುಗಳು ಯೋಗರಹಸ್ಯಗಳನ್ನು ಪೂರ್ಣವಾಗಿ ತಿಳಿದ ಯೋಗಿ ಶ್ರೇಷ್ಠರು. ದೇವಗಾನವಿಧಾನವನ್ನರಿತ ಯೋಗಗಾಯಕರು. ಮಹರ್ಷಿಗಳ ಜ್ಞಾನ-ವಿಜ್ಞಾನಗಳ ಮರ್ಮವನ್ನು ಸಂಪೂರ್ಣವಾಗಿ ತಿಳಿದವರು. ಅದ್ಭುತವಾಅ ಮೇಧಾಶಕ್ತಿ ಸಂಪನ್ನರು. ಪ್ರಯೋಗ-ವಿಜ್ಞಾನಕುಶಲರು. ಅದ್ಭುತವಾದ ವಾಗ್ವಿಭೂತಿಯಿಂದ ಪ್ರಕಾಶಿಸಿದವರು. ತಮ್ಮ ಅದ್ಬುತವಾದ ಯೋಗಬಲದಿಂದ ಸಾವಿರಾರು ಮಂದಿ ಸುಕೃತಿಚೇತನರನ್ನು ಅನುಗ್ರಹಿಸಿ ಆತ್ಮಮಾರ್ಗದರ್ಶನವನ್ನು ಮಾಡಿಸಿದರು. ಸನಾತನವೂ ಆರ್ಷವೂ ಆದ ಧರ್ಮಸಂಸ್ಕೃತಿಗಳ ಬಗ್ಗೆ ಮೌಲಿಕವಾದ ದೃಷ್ಟಿಯನ್ನು ತೆರೆಸುವಂತಹ ಸರಳವೂ ಪ್ರಭಾವಪೂರ್ಣವೂ ಆದ ಪ್ರವಚನಗಳಿಂದ ಶಿಷ್ಯರ ಹೃದಯಗಳಿಗೆ ಅಮರಭಾವವನ್ನು ಕರುಣೆಯಿಂದ ಹರಿಸಿದ ಅವತಾರಪುರುಷರು. ಅವರ ವಾಣೀ ಯಥಾರ್ಥವಾದ ``ಅಮರವಾಣೀ". ಆ ಮಹಾಗುರುಗಳ ಜ್ಞಾನ-ವಿಜ್ಞಾನಯಜ್ಞದಲ್ಲಿ ಸಹಧರ್ಮಿಣಿಯಾಗಿ ಅಷ್ಟಾಂಗಯೋಗವಿಜ್ಞಾನಮಂದಿರದ ಅಧ್ಯಕ್ಷಪೀಠದಲ್ಲಿ ವಿರಾಜಿಸುತ್ತಿರುವ ಶ್ರೀಮಾತೆ ವಿಜಯಲಕ್ಷ್ಮೀ ಮಹಾತಾಯಿಯವರಿ ಗೂ ಅವರ ಪಾದಾರವಿಂದಗಳಿಗೆ ಅರ್ಪಿಸುತ್ತೇನೆ. 

                             ``ಸತ್ಯಂ ಪರಮ್ ಧೀಮಹಿ"
 
\hfill{ಶೇಷಾಚಲಶರ್ಮಾ}                             
