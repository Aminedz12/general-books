\chapter{ಹಿಂದಿನದು ಏಕೆ ಉಳಿಯಬೇಕು?} 

(ಬೆಂಗಳೂರಿನಲ್ಲಿ ನಡೆದ ಗಣಪತಿಯ ಹಬ್ಬದ ಉಪನ್ಯಾಸದ ವಿವರವನ್ನು ಶ್ರೀ ಶ್ರೀಕಂಠರು ಪತ್ರದಲ್ಲಿ ತಿಳಿಸಿದ್ದರು. 

ತತ್ಸಂಬಂಧವಾಗಿ ಮಾತನಾಡುತ್ತಾ ಪ್ರಾಸಂಗಿಕವಾಗಿ `ಹಿಂದಿನದು ಏಕೆ ಉಳಿಯಬೇಕು?' ಎಂಬುದರ ವಿಷಯವಾಗಿ ಶ್ರೀ ಗುರುಭಗವಂತನು ನುಡಿದ ಮಾತುಗಳು. ಸಂಗ್ರಹ-ಶ್ರೀ ಛಾಯಾಪತಿಗಳಿಂದ) 

\large{\bf{ಆರೋಗ್ಯದ ಅಪೇಕ್ಷೆ}} 

ಹಿಂದಿನದು ಏಕೆ ಉಳಿಯಬೇಕು? ಹಿಂದಿನದು ಎಂದೇ ಅದಕ್ಕೆ ಬೆಲೆಯಿಲ್ಲ, ಅದರಿಂದಾಗಬೇಕಾದ ಪ್ರಯೋಜನವಿದ್ದರೆ ಉಳಿಸಲು ಪ್ರಯತ್ನಿಸಿ, ಮೊಟ್ಟಮೊದಲು ಇದ್ದದ್ದು ಆರೋಗ್ಯ. ಅನಾರೋಗ್ಯವು ಮಧ್ಯದಲ್ಲಿ ಬಂತು. ಹಾಗೆ ಮಧ್ಯದಲ್ಲಿ ಬಂದ ಅನಾರೋಗ್ಯವು ಆರೋಗ್ಯವನ್ನು ಕೆಡಿಸುತ್ತದೆ. ಆದ್ದರಿಂದ ಆರೋಗ್ಯವನ್ನು ಉಳಿಸಿಕೊಳ್ಳಲು ಹಿಂದಿನವರ ಅಪೇಕ್ಷೆ. ಕಾರಣ-ಆರೋಗ್ಯವು ಚೆನ್ನಾಗಿದೆ, ನಮಗೆ ನೆಮ್ಮದಿಯನ್ನು ಕೊಡುತ್ತದೆ-ಎಂದು. 

`ಆರೋಗ್ಯವೇನು? ಅನಾರೋಗ್ಯವೇನು? ಹಿಂದಿದ್ದ ಆರೋಗ್ಯ ಹಿಂದಿನದು, ಇತ್ತೀಚೆಗೆ ನಮ್ಮ ಕಾಲಕ್ಕೆ ಬಂದುದನ್ನು ನಾವು ರಕ್ಷಿಸೋಣ. ಯಾವುದೋ ಕಾಲದ ಆರೋಗ್ಯವನ್ನು ಕಟ್ಟಿಕೊಂಡು ನಮಗೆ ಆಗಬೇಕಾದುದೇನು?' ಎಂದು ಒಂದು ವೇಳೆ ವೈದ್ಯರು ಆರೋಗ್ಯವನ್ನು ಹೊಡೆದು ಹಾಕಿ ಅನಾರೋಗ್ಯವನ್ನೇ ಬದುಕಿಸಲು ತೊಡಗಿದರೆ ಯಾರೂ ಒಪ್ಪಲಾರರು. ಹಿಂದಿನದಾದರೂ ಬಾಧಕವಿಲ್ಲ, ನಮಗೆ ಆರೋಗ್ಯವೇ ಇರಲಿ ಎಂದೇ ಎಲ್ಲರೂ ಹೇಳುತ್ತಾರೆ. ಮೊದಲಿನ ಆರೋಗ್ಯದಿಂದ ಸುಖ-ನೆಮ್ಮದಿಗಳನ್ನು ಕಂಡಿದ್ದರಿಂದಲೂ, ಮಧ್ಯದಲ್ಲಿ ಬಂದ ಅನಾರೋಗ್ಯದಿಂದ ಅದು ತಪ್ಪಿಹೋದುದರಿಂದಲೂ ಮತ್ತೆ ಹಿಂದಿನ ಸುಖಕ್ಕಾಗಿ ಆರೋಗ್ಯವನ್ನು ಬದುಕಿಸಿಕೊಳ್ಳಲು ಪ್ರಯತ್ನಿಸುತ್ತಾರೆ. ಆದ್ದರಿಂದ ವೈದ್ಯನಾದವನು ಆರೋಗ್ಯವನ್ನು ಬದುಕಿಸಿ ಅನಾರೋಗ್ಯವನ್ನು ಹೋಗಲಾಡಿಸಬೇಕು. 


\large{\bf{ಆತ್ಮಾನುಭವದ ಅಪೇಕ್ಷೆ}} 


ಅಂತೆಯೇ ಸನಾತನವೂ ಸದಾತನವೂ ಆದ ವಸ್ತುವು ಸುಖವನ್ನು ಕೊಡುವುದರಿಂದ ಹಿಂದಿನದು ಬೇಕು ಎಂಬ ಅಪೇಕ್ಷೆ ಯುಕ್ತವಾಗಿದೆ. ಬ್ರಹ್ಮನು ಹೇಗೆ ಸೃಷ್ಟಿಮಾಡಿಕೊಂಡಿದ್ದಾನೆಯೋ ಅದು ಅಂತೆಯೇ ಇರಬೇಕು, ಹಾಗೆಯೇ ಅದನ್ನು ತರಬೇಕು. ಆದ್ದರಿಂದಲೇ ಬ್ರಹ್ಮಜ್ಞಾನವುಳ್ಳವನ `ಗೈಡೆನ್ಸ್‍' ಬೇಕು. ಇಂಜಿನಿಯರ್‍ ಆದವನು ಒಂದು ಪ್ಲಾನನ್ನು ಹಾಕಿಕೊಟ್ಟರೆ ಮೇಸ್ತ್ರಿಯು ಹಾಗೆಯೇ ಅದನ್ನು ಹೊರರೂಪಕ್ಕೆ ತರುತ್ತಾನೆ. ಅಂತೆಯೇ ಜ್ಞಾನಿಯಾದವನು ಸೃಷ್ಟಿಯ ಸಹಜವಾದ ಸನ್ನಿವೇಶವನ್ನು ತನ್ನ ಮನಸ್ಸಿಗೆ ತಂದುಕೊಂಡು ಒಂದು ಯೋಜನೆಯನ್ನು ಹಾಕಿ ಕೊಟ್ಟರೆ, ಇತರರು ಅಂತೆಯೇ ನಡೆಯಬೇಕು. ಹಿಂದಿನದನ್ನು ಹೊಗಳುವುದು ಏತಕ್ಕೆ? ಎಂದರೆ, ಎಲ್ಲಕ್ಕೂ ಹಿಂದಿನದು ಆತ್ಮ, ಅದನ್ನು ಅರಿತರೆ ಸುಖ, ಅದರಿಂದ ದೊರಕುವ ಸುಖಕ್ಕಾಗಿ ಅದಕ್ಕೆ ಹೊಗಳಿಕೆ-ಅಷ್ಟೇ, ಇತರ ಯಾವ ಪ್ರಯೋಜನದಿಂದಲೂ ಅಲ್ಲ. 


\large{\bf{ಸನಾತನವಾದದ್ದುನ್ನು ರಕ್ಷಿಸಿಕೊಳ್ಳುವ ಕರ್ತವ್ಯ}} 


ಒಬ್ಬ ವೈದ್ಯನು `ಮೃತ್ಯುಂಜಯರಸ'ವನ್ನು ತಯಾರುಮಾಡಬೇಕೆಂದು ಮೂಲಿಕೆಯನ್ನು ತರಲು ತನ್ನ ಶಿಷ್ಯರನ್ನು ಕಳುಹಿಸಬಹುದು. ಅವರ ದುಡಿಮೆಯು ಮೃತ್ಯುಂಜಯರಸಕ್ಕೋಸುಗ ಅಷ್ಟೇ ಹೊರತು ಸಂಬಳಕ್ಕಾಗಿ ಅಲ್ಲ. ಕೇವಲ ಸಂಬಳಕ್ಕೇ ಆದರೆ, `ನಿಮ್ಮ ಸಂಬಳ ನೀವು ತೆಗೆದುಕೊಳ್ಳಿ, ಮೃತ್ಯುಂಜಯರಸದಲ್ಲಿ ಮಾತ್ರ ಪಾಲಿಲ್ಲ.' ಎನ್ನಬೇಕಾಗುತ್ತದೆ. ಅಂತೆಯೇ ಒಬ್ಬ ಪರಮಗುರುವೂ ಮೃತ್ಯುವನ್ನು ಜಯಿಸಿ ಅಮೃತರಾಗಿ ಬಾಳಲು ಹೇಳಿದರೆ ಅದಕ್ಕೂ ಕೆಲವು ಕೆಲಸಗಳು ಬೇಕು. ಹಿಂದಿನದನ್ನು ಯಥಾಪೂರ್ವವಾಗಿ ತರಬೇಕು. ಅದು ಸನಾತನವೂ ಸದಾತನವೂ ಆದುದರಿಂದ ಯಾವಾಗಲೂ ಇರುವಂತಹುದು. 


\newpage 

\begin{center} 

{\LARGE ಇತಿಹಾಸ-ಪುರಾಣ}\\ 

{\LARGE (ರಾಮಾಯಣ-ಭಾಗವತ)} 

\end{center} 
