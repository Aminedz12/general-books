\chapter{ಪುಸ್ತಕಗಳಿಂದ ಆತ್ಮಸಂಸ್ಕರಣ}

(ವಿಜ್ಞಾನಮಂದಿರದ ಸದಸ್ಯರಲ್ಲೋಬ್ಬರಾದ ಶ್ರೀಯುತ ಕೆ.ಎಸ್. ವರದಾಚಾರ್ಯರು ಸಮರ್ಪಣಾಭಾವದಿಂದ ಶ್ರೀಗುರುವಿಗೆ ಮಹಾಭಾರತ ಪುಸ್ತಕಗಳ ಸೆಟ್ಟನ್ನು ಕೊಟ್ಟಾಗ ಪುಸ್ತಕ ಶೇಖರಣೆಯ ಬಗ್ಗೆ  ಅವರಿಗೆ ಅನುಗ್ರಹಿಸಿದ ಪ್ರವಚನ. ಸಂಗ್ರಾಹಕರು ಶ್ರೀಯುತ ಕೆ. ಎಸ್. ವರದಾಚಾರ್ಯರು.)

\section*{ವಿಷಯವು ಪುಸ್ತಕಕ್ಕೆ ಬರಬೇಕಾದರೆ ಮೊದಲು ಆತ್ಮಸ್ಥವಾಗಿರಬೇಕು}

ಯಾವುದೇ ಒಂದು ಪುಸ್ತಕವನ್ನು ಮುದ್ರಿಸಿ ಹೊರತಂದಾಗ ಎಷ್ಟನೆಯ ಮುದ್ರಣವಿದು? ಎಂದು ಜನರು ಕೇಳುವುದುಂಟಪ್ಪಾ! ಹೀಗೆ ಬಂದಾಗ (ಮುದ್ರಣವೆಂದರೆ ಪ್ರಕಾಶನಕಾರ್ಯ) ಅದರ ಆವಿಷ್ಕಾರದ ಬಗ್ಗೆ ಹೇಳಹೊರಡುವಾಗ, ನಿಜಾಂಶವನ್ನು ಹೇಳುವುದಾದರೆ ಮೊದಲನೆಯ ಪ್ರಕಾಶನವು ಆತ್ಮನಲ್ಲೇ ಆಗಬೇಕಾದುದು. ಅದು ಪುಸ್ತಕಕ್ಕಿಳಿಯುವಂತಾದರೆ, ಅಲ್ಲಿ ಎರಡನೆಯ ಪ್ರಕಾಶನವೆಂಬುದರಲ್ಲಿ ನನಗೆ ಸಂಶಯವಿಲ್ಲ. ಏಕೆಂದರೆ ಯಾವುದೇ ಒಂದು ವಿಶಯವು ಪುಸ್ತಕಕ್ಕೆ ಬರಬೇಕಾದರೆ ಮೊದಲಿಗೆ ಆತ್ಮಸ್ಥವಾಗಬೇಕು. ಆತ್ಮನಲ್ಲಿ ಅಭಿವ್ಯಕ್ತವಾಗಿ ಮುಂದಕ್ಕೆ ಬರಬೇಕು. ನಿಮ್ಮ ಅಭಿಪ್ರಾಯವನ್ನರ್ಥ ಮಾಡಿಕೊಂಡು, ನಿಮ್ಮ ಮನಸ್ಸಿನಂತೆ ಮಂದಿರದ ಪುಸ್ತಕಾಲಯ ಸೇರಿಸುವುದಕ್ಕೆ ನಾನು ಕೈಯಿಂದ ತೆಗೆದುಕೊಂಡಿದ್ದೇನೆ. ಆದರೆ ಪುಸ್ತಕದ ಸ್ಥಾನವನ್ನು ನೀವು ಅರ್ಥಮಾಡಿಕೊಳ್ಳಲು ಆ ಬಗ್ಗೆ ಜವಾಬ್ದಾರಿಗಾಗಿ ಕೆಲವು ಮಾತುಗಳನ್ನಾಡುತ್ತೇನೆ. ಮನಸ್ಸಿಗೆ ತಂದುಕೊಳ್ಳಿ. ವಿಷಯದ ಬಗ್ಗೆ ಸಾವಧಾನತೆಯಿರಲಿ.

\section*{ಪುಸ್ತಕ ಶೇಖರಣೆಯ ಪ್ರವೃತ್ತಿಯ ಆರಂಭ}

ಸಾಮಾನ್ಯವಾಗಿ ಲೋಕದಲ್ಲಿ ಪುಸ್ತಕ ಶೇಖರಣೆಯ ಪ್ರವೃತ್ತಿಯು ಎಲ್ಲಿಂದ ಪ್ರಾರಂಭವಾಯಿತೆಂದು ವಿಮರ್ಶಿಸಹೊರಟರೆ, ಅದು ಪ್ರಪ್ರಥಮವಾಗಿ ವಿದ್ಯಾರಾಜನಾದ ಹಯಗ್ರೀವನಿಂದಲೇ ಎಂದು ಹೇಳಬೇಕಾಗುವುದು. ಹಯಗ್ರೀವ ದೇವನನ್ನು ಸ್ತುತಿಸುವಾಗ `ವ್ಯಾಖ್ಯಾಮುದ್ರಾಂ ಕರಸರಸಿಜೈಃ ಪುಸ್ತಕಂ ಶಂಖಚಕ್ರೇ'\label{100} ಎಂದಿದ್ದಾರೆ. ನಮ್ಮ ಜಾಡ್ಯವನ್ನು ಹೋಗಲಾಡಿಸಲು ಯಾವ ದೇವತೆಯನ್ನು ಕುರಿತು `ಸಾ ಮಾಂ ಪಾತು ಸರಸ್ವತೀ ಭಗವತೀ ನಿಶ್ಯೇಷಜಾಡ್ಯಾಪಹಾ'\label{100a} ಎಂದು ಪ್ರಾರ್ಥಿಸುತ್ತೇವೋ, ಆ ಸರಸ್ವತೀದೇವಿಯೂ `ವೀಣಾ ಪುಸ್ತಕಧಾರಿಣೀ' ಯಾಗಿರುವುದರಿಂದ ಆ ತಾಯಿಯಲ್ಲೂ ಪುಸ್ತಕಜಾಡ್ಯ ಇದ್ದೇ ಇದೆಯೆನ್ನಬೇಕು.

\section*{ಪುಸ್ತಕದ ಉಪಯೋಗ ಯಾರಿಗೆ? ಯಾವಾಗ? ಎಷ್ಟು?}

`ಪುಸ್ತಕಕ್ಕೆ  ಪ್ರಾಮುಖ್ಯಕೊಡಬೇಡಿ' ಎಂದು ಸದಾ ಎಚ್ಚರಿಸುವ ನಿಮ್ಮ ಮುಂದಿರುವ ವ್ಯಕ್ತಿಯೂ ಇಷ್ಟಾರು ಪುಸ್ತಕ ರಾಶಿಹಾಕಿದ್ದಾನೆ. ಇದು ಹೇಗೆ? ಇದರ ಉಪಯೋಗ ನಿಮಗೆ ಹೇಗೆ? ಎಂಬ ಪ್ರಶ್ನೆ ಬರಬಹುದು. ಆದರೆ ಪ್ರಶ್ನೆಯು ನಿಮ್ಮಿಂದ ಬರಲಿ, ಬಿಡಲಿ, ಆ ಬಗ್ಗೆ ಜವಾಬ್ದಾರಿಯ ದೃಷ್ಟಿಯಿಂದ ನನ್ನದಾಗಿ ವಿವರಣೆಯೊಂದಿರಬೇಕೆಂದು ನಿಮ್ಮ ಮುಂದೆ ವಿವರಣೆ ಕೊಡುತ್ತಿದ್ದೇನೆ.

ನನ್ನನ್ನೇ ಕೇಳುವುದಾದರೆ ನನಗೆ ಈ ಪುಸ್ತಕದ ಭ್ರಮೆಯಿಲ್ಲಾಪ್ಪ. ಅನವರತವೂ ನನ್ನ ಹೃದಯಾಂತರಾಳದಲ್ಲಿ ಯಾವುದು ಕ್ಷರಭಾವವಿಲ್ಲದೇ ಪ್ರಕಾಶಿಸುತ್ತಿದೆಯೋ, ಅದೇ ನನ್ನ ಪುಸ್ತಕವಾಗಿದೆ. ಈ ಜಗತ್ತಿಗೆಲ್ಲಾ ಬಾಂಬು ಬಿದ್ದರೂ ಯಾವುದು ಮಾತ್ರ ಅಳಿವಿಲ್ಲದೇ ಎಂದೆಂದಿಗೂ ಅಚ್ಯುತವಾಗಿ ರಾರಾಜಿಸುತ್ತಿದೆಯೋ, ಅದೇ ನನ್ನ ಅಂತರಂಗದ ಪುಸ್ತಕವಾಗಿದೆ. ಲೋಕದಲ್ಲಿ ಹೇಗಾದರೆ, ಕಣ್ಣಿನಲ್ಲಿ ನೋಡುವ ಶಕ್ತಿ ಕುಗ್ಗಿದಾಗ, ಕನ್ನಡಕವೂ ಪಕ್ಕದಲ್ಲಿಲ್ಲವಾದಾಗ `ನೀವು ಓದಿ ಹೇಳಿ' ಎಂದು ಮತ್ತೊಬ್ಬರಲ್ಲಿ ಹೇಳಿ ಅವರಿಂದ ಓದಿಸಿ ಅವರ ದ್ವಾರಾ ಕೇಳಿ ತಿಳಿದು ಕೊಳ್ಳುವುದುಂಟೋ ಹಾಗೆ, ಸನ್ನಿವೇಶವಿಶೇಷದಲ್ಲಿ ಈ ಪುಸ್ತಕವೂ ಪಕ್ಕದಲ್ಲಿದ್ದ ಸ್ನೇಹಿತರಂತೆ ದ್ವಾರವಾಗಿದ್ದು ಉಪಯೋಗಕ್ಕೆ ಬರಬಹುದಷ್ಟೆ. ಅದೂ ಕೂಡ ಪ್ರಾಮಾಣಿಕತೆಯೊಡನೆ ಅನುಭವಿಗಳಿಂದ ಬಂದ ಅಂಶವೇ ಪ್ರಕಟವಾಗಿದ್ದರೆ ಮಾತ್ರ. ಹಾಗಿಲ್ಲದೇ ಇದ್ದಾಗ ಸ್ವಲ್ಪ ಮಾತ್ರಕ್ಕೂ ಉಪಯೋಗಾರ್ಹವಾಗುವುದಿಲ್ಲ

\section*{ಮುದ್ರೆಯ ಸ್ವರೂಪವನ್ನು ಶೋಧನೆ ಮಾಡಿ ಹಿಂದಿರುವ ಭಾವವನ್ನರಿಯಬೇಕು}

ನಮ್ಮ ದೇಹದಲ್ಲಿ ಭೌತಿಕವಾಗಿಯೇ ಅನೇಕ ಮುದ್ರೆಗಳು ಸಂತೋಷ ಮತ್ತು ಸಂಕಟಗಳಿಂದ ಹುಟ್ಟಿಕೊಳ್ಳುತ್ತವೆಯಾದರೂ, ಅದು ಯಾವ ಆಶಯದ ಅಭಿವ್ಯಕ್ತಿಯಾಗಿ ಹೊರಬಂತೋ ಆ ಆಶಯವನ್ನೇ ಹೊರಹೊಮ್ಮಿದ ಮುದ್ರೆಗಳಿಂದ ತೆಗೆದುಕೊಳ್ಳಬೇಕು. ಕಣ್ಣಿನಲ್ಲಿ ನೀರು ಬರುವುದುಂಟು. ಅದಕ್ಕೆ ಕೆಲವು ವೇಳೆ ದುಃಖವೂ, ಕೆಲವು ವೇಳೆ ಸಂತೋಷವೂ ಕಾರಣವಾಗುವುದು. ಅದೂ ಒಂದು ಮುದ್ರೆಯೇ. ಅದನ್ನು ನಾವು ನೋಡಿದಾಗ ಅದು ನಮ್ಮನ್ನು ಆಯಾ ಮೂಲಸನ್ನಿವೇಶಕ್ಕೆ ಒಯ್ಯಬೇಕಾದರೆ ಇದರ ಜೊತೆಗೇ ಇರುವ ಇನ್ನೂ ಬೇರೆ ಬೇರೆ ಮುದ್ರೆಗಳೊಡನೆ ಈ ಕಣ್ಣೀರನ್ನು  ಅರ್ಥಮಾಡಿಕೊಳ್ಳಬೇಕಾಗುವುದು. ಸ್ವಲ್ಪ ಕತ್ತಲೆಯಾಗಿದ್ದು ಅದರ ಉಳಿದ ಅಂಶಗಳು ಕಾಣದೇ ಇದ್ದಾಗ ಕಣ್ಣಿನ ಹತ್ತಿರ ಕೈಯಿಟ್ಟು ನೋಡಿದರೆ ನೀರಲ್ಲಿ ಬಿಸಿಕಂಡರೆ ದುಃಖಾಶ್ರುವೆಂದೂ, ತಂಪಾಗಿದ್ದರೆ ಆನಂದಾಶ್ರುವೆಂದೂ ತಿಳಿಯಬೇಕು. ಹೀಗೆ ಶೋಧನೆ ಯಿಂದ ಎರಡಕ್ಕೂ ಪರಸ್ಪರ ಭೇದವನ್ನು ಗ್ರಹಿಸಬೇಕು. ಇಲ್ಲದಿದ್ದರೆ ಎರಡೂ ಸನ್ನಿವೇಶದಲ್ಲೂ ದುಃಖಾಶ್ರುವೇ ಎಂದೋ ಇಲ್ಲದಿದ್ದರೆ ಆನಂದಾಶ್ರುವೇ ಎಂದೋ ಗ್ರಹಿಸಬೇಕಾದೀತು.

\section*{ಲಕ್ಷ್ಯವೊಂದಿದ್ದರೆ ಅದರ ಹೊರ ಗುರುತು ಲಕ್ಷಣ}

ಹೃದಯದಲ್ಲಿ ಸಂತೋಷವೇ ಆಗಲೀ, ದುಃಖವೇ ಆಗಲೀ ಮೂಡಿಬಂದಾಗ ಅವೆರಡನ್ನು ಇರುವಂತೆಯೇ ನೈಜವಾಗಿ ತಿಳಿದುಕೊಳ್ಳುವುದು ಯಾವಾಗ ಸಾಧ್ಯವೆಂದರೆ, ಅಭಿವ್ಯಕ್ತಿಯಲ್ಲೂ ಪ್ರಾಮಾಣಿಕತೆಯಿದ್ದಾಗ ಮಾತ್ರ. ಅದಕ್ಕಾಗಿ ಸಂತೋಷ ಸಂಕಟಗಳ ಲಕ್ಷಣ ಜ್ಞಾನವೂ ಇರಬೇಕು. ಸಂತೋಷಕ್ಕೇನು ಲಕ್ಷಣ? ಎಂದರೆ, `ತುಟಿಗಳೆರಡೂ ಬಿಚ್ಚಿ ಹಲ್ಲು ಕಾಣುತ್ತಿರಬೇಕು. ಹಹಹಹ ಶಬ್ದ ಬರಬೇಕು' ಎಂದಿಷ್ಟೇ ಆಗುವುದಾದರೆ, ಆ ತರಹ ಮಾಡಿ ಮೇಣದ ಗೊಂಬೆಯೊಂದನ್ನಿಟ್ಟಿದ್ದರೆ, ಅದಕ್ಕೂ ಸದಾ ಎಷ್ಟು ಸಂತೋಷ? ಎನ್ನಬೇಕಾಗುತ್ತೆ. ಆದರೆ ಅದಕ್ಕೆ ಸಂತೋಷವಿರಲು ಸಾಧ್ಯವೇ? ಸಂತೋಷವಾಗಲೀ ಸಂಕಟವಾಗಲೀ ಹೃದಯವಿದ್ದೆಡೆಯಲ್ಲಿ ಮಾತ್ರವೇ. ಇಲ್ಲಿ ಗೊಂಬೆಗೆ ಹೃದಯವಿಲ್ಲದ್ದರಿಂದ ಎಲ್ಲಾ ಬಹಿರ್ಲಕ್ಷಣವೇ ಹೊರತು ಒಳಲಕ್ಷಣವಿಲ್ಲ. ಹಾಗಿದ್ದ ಮೇಲೆ ಲಕ್ಷಣಕ್ಕೆ ಲಕ್ಷ್ಯವೇ ಇಲ್ಲ. ಲಕ್ಷ್ಯವೊಂದಿದ್ದರೇ ಅದರ ಹೊರಗುರುತು ಲಕ್ಷಣವೆನಿಸುವುದಷ್ಟೆ. ಲಕ್ಷ್ಯವಿಲ್ಲದಿದ್ದ ಮೇಲೆ ಅದು ಲಕ್ಷಣವೂ ಅಲ್ಲ. ಉದಾಹರಣೆಗೆ ಕಪಟಿಯನ್ನು ತೆಗೆದುಕೊಂಡರೆ, ಅವನಲ್ಲಿ ಪ್ರಾಮಾಣಿಕತೆ ಇರುವುದಿಲ್ಲವಾದ್ದರಿಂದ ಅವನ ಹೃದಯವೇ ಬೇರೆ ಅವನ ಬಾಯೇ ಬೇರೆ. `ಮನಸ್ಯನ್ಯತ್ ವಚಸ್ಯನ್ಯತ್'.\label{102} ಅವನ ನಗುವನ್ನೂ ಅಳುವನ್ನೂ ಯಾವುದನ್ನೂ ನಂಬಲಾಗುವುದಿಲ್ಲ. ಅವನು ಏನನ್ನು  ಹೊರಪಡಿಸುತ್ತಾನೋ ಅದು ಅವನದೇ ಆಗಿರಬೇಕೆಂಬ ನಿರ್ಬಂಧವಿಲ್ಲ. ಅವನಿಗೆ ಅಳಲು ಕಾರಣವೇ ಇಲ್ಲದಿದ್ದರೂ ಸಂದರ್ಭಾನುಗುಣವಾಗಿ ಕೃತಕವಾದ ಅಳು ಅವನಲ್ಲಿರುತ್ತದೆ. ಅನೇಕ ವೇಳೆ ಎದುರಾಳಿಯ ಅಳುವನ್ನು ನೋಡುತ್ತ ಎದುರಾಳಿಗೂ ಅಳು ಬರುವುದೂ ಉಂಟು. ಹಾಗೆಯೇ ಎದುರಾಳಿ ನಕ್ಕಾಗ ಅವನ ಜತೆಗೆ ಸೇರಿಕೊಂಡು ಮತ್ತೊಬ್ಬನೂ ನಗಬಹುದು. ಅಷ್ಟೇ ಅಲ್ಲದೇ ಲೋಕಕ್ಕೋಸ್ಕರ ತಾತ್ಕಾಲಿಕ ಅಳುನಗುಗಳನ್ನು ತನಗೆ ತಂದುಕೊಂಡು ಬುದ್ಧಿಪೂರ್ವಕವಾಗಿ ಅಳುವುದೂ ನಗುವುದೂ ಉಂಟು. ಇಲ್ಲೆಲ್ಲಾ ಇವುಗಳ ಹಿಂದೆ ಮೂಲವು ಸಿಗುವುದಿಲ್ಲ. ಏಕೆಂದರೆ ನಿಜವಾಗಿ ಅಳುವವನಲ್ಲಿ ಶೋಕಕಾರಣವಾವುದು ಇರುವುದೋ ಅದು ಇವನಲ್ಲಿ ಒದಗಿಲ್ಲವಾದ್ದರಿಂದ, ಹೊರಗಡೆ ಕಂಡದ್ದು ಒಳಗಿನದರ ಪರಿಣಾಮವಾಗಿ ಬಂದಿಲ್ಲ. ಹೊರಗಿನಿಂದಲೇ ಬಂದು ಹೊರಗೇ ನಿಂತು ಹೋಗುತ್ತೆ. ಹಾಗೆಯೇ ಗಂಡ ಹೆಂಡಿರ ಜಗಳ ಕೆಲವು ವೇಳೆ ಭಯಂಕರವಾಗಿ ಬೆಳೆಯುತ್ತೆ. ಸಿಕ್ಕಾಪಟ್ಟೆ ಕೋಪ, ಬಿರುಸು ಮಾತು, ತಲೆಚಚ್ಚಿಕೊಳ್ಳುವುದು ಇತ್ಯಾದಿಯಾಗಿ ಎಲ್ಲಾ ಸೇರಿಕೊಂಡು ಜಗಳವಾಗುತ್ತಿರುತ್ತೆ. ಆ ಸಮಯದಲ್ಲಾರಾದರೊಬ್ಬರು ಪರಿಚಿತರೂ ಗೌರವಸ್ಥರೂ ಬಾಗಿಲಲ್ಲಿ ಬಂದು ನಿಂತು, `ಒಳಗಿದ್ದಾರೆಯೇ ಯಜಮಾನ್ರು?' ಎಂದು ಕೇಳಿದೊಡನೆಯೇ ಕೂಡಲೇ, `ನಿಮಗೆ ದಮ್ಮಯ್ಯಾ ಅಂತೇನೇ ಸ್ವಲ್ಪಹೊತ್ತು ಸುಮ್ಮನಿರಿ! ಮಾನ ಹೋಗುತ್ತೆ' ಎಂದು ಒಬ್ಬರಿಗೊಬ್ಬರು ಹೇಳಿಕೊಂಡು, ರಾಜೀ ಮಾಡಿಕೊಂಡು ಕಣ್ಣೀರನ್ನೊರಸಿ, ತಮ್ಮ ಈಗಿದ್ದ ಕೋಪ ತಾಪ ಜಗಳಗಳ ಮುದ್ರೆಗಳನ್ನೆಲ್ಲಾ ಮಾರ್ಪಡಿಸಿಕೊಂಡು, ತಾತ್ಕಾಲಿಕವಾಗಿ ನಗುವಿನಂತೆ ತೋರಿಸಿಕೊಳ್ಳುತ್ತಾ, `ಬನ್ನಿ! ಆರೋಗ್ಯವೇ?' ಇತ್ಯಾದಿ ಯೋಗಕ್ಷೇಮ ವಿಚಾರ ಶುರು ಮಾಡುತ್ತಾರೆ. ಅವರು ಮಾತುಮುಗಿಸಿ ಹಿಂತಿರುಗಿ ಹೋದರೆ ಆ ಕೂಡಲೇ `ಅಲ್ಲಾಂದ್ರೇ' ಅಂತ ಮತ್ತೆ ಶುರು ಜಗಳ. ಹೋದವರೇನಾದರೂ `ಟವೆಲ್ ಬಿಟ್ಟಿದ್ದೆ, ಬ್ಯಾಗ್ ಬಿಟ್ಟಿದ್ದೆ' ಎಂದು ಹೇಳುತ್ತ ಬಂದು ಬಿಟ್ಟರೆ ಮತ್ತೆ ತಾತ್ಕಾಲಿಕ ರಾಜಿ. ಹೀಗೆಲ್ಲಾ ಇರುತ್ತೆ ಜನರ ಜೀವನ.

\section*{ಪ್ರಾಮಾಣಿಕತೆಯ ಲಕ್ಷಣ}

ಪ್ರಾಮಾಣಿಕತೆಯ ಲಕ್ಷಣವಾದರೂ ಏನು? ಸುಖವೇ ಆಗಲೀ ದುಃಖವೇ ಆಗಲೀ ಯಾವುದೇ ಭಾವವೂ ವಾಸ್ತವವಾಗಿ ಹೃದಯದಲ್ಲಿ ಮೂಡುವುದಾದರೆ, ಹೃದಯದಲ್ಲಿ ಮೂಡಿದ್ದೇ ಹೊರಗಡೆ ನೀತಿಯನ್ನು ಹೊಂದಿ ಪ್ರಕಟವಾಗುವಂತಿರಬೇಕು. ಹಾಗಿಲ್ಲದಾದಾಗ ಪ್ರಾಮಾಣಿಕತೆಗೆ ಎಡೆಯಿಲ್ಲ. ಉದಾಹರಣೆಗೆ ಒಬ್ಬ ಮನುಷ್ಯನಿಗೆ ಯಾರೋ ಒಬ್ಬನ ಮೇಲೆ ಹೆಚ್ಚಾದ ಕೋಪವಿದೆಯೆಂದುಕೊಳ್ಳಿ. ಆವಾಗ ಅವನು `ಅಂದುಕೊಳ್ಳುತ್ತಾನೆ ಅವನು ಬರಲಿ, ನಾಲ್ಕು ಬಾರಿಸಿ ಕಳುಹಿಸುತ್ತೇನೆ-' ಎಂದು. ಆವಾಗ ಯೋಚಿಸುತ್ತಾನೆ ಹೀಗೆಯೇ ಕೂಗಿದರೆ ಅವನು ಬರುವುದಿಲ್ಲ. ಆದ್ದರಿಂದ ನಗುಮುಖವಾಗುತ್ತ ಸ್ವಾಗತಕೊಟ್ಟು ಕರೆಯುತ್ತೇನೆ ಎಂದು. ಹಾಗೆ ಕರೆದಾಗ, ಅವನು ಧೈರ್ಯವಾಗಿ ಬಂದಾಗ ಪಟಪಟ ಏಟು ಬೀಳುತ್ತೆ. ಆವಾಗ ಏಟು ತಿಂದವನಿಗೆ ಕಕ್ಕಾಬಿಕ್ಕಿಯಾಗುತ್ತದೆ. ಇದೇನಿದು? ಹೀಗಾಯಿತು? ಎಂದು ಕೊಳ್ಳುತ್ತನೆ. ಅವನ ನಗುವನ್ನು ಸರಿಯಾಗಿ ಲಕ್ಷಣದೊಂದಿಗೆ ಅರ್ಥಮಾಡಿಕೊಂಡಿದ್ದರೆ, ಅದರಲ್ಲಿ ಪ್ರಾಮಾಣಿಕತೆಯಿಲ್ಲವೆಂದು ಗೊತ್ತಾಗುತ್ತಿತ್ತು. ಹೊರಗಿನ ನಗುವಿಗೆ ಮೋಸ ಹೋದದ್ದರಿಂದ ಹೀಗಾಯಿತು. ಹೀಗೆ ಹೊರಗಿನ ಶಿರಪ್ಪು (ನಗುವು) ನೆರುಪ್ಪಾಯಿತು (ಬೆಂಕಿಯಾಯಿತು).

\section*{ಗ್ರಂಥದ ಲಕ್ಷಣ}

ಏತಕ್ಕೆ ಈ ಮಾತನ್ನಾಡಿದ್ದು? ಎಂದರೆ ಹೃದಯದ ಭಾವವನ್ನು ಹೊರಪಡಿಸುವ ಪ್ರಾಮಾಣಿಕತೆಯಿಂದಿರಬೇಕಾದ್ದು ಗ್ರಂಥ. ಹಾಗಿದ್ದಾಗ ಮಾತ್ರ ಜೀವನಯಾತ್ರೆಗೆ ಒಂದು ಗೈಡೆನ್ಸಾಗಿರುತ್ತೆ. ಪುಸ್ತಕವು ಅಂತಹ ಗಗೈಡಾಗಿರುವುದಾದರೆ ನಮಗೆ ಸಹಕರಿಸುವಂತಾಗುತ್ತೆ. ಜೀವನವೇ ವಿಸ್ಮೃತಿಮಯವಾದ್ದರಿಂದ `ನಷ್ಟೋ ಮೋಹಃ ಸ್ಮೃತಿರ್ಲಬ್ಧಾ'\label{104} ಎನ್ನುವಂತೆ ಮಾಡಲು ಪೂರ್ವಾನುಭೂತವಾದ ವಿಷಯವನ್ನು ಮನಸ್ಸಿನಿಂದ ತೆಗೆದುಕೊಂಡು ಒಂದೆಡೆ ಗುರುತು ಹಾಕಿದ್ದು ನಮಗೇ ಆಗಲೀ ಮತ್ತೆ ಬೇರೆಯವರಿಗೇ ಆಗಲೀ ಅನುಭೂತವಿಷಯಾಸಂಪ್ರಮೋಷಕ್ಕೆ- ಸ್ಮೃತಿಗೆ ಅನುಕೂಲ ಮಾಡುವುದು ತಪ್ಪೇನಲ್ಲ. ಅನುಭೂತವಾದ ಅಂಶವು ಒಂದೆಡೆ ಗುರುತುಹಾಕಿದ್ದರೆ, ಆಗಾಗ್ಗೆ ಮರೆವೆಯೊದಗಿದಾಗ ಗುರುತನ್ನು ನೋಡಿ ಪೂರ್ವಸ್ಮೃತಿ ತಂದುಕೊಳ್ಳಬಹುದು. ಆವಾಗ ಆ ಗುರುತು ಆ ನೆಲೆಗೆ ನಮ್ಮನ್ನು ಕೊಂಡೊಯ್ಯುತ್ತೆ. 

\section*{ಗ್ರಂಥಗಳೂ ಸಂಸ್ಕಾರವಿಲ್ಲದವನಿಗೆ ಗ್ರಂಥಿಯಾಗಿಯೇ ನಿಲ್ಲುತ್ತೆ}

ಆದರೆ ಆ ಗುರುತನ್ನು ನೋಡಲೂ ಸಂಸ್ಕಾರವಿರಬೇಕು. ಸಂಸ್ಕಾರವಿಲ್ಲದೇ ಇದ್ದರೆ ಗುರುತು ಯಾವುದೇ ಕೆಲಸವನ್ನೂ ಮಾಡುವುದಿಲ್ಲ. ಉದಾಹರಣೆಗೆ ಸಂಗೀತಗಾರನೊಬ್ಬನಿಗೆ ಅಂತರಂಗದಲ್ಲಿ ಒಂದು ರಾಗವು ಹುಟ್ಟಿಕೊಂಡಾಗ, ಒಂದು ಪುಸ್ತಕ ತೆಗೆದುಕೊಂಡು ಅಡೈಯಾಳ ಗುರುತುಮಾಡಿಟ್ಟರೆ, ಮತ್ತೊಮ್ಮೆ ಅದನ್ನು ನೋಡಿದ ಕೂಡಲೇ ಮತ್ತೆ ಅಲ್ಲಿಗೇ ಕರೆದೊಯ್ಯುತ್ತೆ. ಅದೇ ಸಂಗೀತ ಸಂಸ್ಕಾರವೇ ಇಲ್ಲದವರು ಆ ಗುರುತನ್ನು ನೋಡಿದಾಗ `ನೀದದ್ದಪ್ಪ ನಿಮ್ಮಮ್ಮದದ್ದಮ್ಮ' `ನಿದದಪ ನಿಮಮದದಮ' ಇಷ್ಟೇ ತಾನೇ ಅವರಿಗೆ ಕಾಣುವುದು. ಅದಕ್ಕೆ ಅರ್ಥವೇನಿರುತ್ತೆ? ಅಲ್ಲಿ ಯಾವ ರಾಗ ಸೌಂದರ್ಯಕ್ಕೂ ಜಾಗವಿಲ್ಲ. ಹಾಗೇ ಗ್ರಂಥಗಳೂ ಸಂಸ್ಕಾರವಿಲ್ಲದವನಿಗೆ ಗ್ರಂಥಿಯಾಗಿಯೇ ನಿಲ್ಲುತ್ತೆ.  ಸಂಗೀತಜ್ಞಾನವಿಲ್ಲದ ಅಸಂಸ್ಕಾರಿಗೆ `ನಿಮಮ ದದಪ' ಹೇಗೋ ಹಾಗಾಗುತ್ತೆ. ಗ್ರಂಥಿಯೂ ಸಹ ಕೆಲವೆಡೆ ಸಹಕಾರಿಯಾಗುತ್ತೆ. ಉದಾಹರಣೆಗೆ ಅಂಗಡಿಗೆ ಹೊರಡುವಾಗ ಮುಖ್ಯವಾದ ಯಾವುದೋ ಒಂದು ಸಾಮಾನು ತರಬೇಕಾದ್ದು ಮರೆಯದೇ ಇರಲು ನಮ್ಮ ಬಟ್ಟೆಯ ಅಂಚಿನಲ್ಲಿ ಗಂಟು ಹಾಕಿಕೊಂಡು ಹೊರಡುವ ರೂಢಿ ಹಳೆಯ ಕಾಲದಲ್ಲಿತ್ತು. ಯಾರು ಆ ಗಂಟು ಹಾಕಿಕೊಂಡನೋ ಅವನಿಗೆ ಆ ಸಂಸ್ಕಾರವಿರುವುದರಿಂದ ಗಂಟು ಕೈಗೆ ತಗುಲಿದ ಕೂಡಲೇ, ತರಬೇಕಾದ ಸಾಮಾನು ನೆನಪಿಗೆ ಬರುತ್ತೆ. ಅದೇ ಆ ಬಟ್ಟೆಯಗಂಟು ಒಬ್ಬ ಕಳ್ಳನ ಕೈಗೆ ಸಿಕ್ಕಿದರೆ, ಅದರಲ್ಲೇನಾದರೂ ದುಡ್ಡಿಟ್ಟಿರಬಹುದೇ? ಎಂದು ಅವನು ನೋಡುತ್ತಾನೆಯೇ ಹೊರತು ತರಬೇಕಾದ ಸಾಮಾನಿನ ನೆನಪು ಅವನಿಗೆ ಬರದು. ಹಾಗೆಯೇ ಈ ಗ್ರಂಥಗಳು ಗುರುತು ಹಾಕಬೇಕಾದ್ದನ್ನು  ನೆನಪಿಸುವ ಗ್ರಂಥಿಗಳೇ ಆಗಿವೆಯಾದ್ದರಿಂದ, ಅದರ ಬಗ್ಗೆ ಪೂರ್ವ ಸಂಸ್ಕಾರವಿಲ್ಲದವರಿಗೆ ಅದು ಗಂಟಾಗಿ ಉಳಿಯುತ್ತೆಯೇ ವಿನಹ ಯಾವ ನೈಜವಾದ ಅಂಶಕ್ಕೂ ಸ್ಮಾರಕವಾಗುವುದಿಲ್ಲ. ಕ್ರಮೇಣ ಒಬ್ಬೊಬ್ಬರಿಗೆ ಒಂದೊಂದು ಅವರ ಸಂಸ್ಕಾರಾನುಗುಣವಾಗಿ ಹೊಳೆಯಲು ಶುರುವಾಗುತ್ತೆ. ಹಿಂದೆ ಹೇಳಿದ ಬಟ್ಟೆ ಅಂಚಿನ ಗಂಟು ಸೀರೆಯಲ್ಲಿದ್ದು ಹೆಂಗಸರಿಗೆ ಸಿಕ್ಕಿದರೆ, ಸ್ನಾನಕ್ಕೆ ಹೋಗುವಾಗ ಹಾಕಿಕೊಂಡ ಅರಿಶಿನದ ಗಂಟಿರಬಹುದೇ? ಎಂಬ ಜಿಜ್ಞಾಸೆ. ಸೀರೆ ಒಣಗಿಸಲು ಮನೆಯ ಸೂರಿಗೋ ಗಿಡಕ್ಕೋ ಕಟ್ಟಿದ್ದನ್ನು ಹಾಗೆಯೇ ಎಳೆದುಕೊಂಡು ಬಿಚ್ಚದೇ ಇದ್ದಿರಬಹುದೇ? ಹೀಗೆಲ್ಲಾ ಹುಟ್ಟಿದರೆ `ಅಥತೋ ಗ್ರಂಥಿಜಿಜ್ಞಾಸಾ' ಎಂದಾಗಲೀ `ಅಥಾತೋ ಗ್ರಂಥಿಂ ವ್ಯಾಖ್ಯಾಸ್ಯಾಮಃ' ಎಂದೋ ಶುರುವಾಗಬೇಕಾಗುತ್ತದೆ. ಹೀಗೆ ಗ್ರಂಥಿಯನ್ನು ಬಿಡಿಸಹೊರಟು ಮತ್ತಷ್ಟು ಗ್ರಂಥಿಗಳಾಗುವುದೂ ಉಂಟು. ಹಾಗೆಯೇ ಗ್ರಂಥಗ್ರಂಥಿಯೂ, ಎಣೆಯಿಲ್ಲದ ಸಂಶಯಗ್ರಂಥಿಗಳೂ, ಅವನ್ನು ಬಿಡಿಸಲು ಹೊರಟು ಮತ್ತಷ್ಟು  ಗ್ರಂಥಿಗಳೂ ಬೆಳೆದರೆ, ಅದಕ್ಕಿಂತ ಆಗಬೇಕಾದ ಪ್ರಮಾದ ಇನ್ನ್ನೇನಿದೆ? ಆದ್ದರಿಂದಲೇ `ಗ್ರಂಥತಃ ಅರ್ಥತಶ್ಚ ಶಾಸ್ತ್ರಪ್ರಾಪ್ತಿಃ'\label{105} ಎಂಬ ಸಂಪ್ರದಾಯ ಶುರುವಾಗಿ, ಅದೂ ಅವರವರ ಕೈ ತಗುಲಿ, ಅದರ ಅಂಶವನ್ನೇ ಮುಂದಕ್ಕೆ ತರುತ್ತಾ, ಅವರಿವರ ಕೈ ಗುರುತೇ ಸಂಪ್ರದಾಯವಾಗಲಾರಂಭಿಸಿತು. ಆದ್ದರಿಂದ ಗ್ರಂಥದ-ಪುಸ್ತಕದ ಬಗ್ಗೆ ಎಷ್ಟು ಎಚ್ಚರಿಕೆ ಬೇಕು ನೋಡೀಪ್ಪಾ!.

\section*{ಮಹಾಗುರುವು ಪುಸ್ತಕಗಳನ್ನು ಸಂಗ್ರಹಿಸಿರುವ ಉದ್ದೇಶ}

ಇಂತಹ ಸನ್ನಿವೇಶದಲ್ಲಿ ನೀವೇಕೆ ಇಷ್ಟು ಪುಸ್ತಕಗಳನ್ನು ರಾಶಿಹಾಕಿಕೊಂಡಿದ್ದೀರಿ? ನಿಮ್ಮ ಉದ್ದೇಶವೇನು? ಎಂದರೆ ಲ್ಯಾಬೋರೇಟರಿಯಲ್ಲಿ ಪ್ರಾಕ್ಟಿಕಲ್  ಆಗಿ ತೋರಿಸುವವರಾಗಿದ್ದರೂ, ಕೆಲವು ಥಿಯರಿಗಳನ್ನು ತಿಳಿಸಿ ನಂತರ ಪ್ರಯೋಗದಲ್ಲಿ ತಿಳಿಸುತ್ತಾರೆ. ಪ್ರಯೋಗಕ್ಕೆ ಹೊಂದಿಕೊಳ್ಳದ ಅಂಶಗಳೇನಾದರೂ ಥಿಯರಿಗಳಲ್ಲಿ ಕಂಡು ಬಂದರೆ, ಇದು ಪ್ರಯೋಗ ಸಿದ್ಧವಾಗುವುದಿಲ್ಲ, ಪ್ರಮಾದ ಸಿದ್ಧವಾಗಿದೆ ಎಂದು ತಿಳಿಸಿ ಥಿಯರಿಗಳಲ್ಲಿ ಸಂಭವಿಸಬಹುದಾದ ಭ್ರಮ ಪ್ರಮಾದಗಳ ಬಗೆಗೆ ಎಚ್ಚರಿಕೆ ಕೊಡುತ್ತಾರೆ. ಹಾಗೆ ಋಷಿಗಳೋ ಆ ಪೀಳಿಗೆಯಲ್ಲಿ ಬಂದವರೋ ತಂದ ಶಾಸ್ತ್ರಗ್ರಂಥಗಳು ನಮಗೆ ಮೊದಲೇ ಬಂದಿರುವುದರಿಂದ, ಅವರು ಹೀಗೆ ನಿಜಾಂಶವನ್ನು ತಿಳಿದು ವ್ಯಕ್ತಪಡಿಸಿದ್ದಾರೆ. ಅದರ ಪ್ರಯೋಗ ಹೀಗೆ, ಈ ವಿಷಯ ಅಂತರ್ಮಾರ್ಗದಲ್ಲಿ ಸಾಗಿದವರಿಗೆ ಹಿಂದೂ ಗೋಚರವಾಗಿರುವ ವಿಷಯ ಎಂದೂ, ಮಧ್ಯಕಾಲದ ಕೈವಾಡದಿಂದ ರೂಪುಗೆಟ್ಟಿದ್ದರೆ, ಹೀಗೆ ಸತ್ಯಕ್ಕೆ ದೂರವಾದ ವಿಷಯಗಳೂ ಪುಸ್ತಕದಲ್ಲಿ ಸೇರುತ್ತವೆ, ಪುಸ್ತಕಕ್ಕೆ ಶರಣಾಗಬೇಡಿ! ಸಾವಧಾನತೆಯಿರಲಿ! ಎಂದು ತಿಳಿಸುವುದಕ್ಕಾಗಿಯೂ ಬಳಸಿಕೊಳ್ಳುತ್ತೇನೆಪ್ಪ!

\section*{ಪುಸ್ತಕವು ಸತ್ಯಕ್ಕೆ ಲೀಡ್ ಮಾಡುವಂತಿರಬೇಕು}

ಪುಸ್ತಕವನ್ನು ನೋಡುವುದೇ ತಪ್ಪೆಂದಲ್ಲ. ಪುಸ್ತಕವು ಸತ್ಯಕ್ಕೆ ಲೀಡ್ ಮಾಡುವಂತಿದ್ದರೆ ಸರಿ. ಉದಾಹರಣೆಗೆ ಪ್ರಣವದ ಸ್ವರೂಪವೇನು? ಎಂಬ ಪ್ರಶ್ನೆ ಬಂದಾಗ-

\begin{shloka}
`ಪ್ರತ್ಯಗಾನಂದಂ ಬ್ರಹ್ಮಪುರುಷಂ ಪ್ರಣವಸ್ವರೂಪಮ್ |\label{106a}\\
ಅಕಾರ ಉಕಾರ ಮಕಾರ ಇತಿ| ತಾನೇಕಧಾ ಸಮಭರತ್ತದೇತದೋಮಿತಿ |\\
ಯಮುಕ್ತ್ವಾ ಮುಚ್ಯತೇ ಯೋಗೀ ಜನ್ಮಸಂಸಾರಬಂಧನಾತ್ ||
\end{shloka}

ಎಂದು ಋಷಿವಾಣಿಯಿಂದ ಗ್ರಹಿಸಿದರೂ ತಪ್ಪೇನಿಲ್ಲ. ವಿಷಯವು ನೈಜವಾದ್ದರಿಂದ ಅದನ್ಮ್ನು ಪುಸ್ತಕದ ಮೂಲಕವಾಗಿಯೂ ಗ್ರಹಿಸಬಹುದು. ಆದರೆ ಪುಸ್ತಕದಲ್ಲೇ ನಿಂತು ಬಿಡದೇ ಅದರ ನೆಲೆಗೆ ಹಿಂತಿರುಗಬೇಕು. `ಪ್ರತ್ಯಕ್' ಎಂಬುದನ್ನು ಕೇಳಿ ಹಿಂತಿರುಗುವ ಅಂಶವನ್ನು ಮರೆತು, ಪ್ರತ್ಯಕ್ಕೆಂದರೇನು? ಪ್ರತಿ-ಅಕ್ಕೇ? ಪ್ರತ್ಯಕ್ಕೇ? ಇಲ್ಲಿ ಸಂಧಿಯು ಯಾವುದು? ಉಪಸರ್ಗಕ್ಕೆ ಅರ್ಥವೇನು? ಧಾತುವಾವುದು? ಅದಕ್ಕೇನು ಅರ್ಥ? ಯಾವ ಪಾಣಿನಿ ಸೂತ್ರದಿಂದ ಇದೆಲ್ಲದರ ರೂಪವಾಯಿತು? ಆ ಸೂತ್ರದ ಅರ್ಥತಾತ್ಪರ್ಯಗಳೇನು? ಆ ಸೂತ್ರಕ್ಕೆ ನೀವು ಹೇಳುವ ಅರ್ಥವೇ ಹೌದು ಎಂದೇನು ನಿರ್ಥಾರ? ಮತ್ತೊಂದು ತರಹ ಅರ್ಥವೇಕಿರಬಾರದು? ಭಾಷ್ಯ ಹೀಗಿದೆ. ಕೈಯಟ ಹಾಗಿದೆ, ಎಂದು ಎಲ್ಲವನ್ನೂ ಕಸಕಲು ಹೊರಟರೆ ಅಡ್ಡದಾರಿ ಹಿಡಿದಂತಾಗುವುದು. ಅಲ್ಲಿ ಋಷಿವಾಣಿಯ ಪ್ರಯೋಜನವು ನಯೇ ಪೈಸೆಯಷ್ಟೂ ದೊರಕದು. ನಿಜವಾದ ಯಾವುದಾದರೂ ಹಣ್ಣನ್ನು ಹೀಗೆ ಕಸಕಿದ್ದರೆ ಒಂದಿಷ್ಟು ರಸವಾದರೂ ಸಿಗುತ್ತಿತ್ತು. ಈ ಪುಸ್ತಕದ ಹಣ್ಣನ್ನು ಕಸಕಿದರೆ ಯಾವ ಲಾಭವೂ ಇಲ್ಲ. ಆಯುಸ್ಸಿಗೇ ಅಪವ್ಯಯ. ಪಂಚಾಂಗ ನೋಡಿದರೆ ವರ್ಷಪೂರ್ತಿ `ಕೃತ್ತಿಕೆ ಮಳೆ, ಪುಬ್ಬೇ ಮಳೆ' ಎಂದು ಏನಾದರೊಂದು ಮಳೆಯಿಲ್ಲದೇ ಯಾವ ಹಾಳೆಯೂ ಇಲ್ಲ. ಆದರೆ ಒಂದು ಚಮಚಾ ನೀರೂ ಸಿಗುವುದಿಲ್ಲ. ಆದ್ದರಿಂದ ಪಂಚಾಂಗ ಹಿಂಡಿದರೆ ವ್ಯರ್ಥ. ಆಕಾಶದ ಕಡೆಗೆ ತಿರುಗಿ ನೋಡಬೇಕು. ಪಂಚಾಂಗ ಹಿಂಡಿದರೆ ಸುಸ್ತಾಗಿ ಕಣ್ಣಲ್ಲಿ ನೀರು ಬರಬಹುದಷ್ಟೆ. ಆದ್ದರಿಂದಲೇ-

\begin{shloka}
`ನಾಽವೇದವಿನ್ಮನುತೇ ತಂ ಬೃಹಂತಮ್ |\label{106}\\
ಸರ್ವಾನುಭುಮಾತ್ಮಾನಗ್ಂ ಸಂರಾಯೇ ||'
\end{shloka}
ಎಂದು ಶ್ರುತಿಯು ಹೇಳುತ್ತಿದೆ. ಅಂದರೆ ವೇದಜ್ಞನಲ್ಲದವನು- ವೇದವನ್ನು ಪಡೆಯದವನು ಬ್ರಹ್ಮರೂಪಿಯಾದ ಅವನನ್ನರಿಯಲಾರ. ವಿದ್ಯಾಪೂರ್ತಿಯಾದಾಗ ಎಲ್ಲರಿಗೂ ಅನುಭವೈಕಗಮ್ಯನಾದ ಆತ್ಮಸ್ವರೂಪನಾಗಿರುವನು- ಎಂದದರ ಆಶಯ.

\section*{ಮೂಲಭೂತವಾದ ದರ್ಶನ ಹಾಗೂ ಸಂಸ್ಕಾರಗಳಿದ್ದಾಗ ಪುಸ್ತಕವು ಉದ್ಬೋಧಕವಾಗಬಹುದು}

`ಪ್ರತ್ಯಗಾನಂದಂ' ಎಂದು ನಾವೇ ಬರೆದಾಗಲೂ ಅಲ್ಲಿ ಆನಂದ- ರಸ- ಭಾವಗಳಾವುವೂ ಇರುವುದಿಲ್ಲ. ಅದು ಇರುವ ಜಾಗವೇ ಬೇರೆ. ನಮ್ಮ ಲಿಪಿ (ಭಾರತೀಯ ಲಿಪಿ)ಯಲ್ಲಿ ಬರೆದರೆ ಭಾಷೆಯಾದರೂ ಕಾಣುತ್ತೆ. ಅದೇ ಇಂಗೀಷ್ ಲಿಪಿಯಲ್ಲಿ ಬರೆದರೆ ಅದೂ ಇಲ್ಲ. `ಪ್ರಟ್ಯಗಾಣಂಡಂ' ಆಗಿಬಿಟ್ಟು  ಅದರ ಕಥೆಯೇ ಬೇರೆ. `ಪುಸ್ತಕವು ಸಹಕರಿಸಬಹುದು, ಆದರೆ ಪ್ರಾಕ್ಟಿಕಲ್ ಇದ್ದಾಗ' ಎಂಬುದಕ್ಕಾಗಿ ಈ ಮಾತು ಹೇಳಿದ್ದು. ಇಲ್ಲಿ ಈ ಋಷಿಸಾಹಿತ್ಯದ ಜೊತೆಗೆ ಒಬ್ಬ ಗುರುವು ಅನುಭವಿಯಾದವನು ಲಭಿಸಿದಾಗ, ಋಷಿಹೃದಯಗಳಲ್ಲಿ ನೆಲೆ ನಿಂತಿರುವ ವಿಷಯವನ್ನು ತನ್ನ ಬುದ್ಧಿಗೂ ಹೃದಯಕ್ಕೂ ತಂದುಕೊಂಡು, ಅದನ್ನು ಅಲ್ಲಿಂದ ಶಿಷ್ಯನಲ್ಲಿಗೂ ತಲಪಿಸಿ ತನ್ನ ಹೃದಯಸಂಶ್ಲಿಷ್ಟನನ್ನಾಗಿಸಿದಾಗ ಶಿಷ್ಯತ್ವಪೂರ್ತಿಯಾಗುವುದು. ಶಿಷ್ಯನೂ ಗುರುಹೃದಯದೊಡನೆ ಒಂದುಗೂಡುವಂತಾಗುವುದು. ಅವನೊಡನೆ ಕಲೆತು ಕಲಾವಂತನಾಗಬಹುದು. ಅಲ್ಲಿ ತಾನೇ ನಿಜವಾದ ರಸಭಾವಗಳ ಹರಿದಾಟಕ್ಕೆ ವಿಷಯವಾಗುವುದು. ಅನುಭವಿಗೂ ವ್ಯವಹಾರ ಜೀವನದಲ್ಲಿದ್ದಾಗ ಮತ್ತೆ ಸ್ಮೃತಿಗೆ ಸಹಾಯವಾಗುವುದು ಪುಸ್ತಕ. ಆದರೆ ಪುಸ್ತಕದರ್ಶನಕ್ಕೆ ಮೊದಲು ಒರಿಜಿನಲ್ಲಾದ ದರ್ಶನ ಮತ್ತು ಅದರಿಂದಾದ ಸಂಸ್ಕಾರ ಇವುಗಳಿದ್ದಿದ್ದರೆ, ಪುಸ್ತಕವು ಸಂಸ್ಕಾರೋದ್ಭೋಧಕವಾಗಬಹುದು. ಇಲ್ಲದಿದ್ದರೆ ನೀರಸ.

ಉದಾಹರಣೆಗೆ ಒಬ್ಬ ರಾಜನನ್ನು ಶಿಲೆಯಲ್ಲಿ ಕಲಾಕೃತಿಯಾಗಿ ಮಾಡಿ ಸ್ಟ್ಯಾಚ್ಯು ಮಾಡಿದ್ದಾರೆ. ಅದನ್ನು ಸಾವಿರಾರು ಜನರು ಬಂದು ನೋಡುತ್ತಾರೆ. ಆದರೆ ಅವರೆಲ್ಲಾ ಅದರಲ್ಲಿ ಏನನ್ನು ನೋಡುತ್ತಾರೆ? ಶಿಲೆಯಲ್ಲಿ ಬಿಡಿಸಿರುವ ಕೋಟು ಶರಟುಗಳೋ, ರುಮಾಲೋ, ಆಭರಣವೋ ನೋಡಿ ಎಂತಹ ಶಿಲ್ಪ? ಎಂದು ಅಚ್ಚರಿಯಿಂದ ಕಲ್ಲು ಕುಟುಕನನ್ನು ಹೊಗಳುತ್ತಾರೆ. ಹಾಗೂ ಹೀಗೂ ಸುತ್ತಡಿ ಕೊನೆಗೆ ಜಾಗವನ್ನು ಖಾಲಿಮಾಡುತ್ತಾರೆ. ಅದೇ ಅವನ ಜೊತೆಯಲ್ಲಿ ಬಾಳಿ ಬದುಕಿದ ಮತ್ತು ಬದುಕನ್ನು ಸವಿದಂತಹ ಅವನ ಸತಿಯೇನಾದರೂ ಆ ಸ್ಟ್ಯಾಚ್ಯುವನ್ನು ನೋಡಿದರೆ, ಅವಳ ದೃಷ್ಟಿಯು ಆ ಕಲ್ಲು ಕೆತ್ತನೆಯ ಮೇಲಷ್ಟೇ ನಿಲ್ಲುವುದಿಲ್ಲ. ಅದನ್ನು ಕಂಡು ಅವಳ ಮನಸ್ಸು ಆ ಜಾಗ ಬಿಟ್ಟು, ಅವನೊಡನಾಡಿದ ಬಾಳುವೆ, ಹಾವಭಾವಗಳು, ಪ್ರೇಮರಸ, ಹೃದಯಕ್ಕೆ ಹೃದಯ ಕೊಟ್ಟುಕೊಂಡು ಅಗಲಲಾಗದಾಗಿದ್ದ ಸಹ ಬಾಳುವೇ, ಅಲ್ಲಿಗೆಲ್ಲಾ ಜೀವಂತವಾಗಿದ್ದೆಡೆಗೆ ಸಾಗಿ, ಆ ಮೂಲಕ ಉಕ್ಕಿಬಂದ ಕಣ್ಣೀರಿನಿಂದ ನಿಂತ ಜಾಗ ಸ್ವಲ್ಪವೂ ತೋಯದೇ ಮುಂದುವರಿಯಲಾರಳವಳು. ಆ ರಾಜನ ಅಂತಃಪುರದ ಸ್ತ್ರೀಯರಿಗೆ ಮಾತ್ರ ಇದರ ಮರ್ಮ ಗೊತ್ತಾಗುವುದು. ಅದೇ ಅವನ ಬಾಹ್ಯಪುರದಲ್ಲಿ ಮಾತ್ರ ಓಡಾಡುವವರಿಗೆ ಇದರ ನಡೆ ಅರಿವಿಗೆ ಬರದು . ಹಾಗೆಯೇ ಈ ಹೃದಯಾಂತಃ ಪುರದಲ್ಲೂ ಅಂತರಾತ್ಮನೊಡನೆ ಅವನ ಮನಮೋಹಕವಾದ ಸೌಂದರ್ಯ ಸೌಕುಮಾರ್ಯಾದಿಗಳನ್ನು ಸವಿದು ಆಡಿಪಾಡಿಕೊಂಡಾಡಿ ಮುದ್ದಾಡಿ ತನ್ಮಯತೆಯನ್ನೂ ತರುವ ಆನಂದಕ್ಕೆ ಪರವಶರಾಗಿ ಅವನೊಡನೆ ಸಹ ಬಾಳ್ವೆ ನಡೆಸಿದ್ದರೆ-

\begin{shloka}
`ಯೋವೇದ ನಿಹಿತಂ ಗುಹಾಯಾಂ ಪರಮೇ ವ್ಯೋಮನ್ |\label{108}\\
ಸೋಽಶ್ನುತೇ ಸರ್ವಾನ್ ಕಾಮಾನ್ ಸಹ | ಬ್ರಹ್ಮಣಾ' |
\end{shloka}
ಎನ್ನುವಂತಾಗಿದ್ದರೆ, ಅವನ ಬಗ್ಗೆ ಸಾಹಿತ್ಯವನ್ನು ಪುಸ್ತಕಗಳಲ್ಲೇ ಕಂಡರೂ, ಅವನ ಒಳಹೊರಗಣ್ಣುಗಳು ತೋಯದೇ ಇರಲಾರವು. ಬೇರೆಯವರಾರಾದರೂ ಪಕ್ಕದಲ್ಲಿದ್ದರೆ, ಪುಸ್ತಕ ನೋಡಿ ಇವನೇಕೆ ಕಣ್ಣೀರು ಬಿಡುತ್ತಾನೆ? ಎಂದು ಯೋಚಿಸುತ್ತಾ, ಬುದ್ಧಿವೈಕಲ್ಯ ವಿರಬಹುದೆಂದೂ ಭಾವಿಸುವರು ಹೊರ ಜೀವಿಗಳು.

\section*{ಆತ್ಮನನ್ನೇ ಬೆಳೆಸಿಕೊಳ್ಳಬೇಕಾದರೆ ಒಂದು ಸರಿಯಾದ ಗುರುಕುಲವನ್ನೇರ್ಪಡಿಸಬೇಕು}

ಅಂತಹ ಆತ್ಮನನ್ನು ಬೆಳೆಸಿಕೊಳ್ಳುವ ಸಲುವಾಗಿಯೇ ಈ ಪುಸ್ತಕಗಳು ಹೊರಟಿದ್ದು. ಆದರೆ ಇಂದು ಆತ್ಮನ ಬೆಳವಣಿಗೆಯನ್ನು ಪೂರ್ತಿಯಾಗಿ ನಿಲ್ಲಿಸಿಬಿಟ್ಟು ತಾನೇ ಬೆಳೆದುಕೊಳ್ಳುತ್ತಿದೆ. ಮತ್ತೆ ಇಂದು ಇದರ (ನಿರ್ಜೀವವಾದ ಪುಸ್ತಕರಾಶಿಯ) ಬೆಳವಣಿಗೆಯನ್ನು ತಡೆದು ಆತ್ಮನನ್ನೇ ಬೆಳಸಿಕೊಳ್ಳುವಂತಾಗಬೇಕಾದರೆ, ಒಂದು ಸರಿಯಾದ ಗುರುಕುಲವನ್ನೇರ್ಪಡಿಸಬೇಕು. ಅಡಿಯಿಂದಲೇ ಶುರುಮಾಡಿ ಅಕ್ಷರಾಭ್ಯಾಸದಿಂದ ಪ್ರಾರಂಭಿಸಿ ಮಾತು ಕತೆ ವ್ಯವಹಾರ ಎಲ್ಲವೂ ಪರಿಷ್ಕಾರಗೊಳ್ಳಬೇಕಾಗಿದೆ. ಏಕೆಂದರೆ ಪ್ರತಿಯೊಂದು ಮಾತಿನಲ್ಲೂ, ಪ್ರತಿಯೊಂದು ಅಕ್ಷರದಲ್ಲೂ ಸ್ವರಸ್ಥಾನವೇ ಮಾರ್ಪಟ್ಟುಹೋಗಿದೆ. ನಿಮಗೆ ತಿಳಿದ ಹತ್ತಾರು ಮಕ್ಕಳಿಗೆ `ಸಹಸ್ರಶೀರ್ಷಾಪುರುಷಃ'\label{108a} ಎಂಬುದನ್ನು ಹೇಳಿಕೊಟ್ಟು ನೋಡಿ! ಅಷ್ಟೂ ಮಕ್ಕಳೂ ಅಷ್ಟೂ ತರಹ ಹೇಳುತ್ತವೆ. ಹೇಳಿಕೊಡುವವರು ಸ್ವರಸ್ಥಾನಾದಿಗಳನ್ನು ಸರಿಯಾಗಿ ಅರ್ಥಮಾಡಿಕೊಂಡಿದ್ದು ಮಂತ್ರದ ಭಾಷೆಯನ್ನು ಹೇಳಿಕೊಡುವುದರ ಜೊತೆಗೆ ಅವುಗಳನ್ನು ಮಕ್ಕಳಲ್ಲೂ ಸಂಕ್ರಾಂತಗೊಳಿಸಲು ತೀವ್ರತೆ ವಹಿಸಿದರೆ ಭಾಷಾನುಪೂರ್ವಿಯು ಬರುವಂತೆ ಅವುಗಳೂ ಬರುತ್ತಿತ್ತು. ವೇದಮಂತ್ರಗಳಲ್ಲಿ ಮಾತ್ರವೇ ಅಲ್ಲ, ಆಡುವ ಭಾಷಯಲ್ಲೂ ಈ ಐಬು ಇದ್ದೇ ಇರುತ್ತೆ. ನಿಮ್ಮ ಹೆಸರು ವರದಾಚಾರ್ಯರು ಎಂದು ತಾನೇ. ಅದೇ ಹೆಸರನ್ನು ಹೇಳುವವರೆಲ್ಲ ಏಕರೂಪವಾಗಿ ಹೇಳುತ್ತಾರೆಯೇ? ಹಿರಿಯರು ಕಿರಿಯರು ಸ್ನೇಹಿತರು ಸಹೋದರರು ಸಹೋದ್ಯೋಗಿಗಳು ಹೆಂಡತಿಮಕ್ಕಳು ಇಷ್ಟೆಲ್ಲಾ ಜನರು ನಿಮ್ಮನ್ನು ಸಂಬೋಧಿಸುವವರಾದರೂ ಒಬ್ಬೊಬ್ಬರು ಒಂದೊಂದು ಬಗೆಯಲ್ಲಿ ಹೇಳುತ್ತರಲ್ಲವೇ?

ಹಾಗೆಯೇ ಅಧ್ಯಾತ್ಮ  ಪ್ರಪಂಚದಲ್ಲೂ ಇಷ್ಟಾರು ಜನ ವೇದಶಬ್ದಗಳನ್ನು ಬಳಸುತ್ತಲೇ ಇದ್ದರೂ ಯಾವುದೇ ಒಂದು ನಿಲುವನ್ನೂ ಪಡೆಯದ ಕಾರಣ ಇಂತಹ ದೋಷವೇರ್ಪಟ್ಟಿದೆಯಾದ್ದರಿಂದ, ಇದೆಲ್ಲಾ ಸರಿಯಾಗಿ ಮನದಟ್ಟಾಗಬೇಕಾದರೆ, ಈ ಬಗ್ಗೆ ಮೊದಲಿಗೆ ಒಂದು ನಿಲುವಿಗೆ ಬರಬೇಕು. ಗುರುಕುಲವನ್ನು ಏರ್ಪಡಿಸಿಕೊಂಡು ಅಲ್ಲಿ ಅಂತೇವಾಸಿಯಾಗಿ ಅಲ್ಲಿಂದ ಸರಿಯಾದ ಉಪದೇಶ, ಸಾಧನೆ, ಅನುಭವ ಇದನ್ನೆಲ್ಲಾ ಪಡೆದುಕೊಳ್ಳಳೇ ಬೇಕು. ಇವುಗಳಿಲ್ಲದೇ ಏನೇ ಲಾಗ ಹೊಡೆದರೂ ಅದರ ಸುಳಿವೂ ಮನಸ್ಸಿಗೆ ಬರುವುದಿಲ್ಲಾಪ್ಪಾ!

\section*{ಪುಸ್ತಕಕ್ಕಾಗಲೀ ಉಪನ್ಯಾಸಕ್ಕಾಗಲೀ ಬೆಲೆ ಯಾವಾಗ}

ಈ ರೀತಿಯ ವಿಷಯದ ನೈಜವಾದ ಹಿರಿಮೆಯನ್ನು ನಾವು ಮನದಟ್ಟು ಮಾಡಿಕೊಳ್ಳಬೇಕಾದರೆ, ನಾವು ಆಶ್ರಯಿಸಬೇಕಾದ ಜಾಗವನ್ನು ಪುಸ್ತಕವು ಹೊರಗಡೆ ಚಿತ್ರಿಸಿ ಮನಕ್ಕೆ ತರುವಂತಿದ್ದರೆ, ಆವಾಗತಾನೇಪ್ಪ ಪುಸ್ತಕಕ್ಕೆ ಒಂದು ಬೆಲೆ. ಆ ಚಿತ್ರಣವನ್ನು ಪುಸ್ತಕವು ಮಾಡದಿರುವುದಾದರೆ ಅದಕ್ಕೇನು ಬೆಲೆ? ಕಾಗದ ಪ್ರಿಂಟ್‌ಗಳದಾದ ಖರ್ಚೇ ಅದರ ಬೆಲೆಯೇ ಹೊರತು, ಜೀವನದಲ್ಲಿ ಸ್ವಾಭಾವಿಕವಾಗಿ ಬರುವ ಹಸಿವೆಯನ್ನು ತಣಿಸಿ ತೃಪ್ತಿತರುವಂಥಹ ಬೆಲೆ-ಆಕೃತಕವಾದ ಬೆಲೆ ಅದಕ್ಕೆ ಬರುವುದಿಲ್ಲಪ್ಪ!. ಪುಸ್ತಕಕ್ಕೆ ಹೇಗೋ ಉಪನ್ಯಾಸಗಳಿಗೂ ಹಾಗೆಯೇ. ಉಪನ್ಯಾಸ ಪ್ರಯೋಜನದ ವೈವಿಧ್ಯವನ್ನ್ನು ಗಮನಿಸುವುದಾದರೆ, ಮಹಾ ಮಹಾ ಉಪನ್ಯಾಸಗಳೇ ಜರುಗಿದರೂ, ಜನರು ಕಿಕ್ಕಿರಿದು ತುಂಬಿ ಕೇಳಿದರೂ ಎಷ್ಟೋ ಜನರಿಗೆ ಉಪನ್ಯಾಸದಲ್ಲಿ ಏನು ಹೇಳಿದರು? ಎಂಬುದು ಮನಸ್ಸಿಗೆ ಅರ್ಥವೇ ಆಗುವುದಿಲ್ಲ. ಕೆಲವರಿಗೆ ಕೆಲವಂಶ ಮಧ್ಯೇ ಮಧ್ಯೇ ಅರ್ಧಮರ್ಧವಾಗಿ ಅರ್ಥವಾಗುತ್ತೆ. ಉಪನ್ಯಾಸಗಳು ಹೀಗಾಗುವುದಕ್ಕೆ ಯೋಗ್ಯವಾಗೇ ಇರುತ್ತೆ. ಇಂಥೆ ಉಪನ್ಯಾಸಗಳಿಗೆ ಹೇಗೆ ಬೆಲೆಕಟ್ಟುವುದು? ಉಪನ್ಯಾಸ ಕೇಳಿದವನ ಮೇಲೆ ಯಾವುದೇ ಚಿತ್ರಣವನ್ನೂ ಮಾಡದೇ, ಪ್ರೋಗ್ರಾಂಗಳ ಪಟ್ಟಿಗಷ್ಟೇ ಸೇರಿದ್ದಾದರೆ ಲಾಭವಾದರೂ ಏನು? `ಕಾಲಕ್ಷೇಪ' ಎಂಬ ಹೆಸರೇ ಹೇಳುತ್ತದೆ, ಕಾಲತಳ್ಳುವುದು ಎಂಬ ಅರ್ಥವನ್ನು.

\section*{ಶೋಧನಾಬುದ್ಧಿಯಿಂದ ಮಾತ್ರ ಹೃದಯದ ಪರಿಚಯವಾಗಲು ಸಾಧ್ಯ}

ಯಾವ ಬಗೆಯ ಮಾತು ಕತೆಯೇ ಆಗಲೀ, ಕೈ ಸನ್ನೆಯೇ ಆಗಲೀ, ಲೇಖನಗಳೇ ಆಗಲೀ, ಎಲ್ಲದರಲ್ಲೂ, ಇವೆಲ್ಲಕ್ಕೂ ಮೂಲದಲ್ಲಿರುವವನು ಯಾರೋ ಅವನನ್ನು ಮುಟ್ಟಿ ಹೃದಯದಿಂದ ಬಂದ ವಿಷಯವೆಷ್ಟು? ಅಥವಾ ಅವನ ಸ್ವಭಾವವಾದ ಆಶಯದಿಂದ ಬಂದ ವಿಷಯವೆಷ್ಟು? ಪ್ರಕೃತಿಗೆ ಸಂಬಂಧಪಟ್ಟಾ ವಿಷಯವೆಷ್ಟು? ಎಂಬುದನ್ನು ವಿಭಜನೆ ಮಾಡಿಕೊಂಡು ವಿಷಯತೆಗೆದುಕೊಳ್ಳಬೇಕು. ಏಕೆಂದರೆ ಒಬ್ಬನಿಗೆ ಕೈಯಲ್ಲಿ ಬಾತಚೇಷ್ಟೆಯು ಆಡುತ್ತಿರುತ್ತದೆ. ಅವನು ಆ ಸಮಯದಲ್ಲಿ ಕೈಯಾಡಿಸಿದ್ದನ್ನು  ನೋಡಿದರೆ, ನಿಮಿಷನಿಮಿಷಕ್ಕೂ ಯಾರನ್ನೋ `ಹೋಗು! ಹೋಗು!' ಅನ್ನುತ್ತಿದ್ದಾನೇನೊ ಎನ್ನಿಸುತ್ತಿರುತ್ತೆ. ಸತತವಾಗಿ ಆ ಬಗೆಯ ಕೈ ಒದರಾಟ, ಅಂತಹವನು ಅಕಸ್ಮಾತ್ತಾಗಿ ಸ್ನೇಹಿತನೊಬ್ಬನನ್ನು `ಬಾ' ಎಂದು ಬಾಯಿಂದಲೂ ಕೈಸನ್ನೆಯಿಂದಲೂ ಕೂಗಿ, ಕೂಡಲೇ ವಾತಚೇಷ್ಟೆಯ ಕೈ ಒದರಾಟವನ್ನು ಮಾಡಿಬಿಟ್ಟರೆ, ಆ ಸ್ನೇಹಿತನು ಆ ಸಮಯದಲ್ಲೇನು ಮಾಡಬೇಕು? `ಬಾ' ಎಂಬ ಶಬ್ದ ಮತ್ತು ಮೊದಲ ಕೈ ಸನ್ನೆಗಳನ್ನು ನೋಡಿ ಒಳಕ್ಕೆ ಬರಲು ಶುರುಮಾಡಿದನೋ ಪಕ್ಕದಲ್ಲೇ ಬಂದ `ಹೋಗು' ಎಂಬ ಚೇಷ್ಟೆ  ನೋಡಿದಾಗ ವಾಪಸ್ ಹೊರಡಲು ಸಿದ್ಧವಾಗಬೇಕಾಗುತ್ತೆ. ಇಲ್ಲಿ ವಿಷಯವನ್ನು ವಿಂಗಡಿಸಿಕೊಳ್ಳುವುದಾದರೆ, `ಬಾ' ಎಂಬ ಕೈಸನ್ನೆ ಇವೆರಡೂ ಅವನ ಹೃದಯ-ಆಶಯಕ್ಕೆ ಸಂಬಶಿಸಿದ್ದು, `ಹೋಗು' ಎಂದು ಬಂದದ್ದು ಕೇವಲ ಪ್ರಕೃತಿಗೆ ಸಂಬಂಧಿಸಿದ ವಾತಚೇಷ್ಟೆ ಎಂದು ನಿಶಯಿಸಿಕೊಳ್ಳಬೇಕಾಗುತ್ತೆ. ಇಲ್ಲಿ ಮೊದಲನೆಯದಕ್ಕೆ ಬೆಲೆ ಕೊಟ್ಟು, ಅವನು ನಮ್ಮ ಬರುವಿಕೆಯನ್ನು ಬಯಸಿದ್ದಾನೆಂದು ತಿಳಿದು ಒಳಕ್ಕೆ  ಬರಬೇಕಾಗುತ್ತೆ. ಕೇವಲ ಪ್ರಕೃತಿಯ ಕಾರ್ಯವನ್ನು ನಿರ್ಲಕ್ಷಿಸಬೇಕಾಗುತ್ತೆ. `ಹೋಗು' ಎಂಬುದನ್ನು ತೋರಿಸಲು ಮಾಡಬೇಕಾದ ಕೈಸನ್ನೆಯೂ, ವಾತಚೆಷ್ಟೆಯೂ ಏಕಾಕಾರವಾಗಿಯೇ ಇರಬೇಕಾಗಿ ಬಂದದ್ದರಿಂದ ಈ ಮುದ್ರೆಯು ಸಾಂಕರ್ಯಕ್ಕೆ ಕಾರಣವಾಗಿ ಭ್ರಮಜನಕವಾಗ್ಯಿತು. ಹೀಗೆಲ್ಲಾ ಶೋಧನೆ ಮಾಡಿಕೊಂಡಾಗ ತಾನೇ ಅಲ್ಲಿಯ ನಿಜ ಹೊರಬರಲು ಸಾಧ್ಯವಾಗುವುದು. ಈ ಬಗೆಯ ಶೋಧನಾ ಬುದ್ದಿಯನ್ನು ಬೆಳೆಸಿಕೊಂಡರೆ ಮಾತ್ರ ಹೃದಯ ಪರಿಚಯವಾಗಲು ಸಾಧ್ಯ. 

\section*{ನಿಜವು ಕೈಬಿಟ್ಟು ಹೋಗಲು ಹಲವು ಕಾರಣಗಳುಂಟು}

ಯಾವುದೇ ವಿಷಯದ ಯಥಾರ್ಥತೆಯನ್ನು ಗ್ರಹಿಸಲು ಕೇವಲ ಪುಸ್ತಕಪಂಕ್ತಿಗಳ ಆಧಾರದ ಮೇಲೆಯೇ ಹೊರಟರೆ ಹೃದಯವು ಕೈಕೊಟ್ಟು ಹೋಗಿ ವಿಧವಿಧವಾದ ಅರ್ಥಕಲ್ಪನೆಗೆ ಶುರುವಾಗುತ್ತೆ. ಇದಾರೋ ಒಬ್ಬ ವ್ಯಕ್ತಿಯದಷ್ಟೇ ದೋಷವಲ್ಲ. ಅದಕ್ಕೆ ಬೇರೆ ಬೇರೆ ಕಾರಣಗಳೂ ಇರುತ್ತವೆ. ಒಂದು ಕಾರಣ-ಮುಖ್ಯವಾಗಿ ನೋಡುವವನಿಗೆ ಹೃದಯಬೇಕು. ಸಹೃದಯತೆ ಇಲ್ಲದಿರುವುದರಿಂದಲೂ ನಿಜ ಕೈಬಿಟ್ಟು ಹೋಗುತ್ತೆ. ಮತ್ತೊಂದು ಕಾರಣವೆಂದರೆ-ಈ ಸಂಸ್ಕೃತ ಭಾಷೆಯ ಸ್ವಾಭಾವ. ಸಂಸ್ಕೃತ ಭಾಷೆಯನ್ನು ಹೇಗೆ ಬೇಕಾದರೂ ತಿರುಗಿಸಬಹುದು ನೋಡಿ! ಉದಾಹರಣೆಗೆ ಒಬ್ಬನನ್ನು ಕರೆದು, `ಬಾ ಇಲ್ಲಿ, ನಿನಗೆ ಕಂಕಣ ತರಿಸಿಕೊಡುತ್ತೇನೆ' ಎಂದು ಹೇಳಿದಾಗ ಅವನು ಹತ್ತಿರ ಬರಲು ನಾಲ್ಕು ಬಾರಿಸುತ್ತಾನೆ. ಅದಕ್ಕವನು `ಏನಯ್ಯಾ ಕಂಕಣ ತರಿಸಿಕೊಡುತ್ತೇನೆ ಎಂದು ಹೊಡೆದು ಬಿಟ್ಟೆ?' ಎಂದರೆ `ಕಂಕಣ ತರಿಸಿಕೊಡುತ್ತೇನೆಂದು ಹೇಳಿದಂತೆಯೇ ಮಾಡಿದ್ದೇನೆ, ಕಣ್ಮುಟ್ಟಿನೋಡಿಕೋ?' ಎಂದರೆ `ಕಣ್ಣಿಗೂ ಕಂಕಣಕ್ಕೂ ಏನು ಸಂಬಂಧವಯ್ಯಾ?, ಎಂದಾಗ `ಕಣ್+ಕಣ=ಕಂಕಣ ಅಲ್ಲವೇ? ಕಣ ಎಂದರೆ ನೀರಿನ ಹನಿ. ಕಂಕಣ ಎಂದರೆ ಕಣ್‌ಕಣ-ಕಣ್ಣೀರು' ಎನ್ನುತ್ತಾನೆ. ಹೀಗೆಲ್ಲಾ ಶುರುವಾಗಿಬಿಟ್ಟರೆ, ಇಂಥದನ್ನೇ ಪ್ರಯೋಗಕ್ಕೆ ತಂದುಬಿಟ್ಟರೆ ಬಹಳ ಕಷ್ಟದಲ್ಲಿ ಪರ್ಯವಸಾನವಾಗುವುದು.

ಹೀಗೆಯೇ ಹಂಡೆಯಲ್ಲಿ ನೀರು ಕುದಿಯುತ್ತಿದ್ದಾಗ ಒಬ್ಬನನ್ನು ಕರೆದು `ಬಾಯ್ಯ ಸಾನಕ್ಕೆ ನೀರು ಹಾಕುತ್ತೇನೆ' ಎನ್ನ್ನುತ್ತಾನೆ ಮತ್ತೊಬ್ಬ. ನೀರಿನಲ್ಲಿ ಬರುವ ಹೊಗೆಯನ್ನು ನೋಡಿ `ಬಹಳ ಬಿದಿ ಎಂದು ಕಾಣುತ್ತೇಪ್ಪ! ಹೇಗೆ ಸ್ನಾನಕ್ಕೆ ಬಿಡುತ್ತೀಯೇ?' ಎಂದು ಅವನು ಕೇಳಿದರೆ `ಇಲ್ಲಯ್ಯ ನೀರು ತುಂಬ ಪದವಾಗಿದೆ. ಅದನ್ನು ತೋರಿಸುತ್ತೇನೆ ಬಾ!' ಎಂದು ಕುಳ್ಳಿರಿಸಿ ಮೈಮೇಲೆ ಕುದಿಯುವ ನೀರು ಹಾಕುತ್ತಾನೆ. ಆವಾಗ ಇವನು `ಏನಯ್ಯಾ ಇಷ್ಟು ಬಿಸಿಯಾಗಿದೆ, ತುಂಬ ಪದವಾಗಿದೆಯೆನ್ನುತ್ತೀಯಲ್ಲಾ!' ಎಂದರೆ `ಹೌದಯ್ಯಾ! ತುಂಬ ಎನ್ನುವುದಕ್ಕೆ ಸಂಸ್ಕೃತದಲ್ಲಿ `ಪರಂ' ಎಂದಲ್ಲವಾ? ಎರಡನ್ನೂ ಸೇರಿಸಿಕೊಂಡರೆ `ಪರಂಪದ ತೋರಿಸುತ್ತೇನೆ' ಎಂದೆ. ಈ ಬಿಸಿ ನೀರಲ್ಲಿ ಸ್ನಾನ ಮಾಡಿದರೆ (ನೀನು ಬದುಕುವುದಿಲ್ಲವಾದ್ದರಿಂದ) ಪರಂಪದ ಸೇರಬೇಕಾಗುತ್ತೆ. ಅದನ್ನು ತೋರಿಸುತ್ತೇನೆ ಎಂದೆ. ನನ್ನ ಮಾತಲ್ಲಿ ತಪ್ಪೇನಿದೆ? ಎಂದರೆ- ಹೀಗಾದರೆ ಏನುಗತಿ? ಪದಗಳಿಗೆ ಕೋಶ, ಡಿಕ್ಷ್ನೆರಿಗಳನ್ನೆಲ್ಲಾ ಸೇರಿಸಿ ಮಾತಾಡುವ ಋಷಿಯಾಯಿತೇ ಹೊರತು, ಅನುಭವಿಸುವವರ ಗತಿಯೇನು?

\section*{ನಿರ್ದೇಶನವು ಹೇಗಿರಬೇಕು}

ಯಾವುದೇ ಒಂದು ವಿಷಯವನ್ನು ಪ್ರಸ್ತಾಪಿಸಿ `ಇದು ಹೇಗೆ? ಇದರ ಅರ್ಥವೇನು? ಎಂದು ಪ್ರಶ್ನಿಸಿದಾಗ `ಬ್ರಹ್ಮಸೂತ್ರದಲ್ಲಿ ಹೀಗೆ ಹೇಳುತ್ತೆ' ಎಂದು ಉತ್ತರ ಕೊಡುವುದಾದರೆ ಉಪಯೋಗವಾಗಲಾರದು. ಏಕೆಂದರೆ ಆ ಸೂತ್ರಕ್ಕೆ ಮೂವತ್ತೆಂಟು ಅರ್ಥವಾಗಬಹುದು. ಅದರಲ್ಲಿ ಯಾವುದು ಸರಿಯೆಂದು ನಂಬುವುದು? ಜೊತೆಗೆ ಅಲ್ಲಿ ಬೇರೆ ಬೇರೆ ಪಾಠಾಂತರಗಳು `ಕ' ಪ್ರತಿ `ಖ' ಪ್ರತಿ `ಗ' ಪ್ರತಿ ಹೀಗೆಲ್ಲಾ ಇದೆ. ಎಲ್ಲಾ ಪ್ರತಿಗಳೂ ಒಂದಕ್ಕೊಂದು ವಿರುದ್ಧವಾಗಿದ್ದು `ನನ್ನಕಡೆ ನೋಡು ನನ್ನಕಡೆ ನೋಡು' ಎಂದು ಕರೆಯುತ್ತವೆ. ಯಾವ ಪಾಠವನ್ನು ನಂಬಿ, ಏನನ್ನು ಹಿಡಿದುಕೊಂಡು ಮುಂದುವರಿದು ವಿಷಯ ಮುಟ್ಟಬೇಕು? `ಮೈಸೂರಿಗೆ ಹೋಗಲು ದಾರಿ ಯಾವುದು'? ಎಂದು ಯಾರಾದರೂ ಪ್ರಶ್ನಿಸಿದರೆ, ಒಂದು ದಾರಿಯನ್ನು ನಿಶ್ಚಿತವಾಗಿ ತೋರಿಸಬೇಕೇ ಹೊರತು, ಹೀಗೂ ಹೋಗಬಹುದು ಹಾಗೂ ಹೋಗಬಹುದು ಎನ್ನುವಂತೆ ಹತ್ತಾರು ಕೈ ಮರಗಳಿದ್ದುಬಿಟ್ಟರೆ, ಅಲ್ಲಿ ಆ ಕೈಮರಗಳಿಂದಾದರೂ ಏನು ಉಪಯೋಗವಾದೀತು? ಅದರ ಬದಲು ಯಾವ ಕೈಮರವೂ ಇಲ್ಲದಿರುವುದೇ ವಾಸಿಯಲ್ಲವೇ? ಹೇಗೆ ಬೇಕಾದರೂ ಹೋಗಬಹುದು ಎನ್ನುವುದಕ್ಕೆ ನಿರ್ದೇಶನವೇಕೆ? ನಿರ್ದಿಷ್ಟವಾಗಿ ತಿಳಿಸುವುದಕ್ಕಾಗಿ ತಾನೇ ನಿರ್ದೇಶನ ಬೇಕಾಗುವುದು. ಅನಿರ್ದಿಷ್ಟಕ್ಕೂ ಒಂದು ಗೈಡೆನ್‌ಸ್ ಬೇಕೆ? ಸ್ವಚ್ಛಂದವಿಲ್ಲ. ಒಂದು ಜಾಗದಿಂದ ಹೊರಟು ಮುಂದೆ ಸಾಗುತ್ತಲೇ ಇದ್ದರೆ, ಎಂದಾದರೊಂದು ದಿನ ಹೊರಟ ಜಾಗಕ್ಕೆ ಬಂದು ತಲಪಿಯೇ ತಲಪುತ್ತಾನೆಂಬ ಸಿದ್ಧಾಂತವನ್ನಿಟ್ಟುಕೊಂಡು ಹೊರಟರೆ ಸಾಕು. ಇಂತಹ ಪ್ರಯಾಣಕ್ಕೆ ನಿರ್ದೇಶಕರಾರೂ ಬೇಕಿಲ್ಲ. ಇಲ್ಲಿ ಪ್ರಶ್ನೆ ಉತ್ತರಗಳಾವುದೂ ಬೇಕಿಲ್ಲ. ಆದ್ದರಿಂದ ಹೊರಡುವವನು ಹೇಗೆ ಹೊರಡಬೇಕು? ಎಂದರೆ, ಒಳಹೊಕ್ಕು ನೋಡಿ ಸತ್ಯವಿರುವ ಜಾಗವನ್ನು ಕಂಡು ಹಿಡಿಯಲು ಹೊರಡಬೇಕಾಗಿದೆ. ಹಾಗೆ ಹೊರಡುವಾಗ ನಿರ್ದೇಶಕನ ಮುಖಮುದ್ರೆ, ಮಾತಿನ ಶೈಲಿ, ಪೂರ್ವಾಪರ ಸಂದರ್ಭಗಳು ಎಲ್ಲವನ್ನೂ ಸಾವಧಾನವಾಗಿ ಗಮನಿಸಿ ಮನಸ್ಸಿಗೆ ತಂದುಕೊಂಡು, ನಂತರ ಹೊರಡಲು ಯತ್ನಿಸಬೇಕು.

\section*{ಸಾಹಿತ್ಯ, ಆಕಾರ, ಚಿತ್ರ, ಸನ್ನೆ ಇವೆಲ್ಲಾ ಯಾವಾಗ ಉಪಯೋಗವಾಗುವುದೆಂಬ ಬಗ್ಗೆ ಒಂದು ಪ್ರಸಂಗ}

ಒಂದು ಘಟನೆಯು ಈ ಸನ್ನಿವೇಶದಲ್ಲಿ ಜ್ಞಾಪಕಕ್ಕೆ ಬರುತ್ತೆ. ಹಿಂದೊಮ್ಮೆ ನಾನು ಮತ್ತು ಕುಟ್ಯಣ್ಣ ಇಬ್ಬರೂ ಪೂನಾ ಕಡೆಗೆ ಪ್ರವಾಸ ಹೋಗಿದ್ದೆವು. ಅಲ್ಲಿ ಯಾವುದೇ ಆಹಾರ ಸೌಕರ್ಯವಿರಲಿಲ್ಲ. ಆದ್ದರಿಂದ ಹಸಿವು ತೀರಿಸಿಕೊಳ್ಳಲು ಕಡ್ಲೇಕಾಯಿ ತೆಗೆದುಕೊಂಡು ಬರೋಣವೆಂದು ಅಂಗಡಿಗೆ ಹೊರಟೇವು. ಯಾವುದೊ ಒಂದು ಅಂಗಡಿಯ ಮುಂದೆ ವಿಚಾರಿಸಲು ನಿಂತೆವು. ನಮ್ಮಿಬ್ಬರಿಗೂ ಮರಾಠಿ ಭಾಷೆ ಗೊತ್ತಿರಲಿಲ್ಲ. ಪೂನಾ ಮಹಾರಾಷ್ಟ್ರದ್ದು. ಅವರಿಗೋ ಕನ್ನಡ ಗೊತ್ತಿಲ್ಲ. ಹೀಗಿರುವಾಗ ಕಡ್ಲೇಕಾಯನ್ನು ವಿಚಾರಿಸುವುದು ಹೇಗೆ? ಏನೂ ತೋರದೇ ಯೋಚಿಸುತ್ತಿದ್ದೆವು. `ನೆಲಗಡಲೇಕಾಯಿ' ಎಂಬುದನ್ನು ಕುಟ್ಯಣ್ಣ ಅವರು `ಭೂಚಣಕ' ಎಂದು ಸಂಸ್ಕೃತಕ್ಕೆ  ಭಾಷಾಂತರಿಸಿದರು. ನಂತರ ಆ ಪದವನ್ನೇ ಬಳಸಿ ಅಂಗಡಿಯಲ್ಲಿ ಕೇಳಿದೆವು. ನಾವು ಯಾವುದನ್ನು ಹೀಗೆ ಕೇಳುತ್ತಿದ್ದೇವೆಂದು ಅಂಗಡಿಯವರಿಗೆ ಅರ್ಥವಾಗಲಿಲ್ಲ. ಅಂಗಡಿಯಲ್ಲಿದ್ದ ಯಾವಯಾವುದೋ ಪದಾರ್ಥವನ್ನು ತೋರಿಸುತ್ತಾ `ಇದೇ? ಇದೇ? ಎಂದು ಕೇಳಿದರು. ಅದಾವುದೂ ಆಕೃತಿಯನ್ನು ವಿವರಿಸಿದ್ದಾಯಿತು. ಇಷ್ಟು ಉದ್ದವಿದೆ, ಎರಡೆರಡು ಬೀಜಗಳಿವೆ. ಕೆಲವಲ್ಲಿ ಮೂರು ಬೀಜವೂ ಉಂಟು ಎಂದೆಲ್ಲಾ ಸನ್ನೆ ಮತ್ತು ಮಾತಿನ ಮೂಲಕ ಅರ್ಧಂಬರ್ಧ ಹೇಳಿದೆವು. ನಾವು ತೋರಿಸಿದ್ದು ಅವರಿಗೆ ಯಾವುದೋ ಅಸಭ್ಯವಾಗಿ ಕಂಡಿತೆಂದು ತೋರುತ್ತೆ. ಏನು ಮಾಡಿದರು? ಎಲ್ಲರೂ ನಕ್ಕರು. ನಾವು ಹೇಳಿದ್ದು ನಮಗೆ ಬೇಕಾದ ಪದಾರ್ಥಕ್ಕೆ ಹೊಂದಿಕೊಂಡಿದ್ದರೂ ಅವರಿಗೆ ಅದು ಮನಸ್ಸಿಗೆ ಬಾರದೆ ಅಸಭ್ಯ ಚಿತ್ರವಾಗಿ ನಗುವಿಗೆ ಕಾರಣವಾಯಿತು. ಇದೇನು ಗ್ರಹಚಾರ? ಏನೇನು ಮಾಡಿದರೂ ಇವರಿಗರ್ಥವಾಗುವುದಿಲ್ಲವಲ್ಲಾ! ಎಂದು ಕೊಂಡು ನೆಲಗಡಲೆಕಾಯಿನ ಚಿತ್ರಬರೆದು ತೋರಿಸಿದೆವು. ಆದರೂ ಅವರಿಗೆ ಅರ್ಥವಾಗಲಿಲ್ಲ. ಏನು ಮಾಡುವುದು? ಎಂದು ಯೋಚಿಸುತ್ತಾ ಅಕಸ್ಮಾತ್ ಕೆಳಗೆ ನೋಡಿದಾಗ ಅಲ್ಲೊಂದು ಸಿಪ್ಪೆ ಬಿದ್ದಿತ್ತು. ಅದನ್ನು ತೋರಿಸಿ, ಈ ಕಾಯಿ ಎಂದೆವು. ಆವಾಗ ಅವರಿಗರ್ಥವಾಗಿ ಕೊಟ್ಟರು. ಹೇಗೋ ಕಡ್ಲೇಕಾಯಿ ಸಿಕ್ಕಿಬಿಟ್ಟಿತು. ಇಷ್ಟೆಲ್ಲಾ ಕಥೆಯನ್ನೇಕೆ ಹೇಳಿದ್ದು? ಎಂದರೆ ಸಂದರ್ಭವೊಂದು ಮನಸ್ಸಿಗೆ ಬಾರದಿದ್ದಗ, `ಸಾಹಿತ್ಯ-ಆಕಾರ-ಚಿತ್ರ-ಸನ್ನೆ' ಇದಾವುದೂ ಉಪಯೋಗವಾಗದೇ ಹೋಗುತ್ತೆ ಎಂಬುದಕ್ಕಾಗಿ.

\section*{ಪುಸ್ತಕದಿಂದ ವಿಷಯವು ಪೂರ್ಣವಾಗಿ ಅಭಿವ್ಯಕ್ತವಾಗಬಹುದೇ?}

ಯಾವುದೇ ವಿಷಯ ಹೇಳುವವನೂ ಕೇಳುವವನೂ ಎದುರೆದುರಿದ್ದಾಗ ಹೇಗೆ ಒಬ್ಬರ ಮಾತು ಮತ್ತೊಬ್ಬರಿಗೆ ಅರ್ಥವಾಗಬಹುದೋ, ಹಾಗೆಯೇ ಕಾಗದ ಪತ್ರಗಳ ಮುಖೇನ ಬಂದಾಗ ಅದೇ ವಿಷಯವು ಅರ್ಥವಾಗುತ್ತೆಯೆಂಬ ನಿಶ್ಚಯವಿಲ್ಲ. ಅಂತೆಯೇ ಒಂದು ವಸುತ್ಕ್ಷ್ವನ್ನು ಎದುರಿಗೇ ಕಣ್ಣಿಂದ ನೋಡುವುದಕ್ಕೂ, ಅದೇ ವಸ್ತುವಿನ ವಿಷಯವಾಗಿ ಪುಸ್ತಕಗಳ ಮೂಲಕ ತಿಳಿಯುವುದಕ್ಕೂ ಮಹದಂತರವಿದೆ. ಏಕೆಂದರೆ
ಪುಸ್ತಕದಲ್ಲಿ ನೋಡಿದಾಗ ಕಾಣುವುದು ಪಂಕ್ತಿ, ಅಕ್ಷರಕ್ಕೆ ಸೂಚಕವಾದ ರೇಖೆ (ಲಿಪಿ) ಇಷ್ಟು ಮಾತ್ರವೇ ಕಾಣುವುದು. ಅದೇ ಕಲ್ಲಿನ ಶಿಲ್ಪದ ಮೂಲಕ  ನೋಡಿದಾಗ, ವಸ್ತುವಿನ ಆಕಾರ ಮಾತ್ರ ಕಾಣುವುದು. ಇದಾವುದರಲ್ಲೂ ಸ್ಪರ್ಶ-ಗಂಧ-ರಸ-ನಗು- ವಿಕಾಸ ಇದಾವುದೂ ಇರುವುದಿಲ್ಲ. ಒಂದು ವಸ್ತುವಿನಲ್ಲಿ ಸಹಜವಾಗಿರಬೇಕಾದ್ದೆಲ್ಲಾ ಇರದೇ ಕೆಲವು ಮಾತ್ರವೇ ಇದ್ದಾಗ ಹೇಗೆ ಅಪೂರ್ಣವೋ-ಊನವೋ, ಹಾಗೆಯೇ ಪುಸ್ತಕಾದಿಗಳಲ್ಲಿ ಊನವೇ ಇರುವುದು. ಆದ್ದರಿಂದ ಪುಸ್ತಕಾದಿಗಳ ಮೂಲಕ ಗ್ರಹಿಸಿದ್ದು, ಊನವಿರುವಂತಹ ಅಲ್ಲಿಯೇ ನಿಲ್ಲದೇ, ಪೂರ್ಣಮಾಡಿಕೊಳ್ಳಬಲ್ಲ ಹೃದಯದೆಡೆಗೆ ಸಾಗಿ ಹೃದಿಸ್ಥವಾದಾಗ ಮತ್ತು ಆತ್ಮಸ್ಥವಾದಾಗ ಪೂರ್ಣತೆಯನ್ನು ಪಡೆಯುತ್ತದೆ. ನಿಜ-ಬಯಸುವನ ಮನಸ್ಸು ಪುಸ್ತಕಸ್ಥವಾಗಿಯೇ ನಿಂತುಬಿಡಬಾರದು.

\section*{ಕೇವಲ ಪುಸ್ತಕಾವಲಂಬಿಗಳಾದಾಗ ವಂಚನೆಗೆ ಎಡೆಯಿದೆ}

ಒರಿಜಿನಲ್ಲಾದ ವಿಷಯವನ್ನು ಬಿಟ್ಟು ಪುಸ್ತಕಸ್ಥವಾಗಿ ನಿಂತವರಿಗೆ, ಮೋಸಗಾರರು ಬಹಳ ಮೋಸಮಾಡುತ್ತಾರೆ. ಕೆಲವರು ಒಂದು ವಿಷಯವನ್ನು ತೋಚಿದಂತೆ ಹೇಳಿಬಿಡುವುದು. ಎದುರಿಗಿದ್ದವರು `ನೀವು ಹೇಳಿದ್ದಕ್ಕೇನಾದರೂ ದಾಖಲೆ (ಆಧಾರ) ಉಂಟೇ?' ಎಂದು ಕೇಳಿದರೆ, ಯಾವುದೋ ಪುಸ್ತಕವೊಂದನ್ನು ಹೆಸರಿಸಿಬಿಟ್ಟು `ಇದರ ೫೦ನೇ ಪತ್ರದಲ್ಲಿದೆ' ಎನ್ನುವುದು. ಅವರು ಹೇಳಿದ ಪುಸ್ತಕದಲ್ಲಿರುವ ಪತ್ರ ಸಂಖ್ಯೆಯ ಮೊತ್ತ ೨೦ ಮಾತ್ರ. `ಇದೇನು? ನೀವು ಹೇಳಿದ ಪುಸ್ತಕದಲ್ಲಿ ೨೦ ಪತ್ರ ಮಾತ್ರವಿದೆ. ನೀವು ೫೦ನೇ ಪತ್ರ ಎಂದರಲ್ಲಾ!' ಎಂದರೇ ಆ ಎಡೀಷನ್ ಬೇರೆ. ಅದು ಲೈಬ್ರರಿಯಲ್ಲಿತ್ತು, ಎನ್ನುವುದು. ಆ ಲೈಬ್ರರಿಗೆ ಹೋಗಿನೋಡಿ `ಅಲ್ಲಿಲ್ಲವಲ್ಲಾ ಸ್ವಾಮಿ ನೀವು ಹೇಳಿದ ಪುಸ್ತಕ ಅಂದರೆ `ಹಾಗಾದರೆ ಯಾರೋ ಕದ್ದು ಬಿಟ್ಟಿದ್ದಾರೆ' ಎನ್ನುವುದು. ಹೀಗೆಲ್ಲಾ ವಂಚನೆಮಾಡುತ್ತಾರೆ. ಬೀಸೋದೊಣ್ಣೆ ತಪ್ಪಿದರೆ' ಎನ್ನುತ್ತಾರಲ್ಲಾ ಹಾಗೆ. ಆದ್ದರಿಂದ ವಿಷಯದ ಮೇಲೆ ನಿಂತು ವ್ಯವಹಾರ ಮಾಡುವುದನ್ನು ರೂಢಿಸಿಕೊಳ್ಳಬೇಕು. `ಪುಸ್ತಕಸ್ಥಾತು ಯಾ ವಿದ್ಯಾ\label{114} ತಯಾ ಮೂಢಃ ಪ್ರತಾರ್ಯತೇ' ಎನ್ನುತ್ತಾರಲ್ಲಾ!

\section*{ವಿಷಯವು ಮನಸ್ಸಿಗೆ ಬಂದ ಮೇಲೆ ಪುಸ್ತಕದ ಉಪಯೋಗ ಹೇಗೆ?}

ವಿಷಯವೇ ಮನಸ್ಸಿಗೆ ಬಂದ ಮೇಲೆ ಪುಸ್ತಕವೇತಕ್ಕಾಗಿ? ಎಂದರೆ, ಹೌದು. ನಿಜವಾಗಿ ಮನಸ್ಸಿಗೆ ವಿಷಯ ಬಂದಿದ್ದರೂ ಲೋಕದಲ್ಲಿ ಜನರಿಗೆ ಒಂದು ಗೈಡೆನ್ಸ್  ಕೊಡುವುದಕ್ಕಾಗಿ ಪುಸ್ತಕವನ್ನು ಬಳಸಿಕೊಳ್ಳಬೇಕು. ಮತ್ತೇನೆಂದರೆ ಲೋಕದಲ್ಲಿ ಪುಸ್ತಕಸಿಗುವಷ್ಟು ಸುಲಭವಾಗಿ ನಿಜವು ಸಿಗುವುದಿಲ್ಲವಲ್ಲಾ! ಸುಲಭವಾಗಿ ಸಿಗುತ್ತೆಯೆಲ್ಲಾ! ಎಂದು ಪುಸ್ತಕ ನಂಬಿದರೆ, ಮಿಸ್‌ಲೀಡ್ ಮಾಡುತ್ತದೆ. ಆದ್ದರಿಂದ ಆಯಾಪುಸ್ತಕಗಳನ್ನೂ ನೋಡಿ, ಎಲ್ಲಿ ನಿಜಕ್ಕೆ ಹೊಂದಿ ಕೊಂಡಿದೆ? ಎಲ್ಲೆಲ್ಲಿ ಮಿಸ್‌ಲೀಡ್ ಮಾಡುತ್ತಿದೆ? ಎಂಬುದನ್ನೆಲ್ಲಾ ಶೋಧನೆ ಮಾಡಿ ಅಲ್ಲಲ್ಲೇ ಸ್ವಯಂಬುದ್ದಿ ಬೆಳೆಯುವಂತೆಯೂ ನಿಜಕ್ಕೆ ಹೊಂದಿಕೊಳ್ಳದ್ದು ಯಾವುದು? ಎಂಬ ಬಗ್ಗೆಯೂ ಲೋಕಕ್ಕೆ ವಿವರಣೆಕೊಟ್ಟು ಜನ ಮೋಸ ಹೋಗದಂತೆ ಎಚ್ಚರಿಸಬೇಕಾಗಿದೇಪ್ಪ!

\section*{ಜನ ಸಾಮಾನ್ಯರ ಅರ್ಥಕಾಮಗಳಿಗೆ ವಿರೋಧವಾಗದಂತೆ ಲೋಕವನ್ನು ಧರ್ಮಮೋಕ್ಷಗಳ ಕಡೆಗೆ ಲೀಡ್ ಮಾಡುವ ಕರ್ತವ್ಯವಿದೆ}

ಇತ್ತೀಚೆಗಂತೂ ಬರಬರುತ್ತಾ ಜನರಿಗೆ ಗ್ರಂಥವನ್ನು- ಪುಸ್ತಕರಾಶಿಯನ್ನು ಬೆಳೆಸಿಕೊಂಡು ಹೋಗುವುದೇ ಹೆಚ್ಚಾಗಿ ರೂಢಿಯಾಗಿಬಿಟ್ಟಿದೆ. ಇಂದಿನ ಲೋಕಕ್ಕೆ ಇದೇ ಒಂದು  ವೃತ್ತಿಯಾಗಿ ಹೋಗಿದೆ. `ಎಲ್ಲಾ ಶ್ರುತಿಗಳೂ, ಸ್ಮೃತೀತಿಹಾಸಪುರಾಣಾದಿಗಳೂ ಕಂಠಸ್ಥವೇ ಆಗಿಬಿಟ್ಟಿವೆ' ಎಂದಾಗಿ ಬಿಟ್ಟರೆ, ಈಗಿನ ಲೈಬ್ರರಿಗಳೂ, ಮುದ್ರಣಾಲಯಗಳೂ, ಸೀಸದ ಕಾರ್ಖಾನೆ, ಪೇಪರ್ ಮಿಲ್, ಮತ್ತು ಸಂಶೋಧಕರ ತಂಡ ಇವೆಲ್ಲಕ್ಕೂ ಮುಕ್ತಾಯವಾಗಬೇಕಾಗಿ ಬರುತ್ತೆ. ಎಲ್ಲಾ ಸ್ಟಾಪ್. ಮುಕ್ಕಾಲುಪಾಲು ಜನರ ಜೀವನಕ್ಕೆ ಕಲ್ಲು ಬೀಳುತ್ತೆ. ಎಲ್ಲರಿಗೂ ಪ್ರಮಾರ್ಥ ತಾನೇ ಬೇಕಾದುದು! ಉಳಿದುದೆಲ್ಲ ವ್ಯರ್ಥ ಎಂದು ನಾವು ಹೊರಟರೆ, ಈಗ ನಡೆಯುತ್ತಿರುವ ಪ್ಲಾನ್ಗಳಿಗೆ, ಅವಾಂತರಗಳಿಗೆಲ್ಲಾ ಏನು ವಿಷಯವಿರುತ್ತೆ?. ಆದ್ದರಿಂದ ಸಡನ್ನಾಗಿ ಪುಸ್ತಕವನ್ನೆಲ್ಲಾ ಸ್ಟಾಪ್ ಮಾಡುವ ಹಂತೆಕ್ಕೆ ಹೊರಡಲಾಗುವುದಿಲ್ಲ. ಅವಸರಕ್ಕೆ ಹಾಗೆ ಮಾಡುವುದು ನೀತಿಯೂ ಅಲ್ಲ. ದೇಶಕಾಲಗಳನ್ನನುಸರಿಸಿ ನೀತಿಯರಿತು, ಜನ ಸಾಮಾನ್ಯರ ಅರ್ಥಕಾಮಾದಿಗಳಿಗೆ ವಿರೋಧವಾಗದಂತೇ, ಕೊಂಚ ಕೊಂಚವಾಗಿ ಹೆಜ್ಜೆಯಿಡಿಸುತ್ತಾ ಲೋಕವನ್ನು ಧರ್ಮ ಮೋಕ್ಷಗಳ ಕಡೆಗೆ ಲೀಡ್ ಮಾಡಬೇಕು. ಅಂತಹ ಗೈಡೆನ್ಸ್  ಎರ್ಪಡಿಸಿಕೊಡಬೇಕು. ಈ ವಿಚಾರ, ನನ್ನನ್ನೂ ಸೇರಿಸಿಕೊಂಡು ಭಗವತ್ಪ್ರಜೆಗಳಾದ ನಮ್ಮೆಲ್ಲರ ಕರ್ತವ್ಯದ ಜತೆಗೆ ಸೇರಿಕೊಳ್ಳುತ್ತೆಪ್ಪ! `ಭೂತದಯಾಂ ವಿಸ್ತಾರಯ'\label{115b} ಎಂದು ಅವನನ್ನು ಕೇಳಿಕೊಂಡು ಅವನ ಪ್ರೇರಣೆ ಅವನು ಕೊಡುವ ಬಲ ಶಕ್ತ್ಯುತ್ಸಾಹಗಳು ಎಲ್ಲವನ್ನೂ ಅವಲಂಬಿಸಿ ನಡೆಸಬೇಕಾದ ವಿಷಯವಾಗಿದೆಪ್ಪ! ಇದು.

\section*{ಶ್ರೀಗುರುವಿನ ಅನುಗ್ರಹಭಾಗಿಗಳಿಗೆ ಭವಭಯವಿಲ್ಲ}

ಈ ಮಾತನ್ನು ನಾನು ಹೇಳುತ್ತಿರುವುದು `ನಿವೆಲ್ಲರೂ ಇದಕ್ಕಾಗಿ ದುಡಿಯಲೇಬೇಕು, ಇಲ್ಲದಿದ್ದರೆ, ಭಗವನ್ನಿಗ್ರಹವಾಗುವುದು' ಎಂಬ ಅರ್ಥದಲ್ಲಿ ಅಲ್ಲಾಪ್ಪ! ನೀವುಗಳು ಭಗವಂತನ ಕೈಗೆ ಪೂರ್ಣವಾಗಿಸಿಕ್ಕಿದ್ದೀರಿ. ಯಾವ ಪುಸ್ತಕದ ಬೆಂಬಲವೂ ನಿಮಗೆ ಬೇಕಾಗಿಲ್ಲ. ಯಾವ ಭಾಷ್ಯ-ವ್ಯಾಖ್ಯಾನಗಲ ಅವಶ್ಯಕತೆಯೂ ನಿಮಗೆ ಇಲ್ಲ. ನೀವು ಅವನ ಕೃಪೆಯಿಂದ ಗೆದ್ದುಬಿಟ್ಟಿರಿ. `ನಿರ್ಭರೋ ನಿರ್ಭಯೋಽಸ್ಮಿ'\label{115} ಎಂದು ಇದ್ದು ಬಿಡಬಹುದಪ್ಪ! ನಿಮಗೆ ಯಾವ ಭವಭಯವೂ ಇಲ್ಲ. ನಿಮಗೆ ಭಗವತ್ಕಾರ್ಯವೆಂದು ಹೇಳಿದಂತೆ ಇದನ್ನು ನೀವು ಮಾಡಲಾಗದೇ ಹೋದರೂ ಯಾವ ಭಗವನ್ನಿಗ್ರಹವೂ ಇಲ್ಲಾಪ್ಪ!

\section*{ಹೃದ್ಯಾಗದ ನಂತರ ಸೌಲಭ್ಯಾನುಸಾರವಾಗಿ ಭಗವಂತನ ಬಾಹ್ಯವಾದ ಪೂಜೆ}

ಆದರೆ ನಿಮ್ಮ ಜೀವನ ವ್ಯವಹಾರದ ಕಾರ್ಯಗಳೂ ಮುಗಿದು, ಬಿಡುವು ಉತ್ಸಾಹ ಇತ್ಯಾದಿ ಎಲ್ಲವೂ ಕೂಡಿಬಂದರೆ ಹಗುರವಾಗಿ ಮನಸ್ಸು ಬಿಟ್ಟುಕೊಟ್ಟಾಗ ಭಗವಂತನ ಕೆಲಸವೆಂದು ಮಾಡಬಹುದಪ್ಪಾ! ಇಷ್ಟು ಮಾತ್ರವೇ ಇಲ್ಲಿ ನನ್ನ ಮಾತು.

ಸಾಂಪ್ರದಾಯಿಕವಾಗಿ ನೀವೂ ನಿತ್ಯವೂ ಹೇಳುವ ಮಾತೊಂದಿದೇಪ್ಪ! ಜ್ಞಾಪಿಸಿಕೊಳ್ಳಿ!-

\begin{shloka}
`ಭಗವನ್ ಪುಂಡಾರೀಕಾಕ್ಷ ಹೃದ್ಯಾಗಂ ತು ಮಯಾ ಕೃತಮ್ |\label{115a}\\
ಆತ್ಮಸಾತ್ಕುರು ದೇವೇಶ ಬಾಹ್ಯೇ ತ್ವಾಂ ಸಮ್ಯಗರ್ಚಯೇ ||
\end{shloka}

(ಅಪ್ಪಾ! ಭಗವಂತ! ಇದುವರೆಗೂ ಅಂತರಂಗದ ಪೂಜೆಯನ್ನು ಮಾಡಿದ್ದಾಯಿತು. ಇನ್ನು ನಿನ್ನ ಮುಂದಿನ ಮಕ್ಕಳಿಗೂ ಈ ಸವಿಯನ್ನು ಒಂದು ರೀತಿ ಉಣಿಸಲು, ಹೊರಗಡೆಯೂ ನಿನ್ನನ್ನು ಚೆನ್ನಾಗಿ ಅರ್ಚಿಸುವೆನಪ್ಪ!) ಎಂದು.

\section*{ಮುಂದಿನ ಪೀಳಿಗೆಗೆ ಸನ್ಮಾರ್ಗದರ್ಶನದ ಹೊಣೆ}

ಕ್ಷಣೇ ಕ್ಷಣೇ ವಿಕಾಸಗೊಳ್ಳುತ್ತಿರುವ ಈ ಮುಂದಿನ ಮಕ್ಕಳ ಭೌತಿಕ ದೃಷ್ಟಿಯಲ್ಲಿ ನಮ್ಮ ಆ ಅಂತರ್ದೃಷ್ಟಿಯನ್ನು ಸೇರಿಸಿ ಬೆಳೆಸುವುದಕ್ಕಾಗಿಯೇ ತಾನೇಪ್ಪ ಈ ಕೆಲಸ. ನೀವೇನೋ ಮಹಡಿಗೆ ಆತ್ಮಸೌಧಕ್ಕೆ ಹತ್ತಿಬಿಟ್ಟಿರಿಪ್ಪ! ಆದರೆ ನಿಮ್ಮ ಮುಂದಿನ ಮಕ್ಕಳು `ಅಪ್ಪಾ! ನಾನೂ ಅಲ್ಲಿಗೆ ಬರಬೇಕು, ನನ್ನನ್ನೂ ಕರೆದುಕೋ~' ಎಂದಾಗ, ಅಲ್ಲಿಗೊಂದು ನಿರಪಾಯವಾದ ಏಣಿಯನ್ನು ಹಾಕಿಕೊಡಬೇಕಲ್ಲವೇ? ಮಗುವು ತಾಯಿಯ ಹತ್ತಿರ ಹೋಗಬೇಕೆಂದು ಇಚ್ಛೆಪಟ್ಟಾಗ, ತಾಯಿಯಾದರೂ ಓಡಿಬಂದು ಮಗುವನ್ನಪ್ಪಿಕೊಳ್ಳಬೇಕು, ಅಥವಾ ಮಗುವಾದರೂ ತಾಯಿಯ ಹತ್ತಿರಕ್ಕೆ ನಡೆದುಹೋಗಿ ತಾಯಿಯನ್ನು ತಬ್ಬಿಕೊಳ್ಳಬೇಕು. ಈ ಎರಡೂ ಬಗೆಯ ಕೆಲಸವೂ ಆಗದಿದ್ದರೆ, ಆ ಮಗುವು ತಾಯಿಯನ್ನು ತಲಪುವುದಾದರೂ ಹೇಗೇಪ್ಪ? ಮಗುವಿಗೆ ನಡಿಗೆ ಚೆನ್ನಾಗಿ ಬಂದು ಬಿಟ್ಟಿದ್ದರೆ, ತಾಯಿಯೇನೂ ಓಡಿ ಬಂದು ಮಗುವನ್ನು ಎತ್ತಿಕೊಳ್ಳಬೇಕಾಗಿರುವುದಿಲ್ಲ. ಅದೇ ಮೆಲ್ಲಗೆ ಬರಬಹುದು. ಆದರೆ ನಡಿಗೆ ಬಂದಿದ್ದರಿಂದಲೇನೇ, ಅದು ತಾಯಿಯಲ್ಲಿಗೆ ಬರದೆ ಬೇರೆ ಕಡೆ ಹೋಗುವ ಸಂಭವವೂ ಇರುವುದರಿಂದ ಆ ಬಗ್ಗ್ಗೆ ಸರಿಯಾಗಿ ಗಮನವಿಟ್ಟು ತನ್ನಿಂದ ದೂರವಾಗದಂತೆ ನೋಡಿಕೊಂಡರೆ ಸಾಕು. ಅಥವಾ ಮನೆಯ ಹಿರಿಯಮಕ್ಕಳಾದರೂ ಆ ಕೆಲಸ ಮಾಡಬೇಕಾಗುವುದು, ಮಗು ಅಮ್ಮನಿಂದ ದೂರವಾಗದಂತೆ.

\section*{ಲೋಕದ ದೃಷ್ಟಿಯಿಂದ ಗುರ್ವನುಗ್ರಹಭಾಗಿಗಳಾದವರ ಕರ್ತವ್ಯ}

ಈಗಂತೂ ಎಲ್ಲರೂ ನಡಿಗೆ ಬಂದಿದೆಯೆಂದುಕೊಂಡು ತೋಚಿದಂತೆ ನಡೆಯಲು ಹೊರಟುಬಿಟ್ಟಿದ್ದಾರಪ್ಪ! ಅದನ್ನು ಯಾರೂ ಹೇಗೂ ನಿಲ್ಲಿಸಲಾಗುವುದಿಲ್ಲ. ಅವರು ನಡೆಯುತ್ತಿರುವ ನಡೆಗೇ ಒಂದು ಸನ್ಮಾರ್ಗವನ್ನು ತೋರಿಸಿ ಕೊಡಬೇಕಷ್ಟೆ. ನೀವುಗಳೂ `ಓದು' ಎಂಬುದಾಗಿ ಹತ್ತಾರು ವರ್ಷಶ್ರಮಪಟ್ಟಿದ್ದೀರಿ. ಅದರ ಜೊತೆಗೆ ನಿಮ್ಮ ಗುರು ಕೊಟ್ಟ ದೃಷ್ಟಿಯನ್ನೂ ಸೇರಿಸಿಕೊಂಡು, ಗುರುದೃಷ್ಟಿಗೆ ನಿಮ್ಮ ದೃಷ್ಟಿಯನ್ನು- ಪನಯನಮಾಡಿಕೊಂಡು, ಆ ದೃಷ್ಟಿಯ ಬೆಳಕಿನಲ್ಲಿ ನಮ್ಮ ಅಂತರಂಗದ ಪುರುಷನ ಪರಮಪುರುಷನ ಇಚ್ಛೆಯೆಂದು ಭಾವಿಸಿ ಉತ್ಸಾಹಗೊಂಡು ಅವನ ಕಾರ್ಯವನ್ನು ಯಾವಚ್ಛಕ್ಯ ಮಾಡೀಪ್ಪ!

ಮನೆಗೆ ಸೊಸೆ ಬರುತ್ತಾಳೆ ಮುಂದೆ ಸಂಸಾರ ಬೆಳೆಯುತ್ತೆ, ಮಕ್ಕಳು ಮರಿ ಆಗುತ್ತೆ ಎಂದಾದಾಗ, ಸಂಸಾರಕ್ಕೆ ಬೇಕಾದ ಸಾಮಾನುಗಳನ್ನು ಶೇಖರಿಸುವ ಸನ್ನಾಹ ಯುಕ್ತವಾಗುತ್ತೇಪ್ಪ! ಮಾನ ಮರ್ಯಾದೆಯಾಗಿ ಬಿಸಿಲು ಮಳೆ ಛಳಿ ಗಾಳಿಗಳಿಗೆ ಮರೆಯಾಗುವಂತೆ ನೆರಳೊಂದಿದೆ. ಸ್ವಲ್ಪ ನೆಲವೊಂದು ತೆಗೆದುಕೊಂಡು ಬಿಟ್ಟರೆ ಸಂಸಾರವು ಸುಖವಾಗಿ ಸಾಗಿ ಮುಂದುವರಿಯುತ್ತೆ ಅಂದಾಗ ಹೇಗೆ ಅದರ ಸಿದ್ಧತೆಗಳನ್ನು ಮಾಡಿಕೊಳ್ಳುತ್ತಾರೋ ಹಾಗೆ ದಿನೇ ದಿನೇ ಭಗವತ್ಪ್ರಜೆಗಳು ಬೆಳೆಯುತ್ತಿದ್ದಾರೆ. ಭಗವಂತನ ಸಂಸಾರ ಹೆಚ್ಚುತ್ತಿದೆ ಎಂದಾಗ ಕೆಲವು ಜವಾಬ್ದಾರಿಗಳು ಮನೆಯ ವೃದ್ಧರ ಮೇಲೂ ಹಿರಿಯಮಕ್ಕಳ ಮೇಲೂ ಬೀಳುತ್ತೆಪ್ಪಾ|

\section*{ಯಾವುದು ಲೈಬ್ರರಿಯಾಗಬೇಕು?}

ನೀವೂ ಭಗವಂತನ ಸಂಸಾರದಲ್ಲಿದ್ದೀರಿ. ಮನೆಗೊಂದು ಮುದಿಗೊರಡು ಬೇಕು, ಒಬ್ಬ ಅನುಭವಿಬೇಕು ಅನ್ನುವಂತೆ ನಮ್ಮನ್ನು ಉಪಯೋಗಿಸಿಕೊಳ್ಳಿ! ನಿಮ್ಮ ಸಾಧನೆಯಿಂದ ಮತ್ತು ನಮ್ಮ ಸಹಕಾರದಿಂದ ಬೆಳೆಯುವ ನಿಮ್ಮ ಅನುಭವದ ಸತ್ಯಸಾಹಿತ್ಯಗಳು ನಿಮ್ಮ ಲೈಬ್ರರಿಯಾಗಬೇಕಪ್ಪ! ಅದೇ ನಿಮ್ಮ ಮುಂದಿನ ಮಕ್ಕಳ ಜೀವನಕ್ಕೆ ಆಸ್ತಿಯಾಗುತ್ತೆಪ್ಪ! ನಮ್ಮ ಪ್ರಾಚೀನ ಮಹರ್ಷಿಗಳು ಭಾರತೀಯರ ಪೂರ್ಣಜೀವನಕ್ಕಾಗಿ ಯಾವ ದಿವ್ಯವಾಣಿಯನ್ನು ಅಪರಿಮಿತವಾದ ಆಸ್ತಿಯಾಗಿ ಮಾಡಿ ಇಟ್ಟುಹೋದರೋ, ಅದೆಲ್ಲವನ್ನೂ ಅಸಂಸ್ಕೃತರಾದ ನಮ್ಮವರೂ ಮತ್ತು ಇತರ ದೇಶೀಯರೂ ಸೇರಿಕೊಂಡು ಮೆಟ್ಟಿ ಮಣ್ಣುಪಾಲು ಮಾಡಿ ಬಿಟ್ಟಿರುವಾಗ, ಅದರ ಅಂತರಾಳದಲ್ಲಿರುವ ಸೊಬಗನ್ನು ಅಂದರೆ ಜೀವನೌಷಧವಾದ ಅಂಶವನ್ನು ಮತ್ತೆ ಎತ್ತಿಹಿಡಿದು, ಇದು (ಋಷಿಗಳಿಂದ ಬಂದದ್ದು ಇಂಥದು) ಈ ರೀತಿ ಇದೆ ಎಂದು ಸತ್ಯವನ್ನು ಹೊರಪಡಿಸುವ ರೀತಿ ಇರಬೇಕಪ್ಪ! ನಮ್ಮ ಅನುಭವ ಸಾಹಿತ್ಯದ ಲೈಬ್ರರಿ. ಕಾಳು ಯಾವುದು? ಜಳ್ಳು ಯಾವುದು? ಎಂದರಿಯದೇ ರಾಶಿಹಾಕಿದಂತೆ, ಕಾಗದ ಕಟ್ಟಿನ ರಾಶಿಯಾದರೆ ಲಾಭವಿಲ್ಲ.

\section*{ಶ್ರೀಗುರುವು ತಾನು ನಡೆಸಿದ ಪುಸ್ತಕ ಸಂಗ್ರಹಣೆಯ ಕುರಿತು}

ಯಾರಾದರೊಬ್ಬರಿಗೆ ಒಂದು ಒಳ್ಳೆಯ ಮಾದರಿಯ ದೇವರ ಮಂದಾಸನವನ್ನು ಮಾಡಬೇಕೆಂದು ಮನಸ್ಸಿಗೆ ತೋರಿಬಿಟ್ಟಾಗ, ಅದಕ್ಕೆ ಒಂದು ಪ್ಲಾನ್ೞ್ ಮತ್ತು ಸ್ಕೆಚ್ ಅನ್ನು ಮನಸ್ಸಿಗೆ ಅವನು ತಂದುಕೊಳ್ಳುತ್ತಾನೆ. ಆಗ ಇಲ್ಲಿಲ್ಲಿಗೆ ಇಂಥ ಮರದ ಪಟ್ಟಿಕೊಡಬೇಕು, ಇಲ್ಲಿಗೆ ಇಂಥ ಚಿತ್ರ ಬರಬೇಕು ಎಂದೆಲ್ಲಾ ಭಾವಿಸಿಕೊಂಡು, ಇಷ್ಟು ಮರ ಬೇಕಾಗುತ್ತೆ ಈ ಸೈಜಿನ ಪಟ್ಟಿಗಳು ಬೇಕಾಗುತ್ತೆಂದು ಎಲ್ಲಿ ಯಾವ ಮರದ ತುಂಡು ಚೂರು ಮೊದಲಾದ್ದು ಏನೇನು ಸಿಕ್ಕಿದರೂ ಇದು ಇಲ್ಲಿಗಾಗುತ್ತೆ ಇದಕ್ಕಾಗುತ್ತೆ ಎಂದುಕೊಂಡು ಮನಸ್ಸಿನಲ್ಲೇನೇನೋ ಲೆಖ್ಖಾಚಾರ ಹಾಕುತ್ತಾ, ಎಲ್ಲವನ್ನೂ ತಂದು ಒಂದೆಡೆಯಲ್ಲಿ ರಾಶಿಹಾಕುತ್ತಾನೆ. ಅವನ ಮನಸ್ಸಿನಲ್ಲಿರುವ ಈ ಗುಟ್ಟು ಗೊತ್ತಾಗದೇ ಮನೆಯ ಮಕ್ಕಳೋ ಹೆಂಡತಿಯೋ ಮಾತಾಡಿಕೊಂಡು, ಇವನಿಗೇನೋ ಮರದ ಹುಚ್ಚು ಹಿಡಿದಿದೆ ಎಂದು ಭಾವಿಸಿ ಲೇವಡಿ ಮಾಡಬಹುದು ಅಥವಾ ಅದನ್ನೆಲ್ಲಾ ಕಸವೆಂದು ಯೋಚಿಸಿ ತಿಪ್ಪೆಗೆಸೆಯಲೂ ಪ್ರಯತ್ನಿಸಬಹುದು. ಆದರೆ ಇವೆಲ್ಲ ಚೂರುಗಳ ಜೋಡಣೆಯ ಪ್ಲಾನು ಇವನಿಗೆ ಮಾತ್ರ ಗೊತ್ತಿರುತ್ತೆ. ಹಾಗೇನಾದರೂ ನಮ್ಮ ಅಂತರಂಗದಲ್ಲಿ ತೋರುವಂತಹ ಪೆರುಮಾಳ್ ಕಾರ್ಯವೇನಾದರೂ ಆಗುವುದಾದರೆ, ಸಮಯಕ್ಕುಪಯೋಗವಾಗಲೆಂದು, ಈ ಪುಸ್ತಕಗಳನ್ನು ಸಂಗ್ರಹಿಸುತ್ತಿದ್ದೇನೆಪ್ಪ! ನಾವೇನೂ ಹೋಗುವಾಗ ಯಾವುದನ್ನೂ ತಲೆಯಲ್ಲಿ ಹಾಕಿಕೊಂಡು ಹೋಗುವುದಿಲ್ಲ. ನೀವೂ ಅಷ್ಟೆ. ಎಲ್ಲವನ್ನೂ ಬಿಟ್ಟೇ ಹೋಗಬೇಕು. ಆ ನಮ್ಮ ಪ್ರಭುವೂ ಸಹ `ಇದನ್ನೆಲ್ಲಾ ಬಿಟ್ಟುತೊಲಗಿಸಿ ಬಾ' ಎನ್ನುತ್ತಾನೆಯೇ ಹೊರತು, ತಲೆಯ ಮೇಲೆ ಹಾಕಿಕೊಂಡು ಬನ್ನಿ ಎನ್ನುವುದಿಲ್ಲ. ಇಷ್ಟು  ಪ್ರಮಾಣದಲ್ಲಿ ತೆಗೆದುಕೊಂಡಾಗ ಲಿಪಿಯನ್ನೊಳಗೊಂಡ ಕಾಗದ ಮತ್ತು ಪುಸ್ತಕಗಳ ಆವಶ್ಯಕತೆಯು ಲೌಕಿಕವಾಗಿಯೂ ಮತ್ತು ಆಧ್ಯಾತ್ಮಿಕವಾಗಿಯೂ ಇರುತ್ತೆಪ್ಪ!.

\section*{ಮಹಾಭಾರತವನ್ನು ಕುರಿತು}

ನೀವು ಮಹಾಭಾರತ ಪುಸ್ತಕವನ್ನು ಅರ್ಪಿಸಿದಿರಿ! ಈ ಮಹಾಭಾರತದ ವಿಷಯವನ್ನೇ ಆಲೋಚಿಸಿದರೂ ಇದು ಕೇವಲ `ಆತ್ಮಕ್ರೀಡಃ ಆತ್ಮರತಿಃ\label{118} ಕ್ರಿಯಾವಾನೇಷ ಬ್ರಹ್ಮವಿದಾಂ ವರಿಷ್ಠಃ' ಎನ್ನುವಂತೆ, ಆ ಒಂದು ಭಾರೂಪವಾದ ಆತ್ಮವಸ್ತುವಿನಲ್ಲಿ ಸತ್ಯಪ್ರಕಾಶದಲ್ಲಿ ರಮಿಸುವ-ರತಿಮಾಡುವ ಕ್ರಿಯೆಗೆ- ರಮಣೀಯವಾದ ಕ್ರಿಯೆಗೆ ಒತ್ತಾಸೆಯಾಗಿದೆ. ಅದರ ಸಲುವಾಗಿ `ಭಾರತ' ಎಂಬ ಹೆಸರು ಪಡೆದಿದೆ. ಇದು ಭುವಿಯಲ್ಲಿರುವುದಾಗಿ ಕಂಡುಬಂದರೂ, ಮತ್ತೊಂದು ದೃಷ್ಟಿಯಿಂದ ನೋಡಿಗಾದ ಭೂರ್ಭುವಸ್ಸುವರ್ಲೋಕಗಳಲ್ಲಿಯೂ ನೆಲೆಸಿದ್ದು-

\begin{shloka}
`ಇದಂ ಹಿ ತ್ರಿಷು ಲೋಕೇಷು\label{118a} ಮಹದ್‌ಜ್ಞಾನಂ ಪ್ರತಿಷ್ಠಿತಮ್' ಎನ್ನುವಂತಾಗಿದೆ.
\end{shloka}

\section*{ಮಹಾಭಾರತದ ರಚನೆಯಲ್ಲಿ ತೊಡಗಿದ ವ್ಯಾಸರ ಸ್ಥಿತಿ ಗತಿಗಳನ್ನು ಕುರಿತ ಉಲ್ಲೇಖ ಪರಿಶೀಲನೆ}

ವ್ಯಾಸಮಹರ್ಷಿಗಳೂ ಸಹ ಈ ಮಹಾಭಾರತವನ್ನು ಯಾವ ಕಂಡೀಷನ್ ನಲ್ಲಿದ್ದ್ದುಕೊಂಡು ರಚನೆಮಾಡಿದರು? ಎಂದರೆ-

\begin{shloka}
`ಪುಣ್ಯೇ ಹಿಮವತಃ ಪಾರ್ಶ್ವೇ ಮಧ್ಯೇ ಗಿರಿಗುಹಾಲಯೇ |\label{118b}\\
ವಿಶೋಧ್ಯ ದೇಹಂ ಧರ್ಮಾತ್ಮಾ ದರ್ಭಸಂಸ್ತರಮಾಶ್ರಿತಃ ||\\
ಶುಚಿಃ ಸನಿಯಮೋ ವ್ಯಾಸಃ ಶಾಂತಾತ್ಮಾ ತಪಸಿ ಸ್ಥಿತಃ |\\
ಭಾರತಸ್ಯೇತಿಹಾಸಸ್ಯ ಧರ್ಮೇಣಾಽನ್ವೀಕ್ಷತೇ ಗತಿಮ್ ||\\
ಪ್ರವಿಶ್ಯ ಯೋಗಂ ಜ್ಞಾನೇನ ಸೋಽಪಶ್ಯತ್ಸರ್ವಮಂತತಃ |\\
ತಪಸಾ ಬ್ರಹ್ಮಚರ್ಯೇಣ ವ್ಯಸ್ಯ ವೇದಂ ಸನಾತನಮ್ ||\\
ಇತಿಹಾಸಮಿಮಂ ಚಕ್ರೇ ಪುಣ್ಯಂ ಸತ್ಯವತೀಸುತಃ ||'
\end{shloka}
ಎಂದು ಅದರ ಹಿನ್ನೆಲೆಯನ್ನು ಹೇಳಿಕೊಂಡಿದ್ದಾರೆ.

\section*{ಲೇಖಕನಾಗಲು ಒಪ್ಪಿಕೊಂಡ ಗಣೇಶನೊಡನೆ ಮಾಡಿಕೊಂಡ ನಿಯಮದ ಹಿನ್ನೆಲೆ}

ಇದನ್ನು ಯಾರ ಮೂಲಕ ಬರೆಯಿಸಿದ್ದು ಎಂದರೆ-

\begin{shloka}
`ಮಯೈವ ಪ್ರೋಚ್ಯಮಾನಸ್ಯ ಮನಸಾ ಕಲ್ಪಿತಸ್ಯ ಚ |\label{119c}\\
ಲೇಖಕೋ ಭಾರತಸ್ಯಾಸ್ಯ ಭವ ತ್ವಂ ಗಣನಾಯಕ ||
\end{shloka}

ಎಂದು ಗಣೇಶನಿಗೆ ಹೇಳಿ ಅವನ ಮೂಲಕ ಬರೆಯಿಸಿದರೆಂದು ಹೇಳುತ್ತೆ. ಅವನಿಂದ ಬರೆಸುವಾಗ ಅವನಿಗೆ ಹಾಕಿದ ನಿಯಮವಾದರೂ ಹೇಗಿತ್ತೆಂದರೆ- `ಅಬುದ್ಧ್ವಾ ಮಾಲಿಖ ಕ್ವಚಿತ್'\label{119a} (ಅಯ್ಯಾ ಗಣೇಶ! ಏನನ್ನು ಬರೆಯುವುದಾದರೂ ಅದರ ಅಭಿಪ್ರಾಯವನ್ನು ಮನಸ್ಸಿಗೆ ತಂದುಕೊಳ್ಳದೇ ಬರೆಯಬಾರದಪ್ಪ!) ಎಂದರು. ಆ ನಿಯಮಕ್ಕೆ ಅವನೂ ಒಪ್ಪಿದ.

\begin{shloka}
`ಓಮಿತ್ಯುಕ್ತ್ವಾ ಗಣೇಶೋಽಪಿ ಬಭೂವ ಕಿಲ ಲೇಖಕಃ'|\label{119b}
\end{shloka}

ನಂತರ ಗಣೇಶನೂ ಒಬ್ಬ ದೇವನೇ ಆದರೂ ದೈವೀಕ್ಷೇತ್ರದಲ್ಲೇ ನಿಂತು ಭೇದಿಸಬೇಕಾದ ರೀತಿಯಲ್ಲಿ, ಹಲವಾರು ಶ್ಲೋಕಗಳನ್ನು ಮಧ್ಯೇ ಮಧ್ಯೇ ರಚನೆ ಮಾಡಿದರು. 

\begin{shloka}
`ಸ್ವಾದುಮೇಧ್ಯರಸೋಽಪೇತಂ ಅಚ್ಛೇದ್ಯಮಮರೈರಪಿ |\label{119d}\\
ಗ್ರಂಥಗ್ರಂಥಿಂ ತದಾಚಕ್ರೇ ಶಾಖಾಪುಷ್ಪಫಲೋದಯಮ್ ||'\label{119}
\end{shloka}
ಎಂದು ವ್ಯಾಸಗುಟ್ಟೆನ್ನುವಂತಹ ಶ್ಲೋಕಗಳನ್ನೂ ರಚನೆಮಾಡಿದರು. ವಿಷಯ ಗಾಂಭೀರ್ಯಕ್ಕೆ ತಕ್ಕಂತೆ ಗಂಭೀರಸರಣಿಯಲ್ಲಿ ರಚಿತವಾದ ಮಹಾಗ್ರಂಥವನ್ನು ಅಭಿಪ್ರಾಯ ತಿಳಿಯದೇ ಬರೆದರೆ ಅನೇಕ ಲೋಪದೋಷಗಳಿಗೆ ಅವಕಾಶವಾಗುತ್ತೆಯಾದ್ದರಿಂದ ಲೇಖನಕ್ಕೆ ಮೀಸಲಾದ ಕಲೆಯೊಂದಿಗೆ ಬರೆಯಬೇಕಾಗಿದೆ ಯೆನ್ನುವುದೂ ಸರಿಯಾದ ಅಂಶವೇ.

\section*{ಲೇಖನವೂ ಒಂದು ಕಲೆ}

ಇದೇನಿದು? ಲೇಖನ ಕಲೆ ಎನ್ನುತ್ತಿರಲ್ಲಾ! ಅದಕ್ಕ್ಯಾವ ಕಲೆ? ಹೇಗೆ ಬರೆದರೂ ಅದೇ ಅಕ್ಷರಗಳೇ ಅನ್ನುವಂತಿಲ್ಲ. ನಿಜಕ್ಕೂ ಲೇಖನಕ್ಕಾಗಿ ಕಲೆಯುಂಟು. ಯಾವ ವಿಷವನ್ನು ಹೊರಗೆ ವ್ಯಕ್ತಪಡಿಸುತ್ತೇವೋ, ಅದರ ಆಶಯದೊಡನೆ ಕಲೆತುಕೊಳ್ಳುವಂತೆ ಬರೆದರೆ ಅದು ಲೇಖನ ಕಲೆ. ಬರಹಕ್ಕೆ ಬಂದ ಪಂಕ್ತಿಯನ್ನು ನೋಡಿದಾಗ ಅವನ ಹೃದಯವು ವಿಷಯದೊಡನೆ ಕಲೆದುಕೊಳ್ಳಬೇಕು. ಆವಾಗ ಕಲೆ.

\section*{ಹೃದಯವು ವಿಷಯದೊಡನೆ ಕಲೆತುಕೊಳ್ಳದಿದ್ದಾಗ ಲೇಖನದಲ್ಲಿ ದೋಷವಾಗುವುದೆಂಬ ಬಗ್ಗೆ ಉದಾಹರಣೆ}

ಇಲ್ಲದಿದ್ದರೆ, ಕಲೆಯಲ್ಲ ಕೊಲೆಯಾದೀತು. ಉದಾಹರಣೆಗೆ ರಾಮಾಯಣದ ಒಂದು ಶ್ಲೋಕ-

\begin{shloka}
`ರಾವಣೇನ ರಣೇ ಶಕ್ತಿಃ ಕ್ರುದ್ಧೇನಾಶೀವಿಷೋಪಮಾ |\label{120}\\
ಮುಕ್ತಾ ಶೂರಸ್ಯ ಭೀತಸ್ಯ ಲಕ್ಷ್ಮಣಸ್ಯ ಮಮಜ್ಜ ಸಾ ||
\end{shloka}
ಎಂಬುದಾಗಿದೆ. ಇಲ್ಲಿ `ಕ್ರುದ್ಧನಾದ ರಾವಣನಿಂದ, ಸರ್ಪದಂತಿರುವ ಒಂದು ಶಕ್ತಿಯು ಯುದ್ಧದಲ್ಲಿ ಬಿಡಲ್ಪಟ್ಟಿತು. ಅದು ಶೂರನೂ ಭೀತನೂ ಆದ ಲಕ್ಷ್ಮಣನಲ್ಲಿ ಮುಳುಗಿತು' ಎಂದು ಅರ್ಥ ಬರುತ್ತದೆ. ಲಕ್ಷ್ಮಣನು ಶೂರನೂ ಭೀತನೂ ಆಗಿರುವನೆಂದು ವಾಲ್ಮೀಕಿ ವರ್ಣಿಸುವರೇ? ಮತ್ತು ಮುಳುಗಿತು ಎಂದರೆ ಎಲ್ಲಿ? ಎಂಬುದರ ವಿವರಣೆ ಇಲ್ಲ. ಏಕೆ ಹೀಗಾಯಿತು? ಶ್ಲೋಕವೇ ಅಸಂಬದ್ಧವೆನ್ನೋಣವೇ? ಸುಸಂಬದ್ಧವಾಗಿ ಬರೆಯದೇ ಇದ್ದುದರ ಅಪರಾಧವಿದು ಎನ್ನೋಣವೇ? ನೋಡಿಪ್ಪಾ! ಇದಕ್ಕಾಗಿ ಲೇಖನವನ್ನು ಕಲೆಯೊಡನೇ ಮಾಡಬೇಕೆಂಬುದು. ಅರ್ಥಜ್ಞಾನವಿಲ್ಲದವನು ಅಷ್ಟುವರ್ಣಗಳನ್ನೂ ಲಿಪೀಕರಣ ಮಾಡುತ್ತನಷ್ಟೆ. ಅರ್ಥಜ್ಞಾನವಿದ್ದವನು ಒಂದು ವಿಷಯದ ಬರವಣಿಗೆಯಾಗಲೀ ಭಾಷೆಯಾಗಲೀ ಯಾವ ಉದ್ದೇಶದಿಂದ  ಹೊರಡುವುದೋ ಆ ಉದ್ದೇಶ ಈಡೇರುವಂತೆ ಕಲಾತ್ಮಾಕವಾಗಿ ಬರೆಯುತ್ತಾನೆ. ಶ್ಲೋಕದ ಅಭಿಪ್ರಯವಾದರೂ ಏನೂ? ಎಂದರೆ ಕ್ರುದ್ಧನಾದ ರಾವಣನಿಂದ ಯುದ್ದದಲ್ಲಿ ಬಿಡಲ್ಪಟ್ಟ ಸರ್ಪಸಮಾನವಾದ ಆ ಶಕ್ತಿಯು ನಿರ್ಭೀತನೂ ಶೂರನೂ ಆದ ಲಕ್ಷ್ಮಣನ ಎದೆಯಲ್ಲಿ ತ್ವರಿತವಾಗಿ ಸೇರಿ ಮುಳುಗಿ ಹೋಗಿ ಮರೆಯಾಯಿತು ಎಂದು. ಈ ಅಭಿಪ್ರಯವು ಹೊರಬರಲು, ಸಂಸ್ಕೃತಭಾಷೆಯ ವ್ಯಾಕರಣ, ಸಂಧಿ ಮೊದಲಾದವುಗಳ ಸ್ಫೂರ್ತಿಯ ಜೊತೆಗೆ ಲೇಖನವೂ ಸ್ವಲ್ಪಮಟ್ಟಿಗೆ ಪರಿಷ್ಕೃತವಾಗಿರಬೇಕು.

\begin{shloka}
`ರಾವಣೇನ ರಣೇ ಶಕ್ತಿಃ ಕ್ರುದ್ದೇನಾಶೀವಿಷೋಪಮಾ |\\
ಮುಕ್ತಾಽಽ ಶೂರಸ್ಯಭೀತಸ್ಯ ಲಕ್ಷ್ಮಣಸ್ಯ ಮಮಜ್ಜ ಸಾ ||'
\end{shloka}
ಇಲ್ಲಿ `ಮುಕ್ತಾ ಆಶು ಉರಸಿ ಅಭೀತಸ್ಯ' ಎಂಬ ಪದ ವಿಭಾಗವನ್ನು ಜ್ಞಾಪಿಸಲು ಸಹಕಾರಿಯಾಗಿರಬೇಕು ಈ ಲೇಖನ. ಕನ್ನಡ ಭಾಶೇಯ ಒಂದು ಪದ ನೋಡಿ! `ನಿಂಗಣ' ಎಂದು, ಇಲ್ಲಿ  `ನಿಂಗ+ಅಣ್ಣ' ಎಂಬ ಅರ್ಥವೂ ಬರಬಹುದು. `ನಿನಗೆ ಅಣ್ಣ' ಎಂಬರ್ಥವೂ‌ ಬರಬಹುದು. `ನಿಂಗ ಅಣ್ಣ' ಎಂದಾದರೆ `ನಿಂಗಣ್ಣ' ಬರೆದರೆ ಸಾಕು. ನಿನಗೆ ಅಣ್ಣ ಎಂದಾದಾಗ `ನಿನ್ಗಣ್ಣ' ಎಂದು ಬರೆಯಬೇಕಾಗುತ್ತೆ. ಇಲ್ಲದಿದ್ದರೆ ಅರ್ಥ ಬರುವುದಿಲ್ಲ. ಅನೇಕ ವೇಳೆ ವಿಷಯವನ್ನರ್ಥಮಾಡಿಕೊಳ್ಳದೇ ಹೇಗೆ ಕಿವಿಗ್ರಹಿಸಿತೋ ಹಾಗೆ ಬರೆದುಬಿಟ್ಟು ಆಮೇಲೆ ವಿಷವನ್ನರಿತ ಮೇಲೆ ಹೀಗೇಕೆ ಬರೆದೆ? ಎಂದು ಒದ್ದಾಡಬಹುದು. ಕೆಲವು ವೇಳೆ ತಪ್ಪುತಪ್ಪಾಗಿ ಬರೆದು ತಾನೇ ಓದಲಾಗದೇ ಭ್ರಮಿಸುವುದು ಎಲ್ಲಾ ಉಂಟು. ಒಂದೇ ಪಂಕ್ತಿಗೆ ಹತ್ತಾರು ಪಾಠಾಂತರದ ಖಾಯಿಲೆ ಬಂದ ಕ್ರಮ ಇದೇ ತಾನೆ. ಹಿಂದೆ ಹೇಳಿತೆಲ್ಲಾ, `ಕ' ಪ್ರತಿ `ಕ' ಪ್ರತಿ `ಗ' ಪ್ರತಿ ಎಂದೆಲ್ಲಾ. ಅರಿತು ಬರೆದರೆ ನಾನಾ ಬಗೆಯ ಅಬದ್ಧ ಪಾಠಾಂತರಗಳೇಕೆ ಬರುತ್ತಿದ್ದವು? ಯಾವಳೋ ಒಬ್ಬಳು ಹೇಳಿದಳಂತೆ `ನನ್ನ ಗಂಡ ಬಹಳವೇ ಜಾಣ. ಆದರೆ ತಾನು ಬರೆದುದನ್ನು ಬಿಟ್ಟು, ಮತ್ತ್ಯಾರು ಬರೆದುದನ್ನೂ ಓದಲಾರ' ಎಂದು. ಮತ್ತೊಬ್ಬಳು ಹೇಳಿದಳಂತೆ `ನನ್ನ ಗಂಡ ಇನ್ನೂ ಜಾಣ ಅವನು ಬರೆದುದನ್ನು ಅವನು ಬಿಟ್ಟು ಇನ್ನಾರೂ ಓದಲಾರದು' ಎಂದು. ಇನ್ನೊಬ್ಬಳು ಹೇಳಿದ್ದು ಹೇಗೆಂದರೆ `ನನ್ನ ಗಂಡ ಮತ್ತೂ ಜಾಣ. ಅವನು ಬರೆದದ್ದನ್ನು ಅವನೇ ಓದಲಾರ' ಎಂದು ಹೀಗೆಲ್ಲಾ ಆಗುತ್ತೆ ಲೇಖನ ಕಲೆಯನ್ನು ಬಳಸದೇ ಲಿಪಿ ಬರೆದರೇ ಸಾಕೆಂದು ಕೊಂಡರೆ ಆದ್ದರಿಂದ ಅರ್ಥವನ್ನರಿತು ಲೇಖನ ಮರ್ಯಾದೆಯನ್ನನುಸರಿಸಿ ಬರೆದರೆ ಅರ್ಥವತ್ತಾಗಿರುತ್ತೆ.

\section*{ತತ್ತ್ವವನ್ನು ಮರೆತಾಗ ಹುಟ್ಟಿದ ಕಲ್ಪನಾ ವೈಭವಗಳು}

ಮಹಾಭಾರತದ ಬಗ್ಗೆ ಹೇಳುವುದಾದರೆ, ಗಣಪತೆಯೇ ಸಾಕ್ಷಾತ್ತಾಗಿ ಬರೆದ ಪುಸ್ತಕವೇ ಸಿಕ್ಕಿಬಿಟ್ಟರೆ, ಇಷ್ಟಾರು ಅಪವ್ಯಾಖ್ಯಾನಗಳಿಗೆ ಅವಕಾಶವಿರುತ್ತಿರಲಿಲ್ಲವೋ ಏನೋ? ಎಂದು ತೋರುತ್ತೆ. ಆದರೆ ಈ ಶ್ರೀವೈಷ್ಣವರು ಕಲ್ಪಿಸಿಕೊಂಡ ಪರಿಪಾಟಿಯಿಂದ ಅವರು ಗಣಪತಿ ಬರೆದ ಪುಸ್ತಕವನ್ನೆದುರು ನೋಡುವಂತಿಲ್ಲ. ವಿಷ್ವಕ್ಸೇನರ ಕೈಗೆ ವ್ಯಾಸರು ಕೊಡಲಿಲ್ಲವೇಕೆ? ಎಂಬ ಚಿಂತೆ ಹುಟ್ಟಿದರೆ, ಅದಕ್ಕೆ ಉತ್ತರ ಹೇಳುವವರಿಲ್ಲ. ವಿಷ್ವಕ್ಸೇನರಿಂದಲೇ ಬರೆದಿದ್ದಕ್ಕೆ ಸ್ಮಾರ್ತರು ಹೆಸರು ಬದಲಾಯಿಸಿ ಶ್ಲೋಕಮಾಡಿ ಗಣೇಶನನ್ನು ಹಾಕಿರಲೂ ಸಾಧ್ಯ ಎಂದು ಇವರು ಶಂಕಿಸಲೂಬಹುದು. ಒಟ್ಟಿನಲ್ಲಿ ಹೇಳುವುದಾದರೆ ತತ್ತ್ವವನ್ನು ಮರೆತ ಕಲ್ಪನಾವೈಭವಗಳೆಲ್ಲಾ ಸೇರಿ ನಗೆಗೇಡಿನ ಜೀವನವಾಗಿದೆ.

ಇನ್ನು ಇದರಲ್ಲಿ ಕಾಣುವ ಕಥೆಗಳ ಮೇಲೆ ವಿಧವಿಧವಾದ ಟೀಕೆಗಳು, ವಿಧವಿಧವಾದ ವಿಮರ್ಶೆಗಳು ಬೇರೆ ಅವಾಂತರ. ಒಬ್ಬನು ಇದೆಲ್ಲಾ ಹಾದರದ ಕಥೆ, ಇದನ್ನು ಮುಟ್ಟಬಾರದೆನ್ನುವುದು. ಇನ್ನೊಬ್ಬ ಇದು ನಮ್ಮ ದೇಶದ ಸಂಸ್ಕೃತಿಯೇ ಅಲ್ಲ, ಟಿಬೆಟ್ಟಿನ ಸಂಸ್ಕೃತಿ, ಅಲ್ಲಿ ಒಬ್ಬ ಹೆಂಗಸನ್ನೇ ನಾಲ್ಕೈದು ಜನರು ಬಳಸಿಕೊಳ್ಳುವ ರೂಢಿಯಿತ್ತು, ಅದರ ಇನ್‌ಪ್ಲುಯೆನ್ಸ್ ಇದರ ಮೇಲೆ ಬಿದ್ದಿದೆ ಎನ್ನುವುದು. ದ್ರೌಪತಿಯ ವಸ್ತ್ರಾಪಹರಣ ಪ್ರಸಂಗದ ಸಾಹಿತ್ಯವನ್ನೋದಿ ಬಿಟ್ಟು, ಅವಳು ಉಟ್ಟಿದ್ದ ವಸ್ತ್ರ ಚೈನಾವಸ್ತ್ರವೇ? ಜಪಾನಿನದೇ? ಇಂಗ್ಲೀಷಿನವರ ತಯಾರಿಕೆಯೇ? ಒಂದು ಕಥೆ ಇತ್ತು- ಒಂದು ಮ್ಯಾಚ್‌ಬಾಕ್ಸ್‌ನಲ್ಲಿ ೧೮ ಮೊಳದ ಸೀರೆ ಮಡಸಿ ಇಟ್ಟಿದ್ದರಂತೆ ಎಂದು, ಅಂತಹ ವಸ್ತ್ರವೇ? ಯಾವುದು ಆ ವಸ್ತ್ರ? ಎಂದು ಬೇರೆ ಪರಿಶೀಲನೆ, ಹೀಗಾಗಿ ಹದಗೆಟ್ಟ ಚಿತ್ರಣ ಮಾಡಿಕೊಳ್ಳುತ್ತಿದ್ದಾರೆ. ಮೈ ಮನಗಳನ್ನು ಸತ್ಯಾನ್ವೇಷಣೆಗೆ ಕೊಟ್ಟುಕೊಳ್ಳಲಾಗದ ಚಪಲಜೀವಿಗಳ ಕಪಿಚೇಷ್ಟೆಯಾಗಿದೆ ಇದೆಲ್ಲ.

\section*{ಮಹಾಭಾರತವು ಪಂಚಮವೇದವಾಗಿದೆ}

ನಿಜವಾಗಿ ನೋಡುವುದಾದರೆ, ಸಕಲವೇದಗಳ ಅಭಿಪ್ರಯವನ್ನೂ ಒಂದು ಉತ್ತಮವಾದ ರೀತಿಯಲ್ಲಿ ಜನರ ಮನಸ್ಸಿಗೆ ತರಲು ಹೊರಟಿರುವ ಈ ಪವಿತ್ರ ಗ್ರಂಥವು ಆಸೇತುಶೀತಾಚಲ ಬಾಳುವ ಪ್ರತಿಯೊಬ್ಬ ಭಾರತೀಯ ಆರ್ಯಜನದ ಪೂರ್ಣ ಜೀವನಕ್ಕೆ ಬೇಕಾದ ಒಂದು ಹೊತ್ತಿಗೆಯಾಗಿದೆ. ಆದ್ದರಿಂದ `ಭಾರತಃ ಪಂಚಮೋ ವೇದಃ'\label{122} ಎಂಬ ಅಮರವಾಣಿಗೆ ವಿಷಯವಾಗಿದೆ.

\section*{ಒಳ್ಳೆಯದನ್ನು ಬೆಳೆಸಬೇಕಾದರೆ ಅದರ ಬಗ್ಗೆ ಇರುವ ಅಪಪ್ರಚಾರವನ್ನು ತಿಳಿಯಬೇಕಾಗುವುದು}

ಇಂತಹ ಗ್ರಂಥದ ವಿಷಯದಲ್ಲಿ ಯಾರು ಯಾರು ಏನೇನು ಅಪಪ್ರಚಾರವನ್ನು ಮಾಡುತ್ತಿದ್ದಾರೆಂಬ ಬಗ್ಗೆ ಪ್ರಧಾನವಾಗಿ ತಿಳಿದುಕೊಳ್ಳಬೇಕು. ಲೋಕದಲ್ಲಿಯ ಜನರ ಆರೋಗ್ಯವನ್ನು ತಿಳಿದುಕೊಳ್ಳುವ ಮೊದಲು, ಅನಾರೋಗ್ಯವನ್ನು ತಿಳಿದುಕೊಳ್ಳುವ ಅವಶ್ಯಕತೆ ಇರುವುದರಿಂದ, ಸತ್ಪ್ರಚಾರವನ್ನು ಗಮನಿಸುವ ಮೊದಲೇ ಅಪಪ್ರಚಾರವನ್ನು ತಿಳಿದುಕೊಳ್ಳಬೇಕಪ್ಪ. ಅದಕ್ಕಾಗಿ ನಾನು ಒಳ್ಳೆಯ ಪುಸ್ತಕಗಳನ್ನು ತೆಗೆಯುವುದಕ್ಕಿಂತ ಮೊದಲು. ಅಂತಹ ಪುಸ್ತಕ ತೆಗೆಯುತ್ತೇನೆ. ಏಕೆಂದರೆ ಎಷ್ಟು ಬಗೆಯ ಅನಾರೋಗ್ಯಗಳಿವೆ? ಎಂಬುದನ್ನು ತಿಳಿದು, ಅದೆಲ್ಲಾ ಹೇಗೆ ಹೇಗೆ ಎಲ್ಲೆಲ್ಲಿ ಹರಡಿದೆಯೆಂದು ಗಮನಿಸಿ, ಅದರ ನಿವಾರಣೆಗೆ ಬೇಕಾಗುವ ಔಷಧಿಗಳನ್ನುಪಯೋಗಿಸಿ ವ್ಯಾಧಿನಿವಾರಣೆಯಾದರೆ, ಆರೋಗ್ಯವು ತಾನಾಗಿಯೇ ಬರಬಹುದಪ್ಪ!

\section*{ಪುಸ್ತಕಸ್ವೀಕರಣದ ಹಿಂದಿರುವ ಶ್ರೀಗುರುವಿನ ಮನಸ್ಸು}

ನಾನು ಪುಸ್ತಕಗಳನ್ನು ತೆಗೆಯಲು ಹೊರಡುವಾಗ ಇಂಥದೆಲ್ಲಾ ಆಲೋಚನೆಯನ್ನಿಟ್ಟುಕೊಂಡು ತೆಗೆಯುತ್ತೇನೆ. ನಿಮಗೂ ಇಷ್ಟೇ ಅಭಿಪ್ರಾಯವಿರಬೇಕು. ನೀವೂ ಈ ಬಗೆಯ ಅಭಿಪ್ರಾಯವನ್ನು ಹೊಂದಿ ಇಲ್ಲಿಗೆ  ಅರ್ಪಣಾ ಬುದ್ಧ್ಯಾ ತಂದಿದ್ದರೆ, ಅದಕ್ಕಾಗಿಯೇ ತೆಗೆದುಕೊಳ್ಳುತ್ತೇನೇಪ್ಪಾ! ಇಷ್ಟು ಅಭಿಪ್ರಾಯವನ್ನು ಬಿಟ್ಟು ಮತ್ತಾವುದೇ ರೀತಿಯ ಪುಸ್ತಕಹವ್ಯಾಸ ನನಗೆ ಇಲ್ಲ. ನೀವೂ ಹಾಗೆಲ್ಲಾ ತರಲೂ ಬೇಡಿ.

\section*{ಆತ್ಮಚ್ಛಂದಾನವರ್ತನೆಯಿಂದ ಸುಖಿಗಳಾಗುವಂತೆ ಸಂದೇಶ}

ಜೀವನದಲ್ಲಿ ಸ್ವಾಚ್ಛಂದ್ಯದಿಂದ ಸಂಭವಿಸುವ ಬಗೆ ಬಗೆಯ ಕಷ್ಟನಷ್ಟಗಳನ್ನೆಲ್ಲಾ ಮನಗಂಡು, ಅದನ್ನೆಲ್ಲಾ ಕಳೆದು ಅಂದವಾಗಿ ಬಾಳಾಟ ಮಾಡಲು, ಸತ್ಯ-ಸೌಂದರ್ಯ-ಮಾಂಗಲ್ಯಗಳೊಂದಿಗೆ ಬಾಳುವಂತಾಗಲು, ಸ್ವಾಚ್ಛಂದ್ಯವನ್ನು ತೊರೆದು ಆತ್ಮಚ್ಛಂದಾನುವರ್ತನೆಗಾಗಿ ಈ ಜೀವನವನ್ನೇ ಇನ್ನಾರೋ ಒಬ್ಬನಿಗೆ ಒಪ್ಪಿಸಿಕೊಂಡದ್ದಾಗಿದೆ. ಅವನ ಛಂದದಂತೆ ವರ್ತಿಸಿ ಅವನಂತೆ ಆಡಿ ಸುಖವಾಗಿ ಬಾಳೀಪ್ಪ!

\section*{ಕಾಲದೇಶಾದಿಗಳು ಅನುಕೂಲಿಸಿದಾಗ ಆಗಬೇಕಾದ ಕಾರ್ಯವನ್ನುಕುರಿತು ಶ್ರೀಗುರುವಿನ ಆಶಯ}

ಹೊರಗಡೆಯ ಜೀವನವನ್ನೂ ಈ ಹೇಳಿದ ಅಭಿಪ್ರಾಯಕ್ಕೆ ತಕ್ಕಂತೆ ಇಟ್ಟುಕೊಳ್ಳಲು ಅನುಕೂಲಕರವಾದ ಸನ್ನಿವೇಶಗಳಿದ್ದಿದ್ದರೆ ಮುಂದೆ ಬರಲಿರುವ ಭಗವತ್ಪ್ರಜೆಗಳಿಗಾಗಿ ಕೆಲವು ಕೆಲಸಗಳನ್ನು ಮಾಡಬಹುದಾಗಿತ್ತು. ಆದರೆ ಎಲ್ಲರೂ ನಿಮ್ಮ ನಿಮ್ಮ ಜೀವನಗಳನ್ನೆಲ್ಲಾ ಯಾರೋ ಒಬ್ಬ ಪುರುಷನಿಗೆ ತಗಲಿಸಿಕೊಂಡು ಜೀವನ ನಡೆಸಬೇಕಾಗಿರುವುದರಿಂದ, ನಮ್ಮ ನಿರೀಕ್ಷಣೆಯ ಯಾವುದೇ ಕೆಲಸವೂ ಎಷ್ಟು ಸಾಗಬಹುದೆಂಬ ಬಗೇ ಯಾವುದೇ ನಿರೀಕ್ಷಣೆ ಭರವಸೆಗಳು ಇಟ್ಟುಕೊಳ್ಳುವುದಕ್ಕಾಗುವುದಿಲ್ಲ.

ನಮ್ಮ ಮನಸ್ಸಿನಲ್ಲಿ ಕೆಲವು ಆಶಯಗಳೇನೋ ಇವೆ. ಭಾರತೀಯ ಜೀವನಕ್ಕೆ ಕೈದೀವಿಗೆಯಂತಿರುವ ಹೊತ್ತಿಗೆಯಾದ ಮಹಾಭಾರತ-ಭಾಗವತ-ರಾಮಾಯಣ ಮೊದಲಾದ ಕೆಲವು ಗ್ರಂಥಗಳನ್ನು ಒಂದು ಬಗೆಯ ಹಿನ್ನೆಲೆ, ವಿಷಯವಿಭಾಗ ಮತ್ತು ಆವಶ್ಯಕ ವಿವರಣೆಗಳೊಂದಿಗೆ ಲೋಕದ ಮುಂದಿಡಬೇಕೆಂಬುದಾಗಿ, ಸಮಯ ದೇಶ ಕಾಲಗಳು ಒದಗಿಬಂದಾಗ ನೋಡಿಕೊಳ್ಳೋಣ.

\section*{ಇಂದಿನ ಪುಸ್ತಕ ಮುದ್ರಣಗಳಲ್ಲಿ ಜಾಹಿರಾತುಗಳ ಅನೌಚಿತ್ಯ}

ಇತ್ತೀಚೆಗೆ ಜನರ ಪ್ರವೃತ್ತಿ ಏನಾಗಿದೆಯೆಂದರೆ ಯಾವುದೇ ಪುಸ್ತಕವನ್ನು ಮುದ್ರಿಸಲು ತೆಗೆದುಕೊಂಡರೂ ಅದರ ಅಭಿಪ್ರಾಯವನ್ನು ಚಿತ್ರಿಸುವ ಹಿನ್ನೆಲೆಯ ಸ್ಥಾನದಲ್ಲಿ ಪ್ರಬಲವಾದ ಜಾಹೀರಾತುಗಳು ಎದ್ದು ಕಾಣಲು ಶುರುವಾಗಿರುತ್ತವೆ. ಒಂದು ವೇದ ಪುಸ್ತಕವನ್ನೇ ನೋಡಿ. ಮೊದಲಿಗೆ ಆಂಡಾಳ್ ನಶ್ಯಗಳ ಅಡ್ವರ್ಟೈಸ್‌ಮೆಂಟ್. ನಮ್ಮವರೇ ಒಬ್ಬರು ಮಹಾಭಾರತ ಪುಸ್ತಕವನ್ನು ಮುದ್ರಿಸಲು ಇಲ್ಲಿ ದೈವಾನುಮತಿ ಆಶೀರ್ವಾದಗಳನ್ನು ಬಂದು ಪ್ರಾರ್ಥಿಸಿದ್ದರು. ಅದಕ್ಕೆ ಅವರಲ್ಲಿ ಅಪ್ಪಾ ಯಾವ ವಿಷಯವನ್ನು ಪ್ರಕಟಿಸಲು ಹೊರಡುತ್ತೇವೋ ಅದು ವಿರೂಪಗೊಳ್ಳದಿರಲು ಜಾಹಿರಾತುಗಳನ್ನು ಮೊದಲು ದೂರಮಾಡಬೇಕು, ಸಭಾಚರರಾಗಿ ಪುಸ್ತಕ ಪ್ರಚಾರಮಾಡಬೇಕು, ಇತ್ಯಾದಿ ಜವಾಬ್ದಾರಿ ಹೇಳಿದ್ದೆ. ಅದಾವುದೂ ಉಪಯೋಗಕರವಾಗಲಿಲ್ಲ. ಮತ್ತವರು ಜಾಹೀರಾತುಮಯವಾಗಿ ಪುಸ್ತಕ-ಸಂಚಿಕೆಗಳನ್ನು ಹೊರಹಾಕುತ್ತಿದ್ದಾರೆ. ಅವರ ಪ್ರಕಟಣೆಯನ್ನು ಲೋಕಕ್ಕೆ ದರ್ಶನ ಮಾಡಿಸಬೇಕಾದರೆ, ಜಾಹೀರಾತುಗಳ ದರ್ಶನವಿಲ್ಲದೇ ಸಾಧ್ಯವಿಲ್ಲವಾಗಿದೆ. ಹೀಗಾಗಿದೆಪ್ಪ ಸನ್ನಿವೇಶ. ಆದರೆ ಗೋರಖಪುರ್ ಗೀತಾ ಪ್ರೆಸ್ ನವರು ತಕ್ಕಮಟ್ಟಿಗೆ ಪ್ರಾಮಾಣಿಕವಾಗಿ ಕೆಲಸಮಾಡುತ್ತಾರೆ. ಯಾವ ಜಾಹಿರಾತಿಗೂ ಜಾಗಕೊಡುವುದಿಲ್ಲ. ಪ್ರಿಂಟ್ ತಪ್ಪುಬಿದ್ದ ಜಾಗದಲ್ಲಿ ತಾವೇ ಇಂಕಿನಿಂದ ತಿದ್ದುತ್ತಾರೆ. ಲಾಭಾಪೇಕ್ಷೆಯೂ ಇಲ್ಲದೇ ಕೆಲಸಮಾಡುತ್ತಾರೆ. ಐಡಿಯಾ ಚೆನ್ನಾಗಿದೆ. 

\section*{ತನ್ನ ದೃಷ್ಟಿಯನ್ನೇ ಆರ್ಷಗ್ರಂಥಗಳ ಮೇಲೂ ಚೆಲ್ಲುವುದರ ಲಾಭವೇನು?}

ಆದರೆ ಲೋಕಕ್ಕೆ ಒಂದು ತಾತ್ತ್ವಿಕ ದೃಷ್ಟಿಯನ್ನು ಕೊಡಬೇಕು. ಅದಕ್ಕವಕಾಶವಿಲ್ಲದೆ ತನ್ನ ದೃಷ್ಟಿಯನ್ನೇ ಆರ್ಷಗ್ರಂಥಗಳ ಮೇಲೂ ಚೆಲ್ಲುತ್ತಿದೆಯಪ್ಪ ಲೋಕ. ಸ್ವಚ್ಛಂದವಾದ ನಮ್ಮ ನಮ್ಮ ದೃಷ್ಟಿಯನ್ನೇ ಬೆಳೆಸುವುದಕ್ಕೆ  ವೇದವು ಏಕೆ? ಸ್ಮೃತೀತಿಹಾಸ ಪುರಾಣಗಳೇಕೆ?

\section*{ಎಲ್ಲರಿಗೂ ತಾತ್ತ್ವಿಕ ದೃಷ್ಟಿಯನ್ನು ಕೊಡಲಾಗುವುದಿಲ್ಲ}

ಎಷ್ಟೋಮಕ್ಕಳಿಗೆ ಏನೇನೇ ಕಷ್ಟಪಟ್ಟರೂ ಕೂಡುವ ಲೆಖ್ಖ, ಕಳೆಯುವ ಲೆಖ್ಖ್ ತಿಳಿಸುವುದಕ್ಕಾಗುವುದಿಲ್ಲ. ಅದರಂತೆ ಲೋಕದಲ್ಲಿ ಎಷ್ಟೋ ಜನರಿಗೆ ತಾತ್ತ್ವಿಕ ದೃಷ್ಟಿಯನ್ನು ಕೊಡಲಾಗುವುದಿಲ್ಲ. ಯಾವ ಯಾವ ಬಗೆಯಲ್ಲಿ ಯತ್ನಿಸಿದರೂ ಕೆಲವರು ತಮ್ಮ ಮೌರ್ಖ್ಯದ ರೇಖೆಯನ್ನು ದಾಟಿ ತಾತ್ತ್ವಿಕತೆಯ ಕಡೆಗೆ ತಿರುಗುವುದೇ ಇಲ್ಲ. ಅವರಿಗಾಗಿ ಈ ಪ್ರಯತ್ನವಲ್ಲ. ಒಳಗೆ ಒಂದಿಷ್ಟು ಲಾಲಾಜಲವಿದ್ದು ಬಾಯಾರಿದ್ದವರಿಗೆ ಒಂದು ಲೋಟ ನೀರನ್ನು ಕೊಡಬೇಕಲ್ಲವೇ? ಅಂಥವರು ಕುಡಿದ ನೀರಿಂದ ಬದುಕಬಹುದು. ಅಂಥವರಾಗಿ ಗೂಢಸಂಸ್ಕಾರವಿದ್ದು ಈ ಪಿಪಾಸೆಯುಳ್ಳ ಜೀವಿಗಳೂ ಲೋಕದಲ್ಲಿದ್ದೇ ಇರುವರು. ಅಂಥವರಿಗಾಗಿ ಈ ಮಾತು.

\section*{ನಿಸರ್ಗದ ಅಭಿಪ್ರಾಯಕ್ಕೆ ವಿರೋಧವಿಲ್ಲದ ವರ್ಣನೆಗಳು ಸ್ವಾಭಾವಿಕವೇ}

ಹಾಗೆ ಲೋಕಕ್ಕೆ (ಕಾವ್ಯೇತಿಹಾಸ ಪುರಾಣದಿಗಳಗೆ ಉತ್ತಮಸಾಹಿತ್ಯಗಳದ್ದಾರಾ) ಒಂದು ಹಿನ್ನೆಲೆಯನ್ನು ಕೊಡಲು ಹೊರಟಾಗ, ಅದೇ ಇತಿಹಾಸ ಪುರಾಣಾದಿಗಳಲ್ಲಿ ಕಚ-ಗುಚ ವರ್ಣನೆಗಳೂ ಗೋಚರವಾಗಬಹುದು. ಅವುಗಳು ಸೃಷ್ಟಿಗೆ ಅವಿನಾಭೂತವಾಗಿರುವಂತೆ, ಕಾವ್ಯಕ್ಕೂ ಅವಿನಾಭೂತವಾಗಿ ಬಂದಾಗ, ಅದನ್ನೇನೂ ನಾವು ದೂಷಿಸಬೇಕಾಗಿಲ್ಲ. ಅಂತಹ ವರ್ಣನೆಗಳು ಅಂತರಂಗ ಬಹಿರಂಗಗಳ ಸಾಮರಸ್ಯ ಯುಕ್ತವಾದ ಜೀವನಕ್ಕೆ ಅವಿರುದ್ಧವಾಗಿದೆಯೇ? ಎಂಬುದನ್ನಷ್ಟೇ ಗಮನಿಸಬೇಕು. ಸೃಷ್ಟಿಯಲ್ಲಿ ಸಹಜಸಿದ್ಧವಾದ ಆಯಾ ಲಕ್ಷಣಗಳನ್ನು ನಿಸರ್ಗದ ಪೂರ್ಣಾಭಿಪ್ರಾಯಕ್ಕೆ  ವಿರೋಧವಾಗದಂತೆ ಉಪಯೋಗಿಸಿಕೊಳ್ಳುವ ಜಾಗದಲ್ಲಿ ಆಯಾ ವರ್ಣನೆಗಳೆಲ್ಲಾ ಸ್ವಾಭಾವಿಕತೆಗೇ ಸೇರುತ್ತವೆ.

ವರ್ಣನೆಗಳೆಲ್ಲವೂ ಕಾಮಪ್ರೇರಿತವಾಗಿಯೇ ಹುಟ್ಟಿವೆಯೆಂದೇನೂ ಹೇಳಲಾಗುವುದಿಲ್ಲ. ಆದ್ದರಿಂದಲೇ ಕಚಕುಚಾದಿಗಳನ್ನು ವರ್ಣಿಸುವವರೆಲ್ಲರೂ ಕಾಮುಕರೇ ಎಂದೂ ಹೇಳಲಾಗುವುದಿಲ್ಲ. ಹೊಟ್ಟೆಯಲ್ಲಿ ಹುಟ್ಟಿದ ಮಗಳ ತಾರುಣ್ಯವನ್ನು ನೋಡಿ ಸಂತೋಷಪಡುವ ತಂದೆ ತಾಯಂದಿರಿಗೆ ಮಗಳ ಮೇಲೆ ಕಾಮವಿದೆಯೆನ್ನಲಾಗದು. ಅಂತೆಯೇ ಶಂಕರಭಗವತ್ಪಾದರು ಸೌಂದರ್ಯ ಲಹರಿಯಲ್ಲಿ ದೇವಿ ಜಗನ್ಮಾತೆಯ ಅಪಾದಮಸ್ತಕ ವರ್ಣನೆಮಾಡಿದ್ದಾರೆ. ಅದೂ ಕಾಮುಕತೆಯೆನ್ನೋಣವೇ? ನೈಸರ್ಗಿಕವಾದ ಅಂಗಾಂಗಳ ಬಗೆಗೆ ಕಥನವೇ ಆಭಾಸ, ಕಾಮುಕತ್ವ, ದೋಷಪೂರ್ಣ ಎನ್ನುವುದಾದರೆ, ಅಂತಹ ಅಂಗಗಳಾದ ಶಿಶ್ನೋಪ ಸ್ಠಾದಿಗಳನ್ನೆಲ್ಲ ಸೃಷ್ಟಿಮಾಡಿದ ಬ್ರಹ್ಮಾದಿಗಳ-ಸೃಷ್ಟಿಕರ್ತರ ಸ್ಥಾನಮಾನಗಳಾದರೂ ಏನು? ಸೃಷ್ಟಿಕಾರ್ಯದಲ್ಲಿ ಸ್ವತಂತ್ರರಾದ ಅವರು ಇಷ್ಟಾರು ಕೋಟಿಕೋಟ್ಯಂತರ ಜೀವಿಗೂ ಆಯಾ ಅಂಗಗಳನ್ನು ಸೃಷ್ಟಿಸದೆಯೂ ಇರಬಹುದಿತ್ತಲ್ಲವೇ?

\section*{ಪರಮೈಕಾಂತಿಗಳೂ ವಿರಕ್ತರೂ ಆದ ಆಚಾರ್ಯರಿಗೆ  ಸ್ತ್ರೀಪುರುಷ ಚಿಹ್ನೆಗಳ ಮೇಲೆ ತ್ಯಾಜ್ಯತಾಬುದ್ಧಿಯು ಇರುವುದಾದರೆ ಆದರ್ಶವೆ?}

ಪರಮೈಕಾಂತಿಗಳೂ ಪರಮವಿರಕ್ತರೂ ಆದ ಆಚಾರ್ಯರು ಲೋಕಕ್ಕೆ ವಿರಾಗವನ್ನು ಉಪದೇಶಿಸುವಾಗ, ಅವರ ಮನಸ್ಸಿನಲ್ಲಿ ಲೋಕದಲ್ಲಿಯ ಸ್ತ್ರೀಪುರುಷ ಚಿಹ್ನೆಗಳ ಮೇಲೆಯೇ ತ್ಯಾಜ್ಯತಾಬುದ್ಧಿಯಿದೆಯೆಂದು ಅವರ ಆಶಯವೆಂದು ಹೇಳಬಹುದೇ? ಅಂತಹ ಆಶಯವಾದರೆ ಅದು ಆದರ್ಶಕ್ಕೆ ಹೊಂದುವಂಥದಲ್ಲ್. ಜಗದಾದಿ ದಂಪತಿಗಳಿಂದ ಹೊರಟು ಬೆಳೆಯುತ್ತಿರುವ ಸೃಷ್ಟಿಯಲ್ಲಿ, ಎಲ್ಲಾ ಪದಾರ್ಥಗಳೂ ಜಗತ್ತಿನ ಬೆಳವಣಿಗೆಗೆ ಬೇಕಾದ ಲಕ್ಷಣಗಳೋಡನೆ ಹುಟ್ಟಿಕೊಳ್ಳುತ್ತಿರುತ್ತವೆ. ಒಂದು ವೃಕ್ಷದ ಪರಿಪೂರ್ಣವಾದ ಬೆಳವಣೀಗೆಯೂ ಸಹ (ಅದರ ಅಂಗವಾದ ಯಾವುದೇ ಭಾಗವನ್ನೂ ಬಿಡದೆ ಎಲ್ಲವೂ) ಅದರ ಮೂಲವಾದ ಬಿಜಾವಸ್ಥೆಗೆ ಬಂದು ಸೇರುವುದಕ್ಕಾಗಿಯೇ ಹೊರತು, ಇದು ಇರಲೆಂದು ಯಾವುದೊಂದನ್ನೂ ಅಂಟಿಸುವುದಕ್ಕಾಗಿ ಅಲ್ಲ ಎಂಬುದು ನಿಸರ್ಗದ ಮರ್ಮ ಎಂದರಿತು, ಬಹಿರಂಗದಿಂದ ಅಂತರಂಗಕ್ಕೂ ಅಂತರಂಗದಿಂದ ಬಹಿರಂಗಕ್ಕೂ ಹೊಂದಿಕೆಯನ್ನು ತರುವ ಒಂದು ಸೋಪಾನವನ್ನೇರ್ಪಡಿಸುವಂತಹ ಕವಿಗಳಾಗಿ ನಿಂತು, ಸೃಷ್ಟಿಯು ಹಾಕಿದ ಈಶ್ವರ ಸಂಕಲ್ಪಾಯುತ್ತವಾಗಿ ಬಂದ ಗುರುತನ್ನು ಅಂತೆಯೇ ವರ್ಣಿಸಿದ್ದಾರೇಪ್ಪ!

ನಮ್ಮ ನಮ್ಮ ಮನಸ್ಸುಗಳು ಕಾಮವಾಸನೆಯನ್ನೇ ಪ್ರಧಾನವಾಗಿಟ್ಟುಕೊಂಡಿದ್ದ ಕಾಲದೇಶಗಳಲ್ಲಿ, ಹಾಗೆ ಭಾವಿಸುವಂತಾಗುವುದು ನಮ್ಮ ಕಡೆಯ ದೋಷವಾದೀತೇ ಹೊರತು ಕವಿಗಳದಲ್ಲವೆಂಬುದನ್ನು ವಿಮರ್ಶೆಯಿಂದ ನಾವು ಮನವರಿಕೆ ಮಾಡಿಕೊಳ್ಳಬೇಕು.

\section*{ಆಯಾ ವಸ್ತುಧರ್ಮವನ್ನರಿತು ಸರಿಯಾದ ಸ್ಠಾನಮಾನಗಳೊಡನೆ ಉಪಯೋಗಿಸಿಕೊಳ್ಳಬೇಕು}

ಆಯಾಧರ್ಮಗಳನ್ನೂ ಸರಿಯಾದ ಸ್ಥಾನಮಾನಗಳೊಡನೇ ಉಪಯೋಗಿಸಿಕೊಳ್ಳದಿರುವುದೂ ಸೃಷ್ಟಿಗೆ ಅಪಚಾರವೇ ಆದೀತಲ್ಲವೆ? ಕಣ್ಣು ನೋಡುವುದಕ್ಕಾಗಿಯಲ್ಲವೇ ಇರುವುದು? ಆದರೆ ಅದನ್ನು ದುರುಪಯೋಗ ಮಾಡದೇ, ಸದ್ವಿಷಯದಲ್ಲಿ ಬಳಸಿದ್ದೇ ಆದರೆ ಅದು ದೈವೇಚ್ಛೆಯ ಅತಿಕ್ರಮವಾಗಲಾರದು. `ಕೃಷ್ಣಂ ಲೋಕಯಲೋಚನದ್ವಯ!' ಹೀಗೆ ಎಲ್ಲವೂ ಸತ್ಯಕ್ಕಾಗಿ. ಅದನ್ನು ಮರೆತರೆ ಎಲ್ಲವೂ ತಪ್ಪೇ. ಆದ್ದರಿಂದ ಎಲ್ಲದರಲ್ಲೂ ಅಪ್ರಮಾದ ಮುಖ್ಯ.

(ಅಪ್ರಮಾದ ಬಿಟ್ಟರೆ ಪ್ರಮಾದ, ಅದೇಮೃತ್ಯು. ಇದನ್ನೇ ಸನತ್ಸುಜಾತರು ಹೇಳುವಾಗ-`ಪ್ರಮಾದಂ ವೈ ಮೃತ್ಯುಮಹಂ\label{126} ಬ್ರವೀಮಿ ಸದಾಽಪ್ರಮಾದಮಮೃತತ್ವಂ ಬ್ರವೀಮಿ' ಎಂದರು. ಈ ಮೃತ್ಯುವಿನಿಂದ ಪಾರುಮಾಡಲೋಸುಗವೇ ಋಷಿಗಳ ಎಚ್ಚರಿಕೆಯ ವಾಣಿಗಳು. ನಿಸರ್ಗದಲ್ಲಿ ಯಾವ ತಪ್ಪೂ ಇಲ್ಲ. ಎಲ್ಲಿ ಯಾವ ವರ್ತನೆಯು ಬಿಷಪರಿಣಾಮಿಯೋ ಅದನ್ನು ಜ್ಞಾಪಿಸಿದರು ಋಷಿಗಳು. ಅವರ ಅಭಿಪ್ರಾಯಗಳನ್ನು ಗ್ರಹಿಸದೇ ನಾವೇ ಏನೇನೋ ಅಂದುಕೊಂಡುಬಿಟ್ಟರೆ, ಅದೆಲ್ಲಾ ಶಾಸ್ತ್ರಗಳ ಆಶಯವಲ್ಲ.)

\section*{ಮೂಲದೊಡನೆ ಸಂಬದ್ಧವಾದಾಗ ಯಾವ ವರ್ಣನೆಯೂ ಆಭಾಸವಾಗುವುದಿಲ್ಲ}

ಸಂನ್ಯಾಸಿಗಳಿಗೂ ಉಪಾಸ್ಯವಾದ ದೈವವು ಲಕ್ಷ್ಮೀನಾರಾಯಣರೇ ಆಗಿರುವಾಗ, ತಾಯಿಯಾದ ಮಹಾಲಕ್ಶ್ಮಿಯಲ್ಲೂ ಸ್ತ್ರೀ ಚಿಹ್ನೆಯಿದೆಯಲ್ಲ! ಎಲ್ಲವನ್ನೂ ತ್ಯಾಗ ಮಾಡಿದ ಅವರಿಗೂ ಅವರೇ ದೈವವಾಗಬೇಕೆ? ಅಂದರೆ, ಏನು ಹೇಳಬೇಕು? ಆದ್ದರಿಂ ಕಚಕುಚವರ್ಣನೆ ಮಾತ್ರದಿಂದಲೇ ದೂರ ಸರಿಯಬೇಕಾಗಿಲ್ಲ. ಉದಾಹರಣೆಗೆ ಮನುಷ್ಯನ ಅಂಗೋಪಾಂಗಗಳನ್ನೆಲ್ಲಾ ಬೇರ್ಪಡಿಸಿ ನೋಡಿದಾಗ ಎಷ್ಟು ಆಭಾಸವಾಗಿ ಕಾಣುವುದು? ನೋಡಿ. ಅದೇ ಅಂಗಾಂಗಳಷ್ಟೂ ಸಜೀವವಾಗಿ ಕಂಡಾಗ, ಸುಂದರಾಂಗನೆಂಬ ವರ್ಣನೆಯು ಹುಟ್ಟಿಕೊಳ್ಳುತ್ತದೆ. ಅಲ್ಲಿ ಆಭಾಸವು ನೆನಪಿಗೂ ಬರುವುದಿಲ್ಲ. ಅದರಂಟೆ ಇದೆಲ್ಲವನ್ನೂ ಸಜೀವವಾಗಿ, ಅಂದರೆ ಅದರ ಬೇರಿನೊಡನೆ ತೆಗೆದುಕೊಂಡಾಗ ಯಾವ ವಸ್ತುವೇ ಆಗಲೀ, ಯಾವ ವರ್ಣನೆಯೇ ಆಗಲೀ ಆಭಾಸವಾಗಲಾರದು. ಅದಕ್ಕಾಗಿ ಇವೆಲ್ಲದರ ಬೇರು ಯಾವುದು? ಎಂಬುದನ್ನು ಮೊದಲು ಕಂಡು ಹಿಡಿಯಬೇಕು. ಬೇರಿನೊಡನೆ ತೆಗೆದುಕೊಂಡ ಗಿಡವು ತಾನೇ ಸಜೀವವಾಗಿ ಉಳಿಯುವುದು. ಈ ದಾರಿಯೊಂದನ್ನು ಹಿಡಿದರೆ ಎಲ್ಲವೂ ಸುಗಮವಾಗುವುದು.

\section*{ಆರ್ಷವಾಣಿಯ ಯಥಾರ್ಥತೆ ಅರಿವಾಗಿದ್ದರೆ, ಲೋಕಕ್ಕದು ಉಪಯುಕ್ತವೆನಿಸಿದರೆ ಅದರ ಬಗ್ಗೆ ಜೀವಿಗಳ ಕಣ್ತೆರೆಸುವ ಕಾರ್ಯ ಕೈಗೊಳ್ಳಬೇಕು}

ಹೀಗೆ ಹೊರಟು, ಮಹಾಭಾರತದಲ್ಲೇ ಆಗಲೀ, ವೇದಗಳಲ್ಲೇ ಆಗಲೀ, ವೇದವೆನ್ನುವ ಭಾಗ ಯಾವುದಿದೆ? ಅದರ ಸಾರ ಯಾವುದು? ಮೂಲವೆಲ್ಲಿದೆ? ಎಂದು ಶುದ್ಧಬುದ್ಧಿಯಿಂದ ಮಥನಮಾಡಿ ತೆಗೆದುಕೊಳ್ಳಬೇಕು. ಕಣ್ಣೊಂದು (ದೃಷ್ಟಿಯು) ಚೆನ್ನಾಗಿರುವಾಗಲೇ ಈ ಬಗೆಯ ಕೆಲಸಗಳನ್ನು ಮಾಡಬೇಕು. ಅಲ್ಲಿ ಉದಾಸೀನವು ಸಲ್ಲದು. ಒಂದು ವಸ್ತುವು ಹೋಗಿಯೇ ಹೋಗಲೆಂದು ಮನಸ್ಸಿಗೆ ತೀರ್ಮಾನವಾಗಿ ಬಿಟ್ಟರೆ ಅಲ್ಲಿಂದ ಮುಂದೆ ವೈದ್ಯ, ಚಿಕಿತ್ಸೆ ಯಾವುದೂ ಬೇಡ. ಆ ವಸ್ತುವು ಬದುಕಬೇಕೆಂಅ ಆಸೆಯಿದ್ದಾಗ ಮಾತ್ರ ಅದನ್ನುಳಿಸಿಕೊಳ್ಳಲು ವೈದ್ಯ, ಚಿಕಿತ್ಸೆ ಎಲ್ಲಾ ಬೇಕೇ ಬೇಕು. ಅದರಂತೆ ಆರ್ಷವಾಣಿಗಳೆಲ್ಲಾ ಸಜೀವವಾಗಿರುವ ಅಂಶ ಎಲ್ಲರಿಗೂ ಅರ್ಥವಾಗಬೇಕೆಂದೂ, ಅದರಿಂದ ಜೀವಕೋಟಿಗಳೆಲ್ಲಾ ಬದುಕಬೇಕೆಂದೂ ಆಸೆಯೇನಾದರೂ ಇದ್ದರೆ, ಅದರಲ್ಲಿ ನಾವು ಯಥಾರ್ಥತೆಯನ್ನು ಕಂಡಿದ್ದರೆ, ಆ ಬಗ್ಗೆ ಜೀವಿಗಳಿಗೆ ಕಣ್ತೆರೆಯಿಸುವ ಚಿಕಿತ್ಸಾ ಕಾರ್ಯವನ್ನು ಕೈಕೊಳ್ಳಲೇಬೇಕು.

\section*{ಸಂಪ್ರದಾಯ ಹಾಗೂ ಆಧುನಿಕವಿಜ್ಞಾನಗಳ ಮಧ್ಯೇ ಯಾವುದನ್ನವಲಂಬಿಸಬೇಕೆಂಬ ಬಗ್ಗೆ ಸಂದಿಗ್ಧ ಸ್ಥಿತಿ}

ನಾವು ಸುಮಾರು ಕಾಲಗಳಿಂದ ತಲೆಮಾರು ತಲೆಮಾರಾಗಿ ಹೇಳಿಕೊಂಡು ಬಂದಿರುವ ಸಂಪ್ರದಾಯವೆಲ್ಲವೂ ಬರುತ್ತಲೇ ಇವೆಯಾದರೂ, ಆರ್ಷವಾಣಿಯ ಸಜೀವತೆ ಯಾರ ಮನಸ್ಸಿಗೂ ಬರಲಿಲ್ಲವೇಕೆ? ಋಷಿ ಸಾಹಿತ್ಯಗಳನ್ನೆಲ್ಲಾ ಕಂಡು ಮೃತಭಾಷೆಯೆನ್ನುವ ಕೂಗು ದೇಶದಲ್ಲಿ ಹೇಗೆ ಹುಟ್ಟಿಕೊಂಡಿತು? ಅವರ ಸಾಹಿತ್ಯಗಳು ಅಮರವಾಣಿಯೇ ಆಗಿರುವುದು ಈ ಸಂಪ್ರದಾಯಗಳಿಂದ ಲೋಕಕ್ಕೇಕೆ ಅರ್ಥವಾಗಲಿಲ್ಲ? ಇದರಲ್ಲಿ ಎಲ್ಲಿ ಐಬಾಗಿದೆ? ಎಂದೆಲ್ಲಾ ನಾವೇ ಪ್ರಶ್ನೆ ಹಾಕಿಕೊಳ್ಳಬೇಕು. ಈಗಿನ ಸನ್ನಿವೇಶ ಹೇಗಾಗಿ ಬಂದು ನಿಂತಿದೆಯೆಂದರೆ, `ಹಿಂದಕ್ಕೆ ಹೋದರೆ ಭಾರೀ ಹಳ್ಳ, ಮುಂದಕ್ಕೆ ಹೆಜ್ಜೆಯಿಟ್ಟರೆ ಹುಲಿ'  ಎಂದಾಗಿದೆ. ಪೂರ್ವೀಕರು ಹೋದ ಮಾರ್ಗವೆಂದು ಹಿಂತಿರುಗಿದರೆ ಅಲ್ಲಿ ಮಾರ್ಗವೇ ಇಲ್ಲ. ಹಾಳುಬಾವಿಯಾಗಿ ನಿಂತಿದೆ. ಅತ್ತ ಹೋದವರೇ ಯಾರೂ ಮೇಲಕ್ಕೆ ಹತ್ತಲಾರದೇ ಕಣ್ಣೀರು ಬಿಡುವ ಸ್ಥಿತಿ ಬಿಡುವ ಸಮಾಜವೂ, ಇನ್ನೂ ಏನೇನೋ ಪರಿಸ್ಥಿತಿಗಳೂ ಬಾಯ್ತೆರೆದು ಕಾಯುತ್ತಿವೆ. ಒಂದೊಂದು ಹೆಜ್ಜೆಗೂ ಬೇಕಾದಷ್ಟು ಪ್ರಶ್ನೆಗಳೂ ಸಮಸ್ಯೆಗಳೂ ಒಂದರಮೆಲೊಂದಾಗಿ ನುಗ್ಗುತ್ತಿವೆ. ಹೀಗಿರುವಾಗ ತ್ರಿಶಂಕುಸ್ವರ್ಗದಲ್ಲಿ ನಿಂತು ಮುಂದಿನ ಹೆಜ್ಜೆಗಾಗಿ ಬುದ್ಧಿಯನ್ನೆಲ್ಲಾ ಖರ್ಚುಮಾಡಿ ಆಲೋಚಿಸಬೇಕಾಗಿದೆಪ್ಪಾ!

\section*{ಲೋಕದ ಹಾಗೂ ನಮ್ಮ ಮುಂದಿನ ಪ್ರಜೆಗಳ ದೃಷ್ಟಿಯಿಂದಲೂ ಕರ್ತವ್ಯವಿದೆ}

ಆದ್ದರಿಂದ ಈ ಆಲೋಚನೆಗಾದರೂ ಸಹಕಾರಿಯಾಗುವಂತೆ ಕರನಕಳೇಬರಗಳು ಗಟ್ಟಿಯಾಗಿರುವಾಗ ಯಥಾಶಕ್ತಿ ಕೆಲಸಮಾಡಬೇಕಾಗಿದೆ. ನೀವು ಕೇಳಬಹುದು `ನಮಗೆ ನೀನೇ ಒಳಗೆ ನೆಮ್ಮದಿಯನ್ನು ಕೊಟ್ಟಿದ್ದೀಯೆಲ್ಲಾಪ್ಪಾ! ಇದೊಂದು ದೃಧವಾದರೆ ಸಾಕು. ನಾವಿನ್ನೇಕೆ ಬೇರೆ ಕೆಲಸಮಾಡಬೇಕು?- ಎಂದು. ಆಯಿತು. ಎಲ್ಲಾ ಕೆಲಸವನ್ನೂ ಬಿಟ್ಟು ಸುಮ್ಮನೆ ಕುಳಿತಿರಿ ಅನ್ನಿ. ಅಷ್ಟರಿಂದಲೇ ಪೂರ್ತಿ ನೆಮ್ಮದಿಯಾಯಿತೆಂದು ಧೈರ್ಯವೇ? ಅಥವಾ ಒಂದು ಬೇಳೆ ನಿಮಗೆ ನೆಮ್ಮದಿ ದೊರಕಿಯೇ ಬಿಟ್ಟಿತು ಅನ್ನಿ, ಸಂತೋಷ ನಿಮ್ಮ ಮುಂದಿನ ಪ್ರಜೆಗಳ ಚಿಂತೆಬೇಡವೇ? ನಿಮ್ಮ ಮುಂದಿನ ಪ್ರಜೆಗಳು ಅನಾಥರಂತೆ ದುರ್ಗತಿಯನ್ನು ಹೊಂದುವುದು, ನಿಮಗೆ ನೆಮ್ಮದಿಯನ್ನು ಕೊಡುವುದೇ? ಇದೆಲ್ಲಾ ಯೋಚಿಸಬೇಕಾದ ವಿಷಯ.

\section*{ಸೃಷ್ಟಿಯ ನಡೆಯಾಗಿ ಜ್ಞಾನವಂಶವು, ವಿದ್ಯಾಸಂತಾನವು ಹರಿಗೆಡಹಿಲ್ಲದೇ ಬೆಳೆಯುವುದೆಂಬ ಆಶ್ವಾಸನೆ}

ನೀವುಗಳೆಲ್ಲಾ ಕೆಲಸ ಮಾಡದಿದ್ದರೆ ಅವರೆಲ್ಲಾ ಹೋಗಿಯೇ ಬಿಡುತ್ತಾರೆಂದೇನೂ ಅರ್ಥವಲ್ಲ. ಉದಾಹರಣೆಗೆ ಹುಟ್ಟಿದ ಮಗುವಿಗೆ ಹಸಿವು ಪೂರೈಸಲು ನೀವೇನೂ ಕಿಂಚಿತ್ತೂ ಮಾಡದೇ ಇದ್ದರೂ, ನಿಮ್ಮನ್ನೆದುರು ನೋಡದೇ ತಾಯಿಯಸ್ತನಗಳಿಂದ ತನಗೆ ತಾನೇ ಕ್ಷೀರಧಾರೆಯು ದ್ರವಿಸಿ ಸ್ರವಿಸಲಾರಂಭಿಸುತ್ತದೆ. ಅದರಿಂದದು ಬದುಕುತ್ತದೆ. ಹಾಗೆಯೇ ಪ್ರತಿಯೊಂದು ಸಂತಾನಪರಂಪರೆಗೂ ಆಯಾ ಸೃಷ್ಟಿಯೇ ತನ್ನ ಗುಟುಕನ್ನು ಕೊಟ್ಟು ಬದುಕಿಸುತ್ತದೆ. ಈ ಪಂಡಿತರನ್ನು ಕಾದು ಏನೂ ತೆಗೆದುಕೊಳ್ಳಲಾಗುವುದಿಲ್ಲ. ಇವರನ್ನು ಕಾಯಲು ಹೊರಟರೆ, ಆ ವೇಳೆಗೆ ಜೀವವೇ ಹಾರಿ ಹೋಗಿರುತ್ತೆ. ಆದ್ದರಿಂದ ನೀವು ಇಂದು ಈ ಕೆಲಸವನ್ನು ನಡೆಸಿ ಅವರನ್ನೆಲ್ಲಾ ಬದುಕಿಸಬೇಕೆಂದು ಹೇಳುತ್ತಿಲ್ಲ. ನಿಮ್ಮ ಅಂತರಂಗದಲ್ಲಿ ಯಾವ ಪಂಡವಿರುವುದೋ (ಗುರುಕೊಟ್ಟ ಜ್ಞಾನಶಕ್ತಿ) ಅದನ್ನು ಮುಂದಿನ ಮಕ್ಕಳಿಗೂ ಕೊಡುವಂತಾದರೆ, ಅಲ್ಲಿಗೆ ಒಂದು ನೆಮ್ಮದಿ. ಪ್ರಕೃತಿಗೂ ನೆಮ್ಮದಿ, ಆತ್ಮನಿಗೂ ನೆಮ್ಮದಿ, ಪರಮಾತ್ಮನಿಗೂ ನೆಮ್ಮದಿ. ಇಷ್ಟು ಅರ್ಥದಿಂದ ಹೇಳಿದ ಮಾತು. ಅವರವರಿಗೆ ದೊರಕಿದ ಐಶ್ವರ್ಯವೇನಿದೆಯೋ (ಈಶ್ವರನ ಸಂಪತ್ತು) ನಿತ್ಯೈಶ್ವರ್ಯ ನಿರುಪಾಧಿಕೈಶ್ವರ್ಯ, ಅದು ಅವನ ಮುಂದಿನ ಪ್ರಜೆಗಳಿಗೂ ಬರುವಂತಾಗಲೀಪ್ಪ! ಅವನ ಜ್ಞಾನವಂಶವು- ವಿದ್ಯಾಸಂತಾನವು ಹರಿಗೆಡಹಿಲ್ಲದೇ ಬೆಳೆಯಲೀಪ್ಪ! ಎಂಬ ಆಶಯವನ್ನು ತುಂಬಿಕೊಂಡು. ಅವನು ನನ್ನಲ್ಲಿಟ್ಟ ನ್ಯಾಸವನ್ನು ನಿಮಗೆ ಕೊಡುತ್ತಿದ್ದೇನೆಪ್ಪ.

\section*{ಒಳಗೆ ಬೆಳಗುವ ಪರಮಗುರುವಿನ ಗುರುತ್ವವನ್ನರ್ಥ ಮಾಡಿಸುವ ಕರ್ತವ್ಯವಿದೆ}

ನೋಡಿಪ್ಪ! ನನ್ನ ಪ್ರಭುವು ಎಂದೆಂದಿಗೂ ಪರಮಗುರುವಾಗಿಯೇ ಇರುವನಪ್ಪ! `ಸ ಪೂರ್ವೇಷಾಮಪಿ\label{129} ಗುರುಃ' ಹೀಗಿದ್ದರೂ ಅವನಿಗೆ ಮುಂದಿನ ಪ್ರಜೆಗಳ ಬಾಯಲ್ಲಿ `ಗುರು' ಎಂಬ ಶಬ್ದಕೇಳಿ ಸಂತೋಷಪಡುವ ಆಸೆಯಿದೆ ಎಂದರೆ ತಪ್ಪೇನಿಲ್ಲ. ಏಕೆಂದರೆ ಒಬ್ಬ ಗೃಹಿಣೀಯು ಗರ್ಭವತಿಯಾಗಿ ಹೊಟೇಯಲ್ಲಿ ಮಗುವೊಂದು ಹುಟ್ಟಿಕೊಂಡ ಕ್ಷಣಕ್ಕೇ, ಅವಳು ಅಮ್ಮನಾಗಿಬಿಟ್ಟಳು. ಆದರೂ ಸಹ ಅವಳಿಗೆ ಮಗು ಹುಟ್ಟಿದ ಮೇಲೆ `ಅಮ್ಮ' ಎಂದೆನಿಸಿಕೊಳ್ಳುವ ಆಸೆಯಿಂಡ ಅದಕ್ಕೆ `ಅಮ್ಮ ಅಮ್ಮ' ಎಂದು ಉರುಹಾಕಿಸಿ ಅದಕ್ಕೆ ಆ ಶಬ್ದವನ್ನು ಕಲಿಸಿ ಅದರ ಬಾಯಿಂದ `ಅಮ್ಮ' ಎಂದು ಕರೆಸಿಕೊಂಡು ಹಿಗ್ಗುತ್ತಾಳೆಯಲ್ಲವೇ? ಹಾಗೆ ನನ್ನ ಪ್ರಭುವಿನ ಆಸೆಯನ್ನು ಗುರು ಎನಿಸಿಕೊಳ್ಳುವ ಆಸೆಯನ್ನು ಪೂರ್ತಿಮಾಡಬೇಕಾದರೆ, ಲೋಕಕ್ಕೆ ಅವನ ಗುರುತ್ವವನ್ನರ್ಥಮಾಡಿಕೊಡುವ ಈ ಕೆಲಸವನ್ನು ಲೋಕದಲ್ಲಿ ಮಾಡಿಕೊಂಡು ಬರಬೇಕಪ್ಪ!

\section*{ಪ್ರಾಮಾಣಿಕವಾಗಿ ವಿದ್ಯಾವಂಶವನ್ನು ಬೆಳೆಸುವ ಮಕ್ಕಳಾಗಬೇಕು}

ಇದರ ಜೊತೆಗೆ ಮತ್ತೊಂದು ವಿಷಯವನ್ನೂ ಜ್ಞಾಪಿಸುತ್ತೇನೆಪ್ಪ! ನಾವು ಎಲ್ಲಿಂದೆಲ್ಲಿಯವರೆಗೆ ಹೋದರೂ, ಹೇಗೆ ಹೇಗೆ ಜೀವನ ಮಾಡಿದರೂ, ಪ್ರಾಮಾಣಿಕತೆಯನ್ನು ಪಡೆಯುವುದಕ್ಕೆ ಶ್ರಮಿಸಿ, ಅದನ್ನು ಪಡೆದುಕೊಂಡು ಆ ಪ್ರಾಮಾಣಿಕತೆಯನ್ನುಳಿಸಿಕೊಂಡೇ ಬಾಳಬೇಕು. ಜೀವನಕ್ಕೆಲ್ಲಾ ಒಂದು ಪುಸ್ತಕವೂ ಒಬ್ಬ ಗುರುವೂ ಒಬ್ಬನೇ ಶಿಷ್ಯನೂ ಸಾಕು. ಸಾವಿರದೆಂಟು ಪುಸ್ತಕವಾಗಲೀ ಗುರುವಾಗಲೀ ಶಿಷ್ಯನಾಗಲೀ ಬೇಕಿಲ್ಲ. ನಂಬರುವಾರಾಗಿ ಶಿಷ್ಯರ ಪಟ್ಟಿಯನ್ನೇನೂ ಬೆಳೆಸಬೇಕಾಗಿಲ್ಲ. ಸುಳ್ಳಂಪುಳ್ಳೆಗಳನ್ನು ನೂರಾರು ಕಟ್ಟಿಕೊಂಡು ಮಾಡುವುದಾದರೂ ಏನು? (`ಒಂದು ಕಣ್ಣು ಕಣ್ಣಲ್ಲ, ಒಬ್ಬ ಮಗ ಮಗನಲ್ಲ' ಎಂಬ ಗಾದೆಯಮಾತಿದ್ದರೂ ಅಯೋಗ್ಯರಾದ ನೂರಾರು ಮಕ್ಕಳಿಗಿಂತ ವಂಶದ ಹೆಸರನ್ನು ಮುಂದಕ್ಕೆ ತರುವ ಒಬ್ಬ ಮಗನೇ ಆದರೂ ಸಾಕು ಎನ್ನುವುದೂ ಉಂಟು.) ಹಾಗೆ ಪ್ರಾಮಾಣಿಕವಾಗಿ ವಿದ್ಯಾವಂಶವನ್ನು ಬೆಳೆಸುವ ಮಕ್ಕಳಾಗಬೇಕಾಪ್ಪ!

\section*{ಜೀವನದಲ್ಲಿ ಸುಳ್ಳಿಗೆ ಅವಕಾಶ ಕೊಡಬಾರದು}

ಪ್ರಾಮಾಣಿಕತೆಯನ್ನು ಬಿಟ್ಟು ಒಂದು ಸುಳ್ಳು ಹೇಳಿದರೂ ಅದು ಸುಳ್ಳಿನ ಪರಂಪರೆಯನ್ನು ಬೆಳೆಸುತ್ತೆ. ಮೊದಲೆನೇ ಸುಳ್ಳನ್ನು ಮರೆಮಾಡಿ ಸತ್ಯವೆಂದು ಮಾಡಲು ಒಂದರ ಮೇಲೊಂದಾಗಿ ಸಾವಿರಾರು ಸುಳ್ಳುಹುಟ್ಟಿಕೊಳ್ಳಬೇಕಾಗುತ್ತೆ. ಉದಾಹರಣೆಗೆ- ಒಬ್ಬ ತಂದೆ, ಉಪನೀತನಾದ ಮಗನನ್ನು ಕರೆದುಕೇಳಿದ- "ಸಂಧ್ಯಾವಂದನೆ ಮಾಡಿದೆಯೇನೋ" ಎಂದು. ಮಗ ಹೇಳಿದ- ಮಾಡಿದೆ ಎಂದು. 

ಎಲ್ಲಿಮಾಡಿದೆಯೋ? `ಕೆರೆಯಲ್ಲಿ' `ಕೆರೆಯಲ್ಲಿ ನೀರೇ ಇಲ್ಲವಲ್ಲಾ!' `ನೀರಿಲ್ಲದೆ ಹೇಗೆ ಸಂಧ್ಯಾವಂಧನೆ ಮಾಡಿದೆಯೋ?' ಕೆರೆಯ ಹತ್ತಿರ ಯಾರೋ ನೀರು ತರುತ್ತಿದ್ದರು,' ಅದರಿಂದ ಸಂಧ್ಯಾವಂದನೆ ಮಾಡಿದೆ. `ಪಾತ್ರೆಯೇ ಇಲ್ಲವಲ್ಲೊ~' `ಅಲ್ಲೊಂದು ಎಲೆಯನ್ನು ತೆಗೆದುಕೊಂಡೆ,' ಅದಕ್ಕೆ ಅವರಿಂದ ನೀರು ಬಿಡಿಸಿಕೊಂಡೆ, ಮಾಡಿದೆ- ಎಂದ ಹೀಗೆ ಮಾಡದೇ ಇರುವ ಸಂಧ್ಯಾವಂದನೆಯನ್ನು ಸಮರ್ಥಿಸಿಕೊಳ್ಳಲು ಎಷ್ಟು ಸುಳ್ಳಾಯಿತು? ನೋಡಿ ಜೀವನದಲ್ಲಿ ಸುಳ್ಳಿಗೇ ಅವಕಾಶ ಕೊಡಬಾರದು. ಅದರಲ್ಲಿಯೂ ಗುರುವೆನಿಸಿಕೊಳ್ಳುವವನು ಸರ್ವಥಾ ಸುಳ್ಳಿಗೆ ಎಡೆಗೊಡಬಾರದು ಅವನೇ ಹಾಗಾಗಿಬಿಟ್ಟರೆ ಪರಮಪಾತಕವನ್ನೇ ಬೆಳೆಸಿದಂತಾಗಿ ಬಿಡುತ್ತೆ. ಎಚ್ಚರಿಕೆಯಿಂದಿರಬೇಕು.

\section*{ಸಹ ವೀರ್ಯಂ ಕರವಾವಹೈ' ಎಂಬ ಶಾಂತಿ ಪಾಠಕ್ಕೆ ವಿಷಯ}
\label{130b}

ಮನಸ್ಸಿನಲ್ಲೊಂದು ಅಭಿಪ್ರಾಯವೂ ಹುಟ್ಟಿಕೊಂಡರೆ, ಅದನ್ನು ವ್ಯಕ್ತಗೊಳಿಸಲು ಬಾಯಿ, ಕೈ ಮೊದಲಾದ ಅಂಗಗಳು ಬೇಕು. ಮನಸ್ಸಿಗೆ `ಬಾ' ಎಂದು ಕರೆಯಬೇಕೆನಿಸಿದಾಗ, `ಬಾ' ಎಂಬ ಶಬ್ದವನ್ನು ಮನಸ್ಸೇ ಹೇಳಲಿ! ಅಂದರೆ ಹೇಳುವ ಕೆಲಸ ಬಾಯಿನದು, ಮನಸ್ಸೇ `ಬಾ' ಎಂದು ತೋರಿಸಲಿ! ಎಂದರೆ ತೋರಿಸುವ ಕೆಲಸ ಕೈಯಿನದು. ಆದ್ದರಿಂದ ಒಳಗಿನ ಅಭಿಪ್ರಾಯವನ್ನು ಹೊರತರಲು ಬಾಯಿ ಮತ್ತು ಕೈಗಳು ಬೇಕು. ಅದರಂತೆ ಅಂತರಂಗದಲ್ಲಿ ಪ್ರಾಮಾಣಿಕತೆಯೊಡನೆ ತೀರ್ಮಾನಕ್ಕೆ ಬಂದ ವಿಷಯವನ್ನು , ಅವನ-ಪರಮಪುರುಷನ ಸಾಕ್ಷಿಯಾಗಿ ನಿಶ್ಚಯಕ್ಕೆ ಬಂದ ವಿಷಯವನ್ನು ಹೊರತರಲು ಹೊರ ವ್ಯಾಪಾರ ಮತ್ತು ಹೊರ ಅಂಗಗಳೂ ಬೇಕಾಗುವುದಪ್ಪ! `ಸತ್ಯಂ ಸತ್ಯಂ ಪುನಸ್ಸತ್ಯಂ'\label{130a} ಎಂದು ಒಳಗೆ ಅನಿಸಿದ್ದನ್ನು `ಉದ್ಧ್ಯತ್ಯ ಭುಜಮುಚ್ಯತೇ'\label{130} ಎನ್ನುವುದಕ್ಕೆ ಭುಜವು ಹೇಗೆ ಬೇಕೊ ಹಾಗೆ. ಒಂದು ವೇಳೆ ಅಂತರಂಗವೇ ಅಪ್ರಾಮಾಣಿಕವಾಗಿಬಿಟ್ಟರೆ, ಇನ್ನು ಬಾಯಿ ಕೈ ತೋಳು ಇದೆಲ್ಲಾ ಯಾವ ಉಪಯೋಗಕ್ಕೂ ಬರುವುದಿಲ್ಲ. ಅಂತಹವರನ್ನು ಕುರಿತೇ ಇಲ್ಲಿಯವರೆಗೂ ಬೇಕಾದಷ್ಟು ಮಾತಾಡಿಯಾಯಿತು. ಅಂತಃಕರಣಕ್ಕೆ ಬಾಯಿ ಕೈ ಎಲ್ಲಾ ಶಿಷ್ಯನಾಗಿ ಶ್ಲಿಷ್ಟವಾಗಿದ್ದು ಮುಂದುವರಿದರೆ ಪ್ರಾಮಾಣೀಕತೆ, ಗುರುವೇ ಪರಮಪುರುಷನಿಗೆ ಅಂತಃಕರಣವಾದರೆ, ಆ ಅಂತಃಕರಣಕ್ಕೆ- ಗುರುವಿಗೆ ಶಿಷ್ಯನಿರಬೇಕು. ಹಾಗೆ ಗುರುಶಿಷ್ಯರೀರ್ವರಲ್ಲೂ ಪ್ರಾಮಾಣಿಕತೆಯಿದ್ದಾಗ `ಸಹವೀರ್ಯಂ ಕರವಾವಹೈ, ತೇಜಸ್ವಿನಾವಧೀತಮಸ್ತು'  ಎಂಬ ಶಾಂತಿಪಾಠಕ್ಕೆ ವಿಷಯ. `ಆ ಪರಮಗುರುವಿನೊಡನೇ ಸೇರಿ ಶ್ಲಿಷ್ಟವಾಗಿ ನಿಮ್ಮ ಗುರು, ನಿಮ್ಮ ಗುರುವಿನೊಡನೇ ನೀವು ಶ್ಲಿಷ್ಟವಾಗಿ' ಹೀಗೆ ಒಂದು ಜಾಗದಲ್ಲಿ ವಿಷಯವೂ ಅದರ ಬೆಳವಣಿಗೆಯೂ ಸಿದ್ಧವಾಗುವುದಾದರೆ ಆಗಲಿ ಎಂದು, ಅಲ್ಲೇನೋ ಒಂದು ಪ್ಲಾನ್ ಹಾಕಿಕೊಂಡು, ಹಿಂದೆ ಹೇಳಿದಂತೆ ದೇವರ ಮಂದಾಸನಕ್ಕೆ ಬೇಕಾದ ಮರದ ಚೂರುಗಳನ್ನು ಸಂಗ್ರಹಿಸುವಂತೆ, ಈ ಕೆಲವಾರು ಪುಸ್ತಕಗಳನ್ನೂ ಸಂಗ್ರಹಿಸುತ್ತಿದ್ದೇನೆಪ್ಪ!

\section*{ಕಾಲ ವಿಳಂಬವಾಗದಿರಲೆಂದು ಈ ಮಾರ್ಗದರ್ಶನ}

ಇಷ್ಟೆಲ್ಲಾ ಮಾತಾಡಿ ಪುಸ್ತಕ ಸಂಗ್ರಹಣಕ್ಕೆ ಕೊಡುವ ಹಿನ್ನೆಲೆ, ನಿಮಗೆ ಒಂದು ಸೋಜಿಗವಾಗಿ ತೋರುವುದಾದರೂ, ನಿಮ್ಮಲ್ಲೂ ಈ ಧರ್ಮವು ಹುಟ್ಟಿಕೊಂಡಾಗ, ನೀವೂ ಹೀಗೆಯೇ ಮಾಡುವಂತಾಗುತ್ತೆ. ಖಂಡಿತವಾಗಿಯೂ ನೀವೂ ಹೀಗೆ ಮಾಡುತ್ತೀರಿ. ಆದರೆ ಅಷ್ಟುಕಾಲ ಕಾಯುವುದರೊಳಗೆ ಕಾಲ ದಾಟಿಹೋಗಿರುತ್ತೆ. ಆದ್ದರಿಂದ ಈಗಿನಿಂದಲೇ ನಾನು ಈ ಕೆಲಸವನ್ನು ಶುರು ಮಾಡಿರುತ್ತೇನೆ. ನನ್ನ ಆಶಯವನ್ನು ಯಥಾವತ್ತಾಗಿ ತೆಗೆದುಕೊಳ್ಳೀಪ್ಪ!

\section*{ಎಲ್ಲರ ಪ್ರಯೋಗಾನುಭವಗಳಿಗೆ ಎಲ್ಲ ವಿಷಯಗಳೂ ಲಭಿಸುವುದು ಸಾಧ್ಯವಿಲ್ಲ}

ನಮ್ಮ ಅಂತರಂಗದ ಅನುಭವವನ್ನೂ ಈ ಪುಸ್ತಕಗಳಲ್ಲಿರತಕ್ಕ ಶ್ರುತಿ ಸ್ಮೃತ್ಯಾದಿಗಳನ್ನೂ ಲೋಕದ ಮುಂದೆ ಇಟ್ಟರೂ, ಕೆಲವು ಅನುಭವಕ್ಕೆ ಬರಬಹುದು, ಇನ್ನು ಕೆಲವು ಅನುಭವಕ್ಕೆ ಬಾರದೆಯೂ ಉಳಿಯಬಹುದು. ಆದ್ದರಿಂದ ಪ್ರಯೋಗಾನುಭವಗಳಿಗೆ ವಿಷಯವೆಂದು ನಮ್ಮ ಅರಿವಿಗೆ ಬಾರದ ವಿಷಯಗಳನ್ನು ಹೇಗೆ ಇಟ್ಟುಕೊಂಡು ವ್ಯವಹರಿಸುವುದು? ಎಂಬ ಪ್ರಶ್ನೆ ಬರಬಹುದು ನಿಮಗೆ. ನೀವು ಇಟ್ಟುಕೊಳ್ಳುವ ಮತ್ತು ಬಿಡುವ ವಿಚಾರ ಬೇರೆ. ಋಷಿಗಳ ಅನ್ತರ್ವಾಣಿಗಳು ಹೇಗೆ ಅಂತರಾಳದಲ್ಲಿದ್ದ ವಿಷಯಗಳನ್ನು  ವ್ಯಕ್ತಪಡಿಸಿತೋ ಹಾಗೆ ಅತ್ತಕಡೆಗೆ ಅಭಿಮುಖ್ಯವನ್ನುಂಟು ಮಾಡುವ ಕಾರ್ಯವನ್ನು ಮಾಡಬೇಕು. ಪ್ರಯೋಗಾನುಭವಗಳಿಗೆ ಬಂದ ಮಾತ್ರಕ್ಕೆ ಎಲ್ಲರೂ ಅಂಥದನ್ನು ಇಟ್ಟುಕೊಳ್ಳುತ್ತಾರೆಂದೂ ಹೇಳಲಾಗುವುದಿಲ್ಲ. ರಾಕೆಟ್ಟು-ಆಕಾಶಸಂಚಾರ-ಭೂಪ್ರದಕ್ಷಿಣೆ-ಇತ್ಯಾದಿ ವಿಷಯಗಳು ಪ್ರಯೋಗಾನುಭವಗಳಿಗೆ ಬಂದಿದ್ದರೂ, ಅದೆಲ್ಲಾ ವೈಜ್ಞಾನಿಕ ಎಂದೇ ತಿಳಿದಿದ್ದರೂ, ಎಷ್ಟು ಜನರು ಮನೆಗೆ ತಂದಿಟ್ಟು ಕೊಂಡಿದ್ದಾರೆ? ಆ ವಿಷಯವೇ ಬೇರೆ.

\section*{ನಿಸರ್ಗವೇ ಸಹಜವಾದ ಪುಸ್ತಕ. ಅತ್ತ ಅಭಿಮುಖ್ಯವನ್ನುಂಟುಮಾಡಲು ಪುಸ್ತಕ ಸಹಕರಿಸಬಹುದು}

ಹಾಗೆ ಯೋಚಿಸಿದರೆ ಪುಸ್ತಕಗಳಿಂದಲೇ ಯಾವ ಅನುಭವವೂ ಬರುವುದೂ ಇಲ್ಲ. ಹಾಗೆ ಅನುಭವವನ್ನು ಕೊಡುವ ಪುಸ್ತಕವೇ ಬೇರೆ. ಪ್ರಪಂಚದಲ್ಲಿಯ ನಿಸರ್ಗಮೂಲದ ಆಶಯವನ್ನೆಲ್ಲಾ ಒಂದು ಗೂಡಿಸಿ ತನ್ನ ಕೈಯಲ್ಲಿಟ್ಟುಕೊಳ್ಳುವುದೇ  ಒಂದು ನಿಜವಾದ ಪುಸ್ತಕ. ನಿಜವಾದ ಅನುಭವಬೇಕೆನ್ನುವವರು ಅದನ್ನು ಸಂಗ್ರಹಿಸಿ ಕೊಳ್ಳಬೇಕು. ಆವಾಗ ತಾನೇ `ಅನು-ಭವ' ಎಂಬುದಕ್ಕೆ, ಆ ಆಕ್ಷರಗಳನ್ನು ಬಳಸುವುದಕ್ಕೆ ಒಂದು ವಿಷಯ. ಆದರೆ ಈ ಮೊದಲು ಹೇಳಿದಂತೆ ಅನುಭವವಾಗಬೇಕಾದ ವಿಷಯದ ಕಡೆಗೆ ಆ ವಿಷಯದ ಅನುಭವದ ದಿಕ್ಕಿಗೆ ಅಭಿಮುಖ್ಯವನ್ನುಂಟು ಮಾಡಲು, ನಾವು ಸಂಗ್ರಹಿಸಬೇಕಾದ ಮತ್ತು ಸಂಗ್ರಹಿಸಿದ ಪುಸ್ತಕಗಳು ಸಹಕರಿಸಬಹುದೆಂಬುದಷ್ಟೇ ನಮ್ಮ ಉದ್ದೇಶ. ನಮ್ಮ ಕರ್ತವ್ಯವು ಎಷ್ಟರಮಟ್ಟಿಗೆ ಇದೆಯೆಂದರೆ ಹೇಳತೀರದು.

\section*{ಆಶಯ ಮತ್ತು ಪ್ರಾಣಗಳೇ ಮೂರ್ತಿಯ ಜೀವ}

ನೋಡೀಪ್ಪ ನಾಟ್ಯ ಸರಸ್ವತೀ ವಿಗ್ರಹವನ್ನು ಅವಳ ಬಲಭಾಗವೆಲ್ಲಾ ಮೇಲಕ್ಕೆ ಎದ್ದು ದಿವಿಯನ್ನು ತೋರಿಸುತ್ತಿದೆ. ಬಲಗಾಲು, ಬಲತೋಳು, ಬಲಭುಜ, ಬಲಗೈ, ಬಲಗಣ್ಣು, ಬಲಹುಬ್ಬು ಎಲ್ಲವೂ ಮೇಲ್ಮುಖವಾಗಿದೆ. ಹಾಗೆಯೇ ಮತ್ತೊಂದು ಕಡೆ ಎಡಭಾಗವೆಲ್ಲವೂ ಕೆಳಮುಖವಾಗಿದ್ದು ಪ್ರವೃತ್ತಿ ಮಾರ್ಗಕ್ಕಿಳಿದು ಭುವಿಯನ್ನು ತೋರಿ ಶಿಲ್ಪಿಯು ಈ ರೀತಿಯಾಗಿ ವಿಗ್ರಹವನ್ನು ಕೆತ್ತಿ ಸಿದ್ಧಪಡಿಸಿದರೂ, `ಸ್ವಲ್ಪ ನಿಲ್ಲಿ. ಪ್ರಾಣಪ್ರತಿಷ್ಠೆಯಾಗಲಿ, ನಂತರ ಆರಾಧನೆಯನ್ನೂ ಉತ್ಸವವನ್ನೂ ನಡೆಸೋಣ,' ಎನ್ನುತ್ತಾರೆ. ಏನು ಕಾರಣ? ವಿಗ್ರಹವು ಬಾಹ್ಯಕ್ಕೇನೋ ವಿಷಯವಾಯಿತು. ಅಲ್ಲಿ ನಿಮ್ಮ ಅಂತರಂಗಕ್ಕೇನು ವಿಶಯ? ಆ ದೇವಾತಾ ಮೂರ್ತಿಯ ಆಶಯವೂ ಅದರ ಪ್ರಾಣವೂ ನಿಮ್ಮ ಅಂತರಂಗಕ್ಕೆ ವಿಷಯವಾಗಬೇಕು. ಆಶಯ ಮತ್ತು ಪ್ರಾಣಗಳು ತಾನೇ ಮೂರ್ತಿಯ ಜೀವ ಅವೆರಡನ್ನೂ ಕಳೆದುಕೊಂಡರೆ ಮೂರ್ತಿಯು ನಿರ್ಜೀವ ತಾನೇ.

\section*{ಅರ್ಚಕ ಪರಿಚಾರಕಾದಿಗಳಲ್ಲಿ ಸಂಭವಿಸಿರುವ ಜ್ಞಾನದ ಕೊರತೆ ಹಾಗೂ ಅದರ ಪರಿಣಾಮಗಳು}

ಅರ್ಚಕರಾಗಿ ಪರಿಚಾರಕರಾಗಿ ಅಲ್ಲೇ ನಿರಂತರ ಕೆಲಸ ಮಾಡುತ್ತಿರುವವರಿಗೆ ನಾವು ಪೂಜಿಸುತ್ತಿರುವ ದೇವತಾ ಮೂರ್ತಿಯು ತನ್ನ ಜೀವದೊಂದಿಗಿದೆಯೇ? ಎಂಬ ಪರಿಜ್ಞಾನ ಯಾರಿಗಿದೆ? ವಿಗ್ರಹದಲ್ಲಿ ಭಗವತ್ಕಳೆಯಿದೆಯೇ? ಭಗವಂತನ ಆಲಯವಾಗಿ ಇದನ್ನುಳಿಸಿಕೊಳ್ಳುವ ಬಗೆಯೇನು? ಇದಾವ ಪರಿಜ್ಞಾನವೂ ಇಲ್ಲದೇ, ಭಗವದ್ಧರ್ಮದ ಲೇಶವನ್ನೂ ಹೊಂದದೇ, ಧರ್ಮಘಾತುರೂ ದೈವದ್ರೋಹಿಗಳೂ ಆಗಿ ದೇವಾಲಯವನ್ನು ಹೊಟ್ಟೆಹೊರೆಯುವ ತಾಣವಾಗಿ ಮಾಡಿಕೊಂಡು, ರಾಷ್ಟ್ರಕಂಟಕರಾಗಿ ಮೆರೆಯುತ್ತಾ, ಮುಂದಿನ ಘಟ್ಟದಲ್ಲೇನು ವಿಪತ್ತು ಕಾದಿದೆ? ಎಂಬುದನ್ನೂ ಲಕ್ಷಿಸದೇ, ತಾವೂ ಕೆಟ್ಟು ಏಳುನೆರೆಯನ್ನೂ ಕೆಡಿಸುತ್ತಿದ್ದಾರಪ್ಪ!

\section*{ಇಂದಿನ ವಿಷಮ ಪರಿಸ್ಥಿತಿಯಲ್ಲಿ ದೈವೀಧರ್ಮವರಿತವರ ಜವಾಬ್ದಾರಿ}

ಹೀಗಿರುವಾಗ ನಾವೇನಾಗಬೇಕು? ದೇಶಕ್ಕೇನು ಕೊಡಬೇಕು? ಎಂಬುದನ್ನರಿತು ನಮ್ಮ ಧ್ಯೇಯ ಧೋರಣೆಗಳನ್ನು ಹದಗೆಡಿಸಿಕೊಳ್ಳದೇ ದೈವಕೊಟ್ಟ ಧರ್ಮದೊಡನೆ ಸರಿಯಾದ ದಾರಿಯಲ್ಲಿ ಸಾಗುತ್ತಾ, ವಿಷಯಗಳನ್ನು ವೈಜ್ಞಾನಿಕವಾದ ನಡೆಯೊಂದಿಗೆ ತೆಗೆದುಕೊಂಡು ಬಹಳ ಜವಾಬ್ದಾರಿಯಿಂದ ಹೆಜ್ಜೆಯಿಡಬೇಕಾಗಿದೆ.

\section*{ನಾಟ್ಯ ಸರಸ್ವತಿಯ ಕೈಯಲ್ಲಿರುವ ವಿದ್ಯಾಮಯವಾದ ಪುಸ್ತಕವೇ ನಿಜವಾದ ಪುಸ್ತಕ} 

ಸಾರಭೂತವಾಗಿ ನಾವುಗಳು ಏನುಮಾಡಬೇಕೆಂದರೆ ಆ ಪರಮಪುರುಷನ ಆಶಯಕ್ಕೂ, ಅದನ್ನು ಹೊತ್ತುಕೊಂಡು ಮೌನವಾಗಿ ರಮಣೀಯವಾಗಿ ಸಾಗುತ್ತಿರುವ ನಿಸರ್ಗಕ್ಕೂ ತಕ್ಕಂತೆ ನಮ್ಮ ಹೃದಯರಂಗದಲ್ಲಿ ಪರಮಾನಂದಭರಿತಳಾಗಿ ನಾಟ್ಯವಾಡುತ್ತಿರುವಳೋ, ಆ ಮಹಾನಾಟ್ಯಭಾರತಿಯ ಜಾಡನ್ನು ಹಿಡಿದು ನಾವು ಸಾಗಿ, ಆ ಮಾತೆಯು ತನ್ನ ಕೈಯಲ್ಲಿ ಯಾವ ಪುಸ್ತಕವನ್ನು ಹಿಡಿದಿದ್ದಾಳೋ, ಆ ವಿದ್ಯಾಮಯವಾದ ಪುಸ್ತಕವನ್ನು ತಾನೆಪ್ಪ ಸಂಗ್ರಹಿಸಬೇಕಪ್ಪ! ಅದೇ ನಿಜವಾದ ಪುಸ್ತಕ. ಯಾವುದೇ ಪುಸ್ತಕವಾದರೂ, ಅವಳ ಕೈಪುಸ್ತಕದಲ್ಲಿರುವ ರಹಸ್ಯವನ್ನೇ ನಮಗೆ ತಿಳಿಸಬೇಕು.

\section*{ಇಂದಿನ ಪುಸ್ತಕ ಪ್ರಪಂಚ}

ಈಗ ಆ ಪುಸ್ತಕಕ್ಕೂ ಬೆಳೆಯುತ್ತಿರುವ ಪುಸ್ತಕಗಳಿಗೂ ಸಂಬಂಧವೇ ಇಲ್ಲವಾಗಿದೆ. ಎಲ್ಲಾ ಪುಸ್ತಕಗಳೂ ಜೀವನವನ್ನೇ ಮಿಸ್‌ಲೀಡ್ ಮಾಡುತ್ತಿವೆ. ಪುಸ್ತಕದಲ್ಲಿರುವುದೆಲ್ಲ ವಿದ್ಯೆಯಲ್ಲ. ವಿದ್ಯೆಯನ್ನೇ ಪುಸ್ತಕಕ್ಕಿಳಿಸಿದಾಗ ಮಾತ್ರ ಪುಸ್ತಕದಲ್ಲಿ ವಿದ್ಯೆ ಸಿಗುವುದು. ಅವಿದ್ಯೆಯನ್ನು, ವಿದ್ಯಾಭ್ಯಾಸವನ್ನು ಪುಸ್ತಕಕ್ಕಿಳಿಸಿಬಿಟ್ಟು ಅಲ್ಲಿ ವಿದ್ಯೆಯನ್ನ್ಹು ಹುಡುಕಲಾಗದು.

\section*{ನಿಜವಾದ ಪುಸ್ತಕವು ರೂಪುಗೊಳ್ಳುವ ಬಗೆ ಹಾಗೂ ಅದರ ನಡೆ}

ವಿದ್ಯೆಯಿಂದಲೇ ಸೃಷ್ಟಿ ವಿದ್ಯೆಯಿಂದಲೇ ಸ್ಥಿತಿ. ವಿದ್ಯೆಯಿಂದಲೇ ಲಯ.

\begin{shloka}
`ವಿದ್ಯಾಮಯೋಽಯಂ ಪುರುಷಃ'|\label{133b}\\
`ವಿದ್ಯಯಾ ತಾತ ಸೃಷ್ಟಾನಾಂ ವಿದ್ಯೈವೇಹ ಪರಾಗತಿಃ'||\label{133}\\
`ವಿದ್ಯಯಾಽಮೃತಮಶ್ನುತೇ'||\label{133a}
\end{shloka}
ಅಂತಹ ವಿದ್ಯೆಯನ್ನು ಹೊರತಲು ಹೊರಟಾಗ, ಆ ಬಗೆಯ ಹಾವ-ಭಾವ-ಮುದ್ರೆಗಳಿಂದ ಕೂಡಿ ಗಂಭೀರವಾದ ಸ್ವರೂಪವನ್ನು ತಳೆಯುವ ಈ ಮುಖವೇ ನಿಜವಾದ ಒಂದು ಪುಸ್ತಕ. ಅಲ್ಲಿಂದ ಮುಂದಕ್ಕೆ ಅದನ್ನು ವಿವರಿಸಲು ಹೊರಡುವ ನಾಲಗೆಯೂ ಪುಸ್ತಕವೇ. ಅನಂತರ ಅದಕ್ಕೆ ತಕ್ಕ ನಿಲುವನ್ನು ಹೊಂದುವ ಈ ಕೈಯೂ ಪುಸ್ತಕ. ಅನಂತರ ಈ ಪೇಪರಿನ ಪುಸ್ತಕ ಏರ್ಪಡುವುದು. ಎಲ್ಲಕ್ಕೂ ಮೂಲವಾದ ಹೃದಯದ ದೈವೀ ಪುಸ್ತಕವೊಂದಿಲ್ಲದಿದ್ದರೆ ಇದಾವುದೂ ಉಪಯೋಗಕ್ಕೆ ಬರುವ ಪುಸ್ತಕವಲ್ಲಾಪ್ಪ! ಆ ಪುಸ್ತಕವು ಇಲ್ಲದೇ ನಮ್ಮ ಹೃದಯವು ನೆಲೆಗಾಣದೇ ದಿಕ್ಕೆಟ್ಟು ಭೀರುವಾಗಿರುವಾಗ, ಹೊರಗಡೆ ಬೀರುವನ್ನು ಸಂಪಾದಿಸುವುದೇಕೆ?

\section*{ಯಾವ ಪುಸ್ತಕವನ್ನು ಪೂಜಿಸೋಣ}

ನಾವು ಭಾವುಕರಾಗಿ ಬೇಡಿಕೊಂಡರೆ ತನ್ನ ಕೈಪುಸ್ತಕವನ್ನಾದರೂ ಕೊಟ್ಟು ಬಿಡುವಳು. ಆದರೆ ನಮ್ಮ ಕೈಪುಸ್ತಕವನ್ನವಳು ಮುಟ್ಟುವುದೂ ಇಲ್ಲ. ಆದ್ದರಿಂದ ಅವಳು ಮುಟ್ಟದಂತಹ ನಮ್ಮ ಪುಸ್ತಕವನ್ನು ಅವಳ ಪಕ್ಕದಲ್ಲಿಟ್ಟು ಪೂಜಿಸಲು ಯತ್ನಿಸದೇ ಅವಳ ಕೈಪುಸ್ತಕವನ್ನೇ ಅವಳೊಂದಿಗೆ ನಮ್ಮ ಮುಂದಿಟ್ಟುಕೊಂಡು ಪೂಜಿಸುವವರಾಗೋಣಾಪ್ಪ! ಇಷ್ಟಂಶವನ್ನು ಧೃಢವಾಗಿ ಕಾಪಿಟ್ಟುಕೊಳ್ಳಿ. ಬಾಕಿಯೆಲ್ಲವೂ ಎಷ್ಟೆಷ್ಟು ಕೂಡಿಬರುತ್ತೋ ಅಷ್ಟಷ್ಟಾಗಲೀ! ನಡೆಯು ಮಾತ್ರ ಗಂಭೀರವಾಗಿರಲಿ!. ನನ್ನ ಮನಸ್ಸನ್ನು ನಿಮ್ಮ ಮುಂದಿಟ್ಟಿದ್ದೇನೆಪ್ಪ! ಪುಸ್ತಕದ ಬಗ್ಗೆ ಇಷ್ಟು ಆಶಯವನ್ನು ಮುಂದಿಟ್ಟುಕೊಂಡು ಈ ಮಹಾಭಾರತ ಪುಸ್ತಕವನ್ನು ಮುಟ್ಟಿ, ಅದನ್ನು ಮಂದಿರದ ಪುಸ್ತಕವಾಗಿ ತೆಗೆದು ಕೊಳ್ಳುತ್ತೇನೆ. ಕೃಷ್ಣಾ! ಕೃಷ್ಣಾ!



