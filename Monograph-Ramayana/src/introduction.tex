\chapter*{Introduction}\label{introduction}

\lhead[\small\thepage\quad Manjushree Hegde]{}
\rhead[]{Introduction\quad\small\thepage}

\addtoendnotes{\protect\bigskip{\noindent\Large\bfseries Introduction}\bigskip}

\begin{flushright}
{\sl\bfseries ``The worst readers are those who go about it like marauding troops: they remove what they can make use of, befoul and derange the rest, and blaspheme the whole.''}

\vskip .1cm
{\sl\bfseries -~ Nietzsche}
\end{flushright}



The {\sl Rāmāyaṇa}\index{Ramayana@\textit{Rāmāyaṇa}} has commanded attention from Western Indologists\index{Western Indologists} ever since their introduction to the “epic” at the turn of the nineteenth century. Even now, at the turn of the twenty-first century, that attention has little turned elsewhere.  

Early European scholarship was often harshly critical of the epic : they displayed, almost invariably, a characteristic uneasiness when faced with the “chaotically–structured” text filled with “ogres, magicians and talking beasts” (Goldman 1984: 27).\index{Goldman, Robert} Consequently, they sought to identify an “epic nucleus” --- some {\sl one} great and complex action --- embedded, though transformed, within the ample tapestry of the {\sl Rāmāyaṇa}. Most of their researches, therefore, were attempts to cut away what was to them fantastic figments, the winding disquisitions, and the complex didactic material --- all in order to restore the “original” story --- perhaps a sombre “heroic tale of love, loss and recovery” (Pollock 1993: 262). V.S. Sukthankar\index{Sukthankar, V.S.} sarcastically commented,  

\begin{myquote}
“Modern criticism begins with the assumption that... the “nucleus”... was unfortunately used --- or rather misused --- by wily priests, tedious moralists and dogmatizing lawyers as a convenient peg on which to hang their didactic discourses and sacerdotal legends ... it is a great pity that a fine heroic poem, which may even be found to contain precious germs of ancient Indian history, should have thus been ruined by its careless custodians. But it is not quite beyond redemption. A skillful surgical operation --- technically called “Higher Criticism”— could still disentangle the submerged “epic core” from the adventitious matter...”

\hfill Sukthankar (1957:10)
\end{myquote}

So, the chief European critics of the {\sl Rāmāyaṇa}\index{Ramayana@\textit{Rāmāyaṇa}!European critics} — Lassen, Weber, Jacobi, Schlegel, Gorresio etc --- subscribed to this “higher criticism”\endnote{See Sukthankar (1957)}. 

Twenty-first century studies of the {\sl Rāmāyaṇa}, on the other hand, are mostly focused on the topic of “folk and vernacular\index{Ramayana@\textit{Rāmāyaṇa}!vernacular versions} versions of the Rāma story in which the hegemonic discourses\index{Ramayana@\textit{Rāmāyaṇa}!hegemonic discourses} of patriarchy and social hierarchy that lie close to the heart of Vālmīki’s {\sl Rāmāyaṇa} are contested or resisted in a variety of subaltern and/or regional retellings” (Goldman 2004: 19). In other words, modern critics of Vālmīki’s {\sl Rāmāyaṇa} fixate on identifying elements of “hegemonic and comprehensive regimes of patriarchal dominance” (Goldman 2004: 20)\index{Goldman, Robert} in the text, and address only issues of gender, power, hierarchy and authority in it.  

Contrarily, Professor Sheldon Pollock,\index{Pollock, Sheldon} the Arvind Raghunathan Professor of South Asian Studies, Columbia University, (henceforth Pollock) is in a league of his own. Pollock recognized early on that Indologists were either concerned with a text’s existence in its “moment of genesis” (Pollock 2014:399), or in its relevance to present-day scenario. Both methods --- independently considered --- were, he mused, wholly flawed. In his extensive works on philology, Pollock carefully examined the two methods, and held them up to the light to show the cracks --- the barest chiaroscuro of light --- in them.

Accordingly, if an Indologist subscribes to the first method (i.e., treatment of a text in its “moment of genesis”), he is implicitly committed to the belief that the text is a purely historical object, and has zero meaning to him in his present. Consequently, his goal is to “erase any living critical appropriation of the past text” (Pollock 2014:401) to unearth the “mind of its author”. Its pitfall, according to Pollock, lies in its exclusivity. It (a) neglects the tradition’s reading of the text (b) ignores the presence of past texts in contemporary times, and (c) presupposes, rather arrogantly, that “our own historical being can be erased in grasping that past historical meaning” (Pollock 2014:401). 

An Indologist committed to a “present-reading” of the text, on the other hand, attempts only to problematize the participants’ narrative and decode meanings that are “disguised” in the text and beyond it; consequently, the texts’ meaning to a different (or rather, original) participant/audience is wholly ignored. In Pollock’s words, 

\begin{myquote}
“This type of deeply\index{Pollock!antihistoricist approach} antihistoricist approach typical of our students is the most powerful impression many take away from the experience of teaching the text, but it is an approach that carries its own kind of truth, measuring (positively) the distance in consciousness between now and then but also (negatively) the failure of students to register that distance and enter into other planes\index{Pollock, Sheldon} of reading.”
\hfill Pollock (2014:408)
\end{myquote}

According to Pollock, then, a “tension” exists between the two modes of philology\index{Pollock!philology} for they are viewed as mutually exclusive to each other --- {\sl only} one is often construed to be true. Moreover, a striking feature of philological studies is, he writes, the complete neglect of the readings offered by the tradition itself: 

\begin{myquote}
“Most scholars simply ignore these, as my classics teachers always did, for whom no traditional interpretation, whether of Hellenistic scholiast, Roman commentator or medieval scribe, could make any claim to truth. Even those who do not ignore them, like my Indian teachers or Sanskrit colleagues, rarely offer an account of why we should take the meanings, or the truths, of tradition seriously.”  			       		     						    
\hfill Pollock (2014:402)
\end{myquote}

First of all, then, Pollock wishes to draw\index{traditionist plane} attention to the “traditionist-plane of reading” of a text: a text’s receptive history, he says, is just as important as its genitive history. Of the reason of its importance --- one that his Indian teachers and Sanskrit colleagues failed to muster --- he proposes that every interpretation\index{interpretation of text} of a text is {\sl enabled} because of the presence of certain elements in the text that allows for it --- one would not be possible without the other. It is, therefore, the task of a philologist to search for and retrieve these elements of the text --- to what end is not clear, though. 

Secondly, Pollock wishes to find a delicate way to weave together the three apparently separate strands of philological modes --- reading of a text in “its moment of genesis, its reception over time, and its presence to personal subjectivity” (Pollock 2014:399). Accordingly, there are three, “potentially radically different, dimensions of meanings” to a text --- “the original author’s, the tradition’s and my own” (Pollock 2014:401). Each of these meanings, writes Pollock, are true: {\sl none more or less than the others}. Each contributes towards a subtler/deeper understanding of a text, and must be taken into serious account. Philology thus “resides in the sum total of the varied senses generated on these three planes, their lively copresence to our mind.” (Pollock 2014:400)  

In this manner, Pollock crafts for himself an exquisitely delicate tool --- a pluralistic and inclusive mode of philology that allows him to analyze the {\sl Rāmāyaṇa}\index{Ramayana@\textit{Rāmāyaṇa}} on three different, yet accommodating, “planes of truths”, “feeling no compulsion to rank or even to reconcile them” (Pollock 2014:400). A\index{planes of truth} close reading of his {\sl Rāmāyaṇa}, then, lays bare the effortless ease with which he alternates between the “planes of truth”--- choosing this one or the other in conformity to, as this monograph aims to show, his own implicit design. 

In this monograph I will explore these readings of Pollock: in the first chapter, his “Plane 1” reading of the {\sl Rāmāyaṇa}; in the second, his “Plane 2” reading, and in the third, his “Plane 3” reading of the text. I will employ three angles to analyze and critique his different readings: 
\begin{itemize}
\itemsep=1pt
\item[1)] Philological --- analysis of the method employed to “make sense”\index{analysis!philological@-philological} of the text; 

\item[2)] Semantic --- analysis\index{analysis!semantic@-semantic} of factuality of certain data, translations, interpretations, etc; 

\item[3)] Linguistic --- analysis\index{analysis!linguistic@-linguistic} of language employed for sophisticated hypothesizing.
\end{itemize}

The {\sl Rāmāyaṇa} edition that I have referrred to is the Gita Press (Gorakhpur) edition.


