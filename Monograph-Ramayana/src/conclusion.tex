\chapter*{Conclusion}\label{conclusion}

\addtoendnotes{\protect\bigskip{\noindent\Large\bfseries Conclusion}\bigskip}
\addtocontents{toc}{\protect\contentsline{chapter}{\protect Conclusion}{\thepage}}

\lhead[\small\thepage\quad Reclaiming~ {\sl Rāmāyaṇa}]{}
\rhead[]{Conclusion\quad\small\thepage}


\begin{flushright}
\begin{tabular}{r@{}}
{\sl\bfseries Kāvye rasayitā sarvo na boddhā na niyogabhāk}\\[3pt]
{\sl\bfseries  Bhaṭṭa Nāyaka}\index{Bhatta Nayaka@Bhaṭṭa Nāyaka}
\end{tabular}
\end{flushright}

“Arms and the man I sing”— so run the opening lines of Virgil’s\index{Virgil} {\sl Aeneid}.\index{Aeneid@\textsl{Aeneid}} These few words reflect, wrote Paul Cantor,\index{Cantor, Paul} the whole essence of (ancient) epic poetry: warfare and politics. In his words, 

\vskip .1cm

\begin{myquote}
“...Homer\index{Homer} and Virgil… they do single out the warrior’s life as the central theme of epic poetry. Even Shakespeare,\index{Shakespeare} with his wider range as a poet, focuses his serious plays, his histories and tragedies, on public figures and the central political issue of war and peace… The traditional concept of epic and tragedy as the supreme genres and the pinnacle of literary achievement effectively placed political life at the center of poetic concern.”\hfill (Cantor 2007:375)
\end{myquote}

\medskip
Perhaps it is true of Western Epics. Perhaps it is not. What is certain is that it does not reflect the tenor of Sanskrit {\sl kāvya}-s — and certainly not of the {\sl Rāmāyaṇa}. For Vālmīki’s\index{Valmiki@Vālmīki} {\sl Rāmāyaṇa} is, first \& foremost, a {\sl kāvya}\endnote{In no uncertain terms, the {\sl Rāmāyaṇa} characterizes itself as a {\sl kāvya}. See Kane\index{Kane, P. V.} (1966), Hiltebeitel (2005), Pathak (2007). In his preface to “The Sanskrit Epics”, J. L. Brockington\index{Brockington, J. L.} raises a question: “Is it... worth asking from the start whether the designation of the {\sl Mahābhārata}\index{Mahabharata@\textsl{Mahābhārata}} and the {\sl Rāmāyaṇa} as “epics” affects our understanding of them, generating expectations derived from ideas about the {\sl Iliad}\index{Iliad@\textsl{Iliad}} and {\sl Odyssey}”\index{Odyssey@\textsl{Odyssey}}.} — and {\sl kāvya} — like the other arts — is/was considered a magnificent form of {\sl yoga} in India. 

\smallskip
In his essay, {\sl The Theory of Art in Asia}, Coomaraswamy\index{Coomaraswamy, Ananda} demonstrated the formal steps in the {\sl yoga} of “making of an artifact”: 

\smallskip
\begin{myquote}
“...[the artist], having by various means proper to the practice of {\sl Yoga} eliminated the distracting influences of fugitive emotions and creature images, self-willing and self-thinking, proceeds to visualize the form... The mind “produces” or “draws” ({\sl ākarṣati}) this form to itself, as though from a great distance. Ultimately, that is, from Heaven, where the types of art exist in formal operation; immediately, from the “immanent space in the heart” ({\sl antar-hṛdaya-ākāśa}), the common focus ({\sl saṁstāva}, “concord”) of seer and seen, at which place the only possible experience of reality takes place. The true-knowledge-purity-aspect ({\sl jñāna-sattvarūpa}) thus conceived and inwardly known ({\sl antar-jñeya}) reveals itself against the ideal space ({\sl ākāṡa}) like a reflection ({\sl pratibimbavat}) or as if seen in a dream ({\sl svapnavat}). The imager must realize a complete self-identification with it ({\sl ātmānam… dhyāyāt, bhāvayet}), whatever its peculiarities, even in the case of the opposite sex or when the divinity is provided with terrible supernatural characteristics; the form thus known in an act of non-differentiation, being held in view as long as may be necessary ({\sl evam rūpam yāvad icchati tāvad bhāvayet}), is the model from which he proceeds to execution in stone, pigment, or other material.” 

\hfill (Coomaraswamy\index{Coomaraswamy, Ananda} 1934:5)
\end{myquote}

In an uncannily similar terminology, we are told in the first {\sl kāṇḍa} of the {\sl Rāmāyaṇa} ({\sl Rāmāyaṇa} 1.3.2-8):

\begin{myquote}
“Vālmīki,\index{Valmiki@Vālmīki} although he was already familiar with the story of Rāma,\index{Rama@Rāma} before composing his own {\sl Rāmāyaṇa} sought to realize it more profoundly, and seating himself with his face towards the East and sipping water according to rule (i. e. ceremonial purification), he set himself to yoga-contemplation of his theme. By virtue of his yoga-power he clearly saw before him Rāma, Lakṣmaṇa\index{Laksmana@Lakṣmaṇa} and Sītā,\index{Sita@Sītā} and Daśaratha,\index{Dasaratha@Daśaratha} together with his wives, in his kingdom laughing, talking, acting and moving as if in real life ... by yoga-power that righteous one beheld all that had come to pass, and all that was to come to pass in the future, like a {\sl nelli} fruit on the palm of his hand. And having truly seen all by virtue of his concentration, the generous sage began the setting forth of the history of Rāma”.   
\hfill (Coomarswamy 1918:23)
\end{myquote}

“As a man among men,” F. M. Cornford\index{Cornford, F. M.} writes in his discussion of inspiration among the Greeks, “the poet depends upon hearsay [as when Vālmīki\index{Valmiki@Vālmīki} heard the story of Rāma\index{Rama@Rāma} from Nārada]; but as divinely inspired [Vālmīki in {\sl yogic} vision], he has knowledge of an eyewitness, ‘present’ at the feats he illustrates.” (Cornford 1952:76-77) To Pollock, this is simply a romantic, “sentimental” narration (Pollock 2006:99). But we must concede that when the ancients — Indians, Greeks, or others — called upon the Muses — who were ‘present and knew all things’— to tell them what common mortals like themselves could not know, it is likely that their meaning was more serious than we suppose. Knowledge was understood to be directly accessible to the concentrated and ‘one-pointed’ mind, without the direct intervention of the senses. In the language of psycho-analysis,  “the willed introversion of a creative mind, which, retreating before its own problem and inwardly collecting its forces, dips at least for a moment into the source of life, in order there to wrest a little more strength from the mother for the completion of its work,” (Hinkle 1965:337). 

Indian theory of art is very clear that instruction is not the primary purpose of art. Abhinavagupta shows in the clearest terms that {\sl vyutpatti}\index{vyutpatti@\textsl{vyutpatti}} (didactic education) is — not consciously aimed at, but always is — a by-product of {\sl rasānubhava}\index{rasanubhava@\textsl{rasānubhava}} of a {\sl kāvya}:

\begin{myquote}
{{\sl na hi teṣāṁ vākyānām agniṣṭomādi-vākyavat satyārtha-pratipādana-dvāreṇa pravartakatvāya prāmāṇyam anviṣyate, prītimātra-paryavasāyitvāt / prītereva cālaukika-camatkāra-rūpāyā vyutpatty-aṅgatvāt}} ({\sl Locana,} on {\sl Dhvanyāloka}\index{Dhvanyaloka@\textsl{Dhvanyāloka}} 3.33)

 “... from the sentences of poetry we do not seek for the performance of certain acts on the basis of the tranmission by the sentence of a meaning that is true, as we do from such Vedic setences as “{\sl agniṣṭomam juhuyāt}” (one must offer a fire sacrifice). This is because the end of poetry is pleasure, for it only by pleasure, in the form of an otherworldly delight, that it can serve to instruct us.”
\hfill  [{\sl Trans.} Ingalls~\index{Ingalls, D H} {\sl et al}]
\end{myquote}

So, the “purpose” of a {\sl kavi} is never self-expression/political-agenda/\-social-propaganda. It is unfortunate and ironical in equal measure that the “application” of Pollock’s “inclusive” and “pluralistic” philological tool leads him, not to three “radically different dimensions of truth(s)”, but to a {\sl singular} truth on all three modes of philology:\index{philology!three dimensional@3 dimensional} that the {\sl Rāmāyaṇa} is a work that deals with power-dynamics. 

Yet, it must be admitted that ‘interpretation’ is a landscape peppered with difficulties — it is the famous “hermeneutic circle” that perplexed Dilthey, {\sl et al}. Perhaps it must also be admitted that it is impossible to ascribe a single meaning to a text. But in Matthew Kapstein’s\index{Kapstein, Matthew} words, 

\begin{myquote}
“...if a limitless horizon of possible understandings begins to open before us, we balk nevertheless at the thought that any understanding is just as good as any other. Even if guided by a Kabbalistic conception of the plenitude revealed in each letter of scripture, we retreat before the prospect that all interpretive possibilities must be treated as equal. However, we find ourselves at a loss to specify sure principles that would permit us to delineate between an unlimited range of acceptable or fruitful understandings and an unending field of fantasies that we wish to rule out of court.”
\hfill (cited in Ganeri 2017:15) 
\end{myquote}

Nevertheless, Kapstein offers a response to this “conundrum”: he calls the Indologists to play this game (of interpretation) by employing traditional Indian hermeneutics — “whether embodied in written commentaries or in the living expertise of traditionally educated scholars” — as a compass to “forge pathways through a conceptual topography” (Ganeri 2017:16). Going the Kapstein\index{Kapstein, Matthew} way, then, and locating the {\sl Rāmāyaṇa} within the aims of Indian tradition, we can see that the text is not simply a social or political enterprise, but a magnificent work that is aligned to the ultimate purpose of life. Pollock, on the other hand, is clear:

\begin{myquote}
“These problems can be formulated through a large comparative generalization. If Homer,\index{Homer} for example, addresses, a transcendent problem, showing us what makes life finally impossible— in the words of one writer, “the universality of human doom”— Vālmīki\index{Valmiki@Vālmīki} poses the more difficult question: What is it that makes life possible? This is more difficult because {\sl it is a social, not a cosmic question}.” 

\hfill Pollock (2007a:4) [{\sl italics ours}]
\end{myquote}

To Pollock, the question of {\sl dharma}\index{dharma@\textit{dharma}} is social and political; to Vālmīki, it is universal and trans-mundane. From this position, both stand opposite to each other locked, as it were, in a zero-sum struggle for meaning — and the twain may never meet. 

\newpage
\label{notes}
\lhead[\small\thepage\quad Reclaiming~ {\sl Rāmāyaṇa}]{}
\rhead[]{Notes\quad\small\thepage}
\addtocontents{toc}{\protect\contentsline{chapter}{\protect Notes}{\thepage}}
\theendnotes
