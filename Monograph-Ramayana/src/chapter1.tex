\chapter{“The Mind of Vālmīki”}\label{chapter1}

\addtoendnotes{\protect\bigskip{\noindent\Large\bfseries\thechapter. “The Mind of Vālmīki”}\bigskip}
\index{Valmiki@Vālmīki}

\section{The Problematic}\label{sec1.1}

Pollock's introductory essays to his translation of the Ayodhyā-kāṇḍa\index{Ayodhyakanda@Ayodhyā-kāṇḍa} (Clay Sanskrit Library, 1986) are his own deconstructions of Vālmīki’s {\sl Rāmāyaṇa} upon “Plane 1 Philology”\index{philology@Plane 1}\endnote{Sheldon Pollock offers a deconstruction of Vālmīki’s\index{Valmiki@Vālmīki} {\sl Rāmāyaņa} on three different planes: in its “moment of genesis” (Plane 1),\index{philology!Plane 1} its reception over time (Plane 2),\index{philology!Plane 2} and its presence to personal subjectivity (Plane 3)\index{philology!Plane 3}.}. “Plane 1 philology” is essentially an effort “to understand the text at first as well as, and then even better than, its author” (Schleiermacher\index{Schleiermacher, Friedrich} 1997:112). It is an effort to theorize about the “original nucleus” of a text --- to distinguish between the “old” and the “new”, the “original” and the “spurious” sections of a text, to reconstruct --- psychologically and historically --- the original author's mind. In order to realize this goal, Pollock locates Vālmīki’s\index{Valmiki@Vālmīki} {\sl Rāmāyaṇa} in its (1) literary and (2) historical climate/context. In his own words,  

\medskip
\begin{myquote}
“The meaning and significance of the Ayodhyākāṇḍa will come into focus more readily if we view the story against the background of other ancient Indian epic narratives and locate those narratives in their historical context... Just as, according to the first rule of  interpretation we determine the signification of a word in reference to the words surrounding it, so the literary text can be viewed as a semantic entity that acquires specific meaning in reference to the “texts” in which it is embedded: the literary genre, for example, and its historical situation.” 		
	       							         				      
\hfill  Pollock (2007a:9)
\end{myquote}


Pollock's {\sl Rāmāyaṇa}, therefore, finds itself betwixt\\[-20pt] 
\begin{itemize}
\itemsep=1pt
\item[(a)] “other ancient Indian epic narratives” and

\item[(b)] at the tail-end of “a dynamic period of transition”--- the middle Vedic age. Let us, then, examine each in its turn.\\[-18pt]
\end{itemize}

\subsection{Literary Climate}\label{sec1.1.1}

Most if not every ancient Indian epic, reflects Pollock, is at its core, a confrontation of a “struggle among “brothers” for succession to the hereditary throne” (Pollock 2007a:10).  In his words, 
\vskip 1pt

\begin{myquote}
“... an integral theme of Sanskrit epic literature is kingship ... the acquisition ... the legitimacy of succession, the predicament of transferring hereditary power within a royal dynasty...”

\hfill Pollock (2007a:10)
\end{myquote}

So, according to him, the {\sl Mahābhārata}\index{Mahabharata@\textsl{Mahābhārata}} is essentially a struggle between Yudhiṣṭhira\index{Yudhisthira@Yudhiṣṭhira} and Duryodhana\index{Duryodhana} for the succession to the Kuru throne; the {\sl Harivaṁśa}\index{Harivamsa@\textsl{Harivaṁśa}} is a story of Kṛṣṇa\index{Krsna@Kṛṣṇa} slaying Kaṁsa\index{Kamsa@Kaṁsa} in order to reinstate the latter’s father on the throne; the {\sl Nalopākhyāna}\index{Nalopakhyana@\textsl{Nalopākhyāna}} treats of the struggle between King Nala\index{Nala} and his brother, Puṣkara,\index{Puskara@Puṣkara} for the throne of Niṣadha; stories of Yayāti,\index{Yayati@Yayāti} Śakuntalā,\index{Sakuntala@Śakuntalā} and Devavrata\index{Devavrata} are quoted by him as further examples of his stand. 
\vskip 1pt

It is at once clear that Pollock makes this conclusion by discarding the “unnecessary” “appropriation(s) by brahmanical orthodoxy” (Pollock 2007a:9) --- in other words, by an employment of “higher criticism”.\index{Higher Criticism} In his own words, 
\vskip 1pt

\begin{myquote}
“In the course of their transmission, in oral and afterwards in written form, and as a result of their appropriation by brahmanical orthodoxy, a congeries of topics --- mythological, philosophical, religious, and so on --- was incorporated into them. But at the root, in the very heart of many of these epic narratives, can be found a political problem...”	

\hfill Pollock (2007a:9)
\end{myquote}

Scholars who employ “higher criticism” must illustrate their methods for whatever it is worth. Ludwig,\index{Ludwig, A} for example, specified thus:

\begin{myquote}
“... special attention must be given to the way in which the various episodes have been joined together— whether they have been welded into a harmonious whole or whether they have been pieced together clumsily. The critic must be on the look-out for ‘misconceived-links’, ‘striking laboriousness’, ‘absolute superfluity’, ‘repetition of the theme’, ‘unnatural and farfetched motivation’, ‘incongruity between explanation and matter to be explained’, and so on and so forth. These are all indications of unoriginality and interpolation...”	

\hfill Cited in Sukthankar\index{Sukthankar, Vishnu Sitaram} (1958:7)
\end{myquote}

Unfortunately, Pollock neither illustrates nor justifies the criteria he uses to “cut away”  ``extraneous'' portions so as to determine the “core” of these texts. Based, instead, on this alleged primary concern of kingship in the afore-quoted stories, he deduces that the {\sl Rāmāyaṇa} is also, at its core, a determinate treatment of kingship and its politics. It may be useful to trace here in brief a history of the method that Pollock subscribes to --- its application and its pitfalls.\\[-21pt]  

\subsubsection{Excursus: A   Brief History of Higher Criticism}\label{sec1.1.1.1}\index{Higher Criticism}

Friedrich August Wolf\index{Wolf, Friedrich August} (1754-1824) is generally considered to be the first to tread in a definitive direction towards higher criticism.\index{Higher Criticism} Wolf was deeply skeptical of ancient literary works. In 1795, he published a work, {\sl Prolegomena ad Homerum}, where he challenged the notion that Homer\index{Homer} wrote the {\sl Iliad}\index{Iliad@\textsl{Iliad}} or the {\sl Odyssey}\index{Odyssey@\textsl{Odyssey}}\endnote{See Grafton\index{Grafton, Anthony} (1795).}; these epics, according to Wolf, were gradually built/accumulated in course of time by storytellers and wandering Greek minstrels, and could not be ascribed to any one author. Following the lead of Wolf, Karl Lachmann (1793-1851)\index{Lachmann, Karl} dissected the German epic, {\sl Nibelungenlied},\index{Nibelungenlied@\textsl{Nibelungenlied}} into twenty short “lays”, and the {\sl Iliad} into eighteen “lays”. These lays, he claimed, were variously augmented in course of time to finally attain\index{Everts, William Wallace} epic proportions\endnote{See Grote (1850).}. In William Wallace Everts’\index{Everts, William Wallace} scathing words, 

\begin{myquote}
“Lachmann adopted Wolf’s theory that epics are lays reduced to order, and applied it first to the {\sl Nibelungenlied}. He restored that long epic into its original form of twenty lays, he exposed to view the fissures that the redactor had vainly sought to hide --- fissures whose existence no one had suspected till Lachmann called attention to them, defects such as a childish mind is delighted to have pointed out in a work that appears to be perfect. His keen eye discovered in the poem awkward junctures and apparent discrepancies that revealed to him diversity of origin... He found the glory of the {\sl Iliad}\index{Iliad@\textsl{Iliad}} not in the whole epic with its symmetry and unity, but in the separate lays. The parts he accounted greater than the whole, for the lays were spoiled when they were squeezed into the mold of the epic...  In his work of disintegration, he drew sweeping conclusions from trivial objections... he did not hesitate to say that the parts were ill adjusted, that connections were senseless, that personages were inserted in the wrong place, that the description of arms, raiment, and feasts were useless, that there were frequent gaps, inequalities, yes contradictions...  This {\sl tour de force} had its admirers, the crowd that counts more than it weighs, that worships a great name and wants to be considered in line with the advanced thought of the hour. The followers of Lachmann\index{Lachmann, Karl} carried their master's method to further extremes, making breaks where they could not find joints, and subdividing his divisions, until by their complications they made his theory ridiculous...”
\hfill Everts\index{Everts, William Wallace} (1908:540)
\end{myquote}

Indeed, after the publication of these works, most literary and religious works were scanned for the “original” and the “interpolations” --- the Old Testament,\index{Old Testament} Greek New Testament,\index{New Testament} {\sl Aeneid, Beowulf},\index{Beowulf@\textsl{Beowulf}} works of Chaucer,\index{Chaucer} Shakespeare,\index{Shakespeare} etc. Yet, only a few decades later, this method fell into great disrepute. Men like Humboldt,\index{Humboldt, Alexander Von} Goethe,\index{Goethe} and Boeckh,\index{Boeckh, August} who had first endorsed the idea, rejected it upon further reflection, (Everts 1908:535). Goethe\index{Goethe} wrote (in his communication to Schiller dt. 16 May 1798),

\begin{myquote}
“I am more than ever convinced of the unity and indivisibility of the {\sl Iliad},\index{Iliad@\textsl{Iliad}} and there is no man living or ever will live who can change this conviction. I prefer to think of the {\sl Iliad} as a whole, to feel it joyfully as a whole. There is too much subjective in this whole business. It is interesting to doubt, but it is not edifying. These gentlemen are laying waste the most fruitful garden of the earthly kingdom. They have taken away from us all veneration... The {\sl Iliad} is so round and complete, they may say what they will, that nothing can be added to it, or taken from it.”
\hfill Calvert (1846:153)
\end{myquote}

Of the {\sl Nibelungenlied}\index{Nibelungenlied@\textsl{Nibelungenlied}} and {\sl Beowulf},\index{Beowulf@\textsl{Beowulf}} too, R. W. Chambers\index{Chambers, R. W.} and others pointed out that the quality of the compositions do not allow for the mere sticking of lays together. Chambers famously remarked, “Half a dozen motorbikes cannot be combined to make a Rolls-Royce” (Everts\index{Everts, William Wallace} 1908:553). Of the {\sl Mahābhārata},\index{Mahabharata@\textsl{Mahābhārata}} too, Charles Drekmeier\index{Drekmeier, Charles} commented:

\begin{myquote}
“Hopkins\index{Hopkins, E. Washburn} many years ago concluded that the original narrative core of the epic is impossible to isolate from the later mythical and moralistic accretions, and few present-day students of the Mahabharata would question his judgment.”
\hfill Drekmeier\index{Drekmeier, Charles} (1962:132)
\end{myquote}

Much research --- literary, historical and archeological --- has now led to a generally accepted consensus that the pursuit of the “epic core” is, at best, a dubious and unreliable inquiry. If one continues on its path for a stretch of time, one would have to conclude, like E. Washburn Hopkins, that “there was no text there at all”!\endnote{Sukthankar quotes Hopkins\index{Hopkins, E. Washburn} as saying, “In what shape has epic poetry (in India) come down to us? A text that is no text, enlarged and altered in every recession, chapter after chapter recognized even by native commentaries as {\sl prakṣipta}, in a land without historical sense or care for preservation of popular monuments, where no check was put on any reciter or copyist who might add what beauties or polish what parts he would, where it was a merit to add a glory to the pet god, where every popular poem was handled freely and is so to this day.” (Sukthankar\index{Sukthankar, Vishnu Sitaram} 1957:8)} In Kuntaka's\index{Kuntaka} words, 
\begin{quote}
{{\sl nirantara-rasodgāra-garbha-sandarbha-nirbharāḥ}} |\\
{\sl giraḥ kavīnāṁ jīvanti na kathā-mātram āśritāḥ} || 

\hfill({\sl Vakrokti-jīvita} 4.4, {\sl antara-śloka} 13)
\end{quote}

\begin{myquote}
``Those words of poets which describe episodes delineating poetic relish remain [in the hearts of men] --- not [merely] plot-driven stories'' ({\sl Trans.} ours).
\end{myquote}

V. S. Sukthankar\index{Sukthankar, Vishnu Sitaram} beautifully writes, 

\begin{myquote}
“Notwithstanding the high-sounding phrases in which it is couched, it is easy to see that this critique cannot give absolutely certain and dependable results, it being merely the exploitation of individual opinion, which selects what it pleases and rejects, on insufficient evidence, what is incompatible with a preconceived subjective scheme... the residue no more represents the “original” heroic poem than a mangled cadaver, lacking the vital elements, would represent the organism in its origin or infancy.”
\hfill Sukthankar (1957:7,5)\\[-.6cm]
\end{myquote}

\subsubsection{Pollock \& Higher Criticism}\label{sec1.1.1.2}
\index{Higher Criticism}

Yet, it does not deter Pollock from scrounging for the “original” in the {\sl Rāmāyaṇa} and other Indian epics to form, what can, at best, be called vague and incorrect impressions. 

Not by the furthest stretch of imagination can the {\sl Mahābhārata}\index{Mahabharata@\textsl{Mahābhārata}} be considered only a “struggle among “brothers” for succession to the hereditary throne”. Pollock’s hypothesis throws up a few questions: Why then did each of Arjuna,\index{Arjuna} Bhīma,\index{Bhima@Bhīma} Nakula,\index{Nakula} and Sahadeva\index{Sahadeva} --- Yudhiṣṭhira’s\index{Yudhisthira@Yudhiṣṭhira} brothers --- not contest his right to the throne? Why, again, did each of Duryodhana’s\index{Duryodhana} ninety-nine brothers not fight him for it? 

Furthermore, this is a text that has painted the Indian tradition with intense colors: the temples and art-traditions, the literature and poetry, the festivals and daily-life of India --- are deeply informed by the {\sl Mahābhārata}. In Stanley Rice’s words,\index{Rice, Stanley}

\begin{myquote}
“[the Indian epics] are living and throbbing in the lives of the people of India, even of those illiterate masses that toil in the fields or maintain a drab existence in the ghettos of the towns. To such as these the famous old stories are the music and color of life. They are the perennial fount from which the oft-repeated draughts never quench an insatiable thirst. In the king's palaces and in the peasant's huts you may still hear the grand legends of the Great War and the pathetic sufferings of Rāma...”

\hfill Rice (1924:9)
\end{myquote}

If the {\sl Mahābhārata}\index{Mahabharata@\textsl{Mahābhārata}} was a mere story of jealousy and internecine family strife, how could it fascinate, for more than two millennia, the minds of generations of men? What explains its vitality, its universality, its immortality? No, the {\sl Mahābhārata} is not simply a story of a futile war of annihilation; it is, {\sl at the very least}, a struggle between justice and injustice, between {\sl dharma}\index{dharma@\textsl{dharma}} and {\sl adharma}, between Self and not-Self. In Sukthankar’s\index{Sukthankar, Vishnu Sitaram} words,

\begin{myquote}
“What gives [the {\sl Mahābhārata}] real depth and significance is the projection of the story on to a cosmic background, by its own interpretation of the Bharata War as a mere incident in the ever recurring struggle between the {\sl Devas} and the {\sl Asuras}; in other words, as a mere phase in cosmic evolution... (the war) is the expression of a state of tension between two ideal order of beings, a moral type ...  and an immoral --- or rather an unmoral --- type which it is the object of the former to destroy. This war is an eternal recurrence, a phenomenon assuming in the time-space continuum the most diverse forms and aspects...” 
\hfill Sukthankar (1957:66)
\end{myquote}

Capturing perfectly the true essence of the {\sl Mahābhārata} in his {\sl magnum opus, Dhvanyāloka},\index{Dhvanyaloka@\textsl{Dhvanyāloka}} Ānandavardhana\index{Anandavardhana@Ānandavardhana} wrote, 

\begin{quote}
{{\sl mahābhārate'pi śāstra-rūpa-kāvya-cchāyā’nvayini vṛṣṇi-pāṇḍava-virasā’vasāna-vaimanasya-dāyinīṁ samāptim upanibadhnatā mahā-muninā vairāgya-janana-tātparya-prādhānyena sva-\break prabandhasya darśayatā mokṣa-lakṣaṇaḥ puruṣārthaḥ\index{purusartha-s@\textsl{puruṣārtha}-s}\index{santarasa@\textsl{śānta-rasa}}
 śānto rasaś ca mukhyatayā vivakṣā-viṣayatvena sūcitaḥ}} |\relax

\hfill {\sl Dhvanyāloka} 4.5, {\sl Vṛtti} thereon
\end{quote}

\begin{myquote}
“Again, in the {\sl Mahābhārata},\index{Mahabharata@\textsl{Mahābhārata}} which has the form of a didactic work although it contains poetic beauty, the great sage who was its author, by his furnishing a conclusion that dismays our hearts by the miserable end of the Vṛṣṇis\index{Vrsnis@Vṛṣṇi-s} and Pāṇḍavas,\index{Pandavas@Pāṇḍava-s} shows that {\sl the primary aim of his work has been to produce a disenchantment with the world} and that he has intended {\sl his primary subject to be liberation (mokṣa) from worldly life and the rasa of peace.}”

\hfill [{\sl Trans.}~ Ingalls\index{Ingalls, D H} {\sl et al}] [{\sl italics ours}]
\end{myquote}

If, then, there is a “core” to the {\sl Mahābhārata}, it is verily this.  

Coming now to {\sl Harivaṁśa},\index{Harivamsa@\textsl{Harivaṁśa}} similarly, it is hardly a story of fratricide\index{fratricide} --- it is, in three parts (Harivaṁśa-parvan, Viṣṇu-parvan, Bhaviṣyat-parvan), a rather winding account of Kṛṣṇa’s\index{Krsna@Kṛṣṇa} life. It resumes the {\sl Mahābhārata}’s\index{Mahabharata@\textsl{Mahābhārata}} framing dialogue between Śaunaka\index{Saunaka@Śaunaka} and Ugraśravas:\index{Ugrasravas@Ugraśravas} Śaunaka asks to hear more about the Vṛṣṇi-s,\index{Vrsnis@Vṛṣṇi-s} and Ugraśravas relates what Vaiśampāyana\index{Vaisampayana@Vaiśampāyana} told King Janamejaya\index{Janamejaya} in response to the same question. The Harivaṁśa-parvan follows --- and it contains the details of the creation of the cosmos, of Pṛthu Vainya\index{Prthuvainya@Pṛthu Vainya} (the first king), particulars of the scheme of successive Manu-s, of the tradition of ancestor worship, and of the Solar and Lunar dynasties etc. Janamejaya then asks about Viṣṇu\index{Visnu@Viṣṇu} Nārāyaṇa’s appearances in the world, and especially about his appearance as Kṛṣṇa Vāsudeva.\index{Vasudeva@Vāsudeva} Vaiśampāyana then describes the battle between gods and demons, explains why Viṣṇu took form as Kṛṣṇa, and narrates Kṛṣṇa’s life in detail. The Viṣṇu-parvan consists largely of that narration presenting Kṛṣṇa in his own family context (rather than that of his Pāṇḍava\index{Pandavas@Pāṇḍava-s} cousins) and narrating his birth, his and his brother Balarāma’s\index{Balarama@Balarāma} childhood exploits among the cowherds, his defeat of King Kaṁsa,\index{Kamsa@Kaṁsa} his role within the Vṛṣṇi clan, and his role in his sons’ and grandsons’ affairs. The Bhaviṣyat-parvan reverts to the Śaunaka-Ugraśravas\index{Saunaka@Śaunaka} dialogue: Ugraśravas gives details of Janamejaya’s\index{Janamejaya} descendants, and of his Aśvamedha sacrifice, and so ends the tale.
\vskip 1.3pt

On the vast canvas of the {\sl Harivaṁśa} --- containing 16,374 {\sl śloka}-s --- the episode of Kaṁsa’s defeat occurs within a mere 93 {\sl śloka}-s. In a sum total of 318 {\sl sarga}-s, Kaṁsa’s name occurs in ten. It is amply clear that the Kaṁsa-episode, albeit important, is hardly the “kernel” of the story of {\sl Harivaṁśa} --- so contrary to the claims of Pollock. 
\vskip 1.3pt

Coming now to {\sl Nalopākhyāna}. {\sl Nalopākhyāna}\index{Nalopakhyana@\textsl{Nalopākhyāna}} is a tale of the love of Nala\index{Nala} and Damayantī: Damayantī,\index{Damayanti@Damayantī} the lovely daughter of Bhīma\index{Bhima (father of Damayanti)@Bhīma (father of Damayantī)} (the king of Vidarbha),\index{Vidarbha} and Nala, the prince of Niṣadha, are mutually enamored by an intervention of swans. Very soon, king Bhīma arranges for the {\sl svayaṁvara} of Damayantī --- many princes and gods assemble in the court, and about the neck of the chosen, Damayantī must place a garland of flowers as a token of acceptance. But Damayantī is faced with the problem of distinguishing her mortal lover, Nala, from four of the gods who have assumed forms identical with that of Nala. She accomplishes this, however, by what is called “an Act of Truth”, and the gods assume their true forms, and congratulating Nala, take their departure to their own places. The following story --- spanning twelve years --- shows us the idyllic life of the royal lovers at their court. Yet, trouble brews elsewhere; at the time of the {\sl svayaṁvara}, Kali,\index{Kali} the Fourth Age of the world, had set out for Vidarbha\index{Vidarbha} to marry Damayantī;\index{Damayanti@Damayantī} {\sl en route}, he heard of the wedding of Nala\index{Nala}-Damayantī, and swore, out of jealousy, to work the ruin of Nala. Thus, owing to Kali's malevolence, Nala goes through a series of misfortunes: he loses his kingdom to his brother Puṣkara;\index{Puskara@Puṣkara} is driven to exile with Damayantī; wanders hungry and half-naked in the forest; loses Damayantī, etc. What follows is a story of how he wins back control on his body and on his life, and is finally united with his wife and restored to glory. 
\vskip 1.3pt

Similar is the story of Yayāti.\index{Yayati@Yayāti} It is primarily a story that conveys the timeless message of the value of self-restraint. {\sl Abhijñāna Śākuntala}\index{Sakuntalaofkalidasa@\textsl{Śākuntala} of Kālidāsa} is a heart rendering {\sl bildungsroman}. Devavrata,\index{Devavrata} “the noble scion of a royal family renounces, of his own accord, the kingdom which is his by right, and also vows to absolute celibacy. His sacrifice is so vital and the provocation to it so trivial that when he made this vow, flowers rained down on him from the sky and voices were heard from all the directions murmuring the words “Bhīṣma!\index{Bhisma@Bhīṣma} Bhīṣma!” (Sukthankar\index{Sukthankar, Vishnu Sitaram} 1957:46)
\vskip 1.3pt

Pollock fixes his attention on minor, insignificant episodes and juxtaposes them to form a --- at first glance, convincing --- context for his theory. Note that Pollock wrongly depicts Kaṁsa\index{Kamsa@Kaṁsa} --- Kṛṣṇa’s\index{Krsna@Kṛṣṇa}\index{Kamsa@Kaṁsa} maternal uncle --- as Kṛṣṇa's “brother”\endnote{Pollock quotes {\sl Harivaṁśa}\index{Harivamsa@\textsl{Harivaṁśa}} (65.77, 88) where Kṛṣṇa’s\index{Krsna@Kṛṣṇa} father Vasudeva\index{Vasudeva} is called the husband of Kaṁsa’s\index{Kamsa@Kaṁsa} father’s sister--- but this is clearly an aberration. Vasudeva is popularly Kaṁsa’s sister’s husband.}! Also note the sly and repeated usage of the word “{\sl bhrātṛ}” in the paragraph that follows --- a linguistic stratagem that draws a rather convincing picture of a “struggle among brothers”: 
\vskip 1.3pt

\begin{myquote}
“Here [in {\sl Mahābhārata}],\index{Mahabharata@\textsl{Mahābhārata}} two claimants, Yudhiṣṭhira\index{Yudhisthira@Yudhiṣṭhira} and Duryodhana,\index{Duryodhana} contend for the succession to the Kuru throne. They are related as {\sl\bfseries bhrātṛvya}, first cousins in the male line, called also {\sl\bfseries bhrātṛ}, which signifies in the first instance “{\bf brother}”, half- as well as full (Yudhiṣṭhira refers to Duryodhana’s father as {\sl\bfseries pitṛ}, “father”). Their contention leads to the division of the kingdom and eventually to the exile of Yudhiṣṭhira. The dispute is finally resolved only by a cataclysmic struggle that spirals out to engulf the entire Indian world, besides resulting in the extermination of Duryodhana and all his ninety-nine brothers. The {\sl Harivaṁśa},\index{Harivamsa@\textsl{Harivaṁśa}} on the other hand, tells the story of Kṛṣṇa,\index{Krsna@Kṛṣṇa} the {\sl\bfseries bhrātr} of the usurping prince Kaṁsa.\index{Kamsa@Kaṁsa} He is likewise exiled, in a sense, but returns to slay the tyrant and reinstate the deposed ruler, the father of Kaṁsa himself. King Nala,\index{Nala} in the Nalopākhyāna, finds his position usurped (by his {\sl\bfseries bhrātṛ}) and is driven into exile…”
\hfill Pollock (2007a:10)
\end{myquote}

If these are factual inaccuracies, the historical climate in which Pollock chooses to situate the {\sl Rāmāyaṇa} is also questionable. This is looked into in the next section.

\subsection{Historical Climate}\label{sec1.1.2}

Working now from an {\sl a priori} that kingship\index{kingship in poetry} is indeed a central concern of Indian epic poetry, Pollock wonders why it was that Indian poets were interested in this particular subject:

\begin{myquote}
“We are naturally led to wonder why this question should assume such importance for the Indian epic. We are not dealing here, as in other epic traditions, with just the heroic deeds of warrior kings, but with the nature and function of kingship as such; and these questions are not tangentially significant but central to the structure of the epic...”

\hfill Pollock (2007a:10)
\end{myquote}

\vskip .2cm

In order to arrive at some plausible answers, Pollock tries to reconstruct the historical setting of the {\sl Rāmāyaṇa}: According to him, “major urban cities of Aryan India came into existence” in approx. 700 B.C.E; so, the three-four hundred years following the middle Vedic age (approx. 800 B.C.E, according to him) were, Pollock writes, a crucial period in the Indian society --- it was at this juncture that some “fundamental and lasting transitions occurred in the Indian way of life”. One such “transition” was, he claims, the “extraordinary expansion of the role of the king” (Pollock 2007a:11). 

\vskip .1cm

It is improbable, says Pollock, that exclusive royal dynasties (associated with monarchy) existed in the Vedic period --- there are no accounts of struggles for succession to the throne within families, there is not a word that carries the connotation of “royal dynasties”. So, he says, it must be a later development. But by the time the {\sl Rāmāyaṇa}\index{Ramayana@\textsl{Rāmāyaṇa}} was composed, he writes, a type of monarchy had come about that was unlike anything before --- the welfare of the State had come to depend exclusively on the king, and political power had become entirely concentrated in the hands of royalty. For the first time, he says, it became a practice to transfer royal power through heredity (Pollock 2007a:13). Consequently, many complications arose in its transference, and so the epic poets found it useful, he says, to occupy themselves with its concerns.

Now, Pollock’s conclusion rests primarily on three chronological details that, in turn, rest on the disreputed Aryan Invasion Theory (AIT)\index{Aryan Invasion Theory (AIT)} / Aryan Migration Theory (AMT):\index{Aryan Migration Theory (AMT)}
\begin{itemize}
\itemsep=0pt
\item[(a)] that the Vedic period extends from 1500 B.C.E --- 500 B.C.E; 

\item[(b)] major cities of “Aryan India” can be dated to approx 700 B.C.E.; and

\item[(c)] the {\sl Rāmāyaṇa} was composed in the “last centuries B.C.E”. Let us examine these closely.\\[-.9cm]
\end{itemize}

\subsubsection{Dating of the Veda-s \& cities of India}\label{sec1.1.2.1}
\index{Veda-s, dating of}

\vskip -.2cm

A brief summary of the AIT is in order. David Frawley writes, \index{Frawley, David}

\vskip .1cm

\begin{myquote}
“According to this account [AIT] ... India was invaded and conquered by nomadic light-skinned Indo-European tribes (Aryans) from Central Asia around 1500-1000 BC. They overran an earlier and more advanced dark-skinned Dravidian civilization from which they took most of what later became Indian civilization. In the process they never gave the indigenous people whom they took their civilization from, the proper credit but eradicated all evidence of their conquest. All the Aryans really added of their own was their language (Sanskrit, of an Indo-European type) and their priestly cult of caste that was to become the bane of later Indic society. The so-called Aryans, the original people behind the Vedas, the oldest scriptures of Hinduism,\index{Hindu!scriptures} were reinterpreted by this modern theory [i.e. AIT] not as sages and seers - the {\sl rishis} and {\sl yogis} of Hindu\index{Hindu!tradition} historical tradition - but as primitive plunderers. Naturally this cast a shadow on the Hindu religion and culture as a whole. The so-called pre-Aryan or Dravidian civilization is said to be indicated by the large urban ruins of what has been called the “Indus Valley culture”\index{Indus Valley culture@``Indus Valley culture''} (as most of its initial sites were on the Indus river), or “Harappa and Mohenjodaro”,\index{Mohenjo-daro}\index{Harappa} after its two initially largest sites. In this article we will call this civilization the “Harappan” as its sites extend far beyond the Indus river. It is now dated from 3100-1900 BC. By the invasion theory Indic civilization is proposed to have been the invention of a pre-Vedic civilization and the Vedas, however massive their literature, are merely the products of a dark age following its destruction. Only the resurgence of the pre-Vedic culture in post-Vedic times is given credit for the redevelopment of urban civilization in India...”

\hfill Frawley (1997:3)
\end{myquote}

In the centuries following the establishment of AIT\index{Aryan Invasion Theory (AIT)} theory, the {\sl Veda}-s\index{Vedas} were dated based either on linguistic considerations or on astronomical calculations, and admittedly, “the opinions of best researchers in the matter of the age of the {\sl Ṛgveda}\index{Rgveda@\textsl{Ṛgveda}!age of} differed not by a few centuries but by a few thousands of years” (Ketkar 1987:270). A review of their conclusions, therefore, is in order. 

Max Muller\index{Muller, Max} first dated the {\sl Ṛgveda}, on apparently linguistic grounds, to approx 1200 B.C.E What he did was classify ancient literature sequentially into four distinct periods --- period of the {\sl chandas, mantra, brāhmaṇa}, and the {\sl sūtra}. First of all, the most important names of the Sūtra period\index{Sutraperiod@Sūtra period} were identified as Śaunaka\index{Saunaka@Śaunaka} and Kātyāyana.\index{Katyayana@Kātyāyana} Next, Kātyāyana was claimed (not on absolute grounds) as identical with Vararuci.\index{Vararuci} Afterwards, according to a detail furnished by Somadeva’s\index{Somadeva} {\sl Kathā-sarit-sāgara}\index{Kathasaritsagara@\textsl{Kathā-sarit-sāgara}} that Vararuci was a minister to the Nanda-s of Pāṭalīputra,\index{Pataliputra@Pāṭalīputra} Max Muller pinned Vararuci/Kātyāyana’s\index{Katyayana@Kātyāyana} date to 325 B.C.E. Calculating on these lines, Muller concluded that the {\sl sūtra} period extended from approx. 600 B.C.E to 200 B.C.E. Moving backwards, the Brāhmaṇa period\index{Brahmana period@Brāhmaṇa period} extended from 800 B.C.E to 600 B.C.E, the Mantra period\index{Mantra period} from 1000 B.C.E to 800 B.C.E and finally the Chandas\index{Chandas period} from 1200 B.C.E to 1000 B.C.E. Based on these calculations, Max Muller concluded that the {\sl Ṛgveda} belonged to 1200 B.C.E.  

Muller’s theory, it is obvious, is purely {\sl ad hoc}. Firstly, the theory stands entirely on the identity of Kātyāyana with Vararuci which is hardly beyond question; it is, in fact, pure speculation. Secondly, assigning 200 years for a literary period is extremely arbitrary and without grounds. Winternitz remarks, “Now it is clear that the presumption of 200 years for each of the literary epochs in the birth of the Veda is purely arbitrary... it was strangely forgotten on how weak a footing the prevailing view actually stood...” (Ketkar 1987:272) Eventually, Max Muller withdrew his own claim. He said,

\begin{myquote}
“If now we ask as to how to fix the dates of these periods, it is quite clear that we cannot hope to fix a {\sl terminum a qua}. Whether the Vedic hymns were composed in 1000 or 2000 or 3000 B.C. no power on earth will ever determine”
\hfill Muller (1979:138)
\end{myquote}

Unfortunately, in spite of such a clear-cut retreat by the clergy himself, his earlier hypothesis holds solid ground; many --- if not most --- Western scholars and their Indian followers continue to swear by 1200 B.C.E as the date of the {\sl Ṛgveda}.\index{Rgveda@\textsl{Ṛgveda}!date of} 

On the other hand, two important scholars,\, Jacobi\index{Jacobi, Hermann} and Balagangadhara Tilak,\index{Tilak, Balagangadhara} worked to date the Veda from a different angle --- that of astronomy.\index{astronomy, dating of the Vedas using} In 1899, they worked simultaneously, yet separately --- one in Bonn, and the other in Bombay --- to come to almost identical conclusions. Accordingly, they noticed that in the Brāhmaṇa-s, the Kṛttikā constellation\index{Krttikaconstellation@Kṛttikā constellation} coincided with the Meṣarāśi;\index{Mesarasi@Meṣarāśi} but in the \hbox{Saṁhitā-s}, the Meṣa coincided with the Mṛgaśirā.\index{Mrgasira@Mṛgaśirā} A change of Meṣa from Kṛttikā to Mṛga would take at least 2000 years. Calculating on these lines, the two scholars dated the {\sl Ṛgveda}\index{Rgveda@\textsl{Ṛgveda}!date of} to beyond 5000 B.C.E.  When these results were published, “a hue and cry was raised by the scholars against such a heretic step...” (Ketkar 1987:273), and today, it is barely recognized as valid in the academic circles. Be that as it may, recent advancement in archeological and geological survey has demonstrated --- with ample proof--- that the {\sl Ṛgveda}\index{Rgveda@\textsl{Ṛgveda}} must be of an earlier date. Peter Clift\index{Clift, Peter} and others show, for example, that the Sarasvatī river\index{Sarasvatiriver@Sarasvatī river} in its full-fledged flow (as described in the {\sl Ṛgveda}) can be dated to before 47,000 B.C.E: 

\begin{myquote}
“Our provenance studies now allow some important constraints to be placed on reconstruction of the river systems since the mid-Holocene. Our data show that the Yamuna likely flowed west, not east as it does now, at least prior to 49 ka [ka = kilo-annum, thousand years]. Such a change in drainage pattern is possible because the Yamuna reaches the Himalayan foreland close to the crest of the drainage divide. Why the switch from Indus to Ganges occurred is unknown but could reflect a number of processes diverting the river east, such  as an avulsion event driven by autocyclic processes, as seen for example by the 120 km shift of the Kosi River in 2008. The Beas River was delivering material directly to Tilwalla prior to 10 ka, which in turn would require the Sutlej to have flowed into the Ghaggar-Hakra east of Marot. This means the northern Thar region must have been an area with several major confluences and a large river with a combined flow arguably sufficient to reach as far as the Arabian Sea … While drainage from the Yamuna may have been lost from the Ghaggar-Hakra well before development of the Harappan\index{Harappa} Civilization, flow from the Beas and Sutlej may have been more recent in Cholistan, if still prior to 10 ka. Loss of these rivers might be expected to have had a catastrophic effect on sustaining settlement in this region, but our evidence\index{evidence!archeological} argues against this. Water in the small Ghaggar-Hakra\index{Ghaggar-Hakra} (or Sarasvati)\index{Sarasvatiriver@Sarasvatī river} River would have been further reduced by monsoon weakening from 4.2 ka (Enzel et al., 1999; Staubwasser et al., 2003; Wünnemann et al., 2010), but evidence for dramatic changes in water sources was much earlier. While drainage capture is dramatic in the eastern Indus Basin in the late Quaternary, it appears to have occurred prior to human settlement and not to have directly caused the Harappan\index{Harappa} collapse.”
\hfill Clift {\sl et al} (2012:213)
\end{myquote}

At the very least, ample proof has been given for the “Harappan” civilization’s identity with the Vedic civilization\endnote{See the works of Danino\index{Danino, Michel} (2010), B. B. Lal (2002), David Frawley\index{Frawley, David} (1997), Chakravarthy (1990), A.K.Sharma (1974), etc.}. Equipped with the new data, then, one can confidently say that the “major cities of India came into existence”, not in 700 B.C.E, but, {\sl at the very least}, 2600 B.C.E. For at least four millennia before, writes Michel Danino,\index{Danino, Michel} many regions on the Indo-Gangetic belt had already harbored settled village communities --- settled, but slowly evolving new practices of agriculture, technology (metallurgy in particular), and crafts. In his words, 

\begin{myquote}
“... Extensive, planned cities, rising almost at the same time hundreds of kilometers apart, fully functional by 2600 B.C. and interacting with each other through a tight network....” \hfill Danino (2010:83)
\end{myquote}

If the chronology is corrected, Pollock’s theories, it can be shown, do not hold water at all.\\[-25pt]

\subsubsection{Royal dynasties \& the Veda-s}\label{sec1.1.2.3}

For Pollock, “there had actually (never) been any clear conception of an exclusive royal dynasty in the Vedic period”,\index{Vedic period} and therefore, 

\begin{myquote}
“...we have good reason to suppose that the special prominence dynasties and dynastic succession acquire in the epic texts is at least partly the result of major changes in the structure of political power in late vedic times.”
\hfill Pollock (2007a:13)
\end{myquote}

This is the primary reason he cites for Vālmīki’s\index{Valmiki@Vālmīki} purported exclusive treatment of kingship in the {\sl Rāmāyaṇa}. Contrarily, references to dynastic kings and hereditary monarchy abound in the Veda-s. Many passages of the {\sl Ṛgveda}\index{Rgveda@\textsl{Ṛgveda}!dynasties mentioned in} are proof of this --- we see, for example, that the throne is passed on from father to son for at least four generations among the Tritus (Sharma 1988:114)\endnote{See Majumdar (2010:356), Macdonell and Keith (1912:211).}. The {\sl Śatapatha-brāhmaṇa} refers to a kingdom of ten generations, {\sl daśa-puruṣam rājyam} ({\sl Śatapatha-brāhmaṇa}\index{Satapathabrahmana@\textsl{Śatapathabrāhmaṇa}}\index{Brahmana@\textsl{Brāhmaṇa}!Satapatha@\textsl{Śatapatha}} 12.9.3.1-3), and we hear of the dynastic rulers, \hbox{Pārikṣita-s} and the rulers of the Janaka’s line. In the {\sl Aitareya-brāhmaṇa}\index{Aitareyabrahmana@\textsl{Aitareyabrāhmaṇa}} \index{Brahmana@\textsl{Brāhmaṇa}!Aitareya@\textsl{Aitareya}} is a reference to the birth of an heir to the throne ({\sl Aitareya-brāhmaṇa} 8.9) and we see also that heredity monarchy did not preclude the principle of election or popular selection\endnote{Cf. Raychaudhuri (1953), pp. 160-61, and n.8 {\sl on} p. 160.}--- the tale of Devāpi\index{Devapi@Devāpi} and Śantanu\index{Santanu@Śantanu} is an illustration of the fact that choice was often limited to the royal-family members ({\sl Nirukta}\index{Nirukta@\textsl{Nirukta}} 2.10). We also see that in the {\sl Śatapatha-brāhmaṇa},\index{Satapathabrahmana@\textsl{Śatapathabrāhmaṇa}}
\index{Brahmana@\textsl{Brāhmaṇa}!Satapatha@\textsl{Śatapatha}} Duṣṭarītu Pauṁsāyana\index{Dustaritu Paumsayana@Duṣṭarītu Pauṁsāyana} is banished from his kingdom that had come to him through ten generations ({\sl Śatapatha-brāhmaṇa} 13.9.3.1; cf. {\sl Aitareya-brāhmaṇa}\index{Aitareyabrahmana@\textsl{Aitareyabrāhmaṇa}}\index{Brahmana@\textsl{Brāhmaṇa}!Aitareya@\textsl{Aitareya}} 8.10). A short list of the references to monarchs of a prominent dynasty of {\sl Ṛgveda},\index{Rgveda@\textsl{Ṛgveda}!dynasties mentioned in} the Bharata dynasty, is as follows (Talageri 2000:142):\index{Talageri, Srikanth}
\begin{enumerate}
\itemsep=0pt
\item Bharata:\index{Bharata (of \textsl{Ṛgveda})} 6.16.4;
\item Devavāta:\index{Devavata@Devavāta} 3.23.2,3; 

4.15.4; 

6.27.7; 

8.18.22;

\item Sṛñjaya:\index{Srnjaya@Sṛñjaya} 4.15.4; 

6.27.7; 6.47.25;
\item Vadhryaśva:\index{Vadhryasva@Vadhryaśva}  6.61.1; 

10.69.1,2,4,5,9,12. 
\item Divodāsa:\index{Divodasa@Divodāsa} 1.112.14; 1.16.18; 1.19.4; 1.30.7,10; 

2.19.6; 

4.26.3; 4.30.20; 

6.16.5; 6.26.5; 6.31.4; 6.43.1; 6.47.22,23; 6.61.1,

8.18.25; 8.103.2; 8.9.61;

\item Pratardana:\index{Pratardana} 6.26.8; 

7.33.14;

\item Pijavana:\index{Pijavana} 8.18.22,23,25;

\item Devaśravas:\index{Devasravas@Devaśravas} 3.23.2,3;

\item Sudās:\index{Sudas@Sudās} 1.47.6; 1.63.7; 1.12.19;

3.53.9,11;

5.53.2;

7.8.5,9,15,17,22,23,25; 7.19.3,6; 7.20.2; 7.25.3; 7.32.10; 7.33.3; 7.53.3; 7.60.8,9; 7.64.3; 7.83.1,4,6-8; 

\item Sahadeva:\index{Sahadeva (of \textsl{Ṛgveda})} 1.100.17;

4.15.7-10;
\item Somaka:\index{Somaka} 4.15.9;
\end{enumerate}

Talageri shows that while these kings are descendents of Bharata, they are also 1. called Pūru-s:\index{Puru-s@Pūru-s} according to the Purāṇa-s, the \hbox{Bharata-s}\index{Bharata-s} are a branch of the Pūru-s; this is further confirmed in the {\sl Ṛgveda}\index{Rgveda@\textsl{Ṛgveda}} where both Divodāsa\index{Divodasa@Divodāsa} ({\sl Ṛgveda} 1.130.7) and Sudās\index{Sudas@Sudās} (1.63.7) are called Pūru-s, and Parucchepa Daivodāsi\index{Parucchepa Daivodasi@Parucchepa Daivodāsi} repeatedly speaks as a Pūru ({\sl Ṛgveda} 1.129.5; 131.4). The other prominent dynasty in the {\sl Ṛgveda} is the Tṛkṣi dynasty\index{Trksi dynasty@Tṛkṣi dynasty} of Māndhātā,\index{Mandhata@Māndhātā} identifiable as a branch of the Ikṣvāku-s.\index{Iksvaku-s@Ikṣvāku-s} An extensive analysis of the dynasties is made in Talageri (2000). From the Purāṇa-s, a more exhaustive list can be compiled--- although the value of historical material from the Purāṇa-s is sometimes contested in Western academia.

All of the afore-quoted data goes to show that the “historical climate” in which Pollock wishes to situate the {\sl Rāmāyaṇa} is absolutely baseless. No substantial proof exists at all to show that kingship and its “attendant problems” were “new and, in their very nature, urgent” in the purported epic period. It is, at best, a speculative theory that may have been worked backwards to fit the idea of kingship as the “core” of Indian epics.\\[-21pt]  

\section{The Problematic: An analysis}\label{sec1.2}

After establishing heredity\index{heredity monarchy} monarchy as a matter of {\sl utmost} concern of Indian epic poets, Pollock discusses three of its major “problems” at length:\\[-21pt]   

\subsection{{\sl\bfseries Yauvarājya} (crown-prince-hood):}\label{sec1.2.1}

{\sl Yauvarājya},\index{yauvarajya@\textsl{yauvarājya}} defines Pollock, is a situation where a still-reigning king must appoint his successor. Quoting Jack Goody, he draws attention to the “difficulties” of this type of “premortem succession”:

\begin{myquote}
 “... For the sharp contrast that exists between king and ex-king... makes it well-nigh impossible for a man easily to cast off the authority he has held by right of birth... this aura that attaches to the ex-king creates problems... which can often be resolved only by the banishment or the killing of the king.” 
\hfill (cited in Pollock 2007a:12)
\end{myquote}

Quoting Tod, he continues,

\begin{myquote}
“It is a rule (for an ex-king) never to enter the capital after abandoning the government; he is virtually defunct; he cannot be a subject, and he is no longer a king.”
\hfill (cited in Pollock 2007a:12)
\end{myquote}

It must be noted here at once that the collection of essays that Pollock refers to --- {\sl Succession to High Office} (1979) edited by Jack Goody\index{Goody, Jack} --- deals specifically with the difficulties of transfer of power in Basutoland, Buganda, Northern Unyamwezi, and Gonja. In the context of ancient India --- where patricide was virtually absent --- it is simply irrelevant. Ajātaśatru\index{Ajatasatru@Ajātaśatru} and Aśoka\index{Asoka@Aśoka} — of a later age — are the only two kings associated with patricide; otherwise, very rarely are such accounts to be found in ancient India. We see that Daśaratha,\index{Dasaratha@Daśaratha} for example, looks anxiously forward to the day he would see his son crowned as prince-regal\endnote{Examples can, of course, be multiplied – Kālidāsa,\index{Kalidasa@Kālidāsa} Bāṇabhaṭṭa,\index{Banabhatta@Bāṇabhaṭṭa} etc.}:
\begin{quote}
{{\sl eṣā hy asya parā prītir hṛdi saṁparivartate}} |\\
{{\sl kadā nāma sutaṁ drakṣyāmy abhiṣiktam ahaṁ priyam}} || 

\hfill({{\sl Rāmāyaṇa}}\relax \ 2.1.36)
\end{quote}

\begin{myquote}
“In his heart he cherished this single joyous thought: When shall I see my dear son consecrated?” [{\sl Trans.} Pollock]
\end{myquote}

With utter contempt for facts, Pollock writes, “the Ikṣvāku\index{Iksvaku-s@Ikṣvāku-s} dynasty confronted this [virtually nonexistent] problem [of {\sl yauvarājya}]\index{yauvarajya@\textsl{yauvarājya}} by the institutionalized ritual exile of the king” (Pollock 2007a:12) ({\sl words in [~] supplied by us})--- by his entering of the {\sl vānaprasthāśrama}.\index{vanaprastha@\textsl{vānaprastha}} He implies {\sl vānaprasthāśrama} was a stratagem that (only) the  Ikṣvāku-s adopted to allow a “dignified” exit to the ex-king. This is, of course, a rather ludicrous claim. It would be useful to trace here the idea of {\sl vanaprasthāśrama} to the \hbox{Veda-s} to illustrate that it was certainly not an uncommon practice --- and certainly not unique to the Ikṣvāku-s. In the Vedic texts, the word “{\sl vaikhānasa}”\index{vaikhanasa@vaikhānasa} refers to “{\sl vānaprastha}”. P. V. Kane notes,\index{Kane, P. V.} 

\begin{myquote}
{{\sl “There is nothing in the Vedic Literature expressly corresponding to the vānaprastha. It may however be stated that the Tāṇḍya Mahābrāhmaṇa\index{Tandyamahabrahmana@\textsl{Tāṇḍya-mahābrāhmaṇa}} (14. 4. 7) says that {\sl vaikhānasa} sages were the favorites of Indra\index{Indra} and that one Rahasya Devamalimluc killed them in a place called Munimaraṇa. Vaikhānasa means ‘vānaprastha’ in the sūtras and it is possible that this is the germ of the idea of vānaprastha. …”}}
\hfill  Kane (1941:418)
\end{myquote}

Kane cites another reference to {\sl vānaprasthāśrama}\index{vanaprastha@vānaprastha} in {\sl Aitareya-brāhmaṇa}\index{Aitareyabrahmana@\textsl{Aitareyabrāhmaṇa}}\index{Brahmana@\textsl{Brāhmaṇa}!Aitareya@\textsl{Aitareya}}
 33. 11:

\begin{myquote}
“What (use is there) of dirt, what use of antelope skin, what use of (growing) the beard, what is the use of {\sl tapas}? O! brāhmaṇas! Desire a son, he is a world that is to be highly praised.” Here it is clear that {\sl ajina} refers to brahmacarya, śmaśrūṇi to vānaprasthas (since according to Manu VI. 6 and Gaut III. 33, the vānaprasthas had to grow his hair, beard and nails). Therefore `malam’ and `tapas' must be taken respectively as indicating the householder and the saṁnyāsin. A much clearer reference to three āśramas occurs in the Chāndogya\index{Chandogyopanishad@\textsl{Chāndogyopaniṣad}} Up. II. 23. 1 --- “there are three branches of dharma, the first (is constituted by) sacrifice, study and charity (i. e. by the stage of householder), the second is (constituted by the performance of) {\sl tapas} (i. e. the vānaprastha), the third is the brahmacārī staying in the house of his teacher and wearing himself out till death in the teacher's house; all these attain to the worlds of the meritorious; but one who (has correctly understood brahman) and abides in it attains immortality.”
\hfill Kane (1941:420)
\end{myquote}

So the institution of {\sl vānaprastha} was a well-known one. Other kings who accepted {\sl vānaprastha}\index{vanaprastha@\textsl{vānaprastha}} include Bṛhadaśva,\index{Brhadasva@Bṛhadaśva} Trayyāruṇa,\index{Trayyaruna@Trayyāruṇa} Viśvāmitra,\index{Visvamitra@Viśvāmitra} Kapila,\index{Kapila} Bali,\index{Bali} Manu,\index{Manu} Saṁyāti,\index{Samyati@Saṁyāti} Yayāti,\index{Yayati@Yayāti} Devāpi,\index{Devapi@Devāpi} etc\endnote{See Patil (1946).}. That it was amply respected even at a later age is deducible from the Greek traveller, Megasthenes’\index{Megasthenes} account of it --- he writes the {\sl vānaprasthas}\index{vanaprastha@\textsl{vānaprastha}} ({\sl hylobioi}) who “live in woods where they subsist on leaves of trees and wild fruits, and wear garments made from the bark of tress” were the most respected of the society (McCrindle 2000:102).  We see, again, that Pollock tries to locate the {\sl Rāmāyaṇa} in an incorrect historical climate that is not substantiated with proof of any sort.\\[-21pt] 
\vskip 4pt

\subsection{Interstate Marriage}\label{sec1.2.2}

Another “problem”--- or reason, rather--- that partly led to the “later” hereditary transference of power was, cites Pollock, interstate marriage. Interstate marriage was basically a political alliance; it could occur “only if high office is transferred within a single dynasty.” Pollock writes further, 

\begin{myquote}
“Closely related to this is the politically significant practice of {\sl rājyaśulka},\index{rajyasulka@\textsl{rājya-śulka}} the bride price consisting of the kingship (or kingdom)... It is clearly practicable only when kingship is proprietarily controlled.”

\hfill Pollock (2007a:12)

\vskip -1cm
\end{myquote}


\subsection{Dynastic Struggle}\label{sec1.2.3}

The “final, and critical, intrinsic problem of kingship”, says Pollock, is the divisive and usually violent dynastic struggle that accompanies the hereditary transference of power. In the earlier tradition, Pollock says (referring to the {\sl Mahābhārata}\index{Mahabharata@\textsl{Mahābhārata}} and the {\sl Harivaṁśa}),\index{Harivamsa@\textsl{Harivaṁśa}} there was only one method to resolve this difficulty of transference of hereditary power\index{hereditary power} --- that of armed combat. He writes,

\begin{myquote}
“The {\sl Mahābhārata} is no doubt sensitive to the desperate dilemma of living made possible only through killing. But its interrogations are indecisive; it can conceive of no solution except the final one in heaven. Political violence is no less necessary for its impossibility. That the fratricidal\index{fratricide!doctrine of} doctrine is so often and positively enunciated and defended in the {\sl Mahābhārata} suggests that for this and the other epic stories, as for the historical kings of ancient India, the acquisition and retention of political power ultimately if tragically superseded all other concerns...”

\hfill Pollock (2007a:18) 
\end{myquote}

Vālmīki,\index{Valmiki@Vālmīki} on the other hand, he says, was dissatisfied with this, and suggested another solution to this problem: submission to hierarchy. Accordingly, the only way to obviate deadly antagonism (between family members) was by the doctrine of “unqualified submission” of the younger to the elder prince, the eldest to his father, and so on. So, by setting forth the example of Bharata\index{Bharata} who chose to bow before his brother and not contest the throne (when everyone expected otherwise), Vālmīki,\index{Valmiki@Vālmīki} according to Pollock, made the first literary attempt to “moralize” the exercise of political power. In his words,

\begin{myquote}
“For civilized society, the poet inculcates, by positive precept and negative example, and with a sometimes numbing insistence, a powerful new code of conduct: hierarchically ordered, unqualified submission.''

\hfill Pollock (2007a:16)
\end{myquote}

For Vālmīki, then, he says, violence was literally the strategy of the inhuman, and this, Pollock demonstrates with examples of Sugrīva\index{Sugriva@Sugrīva} and Vālin\index{Valin@Vālin} and also of Vibhīṣaṇa\index{Vibhisana@Vibhīṣaṇa} and Rāvaṇa:\index{Ravana@Rāvaṇa} 
\begin{itemize}
\item[(a)] Sugrīva and Vālin:

Sugrīva’s position, Pollock writes, was like that of Rāma --- he was ousted from his kingdom by his brother, Vālin, and the throne was occupied by the latter. With Rāma’s aid, Sugrīva\index{Sugriva@Sugrīva} {\sl forcibly} seized the throne of the monkeys; here, Rāma had no qualms in deploying arms, says Pollock, for he was dealing with an infringement of righteousness on the part of a monkey. 

\newpage

\item[(b)] Vibhīṣaṇa and Rāvaṇa.\index{Ravana@Rāvaṇa}\index{Vibhisana@Vibhīṣaṇa} 

Vibhīṣaṇa, in a similar plight, takes refuge under Rāma, (not, Pollock remarks, for altruistic reasons, but out of his “desire for the throne”.) When he expresses before Rāma his wish for the throne, the latter immediately crowns him as the king of Laṅkā.\index{Lanka@Laṅkā} Here, Pollock wonders if Rāma does so “in order to gain Vibhīṣaṇa’s continued support”. Rāma crowns him again after Rāvaṇa’s\index{Ravana@Rāvaṇa} death. Here also, Rāvaṇa’s unscrupulous ways are cited as the reason for the use of the sword. As a side note, Pollock observes that there is no mention of Vaiśravaṇa\index{Vaisravana@Vaiśravaṇa} who is “theoretically the rightful heir to the throne” and subtly implies that there is injustice at play here. In conclusion, he says that “submission to hierarchy” was Vālmīki’s\index{Valmiki@Vālmīki} way of resolving the complications that arose in the transference of hereditary power. 
\end{itemize}

\subsubsection{An(other) Incorrect “Literary” \& “Historical” Climate}\label{sec1.2.3.1}

Pollock’s deductions are, {\sl again}, drawn primarily by situating the {\sl Rāmāyaṇa} in an incorrect literary and historical climate. Here, again, Pollock’s {\sl Rāmāyaṇa} finds itself located in the general genre of epics:

\begin{myquote}
“... the genre itself and its primary social context restricted Vālmīki,\index{Valmiki@Vālmīki} like his predecessors and contemporaries to a set of themes. But when we compare the {\sl Rāmāyaṇa} with other examples of epic literature, it seems evident that Vālmīki found the previous treatments deficient not only aesthetically but ethically as well”
\hfill Pollock (2007a:15)
\end{myquote}

But what, exactly, are “previous treatments”? It is widely accepted that Vālmīki is the first poet, the {\sl ādikavi}. Bhoja,\index{Bhoja} for example, reverentially refers to him as the foremost among eloquent poets:

\begin{quote}
{\sl madhumaya-bhaṇitīnāṁ mārgadarśī maharṣiḥ} 

\hfill ({\sl Campū-rāmāyaṇa}\index{Campuramayana@\textsl{Campū-rāmāyaṇa}} 1.8)
\end{quote}

Gaṅgādevī\index{Gangadevi@Gaṅgādevī} writes of Vālmīki’s footsteps as the first in the realm of poetry-writing:   

\begin{quote}
{\sl pṛthivyāṁ padya-nirmāṇa-vidyāyāḥ prathamaṁ padam} 

\hfill ({\sl Madhurā-vijaya}\index{Madhuravijaya@\textsl{Madhurāvijaya}} 1.5) 
\end{quote}

In Kṣemendra’s\index{Ksemendra@Kṣemendra} words, Vālmīki,\index{Valmiki@Vālmīki} like the foremost syllable ‘Om’, is the foremost poet:

\begin{quote}
{\sl oṁkāra iva varṇānāṁ kavīnāṁ prathamaḥ kaviḥ}

\hfill (verse 2, Epilogue post Uttara-kāṇḍa,\index{Uttarakanda@Uttara-kāṇḍa} {\sl Rāmāyaṇa-mañjarī})\index{Ramayanamanjari@\textsl{Rāmāyaṇa-mañjarī}}
\end{quote}

Pollock’s answer is: the {\sl Mahābhārata}\index{Mahabharata@\textsl{Mahābhārata}} and the {\sl Harivaṁśa}:\index{Harivamsa@\textsl{Harivaṁśa}}

\begin{myquote}
“As the {\sl Mahābhārata} and {\sl Harivaṁśa} makes clear, the early epic tradition had acknowledged, if sometimes reluctantly, only one means for the resolution of political and dynastic conflict: armed combat.”

\hfill Pollock (2007a:15)
\end{myquote}

A considerable body of scholarly writing has been devoted to the question of precedence between the two texts, and it is now generally accepted that the {\sl Rāmāyaṇa} is “somewhat older than the {\sl Mahābhārata}”\index{Mahabharata@\textsl{Mahābhārata}} (Goldman 1984:38).\index{Goldman, Robert} Yet, in order to account for the “great synchrony” between the {\sl Mahābhārata} and the {\sl Rāmāyaṇa}, Pollock decides that the oral traditions of the particular works were “co-extensive processes” that were continuously interactive and cross fertilizing ---

\begin{myquote}
“[this and many other reasons] forces us to think of the two epic traditions as co-extensive processes that were underway throughout the second half of the first millennium B.C, until the monumental poet of the {\sl Rāmāyaṇa}, and the redactors of the {\sl Mahābhārata} authoritatively synthesized their respective materials and thereby in effect terminated the creative oral process.”
\hfill Pollock (2007a:43)
\end{myquote}

Pollock treats of the “great symphony” between the Ayodhyā-kāṇḍa\index{Ayodhyakanda@Ayodhyā-kāṇḍa} and the Sabhā-parvan\index{Sabhaparvan@Sabhā-parvan} in a separate segment; his list of “similarities”\index{misinterpretation!techniques of!locating ``similarities''} is as follows: 
\begin{itemize}
\item[(a)] In both the {\sl Rāmāyaṇa} and the {\sl Mahābhārata}, Pollock says, the chapters Ayodhyā-kāṇḍa and Sabhā-parvan constitute the “true commencement” of their respective stories: all that happens in the following chapters depend entirely on what happens here; and all that happened in the preceding chapters serve no other purpose than to prepare for this.

\item[(b)] It is in these chapters that the problem of royal power is abruptly brought into sharp focus: the decision of royal consecration leads to immediate opposition from the rival claimant --- Kaikeyī\index{Kaikeyi@Kaikeyī} in the {\sl Rāmāyaṇa}, and Duryodhana\index{Duryodhana} in the {\sl Mahābhārata}.\index{Mahabharata@\textsl{Mahābhārata}} 

\item[(c)] Kaikeyī’s trickery is similar to that of Śakuni’s\index{Sakuni@Śakuni} --- both of them wish for the long exile of their opponents, so that their {\sl protégé} (Bharata\index{Bharata} and Duryodhana, respectively) can “sink deep roots in the kingdom”. 

\item[(d)] In both cases, the one voice that could control the situation was silenced by a weakness born of excessive love --- Daśaratha’s\index{Dasaratha@Daśaratha} for Kaikeyī\index{Kaikeyi@Kaikeyī} and Dhṛtarāṣṭra’s\index{Dhrtarastra@Dhṛtarāṣṭra} for Duryodhana.  

\item[(e)] Both kings justify their actions (inactions, rather) on grounds of fatalism.

\item[(f)] Yudhiṣṭhira,\index{Yudhisthira@Yudhiṣṭhira} like Rāma, acts out of obedience to his father’s orders. 

\item[(g)] Like Lakṣmaṇa,\index{Laksmana@Lakṣmaṇa} Yudhiṣṭhira’s\index{Yudhisthira@Yudhiṣṭhira} brothers plead with him in vain to take to arms. Yudhiṣṭhira’s response is “uncannily similar” to that of Rāma’s.  
\end{itemize}

Surely, these “similarities” do not suffice to locate the two texts in a single time-frame? Most of the “similarities” --- especially the behavior of different characters --- are simply the characteristic features of the cultural context in which they are situated; in V. K. Gokak’s\index{Gokak, V. K.} words, “the surprising uniformity in the matter of the choice and treatment of themes” (Gokak 1979:72) simply defines the “Indian” in “Indian literature”. Yet, Pollock writes of Vālmīki’s\index{Valmiki@Vālmīki} work as if it were simply one link in a chain of many. 

While this constitutes an incorrect “literary climate”, an incorrect “historical climate” is the “rampant” dynastic struggle that Pollock constantly refers to --- in a Western context, “the Cain\index{Cain Syndrome, the} syndrome” is a foregone conclusion: many references can be found in the Genesis of the jealousy that Cain\index{Cain} bore towards Abel,\index{Abel} and finally, the murder of the latter by Cain. On the other hand, there are more instances of unity than fratricidal\index{fratricide} wars in ancient Indian literature; yet, Pollock chooses to project that syndrome on to the {\sl Rāmāyaṇa}: 

\begin{myquote}
“Everybody in the Ayodhyākāṇḍa\index{Ayodhyakanda@Ayodhyā-kāṇḍa} expects Bharata\index{Bharata} to mount a struggle for power: Daśaratha\index{Dasaratha@Daśaratha} ({\sl sarga} 4), Kausalyā\index{Kausalya@Kausalyā} (69), Guha\index{Guha} (78), Bharadvāja\index{Bharadvaja@Bharadvāja} (84-85), and, of course, Lakṣmaṇa\index{Laksmana@Lakṣmaṇa} (90). {\sl This was the established pattern of behavior.”}  
\hfill (Pollock 2007a:17) [{\sl italics ours}]
\end{myquote}

He makes these conclusions based on certain misinterpretations and mistranslations. Let us, therefore, consider each of Pollock’s claims in their own context separately under four headings viz. (a) Daśaratha (b) Bharadvāja (c) Kausalyā (d) Vibhīṣaṇa.\index{Vibhisana@Vibhīṣaṇa} {\sl\bfseries Please note that we proffer Pollock’s own translations} of the Sanskrit texts to expose the shortcomings in his translations/interpretations. 


\smallskip
\noindent
\textbf{(a) Daśaratha:}\index{Dasaratha@Daśaratha} Pollock’s interpretation of Daśaratha’s expectation for Bharata\index{Bharata} to mount a struggle for the throne is primarily based on verse 2.4.25ff (cited {\sl infra}). Let us examine the different events that unfold prior to this: 

In the said {\sl sarga}, we see that Daśaratha first calls a counsel and tells them that he seeks respite from his kingly duties, and wishes to entrust his subject’s welfare to the care of his son --- with, of course, the approval of the brahmins assembled in court (2.2.10). At this, the assembly applauds his decision, and rejoices “like peacocks at the rumble of a rain-laden cloud” (2.2.17). As it was the auspicious month of Caitra, Daśaratha\index{Dasaratha@Daśaratha} decides to crown Rāma as his heir on the day of Puṣya of the same month\endnote{Puṣya is considered the most auspicious especially for events like marriages or consecration as king.}. He asks sage Vaśiṣṭha to make the required preparations (2.3.4), and later informs his son of his decision. Afterwards --- after everybody empties the hall --- the king holds further consultation with his councilors. {\sl It is then that he realizes that the auspicious day of Puṣya falls immediately the next day} ({\sl Rāmāyaṇa} 2.4.1-2):

\begin{quote}
{{\sl gateṣv atha nṛpo bhūyaḥ paureṣu saha mantribhiḥ}} |\\
{{\sl mantrayitvā tataś cakre niścaya-jñaḥ sa niścayam}} ||

{\sl śva eva puṣyo bhavitā śvo’bhiṣecyas tu me sutaḥ} |\\
{\sl rāmo rājīva-tāmrākṣo yauvarājya iti prabhuḥ} ||
\end{quote}

\begin{myquote}
“After the townsmen had gone, the king held further consultation with his counselors. When he learned what they had determined the lord declared with determination: Tommorrow is Puṣya, so tomorrow my son, Rāma, his eyes coppery as lotuses, shall be consecrated as the crown-regent” [{\sl Trans.} Pollock]
\end{myquote}

So, he sends {\sl again} for Rāma to inform him of this {\sl new} decision: that he will be crowned heir-prince immediately the next day. Further, Daśaratha tells Rāma that he has lately had inauspicious and ominous dreams --- {\sl svapne paśyāmi dāruṇān} (2.3.27). When astrologers were consulted, they informed Daśaratha that his birth star was obstructed by hostile planets which implied that he may die or meet with other such dreadful misfortune --- {\sl rājā hi mṛtyum āpnoti ghoraṁ vāpadam ṛcchati} (2.4.19). Daśaratha is afraid, and wishes to see his son take the reins before any calamity strikes him. Bharata’s\index{Bharata} absence, we see, had nothing to contribute to the urgency of the situation. 

But before concluding the conversation, the king says the following (Daśaratha to Rāma) ({\sl Rāmāyaṇa} 2.4.25-27), 
\begin{quote}
{{\sl viproṣitaś ca bharato yāvad eva purād itaḥ}} |\\
{{\sl tāvad evābhiṣekas te prāpta-kālo mato mama}} || 25 ||\\
{\sl kāmaṁ khalu satāṁ vṛtte bhrātā te bharataḥ sthitaḥ} |\\
{\sl jyeṣṭhānuvartī dharmātmā sānukrośo jitendriyaḥ} || 26 ||\\
{\sl kintu cittaṁ manuṣyāṇām anityam iti me matiḥ} |\\
{\sl satāṁ ca dharma-nityānāṁ kṛta-śobhi ca rāghava} || 27 ||
\end{quote}

Pollock interprets it in order as follows:

\begin{myquote}
“I believe the best time for your consecration is {\sl precisely} while Bharata\index{Bharata} is away.” [{\sl italics ours}]
								 	 
“Granted your brother keeps to the ways of the good, defers to his elder brother, and is righteous, compassionate, and self-disciplined.”
		 
“Still, Rāghava, it is my firm belief that the mind of man is inconstant, even the mind of a good man constant in righteousness. Even such a man is best presented with an accomplished fact.”	
\end{myquote}

Pollock repeatedly alludes to this situation to draw attention to an “inner strife”\index{misinterpretation!techniques of!repeated allusions to ``inner strife''} between the members of the family: for example, in his summary of the Ayodhyā-kāṇḍa,\index{Ayodhyakanda@Ayodhyā-kāṇḍa} he writes, “... in a private meeting with his father [Rāma] learns that the ceremony is to take place the following day {\sl lest} Bharata have time to return and contest the succession.” (Pollock 2007a:6)

First of all, the word {\sl prāpta-kāla} must be interpreted as destiny, not “best time”; so 2.4.25 reads, “I think it is destiny that your consecration should take place when Bharata is away.” It is {\sl in a tone of regret} that Daśaratha\index{Dasaratha@Daśaratha} says these words, {\sl not craftily}. Secondly, Pollock translates the work {\sl kṛta-śobhi} (2.4.27) as “accomplished act”. In doing so, he blatantly ignores the epithet “{\sl śobhi}”. The word literally refers to the splendid nature/praiseworthiness of an act. Hence a more (contextually) appropriate translation would be as follows: “O Rāma! It is my opinion that minds of men are inconstant. But ({\sl kintu}) a virtuous person (like Bharata - {\sl dharma-nityānām satām ca}) will only praise a good that has been done ({\sl kṛta-śobhi}), [i.e. the consecration ceremony]”\endnote{Though some commentators have interpreted this verse in a manner similar to what Pollock has done, our interpretation is supported by a few commentaries too, for instance, {\sl Śiromaṇi}.\index{Siromani@Śiromaṇi}
\begin{quote}
{{\sl nanu sarveṣāṁ janānāṁ khaṇḍa-maṇḍaleśvarāṇāṁ ca tvad-ājñānuvartitvāt vighna-śaṅkāyā abhāvena kālāntare’bhiṣeko bhaviteti kiṁ druta-pravṛttyety ata āha – kiṁ nv iti | manuṣyāṇāṁ cittam anityaṁ prati-kṣaṇaṁ pariṇāmi ataḥ kiṁ nu cittaṁ kutsitam ity arthaḥ | iti me mataṁ niścayaḥ dharma-nityānāṁ dharma-pālakānāṁ satāṁ cittaṁ tu}} {\sl\bfseries kṛta-śobhi kṛtena kartavyakarmaṇaḥ siddhyaiva śobhate tacchīlam etenāvaśyaka-maṅgala-kṛtye vilambo na kārya vyañjitam} |\hfill ({\sl Śiromaṇi} on {\sl Rāmāyaṇa} 2.4.27)
\end{quote}}.

{\sl Literally}, there is no evidence\index{evidence!lack of} to support Pollock’s stand that Daśāratha expected Bharata to contest the throne. 

\noindent
\textbf{(b) Bharadvāja:}\index{Bharadvaja@Bharadvāja} Pollock relies on a few statements of Bharadvāja’s in order to conclude that he expected Bharata to mount a “struggle for kingship”. Let us consider those verses: Bharadvāja says these words to Bharata (Bharadvāja to Bharata)\index{Bharata} ({\sl Rāmāyaṇa} 2.90.10-13):\\[-21pt] 
\begin{quote}
{{\sl kim ihāgamane kāryam tava rājyaṁ praśāsataḥ}} |\\
{{\sl etad ācakṣva me sarvam na hi me śuddhyate manaḥ}} || 10 ||\\
{\sl suṣuve yam amitraghnam kausalyānanda-vardhanam} |\\
{\sl bhrātrā saha sabhāryo yaś ciram pravrājito vanam} || 11 ||\\
({\sl niyuktaḥ strī-niyuktena pitrā yo’sau mahā-yaśāḥ} |\\
{\sl vana-vāsī bhavetīha samāḥ kila caturdaśa})\endnote{Note that this verse is present in the Gita Press edition and not in the edition considered by Pollock and is hence given in brackets.} || 12 ||\\
{\sl kaccin na tasyāpāpasya pāpaṁ kartum ihecchasi} |\\
{\sl akaṇṭakam bhoktumanā rājyaṁ tasyānujasya ca} || 13 ||
\end{quote}

Pollock translates the verses in order thus:

\begin{myquote}
“What is your business in coming here when you should be ruling the kingdom? Explain this to me fully, for my mind is unclear on it.” 

“The son Kausalyā\index{Kausalya@Kausalyā} bore, a slayer of enemies and the one source of delight, has been banished for a long time to the forest with his wife and brother.”			 

“I trust you have no intention of harming this innocent man and his younger brother, thinking thereby to enjoy unchallenged kingship.”
\end{myquote}

If one stops here, as Pollock surely does, one may garner the wrong picture of the situation. Let us look, therefore, at the ensuing verses.\index{misinterpretation!techniques of!selective quotation} When Bharadvāja\index{Bharadvaja@Bharadvāja} thus addresses Bharata,\index{Bharata} the latter is reduced to tears ({\sl Rāmāyaṇa} 2.84.14):
\begin{quote}
{{\sl evam ukto bharadvājam bharataḥ pratyuvāca ha}} |\\
{\sl paryaśru-nayano duḥkhād vācā saṁsajjamānayā} || 
\end{quote}

\begin{myquote}
“So Bharadvāja spoke, and with tears in his eyes Bharata replied to him in a voice breaking with sorrow”. [{\sl Trans.} Pollock]  
\end{myquote}

Bharata defends himself and his intensions in the verses 14---18 to which Bharadvāja replies ({\sl Rāmāyaṇa} 2.84.20):

\newpage

\begin{quote}
{{\sl jāne caitan manaḥsthaṁ te dṛḍhīkaraṇam astv iti}} |\\
{\sl apṛccham tvāṁ tavātyartham kīrtiṁ samabhivardhayan} || 
\end{quote}

\begin{myquote}
“I knew what was in your heart and only questioned you to hear it openly confirmed and to see your fame magnified to the highest degree.” 

\hfill[{\sl Trans.} Pollock]
\end{myquote}

So, the sage did not at all expect a “dynastic struggle”; nor was this the “established custom”. He simply wanted “to hear it openly confirmed and to see your fame magnified to the highest degree”. It is with this same intention that Guha\index{Guha} poses the same question to Bharata,\index{Bharata} and finally praises him thus (Guha to Bharata) ({\sl Rāmāyaṇa} 2.83.5): 
\begin{quote}
{{\sl śāśvatī khalu te kīrtir lokān anucariṣyati}} |\\
{\sl yas tvam kṛcchra-gataṁ rāmaṁ pratyānayitum icchasi} || 
\end{quote}

\begin{myquote}
“You are prepared to bring back Rāma when he is in such a plight, and for this you shall win everlasting fame throughout the worlds”

\hfill [{\sl Trans.} Pollock]
\end{myquote}

\noindent
\textbf{(c) Kausalyā:}\index{Kausalya@Kausalyā}  When Bharata\index{Bharata} and Śatrughna\index{Satrughna@Śatrughna} return from visiting their maternal uncle and hear of the misfortunes that elapsed in their presence, they rush to meet Kausalyā. Upon seeing them, Kausalyā vents out (Kausalyā to Bharata) ({\sl Rāmāyaṇa} 2.69.6): 
\begin{quote}
{{\sl idaṁ te rājya-kāmasya rājyaṁ prāptam akaṇṭakam}} |\\
{\sl samprāptaṁ bata kaikeyyā śīghraṁ krūreṇa karmaṇā} || 
\end{quote}

Pollock translates it thus:

\begin{myquote}
“Then, in deep sorrow, Kausalyā spoke to Bharata. “You lusted for the kingship and here you have it unchallenged--- and how quickly Kaikeyī\index{Kaikeyi@Kaikeyī} secured it for you by her savage deed.”
\end{myquote}

From this, Pollock concludes Bharata’s purported hostility. It must be noted here that Kausalyā was in the clutches of extreme grief when she spoke these words --- she was hardly in possession of her mind. Vālmīki\index{Valmiki@Vālmīki} repeatedly draws attention to her miserable condition by the use of words such as “{\sl vicetanā}”, “{\sl malina-ambarā}”, “{\sl naṣṭa-cetanām}”, etc --- in Pollock’s own translation “insensible”, “dropped down unconscious in the anguish of her sorrow”, etc ({\sl Rāmāyaṇa} 2.69.3-6). 
\begin{quote}
{{\sl evam uktvā sumitrāṁ sā vivarṇā malināmbarā}} |\\
{\sl pratasthe bharato yatra vepamānā vicetanā} ||\\
({\sl sa tu rāmānujaś cāpi śatrughna-sahitas tadā} | 
{\sl pratasthe bharato yatra kausalyāyā niveśanam} ||\\ 
{\sl tataḥ śatrughna-bharatau kausalyāṁ prekṣya duḥkhitau} |)\\
{\sl paryaṣvajetāṁ duḥkhārtāṁ patitāṁ naṣṭa-cetanām} |\\
{\sl rudantau rudatīṁ duḥkhāt sametyāryāṁ manasvinīm} ||\\
{\sl bharataṁ pratyuvācedam kausalyā bhṛśa-duḥkhitā} || 
\end{quote}
Pollock’s translates it thus:

\begin{myquote}
“With this, she set out to Bharata, her face drained of color and her garment filthy, trembling and almost insensible.”

\vskip .1cm
“But at that same moment Rāma’s younger brother was setting out to Kausalyā's\index{Kausalya@Kausalyā} residence, accompanied by Śatrughna.”\index{Satrughna@Śatrughna}

\vskip .1cm
%Perhaps a note can be added here that Pollock gives bhāvārtha and not an exact translation.

“And when Śatrughna and Bharata saw Kausalyā, they were overcome with sorrow. She dropped down, unconsciousness in the anguish of her sorrow, and they took her in their embrace.” 
\end{myquote}

\vskip .1cm

When this context is kept in focus, it is reasonable to assume that Kausalyā\index{Kausalya@Kausalyā} bore no ill will towards Bharata,\index{Bharata} nor did she suspect him of trying to usurp the throne; these were harsh words spoken in extraordinary grief, and ought not to be taken literally.\index{misinterpretation!techniques of!interpreting without considering context} The interpretations of Pollock are far from dignified.

\vskip .1cm
\smallskip
\noindent
{\bf (d) Vibhīṣaṇa:}\index{Vibhisana@Vibhīṣaṇa} Pollock writes, “In Laṅkā\index{Lanka@Laṅkā} once more, the struggle for political power among brothers is settled by the sword”. This is simply a ridiculous claim--- the war at Laṅkā was hardly a “struggle for political power”.\index{misinterpretation!techniques of!finding ``struggle for political power''} It must be noted here that Kumbhakarṇa,\index{Kumbhakarna@Kumbhakarṇa} Rāvaṇa’s\index{Ravana@Rāvaṇa} brother, fought for him until the end --- sacrificed his life, too --- {\sl despite} the fact that he had utter scorn for Rāvaṇa’s misdeeds (6.63.1-21). Vibhīṣaṇa,\index{Vibhisana@Vibhīṣaṇa} too, took refuge under Rāma only {\sl after} Rāvaṇa\index{Ravana@Rāvaṇa} sneeringly rejected his advice and declared (in assembly) that a death penalty ought to be meted out to him (6.16.26).

Further, Pollock takes the statement, “There is no brotherly love among heroes” (7.11.12 as per Pollock), to make a strong case for himself --- but this statement was made by the {\sl rākṣasa}, Prahasta,\index{Prahasta} chief of Rāvaṇa’s army in the Uttara-kāṇḍa,\index{Uttarakanda@Uttara-kāṇḍa} and so by no means can it stand for the “established pattern” of the times. 

Also, this statement is taken out of its context to be misinterpreted by Pollock. Here, the situation is: Rāvaṇa’s maternal grandfather urges Rāvaṇa to seize the throne of Laṅkā --- by force, if necessary (for Rāvaṇa’s brother, Kubera,\index{Kubera} ruled Laṅkā then). Rāvaṇa {\sl rejects} the idea because “Kubera is my elder brother!”, and to this, Prahasta, echoing Rāvaṇa’s\index{Ravana@Rāvaṇa} maternal grandfather’s views, says, “there is no brotherly love among heroes” (7.11.9-12 as per Pollock).

How this statement of the {\sl rākṣasa} chief can come to be the “established pattern”,\index{misinterpretation!techniques of!claiming ``established pattern''} only Pollock can tell.\\[-21pt] 

\subsubsection{“A Problem of Narrative”}\label{sec1.2.3.2}

Another reason for Pollock’s anticipation of a “dynastic strife” is Bharata’s\index{Bharata} alleged legitimate claim to the throne. Pollock’s position may be summarized thus: 

According to Pollock (and according to “higher criticism”\index{Higher Criticism}\endnote{See Section 1.1.1.1.}), the {\sl Rāmāyaṇa} is a middle point of a work in progress: Vālmīki,\index{Valmiki@Vālmīki} he says, perhaps drew inspiration from a body of ballads or legends about heroism and self-sacrifice that are now irrecoverable; and the poem continued to grow even after Vālmīki fixed its essential contours. So, he says, we see a number of discrepancies in the work that abound in any work of synthesis. One major discrepancy, according to Pollock, is Bharata’s\index{Bharata} claim to the throne of Ayodhyā: 

We know that Mantharā\index{Manthara@Mantharā} tells Kaikeyī\index{Kaikeyi@Kaikeyī} that when Daśaratha\index{Dasaratha@Daśaratha} went to fight the war between the gods and the {\sl asura}-s, he took Kaikeyī with him. In the battle that followed, Daśaratha was struck unconscious, and Kaikeyī conveyed him out of it, and saved her husband. Out of gratitude, Daśaratha bestowed on her two boons which now Mantharā urges she use to install Bharata on the throne. This story, Pollock says, has an unusually contrived appearance:
\begin{itemize} 
\itemsep=1pt
\item[(a)] The presence of a queen at a battle is, he says, extraordinary, and virtually unparalleled in Sanskrit literature.  

\item[(b)] Kaikeyī, he says, oddly does not remember the incident at all, and even when she is reminded of it, she simply does not claim the boons as her due. 

\item[(c)] Mantharā\index{Manthara@Mantharā} tells her, “use the power of your beauty”: And this is what Kaikeyī does --- she “ensnares” him with just this power. She makes Daśaratha promise before the Gods that he will fulfill any wish of hers, and when he does this, “with quite a casual indifference to the narrative”, Pollock points out, Kaikeyī says “I will now claim the two boons you had once granted me”. 
\end{itemize}

Now, according to Pollock, there is another “knot” in this matter: 

We find at the end of the story that Daśaratha had, in order to gain the hand of the beautiful princess, agreed to pay the highest bride-price, {\sl rājya-śulka}\index{rajyasulka@\textsl{rājya-śulka}} --- a promise to the woman’s male kin that her son shall succeed the throne. If this was true, Pollock says, Kaikeyī\index{Kaikeyi@Kaikeyī} only demanded what was rightfully hers, and must be exculpated of all blame/taint. This, he insists, must have been the original version of the story. So, according to Pollock, Vālmīki,\index{Valmiki@Vālmīki} in all likelihood, revised the story by introducing the incident of the two boons, and tried to minimize, if not eliminate altogether, the incident of the {\sl rājya-śulka}.\index{rajyasulka@\textsl{rājya-śulka}}

Problems abound in Pollock’s interpretations— all his speculations stand on {\sl one and only one} verse said by Rāma to Bharata (Rāma to Bharata)\index{Bharata} ({\sl Rāmāyaṇa} 2.107.3): 
\begin{quote}
{{\sl purā bhrātaḥ pitā naḥ sa mātaraṁ te samudvahan}} |\\
{\sl mātāmahe samāśrauṣīd rājya-śulkam anuttamam} || 
\end{quote}

\begin{myquote}
“Long ago, dear brother, when our father was about to marry your mother, he made a brideprice pledge to your grandfather--- the ultimate price, the kingship” [{\sl Trans.} Pollock]
\end{myquote}

There is nothing in the {\sl Rāmāyaṇa} other than these words of Rāma (2.107.3) that even remotely suggests that Daśaratha\index{Dasaratha@Daśaratha} made such a promise of {\sl rājya-śulka}.\index{rajyasulka@\textsl{rājya-śulka}} Let us review the connected scenes under these headings: 
\begin{itemize}
\itemsep=0pt
\item[(e)] Mantharā\index{Manthara@Mantharā} and Kaikeyī\index{Kaikeyi@Kaikeyī} 
\item[(f)] Daśaratha and Kaikeyī and 
\item[(g)] Rāma and Bharata.\index{Bharata}
\end{itemize}

\noindent
\textbf{(e) Mantharā and Kaikeyī:}\index{Kaikeyi@Kaikeyī}\index{Manthara@Mantharā} It is, indeed, rather doubtful that Daśaratha had earlier made an offer of two boons to Kaikeyī as Mantharā alleges --- for it {\sl is} rather odd for a queen to accompany a king in a battle. But we need not accept Mantharā’s words as the absolute truth --- it is important enough to remember that she is a wily woman, trying to poison Kaikeyī’s mind.  More importantly --- and irrespective of the two boons --- Mantharā does not mention {\sl rājya-śulka}\index{rajyasulka@\textsl{rājya-śulka}} to Kaikeyī; if Daśaratha had, indeed, made an offer of the kingdom to Kaikeyī’s kinsmen, it is impossible that she (or Kaikeyī) does not know of it/remember it. Pollock translates 2.8.23 in a way to implicate that she did, in fact, know of the {\sl rājya-śulka}. The verse reads (Mantharā to Kaikeyī) ({\sl Rāmāyaṇa} 2.8.34), 
\begin{quote}
{{\sl evaṁ te jñāti-pakṣasya śreyaś caiva bhaviṣyati}} |\\
{\sl yadi ced bharato dharmāt pitryaṁ rājyam avāpsyati} || 
\end{quote}

\begin{myquote}
 “For in this way, good fortune may still befall your side of family --- if, that is, Bharata\index{Bharata} secures, as by rights he should, the kinship of his forefathers.” 

\hfill [{\sl Trans.} Pollock]
\end{myquote}

Here, {\sl dharmāt} does not translate to “as by rights” to imply a {\sl rājya-śulka},\index{rajyasulka@\textsl{rājya-śulka}} but rather, “legally”--- if Daśaratha, indeed, conferred the throne onto Bharata on account of Kaikeyī,\index{Kaikeyi@Kaikeyī} the throne would be his legally. 

\smallskip
\noindent
\textbf{(f) Daśaratha and Kaikeyī:} If Daśaratha, indeed, had schemed to snatch the throne that was rightfully Bharata’s, he would not have “gladly entered the inner chamber to tell his beloved wife the good news.” (Daśaratha\index{Dasaratha@Daśaratha} to Kaikeyī)\index{Kaikeyi@Kaikeyī} ({\sl Rāmāyaṇa} 2.10.10):
\begin{quote}
{{\sl adya rāmābhiṣeko vai prasiddha iti jajñivān}} |\\
{\sl priyārhaṁ priyam ākhyātuṁ viveśāntaḥpuraṁ vaśī} || 
\end{quote}

\begin{myquote}
“Now, when the great king had given orders for Rāghava’s consecration, he gladly entered the inner chmber to tell his beloved wife the good news” [{\sl Trans.} Pollock]
\end{myquote}

Rush, he did, to tell his good wife of his decision; when, instead, he learnt that Kaikeyī was in a fit of grief, lying on the ground in a posture so ill-befitting her, he was consumed with sorrow, and offered to do what she wished; he, in fact, swore by Rāma to fulfill her wishes (Daśaratha to Kaikeyī) ({\sl Rāmāyaṇa} 2.11.5-6): 
\begin{quote}
{{\sl avalipte na jānāsi tvattaḥ priyataro mama}} |\\
{\sl manujo manuja-vyāghrād rāmād anyo na vidyate} || 

{\sl tenājayyena mukhyena rāghaveṇa mahātmanā} |\\
{\sl śape te jīvanārheṇa brūhi yan manasecchasi} ||
\end{quote}

\begin{myquote}
“Oh proud woman! Don't you know that there is nobody on this earth dearer to me than you --- except Rāma, the best among men”.

\hfill [{\sl Trans.} ours]


“I swear in the name of the scion of the Raghu dynasty, the invincible, broad-minded Rāma, the best among men worthy of long life. Tell me what you have in mind.”\hfill [{\sl Trans.} IIT website]
\end{myquote}

It is then that Kaikeyī demands the fulfillment of her wishes. But {\sl nowhere} is a mention of {\sl rājya-śulka} made. It is hard to conceive of any legitimate reason for the conspicuous absence of its mention.

\smallskip
\noindent
\textbf{(g) Rāma and Bharata:}\index{Bharata}  Why, then, did Rāma tell Bharata of a {\sl rājya-śulka}?\index{rajyasulka@\textsl{rājya-śulka}} Let us examine the relevant verses (Rāma to Bharata) ({\sl Rāmāyaṇa} 2.107.3):\\[-21pt] 
\begin{quote}
{{\sl purā bhrātaḥ pitā naḥ sa mātaraṁ te samudvahan}} |\\
{\sl mātāmahe samāśrauṣīd rājya-śulkam anuttamam} || 
\end{quote}

\vskip -.2cm

\begin{myquote}
“Long ago, dear brother, when our father was about to marry your mother, he made a bride price pledge to your grandfather --- the ultimate price, the kingship.” [{\sl Trans.} Pollock]
\end{myquote}

We may furnish an explanation for the above: in order to console Bharata,\index{Bharata} Rāma told him an untruth. Consider the situation:

When Bharata\index{Bharata} met Rāma in Citrakūṭa, he [Bharata] was railing against Daśaratha\index{Dasaratha@Daśaratha} and Kaikeyī,\index{Kaikeyi@Kaikeyī} accusing them both of unrighteousness (2.106.8-14). He felt great anger against his parents, and also a profound guilt for what had transpired; Rāma tried to console him by reassuring him that he was not at fault --- and that his parents, too, were not unrighteous (2.105.32-37). Bharata was not to be pacified --- he continued to lament piteously (2.106.8ff), while Rāma tried to calm him in every way possible. 

Finally, with utmost compassion, Rāma told Bharata of the alleged {\sl rājya-śulka},\index{rajyasulka@\textsl{rājya-śulka}} concluding his sentence with “{\sl mā viṣādam}”--- “do not despair” (2.107.19). This utterance, therefore, need not be taken literally; it was {\sl solely to console Bharata}\index{Bharata} that Rāma said them; note that in 2.107.19, in a similar situation with respect to Daśaratha, Rāma tells Sumantra, “You can tell the king you did not hear [Daśaratha’s\index{Dasaratha@Daśaratha} command to stop the chariot]... [for] to prolong sorrow is the worst thing of all.” Rāma has employed such words more than once: 

\newpage

\begin{quote}
{\sl ciraṁ duḥkhasya pāpiṣṭham} (2.40.47)\\ 
{\sl ciraṁ duḥkhasya pāpīyaḥ} (2.50.5). 
\end{quote}

Pollock’s allegations of an imminent dynastic struggle, we see, are absolutely without any grounds. 

\section{Vālmīki’s “Solution”}\label{sec1.3}
\index{Valmiki@Vālmīki}

In order to address the above-stated “problems”, Vālmīki allegedly proposed a “solution” as per Pollock: to obey elders without deliberation. According to Pollock, this instruction of “submission to hierarchy”\index{hierarchy} was not limited to the {\sl kṣatriya}\index{ksatriya@\textsl{kṣatriya}} class alone --- it was addressed to the society at large. By modeling a protagonist whose status was that of “absolute heteronomy”, Vālmīki, Pollock says, glorified its value and instructed the society to follow suit. He writes,

\begin{myquote}
“The state of junior members of the Indian household was, historically, not dissimilar to that of slaves (as was also the case in ancient Rome), both with respect to the father, and again, hierarchically among themselves… More generally, like the slave, Rāma is ‘not his own master, he is subordinate to others and cannot go where he wishes’ as an early Buddhist text defines the condition of slavery.” 				                                        
\hfill Pollock (2007a:20)
\end{myquote}

So, when one was told to “behave like Rāma”, it meant “submit to hierarchy”. It was a formula that, Pollock says, was inculcated into the public memory by innumerous public recitations and performances of the poem, and he refers to the large endowments made by Pallavas,\index{Pallava-s}\index{Ramayana@\textsl{Rāmāyaṇa}!endowments made for the performance of} Coḷas\index{Colas@Coḷas} and Pānḍyas\index{Pandyas@Pānḍyas} for this purpose in support of his argument.

By Pollock’s verdict, one would have to conclude that Vālmīki\index{Valmiki@Vālmīki} conjured the idea of filial piety\index{political!tool of subjugation!filial piety as} as a “political tool of subjugation”. Without doubt, this accusation borders on the ludicrous. Filial piety is/was a ubiquitous value: in Chinese culture, for example, filial piety\index{filial piety} is considered the {\sl first} virtue; historian Hugh D.R. Baker\index{Baker, Hugh D R} treads so far as to say that familial-respect is the only element common with the different traditions in China (Baker 1979:102). It was often enunciated by Confucius,\index{Confucius} and the oft repeated story goes thus:\index{filial piety!in Chinese} 

\begin{myquote}
“Once upon a time Confucius was sitting in his study, having his disciple Tseng Tsan to attend upon him. He asked Tseng Ts' an : “Do you know by what virtue and power the good Emperors of old made the world peaceful, the people to live in harmony with one another, and the inferior contented under the control of their superiors?” To this Tseng Ts'an, rising from his seat, replied: “I do not know this, for I am not clever.” Then said Confucius: “The duty of children to their parents is the fountain whence all other virtues spring and also the starting point from which we ought to begin our education. Now take your seat, and I will explain this. Our body and hair and skin are all derived from our parents, and therefore we have no right to injure any of them in the least. This is the first duty of a child. To live an upright life and to spread the great doctrines of humanity one must win good reputation after death, and reflect great honor upon our parents. This is the last duty of a son.” Hence the first duty of a son is to pay a careful attention to every want of his parents. The next is to serve his government loyally; and the last to establish a good name for himself. “So it is written in the {\sl Ta Ya}: You must think of your ancestors and continue to cultivate the virtue which you inherit from them”.
\hfill Chen (1908:16)	 
\end{myquote}

Similarly, {\sl pietas erga parentes} (piety towards parents) was one of the strongest instincts in the Roman people:\index{filial piety!in Romans}  

\begin{myquote}
“This relationship is the natural home of the Roman {\sl pietas}. To be pious meant to be “son”, and lovingly to fulfill the duties of the filial relationship. Love fulfilling duties, or rather the loving fulfillment of duties, this is the meaning of {\sl pietas}.”
\hfill (Haecker 1934:62)
\end{myquote}

Aeneas, the hero of Virgil’s\index{Virgil} {\sl Aeneid},\index{Aeneid@\textsl{Aeneid}}\index{filial piety!in \textsl{Aeneid}} is famously known as “{\sl pius}”--- in the course of the twelve books of the {\sl Aeneid}, Virgil applies to Aeneas the epithet {\sl pius} fifteen times in the narrative, has the other characters refer to him as {\sl pius, pietate insignis} or some equivalent expression eight times, and finally has Aeneas speak of himself twice as {\sl pius} (Moseley 1925:387). In the Tenth Book, particularly, is an episode of exquisite depiction of {\sl pietas erga parentes} (769-832). If this idea was but of one man --- Vālmīki\index{Valmiki@Vālmīki} --- how can we account for its universality? 

Pollock further extrapolates the “submit to hierarchy” formula to the Indian social-life at large, and writes that post-Vālmīki, this “inflexible hierarchy based on birth” became the “norm”. He quotes from the {\sl Rāmāyaṇa} to substantiate his claim: “the {\sl kṣatriya}-s\index{ksatriya@\textsl{kṣatriya}} accepted the brahmins as their superiors, and the {\sl vaiśya}-s were subservient to the {\sl kṣatriya}-s. The {\sl śūdra}-s, devoted to their proper duty, served the other three classes (1.6.19)'' (Pollock 2007a:11). In the light of Pollock’s statements, it becomes imperative to understand the idea behind “submit to hierarchy”— whether filial, social or otherwise.\\[-20pt] 

\subsection[Social Hierarchy in India]{Social Hierarchy in India\endnote{This whole section draws rather heavily upon Coomaraswamy’s writings.}}\label{sec1.3.1}

According to the Hindus,\index{Hindu!definition of Purpose of Life} the purpose of life is defined in a fourfold way. On the one hand, the purposes of life\index{purusartha-s@\textsl{puruṣārtha}-s} are the satisfaction of desire ({\sl kāma}),\index{kama@\textsl{kāma}} the pursuit of the means thereof ({\sl artha}),\index{artha@\textsl{artha}} and the fulfillment of function ({\sl dharma},\index{dharma} in the sense of duty); on the other hand, the final purpose of life is to attain liberation ({\sl mokṣa})\index{moksa@\textsl{mokṣa}} from all wanting, valuation and responsibilities. These immediate and final ends are not independent of, or fundamentally opposed to, one another, and provision is made in the society for both the active life of the householder ({\sl pravṛtti}),\index{pravrtto@\textsl{pravṛtti}} and for the contemplative life ({\sl nivṛtti})\index{nivrtti@\textsl{nivṛtti}} of the {\sl saṁnyāsin}.\index{samnyasin@\textsl{saṁnyāsin}}  

For a person leading an active life, spiritual progress is to be attained by the study of Scriptures, and the fulfillment of one’s own proper functions ({\sl sva-karman})\index{sva-karman@\textsl{sva-karman}} in the {\sl āśrama} that one may be living in at the time. In Kṛṣṇa’s\index{Krsna@Kṛṣṇa} words, {\sl sve sve karmaṇy abhirataḥ saṁsiddhiṁ labhate naraḥ} ({\sl Bhagavad-gītā}, 18.45) — Man reaches perfection by his loving devotion to his own work ({\sl sva-karma}) [{\sl Trans.} Coomaraswamy\index{Coomaraswamy, Ananda} (1946:28)]. {\sl Sva-karman} may be understood as performance of an activity in conformity with one’s own nature. In Rene Guenon’s words,\index{Guenon, Rene} 

\begin{myquote}
“... everyone must normally fulfill the function for which he is destined by his very nature, and he cannot fulfill any other function without a resulting grave disorder, which will have its repercussion on the whole social organization to which he belongs. Even more than this, if such a disorder becomes general, it will have its effects on the cosmic realm itself, all things being linked together according to strict correspondences”
\hfill (Fohr {\sl et al} 2001:58)  
\end{myquote}

According to tradition, {\sl dharma}\index{dharma@\textsl{dharma}}\endnote{``{\sl Dharma}”, Coomaraswamy writes, “is a pregnant term, difficult to translate in the present context; cf. {\sl eidos} in {\sl Republic}, 434A. In general, {\sl dharma} (literally ‘support’, {\sl dhṛas} in {\sl dhruva}, ‘fixed’, ‘Pole Star’, and Gr. {\sl thronos}) is synonymous with ‘Truth’. Than this ruling principle there is ‘nothing higher’ ({\sl Bṛhadāraṇyakopaniṣad},\index{Brhadaranyakopanisad@\textsl{Bṛhadāraṇyakopaniṣad}}\index{Upanisad@\textsl{Upaniṣad}!Brhadaranyaka@\textsl{Bṛhadāraṇyaka}} I.4.14); {\sl dharma} is the ‘king’s King’ ({\sl Aṅguttara Nikāya},\index{Anguttara Nikaya@\textsl{Aṅguttara Nikāya}} 1.109), i.e. ‘King of kings’; and there can be no higher title than that of {\sl dharma-rāja}, ‘King of Justice’. Hence the well-known designation of the veritable Royalty as Dharmarāja, to be distinguished from the personality of the king in whom it temporarily inheres”. (Coomaraswamy 2004:211)} and {\sl sva-karman}\index{sva-karman@\textsl{sva-karman}} are interwoven concepts: the one is the Universal and Eternal law; “the other is that share of this Law for which every man is made responsible by his physical and mental constitution” (Coomaraswamy\index{Coomaraswamy, Ananda} 1940:41). In other words, through the performance of his {\sl sva-karman}, an individual participates in the Universal {\sl dharma}, and so it is that {\sl sva-karman} is translatable as {\sl sva-dharma}.\index{sva-dharma@\textsl{sva-dharma}} Coomaraswamy writes,

\begin{myquote}
“Nothing will be more ruinous to the state than for the cobbler to attempt to do the carpenter’s work, or for an artisan or money maker led on by wealth or by command of votes or by his own strength to take upon himself the soldier’s form, or for a soldier to take upon himself that of a counselor or warden, for which he is not fitted, or for one man to be a jack-of-all-trades; and he [Plato]\index{Plato} says that wherever such perversions occur, there is injustice. He points out that our several natures are not all alike, but different, and maintains that everyone is bound to perform for the state one social service, that for which his nature is best adapted. And in this way more will be produced, and of a better sort, and more easily, when each one does one work, according to his own nature, at the right time and being at leisure from other tasks. In other words, the operation of Justice provides automatically for the satisfaction of all the real needs of a society.”
\hfill Coomaraswamy (1977:23)
\end{myquote}

Traditionally, the inheritance of functions is a matter of re-birth --- “not in the current misinterpretation of the word, but as rebirth is defined in Indian scriptures and in accordance with the traditional assumption that the {\sl father himself is reborn in his son}” (Coomaraswamy 1977:18). According to Manu,\index{Manu}\\[-20pt] 
\begin{quote}
{{\sl patir bhāryām sampraviśya garbho bhūtveha jāyate}} |\\
{\sl jāyāyās taddhi jāyātvaṁ yad asyāṁ jāyate punaḥ} || 

\hfill ({\sl Manusmṛti}\index{Manusmrti@\textsl{Manusmṛti}} 9.8) 
\end{quote}

The husband is re-born from his wife --- and for this reason is she called “{\sl jāyā}” (from her is he born again). According to this conception, 

\begin{myquote}
“the father, as regards his empirical personality or “character”, is reborn in his son, who is to all intents and purposes identified with him and takes his place in the community when he retires or dies; and that this natural succession is confirmed by formal rites of transmission. The vocational function is a form of divine service, and the {\sl métier}, i.e., “ministry”, a work that at the same time honors God and serves man’s present needs: and so it is that in India, as it was for Plato,\index{Plato} the first reason for which one ought to beget children is in order to, ‘carry on from generation to generation the good work’.”
\hfill Coomaraswamy\index{Coomaraswamy, Ananda} (1940:39)
\end{myquote}

Coomaraswamy writes,  {\sl duo sunt in homine}\endnote{{\sl dvā suparṇā sayujā sakhāyā samānaṁ vṛkṣaṁ pariṣasvajāte}|
 
{\sl tayor anyaḥ pippalam svādv atty anaśnann anyo abhicākaśīti} || ({\sl Muṇḍakopaniṣad}\index{Mundakopanisad@\textsl{Muṇḍakopaniṣad}}\index{Upanisad@\textsl{Upaniṣad}!Mundaka@\textsl{Muṇḍaka}} 3.1.1)}--- there are two (natures) in man. Of these two, one is the mortal/individual personality or character of the man; the other, the Immortal and the very Person of the man himself. It is only to the former, individual nature that {\sl varṇa} can be applied; the word {\sl varṇa} itself could, indeed, be rendered not inaccurately by “individuality,” inasmuch as color arisen from the contact of light with a material, which then exhibits a color that is determined not by the light, but by its own nature. In other words, 

\begin{myquote}
“My” individuality or psychophysical constitution is not, from this point of view, an end in itself either for me or for others, but always a means, garment, vehicle, or tool to be made good use of for as and for so long as it is “mine”; it is not an absolute, but only a relative value, personal insofar as it can be utilized as means to the attainment of man’s last end of liberation, and social in its adaptation to the fulfillment of this, that, or the other specialized function. It is the individuality, and not the Person, that is bequeathed by the father to his son, in part by heredity, in part by example, and in part by formal rites of transmission: when the father retires, or at his death, the son inherits his position, and, in the widest sense of the word, his debts, i.e., social responsibilities. This acceptance of the paternal inheritance sets the father free from the burden of social responsibility that is attached to him as an individual; having done what there was for him to do, the very man departs in peace… When, now, we have forgotten who we are and, identifying ourselves with our “outer man”, have become lovers of our own individual-selves, we imagine that our whole happiness is contained in the freedom of this “outer man” to go his own way and find pasture where he will. There, in ignorance and in desire, lie the roots of individualism. Thus, the traditional concept of liberty goes far beyond, in fact, the demand of any anarchist; it is the concept of an absolute, unfettered freedom to be as, when, and where we will. All other and contingent liberties, however desirable and right, are derivative and to be valued only in relation to this last end.”

\hfill (Coomaraswamy 1977:18).
\end{myquote}

Studied in this light, it can be shown that the Indian caste-system --- the alleged “hierarchical” system --- is everyman’s Way to realize the Last End --- knowing his Self. To understand this better, we must understand the word “{\sl karman}” better. {\sl Karman}\index{karman@\textsl{karman}} is derived from the verbal root {\sl kṛ}. Coomaraswamy\index{Coomaraswamy, Ananda} shows that significantly\endnote{Cf. Latin {\sl creare} with the same values; Cf. Latin {\sl facere}, originally {\sl sacra facere} (to make sacred)}, {\sl karman} is not merely “work” or “action,” but it is synonymous with {\sl yajña} and also with {\sl vrata} (sacred operation). Indeed, the primary reference of {\sl karman} is to the performance of sacrificial rites, and it must be understood that while we render {\sl karman} by “action,” it is actually impossible to make any essential distinction of the meaning “sacrificial operation” for that of simply “operation”.

Accordingly, all trades of a traditional society are considered sacred, and as a sort of liturgy. Kṛṣṇa\index{Krsna@Kṛṣṇa} says, “inasmuch as by his own work, he [a man] is praising Him”. It is precisely this idea that finds such vivid expression in the well-known Indian philosophy of action, the {\sl karma-mārga} of the {\sl Bhagavad-gītā}.\index{Bhagavadgita@\textsl{Bhagavadgītā}} Kṛṣṇa says {\sl yajñārthāt karmaṇo’nyatra loko’yaṁ karma-bandhanaḥ} ({\sl Bhagavad-gītā} 3.9) --- the world is enchained by whatever is done, unless it be made a {\sl yajña}. So, then, we are to do whatever Nature bids us do, whatever ought to be done; but without anxiety about the consequences over which we have no control. We are to surrender all activities to Him, that they may be His and not ours; they will no more affect Him than a drop of water sticks to the shiny lotus leaf. There is no liberation by merit, but only by working without ever thinking that “I”, that which I call “myself”, is the actor.

In other words, everyman’s Way to become what he is --- what he has it in him to become --- is one of perfectionism in that station of life to which his own nature imperiously summons him. The pursuit of perfection is everyman’s “equality of opportunity”; and the goal is the same for all, for the miner and the professor alike, because there are no degrees of perfection. 

This metaphysic of action underlies the whole Indian vocational system, and it is from this point of view that in India, every profession is a “priesthood”. In the more unified life of India it is not only in special rites that the meaning of life has been focused; this life itself has been treated as a significant ritual, and so sanctified. This sacrificial interpretation of life can be best explained by quoting the doctrine itself as expounded by Ghora Āṅgirasa\index{Ghora Angirasa@Ghora Āṅgirasa} to Kṛṣṇa:\index{Krsna@Kṛṣṇa} 

\begin{myquote}
“When one hungers and thirsts and has no pleasure, that is his initiation. 

When one eats and drinks and takes one’s pleasure, that is his participation in the sacrificial-sessions. 

When one laughs and feasts and goes with a woman, that is his participation in the liturgy. 

When one is fervent, or generous, or does right, or does no hurt, or speaks the truth, these are his fees to the priests. 

Wherefore they say: He will beget, he has begotten - and that is his being born again. 

Death is the final ablution.” 

\hfill({\sl Chāndogyopaniṣad}\index{Chandogyopanishad@\textsl{Chāndogyopaniṣad}}\index{Upanisad@\textsl{Upaniṣad}!Chandogya@\textsl{Chāndogya}}
 III. 17.1-5) [{\sl Trans.} Coomaraswamy\index{Coomaraswamy, Ananda} 1946:18] 
\end{myquote}

So, the caste-system\index{caste} is not like the class-system; rather, to quote A. M. Hocart, “a four-fold vertical division of humanity” (Hocart 1950:163). Kṛṣṇa\index{Krsna@Kṛṣṇa} says, {\sl cāturvarṇyaṁ mayā sṛṣṭaṁ guṇa-karma-vibhāgaśaḥ} ({\sl Bhagavad-gītā}\index{Bhagavadgita@\textsl{Bhagavadgītā}} 4.13) --- ``I emanated the Four {\sl varṇa}-s, distributing qualities and operations'' [{\sl Trans.} Coomaraswamy] (Coomaraswamy 1946:6). Accordingly, the individual natures which are the most spiritually evolved, the ones closest to the ultimate goal of {\sl mokṣa} --- the Brahmins --- occupy the first rung of the ladder. Such natures are qualified thus:
\begin{quote}
{{\sl śamo damas tapaḥ śaucaṁ kṣāntir ārjavam eva ca}} |\\
{\sl jñānaṁ vijñānam āstikyaṁ brahma-karma svabhāvajam} || 

\hfill ({\sl Bhagavad-gītā} 18.42)
\end{quote}

\begin{myquote}
“Control of internal and external sense-organs, austerity, purity, forbearance, straightforwardness, knowledge, wisdom, and faith are the duties of brahmins, born of their inherent nature.”\hfill [{\sl Trans.} ours]
\end{myquote}

It must be noted here that any individual who bears these qualities may be perceived as a brahmin. We see, for example, Satyakāma Jābāla of {\sl Chāndogyopaniṣad},\index{Chandogyopanishad@\textsl{Chāndogyopaniṣad}}\index{Upanisad@\textsl{Upaniṣad}!Chandogya@\textsl{Chāndogya}} goes to a brahmin teacher and asks to be his disciple. When asked to what lineage he belongs, Satyakāma can only answer that he is the son of his mother, and knows not who his father may have been: the teacher accepts him on the ground that such candor is tantamount to a brahmin lineage. 

On the next rung of the ladder, then, are the {\sl kṣatriya}-s:\index{ksatriya@kṣatriya}
\begin{quote}
{{\sl śauryaṁ tejo dhṛtir dākṣyaṁ yuddhe cāpy apalāyanam}} |\\
{\sl dānam īśvara-bhāvaś ca kṣātraṁ karma svabhāvajam} || 

\hfill ({\sl Bhagavad-gītā}\index{Bhagavadgita@\textsl{Bhagavadgītā}} 18.43)
\end{quote}

\begin{myquote}
“Valour, invincibility, steadiness, adroitness and not retreating in battle, generosity and lordliness are the duties of {\sl kṣatriya}-s born of their inherent nature.”\hfill [{\sl Trans.} ours]
\end{myquote}

{\sl Vaiśya}-s occupy the third, and the {\sl śūdra}-s, the last rung of the ladder: 

\begin{quote}
{{\sl kṛṣi-gaurakṣya-vāṇijyaṁ vaiśya-karma svabhāvajam}} |\\
{\sl paricaryātmakaṁ karma śūdrasyāpi svabhāvajam} || 

\hfill ({\sl Bhagavad-gītā} 18.44)
\end{quote}

\begin{myquote}
“Agriculture, cattle-tending and trade are the duties of the {\sl vaiśya}, born of his inherent nature. And the duty of a {\sl śūdra} is one of service, born of his inherent nature.”\hfill [{\sl Trans.} ours]
\end{myquote}

In a vocational society uncorrupted by ideas of social ambition, it is taken for granted that “everyone is very proud of his hereditary science” ({\sl kula-vidyā}) ({\sl Mālavikāgnimitra} 1.4).\endnote{So, Philo, pointing out that when the king asks, “What is your work?” he receives the answer, “We are shepherds, as were our fathers”, comments: “Aye, indeed! Does it not seem that they were more proud of being shepherds than is the king, who is talking to them, of his sovereign power?” In one of Dekker’s plays, he makes his grocer express the fervent wish, May no son of mine ever be anything but a grocer!}

In a vocationally integrated society, a proportionate equality is practiced.\endnote{Cf. Aristotle’s words, “Everything is ordered together to one end; but just as in a household, the free have the least authority to act at random, and have most or all of their actions arranged for them, whereas the footmen and animals have but little common (responsibility) and act for the most part at random” ({\sl Metaphysics} 12.10.3).} Accordingly, the liberty of choice is more and more restricted the higher one’s status: {\sl noblesse oblige}. (Coomaraswamy 1940:40)\index{Coomaraswamy, Ananda}

It must be noted here that {\sl varṇa} is not, in Hindu\index{Hindu!law} law, a legal disability; men of any {\sl varṇa} may act as witnesses in suits ({\sl Manusmṛti}\index{Manusmrti@\textsl{Manusmṛti}} 8.61-63). Furthermore, according to this proportionate law, a king is to be fined a thousand times as much as a {\sl śūdra} for the same offence (for the consequences or repercussions of a King’s offence is far-reaching in effect than that of a common man.) Similarly, a brahmin’s punishment is also very much heavier than a {\sl śūdra’s} for the same offence, and many things are allowed to the {\sl śūdra} that a brahmin or the wife of a brahmin may not do, like remarriage (Coomaraswamy 1946:16). 

It must also be noted that caste discrimination is strict only in terms of rules against intermarriage and inter-dining. For example, not even the king can aspire to marry his own brahmin cook’s daughter. As for inter-dining, a Hindu does not inter-dine even with his own wife, or his own caste, and that this has nothing whatever to do with social prejudice of any kind, but reflects a functional differentiation.\endnote{T.W. Rhys Davids remarks, “Evidence\index{evidence!regarding intermarriage} has been yearly accumulating on the existence of restrictions as to intermarriage, and as to the right of eating together among other tribes---Greeks, Germans, Russians and so on. Both the spirit, and to a large degree, the actual details of modern Indian caste usages, are identical with these ancient, and no doubt universal, customs.”(Rhys Davids 1899:98)} The {\sl varṇa} system has been painted in such dark colors only because it is incompatible with the existing industrial system; it ought not to be judged by concepts of success that govern life in a society organized for overproduction and profit at any price, and where it is everyone’s ambition to rise on the social ladder, rather than to realize his own perfection. Indeed, this scheme is the nearest and only approach to a workable socialism that has been tried in our race, and that succeeded for hundreds of years. Sir George Birdwood’s words:\index{Birdwood, Sir George} 

\begin{myquote}
“In that [Hindu] life all are but co-ordinate parts of one undivided and indivisible whole, wherein the provision and respect due to every individual are enforced, under the highest religious sanctions, and every office and calling perpetuated from father to son by those cardinal obligations of caste on which the whole hierarchy of Hinduism hinges... We trace there the bright outlines of a self-contained, self-dependent, symmetrical and perfectly harmonious industrial economy, deeply rooted in the popular conviction of its divine character, and protected, through every political and commercial vicissitude, by the absolute power and marvelous wisdom and tact of the brāhminical priesthood. Such an ideal order we should have held impossible of realization, but that it continues to exist and to afford us, in the yet living results of its daily operation in India, a proof of the superiority, in so many unsuspected ways, of the hierarchic civilization of antiquity over the secular, joyless, inane, and self-destructive, modern civilization of the West”.
\hfill Birdwood (1915:76, 83-84)
\end{myquote}

\section{The “Crux” of the Problematic}\label{sec1.4}

An(other) important “achievement” of the {\sl Rāmāyaṇa} was, Pollock says, the establishment of paternalistic formulation of the political society. Here, the king is portrayed as the father of the state, and the people as his children. It is an “alluring image” that establishes, Pollock says, a kinship bond between the two, and works towards the “institutionalization of dependency and loyalty… for the centralization of power” (Pollock 2007a:21).  Here, too, Pollock seems to imply that Vālmīki\index{Valmiki@Vālmīki} perhaps {\sl conjured} this idea for attaining “political domination”. 

In another context, Pollock contradicts himself by saying “Kingship,\index{kingship} as an institution, has no authority and legitimacy of its own. It is dependent on the uneasy relationship between the king and the brahmin ...” (Pollock 2007a:69). Accordingly, in Indian tradition there has always existed, Pollock claims, a “dichotomy of power”--- while {\sl kṣatriya}-s\index{ksatriya@\textsl{kṣatriya}} ruled their kingdoms, they were in turn ruled by brahmins.  Power lay in their hands--- but its legitimization was always, according to Pollock, in the hands of brahmins.\index{brahmin} This legitimizing power exercised by them was, he says, derived from their magical control of the sacrifice, which they could manipulate for political purposes.

\begin{myquote}
“Kingship as an institution has no authority and legitimacy of its own. It is dependent on the uneasy relationship between the king and the brahmin... the brahmin’s monopoly of the source of authority and legitimization leaves the king with mere power and effectively bars kingship from developing its full potential as the central regulating force.” 	
\hfill (Heeterman cited in Pollock 2007a:69)
\end{myquote}

In Rāma, Pollock claims, there is a response to this “potentially incapacitating bifurcation”. Indian tradition defines a \hbox{{\sl kṣatriya}’s} {\sl dharma}\index{dharma@\textsl{dharma}} as a readiness to fight/destroy the enemy, to show bravery in battle, to live by the sword, etc. But Rāma, a {\sl kṣatriya}, Pollock says, does not accept this. He quotes Rāma advising Lakṣmaṇa,\index{Laksmana@Lakṣmaṇa} “...give up this ignoble notion that is based on code of {\sl kṣatriya}-s...base your actions on righteousness, not violence.” ({\sl Rāmāyaṇa} 2.18.36) [{\sl Trans.} Pollock]. He quotes another instance where Rāma calls {\sl kṣātra dharma}\index{ksatra-dharma@\textsl{kṣātra-dharma}} as debased, vicious, covetous, and one that evil men observe. ({\sl Rāmāyaṇa} 2.101.20) For Rāma, then, according to Pollock, {\sl dharma} is associated only with truth and righteousness\endnote{Pollock writes that in the {\sl Mahābhārata}\index{Mahabharata@\textsl{Mahābhārata}} also, Yudhiṣṭhira\index{Yudhisthira@Yudhiṣṭhira} voices the same opinion as Rāma--- but here, he says, Yudhiṣṭhira is branded as a {\sl nāstika}, and his position is used as a {\sl pūrvapakṣa} for Kṛṣṇa\index{Krsna@Kṛṣṇa} and others to reassert {\sl kṣātra dharma}.\index{ksatra-dharma@\textsl{kṣātra-dharma}} So, he says, although both Rāma and Kṛṣṇa wish to stand as models for the society to emulate, the ideals they stand for are “in absolute opposition to each other”.}.

Pollock proceeds to muse on the implications of Rāma’s alleged position: According to him, Rāma simply “enlarged his code” to become righteous. He made his “{\sl kṣatriya dharma} to absorb brahmanical {\sl dharma} and its legitimizing ethics--- nonviolence and spirituality. In this way, the {\sl kṣatriya} could become self-legitimizing and the full potential of kingship could be activated at last” (Pollock 2007a:71). So, according to Pollock, “{\sl the hierarchical subordination that Rāma and the text explicitly upholds, is implicitly opposed by his spiritual commitment}” (Pollock 2007a:72) [{\sl italics ours}]. Summarizing Pollock's view, although Vālmīki\index{Valmiki@Vālmīki} established this formula for Indian society, by implication he decried the {\sl varṇa} system in the end by allowing Rāma to discard his {\sl dharma}\index{dharma@\textsl{dharma}} for “loftier” views. According to Pollock, it is only such a “spiritualized” {\sl dharma} that can make social peace possible.

First of all, to show Rāma as contemptuous of {\sl kṣātra-dharma}\index{ksatra-dharma@\textsl{kṣātra-dharma}} is groundless and ridiculous in equal measure. In the very beginning of Ayodhyā-kāṇḍa,\index{Ayodhyakanda@Ayodhyā-kāṇḍa} Rāma is described thus ({\sl Rāmāyaṇa} 2.1.16):
\begin{quote}
{{\sl kulocita-matiḥ kṣātraṁ dharmaṁ svaṁ bahu manyate}} |\\
{\sl manyate parayā kīrtyā mahat svarga-phalaṁ tataḥ} ||  
\end{quote}

\begin{myquote}
“[Rama] entertained thoughts befitting his (great) dynasty and honoured the code of conduct of a kshatriya; he believed one could attain heavenly abode through his great achievements.”

\hfill [{\sl Trans.} IIT website] [{\sl diacritics not used in the original}]
\end{myquote}

Rāma is first, before anything else, a {\sl kṣatriya}\index{ksatriya@\textsl{kṣatriya}}--- immortalized in his image of Kodaṇḍa Rāma. Rāma’s prowess comes to light in most of the {\sl kāṇḍa}-s --- but we realize his indomitable prowess in the Araṇya-kāṇḍa\index{Aranyakanda@Araṇya-kāṇḍa} in particular: Mārīca,\index{Marica@Mārīca} we see, did not forget the taste of Rāma’s wrath till the end --- his description of Rāma must be recounted {\sl verbatim} (Mārīca to Rāvaṇa)\index{Ravana@Rāvaṇa} ({\sl Rāmāyaṇa} 3.37.16 and 3.39.14 respectively): 
\begin{quote}
{{\sl dhanur vyādita-dīptāsyam śarārciṣam amarṣaṇam}} |\\
{\sl cāpa-bāṇa-dharam tīkṣṇam śatru-sainya-prahāriṇam} ||
\end{quote}

\begin{myquote}
“His bow is like on open burning mouth, and his flaming arrows are like fire. He is all anger. He is the wielder of bow and arrows. He can strike the enemy army (alone)”.\hfill [{\sl Trans.} IIT website] [{\sl diacritics not used in the original}] 
\end{myquote}

\begin{quote}
{{\sl vṛkṣe vṛkṣe ca paśyāmi cīra-kṛṣṇājināmbaram}} |\\
{\sl gṛhīta-dhanuṣam rāmam pāśa-hastam ivāntakam} || 
\end{quote}

\begin{myquote}
“In every tree, I see Rāma clad in bark and deerskin, wielding the bow, holding a noose in hand like the god of death.” 

\hfill [{\sl Trans.} IIT website] [{\sl diacritics not used in the original}]
\end{myquote}

Rāvaṇa\index{Ravana@Rāvaṇa} does not understand Mārīca’s fears till he himself comes face to face with Rāma in battle ({\sl Rāmāyaṇa} 6.99.12): 

\begin{quote}
{{\sl sa dadarsa tato rāmam tiṣṭhantam aparājitam}} |\\
{\sl lakṣmaṇena saha bhṛātrā viṣṇunā vāsavam yathā} || 
\end{quote}

\begin{myquote}
“[Rāvaṇa]\index{Ravana@Rāvaṇa} saw Rāma with Lakṣmaṇa\index{Laksmana@Lakṣmaṇa} like Viṣṇu\index{Visnu@Viṣṇu} with Indra\index{Indra} (with his Kodaṇḍa in hand, touching the sky”. [{\sl Trans.} ours]
\end{myquote}

An incident that illustrates well Rāma’s prowess is that of Khara and the 14,000 {\sl rākṣasa}-s who are decimated by Rāma’s arrows. Recounting the incident, Śūrpaṇakhā tells Rāvaṇa (Śūrpaṇakhā to Rāvaṇa) ({\sl Rāmāyaṇa} 3.34.7-10): 
\begin{quote}
{{\sl nādadānam śarān ghorān na muñcantam śilīmukhān}} |\\
{\sl na kārmukam vikarṣantam rāmam paśyāmi saṁyuge} ||\\
{\sl hanyamānam tu tat sainyam paśyāmi śara-vṛṣṭibhiḥ} |\\
{\sl indreṇevottamam sasyam āhataṁ tv aśma-vṛṣṭibhiḥ} ||\\
{\sl rakṣasām bhīma-rūpāṇām sahasrāṇi caturdaśa} |\\
{\sl nihatāni śarais tīkṣṇaiḥ tenaikena padātinā} ||\\
{\sl ardhādhika-muhūrtena kharaś ca saha-dūṣaṇaḥ} ||
\end{quote}

\begin{myquote}
(Summary [{\sl ours}]): None saw how Rāma took the arrows, how he directed them, or how he bent the bow. The {\sl rākṣasa}-s felt the scorching arrows killing them. Rāma alone put an end to the life of Khara and Dūṣaṇa.)  
\end{myquote}

It is not for nothing that Kṛṣṇa\index{Krsna@Kṛṣṇa} salutes Rāma’s invincibility with the words ({\sl Bhagavadgītā} 10.31), “[{\sl pavanaḥ pavatām asmi}] {\sl rāmaḥ śastra-bhṛtām aham}” --- “[Of the purifying elements, I am the wind.]; among the wielders of weapons, I am Rāma”. 

Secondly, Pollock wrongly associates “truth and righteousness” {\sl only} with brahmins,\index{brahmin} and violence only with {\sl kṣatriya}-s\index{ksatriya@\textsl{kṣatriya}} --- if he is taken literally, we must conclude that {\sl kṣātra} has not an iota of truth or righteousness in it. This is, (un)fortunately, ridiculous. Contrary to his wilful misrepresentations, {\sl brāhma}\index{brahma@\textsl{brāhma}} and {\sl kṣātra} are both grounded in “truth and righteousness”, and differ, primarily, only in function (and also, a predominance of one of the {\sl sattva/rajas guṇa}-s). {\sl Brāhma}’s essential function is to safegaurd the fundamental principles on which stands a tradition or a society --- so its “true function” is knowledge and teaching. {\sl Kṣātra},\index{Ksatra@\textsl{Kṣātra}} on the other hand, is “government” and it is two-fold in nature: administrative/judicial on the one hand, and military on the other --- and these two elements are represented in various traditions as “the scales and the sword”. The {\sl Bṛhadāraṇyakopaniṣad}\index{Brhadaranyakopanisad@\textsl{Bṛhadāraṇyakopaniṣad}}\index{Upanisad@\textsl{Upaniṣad}!Brhadaranyaka@\textsl{Bṛhadāraṇyaka}} asserts that {\sl kṣatra} is the life-breath ({\sl prāṇa}) that protects ({\sl trāyate}) one from being hurt ({\sl kṣaṇitoḥ}). ({\sl Bṛhadāraṇyakopaniṣad} 5.13.4)\endnote{Cf. {\sl Śatapatha-brāhmaṇa}\index{Satapathabrahmana@\textsl{Śatapathabrāhmaṇa}}\index{Brahmana@\textsl{Brāhmaṇa}!Satapatha@\textsl{Śatapatha}}.14. 8.14.4.}.  A well-established {\sl kṣatra}\index{ksatra@\textsl{kṣatra}} ensures a well-nourished community\endnote{{\sl Aitareya-brāhmaṇa.\index{Aitareyabrahmana@\textsl{Aitareyabrāhmaṇa}}\index{Brahmana@\textsl{Brāhmaṇa}!Aitareya@\textsl{Aitareya}} 8.7.10, kṣatra-rūpaṁ tat}; cf. {\sl Taittirīya-brāhmaṇa}.\index{Taittiriyabrahmana@\textsl{Taittirīyabrāhmaṇa}}\index{Brahmana@\textsl{Brāhmaṇa}!Taittiriya@\textsl{Taittirīya}} 3.8.23.3, {\sl rājanyo bāhubalī bhāvukaḥ}.}. A difference only of function between the {\sl brahma}\index{brahma@\textsl{brahma}} and {\sl kṣatra}\index{ksatra@\textsl{kṣatra}} can be seen in the Puruṣa-sūkta\index{Purusasukta@Puruṣa-sūkta} of {\sl Ṛgveda}\index{Rgveda@\textsl{Ṛgveda}!brahma and kṣatra in} (10.190): here, brahmins are shown as corresponding to the mouth of the {\sl Puruṣa} representing teaching whereas the {\sl kṣatriya}-s\index{ksatriya@\textsl{kṣatriya}} correspond to his arms because their functions relate essentialy to action. 

Indian kingship, therefore, implied anything but a personal autocracy. Kings required the vocal consent of the people who “made him the guardian of their comfort” ({\sl Ṛgveda} 1.100.7) --- and participation in government meant the representation of at least three, if not all the four {\sl varṇa}-s making up the multitude of the {\sl viś} --- the people. The {\sl sabhā}\endnote{Cf. {\sl Atharva-veda}\index{Atharvaveda@\textsl{Atharva-veda}} 2.24.13; 6.28.6; 7.1.4; 8.4.9; 10.34.6.} and the {\sl samiti}\endnote{Cf {\sl Ṛg-veda}.1.95.8; 9.92.6; 10.97.6; 166.4; 191.3} were the forums of {\sl vox populi}, which served as checks on the authority of the king. The popular basis of royal selection finds expression in the {\sl Atharva-veda}\index{Atharvaveda@\textsl{Atharva-veda}} (6.87-88):

\smallskip
\begin{myquote}
{{\sl “... viśas tvā sarvā vāñchantu}}\\
{{\sl mā tvad rāṣṭram adhi bhraśat.}}\\
{{\sl dhruvo’cyutaḥ pramiṇīhi}}\\
{{\sl śatrūñ chatrūyato’dharān pādayasva /}}\\
{{\sl sarvā diśaḥ sammanasaḥ}}\\
{{\sl sadhrīcīr-dhruvāya te samitiḥ kalpatām iha...”}}
\end{myquote}
\smallskip

“Gladly you come among us; remain firmly without faltering; may all the people want you; may you not fall off the state... Vanquish you firmly, without falling, the enemies, and those behaving like enemies crush you under your feet. All the quarters unanimously honor you and for firmness the assembly here creates you.” [{\sl Trans.} Jayaswal]

A king was called {\sl rāṣṭrabhṛt --- literally}, “one who bears, supports, and maintains the kingdom”.  The primacy of the king’s protective power is as old as the {\sl Ṛgveda}\index{Rgveda@\textsl{Ṛgveda}!primacy of king's power in} as is inherent in such epithets as {\sl gopā janasya} ({\sl Ṛgveda} 3.43.5) and {\sl janasya gopatiḥ} ({\sl Ṛgveda} 9.35.5). The king’s obligation to his subjects comes forcefully to the fore in the oath he takes before the consecrating priest in the {\sl aindra-mahābhiṣeka} ceremony. “From the night of my birth, the royal birth”, says he, “to that of my death, for the space between these two, my sacrifice and my gifts, my place, my good deeds, my life and my offspring, mayest thou take if I play thee false”. ({\sl Aitareya-brāhmaṇa}\index{Aitareyabrahmana@\textsl{Aitareyabrāhmaṇa}}\index{Brahmana@\textsl{Brāhmaṇa}!Aitareya@\textsl{Aitareya}} 8.15) [{\sl Trans.} Jayaswal]

A king is {\sl dharmasya goptā}, the “protector of {\sl dharma}”,\index{dharma@\textsl{dharma}}\index{dharma@\textsl{dharma}!king as a protector of} and {\sl dhṛta-vrata}\endnote{{\sl Taittirīya-saṁhitā}\index{Taittiriyasamhita@\textsl{Taittirīyasaṁhitā}} 1.8.16;  {\sl Taittirīya-brāhmaṇa}. 1.7.10.2;} the “upholder of the sacred law”. Elsewhere, he is called {\sl satya-sava}, “of true sacrifice”, {\sl satya-dharma}, “of true conduct”, {\sl satyānṛte}, an “authority in truth and falsehood” like Varuṇa,\index{Varuna@Varuṇa} and a {\sl satya-rājan}, a “true king” ({\sl Taittirīya-brāhmaṇa}\index{Taittiriyabrahmana@\textsl{Taittirīyabrāhmaṇa}}\index{Brahmana@\textsl{Brāhmaṇa}!Taittiriya@\textsl{Taittirīya}} 1.7.10.1-6). The emphasis therefore is clearly on truth and duty. King Aśvapati proudly declares: “Within my realm there is no thief, no miser, nor a drinking man, none altar-less, none ignorant, no man unchaste, no wife unchaste”. ({\sl Chāndogyopaniṣad}\index{Chandogyopanishad@\textsl{Chāndogyopaniṣad}}\index{Upanisad@\textsl{Upaniṣad}!Chandogya@\textsl{Chāndogya}} 5.11.5) 

An interesting and important ceremony in the Rājasūya consists in the priests silently striking the king with sticks on the back. The rite is construed as the height of priestly authority by Weber; but the rod or {\sl daṇḍa} touching the king is understood as the “symbolic Scepter of Justice” by Jayaswal, “conveying by the action the view of the sacred common law that the king was not above but under the law” (Jayaswal 2006:208-209). 

So, the last thing expected of the Indian king was to “do as he liked”--- he had to always do what was “correct”. A large section of {\sl Artha-śāstra},\index{Arthasastra@\textsl{Arthaśāstra}} for example, deals with the characteristics of a just monarch, and the methods to train him to be one. In the opening pages of this text, Kauṭilya says, “In the happiness of his population, rests the ruler’s own happiness; in their welfare lies his welfare; he shall not consider as good whatever pleases him, but he shall consider as good whatever pleases his population.”\endnote{{{\sl prajā-sukhe sukhaṁ rājñaḥ prajānāṁ ca hite sukham}} |\\ {\sl nātma-priyaṁ hitaṁ rājñaḥ prajānāṁ tu priyaṁ hitam} || (Artha-śāstra 1.19)} In another place, Kauṭilya says that the king functions as a public servant--- “he is on the same footing as his soldiers, both receiving their different wages and both being entitled to share the assets of the nation.” ({\sl Artha-śāstra} 10.3) The king was regarded as a trustee; he was particularly enjoined to note that the treasury was not his private or personal property--- it was a public trust to be utilized for public purposes. ({\sl Mahābhārata} 4.2.3-5)\endnote{{{\sl bala-prajā-rakṣaṇārthaṁ dharmārthaṁ koṣa-saṅgrahaḥ}} |\\
{\sl paratreha ca sukhado nṛpasyānyas tu duḥkhadaḥ} ||\\
{\sl strī-putrārthaṁ kṛto yaś ca svopabhogāya kevalam} |\\
{\sl narakāyaiva sa jñeyo na paratra sukhapradaḥ} ||\\ 
({\sl Mahābhārata} 4.2.3-5)

\noindent
Also cf.\\ 
{{\sl svabhāga-bhṛtyā dāsyatve prajānāṁ ca nṛpaḥ kṛtaḥ}} |\\
{\sl brahmaṇā svāmi-rūpas tu pālanārthaṁ hi sarvadā} || ({\sl Mahābhārata} 4.2.137)}

No Indian texts on polity endorsed the subjects to tolerate tyranny --- if the king did not adhere to {\sl dharma}, the subjects were, at first, recommended to threaten the tyrant that they would migrate from the country if he did not mend his ways ({\sl Śukra-nīti-sāra} 1.277-78). If this did not produce any effect, the subjects were to dethrone the king, and replace him with a person of their choice ({\sl Śukra-nīti-sāra} 1.351-53). The {\sl Mahābhārata},\index{Mahabharata@\textsl{Mahābhārata}} too, explicitly recognizes the subject’s right to tyrannicide (albeit as a last resort) ({\sl Mahābhārata} 13.86.35-6). So, the distinctions between monarchy and tyranny were sharply drawn --- the king was certainly no despot riding roughshod over the susceptibilities of his people. If he did, he came to grief. We hear of many kings deposed by their people; and of their efforts to resume their reign\endnote{Cf. {\sl Atharva-veda}.3.3.4;}.  The {\sl Śatapatha-brāhmaṇa}\index{Satapathabrahmana@\textsl{Śatapathabrāhmaṇa}}\index{Brahmana@\textsl{Brāhmaṇa}!Satapatha@\textsl{Śatapatha}} refers to Duṣṭarītu Pauṁsāyana\index{Dustaritu Paumsayana@Duṣṭarītu Pauṁsāyana} being banished from his kingdom, though it had come to him through ten generations. 

None can perhaps surpass Coomaraswamy,\index{Coomaraswamy, Ananda} who writes with uncanny insight and uncanny precision: 

\begin{myquote}
“...the absolute monarchies of the Orient are not comparable to that of France immediately before the revolution. The normal Oriental monarchy is really a theocracy, in which the king’s position is that of an executive who may do only what ought to be done and is a servant of justice ({\sl dharma})\index{dharma@\textsl{dharma}} of which he is not himself the author. The whole prosperity of the state depends upon the king’s virtue --- just as for Aristotle, the monarch who rules in his own interest is not a king but a tyrant, and may be removed like a mad dog.”

\hfill Coomaraswamy (1977:48)
\end{myquote}

Thus, the kingship envisaged by the Indian traditional doctrine was far removed from despotism, and only a ruler who ruled himself could rule others.  If this subordination of temporal power to spiritual authority is understood, the issue will be resolved.  

The healthy development of kingship to its full potential was achieved thus nowhere else perhaps as in India --- exactly contrary to the contention of Pollock.

