\chapter{{\sl\bfseries Rāmāyaṇa} Casts its Ancient Spell?}\label{chapter3} 


On Sept 25, 1990, L K Advani\index{Advani, L.K.} started on a {\sl Rām-ratha-yātrā} from Somnath to Ayodhya to call for the rebuilding of the Rām {\sl mandir} in Ayodhya. The {\sl ratha} — which resembled that of Arjuna’s in the tele-serial {\sl Mahābhārata} — was welcomed enthusiastically — with zealous cries of “Jai Śrī Rām!” Its route was strewn with flowers, women carried coconuts, incense sticks and sandalwood paste to offer {\sl ārati} to the {\sl ratha} which was then smeared with {\sl tilak} and the dust off its wheels was taken religiously. It did much to stir national unity — also, it allegedly catalyzed the destruction of the Babri Masjid\index{Babri Masjid} two years later.   

Pollock’s article, “{\sl Rāmāyaṇa} and Political Imagination in India” (1993)\endnote{This is Pollock’s only essay that operates on Plane 3 philology}, commences with Advani’s {\sl ratha-yātrā}. In what follows, Pollock attempts to show how the {\sl Rāmāyaṇa}, “a heroic tale of love, loss, and recovery from the classical past should be invoked to empower and give substance to the politics of the present” (Pollock 1993:262). From historic sources — archeological, epigraphical, and literary — he tries to demonstrate that the “{\sl Rāmāyaṇa} imaginary”\index{imaginary} (an image of Rāma as a “destroyer” of {\sl rākṣasa}-s)\index{rakshasa} came to occupy “a public political space” from the twelfth century onwards — when India faced a cultural and political confrontation by the Muslims/Turks. At this convenient juncture, he says, the Hindu rulers used the {\sl Rāmāyaṇa} to consolidate their power — they cast themselves in the role of Rāma and the foreign invaders as {\sl rākṣasa}-s; and this trend, Pollock says, continues to this day. Furthermore, Pollock claims that Vālmīki’s {\sl Rāmāyaṇa} itself carries certain elements/instruments in its narrative that allows for an “easy deployment” of itself for “dangerous” political purposes:\index{political purposes} it is for this reason that it was employed in the twelfth century, and it is for this reason that it continues to be employed for perpetrating “political outrages” today. In his words,   

\begin{myquote}
“I suggest in what follows that the {\sl Rāmāyaṇa} came alive in the realm of public political discourse in western and central India in the eleventh to fourteenth centuries in a dramatic and unparalleled way. I believe the text offers unique imaginative instruments - in fact, two linked instruments - whereby, on the one hand, a divine political order can be conceptualized, narrated, and historically grounded, and, on the other, a fully demonized Other can be categorized, counterposed, and condemned. The makers of elite culture in medieval South Asia chose these instruments for the work of divinization and demonization at this historical moment because of the emergence of two enabling conditions. One was the peculiar salience that a far older political theology now seems to have achieved in the service of the legitimation or enhancement or perhaps just self-understanding of kingship. The other was the appearance of Others who - whether, in fact, they presented an unprecedented unassimilability or could opportunistically be represented as such - were especially vulnerable to the demonizing formulation the {\sl Rāmayana} made available.”
\hfill Pollock (1993:264)
\end{myquote}

Pollock’s views may be summarized thus:

\begin{itemize}
\item[(a)] Vālmīki’s {\sl Rāmāyaṇa} allows for the “divinization” of Hindu political order, and the “demonization” of the Other.\index{Other}  Accordingly, then, the {\sl Rāmāyaṇa} has — for a thousand years — served as a code in which “proto-communist relations could be activated and theocratic legitimation could be rendered” (Pollock 1993:288). 

\item[(b)] From the twelfth century onwards — i.e., simultaneous with the “appearance of the Others” — it came to be employed in equations of politics and power: Hindu kings could be portrayed as the protector/“divine-king” Rāma, and the “outsiders” were {\sl rākṣasa}-s.  

\item[(c)] For this reason, it continues to be deployed today for “dangerous” Hindu-Muslim politics. 
\end{itemize}

Let us examine carefully the substantiation that Pollock offers for his claims.

\section{Archeological evidence (Rise of Rāma temple cult)}\label{sec3.1}
\index{archeological evidence}

Pollock’s first evidence is that of archeology — the “rise of Rāma temples”\index{Rama temples} after the twelfth century. Rāma-worship and “Rāma-cult” commences, says Pollock, somewhere in the twelfth century, “assumes a prominent place within the context of a political theology”, and attains a “centrality” by the fourteenth century. Its growth can, he claims, be traced in virtual synchrony with “a set of particular historical events”— i.e. the “appearance of the Others”. 

In actuality, there is little work that traces the “rise of Rāma cult” in India — and it is perhaps Hans Bakker’s\index{Bakker, Hans} work that we must resort to. Bakker’s argument is similar to that of Pollock’s — he postulates the emergence of Rāma worship “in the latest period of independent Hindu rule in north India” and before the “firm establishment of Muslim power” (Bakker 1986: 66). According to him, worship of the other {\sl avatāra}-s\index{avatara} of Viṣnu were based on “regional, popular, and not specifically Vaiṣnava traditions”— the Rāma cult had to wait for “favorable historical circumstances”. He writes,  

\begin{myquote}
“This seems to have occurred when Hindus were driven into a defensive position by Muslim power, but this factor would never have led to a cult of such dimensions, impact and importance, had not a wave of emotional devotion ({\sl bhakti}) a particular kind completely transformed the outlook and character of Hindu religion, in particular of Vaiṣnavism”.

\hfill Bakker (1986:66)
\end{myquote}

Pollock finishes what Bakker started. His “evidence” is twelfth-century inscriptional references to 

\begin{itemize}
\item[(1)] two temples dedicated to Rāma in the kingdom of the Kalacuris of Ratanpur (in the Chhattisgarh area of Madhya Pradesh), 

\item[(2)] the Rāma complex at Ramtek and 

\item[(3)] the Rāmacandra shrine at Hampi in the kingdom of Vijayanagara\index{Vijayanagara}  (Pollock 1993:266-9). Pollock talks about the construction or reinvigoration of “several major cultic centers” devoted to Rāma after the twelfth-thirteenth centuries, but he does not cite them. He is not sure himself what the situation in the Gāhaḍavāla kingdom of Uttar Pradesh was like — he refers to Bakker’s work on Ayodhya to point out that the “Gāhaḍavāla dynasty begins to develop Ayodhya\index{Ayodhya} as a major Vaiṣṇava center by way of a substantial temple building program” (Pollock 1993:266). He does not cite any inscriptional evidence but nevertheless states with confirmation that a Rāma temple was constructed at Svargadvara ghat, probably by Jayacandra. 
\end{itemize}

Pollock’s substantiation for his grand claims of significant growth of building Rāma temples\index{Rama temples} all over India from the Gupta period onward is decidedly thin. On the other hand, the available documentation shows that temples of Śiva and Viṣṇu were built more than other deities in the period under examination. We see that although the Rāmacandra temple was located in the nucleus of the royal complex at Vijayanagara, the official epigraphic records of the Vijayanagara\index{Vijayanagara} kings mention often their {\sl rāṣṭra-devatā}, Virūpākṣa. Anila Verghese notes, “Pampā Virūpākṣa has undoubtedly been the principal deity at the site from before the founding of the empire onwards, and he was adopted as the guardian deity of the Vijayanagara state” (Verghese 1995:132). Another scholar, G. Michell, writes that the nucleus of Vijayanagara’s early sacred center at Hampi consisted of Śaivite shrines (Michell 1995:31-32). A sample of epigraphical record of the time is given below:

\begin{myquote}
“...  Oh wonder! Though (like Kṛṣṇa) he (King Īśvara) was the son of the glorious Devakī, though (like Viṣṇu) he had lotus eyes, though he acquired tribute ({\sl bali}) by his valor which was able to subdue the three worlds, (just as Viṣṇu in his Vāmanāvatāra acquired the three worlds from Bali by his three steps), and though he bore (the auspicious marks of) the conch and the discus in his hand— he became still more famous by the name of Īśvara, as he obtained prosperity ({\sl bhūti}), universal worship, and the daughter of a king, (just as the god Īśvara wears ashes ({\sl bhūti}) is universally worshipped and is the husband of the daughter of the mountain).\hfill Hultzsch (1892:367)
\end{myquote}

Compare this with the landscape of sacred centers in late thirteen the early fourteenth centuries Andhra. Keilhorn\index{Kielhorn} notes, 

\begin{myquote}
“From the above abstract it will appear that most of the donations recorded here were made in the favor of the god Viṣṇu, under the names of Viṣṇu-bhaṭṭāraka, Nārāyaṇa- bhaṭṭāraka, Vāmanasvāmideva and Chakravāmideva. The same divinity I understand to be denoted by the name Tribhuvanasvāmideva. But besides him, we find among the donees also Umāmaheśvara, clearly a form of the god Śiva, and Bhailasvāmideva, a name in a fragmentary inscription from Bhilda, mentioned by Dr. Hall in the {\sl Journ.Beng}. As. Soc, vol XXXI, p.112, is distinctly given as a designation of Ravi, the Sun’.”

\hfill Hultzsch  (1892:168)
\end{myquote}

Both these records were inscribed at the height of {\sl turuṣka}\index{turuska} penetration into the Deccan. Yet, we see no reference to Rāma or such “political imagination”\index{political imagination} in them. A cursory glance at the list of “royal cults” of the period (epigraphy and actual sites) shows, in fact, that a very striking variety of deities were worshipped in different regions\endnote{See Brajdulal Chattopadhyaya(1998)}: Bṛhadīśvara and Gaṅgaikoṇḍacolapuram were the royal “cult-centers” of Rājarāja Coḷa and Rājendra Coḷa of Tamil Nadu; the coins and epigraphs of the Kadambas consistently refer to Śrīsaptakoṭīśvara as their deity; the Śilāhāra-s always invoked Mahālakṣmī in their inscriptions; the Cālukya-s and Vāghela-s of Gujarat considered Somanātha as their most important deity. In Rajasthan, the site of Ekaliṅga emerged as a major centre of royal cult in the kingdom of the Guhila-s of Mewar; in Orissa, the Bhañja-s worshipped Śiva and Stambheśvarī, and Jagannātha.

Of course these are but a few examples. Yet, it shows that the formation of royal or regional cults cannot be pinned down to a particular time-period nor can it be categorized as “reactionary”.  What Bakker and Pollock do is choose certain insignificant incidents and piece them together to form a convenient narrative. C. Talbot\index{Talbot, C.} also notes, “Inscriptions from Andhra provide little support for Pollock’s thesis, as far as the {\sl Rāmāyaṇa} itself is concerned, for there are few direct references to the epic story.” (Talbot 1995: 696) 

In Chattopadhyaya’s\index{Chattopadhyaya, Brajdulal} words, “the evidence adduced by Bakker and Pollock does not appear extraordinarily significant and can be explained in ways other than what have been advocated by them” (Chattopadhyaya 1998:106). 

\section{Textual evidence (epigraphical)}\label{sec3.2}
\index{epigraphical evidence}

Pollock’s second category of evidence is that of inscriptions. In his words, 

\begin{myquote}
“If the architectural remains associated with Rāma have yet to be systematically worked through and synthetically analyzed, this is even more the case with the inscriptional materials (beyond those associated with a temple cult) that refer to or invoke the god-king or in one way or another process {\sl Rāmāyaṇa} themes (Sircar 1980 and Diskelkar 1960 are the sole, unhelpful guides). So here, too, my findings have to be regarded as provisional, but again I would be surprised if further work would require fundamental revision of my conclusion: The {\sl Rāmāyaṇa} supplies serious material to the political imagination of pre-modern India as coded in the inscriptional record only from the later medieval period on; references in the first millennium are remarkably few but gain in frequency and complexity especially after the twelfth century.”
\hfill Pollock (1993:270)
\end{myquote}

Before the twelfth century, Pollock says, references to Rāma in inscriptions\index{inscriptions} are “static, formulaic allusions”. In this period, he writes, “Rāma and {\sl Rāmāyaṇa} mythemes function as peripheral rhetorical embellishments, inflecting and texturing a given discourse but not constituting it” (Pollock 1993:272). Very different are, according to him, the materials of the succeeding period. After the twelfth century, he says, “the political world comes to be read through-identified with, cognized by-the narrative provided by the epic tale.” (Pollock 1993:272). He furnishes two evidences in support of his stand: 
\begin{itemize}
\item[(a)] the Dabhoi stone inscription of the Vāghelā family of Gujarat (AD 1253) 
\item[(b)] Hansi inscription of 1167, which can be regarded as a {\sl praśasti} of Cāhamāna Pṛthvīrāja II. 
\end{itemize}
Let us examine each in its own context. 

\smallskip
\noindent
{\bf Dabhoi inscription:}\index{Dabhoi inscription} The Dabhoi inscription is a {\sl praśasti} (panegyric) composed by Someśvaradeva, a {\sl purohita} of the Rāṇaka-s of Ḍholkā (author of {\sl Kīrti-kaumudī}), and is dated 14 May 1253 ({\sl vikrama saṁvat 1811 jyeṣṭha śudi 15 vudhadine}). Vīsaladeva, who was then the ruler of Gujarat, had, at the time, ordered the restoration of a Śiva temple at Dabhoi, and in his honor was this {\sl praśasti} composed/inscribed. It contains a total of 116 verses. Let us look at its details verse-by-verse:

\smallskip
\noindent
{\bf (Verse no.) 1-3:} A {\sl maṅgala} to Śiva-Vaidyanātha, and a fragment of it reads, “May glorious Vaidyanātha himself with his eight bodies grant their desires to the creatures.”

\smallskip
\noindent
{\bf 4:} Description of Vīsaladeva’s ancestors— “the line of the progeny of that (man), the good deeds of which (line)… (cannot be described—) even by eloquent men.”

\smallskip
\noindent
{\bf 5:} <Lost>

\smallskip
\noindent
{\bf 6:} “Won over by the eminent qualities of this conqueror of his foes, the guardian goddess (Śrī) of the Gūrjara princes became of her own choice his bride, just as (the goddess Śrī became the bride) of (Viṣṇu), the foe of Bāṇa (at the churning of the ocean).”\endnote{Bühler notes that these verses are identical to {\sl Kīrti-kaumudī} II.2. Here, the lines refer to Mūlarāja, the founder of the Chalukya dynasty of Aṇhilvād, and hence, the verse in the {\sl praśasti} must also refer to the same person (Hultzsch 1982: 21). } 

\smallskip
\noindent
{\bf 7:} (Continues description of Mūlarāja) – (the wives of his enemies tremble or fly into the jungles), “when he, an embodied stream of the sentiment of heroism, stands on the back of… with the intention of fighting.”

\smallskip
\noindent
{\bf 8:} <Lost>

\smallskip
\noindent
{\bf 9:} (Description of Vāghelās, beginning with Arṇorāja) — “By whom, even the son of Dhavala, an imitator of Kṛṣṇa, this realm of famous Gūrjara land was made free from thorns.”\endnote{Bühler notes that this is in line with {\sl Kīrti-kaumudī} 2.63, “By that good warrior who imitated Kṛṣṇa, even by the son of Dhavala, was begun the clearance of the kingdom from thorns.” (Hultzsch 1982: 21).}

\smallskip
\noindent
{\bf 10:} “[Arṇorāja] slew on the battle field Raṇasimha who resembled Rāvaṇa”

\smallskip
\noindent
{\bf 11:} “Now when his son valiant Lavaṇaprasāda \index{Lavanaprasada} (was able to sustain) the load of Gūrjara land, he (Arṇorāja) offered, his heart being averse to the world, a battle-sacrifice at which he which he gave his life as fee.”

\smallskip
\noindent
{\bf 12:} “(Owing to some deeds of Lavaṇaprasāda) the kingdom of the Gūrjaras was even greater than that of Rāma”. 

\smallskip
\noindent
{\bf 13:} Mention of a fight near Vardhamāna (the modern town of Vaḍhvān in North-eastern Kāṭhiāvāḍ) with some unnamed powerful foes.

\smallskip
\noindent
{\bf 14:} “By whom the chief of Naḍūla was deeply wounded with his mighty sword; owing to this (severe stroke), yon kings quake even today, just as the mountains at a thunder-clap”\endnote{Bühler notes that this is identical with {\sl Kīrti-kaumudī}, 2. 69 (Hultzsch 1982: 22). }

\smallskip
\noindent
{\bf 15:} “How many godlike kings are there not on earth? But even all of them became troubled by the mere mention of the king of the {\sl turuṣka}-s. When that ({\sl turuṣka} king), excessively angry, approached in order to fight, (it was Lavaṇaprasāda) who placed only…”

\smallskip
\noindent
{\bf 16:} “By whom (Lavaṇaprasāda), the king of the {\sl turuṣka}-s... who has spattered the earth with the blood flowing from the cut-off heads of numerous kings— when he came in front, with dry lips, full of doubt— was conquered at Stambha with his arm (strong) like a post ({\sl stambha}) and terrible through the sword”.


\smallskip
\noindent
{\bf 17:} ... “if he (Lavaṇaprasāda) is a mortal, how is it that he conquered the lord of the {\sl mlechcha}-s?”

\smallskip
Note here that in Lavaṇaprasāda’s time, there were three Muslim\index{Muslim} attacks on Gujarat: 
\begin{itemize}
\item[(a)] the (unsuccessful) expedition of Shahabuddin Ghori\index{Shahabuddin Ghori}  (1178 C.E) 
\item[(b)] the first expedition of Qutbuddin\index{Qutbuddin} in 1194 C.E (during which Aṇhilvāḍ was sacked), and 
\item[(c)] the second expedition of Qutbuddin in 1196 C.E, which was unsuccessful but led to the temporary conquest of Gujarat (and the temporary occupation of Aṇhilvāḍ). G. Bühler notes that this inscription perhaps does not refer to any of the three battles. He writes, “… the most probable conjecture seems to me that it [the battle that the Dabhoi inscription refers to] happened after the occupation of Aṇhilvāḍ in 1196. Sometime later the Muhammadans did suffer a defeat in Gujarat and the province shook their yoke off. The details of these events are not given either by the Muhammadan or the Hindu authors; but our passage probably contains an allusion to them, and it may be that Lavaṇaprasāda was the liberator of his country.” (Hultzsch 1892: 23). 
\end{itemize}

\smallskip
\noindent
{\bf 18:} “(Lavaṇaprasāda), a repository of medicine like valor, cured (his country) when the crowd of the princes of Dhārā, of the Dekhan and of Maru, who resembled diseases (attacked it)”.

\smallskip
\noindent
{\bf  19:} “He (Lavaṇaprasāda)\index{Lavanaprasada@Lavaṇaprasāda} who raises his race, seems to me greater than Yudhiṣṭhira, whose relatives were all destroyed, though their power to remove a Salya is equal”. 

\smallskip
\noindent
{\bf 20:} <Lost>

\smallskip
\noindent
{\bf 21:} (referring to the erection of a temple in Kumāra near Vaḍhvāṇ), “Who (Lavaṇaprasāda) caused to be erected in the neighborhood of Vardhamāna, a (temple of) Kumāra rivaling the ocean (in the possession of treasures) and surpassing the moon (in brilliancy)”.

\smallskip
\noindent
{\bf 22-24:} Not clear

\smallskip
\noindent
{\bf 25:} (Praise of Vīradhavala) “From him sprang a son, who was the image of Daśaratha and Kakutstha, who swallowed like a mouthful the armies of hostile kings— Vīradhavala. When the flood of his fame spread, the cleverness of faithless women, whose minds are distressed with the longing after enjoyments— in the art of approaching (their lovers) was destroyed”. 

\smallskip
Bühler notes that in “the remaining verses referring to Vīradhavala, 26-51, little more than single letters or words are legible, except in verse 45, where an unsuccessful combined attack of the lord of Dhārā and of the ruler of the Dekhan is mentioned. The portion of the {\sl praśasti} which celebrates Vīsaladeva’s great deeds and virtues, verses 52 – 108, is likewise in a very bad condition…” (Hultzsch 1892: 23)

So the Dabhoi inscription\index{Dabhoi inscription} refers to Gūrjara kingdom\index{Gurjara kingdom@Gūrjara kingdom} ruled over by Lavaṇaprasāda as “greater than {\sl Rāma-rājya}” (verse 12). It also mentions the defeat of the {\sl turuṣka} king by Lavaṇaprasāda who, the inscription asserts, could not be a mere mortal (verse 17). However, the “meaning-conjuncture” — an expression which Pollock uses to point to the identity of the king as victor over the {\sl turuṣka}-s with Rāma, the slayer of Ravaṇa — does not take place in this record.  

Interestingly, the record refers to Arṇorāja, founder of the Vāghela line, as imitating the feats of Kṛsna, and his adversary Raṇasiṁha (not a {\sl turuṣka}), slain on battlefield, is called Rāvaṇa (verse 10). Lavaṇaprasāda, victor over the {\sl turuṣka} king, is mentioned to be more famous than Yudhiṣthira (verse 19). His son Vīradhavala, was “the image of Daśaratha and Kākutstha” (verse 25). We see that the composer of the record drew upon a repertoire of available motifs — and the “political mytheme” of Rāma v/s Rāvaṇa was not one of them. 

How this substantiates Pollock’s claim, only Pollock can explain. 

\smallskip
\noindent
{\bf Hansi record:}\index{Hansi record} The Hansi record (1224) constitutes 22 lines of writings — partly prose, partly verse. It is also a {\sl praśasti},\index{prasasti@praśasti} and its aim was to describe Kilhaṇa’s conquest of Paṁchapura. Kilhaṇa was a maternal uncle and feudatory of the Cāhāmana king, Pṛthvirāja. Kilhaṇa was put in charge of the fort of Hansi to check the progress of Hammīra, the Muslim emperor. Let us examine its details— it reads as follows:

\newpage

\noindent
{\bf (Verse) 1:} Obeisance to [an unspecified] Goddess, and invokes the blessings of the god, Murāri.

\smallskip
\noindent
{\bf 2:} Informs us that there was a king of the Cāhamāna lineage called Pṛthvīrāja, and his maternal uncle called Kilhaṇa. 

\smallskip
\noindent
{\bf 3:} Informs that Hammīra had become the cause of anxiety to the world, and Pṛthvirāja put Kilaṇa in charge of the fort of Hansi. 

\smallskip
\noindent
{\bf 4:} Kilhaṇa belonged to the race of Guhilauta. 

\smallskip
\noindent
{\bf 5:} Kilhaṇa erected a {\sl pratolī} (gateway) which with its flags set Hammīra as it were at defiance. 

\smallskip
\noindent
{\bf 6:} Near the gateway were constructed two {\sl koṣṭhaka}-s (granaries)

\smallskip
\noindent
{\bf Lines 9-10} (prose) speak of a letter sent to Kilhaṇa\index{Kilhana@Kilhaṇa} by Vibhīṣaṇa.

\smallskip
\noindent
{\bf 7:} “(the letter begins) The lord of demons (Vibhīṣaṇa)\index{Vibhisana@Vibhīṣaṇa} who has obtained a boon from Rāma, the crest jewel of the lineage of Raghu, respectfully speaks thus to Kilhaṇa staying in the fort of Hansi”.  

\smallskip
\noindent
{\bf 8:} “In the work of the building the bridge, we both assisted the leaders of the monkeys and bears. And you yourself (Kilhaṇa) have written saying that the lord of Paṁchapura, a string of pearls and this city, had been given to you by the Omnipresent Rāma.”

\smallskip
\noindent
{\bf 9:} In this verse, Pṛthvirāja is compared to Rāma and Kilhaṇa to Hanumān. 

\smallskip
\noindent
{\bf 10:} Vibhīṣaṇa bestows conventional praise on Kilhaṇa. 

\smallskip
\noindent
{\bf 11:} Refers to his having burnt Paṁchapura, and captured but not killed its lord.

\smallskip
\noindent
{\bf 12:} Eulogy of Kilhaṇa

\smallskip
\noindent
{\bf 13:} Vibhīṣaṇa requests Kilhaṇa to accept the string of pearls or even Laṅkā but promise safety to him.

\smallskip
\noindent
{\bf Line 19-20} is prose – it informs us that this string of pearls was presented by the ocean to Rāmabhadra when he was intent upon constructing the bridge.

\newpage

\noindent
{\bf 14-15:} One Valha who belonged to the Ḍoḍa race and who was a subordinate of Kilhaṇa and that his son was Lakṣmaṇa under whose auspices the praśasti was composed.

This is followed by the date: Thursday, 7th of the bright half of Māgha of the Vikrama year 1224 (1170 C.E).

We see that this {\sl praśasti} draws an identification of the Cāhamāna king Pṛthvīrāja with Rāma, and of Kilhaṇa, Pṛthvīrāja’s maternal uncle, with Hanumān. However, if the record is located within the context of other contemporary inscriptions, it yields a different picture: Usually, in most Indian texts, heroes are identified with Viṣnu, or Mahāvarāha (who lifts the earth submerged in the ocean of {\sl turuṣka} rule), or as Agastya (who is the swallower of the ocean) (Chattopadyaya 1998:110). In the medieval epigraphs, therefore, “whether in the context of Yavana raids or outside them, the king— as a hero and a ruler— has many identities: Indra, Viṣnu, Trivikrama, Mahāvarāha, Śiva, Pṛthu, Agastya, Kāma, Revanta, Yudhiṣṭhira, Bhīma, Rāma, and so on” (Chattopadyaya 1998: 110). This point is better illustrated by the juxtaposition of extracts from other inscriptions of the period under examination. 

Consider the evidence furnished by the following five inscriptions: 
\begin{itemize}
\item[{\bf 1.}] Ajaygadh rock inscription\index{Ajaygadh rock inscription} of Candella Vīravarman (1261 C.E) 

“... After him, Pṛthvīvarman was king, similar to Pṛthu; and then Madana ruled over the kingdom, a god of love to the opponents. Then came the illustrious king Paramārdin, who, as a leader, even in his youth, struck down opposing heroes… then the prince Trailokyavarman ruled the kingdom, a very creator in providing strong places. Like Viṣnu he was, in lifting up the earth, immerged in the ocean formed by the streams of {\sl turuṣkas}. Victorious is his son Vīṛa, that ruler of the earth of spotless bravery who has delighted the damsels of heaven by sending them, as lovers, the hostile heroes whom he cut down on the field of battle. Victorious (and) to be worshipped by all men is he whom when he strikes down the wicked (and) disperses crowds of opponents, people gaze at— wondering whether he be Viṣnu riding on Garuḍa or Śiva about on his bull.” 

\hfill (Hultzsch 1892: 329)

\newpage

\item[{\bf 2.}] Bitragunta Grant\index{Bitragunta grant} of Saṅgama II (1278 C.E): 

“... from him were produced five heroic sons, as, formerly, the (five) celestial trees from the milk ocean— first, king Harihara; the, the ruler of the earth, Kampa; then, the protector of the earth, Bukka; (and) afterwards, Mārapa and Muddapa. Of these, king Harihara— by whom the Sultan (Suratrāṇa), who resembled Sutrāman (Indra) was defeated— ruled the earth for a long time. His younger brother, king Kampana, whose name became true to its meaning as he made the enemies tremble, ruled the earth for a long time…. Into the flames of his [Harihara II’s] valor the {\sl yavana, turushka} and Andhra hostile kings fell like moths.. (Heras 1929: 120) 

\item[{\bf 3.}] Satyamangalam Plates\index{Satyamangalam plates} of Devaraya II (1346 C.E): 

“Through the wind (which was produced) by the flapping of the ears of his elephants on the field of battle, the {\sl tulushka} horsemen experienced the fate of cotton (were blown away)…” (Hultzsch 1982: 40)

\item[{\bf 4.}] Machchishahr Copper Plate\index{Machchishahr copper plate} Inscription (1197 A.D): 

“To him was born a king called Vijayacandra who was capable ({\sl dakṣa}) of destroying ({\sl viccheda}) the allies ({\sl pakṣa}) of (enemy) kings ({\sl bhūbhṛt}); just as Indra is capable ({\sl dakṣa}) of cutting under the wings of the (fabulous flying) mountains and who (Indra), had washed off the heat of the terrestrial world with streams of water from the clouds in the shape of the eyes of the Hammīra women, when he was indulging in the sport of subduing the world (?).”
\hfill (Prasad 1990: 67)

\item[{\bf 5.}] Batesvar Chandella inscription\index{Batesvar Chandella inscription} of Paramardideva (1195): 

“Among them appeared the lord of the earth Madanavarman, who with his flashing sword scattered (his) adversaries (and) whose vigor became known by his onslaught on hostile kings, elated with pride; (resembling) the great Indra who cut off the wings of the mountains with his thunderbolt (and) whose might became famous by his killing (the demon) Vala”.

\hfill (Hultzsch 1892: 212)
\end{itemize}

The few samples suffice to show that Pollock has arbitrarily isolated Rāma from the variegated world of a host of divinities and legendary kings, just to suit his theory/purpose and has wilfully and skillfully avoided any reference to others. 

\section{Textual Evidence (Literary)}\label{sec3.3}
\index{literary evidence}

Pollock’s last category of evidence is what he calls “historiographical” or literary evidence (Pollock 1993:273). Pollock takes just two literary examples to strengthen his argument: 
\begin{itemize}
\item[(a)] Hemacandra’s {\sl Dvyāśrayakāvya}
\item[(b)] Jayāṅka’s {\sl Pṛthvīrāja-vijaya.}
\end{itemize}

Admittedly, the two texts consider their respective kings, Jayasiṁha Siddharāja and Pṛthvīrāja III Cāhamāna as incarnations of Rāma, and the latter text, in particular, dwells at length on the depredations by the {\sl turuṣka}-s in the region of Ajayameru, Rajasthan. Yet, to say that this lends incontestable support to Pollock’s conjecture of “mythopolitical equivalence” is a stretch of imagination. In the period under examination, there were innumerous forms of comparisons, and no one form dominated the other — for example, {\sl kāvya}-s like Gaṅgādevī’s {\sl Madhurā-vijaya}, Nayacandra’s {\sl Hammīra Mahākāvya}, etc are literary works that talk of {\sl turuṣka} invasions without reference to Rāma, and texts like Sandhyākaranandin’s {\sl Rāma-carita} use the “{\sl Rāmāyaṇa}-mytheme” in the context of a local warfare that is not {\sl turuṣka}.

Let us look at Gaṅgādevi’s {\sl Madhurā-vijaya} to understand how other images (other than the “{\sl Rāmāyaṇa} imaginary”) were used to depict the Muslims. {\sl Madhurā-vijaya} is a {\sl mahā-kāvya} written in the second half of the fourteenth century in celebration of Gaṅgādevī’s husband’s victory over the {\sl turuṣka}-s of Madurai. Its eighth {\sl sarga reads} thus: 

\begin{myquote} 
............................................... {{\sl vyāghra-purīti sā yathārtham}} || 1 ||\\
{\sl adhiraṅgam avāpta-yoganidraṁ harim udvejayatīti jāta-bhītiḥ} |\\
{\sl patitaṁ muhur iṣṭakā-nikāyaṁ phaṇa-cakreṇa nivārayaty ahīndraḥ} || 2 ||\\
{\sl ….nughūrṇad ūrṇanābhaṁ vana-vetaṇḍa-vimardinīm avasthām} |\\
{\sl viratāny aparicchada-prapañco bhajate hanta ! gaja-pramāthi-nāthaḥ} || 3 ||\\ 
{\sl ghuṇa-jagdha-kavāṭa-sampuṭāni sphuṭa-dūrvāṅkura-sandhi-maṇḍapāni} |\\
{\sl ślatha-garbha-gṛhāṇi vīkśya dūye bhṛśam anyāny api devatā-kulāni} || 4 ||\\
{\sl mukharāṇi purā mṛdaṇga-ghoṣair abhito deva-kulāni yāny abhūvan} |\\
{\sl tumulāni bhavanti pheravāṇāṁ ninadais tāni bhayaṅkarair idānīm} || 5 ||\\
{\sl satatādhvara-dhūma-sourabhaiḥ prāṅ-nigamodghoṣaṇavadbhir agrahāraiḥ} |\\
{\sl adhunājani visra-māṁsa-gandhair adhika-kśība-tuluṣka-siṁha-nādaiḥ} || 6 ||\\
\hspace{5cm} ...\qquad
\end{myquote}

The verses may be translated as, 

\begin{myquote}
“O King! That city, which was called “Madhurapurī” for its sweet beauty, has now become the city of wild animals, making true its older name “Vyāghra-purī”, the city of tigers, for humans dwell there no longer”. (1) 

“The famed temple of Śrī Raṅgapaṭtana has fallen to decay, and its structure being reduced to rubble. So much that Viṣnu who famously slept there in his deep {\sl yoga-nidrā}, has now literal protection only of the hood of Ādiśeṣa who has to be ever cautious from the falling bricks of the debris”. (2) 

“How do I describe the condition of the abode of the slayer of Gajāsura! In the bygone days, after slaying Gajāsura, Lord Śiva had taken its skin for his garment. And now being stripped he has gone back to being {\sl digambara}. Wild elephants have now made the Śiva-liṅga their plaything, and all but spider-webs are the decorations of his abode.” (3) 

“[when such is the state of those famous temples] how would other {\sl devasthāna}-s be any better! Moth have eaten away the once-beautiful wooden structures, the maṇdapa-s have developed cracks in which now grass grows, and {\sl garbha-gṛha}-s of many others are dilapidated and crumbling. My Lord, my heart is crying as I describe to you the situation of our beloved {\sl devatā-kula}.” (4) 

“Those {\sl deva-mandira}-s which used to resound with the joyous and pious beats of {\sl mṛdaṅga}, today only the echo of fearful howls of jackals can be heard there.” (5) 

That Gaṅgā of South, mighty Kāverī, which used to earlier flow in proper channels curbed with dams created by our noble rulers of past – she now flows like a vagabond without discipline; like her new lords, these {\sl turuṣka}-s, her dams being breached beyond repair. (6)…” 

\hfill {\sl Madhurā-vijaya} (8.1-6) [{\sl Trans. S.K  Tiwari}]
\end{myquote}

So it continues, and further the situation is shown to escalate to cataclysmic proportions, and Gaṅgādevī’s husband takes the form of Mahā-varāha to restore peace. 

Nowhere, we can see, are motifs from the {\sl Rāmāyaṇa} used in the descriptions here.  

On the other hand, let us take the example of Sandhyākaranandin's {\sl Rāma-caritam} – this is a {\sl śleṣa kāvya} (paronomastic) that produces two levels of meanings at the same time: on one level, the kāvya narrates Sītā’s deliverance after the slaying of Rāvaṇa by Rāma.  On the second level, it narrates the account of the Pāla ruler (Sandhyākara’s patron), Rāmapāla, and his slaying of the Kaivarta king, Bhīma, (who usurped the territory of Varendrī for a short time). We can see an application of Pollock’s “{\sl Rāmāyaṇa} mytheme”, if that, in a context that does not involve Muslims. 

Moreover, {\sl Pṛthvīrāja-vijaya}’s “demonization” of the Ghurids cannot be interpreted as a reflection of a true Hindu “abhorrence” of Muslims. Jayāṅka was simply engaging in a mode of dehumanizing the Ghaznavid and Ghurid rivals of the Cāhamānas that was frequently employed to designate foes in literature — Muslim or otherwise. 

Note, for example, the following four references in Śyāmilaka’s {\sl Pāda-tāḍitaka}, a {\sl bhāṇa} of the Gupta period: 
\begin{itemize}
\item[(1)] “Aha, this is the character of a Diṇḍi. The Diṇḍi-s are not very different from the monkeys, what does he then find lovable in the Diṇḍi-s?”

\item[(2)] “Here is a man with the face of a he-goat, whose loins are covered with a piece of cloth, and whose shoulders are full of thick hairs. (Besides) he comes biting a radish. If he is not a Dāśeraka then he must be a devil” 

\item[(3)] “What merit has he discovered in this (slave) maid Barbarikā...? Moreover, this Barbarī, the veritable goddess of darkness with whiteness in the teeth and eyes only, appears like night with a very strip of the crescent moon. But this is not strange. For the men of Saurāṣṭra and the monkeys are all of the same class” 

\item[(4)] “ ... who will listen to the Yavana courtesans’ words which are like the chattering of a monkey, full of shrill sounds and of indistinguishable consonants and which are interspersed with the (occasional) display of the forefingers?” (Chattopadhyaya 1998:81).\index{Chattopadhyaya, Brajdulal} 
\end{itemize}

Similarly, in the same Hemacandra’s {\sl Dvayāśraya mahākāvya} that Pollock cites, the character of Grāharipu, ruler of Saurāṣṭra-deśa, is described as a cruel tyrant, anti-religious, killer of pilgrims, and he is one who causes calamities, plunders people and destroys forts and important place. (Chattopadhyaya 1998: 81). But Pollock admits that, “True the Cālukyas could imagine the Colas as {\sl rākṣasa}-s, or the Colas could thus position the Sinhalese” (Pollock 1993:283). It is, however, not enough to simply refer to counter-evidence; it must be shown how the same can be reconciled with the notion of “the utter dichotomization of the enemy”. 

Muslims, a closer scrutiny will show, were not universally reviled. Representations of Muslims generally oscillated depending on the prevailing political conditions: in times of military conflict and radically fluctuating spheres of influence, the rhetoric was often negative in tone; whereas long established Muslim rulers were conceptually assimilated into the Sanskritic political imagination. 

Cynthia Talbot shows in her extensive work on “Hindu-Muslim identities in pre-colonial India” (1995) that anti-Muslim polemics reflected a defensive stance of the Hindus, but “from the early fifteenth through mid-sixteenth centuries, there was little dramatic change in the power balance, and tensions subsided momentarily… in this… phase, we witness no demonization of Muslim” (Talbot 1995:706). 

Chattopadhyaya also notes, “… the reality… could be represented in two ways. In one representation, the destroyer of the Yavana who is a destroyer of social order is comparable to Viṣṇu; in another, Yavana, as a benign ruler, gives succor to Viṣṇu who, leaving the burden of preservation to the ruler, retires to peaceful sleep in the ocean of milk. It cannot be argued that chronologically, one representation replaces the other” (Chattopadhyaya 1998:60). 

We see, therefore, that Pollock’s method of isolating references relating to Rāma from their larger contexts has the effect of exaggerating their importance. One must not settle for such neat narratives, and must, instead, work towards a more sophisticated theorizing. 


\section*{Conclusion}

\medskip
\begin{flushright}
\begin{tabular}{c@{}}
{\sl\bfseries Kāvye rasayitā sarvo na boddha na niyogabhāk}\\[3pt]
{\sl\bfseries  Bhaṭṭa Nāyaka}\index{Bhatta Nayaka@Bhaṭṭa Nāyaka}
\end{tabular}
\end{flushright}


“Arms and the man I sing”— so run the opening lines of Virgil’s\index{Virgil} {\sl Aeneid}. These few words reflect, wrote Paul Cantor, the whole essence of (ancient) epic poetry: warfare and politics. In his words, 

\begin{myquote}
“...Homer and Virgil… they do single out the warrior’s life as the central theme of epic poetry. Even Shakespeare, with his wider range as a poet, focuses his serious plays, his histories and tragedies, on public figures and the central political issue of war and peace… The traditional concept of epic and tragedy as the supreme genres and the pinnacle of literary achievement effectively placed political life at the center of poetic concern.”\hfill (Cantor 2007: 375)
\end{myquote}

\medskip
Perhaps it is true of Western Epics. Perhaps it is not. What is certain is that it does not reflect the tenor of Sanskrit {\sl kāvya}-s — and certainly not of the {\sl Rāmāyaṇa}. For Vālmīki’s {\sl Rāmāyaṇa} is, first \& foremost, a {\sl kāvya}\endnote{In no uncertain terms, the {\sl Rāmāyaṇa} characterizes itself as a {\sl kāvya}. See Kane (1966) Alf Hiltebeitel (2005), Shubha Pathak (2007). In his preface to “The Sanskrit Epics”, J.L. Brockington raises a question: “Is it... worth asking from the start whether the designation of the {\sl Mahābhārata} and the {\sl Rāmāyaṇa} as “epics” affects our understanding of them, generating expectations derived from ideas about the {\sl Illiad} and {\sl Odyssey}”.} — and {\sl kāvya} — like the other arts — is/was considered a magnificent form of {\sl yoga} in India. 

\smallskip
In his essay, {\sl The Theory of Art in Asia}, Coomaraswamy demonstrated the formal steps in the {\sl yoga} of “making of an artifact”: 

\smallskip
\begin{myquote}
“...[the artist], having by various means proper to the practice of {\sl Yoga} eliminated the distracting influences of fugitive emotions and creature images, self-willing and self-thinking, proceeds to visualize the form... The mind “produces” or “draws” ({\sl ākarṣati}) this form to itself, as though from a great distance. Ultimately, that is, from Heaven, where the types of art exist in formal operation; immediately, from the “immanent space in the heart” ({\sl antar-hṛdaya-ākāṣa}), the common focus ({\sl samstāva}, “concord”) of seer and seen, at which place the only possible experience of reality takes place. The true-knowledge-purity-aspect ({\sl jnana-sattvarūpa}) thus conceived and inwardly known ({\sl antar-jneya}) reveals itself against the ideal space ({\sl ākāṣa}) like a reflection ({\sl pratibimbavat}) or as if seen in a dream ({\sl svapnavat}). The imager must realize a complete self-identification with it ({\sl ātmānam… dhyāyāt, bhāvayet}), whatever its peculiarities, even in the case of the opposite sex or when the divinity is provided with terrible supernatural characteristics; the form thus known in an act of non-differentiation, being held in view as long as may be necessary ({\sl evam rūpam yāvad ichchati tāvad bhāvayet}), is the model from which he proceeds to execution in stone, pigment, or other material.” 
\hfill (Cooomaraswamy 1934: 5)
\end{myquote}

In an uncannily similar terminology, we are told in the first {\sl kāṇḍa} of the {\sl Rāmāyaṇa} ({\sl Rāmāyaṇa} 1.3.2-8):

\begin{myquote}
“Vālmīki, although he was already familiar with the story of Rāma, before composing his own {\sl Rāmāyaṇa} sought to realize it more profoundly, and seating himself with his face towards the East and sipping water according to rule (i. e. ceremonial purification), he set himself to yoga-contemplation of his theme. By virtue of his yoga-power he clearly saw before him Rāma, Lakṣmaṇa and Sītā, and Daśaratha, together with his wives, in his kingdom laughing, talking, acting and moving as if in real life ... by yoga-power that righteous one beheld all that had come to pass, and all that was to come to pass in the future, like a {\sl nelli} fruit on the palm of his hand. And having truly seen all by virtue of his concentration, the generous sage began the setting forth of the history of Rāma”.   
\hfill (Coomarswamy 1918: 23)
\end{myquote}

“As a man among men,” F. M. Cornford writes in his discussion of inspiration among the Greeks, “the poet depends upon hearsay [as when Vālmīki heard the story of Rāma from Nārada]; but as divinely inspired [Vālmiki in {\sl yogic} vision], he has knowledge of an eyewitness, ‘present’ at the feats he illustrates.” (Cornford 1952:76-77) To Pollock, this is simply a romantic, “sentimental” narration (Pollock 2006:99). But we must concede that when the ancients — Indians, Greeks, or others — called upon the Muses — who were ‘present and knew all things’— to tell them what common mortals like themselves could not know, it is likely that their meaning was more serious than we suppose. Knowledge was understood to be directly accessible to the concentrated and ‘one-pointed’ mind, without the direct intervention of the senses. In the language of psycho-analysis,  “the willed introversion of a creative mind, which, retreating before its own problem and inwardly collecting its forces, dips at least for a moment into the source of life, in order there to wrest a little more strength from the mother for the completion of its work,” (Hinkle 1965: 337). 

Indian theory of art is very clear that instruction is not the primary purpose of art. Abhinavagupta shows in the clearest terms that {\sl vyutpatti} (didactic education) is — not consciously aimed at, but always is — a by-product of {\sl rasānubhava} of a {\sl kāvya}:

\newpage

\begin{myquote}
{{\sl na hi teṣāṃ vākyānām agniṣṭomādi-vākyavat satyārtha-pratipādana-dvāreṇa pravartakatvāya prāmāṇyam anviṣyate, prītamātra-paryavasāyitvāt / prītereva cālaukika-camatkāra-rūpāyā vyutpattyaṅgatvāt / Locana,}} Uddyota 3.33)


 “... from the sentences of poetry we do not seek for the performance of certain acts on the basis of the tranmission by the sentence of a meaning that is true, as we do from such Vedic setences as “{\sl agniṣṭomam juhuyāt}” (one must offer a fire sacrifice). This is because the end of poetry is pleasure, for it only by pleasure, in the form of an otherworldly delight, that it can serve to instruct us.”
\hfill  [{\sl Trans. Ingalls et al}]
\end{myquote}

So, the “purpose” of a {\sl kavi} is never self-expression/political-agenda/\-social-propaganda. It is unfortunate and ironical in equal measure that the “application” of Pollock’s “inclusive” and “pluralistic” philological tool leads him, not to three “radically different dimensions of truth(s)”, but to a {\sl singular} truth on all three modes of philology: that the {\sl Rāmāyaṇa} is a work that deals with power-dynamics. 

Yet, it must be admitted that ‘interpretation’ is a landscape peppered with difficulties — it is the famous “hermeneutic circle” that perplexed Dilthey, {\sl et al}. Perhaps it must also be admitted that it is impossible to ascribe a single meaning to a text. But in Matthew Kapstein’s words, 

\begin{myquote}
“...if a limitless horizon of possible understandings begins to open before us, we balk nevertheless at the thought that any understanding is just as good as any other. Even if guided by a Kabbalistic conception of the plenitude revealed in each letter of scripture, we retreat before the prospect that all interpretive possibilities must be treated as equal. However, we find ourselves at a loss to specify sure principles that would permit us to delineate between an unlimited range of acceptable or fruitful understandings and an unending field of fantasies that we wish to rule out of court.”
\hfill (Ganeri 2017: 15) 
\end{myquote}

Nevertheless, Kapstein offers a response to this “conundrum”: he calls the Indologists to play this game (of interpretation) by employing traditional Indian hermeneutics — “whether embodied in written commentaries or in the living expertise of traditionally educated scholars” — as a compass to “forge pathways through a conceptual topography” (Ganeri 2017: 16). Going the Kapstein way, then, and locating the {\sl Rāmāyaṇa} within the aims of Indian tradition, we can see that the text is not simply a social or political enterprise, but a magnificent work that is aligned to the ultimate purpose of life. Pollock, on the other hand, is clear:

\newpage

\begin{myquote}
“These problems can be formulated through a large comparative generalization. If Homer, for example, addresses, a transcendent problem, showing us what makes life finally impossible— in the words of one writer, “the universality of human doom”— Vālmīki poses the more difficult question: What is it that makes life possible? This is more difficult because {\sl it is a social, not a cosmic question}.” 

\hfill Pollock (2007a: 4) [{\sl italics ours}]
\end{myquote}

To Pollock, the question of {\sl dharma}\index{dharma} is social and political; to Vālmīki, it is universal and trans-mundane. From this position, both stand opposite to each other locked, as it were, in a zero-sum struggle for meaning — and the twains may never meet. 


\begin{thebibliography}{99}
\bibitem[]{chap3_item0}
(This is a dual reference bibliography. Sanskrit texts are noted both under their own names (nominally) and under the names of their editors (in detail). This collapses the usual double lists into one – the primary sources (Sanskrit) and the secondary sources.)
\itemsep=2pt
\bibitem[]{chap3_item1}
{\sl Aurobindo}, S. (1995, 1922$^{1}$). {\sl Essays on the Gita}. New Delhi. Lotus Press. 

\bibitem[]{chap3_item2}
{\sl Abhijñāna-śākuntalam} of Kālidasa. See Kale (2010). 

\bibitem[]{chap3_item3}
{\sl Abhinava-bhārati} of Abhinava Gupta. See Pande (1997)

\bibitem[]{chap3_item4}
{\sl Aitareya-brāhmaṇa}. See Haug (1863)

\bibitem[]{chap3_item5}
{\sl Āpastamba Dharma-sūtra}. See Pandeya (2006)

\bibitem[]{chap3_item6}
Aristotle’s {\sl Metaphysics}. See Sachs, Joe. (1999)

\bibitem[]{chap3_item7}
{\sl Artha-śāstra} of Kauṭilya. See Kangle (2014)  

\bibitem[]{chap3_item8}
{\sl Atharvaveda} See Griffith, Ralph; Keith, Arthur. B (2017).

\bibitem[]{chap3_item9}
Bailey, Cyril. (1921). {\sl Lucretius’ On the Nature of Things}. Oxford. Clarendon Press. 

\bibitem[]{chap3_item10}
Baker, Hugh. D. R. (1979). {\sl Chinese Family and Kinship}. New York. Columbia University Press. 

\bibitem[]{chap3_item11}
Bakker, Hans. (1986). {\sl Ayodhyā}. Netherlands. Groningen. 

\bibitem[]{chap3_item12}
Bakker, Hans. (1987). “Reflections on the evolution of Rāma devotion in the light of textual and archaeological evidence”, {\sl Wiener Zeitschrift für die Kunde Südasiens und Archiv für indische Philosophie}, Band XXXI (1987), pp. 9-42. 

\bibitem[]{chap3_item13}
{\sl Bṛhadāraṇyakopaniṣad}. See Swahananda (2016)

\bibitem[]{chap3_item14}
Belvalkar, S. K (Ed.) (1924). {\sl Kāvyādarśa} of Danḍin.  Poona. Bhandarkar Oriental Research Institute. 

\bibitem[]{chap3_item15}
Belvalkar, S. K (Ed.) (1945). {\sl Bhagavad-gītā}. Poona. Bhandarkar Oriental Research Institute. 

\bibitem[]{chap3_item16}
{\sl Bhagavad-gītā}. See Belvalkar (1945)

\bibitem[]{chap3_item17}
{\sl Bhāgavata-purāṇa}. See Tapasyananda (2003). 

\bibitem[]{chap3_item18}
Birdwood, George. (1915). {\sl Sva}. London. London Philip Lee Warner. 

\bibitem[]{chap3_item19}
Brockington, J.L.(1998). {\sl The Sanskrit Epics}. Netherlands. Brill.

\bibitem[]{chap3_item20}
Buhler, George (Trans). (1882). {\sl Vaṣiṣṭha Dharma-sūtra}. Bombay. Bombay Sanskrit and Prakrit Series. 

\bibitem[]{chap3_item21}
Caland, W., Vira, Raghu. (1983, 19261). {\sl The Śatapatha Brāhmaṇa in the Kāṇvīya Recesion}. New Delhi. Motilal Banarsidass. 

\bibitem[]{chap3_item22}
Calvert, George, H (trans). (1846). {\sl Correspondence Between Schiller and Goethe, from 1794 to 1805 (Vol 2)}. New York \& London. Wiley and Putnam.  

\bibitem[]{chap3_item23}
Cantor, Paul. (2007). “The Politics of the Epic: Wordsworth, Byron, and the Romantic Redefinition of Heroism”. {\sl The Review of Politics}, Vol. 69, No. 3, Special Issue on Politics and Literature (Summer, 2007). pp. 375-401.

\bibitem[]{chap3_item24}
Chakravarti, Dilip. K. (1990) {\sl The External Trade of the Indus Civilization}. New Delhi. Munshiram  
Manoharlal. 

\bibitem[]{chap3_item25}
{\sl Chāndogyopaniṣad}. See Swahananda (2016)

\bibitem[]{chap3_item26}
Chattopadhyaya, Brajdulal. (1998). {\sl Representing the Other?} New Delhi. Manohar Publishers. 

\bibitem[]{chap3_item27}
Chen, Ivan. (1908). {\sl Book of filial duty}. London. Library of Alexandria. 

\bibitem[]{chap3_item28}
Clift, P.D, Carter A, Giosan L, Duncan J, Duller GAT, Mcklin MG, Alizai A, Tabrez AR, Danish M, Vanlaningham S, Fuller DO. (2012). “U-Pb zircon dating evidence for a Pleistocene Sarasvati River and capture of the Yamuna River”. In Geology 40(2): 211-214

\bibitem[]{chap3_item29}
Chowdhury, A.M. (2012). {\sl The Ramacharitam}. Bangladesh. Asiatic Society of Bangladesh. 

\bibitem[]{chap3_item30}
Concord, F. M. (1952). {\sl Principium Sapientiae: The origins of Greek Philosophical thought}. Cambridge. Cambridge University Press. 

\bibitem[]{chap3_item31}
Coomaraswamy, A. K. (1918). {\sl The Dance of Shiva}. New York. The Sunwise Turn Inc. 

\bibitem[]{chap3_item32}
Coomaraswamy, A.K. (1934). {\sl The Transformation of Nature in Art}. New York. Dover Publications.

\bibitem[]{chap3_item33}
Coomaraswamy, A.K. (1940). {\sl East and West and Other Essays}. Colombo. Ola Books Ltd. 

\bibitem[]{chap3_item34}
Coomaraswamy, A.K. (1946). {\sl Religious Basis of the Forms of Indian Society}. Montana. Literary Licensing, LLC

\bibitem[]{chap3_item35}
Coomaraswamy, A.K. (1977). “The Bugbear of Democracy, Freedom, and Equality”. {\sl Studies in Comparative Religion}, Vol. 11, No. 3. (Summer, 1977). 

\bibitem[]{chap3_item36}
Danino, Michel. (2010). {\sl The Lost River: On the Trail of the Sarasvati}. Haryana. Penguin India.  

\bibitem[]{chap3_item37}
Davids, Rhys. T. W. (1899). {\sl Dialogues of the Buddha}. New York. Oxford University Press. 

\bibitem[]{chap3_item38}
{\sl Dhvanyāloka} of Anandavardhana. See Pathaka  (1965)

\bibitem[]{chap3_item39}
Drekmeier, Charles. (1962). {\sl Kingship and Community in Early India}. Stanford University Press. California. Eggeling, Julius (Trans). (2010) {\sl Śatapatha Brāhmaṇa (According to the School of Mādhyandina)}. Nabu   Press. USA. 

\bibitem[]{chap3_item40}
Eliade, Mircea. (1959). {\sl Cosmos and History: The Myth of the Eternal Return}. New York. Harper Torchbooks. 

\bibitem[]{chap3_item41}
Everts, W. W. (1908). “Homer and the Higher critics”. {\sl Bibliotheca Sacra}. Vol BSAC 065:259 (July 1908). pp 531 – 556. 

\bibitem[]{chap3_item42}
Frawley, David. (1997). {\sl The Myth of Aryan Invasion of India}.  New Delhi. Voice of India. 

\bibitem[]{chap3_item43}
Frye, Northrop. (2006). {\sl Anatomy of Criticism: Four Essays}. Toronto. University of Toronto Press. 

\bibitem[]{chap3_item44}
{\sl Gautama Dharma-sūtra}. See Pandeya  (2013). 

\bibitem[]{chap3_item45}
Ganeri, Jonardon (Ed). (2017). {\sl The Oxford Handbook of Indian Philosophy}. New York. Oxford University Press. 

\bibitem[]{chap3_item46}
Ghoshal, U.N. (1923). A History of Hindu Political Ideas. Calcutta. Oxford University Press. 

\bibitem[]{chap3_item47}
Giles, Herbert Allen. (1889). {\sl Chuang Tzu, Mystic, Moralist, and Social Reformer}. London. Bernard Quaritch. 

\bibitem[]{chap3_item48}
Gokak, V. K.  (1979). {\sl The Concept of Indian Literature}. New Delhi. Munshiram Manoharlal. 

\bibitem[]{chap3_item49}
Goldman, Robert. (1984). {\sl The Rāmāyaṇa of Vālmīki: an Epic of Ancient India Volume I: Bālakāṇḍa}. New Jersey. Princeton University Press. 

\bibitem[]{chap3_item50}
Goldman, Robert. (2004). “Resisting Rāma: Dharmic debates on Gender and Hierarchy and the Work of the Vālmīki’s {\sl Rāmāyaṇa}”. In Mandakranta Bose (eds) {\sl Revisiting Rāmayaṇa}. New York. Oxford University Press. 

\bibitem[]{chap3_item51}
Goldman, Robert. (2005). “Historicising the Ramakathā: Vālmīki's {\sl Rāmāyaṇa} and its Medieval commentators”. In {\sl India International Centre Quarterly, Vol. 31, No. 4 (Spring)}. pp 83-97

\bibitem[]{chap3_item52}
Goody, Jack (Ed). (1979, 19671). {\sl Succession to high office}. Cambridge. Cambridge University Press. 

\bibitem[]{chap3_item53}
Grafton, Anthony. (1985). “Renaissance Readers and Ancient Texts: Comments on some Commentaries”. In {\sl Renaissance Quarterly}, 38(4). pp 615-649.

\bibitem[]{chap3_item54}
Grafton, Anthony. (2016). {\sl Friedrich August Wolf’s Prolegomena to Homer, 1795}.  New Jersey.  Princeton University Press. 

\bibitem[]{chap3_item55}
Griffith, Ralph; Keith, Arthur. B (Trans). (2017). {\sl The Vedas: The Samhitas of Ṛig, Yajur, Sama and Atharva Vedas}. Createspace Independent Pub. (Self-publishing) (Combined volumes of comprising reprints of translations of these authors done over a century ago; spelling of the title as in the printed version).

\bibitem[]{chap3_item56}
Grote, George. (1850). {\sl History of Greece}.  London. J. Murray. 

\bibitem[]{chap3_item57}
Fohr, H.D., Bethell, C., Moore, P., Schiff. H. (Trans) (2001). {\sl Rene Guenon’s Miscellanea}. New York. Sophia Perennis. 

\bibitem[]{chap3_item58}
Haecker, Theodor. (1934). {\sl Virgil, Father of the West}. London. Sheed and Ward. 

\bibitem[]{chap3_item59}
Halbfass, Wilhelm. (1995). {\sl Philology and Confrontation: Paul Hacker on Traditional and Modern Vedanta}. New York. State University of New York Press. 

\bibitem[]{chap3_item60}
{\sl Harivaṁśa}. See Nagar (2012). 

\bibitem[]{chap3_item61}
Havell, E. B. (1928). {\sl Indian Sculpture and Painting}. London. John Murray. 

\bibitem[]{chap3_item62}
Haug, Martin (Ed) (Trans). (1863). {\sl Aitareya Brahmanam of the Rigveda}. London. Trubner and Co. 

\bibitem[]{chap3_item63}
Heras, H. Rev. (1929). {\sl Beginnings of Vijayanagara History}. Bombay. Indian Historical Research Institute. 

\bibitem[]{chap3_item64}
Hiltebeitel, Alf. (2005). “Not Without Subtales: Telling Laws and Truths in the Sanskrit Epics”. {\sl Journal 
of Indian Philosophy} (2005) 33: pp 455–511

\bibitem[]{chap3_item65}
Hinkle, Beatrice. (1965). {\sl Carl Jung’s The Psychology of the Unconscious}. New York.Dood, Mead and co. 

\bibitem[]{chap3_item66}
Hocart, A.M. (1950). {\sl Caste: A Comparitive Study}. London. Methuen \& Co. Ltd. 

\bibitem[]{chap3_item67}
Hultzsch, E. (Ed). (1982). {\sl Epigraphica Indica Volume 1}. Delhi. Archeological Survey of India. 

\bibitem[]{chap3_item68}
Ingalls, D. H,  Masson, Jeffrey Moussaieff., and Patwardhan. M. V. (1990). {\sl The Dhvanyaloka of 
Anandavardhana with the Locana of Abhinavagupta}. England. Harvard University Press. 

\bibitem[]{chap3_item69}
Jayaswal, K. P. (2006, 19241). {\sl Hindu Polity: A Constitutional History of India in Hindu Times}. Varanasi. Chaukhamba Sanskrit Pratishthan Oriental Publishers \& Distributors. 

\bibitem[]{chap3_item70}
Joglekar, K. M (Ed). (1916). {\sl Raghuvaṁśa} of Kālidāsa (with Mallinātha’s Sañjīvinī). Bombay. Nirnaya Sagar Press.  

\bibitem[]{chap3_item71}
Jowett, Benjamin. (1960). {\sl Plato’s Republic}. New York. Anchor Books. 

\bibitem[]{chap3_item72}
Jowett, Benjamin. (2016). {\sl Plato’s Ion}. Greece. Demosthenes Koptis. 

\bibitem[]{chap3_item73}
Kale, M. R (Trans). (2010). {\sl Abhijñāna Śākuntalam} of Kālidāsa. New Delhi. Motilal Banarasidass. 

\bibitem[]{chap3_item74}
Kane, P.V. (1941). {\sl History of Dharmaśāstra} Pune. Bhandarkar Oriental research institute. 

\bibitem[]{chap3_item75}
Kane, P.V. (1966). “The two epics”. {\sl Annals of the Bhandarkar Oriental Research Institute}, Vol. 47, No. 1/4 (1966). pp. 11-58 

\bibitem[]{chap3_item76}
Kangle, R. P ed. (2014). {\sl Kauṭilya’s Artha-śāstra}. New Delhi. Motilal Banarasidass. 

\bibitem[]{chap3_item77}
Kashyap, R.L (Ed) (Trans). (2017). {\sl Taittirīya Brāhmaṇa}. Bangalore. Sri Aurobindo Kapali Sastry Institute of Vedic Culture. 

\bibitem[]{chap3_item78}
{\sl Kāvyādarśa} of Danḍin. See Belvalkar (1924)  

\bibitem[]{chap3_item79}
{\sl Kavyālamkāra} of Bhāmaha. See Suri (1909).

\bibitem[]{chap3_item80}
{\sl Kāvyaprakāśa} of Mammaṭa. See Venkatanathacarya (1974)

\bibitem[]{chap3_item81}
Ketkar, S. (1987, 19271) (Trans). {\sl Maurice Winternitz’s A History of Indian Literature}. Calcutta. University of Calcutta. 

\bibitem[]{chap3_item82}
Keynes, Georffrey. (1946). {\sl Poetry and Prose of William Blake (Vol 1)}. London. The Nonsuch Press. 

\bibitem[]{chap3_item83}
Krishnamoorthy, K (Ed.) (1977). {\sl Vakrokti-jīvita of Kuntaka}. Dharwad. Karnatak University. 

\bibitem[]{chap3_item84}
Lal. B. B (2002). {\sl The Saraswati Flows on the Continuity of Indian Culture}. New Delhi. Aryan Books  
International. 

\bibitem[]{chap3_item85}
{\sl Mahābhārata} of Vyāsa. See Sukthankar (1942)

\bibitem[]{chap3_item86}
{\sl Madhurā-vijayam} of Gaṅgādevi. See Sastri (1924)

\bibitem[]{chap3_item87}
{\sl Mālavikāgnimitra}. See Parab K P (Ed). (1915).

\bibitem[]{chap3_item88}
Majumdar, R.C. (2010). {\sl The History and Culture of the Indian People: Volume 1. The Vedic Age}. New Delhi. Munshiram Manoharlal. 

\bibitem[]{chap3_item89}
Mallette, K. (2010). {\sl European Modernity and the Arab Mediterranean: Toward a New Philology and a Counter-orientalism}. University of Pennsylvania Press. Pennsylvania.

\bibitem[]{chap3_item90}
{\sl Manusmṛti}. See Shastri, J.L. (1983).

\bibitem[]{chap3_item91}
Martin, R. P. (1993). “Telemach us and the last hero song”. In {\sl Colby Quarterly}, Vol 29, Issue 3. pp 222-240.

\bibitem[]{chap3_item92}
McCrindle, J. W. (2000). {\sl Ancient India: As Described by Megasthenes and Arrian}. New Delhi. Munshiram Manoharlal. 

\bibitem[]{chap3_item93}
McEnerney, John I. (1986). {\sl St. Cyril of Alexandria: Letters 1-50}. Washington D.C.  Washington D.C. The Catholic University of America Press. 

\bibitem[]{chap3_item94}
Michell, G. (1995). {\sl Architecture and Art of Southern India}. Cambridge. Cambridge University Press.

\bibitem[]{chap3_item95}
Moseley, Nicholas. (1925). “Pius Aeneas.” {\sl The Classical Journal (The Classical Association of the Middle West and South)} 20, no. 7 (April 1925): pp.387-400.

\bibitem[]{chap3_item96}
Mudholkar, S.S.Katti (Ed). (1914-1920) {\sl Rāmāyaṇa of Vālmīki (with three commentaries called Tilaka,    Shiromani and Bhooshana) (7 Vols)}. Bombay. Gujarati Printing Press. 

\bibitem[]{chap3_item97}
Muller, F. Max. (1979). {\sl Physical Religion} (First ed 1890). New Delhi. Asian Educational Service. 

\bibitem[]{chap3_item98}
Nagar, Shanti Lal. (2012). {\sl Harivaṁśa}. Delhi. Eastern Book Linkers. 

\bibitem[]{chap3_item99}
Okakura, Kakuzo. (1904). {\sl Ideals of the East}. New York. E. P. Dutton \& co. 

\bibitem[]{chap3_item100}
{\sl Pāda-tāḍitaka of Śyāmilaka}. See Schokker, G.H (1966).

\bibitem[]{chap3_item101}
Pande, Anupa. (1997). {\sl Abhinava-bhāratī of Abhinava Gupta}. Allahabad. Raka Prakashan. 

\bibitem[]{chap3_item102}
Pandeya, Umesha Chandra (Ed). (2013). {\sl Gautama Dharma-sūtras}. Delhi. Chaukambha Sanskrit Sansthan. 

\bibitem[]{chap3_item103}
Parab K P (Ed). (1915). {\sl Kālidāsa’s Mālavikāgnimitra}. Mumbai. Nirnaya Sagar Press. 

\bibitem[]{chap3_item104}
Pathak, Subha. (2006). “Why Do Displaced Kings Become Poets in Sanskrit Epics? Modeling Dharma in the affirmative Rāmāyaṇa and the Interrogative Mahābhārata.” {\sl International Journal of Hindu Studies} Vol. 10, No. 2 (August), pp. 127-149

\bibitem[]{chap3_item105}
Pathaka, Jagannatha (Ed). (1965). {\sl Ānandavardhana’s Dhyanyāloka (with Abhinavagupta’s Locana)}. Varanasi. Chowkhamba Vidya Bhavan. 

\bibitem[]{chap3_item106}
Patil, D.R. (1973). {\sl Cultural History from the Vāyu-Purāna}. Delhi. Motilal Banarasidas. 

\bibitem[]{chap3_item107}
Philo. See Yonge, C. D (1993).

\bibitem[]{chap3_item108}
Pollock, Sheldon. (1993). “Rāmāyaṇa and Political Imagination in India”. {\sl The Journal of Asian Studies} 52, no 2 (May 1993). pp 261-297

\bibitem[]{chap3_item109}
Pollock, Sheldon. (2006). {\sl Language of the Gods in the World of Men}. California. University of California. 

\bibitem[]{chap3_item110}
Pollock, Sheldon. (2007a). {\sl The Rāmāyaṇa of Vālmīki, Volume II – Ayodhyākāṇḍa}. Delhi.  Motilal Banarasidass Publishers. 

\bibitem[]{chap3_item111}
Pollock, Sheldon. (2007b). {\sl The Rāmāyaṇa of Vālmīki, Volume III – Araṇyakāṇḍa}. Delhi. Motilal Banarasidass Publishers. 

\bibitem[]{chap3_item112}
Pollock, Sheldon. (2014). “Philology in three dimensions”. {\sl Postmedieval: a journal of medieval cultural studies}, Vol. 5, 4. pp 398–413

\bibitem[]{chap3_item113}
Prasad, Pushpa. (1990). {\sl Sanskrit Inscriptions of Delhi Sultanate, 1191-1526}. New York. Oxford University Press. 

\bibitem[]{chap3_item114}
{\sl Pṛthvīrāja-vijaya} of Jayāṅka. See Belvalkar, S. K (1914) 

\bibitem[]{chap3_item115}
{\sl Raghuvaṁśa} of Kālidāsa. See Joglekar (1916)

\bibitem[]{chap3_item116}
{\sl Rāmacaritam} of Sandhyākaranandin. See Chowdhury, A.M (2012). 

\bibitem[]{chap3_item117}
Ramanujan, A.K. (1989). “Is there an Indian Way of Thinking? An Informal Essay”. In {\sl Contributions to Indian Sociology} 1989;23;41 DOI: 10.117/006996689023001004. pp 41-58.

\bibitem[]{chap3_item118}
{\sl Ramāyaṇa} of Vālmīki. See Mudholkar and S.S.Katti (1914-1920).

\bibitem[]{chap3_item119}
{\sl Ramāyaṇa} of Vālmīki. (2006). Gorakhpur. Gita Press. (No editor mentioned)

\bibitem[]{chap3_item120}
{\sl Rāmāyaṇa-mañjari} of Kṣemendra. See, Sastri, Bhavadatta  (1903)

\bibitem[]{chap3_item121}
Raychaudari, Hemchandra. (1953). {\sl Political History of Ancient India}. Calcutta.University of Calucutta.

\bibitem[]{chap3_item122}
Rice, Stanley. (1924). {\sl Tales from the Mahābhārata}. London. Selwyn \& Blount. 

\bibitem[]{chap3_item123}
Sachs, Joe. (1999). {\sl Aristotle’s Metaphysics}. New Mexico. Green Lion Press. 

\bibitem[]{chap3_item124}
Said, Edward W. (2004).  {\sl Humanism and Democratic Criticism}. New York. Columbia University Press. 

\bibitem[]{chap3_item125}
Sarkar, B. K. (1913). {\sl Śukra-nīti-sāra}. Allahabad. Sudhindranatha Vasu. 

\bibitem[]{chap3_item126}
Sastri, Bhavadatta (Ed). (1903). {\sl Rāmāyaṇa-mañjari} of Kṣemendra. Mumbai. Tukaram Javaji. 

\bibitem[]{chap3_item127}
Sastri, Ganapati T. (1909). {\sl The Vyakti-viveka of Rājānaka Mahimabhaṭṭa and its commentary of Rājānaka Ruyyaka}. Tranvancore. Trivandrum Sanskrit Series. 

\bibitem[]{chap3_item128}
Sastri, Harihara (Ed). (1924). {\sl Madhurāvijayam} of Gaṅgādevi. Trivandrum. Sridhara Power Press. 

\bibitem[]{chap3_item129}
{\sl Śatapatha-brāhmaṇa}. See Caland, W., Vira, Raghu (1983, 19261)

\bibitem[]{chap3_item130}
Schleiermacher, Friedrich, Duke, J., Forstman, J., \& Kimmerle, H. (1997). {\sl Hermeneutics: The handwritten manuscripts}. Atlanta. Scholars Press.  

\bibitem[]{chap3_item131}
Schokker, G.H. (1966). {\sl The Pādatāḍitaka of Śyāmilaka (Part 1)}. Netherlands. Hague Publication. 

\bibitem[]{chap3_item132}
Schokker, G.H. (1976). {\sl The Pādatāḍitaka of Śyāmilaka (Part 2)}. Netherlands.  Dordrecht. 

\bibitem[]{chap3_item133}
Schuon, Frithjof. (2007). {\sl Art from the Sacred to the Profane: East and West}. Indiana. World Wisdom Inc. 

\bibitem[]{chap3_item134}
Scodel, Ruth. (2009). {\sl Listening to Homer: Tradition, Narrative, and Audience}. Ann Arbor. University of Michigan Press. 

\bibitem[]{chap3_item135}
Sharma, Ravindra. (1988). {\sl Kingship in India from Vedic age to Gupta age}. New Delhi. Atlantic Publishers \& Distributers. 

\bibitem[]{chap3_item136}
Sharma, A. K. (1974). “Evidence of Horse from the Harappan Settlement at Surakoṭaḍā”. {\sl Purātattva No}. 
7, pp 75-76

\bibitem[]{chap3_item137}
Shastri, J.L. (1983). {\sl Manusmriti}. Delhi. Motilal Banarasidass. 

\bibitem[]{chap3_item138}
{\sl Śukra-nīti-sāra}. See Sarkar, B. K. (1913). 

\bibitem[]{chap3_item139}
Sukthankar, V.S. et al. (Ed.) (1933-1966). {\sl Mahābhārata} (19 vols). Poona. Bhandarkar Oriental Research Institute. 

\bibitem[]{chap3_item140}
Sukthankar, V.S. (1957). {\sl On the Meaning of the Mahābhārata}. Bombay. The Asiatic society of Bombay. 

\bibitem[]{chap3_item141}
Suri Srikrishna (Ed) (1909). {\sl Kavyālaṅkāra of Bhāmaha}. Srirangam. Vani Vilas Press. 

\bibitem[]{chap3_item142}
Swahananda, Swami (Trans). (2016). {\sl Chandogya and Brihadaranyaka Upanishads with Short Commentaries}. Createspace Independent Publishing Platform. (Self-Publishing). 

\bibitem[]{chap3_item143}
Swenson, David. F. (1936). {\sl Philosophical Fragments by Soren Keirkegaard}. New Jersey. Princeton University Press. 

\bibitem[]{chap3_item144}
{\sl Taittirīya-brāhmaṇa}. See Kashyap, R.L (Trans). (2017)

\bibitem[]{chap3_item145}
Talageri, Srikanth. (2000). {\sl Rigveda: A Historical Analysis}. New Delhi. Aditya Prakashan. 

\bibitem[]{chap3_item146}
Talbot, Cynthia. (1995). “Inscribing the other, inscribing the self: Hindu-Muslim identities in pre-colonial India.” {\sl Comparative Studies in Society and History} Vol. 37, No. 4 (Oct., 1995). pp. 692-722

\bibitem[]{chap3_item147}
The {\sl Vedas}. See Griffith, Ralph; Keith, Arthur. B (2017).

\bibitem[]{chap3_item148}
{\sl Vakrokti-jīvita} of Kuntaka. See Krishnamoorthy (1977)

\bibitem[]{chap3_item149}
{\sl Vaṣiṣṭha Dharma-sūtra}. See Buhler (1882)

\bibitem[]{chap3_item150}
Venkatanathacarya, N S (Ed.) (1974). {\sl Mammaṭa’s Kāvyaprakāśa}. Mysore. Mysore Oriental Research Institute

\bibitem[]{chap3_item151}
Verghese, Anila. (1995). {\sl Religious Traditions at Vijayanagara: as revealed through its monuments}. New  Delhi. Manohar Publishers. 

\bibitem[]{chap3_item152}
Wildman, S., Burne-Jones, E. C., Christian, J., Crawford, A., Des, C. L. (1998). {\sl Edward Burne-Jones, Victorian artist-dreamer}. New York. Metropolitan Museum of Art. 

\bibitem[]{chap3_item153}
Yonge, C. D. (1993). {\sl The works of Philo. Massachussets}. Hendrickson Publishers.
\end{thebibliography}



\theendnotes

