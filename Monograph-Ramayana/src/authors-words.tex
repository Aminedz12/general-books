\makeatletter
\def\@makeschapterhead#1{%
  \vspace*{50\p@}%
  {\parindent \z@ \raggedleft
    \normalfont
    \interlinepenalty\@M
    \LARGE \bfseries  #1\par\nobreak
    \vskip 30\p@
  }}
\makeatother

\chapter*{Author's Words}\label{authors-words}

\lhead[\small\thepage\quad Reclaiming the {\sl Rāmāyaṇa}]{}
\rhead[]{Author's Words\quad\small\thepage}

In the summer of 2014, a serious pursuit of {\sl vyākaraṇa} studies had taken me to a {\sl gurukula} in Chattisgadh, and it was here that I first heard of Prof. Sheldon Pollock. A student of his had just returned (after her Post Graduation in Sanskrit) from Columbia University to Chattisgadh to “really study Sanskrit”, and she had professed effusive regard for her mentor, ‘Shelly’. I had then --- out of curiosity --- skimmed through a few of his writings but dismissed them out of some vague and indistinct notion of its unreliability. So it was during the months leading up-to the first Swadeshi Indology Conference (which was held in July 2016) --- two years later --- that I first read with seriousness some writings of Pollock --- but the same unsettling, indistinct notion of its incorrectness bothered me. It was, finally, at the Swadeshi Indology Conference that that vague notion was made concrete and given distinct, if geometric, proportions; and this book is its result.  

I chose to problematize Pollock’s reading of the {\sl Rāmāyaṇa} partly because of my own long and loving acquaintance with the text, and partly --- but more importantly --- because of the vital space it occupies in the consciousness of the Indian mind. For Indians, the {\sl Rāmāyaṇa} is their sacred {\sl itihāsa}, the fabric of their whole life, “as much a piece of unfinished business as it was for the learned commentators on the poem of the thirteenth through the eighteenth centuries… as vital a work as it had been for its original audiences who preceded them by millennia” (Goldman 2005: 83). Indians still celebrate the birth of Rāma (the pan-Indian Rāmanavami festival) with fasts, prayers and recitations of the Rāma-kathā; Dīpāvali is still celebrated to commemorate Rāma’s return from exile to Ayodhyā; the nine nights of Navarātri see a re-enactment of the {\sl Rāmāyaṇa} --- and on the tenth night (Duśśera), an effigy of Rāvaṇa is burnt symbolically to remember Rāma’s victory. It is also the practice of many to do the Rāmāyaṇa-pārāyaṇa those ten days. Furthermore, in Coomaraswamy’s words, “to be such a man as Rāma, such a wife as Sītā, rather than to express ‘oneself’, is the aim [of every Indian]” (Coomaraswamy 1918: 87).

In his ‘{\sl The Myth of the Eternal Return}’, Mircea Eliade demonstrated that in archaic societies, ‘reality’ was an “unending recurrence of archetypal paradigms played out in the cosmos”, and that “archaic ontology” “tolerates ‘history’ with difficulty and attempts to periodically abolish it” through a “reduction of events to categories and of individuals to archetypes” (Eliade 1959: 36). Much of India still retains of this archetypal-mythical world-view, and in Eliade’s words,  

\begin{myquote}
“It is not our part to decide whether such motives were puerile or not, or whether such a refusal of history always proved efficacious. In our opinion, only one fact counts: by virtue of this view, tens of millions of men were able, for century after century, to endure great historical pressures without despairing, without… falling into that spiritual aridity that always brings with it a relativistic or nihilistic view of history.”

\hfill Eliade (1959:152)
\end{myquote}

Unfortunately, Pollock’s reading of the {\sl Rāmāyaṇa} does much harm to the traditional hermeneutics of the text, and this monograph is an attempt to draw attention to it. Of course, it is neither an ideal exposition of the traditional explanative nor an ideal refutation of Pollock’s --- yet, it is an attempt to begin thinking through a number of concerns and questions connected with “hermeneutics”; it is an attempt to draw attention to the challenges --- internal and external, intellectual and institutional --- facing India studies. If this modest attempt can function as a starter for serious traditionalists to offer more thorough rebuttals, I will feel well-rewarded. 

Italicized words in quotations are emphasized in the original unless otherwise noted. Non-English words are italicized and marked with diacritics. While quoting other sources {\sl verbatim}, however, the diacritics, italicization, etc, are given as they are in the original. 

%\hfill {\bf Manjushree Hegde}

\section*{Acknowledgements}\label{ack1}

%\lhead[\small\thepage\quad Reclaiming Rāmāyaṇa : Disentangling the Discourses]{}
%\rhead[]{Acknowledgements\quad\small\thepage}

To Sri Rajiv Malhotra I owe an immense debt of gratitude. It was he who first drew my attention --- through his books --- to the existence of the Western-academia-challenge that looms before us. It was at his conference that I learnt of the severity of the problem --- and the urgency of a rebuttal. It was he who first suggested that I write this monograph --- and graciously made place in his institution for me. It was he who channeled the financial support to complete it --- without him, and his institution --- Infinity Foundation India --- this monograph would truly be inconceivable.   

I am deeply indebted to Prof. K. S. Kannan who was equally --- if not more --- instrumental in the undertaking of this monograph. But for him, I would not have met Sri Rajiv Malhotra. But for him, I would not have been chosen to undertake this work. But for him, I would not have had the courage or confidence to take it up --- or do justice to it. His support has been unflagging --- I have depended on his learning in moments of confusion, on his ingenuity to counter that of Prof. Sheldon Pollock. A constant source of encouragement, he has, with utmost patience, discussed with me my ideas, offered constructive criticism and made countless suggestions for improvements throughout the months that I have worked on this monograph.  I gratefully acknowledge Ms. H. R. Meera’s support throughout this project. I have had endlessly enriching discussions with her; many --- if not most --- ideas have stemmed from them.  She has also expertly dealt with the final re-touching of the monograph, laboring hard to ensure editorial consistency --- I am very grateful for it. 

Each of the following friends\,/\,IFI team members have contributed in various ways, and I regret I do not have the space to explain how valuably: Śatāvadhānī Dr. R. Ganesh, Alok Mishra, Shashi Kiran B.N., Shalini Puthiyedam, Vijaya Vishwanathan, and Sudarshan T. N. A generous funding from Vellayan Chettiar Trust has facilitated this work --- I am deeply grateful for it. 


\bigskip
\hfill {\bf Manjushree Hegde}
