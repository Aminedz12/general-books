\thispagestyle{plain}

\lhead[\small\thepage\quad Reclaiming~ {\sl Rāmāyaṇa}]{}
\rhead[]{Bibliography\quad\small\thepage}

\begin{thebibliography}{99}
\addtocontents{toc}{\protect\contentsline{chapter}{\protect Bibliography}{\thepage}}
\label{bibliography}
\itemsep=2pt
\bibitem[]{chap3_item1}
{Aurobindo, S.} (1995, 1922$^{1}$). {\sl Essays on the Gita}. New Delhi: Lotus Press. 

\bibitem[]{chap3_item2}
{\sl\bfseries Abhijñāna-śākuntala} of Kālidāsa. See Kale (2010). 

\bibitem[]{chap3_item3}
{\sl\bfseries Abhinava-bhāratī} of Abhinava Gupta. See Pande (1997).

\bibitem[]{chap3_item4}
{\sl\bfseries Aitareya-brāhmaṇa}. See Haug (1863).

\bibitem[]{chap3_item5}
{\sl\bfseries Āpastamba Dharma-sūtra}. See Pandeya (2006).

\bibitem[]{chap3_item6}
{\sl Metaphysics} of Aristotle. See Sachs (1999).

\bibitem[]{chap3_item7}
{\sl\bfseries Artha-śāstra} of Kauṭilya. See Kangle (2014).  

\bibitem[]{chap3_item8}
{\sl\bfseries Atharvaveda}. See Griffith and Keith (2017).

\bibitem[]{chap3_item9}
Bailey, Cyril.\ (1921). {\sl Lucretius’ On the Nature of Things}. Oxford: Clarendon Press. 

\bibitem[]{chap3_item10}
Baker, Hugh. D. R. (1979). {\sl Chinese Family and Kinship}. New York: Columbia University Press. 

\bibitem[]{chap3_item11}
Bakker, Hans.\ (1986). {\sl Ayodhyā}. Netherlands: Groningen. 

\bibitem[]{chap3_item12}
---\kern3pt(1987). “Reflections on the evolution of Rāma devotion in the light of textual and archaeological evidence”, {\sl Wiener Zeitschrift für die Kunde Südasiens und Archiv für indische Philosophie}, Band XXXI (1987), pp.~9--42. 

\bibitem[]{chap3_item13}
{\sl\bfseries Bṛhadāraṇyakopaniṣad}. See Swahananda (2016).

\bibitem[]{chap3_item13_1}
Belvalkar, S. K. (Ed.) (1914). {\sl Pṛthvīrājavijaya.} Calcutta: Asiatic Society.

\bibitem[]{chap3_item14}
---\kern3pt(Ed.) (1924). {\sl Kāvyādarśa of Danḍin}.  Poona: Bhandarkar Oriental Research Institute. 

\bibitem[]{chap3_item15}
---\kern3pt(1945). {\sl Bhagavad-gītā}. Poona: Bhandarkar Oriental Research Institute. 

\bibitem[]{chap3_item16}
{\sl\bfseries Bhagavad-gītā}. See Belvalkar (1945).

\bibitem[]{chap3_item17}
{\sl\bfseries Bhāgavata-purāṇa}. See Tapasyananda (2003). 

\bibitem[]{chap3_item17_1}
Bhandarkar, D. R, Temple, Richard Carnac (Ed.) (1912). {\sl The Indian Antiquary}, Vol 41. Bombay: British India Press. 

\bibitem[]{chap3_item18}
Birdwood, George. (1915). {\sl Sva}. London: London Philip Lee Warner. 

\bibitem[]{chap3_item18_1}
Bose, Mandakranta (Ed.) (2004). {\sl Revisiting Rāmayaṇa}. New York: Oxford University Press.

\bibitem[]{chap3_item19}
Brockington, J. L. (1998). {\sl The Sanskrit Epics}. Netherlands: Brill.

\bibitem[]{chap3_item20}
Bühler, George (Trans.) (1882). {\sl Vaśiṣṭha Dharma-sūtra}. Bombay: Bombay Sanskrit and Prakrit Series. 

\bibitem[]{chap3_item21}
Caland, W., Vira, Raghu. (1983, 19261). {\sl The Śatapatha Brāhmaṇa in the Kāṇvīya Recesion}. New Delhi: Motilal Banarsidass. 

\bibitem[]{chap3_item22}
Calvert, George, H (Trans.) (1846). {\sl Correspondence Between Schiller and Goethe, from 1794 to 1805 (Vol~2)}. New York \& London: Wiley and Putnam.  

\bibitem[]{chap3_item22_1}
{\sl\bfseries Campū-rāmāyaṇa}. See Pansikar (1982).

\bibitem[]{chap3_item23}
Cantor, Paul. (2007). “The Politics of the Epic: Wordsworth, Byron, and the Romantic Redefinition of Heroism”. {\sl The Review of Politics}, Vol.~69, No.~3, Special Issue on Politics and Literature (Summer, 2007). pp.~375--401.

\bibitem[]{chap3_item24}
Chakravarti, Dilip. K. (1990) {\sl The External Trade of the Indus Civilization}. New Delhi: Munshiram Manoharlal. 

\bibitem[]{chap3_item25}
{\sl\bfseries Chāndogyopaniṣad}. See Swahananda (2016).

\bibitem[]{chap3_item26}
Chattopadhyaya, Brajdulal. (1998). {\sl Representing the Other?} New Delhi: Manohar Publishers. 

\bibitem[]{chap3_item27}
Chen, Ivan.\ (1908). {\sl Book of filial duty}. London: Library of Alexandria. 

\bibitem[]{chap3_item28}
Clift, P.D, Carter A, Giosan L, Duncan J, Duller GAT, Mcklin MG, Alizai A, Tabrez AR, Danish M, Vanlaningham S, Fuller DO. (2012). “U-Pb zircon dating evidence for a Pleistocene Sarasvati River and capture of the Yamuna River”. In Geology 40(2): 211--214.

\bibitem[]{chap3_item29}
Chowdhury, A.M. (2012). {\sl The Ramacharitam}. Bangladesh: Asiatic Society of Bangladesh. 

\bibitem[]{chap3_item30}
Concord, F. M. (1952). {\sl Principium Sapientiae: The origins of Greek Philosophical thought}. Cambridge: Cambridge University Press. 

\bibitem[]{chap3_item31}
Coomaraswamy, A. K. (1918). {\sl The Dance of Shiva}. New York: The Sunwise Turn Inc. 

\bibitem[]{chap3_item32}
---\kern3pt(1934). {\sl The Transformation of Nature in Art}. New York: Dover Publications.

\bibitem[]{chap3_item33}
---\kern3pt(1940). {\sl East and West and Other Essays}. Colombo: Ola Books Ltd. 

\bibitem[]{chap3_item34}
---\kern3pt(1946). {\sl Religious Basis of the Forms of Indian Society}. Montana: Literary Licensing, LLC.

\bibitem[]{chap3_item35}
---\kern3pt(1977). “The Bugbear of Democracy, Freedom, and Equality”. {\sl Studies in Comparative Religion}, Vol.~11, No.~3. (Summer, 1977). 

\bibitem[]{chap3_item35_1}
Coomaraswamy, R (Ed.) (2004). {\sl The Essential Ananda K Coomaraswamy}. Indiana: World Wisdom Inc.

\bibitem[]{chap3_item36}
Danino, Michel. (2010). {\sl The Lost River: On the Trail of the Sarasvati}. Haryana: Penguin India.  

\bibitem[]{chap3_item37}
Davids, Rhys. T. W. (1899). {\sl Dialogues of the Buddha}. New York: Oxford University Press. 

\bibitem[]{chap3_item38}
{\sl\bfseries Dhvanyāloka} of Anandavardhana. See Pathak (1965).

\bibitem[]{chap3_item39}
Drekmeier, Charles.\ (1962). {\sl Kingship and Community in Early India}. Stanford University Press. California. Eggeling, Julius (Trans.) (2010). {\sl Śatapatha Brāhmaṇa (According to the School of Mādhyandina)}. Nabu Press. USA. 

\bibitem[]{chap3_item40}
Eliade, Mircea.\ (1959). {\sl Cosmos and History: The Myth of the Eternal Return}. New York: Harper Torchbooks. 

\bibitem[]{chap3_item41}
Everts, W. W. (1908). “Homer and the Higher critics”. {\sl Bibliotheca Sacra}. Vol BSAC 065:259 (July 1908). pp~531--556. 

\bibitem[]{chap3_item42}
Frawley, David. (1997). {\sl The Myth of Aryan Invasion of India}.  New Delhi: Voice of India. 

\bibitem[]{chap3_item43}
Frye, Northrop.\ (2006). {\sl Anatomy of Criticism: Four Essays}. Toronto: University of Toronto Press. 

\bibitem[]{chap3_item44}
{\sl\bfseries Gautama-dharma-sūtra}. See Pandeya (2013). 

\bibitem[]{chap3_item45}
Ganeri, Jonardon. (Ed.) (2017). {\sl The Oxford Handbook of Indian Philosophy}. New York: Oxford University Press. 

\bibitem[]{chap3_item46}
Ghoshal, U.N. (1923). A History of Hindu Political Ideas. Calcutta: Oxford University Press. 

\bibitem[]{chap3_item47}
Giles, Herbert Allen.\ (1889). {\sl Chuang Tzu, Mystic, Moralist, and Social Reformer}. London: Bernard Quaritch. 

\bibitem[]{chap3_item48}
Gokak, V. K.  (1979). {\sl The Concept of Indian Literature}. New Delhi: Munshiram Manoharlal. 

\bibitem[]{chap3_item49}
Goldman, Robert.\ (1984). {\sl The Rāmāyaṇa of Vālmīki: an Epic of Ancient India Volume I: Bālakāṇḍa}. New Jersey: Princeton University Press. 

\bibitem[]{chap3_item50}
---\kern3pt(2004). “Resisting Rāma: Dharmic debates on Gender and Hierarchy and the Work of the Vālmīki’s {\sl Rāmāyaṇa}”. In Bose (2004). pp~19--46. 

\bibitem[]{chap3_item51}
---\kern3pt(2005). “Historicising the Ramakathā: Vālmīki's {\sl Rāmāyaṇa} and its Medieval commentators”. In {\sl India International Centre Quarterly, Vol.~31, No.~4 (Spring)}. pp~83--97.

\bibitem[]{chap3_item52}
Goody, Jack (Ed.) (1979, 1967$^{1}$). {\sl Succession to high office}. Cambridge: Cambridge University Press. 

\bibitem[]{chap3_item53}
Grafton, Anthony.\ (1985). “Renaissance Readers and Ancient Texts: Comments on some Commentaries”. In {\sl Renaissance Quarterly}, 38(4). pp~615--649.

\bibitem[]{chap3_item54}
---\kern3pt(2016). {\sl Friedrich August Wolf’s Prolegomena to Homer, 1795}.  New Jersey: Princeton University Press. 

\bibitem[]{chap3_item55}
Griffith, Ralph and Keith, Arthur. B (Trans.) (2017). {\sl The Vedas: The Samhitas of Rig, Yajur, Sama and Atharva Vedas}. Createspace Independent Pub. (Self-publishing) (Combined volumes of comprising reprints of translations of these authors done over a century ago; spelling of the title as in the printed version).

\bibitem[]{chap3_item56}
Grote, George.\ (1850). {\sl History of Greece}.  London: J. Murray. 

\bibitem[]{chap3_item57}
Fohr, H.D., Bethell, C., Moore, P., Schiff. H. (Trans.) (2001). {\sl Rene Guenon’s Miscellanea}. New York: Sophia Perennis. 

\bibitem[]{chap3_item58}
Haecker, Theodor. (1934). {\sl Virgil, Father of the West}. London: Sheed and Ward. 

\bibitem[]{chap3_item59}
Halbfass, Wilhelm. (1995). {\sl Philology and Confrontation: Paul Hacker on Traditional and Modern Vedanta}. New York: State University of New York Press. 

\bibitem[]{chap3_item60}
{\sl\bfseries Harivaṁśa}. See Nagar (2012). 

\bibitem[]{chap3_item61}
Havell, E. B. (1928). {\sl Indian Sculpture and Painting}. London: John Murray. 

\bibitem[]{chap3_item62}
Haug, Martin. (Ed.) (Trans.) (1863). {\sl Aitareya Brahmanam of the Rigveda}. London: Trubner and Co. 

\bibitem[]{chap3_item63}
Heras, H. Rev.\ (1929). {\sl Beginnings of Vijayanagara History}. Bombay: Indian Historical Research Institute. 

\bibitem[]{chap3_item64}
Hiltebeitel, Alf. (2005). “Not Without Subtales: Telling Laws and Truths in the Sanskrit Epics”. {\sl Journal 
of Indian Philosophy} (2005) 33: pp~455--511.

\bibitem[]{chap3_item65}
Hinkle, Beatrice. (1965). {\sl Carl Jung’s The Psychology of the Unconscious}. New York: Dood, Mead and co. 

\bibitem[]{chap3_item66}
Hocart, A.M. (1950). {\sl Caste: A Comparitive Study}. London: Methuen \& Co. Ltd. 

\bibitem[]{chap3_item67}
Hultzsch, E. (Ed.) (1892). {\sl Epigraphica Indica Volume 1}. Delhi: Archeological Survey of India. 

\bibitem[]{chap3_item68}
Ingalls, D. H,  Masson, Jeffrey Moussaieff., and Patwardhan. M. V. (Ed.) (Trans.) (1990). {\sl The Dhvanyaloka of 
Anandavardhana with the Locana of Abhinavagupta}. England: Harvard University Press. 

\bibitem[]{chap3_item69}
Jayaswal, K. P. (2006, 1924$^{1}$). {\sl Hindu Polity: A Constitutional History of India in Hindu Times}. Varanasi: Chaukhamba Sanskrit Pratishthan Oriental Publishers \& Distributors. 

\bibitem[]{chap3_item70}
Joglekar, K. M (Ed.) (1916). {\sl Raghuvaṁśa} of Kālidāsa (with Mallinātha’s {\sl Sañjīvinī}). Bombay: Nirnaya Sagar Press.  

\bibitem[]{chap3_item70_1}
Johnson, W. J. (2005). {\sl Mahābhārata Book Three: The Forest: Volume Four}. New York: New York University Press. 

\bibitem[]{chap3_item71}
Jowett, Benjamin. (1960). {\sl Plato’s Republic}. New York: Anchor Books. 

\bibitem[]{chap3_item72}
---\kern3pt(2016). {\sl Plato’s Ion}. Greece: Demosthenes Koptis. 

\bibitem[]{chap3_item73}
Kale, M. R (Trans.) (2010). {\sl Abhijñāna Śākuntalam} of Kālidāsa. New Delhi: Motilal Banarsidass. 

\bibitem[]{chap3_item74}
Kane, P. V. (1941). {\sl History of Dharmaśāstra}. Vol. 2, Part 1. Pune: Bhandarkar Oriental Research Institute. 

\bibitem[]{chap3_item75}
---\kern3pt(1966). “The two epics”. {\sl Annals of the Bhandarkar Oriental Research Institute}, Vol.~47, No.~1/4 (1966). pp.~11--58 

\bibitem[]{chap3_item76}
Kangle, R P (Ed.) (2014). {\sl Kauṭilya’s Artha-śāstra}. New Delhi: Motilal Banarsidass. 

\bibitem[]{chap3_item77}
Kashyap, R L (Ed.) (Trans.) (2017). {\sl Taittirīya Brāhmaṇa}. Bangalore: Sri Aurobindo Kapali Sastry Institute of Vedic Culture. 

\bibitem[]{chap3_item78}
{\sl\bfseries Kāvyādarśa} of Danḍin. See Belvalkar (1924).  

\bibitem[]{chap3_item79}
{\sl\bfseries Kavyālaṅkāra} of Bhāmaha. See Suri (1909).

\bibitem[]{chap3_item80}
{\sl\bfseries Kāvyaprakāśa} of Mammaṭa. See Venkatanathacarya (1974).

\bibitem[]{chap3_item81}
Ketkar, S. (Trans.) (1987, 1927$^{1}$).  {\sl Maurice Winternitz’s A History of Indian Literature}. Calcutta: University of Calcutta. 

\bibitem[]{chap3_item82}
Keynes, Georffrey. (1946). {\sl Poetry and Prose of William Blake (Vol 1)}. London: The Nonsuch Press. 

\bibitem[]{chap3_item83}
Krishnamoorthy, K (Ed.) (1977). {\sl Vakrokti-jīvita of Kuntaka}. Dharwad: Karnatak University. 

\bibitem[]{chap3_item84}
Lal. B. B (2002). {\sl The Saraswati Flows on the Continuity of Indian Culture}. New Delhi: Aryan Books International. 

\bibitem[]{chap3_item84_1}
Limaye, Acharya V. P. and Vadekar, R. D. (Ed.) (1958). {\sl Eighteen Principal Upanishads}, The. Vol. 1. Poona: Vaidika Samodhana Mandala.

\bibitem[]{chap3_item84_2}
Macdonell, A. A. and Keith, A. B. (1912). {\sl Vedic Index of Names and Subjects}. Vol. 2. London: John Murray.

\bibitem[]{chap3_item85}
{\sl\bfseries Madhurā-vijaya} of Gaṅgādevi. See Sastri (1924).

\bibitem[]{chap3_item86}
{\sl\bfseries Mahābhārata} of Vyāsa. See Sukthankar (1942).

\bibitem[]{chap3_item87}
Majumdar, R. C. (2010, 1951$^{1}$). {\sl The History and Culture of the Indian People: Volume~1. The Vedic Age}. New Delhi: Munshiram Manoharlal. 

\bibitem[]{chap3_item88}
{\sl\bfseries Mālavikāgnimitra} of Kālidāsa. See Parab (1915).

\bibitem[]{chap3_item89}
Mallette, K. (2010). {\sl European Modernity and the Arab Mediterranean: Toward a New Philology and a Counter-orientalism}. University of Pennsylvania Press. Pennsylvania.

\bibitem[]{chap3_item90}
{\sl\bfseries Manusmṛti}. See Shastri (1983).

\bibitem[]{chap3_item91}
Martin, R. P. (1993). “Telemach us and the last hero song”. In {\sl Colby Quarterly}, Vol~29, Issue~3. pp~222--240.

\bibitem[]{chap3_item92}
McCrindle, J. W. (2000). {\sl Ancient India: As Described by Megasthenes and Arrian}. New Delhi: Munshiram Manoharlal. 

\bibitem[]{chap3_item93}
McEnerney, John I. (1986). {\sl St. Cyril of Alexandria: Letters 1-50}. Washington D.C: The Catholic University of America Press. 

\bibitem[]{chap3_item94}
Michell, G. (1995). {\sl Architecture and Art of Southern India}. Cambridge: Cambridge University Press.

\bibitem[]{chap3_item95}
Moseley, Nicholas. (1925). “Pius Aeneas.” {\sl The Classical Journal (The Classical Association of the Middle West and South)} 20, no.~7 (April 1925): pp.~387--400.

\bibitem[]{chap3_item96}
Mudholkar, S. S. Katti (Ed.) (1914--1920). {\sl Rāmāyaṇa of Vālmīki (with three commentaries called Tilaka,    Shiromani and Bhooshana) (7 Vols)}. Bombay: Gujarati Printing Press. 

\bibitem[]{chap3_item96_1}
{\sl\bfseries Muṇḍakopaniṣad}. See Limaye and Vadekar (1958).

\bibitem[]{chap3_item97}
Muller, F. Max.\ (1979). {\sl Physical Religion} (First ed 1890). New Delhi: Asian Educational Service. 

\bibitem[]{chap3_item98}
Nagar, Shanti Lal.\ (2012). {\sl Harivaṁśa}. Delhi: Eastern Book Linkers. 

\bibitem[]{chap3_item98_1}
{\sl\bfseries Nirukta}. See Sarup (1967).

\bibitem[]{chap3_item99}
Okakura, Kakuzo. (1904). {\sl Ideals of the East}. New York: E. P. Dutton \& co. 

\bibitem[]{chap3_item100}
{\sl\bfseries Pāda-tāḍitaka} of Śyāmilaka. See Schokker (1966) and (1976).

\bibitem[]{chap3_item101}
Pande, Anupa.\ (1997). {\sl Abhinava-bhāratī of Abhinava Gupta}. Allahabad: Raka Prakashan. 

\bibitem[]{chap3_item102}
Pandeya, Umesha Chandra (Ed.) (2013). {\sl Gautama Dharma-sūtras}. Delhi: Chaukambha Sanskrit Sansthan. 

\bibitem[]{chap3_item102_1}
Pansikar, W. L. S. (Ed.) (1982, 1917$^{1}$). {\sl Campū Rāmāyaṇa of Bhoja}. Varanasi: Chaukhambha Sanskrit Sansthan.

\bibitem[]{chap3_item103}
Parab, K. P. (Ed.) (1915). {\sl Kālidāsa’s Mālavikāgnimitra}. Mumbai: Nirnaya Sagar Press. 

\bibitem[]{chap3_item104}
Pathak, Subha.\ (2006). “Why Do Displaced Kings Become Poets in Sanskrit Epics? Modeling Dharma in the affirmative Rāmāyaṇa and the Interrogative Mahābhārata.” {\sl International Journal of Hindu Studies} Vol.~10, No.~2 (August), pp.~127--149

\bibitem[]{chap3_item105}
Pathak, Jagannatha (Ed.) (1965). {\sl Ānandavardhana’s Dhyanyāloka (with Abhinavagupta’s Locana)}. Varanasi: Chowkhamba Vidya Bhavan. 

\bibitem[]{chap3_item106}
Patil, Devendrakumar Rajaram (1973). {\sl Cultural History from the Vāyu-Purāna}. Delhi: Motilal Banarasidas. 

\bibitem[]{chap3_item107}
Philo. See Yonge (1993).

\bibitem[]{chap3_item108}
Pollock, Sheldon. (1993). “Rāmāyaṇa and Political Imagination in India”. {\sl The Journal of Asian Studies} 52, no~2 (May 1993). pp~261--297

\bibitem[]{chap3_item109}
---\kern3pt(2006). {\sl Language of the Gods in the World of Men}. California: University of California. 

\bibitem[]{chap3_item110}
---\kern3pt(2007a). {\sl The Rāmāyaṇa of Vālmīki, Volume II – Ayodhyākāṇḍa}. Delhi: Motilal Banarsidass Publishers. 

\bibitem[]{chap3_item111}
---\kern3pt(2007b). {\sl The Rāmāyaṇa of Vālmīki, Volume III – Araṇyakāṇḍa}. Delhi: Motilal Banarsidass Publishers. 

\bibitem[]{chap3_item112}
---\kern3pt(2014). “Philology in three dimensions”. {\sl Postmedieval: a Journal of Medieval Cultural Studies}, Vol.~5, 4. pp~398--413

\bibitem[]{chap3_item113}
Prasad, Pushpa. (1990). {\sl Sanskrit Inscriptions of Delhi Sultanate, 1191-1526}. New York: Oxford University Press. 

\bibitem[]{chap3_item114}
{\sl\bfseries Pṛthvīrāja-vijaya} of Jayāṅka. See Belvalkar (1914).

\bibitem[]{chap3_item115}
{\sl\bfseries Raghuvaṁśa} of Kālidāsa. See Joglekar (1916).

\bibitem[]{chap3_item116}
{\sl\bfseries Rāmacarita} of Sandhyākaranandin. See Chowdhury (2012).

\bibitem[]{chap3_item117}
Ramanujan, A.K. (1989). “Is there an Indian Way of Thinking? An Informal Essay”. In {\sl Contributions to Indian Sociology} 1989;23;41 DOI: 10.117/006996689023001004. pp~41--58.

\bibitem[]{chap3_item118}
{\sl\bfseries Rāmāyaṇa} of Vālmīki. (2006). Gorakhpur: Gita Press.

Also see Mudholkar (1914-1920).

Also see Goldman (1984).

Also see Pollock (2007). 

Also see ``Valmiki Ramayana".



%{\sl\bfseries Ramāyaṇa} of Vālmīki. See Mudholkar and Katti (1914--1920).
%See Goldman (1984).
%See Gita Press (2006).
%See Pollock (2007).
%See "Valmiki Ramayana".

%\bibitem[]{chap3_item118_1}
%Gita Press. (2006). Rāmāyaṇa of Vālmīki. Gorakhpur: Gita Press.

%
%{\sl\bfseries Ramāyaṇa} of Vālmīki. (2006). Gorakhpur: Gita Press. (No editor mentioned)

\bibitem[]{chap3_item119}
{\sl\bfseries Rāmāyaṇa-mañjarī} of Kṣemendra. See Sastri (1903).

\bibitem[]{chap3_item120}
Raychaudari, Hemchandra. (1953). {\sl Political History of Ancient India}. Calcutta: University of Calcutta.

\bibitem[]{chap3_item121}
{\sl\bfseries Ṛgveda}. See Griffith and Keith (2017).

%\bibitem[]{chap3_item121_1}
%Davids, Rhys. T. W. (1899). {\sl Dialogues of the Buddha}. Oxford University Press. New York.

\bibitem[]{chap3_item122}
Rice, Stanley. (1924). {\sl Tales from the Mahābhārata}. London: Selwyn \& Blount. 

\bibitem[]{chap3_item123}
Sachs, Joe. (1999). {\sl Aristotle’s Metaphysics}. New Mexico: Green Lion Press. 

\bibitem[]{chap3_item124}
Said, Edward W. (2004).  {\sl Humanism and Democratic Criticism}. New York: Columbia University Press. 

\bibitem[]{chap3_item125}
Sarkar, B. K. (1913). {\sl Śukra-nīti-sāra}. Allahabad: Sudhindranatha Vasu. 

\bibitem[]{chap3_item125_1}
Sarup, Lakshman (Ed.) (1967). {\sl The Nighaṇṭu and the Nirukta}. Delhi: Motilal Banarsidass.

\bibitem[]{chap3_item126}
Sastri, Bhavadatta (Ed.) (1903). {\sl Rāmāyaṇa-mañjari} of Kṣemendra. Mumbai: Tukaram Javaji. 

\bibitem[]{chap3_item127}
Sastri, Ganapati T. (1909). {\sl The Vyakti-viveka of Rājānaka Mahimabhaṭṭa and its commentary of Rājānaka Ruyyaka}. Tranvancore: Trivandrum Sanskrit Series. 

\bibitem[]{chap3_item128}
Sastri, Harihara (Ed.) (1924). {\sl Madhurāvijaya} of Gaṅgādevi. Trivandrum: Sridhara Power Press. 

%\bibitem[]{chap3_item128_1}
%Sastri, Katti Srinivasa (Ed.) (1912). {\sl Valmiki Ramayana with Three Commentaries of Tilaka, Shiromani and Bhooshana}. Bombay: Gujarati Printing Press.

\bibitem[]{chap3_item129}
{\sl\bfseries Śatapatha-brāhmaṇa}. See Caland and Vira (1983).

\bibitem[]{chap3_item129_1}
Sathwalekar, S. D. (Ed.) (1957). {\sl Taittirīya-saṁhitā}. Surat: Swadhyaya Mandala.

\bibitem[]{chap3_item130}
Schleiermacher, Friedrich, Duke, J., Forstman, J., \& Kimmerle, H. (1997). {\sl Hermeneutics: The handwritten manuscripts}. Atlanta: Scholars Press.  

\bibitem[]{chap3_item131}
Schokker, G. H. (1966). {\sl The Pādatāḍitaka of Śyāmilaka (Part 1)}. Netherlands: Hague Publication. 

\bibitem[]{chap3_item132}
---\kern3pt(1976). {\sl The Pādatāḍitaka of Śyāmilaka (Part 2)}. Netherlands: Dordrecht. 

\bibitem[]{chap3_item133}
Schuon, Frithjof. (2007). {\sl Art from the Sacred to the Profane: East and West}. Indiana: World Wisdom Inc. 

\bibitem[]{chap3_item134}
Scodel, Ruth.\ (2009). {\sl Listening to Homer: Tradition, Narrative, and Audience}. Ann Arbor: University of Michigan Press. 

\bibitem[]{chap3_item135}
Sharma, Ravindra.\ (1988). {\sl Kingship in India from Vedic age to Gupta age}. New Delhi: Atlantic Publishers \& Distributers. 

\bibitem[]{chap3_item136}
Sharma, A. K. (1974). “Evidence of Horse from the Harappan Settlement at Surakoṭaḍā”. {\sl Purātattva}. No~7, pp~75--76

\bibitem[]{chap3_item137}
Shastri, J. L. (1983). {\sl Manusmṛti}. Delhi: Motilal Banarsidass. 

\bibitem[]{chap3_item137_1}
{\sl\bfseries Śiromani}. See Mudholkar (1914--1920).

\bibitem[]{chap3_item138}
{\sl\bfseries Śukra-nīti-sāra}. See Sarkar (1913). 

\bibitem[]{chap3_item139}
Sukthankar, V.S. (1957). {\sl On the Meaning of the Mahābhārata}. Bombay: The Asiatic society of Bombay. 

\bibitem[]{chap3_item140}
Sukthankar, V.S. {\sl et al.} (Ed.) (1933--1966). {\sl Mahābhārata} (19 vols). Poona: Bhandarkar Oriental Research Institute. 

\bibitem[]{chap3_item141}
Suri Srikrishna (Ed.) (1909). {\sl Kavyālaṅkāra of Bhāmaha}. Srirangam: Vani Vilas Press. 

\bibitem[]{chap3_item142}
Swahananda, Swami (Trans.) (2016). {\sl Chandogya and Brihadaranyaka Upanishads with Short Commentaries}. Createspace Independent Publishing Platform. (Self-Publishing). 

\bibitem[]{chap3_item143}
Swenson, David. F. (1936). {\sl Philosophical Fragments by Soren Keirkegaard}. New Jersey: Princeton University Press. 

\bibitem[]{chap3_item144}
{\sl\bfseries Taittirīya-brāhmaṇa}. See Kashyap (2017).

\bibitem[]{chap3_item144_1}
{\sl\bfseries Taittirīya-saṁhitā.} See Sathwalekar (1957).

\bibitem[]{chap3_item145}
Talageri, Srikanth. (2000). {\sl Rigveda: A Historical Analysis}. New Delhi: Aditya Prakashan. 

\bibitem[]{chap3_item146}
Talbot, Cynthia. (1995). “Inscribing the other, inscribing the self: Hindu-Muslim identities in pre-colonial India.” {\sl Comparative Studies in Society and History} Vol.~37, No.~4 (Oct., 1995). pp.~692--722.

\bibitem[]{chap3_item147}
{\sl\bfseries Vakrokti-jīvita} of Kuntaka. See Krishnamoorthy (1977).

\bibitem[]{chap3_item147_1}
"Valmiki Ramayana -- Developed and Maintained by IIT Kanpur". \url{https://www.valmiki.iitk.ac.in/}. Accessed on 15th February 2018.

\bibitem[]{chap3_item148}
{\sl\bfseries Vaśiṣṭha-dharma-sūtra}. See Bühler (1882).

\bibitem[]{chap3_item149}
{\sl\bfseries Veda-s}, The. See Griffith and Keith (2017).

\bibitem[]{chap3_item150}
Venkatanathacarya, N S (Ed.) (1974). {\sl Mammaṭa’s Kāvyaprakāśa}. Mysore: Mysore Oriental Research Institute

\bibitem[]{chap3_item151}
Verghese, Anila. (1995). {\sl Religious Traditions at Vijayanagara: as revealed through its monuments}. New  Delhi: Manohar Publishers. 

\bibitem[]{chap3_item152}
Wildman, S., Burne-Jones, E. C., Christian, J., Crawford, A., Des, C. L. (1998). {\sl Edward Burne-Jones, Victorian artist-dreamer}. New York: Metropolitan Museum of Art. 

\bibitem[]{chap3_item153}
Yonge, C. D. (1993). {\sl The works of Philo}. Massachussets: Hendrickson Publishers.
\end{thebibliography}
