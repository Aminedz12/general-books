\chapter{peYthAgorasfna saMKeyxgaLu}
\vskip -20pt

laMbakoVna tirxBujadalilx bAhugaLelAlx pUNARkagaLAguvaMte Arisuvudu oMdu muKayxvAda parxshenx. idakekx saMbaMdhisidaMte barxhamxgupatxna garxMthadalilx kaMDubarutatxde.

yAvudAdarU $3$ pUNaRsaMKeyxgaLa peYki, eraDU saMKeyxgaLa vagaRgaLa motatxvu, mUraneya saMKeyxya vagaRkekx samanAdare A saMKeyxgaLige peYthAgorasfna saMKeyxgaLu enunxtetxVve.

\textbf{udAharaNe:} $3,4$ matutx $5$  peYthAgorasf saMKeyxgaLu.
$$
\text{EkeMdare}\quad 3^2+4^2=5^2
$$
$3,4$ matutx $5$ saMKeyxgaLanunx mUla peYthAgorasf saMKeyxgaLu enunxtetxVve. ililx $3$ saMKeyx\-gaLU karxmAgatavAgi baMdive. I guNavanunx hoMdiruva saMKeyxgaLu idoMdeV eMdu kANisutatxde.
$$
a^2+b^2 = c^2
$$

idanunx peYthAgorasf  na samiVkaraNa enunxtetxVve. $a,b$ matutx $c$ gaLige oMdu nidiRSaTx belegaLanunx koTATxga mAtarx peYthAgorasf saMKeyxgaLu dorakutatxve.

biVjagaNitada riVti idakekx oMdu sAmAnayx sUtarxvide.
$$
a=m, \quad b= \frac{m^2-1}{1}, \quad\text{matutx} \quad c=\frac{m^2+1}{2} \quad \text{Agirali}
$$
Iga `$m$' ge pUNaRsaMKeyxgaLanunx AdeVshisidAga peYthAgorasf na saMKeyxgaLu namage dorakutatxve.
$$
a^2+b^2=c^2 \quad\text{sUtarxkekx AdeVshisidare}
$$
\begin{align*}
\left(m^2\right)^2+\left(\frac{m^2-1}{2}\right)^2 &= \left(\frac{m^2+1}{2}\right)^2\\
m^2+ \frac{m^4-2m^2+1}{4} &= \frac{m^4+2m^2+1}{4}\\
\frac{4m^2+m^4-2m^2+1}{4} &= \frac{m^4+2m^2+1}{4}\\
\frac{m^4+2m^2+1}{4} &= \frac{m^4+2m^2+1}{4}
\end{align*}
Iga $m=3$ Adare
\begin{align*}
(3)^2+\left(\frac{3^{2}-1}{2}\right)^2 &= \left(\frac{3^2+1}{2}\right)^2\\
(3)^2+(4)^2 &= (5)^2
\end{align*}

aMdare $m=3$ AdAga $3,4,5$ eMba {\bf peYthAgorasf saMKeyxgaLu} dorakutatxve.

$m=5$ \quad AdAga
\begin{align*}
(5)^2 + \left(\frac{5^2-1}{2}\right)^2 &= \left(\frac{5^2+1}{2}\right)^2\\
5^2 +(12)^2 &= (13)^2
\end{align*}

$m=5$ AdAga $5,12,13$ peYthAgorasf saMKeyxgaLu dorakutatxve.

meVlina eraDu saMdaBaRgaLalUlx $m$ ge besasaMKeyxgaLanenxV AdeVshisidevu.
sama\-saMKeyxgaLanunx AdeVshisidare
$$
m=4 \quad \text{AdAga}\quad \frac{m^2-1}{2}=\frac{15}{2} \quad \frac{m^2+1}{2} = \frac{17}{2}
$$

aMdare $4$, $\frac{15}{2}$, $\frac{17}{2}$ peYthAgorasf saMKeyxgaLu laBisiduvu. AdarU samiVkaraNa sari eMdu toVrisabahudu
\begin{align*}
(4)^2+\left(\frac{15}{2}\right)^2 &= \left(\frac{17}{2}\right)^2\\
\frac{16}{1}+\frac{225}{4} &= \frac{289}{4}\\
\frac{64+225}{4} &= \frac{289}{4}\\
\frac{289}{4}&= \frac{289}{4}
\end{align*}

peYthAgorasf na saMKeyxgaLanunx kaMDuhiDiyalu matotxMdu sUtarxvide.
$$
\left(m^2-n^2\right)^2 + (2mn)^2 = (m^2+n^2)^2
$$

aMdare ililx\; $a=m^2-n^2$, \quad $b=2mn$ \quad \text{matutx} \quad $c=m^2+n^2$

meVlina samiVkaraNadalilx $m=2$ matutx $n=1$ Adare
$$
a=3,\quad b=4, \text{matutx} \quad c=5 \; \text{dorakutatxde}.
$$

idu namage gotitxruvaMte peYthAgorasf saMKeyxgaLu.

$m=3$ \quad matutx \quad $n=2$ \quad AdAga

$a=5$, \quad $b=12$ \quad matutx\; $c=13$ \quad dorakutatxde. idU saha peYthAgorasf saMKeyxgaLeV. meVle baMdiruva samiVkaraNagaLanunx gamanisidAga eraDu viSayagaLanunx nAvu gamanisabahudu.
\begin{enumerate}
\item[{\rm 1)}] mUru saMKeyxgaLU samanAgiralu sAdhayxvilalx.
\item[{\rm 2)}] $a,b,c$ gaLa belegaLu AroVhaNa karxmadalelxV baMdive.
\begin{itemize}
\item \begin{tabbing}
sAvaRtirxkavAgi \;\;
\= \= $a\neq  b \neq c$ \\
\> \> $a < b < c$
 \end{tabbing} 
\end{itemize}
\end{enumerate}

$1$ matutx $2$ peYthAgorasf na saMKeyxgaLalalx.
$$
\begin{array}{lclclc}
a=6 & \text{AdAga} &  b=8  & \text{matutx}  & c=10 &\text{Agutatxde}\\  
a=10 & \text{AdAga} & b=24  & \text{matutx} & c=26 &\text{Agutatxde}\\ 
a=12 & \text{AdAga} & b=35  & \text{matutx} & c=37 &\text{Agutatxde}
\end{array}
$$

\centerline{ivelAlx peYthAgorasf na saMKeyxgaLu.}
$$
\begin{array}{llll}
a=7 & \text{AdAga} &  b=24  &  c=25 \\  
a=13 & \text{AdAga} & b=84  &  c=85 \\ 
a=17 & \text{AdAga} & b=144 &  c=145 
\end{array}
$$

\centerline{punaH  ivelAlx peYthAgorasf saMKeyxgaLu.}
\begin{flalign*}
\text{AgaleV heVLidaMte}\qquad & 3^2+4^2=5^2 &\\
&(33)^2+(44)^2  =(55)^2\\
&(333)^2+(444)^2 =(555)^2\\
&(3333)^2+(4444)^2 =(5555)^2
\end{flalign*}
idu hAgeyeV muMduvariyutatxde

peYthagorasf na saMKeyxgaLanunx beVroMdu riVtiyalilx paDeyabahudu

\begin{center}
\begin{tabular}{>{$}l<{$}>{$}l<{$}>{$}l<{$}>{$}l<{$}}
1\frac{2}{3} = \frac{5}{3} & 5-1=4 & 3,4,5 &{\frac{5}{3}\text{eMbudaralilx aMshavutirxBujada }}\\ 
                           &       &       &\text{kaNaRvAgiyU, CeVdavu oMdu}\\
                            &      &       &\text{bAhuvAgiyU iruvudu.}\\ [0.1cm]
2\frac{3}{5} = \frac{13}{5}& 13-1=12 & 5,12,13 & \text{ideV riVti}\frac{13}{5}\;\text{eMbudaralilx}\\[0.2cm] 
3\frac{4}{7} = \frac{25}{7}& 25-1=24 & 7,24,25& \frac{25}{4}\;\text{eMbudaralilx}\\[0.2cm]
4\frac{5}{9} = \frac{41}{9}& 41-1=40 & 9,40,41 & \frac{41}{9}\;\text{eMbudaralilx}
\end{tabular}
\end{center}

$3 \cdot 4\cdot 5$, $5\cdot 12 \cdot 13$, $7\cdot 24\cdot 25$, $9\cdot 40 \cdot 41$ ivugaLelalx peYthagorasf saMKeyxgaLu.

$2n+1$, $2n^{2}+2n$, $2n^2+2n+1$ gaLu 

peYthAgorasf saMKeyxgaLanunx paDeyalu matotxMdu sUtarxvide Adare idu sAvaRtirxkavalalx. (nidiRSaTx belegaLige mAtarx anavxya) 

$$
\begin{array}{lll}
n=1 & \text{AdAga} & 3 \cdot 4 \cdot 5 \\
n=2 & \text{AdAga} & 5 \cdot 12 \cdot 13 \\
n=3 & \text{AdAga} & 7 \cdot 24 \cdot 25 \\
\end{array}
$$
\centerline{hiVgeyeV muMduvariyutatxde.}

peYthAgorasfna parxkAra laMbakoVnada bAhugaLanunx kaMDuhiDiyalu I sUtarx\-vide $2n+1, 2n^{2}+2n, 2n^{2}+2n+1$  \quad $n=2$ \;AdAga $5,12,13$ eMdu barutatxde. Adare peYthAgorasfna utatxradalilx kaDeya saMKeyxgaLa aMtaravu $1$ AdadxriMda utatxravu oMdu visheVSa saMdaBaRvanunx mAtarx tiLisutatxde. sAvaRtirxkavAda utatxravalalx. peYthAgorasf na parxmeVya eMdu iMdu nAvu kareyutitxruvudanunx bahaLa hiMdiniMdalU IjipfTx, iMDiya, ciVnAgaLalilx baLakeyalilxtutx yUkilxDfna modalaneya pusatxkadalilx kaMDubaruva sAdhaneyu yUkilxDaDxneV savxtaH koTiTxruvudu. adu peYthAgorasf nadalalx. peYthAgorasf iMDiyakekx baMdu idanunx BAratiVyariMda kalitukoMDa eMdu heVLutAtxre.













