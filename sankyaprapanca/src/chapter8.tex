\chapter{aviBAjayx saMKeyxgaLa anaMta sAmArxjayx}


gaNitavanunx `vijAcnxnada rANi' eMdu kareyutAtxre. BwtavijAcnxna matutx gaNita \-parasapxra avalaMbita viSayagaLu. gaNitavilalxde BwtavijAcnxna beLeyalu sAdhayxvAguvudilalx. AdarU saMKAyxsidAdhxMta mAtarx hiMdiniMdalU tananx shudadhxteyanunx uLisikoMDu baMdide. aSeTxV alalx gahanavU, shudadhxvU Ada gaNita ciMtaneya utapxnanx eninxsikoMDide.

saMKAyxsidAdhxMtada halavAru parxmeVyagaLanunx inUnx sAdhisabeVkAgide. 

\textbf{udAharaNege:} aviBAjayx saMKeyxgaLu anaMtave athavA avugaLige miti ideyeV? eMba parxshenxge utatxra kaMDukoLaLxbeVkAgide. 

oMdu matutx adeV saMKeyxyanunx biTuTx beVre yAva dhanapUNARMkagaLiMdalU BAgisalapxDada saMKeyxgaLeV aviBAjayx saMKeyxgaLu.

aMkagaNitada mUlaBUta parxmeVyada parxkAra aviBAjayx saMKeyxgaLu oMdu eMba saMKeyxgiMta doDaDxdAgirabeVku. saMKeyxgaLa adhayxyanadalilx yUkilxDf I viSayavanunx pArxtayxkiSxkeya mUlaka tiLisidAdxne. AdadxriMda oMdanunx I sherxVDhiyiMda keYbiDalAgide.
 
 $2,3,5,7,11,13,17\cdots$ muMtAda saMKeyxgaLanunx apavataRnagaLAgi oDe\break yalu sAdhayxvilalxvAdadxriMda ivugaLanunx aviBAjayx saMKeyxgaLenunxtetxVve $(prime\- Numbers)$.

sAvxBAvika saMKeyxgaLu $(Natural Numbers)$ anaMtavAgiruvudariMda, avi\-BAjayx saMKeyxgaLu saha anaMtavAgirutatxve. hAgAdare I elAlx aviBAjayx saMKeyxgaLanunx kaMDukoLaLxlu sAdhayxveV? eMdu parxshenx hAkikoMDAga, inUnx saMpUNaR aviBAjayx saMKeyxgaLanunx kaMDuhiDiyuva sUtarxvanunx kaMDukoLaLxlu sAdhayxvAgilalx. Adare aneVka parxyatanxgaLu ililxyavarege naDedide.

$1,2,3,4,5\cdots$ muMtAduvu dhana pUNARMkagaLa sherxVDhi. idu anaMtavAgi kone eMbudilalxde sAgutatxde. idaralilx $1,3,5,7\cdots$ muMtAduvu besasaMKeyxgaLa sherxVDhi $2,4,6,8\cdots$ muMtAdavu samasaMKeyxgaLa sherxVDhi.

samasaMKeyxgaLa peYki $2$ nunx biTuTx uLida elAlx saMKeyxgaLanunx apavataRnagaLAgi oDeyabahudu. ivugaLelAlx BAjayxsaMKeyxgaLu.

besasaMKeyxgaLa sherxVDhiyalilx kelavu BAjayx saMKeyxgaLu matetx kelavu aviBAjayx saMKeyx\-gaLu.

$2,3,5,7,11,13,17,19,23\cdots$ muMtAda sherxVDhiyalilxruva parxtiyoMdu saMKeyxyU oMdariMda matutx adeV saMKeyxyiMda mAtarx BAgavAgutatxve. beVre yAva saMKeyxyiMdalU BAgavAguvudilalx.

kirx.~pU.\ $230$ ralilx yUkilxDfna samakAliVnanAda eraToVsatxniVsf eMbuvanu aviBAjayx saMKeyxgaLanunx tiLiyuva karxmavanunx pArxyoVgikavAgi toVrisidAdxne. avana karxma eraToVsatxniVsf jaraDi eMdu parxsidadhxvAgide. avanu hoDeduhAkuva vidhAna\-vanunx upayoVgisi oMdariMda, oMdanUru saMKeyxgaLa naDuve $25$ aviBAjayx saMKeyxgaLanunx kaMDukoMDa. heVge baMtu? avanige noVDoVNa $1$ riMda $100$ ravaregina pUNaRsaMKeyxgaLanunx baredu, $2$ nunx biTuTx $2$ riMda nisheshxVSavAgi BAgavAguva elAlx saMKeyxgaLanunx hoDeda, naMtara $3$ nunx biTuTx $3$ riMda nisheshxVSavAgi BAgavAguva elAlx saMKeyxgaLanunx hoDeda, anaMtara $5$ nunx biTuTx $5$ riMda nisheshxVSavAgi BAgavaguva elAlx saMKeyxgaLanunx hoDeda, anaMtara $7$ nunx biTuTx $7$ riMda nisheshxVSavAgi BAgavAguva elAlx saMKeyxgaLanunx hoDeda. Iga uLida saMKeyxgaLeV aviBAjayx saMKeyxgaLu. saMKeyx jAsitxyAdAga $11$ matutx $13$ riMda hiVgeyeV muMduvarisabeVku. hiVge avanu oMdariMda , oMdanUraravarege avanunx $25$ aviBAjayx saMKeyxgaLanunx paDeda.

saMKeyxgaLu hecucx idAdxga I riVtiya hoDeyuva karxmadiMda aviBAjayx saMKeyxgaLa pUNaRvAda sherxVDhi paDeyalu sAdhayxvAguvudilalx nijavAgi heVLuvudAdare, aviBAjayx saMKeyxgaLa sherxVDhige kone eMbudeVyilalx. ivugaLa haMcikeyalilx niVtiniyamagaLilalx. samapaRka sUtarx kaMDuhiDiyalu yAva gaNitajacnxnigU sAdhayxvAgilalx. 

kaniSaThx aviBAjayx saMKeyx $2$, hAgAdare gariSaThx aviBAjayxsaMKeyx yAvudu? I samaseyxya bagegx talekeDesikoMDavanu girxVsf deVshada pArxciVna gaNitajacnx yUkilxDf. ivana parxkAra gariSaThx aviBAjayxsaMKeyx eMbudeV ilalx. niVvu eSeTxV doDaDxdAda aviBAjayxsaMKeyxyanunx koTaTxre. adakikxMta doDaDxdAda aviBAjayx saMKeyx ide eMdu nAnu sAdhisabalelx eMdu heVLi jotege niKaravAda gaNita sAdhane niVDida.

ideV riVti $6n^2+6n+31$ eMba sUtarxdalilx $n$ ge kelavu nidiRSaTx belegaLanunx koTATxga mAtarx $74$ aviBAjayx saMKeyxgaLu dorakutatxve.

$3n^2+3n+23$ eMba sUtarxdalilx $n$ ge kelavu nidiRSaTx belegaLanunx koTATxga mAtarx $71$ aviBAjayx saMKeyxgaLu laBayxvAgutatxde.

$5n+3$ eMba sUtarxdalilx $n$ na elAlx belegaLigU aviBAjayx saMKeyxgaLu bAradidadxrU. I sUtarxvU hecicxna aviBAjayxsaMKeyxgaLanunx niVDutatxde. eMdu tiLidu baMdide.

$k=1$ athavA $3$ matutx $n$ yAvudeV pUNARMkavAdAga $2^{n}\cdot k+1$ eMbudu aviBAjayx saMKeyx eMdu rAbinasxnf sUtarx tiLisutatxde. Adare $n=5$ matutx $k=1$ AdAga baruva $33$ aviBAjayx saMKeyxyalalx, $n=5$ matutx $k=3$ AdAga baruva $97$ aviBAjayxsaMKeyx.

IgAgaleV dhanapUNARMkagaLa sherxVDhiyanunx anukUla aMtaragaLAgi viBAgisi parxtiyoMdu aMtaradalilxyU eSeTxSuTx aviBAjayx saMKeyxgaLive eMbudanunx patetx mADidAdxre.

$2$ riMda $1000$ ravarege $168$.
 
$2$ riMda $5000$ ravarege $1146$. 

$2$ riMda $50,000$ ra varege $5133$ aviBAjayxsaMKeyxgaLive.

idara jotege aviBAjayx saMKeyxgaLa EkasAthxnadalilx $1,3,7,9$ enunxva aMkagaLeV irabeVku eMbudanunx tiLisidAdxre.

aviBAjayx saMKeyxgaLanunx kaMDuhiDiyalu $n^2+n+11$ matutx $n^2+n+17$ eMba matetxraDu sUtarxgaLive. $n=0$ matutx $n=79$ I avadhiyalilx $80$ aviBAjayx saMKeyxgaLu dorakutatxve. Adare $n=80,81,84,89,96$ belegaLanunx koTATxga BAjayxsaMKeyxgaLu laBayxvAgutatxve.

elAlx aviBAjayx saMKeyxgaLanUnx, koneya pakaSx hecucx, aviBAjayx saMKeyxgaLanunx niVDuva oMdu sUtarxvanunx tiLiyalu gaNita saMshoVdhakaru iMdigU parxyatanx mADutitxdAdxre. aviBAjayx saMKeyxgaLa saMBavaniVyate hecucx anishacxteyiMda kUDide.

$10^{2n}-10^n+1$ eMba sUtarxdalilx, $n=1$ AdAga BAjayxsaMKeyx. 

$n=2$ AdAga aviBAjayxsaMKeyx.

$n=3$ AdAga BAjayxsaMKeyx.

$n=4$ AdAga aviBAjayxsaMKeyx.

hiVge biTuTx biTuTx aviBAjayx saMKeyxgaLu baMdiruvudanunx nAvu noVDabahudu. hiVge aviBAjayxsaMKeyxgaLanunx paDeyalu aneVka sUtarxgaLidadxrU yAvudU samapaRka\-vAgilalx. I samaseyx iMdigU jaTila. saMKAyxparxpaMcadalilx idoMdu BAriV savAlu.

PamaRTf mitarxnAda $17$ neya shatamAnada PArxnfsxdeVshada pAdirx maseRnf matotxMdu sUtarxvanunx nirUpisida. A sUtarxveV 
$$
2^n-1
$$
I sUtarxdalilx $n=2$ matutx $n=3$ AdAga baruva $3$ matutx $7$ aviBAjayx saMKeyxgaLu.

$2^n-1$ aviBAjayx saMKeyxyAgabeVkAdare avashayxkavAgi $n$ oMdu aviBAjayx saMKeyxyAgirabeVku eMdu maseRnf tiLisida. sUtarxdalilx $n=2$, $n=3$, $n=5$, $n=11$ AdAga karxmavAgi $3,7,31,2047$ aviBAjayx saMKeyxgaLu dorakutatxve.

Adare $n= 179,181,197$ matutx $233$ AdAga $2^{n}-1$ eMbudu oMdu BAjayx saMKeyx eMdu tiLidubaMdide.

$n$ na bele $2,19,23,317$ AdAga $\frac{10^n-1}{9}$ eMbudu aviBAjayx saMKeyxyAgiru\-tatxde eMdu tiLidubaMdide. idanunx riponeTf sUtarx enunxtAtxre.

inunx atayxMta doDaDx aviBAjayx saMKeyx yAvudu? eMbudara bagegx gaNitajacnxra gamana hariyalu AraMBavAyitu.

AraMBadalilx $2^{127}-1$ doDaDx aviBAjayx saMKeyx eMdaru.

anaMtara $2^{11,213}-1$ doDaDx aviBAjayx saMKeyx eMdaru.

kirx.~sha.\ $1971$ ralilx, $24$ neya maseRnf aviBAjayx saMKeyxyanunx kaMDuhiDilAyitu.

AsaMKeyxyeV $2^{19,937}-1$ idaralilx $6002$ aMkagaLive.

anaMtara $2^{21,701}-1$ idaralilx $6530$ kikxMtalU hecucx aMkagaLive.

AgasfTx $23$, $2008$ iMTarfneTf maseRnf aviBAjayx saMKeyxgaLa \-saMshoVdhaneyiMda laBisida atayxMta doDaDx aviBAjayx saMKeyx
$$ 
2^{4,31,12,609}-1
$$

($2$ ra GAta $4$ koVTi, $31$lakaSx, $12$sAvirada $609-1$)

idaralilx $1,29,78,189$ ($1$koVTi, $29$ lakaSx, $78$ sAvirada, $189$ aMkagaLive.)

idu maseRnf na $47$ neya aviBAjayx saMKeyx. Adare sepeTxMbarf $6$, $2008$ matutx Epirxlf $12$, $2009$ ralilx aviBAjayx saMKeyxgaLa bagegx saMshoVdhane naDedarU AgasfTx $23$, $2008$ ralilx tiLida maseRnf aviBAjayxveV atayxMta doDaDx aviBAjayx saMKeyx eMdu tiLiyitu. 

ititxVcina saMshoVdhaneyiMda $58$ neya maseRnf na atidoDaDx aviBAjayx saMKeyx labadhxvAgide. A saMKeyxyeV 
$$
2^{5,78,85,161}-1
$$

($2$ ra GAta $5$ koVTi, $78$ lakaSx, $85$, sAvira,$161 - 1$)\\
idaralilx $17$ miliyanf aMkagaLive eMdugotAtxgide aMtU atayxMta doDaDx aviBAjayx saMKeyxyanunx huDukuva oMdu parxyatanxmAtarx muMduvariyutitxde.

gaNitajacnxru elAlx aviBAjayx saMKeyxgaLige anavxyavAguva vicitarx lakaSxNagaLanunx adhayxyana mADi gurutisidAdxre. idanunx aviBAjayx saMKeyxgaLa savxBAva eMdu heVLa\-bahudu.

\begin{enumerate}[{\rm 1)}]
\itemsep=0pt
\item eraDakikxMta doDaDxdAda yAvudeV samasaMKeyxyanunx eraDu aviBAjayxsaMKeyxgaLa motatxvAgi bareyabahudu.

\textbf{udAharaNege:}\quad $11+13=24$.

idu raSAyxdeVshada gaNitajacnx goVlfDxbAkf eMbuvana UhAparxmeVya.

\item saMKeyx $3$ kikxMta adhikavAgiruva yAvudAdarU eraDu aviBAjayx saMKeyxgaLa vagaR\-gaLa vayxtAyxsavu $24$ eMba saMKeyxyiMda nisheshxVSavAgi BAgavAgutatxde.

\textbf{udAharaNege:} \qquad $7^2-5^2 = 49-25 = 24$.

\item eraDu eMba saMKeyxyanunx biTuTx uLida yAvudeV eraDu karxmAgata aviBAjayx saMKeyxgaLa 
vayxtAyxsa yAvAgalU oMdu samasaMKeyxyAgirutatxde.

\textbf{udAharaNege:}\qquad $19-17=2$.

\item $1,3$ matutx $5$ eMba $3$ saMKeyxgaLanunx biTuTx uLida elAlx saMKeyxgaLU mUru aviBAjayx saMKeyxgaLa motatx eMdu toVrisabahudu.

\textbf{udAharaNege:} \qquad $3+5+7=15$.

\item elAlx BAjayx saMKeyxgaLanUnx aviBAjayx saMKeyxgaLa apavataRnagaLa \-guNalabadhxvanAnxgi nirUpisabahudu.

\textbf{udAharaNege:}\qquad $15=3 \times 5$.

\item eraDu eMba saMKeyxyanunx biTuTx yAva aviBAjayx saMKeyxyanAnxgali eraDu beVre beVre vagaR saMKeyxgaLa motatx athavA vayxtAyxsa eMdu toVrisabahudu.

\begin{tabular}{@{}ll}
\textbf{udAharaNege:} & $3 = 2^2-1^1$\\
					  & $5 = 2^2+1^2$
\end{tabular}
\item eraDariMda AraMBisi elAlx karxmAgata aviBAjayx saMKeyxgaLa guNalabadhxkekx oMdanunx seVrisidare aviBAjayx saMKeyxyeV laBayxvAgutatxde.
\begin{flalign*}
\textbf{udAharaNege:} \qquad &2\times 3+1 = 7&\\
&2\times 3\times 5+1 = 31
\end{flalign*}

\item $4n+1$ eMba rUpavuLaLx parxtiyoMda aviBAjayx saMKeyxyanunx eraDU vagaRgaLa motatxdaMte bareyabahudu.

\textbf{udAharaNege:}\qquad $13 = 2^2 +3^2$.

I parxmeVyavanunx Ayilaranu sAdhisi, $2^n(4n+1)$ eMba rUpavuLaLx parxtiyoMdu saMKeyxgU ideV guNavide eMdu tiLisidanu.

\item kelavu saMdaBaRgaLalilx aviBAjayx saMKeyxgaLu matutx avugaLa viloVmagaLu aviBAjayx saMKeyxgaLAgirutatxve.
\begin{flalign*}
\textbf{udAharaNege:}\qquad &13\; \text{matutx}\; 31&\\
&17\; \text{matutx}\; 71
\end{flalign*}
\item eraDu matutx mUranunx biTuTx elAlx aviBAjayx saMKeyxgaLa vayxtAyxsavu besasaMKeyx Agiralu sAdhayxvilalx.

\textbf{udAharaNege:}\qquad $5,7,17,19$

\item kelavu aviBAjayx saMKeyxgaLu aMkagaNita sherxVDhiyalilxve.

\textbf{udAharaNege:}\qquad $7,37,67,97$

\item $2$ matutx $5$ nunx biTuTx yAvudeV aviBAjayx saMKeyxyiMda oMdanunx BAgisidAga oMdu AvataRka dashamAMsha doreyutatxde.
\begin{flalign*}
\textbf{udAharaNege:} \qquad \frac{1}{7} &= 0.142857&\\
\frac{1}{17} &=0.0588235294117647
\end{flalign*}

\item kelavu aviBAjayx saMKeyxgaLu mAlA aviBAjayxsaMKeyxgaLAgirutatxve.

\textbf{udAharaNege:}\qquad $101,131,181,313$

\item $2$ matutx $3$ ivu aviBAjayx saMKeyxgaLu akakxpakakxdalilx baMdive. Adare I riVtiya matotxMdu udAharaNe siguvudu kaSaTx. hiVge aviBAjayx saMKeyxgaLa yAterx anaMta.
\end{enumerate}

