\makeatletter
\def\@makeschapterhead#1{%
  \vspace*{50\p@}%
  {\parindent \z@ \raggedleft
    \normalfont
    \interlinepenalty\@M
    \LARGE \bfseries  #1\par\nobreak
    \vskip 20\p@
  }}
\makeatother

\chapter*{Our Contributors\\ {\rm\sl\small (in alphabetical order of last names)}}\label{contributors}

\lhead[\small\thepage]{}
\rhead[]{\small\thepage}
\chead[]{}
\cfoot[]{}

\section*{Vrinda Acharya}

Vrinda Acharya is an M.Com degree holder, an M.A. in Sanskrit and has completed Vidwat in Carnatic Classical Music. She is a full time professional Carnatic vocalist and musicologist of repute, and has received several awards, and performed across India and abroad. She has presented papers in national and international conferences on music and the Vedic heritage. She has given lectures on Indian Music in some universities in the US and is a recipient of research fellowship from Karnataka State Government. She has worked earlier for several years as a commerce faculty in reputed colleges and business schools in Bangalore.

\section*{Manjushree Hegde}

Manjushree Hegde is a researcher of Indology with a swadeshi perspective. She is currently employed as Assistant Professor at Amrita University in Coimbatore. She has completed her M.A. in Sanskrit from KSOU, Mysuru. She is a Mechanical Engineer from PESIT, Bengaluru by training, but finds herself more at home working with Sanskrit texts. She has studied Vyākaraṇa under Dr. Pushpa Dixit in Bilaspur, Chattisgarh, and has pursued Vedānta studies under Vidwan Ashwathnarayana Avadhani at Mattur, Karnataka. She has also been designated as Infinity Foundation India Research Fellow subsequent to her presentation at the first edition of the Swadeshi Indology Conference Series in July 2016.

\section*{Sowmya Krishnapur}

Dr.~Sowmya Krishnapur was a UGC Senior Research Fellow at the Department of Sanskrit, University of Madras. She has a PhD in the area of Sanskrit Grammar. After acquiring an M.S. in Biochemistry from Central College, Bengaluru, she did the Acharya (M.A.) degree at Rashtriya Sanskrit Sansthan, Sringeri. She has worked as Senior Computational Linguist and Subject Matter Expert at Vyoma Linguistic Labs Foundation, Bengaluru, for the production of several multimedia resources for Sanskrit learning. She has also presented and published several papers, notably one on Vyakarana at the 16th World Sanskrit Conference in Bangkok in 2015.

\section*{Subhodeep Mukhopadhyay}

Subhodeep Mukhopadhyay is an independent management consultant associated with education and agriculture sector, with a keen interest in Indian history from a civilizational perspective, Hindu science and technology, and also Tantra Śāstra. He is an MBA in Finance with a B.Tech in Computer Engineering. He is currently pursuing an Advanced Sanskrit course from Ramakrishna Mission, Kolkata, and is\break presently developing a rule-based Bengali to Sanskrit translation software, which could potentially speed up translation projects and contribute to the production of new Sanskrit works.


\section*{Sudarshan Therani}

Therani Nadathur Sudarshan is a computer scientist, programmer and hands-on technologist / engineer, startup founder, entrepreneur and technology consultant. He is deeply interested in discovering the immense “practicality” of the Indian Knowledge Systems. His primary research interests lie in Symbol Systems for representation and\break intelligence-spanning man-made material systems (AI), naturally occurring systems (biological) and the Indic symbol systems. {\sl Sampradāya} studies ({\sl viśisṭādvaita}) and practices are part of his upbringing, and immensely influence all his activities. He is also an active participant in the vibrant temple culture /events at many Vishnu temples in Tamil Nadu.

\section*{Surya K.}

Surya completed his B.Tech from IIT Madras, and his graduate studies in the US. He is currently settled in the US in a professional career. He is a member of Indian History Awareness \& Research (IHAR), an initiative of Arsha Vidya Satsanga. He also volunteers for North South Foundation (NSF) to raise money for scholarships for poor students in India. 

\section*{Rajath Vasudevamurthy}

Dr.~Rajath Vasudevamurthy is an Assistant Professor at Amrita School of Engineering, Amrita University, Bengaluru campus, in the ECE department. He has a PhD from the Department of Electrical Communication Engineering, IISc (Indian Institute of Science), Bengaluru, and is a Post Doc from The Pennsylvania State University, State College, USA. He is a keen student of Saṁskṛta and Vedānta, and actively involves himself in related learning and propagation activities.

