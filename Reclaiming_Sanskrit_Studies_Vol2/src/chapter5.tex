\chapter{{{\sl\bfseries Śāstra}\relax} -- an impediment to progress?$^{*}$}\label{chapter\thechapter:begin}
\vskip -10pt

\Authorline{Rajath Vasudevamurthy}
\lhead[\small\thepage\quad Rajath Vasudevamurthy]{}

\hfill{\sl(\url{rajath.v7@gmail.com})}

\footnotetext[1]{pp.~\pageref{chapter\thechapter:begin}--\pageref{chapter\thechapter:end}. In: Kannan, K S (Ed.) (2018) {\sl Śāstra-s Through the Lens of Western Indology - A Response}. Chennai: Infinity Foundation India.}

\rhead[]{\small \thechapter. {\sl Śāstra} -- an impediment to progress?\quad \thepage}

\vskip .7cm

\section*{Abstract}
\index{Rajath, V}

{\sl Śāstra} is of two types -- {\sl pauruṣeya}\index{pauruseya@\textsl{pauruṣeya}} and {\sl apauruṣeya} -- \index{apauruseya@\textsl{apauruṣeya}}  based on whether the subject matter revealed therein is accessible to be verified by the human mind or not; and whether the said {\sl śāstra} has an author (human) or not. Veda is {\sl apauruṣeya} since the subject matter it reveals, like {\sl dharmādharma-vyavasthā, punar-janman etc.}, can neither be proved nor disproved by the five senses and/or inference. Vedāṅga-s are ancillary sciences\index{science!ancillary} which help one in understanding the intended meaning of the Veda, while the Upaveda-s, Itihāsa-s, Purāṇa-s, Smṛti-s etc. are authored to help percolate the vision of the Veda to all sections of society, and so are considered {\sl pauruṣeya}.\index{Purana}

The contrarian view is that the Vedāṅga-s\index{Vedanga@Vedāṅga} - Vyākaraṇa\index{Vyakarana@\textsl{Vyākaraṇa}} and Chandas especially, were meant to describe the grammar\index{grammar} and prosody used in the Veda; somehow, over time, they became prescriptive\index{prosody} to the extent that future users of the language were forced to adhere to them more strictly than they are adhered to in the Veda itself. Such impositions can be found in the context of Dharmaśāstra-s, Āyurveda\index{Ayurveda@Āyurveda} etc. It is claimed thereby that {\sl śāstra}-s which were descriptive and flexible in spirit to begin with, became over time prescriptive and rigid, thereby stifling the creativity of the future generations.

On the one hand, we see that Indian society has been vigorous all through, making original contributions in fields such as science and technology, mathematics, economics, and philosophy and language; while on the other hand, it is said to have undergone deterioration for a thousand years. This contradiction begs an effort to understand Indian history and culture in a holistic manner, paying adequate heed to all the underlying aspects.

\section{Introduction}\label{art12-sec1}

A critical and honest examination of Pollock's paper on {\sl Śāstra}-s (Pollock 1985) will motivate the reader to undertake a thorough study of Indian history and the unique place {\sl Śāstra}-s hold in the Indian intellectual tradition. The study of Indian history is made complicated by the presence of various kinds of history -- Marxist,\index{Marxist!history} Nationalist etc. In the context of the field of Indology\index{Indology} being dominated by Westerners, many of whom are not sympathetic to the Indic civilization, Sri Aurobindo remarks ``...a time must come when the Indian mind will shake off the darkness that has fallen upon it, cease to think or hold opinions at second and third hand and reassert its right to judge and enquire in a perfect freedom into the meaning of its own Scriptures....'' (Aurobindo\index{Aurobindo, Sri} 1994:130). 

In recent times, a lot of new evidence is surfacing which challenges the long held positions about Indian history; prominent among them being ---
\begin{itemize}
\itemsep=1pt
\item[$\bullet$] Study of world economic history where India led the world GDP for at least 1500 years (Maddison\index{Maddison, Angus} 2001)

\item[$\bullet$] Influence of Indian mathematics\index{Indian!Mathematics} on Europe (Raju\index{Raju, C K} 2007)

\item[$\bullet$] Archaeological evidence from Bronze-age Harappan civilization\index{Harappan civilization} suggesting that it is at least 8000 year-old and not just 5500 (Sarkar {\sl et al.} 2016)

\item[$\bullet$] Re-examination of evidence; relating to Sandrokottos\index{Sandrokottos} and Devānāṁpriya Priyadarśī,\index{Devanampriya Priyadarsi@Devānāṁpriya Priyadarśī} and Buddhist chronicles\index{Buddhist chronicles} such as {\sl Dīpavaṁśa, Mahāvaṁśa}; pushing the date of the Buddha back by about 500 years (Arya 2015:286-292, Roy 2016)
\end{itemize}

Anybody who gets to know about the science \& technology, mathematics \& astronomy and economic history of ancient India and China seeks an explanation as to how and why the two countries paled into insignificance at the turn of the 20th century. In the Indian context, many scholars lay the blame for this decline at the doors of the invasions starting from Mahmud Ghazni\index{Ghazni Mahmud} to the advent of the British. This explanation is not entirely satisfying; for, prior to those events many invasions (Greek, Huns, Shakas etc.) were warded off and the refugees (Jews, Parsees etc.) were seamlessly absorbed into the Indian community (Sanyal 2008:5-19). Swami Vivekananda \index{Vivekananda, Swami} observes that with the overemphasis of {\sl nirvāṇa} for the masses in Buddhism \& Jainism, the society lost its {\sl kṣātra} and became vulnerable militarily to foreign invasions\endnote{Swami Vivekananda says thus in his address to junior {\sl Sannyāsin}-s at the Belur Math speaking on `{\sl Saṁnyāsa}: Its Ideal and Practice' --- ``First, we have to understand that we must not have any impossible ideal. An ideal which is too high makes a nation weak and degraded. This happened after the Buddhistic and the Jain reforms. On the other hand, too much practicality is also wrong. If you have not even a little imagination, if you have no ideal to guide you, you are simply a brute. So we must not lower our ideal, neither are we to lose sight of practicality.'' (Vivekananda\index{Vivekananda, Swami} 2007:402)}; and in turn led to an erosion of value systems with extraneous lifestyle, definitions of life goals and thought systems being imitated without proper examination. 

Sanyal\index{Sanyal, Sanjeev}  argues that the decline in the Indian civilization was due to a change in mindset; described as an inward-looking tendency, not open to comparison and sharing knowledge with others. Its most glaring effect is that crossing the ocean became a taboo to the people who had a great ship-building industry and maritime expertise (Sanyal 2010:19-24)!

In this paper, we closely examine another interesting idea proposed to explain this decline - forwarded by Pollock\endnote{Pollock says thus in the abstract that in Sanskritic tradition, ``Theory is held always and necessarily to precede and govern practice; there is no dialectical interaction between them. Two important implications of this fundamental postulate are that all knowledge is pre-existent, and that progress can only be achieved by a regressive re-appropriation of the past.'' (Pollock 1985:499)} -- that {\sl śāstra}-s are an impediment to creativity and progress. The rest of the paper is organized as follows: \S\ref{art12-sec2} (the next one) presents a traditional overview of {\sl śāstra}-s, \S\ref{art12-sec3} presents some of Pollock's views that {\sl śāstra}-s come in the way of progress and \S\ref{art12-sec4} presents the Conclusion.\\[-20pt]

\section{{{\sl\bfseries Śāstra}\relax}-s --- Insiders' view}\label{art12-sec2}

The basic details of how tradition views {\sl śāstra} will not be found explicitly discussed in the main source texts such as the Veda-s, the {\sl Bhagavad Gītā}\index{Bhagavadgita@\textsl{Bhagavadgītā}} or the {\sl Brahmasūtra}-s.\index{Brahmasutras@\textsl{Brahma-sūtra}-s} That is one of the reasons why it has led to so many concepts and interpretations different from the intention of {\sl śāstra}. Hence it has been mandatory to learn the {\sl śāstra}-s from a guru who is grounded in the tradition, who may be described as an\break ``insider'' (Malhotra 2016:24-28). Some of the very important features of {\sl śāstra} are discussed in this section according to the tradition of Advaita Vedānta (since Pollock quotes {\sl Śaṅkarabhāṣya} in a few places), but I believe these details are acceptable to other Vedāntic traditions by and large.\\[-20pt]

\subsection{Definition \& Scope}\label{art12-sec2.1}

Etymologically, the word {\sl Śāstra} is derived in two ways ``{\sl śaṁsanāt śāstram}'' and ``{\sl śāsanāt śāstram}'', which indicate the {\sl pāramārthika}\index{paramarthika} (transcendental) and {\sl vyāvahārika}\index{vyavaharika@\textsl{vyāvahārika}} (transactional) relevance of {\sl śāstra} respectively\endnote{The essence is captured beautifully by the Kannada words ``{\sl kāṇme}'' and ``{\sl jāṇme}'', corresponding to the Sanskrit words {\sl darśana} and {\sl vidhi-niṣedha} (Ganesh 2008:xv).}. In its restricted sense, {\sl śāstra} refers to the Veda ({\sl literally}, `knowledge'), which primarily reveals those facts which cannot otherwise be known; and the Veda has no role in revealing what can be known by other available means (\S\ref{art12-sec2.2}). The subject matter of the Veda is: the nature of the {\sl jīva}\index{jiva@\textsl{jīva}} (individual), reason for a particular birth, what happens after death, what means can be adopted while living to reach favorable ends after death etc. These are termed {\sl dharmādharma-vyavasthā} (cosmic moral order), {\sl gati} (travel after death), {\sl loka} (other worlds) etc. Therefore, unless one has total {\sl śraddhā}\index{sraddha@\textsl{śraddhā}} in the words of the Veda, one cannot commit oneself wholeheartedly to understand and live this vision. 

The first part of the Veda, called Karma-kāṇḍa reveals relation between  {\sl sādhana} and {\sl sādhya} (means and ends) and is put in the form of injunctions\index{injunctions} and prohibitions\index{prohibitions} ({\sl vidhi} and {\sl niṣedha}). Knowledge of this part of Veda, called Aparā-vidyā ({\sl literally}, Lower Knowledge) serves to fulfill the desires of man - for a better body, better {\sl loka} \& better possessions. This is achieved {\sl via}
\begin{itemize}
\itemsep=0pt
\item[$\bullet$] {\sl yajña}\index{yajna@\textsl{yajña}} - performance of the sacred rites with a reverential attitude 
\item[$\bullet$] {\sl dāna} - sharing one's possessions\index{dana@\textsl{dāna}}
\item[$\bullet$] {\sl tapas}\index{tapas@\textsl{tapas}} - austerities\endnote{{\sl yajña-dāna-tapaḥ-karma na tyājyaṁ kāryam eva tat} |\newline 
{\sl yajño dānaṁ tapaś caiva pāvanāni manīṣiṇām} || ({\sl Bhagavadgītā} 18.5)} 
\end{itemize}
This brings about a tremendous change in the doer: committed to such a lifestyle, he stands to gain freedom from pride, arrogance, violent reactions, hoarding, selfishness, pettiness etc., which in turn gives a mind, subtle enough for (or prepared to be open to) the second part of the Veda called Jñāna-kāṇḍa or Vedānta, where there is only revelation of a fact already existing and self-evident. In this right seeing, called Parā-vidyā\index{Para-vidya@\textit{Parā-vidyā}} ({\sl literally}, Higher Knowledge), the burden of do’s \& don't's have no more relevance. Without Aparā-vidyā,\index{Apara-vidya@Aparā-vidyā} Parā-vidyā is impossible and without Parā-vidyā, Aparā-vidyā is incomplete. Use of the `faculty of choice' (unique to man) to gain and live this vision is the highest contribution one can make to oneself and to the entire creation at a very fundamental level.\\[-20pt]

\subsection{{{\sl\bfseries Pramāṇa-śāstra}\relax} or Epistemology}\label{art12-sec2.2}
\index{epistemology}

That by which the existence of a thing is proved is called {\sl pramāṇa}. In other words, any knowledge is valid only if it is born out of a {\sl pramāṇa}. The definition of a {\sl pramāṇa}\index{pramana}  is {\sl anadhigata-abādhita-arthāvabodhaka. Avabodhaka} indicates the revealing nature of a {\sl pramāṇa}. The other three words are explained thus:

{{\sl\bfseries Anadhigata}\relax} \index{anadhigata@\textsl{anadhigata}} -- in the absence of a given {\sl pramāṇa}, one cannot even suspect ignorance of the corresponding object. For example, a blind person cannot even suspect his ignorance of color or form; nor does a person unexposed to the Veda suspect his ignorance of {\sl dharmādharma-vyavasthā} and Brahman.

{{\sl\bfseries Abādhita}\relax}\index{abadhita@abādhita} -- one {\sl pramāṇa} can neither verify nor negate the knowledge acquired from another. For example, what the eye reveals (forms and colors) can neither be verified nor negated by the ear, the nose etc. Similarly, what the Veda reveals can never be subjected to verification by empirical sciences;\index{science!empirical} the latter are based on data gathered through the five senses, or their extensions (telescope/microscope etc.), and inference based on such data. This is in contrast with modern science\index{science!verification in modern} where a huge premium is placed on verification.

{{\sl\bfseries Arthavat}\relax}\index{arthavat@\textsl{arthavat}} -- what is revealed by every {\sl pramāṇa} has some utility. 

The following six {\sl pramāṇa}-s are accepted in varying degrees in the different {\sl darśana}-s in keeping with their disposition and reasoning (Satprakashananda 2005):
\begin{itemize}
\itemsep=0pt
\item[(a)] {\sl Pratyakṣa}:\index{pramana@\textsl{pramāṇa}!pratyaksa@\textsl{pratyakṣa}}\index{pratyaksa@\textsl{pratyakṣa}} Direct Perception (via the five senses)
\item[(b)] {\sl Anumāna}:\index{pramana@\textsl{pramāṇa}!anumana@\textsl{anumāna}}\index{anumana@\textsl{anumāna}} Inference
\item[(c)] {\sl Upamāna}:\index{pramana@\textsl{pramāṇa}!upamana@\textsl{upamāna}}\index{upamana@\textsl{upamāna}} Example
\item[(d)] {\sl Arthāpatti}:\index{pramana@\textsl{pramāṇa}!arthapatti@\textsl{arthāpatti}}\index{arthapatti@\textsl{arthāpatti}} Two step inference (usually of the form `unless-otherwise')
\item[(e)] {\sl Anupalabdhi}:\index{pramana@\textsl{pramāṇa}!arthapatti@\textsl{anupalabdhi}}\index{arthapatti@\textsl{arthāpatti}} Non-perception
\item[(f)] {\sl Śabda}:\index{pramana@\textsl{pramāṇa}!sabda@\textsl{śabda}}\index{sabda@\textsl{śabda}} Verbal testimony
\end{itemize}

In the cognition of any object, the instrument of knowledge is called {\sl pramāṇa}, the object is called {\sl prameya}, the cognizer is called {\sl pramātṛ}. When a {\sl prameya} is not amenable to {\sl pratyakṣa}, it is amenable to either {\sl parokṣa}\index{paroksa@\textsl{parokṣa}} (remote) or available as {\sl aparokṣa}\index{aparoksa@\textsl{aparokṣa}} (of the nature of {\sl pramātṛ}). {\sl Parokṣa} itself can be either {\sl laukika parokṣa} (which can potentially become available to {\sl pratyakṣa})\index{pramana@\textsl{pramāṇa}!pratyaksa@\textsl{pratyakṣa}}\index{pratyaksa@\textsl{pratyakṣa}} or {\sl nitya-parokṣa} (eternally remote, can never become open to {\sl pratyakṣa}) such as {\sl dharmādharma-vyavasthā, gati}, other {\sl loka}-s etc. (\S\ref{art12-sec2.1}). The role of {\sl śabda-pramāṇa} or the Veda is in revealing items which are {\sl nitya-parokṣa} and {\sl aparokṣa}.

When a person knows a tree, he also knows that he has the knowledge of the tree. How does he know this? Advaita-vedānta says this knowledge is {\sl svataḥ-siddha} (self-existent) and {\sl svataḥ-prakāśa} (self-effulgent) which illumines all thoughts and objects (Satprakashananda 2005:110-112). Such a self-existent, self-effulgent reality is called `Brahman' in the Upaniṣad-s -- ``{\sl prajñānaṁ brahma}'' ({\sl Aitareya Upaniṣad}\index{Upanisad@Upaniṣad!Aitareya@\textsl{Aitareya}}\index{Aitareyaupanisad@\textsl{Aitareya Upaniṣad}} 3.1.3), which obtains as the Self in all and is the real {\sl pramāṇa} (proof or evidence); since it alone reveals {\sl sukha} and {\sl duḥkha} (pleasure and pain), the states of dream, sleep etc., while the {\sl pramāṇa}-s listed above are confined to objects in the waking state only. Nonetheless, all the {\sl pramāṇa}-s indirectly intimate the Self alone -- just as the cognition of all objects in a room intimate the presence of light--without which no knowledge can arise. It is for this reason that all branches of knowledge are revered in Indian culture, and are spoken of as divine in origin. As long as a knowledge born out of {\sl pra\-mā\-ṇa}-s does not resolve into the Self-evident, there is no end to seeking its verification.

The subject matter of the Upaniṣad is so subtle that the Upaniṣad itself says that neither word nor thought can reveal it (\S\ref{art12-sec2.4}), but yet it is possible to understand (``{\sl ānandaṁ brahmaṇo vidvān}'', {\sl Taittirīya Upaniṣad}\index{Upanisad@Upaniṣad!Taittiriya@\textsl{Taittirīya}}\index{Taittiriyaupanisad@\textsl{Taittirīya Upaniṣad}} 2.9.1). This is not a contradiction but a paradox, which the {\sl śāstra} itself resolves by resorting to a unique {\sl prakriyā} called {\sl adhyāropa-apavāda} (superimposition-negation) to communicate that which cannot be objectified by words ``{\sl adhyāropāpavādābhyāṁ niṣprapañcaṁ prapañcyate}''\endnote{This is quoted by Śaṅkarācārya as `{\sl sampradāya-vidāṁ vacanam}' (quotation extant in the tradition) in his {\sl bhāṣya} on ``{\sl sarvataḥ pāṇi-pādam ... sarvam āvṛtya tiṣṭhati}'' {\sl Bhagavadgītā} 13.13.}. 

At a superficial glance, the statements belonging to the genres of\break {\sl adhyāropa} and {\sl apavāda} appear contradictory and might lead to confusion about the {\sl tātparya} or {\sl vivakṣā}, the intended meaning. Therefore, one has to carefully understand each statement in its context and also know the vision of the entire {\sl śāstra}, hence the need for a {\sl śrotriya-guru} (traditional master).\\[-20pt]

\subsection{Vision about creation}\label{art12-sec2.3}

As the dictum of the Sāṅkhya-s\index{Sankhya} viz. ``{\sl bhogāpavargārthaṁ sṛṣṭiḥ}''\endnote{``{\sl pradhānaṁ hi puruṣasyātmano bhogāpavargau kurvad upakaroti}'' ({\sl Śaṅkara Bhāṣya} on {\sl Brahma Sūtra}\index{Brahmasutras@\textsl{Brahma-sūtra}-s} 1.1.6)} expounds, ``creation'' is for the sake of the {\sl jīva} for the two-fold purpose of 
\begin{itemize}
\itemsep=0pt
\item[(a)] {\sl bhoga} -- (in the form of or towards the end of {\sl karma-kṣaya}/experiences of pain and pleasure) and 

\item[(b)] {\sl apavarga} (emancipation). 
\end{itemize}
But at the same time, since no single {\sl jīva} is responsible for the creation, there must be an Intelligence which has given rise to this magnificent, enormous and orderly creation; it is posited as Īśvara -- a case of {\sl cetana-kārya-vāda} (since Īśvara cannot be inert) (Cf. ``{\sl janmādy asya yataḥ}'' {\sl Brahma Sūtra}\index{Brahmasutras@\textsl{Brahma-sūtra}-s} 1.1.2)\endnote{While {\sl Sāṅkhya-darśana} says that creation is due to modification of {\sl Prakṛti} in the mere presence of {\sl Puruṣa}, Advaita Vedānta says that Brahman is both the {\sl upādāna-kāraṇa} (material cause) and {\sl nimitta-kāraṇa} of the Universe.}. This is in stark contrast with modern science,\index{science!modern} where the purpose of creation is not known, and creation is said to have come about from inert matter. 

Secondly, just as a pot cannot be brought into manifestation by a potter unless it is potential (unmanifest) in clay; similarly creation is unmanifest alone brought to manifestation. In other words, no new object is ever created, and whatever comes into existence is that which is already in existence in an unmanifest, but potential, form. So, essentially creation is really a manifestation of names and forms, while dissolution is unmanifestation; and this cycle of creation and dissolution (manifestation and unmanifestation) proceeds according to the inviolable law of {\sl karman}. This view, which is the ultimate expression of the Law of Conservation, is called {\sl Satkārya-vāda},\index{Satkaryavada@\textsl{Satkāryavāda}} and is held by the Sāṅkhya and Vedānta\index{Vedanta} {\sl darśana}-s.\\[-20pt]

\subsection{{{\sl\bfseries Śruti-Smṛti}\relax} demarcation}\label{art12-sec2.4}
 \index{Sruti-Smrti demarcation@Śruti-Smṛti demarcation}

Brahman, says the Upaniṣad, cannot be objectified by word or thought ``{\sl yato vāco nivartante aprāpya manasā saha ...}'' ({\sl Taittirīya Upaniṣad}\index{Upanisad@Upaniṣad!Taittiriya@\textsl{Taittirīya}} 2.9.1); but is very much the reason for the functioning of the speech and mind ``... {\sl manaso mano yad vāco ha vācam} ...'' ({\sl Kena Upaniṣad}\index{Upanisad@Upaniṣad!Kena@\textsl{Kena}}\index{Kenaupanisad@\textsl{Kena Upaniṣad}} 1.2); and therefore is {\sl apauruṣeya-vastu}.\index{apauruseya@\textsl{apauruṣeya}} That {\sl śāstra} which reveals this {\sl apauruṣeya-vastu} is called {\sl apauruṣeya-śāstra}, the same as the {\sl Śruti} or the Veda in the tradition. It consists of the eternal truths discovered or ``seen'' by {\sl ṛṣi}-s ({\sl mantra-dṛaṣṭṛ}-s) about {\sl dharmādharma-vyavasthā, gati, loka} etc.\ as discussed earlier (\S\ref{art12-sec2.1}). Works authored by sages by applying the revelations of {\sl Śruti} to contemporary society are called {\sl Smṛti}-s. {\sl Smṛti}-s are subservient to {\sl Śruti} since every author comes with his/her own interpretations and opinions. This distinction of {\sl Śruti}  and {\sl Smṛti}, of the separation the eternal principles and time-space dependent social norms, is unique to Hinduism; and it is this feature that enables Hinduism to constantly adapt to changing circumstances and be a living tradition continuously. Over time, unfortunately, this clarity of understanding was lost, and as a consequence, {\sl śraddhā}, the sacred feeling attached to this wisdom, was also lost; and blind, mechanical ritualistic beliefs stand to represent the most exalted tradition, giving place to contradictory opinions about {\sl śāstra}\endnote{Swami Vivekananda, addressing the people of Calcutta, says --- ``... what the Bible is to the Christians, what the Koran is to the Mohammedans, what the Tripitaka is to the Buddhist, what the Zend Avesta is to the Parsees, these Upanishads are to us. These and nothing but these are our scriptures. The Purânas, the Tantras, and all the other books, even the Vyasa-Sutras, are of secondary, tertiary authority, but primary are the Vedas. Manu,\index{Manu} and the Puranas, and all the other books are to be taken so far as they agree with the authority of the Upanishads, and when they disagree they are to be rejected without mercy. This we ought to remember always, but unfortunately for India, at the present time we have forgotten it. A petty village custom seems now the real authority and not the teaching of the Upanishads. A petty idea current in a wayside village in Bengal seems to have the authority of the Vedas, and even something better. And that word ``orthodox'', how wonderful its influence! To the villager, the following of every little bit of the Karma Kanda is the very height of ``orthodoxy'', and one who does not do it is told, ``Go away, you are no more a Hindu.'' So there are, most unfortunately in my motherland, persons who will take up one of these Tantras and say, that the practice of this Tantra is to be obeyed; he who does not do so is no more orthodox in his views. Therefore it is better for us to remember that in the Upanishads is the primary authority, even the Grihya and Shrauta Sutras are subordinate to the authority of the Vedas....'' (Vivekananda\index{Vivekananda, Swami} 2007:268-269).}.

%\subsection{{{\sl\bfseries Prakriyā}\relax} -- methodology}\label{art12-sec2.5}
% \index{prakriya}

%\subsubsection{{{\sl\bfseries Adhyāropa-apavāda}\relax} (deliberate superimposition and subsequent negation)}\label{art12-sec2.5.1}

%\subsubsection{Ontology -- {{\sl\bfseries Śruti, Yukti}\relax} and {{\sl\bfseries Anubhava}\relax}}
%\index{ontology} 

%A marked difference between ontology in Indian traditions and that of the West is that, the Indian tradition takes into account all the three states of experience -- waking, dream \& sleep -- into consideration; whereas Western Science and philosophies examine only the waking state predominantly. Three realities are accepted in Advaita-vedānta viz.,
%\begin{itemize}
%\item[(a)] {{\sl\bfseries Prātibhāsika-satya}\relax}\index{pratibhasika-satya}  available only for perception but lacks material existence -- like mirage-water, bending of stick partially immersed in water (due to refraction) etc.

%\item[(b)] {{\sl\bfseries Vyāvahārika-satya}\relax} \index{vyavaharika-satya}\index{realities, types of} the material world that obtains in the waking state and is available for transaction.

%\newpage

%\item[(c)] {{\sl\bfseries Pāramārthika-satya}\relax}\index{paramarthika-satya} that which is unnegatable in the three periods of time (past-present-future), and the ground in which waking, dream and sleep are all available.
%\end{itemize}

%The above three {\sl satya}-s with respect to a {\sl jīva} are his individuality, his body-mind-sense complex and Brahman (the essential nature) respectively. There are differences between {\sl darśana}-s on the nature and existence of {\sl pāramārthika-satya}, however all of them agree that the ``reality'' has to be in accordance with {\sl Śruti} (Veda), {\sl yukti} (logic) and {\sl anubhava} (experience). (``{\sl anubhavānusārī śrutiḥ.śrutyanusārī yuktiḥ}.'')

%~
%\subsubsection{{{\sl\bfseries Adhikaraṇa}\relax}-method}
%\index{adhikarana}

%To resolve apparent contradictions in any work where the author is not around, or for which there is no single author; the tradition has come up with the {\sl adhikaraṇa} method. In the case of the Veda-s, Jaimini and Bādarāyaṇa have composed their {\sl sūtra}-s keeping this methodology in mind which is adhered to by the commentators also; the salient features comprising the following:
%\begin{itemize}
%\item[(a)] {{\sl\bfseries Viṣaya}\relax} subject matter to be communicated

%\item[(b)] {{\sl\bfseries Viśaya}\relax} possible doubts

%\item[(c)] {{\sl\bfseries Pūrvapakṣa}\relax}\index{purvapaksa} presentation of an opposing viewpoint (or objection)

%\item[(d)] {{\sl\bfseries Siddhānta}\relax}\index{siddhanta} reconciliation \& presentation of the intended meaning of the {\sl śāstra}

%\item[(e)] {{\sl\bfseries Saṅgati}\relax} connection between a topic and the next 
%\end{itemize}

%The feature that is most relevant in the context of this paper is {\sl pūrvapakṣa. Pūrvapakṣa} is an understanding or presentation of the opposing viewpoint objectively using its own terminology backed by its own reasoning. Once this is accomplished, the loopholes or limitations of pūrvapakṣa become clear in the light of the {\sl siddhanta} and a wrong understanding is resolved. For the {\sl pūrvapakṣa-siddhānta} dialogue to be fruitful, both parties must base their arguments on a methodology and a set of {\sl pramāṇa}-s (\$\ref{art12-sec2.2}) agreed upon by both. 

%\subsubsection{{\sl\bfseries Anubandha-catuṣṭaya}}\label{art12-sec2.5.4}
%\index{anubandha-catustaya}

%{\sl Śāstra} is intended for the ultimate well-being ({\sl parama-hita}) of humans. The fourfold pre-requisites of every {\sl śāstra} are:
%\begin{itemize}
%\item[(a)] {{\sl\bfseries Prayojana}\relax} primary requirement on the part of {\sl śāstra} (and {\sl upadeśa}) is to point out the utility of the study at hand

%\item[(b)] {{\sl\bfseries Adhikārin}\relax} qualifications of the seeker.

%\item[(c)] {{\sl\bfseries Viṣaya}\relax} subject matter

%\item[(d)] {{\sl\bfseries Sambandha}\relax} connection between {\sl viṣaya} and prayojana\endnote{These pre-requisites for {\sl Advaita Vedānta} are presented in Paramānanda Bhāratī \hbox{2014:2-5}.}
%\end{itemize}

%It is very important to recognize the role of {\sl sambandha}--In one type of {\sl śāstra}, the knowledge gained has to be followed by action in order to achieve the desired end. For example, in the case of {\sl pākaśāstra} (treatise on cookery), after knowing a recipe, appropriate materials, utensils have to be procured and procedure followed till the dish is ready and finally has be eaten for satisfaction of hunger. The procedure to be followed ({\sl vidhi-niṣedha}) is a necessity for the desired end, and not to be looked upon as authoritative injunction. On the other hand, in the case of Vedānta, acquiring Self-knowledge alone is sufficient for {\sl mokṣa}, without needing anything else; since the end to be gained is not away from the seeker and the cause of his suffering was his mistaken identity. Therefore, Self-knowledge alone is the means and the end to see things as they are.  ``{\sl tarati śokam ātmavit}'' ({\sl Chāndogya Upaniṣad} 7.1.3)

%In Vedānta, the qualifications of an {\sl adhikārin} are mentioned as {\sl mumukṣutva, vairāgya, viveka} and {\sl śama-dama-uparati-titikṣā-śraddhā-samādhāna}. {\sl Śraddhā} is often translated as `faith' or `belief'; but the correct translation in keeping with the spirit of the tradition is `openness'; an openness to the possibility of an intelligence not confined to mind/person. Juluri presents this succinctly ``Hinduism to me is more than a religion, and even more than a way of life. To me, Hinduism is way of knowing life, intelligently.'' (Juluri 2015:6)

\vskip -1.3cm

\section{Pollock's views}\label{art12-sec3}

Professor Sheldon Pollock is a prominent American Indologist who speaks of a new outlook called `Political Philology'\index{misinterpretation!techniques of!political philology}\index{philology!political}\endnote{Pollock says -- ``Culture and power are two sides of the same coin, and it is the task of a critical philology\index{philology!critical} to read both -- not just the texts of literature but the texts of the political, too.'' (Pollock 2008:58); and he is disturbed that D.\ D.\ Kosambi's quest for a political philology has had not a single successor in India.}, according to which, the sacred texts of a civilization under study are examined from a political lens, where one set of people oppress another, pretty much in line with the famous `class-struggle', propounded by Karl Marx. 

His views on {\sl śāstra} v/s {\sl prayoga}, division of {\sl śāstra} into {\sl pauruṣeya} and {\sl apauruṣeya} role of reason as against {\sl śāstra}, progress and regress up are discussed below.

%Unlike many academicians who are satisfied with publishing papers and books, Pollock strives to bring about a positive change in society worldwide in general, and in marginalized sections of Indian society in particular; an exercise which he calls `liberating philology'\index{philology!liberating} (Pollock 2015).

%Elsewhere in his literature, Pollock accepts that all narratives are equally valid (Naik 2016) --the insider's narrative is biased towards a sympathetic view of his civilization, while the outsider's narrative is biased towards a critical view of the same. He then argues that his own bias is better since it releases the marginalized sections from oppression and sets them free.

\subsection{Pollock's Paper on {{\sl\bfseries Śāstra}\relax}}\label{art12-sec3.1}

Pollock starts on the note that ``{\sl śāstra} is one of the most fundamental features and problems of Indian civilization in general and of Indian intellectual history\index{intellectual history!sastra as a problem of@{\sl śāstra} as a problem of} in particular'' (Pollock 1985:499); and goes on to say that he will examine the idea and nature of {\sl śāstra} which has not been the object of sustained Indological scrutiny. Theory, he says, has been held to precede and govern practice, which is diametrically opposite to that in the West. The implication has been the idea that all knowledge is pre-existent and therefore progress consists in a regressive re-appropriation of the past. The eternality of the Veda is used to naturalize and de-historicize cultural practices -- two components in a larger discourse of power (Pollock 1985:499).\index{misinterpretation!techniques of!theory and practice}

\subsection{Naipaul's complaint about Indian art}\label{art12-sec3.2}

At the outset, Pollock observes a trend of strict {\sl śāstric} codification across the entire cultural spectrum of India leading him to concur with V. S. Naipaul's complaint that the art\index{art} of India was ``limited by the civilization, by an idea of the world in which men were born only to obey the rules.'' But this quote of Naipaul\index{Naipaul, V.S.} actually appears in his review of the book {\sl Imperial Mughal\index{Mughal!painting} Painting} by Stuart Cary Welch,\index{Welch, Stuart Cary} in the following passage:
\begin{myquote}
``...He has chosen forty pictures to illustrate the glory, limitations, and decline of Mughal court painting. An art that developed so fast, had Persia, India, and Europe to draw on...why didn't it do more? Why did this art, so human in the beginning, in the late sixteenth century, and so full of possibility, exhaust itself so quickly by the end of the seventeenth?

The answer can be inferred from Imperial Mughal Painting.\index{Mughal!painting} The art was limited by the civilization, by an idea of the world in which men were born only to obey the rules - Islam was always in the wings, waiting to resimplify and stifle. The art was limited by the despotism that went with this idea, the despotism that dealt only in power and glory but could create no nation. The art was limited by the ignorance and absurd conceit of a court dazzled by its own glitter. It shows in the paintings...'' (Naipaul 1979:10)
\end{myquote}

In this case, Pollock seems to be falling into the trap he is accusing the {\sl śāstra}-s of committing -- ignoring historicity, cultural change and socio-political conditions; thereby {\sl quoting out of context}.\index{misinterpretation!techniques of!error/out-of-context quotation} It is rather strange that Pollock does not find any other observation of Naipaul worthy of quote, although Naipaul had reviewed at least four other books on Indian art in the very same article.

\subsection{Cultural grammars}\label{art12-sec3.3}
\index{grammar!cultural@- cultural}

Pollock states that it is clearer to him that ``everywhere civilization as a whole...is constrained by rules of varying strictness, and indeed, may be accurately described by an accounting of such rules.'' (Pollock 1985:499); and goes on to compare the regulation of behavior between the {\sl Manu-smṛti}\index{Manusmrti@\textsl{Manu-smṛti}} and Amy Vanderbilt's\index{Vanderbilt, Amy}\index{Everyday Etiquette@\textsl{Everyday Etiquette}} in the case of one person meeting another. He further goes on to say that while in the West, such codes of etiquette have largely remained `tacit' knowledge existing on the level of practical and not discursive awareness; they were textualized in India, at an early date and had ``to be learned rather than assimilated by a natural process of cultural osmosis.'' (Pollock 1985:500) 

While there may be some truth in the above analysis, it has more to do with a misunderstanding of {\sl śāstra} as pointed out in \S\ref{art12-sec2.4}. The text of the {\sl Manu-smṛti}\index{Manusmrti@\textsl{Manu-smṛti}} consists of four major divisions ---
\begin{itemize}
\itemsep=1pt
\item[(i)] {\sl sṛṣṭi-prakriyā} (origin of the world)

\item[(ii)] sources of {\sl dharma}

\item[(iii)] {\sl varṇāśrama-dharma} ({\sl dharma} of the social classes and stages of life)

\item[(iv)] {\sl karma-niyati}\index{karma@\textsl{karma}} and {\sl mokṣa}\index{moksa@\textit{mokṣa}} (law of {\sl karma} and liberation).\endnote{Olivelle 2004, pp. 8--9. Justice M. Rama Jois lists the topics of each of the twelve chapters of the {\sl Manu-smṛti}\index{Manusmrti@\textsl{Manu-smṛti}} (Jois 2004:28-29).}
\end{itemize}

The fourfold sources of {\sl dharma}\index{dharma} are presented by the {\sl Manu-smṛti}\index{Manusmrti@\textsl{Manu-smṛti}} as -- Veda (or {\sl Śruti}), {\sl Smṛti, sadācāra} (conduct of the noble people) and {\sl ātmanas-tuṣṭi} (satisfaction of one's conscience) ({\sl Manu-smṛti} 2.6 and 2.12). Pollock seems to suggest that because these \hbox{{\sl dharma-śāstra}-s} are influenced by the paradigm deriving from strict regulation of ritual action in Vedic ceremonies, these grammars were invested with massive authority ensuring a nearly unchallengeable claim to normative control of cultural practices. In fact, the view of {\sl Manu-smṛti}\index{Manusmrti@\textsl{Manu-smṛti}} itself is quite the opposite; that when social conditions change, the {\sl Manu-smṛti} can be and has to be given up in favor of a better {\sl smṛti}. Justice M. Rama Jois\index{Jois, Justice M. Rama} states that
\begin{myquote}
``Yajnavalkya follows the same pattern as in Manu with slight modifications. On matters such as women's rights of inheritance and right to hold property, status of Sudras, and criminal penalty, Yajnavalkya is more liberal than Manu...''\hfill (Jois 2004:31-32)
\end{myquote}
and suggests that the liberal nature of {\sl Yājñavalkya-smṛti} may have been influenced by Buddhism in ancient India; and perhaps that is the reason it includes prescriptions on the organization of monasteries, land grants, execution of deeds and other matters. (Jois 2004:31-32)

In the context of a {\sl vānaprasthin}, the {\sl Manu-smṛti}\index{Manusmrti@\textsl{Manu-smṛti}} states that he must gradually give up all attachments and being freed from opposites, remain in his true nature (viz.\ Brahman) ({\sl Manu-smṛti} 6.81); and ends on the note that one who sees the Self in all through understanding, attains to the highest bliss, which has always been his true nature called Brahman ({\sl Manu-smṛti}\index{Manusmrti} 12.125) (perfectly in line with the {\sl Upaniṣad}-s). Therefore, while the {\sl Manu-smṛti} does prescribe detailed regulation in certain seemingly trivial matters (as Pollock presents the case of a meeting between two people); it is important to note that all the salient features of the Vedic vision are very well presented. It is a classic example, where the broad over-arching vision is kept in mind even while discussing the minutest detail.

\subsection{{{\sl\bfseries Śāstra}\relax}-s -- Definition \& Scope}\label{art12-sec3.4}
\index{sastra@\textsl{śāstra}!definition and scope}

Pollock states that although the word {\sl śāstra} has occurred in ancient texts starting from the {\sl Ṛgveda}, no comprehensive definition is offered till the medieval period! He then encounters two definitions of {\sl śāstra} -- ``authoritative rule'' and ``philosophical system''. Pollock seems to `struggle' to reconcile the above two definitions of {\sl śāstra}, and rather hastily concludes that the aspect of ``regulating'' or ``codifying'' is found to be prominent quoting Kumārila-bhaṭṭa's\index{Kumarilabhatta@Kumārila Bhaṭṭa}   {\sl Śloka-vārtika}. But this two-fold nature of {\sl śāstra} of revealing the ``vision'' and describing the ``means'' to realize it is very clear in the tradition (\S\ref{art12-sec2.1}) and is also succinctly captured in the {\sl Muṇḍakopaniṣad}\index{Upanisad@Upaniṣad!Mundaka@\textsl{Muṇḍaka}} and {\sl Bhagavad-gītā}\index{Bhagavadgita@Bhagavadgītā} respectively as: 
\begin{quote}
``{{\sl parīkṣya lokān karmacitān}}\\
{\sl brāhmaṇo nirvedam āyāt}\\
{\sl nāstyakṛtaḥ kṛtena} |\\
{\sl tad-vijñānārthaṁ...}'' ({\sl Muṇḍaka Upaniṣad} 1.2.12)\\
``{\sl ārurukṣor muner yogaṁ karma kāraṇam ucyate} |\\
{\sl yogārūḍhasya tasyaiva śamaḥ kāraṇam ucyate} ||'' 

\hfill ({\sl Bhagavad Gītā} 6.3)
\end{quote}

Then the {\sl vidyā-sthāna}-s\index{vidya@\textsl{vidyā}} are presented by Pollock in line with Rājaśekhara's\index{Rajasekhara} taxonomy, as follows ---

\newpage

{\sl Apauruṣeya}\index{apauruseya@\textsl{apauruṣeya}}\\[-20pt]
\begin{itemize}
\topsep=0pt
\itemsep=0pt
\item[$\bullet$] 4 {\sl Vedas -- Ṛg, Yajus, Sāma, Atharva}

\item[$\bullet$] 6 {\sl Vedāṅga}-s -- {\sl Śikṣā, Kalpa, Vyākaraṇa,\index{Vyakarana@\textsl{Vyākaraṇa}} Jyautiṣa, Chandas, Nirukta}

\item[$\bullet$] 4 {\sl Upavedas} -- {\sl Āyurveda,\index{Ayurveda@\textsl{Āyurveda}} Dhanur-veda, Gāndharva-veda, Sthāpatya-veda}\\[-17pt]
\end{itemize}
{\sl Pauruṣeya}\index{pauruseya@\textsl{pauruṣeya}}\\[-20pt]
\begin{itemize}
\topsep=0pt
\itemsep=0pt
\item[$\bullet$] {\sl Purāṇa and Itihāsa}

\item[$\bullet$] {\sl Ānvīkṣikī}\index{anviksiki@\textsl{ānvīkṣikī}} -- logic \& philosophy

\item[$\bullet$] ({\sl Karma} and {\sl Brahma}) {\sl Mīmāṁsā}

\item[$\bullet$] {\sl Smṛti -- Tantra -- Dharma-śāstra}-s\\[-17pt]
\end{itemize}
{\sl Others}\\[-20pt]
\begin{itemize}
\topsep=0pt
\itemsep=0pt
\item[$\bullet$] {\sl Vārttā} -- commerce

\item[$\bullet$] {\sl Daṇḍanīti} -- law

\item[$\bullet$] Erotology, Art, Architecture
\end{itemize}

In spite of noting that Rājaśekhara makes a distinction between {\sl apauruṣeya} and {\sl pauruṣeya śāstra}-s, Pollock again refers to the {\sl Śloka-vārtika}: ``{\sl śāstra} is that which teaches people what they should and should not do. It does this by means of eternal [words] or those made [by men]'' (Pollock 1985:501) and concludes that it undermines the dichotomy between {\sl pauruṣeya} and {\sl apauruṣeya} and claims that Madhusūdana Sarasvatī's \index{Madhusudana Sarasvati@Madhusūdana Saravatī} {\sl Prasthāna-bheda} too abandons the said distinction. 

It is quite well known in the tradition that Pūrva-mīmāṁsā gives importance to Vedic ceremonies. As Jaimini puts it, ``{\sl āmnāyasya kriyārthatvād ānarthakyam atadarthanām}'' ({\sl Jaiminī\index{Jaimini} Sūtra} 1.2.1) (``The purpose of the Veda being in the enjoining of actions, those parts of the Veda which do not serve that purpose are useless'' (Jha 1933:51)). It is therefore very clear that {\sl Śloka-vārtika} would not pay much heed to the ``vision'' aspect of {\sl śāstra} since there is no place for injunctions there. It has been shown how these injunctions too are not really commandments, but a revelation of means for satisfying one's desires (\S\ref{art12-sec2.1}). While it is true that {\sl Prasthāna-bheda}\index{Prasthanabheda@\textsl{Prasthāna-bheda}} does not explicitly mention {\sl pauruṣeya}\index{pauruseya@\textsl{pauruṣeya}} and {\sl apauruṣeya},\index{apauruseya@\textsl{apauruṣeya}} a clear distinction is made between {\sl śāstra}-s which are authored and those which are not; with a list of authors and {\sl śāstra}-s composed by them. 

Pollock very simplistically translates {\sl pauruṣeya} and {\sl apauruṣeya}\index{apauruseya}  as human and transcendent respectively. The word ``transcendent'' does not capture the meaning of {\sl apauruṣeya}, since according to the Veda (especially the Upaniṣad-s), the {\sl apauruṣeya-vastu} (not available for perception and inference) called Brahman is both transcendent and immanent, and is not away from any particular experience, hence {\sl aparokṣa}.\index{aparoksa@\textsl{aparokṣa}}

This is in stark contrast with the Semitic idea of a transcendent God completely removed from the world (\S\ref{art12-sec2.2}, \ref{art12-sec2.4}). 

Returning to the definition of {\sl śāstra} as ``philosophical system'', Pollock observes that there must be a possibility of `false {\sl śāstra}-s'\index{sastra@\textsl{śāstra}} or ``{\sl asac-chāstra}-s'' and complains that while {\sl Manu-smṛti}\index{Manusmrti@\textsl{Manu-smṛti}} talks about it, Rājaśekhara's system cannot accommodate it. In addition, he says Mīmāṁsā too supports Rājaśekhara's system and invests {\sl śāstra}-s with inerrancy and paramount authority. (Pollock 1985:503)

The above statement of Mīmāṁsā supporting inerrancy of {\sl śāstra} is unfounded since Kumārila-bhaṭṭa's\index{Kumarilabhatta@\textsl{Kumārila Bhaṭṭa}} view on {\sl verbal} ({\sl scriptural}) {\sl cognition} is presented thus --
\begin{myquote}
“...the Vedas are not the work of a Personal Author; and being thus free from any defects due to such authorship, the Vedas must be regarded as the only source of knowledge (relating to {\sl Dharma}), which is infallible in its Self-Sufficient Validity.''\hfill (Jha 1964:153)
\end{myquote}

Moreover, Rājaśekhara does not claim to propound a philosophical system; his intention is to present an analysis of literature; and drawing far-reaching conclusions based on his taxonomy is tantamount to putting words in his mouth. Therefore, the further claim that all {\sl śāstra}-s enjoy the unique qualities of inerrancy and paramount authority does not hold water.\\[-20pt]

\subsubsection{Change in character of {{\sl\bfseries Śāstra}\relax} from description to prescription}\label{art12-sec3.4.1}

\vskip -5pt

Pollock observes that the oldest {\sl śāstra}-s, namely the \hbox{Vedāṅga-s};\index{Vedanga} especially Chandas (prosody),\index{prosody} Vyākaraṇa\index{Vyakarana@\textsl{Vyākaraṇa}} (grammar)\index{grammar} and Śikṣā (phonetics),\index{phonetics} started out as descriptive in character in helping to understand and assimilate the Vedic mantras. But, somehow over time, they changed their character to becoming prescriptive where future use of the language came to be governed by the above three {\sl śāstra}-s.

While it is not completely untrue that composers in Sanskrit adhered to Chandas, Vyākaraṇa and Śikṣā; it was largely because of the Sanskrit language being rich in vocabulary with hundreds of synonyms even for water, fire etc. and a multitude of meters to choose from. Although it is argued that strict adherence to {\sl vyākaraṇa} prevented the language from evolving freely, on the other hand Sanskrit works composed from the time of Pāṇini onward are perfectly intelligible today; unlike the vernaculars where, study of works about a thousand years old is not straightforward. In addition, grammar\index{grammar} fixed the framework of Sanskrit, but also provided for the creation of new words. There are examples of Sanskrit words ({\sl gṛha}) that got into Prakrit ({\sl geha}) and were borrowed back into Sanskrit (Deshpande\index{Deshpande, Madhav M.} 1993:74). In pre-modern India, since a majority of compositions were committed to memory, they were either in the form of cryptic pithy statements called {\sl sūtra}-s,\index{sutra} or metrical in nature called {\sl śloka}-s; and since knowledge was transferred from {\sl guru} to {\sl śiṣya}, any possible misunderstanding would be taken care of.\\[-22pt]

\subsection{Generalizing that theory is superior to practice}\label{art12-sec3.5}

\vskip -.1cm

Pollock quotes the statement of the {\sl Upaniṣad}\index{Upanisad@Upaniṣad!Chandogya@\textsl{Chāndogya}} (``{\sl yad eva vidyayā karoti śraddhayā upaniṣadā tad eva vīryavattaraṁ bhavati}'' {\sl Chāndogya Upaniṣad} 1.1.10) that practice with {\sl vidyā} (knowledge of Om),\index{Om} {\sl śraddhā} (faith) and {\sl upaniṣad} (meditation) is more efficacious than mere practice without knowledge; and goes on to generalize that the intention of the \hbox{{\sl śāstra}-s} is to say that theory is superior to practice, citing examples from Vyākaraṇa, Āyurveda,\index{Ayurveda@Āyurveda} {\sl Arthaśāstra} and {\sl Kāmasūtra}.\index{Kamasutra@\textsl{Kāma-sūtra}}

The commentary on this mantra by Śaṅkarācārya\index{Sankara/Sankaracarya@\textsl{Śaṅkara/Śaṅkarācārya}} goes thus --

\begin{myquote}
``The Knowledge of the syllable `Om' being the essence of the `Essences', its being endowed with the qualities of fulfillment and prosperity does not consist merely in Knowledge of that syllable being a factor of the sacrifice: it is much more than that ... In ordinary life also, it is found that when a merchant and a forester sell pieces of ruby and other gems, the former (who knows the real character of the gems) always obtains a higher price than the	latter (who is ignorant); and this is due to the superior Knowledge possessed by the merchant... The assertion that the Act of the man with Knowledge is `more effective' than that of the ignorant man means that even when done by the ignorant person, the act is effective; so that it does not mean that the ignorant man is not fit to perform the act. In fact in the section dealing with {\sl Uṣasti} (later on) it is described that even ignorant persons have performed the priestly functions.

The act of meditating upon `Om' as the `essence of essences', as `fulfillment' and as `prosperity' forms a single act of `meditation' (and worship); as there are no efforts intervening in between these...'' 

\hfill (Jha\index{Jha, Ganganatha} 1942:14-15)
\end{myquote}

The last statement forms the basis of the inclusiveness of `Hinduism', where all people are accepted as they are and an appropriate {\sl upāya} or {\sl sādhana} is presented for their gradual evolution. In a related context, the {\sl Bṛhadāraṇyaka Upaniṣad}\index{Upanisad@Upaniṣad!Brihadaranyaka@\textsl{Bṛhadāraṇyaka}}\index{Brhadaranyakaupanisad@\textsl{Bṛhadāraṇyaka Upaniṣad}} states even more explicitly --
\begin{quote}
``{\sl atha trayo vāva lokā - manuṣya-lokaḥ pitṛ-loko deva-loka iti...
karmaṇā pitṛ-loko, vidyayā deva-loko, deva-loko vai lokānāṁ śreṣṭhas - tasmād vidyāṁ praśaṁsanti}.'' 

\hfill({\sl Bṛhadāraṇyaka Upaniṣad}\index{Upanisad@Upaniṣad!Brihadaranyaka@\textsl{Bṛhadāraṇyaka}} 1.5.16)
\end{quote}
that the three {\sl loka}-s viz.\ {\sl manuṣya-loka, pitṛ-loka}, and {\sl deva-loka}--are gained by {\sl putra} (begetting and taking care of progeny), {\sl karma}\index{karma@\textit{karma}} (ritualistic action) and {\sl vidyā}\index{vidya@\textsl{vidyā}} (meditation) respectively; and since {\sl deva-loka} is the highest among the three, the means of attaining it viz.\ {\sl vidyā} is praised. The meaning of the word {\sl vidyā} in this context is, therefore, to be taken as {\sl upāsanā} or meditation, since mere knowledge cannot produce attainment of a {\sl loka} as its result.

Talking about the dialectical orientation prevalent in the medical tradition, Pollock says

\vskip .1cm

\begin{myquote}
``... As is well known, the {\sl Suśrutasaṁhitā}\index{Susrutasamhita@\textsl{Suśrutasaṁhitā}} urges the student to examine cadavers carefully and subject them to anatomical scrutiny, ``For it is only by combining both direct observation and the information of the {\sl śāstra} that thorough knowledge is obtained.'' The {\sl Carakasaṁhita}\index{Carakasamhita@\textsl{Carakasaṁhita}} confirms this view, ... proclaiming that ``The wise understand that their best teacher is the very world around them.''

\vskip .1cm

Such voices ... may with some justice be viewed as oppositional, and in any case are pretty much in the minority. The dominant ideology is that which ascribes clear priority and absolute competence to shastric codification...''\hfill (Pollock 1985:510)
\end{myquote}

All {\sl darśana}-s accept {\sl pratyakṣa}\index{pramana!pratyaksa@\textsl{pratyakṣa}}\index{pramana@\textsl{pramāṇa}!pratyaksa@\textsl{pratyakṣa}}\index{pratyaksa@\textsl{pratyakṣa}} as a valid {\sl pramāṇa}, and hence saying that validity given to ``directly observed evidence'' is a minority or oppositional view is totally wrong. In fact, it  would be very interesting to examine the Buddhist and Jaina {\sl śāstra}-s\index{sastra@\textsl{śāstra}} since they do not accept {\sl śabda} as {\sl pramāṇa}; but this is conspicuously absent from this paper.

The word `theory' is etymologically derived from {\sl theōros} (Greek), meaning spectator; but in English, it has acquired the meaning of `a supposition or a system of ideas intended to explain something'. But, {\sl śāstra},\index{sastra@\textsl{śāstra}!as applied science} as discussed in \S\ref{art12-sec2.1}, has a two-fold meaning -- one, revealing facts as they are; and the other, a manual for practical application. In other words, one part of {\sl śāstra} reveals the nature of existence as it is; and another part reveals the connection between a known action and a desired result\endnote{Every {\sl jīva}, more so a human being, has the two-fold capabilities of `knowing' and `doing'. `Knowing' is about the nature of existence and `doing' is to bring about a desired change. Every `doing' involves some amount of `knowing', and every process of `knowing' involves some amount of `doing'. The relevance of the {\sl śāstra} here is therefore in showing `what is to be known' and ‘which action needs to be done in what manner'. This understanding is thanks to Dr.~Ramā Phaṇirāj.}. Based on the etymological derivation viz.\ ``{\sl śāsanāt śāstram}'', {\sl śāstra} is said to be closer in meaning to `applied science'\index{science!applied}\endnote{This opinion was communicated by Śrīramaṇa Śarmā, a disciple of Śrī Maṇidraviḍa Śāstrī. It is also available on YouTube here - \url{https://youtu.be/o3d-q30zK9U?t=119} accessed on 30th October 2017.}. Therefore, the usage of the words ‘theory’ and ‘practice’ in the context of {\sl śāstra} is too simplistic and in most occasions horribly wrong. 

In summary, the thesis that ``theory is superior to practice'' rests on lack of clarity in the meanings of words used, such as {\sl vidyā, śāstra},\index{sastra@\textsl{śāstra}!theory and practice@`theory and practice'} theory etc. (or is quoted out of context); and the omission of a discussion on epistemology\index{epistemology} (the very basis for validity of {\sl śāstra}).\\[-18pt]

\subsection{Extrapolation of texts by {{\sl\bfseries Śiṣṭa}\relax}-s}\label{art12-sec3.6}

In the case of Dharma-śāstra-s,\index{Dharmasastra@Dharmaśāstra} in a situation when the {\sl śruti} and {\sl smṛti} do not provide an explicit conclusion, one must look to {\sl śiṣṭācāra. Baudhāyana-dharmasūtra} defines the {\sl śiṣṭa}-s\index{sista@\textsl{śiṣṭa}} as devoid of jealousy, ego and other vices; and as those well-versed in {\sl dharma} who can extrapolate {\sl śāstra} to matters it does not explicitly cover\endnote{``{\sl śiṣṭāḥ khalu vigata-matsarā nirahaṅkārāḥ kumbhī-dhānyā alolupā dambha-darpa-lobha-moha-krodha-vivarjitāḥ.}\newline
... {\sl śiṣṭās-tad-anumāna-jñāḥ śruti-pratyakṣa-hetavaḥ}.'' ({\sl Baudhāyana Dharmasūtra}\index{Baudhayana Dharmasutra@\textsl{Baudhāyana Dharmasūtra}} 1.1.5-6)}. Based on {\sl Śābara-bhāṣya} on the {\sl Jaimini-sūtra}-s -- ``{\sl dharmasya śabda-mūlatvād aśabdam anapekṣaṁ syāt}''  and ``{\sl api vā kartṛ-sāmānyāt pramāṇam anumānaṁ syāt}'' ({\sl Jaiminī Sūtra} 1.3.1-2), Pollock states that ---

\begin{myquote}
``... the ``learned'' do not creatively reason from and extend {\sl śāstra} to illuminate obscure areas of social or moral conduct; on the contrary, their behavior derives directly from and fully conforms with the texts as codified, but these texts are ones to which we no longer have access.'' 

\hfill (Pollock 1985:506)
\end{myquote}

\newpage

But the {\sl Sūtra} and {\sl Śābara-bhāṣya}\index{Sabarabhasya@\textsl{Śābara-bhāṣya}} actually state --
\begin{myquote}
[{\sl Sūtra} 1.3.1]: ({\sl Pūrvapakṣa})\index{purvapaksa@\textsl{pūrvapakṣa}} ``Inasmuch as dharma is based upon the Veda, what is not Veda should be disregarded.'' (Jha 1933:87) \index{Jha, Ganganatha}

[{\sl Sūtra} 1.3.2]: ({\sl Siddhānta}) ``But ({\sl Smṛti}) is trustworthy, as there would be inference (assumption, of the basis in the Veda) from the fact of the agent being the same.'' 

[{\sl Bhāṣya}]: ``Even if they do not actually find it (a Vedic text as basis for the {\sl Smṛti} in question), they would infer it. It is quite possible also that the text upon which the {\sl Smṛti} is based was actually known to the {\sl Smṛti}-writer, but has since been forgotten.'' (Jha 1933:89)
\end{myquote}

The response given in this {\sl sūtra} and {\sl bhāṣya} to the objection presented in the previous {\sl sūtra} - that `no non-Vedic text ({\sl smṛti}) is valid,' - is more in support of inference -- in line with what Bodhāyana\index{Bodhayana} has said---than what is suggested by Pollock here.\\[-18pt]

\subsection{Role of reasoning being condemned in {{\sl\bfseries Śāstra}\relax}}\label{art12-sec3.7}

Pollock quotes from the {\sl Bhagavadgītā}\index{Bhagavadgita} and Abhinavagupta's {\sl saṅ\-graha-śloka} as ---
\begin{myquote}
``Whoever abandons the injunctive rules of {\sl śāstra} and proceeds according to his own will never achieve success, or happiness, or final beatitude...”

\hfill (Pollock 1985:510)

``...one must never contemplate action according to one's own lights, but must instead follow shastric injunction...'' (Pollock 1985:510)
\end{myquote}
with the original {\sl śloka}-s being ---
\begin{quote}
``{{\sl yaḥ śāstra-vidhim utsṛjya vartate kāma-kārataḥ}} |\\ 
{\sl na sa siddhim avāpnoti na sukhaṁ na parāṁ gatim} ||\\
{\sl tasmāc chāstraṁ pramāṇaṁ te kāryākārya-vyavasthitau} |'' 

\hfill ({\sl Bhagavad Gītā} 16.23-24)

``{\sl abodhe svātma-buddhyaiva kāryaṁ naiva vicārayet} |\\
{\sl kintu śāstrokta-vidhinā śāstraṁ bodha-vivardhanam} ||'' 

\hfill (Raina 1933:162)
\end{quote}

The above {\sl Gītā śloka} is said in this context: under the spell of ignorance, when one is overcome by {\sl kāma} (passion), {\sl krodha} (anger) and {\sl lobha} (greed) -- beautifully summed up by Abhinavagupta in one word, {\sl abodha} (non-understanding) -- {\sl śāstra} must be the guiding light in determining what to do and what not to do. 

Here is a case of {\sl quoting out of context} by Pollock\index{misinterpretation!techniques of!error/out-of-context quotation} to suit his `theory'.

\subsection{Implications of Pre-existence of Theory}\label{art12-sec3.8}

\subsubsection{Creation of knowledge as divine activity}\label{art12-sec3.8.1}
\index{knowledge!creation} 

Having ``established'' that theory pre-exists practice in pre-modern India, Pollock goes on to talk about the implications of such an idea. Pollock states that practically all {\sl śāstra}-s, starting from the {\sl Kāma-sūtra}\index{Kamasutra@\textsl{Kāma-sūtra}} to Purāṇa to {\sl Māna-sāra} (an early text on architecture and town-planning) to Āyurveda\index{Ayurveda@Āyurveda} - all trace their origins \index{sastra@\textsl{śāstra}!origins of} to either Śiva, Viṣṇu or Brahmā. Thus, the origin of knowledge is traced through an unbroken succession of teachers, {\sl guru-śiṣya-paramparā}, as an unabridged, complete transmission of the divine prototype -- or by sudden revelation. Based on these observations, Pollock concludes that ---

\begin{myquote}
``... First, the ``creation'' of knowledge is presented as an exclusively divine activity, and occupies a structural cosmological position suggestive of the creation of the material universe as a whole ... knowledge ... is frozen for all time in a given set of texts that are continually made available to human beings ... If any sort of amelioration\index{amelioration} is to occur, this  can only be in the form of a ``regress,'' a backward movement aiming at a closer and more  faithful approximation to the divine pattern ...  According to his own self-representation, there can be for the thinker no originality of thought, no brand-new insights, notions, perceptions, but only the attempt better and more clearly to grasp and explain the antecedent, always already formulated truth. All Indian learning, accordingly, perceives itself and indeed presents itself largely as commentary on the primordial {\sl śāstra}-s.''\hfill \hbox{(Pollock 1985:515)}
\end{myquote}

Although Pollock is right in pointing out that virtually all \hbox{{\sl śāstra}-s} present themselves as of divine origin;\index{sastra@\textsl{śāstra}!divine origins}\index{divine origin} he does not state that the activities pertaining to sculpture,\index{sculpture} dance,\index{dance} music,\index{music} architecture,\index{architecture} astronomy\index{astronomy} and even metallurgy\index{metallurgy} were ultimately God-centered. 

Dance and music are filled with themes concerned with the divine. Sculpture and architecture were used in the building of huge temple complexes with intricate carvings and construction of complicated {\sl yajña-vedi}-s. Astronomy was predominantly used for fixing the calendar in order to determine the auspicious {\sl muhūrta} for certain ceremonies, besides agriculture and sea-voyage; and one of the greatest achievements of metallurgy being the rust-free Iron pillars\index{Iron pillar} at Delhi and Koḍacādri (near Kollur) are located in temple complexes at Mehrauli and {\sl Mūla-Mūkāmbikā} temple respectively\endnote{Another colossal rust-free Iron pillar can be found today in the premises of {\sl Lāṭ} Masjid at Dhar (Madhya Pradesh); the masjid is said to have been a temple earlier.}. 

The reason for the reverential attitude towards knowledge and the transcendence\index{transcendence} and immanence of ``God'' who is {\sl jñāna-svarūpa} is pointed out in \S\ref{art12-sec2.2}.\\[-23pt]

\subsubsection{Denial of Discovery \& Innovation}\label{art12-sec3.8.2}

\vskip -.2cm

Pollock quotes Matilal's\index{Matilal, Bimal Krishna} translation of Jayantabhaṭṭa's\index{Jayantabhatta} {\sl Nyāyamañjarī} as
\begin{myquote}
``How can we {\sl discover} any new fact or truth? One should consider novelty only in rephrasing the older truths of the ancients in modern terminology.''\hfill (Pollock 1985:515) ({\sl italics ours})
\end{myquote}

But Matilal's actual translation of the same is

\begin{myquote}
``How can we {\sl discuss} any new fact or truth? One should consider novelty only in rephrasing the older truths of the ancient in modern terminology.''\hfill (Matilal 1977:93) ({\sl italics ours})
\end{myquote}

This is an {\sl error in quotation} \index{misinterpretation!techniques of!error/out-of-context quotation} and not even in translation. And based on this, Pollock goes on to say ---
\begin{myquote}
``... if in certain areas the shastric paradigm did encourage or enforce a certain stasis (as in language and literature), elsewhere Indian cultural history in the classical and medieval period is crowded with exciting discovery and innovation (as in mathematics and architecture). These are not, however, perceived to be such; they are instead viewed, through the inverting lens of ideology, as renovation and recovery...'' 

\hfill (Pollock 1985:515)
\end{myquote}


{\sl Prasthāna-bheda} of Madhusūdana Saravatī\index{Madhusudana Sarasvati@Madhusūdana Saravatī} refers to the preeminent composers of the texts of Vyākaraṇa,\index{Vyakarana@\textsl{Vyākaraṇa}} Nirukta, etc. as Bhagavān Pāṇini,\index{Panini} Bhagavān Yāska\index{Yaska@Yāska} etc. respectively. If indeed the insiders viewed contributions to various fields of knowledge as mere recovery, how can one explain the high pedestal on which the above contributors are placed. 

In fact, Matilal mentions the humility of Jayanta-bhaṭṭa in the face of an important and novel contribution as the {\sl Nyāya-mañjarī}. Similarly, it is very interesting to note in this context that Āryabhaṭa\index{Aryabhata} in his  {\sl Āryabhaṭīya} states that the Earth is a sphere ({\sl Āryabhaṭīya} 4.6 {\sl bhūgolaḥ sarvato vṛttaḥ}) and that the Earth goes round the Sun, rather casually, without any trace of claiming that he has discovered a great truth; but when the heliocentric theory\index{heliocentric theory} is proposed by Copernicus a thousand years later, it is Copernicus\index{Copernicus} that claims it as a very important truth newly discovered by him. 

Similarly, Sāyaṇācārya\index{Sayanacarya@Sāyaṇācārya} simply states, addressing the Sun - it is well known that you travel at the speed of so many {\sl yojana}-s per {\sl nimeṣa} ({\sl Sāyaṇa-bhāṣya} on {\sl Ṛgveda} 1.9.50.4). But the discovery of  the speed of light was given great importance in the West as starting from the days of Galileo.\index{Galileo} 

This difference needs to be examined more carefully rather than merely being written off as renovation or recovery of knowledge. 

It seems that in the light of {\sl śāstra} - which brings the attention of\break the human mind towards the most intimate and immediate reality\break ({\sl aparokṣa}),\index{aparoksa@\textsl{aparokṣa}} which is timeless and of the nature of knowledge -- such humility is but a natural consequence.\\[-25pt]

\subsubsection{{{\sl\bfseries Śāstra}\relax} -- Practical Discourse of Power}\label{art12-sec3.8.3}

Pollock  says
\begin{myquote}
``... all contradiction between the model of cultural knowledge and actual cultural change is thereby at once transmuted and denied; creation is really re-creation, as the  future is, in a sense, the past. Second, the living, social, historical, contingent tradition is naturalized, becoming as much a part of the order of things as the laws of nature  themselves... And finally, through such denial of contradiction and reification of tradition, the sectional interests of pre-modern India are universalized and valorized. The theoretical discourse of {\sl śāstra} becomes in essence a practical discourse of power. ...''\hfill (Pollock 1985:516)
\end{myquote}

\vskip .1cm

Any close observer of Indian history will note how the practices of fire rituals, symbolism,\index{symbolism} temples, {\sl pīṭha}-s and {\sl maṭha}-s, etc. have evolved and are continuing to evolve; and in many cases one has given way, largely, to the succeeding. India has always been known as a land of diverse customs, traditions etc.; and hence the statement that sectional interests were universalized does not hold water. 

Furthermore, Indian society has always been constantly churning with activity -- whenever an intellectual challenged/criticized the Veda, another intellectual would stand up to defend the Vedic vision. For example, when Buddhism\index{Buddhism} and Jainism\index{Jainism} challenged the Veda, \hbox{Vaiśeṣika-s},\index{Vaisesika@Vaiśeṣika} Pūrvamīmāṁsaka-s and Vedantin-s responded; and later the Bhakti movement flourished, and various socio-religious reform movements happened, during and after the British era.

\vskip .1cm

The idea that `knowledge is power' gives a status to the ego, whereas the tradition has always defined ``knowledge'' as that which liberates (``{\sl sā vidyā\index{vidya@vidyā} yā vimuktaye}'' {\sl Viṣṇu Purāṇa}.1.19.41) {\sl i.e}.\ it frees one from a sense of inadequacy, helps one to be simple, and hold a reverential outlook towards life. In the understanding that {\sl śāstra} is a revealer, meant for the ultimate well-being of the {\sl jīva} - \index{jiva@\textsl{jīva}} who has been empowered in the sense in which Pollock speaks of it? On the other hand, it is the one who works towards gaining the vision and lives the vision is truly ``empowered'' i.e., fulfilled ({\sl kṛtārtha}).\\[-22pt]

\subsection{The Critical Presupposition}\label{art12-sec3.9}


In the last section of the paper, Pollock sets out to find a justification for the ``priority of {\sl śāstra} to all and every practical application and activity''. He quotes Chattopadhyaya's\index{Chattopadhyaya, Debiprasad} translation of the {\sl Caraka-saṁhitā}\index{Carakasamhita@\textsl{Caraka-saṁhitā}} (see Chattopadhyaya 2014:78, a later reprint) ---
\begin{myquote}
``Of these three ways of knowing, the starting point is the knowledge derived from authoritative instruction.\index{authoritative instruction} At the next step, it has to be critically examined by perception and inference. Without there being some knowledge obtained from authoritative instruction, what is there for one to examine critically by perception and inference?'' 

\hfill (Pollock 1985:516-7)	
\end{myquote}

This is again a case of {\sl selective quotation},\index{misinterpretation!techniques of!selective quotation} while the actual text from the same page under the topic of ``The extension of sense-knowledge: Diagnosis'' says 
\begin{myquote}
``...\,The ancient doctors are not running after empty metaphysical postulates. Their main theoretical drive is determined by imposing amount of empirical data. 

Such data, it is true, are compiled mainly by unaided sense-organs. But this again is not to be misunderstood. In spite of being inevitably dependent on unaided sense-organs, the physicians also feel the need of extending their knowledge beyond the limits of bare sense-perceptions.

...\,The diagnosis of a disease is faultless only after the disease has been fully examined in all its aspects with the help of these three ways of knowing. The full knowledge of an object cannot be obtained by only one of these ways of knowing.

...\,For the learned, therefore, there are two modes of critical examination, viz. perception and inference. Or, if authoritative instruction\index{authoritative instruction} also is included, the modes of critical examination are three....'' 

\hfill (Chattopadhyaya 2014:77-8)
\end{myquote}
This places placing more importance on direct perception and inference than what Pollock suggests. He further quotes Chattopadhyaya (Chattopadhyaya 2014:160, a later reprint) {\sl selectively}\index{misinterpretation!techniques of!selective quotation} again: 
\begin{myquote}
``Ayurveda\index{Ayurveda@Āyurveda} is called eternal, because it is without beginning, because it is nothing but the laws inherent in nature and because the natural properties of the real substances are unalterable... Apart from the restricted sense of acquiring this knowledge and of spreading it, there is no meaning in saying that medical science\index{science!medical} came into being having been non-existent before.''\hfill (Pollock 1985:517)
\end{myquote}
It has however been made clear on the same page that the intended meaning is

\begin{myquote}
``...Āyurveda--in the sense of a body of natural laws -- is beginningless... Medical science can be said to have a beginning only from the standpoint of acquiring the knowledge of these laws or of spreading the knowledge...Diseases are cured not by any artificial technique of which the doctors are the inventors. These are cured by the laws inherent in nature, which the doctors can only know and rightly apply...'' 

\hfill (Chattopadhyaya 2014:160)
\end{myquote}

The intended meaning of Āyurveda\index{Ayurveda@Āyurveda} is a body of laws, whereas Pollock takes it as a text; and further uses it as a brick to construct his thesis. 

Pollock then restates the following Western philosophical arguments:
\begin{itemize}
\itemsep=1pt
\item[(a)] {\bf Meno's paradox} \index{Meno’s paradox} ``A man cannot inquire either about that which he knows (no need), or about that which he does not know (not possible)''.

\item[(b)] {\bf Socratic\index{Socrates} merging}\index{Socrates} of {\sl mathesis-anamnesis} where the source of knowledge is the psyche itself (and not text) since the soul is immortal and takes multiple births in different worlds and therefore ``all learning is really recollection''.

\newpage

\item[(c)] {\bf Popper's neopositivism}\index{Popper, K R} ``All acquired knowledge, all learning, consists of the modification (possibly the rejection) of some form of knowledge, or disposition, which was there previously...''
\end{itemize}
Pollock continues:
\begin{myquote}
``Whatever the cogency of these more philosophical explanations for the special character attributed to {\sl śāstra}, a historical-cultural consideration seems to me somewhat more persuasive ... the peculiar traits {\sl śāstra} is invested with in the classical period are easily related to ... widespread belief in the transcendent character of ... the Vedas.'' (Pollock 1985:518)
\end{myquote}

In fact, all the above philosophical arguments have been accounted for in the {\sl śāstra}-s themselves, respectively as---
\begin{itemize}
\itemsep=1pt
\item[(a)] knowledge arises as the {\bf connection between the known and the }\index{knowledge!connection of known and unknown} (knowable) {\bf unknown}. For example, attainment of {\sl svarga} by offering milk in fire ({\sl jyotiṣṭoma}).

\item[(b)] {\sl saṁskāra}-s of {\sl jīva}-s are preserved across {\sl janman}-s (births), and manifest themselves when suitable situations arise in accordance with {\sl karma-niyati}

\item[(c)] error, presented as {\sl adhyāsa} (wrong super-imposition) in {\sl Yoga-\break sūtra}\index{Yogasutra@\textsl{Yoga-sūtra}} and {\sl Adhyāsa-bhāṣya}\index{Adhyasabhasya@\textsl{Adhyāsabhāṣya}}
\end{itemize}

Pollock does not seem interested in examining the cogency of these arguments. He mentions only (a) in a footnote; and wants to use the ``transcendent'' character attributed to the Veda to further his ``theory–practice'' thesis with the Veda itself as `theory' and ``cosmic creation'' as `practice' which proceeds according to Veda.\index{Veda} Again, the Veda is looked upon as a text, and hence his wrong conclusions. In this context of creation proceeding from [the Vedic] word, there is a huge discussion considering the following contrary statements from within the Veda itself ---
\begin{itemize}
\itemsep=1pt
\item[$\bullet$] ``that, from which beings originate, through which they live, and in which they re-enter after death, explore that [because] that is Brahman'' (Deussen\index{Deussen, P} 1986:241-6)

\item[$\bullet$] ``Whence all creation had its origin; 

he, whether he fashioned it or whether he did not;

\newpage

he, who surveys it all from highest heaven; 

he knows - or maybe even he does not know.'' 

\hfill (Basham\index{Basham, A L} 1954:250)
\end{itemize}
However, Pollock seems to be content with {\bf selectively quoting}\index{misinterpretation!techniques of!selective quotation} those portions which suit his thesis. Finally, Pollock concludes his paper quoting Manu's\index{Manu} position about the status given to the Veda - to suit his own position ---
\begin{myquote}
``Secular {\sl śāstra} in general, consequently, as a portion of this (Vedic) corpus (and were it not, it would be ``worthless and false,'' as Manu says, ``being of modern date''), comes to share the Veda's transcendent attributes. Just like the Veda, too, it thereby establishes itself as an essential {\sl a priori} of every dimension of practical activity ...'' 

\hfill (Pollock 1985:519)
\end{myquote}

It is important to note that the Veda does not talk of a sacred-secular divide; and {\sl Manu-smṛti}\index{Manusmrti@\textsl{Manu-smṛti}} is only pointing out that those doctrines contrary to the vision of the Veda are `worthless and false'. No other {\sl śāstra} -- secular or otherwise -- gets a status like the Veda. Hence, the last statement of Pollock is unfounded and wrong.

\section{Conclusion}\label{art12-sec4}

This paper starts with the motivation to understand the history of Indian civilization in the face of new evidence, and having to explain the relative decline in the recent past. The views expressed by Pollock on {\sl śāstra}-s are shown to rely on wrong quotations, quotations out of context, carefully cherry-picking statements in {\sl śāstra}, over-generalizing to suit his materialistic insinuations and glaring omissions\index{misinterpretation!techniques of!omissions} such as discussions on the {\sl Ṛṅ-mantra} ``{\sl ā no bhadrāḥ kratavo yāntu viśvataḥ}'', and the Buddhist and the Jaina {\sl śāstra}-s, for example.

Pollock tries to collapse ---
\begin{itemize}
\itemsep=1pt
\item[$\bullet$]  ``vision'' and ``regulation'' meanings of ``{\sl śāstra}'', to just ``regulation'',

\item[$\bullet$] ``{\sl pauruṣeya}''\index{pauruseya@\textsl{pauruṣeya}} and ``{\sl apauruṣeya}''\index{apauruseya@\textsl{apauruṣeya}} {\sl śāstra}-s into one category,

\item[$\bullet$] Veda and non-Vedic texts to same ``level'' of ``authority''.
\end{itemize}
--- All this in an effort to portray ancient Indian culture as relying on pre-existing authoritative injunctions, thereby denying creativity and progress!

While Pollock accepts all narratives as equally valid; such narratives are to be rejected outright which do not care to take into account the contextual meanings and the overall vision of the scriptures under examination; and further, contain the errors pointed out above. If an eye develops cataract, nobody seeks to use ears or nose to see forms and colors, but get the eye operated upon; similarly a misunderstanding of {\sl śāstra} has to be cleared by right understanding of the same using the tools of textual analysis in an objective manner; and not seek to replace it. 

Under the sway of Western universalism\index{Western!Universalism} in science\index{science!education} education, the scientist heroes of yore of various non-Western cultures\index{cultures!non-Western} have faded away from memory (Joseph\index{Joseph, George Gheverghese} 2001:1-24). 

In order to prevent further damage to indigenous cultures and overall loss of knowledge and diversity to humanity, the insiders' view must be encouraged to be studied, preserved and propagated in all cultures. 


\section*{Acknowledgements}

The author acknowledges that the understanding out of which this paper has been written is thanks to the exposition of Vedānta by Śrī Phaṇirāj Kumble, Dr. Ramā Phaṇirāj, Swāmī Paramānanda Bhāratī, Swāmī Paramasukhānanda, Swāmī Paramārthānanda Sarasvatī; and discussions with Shri Gokulmuthu Narayanaswamy, Dr.~T.~S. Mohan and Dr. Shankar Rajaraman. A special gratitude to Dr.~Ramā Phaṇirāj for taking the pains to go through the draft and suggest critical improvements and corrections. Thanks are due to Mrs.~H.~R. Meera and Mrs.~Shalini Puthiyedam for encouraging me to take up this work. Thanks are due to Vikas Veshishth for maintaining an informative blog.

\newpage

\begin{thebibliography}{99}
\itemsep=1pt
\bibitem[]{chap12_item1}
{{\sl\bfseries Aitareya Upaniṣad}}. See (Radhakrishnan 2011:523).

\bibitem[]{chap12_item2} 
Arya, Vedveer (2015). {\sl The Chronology of Ancient India: Victim of Concoctions and Distortions}. Hyderabad: Aryabhata Publications.

\bibitem[]{chap12_item3}
Aṇṇaṅgarācārya (Ed.) (1972). Śrīviṣṇupurāṇam. Kanchi: Liberty Printers. 

\bibitem[]{chap12_item4}
{{\sl\bfseries Āryabhaṭīya}}. See Shukla (1976).

\bibitem[]{chap12_item5} 
Aurobindo, Sri (1994). {\sl India’s Rebirth} Paris: Institut de Recherches Volutives. (Institute for Evolutionary Research).

\bibitem[]{chap12_item6}
Basham, A L (1954). {\sl The Wonder that was India: A survey of the history and culture of the Indian sub-continent before the coming of the Muslims}. Delhi: Rupa \& Co.

\bibitem[]{chap12_item7}
Basu, B D (Ed.) and Sandal, Pandit Mohan Lal (Tr.) (1923). {\sl The Mīmāṁsā Sūtras of Jaiminī} Allahabad: Dr. Sudhindhre Nath Basu, The Panini Office.

\bibitem[]{chap12_item8}
{{\sl\bfseries Bhagavad Gītā}}. See (Radhakrishnan 2005).

\bibitem[]{chap12_item9}
{\sl\bfseries Bṛhadāraṇyaka Upaniṣad}. See (Radhakrishnan 2011).

\bibitem[]{chap12_item10}
Chattopadhyaya, Debiprasad (2014). {\sl Science and Society in Ancient India}. Kolkata: K P Bagchi \& Company (First Edition, Second Reprint).

\bibitem[]{chap12_item11}
{\sl\bfseries Chāndogya Upaniṣad}. See (Radhakrishnan 2011:339)

\bibitem[]{chap12_item12} 
Deshpande, Madhav M. (1993). {\sl Sanskrit \& Prakrit: Sociolinguistic Issues}. Delhi: Motilal Banarsidass.

\bibitem[]{chap12_item13} 
Deussen, Paul (1986). {\sl Sixty Upanishads of the Veda, Volume 1}. Delhi: Motilal Banarsidass.

\bibitem[]{chap12_item14} 
Ganesh, Shatavadhāni Dr. R. (2008). {\sl Sandhyā-darśana}. Bengaluru: Śrī Nityānanda Prakāśana.

\bibitem[]{chap12_item15} 
{{\sl\bfseries Jaiminī Sūtra}}. See Basu (1923).

\bibitem[]{chap12_item16} 
Jha, Ganganatha (Tr.) (1933). {\sl Śābara-bhāṣya (English translation)}. Baroda: Oriental Institute.

\bibitem[]{chap12_item17} 
--- (1942). {\sl The Chāndogyopanishad (A treatise on Vedānta Philosophy translated into English with The Commentary of Śankara)}. Poona: Oriental Book Agency.

\bibitem[]{chap12_item18} 
--- (1964). {\sl Pūrva-Mīmāṁsā in its Sources}. Varanasi: The Banaras Hindu University.

\bibitem[]{chap12_item19} 
Jois, M Rama (2004). {\sl Legal and Constitutional History of India}. Allahabad: Universal Law Publishing.

\bibitem[]{chap12_item20}
Joseph, George Gheverghese (2001). {\sl The Crest of the Peacock --- Non-European Roots of Mathematics}. Princeton University Press.

\bibitem[]{chap12_item21} 
Juluri, Vamsee Krishna (2015). {\sl Rearming Hinduism}. Mumbai: Westland.

\bibitem[]{chap12_item22} 
{{\sl\bfseries Kaṭha Upaniṣad}}. See (Radhakrishnan 2011).

\bibitem[]{chap12_item23} 
{{\sl\bfseries Kena Upaniṣad}}. See (Radhakrishnan 2011).

\bibitem[]{chap12_item24} 
Maddison, Angus (2001). {\sl The World Economy --- A Millenium Perspective}. The Organisation for Economic Cooperation and Development.

\bibitem[]{chap12_item25} 
Malhotra, Rajiv (2011). {\sl Breaking India}. Manjul Publishing House Pvt. Ltd.

\bibitem[]{chap12_item26}
--- (2016). {\sl The Battle for Sanskrit}. HarperCollins India.

\bibitem[]{chap12_item27}
{{\sl\bfseries Manu-smṛti.}} See Sharma (2012).

\bibitem[]{chap12_item28} 
Matilal, Bimal Krishna (1977). {\sl Nyāya-Vaiśeṣika}. Wiesbaden: Otto Harrassowitz.

\bibitem[]{chap12_item29} 
{{\sl\bfseries Muṇḍaka Upaniṣad}}. See (Radhakrishnan 2011).

\bibitem[]{chap12_item30}
Naik, Ashay (2016). ``The Peculiarity Of The Pollock Challenge''. {\sl Swarajya Magazine} 
(published on 30th March 2016) accessed on 7th December 2017.

\bibitem[]{chap12_item31}
Naipaul, V. S. (1979). {\sl New York Review of Books}. Rea S. Hederman. March 22.

\bibitem[]{chap12_item32} 
Olivelle, Patrick (2004). {\sl Manu’s Code of Law: A Critical Edition and Translation of the Mānava-Dharmaśāstra}. Oxford: Oxford University Press.

\bibitem[]{chap12_item33} 
Paramānanda-Bhāratī, Swāmī (2014). {\sl Védānta Prabodha}. Bengaluru: Jnanasamvardhani Pratishthanam.

\bibitem[]{chap12_item34} 
Pollock, Sheldon (1985). ``On the Theory of Practice and the Practice of Theory in Indian Intellectual History''. {\sl Journal of the American Oriental Society} 105.3, pp. 499--519.

\bibitem[]{chap12_item35} 
--- (2008). ``Towards a Political Philology: D D Kosambi and Sanskrit''. {\sl Economic \& Political Weekly} 43.30, pp. 52--59.

\bibitem[]{chap12_item36} 
--- (2015). ``Liberating Philology''. {\sl Verge: Studies in Global Asias} 1.1, pp. 16--21.

\bibitem[]{chap12_item37} 
Radhakrishnan, Sarvepalli (2005). {\sl The Bhagavadgita}. New Delhi: HarperCollins. First Edition, Twenty second impression.

\bibitem[]{chap12_item38} 
--- (2011). {\sl The Principal Upaniṣads}. Noida: HarperCollins. First Edition, Twenty second.

\bibitem[]{chap12_item39} 
Raina, Pandit Lakshman (1933). {\sl Srimad Bhagavad Gita With Commentary by Mahāmāheshvara Rājānaka Abhinava Gupta Edited with notes}. Srinagar: Kashmir Pratap Stem Press.

\bibitem[]{chap12_item40} 
Raju, Chandra Kant (2007). {\sl Cultural Foundations of Mathematics: The Nature of Mathematical Proof and the Transmission of the Calculus from India to Europe in the 16th cCE}. Pearson Education India.

\bibitem[]{chap12_item41} 
Roy, Dr. Raja Ram Mohan (2016). ``A new dating of Buddha based on the evidence of Sumatitantra''. {\sl India Facts Indology}.
(\url{http://indiafacts.org/a-new-dating-of-buddha-based-on-the-evidence-of-sumatitantra/}). Accessed on 30th October 2017. 

\bibitem[]{chap12_item42} 
Sanyal, Sanjeev (2008). {\sl The Indian Renaissance --- India's Rise after a Thousand Years of Decline}. Gurgaon: Penguin.

\bibitem[]{chap12_item43} 
Sarkar, Anindya, Arati Deshpande Mukherjee, M. K. Bera, B. Das, Navin Juyal, P. Morthekai, R. D. Deshpande, V. S. Shinde and L. S. Rao. (2016). ``Oxygen isotope in archaeological bioapatites from India: Implications to climate change and decline of Bronze Age Harappan civilization''. {\sl Nature Scientific Reports} 26555.6, 10.1038/srep26555 (2016).

\bibitem[]{chap12_item44} 
Satprakashananda, Swami (2005). {\sl Methods of Knowledge according to Advaita Vedanta}. Kolkata: 
Advaita Ashrama (First Edition, Fifth Indian Reprint).

\bibitem[]{chap12_item45}
Sharma, R N (Ed.) (2012).  {\sl Manusmṛti}. Varanasi: Chowkhamba Vidya Bhawan.

\bibitem[]{chap12_item46}
{{\sl\bfseries Taittirīya Upaniṣad}}. See Radhakrishnan (2011).

\bibitem[]{chap12_item47}
{{\sl\bfseries Viṣṇupurāṇa}}. Aṇṇaṅgarācārya (1972).

\bibitem[]{chap12_item48} 
Vivekananda, Swami (2007, 1897$^1$). {\sl Lectures from Colombo to Almora}. Kolkata: Advaita Ashrama.
\end{thebibliography}


\theendnotes
\label{chapter\thechapter:end}

