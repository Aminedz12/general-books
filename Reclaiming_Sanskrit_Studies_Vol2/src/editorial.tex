\chapter*{Series Editorial}\label{gen_editorial}

\lhead[\small\thepage\quad K S Kannan]{}
\rhead[]{Series Editorial\quad\small\thepage}


It is a tragedy that many among even the conscientious Hindu scholars of~Sanskrit and Hinduism\index{Hinduism} still harp on Macaulay, and ignore others while accounting for the ills of the current Indian education system, and the consequent erosion of Hindu values in the Indian psyche. Of course, the machinating Macaulay brazenly declared that a single shelf of a good European library was worth the whole native literature of India, and sought accordingly to create “a class of persons, Indian in blood and colour, but English in taste, in opinions, in morals and in intellect” by means of his education system - which the system did achieve. 

An important example of what is being ignored by most Indian scholars is the current American Orientalism\index{American Orientalism}. They have failed to counter it on any significant scale. 

It was Edward Said (1935-2003) an American professor at Columbia University who called the bluff of “the European interest in studying Eastern culture and civilization” (in his book {\sl Orientalism} (1978)) by showing it to be an inherently political interest; he laid bare the subtile, hence virulent, Eurocentric prejudice aimed at twin ends – one, justifying the European colonial aspirations and two, insidiously endeavouring to distort  and delude the intellectual objectivity of even those who could be deemed to be culturally considerate towards other civilisations. Much earlier, Dr.\ Ananda Coomaraswamy\index{Coomaraswamy, Ananda K} (1877--1947) had shown the resounding hollowness of the {\sl leitmotif} of the “White Man’s Burden.” 

But it was given to Rajiv Malhotra, a leading public intellectual in America, to expose the Western conspiracy on an unprecedented scale, unearthing the {\sl modus operandi} behind the unrelenting and unhindered program for nearly two centuries now of the sabotage of our ancient civilisation yet with hardly any note of compunction.  One has only to look into Malhotra’s seminal writings - {\sl Breaking India} (2011), {\sl Being Different} (2011), {\sl Indra’s Net} (2014), {\sl The Battle for Sanskrit} (2016), and {\sl The Academic Hinduphobia} (2016) - for fuller details.
\vskip 1.5pt

This pentad - preceded by {\sl Invading the Sacred} (2007) behind which, too, he was the main driving force - goes to show the intellectual penetration of the West, into even the remotest corners (spatial/temporal/\break thematic) of our hoary heritage. There is a mixed motive in the latest Occidental enterprise,  ostensibly being carried out with pure academic concerns. For the American Orientalist doing his ``South Asian Studies'' (his new term for “Indology Studies”), Sanskrit is inherently oppressive - especially of Dalits, Muslims and women; and as an antidote, therefore, the goal of Sanskrit studies henceforth should be, according to him, to ``exhume and exorcise the barbarism'' of social hierarchies and oppression of women happening ever since the inception of Sanskrit - which language itself came, rather, from outside India. Another important agenda is to infuse/intensify animosities between/among votaries of Sanskrit and votaries of vernacular languages in india. A significant instrument towards this end is to influence mainstream media so that the populace is constantly fed ideas inimical to the Hindu heritage. The tools being deployed for this are the trained army of “intellectuals” - of leftist leanings and “secular” credentials.
\vskip 1.5pt

Infinity Foundation (IF), the brainchild of Rajiv Malhotra, started 25 years ago in the US, spearheaded the movement of unmasking the “catholicity” (- and what a euphemistic word it is!) of Western academia. The profound insights provided by the ideas of ``Digestion''\index{digestion} and the “U-Turn Theory” propounded by him remain unparalleled.
\vskip 1.5pt

It goes without saying that it is {\sl ultimately the Hindus in India who ought to be the real caretakers of their own heritage}; and with this end in view, {\bf Infinity Foundation India (IFI)} was started in India in 2016. IFI has been holding a series of Swadeshi Indology Conferences. 
\vskip 1.5pt

Held twice a year on an average, these conferences focus on select themes and even select Indologists of the West (sometimes of even~the East), and seek to offer refutations of mischievous  and misleading misreportages/misinterpretations bounteously brought out by these Indologists - by way of either raising red flags at, or giving intellectual responses to, malfeasances inspired in fine by them. To employ Sanskrit terminology, the typical secessionist misrepresentations presented by the West are treated here as {\sl pūrva-pakṣa}, and our own responses/rebuttals/rectifications as {\sl uttara-pakṣa} or {\sl siddhānta}. 

The first two conferences focussed on the writings of Prof.\ Sheldon Pollock, the outstanding American Orientalist (also of Columbia University, ironically) and considered the most formidable and influential scholar of today. There can always be deeper/stronger responses than the ones that have been presented in these two conferences, or more insightful perspectives; future conferences, therefore, could also be open in general to papers on themes of prior conferences.
\bigskip

\noindent
Vijayadaśamī\hfill	{\bf Dr.~K S Kannan}\\
Hemalamba Saṃvatsara\hfill Academic Director\\
Date 30-09-2017\hfill and\\	
\phantom{.}~\hfill General Editor of the Series                  
               

\chapter*{Volume Editorial}\label{editorial}

\vskip 9pt
\lhead[\small\thepage\quad K S Kannan]{}
\rhead[]{Volume Editorial\quad\small\thepage}

\begin{flushright}
{{\sl “For in my sight, the villain with a crafty tongue}}\\
{\sl Incurs a most heavy retribution: he boasts that his words}\\
{\sl Will cloak deceit decently, and boldly pursues his wicked end:}\\
{\sl Yet in fact he is not very wise.”}\\
{\sl {\bfseries Medea}} (of Euripedes) 580--83, (trans.) D L Page\\[5pt]
{\sl The fashions of Western Indologists may change, but their designs remain the same.}\\[7pt]
--- {\bf K S Kannan}
\end{flushright}

It gives us pleasure to place before the public the second volume of the proceedings of the Swadeshi Indology Conferences, entitled\\   
{\sl Śāstra-s through the Lens of Western Indology -- A Response.}\\
As noted in the Volume Editorial of the first volume ({\sl Western Indology and its Quest for Power, 2017}), two Swadeshi Indology conferences were conducted by IFI (Infinity Foundation India) in 2016 July and 2017 February, which sought to examine some of the writings of Professor Sheldon Pollock of Columbia University. The first conferences had four themes, and the second one had six more. 

It goes without saying that there has been no intention of targetting one individual. It only happens that Pollock is the most formidable of the American Orientalists today, and his approaches and interpretations, pervasive as they are, are quite pernicious and inimical to Hindu heritage. He has been academically active for decades now and his sphere of influence is by no means small. There are hardly many that contest his views even in America. 
\eject

Much as Prof.~Wendy Doniger O’Flaherty is obsesse with sex, so is Pollock with power: Rajiv Malhotra, a leading public intellectual of America, notices how the word “power” recurs 600 times in Pollock’s {\sl Language of Gods...} book (2006), and the words “political” and “politics” over 900 times! As we remarked in the Volume Editorial of {\sl Western Indology and its Quest for Power}, there is perhaps no event or utterance where Pollock cannot perceive some vicious play of power.

We have planned a few more volumes in this series, and Volume 2 is being presented now. This volume deals primarily with {\sl śāstra}-s, a theme that Pollock has dealt with in some detail in more than a dozen of his articles/books (from 1985-2015) which have been looked into by the over half-a-dozen scholars who have presented papers in this volume.

A conspectus of the papers in this volume is in order here. There are seven papers in this volume.

The opening paper by {\bf Sowmya Krishnapur (Ch.~1)} analyses the claims of Pollock with respect to grammar, or rather Vyākaraṇa, in relation to political (which is to say, royal) power. What puzzles Pollock is the “widely shared, largely uniform cosmopolitan style of Sanskrit inscriptional discourse” -- reaching even up to the farthest corner of the Far East: of the Sanskrit Cosmopolis. What touches him is that the style of the inscriptions there bears enormous kinship to that of standard Sanskrit poetry. The broadcasting of the philological instruments in order for this to come about can by no means by one on any small scale. He is acutely aware that Daṇḍin’s\index{Dandin@Daṇḍin} {\sl Kāvyādarśa} was “probably the most influential work on literary science\index{science!literary} in world history after Aristotle’s\index{Aristotle} {\sl Poetics}”.\index{Poetics of Aristotle@\textsl{Poetics} (of Aristotle)} For him, it was the spread of (Sanskrit) grammar that chiefly carried cultural and political associations; nowhere else in the world, after all, was the study of language so highly developed: the roots of many basic conceptual components of Western modernity are traceable, after all, to substantive and theoretical formulations of premodern Indian linguistic thought. Grammar and rulership were for him mutually constitutive in India all through. He presents half a dozen inscriptions as “evidence” for this assertion, while, as Sowmya exposes, not one of these is commensurate with the demand. He presents puerile proofs of Pāṇini and Patañjali as patronised by kings! The “evidence” of other grammarians is also equally naive. He beams with confidence characterising his evidence as “slender but suggestive”, “rather vague data” etc. Sowmya has laid bare his sleight of tongue, while he is thus only spinning the yarn – weaving tapestries of theories out of tenuous and disjointed threads of factoids. If one wanted stately examples of equivocation and prevarication in academic papers, his papers offer them generously. The American Indologist is not just a {\sl vaitaṇḍika} but an adept at {\sl chala} and {\sl jāti} in the craft of debate. To intertwine Vyākaraṇa\index{Vyākaraṇa@\textsl{Vyākaraṇa}} -- the “{\sl ajihmā rāja-paddhati}” ({\sl after} Bhartṛhari) with {\sl parātisandhāna-vidyā} ({\sl after} Kālidāsa), requires a {\sl jihma-vṛtti} nonpareil that Pollock’s writings uniformly betray.
\vskip 8pt

The paper by {\bf Subhodeep Mukhopadhyay (Ch.~2)} is a rebuttal of Pollock’s practice {\sl v/s} theory [hypo]thesis. The West is always good at driving a wedge where there is unity, and Pollock is West at its best. He pits {\sl śāstra} against  {\sl prayoga},\index{sastra@\textit{śāstra}!and {\sl prayoga}} by first befuddling the issue by his loose translation of the two words as “theory” and “practice”; such is his practice. Pollock alludes to {\sl śāstra}-s such as {\sl kṛṣi-śāstra, gaṇita-śāstra, gandha-śāstra} etc., and indicates how they cover “virtually every activity”; for Pollock, {\sl śastra}-s are cultural grammars,\index{grammar!cultural@- cultural} and the Veda is {\sl śāstra} par excellence. He postulates that in Sanskritic culture, theory was “held always as necessarily to precede and govern practice'': that the secular life of an individual is subjected to an all-pervasive ritualization, leading to misery and entrapment of people”; and the West is eternally progressive and forward-looking as it is not dependent of {\sl śāstric} norms which are diametrically opposed to the Aristotelian approach. Subhodeep points out how Pollock uses selective observation, indulges in over-generalization, and summarily rejects all counter-evidence. Pollock conveniently overlooks the fact that Buddhist and Jain scholars made great advancements in Gaṇita despite according no special place for the Veda-s; whence there is no question of their works being treated as commentaries\index{commentaries} on the Veda-s. The {\sl Śulba-sūtra}-s, the oldest extant mathematical texts already deal with design and architectural problems in ritual. Seidenberg\index{Seidenberg, A} spoke of geometry as traceable to ritual origins. Subhodeep turns the tables against Pollock’s charge - that conformity with {\sl śāstric} norms implied no new discoveries, but a mere recovery of what was past knowledge; it is Mathematical Platonism that rules mathematics today, which has disdain for empirical evidence: a veritable case, then, of those in glass houses throwing stones at others. What is much in evidence is the relish with which the Western Indologist loves to lampoon Hindu heritage: what if fidelity to facts is the first casualty, after all?
\vskip 8pt

The paper by {\bf K Surya (Ch.~3)} on {\sl śāstra} and {\sl prayoga}\index{sastra@\textit{śāstra}!and {\sl prayoga}} as handled by Pollock, contains a rather powerful exposé of Pollockian fallacies. {\sl Śāstra}-s crippled the innovative spirit of Indian intellectuals, bewails Pollock; he iterates and reiterates in manifold manners that these intellectuals could after all do little else than just uncover {\sl knowledge already pre-existent in the śāstra}-s. Theory and practice as carried out in the West are diametrically opposed to their counterparts in the East, holds Pollock. So he rules out any role for experience, experiment, invention, discovery, innovation in traditional Indian learning.
\vskip 3pt

He exploits {\sl Satkārya-vāda} by first stretching it to illegitimate limits. Effects are inherent in causes, says the {\sl Satkārya}\index{Satkaryavada@\textsl{Satkāryavāda}} theory, but Pollock goes to argue that knowledge must be inherent in prior textual material viz. the Veda -- a logical extrapolation unheard of in the vast philosophical literature of the Hindus: (this, notwithstanding the enormity of the Vedic literature lost, already by the time of Patañjali (of the pre-Christian era), which is itself of no mean order). Neither on the basis either of the commentary of Śaṅkara, nor of the {\sl Brahma-sūtra}-s\index{Brahmasutra@\textsl{Brahma Sūtra}} Śaṅkara was commenting upon, nor of the Upaniṣadic texts that the {\sl sūtra} text was drawing upon -- is it inferible that “all one has to do is to fall back upon nothing more than prior textual materials for any derivation or enhancement of knowledge”.
\vskip 3pt

As to the idea of divine origin claimed in the great and ancient Āyurveda\index{Ayurveda@Āyurveda} texts such as {\sl Caraka-saṁhitā}\index{Carakasamhita@\textsl{Caraka-saṁhitā}} and {\sl Suśruta-saṁhitā},\index{Susrutasamhita@\textsl{Suśruta-saṁhitā}} Surya points out how even those texts have undergone further redactions/revisions. The author cites “the staggering volume and diversity of scientific literary productions” post {\sl bṛhat-trayī} : thousands of medical texts produced 600 CE onwards describing “new diseases, new theories, new treatments and new medicines” - drawing upon the writings of Wujastyk.
\vskip 3pt

The American Indologist seems to read into the Hindu context what just happened with the Christian faith - viz. that valid knowledge had perforce to issue from the Bible, {\sl the} “Book”. On the contrary, Hindu India never witnessed any Inquisition -- no Galileo incarcerated, no Bruno burnt; and no Socrates\index{Socrates} made to drink hemlock on grounds of {\sl asabeia} -- “not believing in the god of the State”. Too, there is nothing in the Hindu scriptures, after all, that can be opposed to, or be threatened by, the theories of Darwin.

Nothing in academics can be more brazen than the leading Indologist’s chicanery in quoting from Caraka\index{Caraka} blatantly selectively lopping off key chunks of even adjacent texts; it is as simple and as hypocritical as deriving “{\sl Socrates is}”\index{Socrates} from “{\sl Socrates is dead}”. As is his wont, Pollock routinely misleads by partial citations. While the text lays due stress on perception and inference, he withholds it from the readers’ view. Authoritative testimony is misinterpreted by him as a “beginningless text”, and Surya proffers citations from Sadegh-Zadeh, versed in modern medical literature - exposing thereby the relentless and warrantless diatribe of the celebrated Indologist against {\sl śastra}-s in general and the Veda-s in particular.
\vskip 8pt

The paper by {\bf Vrinda Acharya (Ch.~4)} examines the role of {\sl śāstra}-s against the background of the verdicts of Pollock viz. that {\sl śāstra}-s hinder genuine creativity, practical innovation, original thinking and progressive growth -- as “the {\sl śāstra}-s are straight-jacketed by the Vedic worldview”; that there is no dialectical interaction\index{sastra!dialectical interaction} between theory and practices; that {\sl śāstra}-s instigate authoritarianism; and that the relationship between {\sl śāstra} and {\sl prayoga}\index{sastra@\textit{śāstra}!and {\sl prayoga}} is diametrically opposed to what is found in the West.

Not claiming to be exhaustive and complete in her investigation, Vrinda seeks to show certain arguments which boomerang on Pollock, certain assertions as lacking logic and proof, certain essential aspects overlooked by him, and certain quotations of his being plain wrong or selective. Vrinda cites with approval the remark of Hanneder - that Pollock’s argumentation is often arbitrary, given his (ie Pollock's) tendency to over-interpret the evidence in support of his theory. Vrinda espies his rendering of the word {\sl dharma} as ‘rule-boundedness’ as very narrow and parochial, given the breadth of the concept. Pollock complains about the centrality of rule governance in human behaviour as set forth in {\sl Manu-smṛti};\index{Manusmrti@\textsl{Manu-smṛti}} endeavouring, all the same, to appear balanced and fair, he cites from Amy Vanderbilt’s {\sl Everyday Etiquette} which abounds in similar rules - thus himself nullifying in effect the case he seeks to make out. He says students in the West are amazed to find a whole battery of rules in {\sl Manusmṛti}. One is inspired to suspect arrogance on the part of Pollock: what if the West thinks of the East as having a very peculiar cultural grammar?\index{grammar!cultural@- cultural} Can not (or should not) the East think similarly of the West? Is it settled for him that the West is wise and all else otherwise? Branding {\sl śāstra}-s as frozen in time, Pollock himself cites several scholars who present different lists of {\sl śāstra}-s -- thus proving himself a {\sl vadato-vyāghāta}! His grouse that Indian art-making is too constrained by rules is contradicted by the fact that it is Western music which is very rigid in practice. Vrinda knows this as she is a musician herself. Is Pollock blind, then, to the limitations that Western (read American) civilization possesses?
\vskip 8pt

The paper of {\bf Rajath Vasdevamurthy (Ch.~5)} deals with the issue posited by Pollock  of {\sl śāstra} as in impediment to progress.

Rajath shows how Pollock is playing his mischievous game - quoting statements out of context. Reviewing the book {\sl Imperial Mughal Painting},\index{Mughal!painting} Naipaul remarks: “Art then was limited by the civilisation, by an idea of the world in which men were born only to obey the rules”; slyly, Pollock cites this as a statement made in connection with Indian art, setting aside the other writings of Naipaul.

Rajath also shows how, while quoting Matilal’s translation of a verse from Jayanta Bhaṭṭa’s {\sl Nyāya-mañjarī}, Pollock substitutes a word (- the key word, the very verb!) and has unscrupulously manipulated and thus misrepresented Matilal. Pollock will have to be accused of double injustice -- injustice to Matilal and injustice to fellow scholars, and to even the future readers. There are more issues in the paper that the reader can peruse by himself.
\vskip 8pt

The paper of {\bf T N Sudarshan (Ch.~6)} points to nearly a dozen flaws in the reasoning methods of Pollock. For example he refers to four types of dissonance -- epistemological, ontological, interpretive, philosophical (issues that have to do respectively with {\sl pramāṇa, tattva, mīmāṁsā} and {\sl darśana}). He proposes that the current Western theses on {\sl śāstra} derive from deep ignorance -- “a veritable nescience”, as Sudarshan would say. Interpretation of Sanskrit texts involves an inalienable {\sl saṁskāra} which the Western Indologist lacks. Pollock’s philology is flawed and nebulous. The current “scientific method” is itself inadequate in respect of the interpretation of {\sl śāstra}-s; neither does Pollock measure up to the task of their proper interpretation. The {\sl etic} modes of analysis are often errant. Fallacious assumptions, institutionalized biases and insidious methods consistently vitiate the Pollockian program.

Western Indology has not remained as bad as it used to be; it has steadily worsened and is getting the more and the more debased - being answerable first of all to the exigencies and executions of the military-political-industrial complex: when have academia in the West not been a tool of their funding agencies? Modern science\index{science!modern} is not without biases; social sciences are more often than not biases sophisticated; and the “humanities are their own god”.

What is remarked of history is truer of humanities: they usually are “hard-core interpretation surrounded by a pulp of disputable facts”. On the other hand, governed as they essentially are “by first-person empiricism” {\sl śāstra}-s are closer to the spirit of science than is current science. Critical theory\index{Critical theory} seeks to “emancipate” (i.e. in a Western sense) all non-Western societies - as the current normative is “Western Universalism” (as aptly labelled by Rajiv Malhotra). Pollock’s socio-economic analysis subscribes to a Marxist-driven\index{Marxist!framework} framework, notes Sudarshan. The power discourse that Pollock is obsessed with is influenced by post-modernistic perspectives. The new brand philology, Pollock claims, merits the same centrality among disciplines as philosophy or mathematics! Pollock’s philology is the very antithesis of scientific method as it is an {\sl anything-goes style} “discipline”. Adluri and Bagchee have exposed the earlier critical theories. The equation of {\sl śāstra} with theory is itself problematic, and the ontological relation Pollock speaks of in its connection with {\sl prayoga}\index{sastra@\textit{śāstra}!and {\sl prayoga}} is even more so, betraying mischief than confusion. The mutually influential nature of theory and practice has been well-taken care of by the welll-seasoned ancient authorities such as Kauṭalya; and Ryle or Pollock have hardly anything to add to here. In any case, and as is his wont, Pollock links {\sl śāstra}-s in general somehow to legislative control, and makes it an issue of prioritizing theory over practice. Reducing {\sl śāstra}-s to mere theory first, Pollock takes next the step of branding {\sl śāstra}-s as myths, thereby rendering their very credibility questionable. And all this exercise is spurred by the interior motive-of making the Western norms look superior.

Even the enormous secular knowledge deposited in the Purāṇa-s are made to look mythical and shown as frozen for all time, allowing therefore neither change nor growth, and so there came to be no conception of progress. All this is aimed at characterizing the Purāṇa-s as encumbered by ideological hindrances and as being against the idea of progress in the West. Championing regress further, the {\sl śāstra}-s pave the way for the universalisation and valorisation of the sectional interests of premodern India, and are in essence a practical discourse of power: so run Pollock’s convoluted concoctions.

In his response to these complicated arguments of Pollock, Sudarshan takes a primary note of certain vital factors here: of the positive role of the required {\sl saṁskāra-s}; of the negative impact of centuries of colonial rule; and of the consequent slavery and material impoverishment that have taken their own toll. The West is essentially āsuric as against the East which is essentially sāttvic: Will Durant expresses his indignation at the deliberate bleeding of India by England as {\sl the greatest crime in all of history}. What is generally not easily noted is, Sudarshan aptly notes, that the same crime continues unabated to this day though essentially on an intellectual plane.
\vskip 8pt

The last paper of {\bf Manjushree Hegde (Ch.~7)} concerns with “Project SKSEC\index{SKSEC Project} (Sanskrit Knowledge Systems on the Eve of Colonialism)”,\index{colonialism} the international collaborative project of Pollock. The project seeks to draw a picture of the “death of Sanskrit” in the face of European modernity. The paper holds a mirror to Pollock's predetermined conjectures and conclusions. “Navya” (as in Navya Nyāya) is a precious discovery for him, and he loves to put it into some use somewhere, and what can be better than concocting a projectile in India synchronising with the Renaissance in Europe?; after all, {\sl navya} and {\sl renaissance} both convey the idea of the new? And what better opportunity to kill two birds with one stone (much better, indeed, than the {\sl niṣāda} of the {\sl Rāmāyaṇa}\index{Ramayana@\textsl{Rāmāyaṇa}} who killed ({\sl jaghāna vaira-nilayaḥ - Rāmāyaṇa} 1.2.10) but one curlew with one arrow): claiming that India repudiated {\sl navya}, and that the fillip for this came from within. And add a third “thesis” to cap it all: the creative upsurge in Indian science\index{science!Indian} happened during Muslim rule.

Manjushree calls the SKSEC enterprise a “shockingly mischievous endeavour”. Putting up a show of opening a conversation, this ambitious international collaborative project {\sl works within} a fixed framework of predetermined conjecture, and {\sl works towards} a set of predetermined conclusions. While he speaks of a pre-Colonial homomorphism between the East and the West, she unearths the depths of deviousness in the tall claims made of the very Renaissance. He sees an explosion of scholarly production beginning in the (pre-colonial) 16th century, but she produces a statistical analysis (based on sources such as the celebrated work of Karl Potter), to show that literary production in numerous disciplines has continued to flourish well beyond that date and in full measure and vigour - belying thence Pollock's fancy claims. Contrary to his contention, the Navya Nyāya language easily and soon became the technical language coming in handy as a medium for all serious philosophical discussions - essentially owing, as she well points out, to its ontological neutrality. Not a shred of proof has been, or can be, offered to show the pervasion of this mathematical format to Persian\index{Persian!influence} influence. Energetically sweeping facts away with his discriminatory broom, the American Indologist pretends ignorance of the Hindu polymaths -- Abhinavagupta/Hemacandra/Vedānta Deśika {\sl et al}. Manjushree illuminates his ignorance through bar charts chronicling the abundance of multidisciplinary productions right in the period he claims it was all so poor.

The pitiable paucity of armaments in his arsenal is pointedly referred to by her by alluding to his reference to but two, {\sl just two}, examples (of Siddhicandra and Jagannātha) in {\sl every paper} of his on the theme of subjectivity.\index{subjectivity} The strikingly fanciful interpretation of Jagannātha’s verses - as autobiographical in nature and as betraying an illicit sexual relationship - bear no scrutiny if one goes by her analysis, drawing as she does upon the careful explanations of P V Kane in his work on Sanskrit poetics. Rarely does one find a keen parallel in the academic world to Pollock’s craft - of flip phraseology (as Hatcher discovers it) and of the art of self-contradiction (as Manjushree uncovers it) -- the regular weapons leftists dexterously deploy. She reins in the fanciful flights of Pollock (in inventing unwarranted parallels with the West) by showing how India had after all no church to produce heresy and no excommunication and censorship to provoke Reformation and religious wars. She cites Danino\index{Danino, Michel} (who in turn is citing Vincent Smith) or Bronner and Shulman (who marshal sound facts and arguments) to detonate the myths devised by him. 
\vskip 10pt

\centerline{*\quad*\quad*}
\vskip 5pt

Swadeshi Indology champions the cause of contending with and combating, on an intellectual plane, the abominable iniquities that Western Indology in general, and American Indology in particular, have been perpetrating without compunction on our heritage.
\eject

A word about the content and style in the articles to follow. It goes without saying that the opinions expressed in the papers presented here are those of the respective authors. The authors hold themselves responsible for the veracity of their statements. And lastly, in the case of quotes from modern works, we have retained the nonstandard/inaccurate/spelling/diacritics/punctuation that may sometimes be present in the source.

We have also introduced a new feature in the Bibliography. Rather than presenting the items as Primary(i.e.\ Sanskrit) sources and Secondary sources (books and articles), we have merged the two, but shown the former in Bold, so that they can be quickly surveyed independently too.

Suggestions for improvement of the book in every aspect are most welcome.

Long ago did Manu record his caveat regarding the iniquitous implications of linguistic larceny:

\begin{quote}
{{\sl vācy arthā niyatāḥ sarve, vāṅ-mūlā, vāg-viniḥsṛtāḥ}} |\\
{\sl tāṁ tu yas stenayed vācaṁ sa sarva-steya-kṛn naraḥ!} ||
\end{quote}

\begin{quote}
“It is in speech that all ideas are settled/regulated; rooted in speech as they are, they issue from speech. And he who theives [which is to say, misinterprets] such speech is verily guilty of stealing everything!”\hfill	- {\sl Manu-smṛti}\index{Manusmrti@\textsl{Manu-smṛti}} 4.256\qquad\,
\end{quote}
\bigskip

\noindent
Makara-Saṅkrānti,\hfill	{\bf Dr.~K S Kannan}\\
Hemalamba Saṁvatsara\hfill Academic Director\\
15th January 2018\hfill and\\	
\phantom{.}~\hfill General Editor of the Series
