\chapter[Practice {\sl  v/s} Theory: Gaṇita and Mathematics]{Practice {\sl\bfseries v/s} Theory:\\ Gaṇita and Mathematics$^{*}$}\label{chapter\thechapter:begin}
\footnotetext[1]{pp.~\pageref{chapter\thechapter:begin}--\pageref{chapter\thechapter:end}. In: Kannan, K S (Ed.) (2018) {\sl Śāstra-s Through the Lens of Western Indology - A Response}. Chennai: Infinity Foundation India.}

\Authorline{Subhodeep Mukhopadhyay}

\hfill{\sl(\url{benny.ya@gmail.com})}

\lhead[\small\thepage\quad Subhodeep Mukhopadhyay]{}

\section*{Abstract}

{\sl Śāstra}-s are an integral part of Indian knowledge systems, and provide systematic procedures to accomplish specific objectives in diverse\break fields - like mathematics, philosophy, architecture, politics, economy and others. Noted Sanskrit scholar Professor Sheldon Pollock however views {\sl śāstra}-s as a problem, and sees a dichotomy between {\sl śāstra} (``theory'')\index{sastra@\textsl{śāstra}!theory and practice@`theory and practice'} and {\sl prayoga}\index{sastra@\textsl{śāstra}!and {\sl prayoga}}\index{prayoga@\textsl{prayoga}} (``practical activity'') in Sanskritic culture, and considers them to be a regressive reformulation of the contents of the Veda-s - the {\sl śāstra par excellence}. This paper is meant to be a refutation of some of Pollock's core assumptions, by way of taking into account Gaṇita-śāstra/Indian arithmetic as a case study. We demonstrate that from the earliest times to premodern times, Gaṇita\index{Ganita@Gaṇita} in India has relentlessly focused on real-life problems, developing logical and efficient algorithms for problem solving, - and this, even among Jain and Buddhist scholars, who do not regard Veda-s as a {\sl pramāṇa}.\index{pramana@\textsl{pramāṇa}} Akin to all Indian schools of thought and even modern science,\index{science!modern} Gaṇita accepts {\sl pratyakṣa-pramāṇa}\index{pratyaksa-pramana@\textsl{pratyakṣa-pramāṇa}}\index{pramana@\textsl{pramāṇa}!pratyaksa@\textsl{pratyakṣa}}\index{pratyaksa@\textsl{pratyakṣa}} or empirical evidence as the first means of knowledge. Depending, however, entirely on axioms, binary logic\index{binary logic} and eternally valid proofs (theory) - as opposed to calculations ({\sl prayoga})\index{sastra@\textsl{śāstra}!and {\sl prayoga}} -  it is formal Western mathematics that categorically rejects the empirical world, and is imbued with theological dogma. Binary logic cannot be taken as a normative logic, as the Buddhist {\sl Catuṣkoṭi}\index{Catuskoti@\textsl{Catuṣkoṭi}} and the Jain {\sl Syādvāda}\index{Syadvada@\textsl{Syādvāda}} schema follow different approaches. Western mathematics, which is the only type of mathematics taught in schools today, therefore comes across as arcane, abstract and complicated to most non-specialists, and has become a tool of little more than cultural hegemony.\index{cultural!hegemony}\\[-20pt] 

\section*{{\sl\bfseries Śāstra} and Indian Knowledge Systems}
\index{sastra@\textsl{śāstra}!pedagogy}

For purposes of pedagogy, knowledge is organized into three categories, {\sl śāstra} (primary sources), (compendiums) and {\sl ṭīkā}\index{Tika@\textsl{ṭīkā}} (commentaries)\index{commentaries} within the tradition of Indian knowledge systems. {\sl Śāstra}-s are broadly categorized as {\sl apauruṣeya}\index{apauruseya@\textsl{apauruṣeya}} and {\sl pauruṣeya}.\index{apauruseya@\textsl{pauruṣeya}}\endnote{``{\sl Apauruṣeya}\index{apauruseya@apauruṣeya} discourse is non-contingent, and its assertions, like those of science, are not dependent on an individual for their truth.'' (Kapoor 2005:21)} {\sl Apauruṣeya} includes the Veda-s and notionally the {\sl vedāṅga}-s, while {\sl pauruṣeya}\index{pauruseya@\textsl{pauruṣeya}} includes\break {\sl purāṇa}-s\index{Purana@Purāṇa} including ({\sl itihāsa}), {\sl ānvīkṣīkī}\index{anviksiki@\textsl{ānvīkṣīkī}} (logic), {\sl mīmāṁsā}\index{Mimamsa@Mīmāṁsā} (analysis and interpretation), {\sl dharma-śāstra} (sociology/law/ritual), {\sl kāvya vidyā}\index{vidya@\textsl{vidyā}} and all other {\sl vidyā}-s and {\sl kalā}-s (Kapoor\index{Kapoor, Kapil} 2005:22). {\sl Śāstra}-s are texts dealing with specialized\index{Pollock, Sheldon passim} technical knowledge in diverse fields like mathematics, physics, chemistry, arts, architecture, astronomy, philosophy, politics, and a host of other subjects (Malhotra\index{Malhotra, Rajiv} 2016:37) (Lochtefeld\break 2002:626). 

``{\sl Śāstra}'' is, in general, used to refer to any science.\index{science!sastra as@\textsl{śāstra} as} Scholars have translated, for example, {\sl Artha-śāstra}\index{Arthasastra@\textsl{Artha-śāstra}} variously as the ``Science of Politics'', ``Treatise on Polity'', ``Science of Material Gain'', ``Science of Polity'', and ``Science of Political Economy'' (Boesche 2003). While discussing the vast scope and breadth of {\sl śāstra}-s, Professor Sheldon Pollock (hereafter Pollock) observes that they cover ``virtually every activity'' ranging from ``cooking, sexual intercourse, elephant-rearing, thievery, to mathematics, logic, ascetic renunciation, and spiritual liberation.'' (Pollock 1985:502). He enumerates among others, \hbox{{\sl śāstra}-s} like Kṛṣi-śāstra, Gaṇita-śāstra, Gandha-śāstra, Kalā-śāstra, Matysa-śāstra and Śakuna-śāstra. Śūdraka,\index{Sudraka@Śūdraka} who thrived somewhere between 3$^{\text{rd}}$ century BCE and 5$^{\text{th}}$ century CE, even considers thievery a science in his plays {\sl Mṛc-chakaṭika}\index{Mrcchakatika@\textsl{Mṛc-chakaṭika}} and {\sl Padma-prābhṛtaka},\index{Padmaprabhrtaka@\textsl{Padma-prābhṛtaka}}  and alludes to a book on Caurya Śāstra called {\sl Steya-sutta}\index{Steyasutta@\textsl{Steya-sutta}} (Varadpande\index{Varadpande, M.L.} 2005:158-159).

\newpage

Unlike formal mathematics and theoretical sciences,\index{science!theoretical} {\sl śāstra}-s do not follow the proposition-axiom-and-proof model, and instead, comparable to applied sciences\index{science!applied} and engineering today, present ``a series of rules, which serve to characterize, and carry out systematic procedures to accomplish various ends. These systematic procedures are variously referred to as {\sl vidhi, kriyā} or {\sl prakriyā, sādhanā, karma} or {\sl parikarma, karaṇa}, etc., in different disciplines'' (Srinivas\index{Srinivas, M D} {\sl et al.} 2014). In his treatise {\sl Vākya-padīya},\index{Vakyapadiya@\textsl{Vākya-padīya}}\index{Vyakarana@\textsl{Vyākaraṇa}!texts!Vakyapadiya@\textsl{Vākya-padīya}} Bhartṛhari\index{Bhartrhari@Bhartṛhari} (ca. 500 CE) states clearly that the procedures taught in the {\sl śāstra}-s are simply certain means ({\sl upāya}) towards certain desired ends, and must not be viewed as constraints or regulations.\endnote{``{\sl Upāya}-s (procedures taught in {\sl śāstra}-s) are to be discarded, even though they are to be used for accomplishing an objective. There is no necessary limitation on such {\sl upāya}-s. One accomplishes objectives by one means or the other.'' Discussed in (Srinivas\index{Srinivas, M D} 2016:9)} This is a continuation of Patañjali's (c.2$^{\text{nd}}$ century BCE) assertion in his commentary {\sl Mahā-bhāṣya}\index{Mahabhasya@\textsl{Mahābhāṣya}}\index{Vyakarana@\textsl{Vyākaraṇa}!texts!Mahabhasya@\textsl{Mahābhāṣya}} (on Pāṇini’s {\sl Aṣṭādhyāyī})\index{Astadhyayi@\textsl{Aṣṭādhyāyī}}\index{Vyakarana@\textsl{Vyākaraṇa}!texts!Astadhyayi@\textsl{Aṣṭādhyāyī}} to the effect that ``utterances and their meanings are actually established in the world: one does not, after all, go to a grammarian to make utterances for him, as one goes to a potter for pots.'' Others like Puṇyarāja\index{Punyaraja@Puṇyarāja} (commentator on {\sl Vākyapādiyā}) and the 18$^{\text{th}}$ century scholar Nāgeśa-bhaṭṭa\index{Nagesabhatta@Nāgeśa-bhaṭṭa} reiterate this same practical approach to scientific theorization (Srinivas 2016:8-10). 


Kapil Kapoor\index{Kapoor, Kapil} has highlighted this fact that Indian thought being pluralistic, is not obsessed with a ``one given truth'' as in the West (Kapoor 2005:26). He says that while Indian thinkers acknowledge the existence of some truth, they have reservations about the possibility of getting access to or knowing it via single/exclusive paths/methods, and thus there can be different ways to know the truth. This allows for multiple world-views, ontologies and epistemologies; and therefore, an individual is not subject here to the pressures of societal/communal norms.  

\section*{Pollock’s Views on {\sl\bfseries Śāstra}}

Pollock views Sanskrit as an artificial enterprise, and sees Sanskrit knowledge systems, as a derivative of this non-natural language, and even as a mere continuation of the revelation of the Veda-s. Sanskrit knowledge systems include, according to him,  ``forms of thought about Sanskrit, about the language's particular linguistic identity, peculiar social and ideological history (its connection with old revelation of the Veda-s), and specialized resource (such as the hyper-synonymy of a non-natural language)'' (Pollock 2005:10). 

{\sl Śāstra}-s are, according to him, the Sanskrit equivalent of cultural grammars,\index{cultural!grammars} which are, in turn, the ``codes defining a given culture''; and thus {\sl śāstra} is also in a way ``practical knowledge, mastery of which makes one a competent member of the culture in question'' (Pollock 1985:500). Pollock concedes that the word {\sl śāstra} traditionally has had different shades of meaning - such as rules, regulations, system of ideas, and philosophical system; but he focuses exclusively on the regulatory aspect of {\sl śāstra}-s, and especially as applicable to the classical period, - all in order to establish his thesis. He summarizes {\sl śāstra} as ``a verbal codification of rules, whether of human or divine provenance, for the positive and negative regulation of some given human practices'' (Pollock 1985:501). However he goes further and zeroes in on what he believes to be ``an old meaning of the term, preserved in the classical period, above all, in the Pūrva- and Uttara-mīmāṁsā\index{Mimamsa@Mīmāṁsā} tradition.'' He now explicitly says: 
\begin{myquote}
``...{\sl śāstra}\index{sastra@\textsl{śāstra}} refers more specifically to {\sl veda}, as when, for example, in the {\sl Brahmasūtras,\index{Brahmasutras@\textsl{Brahma-sūtra}-s} brahma} is described as {\sl śāstrayoni}-, ``that, the source of our knowledge of which is {\sl śāstra}'' (that is, the vedas and in particular the Upaniṣads).\index{Upanisad@Upaniṣad} Such a shared signifier for the two domains (``rule'' or ``book of rules'' on the one hand and ``revelation''\index{revelation} on the other) bespeaks an important rapprochement, or even convergence, between them. The bivalency may have been more than symptomatic, having perhaps fostered a postulate of critical importance in Indian intellectual\index{intellectual history!Indian} history (below, pp. 518-19), unless it is more properly viewed as an effect rather than the cause of that postulate.”\hfill 	(Pollock 1985:502)
\end{myquote}

In his view, this belief of traditional India in the eternal nature of the Veda-s and as a fountainhead of all knowledge, constrains the {\sl Śāstra}-s in the production of new knowledge, as {\sl Śāstra} texts can at best reformulate what is already in the Veda-s. Pollock thus considers {\sl śāstra}-s as theory and injunctions, and treats them separately from {\sl prayoga}\index{prayoga@\textsl{prayoga}}\index{sastra@\textsl{śāstra}!and {\sl prayoga}} (``practical activity''); and postulates that in Sanskritic culture, theory was ``held always and necessarily to precede and govern practice'' and that ``all knowledge is pre-existent, and that progress can only be achieved by a regressive re-appropriation of the past.'' This, Pollock believes, leads to a severe curtailment of individual agency, which he describes as ``the most exquisite expression of centrality of rule-governance in human behavior'' in Indic culture (Pollock 1985:499-500). 

\newpage

Pollock sees in all this, a kind of bondage emanating from ``a structure of religious dogma'' (Veda-s), such that the secular life of an individual is subjected to an all-pervasive ritualization,\index{ritualization} ``leading to misery and entrapment of people'', and laments that these dogmas have been a major problem in Hindu society (Malhotra\index{Malhotra, Rajiv} 2016:115). He seems to consider the Veda-s as books of revelation, perhaps similar to monotheistic revealed books like the Bible or the Koran, and sees a convergence of rules as well as revelations in {\sl śāstra}-s. Finally, he argues that while {\sl śāstra}-s survive in practice in varying degrees in Hindu society today, Sanskrit intellectual history\index{intellectual history!of Sanskrit} and knowledge system came to an inevitable end in the eighteenth century, almost a century before ``interactions with colonial knowledge'' and despite the prevalence of a ``dynamic era of intellectual inquiry'' ({\sl pax mughalana})\index{pax mughalana\textsl{pax mughalana}} since 1590 under the Muslim rule, and ``created a vacuum for Western knowledge to fill'' the gap (Pollock 2005:82-83) (Pollock 2011:5).

Pollock's approach may be succinctly captured as follows:
\begin{itemize}
\item Out of the many different shades of meaning of ``{\sl śāstra}'' ranging from science, procedure, philosophical system to regulations, he selects the ``regulating'' or ``codifying'' aspect of {\sl śāstra} as representative because he ``seems to find it prominent in what seems to be the first formal definition of {\sl śāstra}'' (Pollock 1985:501).

\item Within this specific context, he investigates a few {\sl śāstra}-s in the domains of language, social relations and sexuality during the ``classical age''; and concludes that all knowledge derives from {\sl śāstra}, and that {\sl śāstra}-s refer more specifically to the Veda-s.\index{sastra!refer to Veda-s} 
\end{itemize}

Based on the above approach, Pollock's hypothesis and chain of reasoning may be summarized as follows (Pollock 1985:510-515): 
\begin{itemize}
\item In the Indian tradition, knowledge creation is viewed as a divine activity and Indian learning therefore ``perceives itself and indeed presents itself largely as commentary'' on the \hbox{Veda-s}. Anything new is therefore simply perceived as ``backward movement aiming at a closer and more faithful approximation to the divine pattern.''

\newpage

\item According to Pollock, ``experience, experiment, invention, discovery, innovation'' are epistemologically meaningless in the context of Indian knowledge systems.

\item The Indian tradition thus views all new discoveries and innovation ``through the inverting lens of ideology, as renovation of recovery'', and thus there can be no conception of progress.

\item On the other hand, the dominant position of Western thinking from the time of Aristotle\index{Aristotle} has been that ``efficient practice precedes the theory of it''; and that intelligent performance of actions is an outcome of such ``commonsense assessment'' of how conceptual systems and behavior actually interact; and are ``diametrically opposed'' to the regressive back-ward looking Indian approach. 

\item The practice of any art or science in the West, not being dependent on {\sl śāstric} norms, is eternally progressive and forward-looking - with no ``mythic crystallizations of the postulate of shastric priority, namely the accounts of their origins.''
\end{itemize}

\section*{Approach of this Paper}

We will first examine the methodological approach\index{misinterpretation!techniques of!methodical approach} that Pollock has employed. At the very outset, we come across three main problems with the very framework of his reasoning.\index{misinterpretation!techniques of!flawed arguments}
\begin{itemize}
\item Pollock uses selective observations to form a hypothesis - by virtue of using the most restrictive definition of {\sl śāstra}.

\item He indulges in over-generalization\index{misinterpretation!techniques of!over-generalization} by applying this narrow definition to all the {\sl śāstra}-s: Mīmāṁsā subscribes to this restrictive definition, and since in his view, Mīmāṁsā is ``the most orthodox and in many respects most representative of Indian traditions'' (Pollock\index{misinterpretation!techniques of!rejecting counter-evidence} 1985:503), this definition, he says, applies to all Indian knowledge traditions universally.

\item He succumbs to premature closure when he summarily rejects all counter-evidence\index{misinterpretation!techniques of!rejecting counter-evidence} - by declaring that such voices are ``pretty much in minority'' (Pollock 1985:510).
\end{itemize}

Now that we see the flaws in his basic approach, we need to examine his chain of reasoning within this broad framework. We need to understand whether the broad conclusions that he has reached are universal or context-based and specific to perhaps some {\sl śāstra}-s only under some circumstances.  We also need to inspect his assumptions of Western knowledge systems, as they can be said to constitute the ``gold standard'' for him, and against which he is directly as well as indirectly benchmarking the Indian knowledge systems.

While discussing Rājaśekhara’s\index{Rajasekhara@Rājaśekhara} enumerations of {\sl śāstra}-s he observes that even Gaṇita\index{Ganita@Gaṇita} is ``amenable to treatment in {\sl śāstra}'' and is one of the many practices for which the rules provided by {\sl śāstra} (as per his narrow definition) also apply (Pollock 1985:502). 

Therefore we will examine in this paper Pollock’s chain of reasoning in the context of Gaṇita-śāstra (Indian Arithmetic).\index{Indian!Arithmetic} Continuing in the same vein, we will then apply his framework of Western knowledge systems and his views of {\sl śāstra} to modern Formal Mathematics, and analyze the implications. We will suitably demonstrate in this paper that his views on {\sl śāstra}-s can be applied remarkably well to modern Western formal mathematics but not to Indian arithmetic, thereby rendering his generalizations questionable. On the other hand, as we shall demonstrate, the modern scientific method of research lends itself quite well to understanding and practicing Gaṇita but not mathematics.

\section*{Gaṇita-śāstra as a commentary on the Veda-s}

Gaṇita (arithmetic) is an important {\sl śāstra} for Hindus, Jains as well as Buddhists. Jains regard the knowledge of {\sl saṅkhyāna} (the science\index{science!of numbers} of numbers) as a particularly important accomplishment, and in fact one of the four branches of their religious literature is called {\sl gaṇitānuyoga} (Datta and Singh 1962:4). As per Mahāvīrācārya\index{Mahaviracarya@Mahāvīrācārya} (a 9$^{\text{th}}$ century Jain mathematician), Gaṇita has immense practical applications across a diverse range of real-life situations: 
\begin{myquote}
``All activities which relate to worldly, Vedic or religious affairs make use of enumeration. In the art of love, economics, music, dramatics, in the art of cooking, in medicine, in architecture and such other things, in prosody, in poetics and poetry, in logic, grammar and such other things, and in relation to all that constitute the peculiar value of the arts, the science of calculation (Gaṇita) is held in high esteem. In relation to the movement of the Sun and other heavenly bodies, in connection with eclipses and conjunction of planets, and in the determination of direction, position and time and in (knowing) the course of the Moon -- indeed in all these it (Gaṇita) is accepted (as the sole means).''

\hfill (Srinivas {\sl et al.} 2014:7)
\end{myquote}

In Buddhist thought, special emphasis was given on arithmetic ({\sl gaṇanā/saṅkhyāna}) and even Gautama the Buddha started his education when he was eight years old ``firstly (with) writing and then arithmetic as the most important of the 72 sciences and arts'' (Datta and Singh 1962:4-6). Gaṇeśa Daivajña,\index{Ganesa Daivajna@Gaṇeśa Daivajña} in his commentary {\sl Buddhi-vilāsinī}\index{Buddhivilasini@\textsl{Buddhi-vilāsinī}} (c. 1540) on the 12$^{\text{th}}$ century text {\sl Līlāvatī}, defines Gaṇita as the science of computation\endnote{{\sl gaṇyate saṅkhyāyate tad gaṇitam} | {\sl tat-pratipādakatvena tat-saṁjñaṁ śāstram ucyate} | Quoted in (Srinivas {\sl et al.} 2014:8)}, and we see demonstrable evidence right down to early pre-modern times of the {\sl prayoga}\index{sastra@\textsl{śāstra}!and {\sl prayoga}} of Gaṇita in almost all aspects of life. M.D. Srinivas says in this regard that Indian mathematical texts, similar to {\sl śāstra-s} in general, follow a computational and algorithmic approach rather than being all about mathematical propositions, theorems, and proofs. 
\begin{myquote}
The canonical texts of different disciplines in Indian tradition present rules, which are generally called as {\sl sutras} or {\sl lakshanas}. Most of these rules serve to characterize systematic procedures which are designed to accomplish specific ends. In this way the Indian Sastras are always rooted in {\sl vyavahāra} or practical applications.''
\hfill (Srinivas\index{Srinivas, M D} 2015:1)
\end{myquote}

Both Jainism and Buddhism reject the Veda-s (``the primordial {\sl śās\-tra}-s'' as Pollock puts them) as {\sl pramāṇa} (valid means of knowledge), and thus there is no question of Gaṇita texts being commentaries\index{commentaries} on the eternal knowledge encapsulated in the Veda-s, or of new discoveries of being an approximation to some Vedic divine pattern.

\section*{Epistemology of Gaṇita-śāstra and Mathematics: {\sl\bfseries Upapatti versus} Proof}
\index{Ganita@Gaṇita}\index{upapatti@\textsl{upapatti}}

Modern mathematics is about proof and not computation. A person has to {\sl irrefutably} prove every mathematical proposition, and this philosophy of irrefutable proof is also what is taught in schools and colleges. Empirical observation and inference is not a valid means of knowledge in formal mathematics (unlike science),\index{science!contrasting with mathematics} and computation does not play any significant role in the philosophy of mathematics.\endnote{See (Horsten\index{Horsten, Leon} 2016) for more details.} 

This Platonist\index{Plato} School of Mathematics,\index{mathematics} which is the dominant school, adheres to the metaphysical view that mathematics exists in relation to eternal abstract objects, beyond space and time, and mathematics is simply the means to discover such truths.\endnote{``Mathematical truths are therefore discovered, not invented.'' (Linnebo\index{LinneboOystein@Linnebo, Øystein} 2013)},\endnote{``Mathematical platonism is any metaphysical account of mathematics that implies mathematical entities exist, that they are abstract, and that they are independent of all our rational activities... Mathematical platonists are often called `realists''' (Cole 2016)} Numbers are treated as abstractions according to David Hilbert\index{Hilbert, David} who is considered to be one of the most influential and universal mathematicians of the 20$^{\text{th}}$ Century; and Horsten\index{Horsten, Leon} (2016) an adherent of mathematic formalism, argues that symbols are abstract entities, and higher mathematics is a formal game, and the statements of higher-order mathematics are uninterpreted strings of symbols whose proofs are nothing more than manipulation of these symbols using some fixed rules. He adds that the point of this game, “in Hilbert's\index{Hilbert, David} view, in proving statements of elementary arithmetic, which do have a direct interpretation.”

Gaṇita-śāstra is not based on the Western notion of axiomatic proofs but rather on what is known as {\sl upapatti},\index{upapatti@\textsl{upapatti}} whose purpose is 

``(i) to remove confusion and doubts regarding the validity and interpretation of mathematical results and procedures; and, 

(ii) to obtain assent in the community of mathematicians'' (Srinivas\index{Srinivas, M D} 2016:4). Moreover an {\sl upapatti} may involve empirical observations and experimental verifications, and is subject to all the pitfalls of any empirical sciences\index{science!empirical} like the construction of a bridge or a rocket-launch, and in this sense Gaṇita is quite distinct from Greco-European mathematics which demands infallible certainty.

Therefore Pollock's second observation - that experience, experiment, invention, discovery, innovation are meaningless in the context of Indian knowledge systems (Pollock 1985:502) - must be rejected; and in fact it is modern day Western mathematics, on the other hand, where empirical observations and experience play no role.

\section*{Gaṇita and Progress: {\sl\bfseries Śulba-sūtra}-s, Geometry and Town Planning}
\index{Sulbasutras@\textsl{Śulba-sūtra}-s}

The {\sl Śulba-sūtra}-s\index{Sulbasutras@\textsl{Śulba-sūtra}-s} are among the oldest known mathematics texts in existence and the earliest among them have been estimated to have been composed in 800 BCE or earlier, although the principles espoused were probably in use from the much earlier Vedic times. They are manuals dealing with the construction of elegant and complex fire-altars used in Vedic {\sl yajña}-s,\index{yajna@\textsl{yajña} altar!construction of} and pre-suppose the knowledge of practical constructive geometry and abstract algebraic formulations (Dani\index{Dani, Shrikrishna} 2010:10). 

Being engineering manuals, they provide the required mathematical guidelines for the construction of accurate brick-altars,\index{fire-altars} some of which have complex shapes like ``a falcon\index{yajna@\textsl{yajña} altar!shapes of} in flight with curved wings, a chariot-wheel complete with spokes or a tortoise with extended hands and legs''\index{yajna@\textsl{yajña} altar!shapes of} (Dutta\index{Dutta, Bibhutibhushan} 2002:5). The {\sl Śulba-sūtra}-s\index{Sulbasutras@\textsl{Śulba-sūtra}-s} deal with practical and real-world design and architectural problems. The construction of altars of various complex shapes like the isosceles trapezium, falcon or tortoise shape involved the use of square roots, irrational numbers, Pythagorean relations and advanced geometric concepts like area transformation (Bag 1990:4).

Compared to contemporary Egyptian and Greek geometry of those times, the Indian technique was distinct and more practice-oriented ``because of its concept of both rational and irrational numbers and their application of for verifying the truth of the theorem of the square of the diagonal'' (Bag 1990:17). Seidenberg\index{Seidenberg, A} the well-known mathematician and math-historian,  posited that geometry as a subject had a ritual origin, and that constructive geometry and geometric algebra can trace their origin to an ancient culture resembling Vedic culture (Dutta 2002:4). Since Gaṇita played an important role in the Vedic way of life and in Vedic geometry in particular, we may conclude, extending Seidenberg’s argument, that geometry as a scientific discipline of varied practical applications most likely originated from the Gaṇita-like {\sl śāstra} of a culture resembling Vedic culture.\endnote{So far we are not aware of any other culture resembling Vedic culture outside India.}

Moreover there is no reason to believe that this sophistication in geometry of ancient Hindus was restricted to design and construction of fire altars only. Both the {\sl Baudhāyana Śulba-sūtra}\index{Sulbasutras@\textsl{Śulba-sūtra}-s!Bauddhayana@\textsl{Baudhāyana}} and the {\sl Āpastamba Śulba-sūtra}\index{Sulbasutras@\textsl{Śulba-sūtra}-s!Apastambha@\textsl{Āpastamba}} enumerate the different units of measurement used for the purpose of construction, and in fact units like ``{\sl ang, pada, prakrama, prādeśa, bāhu, aratni} carry long tradition and have been used earlier in {\sl Saṁhitā} and {\sl Brāhmaṇic} literature in the same sense as these have been used in the {\sl Śulba-sūtras}'' (Bag 1990:10). Studies have shown that in terms of architecture and large-scale town-planning, and especially with respect to the use of precise geometric ratios, the Harappan civilization\index{Harappan civilization} scheme of units is related to classical Indian historical unit systems.\endnote{See (Danino\index{Danino, Michel} 2003) for details.} Specifically, studies across different sites like Dholavira,\index{Sulbasutras@\textsl{Śulba-sūtra}-s!linear measures from@-linear measures from!Indus-Sarasvati sites@- Indus-Sarasvati sites}\index{Sulbasutras@\textsl{Śulba-sūtra}-s!linear measures from@-linear measures from!Artha-sastra scheme@- \textsl{Artha-śāstra} scheme} Lothal, Mohenjo-daro\index{Mohenjo-daro} have confirmed that the much later {\sl Artha-śāstra’s}\index{Arthasastra@\textsl{Arthaśāstra}} scheme of linear measures, which itself is a derivative of the {\sl Śulba-sūtra}-s system, can be traced archaeologically to Harappan ratios and proportions (Danino\index{Danino, Michel} 2008:66).

Thus we see that Gaṇita as a {\sl śāstra} (science) of immense practical value was cultivated and practiced over a long period of time in India. While discussing the implications of the priority of theory over practice, Pollock observes: 
\begin{myquote}
``That the practice of any art or science, that all activity whatever succeeds to the degree it achieves conformity with shastric norms would imply that the improvement of any given practice lies, not in the future and the discovery of what has never been known before, but in the past and the more complete recovery of what was known in full in the past.''

\hfill (Pollock 1985:512)
\end{myquote}

This observation is not applicable to empirical sciences\index{science!empirical} like Gaṇita-śāstra which follow the experiment - observe - and - theorize paradigm - which is a bottom-up approach rather than the axiom-and-proof\break paradigm, which is a top-down approach of formal mathematics. Like science and engineering, empirical observations were in fact the starting point of Indian arithmetic, followed by inferences and formulation of algebraic abstractions and algorithms, which could then be applied to other areas. This ties in very closely to the scientific method of research in contemporary Western knowledge systems.\\[-20pt]

\section*{Applying Pollock’s {\sl\bfseries Śāstra} Framework to\\ Formal Mathematics}

In this section I will demonstrate that Pollock’s chain of reasoning with regards to {\sl śāstra}-s and {\sl prayoga}\index{sastra@\textsl{śāstra}!and {\sl prayoga}} can, and in fact should be, applied to formal Western mathematics.

The important building blocks of modern mathematics are statements, tautology, contradictions, axioms and proof.\endnote{A tautology is a general statement that is always true; and a logical deduction is obtained by substituting in a tautology. Contradictions are the opposite of a tautology and are false under all circumstances. (Goldberger 2015:5)} As per the Law of the Excluded Middle, every statement must be either true or false, but never both or none.  While logic helps in deduction, some statements are required to begin with. These are axioms and are considered eternal truths. With these axioms and logic, on application of certain pre-defined operations, one arrives at a proof.\endnote{``A proof of a theorem is a finite sequence of claims, each claim being derived logically (i.e. by substituting in some tautology) from  the  previous  claims,  as  well  as  theorems  whose  truth  has  been already established.  The last claim in the sequence is the statement of the theorem, or a statement that clearly implies the theorem.'' (Goldberger 2015:6)} 

Noted mathematician, S.G. Krantz\index{Krantz, S G} while delineating the difference between science (which relies on reproducible experiments with control) and mathematics (which relies on axioms) says that in mathematics one must set definitions and axioms in place before any other action, and that any statement without proof has no currency (Krantz 2007:7-8).\\[-20pt]

\section*{Mathesis and Eternal Truths}
\index{mathesis}

While discussing the epistemological implications of {\sl satkārya-vāda}\index{Satkaryavada@\textsl{Satkāryavāda}} Pollock assumes that they ``operated subliminally in the mythic representation of the transcendent provenance and authority of {\sl sastra}'', and in fact, in a Western context the same is seen in the ``Socratic\index{Socrates} merging of {\sl mathesis}\index{mathesis@\textsl{mathesis}} and {\sl anamnesis}''\index{anamnesis@\textsl{anamnesis}} (Pollock 1985:518). {\sl Satkārya-vāda}\index{Satkaryavada@\textsl{Satkāryavāda}} is out of scope for this paper, but the notion of {\sl mathesis} has a very important role to play in Western mathematics. 

The early Greeks\index{Greek mathematics} believed that the knowledge on mathematics was innate and eternal and that even a slave boy with skillful probing could recall this eternal innate knowledge of mathematics from previous lives. This is known as {\sl mathesis} or recollection of knowledge from previous lives, and is the origin of Western mathematics (Raju\index{Raju, C K} 2011a:274). Even today, Mathematical Platonism,\index{Mathematical Platonism} which is the dominant school of mathematical thought, views all mathematical objects as already pre-existing and mathematics as a tool to discover the hidden reality and “that mathematical entities are not constituents of the spatio-temporal realm” (Cole 2016). 

Mathematics as a subject and the geometric notion of eternal truths, faced grave conceptual challenges in the 19$^{\text{th}}$ century - with the discovery of non-Euclidean geometry, space-filling curves, continuous nowhere-differentiable curves and other findings opposed to geometric intuition and the Euclidean certainty of Western mathematics.
\begin{myquote}
``The situation was intolerable because geometry had served, from the time of Plato,\index{Plato} as the supreme exemplar of the possibility of certainty in human knowledge. Spinoza\index{Spinoza} and Descartes\index{Descartes, Rene} followed the {\sl``more geometrico''} in establishing the existence of God, as Newton followed it in establishing his laws of motion and gravitation. The loss of certainty in geometry was philosophically intolerable, because it implied the loss of all certainty in human knowledge.''	
\hfill (Hersh\index{Hersh, Reuben} 1979:35-36)
\end{myquote}

After this point, mathematicians started considering arithmetic and not geometry as the foundation of mathematics and this gave rise to formal set theory, proof theory of Hilbert\index{Hilbert, David} and formal logic, as mathematicians went on a quest of completeness and certainty in mathematics.\endnote{“I wanted certainty in the kind of way in which people want religious faith. I thought that certainty is more likely to be found in mathematics than elsewhere.” (Russell\index{Russell, Bertrand} 1956:54).} In 1898, Hilbert published his book {\sl Grundlagen der Geometrie}\index{Grundlagen der Geometrie@\textsl{Grundlagen der Geometrie}} which makes use of the axiomatic approach; and eventually from the 1920s, when he fully adopted the formalistic system, he started treating the basic concepts of geometry, the axioms and the theorems as “purely formal constructs, having no direct, intuitive meaning whatsoever” (Corry 1997:258-259).

\section*{Rejection of Empirical Evidence}

%\begin{myquote}
In ordinary circumstances, if a child were asked to prove that $2 + 2 = 4$, she or he could demonstrate the same by counting four oranges or four pencils; and in experimental science,\index{science!empirical} engineering, Gaṇita,\index{Ganita@Gaṇita} real life situations or applications, this enumeration is a perfectly valid {\sl pramāṇa}. Such empirical proof is however not accepted in formal mathematics, which demands abstract, eternal, and infallible proof based on axioms - like Peano’s axioms.\index{Peano’s axioms} Since most people do not know what Peano’s axioms are, they are forced to rely on mathematical authority (Raju\index{Raju, C K} 2012: Chapter 12, Section 7).
%\end{myquote}

Noted Western philosopher Kant, for example, rejects empirical knowledge outright, saying that ``an empirical proposition cannot possess the quality of necessity and absolute universality, which, nevertheless, are the characteristics of all geometrical propositions'' (Kant\index{Kant, Immanuel} 1855:39). 

This disdain of modern mathematics for empirical evidence may be traced to the work of Greek mathematician Euclid\index{Euclid} (who supposedly lived in 3$^{\text{rd}}$ century BCE), and is said to be the author of The {\sl Elements}, which is considered one of the most influential works in the history of mathematics\endnote{There is actually no evidence that there was ever any person called Euclid.\index{Euclid} Noted historian of Greek mathematics, David\index{Fowler, David} Fowler admitted that that almost all texts supposedly authored by Euclid ``come from Constantinople, the earliest from 888 AD, closer time to us than to the supposed date of Euclid!'' (Fowler 2002). It is more likely that {\sl Elements}\index{ElementsofEuclid@\textsl{Elements} of Euclid} was authored as a religious book by Theon or his daughter Hypatia, native Egyptians and residents of Alexandria, and proponents of {\sl mathesis}.\index{mathesis@\textsl{mathesis}} See (Raju\index{Raju, C K} 2009).}. The history of Western scientific temper and mathematical spirit, the very notion of irrefragable demonstration, axioms and non-empirical proofs is said to be traceable to Euclid’s work.\endnote{The very notion of ``irrefragable demonstration'' of Euclid is based on a single spurious isolated passage in very a late manuscript of Proculus' {\sl Commentary}. See (Raju 2011a).} Ironically, the {\sl Elements} does make use of empirical insights in the  1$^{\text{st}}$ and $4^{\text{th}}$ proposition.

%\begin{myquote}
Raju points out that the 47$^{\text{th}}$ proposition (``Pythagorean theorem'') could easily be solved in one step, like Indian texts, if empirical methods were used throughout, and thus the ``myth (that the Elements\index{ElementsofEuclid@\textsl{Elements} of Euclid} concerns metaphysical proof) makes the Elements either trivial or inconsistent'' (Raju 2011a:276).
%\end{myquote}

Moreover, as a result of Christian rational theology, it was believed from the time of Aquinas\index{Aquinas, Thomas} (13$^{\text{th}}$ century CE): 
\begin{itemize}
\item[-] that the Bible\index{Bible} was the word of God who rules the world with eternal laws,
\item[-] that nature was the work of God written in the language of mathematics, and 
\item[-] that ``logic which binds God is more powerful than empirical facts which don’t'' (Raju\index{Raju, C K} 2012:Chapter 13, Section 2). 
\end{itemize}

In order to perpetuate the alleged metaphysical and eternal nature of {\sl Elements}, Hilbert\index{Hilbert, David} in the 20$^{\text{th}}$ century reinterpreted Euclid’s\index{Euclid} {\sl Elements}\index{ElementsofEuclid@\textsl{Elements} of Euclid} with all occurrences of the word ``equality'' reading as ``congruence''

\begin{myquote}
``In the process, mathematics was transformed: from the ``science of learning'' or the ``science of arousing the soul'' to ``a method of compelling argument''. That is also how mathematics is taught today, as a doctrine of proof. The special thing about this deductive proof (or axiomatic-deductive method) is that it is purely metaphysical. Western philosophy believed that such metaphysical proofs are ``perfect'' and infallible, in contrast to empirical proofs which are fallible.''

\hfill 	(Raju 2012: Chapter 12, Section 5)
\end{myquote}

The result of all this is seen in Bertrand Russell\index{Russell, Bertrand} and Alfred North Whitehead's\index{Whitehead, Alfred North} monumental work {\sl Principia Mathematica},\index{Principia Mathematica@\textsl{Principia Mathematica}} where it takes over 360 pages to prove that $1 + 1 = 2$. On the other hand, practitioners of Gaṇita\index{Ganita@Gaṇita} would have perhaps objected to the very notion of $2 + 2 =4$ as a universal notion on three grounds:

%\begin{myquote}
{{\bf Abstraction}}: Adding two coins and two horses, do not lead to anything meaningful and useful, unless one abstracts the horses and coins to generic objects.

{{\bf Base}}: Two would mean something different when the base is changed from decimal to trinary or hexadecimal. Even on an abstract level, two and two would still not make four in such a case.

{{\bf Units}}: Two five rupee coins and two one rupee coins equal twelve rupees, which may be expressed as either six two rupees coins, or twelve one rupee coins, but not as four three rupees coins or three four rupee coins, as three rupees coins or four rupee coins do not exist in the real world.
%\end{myquote}

A direct outcome of this formalization of mathematics has been the separation of practice and theory in mathematics. Practical disciplines like engineering involve algorithmic computations and have minimal or no requirement of exact proofs. On the other hand, theoretical mathematics has become exceedingly complicated, unintuitive and laborious on account of excessive formalism. Raju explains why such formalism is anti-utilitarian and culturally biased: 
\begin{myquote}
``The Western philosophy of math originated in religious beliefs about {\sl mathesis},\index{mathesis@\textsl{mathesis}} cursed by the church. Later, mathematics was ``reinterpreted'', in a theologically-correct way, using the myth of ``Euclid'' and his deductive proofs. The {\sl fact} of the empirical proofs in the {\sl Elements},\index{ElementsofEuclid@\textsl{Elements} of Euclid} was, however, contrary to this myth. The discrepancy was resolved by Hilbert\index{Hilbert, David} and Russell\index{Russell, Bertrand} who rejected empirical proofs as unsound, reducing all mathematics to metaphysics... Historically, most school-level math originated in the non-West with a practical epistemology, but was absorbed in the West after superimposing on it an incompatible Western metaphysics, still used to teach it. This has made mathematics needlessly complex.~Accordingly, math can be made easy and more universal by reverting to a more practical epistemology.''
\hfill \hbox{(Raju\index{Raju, C K} 2011a:274)}
\end{myquote}

\section*{Binary Logic of Formal Math {\sl\bfseries versus} Buddhism, Jainism and Quantum Logic}
\index{binary logic}
\index{Quantum logic}

The Law of Excluded Middle states that every statement must be either true or false, but never both or neither. Mathematical proofs are based solely on this binary logic. It must however be clearly noted that binary logic is neither universal nor normative. There is no requirement that a statement must be either only true or only false. The Buddhist {\sl Catuṣkoṭi}\index{Catuskoti@\textsl{Catuṣkoṭi}} logic system (four-valued), the Jain {\sl Syādvāda}\index{Syadvada@\textsl{Syādvāda}} system (seven-fold predication) and even quantum logic system are all valid non-binary\index{binary logic!as opposed to non-binary} logic systems. In {\sl Catuṣkoṭi} system, a statement can be true, false, both true and false and neither true nor false. In the Jain {\sl Syādvāda} system, a statement can take seven states like {\sl syād-asti} (perhaps true),\break {\sl syān-nāsti} (perhaps not true), {\sl syād-asti-nāsti} (perhaps true as well as not true), {\sl syād-asti-avaktavyaḥ} (perhaps true and indescribable), {\sl syān-nāsti-avaktavyaḥ} (perhaps not true and indescribable), {\sl syād-asti-nāsti-avaktavyaḥ} (perhaps both true and untrue and indescribable) and {\sl syād-avaktavyaḥ} (perhaps indescribable).

The sole reliance on binary logic as a fundamental universal postulate automatically excludes the four-valued Buddhist logic,\index{binary logic!as opposed to non-binary} or the seven-fold Jain predication.\index{binary logic!as opposed to non-binary} The imposition of a binary logic\index{binary logic!imposition of} system in mathematics and claiming it to be universal is another attempt at cultural hegemony, as it delegitimizes the presence of any arithmetic with non-binary logic systems. The question of domination thus remains an important aspect of modern formal Western mathematics where an academic elite controls the curriculum, the discourse as well as the spread of mathematical knowledge of a very specific kind, and adhering to a specific ideology. Raju\index{Raju, C K} explains this point: 
\begin{myquote}
``Now, deductive inference is based on logic, but which logic? Deductive proof lacks certainty unless we can answer this question with certainty. Russell\index{Russell, Bertrand} thought, like Kant,\index{Kant, Immanuel} that logic is unique and comes from Aristotle.\index{Aristotle} However, one could take instead Buddhist or Jain logic, or quantum logic, or the logic of natural language, none of which is 2-valued. The theorems that can be inferred from a given set of postulates will naturally vary with the logic used: for example, all proofs by contradiction would fail with Buddhist logic. One would no longer be able to prove the existence of a Lebesgue nonmeasurable set, for example. This conclusively establishes that the metaphysics of formal math is religiously biased, for the theorems of formal mathematics vary with religious beliefs. Furthermore, the metaphysics of formal math has no other basis apart from Western culture: it can hardly be supported on the empirical grounds it rejects as inferior!''
\hfill (Raju 2011a:277)
\end{myquote}

\medskip

\section*{Conclusion}

Indian tradition has been very clear that Gaṇita had wide applications in worldly, Vedic, and religious affairs; and its development through experimentations, new inventions, discoveries, and innovation - all stemmed from this all-pervasive need, and not merely as a commentary on the Veda-s. From ancient times to pre-colonial times, Gaṇita\index{Ganita@Gaṇita} as a {\sl Śāstra} (science) has relentlessly focused on practical issues - developing logical and efficient algorithms for problem solving, and was viewed as an application-oriented subject by its practitioners. Gaṇita-śāstra\index{sastra@\textsl{śāstra}} was practiced even among Jain and Buddhist scholars, who do not regard the Vedas as a {\sl pramāṇa}.\index{pramana@\textsl{pramāṇa}} It is also believed that geometry\index{geometry} as a discipline, most likely originated from the Gaṇita - like {\sl śāstra} of a culture resembling Vedic culture. Like any scientific discipline, Gaṇita-śāstra also operates on real entities, relies on empirical data and is falsifiable. Based on all the evidence furnished so far, Pollock’s claim - that Indian tradition did not view the discoveries and innovations of Indian arithmetic as scientific achievements and progress, but rather as recovery of past truths - must be categorically rejected.

On the other hand, it is formal mathematics which is based on abstractions, on formalism, on the notion of eternal truths and falsehoods, and on a complete rejection of the empirical world. Unlike science,\index{pramana@\textsl{pramāṇa}!anumana@\textsl{anumāna}} experimentations and inferences ({\sl pratyakṣa-\index{pratyaksa@\textsl{pratyakṣa}}pramāṇa}\index{pramana@\textsl{pramāṇa}!pratyaksa@\textsl{pratyakṣa}} and {\sl anumāna})\index{anumana@\textsl{anumāna}} play no role in formal mathematics. Since there is no room for the empirical, there is no question of empirical falsifiability\endnote{Science\index{science!falsifiability} is falsifiable, i.e. new observations can falsify previous conclusions.}. Pollock’s observation below on {\sl śāstra}-s as a fountainhead of all knowledge, can be applied equally well to modern mathematics as a fountainhead of all truths.
\begin{myquote}
``All knowledge derives from {\sl śāstra}; success ... is achieved only because the rules governing these practices have percolated down to the practitioners - not because they were discovered independently through the creative power of practical consciousness ``however far removed'' from the practitioners the {\sl śāstra} may be. As for learning the {\sl śāstra} itself, this is the necessary commencement of the tradition, and later serves to enhance the efficacy of the practice.''\hfill (Pollock 1985:507)
\end{myquote}

Formal mathematics perceives and indeed presents itself largely as commentary on the primordial axioms and laws already pre-existing, and anything new is simply a backward movement in order to better represent a divine pattern. Mathematics does not produce any new knowledge, but is a process of re-discovering and recovering external pre-existent truths. All advances in theoretical mathematics are viewed by its practitioners through an inverting lens of ideology, as renovation of recovery, since {\sl pratyakṣa-pramāṇa} and {\sl anumāna-pramāṇa} are epistemologically meaningless in the context of formal mathematics. 

\newpage

Noted Israeli historian of science,\index{science!history of} Leo Corry\index{Corry, Leo} traces the origins of ``Eternal Truth'' in modern mathematics from Hilbert\index{Hilbert, David} to the latter-day Bourbaki School,\index{Bourbaki School} and says that while earlier mathematicians believed in eternal mathematical truths, starting from mid-1930s, it came to be believed that mathematics has reached an “ultimate stage of evolution” and was unlikely to change “by any future development of this science” (Corry 1997:253).
%\begin{myquote}
%``The belief in the existence of eternal mathematical truth has been part of this science throughout history. Bourbaki, however, introduced an interesting, and rather innovative twist to it, beginning in the mid-1930s. This group of mathematicians advanced the view that mathematics is a science dealing with structures, and that it attains its results through a systematic application of the modern axiomatic method. Like many other mathematicians, past and contemporary, Bourbaki understood the historical development of mathematics as a series of necessary stages inexorably leading to its current state--meaning by this, the specific perspective that Bourbaki had adopted and were promoting. But unlike anyone else, Bourbaki actively put forward the view that their conception of mathematics was not only illuminating and useful to deal with the current concerns of mathematics, but in fact, that this was the ultimate stage in the evolution of mathematics, bound to remain unchanged by any future development of this science.''
%\hfill (Corry 1997:253)
%\end{myquote}

A final consequence of this is, as I have suggested earlier, that unlike in Gaṇita-śāstra, there can be no conception of progress in formal mathematics as in the evolutionary scale, mathematics has already reached an ``ultimate stage''; and any movement that occurs is simply a backward movement towards an eternal mathematical truth. The origins of mathematics is, on the other hand, as we have seen, {\sl mathesis}\index{mathesis@\textsl{mathesis}} (interrogating the soul to rediscover eternal truths) and not from {\sl mathema} (knowledge). The Greeks never saw any practical use of arithmetic apart from military use; arithmetic was intended for spiritual elevation only (Jowett\index{Jowett, Benjamin} 1892:585). Moreover the development of Western mathematics over the centuries has been closely linked to Biblical theology in one way or the other till the present day, and contrives to trace its origin to the mythical Euclid:\index{Euclid} a clear case of mythic self-representation. While discussing the various dimensions of hegemony in mathematics, Raju\index{Raju, C K} makes the following observation:

\vskip .1cm

\begin{myquote}
``Identifying the difficulties with math learning, and proposing a solution, does represent a major advance. But there are difficulties in implementing the solution. Various stakeholders (such as students afraid of math, or their parents) are never consulted to decide what sort of math to teach. Even scientists and engineers are rarely consulted regarding what sort of math ought to be taught to them. However, if all decisions regarding the math curriculum are left solely to math ``experts'', there is an obvious conflict of interests: for these experts were brought up on the older tradition of formal mathematics, and rejecting formal mathematics may well make their past work valueless.'' 

\hfill (Raju 2011b:283) 
\end{myquote}

\vskip .1cm

Sadly, the only form of arithmetic taught in schools and colleges is formal mathematics, which as we have shown, is theoretical, abstract, divorced from reality, dogmatic and metaphysical - all characteristics which a ``science''\index{science!definition of} should NOT have. Western mathematics therefore comes across as arcane, abstract and complicated to most non-specialists, and has become a tool of academic imperialism and cultural hegemony.\index{cultural!hegemony}

\vskip .1cm

In conclusion: Pollock makes use of selective observations\index{misinterpretation!techniques of!cherry-picking data} to form a hypothesis by using the most restrictive definition\index{restrictive definition} of {\sl śāstra} - in the sense of injunction, as applicable to Mīmāṁsā and the transcendental realm. He then overgeneralizes\index{misinterpretation!techniques of!over-generalization} - by applying this narrow definition to all {\sl śāstra}-s including those in the practical realm, since in his view, Mīmāṁsā is the most orthodox and most representative of Indian traditions, and even suggests that scientific {\sl śāstra}-s are amenable to the same logic and treatment as other {\sl śāstra}-s. Moreover, he summarily rejects all counter-evidence\index{summarily rejects all counter-evidence} because he feels that the usage of {\sl śāstra}-s in a non-restrictive, non-binding and non-injunctive manner is not representative of the whole tradition. 

Against the mass of evidence we have furnished above, it is evident that Pollock’s approach as well as chain of reasoning and conclusion are not valid universally for all {\sl śāstra}-s, especially those which rely on empirical knowledge as their fundamental principle. It may be fair to say that what may be applicable to transcendental knowledge may not be applicable to the so-called hard sciences like Gaṇita\index{Ganita@Gaṇita} or astronomy,\index{astronomy} which are heavily computational in nature.

We have also examined his assumptions of Western knowledge systems which he considers to be the gold standard, and against which, he is directly as well as indirectly benchmarking the Indian knowledge systems; and we have shown that his thesis of a theoretical, backward-looking\endnote{(Pollock 1985:507).}, dogmatic and metaphysical {\sl Śāstra}\index{sastra@\textsl{śāstra}} with mythic\endnote{(Pollock 1985:512).} self-representation, applies remarkably well to modern formal mathematics.

\begin{thebibliography}{99}
\itemsep=2pt
\bibitem[]{chapter5_item1} 
Bag, A.K. (1990) ``Ritual Geometry in India and Its Parallelism in Other Cultural Areas.'' {\sl Indian Journal of History of Science} 25(1-4). pp 4-19.

%\bibitem[]{chapter5_item2} 
%Bhatia, Rajendra (Ed.) (2010) {\sl Study in the History of Indian Mathematics}. New Delhi: Hindustan Book Agency.

\bibitem[]{chapter5_item3} 
Boesche, Roger (2003) ``Kauṭilya's {\sl Arthaśāstra} on War and Diplomacy in Ancient India.'' {\sl The Journal of Military History} 67(1). pp 9-37.

\bibitem[]{chapter5_item4} 
Cole, Julian C (2016) ``Mathematical Platonism.'' {\sl Internet Encyclopedia of Philosophy} (IEP). (\url{http://www.iep.utm.edu/mathplat/})\break 
accessed on 16 Aug 2016.

\bibitem[]{chapter5_item5} 
Corry, Leo (1997) ``The origins of eternal truth in modern mathematics: Hilbert to Bourbaki and beyond.'' {\sl Science in Context} 10(2). pp 253-296.

\bibitem[]{chapter5_item6} 
Dani, Shrikrishna G (2010) ``Geometry in the Śulvasūtras.'' pp 9-38. In Seshadri (2010).

\bibitem[]{chapter5_item7} 
Danino, Michel (2003) ``The Harappan Heritage and the Aryan Problem.'' {\sl Man and Environment} 28. pp 21-32.

\bibitem[]{chapter5_item8} 
--- (2008) ``New Insights into Harappan Town-Planning, Proportions and Units, with Special Reference to Dholavira.'' {\sl Man and Environment} XXXIII(1). pp 66-79.

\bibitem[]{chapter5_item9} 
Datta, Bibhutibhushan and Avadhesh N. Singh (1962) {\sl History of Hindu Mathematics: A Source Book. Parts I and II}. Bombay: Asia Publishing House.

\bibitem[]{chapter5_item10} 
Dutta, Amartya K (2002) ``Mathematics in Ancient India.'' {\sl Resonance} 7(4). pp 4-19.

\bibitem[]{chapter5_item11} 
Fowler, David (2002) ``Euclid.'' The Math Forum. (\url{http://mathforum.org/kb/message.jspa?messageID=1175734}) accessed on 16 Aug 2016.

\bibitem[]{chapter5_item12} 
Goldberger, Assaf (2015) ``What are mathematical proofs and why they are important? Math 216 class, University of Connecticut.'' CS 2800: Discrete Structures. (\url{http://www.cs.cornell.edu/courses/cs2800/2015fa/handouts/goldberger_proof_notes.pdf}) accessed on 16 Aug 2016.

\bibitem[]{chapter5_item13} 
Hersh, Reuben (1979) ``Some proposals for reviving the philosophy of mathematics.'' {\sl Advances in Mathematics} 31(1). pp 31-50.

\bibitem[]{chapter5_item14} 
Horsten, Leon (2016) ``Philosophy of Mathematics.'' {\sl The Stanford Encyclopedia of Philosophy} (Summer 2016 Edition). Retrieved August 2016 (\url{http://plato.stanford.edu/archives/sum2016/entries/philosophy-mathematics/}) accessed on 16 Aug 2016.

\bibitem[]{chapter5_item15} 
Jowett, Benjamin (1892) {\sl The Dialogues of Plato (The Republic, Timaeus, Critias)}. London: Oxford University Press.

\bibitem[]{chapter5_item16} 
Kant, Immanuel (1855) {\sl Critique of Pure Reason}. London: Henry G. Bohn. 

\bibitem[]{chapter5_item17} 
Kapoor, Kapil (2005) {\sl Indian Knowledge Systems}. Shimla: D.K. Print World Ltd.

\bibitem[]{chapter5_item18} 
Krantz, Steven G (2007) ``The history and concept of mathematical proof.'' 
(\url{http://www.math.wustl.edu/~sk/eo lss.pdf})

accessed on 16 Aug 2016.

\bibitem[]{chapter5_item19} 
Linnebo, Øystein (2013) ``Platonism in the Philosophy of Mathematics.'' {\sl The Stanford Encyclopedia of Philosophy} (Winter 2013 Edition). (\url{http://plato.stanford.edu/archives/win2013/entries/platonism-mathematics/}) accessed on 16 Aug 2016.

\bibitem[]{chapter5_item20} 
Lochtefeld, James (2002) {\sl The Illustrated Encyclopedia of Hinduism: N-Z}. New York: Rosen Publishing.

\bibitem[]{chapter5_item21} 
Malhotra, Rajiv (2016) {\sl The Battle for Sanskrit}. Noida: HarperCollins Publishers.

\bibitem[]{chapter5_item22} 
Pollock, Sheldon (1985) ``The Theory of Practice and the Practice of Theory in Indian Intellectual History.'' {\sl Journal of the American Oriental Society} 105(3). pp 499-519. (\url{http://www.jstor.org/stable/601525}) accessed on 16 Aug 2016.

\bibitem[]{chapter5_item23} 
--- (2005) {\sl The ends of man at the end of premodernity}. Amsterdam. Royal Netherlands Academy of Arts and Sciences.

\bibitem[]{chapter5_item24} 
--- (2011) Forms of Knowledge in Early Modern Asia: Explorations in the Intellectual History of India and Tibet, 1500--1800. Duke University Press.

\bibitem[]{chapter5_item25} 
Raju, C K (2009) ``Good-Bye Euclid!'' {\sl Bhartiya Samajik Chintan} 7(4). pp 255-264.

\bibitem[]{chapter5_item26} 
--- (2011a) ``Teaching mathematics with a different philosophy Part 1: Formal mathematics as biased metaphysics.'' {\sl Science and Culture} 77(7-8). pp 274-79. 
(\url{http://www.scienceandculture-isna.org/July-aug-2011/03%20C%20K%20Raju.pdf}) 
accessed on 16 Aug 2016.

\bibitem[]{chapter5_item27} 
--- (2011b) ``Teaching Mathematics with a Different Philosophy - II Calculus Without Limits.'' {\sl Science and Culture} 77(7-8). pp 280-285. 
(\url{http://www.scienceandculture-isna.org/July-aug-2011/04%20C%20K%20Raju2.pdf}) 
accessed on 16 Aug 2016.

\bibitem[]{chapter5_item28} 
--- (2012) {\sl Euclid and Jesus - How and why the church changed mathematics and Christianity across two religious wars}. Multiversity and Citizens International.

\bibitem[]{chapter5_item29} 
Russell, Bertrand (1956) {\sl Portraits from Memory and Other Essays}. New York: Simon and Schuster.

\bibitem[]{chapter5_item30} 
Seshadri, C S (ed) 2010 {\sl Studies in the History of Indian Mathematics}. New Delhi: Hindustan Book Agency.

\bibitem[]{chapter5_item31} 
Srinivas, M D (2015) ``The Algorithmic Approach of Indian Mathematics.'' Centre for Policy Studies. Chennai. (\url{http://iks.iitgn.ac.in/wp-content/uploads/2016/02/The-Algorithmic-Approach-of-Indian-Mathematics-MD-Srinivas-2015.pdf}) accessed on 16 Aug 2016.

\bibitem[]{chapter5_item32} 
Srinivas, M D (2016) ``On the Nature of Mathematics and Scientific Knowledge in Indian Tradition.'' Centre for Policy Studies. Chennai. (\url{http://iks.iitgn.ac.in/wp-content/uploads/2016/02/On-the-Nature-of-Mathematics-and-Scientific-Knowledge-in-Indian-Tradition-MD-Srinivas-2016.pdf}) accessed on 16 Aug 2016.

\bibitem[]{chapter5_item33} 
Srinivas, M D., K Ramasubramanian, and M S. Sriram (2014) ``Mathematics in India - From Vedic Period to Modern Times.'' National Programme on Technology Enhanced Learning (NPTEL). 
(\url{http://nptel.ac.in/courses/111101080/Downloads/01%20Overview%20(MDS).pdf}) 
accessed on 16 Aug 2016.

\bibitem[]{chapter5_item34} 
Varadpande, Manohar L (2005) {\sl History of Indian Theatre Vol. III}. Delhi: Abhinav Publication.
\end{thebibliography}

\theendnotes
\label{chapter\thechapter:end}
