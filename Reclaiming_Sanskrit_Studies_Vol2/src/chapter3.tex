\chapter{Is Practice already in {{\sl\bfseries Śāstra}\relax}-s}\label{chapter\thechapter:begin}
\vskip -10pt

\Authorline{Surya K.\footnote{pp.~\pageref{chapter\thechapter:begin}--\pageref{chapter\thechapter:end}. In: Kannan, K S (Ed.) (2018) {\sl Śāstra-s Through the Lens of Western Indology - A Response}. Chennai: Infinity Foundation India.}}
\lhead[\small\thepage\quad Surya K.]{}

\vskip -10pt

\rhead[]{\small \thechapter. Is Practice already in {{\sl Śāstra}\relax}-s\quad \thepage}

\section*{Introduction}
\index{Surya K.}

In his 1985 paper (viz.\ ``Theory of Practice and Practice of Theory in Indian Intellectual History''), Professor Sheldon Pollock {\sl concludes} that pre-modern Indian intellectuals held the belief that all knowledge of practice is contained in {\sl śāstra}-s (``theory \index{theory} of practice'').  Pollock says that this belief crippled the innovative spirit of Indian intellectuals and limited their practice to the merely uncovering and examining knowledge concealed in the {\sl śāstra}-s. He restates this conclusion multiple times in his paper showing the centrality of the conclusion to the paper's thesis. 

Here are some of the quotes from this paper:
\begin{itemize}
\itemsep=1pt
\item[(a)] The understanding of the relationship of {\sl śāstra} (``theory'') to {\sl prayoga}\index{prayoga@\textsl{prayoga}} (``practical activity'') in Sanskritic culture is shown to be diametrically opposed to that usually found in the West.

\hfill (Pollock 1985:499)

\item[(b)] Two important implications of this fundamental postulate are that all knowledge is pre-existent [...] The eternality of the Veda-s, the {\sl śāstra par excellence}, is one presupposition or justification for this assessment of {\sl śāstra}.\hfill (Pollock 1985:499)

\item[(c)] All Indian learning presents itself largely as commentary on the primordial {\sl śāstra}-s.

Logically excluded from epistemological meaningfulness are likewise experience, experiment, invention, discovery, innovation.

\hfill (Pollock 1985:515)

\item[(d)] If any sort of amelioration\index{amelioration} is to occur, this can only be in the form of a regress, a backward movement aiming at a closer and more faithful approximation to the divine pattern.

\hfill (Pollock 1985:515)

\item[(e)] From the conception of an {\sl a priori śāstra} it logically follows [...] that there can be no conception of progress of the ``forward movement from worse to better'', on the basis of innovations in practice.\hfill (Pollock 1985:515)

\item[(f)] All knowledge derives from {\sl śāstra}; success [...] is achieved only because the rules governing these practices have percolated\break down to the practitioners - not because they were discovered independently through the creative power of practical consciousness -``however far removed'' from the practitioners the {\sl śāstra} may be.
\hfill (Pollock 1985:507)

\item[(g)] We ourselves do not ``create'' knowledge, but merely bring it to manifestation from the (textual) materials in which it lies concealed from us.
\hfill (Pollock 1985:517) 
\end{itemize}

This paper critiques four excerpts from Pollock's paper which together argue for his conclusion. Summarized at the end is Pollock's core argument embedded in those excerpts and the rebuttal argued for in this paper.

\section*{I. Pollock's view:}  

In accordance with the {\sl Satkāryavāda}\index{Satkāryavāda@\textsl{Satkāryavāda}} theory of causation, Indian intellectuals believed that all knowledge pre-exists in eternal texts. As a consequence, Indian intellectuals voluntarily limited their knowledge acquisition to merely bringing knowledge to manifestation from textual materials.

\newpage

To quote Pollock:
\begin{myquote}
In traditional India, the causal doctrine associated especially with Sāṅkhya\index{Sankhya@\textsl{Sāṅkhya}} and early Vedānta\index{Vedanta@\textsl{Vedānta}} would seem to have particular relevance here ... This is the notion of {\sl Satkāryavāda}: As a pot, for example, must pre-exist\index{knowledge!pre-exist} in the clay (since otherwise it could never be brought into existence or could be brought into existence from some other material. e.g.  threads), {\sl so knowledge must pre-exist in something in order that we may derive it thence} (thus in part the postulates of the {\sl a priori} and finally transcendent {\sl śāstra}): like the clay, which {\sl ex hypothesi} must in some form exist eternally, that from which our knowledge comes must be eternal: and like the potter, {\sl we ourselves do not ``create'' knowledge, but merely bring it to manifestation\index{knowledge!manifestation} from the (textual) materials in which it lies concealed from us}. ({\sl Footnote here}: For a good synopsis of the doctrine of {\sl Satkāryavāda}, see Śaṅkara on {\sl Brahma Sūtra} 2.1.18).''  
\hfill (Pollock 1985:517) ({\sl Italics ours})
\end{myquote}

\section*{Our Response:}

In this excerpt, Pollock refers to Śaṅkara's Bhāṣya on {\sl Brahma Sūtra}\index{Brahmasūtra@\textsl{Brahmasūtra}} 2.1.18, giving the reader the distinct sense that underlined parts are either stated explicitly in, or inferred implicitly from, the {\sl Sūtra} or the {\sl Bhāṣya}.  As we will see, underlined parts cannot be inferred from either the {\sl Sūtra} or the {\sl Bhāṣya}.

Śaṅkara's {\sl Bhāṣya} elaborates on the {\sl Satkāryavāda} theory of causation (Gambhirananda\index{Gambhirananda} 1965:339-345) at length according to which, material can neither be created nor destroyed; something cannot come from nothing. Therefore, all material pre-exists in some form.  We know from common experience that a pot (an effect) is a manifestation of clay\index{clay and pot} (its cause); and that curds (an effect)\index{cause and effect} are a manifestation of milk\index{milk and curd} (their cause). In addition, there must be some special potency in milk -- which is not in clay - to manifest as curd.  Furthermore, neither the potency nor the effect is independent of the cause.  Accordingly, Śaṅkara concludes that the effect is pre-existent in the cause.

It is noteworthy that Śaṅkara\index{Sankaracarya@\textsl{Śaṅkarācārya}} does not simply invoke {\sl śruti} texts to make this claim.  Śaṅkara's {\sl Bhāṣya} grounds its reasoning on everyday common experiences, and what appear probable on that basis.  Śaṅkara's approach is common to the Indian intellectual tradition of the past.

\newpage

Indian intellectual tradition requires that {\sl śruti} texts be interpreted according to three conditions (Hiriyanna 1932:180-182):\index{Hiriyanna, M.}
\begin{enumerate}
\item Revealed truth should be new or extra-empirical ({\sl a-laukika}),\index{alaukika@\textsl{alaukika}} i.e. otherwise unattained and unattainable.\index{unattainable}

\item What is revealed should be un-contradicted ({\sl a-bādhita})\index{abādhita@\textsl{abādhita}} by any other means of knowing. Interpretation of revealed knowledge should be internally consistent.

\item Reason should foreshadow what revelation\index{revelation} teaches. That is, revealed truth must appear probable on the basis of common experiences in the empirical sphere. They serve to remove any `antecedent improbability' that may be felt to exist about the truth in question.
\end{enumerate}

Śaṅkara's {\sl Bhāṣya} on 2.1.18 clearly shows adherence to these conditions.

In the excerpt, Pollock somehow infers from {\sl Satkāryavāda} that [all] knowledge pre-exists in {\sl something} and that that something is ``{\sl textual materials}''. To quote him
\begin{myquote}
As a pot, for example, must pre-exist in the clay {\sl so knowledge must pre-exist in something}... We ourselves do not ``create'' knowledge, but {\sl merely bring it to manifestation from the (textual) materials} in which it lies concealed from us.\hfill (Pollock 1985:517) ({\sl italics ours})
\end{myquote}

Let us briefly examine the {\sl Sūtra} and Śaṅkara's {\sl Bhāṣya} to see if Pollock can find support for his inference.

{\sl Brāhma Sūtra} 2.1.18 (Gambhirananda 1965:339):
\begin{myquote}
[The preexistence and non-difference of the effect are established] \textbf{from reasoning and another Upaniṣadic text}.
\end{myquote}

We pose three questions below in order to look into the questions raised/implied by Pollock:
\begin{itemize}
\item[(a)] Could the Indian intellectual infer from the {\sl sūtra} that all knowledge comes ``from reasoning and Upaniṣadic texts''?

Sankara's {\sl Bhāṣya} on {\sl sūtra} 2.1.18 validates that the context of the {\sl sūtra} is {\sl Satkāryavāda}.  Gambhirananda\index{Gambhirananda} clarifies that the phrase ``another Upaniṣadic text'' in the {\sl sūtra} is to be understood to mean ``another passage'' from an Upaniṣadic text. Indeed Śaṅkara's Bhāṣya on {\sl Sūtra}-s 2.1.17 and 2.1.18 reference different passages from {\sl Chāndogya Upaniṣad}.\index{Upanisad@\textsl{Upanisad}!Chandogya@\textsl{Chāndogya}} Thus, according to Śaṅkara's Bhāṣya, {\sl Sūtra} 2.1.18 is saying that ``Pre-existence of effect and non-difference of effect from the cause are established from reasoning and another passage in {\sl Chāndogya Upaniṣad}.''  There is simply no scope for the Indian intellectual to infer from tradition (which is to say Śaṅkara's {\sl Bhāṣya}) - that the {\sl Sūtra} is saying that all knowledge comes from reasoning and ``textual'' materials.

\item[(b)] If Indian intellectuals believe that knowledge is only discovered and not created, does that belief limit their ``creative'' capability?

History illustrates that, even in purely creative fields such as art, such a belief did not hinder ``creative'' excellence.  Modern day researchers in science and mathematics\index{mathematics} commonly believe that they only discover -- and not create - knowledge in their fields.  Modern day engineers and craftsmen only change ``name and form'' of materials and energy they work with, using principles derived from their knowledge of science and mathematics. There is nothing limiting to an intellectual's practical ``creative'' capability\index{creative capability} if the intellectual believes that knowledge is only discovered and not created.

But Pollock does not stop there.  He says that, in accordance with {\sl Satkāryavāda},\index{Satkāryavāda@\textsl{Satkāryavāda}} Indian intellectuals believed - that all knowledge, including that of practice not only pre-exists, but that it pre-exists in something; specifically in textual materials.

\item[(c)] Does Śaṅkara's {\sl Bhāṣya} allow Indian intellectuals to infer that all knowledge pre-exists \index{knowledge!pre-exist} in {\bf textual} materials?

In conformance with his {\sl Bhāṣya}, Śaṅkara would agree that {\sl pāramārthika}\index{paramarthika@\textsl{pāramārthika}} knowledge pre-exists in {\sl śruti} texts, but that {\sl laukika}\index{laukika@\textsl{laukika}} (worldly) knowledge comes from common experience and not from any texts. Śaṅ\-kara's {\sl Bhāṣya} neither says anywhere nor gives scope for the Indian intellectual to infer that {\sl laukika} knowledge pre-exists in textual materials.  Śaṅkara does not - and would not - make such a claim, simply because he would find the claim unreasonable and untenable - based on common experience. For example, Śaṅkara's reasoning supports the idea that milk - and not clay -- manifests as curd is something that follows from common experience, and not from any texts. In fact, the claim that all knowledge pre-exists in texts is completely unnecessary for, and even extraneous to, the {\sl Satkāryavāda} theory of causation. In particular, {\sl Satkāryavāda} does not require {\sl laukika} knowledge, which is available to one and all through common experiences, to pre-exist in textual form. Thus, there is no credible reason for Pollock to infer that {\sl Satkāryavāda} led Indian intellectuals to believe that all knowledge pre-exists in textual materials.

Thus, there is no justification for Pollock to insert the word ``textual'' into the context.

While there is no justification, inserting the word ``textual'' does serve an important purpose for Pollock: it supports his conclusion that Indian intellectual practically lived by the belief that all knowledge, including that of practice, pre-exists in eternal ``textual materials'', an absurd but foundational assertion in Pollock's 1985 {\sl Śāstra}-s paper.
\end{itemize}


\section*{II. Pollock's view:}

Indian intellectuals believed that {{\sl śāstra}\relax}-s\index{sastra@\textsl{śāstra}} of practice are divine in origin; and therefore that they are perfect. Since {{\sl śāstra}\relax}-s are perfect, there is no scope for improving them. Therefore, there is no scope for original research.

To quote Pollock:
\begin{myquote}
Our last examples of the individual {\sl śāstra} as deriving from some {\sl primordial text}

[...]

Finally, the most important of the medical texts, the {\sl Caraka-saṃhitā},\index{Caraka-samhita@\textsl{Caraka-samhitā}} claims to be Agniveśa's transcription of the teachings of Ātreya, which were received, through Bhāradvāja, Prajāpati, and the Aśvins, ultimately from Brāhma, while the second major text, the {\sl Susruta-saṃhitā}, similarly begins with a mythological introduction concerning the origin of medicine, and claims that ``Brāhma it was who enunciated this Vedāṅga,\index{Vedanga@\textsl{Vedāṅga}} this eight-fold Āyurveda.\index{Ayurveda@Āyurveda}''

[...]

\newpage

From the conception of an a {\sl priori śāstra} it logically follows [...] that there can be no conception of progress of the forward movement from worse to better, on the basis of innovations in practice. 

\hfill (Pollock 1985:513, 515) ({\sl Italics ours})
\end{myquote}

\section*{Our Response:}

In the above excerpt, Pollock presents {\sl Caraka-saṃhitā} and {\sl Susruta-saṃhitā} as two {\sl śāstra}-s of practice which claim to have divine origin.  With this excerpt, Pollock is persuading the reader\index{Pollock!persuading the reader} to accept that {\sl śāstra}-s of practice claimed divinity to assert their impeccable epistemic credentials. Pollock wants the reader to accept as a consequence that pre-modern Indian intellectuals never upgraded {\sl śāstra}-s of practice and voluntarily limited themselves to merely bringing knowledge to manifestation from the {\sl śāstra}-s.

In this discussion, we see that claims of divine origin did not render {\sl śāstra}-s of practice unquestionable. We will look at an example of how tradition intentionally modified a divine {\sl śāstra} of practice. We will also see that Indian intellectuals contributed to vibrant original research long after the various {\sl śāstra}-s were established in the tradition.

Mīmāṃsā philosophers divided testimonial knowledge into {\sl pauruṣeya} (authored) and {\sl apauruṣeya} (eternal, authorless) (Sharma 2013:220-222).\index{Sharma, Chandradhar} For Mīmāṃsā philosophers, Vedas are {\sl apauruṣeya}; eternal and authorless, therefore their epistemic credentials are impeccable. All other texts are creations of human authors and such works are subject to defects, doubts, and errors.

While Mīmāṃsā\index{Mīmāṃsā@\textsl{Mīmāṃsā}} philosophers claimed impeccable credentials for the Veda-s, Nyāya philosophers located assent-worthiness of {\sl all} knowledge in the epistemic credentials of the speaker as mentioned in the following {\sl sūtra} (Ganeri 2010:10):\index{Ganeri, Jonardon}
\begin{myquote}
{{\sl Nyāya Sūtra}\relax}\index{Nyaya Sutra@\textsl{Nyāya Sūtra}}  2.1.68 ``Just as with the assent-worthiness of medical treatises and {\sl mantra}-s, the assent worthiness of the Veda is a function of the credibility of the testifier.
\end{myquote}

This {\sl sūtra} is implying that there was no controversy insofar as assent-worthiness of medical treatises ({\sl Caraka-saṃhitā}\index{Caraka-samhita@\textsl{Caraka-samhitā}} and {\sl Suśruta-saṃhitā}) was concerned; their assent-worthiness depended on the credibility of their testifier. While Mīmāṃsā and Nyāya philosophers disagreed on epistemic credentials of {\sl śruti} texts, they both agreed that assent-worthiness of {\sl śāstra}-s of practice depended on the credibility of the testifier. Thus, the assent-worthiness of a {\sl śāstra} of practice could be questioned in the Indian intellectual tradition even if that {\sl śāstra} claimed a divine origin.

{\bf (a) History of revisions: Divine {\sl\bfseries śāstra}-s of practice are not written on stone tablets}

In the Indian tradition, all knowledge is considered divine.  Thus, attributing divine origin to a {\sl śāstra} of practice is an expression of respect.  It does not follow that the knowledge of that {\sl śāstra} is perfect or that it is complete.  For example, {\sl Caraka-saṃhitā}\index{Caraka-samhita@\textsl{Caraka-samhitā}} was revised first by Caraka, and later by Dṛḍhabala several centuries after Caraka.

Revisions to {\sl śāstra}-s of practice of divine origin, while not common, were certainly not prohibited in the tradition as seen by Dṛḍhabala's (5$^{\text{th}}$ century CE) comment:
\begin{myquote}
A redactor expands what was stated [too] briefly, abbreviates what is too extensive and [thereby] makes an ancient corpus of knowledge ({\sl tantra}) new again. Therefore Caraka, who was exceedingly intelligent, revised this highest corpus of knowledge.\hfill (Maas 2010:2)
\end{myquote}

In contrast, tradition would not accept similar redactions to {\sl śruti} texts.

While citing Pollock's 1985 {\sl śāstra}-s paper, Phillip Maas\index{Maas, Phillip} (Maas 2010) notes that the history of literary works shows that {\sl Caraka-saṃhitā} was indeed updated later.  Dṛḍhabala explains\endnote{Michelangelo held the belief that statues pre-existed in the marble that he was carving from; however, the belief did not stop Michelangelo from carving some of the most celebrated sculptures in history.  ``All the unfinished statues at the Accademia [Gallery in Florence, Italy] reveal Michelangelo's approach and concept of carving. Michelangelo believed the sculptor was a tool of God, not creating but simply revealing the powerful figures already contained in the marble. Michelangelo's task was only to chip away the excess, to reveal'' (Accademia Gallery in Florence 1520-1540).}\endnote{Dṛḍhabala writes:In this corpus of knowledge [composed] by Agniveśa, which was revised by Caraka, seventeen chapters as well as the Kalpa- and the Siddhi[sthāna-chapters] were found to be missing. These remaining [chapters] of that important corpus of knowledge were properly composed by Dṛḍhabala, son of Kapilabala, in order to complete it.\hfill (Maas 2010:4)} that the copy of {\sl Caraka-saṃhitā} as available to him was incomplete.  In a bid to complete the text, after propitiating Lord Siva, Dṛḍhabala says that he {\sl added seventeen chapters} to {\sl Caraka-saṃhitā} on the subject of medical substances.

Commenting on Dṛḍhabala's own explanations for changes to {\sl Caraka-saṃhitā}, Maas writes: 
\begin{myquote}
Dṛḍhabala tells us that the older work was in need of revision. The reason for this, however, is not some deficiency in content. [...] The fact that Dṛḍhabala does not refer to a qualitative change of medical knowledge in time is not surprising at all if we remember the traditional account of how Āyurveda\index{Ayurveda@Āyurveda} came to be known to mankind.\hfill (Maas 2010:2)
\end{myquote}

{\sl Caraka-saṃhitā}'s claim to divine origin did not prevent future modifications (however, the claim helped limit the extent of modifications).  Furthermore, knowledge underlying those modifications came from practical discoveries.  In other words, Indian intellectuals did not deem {\sl Caraka-saṃhitā} as an eternal {\sl śāstra} beyond question; they did not consider {\sl Caraka-saṃhitā} ``text'' as the only, or the final, source of knowledge.

{\sl Caraka-saṃhitā}'s\index{Caraka-samhita@\textsl{Caraka-samhitā}} claim of divine origin restrained modifications, ensuring its survival to our times. In addition, Indian tradition did not have ``ownership rights'', ``copyrights'', or ``editorial rights''.  Taking over someone else's work and developing\endnote{Gray's {\sl Anatomy} is the Western ``{\sl śāstra} of practice'' first published in 1858. {\sl Gray's Anatomy} became an authoritative testimony for anatomy and it still remains so after all these years. The second edition, corrected and revised by Gray, was published before his death in 1860. Following Gray's death, ``rights'' to {\sl Gray's Anatomy} changed hands. The newest US edition is the 41st edition published in 2015.  UK editions of {\sl Gray's Anatomy} have been published in parallel to the US edition.} new editions is alien to Indian tradition.  However, absence of future editions did not translate into lack of original research in future.  Pollock's paper seems to make the unstated assumption that new knowledge production can only happen through revisions of existing texts. As we see next, authority of {\sl Caraka-saṃhitā} and {\sl Susruta-saṃhitā} certainly did not suppress the vibrancy of intellectual culture of original discoveries in India.

{\bf (b) Dynamic history of vibrant original medical research}

In his review of Meulenbeld's monumental work {\sl A History of Indian Medical Literature}\endnote{Meulenbeld's monumental work {\sl A History of Indian Medical Literature} (HIML) (Meulenbeld 1999-2002) comes in 5 volumes and spans over 4000 pages.  Volume IA covers classical {\sl śāstra}-s of medical practice. Volumes IIA and IIB show a vibrant intellectual history of original medical research from 600 CE until recent times. Volume 1A: 7-180 covers {\sl Caraka-saṃhitā}.  Volume 1A: 130-141 covers Dṛḍhabala exclusively.} ({\bf HIML}) (Meulenbeld 1999-2002), Dominik Wujastyk\index{Wujastyk, Dominik} writes:
\begin{myquote}
The early compendia called `the great triad ({\sl bṛhat-trayī})'--those ascribed to Caraka, Suśruta, and Vāgbhaṭa\index{Vagbhata@\textsl{Vāgbhaṭa}}---are works that at least most Indologists have heard of, if not studied. Volumes IIa and IIb of HIML will reveal to many for the first time {\sl the staggering volume and diversity of scientific literary production in the post-classical period. They survey the thousands of Indian medical works written from about AD 600 up to the present.} 

Authors in the sixteenth, seventeenth and eighteenth centuries produced a rich and important crop of diverse medical treatises, often describing new diseases, new theories, new treatments, and new medicines. {\sl These facts decisively contradict the two common opinions that post-classical Indian medicine was static and unchanging, and that medical creativity entered a dark age after Vāgbhaṭa.} 

\hfill (Wujastyk 2004:405) ({\sl italics ours})
\end{myquote}

Indian intellectuals were not merely ``bringing knowledge to manifestation from {\sl śāstra}-s of practice'' as Pollock demeaningly presents.  Long after medical {\sl śāstra}-s were fully established in the tradition, the tradition of literary production of medical research continued unabated in the post-Classical period up to Colonial times.

\newpage

Next, we will see that the Indian intellectual actually believed that the quest for worldly knowledge is, in fact, not ``centered on texts'' at all.

\section*{III. Pollock's view:}

Indian intellectuals inferred that {{\sl śāstra}\relax}-s of practice are {\sl a priori} texts (eternal), and as a consequence are perfect and all-encompassing.\break  Therefore, experience and experimentation are useless for discovering {{\sl laukika}\relax} (worldly) knowledge.

To quote Pollock:\index{Pollock, Sheldon}
\begin{myquote}
Logically excluded from epistemological meaningfulness are likewise experience, experiment, invention, discovery, innovation.  According to his own self-representation, there can be for the thinker no originality of thought, no brand-new insights, notions, perceptions, but only the attempt better and more clearly to grasp and explain the antecedent, always already formulated truth. All Indian learning, accordingly, perceives itself and indeed presents itself largely as commentary on the primordial {\sl śāstra}-s.\hfill (Pollock 1985:515)
\end{myquote}

\section*{Our Response:}

From early times in ancient India, Indian intellectuals recognized that there are different means for knowing different kinds of knowledge. {\sl ``Pramā''} means valid knowledge.\index{knowledge!valid}  {\sl ``Pramāṇa''} is the means for {\sl pramā}.\index{pramā}  Without {\sl pramāṇa}\index{pramana@\textsl{pramāṇa}} there cannot be any {\sl pramā}.  For any {\sl pramā}, there can be {\sl one and only one pramāṇa}\endnote{To avoid misunderstanding, it is important to clarify at this point that we are not saying that all {\sl laukika} knowledge comes from {\sl pratyakṣa pramāṇa}. For example, when smoke is seen in the mountains but not the source fire, one relies on {\sl anumāna-pramāṇa} to draw the inference that there is fire. This simple example shows us that {\sl anumāna-pramāṇa} is an integral part of sourcing {\sl laukika} knowledge.  In addition, as will be discussed in detail later, authoritative testimony ({\sl āpta-upadeśa}) is also an important {\sl pramāṇa} for {\sl laukika} knowledge.}.  Indian knowledge systems collectively recognize six {\sl pramāṇa}-s. Each knowledge system relied on a subset of the six {\sl pramāṇa}-s as the means for its valid knowledge.

Three {\sl pramāṇa}-s are of concern in this article:
\begin{myquote}
{{\sl Pratyakṣa-pramāṇa}\index{pratyaksa@\textsl{pratyakṣa}}} means the means for knowledge by direct perception. 

{{\sl Anumāna-pramāṇa}\index{anumāna@\textsl{anumāna}}} means the means for knowledge based on inference or reasoning.  

Authoritative testimony {({\sl āpta-upadeśa})\index{apta-vacana@āpta-vacana}} means the testimonial knowledge of assent-worthy men.
\end{myquote}

As an illustration, imagine that you are standing on a roadside looking at a house. Smoke coming out of its smoke stack is directly visible to your naked eye but the fire causing the smoke cannot be seen as it is inside the house.  You have learnt from common experience that there cannot be smoke without fire.  Therefore, you infer that there is fire in the house.  In this case, the means for knowing about the smoke is {\sl pratyakṣa-pramāṇa}, and the means for knowing about the fire is {\sl anumāna-pramāṇa}.  Since the knowledge of smoke is gained by {\sl pratyakṣa-pramāṇa}, direct perception is the sole authority of this knowledge; all other means for knowledge of smoke are secondary; they have no authority.  

In case fire is directly perceived as when we can see it through glass walls, then the {\sl pramāṇa} for knowledge of fire is {\sl pratyakṣa} and not {\sl anumāna}.  

In either case, any way, {\sl śruti} texts are not the {\sl pramāṇa} for either smoke or fire.  For all {\sl laukika}\index{laukika@\textsl{laukika}} (worldly) knowledge, {\sl pratyakṣa} and {\sl anumāna} are the appropriate\endnote{Indian intellectual tradition was clear on the concept of {\sl pramāṇa} from early times.  {\sl Sāṁkhya-kārikā} of Īśvara-kṛṣṇa is the earliest extant text of the Sāṅkhya philosophy. Verses 4-6 from his translation show the essence of epistemological thought in ancient Indian intellectual tradition: Perception, inference and right affirmation are admitted to be threefold proof; for they (are by all acknowledged, and) comprise every mode of demonstration. It is from proof that belief of that which is to be proven results. Perception is ascertainment of particular objects. Inference, which is of three sorts, premises an argument, and deduces that which is argued by it. Right affirmation is true revelation ({\sl āpta-vacana} and {\sl śruti}, testimony of reliable source and the Veda-s).  Sensible objects become known by perception; but it is by inference or reasoning that acquaintance with things transcending the senses is obtained. A truth which is neither to be directly perceived, nor to be inferred from reasoning, is deduced from {\sl āptava-cana} or {\sl śruti}.\newline The English translation of {\sl Sāṅkhya Kārikā} by Henry Colebrooke was published in 1837.} {\sl pramāṇa}-s. In matters of knowledge beyond perception, {\sl pratyakṣa} cannot be a {\sl pramāṇa}.

(a) Śaṅkara unequivocally says that {{\sl śruti}\relax} is not a {{\sl pramāṇa}\relax} for {{\sl laukika}\relax} knowledge.

Śaṅkara explains this in his {\sl Bhāṣya} on {\sl Bhagavād Gītā}\index{Bhagavadgita@\textsl{Bhagavadgīta}} 18.66:\index{Sankaracarya@\textsl{Śaṅkarācārya}}
\begin{myquote}
{\textsl{Śruti is an authority only in matters not perceived by means of ordinary instruments of knowledge, such as pratyakṣa or immediate perception;}} [...] A hundred {\sl śruti}-s may declare that fire is cold or that it is dark; still they possess no authority in the matter. If {\sl śruti} should at all declare that fire is cold or that it is dark, we would still suppose that it intends quite a different meaning from the apparent one; for its authority cannot otherwise be maintained: we should in no way attach to {\sl śruti} a meaning which is opposed to other authorities or to its own declaration.

\hfill (Sastry 1977:513) ({\sl italics ours})
\end{myquote}

Śaṅkara emphasizes the importance of {\sl pratyakṣa-pramāṇa} for knowledge that pertains to the physical world when he says, ``A hundred {\sl śruti}-s may declare that fire is cold or that it is dark; still they possess no authority in the matter.''  In particular, Śaṅkara does not say anywhere that all knowledge is in {\sl śruti} texts. Instead, Śaṅkara says, ``{\sl Śruti} is an authority only in matters not perceived by means of ordinary instruments of knowledge, such as {\sl pratyakṣa} (immediate perception)''. 

Thus, the real issue is not whether all knowledge is in {\sl śruti} texts, but whether {\sl śruti} texts have authority on all knowledge.  Śaṅkara is saying unequivocally that {\sl śruti} texts do not have authority on knowledge where {\sl pratyakṣa} is the {\sl pramāṇa}. Therefore, ``texts'' cannot limit Hindu minds in {\sl laukika}\index{laukika@\textsl{laukika}} matters.

Thus, Śaṅkara's {\sl Bhāṣya} on {\sl Bhagavād Gīta} 18.66 debunks Pollock's conclusive assessment viz. Indian intellectuals excluded experience and experimentation from epistemological meaningfulness. (Pollock 1985:515):
%\begin{myquote}
%Logically excluded from epistemological meaningfulness are likewise experience, experiment, invention, discovery, innovation.  According to his own self-representation, there can be for the thinker no originality of thought, no brand-new insights, notions, perceptions, but only the attempt better and more clearly to grasp and explain the antecedent, always already formulated truth.
%\end{myquote}

Pollock's harsh assessment ignores {\sl implications of} the pioneering efforts in the intellectual traditions of India to systematize\index{knowledge!systematize} knowledge based on the nature of knowledge, and the means for arriving at valid knowledge.  Pre-modern Indian intellectuals believed in {\sl pramāṇa}\index{pramana@\textsl{pramāṇa}} theory and, as a result, understood that discovery of {\sl laukika} knowledge was not confined to any ``texts''. 

In contrast, in the Middle Ages, Western intellectual tradition vested\endnote{Augustine (5$^{\text{th}}$ century CE) took the position that faith comes first if reason disagreed with scripture. Augustine held that the Church was the ultimate arbiter to decide how reason should conform to scripture. Thomas Aquinas (13$^{\text{th}}$ century CE) introduced to the Western world the notion that philosophy (reason, secular natural laws) has its own useful place separate from scriptural knowledge (religious eternal laws). Scriptural knowledge has certainty; unlike scriptures, philosophy can be erroneous. Thus, neither Augustine nor Aquinas allowed reason to independently evaluate {\sl laukika} claims in scripture.  On Aquinas' views on the relationship between theology and reason, Jan Aerstson writes:
\begin{myquote}
{\fontsize{8pt}{10pt}\selectfont
And even concerning those truths about God which human reason is able to attain, divine revelation is not superfluous, for those truths are known only to a few people, and mingled with a great deal of error.  For these reasons theology, a rational inquiry based on revelation, is necessary.\hfill (Aertsen 1993:34-35)

...

The first principle is that there is harmony between philosophy, guided by the light of natural reason, and theology, guided by the light of faith. ...``If, however, in the writings of the philosophers one finds anything contrary to faith, it is not philosophy, but rather an abuse of philosophy stemming from a defect of reason.''}
\end{myquote}
In other words, scriptures are never wrong even if the knowledge in scriptures differs with {\sl laukika} experience.  Philosophy and reason then are useful insofar as they support scriptures.} scriptures with higher authority and assent-worthiness over reason (even in matters of {\sl laukika} knowledge) when the two were in conflict. To this day, Christian apologists\index{Christian apologists} continue to affirm scriptural beliefs over established science when the two are in conflict in matters of {\sl laukika} knowledge.

The subject matter of {\sl pramāṇa}\index{pramana@\textsl{pramāṇa}} is known as epistemology\index{epistemology} in Western philosophy.\index{Western philosophy}  Epistemology arrived in Western thought relatively recently (apart from early Greek efforts to separate {\sl scientia} (principles) from {\sl opinio} (opinions of authority))\endnote{Nancey Murphy writes (Murphy 1990:4): The crisis consisted in the simultaneous erosion of both of the epistemic categories available ...{\sl scientia} and {\sl opinio}. Trouble with {\sl scientia} had begun in the late medieval period.  The voluntarists' elevation of divine omnipotence and freedom threatened to narrow {\sl scientia} to triviality: What could be deduced about the natural order if God could intervene to change it at any time?  The only consequences one could rely on, it seemed, were those of the law of noncontradiction.}. There were no systematic efforts to separate worldly experiential ({\sl laukika}) knowledge from metaphysical ({\sl alaukika}) knowledge in the Western intellectual tradition until the times of empiricists\endnote{Empiricists hold that sensory experience is the only means of knowledge. While rationalists hold that pure thought can be a means of knowledge, empiricists reject that possibility.} viz. John Locke\index{Locke, John} (1632-1704) and David Hume (1859-1952).\index{Hume, David}  Incidentally, David Hume is known to have had knowledge of Buddhist philosophical views (Gopnik 2009).\index{Gopnik, Alison}

Church publicity opposed discoveries of {\sl laukika} knowledge (for example heliocentrism,\index{heliocentrism} the theory of evolution by natural selection, and the big bang theory\index{big bang theory}) which conflicted with the religious scripture: Galileo\index{Galileo} was persecuted by the Church for championing new discoveries of {\sl laukika} knowledge that conflicted with scriptural knowledge on {\sl laukika} matters.  Brian Baigrie\index{Baigrie, Brian} writes:
\begin{myquote}
The tragedy that descended on Galileo\index{Galileo} has been described in many places. Briefly, he was warned in 1616 by the Inquisition to cease teaching the Copernican\index{Copernicus} theory, for it was now held ``contrary to Holy Scripture.'' [...] Galileo was ordered by the Pope to travel to Rome where he was confined for a few months, threatened with torture, and forced to make an elaborate formal renunciation of the Copernican theory.  He was sentenced to perpetual confinement and forbidden to publish anything on Copernicanism.\hfill (Baigrie 2002:54)
\end{myquote}

Given the belief in {\sl ex nihilo} creation (i.e., creation from nothing), Western civilization had to wait until Enlightenment to discover the law of conservation of matter that Satkāryavādin-s\index{Satkaryavada@\textsl{Satkāryavāda}} intuitively understood\endnote{Rajiv Malhotra wrote an illuminating article about Vivekananda's contributions to knowledge of the West in celebration of Vivekananda's 150$^{\text{th}}$ birth anniversary.  Vivekananda says: 
\begin{myquote}
{\fontsize{8pt}{10pt}\selectfont
``Thousands of years ago, it was demonstrated by Kapila, the great father of all philosophy, that destruction means going back to the cause.  If this table here is destroyed, it will go back to its cause, to those fine forms and particles which, combined, made this form which we call a table. If a man dies, he will go back to the elements which gave him his body; if this earth dies, it will go back to the elements which gave it form.  This is what is called destruction, going back to the cause.  Therefore, we learn that the effect is the same as the cause, not different. It is only in another form.''\hfill (Malhotra 2013:568)}\relax
\end{myquote}
Vivekananda says:
\begin{myquote}
{\fontsize{8pt}{10pt}\selectfont
``The effect is the cause manifested.  There is no essential difference between the effect and the cause ... We have seen that everything in the universe is indestructible. There is nothing new; there will be nothing new.''\hfill (Malhotra 2013:567)}\relax
\end{myquote}} to be true for over two millennia before. In spite of developments in Western epistemology, conflict between science and scripture continued unabated. Science did not always enlighten Abrahamic religious\index{Abrahamic religion} views in {\sl laukika} matters.

Even in the 21$^{\text{st}}$ century, educated Christians and Muslims routinely make assertions in {\sl laukika} matters with religious scriptures as their {\sl pramāṇa}.  A prominent example is the argument for intelligent design\endnote{For example, Dembski argues extensively - using his background in mathematics and computer science - for how nature shows the hand of an intelligent designer and how ``God's design is accessible to scientific inquiry'' and that it is ``empirically detectable''.\hfill (Dembski 1999)} and against the theory of evolution by natural selection.  Because Abrahamic religious scriptures took a position on creation of Adam\endnote{One of the memorable exhibits at the Vatican is Michelangelo's Renaissance fresco painting on the ceiling of The Sistine Chapel. The painting depicts creation of Adam - the very first man - by God.}- the first man - by God, many followers of those religions reject\endnote{Belief in creation of Adam, the first man, by God is core to the Christian doctrine of ``Original Sin'' and to the related concept of salvation by faith and grace.  Rajiv Malhotra coined the term history-centrism (Malhotra 2011) to highlight the inextricable link between Abrahamic religious core doctrinal beliefs and history (including the ``history'' of creation of the first man in God's image). (Barooah 2012) ``Regular church attendance is strongly positively correlated with believing in creationism, and negatively correlated with believing in theistic evolution and evolution [by natural selection]. Among those who attend the church weekly, two-thirds believe in creationism, 25 percent believe in theistic evolution and a mere 3 percent believe in evolution.''  

Between 1982 and 2014, Gallup poll surveys found that more than 40\% of adults in the U.S. still believe that man was created by God less than 10,000 years ago. (Newport 2014)} extensive evidence ({\sl pratyakṣa-pramāṇa}) from a wide variety of scientific disciplines such as paleontology, geology, genetics, and evolutionary biology - thus undermining the theory of evolution by natural selection that explains known evidences ({\sl anumāna-pramāṇa}) while also making predictions yet to be confirmed by future discoveries. It was only two years ago (2014) that the Pope\index{Pope} made a nominal gesture towards reconciliation with the theory of evolution and the big bang theory\index{big bang theory} (Withnall 2014).\index{Withnall, Adam}

In contrast, Indian tradition has no issues with new discoveries of {\sl laukika} knowledge which contradict existing {\sl śāstra}-s.  We know that Bhāskara,\index{Bhāskara} Āryabhaṭa,\index{Aryabhata@Āryabhaṭa} Varāhamihīra,\index{Varahamihira@\textsl{Varāhamihīra}} and others made their intellectual pursuits in astronomy while being within the milieu of Indian tradition.  For example, there are no records that Āryabhaṭa underwent mental conflict when his discovery of the mechanism of eclipses differed from corresponding ideas in established \hbox{{\sl śāstra}-s} (e.g. {\sl Vedāṅga}\index{Vedanga@\textsl{Vedāṅga}} {\sl Jyautiṣa}). As Śaṅkara said, \index{Sastry, Alladi Mahadeva} ``We should in no way attach to {\sl śruti} a meaning which is opposed to other authorities or to its own declaration'' (Sastry 1977:513). Indian intellectual tradition is clear on who has authority over {\sl laukika} knowledge.  If a scientific discovery ({\sl pratyakṣa-pramāṇa}) from an assent-worthy person differs with interpretations of {\sl śruti} texts, tradition requires that {\sl śruti} text be reinterpreted to stay consistent.

\section*{IV. Pollock's view:}

If {{\sl śāstra}\relax}-s of practice are not truly eternal texts then they cannot be {{\sl pramāṇa}\relax}.\index{pramana@\textsl{pramāṇa}}  Therefore, they had no choice but to claim divine origin.

To quote Pollock:\index{Pollock, Sheldon}
\begin{myquote}
The medical tradition, which as we saw shares the paradigmatic mythic conception of the origins of knowledge, offers an epistemological analysis that may be extended to other {\sl śāstra}-s in its discussion of the {\sl pra\-māṇa āpta-vacanam},\index{pramana@\textsl{pramāṇa}} ``authoritative testimony.''  After defining and describing the various ``sources of valid knowledge,'' the {\sl Caraka-saṃhitā}\index{Caraka-samhita@\textsl{Caraka-samhitā}} remarks, ``Of these three ways of knowing, {\sl the starting point is the knowledge derived from authoritative instruction. At the next step, it has to be critically examined by perception and inference. Without there being some knowledge obtained from authoritative instruction, what is there for one to examine critically by perception and inference?''}

{{\sl Here we are given what seems essentially an epistemological response to the paradox}} [...] Since theoretically no one is exempt from the paralyzing effects of this paradox, it is impossible to imagine how a body of knowledge such as medicine could ever have developed and been transmitted without positing the existence of some prior beginningless and unbroken ``authoritative instruction.'' {\sl This enables the student to escape the circle by having the scope and object of his discipline defined for him, and learning what in fact it is that he must bring his powers of perception and inference to bear on.}
\hfill (Pollock 1985:516-517) ({\sl italics ours})
\end{myquote}

\section*{Our Response:}

In order to properly understand Pollock's argument, it is helpful to lay out his argument in three parts:

{\bf Part 1:}
According to Pollock, {\sl Caraka-saṃhitā} (the Āyurveda {\sl śāstra}) says:
\begin{myquote}
Authoritative instruction ({\sl śāstra}) is the starting point of medical knowledge. 

At the next step, knowledge of {\sl śāstra} is examined by direct perception and inference.

Without the {\sl śāstra}, there is nothing to really examine by perception and inference.  
\end{myquote}

\newpage

Pollock argues that students of {\sl Caraka-saṃhitā} accordingly limited their own intellectual pursuits to merely examining knowledge concealed in the {\sl śāstra}.  This view of Pollock is clear from the italicized text in the excerpt: The scope and object of discipline is studying the {\sl śāstra} by bringing his powers of perception\index{perception} and inference\index{inference} to bear on it; that is, subordinating perception and inference to {\sl śāstra}.

{\bf Part 2}: {\sl Caraka-saṃhitā},\index{Caraka-samhita@\textsl{Caraka-samhitā}} the authoritative instruction, confers upon itself the status of a {\sl pramāṇa}, an authoritative text.  However, authoritative instruction, given its human  origin, has to be ultimately sourced from perception and inference.  As {\sl laukika}\index{laukika@\textsl{laukika}} knowledge, where else can it come from? In order to avoid this paradox of one {\sl pramāṇa} depending on another {\sl pramāṇa}, authoritative instruction has to be somehow established as being independent of other {\sl pramāṇa}-s.\index{pramana@\textsl{pramāṇa}}

{\bf Part 3}: The only way that an authoritative instruction can have a standing as a {\sl pramāṇa} is, if it is posited to be a beginningless text which is to say it is authorless, which amounts to saying it has a divine origin.  Pollock calls this the ``epistemological response to the paradox''. {\sl Caraka-saṃhitā} indeed claims its origin in Brāhma (see the second excerpt).

Pollock misrepresents {{\sl Caraka-saṃhitā}\relax} by selectively quoting\index{Pollock!selective quotation} from its commentary, excluding its important context, thereby altering its meaning.

Pollock quotes only a specific part of the commentary from {\sl Caraka-saṃhitā} presented in Debiprasad Chattopadhyaya's\index{Chattopadhyaya, Debiprasad} book {\sl Science and Society in Ancient India} (1978).  Fortunately, Chattopadhyaya's book provides extensive commentary surrounding the small quote that Pollock used. As we look at the broader context of the quote, we see that Pollock's argument is untenable in the context of what {\sl Caraka-saṃhitā} says: Why did Pollock then selectively quote from {\sl Caraka-saṃhitā} deliberately excluding a few words thus materially altering the meaning?

Firstly, by ``authoritative instruction'', {{\sl Caraka-saṃhitā}\relax} does not mean a beginningless text we substantiate by quoting.

{\sl Caraka-saṃhitā}:
\begin{myquote}
``[...] authoritative instruction means knowledge imparted by authoritative persons. Authoritative persons, again, are those who possess undisputed knowledge and memory, the technique of classification and whose observations are not affected by subjective factors called likes and dislikes. Because of being thus characterized, what they say is authoritative.  By contrast, the words coming from persons that are inebriated, insane, stupid, subjectively inclined and given to half-truths are unauthoritative.''\hfill \hbox{(Chattopadhyaya 1978:89)}
\end{myquote}

{\sl Caraka-saṃhitā} is saying that authoritative instruction comes from authoritative persons ({\sl āpta-upadeśa})\index{apta@āpta} who possess undisputed knowledge and memory. Pollock leaves out this definition of authoritative instruction - which is {\sl the exact opposite of what Pollock is persuading the reader} to understand ``authoritative instruction'' to mean: ``some prior beginningless and unbroken instruction''!

{\sl Caraka-saṃhitā} holds that full-knowledge for diagnosis cannot be obtained from any one of the three {\sl pramāṇa}-s: authoritative testimony, perception, and inference.

Let us look at the explanation in {\sl Caraka-saṃhitā} immediately {\sl preceding the isolated quote} used by Pollock:
\begin{myquote}
Three indeed are the modes of ascertaining the specific nature of a disease. These are 

(a) authoritative instruction ({\sl āpta-upadeśa}),\\ 
(b) perception ({\sl pratyakṣa}), and\\ 
(c) inference ({\sl anumāna}).

[...]

The diagnosis of a disease is faultless only after the disease has been fully examined in all its aspects {\sl with the help of these three ways of knowing. The full knowledge of an object cannot be obtained by only one of these ways of knowing.}\hfill (Chattopadhyaya 1978:89) ({\sl italics ours})
\end{myquote}

Here, {\sl Caraka-saṃhitā} is explaining that it considers authoritative instruction a {\sl pramāṇa} along with perception and inference. In other words, all the three {\sl pramāṇa}-s provide relevant knowledge to decipher the nature of a specific disease; all the three {\sl pramāṇa}-s have independent authority over parts of knowledge required for diagnosis. Pollock leaves out this commentary - which appears immediately before his chosen quote -- as it conflicts with his thesis -- viz. perception and inference are subordinated to authoritative instruction, and they merely further examine knowledge already pre-existing in the authoritative instruction.

\newpage

{\sl Caraka-saṃhitā} holds that authoritative testimony is optional for the learned person, but perception and inference are critical.

What immediately follows the isolated quote used by Pollock in {\sl Caraka-saṃhitā} is 
\begin{myquote}
For the learned, therefore, there are two modes of critical examination, viz. perception and inference. Or, if authoritative instruction also is included, the modes of critical examination are three.

\hfill (Chattopadhyaya 1978:89)
\end{myquote}

According to the above comment from {\sl Caraka-saṃhitā}, only two {\sl pramāṇa}-s are critical for a learned person to examine a patient: namely, perception and inference. Pollock conveniently leaves out this comment - which appears immediately after his chosen quote -- as it conflicts with his thesis that {\sl Caraka-saṃhitā} subordinates direct perception\index{perception} and inference\index{inference} to authoritative instruction.  

{{\sl Caraka-saṃhitā}\relax} gives direct perception the same level of importance that a modern medical practitioner would give. Speaking on the critical importance of direct perception in medical diagnosis, {\sl Caraka-saṃhitā} says 
\begin{myquote}
One wanting to know the nature of a disease by perception should examine everything perceptible in the body of the patient [...].

Thus for instance one should examine with the {\bf auditory sense} the intestinal sounds, the sounds of the joints and finger knuckles, variations in the patient's voice, or any other sound that may be present in any part of the body.  

With the {\bf visual sense} are to be examined the color, shape, proportion and the general appearance as well as changes in physique and behavior of the patient.  Besides, whatever else can be the object of visual knowledge should also be similarly examined.

[...]

Such then are the ways of examining the patient by perception, inference and instruction (authority).\hfill (Chattopadhyaya 1978:90) ({\sl emphasis ours})
\end{myquote}

While detailed commentary has been truncated by us for the sake of brevity, the essence however is clear that diagnosis relies heavily on direct perception and inference.  In fact, what Caraka says on the importance of direct perception and inference is what any modern physician would find agreeable.

{{\sl While Susruta-saṃhitā\index{Susruta-samhita@\textsl{Susruta-saṃhitā}} calls itself\endnote{Pollock writes: ``...{\sl Susruta-saṃhitā}, similarly begins with a mythological introduction concerning the origin of medicine, and claims that ``Brāhma it was who enunciated this Vedāṅga, this eight-fold Āyurveda.'' (Pollock 1985:513)

Kunja Lal (1907:2) translates the original thus;

The Āyurveda (which forms the subject of our present discourse), originally formed one of the subsections of the {\sl Atharva Veda}; and even before the creation of mankind, the self-begotten Brahmā strung it together into a hundred thousand couplets ({\sl śloka}-s), divided into a thousand chapters.} a Vedāṅga, it is aggressive in asserting the importance of direct perception. It does not ask the reader to look up its ``text''.}}

Chattopadhyaya writes:
\begin{myquote}
Though this discussion of the {\sl Susruta-saṃhitā} is less detailed than that of {\sl Caraka-saṃhitā} just quoted, it needs to be noted that it seems to go a step further than the latter, inasmuch as it does not hesitate to use the sense of taste also for diagnostic purposes.  However, what is much more remarkable is that [...] {\sl carried by its zeal for direct sense-perception, it goes to the extent of emphasizing the importance of the practice of dissecting human corpse, without which as the text claims, the knowledge of anatomy can never be satisfactory.}
\hfill (Chattopadhyaya 1978:93) ({\sl Italics ours})
\end{myquote}

(c) Can authoritative instruction truly be a {{\sl pramāṇa}\relax}?  Is there a paradox in thinking so?

How can authoritative instruction be a {\sl pramāṇa} independent of perception and inference? As a product authored by humans, does not authoritative instruction ultimately trace its origin to perception and inference of individuals? Is {\sl Caraka-saṃhitā} stuck as Pollock insists in a paradox unless it also claims to be a beginningless text?  Does {\sl Caraka-saṃhitā} have no choice but claim to be a beginningless text - {\sl apauruṣeya}, that is - in order to maintain its status as a {\sl pramāṇa}?

Authoritative instruction is indeed an independent {{\sl pramāṇa}\relax}.  Authoritative instruction does not have to be a beginningless text in order to avoid the paradox.

Contemporary medical philosophical discourse (Sadegh-Zadeh 2015)\index{Sadegh-Zadeh, Kazem} recognizes the critical importance of testimonial knowledge and its status as a {\sl pramāṇa}.  Brief excerpts are provided below from (Sadegh-Zadeh 2015) with italicized text to highlight key points. For details, please read the original.
  
Medical knowledge, is essentially testimonial knowledge.\index{knowledge!testimonial and non-testimonial} Sadegh-Zadeh  asserts that medicine as a scientific discipline cannot exist without authoritative testimony. In fact, the only source for scientific knowledge is testimony. Without our collective testimonial knowledge, human beings would not be much better off than other animals.  

Sadegh-Zadeh explains:
\begin{myquote}
Most medical experts believe that medicine is primarily and exclusively an empirical science.  They therefore assume that their expert knowledge stems from the following three sources: experience, memory, and reasoning. {\sl If this assumption were true, however, medicine as a scientific discipline could never exist}. The assumption ignores collaboration, and hence, communication between medical experts as a source of knowledge. Yet, collaboration enables testimony, and as was anticipated above, {\sl testimony is the only source of scientific knowledge.}

[...]

That means that if experience, memory, and reasoning are our primary {\sl individual} sources of knowledge, {\sl testimony} is our primary {\sl social} source of knowledge.  And {\sl without this social source of knowledge, with regard to intellect human beings would not be so very distant from animals, and scientific knowledge would not exist at all.}

\hfill (Sadegh-Zadeh 2015:536-537) ({\sl italics ours})
\end{myquote}

Testimonial knowledge cannot be reduced to non-testimonial, evidential knowledge. It cannot be subordinated to direct perception and inference.

Sadegh-Zadeh explains that medical expertise has epistemic dependence on testimonial knowledge.  The origin of testimonial knowledge is not experience, memory, or reasoning but a community i.e., society.  One cannot practically regress back through the long chains of information-flow to arrive at original direct perceptions and inferences of individuals.  It is practically impossible.  Thus, the initial source of testimonial knowledge for medical expertise is not other non-testimonial epistemic means of knowing for the individual but the society itself.  In other words, testimony is an essential epistemic means for a society of many.

Sadegh-Zadeh explains:
\begin{myquote}
how the paradox that Pollock refers to is not a concern for testimonial knowledge.  

[...]

Testimony can exist in indefinitely long chains. [...]  However, the chain must start in some initial source so as to avoid infinite regress and vicious circles.  [...] this origin of the chain i.e., the initial source of testimonial knowledge is not `experience, memory, or reasoning' as one would suppose, but a community i.e., the society.

\hfill (Sadegh-Zadeh 2015:536-537)
\end{myquote}

\newpage

Medicine\index{medicine} is a community-based discipline and its knowledge cannot be reduced to direct perception and inference. 

Sadegh-Zadeh explains that innovation in medical knowledge is essentially a community-based activity that relies on testimonial knowledge at every step (Sadegh-Zadeh 2015:537).

An author of a medical book draws on two types of second-hand information that he/she is not directly involved in: (a) non-medical knowledge such as statistical methods, biological or physical theories; (b) medical knowledge from other literature sources.  Such second-hand information is testimonial knowledge.

Even a researcher directly involved in laboratory research is not free from reliance on testimonial knowledge. The researcher relies on findings of other theories to conduct their experiments or interpret their laboratory results. In such a case, the researcher is not likely to have first-hand information on those theories.

Medical journal articles can have as many as a hundred authors. Some of the authors will not even know how some of the results or numbers used in the publication are arrived at. Any one author can know that other authors did such and such work to arrive at a particular result.

In summary, contemporary medical scholarship recognizes authoritative testimony as a {\sl pramāṇa} along with direction perception and inference. There is no paradox in considering authoritative testimony a {\sl pramāṇa} as Pollock imagines.

\section*{Summary of Pollock's Argument and its Refutation}

{\bf Pollock's Argument}

Pre-modern Indian intellectuals believed the following:

{\bf Claim 1}: All knowledge pre-exists in eternal ``texts''.  

{\bf Claim 2}: {\sl Śāstra}-s of practice are eternal ``texts'' for their respective domains.

{\bf Conclusion}:  All knowledge of practice pre-exists in respective {\sl śāstra}-s of practice. 

\newpage

{\bf Refutation of Pollock's Argument}

Pre-modern Indian intellectuals did not believe in Claim 1.  They believed that:
\textsl{Satkāryavāda} does not claim anywhere that all knowledge pre-exists in ``textual materials'' (verbal or written form).

\textsl{Pramāņa}-s are independent means for valid knowledge. Thus, {\sl pramāṇa}- and not ``texts''- are authorities of knowledge.

Pre-modern Indian intellectuals did not believe in Claim 2.  They believed that:
Direct perception and inference are essential {\sl pramāṇa}-s for knowledge of practice.  Because of their very nature, and because of their authority as {\sl pramāṇa} and independence from other {\sl pramāṇa}-s, valid knowledge from direct perception and inference cannot alternatively come from ``texts'', even if those texts have divine origin.

Authoritative testimonies are indispensable to growth of human knowledge and are a {\sl pramāṇa}.  {\sl Śāstra}-s of practice such as {\sl Caraka-saṃhitā} were best-in-class for many centuries; they were the authoritative testimonies.

Not just {\sl śāstra}-s, all knowledge is divine. It does not follow that {\sl śāstra}-s of practice have perfect knowledge or that they are complete.
Assent-worthiness\index{sastra@\textsl{śāstra}!assent-worthiness} of {\sl śāstra}-s of practice can be questioned even if the {\sl śāstra} claims divine origin.

\section*{Proper Conclusion}

Pre-modern Indian intellectual believed that {\sl śāstra}-s of practice (``texts'') are authorities for testimonial knowledge. However, authoritative testimonial ``texts'' are not the sole means of knowledge of practice. Direct perception, inference, and authoritative testimony are all indispensable means for practical knowledge.

Pre-modern Indian intellectual did not merely bring knowledge of practice to manifestation from the textual materials in which it lies concealed from us. There is historical evidence that they did extensive research and published original works in post-classical period well into recent times.

\newpage

\begin{thebibliography}{99}
\itemsep=2pt
\bibitem[]{chap10_item1}
Accademia Gallery in Florence, Italy. {\sl Michelangelo's prisoners or slaves}. 1520--1540. \url{http://www.accademia.org/explore-museum/artworks/michelangelos-prisoners-slaves/} (accessed 20 Jun 2016).

\bibitem[]{chap10_item2}
Aertsen, Jan A. (1993) ``Aquinas's' philosophy in its historical setting.'' In {\sl The Cambridge Companion to Aquinas}, by Norman Kretzmann and Eleonore Stump. Cambridge: Cambridge University Press.

\bibitem[]{chap10_item3}
Baigrie, Brian. (2002) ``The New Science: Kepler, Galileo, Mersenne.'' {\sl In Blackwell Companions to Philosophy: A Companion to Early Modern Philosophy}, by Steven Nadler (Editor), 45--59. Oxford: Blackwell Publishers Ltd.

\bibitem[]{chap10_item4}
Barooah, Jahnabi. {\sl 46\% Americans Believe In Creationism According To Latest Gallup Poll}. 05 June 2012. \url{http://www.huffingtonpost.com/2012/06/05/americans-believe-in-creationism_n_1571127.html} (accessed 20 June 2016).

\bibitem[]{chap10_item5}
Chattopadhyaya, Debiprasad. (1978) {\sl Science and Society in Ancient India}. Amsterdam: John Benjamins Publishing.

\bibitem[]{chap10_item6}
Dembski, William A. (1999) {\sl Intelligent Design: The Bridge Between Science \& Theology}. Downers Grove, Illinois: InterVarsity press.

\bibitem[]{chap10_item7}
Gambhirananda. (1965) {\sl Brahma Sutra Bhāṣya of Śaṅkarācārya}. Advaita Ashrama.

\bibitem[]{chap10_item8}
Ganeri, Jonardon. (2010) ``Hinduism.'' In {\sl A Companion to Philosophy of Religion}, edited by Charles Taliaferro, Paul Draper and Philip L. Quinn. West Sussex: Wiley-Blackwell.

\bibitem[]{chap10_item9}
Gopnik, Alison. (2009) ``Could David Hume Have Known about Buddhism? Charles Francois Dolu, the Royal College of La Flèche, and the Global Jesuit Intellectual Network.'' {\sl HUME STUDIES, Volume 35, Number 1\&2}, pp5--28.

\bibitem[]{chap10_item10}
Hiriyanna, Mysore. (1932) {\sl Outlines of Indian Philosophy}. London: George Allen \& Unwin Ltd..

\bibitem[]{chap10_item11}
Kunja Lal, Kaviraj Bhishagratna, (1907). {\sl An English Translation of the Sushruta Samhita}, Vol. 1, Sūtrasthānam. Calcutta: J.N. Bose.

\bibitem[]{chap10_item12}
Maas, Phillip A. (2010) ``On what became of the {\sl Caraka-saṃhitā} after Dṛḍhabala's revision.'' {\sl eJournal of Indian Medicine, Volume 3}, pp1--22.

\bibitem[]{chap10_item13}
Malhotra, Rajiv. (2011) {\sl Being Different: An Indian Challenge to Western Universalism}. New Delhi: HarperCollins India.

\bibitem[]{chap10_item14}
Malhotra, Rajiv. (2013) ``Vivekananda's ideas and the two revolutions in Western thought.'' In {\sl Vivekananda as the turning point: The Rise of a new spiritual wave}, edited by Swami Shuddhidananda, pp. 559--583. Kolkota: Advaita Ashrama.

\bibitem[]{chap10_item15}
Meulenbeld, Gerrit Jan. (1999-2002) {\sl A History of Indian Medical Literature. (Groningen Oriental Studies 15) (5 Volumes)}. Groningen, Netherlands: Egbert Forsten Publishing.

\bibitem[]{chap10_item16}
Murphy, Nancey C. (1990) {\sl Theology in the Age of Scientific Reasoning. Ithaca}, New York: Cornell University Press.

\bibitem[]{chap10_item17}
Newport, Frank. (2014) In {\sl U.S., 42\% Believe Creationist View of Human Origins}. 02 June 2014. \url{http://www.gallup.com/poll/170822/believe-creationist-view-human-origins.aspx} (accessed 20 June 2016).

\bibitem[]{chap10_item18}
Pollock, Sheldon. (1985) ``The Theory of Practice and the Practice of Theory in Indian Intellectual History.'' {\sl Journal of the American Oriental Society} 105, no. 3, pp. 499--519.

\bibitem[]{chap10_item19}
Sadegh-Zadeh, Kazem. (2015) {\sl Handbook of Analytic Philosophy of Medicine}. Dordrecht: Springer.

\bibitem[]{chap10_item20}
Sastry, Alladi Mahadeva. (1977, 1897$^1$) {\sl The Bhagavad Gita with the commentary of Sri Sankaracharya, 7th Edition}. Mysore: Author.

\bibitem[]{chap10_item21}
Sharma, Chandradhar. (2013, 1960) {\sl A Critical Survey of Indian Philosophy}. Delhi: Motilal Banarsidass Publishers Private Limited.

\bibitem[]{chap10_item22}
Sherma, Rita and Sharma, Arvind (2008) {\sl Hermeneutics and Hindu Thought: Toward a Fusion of Horizons}. Netherlands; Springer.

\bibitem[]{chap10_item23}
Withnall, Adam. (2014) ``Pope Francis declares evolution and Big Bang theory are real and God is not `a magician with a magic wand'.'' {\sl Independent}, 28 Oct 2014. (accessed 20 June 2016).

\bibitem[]{chap10_item24}
Wujastyk, Dominik. (2004) ``Review of G. Jan Meulenbeld's A History of Indian Medical Literature. (Groningen Oriental Studies Volume XV/I--III.).'' {\sl Bulletin of the School of Oriental and African Studies}, pp. 404--407.
\end{thebibliography}

\theendnotes
\label{chapter\thechapter:end}
