\chapter{On the role of {\sl\bfseries Śāstra} : an Examination}\label{chapter4}
\vskip -10pt

\Authorline{K. Vrinda Acharya}
\lhead[\small\thepage\quad K. Vrinda Acharya]{}
\vskip -10pt

\section*{Abstract}
%\index{Acharya, K Vrinda}

One of the major areas of Professor Sheldon Pollock's (hereafter Pollock) work is his writings on `the relationship between {\it Śāstra}\index{sastra@\textit{śāstra}} and {\it Prayoga}'\index{prayoga@\textsl{prayoga}} which need serious and considerable {\it pūrvapakṣa}\index{purvapaksa@\textsl{purvapaksa}} and {\it uttarapakṣa}.\index{uttarapaksa@\textsl{uttarapakṣa}} Pollock alleges that {\it śāstra}-s hinder genuine creativity,\index{creativity} practical innovation,\index{innovation} original thinking and progressive growth, since they are straitjacketed by the Vedic worldview. On the contrary, there is abundant counter-evidence which shows that Indians have always been innovative in producing and applying {\it śāstra}-s to both empirical and spiritual domains. 

His main contention in his article (1985) titled `The Theory of Practice and the Practice of Theory in Indian Intellectual History' is that ``{\it Śāstra} is one of the fundamental features and problems of Indian civilization in general, and of Indian intellectual history in particular.''-- wherein his terming of {\it Śāstra}\index{sastra@\textit{śāstra}!problem of} as a `problem' is itself weird and objectionable.  

This paper intends to review some of the striking points made by Pollock in this article, which are as follows.
\begin{enumerate}
\item In India, there is no dialectical interaction between theory and practice.

\item All knowledge is pre-existent, and progress can only be achieved by a regressive re-appropriation of the past.

\item The eternality of the Vedas,\index{Vedas!eternality} the {\it śāstra} par excellence, is one presupposition or justification for this assessment of {\it Śāstra}.

\item Indian civilization is constrained by rules of varying strictness.

\item {\it Śāstra}-s instigate `authoritarianism',\index{sastra@\textit{śāstra}!authoritarianism} which further leads to social oppression.\index{sastra@\textit{śāstra}!social oppression}

\item The understanding of the relationship between {\it Śāstra} and {\it Prayoga}\index{prayoga@\textsl{prayoga}} in India is  diametrically opposed to what is usually found in the West. 
\end{enumerate}

This paper proposes to critically examine the nature, rationality, motive, and more importantly, the need (if any) for such an analysis made by Pollock. It tries to throw some light on the shortcomings in his assumptions, his logic and his conclusions. Also highlighted in this paper are -- some of his arguments which actually boomerang upon himself; some of the assertions which seem to have no basis whatsoever; some essential aspects overlooked by him; and a few of his selective/misleading citations.

\vskip -12pt

\section*{Keywords}

Sheldon Pollock, \textsl{Shastra, śāstra}, theory, practice, Sanskrit, Vedic, traditional India, regression, West, Western Indology, Swadeshi Indology, rebuttal, \textsl{poorvapaksha, pūrvapakṣa, uttarapaksha/uttarapakṣa}, sacred, secular.

\vskip -12pt

\section*{Introduction}

Professor Sheldon Pollock, the well-known American Sanskrit scholar and Indologist, the chief protagonist of the American Orientalist movement (as Rajiv Malhotra bears him out to be), is naturally one who cannot be ignored or taken lightly, considering the immense academic work he has done in the field of Sanskrit studies; these studies of his are\index{Pollock!agenda} quite unfavourable to the greatness and sanctity of our Sanskrit tradition. One cannot also miss to note the recognition, support and funding he gets from the Indian government/Indian academia/ traditional mutts/ Indian corporate/ secularists/ leftists.

Pollock's main agenda seems to be demeaning, or altogether dismissing, the transcendental nature of Sanskrit language; and arguing that Sanskrit texts and scriptures were meant to be used as a tool of social oppression and discrimination. He explicitly praises the language for its {\sl kāvya}, its beauty, aesthetics and structure; and at the same time constantly and implicitly disregards the Veda-s and other {\sl śāstra}-s, as a part of his mission to ``secularise'' Sanskrit.\index{Sanskrit!secularisation of} 

One of the major areas of Pollock's work is his writings on `the relationship between {\sl Śāstra} and {\sl Prayoga}' which call for serious and considerable {\sl Pūrvapakṣa} and {\sl Uttarapakṣa}. In the words of Rajiv Malhotra ``His extreme views on {\sl śāstra} provide insight into the deep-rooted prejudices and chauvinism\index{Western chauvinism} of the West.'' (Malhotra 2016:115) He alleges that {\sl śāstra}-s hinder genuine creativity, practical innovation, original thinking and progressive growth, since they are straitjacketed by the Vedic worldview; whereas there is abundant counter-evidence which shows that Indians have always been innovative\index{innovation} in producing and applying {\sl śāstra}-s to both empirical and spiritual domains.

My paper is an attempt at a {\sl pūrvapakṣa} and {\sl uttarapakṣa} {\em vis-á-vis} Pollock's views on Indian {\sl śāstra}-s. It simply tries to read and understand what he has to say, as also why he says so; and further tries to critically examine whether he is justified in saying so. It tries to throw some light on the drawbacks in his assumptions, his logic and his conclusions. Some of his arguments which boomerang on him, some of his assertions which have no logic or valid proof, some essential aspects overlooked by him and a few of his wrong or selective quotations \index{quotations!selective} - are also highlighted in this paper. 

I do not claim this to be an exhaustive and complete investigation of his writing on the subject. Within the limited time and resources available to me, I attempt to evaluate a few of his assertions in the light of their nature, basis, rationality, motive, and more importantly, their very need, real or supposed.

\vskip -12pt

\section*{Pollock's assessment of {\sl\bfseries Śāstra}}

In his article titled `The Theory of Practice and the Practice of Theory in Indian Intellectual History', which was written as early as 1985, the very first sentence of his Abstract reads thus -- ``{\it Śāstra} is one of the fundamental features and problems of Indian civilization in general and of Indian intellectual history in particular.'' (Pollock 1985:499) -- wherein his terming of {\it Śāstra} as a `problem' \index{sastra@\textit{śāstra}!problem} is itself weird and objectionable. His Abstract summarizes his position as follows.
\begin{myquote}
``The understanding of the relationship of {\sl śāstra} (``theory'') to {\sl prayoga}\index{sastra@\textit{śāstra}!and prayoga@\textsl{and prayoga}}  (``practical activity'') in Sanskritic culture is shown to be diametrically opposed to that usually found in the West. Theory is held always and necessarily to precede and govern practice, there is no dialectical interaction between them. Two important implications of this fundamental postulate are that all knowledge is pre-existent, and that progress can only be achieved by a regressive re-appropriation of the past. The eternality of the Veda-s, the {\sl śāstra} par excellence, is one presupposition or justification for this assessment of {\sl śāstra}. Its principal ideological effects are to naturalize and de-historicize cultural practices, two components in a larger discourse of power.''\hfill (Pollock 1985:499)
\end{myquote}
 
Pollock condemns the {\sl śāstric} codification and its normative character, and says that it was this attitude that prompted him to further study in the area of {\sl śāstric} regulation, conceived accordingly as an analysis of the components of cultural hegemony\index{cultural hegemony} or at least authoritarianism.\index{authoritarianism} He asserts that the question of domination remains important for several areas of pre-modern India, and everywhere civilisation as a whole has been constrained by rules of varying strictness. (Pollock 1985:499)

Though he agrees that cultural grammars\index{cultural grammars} exist in every society, he declares that classical Indian civilisation offers what may be the most exquisite expression of the centrality of rule-governance in human behavior. He goes on to say that, due to the influence of strict regulation of ritual action imposed in Vedic ceremonies, secular life as a whole was subject to a kind of ritualisation,\index{ritualization} whereby all its performative gestures and signifying practices came to be encoded in texts. He also adds that grammars in India were invested with massive authority, ensuring a nearly unchallengeable claim to normative control of cultural practices. (Pollock 1985:500)

Further, he tries to analyse the so-called `problem'\index{sastra@\textit{śāstra}!problem} of {\sl śāstra} with three connected questions he raises for himself.
\begin{enumerate}
\item How does the tradition view the relationship of a given {\sl śāstra} to its object?

\newpage

\item What are the implications of this view for the concept of cultural change?

\item Is there some traditional presupposition, or justification, for the previous two notions?\hfill (Pollock 1985:501)
\end{enumerate}

At first Pollock goes on to ``exhume'' a few interpretations of the term `{\it śāstra}', and its classifications and lists given by several traditional thinkers. Here he opines that the very notion of a finite set of `topics of knowledge' implies an attempt at an exhaustive classification of human practices and also that literally every activity of pre-modern life in India was subject to the strict codification and paramount authority of the {\it śāstra}. He analyses Vyākaraṇa, Dharmaśāstra, and {\sl Kāma Sūtra} among other {\it śāstra}-s through his lens, and concludes that the śāstric discourse\index{sastra@\textit{śāstra}!discourse} is injunctive - thereby giving no scope whatsoever for any individual for creative thinking. Traditional India's own understanding of this aspect is, according to him, completely opposite to that of the West. He finally says that the technical {\it śāstra}-s on the one hand prescribe `the only one correct way' to do a particular thing;  whereas the regulative {\it śāstra}-s try to acquire a technical and constructive aspect by imposing codified legislative control on obvious and natural human activities, as if they are impossible to exist in its absence. (Pollock 1985:501-502)

In his attempt to evaluate the implications of this priority of theory, he says
\begin{myquote}
``That the practice of any art or science, that all activity whatever succeeds to the degree it achieves conformity with {\em śāstric} norms would imply that the improvement of any given practice lies, not in the future and the discovery of what has never been known before, but in the past and the more complete recovery\index{recovery of past} of what was known in full in the past.'' 

\hfill (Pollock 1985:512)
\end{myquote}

What he means to say here is that no progress has happened in the Indian intellectual history for thousands of years owing to this `regression'\index{knowledge!regression} which in turn is a result of the unchallenged and undisputed supremacy\index{sastra@\textit{śāstra}!supremacy} of the {\it śāstra}-s. He attributes this supremacy to what he calls the `mythic crystallization' with regard to the origin of {\it śāstra}-s. He goes on to say that extant {\it śāstra}-s are either considered as abridged versions of the divine prototypes, or as exact reproductions of the originals; and quotes several examples. As a result of this, there has always been a backward movement only trying to be as close as possible to the divine paradigm, but no forward movement towards any betterment based on practical innovations. (Pollock 1985:512-515)

The traditional pre-supposition or justification for all this, he argues, is that the {\it Veda}-s are considered eternal, infinite, self-existent and infallible; and they represent knowledge which is always pre-existent,\index{knowledge!pre-existent} without a beginning or an end, and as such, creation of knowledge\index{knowledge!creation} is not possible. What is possible is only acquisition,\index{knowledge!acquisition} modification or sometimes rejection of knowledge\index{knowledge!rejection} which already exists. All secular {\it śāstra}-s are carved out of the Vedic corpus, and share the Veda's transcendent attributes. (Pollock 1985:517-519)

\section*{Analysing Pollock's approach}

Starting with the positives first, Pollock\index{Pollock!approach} is undoubtedly a very well-read academician, who has done his homework extremely well in the preparation of this article. He quotes his references with reasonable accuracy (with an exception of a very few which I show a little later in this paper) and uses them intelligently to the best advantage of his line of argument. 

However, on the other side, the first major, easily noticeable, drawback in Pollock's approach is that the fundamental premise of the paper is not clear. He labels {\it śāstra} as a problem, but doesn't state what the problem actually is. Further, he says
\begin{myquote}
``In the light of the major role it appears to play in Indian civilization, it is surprising to discover that the idea and nature of {\sl śāstra} in its own right, as a discrete problem of intellectual history, seem never to have been the object of sustained scrutiny. Individual {\sl śāstra}-s have of course received intensive examination, as have certain major sub-genres, such as the {\sl sūtra}. But a systematic and synthetic analysis of the phenomenon as a whole, as presenting a specific and unique problematic of its own, has not to my knowledge been undertaken.''\hfill (Pollock 1985:500-501)
\end{myquote}

{\sl Śāstra} seems to be a big problem to him, but it has not been one to Indians. Where then is the question of making it an object of any scrutiny? 

My assessment of Pollock writing on this subject is that he is ceaselessly wandering and searching for loopholes\index{Pollock!loopholes} in the Indian system to prove his point. No doubt he seeks to substantiate his thesis by drawing together evidence from a wide variety of sources. But, as Hanneder\index{Hannedar, Jurgen} says, Pollock's argumentation is ``often arbitrary''; in his review of Pollock's paper on ``The Death of Sanskrit'', Hanneder says ``...it is my impression that Pollock has over-interpreted the evidence to support his theory...'' (Hanneder 2002:294). Even here, in most cases, where he is unable to convince the reader, he tries to confuse! First of all, the very basic idea of {\it śāstra} that he uses to start his argument is not well-defined. He quotes from a wide range of texts and tries to arrive at a single definition of {\it śāstra}, which he can then conveniently use to drive his point. 

In fact, the term {\it śāstra} is seen to have more than one dimension\index{sastra@\textit{śāstra}!dimensions} in the Indian tradition. In the context of Alaṅkāra, Nyāya or Vyākaraṇa, the term {\it śāstra} is seen as theory, whereas with reference to philosophy, it refers to {\it sacchāstra} or {\it sadāgama}. (Refer {\it Viṣṇutattvavinirṇaya} (of Ācārya Madhva) 1.3, 1.4).  In some other context, it is confined to {\it dharma-śāstra} alone. In common usage, {\it śāstra} can also mean anything which has a scientific basis or even the {\it ācāra}-s and {\it vicāra}-s (``specified/regulated conduct'', and ``trains of thought'') of an average Hindu household. In general,  śāstra-jñāna means the technical know-how or specialized knowledge in a defined area of practice. Though Pollock quotes several references from different disciplines, he ultimately sticks to one meaning as `theory' and proceeds with his argument. 

Secondly, he doesn't seem to have really understood the meaning of the words `{\it pauruṣeya}' and `{\it apauruṣeya}'. He translates them as `human' and `transcendent' (Pollock 1985:502). In reality, `{\it pauruṣeya}'\index{pauruseya@\textit{pauruṣeya}} refers to that which is created or composed by somebody, human or divine. ({\it Puruṣa} meaning either human being or the Supreme Being) (Ref. Apte 2007; also see Puruṣa-sūkta, 1-3). And `{\it apauruṣeya}'\index{apauruseya@\textsl{apauruṣeya}} refers to that which is eternal or timeless, that which has always been there - for it is not created by anybody at all. This should sort out all the confusion he seems to have about which {\it śāstra}-s are {\it pauruṣeya}' and which {\it śāstra}-s are `{\it apauruṣeya}'. 

Thirdly, he calls {\it dharma} `rule-boundedness' (Pollock 1985:511) which is a very narrow and parochial interpretation.\index{dharma@\textsl{dharma}!interpretation} {\it Dharma}, the pivotal concept of Indian philosophy, is a very broad concept which has no single definition. It is derived from the root `\textsl{dhṛ dhāraṇe}' which means `that which holds, supports, sustains, maintains'--``\textsl{dhāraṇād dharma ity āhuḥ}'' ({\it Mahābhārata} 8.69.58). So, essentially it means the eternal unvarying cosmic law,\index{cosmic law} inherent in the very nature of things. In short, it means the all-pervading cosmic order that sustains the entire creation. There is no single equivalent word which translates `{\it dharma}' in any Western language. In the words of Rajiv Malhotra, it is a `non-translatable' (Malhotra 2013:259).

After citing Kumārilabhaṭṭa's\index{Kumarila Bhatta@Kumārila Bhaṭṭa} and Rājaśekhara's dichotomy of {\it śāstra}-s into {\it pauruṣeya} and {\it apauruṣeya}, Pollock says,
\begin{myquote}
``The postulate of a single source of both sorts of knowledge was far more widespread, and is the dominant presupposition when not clearly enunciated (as it is in the most elaborate survey of sciences, Madhusūdhana Sarasvati's\index{Madhusudana Sarasvati@Madhusūdhana Sarasvati} 16th century {\sl Prasthāna-bheda},\index{prasthanabheda@\textsl{prasthanabheda}} where the division between human and transcendent is altogether abandoned.'' 

\hfill (Pollock 1985:503)
\end{myquote}

He does this in order to show that there is a lot of confusion in the Indian tradition with respect to the origin of {\sl śāstra}-s. But {\sl Prasthāna-bheda} only says ``All the scriptures have the Supreme Lord as their purport'' -- 

\begin{quote}
``atra sarveṣāṁ śāstrāṇāṁ bhagavaty eva tātparyaṁ sākṣāt-paramparayā veti samāsena teṣāṁ prasthāna-bhedo `troddiśyate'' P.1. {\sl Prasthāna-bheda}, 
\end{quote}
\noindent
and does not speak about the source of the scriptures. The same text later confirms the {\sl apauruṣeyatva} of the {\sl Veda}-s. Says {\sl Prasthāna-bheda} (p3) -- 

\begin{quote}
``{tatra dharma-brahma-pratipādakam apauruṣeyaṁ pramāṇa-vākyaṁ vedaḥ}'' 
\end{quote}

In his thesis about grammar and language usage, Pollock analyses Patañjali’s\index{Patanjali@Patañjali} view on acceptable usage outside the framework of grammar, and concludes thus
\begin{myquote}
``Acceptability and grammaticality exist concurrently as separate entities in any speech community (indeed, within each individual speaker). This co-existence may be uneasy, with grammaticality finally assimilating or annihilating acceptability as a category, though the latter, in a continually recurring process, will resurface in another form.'' 

\hfill (Pollock 1985:505)
\end{myquote}

Pollock's basic complaint throughout is that there is no dialectical interaction between {\it śāstra} and {\it prayoga}.\index{sastra@\textit{śāstra}!and prayoga} When Patañjali’s position clearly allows certain usages outside grammar, thereby showing that both acceptability and grammaticality mostly co-exist (which he himself agrees in the first statement above), is it not enough to show that there is indeed a natural, dialectical, interaction between {\it śāstra} and {\it prayoga}? Moreover, if grammaticality assimilates acceptability from time to time, where is any question of the coexistence becoming uneasy? Many more such instances of contradicting and confusing arguments can be pointed out in Pollock's paper.

Pollock declares that he is not trying to say what is right or wrong, but is only trying to see how all this was perceived in traditional India. 
\begin{myquote}
``... I do not, however mean to argue here the question of the truthfulness of this assessment, but simply to adumbrate the usual Western opinion as a backdrop to what I take to be the development and final position of the ancient Indian attitude toward it. For what is presently at issue is not the veracity of this or that model of the origins and transmission of cultural knowledge, but rather how such things were understood in traditional India. This understanding, as should now be clear, is diametrically opposed to that commonly found in the West..'' \hfill (Pollock 1985:511)
\end{myquote}

In reality, he is not making any objective study, though he claims he is. It is very clear that he is bent upon branding {\it śāstra} as `a problem', as `regressive', etc., and throughout his thesis, all his postulates \index{Pollock!postulates} and logic are directed towards the same. He does not really seem to be making any attempt to know how {\it śāstra} was understood in India. Actually, what he \textsl{`wants' to conclude} is decided in the beginning itself. 

In order {\sl to appear} balanced, fair, and just, Pollock acknowledges several aspects that contradict his claims, but later on very smartly sidelines them altogether, and ultimately concludes what he initially wanted to conclude. 

For instance, he quotes a passage from the {\it Manu-smṛti} which talks about a certain code of conduct with respect to greeting another person. He then exclaims that students in the West ``are amazed to find even so apparently simple an act as meeting another person encumbered with a whole battery of rules'', and thus find Indian culture very alien. He then goes on to quote a section from a book called \textsl{Amy Vanderbilt’s Everyday Etiquette},\index{Amy Vanderbilt’s \textsl{Everyday Etiquette}} which prescribes rules for shaking hands, smiling, removing hats, etc., while meeting somebody. 

Surprisingly, he agrees that such cultural grammars,\index{grammar!cultural} the mastery of which makes one a competent member of the culture in question, exist in every society, and that they are the code defining the given culture as such. But finally he concludes that it is only Indian civilisation that offers the most exquisite expression of the centrality of rule-governance in human behavior, and it is thus that {\sl śāstra} which represents these grammars became a serious problem. (Pollock 1985:500)

To point out another example: towards the end of his paper, he draws inferences from \textsl{Caraka-saṃhitā}\index{Caraka-samhita@\textsl{Caraka-saṃhitā}} to the effect that all knowledge is eternal and pre-existent, and there can be no `creation' of knowledge\index{knowledge!creation} as such. He continues by citing Western philosophers like Socrates\index{Socrates} and Popper\index{Popper} who also subscribe to the same view but in different words. 

But the clever conclusion that he makes is ``Whatever the cogency of these more philosophical explanations for the special character attributed to {\it śāstra}, a historical-cultural consideration seems to me somewhat persuasive.'' (Pollock 1985:518) 

This clearly shows that he eventually concludes just the way he wants, but simply gives references from here and there in order to make a show that he is following a very analytical approach. I will offer a few more of such examples a little later.

Pollock is usually seen following a very tactful and oblique approach.\index{Pollock!approach} He generally starts by admiring and appreciating the Indian tradition. Those who read him only superficially will believe he is all praise for the same. There are several instances, in most of his writings, and particularly in this paper, where, in the same pages in which he criticises {\it śāstra}-s as being dogmatic and oppressive, he also admires them for their monumental and unparalleled intellectual accomplishment, and impact on culture in traditional India (Pollock 1985:500). 

Only an in-depth and comprehensive study of this will reveal to us that he is saying, implicitly though, that these monumental and unparalleled śāstra-s are to be seen as a prison in which Hindus have trapped themselves (Malhotra 2016:117).\index{Malhotra, Rajiv}

Another such example is his phrase ``exquisite expression of the centrality of rule-governance in human behavior'' which he uses to describe classical Indian civilisation. This seems to be Pollock's characteristic style. \index{Pollock!characteristic style}

After making his conclusion about the superiority of theory over practice in the Indian tradition, he finally says ---
\begin{myquote}
`` ...But most people today I think would readily accept the common-sense assessment of Ryle,\index{Ryle, Gilbert} that ``efficient practice precedes the theory of it''... And this is the position that has been dominant in Western thinking from the time of Aristotle.''  \index{Aristotle}\hfill (Pollock 1985:510-511)
\end{myquote}

Does he mean to say that a civilisation - which is known all over the world over for its richness in philosophy, culture, religion, knowledge and education; which had seen the light of wisdom several centuries ago, when most countries in the world were steeped in the darkness of ignorance and primitiveness; and which gave to the world the most profound knowledge systems related to all possible domains of human life - does not have common sense?

Further, he quotes from the book `\textsl{The Concept of Mind}'\index{The Concept of Mind} by Gilbert Ryle, the well known British philosopher, thus: 
\begin{myquote}
``Theorization translates what is discovered by actions into concepts and doctrines... In every field, action comes first, classification and conceptualization come later.... Is it not more intuitive, however to think that theory evolves out of practice and will itself evolve as practice, refines and modifies itself?'' 
\hfill (Pollock 1985:511)
\end{myquote}

This is of course true even in India. Pollock has completely failed to recognize that those who originally wrote or composed the {\it śāstra}-s had indeed first practiced and experienced certain truths themselves; and only later, went on to textualise them. In fact, it is the basic eligibility criterion for anybody to write a {\it śāstra}. Only a {\it śāstra} written by a {\it ṛṣi}/mystic/seer with a deep transcendental and experiential yogic insight\index{yogic insight} (Pollock seems to acknowledge this elsewhere but with a tone of subtle sarcasm) will have the real substance, and thus will stand the test of time, whereas others will gradually fade away. I am reminded of Swami Vivekānanda's \index{Vivekananda, Swami} words -- ``Preach what you practice''.

\section*{Arguments lacking logic/proof}

\begin{enumerate}
\item Pollock says ``In the West, codes have largely remained ``tacit'' knowledge, existing on the level of practical and not discursive awareness. In India, by contrast, they were textualized, many of them at an early date, and had consequently to be learned rather than assimilated by a natural process of cultural osmosis.'' \index{cultural osmosis} (Pollock 1985:500) 

\newpage

This doesn't make much sense to me. First of all in India, cultural values are more assimilated than formally learnt from texts. Those who have studied the {\it śāstra}-s are only a few, but cultural codes\index{cultural codes} and practices are an inseparable aspect of almost every Indian/Hindu which he/she learns by living the culture.  Secondly, if Western codes are `tacit' as he calls them, where was the need for a book like \textsl{Amy Vanderbilt's Everyday Etiquette} \index{Amy Vanderbilt’s \textsl{Everyday Etiquette}} which dictates cultural grammars in great detail?

\item According to him, the {\it Vedāṅga}-s, \index{Vedanga@Vedāṅga} primarily, were in their very nature only taxonomical and descriptive; and only later transformed themselves from a descriptive catalogue to a prescriptive or injunctive system. He uses Geertz's concept of ``models of'' as against ``models for'', and says texts that had initially shaped themselves to reality so as to make it ``graspable'', end by asserting the authority to shape reality to themselves (Pollock 1985:503-504). 

But what is the basis for such an argument? In the absence of any valid historical or textual proof, this seems more like his own baseless assumption. Why, when, how, and on account of whom did such a transformation in the nature of {\it Vedāṅga}-s happen? - is a question he has to address here.

\item Pollock first gives many quotations which indicate that {\it śāstra} had to always precede, guide, regulate, control and dominate practice in all domains of pre-modern Indian life. Then he goes on to cite some counter-examples by Kauṭilya, Manu, Daṇḍin, Suśruta, and Caraka among others, showing that experience or actual practice was considered more important than mere theoretical knowledge from texts/books. 

He cites these examples - which actually disprove his own line of argument - just in order to give the reader an impression that he is unbiased and is doing a very truthful and impartial scrutiny. However, he later dismisses all these examples calling them a `minority'. He says ``Such voices...are pretty much in the minority. The dominant ideology is that which ascribes clear priority and absolute competence to śāstric codification'' (Pollock 1985:510). 

Here again, he does not give any convincing basis - for writing off these as minority views, and hence insignificant; or for claiming, consequently, the {\it śāstra}-s were, mostly and by majority, authoritarian.\index{sastra@\textit{śāstra}!authoritarian} 
\end{enumerate}

\section*{Arguments which boomerang}
\index{Pollock!arguments boomerang}

\begin{enumerate}
\item He first says that although the word {\it śāstra} is attested from the time of the earliest literary monuments, no comprehensive definition is offered until the medieval period (Pollock 1985:501). But later on, while talking of Madhva's\index{Madhvacharya@Madhvācārya} definition of {\it śāstra}, he himself says it is that which is found in the {\it Skāndapurāṇa} (Pollock 1985:503), which definitely belongs to a period much earlier than the so-called medieval period. He also quotes definitions of {\it śāstra} from Pūrva Mīmāṁsā and Patañjali. 

Moreover, much of Sanskrit vocabulary is amenable to  etymology,\index{sastra@\textit{śāstra}!etymology} and every word that is used has to be well-defined. {\it Śāstra} is derived from the root `\textsl{śāsu --anuśiṣṭau}'. It is defined as ``\textsl{śiṣyate anena iti śāstraṁ}'' or ``\textsl{śāsti, trāyate ca iti śāstraṁ}'' or ``\textsl{śāsanāc chaṁsanāc caiva śāstram ity abhidhīyate}'', (he himself quotes this last definition in a footnote) meaning ``that which instructs or directs'', and certainly not ``that which restricts''!

\item Pollock argues that {\it śāstra}-s are fixed or rigid, and frozen in time, and hence do not offer any scope for creativity or innovation. He goes on to quote several traditional mystics, scholars and authors belonging to various times in history, who have listed the {\it śāstra}-s which they approve or consider as valid. 

But the very fact that so many scholars or authors have given different lists of {\it śāstra}-s itself proves that {\it śāstra}-s were in no way fixed or rigid. From time to time, newer {\it śāstra}-s were produced, sometimes supplementing, and sometimes refuting or competing with, each other. Where then is the question of any restriction?
\end{enumerate}

\section*{Essential aspects overlooked}
\index{Pollock!overlook}

\begin{enumerate}
\item Pollock repeatedly emphasises the top-down nature of the flow of all knowledge in {\it śāstra}-s. He speaks extensively about Sanskrit grammar and feels that the usage of language is bound by strict grammar. 

He completely ignores the fact that there has always been a reciprocal flow of influence between grammar and usage in Sanskrit language, and so is the case in any \textsl{śāstra-prayoga} pair. 

Some well known axioms followed by Sanskrit grammarians prove this fact adequately --- 
\begin{itemize}
\item[$\bullet$] ``\textsl{prayoga-śaraṇāḥ vaiyyākaraṇāḥ}:'' $\{$Grammarians rely on practice (\textsl{Translation mine})$\}$, 

\item[$\bullet$] ``\textsl{sarve niyamāḥ sāpavādāḥ}'' $\{$All rules have exceptions (\textsl{Translation mine})$\}$, 

\item[$\bullet$] ``\textsl{yogād rūḍhir balīyasī}'' $\{$Connoted sense is stronger than (or preferred to) the etymological meaning (\textsl{Translation mine})$\}$-- 
\end{itemize}

these provide sufficient evidence to show that Sanskrit grammar was never rigid, but was always open to be influenced by practice. 

Also, in the development of Sanskrit grammatical tradition, the  {\it Vārttika}-s of Vararuci\index{Vararuci} enunciate as much as 5000 exceptions to rules made by Pāṇini,\index{Panini@Pāṇini} out of which {\it Mahābhāṣya} of Patañjali\index{Patanjali@Patañjali} negates 3000, and accepts the other 2000. Down the line, there were several other illustrious grammarians like Jayāditya, Nāgeśabhaṭṭa,\index{Nagesabhatta@Nāgeśabhaṭṭa} Kaiyyaṭa\index{Kaiyyata@Kaiyyaṭa}  and so on, who made significant contributions to Sanskrit grammar. This proves the fact that Sanskrit grammar\index{Sanskrit!grammar}  was never fixed or frozen.

\item Pollock says that the texts on metrics\index{metrics (\textsl{chandas})} were nomological in character, which made the subject to be treated with homogeneity for over some two thousand years and there has been a very keen attempt throughout on the part of poets to approximate their work as closely as possible to the śāstric stipulations (Pollock 1985:499). 

Firstly, {\it chandas}\index{chandas@\textsl{chandas}} or metrics did not stop with Piṅgala.\index{Pingala@Piṅgala} Kedārabhaṭṭa's\index{Kedarabhatta@Kedārabhaṭṭa} \textsl{Vṛtta-ratnākara}\index{Vrtta-ratnakara@\textsl{Vṛtta-ratnākara}} and many other texts on metrics can be identified with the middle ages of Sanskrit prosody. 

Secondly, {\it Vṛtta-ratnākara} which is the most exhaustive compilation containing over 600 metres,  shows a substantially larger repertoire than that found in any other metrical tradition. 

When almost all possible aesthetic combinations of metres are already dealt with, is there any logic in simply trying to think out of the box?  

\item While speaking about arts,\index{arts} Pollock says art-making is constrained by rules of varying strictness (Pollock 1985:499). He doesn't seem to acknowledge the fact that India's contribution to the world in the field of music, dance and other forms of art is one of the greatest. Arts, in India, have always been the most creative and innovative;\index{innovation} Indian artists have the maximum liberty,\index{sastra@\textit{śāstra}!artistic liberty} and the {\it śāstra}-s or the rules only go to provide a broad framework or boundary, and an open architecture within which art is expected to thrive. Tradition is dynamic and keeps evolving, and it has taken centuries to define and refine these boundaries. 

A celebrated saying ``\textsl{prayoga-pradhānaṁ hi nāṭyaśāstram}'' ({\it Māla\-vikāgnimitra}\index{Malavikagnimitra@Mālavikāgnimitra} 1.15+) $\{$The science of dramaturgy is predominated by practice (\textsl{Translation mine})$\}$ very well proves the fact that, in Indian art, it is practice that makes theory, and not vice-versa. Boundaries are a must to bring in some orderliness and discipline in any art; drawn sensibly, they do not restrict creativity. What is the fun in creating something when it's totally free ended?! 

Quite on the other hand, it is when boundaries are well-defined, that the challenge is immense, and any creation is bound to be amazing only when it is within a refined, well tested, space. If a certain artist is unable to explore his creativity within the boundary, that reflects on his own inability, and not on the limitations of the system itself. 

When we talk of music, it is Western Music in fact, which is very rigid in its practice, where a composition of Beethoven or Mozart has to be played exactly as it is, or rather, as it was when composed. Not so at all in Indian music:\index{music} for example, the same composition of Tyāgarāja is presented by different artistes in different ways. Each artist adds his own special touch to the composition, retaining however, the original structure. Not just that, it is only in Indian music that we find a unique concept of `{\it mano-dharma}' which offers maximum liberty for an artist\index{artistic liberty} to express his art - in the most spontaneous, creative, and imaginative manner. 

\item Pollock looks at {\it śāstra} only from a single and limited angle; and repeatedly blames {\it śāstra} as being too normative. Whereas, while speaking about grammar,\index{grammar} he himself says ``...However detailed a descriptive grammar may be, it can never be totally exhaustive of linguistic practice. There are indeed passages in Pāṇini\index{Panini@Pāṇini} where this is evidently well understood, and where consequently it cannot be claimed that {\it śāstra} has made provision for all acceptable usage'' (Pollock 1985:504-505). 

This is absolutely true, and it is only our {\it ṛṣi}-s or masters who had the vision to realise such a pragmatic and profound truth many centuries earlier than the rest of the world realised it.  Yet, our {\it śāstra}-s are special and unique due to their greatness, and grandness, vastness, to a much greater extent than their counterparts in any other society in the world, ancient or modern. Pollock altogether fails to see this dimension of {\it śāstra}. {\it Śāstra} in India, as opposed to rules or codes of any other civilisation, is much broader, comprehensive and wide-ranging; and this shows it gives utmost freedom to the practitioner.

\item Though Pollock recognizes the importance of taking account of traditional categories and concepts when attempting to understand the cultural achievements of ancient India, he doesn't seem to see that in India, {\it śāstra}-s are always considered as collection of experiences of predecessors which they express in a well-defined and systematic manner, and which serve as guidelines\index{sastra@\textit{śāstra}!guidelines} to the successors. 

In a way, {\it śāstra}-s make our job easy in any field of activity by showing us the trodden path and providing us with time-tested principles and techniques (and some {\it śāstra}-s even tell us short cuts) which can be straightaway employed. This only enables us to take off from the point where our ancestors had stopped, to improve upon what they have done, rather than repeating all their earlier trials, and learning things the long way. Thus, knowledge has always been incessantly evolving in India. This aspect of `evolution'\index{sastra@\textit{śāstra}!evolution} is completely ignored by Pollock.

\item Pollock seems to be not in favour of the idea of divine origin of all knowledge. But it is in fact the very bedrock of Indian philosophical thought. The {\it Upaniṣad}-s proclaim ``\textsl{yasmin vijñāte sarvam idaṁ vijñātaṁ bhavati}'' (\textsl{Muṇḍaka Upaniṣad}\index{Upanisad!Mundaka@Muṇḍaka} 1.3) $\{$That by knowing which everything else becomes known (\textsl{Translation mine})$\}$ indicating an all-inclusive broad umbrella, within the purview of which everything else falls. A similar expression can be found in the \textsl{Bhagavad-gītā}\index{Bhagavadgita@\textsl{Bhagavadgītā}} (7.2)
\begin{quote}
{{\sl jñānaṁ te'ham sa-vijñānam idaṁ vakṣyāmy aśeṣataḥ}} |\\
{\sl yaj jñātvā neha bhūyo'nyaj jñātavyam avaśiṣyate} ||
\end{quote}
$\{$I shall tell you without reserve about this knowledge ({\it jñāna}) together with [its] realization ({\it vijñāna}), knowing which there remains nothing further to be known here. ({\em Translation Swami Viveswarananda})$\}$

\item Even in the {\it kāvya}-s there are ample references that show that ours was a dynamic tradition that always adored, encouraged and honoured creativity and novelty. Says Kālidāsa:
\begin{quote}
purāṇam ity eva na sadhu sarvaṁ\\
na cāpi kāvyaṁ navam ity avadyaṁ | (\textsl{Mālavikāgnimitra} \index{Malavikagnimitra@Mālavikāgnimitra} 1.2)
\end{quote}
$\{$The fact of being anterior is not a validation for aptness. Poetry is not to be rejected just because it is novel. (\textsl{Translation mine})$\}$

\item Throughout his paper, Pollock attacks\index{sastra@\textit{śāstra}!attacks on} {\it śāstra} with words like \textsl{rigid, frozen, fixed, authoritative, cultural hegemony,\index{sastra@\textit{śāstra}!!hegemony} domination, dogmatic, prescriptive, rule-governance, regulation, injunctive, stipulative, legislative} and so on; but nowhere in his paper has he given even a single example from any {\it śāstra} which imposes stringent rules, or instigates `authoritarianism', or intimidates the practitioners of any dreadful consequences and/or leads to social oppression. Neither has he given any instance from Western culture, which he claims to give all scope for freedom and individual creativity. At least a couple of parallel examples in specific spheres of activity would have enhanced the efficacy of his assertions.

\item Well does Rajiv Malhotra say ``Pollock is unable to see that transcendence\index{sastra@\textit{śāstra}!tool to transcendence} is the fountain-head of creativity. For Hindus, a {\it śāstra} can be a tool leading to transcendence, be it {\it rasa}\index{rasa@\textsl{rasa}} in the performing arts, \ {\it jñāna} in {\it Vedānta},\index{Vedanta@Vedānta} or {\it samādhi}\index{samadhi@\textsl{samādhi}} in {\it yoga}.'' (Malhotra\index{Malhotra, Rajiv} 2016:116)

\item If Western students find Indian culture alien and strange, as claimed by Pollock, it is equally true that Indians find Western culture\index{Western culture} alien and strange. I think it is a matter of common sense that for anybody in this world, any new, hitherto unknown or unfamiliar culture will always seem alien. Thus, it is no big deal that Westerners are amazed by Indian culture and its practices.

\item Pollock is seen generalising the concept of `the West'. It is not clear what he actually means by the West and its culture. European countries such as Britain, France, Greece, Germany have their own independent and separate cultures, which are quite different, and sometimes totally opposed to the culture of America. Similarly, Central American and South American countries (the Incas,\index{Incas} Mayas and Aztecs)\index{Aztecs} also have their own distinct and unique cultures. Pollock seems to be a representative of the American culture by and large, and implies merely `America' when he says `the West'. Therefore, all his arguments in support of the `West' also become narrow and constricted. 
\end{enumerate}

\subsection*{Contrary references cited yet ignored}

\begin{enumerate}
\item He quotes Vātsyāyana's words from the {\it Kāma-sūtra} thus ``...all over the world there are only a handful of people who know the {\it śāstra}, while the practice (of it, or, the application of its principles) is within the grasp of many people.'' He goes on to say that, as for learning the {\it śāstra} itself, this is the necessary commencement of the tradition, and later serves to enhance the efficacy of the practice. 

He also mentions \textsl{Chāndogya Upaniṣad's}\index{Upanisad!Chandogya@Chāndogya} and Yaśodhara's\index{Yashodhara@Yaśodhara} view that {\it śāstra} makes practice `stronger', reliable, and consistent, unlike uninformed practice, whose effectivity is altogether fortuitous. (Pollock 1985:507)

\item While speaking about grammar,\index{grammar} he himself says ``...However detailed a descriptive grammar may be, it can never be totally exhaustive of linguistic practice. There are indeed passages in Pāṇini\index{Panini@Pāṇini} where this is evidently well understood, and where consequently it cannot be claimed that {\it śāstra} has made provision for all acceptable usage.'' (Pollock 1985:504-505)

\item He points out Kauṭilya's\index{Kautilya@Kauṭilya} expression -- ``One who knows {\it śāstra} but is inexperienced will come to grief in practical application.'' He refers to Kauṭilya's words ``\textsl{sarva-śāstrāṇy anukramya prayogam upalabhya ca}'' (\textsl{Artha-śāstra}\index{Arthasastra@\textsl{Arthaśāstra}} 1.8.25) -- meaning he ran through all the {\it śāstra}-s and observed actual practice. (Pollock 1985:509-510)

\item In the context of learning the science of economics, Manu\index{Manu} is quoted thus ``(this has to be learned) from the common people themselves.'' (Pollock 1985:509-510)

\item Pollock says Daṇḍin\index{Dandin@\textsl{Daṇḍin}} offers his own definition of the nature of poetry after having brought together (summarizing, or, consulting) the earlier {\it śāstra}-s on the subject, and observing actual practices -- 

``\textsl{pūrva-śāstrāṇi samhṛtya prayogān upalakṣya ca} | 

\textsl{yathā-sāmarthyam asmābhiḥ kriyate kāvya-lakṣaṇaṁ} ||'' 

\hfill (\textsl{Kāvyādarśa} 1.2).

In his footnote, he also makes a mention of Ānanda-vardhana who he says is far more descriptive than other {\it ālaṅkārika}-s. (Pollock 1985:510)

\item He quotes from \textsl{Suśruta-saṃhitā}\index{Susruta-samhita@Suśruta-saṃhitā} thus ``For it is only by combining both direct observation and the information of the {\it śāstra}-s that thorough knowledge is obtained.'' (Pollock 1985:510)

\item He cites from \textsl{Caraka-saṃhitā}\index{Caraka-samhita@\textsl{Caraka-saṃhitā}} thus ``Of all types of evidence, the most dependable is that directly observed.'' (Pollock 1985:510)
\end{enumerate}

\section*{Conclusion}

A thorough and more detailed investigation of Pollock's paper will reveal that there are some more serious shortcomings in his assumptions, logic as well as conclusions. In my opinion he has not been quite successful in proving that {\it śāstra}-s are opposed to progress, and thus hinder creativity. Neither has he provided evidence to show that {\it śāstra}-s in India inspire hegemony\index{hegemony} and social oppression,\index{social oppression} nor has he demonstrated the inherent nature of freedom and progress which he claims to be the crux of the Western civilisation.

The question that is persistently bothering me is -- has there been no {\it pūrvapakṣa} done by our traditional scholars for so many years for an article written as early as 1985? If yes, has it been effective? If not, it is highly unfortunate! The Swadeshi Indology conference is a great step towards filling this lacuna and building a home team of scholars who can give appropriate rebuttals to the likes of Pollock.

\section*{Acknowledgements}

I am extremely thankful to Prof.~K.~S.~Kannan for the faith he had in my abilities and for motivating me to take up this task - of studying and analysing an Indologist as big as Sheldon Pollock. I extend my thanks to Dr.~Vinay Acharya, Assistant Professor, Faculty of Vedanta, Karnataka Samskrit University, Bangalore, for his inspirational and invaluable guidance, inputs and reviews in the preparation of this paper. Thanks to the one and only, Sri Rajiv Malhotra, whose videos and books (especially \textsl{The Battle for Sanskrit}) opened the eyes of thousands of insiders like me, and but for whom this project of {\it pūrvapakṣa} and {\it uttarapakṣa}, would not have taken off on this massive scale.

\begin{thebibliography}{99}
\itemsep=2pt
\bibitem[]{chap9_item1}
Apte, Vasudeo Govind. (2007) (1933$^{2}$) \textsl{The Concise Sanskrit-English Dictionary}. Delhi: Motilal Banarsidass.

\bibitem[]{chap9_item2}
Dalal, C.D and Sastry, R.A. (1934) \textsl{Kāvyamīmāṁsā of Rājaśekhara}. (Revised and enlarged by K.S.~Ramaswami Sastri Siromani). 3rd Ed. Baroda: Oriental Institute.

\bibitem[]{chap1_item3}
Hanneder, Jurgen (2002) ``On the Death of Sanskrit''. \textsl{Indo-Iranian Journal}. pp 293-310. Brill Academic Publishers. 

\bibitem[]{chap9_item4}
\textsl{Kāvyamīmāṁsā} of Rājaśekhara. See Dalal and Sastry. 


\bibitem[]{chap1_item5}
Mahābhārata (1958). (No editor). Gorakhpur: Gita Press.

\bibitem[]{chap1_item6}
\textsl{Mālavikāgnimitra}. See Parab. 

\bibitem[]{chap9_item7}
Malhotra, Rajiv (2016) \textsl{The Battle for Sanskrit}. Noida: HarperCollins Publishes India.

\bibitem[]{chap9_item8}
Malhotra, Rajiv (2013) \textsl{Being Different}. Paperback edition. Noida: HarperCollins Publishes India.

\bibitem[]{chap9_item9}
\textsl{Muṇḍaka Upaniṣad}. See Singh. 

\bibitem[]{chap9_item10}
Parab, Kasinath Pandurang (Ed.). (1924) \textsl{The Mālavikāgnimitra of Kālidāsa}. (Revised by Wasudev Laxman Sastri Pansikar). 6th ed. Bombay: Nirnayasagar Press.

\bibitem[]{chap9_item11}
Pollock, Sheldon. (1985) ``The Theory of Practice and the Practice of Theory in Indian Intellectual History''. \textsl{Journal of the American Oriental Society} 105.3: pp 499-519.

\bibitem[]{chap9_item12}
Prahladacharya, D and Bhat, Haridasa (Ed.s) (2008) (1969$^{1}$) \textsl{Śrimad-viṣṇu-tattva-vinirṇayaḥ}. Bangalore: Poornaprajna Samshodhana Mandiram.

\bibitem[]{chap9_item13}
{\it Prasthāna-bheda} of Madhusūdhana Sarasvatī. (1912) (No editor). Srirangam: Sri Vani Vilas Press.

\bibitem[]{chap9_item14}
Ramachandra Rao, S.K. (2006) \textsl{Puruṣa Sūkta} (Text, Transliteration, Translation and Commentary). Bangalore: Sri Aurobindo Kapali Sastry Institute of Vedic Culture. 

\bibitem[]{chap9_item15}
Singh, Karan (Trans.). (1987) \textsl{Mundaka Upanishad} -- \textsl{The Bridge to Immortality}. Bombay: Bharatiya Vidya Bhavan.

\bibitem[]{chap9_item16}
\textsl{Śrīmad Bhagavad-gītā}. See Swami Vireswarananda. 

\bibitem[]{chap9_item17}
Swami Vireswarananda (Trans.) (1989) \textsl{Srimad Bhagavad-gītā}. Madras: Sri Ramakrishna Math.

\bibitem[]{chap9_item18}
\textsl{Viṣṇutattva-vinirṇaya}. See Prahladacharya and Bhat.
\end{thebibliography}
