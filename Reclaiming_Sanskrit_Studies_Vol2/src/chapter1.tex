\chapter{On Grammar and Royal Power}\label{chapter1}

\Authorline{Sowmya Krishnapur}
\lhead[\small\thepage\quad Sowmya Krishnapur]{}

\section*{Introduction}

In the fourth chapter of his {\sl The Language of Gods in the World of Men} (Pollock 2006), Professor Sheldon Pollock\index{Pollock, Sheldon} (hereafter Pollock) tries to establish the relation between Grammar\index{grammar} and royal/political power.\index{political power}  In the first section of the chapter, titled {\sl ``Grammatical and Political Correctness: The Politics of Grammar''} (Pollock 2006:162) Pollock tries to trace the evolution of this relationship. He states at the beginning of this section, 
\begin{myquote}
``The spread of a widely shared, largely uniform cosmopolitan style of Sanskrit inscriptional discourse would have been impossible without an equally vast circulation of the great kāvya exemplars of that style, accompanied by the philological instruments without which the very existence of such texts was unthinkable''.
\hfill (Pollock 2006:162) 
\end{myquote}

He gives several examples to show the spread of Sanskrit literature. Stating that along with literary works, the texts of literary art, metrics, lexicography, and related knowledge systems circulated throughout the Sanskrit Cosmopolis\index{Sanskrit cosmopolis} with the status of ``precious cultural commodities'', he says that they ``...came to provide a general framework within which a whole range of vernacular literary practices could be theorized'' (Pollock 2006:163). Taking the {\sl Kāvyādarśa} of Daṇḍin as an example, he traces its spread through South India, Srilanka, Tibet, and China. He concludes that ``All this makes Daṇḍin's {\sl Mirror} probably the most influential work on literary science in world history after Aristotle's Poetics'' (Pollock 2006:163). He makes similar observations about the spread of lexicons like {\sl Amarakośa}\index{Amarakosa@Amarakośa} and metrical texts like {\sl Piṅgaḷasūtra}\index{Pingalasutra@Piṅgaḷasūtra} and {\sl Vṛttaratnākara}.\index{Vrttaratnakara@Vṛttaratnākara} 

Pollock begins the discussion on grammar with the statement ``All that has just been described for {\sl kāvya} itself and for its other ancillary practices is equally true of the knowledge system known as {\sl vyākaraṇa},\index{vyakarana@vyākaraṇa} language analysis or, more simply if less precisely, grammar'' (Pollock 2006:164). In fact, he says that his observations hold truer for Grammar, ``...since it carried cultural and political associations in premodern South Asia far more potent than any other form of knowledge'' (Pollock 2006:164). According to Pollock, this qualifies Grammar to be considered separately, and by implication, exclusively, for a study of how Sanskrit knowledge systems circulated in the cosmopolis. This statement, in fact, serves as the anchor for this section. Saying ``It merits separate consideration'' (Pollock 2006:164), Pollock clearly states that his intention is to look at Grammar exclusively. Yet, at each step, he brings in a random mix of all different disciplines, while maintaining the focus on Grammar in the conclusions that he draws. A close analysis of the chapter reveals that the dexterity of Pollock lies in the sleight of tongue,\index{Pollock!Sleight of tongue} and not in appropriateness or accuracy of the examples that he chooses. Hence, great care needs to be exerted in analysing his examples, and the words or phrases that he employs.

In the very next paragraph, Pollock observes that the study of language was more highly developed in South Asia than anywhere else in the premodern world, and traces the roots of many basic conceptual components of Western modernity to substantive and theoretical learnings that their proponents gathered from premodern Indian linguistic thought. However, he cautions that this development of the study of language should not be seen as a purely abstract intellectual discipline ``as it is typically seen''. Clearly, Pollock is paving the way here to drag into this his favourite lens of political philology,\index{political philology} and give a whole new dimension to grammatical studies in India. He says, 
\begin{myquote}
``Understanding the Indian care for language also depends, to a significant degree, on understanding the place of language care in the Indian social-moral order, and that in part means grasping its relationship to political power\index{political power}'' 
\hfill (Pollock 2006:165)
\end{myquote}

\newpage

This clearly shows that the focus of Pollock is not on analysing the merits or otherwise of the millenia old tradition of Sanskrit grammar, not on the contributions of the Indian grammarians to the study of language, and certainly not on understanding the tenets of Vyākaraṇa-śāstra.\index{Vyakaranasastra@Vyākaraṇaśāstra} His sole objective is to establish a link between Grammar and political power, so as to further some of the claims that he makes in his pet theories of political philology.

Pollock reiterates this saying ``...rulership and Sanskrit grammaticality and learning were more than merely associated; they were to some degree mutually constitutive.'' (Pollock 2006:165). The main aim of this section of his text is to demonstrate this mutually constitutive relationship between grammar and political power.

This is demonstrated, according to Pollock, by three features (Pollock 2006:165):
\begin{enumerate}
\item Celebration of grammatical learning especially in kings.
\item Royal patronage of grammatical learning.
\item Competitive zeal among rulers everywhere to encourage grammatical creativity and adorn their courts with scholars who could exemplify the same. 
\end{enumerate}

Since Pollock claims that it was Grammar exclusively which enjoyed the privilege of such a close, mutually constitutive, association with political power, tradition expects him to defend his claims via the twin modes of marking off in the śāstric tradition, viz. {\sl ayoga-vyavaccheda} and {\sl anyayoga-vyavaccheda}\endnote{{\sl ayoga-vyavaccheda}\index{ayoga-vyavaccheda} and {\sl anyayoga-vyavaccheda}\index{anyayoga-vyavaccheda} are two standard techniques used in the {\sl śāstra}-s to demonstrate exclusivity. To prove an exclusive relationship between two entries A and B, one would have to prove two points -- first, that A is always associated with B, and that there is no case where the association is absent. This is termed {\sl a-yoga-vyavaccheda}, literally, ruling out non-association. The second is that A is never found in association with entities other than B. This is termed {\sl anya-yoga-vyavaccheda}, literally, ruling out association with any non-B.}. To elucidate: demonstrating the continuous, exception-free association of Vyākaraṇa with these features would be {\sl ayoga-vyavaccheda}; demonstrating that no other {\sl śāstra} enjoyed such association with political power would be {\sl anyayoga-vyavaccheda}. It is only when both these are demonstrated satisfactorily, would his observations, and the conclusions that he draws from them, be deemed acceptable by a discerning reader. 

Pollock himself declares that he is not aiming for {\sl anyayoga-vyavaccheda} -- i.e proving his claims with reference only to Vyākaraṇa, to the exclusion of other {\sl śāstra}-s. This is done by a very clever use of circuitous language,\index{Pollock!circuitous language} saying 
\begin{myquote}
``...Sanskrit learning itself became an essential component of power. The figure of the learned king became quickly established, especially the king learned in Sanskrit philology (and we may with justice speak of ``philology'' since ``grammar'' is often found to be used metonymically, standing for knowledge of lexicology, prosody, and the like, including literature)''
\hfill (Pollock 2006:166)
\end{myquote}

This unravels the first trick of his writing : Despite focussing exclusively on grammar in his preface and his conclusion, Pollock freely utilises all the different {\sl śāstra}-s, and even {\sl kāvya}-s, in drawing his examples and evidences. By using the term `metonymically', Pollock tries to justify this implicitly. But then, on what grounds would evidences built on a wide range of disciplines justify the conclusions that he draws pertaining to a specific and single discipline?

\section*{Celebration of Grammatical learning in kings}

For illustrating his first feature, i.e, celebration of grammatical learning in kings, Pollock mentions seven inscriptions,\index{inscriptions} through which he wishes to demonstrate that grammatical learning among kings was held in high esteem across the Sanskrit cosmopolis. These illustrations are examined below to see if they do actually serve his stated purpose. 

\section{Praśasti of Kṣatrapa Rudradāman,\newline Junagadh, 150 CE}\label{chap3-sec1}\index{inscriptions!Junagadh}

The relevant portion of the inscription which is referred to by Pollock reads,
\begin{myquote}
{{\sl ``hastocchrayārjitorjita-dharmānurāgeṇa śabdārtha-gāndharva-nyāyādyānāṁ vidyānāṁ mahatīnāṁ pāraṇa-dhāraṇa-vijñāna-prayogāvāpta-vipulakīrtinā''}}

\hfill (Hultzsch 1905-06:44) (hyphenation ours)
\end{myquote}

Pollock himself translates it thus, ``one who has won wide fame by his theoretical and practical mastery and retention of the great knowledges, grammar, polity, music, systematic thought, and so on'' (Pollock 2006:166). While there is no doubt that the word {\sl śabdārtha} used in the inscription does refer to Grammar, one must note that Pollock ignores the other streams of knowledge such as music and logic, all {\sl equally} held aloft as great ({\sl mahatī}) which are mentioned alongside, and keeps his focus exclusively on grammar. 

\section{Praśasti of Samudragupta, Allahabad, 4th cent. CE}\label{chap3-sec2}\index{inscriptions!Allahabad}

Pollock quotes three statements from this inscription -- 
\begin{myquote}
{\sl ``yasya prajñā-niṣaṅgocita-sukha-manasaḥ śāstra-tattvārtha-bhartuḥ...'', ``vaiduṣyaṁ\break tattva-bhedi...'', ``ādhyeyaḥ sūkta-mārgaḥ kavi-mati-vibhavotsāraṇaṁ cāpi kāvyam''} (Fleet 1960:6). 
\end{myquote}

Even going by Pollock's own translations of the sentences [``master of the true meanings of the {\sl śāstras}'', ``a man of truth-piercing learning'', whose ``way of poetry merits the closest study, and whose literary work puts to shame the creative powers of [other] poets.'' (Pollock 2006:166)], there is no reference to Grammar, either explicit or implicit, in this inscription.

\section{Copper plate record of Durvinīta, 6th cent. CE}\label{chap3-sec3}\index{inscriptions!Copper plate of Durvinita}

Pollock quotes a portion of the text of this inscription which reads 
\begin{myquote}
{{\sl ``śabdāvatāra-kareṇa devabhāratī-nibaddha-vaḍḍa-kathena kirātārjunīye pañca\-daśa-sarga-ṭīkākāreṇa durvinīta-nāmadheyena''}} 
\hfill \hbox{({\sl hyphenation ours})}
\end{myquote}

In Pollock's own words, ``the king is praised as the man who composed the Descent of Language [now lost], and rewritten the [Paishachi] {\sl Bṛhatkathā} [Great Tale] in the language of the Gods'' (Pollock 2006:166). The inscription also mentions that he composed a commentary on the fifteenth canto of the {\sl Kirātārjunīya}. However, none of the works of Durvinīta are now available, and there is no evidence to even guess what kind of a work the {\sl Śabdāvatāra} was. This being the case, it seems rather far-fetched to connect Durvinīta\index{Durvinita} with Grammar, going merely by the title of an unknown text. 

\section{{\sl\bfseries Praśasti} of King Sañjaya of Java, 732 CE}\label{chap3-sec4}\index{inscriptions!in Java}

This inscription praises the king as {\sl ``śāstra-sūkṣmārtha-vedī'',} ``one who understood the subtle points of {\sl śāstra}-s''. Why should the term {\sl śāstra} used here be interpreted as a reference to Vyākaraṇa of all? It is a term which can encompass all disciplines of knowledge. In relation to a king, if we need to associate a particular discipline, isn't it more logical to associate Arthaśāstra (or Dharmaśāstra or Nītiśāstra), rather than Vyākaraṇa?

\section{Praśasti of Jaya-Indravarman I, Champa, 970 CE}\label{chap3-sec5}

Pollock says that this king is celebrated explicitly as an expert in Pāṇini's grammar and the {\sl Kāśikā}.  The original text of this inscription\index{inscriptions!in Champa} could not be traced by me. However, a similar wording is found in the {\sl praśasti} of 840 CE, related to Indravarman III of Champa. This inscription reads, 
\begin{myquote}
{\sl ``mīmāṁsa-ṣaṭtarka-jinendra-sūrmiḥ}

{\sl sa-kāśikā-vyākaraṇodakaughaḥ |}

{\sl ākhyāna-śaivottara-kalpa-mīnaḥ}

{\sl paṭiṣṭha eteṣviti satkavīnām''}

\hfill (Majumdar 1927:III 138) ({\sl hyphenation ours})
\end{myquote}

Since this inscription specifically mentions Vyākaraṇa and {\sl Kāśikā},\index{Kasika@Kāśikā}  we can freely admit this as an illustration for Pollock's point. However, it must be noted that even this does not exclusively refer to Vyākaraṇa. Among all the {\sl śāstra}-s mentioned here, including Mīmāṃsā, Tarka, Śaiva, Uttarakalpa, and so on, why is Pollock's spotlight only on Vyākaraṇa?

\section{Praśasti of Sūryavarman I, Angkor, 1002 CE}\label{chap3-sec6}\index{inscriptions!in Angkor}

Pollock himself provides the original reading of this inscription --
\begin{myquote}
{\sl ``bhāṣyādi-caraṇā kāvya-pāṇiḥ ṣaḍ-darśanendriyā.}

{\sl yanmatir dharmaśāstrādi-mastakā jaṅgamāyate''}

\hfill \hbox{(Pollock 2006:167 FN) ({\sl hyphenation ours})}
\end{myquote}

The English translation of this verse, as provided by Pollock, is a beauti\-ful illustration of his misleading usage of language.\index{Pollock!misleading language}  He translates the verse as, 
\begin{myquote}
``one whose mind itself truly seemed a body that could move, with the [Great] Commentary [of Patañjali on Pāṇini’s grammar] and the rest [of the grammatical treatises] for its feet, [the two kinds of] literature [prose and  verse]  for  its  hands,  the  six  systems  of  philosophy  for  its  senses,  and {\sl dharma} and the other {\sl śāstra}-s for its head.'' \hfill (Pollock 2006:166)
\end{myquote}

Words supplied by Pollock, placed by him within square brackets, merit careful reading. While the original verse mentions {\sl bhāṣya} without any attributes, Pollock translates it as `the [Great] Commentary [of Patañjali on Pāṇini's grammar]'. It is well known that the epithet {\sl mahat} on the term {\sl bhāṣya} is used exclusively for the Vyākaraṇa {\sl bhāṣya}. However, without this epithet, the word {\sl bhāṣya} is a generic term whose instances are found in all {\sl śāstra}-s. This is also backed by the fact that the other terms used alongside in the inscription, namely {\sl kāvya} and {\sl darśana} are also generic terms, not referring to any specific texts. Pollock provides the epithet `Great' himself to drag the {\sl Mahābhāṣya} into this, and using that, he extrapolates the term {\sl ādi} to refer to the rest of grammatical treatises. Clearly, he is resorting to flights of fancy,\index{Pollock!flights of fancy} even while there is no firm ground to leap from!

\section{Description of the Veṅgi Cālukya king Rājarājanarendra, by the Telugu poet Nannayya}\label{chap3-sec7}

Here, Pollock says that the king is described as 
\begin{myquote}
``lucid in thought, trained in the science of Kumāra [the Kātantra of Śarvavarman], a good Cāḷukya, luminous as the moon, [who] finds peace in studying the ancient texts.''
\hfill (Pollock 2006:166)
\end{myquote}

{\sl Kātantra}\index{Katantra@Kātantra}  is of course a grammatical work. 

Thus, out of a total of seven examples quoted by Pollock, there are only three instances of specific mention of Vyākaraṇa and its texts. Even among those, except in the last instance, Vyākaraṇa is mentioned alongside and on par with the other {\sl śāstra}-s -- neither exclusively, nor as the most important. The other four examples are related to Vyākaraṇa only by a certain stretch of imagination. 

It is left to discerning readers to decide if Pollock is justified in claiming that `celebration of grammatical learning in kings' is one of the features which demonstrates ``a mutually constitutive association between Sanskrit grammaticality and rulership''.

\section*{Royal patronage of Grammatical Learning}

Perhaps realising the weakness of his position, Pollock tries to prove in the next section  that the kings owed their excellence of language to the mastery of grammar. It is the case that mostly rather rare is the knowledge of {\sl śāstra}-s\index{sastra (śāstra)} and the poetic skills of kings which are praised in the {\sl praśasti}-s,\index{prasasti@praśasti} and a direct acknowledgement of their grammatical accomplishments is rather rare; how else then can he build on (non-existent) pairing between royalty and Vyākaraṇa? He says that the rulers may have been thought to possess ``some natural capacity for realizing the linguistic norms\index{linguistic norms} of Sanskrit''.

As in any other language, excellence in Sanskrit is accomplished by practice and application to the language, and not just by a study of grammar. However, Pollock denies this by using his own theory which he propounds in his paper on {\sl Śāstra}-s, saying ``In the Sanskrit thought world, normativity\index{sastra (śāstra)!normitivity} was always conceived of as pre-existent to any actual instantiation: practices conformed to rules, while rules were never constituted out of practices'' (Pollock 2006:167). Using this ``norm'', he builds a reverse relationship where kings derived their power through linguistic excellence, and such linguistic excellence was built through a thorough study of grammatical texts.
\begin{myquote}
``Excellence in the command of the Sanskrit language was therefore something kings had to achieve through mastery of a theoretical body of material that already established that excellence, and all of them everywhere could achieve this to the same degree and in the same manner, assuming they were in possession of the right textual instruments. This attainment, as demonstrated by the references just cited, was one among other celebrated royal attributes and so was as essential to kingship as the martial power, political sagacity, physical beauty, fame, and glory''
\hfill (Pollock 2006:167)
\end{myquote}

Thus emerges his ``mutually constitutive'' relationship between grammar and political power, wherein political correctness derives from grammatical correctness, and which is why kings went out of their way to encourage grammar! To quote Pollock:
\begin{myquote}
``Since it was theory that underlay the royal practice of grammatical correctness, which itself was seen as a component of political correctness, it stands to reason that power should have actively cared for grammar by sponsoring the production of grammatical texts and ensuring their continued study.''
\hfill (Pollock 2006:167)
\end{myquote}

Pollock quotes Hartmut Scharfe as providing the initial spark of such thought, where Scharfe opines that a strong case could be made for the importance of princely patronage of grammatical studies, and traces three spurts of activity in the 5th, 11th and 17th centuries coinciding with the period of strong kingdoms. Scharfe states this, as Pollock himself admits, ``in passing'', with no elaboration on grounds of historical data only or substantiation. However, Pollock is not satisfied by merely acknowledging that royal encouragement was (as in the case of all other literary/cultural activity) useful in promoting grammatical activity to an extent. He goes to the extent of claiming that,
\begin{myquote}
``...princely patronage was not just vaguely ``important'' to Sanskrit philology, and the history of the relationship between polity and philology was not just episodic, punctuated by spurts that nevertheless remain obscure in their origins and mysterious in their effects. On the contrary, royal power seems to have provided the essential precondition for the flourishing of the postliturgical philological tradition---as philology likewise provided a precondition for power--- from the birth of the Sanskrit cosmopolitan order throughout its lifetime.''

\hfill (Pollock 2006:168)
\end{myquote}

These sentences are crucial in understanding the implications of Pollock, which are threefold:
\begin{enumerate}
\item There would be no existence of Vyākaraṇa without royal patronage. 
\item The power of kings hinged on flourishing grammatical studies.
\item Such a relationship between grammar and political power can be traced through the entire lifetime of the Sanskrit cosmopolitan order (and, by implication, the entire lifetime of Vyākaraṇa-śāstra itself).
\end{enumerate}

It is needless to state that Pollock would have to produce strong supporting material while making such claims - claims which go against everything that traditional students and practitioners of the system believe in. However, Pollock is wont to making strong asserverations based on flimsy foundations. He ventures on relentlessly, however, to elaborate on the following instances in order to prove the existence of his fancied concomitance\index{Pollock!fancied concomitance} --

\section{Pāṇini’s relation with power}\label{chap3-sec8}

Pollock admits that, ``for the earliest period of Sanskrit grammar, the historical data are too thin to demonstrate the mutually constitutive relationship of grammar and power with much cogency'' (Pollock 2006:168). Yet ``for what it is worth'', he reports a legend mentioned by the seventh century Chinese pilgrim Xuangzang.\index{Xuangzang} The use of such a phrase (cited above in inverted commas) in serious academic writing by a veteran scholar is surprising, nay shocking. Doesn't it make it amply clear that the author is stating something despite knowing that it is by no means a trustworthy statement? 

The legend in question states that on completing his grammar, Pāṇini\index{Panini@Pāṇini} offered it to his king, who regarded it highly, and ordered that all people in the country should learn the book, and offered a reward of one thousand gold coins for anybody who could recite it by heart. 

This association of Pāṇini with a king is unsupported by anything in his text, or in the vast commentarial literature on his text, or any stray historical references either. Yet, Pollock tries to build this relationship based on a legend heard somewhere, more than a thousand years after Pāṇini's time. In addition, he also uses this legend to hint that the study of grammar in India was, right from the beginning, imposed by rulership. Does a trifling hearsay suffice to support the extraordinary claim that Vyākaraṇa-śāstra owes its very existence to political patronage?

\section{Patañjali and political power}\label{chap3-sec9}

Pollock again admits that the evidences offered by him are still ``slender but suggestive'' (Pollock 2006:168). He builds his case on two quotations from the {\sl Mahābhāṣya}\index{Mahabhasya@\textit{Mahābhāṣya}} which have been widely discussed by scholars since a long time. These are the expressions 
\begin{myquote}
{\sl `iha puṣyamitraṃ yājayāmaḥ'}

(``Here we conduct a sacrifice on behalf of Puṣyamitra''), and

{\sl `aruṇad yavanaḥ sāketam'}

(``The {\sl yavana} besieged Sāketa''),
\end{myquote}
which are offered by Patañjali\index{Patanjali@Patañjali} as illustrations for certain rules. Pollock particularly relies on the second statement and extrapolates from later practices to assert that through this, Patañjali is trying to identify himself.

Let us examine the context of the two citations from the {\sl Mahābhāṣya} - {\sl `aruṇad yavanaḥ sāketam'} is offered by Patañjali as an illustration to the {\sl vārttika `parokṣe ca lokavijñāte prayoktur darśanaviṣaye laṅ vaktavyaḥ'}, under {\sl `anadyatane laṅ'} ({\sl Aṣṭādhyāyī} 3.2.111). Pānini prescribes {\sl liṭ lakāra} for cases of past tense where the event was not witnessed directly by the speaker ({\sl parokṣa}). Regarding that, the {\sl vārttika} states that famous events happening in the lifetime of the speaker, which could have been witnessed by him, govern the usage of {\sl laṅ} rather than {\sl liṭ}. Here, two examples are offered by Patañjali -- {\sl `aruṇad yavanaḥ sāketam'}, and {\sl `aruṇad yavano mādhyamikām'}. The implication is that the attack on Sāketa and Mādhyamikā happened during the lifetime of Patañjali (and were not witnessed by him directly). These two statements have been discussed widely over the last two centuries for their grammatical/semantic implications, and those discussions are not relevant here.

To pursue Pollock's argument: wherever later grammarians wish to illustrate this rule, they resort to one of the two ways -- either repeat the examples in the {\sl Mahābhāṣya}, or come up with their own examples. Some, like the author of {\sl Kāśikā}\index{Kasika@Kāśikā}, and the author of {\sl Prasāda}, a commentary on {\sl Prakriyākaumudī}, simply repeat the examples of the {\sl Mahābhāṣya}. However, these would not be strictly correct, as these events are not happening during their lifetime. Some other grammarians proffer their own examples, which they use for self-identification (Burgess 1874:266-7).

For example, Abhayanandin's {\sl Jainendra-Vyākaraṇa} has {\sl `aruṇan mahendro mathurām'} (``Mahendra attacked Mathurā''); {\sl Hemacandra-Vyākaraṇa} has {\sl `aruṇat siddharājo’vantīn'} (``Siddharāja attacked Avanti'') and {\sl `ajayat siddhaḥ saurāṣṭrān'} (``Siddha conquered Saurāṣṭra''). The {\sl Śākaṭāyana-śabdānuśāsana}, mentioned by Pollock later, has {\sl `adahad amoghavarṣo'\-rātīn'} (``Amoghavarṣa burnt the enemies''). In these examples, Siddharāja quoted by Hemacandra was none other than his own patron, viz. the Cālukya monarch, Jayasiṃhasiddharāja the second. Śākaṭā\-yana mentions his own patron, the Rāṣṭrakūṭa king Amoghavarṣa.\break Based on this, Pollock draws the conclusion that 
\begin{myquote}
``...Patañjali -- or the earlier grammarian he may have been citing - was seeking, in a very subtle way that virtually all later grammarians were to adopt, to identify himself, his patron and the place where he worked. That location was obviously courtly, whether it was the court of the Śuṅga overlords (the dynasty to which Puṣyamitra belonged) who succeeded the Maurya kings or another court three centuries later''

\hfill (Pollock 2006: 169)
\end{myquote}

However, in drawing the above conclusion, Pollock ignores\index{Pollock!ignores difference in examples} the fundamental difference between the examples offered by Patañjali and those offered by the later court-grammarians. Patañjali is stating the case of a foreign invasion, whereas the later grammarians are citing the conquests carried out by their patrons. If the examples were exactly similar, we would have to conclude that Patañjali was a grammarian in the court of the invading {\sl yavana}. Is this accepted by Pollock? Obviously not, as he says that the location of Patañjali was the court of either Puṣyamitra or some other later king.

When the difference in approach is so glaring, and when there is no mention of any ``patron king'' in Patañjali's statements, how can Pollock be justified in concluding that his location was ``obviously courtly''? Why can it not be the case that Patañjali is simply stating a famous event of his lifetime he was no witness to?

\section{The case of Kumāralāta}\label{chap3-sec10}

The next two grammarians that Pollock considers are Kumāralāta\index{Kumaralata@Kumāralāta} and Śarvavarman. Students of grammar should be excused for wondering who Kumāralāta is, as his name is unheard of in the vast grammatical literature, both Pāṇinian and non-Pāṇinian. Pollock himself says that 
\begin{myquote}
``Kumāralāta is known as a grammarian only through fragments of his work discovered in central Asia and brilliantly analyzed at the beginning of the century by Heinrich Lüders.''
\hfill (Pollock 2006:169)
\end{myquote}

The Buddhist Kumāralāta is famous as the founder of the {\sl Sautrāntika} sect. Belonging to the genre of {\sl Dṛṣṭānta-paṅkti} (collection of moral stories which illustrate the principles of Buddhism), his work {\sl Kalpanā-maṇḍitaka} is also known as {\sl Sūtrālaṅkāra}. Pollock says, ``The Kuśāṇa emperor Kaniṣka appears in two of the tales in Kumāralāta's story collection, but no further evidence is available to determine just how close the grammarian's association with the court may have been'' (Pollock 2006:169).

We need to consider the following facts here --

The authorship of {\sl Sūtrālaṅkāra} itself is not settled; and this text is available fully in Chinese and only partially in Sanskrit, and is considered by many to have been penned by Aśvaghoṣa.

\newpage

Even if {\sl Sūtrālaṅkāra} is accepted to be by Kumāralāta, it does not offer any conclusive evidence of the author's relationship with the court of Kuśāṇa emperor. A few points omitted\index{Pollock!points omitted} by Pollock, are also worth noting. The {\sl Sūtrālaṅkāra} is a text with nearly two hundred stories, mostly featuring heroes from all walks of life, including brahmins, {\sl saṁnyāsin}-s, {\sl bhikṣu}-s, merchants, painters, washermen and so on. Further, even the Mauryan emperor Aśoka figures in three of the stories of this collection (Nariman 1923:196). Ignoring all this, how can Pollock postulate that Kumāralāta was in the court of Kaniṣka, just because Kaniṣka features in two stories of the collection?

So, in this illustration offered by Pollock, Kumāralāta's status as a grammarian itself is based on feeble evidence as only portions of the text are available. Further, Kumāralāta's association with Kuśāṇa emperor Kaniṣka is based on flimsy evidence from another text supposedly authored by him, but whose very authorship is debated. Thirdly, in the very next paragraph, Pollock remarks on the similarities between this text and the {\sl Kātantra-Vyākaraṇa}, and concludes that it makes more sense to ``assume that the Kātantra was adopted and expanded by the northern Buddhists than the other way around'' (Pollock 2006: 170). Pollock is also certain that the {\sl Kātantra} antedates Candragomin's\index{Candragomin} work, which is reasonably securely assigned to the mid-fifth century. 

All this means that, in one paragraph, Pollock forcibly associates\index{Pollock!forcibly associates} Kumāralāta with the Kuśāṇa emperor Kaniṣka of the second century CE. And in the very next paragraph, he pushes him to later than fifth century CE. Though Pollock is very clever in that he does not explicitly associate Kumāralāta with Kaniṣka, the implication is very obvious. Also, if that is not what he is implying, why bring in Kumāralāta at this point at all?

\section{Śarvavarman, author of the {\sl\bfseries Kātantra-Vyākaraṇa}}\label{chap3-sec11}

The next grammarian considered by Pollock is Śarvavarman,\index{Sarvavarman@Śarvavarman} the founder of a non-Pāṇinian system of grammar called {\sl Kātantra-Vyākaraṇa}. This system is popularly believed to have originated by the grace of Lord Kārttikeya. Thus, scholars hold that it is this system which is intended by the word {\sl kaumāra} in the popular verse which enumerates the nine traditions of Vyākaraṇa --
\begin{myquote}
{{\sl aindraṃ cāndraṃ kāśakṛtsnaṃ kaumāraṃ śākaṭāyanam}} |\\ 
{{\sl sārasvataṃ cāpiśalaṃ śākalaṃ pāṇinīyakam}} || 
\end{myquote}
Pollock fills two pages with details on the possible reason for the origin of {\sl Kātantra-Vyākaraṇa}, the relationship between {\sl Kātantra} and {\sl Kumāralāta-Vyākaraṇa}, the extent of spread of {\sl Kātantra}, endowments in South India specifically for its study etc. 

Wondering about the relevance of all this to the matter at hand, astute readers are apt to notice this -- Pollock opens his writing on Śarvavarman by stating that ``the grammarian Śarvavarman, author of the {\sl Kātantra}, is placed by legendary accounts at the Sātavāhana court in perhaps the second century.'' (Pollock 2006:169). Further, he says, ``This location may receive some confirmation in a remark of Xuanzang's biographer, who reported that ``recently a Brahman of southern India again shortened [Pāṇini’s grammar] to twenty-five hundred stanzas for the king of South India.'' (Pollock 2006:170). 

That Pollock himself does not consider this report to be authentic stands proved - both by the fact that he uses the term `legendary accounts', and by his explicit statement about the certainty of {\sl Kātantra} antedating Candragomin's work of mid-fifth century CE. There is no evidence offered in the entire length of the next two pages of writing, for connecting Śarvavarman to any political power. Yet, at the end, Pollock concludes 
\begin{myquote}
``Even so, whether it was Śarvavarman or Kumāralāta who composed the original Kātantra, there is little doubt that the author was closely associated with a ruling power, whether in the South or in the North.''

\hfill (Pollock 2006:171) 
\end{myquote}
This is an extraordinary style\index{Pollock!extraordinary style}  of presenting a case, where a claim made in the introduction is simply reiterated at the end as the conclusion, without even an attempt to defend it in the span in between. The ``proof'' for Śarvavarman’s association with royal power is a legendary account (which doesn't even contain any specific reference to him or his patron); the ``proof'' for Kumāralāta's association with royal power, as demonstrated above, is the mention of a king in but two among two hundred stories of a story-collection of debated authorship. Yet, Pollock confidently states that ``there is little doubt that the author was closely associated with a ruling power''. 

\section{Revival of the study of Mahābhāṣya by two kings of Kashmir}\label{chap3-sec12}

Pollock quotes two portions from the {\sl Rāja-taraṅgiṇī}\index{Raja-tarangini@Rāja-taraṅgiṇī} which narrate accounts of revival of the study of the {\sl Mahābhāṣya} by two separate kings. The first is the instance of King Jayāpīḍa,\index{Jayapida@Jayāpīḍa}  which is given thus ({\sl Rāja-taraṅgiṇī}) --
\begin{myquote}
{{\sl utpatti-bhūmau deśe’smin dūradūra-tirohitā}} |\\
{{\sl kaśyapena vitasteva tena vidyāvatāritā}} || 4.486\\
{{\sl deśāntarād āgamayya vyācakṣāṇān kṣamāpatiḥ}} |\\ 
{{\sl prāvartayata vicchinnaṁ mahābhāṣyaṁ sva-maṇḍale}} || 4.488\\
{{\sl kṣīrābhidhāc chabdavidyopādhyāyāt sambhṛta-śrutaḥ}} |\\
{{\sl budhaiḥ saha yayau vṛddhiṁ sa jayāpīḍa-paṇḍitaḥ}} || 4.489  
\end{myquote}
This story states the the knowledge of Vyākaraṇa had disappeared from its very birthplace, and the king brought scholars from a different place to revive the study of the {\sl bhāṣya} in his kingdom. He himself learnt from a teacher named Kṣīra, and, due to his association with scholars, became a learned man himself, honoured as Jayāpīḍa-paṇḍita. Jayāpīḍa's reign has been determined by historians to be mid-eighth century CE. A second passage from the {\sl Rāja-taraṅgiṇī} is also mentioned by Pollock, where an earlier king by name Abhimanyu performed a similar feat in the fourth century CE ({\sl Rāja-taraṅgiṇī}).
\begin{myquote}
{{\sl candrācāryādibhir labdhvā deśāt tasmāt tadāgamam}} |\\ 
{{\sl pravartitaṁ mahābhāṣyaṁ svaṁ ca vyākaraṇaṁ kṛtam}} || 1.176
\end{myquote}
Here, it is stated that upon receiving his command, Candra and other preceptors started the study of {\sl Mahābhāṣya}, and also created their own system of grammar. Based on this, Pollock states that 
\begin{myquote}
``clearly, for Kalhaṇa\index{Kalhana@Kalhaṇa}  at least, the stories of the kings Abhimanyu and Jayāpīḍa are meant to be symmetrical as well as to convey a sense of the central place of royal patronage in the fostering of systematic Sanskrit knowledge, especially philological knowledge.''
\hfill (Pollock 2006:172)
\end{myquote}
While this may appear true at first sight, two points are to be noted. First is the fact that Jayāpīḍa was interested not just in grammar, but in all disciplines of knowledge. In the following verses of the {\sl Rāja-taraṅgiṇī} (4.494 -- 4.497), Kalhaṇa gives a long list of scholars patronised by Jayāpīḍa. They include the rhetoricians Udbhaṭa and Vāmana, Dāmodara-gupta (author of {\sl Kuṭṭinī-mata}, a text on Kāma-śāstra), and poets like Manoratha. Also, going by the description, the scholarly Jayāpīḍa appears to be more of an exception rather than the rule. If all the rulers were uniformly interested in encouraging grammar, or indeed, as Pollock puts it, ``it was a royal obligation to ensure the stability and continuation of the grammatical order'' (Pollock 2006:173), how is it that the study of {\sl Mahābhāṣya}, which was revived by Abhimanyu in the fourth century CE, utterly disappear ({\sl `dūradūra-tirohitā'}) in the first place, within a span of three centuries when Jayāpīḍa had to revive it again? Indeed, finding no scholars of grammar in Kashmir, Jayāpīḍa had to import them from other places. 

If Pollock accepts both instances mentioned in the {\sl Rāja-taraṅgiṇī}, he would have to accept that, rather than being fervent supporters of grammatical knowledge, the kings of Kashmir were so averse to it that the tradition of grammatical studies was completely uprooted twice within the span of a few centuries.

\section{Śākaṭāyana Vyākaraṇa}\label{chap3-sec13}

Pollock then mentions Śākaṭāyana,\index{Sakatayana@Śākaṭāyana}  the court-grammarian patronised by the Rāṣṭrakūṭa king Amoghavarṣa\index{Amoghavarsha@Amoghavarṣa} of the eighth century CE. He composed a new system of grammar called the {\sl Śabdānuśāsana}, and also an auto-commentary on it which was titled {\sl Amogha-vṛtti} in honour of his patron. 

\section{Halāyudha, the author of {\sl Kavirahasya}}\label{chap3-sec14}

Patronised by the Rāṣṭrakūṭa king Kṛṣna the third, Halāyudha\index{Halayudha@Halāyudha} composed a {\sl praśasti} titled {\sl Kavi-rahasya} in praise of his patron. The unique feature of this poem is that it attempts to illustrate the various forms of meanings of all verbal roots. For example, this is the verse quoted by Pollock. 
\begin{myquote}
{{\sl vetti sarvāṇi śāstrāṇi garvo yasya na vidyate}} |\\
{{\sl vintte dharmaṃ sadā sadbhis teṣu pūjāṃ ca vindati}} || 49 
\end{myquote}
This verse demonstrates the use of the verbal root {\sl vid} in four different senses, along with its form (in the Present Tense) in each sense.

Poems like {\sl Kavi-rahasya}, composed with an intention of illustrating the rules of Vyākaraṇa or Alaṅkāra-śāstra, belong to a popular genre of {\sl kāvya}-s called {\sl śāstra-kāvya}. Under the pretext of {\sl śāstra-kāvya}-s, Pollock also mentions {\sl Bhaṭṭi-kāvya}.\index{Bhatti-kavya@Bhaṭṭi-kāvya} {\sl Bhaṭṭi-kāvya}, also known as {\sl Rāvaṇa-vadha}, is one of the oldest and the foremost of {\sl śāstra-kāvya}-s, where Bhaṭṭi illustrates various Pāṇinian {\sl sūtra}-s topicwise, even as he narrates the story of Śrīrāma. The final verse of this poem is,
\begin{myquote}
{{\sl kāvyam idaṁ vihitaṁ mayā}}\\
\phantom{\quad} {{\sl valabhyāṁ śrīdharasena-narendra-pālitāyām}} |\\
{{\sl kīrtir ato bhavatān nṛpasya tasya}}\\ 
\phantom{\quad} {{\sl  premakaraḥ kṣitipo yataḥ prajānām}} || 22.35 
\end{myquote}
It means, ``This poem was composed by me in Valabhī, which is ruled by Śridharasena. May the king attain fame through this, since he is engaged in delighting the subjects''. By mentioning this poem in this context, Pollock is clearly implying a royal association, though, as he himself admits, ``...so far as can be determined from the text itself the narrative was not specifically intended to map against the life of the ruling overlord and cannot easily be read that way.'' (Pollock 2006:174). 

\section{Hemacandra, author of the {\sl\bfseries Dvyāśraya-kāvya}}\label{chap3-sec15}

At the end of the twelfth century CE, Hemacandra\index{Hemacandra} composed this poem illustrating the rules of both Sanskrit and Prakrit grammars composed by himself, while narrating the history of his patrons -- Jayasiṃha-siddharāja and Kumārapāla.

Thus, among the eight evidences offered so far by Pollock in support of his second claim (that rulers went out of their way to encourage and promote grammatical studies), it has been demonstrated that the first four instances show no definite association with royal power. The fifth case of two kings of Kashmir is more illustrative of the apathy of kings towards grammar, rather than encouragement. It is only the last three examples which demonstrate definite cases of grammatical texts being composed by courtly scholars.

Even Pollock must have been apprehensive as to whether his arguments suffice to demonstrate the association of grammar with royal power `throughout its lifetime'. Hence, he seeks to expand his domain by stating ``If for the purposes of analyzing the interrelationship between power and philology we widen the genre-domain of {\sl śāstrakāvya} to include the illustration of not only grammatical but also rhetorical norms ({\sl alaṅkāra-śāstra}\index{alankarasastra@alaṅkāra-śāstra}), as the latter part of Bhaṭṭi's work in fact does, we perceive a vast field of scholarly poetic texts on kings and literary culture'' (Pollock 2006:174). Having set out to portray the exclusive and superlative association of grammar with political power, why this reliance now on just the other streams of knowledge?

Yet, Pollock announces confidently that ``This brief survey of power and philology could easily be extended to include almost every important intellectual who wrote on grammar in the Sanskrit Cosmopolis'' (Pollock 2006:175). To this end, he throws in the names of Nārāyaṇa Bhaṭṭatiri, Bhaṭṭojī Dīkṣita, Kauṇḍabhaṭṭa and Nāgeśabhaṭṭa.

Nārāyaṇa Bhaṭṭatiri\index{Narayana Bhattatiri@Nārāyaṇa Bhaṭṭatiri} (16$^{\text{th}}$ century CE) composed {\sl Prakriyā-sarvasva} at the insistence of Devanārāyaṇa, the ruler of Ambalapuzha. He mentions this fact right at the beginning of his text. Bhaṭṭojī Dīkṣita\index{Bhattoji Diksita@Bhaṭṭojī Dīkṣita} and Kauṇḍabhaṭṭa,\index{Kaundabhatta@Kauṇḍabhaṭṭa} two famed names in the field of Pāṇinian grammar, were patronised by the post-Vijayanagara Nāyaka kings of Keladi in the 17$^{\text{th}}$ cent. CE, as stated by Pollock. However, there is not even a mention of their patron kings anywhere in their texts. Bhaṭṭojī Dīkṣita does not take the names of any patron even among the examples that he gives to illustrate rules in his text. It is highly doubtful as to what political mileage or power the kings drew by patronising these grammarians. It is more likely that they considered the patronage of all learning and arts as an essential component of good kingship, and not as any potent source of royal power. Nāgeśabhaṭṭa,\index{Nagesabhatta@Nāgeśabhaṭṭa} the foremost among modern grammarians, lived in the 18$^{\text{th}}$ century, when ``the sun of Sanskrit cosmopolitanism had already set'' (Pollock 2006:175). He was patronised by a king called Rāma in the kingdom of Śṛṅgaverapura, as declared by himself in the prelude to one of his texts ({\sl Laghu-śabdendu-śekhara} verse 2). However, even his texts have absolutely no reference to his patron king any more than a mention of his name. 

After this, in line with his usual practice, Pollock provides a long list of rhetoricians who worked in royal courts, which fills a good deal of space but adds nothing to his aim of demonstrating that grammar carried political associations more potent than any other form of knowledge.

The inherent weakness in these evidences\index{Pollock!weakness in evidences} need not be proved by anybody else. Pollock himself admits it as ``rather vague data'' (Pollock 2006:175). To make it more concrete, Pollock then proceeds to provide examples of royal endowments aimed at supporting the reproduction of grammatical knowledge. His intention is to use these to further strengthen his case for royal patronage of grammatical learning. However, as in the previous cases, they turn out to be nothing more than a flow of words which carries away the facts in its force.

In this context, Pollock quotes three inscriptions -- two from Karnataka and one from Tamilnadu.

\section{Inscription of Govinda IV of Rāṣṭrakūṭa dynasty, 929 CE}\label{chap3-sec16}

Pollock makes a mention of land gifted by Govinda the Fourth to two hundred brahmins in Puligere for their study of grammar and (as Pollock puts within brackets) for the study of political theory, literary criticism, history, logic and commentary writing. 

Clearly, this is not an endowment purely for the purposes of grammatical learning.

\section{Inscription of King Bijjaḷa of Kalācūri lineage of Kalyāṇa, 1162 CE}\label{chap3-sec17}\index{inscriptions!Rashtrakuta}

Pollock says that this grant of Bijjaḷa was given to a Kāḷāmukha Śaiva college in Kodhimaṭha, whose syllabus included ``analysis of the {\sl Kaumāra}\index{Kaumara grammar@Kaumāra grammar} [i.e. {\sl Kātantra}\index{Katantra@Kātantra}], {\sl Pāṇinīya, Śākaṭāyanaśabdānuśāsana}, and other grammars'' (Pollock 2006:176). He does not mention what else was included in the syllabus. A look into the original text of the grant reveals that it included all extant streams of learning, including the four Veda-s, {\sl vedāṅga}-s, {\sl darśana}-s such as Nyāya, Vaiśeṣika, Mīmāṃsā, Sāṅkhya, Bauddha etc., along with their commentaries, Yogaśāstra-s such as Lākula-siddhānta and Pātañjala, the eighteen {\sl purāṇa}-s, {\sl dharmaśāstra, kāvya} and {\sl nāṭaka}. (Rice 1902:28 SK.102).

How can this be treated as an illustration for patronage of grammatical learning exclusively?

\section{Inscription of Coḷa king Kulottuṅga the third, 1235 CE}\label{chap3-sec18}

Pollock says that this Coḷa king donated a land of 400 acres for the construction of a ``hall for the analysis of the gift of grammar'' ({\sl Vyākaraṇa-dāna-vyākhyāna-maṇḍapa}), giving the impression that such a vast space was allocated for grammatical learning. But actually, this land was given for maintaining the {\sl Vyākaraṇa-dāna-vyākhyāna-maṇḍapa}, which is a part of the {\sl Tyāgarāja} temple of Tiruvotriyur (Ayyar 1993:61). But it has to be accepted that this is one case where Vyākaraṇa is indeed treated exclusively.

Thus, among the three illustrations provided by Pollock here, only one records an instance of a special privilege accorded to Vyākaraṇa. Despite this, Pollock believes that ``further amassing data would only be redundant'' (Pollock 2006:176). He concludes this section of the chapter saying
\begin{myquote}
``the main point should be clear: that power’s concern with grammar, and to a comparable degree grammar's concern with power, comprised a constitutive feature of the Sanskrit cosmopolitan order.''

\hfill (Pollock 2006:176)
\end{myquote}
The third feature claimed by him, namely ``competitive zeal among rulers everywhere to encourage grammatical creativity and adorn their courts with scholars who could exemplify it'', is elaborated in the second section of the chapter, and it will be taken up for analysis by me in a future article.

\section*{Conclusion}

Pollock himself admits that Indian grammatical studies are held in worldwide esteem purely based on their intellectual depth, and nothing else. References to politics or royalty in works of grammar are extremely rare, even in works which were probably produced through royal patronage. Yet, Pollock tries to build an entirely imaginary relationship between grammar and political power using a convoluted language which conceals much more than it reveals. This article has shown that the evidences that he brings forth are mostly weak, and sometimes totally irrelevant. Based on such unreliable evidences,\index{Pollock!unreliable evidences} he makes sweeping claims to support his grand theories of political philology. 

It is unfortunate that Vyākaraṇa-śāstra, which has been hailed by Bhartṛhari as the ``straight path leading to the Light Sublime'' has been looked at by Pollock entirely through a distorting lens, deceiving himself thereby and taking his readers for a grand ride. 

\begin{thebibliography}{99}
\itemsep=2pt
\bibitem[]{chap3_item1}
Ayyar, P.V. Jagadisa (1993). {\sl South Indian Shrines}. New Delhi: Asian Educational Services. 

\bibitem[]{chap3_item2}
{\sl Bhaṭṭikāvya}. See Trivedi

\bibitem[]{chap3_item3}
Burgess, Jas (Ed.) (1874). {\sl Indian Antiquary} Vol.7. Bombay: The Education Society.

\bibitem[]{chap3_item4}
{\sl Epigraphia Carnatika} See Rice. 

\bibitem[]{chap3_item5}
{\sl Epigraphia Indica} See Hultzsch. 

\bibitem[]{chap3_item6}
Fleet, John Faithfull. {\sl Corpus Inscriptionum Indicarum - Inscriptions of the early Gupta kings and their successors}. Vol. 3. Indological Book House, Varanasi.

\bibitem[]{chap3_item7}
Heller, Ludwig (Ed.) (1900) {\sl Halāyudha's Kavīrahasya in beiden Recensionen}. Greifswald: J.Abel.

\bibitem[]{chap3_item8}
Hultzsch, E. (Ed.) (1905-06) {\sl Epigraphia Indica}. Vol.8. Calcutta: Archeological Survey of India.

\bibitem[]{chap3_item9}
{\sl Indian Antiquary} See Burgess.

\bibitem[]{chap3_item10}
{\sl Kavirahasya}. See Heller. 

\bibitem[]{chap3_item11}
{\sl Laghuśabdenduśekhara}. See Shastri, Nandakishore.

\bibitem[]{chap3_item12}
Majumdar, R.C (1927) {\sl Ancient Indian Colonies in the Far East}. Vol. 1 Champa. Lahore: The Punjab Sanskrit Book Depot. 

\bibitem[]{chap3_item13}
Nariman, J.K (1923) {\sl Literary History of Sanskrit Buddhism}. Ed. 2, Reprint 1992. Delhi: Motilal Banarsidass Publishers Pvt. Ltd. 

\bibitem[]{chap3_item14}
Pollock, Sheldon (2006) {\sl The Language of the Gods in the World of Men}. California: University of California Press. 

\bibitem[]{chap3_item15}
{\sl Prakriyā-sarvasva}. See Sastri.

\bibitem[]{chap3_item16}
{\sl Rāja-taraṅgiṇī}. See Shastri, Pandeya Ramtej.

\bibitem[]{chap3_item17}
Ramesh, K.V (1984) {\sl Inscriptions of the Western Gangas}. New Delhi: Indian Council of Historical Research.

\bibitem[]{chap3_item18}
Rice, Lewis (Ed.) (1902) {\sl Epigraphia Carnatika} Vol. 7 Part 1. Bangalore: Mysore Government Central Press. 

\bibitem[]{chap3_item19}
Sastri, Sambasiva (Ed.) (1951) {\sl The Prakriyāśarvasva of Śrī Nārāyaṇa Bhaṭṭa}. Part 1. Trivandrum: Government Press. 

\bibitem[]{chap3_item20}
Scharfe, Hartmut (1977) {\sl Grammatical Literature}. Wiesbaden: Otto Harrassowitz. 

\bibitem[]{chap3_item21}
Shastri, Nandkishore (Ed.) (1936) {\sl Laghu Shabdendu Shekhara Diacrities of M.M.Nagesh Bhatt}. Benares: Bhargava Pustakalaya. 

\bibitem[]{chap3_item22}
Shastri, Pandeya Ramtej (Ed.) (1960) {\sl Kalhana’s Rajatarangini Diacrities}.\index{Diacrities@\textsl{Diacrities}} Kashi: Pandit Pustakalaya. 

\bibitem[]{chap3_item23}
Trivedi, K.P (Ed.) (1898) {\sl The Bhaṭṭi-kāvya or Rāvaṇa-vadha}. Vol.2. Bombay: Government Central Book Depot. 
\end{thebibliography}

\theendnotes
