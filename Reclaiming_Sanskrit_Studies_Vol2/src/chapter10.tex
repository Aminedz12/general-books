\chapter{Did Indian Intellectual tradition hold that all knowledge of practice is contained in \textit{śāstra}-s?
A Critique of Pollock's 1985 \textit{śāstra}-s paper}\label{chapter10}
\vskip -10pt

\Authorline{Surya K.}
\lhead[\small\thepage\quad Surya K.]{}
\vskip -10pt

\section*{Introduction}

In his 1985 \textit{Śāstra}-s paper (``Theory of Practice and Practice of Theory in Indian Intellectual History''), Pollock \textit{concludes} that pre-modern Indian intellectuals held the belief that all knowledge of practice is contained in \textit{śāstra}-s (``theory of practice'').  Pollock says that this belief crippled innovative spirit of Indian intellectuals and limited their practice to merely uncovering and examining knowledge concealed in \textit{śāstra}-s. Pollock restates this conclusion multiple times in his paper showing the centrality of the conclusion to the paper's thesis. Here are some of the quotes from his paper:
\begin{myquote}
The understanding of the relationship of \textit{śāstra} (``theory'') to \textit{prayoga} (``practical activity'') in Sanskritic culture is shown to be diametrically opposed to that usually found in the West. (p. 499)

Two important implications of this fundamental postulate are that all knowledge is pre-existent ... The eternality of the Veda-s, the \textit{śāstra par excellence}, is one presupposition or justification for this assessment of \textit{śāstra}. (p. 499)

All Indian learning presents itself largely as commentary on the primordial \textit{śāstra}-s.

Logically excluded from epistemological meaningfulness are likewise experience, experiment, invention, discovery, innovation. (p. 515)

If any sort of amelioration is to occur, this can only be in the form of a regress a backward movement aiming at a closer and more faithful approximation to the divine pattern. (p. 515)

From the conception of \textit{an a priori śāstra} it logically follows ... that there can be no conception of progress of the ``forward movement from worse to better'', on the basis of innovations in practice. (p. 515)

All knowledge derives from \textit{śāstra}; success ... is achieved only because the rules governing these practices have percolated down to the practitioners - not because they were discovered independently through the creative power of practical consciousness -``however far removed'' from the practitioners the \textit{śāstra} may be. (p. 507)

We ourselves do not ``create'' knowledge, but merely bring it to manifestation from the (textual) materials in which it lies concealed from us. (p. 517) 
\end{myquote}

This paper critiques four excerpts from Pollock's paper which together argue for his conclusion. Summarized at the end is Pollock's core argument embedded in those excerpts and the rebuttal argued for in this paper.

\section*{I. Pollock's view:  In accordance with the Satkāryavāda theory of causation, Indian intellectuals believed that all knowledge pre-exists in eternal texts. As a consequence, Indian intellectuals voluntarily limited their knowledge acquisition to merely bringing knowledge to manifestation from textual materials}

Sheldon Pollock writes (Pollock 1985:517):
\begin{myquote}
In traditional India, the causal doctrine associated especially with Sāṃkhya and early Vedānta would seem to have particular relevance here ... This is the notion of \textit{Satkāryavāda}: As a pot, for example, must pre-exist in the clay (since otherwise it could never be brought into existence or could be brought into existence from some other material. e.g.  threads), \textbf{so knowledge must pre-exist in something} in order that we may derive it thence (thus in part the postulates of the a priori and finally transcendent \textit{śāstra}): like the clay, which \textit{ex hypothesi} must in some form exist eternally, that from which our knowledge comes must be eternal: and like the potter, \textbf{we ourselves do not ``create'' knowledge, but merely bring it to manifestation from the (textual) materials in which it lies concealed from us}. (Footnote: For a good synopsis of the doctrine of \textit{Satkāryavāda}, see Śaṅkara on \textit{Brahma Sūtra} 2.1.18).'' (Emphasis ours) 
\end{myquote}

\section*{Response}

In this excerpt, Pollock refers to Śaṅkara's Bhāṣya on {\sl Brahma Sūtra} 2.1.18 giving reader the sense that underlined parts are somehow implied either by the {\sl sūtra} or the Bhāṣya.  As we will see, underlined parts cannot be inferred from either the {\sl sūtra} or the Bhāṣya.

Śaṅkara's Bhāṣya elaborates at length on Satkāryavāda theory of causation (Gambhirananda 1965:339-345) according to which material can neither be created nor destroyed; something cannot come from nothing. Therefore, all material pre-exists in some form.  We know from common experience that a pot (an effect) is a manifestation of clay (its cause) and that curd (an effect) is a manifestation of milk (their cause). In addition, there must be some special potency in milk -- but not in clay - to manifest as curd.  Furthermore, neither the potency nor the effect is independent of the cause.  Accordingly, Śaṅkara concludes that the effect is pre-existent in the cause.

It is noteworthy that Śaṅkara does not simply invoke {\sl śruti} texts to make the claim.  Śaṅkara's Bhāṣya grounds its reasoning on everyday common experiences and what appear probable on that basis.  Śaṅkara's approach is common to Indian intellectual tradition of the past.

Indian intellectual tradition requires that {\sl śruti} texts be interpreted according to three conditions (Hiriyanna 1932:180-182):
\begin{enumerate}
\item Revealed truth should be new or extra-empirical ({\sl alaukika}), i.e. otherwise unattained and unattainable.

\item What is revealed should not be contradicted ({\sl a-bādhita}) by any other means of knowing. Interpretation of revealed knowledge should be internally consistent.

\item Reason should foreshadow what revelation teaches. That is, revealed truth must appear probable on the basis of common experiences in the empirical sphere. They serve to remove any `antecedent improbability' that may be felt to exist about the truth in question.
\end{enumerate}

Śaṅkara's Bhāṣya on 2.1.18 clearly shows adherence to these conditions.

In the excerpt, Pollock somehow infers from Satkāryavāda that(all) knowledge pre-exists in {\sl something} and that that something is ``{\sl textual materials}'':
\begin{myquote}
As a pot, for example, must pre-exist in the clay so knowledge must pre-exist in {\bf something}... We ourselves do not ``create'' knowledge, but merely bring it to manifestation from the {\bf (textual) materials} in which it lies concealed from us. 

\hfill (Emphasis ours)
\end{myquote}

Let us briefly examine the {\sl sūtra} and Śaṅkara's Bhāṣya to see if Pollock can find support for his inference.

{\sl Brāhma Sūtra} 2.1.18 (Gambhirananda 1965, 339):
\begin{myquote}
(The preexistence and non-difference of the effect are established) \textbf{from reasoning and another Upaniṣadic text}.
\end{myquote}

\textbf{Could the Indian intellectual infer from the {\sl sūtra} that all knowledge comes ``from reasoning and Upaniṣadic texts''?}

Sankara's Bhāṣya on {\sl sūtra} 2.1.18 validates that the context of {\sl sūtra} is Satkāryavāda.  Gambhirananda clarifies that the phrase ``another Upaniṣadic text'' in the {\sl sūtra} is to be understood to mean ``another passage'' from an Upaniṣadic text. Indeed Śaṅkara's Bhāṣya on {\sl sūtra}-s 2.1.17 and 2.1.18 reference different passages from {\sl Chāndogya Upaniṣad}. Thus, according to Śaṅkara's Bhāṣya, {\sl sūtra} 2.1.18 is saying that ``Pre-existence of effect and non-difference of effect from the cause are established from reasoning and another passage in Chāndogya Upaniṣad.''  There is simply no scope for the Indian intellectual to infer from tradition -Śaṅkara's Bhāṣya - that the {\sl sūtra} is saying that all knowledge comes from reasoning and ``textual'' materials.

\textbf{If Indian intellectuals believe that knowledge is only discovered and not created, does that belief limit their ``creative'' capability?}

History illustratesi that, even in purely creative fields such as art, such a belief did not hinder ``creative'' excellence.  Modern day researchers in science and mathematics commonly believe that they only discover -- and not create - knowledge in their fields.  Modern day engineers and craftsmen only change ``name and form'' of materials and energy they work with using principles derived from knowledge of science and mathematics. There is nothing limiting to an intellectual's practical ``creative'' capability if the intellectual believes that knowledge is only discovered and not created.

But Pollock does not stop there.  He says that, in accordance with Satkāryavāda, Indian intellectuals believed that all knowledge, including that of practice not only pre-exists but pre-exists in something; specifically in textual materials.

\textbf{Does Śaṅkara's Bhāṣya allow Indian intellectuals to infer that all knowledge pre-exists in textual materials?}

In conformance with his Bhāṣya, Śaṅkara would agree that {\sl pāramārthika} knowledge pre-exists in {\sl śruti} texts, but that {\sl laukika} (worldly) knowledge comes from common experience and not from any texts. Śaṅkara's Bhāṣya neither says anywhere nor gives the scope for Indian intellectual to infer that {\sl laukika} knowledge pre-exists in textual materials.  Śaṅkara does not -- and would not - make such a claim simply because he would find the claim reasonable based on common experience. For example, knowledge that milk - and not clay -- manifests as curd follows from common experience and not from any texts. In fact, the claim that all knowledge pre-exists in texts is completely unnecessary for, and even extraneous to, the Satkāryavāda theory of causation. In particular, Satkāryavāda does not require {\sl laukika} knowledge, which is available to one and all through common experiences, to pre-exist in textual form. Thus, there is no credible reason for Pollock to infer that Satkāryavāda led Indian intellectuals to believe that all knowledge pre-exists in textual materials.

\textbf{Thus, there is no justification for Pollock to insert the word ``textual'' into the context.}

While there is no justification, inserting the word ``textual'' serves an important purpose for Pollock: it supports his conclusion that Indian intellectual practically lived by the belief that all knowledge, including that of practice, pre-exists in eternal ``textual materials'', an absurd but foundational assertion in Pollock's 1985 {\sl śāstra}-s paper.

\section*{II. Pollock: Indian intellectual believed that {\sl śāstra}-s of practice are divine in origin; therefore they are perfect. Since {\sl śāstra}-s are perfect, there is no scope for improving them. Therefore, there is no scope for original research}

Pollock writes (Pollock 1985:513, 515):
\begin{myquote}
Our last examples of the individual {\sl śāstra} as deriving from some primordial text are presented by the sciences of architecture, astronomy, and medicine ...

Finally, the most important of the medical texts, the {\sl Caraka-saṃhitā}, claims to be Agniveśa's transcription of the teachings of Ātreya, which were received, through Bhāradvāja, Prajāpati, and the Aśvins, ultimately from Brāhma, while the second major text, the {\sl Susruta-saṃhitā}, similarly begins with a mythological introduction concerning the origin of medicine, and claims that ``Brāhma it was who enunciated this Vedāṅga, this eight-fold Āyurveda.''

...

From the conception of an a priori {\sl śāstra} it logically follows - and Indian intellectual history demonstrates that this conclusion was clearly drawn - that there can be no conception of progress of the forward movement from worse to better, on the basis of innovations in practice. 
\end{myquote}

\section*{Response}

In the above excerpt, Pollock presents {\sl Caraka-saṃhitā} and {\sl Susruta-saṃhitā} as two {\sl śāstra}-s of practice which claim to have divine origin.  With this excerpt, Pollock is persuading the reader that {\sl śāstra}-s of practice claimed divinity to assert their impeccable epistemic credentials. Pollock wants the reader to accept as a consequence that pre-modern Indian intellectuals never upgraded {\sl śāstra}-s of practice and voluntarily limited themselves to merely bringing knowledge to manifestation from the {\sl śāstra}-s.

In this discussion, we see that claims of divine origin did not render {\sl śāstra}-s of practice unquestionable. We will look at an example of how tradition intentionally modified a divine {\sl śāstra} of practice. We will also see that Indian intellectuals contributed to vibrant original research long after {\sl śāstra}-s were established in tradition..

%raghu, 
