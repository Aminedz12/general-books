\chapter{The Science and Nescience of Śāstra}\label{chapter\thechapter:begin}
\vskip -10pt

\Authorline{Sudarshan Therani Nadathur\footnote{pp.~\pageref{chapter\thechapter:begin}--\pageref{chapter\thechapter:end}. In: Kannan, K S (Ed.) (2018) {\sl Śāstra-s Through the Lens of Western Indology - A Response}. Chennai: Infinity Foundation India.}}
\lhead[\small\thepage\quad T.N. Sudarshan]{}

\vskip -10pt

\rhead[]{\small \thechapter. The Science and Nescience of Śāstra\quad \thepage}

\section*{Abstract}

The interpretations of {\sl śāstra} as done by Professor Sheldon Pollock (hereafter Pollock) are critically appraised in this paper, with a firm grounding in the traditional perspectives and vocabularies of the {\sl vidyā}-s (poorly translated by Pollock as theory) - as practiced ({\sl śāstra}) by actual practitioners of the Vedic tradition. The foundational perspectives and motivations that have driven the theorization of the grammars and metaphors of Indian Knowledge Systems, and the practices derived thereof are examined. It is proposed that the current Western theses on {\sl śāstra}\index{sastra@\textsl{śāstra}} derive from a deep ignorance -- a veritable nescience. The primacy, undilutability, and non-negotiable nature of a sacred perspective ({\sl saṃskāra}) whilst interpreting Sanskrit texts on Indic knowledge systems are established. The flawed and incorrect use of philology by Pollock and its overall nebulous nature is explicated. The limitation of the scientific method in interpreting {\sl śāstra} is discussed. The non-empirical, non-verifiable and unscientific nature of the methods used by Pollock to make his claims, are highlighted. 

The aim, purpose and science\index{science} of any {\sl śāstra} is to lead the practitioner on the path to a holistically (nature included) harmonious existence. The scope and role of {\sl śāstra} is beyond that of Science or Religion\index{religion} (as the West currently knows/interprets). Unless this universality of aim is acknowledged and more importantly reinforced and realized by its practice - Western scholarship will continue to provide nebulous and incorrect {\sl etic}\endnote{{{\sl Etic vs Emic}} - The terms were coined in 1954 by linguist Kenneth Pike, who argued that the tools developed for describing linguistic behaviors could be adapted to the description of any human social behavior. {\sl Emic} and {\sl etic} are derived from the linguistic terms {\sl phonemic} and {\sl phonetic} respectively, which are in turn derived from Greek roots. The possibility of a truly objective description was discounted by Pike himself in his original work; he proposed the emic/etic dichotomy in anthropology as a way around philosophic issues about the very nature of objectivity. See Pike(1967). In Malhotra(2016), the description  of traits of  the “insider” and “outsider” vis-a-vis Indology are explicitly stated and is the definition applicable in this paper. Pollock is an outsider.} interpretations of Indic knowledge systems driven by nescience.

\section*{Introduction}

This paper aims to highlight fundamental issues in the method chosen, and the fallacy of the assumptions made, in Pollock's paper entitled ``The Theory of Practice and the Practice of Theory in Indian Intellectual History'' (Pollock 1985) on Indian {\sl śāstra}. Even after thirty years of the publication of the said paper, there has been no rigorous examination of the interpretation/theses made on this paper. That the basis of Western Indology in general and of Pollock's American school of Orientalism/Indology in particular, is ignorant of the methods of Indian ({\sl śāstra}) science (knowledge systems), and is based on a flawed understanding (nescience)\index{nescience} needs to be called out. It is posited here that the entire edifice of the Pollock school is built on questionable claims, is driven by deep underlying institutionalized biases, and is manufactured by insidious methods of philology. 

There have been numerous erudite discussions, deep analyses and critical perspectives post-Said, i.e. after Said \index{Said, Edward} addressing Western scholarship directed toward the East (1979), in the general spirit of inquiry, and in the larger context of the problems of Western Indology. As indicated earlier, we examine in the spirit of the traditional Indian argumentative method of {\sl pūrvapakṣa},\index{purvapaksa@pūrvapakṣa} the work of Pollock, his methods, and his specific theses on Indian {\sl śāstra}.  

In the first two sections, we set the stage for a deeper examination of Pollock (1985): in the first, comparing the Western and traditional Indian constructs and world-views (epistemology),\index{epistemology} we highlight the foundational incompatibilities; in the second, very much in the spirit of {\sl pūrvapakṣa} Western theory ({\sl śāstra}) and Western practice ({\sl prayoga}), we examine with specific focus on Pollock's methods (his {\sl śāstra} and its {\sl prayoga}). In the third section, we `read' and examine ({\sl pūrva-pakṣa}) the specific paper in question. 

Wherever relevant, specific dissonances with Indian perspectives are identified, highlighting foundational issues in method and approach. We conclude with a discussion on the implications of this {\sl pūrva-pakṣa}, and a brief summary on the Pollockian methods; we also identify some possible classes of refutations ({\sl uttara-pakṣa}).\index{uttarapaksa@uttarapakṣa}

\section*{Preliminaries}

Western social sciences freely use and refer to the methods and theories of modern mathematics and science, and follow their evolution in an implicit manner.  By themselves, Western social sciences, unlike the (Western) natural sciences, do not have any foundational scientific construct on which their reasoning can be based. We briefly discuss epistemology in Indian Knowledge Systems, and attempt to describe Western Indology in its theory and practice in terms of Indian categories.
\vskip -10pt

\section*{Traditional Indian Epistemology}


{\sl Tattva}-s - Principles of Reality

For a detailed description of the categories and for a discussion of the salient issues/points and a treatment of the evolution of Indian Epistemology, see Ganeri (2005).\index{Matilal, Bimal Krishna} I quote from the Introduction therein:
\begin{myquote}
Epistemology, , Matilal argues, is the key philosophical discipline in the Indian debate, not metaphysics, a claim that does not preclude the discussion of metaphysical questions, but sees their resolution as lying in an analysis of the structures of knowledge and language.
\hfill (Ganeri 2005:x)
\end{myquote}

Given the variances and distinctness among the different schools and perspectives, there are a set of common presuppositions that can be considered to be the ``basis'', whenever one discusses Indian categories in an overarching manner.

{\sl Darśana}-s often describe their epistemology, based on their unique interpretation of the aforementioned common ``basis''; and almost all of them refer to the categories from the Sāṅkhya\index{Sankhya} school, or base their variations upon it. Both the {\sl āstika}\index{darsana@darśana!astika@āstika} (orthodox) and {\sl nāstika}\index{darsana@darśana!nastika@nāstika} (heterodox) {\sl darśana}-s interpret the basic framework of the Sāṅkhya. Its influence on other schools like Tantra-based Kashmir Śaivism and the various other schools of Śaivism is known.

The well-known Nāsadīya Sūkta\index{Nasadiyasukta@Nāsadīya Sūkta} ({\sl Ṛg Veda} 10.129) is considered as the origin for the later development of the Sāṅkhya as a distinct {\sl darśana}. The oft-quoted verse from the Śānti Parvan of the {\sl Mahābhārata}\index{Mahabharata@Mahābhārata} on the Sāṅkhya
\begin{myquote}
There is no knowledge that is equal to this. The knowledge, which is described in the system of the Sāṅkhya, is regarded as the highest.

\hfill (Mahābhārata 12.302.xx)
\end{myquote}

In the epistemology in the {\sl Bhagavad Gītā},\index{Bhagavadgita@Bhagavadgītā} Lord Kṛṣṇa uses Sāṅkhya as the basis, and uses it to reason about concepts and causalities in and across the categories. The second chapter ({\sl adhyāya}) which summarises the entirety of the text, is sometimes referred to as the Sāṅkhya Yoga or the Yoga of Knowledge. 

It can safely be assumed that the categories of Sāṅkhya can be used as a basis in a broad sense to reason about perspectives based on Indian systems. For the purposes of this discussion, we can probably also refer to such a perspective as a {\sl Dhārmic worldview} as enunciated in the book {\sl Being Different} (Malhotra 2011).

The Sāṅkhya ontology explicitly acknowledges the existence of the {\sl Puruṣa}\index{Purusa@Puruṣa} or a Supreme Consciousness and the manifestation of material existence as the {\sl Prakṛti}.\index{Prakrti@Prakṛti} Such an existence of a ``supreme'' is acknowledged in almost all schools of Indian thought excepting a few categorised as the materialist schools. The Supreme Good is {{\sl mokṣa}\relax}\index{moksa@mokṣa} which consists in the permanent impossibility of the incidence of pain... in the realisation of the Self as Self pure and simple.

See Heera (2011) for a preliminary treatment of Indian materialism esp. the Cārvāka school for an example of a {\sl darśana} which rejects this.

Given the prevalence of the ``sacred'' dimension in almost all schools of Indian thought, and the `sacred' being the recommended mental disposition when `reading' them, it is apt that  we bring attention to a term called Sacred Philology\index{Sacred philology} in an Indian context, coined and explained in Malhotra  (2016:362) as\index{Malhotra, Rajiv} a philology rooted in the conviction that Sanskrit cannot be divorced from its matrix in the Veda-s and other sacred texts, or from an orientation towards the transcendent realm.

\newpage

{\bf {\sl\bfseries Pramāṇa} \index{pramana@pramāṇa} -- ``Means to acquire knowledge''}

In the Stanford Encyclopedia\index{Stanford Encyclopedia} of Philosophy (2015) (SEP) entry on {\sl Epistemology in Classical Indian Philosophy}, (Philips says)
\begin{myquote}
Commonalities in the classical Indian approaches to knowledge and justification frame the arguments and refined positions of the major schools. Central is a focus on occurrent knowledge coupled with a theory of ``mental dispositions'' called {\sl saṁskāra}.
\end{myquote}

We have dealt with the issue of mental disposition in the previous section on {\sl tattva}-s.\index{tattva} Unless there is an acknowledgement of a world-view similar to a Sāṅkhya view, such a {\sl saṁskāra}\index{samskara@saṁskāra} will be, for practical purposes, irrelevant. For purposes of interpreting Indian texts that deal with the primary texts and commentaries of the {\sl vidyā}-s and {\sl śāstra}-s, unless there is a rootedness, an awareness of the sacred, the interpretive exercise and the inferences drawn thereby can be considered to be done from a position of {\sl nescience}. The sacred position could thence be considered {\sl scientific}.
\begin{myquote}
Epistemic evaluation of memory, and indeed of all standing belief, is seen to depend upon the epistemic status of the occurrent cognition or awareness or awarenesses that formed the memory, i.e., {\sl the mental disposition, in the first place}. Occurrent knowledge in turn must have a knowledge source, {\sl pramāṇa}.\hfill (Phillips 2015)\index{Phillips, Stephen}
\end{myquote}

The principal means ({\sl pramāṇa}),\index{pramana@pramāṇa} recognized across most Indian schools of thought are Perception ({\sl Pratyakṣa}),\index{pratyaksa@Pratyakṣa} Inference ({\sl Anumāna}),\index{anumana@anumāna} Analogy ({\sl Upamāna}),\index{upamana@upamāna} Postulation ({\sl Arthāpatti}),\index{arthapatti@arthāpatti} Cognitive Proof ({\sl Anupalabdhi})\index{anupalabdhi} and Testimony ({\sl Śabda}).\index{sabda@śabda} For purposes of the current discussion (evaluating the validity of the claims made by Western Indology), it will suffice to know that most of Western Indology is based on the West's own internal {\sl Śabda} and Western {\sl siddhānta} and use of the (traditional) sources of knowledge in a dissonant manner,  primarily with a non-sacred {\sl saṃ\-skāra}. The {\sl pramāṇa} need to be constrained by the appropriate {\sl saṁskāra}.
\begin{myquote}
Logic is developed in classical India within the traditions of epistemology. Inference is a second knowledge source, a means whereby we can know things not immediately evident through perception. Oetke\index{Oetke, C.} (1996) finds three roots to the earliest concerns with logic in India: {\sl (1) common-sense inference, (2) establishment of doctrines in the frame of scientific treatises (\sl śāstra\relax), and (3) justification of tenets in a debate}. The three of these come together (though the latter two are predominant) within the epistemological traditions in an almost universal regard of inference as a knowledge source. ({\sl italics ours})
\hfill \hbox{(Phillips 2015)}\index{Phillips, Stephen}
\end{myquote}

\medskip
{\bf {\sl\bfseries Mīmāṁsā} - Methods of Investigation}

Though Mīmāṁsā\index{Mimamsa@Mīmāṁsā} is considered to be a {\sl darśana} (school of philosophy) in its own right, it is critical to note that it lays down the rules of interpretation of Sanskrit sentences and the derivation of context. It involves a deep understanding of the techniques of grammar (Vyākaraṇa), prosody (Chandas), and etymology (Nirukta). The Pūrva-mīmāṁsā describes rules of interpretation, causality, {\sl pramāṇa}-s in the context of the {\sl karma-śāstra} of {\sl yajña}-s. This framework has in general been extended to other contexts too such as the {\sl mokṣa-śāstra} of Vedānta.

The range and scope of Mīmāṁsā is immense and there is no such coherent equivalent discipline in the Western system of knowledge. Given this, we note that when it comes to interpretation of Sanskrit texts, unless based on a solid ground of the \hbox{{\sl mīmāṁsā-adhikaraṇa}-s,} any interpretation of Sanskrit text is decidedly incomplete (even if partially correct), incorrect at best, and at worst irrelevant. Pandurangi\index{Pandurangi, K. T.} writes,
\begin{myquote}
The semantics, considering the language, autonomy at word and meaning level and sentence level is an important contribution of Pūrva-mīmāṁ\-sā. The two theories of sentence meaning: Abhihitānvaya-vāda (Kumārila's independent meaning) and Anvitābhidhāna-vāda (Prabhākara's dependent meaning) is another contribution. All systems of Indian philosophy have adopted these two theories with some modifications. There are more than a hundred maxims crystallising the guidelines for interpretation. \hfill \hbox{(Pandurangi 2013:57-58)}
\end{myquote}

Western Indology, with its large body of literature and academic work, is mostly in non-Sanskrit and non-Indian languages and can be summarily dismissed as {\bf irrelevant} if one were to take the orthodox\break Mīmāṁsaka view. `Readings' of Sanskrit are currently being done with `free-style' techniques like Pollockian philology. For such ``readings'' to be acknowledged as valid by any serious traditionalist/Swadeshi/\-emic Indologist, they should at least be justified in the Mīmāṁsā (however rudimentary) interpretive framework. 

\newpage

\section*{Epistemology of Western Indology (according to Indian categories)}

Methods, techniques, assumptions, worldviews of the etic (outsider/\-Western) interpreters do not easily conform to the Indian/Vedic epistemologies. The principles of reality, the sources of knowledge, and the techniques of interpretation do not have much commonality. Most of Western Indology scholarship can be categorised as materialist and\break anthropocentric. They acknowledge only the gross realities of the\break world and their relation to man. The approach is closest to the Indian School of Cārvāka\index{Carvaka@Cārvāka} or Lokāyata.\index{Lokayata@Lokāyata} Categorised interestingly as Cārvāka 2.0 in (Malhotra\index{Malhotra, Rajiv} 2016:90-91), they do not acknowledge the existence of the sacred, and are sophisticated materialists.

In summary, the {\sl tattva}-s and {\sl pramāṇa}-s (see section on scientific nature of {\sl śāstra} for further discussion) of Indian systems do not have parallels in Western methods. The inapplicability of the techniques of Mīmāṁsā and the limited logics of the Western interpretive framework make it difficult to view the repository of Western Indological scholarship in serious light. Furthermore, we discuss the limitations of the ``Western scientific method'' on its own and in the context of Western social sciences in the ensuing section.

\section*{{{\sl\bfseries Pūrva-pakṣa}\relax} of the Western Methods}

As the needs and goals of Western Indology have kept changing, the methods have also evolved and the resulting commentary on India via Indological methods has also been continually changing. American Orientalism,\index{American Orientalism} the latest avatar, has been influenced by the policies of dual-use anthropology, and is highlighted in a recent book entitled {\sl Cold War Anthropology} (Price\index{Price, D. H.} 2016). Much of the topics of study and overall direction of research is guided by American policy needs and requirements of the military-industrial complex and aided by the military-politico-academic nexus. Data and fact collections are made to fit the conclusions and inferences that have already been made so as to conform to existing, or new, policy decisions. There has been no consistent set of axiomatic assumptions or canonical methods which may be (characterised as `Western') in use over the entire period of Western Indology. In the post-Said, postmodernism-influenced Pollock school, the overarching consistency has been in the presence of the ``political'' sensibility in the style of `readings' and the sordid manufacture of `political {\sl literarization}'\index{political literization} from historical text.

\section*{A Brief History of the Western method of `Science'}

According to Western self-hagiographic accounts, humanity owes it to the Greeks for the origins of logical methods. Aristotle's\index{Aristotle} {\sl Organon} is supposed to have influenced the work {\sl Novum Organum} of Francis Bacon. \index{Bacon, Francis} The Baconian method (inductive reasoning via axioms) influenced the development of the scientific method of modern science. See Applebaum\index{Applebaum, W.} (2000) and Applebaum (2005) for a historical account of the methods of modern science. Most modern scientific methods have abandoned the classical methods and logical empiricism, and depend more on hypothesis-formation and its experimental verification (``reproducibility''). The Popperian principle of falsifiability holds sway, wherein every theory is subject to verifiability (Popper\index{Popper, K.R.} 1968): No theory can be proven to be true. It can only proven to be not false - yet.  Another key principle (via Kuhn 1996) is commensurability, wherein, almost always, rival theories are incommensurable. It is not possible to understand one theory in terms of the other leading to relativism and irrationality of theories. To defend this viewpoint and provide a framework to validate theories, Kuhn cited five criteria - accuracy, consistency, broadness, simplicity, fruitfulness - that determine choice of theory. Kuhn states
\begin{myquote}
``When scientists must choose between competing theories, two men fully committed to the same list of criteria for choice may nevertheless reach different conclusions. I am suggesting, of course, that the criteria of choice with which I began function not as rules, which determine choice, but as {\sl values}, which influence it''\hfill (Kuhn\index{Kuhn, T. S.} 1977:324)
\end{myquote}

That Western science is not far off from the influence of the {{\sl saṁskāra}\relax}\index{samskara@saṁskāra} (values/mental disposition) of the scientist is something that is to be internalised. Populist accounts of the supposed rationalism and objectivity of science gloss over this. Given the variance in the methods of scientific inquiry, there are some basic components of method that the community agrees upon (for natural sciences but {\sl not} for social sciences). The four essential elements are observations, hypotheses, predictions and experiments. The repeated cycle of these four elements when subject to peer review comprise the scientific method (in current practice).

In general, the social sciences use much more {\sl subjective} and {\sl fundamentally unverifiable} methods. The misuse and misunderstanding of quantitative analysis/statistics and the lack of statistical rigor in ill-defined ``experiments'' (which only add a veneer of formality) in the social sciences to convince the lay person of its ``scientific'' nature deserves mention.

\section*{The Scientific Nature of {\sl\bfseries śāstra}}

At this juncture, it is essential to understand the deeply scientific nature of {\sl śāstra}. One must closely observe that there is {\sl no dichotomy} of natural-sciences vs social-sciences in the Indian traditional knowledge systems (See Kapoor\index{Kapoor, Kapil} and Singh\index{Singh, A.K.} 2005). Much of the Indian {\sl śāstra}-s have their origins in the experimental verification (first person empiricism) of the methods by the seers ({\sl ṛṣi}-s). The {\sl ṛṣi}-s could achieve higher states of consciousness by various inner methods of {\sl dhyāna} (conferring {\sl divya-dṛṣṭi}) and experience the realities of the {\sl śāstra}-s first-hand. The {\sl śāstra}-s are not the result of arbitrary theorising and hypothesising. Scientific empiricism\index{sastra@\textsl{śāstra}!first-person empiricism} by contrast is predominantly third-person and only examines phenomena and its inherent ``causalities'' as an observer, never as a {\sl subject}.

For the purposes of this paper, it would not be wholly incorrect to state that guesswork driven science (see Feynman\index{Feynman, Richard} 2013) and the first person empirical nature of Indian ``science ({\sl śāstra})'' are incommensurable. On closer examination, one can very well claim that in their genesis, methodology and  evolution, the methods that exemplify  Indian  {\sl śāstra} are {\sl closer} to ``{\sl science}'' in a deeper sense than actual Western science itself - they do not have the intermediate third-party steps of hypothesis and theory-building. The phenomenologies are directly derived from experience (experimental verification) of the actual nature of reality. Though the practice of {\sl śāstra} is traditionally aligned with the overall sacred perspective, it is possible to abstract the inferences and models (as in Yoga, Āyurveda, which can be considered to align with the ``natural sciences''), even if not to their full potential but at least to ``materially'' tractable levels (health/wellness). They can be taught to the actual practitioner who has minimal qualifications or someone who even does not subscribe to the sacred perspective and one can still observe ``visible'' results.

However, such an approach to transfer ``models'' in the context of the social sciences (which deal with the inner states of the individual and thereby collectively of society) fails as it critically depends on ``first person empiricism''. For a novel and probably first-of-a-kind treatment of this perspective, the use of Indian models in applicative social sciences, see Cornelissen\index{Cornelissen, M.} (2013). To illustrate - a sacred {\sl saṁskāra}\index{samskara@saṁskāra} cannot be faked, meditative states cannot be assumed, {\sl cakra} influences cannot be chemically induced etc. There are also other limitations imposed by the (third person empiricism) scientific method, when it comes to the application of abstracted/lifted models of Indian {\sl śāstra} into Western social sciences. The need for an observer, an external reviewer, and also the Western model of peer-review and ``published'' scholarship do not allow for ``primacy'' of individual experience. Western social sciences is mostly a ``social'' and theoretical activity which then supposedly percolates to/affects the individual via the State, unlike the underlying assumptions of the Indian anthropological models, which seamlessly traverse between the individual and society in both directions (see Gurumurthy\index{Gurumurthy, S.} (2014)).

\section*{A Brief History of Western Humanities}

Western humanities\index{Western humanities} is deeply rooted in the manufactured history of the West and the attempts to understand the present in terms of this past.
%\begin{myquote}
%The word ``humanities'' is derived from the Renaissance Latin expression {\sl studia humanitatis}, or ``study of humanitas'' (a classical Latin word meaning--in addition to ``humanity'' -- ``culture, refinement, education'' and, specifically, an ``education befitting a cultivated man'').\hfill (Ref: ``Humanities'')
%\end{myquote}

Humanities is mostly about study of theories and interpretations, and has very little to do with actual realities in terms of practice and experiential knowledge. It was during the post-renaissance period that  humanities became essentially theoretical -- subjects to study rather than of practice. This is seen in the shift of interest of the humanities into Literature and History.
%\begin{myquote}
%A major shift occurred with the Renaissance humanism of the fifteenth century, when the humanities began to be regarded as subjects to {{\sl\bfseries study rather than practice}\relax}, with a corresponding shift away from traditional fields into areas such as literature and history.\hfill (Ref: ``Humanities'')
%\end{myquote}

In his blog, {\sl Think Again} in the New York Times, Stanley Fish delivers a devastating critique on the humanities and its overall relevance to modern society. He attempts to place it in perspective and has much to say about its overall utility in general – Humanities can be justified neither by its benefits to society or even by its effects on the moral fabric (ennobling qualities as he calls it) of the individual. It does not offer recipes for  wisdom nor does it offer an understanding of prejudice. These are, he says, impossible goals for humanities to deliver on. 
%\begin{myquote}
%Any attempt to justify the humanities in terms of outside benefits such as social usefulness (say increased productivity) or in terms of ennobling effects on the individual (such as greater wisdom or diminished prejudice) is ungrounded, and simply places impossible demands on the relevant academic departments.\hfill (Fish\index{Fish, Stanley} 2008)
%\end{myquote}

Fish concludes
\begin{myquote}
To the question ``of what use are the humanities?'', the only honest answer is {\sl none whatsoever}. And it is an answer that brings honor to its subject. Justification, after all, confers value on an activity from a perspective outside its performance. An activity that cannot be justified is an activity that refuses to regard itself as instrumental to some larger good. The humanities are their own good.\hfill (Fish 2008)
\end{myquote}

\section*{Study of Text - Critical Theory}

Scholarship in the Western Humanities use techniques from various schools of Critical Theory to develop arguments and examine a subject in a supposedly unbiased and objective manner (a diluted version of scientific methods that depend on the critic). Critical Theorists including Hegel {\sl rejected} the ``objective'', scientific approach. They sought to frame theories within ideologies of human freedom. Most humanities (and Indology can be seen to be using techniques that originate from both sociology and literary criticism) scholarship can be classified by ``method'' to be using some combination of the methods listed here.

According to Bohman\index{Bohman, James}:
\begin{myquote}
``Critical Theory'' as a ``method'' derives from the works of several generations of German philosophers and social theorists in the Western European Marxist tradition known as the Frankfurt school. It supposedly is ``critical'' as it has specific practical purpose: -- goals like human ``emancipation from slavery'', and its ability to act as a ``liberating influence''.\hfill (Bohman 2015)
\end{myquote}

He further states as definition
\begin{myquote}
A critical theory is adequate only if it meets three criteria: it must be explanatory, practical, and normative, all at the same time. That is, it must explain what is wrong with current social reality, identify the actors to change it, and provide both clear norms for criticism and achievable practical goals for social transformation\hfill (Bohman 2015)
\end{myquote}

\newpage

The hoary aim of the methods of the Critical Theory is to ``emancipate'', in a Western sense, the societies and systems under study. Implicit in the definition is that it is only the ``West'' that gets to comment and critique, based on methods created by the West. In the specific case of Indology - most of the Indologists are {\sl not} practising adherents of Indic lifestyles, and do not adhere to or live by {\sl dhārmic} notions of the world, and do not have world views which originate from these alternative (non-Western and {\sl dhārmic}) epistemologies. Is not such a social inquiry made via these methods poisoned by existing biases?

Bohman adds in the conclusion to his entry
\begin{myquote}
that the evolution of critical methods has made it multi {\sl thoroughly pluralistic} meaning there is no single perspective that dominates the discourse.\hfill (Bohman 2015)
\end{myquote}

Whether these perspectives on actual methods are subscribed to by real-world  academics and departments of Indology is something that could be debated on - but just in case we were in doubt whether a {\sl dhārmic perspective} on Indology would be  not ``allowed'' in a strict sense of ``method'' - it seems, it would be allowed. See Critchley (1992) for detailed discussions on the {{\sl\bfseries ethics}\relax} and {{\sl\bfseries politics}\relax} (Lüdemann 2014) of these Western methods.

The issue of what is considered ``normal'' in the current modern context (circa 2016), the so-called normative is mostly what Malhotra explicitly identifies as {\sl Western Universalism} (Malhotra\index{Malhotra, Rajiv} 2011). Be that as it may, we will still need to examine whether Critical theory with a {\sl dhārmic} perspective is indeed a valid one, if one were to subscribe to the validity of the methods of Western\index{Western Universalism} Critical Theory.

It turns out that it could indeed be a possibility, the periodically updated SEP entry by Bohman has this to say on the ``normative''
\begin{myquote}
Critical Theory offers an approach to distinctly normative issues that cooperates with the social sciences in a nonreductive way. Its domain is inquiry into the normative dimension of social activity, in particular how actors employ their practical knowledge and normative attitudes from complex perspectives in various sorts of contexts. It also must consider social facts as problematic situations {{\sl from the point of view of variously situated agents}\relax}.
\end{myquote}

\newpage

\section*{Other methods (deriving from Marx)}

It is well known that the methods deriving  from the Dialectic view of society and social transformation that are the core ``methods'' of Marxist theoretical toolset. These are used to analyse class/social-relations and every other phenomenon of interest both in contemporary society and also historically (as we see in the case of Indology).

Variations of the Marxist theories influenced by the Russian dialectics of Lenin, and Stalin, Marxist Critical Theory of the Frankfurt school are generally the most widespread versions used as the basis in Western (social science) academia. There exist world-wide variations as in China's Marxism influenced by Mao, the Guevara influenced Latin-American version, North Korean (Juche) not to mention the various regional Indian versions.

Pollock's socio-economic analysis of historical India can be seen to be primarily subscribing to a Marxist-driven framework. Applying the Marxist lenses to the past and also to his (political and liberation) philology, he is able to give novel readings and perspectives of Indian texts that pass for Indology scholarship.

\section*{Postmodernism}
\index{postmodernism}

As a movement which is a reaction to the assumptions and values of the Modern West (16th-20th century), many of its characteristics derive from the denial of the enlightenment values and principles. There is no objective natural reality, there is nothing like the truth, and no faith in science and technology as instruments of human progress; add to this the relativism of logic and reason and, importantly, the view that language is freely interpretable (via deconstruction). 

The entry on the subject in the Encyclopedia Britannica says:
\begin{myquote}
In the 1980s and '90s, academic advocates on behalf of various ethnic, cultural, racial, and religious groups embraced postmodern critiques of contemporary Western society, and postmodernism became the unofficial philosophy of the new movement of ``identity'' politics. 

\hfill \hbox{(Dulgnan 2014)}
\end{myquote}

\newpage

Pollock derives the ``power'' of his philology from the Post-modern views of language. The `power discourse'\index{power discourse} that he attempts to `see' in historical text, is also influenced by these Postmodern perspectives.

\section*{Philology - Method to the mischief}

The overall framework and argumentative methods of Critical Theory (described in brief in the previous section) along with the tools and techniques provided by Philology,\index{Pollock!philology} largely contribute to the bodies of scholarship created by Pollock and others in this new modern school of American Orientalism. Philology is generally considered to be the study of language from historical text sources. Interpretation of text, establishment of authority and authenticity also come under its purview.
%\begin{myquote}
%Philology is the study of language in written historical sources; it is a combination of literary criticism, history, and linguistics. It is more commonly defined as the study of literary texts and written records, the establishment of their authenticity and their original form, and the determination of their meaning.\hfill (See ``Philology'')
%\end{myquote}

With his background of training in the Greek classics, Pollock considers the use of Philology and its methods (according to his definitions) as being critical to understanding the meaning (hidden or otherwise) of texts.

He says 
\begin{myquote}
...philology is, or should be, the discipline of making sense of texts.

\hfill (Pollock 2009:934)
\end{myquote}

He is vehement in that it must have {{\sl nothing to do with meaning or truth}\relax}, as those do not comprise its working definition but one should view it as follows
\begin{myquote}
It is not the theory of language - that's linguistics - or the theory of meaning or truth - that's philosophy - but the theory of textuality as well as the history of textualized meaning.\hfill (Pollock 2009:934)
\end{myquote}

Additionally he opines that
\begin{myquote}
...if mathematics is the language of the book of nature, as Galileo taught, philology is the language of the book of humanity.\hfill (Pollock 2009:934)
\end{myquote}

His ``working'' definition of Philology reaches a climax with this rather breathless conclusion
\begin{myquote}
Thus, both in theory and in practice across time and space, philology merits the same centrality among the disciplines as philosophy or mathematics.\hfill (Pollock 2009:934)
\end{myquote}

This begs the question - when will the West have a Nobel for philology? No prizes for guessing its first recipient.

In Section 3 of this paper, Pollock unravels some of the methods of his analysis
\begin{myquote}
I map out three domains of history, or rather of meaning in history, that are pertinent to philology: {\sl textual meaning, contextual meaning, and the philologist's meaning}. I differentiate the first two by a useful analytical distinction drawn in Sanskrit thought between {\sl pāramarthika sat}\index{paramarthika@pārāmarthika} and {\sl vyāvahārika sat} - \index{vyavaharika@vyāvahārika} ultimate and pragmatic truth, perhaps better translated with Vico's\index{Vico} {\sl verum} and {\sl certum}. ({\sl italics ours})
\hfill (Pollock 2009:950) 
\end{myquote}

On textual meaning,\index{Pollock!textual meaning} he says
\begin{myquote}
People often lie,.... and so do texts. It may not be very fashionable to say so these days, but the lies and truths of texts must remain a prime object of any future philology.....

We should not throw out the baby of textual truth, however, with the bathwater of Orientalism past or present.\hfill (Pollock 2009:951-952)
\end{myquote}

Pollock thus opens the Pandora's box of ``free'' interpretation that can be assigned to any text, to mean anything independent of context - by ascribing it to {\sl some sort of hidden intent} by the author.

On contextual meaning,\index{Pollock!contextual meaning} he says,
\begin{myquote}
Here what has primacy is ``seeing things their way,'' ... that is, the meaning of a text for historical actors.\hfill (Pollock 2009:954)
\end{myquote}

Again, strangely, he introduces the method of adding agents into the frame of text, other than the author, to whom the actual text can supposedly refer to. This gives the ``interpreter'' additional degrees of freedom to build context around any text.

On the third and final meaning, viz the Philologist's meaning,\index{Pollock!philologist’s meaning} Pollock says
\begin{myquote}
The interpretive circle here can be a virtuous one, and we can tack back and forth between prejudgment and text to achieve real historical understanding...

... We somehow assume we can escape our own moment in capturing the moment of historical others, and we elevate the knowledge thereby gained into knowledge that is supposed to be not itself historical, but unconditionally true.\hfill (Pollock 2009:957)
\end{myquote}

\newpage

The act of interpreting a text to make it mean what we want it to mean based on some pre-existing ideology, personal affiliation can somehow be elevated as historical truth.

\vskip .1cm

It gets murkier:
\begin{myquote}
Discovering the meaning of such texts by understanding and interpreting them and discovering how to apply them ... in relation to one's own life, are not separate actions but a single process. And the principle here holds for all interpretation; {\sl applicatio} is not optional but integral to understanding. Historical objects of inquiry, accordingly, do not exist as natural kinds, but, on the contrary, they only emerge as historical objects from our present-day interests.\hfill \hbox{(Pollock 2009:958)}
\end{myquote}

So now, any kind of meaning that the interpreter (Philologist) wants to see or ascribe, based on personal prejudice or understanding of the subject matter becomes possible. Present day context can be reflected back to historical events. Examples include theories such as the Rāmāyaṇa\index{Ramayana@Rāmāyaṇa} being used as a rabble-rousing  political text - then, applying it  to the events in Ayodhya, the conclusions of Deep Orientalism - then linking Sanskrit to the rise of the Nazis\index{Nazi} etc. (See (Pollock 2014), section on {\sl Reading the Sanskrit Tradition} for proof of this method in action).

The most intriguing part, however, is the conclusion:
\begin{myquote}
There is, thus, no inherent contradiction between historical truth and application... It's time we got clear on two things. Historical knowledge does not stand in some sort of fundamental contradiction with truth. Nor does it demand our impartiality; objectivity does not entail neutrality.

\hfill (Pollock 2009:958)
\end{myquote}

So, what this means is that using these methods any sort of contextual interpretation and conclusion can be created. Based purely on application of one's own interest and life experiences, any sort of fabricated interpretations of ``text'' becomes somehow valid. The resulting interpretation can now be called the historical truth!

\vskip .1cm

At this point, it must be asserted, the logic governing Pollock's philology and the scientific method are poles apart. {\sl A serious re-consideration of such fallacious methods\index{Pollock!fallacious methods} by scholars like Sheldon Pollock is necessary}.

\vskip .1cm

That Pollock prefers to read texts, primarily with a political lens, is an observation made in Malhotra (2016:205). In a relatively recent paper (Pollock 2014), Pollock seems to have developed his theories a bit more, and couched them in new vocabulary. He calls it the three dimensions of Philology.\index{Pollock!three-dimensional philology}  His methods and intent are well camouflaged this time around too, but he does rather bravely reveal his motive in the abstract.
\begin{myquote}
Enacting philology in three dimensions requires a delicate balance -- essential if we are to cultivate the important {{\sl political--ethical}\relax} values that are only possible by learning to read well.\hfill (Pollock 2014:398)
\end{myquote}

In the conclusion, he aptly summarizes the absolutely ``free-style'' intentional nature of his philology
\begin{myquote}
At the same time, and more positively now, philology on Plane 1 (historicism) helps us to better comprehend the nature, or natures, of human existence and the radical differences it has shown over time, that is, the vast variety of ways of being human. Philology on Plane 2 (traditionism) helps us to better understand and to develop patience for the views of others, and so to expand the possibilities of human solidarity ({\sl this is the great value of reading a deep and distant past like India's, since it is precisely the presence of a long and very unfamiliar history of reading and interpretation that lets us exercise so effectively the virtues of the quest for understanding and solidarity}). Philology on Plane 3 (presentism) helps us to come to understand our own historicity and our relationship to all earlier historical interpretations, including the originary, and thereby to gain a new humility for the limits of our capacity to know and a new respect for the importance to keep trying. It may well be there are other intellectual practices that can teach us these lessons both negative and positive, but {\sl I know of none that can do so as consistently and immediately as reading well through the discipline of philology}. ({\sl italics ours})

\hfill (Pollock 2014:411) 
\end{myquote}

These methods (diagnostic political philology\index{Pollock!political philology} and prescriptive liberation philology)\index{Pollock!liberation philology} of Pollock are going global and are being used rather ``freely'' to interpret in what could only be called as an {\sl anything-goes} style. For more on the growing global footprint of these rather questionable methods, see Pollock {\sl et al.} (2015). For a discussion of similar critical methods used in the earlier German school refer to Adluri and Bagchee (2013).

\section*{{\sl\bfseries Pūrva-pakṣa}}

In the previous section, we briefly examined some of the methods in Western Indology, specifically those of Pollock.  We now examine his methods and work using the lenses of Western methods on the one hand and those of traditional Indian perspectives on the other. We use this new understanding to throw light on Pollock's views on Indian {\sl śāstra} (Pollock 1985).

The aforementioned paper is presented in three distinct sections besides the Introduction in which Pollock intentionally misuses a quote by Naipaul,\index{Naipaul, V.S.} see the blogpost ``Pūrvapaksha of Sheldon Pollock'' (2016). Pollock begins to frame the argument for {\sl śāstra} being theory by referring to bodies of rules and codes in the {\sl dharma-śāstra}-s and {\sl Manu-smṛti}.\index{Manusmrti@Manusmṛti} He creates an initial bias in the reader by citing two examples from texts which were meant to be a collection of rules by their very design.

The next three sections mirror the appropriately titled sections in Pollock's paper. By his own admission, it is only the first section that is dealt with in detail wherein he makes some logically valid claims and the remaining two are dealt with cursorily and speculatively.

\section*{1. The Relationship of {{\sl\bfseries Śāstra}\relax} to its Object}

We are informed further by Patañjali that  ``{\sl śāstra} is that from which there derives regulation [definite constraints on usage]" ({\sl śāstrato hi nāma vyavasthā}). 'Outside the grammatical tradition, the term embraced more broadly the notion of "system of ideas," (Pollock 1985:501)

To establish that {\sl śāstra} is a set of rules, Pāṇini and Patañjali\index{Patanjali@Patañjali} are quoted selectively - {\sl śāstrārtha-sampratyaya} (intention of a rule), {\sl śāstrato hi nāma vyavasthā} (usage constraints) respectively (Pollock 1985:501). That their domain is {\sl vyākaraṇa} or grammar, which is an encapsulation of language understanding in terms of ({\sl sūtra}-s) rules is conveniently ignored: What does one have in grammar if not rules? 

The first formal definition of {\sl śāstra} as theory is supposedly by the Mīmāmsāka viz. Kumārila-bhaṭṭa\index{Kumarila Bhatta@Kumārila Bhaṭṭa}

\newpage

\begin{myquote}
%``Śāstra'', we are told by the great eighth-century Mīmāṁsāka Kumārila Bhaṭṭa, ``is that which teaches people what they should and should not do. It does this by means of eternal [words] or those made [by men]. Descriptions of the nature [of things, states] can be embraced by the term {\sl śāstra}, insofar as they are elements subordinate [to injunctions to action].''
{{\sl Śāstra}} is thus, according to the standard definition, a verbal codification of rules, whether of divine or human provenance, for the positive and negative regulation of some given human practices.\hfill (Pollock 1985:501)
\end{myquote}

Pollock very correctly identifies the Mīmāṁsā\index{Mimamsa@Mīmāṁsā} interpretation of {\sl śāstra vis-à-vis śruti} - the Veda and also of the relationship to the {\sl upaveda}-s, {\sl vedāṅga}-s\index{Vedanga@Vedāṅga} and  {\sl vidyā-sthāna}-s. He exhaustively lists multiple catalogues of {\sl śāstra}\index{sastra@\textsl{śāstra}} and their evolution.

That {\sl śāstra} is the original source of knowledge is acknowledged but then he digresses on some supposed {\sl bivalency} between rules and revelation. It appears that this is due to Pollock's misunderstanding of revelation of {\sl śruti} being similar to revelation akin to Abrahamic books.

After this point, inexplicably, Pollock assumes that {\sl śāstra} means {\sl theory}, which is incorrect. The {\sl śāstra}-s have been derived and codified based on actual experience, and are encoded experiences driven by practice, and not some arbitrary theory (like scientific theories) based on guesswork, the principles of falsifiability and third person empiricism.

We need to internalize what Aurobindo\index{Aurobindo, Sri} has to say about this :
\begin{myquote}
{{\sl Śāstra}} is not some blind uncritical following of customs, (however good or bad). It essentially is the knowledge and teaching laid down by millennia of intuition, experience and wisdom, some of the best standards available to humanity.\hfill (Aurobindo 1997:229)
\end{myquote}

Also, Gandhi\index{Gandhi, M. K.} in his lectures on the Gītā,\index{Bhagavadgita} explicitly makes it clear, {\sl śāstra} is very much contextual, and deeply influenced by its practice
\begin{myquote}
He who forsakes the rule of {\sl śāstra} and does but the bidding of his selfish desires, gains neither perfection, nor happiness, nor the highest state.

{\sl ...Śāstra} does not mean the rites and formulae laid down by so called dharma shastra, but {\sl the path of self-restraint laid down by the seers and the saints.}

...Therefore let {\sl Sastra} be thy authority for determining what ought to be done and what ought not to be done; {{\sl ascertain}\relax} {\sl thou the role of the} {{\sl Shastra}\relax} {\sl and do thy task here} (accordingly).
\hfill ({\sl italics ours}) (Sahadeo 2012:115)
\end{myquote}

Pollock then makes a rather dubious claim that the Indian intellectual history somehow assumes a `finite set of topics of knowledge'. No such claims were made by Rājaśekhara\index{Rajasekhara@Rājaśekhara} whom he quotes as an authority (cataloging the types of {\sl śāstra}).

\newpage

Now that he assumes that {\sl śāstra} means theory, albeit without properly establishing it, Pollock then addresses the relationship between theory and its practice ({\sl prayoga}). He poses the question thus 
\begin{myquote}
If {\sl śāstra} is the systematic exposition of some knowledge, what does the Indian intellectual tradition conceive to be the relationship of this exposition to the actual enactment of the knowledge? How, that is, are theory ({\sl śāstra}) and practice ({\sl prayoga}) viewed as interrelated'? What is the causal -- or more grandly, ontological -- relation that is thought to subsist between the two?\hfill (Pollock 1985:504)
\end{myquote}

He then goes on to examine specific cases and cites selective examples\index{Pollock!selective examples} from Vyākaraṇa,\index{Vyakarana@Vyākaraṇa} Dharmaśāstra,\index{Dharmasastra@Dharmaśāstra} and Kāmasūtra\index{Kamasutra@Kāmasūtra} which supposedly exhort the practitioner to abide strictly by injunction and rules. He implies that any sort of failure experienced by the practitioner is exclusively ``his'' and not of the {\sl śāstra}.

He then proceeds to make cryptic statements such as 
\begin{myquote}
All knowledge derives from {\sl śāstra}; success in astrology or in the training of horses and elephants, no less than in language use and social intercourse, is achieved only because the rules governing these practices have percolated down to the practitioners - not because they were discovered independently through the creative power of practical consciousness - ``however far removed'' from the practitioners the {\sl śāstra} may be.\hfill (Pollock 1985:507)
\end{myquote}

The point being made is not clear at all - Does Pollock expect every individual to write his/her own {\sl śāstra}? Every civilisation has rules which encode its cultural goals and orients its practitioners to them. In the {\sl dharmic} way of living, human progress via pursuit of {\sl puruṣārtha}-s (the goals of existence) toward possibly a final {\sl Mokṣa} (liberation) is the overarching goal of all {\sl śāstra}-s. The {\sl kārmic-dhārmic} worldview which includes causal chains across multiple births is fundamental to Vedic cosmology\index{Vedic cosmology} without which, any reading of any {\sl śāstra} will be flawed. The deliberate sidelining of this perspective - which is possibly beyond the scope of Pollock's gaze - seems to be one of the principal causes of the nescience.

He then proceeds to cite Rāmānuja\index{Ramanujacarya@Rāmānujacārya} from the {\sl Śrī Bhāṣya} which is a commentary on the {\sl Brahma Sūtra}-s\index{Brahmasutras@Brahmasūtras}, a {\sl mokṣa śāstra}, where such emphasis is being made.
\begin{myquote}
``{{\sl Śāstra}} is so called because it instructs; instruction leads to action, and {\sl śāstra} has this capacity to lead to action by reason of its producing knowledge.'' The actual program of spiritual liberation enacts this postulate, since for Rāmānuja, {\sl śāstra} forms the sole means for attaining {\sl mokṣa}.\hfill (Pollock 1985:509)
\end{myquote}

The proverbial elephant in the room is staring right at him - So how does Pollock react? - he does not. He moves on, instead, to cite the rules from Kauṭilya's {\sl Artha-śāstra}.

To juxtapose this new construct of the traditional Indian way of theory over practice with Western thinking, Pollock quotes Ryle and Aristotle, but an actual full reading of the reference implies something else.

He quotes Ryle saying,
\begin{myquote}
But most people today I think would readily accept the commonsense assessment of Ryle, that ``efficient practice precedes the theory of it''.

\hfill (Pollock 1985:510)
\end{myquote}

The same reference by Ryle also says this,
\begin{myquote}
The common-sense position has not, however, gone unchallenged. Suggestive is Popper's epistemological conjecture that theories or expectations, logically speaking, must predetermine experience; that our dispositions and in fact senses are ``{\sl theory-impregnated}''. ({\sl italics ours})

\hfill (Pollock 1985:511)
\end{myquote}

To sum up, Ryle says any practice is a {\sl priori} influenced by theory.

\vskip .1cm

The fact that theory and practice are a feedback-driven continuum over multiple generations in dhārmic traditions, encoding newer experiences into {\sl śāstra}-s which are rewritten according to the times, is something that Pollock has unfortunately ignored. A similar sentiment is echoed by Ryle.
\begin{myquote}
...Still others wonder whether the dichotomy between theory and practice may not itself be more theoretical than practical... Is it not more intuitive, however, to think that theory evolves out of practice and will itself evolve as practice refines and modifies itself'?\hfill (Pollock 1985:511)
\end{myquote}

Now that he realizes that he has possibly argued himself into a corner he concludes the first section by suddenly switching over to {\sl dharma} -\index{dharma}
\begin{myquote}
To simplify a complex argument, we may say that {\sl dharma} in the largest sense connotes the correct way of doing anything. From the Mīmāṁsā\index{Mimamsa@Mīmāṁsā} perspective, the prevailing one from which the rest of shastric discourse is extrapolated, {\sl dharma} is by definition ``rule-boundedness'' ({\sl codanā-lakṣaṇa}), and the rules themselves are encoded in {\sl śāstra} ({\sl upadeśa}).

\hfill (Pollock 1985:511)
\end{myquote}

He seems to have made a deep inference that Indians follow {\sl dharma} $\to$ {\sl Dharma} is a function of rule-boundedness $\to$ Rules are encoded in {\sl śāstra}. So - what happened to the claim of an underlying operational {\sl theory} ?

This is followed by what could only be called an act of pure reasoning, a hail-mary\endnote{A Hail Mary Pass is a very long forward pass in American football, made in desperation with only a small chance of success.} pass to Kant\index{Kant, Immanuel}
\begin{myquote}
But rules, as we have known since Kant, are either constitutive or regulative, (the rules of chess and those of dinner table etiquette would be respective examples). Shastric discourse collapses the two, enunciating both in the same injunctive mood.\hfill (Pollock 1985:511)
\end{myquote}

By this sleight of hand, Pollock with some philological magic and invocation of Kant has transformed {\sl śāstra}-s to become objects of political control. So now {\sl śāstra}, somehow because of their supposed regulative nature, make it possible for human social and sexual intercourse to {\sl become amenable to codified legislative control}.

The objective of this section of Pollock's paper was to lay out the relationship of {\sl śāstra} to its object - Pollock has argued for interpreting {\sl śāstra} with a political reading and successfully places it in what he calls the larger discourse of power.

Although Pollock has already mentioned that the remaining two sections are dealt with cursorily and speculatively, we shall continue to examine the arguments made.

\section*{2. The Implications of the Priority of Theory}

Having somehow ``established'' the fact that {\sl śāstra} is theory bereft of actual practice (via Pollockian logic), The intent of Pollock in the second section is to show the effects of prioritizing theory over practice. (Ref. Pollock 1985:512-516)

{\sl Śāstra}-s ascribe their own origins to seers and {\sl ṛṣi}-s of the past, individuals with a higher consciousness, those who could access and experience and verify by practice such knowledge; and the {\sl śāstra}-s are now categorised as {\sl myth}!

\newpage

Myth in the Judeo-Christian (and postmodern) discourse refers to any text/work, that does not have official sanction and that which is not considered true or factual by some central authority or history - controlling ``religious'' body ex: Church, Rabbinic Council etc. Non-Abrahamic traditions (especially dhārmic ones) are not ``centrally'' controlled nor are they history-centric nor do they have centralised institutions. By labeling Indian {\sl śāstra} as {\sl myths}, the credibility of these ``so-labelled'' works are automatically questioned. For a more comprehensive discussion of this label -  ``myth'' - and for more on the current normative on the ``scientific'' view of myth in Western scholarship see Jung\index{Jung, Carl} and Kerenyi (1963).

The modus is generally something like this --
\begin{itemize}
\itemsep=1pt
\item[{\sl(1)}] {\sl A people P (non-West) refer to X (a body of knowledge unknown to the West) as a guideline/reference}

\item[{\sl(2)}] {\sl Call X  a myth} 

\item[{\sl(3)}] {\sl Imply that X is probably untrue}

\item[{\sl(4)}] {\sl People P who refer to X are primitive and regressive}

\item[{\sl(5)}] {\sl The West (which does not use/refer to X) is superior to P}

\item[{\sl(6)}] {\sl In case there is an inkling of something good/monetizable, appropriate X, rename X, call X a Western invention} 
\end{itemize}

(The last step is a phenomenon called digestion and is part of a larger phenomenon of a cultural U-turn.\index{U-turn} See Malhotra (2011) for definitions and discussion).

This method has been repeatedly used in Indological studies and is also seen as a general phenomenon in the appropriation of traditional knowledge and intellectual property of non-Western civilisations into the larger capitalist Western universal narrative called Science. This is nothing new in the history of the encounter of the West with the non-West. See Raju (2009) for a discussion. The deconstructive reader should be aware that such a mental disposition is in most cases the primary {\sl saṁskāra} of the Western academic/intellectual.

The {\sl Purāṇa}-s encapsulate enormous amount of secular knowledge, which Pollock acknowledges.

\newpage

On the {\sl Agni Purāṇa},\index{Agni Purana} he says
\begin{myquote}
What Agni goes on to reveal is an encyclopedic synthesis of human knowledge, including what is in fact a vast array of discrete {\sl śāstra}-s on topics as diverse as {\sl dharma}, architecture and iconology, astronomy, divination, the lapidary's art, the science of weapons, arboriculture, veterinary medicine, metrics, phonetics, grammar, and rhetoric.

\hfill (Pollock 1985:514)
\end{myquote}

\vskip .1cm

Using  selective quoting,\index{Pollock!selective quoting} the {\sl Mahābhārata}, {\sl Nāradasmṛti},  the {\sl dharma}\-\break\hbox{śāstra} texts,  {\sl Kāvyamīmāṁsā} (poetics), {\sl Kāmasūtra},  {\sl Matsyapurāṇa},  {\sl Caraka\-saṁhitā} (Āyurveda/medicine),  {\sl Suśrutasaṁhitā} (surgery/\-medicine),\break  {\sl Agni\-purāṇa}, {\sl Bharata-Nāṭyaśāstra} (dance, performing art), the {\sl Dhanurveda} texts (archery), {\sl Pākadarpaṇa} (cooking) -- have all been labelled as myth.

\vskip .1cm
\begin{myquote}
Numerous individual \textsl{śāstras} adopt this mythological self-understanding and represent themselves as the outcome of a similar process of abridgement.
\hfill	(Pollock 1985:512) 
\smallskip

Such consequences of the priority of {\sl śāstra} are I think clearly expressed in the mythic crystallizations of the postulate of shastric priority, namely the accounts of their origins the different {\sl śāstras} contrive for themselves. It would be possible to trace historically the growth of this self-representation. In {\sl dharmaśāstra}, for example, such a mythic orientation appears to set in after the composition of the major {\sl sūtras} ({\sl Āpastamba} and {\sl Gautama}).\hfill ({\sl italics ours}) (Pollock 1985:512) 

\smallskip

There are several {\sl conclusions to be drawn from such mythic representations of the origins of knowledge}, which I think comprise considerably more than simple ``literary transposition[s] of speculations on the Golden Age.'' {\sl First, the ``creation''\index{knowledge!creation} of knowledge is presented as an exclusively divine activity, and occupies a structural cosmological position suggestive of the creation of the material universe as a whole}. Knowledge, moreover - and again, this is knowledge of every variety, from the transcendent sort ``whose purposes are uncognizable'' [{\sl adṛṣṭārtha}-] to that of social relations, music, medicine (and evidently even historical knowledge) -- is by and large viewed as permanently fixed in its dimensions; knowledge, along with the practices that depend on it, does not change or grow, but is frozen for all time in a given set of texts that are continually made available to human beings in whole or in part during the ever repeated cycles of cosmic creation. \hfill (Pollock 1985:515)
\end{myquote}

\newpage

The conclusions he draws are not surprising:
\begin{myquote}
A final consequence is one I suggested earlier. From the conception of an {\sl a priori śāstra} it logically follows - and Indian intellectual history demonstrates that this conclusion was clearly drawn - that there can be no conception of progress, of the forward ``movement from worse to better,'' on the basis of innovations in practice. 
\hfill \hbox{(Pollock 1985:515)}
\end{myquote}

The West provides a big contrast, he says: 
\begin{myquote}
Undoubtedly the idea of progress in the West germinated in a soil made fertile by a peculiar constellation of representations, about time, history, and eschatology. Whatever may be the possibility of the idea's growth in the absence of these concepts, it is clear that in traditional India there were at all events ideological hindrances in its way.	
\hfill \hbox{(Pollock 1985:515)}
\end{myquote}

Expanding the scope of the {\sl pūrva-pakṣa} to address these claims, pointers to alternate viewpoints are presented and the reader is encouraged to examine the claims of Western growth and superiority of Western lifestyles. 

Driving the final set of nails into the proverbial coffin, Pollock posits that any expectation of progress using Indian methods is not possible, because
\begin{myquote}
If any sort of amelioration is to occur, this can only be in the form of a ``regress,'' a backward movement aiming at a closer and more faithful approximation to the divine pattern.\hfill (Pollock 1985:515)
\end{myquote}

His attacks on the `{\sl sanātana}' nature of Indian knowledge (timeless truths that are valid for all time) reach a climax in this section:
\begin{myquote}
First, all contradiction between the model of cultural knowledge and actual cultural change is thereby at once transmuted and denied; creation is really re-creation, as the future is, in a sense, the past. Second, the living, social, historical, contingent tradition is naturalized, becoming as much a part of the order of things as the laws of nature themselves. Just as the social, historical phenomenon of language is viewed by Mīmāṁsā\index{Mimamsa@Mīmāṁsā} as natural and eternal, so the social dimension and historicality of all cultural practices are eliminated in the shastric paradigm. And finally, through such denial of contradiction and reification of tradition, the sectional interests of pre-modern India are universalized and valorized. The theoretical discourse of {\sl śāstra} becomes in essence a {\sl practical discourse of power}.\index{sastra@\textsl{śāstra}!discourse of power}
\hfill \hbox{(Pollock 1985:516)}
\end{myquote}

\section*{Discussion}

Pollock's denial of Indian cosmological precepts and the lack of any discussion of the background basis or {\sl saṁskāra} (in a Mīmāṁsā sense\endnote{See discussion on “mental disposition”. (Stephen 2015)}) is apparent here. Pollock's {\sl saṁskāra} denies the existence of such a cosmology and sidesteps the issue either out of arrogance or ignorance.

Denying this sacred cosmology, or viewing humans as fallen (sinners) or the nihilism of Western liberalism/secularism -- these are not the {\sl śāstraic} way, which conceives of humanity as divine in essence. There have been materialist viewpoints and lifestyles in India, and they have had their following. What is important is what lasts and what is preserved by practice. Despite all its material poverty in recent times (forced by centuries of slavery and colonialism) modern India is a living testimony to the {\sl sanātanic} nature of Indian living. The long timescales\endnote{There is no finality of actual numbers arrived at by various modern methods - genetics, archaeo-astronomy,  radiocarbon-dating etc. (It is to be noted that traditional dates, measures of time, calendars have spectacular time-scales beyond the conception of modern science). We can safely say that it is at least more than 10 millennia old.} of a stable Indic civilisation have led to such a sense of {\sl sanātana} and the observance of existence of imperishable truths.

This has been something Western intellectualism finds difficult to internalise. Pursuit of lifestyles that conform to a Vedic cosmology (=beliefs in {\sl karman, dharma, punar-janman} (rebirth) etc.) require stable societies: stability in time, place for humans to evolve spiritually, become aware of higher planes of consciousness, and practice life (as per {\sl śāstraic} guidelines ) so that the traversal across the \hbox{{\sl puruṣārtha}-s}\index{purusarthas@puruṣārthas} (goals and aims of life) is as it ought to be, leading to  possible transcendence in the current life and possibly lead to a final {\sl mokṣa}  (liberation / escape from the cycles of rebirth and the attainment of a higher consciousness).

In terms of {\sl guṇa}\index{guna@guṇa} (behaviors) such societies had to inculcate and encourage a {\sl sāttvic} orientation (purity) to create the proper {\sl saṁskāra} - unlike the mostly {\sl āsuric} (demonic) materialist/egoistic motivations of anthropocentric societies. For a much deeper elaboration of this perspective - See {\sl Deva} and {\sl Asura} in Aurobindo\index{Aurobindo, Sri} (1997:475) -- where Aurobindo describes the essential nature of the structure of {\sl śāstra} as being built upon a number of preparatory conditions or {\sl dharma}-s - it is a means to an end (that of freedom and the discovery of the divine nature in man) – not an end in itself.
%\begin{myquote}
%For the {\sl śāstra} in its ordinary aspect is not that spiritual law, although at its loftiest point, when it becomes a science and art of spiritual living, {\sl Adhyatma-Shastra}, -- the Gita itself describes its own teaching as the highest and most secret {\sl śāstra}, -- {{\sl\bfseries it formulates a rule of the self-transcendence of the sattvic nature}\relax} and develops the discipline which leads to spiritual transmutation. Yet all {\sl śāstra} is built on a number of preparatory conditions, dharmas; it is a means, not an end. The supreme end is the freedom of the spirit when abandoning all dharmas the soul turns to God for its sole law of action, acts straight from the divine will and lives in the freedom of the divine nature, not in the Law, but in the Spirit. (Emphasis mine)
%\end{myquote}

The land we currently know as India has been a place on this planet which has allowed this to transpire for a long, long time. India has borne the brunt of the deep Western desire to ``conquer'' - militarily and politically during the colonial era, as is borne out by W. Durant:
\begin{myquote}
The more I read the more I was filled with astonishment and indignation at the apparently conscious and deliberate bleeding of India by England throughout a hundred and fifty years. I began to feel that I had come upon the greatest crime in all of history.\hfill  (see Durant(1930) :x)
\end{myquote}
The same is being done intellectually through the weapon of Western Indology today. The West has the tendency to frame every encounter with the `other', as some sort of `clash' of civilizations, so that the discourse forever remains a `{\sl discourse of power}'. It is time that the deeper reality is acknowledged : {\sl it is a clash between the nescience of the West and the science of the} {{\sl śāstra}\relax} - a clash of evil {\sl saṁskāra} (mental disposition) and {\sl jñāna} (knowledge).

\section*{3. The Critical Presupposition:}

\subsection*{The transcendence of {{\sl\bfseries Śāstra}\relax}}
\index{sastra@\textsl{śāstra}!transcendence of}

In the final section of his paper, Pollock makes serious charges against the very nature of the Veda, its primacy and its primordial origins and thereby of all {\sl śāstra}-s. He quotes {\sl Caraka-saṁhitā}\index{Caraka-Samhita@Caraka-saṁhitā} - 
\begin{myquote}
``Apart from the restricted sense of acquiring this knowledge and of spreading it, there is no meaning in saying that medical science came into being having been non-existent before''.\hfill (Pollock 1985:517)
\end{myquote}

Pollock posits that there is a serious  epistemological issue, and that this leads to a greater problem of causality. He says:

\begin{myquote}
This is the notion of {\sl satkārya-vāda}:\index{Satkaryavada@Satkāryavāda} As a pot, for example, must pre-exist in the clay (since otherwise it could never be brought into existence or could be brought into existence from some other material, e.g., threads), so knowledge must pre-exist in something in order that we may derive it thence (thus in part the postulates of the {\sl a priori} and finally transcendent {\sl śāstra}); like the clay, which {\sl ex hypothesi} must in some form exist eternally, that from which our knowledge comes must be eternal; and like the potter, we ourselves do not ``create'' knowledge, but merely bring it to manifestation from the (textual) materials in which it lies concealed from us. 
\hfill	(Pollock 1985:517)
\end{myquote}

He uses the Sāṅkhya notion of {\sl satkārya-vāda}  and extends it to all philosophical literature. He uses this to claim that all the efforts of traditional scholarship have been only to rediscover what has previously existed. This sense of mere `rediscovery' (read ``lack of innovation'') is also attributed to the {\sl śāstra}-s.

To make the claim that even in this presupposition, theory precedes practice, Śaṅkara's {\sl Bhāṣya} on the {\sl Brahma Sūtra}-s\index{Brahmasutra} is quoted selectively and out of context.\index{Pollock!quoting out of context} Pollock now claims 
\begin{myquote}
In his commentary on the {\sl Brahmasūtras}, where several of the passages cited above are marshaled, {\sl Śaṅkara rather clearly makes the connections} that relate such speculation on the cosmogonic Logos {\sl to the problems of śāstra} raised in this paper.
\hfill	({\sl italics ours}) (Pollock 1985:519)
\end{myquote}

This is a very serious claim, that requires a detailed rejoinder, which is beyond the scope of the current discussion. 

As this is the presupposition, it implicates {\sl śāstra} as a whole and lends support to the arguments and the inferences of the earlier two sections. Indian intellectual history has shown no ``newness'', because by its very nature, its dependence on theory ({\sl śāstra}), depends on transcendence of Vedas.  Since the Vedas and {\sl śāstra}-s are infallible and not of recent origin, Indian intellectual history has only been rediscovering what has been stated in the {\sl Veda}-s, there has been no innovation in Indian intellectual history and by implication Indian civilisation!

\section*{The Implications of {{\sl\bfseries Pūrva-pakṣa}\relax} (Summary of Pollockian methods and scholarship)}

On a deep reading of Pollock's\index{Pollock!foundational claims} other works, we find that he makes various foundational claims similar to the ones in this paper on {\sl śāstra} and practice of theory, viz. in (Pollock 2004) on {\sl Dharma} and {\sl Mīmāṁsā}, (Pollock 1989) on {\sl Mīmāṁsā} and the problem of history -- where we see similar methods and techniques, and a need  to frame interpretations into a ``discourse of power''.

The degrees of freedom in his philological style (the method to his mischief), gives the ability to ascribe any motive to the writer of a historical text, freely place other agents in the frame of text - preferably to illustrate victimhood, and then the third plane is whatever else Pollock wants to add as the philologist's meaning - context/meanings that he wants to project backwards onto historical text (Nazism, events of Ayodhya etc.). This free-style context building, incoherent and incomplete chains of reasoning, a marked lack of {\sl sāttvic saṁskāra} - are hallmarks of this style. 

In summary, this {\sl pūrvapakṣa}\index{purvapaksa@pūrvapakṣa} is a non-trivial examination of the methods used by Pollock in manufacturing his theses\endnote{Pollock’s claims summarised (from Conference - Call for Papers) The relationship between {\sl śāstra} and {\sl prayoga} (theory and practical activity) is one which is diametrically opposed to what it is in the West. 

{\leftskip=10pt \rightskip=10pt
In the West there is progress because new experience and practical considerations inform the thinkers who can change and develop new thought based on such empirical evidence. On the other hand, the Veda-s are deemed as {\sl śāstra} par excellence, and as already containing all the knowledge. The Veda-s are thus opposed to all progress. {\sl śāstra}-s are frozen in time; hence they hinder creativity, and are inherently regressive. Added to this, {\sl śāstra}-s engender authoritarianism and inspire social oppression. In contrast, Western civilisation is based on freedom. As a result, {\sl śāstra}-s are to be seen as a major cause - of Indian lack of creativity and freedom, and for the existence of oppression.
\par}} on Indian {\sl śāstra}. The conclusions drawn are -
\begin{enumerate}
\itemsep=1pt
\item Pollock's approach is not based on the methods of science (the scientific method).

\item The hypotheses and claims he makes are not based on established methods of Western social sciences, even which we claim as being questionable in their rigor. 

\item The claims do not have any empirical (factual) grounding.

\item They are not based on methods of Indian knowledge traditions.

\item The assumptions and theses are driven by historically institutionalised biases and deplorable ``free-style'' methods (philology) of interpreting text.
\end{enumerate}

\section*{Refutations - Classes of {{\sl\bfseries Uttara-pakṣa}\relax}}
\index{uttarapaksa}

The {\sl pūrva-pakṣa} we have made has highlighted some peculiarities of Pollock's style and method. Based on the causal chains of reasoning in Pollock's logic, certain arguments have been built, and guidelines have been presented which could be used as a basis for future {\sl uttara-pakṣa}/responses. A cogent response to the claims made by Pollock can possibly be constructed using some combination of these methods. It is to be noted that the list below is only indicative and by no means exhaustive.
\begin{enumerate}
\item {\bf Basic logical flaws\index{Pollock!logical flaws} in causal chains of reasoning} - Through the use of the scientific method of falsifiability, researchers can provide counter-examples to Pollock's claims.  But given the fact that social sciences and humanities are often pseudo-scientific and pseudo-analytic this method might need to backed up with other lines of argumentation as well.

\item {\bf Flawed use of Critical Theory} - The methods and lines of argumentation used in free-style Pollockian deconstruction via philology, are deeply flawed as a method. A detailed case expanding the section `Method to the mischief' can be made. 

\item {\bf Flawed use of Philology}\index{Pollock!flawed use of philology} - Similar to the above, scholars can construct arguments on the flawed uses of Philology, Deconstruction and the ``Three Dimensions'' in Pollock's approach.

\item {\bf Data-centric Analysis (Empirical methods)} - By interacting with actual practitioners of {\sl śāstra}, insiders to the Indian tradition and intellectual practice may bring out via a sufficiently detailed questionnaire, with both objective and subjective focus\endnote{Digital Humanities means the practice and use of techniques from computing in the humanities -  text analysis, data and information mining from both traditional and newer data sources like social media conversations, blogs and other digital and digitised sources and their visualisation and dissemination techniques. For a good introduction  See Burdick (2012)} (see Note on digital humanities), the dissonance between the claims made by Pollock and  actual real-world realities.

\item {\bf Use of techniques like Discourse Representation Theory} which are not used in Western Critical Theory {\sl per se} - but which are used in semantic analysis and for encoding of knowledge (see Geurts {\sl et al} (2015)). Diagrammatic representations used in this framework can be used as part of relatively easier pedagogical methods - making arguments and criticism ``visible'' and accessible to non-experts too.

The methods listed below would rely on the traditional Indian frameworks as basis

\item {\bf Epistemological dissonance ({{\sl\bfseries Pramana}\relax})}\index{epistemological dissonance} - The sources of knowledge and the meanings ascribed by the West to Sanskrit non-translatables can be discussed and arguments can be made based on his incomplete understanding of the semantics.
\begin{itemize}
\item[a.]  Testimonial ({\sl Śabda}) dissonance and documented views from centuries of commentaries - Pollock's claims are simply different from those based on the commentaries on the {\sl śāstra}-s.

\item[b.] Use of Indian theories of error ({\sl khyāti-vāda}) must be brought to bear upon the lines of his reasoning and argument.
\end{itemize}

\item {\bf Ontological dissonance ({{\sl\bfseries Tattva}\relax})}\index{ontological dissonance} -  Western secular worldview does not believe in {\sl ātman}, the {\sl manas}, the {\sl guṇa}-s and other Indian principles of reality. There are no equivalent categories in science. They have been at best ignored or at worst collapsed into other categories. Arguments can be built on the basis of this dissonance.

\item {\bf Interpretive dissonance ({{\sl\bfseries Mīmāṁsā}\relax})}\index{interpretative dissonance} - The disposition of the interpreter of a text ({\sl saṁskāra}) is a key factor in the outsider-insider framework of anthropology and Western social sciences. Pollock is an outsider, and as an outsider he will make such claims. Arguments can be built on the basis of this dissonance. The resulting subjectivity and lack of objectivity of this method can be argued out.

\item {\bf Philosophical dissonance ({{\sl\bfseries Darśana}\relax})}\index{philosophical dissonance} - This would be a class of heavyweight methods of refutation. All the above three dissonances, ontological, epistemological and interpretive - from the perspective of a specific {\sl vedantic darśana}, for example: a Dvaita perspective of {\sl śāstra} or an Advaita perspective, can be brought to bear on the claims made in the paper. As each {\sl darśana} is complete in itself in term of its methods of arguments, its {\sl pramāṇa}-s and {\sl śāstra}, and have their own massive repositories of {\sl vāda}-s (arguments), each such refutation would be complete in itself. Other perspectives like those from {\sl Jaina} and {\sl Bauddha darśana} can also be brought to bear independently.

\item {\bf Reflective methods} - The selfsame methods used by Pollock can be used as basis of argumentation. As suggested in Malhotra (2016:354),\index{Malhotra, Rajiv} we can use Pollockian methods of philology on his work and also use his thesis of the aesthetization of power.\index{anesthetization of power} Pollock and Western academia (the first world) uses English (language of power) to interpret native bodies of knowledge. Arguments can be built based on such a narrative.
\end{enumerate}

\section*{Concluding Remarks}

We have critically appraised Pollock's theses on {\sl śāstra} using traditional Indian perspectives and some Western methods. The claims made in the original paper and the methods used therein have impacted the individual and collective psyche of the followers of {\sl dharma} via its influence through decades on academics and intellectuals. The sophisticated Western training and colonized {\sl saṁskāra} of many Indian-born academics and intellectuals influence not only the global but also the domestic discourse in India. It has had secondary and tertiary effects through popular culture and dissemination of its interpretations through the leftist control of the official Indian narrative through activities like preparation of textbooks and history-writing. As a direct follow-up based on approaches used in this paper a formalized dialectic tool, a Pollock {\sl Anti-Reader} (diagnostic guidelines)\index{Pollock!diagnostic guidelines} is suggested. This can take the form of a book or a digital tool driven by topic-models based on the writings of Pollock.

The Indian self-narrative academic social sciences as practised in India in English, traditional critical scholarship in Sanskrit and the regional languages  - these have not been able to shake off the influences of the Western models; or are able to even interpret or adapt them to the traditional perspectives - so much so that the very idea of ``traditional'' is being redefined.

This existing narrative needs to be challenged in many more ways and a sustained response mounted via new hybrid interpretive techniques based on the traditional Indian approaches. 

\begin{thebibliography}{99}
\itemsep=2pt
\bibitem[]{chap11_item1} 
Adluri, V. P. (2011). ``Pride and Prejudice: Orientalism and German Indology''. {\sl Hindu Studies International Journal of Hindu Studies}, 15(3); pp. 253--292. doi:10.1007/s11407-011-9109-4

\bibitem[]{chap11_item2}
--- and Bagchee, J. (2013). {\sl The Nay Science: A History of German Indology}. 
Oxford : Oxford University Press

\bibitem[]{chap11_item3}
Applebaum, W. (2000). {\sl Encyclopedia of the Scientific Revolution: From Copernicus to Newton}. New York: Garland Pub.

\bibitem[]{chap11_item4}
--- (2005). {\sl The Scientific Revolution and the Foundations of Modern Science}. Westport, CT: Greenwood Press.

\bibitem[]{chap11_item5}
Aurobindo. (1997). {\sl The Complete Works of Sri Aurobindo. Essays on the Gita}. Pondicherry: Sri Aurobindo Ashram Publ. Dept.

\bibitem[]{chap11_item6}
Bohman, James (2015) ``Critical Theory''. {\sl The Stanford Encyclopedia of Philosophy} (Winter 2015 Edition); Edward N. Zalta (ed.). URL = \url{<http://plato.stanford.edu/archives/win2015/entries/critical-theory/>}

\bibitem[]{chap11_item7}
Burdick, A. (2012). {\sl Digital Humanities}. Cambridge, MA: MIT Press.

\bibitem[]{chap11_item8}
Cahoone, L. E. (2010). {\sl The Modern Intellectual Tradition: From Descartes to Derrida, Course Guidebook}. Chantilly, VA: The Teaching Company.

\bibitem[]{chap11_item9}
Castle, G. (2007). {\sl The Blackwell Guide to Literary Theory}. Malden, MA: Blackwell.

\bibitem[]{chap11_item10}
``Chapter Two: Contents of the Gītā Summarized''. In {\sl Bhaktivedanta VedaBase}. \url{http://www.vedabase.com/en/bg/2}. Accessed on 10 May, 2016

\bibitem[]{chap11_item11}
Chatterjee, S. (1950). {\sl The Nyāya Theory of Knowledge, a Critical Study of Some Problems of Logic and Metaphysics}. Calcutta: University of Calcutta.

\bibitem[]{chap11_item12}
Cornelissen, M. (2013). {\sl Foundations and Applications of Indian Psychology}. Pearson India.

\bibitem[]{chap11_item13}
Critchley, S. (1992). {\sl The Ethics of Deconstruction: Derrida and Levinas}. Oxford, UK: Blackwell.

\bibitem[]{chap11_item14}
Dulgnan, Brian (Last modified on 31 Oct, 2014) ``Postmodernism''. In {\sl Encyclopedia Brittanica}. \url{https://www.britannica.com/topic/postmodernism-philosophy}.  Accessed on 18 March, 2017.

\bibitem[]{chap11_item15}
Durant, W. (1930). {\sl The Case for India}. New York: Simon and Schuster.

\bibitem[]{chap11_item16}
Emch, G. G., Sridharan, R., \& Srinivas, M. D. (2005). {\sl Contributions to the History of Indian Mathematics}. New Delhi: Hindustan Book Agency.

\bibitem[]{chap11_item17}
Feynman, R (published 2013, January 20). ``The Scientific Method - Richard Feynman'' \url{https://www.youtube.com/watch?v=OL6-x0modwY}. Accessed on 18 May, 2016.

\bibitem[]{chap11_item18}
Fish, Stanley (06 Jan, 2008) ``Will the Humanities Save Us?''. In {\sl The New York Times}. \url{http://fish.blogs.nytimes.com/2008/01/06/will-the-humanities-save-us/}. Accessed on 18 March, 2017.

\bibitem[]{chap11_item19}
Geurts, Bart, Beaver, David I. and Maier, Emar (2016) ``Discourse Representation Theory'', {\sl The Stanford Encyclopedia of Philosophy} (Spring 2016 Edition); Edward N. Zalta (ed.). URL = \url{<http://plato.stanford.edu/archives/spr2016/entries/discourse-representation-theory/>}.

\bibitem[]{chap11_item20}
Gillon, B. (First version on 19 Apr, 2011). ``Logic in Classical Indian Philosophy''. \url{http://plato.stanford.edu/entries/logic-india/} Accessed on 10 May, 2016.

\bibitem[]{chap11_item21}
Gurumurthy, S. (08 Oct, 2014) ``S Gurumurthy's lecture 12'', \url{https://www.youtube.com/watch?v=bVxm7GuNys0&t=295}. Accessed on May 20, 2016. 

\bibitem[]{chap11_item22}
Heera, B. (2011). {\sl Uniqueness of Cārvāka philosophy in traditional Indian thought}. New Delhi: Decent Books.

%\bibitem[]{chap11_item24}
%``Humanities''  \url{https://en.wikipedia.org/wiki/Humanities}. May 09, 2016. 

%\bibitem[]{chap11_item25}
%``Indology'' \url{ https://en.wikipedia.org/wiki/Indology}. Accessed on May 13, 2016.

\bibitem[]{chap11_item23}
Jacob, G. A. (1983). {\sl A Handful of Popular Maxims: Parts I, II \& III}. Delhi: Niranjana.

\bibitem[]{chap11_item24}
Jung, C. G., \& Kerenyi, C. (1963). {\sl Science of Mythology: Essays on the Myth of the Divine Child and the Mysteries of Eleusis}. London: Routledge. 

\bibitem[]{chap11_item25}
Kannan Swami (06 Oct, 2013) ``Part 001 - Modern Mathematics in Ancient India - Upanyasams by Sri U Ve Navalpakkam Dr. Kannan Swami.'' \url{https://www.youtube.com/watch?v=EO3g73gyJe8&list=PLlwAjY5wheXD4jls0pJom1SGkjUfOBSim}. Accessed on 07 May, 2016

\bibitem[]{chap11_item26}
Kapoor, K. (2005a). {\sl Dimensions of Pāṇini grammar: The Indian grammatical system}. New Delhi: D.K. Printworld.

\bibitem[]{chap11_item27}
--- (2005b). {\sl Text and interpretation: The Indian tradition}. New Delhi: D.K. Printworld. 

\bibitem[]{chap11_item28}
--- and Singh, A. K. (2005). {\sl Indian Knowledge Systems}. New Delhi: D.K. Printworld.

\bibitem[]{chap11_item29}
Kashyap, R. L., and Sadagopan, S. (1998). {\sl Rig Veda Samhita}. Bangalore: Sri Aurobindo Kapāli Sāstry Institute of Vedic Culture.

\bibitem[]{chap11_item30}
Kazanas, N. (2010). {\sl Economic principles in the Vedic tradition}. New Delhi: Aditya Prakashan.

\bibitem[]{chap11_item31}
Kuhn, T. S. (1977). {\sl The Essential Tension: Selected Studies in Scientific Tradition and Change}. Chicago: University of Chicago Press.

\bibitem[]{chap11_item32}
--- (1996). {\sl The Structure of Scientific Revolution}. 3$^{\text{rd}}$ Edn. Chicago: University of Chicago Press.

\bibitem[]{chap11_item33}
Lindlof, T. R., \& Taylor, B. C. (2011). {\sl Qualitative Communication Research Methods}. London: SAGE.

\bibitem[]{chap11_item34}
Lüdemann, S. (2014). {\sl Politics of Deconstruction: A New Introduction to Jacques Derrida}. Stanford : Stanford University Press

\bibitem[]{chap11_item35}
Malhotra, R. (2011). {\sl Being Different: An Indian Challenge to Western Universalism}. New Delhi: HarperCollins  India. 

\bibitem[]{chap11_item36}
--- (2014). {\sl Indra's Net: Defending Hinduism's Philosophical Unity}. India : Harper Collins.

\bibitem[]{chap11_item37}
--- (2016). {\sl The Battle for Sanskrit: Is Sanskrit political or sacred, oppressive or liberating, dead or alive?} New Delhi: HarperCollins India.

\bibitem[]{chap11_item38}
Malhotra, Rajiv  (2016) ``Why Sheldon Pollock is a very important Indologist to engage''. \url{http://rajivmalhotra.com/books/the-battle-for-sanskrit/why-sheldon-pollock-is-a-very-important-indologist-to-engage/} Accessed on 13May, 2016.

\bibitem[]{chap11_item39}
--- and Neelakandan, A. (2011). {\sl Breaking India: Western Interventions in Dravidian and Dalit Faultlines}. New Delhi: Amaryllis.

\bibitem[]{chap11_item40}
Matilal, B. K. (1971). {\sl Epistemology, Logic, and Grammar in Indian Philosophical Analysis}. The Hague: Mouton.

\bibitem[]{chap11_item41}
--- (1990). {\sl The Word and the World: India's Contribution to the Study of Language}. Delhi: Oxford University Press.

\bibitem[]{chap11_item42}
McQuillan, M. (2007). {\sl The Politics of Deconstruction: Jacques Derrida and the Other of Philosophy}. London: Pluto Press.

\bibitem[]{chap11_item43}
Oetke, C. (1996, 447). ``Ancient Indian logic as a theory of nonmonotonic reasoning''. {\sl Journal of Indian Philosophy}, 24(5). doi:10.1007/bf00200326

%\bibitem[]{chap11_item46}
%``Outline of Critical Theory''  \url{https://en.wikipedia.org/wiki/Outline_of_critical_theory}. Accessed on 09 May, 2016.

\bibitem[]{chap11_item44}
Pandurangi, K. T. (2013). {\sl Critical Essays on Pūrvamīmāmsā}. Bangalore : Vidyadhisa Post Graduate Sanskrit Research Centre.

\bibitem[]{chap11_item45}
Phillips, Stephen (2015) ``Epistemology in Classical Indian Philosophy''. {\sl The Stanford Encyclopedia of Philosophy} (Spring 2015 Edition); Edward N. Zalta (ed.). URL = \url{<http://plato.stanford.edu/archives/spr2015/entries/epistemology-india/>}

\bibitem[]{chap11_item46}
--- and Tatacharya, N.S.R (2004). {\sl Epistemology of Perception: Gaṅgeśa's} Tattva-cintāmaṇi: {\em Jewel of Reflection on the Truth (About Epistemology), the Perception Chapter (Pratyakṣa-khaṇḍa).} New York: American Institute of Buddhist Studies.

\bibitem[]{chap11_item46a}
--- ``The Philosophy of Sāṅkhya''. In {\sl Bhaktivedanta VedaBase}. \url{http://www.vedabase.com/en/sb/11/24}. Accessed on May 10, 2016.

%\bibitem[]{chap11_item50}
%``Philology''. \url{https://en.wikipedia.org/wiki/Philology}. Accessed on 09 May, 2016.

\bibitem[]{chap11_item47}
Pike, K. L. (1967). {\sl Language in Relation to a Unified Theory of the Structure of Human Behavior}. The Hague: Mouton.

\bibitem[]{chap11_item48}
Pollock (1985, 07). ``The Theory of Practice and the Practice of Theory in Indian Intellectual History''. {\sl Journal of the American Oriental Society}, 105(3). pp. 499--519. doi:10.2307/601525

\bibitem[]{chap11_item49}
--- (1989). ``Mīmāṃsā and the Problem of History in Traditional India''. {\sl Journal of the American Oriental Society}, 109(4), pp. 603--610. doi:10.2307/604085

\bibitem[]{chap11_item50}
--- (2004). ``The Meaning of Dharma and the Relationship of the Two Mīmāṁsās: Appayya Dīkṣita's Discourse on the Refutation of a Unified Knowledge System of Pūrva-mīmāṁsā and Uttara-mīmāṁsā''. {\sl Journal of Indian Philosophy}, 32(5-6); pp. 769--811. doi:10.1007/s10781-004-8650-5

\bibitem[]{chap11_item51}
--- (2006). {\sl The Language of the Gods in the World of Men: Sanskrit, Culture, and Power in Premodern India}. Berkeley: University of California Press.

\bibitem[]{chap11_item52}
Pollock, S. (2007). ``Pretextures of Time''. {\sl History and Theory}, 46(3); pp. 366--383. doi:10.1111/j.1468-2303.2007.00415.x

\bibitem[]{chap11_item53}
--- (2009). ``Future Philology? The Fate of a Soft Science in a Hard World''. {\sl Critical Inquiry}, 35(4); pp. 931--961. doi:10.1086/599594

\bibitem[]{chap11_item54}
---. (2014). ``Philology in Three Dimensions''. {\sl Postmedieval: A Journal of Medieval Cultural Studies Postmedieval}, 5(4); pp. 398--413. doi:10.1057/pmed.2014.33

\bibitem[]{chap11_item55}
Pollock, S. I., Elman, B. A., \& Chang, K. K. (2015). {\sl World Philology}. Harvard : Harvard University Press.

\bibitem[]{chap11_item56}
Popper, K. R. (1968). {\sl The Logic of Scientific Discovery}. London: Hutchinson.

\bibitem[]{chap11_item57}
Price, D. H. (2016). {\sl Cold War Anthropology: The CIA, the Pentagon, and the Growth of Dual use Anthropology}. Durham : Duke University Press

\bibitem[]{chap11_item58}
``Pūrvapaksha of Sheldon Pollock: Misappropriating text from VS Naipaul's article'' (Last modified on 07 May, 2016). \url{http://purvapaksha-pollock.blogspot.in/2016/05/misappropriating-text-from-vs-naipauls.html}. Accessed on 07 May, 2016.

\bibitem[]{chap11_item59}
Raju, C. K. (2003). {\sl The Eleven Pictures of Time: The Physics, Philosophy, and Politics of Time Beliefs}. New Delhi: Sage Publications.

\bibitem[]{chap11_item60}
--- (2007). {\sl Cultural Foundations of Mathematics: The Nature of Mathematical Proof and the Transmission of the Calculus from India to Europe in the 16th c. CE}. Delhi: Pearson Longman.

\bibitem[]{chap11_item61}
--- (2009). {\sl Is Science Western in Origin?} Delhi: Daanish Books.

\bibitem[]{chap11_item62}
Sahadeo, R. (2012). {\sl Mohandas K. Gandhi: Thoughts, Words, Deeds: His inspiration: Bhagavad-Gita}. Xlibris Corporation: Ramnarine Sahadeo.

\bibitem[]{chap11_item63}
Said, E. W. (1979). {\sl Orientalism}. New York: Vintage Books.

\bibitem[]{chap11_item64}
{\sl Saṃkhyakarikā}. See Śāstri (1941).

\bibitem[]{chap11_item65}
Śastri D. (Ed.) (1941) {\sl Saṃkhyakarikā} with commentary of Gauḍapādācārya (with Notes and Hindi Translation). Benares: Chowkhamba Sanskrit Series Office.

\bibitem[]{chap11_item66}
``Shanti Parva Part III''. \url{http://www.sacred-texts.com/hin/m12/m12c001.htm}. Accessed on May 10, 2016.


\bibitem[]{chap11_item67}
Shastri, D. (1957). {\sl A Short History of Indian Materialism, Sensationalism and Hedonism}. Calcutta: Bookland Private.

\bibitem[]{chap11_item68}
``Sheldon Pollock: Liberation Philology.'' (Last modified on May 10, 2016.). \url{http://www.csds.in/events/sheldon-pollock-liberation-philology}. Accessed on 10 May, 2016.

\bibitem[]{chap11_item69}
Stocker, B. (2006). {\sl Routledge Philosophy Guidebook to Derrida on Deconstruction}. Oxon : Routledge

%\bibitem[]{chap11_item73}
%Sudarshan, T.N. (Last modified on 17 May, 2016) ``The Science and Nescience of {\sl śāstra}'' (Web Version). \url{https://sudarshanauvaca.wordpress.com/}. Accessed on 18 May, 2016.
\end{thebibliography}

\theendnotes
\label{chapter\thechapter:end}
