%%importing packages
\usepackage[utf8]{inputenc}
\usepackage{fontspec}
\usepackage[xetex,dvips]{graphicx}
\usepackage{fancyhdr}
\usepackage{setspace}
\usepackage{xstring}
\usepackage{polyglossia}
\usepackage{titlesec}
\usepackage{xpatch}
\usepackage{endnotes}
\usepackage{array}
\usepackage{longtable}
\usepackage{enumitem}
\usepackage{color}
\usepackage{longfbox}
\usepackage{ifthen}
\usepackage{pifont}
\usepackage{float}
\usepackage{imakeidx}

\def\Break{\break }
\def\passim#1{\textsl{passim}}
\indexsetup{othercode=\small}
\makeindex[title=Index,program=xindy,options=-C utf8 -M mystyle.xdy -L Kannada,columnseprule=true,columns=3]



\setlist[itemize]{noitemsep, topsep=0pt}
\setlist[enumerate]{noitemsep, topsep=0pt}

\setcounter{tocdepth}{1}

%%longtable settings
\setlength\LTleft{30pt}
\setlength{\LTpre}{0pt}
\setlength{\LTpost}{0pt}



\xpatchcmd{\verse}{\itemsep}{\advance\topsep-0.6em\itemsep}{}{}
\xpatchcmd{\flushright}{\itemsep}{\advance\topsep-1em\itemsep}{}{}
\xpatchcmd{\flushleft}{\itemsep}{\advance\topsep-1em\itemsep}{}{}

\setmainlanguage[numerals=Kannada]{kannada}
\setotherlanguages{english}

%%page settings for the book with zwpagelayout package
\usepackage[papersize={180mm,250mm},textwidth=150mm,
textheight=210mm,headheight=6mm,headsep=4mm,topmargin=15mm,botmargin=15mm,
leftmargin=15mm,rightmargin=15mm,cropmarks]{zwpagelayout}

%%page settings for the book with geometry package
%~ \usepackage{geometry}
%~ \usepackage{atbegshi}
%~ \geometry{paper=a5paper,layoutsize={120mm,182mm},layouthoffset=13.5mm,layoutvoffset=14mm,width=90mm,left=15mm,right=15mm,height=146mm,headheight=5mm,headsep=4mm,top=15mm,bottom=12mm,includehead=false,includefoot=true,showcrop=true}

%%defining fonts
\setmainfont[
	Script=Kannada,
	BoldFont=SHREE-KAN-OTF-0850-Bold,
	ItalicFont=SHREE-KAN-OTF-0850-Italic,
	BoldItalicFont=SHREE-KAN-OTF-0850-Bold-Italic,
	HyphenChar="200C
]{SHREE-KAN-OTF-0850}

\newfontfamily\kannadafont[
	Script=Kannada,
	BoldFont=SHREE-KAN-OTF-0850-Bold,
	ItalicFont=SHREE-KAN-OTF-0850-Italic,
	BoldItalicFont=SHREE-KAN-OTF-0850-Bold-Italic,
	HyphenChar="200C
]{SHREE-KAN-OTF-0850}

\defaultfontfeatures{Ligatures=TeX}

\newfontfamily\englishfont[
	Script=Latin,
	Ligatures=TeX,
	BoldFont=GentiumBasic-Bold,
	ItalicFont=GentiumBasic-Italic,
	BoldItalicFont=GentiumBasic-BoldItalic,
]{GentiumBasic}


%%user defined commands
\long\def\bookTitle#1{\vfill\centerline{{\fontsize{30}{32}\selectfont\textbf{#1}}}\vfill}
\def\titleauthor#1{\centerline{{\LARGE\textbf{#1}}}\vfill}
\newenvironment{myquote}[1]{\medskip\par\bgroup\fontsize{10}{12}\selectfont\noindent\leftskip=10pt\rightskip=10pt#1}{\par\egroup\medskip}
\def\delimiter{\bigskip\centerline{*\quad*\quad*}\bigskip}
\def\general#1{#1}
\def\vauthor#1{{\hfill #1}}
\def\supskpt#1{$^{#1}$}

\def\enginline#1{{\fontsize{11}{13}\selectfont\eng{#1}}}
\def\publisher#1{{\fontsize{16pt}{20pt}\selectfont\bfseries #1}}
\def\place#1{{\fontsize{14pt}{16pt}\selectfont #1}}

\def\eng#1{{\englishfont\textenglish{#1}}}
\def\kan#1{{\kannadafont\textkannada{#1}}}
\def\engfoot#1{\eng{#1}}

\def\notesname{ಅಡಿ ಟಿಪ್ಪಣಿಗಳು}%
\renewcommand{\contentsname}{ವಿಷಯಾನುಕ್ರಮಣಿಕೆ}

%%fancy header settings
\fancypagestyle{plain}{%
\chead[]{}
\lhead[]{}
\rhead[]{}
\cfoot[]{}
}
\lhead[{\fontsize{12}{12}\selectfont\thepage\quad ಮಂಡ್ಯ ಜಿಲ್ಲೆಯ ಶಾಸನ ಮತ್ತು ಸಂಸ್ಕೃತಿ}]{}
\rhead[]{{\fontsize{12}{12}\selectfont\leftmark\quad\thepage}}
\chead[]{}
\lfoot[]{}
\rfoot[]{}
\cfoot[]{}

\renewcommand{\headrulewidth}{0pt}

\pagestyle{fancy}

\renewcommand{\thefootnote}{\fnsymbol{footnote}}

%%redefining macros
\renewcommand\chaptermark[1]{\markboth{#1}{}}
\makeatletter

\titleformat*{\section}{\fontsize{15}{17}\fontseries{bx}\selectfont}
\titlespacing*{\section}{0pt}{*1.5}{*1}

\def\@seccntformat#1{%
  \expandafter\ifx\csname c@#1\endcsname\c@section\else
  \csname the#1\endcsname\quad
  \fi}

\def\@makechapterhead#1{%
  \vspace*{50\p@}%
  {\parindent \z@ \raggedright \normalfont
    \ifnum \c@secnumdepth >\m@ne
      \if@mainmatter
	\setcounter{endnote}{0}%
        \fontsize{18}{20}\fontseries{bx}\selectfont \@chapapp\space – \thechapter\space #1
        \par\nobreak
        \vskip 40\p@
      \fi
    \fi
    \interlinepenalty\@M
    %~ \Huge \bfseries \par\nobreak
    %~ \vskip 40\p@
  }}
  
\renewcommand\labelitemi{\eng{\textbullet}}

%\renewcommand\chapter{\if@openright\cleardoublepage\else\ifthenelse{\arabic{chapter} > 0}{\chapterend}{}\clearpage\fi
%                    \thispagestyle{plain}%
%                    \global\@topnum\z@
%                    \@afterindentfalse
%                    \secdef\@chapter\@schapter}



%\def\@makeenmark{\relax{\kern1pt}\small\@theenmark}
\def\enoteformat{\rightskip\z@ \leftskip\z@ \parindent=1.8em
  \leavevmode\llap{\theenmark. }}
                      
\makeatother

\linespread{1.04}
\pretolerance=-1
\tolerance=4000
\hfuzz=1pt
%~ \hbadness=10000
%~ \tolerance=1
%~ \emergencystretch=\maxdimen
%~ \hyphenpenalty=10000
%~ \hbadness=10000

\setlength{\parindent}{3.2em}
\setlength{\parskip}{.5em}


\brokenpenalty9999\relax

\newcommand{\sethyphenation}[3][]{%
  \sbox0{\begin{otherlanguage}[#1]{#2}
    \hyphenation{#3}\end{otherlanguage}}}

%\def\passim#1{\textsl{passim}}
%~ \makeindex
%~ \makeindex[title=Index,program=makeindex,options=-s mystyle.ist,columnseprule=true,columns=2]
%\makeindex[title=Index,program=xindy,options=-C utf8 -M mystyle.xdy -L kannada,columnseprule=true,columns=2]

\input dictionary1

