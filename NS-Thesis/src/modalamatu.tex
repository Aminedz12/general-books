
\chapter*{ಮೊದಲ ಮಾತು}
\addcontentsline{toc}{chapter}{ಮೊದಲ ಮಾತು}

ಮಂಡ್ಯ ಜಿಲ್ಲೆಯ, ಕೃಷ್ಣರಾಜಪೇಟೆ ತಾಲ್ಲೂಕಿನ, ಸಂತೇಬಾಚಹಳ್ಳಿ ನಮ್ಮ ಊರು.  ಮಂಡ್ಯ{\rm -}ಹಾಸನ ಜಿಲ್ಲೆಯ  ಗಡಿ ಗ್ರಾಮವಾದ ನಮ್ಮ ಊರಿನ ಸುತ್ತಲೂ ಸುಮಾರು 5 ಮೈಲಿಯಿಂದ 15 ಮೈಲಿ ಫಾಸಲೆಯೊಳಗೆ,  ನೂರಾರು ಶಾಸನಗಳು, ಸುಂದರ ದೇವಾಲಯಗಳು, ಬಸದಿಗಳು ಮತ್ತು ಸ್ಮಾರಕ ಶಿಲೆಗಳನ್ನು ಹೊಂದಿರುವ, ಐತಿಹಾಸಿಕವಾಗಿ ಮತ್ತು ಸಾಂಸ್ಕೃತಿಕವಾಗಿ ಬಹಳ ಮಹತ್ವವನ್ನು ಪಡೆದಿರುವ ಶ್ರವಣಬೆಳಗೊಳ, ಕಂಬದಹಳ್ಳಿ, ಮೇಲುಕೋಟೆ, ಅಗ್ರಹಾರ ಬಾಚಹಳ್ಳಿ, ಕಿಕ್ಕೇರಿ, ಹೊಸಹೊಳಲು, ನಾಗಮಂಗಲ, ಸಿಂದಘಟ್ಟ ಮೊದಲಾದ ಊರುಗಳಿವೆ.  ಈ ಊರುಗಳು ಮೊದಲಿನಿಂದಲೂ ನನ್ನನ್ನು ಸೆಳೆಯುತ್ತಿದ್ದವು. ನಮ್ಮ ಊರಿನಲ್ಲಿಯೂ ಹೊಯ್ಸಳರ ಕಾಲದ ಮಹಲಿಂಗೇಶ್ವರ, ವಿಜಯನಗರ ಕಾಲದ ವೀರನಾರಾಯಣ, ವೀರಭದ್ರೇಶ್ವರ ದೇವಾಲಯಗಳೂ, ಶಾಸನಗಳೂ ಇವೆ.  ನಮ್ಮ ಊರಿನ ಬಳಿ ಇರುವ ಬೆಣ್ಣೆ ಸಿದ್ಧನ ಗುಡ್ಡದಲ್ಲಿ ಕಲ್ಲಿನ ಸಮಾಧಿಗಳಿವೆ.  ಹೀಗಾಗಿ ಮೊದಲಿನಿಂದಲೂ, ದೇವಾಲಯಗಳು, ಶಾಸನಗಳ ಬಗ್ಗೆ ನನಗೆ ಆಸಕ್ತಿ ಬೆಳೆದಿತ್ತು. ಆದರೆ ಮೊದಲು ಆ ದಿಕ್ಕಿನಲ್ಲಿ ನನ್ನ ಓದು ಸಾಗಲಿಲ್ಲ. ನಮ್ಮ ಮಂಡ್ಯ ಜಿಲ್ಲೆ ಎಂದರೆ ನನಗೆ ಅಭಿಮಾನ. ಜಿಲ್ಲೆಯ ಎಲ್ಲ ತಾಲ್ಲೂಕುಗಳಲ್ಲೂ ತಿರುಗಾಡಿದ್ದೇನೆ.

ಕನ್ನಡ, ಇಂಗ್ಲಿಷ್​, ಬೆರಳಚ್ಚು ಮತ್ತು ಶೀಘ್ರಲಿಪಿ ಪರೀಕ್ಷೆಗಳಲ್ಲಿ ತೇರ್ಗಡೆಯಾಗಿ, ಜೀವನೋಪಾಯಕ್ಕಾಗಿ ಸರ್ಕಾರಿ ಸೇವೆಗೆ ಸೇರಿದೆ. ನಂತರ ಸೇವೆಯಲ್ಲಿದ್ದಾಗಲೇ ಕನ್ನಡ ಮೇಜರ್​, ಇತಿಹಾಸ ಮತ್ತು  ಸಮಾಜ ಶಾಸ್ತ್ರವನ್ನು ಮೈನರ್​ ವಿಷಯವಾಗಿ ತೆಗೆದುಕೊಂಡು ಬಿ.ಎ. ಪದವಿ ಪಡೆದೆ. ಮೈಸೂರು ವಿ.ವಿ. ಅಂಚೆ ಮತ್ತು ತೆರಪಿನ ಶಿಕ್ಷಣ ಸಂಸ್ಥೆಯಿಂದ ಶಾಸನಶಾಸ್ತ್ರ ಮತ್ತು ಕರ್ನಾಟಕದ ಸಾಂಸ್ಕೃತಿಕ ಚರಿತ್ರೆಯನ್ನು ಮುಖ್ಯ ವಿಷಯವನ್ನಾಗಿ ತೆಗೆದುಕೊಂಡು ಎಂ.ಎ. ಮಾಡಿದೆ. ಕನ್ನಡ ಸಾಹಿತ್ಯ ಪರಿಷತ್ತಿನ ಮೂಲಕ ಹಂಪಿ ಕನ್ನಡ ವಿಶ್ವವಿದ್ಯಾನಿಲಯವು ನಡೆಸುತ್ತಿದ್ದ ಶಾಸನ ಶಾಸ್ತ್ರ ಡಿಪ್ಲೊಮಾ ಪರೀಕ್ಷೆಯಲ್ಲಿ ತೇರ್ಗಡೆಯಾದೆ. ಅಂಚೆ ಮತ್ತು ತೆರಪಿನ ಶಿಕ್ಷಣ ಸಂಸ್ಥೆಯಲ್ಲಿ ಎಂ.ಎ. ಅಧ್ಯಯನ ಮಾಡುವಾಗ,  ನನಗೆ ಪರಿಚಯವಾದವರು ನಮಗೆ ಪಾಠ ಮಾಡುತ್ತಿದ್ದ ಡಾ. ಡಿ.ಟಿ. ಬಸವರಾಜ್​ ಅವರು.   ನಂತರ ಅವರು ಕರ್ನಾಟಕ ರಾಜ್ಯ ಮುಕ್ತ ವಿಶ್ವವಿದ್ಯಾನಿಲಯಕ್ಕೆ ರೀಡರ್​ ಆಗಿ ಬಂದರು. ಮುಕ್ತ ವಿ.ವಿ. ಯಿಂದ ಅವರ ಮಾರ್ಗದರ್ಶನದಲ್ಲಿ ಎಂ.ಫಿಲ್​. ಪರೀಕ್ಷೆ ಮುಗಿಸಿದೆ. ಮುಕ್ತ ವಿ.ವಿ.ಯಲ್ಲಿ ಪಿಎಚ್​.ಡಿ. ಅಧ್ಯಯನಕ್ಕೆ ಅವಕಾಶ ಕಲ್ಪಿಸಲಾಯಿತು, ಅರ್ಜಿ ಹಾಕಲು ಸಿದ್ಧತೆ ಮಾಡಿಕೊಂಡು ಮೈಸೂರಿಗೆ ಹೋಗಿ, ಡಾ. ಡಿ. ಟಿ. ಬಿ. ಅವರನ್ನು ಭೇಟಿಮಾಡಿ ಸರ್​ ತಾವೇ ನನಗೆ ಗೈಡ್​ ಮಾಡಬೇಕೆಂದೆ. ನನ್ನ ಪಿಎಚ್. ಡಿ. ಅಧ್ಯಯನಕ್ಕೆ  ಮಾರ್ಗದರ್ಶಕರಾಗಿರಲು ಒಪ್ಪಿಕೊಂಡರು. ಡಾ. ಡಿ. ಟಿ. ಬಿ. ಅವರು ಖ್ಯಾತ ವಿದ್ವಾಂಸರಾದ ಡಾ. ಎಲ್​. ಬಸವರಾಜು ಮತ್ತು ಡಾ. ಎನ್​.ಎಸ್​. ತಾರಾನಾಥ್​ ಅವರ ಆತ್ಮೀಯ ಶಿಷ್ಯರು. ಹಮ್ಮು ಬಿಮ್ಮಿಲ್ಲದ, ಸ್ನೇಹಪರರಾದ, ಕನ್ನಡ ಸಾಹಿತ್ಯ ಮತ್ತು ಸಂಸ್ಕೃತಿಯ ಬಗ್ಗೆ ಅಪಾರ ತಿಳಿವಳಿಕೆಯುಳ್ಳ ಅವರು ನನಗೆ ಮಾರ್ಗದರ್ಶಕರಾದುದು ನನ್ನ ಅದೃಷ್ಟ.

ಮಂಡ್ಯ ಜಿಲ್ಲೆಯ ಶಾಸನಗಳು{\rm -}ಒಂದು ಅಧ್ಯಯನ ಎಂಬ ವಿಷಯದ ಮೇಲೆ ಪಿಎಚ್​.ಡಿ. ಅಧ್ಯಯನಕ್ಕೆ ರಿಜಿಸ್ಟರ್​ ಮಾಡಿಸಿ, ಮುಂದೆ ಐದು ವರ್ಷಗಳ ಕಾಲ ಸತತವಾಗಿ ಕುಳಿತು ಅಧ್ಯಯನ ಮಾಡಿದೆ, ಪ್ರಮುಖ ಶಾಸನಗಳು, ಸ್ಮಾರಕಗಳು ಇರುವ ಸುಮಾರು 200ಕ್ಕೂ ಹೆಚ್ಚು ಹಳ್ಳಿಗಳಿಗೆ ಭೇಟಿ ನೀಡಿ ಕ್ಷೇತ್ರ ಕಾರ್ಯ ಮಾಡಿದೆ. ಇದು ಒಂದು ಅಗಾದವಾದ ಕಾರ್ಯ ಎನಿಸಿತು. ಪ್ರತಿ ಅಧ್ಯಾಯದ ಬರವಣಿಗೆ ಮುಗಿದ ತಕ್ಷಣ ಮಾರ್ಗದರ್ಶಕರಿಗೆ ಸಲ್ಲಿಸುತ್ತಿದ್ದೆ. ನನ್ನ ಮಾರ್ಗದರ್ಶಕರಾದ ಡಾ. ಡಿಟಿಬಿ ಅವರು ಪ್ರತಿ ಅಧ್ಯಾಯವನ್ನು ಪರಿಶೀಲಿಸಿ, ಸೂಕ್ತ ಸೂಚನೆಗಳನ್ನು ನೀಡಿ, ಬರವಣಿಗೆಯ ಬಗ್ಗೆ ಮಾರ್ಗದರ್ಶನ ಮಾಡುತ್ತಿದ್ದರು. ಅದರ ಜೊತೆಗೆ ಶಾಸನ ಶಾಸ್ತ್ರ ತರಗತಿಗಳಲ್ಲಿ ನನ್ನ ಗುರುಗಳಾಗಿದ್ದ ಪ್ರಖ್ಯಾತ ಶಾಸನ ಶಾಸ್ತ್ರ ವಿದ್ವಾಂಸರು, ಪ್ರಾಚಾರ್ಯರೂ ಆದ ಡಾ. ಕೆ.ಆರ್​. ಗಣೇಶ್​ ಅವರಿಗೂ ನಾನು ಬರೆದ ಅಧ್ಯಾಯಗಳನ್ನು ಒಪ್ಪಿಸುತ್ತಿದ್ದೆ. ಅವರು ಅವುಗಳನ್ನು ಆಮೂಲಾಗ್ರವಾಗಿ ಪರಿಶೀಲಿಸಿ, ವಿಷಯ ಮಂಡನೆ ಬಗ್ಗೆ, ಬರವಣಿಗೆ ಬಗ್ಗೆ, ಪರಾಮರ್ಶೆಯ ಬಗ್ಗೆ ಸೂಕ್ತ ಸಲಹೆ ಸೂಚನೆಗಳನ್ನು ನೀಡುತಿದ್ದರು.

ಮಂಡ್ಯ ಜಿಲ್ಲೆಯ ಇತಿಹಾಸ ಮತ್ತು ಪುರಾತತ್ವದ ಬಗ್ಗೆ, ಮಂಡ್ಯದಲ್ಲಿ ದಿನಾಂಕ 11 ಮತ್ತು 12 ಆಗಸ್ಟ್​ 2007 ರಂದು ಏರ್ಪಡಿಸಲಾಗಿದ್ದ ವಿಚಾರ ಸಂಕಿರಣದಲ್ಲಿ ಹಾಗೂ ಅದರಲ್ಲಿ ಮಂಡಿಸಲಾದ ಪ್ರಬಂಧಗಳ ಸಂಕಲನ “ಮಂಡ್ಯ ಜಿಲ್ಲೆಯ ಇತಿಹಾಸ ಮತ್ತು ಪುರಾತತ್ವ”  ಎಂಬ ಗ್ರಂಥದಲ್ಲಿ ಪ್ರಕಟವಾಗಿರುವಂತೆ, ತಮ್ಮ ಅಧ್ಯಕ್ಷ ಭಾಷಣದಲ್ಲಿ ಖ್ಯಾತ ಪ್ರಾಚೀನ ಇತಿಹಾಸ ಮತ್ತು  ಪುರಾತತ್ವ ವಿದ್ವಾಂಸರಾದ ಡಾ. ಎಂ.ಎಸ್​. ಕೃಷ್ಣಮೂರ್ತಿಯವರು, “ಇಷ್ಟು ಪ್ರವರ್ಧಮಾನವಾಗಿ, ಸಂಪದ್ಭರಿತವಾಗಿ ನೆಲೆಸಿದ್ದ ನಾಡಿನ ಇತಿಹಾಸವನ್ನೂ, ಸಂಸ್ಕೃತಿಯನ್ನೂ ಪ್ರತ್ಯೇಕವಾಗಿ ಆಳವಾಗಿ ಅಧ್ಯಯನ ಮಾಡಲು ಯಾರೂ ಪ್ರಯತ್ನಿಸದಿರುವುದು ಅಚ್ಚರಿಯ ಸಂಗತಿಯೇ ಸರಿ...... “ಮಂಡ್ಯ ಜಿಲ್ಲೆಯಲ್ಲಿ ಹೊಯ್ಸಳ ಸಂಸ್ಕೃತಿಯ ವಿವಿಧ ಮುಖ\-ಗಳನ್ನು ಪರಿಚಯ ಮಾಡಿಕೊಡುವ ಗ್ರಂಥ ಇನ್ನೂ ಪ್ರಕಟವಾಗಬೇಕಾಗಿದೆ, ಇದೇ ಅವಲೋಕನ, ಅಭಿಪ್ರಾಯ ವಿಜಯನಗರ ಮತ್ತು ವಿಜಯನಗರೋತ್ತರ ಕಾಲದ ಸಂಶೋಧನೆಗಳಿಗೂ ಅನ್ವಯಿಸುತ್ತದೆ”.... \textbf{“ಈ ಎಲ್ಲಾ ಬದಲಾವಣೆ, ಬೆಳವಣಿಗೆಗಳ ಒಂದು ಜ್ಞಾನ, ಸಮಗ್ರ ಅಧ್ಯಯನಗಳಿಂದ ಬರಬೇಕೇ ಹೊರತು, ಆಳವಾದರೂ ಛಿದ್ರ{\rm -}ಛಿದ್ರ ಅಧ್ಯಯನಗಳಿಂದ ಖಂಡಿತಾ ಅಲ್ಲ”} ಎಂದು ತಮ್ಮ ಅಭಿಪ್ರಾಯವನ್ನು ವ್ಯಕ್ತಪಡಿಸಿದ್ದಾರೆ. ಈ ಅಭಿಪ್ರಾಯ ನನ್ನ ಪಿಎಚ್​.ಡಿ.ಅಧ್ಯಯನಕ್ಕೆ ಹಾಗೂ ಈಗ ನನ್ನ ಈ  ಕೃತಿ ರಚನೆಗೆ ಇಂಬು ನೀಡಿದೆ. ಈ ನಿಟ್ಟಿನಲ್ಲಿ ನನ್ನ ಅಧ್ಯಯನ ಮೊದಲ ಮೆಟ್ಟಿಲು ಎಂದು ಹೇಳಬಹುದು.

ಬಿಡುವಿಲ್ಲದ ಕಚೇರಿ ಕೆಲಸಗಳು, ಪ್ರವಾಸ, ಸಾಂಸಾರಿಕ ವಿಷಯಗಳು ಇವುಗಳ ನಡುವೆ,  ಐದು ವರ್ಷಗಳ ಕಾಲ ಸತತವಾಗಿ ಬಿಡುವಿನ ಸಮಯದಲ್ಲಿ, ಕೆಲಸ ಮಾಡಿ, ಪ್ರಬಂಧವನ್ನು ಸಿದ್ಧಪಡಿಸಿದೆ. 2012 ರಲ್ಲಿ ವಿಶ್ವವಿದ್ಯಾನಿಲಯಕ್ಕೆ ನನ್ನ ಪ್ರಬಂಧವನ್ನು ಸಲ್ಲಿಸಿದೆ. 2013 ರಲ್ಲಿ ನನಗೆ ಪಿಎಚ್. ಡಿ. ಪದವಿ ಪ್ರದಾನ(ಅವಾರ್ಡ್) ಆಯಿತು. ಮೈಸೂರು ವಿ.ವಿ. ಕನ್ನಡ ಅಧ್ಯಯನ ಸಂಸ್ಥೆಯ ಪ್ರೊಫೆಸರ್​, ಶಾಸನ ಶಾಸ್ತ್ರ ಕ್ಷೇತ್ರದ ಯುವ ವಿದ್ವಾಂಸ,  ಡಾ. ಎಂ.ಜಿ. ಮಂಜುನಾಥ್​ರವರು ಹಂಪಿ ವಿ.ವಿ.ಯವರು ಶ್ರವಣಬೆಳಗೊಳದಲ್ಲಿ ನಡೆಸಿದ ಶಾಸನಶಾಸ್ತ್ರ ಅಧ್ಯಯನದ ಮೂರು ದಿನಗಳ ಕಮ್ಮಟದಲ್ಲಿ ನಮಗೆ ಪ್ರಾಯೋಗಿಕವಾಗಿ ಹಾಗೂ ತರಗತಿಗಳಲ್ಲಿ ಶಾಸನಗಳ ಬಗ್ಗೆ ಉಪನ್ಯಾಸ ನೀಡಿದ್ದರು. ಅವರ ಸಂಶೋಧನಾ ಕೃತಿಗಳು, ಲೇಖನಗಳನ್ನೆಲ್ಲಾ ನಾನು ಓದಿದ್ದೇನೆ. ನನ್ನ ಪ್ರಬಂಧದ ಬಗ್ಗೆ ಅವರೂ ಮೆಚ್ಚುಗೆಯ ಮಾತುಗಳನ್ನು ಆಡಿದ್ದರು. ನನ್ನ ಈ ಕೃತಿ ಅಚ್ಚಿಗೆ ಬಂದಮೇಲೆ ಅವರು ತಮ್ಮ ಅಭಿಪ್ರಾಯವನ್ನು ಸಂತೋಷದಿಂದ ಬರೆದುಕೊಟ್ಟರು. ಅವರಿಗೆ ನಾನು ಅತ್ಯಂತ ಕೃತಜ್ಞನಾಗಿದ್ದೇನೆ.

ಕನ್ನಡ ಎಂ.ಎ., ಶಾಸನಶಾಸ್ತ್ರ, ಎಂ.ಫಿಲ್​. ತರಗತಿಗಳಲ್ಲಿ ನಾನು ಹೆಚ್ಚಾಗಿ ಓದುತ್ತಿದ್ದುದು, ಕನ್ನಡ ಶಾಸನಗಳ ಸಾಂಸ್ಕೃತಿಕ ಅಧ್ಯಯನ ಲೋಕದ ಹೆಬ್ಬಾಗಿಲನ್ನು ತೆರೆದ, ಪ್ರಖ್ಯಾತ ಸಂಶೋಧಕರು, ಚಿಂತಕರು,  ಕರ್ನಾಟಕದ ಇತಿಹಾಸ ಮತ್ತು ಸಂಸ್ಕೃತಿಯ ವಿದ್ವಾಂಸರು, ಕನ್ನಡ ನಾಡು ನುಡಿಯ ಅನನ್ಯ ಪ್ರೇಮಿಗಳೂ ಆದ ಡಾ. ಎಂ. ಚಿದಾನಂದ ಮೂರ್ತಿಯವರ ಸಂಶೋಧನಾ ಕೃತಿಗಳನ್ನು.  ಅವರನ್ನು ಅನೇಕ ಸಭೆ ಸಮಾರಂಭದಲ್ಲಿ ನೋಡಿದ್ದೆ. ಅವರ ಭಾಷಣಗಳನ್ನು, ಉಪನ್ಯಾಸಗಳನ್ನು ಕೇಳಿದ್ದೆ. ನನಗೆ, ನನ್ನಂತಹ ಸಾವಿರಾರು ಶಾಸನಶಾಸ್ತ್ರ ಅಧ್ಯಯನ ಆಸಕ್ತರಿಗೆ ಅವರು ಮಾನಸ ಗುರುಗಳು. ನನ್ನ ಪ್ರಬಂಧವನ್ನು ಅವರಿಗೆ ತೋರಿಸಿ, ಅವರ ಅಭಿಪ್ರಾಯವನ್ನು ಪಡೆದುಕೊಳ್ಳಬೇಕೆಂದು ಕಾತರನಾಗಿದ್ದೆ. ನನ್ನ ಮಿತ್ರರ ಮೂಲಕ ಅವರನ್ನು ಭೇಟಿ ಮಾಡಿ ಪರಿಚಯ ಮಾಡಿಕೊಂಡು ನನ್ನ ಪ್ರಬಂಧದ ಬಗ್ಗೆ ವಿವರಿಸಿದೆ. ನನ್ನ ಪ್ರಬಂಧವನ್ನು ಪರಿಶೀಲಿಸಲು ಒಪ್ಪಿಗೆಯನ್ನು ನೀಡಿದರು. ಒಂದೆರಡು ತಿಂಗಳ ನಂತರ ನನ್ನ ಮಿತ್ರರು ದೂರವಾಣಿ ಮಾಡಿ ಡಾ. ಎಂ. ಚಿದಾನಂದಮೂರ್ತಿಯವರು ನಿಮ್ಮ ಪ್ರಬಂಧವನ್ನು ನೋಡಿ, ತಮ್ಮ ಅಭಿಪ್ರಾಯವನ್ನು ಬರೆದಿಟ್ಟಿದ್ದಾರೆ, ಹೋಗಿ ತೆಗೆದುಕೊಂಡು ಬನ್ನಿ ಎಂದರು. ತಮ್ಮ ಲೆಟರ್​ಹೆಡ್​ನಲ್ಲಿ ಕೈಬರಹದಲ್ಲೇ ಅವರ ಅಭಿಪ್ರಾಯವನ್ನು ಬರೆದಿಟ್ಟಿದ್ದರು.  ಪರವಾಗಿಲ್ಲ, ಆದರೆ ಪ್ರಬಂಧ ಪ್ರಕಟವಾಗಬೇಕು, ಪ್ರಕಟವಾಗದಿದ್ದರೆ ಏನೂ ಪ್ರಯೋಜನವಿಲ್ಲದಂತಾಗುತ್ತದೆ ಎಂದು ಪ್ರೋತ್ಸಾಹದ ಮಾತುಗಳನ್ನು ಆಡಿದರು. ಬೆನ್ನುತಟ್ಟಿದರು. ಅವರು ನನ್ನ ಪ್ರಬಂಧಕ್ಕೆ ಬರೆದ ಅಭಿಪ್ರಾಯವನ್ನು ಈ ಕೃತಿಯಲ್ಲಿ ಅಚ್ಚು ಹಾಕಿಸಿದ್ದೇನೆ.

ಪ್ರಬಂಧ ರಚನೆಯ ನಂತರವೂ ಮೂರು ನಾಲ್ಕು ವರ್ಷಗಳ ನಾನು ಶಾಸನಗಳ ಬಗ್ಗೆ ಹೆಚ್ಚಿನ ಅಧ್ಯಯನ, ಸಂಶೋಧನೆ ಕೈಗೊಂಡೆ. ಕ್ಷೇತ್ರ ಕಾರ್ಯ ನಡೆಸಿದೆ. ಇದರಿಂದ ನನಗೆ ಅನೇಕ ಹೊಸ ವಿಷಯಗಳು ತಿಳಿದು ಬಂದವು. ಕೆಲವು ಕಡೆ ನನ್ನ ಅಭಿಪ್ರಾಯಗಳನ್ನು ಬದಲಾಯಿಸಬೇಕಾಗಿ ಬಂದಿತು. ಈ ನನ್ನ ಮುಂದುವರಿದ ಸಂಶೋಧನೆಯ ಫಲಶ್ರುತಿಯೇ,"ಮಂಡ್ಯ ಜಿಲ್ಲೆಯ ಶಾಸನ ಮತು ಸಂಸ್ಕೃತಿ" ಎಂಬ ಈ ನನ್ನ ಕೃತಿಯಾಗಿದೆ.

ನನ್ನ ಕೃತಿಯೇನೋ ಸಿದ್ಧವಾಯಿತು. ಆದರೆ ಪ್ರಕಟಣೆ ಸಾಧ್ಯವಿಲ್ಲದ ಮಾತು. ಇಂತಹ ಕೃತಿಗಳನ್ನು ಯಾರೂ ಪ್ರಕಟಿ\-ಸಲು ಮುಂದೆ ಬರುವುದಿಲ್ಲ. ಹೀಗಾಗಿ ಐದಾರು ವರ್ಷ ಸುಮ್ಮನಿದ್ದೆ.  ಕಾಮಧೇನು ಪ್ರಕಾಶನದ ಶ್ರೀ ಶಾಮಸುಂದರ ರಾವ್​ ಅವರು ನನಗೆ ಪರಿಚಿತರು.  ಸರ್ಕಾರಿ ಪದವಿ ಕಾಲೇಜಿನಲ್ಲಿ ಕನ್ನಡ ರೀಡರ್​ ಆಗಿದ್ದ ಅವರು ಸರಳರು, ಸಜ್ಜನರು. ಶಾಸಕರ ಭವನದಲ್ಲಿ ಪುಸ್ತಕದ ಅಂಗಡಿಯನ್ನು ತೆರೆದಾಗಿನಿಂದ ಅವರು ನನಗೆ ಪರಿಚಿತರಾದರು. ಸಾಧ್ಯ\-ವಾದಾಗಲೆಲ್ಲಾ ಅವರ ಪುಸ್ತಕಾಲ\-ಯಕ್ಕೆ ಹೋಗುತ್ತಿದ್ದೆ. ಅನೇಕ ಪುಸ್ತಕಗಳನ್ನು ಖರೀದಿಸುತ್ತಿದ್ದೆ. ಅವರು ಅನೇಕ ಮೌಲ್ಯಯುತ, ಸತ್ವಯುತ ಪ್ರಾಚೀನ, ಅರ್ವಾಚೀನ ಕೃತಿಗಳನ್ನು ತಮ್ಮ ಪ್ರಕಾಶನದ ಮೂಲಕ ಪ್ರಕಟಿಸಿ ಪ್ರಖ್ಯಾತರಾಗಿದ್ದಾರೆ. ಲಾಭ\-ನಷ್ಟದ ಲೆಕ್ಕಾಚಾರವಿಲ್ಲದೆ ಕನ್ನಡ ಕೃತಿಗಳು ಪ್ರಕಟವಾಗಬೇಕು, ಓದುಗರ ಕೈಸೇರಬೇಕು ಎಂಬ ಒಂದೇ ಉದ್ದೇಶದಿಂದ ಅವರು ಪುಸ್ತಕ ಪ್ರಕಟಣೆಯನ್ನು ಮಾಡುತ್ತಿದ್ದಾರೆ.  ಒಂದು ವರ್ಷದ ಹಿಂದೆ ಇರಬಹುದು.  ಒಂದು ದಿನ ಅವರು ನಿಮ್ಮ ಪ್ರಬಂಧ ಪ್ರಕಟಿಸು\-ತ್ತೀರಾ, ಯಾರಿಗಾದರೂ ಕೊಟ್ಟಿದ್ದೀರಾ, ಎಂದು ಕೇಳಿದರು. ಇಲ್ಲ ಅದನ್ನು ಯಾರು ಪ್ರಕಟಿಸುತ್ತಾರೆ, ಬಹಳ ದೊಡ್ಡದು ಎಂದೆ. ನಮ್ಮ ಕಡೆಯವರ ಒಂದು ಪ್ರಕಾಶನ ಸಂಸ್ಥೆ ಇದೆ. ಅದರ ಮೂಲಕ ಪ್ರಕಟಿಸೋಣ, ಕೃತಿ ರೂಪದಲ್ಲಿ ಬರೆದು ಸಿದ್ಧಪಡಿಸಿಕೊಡಿ ಎಂದರು. ಅವರ ಮಾತನ್ನು ಕೇಳಿ ನಾನು ಆಶ್ಚರ್ಯಚಕಿತನಾದೆ. ಪ್ರಖ್ಯಾತ ಲೇಖಕರು, ವಿದ್ವಾಂಸರನ್ನು ಹೊರತುಪಡಿಸಿ, ನನ್ನಂತಹ ಅಜ್ಞಾತ ಲೇಖಕರ, ಅದರಲ್ಲೂ ಬೇಡಿಕೆ ಇಲ್ಲದ ಸಂಶೋಧನಾ ಕೃತಿಯನ್ನು ಪ್ರಕಟಿಸಲು ಕೇಳಿದರೆ, ಮುಂದೆ ಬರದಿರುವ, ಬಂದರೂ ನಾನಾ ರೀತಿಯ ಕಂಡೀಷನ್​ಗಳನ್ನು ಹಾಕುವ ಪ್ರಕಾಶಕರು ಇರುವಾಗ, ಇವರು ಇಷ್ಟೊಂದು ಸುಲಲಿತವಾಗಿ ಈ ರೀತಿ ಹೇಳಿದ್ದನ್ನು ನಂಬಲು ಸಾಧ್ಯವಾಗಲೇ ಇಲ್ಲ. ಮೂಕನಾಗಿ\break ಆಯಿತು ಎಂದೆ. ನಮ್ಮ ಗುರುಗಳಾದ ಡಾ. ಕೆ.ಆರ್​.ಗಣೇಶ್​ ಅವರು ನನ್ನ ಕೃತಿಯನ್ನು ಮತ್ತೊಮ್ಮೆ ಆಮೂಲಾಗ್ರವಾಗಿ ನೋಡಿ ಸಲಹೆ ಸೂಚನೆಗಳನ್ನು ನೀಡಿದರು. ಮತ್ತೆ ಅನೇಕ ಅಧ್ಯಾಯಗಳನ್ನು ಹೊಸದಾಗಿ ರಚಿಸಿದೆ. ಕೆಲವು ಅಧ್ಯಾಯಗಳನ್ನು, ಅನು\-ಬಂಧಗಳನ್ನು ಪೂರ್ಣವಾಗಿ ಕೈಬಿಟ್ಟೆ, ಸಂಶೋಧನೆಯ ನಂತರದಲ್ಲಿ  ಅಧ್ಯಯನ ಹಾಗೂ ಕ್ಷೇತ್ರಕಾರ್ಯದ ಮೂಲಕ  ನನ್ನ ಗಮನಕ್ಕೆ ಬಂದ ಅನೇಕ ಹೊಸ ವಿಷಯಗಳನ್ನು, ಛಾಯಾಚಿತ್ರಗಳನ್ನು ಸೇರಿಸಿದೆ, ಹೀಗೆ ನನ್ನ ಪ್ರಬಂಧವು ಹೊಸ ರೂಪು ತಳೆಯಿತು. ಮೂಲ ಪ್ರಬಂಧಕ್ಕಿಂತ ಈ ನನ್ನ ಕೃತಿಯು ಪೂರ್ಣವಾಗಿ ವಿಭಿನ್ನವಾಗಿಯೇ ನಿಲ್ಲುತ್ತದೆ.

ವಿದ್ವಾಂಸರ ಜೊತೆಗೆ ಜನಸಾಮಾನ್ಯರನ್ನು ದೃಷ್ಟಿಯಲ್ಲಿಟ್ಟುಕೊಂಡು ಹೊಸದಾಗಿ ಬರೆದು ಸಿದ್ಧಪಡಿಸಿದ ಈ ಕೃತಿಯಲ್ಲಿ ಅಡಿಟಿಪ್ಪಟಣಿಗಳನ್ನು ಉಳಿಸಿಕೊಳ್ಳಬೇಕೋ, ಬಿಡಬೇಕೋ ಎಂಬ ಜಿಜ್ಞಾಸೆ ಶುರುವಾಯಿತು. ಕೊನೆಗೆ ಸಂಶೋಧನಾತ್ಮಕ ಕೃತಿ\-ಯಾಗಿದ್ದರಿಂದ ಅವುಗಳನ್ನು ಹಾಗೇ ಉಳಿಸಿಕೊಂಡು, ಆಯಾ ಅಧ್ಯಾಯಗಳ ಕೊನೆಗೆ ತಂದು ಜೋಡಿಸಿದೆ. ಅದರಿಂದ ಕೃತಿ ಮತ್ತೆ ದೊಡ್ಡದಾಗಿ ಈಗಿನ ಸ್ವರೂಪ ಪಡೆಯಿತು. ಜಿಲ್ಲೆಯ ಒಳನಾಡಿನಲ್ಲಿರುವ ನೂರಾರು ಹಳ್ಳಿಗಳಿಗೆ ಹೋಗಿ ತೆಗೆದಿರುವ ಅಪರೂಪದ ಸ್ಮಾರಕಗಳ ಸುಮಾರು 300ಕ್ಕೂ ಛಾಯಾಚಿತ್ರಗಳನ್ನು ಈ ಕೃತಿಯಲ್ಲಿ ಅಳವಡಿಸಿದ್ದೇನೆ.

ಜನಸಾಮಾನ್ಯರನ್ನು ದೃಷ್ಟಿಯಲ್ಲಿಟ್ಟುಕೊಂಡು ಬರೆದಿರುವುದರಿಂದ ಕೆಲವೆಡೆ ನಿರೂಪಣೆ ಹೆಚ್ಚಾಯಿತು ಎಂದೆನಿಸು\-ತ್ತದೆ. ಕೆಲವು ಕಡೆ ಪುನರಾವರ್ತನೆಯಾಗಿರುವಂತೆ ತೋರುತ್ತದೆ.   ಶಾಸನಗಳನ್ನು ಆಧರಿಸಿ ಕೃತಿ ರಚನೆ ಮಾಡುವಾಗ ಕೆಲವು ಕಡೆ ಈ ರೀತಿ ವಿಷಯಗಳ ಪುನರಾವರ್ತನೆ ಆಗೇ ಆಗುತ್ತದೆ. ಇದರಿಂದ ಆಗುವ ಪ್ರಯೋಜನ ಎಂದರೆ ಪ್ರತಿ ಅಧ್ಯಾಯಗಳನ್ನು ಓದುವಾಗ  ಅವು ಪ್ರತ್ಯೇಕವಾಗಿಯೇ ನಿಲ್ಲುತ್ತವೆ. ಇಂತಹ ಕೃತಿಗಳ ಅಧ್ಯಯನದಿಂದ ಜನಸಾಮಾನ್ಯರಿಗೆ ಶಾಸನಗಳು ಹಾಗೂ ಸ್ಮಾರಕಗಳ ಬಗ್ಗೆ, ತಮ್ಮ ಊರಿನ ಬಗ್ಗೆ ಒಂದು ರೀತಿಯ ಅಭಿಮಾನ ಮೂಡಬೇಕು, ಅದರಿಂದ ಶಾಸನಗಳು ಮತ್ತು ಸ್ಮಾರಕಗಳ ರಕ್ಷಣೆಗೆ ದಾರಿಯಾಗುತ್ತದೆ. ತಮ್ಮ ಊರು ಇಷ್ಟೊಂದು ಪ್ರಾಚೀನವಾದುದು, ಐತಿಹಾಸಿಕವಾಗಿ ಇಷ್ಟೊಂದು ಮುಖ್ಯವಾದುದು ಎಂಬ ಅಭಿಮಾನ ಮೂಡುತ್ತದೆ.  ಇಂತಹ ಒಂದು ಮನೋಭಾವನೆ ಇಂದು ನಮ್ಮ ಯುವಕರಲ್ಲಿ, ಜನಸಾಮಾನ್ಯರಲ್ಲಿ ಮೂಡು\-ತ್ತಿದೆ. ನನ್ನ ಈ ಒಂದು ಕೃತಿಯಿಂದ ಅದಕ್ಕೆ ಇಂಬು ದೊರೆತರೆ ಅದೇ ನನಗೆ ತೃಪ್ತಿ. ಅಲ್ಲಲ್ಲಿ ಶಾಸನೋಕ್ತ ಪದ್ಯಗಳನ್ನು, ಶಾಸನದ ಪದಗಳನ್ನು, ವಾಕ್ಯಗಳನ್ನು ಹಾಗೆಯೇ ಉಳಿಸಿಕೊಂಡು ಉಲ್ಲೇಖಿಸಿದ್ದೇನೆ. ಇದರಿಂದ ನಿರೂಪಣೆಗೆ ಖಚಿತತೆ ಬಂದಿದೆ.  ಕನ್ನಡ ಸಾಹಿತ್ಯ ಪರಿಷತ್ತಿನಲ್ಲಿ,  ಶಾಸನ ಶಾಸ್ತ್ರ ತರಗತಿಯಲ್ಲಿ ಅತ್ಯುತ್ತಮ ರೀತಿಯಲ್ಲಿ ಬೋಧನೆ ಮಾಡಿ, ಈ ಕ್ಷೇತ್ರದ ಕಡೆಗೆ \hbox{ನಮ್ಮನ್ನು} ಸಂಪೂರ್ಣವಾಗಿ ಆಕರ್ಷಿಸಿದವರು, ಶಾಸನಗಳ ಸಂಶೋಧನೆಯ ಮೂಲಪಾಠಗಳನ್ನು ಕಲಿಸಿದವರು, ಕರ್ನಾಟಕ ಶಾಸನ ಅಧ್ಯಯನ ಕ್ಷೇತ್ರದಲ್ಲಿ ಅತ್ಯಂತ ಪ್ರಸಿದ್ಧರಾದ  ಡಾ.ಕೆ.ಆರ್​.ಗಣೇಶ್​, ಡಾ.ದೇವರ ಕೊಂಡಾರೆಡ್ಡಿ, ಡಾ. ಪಿ.ವಿ. ಕೃಷ್ಣಮೂರ್ತಿ,\break ಡಾ.ಹೆಚ್​.ಎಸ್​. ಗೋಪಾಲರಾವ್​, ಡಾ.ಎಲ್​.ಎಸ್​. ಶ್ರೀನಿವಾಸಮೂರ್ತಿ, ಪ್ರೊ. ಲಕ್ಷ್ಮಣತೆಲಗಾವಿ, ಡಾ. ಎಂ. ಜಿ. \hbox{ನಾಗರಾಜ್} ಈ ಗುರುಗಳ ಸಮೂಹ. ಇವರಿಗೆ ನಾನು ಅತ್ಯಂತ ವಿನೀತನಾಗಿ ನನ್ನ ವಂದನೆಗಳನ್ನು ಸಲ್ಲಿಸುತ್ತೇನೆ. ನಮ್ಮ ಊರಿನವರು, ನನ್ನ ಸಹಪಾಠಿಗಳೂ ಮಿತ್ರರೂ ಆದ, ನಿವೃತ್ತ ಇತಿಹಾಸ ಪ್ರಾಧ್ಯಾಪಕ ಡಾ. ಎಸ್​.ಎನ್​. ಶಿವರುದ್ರಸ್ವಾಮಿಯರೂ, ಈ ಕ್ಷೇತ್ರದಲ್ಲಿ ನಾನು ಅಧ್ಯಯನ ಮಾಡಿ ಬರವಣಿಗೆ ಮಾಡಲು ಕಾರಣಕರ್ತರು. ಕರ್ನಾಟಕದ ಪ್ರಖ್ಯಾತ ಗಮಕಿಗಳು, ಹಿರಿಯ ಅಧಿಕಾರಿಗಳೂ ಆದ ಡಾ. ಎ.ವಿ. ಪ್ರಸನ್ನ ಅವರು ನನಗೆ ಪರಿಚಿತರು. ಅವರಿಗೂ ನನ್ನ ಪಿಎಚ್​.ಡಿ. ಅಧ್ಯಯನದ ಬಗ್ಗೆ ಹೇಳಿದ್ದೆ. ನನ್ನ ಪ್ರಬಂಧವನ್ನು ಅವರ ಪರಾಮರ್ಶೆಗೆ ನೀಡಿದ್ದೆ. ಈಗಲೂ ಹೊಸದಾಗಿ ರಚಿತವಾದ ಈ ಕೃತಿಯನ್ನು ಅವರೂ ಪರಿಶೀಲಿಸಿ ಪ್ರೋತ್ಸಾಹದಾಯಕ ಮಾತುಗಳನ್ನು ಆಡಿ, ತಮ್ಮ ಅಭಿಪ್ರಾಯವನ್ನು ನೀಡಿದ್ದಾರೆ. ಅವರಿಗೂ ನಾನು ಆಭಾರಿಯಾಗಿದ್ದೇನೆ. ನಮ್ಮ ಗುರುಗಳಾದ ಡಾ.ಕೆ.ಆರ್​.ಗಣೇಶ್​ ಅವರು ಕೇಳಿದ ತಕ್ಷಣ ತಮ್ಮ ಮೆಚ್ಚುಗೆಯ ನುಡಿಯನ್ನು ಬರೆದು\-ಕೊಟ್ಟರು. ಅವರೆಲ್ಲರಿಗೂ ನಮ್ರತಾಪೂರ್ವಕ ವಂದನೆಗಳು.

ನನ್ನ ಈ ಒಂದು ಸಂಶೋಧನಾತ್ಮಕ ಕೃತಿಯನ್ನು ಬಹಳ ರಿಸ್ಕ್​ ತೆಗೆದುಕೊಂಡು ಪ್ರಕಟಿಸುತ್ತಿರುವ ಸುಮೇರು ಸಾಹಿತ್ಯದ ಶ್ರೀಮತಿ ಸುಮಿತ್ರಾ ವಿ. ದರ್ಶನ್​ ಅವರಿಗೆ ನಾನು ಅತ್ಯಂತ ಕೃತಜ್ಞನಾಗಿದ್ದೇನೆ. ಸುಮೇರು ಸಾಹಿತ್ಯದ ಮೂಲಕ ಅವರು ಇಂತಹ ಅನೇಕ ಮೌಲ್ಯಯುತವಾದ ಮೇರು ಕೃತಿಗಳನ್ನು ಪ್ರಕಟಿಸಿದ್ದಾರೆ.  ಅವರಿಗೆ ನನ್ನನ್ನು ಪರಿಚಯಿಸಿ, ನನ್ನ ಕೃತಿಯು ಪ್ರಕಟ\-ಗೊಳ್ಳಲು ಅನುವು ಮಾಡಿಕೊಟ್ಟ ಶ್ರೀ ಶಾಮಸುಂದರ ರಾಯರಿಗೆ, ಶ್ರೀಮತಿ ಮುದ್ದಮ್ಮ ಶಾಮಸುಂದರರಾಯರಿಗೆ ನಾನು ಅತ್ಯಂತ ಆಭಾರಿಯಾಗಿದ್ದೇನೆ. ಅವರು ಈ ಕೃತಿಯನ್ನು ಸಿದ್ಧಪಡಿಸಿಕೊಡಬೇಕೆಂದು ಹೇಳಿದ ಕೊನೆಯ ಗಡುವುಗಿಂತ ಆರು ತಿಂಗಳ ನಂತರ ನಾನು ಈ ಕೃತಿಯನ್ನು ನೀಡಿದರೂ, ಬೇಸರ ಪಟ್ಟುಕೊಳ್ಳದೆ ಪ್ರಕಟಣೆ ಮಾಡಿದ್ದಾರೆ.

ಶ್ರೀರಂಗಪಟ್ಟಣದ ಶ್ರೀರಂಗ ಡಿಜಿಟಲ್ಸ್​ನ ಮಾರ್ಗದರ್ಶಕರಾದ ಪ್ರೊ. ಡಾ. ಸಿ.ಎಸ್​.ಯೋಗಾನಂದ್​ ಹಾಗೂ ವ್ಯವಸ್ಥಾಪಕರಾದ ಶ್ರೀ ಅರ್ಜುನ್​ ಅವರು. ಈಗಲೂ ಅವರು ಬದಲಾದ, ಕ್ಲಿಷ್ಟಕರವಾದ ಈ ಕೃತಿಯನ್ನು, ತಮ್ಮ ಬಿಡುವಿಲ್ಲದ ಕಾರ್ಯಭಾರದ ನಡುವೆಯೂ, ಪ್ರೀತಿಯಿಂದ ಬಹಳ ಸುಂದರವಾಗಿ ಟೈಪ್​ಸೆಟ್ಟಿಂಗ್​ ಮಾಡಿಸಿಕೊಟ್ಟಿದ್ದಾರೆ. ಅವರಿಗೆ ನಾನು ಎಷ್ಟು ಆಭಾರಿ\-ಯಾಗಿದ್ದರೂ ಸಾಲದು. ಅವರ ಸಂಸ್ಥೆಯಲ್ಲಿ ಈ ಕಾರ್ಯಭಾರವನ್ನು ಅವರ ನಿರ್ದೇಶನದಂತೆ ಸುಂದರವಾಗಿ ಮಾಡಿಕೊಟ್ಟ,  ಶ್ರೀ ಶಿವಶಂಕರ್​, ಶ್ರೀ ರಾಘವೇಂದ್ರ ಮತ್ತು ಅವರ ಸಹೋದ್ಯೋಗಿಗಳಿಗೆ ನಾನು ಚಿರಋಣಿಯಾಗಿದ್ದೇನೆ. ಈ ಕೃತಿಯ ಛಾಯಾ ಚಿತ್ರಗಳನ್ನು ಜೋಡಿಸಿ ಲೇಔಟ್​ ಮಾಡಿಕೊಟ್ಟ, ಮಂಜುಶ್ರೀ ಎಂಟರ್​\-ಪ್ರೈಸಸ್​ನ ಬಿ.ವಿ. ಗೋಪಾಲಕೃಷ್ಣ ಅವರಿಗೆ ನನ್ನ ವಂದನೆಗಳು.  ಇಂತಹ ಕೃತಿಯನ್ನು ಸುಂದರವಾಗಿ ಮುದ್ರಣ ಮಾಡುವುದೂ ತಾಂತ್ರಿಕವಾಗಿ  ಒಂದು ಮಹತ್ವದ\break ಕೆಲಸ.  ಸುಮಾರು 25{\rm -}30 ವರ್ಷಗಳಿಂದ ನನಗೆ ಪರಿಚಿತರಾದ, ಸಜ್ಜನರಾದ, ಹೆಸರಾಂತ ಪರಿಮಳ ಮುದ್ರಣಾಲಯದ ಮಾಲೀಕ\-ರಾದ ಶ್ರೀ ಕೆ.ಸಿ. ಪ್ರಭಾಕರ್​ ಅವರು ಅಲ್ಪ ಅವಧಿಯಲ್ಲಿ ಬಹಳ ಸುಂದರವಾಗಿ ಈ ಕೃತಿಯನ್ನು ಮುದ್ರಿಸಿ ಕೊಟ್ಟಿದ್ದಾರೆ. ಅವರಿಗೆ ಅವರ ಮುದ್ರಣಾಯಲದ ಶ್ರೀಕಾಂತ್​ ಅವರಿಗೆ ಹಾಗೂ ಅವರ ಸಹೋದ್ಯೋಗಿ\-ಗಳಿಗೆ ನಾನು ಆಭಾರಿಯಾಗಿದ್ದೇನೆ. ಈ ಕೃತಿಯನ್ನು ಆಮೂಲಾಗ್ರವಾಗಿ ಓದಿ ಕರಡಚ್ಚನ್ನು ತಿದ್ದಿಕೊಟ್ಟ ನನ್ನ ಆತ್ಮೀಯ ಮಿತ್ರರಾದ ಶಾಸನಶಾಸ್ತ್ರ ಮತ್ತು ಎಂ.ಫಿಲ್. ಸಹಪಾಠಿಗಳಾದ ಡಾ. ಕೆ.ವಿ. ಅನಂತಪದ್ಮನಾಭ ಎಂ.ಎ., ಎಂ.ಫಿಲ್., ಪಿಎಚ್.ಡಿ., ಇವರಿಗೆ ನಾನು ಕೃತಜ್ಞನಾಗಿದ್ದೇನೆ. ನನ್ನ ಬರವಣಿಗೆಯನ್ನು ಪ್ರೋತ್ಸಾಹಿಸುತ್ತಿರುವ ಪ್ರಖ್ಯಾತ ಕನ್ನಡ ನವೋದಯ ಕವಿಗಳು, ಕರ್ನಾಟಕ ವಿಧಾನ ಪರಿಷತ್ತಿನ ಮಾಜಿ ಸದಸ್ಯರು ಆದ ಪದ್ಮಶ್ರೀ ಡಾ. ದೊಡ್ಡರಂಗೇಗೌಡರವರಿಗೆ, ಖ್ಯಾತ ಕನ್ನಡಪರ ಹೋರಾಟಗಾರ ರಾ. ನಂ. ಚಂದ್ರಶೇಖರ್‌ರವರಿಗೆ ನಾನು ಆಭಾರಿಯಾಗಿದ್ದೇನೆ.

ನಮ್ಮನ್ನು ಸಾಕಿ ಬೆಳೆಸಿ, ನಮ್ಮ ತಾಲ್ಲೂಕಿನ ಹಳ್ಳಿಗಳನ್ನು ಕಾಲ್ನಡಿಗೆಯಲ್ಲಿ ಸುತ್ತಿಸಿ ನನಗೆ ಇತಿಹಾಸ ಮತ್ತು ಭೂಗೋಳ\break ಜ್ಞಾನವನ್ನು ಕಲಿಸಿದ ನಮ್ಮ ತಂದೆ ಸಂತೇ ಬಾಚಹಳ್ಳಿಯ ಮಾಜಿ ಚೇರ್ಮೆನ್, ಮಾಜಿ ಪೊಸ್ಟ್ ಮಾಸ್ಟರ್ ದಿವಗಂತ ಎಸ್.\break ಸುಬ್ಬರಾಯರ ದಿವ್ಯ ಸ್ಮರಣೆಗಳು.

ಮನೆಯ ಕಡೆ ಎಲ್ಲಾ ಜವಾಬ್ದಾರಿಯನ್ನೂ ಹೊತ್ತುಕೊಂಡು, ಯಾವಾಗಲೂ ನನ್ನ ಓದುಬರಹಕ್ಕೆ ಒತ್ತಾಸೆಯಾಗಿರುವ ನನ್ನ ಪತ್ನಿ ಶ್ರೀಮತಿ ರಾಜೇಶ್ವರಿ, ನನ್ನ ಮಕ್ಕಳು, ಚಂದ್ರಮೌಳಿ, ಯಶಸ್ವಿನಿ, ನನ್ನ ಸೊಸೆ ಸುಮನಾ ಇವರಿಗೆ ನನ್ನ ಹೃತ್ಪೂರ್ವಕ ಪ್ರೀತಿ ಭಾವನೆಯನ್ನು ವ್ಯಕ್ತಪಡಿಸುತ್ತೇನೆ. ನನ್ನನ್ನು ಹಳ್ಳಿಗಳಿಗೆ ಬೈಕ್​ ಮೇಲೆ ಕರೆದುಕೊಂಡು ಹೋಗಿದ್ದ ನನ್ನ ತಮ್ಮನ ಮಕ್ಕಳಾದ  ರೋಹಿತ್​ ಶ್ರೀಪತಿ, ಶ್ರೀಕಾಂತ್​ ಸುಬ್ರಹ್ಮಣ್ಯ, ಶಶಾಂಕ್ ರವಿಶಂಕರ್, ಎಸ್. ಎನ್. ವಿಶ್ವನಾಥ ನಾಗರಾಜ್ ಮಂಡ್ಯದ ಮಿತ್ರರಾದ ಶ್ರೀನಿವಾಸಮೂರ್ತಿ ಹಾಗೂ ಹಿರಿಯರಾದ ಬೆಳ್ಳೂರಿನ ಶ್ರೀ ಲಕ್ಷ್ಮೀನಾರಾಯಣ ಅವರಿಗೆ ನನ್ನ ವಂದನೆಗಳು. ವೃತ್ತಿ ಜೀವನದಲ್ಲಿ ಮತ್ತು ವೈಯುಕ್ತಿಕ ಜೀವನದಲ್ಲಿ ನನ್ನ ಹಿತೈಷಿಗಳಾಗಿರುವ, ಆತ್ಮೀಯರಾದ ಕರ್ನಾಟಕ ಸರ್ಕಾರದ ನಿವೃತ್ತ ಅಪರಕಾರ್ಯದರ್ಶಿ\-ಗಳೂ ಆದ ಶ್ರೀ ಶಿವಕುಮಾರ್ ಹಾಗೂ ಅವರ ಕುಟುಂಬವರ್ಗದವರಿಗೆ ನಾನು ಆಭಾರಿಯಾಗಿದ್ದೇನೆ.

ಸುಮಾರು 38 ವರ್ಷಗಳ ಕಾಲ ಸರ್ಕಾರಿ ನೌಕರನಾಗಿದ್ದು, ಶಾಸನಗಳ ಅಧ್ಯಯನ ಮತ್ತು ಸಂಶೋಧನಾ ಕ್ಷೇತ್ರಕ್ಕೆ, ಕೃತಿ ರಚನೆಗೆ ತೀರಾ ಹೊಸಬನಾದ ನಾನು, ನಮ್ಮ ಗುರುಗಳ ಬೋಧನೆ ಮತ್ತು ಮಾರ್ಗದರ್ಶನದಿಂದ, ಈ ವಿಷಯದಲ್ಲಿ ಅಧ್ಯಯನ ಮಾಡಿ, ನಮ್ಮ ಮಂಡ್ಯ ಜಿಲ್ಲೆಯ ಮೇಲಿನ ಅಭಿಮಾನದಿಂದ, ಒಂದು ಕೃತಿಯನ್ನು ರಚಿಸಿದ್ದೇನೆ. ಈ ಕ್ಷೇತ್ರದಲ್ಲಿ ವಿದ್ವಾಂಸ\-ರಾಗಿರುವ ಮಹನೀಯರಿಗೆ, ಈ ಕೃತಿಯಲ್ಲಿ ಅನೇಕ ಲೋಪದೋಷಗಳು ಕಾಣಬಹುದು. ದಯಮಾಡಿ ಅದನ್ನು ಮನ್ನಿಸಿ ಈ ಕೃತಿಯನ್ನು ಪರಾಂಬರಿಸಿ ಮಾರ್ಗದರ್ಶನ ಮಾಡಬೇಕೆಂದು ವಿನಂತಿಸುತ್ತೇನೆ.  ಸಾಹಿತ್ಯ, ಇತಿಹಾಸ ಮತ್ತು ಸಂಸ್ಕೃತಿ ಪ್ರಿಯರಾದ ಕನ್ನಡ ನಾಡಿನ ಜನರು,  ಮಂಡ್ಯ ಜಿಲ್ಲೆಯ ಜನರು, ಇತಿಹಾಸ ಮತ್ತು ಸಂಸ್ಕೃತಿಯ ಕ್ಷೇತ್ರದ ವಿದ್ವಾಂಸರು, ವಿದ್ಯಾರ್ಥಿಗಳು, ಈ ನನ್ನ ಕೃತಿಯನ್ನು ಆದರದಿಂದ ಬರಮಾಡಿಕೊಳ್ಳುವರೆಂಬ ವಿಶ್ವಾಸವನ್ನು ವ್ಯಕ್ತಪಡಿಸುತ್ತೇನೆ.

\bigskip

\begin{flushright}
\textbf{ಸಂತೇಬಾಚಹಳ್ಳಿ ಡಾ. ಎಸ್​. ನಂಜುಂಡಸ್ವಾಮಿ.}
\end{flushright}

