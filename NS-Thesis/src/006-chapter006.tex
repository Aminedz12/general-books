\chapter{ಕೃಷಿ ಸಂಸ್ಕೃತಿ}

\vskip 4pt

ಪ್ರಪಂಚದ ಪುರಾತನ ನಾಗರಿಕತೆಗಳಲ್ಲಿ ಭಾರತದ ಸಿಂಧೂಕಣಿವೆಯ ನಾಗರಿಕತೆ ಬಹಳ ಪ್ರಮುಖವಾದುದು. “ಈ ನಾಗರಿಕತೆಯ ಜನರ ಮುಖ್ಯ ಕಸುಬು ಕೃಷಿಯಾಗಿದ್ದಿತು. ಭೂಮಿಯ ಫಲವತ್ತು ಮತ್ತು ನೀರಾವರಿ ಸೌಲಭ್ಯ ಕೃಷಿಯನ್ನು ಲಾಭದಾಯಕ\break ಉದ್ಯಮವನ್ನಾಗಿ ಮಾಡಿದ್ದಿತು”.\endnote{ ರಾಮಲಿಂಗಪ್ಪ, ಎಚ್​., ನಾಗರಿಕತೆಗಳ ಸಮೀಕ್ಷೆ, ಪುಟ 109} ಕೃಷಿ ಮತ್ತು ಪಶುಪಾಲನೆ ಮುಖ್ಯ ವೃತ್ತಿಗಳಾಗಿದ್ದವು. ಜನರು ಸಾಮಾನ್ಯವಾಗಿ ತಮ್ಮ\break ಜಮೀನುಗಳ ಹತ್ತಿರ ಮನೆಮಠಗಳನ್ನು ಕಟ್ಟಿಕೊಂಡು ವಾಸ ಮಾಡುತ್ತಿದ್ದರು. ಕೃಷಿಗೆ ಬಳಸುತ್ತಿದ್ದ ಭೂಮಿಗೆ “ಕ್ಷೇತ್ರ” ಅಥವಾ “ಉರ್ವರ” ಎಂದು ಹೆಸರಿದ್ದಿತು. ನೀರಾವರಿಯ ಪ್ರಯೋಜನ ಇವರಿಗೆ ತಿಳಿದಿದ್ದಿತು”\endnote{ ರಾಮಲಿಂಗಪ್ಪ, ಎಚ್​., ಪೂರ್ವೋಕ್ತ, ಪುಟ 119-120}. “ರಾಷ್ಟ್ರಕೂಟ ಅಮೋಘವರ್ಷನ ಶಾಸನಗಳಲ್ಲಿ ನೇಗಿಲಿನ ಗುಳದ ಕೆತ್ತನೆಯು ಸಾಧಾರಣವಾಗಿ ಕಾಣಬರುವ ರಾಜಚಿಹ್ನೆಯಾಗಿದೆ, ಮಾವಳಿ ಶಾಸನದಲ್ಲಿ ನೇಗಿಲಿನ ಚಿತ್ರವನ್ನೇ ಬಿಡಿಸಲಾಗಿದೆ, ಇದು ಕೃಷಿಗೆ ಸಲ್ಲುತ್ತಿದ್ದ ಪೂಜ್ಯತೆಯನ್ನು ಕುರುಹು” ಎಂಬ ಉಲ್ಲೇಖಗಳನ್ನು ಗಮನಿಸಬಹುದು\endnote{ ಸ್ವಾಮಿ ಡಾ॥ ಬಿ.ಜಿ.ಎಲ್​., ಶಾಸನಗಳಲ್ಲಿ ಗಿಡಮರಗಳು, ಪುಟ 2}.

ಪ್ರಾಚೀನ ಕಾಲದಂತೆ ಇಂದಿಗೂ ಮಂಡ್ಯ ಜಿಲ್ಲೆಯ ಜನರ ಮುಖ್ಯ ಕಸುಬು ಕೃಷಿಯಾಗಿದೆ. ಇವತ್ತೂ ಹೆಚ್ಚಿನ ಪ್ರಮಾಣದ ರೈತರು ಪರಂಪರಾಗತವಾದ ರೀತಿಯಲ್ಲಿ ಬೇಸಾಯ ಮಾಡುತ್ತಿದ್ದಾರೆ. ಕೃಷಿ ವ್ಯವಸಾಯ ಎಂಬ ಶಬ್ದಗಳು ಹೊಯ್ಸಳರ ಕಾಲದ ಶಾಸನಗಳಲ್ಲಿ ಬಳಕೆಯಾಗಿವೆ. \textbf{“ಆ ತೋಟಂಗಳ ಕ್ರಿಷಿಬ್ಯವಸಾಯ\index{ಕ್ರಿಷಿಬ್ಯವಸಾಯ} ಏತ ನಿರಧ್ವಾನವೊಳಗಾದ ಸಮಸ್ತ ಭ್ಯವಸಯವನೂ ತಾವೇ ಮಾಡುವರು”, “ಆ ತೋಂಟದ ಫಲಂಗಳು ಅದಿಕ್ರಯದಾನಕ್ಕೆ ಸಲೂಉದು ಕ್ರಿಸಿ ಬ್ಯವಸಾಯವ ಮಾಡದಿದ್ದರಾದಡೆ ಅವರ ಪಡಿ ಜೀವಿತದಲೇ ಯಿಕ್ಕಿ ಮಾಡಿಸುವರು”} ಎಂದು ಹಾರನಹಳ್ಳಿ ಶಾಸನದಲ್ಲಿ ಹೇಳಿದೆ.\endnote{ ಎಕ 10 ಅರಸೀಕೆರೆ 233 ಹಾರನಹಳ್ಳಿ 1265}\textbf{ಕೃಷಿ ಕ್ಷೇತ್ರಕ್ಕೆ ಸಂಬಂಧಿಸಿದಂತೆ ಜಿಲ್ಲೆಯ ಶಾಸನಗಳಲ್ಲಿ ಉಲ್ಲೇಖವಾಗಿರುವ ವಿವರಗಳನ್ನು ಸಂಗ್ರಹಿಸಿ, ವಿಶ್ಲೇಷಿಸಿ ಈ ಕೆಳಗಿನಂತೆ ಕ್ರಮಬದ್ಧವಾಗಿ ನಿರೂಪಿಸಲಾಗಿದೆ.} ಸಂದರ್ಭೋಚಿತವಾಗಿ ಅಕ್ಕಪಕ್ಕದ ಜಿಲ್ಲೆಯ ಶಾಸನಗಳನ್ನೂ ಉಲ್ಲೇಖಿಸಲಾಗಿದೆ.

\section*{ಭೂಮಿಯ ವಿಧಗಳು – ಗದ್ದೆ, ಬೆದ್ದಲು,ತೋಟ, ತುಡಿಕೆ, ಮರ}
\index{ಗದ್ದೆ} \index{ಬೆದ್ದಲು (ಬೆದ್ದಲೆ)} \index{ತೋಟ (ತೋಟದ ಸ್ಥಳ - ಕ್ಷೇತ್ರ, ತೋಂಟ)} \index{ತುಡಿಕೆ} \index{ಮರ}

ಗಂಗರು ಮತ್ತು ಹೊಯ್ಸಳರ ಕಾಲದ ಶಾಸನಗಳಲ್ಲಿ ಭೂಮಿಯನ್ನು ಪ್ರಧಾನವಾಗಿ ಗದ್ದೆ, ಬೆದ್ದಲು ಮತ್ತು ತೋಟ(ತೋಂಟ) ಎಂದು ಮೂರು ವಿಧವಾಗಿ ವಿಂಗಡಿಸಿರುವುದು ಕಂಡುಬರುತ್ತದೆ. ನಂತರ ತುಡಿಕೆ ಎಂಬ ಒಂದು ವಿಧವೂ ವರ್ಗೀಕರಿಸಲ್ಪಟ್ಟಿತು. ನೀರಿನ ಆಸರೆಯಿಂದ ಭತ್ತವನ್ನು(ನೆಲ್ಲು, ಅಕ್ಕಿ) ಬೆಳೆಯುತ್ತಿದ್ದ ಪ್ರದೇಶವೇ ಗದ್ದೆ, ಮಳೆಯ ಆಶ್ರಯದಲ್ಲಿ ವ್ಯವಸಾಯ ಮಾಡಿ ಇತರ ಧಾನ್ಯಗಳನ್ನು ಬೆಳೆಯುತ್ತಿದ್ದ ಪ್ರದೇಶವೇ ಬೆದ್ದಲು, ಕೆಯಿ ಅಥವಾ ಹೊಲ.\endnote{ ಎಕ 6 ಕೃಪೇ 56 ಭದ್ರನಕೊಪ್ಪಲು 12ನೇ ಶ.} ವಿಜಯನಗರ ಮತ್ತು ಒಡೆಯರ ಕಾಲದ ಶಾಸನಗಳಲ್ಲಿ ನೀರಾವರಿಯ ಬೇಸಾಯವನ್ನು “ನೀರಾರಂಭ”\index{ನೀರಾರಂಭ} ಎಂದು, ಮಳೆ ಆಶ್ರಯದ ವ್ಯವಸಾಯವನ್ನು “ಕಾಡಾರಂಭ”\index{ಕಾಡಾರಂಭ} ಎಂದು ಹೇಳಿದೆ. ತೆಂಗು, ಅಡಿಕೆ, ಬಾಳೆ ಇವುಗಳನ್ನು ಬೆಳೆಯುವ ಭೂಮಿಯನ್ನು ತೋಟವೆಂದೂ, ಇವುಗಳನ್ನು ಬಿಟ್ಟು ಫಲವೃಕ್ಷಗಳೂ\index{ಫಲವೃಕ್ಷ} ಸೇರಿದಂತೆ ಇತರೆ ವೃಕ್ಷಗಳನ್ನು ಬೆಳೆಯುವ ಪ್ರದೇಶವೇ ತುಡಿಕೆ ಎಂದು ಕರೆಯಲಾಗಿದೆ ಎಂದು ಹೇಳಬಹುದು. “ತೆಂಗು, ಅಡಿಕೆ, ಬಾಳೆ ಮುಂತಾದವನ್ನು ಬೆಳೆಸುವ ಭೂಮಿಯನ್ನು ತೋಟವೆಂದೂ, ಹೂಗಿಡಗಳನ್ನು ಬೆಳೆಸುವ ಉದ್ಯಾನವನ್ನು ತುಡಿಕೆ ಎಂದೂ ಕರೆಯುತ್ತಿದ್ದರು ಎಂದು ಅಭಿಪ್ರಾಯ ಪಡಲಾಗಿದೆ”.\endnote{ ವೆಂಕಟಾಚಲಶಾಸ್ತ್ರೀ, ಡಾ॥ ಟಿ.ವಿ., ಕನ್ನಡ ಯಮಳಶಬ್ದಗಳು, ಹಳೆಯ ಹೊನ್ನು, ಪುಟ 194-95} ಆದರೆ ಹೂವನ್ನು ಬೆಳೆಸುತ್ತಿದ್ದ ಭೂಮಿಯನ್ನು ಜಿಲ್ಲೆಯ ಅನೇಕ ಶಾಸನಗಳಲ್ಲಿ ಪುಷ್ಪೋದ್ಯಾನ,\index{ಪುಷ್ಪೋದ್ಯಾನ} ನಂದನವನ\index{ನಂದನವನ} ಎಂದು ಕರೆಯಲಾಗಿದೆ. “ದೇವರ ಹೂದೋಟದ\index{ದೇವರ ಹೂದೋಟ} ಭೂಮಿ”,\endnote{ ಎಕ 7 ಮಂ 29 ಬಸರಾಳು 1234} “ಕೆರೆಯೊಳಗಣ ಹೂದೋಟವೊಂದು। ಯಪ್ಪತ್ತು ಕಂಬದ ತುಡಕೆವೊಂದು” ಎಂದು ಹೂದೋಟ ಮತ್ತು ತುಡಕೆ ಎರಡನ್ನೂ ಬೇರೆ ಬೇರೆಯಾಗಿಯೇ ಹೇಳಿದೆ.\endnote{ ಎಕ 10 ಅರ 184 ಹುಲ್ಲೇಕೆರೆ 1162} ಆದುದರಿಂದ ತುಡಕೆ ಎಂದರೆ ಫಲವೃಕ್ಷಗಳನ್ನು ಬೆಳೆಸುತ್ತಿದ್ದ ಪ್ರದೇಶ ಎಂದು ಹೇಳಬಹುದು. ನೆಲ್ಲು(ಭತ್ತ), ರಾಗಿ ಜೋಳ ಧಾನ್ಯದ ಹೆಸರುಗಳು ಶಾಸನಗಳಲ್ಲಿ ಉಲ್ಲೇಖವಾಗಿವೆ.\endnote{ ಎಕ 7 ನಾಮಂ 67 ದಡಗ 1285} ರಾಗಿಯಹಳ್ಳಿ ಎಂಬ ಊರೂ ನಾಗಮಂಗಲ ತಾಲ್ಲೂಕಿನಲ್ಲಿ ಶಾಸನೋಕ್ತವಾಗಿದೆ.\endnote{ ಎಕ 7 ನಾಮಂ 73 ಬೆಳ್ಳೂರು 1284}

ಮರ ಎಂಬ ಭೂಮಿಯ ವಿಧವೂ ಶಾಸನಗಳಲ್ಲಿ ಹೇಳಲ್ಪಟ್ಟಿದೆ. ಹೊಯ್ಸಳರ ಒಂದು ಶಾಸನದಲ್ಲಿ “ಹಳ್ಳಿಯ ಚತುಸ್ಸೀಮೆ\-ಯೊಳಗಣ ಮರ,\index{ಮರ} ಹಳ್ಳಿ, ಕೆರೆ, ತೋಟ, ತೆಂಗು ಕಉಂಗು, ಗದ್ದೆ, ಬೆದ್ದಲು ಸಹಿತ ನಾಲ್ಕರಲೊನ್ದು ಭಾಗ ಭೂಮಿ\-ಯನು” ಎಂದು ಹೇಳಿದೆ. ಇದೇ ಕಾಲದ ಒಂದು ತಮಿಳು ಶಾಸನದಲ್ಲಿ “ಪಳ್ಳಿಗಳ್​ ತೋಟಂಗಳ್​ ಮೇನೊಕ್ಕಿನ ಮರಮ್ ಕೀಣೊಕ್ಕಿನ ಕಣರುಗಳನ್ದ ಸೇನಾಪತಿಎಡುತ್​ ತಿರುನನ್ದವನ”\index{ತಿರುನಂದನವನ (ನಂದಾವನ-ತಿರುನಂದನ)} ಎಂದು ಹೇಳಿದೆ.\endnote{ ಎಕ 7 ಮ 101 ಬನ್ನಹಳ್ಳಿ 1314} ಇಲ್ಲಿ ತೋಟ ಮತ್ತು ಮರ ಎರಡನ್ನೂ ಬೇರೆಯಾಗಿ ಹೇಳಿದೆ. ವಿಜಯನಗರ ಶಾಸನಗಳಲ್ಲಿ ವ್ಯವಸಾಯದ ಭೂಮಿಯನ್ನು “ಗದ್ದೆ ತೋಟ, ಮರ” ಎಂದು ವಿಂಗಡಿಸಿರುವುದು ಕಂಡು ಬರುತ್ತದೆ.\endnote{ ಎಕ 7 ಮವ 10 ಸಶ್ಯಾಲಪುರ 1517} “ಮರ” ಎಂಬುದು ನಾನಾ ರೀತಿ ಫಲವೃಕ್ಷಗಳನ್ನು, ನಿರ್ಮಾಣಕ್ಕೆ ಅಗತ್ಯವಾದ ಮರಮುಟ್ಟುಗಳನ್ನು ಬೆಳೆಸುತ್ತಿದ್ದ ಭೂಮಿ ಎಂದು ಹೇಳಬಹುದು. ಮರಗಳನ್ನು ಕಡಿದಾಗ ವಿಧಿಸುತ್ತಿದ್ದ ತೆರಿಗೆಯನ್ನು, ‘ಮರವಳಿ’, ‘ಮರವಳಿಯ ಸುಂಕ’ ಎಂದು ಕರೆಯಲಾಗಿದೆ.\endnote{ ಎಕ 7 ಮವ 139 ಸುಜ್ಜಲೂರು 1473} ಮರಗಳನ್ನು ಹೆಚ್ಚಾಗಿ ಬೆಳೆದು, ಮರಗೆಲಸವನ್ನು ವಿಶೇಷವಾಗಿ ಮಾಡುತ್ತಿದ್ದ ಹಳ್ಳಿಗಳನ್ನು ‘ಮರವೂರು’,\endnote{ ಎಕ 7 ಮಂ 14 ಹುಳ್ಳೇನಹಳ್ಳಿ 8ನೇ ಶ.} ‘ಮರಹಳ್ಳಿ’,\endnote{ ಎಕ 7 ಮವ 14 ಮಗ್ಗನಹಳ್ಳಿ 10ನೇ ಶ.} ಎಂದು ಕರೆಯುತ್ತಿದ್ದರೆಂದು ಊಹಿಸಬಹುದು.

ಜಿಲ್ಲೆಯಲ್ಲಿರುವ, ವಿಜಯನಗರದ ಮತ್ತು ಮೈಸೂರು ಅರಸರ ಕಾಲದ ಶಾಸನಗಳಲ್ಲಿ ಕೃಷಿ ಭೂಮಿಯನ್ನು ಕ್ರಮಬದ್ಧವಾಗಿ ವಿಂಗಡಿಸಿರುವುದು ಕಂಡುಬರುತ್ತದೆ. ಕ್ರಿ.ಶ.1469ರ ಮೇಲುಕೋಟೆ ಶಾಸನದಲ್ಲಿ, \textbf{“ಗದ್ದೆ\index{ಗದ್ದೆ} ಬೆದ್ದಲು\index{ಬೆದ್ದಲು (ಬೆದ್ದಲೆ)} ತೋಟ\index{ತೋಟ (ತೋಟದ ಸ್ಥಳ - ಕ್ಷೇತ್ರ, ತೋಂಟ)} ತುಡಿಕೆ\index{ತುಡಿಕೆ} ಅಣೆ ಅಚ್ಚುಕಟ್ಟು\index{ಅಣೆ ಅಚ್ಚುಕಟ್ಟು} ಆಗಾಮಿ ಮುಂತಾದ ಕ್ಷೇತ್ರದ”} ಎಂದು,\endnote{ ಎಕ 6 ಪಾಂಪು 163 ಮೇಲುಕೋಟೆ 1469} ಕ್ರಿ.ಶ.1477ರ ಮಂಡ್ಯ ತಾಲ್ಲೂಕು ಚಾಮಲಾಪುರ ಶಾಸನದಲ್ಲಿ “ಗದ್ದೆಬೆದಲು ಅಣೆ ಅಚ್ಚುಕಟ್ಟು ಕಟ್ಟೆ ಕಾಲುವೆ” ಎಂದು,\endnote{ ಎಕ 7 ಮಂ 42 ಚಾಮಲಾಪುರ 1477} ಕ್ರಿ.ಶ.1524ರ ನಾಗಮಂಗಲ ತಾಲ್ಲೂಕು ತಿಬ್ಬನಹಳ್ಳಿ ಶಾಸನದಲ್ಲಿ \textbf{“ಗದ್ದೆ ಬೆದ್ದಲು ತೋಟ ತುಡಿಕೆ ಅಣೆ ಅಚ್ಚುಕಟ್ಟು”}\endnote{ ಎಕ 7 ನಾಮಂ 164 ತಿಬ್ಬನಹಳ್ಳಿ 1524}, ಎಂದು, ಕ್ರಿ.ಶ.1528ರ ಶ‍್ರೀರಂಗಪಟ್ಟಣ ಶಾಸನದಲ್ಲಿ \textbf{“ಗದ್ದೆ ಬೆದ್ದಲು ತೋಟ ತುಡಿಕೆ ಅಣೆ ಅಚ್ಚುಕಟ್ಟು ಏತಗುಯ್ಯಲು ಕಾಡಾರಂಭ\index{ಕಾಡಾರಂಭ} ನೀರಾರಂಭ\index{ನೀರಾರಂಭ} ಕಳ\index{ಕಳ} ಕೊಠಾರ”}\index{ಕೊಠಾರ} ಎಂದು\endnote{ ಎಕ 6 ಶ‍್ರೀಪ 7 ಶ‍್ರೀರಂಗಪಟ್ಟಣ 1528}, ಕ್ರಿ.ಶ.1623ರ ಹೊನ್ನಲಗೆರೆ ಶಾಸನದಲ್ಲಿ \textbf{“ಗದ್ದೆ ಬೆದ್ದಲು ತೋಟ ತುಡಿಕೆ ಅಡು ಮನೆ ಅಣೆ ಅಚ್ಚುಕಟ್ಟು\index{ಅಚ್ಚುಕಟ್ಟು} ಕಾಡಾರಂಭ, ನೀರಾರಂಭ, ಗೂಡೆ,\index{ಗೂಡೆ}\general{\break } ಗುಯ್ಯಲು,\index{ಗುಯ್ಯಲು} ಕಳ ಕೊಠಾರ”} ಎಂದು, ಕ್ರಿ.ಶ.1680ರ ಬೆಳ್ಳೂರು ತಾಮ್ರಶಾಸನದಲ್ಲಿ \textbf{“ಈ ಗ್ರಾಮಕ್ಕೆ ಸಲುವಂತಾ ಯರೆನೆಲ,\general{\break } ಕೆಂನೆಲ, ಕಾಡಾರಂಭ, ನೀರಾರಂಭ, ಅಣೆ ಅಚ್ಚುಕಟ್ಟು, ಯಾತ\index{ಯಾತ} ಕಪಿಲೆ\index{ಕಪಿಲೆ} ಗೂಡೆ ಗೂಯಿಲು ಕೆರೆ ಕುಂಟೆ ಕಾಲುವೆ ಮೊದಲಾಗಿ”} ಎಂದು ವಿಂಗಡಿಸಲಾಗಿದೆ.\endnote{ ಎಕ 7 ನಾಮಂ 96 ಬೆಳ್ಳೂರು 1680} ಈ ರೀತಿಯ ಕ್ರಮಬದ್ಧವಾದ ಭೂವಿಂಗಡಣೆಯ ಉಲ್ಲೇಖಗಳು ವಿಜಯನಗರ ಕಾಲದಿಂದ ಮೈಸೂರು ಒಡೆಯರ ಕಾಲದವರೆಗೆ, ಜಿಲ್ಲೆಯ ಶ‍್ರೀರಂಗಪಟ್ಟಣ,\endnote{ ಎಕ 6 ಶ‍್ರೀಪ 2 ಶ‍್ರೀರಂಗಪಟ್ಟಣ 1528, 7-1528, 8-1528} ಮೇಲುಕೋಟೆ,\endnote{ ಎಕ 6 ಪಾಂಪು 163 ಮೇಲುಕೋಟೆ 1391, ಮೇಲುಕೋಟೆ 132-1530 129-1545, 179-1459, 128-1564} ನೆಟ್ಟಕಲ್ಲು,\endnote{ ಎಕ 7 ಮವ 86 ನೆಟ್ಟಕಲ್ಲು 1532} ಮೇನಾಗರ,\endnote{ ಎಕ 6 ಪಾಂಪು 230 ಮೇನಾಗರ 1584} ನರಿಹಳ್ಳಿ,\endnote{ ಎಕ 6 ಪಾಂಪು 223 ನರಿಹಳ್ಳಿ 1585} ಹೊನ್ನಲಗೆರೆ,\endnote{ ಎಕ 7 ಮ 64 ಹೊನ್ನಲಗೆರೆ 1623} ಮಾದಾಪುರ,\endnote{ ಎಕ 6 ಕೃಪೇ 45 ಮಾದಾಪುರ 17ನೇ ಶ.} ಕುಡುಗುಬಾಳು,\endnote{ ಎಕ 7 ನಾಮಂ 165 ಕುಡುಗುಬಾಳು 1640} ಶ‍್ರೀರಂಗಪಟ್ಟಣ,\endnote{ ಎಕ 6 ಶ‍್ರೀಪ 23 ಶ‍್ರೀರಂಗಪಟ್ಟಣ 1664} ಬೆಳ್ಳೂರು,\endnote{ ಎಕ 7 ನಾಮಂ 96 ಬೆಳ್ಳೂರು 1680} ತೊಣ್ಣೂರು,\endnote{ ಎಕ 6 ಪಾಂಪು 99 ತೊಣ್ಣೂರು 1722} ಮೇಲುಕೋಟೆ,\endnote{ ಎಕ 6 ಪಾಂಪು 129 ಮೇಲುಕೋಟೆ 1545

ಎಕ 6 ಪಾಂಪು 215 ಮೇಲುಕೋಟೆ 1724

ಎಕ 6 ಪಾಂಪು 216 ಮೇಲುಕೋಟೆ 1725} ಮೊದಲಾದ ಅನೇಕ ಶಾಸನಗಳಲ್ಲಿ ಕಂಡುಬರುತ್ತದೆ.

\textbf{ಗೋಮಾಳ\index{ಗೋಮಾಳ}/ಗೋಭೂಮಿ\index{ಗೋಭೂಮಿ}/ಸೊಪ್ಪಿನಗುಡ್ಡೆ:}\index{ಸೊಪ್ಪಿನಗುಡ್ಡೆ} ಊರಿನ ದನಕರುಗಳನ್ನು ಮೇಯಿಸುತ್ತಿದ್ದ ಗೋಮಾಳವನ್ನು ಜಿಲ್ಲೆಯ ಶಾಸನಗಳಲ್ಲಿ “ಯರಡುಕೋಡಿಗಳಿಂದೊಳಗುಳ್ಳ ಯೆಲ್ಲೆತಿಟ್ಟು ಗೋಭೂಮಿ”,\endnote{ ಎಕ 7 ಮವ 139 ಸುಜ್ಜಲೂರು 1473} “ನೆಲಮನೆ ಬಲ್ಲೇನಹಳ್ಳಿಗೆ ಸಲ್ಲುವ ಗೋಭೂಮಿಯಂನೂ ಯೆಡದಲ್ಲಿ ಹಾಯ್ಕಿಕೊಂಡು”,\endnote{ ಎಕ 6 ಶ‍್ರೀಪ 93 ನೆಲಮನೆ 1458} ಎಂಬ ಉಲ್ಲೇಖಗಳಲ್ಲಿ ಗೋಭೂಮಿ ಎಂದು ಹೇಳಿದೆ. ಇದನ್ನು “ಸೊಪ್ಪಿನಗುಡ್ಡೆ”,\endnote{ ಎಕ 7 ಮ 64 ಹೊನ್ನಲಗೆರೆ 1623} ಬಿಳಲ(ಬೀಳಿಲು)(ಬೀಳುಭೂಮಿ)ಎಂದೂ ಶಾಸನಗಳಲ್ಲಿ ಹೇಳಿದೆ.\endnote{ ಚಿದಾನಂದಮೂರ್ತಿ,ಡಾ॥ ಎಂ., ಕನ್ನಡ ಶಾಸನಗಳ ಸಾಂಸ್ಕೃತಿಕ ಅಧ್ಯಯನ, ಪುಟ 368}

\textbf{ಕಾಡನ್ನು ಕಡಿದು ವ್ಯವಸಾಯ ಭೂಮಿ:} ಕಾಡನ್ನು ಕಡಿದು ಗದ್ದೆ, ಬೆದ್ದಲು, ತೋಟಗಳನ್ನು ಆಬಾದು ಮಾಡಲಾಗುತ್ತಿತ್ತು. “ಆ ಮರ \textbf{ಕಾಡನು} ಗದೆ ಬೆದಲೊಳಗಾದ ಭೂಮಿಯೊಳಗೆ ತೆಂಗು ಕವುಂಗು ಮುಖ್ಯವಾದ ಸಮಸ್ತ ಸ್ಥಾವರವಹ ಫಲವ್ರುಕ್ಷಂ\break ಗಳನೂ ಯಿಕ್ಕಿಕೊಂಡು ಕೆರೆಯ ಕಟ್ಟಿಕೊಂಡು ಕಾಲವೆಯನು ತಂದುಕೊಂಡು ಸಂತಾನಗಾಮಿಯಾಗಿ ಭೋಗಿಸುವರು”,\endnote{ ಎಕ 7 ಮಂ 56 ಹೊಸಬೂದನೂರು 1276} “ಮಯಿಲನಹಳ್ಳಿಯ ಕೆರೆಯ ಕೆಳಗಣ ಕುಂಬಾರಕಟ್ಟೆಯ ಕಾಡೊಳಗುಳ್ಳ ಆ ಭೂಮಿ ಮಾಡುವವರಿಗೆ”,\endnote{ ಎಕ 6 ಪಾಂಪು 164 ಮೇಲುಕೋಟೆ 1369} ಎಂಬುದು ಕಾಡನ್ನು ಕಡಿದು ಭೂಮಿಯನ್ನು ವ್ಯವಸಾಯಕ್ಕೆ ಸಿದ್ಧಪಡಿಸಿಕೊಳ್ಳುತ್ತಿದ್ದುದನ್ನು ಹೇಳುತ್ತದೆ. ವೀರಯ್ಯ ದಂಡನಾಯಕನು ಕಾಡನ್ನು ಕಡಿದು ವೀರಬಲ್ಲಾಳಪುರವನ್ನು ಮಾಡಿ ಮೂರು ಕೆರೆಗಳನ್ನು ಕಟ್ಟಿಸಿದನೆಂದು ತಿಳಿದುಬರುತ್ತದೆ.\endnote{ ಎಕ 9 ಬೇಲೂರು 438 ವೀರದೇವನಹಳ್ಳಿ 1186}

\section*{ನೀರಾವರಿ ಬೇಸಾಯ \enginline{-} ಕೆರೆಯ ಹಿಂದಿನ ಗದ್ದೆಗಳು \enginline{-} ಗದ್ದೆಯಬಯಲು \enginline{-} ಅಚ್ಚುಕಟ್ಟು:}

ನೀರಾವರಿ ಸೌಲಭ್ಯದಿಂದ ಭತ್ತವನ್ನು ಬೆಳೆಯುತ್ತಿದ್ದ ಭೂಮಿಯನ್ನು ಜಿಲ್ಲೆಯ ಬಹುತೇಕ ಶಾಸನಗಳಲ್ಲಿ, ಗೞ್ದೆ, ಗದ್ದೆ, ಗರ್ದ್ದೆ, ಗದ್ದೆಯ ಸ್ಥಳ, ಗದ್ದೆಯ ಕ್ಷೇತ್ರ, ಎಂದು ಕರೆಯಲಾಗಿದ್ದು, ಇನ್ನು ಕೆಲವು ಶಾಸನಗಳಲ್ಲಿ, ಮಣ್ಣು,\endnote{ ಎಕ 7 ಮ 56 ತಾಯಲೂರು 907, ಎಕ 7 ಮ 42 ಆತಕೂರು 949} ನಿಮಣ್ನು(ನೀರ್ಮ್ಮಣ್ಣು),\endnote{ ಎಕ 6 ಪಾಂಪು 44 ಕನ್ನಂಬಾಡಿ 1114}\break ನೀರ್ಣ್ನೆಲ,\endnote{ ಎಕ 7 ನಾಮಂ 26 ಕಂಬದಹಳ್ಳಿ 1168} ಎರೆ,\endnote{ ಎಕ 7 ನಾಮಂ 149 ದೇವರಹಳ್ಳಿ 776} ತರಿ,\index{ತರಿ}\endnote{ ಎಕ 6 ಕೃಪೇ 59 ಸಾಸಲು 1121} ಎಂದು ಕರೆಯಲಾಗಿದೆ. ತಮಿಳು ಶಾಸನದಲ್ಲಿ “ಪುನ್ಶೆಯ್​”,\endnote{ ಎಕ 6 ಶ‍್ರೀಪ 114 ಅರಕೆರೆ 11ನೇ ಶ.} ಎಂದು ಹೇಳಿದ್ದು ಇದನ್ನು\break ನೀರ್ಮಣ್ಣು\index{ನೀರ್ಮಣ್ಣು} ಅಥವಾ ನೀರ್ನೆಲ\index{ನೀರ್ನೆಲ} (wet land) ಎಂಬುದಾಗಿ ಎಪಿಗ್ರಾಫಿಯಾ ಸಂಪಾದಕರು ಹೇಳಿದ್ದಾರೆ. ಗದ್ದೆಯನ್ನು\break ‘ನೆಲು(ಲ್ಲು)’ - “ಕಞ್ಡುಗ ನೆಲು” ಎಂದು ಹೇಳಿದೆ.\endnote{ ಎಕ 7 ಮವ 143 ಕಲ್ಕುಣಿ 13–14ನೇ ಶ.} ಈಗಲೂ ಭತ್ತದ ಹುಲ್ಲಿಗೆ ನೆಲ್ಲು ಹುಲ್ಲು ಎಂದು ಹೇಳುವರು. ಕಳನಿ ಎಂಬ ಪದ ಪ್ರಯೋಗವಿದ್ದು ಇದು ಕೇವಲ ಭತ್ತವನ್ನು ಬೆಳೆಯುವ ಗದ್ದೆಯಾಗಿರಬಹುದು. “ಗದ್ದೆ ಸಲಗೆ ಮೂರನು ಉತ್ತಮವಹ ಠಾವಿನಲು ಶಂಖಚಕ್ರದ ಕಲ್ಲನು ನಡ್ಸಿ ಕೊಟ್ಟರು”,\endnote{ ಎಕ 7 ನಾಮಂ 73 ಬೆಳ್ಳೂರು 1284} ಎಂದು ಬೆಳ್ಳೂರು ಶಾಸನದಲ್ಲಿ ಹೇಳಿದ್ದು, ದಾನವನ್ನು ನೀಡುವಾಗ ಉತ್ತಮವಾದ ಸ್ಥಳದಲ್ಲಿ ಗದ್ದೆಯನ್ನು ದತ್ತಿಯಾಗಿ ಬಿಡುತ್ತಿದ್ದರು.

\vskip 2pt

\textbf{ಕೆರೆಯ ಬಯಲು/ಅಚ್ಚುಕಟ್ಟು:} ಕೆರೆಯ ತೂಬಿನಿಂದ ಹೊರಟ ಕಾಲುವೆಗಳ ಕೆಳಗೆ ಗದ್ದೆಗಳಿರುತ್ತಿದ್ದವು “ಮೞ್ತೆ ಕಾಲಂಗಳೊಳ್​ ಒರ್ಕ್ಕಣ್ಡುಗ ಗದ್ದೆ”, ಅಂದರೆ ಮತ್ತಿಮರಗಳ ಸಾಲಿನಿಂದ ಕೂಡಿದ್ದ ಕಾಲುವೆಯ ಕೆಳಗೆ ಒಂದು ಕೊಳಗ ಗದ್ದೆ ಎಂದು ಆತಕೂರು ಶಾಸನದಲ್ಲಿ ಹೇಳಿದೆ.\endnote{ ಎಕ 7 ಮ 42 ಆತಕೂರು 949} “ಈ ಕೆರೆಗೆ ಕಾಲುವೆ ಸಹ ಪೂರ್ವ ಮರ್ಯಾದೆ” ಎಂದು ಸುಜ್ಜಲೂರು ಶಾಸನದಲ್ಲಿ,\endnote{ ಎಕ 7 ಮವ 139 ಸುಜ್ಜಲೂರು 1473} “ಕಟ್ಟು ಕುಲ್ಯಾ ಸಮಾಯುಕ್ತಾ ತಟಾಕೇಸ್ಯತು” ಎಂದರೆ ಕಟ್ಟುಕಾಲುವೆಗಳಿಂದ ಕೂಡಿದ ಕೆರೆ ಎಂಬುದಾಗಿ ಬ್ಯಾಲದಕೆರೆ ಶಾಸನದಲ್ಲಿ ಹೇಳಿದೆ.\endnote{ ಎಕ 6 ಕೃಪೇ 99 ಬ್ಯಾಲದಕೆರೆ 1532} ಅಣೆಕಟ್ಟು ಅಥವಾ ಕೆರೆಯ ಕಾಲುವೆಗಳಿಂದ ನೀರಾವರಿ ವ್ಯವಸಾಯಕ್ಕೆ ಒಳಗಾದ ಭೂಮಿಯನ್ನು ‘ಅಚ್ಚುಕಟ್ಟು’\index{ಅಚ್ಚುಕಟ್ಟು} ಅಥವಾ ‘ಬಯಲು’ ಎಂದು ಕರೆಯಲಾಗಿದೆ. “ದುಂಡುಸಮುದ್ರದಾ ವಯಲು”\index{ದುಂಡುಸಮುದ್ರದಾ ವಯಲು},\endnote{ ಎಕ 7 ನಾಮಂ 149 ದೇವರಹಳ್ಳಿ 776} “ಮರಿಯಾನೆ ಸಮುದ್ರದ ಬಯಲುಮಂ”,\index{ಮರಿಯಾನೆ ಸಮುದ್ರದ ಬಯಲು}\endnote{ ಎಕ 7 ನಾಮಂ 69 ದಡಗ 13ನೇ ಶ.} “ಕಬಿನಕೆರೆಯ ಮೂಡಣ ಕೋಡಿಯಿಂದವಾ ಬಯಲು”,\index{ಬಯಲು}\endnote{ ಎಕ 7 ನಾಮಂ 64 ಯಲ್ಲಾದಹಳ್ಳಿ 1145} “ಇತರೆ ಗದ್ದೆ ಕೆಳಗೆ ಪೂರ್ವ್ವದಲ್ಲಿ ವುಳ್ಳ ಅಚ್ಚುಕಟ್ಟು ಮುಖ್ಯವಾದ”,\endnote{ ಎಕ 7 ನಾಮಂ 83 ಬೆಳ್ಳೂರು 1269} ಎಂದು ಹೇಳಿದೆ. “ಈ ಯೆಲ್ಲಾ ಸ್ಥಳಂಗಳ ಪ್ರಸಿದ್ಧ ಸೀಮಾಸಮಂನ್ವಿತವಹ ಕ್ಷೇತ್ರಂಗಳೊಳಗೆ ಕೆರೆ ಕಟ್ಟೆ ಕಾಲುವೆ ಮುಖ್ಯವಾದವನು ಕಟ್ಟಿಸಿ”,\endnote{ ಎಕ 7 ನಾಮಂ 76 ಬೆಳ್ಳೂರು 1284} ಎಂದು ಬೆಳ್ಳೂರು ಶಾಸನದಲ್ಲಿ ಹೇಳಿದ್ದು, ಕೆರೆ ಕಟ್ಟೆಗಳನ್ನು ಕಟ್ಟಿಸಿ ಕಾಲುವೆಯನ್ನು ತೋಡಿಸಿ ನೀರಾವರಿಗೆ ಅನುಕೂಲ ಮಾಡಿಕೊಡಲಾಗುತ್ತಿತ್ತು. ಕಾಲವೆಯನ್ನು “ನೀರುಗಾಲುವೆ” ಎಂದೂ ಕರೆಯಲಾಗಿದೆ.\endnote{ ಎಕ 7 ನಾಮಂ 84 ಬೆಳ್ಳೂರು 1269}

\vskip 2pt

\textbf{ಕೀಳೇರಿಯ ಗದ್ದೆ,\index{ಕೀಳೇರಿಯ ಗದ್ದೆ} ಮೊದಲೇರಿಯ ಗದ್ದೆ:}\index{ಮೊದಲೇರಿಯ ಗದ್ದೆ} ಕೆರೆ ಬಯಲಿನ ಗದ್ದೆಗಳಲ್ಲಿ ಕೀಳೇರಿಯ ಗದ್ದೆ, ಮೊದಲೇರಿಯ ಗದ್ದೆ ಎಂದು, ಎರಡು ವಿಧವನ್ನು ಹೇಳಿದೆ. ಏರಿಯು ಪ್ರಾರಂಭವಾಗುವ ಎರಡೂ ಕಡೆ, ಏರಿಗಳು ಎತ್ತರವಾಗಿರುವದಿಲ್ಲ, ಇದನ್ನು ಮೊದಲೇರಿ ಎಂದು ಕರೆಯಲಾಗಿದೆ. ಈ ಮೊದಲೇರಿಯ ಹಿಂದಿನ ಗದ್ದೆಗಳು ಸ್ವಲ್ಪ ಎತ್ತರ ಪ್ರದೇಶದಲ್ಲಿರುತ್ತವೆ. ಇದನ್ನು ಮೊದಲೇರಿಯ ಗದ್ದೆ ಎಂದು ಕರೆಯಲಾಗಿದೆ. ಕೆರೆಯ ಮಧ್ಯಭಾಗಕ್ಕೆ ಬಂದಂತೆಲ್ಲಾ, ಏರಿಯು ಬಹಳ ಎತ್ತರವಾಗಿದ್ದು, ಅದರ ಹಿಂದೆ ಹಳ್ಳದಂತಹ ಪ್ರದೇಶದಲ್ಲಿರುವ ಗದ್ದೆಗೆ ಕೀಳೇರಿಯ\index{ಕೀಳೇರಿ} ಗದ್ದೆ ಎಂದು ಕರೆಯಲಾಗಿದೆ. ಕೀಳ್​ ಎಂದರೆ ಕೆಳಗೆ ಎಂಬ ಅರ್ಥವಿದೆ. ಮೊದಲೇರಿಯ ಗದ್ದೆಗೆ ನೀರಿನ ಹರಿವು ಕಡಿಮೆ, ಕೀಳೇರಿಯ ಗದ್ದೆಗೆ ನೀರಿನ ಹರಿವು ಜಾಸ್ತಿ. “ಕೀಳೇರಿಯೊಳು ದೇವರ ತೆಂಕಣದೆಸೆಯೊಳು ಗೞ್ದೆ ಸಲಗೆ ನಾಲ್ಕುಂ”,\endnote{ ಎಕ 6 ಕೃಪೇ 73 ಹಿರಿಕಳಲೆ 12ನೇ ಶ.(1118)} “ಕೀಳೇರಿಯ ಗದ್ದೆ ಸಲಗೆ ಒಂದು ಕೊಳಗ ಹತ್ತು”,\endnote{ ಎಕ 7 ನಾಮಂ 72 ಅಳೀಸಂದ್ರ 1183} “ಹೊಯ್ಸಳ ಸಮುದ್ರದ ಮೊದಲೇರಿಯೊಳೊರ್ಖಂಡುಗ ನೀರ್ವ್ವರೆಯುಮಂ, ತೆಂಕಣಸೆಟ್ಟಿಯ ಕೆರೆಯ\index{ತೆಂಕಣಸೆಟ್ಟಿಯ ಕೆರೆ} ಮೊದಲೇರಿಯೊಳ್​ ಖಂಡುಗ ಗದ್ದೆ”,\endnote{ ಎಕ 7 ನಾಮಂ 118 ಹಟ್ಟಣ 1178} “ಯೆಕಟ್ಟೆಯ ಕೆರೆಯ ಕೆಳಗೆ ಮೊದಲೇರಿಯಲ್ಲಿ ಗದ್ದೆ,\endnote{ ಎಕ 7 ಮಂ 32 ಬಸರಾಳು 1507} “ಹಿರಿಯ ಕೆರೆಯ\index{ಹಿರಿಯ ಕೆರೆ} ಮೊದಲಗದ್ದೆ”,\endnote{ ಎಕ 7 ನಾಮಂ 62 ಲಾಳನಕೆರೆ 1218} “ಹಿರಿಯ ಕೆರೆಯ ಕೆಳಗೆ ಮೊದಲೇರಿಯಲು ತೋಟಸ್ಥಳ ಗದ್ದೆ”,\endnote{ ಎಕ 6 ಕೃಪೇ 98 ಭೈರಾಪುರ 1267} “ಹಿರಿಯ ಕೆರೆಯ ಕೆಳಗಣ ಮೊದಲ ಮಡವೆಯಲಿ ಗದ್ದೆ ಐದು ಸಲಗೆಯೂ”,\endnote{ ಎಕ 7 ಮ 110 ಬೊಪ್ಪಸಂದ್ರ 1388} ಮೊದಲ ಹೊಸ ಬಯಲಲು,\endnote{ ಎಕ 7 ನಾಮಂ 65 ದಡಗ 1400} ಪೆರಿಯೇರಿಯ ಕೀೞಿಲ್​ ಇರುಬದಿನ್​ ಕಣ್ಡಗ”,\endnote{ ಎಕ 7 ಮಂ 70 ಬೇಲೂರು 1162} ಈ ರೀತಿ ಜಿಲ್ಲೆಯ ಶಾಸನಗಳಲ್ಲಿ ಅನೇಕ ಉದಾಹರಣೆಗಳಿವೆ.

\vskip 2pt

\textbf{ತೂಬಿನ ಬಾಯಿನ ಗದ್ದೆಗಳು:} “ಕೆರೆಯ ತೂಬಿನಿಂದ ಹೊರಟ ನೀರು ಕಾಲುವೆಗಳಲ್ಲಿ (ಬಾಯ್ಕಲ್​ ಅಥವಾ ಕಾಲ್​) ಹರಿದು ಗದ್ದೆಗಳನ್ನು ಸೇರುತ್ತಿತ್ತು.\endnote{ ಚಿದಾನಂದಮೂರ್ತಿ ಡಾ॥ ಎಂ., ಕನ್ನಡ ಶಾಸನಗಳ ಸಾಂಸ್ಕೃತಿಕ ಅಧ್ಯಯನ, ಪುಟ 367} ಕೆರೆಯ ನೀರು ತೂಬಿನಿಂದ ಹೊರಬರುವ ಜಾಗದಲ್ಲಿರುವ ಗದ್ದೆಗೆ ಹೆಚ್ಚಿನ ಮಹತ್ವ ಇತ್ತು. ಕಾರಣ ಈ ಗದ್ದೆಗೆ ನೀರಿನ ಕೊರತೆ ಬೀಳುತ್ತಿರಲಿಲ್ಲ, ನೀರು ಹಾಯಿಸುವ ಕೆಲಸವೂ ಸುಲಭವಾಗಿರುತ್ತಿತ್ತು. ಇದನ್ನು ತೂಂಬಿನ ಗದ್ದೆ\index{ತೂಂಬಿನ ಗದ್ದೆ} ಎಂದು ಕರೆಯಲಾಗಿದೆ. “ತುಂಬಿನ ಮೊದಲೇರಿಯಲಿ\index{ತುಂಬಿನ ಮೊದಲೇರಿ}.... ಖಂಡುಗ ಗದ್ದೆಯಂ... ಬಿಟ್ಟ ದತ್ತಿ”,\endnote{ ಎಕ 7 ನಾಮಂ 29 ಕಂಬದಹಳ್ಳಿ 1174} “ತುಂಬಿನ ಮೊದಲಲು ಗೞ್ದೆ ಸಲಗೆ ಯೆರಡು”,\endnote{ ಎಕ 6 ಕೃಪೇ 73 ಹಿರಿಕಳಲೆ 12ನೇ ಶ.} “ಬಾಯಿಕಾಲಠಾವಿನಲು\index{ಬಾಯಿಕಾಲಠಾವಿನಲು} ಎಮ್ಮ ಗದ್ದೆ”,\endnote{ ಎಕ 7 ನಾಮಂ 84 ಬೆಳ್ಳೂರು 1269} ಕಲ್ಲತುಂಬಿನ ಮೊದಲಲು ಗದ್ದೆ,\endnote{ ಎಕ 7 ನಾಮಂ 98 ಮುದಿಗೆರೆ 1139} “ಅವ್ವೆಯರ ಕೆರೆಯ ಬಡಗಣ ಕೋಡಿಯಲ್ಲಿನ ತುಂಬಿನ ಬಾಯಿಕಾಲಿಂ ಪಡುವಲು ಗದ್ದೆ, ಹೊಸಕೆರೆಯ ತುಂಬಿನ ವೋದಕುಳಿಯಿಂ ಮೂಡಲು ಕ್ಷೇತ್ರ”,\endnote{ ಎಕ 7 ನಾಮಂ 73 ಬೆಳ್ಳೂರು 1284} ಎಂದು ಶಾಸನಗಳಲ್ಲಿ ಉಲ್ಲೇಖಿಸಲಾಗಿದೆ. ‘ತೂಬಿನ ಬಾಯಿಕಾಲು’ ಎಂಬ ಪ್ರಯೋಗ ಗಮನಾರ್ಹ.

\vskip 2pt

\textbf{ಕೋಡಿಯ ಗದ್ದೆಗಳು:}\index{ಕೋಡಿಯ ಗದ್ದೆಗಳು} ಕೆರೆಯಿಂದ ಹೊರಡುವ ಕೋಡಿಯ ಹಳ್ಳದ ಪಕ್ಕದಲ್ಲಿದ್ದ ಗದ್ದೆಗಳೂ ಕೂಡಾ ಶಾಸನಗಳಲ್ಲಿ ಉಲ್ಲೇಖವಾಗಿವೆ. “ಹಿರಿಯ ಕೆರೆಯ ಪಡುವಣ ಕೋಡಿಯಳರೆಯ ಕಯಿ ಬೆದ್ದಲ್​ ಕೊಳಗ ಹತ್ತು”,\endnote{ ಎಕ 6 ಕೃಪೇ 66 ಮಾಳಗೂರು 1117} “ಅನ್ತಾ ಕೆರೆಯ ಮೂಡಣ ಕೋಡಿಯಲು ಗದ್ದೆ ಮೂವತ್ತು ಕೊಳಗ, ಮತ್ತಮಾ ಕೆರೆಯ ಪಡುವಣ ಕೋಡಿಯಲು ಮೂವತ್ತು ಕೊಳಗಂ”,\endnote{ ಎಕ 6 ಕೃಪೇ 51 ತೊಣಚಿ 10-11ನೇ ಶ.} “ತೆಂಕಲು ಕೋಡಿಯ (ಹಳ್ಳದ) ಹಿರಿಯಪ್ಪಗಳ ಗದ್ದೆ”, ಎಂಬ ಉಲ್ಲೇಖಗಳಿವೆ.\endnote{ ಎಕ 6 ಕೃಪೇ 95 ಭೈರಾಪುರ 1312}

\textbf{ಮಣಲಗದ್ದೆ:}\index{ಮಣಲಗದ್ದೆ} ‘ಮಣಲಗದ್ದೆ’ ಎಂಬ ಒಂದು ವಿಧವಾದ ಗದ್ದೆಯನ್ನು ಶಾಸನಗಳು ಉಲ್ಲೇಖಿಸುತ್ತವೆ. ಇದು ಮರಳಿನಿಂದ ಕೂಡಿದ ಗದ್ದೆಯಾಗಿರಬಹುದು. “ಅಡುವಿನ ಮಣ್ನ ಗೞ್ದೆ,\endnote{ ಎಕ 6 ಕೃಪೇ 73 ಹಿರಿಕಳಲೆ 12ನೇ ಶ.} “ಹಿರಿಯಕೆರೆಯ ಕೆಳಗಣ ಬಾಯಲ ಮಣಲಗದ್ದೆ,\endnote{ ಎಕ 10 ಅರ 59 ಕುರುವಂಕ 1186} “ಕೊಡಗಿಯ ಮಣಲ ಗದ್ದೆ,\endnote{ ಎಕ 10 ಅರ 120 ಹೊಳಲಕೆರೆ 1185} “ಅತ್ತಿಯ ಮಣಲ ಗದ್ದೆ,\endnote{ ಎಕ 10 ಅರ 128 ರಾಮಪುರ 1196} ಎಂದು ಹೇಳಿದೆ. “ದೇವರಿಗೆ ಮಣಲಗದೆಯಲು ಯಿಕ್ಕುವರು”,\endnote{ ಎಕ 10 ಅರಸೀಕೆರೆ 233 ಹಾರನಹಳ್ಳಿ 1265} “ದಡಿಗನ ಕೆರೆಯ ಮಂಣ ಮೆದೆಯ ಭೂಮಿ”,\endnote{ ಎಕ 7 ನಾಮಂ 162 ತಿಬ್ಬನಹಳ್ಳಿ 13ನೇ ಶ.} ಎಂದು ಹೇಳಿದ್ದು ಮಣಲ ಅಥವಾ ಮರಳುನೆಲದಿಂದ ಕೂಡಿದ ಗದ್ದೆಯಾಗಿದ್ದು, ಕಪ್ಪು ಎರೆಯ ಭೂಮಿಯಂತೆ ಇದೂ ಮರಳು ಮಣ್ಣಿನಿಂದ ಕೂಡಿದ ಭೂಮಿಯಲ್ಲಿದ್ದ ಗದ್ದೆ ಇರಬಹುದು. ಮಂಡ್ಯ ಜಿಲ್ಲೆಯಲ್ಲಿ ಕಪ್ಪು ಮಣ್ಣಿನ ಗದ್ದೆ ಹೆಚ್ಚಾಗಿಲ್ಲ. ಮರಳು ಮಣ್ಣಿನ ಗದ್ದೆಗಳು ಹೆಚ್ಚಾಗಿವೆ.

\textbf{ಬೀಜವರಿ ಗದ್ದೆಗಳು\index{ಬೀಜವರಿ ಗದ್ದೆ} \general{\enginline{-}} ಬಿತ್ತುವಟ್ಟ\index{ಬಿತ್ತುವಟ್ಟ (ಬಿತ್ತುವಾಟ)}:} ಬೀಜವರಿ ಗದ್ದೆಯ ಉಲ್ಲೇಖ ಅನೇಕ ಶಾಸನಗಳಲ್ಲಿ ಬರುತ್ತದೆ. ಇದನ್ನು ಬೀಜೋತ್ಪಾದನೆಗಾಗಿ ಮೀಸಲಿರಿಸಿದ ಗದ್ದೆ ಎಂದು ಹೇಳಬಹುದು. ಈ ಇತ್ತೀಚಿನ ವರ್ಷದವರೆಗೆ ಮಂಡ್ಯ ಜಿಲ್ಲೆಯ ರೈತರು ತಮ್ಮ ಜಮೀನಿನಲ್ಲಿ ಬೆಳದ ದವಸಧಾನ್ಯಗಳ ಪೈಕಿ ಒಂದೆರಡು ಗುಂಟೆಯ ಬೆಳೆಯನ್ನು ಬೀಜಕ್ಕೆ ಎಂದು ಬಿಡುತ್ತಿದ್ದರು. “ಗ್ರಾಮ ಕೊಳಗದಲು ಆಯಿಘಂಡುಗ ಗದ್ದೆಯ ಬೀಜವರಿಯನು”,\endnote{ ಎಕ 6 ಕೃಪೇ 95 ಭೈರಾಪುರ 1312} “ಬೀಜವರಿ ಯಿಪ್ಪತ್ತು ಖಂಡುಗ ಗದ್ದೆಯನೂ ಸುರಕ್ಷಿತವಾಗಿ”, “ಬೀಜವರಿ ಎಂಟು ಖಂಡುಗ ಗದ್ದೆಯೂ ಅವರಿಗೆ ಪೂರ್ವ್ವಮರ್ಯ್ಯಾದೆಯಾಗಿ”,\endnote{ ಎಕ 6 ಪಾಂಪು 19 ಸೀತಾಪುರ 1467}, “ಸೇನಬೋವ ರಾಮಾನುಜಗೆ ಬೀಜವರೀ ಗದೆ”,\endnote{ ಎಕ 6 ಪಾಂಪು 163 ಮೇಲುಕೋಟೆ 1319} “ಕೆರೆ ಕೆಳಗೆ ಬೀಜವರಿ ಗದ್ದೆ ಖಂಡುಗ 10”,\endnote{ ಎಕ 6 ಶ‍್ರೀಪ 23 ಶ‍್ರೀರಂಗಪಟ್ಟಣ 1664} “ಚಿಕಸಿಂಗರಾಯಗೆ ಕೆರೆಯ ಕೆಳಗೆ ಬೀಜವರಿ ಖ 1, ಕಾಲುವೆಯ ಕೆಳಗೆ ಬೀಜವರಿ ಖ 1” \endnote{ ಎಕ 6 ಶ‍್ರೀಪ 101 ಅರಕೆರೆ 17ನೇ ಶ.}, ಈ ರೀತಿ ಅನೇಕ ಶಾಸನಗಳಲ್ಲಿ ಬೀಜವರಿ ಗದ್ದೆಯ ಉಲ್ಲೇಖಗಳಿವೆ.

ಗಂಗರ ಕಾಲದ ಶಾಸನಗಳಲ್ಲಿ ಬೀಜವರಿ ಗದ್ದೆಗೆ ‘ಬಿತ್ತುವಟ್ಟ’\index{ಬಿತ್ತುವಟ್ಟ (ಬಿತ್ತುವಾಟ)} ಎಂಬ ಪ್ರಯೋಗ ಹೆಚ್ಚಾಗಿದೆ. “ಹತ್ತು ಕೊಳಗ ಮಣ್ನುಂ ಕೆರೆಗೆ ಬಿತ್ತುವಟ್ಟಮುಮಂ ಬಿಟ್ಟರ್​”,\endnote{ ಎಕ 7 ಮ 67 ಬೇಲೂರು 997} “ಆರಣಿಯೂರ ಕೆರೆಗಳ ಬಿತ್ತುವಟ್ಟವಂ ಬಿಟ್ಟರ್​”,”\endnote{ ಎಕ 7 ಮ 117 ಶೆಟ್ಟಿಹಳ್ಳಿ 10ನೇ ಶ.} “ಅರಿಯಮ್ಮಸೆಟ್ಟಿ ಬಿತ್ತುವಟ್ಟಮಂ ಕೊಟ್ಟ”,\endnote{ ಎಕ 7 ನಾಮಂ 99 ಆರಣಿ 972} “ಅಸವಯ್ಯನು ದುಗ್ಗಯ್ಯಂ ನುಳಮ್ಬನುಂ ಈ ಮುವರುಮಿಳ್ದು ಬಿತ್ತವಟ್ಟಕೆ ನಟ್ಟ ಕಲ್ಲು”\endnote{ ಎಕ 7 ಮವ 17 ರಾವಂದೂರು 9-10ನೇ ಶ.} “ನೀರುಸಂಧಿಗೆ ಬಿತ್ತುವಟ್ಟವಾಗಿ”\endnote{ ಎಕ 7 ನಾಮಂ 83 ಬೆಳ್ಳೂರು 1269} “ಇವೆಲ್ಲ ದೇವಕೆರೆಗೆ ಸಲ್ವುದು ಖ॥0 ಬಿತ್ತುವಟ್ಟ”\endnote{ ಎಕ 6 ಶ‍್ರೀಪ 122 ವೊಡೇರಿ 11ನೇ ಶ.} ಎಂದು ಹೇಳಿದೆ.

\section*{ಗದ್ದೆಯ ಅಳತೆಗಳು}

ಜಿಲ್ಲೆಯ ಶಾಸನಗಳಲ್ಲಿ ಗದ್ದೆಯನ್ನು ಸಲಗೆ, ಕಂಬ, ಕಮ್ಮ, ಕೊಳಗ, ಕುಳ/ಗುಳ, ಗಳ/ಗಳೆ ಮತ್ತು ಖಂಡುಗದ ಅಳತೆಯ ಮಾನ\-ದಲ್ಲಿ ಅಳೆದಿರುವುದು ಕಂಡು ಬರುತ್ತದೆ. ಆದರೆ ಗದ್ದೆಯನ್ನು ಹೆಚ್ಚಾಗಿ ಸಲಗೆಯ ಲೆಕ್ಕದಲ್ಲಿ ಅಳೆಯಲಾಗಿದೆ. ಗದ್ದೆಯಲ್ಲಿ ಉತ್ಪಾದನೆ\-ಯಾಗುವ ಭತ್ತದ ಪ್ರಮಾಣದ ಮೇಲೆ ಈ ಅಳತೆಯನ್ನು ಮಾಡಲಾಗುತ್ತಿತ್ತೆಂದು ವಿದ್ವಾಂಸರು ಅಭಿಪ್ರಾಯ ಪಟ್ಟಿದ್ದಾರೆ.\endnote{ Shivanna, Dr. K.S., The Agrarian System of Karnataka, pp.65-66}

\textbf{ಸಲಗೆ: ಜಿಲ್ಲೆಯ ಶಾಸನಗಳಲ್ಲಿ ಗದ್ದೆಯನ್ನು ಹೆಚ್ಚಾಗಿ ಸಲಗೆಯ\index{ಸಲಗೆ (ಸಲಿಗೆ)} ಲೆಕ್ಕದಲ್ಲೇ ಹೇಳಿದೆ.} ಸಲಗೆಯ ನಂತರ ಅದಕ್ಕಿಂತ ಕಡಿಮೆ ಅಳತೆಯ ಕೊಳಗ ಎಂಬ ಅಳತೆಯ ಮಾನವನ್ನು ಬಳಸಿದೆ. ‘ನಿವೇದ್ಯದ ಗದ್ದೆ ಸಲಗೆ 1’,\endnote{ ಎಕ 7 ನಾಮಂ 63 ಲಾಳನಕೆರೆ 1165} ‘ಗದ್ದೆ ಸಲಗೆ ಎರಡು, ಗದ್ದೆ ಸಲಗೆ ಮೂರು’,\endnote{ ಎಕ 7 ನಾಮಂ 7 ನಾಗಮಂಗಲ 1134} ‘ಗೞ್ದೆ ಸಲಗೆ ನಾಲ್ಕುಂ’,\endnote{ ಎಕ 6 ಕೃಪೇ 73 ಹಿರಿಕಳಲೆ 12ನೇ ಶ.} ‘ಗದ್ದೆ ಸಲಿಗೆ\index{ಸಲಗೆ (ಸಲಿಗೆ)} ಮುವತ್ತು’,\endnote{ ಎಕ 7 ನಾಮಂ 33 ಕಂಬದಹಳ್ಳಿ 1118} “ಹಿರಿಯ ಕೆರೆಯ ಕೆಳಗೆ ಎರಡು ಸಲಗೆ ಗದ್ದೆ, ಕಿರುಕೆರೆಯ ಕೆಳಗೆ ನಾಲಕು ಸಲಗೆ ಗದ್ದೆ”,\endnote{ ಎಕ 6 ಕೃಪೇ 48 ತೊಣಚಿ 1191} “ಹೊಯ್ಸಳೇಶ್ವರ ದೇವರ್ಗ್ಗೆ ಬಿಟ್ಟ ಗದ್ದೆ ಸಲಗೆ 2, ಕೊಳಗ\index{ಕೊಳಗ} 10”,\endnote{ ಎಕ 6 ಕೃಪೇ 42 ತೆಂಗಿನಘಟ್ಟ 1117} “ಗದ್ದೆ ಸಲಗೆ 2 ಕೊಳಗ 14”,\endnote{ ಎಕ 6 ಕೃಪೇ 42 ತೆಂಗಿನಘಟ್ಟ 1117} ಮೊದಲಾದ ಉಲ್ಲೇಖಗಳಿವೆ. ಬಸರಾಳು ಶಾಸನದಲ್ಲಿ “ಬಸರಿವಾಳದ ಹಿರಿಯಕೆರೆಯ ಕೆಳಗೆ ಗದ್ದೆ ಸ 6 ಕೊ 10, ಹಡವನಹಳ್ಳಿಯ ಕೆರೆಯ ಕೆಳಗೆ ಗದ್ದೆ ಸ 6 ಕೊ 10, ಗುಜ್ಜವ್ವೆ ನಾಯಕಿತ್ತಿಯ ಕೆರೆಯ ಕೆಳಗೆ ಸ 4, ಎಕ್ಕೆಹಟ್ಟಿಯ ಕೆರೆಯ ಕೆಳಗೆ ಸ 3 ಅನ್ತು ಗದ್ದೆ ಸಲಗೆ 20” ಎಂದು ಹೇಳಿದೆ.\endnote{ ಎಕ 7 ಮಂ 29 ಬಸರಾಳು 1234} ಇದನ್ನು ಲೆಕ್ಕ ಹಾಕಿದಾಗ 19 ಸಲಗೆ 20 ಕೊಳಗ ಆಗುತ್ತದೆ. 20 ಕೊಳಗಕ್ಕೆ ಒಂದು ಸಲಗೆ ಎಂಬ ಲೆಕ್ಕದ ಮೇಲೆ ಒಟ್ಟು 20 ಸಲಗೆ ಗದ್ದೆ ಎಂದು ಹೇಳಿದೆ. ಇದೇ ರೀತಿ 20 ಕೊಳಗಕ್ಕೆ\index{ಕೊಳಗ} ಒಂದು ಸಲಗೆ\index{ಸಲಗೆ (ಸಲಿಗೆ)} ಎಂಬ ಲೆಕ್ಕ ಅರಸೀಕೆರೆ ಶಾಸನದಲ್ಲಿದೆ.\endnote{ ಜಗದೀಶ, ಡಾ॥, ಮಂಡ್ಯ ಜಿಲ್ಲೆಯ ಮಾಪನ ಶಾಸ್ತ್ರ, ಮಂಡ್ಯ ಜಿಲ್ಲೆಯ ಇತಿಹಾಸ ಮತ್ತು ಪುರಾತತ್ವ, ಪುಟ 518-19} ಅರಸೀಕೆರೆಯ ಇನ್ನೊಂದು ಶಾಸನದಲ್ಲಿ “ಅಗ್ಗುಳಿಯ ಕೆರೆಯ ನಡುಬಯಲಲ್ಲಿ ಸಲಗೆಯಾರಕ್ಕಂ 6 ಕಂಬ 240” ಎಂದು ಹೇಳಿದೆ. ಇದರಿಂದ 40 ಕಂಬಕ್ಕೆ ಒಂದು ಸಲಗೆ ಎಂಬ ಅಳತೆ ಸಿಗುತ್ತದೆ.\endnote{ ಎಕ 10 ಅರ 7 ಅರಸೀಕೆರೆ 1189} “ಹಿರಿಯಕೆರೆಯ ಮೂಡಗೋಡಿಯ ಬಯಲ ನಡುಸ್ಥಳದ ಗದ್ದೆ ಸಲಗೆ 7 ಕ್ಕಂ ಕಂಬವಿಂನೂರೆಂಭತ್ತು 280” ಇಲ್ಲೂ ಕೂಡಾ 40 ಕಂಬಕ್ಕೆ ಒಂದು ಸಲಗೆ ಎಂಬ ಅಳತೆ ಸಿಗುತ್ತದೆ.\endnote{ ಎಕ 10 ಅರ 23 ಅರಸೀಕೆರೆ 1173}

\vskip 2pt

\textbf{ಸಲಗೆಯ ಅಳತೆ ಪ್ರಮಾಣವನ್ನು ಕೆಲವು ಶಾಸನಗಳಲ್ಲಿ ನೀಡಲಾಗಿದೆ. “ಬೀಜವರಿ ಸಲಗೆ\index{ಸಲಗೆ (ಸಲಿಗೆ)} ವೊಂದಕ್ಕೆ ಮೂವತ್ತೆರಡು\general{\break } ಮೆಟ್ಟಿನಗಳೆಯಲ್ಲಿ\index{ಮೆಟ್ಟಿನಗಳೆ} ನಾಲ್ವತ್ತುಯೆಂಟು ಕಂಬದ ಮರ್ಯದೆಯಲು ಹನ್ನೆರಡು ಸಲಗೆ ಗದ್ದೆಯನಳ್ಸಿಕೊಂಡು ಆ ಚತುಸೀಮೆಯಲು ಕಲ್ಲು ನಡಿಸಿಕೊಂಡೆವಾಗಿ”} ಎಂದು ಬೆಳ್ಳೂರು ಶಾಸನದಲ್ಲಿ ಹೇಳಿದ್ದು, ಒಂದು ಸಲಗೆಗೆ 32 ಮೆಟ್ಟಿನ ಗಳೆ ಅಥವಾ 48 ಕಂಬ ಎಂಬ ಅಳತೆ ದೊರೆಯುತ್ತದೆ.\endnote{ ಎಕ 7 ನಾಮಂ 82 ಬೆಳ್ಳೂರು 1269} “ಬ್ರಹ್ಮಸಮುದ್ರವಾದ ಹಿರಿಯಕೆರೆಯ ಕೆಳಗೆ ಮೊದಲೇರಿಯ ಹಳುಗಿನಲು ಏಕಸ್ಥಳವಾಗಿ\break ಮೂವತ್ತಯೆರಡು ಮೆಟ್ಟಿನಗಳೆಯಲು ಕಂಬ ನಾನೂರಕ್ಕಂ ದೇವರ ಪಡಿಯಕೊಳಗದಲು ಬೀಜವರಿಯ ಎಂಟು ಸಲಗೆ ಗದ್ದೆಯನು” ಎಂದು ಚನ್ನರಾಯಪಟ್ಟಣ ತಾಲ್ಲೂಕಿನ ನುಗ್ಗೆಹಳ್ಳಿ ಶಾಸನದಲ್ಲಿ ಇದೇ ಅಳತೆಯನ್ನು ಹೇಳಿದೆ.\endnote{ ಎಕ 10 ಚರಾಪ 112 ನುಗ್ಗೇಹಳ್ಳಿ 1252} “ಅಣ್ನಿಯ ಸಮುದ್ರ\-ದಲು ಗದ್ದೆ ಸಲಗೆಯ ಭೂಮಿಗೆ ಕೊಳಗ ಹತ್ತು” ಎಂದು ಮರಸೆ ಶಾಸನದಲ್ಲಿ ಹೇಳಿದೆ.\endnote{ ಎಕ 5 ಮೈ 188 ಮರಸೆ 1191} ಅರಕೆರೆ ಶಾಸನದಲ್ಲಿ “\hbox{ಕೊಡಂಗೆ} ಗೞ್ದೆ ಕಡಹು ಮೂಱು, ಸಲಗೆ ನಾಲ್ಕು, ಬೆದ್ದಲೆ\index{ಬೆದ್ದಲು (ಬೆದ್ದಲೆ)} ಸಾಯಿರದಯಿನೂಱು” ಎಂದು ಹೇಳಿದ್ದು, ಗದ್ದೆಯನ್ನು ಕಡಹು ಲೆಕ್ಕದಲ್ಲಿ ಅಳೆದಿದ್ದು, ಇದು ಸಲಗೆಗಿಂತ ದೊಡ್ಡ ಅಳತೆಯಾಗಿದೆ ಎಂದು ಹೇಳಬಹುದು.\endnote{ ಎಕ 6 ಶ‍್ರೀಪ 113 ಅರಕೆರೆ 1108}

\vskip 2pt

\textbf{ಖಂಡುಗ:}\index{ಖಂಡುಗ} ಜಿಲ್ಲೆಯ ಶಾಸನಗಳಲ್ಲಿ ಖಂಡುಗ ಲೆಕ್ಕದಲ್ಲಿ ಗದ್ದೆಯ ಅಳತೆಯನ್ನು ಹೇಳಿದೆ. ಗಂಗರ ಕಾಲದ ಪ್ರಾಚೀನ ಶಾಸನಗಳಲ್ಲಿ ಖಂಡುಗದ ಉಲ್ಲೇಖ ಬಹಳ ಕಡೆ ಬಂದಿದೆ. “ಶ‍್ರೀವುರದಾ ವಯಲುಳ್ಕಮ್ಮರ್ಗ್ಗಟ್ಟಿನಲ್ಲಿ\index{ಶ‍್ರೀವುರದಾ ವಯಲು} ಇರ್ಕ್ಕಣ್ಡುಗಂ ಕಳನಿ, ಪೆರ್ಗ್ಗೆರೆಯಾ ಕೆಳಗೆ ಅರುಗಣ್ಡುಗಂ ಎರೆ”, ಪುಲಿಗೆರೆಯಾ ಕೋಯಿಲ್ಗೋಡಾ ಎಡ ಇರ್ಪ್ಪತ್ತು ಗಣ್ಡುಂಗಂ ಬ್ಬೆದೆ,\endnote{ ಎಕ 7 ನಾಮಂ 149 ದೇವರಹಳ್ಳಿ 776} ಎಂದು ಹೇಳಿದ್ದು, ಕಳನಿ, ಎರೆ, ಬೆದೆ ಇವು ಗದ್ದೆಯನ್ನು ಸೂಚಿಸುತ್ತಿವೆ. “ಮೞ್ತಿ(ಳ್ತಿ) ಕಾಲಙ್ಗಳೊಳಿರ್ಕ್ಕಣ್ಡುಗ ಮಣ್ನ ಕೊಟ್ಟರಾ ಮಣ್ನ”,\endnote{ ಎಕ 7 ಮ 42 ಆತಕೂರು 949} “ದೇವರಿಗೆ ಬಿಟ್ಟ ಗದ್ದೆ ಕಣ್ಡುಗ 15”,\endnote{ ಎಕ 6 ಕೃಪೇ 37 ಕಿಕ್ಕೇರಿ 1095} ಶಿವಕೆರೆಯೊಳು ಪನ್ನಿಕ್ಕಂಡುಗ ಬಿಟ್ಟ ನಿಮ್ಮಣ್ನು, ಹರಿಯ ಕೆರೆಯಲ್ಲಿ ಕಂಡುಗ \hbox{ವೆದೆಯ}(ಗದೆಯ)ಬಿಟ್ಟ”,\endnote{ ಎಕ 6 ಪಾಂಪು 44 ಕನ್ನಂಬಾಡಿ 1114-15} “ಇರ್ಪ್ಪತ್ತುನಾಲ್ಕಂಡುಗಂ ನೀರ್ಣ್ನೆಲನಂ”,\endnote{ ಎಕ 7 ನಾಮಂ 26 ಕಂಬದಹಳ್ಳಿ 1168} ಎಂದು ಹೇಳಿದೆ. “ಗದ್ದೆ ಖ 3”,\endnote{ ಎಕ 7 ಮ 113 ಬೊಪ್ಪಸಂದ್ರ 1363} “ಹಿರಿಯರ್ಸಿನಕೆರೆಯಲಿ ಖ 1 ಗದ್ದೆಯ ವುಂಬಳಿ”,\endnote{ ಎಕ 7 ಮ 131 ದೊಡ್ಡರಸಿನಕೆರೆ 1437} “ಹರಹಿನ ಕಾಲುವೆಯ ಕೆಳಗಣ ಶ‍್ರೀರಂಗಪುರದ ಮಹಾಜನಗಳಿಂದ ನಮಗೆ ಕ್ರಯವಾಗಿ ಬಂದ ಹತ್ತು ಖಂಡುಗ ಗದ್ದೆ”,\endnote{ ಎಕ 6 ಶ‍್ರೀಪ 31 ಶ‍್ರೀರಂಗಪಟ್ಟಣ 1517} “ಮದ್ದೂರು ತಾವರೆಕಟ್ಟೆಯ ಕೆಳಗೆ ಕೊಟ್ಟ ಗದ್ದೆ ಖಂ 5”,\endnote{ ಎಕ 7 ಮ 64 ಹೊನ್ನಲಗೆರೆ 1623} ಎಂದು ಹೇಳಿದೆ.

\vskip 2pt

ಶ‍್ರೀರಂಗಪಟ್ಟಣ ತಾಲ್ಲೂಕು, ಬೆಳಗೊಳ ಶಾಸನದಲ್ಲಿ ಖಂಡುಗವನ್ನು ದೊಡ್ಡ ಪ್ರಮಾಣದಲ್ಲಿ ಹೇಳಿದೆ. “ಬಳಗುಳದಲೆ ಜಂನರಾಯ ಹೆಬಾರುವರ ಕಯ್ಯ ಕ್ರಯವಾಗಿ ಬಂದದು ಆ. ಗಟನ \textbf{ಗದೆಯಲಿ ಖ 600} ಚಿಕ್ಕ ಹೆಬಾರುವರ ಜಂನಯನಿಂದ ಕ್ರಯವಾಗಿ ಬಂದ \textbf{ಗದೆ 55, ಉಭಯಂ ಖ 11... ಉಭಯ ಕೂಡಿದ ಗದೆ 2550} ಅಂತು ಕ... ಬಳಗುಳದಲಿ ತೋಟವ್ರಿತ್ತಿ 2, \textbf{ಗದ್ದೆ ಖ 2550”} ಎಂದು ಹೇಳಿದೆ.\endnote{ ಎಕ 6 ಶ‍್ರೀಪ 71 ಬೆಳಗೊಳ 1598} ಬಹುಶಃ ಆ ಭೂಮಿಯಲ್ಲಿ ಬರುತ್ತಿದ್ದ ಬೆಳೆಯ ಪ್ರಮಾಣದ ಮೇಲೆ ಈ ಅಳತೆಯನ್ನು ಹೇಳಿರಬಹುದು.

\vskip 2pt

ಗದ್ದೆಗೆ ಮಣ್ಣು ಮತ್ತು ತರಿ ಎಂಬ ಶಬ್ದವನ್ನು ಉಪಯೋಗಿಸಿರುವುದು, ಗದ್ದೆಯ ಅಳತೆಗೆ ಖಂಡುಗ, ಕೊಳಗ ಎರಡೂ ಅಳತೆಯನ್ನು ಒಟ್ಟಿಗೆ ಉಪಯೋಗಿಸಿರುವುದು ಸಾಸಲು ಶಾಸನದಲ್ಲಿ ಕಂಡುಬರುತ್ತದೆ. “ಭೋಗೇಶ್ವರ ದೇವ(ರಿಗೆ) ಮ. ಖಣ್ಡಗ 2....... ತರಿ ಖಣ್ಡಗ 4, ಅನ್ತಾ ಕೆರೆಯ ಮೂಡಣ ಕೋಡಿಯಲಿ ಗದ್ದೆ ಮೂವತ್ತು ಕೊಳಗ, ಮತ್ತಮಾ ಕೆರೆಯ ಪಡುವಣ ಕೋಡಿಯಲಿ ಮೂವತ್ತು ಕೊಳಗಂ, ಮೂಡಣ ಏರಿಯ ಗದ್ದೆಯಲ್ಲಿ ಖ 1” ಎಂದು ಹೇಳಿದೆ.\endnote{ ಎಕ 6 ಕೃಪೇ 59 ಸಾಸಲು 1121} ಇಲ್ಲಿ ಒಟ್ಟು ಆರು ಖಂಡುಗ ಗದ್ದೆ ಎಂದು ಹೇಳಿದ ನಂತರ ಅದನ್ನು ಅರವತ್ತು ಕೊಳಗ ಎಂದು ಹೇಳಲಾಗಿದೆ ಎಂದು ಊಹಿಸಬಹುದು. ಅಲ್ಲಿಗೆ ಒಂದು ಖಂಡುಗಕ್ಕೆ ಹತ್ತು ಕೊಳಗದಂತೆ ಲೆಕ್ಕ ಹಾಕಲಾಗಿದೆ ಎಂದು ಹೇಳಬಹುದು.

\vskip 2pt

\textbf{ಕಂಬ\index{ಕಂಬ}/ಕಂಮ/ಕಮ್ಮ\index{ಕಮ್ಮ}/ಕವ:}\index{ಕವ} ಕಂಬದ ಲೆಕ್ಕದಲ್ಲೂ ಕೂಡಾ ಗದ್ದೆಯನ್ನು ಅಳತೆ ಮಾಡಲಾಗುತ್ತಿತ್ತು. “ತಗ್ಗಿನಲ್ಲಿ ಗದ್ದೆ ಕಂಬ 60” “ಖಂಡೀಕದ ಗದೆ ಕವ ಎಂಭತ್ತನು”,\endnote{ ಎಕ 6 ಶ‍್ರೀಪ 108 ಅರಕೆರೆ 13ನೇ ಶ.} “ಅಕರವಾಗಿ ಕೊಂಡು ಕೊಟ್ಟ ಗದ್ದೆ ಕಂಬ 60, ಮಾಯಂಣ ಕೊಂಡು ಕೊಟ ಗದೆ ಕಂಬ 50”\endnote{ ಎಕ 6 ಶ‍್ರೀಪ 110 ಅರಕೆರೆ 1512}. “ಬಳ್ಳೀ ಮಡವಯ ಖಂಡೀಕದ ಗದೆ ಕವ\index{ಖಂಡೀಕದ ಗದೆ ಕವ} ಎಂಭತ್ತನು ಕ್ರಯದಾನವಾಗಿ ಕೊಂಡು”,\endnote{ ಎಕ 6 ಶ‍್ರೀಪ 108 ಅರಕೆರೆ 13ನೇ ಶ.} ಎಂದು ಅರಕೆರೆ ಶಾಸನದಲ್ಲಿ ಹೇಳಿದೆ. ಆದುದರಿಂದ ಕಂಬ ಕವ ಎರಡೂ ಒಂದೇ ಅಳತೆಯ ಮಾನಗಳು ಎಂದು ಹೇಳಬಹುದು ಕಂಮ ಎಂದರೂ ಕಂಬವೇ ಎಂದು ಹೇಳಬಹುದು. “ಬೆಳ್ದಲೆ ಕಮ್ಮ 500, ಬೆದ್ದಲೆ ಕಮ್ಮ 400” ಎಂದು ಅಗ್ರಹಾರ ಬೆಳಗುಲಿ ಶಾಸನದಲ್ಲಿ ಹೇಳಿದೆ.\endnote{ ಎಕ 10 ಚರಾಪ 98 ಅಗ್ರಹಾರಬೆಳಗಲಿ 1133}

\textbf{ಕೊಳಗ:}\index{ಕೊಳಗ} 9–10ನೇ ಶತಮಾನದ ಶಾಸನಗಳಲ್ಲಿ ಶಾಸನಗಳಲ್ಲಿ ಗದ್ದೆಯನ್ನು ಕೇವಲ ಕೊಳಗದಲ್ಲಿ ಮಾತ್ರ ಅಳೆದಿರುವುದು ಕಂಡಬರುತ್ತದೆ. “ಹತ್ತು ಕೊಳಗ ಮಣ್ನುಂ”,\endnote{ ಎಕ 7 ಮಂ 67 ಬೇಲೂರು 997} “ತಮ್ಮ ಮೊದಲ ಒಕ್ಕಲೊಳ್​ ಪತ್ತು ಕೊಳಗ ಗದ್ದೆ”,\endnote{ ಎಕ 7 ಮವ 60 ಮಾರೆಹಳ್ಳಿ 1014} “ದೇವ\-ಪರ್ಬ್ಬ ನಿಮಿತ್ತ ಮಣ್ನುಂ ಕೊ 30”,\endnote{ ಎಕ 6 ಕೃಪೇ 51 ತೊಣಚಿ 10-11ನೇ ಶ.} “ಬ್ರಾಹ್ಮಣರ್ಗ್ಗೆ ಗೞ್ದೆ ಕೊಳಗ ಮೂರು”,\endnote{ ಎಕ 6 ಶ‍್ರೀಪ 113 ಅರಕೆರೆ 1108} “ಆಯ್ವತ್ತು ಕೊಳಗ ಗದ್ದೆಯಂ”,\endnote{ ಎಕ 6 ಪಾಂಪು 15 ಕ್ಯಾತನಹಳ್ಳಿ 1175} “ಕೆರೆಯ ಮೊದಲ ಗದ್ದೆ ಅತ್ತಿಯಗದ್ದೆ ಸಹಿತ ಸೂ.ಕೊ 10, ಮತ್ತಿಯಕೆರೆಯಲಿ ಕೊ10”,\endnote{ ಎಕ 7 ನಾಮಂ 62 ಲಾಳನಕೆರೆ 1219} ಎಂದು ಹೇಳಿದ್ದು ಇಲ್ಲಿ ‘ಕೊ’ ಎಂಬುದು ಕೊಳಗ ಎಂಬುದರ ಸಂಕ್ಷೇಪ. “ಕೀಳೇರಿಯ ಗದ್ದೆ ಸಲಗೆ ಒಂದು ಕೊಳಗ ಹತ್ತು”,\endnote{ ಎಕ 7 ನಾಮಂ 72 ಅಳೀಸಂದ್ರ 1183} “ಮಾಚೇಶ್ವರ ಮೂಲಸ್ಥಾನಕ್ಕೆ\break ಸ 2 ಕೊ10”,\endnote{ ಎಕ 7 ನಾಮಂ 81 ಬೆಳ್ಳೂರು 1223} ಎಂದು ಹೇಳಿದ್ದು, ಒಂದು ಸಲಗೆಯು ಹತ್ತು ಕೊಳಗಕ್ಕಿಂತ ಜಾಸ್ತಿ ಇತ್ತು ಎಂಬ ಅರ್ಥ ಬರುತ್ತದೆ.

ಒಂದೇ ಶಾಸನದಲ್ಲಿ ಕೆಲವು ಸಲ ಗದ್ದೆಯನ್ನು ಕೊಳಗ ಸಲಗೆ ಎರಡರಲ್ಲೂ ಅಳೆಯಲಾಗಿದೆ. “ಹಿರಿಯ ಕೆರೆಯಲಿ ಗದ್ದೆ\break ಸ 3, ಬೆದ್ದವ್ವೆಯ ಕೆರೆಯಲಿ ಕೊ 10, ಹಾಡುವನ ಕೆರೆಯಲಿ ಕೊ10 ಹೊಲಗೆರೆಯಲಿ ಕೊ10”,\endnote{ ಎಕ 7 ನಾಮಂ 173 ಭೀಮನಹಳ್ಳಿ 1230} ಎಂದು ಭೀಮನಹಳ್ಳಿ ಶಾಸನದಲ್ಲಿ, “ಲಿಂಗಮುದ್ರೆಯ ನಟ್ಟ ಕಲ್ಲಿಂ ಗದ್ದೆ ಕೊಳಗ 10, ಕಂಬೆಗೆರೆಯ ತೂಬಿನ ಮೊದಲಲಿ ಗದೆ ಸ 1 ”\endnote{ ಎಕ 7 ನಾಮಂ 90 ಬೆಳ್ಳೂರು 15ನೇ ಶ.}, “ಮಾಣಿಮಾಡೆಗೆ ಬಿಟ್ಟ ಗದ್ದೆ ಹತ್ತುಕೊಳಗ”,\endnote{ ಎಕ 7 ಮವ 39 ಹುಲ್ಲೇಗಾಲ 1177} ಬಸದಿಯ ಗದ್ದೆಯ ತೆಂಕಣ ಹಳ್ಳ ಕೊ 10, ತೆಂಕಣ ಕೋಡಿಯ ಕೆಳಗೆ ಮೂಡಣ ಹಳ್ಳಿಯ ಸೊಲ್ಲಗೆಯ ಗದೆ”,\endnote{ ಎಕ 7 ನಾಮಂ 90 ಬೆಳ್ಳೂರು 15ನೇ ಶ.} ಎಂದು ಹೇಳಿದ್ದು ಇದರಿಂದ ಖಂಡುಗ, ಕೊಳಗ, ಸಲಗೆ ಇವು ಗದ್ದೆಯ ಅಳತೆಗಳಾಗಿವೆ.

\textbf{ಗುಳ/ಕುಳ:}\index{ಗುಳ}\index{ಕುಳ} ಗದ್ದೆಯನ್ನು ಕುಳ ಅಥವಾ ಗುಳದಲ್ಲಿಯೂ ಅಳೆದಿದೆ. “ಸಾವಿಯಬ್ಬೇಸ್ವರಕ್ಕೆ ಬಿಟ್ಟ ಮಣ್ನಾವುದೆಂದೊಡೆ ಸಿವರಾಯ ಕೆರೆಯೊಳ್ಪದಿರ್ಕ್ಕುಳವೆಡೆ”,\endnote{ ಎಕ 6 ಪಾಂಪು 43 ಕನ್ನಂಬಾಡಿ 10-11ನೇ ಶ.} ಎಂದು ಕನ್ನಂಬಾಡಿ ಶಾಸನದಲ್ಲಿ, “ಹದಿಮೂ(ರು) ಗುಳ ಗದ್ದೆ” ಎಂದು ಹುಲ್ಲೇಗಾಲ ಶಾಸನದಲ್ಲಿ ಹೇಳಿದೆ.\endnote{ ಎಕ 7 ಮವ 36 ಹುಲ್ಲೇಗಾಲ 1291} “ಸೇನಬೋವ ರಾಮಾನುಜಗೆ ಬೀಜವರೀ ಗದೆ ಖಂ 1.0 ಅಕ್ಷರದಲೂ ಆಯಿಗುಳದ ಗದೆಯನು” ಎಂದು ಮೇಲುಕೋಟೆ ಶಾಸನದಲ್ಲಿ ಹೇಳಿದೆ.\endnote{ ಎಕ 6 ಪಾಂಪು 163 ಮೇಲುಕೋಟೆ 1469}\textbf{ಒಂದು ಖಂಡುಗ ಗದ್ದೆಗೆ ಐದು ಗುಳ ಎಂಬ ಲೆಕ್ಕ ಇದರಿಂದ ಹೊರಡುತ್ತದೆ.} ರೈತರು ಹೊಲ ಉಳಲು ನೇಗಿಲಿಗೆ ಸೇರಿಸುವ ಒಂದು ಗೊತ್ತಾದ ಉದ್ದದ ಕಬ್ಬಿಣದ ಪಟ್ಟಿಗೆ ಗುಳ/ಕುಳ ಎಂದು ಹೇಳುತ್ತಾರೆ.

\textbf{ಗಳೆ/ಗಳ:}\index{ಗಳ (ಗಳೆ)}\index{ಗಳ (ಗಳೆ)} ಗಳೆ, ಗಳ ಎಂಬ ಅಳತೆಯು ಜಿಲ್ಲೆಯ ಶಾಸನಗಳಲ್ಲಿ ಉಲ್ಲೇಖವಾಗಿದೆ. ಮೇಲೆ ತಿಳಿಸಿದಂತೆ ಬೆಳ್ಳೂರು ಶಾಸನದಲ್ಲಿ \textbf{ಸಲಗೆವೊಂದಕ್ಕೆ ಮೂವತ್ತೆರಡು ಮೆಟ್ಟಿನಗಳೆಯಲ್ಲಿ\index{ಮೆಟ್ಟಿನಗಳೆ} ನಾಲ್ವತ್ತುಯೆಂಟು ಕಂಬದ ಮರ್ಯಾದೆ} ಎಂದು ಹೇಳಿದೆ. \textbf{ಒಂದು ಸಲಗೆಗೆ 32 ಮೆಟ್ಟಿನ ಗಳೆ ಎಂಬ ಅಳತೆಯು ಸಿಗುತ್ತದೆ}. ಭೈರಾಪುರ ದೇವಾಲಯದ ಅಧಿಷ್ಠಾನದ ಮೇಲೆ \textbf{“ಅರೆಗಳೆ”}\index{ಅರೆಗಳೆ} ಎಂಬ ಒಂದು ಅಳತೆಯನ್ನು ಗುರುತಿಸಿದೆ. ಅದರ ಉದ್ದ ಹತ್ತು ಮೊಳ ಅಥವಾ 15 ಅಡಿ ಎಂದು ತಿಳಿದುಬರುತ್ತದೆ.\endnote{ ರಾಜೇಂದ್ರಪ್ಪ, ಸಿ.ವಿ., ಪ್ರೇಕ್ಷಣೀಯ ಸ್ಥಳಗಳು, ಮಂಡ್ಯ ಜಿಲ್ಲಾ ಗೆಜೆಟಿಯಾರ್​ (ಪರಿಷ್ಕೃತ), ಪುಟ 948} ಅರ್ಧ ಗಳೆಗೆ\break ಹದಿನೈದು ಅಡಿಯಾದರೆ, ಒಂದು ಗಳೆಗೆ 30 ಅಡಿ ಉದ್ದವಾಗುತ್ತದೆ. ನಿರ್ದಿಷ್ಟ ಉದ್ದದ ಬಿದಿರಿನ ಸೀಳಿಗೆ “ಗಳ” ಎಂದು ಹಳ್ಳಿಗಳ ಕಡೆ ಹೇಳುತ್ತಿದ್ದರು, ಈಗಲೂ ಹೇಳುತ್ತಾರೆ. ಅದರಲ್ಲಿ ಬಿದಿರಿನ ಇಷ್ಟೇ ಗೇಣುಗಳು ಇರಬೇಕೆಂಬ ಲೆಕ್ಕ ಇತ್ತು. \textbf{“ಗಂಗನ ಗಳೆ ಗದ್ದೆ ಮತ್ತರೊಂದು”,\index{ಮತ್ತರು} “ಗದ್ದೆ ಗಂಗನ ಗಳೆಯಲು ಕಂಬ\index{ಕಂಬ} ಹಂನೆರಡು”} ಎಂಬ ಒಂದು ಅಳತೆಯು ಅಮೃತಾಪುರ ಶಾಸನದಲ್ಲಿ ಉಲ್ಲೇಖವಾಗಿದೆ.\endnote{ ಹಿರೇಮಠ್​ ಡಾ॥ ಆರ್​.ಸಿ., ಕಲಬುರ್ಗಿ ಡಾ॥ ಎಂ.ಎಂ., ಕನ್ನಡ ಶಾಸನ ಸಂಪದ, ಪುಟಗಳು 83,84 ಮತ್ತು 137}


\section*{ಬೆದ್ದಲು/ಕೆಯ್​/ಹೊಲ}
\index{ಬೆದ್ದಲು (ಬೆದ್ದಲೆ)}\index{ಕೆಯ್​}\index{ಹೊಲ}

ಬೆದ್ದಲು ಎಂದರೆ ಮಳೆಯ ಆಶ್ರಯದಲ್ಲಿ ವ್ಯವಸಾಯ ಮಾಡುವ ಪ್ರದೇಶ ಎಂದು ಹೇಳಬಹುದು. ಕಾರಣ ಇದನ್ನು ಗದ್ದೆಗಿಂತ ಹೆಚ್ಚು ದೊಡ್ಡ ಅಳತೆಯ ಪ್ರಮಾಣದಲ್ಲಿ ದತ್ತಿ ಬಿಡಲಾಗಿದೆ. ಬೆದ್ದಲನ್ನು ಕೆಯ್​, ಹೊಲ ಎಂದೂ ಹೇಳಿದೆ. “ಯೋಗಂಣಗಳ ಕಟ್ಟೆಯ ಕೆಳಗೆ ಹೊಯಸಲನ ಗದೆಯಿಂ ಬಡಗಲು ಎರೆಯ ಹೊಲ”,\index{ಎರೆಯ ಹೊಲ}\endnote{ ಎಕ 6 ಕೃಪೇ 95 ಭೈರಾಪುರ 1312} “ಹೊಲ”,\endnote{ ಎಕ 6 ಕೃಪೇ 64 ಸಂತೇಬಾಚಹಳ್ಳಿ 1553} ತೊಣಚಿ ಶಾಸನದಲ್ಲಿ ದೇವಾಲಯದ ಮುಂದಣ ನರುವಲ ಹಾಳ ಹೊಲ\index{ನರುವಲ ಹಾಳ ಹೊಲ} ಎಂದು ಹೇಳಿದೆ. ಹೊಲವನ್ನು ಕೆಯ್​, ಕೆಯಿ ಎಂದೂ ಕರೆಯಲಾಗಿದೆ. “ಕೆಯಿ ಕಂಬ 100, ಹಿತ್ತಿಲ ಕೆಯಿ ಕಂಬ 80”,\endnote{ ಎಕ 7 ನಾಮಂ 62 ಲಾಳನಕೆರೆ 1219} ಕೆಯಿ”,\endnote{ ಎಕ 7 ನಾಮಂ 62 ಲಾಳನಕೆರೆ 1219} ಎಂಬ ಉಲ್ಲೇಖಗಳಿವೆ.

ಬೆದ್ದಲುಗಳು ಸಾಮಾನ್ಯವಾಗಿ ಕೆರೆ ಕಟ್ಟೆ ಕಾಲುವೆಗಳ ನೀರು ಹರಿಯದ ಎತ್ತರ ಪ್ರದೇಶದಲ್ಲಿ ಇರುತ್ತಿದ್ದವು.\break “ದೇವಾಲ್ಯದ ಮುಂದಣ ನಱುವಲ ಹಾಳ ಹೊಲದೊಳಗೆ ಸಮಸ್ತ ಪ್ರಭುಗಾವುಂಡುಗಳುಂ ಬಿಟ್ಟ ಭೂಮಿ ನಾಲ್ಕುಸಾಯಿರಂ”,\endnote{ಎಕ 6 ಕೃಪೇ 48 ತೊಣಚಿ 1191} “ಬಸದಿಯ ನಾಲ್ದೆಸೆಯ ಬೆದ್ದಲೆಯುಮಂ”,\endnote{ ಎಕ 7 ನಾಮಂ 118 ಹಟ್ಟಣ 1178} “ಗ್ರಾಮಕ್ಕೆ ಮೂಡಲು ನೀರ ತರಹೋದ ದಾರಿಯಿಂ ತೆಂಕಣ ತಿಟ್ಟಿನ ಸಾರಿಕೆಯೊಳಗೆ ಬೆದ್ದಲು 500”,\endnote{ ಎಕ 7 ಮ 110 ಬೊಪ್ಪಸಂದ್ರ 1388} ಎಂದು ಹೇಳಿದ್ದೆ. ತಿಟ್ಟು\index{ತಿಟ್ಟು} ಎಂದರೆ ಎತ್ತರವಾದ ಪ್ರದೇಶ. \textbf{“ಈ ಕೆರೆಗಳ ಕಟ್ಟೆ ಕಾಲುವೆಯ ಪಂಥದಲಿ ಯಿದ್ದ ಯೆಮ್ಮ ಬೆದ್ದಲುಗಳಂ ಬಿಟ್ಟು ಅದಕ್ಕೆ ಪ್ರತಿಕ್ಷೇತ್ರವನು ಬಿಟ್ಟೆವಾಗಿ ಯೀ ಬೆದ್ದಲ ಬಿಟ್ಟುಕೊಂಡೆವು ಇಂತೀ ಗದೆ ಕೆಳಗೆ ಪೂರ್ವ್ವದಲ್ಲಿ ವುಳ್ಳ ಅಚ್ಚುಕಟ್ಟು ಮುಖ್ಯವಾದ ಶಾಸನ ಮರ್ಯ್ಯಾದೆಯಲುಳ್ಳ ಬೆದ್ದಲೆಗೆವೂ ಮಾಧವದೇವರ ಗದ್ದೆಗೆವೂ ನೀರು ಸಂಧಿಗೆ ಬಿತ್ತುವಟ್ಟವಾಗಿ”},\endnote{ ಎಕ 7 ನಾಮಂ 83 ಬೆಳ್ಳೂರು 1269} ಎಂದು ಬೆಳ್ಳೂರು ಶಾಸನದಲ್ಲಿ ಹೇಳಿದ್ದು ಗದ್ದೆ ಬೆದ್ದಲೆಗಳೆರಡಕ್ಕೂ ಕಾಲುವೆಯಿಂದ ನೀರನ್ನು ಹರಿಸಿ\-ಕೊಳ್ಳಲಾಗುತ್ತಿತ್ತೆಂದು ಹೇಳಬಹುದು. “ಹೊಲಗುತ್ತಗೆಯಂ” ಎಂಬ ಪ್ರಯೋಗವಿದೆ.\endnote{ ಎಕ 7 ನಾಮಂ 68 ದಡಗ 13ನೇ ಶ.} ಇದರಿಂದ ಹೊಲವನ್ನು ಬೇಸಾಯ ಮಾಡಲು ಗುತ್ತಗೆಗೆ ನೀಡಲಾಗುತ್ತಿತ್ತೆಂದು ಹೇಳಬಹುದು.


\section*{ಬೆದ್ದಲು ಅಥವಾ ಹೊಲದ ಅಳತೆ}
\index{ಬೆದ್ದಲು (ಬೆದ್ದಲೆ)}\index{ಹೊಲ}

ಬೆದ್ದಲು ಅಥವಾ ಹೊಲವನ್ನು ಸಾಮಾನ್ಯವಾಗಿ ಕಂಬದ ಲೆಕ್ಕದಲ್ಲಿ ಅಳೆಯಲಾಗಿದೆ. ಕೆಲವು ಸಲ ಕಂಬಕ್ಕೆ ಬದಲಾಗಿ ಕಮ್ಮ ಎಂದೂ ಉಪಯೋಗಿಸಿದೆ. ಕೆಲವು ಕಡೆ ಬೆದ್ದಲೆಯನ್ನು ಸಲಗೆ ಕೊಳಗದಲ್ಲೂ ಅಳೆದಿರುವುದು ಕಂಡು ಬರುತ್ತದೆ. ಮೇಲುಗೊಡಗಿ ಕಂ 200, ದಿಣ್ಣೆಯ ಮೇಲೆ ಸೊಲಗೆ ಕೊಡಗಿ ಕಂ 60 ಎಂದು ಹೇಳಿದ್ದು, ಹೊಲವು ಯಾವ ರೀತಿಯ ಕೊಡಗೆ ಎಂಬುದನ್ನು ಸೂಚಿಸಿದೆ\endnote{ ಎಕ 7 ನಾಮಂ 46 ಸುಬ್ಬರಾಯನಕೊಪ್ಪಲು 1460}.

\textbf{ಕಂಬ/ಕಮ್ಮ:}\index{ಕಂಬ}\index{ಕಮ್ಮ} “ಕಂಬ 4 ಭೂಮಿ.......ಬೆದ್ದಲೆ ಸಹಿತ”\endnote{ ಎಕ 6 ಪಾಂಪು 252 ತಿರುಮಲಸಾಗರ ಛತ್ರ 1125} “ಆ ಬೆಳ್ಳೂರಲಿ ಕಲ್ಲು ನಡಿಸಿಕೊಟ್ಟ ಸೀಮೆಯಿಂದೊಳಗಣ ಗದ್ದೆ ಸ 36ನೂ ಆ ಬೆದ್ದಲು ಕಂಬ 1850ನೂ ಸರ್ವಮಾನ್ಯವಾಗಿ” ಬಿಡಲಾಯಿತು ಎಂದು ಹೇಳಿದೆ.\endnote{ ಎಕ 7 ನಾಮಂ 74 ಬೆಳ್ಳೂರು 1271} “ಬೆದ್ದಲೆ ಕಂಬ 400..... ಕಮ 15”,\endnote{ ಎಕ 7 ನಾಮಂ 80 ಬೆಳ್ಳೂರು 1199} ಬೆದ್ದಲೆ ಕಂಭ 500,\endnote{ ಎಕ 7 ಮ 113 ಬೊಪ್ಪಸಂದ್ರ 1363} “ಆ ಕೋಡಿಯ ತೆಂಕ ಬೆದ್ದಲು ಕಂಬ 3000”,\endnote{ ಎಕ 7 ನಾಮಂ 81 ಬೆಳ್ಳೂರು 1223} ಎಂದು ಹೇಳಿದೆ. “ಪುರದ ಮುಂದೆ\break ಬೆದಲು ಕಂಬ 600, ಬರಗಿನ ಹೊಲ ಕಂಬ 600”,\endnote{ ಎಕ 7 ಮವ 24 ಸಾಹಳ್ಳಿ 1573} “ಬೇಬಿ ಹೊಲದಲಿ ಸಲುವ ಬೆದಲು ಕಂ 600, ಆಲೆವನೆ ಸಂತೆಗುಡಿ ಸಲುವ ಕಂಬ 100”,\endnote{ ಎಕ 7 ಮವ 31 ಹುಸ್ಕೂರು 1313} ಬೆದ್ದಲು ಕಂಬ100,\endnote{ ಎಕ 7 ಮಂ 30 ಬಸರಾಳು 1237} ಎಂದು ಬೆದ್ದಲೆ ಅಥವಾ ಹೊಲವನ್ನು ಕಂಬದಲ್ಲಿ ಅಳೆಯಲಾಗಿದೆ. ಬೆದ್ದಲನ್ನು ಕಮ್ಮದ ಲೆಕ್ಕದಲ್ಲೂ ಅಳೆದಿದೆ. “ಬೆದ್ದಲು ನೂರು ಕಮ್ಮ”,\endnote{ ಎಕ 7 ನಾಮಂ 7 ನಾಗಮಂಗಲ 1134} ನೀರಾವರಿ ಇಲ್ಲದ ಸ್ಥಳದಲ್ಲಿ ಹೆಚ್ಚು ಪ್ರಮಾಣದ ಬೆದ್ದಲನ್ನು ದತ್ತಿಯಾಗಿ ನೀಡಲಾಗುತ್ತಿತ್ತೆಂದು ಹೇಳಬಹುದು. “ಹಿರಿಯೂರಿಂ ಹಡವನಹಳ್ಳಿಗೆ ಹೋದೋಣಿಯಿಂ ತೆಂಕಣ ಹಾಳಿನಲ್ಲಿ ಬೆದ್ದಲು ಕಂಬ 750, ಹಡವನಹಳ್ಳಿಯ ಬಡಗಣ ನೀರುಗಲ್ಲ ಹಾಳಿನಲ್ಲಿ ಕಂಬ 750, ಕೋಡಿಯಹಳ್ಳಿಯಿಂ ಮೂಡಣ ಹುಲ್ಲೆಯ ಹಾಳಲು ಕಂಬ 500, ಅಂತು ಬೆದ್ದಲು 2000”,\endnote{ ಎಕ 7 ಮಂ 29 ಬಸರಾಳು 1234} ಎಂದು ಹೇಳಿದ್ದು, 2000 ಸಂಖ್ಯೆಯು ಕಂಬವನ್ನು ಸೂಚಿಸುತ್ತಿದೆ ಎಂದು ಹೇಳಬಹುದು. “ಕಂಬ ನಾಲ್ವತ್ತಾರಲು ಲೆಕದಲು 46.......(ಇ)ಡು ಪಡಿಯಾಗಿ ಇಕ್ಕೇರಿ ಕಂಬ\index{ಇಕ್ಕೇರಿ ಕಂಬ} ಐವತ್ತು ಅಂತು ಸಾವಿರದ..... ಕಂಬ” ಎಂದು ಹೇಳಿದ್ದು ಇಕ್ಕೇರಿ ಕಂಬದ ಅಳತೆಯನ್ನು ಸ್ಥಳೀಯ ಕಂಬದ ಅಳತೆಯೊಡನೆ ಹೊಂದಾಣಿಕೆ ಮಾಡಿರುವುದು ಕಂಡುಬರುತ್ತದೆ.\endnote{ ಎಕ 6 ಕೃಪೇ 10 ಹರಿಹರಪುರ 1311} “ಕೆಳದಿ ವೀರಭದ್ರ ದೇವಾಲಯದ ಗೋಡೆಯಲ್ಲಿರುವ 2 ಅಡಿ 7.2 ಇಂಚು ಉದ್ದ, 1.1.ಇಂಚು ಅಗಲವಿರುವ ಅಡಿಕೋಲು ಒಂದರ ಕೆತ್ತನೆ ಇದೆ. ತೋಟದ ಅಳತೆ ಮಾಡಲು 18 ಕಾಲು ಅಡಿ ಉದ್ದದ ಕೋಲನ್ನು ಬಳಸುತ್ತಿದ್ದರು, ಇದು 18 ಅಡಿ ಎಂದು ಕೆ.ಎನ್​. ಚಟ್ನೀಸ್​ ಅವರು ಅಭಿಪ್ರಾಯ ಪಟ್ಟಿದ್ದಾರೆ, ಇಕ್ಕೇರಿ ದೇವಾಲಯದ ಮುಂಭಾಗದ ಮೆಟ್ಟಿಲ ಅಳತೆಯನ್ನು ತೋಟದ ಅಳತೆಗಾಗಿ ಅಳತೆಗೋಲಾಗಿ ಬಳಸುತ್ತಿದ್ದರು ಎಂದು ರೈಸ್​ ಬರೆದಿದ್ದಾರೆ ಎಂದು ತಿಳಿದುಬರುತ್ತದೆ”.\endnote{ ವೆಂಕಟೇಶ್​ ಡಾ|| ಕೆ.ಜಿ., ಕೆಳದಿ ಶಾಸನಗಳ ಸಾಂಸ್ಕೃತಿಕ ಅಧ್ಯಯನ, ಪುಟ 192} ಇದೇ ಇಕ್ಕೇರಿ ಕಂಬ ಇರಬಹುದು.

\textbf{ಸಲಗೆ/ಖಂಡುಗ:}\index{ಸಲಗೆ (ಸಲಿಗೆ)}\index{ಖಂಡುಗ} ಬೆದ್ದಲೆಯನ್ನು ಸಲಗೆ, ಕೊಳಗ ಮತ್ತು ಖಂಡುಗದಲ್ಲಿ ಕೂಡಾ ಅಳೆದಿದೆ. “ಹೊಯ್ಸಳೇಶ್ವರ\break ದೇವರ್ಗ್ಗೆ ಬಿಟ್ಟ ಬೆದ್ದಲೆ ಸಲಗೆ 2 ಕೊಳಗ 14”\endnote{ ಎಕ 6 ಕೃಪೇ 42 ತೆಂಗಿನಘಟ್ಟ 1117} “ಬೆದ್ದಲೆ ಸಲಗೆ 1”,\endnote{ ಎಕ 7 ನಾಮಂ 61 ಲಾಲನಕೆರೆ 1138} “ಬಸದಿಯಿಂ ಬಡಗಲು ಬೆದ್ದಲೆ ಬೆದೆ ಖಂಡುಗ ಎರಡು”,\endnote{ ಎಕ 7 ನಾಮಂ 14 ಸುಕಧರೆ 12ನೇ ಶ.} ಎಂಬ ಉಲ್ಲೇಖವಿದೆ. ಆದರೆ ಇದು ಬಹಳ ಅಪರೂಪ.

\textbf{ಕೊಳಗ:}\index{ಕೊಳಗ} ಬೆದ್ದಲೆಯನ್ನು ಕೇವಲ ಕೊಳಗದಲ್ಲೇ ಅಳೆದಿರುವುದು ಕಂಡು ಬರುತ್ತದೆ. “ಆ ಹಿರಿಯ ಕೆರೆಯ ಪಡುವಣ ಕೋಡಿಯಳರೆಯ ಕಯಿ ಬೆದ್ದಲ್​ ಕೊಳಗ ಹತ್ತು”\endnote{ ಎಕ 6 ಕೃಪೇ 66 ಮಾಳಗೂರು 1117}, “ಕಲಿಯಣ್ನನ ಕೊಡುಗೆಯೊಳು ಆಯ್ವತ್ತು ಕೊಳಗ ಗದ್ದೆಯುಂ ಸಾಯಿರ ಕೊಳಗ ಬೆದ್ದಲೆಯುಂ”,\endnote{ ಎಕ 6 ಪಾಂಪು 15 ಕ್ಯಾತನಹಳ್ಳಿ 1175} “ಬೆದ್ದಲೆ ದೇವರ ಮುಂದೆ ಕೊಳಗಂ ಹತ್ತು, ಕಬ್ಬಿನಕೆರೆಯ ಕೋಡಿಯಲು ಕೊಳಗಂ ಹತ್ತು”,\endnote{ ಎಕ 7 ನಾಮಂ 61 ಲಾಲನಕೆರೆ 1138} “ಬೆದ್ದಲೆ ದೇವರ ಸನ್ನಿಧಾನದಲು ಕೊಳಗಂ 10”,\endnote{ ಎಕ 7 ನಾಮಂ 63 ಲಾಳನಕೆರೆ 1165} ಎಂದು ಹೇಳಿದೆ. ಈ ಅಳತೆಯು ಹೊಲದಲ್ಲಿ ಇಳುವರಿಯಾಗುವ ಧಾನ್ಯದ ಪ್ರಮಾಣದಲ್ಲಿತ್ತೋ ಅಥವಾ ಉದ್ದ ಅಗಲದ ಅಳತೆಯಲ್ಲಿತ್ತೋ ತಿಳಿದುಬರುವುದಿಲ್ಲ.

\textbf{ಮತ್ತರು:}\index{ಮತ್ತರು} ಬೆದ್ದಲೆಯನ್ನು ಮತ್ತರಿನ ಲೆಕ್ಕದಲ್ಲಿ ಅಳೆಯಲಾಗಿದೆ. ಮಂಡ್ಯ ಜಿಲ್ಲೆಯ ಪ್ರದೇಶದಲ್ಲಿ ಮತ್ತರಿನ ಅಳತೆ ಶಾಸನಗಳಲ್ಲಿ ಹೆಚ್ಚಾಗಿ ಉಲ್ಲೇಖವಾಗಿಲ್ಲ. “ಸಿವಾಲ್ಯದ ಪಡುವಣ ಮೆಯ್ಯ ಬೆದ್ದಲೆ ಮತ್ತರೊಂದು”,\endnote{ ಎಕ 7 ನಾಮಂ 98 ಮುದಿಗೆರೆ 1139} ಎಂದು ಹೇಳಿದೆ. ಮತ್ತರು ಬಹಳ ದೊಡ್ಡ ಅಳತೆಯಾಗಿರಬಹುದೆಂದು ಕಂಡು ಬರುತ್ತದೆ. ಕಾರಣ ಬೆದ್ದಲೆಯನ್ನು ಮತ್ತರಿನಲ್ಲಿ ಹೇಳುವಾಗ ಐದು, ಒಂದು ಎಂದು ಹೇಳಿದೆ. “ಇದಕ್ಕೆ ಮತ್ತಲ್​ ಮತ್ತರ್​ ಎಂದೂ ಹೆಸರುಗಳುಂಟು. ಮತ್ತ, ಮ ಎಂಬ ಸಂಕ್ಷೇಪ ರೂಪಗಳೂ ಉಂಟು”. “ಮತ್ತರಕ್ಕಿಂತ ಸಣ್ಣ ಅಳತೆಯ ಪ್ರಮಾಣವೆಂದರೆ ಕಂಬ, ಕಮ್ಮ. ಅಮೃತಾಪುರ ಶಾಸನದ ಪ್ರಕಾರ 1 ಮತ್ತರಕ್ಕೆ ನೂರು ಕಂಬ” ಹರಪನಹಳ್ಳಿ ತಾಲ್ಲೂಕು ಕುರುವತ್ತಿ\index{ಕುರುವತ್ತಿ} ಶಾಸನದಿಂದ “ಒಂದು ಮತ್ತರಕ್ಕೆ 900 ಕಂಬಗಳೆಂದೂ\index{ಕಂಬ} ತಿಳಿದುಬರುತ್ತದೆ”, “ಮತ್ತರದ ಕ್ಷೇತ್ರ ಸಾಧಾರಣವಾಗಿ ಈಗಿನ ಒಂದು ಎಕರೆಯಿಂದ ಎಂಟು ಎಕರೆವರೆಗೆ ಆಗುತ್ತದೆ” ಎಂಬುದನ್ನು ವಿದ್ವಾಂಸರು ಗುರುತಿಸಿದ್ದಾರೆ.\endnote{ ಹಿರೇಮಠ್​ ಡಾ|| ಆರ್​.ಸಿ., ಕಲಬುರ್ಗಿ ಡಾ|| ಎಂ.ಎಂ., ಕನ್ನಡ ಶಾಸನ ಸಂಪದ, ಪುಟ 145-46}


\section*{ಬೆದ್ದಲ ಅಳತೆ ಸಂಖ್ಯೆಯಲ್ಲಿ}

ಬೆದ್ದಲನ್ನು ಅಳೆಯುವಾಗ ಕೆಲವು ಕಡೆ ಕೇವಲ ಅಂಕಿ ಸಂಖ್ಯೆಯನ್ನು ಹೇಳಿದೆಯೇ ಹೊರತು ಅದು ಏನನ್ನು ಸೂಚಿಸುತ್ತದೆ ಎಂದು ಹೇಳಿಲ್ಲ. “ಅಯ್ವತ್ತರಂ ಬೆದ್ದಲೆಯನ್​”,\index{ಅಯ್ವತ್ತರಂ ಬೆದ್ದಲೆ} “ಬೆದ್ದಲು ಸಾಯಿರದಯಿನೂರು”,\endnote{ ಎಕ 6 ಶ‍್ರೀಪ 113 ಅರಕೆರೆ 1108} “ಬೆದ್ದಲೆ 600”,\endnote{ ಎಕ 7 ನಾಮಂ 123 ಭೀಮನಹಳ್ಳಿ 1230} ಊರಿಂದ ಬಡಗಲುಬೆದ್ದಲು ಎರಡುಸವಿರ”,\endnote{ ಎಕ 7 ಮವ 23 ಗೌಡಗೆರೆ 1253} “ಬೆದಲು ಆಯಿನೂರನೂ\index{ಬೆದ್ದಲು (ಬೆದ್ದಲೆ)} ಕೊಂಡು”,\endnote{ ಎಕ 7 ಮಂ 56 ಹೊಸಬೂದನೂರು 1276} “ಸಾಯಿರ ಬೆದ್ದಲು”,\index{ಸಾಯಿರ ಬೆದ್ದಲು}\endnote{ ಎಕ 7 ಮ 114 ಬೊಪ್ಪಸಂದ್ರ 14ನೇ ಶ.} “ಬೆದ್ದಲು 1200”,\endnote{ ಎಕ 7 ಮ 52 ತಿಪ್ಪೂರು 12ನೇ ಶ.}ಎಂದು ಹೇಳಿದೆ. ಈ ಸಂಖ್ಯೆಯನ್ನು ಕಂಬ ಎಂದೇ ಅರ್ಥೈಸಬಹುದು. ಬೆದ್ದಲು ಅಥವಾ ಹೊಲವನ್ನು ಮಂಣು ಎಂದು ಕೂಡಾ ಕರೆಯಲಾಗಿದೆ. “ನೆತ್ತರುಗೊಡಗೆ ನಾನೂರು ಮಂಣು, ಹದಿಮೂ ಗುಳ ಗದ್ದೆ”,\endnote{ ಎಕ 7 ಮವ 36 ಹುಲ್ಲೇಗಾಲ 1219} ಎಂದು ಹೇಳಿದೆ. ಇಲ್ಲಿ ಗದ್ದೆಯನ್ನು ಗುಳದಲ್ಲಿ ಅಳೆದಿದ್ದು ಮಣ್ಣು ಎಂದರೆ ಹೊಲವನ್ನು ನಾನೂರು ಎಂದು ಹೇಳಿದೆ. ನಾನೂರು ಎಂಬುದು ಕಂಬದ ಅಳತೆಯೇ ಆಗಿರಬಹುದು.


\section*{ತೋಟಗಳು}
\index{ತೋಟ (ತೋಟದ ಸ್ಥಳ - ಕ್ಷೇತ್ರ, ತೋಂಟ)}

ಶಾಸನಗಳಲ್ಲಿ ತೋಟಗಳನ್ನು, ತೋಣ್ಟ (ತೋಂಟ),\endnote{ ಎಕ 7 ಮ 56 ತಾಯಲೂರು 907} ತೋಟ, ತೋಟದ ಸ್ಥಳ,\index{ತೋಟ (ತೋಟದ ಸ್ಥಳ - ಕ್ಷೇತ್ರ, ತೋಂಟ)}\endnote{ ಎಕ 7 ನಾಮಂ 81 ಬೆಳ್ಳೂರು 1224} ತೋಟದ ಕ್ಷೇತ್ರ,\index{ತೋಟ (ತೋಟದ ಸ್ಥಳ - ಕ್ಷೇತ್ರ, ತೋಂಟ)} ತೆಂಗು,\index{ತೆಂಗು} ಕವುಂಗು ಕ್ಷೇತ್ರ,\index{ಕವುಂಗು ಕ್ಷೇತ್ರ} ಕೌಂಗಿನ ತೋಟ, ಅಡಕೆಯ ಮರದ ತೋಟ, ತೋಟಸ್ಥಳ ಎಂದು ಹೇಳಲಾಗಿದೆ. ಜಿಲ್ಲೆಯ ಶಾಸನಗಳಲ್ಲಿ ಸಾಮಾನ್ಯವಾಗಿ ತೋಟ ತುಡಿಕೆಗಳನ್ನು ಒಟ್ಟಾಗಿಯೇ ಉಲ್ಲೇಖಿಸಿದೆ. ಆದರೆ ಇವುಗಳಿಗಿರುವ ವ್ಯತ್ಯಾಸದ ವಿವರಗಳಿಲ್ಲ. ತೋಟಗಳು ಸಾಮಾನ್ಯವಾಗಿ ಕೆರೆಯ ಕೆಳಗೆ, ಹಳ್ಳಗಳ ಬದಿಯಲ್ಲಿ ಇರುತ್ತಿದ್ದವು. ಗದ್ದೆಯ ಭೂಮಿಯನ್ನೇ ತೋಟಗಳನ್ನು ಬೆಳೆಸಲು ಉಪಯೋಗಿಸಲಾಗುತ್ತಿತ್ತು. ಇಂದಿಗೂ ಕೆರೆಯ ಹಿಂದೆಯೇ ತೋಟಗಳು ಇರುವುದನ್ನು ನೋಡಬಹುದು. “ಕಮ್ಮಟೇಶ್ವರ ದೇವರ ಮುನ್ತಣ ಕೆರೆಯಞ್ಚ ಕೆಲದ ತೋಟದ ಮಣ್ಣು ಹಳ್ಳದ ಗರ್ದ್ದೆಯುಂ”,\index{ಹಳ್ಳದ ಗರ್ದ್ದೆ}\endnote{ ಎಕ 6 ಕೃಪೇ 66 ಮಾಳಗೂರು 1117} ಎಂದು ಮಾಳಗೂರು ಶಾಸನದಲ್ಲಿ ಹೇಳಿದ್ದು, ತೋಟ ಗದ್ದೆಯ ಭೂಮಿ ಒಂದೇ ಆಗಿತ್ತು ಎಂಬುದನ್ನು ತೋರಿಸುತ್ತದೆ. ಅದೇ ರೀತಿ ಕನ್ನಂಬಾಡಿಯ ಶಾಸನದಲ್ಲಿ “ತೋಂಟದ ಗದ್ದೆ ಕೊ 10”,\endnote{ ಎಕ 6 ಪಾಂಪು 34 ಕನ್ನಂಬಾಡಿ 13ನೇ ಶ.} ಹುಬ್ಬನಹಳ್ಳಿ ಶಾಸನದಲ್ಲಿ “ಹಿರಿಯ ಕೆರೆಯ ಕೆಳಗೆ ತೋಂಟದ ಗದ್ದೆ ಸಲಗೆ 1”,\endnote{ ಎಕ 6 ಕೃಪೇ 62 ಹುಬ್ಬನಹಳ್ಳಿ 1140} ಎಂದು ಹೇಳಿದೆ. ದೇವಲಾಪುರದ ಶಾಸನದಲ್ಲಿ “ದೇವಲಾಪುರದ ಹಿರಿಯ ಕೆರೆಯ ಕೆಳಗಳ ಮುತ್ತೇರಿಯ ಸ್ಥಳನಿರ್ದೇಶದ ಕರಿಯನ ತೋಟದಿಂ, ಪಡುವಲು ಸಂಣಬೊಂಮಂಣನ ತೋಟದಿಂ, ಬಡಗಲು ಗಡದ ಗಂಗಂಗಣಗಳ ತೋಟದಿಂ, ಮೂಡಲು ಬಲಭದ್ರದಾಸರ ತೋಟದಿಂ ತೆಂಕಲು ಯಿಂತೀ ಚತುಸೀಮೆಯ ಒಳಗುಳ ಅಡಕೆಯ ಮರ\index{ಅಡಕೆಯ ತೋಟ} ಯಿಂನೂರು ಮರದೊಳಗುಳ”,\endnote{ ಎಕ 7 ನಾಮಂ 157 ದೇವಲಾಪುರ 1472} ಎಂದು ಹೇಳಿದ್ದು ಕೆರೆಯ ಹಿಂದೆಯೇ ತೋಟಗಳಿರುತ್ತಿದ್ದವೆಂದು ಖಚಿತವಾಗಿ ಗುರುತಿಸಬಹುದು. ತೋಟಗಳಿಗೂ ಕಾಲುವೆಯಿಂದ ನೀರನ್ನು ಹರಿಸುತ್ತಿದ್ದರು. “ಮಾಸಳೆನಟಿ ತೋಟಕ್ಕೆ ಕಾಲುವೆಯ ನೀರಸರತಿಗಂ 2 ಬಿಟ್ಟೆವು”,\endnote{ ಎಕ 6 ಪಾಂಪು 30 ಕನ್ನಂಬಾಡಿ 1553} “ಕಾಲುವೆಯ ಗದ್ದೆ ಬೆದ್ದಲು ತೋಟ ತುಡುಕೆ”,\index{ತೋಟ ತುಡುಕೆ}\endnote{ ಎಕ 6 ಪಾಂಪು 19 ಸೀತಾಪುರ 1467} ಎಂಬ ಉಲ್ಲೇಖಗಳಿದ್ದು, ಇದರಿಂದ ತೋಟಕ್ಕೂ ಕಾಲುವೆಯ ಮೂಲಕ ನೀರನ್ನು ಹರಿಸುತ್ತಿದ್ದರು ಎಂದು ತಿಳಿದುಬರುತ್ತದೆ. “ದೇವರಿಗೆ ಬಿಟ್ಟ ಕಉಂಗಿನ ತೋಂಟ ಗದ್ದೆ ಕಣ್ಡುಗ 15”,\endnote{ ಎಕ 6 ಕೃಪೇ 37 ಕಿಕ್ಕೇರಿ 1095} ಎಂದು ಹೇಳಿದ್ದು ಇದು ತೋಟದ ಗದ್ದೆಯಾಗಿದ್ದು ತೋಟವನ್ನು ಖಂಡುಗದಲ್ಲಿ ಅಳೆಯಲಾಗಿದೆ ಎಂದು ಹೇಳಬಹುದು. “ಪಿರಿಯಕೆರೆಯ ತೂಂಬಿನಿಂ ಬಡಗಣ ಹಳ್ಳದಿಂ ತೆಂಕ ಕೌಂಗಿನ ತೋಂಟ”,\index{ಕೌಂಗಿನ ತೋಂಟ}\endnote{ ಎಕ 7 ನಾಮಂ 33 ಕಂಬದಹಳ್ಳಿ 1118} ಅಂದರೆ ಕೆರೆಯ ತೂಬಿನ ಹಳ್ಳದ ದಂಡೆಯಲ್ಲಿ ಅಡಿಕೆಯ ತೋಟ ಇತ್ತೆಂದು ಹೇಳಿದೆ. “ಅಡಕೆಯ ಮರದ ತೋಂಟದ ಸ್ಥಳ ಮರ 1000”,\endnote{ ಎಕ 7 ನಾಮಂ 61 ಲಾಲನಕೆರೆ 1138} ಎಂಬ ಉಲ್ಲೇಖವಿದ್ದು ತೋಟದಲ್ಲಿ ಎಷ್ಟು ಅಡಕೆಯ ಮರಗಳು ಇದ್ದವು ಎಂಬುದನ್ನೂ ಹೇಳಿದೆ. “ಹುಲ್ಲೆಯ ಕೆಳಗೆ ಅಡಕೆಯ ಮರ\index{ಅಡಕೆಯ ಮರ} ಯಿಂನೂರರ ಸ್ಥಳದ ತೋಟವೊಂದು”,\endnote{ ಎಕ 10 ಅರ 184 ಹುಲ್ಲೇಕೆರೆ 1162} ಎಂಬ ಉಲ್ಲೇಖವನ್ನೂ ನೋಡಬಹುದು. ದಡಗ ಶಾಸನದಲ್ಲಿ “ಹಿರಿಯಕೆರೆಯ ಕೆಳಗಣ ಅಡಕೆಯ ತೋಟ”ದ\index{ಅಡಕೆಯ ತೋಟ} ಉಲ್ಲೇಖವಿದೆ.\endnote{ ಎಕ 7 ನಾಮಂ 68 ದಡಗ 13ನೇ ಶ.} “ಭೂಮಿಯೊಳಗೆ ತೆಂಗು ಕವುಂಗು ಮುಖ್ಯವಾದ ಸಮಸ್ತ ಫಲವ್ರುಕ್ಷಂಗಳನೂ ಯಿಕ್ಕಿಕೊಂಡು”,\endnote{ ಎಕ 7 ಮಂ 56 ಹೊಸಬೂದನೂರು 1276} ಎಂದು ಹೇಳಿದ್ದು, ತೆಂಗು ಅಡಿಕೆಯ ಜೊತೆಗೆ ಹಣ್ಣನು ನೀಡುವ ಮರಗಳನ್ನೂ ಕೂಡಾ ಬೆಳೆಸುತ್ತಿದ್ದರು. “ತೋಟದ ಅಡಕೆ ಹುಟ್ಟುವಳಿಯ” ಉಲ್ಲೇಖ ಚಟ್ಟಮಗೆರೆ ಶಾಸನದಲ್ಲಿದೆ.\endnote{ ಎಕ 6 ಕೃಪೇ 103 ಚಟ್ಟಂಗೆರೆ 1759} ಜಕ್ಕನಹಳ್ಳಿ ತೋಟ, ಅಸಗರಹಳ್ಳಿ ತೋಟಗಳ ಉಲ್ಲೇಖ ತೊಣ್ಣೂರು ಶಾಸನದಲ್ಲಿದೆ.\endnote{ ಎಕ 6 ಪಾಂಪು 99 ತೊಣ್ಣೂರು 1722}

ವೀಳ್ಯದೆಲೆ\index{ವೀಳ್ಯದೆಲೆ} ತೋಟವನ್ನು ಕುಳಿಗಳ\index{ಕುಳಿ} ಮೇಲೆ ಲೆಕ್ಕಹಾಕಲಾಗಿದೆ. ಕುಳಿ ಎಂದರೆ ವೀಳ್ಯದೆಲೆಯ ಹಂಬನ್ನು ನೆಟ್ಟು ಅದನ್ನು ಒಂದು ಮರಕ್ಕೆ ಹಬ್ಬಿಸುವ ಜಾಗ. “ಪಡಿಯೆಲೆಯ ವೀಳೆಯಕ್ಕ ಬಿಲ್ಕುಂಬಳಿಯ ತೋಟದಲು ಎಲೆಯ ಗುಳಿ ಇನೂರು ಗುಳಿಯಂ” “ಗಾವಡಗದೆ ಮೂವತ್ತು ಗುಳಿಯುಂ ಅನ್ತು ಇಂನೂರಎಂಬತ್ತು ಗುಳಿಯ ಪಂನಾಯವಂ”,\index{ಗುಳಿಯ ಪಂನಾಯ}\endnote{ ಎಕ 6 ಪಾಂಪು 79 ತೊಣ್ಣೂರು 1175} ಎಂದು ಹೇಳಿದೆ. “ಹದಿನಯ್ಗುಳ ಎಲೆಯತ್ತಿಯನು”,\endnote{ ಎಕ 6 ಪಾಂಪು 161 ಮೇಲುಕೋಟೆ 14ನೇ ಶ.} ಎಂದು ಹೇಳಿದ್ದು ಇದು ಹದಿನೈದು ಕುಳಿ ಎಲೆಯ ತೋಟವನ್ನು ಸೂಚಿಸುತ್ತದೆ.

ತೋಟವನ್ನೂ ಕೂಡಾ ಸಲಗೆಯಲ್ಲೂ ಕೂಡಾ ಅಳೆಯಲಾಗುತ್ತಿತ್ತೆಂದು ಕಂಡುಬರುತ್ತದೆ. “ಅಡಕೆದೋಂಟವೊಳ\-ಗಾಗಿ ಸಲಗೆ ವೊಂಬತ್ತು” ಎಂದು ಅಗ್ರಹಾರಬೆಳಗಲಿ ಶಾಸನದಲ್ಲಿ ಹೇಳಿದೆ.\endnote{ ಎಕ 10 ಚರಾಪ 94 ಅಗ್ರಹಾರಬೆಳಗಲಿ 1209} “ಕೆರೆಗೊಡಗಿ\index{ಕೆರೆಗೊಡಗಿ} ತೋಟಸ್ಥಳವೊಳಗಾಗಿ ಸ 125”,\endnote{ ಎಕ 7 ನಾಮಂ 81 ಬೆಳ್ಳೂರು 1223}\break “ಮತ್ತಿಯಕೆರೆಯ ಕೆಳಗೆ ಗದ್ದೆ ತೋಟದಿಂ ಪಡುವಲು ಸಲಗೆ 2 ಎಂದು ಹೇಳಿದ್ದು ಇಲ್ಲಿ ಗದ್ದೆ ತೋಟ ಎರಡನ್ನೂ ಒಟ್ಟಿಗೆ ಸಲಗೆಯಲ್ಲಿ ಅಳೆದಿದೆ ಎಂದು ಹೇಳಬಹುದು.\endnote{ ಎಕ 7 ನಾಮಂ 63 ಲಾಳನಕೆರೆ 1165} “ತೆಂಗಣ ಮಾವಿನ ಬನಂ\index{ಮಾವಿನ ಬನ} ವೊಂಭೈನೂರ ಇಪ್ಪತ್ತ ಎಣ್ಟು” ಎಂದು ತೊಣ್ಣೂರು ಶಾಸನದಲ್ಲಿ ಹೇಳಿದ್ದು 928 ಮಾವಿನ ಮರಗಳಿದ್ದ ತೋಟವನ್ನು ತೊಂಡನೂರಿನ ವಿತ್ತಿರುಂದ ಪೆರುಮಾಳೆ ದೇವರಿಗೆ ದತ್ತಿ ಬಿಡಲಾಗಿದೆ.\endnote{ ಎಕ 6 ಪಾಂಪು 88 ತೊಣ್ಣೂರು 1157}

\section*{ಭೂಮಿಯ ಗುತ್ತಗೆ/ವಾರ}
\index{ಗುತ್ತಗೆ/ವಾರ}

ಭೂಮಿಯನ್ನು ಬೇಸಾಯ ಮಾಡಲು ಗುತ್ತಿಗೆಗೆ ಕೊಡಲಾಗುತ್ತಿತು. “ಅಲ್ಲಿಯ ಹೊಲಗುತ್ತಗೆಯಂ”,\endnote{ ಎಕ 7 ನಾಮಂ 68 ದಡಗ 12ನೇ ಶ.} ಎಂಬ ಉಲ್ಲೇಖವು ಗುತ್ತಿಗೆಗೆ ನೀಡಿದ್ದ ಹೊಲವನ್ನು ಸೂಚಿಸುತ್ತದೆ. “ಆ ಕ್ಷೇತ್ರವ\index{ಕ್ಷೇತ್ರ} ಮಾಡುವ ಗುತ್ತಗೆಕಾರರು\index{ಗುತ್ತಗೆಕಾರರು} ಆ ಗುತ್ತಗೆಯ ವಸ್ತುವನು ಆ ಶ‍್ರೀ ಪ್ರಸಂನ ಮಾಧವದೇವರು ಶ‍್ರೀ ರಾಮಕೃಷ್ಣದೇವರು, ಶ‍್ರೀ ವರದಅಲ್ಲಾಳನಾಥದೇವರುಗಳ ಅಮ್ರಿತಪಡಿಗೆವೂ ಆ ಕೆರೆಯ ಭಂಡಿಯ ಧರ್ಮ್ಮಕೆವೂ ಸರಿಯಾಗಿ ಯಿಕ್ಕುತಬಹುರು”,\endnote{ ಎಕ 7 ನಾಮಂ 76 ಬೆಳ್ಳೂರು 1284} “ಆ ಗದ್ದೆ ಸ 36ನೂ ತೋಟಗದ್ದೆ ಮುಖ್ಯವಾಗಿ ಏನನಾದಡಂ ಬಿತ್ತಿ ಬೆಳೆದುಕೊಂಡು ತಾರಣ ಸಂವತ್ಸರ ಮೊದಲಾಗಿ ವರುಷ ಮೂರಕ್ಕಂ ವರುಷಂಪ್ರತಿ ಗುತ್ತಗೆಯಾಗಿ ಗ37 ಪ2 ರ ಲೆಕ್ಕದೆ ತೆರುತ ಬಹರು”,\endnote{ ಎಕ 7 ನಾಮಂ 74 ಬೆಳ್ಳೂರು 1271} ಎಂದು ಹೇಳಿರುವುದರಿಂದ ಗುತ್ತಗೆಯನ್ನು ಆ ಭೂಮಿಯ ಉತ್ಪನ ಹಾಗೂ ಗುತ್ತಗೆಯ ಅವಧಿಗೆ ತಕ್ಕಂತೆ ಲೆಕ್ಕಹಾಕಿ ನಿಗದಿಪಡಿಸಲಾಗುತ್ತಿತ್ತು ಮತ್ತು ನಿರ್ದಿಷ್ಟಪಡಿಸಿದ ಗುತ್ತಗೆಯನ್ನು ಸರಿಯಾಗಿ ಸಂದಾಯ ಮಾಡಬೇಕಾಗುತ್ತಿತ್ತೆಂದು ಹೇಳಬಹುದು. “ರಂಗಸಮುದ್ರ ಕೆರೆಯ ಕೆಳಗೆ ಶ‍್ರೀ ಚೆಲಪಿಳೆರಾಯರ ಭಂಡಾರಕ್ಕೆ ಉತ್ತು ಬಿತ್ತಿ ವಾರವನಿಕ್ಕುವ\index{ವಾರವನಿಕ್ಕುವ} ಗದೆಯ ಕಳೆದು” ಎಂದು ಮೇಲುಕೋಟೆ ಶಾಸನದಲ್ಲಿ ಹೇಳಿದ್ದು ದೇವರ ಗದ್ದೆಗಳನ್ನು ವಾರಕ್ಕೆ (ಗುತ್ತಿಗೆ) ನೀಡುತ್ತಿದ್ದುದು ತಿಳಿದುಬರುತ್ತದೆ.\endnote{ ಎಕ 6 ಪಾಂಪು 199 ಮೇಲುಕೋಟೆ 1458}

\section*{ಬೆಳೆಗಳ ಮೇಲೆ ತೆರಿಗೆ-ಸುಂಕ}

ನೀರಾವರಿಗೆ ಸುಂಕವನ್ನು ವಿಧಿಸುತ್ತಿದ್ದರು. ಇದನ್ನು “ನೀರ್ವ್ವರಿ ಸುಂಕ”\index{ನೀರ್ವ್ವರಿ ಸುಂಕ} ಎಂದು ಶಾಸನಗಳಲ್ಲಿ ಹೇಳಿದೆ.\endnote{ ಎಕ 7 ನಾಮಂ 100 ಆರಣಿ 1141} ಅದೇ ರೀತಿ ಕಳ ಕೊಠಾರ ಎಂಬುದು ಹೊಲ ಗದ್ದೆಗಳಲ್ಲಿ ಹಾಕುತ್ತಿದ್ದ ಕಣವನ್ನು ಕುರಿತು ಹೇಳಿದೆ. ಇದಕ್ಕೆ “ಕಣಸಲಿಗೆ”,\index{ಕಣಸಲಿಗೆ}\endnote{ ಎಕ 6 ಕೃಪೇ 39 ಗೋವಿಂದನಹಳ್ಳಿ 1236} ಎಂಬ ತೆರಿಗೆಯೂ ಇತ್ತು. ವ್ಯವಸಾಯದಿಂದ ಬರುತ್ತಿದ್ದ ಆದಾಯವನ್ನು ಒಡೆಯರ ಕಾಲದ ಶಾಸನದಲ್ಲಿ “ಸಕಲ ದವಸಾದಾಯ”\index{ದವಸಾದಾಯ} ಎಂದು ಕರೆದಿದೆ.\endnote{ ಎಕ 7 ಮವ 9 ಸಶ್ಯಾಲಪುರ 1672} “ಆ ಸಕಲ ಪೈರು ಪೊಂಮಿಗೆ ಸಲುವ ಜವಳಿ ಲಾಭಾದಾಯದ ಪೊಂಮ, ಕಬ್ಬಿನ ಪೊಂಮ,\index{ಕಬ್ಬಿನ ಪೊಂಮ} ಹೊಗೆಸೊಪ್ಪಿನ ಪೊಂಮ,\index{ಹೊಗೆಸೊಪ್ಪಿನ ಪೊಂಮ} ಅಠ್ಠವಣೆಗೆ ಸಲುವ ಪೈರುಸುಂಕ\index{ಪೈರು ಸುಂಕ} ಪೊಂಮ, ದೇವಸ್ಥಾನಗಳಿಗೆ ಸಲ್ವ ಪೈರುಗಳು” ಎಂಬ ಉಲ್ಲೇಖ ಮೈಸೂರು ಅರಸರ ಕಾಲದ ಶಾಸನದಲ್ಲಿದ್ದು,\endnote{ ಎಕ 6 ಪಾಂಪು 215 ಮೇಲುಕೋಟೆ 1724} ಇವೆಲ್ಲಾ ಬೆಳೆಯ ಮೇಲಿನ ತೆರಿಗೆಗಳು. ಪೊಂಮ ಚಾವಡಿ ಎಂಬ ತೆರಿಗೆಗಳನ್ನು ಸಂಗ್ರಹಿಸುವ ವಿಭಾಗವೇ ಮೈಸೂರು ಅರಸರ ಕಾಲ\-ದಲ್ಲಿತ್ತು. “ತೋಣ್ಟದಾಯ\index{ತೋಣ್ಟದಾಯ} ವೊಳಗಾದ”,\endnote{ ಎಕ 7 ಮವ 144 ಕಲ್ಕುಣಿ 1318} ಎಂದು ಒಂದು ಶಾಸದಲ್ಲಿ ಹೇಳಿದ್ದು ತೋಟದ ಆದಾಯದ ಮೇಲೆ ಸುಂಕವನ್ನು ವಿಧಿಸಲಾಗುತ್ತಿತ್ತು.

\newpage

\section*{ವ್ಯವಸಾಯಗಾರರು/ಒಕ್ಕಲುಮಕ್ಕಳು}
\index{ಒಕ್ಕಲು ಮಕ್ಕಳು}

ಜಮೀನುಗಳಲ್ಲಿ ಉತ್ತು ಬಿತ್ತು ಮಾಡುವ ರೈತರನ್ನು, ಒಕ್ಕಲು, ಒಕ್ಕಲು ಮಕ್ಕಳು, ಭೂಮಿದೇವಿಯ ಮಕ್ಕಳು, ಎಂದು ಕರೆಯಲಾಗಿದೆ.\endnote{ ಸ್ವಾಮಿ ಡಾ|| ಬಿ.ಜಿ.ಎಲ್​., ಶಾಸನಗಳಲ್ಲಿ ಗಿಡಮರಗಳು, ಪುಟ 5

Gururajachar Dr.S., Some aspects of economic and social life in Karnataka, pp.40} ಇವರನ್ನು ಬಹಳ ಗೌರವದಿಂದ ಕಾಣಲಾಗುತ್ತಿತ್ತು ಎಂದು ಹೇಳಿದೆ. “ತ್ರಿಭುವನಮಲ್ಲ ಪೊಯ್ಸಳದೇವನ ಪಾದಪದ್ಮೋಪಜೀವಿ ಶ್ರಿಮತೊಕ್ಕಲ ಬಲ್ಲಹ ಚಕ್ಕರ ಭೀಮ”\index{ಶ್ರಿಮತೊಕ್ಕಲ ಬಲ್ಲಹ ಚಕ್ಕರ ಭೀಮ} ಎಂಬುವವನ ಉಲ್ಲೇಖ ಕಡೂರು ಶಾಸನದಲ್ಲಿದೆ.\endnote{ ಎಕ 12 ಕಡೂರು 76 ಗರ್ಜಿ 1089} ಒಕ್ಕಲು ಎಂಬುದು ವ್ಯವಸಾಯಗಾರರ ಸಂಘ\index{ವ್ಯವಸಾಯಗಾರರ ಸಂಘ} ಎಂದೂ ಹೇಳಿದ್ದೂ, ಶಾಸನಗಳಲ್ಲಿ ‘ಐವತ್ತೊಕ್ಕಲು’,\index{ಐವತ್ತೊಕ್ಕಲು} ‘ಮೂವತ್ತೊಕ್ಕಲು’ಗಳ\index{ಮೂವತ್ತೊಕ್ಕಲು} ಉಲ್ಲೇಖಗಳಿವೆ.\endnote{ Venkatarathnam, Dr.A.V., Local Government in the Vijayanagara Empire, pp. 25, 27} “ತೋಟ ತುಡಿಕೆ ಕಳಮನೆ ಒಕ್ಕಲುಮಕ್ಕಳು ಸುಂಕ ಸುವರ್ಣ್ನಾದಾಯ”,\endnote{ ಎಕ 6 ಪಾಂಪು 139 ಮೇಲುಕೋಟೆ 1450

ಎಕ 6 ಪಾಂಪು 19 ಸೀತಾಪುರ 1467

ಎಕ 6 ಪಾಂಪು 132 ಮೇಲುಕೋಟೆ 1530

ಎಕ 6 ಪಾಂಪು 129 ಮೇಲುಕೋಟೆ 1545

ಎಕ 6 ಪಾಂಪು 128 ಮೇಲುಕೋಟೆ 1564} ಎಂದು ಅನೇಕ ಶಾಸನಗಳಲ್ಲಿ ಹೇಳಿದೆ. ಇದು ಒಕ್ಕಲು/ಒಕ್ಕಲುಮಕ್ಕಳ ಮೇಲೆ ವಿಧಿಸುತ್ತಿದ್ದ ತೆರಿಗೆ ಎಂದು ಹೇಳಬಹುದು. ಒಕ್ಕಲು ಮಕ್ಕಳನ್ನು ಬಹಳ ಗೌರವದಿಂದ ಕಾಣಲಾಗುತ್ತಿತ್ತೆಂದು ವಿದ್ವಾಂಸರು ಹೇಳಿದ್ದಾರೆ.\endnote{ Gururajachar Dr.S., Agriculture, Economic and Social Life in Karnataka, pp.40-41} ಮುಖ್ಯವಾಗಿ ಶೂದ್ರರು, ಗೌಡರು, ಒಕ್ಕಲಿಗರು ಮತ್ತು ಹೊಲೆಯರು ಒಕ್ಕಲುತನದಲ್ಲಿ ಅಂದರೆ ವ್ಯವಸಾಯದಲ್ಲಿ ತೊಡಗಿರುತ್ತಿದ್ದರೆಂದು ಹೇಳಲಾಗಿದೆ.\endnote{ Shivanna, Dr.K.S., The Agrarian System of Karnataka, pp.18-19} ತೊಂಡನೂರಿನ ತಿಲ್ಲೆಕೂತ್ತವಿಣ್ನಘರ್​ ದೇವರಿಗೆ ದತ್ತಿ ಬಿಡುವಾಗ ಮೂವತ್ತೂರ ಪ್ರಭುಗವುಡುಗಳು ಒಕ್ಕಲುಗೂಡಿದ್ದರೆಂದು\index{ಒಕ್ಕಲುಗೂಡಿ} ಹೇಳಿದೆ.\endnote{ ಎಕ 6 ಪಾಂಪು 88 ತೊಣ್ಣೂರು 1157} ಒಕ್ಕಲುಗೂಡುವುದು ಎಂದರೆ ಒಕ್ಕಲು ಮಕ್ಕಳು ಸಭೆ ಸೇರುವುದು ಎಂಬ ಅರ್ಥ ಬರುತ್ತದೆ. “ಆ ಕ್ಷೇತ್ರ ಮಾಡುವ ವೊಕ್ಕಲು\index{ವೊಕ್ಕಲು} ರಂಗಗವುಡ। ಹಿರಿಯಂಣ್ನನ ಮಗ ಚವುಡಯ್ಯ। ಬೀರಗವುಡ। ಪಟ್ಟಣಸ್ವಾಮಿ ಮಾಚಿಸೆಟ್ಟಿ। ಬಾಚೆಯಸಾಹಣಿ। ಬಲ್ಲೆಯಹೆಗ್ಗಡೆಯ ಅಲ್ಲಪ್ಪ। ಆ ಚಿಕ್ಕಬಲ್ಲಯ್ಯ।ಸಂಚಿಯಂಣ್ನ। ತಿಪ್ಪನು। ಆ ಪೆರುಮಾಳೆದೇವ ದಂಣ್ನಾಯಕರ ಮಗ ನಾರಣದೇವಂಣ್ನ। ಅದ್ದಿಕದ ಸಿಂಗಣ್ನ।ನಾಣಗೌಡಿಯ ಗೋಪಾಳದೇವ। ಅನ್ತೂ ಈ ಭಾಗೆಯ ಒಕ್ಕಲು ಹನ್ನೆರಡಕ್ಕಂ ಮನೆದೆಱೆ ಮಾನ್ಯ।” ಎಂದು ಬೆಳ್ಳೂರು ಶಾಸನದಲ್ಲಿ ಹೇಳಿದ್ದು, ಇದರಿಂದ ಎಲ್ಲ ಕೋಮಿನವರೂ ಒಕ್ಕಲುಗಳಾಗಿ ವ್ಯವಸಾಯ ಮಾಡುತ್ತಿದ್ದರು ಎಂದು ಹೇಳಬಹುದು.\endnote{ ಎಕ 7 ನಾಮಂ 74 ಬೆಳ್ಳೂರು 1271} ಒಕ್ಕಲುಗಳು ಅವರ ಕೈಕೆಳಗೆ ಜೀತಗಾರರು ಅಥವಾ ಕೆಲಸಗಾರರನ್ನು ಇಟ್ಟುಕೊಂಡು ದೊಡ್ಡ ದೊಡ್ಡ ಕ್ಷೇತ್ರವನ್ನು ವ್ಯವಸಾಯ ಮಾಡಿಸುತ್ತಿದ್ದರೆಂದು ಹೇಳಬಹುದು. ಇವರು “ಒಕ್ಕಲುದೆರೆ”\index{ಒಕ್ಕಲುದೆರೆ} ಎಂಬ ತೆರಿಗೆಯನ್ನು ನೀಡಬೇಕಾಗುತ್ತಿತ್ತು.\endnote{ ಗುರುರಾಜಾಚಾರ್​, ಪೂರ್ವೋಕ್ತ, ಪುಟ 161} “ಜೀತಗಾರರ\index{ಜೀತಗಾರರು} ಕಂಬಳಗಾರರ\index{ಕಂಬಳಗಾರರು} ಸುಂಕ”,\endnote{ ಎಕ 6 ಪಾಂಪು 215 ಮೇಲುಕೋಟೆ 1724} ಎಂದು ಹೇಳಿದ್ದು, ಇವರೂ ಕೂಡಾ ವ್ಯವಸಾಯ ಮಾಡುವವರಾಗಿದ್ದು, ಇವರ ಮೇಲೆ ಸುಂಕವನ್ನು ವಿಧಿಸಲಾಗುತ್ತಿತ್ತೆಂದು ಹೇಳಬಹುದು. ಕಂಬಳಗಾರ ಎಂದರೆ ಎಮ್ಮೆ ಕೋಣಗಳ ಮೂಲಕ ಗದ್ದೆಯನ್ನು ಉಳುತ್ತಿದ್ದವರೆಂದು ಊಹಿಸಬಹುದು. “ಒಕಲಲ್ಲದೆ ಹೊಸ ಒಕಲಿಂಗೆ”,\endnote{ ಎಕ 6 ಪಾಂಪು 31 ಕನ್ನಂಬಾಡಿ 12-13ನೇ ಶ.} ಎಂದರೆ ಆ ಊರಿನ ಮೂಲ ಒಕ್ಕಲುಗಳಲ್ಲದೆ ಹೊಸದಾಗಿ ಬಂದು ಸೇರಿಕೊಂಡ ಒಕ್ಕಲುಗಳಿಗೆ ತೆರಿಗೆಯನ್ನು ವಿಧಿಸಲಾಗಿದೆ. “ವೊಕ್ಕಲು ಬಹಸುಂಕ\index{ವೊಕ್ಕಲು ಬಹಸುಂಕ} ವೊಕಲು ಹೋಹಸುಂಕ”,\index{ವೊಕಲು ಹೋಹಸುಂಕ}\endnote{ ಎಕ 8 ಅರ 12 ಬೋಳಕ್ಯಾತನಹಳ್ಳಿ 1422} ಎಂಬ ಸುಂಕದ ಉಲ್ಲೇಖವೂ ಇದ್ದು, ಇದರಿಂದ ಒಂದು ಊರಿಗೆ ಹೊರಗಿನಿಂದ ಒಕ್ಕಲುಗಳನ್ನು ಕರೆದುಕೊಂಡು ಬಂದಾಗ, ಬೇರೆ ಊರುಗಳಿಗೆ ಒಕ್ಕಲುಗಳನ್ನು ಕರೆದುಕೊಂಡು ಹೋಗುವಾಗ ಸುಂಕವನ್ನು ನೀಡಬೇಕಾಗುತ್ತಿತ್ತು ಎಂದು ಹೇಳಬಹುದು. ಅಥವಾ ಒಕ್ಕಲುಗಳ ಅಥವಾ ಅದರ ಸಂಘದ ಸದಸ್ಯರುಗಳು ಸುಂಕವನ್ನು ನೀಡಬೇಕಾಗುತ್ತಿತ್ತೆಂದು ಹೇಳಬಹುದು. ಕೃಷ್ಣದೇವರಾಯನ ಮೇಲುಕೋಟೆಯ ಒಂದು ಶಾಸನದಲ್ಲಿ ಸಂಪತ್ಕುಮಾರನಾದ ಶ‍್ರೀ ಚೆಲ್ಲಪಿಳ್ಳೆರಾಯ ದೇವರ ಅಮೃತಪಡಿ ನೈವೇದ್ಯಕ್ಕೆ ದತ್ತಿಯನ್ನು ಬಿಡುವಾಗ “ಹಿಂದಕೆ ದೇವಸ್ಥಾನದ ಸೀಮೆಯವರನೂ ವೊಕ್ಕುಳಾರಿ\index{ವೊಕ್ಕುಳಾರಿ}(ಒಕ್ಕಲಾರಿ-ಒಕ್ಕಲಾಗಿದ್ದವರನ್ನು) ಹಿಡಿದುಕೊಂಡು ಹೋಗಿ ತೊಂಡನೂರ ಗದ್ದೆನೂ ವುಳಿಸುವ ಸಂಮಂಧ ಅದನೂ ನಾವು ಕುಳವಕಡಿದು ಬಿಟ್ಟೆವು” ಎಂದು ಹೇಳಿದೆ.\endnote{ ಎಕ 6 ಪಾಂಪು 134 ಮೇಲುಕೋಟೆ 1528} ಇದನ್ನು ಎಪಿಗ್ರಾಫಿಯಾ ಸಂಪಾದಕರು ಈ ರೀತಿ ಅರ್ಥೈಸಿದ್ದಾರೆ. “ಮೇಲುಕೋಟೆ ದೇವಸ್ಥಾನದ ಭೂಮಿಗಳನ್ನು ಉಳುತ್ತಿದ್ದ ಒಕ್ಕಲಿಗರನ್ನೇ ತೊಂಡನೂರಿನ ದೇವಸ್ಥಾನಕ್ಕೆ\break ಸೇರದೇ ಇರುವ ಭೂಮಿಗಳನ್ನು ಉಳಲು ಬಲವಂತವಾಗಿ(?) ಕರೆದೊಯ್ಯುತ್ತಿದ್ದ ಕಾರಣ ಈಗ ತೊಂಡನೂರಿನ ಭೂಮಿಗಳನ್ನೂ ದೇವಸ್ಥಾನಕ್ಕೇ ನೀಡಿದುದಾಗಿ, ಇದನ್ನು ಅರ್ಥೈಸಬಹುದು, ಇದರಿಂದ ಈ ಎಲ್ಲ ಭೂಮಿಗಳ ಒಡೆತನ ದೇವಸ್ಥಾನಕ್ಕೇ ಸೇರಿದ್ದು ಅವುಗಳನ್ನು ಉಳಲು ಒಕ್ಕಲಿಗರನ್ನು ಸೂಕ್ತವಾಗಿ ನಿಯಮಿಸಿಕೊಳ್ಳುವ ಹೊಣೆ ದೇವಸ್ಥಾನದ್ದೇ. ಆದುದರಿಂದ ಒಕ್ಕಲುಗಳನ್ನು ಬಲವಂತದಿಂದ ಬೇರೆಡೆಗೆ ಕರೆದೊಯ್ಯುವ ಅಭ್ಯಾಸಕ್ಕೆ ಮುಕ್ತಾಯ ಹಾಕಿದಂತಾಯಿತೇನೋ”.\endnote{ ಎಪಿಗ್ರಾಫಿಯಾ ಕರ್ನಾಟಿಕಾ, ಸಂಪುಟ 6, ಪೀಠಿಕೆ, ಪುಟ liii} ಬಲವಂತವಾಗಿ ಕರೆದುಕೊಂಡು ಹೋಗಿ ಎನ್ನುವ ಬದಲು, ಈ ಹಿಂದೆ ಯಾರು ಈ ಭೂಮಿಯಲ್ಲಿ ಒಕ್ಕಲುತನ ಮಾಡುತ್ತಿದ್ದರೋ ಅವರನ್ನು ಗುರುತಿಸಿ, ಈ ಭೂಮಿಯನ್ನು ಮತ್ತೆ ಅವರಿಗೇ ಒಕ್ಕಲುತನ ಮಾಡಲು ನೀಡಿರಬಹುದೆಂದು ಊಹಿಸಬಹುದು.

\newpage

\section*{ಬೀಳುಭೂಮಿ/ಮೊರಡಿ/ಬೋರೆ}
\index{ಮೊರಡಿ}\index{ಬೋರೆ}

ವ್ಯವಸಾಯಕ್ಕೆ ಯೋಗ್ಯವಲ್ಲದ ಬೀಳು ಭೂಮಿಯನ್ನು ಮೊರಡಿ, ಬೋರೆ ಎಂದು ಕರೆದಿದೆ. ”ಕಗ್ಗಲ್ಲಮೊರಡಿ”,\index{ಕಗ್ಗಲ್ಲಮೊರಡಿ}\endnote{ ಎಕ 7 ನಾಮಂ 1 ನಾಗಮಂಗಲ 1171

ಎಕ 7 ನಾಮಂ 61 ಲಾಲನಕೆರೆ 1138} “ತರಗೆಲೆಯ ಮೊರಡಿ”,\endnote{ ಎಕ 7 ನಾಮಂ 64 ಲಾಳನಕೆರೆ 1145} ದಿಣ್ಣೆ, ಹಾಳು ಎಂದೂ ಕರೆಯಲಾಗಿದೆ. “ಹೊತ್ತಿಯ ಮರಡಿ\index{ಹೊತ್ತಿಯ ಮರಡಿ} ಎಂಬ ಬೆಣಚುಕಲ್ಲು ಮರಡಿಯನ್ನು\index{ಬೆಣಚುಕಲ್ಲು ಮರಡಿ} ಬಲದಲ್ಲಿ ಹಾಯ್ಕಿಕೊಂಡು”, “ಕಳ್ಳಿವಬ್ಬೆಯ\index{ಕಳ್ಳಿವಬ್ಬೆ} ಪೊದೆ ಬೆಳೆದಿದ್ದ ತೆವರೇ ಮೇರೆಯಾಗಿ ನಡೆದು ಹುಟ್ಟುಗಲ್ಲ ಕಿರು ಮರಡಿ\index{ಕಿರು ಮರಡಿ} ಮೇರೆ ಆ\break ತೆವರೇ ಮೇರೆಯಾಗಿ”,\endnote{ ಎಕ 6 ಶ‍್ರೀಪ 93 ನೆಲಮನೆ 1458} “ಅಥ ತೀವನ ಬೋರೆ\index{ತೀವನ ಬೋರೆ} ಇತ್ಯಸ್ಮಿಂ”,\endnote{ ಎಕ 6 ಕೃಪೇ 71 ಕೈಗೋನಹಳ್ಳಿ 1462} ಎಂದು ದತ್ತಿಯ ಭೂಮಿಯ ಮೇರೆಯನ್ನು \hbox{ಮೊರಡಿ}/ಮರಡಿಗಳ ಮೂಲಕ ಗುರುತಿಸಲಾಗಿದೆ.


\section*{ನಂದನವನಗಳು/ಪುಷ್ಪೋದ್ಯಾನಗಳು/ತೋಪುಗಳು}
\index{ನಂದನವನ}\index{ಪುಷ್ಪೋದ್ಯಾನ}\index{ತೋಪುಗಳು}

ದೇವಾಲಯದ ದಿನನಿತ್ಯದ ಪೂಜೆಗೆ ಅಗತ್ಯವಾದ ಹೂವುಗಳನ್ನು ಬೆಳೆಸುವುದಕ್ಕೆ ಭೂಮಿಯನ್ನು ಮೀಸಲಾಗಿ ಬಿಟ್ಟಿದ್ದು, ಇವುಗಳನ್ನು ಪುಷ್ಪೋದ್ಯಾನಗಳು, ನಂದನವನ, ಹೂತೋಟ ಎಂದು ಕರೆದಿದೆ. ಅದೇ ರೀತಿ, ಊರಿನ ಮುಂದೆ ಮರದ ತೋಪು\-ಗಳನ್ನು ಮತ್ತು ದಾರಿಯ ಬದಿಯಲ್ಲಿ ವಿಶ್ರಮಿಸಲು ಮರದ ತೋಪುಗಳನ್ನು ಬೆಳೆಸಲಾಗುತ್ತಿತ್ತು ಎಂಬುದು ಜಿಲ್ಲೆಯ ಶಾಸನಗಳಿಂದ ತಿಳಿದುಬರುತ್ತದೆ.

ಆರಣಿಯಲ್ಲಿ ಸಂತೆಯ ಮೈದಾನದ ಮುಂದೆ ಮಂಡಲಸ್ವಾಮಿಯು ಕವಿಗಳ ಮನವನ್ನೂ ಸೆಳೆಯುವ ಸುಂದರವಾದ ತಳ್ತಾರವೆಯನ್ನು (ಮರಗಳ ತೋಪು ಉದ್ಯಾನವನ) ನಿರ್ಮಿಸಿದ್ದನು. \textbf{“ಆರಣಿಯಲ್ಲಿ ನೀಳ್ದೆಸೆವುವಂ ಕವಿಗಳು ಸಿರಿಗಿರ್ಕ್ಕೆ ದಾರು\-ಮಗಲ್ದು ಪೋಗದಿರೆ ಸಂತೆಯನಂತದರಿಂದ ಮುಂದೆ ತಳ್ತಾರವೆಯೊಪ್ಪುವಂ\index{ತಳ್ತಾರವೆ} ಕರಕಳಂ ಪೆಸರ್ವ್ವೆತ್ತುವನೊಳ್ಪುಮೀರೆ ಬೆಳ್ಳೂರೊಳ\-ಗಿರ್ದ್ದು ಮಾಡಿಸಿದ ಮಂಡಲ ಸಾಮಿ ಸದರ್ತ್ಥನಲ್ಲನೇ”} ಎಂದು ಶಾಸನವು ಹೊಗಳಿದೆ.\endnote{ ಎಕ 7 ನಾಮಂ 80 ಬೆಳ್ಳೂರು 1199} ಆರವೆಯ\index{ಆರವೆ} ಉಲ್ಲೇಖ ಸುಜ್ಜಲೂರು ಶಾಸನದಲ್ಲಿದೆ.\endnote{ ಎಕ 7 ಮವ 139 ಸುಜ್ಜಲೂರು 1473} ವಿಜಯನಗರ ಕಾಲದಲ್ಲಿ ಭಟ್ಟರ ಬಾಚಿಯಪ್ಪನೆಂಬ ಅಧಿಕಾರಿಯು ನಾಲ್ಕು ದಿಕ್ಕಿನಲ್ಲೂ ಸಾಲುಮರ\-ಗಳನ್ನು ನೆಡಿಸಿದನು, ಚತುಸ್ಸೀಮೆಯೊಳಗೆ ನೆಡಿಸಿದ ಅರಳಿಯ ಮರ\-ಗಳಿಗೆ\index{ಅರಳಿಮರ} ಮುಂಜಿಯನ್ನು\index{ಮುಂಜಿ} ಕಟ್ಟಿಸಿದನು.\endnote{ ಎಕ 7 ಮ 93 ಅರುವನಹಳ್ಳಿ 1358} ಅರಳಿಯ ಮರಕ್ಕೆ\break ಮುಂಜಿ(ಮದುವೆ) ಮಾಡುವುದು ಎಂದರೆ ಅದರ ಪಕ್ಕದಲ್ಲಿ ಒಂದು ಬೇವಿನ ಮರವನ್ನು ನೆಡಿಸಿ ಎರಡಕ್ಕೂ ಹಸಿನೂಲನ್ನು ಸುತ್ತಿ ಶಾಸ್ತ್ರವನ್ನು ಮಾಡುತ್ತಾರೆ. ಇದು ಕಳೆದ 40-50 ವರ್ಷಗಳ ಹಿಂದೆಯೂ ರೂಢಿಯಲ್ಲಿತ್ತು. ಈಗಲೂ ಅಲ್ಲಲ್ಲಿ ಇದೆ.

ವಿರ್ರಿರುಂದ ಪೆರುಮಾಳೆ ದೇವರ ನಂದನವನಕ್ಕೆ ಸುಂಕವನ್ನು ದತ್ತಿ ಬಿಡಲಾಗಿದೆ.\endnote{ ಎಕ 6 ಪಾಂಪು 80 ಮೇಲುಕೋಟೆ 1177} ಮೇಲುಕೋಟೆಯ ತಿರು\-ನಾರಾಯಣಪೆರುಮಾಳ್​ ದೇವರ ತಿರುವಿಡೈಯಾಟ್ಟದ\index{ತಿರುವಿಡೈಯಾಟ್ಟ} ಭೂಮಿಯಲ್ಲಿ ತಿರುನಾರಾಯಣನ್​ ತಿರುನಂದನವನಕ್ಕೆ\index{ತಿರುನಂದನವನ (ನಂದಾವನ-ತಿರುನಂದನ)} ನಾಲ್ಕು ಹೊನ್ನನ್ನು ದತ್ತಿಯಾಗಿ ಬಿಡಲಾಗಿದೆ.\endnote{ ಎಕ 6 ಪಾಂಪು 121 ತೊಣ್ಣೂರು 1198} “ಬಸರಿವಾಳದ ಹಿರಿಯಕೆರೆಯ ತುಂಬಿನಿಂ ತೆಂಕಲು ಬಾಯಿಕಾಲ ದೊಡ್ಡರಳಿಯಗೊಂದಿವ\-ರೆಗಂ ದೇವರ ಹೂದೋಟದ\index{ಹೂದೋಟ} ಭೂಮಿ,\endnote{ ಎಕ 7 ಮಂ 29 ಬಸರಾಳು 1234} ಎಂಬ ಉಲ್ಲೇಖವು ದೇವಾಲಯಕ್ಕೆ ಸಾಕಷ್ಟು ದೊಡ್ಡ ಪ್ರಮಾಣದಲ್ಲಿ ಹೂದೋಟದ ಭೂಮಿಯನ್ನು ಬಿಟ್ಟಿರುವುದನ್ನು ಸೂಚಿಸುತ್ತದೆ. “ಎರಹಿಸೆಟ್ಟಿ ಹೂವಿಂಗೆ ಗ 1, ಮತ್ತಂ ಹೂದೋಟದ ಭೂಮಿಗೆ ಗ 1” ಎಂದು ದತ್ತಿಯನ್ನು ಬಿಡಲಾಗಿದೆ.\endnote{ ಎಕ 10 ಅರ 26 ಅರಸೀಕೆರೆ 1204} ಜೊತೆಗೆ ಈ ಹೂದೋಟದ ತೋಟಿಗರಿಗೆ 2 ಗದ್ಯಾಣ ಸಂಬಳವನ್ನು ನಿಗದಿಪಡಿಸಲಾಗಿದೆ.\endnote{ ಎಕ 7 ಮಂ 30 ಬಸರಾಳು 1237} ಅರಕೆರೆಯ ಅಯ್ಯಾದಕ್ಕನು ತನ್ನ ದತ್ತಿಯಲ್ಲಿ ದೇವರ ತಿರುನಂದನವನ\index{ತಿರುನಂದನವನ (ನಂದಾವನ-ತಿರುನಂದನ)} ನಿರ್ವಹಣೆಗೆ ದತ್ತಿ ಬಿಟ್ಟಿದ್ದಾಳೆ. “ಈ ಪಾದವೃತ್ತಿಯ ಪ್ರತಿ ವರ್ಷದ ಸಮಸ್ತ ಬೆಳೆ ಉತ್ಪತ್ತಿಯನು ಆ ದೇವರ ತಿರಿನಂದಾವನ\index{ತಿರುನಂದನವನ (ನಂದಾವನ-ತಿರುನಂದನ)} ಮಾಡುವವರ ಪ್ರತಿವರ್ಷ ಜೀವಿತಮುಖ್ಯವಾಗಿದ್ದ ಆ ದೇವರ ಶ‍್ರೀ ಕಾರ್ಯದ ಅಂಗಭೋಗಕ್ಕೆ ಕೊಡುವರು” ಎಂದು ಹೇಳಿದ್ದು, ದೇವರಿಗೆ ತಿರಿನಂದನವನವನ್ನು ಬಿಟ್ಟಿದ್ದು ಹಾಗೂ ಅದನ್ನು\break ನಿರ್ವಹಿಸುವವರಿಗೂ ಕೂಡಾ ದತ್ತಿಯನ್ನು ಬಿಡಲಾಗಿದೆ.\endnote{ ಎಕ 6 ಶ‍್ರೀಪ 98 ಅರಕೆರೆ 1254} “ತಿರುನಂದನ, ಧರ್ಮ್ಮದ ತೋಪು”ಗಳ,\endnote{ ಎಕ 6 ಪಾಂಪು 215 ಮೇಲುಕೋಟೆ 1724} ಉಲ್ಲೇಖ ಮೇಲುಕೋಟೆಯ ಶಾಸನದಲ್ಲಿದೆ. ತಿರುನಂದಾವನ ಮೊದಲಾದ ಕೈಂಕರ್ಯದ ಸಲುವಾಗಿ ಹಳ್ಳಿಗಳ ಸುಂಕವನ್ನು ದತ್ತಿ ಬಿಡಲಾಗಿದ್ದು, ನಂದಾವನ ಮಾಡುವ ಸಂಬಳ 6 ವರಹವನ್ನು ಗೊತ್ತು ಮಾಡಲಾಗಿದೆ.\endnote{ ಎಕ 6 ಪಾಂಪು 125 ಮೇಲುಕೋಟೆ 1535} “ತಿರುನಂದಾವನ ಸಂಬಳ ಚೆರುಪು ಸಹ ಗ 57ನ್ನು”,\endnote{ ಎಕ 6 ಪಾಂಪು 128 ಮೇಲುಕೋಟೆ 1564} ದತ್ತಿಯಾಗಿ ಬಿಡಲಾಗಿದೆ. ಮೇಲುಕೋಟೆಯಲ್ಲಿ “ಅಪ್ಪಯಂಗಾರೂ ಪಂಚಭಾಗವತ ಸ್ಥಳದಲೂ\index{ಪಂಚಭಾಗವತ ಸ್ಥಳ} ಮಾಡಿದ ತಿರುನಂದಾವನದಲಿ ಚಿಕತಿಂನಾಳ (ತಿರುನಾಳ್) ಆಯಿದನೆಯ ತಿಂನಾಳಿನಲೂ ಸ್ವಾಮಿ ಬಿಜೆ ಮಾಡಿ ಆರೋಗಣೆ ಚಿತಯಿಸಿ”,\endnote{ ಎಕ 6 ಪಾಂಪು 131 ಮೇಲುಕೋಟೆ 1551} ಎಂದು ಹೇಳಿದ್ದು, ಈ ತಿರುನಂದನವನಕ್ಕೆ ವಿಶೇಷ ದಿವಸಗಳಲ್ಲಿ ದೇವರ ಉತ್ಸವವು ಬಿಜಯಿ ಮಾಡಿಸುತ್ತಿತ್ತು. ಅಲ್ಲಿ ಸಮಾರಾಧನೆ ಮತ್ತು ಆರೋಗಣೆಗಳು ನಡೆಯುತ್ತಿದ್ದವು ಎಂಬುದು ತಿಳಿದುಬರುತ್ತದೆ. ಈ ಪಂಚಭಾಗವತ ಸ್ಥಳದಲ್ಲಿ ಈಗಲೂ ನೂರುವರ್ಷ ಹಳೆಯ\-ದಾದ ಮರಗಳಿರುವುದನ್ನು ನೋಡಬಹುದು. ಅರವೀಡು ವಂಶದ ಶ‍್ರೀ ರಂಗರಾಯನ ಕಾಲದಲ್ಲಿ ಚೋಳವೆಂಕಟ\-ಪತಿಯು ಬೇಲೂರು ಚೆನ್ನಕೇಶವ\-ದೇವರಿಗೆ ಚೆಲುಪುಷ್ಪದ ತೋಟ\index{ಚೆಲುಪುಷ್ಪದ ತೋಟ} ಒಂದನ್ನು ದತ್ತಿಯಾಗಿ ಬಿಟ್ಟನೆಂದು ಶ‍್ರೀರಂಗಪಟ್ಟಣ ಶಾಸನದಲ್ಲಿ ಹೇಳಿದೆ.\endnote{ ಎಕ 6 ಶ‍್ರೀಪ 22 ಶ‍್ರೀರಂಗಪಟ್ಟಣ 1662}

ತೋಟಿಗರಲ್ಲದೆ ಮಾಲೆಗಾರರು\index{ಮಾಲೆಗಾರರು} ಎಂಬುವವರೂ ಕೂಡಾ ಇದ್ದರು. “ಪೂನಿರಿವ ಮಾಲಕಾರಂಗೆ” ದತ್ತಿಯನ್ನು ಬಿಟ್ಟಿರುವ ಪ್ರಾಚೀನ ಉಲ್ಲೇಖ ಬಾದಾಮಿ ಶಾಸನದಲ್ಲಿದೆ. ಇವರು ದೇವರ ಹೂದೋಟದ ನಿರ್ವಹಣೆಯನ್ನು ಮಾಡುತ್ತಿರಲಿಲ್ಲ. ಬದಲಿಗೆ ಅದರಿಂದ ಹೂಗಳನ್ನು (ಕುಯ್ದು) ಸಂಗ್ರಹಿಸಿ ಮಾಲೆಯನ್ನು ಕಟ್ಟಿ ದೇವರಿಗೆ ಅರ್ಪಿಸುವ ಕಾಯಕ ಮಾಡುತ್ತಿದ್ದರೆಂದು ಹೇಳಬಹುದು. ಬಸರಾಳು ಶಾಸನದಲ್ಲಿ ಹೂದೋಟದ ತೋಟಿಗರಿಗೆ\index{ತೋಟಿಗ} ಮತ್ತು ಮಾಲೆಗಾರರಿಗೆ ಪ್ರತ್ಯೇಕವಾಗಿ ವೇತನವನ್ನು ನಿಗದಿಪಡಿಸಿರುವುದರಿಂದ ಇದು ಖಚಿತವಾಗುತ್ತದೆ.\endnote{ ಎಕ 7 ಮಂ 30 ಬಸರಾಳು 1237}

\section*{ಜಿಲ್ಲೆಯ ಶಾಸನಗಳಲ್ಲಿ ಮರಗಿಡಗಳು}

ಮರಗಳು ಪರಿಸರ ಅಥವಾ ಪ್ರಕೃತಿಯ ಹಾಗೂ ಮಾನವನ ಜೀವನದ ಅವಿಭಾಜ್ಯ ಅಂಗಗಳಾಗಿವೆ. ಮರಗಳಿಲ್ಲದೆ ಪರಿಸರವೇ ಇಲ್ಲ. ನಮ್ಮ ಪ್ರಾಚೀನರು ಮರಗಳಿಗೆ ಬಹಳ ಪ್ರಾಮುಖ್ಯತೆಯನ್ನು ಕೊಡುತ್ತಿದ್ದರು. ಮರವನ್ನೇ ಸಾಕ್ಷಿಯನ್ನಾಗಿ ಮಾಡಿ ವ್ಯವಹಾರ ನಡೆಯುತ್ತಿತ್ತು. ಅನೇಕ ಕಡೆ ಇತ್ತೀಚಿನವರೆಗೂ ಹಳ್ಳಿಯ ರೈತರು ತಮ್ಮ ಜಮೀನಿನಲ್ಲಿ ಒಂದು ಮರವನ್ನು ದೇವರ ಮರವೆಂದು ಬಿಟ್ಟು ಅದನ್ನು ಯಾವ ಕಾರಣಕ್ಕೂ ಕಡಿಯುತ್ತಿರಲಿಲ್ಲ. ಸಾಮಾನ್ಯವಾಗಿ ಗ್ರಾಮಾಂತರ ಪ್ರದೇಶದ ಮನೆತನಗಳಿಗೆ (ಕುಲಗಳಿಗೆ) ಅವರದ್ದೇ ಆದ ಮರಗಳಿವೆ. ಮದುವೆ ಮುಂಜಿ ಮುಂತಾದ ಸಮಾರಂಭಗಳಲ್ಲಿ ಆ ಮರದ ಕೊಂಬೆಯನ್ನು ತಂದು ಅದಕ್ಕೆ ಸೀರೆ ಉಡಿಸಿ ಪೂಜಿಸುವುದನ್ನು ಇಂದಿಗೂ ಕಾಣಬಹುದು. ಹಿಂದಿನ ಕಾಲದಲ್ಲಿ ಮದುವೆ ಆದಾಗ ಅವರ ಕುಲವೃಕ್ಷದ ಕೊಂಬೆಯನ್ನು ತಂದು ಅದನ್ನು ಮನೆಯ ಮುಂದೆ ಮದುವೆ ಚಪ್ಪರದಲ್ಲಿ ನೆಟ್ಟು, ಅದರ ಕೆಳಗೆ ಹಸೆಮಣೆಯನ್ನು ಮಾಡಿ, ಗಂಡು ಹೆಣ್ಣನ್ನು ಕೂರಿಸಿ, ಮದುವೆ ಮಾಡುತ್ತಿದ್ದ ದೃಶ್ಯಗಳನ್ನು ನೋಡಬಹುದಾಗಿತ್ತು. ಆದರೆ ಈಗ ಈ ಪದ್ಧತಿ ಮರೆಯಾಗುತ್ತಿದೆ. ಅಂತಹ ಮರಗಳೇ ಇಂದು ದೊಡ್ಡ ದೊಡ್ಡ ಮರಗಳಾಗಿ 200-300 ವರ್ಷಗಳಿಂದ ಬೆಳೆದು ನಿಂತಿವೆ. ಸಾಲುಮರಗಳನ್ನು ನೆಡುವ ಪದ್ಧತಿ ಅಶೋಕನ ಕಾಲದಿಂದಲೂ ಇದೆ. \textbf{“ನಾಲ್ಕುದಿಕ್ಕಲಿ ಸಾಲುಮರಗಳನು\index{ಸಾಲುಮರಗಳು} ಇಕ್ಕಿಸಿದೆವು”},\endnote{ ಎಕ 7 ಮ 93 ಅರುವನಹಳ್ಳಿ 1358} ಎಂದು ವಿಜಯನಗರ ಕಾಲದ ಶಾಸನದಲ್ಲಿ ಹೇಳಿದೆ. ಸುಜ್ಜಲೂರು ಶಾಸನದಲ್ಲಿ ಅಗ್ರಹಾರದ ಮೇರೆಯನ್ನು ಹೇಳುವಾಗ \textbf{“ಮರವಳಿಗದ್ದೆ,\index{ಮರವಳಿಗದ್ದೆ} ಯೆಲೆಂಗುಲಿ ತೆಂಗು\index{ತೆಂಗು} ಮಾವು\index{ಮಾವು}\general{\break } ಹಲಸು\index{ಹಲಸು} ಬಾಳೆ\index{ಬಾಳೆ} ಬದನೆ ಕಬ್ಬಿನಾಲೆ, ಕಾಲುಪಚ್ಚೆ, ಯೆಲೆಗುಳಿ ತೋಟಸ್ಥಳ”},\index{ಯೆಲೆಗುಳಿ ತೋಟಸ್ಥಳ}\endnote{ ಎಕ 7 ಮವ 139 ಸುಜ್ಜಲೂರು 1473} ಎಂಬುದಾಗಿ ಮರಗಳ ವಿವರಗಳನ್ನು ನೀಡಿದರೆ, ಮಳವಳ್ಳಿ ಶಾಸನದಲ್ಲಿ \textbf{“ರತಿಪ್ರೇಮಾಸ್ಪದೇ ಮಲ್ಲಿಕಾ ಜಾಜಿ ಚಂಪಕ\index{ಮಲ್ಲಿಕಾ ಜಾಜಿ ಚಂಪಕ} ಮುಖ್ಯಪುಷ್ಪ ನಿವಹೈರತ್ಯಂತ ಸಂಶೋಭಿತೇ”} ಎಂಬುದಾಗಿ ಹೂವಿನ ಗಿಡಗಳ ವಿವರಗಳನ್ನು ನೀಡಿದೆ.

ಮೊದಲಿಗೆ ಎಲ್ಲೆಲ್ಲೂ ಮರಗಿಡಗಳೇ ತುಂಬಿದ್ದು, ಭೂಮಿಯನ್ನು ವ್ಯವಸಾಯಕ್ಕೆ ಒಳಪಡಿಸಲು ಅವುಗಳನ್ನು\break ಕಡಿಯಲಾಯಿತು. ಅದೇ ರೀತಿ “ಶಾಸನೋಕ್ತವಾದ ಸೀಮಾಸಸ್ಯಗಳ ಅಧ್ಯಯನವನ್ನು ಮುಂದುವರಿಸುವುದರಿಂದ ಸಸ್ಯಶಾಸ್ತ್ರದ ಕೆಲವು ತೊಡಕುಗಳು ನಿವಾರಣೆಯಾಗುವ ಸೂಚನೆ ಕಂಡು ಬಂದಿದೆ. ಪೂರ್ವಕಾದಲ್ಲಿ ಬೆಳೆಯುತ್ತಿದ್ದ ಗಿಡಮರಗಳೇ ಇಂದಿಗೂ ನಮ್ಮ ನೆಲದ ಮೇಲೆ ಬೆಳೆಯುತ್ತಿವೆಯಾದರೂ ಅವುಗಳ ಹರವು ನಕಾಸೆಯೂ ಸಾಂಧ್ರತೆಯೂ ಹಾಗೆಯೇ ಉಳಿದುಕೊಂಡು\break ಬಂದಿದೆಯೆಂದು ಭಾವಿಸುವುದು ಸರಿಯಲ್ಲ” ಎಂಬ ಹೇಳಿಕೆಯೂ ಅದೇ ರೀತಿ “9ನೇ ಶತಮಾನದಿಂದ 13ನೇ ಶತಮಾನದವರೆಗೆ ಅಸಂಖ್ಯಾತ ಚತುರ್ವೇದಿ ಮಂಗಲಗಳೂ, ಬ್ರಹ್ಮಪುರಿಗಳೂ ನಿರ್ಮಾಣವಾದವು. ಲೆಕ್ಕವಿಲ್ಲದಷ್ಟು ಭೂವಿಸ್ತಾರಗಳು ಸಾಗು\-ವಳಿಗೆ ಒಳಗಾದವು. ಹೀಗಾಗಿ ಆ ಪ್ರದೇಶದ ವನ್ಯಜಾತಿಯ ಸಸ್ಯಗಳು ನಾಶವಾಗಿ ಹೊಸ ಜಾತಿಗಳು ಆಕ್ರಮಣ ಮಾಡುವುದಕ್ಕೆ ಅವಕಾಶವಾಯಿತು. ಪ್ರಾಚೀನ ಕಾಲದ ಪ್ರದೇಶವೊಂದರ ಸಸ್ಯಸಮೃದ್ಧಿಯನ್ನು ಅದು ವ್ಯತ್ಯಾಸಗೊಂಡ ರೀತಿನೀತಿಗಳನ್ನು ಅರಿ\-ಯಲು ನಮಗೆ ಈಗ ದೊರಕಿರುವ ಏಕಮಾತ್ರ ಸಾಧನವೆಂದರೆ ಶಾಸನಗಳು”\endnote{ ಸ್ವಾಮಿ ಡಾ|| ಬಿ.ಜಿ.ಎಲ್​., ಶಾಸನಗಳಲ್ಲಿ ಗಿಡಮರಗಳು.} ಎಂಬ ಹೇಳಿಕೆಯು ಗಮನಾರ್ಹ. ಗಂಗರಾಜ ಹಸ್ತಿಮಲ್ಲ (ಎರಡನೆಯ ಪೃಥಿವಿಪತಿ)ನ ಉದಯೇಂದಿರಮ್ ಶಾಸನದಲ್ಲಿ ಬ್ರಹ್ಮದೇಯವಾಗಿ ದಾನಕೊಟ್ಟ ಸಂದರ್ಭದಲ್ಲಿ ಆ ಭೂಮಿಯ ಮೇರೆಗಳನ್ನು ಹೇಳುವಾಗ ಉಲ್ಲೇಖಿಸಿರುವ ಆಲದ ಮರ,\index{ಆಲದ ಮರ} ಮರುದು(ಅರ್ಜುನ ಮರ),\index{ಮರುದು(ಅರ್ಜುನ ಮರ)} ಬೇವು,\index{ಬೇವು} ಕುರಾ\break (ಕುರವಕ), ಕಲ್ಲಾಲ, ಕುರಿಂಜಿಲು, ತಣಕ್ಕು, ಕಾರೆ ಇವುಗಳನ್ನು ವಿದ್ವಾಂಸರು ಉಲ್ಲೇಖಿಸಿದ್ದಾರೆ.\endnote{ ಅದೇ, ಪುಟ 6} ಇವುಗಳಿಗೆ ಹೊರತಾದ ಅನೇಕ\break ಮರಗಳು ಶಾಸನಗಳಲ್ಲಿ ಉಲ್ಲೇಖಿತವಾಗಿದೆ. ಮಂಡ್ಯ ಜಿಲ್ಲೆಯ ದೊಡ್ಡ ದೊಡ್ಡ ಶಿಲಾ ಶಾಸನಗಳು ಮತ್ತು ತಾಮ್ರ ಶಾಸನಗಳಲ್ಲಿ ದತ್ತಿ ನೀಡಿದ ಭೂಮಿಯ ಎಲ್ಲೆ ಅಥವಾ ಮೇರೆಗಳನ್ನು ಹೇಳುವಾಗ ಜಿಲ್ಲೆಯಲ್ಲಿ ಬಹಳವಾಗಿ ಬೆಳೆಯುತ್ತಿದ್ದ ಮರಗಳ ಹೆಸರುಗಳನ್ನು ಉಲ್ಲೇಖಿಸಿದ್ದು ಅವುಗಳನ್ನು ಈ ಕೆಳಗಿನಂತೆ ಗುರುತಿಸಬಹುದು.

ಮರಗಳನ್ನು ಒಟ್ಟಾಗಿ ಬೆಳೆಸುತ್ತಿದ್ದರು, ಇದನ್ನು ‘ಬನ’\index{ಬನ} ಅಥವಾ ‘ತೋಪು’\index{ತೋಪುಗಳು} ಎಂದು ಕರೆಯಲಾಗುತ್ತಿತು. “ತರು ನಂದನ ಧರ್ಮದ ತೋಪಿನ” ಉಲ್ಲೇಖ ಮೇಲುಕೋಟೆ ಶಾಸನದಲ್ಲಿದೆ. ರಂಗಾಪುರಕ್ಕೆ ಪಡುವಣ ತೋಪಿನ ಅಂಚಿನಲ್ಲಿ”, “ಈ ತೋಪಿನ ಪಟ್ಟಣದಮ್ಮನ\index{ತೋಪಿನ ಪಟ್ಟಣದಮ್ಮ} ಗುಡಿಗೆ ತೆಂಕಲಾಗಿ”,\endnote{ ಎಕ 6 ಪಾಂಪು 99 ತೊಣ್ಣೂರು 1722} ಎಂದು ತೊಣ್ಣೂರು ಶಾಸನದಲ್ಲಿ ಹೇಳಿದ್ದು, ಹಳ್ಳಿಗಳ ಬಳಿ ಮರದ ತೋಪು\-ಗಳು ಇದ್ದುದನ್ನು ಇದು ಸೂಚಿಸುತ್ತದೆ. ತೊಣ್ಣೂರಿನ ತೋಪಿನ ಪಟ್ಟಲದಮ್ಮನ ಗುಡಿಯ ಸುತ್ತ ಇಂದಿಗೂ ದೊಡ್ಡದೊಡ್ಡ ಮರಗಳ ತೋಪಿದೆ. ‘ಕಲ್ಲತ್ತಿಯಮರ’,\endnote{ ಎಕ 7 ನಾಮಂ 72 ಅಳೀಸಂದ್ರ 1183} ‘ಅತ್ತಿಮರ’,\endnote{ ಎಕ 7 ನಾಮಂ 157 ದೇವಲಾಪುರ 1472} ‘ಅತ್ತಿಹೊಲ’,\endnote{ ಎಕ 7 ಮವ 105 ತಿಗಡಹಳ್ಳಿ 1337} ಇವು ಅತ್ತಿಮರದ ಉಲ್ಲೇಖಗಳು.

ಜಿಲ್ಲೆಯ ಶಾಸನಗಳಲ್ಲಿ ಅರಳೀಮರಗಳ\index{ಅರಳಿಮರ} ಉಲ್ಲೇಖ ಹೆಚ್ಚಾಗಿದೆ. ‘ಹೊಂನರಳಿ’,\endnote{ ಎಕ 7 ನಾಮಂ 61 ಲಾಳನಕೆರೆ 1138} ‘ದೊಡ್ಡರಳಿಯ ಗೊಂದಿ\-ವರೆಗಂ’,\endnote{ ಎಕ 7 ಮಂ 29 ಬಸರಾಳು 1234} ‘ಅರಳಿಯ ಮರಗಳಿಗೆ ಮುಂಜಿಯನು ಕಟ್ಟಿಸಿದೆವು’,\endnote{ ಎಕ 7 ಮ 93 ಅರುವನಹಳ್ಳಿ 1358} “ಚಿಕ್ಕೋಜನಕಟ್ಟೆಯನ್ನೂ ಅರಳಿಮರನಂನೂ ಯೆಡದಲ್ಲಿ ಹಾಯಿಕಿಕೊಂಡು”,\endnote{ ಎಕ 6 ಶ‍್ರೀಪ 93 ನೆಲಮನೆ 1458} “ಮಾಯಣ್ಣನ ಅರಳಿಯ ಬಲಕಿಕ್ಕಿ”,\endnote{ ಎಕ 7 ಮವ 139 ಸುಜ್ಜಲೂರು 1473} ಮುಂತಾದ ಉಲ್ಲೇಖಗಳನ್ನು ನೋಡಬಹುದು. ಅರಳಿಮರಕ್ಕೆ ಶಾಸ್ತ್ರೋಕ್ತವಾಗಿ ಹಸೀನೂಲನ್ನು ಸುತ್ತಿ ಮುಂಜಿ ಮಾಡುವ ಸಂಪ್ರದಾಯ ಇಂದಿಗೂ ಇದೆ. “ಕುರುವದ ಮೂಲೆ ಅರಳಿಮರ”,\endnote{ ಎಕ 6 ಪಾಂಪು 99 ತೊಣ್ಣೂರು 1722} ಮಂಡ್ಯ ಜಿಲ್ಲೆಯಲ್ಲಿ ಹೆಚ್ಚಾಗಿ ಬೆಳೆಯುವ ಇನ್ನೊಂದು ಮರ ಅಂಕೋಲೆ. ಇದರಿಂದ ತಬಲ ಬಾರಿಸುವ ಕೋಲನ್ನು ತಯಾರಿಸು\-ತ್ತಾರೆ. ಇದು ಅನೇಕರ ವಂಶ ವೃಕ್ಷವಾಗಿದೆ. “ಹಿರಿಯ ಕೆರೆಯ ಕೆಳಗೆ ಅಂಕೋಲೆಯ\index{ಅಂಕೋಲೆ} ಮೇಡಿ\-ನಲ್ಲಿ” ಎಂಬುದು ಅಂಕೋಲೆ ಮರ\-ಗಳು ಹೆಚ್ಚಾಗಿ ಬೆಳೆದಿದ್ದ ಜಾಗವನ್ನು ಸೂಚಿಸುತ್ತದೆ.

ಮಂಡ್ಯ ಜಿಲ್ಲೆಯಲ್ಲಿ ಇಂದಿಗೂ ಕೂಡಾ ನೂರಿನ್ನೂರು ವರ್ಷದ ಆಲದ ಮರಗಳನ್ನು ಗ್ರಾಮಾಂತರ ಪ್ರದೇಶದಲ್ಲಿ ಕಾಣಬಹುದು. ಬೆಳ್ಳೂರಿನ ದೊಡ್ಡಜಟಕದ ಹತ್ತಿರ, ಮಳವಳ್ಳಿ ತಾಲ್ಲೂಕು ಕಂದಾಗಾಲದ ಹತ್ತಿರ ಸುಮಾರು 200-300 ವರ್ಷ\-ಗಳಿಗೂ ಹಳೆಯದಾದ ಬೃಹತ್​ ಆಲದ ಮರಗಳಿವೆ. ಸಂಸ್ಕೃತ ಶಾಸನಗಳಲ್ಲಿ ಇದನ್ನು ವಟವೃಕ್ಷ\index{ವಟವೃಕ್ಷ} ಎಂದು ಹೇಳಿದೆ. \textbf{“ಆಗ್ನೇಯಾಂ ವಟವೃಕ್ಷಸ್ಯ ಸಮೀಪೆ” “ಪಶ್ಚಿಮೇ ವಟವೃಕ್ಷಸ್ಯ ಶ್ವೇತಾ ವಾಮನ ರೂಪಿಣೇ”},\endnote{ ಎಕ 6 ಕೃಪೇ 71 ಕೈಗೋನಹಳ್ಳಿ 1462} ‘ಆಲದಮರ’, ‘ಅಡ್ಡವಟ್ಟೆಯಾಲದ ಮರ’,\endnote{ ಎಕ 7 ನಾಮಂ 61 ಲಾಳನಕೆರೆ 1138} ‘ಮೂಡ ಕುಂಟಾಲದಿಂ’,\index{ಮೂಡ ಕುಂಟಾಲ}\endnote{ ಎಕ 7 ನಾಮಂ 1 ನಾಗಮಂಗಲ 1173} ‘ಭಳರಿಯಾಲ,\index{ಭಳರಿಯಾಲ} ಆಲದ ಕಮರಿ’,\endnote{ ಎಕ 7 ನಾಮಂ 72 ಅಳೀಸಂದ್ರ 1183} “ಹಿರಿಯ ಆಲದ ಮರದಡಿಯ ಹಿರಿಯ ಅರೆ”,\endnote{ ಎಕ 7 ನಾಮಂ 168 ಕಸಲಗೆರೆ 1190} “ಬೆಂಗಾಲಗುಂಗೀಶಾನ್ಯ”,\endnote{ ಎಕ 7 ಮ 69 ವೈದ್ಯನಾಥಪುರ 1261} ಈ ರೀತಿ ಅನೇಕ ಉಲ್ಲೇಖಗಳಿವೆ. ಬೆಂಗಾಲಗುಂಗು ಎಂದರೆ ಆಲದ ತೋಪಿರಬಹುದು. ಪುಲಿಗೆರೆ ಕಡೆಯಿಂದ ಬಂದ “ಪ್ರಿಥುವಿಯ ಮಹಾಗಣಗಳು ಹೊಸಒಳಲ ಬಡಗಣ ಹೆಬಾಗಿಲ ಆಲದಮರದೆಲೆ ಸಿಂಹಾಸನದ\index{ಆಲದಮರದೆಲೆ ಸಿಂಹಾಸನ} ಮೇಲೆ ವಜ್ರಬಇಸಣಿಗೆಯನಿಕ್ಕಿ” ಕುಳಿತಿದ್ದರೆಂದು ಹೇಳಿದೆ.\endnote{ ಎಕ 6 ಕೃಪೇ 8 ಹೊಸಹೊಳಲು 1306} ಈಗಲೂ ಈ ಊರಿನ ಸುತ್ತಮುತ್ತ ಆಲದಮರ ಬಹುಸಂಖ್ಯೆಯಲ್ಲಿದೆ. “ಒಳಗೆರೆಯ ಆಲ,\index{ಒಳಗೆರೆಯ ಆಲ} ಅದರಿಂ ಮೂಡಣ ಕಲ್ಲರೆಯ ಆಲ”,\index{ಕಲ್ಲರೆಯ ಆಲ}\endnote{ ಎಕ 7 ಮವ 44 ನಡಗಲ್​ಪುರ 1510} “ಆಲೂರ ಯೆಲ್ಲೆ ಮಧ್ಯದ ಆಲದಮರದಿಂದ, ಬೇಡರಹಳ್ಳಿ ಯೆಲ್ಲೆಯ ಆಲದಿಂದ”,\endnote{ ಎಕ 7 ಮ 64 ಹೊನ್ನಲಗೆರೆ 1623} “ಆಲದ ತಾಳ ಬಳಿಯ ನೆಟ್ಟಕಲ್ಲು”,\endnote{ ಎಕ 6 ಪಾಂಪು 99 ತೊಣ್ಣೂರು 1722} ಇನ್ನೂ ಮುಂತಾಗಿ ಆಲದ ಮರದ ಉಲ್ಲೆಖವಿದೆ. “ಸಸಿಯಾಲದಪುರ” ಎಂಬ ಊರಿನ ಹೆಸರು ಶಾಸನೋಕ್ತವಾಗಿದೆ.\endnote{ ಎಕ 7 ಮವ 9 ಮಳವಳ್ಳಿ 1672} “ಮೇಲಣ ಆಲ, ಬ್ರಹ್ಮಚಾರಿ ಆಲ”ದ ಉಲ್ಲೇಖ ಸುಜ್ಜಲೂರು ಶಾಸನದಲ್ಲಿದೆ.\endnote{ ಎಕ 7 ಮವ 139 ಸುಜ್ಜಲೂರು 1473}

“ಹೆಬ್ಬಳ್ಳದ ತಡಿಯ ಈಚದ ತಾಳೊತ್ತಿನ”,\endnote{ ಎಕ 7 ಮಂ 7 ಮಂಡ್ಯ 1516} ಎಂದು ಮಂಡ್ಯ ತಾಮ್ರಶಾಸನದಲ್ಲಿ ಉಲ್ಲೇಖಿಸಿದ್ದು, ಈಚದ\break ತಾಳೊತ್ತಿನ ಎಂಬುದು ಈಚಲ ಮರದ ತೋಪು ಅಥವಾ ಸಾಲನ್ನು ಸೂಚಿಸುತ್ತದೆ. ಗೊಬ್ಬಳಿ ಮರಕ್ಕೆ ಕಂಟಕದ್ರುಮವೆಂದು ಹೇಳಲಾಗಿದೆ.\endnote{ ಎಕ 6 ಕೃಪೇ 71 ಕೈಗೋನಹಳ್ಳಿ 1462} ಗೋಣಿ ಮರವೂ ಆಲದ ಜಾತಿಗೆ ಸೇರಿದ ಮರ. ಇದರ ಸೊಪ್ಪು ಆಡು ಕುರಿಗಳಿಗೆ ಬಹಳ ಇಷ್ಟ. ಈ ಮರ ಈಚೆಗೆ ಬಹಳ ಕಡಿಮೆ ಆಗುತ್ತಿದೆ. ಗೋಣಿಮರದ\index{ಗೋಣಿಮರ} ಉಲ್ಲೇಖ ತೊಣ್ಣೂರು ಶಾಸನದಲ್ಲಿದೆ.\endnote{ ಎಕ 6 ಪಾಂಪು 99 ತೊಣ್ಣೂರು 1722} ಜಿಲ್ಲೆಯಲ್ಲಿ ಚುಜ್ಜಲ\index{ಚುಜ್ಜಲ} ಮರವು ಬಹಳವಾಗಿ ಬೆಳೆಯುತ್ತದೆ. ಇದರ ಸೊಪ್ಪನ್ನು ತಲೆಕೂದಲನ್ನು ಸ್ವಚ್ಛಗೊಳಿಸಲು ಉಪಯೋಗಿಸುತ್ತಾರೆ. “ಅಲ್ಲಿಂ ಬರಲು ಚುಜ್ಜಲ ಮರ”,\endnote{ ಎಕ 7 ನಾಮಂ 61 ಲಾಳನಕೆರೆ 1138} ಎಂದು ಈ ಮರವನ್ನು ಉಲ್ಲೇಖಿಸಲಾಗಿದೆ.

ತೆಂಗಿನ ಮರದ, ತೆಂಗಿನ ತೋಟಗಳ ಉಲ್ಲೇಖ ಅನೇಕ ಶಾಸನಗಳಲ್ಲಿದೆ. ಈಗಲೂ ಮಂಡ್ಯ ಜಿಲ್ಲೆಯಲ್ಲಿ ತೆಂಗಿನ ತೋಟಗಳ ಸಂಖ್ಯೆ ವಿಪರೀತವಾಗಿದೆ. ತೆಂಗಿನಕಟ್ಟ\index{ತೆಂಗಿನಕಟ್ಟ} (ಇಂದಿನ ತೆಂಗಿನಘಟ್ಟ) ಎಂಬ ಊರಿನ ಉಲ್ಲೇಖ ಶಾಸನದಲ್ಲಿದ್ದು ಇಲ್ಲಿ ಬಹಳವಾಗಿ ತೆಂಗಿನ ಮರಗಳನ್ನು ಬೆಳೆಸುತ್ತಿದ್ದರೆಂದು ತೋರುತ್ತದೆ.\endnote{ ಎಕ 6 ಕೃಪೇ 42 ತೆಂಗಿನಘಟ್ಟ 1117

ಎಕ 6 ಕೃಪೇ 38 ಗೋವಿಂದನಹಳ್ಳಿ 1236} “ನೀರಗುಂಡಿ ರಾಮಚಂದ್ರ ದೇವರ ಸನ್ನಿಧಿ”,\endnote{ ಎಕ 6 ಕೃಪೇ 38 ಕಿಕ್ಕೇರಿ 16ನೇ ಶ.} ಎಂಬ ಉಲ್ಲೇಖವಿದ್ದು ನೀರುಗುಂಡಿ ಎಂಬುದು ಒಂದು ರೀತಿಯ ಮರ. ಇದನ್ನು ನೀರಂಜಿ ಮರ ಎನ್ನುತ್ತಾರೆ. ನೇರಳೆ ಮರಗಳ ಉಲ್ಲೇಖ ಜಿಲ್ಲೆಯ ಶಾಸನಗಳಲ್ಲಿ ಅನೇಕ ಕಡೆ ಬಂದಿದೆ. “ಕಿರುಕೆರೆಯೊಳಗಣ ನೇಱಿಲು”, “ನೇಱಿಲ ಕೆರೆಯ ಗದ್ದೆ”,\endnote{ ಎಕ 6 ಕೃಪೇ 62 ಹುಬ್ಬನಹಳ್ಳಿ 1140}\break “ನೇರಲ ತಾಳಕಟ್ಟೆ”,\endnote{ ಎಕ 7 ಮ 64 ಹೊನ್ನಲಗೆರೆ 1623} “ನೇರಲಕೆರೆ\index{ನೇರಲಕೆರೆ} ಯೆಲ್ಲೆಗೆ ಮೂಡಲು”,\endnote{ ಎಕ 6 ಶ‍್ರೀಪ 216 ಶ‍್ರೀರಂಗಪಟ್ಟಣ 1725} ಈ ಉಲ್ಲೇಖಗಳಲ್ಲಿ ನೇರಳೆಯ ಮರಗಳು ಬೆಳೆದಿದ್ದ ಕಟ್ಟೆ, ಕೆರೆಗಳ ಉಲ್ಲೇಖವಿದೆ. ಹುಬ್ಬನಹಳ್ಳಿಯ ನೇರಲಕೆರೆಯ ಬಳಿ ಇಂದಿಗೂ ನೇರಳೆ ಮರಗಳು\index{ನೇರಳೆ ಮರಗಳು} ಬೆಳೆದಿರುವುದನ್ನು ನೋಡಬಹುದು. “ಮೂಡಾಯ\-ದೊಳಗೆರೆಯ ಪಾದರಿ”\endnote{ ಎಕ 7 ಮಂ 14 ಹುಳ್ಳೇನಹಳ್ಳಿ 8ನೇ ಶ.} ಎಂಬಲ್ಲಿ ಪಾದರಿ ಮರದ ಉಲ್ಲೇಖವಿದೆ. ಬನ್ನಿಮರವು ಪವಿತ್ರ ವೃಕ್ಷ. ಇದರ ಉಲ್ಲೇಖ ಅನೇಕ ಶಾಸನಗಳಲ್ಲಿದೆ. ಸಂಸ್ಕೃತ ಶಾಸನದಲ್ಲಿ ಇದನ್ನು ಶಮೀವೃಕ್ಷವೆಂದು\index{ಶಮೀವೃಕ್ಷ} ಹೇಳಿದೆ. “ಬನ್ನಿರ್ಗ್ಗಾಲ ಕುಪ್ಪೆ”,\endnote{ ಎಕ 7 ಮಂ 14 ಹುಳ್ಳೇನಹಳ್ಳಿ 8ನೇ ಶ.} “ಬನ್ನೀಮರನನ್ನು ವೊಪಚಿ ದೂರವಾಗಿ ಬಲದಲ್ಲಿ ಹಾಯಿಕಿಕೊಂಡು”,\endnote{ ಎಕ 6 ಶ‍್ರೀಪ 93 ನೆಲಮನೆ 1458} ”ಬಂನೀಮರದ\index{ಬಂನೀಮರ} ತಾಳೊತ್ತಿನಲ್ಲಿ ನಟ್ಟಕಲ್ಲು”,\endnote{ ಎಕ 7 ಮಂ 7 ಮಂಡ್ಯ 1516} “ದಿಶೈಶಾನ್ಯಾಂ ಶಮೀ\-ವೃಕ್ಷ ಸಮೀಪೆ”,\endnote{ ಎಕ 6 ಕೃಪೇ 71 ಕೈಗೋನಹಳ್ಳಿ 1462} ಇವೆಲ್ಲಾ ಬನ್ನೀಮರದ ಉಲ್ಲೇಖಗಳು. ಬಸರಿ ಮರವು ಆಲದ ಜಾತಿಗೆ ಸೇರಿದ ಮರ. ಅನೇಕ ಕುಟುಂಬಗಳ ವೃಕ್ಷ. ಈಚೆಗೆ ಈ ಮರ ಕಡಿಮೆಯಾಗುತ್ತಿದೆ. ರಸ್ತೆಗಳ ಬದಿ ಇದನ್ನು ಬೆಳೆಸುತ್ತಾರೆ. “ಹೊಸಹೊಳಲ ಮಾರ್ಗ್ಗದ ಬಸರಿಮರದ ಕೆಳಗೆ”,\endnote{ ಎಕ 6 ಪಾಂಪು 99 ತೊಣ್ಣೂರು 1722} ಎಂಬಲ್ಲಿ ದಾರಿಯುದ್ದಕ್ಕೂ ಬಸರಿಮರ ಬೆಳೆದಿದ್ದರ ಉಲ್ಲೇಖವಿದೆ.

ಬಿದಿರು ಮತ್ತು ಬಿದಿರುಮೆಳೆಗಳ ಉಲ್ಲೇಖ ಜಿಲ್ಲೆಯ ಶಾಸನಗಳಲ್ಲಿ ಕಂಡುಬರುತ್ತದೆ. “ಹೆಬ್ಬಿದರ ಮಡೆಯ\index{ಹೆಬ್ಬಿದರ ಮಡೆ} ತಿಬ್ಬನಹಳ್ಳಿ”,\endnote{ ಎಕ 7 ನಾಮಂ 160 ದೇವಲಾಪುರ 1513} ಎಂಬ ಉಲ್ಲೇಖದಲ್ಲಿ ದೊಡ್ಡ ಬಿದಿರು ಮೆಳೆಗಳ ನಡುವೆ ಇದ್ದ ತಿಬ್ಬನಹಳ್ಳಿ ಎಂಬ ಅರ್ಥ ಹೊರಡುತ್ತದೆ. “ಉತ್ತರ ವಬ್ಬೆ ಬಿದಿರಿನ ಮೆಳೆ”,\index{ಬಿದಿರಿನ ಮೆಳೆ}\endnote{ ಎಕ 6 ಪಾಂಪು 99 ತೊಣ್ಣೂರು 1722} ಎಂಬ ಉಲ್ಲೇಖವಿದೆ. ಕಳಲೆ ಎಂದರೂ ಬಿದಿರು. ಹಿರಿಕಳಲೆ, ಚಿಕ್ಕಕಳಲೆ ಎಂಬ ಊರುಗಳೂ ಮಂಡ್ಯ ಜಿಲ್ಲೆಯಲ್ಲಿವೆ. ಬೇಲದ ಕೆರೆ,\endnote{ ಎಕ 6 ಕೃಪೇ 8 ಹೊಸಹೊಳಲು 1306} ಎಂಬುದು ಬೇಲದ ಮರಗಳಿಂದ ಆವೃತವಾಗಿದ್ದ ಅಥವಾ ಏರಿಯ ಮೇಲೆ, ಏರಿಯ ಹಿಂದೆ ಬೇಲದ ಮರಗಳು ಬೆಳೆದಿದ್ದ ಕೆರೆ ಎಂದು ಹೇಳಬಹುದು. ಬೇಲೆಕೆರೆ\index{ಬೇಲೆಕೆರೆ} (ಬ್ಯಾಲದಕೆರೆ) ಎಂಬ ಗ್ರಾಮವು ಶಾಸನೋಕ್ತವಾಗಿದೆ.\endnote{ ಎಕ 6 ಕೃಪೇ 99 ಬ್ಯಾಲದಕೆರೆ 1532} “ಬೇಲದ ತಾಳಗುಡ್ಡದ\index{ಬೇಲದ ತಾಳಗುಡ್ಡ} ಕಲ್ಲುಗುಡ್ಡ”,\endnote{ ಎಕ 7 ಮ 7 ಮಂಡ್ಯ 1516} ಎಂಬುದು ಬೇಲದ ಮರಗಳು\index{ಬೇಲದ ಮರಗಳು} ಬೆಳೆದಿದ್ದ ಗುಡ್ಡವನ್ನು ಸೂಚಿಸುತ್ತದೆ. “ಬೇವಿನ ಮರದ\index{ಬೇವಿನ ಮರ} ಕಟ್ಟೆ”,\endnote{ ಎಕ 6 ಪಾಂಪು 99 ತೊಣ್ಣೂರು 1722} ಎಂಬುದು ಬೇವಿನಮರಗಳ ಬಳಿ ಇದ್ದ ಕಟ್ಟೆಯನ್ನು ಸೂಚಿಸುತ್ತದೆ. ಮತ್ತಿ\-ಮರವನ್ನು ಸಂಸ್ಕೃತದಲ್ಲಿ ಅರ್ಜುನ ವೃಕ್ಷ ಎಂದು ಕರೆಯಲಾಗುತ್ತದೆ. “ಪಿರಿಯಕೆರೆಯ ಕೆಳಗೆ ಮೞ ಕಾಲಙ್ಗಳೊಳಿರ್ಕ್ಕಣ್ಡುಗ ಮಣ್ನ ಕೊಟ್ಟರಾ”,\endnote{ ಎಕ 7 ಮ 42 ಆತಕೂರು 949} ಎಂಬುದು ಉದ್ದಕ್ಕೂ ಮತ್ತಿ ಮರಗಳು\index{ಮತ್ತಿ ಮರಗಳು} ಬೆಳೆದಿದ್ದ ಕಾಲುವೆಯನ್ನು ಸೂಚಿಸುತ್ತದೆ. “ಮತ್ತಿಯಕೆರೆಯ ಬಡಗಣಕೋಡಿ”,\endnote{ ಎಕ 7 ನಾಮಂ 169 ಕಸಲಗೆರೆ 1142} ‘ಮತ್ತಿಯಕೆರೆ’,\endnote{ ಎಕ 7 ನಾಮಂ 61 ಲಾಳನಕೆರೆ 1138} ಉಲ್ಲೇಖ ಶಾಸನಗಳಲ್ಲಿದೆ. ಮತ್ತಿಕೆರೆ ಎಂಬ ಊರುಗಳೂ ಇವೆ. “ಮಾವಿನಕೆರೆ”ಯ\endnote{ ಎಕ 7 ನಾಮಂ 64 ಯಲ್ಲಾದಹಳ್ಳಿ 1145} ಉಲ್ಲೇಖ ಯಲ್ಲಾದಹಳ್ಳಿ ಶಾಸನದಲ್ಲಿ, “ಮಾವಿನಕೆರೆ”\index{ಮಾವಿನಕೆರೆ} ಗ್ರಾಮದ ಉಲ್ಲೇಖ ಬಸ್ತಿ ಶಾಸನದಲ್ಲಿದೆ.\endnote{ ಎಕ 6 ಕೃಪೇ 107 ಬಸ್ತಿ 12ನೇ ಶ.} ಈ ಕೆರೆ ಹಾಗೂ ಊರುಗಳು ಮಾವಿನಮರಗಳಿಂದ ಕೂಡಿದ್ದವೆಂದು ಊಹಿಸಬಹುದು. 900 ಮಾವಿನ ಮರಗಳಿದ್ದ ಮಾವಿನ ಬನದ\index{ಮಾವಿನ ಬನ} ಉಲ್ಲೇಖ ತೊಣ್ಣೂರು ಶಾಸನದಲ್ಲಿದೆ.\endnote{ ಎಕ 6 ಪಾಂಪು ತೊಣ್ಣೂರು} “ಚೂತವೃಕ್ಷ,\index{ಚೂತವೃಕ್ಷ}\endnote{ ಎಕ 6 ಕೃಪೇ 71 ಕೈಗೋನಹಳ್ಳಿ 1462} ಊರಿಗೆ ಮೂಡಲಾಗಿ ಮಾವಿನಮರದ ಹೊಲ,\index{ಮಾವಿನಮರದ ಹೊಲ}\endnote{ ಎಕ 6 ಶ‍್ರೀಪ 24 ಶ‍್ರೀಪ 1686} “ನೀಲಕಂಠನಹಳ್ಳಿಯ ಮಧ್ಯದ\break ಮಾವಿನಮರದಿಂ”,\endnote{ ಎಕ 7 ಮ 64 ಹೊನ್ನಲಗೆರೆ 1623} “ಮೂಡ ಮಾವಿನಕೆರೆಯ ದಾರಿಯಿಂದ”,\endnote{ ಎಕ 7 ನಾಮಂ 64 ಯಲ್ಲಾದಹಳ್ಳಿ 1145} ಎಂಬ ಉಲ್ಲೇಖಗಳು ಅಲ್ಲಿ ಬೆಳೆದಿದ್ದ ಮಾವಿನ ಮರ\-ಗಳನ್ನು ಸೂಚಿಸುತ್ತವೆ. ಜಿಲ್ಲೆಯ ಬಾರೆಯಂತಹ ಒಣ ಪ್ರದೇಶಗಳಲ್ಲಿ ಮುತ್ತುಗದ ಮರದ ಕಾಡೇ ಇದೆ. ಇದನ್ನು ಸಂಸ್ಕೃತ ಶಾಸನದಲ್ಲಿ ಕಿಂಶುಕ ವೃಕ್ಷವೆಂದು ಸೂಚಿಸಿದೆ.\endnote{ ಎಕ 6 ಕೃಪೇ 71 ಕೈಗೋನಹಳ್ಳಿ 1462}

ಹುಣಸೆ\index{ಹುಣಸೆ} ಮರದ ಉಲ್ಲೇಖ ಅನೇಕ ಶಾಸನಗಳಲ್ಲಿದೆ. “ಪಶ್ಚಿಮೋತ್ತರಸ್ಯಾನ್ದಿಶಿ ಪುಣುಸೆಯ ಗೊಟ್ಟೆಗಾಲ”,\endnote{ ಎಕ 7 ನಾಮಂ 149 ದೇವರಹಳ್ಳಿ 776}\break “ಪಡುವಾಯ್ನೋಡಿ ಪೆರ್ವುಣಸೆ”,\index{ಪೆರ್ವುಣಸೆ}\endnote{ ಎಕ 7 ಮಂ 14 ಹುಳ್ಳೇನಹಳ್ಳಿ 8ನೇ ಶ.} “ಚಂಚರಿವಳ್ಳದ ಪಡುವಣ ಹುಣಸೆ”, ಇವು ಹುಣಸೆಯ ಮರದ ಪ್ರಾಚೀನ ಉಲ್ಲೇಖಗಳು. “ಗದ್ದೆಯ ಬಡಗಣ ಹಲವು ಹುಣಿಸೆಯ ನಟ್ಟಕಲ್ಲು”,\endnote{ ಎಕ 7 ನಾಮಂ 169 ಕಸಲಗೆರೆ 1142} “ಹರಿದ ಹಳ್ಳದ ತಕ್ಕರಹುಣಿಸೆ”,\endnote{ ಎಕ 7 ನಾಮಂ 61 ಲಾಳನಕೆರೆ 1138} “ಅಲ್ಲಿಂ ಮೂಡಣ ಹುಣಿಸೆ”,\endnote{ ಎಕ 7 ನಾಮಂ 72 ಅಳೀಸಂದ್ರ 1183} “ಕಲ್ಲುಸರಡಿನ ಎಣೆಹುಣಿಸೆಗಳಿಂ”,\endnote{ ಎಕ 7 ನಾಮಂ 74 ಬೆಳ್ಳೂರು 1271} ಎಂಬ ಉಲ್ಲೇಖಗಳಿವೆ. ಕೃಷ್ಣರಾಜಪೇಟೆ ತಾಲ್ಲೂಕಿನ ಶಾಸನದಲ್ಲಿ ಪುಣಿಸಮಯ್ಯನ ಹೆಸರಿದೆ. ಅಗ್ರಹಾರಬಾಚಹಳ್ಳಿಯ ಈಶ್ವರನನ್ನು ಹುಣಸೇಶ್ವರ\index{ಹುಣಸೇಶ್ವರ ದೇವಾಲಯ} ಎಂದು ಕರೆಯುತ್ತಾರೆ. ಬೆಂಗಳೂರಿನ ಬಳಿ ‘ತರಹುಣಿಸೆ’ ಎಂಬ ಹಳ್ಳಿ ಇದ್ದು, ತಕ್ಕರಹುಣಸೆಗೂ, ತರಹುಣಸೆಗೂ ಸ್ಥಳನಾಮದ ಸಂಬಂಧ ಇರುವಂತೆ ತೋರುತ್ತದೆ. ಕಿಕ್ಕೇರಿ ಶಾಸನದಲ್ಲಿ “ವೂರಿಂ ತೆಂಕಣ ಬೀರಭದ್ರನ ಹುಣಿಸೆ ಹಿರಿಯಕೆರೆಯ ಕೆಳಗಣ ಗದ್ದೆ”,\endnote{ ಎಕ 6 ಕೃಪೇ 27 ಕಿಕ್ಕೇರಿ 1171} ಎಂದು ಹೇಳಿದ್ದು, ಈಗಲೂ ಕಿಕ್ಕೇರಿಯ ದೊಡ್ಡ ಕೆರೆಯ ಉತ್ತರ ಭಾಗ ಒಂದು ಕಡೆ ಮತ್ತು ಕೆರೆಯ ಏರಿಯ ಮೇಲೆ ದೊಡ್ಡ ದೊಡ್ಡ ಹುಣಸೆ ಮರಗಳಿರುವುದನ್ನು ನೋಡಬಹುದು. ಯಲವದ ಮರದ ಉಲ್ಲೇಖ ಸುಜ್ಜಲೂರು ಶಾಸನದಲ್ಲಿದೆ. ಇದು ಯಾವ ಮರ ಎಂಬುದು ತಿಳಿದುಬರುವುದಿಲ್ಲ.\endnote{ ಎಕ 7 ಮವ 139 ಸುಜ್ಜಲೂರು 1473} ಲಕ್ಕಿ ಪತ್ರೆಯು ಬಹಳ ಪವಿತ್ರವಾದುದು. ಶೈವರ ಕೈಯಲ್ಲಿರುತ್ತಿದ್ದ ಲಗುಡ ಅಥವಾ ಲಾಕುಳ ದಂಡವು ಲಕ್ಕಿ ಮರದ್ದೇ ಆಗಿದೆ. ಹಳ್ಳಿಗಳ ಕಡೆ ಲಕ್ಕಿ\break ಮರವನ್ನು ಲಕುಳೀ ಮರವೆಂದೂ ಕರೆಯುತ್ತಾರೆ. “ಲಕ್ಕೀ ಕಟ್ಟೆ”\index{ಲಕ್ಕೀ ಕಟ್ಟೆ}\endnote{ ಎಕ 6 ಪಾಂಪು 99 ತೊಣ್ಣೂರು 1722

ಎಕ 7 ಮವ 105 ತಿಗಡಹಳ್ಳಿ 1337

ಎಕ 7 ನಾಮಂ 169 ಕಸಲಗೆರೆ 1142

ಎಕ 7 ನಾಮಂ 1 ನಾಗಮಂಗಲ1173

ಎಕ 6 ಶ‍್ರೀಪ 93 ನೆಲಮನೆ 1458} ಉಲ್ಲೇಖ ಶಾಸನದಲ್ಲಿದೆ. ಸಾಮಾನ್ಯವಾಗಿ ಕೆರೆಗಳ ಬಳಿ ಈ ಮರಗಳು ಹೆಚ್ಚಾಗಿ ಬೆಳೆಯುತ್ತಿದ್ದವು. ಇವಲ್ಲದೆ, “ಕಗ್ಗಲಿ”,\index{ಕಗ್ಗಲಿ}\endnote{ ಎಕ 7 ಮವ 105 ತಿಗಡಹಳ್ಳಿ 1337} ಕೇದಗೆಯ ಕೆರೆ”,\index{ಕೇದಗೆಯ ಕೆರೆ}\endnote{ ಎಕ 7 ನಾಮಂ 169 ಕಸಲಗೆರೆ 1142} ‘ಅರಿಕನಕಟ್ಟದ ಕಣಿಗಿಲೆಯ ಕಟ್ಟೆ’,\endnote{ ಎಕ 7 ನಾಮಂ 1 ನಾಗಮಂಗಲ1173}\break ‘ಕಿರುಗಿಡವೊಬ್ಬೆ’, ‘ಕಳ್ಳಿವಬ್ಬೆ ಪೊದೆ ಬೆಳೆದಿದ್ದ’\endnote{ ಎಕ 6 ಶ‍್ರೀಪ 93 ನೆಲಮನೆ 1458} ಉಲ್ಲೇಖಗಳು ಶಾಸನದಲ್ಲಿವೆ. ತಗಚೆ\index{ತಗಚಗೆರೆ} ಗಿಡಗಳು ಬೆಳೆದಿದ್ದ ತಗಚೆಗೆರೆ ಉಲ್ಲೇಖ ಬೆಳ್ಳೂರು ಶಾಸನದಲ್ಲಿದೆ. ತಗಚೆಸೊಪ್ಪನ್ನು ಚರ್ಮದ ಖಾಯಿಲೆಗೆ ಉಪಯೋಗಿಸುತ್ತಾರೆ. ಈಗಲೂ ತಗಚೆ ಗಿಡಗಳು ಬೆಳೆದಿರುವ ಅನೇಕ ಕಟ್ಟೆಗಳು ಜಿಲ್ಲೆಯಲ್ಲಿ ಕಂಡುಬರುತ್ತವೆ.

\begin{center}
***
\end{center}

\theendnotes

