
\chapter{ಶಾಸನ ಕವಿಗಳು ಮತ್ತು ಶಾಸನ ಸಾಹಿತ್ಯ}

ಜಿಲ್ಲೆಯಲ್ಲಿರುವ ಶಾಸನಗಳನ್ನು ಹೆಚ್ಚಾಗಿ ಆಸ್ಥಾನ ಕವಿಗಳು, ಶಾಸನ ಬರಹವನ್ನೇ ವೃತ್ತಿಯಾಗುಳ್ಳ ಅಧಿಕಾರಿಗಳು (ಕರಣಿಕರು) ಮತ್ತು ಸ್ಥಳದ ಸೇನಬೋವರು ಬರೆದಿರುವುದು ಕಂಡುಬರುತ್ತದೆ. ಕವಿಗಳು ರಚಿಸುತ್ತಿದ್ದ ಶಾಸನಗಳಲ್ಲಿ ಆ ಊರಿನ ಪ್ರಾಕೃತಿಕ ವರ್ಣನೆ, ರಾಜನ, ಮಾಂಡಲೀಕರ, ಸಾಮಂತರ, ಅಧಿಕಾರಿಗಳ ಹಾಗೂ ಇತರ ಮಹಾದಾನಿಗಳ ವಂಶಾವಳಿ ಮತ್ತು ಬಿರುದಾವಳಿಗಳ ಸಮೇತ ಅವರ ಸಾಧನೆಗಳ ವರ್ಣನೆಯನ್ನು ಪ್ರಮುಖವಾಗಿ ಹಾಗೂ ಕಾವ್ಯಮಯವಾಗಿ ಮಾಡಿದ ನಂತರ, ದತ್ತಿಯ ಭಾಗವನ್ನು ಸಾಮಾನ್ಯ ರೀತಿಯಲ್ಲಿ ಹೆಚ್ಚು ವಿವರಗಳಿಲ್ಲದೇ ವರ್ಣಿಸಿರುವುದು ಕಂಡು ಬರುತ್ತದೆ. ಇಂತಹ ಕೆಲವು ಶಾಸನಗಳಲ್ಲಿ ಮಾತ್ರ ಕವಿಗಳು ತಮ್ಮ ಹೆಸರನ್ನು ಉಲ್ಲೇಖಿಸಿದ್ದಾರೆ, ಆದರೂ ಕವಿ ಲಿಖಿತವೆಂದು ಹೇಳಬಹುದಾದ ಅನೇಕ ಶಾಸನಗಳಲ್ಲಿ ಕವಿಗಳ ಹೆಸರು ಕಂಡುಬರುವುದಿಲ್ಲ. ಶಾಸನವನ್ನು ಕಲ್ಲ ಮೇಲೆ ಬರೆದವರು ಬರಹಗಾರರು, ಕೆತ್ತಿದವರು ಕಂಡರಣೆಕಾರರು. ಸ್ಪಷ್ಟವಾಗಿ ನಮೂದಿಸದ ಹೊರತು ಇದು ತಿಳಿದುಬರುವುದಿಲ್ಲ.

ಶಾಸನಗಳ ರಚನೆಯನ್ನೇ ವಂಶಪಾರಂಪರ್ಯ ವೃತ್ತಿಯನ್ನಾಗಿಸಿಕೊಂಡಿದ್ದ ಅಧಿಕಾರಿಗಳು (ಶಾಸನ ರಚನೆಕಾರರು) ಕವಿಗಳ ರೀತಿಯಲ್ಲಿ ವರ್ಣನೆಗಳನ್ನು ಮಾಡಿರುವುದು ಕಂಡು ಬರುವುದಿಲ್ಲ. ಆದರೆ ಎಷ್ಟುಬೇಕೋ ಅಷ್ಟರ ಮಟ್ಟಿಗೆ ರಾಜರು, ಅಧಿಕಾರಿಗಳು, ದಾನಿಗಳ ವರ್ಣನೆಯನ್ನು ಮಾಡಿ ದತ್ತಿಯ ಭಾಗವನ್ನು ಬರೆಯುವಾಗ, ಅದರ ಎಲ್ಲೆಗಳನ್ನು ಗುರುತಿಸುವಾಗ ವಿಶೇಷ ಆಸಕ್ತಿ ವಹಿಸಿ ವಿವರಗಳನ್ನು ನೀಡಿರುವುದು ಕಂಡು ಬರುತ್ತದೆ. ಊರಿನ ಭೌಗೋಳಿಕ ಇತಿಹಾಸದ ದೃಷ್ಟಿಯಿಂದ ಇದು ಮುಖ್ಯವಾಗುತ್ತದೆ. ಇಂತಹ ಶಾಸನಗಳನ್ನು ರಾಜಾಸ್ಥಾನದಲ್ಲಿ ವಂಶಪಾರಂಪರ್ಯವಾಗಿದ್ದ ಶಾಸನ ಲೇಖಕರು, ಸ್ಥಳದ ಸೇನಬೋವರು ಮತ್ತು ಸ್ಥಳೀಯ ಬರಹಗಾರರ ಹೆಸರುಗಳು ಕಂಡು\-ಬರುತ್ತವೆ. ಸ್ಥಳೀಯ ಬರಹಗಾರರು ರಚಿಸಿದ ಶಾಸನಗಳು ಬಹಳ ಚಿಕ್ಕದಾಗಿದ್ದು ವ್ಯಾವಹಾರಿಕ ಸ್ವರೂಪದ್ದಾಗಿವೆ.

\section{ಶಾಸನ ಕವಿಗಳು}

ಜಿಲ್ಲೆಯ ಅನೇಕ ಶಾಸನಗಳನ್ನು ಕವಿಗಳು ಬರೆದಿದ್ದಾರೆ. ಅವರಲ್ಲಿ ಬಿಂಡಯ್ಯ, ಚಿದಾನಂದ ಮಲ್ಲಿಕಾರ್ಜುನ, ಸಾಂತಮಹಂತ, ಶಾಂತಿನಾಥ, ನೃಸಿಂಹಸೂರಿ ಮೊದಲಾದವರು ಖ್ಯಾತ ಕವಿಗಳಾಗಿದ್ದಂತೆ ತೋರುತ್ತದೆ.\break ಚಿದಾನಂದ ಮಲ್ಲಿಕಾರ್ಜುನನ “ಸೂಕ್ತಿ ಸುಧಾರ್ಣವ”ವನ್ನು ಬಿಟ್ಟರೆ ಬೇರೆಯವರ ಯಾವುದೇ ಕೃತಿಗಳೂ ದೊರಕಿರುವುದಿಲ್ಲ.


\section{ಬಿಂಡಯ್ಯ (10ನೇ ಶ.)}

ಗಂಗರ ಕಾಲದ ಎಲೆಕೊಪ್ಪ ಬಂಡೆ ಶಾಸನವನ್ನು ‘ಶ‍್ರೀಮದೇಳಾಚಾರ್ಯ್ಯರ ವರಗುಡ್ಡ ಬಿಣ್ಡಯ್ಯ ಬರೆದಂ, ಮಂಗಳಂ’ ಎಂದು ಹೇಳಿದೆ.\endnote{ ಎಕ 7 ನಾಮಂ 122 ಎಲೆಕೊಪ್ಪ 10-11ನೇ ಶ.} ಬಿಂಡಯ್ಯನು ಕವಿಯಾಗಿದ್ದನೆಂದು ಹೇಳಿದೆ.\endnote{ ಮೇವುಂಡಿ ಮಲ್ಲಾರಿ, ಕನ್ನಡ ನಾಡಿನ ಶಾಸನಕವಿಗಳು, ಪುಟ 142} ಈತನು ರಚಿಸಿರುವ ಮೂರು ಕಂದ ಪದ್ಯಗಳು ಉತ್ತಮ ಕವಿತಾ ಗುಣವನ್ನು ಹೊಂದಿವೆ. ಮೇವುಂಡಿಯವರು ಇವನ ಹೆಸರನ್ನು ‘ಬಿಂದಯ್ಯ’ ಎಂದು ಹೇಳಿದ್ದಾರೆ, ಆದರೆ ಶಾಸನದಲ್ಲಿ ಬಿಂಡಯ್ಯ ಎಂಬ ಹೆಸರೇ ಇದ್ದು, ಇದು ಜೈನಕವಿ ಆಂಡಯ್ಯನೆಂಬ ನೆನಪನ್ನು ತರುತ್ತದೆ. ಶ್ರವಣಬೆಳಗೊಳದ ಬಳಿ ಬಿಂಡೇನಹಳ್ಳಿ ಎಂಬ ಊರೂ ಇದೆ. ನರಸಿಂಹಮೂರ್ತಿಯವರು ಬಿಂಡಯ್ಯನನ್ನು ಕವಿ ಎಂದು ಹೇಳದೆ ಕೇವಲ ಶಾಸನ ಲೇಖಕನೆಂದು ಹೇಳಿದ್ದಾರೆ.\endnote{ ನರಸಿಂಹಮೂರ್ತಿ, ಡಾ॥ ಪಿ.ಎನ್​. ಮಂಡ್ಯ ಜಿಲ್ಲೆಯ ಶಾಸನಕವಿಗಳು ಮತ್ತು ಓಜರು, ಪುಟ 87} ಜೈನಕವಿಗಳು ಸಾಮಾನ್ಯವಾಗಿ ಜೈನಯತಿಗಳ ಶಿಷ್ಯರಾಗಿಯೇ ಇರುತ್ತಿದ್ದರು. ಗುಹಾವಾಸಿಗಳಾದ ಗೊಹೆಯ ಭಟ್ಟಾರಕನನ್ನು ಸ್ತುತಿಸಿರುವ ಈ ಶಾಸನದ ಪದ್ಯಗಳು ಉಲ್ಲೇಖಾರ್ಹ. ಎಲೆಕೊಪ್ಪಕ್ಕೆ ಹೋಗಿ ಹುಡುಕಿದರೂ ಈ ಶಾಸನ ಸಿಗಲಿಲ್ಲ.

\begin{verse}
\textbf{ತಳವಳಗನಾಗಿ ಪುಲಿಗಳ} \\\textbf{ಗಳಗರ್ಜ್ಜನೆಗಾದಮಳ್ಕಿ ಗುಹೆಯಿಂದಮಣಂ} \\\textbf{ತಳರದೆ ನೆಲೆಸಿಪ್ಪ ಮನೋ} \\\textbf{ಬಳಮುರ್ಬ್ಬಿಗಗುರ್ವ್ವು ಗೊಹೆಯ ಭಟ್ಟಾರಕನ್​}
\end{verse}

\begin{verse}
\textbf{ಪಕ್ಷೋಪವಾಸಿಗಳಳ್ಪಾ} \\\textbf{ಪಕ್ಷಯಕರ ಮೂರ್ತ್ತಿಗಳ್ಗುಹಾವಾಸಿಗಳ} \\\textbf{ತ್ಯಕ್ಷುಂಣಚರಿತರಿವರೆ} \\\textbf{ನ್ದೀಕ್ಷಿತಿ ಪೊಗಳ್ದಪುದು ಗೊಹೆಯ ಭಟ್ಟಾರಕನ್​}
\end{verse}

\begin{verse}
\textbf{ಪಡೆದುದು ನೊಳಂಬವಾಡಿಯು} \\\textbf{ಮೊಡನೊಡನೀ ದಡಿಗವಾಡಿಯುಂ ಧರ್ಮ್ಮಮಮನೋ} \\\textbf{ಗೆಡಿಸದ ತಪದಿಂ ಭುವನಮ} \\\textbf{ನಡಿಗೆರಗಿಸಿ ನೆಗಳ್ದ ಗೊಹೆಯ ಭಟ್ಟಾರಕನ್​ }
\end{verse}

ಹುಲಿಗಳಿದ್ದ ಗುಹೆಯಲ್ಲಿ ತಪಸ್ಸನ್ನು ಮಾಡುತ್ತಾ, ಹುಲಿಗಳ ಘರ್ಜನೆಗೆ ಸ್ವಲ್ಪವೂ ಅಲುಗಾಡದ, ಪಕ್ಷೋಪವಾಸಿ, ಪಾಪಕ್ಷಯ ಮೂರ್ತಿ, ನೊಳಂಬವಾಡಿ, ದಡಿಗವಾಡಿಯನ್ನು ಪಡೆದರೂ, ಆ ರಾಜ್ಯಗಳ ರಾಜರು ಇವನ ಶಿಷ್ಯರಾಗಿದ್ದರೂ, ಧರ್ಮವನ್ನು ಮನಗೆಡಿಸದೆ, ತನ್ನ ತಪಸ್ಸಿನಿಂದ ಇಡೀ ಭುವನವೇ ತನ್ನ ಅಡಿಗೆರಗುವಂತೆ ಮಾಡಿದ ಗೊಹೆಯ ಭಟ್ಟಾರಕನ ಸ್ತುತಿ ಆರ್ದ್ರತೆಯಿಂದ ಕೂಡಿದ್ದು, ಮನದಲ್ಲಿ ಭಕ್ತಿ ಭಾವನೆಯನ್ನು ಉಂಟು ಮಾಡುತ್ತವೆ.


\section{ಸಾಂತಮಹಂತ(1138)}

ಈತನು ನಾಗಮಂಗಲ ತಾಲ್ಲೂಕು ಲಾಳನಕೆರೆಯ ಶಾಸನವನ್ನು ಬರೆದಿರುತ್ತಾನೆ.\endnote{ ಎಕ 7 ನಾಮಂ 61 ಲಾಳನಕೆರೆ 1138

ಮೇವುಂಡಿ ಮಲ್ಲಾರಿ, ಕನ್ನಡನಾಡಿನ ಶಾಸನಕವಿಗಳು, ಪುಟ 121-22} ಈ ಶಾಸನದಲ್ಲಿ 13 ಕಂದಪದ್ಯಗಳೂ, 7 ವೃತ್ತಗಳೂ ಇವೆ. ಶಾಸನದ ಕೊನೆಯ ಕಂದಪದ್ಯದಲ್ಲಿ ತನ್ನ ವಿವರವನ್ನು ನೀಡಿರುವುದನ್ನು ಬಿಟ್ಟರೆ ಹೆಚ್ಚಿನ ವಿವರಗಳೇನಿಲ್ಲ. ಈತನು ಅರಸೀಕೆರೆ ತಾಲ್ಲೂಕು ಕೆಲ್ಲಂಗೆರೆಯ ಶಾಸನವನ್ನೂ ಬರೆದಿದ್ದಾನೆ.\endnote{ ಎಕ 10 ಅರಸೀಕೆರೆ 152 ಕೆಲ್ಲಂಗೆರೆ 1175} ಅದರಲ್ಲೂ ಈ ಕೆಳಗಿನ ಪದ್ಯ ಒಂದನ್ನು ಬಿಟ್ಟರೆ ಬೇರೆ ವಿವರಗಳಿಲ್ಲ.

\begin{verse}
\textbf{ಕ್ಷಿತಿವಿನುತೆ ಸರ್ಬ್ಬದೇವ} \\\textbf{ನುತೆ ಸಂನುತೆ ಸೋಮಿಯಕ್ಕನಣುಗಿನ ಪುತ್ತ್ರಂ} \\\textbf{ಸತುಕವಿ ಸಾಂತಮಹಂತಂ} \\\textbf{ಸ್ತಿತಿಸಾರಂ ಪೇಳ್ದನೞ್ತಿಯಿಂ ಸಾಶನಮಂ }
\end{verse}

ಸೋಮಿಯಕ್ಕ(ಸೋವಿಯಕ್ಕ)ನ ಪ್ರೀತಿಯ ಪುತ್ರನಾದ (ಕಿರಿಯಪುತ್ರ?) ಇವನು ತನ್ನ ತಾಯಿಯನ್ನು ಬಹಳವಾಗಿ ಹೊಗಳಿದ್ದಾನೆ. ಅವನು ರಚಿಸಿರುವ ಕೆಲವು ಉನ್ನತ ಮಟ್ಟದ ಕವಿತ್ವದದಿಂದ ಕೂಡಿರುವ ಕೆಲವು ಪದ್ಯಗಳನ್ನು ಗುರುತಿಸಬಹುದು. ಎರಡು ಪ್ರಾರ್ಥನಾ ಪದ್ಯಗಳಂತೂ ಶಾಸನ ಸಾಹಿತ್ಯದಲ್ಲಿ ಅತ್ಯುತ್ತಮ ರಚನೆಗಳೆನಿಸಿಕೊಂಡಿವೆ.


\section{ಶಿವನ ಪ್ರಾರ್ಥನೆ:}

\begin{verse}
\textbf{ಪತ್ತಿರೆ ಗಂಗೆಯ ಹನಿಗಳು} \\\textbf{ಮುತ್ತಿನ ಬಾಸಿಗಮಿದೆನಿಸಿ ವಾಸುಗಿ ಸುಖದಿಂ} \\\textbf{ಸುತ್ತಿರ್ದ್ದ ಜೂಟದಾ ದೇ} \\\textbf{ವೋತ್ತಮ ನಮಗೊಸೆದು ಕೊಡುಗೆ ಸುಖಸಂಪದಮಂ}
\end{verse}

\begin{verse}
\textbf{ಭೇದಂ ಮೂರ್ತಿಯೊಳಲ್ಲದೆ} \\\textbf{ಭೇದಂ ಪರಮಾರ್ತ್ಥ ತತ್ವದೊಳು ಸಲ್ಲದೆನಲು} \\\textbf{ಮೂದೇವರಾದಭೇದದ} \\\textbf{ಮಾದೇವಂ ದೇವನೀಗೆ ಸುಖಸಂಪದಮಂ}
\end{verse}


\section{ವಿಷ್ಣುವರ್ಧನ}

\begin{verse}
\textbf{ಕೊಂಡಂ ತಳಕಾಡಂ ಕೈ} \\\textbf{ಕೊಂಡಂ ಮೇಲೆತ್ತಿ ಕೊಂಗನವೆಯವದಿಂದಂ} \\\textbf{ಕೊಂಡಂ ವಿಷ್ಣುವೇ ಚೋಳನ} \\\textbf{ಮಂಡಳೀಕರ ಮುಡೆಗೊಂಡು ತನುಮಂಡಳಮಂ}
\end{verse}


\section{ಯೇಚಿರಾಜ ದಂಡಾಧೀಶ}

\begin{verse}
\textbf{ಅಂಡಲೆದರಿನರಪಾಳರ} \\\textbf{ಹಿಂಡಂ ಬೆಂಕೊಂಡು ಪರಮಂಡಳಮಂ} \\\textbf{ದಂಡೇಷನೇಚಿರಾಜಂ} \\\textbf{ಗಂಡರಗಂಡಂ ಧರಿತ್ರಿಯೊಳುಪೆಸರ್ವ್ವಡೆದಂ}
\end{verse}


\section{ಬಿಟ್ಟಿದೇವ ದಂಡಾಧೀಶ}

\begin{verse}
\textbf{ನುಡಿದ ನುಡಿ ತಾಂಬ್ರಶಾಸನ} \\\textbf{ಪಡೆದ ಧನಂ ಸದುಬುಧರ್ಗ್ಗಮಾತ್ಯರೊಳಧಿಕಂ} \\\textbf{ಪೊಡವಿಯೊಳೆ ತೋರ್ಪ್ಪಸುರತರು} \\\textbf{ಪಡೆ ಮಾತೇಂ ಬಿಟ್ಟಿದೇವ ದಂಡಾಧೀಶಂ}
\end{verse}


\section{ಮಂತ್ರಿ ಬೋಕಂಣ}

\begin{verse}
\textbf{ಬೀರಂ ಬಿಂಕಮದೇವುದೋ} \\\textbf{ಹಾರುವ ನಿನಗೆಂದು ನುಡಿವ ಗಳಹನ ಬಾಯೊಳು} \\\textbf{ಕೂರಲಗನೆಯ್ದೆ ಕುತ್ತುವ} \\\textbf{ವೀರಾಗ್ರಣಿ ಮಂತ್ರಿ ಬೋಕಣಂ ವಸುಮತಿಯೊಳು}
\end{verse}


\section{ಕವಿಕುಳ ತಿಳಕ ಶಾಂತಿನಾಥ (1165)}

ನಾಗಮಂಗಲ ತಾಲ್ಲೂಕು, ಲಾಳನಕೆರೆಯ ಮಧುಸೂದನ ದಂಡನಾಯಕನ 57 ಸಾಲುಗಳ ದೊಡ್ಡ ಶಾಸನವನ್ನು ಶಾಂತಿನಾಥನೆಂಬ ಕವಿ ರಚಿಸಿದ್ದಾನೆ.\endnote{ ಎಕ 7 ನಾಮಂ 63 ಲಾಳನಕೆರೆ 1165}

\begin{verse}
\textbf{ದಕ್ಷಿಣ ಹೆಂಮನ ಮಂಮಂ } \\\textbf{ಲಾಕ್ಷಣಗವಿ ಶಾಂತಿನಾಥ ಕವಿಕುಳತಿಳಕಂ} \\\textbf{ದಾಕ್ಷಿಂಣ್ಯನಿಧಿ ಗುಣೋತ್ಕರ } \\\textbf{ಸೀಕ್ಷಾಗುರು ಹೇಳ್ದನರ್ತಿಯಿಂ ಸಾಶನಮಂ}
\end{verse}

ಈತನು ದಕ್ಷಿಣದಲ್ಲಿ ಪ್ರಸಿದ್ಧನಾಗಿದ್ದ ಹೆಂಮನೆಂಬುವವನ ಮೊಮ್ಮಗನೆಂದೂ, ಲಾಕ್ಷಣ ಕವಿ ಎಂದರೆ ಅಲಂಕಾರಶಾಸ್ತ್ರ, ವ್ಯಾಕರಣ ಮೊದಲಾದವುಗಳಲ್ಲಿ ವಿದ್ವಾಂಸನಾದ ಕವಿ ಎಂದು, ಅನೇಕ ಕವಿಗಳಿಗೆ ಶಿಕ್ಷಾ ಗುರುವಾಗಿದ್ದನೆಂದೂ ತಿಳಿದುಬರುತ್ತದೆ. ಮೇವುಂಡಿ ಮಲ್ಲಾರಿಯವರು “ಈತನು ಅರಸೀಕೆರೆಯ 48ನೆಯ ಶಾಸನವನ್ನು ಬರೆದಿರುತ್ತಾನೆಂದೂ, ಈತನು ಬ್ರಾಹ್ಮಣ ಕವಿಯೆಂದು ತೋರುತ್ತ\-ದೆಂದೂ, ಶಾಂತಿನಾಥನು ಪ್ರೌಢ ಕವಿಯೆಂದೂ” ಹೇಳಿದ್ದಾರೆ. ಈ ಎರಡೂ ಶಾಸನಗಳಿಂದ ಕೆಲವು ಪದ್ಯಗಳನ್ನು ಕವಿಚರಿತೆಕಾರರು ಸಂಗ್ರಹಿಸಿಕೊಟ್ಟಿದ್ದಾರೆ.\endnote{ ಕರ್ನಾಟಕ ಕವಿಚರಿತೆ, ಪ್ರಥಮ ಸಂಪುಟ, ಪುಟ 250-252} ಮೇವುಂಡಿ ಮಲ್ಲಾರಿಯವರು “ಶಾಂತಿನಾಥನು ಹೊಯ್ಸಳ ರಾಜ್ಯದ ದಕ್ಷಿಣ ಭಾಗದವನಾಗಿದ್ದು, ಲಾಳನಕೆರೆಯವನೋ ನೆರೆಹೊರೆಯ ಪ್ರದೇಶದವನೋ ಆಗಿರಬೇಕೆಂದೂ, ಅರಸೀಕೆರೆ ತಾಲ್ಲೂಕು ಕಣಕಟ್ಟೆ ಶಾಸನವು ಕವಿ ಶಾಂತಿನಾಥನ ಅಜ್ಜನ ಹೆಸರಿನ ಬಗ್ಗೆ ವಾದವನ್ನೆಬ್ಬಿಸುತ್ತದೆಂದೂ, ಆದರೂ ಇವರಿಬ್ಬರೂ ಅಭಿನ್ನರೆಂದು ಹೇಳಿದ್ದಾರೆ, ಲಾಳನಕೆರೆ ಶಾಸನವನ್ನು ಬರೆದ 42 ವರ್ಷಗಳ ತರುವಾಯ ಕಣಕಟ್ಟೆ ಶಾಸನವನ್ನು ಇವನು ಬರೆದನೆಂದು ಹೇಳಬಹುದೆಂದೂ ಹೇಳಿದ್ದಾರೆ.\endnote{ ಮೇವುಂಡಿ ಮಲ್ಲಾರಿ, ಕನ್ನಡನಾಡಿನ ಶಾಸನಕವಿಗಳು, ಪುಟ 147-148}” ಕಣಕಟ್ಟೆ ಶಾಸನದ ಪದ್ಯ ಹೀಗಿದೆ.\endnote{ ಎಕ 10 ಅರಸೀಕೆರೆ 88 ಕಣಕಟ್ಟೆ, 1189}

\begin{verse}
\textbf{ದಕ್ಷಿಣ ಸೋಮನ ಮಮ್ಮಂ} \\\textbf{ಲೊಕ್ಕಣಕವಿ ಶಾಂತಿನಾಥ ಕವಿಕುಳತಿಳಕಂ} \\\textbf{ದಾಕ್ಷಿಂಣ್ಯನಿಧಿ ಗುಣಾಕರ} \\\textbf{ಸಿಕ್ಷಾಗುರು ಹೇಳಿದರ್ತಿಯಿಂ ಸಾಸನಮಂ}
\end{verse}

ಕಣಕಟ್ಟೆ ಶಾಸನದಲ್ಲಿ ಸುಮಾರು 12 ಕಂದಪದ್ಯಗಳೂ, 8 ವೃತ್ತಗಳೂ ಇದ್ದರೆ, ಲಾಳನಕೆರೆ ಶಾಸನದಲ್ಲಿ ಸುಮಾರು 11 ಕಂದ ಪದ್ಯಗಳೂ, 13 ವೃತ್ತಗಳೂ ಇವೆ. ಲಾಳನಕೆರೆ ಶಾಸನದ ಪದ್ಯಗಳು ಕಾವ್ಯಮಯವಾಗಿವೆ.

“ಸಂಧಿವಿಗ್ರಹಿ ಪಾರ್ಶ್ವನಾಥನ ತನಯ ಶಾಂತಿನಾಥ” ಎಂಬುವವನು ಚೆಬ್ರೋಲು ಶಾಸನವನ್ನು ಬರೆದಿರುತ್ತಾನೆಂದು ತಿಳಿದುಬರುತ್ತದೆ.\endnote{ ತಾರಾನಾಥ್​ ಡಾ॥ ಎನ್​.ಎಸ್​., ಶಾಂತಿನಾಥ, ಕನ್ನಡ ಸಾಹಿತ್ಯ ಚರಿತ್ರೆ, ಸಂಪುಟ 3, ಪುಟ 246} “ಸುಕುಮಾರ ಚರಿತೆ” ಎಂಬ ಚಂಪೂಕಾವ್ಯವನ್ನು ಬರೆದಿರುವ ಶಾಂತಿನಾಥ ಕವಿಯೂ ಶಿಕಾರಿಪುರದ ಎರಡು ಶಾಸನಗಳನ್ನು ಬರೆದಿದ್ದಾನೆ. ಈತನು ಗೋವಿಂದರಾಜನ ಮಗ.\endnote{ ಮಂಜುನಾಥನ್​, ಶಾಂತಿನಾಥ, ಕನ್ನಡ ಸಾಹಿತ್ಯ ಚರಿತ್ರೆ, ಸಂಪುಟ 3, ಪುಟ 748-762

ಕರ್ನಾಟಕ ಕವಿಚರಿತೆ, ಪ್ರಥಮ ಸಂಪುಟ, ಪುಟ 107-108} ಇವನ ಕಾಲ 1068–76. ಆದುದರಿಂದ ಮೇಲ್ಕಂಡ ಶಾಂತಿನಾಥರಿಬ್ಬರೂ ಲಾಳನಕೆರೆ ಶಾಸನದ ಶಾಂತಿನಾಥನಿಗಿಂತ ಭಿನ್ನರು.


\section{ಚಿದಾನಂದ (ಮಲ್ಲಿಕಾರ್ಜುನ) (1230)}

ಬಸರಾಳಿನ ಮಲ್ಲಿಕಾರ್ಜುನ ದೇವಾಲಯದ ಪ್ರವೇಶ ದ್ವಾರ ಮಂಟಪದಲ್ಲಿ ಇಟ್ಟಿರುವ, ಒಂದು ದೊಡ್ಡ ಶಾಸನ ಶಿಲೆಯಲ್ಲಿರುವ ಎರಡು ಶಾಸನಗಳನ್ನು ಚಿದಾನಂದ ಮಲ್ಲಿಕಾರ್ಜುನನೆಂಬ ಕವಿಯು ಬರೆದಿದ್ದಾನೆ.\endnote{ ಮೇವುಂಡಿ ಮಲ್ಲಾರಿ, ಕನ್ನಡ ನಾಡಿನ ಶಾಸನಕವಿಗಳು, ಪುಟ 255} ಇವನು ತನ್ನನ್ನು ಸತ್ಕವೀಶ್ವರನೆಂದು ಹೇಳಿಕೊಂಡಿದ್ದಾನೆ.\endnote{ ಎಕ 7 ಮಂ 29 ಬಸರಾಳು 1234, ಮಂ. 30 ಬಸರಾಳು 1237} ಇದರ ಜೊತೆಗೆ ಬಸರಾಳಿಗೆ ಸಮೀಪದಲ್ಲಿರುವ ಭೀಮನಹಳ್ಳಿಯ ಕೊಮ್ಮೆಯರ ಕುಲದ ಶಾಸನವನ್ನೂ ಇವನೇ ಬರೆದಿರುವಂತೆ ತೋರುತ್ತದೆ.\endnote{ ಎಕ 7 ನಾಮಂ 173 ಭೀಮನಹಳ್ಳಿ 1230} ಈ ಮೂರು ಶಾಸನಗಳಲ್ಲಿ ಅನೇಕ ಪದ್ಯಗಳು ಪುನರಾವರ್ತನೆಯಾಗಿದ್ದು ಬರವಣಿಗೆಯ ಒಕ್ಕಣೆ ಒಂದೇ ಬಗೆಯಾಗಿದೆ.

\begin{verse}
\textbf{ಧರೆಪೊಗಳೆ ಚಿದಾನನ್ದಂ} \\\textbf{ವಿರಚಿಸಿದಂ ಸತ್ಕವೀಶ್ವರಂ ಶಾಸನವಂ} \\\textbf{ಪರಮಪ್ರಕಾಶ ಯೋಗೀ} \\\textbf{ಶ್ವರ ತನೆಯಂ ಬ್ರಹ್ಮವಿದ್ಯೆಗಾಸ್ಪದ ರೂಪಂ}
\end{verse}

ಕವಿಚರಿತ್ರೆಕಾರರು ಈತನು ಪರಮಪ್ರಕಾಶಯೋಗೀಶ್ವರನ ಮಗನೆಂದು ಹೇಳಿದ್ದಾರೆ. ಈ ಶಾಸನದ ಏಳು ವೃತ್ತಗಳನ್ನು ಸಂಗ್ರಹಿಸಿಕೊಟ್ಟಿದ್ದಾರೆ.\endnote{ ನರಸಿಂಹಾಚಾರ್​, ಆರ್​., ಕರ್ನಾಟಕ ಕವಿಚರಿತೆ, ಪ್ರಥಮ ಸಂಪುಟ, ಪುಟ 419-21} ಸೂಕ್ತಿಸುಧಾರ್ಣವ ಗ್ರಂಥದ ಲೇಖಕ ಮಲ್ಲಿಕಾರ್ಜುನ ಅಥವಾ ಚಿದಾನಂದಮಲ್ಲಿಕಾರ್ಜುನನು ಜೈನಕವಿಯೆಂದೂ, ಈತನು ಕೇಶಿರಾಜನ ತಂದೆಯೆಂದೂ, ಜನ್ನನ ಸೋದರಿಯ ಗಂಡನೆಂದೂ “ಯೋಗಿಪ್ರವರ ಚಿದಾನಂದ ಮಲ್ಲಿಕಾರ್ಜುನ ಎಂಬುದರಿಂದ ಈತನು ಮುನಿಶ್ರೇಷ್ಠನೆಂದು ತಿಳಿಯುತ್ತದೆಂದು” ಈ ಇಬ್ಬರೂ ಬೇರೆಬೇರೆ ಎಂದು ಹೇಳಿದ್ದಾರೆ.\endnote{ ಅದೇ, ಪುಟ 430-441} ಪರಮಪ್ರಕಾಶ ಯೋಗೀಶ್ವರತನಯ ಚಿದಾನಂದ ಮತ್ತು ಮಲ್ಲಿಕಾರ್ಜುನ ಈ ಎರಡೂ ಒಬ್ಬನೇ ಕವಿಯ ಹೆಸರೆಂದು, ಈತನು “ಸೂಕ್ತಿಸುಧಾರ್ಣವ” ಎಂಬ ಪ್ರಾಚೀನ ಕನ್ನಡ ಕಾವ್ಯ ಸಂಕಲನ ಗ್ರಂಥವನ್ನು, ಬಸರಾಳು ಶಾಸನಗಳನ್ನೂ ರಚಿಸಿದ್ದಾನೆಂದೂ, ಶಬ್ದಮಣಿದರ್ಪಣದ ಕರ್ತೃ ಕೇಶಿರಾಜನು ಈತನ ಮಗನೆಂದೂ ಡಾ. ಟಿ.ವಿ. ವೆಂಕಟಾಚಲಶಾಸ್ತ್ರಿಗಳ ವಿವರವಾದ ವಿಶ್ಲೇಷಣೆಯಿಂದ ತಿಳಿದುಬರುತ್ತದೆ. ಬಸರಾಳು ಶಾಸನವನ್ನೂ ಶಾಸ್ತ್ರಿಗಳು ವಿವರವಾಗಿ ವಿಶ್ಲೇಷಣೆ ಮಾಡಿದ್ದಾರೆ.\endnote{ ವೆಂಕಟಾಚಲಶಾಸ್ತ್ರೀ, ಡಾ॥ ಟಿ.ವಿ., ಮಲ್ಲಿಕಾರ್ಜುನ-1237, ಕನ್ನಡ ಸಾಹಿತ್ಯ ಚರಿತ್ರೆ, ಸಂಪುಟ-4, ಭಾಗ-2 ಪುಟ 1581-1606} ಈ ಮೂರೂ ಶಾಸನಗಳಲ್ಲಿ ಕಾವ್ಯಸೌಂದರ್ಯದಿಂದ ಕೂಡಿರುವ ಅನೇಕ ಕಂದಪದ್ಯಗಳು, ಖ್ಯಾತಕರ್ನಾಟಕದ ವೃತ್ತಗಳು ಹೆಚ್ಚಿನ ಸಂಖ್ಯೆಯಲ್ಲಿವೆ. ರಾಜರು ಮತ್ತು ಅಧಿಕಾರಿಗಳ ವ್ಯಕ್ತಿತ್ವ ವರ್ಣನೆಯೇ ಪ್ರಧಾನವಾಗಿದ್ದು ಅನೇಕ ಐತಿಹಾಸಿಕ ಅಂಶಗಳಿಂದ ಕೂಡಿವೆ. ಬಸರಾಳು ದೇವಾಲಯದ ವರ್ಣನೆ ವಿಶಿಷ್ಟವಾಗಿ ಮೂಡಿಬಂದಿದೆ. ಚಿದಾನಂದ ಮಲ್ಲಿಕಾರ್ಜುನನು ಸ್ಮಾರ್ತ\break ಬ್ರಾಹ್ಮಣನಾಗಿದ್ದು, ಬಸರಾಳಿನ ಮಲ್ಲಿಕಾರ್ಜುನ ದೇವಾಲಯದ ನಿರ್ಮಾತೃ ಹರಿಹರ ದಂಡನಾಯಕನಿಗೆ ಗುರುವಾಗಿದ್ದನೆಂದು ಹೇಳಬಹುದು. ಈತನ ಮಗ ಕೇಶಿರಾಜನು ತನ್ನ ಮಾವ ಜನ್ನನ ಪ್ರಭಾವದಿಂದ ಜೈನ ಮತವನ್ನು ಸ್ವೀಕರಿಸಿರಬಹುದು. ಈ ಶಾಸನದ ಕೆಲವು ಪದ್ಯಗಳು.


\section{ಐತಿಹಾಸಿಕವಾಗಿ ಚರ್ಚಾಸ್ಪದವಾಗಿರುವ ಸೋಮೇಶ್ವರನ ವರ್ಣನೆ}

\begin{verse}
\textbf{ಮುಂನಂ ರೂಡಿಯ ಕೃಷ್ಣ ಕಂಧರನುಮಂ ಮಾರ್ಕ್ಕೊಂಡು ಚೋಳೋರ್ವ್ವಿಯಂ \\ ನಿಂನಂತಾರೊಳಪೊಕ್ಕು ಸಾದಿಸಿದರಾರ್ಪ್ಪಾಂಡೇಶನಂ ಶೌರ್ಯದಿಮ \\ ಬೆನ್ನಂ ಪತ್ತಿಸೆ ಸೋವಿದೇವ ಘಟೆಯುಂ ಕೈಕೊಂಡರಾರ್ಚ್ಚೋಳನಂ \\ ತಂನಾಮ್ನಾಯದ ರಾಜ್ಯದೊಳ್​ ನಿರಿಸಿದಸ್ಸೋಮಾನ್ವಯಯೋರ್ವ್ವೀಶ್ವರರ್​}
\end{verse}


\section{ದಕ್ಷಿಣ ಚಕ್ರವರ್ತಿ ಸೋಮೇಶ್ವರನ ಕಡಿತಕ್ಕೇರಿದ ರಾಜ್ಯಗಳು:}

\begin{verse}
\textbf{ಗಡಿ ಮೂಡಲು ಸಲೆ ಕಂಚಿಯಿತ್ತ ಪಡುವಲ್​ ತಳ್ತಿರ್ದ್ದ ವೇಳಾಪುರಂ \\ ಬಡಗಲ್​ ಪೆರ್ದ್ದೊರೆ ತೆಂಕಲಂಕದ ಬಬೆಯನಾಡಾಂಕಿಯಾದೀ ನೆಲಂ \\ ಕಡಿತಕ್ಕೇರಿತು ಸೋವಿದೇವ ನೃಪನಿಂದೇವಣ್ನಿಪೆಂ ರಾಯರೊಳ್​ \\ ಪಡಿಯಾರ್ದ್ದಕ್ಷಿಣ ಚಕ್ರವರ್ತ್ತೀ ತಿಳಕಂಗೀ ವಿಶ್ವಭೂಪಾಳಕರ್​}
\end{verse}


\section{ಸೋಮೇಶ್ವರನ ಮಗ ನರಸಿಂಹ:}

\begin{verse}
\textbf{ಶ‍್ರೀವಧು ಮುತ್ತಿನೆಕ್ಕಸರದಂತುರದೊಳ್​ ನಲಿವಂತು ವಿಕ್ರಮ \\ ಶ‍್ರೀವಧು ಬಾಹುಪೂರಕದವೊಲ್​ ಭುಜದೊಳ್​ ನಲಿವಂತು ಕೀರ್ತಿ ದಿ \\ ಗ್ದೇವಿಯರೊಳ್​ ನಿಜಾಜ್ಞೆವೆರಸಾದರದಿಂ ನಲಿವಂತು ಧರ್ಮಲ \\ ಕ್ಷ್ಮೀವರನಾಗಿ ಪಾಳಿಸಿದನುರ್ವ್ವರೆಯಂ ನರಸಿಂಹ ಭೂಬುಜಂ}
\end{verse}


\section{ಅಡ್ಡಾಯದ ಹರಿಹರ ದಂಡನಾಯಕ:}

\begin{verse}
\textbf{ಚರಿತಂ ಗಂಗಾನದೀಸಂಗಮಸಹಚರಮಾಸ್ಯೇಂದು ಸತ್ಯಾಮೃತಶ‍್ರೀ\\ ಭರಿತಂ ಲಕ್ಷ್ಮೀ ವಿಳಾಸಂ ದ್ವಿಜಗುರುಬುಧಗೋತ್ರಾದಿ ಸದ್ದಾನದೀಕ್ಷಂ \\ ಗುರುಚಿತ್ತಂ ಪಾರ್ವತೀವಲ್ಲಭ ಪದಕಮಳ ಧ್ಯಾನ ಸಂಧಾನ ಸಾರಂ \\ ನರಸಿಂಹೋರ್ವೀಶನಡ್ಡಾಯದದ ಹರಿಹರಂ ಲೋಕದೊಳ್​ ತಾನೆ ಧನ್ಯಂ}
\end{verse}


\section{ಮಲ್ಲಿಕಾರ್ಜುನ ದೇವಾಲಯ}

\begin{verse}
\textbf{ಮೊದಲಿಂದಂ ಕಳಶಂಬರಂ ಮೆರೆವ ನಾನಾಚಿತ್ರಪತ್ರಂಗಳಿಂ \\ ಮುದಮಂ ಬೀರುವ ಭಾರತಾದಿ ಕಥೆಯಿಂ ಮೆಯ್ವೆತ್ತ ಕೂಟಂಗಳಿಂ \\ ದಿದು ಪಾಂಚಾಳಿಕೆ ತಳ್ತ ಮೇರುಗಿರಯೋ ಪೇಳೆಂಬಿನಂ ವಿಭ್ರಮಾ \\ ಸ್ಪದ ಮಾಗಿರ್ಪುದು ಮಲ್ಲಿಕಾರ್ಜುನ ದೇವಾಲಯಂ}
\end{verse}


\section{ನೃಸಿಂಹ (1381)}

ಈತನು ಮದ್ದೂರು ತಾಲ್ಲೂಕು ಅರುವನಹಳ್ಳಿಯ ಭಟ್ಟರ ಬಾಚಿಯಪ್ಪನ ಮರಣವನ್ನು ತಿಳಿಸುವ ಶಾಸನವನ್ನು ಬರೆದಿದ್ದಾನೆ.\endnote{ ಮೇವುಂಡಿ ಮಲ್ಲಾರಿ, ಕನ್ನಡ ನಾಡಿನ ಶಾಸನ ಕವಿಗಳು, ಪುಟ 274} ಈ ಶಾಸನದ ಪದ್ಯಗಳು ಭಟ್ಟರ ಬಾಚಿಯಪ್ಪನ ಚರಮಗೀತೆಯಂತಿದ್ದು ಉತ್ತಮ ಕವಿತಾ ಗುಣವನ್ನು ಹೊಂದಿವೆ.

\begin{verse}
\textbf{ಶೂರತೆಯಂ ಮ್ರಿಗಾಧಿಪನೊಳಾ ಕ್ಷಮೆಯಂ ಕ್ಷಿತಿಯೊಳ್ಗಭೀರಮಂ} \\\textbf{ವಾರಿಧಿಯೊಳ್ಮನೋರತೆಯಂ ಮಕರಧ್ವಜನೊಳ್ಸುಶಾಂತಿಯಂ} \\\textbf{ವಾರಿಜವೈರಿಯೊಳ್ಪಡದು ಪದ್ಮಜ ನಿರ್ಮ್ಮಿಸಿದಂತಿರಂಜಿತಂ} \\\textbf{ಚಾರುಚರಿತ್ರ ಕೀರ್ತಿಯ ತನೂಭವ ಬಾಚನುದಾರನುರ್ವಿಯೊಳು}
\end{verse}

ಈ ಪದ್ಯವು ರನ್ನ ಮಹಾಕವಿಯ ಸಾಹಸಭೀಮ ವಿಜಯದಲ್ಲಿ ಬರುವ “ಶರಸಂದೋಹಮನನ್ಯಸೈನ್ಯದೊಡಲೊಳ್​ ಬಿಲ್ಬಲ್ಮೆಯಂ ತನ್ನ ಶಿಷ್ಯರಮಯ್ಯೊಳ್​ ನಿಜಕೀರ್ತಿಯಂ ನಿಖಿಳ ದಿಕ್ಚಕ್ರಂಗಳೊಳ್​...” ಪದ್ಯದ ಛಾಯೆಯನ್ನು ಹೋಲುತ್ತದೆಂದು ಹೇಳಬಹುದು.

\begin{verse}
\textbf{ಪದ್ಯವ ವಿರಚಿಸಿದಂ ನಿರ} \\\textbf{ವದ್ಯಂ ಸುಕವೀಂದ್ರಲಪವನ ಮಣಿಮಯಮುಕುರಂ} \\\textbf{ಮಾದ್ಯತ್​ ಕಂಠೀರವ ರವ} \\\textbf{ಮಾದ್ಯತ್ತು ಹಿನಾಂಶು ಕಿರಣ ಕೀರ್ತಿ ನೃಸಿಂಹಂ.}\endnote{ ಎಕ 7 ಮ 87 ಅರುವನಹಳ್ಳಿ 1381}
\end{verse}

ಎಂದು ಕವಿ ತನ್ನನ್ನು ಹೊಗಳಿಕೊಂಡಿದ್ದಾನೆ. ಕಂಠೀರವರವ ಎಂಬುದು ರನ್ನನ “ಆ ರವಮಂ, ನಿರ್ಜಿತ ಕಂಠೀರವಮಮ್” ಎಂಬುದರ ಅನುಕರಣೆಯಾಗಿದೆ ಎಂದು ಹೇಳಬಹುದು.


\section{ತಿರುಮಲಾರ್ಯ(1645-1707)}

ಚಿಕದೇವರಾಜ ಒಡೆಯರ ಮಂತ್ರಿಯೂ, ಸ್ನೇಹಿತನೂ, ಪ್ರಸಿದ್ಧ ಕವಿಯೂ ಆಗಿದ್ದನು. ಇವನು ತಿರುಮಕೂಡಲು ನರಸೀಪುರ,\endnote{ ಎಕ 3 ತೀನಪು 23} ಚಾಮರಾಜನಗರ\endnote{ ಎಕ 4 ಚಾನ 11 ಚಾಮರಾಜನಗರ 1675} ಶಾಸನಗಳನ್ನು ಬರೆದಿದ್ದಾನೆ. ಶ‍್ರೀರಂಗಪಟ್ಟಣದ ಕ್ರಿ.ಶ.1686ರ ತಾಮ್ರಶಾಸನವನ್ನೂ,\endnote{ ಎಕ 6 ಶ‍್ರೀಪ 24 ಶ‍್ರೀರಂಗಪಟ್ಟಣ 1686} ಕೂಡಾ ಈ ತಿರುಮಲಾರ್ಯನೇ ಬರೆದಿದ್ದಾನೆಂದು ವಿದ್ವಾಂಸರು ಊಹಿಸಿದ್ದಾರೆ.\endnote{ ಹರಿಶಂಕರ್​, ಎಸ್​.ಎಸ್​., ತಿರುಮಲಾರ್ಯ, ಪುಟ 111} ಪ್ರೌಢ ಸಂಸ್ಕೃತ ಭಾಷೆಯಿಂದ, ಅನೇಕ ಸಂಸ್ಕೃತದ ಅಕ್ಷರ ವೃತ್ತಗಳಿಂದ ಈ ಶಾಸನ ಕೂಡಿದೆ. ಈ ಶಾಸನಗಳ ಅಂತ್ಯದಲ್ಲಿ \textbf{“ಚಿಕದೇವರಾಜ ನೃಪತೇ ಸಭಾಸ್ಸುಧರ್ಮಾಮಿವಾಧ್ಯಾಸ್ತೇ। ತಸ್ಯಾಸ್ಯ ಕೌಶಿಕಾನ್ವಯಸಿಂಧು ವಿಧೋರಲಗ ಸಿಂಗಾರಾರ್ಯ್ಯಸ್ಯ~। ತನಯಸ್ತಿರುಮಯಾರ್ಯ್ಯೋ ವ್ಯತಾನೀತ್ತಾಂಬ್ರಶಾಸನ ಶ್ಲೋಕಾನ್​॥} ಎಂದು ಹೇಳಿದ್ದು, ಇವನು ಚಿಕದೇವರಾಜನ ಸಭಾ ಭೂಷಣನಾಗಿದ್ದನೆಂದು, ಕೌಶಿಕ ಗೋತ್ರದ ಸಿಂಗಾರಾರ್ಯನ ಮಗನೆಂದೂ ತಿಳಿದುಬರುತ್ತದೆ.

ಆದರೆ ಈ ಶಾಸನದ ಕರ್ತೃ ರಾಮಾಯಣಂ ತಿರುಮಲಾರ್ಯನೆಂದು ಶ‍್ರೀ ಹಯವದನರಾಯರು ಹೇಳಿದ್ದಾರೆ. ಆದರೆ ಇವರಿಬ್ಬರೂ ಬೇರೆ. ಈ ತಿರುಮಲಾರ್ಯನ ಬಗ್ಗೆ ಪಿ.ಎನ್​. ನರಸಿಂಹ ಮೂರ್ತಿಯವರು ಉಲ್ಲೇಖಿಸಿಲ್ಲ.


\section{ರಾಮಾಯಣಂ ತಿರುಮಲಾರ್ಯ (1722)}

ತೊಣ್ಣೂರಿನ ಒಂದು ದೊಡ್ಡ ತಾಮ್ರಶಾಸನವನ್ನು ಮತ್ತು ಮೇಲುಕೋಟೆಯ ಎರಡು ತಾಮ್ರಶಾಸನಗಳನ್ನು, ರಾಮಾಯಣ ಮಹಾಭಾರತ ಪಾರಾಯಣ ಮತ್ತು ಕೃತಿರಚನೆಯನ್ನೇ ವೃತ್ತಿಯನ್ನಾಗಿ ಮಾಡಿಕೊಂಡಿದ್ದ, ಕವಿ ತಿರುಮಲೆಯಾರ್ಯನು ಬರೆದಿರುತ್ತಾನೆ. ತೊಣ್ಣೂರಿನ ದೊಡ್ಡ ತಾಮ್ರಶಾಸನದಲ್ಲಿ ಬಂದಿರುವ ಈ ಕೆಳಕಂಡ ವರ್ಣನೆಯನ್ನು ನೋಡಿದರೆ\break ತಿರುಮಲಾರ್ಯನು ಸಂಸ್ಕೃತ, ಕನ್ನಡ ಮತ್ತು ತೆಲುಗು ಭಾಷೆಗಳಲ್ಲಿ ಉದ್ದಾಮ ಪಂಡಿತನಾಗಿದ್ದು, ಸಂಗೀತದಲ್ಲಿ ಪರಿಣತ\-ನಾಗಿದ್ದುದರ ಜೊತೆಗೆ, ಉಭಯ ಭಾಷೆಯ ಕವಿಯಾಗಿದ್ದನೆಂದು ಊಹಿಸಬಹುದು.

\begin{verse}
\textbf{“ಕರ್ನ್ನಾಟಾಂಧ್ರ ಸುಸಂಸ್ಕೃತ ಕವಿತಾ ಗಾಂಧರ್ವ್ವಕೇಷು ಯಃಕ್ಕುಶಲಃ। } \\\textbf{ತೇನೇಮ ರಾಮಾಯಣ ತಿರುಮಲೆಯಾಚಾರ್ಯ ಸೂರಿಣಾ ಫಣಿತಾಃ।} \\\textbf{ಗ್ರಂಥಾಸ್ಸಂತೋಷೌಯಪ್ರಭವಂತ್ವಿಹ ತಾಂಬ್ರಶಾಸನೇ ಲಿಖಿತಾಃ।} \\\textbf{ಸಂಪತ್ಸಾರಸ್ವತ ಬಹುಸಂತಾನಕ್ಷೇಮಸರ್ವ್ವ ಸೌಖ್ಯಾಯ।} \\\textbf{ಕಲ್ಯಾಣಾಯ ಯಥೇಷ್ಟಕಂ ಕಲಿತ ಸಮಪಸ್ತೇಪ್ಸಿತಾರ್ತ್ಥಲಾಭಾಯ।} \\\textbf{ಶ‍್ರೀರಾಮಾಯಣ ಭಾರತ ಪಾರಾಯಣ ನಿಹಿತಾ ವೃತ್ತಿನಾ ಕೃತಿನಾ।} \\\textbf{ಕವಿನಾ ತಿರುಮಲೆಯಾಚಾರ್ಯ್ಯೇಣೇಮ ತಾಂಬ್ರಶಾಸನಂ ಲಿಖಿತಂ।}\endnote{ ಎಕ 6 ಪಾಂಪು 99 ತೊಣ್ಣೂರು 1722}
\end{verse}

\textbf{ಶ‍್ರೀ ರಾಮಾಯಣ ಭಾರತ ಪಾರಾಯಣ ವಿಹಿತ ವೃತ್ತಿನಾ ಕೃತಿನಾ। ಕವಿನಾ ತಿರುಮಲೆಯಾಚಾರ್ಯೇಣೇದನ್ತಾಮ್ರ ಶಾಸನಂ ಲಿಖಿತಂ।,}\endnote{ ಎಕ 6 ಪಾಂಪು 215 ಮೇಲುಕೋಟೆ 1724}\textbf{ ಶ‍್ರೀ ರಾಮಾಯಣ ಭಾರತ ಪಾರಾಯಣ ವಿಹಿತಿ ವೃತ್ತಿನಾ ಕೃತಿನಾ। ಕವಿನಾ ತಿರುಮಲೆಯಾಚಾರ್ಯೇಣದಂ ತಾಮ್ರಶಾಸನಂ ಲಿಖಿತಂ।,}\endnote{ ಎಕ 6 ಪಾಂಪು 216 ಮೇಲುಕೋಟೆ 1725} ಎಂದು ಮೇಲುಕೋಟೆಯ ತಾಮ್ರಶಾಸನಗಳಲ್ಲಿ ರಾಮಾಯಣಂ ತಿರುಮಲಾರ್ಯನ ಹೆಸರು ಉಲ್ಲೇಖವಾಗಿದೆ. “\textbf{ಕೌಂಡಿಣ್ಯ ಗೋತ್ರದ ಅಮಳವೂರ್​ ಮನೀಷಿ ಶಠಕೋಪಾರ್ಯನ ಪೌತ್ರ, ಶ‍್ರೀ ರಾಮಾಯಣಂ ಶಿಂಗಾರಾರ್ಯನ ತನೂಜ ರಾಮಾಯಣಂ ತಿರುಮಳಾಚಾರ್ಯ”} ಎಂದು ತೊಣ್ಣೂರು ಶಾಸನದಲ್ಲಿ ಹೇಳಿದೆ.\endnote{ ಎಕ 6 ಪಾಂಪು 99 ತೊಣ್ಣೂರು 1722} ರಾಮಾಯಣಂ ತಿರುಮಲೆಯಾಚಾರ್ಯನು ಚಿಕ್ಕದೇವರಾಯನ ಆಸ್ಥಾನದಲ್ಲಿದ್ದನೆಂದು ಶ‍್ರೀ ಹಯವದನರಾಯರೂ, ಈತನು ಕೃಷ್ಣರಾಜನ ಆಸ್ಥಾನದಲ್ಲಿದ್ದನೆಂದು ಕವಿಚರಿತೆಕಾರರೂ ಹೇಳಿದ್ದಾರೆ. ಆದರೆ ರಾಮಾಯಣಂ ತಿರುಮಲೆಯಾರ್ಯನು ಚಿಕದೇವರಾಜನ ಆಸ್ಥಾನ ಕವಿಯಾಗಿರಲಿಲ್ಲವೆಂದು, ಈತನು ಶಾಸನ ರಚನೆಯಲ್ಲಿ ತಿರುಮಲಾರ್ಯನನ್ನು ಅನುಸರಿಸಿದ್ದು ತಿರುಮಲಾರ್ಯನು ರಚಿಸಿರುವ ಶಾಸನ ಹಾಗೂ ಕೃತಿಗಳಿಂದ ಕೆಲವು ಪದ್ಯಗಳನ್ನು ತೆಗೆದುಕೊಂಡಿರುವುದು ಕಂಡುಬರುತ್ತದೆಂದು ವಿದ್ವಾಂಸರು ಅಭಿಪ್ರಾಯಪಟ್ಟಿದ್ದಾರೆ.\endnote{ ಹರಿಶಂಕರ್​, ಎಚ್​.ಎಸ್​., ತಿರುಮಲಾರ್ಯ, ಪುಟ 112-113}

ಈ ಶಾಸನದಲ್ಲಿ, ಈ ಕಾಲಕ್ಕೆ ಅಪರೂಪವಾಗಿದ್ದ, ಸಂಸ್ಕೃತ ಅಕ್ಷರ ವೃತ್ತಗಳನ್ನು, ಅದರಲ್ಲೂ ಗೀತಿಕೆಯನ್ನು, ಕಂದ ಪದ್ಯಗಳನ್ನು, ವಿಪುಲವಾಗಿ ಬಳಕೆ ಮಾಡಿದ್ದಾನೆ. ಜೊತೆಗೆ ಶ್ಲೋಕವೂ ಇದೆ. ಅಗ್ರಹಾರದ ಎಲ್ಲೆಗಳನ್ನು ಹೇಳುವಾಗ ಎಲ್ಲವನ್ನೂ ಸ್ಪಷ್ಟವಾಗಿ ಹೇಳಿದ್ದು, ಇವನ ಭೂಗೋಳ ಜ್ಞಾನಕ್ಕೆ ಸಾಕ್ಷಿಯಾಗಿದೆ.


\section{ನೃಸಿಂಹಸೂರಿ(1674)}

ಕಂಠೀರವ ನರಸರಾಜ ಒಡೆಯರ ಕಾಲದ ಮೇಲುಕೋಟೆ ತಾಮ್ರ ಶಾಸನವನ್ನು ಕೌಶಿಕ ವಂಶದ ವೇದ ವಿದ್ವಾಂಸನಾದ, ಶ‍್ರೀನಿವಾಸನ ಪುತ್ರ ನರಸಿಂಹಸೂರಿಯು ಬರೆದಿರುತ್ತಾನೆ. ಈತನೂ ಕೂಡಾ ಒಡೆಯರ ಆಸ್ಥಾನದ ಶಾಸನ ಬರಹಗಾರನಾಗಿದ್ದ\-ನೆಂದು ಹೇಳಬಹುದು.\endnote{ ಎಕ 6 ಪಾಂಪು 214 ಮೇಲುಕೋಟೆ 1674} ಈತನು ಶ‍್ರೀಭಾಷ್ಯಕ್ಕೆ “ಬ್ರಹ್ಮವಿದ್ಯಾಕೌಮುದಿ” ಎಂಬ ಟಿಪ್ಪಣಿಯನ್ನು ಬರೆದಿರುವ ಮೇಲುಕೋಟೆಯ ಶ‍್ರೀನಿವಾಸನ ಪುತ್ರನಾಗಿರಬಹುದು.\endnote{ ಸೂರ್ಯನಾಥ ಕಾಮತ್​ ಡಾ॥, ತಿರುಮಲಾರ್ಯನ ಕಾಲ, ಸನ್ನಿವೇಶ, ತಿರುಮಲಾರ್ಯ, ಸಂ. ವಿಸೀ, ಪುಟ 18}


\section{ಶಾಸನ ಸಾಹಿತ್ಯ:}

ಜಿಲ್ಲೆಯ ಅನೇಕ ಶಾಸನಗಳನ್ನು ಅಜ್ಞಾತ ಕವಿಗಳು, ಬರಹಗಾರರೂ ಬರೆದಿದ್ದು, ಶಾಸನದ ಸೀಮಿತತೆ ಹಾಗೂ ಸಂಕ್ಷಿಪ್ತತೆಯಲ್ಲಿ ಅವರ ಕವಿತ್ವ ಪ್ರದರ್ಶನ ಮಾಡಲು ಅವಕಾಶ ಸಿಕ್ಕಿಲ್ಲ. ಆದರೂ ಮಿಂಚಿನಂತೆ ಅಲ್ಲಲ್ಲಿ ಸಾಹಿತ್ಯದ ದೃಷ್ಟಿಯಿಂದ ಗಮನಾರ್ಹವಾದ, ಮನಸೆಳೆಯುವ, ಕಂದ ಮತ್ತು ವೃತ್ತಗಳನ್ನು ಬಳಸಿದ್ದಾರೆ. ಅವು ಛಂದಸ್ಸಿನ ದೃಷ್ಟಿಯಿಂದ ಅಶುದ್ಧವಾದರೂ ಮನೋಜ್ಞವಾಗಿವೆ. ವಿವಿಧ ಅಧ್ಯಾಯಗಳಲ್ಲಿ ಶಾಸನಗಳಲ್ಲಿ ವಿಶ್ಲೇಶಿಸುವಾಗ ಕವಿತಾ ಗುಣವುಳ್ಳ ಕೆಲವು ಪದ್ಯಗಳನ್ನು ಅಲ್ಲಲ್ಲಿ ಉಲ್ಲೇಖಿಸಲಾಗಿದೆ. ಉಳಿದಂತೆ ಸಾಹಿತ್ಯ ದೃಷ್ಟಿಯಿಂದ ಗಮನಾರ್ಹವಾದ ಕೆಲವು ಪದ್ಯಗಳನ್ನು ಈ ಕೆಳಗೆ ನೀಡಲಾಗಿದೆ. ಕೆಲವು ಶಾಸನ ಗದ್ಯವೂ ಚೆನ್ನಾಗಿದ್ದು, ಅದನ್ನು ವಿಸ್ತಾರದ ಕಾರಣದಿಂದ ಇಲ್ಲಿ ನೀಡಿರುವುದಿಲ್ಲ. ಅಗ್ರಹಾರದ ಅಥವಾ ದತ್ತಿ ನೀಡಿದ ಭೂಮಿಯ ಎಲ್ಲೆಗಳ ವರ್ಣನೆಯು ಭೂಗೋಳ ಶಾಸ್ತ್ರವಾಗಿದ್ದು, ಇಂದಿನ ಮೋಜಿನಿ ಪದ್ಧತಿಯವರು ಇದನ್ನು ಅಧ್ಯಯನ ಮಾಡುವುದು ಒಳ್ಳೆಯದು.

ಜಿಲ್ಲೆಯ ಪ್ರಾಚೀನ ಶಾಸನಗಳಲ್ಲಿ ಒಂದಾದ ಕಂಬದಹಳ್ಳಿ ಶಾಸನಗಳಲ್ಲಿ ಬರುವ ಜೈನ ಯತಿಗಳ ವರ್ಣನೆ.\endnote{ ಎಕ 7 ನಾಮಂ 33 ಕಂಬದಹಳ್ಳಿ, ಕ್ರಿ.ಶ.850-900}

\noindent
\textbf{ಪಾಲ್ಯಕೀರ್ತಿ ದೇವ:}

\begin{verse}
\textbf{ಸುರಕರಿಯ ಕಾಮಧೇನುವ \\ ಸರದಭ್ರಕಾನ್ತಿಯಂ ಪುದುಂಗೊಳಿಸುತ್ತುಂ \\ ಶರದಮಳಚನ್ದ್ರ ಬಿಂಬದ \\ ದೊರೆಮಿಗಿಲ್​ ಪಾಲ್ಯಕೀರ್ತಿ ದೇವರ ಕೀರ್ತಿ}
\end{verse}

\noindent
\textbf{ಪಲ್ಲ ಪಂಡಿತ:}

\begin{verse}
\textbf{ಏವೊಗಳ್ಪುದುಣ್ನ ವಿಬುಧಜ \\ ನಾವಳಿಗಂ ಬೇಡಿದರ್ತ್ಥಿಜನಕನ್ನಿಚ್ಛ \\ ನ್ದೇವತರುಕುಡುವ ತೆರದ \\ ನ್ತಿವರ್ಸ್ಸಲೆ ಪಲ್ಲ ಪಣ್ಡಿತರ್ವ್ವಸುಮತಿಯೊಳ್​}
\end{verse}

\noindent
\textbf{ವಿನಯ ನಂದಿ ಮುನಿ:}

\begin{verse}
\textbf{ನಾಡೊಳಗಿದೆಸೆದ ಗೋಸನೆ \\ ಬಾಡಂಗಳ್ಗೆರಗಿದನ್ದೆ ಮುನಿ ವನಿತೆಯರೊ\\ ಳ್ಕೂಡಿದನೆಂಬೀ ನುಡಿಯದ \\ ನೇಡಿಪುದೆಲೆ ವಿನಯನನ್ದಿದೇವರ ಚರಿತಂ}
\end{verse}

\newpage

\noindent
\textbf{ಎಕವೀರ ಭಟಾರ}

\begin{verse}
\textbf{ದಾನದ ಪೆಮ್ಪು ದೀನಜನಕೋಟಿಗೆ ಕಲ್ಪಕುಜಾಳಿ ನೋಡೆ ಸ \\ ನ್ಮಾನದ ಪೆಂಪು ಭವ್ಯಜನಸಂಕುಳ ಮನ್ತಣಿಪಿತ್ತು ದಾನ ಸ \\ ನ್ಮಾನ ತಪೋಪವಾಸ ಗುಣ ಸನ್ತತಿಯಂ ಸಲೆ ತಾಳ್ದಿದರ್ಜ್ಜಗ \\ ನ್ಮಾನಿಗಳೇಕವೀರ ಮುನಿನಾಥರೆ ಜಂಗ ತೀರ್ತ್ಥವಲ್ಲರೇ}
\end{verse}

ಹೊಸಹೊಳಲು ಶಾಸನದಲ್ಲಿ ಬರುವ ದಿವಾಕರಣಂದಿ ಮತ್ತು ಕುಕ್ಕುಟಾಸನ ಮಲಧಾರಿ ದೇವರ ವರ್ಣನೆ.\break (ಈ ಕಂದಪದ್ಯ ಮೊದಲಿಗೆ ಶಾಂತಿನಾಥನ ಸುಕುಮಾರಚರಿತೆಯಲ್ಲಿ ಕಾಣಿಸಿಕೊಂಡಿದೆ). ಇದೇ ಶಾಸನದಲ್ಲಿ ದೇಮಿಕಬ್ಬೆಯ ವರ್ಣನೆಯೂ ಇದೆ.

\noindent
\textbf{ದಿವಾಕರಣಂದಿ}

\begin{verse}
\textbf{ವಿದಿತ ವ್ಯಾಕರಣದ ತ \\ ರ್ಕ್ಕದ ಸಿದ್ದಾನ್ತದ ವಿಶೇಷದಿಂ ತ್ರೈವಿದ್ಯಾ \\ ಸ್ಪದದಿಂದಿರೆ ಬಣ್ನಿಪು \\ ದು ದಿವಾಕರಣಂದಿ ದೇವಸಿದ್ಧಾಂತಿಗರಂ}
\end{verse}

\noindent
\textbf{ಕುಕ್ಕುಟಾಸನ ಮಲಧಾರಿದೇವನ ವರ್ಣನೆ(ಈ ಪದ್ಯವು ಶ್ರವಣಬೆಳಗೊಳದ ಶಾಸನದಲ್ಲಿಯೂ ಇದೆ)}

\begin{verse}
\textbf{ಬಳಯುತರ ಬಳಲ್ವ ಚಲಂ ತಾಂ ನೇಸರಂಗಿದಿರಾಗಿ ಸಂ \\ ಚಳಿಸೆ ಪಳಂಚಿ ತೂಲ್ದವನನೋಡಿಸಿ ಮೆಯ್ವಗೆಯದ ಧೂಸಱಿಂ \\ ಕೆಳೆಯದೆ ನಿಂದ ಕರ್ವ್ವುನದ ಮಿರ್ಗ್ಗಿಡಸಿರ್ಪ್ಪಿನಮಕ್ಕೆನತ್ತ ಕ \\ ತ್ತಳಮೆನಿಸಿತು ಪುತ್ತದಮೆಯ್ಯ ಮಲಧಾರಿದೇವರಂ}
\end{verse}

\noindent
\textbf{ದೇಮಿಕಬ್ಬೆ}

\begin{verse}
\textbf{ಉತ್ತಮ ವಸ್ತು ಸುವರ್ನ್ನಮ\\ ನುತ್ತಮ ಪಾತ್ರಕ್ಕೆ ಗೊಟ್ಟು ದೇಮಾಂಬಿಕೆ ತಾ\\ ನುತ್ತಮೆಯನೆ ಸಕಳಜನಂ\\ ಕತ್ತರಿಘಟ್ಟದೊಳ್ ಬಸದಿಯಂ ಮಾಡಿಸಿದಳ್}
\end{verse}

\noindent
\textbf{ವಿಷ್ಣುವರ್ಧನನ ವಿಜಯಗಳು}

ವಿಷ್ಣುವರ್ಧನನು ಸಾಧಿಸಿದ ವಿಜಯಗಳನ್ನು ಸುಂಕಾತೊಂಡನೂರು ಶಾಸನವು ಸಂಕ್ಷಿಪ್ತವಾಗಿ ಪಟ್ಟಿಮಾಡಿದೆ.\endnote{ ಎಕ 6 ಪಾಂಪು 236 ಸುಂಕಾತೊಂಡನೂರು 12ನೇ ಶ.} "ಶ‍್ರೀಮನ್ಮಹಾಮಂಡಲೇಶ್ವರಂ ತ್ರಿಭುವನಮಲ್ಲ ಹೊಯ್ಸಳ ಶ‍್ರೀವಿಷ್ಣುವರ್ಧನದೇವರ ಪ್ರತಾಪವೆಂತೆಂದಡೆ:"

\begin{verse}
\textbf{ನಂಗಲಿ ಕೊಂಗು ಸಿಂಗಮಲೆ ರಾಯಪುರಂ ತಳಕಾಡುರೊದ್ದ \\ ಬೆಂಗಿರಿ ಹೊಸಕೊಳ್ಳಗ್ಗಿರಿ ಬಳ್ಳರೆವಲ್ಲುರು ಚಕ್ರಗೊಟ್ಟಮು \\ ಚ್ಚಂಗಿ ವಿರಾಟನಪೊಳಲು ಬಂಕಪುರಂ ಬನವಾಸೆ ಕೊಯತೂ \\ರ್ತ್ತುಂಗಪರಾಕ್ರಮಂ ವಿಜಯವರ್ದ್ಧನನೀ ಕಲಿ ವಿಷ್ಣುವರ್ಧನದೇವ}
\end{verse}

\begin{verse}
\textbf{ನೀಳಾದ್ರೀಪಡಿಯಘಟ್ಟಂ \\ ಏಳುಂಮಲೆ ಕಂಚಿ ತುಳುವರಾಜೇಂದ್ರಪುರಂ \\ ಕೋಳಾಲಬಯಲುನಾಡುಮಂ \\ ನಾಳಾಪದೆ ಕೊಂಡ ವಿಷ್ಣುವರ್ಧನದೇವ}
\end{verse}

\begin{verse}
\textbf{ಹಲಸಿಗೆ ಬೆಳುವಲವೊಪ್ಪುವ \\ ಹುಲಿಗೆರೆಯಾಲೊಕ್ಕಿಗುಂಡಿ ಹೆದ್ದರೆವರೆಗಂ \\ ಕಲಿಗಳನೆ ತಗುಳ್ದು ವಿಕ್ರಮ \\ ಬಲದಿಂ ಕೈಕೊಂಡ ವಿಷ್ಣು ಭೂಮಂಡಳಮಂ}
\end{verse}

\noindent
\textbf{ಮದ್ದೂರಿನ ವೈದ್ಯನಾಥಪುರದಲ್ಲಿರುವ ಶಾಸನದಲ್ಲಿರುವ ಸೋಮ ದಂಡನಾಯಕನ ಅಳಿಯ ಕೇತಚಮೂಪತಿಯ ವರ್ಣನೆ.}\endnote{ ಎಕ 7 ಮ 69 ವೈದ್ಯನಾಥಪುರ 1261}

\begin{verse}
\textbf{ಆತನ ಮಂತ್ರಿ ಲಲಾಮಂ \\ ನೀತಿಗೆ ಚಾಣಕ್ಯನೆನಿಪ ಸೋಮಂಗಳಿಯಂ \\ ಕೇತಚಮೂಪತಿ ಪದಪಿಂ \\ ಖ್ಯಾತಿಯ ಮದ್ದೂರ ವೈಜನಾಥಂಗೊಲವಿಂ}
\end{verse}

\begin{verse}
\textbf{ವೀರಶ‍್ರೀ ವಧೂ ವಲ್ಲಭಂ ಭುಜಬಳಂ ಸಾಹಿತ್ಯನತ್ಯಂತ ಗಂ \\ ಭೀರಂ ಮಾವನ ಗಂಧವಾರಣನುದಾರಂ ನೋಳ್ಪಡ ಸಜ್ಜನಾ \\ ಧಾರಂ ಗೋತ್ರ ಪವಿತ್ರ ಸೌಖ್ಯನೆಸೆವಂ ಭೂಚಕ್ರದೊಳ್ಗಂಡಪೆಂ \\ ಡಾರಂ ಯಾದವ ಮಂತ್ರಿ ಸೋಮನಳಿಯಂ ಶ‍್ರೀಕೇತದಂಡಾಧಿಪಂ}
\end{verse}

\noindent
\textbf{ಅಗ್ರಹಾರಬಾಚಹಳ್ಳಿಯ ಮಹಾಸಾಮಂತ ಗಂಡನಾರಾಯಣಸೆಟ್ಟಿಯ ಶಾಸನದ ಈಶ್ವರ ಸ್ತುತಿ.}\endnote{ ಎಕ 6 ಕೃಪೇ 77 ಅಗ್ರಹಾರಬಾಚಹಳ್ಳಿ 1179}

\begin{verse}
\textbf{ಗಿರಿಜೆಯ ಕಣ್ನ ಬೆಳುಪಿನೊಳಾತ್ಮ ಸರೀರದ ಬೆಳುಹು \\ ಕುಂತಳೋತ್ಕರದ ಕರ್ಪಿನೊಡನುಣ್ಮುವ ಕೆಂಪಿನಳಪ್ಪುವ \\ ಕೆಂಜೆಡೆದೊಂಗಲ ಕೆಂಪು ಕೂಡೆ ತಳುತೆಸೆವವೀಶ್ವರ \\ ಕುಡುಗೆ ಹೊಯ್ಸಳ ಸೆಟ್ಟಿಗಭೀಷ್ಟ ಸಿದ್ಧಿಯಂ}
\end{verse}

\noindent
\textbf{ಕಸಲಗೆರೆ ಕಲ್ಲೇಶ್ವರ ದೇವಾಲಯದಲ್ಲಿರುವ ಅರಸಿಯಕೆರೆಯ ಪಟ್ಟಸಾಹಣಿ ಮಹದೇವಣ್ಣನ ಶಾಸನದ ಶಿವಸ್ತುತಿ.}\endnote{ ಎಕ 7 ನಾಮಂ 168 ಕಸಲಗೆರೆ 1190}

\begin{verse}
\textbf{ಶ‍್ರೀಮದ್ಭಾಳೇಂದು ಲೇಖಾವಳಯವಳೆಯಿತ ವ್ಯೋಮಗಂಗಾತರಂಗ \\ ಸ್ತೋಮೋಧ್ಯಾಮಾಭಿರಾಮಂ ಸ್ತುಳಕಪಿಳ ಜಟಾಜಾಳಕಂ ಕೀರ್ತಿಲಕ್ಷ್ಮೀ \\ ಧ್ವಾಮಂ ಭಕ್ತಿಬ್ರಜಕ್ಕಾಯುವವನ ವಿಚಳಿತಂ ಶ‍್ರೀಯುವಂ ಮಾಳ್ಕಧೀಶಂ\\ ಸೋಮೇಷಂ ಸ್ನಿಗ್ಧಂ ಗೌರೀಸ್ತನಕಳಸರುಚಿವ್ಯಾಪಿವಾಮಾರ್ಧದೇಹಂ}
\end{verse}

\newpage

\noindent
\textbf{ಬೆಳ್ಳೂರಿನ ಮಂಡಲಸ್ವಾಮಿ ಶಾಸನದಲ್ಲಿ ಅಪರೂಪದ ಶಿವಸ್ತುತಿಯ ಶ್ಲೋಕವಿದೆ.}\endnote{ ಎಕ 7 ನಾಮಂ 81 ಬೆಳ್ಳೂರು 1145}

\begin{verse}
\textbf{ಶ‍್ರೀ ವಿಶುದ್ಧ ಜ್ಞಾನದೇಹಾಯ ತ್ರಿವೇದೀ ದಿವ್ಯ ಚಕ್ಷುಷೇ\\ ಸ್ರೇಯಪ್ರಾಪ್ತ ನಿಮಿತ್ತಾಯ ನಮಸೋಮಾರ್ದ್ಧಧಾರಿಣೇ॥}
\end{verse}

\noindent
\textbf{ಬೆಳ್ಳೂರು ಶಾಸನದ ಪೆರುಮಾಳೆ ದೇವ ದಂಡನಾಯಕನ ವರ್ಣನೆ.}

\begin{verse}
\textbf{ಜವನಿಕೆಯೊಡಲಿರ್ವ್ವಲದ ವೀರಭಟಾವಳಿ ನೋಡೆ ಖಳ್ಗದಿಂ\\ ದವೆ ಕಲಿರತ್ನಪಾಲನ ಸಿರೋಂಬುಜಮಂ ಜಯಲಕ್ಷ್ಮಿಗಿತ್ತು \\ ತಜ್ಜವನಿಕೆಗೊಂಡಗಂಡ ಪೆರುಮಾಳೆಚಮೂಪತಿಗಿಂತು ಸಾರ್ದ್ದುದಾ \\ ಜವನಿಕೆ ನಾರಣಾಂಕವಿದು ರಾವುತರಾಯನುದಗ್ರದೊರ್ವ್ವಳಂ}
\end{verse}

\begin{verse}
\textbf{ಮದವದುಗ್ರವೈರಿಮದಮರ್ದ್ಧನ ವೀರನೃಸಿಂಹ ಭೂಬುಜಂ \\ ಗದಿರದೆ ಬಂದು ಸೇವುಣ ಮಹಾಮಹಿಪಂ ಮಹದೇವರಾಣೆಯಿಂ \\ ಕದನದೊಳಾಂತು ನಿತ್ತರಿಸಲಾರದೆ ಬಿಟ್ಟು ತುರುಗಮಂಗಳಂ \\ ಬೆದರೆ ಪಲಾಯನಂ ಕುಶಲಮೆಂದೋಡಿದನೊಂದೆ ರಾತ್ರಿಯೊಳ್​}
\end{verse}

\noindent
\textbf{ಶ‍್ರೀರಂಗ ಪಟ್ಟಣದ ಶಾಸನದಲ್ಲಿರುವ ರಂಗನಾಥನ ವರ್ಣನೆ ಇದೆ.}\endnote{ ಎಕ 6 ಶ‍್ರೀಪ 2 ಶ‍್ರೀರಂಗಪಟ್ಟಣ 1528}

\begin{verse}
\textbf{ಕಾವೇರಿ ವನಮಧ್ಯದೇಶೇ ವಿಲಸ್​ ಶ‍್ರೀರಂಗಪಟ್ಟಣಾಭಿಧೇ \\ ವೈಕುಂಠೇ ಮುನಿಗೌತಮಸ್ಯ ತಪಸಾ ಹೃಷ್ಟಃ ಪುರಾಣಃ ಪುಮಾನ್​ \\ ಶೇತೇ ಸರ್ವವಿಭೂಷಣೋ ಕಮಲಯಾ ಭೂಮ್ಯೇ \\ ಸಮಾರಾಧಿತಾಶೇಷೈರ್ಭೂಸುರಪುಂಗವಾ ವಿಕೃತಿಭಿಃ ಸಂಶೇವಿತಃ ಶಾಶ್ವತಂ}
\end{verse}

\noindent
\textbf{ಜಿಲ್ಲೆಯ ಅನೇಕ ಶಾಸನಗಳಲ್ಲಿ ಕಂಡು ಬರುವ ವರಾಹ ಅವತಾರದ ವರ್ಣನೆ.}\endnote{ ಎಕ 7 ನಾಮಂ 73 ಬೆಳ್ಳೂರು 1284, ನಾಮಂ 74 ಬೆಳ್ಳೂರು 1271

ಎಕ 7 ನಾಮಂ 76 ಬೆಳ್ಳೂರು 1284,}

\begin{verse}
\textbf{ಆದಿಕ್ರೋಡಂ ಧಾತ್ರಿಯ \\ ನಾದರದಿಂ ತಂನ ತೊಳೆಪ ದಂಷ್ಟ್ರಾಗ್ರದೊಳಂ \\ ದಾದಂ ನೆಗಪಿದನಂದಿಂ \\ ಮೇದಿನಿ ಸುಸ್ಥಿರತೆವೆತ್ತುದಾಚಂದ್ರಾರ್ಕ್ಕಂ}
\end{verse}

\noindent
\textbf{ವಿಷ್ಣುವು ಸಾಮಂತ ಕಾಚೀದೇವನಿಗೆ ಒಳ್ಳೆಯದನ್ನು ಮಾಡಲಿ ಎನ್ನುವ ಬೆಳ್ಳೂರು ಶಾಸನದ ವರ್ಣನೆ.}\endnote{ ಎಕ 7 ನಾಮಂ 81 ಬೆಳ್ಳೂರು 1145}

\begin{verse}
\textbf{ಶ‍್ರೀಯುಂ ಪಂಕರುಹೋದರಂ ಕಮಲಜಂ ದೀರ್ಘಾಯುವಂ ವಿಕ್ರಮ \\ ಶ‍್ರೀಯುಂ ನೀಲಗಳಂ ಮನೋಮುದದಿನೀವುತ್ತಿರ್ಕ್ಕೆ ಸಾಮಂತ ರಾ \\ ಧೇಯಂಗಗ್ಗದ ಮೂರುಲೋಕಜಗದಾಳಂ ಕಾಚಿದೇವಂಗೆ ಗಾಂ \\ ಗೇಯದ್ಭಾಸಿಗನೂನ ಪುಂಣ್ಯನಿಳಯಂಗಾಚಂದ್ರ ತಾರಾಂಬರಂ}
\end{verse}

\noindent
\textbf{ಮದ್ದೂರು ತಾಲ್ಲೂಕಿನ ಅರುವನಹಳ್ಳಿಯ ಶಾಸನದ ಭಟ್ಟರ ಬಾಚಿಯಪ್ಪ ಅಥವಾ ಬಾಚಿರಾಜನ ಸ್ತುತಿ.}\endnote{ ಎಕ 7 ಮ 87 ಅರುವನಹಳ್ಳಿ 1381}

\begin{verse}
\textbf{ಜನತಾಧಾರನುದಾರನನ್ಯ ವನಿತಾದೂರಂ ವಚಃಸುಂದರೀ\\ ಘನವೃತ್ತಸ್ತನಹಾರಶೂರನು ಸುಹೃತ್​ವಕ್ತ್ರಾಬ್ಜ ಮಾರ್ತ್ತಾಂಡನುಂ \\ ವನಜಾತಾಯತನೇತ್ರಪುಂಣ್ಯಕ್ರುತ ಗಾತ್ರಂ ನವ್ಯಚಾರಿತ್ರನುಂ \\ ವಿನುತಪ್ರಾಭವಕೀರ್ತ್ತಿರಾಜನ ಸುತಂ ಶ‍್ರೀಬಾಚಿರಾಜಾಹ್ವಯಂ}
\end{verse}

\noindent
\textbf{ಮೇಲುಕೋಟೆಯ ತಿಮ್ಮಣ್ಣ ದಂಡನಾಯಕನ ಶಾಸನದಲ್ಲಿ ದೇವತಾ ಸ್ತುತಿಯ ನಂತರ, ರಾಜನಾದ ಇಮ್ಮಡಿದೇವರಾಯ ಅಥವಾ ವಿರೂಪಾಕ್ಷ ಮತ್ತು ಮಂತ್ರಿ ತಿಮ್ಮಣ್ಣ ದಂಡನಾಯಕ ಮತ್ತು ಅವನ ಹೆಂಡತಿ ರಂಗಮಾಂಬೆಯ ವರ್ಣನೆ.}\endnote{ ಎಕ 6 ಪಾಂಪು 179 ಮೇಲುಕೋಟೆ 1458}

\begin{verse}
\textbf{ಯದುವಂಶಮಹಾಂಭೋದಿಚಂದ್ರಮಾಶ್ಚಂದ್ರಕೀರ್ತಿಮಾನ್​ \\ ಶ‍್ರೀಮಲ್ಲಿಕಾರ್ಜುನೋ ನಿತ್ಯಂ ಜೀಯಾದಾಚ್ಯಂದ್ರತಾರಕಂ॥}
\end{verse}

\begin{verse}
\textbf{ಶ‍್ರೀಮಂನ್​ ತಿಮ್ಮಣ್ಣದಂಡೇಶೋ ಲೋಹಿತಾನ್ವಯ ಶೇಖರಃ \\ ಜೀಯಾತ್​ ತಸ್ಯಾಪಿ ಮಹಿಷೀ ರಂಗಾಂಬಾ ಮಂಗಲಾತ್ಮಿಕಾ॥}
\end{verse}

\noindent
\textbf{ತೊಣ್ಣೂರು ಶಾಸನದಲ್ಲಿರುವ ರಣಧೀರ ಕಂಠೀರವನ ವರ್ಣನೆ}\endnote{ ಎಕ 6 ಪಾಂಪು 214 ತೊಣ್ಣೂರು 1647}

\begin{verse}
\textbf{ಜಯತು ಶ‍್ರೀಪತೇರ್ವಾಮನೇತ್ರವಂಶಾಬ್ಧಿ ಚಂದ್ರಮಾ \\ ಕಲಾನಿಧಿರುದಾರಶ‍್ರೀ ಕಂಠೀರವ ನೃಸಿಂಹರಾಟ್​॥ \\ ಶ‍್ರೀಮತ್ಪಶ್ಚಿಮರಂಗಪಟ್ಟಣವರೇ ಸಿಂಹಾಸನೇ ಸಂಸ್ಫುರನ್​ \\ ಮುಕ್ತಾ ಛತ್ರಶುವರ್ಣಮತ್ಸ್ಯಮಕರಾಕಾರಧ್ವಜೈಶ್ಚಿನ್ಹಿತ॥}
\end{verse}

\begin{center}
***
\end{center}

\theendnotes

