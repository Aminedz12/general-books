\chapter{ನೀರಾವರಿ ವ್ಯವಸ್ಥೆ}

ಮಂಡ್ಯ ಜಿಲ್ಲೆಯು ಹೇಮಾವತಿ,\index{ಹೇಮಾವತಿ} ಕಾವೇರಿ\index{ಕಾವೇರಿ} ನದಿ ನೀರಿನಿಂದ ಸಮೃದ್ಧವಾದ ನೀರಾವರಿ ಕ್ಷೇತ್ರವನ್ನು ಹೊಂದಿರುವ ಜಿಲ್ಲೆ ಎಂದು, ಭತ್ತ ಕಬ್ಬನ್ನು ಅಪಾರ ಪ್ರಮಾಣದಲ್ಲಿ ಬೆಳೆಯುವ ಜಿಲ್ಲೆ ಎಂದು, ಸಕ್ಕರೆಯ ನಾಡು ಎಂದು ಪ್ರಸಿದ್ಧವಾಗಿದೆ. ಆದರೆ ಪ್ರಾಚೀನ ಕಾಲದಿಂದಲೂ ಮಂಡ್ಯ ಜಿಲ್ಲಾ ಪ್ರದೇಶದಲ್ಲಿ ಆಳ್ವಿಕೆ ಮಾಡಿದ ರಾಜರು, ಅಧಿಕಾರಿಗಳು, ಈ ಪ್ರದೇಶದಲ್ಲಿದ್ದ ಜನಸಾಮಾನ್ಯರು ಕಾವೇರಿ, ಹೇಮಾವತಿ, ಲೋಕಪಾವನಿ, ವೀರವೈಷ್ಣವಿ, ಶಿಂಶಾ ನದಿಗಳಿಗೆ ಕಟ್ಟೆಕಾಲುವೆಗಳನ್ನು ನಿರ್ಮಿಸಿ, ನೂರಾರು ಕೆರೆಕಟ್ಟೆಗಳನ್ನು ಕಟ್ಟಿಸಿ, ನೀರಾವರಿ ಬೇಸಾಯಕ್ಕೆ ಅನುಕೂಲ ಮಾಡಿಕೊಂಡಿರುವುದು ಶಾಸನಗಳ ಅಧ್ಯಯನದಿಂದ ತಿಳಿದುಬರುತ್ತದೆ.

ನೀರಾವರಿ ವ್ಯವಸ್ಥೆಯ ಈ ಅಧ್ಯಾಯದಲ್ಲಿ, ಜಿಲ್ಲೆಯಲ್ಲಿರುವ ಶಾಸನೋಕ್ತ ಕೆರೆಗಳು, ಅಣೆಕಟ್ಟುಗಳು, ಕಾಲುವೆಗಳು ಇವುಗಳ ನಿರ್ಮಾಣದ ಹಿನ್ನೆಲೆ ಮತ್ತು ವೈವಿಧ್ಯತೆಗಳನ್ನು, ಶಾಸನೋಕ್ತ ನದಿಗಳು ಮತ್ತು ಹಳ್ಳಕೊಳ್ಳಗಳನ್ನೂ ಅಧ್ಯಯನ ಮಾಡಲಾಗಿದೆ.

\section*{ಕೆರೆಕಟ್ಟೆಗಳು ಹಾಗೂ ಅವುಗಳ ನಿರ್ಮಾಣದ ಹಿನ್ನೆಲೆ}

ಕೃಷಿಗೆ ಹಾಗೂ ದಿನನಿತ್ಯದ ಬಳಕೆಗೆ ನೀರು ಅಗತ್ಯವೆಂದು ಕಂಡುಕೊಂಡ ಮಾನವ, ಆ ನೀರನ್ನು ಹಿಡಿದಿಟ್ಟು ಅಗತ್ಯಕ್ಕೆ ತಕ್ಕಂತೆ ಉಪಯೋಗಿಸಿಕೊಳ್ಳುವುದನ್ನು ಕಲಿತುಕೊಂಡ. ಅದರ ಫಲವೇ ಕೆರೆ, ಕಟ್ಟೆ, ಕುಂಟೆ, ಬಾವಿಗಳು. “ನೀರಾವರಿಯು ಮುಖ್ಯವಾಗಿ ನದಿ, ಬಾವಿ, ಕೆರೆ,\index{ಕೆರೆ} ಕುಂಟೆ, ಕಟ್ಟೆಗಳನ್ನು ಅವಲಂಬಿಸಿದ್ದಿತು”.\endnote{ ರಾಜಾರಾಮ ಹೆಗ್ಗಡೆ, ಕೆರೆ ನೀರಾವರಿ ನಿರ್ವಹಣೆ \engfoot{–} ಚಾರಿತ್ರಿಕ ಅಧ್ಯಯನ, ಪುಟ 36} ಪ್ರತಿಯೊಂದು ಊರಿನಲ್ಲಿಯೂ ಒಂದು ಅಥವಾ ಅದಕ್ಕಿಂತ ಹೆಚ್ಚು ಕೆರೆಗಳಿರುತ್ತಿದ್ದವು.

ಮನುಷ್ಯನಾಗಿ ಹುಟ್ಟಿದ ಮೇಲೆ ಅವನು ಮಾಡಬೇಕಾದ ಸಾಮಾಜಿಕ ಕಾರ್ಯಗಳಲ್ಲಿ ಕೆರೆಯನ್ನು ಕಟ್ಟಿಸುವುದು ಪ್ರಮುಖವಾಗಿತ್ತು.\textbf{“ಅನ್ನಂ, ಸುವರ್ನ್ನ, ಕೆರೆಯುಂ ಸನ್ನುತ ಗೋದಾನ ಭೂಮಿದಾನ ಸಿವಾಲ್ಯ ಕನ್ಯಾದಾನಂಗಳನತ್ಯೋನ್ನದಿಂ ಮಾಡಿಸಿದಂ” }ಎಂದು ಹುಬ್ಬನಹಳ್ಳಿ ಶಾಸನದಲ್ಲಿ ಪಟ್ಟಿಮಾಡಿದೆ.\endnote{ ಎಕ 6 ಕೃಪೇ 62 ಹುಬ್ಬನಹಳ್ಳಿ 1140} ಇದೇ ವಿಚಾರವು ಜಾಗನಕೆರೆ ಶಾಸನದಲ್ಲೂ ಇದೆ. \textbf{“ಅನ್ನಸುವರ್ನ್ನವುದಕಂ ಸಂನ್ನುತ ಗೋದಾನ ಭೂಮಿದಾನ ಸಮೇತಂ ಸಂನ್ನುತ ಸಿವಾಲ್ಯಂಗಳನತ್ಯುಂನದಿಂ ಮಾಳ್ಪ”} ಎಂದು ಹೇಳಿದೆ.\endnote{ ಎಕ 6 ಕೃಪೇ 69 ಜಾಗನಕೆರೆ 1242} ಇಲ್ಲಿ ಉದಕ ಎಂದರೆ ಕೆರೆ ಎಂದು ಅರ್ಥ. ಲಕ್ಷ್ಮೀಧರ ಅಮಾತ್ಯನ ಶಾಸನದಲ್ಲಿ \textbf{“ಕೆರೆಯಂ ಕಟ್ಟಿಸು, ಬಾವಿಯಂ ಸವೆಸು, ದೇವಾಗಾರಮಂ ಮಾಡಿಸು”} ಎನ್ನುವ ಸಾಲೂ ಇದೇ ಆಶಯವನ್ನು ವ್ಯಕ್ತಪಡಿಸುತ್ತದೆ. ಒಂದು ಮುಖ್ಯ ಸಾಮಾಜಿಕ ಕೆಲಸಕ್ಕೆ ಧಾರ್ಮಿಕ ಪರಿವೇಷವನ್ನು ತೊಡಿಸಿ, ಕೆರೆಗಳ ನಿರ್ಮಾಣ ನಿರ್ವಹಣೆಯು ಒಂದು ಪೂಜ್ಯವಾದ ಧಾರ್ಮಿಕ ಕಾರ್ಯ ಎನ್ನುವಂತಹ ಹಿನ್ನೆಲೆಯನ್ನು ಕಲ್ಪಿಸಲಾಗಿತ್ತು. \textbf{“ಬಸದಿ ಕೆರೆ ದೇಗಲು ಮಳಿಗೆ ಸುರಾಸುರಯುದ್ಧ ಕಥೆಯಿವಂ ಮುದುವೊಳಲೊಳ್​ ಪೊಸತಾಗೆ ನಿರ್ಮ್ಮಿಸಿ ಪಡೆದಂ ಜಸದನೆರವನೆಳಗೆರೆಗಾಂಕಂ” ಎಂದು ಹಟ್ಟಣ }ಶಾಸನವು ಸೋವಿಸೆಟ್ಟಿಯನ್ನೂ ವರ್ಣಿಸಿದೆ.\endnote{ ಎಕ 7 ನಾಮಂ 118 ಹಟ್ಟಣ 1178} ಎರಡನೆಯ ವೀರಬಲ್ಲಾಳನ ಮಂತ್ರಿ ವೀರದೇವನು ಕಾಡನ್ನು ಕಡಿದು ವೀರಬಲ್ಲಾಳಪುರವನ್ನೂ ನಿರ್ಮಿಸಿ ಆ ಊರಿನಲ್ಲಿ ವೀರನಾರಾಯಣ ದೇವಾಲಯವನ್ನು ಹಾಗೂ ರುದ್ರಸಮುದ್ರ,\index{ರುದ್ರಸಮುದ್ರ} ಗಂಗಾಸಮುದ್ರ, ಅಚ್ಯುತಸಮುದ್ರ ಮತ್ತು ವೀರಸಮುದ್ರವೆಂಬ ಕೆರೆಗಳನ್ನು ಕಟ್ಟಿಸಿದನು. ಕೆರೆಗಳಲ್ಲಿ ನೀರು ಸ್ವಚ್ಛವಾಗಿರಬೇಕೆಂಬ ದೃಷ್ಟಿಯಿಂದ ಆಮೆಯನ್ನು, ಮೀನನ್ನೂ ಬಿಟ್ಟು, ತಾವರೆಯ ಬಳ್ಳಿಗಳನ್ನೂ ಬೆಳೆಸುತ್ತಿದ್ದರೆಂಬ ಅಂಶ ವೀರದೇವನಪುರ ಶಾಸನದಿಂದ ತಿಳಿದುಬರುತ್ತದೆ.\textbf{ “ಕಟ್ಟಿಸಿದಂ ತತ್ಕ್ರಮಮಮರ್ದೆಸೆವಂತು ಕಂನೆಗೆರೆಯ ವೀರಂ}" ಎಂದೂ ಈ ಶಾಸನದಲ್ಲಿ ಹೇಳಿದೆ.\endnote{ ಎಕ ಬೇಲೂರು} ಕಂನೆಗೆರೆ\index{ಕಂನೆಗೆರೆ} ಎಂದರೆ ಹೊಸದಾಗಿ ನಿರ್ಮಿಸಿದ ಕೆರೆ. ಕೆರೆಗಳನ್ನು ಹೊಸದಾಗಿ ಕಟ್ಟಿಸಿದಾಗ ಶಾಸನಗಳಲ್ಲಿ ‘ಕಂನೆಗೆರೆ’ ಎಂದು, ಅಥವಾ ಕೆರೆಯನ್ನು ಕಟ್ಟಿಸಿದರೆಂದೂ ಸ್ಪಷ್ಟವಾಗಿ ಹೇಳಲಾಗಿರುತ್ತದೆ.

ಕೆರೆಯನ್ನು ಹಾಳು ಮಾಡುವುದು ಪಂಚಮಹಾಪಾತಕಗಳಲ್ಲಿ\index{ಪಂಚಮಹಾಪಾತಕ} ಒಂದೆಂದು ತಿಳಿದುಬರುತ್ತದೆ. ಪಾಂಡವಪುರ ಕ್ಯಾತನಹಳ್ಳಿ ಶಾಸನದಲ್ಲಿ ಬಸದಿಗೆ ಬಿಟ್ಟ ದತ್ತಿಯ ರಕ್ಷಣೆಯ ಬಗ್ಗೆ ಹೇಳುವಾಗ \textbf{“ಇದನಳಿದುಣ್ಡೋನ್​ ಕೊಣ್ಡೋನ್​ ಪಸುವುಂ, ಪಾರ್ವರುಂ, ಕೆರೆಯುಂ, ಅರಮೆಯುಂ ಬಾರಣಾಸಿಯಮವನಳಿದೋಂ ಪಞ್ಚಮಹಾಪಾತಕಂ”} ಎಂದು ಹೇಳಿದೆ.\endnote{ ಎಕ 6 ಪಾಂಪು 16 ಕ್ಯಾತನಹಳ್ಳಿ 9-10ನೇ ಶ.} ಕೂಲಿಗ್ಗೆರೆ ಶಾಸನದಲ್ಲಿ \textbf{“ಇದನಳಿದೋಂ ಕೆರೆಯುಮಾರವೆಯುಮನಳಿದು ಕೋಣ್ಡೋಮ್ಮಹಾಪಾತಕನ್​”} ಎಂದು ಹೇಳಿದೆ.\endnote{ ಎಕ 7 ಮ 100 ಕೂಲಿಗ್ಗೆರೆ 916} ನಮ್ಮ ಪ್ರಾಚೀನರು ಕೆರೆಯ ವಿಚಾರಕ್ಕೆ ಅತ್ಯಂತ ಮಹತ್ವವನ್ನು ಕೊಟ್ಟಿದ್ದು ಇದರಿಂದ ತಿಳಿದುಬರುತ್ತದೆ. ವಿಜಯನಗರ ಕಾಲದ ಕೆಲವು ಶಾಸನಗಳಲ್ಲಿ ಕೆರೆಗಳ ನಾಶಕೃತ್ಯವನ್ನು ಶಿಶುಹತ್ಯೆ, ಗೋಹತ್ಯೆ ಹಾಗೂ ಬ್ರಾಹ್ಮಣ ಹತ್ಯೆಗಳೊಡನೆ ಸಮೀಕರಿಸಲಾಗಿದೆ. ದತ್ತಿಯನ್ನು ಪಾಲಿಸಿದವರಿಗೆ \textbf{“ಕೆರೆದೇವಾಲಯ ಅರವೆಗಳ ಕಟ್ಟಿಸಿದ ಫಲ, ಇದನ್ನು ಅಳಿಪಿದವರು ಕೆರೆದೇವಾಲಯ ಅರವೆಗಳ ಕೆಡಿಸಿದ ಪಾತಕದಲು ಹೋಹರು, ದತ್ತಿಯನ್ನು ಅಳಿದವರು ಕೆರೆದೇವಾಲ್ಯವನಳಿದ ದೋಷದಲು ಹೋಹರು, ಶಿಶುವಧೆ ಗೋವಧೆ ಮುಂತಾದ ದೋಷದಲು ಹೋಹರು, ಈ ಧರ್ಮಕ್ಕೆ ಆರೊಬ್ಬರು ಅಳಿಪಿದರೆ ವಾರಣಾಸಿಯಲು ಸಹಸ್ರ ಬ್ರಾಹ್ಮಣರ, ಸಹಸ್ರ ಕಪಿಲೆಯ ವಧಿಸಿ, ಕೆರೆ ದೇವಾಲ್ಯವನು ಕೆಡಿಸಿದ ಪಾತಕದಲಿ ಹೋಹರು”} ಎಂದು ಶಾಸನಗಳು ಹೇಳುತ್ತವೆ.\endnote{ ಎಕ 13 ತೀರ್ಥಹಳ್ಳಿ 9, 18, 244 - 15ನೇ ಶ.}

ನೀರಾರಂಭಕ್ಕೆ ಎಂದರೆ ನೀರಾವರಿ ಬೇಸಾಯಕ್ಕೆ, ಊರಿನ ಜನ ಜಾನುವಾರುಗಳಿಗೆ ಅಗತ್ಯವಾದ ದಿನಬಳಕೆಯ ನೀರಿಗೋಸ್ಕರ ಕೆರೆಗಳನ್ನು ನಿರ್ಮಿಸುತ್ತಿದ್ದರು. ಮುಖ್ಯವಾಗಿ ರಾಜರೇ ಕೆರೆಗಳನ್ನು ಕಟ್ಟಿಸುತ್ತಿದ್ದರು, ನಂತರ ಮಹಾಪ್ರಧಾನ ದಂಡನಾಯಕರು, ಹಡವಳರು, ಹೆಗ್ಗಡೆಗಳು, ವರ್ತಕರು, ಗಾವುಂಡರುಗಳು, ಸ್ತ್ರೀಯರೂ ಸೇರಿದಂತೆ ಜನಸಾಮಾನ್ಯರುಗಳು, ತಪ್ಪದೆ ಕೆರೆ\index{ಕೆರೆ} ದೇಗುಲಗಳನ್ನು ಕಟ್ಟಿಸುತ್ತಿದ್ದರು. \textbf{“ಎನಿತಾನುಂ ಕೆರೆ ದೇಗುಲಗಳೆನಿತಾನುಂ ಜೈನಗೇಹಗಳಂತೆನೆತುಂ ಸಂತೋಷದಿಂ ಮಾಡಿದಂ ವಿನಯಾದಿತ್ಯ ನ್ರಿಪಾಳಂ”} ಎಂದು ಶ್ರವಣಬೆಳಗೊಳ ಶಾಸನ ವಿವರಿಸುತ್ತದೆ.\endnote{ ಎಕ 2 ಶ್ರಬೆ 176 ಚಿಕ್ಕಬೆಟ್ಟ 1123}

ಹೊಸದಾಗಿ ಕೆರೆಯನ್ನು ನಿರ್ಮಿಸಿದ ಉಲ್ಲೇಖಗಳು ಅನೇಕ ಶಾಸನಗಳಲ್ಲಿವೆ. ಇವುಗಳನ್ನು ಕನ್ನೆಗೆರೆ, ಹೊಸಕೆರೆ ಎಂದು ಕರೆಯಲಾಗಿದೆ.\endnote{ ಎಕ 7 ನಾಮಂ 132 ದೊಡ್ಡಜಟಕ 1179} ಈ ರೀತಿ ಹೊಸದಾಗಿ ಕೆರೆಗಳನ್ನು ನಿರ್ಮಿಸಿದವರನ್ನು \textbf{“ಕನ್ನೆಗೆರೆ ಮಲ್ಲಂ”}\index{ಕನ್ನೆಗೆರೆ ಮಲ್ಲ} ಎಂದು ಶಾಸನಗಳು ವರ್ಣಿಸಿವೆ.\endnote{ ಎಕ 7 ನಾಮಂ 132 ದೊಡ್ಡ ಜಟಕ 1179} ಕೆಲವರನ್ನು \textbf{‘ಒಡಗೆರೆ ಮಲ್ಲ’}\index{ಒಡಗೆರೆ ಮಲ್ಲ}\endnote{ ಎಕ 6 ಕೃಪೇ 78 ಹುಬ್ಬನಹಳ್ಳಿ 1140, ಎಕ 7 ನಾಮಂ 129 ದೊಡ್ಡ ಜಟಕ} ಎಂದು ಕರೆದಿದ್ದು, ಇವರು ಒಡಗೆರೆ ಊರಿನವರೇ, ಅಥವಾ ಒಡೆದುಹೋದ ಕೆರೆಗಳನ್ನು ನಿರ್ಮಿಸುವುದರಲ್ಲಿ ಮುಂದಿದ್ದರೆ ಎಂಬುದು ತಿಳಿದುಬರುವುದಿಲ್ಲ. ಕಾರಣ ಜಿಲ್ಲೆಯ ಕೆಲವು ಶಾಸನಗಳಲ್ಲಿ ಒಡೆದು ಹೋದ ಕೆರೆಗಳನ್ನು ಕಟ್ಟಿಸಿದ್ದ ಉಲ್ಲೇಖವಿದೆ. ಕೆರೆಗಳನ್ನು ನಿರ್ಮಿಸುವುದರ ಜೊತೆಗೆ ಅವುಗಳ ಜೀರ್ಣೋದ್ಧಾರಕ್ಕೆ ಕೆರೆಯ ತೂಬಿನ ಬಳಿಯ ಬೀಜವರಿ ಗದ್ದೆ ಅಥವಾ ಬಿತ್ತುವಾಟವನ್ನು ಅಥವಾ ಬೇರೆ ಕಡೆ ಗದ್ದೆ ಬೆದ್ದಲುಗಳನ್ನು ಅಥವಾ ಆ ಊರಿನ ಕೆಲವು ತೆರಿಗೆಗಳನ್ನು ದತ್ತಿಯಾಗಿ ಬಿಡುತ್ತಿದ್ದರು. ಸಾಮಾನ್ಯವಾಗಿ ದೇವಾಲಯಗಳನ್ನು ಕಟ್ಟಿಸಿದಾಗ ಅದರ ಜೊತೆಗೆ ಕೆರೆಗಳನ್ನು ನಿರ್ಮಿಸಿರುವುದು ಕಂಡು ಬರುತ್ತದೆ.

ಹಳ್ಳಗಳಿಗೆ ಅಡ್ಡಲಾಗಿ ಕೆರೆಗಳನ್ನು ಕಟ್ಟುತ್ತಿದ್ದರು. ಹಳ್ಳಗಳ ನೀರು ಕೆರೆಗೆ ಹರಿದು ಬಂದು ಕೆರೆಯನ್ನು ತುಂಬಿಸುತ್ತಿದ್ದವು. \textbf{“ಕೆರೆಯೊಳ್ಕೂಡುವ ಪೆರ್ವ್ವಳ್ಳಂ”}, ಎಂದು ಹುಳ್ಳೇನಹಳ್ಳಿ ಶಾಸನದಲ್ಲಿ,\endnote{ ಎಕ 7 ಮಂ 14 ಹುಳ್ಳೇನಹಳ್ಳಿ 8ನೇ ಶ (ಸು.750)}\textbf{“ಸೆಟ್ಟಿಯಹಳ್ಳಿಯ ಕೆರೆಗಿಳಿದ ನಾಯಿಹಳ್ಳ”},\index{ನಾಯಿಹಳ್ಳ} ಎಂದು ಬೋಗಾದಿ ಶಾಸನದಲ್ಲಿ ಹೇಳಿದೆ.\endnote{ ಎಕ 7 ನಾಮಂ 183 ಬೋಗಾದಿ 1144}\textbf{‘ಹೆಬ್ಬಳ್ಳವು\index{ಹೆಬ್ಬಳ್ಳ} ಬೇಡಹರಳ್ಳಿ ವೊಳಗೆರೆಯ ಕಗ್ಗಲ್ಲ ಗುಂಡಿನಲ್ಲಿ ಕೂಡಿತ್ತು’} ಎಂದು ವೈದ್ಯನಾಥಪುರ ಶಾಸನದಲ್ಲಿ ಹೇಳಿದೆ.\endnote{ ಎಕ 7 ಮ 69 ವೈದ್ಯನಾಥಪುರ 1261} ಹಳ್ಳವು ಕೆರೆಗೆ ಬಂದು ಸೇರುತ್ತಿದ್ದುದನ್ನು ಈ ಉದಾಹರಣೆಗಳು ಖಚಿತಪಡಿಸುತ್ತವೆ. \textbf{“ಹಳ್ಳವಾದುವಂ ಕೆರೆಯಾಗಿ ಕಟ್ಟಿಸಿದನುದ್ಯೋಗ ಮಲ್ಲಂ”} ಎಂದು ದೊಡ್ಡಜಟಕ ಶಾಸನವು ವರ್ಣಿಸಿದೆ.\endnote{ ಎಕ 7 ನಾಮಂ 132 ದೊಡ್ಡಜಟಕ 1179} ಹಳ್ಳವಾಗಿರುವ ಭೂಮಿಯಲ್ಲಿ ಹಾಗೂ ಹಳ್ಳಗಳಿಗೆ ಅಡ್ಡಲಾಗಿ ಕೆರೆಗಳನ್ನು ನಿಮಿಸುತ್ತಿದ್ದರು ಎಂದು ಇದರಿಂದ ತಿಳಿದುಬರುತ್ತದೆ. \textbf{ಮಲ್ಲಿಗೆದುಡುಪಿನ ಮಾವಿನಹಳ್ಳವು,\index{ಮಾವಿನಹಳ್ಳ} ನವಿಲಹಳ್ಳವು\index{ನವಿಲಹಳ್ಳ} ಕೂಡುವಲ್ಲಿ ಒಂದು ಕೆರೆಯನ್ನು ಕಟ್ಟಲಾಯಿತೆಂದು} ನಾಗಮಂಗಲ ತಾಲ್ಲೂಕಿನ ದೊಂದೆ ಮಾದಿಹಳ್ಳಿ ಶಾಸನದಲ್ಲಿ ಹೇಳಿದೆ.\endnote{ ಎಕ 7 ನಾಮಂ 151 ದೊಂದೆಮಾದಿಹಳ್ಳಿ 1521}\textbf{“ತಟ್ಟೆಹಳ್ಳದ ಕನ್ನೆಗೆರೆ ಮಸಣಸಮುದ್ರದಲು”}\index{ಕನ್ನೆಗೆರೆ ಮಸಣಸಮುದ್ರ}\endnote{ ಎಕ 7 ನಾಮಂ 130 ದೊಡ್ಡಜಟಕ 1179}\textbf{“ಇಟ್ಟಿಗೆಗಂದಗಳ್ದ ಕುಳಿಗಳೇ ಕೆರೆಯಾದವು”} ಎಂದು ಹೇಳಿದ್ದು, ಭೂಮಿಯನ್ನು ಅಗೆದು ಆಳಮಾಡಿ(ದೊಡ್ಡ ದೊಡ್ಡ ಕುಳಿಗಳು) ಕೆರೆಗಳನ್ನು ನಿರ್ಮಿಸುತ್ತಿದ್ದರು ಎಂಬುದನ್ನು ಸೂಚಿಸುತ್ತದೆ.\endnote{ ಎಕ 2 ಶ್ರಬೆ 176 ಚಿಕ್ಕಬೆಟ್ಟ 1123}

\section*{ಕೆರೆಗಳ ವರ್ಗೀಕರಣ – ಹಿರಿಕೆರೆ – ಕೆರೆ – ಕಿರುಕೆರೆ – ಕಟ್ಟೆ – ಕುಂಟೆ – ಕಲ್ಲಣೆ}
\index{ಹಿರಿಕೆರೆ}\index{ಕೆರೆ}\index{ಕಿರುಕೆರೆ}\index{ಕಟ್ಟೆ}\index{ಕುಂಟೆ}

ಕೆರೆಗಳನ್ನು ಅವುಗಳಲ್ಲಿ ತುಂಬುವ ನೀರಿನ ಸಾಮರ್ಥ್ಯ, ನೀರಾವರಿ ಸಾಮರ್ಥ್ಯ, ಅವುಗಳ ಉದ್ದ, ಅಗಲ, ಆಳ, ಉಪಯೋಗ ಇವುಗಳನ್ನು ಅವಲಂಬಿಸಿ ವಿಂಗಡಿಸಬಹುದು. ಶಾಸನಗಳಲ್ಲಿ ಸಾಮಾನ್ಯವಾಗಿ ಹಿರಿಯಕೆರೆ, ಸಾಗರ, (ಸಮುದ್ರ, ಸಂದ್ರ, ಅಂಬುಧಿ), ಕಿರುಕೆರೆ, ಕಟ್ಟೆ, ಎಂದು ನಾಲ್ಕು ಮುಖ್ಯ ವಿಧಗಳನ್ನು ಕಾಣಬಹುದು. ಹಿರಿಯಕೆರೆಯನ್ನು ಪೆರ್ಗ್ಗೆರೆ ಎಂದೂ ಕರೆಯಲಾಗಿದೆ. ಕಲ್ಯಾಣದ ಚಾಲುಕ್ಯರ ಶಾಸನಗಳಲ್ಲಿ ತಟಾಕ ಎಂಬ ಪದ ಪ್ರಯೋಗವಿದೆ. ಹೊಯ್ಸಳರು ಮತ್ತು ವಿಜಯನಗರ ಶಾಸನಗಳಲ್ಲಿ ಸಮುದ್ರ ಎಂಬ ಪದದ ಬಳಕೆ ಜಾಸ್ತಿ ಇದೆ. ಇವುಗಳ ಬಗ್ಗೆ ಖಚಿತವಾದ ವಿವರಣೆ ಶಾಸನಗಳಲ್ಲಿ ಸಿಗುವುದಿಲ್ಲ. “ಬಖೈರುಗಳಲ್ಲಿ ಕೆರೆ, ಕಟ್ಟೆ, ಕುಂಟೆಗಳ ಪಟ್ಟಿ ಇದೆ. ಫ್ರಾನ್ಸಿಸ್​ ಬುಖನಾನ್​ ದನಕರುಗಳಿಗೆ ಕುಡಿಯಲು ನೀರು ಒದಗಿಸುವದೇ ಕಟ್ಟೆ ಎಂದು, ಭೂಮಿಗೆ ನೀರು ಹಾಯಿಸಲು ಉಪಯೋಗಿಸುವ ದೊಡ್ಡ ಜಲಾಶಯವೇ ಕೆರೆ” ಎಂದು ಹೇಳುತ್ತಾನೆ.\endnote{ ದೀಕ್ಷಿತ್​, ಜಿ.ಎಸ್​., ಕರ್ನಾಟಕದಲ್ಲಿ ಕೆರೆ ನೀರಾವರಿ, ಪುಟ 11-12} “ಕೆರೆ ಮತ್ತು ಕಟ್ಟೆ ಎಂಬ ಎರಡು ಬಗೆಯ ಜಲಾಶಯಗಳಿದ್ದವು, ಸಣ್ಣ ಜಲಾಶಯವು ಕಟ್ಟೆ, ದೊಡ್ಡದು ಕೆರೆ”.\endnote{ ಕೃಷಿ ವ್ಯವಸ್ಥೆ, ಪ್ರೊಃ ಕೆ.ಎಸ್​. ಶಿವಣ್ಣ, ಕರ್ನಾಟಕ ಚರಿತ್ರೆ, ಪುಟ 158} ಒಂದೇ ಶಾಸನದಲ್ಲಿ ಕೆರೆ, ಕಿರುಕೆರೆ, ಕಟ್ಟೆ, ಸಮುದ್ರ ಈ ಪದಗಳನ್ನು ಉಪಯೋಗಿಸಿರುವುದರಿಂದ ಇವು ಬೇರೆ ಬೇರೆಯೇ ಸ್ವರೂಪದ ಜಲಾಶಯಗಳು ಎಂದು ಹೇಳಬಹುದು. ಅರಿಕನಕಟ್ಟೆ, ಕಣಿಗಿಲೆಯಕಟ್ಟೆ, ಮತ್ತಿಯಕೆರೆ, ಚೊಕಚಾರ್ಯ್ಯಕಟ್ಟೆ, ಚೆಳೆಯಕಟ್ಟೆ, ಮೊದಲಿಹಳ್ಳಿಯಕೆರೆ”,\endnote{ ಎಕ 7 ನಾಮಂ 1 ನಾಗಮಂಗಲ 1173} “ಮರಿಯಾನೆಸಮುದ್ರದ ಬಯಲು, ಮಳೆಹಳ್ಳಿಯ ಮುಂದಣ ಕಿರುಕೆರೆ, ಕೋಡಿಹಳ್ಳಿಯ ಮುಂದಣ ಕಿರುಕೆರೆ, ಹಿರಿಯಕೆರೆಯ ಕೆಳಗಣ ಅಡಕೆಯತೋಟ”,\endnote{ ಎಕ 7 ನಾಮಂ 68 ದಡಗ 13ನೇ ಶ.}, “ಹಡುವಳಿತಿಯ ಕೆರೆಯ\index{ಹಡುವಳಿತಿಯ ಕೆರೆ} ಪಡುವಣಕೋಡಿ ಬಡಗಲು ಕಿರುಕೆರೆಯೊಳಗಣ ನೇರಲು”,\endnote{ ಎಕ 7 ನಾಮಂ 72 ಅಳೀಸಂದ್ರ 1183} “ರಂಗಸಮುದ್ರ,\index{ರಂಗಸಮುದ್ರ} ನಾಗನಾಗನಕಟ್ಟೆ,\index{ನಾಗನಾಗನಕಟ್ಟೆ} ಜೋಗಿಯಕಟ್ಟೆ,\index{ಜೋಗಿಯಕಟ್ಟೆ} ಲೋಕಪಾವನೆಯಸಾಗರ”,\index{ಲೋಕಪಾವನೆಯಸಾಗರ}\endnote{ ಎಕ 6 ಶ‍್ರೀಪ 93 ನೆಲಮನೆ 1458}ಎಂದು ಹೇಳಿದ್ದು, ಜಿಲ್ಲೆಯ ಶಾಸನಗಳನ್ನು ಪರಿಶೀಲಿಸಿದಾಗ ಈ ರೀತಿ ಕೆರೆ, ಕಟ್ಟೆ, ಸಮುದ್ರ, ಕಿರುಕೆರೆ ಎಂದು ಒಂದೇ ಶಾಸನದಲ್ಲಿ ಬೇರೆಬೇರೆಯಾಗಿ ಹೇಳಿದೆ. ಜಿಲ್ಲೆಯ ಕೆಲವು ಶಾಸನಗಳಲ್ಲಿ ಕಟ್ಟೆಗಳ ಉಲ್ಲೇಖ ಜಾಸ್ತಿ ಇದೆ. ದೊಡ್ಡ ಕೆರೆಗಳಷ್ಟೇ, ಈ ಸಣ್ಣ ಕಟ್ಟೆಗಳಿಗೂ ಪ್ರಾಮುಖ್ಯತೆ ಇತ್ತು ಎಂಬುದು ಇದರಿಂದ ತಿಳಿದುಬರುತ್ತದೆ. ಒಡೆಯರ ಕಾಲದ ಹೊನ್ನಲಗೆರೆ ಶಾಸನದಲ್ಲಿ ಕೊತ್ತಿಬೈಚನಕಟ್ಟೆ,\index{ಕೊತ್ತಿಬೈಚನಕಟ್ಟೆ} ನವುಲೆಸೊಣ್ನನಕಟ್ಟೆ, ಹಾಳಬಸ್ತಿಕಟ್ಟೆ, ಪ್ರಭುದೇವರ ಕಟ್ಟೆ, ಚಿಟ್ಟಿಗವುಡಿಯ ಕಟ್ಟೆ, ದಾಸಗವುಡನ ಕಟ್ಟೆಗಳ\index{ದಾಸಗವುಡನ ಕಟ್ಟೆ} ಉಲ್ಲೇಖವಿದೆ.\endnote{ ಎಕ 7 ಮ 64 ಹೊನ್ನಲಗೆರೆ 1623} ತೊಣ್ಣೂರು ಶಾಸನದಲ್ಲಿ, ಕುಡಿನೀರುಕಟ್ಟೆ, ಹನುಮನಕಟ್ಟೆ, ಮುದೇಗೌಡನಕಟ್ಟೆ, ತೊರೆಯಬಳಿಯ ಕಲ್ಲಣೆ,\index{ಕಲ್ಲಣೆ} ರಿಕಂರಾಜನಕಟ್ಟೆ, ಗಿರೆಗೌಡನಕಟ್ಟೆ, ರಾಮದೇವರಕಟ್ಟೆ, ಬೋಳನಕಟ್ಟೆ, ಬೇವಿನಮರದ ಕಟ್ಟೆ, ಲಕ್ಕಿಕಟ್ಟೆ,\index{ಲಕ್ಕಿಕಟ್ಟೆ} ವಡೆಯರಕಟ್ಟೆ, ಮಾಯಿಗನಕಟ್ಟೆಗಳ ಉಲ್ಲೇಖವಿದ್ದು ಜಲಸಂಗ್ರಹದಲ್ಲಿ ಕಟ್ಟೆಗಳಿದ್ದ ಮಹತ್ವವನ್ನು ಸೂಚಿಸುತ್ತವೆ.\endnote{ ಎಕ 6 ಪಾಂಪು 99 ತೊಣ್ಣೂರು 1722}

ಸಾಮಾನ್ಯವಾಗಿ ಹಿರಿಯಕೆರೆ ಅಥವಾ ದೊಡ್ಡದಾದ ಕೆರೆಗಳನ್ನು ಕೃಷಿಗೂ, ಕಿರುಕೆರೆ ಅಥವಾ ಚಿಕ್ಕಕೆರೆಗಳನ್ನು ದೇವಾಲಯಗಳ ಉಪಯೋಗಕ್ಕೆ ಮತ್ತು ಜನರ ಕುಡಿಯುವ ನೀರಿಗೆ, ಕಟ್ಟೆಗಳ ನೀರನ್ನು ವ್ಯವಸಾಯ ಹಾಗೂ ಜನ ಜಾನುವಾರು ಬಳಕೆಯ ದಿನನಿತ್ಯದ ನೀರಿಗೆ ಉಪಯೋಗಿಸುತ್ತಿದ್ದರೆಂದು ತೋರುತ್ತದೆ. ನಾಗಮಂಗಲ ತಾಲ್ಲೂಕಿನ ಸುಖಧರೆಯಲ್ಲಿ ಬಸದಿಯನ್ನು ಕಟ್ಟಿಸಿ, ಒಂದು ಹಿರಿಯ ಕೆರೆಯನ್ನು, ಎರಡು ಕಿರುಕೆರೆಗಳನ್ನೂ ನಿರ್ಮಿಸಲಾಯಿತೆಂದು ಹೇಳಿದೆ.\endnote{ ಎಕ 7 ನಾಮಂ 14 ಸುಖಧರೆ 12ನೇ ಶ.} ಮಳವಳ್ಳಿಯಲ್ಲಿ ಇಂದಿಗೂ ಒಂದು ಸಣ್ಣ ಕಟ್ಟೆಯನ್ನು ಕುಡಿನೀರುಕಟ್ಟೆ\index{ಕುಡಿನೀರುಕಟ್ಟೆ} ಎಂದು ಕರೆಯುತ್ತಾರೆ. ಇದೇ ಮಳವಳ್ಳಿಯಲ್ಲಿ ಚಿಕ್ಕದೇವರಾಜ ಒಡೆಯನು ಜನಗಳ ಉಪಯೋಗಕ್ಕಾಗಿ ಸುಂದರವಾದ ದೊಡ್ಡದಾದ ಕೊಳವನ್ನು ನಿರ್ಮಿಸಿದನೆಂದು ಹೇಳಿದೆ.\endnote{ ಎಕ 7 ಮವ 2 ಮಳವಳ್ಳಿ 1685} ಇದನ್ನು ಈಗ ಶೃಂಗಾರ ಕೊಳ ಎಂದು ಹೇಳುತ್ತಾರೆ. ಮೊದಲಿಗೆ ಈ ಕೊಳದಿಂದ ದೇವರ ಪೂಜೆಗೆ ಮತ್ತು ಕುಡಿಯಲು ನೀರನ್ನು ತೆಗೆದುಕೊಂಡು ಹೋಗಲಾಗುತ್ತಿತ್ತೆಂದು ತಿಳಿದುಬರುತ್ತದೆ. ಬೆಟ್ಟದ ಚಂದ್ರಪ್ರಭ ಸ್ವಾಮಿಯ ಅರ್ಚನೆಗೆ ಸಮರ್ಪಿಸಿದ ಕೋಮಲಕಟ್ಟೆಯ\index{ಕೋಮಲಕಟ್ಟೆ} ವಿಚಾರ ಕಂಬದಹಳ್ಳಿ ಶಾಸನದಿಂದ ತಿಳಿದುಬರುತ್ತದೆ.\endnote{ ಎಕ 7 ನಾಮಂ 37 ಕಂಬದಹಳ್ಳಿ 17-18ನೇ ಶ.} ಆದುದರಿಂದ ದೇವರ ಪೂಜೆಯ ನೀರಿಗೆ, ಕಿರುಕೆರೆ ಅಥವಾ ಕಟ್ಟೆಗಳನ್ನು ಪ್ರತ್ಯೇಕವಾಗಿ ಕಟ್ಟಿಸುತ್ತಿದ್ದರೆಂದು ಹೇಳಬಹುದು.

“ದೊಡ್ಡದೊಡ್ಡ ಕೆರೆಗಳಿಗೆ ಹೂಳು ತುಂಬಬಾರದೆಂಬ ಉದ್ದೇಶದಿಂದ, ದೊಡ್ಡ ದೊಡ್ಡ ಕೆರೆಗಳ ಹಿಂದೆ ಕಟ್ಟೆಗಳನ್ನು ನಿರ್ಮಿಸಿ ಮಣ್ಣನ್ನು ತಡೆದು, ಆ ಕಟ್ಟೆಗಳ ಹೆಚ್ಚುವರಿ ನೀರು ಕೋಡಿಯ ಮೂಲಕ ಹರಿದು ದೊಡ್ಡ ಕೆರೆಗಳನ್ನು ತುಂಬಿಸುತ್ತಿತ್ತು”.\endnote{ ರಾಜಾರಾಮ ಹೆಗ್ಗಡೆ, ಕೆರೆ ನೀರಾವರಿ ನಿರ್ವಹಣೆ \engfoot{–} ಚಾರಿತ್ರಿಕ ಅಧ್ಯಯನ, ಪುಟ 38} “ಕೆರೆಗೆ ಬರುವ ನೀರು ತನ್ನ ಜತೆ ಪ್ರವಾಹದಲ್ಲಿ ಕಲ್ಲು ಮಣ್ಣನ್ನು ತರುತ್ತದೆ. ಅದು ಕೆರೆ ಅಂಗಳದಲ್ಲಿ ನಿಲ್ಲುತ್ತದೆ. ಕಾಲಕ್ರಮೇಣ ಕೆರೆಯಲ್ಲಿ ನೀರನ್ನು ಸಂಗ್ರಹಿಸುವ ಸಾಮರ್ಥ್ಯ ಕಡಿಮೆಯಾಗುತ್ತದೆ. ನಮ್ಮ ಪೂರ್ವಿಕರು ಕೆರೆಗೆ ಬಂದು ಸೇರುವ ಎಲ್ಲಾ ಹಳ್ಳಗಳಿಗೆ ಅಲ್ಲಲ್ಲೇ ಸಣ್ಣ ಕಟ್ಟೆಗಳನ್ನು ಕಟ್ಟಿ ಪ್ರವಾಹದಲ್ಲಿ ಬರುವ ಕಲ್ಲುಮಣ್ಣನ್ನು ಅಲ್ಲಲ್ಲೇ ತಡೆ ಹಿಡಿದಿಡುತ್ತಿದ್ದರು. ಕೆರೆಯ ಅಂಗಳವನ್ನು ಸಂರಕ್ಷಿಸುತ್ತಿದ್ದರು”.\endnote{ ಮೋಹನ್​, ಎಸ್​.ಕೆ., ಹಳೆಯ ಕೆರೆಗಳ ಕಟ್ಟಡದಲ್ಲಿ ತಾಂತ್ರಿಕ ವಿಶಿಷ್ಟತೆ, ಕೆರೆನೀರಾವರಿ ನಿರ್ವಹಣೆ: ಚಾರಿತ್ರಿಕ ಅಧ್ಯಯನ, ಪುಟ 6} ಗ್ರಾಮೀಣ ಪ್ರದೇಶದಲ್ಲಿ ಇದನ್ನು ಬಸಿಕಟ್ಟೆ\index{ಬಸಿಕಟ್ಟೆ} ಎಂದು ಕರೆಯಲಾಗುತ್ತಿತ್ತು. ಮದ್ದೂರು ತಾಲ್ಲೂಕಿನ ಶಾಸನದಲ್ಲಿ, “ಕೆರೆಯ ಮೇಲಕ್ಕಂ ಬೊಯ್ಸಿ ಕಟ್ಟೆಯಂ\index{ಬೊಯ್ಸಿ ಕಟ್ಟೆ} ಕಟ್ಟಿಸಿ ತೂಂಬನಿಕ್ಕಿಸಿ” ಎಂದು ಶಾಸನದಲ್ಲಿ ಸ್ಪಷ್ಟವಾಗಿ ಹೇಳಿದೆ.\endnote{ ಎಕ 7 ಮದ್ದೂರು 67 ಬೇಲೂರು 997} ಉತ್ತರ ಕರ್ನಾಟಕದಲ್ಲಿ ಇವುಗಳನ್ನು ಜಿನುಗುಕೆರೆ ಎಂದು ಕರೆಯುತ್ತಾರೆ. “ಮಳೆಯನೀರು ಮೊದಲು ಸಣ್ಣ ಕೆರೆ, ಹೊಂಡಗಳಲ್ಲಿ ಶೇಖರಣೆಯಾಗಿ, ಅಲ್ಲಿಂದ ದೊಡ್ಡ ಕೆರೆಗೆ ಹರಿದು ಬರುತ್ತಿತ್ತು. ಹೀಗೆ ಹರಿಯುವಾಗ ಹಳ್ಳಗಳು ಹೊತ್ತು ತರುವ ಹೂಳುಮಣ್ಣು ಸಣ್ಣಕೆರೆಗಳಲ್ಲಿ ಉಳಿದು ದೊಡ್ಡ ಕೆರೆಗೆ ಕೇವಲ ತಿಳಿಯಾದ ನೀರು ಮಾತ್ರ ಶೇಖರಣೆಯಾಗುತ್ತಿತ್ತು” ಎಂಬ ಅಭಿಪ್ರಾಯವೂ ಇದನ್ನೇ ಸಮರ್ಥಿಸುತ್ತದೆ.\endnote{ ವೆಂಕಟೇಶಜೋಯಿಸ್​ ಕೆಳದಿ, ಕೆಳದಿ ಕಾಲದಲ್ಲಿ ನೀರಾವರಿ ವ್ಯವಸ್ಥೆ, ಪೂರ್ವೋಕ್ತ, ಪುಟ 76-77} ಕಟ್ಟೆಗಳನ್ನೇ ಕೆರೆಯಾಗಿ ಬಡಾಯಿಸುತ್ತಿದ್ದರು (ಪುನರ್​ನಿರ್ಮಾಣ ಮಾಡುತ್ತಿದ್ದರು). “\textbf{ಎಳಸನಕಟ್ಟವ ಕೆರೆಯಾಗಿ ಕಟ್ಟಿಸಿ”} ಎಂಬ ವಿಚಾರ ಶ್ರವಣಬೆಳಗೊಳದ ಶಾಂತಲದೇವಿ ಶಾಸನದಲ್ಲಿದೆ.\endnote{ ಎಕ 2 ಶ್ರಬೆ 176 ಚಿಕ್ಕಬೆಟ್ಟ 1123}


\section*{ಕೆರೆಗಳ ಉದ್ದ ಅಗಲ ಆಳ}

ಬುಕ್ಕರಾಯನ ಕಡಪಾಜಿಲ್ಲೆಯ ಪೊರುಮಾಮಿಲ್ಲ ಶಾಸನ ಕೆರೆಯನ್ನು ನಿರ್ಮಿಸುವವರಿಗೆ ಕೈಪಿಡಿಯಂತೆ ಇದೆಯೆಂದು ತಿಳಿದು\-ಬರುತ್ತದೆ.\endnote{ ದೀಕ್ಷಿತ್​, ಜಿ.ಎಸ್​., ಕರ್ನಾಟಕದಲ್ಲಿ ಕೆರೆ ನೀರಾವರಿ, ಪುಟ 6 ಮತ್ತು 106} ಸುಮಾರು 14ನೇ ಶತಮಾನದ ಮದ್ದೂರು\index{ಮದ್ದೂರು} ಶಾಸನದಲ್ಲಿ ಅಮ್ರುತಯ್ಯ ಕೆರೆಯ\index{ಅಮ್ರುತಯ್ಯ ಕೆರೆ} ಉಲ್ಲೇಖ ಬರುತ್ತದೆ. ಈ ಶಾಸನದಲ್ಲಿ ಕೆರೆಯ ಉದ್ದ ಅಗಲಗಳ ಅಳತೆಗಳನ್ನು ಹೇಳಿರುವಂತೆ ತೋರುತ್ತದೆ. “ಕಿಳಕಂಬ..ರದಂ ಕೆಳಗೆ ಕಂಬ 200, ಅಂಗರ 120, ಅತ್ತಿಯಂ...ಕ ಮುಟ್ಟಲು ಕಂಬ 100, ಕೊಯಲು ಕಂಬ 250, ಕಂಬ 200 ಹೊಂಗಿಗಾಲು(ವೆ).. ಕಂಬ 50 ಅಂತು.... ದಪುಯರದಡೆ ಕಂಬ 100 ಎಂದು ಅಳತೆಗಳನ್ನು ಹೇಳಿದೆ. ಬಹುಶಃ ಈ ಅಳತೆಗಳು ಕೆರೆಯ ಉದ್ದಗಲಗಳನ್ನು ಸೂಚಿಸುತ್ತಿರುವಂತೆ ತೋರುತ್ತದೆ.\endnote{ ಎಕ 7 ಮ 9 ಮದ್ದೂರು 14 ನೇ ಶ.}

ಮಳವಳ್ಳಿ ತಾಲ್ಲೂಕಿನ ನಡಗಲ್​ಪುರ ಶಾಸನದಲ್ಲಿ ಕೆರೆಯ ಉದ್ದ ಅಗಲಗಳನ್ನು ಹೇಳಿರುವಂತೆ ತೋರುತ್ತದೆ. “ಆ ಕೆರೆಗುಳ್ಳ ಸೀಮೆ ಗಂಗಣ್ನನ ಕೊಡಗೆಯ ಹೊಲದ ಒಬ್ಬೆ ಮೇರುವೆ ಒಳಗೆರೆಯಿಂದಂ ಮೂಡಲು ಬಿ.ಸ.1 ಒಳಗಗಿ ಅಲಿಂದಲು ದೇವಿಗೆರೆ ಸಹಿತ ಅಲ್ಲಿಂ ಒಳಗೆರೆ ಸಹಿತ ಆ ಕೆರೆಯ ಕೆಳಗಳ ಮಣ್ನು ಮುಂನೂರು” ಎಂದು ಹೇಳಿದೆ.\endnote{ ಎಕ 7 ಮವ 45 ನಡಗಲ್​ಪುರ 1500}, ಇದು ಕೆರೆಯ ಮೇರೆಯನ್ನು ಅದರ ಅಳತೆಯನ್ನೂ ಸೂಚಿಸುತ್ತಿದೆ ಎಂದು ಊಹಿಸಬಹುದು.


\section*{ಪ್ರಾಚೀನ ಮಂಡ್ಯ ಜಿಲ್ಲೆಯ ಪ್ರದೇಶದಲ್ಲಿದ್ದ ಕೆರೆಗಳ ಅಂದಾಜು}

ಹಿಂದಿನ ಮೈಸೂರು ರಾಜ್ಯದ ಮಂಡ್ಯ ಜಿಲ್ಲೆಯಲ್ಲಿದ್ದ ಕೆರೆಗಳ ಬಗ್ಗೆ ಮುಖ್ಯವಾದ ಒಂದೆರಡು ಪ್ರಾಚೀನ ಲಿಖಿತ ಮೂಲಗಳಿಂದ ಮಾಹಿತಿ ದೊರೆಯುತ್ತದೆ. ಸುಮಾರು 1802 ರಲ್ಲಿ ಬುಕ್​ನಾನ್​\index{ಬುಕ್​ನಾನ್​} ಎಂಬುವವನು ಮೈಸೂರು ಮತ್ತು ಕನ್ನಡ ಜಿಲ್ಲೆಗಳಲ್ಲಿ ಪ್ರವಾಸ ಮಾಡಿ, ಗವರ್ನರ್​ ಜನರಲ್​ ವೆಲ್ಲೆಸ್ಲಿಗೆ ಸಲ್ಲಿಸಿದ ವರದಿಯಲ್ಲಿ ಆಗ ಇದ್ದ ಕೆರೆಗಳು ಹಾಗೂ ಅವುಗಳ ಸ್ಥಿತಿಗತಿಗಳ ವಿವರ ಇದೆ ಎಂದು ತಿಳಿದುಬರುತ್ತದೆ. ಎರಡನೆಯದಾಗಿ 1804 ರಲ್ಲಿ ಕರ್ನಲ್​ ಮಾರ್ಕ್ಸ್ವಿಲ್ಕ್ಸ್ ಸಲ್ಲಿಸಿದ ವರದಿಯಲ್ಲಿ ಕೆರೆಗಳ ಸಂಖ್ಯೆ ಮತ್ತು ನೀರಾವರಿಗೆ ಒಳಪಟ್ಟಿದ್ದ ಭೂವಿಸ್ತಾರದ ವಿವರಗಳಿವೆ ಎಂದು ತಿಳಿದುಬರುತ್ತದೆ. ಮೈಸೂರಿನ ಮುಖ್ಯ ಎಂಜಿನಿಯರ್​ ಆಗಿದ್ದ ಮೇಜರ್​ ಸ್ಯಾಂಕಿ\index{ಮೇಜರ್​ ಸ್ಯಾಂಕಿ} 1866 ರಲ್ಲಿ ಸಲ್ಲಿಸಿದ ವರದಿಯಲ್ಲಿ ಅಂದಿದ್ದ ಕೆರೆಗಳ ಸಂಖ್ಯೆ ಹಾಗೂ ನೀರಾವರಿಯ ವಿಸ್ತಾರ ಇವುಗಳ ಬಗ್ಗೆ ಅಂಕಿ ಅಂಶಗಳಿವೆ ಎಂದು ತಿಳಿದುಬರುತ್ತದೆ.\endnote{ ದೀಕ್ಷಿತ್​, ಜಿ.ಎಸ್​., ಕರ್ನಾಟಕದಲ್ಲಿ ಕೆರೆ ನೀರಾವರಿ, ಪುಟ 10} ಟಿಪ್ಪೂಸುಲ್ತಾನನ ಪತನಾನಂತರ, ಮೆಕೆಂಜಿಯು\index{ಮೆಕೆಂಜಿ} 1799 ಮತ್ತು 1806ರ ನಡುವೆ ಮೈಸೂರು ಸಂಸ್ಥಾನದ ಭೂವೈಜ್ಞಾನಿಕ ಸಮೀಕ್ಷೆಯನ್ನು ಕೈಗೊಂಡು ಸಿದ್ಧಪಡಿಸಿದ ವರದಿಯಲ್ಲಿ ಕೆರೆಗಳ ಸಂಖ್ಯೆಯನ್ನು ನಮೂದಿಸಿದೆ.

ಪಟ್ಟಣ ಅಷ್ಟಗ್ರಾಮ \enginline{–} ದೊಡ್ಡಕೆರೆಗಳು 347, ಮೈಸೂರು-ತಲಕಾಡು ದೊಡ್ಡ ಕೆರೆಗಳು\index{ದೊಡ್ಡ ಕೆರೆಗಳು} 32 ಚಿಕ್ಕ ಕೆರೆಗಳು\index{ಚಿಕ್ಕ ಕೆರೆಗಳು} 56, ಮೇಲುಕೋಟೆ \enginline{–} ದೊಡ್ಡಕೆರೆಗಳು 133, ನರಸೀಪುರ- ದೊಡ್ಡಕೆರೆಗಳು 106, ಮದ್ದೂರು- ದೊಡ್ಡಕೆರೆಗಳು 252, \hbox{ಚನ್ನಪಟ್ಟಣ-} ದೊಡ್ಡಕೆರೆಗಳು 209, ಹುಲಿಯೂರುದುರ್ಗ ದೊಡ್ಡಕೆರೆಗಳು 99, ನಾಗಮಂಗಲ-ದೊಡ್ಡಕೆರೆಗಳು 31, ಸಣ್ಣಕೆರೆಗಳು 80. ಕುಣಿಗಲ್​- ದೊಡ್ಡಕೆರೆಗಳು 8 ಸಣ್ಣಕೆರೆಗಳು 424. ಇದರಲ್ಲಿ ಮಳವಳ್ಳಿ ಭಾಗದ ಕೆರೆಗಳ ಸಂಖ್ಯೆಯನ್ನು ನೀಡಿರುವುದಿಲ್ಲ. ಆದರೆ ಮಳವಳ್ಳಿ ಕಸಬಾದಿಂದ ದಕ್ಷಿಣಕ್ಕೆ 1799ರ ಯುದ್ಧರಂಗದ ಬಳಿ ಕಾವೇರಿ ನದಿಯಿಂದ ಕೆರೆಯವರೆಗೆ ಇದ್ದ ಒಂದು ಪುರಾತನ ಅಣೆಕಟ್ಟಿನ ಗುರುತುಗಳು ಇದ್ದವು ಎಂದು ಮೆಕೆಂಜಿ ತನ್ನ ವರದಿಯಲ್ಲಿ ಹೇಳಿದ್ದಾನೆ.\endnote{ ಪೂರ್ವೋಕ್ತ, ಪುಟ 198 ರಿಂದ 204} ಮೈಸೂರಿನ ನಾನಾ ಜಿಲ್ಲೆಗಳಲ್ಲಿ 1871ರ ನಂತರ ಇದ್ದ ಕೆರೆಗಳ ಸಂಖ್ಯೆಯನ್ನು ಈ ಕೆಳಗಿನಂತೆ ನೀಡಲಾಗಿದೆ. ಮೈಸೂರು ಒಡೆಯರ ಆಡಳಿತದಲ್ಲಿ ಮಂಡ್ಯ ಜಿಲ್ಲೆಯು ಅಷ್ಟಗ್ರಾಮ ವಿಭಾಗದ ಮೈಸೂರು ಮತ್ತು ಹಾಸನ ಜಿಲ್ಲೆಗಳಲ್ಲಿ ಸೇರ್ಪಡೆಯಾಗಿತ್ತು. 1871 ರಲ್ಲಿ ಈ ಪ್ರದೇಶದಲ್ಲಿದ್ದ ಕೆರೆಗಳನ್ನು ಈ ಕೆಳಕಂಡಂತೆ ಗುರುತಿಸಲಾಗಿದೆ. ಅಷ್ಟಗ್ರಾಮ ವಿಭಾಗ-ಮೈಸೂರು ಜಿಲ್ಲೆ 1474 ಕೆರೆಗಳು, ಹಾಸನ ಜಿಲ್ಲೆ.\enginline{–} 6324 ಕೆರೆಗಳು.\endnote{ ಪೂರ್ವೋಕ್ತ, ಪುಟ 205}


\section*{ಕೆರೆಗಳ ನಿರ್ಮಾಣ ಮತ್ತು ಜೀರ್ಣೋದ್ಧಾರ}

ಮಂಡ್ಯ ಜಿಲ್ಲೆಯಲ್ಲಿ ಶಾಸನಗಳಲ್ಲಿ ಹೊಸಕೆರೆಗಳ ನಿರ್ಮಾಣ ಮತ್ತು ಹಳೆಯ, ಒಡೆದುಹೋಗಿದ್ದ ಕೆರೆಗಳ ಜೀರ್ಣೋದ್ಧಾರದ ಬಗ್ಗೆ ಸಾಕಷ್ಟು ಉಲ್ಲೇಖಗಳಿವೆ. ಹೊಸದಾಗಿ ಕಟ್ಟಿದ ಕೆರೆಗಳನ್ನು ಕನ್ನೆಗೆರೆ\index{ಕನ್ನೆಗೆರೆ} ಎಂದು ಕರೆದಿದೆ. ಹೊಸದಾಗಿ ಕೆರೆಗಳನ್ನು ಕಟ್ಟಿದವರನ್ನು \textbf{“ಕನ್ನೆಗೆರೆಮಲ್ಲ”}\index{ಕನ್ನೆಗೆರೆಮಲ್ಲ} ಎಂದು ಕರೆದು ಗೌರವಿಸಲಾಗಿದೆ. ಶಾಸನಗಳನ್ನು ಆಧರಿಸಿ ರಾಜವಂಶದ ಕಾಲಾನುಕ್ರಮವಾಗಿ ಅವುಗಳನ್ನು ಈ ಕೆಳಗಿನಂತೆ ಅಧ್ಯಯನ ಮಾಡಬಹುದು. ಶಾಸನಗಳಲ್ಲಿ ಅನೇಕ ಅಧಿಕಾರಿ ಸಾಮಂತರುಗಳನ್ನು \textbf{“ಒಡಗೆರೆಮಲ್ಲಂ”} ಎಂದು ವರ್ಣಿಸಲಾಗಿದೆ.\endnote{ ಎಕ 7 ನಾಮಂ 132 ದೊಡ್ಡಜಟಕ 1179} ಒಡೆದುಹೋದ ಕೆರೆಗಳನ್ನು ಜೀರ್ಣೋದ್ಧಾರ ಮಾಡುವುದರಲ್ಲಿ ಇವರು ಮುಂದಾಗಿದ್ದರು ಎಂಬುದು ಇದರ ಅರ್ಥ ಇರಬಹುದು.

ಕೆರೆಗಳ ಜೀರ್ಣೋದ್ಧಾರಕ್ಕೆ ಮತ್ತು ನಿರ್ವಹಣೆಗೆ ಬಿತ್ತುವಟ್ಟ (ಬೀಜವರಿಗದ್ದೆ) ವನ್ನು ದತ್ತಿಯಾಗಿ ಬಿಡಲಾಗುತ್ತಿತ್ತು ಎಂಬುದು ಕೆರೆಗಳಿಗೆ ಸಂಬಂಧಿಸಿದ ಜಿಲ್ಲೆಯ ಬಹುತೇಕ ಶಾಸನಗಳಿಂದ ತಿಳಿದುಬರುತ್ತದೆ. “10 ಕೊಳಗ ಮಣ್ನುಂ ಕೆಱೆಗೆ ಬಿತ್ತುವಟ್ಟಮುಮಂ ಬಿಟ್ಟರ್​”.\endnote{ ಎಕ 7 ಮಂ 67 ಬೇಲೂರು 997} ಎಂದು ಬೇಲೂರು ಶಾಸನದಲ್ಲಿ ಹೇಳಿದೆ. “ಸೋವರಾಸಿ ಭಟಾರಕ ಕಟ್ಟಿಸಿದ ಕೆರೆಗೆ ಹನ್ನೆರಡು ಕೊಳಗ ಗದ್ದೆಯನ್ನು ಬಿತ್ತುವಟ್ಟಾಗಿ ಬಿಟ್ಟಂತೆ ಗಂಗರ ಕಾಲದ ಹಳೆಬೂದನೂರು ಶಾಸನದಿಂದ ತಿಳಿದುಬರುತ್ತದೆ.\endnote{ ಎಕ 7 ಮಂ 54 ಹಳೇಬೂದನೂರು 10ನೇ ಶ.} ಕೆಲವು ಸಂದರ್ಭಗಳಲ್ಲಿ ತೆರಿಗೆಗಳನ್ನೂ ದತ್ತಿಯಾಗಿ ಬಿಡಲಾಗುತ್ತಿತು. \textbf{“ಚಿಕ್ಕಕಪಯ್ಯನವರು ಕೆರೆಗೋಡ ಹಣುಗನಕೆರೆ ತೂಬಿನ ಕೆಳಗೆ ಹಾಕಿದ ಭೂ ದಸವಂದದ ತೆರಗಣ ನೆರವಿನ ಹಣ ಕೆರೆಗೆ ಅದೆ ಆರು ಕಸುಕೊಂಡ್ರು ನಾಯಮಾಂಸ ತಿಂದಹಾಗೆ”} ಎಂದು ಕೆರಗೋಡು ಶಾಸನದಲ್ಲಿ ಹೇಳಿದೆ.\endnote{ ಎಕ 7 ಮಂ 38 ಕೆರಗೋಡು 15ನೇ ಶ.} ಕೆರೆಯ ಜೀರ್ಣೋದ್ಧಾರ ಕೆಲಸಕ್ಕೆ ಬಂಡಿಗಳನ್ನು ಮೀಸಲಾಗಿಡಲಾಗುತ್ತಿತ್ತು.\endnote{ ಪ್ರಭಾಕರ ರಾವ್​ ಕೆ., ಮಧ್ಯಕಾಲೀನ ಕರ್ನಾಟಕದಲ್ಲಿ ಕೆರೆಗಳ ನಿರ್ವಹಣೆ, ಕೆರೆನೀರಾವರಿ ನಿರ್ವಹಣೆ, ಚಾರಿತ್ರಿಕ ಅಧ್ಯಯನ, ಪುಟ 29} ಕೆರೆಗಳ ಕೆಲಸಕ್ಕೆ ಹೂಡುವ ಬಂಡಿಗಳವರ ಜೀವಿತಕ್ಕೆ ಗದ್ದೆಯನ್ನು ಬಿಟ್ಟ ವಿಚಾರ ಬೆಳ್ಳೂರಿನ ಶಾಸನದಲ್ಲಿದೆ. ಬಂಡಿಗಳನ್ನು ಹೂಡುವವರಿಗೆ ಗದ್ದೆಯನ್ನು ದತ್ತಿಯಾಗಿ ಬಿಡಲಾಗುತ್ತಿತ್ತು. ಇದನ್ನು “\textbf{ಕೆರೆಯ ಬಂಡಿಯ ಗದ್ದೆ}”\index{ಕೆರೆಯ ಬಂಡಿಯ ಗದ್ದೆ} ಎಂದು ಹೇಳಿದೆ.\endnote{ ಎಕ 7 ನಾಮಂ 97 ಕೋಡಿನಾಗಲಾಪುರ 16ನೇ ಶ.}\textbf{"ಕೆರೆಯಂ ಕಟ್ಟಿದವರ್ಗ್ಗೆ ಹಂನೆರಡುವರ್ಷ ಮಾನ್ಯವಲ್ಲಿಂ ಮೇಲೆ ಹತ್ತುಸಲಗೆ ಗದ್ದೆಯಂ ಮಾನ್ಯವಂ ಸಲಿಸಿ"} ಎಂದು ಮದ್ದೂರು ಶಾಸನದಲ್ಲಿ ಹೇಳಿದ್ದು ಕೆರೆಯನ್ನು ಕಟ್ಟುತ್ತಿದ್ದವರಿಗೆ ತೆರಿಗೆಯಿಂದ ವಿನಾಯತಿಯನ್ನು, ಗದ್ದೆಗಳನ್ನೂ ನೀಡುತ್ತಿದ್ದರೆಂದು ತಿಳಿದುಬರುತ್ತದೆ.\endnote{ ಎಕ 7 ಮ 9 ಮದ್ದೂರು 14ನೇ ಶ.} ಕೆರೆಗಳ ನಿರ್ವಹಣೆಗೆ ಬಿಟ್ಟ ಹಣವನ್ನು ಯಾರೇ ಮೋಸಮಾಡಿ ಪಡೆದುಕೊಂಡರೂ ಅದು ನಾಯಿಮಾಂಸ ತಿಂದ ಹಾಗೆ ಎಂದು ಹೇಳಿರುವುದು ಕೆರೆಗಳಿಗೆ ನೀಡಿದ್ದ ಪ್ರಾಮುಖ್ಯತೆಯನ್ನು ಸೂಚಿಸುತ್ತದೆ.


\section*{ಕೆರೆಗಳ ನಿರ್ಮಾಣ ಮತ್ತು ಜೀರ್ಣೋದ್ಧಾರ, ಕೆರೆಯ ನಿರ್ವಹಣೆಗೆ ದತ್ತಿ – ಗಂಗರ ಕಾಲ}
\index{ಕೆರೆಗಳ ನಿರ್ಮಾಣ}\index{ಕೆರೆಯ ನಿರ್ವಹಣೆ}

ಕೆರೆಗಳ ನಿರ್ಮಾಣದ ಬಗ್ಗೆ ಪ್ರಾಚೀನರು ಅನುಸರಿಸುತ್ತಿದ್ದ ವಿವರಗಳನ್ನು ಕ್ರಿ.ಶ. 907ರ ತಾಯಲೂರು(ತೈಲೂರು)\index{ತಾಯಲೂರು(ತೈಲೂರು)} ಶಾಸನವು ನೀಡುತ್ತದೆ. ಈ ಶಾಸನವು ಇಂದಿಗೂ ಕೆರೆಯ ಒಳಗೆ ಬಿದ್ದಿದೆ. ಕದರೂರ ಗಾಮುಂಡಗಳು, ಒಕ್ಕಲು ಮೊದಲಾದವರು ಮಣಾಲರನ ಅನುಮತದಿಂದ ಸಭೆ ಸೇರಿ ಕಚ್ಚವರ ಪೊಳಲಸೆಟ್ಟಿಯು ಕಟ್ಟಿದ ಕೆರೆಯ ಭೂಮಿಯ ಬಗ್ಗೆ ಮತ್ತು ಆ ಕೆರೆ ನಿರ್ಮಿಸಲು ಮಾಡಿದ ವೆಚ್ಚಕ್ಕೆ ಬದಲಾಗಿ ಆ ಕೆರೆಯಿಂದ ನೀರಾವರಿ ಸೌಲಭ್ಯ ಪಡೆಯುವ ಭೂಮಿಯನ್ನು ಹಂಚಿಕೆ ಮಾಡುವ ಬಗ್ಗೆ ತೀರ್ಮಾನಕ್ಕೆ ಬರುತ್ತಾರೆ. ಇದನ್ನು ಶಾಸನದಲ್ಲಿ \textbf{“ಕಾಲಳೆಗೊಟ್ಟ ಕ್ರಮ”}\index{ಕಾಲಳೆಗೊಟ್ಟ ಕ್ರಮ} ಎಂದು ಹೇಳಿದೆ. ಅದರ ಪ್ರಕಾರ 35 ಖಂಡುಗ ಭೂಮಿಯನ್ನು(ಗದ್ದೆ) ಪೊಳಲಸೆಟ್ಟಿಯು ತಾನು ಮೆಚ್ಚಿದಲ್ಲಿ ತೆಗೆದುಕೊಂಡು, ಅದರಲ್ಲಿ 5 ಖಂಡುಗವನ್ನು ಪತ್ತೊನ್ದಿ\index{ಪತ್ತೊನ್ದಿ} ತೆರಿಗೆಯಾಗಿ ಬಿಡುವಂತೆ ಹೇಳಿದೆ. ಮತ್ತೆ ಆ ಕೆರೆಯಿಂದ ಹೊರಟ ಕಾಲುವೆಯ ಎರಡೂ ಪಕ್ಕದಲ್ಲಿ ತನಗೆ ಬೇಕಾದ ಕಡೆ ಮೂರು ಖಂಡುಗ ಮಣ್ಣಿನಲ್ಲಿ (ಭೂಮಿ) ಪೊಳಲಸೆಟ್ಟಿಯು ತೋಟವನ್ನು ಬೆಳೆಸಿಕೊಳ್ಳುವಂತೆ ಹೇಳಿದ್ದು, ಇದನ್ನು ಸರ್ವಮಾನ್ಯವಾಗಿ(ತೆರಿಗೆರಹಿತವಾಗಿ) ಬಿಡಲಾಗಿದೆ. ಇದರ ಬದಲಿಗೆ ಪೊಳಲಸೆಟ್ಟಿಯು ವರ್ಷಕ್ಕೆ 12 ಕುಳ ಮೆಣಸನ್ನೂ, 15 ಪಣ ಲೋಹದ್ರಮ್ಮಗಳನ್ನು ಒಂದು ಕುಳ ತುಪ್ಪವನ್ನೂ ತೆರುವಂತೆ ಹೇಳಿದೆ.\endnote{ ಎಕ 7 ಮ 56 ತಾಯಲೂರು 907}

ಶ‍್ರೀಪುರುಷನ ಕಾಲದಲ್ಲಿ ಊರಳಿವಿನಲ್ಲಿ ಮಡಿದ ವೀರನ ಸ್ಮರಣಾರ್ಥವಾಗಿ ಕೊಂಗಣಿಕೆರೆಯನ್ನು\index{ಕೊಂಗಣಿಕೆರೆ} ನಿರ್ಮಿಸಿ ದೇವರಿಗೆ ಬಿಡಲಾಯಿತೆಂದು ಕ್ರಿ.ಶ. 8ನೇ ಶತಮಾನದ ಪೂರಿಗಾಲಿ ಶಾಸನದಲ್ಲಿ ಹೇಳಿದೆ.\endnote{ ಎಕ 7 ಮವ 122 ಪೂರಿಗಾಲಿ 8ನೇ ಶ.} ಗಂಗಪೆರ್ಮಾನಡಿಯು ಕುಂದೂರನ್ನು ಆಳುತ್ತಿದ್ದಾಗ ಅವನ ಪೆರ್ಗ್ಗಡೆ ಬಾಸಣಯ್ಯನು\index{ಪೆರ್ಗ್ಗಡೆ ಬಾಸಣಯ್ಯ} ಬೇಲೂರಿನ ಕೆರೆಯ ಮೇಲಕ್ಕೆ ಬೊಯ್ಸಿ ಕಟ್ಟೆಯನ್ನು\index{ಬೊಯ್ಸಿ ಕಟ್ಟೆ}(ಬಸಿ ಕಟ್ಟೆ) ಕಟ್ಟಿಸಿ, ಕೆರೆಯ ನಿರ್ವಹಣೆ ಹಾಗೂ ಜೀರ್ಣೋದ್ಧಾರ ಕೆಲಸಗಳಿಗೆ ಬಿತ್ತುವಟ್ಟವಾಗಿ ಹತ್ತು ಕೊಳಗ ಮಣ್ಣನ್ನೂ(ಗದ್ದೆ) ದತ್ತಿಯಾಗಿ ಬಿಡುತ್ತಾನೆ.\endnote{ ಎಕ 7 ಮಂ 67 ಬೇಲೂರು 997} ತಿರಿಯಮ್ಮನು ದೇವಕೆರೆಯೆಂಬ ಕೆರೆಯನ್ನು ಕಟ್ಟಿಸುವಾಗ ತಿಮ್ಮಾರ್ಯನೆಂಬ ಅಧಿಕಾರಿಯು ಕೆರೆಗೊಡಂಗೆಯಾಗಿ (ಕೆರೆಯ ನಿರ್ಮಾಣ ಮತ್ತು ಜೀರ್ಣೋದ್ಧಾರಗಳಿಗೆ ನೀಡುವ ದತ್ತಿ), ಊರಿನ ಪುಟ್ಟಗುರವೆಂಬ\index{ಪುಟ್ಟಗುರ} ತೆರಿಗೆಯನ್ನು, ಊರ ದನದ ಪೆಟ್ಯ (ಸಗಣಿ) ಮತ್ತು ಹದಿಮೂರು ಖಂಡುಗ ಬಿತ್ತುವಟ್ಟ (ಬೀಜವರಿ ಗದ್ದೆ) ಇವುಗಳನ್ನು ದತ್ತಿಯಾಗಿ ಬಿಡುತ್ತಾನೆ. ದನದ ಸಗಣಿಯನ್ನು ಗೊಬ್ಬರಕ್ಕೆ ಬಳಸುತ್ತಿದ್ದು, ಇದರಿಂದ ಬರುವ ಆದಾಯವನ್ನು ಕೆರೆಗೆ ದತ್ತಿ ಬಿಡಲಾಗಿದೆ ಎಂದು ಊಹಿಸಬಹುದು. ಶ‍್ರೀರಂಗಪಟ್ಟಣ ತಾಲ್ಲೂಕು, ವೊಡೇರಿ ಗ್ರಾಮದ ಈ ಶಾಸನವು ಈಗ ಊರಿನ ಕಟ್ಟೆಯ ಕೆಳಗೆ ಬಿದ್ದಿದೆ. ಬಹುಶಃ ಇಲ್ಲಿ ನಿರ್ಮಿತವಾಗಿದ್ದ ದೊಡ್ಡ ಕೆರೆಯು ಇಂದು ಒಂದು ಸಣ್ಣ ಕಟ್ಟೆಯಾಗಿದೆ.\endnote{ ಎಕ 6 ಶ‍್ರೀಪ 122 ವೊಡೇರಿ 11ನೇ ಶ.} ರಕ್ಕಸಗಂಗ ಪೆರ್ಮಾನಡಿಯ ಕಾಲದಲ್ಲಿ ಬೂದನೂರಿನಲ್ಲಿ ಸೋವರಾಸಿ\index{ಸೋವರಾಸಿ} ಭಟಾರಕರನೆಂಬ ಶೈವಯತಿಯು ಕಟ್ಟಿಸಿದ ಕೆರೆಗೆ ಆ ಊರಿನ ಗಾವುಂಡರಾದ ಚಾಮಯ್ಯ ಮತ್ತು ಜಯಮ್ಮ ಬಿತ್ತುವಾಟವನ್ನು (ಹನ್ನೆರಡು ಕೊಳಗ ಗದ್ದೆ) ದತ್ತಿಯಾಗಿ ಬಿಡುತ್ತಾರೆ.\endnote{ ಎಕ 7 ಮಂ 54 ಹಳೇಬೂದನೂರು 10-11ನೇ ಶ.}

ಚೋಳನ ಮಾಂಡಲೀಕನಾದ ಪೋಮನ್​ ಇರ್ರಾಮನ್​ ಎಂಬುವವನು, ಇಡೈತುರೈ ನಾಡಿನ, ಸಿರಿಯಕಲಸತ್ತುಪಾಡಿ\break (ಇಂದಿನ ಕಲಸ್ತವಾಡಿ) ಕೆರೆಯು 200 ವರ್ಷಗಳಿಂದ ಒಡೆದು ಹಾಳಾಗಿರಲು ಅದನ್ನು ಜೀರ್ಣೋದ್ಧಾರ ಮಾಡಿದನು.\endnote{ ಎಕ 6 ಶ‍್ರೀಪ 67 ಬೊಮ್ಮೂರು ಅಗ್ರಹಾರ 1102-03} ಇದೇ ಕಾಲದಲ್ಲಿ, ಮಾದಿಯಣ್ಣ, ಲಕ್ಕಣ್ಣ ಇವರು ಕನ್ನಂಬಾಡಿಯಲ್ಲಿ ಎರಡು ದೇಗುಲಗಳನ್ನು ಮಾಡಿಸಿ ಅದಕ್ಕೆ ಶಿವಕೆರೆ\index{ಶಿವಕೆರೆ} ಮತ್ತು ಹರಿಯಕೆರೆಯಲ್ಲಿ\index{ಹರಿಯಕೆರೆ} ಭೂಮಿಯನ್ನು ಬಿಟ್ಟರೆಂದು ಹೇಳಿದೆ. ಬಹುಶಃ ಇವರು ಶಿವ ಮತ್ತು ವಿಷ್ಣುವಿನ ದೇವಾಲಯವನ್ನು ಕಟ್ಟಿಸಿ ಎರಡೂ ದೇವರ ಹೆಸರಿನಲ್ಲಿ ಶಿವಕೆರೆ ಮತ್ತು ಹರಿಕೆರೆಗಳನ್ನು ಕಟ್ಟಿಸಿರಬಹುದೆಂದು ಊಹಿಸಬಹುದು.\endnote{ ಎಕ 6 ಪಾಂಪು 44 ಕನ್ನಂಬಾಡಿ 1114-15}

ಅರಕೆರೆಯಲ್ಲಿ, ದಮ್ಮಿಸೆಟ್ಟಿಯ ಮಗ ಉದಯಾದಿತ್ಯ ಪಲ್ಲವರಾಯನು ಕೆರೆಯನ್ನು ಕಟ್ಟಿಸಿದನೆಂದು, ಅರಕೆರೆ ನಾಡಾಳುವ ಬೀರಗಾವುಂಡನು, ಈ ಕೆರೆ ನಿರ್ಮಾಣಕ್ಕೆ, ಗದ್ದೆ ಬೆದ್ದಲುಗಳನ್ನು, ಕೊಡುಗೆಯಾಗಿ ಬಿಡುತ್ತಾನೆ.\endnote{ ಎಕ 6 ಶ‍್ರೀಪ 113 ಅರಕೆರೆ 1108} ಉದಯಾದಿತ್ಯ ಪಲ್ಲವರಾಯನು ತಾನು ಕಟ್ಟಿಸಿದ ಈ ಕೆರೆಯ ಜೀರ್ಣೋದ್ಧಾರದ ಕಾರ್ಯಗಳಿಗೆ 1500 ಹೊನ್ನುಗಳನ್ನು, “ಕಟ್ಟುವಿಚ್ಚ ಏರಿಯಿಲ್​” ಎಂದರೆ ಬಹುಶಃ ಕೀಳೇರಿಯ ಕೆಳಗೆ ನಾಲ್ಕು ಸಲಗೆ ಗದ್ದೆಯನ್ನು ಕೊಡುಗೆಯಾಗಿ ಬಿಡುತ್ತಾನೆ.\endnote{ ಎಕ 6 ಶ‍್ರೀಪ 114 ಅರಕೆರೆ 12ನೇ ಶ.}

\section*{ಕೆರೆಗಳ ನಿರ್ಮಾಣ ಮತ್ತು ಜೀರ್ಣೋದ್ಧಾರ-ದತ್ತಿ: ಹೊಯ್ಸಳರ ಕಾಲ}

ಹೊಯ್ಸಳರ ಕಾಲದಲ್ಲಿ ಕೆರೆಗಳ ನಿರ್ಮಾಣ ವ್ಯವಸ್ಥಿತವಾಗಿ ದೊಡ್ಡ ಪ್ರಮಾಣದಲ್ಲಿ ನಡೆಯಿತು. “ಕೆರೆಕಟ್ಟುವ ಕಲೆಯಲ್ಲಿ ಹೊಯ್ಸಳರು ಚಾಳುಕ್ಯರನ್ನೂ ಮೀರಿಸಿದ್ದರು. ವಿಷ್ಣುವರ್ಧನ, ಇಮ್ಮಡಿ ವೀರಬಲ್ಲಾಳ ಹಾಗೂ ಮುಮ್ಮಡಿ ಬಲ್ಲಾಳ ಅತ್ಯಂತ ಕಾರ್ಯಶಾಲಿಗಳಾಗಿದ್ದು ಇವರ ಆಳ್ವಿಕೆಯಲ್ಲಿ ಈ ದಿಸೆಯಲ್ಲಿ ಬಹಳ ಕಾರ್ಯ ನಡೆಯಿತು” ಎಂಬ ವಿದ್ವಾಂಸರ ಅಭಿಪ್ರಾಯಕ್ಕೆ ತಕ್ಕಂತೆ, ಜಿಲ್ಲೆಯಲ್ಲಿ ಇವರ ಕಾಲದಲ್ಲಿ ಹೆಚ್ಚಿನ ಕಟ್ಟೆಕಾಲುವೆಗಳು, ಕೆರೆಗಳು ನಿರ್ಮಾಣವಾಗಿವೆ.\endnote{ ದೀಕ್ಷಿತ್​, ಜಿ.ಎಸ್​., ಕೆರೆಗಳುಃ ಚಾರಿತ್ರಿಕ ಅಂಶಗಳು, ಕೆರೆ ನೀರಾವರಿ ನಿರ್ವಹಣೆ ಚಾರಿತ್ರಿಕ ಅಧ್ಯಯನ, ಪುಟ 2} ಹೊಯ್ಸಳರ ಕಾಲದಲ್ಲಿ 215 ಕೆರೆಗಳ ನಿರ್ಮಾಣ ಹಾಗೂ 36 ಕೆರೆಗಳ ಜೀರ್ಣೋದ್ಧಾರವಾಗಿದೆ ಎಂದು ಅಂದಾಜು ಮಾಡಲಾಗಿದೆ. ಆದರೆ ಆ ಸಂಖ್ಯೆ ಇದಕ್ಕಿಂತ ಹೆಚ್ಚಿದೆ.\endnote{ ಪ್ರಭಾಕರರಾವ್​, ಕೆ., ಮಧ್ಯಕಾಲೀನ ಕರ್ನಾಟಕದಲ್ಲಿ ಕೆರೆಗಳ ನಿರ್ವಹಣೆ, ಕೆರೆ ನೀರಾವರಿ ನಿರ್ವಹಣೆ - ಚಾರಿತ್ರಿಕ ಅಧ್ಯಯನ, ಪುಟ 27} ಹೊಯ್ಸಳರ ಕಾಲದಲ್ಲಿ ನಡೆದ ಕೆರೆಗಳ ನಿರ್ಮಾಣ ಜೊತೆಗೆ, ನದಿಗಳಿಗೆ ಅಣೆಕಟ್ಟುಗಳನ್ನು ಕಟ್ಟಿ ಭೂಮಿಯನ್ನು ನೀರಾವರಿಗೆ ಒಳಪಡಿಸುವ ಕೆಲಸವೂ ನಡೆಯಿತು. ಮುಂದೆ ವಿಜಯನಗರದ ಅರಸರೂ ಇದನ್ನೇ ಮುಂದುವರಿಸಿದರು. ಸಾಮಾನ್ಯವಾಗಿ ಹೊಯ್ಸಳರ ಕಾಲದಲ್ಲಿ ದೇವಾಲಯ, ಬಸದಿಗಳ ಜೊತೆಗೇ, ಕೆರೆಗಳನ್ನು ಕಟ್ಟಿಸಿ, ಅದೇ ಕೆರೆಯ ಕೆಳಗೆ ದೇವಾಲಯ ಮತ್ತು ಬಸದಿಗಳಿಗೆ ಗದ್ದೆ ಬೆದ್ದಲುಗಳನ್ನು ದತ್ತಿಯಾಗಿ ಬಿಡಲಾಗಿರುತ್ತದೆ. ಕೆರೆಯ ನಿರ್ವಹಣೆಗೆ ಬಿಟ್ಟ ದತ್ತಿಯ ವಿಚಾರ ಹೆಚ್ಚಿನ ಶಾಸನಗಳಲ್ಲಿ ಕಂಡು ಬರುವುದಿಲ್ಲ.

\textbf{ವಿನಯಾದಿತ್ಯನ ಕಾಲದ,} ಎಲೆಕೊಪ್ಪದ\index{ಎಲೆಕೊಪ್ಪ} ಮಣಿಯಮರಸ ಕೆರೆಯೇ,\index{ಮಣಿಯಮರಸ ಕೆರೆ} ಜಿಲ್ಲೆಯಲ್ಲಿ ಹೊಯ್ಸಳರ ಕಾಲದ ಕೆರೆಯ ಮೊದಲ ಶಾಸನೋಕ್ತ ಉಲ್ಲೇಖ.\endnote{ ಎಕ 7 ನಾಮಂ 127 ಎಲೆಕೊಪ್ಪ 11ನೇ ಶ.} ಪೆರ್ಗ್ಗಡೆ ಮಲ್ಲಿಯಣ್ಣನು ಸ್ವಧರ್ಮದಿಂದ\index{ಸ್ವಧರ್ಮ} ಕಿಕ್ಕೇರಿಯಲ್ಲಿ\index{ಕಿಕ್ಕೇರಿ} ಕನ್ನೆಗೆರೆಯನ್ನು ಕಟ್ಟಿಸಿದನು. ಇದೇ ಇಂದಿನ ಕಿಕ್ಕೇರಿಯ ಹಿರಿಯಕೆರೆ ಆಗಿದೆ.\endnote{ ಎಕ 7 ಕೃಪೇ 37 ಕಿಕ್ಕೇರಿ 1095} ಕೆರೆಯ ಹಿಂದೆ ಮೂಲಬ್ರಹ್ಮೇಶ್ವರ ದೇವಾಲಯವಿದೆ.

\textbf{ವಿಷ್ಣುವರ್ಧನನ ಕಾಲದಲ್ಲಿ} ಇಂಗಲಿಕನ ಕುಪ್ಪೆಯ ಬಮ್ಮಣ್ಣನು, ಊರಿನ ಉತ್ತರ ದಿಕ್ಕಿನಲ್ಲಿ ಹರಿಯುವ ಹಳ್ಳಕ್ಕೆ ಅಡ್ಡಲಾಗಿ ಕೆರೆಯನ್ನು ನಿರ್ಮಿಸಿದನೆಂದು ಹೇಳಬಹುದು.\endnote{ ಎಕ 6 ಪಾಂಡವಪುರ 252 ತಿರುಮಲಸಾಗರ ಛತ್ರ 1125} ಪೆರ್ಗ್ಗಡೆ ಮಲ್ಲಿನಾಥನು ಮಲ್ಲಘಟ್ಟದಲ್ಲಿ (ಇಂದಿನ ಅಬಲವಾಡಿ) ದೇವರಕೆರೆಯನ್ನು\index{ದೇವರಕೆರೆ} ನಿರ್ಮಿಸಿದನು.\endnote{ ಎಕ 7 ಮ 29 ಅಬಲವಾಡಿ 1131} ಸುಖದೊರೆಯನ್ನು ಆಳುತ್ತಿದ್ದ ಬೋಕಿಸೆಟ್ಟಿ\index{ಬೋಕಿಸೆಟ್ಟಿ} ಹಾಗೂ ಅವನ ತಮ್ಮಂದಿರುಗಳು, ತಮ್ಮ ಅಯ್ಯನ ಹೆಸರಿನಲ್ಲಿ ಮಾಚ ಸಮುದ್ರ\index{ಮಾಚ ಸಮುದ್ರ} ಮತ್ತು ತಮ್ಮ ಅವ್ವೆಯ ಹೆಸರಿನಲ್ಲಿ ಮಾಕಸಮುದ್ರವೆಂಬ ಎರಡು ಕನ್ನೆಗೆರೆಗಳನ್ನು ಕಟ್ಟಿಸಿ ಮೂಲಸ್ಥಾನ ದೇವರಿಗೆ ಧಾರಾಪೂರ್ವಕವಾಗಿ ಬಿಡುತ್ತಾರೆ.\endnote{ ಎಕ 7 ನಾಮಂ 17 ಪುರದಕಟ್ಟೆ (ಬೇಚಿರಾಕ್​) 1139} ಪ್ರಾಯಶಃ ಈ ಕೆರೆಗಳಿಂದ ಬರುವ ತೆರಿಗೆಯು ದೇವಾಲಯಕ್ಕೆ ಮತ್ತು ಕೆರೆಯ ನಿರ್ವಹಣೆಗೆ ಹೋಗುತ್ತಿತ್ತೆಂದು ಊಹಿಸಬಹುದು. ಜಕ್ಕಿಸೆಟ್ಟಿಯು\index{ಜಕ್ಕಿಸೆಟ್ಟಿ} ತನ್ನ ಊರಾದ ಸುಕ್ಕುಧರೆಯಲ್ಲಿ(ಇಂದಿನ ಸುಗಧರೆ) ಬಸದಿಯನ್ನು ಮಾಡಿಸಿ, ಊರ ಈಶಾನ್ಯಕ್ಕೆ ಒಂದು ಕೆರೆಯನ್ನು ಮತ್ತು ಒಂದು ಕಿರುಕೆರೆಯನ್ನೂ ಕಟ್ಟಿಸಿದನು.\endnote{ ಎಕ 7 ನಾಮಂ 14 ಸುಕಧರೆ 12ನೇ ಶ.} ತೆಂಗಿನ ಘಟ್ಟದ ಹಡವಳದ ಕೊಳ್ಳಿಅಯ್ಯ ಹಾಗೂ ಹೆಗ್ಗಡೆ ಮುಂಜಯ್ಯ ಇವರುಗಳು ಸೇರಿ, ಕೆರೆಯನ್ನು ಕಟ್ಟಿಸಿ ಆ ಕೆರೆಯ ಕೆಳಗೆ ಹೊಯ್ಸಳೇಸ್ವರ ದೇವರಿಗೆ ಗದ್ದೆ ಬೆದ್ದಲುಗಳನ್ನು ದತ್ತಿಯಾಗಿ ಬಿಡುತ್ತಾರೆ.\endnote{ ಎಕ 6 ಕೃಪೇ 42 ತೆಂಗಿನಘಟ್ಟ 1117} ಮಾಳಿಗೆಯೂರಿನ ವೃತ್ತಿಯನಾಯಕ ಮಹಾಸಾಮಂತ ಮಾಚೆಯ ನಾಯಕನು, ಹುಬ್ಬನಹಳ್ಳಿಯಲ್ಲಿ ಮಾಕೇಶ್ವರ ದೇವಾಲಯವನ್ನು ಕಟ್ಟಿಸಿ ಕೆರೆಯನ್ನು ನಿರ್ಮಿಸಿದನು. \textbf{“ಮತ್ತಂ ಆ ನಾಯಕರು ಕಟ್ಟಿಸಿದ ಹಿರಿಯ ಕೆರೆಯ ಕೆಳಗೆ”}\index{ಹಿರಿಯ ಕೆರೆ} ಆ ದೇವಾಲಯಕ್ಕೆ ಗದ್ದೆಬೆದ್ದಲುಗಳನ್ನು ಬಿಟ್ಟಿರುವುದರಿಂದ ಮಾಚನೇ ಈ ಹಿರಿಕೆರೆಯನ್ನು ನಿರ್ಮಿಸಿರುತ್ತಾನೆ.\endnote{ ಎಕ 6 ಕೃಪೇ 62 ಹುಬ್ಬನಹಳ್ಳಿ 1140} ಒಂದನೆಯ ನರಸಿಂಹನ ಕಾಲದಲ್ಲಿ ಕಿಕ್ಕೇರಿ ಹಿರಿಯ ಕೆರೆಯ ಉಲ್ಲೇಖವಿದೆ.\endnote{ ಎಕ 6 ಕೃಪೇ 27 ಕಿಕ್ಕೇರಿ 1171} ಈ ಕೆರೆಯು ಮೇಲೆ ತಿಳಿಸಿದಂತೆ ವಿನಯಾದಿತ್ಯನ ಕಾಲದಲ್ಲಿ ನಿರ್ಮಾಣವಾಗಿದೆ.

\textbf{ಎರಡನೆಯ ಬಲ್ಲಾಳನ} ಕಾಲದಲ್ಲಿ ಕೂಡಾ ಜಿಲ್ಲೆಯಲ್ಲಿ ಅನೇಕ ಕೆರೆಗಳು ನಿರ್ಮಾಣವಾಗಿವೆ. ಕುರುಕ್ಕಿನಾಡ ಮಾಳಾನಹಳ್ಳಿಯ ಹರದಗಾವುಂಡನು, ಊರ ತೆಂಕಣಭಾಗದಲ್ಲಿ, ಹರದ ಸಮುದ್ರವನ್ನು\index{ಹರದ ಸಮುದ್ರ} ಕಟ್ಟಿಸುತ್ತಾನೆ.\endnote{ ಎಕ 6 ಪಾಂಪು 20 ಮಾಳಾನಹಳ್ಳಿ 1176} ಹೊಯ್ಸಳ ಪಟ್ಟಣಸ್ವಾಮಿ ಸೋವಿಸೆಟ್ಟಿಯು, ಪಟ್ಟಣದಿಂದ(ಹಟ್ನ) ಬಡಗಣ ನಗರಸಮುದ್ರ,\index{ನಗರಸಮುದ್ರ} ಮೂಡಣ ಹೊಯ್ಸಳ ಸಮುದ್ರ,\index{ಹೊಯ್ಸಳ ಸಮುದ್ರ} ತೆಂಕಣ ಸೆಟಿಯಕೆರೆ\index{ಸೆಟಿಯಕೆರೆ} ಎಂಬ ಹೆಸರಿನ \textbf{“ತಟಾಕತ್ರಯ}”ಗಳನ್ನು\index{ತಟಾಕತ್ರಯ} ಕಟ್ಟಿಸಿದನು.\endnote{ ಎಕ 7 ನಾಮಂ 118 ಹಟ್ಟಣ 1178} ಊರಿನ ಮೂರೂ ದಿಕ್ಕುಗಳಿಗೂ ಕೆರೆ ಕಟ್ಟಿಸಿರುವುದು ವಿಶೇಷ. ಈ ಊರು ಬೆಟ್ಟ ಗುಡ್ಡಗಳ ಬದಿಯಲ್ಲಿದ್ದು, ಕೆರೆಗಳಿಗೆ ಸರಾಗವಾಗಿ ನೀರು ಹರಿದುಬರುತ್ತಿತ್ತೆಂದು ತೋರುತ್ತದೆ. ಬಡಗೆರೆನಾಡಾಳುವ ಗಾವುಂಡರು, ಆಲದಹಳ್ಳಿಯಲ್ಲಿ ಕೆರೆಯನ್ನು ಕಟ್ಟಿಸುತ್ತಾರೆ.\endnote{ ಎಕ 7 ಮಂ 83 ಕೊತ್ತತ್ತಿ 1178} ಮಹಾಸಾಮಂತ ದುಮ್ಮೆಯನಾಯಕನು ಕಲುಕಣಿ ನಾಡ ಜೆಟ್ಟಿಗದಲ್ಲಿ, ಕನ್ನೆಗೆರೆ ದುಮ್ಮಸಮುದ್ರ,\index{ದುಮ್ಮಸಮುದ್ರ} ಒಡೆಯರಹಳ್ಳಿಯ ಹೊಸಕೆರೆ,\index{ಹೊಸಕೆರೆ} ತಟ್ಟಹಳ್ಳದ ಕನ್ನೆಗೆರೆ ಮಸಣಸಮುದ್ರ\index{ಮಸಣಸಮುದ್ರ} ಎಂಬ ನಾಲ್ಕು ಕೆರೆಗಳನ್ನು ನಿರ್ಮಿಸಿದನೆಂದು ತಿಳಿದುಬರುತ್ತದೆ. ಈ ಶಾಸನದಲ್ಲಿ ಆಲಕೆರೆ\index{ಆಲಕೆರೆ} ಮತ್ತು ಮಣಿಯಮ್ಮನ ಕೆರೆಯ\index{ಮಣಿಯಮ್ಮನ ಕೆರೆ} ಉಲ್ಲೇಖವೂ ಇದೆ.\endnote{ ಎಕ 7 ನಾಮಂ 130, 131 ದೊಡ್ಡಜಟಕ 1179} ಬೆಳ್ಳೂರಿನ ಮಂಡಲಸ್ವಾಮಿಯು,\index{ಮಂಡಲಸ್ವಾಮಿ} ಊರಮುಂದೆ ತನ್ನ ಹೆಸರಿನಿಂದಲೇ ಪ್ರಸಿದ್ಧವಾದ ಕೆರೆಯನ್ನು ಕಟ್ಟಿಸಿದನೆಂದು ತಿಳಿದುಬರುತ್ತದೆ. ಈ ಶಾಸನದಲ್ಲಿ ಊರಮುಂದಣ ಕಿರುಕೆರೆಯ\index{ಕಿರುಕೆರೆ} ಉಲ್ಲೇಖವಿದ್ದು, ಈ ಕಿರುಕೆರೆಯ ಕೆಳಗೆ ದೇವರಿಗೆ ಭೂಮಿಯನ್ನು ಬಿಡಲಾಗಿದೆ. ಇದರಿಂದ ಕಿರುಕೆರೆಗಳು ದೇವಾಲಯಕ್ಕೆ ಮೀಸಲಾಗಿದ್ದವೆಂದು ಹೇಳಬಹುದು.\endnote{ ಎಕ 6 ನಾಮಂ 80 ಬೆಳ್ಳೂರು 1199} ಕಲ್ಕುಣಿಯ ಮಾದಿರಾಜ ಹೆಗ್ಗಡೆಯು, ಬಲಸಮುದ್ರ ಕೆರೆಯನ್ನು\index{ಬಲಸಮುದ್ರ ಕೆರೆ} ಕಟ್ಟಿಸಿ, \textbf{ಅದರ ಎಲ್ಲೆಗಳನ್ನು ಕಲ್ಲು ನೆಟ್ಟು ಗುರುತಿಸಿ}, ಆ ಕೆರೆಯ ಜೀರ್ಣೋದ್ಧಾರಕ್ಕೆ ಬಡಗೆರೆ ನಾಡ ಸಿದ್ಧಾಯದಿಂ ಇಪ್ಪತ್ತು ಗದ್ಯಾಣವನ್ನು, ಎರಡು ಖಂಡುಗ ನೆಲವನ್ನು ಕಲುಕಣಿಯ ಸಮಸ್ತ ಪ್ರಭುಗವುಡುಗಳ ಸಮ್ಮುಖದಲ್ಲಿ ದತ್ತಿಯಾಗಿ ಬಿಡುತ್ತಾನೆ.\endnote{ ಎಕ 7 ಮವ 143 ಕಲ್ಕುಣಿ 13-14ನೇ ಶ.}\textbf{ಕೆರೆಯ ಎಲ್ಲೆಯನ್ನು ಗುರುತಿಸಿರುವ ಪ್ರಮುಖವಾದ ವಿಷಯ ಈ ಶಾಸನದಲ್ಲಿದೆ.}

\textbf{ಎರಡನೆಯ ನರಸಿಂಹನ ಕಾಲದಲ್ಲಿ,} ಬೆಳ್ಳೂರಿನಲ್ಲಿ, ಮಾಚಸಮುದ್ರ\index{ಮಾಚಸಮುದ್ರ} ಕೆರೆ, ಸಿರಿರಂಗಪುರ ಕೆರೆ ಮತ್ತು ಕಿರುಕೆರೆ ಮತ್ತು ತಗಚೆಗೆರೆಗಳಿದ್ದವೆಂದು\index{ತಗಚೆಗೆರೆ} ತಿಳಿದುಬರುತ್ತದೆ. ಸಿಂಧೆಯನಾಯಕನು ಕಾಚಿದೇವನ ತಂದೆ ಮಾಚಿಮಯ್ಯನಿಗೆ ಧಾರಾಪೂರ್ವಕವಾಗಿ ನೀಡಿದ ಭೂಮಿಯಲ್ಲಿ, ಕಾಚಿದೇವನು ತನ್ನ ತಂದೆ ಮಾಚಿದೇವನ ಹೆಸರಿನಲ್ಲಿ ಬೆಳ್ಳೂರಿನಲ್ಲಿ ಮಾಚಸಮುದ್ರ ಕೆರೆಯನ್ನು ನಿರ್ಮಿಸಿದನು. ಈ ಕೆರೆಗೆ ಮೂಡಲು ಕುಪ್ಪಗವುಡಿಯಹಳ್ಳಿ, ಬಡಗ ಕರಿಯಜೀಯನಹಳ್ಳಿ, ಕಲ್ಲಹಳ್ಳಿ, ಪಡುವ ತೆಂಕ ಹಿರಿಯಹಳ್ಳಿಗಳೇ ಮೇರೆಯಾಗಿತ್ತೆಂದು ಹೇಳಿದ್ದು, ಇವು ಭಾರೀ ದೊಡ್ಡ ಕೆರೆಗಳಾಗಿರಬಹುದು. ಈ ಕೆರೆಯ ನಿರ್ವಹಣೆಗೆ ಕೆರೆಗೊಡಗಿಯಾಗಿ ಮೂಡಣ ಕೋಡಿಯಲ್ಲಿ 125 ಸಲಗೆ ತೋಟಸ್ಥಳ, 3000 ಕಂಬ ಬೆದ್ದಲುಗಳನ್ನು ಬಿಡಲಾಗಿದೆ.\endnote{ ಎಕ 6 ನಾಮಂ 81 ಬೆಳ್ಳೂರು 1223-24} ಈ ಅವ್ವೆಯರ ಕೆರೆ\index{ಅವ್ವೆಯರ ಕೆರೆ} ಮತ್ತು ತಗಚೆಗೆರೆಗಳನ್ನೇ ಮುಂದೆ ಪೆರುಮಾಳೆ ದೇವ ದಂಡನಾಯಕನು ವಿಸ್ತರಿಸಿ ಜೀರ್ಣೋದ್ಧಾರ ಮಾಡುತ್ತಾನೆ. ಹರಿಹರ ದಂಡನಾಯಕನು, ಬಸುರಿವಾಳದಲ್ಲಿ, ತನ್ನ ತಾಯಿ ಗುಜ್ಜಲೆಯ ಹೆಸರಿನಲ್ಲಿ ಗುಜ್ಜವ್ವೆ ನಾಯಕಿತ್ತಿಯ ಕೆರೆಯನ್ನೂ\index{ಗುಜ್ಜವ್ವೆ ನಾಯಕಿತ್ತಿಯ ಕೆರೆ} ನಿರ್ಮಿಸಿದನು.\endnote{ ಎಕ 7 ಮಂ 29 ಬಸರಾಳು 1234} ಇಂದಿಗೂ ಇದನ್ನು ಗುಜ್ಜವ್ವೆಕೆರೆ ಎನ್ನುತ್ತಾರೆ.

\section*{ಪೆರುಮಾಳೆ ದೇವನಿಂದ ಬೆಳ್ಳೂರಿನ ಕೆರೆಗಳ ವಿಸ್ತರಣೆ}

ಮೂರನೇ ನರಸಿಂಹನ ಕಾಲದಲ್ಲಿ ಪೆರುಮಾಳೆ ದೇವ ದಂಡನಾಯಕನು ಬೆಳ್ಳೂರು ಶಾಸನವು ಅಂದಿನ ಕಾಲದಲ್ಲಿ ಕೆರೆಗಳನ್ನು ವಿಸ್ತರಿಸಿ ಜೀರ್ಣೋದ್ಧಾರ ಮಾಡುತ್ತಿದ್ದ ರೀತಿಯ ಸ್ಪಷ್ಟಚಿತ್ರಣವನ್ನು ನೀಡುತ್ತವೆ. ಆ ಊರಿನ ಕೆರೆಗಳು ಹಾಗೂ ಕಟ್ಟುಕಾಲುವೆಗಳ ವಿಸ್ತರಣೆಯನ್ನು ಮಾಡುವಾಗ ಅನುಸರಿಸಿದ ಕ್ರಮಗಳು ಇಂದಿನ ಭೂಸ್ವಾಧೀನ ಕ್ರಮಕ್ಕಿಂತಲೂ ಆಧುನಿಕವಾಗಿ ಕಂಡುಬರುತ್ತದೆ.

ಪೆರುಮಾಳೆದೇವ ದಂಡನಾಯಕನು ಅಪಾರವಾದ ಹಣವನ್ನು ಖರ್ಚುಮಾಡಿ ಬೆಳ್ಳೂರಿನ ಅಲ್ಲಾಳಸಮುದ್ರ,\index{ಅಲ್ಲಾಳಸಮುದ್ರ} ಅವ್ವೆಯರಕೆರೆ, ತಗಚೆಗೆರೆಗಳನ್ನು \textbf{“ಧ್ರುವಜಲವಹಂತಾಗಿ”}\index{ಧ್ರುವಜಲ} ಅಂದರೆ ಹೆಚ್ಚು ನೀರು ನಿಲ್ಲುವಂತಹ ರೀತಿಯಲ್ಲಿ ವಿಸ್ತರಿಸಿದನು. ಜೊತೆಗೆ ಈ ಕೆರೆಯ ಕಟ್ಟುಕಾಲುವೆಗಳನ್ನು ವಿಸ್ತರಿಸಿ ಜೀರ್ಣೋದ್ಧಾರ ಮಾಡಿದನು. ಈ ಕೆರೆಯ ಮತ್ತು ಕಟ್ಟುಕಾಲುವೆಗಳ ವಿಸ್ತರಣೆಯನ್ನು ಮಾಡುವಾಗ, ಆ ಪ್ರದೇಶದಲ್ಲಿದ್ದ ಆ ಊರಿನ ದೇವಾಲಯಗಳ ದತ್ತಿಯ ಗದ್ದೆಬೆದ್ದಲುಗಳನ್ನು ವಶಪಡಿಸಿಕೊಂಡು ಅದಕ್ಕೆ ಪ್ರತಿಕ್ಷೇತ್ರವಾಗಿ(ಬದಲಾಗಿ) ಬೇರೆಕಡೆ ಬಹುಶಃ ಹೊಸದಾಗಿ ವಿಸ್ತರಿಸಿದ ಈ ಕೆರೆಯ ಕೆಳಗೆ ಬೆದ್ದಲುಗಳನ್ನು ಕೊಟ್ಟನು. ಕೆರೆಯನ್ನು ವಿಸ್ತರಿಸಿ ಕಟ್ಟಿದ ಮೇಲೆ, ದೇವಾಲಯಕ್ಕೆ ನೀಡಿದ ಬೆದ್ದಲುಗಳು(ಹೊಲಗಳು) ನೀರಾವರಿ ಅಚ್ಚುಕಟ್ಟಿಗೆ ಒಳಪಟ್ಟ ಕಾರಣ, ಬಿತ್ತುವಟ್ಟವಾಗಿ, ವರ್ಷನಿಬಂಧಿಯಾಗಿ, ಆ ಗದ್ದೆಗಳಲ್ಲಿ ಬೆಳೆಯುವ ಭತ್ತದಲ್ಲಿ ಖಂಡುಗ ಒಂದಕ್ಕೆ ನಾಲ್ಕುಹಣವನ್ನು ತೆರಲು ಸ್ಥಾನಪತಿಗಳು ಒಪ್ಪಿಕೊಳ್ಳುತ್ತಾರೆ. ಈ ಶಾಸನದ ಕಾಲ ಕ್ರಿ.ಶ. 1269 ಜನವರಿ 6ಕ್ಕೆ ಸರಿಹೋಗುತ್ತದೆ. ಈ ಕೆರೆಗಳ ಮತ್ತು ಕಟ್ಟುಕಾಲುವೆಗಳ ವಿಸ್ತರಣೆ ಮತ್ತು ಜೀರ್ಣೋದ್ಧಾರ ಕಾರ್ಯ ಈ ಕಾಲದಲ್ಲಿ ಪ್ರಾರಂಭವಾಯಿತೆಂದು ಹೇಳಬಹುದು.\endnote{ ಎಕ 7 ನಾಮಂ 83 ಬೆಳ್ಳೂರು 1269 ಜನವರಿ 6} ಮೇಲ್ಕಂಡ ಕೆರೆಗಳ ವಿಸ್ತರಣೆಯ ಕೆಲಸವು ಬಹುಶಃ ಕ್ರಿ.ಶ.1269 ಸೆಪ್ಟೆಂಬರ್​ 2ಕ್ಕೆ ಮುಗಿದಿರುವಂತೆ ಇದೇ ವಿಷಯವನ್ನು ತಿಳಿಸುವ ಅಲ್ಲಿರುವ ಇನ್ನೊಂದು ಶಾಸನದಿಂದ ತಿಳಿದುಬರುತ್ತದೆ. ಬಹುಪಾಲು ಶಾಸನಗಳು ಕೆರೆ ನಿರ್ಮಾಣ ಅಥವಾ ಜೀರ್ಣೋದ್ಧಾರದ ನಿಖರ ದಿನವನ್ನು ತಿಳಿಸುತ್ತದೆ ಎಂಬ ಅಭಿಪ್ರಾಯಕ್ಕೆ ಇದು ಪುಷ್ಟಿ ನೀಡುತ್ತದೆ.\endnote{ ದೀಕ್ಷಿತ್​, ಜಿ.ಎಸ್​., ಕರ್ನಾಟಕದಲ್ಲಿ ಕೆರೆ ನೀರಾವರಿ, ಪುಟ 8} ಈ ಸಂದರ್ಭದಲ್ಲಿ ಮತ್ತೆ, ಈ ಕೆರೆಗಳ ವಿಸ್ತರಣೆ ಮತ್ತು ಭೂಸ್ವಾಧೀನಕ್ಕೆ ಸಂಬಂಧಿಸಿದಂತೆ ಮತ್ತೊಂದು ಶಾಸನವನ್ನು ಪೆರುಮಾಳೆ ದೇವನು ಹಾಕಿಸಿದ್ದಾನೆ. ಈ ಕೆರೆಗಳ ವಿಸ್ತರಣೆ ಕಾರ್ಯಪೂರ್ಣವಾಗಿ ಅಲ್ಲಾಳ ಸಮುದ್ರ ಕೆರೆಯಲ್ಲಿ ನೀರು ನಿಂತಾಗ ಆ ನೀರು, ಕೆರೆಯ ಹಿಂದಿದ್ದ, ಶ‍್ರೀರಂಗಪುರ ಅಗ್ರಹಾರಕ್ಕೆ ಸೇರಿದ್ದ ಬೆದ್ದಲುಗಳವರೆಗೂ ವಿಸ್ತರಿಸಿ, ಆ ಬೆದ್ದಲುಗಳು ಕೆರೆಯಲ್ಲಿ ಮುಳುಗಡೆಯಾದವು. ಆಗ ಮತ್ತೆ ಬೆಳ್ಳೂರು ಅಗ್ರಹಾರದ ಮಹಾಜನಗಳು, ಶ‍್ರೀರಂಗಪುರ\index{ಶ‍್ರೀರಂಗಪುರ} ಅಗ್ರಹಾರದ ಮಹಾಜನಗಳ ಜೊತೆ ಮತ್ತೊಂದು ಒಪ್ಪಂದಕ್ಕೆ ಬರುತ್ತಾರೆ. ಈ ರೀತಿ ನೀರು ನಿಂತು ಕೆರೆಯ ವಿಸ್ತರಣೆಗೆ ಒಳಪಟ್ಟ ಬೆದ್ದಲುಗಳನ್ನು ಕೆರೆಗೆ ಬಿಟ್ಟುಕೊಟ್ಟು, ಉಳಿದ ಬೆದ್ದಲುಗಳನ್ನು ಉಳಿಸಿಕೊಳ್ಳಲು ಮೊದಲು ತೀರ್ಮಾನಿಸುತ್ತಾರೆ. ಮುಳುಗಡೆಯಾದ ಬೆದ್ದಲುಗಳಿಗೆ ಬದಲಿಯಾಗಿ ಅಲ್ಲಾಳ ಸಮುದ್ರ ಕೆರೆಯ ಕೆಳಗೆ ಹನ್ನೆರಡು ಸಲಗೆ ಗದ್ದೆಯನ್ನು, ಮತ್ತು ಹೊಸದಾಗಿ ಕಟ್ಟಿಸಿದ ಅಲ್ಲಾಳಸಮುದ್ರ ಕೆರೆಯ ಮೇಲಣ ಕೋಡಿಯ ಒಳಗಿದ್ದ ಭೂಮಿಗೆ ಬದಲಾಗಿ, ಅಲ್ಲಾಳಸಮುದ್ರ ಕೆರೆಯ ಕೆಳಗೆ ಇನ್ನು ಹನ್ನೆರಡು ಸಲಗೆ ಗದ್ದೆಯನ್ನು ತೆಗೆದುಕೊಂಡು, \textbf{ನೀರೊತ್ತನ್ನು}\index{ನೀರೊತ್ತನ್ನು} ಕಳೆದು, ಉಳಿದಂತೆ ಶಾಸನಸ್ತವಾದ ಕರಗಳನ್ನು ತೆರಲು ಶ‍್ರೀರಂಗಪುರದ ಮಹಾಜನಗಳು ಒಪ್ಪುತಾರೆ.\endnote{ ಎಕ 7 ನಾಮಂ 82 ಬೆಳ್ಳೂರು 1269 ಸೆಪ್ಟೆಂಬರ್​ 2} ಮುಳುಗಡೆಯಾದ ಮತ್ತು ಕೆರೆಗೆ ಬಿಟ್ಟುಕೊಟ್ಟ ಭೂಮಿಗೆ ಬದಲಾಗಿ ಆ ಕೆರೆಯ ಕೆಳಗೇ ಗದ್ದೆಯನ್ನು ನೀಡುವುದು, ಹಾಗೂ ಕೆರೆಯ ನಿರ್ವಹಣೆಗೆ ಬಿತ್ತುವಟ್ಟವಾಗಿ ತೆರಿಗೆಗಳನ್ನು ಹೆಚ್ಚಾಗಿ ತೆರಲು ಒಪ್ಪುವುದು ಈ ಭೂಸ್ವಾಧೀನ ಕಾರ್ಯದ ವಿಶೇಷವಾಗಿದೆ. ನೀರೊತ್ತಿನಲ್ಲಿರುವ ಭೂಮಿಯಲ್ಲಿ\break ಶೈತ್ಯದಿಂದ ಬೆಳೆ ಬೆಳೆಯುವುದಿಲ್ಲ ಆದಕಾರಣ ಅದನ್ನು ಕರದಿಂದ ವಿಮುಕ್ತಗೊಳಿಸಲಾಗಿದೆ.

ಈ ಕೆರೆಗಳ ನಿರ್ವಹಣೆಗೆ ಶಾಶ್ವತವಾದ ವ್ಯವಸ್ಥೆಯನ್ನು ಮಾಡಿರುವುದು ಇದೇ ಊರಿನಲ್ಲಿರುವ ಪೆರಮಾಳೆದೇವನ ಮುಂದಿನ ಶಾಸನಗಳಿಂದ ತಿಳಿದುಬರುತ್ತದೆ. \textbf{“ಕೆರೆಯ ಭಂಡಿಗಳಿಗೆ\index{ಕೆರೆಯ ಭಂಡಿ} ಆ ಬೆಳ್ಳೂರಲಿ ಕಲ್ಲ ನಡಿಸಿ ಕೊಟ್ಟ ಸೀಮೆಯಿಂದೊಳಗೆ ಗದ್ದೆ ಸಲಗೆ 36, ಬೆದ್ದಲು ಕಂಬ 1850, ಆ ಬೆಳ್ಳೂರ ಅಲ್ಲಾಳ ಸಮುದ್ರವೊಳಗಾದ ಕೆರೆಗಳ ಕೆಲಸಕ್ಕೆ ಹೂಡುವ ಭಂಡಿಗಳ ಆಳ ಜೀವಿವೊಳಗಾಯ್ತಕ್ಕೆ ಆ ಮಹಾಜನಂಗಳು ಎಂದೆಂದಿಗೂ ಸರ್ವಬಾಧಾಪರಿಹಾರ ಸರ್ವಮಾನ್ಯವಾಗಿ ನಡೆಸಿಕೊಂಡುಬರುವುದು”} ಎಂದು ಈ ಕೆರೆಗಳ ನಿರ್ವಹಣೆ, ಜೀರ್ಣೋದ್ಧಾರ ಕಾರ್ಯಗಳಿಗೆ ವ್ಯವಸ್ಥೆ ಮಾಡಿರುವುದು ಕಂಡುಬರುತ್ತದೆ.\endnote{ ಎಕ 7 ನಾಮಂ 74 ಬೆಳ್ಳೂರು 1271} ಈ ಭೂಮಿಯನ್ನು ಗುತ್ತಗೆಯಾಗಿ ನೀಡಿ, ಗುತ್ತಗೆಯಿಂದ ಬರುವ ಆದಾಯವನ್ನು ಗುತ್ತಗೆದಾರರು ಭಂಡಿಯ ಜೀವಿತದವರಿಗೆ ನೀಡುವಂತೆ ಏರ್ಪಾಡು ಮಾಡಲಾಗಿದೆ. “ಅಲ್ಲಾಳ ಸಮುದ್ರ ಕೆರೆಯ ಪಡುವಣ ಕೋಡಿಯಲಿ ಶ‍್ರೀರಂಗಪುರದವರಿಗೆ ಬಿಟ್ಟ ಗದ್ದೆಯಿಂದ ಪಡುವಲು, ಪಡುವಣ ಕೋಡಿಯಿಂ ಮೂಡಲುವುಳ್ಳ ಕ್ಷೇತ್ರವು ಆ ಊರ ಪ್ರಸನ್ನ ಮಾಧವದೇವರು, ಶ‍್ರೀ ರಾಮಕೃಷ್ಣ, ಶ‍್ರೀ ವರದ ಅಲ್ಲಾಳನಾಥ ದೇವರುಗಳಿಗೆ ಮತ್ತು ಅಲ್ಲಾಳ ಸಮುದ್ರ ಕೆರೆಯ ಭಂಡಿಯ ಧರ್ಮಕ್ಕೆ\index{ಭಂಡಿಯ ಧರ್ಮ} ಸರಿಯಾಗಿ ಸಲುವುದು. ಆ ಕ್ಷೇತ್ರವ ಮಾಡುವ ಗುತ್ತಗೆಕಾರರು,\index{ಗುತ್ತಗೆಕಾರರು} ಆ ಗುತ್ತಗೆಯ ವಸ್ತುವನು ಆ ಪ್ರಸನ್ನ ಮಾಧವದೇವರು, ಶ‍್ರೀ ರಾಮಕೃಷ್ಣದೇವರು, ಶ‍್ರೀ ಅಲ್ಲಾಳನಾಥ ದೇವರುಗಳ ಅಮೃತಪಡಿಗೆ ಮತ್ತು ಆ ಕೆರೆಯ ಭಂಡಿಯ ಧರ್ಮಕ್ಕೆ ಸರಿಯಾಗಿ ಯಿಕ್ಕುತ ಬಹುದು” ಎಂದು ಹೇಳಿದೆ.\endnote{ ಎಕ 7 ನಾಮಂ 76 ಬೆಳ್ಳೂರು 1284}\textbf{ಹೀಗಾಗಿ ಕೆರೆಯ ನಿರ್ವಹಣಾ ವ್ಯವಸ್ಥೆ ಕಾಲಾನುಕಾಲಕ್ಕೆ ವ್ಯತ್ಯಯವಿಲ್ಲದೆ ನಡೆದುಕೊಂಡು ಬರುತ್ತಿತ್ತು ಎಂಬುದು ಇದರಿಂದ ತಿಳಿದುಬರುತ್ತದೆ. }

ಯಾದವನಾರಾಯಣಪುರವಾದ ಗುತ್ತಲ ಕೇಶವದೇವರ ಸ್ಥಾನೀಕರಾದ ಪುರುಷೋತ್ತಮ ದೇವ ಮತ್ತು ನಂಬಿಪಿಳ್ಳೆ ಇವರುಗಳು, ಕೇಶವ ದೇವರ ದೇವದಾನಕ್ಕೆ ಸೇರಿದ್ದ ಊರು, ಮರಕಾಡನ್ನು ಉದ್ಭವಸರ್ವಜ್ಞಪುರವಾದ ಬೂದನೂರ ಮಹಾಜನ\-ಗಳಿಗೆ ನೀಡಿ, ಆ ಭೂಮಿಯಲ್ಲಿದ್ದ ಮರಕಾಡನ್ನು ಕಡಿದು, ಕೆರೆಯನ್ನು ನಿರ್ಮಿಸಿ ಕಾಲುವೆಯನ್ನು ತಂದುಕೊಂಡು ಸಂತಾನಗಾಮಿ\-ಯಾಗಿ ಉಪಭೋಗಿಸಲು ದತ್ತಿಯಾಗಿ ಬಿಡುತ್ತಾರೆ. ಈ ರೀತಿಯಾಗಿ ದತ್ತಿ ನೀಡಿದ ಭೂಮಿಯಲ್ಲಿ ಕೆರೆಯನ್ನು ಕಟ್ಟಿದಾಗ ಆ ಕೆರೆಯ ಹಿಂದೆ ಕೇಶವದೇವರ ಅಮೃತಪಡಿಗೆ ಮೂರು ಸಲಗೆ ಗದ್ದೆ ಮತ್ತು ಐದುನೂರು ಬೆದ್ದಲನ್ನು ನೀಡುವಂತೆಯೂ ವಿಧಿಸುತ್ತಾರೆ.\endnote{ ಎಕ 7 ಮಂ 56 ಹೊಸಬೂದನೂರು 1276} ಪಾಂಡ್ಯ ರಾಜನಾದ ತ್ರಿಭುವನ ಚಕ್ರವರ್ತಿ ಕೊನೇರಿಮ್ಮೈಕೊಣ್ಡಾಣ್​ ಎಂಬುವವನು, ಹದಿನೈದು ಕಳನಿ ಭೂಮಿಯಲ್ಲಿ ಕೆರೆಯನ್ನು ನಿರ್ಮಿಸುವಂತೆಯೂ, ಕಾಲುವೆಯನ್ನು ತೆಗೆಸುವಂತೆಯೂ, ಏರಿಯ ಮೊದಲ ಭಾಗದಲ್ಲಿ ಬಿತ್ತುವಟ್ಟನ್ನು ಬಿಡುವಂತೆಯೂ, ಮರದೂರು (ಮದ್ದೂರು) ಮಹಾಜನಗಳಿಗೆ ವಿನಂತಿ ಮಾಡಿಕೊಳ್ಳುತ್ತಾನೆ.\endnote{ ಎಕ 7 ಮ 9 ಮದ್ದೂರು 13ನೇ ಶ.}

ಮೂರನೆಯ ಬಲ್ಲಾಳನ ಕಾಲದಲ್ಲಿಯೂ ಕೂಡಾ ಜಿಲ್ಲೆಯಲ್ಲಿ ನಿರ್ಮಿತವಾದ ಅನೇಕ ಕೆರೆಗಳು ಶಾಸನೋಕ್ತವಾಗಿವೆ. ಅನಾದಿ ಅಗ್ರಹಾರ ಮಲ್ಲಿಕಾರ್ಜುನ ಪುರವಾದ ಗುತ್ತಲಿನ ತಾವರೆಕೆರೆಯ,\index{ತಾವರೆಕೆರೆ} ಹಿರಿಯ ತುಂಬಿನ ಕೆಳಗೆ ಕಟ್ಟಕಂಮಹದ ಕೊಡುಗೆಯಾಗಿ ವಿಶ್ವಣ್ಣನಿಗೆ ಗದ್ದೆಯನ್ನು ಬಿಡಲಾಗಿದೆ. ಬಹುಶಃ ಇದು ಈ ಕೆರೆಯ ಜೀರ್ಣೋದ್ಧಾರಕ್ಕೆ ಬಿಟ್ಟ ಕೊಡುಗೆ ಇರಬಹುದು.\endnote{ ಎಕ 7 ಮಂ 60 ಗುತ್ತಲು 1316} ಮುಮ್ಮಡಿ ಬಲ್ಲಾಳನ ಮಹಾಪ್ರಧಾನ ಆದಿಸಿಂಗೆಯ ದಂಡನಾಯಕನು\index{ಆದಿಸಿಂಗೆಯ ದಂಡನಾಯಕ} ರಾಣೀವಾಸ ಅಂದರೆ ಮುಮ್ಮಡಿ ಬಲ್ಲಾಳನ ರಾಣಿ ದೇಮಲಾದೇವಿಯ ಹೆಸರಿನಲ್ಲಿ ಕಲ್ಲಹಳ್ಳಿಯಲ್ಲಿ ದೇಮಲಮಹಾಸಮುದ್ರವೆಂಬ\index{ದೇಮಲಮಹಾಸಮುದ್ರ} ಕೆರೆಯನ್ನು ನಿರ್ಮಿಸುತ್ತಾನೆ.\endnote{ ಎಕ 6 ಕೃಪೇ 108 ವರಾಹನಾಥ ಕಲ್ಲಹಳ್ಳಿ 1334}

\section*{ಕೆರೆಗಳ ನಿರ್ಮಾಣ ಮತ್ತು ಜೀರ್ಣೋದ್ಧಾರ$-$ವಿಜಯನಗರದ ಕಾಲ}

“ವಿಜಯನಗರದ ಅರಸರು ಭಾರೀ ಕೆರೆಗಳನ್ನು ಮತ್ತು ಅಣೆಕಟ್ಟುಗಳನ್ನು ಕಟ್ಟಿ ಈ ಶಾಸ್ತ್ರದಲ್ಲಿ ತಮ್ಮ ಪರಿಣತೆಯನ್ನು ಶಿಖರಕ್ಕೆ ಒಯ್ದಿದ್ದಾರೆ. ಹಂಪಿಗೆ ನೀರನ್ನು ಒದಗಿಸುವ ತುರ್ತು ಅಣೆಕಟ್ಟು ಇಂದಿಗೂ ಉಪಯುಕ್ತ. ಇಂತಹ ಅನೇಕ ಅಣೆಕಟ್ಟುಗಳನ್ನು ವಿಜಯನಗರದ ರಾಜರು ತುಂಗಭದ್ರೆಗೂ ಮತ್ತು ಕಾವೇರಿ ನದಿಗೂ ಕಟ್ಟಿದರು. ಇವತ್ತಿಗೂ ಅವು ಜನಗಳಿಗೆ ನೀರನ್ನು ಕೊಡುತ್ತಿವೆ. ಕೃಷ್ಣದೇವರಾಯನ ಕಾಲ ಕೆರೆ ಕಟ್ಟುವ ಕಾಲದ ಉಚ್ಛ್ರಾಯ ಕಾಲ” ಎಂಬ ವಿದ್ವಾಂಸರ ಅಭಿಪ್ರಾಯಕ್ಕೆ ಜಿಲ್ಲೆಯ ಶಾಸನಗಳು ಇಂಬುಕೊಡುತ್ತವೆ.\endnote{ ದೀಕ್ಷಿತ್​, ಜಿ.ಎಸ್​., ಕೆರೆಗಳುಃಚಾರಿತ್ರಿಕ ಅಂಶಗಳು, ಕೆರೆ ನೀರಾವರಿ ನಿರ್ವಹಣೆ ಚಾರಿತ್ರಿಕ ಅಧ್ಯಯನ, ಪುಟ 2} ವಿಜಯನಗರದ ಆಳ್ವಿಕೆ ಕಾಲದಲ್ಲಿ ಕರ್ನಾಟದಲ್ಲಿ 77 ಕೆರೆಗಳು ನಿರ್ಮಾಣಗೊಂಡು, 19 ಕೆರೆಗಳು ಜೀರ್ಣೋದ್ಧಾರಗೊಂಡವು ಎಂಬುದು ನೀರಾವರಿ ತಜ್ಞರೊಬ್ಬರು ನೀಡಿರುವ ಅಂಕಿಅಂಶ.\endnote{ ಪ್ರಭಾಕರರಾವ್​, ಕೆ., ಮಧ್ಯಕಾಲೀನ ಕರ್ನಾಟಕದಲ್ಲಿ ಕೆರೆಗಳ ನಿರ್ವಹಣೆ, ಅದೇ, ಪುಟ 27}

ಎರಡನೆಯ ಬುಕ್ಕರಾಯನ ಸಚಿವ ಭಟ್ಟರ ಬಾಚಿಯಪ್ಪನು,\index{ಭಟ್ಟರ ಬಾಚಿಯಪ್ಪ} ಬುಕ್ಕರಾಯ ಸಮುದ್ರ,\index{ಬುಕ್ಕರಾಯ ಸಮುದ್ರ} ಕೀರ್ತಿ ಸಮುದ್ರ,\index{ಕೀರ್ತಿ ಸಮುದ್ರ} ಮಾಳವ್ವೆಯಕೆರೆ,\index{ಮಾಳವ್ವೆಯಕೆರೆ} ನಾಗವ್ವೆಯಕೆರೆ,\index{ನಾಗವ್ವೆಯಕೆರೆ} ಬಾಚಪ್ಪನಕೆರೆ\index{ಬಾಚಪ್ಪನಕೆರೆ} ಇವುಗಳನ್ನು ಕನ್ನೆಗೆರೆಯಾಗಿ\index{ಕನ್ನೆಗೆರೆ} ಕಟ್ಟಿಸಿ, ಚವುಡಪ್ಪನ ಕಾಲುವೆಯನ್ನು ನಿರ್ಮಿಸಿದನು. ಮೂರನೆಯ ನರಸಿಂಹನ ಪ್ರಧಾನ ಪೆರುಮಾಳೆ ದೇವನನ್ನು ಬಿಟ್ಟರೆ ಈ ರೀತಿ ಅಧಿಕಾರಿಯೊಬ್ಬ ಐದು ಕೆರೆಗಳನ್ನು ಕಟ್ಟಿಸಿ, ಕಾಲುವೆಗಳನ್ನು ನಿರ್ಮಿಸಿದ ಉದಾಹರಣೆಗಳು ವಿರಳವೆಂದು ಹೇಳಬಹುದು.\endnote{ ಎಕ 7 ಮ 93 ಅರುವನಹಳ್ಳಿ 1358}

ವರದೆಯ ನಾಯಕನು, ಕೆಲ್ಲಂಗೆರೆಯಲ್ಲಿ, ವರದರಾಜ ಸಮುದ್ರವೆಂಬ ಒಂದು ಕನ್ನೆಗೆರೆಯನ್ನು ಕಟ್ಟಿಸಿ ಮಲ್ಲಿಕಾರ್ಜುನ ದೇವರ ಪದಕ್ಕೆ ಸಮರ್ಪಿಸಿದನು.\endnote{ ಎಕ 7 ನಾಮಂ 58 ಕೆಳಗೆರೆ 15ನೇ ಶ.} ಕೆರಗೋಡ ಹಣುಗನಕೆರೆಯ ತೂಬಿನ ಕೆಳಗಿನ ಗದ್ದೆಯ ತೆರಿಗೆಯಲ್ಲಿ \textbf{ದಸವಂದವನ್ನು}\index{ದಸವಂದ} (ಹತ್ತನೆಯ ಒಂದು ಭಾಗ) ಕೆರೆಯ ಜೀರ್ಣೋದ್ಧಾರಕ್ಕೆ ಬಿಡುತ್ತಾರೆ. ಇದನ್ನು \textbf{“ಆರು ಕಸುಕೊಂಡರೂ ನಾಯ ಮಾಂಸ ತಿಂದಹಾಗೆ” }ಎಂದು ಹೇಳಿದೆ.\endnote{ ಎಕ 7 ಮಂ 38 ಕೆರಗೋಡು 15-16ನೇ ಶ.}

\newpage

ಕೃಷ್ಣದೇವರಾಯನ ಕಾಲದಲ್ಲಿ ಜಿಲ್ಲೆಯಲ್ಲಿ ಅನೇಕ ಕೆರೆಗಳು ನಿರ್ಮಾಣವಾಗಿರುವುದು ಶಾಸನೋಕ್ತವಾಗಿವೆ.\break ನಡಗಲ್​ಪುರದ ಶಾಸನದಲ್ಲಿ ಕೆರೆ ಶಿವಾಲಯಕ್ಕೆ\index{ಶಿವಾಲಯ} ಸಲ್ಲುವುದು ಎಂದು ಹೇಳಿದೆ. ಇದೇ ಶಾಸನದಲ್ಲಿ ಹಡುವಳಕೆರೆ,\index{ಹಡುವಳಕೆರೆ} ಬಡಗಣ ಕೆರೆಯ ಉಲ್ಲೇಖವಿದೆ.\endnote{ ಎಕ 7 ಮವ 44 ನಡಗಲ್​ಪುರ 1510} ಇಲ್ಲೇ ಇರುವ ಇನ್ನೊಂದು ಶಾಸನದಲ್ಲಿ ದೇವಿಕೆರೆಯ\index{ದೇವಿಕೆರೆ} ಉಲ್ಲೇಖವಿದೆ. ವಿಲಸಗೆರೆಗೆ\index{ವಿಲಸಗೆರೆ} ಕೊಡುಗೆಯ ಗದ್ದೆಯನ್ನು ಬಿಡಲಾಗಿದೆ. ಕೆರೆಗುಳ್ಳ ಸೀಮೆ ಗಂಗಣ್ಣನ ಕೊಡುಗೆಯ ಹೊಲ ಎಂಬ ಉಲ್ಲೇಖವಿದ್ದು, ಬಹುಶಃ ಕೆರೆಯ ನಿರ್ಮಾಣಕ್ಕೆ ಗಂಗಣ್ಣನು ತನ್ನ ಹೊಲವನ್ನು ಬಿಟ್ಟುಕೊಟ್ಟಿದ್ದು ಅದಕ್ಕೆ ಬದಲು ಆ ಕೆರೆಯ ಕೆಳಗೆ ಗದ್ದೆಯನ್ನು ಪಡೆದಂತೆ ತೋರುತ್ತದೆ.\endnote{ ಎಕ 7 ಮವ 45 ನಡಗಲ್​ಪುರ 1500}

ಒಡೆದು ಹೋಗಿದ್ದ ಕೆರೆಯನ್ನು ಮತ್ತೆ ಹೊಸದಾಗಿ ಕಟ್ಟಿಸಿ ತೂಬನ್ನು ಇಟ್ಟ ವಿಚಾರ ಮೇಲುಕೋಟೆಯ\break ಕೃಷ್ಣದೇವರಾಯನ ಕಾಲದ ಶಾಸನದಿಂದ ತಿಳಿದುಬರುತ್ತದೆ. ಚೆಲುವಪಿಳ್ಳೆ ದೇವರ ತಿರುವಿಡಿಯಾಟಕ್ಕೆ ಸೇರಿದ್ದ ಪುರ ಗ್ರಾಮದ ಕೆರೆಯು ಒಡೆದು ಖಿಲವಾಗಿತ್ತೆಂದೂ, ಅದನ್ನು ಒಡೆಯಾರ ತಿಬ್ಬಸೆಟ್ಟಿಯವರ ಮಗ ಲಕ್ಷ್ಮೀಪತಿಸೆಟ್ಟಿಯು\index{ಲಕ್ಷ್ಮೀಪತಿಸೆಟ್ಟಿ} 300 ಘಟ್ಟಿ ವರಹಗಳನ್ನು ಖರ್ಚುಮಾಡಿ, ಜೀರ್ಣೋದ್ಧಾರ ಮಾಡಿ, ವೊಡವನ್ನು ಕಟ್ಟಿಸಿ, ತೂಬನ್ನು ನಿಲ್ಲಿಸಿ, ಕಡುವನ್ನು ತೆಗೆಸಿ ಜೀರ್ಣೋದ್ಧಾರ ಮಾಡಿದನೆಂದು, ಕೆರೆಯ ನಿರ್ಮಾಣಕ್ಕೆ ಅವನು ಖರ್ಚು ಮಾಡಿದ ಹಣವನ್ನು ಪಡೆದುಕೊಳ್ಳದೇ, ಅದರಿಂದ ಮೇಲುಕೋಟೆ ದೇವಾಲಯದಲ್ಲಿ ತನ್ನ ತಂದೆಯ ಹೆಸರಿನಲ್ಲಿ ಪೂಜಾವ್ಯವಸ್ಥೆಗಳನ್ನು ಮಾಡಿದನೆಂದು ತಿಳಿದುಬರುತ್ತದೆ.\endnote{ ಎಕ 6 ಪಾಂಪು 135 ಮೇಲುಕೋಟೆ 1519} ವೊಡವು ಎಂದರೆ ಏರಿ ಅಥವಾ ಕೆರೆಯ ಕೋಡಿ ಎಂದು ಅರ್ಥ. ಕಡಹು\index{ಕಡಹು} ಎಂದರೆ ಜನಗಳು ದಿನನಿತ್ಯದ ಬಳಕೆಗೆ ಕೆರೆಯ ನೀರನ್ನು ಬಳಸಿಕೊಳ್ಳಲು ಇರುವ ಜಾಗ. ಇಲ್ಲಿಂದ ದೋಣಿಗಳಲ್ಲಿ ಕೆರೆಯನ್ನು ದಾಟಲೂ ಕೂಡಾ ವ್ಯವಸ್ಥೆ ಇರುತ್ತಿತ್ತು.

ವಿಜಯನಗರ ಕಾಲದಲ್ಲಿ ಕೆರೆಗಳ ನಿರ್ಮಾಣದ ಬಗೆಗಿನ ಒಂದು ಸ್ಪಷ್ಟ ಚಿತ್ರಣವನ್ನು ನಾಗಮಂಗಲ ತಾಲ್ಲೂಕು ದೊಂದೆಮಾದಿಹಳ್ಳಿ\index{ದೊಂದೆಮಾದಿಹಳ್ಳಿ} ಶಾಸನವು ನೀಡುತ್ತದೆ. ಅನಾದಿಅಗ್ರಹಾರ ಭಟ್ಟರತ್ನಾಕರವಾದ ನಾಗಮಂಲಗದ ಅಸೇಷ ಮಹಾಜನಂಗಳು, ಹೊಸಹಳ್ಳಿಯ ಜಂನಿಕೂಚಿಗಳ ಮಗ ವಿಠಣ್ಣನಿಗೆ ಒಂದು ಕೆರೆಯನ್ನು ಕಟ್ಟಿಸಿಕೊಡುವಂತೆ ಹೇಳಿ, ಒಂದು ಒಪ್ಪಂದ ಪತ್ರವನ್ನು ನೀಡುತ್ತಾರೆ. ಹೊಸಹಳ್ಳಿಯ (ಇಂದಿನ ದೊಂದೆಮಾದಿಹಳ್ಳಿ) ಮೂಡಣ ಹೊಲದೊಳಗೆ ಮಲ್ಲಿಗೆದುಡುಪಿನ ಮಾವಿನಹಳ್ಳವು,\index{ಮಾವಿನಹಳ್ಳ} ನವಿಲಹಳ್ಳವು ಕೂಡುವಲ್ಲಿ ಒಂದು ಕೆರೆಯನ್ನು ಕಟ್ಟುವಂತೆಯೂ, ಈ ಕೆರೆಯನ್ನು ಕಟ್ಟಿದುದಕ್ಕೆ ಕೆರೆಗೊಡಂಗೆಯ\index{ಕೆರೆಗೊಡಂಗೆ} ಮರ್ಯಾದೆಯಾಗಿ, ಮಾಯಿದೇವರಕೆರೆಯ\index{ಮಾಯಿದೇವರಕೆರೆ} ಕೆಳಗೆ ಕೋಡಿಯ ಹಂತದಿಂದ ತೂಬಿನ ಹಂತದವರೆಗೆ ಎಷ್ಟುಗದ್ದೆ ಇದೆ ಅಷ್ಟರಲ್ಲಿ ನಾಲ್ಕನೆ ಒಂದು ಭಾಗವನ್ನು, ಏರಿಯ ಮೊದಲಲಿ ಪಡುವಲಾಗಿ ಇರುವ ಗದ್ದೆಯಲ್ಲಿ ನಾಲ್ಕನೆಯ ಒಂದು ಭಾಗವನ್ನು ಸರ್ವಬಾಧಾಪರಿಹಾರವಾಗಿ, ಸರ್ವಮಾನ್ಯವಾಗಿ ಕೊಡುತ್ತಾರೆ. ಬಹುಶಃ ಈ ಮಾಯೀದೇವರ ಕೆರೆಯೇ ವಿಠಣ್ಣನು ಕಟ್ಟಿಸಿದ ಕೆರೆಯಾಗಿರಬಹುದು. ಅದರೆ ಈಗ ಆ ಕೆರೆ ಇಲ್ಲ, ಕೇವಲ ಶಾಸನ ಮಾತ್ರ ನಿಂತಿದೆ.\endnote{ ಎಕ 7 ನಾಮಂ 151 ದೊಂದೆಮಾದಿಹಳ್ಳಿ 1521}

ಅಚ್ಯುತರಾಯನ ಕಾಲದಲ್ಲಿ, ಕದ್ದಳಗೆರೆಯ ಹೊಸಕೆರೆಯೂ, ಕೃಷ್ಣದೇವವೊಡೆಯರ ಕೆರೆಯೂ ಒಡೆದು ಖಿಲವಾಗಿ\-ರಲು, ಹರಿಗಿಲ ಅಬ್ಬರಾಜಗಳ ಮಕ್ಕಳು ಪೆರಿರಾಜರು, 100 ಗದ್ಯಾಣವನ್ನು ಖರ್ಚು ಮಾಡಿ ಕದ್ದಳಗೆರೆಯ ಹೊಸಕೆರೆಯನ್ನೂ, 50 ಗದ್ಯಾಣ ಖರ್ಚು ಮಾಡಿ ಕೃಷ್ಣದೇವ ವೊಡೆಯರ ಕೆರೆಯ ವೊಡವುಗಳನ್ನು ಕಟ್ಟಿದರು. ಇದಕ್ಕಾಗಿ ಪೆರಿರಾಜನು ಖರ್ಚುಮಾಡಿದ ಹಣಕ್ಕೆ ಬದಲು ಶ‍್ರೀಭಂಡಾರದಿಂದ ಪ್ರತಿನಿತ್ಯ ದೇವರಿಗೆ ಪೂಜೆ ಮಾಡಿ ಪ್ರಸಾದ ಪಡೆದುಕೊಳ್ಳಲು 52 ಜನ ಶ‍್ರೀವೈಷ್ಣವರು ವ್ಯವಸ್ಥೆ ಮಾಡುತ್ತಾರೆ.\endnote{ ಎಕ 6 ಪಾಂಪು 138 ಮೇಲುಕೋಟೆ 1534}

ಬಳಗುಂದಿಯ ತಿಪ್ಪಯ್ಯನ ಮಗ ಕದರೆನಾಯಕನು\index{ಕದರೆನಾಯಕ} ಕಟ್ಟಿಸಿದ ಕೆರೆ ಎಂದು ನಾಗಮಂಗಲ ತಾಲ್ಲೂಕು, ಶಿವನಹಳ್ಳಿಯ ಬಂಡೆಯ ಮೇಲಿನ ಶಾಸನ ತಿಳಿಸುತ್ತದೆ. ಆದರೆ ಆ ಜಾಗದಲ್ಲಿ ಕೆರೆ ಇರುವುದಿಲ್ಲ.\endnote{ ಎಕ 7 ನಾಮಂ 21 ಶಿವನಹಳ್ಳಿ 1531} ರಾಮಚಂದ್ರ ಹೆಬ್ಬಾರುವ, ದುದ್ದದ ನಂಜಪ್ಪ, ತುರುದೇವರು ಈ ಮೂರು ಏಕಸ್ಥವಾಗಿ ದುದ್ದದ ಕೆರೆಯ\index{ದುದ್ದದ ಕೆರೆ} ಕೆಲಸಕ್ಕೆ ಮೂರು ಭಂಡಿಗೆ ಬೀಚನಹಳ್ಳಿಯ ತೆರಿಗೆಗಳನ್ನು ಬಿಡುತ್ತಾರೆ.\endnote{ ಎಕ 7 ಮಂ 20 ಬೀಚೇನಹಳ್ಳಿ 17ನೇ ಶ.} ಬಹುಶಃ ಹಟ್ಟಣದ ಕೆರೆಯ\index{ಹಟ್ಟಣದ ಕೆರೆ} ನಿರ್ವಹಣೆಗಾಗಿ ದುದ್ದ ಮತ್ತು ಹಟ್ಟಣದ ಮಹಾಜನಗಳು ಸೋವಿಸೆಟ್ಟಿಗೆ ಬೆಟ್ಟಹಳ್ಳಿ ಗ್ರಾಮದ ತೆರಿಗೆಗಳನ್ನು ಕೊಡುಗೆಯಾಗಿ ನೀಡಿದರೆಂದು ಊಹಿಸಬಹುದು.\endnote{ ಎಕ 7 ಮಂ 22 ಬೇವುಕಲ್ಲುಹಟ್ಟಣ 16ನೇ ಶ.}

\section*{ಕೆರೆಗಳ ನಿರ್ಮಾಣ ಮತ್ತು ಜೀರ್ಣೋದ್ಧಾರ - ಮೈಸೂರು ಒಡೆಯರ ಕಾಲ}

ಒಡೆಯರ ಕಾಲದಲ್ಲಿ ಅನೇಕ ಕೆರೆಗಳುನಿರ್ಮಾಣವಾದರೂ ಅವು ಹೆಚ್ಚಾಗಿ ಶಾಸನೋಕ್ತವಾಗಿಲ್ಲ. ದೇವ ಒಡೆಯರು ತಮ್ಮ ಶಿಷ್ಯೆ ಚಿಕ್ಕಿಯರ ಮಗ ಮುದ್ದಣ್ಣನು ದೇವರ ಕಟ್ಟೆ ಕಟ್ಟುವುದಕ್ಕೆ 15 ಗುಳಿಗೆ (ಬಹಶಃ ಕಂಠಿರಾಯ ವರಹ) ಯನ್ನು ನೀಡುತ್ತಾರೆ. ಜೊತೆಗೆ ಈ ಕೆರೆಯ ನಿರ್ಮಾಣಕ್ಕೆ ತಮ್ಮ ಶಿಷ್ಯ ಮುದ್ದಣ್ಣನ ಕಯ್ಯನ್ನು (ಹೊಲ) ತೆಗೆದುಕೊಂಡು ಅದಕ್ಕೆ ಹಣವನ್ನು ನೀಡುತ್ತಾರೆ.\endnote{ ಎಕ 7 ನಾಮಂ 139 ಹಾಲ್ತಿ 1605}

\section*{ಶಾಸನೋಕ್ತ ಕೆರೆಗಳು}

ಒಂದು ಊರು ಎಂದ ಮೇಲೆ ಒಂದು ಕೆರೆ ಇರಲೇಬೇಕಷ್ಟೆ. ಬಹುತೇಕ ಶಾಸನಗಳಲ್ಲಿ ಆ ಊರಿನ ಕೆರೆಯ ಅಥವಾ ಊರಿನ ಸುತ್ತಮುತ್ತ ಇದ್ದ ಕೆರೆಗಳ, ಕಟ್ಟೆಗಳ ಉಲ್ಲೇಖಗಳು ಬಂದೇ ಬರುತ್ತವೆ. ಒಂದೊಂದು ಊರಿನ ಶಾಸನದಲ್ಲಿ ಎರಡರಿಂದ, ಏಳೆಂಟು ಕೆರೆ ಕಟ್ಟೆಗಳ ಉಲ್ಲೇಖವಿದ್ದು ಅಂದಿನ ಜನರು ಕೆರೆಕಟ್ಟೆಗಳಿಗೆ ನೀಡುತ್ತಿದ್ದ ಮಹತ್ವದ ಅರಿವಾಗುತ್ತದೆ. ಹಳ್ಳಿಗಳ ಮತ್ತು ದಾನ ನೀಡಿದ ಭೂಮಿಯ ಎಲ್ಲೆಯನ್ನು(ಮೇರೆ) ಹೇಳುವಾಗ, ಈಗಾಗಲೇ ಇದ್ದ ಕೆರೆಯ ಕೆಳಗೆ ಭೂಮಿಯನ್ನು ದತ್ತಿ ನೀಡಿದಾಗ, ಈ ಕೆರೆಗಳನ್ನು ಉಲ್ಲೇಖಿಸಲಾಗಿದೆ. ಈ ಕೆರೆಗಳು ಶಾಸನದ ಕಾಲಕ್ಕಿಂತ ಹಿಂದೆಯೇ ನಿರ್ಮಿತವಾಗಿದ್ದ ಪ್ರಾಚೀನ ಕೆರೆಗಳಾಗಿವೆಯೆಂದು ಹೇಳಬಹುದು. ಈ ಕೆರೆಗಳ ನಿರ್ಮಾಣ ಕಾಲ ಯಾವುದು, ಅವುಗಳನ್ನು ನಿರ್ಮಿಸಿದವರು ಯಾರು ಎಂಬುದು ತಿಳಿದುಬರುವುದಿಲ್ಲ.

\section*{ಗಂಗರ ಕಾಲದ ಶಾಸನೋಕ್ತ ಕೆರೆಗಳು.}

ಶ‍್ರೀಪುರುಷನ ದೇವರಹಳ್ಳಿ ತಾಮ್ರಪಟಗಳಲ್ಲಿ, ಪಣ್ಯಂಗೆರೆ,\index{ಪಣ್ಯಂಗೆರೆ} ಬೆಳ್ಗಲ್ಲಿಗೆರೆಯ\index{ಬೆಳ್ಗಲ್ಲಿಗೆರೆ} ಒಳಗೆರೆಯಲ್ಲಿ ಹಳ್ಳವು ಕೂಡುವ ಜಾಗ, ಪೊನ್ಕೆವಿ ತಾಳ್ತುವಾಯರಾಕೆರೆ, ದುಂಡು ಸಮುದ್ರದ\index{ದುಂಡು ಸಮುದ್ರ} ಬಯಲುಗಳನ್ನು ಉಲ್ಲೇಖಿಸಿದೆ. ಬಹುಶಃ ಈ ದುಂಡು ಸಮುದ್ರವನ್ನು ಶಾಸನೋಕ್ತ ನೀರ್ಗುಂದ ಯುವರಾಜ ದುಂಡುವೇ ಕಟ್ಟಿಸಿರಬಹುದು.\endnote{ ಎಕ 7 ನಾಮಂ 149 ದೇವರಹಳ್ಳಿ 776-77} ಹುಳ್ಳೇನಹಳ್ಳಿ ತಾಮ್ರಪಟಗಳಲ್ಲಿ ಮೂಡಾಯ ಒಳಗೆರೆಯ, ಮೊರಡೆ ಬೆದಿಕೆರೆ, ಕಡವಿಗೆರೆಗಳ ಉಲ್ಲೇಖವಿದೆ.\endnote{ ಎಕ 7 ಮಂ 14 ಹುಳ್ಳೇನಹಳ್ಳಿ 8 ನೇ ಶ.} ಗಂಜಾಮ್ ತಾಮ್ರ ಶಾಸನದಲ್ಲಿ ಸೆಟ್ಟಿಗೆರೆ, ಮೂಡಗೆರೆ, ಬೊಜ್ಜೆಗೆರೆಯ ಒಳಂಗೆರೆ”ಗಳ ಉಲ್ಲೇಖವಿದೆ.\endnote{ ಎಕ 6 ಶ‍್ರೀಪ 66 ಗಂಜಾಮ್ 8ನೇ ಶ.} ಹಳ್ಳೆಗೆರೆ ತಾಮ್ರಪಟಗಳಲ್ಲಿ ಕಿರುಕೊಣ್ಣಿನ್ದ ತಟಾಕ,\index{ಕಿರುಕೊಣ್ಣಿನ್ದ ತಟಾಕ} ಪರ್ಗ್ಗೊಣ್ಣಿನ್ದ ತಟಾಕ,\index{ಪರ್ಗ್ಗೊಣ್ಣಿನ್ದ ತಟಾಕ} ಕಿರುಬಳ್ಳಿಯೂರು ಕೆರೆ,\index{ಕಿರುಬಳ್ಳಿಯೂರು ಕೆರೆ} ಸೆಳ್ಳೆಕೆರೆ, ನಿಡುವೆತ್ತಕೆರೆಗಳನ್ನು ಉಲ್ಲೇಖಿಸಿದೆ.\endnote{ ಎಕ 7 ಮಂ 35 ಹಳ್ಳೆಗೆರೆ 713}

ಆತಕೂರು ಶಾಸನದಲ್ಲಿ “ಪಿರಿಯಕೆರೆಯ ಕೆಳಗೆ, ಮೞ್ತಿಕಾಲಂಗಳೊಳ್​” ಎರಡು ಖಂಡುಗ ಮಣ್ಣನ್ನು(ಗದ್ದೆ) ದತ್ತಿಬಿಡಲಾಗಿದೆ. ಇದು ಆತಕೂರ ಹಿರಿಯಕೆರೆ\index{ಹಿರಿಯಕೆರೆ} ಮತ್ತು ಅದರಿಂದ ಹೊರಟ ಮತ್ತಿಮರಗಳಿಂದ ಕೂಡಿ ಕಾಲುವೆಯನ್ನು ಸೂಚಿಸುತ್ತದೆ.\endnote{ ಎಕ 7 ಮ 42 ಆತಕೂರು 949} ಮೂರನೇ ಬಲ್ಲಾಳನ ಕಾಲದ ಶಾಸನದಲ್ಲೂ ಕೂಡಾ ಆತಕೂರ ಹಿರಿಯಕೆರೆಯ ಉಲ್ಲೇಖವಿದೆ.\endnote{ ಎಕ 7 ಮ 44 ಆತಕೂರು 1279}

ಆರಣಿಯ ಕೆರೆಗೆ,\index{ಆರಣಿಯ ಕೆರೆ}\endnote{ ಎಕ 7 ನಾಮಂ 99 ಆರಣಿ 972} ಕಿರಿಯ ಬೆಳ್ಗುಂದದ ಕೆರೆಗೆ,\index{ಬೆಳ್ಗುಂದದ ಕೆರೆ}\endnote{ ಎಕ 7 ನಾಮಂ 117 ಬಿಳಗುಂದ 10ನೇ ಶ.} ರಾವಂದೂರಿನ ಕೆರೆಗೆ,\index{ರಾವಂದೂರಿನ ಕೆರೆ}\endnote{ ಎಕ 7 ಮವ 17 ರಾವಂದೂರು 9-10ನೇ ಶ.} ಬಿತ್ತವಟ್ಟವನ್ನು ಬಿಟ್ಟಿರುವ ಉಲ್ಲೇಖಗಳಿವೆ. 11ನೇ ಶತಮಾನದ, ಕನ್ನಂಬಾಡಿಯ ಶಾಸನದಲ್ಲಿ ಸಿವರಾಯಕೆರೆ,\index{ಸಿವರಾಯಕೆರೆ} ಗೊರವರಕೆರೆ,\index{ಗೊರವರಕೆರೆ} ಆಸನಹಾಳ(ಕೆರೆಯ) ದೊಡ್ಡೇರಿಯ ಉಲ್ಲೇಖವಿದೆ.\endnote{ ಎಕ 6 ಪಾಂಪು 43 ಕನ್ನಂಬಾಡಿ 10-11ನೇ ಶ.}

\section*{ಹೊಯ್ಸಳರ ಕಾಲದ ಶಾಸನೋಕ್ತ ಕೆರೆಗಳು}

ವಿಷ್ಣುವರ್ಧನನ ಕಾಲದ ಮಾಳಗೂರು ಶಾಸನದಲ್ಲಿ “ಊರುಂಬ ಹಿರಿಯ ಕೆರೆಯ”\index{ಊರುಂಬ ಹಿರಿಯ ಕೆರೆ} ಉಲ್ಲೇಖವಿದೆ.\endnote{ ಎಕ 6 ಕೃಪೇ 66 ಮಾಳಗೂರು 1117} ಊರುಂಬ ಎಂದರೆ, ಊರಿನವರು ತಮ್ಮ ದಿನಬಳಕೆಗೆ ಬಳಸುವ ಮತ್ತು ನೀರಾವರಿಗೆ ಬಳಸುವ ಎಂದು ಅರ್ಥೈಸಬಹುದು. ಈ ಕೆರೆಯ ಹಿಂದೆ ಸುಮಾರು 200-300 ಎಕರೆ ಇಂದಿಗೂ ನೀರಾವರಿಯಾಗುತ್ತಿದೆ. ಕಂಬದಹಳ್ಳಿ ಶಾಸನದಲ್ಲಿ, ಪಿರಿಯಕೆರೆಯ ತೆಂಕಣತೂಂಬಿನಿಂದ, ಬಡಗಣ ಹಳ್ಳದಿಂದ ತೆಂಕಕ್ಕೆ, ಕೌಂಗಿನ ತೋಟ ಗದ್ದೆಗಳು ಇದ್ದವೆಂದು ಹೇಳಿದೆ.\endnote{ ಎಕ 7 ನಾಮಂ 33 ಕಂಬದಹಳ್ಳಿ 1118} ಇದು ಕಂಬದಹಳ್ಳಿ ಮತ್ತು ಬಿಂಡಿಗನವಿಲೆಯ ಮಧ್ಯೆ ಇಂದಿಗೂ ಇರುವ ದೊಡ್ಡಕೆರೆಯಾಗಿರಬಹುದು. ಹೊಸಹೊಳಲು ಶಾಸನದಲ್ಲಿ ಮೂಡಕೆರೆ ಮತ್ತು ಕೇಣಿಯ ಸಮೀಪದ ಕಡವದಕೊಳದ ಕೆರೆಯ ಕೆಳಗಿನ ಗದ್ದೆಗಳ ಉಲ್ಲೆಖವಿದೆ.\endnote{ ಎಕ 6 ಕೃಪೇ 3 ಹೊಸಹೊಳಲು 1118} ಸಾಸಲಿನ ಕೆರೆಯ ಹಿಂದೆ ಭೋಗೇಶ್ವರ ದೇವರಿಗೆ ಗದ್ದೆ ಬಿಟ್ಟ ಉಲ್ಲೇಖವಿದೆ.\endnote{ ಎಕ 6 ಕೃಪೇ 59 ಸಾಸಲು 1121}

ಹಿರಿಯಕಳಲೆಯ ಶಾಸನದಲ್ಲಿ ಪಿರಿಯಕೆರೆಯ ತುಂಬಿನ(ತೂಬಿನ) ಗದ್ದೆ, ದೇವರ ಮುಂದಣ ಕೆರೆಯೊಳಗಿನ ಗದ್ದೆಗಳ ಉಲ್ಲೇಖವಿದೆ.\endnote{ ಎಕ 6 ಕೃಪೇ 73 ಹಿರೀಕಳಲೆ 12ನೇ ಶ.} ವಿಷ್ಣುವರ್ಧನನ ನಾಗಮಂಗಲ ಶಾಸನದಲ್ಲಿ, ಹೊಸವಳ್ಳಿಯ ಕೆರೆ, ಅರಿಕನಕಟ್ಟದ ಯೆಮ್ಮೆಗೆರೆ, ಮತ್ತಿಯಕೆರೆ, ನಾಗಮಂಗಲದ ಹಿರಿಯಕೆರೆಗಳನ್ನು ಉಲ್ಲೇಖಿಸಿದೆ.\endnote{ ಎಕ 7 ನಾಮಂ 7 ನಾಗಮಂಗಲ 1134} ಈ ಪೈಕಿ ನಾಗಮಂಗಲದ ಹಿರಿಕೆರೆ ಈಗಲೂ ಉಳಿದಿದೆ. ಉಳಿದ ಕೆರೆಗಳನ್ನು ಗುರುತಿಸಲು ಆಗುತ್ತಿಲ್ಲ.

\textbf{ನಾಗಮಂಗಲ ತಾಲ್ಲೂಕು ಲಾಲನಕೆರೆ ಶಾಸನದಲ್ಲಿ ಕೊತ್ತನಕೆರೆ, ಮತ್ತಿಯಕೆರೆ, ದಾವಂಣನ ಕೆರೆ, ಬೀಚೆಯನಕೆರೆ, ದೇವರಕೆರೆ, ಹಿರಿಯಕೆರೆ, ಕಬ್ಬಿನಕೆರೆ\index{ಕಬ್ಬಿನಕೆರೆ} ಎಂಬ ಏಳು ಕೆರೆಗಳನ್ನು ಹೆಸರಿಸಿದೆ.}\endnote{ ಎಕ 7 ನಾಮಂ 61 ಲಾಲನಕೆರೆ 1138} ಇಷ್ಟೊಂದು ಕೆರೆಗಳನ್ನು ಹೆಸರಿಸುವ ಶಾಸನ ಇದೊಂದೇ ಎಂದು ಹೇಳಬಹುದು. ಕಬ್ಬಿನ ಕೆರೆಯ ಕೆಳಗೆ ಕಬ್ಬನ್ನು ಬೆಳೆಯುತ್ತಿದ್ದರೆಂದು ಊಹಿಸಬಹುದು. ಇದೇ ಊರಿನ ಶಾಸನಗಳಲ್ಲಿ ಹಿರಿಯಕೆರೆ ಮತ್ತು ಮತ್ತಿಯಕೆರೆ,\index{ಮತ್ತಿಯಕೆರೆ} ಹೆಗ್ಗಡೆ ಕೆರೆಗಳ\index{ಹೆಗ್ಗಡೆ ಕೆರೆ} ಉಲ್ಲೇಖವಿದೆ.\endnote{ ಎಕ 7 ನಾಮಂ 62 ಲಾಲನಕೆರೆ 1218}\textbf{ಅಂದರೆ ಈ ಊರಿನಲ್ಲಿ ಒಟ್ಟು 10 ಕೆರೆಗಳಿದ್ದು, ಬಹಳ ಸಮೃದ್ಧವಾದ ಹಾಗೂ ದೊಡ್ಡ ಊರಾಗಿತ್ತೆಂದು ಹೇಳಬಹುದು}. ಆದುದರಿಂದ ಈ ಊರನ್ನು \textbf{“ಕಲುಕಣಿ ನಾಡಿಗೆ ಶಿರೋಮಣಿಯಂತಿಪ್ಪ ಲಾಳನಕೆರೆ”} ಎಂದು ಶಾಸನದಲ್ಲಿ ಹೊಗಳಿದೆ.

ದಡಿಗ(ದಡಿಗನಕೆರೆ) ಶಾಸನದಲ್ಲಿ, ಮರಿಯಾನೆಸಮುದ್ರದ\index{ಮರಿಯಾನೆಸಮುದ್ರ} ಬಯಲು, ಮಳೆಹಳ್ಳಿಯ ಮುಂದಣ ಕಿರುಕೆರೆ,\break ಕೋಡಿಹಳ್ಳಿಯ ಮುಂದಣ ಕಿರುಕೆರೆ, ಹಿರಿಯಕೆರೆಯ ಕೆಳಗಣ ಅಡಕೆಯ ತೋಟಗಳನ್ನು ದತ್ತಿಯಾಗಿ ಬಿಡಲಾಗಿದೆ. ಮರಿಯಾನೆ ಸಮುದ್ರವು ಭರತ ಬಾಹುಬಲಿ ದಂಡನಾಯಕರ ತಂದೆಯ ಹೆಸರಿನಲ್ಲಿ ನಿರ್ಮಿತವಾದ ಕೆರೆಯಾಗಿರಬಹುದು. ದಡಗದ ಈ ಕೆರೆಯು ಇಂದಿಗೂ ಬಹಳ ವಿಶಾಲವಾಗಿ ಹರಡಿರುವ ದೊಡ್ಡ ಕೆರೆಯಾಗಿದ್ದು, ಇದೇ ಮರಿಯಾನೆಸಮುದ್ರ.\endnote{ ಎಕ 7 ನಾಮಂ 68 ದಡಗ 12ನೇ ಶ.} ಆದರೆ ಈ ಕೆರೆಯಲ್ಲಿ ಈಗ ಹೂಳು ತುಂಬಿದ್ದು, ಸುತ್ತಲೂ ಒತ್ತುವರಿಯಾಗಿದೆ. ಹುಬ್ಬನಹಳ್ಳಿ ಶಾಸನದಲ್ಲಿ ಹಿರಿಯ ಕೆರೆಯನ್ನು ಕಟ್ಟಿಸಿದ ಮತ್ತು ನೇರಳೆಕೆರೆಯ\index{ನೇರಳೆಕೆರೆ} ಉಲ್ಲೇಖವಿದೆ. ನೇರಳಕೆರೆಯು ಇಂದು ಸಣ್ಣ ಕಟ್ಟೆಯಾಗಿದ್ದು, ಸುತ್ತಲೂ ಬಹುಸಂಖ್ಯೆ ನೇರಳೆ, ಮಾವಿನ ಮರಗಳಿದ್ದವು. ಪಕ್ಕದಲ್ಲಿ ಜೀರ್ಣವಾದ ಮಾಕೇಶ್ವರ ದೇವಾಲಯವಿದೆ. \endnote{ ಎಕ 6 ಕೃಪೇ 62 ಹುಬ್ಬನಹಳ್ಳಿ 1140} ಆರಣಿಯ ಶಾಸನದಲ್ಲಿ ಗೆದನಕೆರೆ,\endnote{ ಎಕ 7 ನಾಮಂ 100 ಆರಣಿ 1141} ಕಸಲಗೆರೆ ಶಾಸನದಲ್ಲಿ ಕನಕಗಟ್ಟದ ಕೆರೆ, ಮತ್ತಿಯಕೆರೆ, ಕೇದಗೆಗೆರೆಗಳ\index{ಕೇದಗೆಗೆರೆ} ಉಲ್ಲೇಖವಿದೆ.\endnote{ ಎಕ 7 ನಾಮಂ 169 ಕಸಲಗೆರೆ 1142}

ಒಂದನೆಯ ನರಸಿಂಹನ ಕಾಲದ ಅನೇಕ ಶಾಸನಗಳಲ್ಲಿ ಕೆರೆಗಳ ಉಲ್ಲೇಖ ಕಂಡುಬರುತ್ತದೆ. ಶಣಬ ಶಾಸನದಲ್ಲಿ \textbf{ಯಾದವನಾರಾಯಣ ಚತುರ್ವೇದಿ ಮಂಗಲದ ಯಾದವ ಸಮುದ್ರ\index{ಯಾದವ ಸಮುದ್ರ} ಕೆರೆಯ ಉಲ್ಲೇಖವಿದೆ.}\endnote{ ಎಕ 6 ಪಾಂಪು 122 ಸಣಬ 12ನೇ ಶ.} ಮೈಸೂರು ಒಡೆಯರ ಶಾಸನದಲ್ಲಿ ಇದನ್ನು \textbf{“ರಾಮಾನುಜಾಂಘ್ರಿ ಶ‍್ರೀ ತೀರ್ಥ ತಟಾಕ”}\index{ರಾಮಾನುಜಾಂಘ್ರಿ ಶ‍್ರೀ ತೀರ್ಥ ತಟಾಕ} ಎಂದು ಕರೆಯಲಾಗಿದ್ದು, ಇದು ರಾಮಾನುರ ಕಾಲದಲ್ಲೇ ನಿರ್ಮಿತವಾದ ಕೆರೆ ಎಂಬುದನ್ನು ಸೂಚಿಸುತ್ತದೆ.\endnote{ ಎಕ 6 ಪಾಂಪು 99 ತೊಣ್ಣೂರು 1722} ಈ ಕೆರೆಯನ್ನು ವಿಷ್ಣುವರ್ಧನ ಮತ್ತು ಒಂದನೇ ನರಸಿಂಹನ ಕಾಲದಲ್ಲಿ ನಿರ್ಮಿಸಲಾಗಿದೆ.\endnote{ ಸ್ವಾಮಿ, ಲ.ನ., ಕೆರೆಕಟ್ಟೆಗಳು, ತೊಣ್ಣೂರು, ಸಂಃ ಡಾ.ಸಿ.ಮಹದೇವ, ಪುಟ 133} ಇದೇ ಇಂದಿನ ಪ್ರಸಿದ್ಧ ತೊಣ್ಣೂರು ಕೆರೆ. ಈ ಕೆರೆಯ ಉಲ್ಲೇಖ ಇರುವ ಶಾಸನಗಳು ಇವೆರಡೇ ಎಂದು ಹೇಳಬಹುದು. ಸಣಬ ಗ್ರಾಮವು ಈ ಕೆರೆಯ ಒಳಗೆರೆಯ ಹಿಂದೆ ಇದೆ. ಈ ಊರಿನವರು ತೊಣ್ಣೂರಿಗೆ ಬರಬೇಕಾದರೆ, ಹರಿಗೋಲಿನ ಮೂಲಕ ಈ ಕೆರೆಯ ಮೇಲೆ ಬರುತಿದ್ದರು. ಈ ಕೆರೆಯ ನೀರು ತಿಳಿಯಾಗಿದ್ದು ಇಡೀ ವರ್ಷ ತುಂಬಿರುತ್ತಿತ್ತು. ಆದರೆ ಈಚೆಗೆ ನೀರಿನ ಮೂಲಗಳು ತಪ್ಪಿಹೋಗಿ, ನೀರು ಕಡಿಮೆಯಾಗುತ್ತಿದೆ. ಈಚೆಗೆ ಚಟ್ಟಮಗೆರೆ ಸುರಂಗದ ಮೂಲಕ ಹೇಮಾವತಿ ನಾಲೆಯ ನೀರನ್ನು ಇದಕ್ಕೆ ಬಿಡಲಾಗಿದೆ. ಇನ್ನೊಂದು ವಿಶೇಷವೆಂದರೆ, ಈ ಕೆರೆಯಲ್ಲಿ ಒಂದು ದಡದ ಕಡೆಗೆ ಮರಳು ಸಂಗ್ರಹವಾಗುತ್ತದೆ. ಇದನ್ನು ಮಳ್ಳಕವ್ವ ಎನ್ನುತ್ತಾರೆ. ಇನ್ನೊಂದು ಕಡೆಗೆ ಸೆತ್ತೆ ಸೊಪ್ಪು, ಕಸಕಡ್ಡಿಗಳೂ ಸಂಗ್ರಹವಾಗುತ್ತದೆ. ಹೊಯ್ಸಳರು ಈ ಕೆರೆಯನ್ನು ನಿರ್ಮಿಸಿದಾಗ ತೂಬನ್ನು ಇಡುವುದನ್ನು ಮರೆತರೆಂದೂ ಅಥವಾ ಇಟ್ಟಿದ್ದ ತೂಬು ಸಾಲದೇ ಕೆರೆಯ ಏರಿಯು ಒಡೆಯುವ ಸ್ಥಿತಿ ಬಂದಿತೆಂದೂ, ಆಗ ರಾಮಾನುಜಾಚಾರ್ಯರು\index{ರಾಮಾನುಜಾಚಾರ್ಯರು} ತಮ್ಮ ಕೈಲಿದ್ದ ದಂಡಿಯಿಂದ ಬಂಡೆಯನ್ನು ಕೊರೆದು ಕೆರೆಯನೀರು ಹೊರಹೋಗುವಂತೆ ಮಾಡಿದರೆಂದೂ ಹೇಳುತ್ತಾರೆ.\endnote{ ಈ ಅಂಶಗಳು ಲ.ನ.ಸ್ವಾಮಿ ಅವರ ಲೇಖನದಲ್ಲಿ ಉಲ್ಲೇಖವಾಗಿಲ್ಲ.} 1890ರ ಮೋಜಿಣಿಯ ನಕ್ಷೆಯಲ್ಲಿ ತೊಂಡನೂರಿನಲ್ಲಿ ಬರುವ ಕೆರೆಕಟ್ಟೆಗಳ, ಕೊಳಗಳು ಬಾವಿಗಳನ್ನು ಗುರುತಿಸಲಾಗಿದ್ದು, ಆ ಪಟ್ಟಿಯಲ್ಲಿ ಯಾದವಸಮುದ್ರ ಅಥವಾ ಮೋತಿತಲಾಬ್​\index{ಮೋತಿತಲಾಬ್​} ಎಂದು ಇದನ್ನು ಗುರುತಿಸಲಾಗಿದೆ.\endnote{ ಸ್ವಾಮಿ, ಲ.ನ., ಕೆರೆಕಟ್ಟೆಗಳು, ತೊಣ್ಣೂರು, ಪುಟ 130}

\newpage

ಟಿಪ್ಪು ಸುಲ್ತಾನನು ಈ ಕೆರೆಯನ್ನು ನೋಡಿ ಇದು ‘ಮೋತಿ ತಲಾಬ್’\index{ಮೋತಿ ತಲಾಬ್} ಅಂದರೆ ಮುತ್ತಿನಕೆರೆ ಎಂದು ಕರೆದನಂತೆ. ಒಂದು ಮುತ್ತನ್ನು ನೀರಿನಲ್ಲಿ ಹಾಕಿದರೆ, ಅದು ಮೇಲಿನಿಂದ ಕಾಣುವಷ್ಟು ಈ ಕೆರೆಯ ನೀರು ತಿಳಿಯಾಗಿತ್ತೆಂದು ಈ ಕಾರಣದಿಂದ ಇದಕ್ಕೆ ಮೋತಿ ತಲಾಬ್ ಎಂದು ಕರೆಯಲಾಯಿತೆಂದು ಸ್ಥಳೀಯ ಹಿರಿಯರು ಹೇಳುತ್ತಾರೆ. ಬ್ರಿಟಿಷರು ದಂಡೆತ್ತಿ ಬಂದ ಸಮಯದಲ್ಲಿ ಅವರಿಗೆ ನೀರು ಸಿಗದೇ ಇರಲಿ ಎಂದು ಟಿಪ್ಪು ಸುಲ್ತಾನ್ ಈ ಕೆರೆಯನ್ನು ಒಡೆಸಿದನೆಂದು, ಆಗ ಈ ಕೆರೆಯ ಏರಿಯ ಮೇಲಿದ್ದ ಹೊಯ್ಸಳೇಶ್ವರ ದೇವಾಲಯ ನಾಶವಾಯಿತೆಂದು ಸ್ಥಳೀಯರು ಹೇಳುತ್ತಾರೆ. ಆ ನಂತರ ಇದನ್ನು ಪುನರ್ ನಿರ್ಮಿಸಲಾಯಿತೆಂದು ಹೇಳುತ್ತಾರೆ. ಆದರೆ ಅದಕ್ಕೆ ಆಧಾರಗಳಿಲ್ಲ.

\vskip 2pt

ನಾಗಮಂಗಲ ತಾಲ್ಲೂಕು ಬೋಗಾದಿ ಶಾಸನದಲ್ಲಿ ತಾಳೆಯಕೆರೆ,\index{ತಾಳೆಯಕೆರೆ} ಸೆಟ್ಟಿಯಹಳ್ಳಿಯ ಕೆರೆ, ದೊಡ್ಡಿನ ಕೆರೆಗಳ ಉಲ್ಲೇಖ\-ವಿದೆ.\endnote{ ಎಕ 7 ನಾಮಂ 183 ಬೋಗಾದಿ 1144} ಯಲ್ಲಾದಹಳ್ಳಿಯ ಶಾಸನದಲ್ಲಿ ದೇವರಕೆರೆ, ಮಾವಿನಕೆರೆ, ದೊಡ್ಡಕಟ್ಟ, ಕೇತನಕಟ್ಟಗಳನ್ನು ಉಲ್ಲೇಖಿಸಿದೆ.\endnote{ ಎಕ 7 ನಾಮಂ 64 ಯಲ್ಲಾದಹಳ್ಳಿ 1145} ಮಂಡ್ಯ ತಾಲ್ಲೂಕು ಬೇಲೂರಿನ ತಮಿಳು ಶಾಸನದಲ್ಲಿ ಆ ಊರ ಕೆರೆಯ ಪೆರಿಯೇರಿಯ ಕೆಳಗೆ 20 ಖಂಡುಗ ಭೂಮಿಯನ್ನು ಬಿಟ್ಟಂತೆ ಹೇಳಿದೆ.\endnote{ ಎಕ 7 ಮಂ 70 ಬೇಲೂರು 1162} ಎರಡನೆಯ ಬಲ್ಲಾಳನ ಕಾಲದ ನಾಗಮಂಗಲ ಶಾಸನದಲ್ಲಿ ಚೆನ್ನಕೇಶವದೇವರ ದೇವದಾನಕ್ಕೆ ಬಿಟ್ಟ ಹಾಲತಿ ಮತ್ತು ಹೊನ್ನಗೊಂಡನಹಳ್ಳಿಯ ಮೇರೆಗಳನ್ನು ಹೇಳುವಾಗ, ಪಡುವಸಾಗರ, ಅರಿಕನಕಟ್ಟದ ಕಣಿಗಿಲೆಯಕಟ್ಟೆ,\index{ಕಣಿಗಿಲೆಯಕಟ್ಟೆ} ಮತ್ತಿಯಕೆರೆ, ಚೊಕ್ಕಚಾರ್ಯಕಟ್ಟೆ, ಚೆಳೆಯಕಟ್ಟೆ, ಸೆಣೆಯಕಟ್ಟೆ, ಮೊದಲಿಹಳ್ಳಿಯಕೆರೆ\index{ಮೊದಲಿಹಳ್ಳಿಯಕೆರೆ} ಈ ರೀತಿ ಎಂಟು ಕೆರೆಕಟ್ಟೆಗಳನ್ನು ಉಲ್ಲೇಖಿಸಲಾಗಿದೆ.\endnote{ ಎಕ 7 ನಾಮಂ 1 ನಾಗಮಂಗಲ 1171-73} ಹಳ್ಳಿಗಳಲ್ಲಿ ಬಟ್ಟೆಯನ್ನು ಒಗೆಯುವುದು ಎನ್ನುವುದಕ್ಕೆ ಬಟ್ಟೆ ಸೆಣೆಯುವುದು ಎಂದು ಹೇಳುತ್ತಾರೆ. ಸೆಣೆಯಕಟ್ಟೆ ಬಟ್ಟೆಗಳನ್ನು ಒಗೆಯುತ್ತಿದ್ದ ಕಟ್ಟೆ ಇರಬಹುದು.

\vskip 2pt

ನಾಗಮಂಗಲ ತಾಲ್ಲೂಕು ಕಂಬದಹಳ್ಳಿ ಶಾಸನದಲ್ಲಿ, ಹಿರಿಯಕೆರೆಯ ತೆಂಕಣ ತುಂಬಿನ ಮೊದಲೇರಿಯಲ್ಲಿ ಗದ್ದೆಯನ್ನು ದತ್ತಿ ಬಿಡಲಾಗಿದೆ.\endnote{ ಎಕ 7 ನಾಮಂ 29 ಕಂಬದಹಳ್ಳಿ 1174} ಇಂದಿಗೂ ಕೂಡಾ ಈ ಕೆರೆಯು ಈ ಕಡೆ ಕಂಬದಹಳ್ಳಿಯಿಂದ, ಆ ಕಡೆ ಬಿಂಡಿಗನವಿಲೆಯವರೆಗೂ ಇದ್ದ ಭಾರೀ ಕೆರೆ ಆಗಿತ್ತು. ಈಗ ಅತಿಕ್ರಮಣಕ್ಕೆ ಒಳಗಾಗಿ ಹೂಳು ತುಂಬಿದೆ. ಇಲ್ಲಿರುವ ಗಂಗರಾಜನ ಶಾಸನದಲ್ಲಿ ಪಂದೂರಕ್ಕಿಯ ಕೆರೆಯ ಉಲ್ಲೇಖವಿದೆ. ಹಟ್ಟಣ ಶಾಸನದಲ್ಲಿ ಸೋವಿಸೆಟ್ಟಿಯು ಕಟ್ಟಿಸಿದ ಮೂರು ಕೆರೆಗಳ ಜೊತೆಗೆ, ಬಳ್ಳೆಯಕೆರೆ,\index{ಬಳ್ಳೆಯಕೆರೆ} ಜತಗರ ಕೆರೆಗಳ\index{ಜತಗರ ಕೆರೆ} ಉಲ್ಲೇಖವಿದೆ. \textbf{ಇದರಿಂದಾಗಿ ಒಂದೇ ಊರಿನಲ್ಲಿ ಐದು ಕೆರೆಗಳಿದ್ದುದು ತಿಳಿದುಬರುತ್ತದೆ.}\endnote{ ಎಕ 7 ನಾಮಂ 118 ಹಟ್ಟಣ 1178}

\vskip 2pt

ಇಮ್ಮಡಿ ಬಲ್ಲಾಳನ ಕಾಲದ ಅಳೀಸಂದ್ರ ಶಾಸನದಲ್ಲಿ, ಅಣವಸಮುದ್ರದ\index{ಅಣವಸಮುದ್ರ} ಹಿರಿಯಕೆರೆ, ಹೊಲೆಗೆರೆ, ಹಾರುವಗೆರೆ,\index{ಹಾರುವಗೆರೆ} ಹಡವಳಿತಿಯ ಕೆರೆ, ಕಿರುಕೆರೆ, ಮಲ್ಲಿನಗೆರೆ, ಹೀಗೆ ಒಟ್ಟು ಆರು ಕೆರೆಗಳನ್ನು ಉಲ್ಲೇಖಿಸಲಾಗಿದೆ.\endnote{ ಎಕ 7 ನಾಮಂ 72 ಅಳೀಸಂದ್ರ 1183} ಇದೇ ಊರಿನ ಇಮ್ಮಡಿ ಬಲ್ಲಾಳನ ಕಾಲದ ಶಾಸನದಲ್ಲಿ ಹಗವಮಗೆರೆಗೆ ಸೇರಿದ, ಕದಲಂಬಳ್ಳೆಯ ಕೆರೆ, ತುಂಬಿನಕೆರೆಗಳ ಉಲ್ಲೇಖವಿದೆ.\endnote{ ಎಕ 7 ನಾಮಂ 168 ಕಸಲಗೆರೆ 1190} ತೊಳಂಚೆಯ(ತೊಳಸಿ) ಗ್ರಾಮದ ಹಿರಿಯ ಕೆರೆ ಮತ್ತು ಕಿರುಕೆರೆಯ ಕೆಳಗೆ ಗದ್ದೆಯನ್ನೂ ಸಿದ್ಧನಾಥ ದೇವರಿಗೆ ದತ್ತಿಯಾಗಿ ಬಿಡಲಾಗಿದೆ.\endnote{ ಎಕ 6 ಕೃಪೇ 48 ತೊಣಚಿ 1191} ಈ ಕೆರೆಯಲ್ಲಿ ಹಂಸಗಳಿದ್ದು, ಅದಕ್ಕೆ ಈ ಊರಿಗೆ ತೊಳಂಚೆ\index{ತೊಳಂಚೆ} ಎಂದು ಹೆಸರು ಬಂದಿರಬಹುದು. ತೊಳಸಿ ದೇವಾಲಯ(ಅಂಕನಹಳ್ಳಿ) ಕಡೆಯಿಂದ ತೊಳಸಿ ಊರಿನವರೆಗೂ ಈ ಕೆರೆ ಹರಡಿದ್ದು, ಭಾರೀ ಕೆರೆಯಾಗಿದ್ದ ಇದು ಅತಿಕ್ರಮಣಕ್ಕೆ ಒಳಗಾಗಿ ಹೂಳುತುಂಬಿದೆ. ಅರಕೆರ ಅಗ್ರಹಾರದ ಬಳಿ ಇದ್ದ ‘ಬಳ್ಳೀ ಮಡವಯ’ ಎಂದರೆ ಬಳ್ಳೀ ಮಡುವಿನ ಉಲ್ಲೇಖವಿದೆ. ಮಡುಗಳು ಒಂದು ರೀತಿಯ ಸಣ್ಣ ಸರೋವರ ಅಥವಾ ಸ್ವಾಭಾವಿಕವಾದ ಕೆರೆಗಳು ಅಥವಾ ದೊಡ್ಡ ಕಟ್ಟೆಗಳು. ಇವುಗಳಲ್ಲಿ ಯಾವಾಗಲೂ ನೀರಿರುತ್ತಿದ್ದು, ಬಹಳ ಆಳವಾಗಿರುತ್ತಿದ್ದವು.\endnote{ ಎಕ 6 ಶ‍್ರೀಪ 108 ಅರಕೆರೆ 13ನೇ ಶ.} ಹೆಮ್ಮನಹಳ್ಳಿಯ ಕದರಿಯೂರ ಕೆರೆ,\index{ಕದರಿಯೂರ ಕೆರೆ}\endnote{ ಎಕ 7 ಮ 48 ಹೆಮ್ಮನಹಳ್ಳಿ 13ನೇ ಶ.} ಹಾದರವಾಗಿಲ ಬೆಳತೂರ ಕೆರೆ,\endnote{ ಎಕ 7 ಮವ 92 ಹೊಸಹಳ್ಳಿ 13-14ನೇ ಶ} ದುಂಡೇನಹಳ್ಳಿಯ ಊರ ಮುಂದಣ ಹಿರಿಯ ಕೆರೆ,\endnote{ ಎಕ 7 ಮ 39 ದುಂಡೇನಹಳ್ಳಿ 13ನೇ ಶ.} ಶಾಸನೋಕ್ತವಾಗಿವೆ.

\vskip 2pt

ಬಸರಾಳಿನ ಸಮೀಪದ ಭೀಮನಹಳ್ಳಿ ಶಾಸನದಲ್ಲಿ, ಹಿರಿಯಕೆರೆ, ಬೆದ್ದವ್ವೆಯಕೆರೆ, ಹಾಡುವನಕೆರೆ ಮತ್ತು ಹೊಲಗೆರೆಯ\index{ಹೊಲಗೆರೆ} ಉಲ್ಲೇಖವಿದೆ.\endnote{ ಎಕ 7 ನಾಮಂ 173 ಭೀಮನಹಳ್ಳಿ 1230} ಬಸರಾಳಿನ ಗುಜ್ಜವ್ವೆನಾಯಕಿತ್ತಿಯ ಕೆರೆಯ ಜೊತೆಗೆ ಬಸುರಿವಾಳದ ಹಿರಿಯಕೆರೆ, ಹಡವನಹಳ್ಳಿ ಕೆರೆ, ಎಕ್ಕೆಹಟ್ಟಿಯಕೆರೆಗಳ ಉಲ್ಲೇಖವಿದೆ.\endnote{ ಎಕ 7 ಮಂ 29 ಬಸರಾಳು 1234} ಎಕ್ಕಟೆಯ ಕೆರೆಯ ಉಲ್ಲೇಖ ಇಲ್ಲೇ ಇರುವ ವಿಜಯನಗರ ಕ್ರಿ.ಶ.1507ರ ಶಾಸನದಲ್ಲೂ ಇದೆ.\endnote{ ಎಕ 7 ಮಂ 32 ಬಸರಾಳು 1504}

\vskip 2pt

ಸೋಮೇಶ್ವರನ ಕಾಲದಲ್ಲಿ, ಎಮ್ಮದೂರಹಳ್ಳಿಯನ್ನು ಪಟ್ಟಣವನ್ನಾಗಿ ಮಾಡಲು ಬಳಗಾರ ಮಲ್ಲಸೆಟ್ಟಿಗೆ ಆ ಊರ ಮಲ್ಲಿಗೆಗೆರೆಯನ್ನು ಹೊರತಾಗಿ ಅನೇಕ ತೆರಿಗೆಗಳನ್ನು ದತ್ತಿ ಬಿಡುತ್ತಾರೆ.\endnote{ ಎಕ 7 ಮಂ 71 ಕನ್ನಲ್ಲಿ 1251} ಕೆರೆಗಳು ಊರ ಆಸ್ತಿಯಾಗಿದ್ದು ಅದನ್ನು ಇತರ ಉದ್ದೇಶಗಳಿಗೆ ನೀಡುತ್ತಿರಲಿಲ್ಲವೆಂಬುದು ಇದರಿಂದ ತಿಳಿಯುತ್ತದೆ. ಮಳವಳ್ಳಿ ತಾ. ಗವುಡಗೆರೆ ಶಾಸನದಲ್ಲಿ ಕುಂಬಗೆರೆ\index{ಕುಂಬಗೆರೆ} ಮತ್ತು ದೇವಿಗೆರೆಗಳ ಉಲ್ಲೇಖವಿದೆ.\endnote{ ಎಕ 7 ಮವ 23 ಗೌಡಗೆರೆ 1253} ಕುಂಬಗೆರೆಯು ಕುಂಬಾರರು ನಿರ್ಮಿಸಿದ ಕೆರೆಯಾಗಿರಬಹುದು.

ವೈದ್ಯನಾಥಪುರ ಶಾಸನದಲ್ಲಿ ಹದಲಿಕೆರೆ, ಬೇಡರಹಳ್ಳಿ ಕೆರೆಗಳ ಉಲ್ಲೇಖವಿದೆ.\endnote{ ಎಕ 7 ಮ 69 ವೈದ್ಯನಾಥಪುರ 1261} ಭೈರಾಪುರದ\break ರೇಕವ್ವೆ ದಂಡನಾಯಕಿಯ ಶಾಸನದಲ್ಲಿ ಭೈರವಪುರದ ಹಿರಿಯಕೆರೆ, ಬೊಮ್ಮಿಯರಕೆರೆಯ ಉಲ್ಲೇಖವಿದೆ. ಬಹುಶಃ ರೇಕವ್ವೆ ದಂಡನಾಯಕಿಯು ಈ ಅಗ್ರಹಾರವನ್ನು ಮಾಡಿದಾಗ, ಈ ಕೆರೆಗಳನ್ನು ಕಟ್ಟಿಸಿರಬಹುದೆಂದು ಊಹಿಸಬಹುದು.\endnote{ ಎಕ 6 ಕೃಪೇ 98 ಭೈರಾಪುರ 1267} ಭೈರಾಪುರದ ಕೆರೆಗೆ ಸುತ್ತಮುತ್ತಲ ಗುಡ್ಡಗಳಿಂದ ನೀರು ಹರಿದು ಬಂದು ಭರ್ತಿಯಾಗಿರುತ್ತಿತ್ತು. ಆದರೆ ಈಗ ಆ ಕೆರೆಯು ಬತ್ತುತ್ತಿದೆ.

ಸಿಂದಘಟ್ಟದಲ್ಲಿ\index{ಸಿಂದಘಟ್ಟ} ಇಂದು ಬಹಳ ದೊಡ್ಡ ಕೆರೆ ಇದ್ದು ಹೇಮಾವತಿ\index{ಹೇಮಾವತಿ} ನಾಲೆ ನೀರಿನಿಂದ ಭರ್ತಿಯಾಗುತ್ತದೆ. ಶಾಸನದಲ್ಲಿ “ಬವೆಯ ಕೆರೆ” ಎಂಬ ಅಸ್ಪಷ್ಟ ಉಲ್ಲೇಖವಿದ್ದು ಅದೇ ಈ ಕೆರೆಯ ಹೆಸರಿರಬಹುದು.\endnote{ ಎಕ 6 ಕೃಪೇ 90 ಸಿಂದಘಟ್ಟ 1299} ಅನಾದಿ ಅಗ್ರಹಾರ ರಾಯಸಮುದ್ರವಾದ ಹೊಸಹೊಳಲಿನ ಶಾಸನದಲ್ಲಿ ಬೇಲದ ಕೆರೆಯ ಉಲ್ಲೇಖವಿದೆ. ಈಗಿನ ಹೊಸಹೊಳಲಿನ ಭಾರೀ ಕೆರೆಯೇ ರಾಯಸಮುದ್ರ ಕೆರೆಯಾಗಿರಬಹು.\endnote{ ಎಕ 6 ಕೃಪೇ 8 ಹೊಸಹೊಳಲು 1306} ಹುಸ್ಕೂರು ಶಾಸನದಲ್ಲಿ ಪೆರ್ಬಾಣನಹಳ್ಳಿಯ ಹಿರಿಯ ಕೆರೆಯ ಉಲ್ಲೇಖವಿದೆ.\endnote{ ಎಕ 7 ಮವ 31 ಹುಸ್ಕೂರು 1313} ದೊಡ್ಡಗರುಡನಹಳ್ಳಿ ಶಾಸನದಲ್ಲಿ ಚಿಕ್ಕಗರುಡನಹಳ್ಳಿಯ ಕೆರೆಯ ಉಲ್ಲೇಖವಿದೆ.\endnote{ ಎಕ 7 ಮಂ 25 ದೊಡ್ಡಗರುಡನಹಳ್ಳಿ 1275}

ಮಹಾಪ್ರಧಾನ ದಂಡನಾಯಕ ಪೆರಮಾಳೆ ದೇವನ\index{ಪೆರಮಾಳೆ ದೇವ} ಬೆಳ್ಳೂರು ಶಾಸನದಲ್ಲಿ, ಬೆಳ್ಳೂರು ಹಿರಿಯಕೆರೆ, ಅಲ್ಲಾಳಸಮುದ್ರ, ಹಳ್ಳಿಕೊಪ್ಪದ ಕೆರೆ, ಕುಂಬಾರಗುಂಡಿಗಳ ಉಲ್ಲೇಖವಿದೆ.\endnote{ ಎಕ 7 ನಾಮಂ 74 ಬೆಳ್ಳೂರು 1271} ಇನ್ನೊಂದು ಶಾಸನದಲ್ಲಿ, ಬಿಲ್ಲಬೆಳಗುಂದದ ಹಿರಿಯಕೆರೆ, ಬೆಳ್ಳೂರಿನ ಅವ್ವೆಯರ ಕೆರೆಗಳ ಉಲ್ಲೇಖವಿದ್ದು, ಪೆರುಮಾಳೆ ದೇವನೇ ಯಜಮಾನನಾಗಿ ನಿಂತು, ಬೆಳ್ಳೂರು ಮತ್ತು ಅದರ ಕಾಲುವಳ್ಳಿಗಳ ಸೀಮೆಯಲ್ಲಿ ಕೆರೆ ಕಟ್ಟೆ ಕಾಲುವೆ ಮುಖ್ಯವಾದುದನ್ನು ಕಟ್ಟಿಸಬೇಕೆಂದು ಹೇಳಿದೆ.\endnote{ ಎಕ 7 ನಾಮಂ 76 ಬೆಳ್ಳೂರು 1284 ಜೂನ್​ 25} ಇವನ ಇನ್ನೊಂದು ಶಾಸನದಲ್ಲಿ, ಗಾಣಸಮುದ್ರದ ಹಿರಿಯಕೆರೆಯ ಉಲ್ಲೇಖವಿದೆ.\endnote{ ಎಕ 7 ನಾಮಂ 73 ಬೆಳ್ಳೂರು 1284 ಅಕ್ಟೋಬರ್​ 11} ಅಲ್ಲಾಳ ಸಮುದ್ರವನ್ನು ದಾಸನಕೆರೆ\index{ದಾಸನಕೆರೆ} ಎಂದು ಕರೆಯುತ್ತಾರೆ.

\section*{ವಿಜಯನಗರದ ಕಾಲದ ಶಾಸನೋಕ್ತ ಕೆರೆಗಳು}

ಒಂದನೇ ಬುಕ್ಕರಾಯನ ಕಾಲದ ಶಾಸನಗಳಲ್ಲಿ, ಮಯಿಲನಹಳ್ಳಿ ಗ್ರಾಮದ ಕೆರೆ ಹಾಗೂ ಅದರ ಕೆಳಗಿನ ಕುಂಬಾರಕಟ್ಟೆ,\endnote{ ಎಕ 6 ಪಾಂಪು 164 ಮೇಲುಕೋಟೆ 1369}\break ಹೊಳಲಿನ ಬಲಿಯಕೆರೆ, ಉಲ್ಲೇಖವಿದೆ.\endnote{ ಎಕ 7 ಮಂ 8 ಹೊಳಲು 14-15ನೇ ಶ.} ಎರಡನೇ ಹರಿಹರನ ಬೊಪ್ಪಸಮುದ್ರ ಶಾಸನದಲ್ಲಿ ಊರಿಂದ ಮೂಡಣ ಹಿರಿಯಕೆರೆಯ ಉಲ್ಲೇಖವಿದೆ.\endnote{ ಎಕ 7 ಮ 110 ಬೊಪ್ಪಸಮುದ್ರ 1388}

ಇಮ್ಮಡಿ ದೇವರಾಯನ ಮಾಚಲಘಟ್ಟ ಶಾಸನದಲ್ಲಿ, ಮಾಚನಕಟ್ಟದ ಹಿರಿಯಕೆರೆ ಹಾಗೂ ಅದರ ಕೆಳಗಿದ್ದ ಗದ್ದೆ, ತೆಂಗಿನ ಮತ್ತು ಅಡಕೆಯ ತೋಟಗಳ ಉಲ್ಲೇಖವಿದೆ.\endnote{ ಎಕ 7 ನಾಮಂ 179 ಮಾಚಲಘಟ್ಟ(ಬೇಚಿರಾಕ್​) 1426} ಈ ಭಾಗದ ಅಡಿಕೆಯು ಇತ್ತೀಚಿನವರೆಗೂ ಬಹಳ ಪ್ರಸಿದ್ಧವಾಗಿತ್ತು. ಪ್ರೌಢದೇವರಾಯನ ಶ‍್ರೀರಂಗಪಟ್ಟಣ ತಾಮ್ರಶಾಸನದಲ್ಲಿ, ಚಂದಿಗಾಲು\index{ಚಂದಿಗಾಲು} ಗ್ರಾಮದ ಎಲ್ಲೆಗಳನ್ನು ಹೇಳುವಾಗ, ಹತ್ತಿಯಕಟ್ಟೆ, ನಗುಲನಹಳ್ಳಿಯ\index{ನಗುಲನಹಳ್ಳಿ} ಕೆರೆ, ವಾಯುವ್ಯಕ್ಕೆ ಕಾವೇರಿ ಹೊಳೆ,\index{ಕಾವೇರಿ ಹೊಳೆ} ಆದಿಗೌಡನಕೆರೆ, ಮೇನಾಪುರದ ಕಾನಿಕೆರೆ ಇವುಗಳನ್ನು ಉಲ್ಲೇಖಿಸಲಾಗಿದೆ.\endnote{ ಎಕ 6 ಶ‍್ರೀಪ 25 ಶ‍್ರೀರಂಗಪಟ್ಟಣ 1430} ಮೇಲುಕೋಟೆ ಶಾಸನದಲ್ಲಿ ಹೊಸಹಳ್ಳಿ ಗ್ರಾಮದ ಹಿರಿಯಕೆರೆ,\endnote{ ಎಕ 6 ಪಾಂಪು 152 ಮೇಲುಕೋಟೆ 1432} ಮತ್ತು ದೊಡ್ಡ ಅರಸಿನಕೆರೆಯ ಶಾಸನದಲ್ಲಿ, ಕಾಮಿಗೆರೆ ಮತ್ತು ಹಿರಿಯರಸಿನ ಕೆರೆಯ ಉಲ್ಲೇಖವಿದೆ. ಶಾಸನವು ಈ ಊರಿನ ಕೆರೆಯ ಹಿಂದಿನ ಗದ್ದೆಯಲ್ಲಿದ್ದು, ಈಗ ಈ ಕೆರೆಯನ್ನು ಧರ್ಮಕೆರೆ ಎಂದು ಕರೆಯುತ್ತಾರೆ.\endnote{ ಎಕ 7 ಮ 131 ದೊಡ್ಡ ಅರಸಿನಕೆರೆ 1437}

ತಿಮ್ಮಣ್ಣ ದಂಡನಾಯಕನ ಕ್ರಿ.ಶ. 1458ರ ನೆಲಮನೆ ಶಾಸನದಲ್ಲಿ, ಬಲ್ಲೇನಹಳ್ಳಿ ಮತ್ತು ಯಲವದಹಳ್ಳಿ ಈ ಎರಡು ಗ್ರಾಮಗಳಿಗೆ ಸೇರಿದ ರಂಗಸಮುದ್ರ ಕೆರೆಯ ಉಲ್ಲೇಖವಿದ್ದು, ಆ ಕೆರೆಯ ಕೆಳಗಣ ಗದ್ದೆಗಳು ಚೆಲುವಪಿಳ್ಳೆರಾಯರ ಭಂಡಾರಕ್ಕೆ ಸೇರಿತ್ತೆಂದು, ಈ ಕೆರೆಯ ನೀರು ವ್ಯವಸಾಯಕ್ಕೆ (ನೀರುವರಿಯು) ಬಲ್ಲೇನಹಳ್ಳಿಯ ಸೀಮೆಯವರೆಗಿನ ಗದ್ದೆಗೆ ಹೋಗುತ್ತಿತ್ತೆಂದು ಹೇಳಿದೆ.\endnote{ ಎಕ 6 ಪಾಂಪು 179 ಮೇಲುಕೋಟೆ 1458} ರಂಗಸಮುದ್ರವನ್ನು\index{ರಂಗಸಮುದ್ರ} ತಿಮ್ಮಣ್ಣದಂಡನಾಯಕನ ಹೆಂಡತಿ ರಂಗಮಾಂಬೆಯು\index{ರಂಗಮಾಂಬೆ} ಕಟ್ಟಿಸಿರಬಹುದು. ಇದೇ ಕಾಲದ ತಿಮ್ಮಣ್ಣ ದಂಡನಾಯಕನ ನೆಲಮನೆ ಶಾಸನದಲ್ಲಿ, ನಾಗನಾಗನಕಟ್ಟೆ, ವೀರಣ್ಣನ ಕುಂಟೆ, ಜೋಗಿಯಕಟ್ಟೆ, ಚಿಕ್ಕೋಜನಕಟ್ಟೆ, ಲೋಕಪಾವನೆಯ ಸಾಗರ\index{ಲೋಕಪಾವನೆಯ ಸಾಗರ} ಇವುಗಳನ್ನು ಉಲ್ಲೇಖಿಸಿದೆ. ಲೋಕಪಾವನಿ ನದಿಯು ಈ ಊರಿಗೆ ಸಮೀಪದಲ್ಲಿ ಹರಿಯುತ್ತಿದ್ದು, ಅದಕ್ಕೆ ಲೋಕಪಾವನೆಯ ಸಾಗರ ಎಂಬ ಅಣೆಕಟ್ಟೆಯನ್ನು ನಿರ್ಮಿಸಿರಬಹುದು. \textbf{ಇಲ್ಲಿ ಕಟ್ಟೆ, ಕುಂಟೆ, ಕೆರೆ, ಸಾಗರ ಇವುಗಳನ್ನು ಪ್ರತ್ಯೇಕವಾಗಿ ಹೇಳಿದೆ}.\endnote{ ಎಕ 6 ಶ‍್ರೀಪ 93 ನೆಲಮನೆ 1458} ಇದೇ ಕಾಲದ ಸುಬ್ಬರಾಯನಕೊಪ್ಪಲು ಶಾಸನದಲ್ಲಿ ಹಿರಿಯಕೆರೆಯ ಉಲ್ಲೇಖವಿದೆ.\endnote{ ಎಕ 7 ನಾಮಂ 46 ಸುಬ್ಬರಾಯನಕೊಪ್ಪಲು 1460} ಮಳವಳ್ಳಿ ಶಾಸನದಲ್ಲಿ ತಮ್ಮಡಿಹಳ್ಳಿಯ ಹಿರಿಯಕೆರೆಯ ಉಲ್ಲೇಖವಿದೆ.\endnote{ ಎಕ 7 ಮವ 3 ಮಳವಳ್ಳಿ 1456}

ಮೇಲುಕೋಟೆಯ ನಾಚಿಯಾರಮ್ಮನ ಶಾಸನದಲ್ಲಿ ನಳನಹಳ್ಳಿಯ ಪರಾಂಕುಶಸಮುದ್ರಕೆರೆ,\index{ಪರಾಂಕುಶಸಮುದ್ರಕೆರೆ} ಕಡಬದಕೆರೆ,\break ಮಳಲನಕೆರೆ, ಲೊಕಿಯಮಲನಕೆರೆಗಳ ಉಲ್ಲೇಖವಿದೆ.\endnote{ ಎಕ 6 ಪಾಂಪು 163 ಮೇಲುಕೋಟೆ 1469} ಆದರೆ ಈ ಗ್ರಾಮಗಳು ಹಾಗೂ ಕೆರೆಗಳು ಎಲ್ಲಿವೆ ಎಂಬುದು ತಿಳಿದುಬರುವುದಿಲ್ಲ. ಇದೇ ಕಾಲದ ಶಾಸನಗಳಲ್ಲಿ, ದೇವಲಾಪುರದ ಹಿರಿಯಕೆರೆ,\endnote{ ಎಕ 7 ನಾಮಂ 158 ದೇವಲಾಪುರ 1472} ಸುಜ್ಜಲೂರು ತಾಮ್ರಶಾಸನದಲ್ಲಿ, ಆಲುಗೊಡು ಅಗ್ರಹಾರದ ಎಲ್ಲೆಗಳನ್ನು ಹೇಳುವಾಗ, ಹಿರಿಯಕೆರೆ, ಕಟ್ಟೆಕೆರೆ, ಕೊಳಿಗೆರೆ, ಕಾಳಿಗೆರೆ, ಹಿರಿಯಕೆರೆ,\endnote{ ಎಕ 7 ಮವ 139 ಸುಜ್ಜಲೂರು 1473}\break ದೊಡ್ಡಅರಸಿನಕೆರೆಯ ಶಾಸನದಲ್ಲಿ ದಂಮಿಗೆರೆ,\endnote{ ಎಕ 7 ಮ 129 ದೊಡ್ಡ ಅರಸಿನಕೆರೆ 15ನೇ ಶ.} ಚಾಮಲಾಪುರದ ಕೆರೆ,\endnote{ ಎಕ 7 ಮಂ 42 ಚಾಮಲಾಪುರ 1477} ದುಗ್ಗನಹಳ್ಳಿಯ ಮರಹಳ್ಳಿ ಕೆರೆಗಳ,\endnote{ ಎಕ 7 ಮವ 14 ಮಗ್ಗನಹಳ್ಳಿ 15ನೇ ಶ.} ಉಲ್ಲೇಖವು ಕಂಡು ಬರುತ್ತದೆ.

ಕೃಷ್ಣದೇವರಾಯನ ಕಾಲದ ಶಾಸನಗಳಲ್ಲಿ ಜಿಲ್ಲೆಯ ಅನೇಕ ಕೆರೆಗಳು ಶಾಸನೋಕ್ತವಾಗಿವೆ. ನಡಗಲ್​ಪುರ ಶಾಸನದಲ್ಲಿ ವಿಲಸಗೆರೆ, ದೇವಿಗೆರೆೆ,\endnote{ ಎಕ 7 ಮವ 45 ನಡಗಲಪುರ 1500}, ಕೊರಟಿಹಳ್ಳಿಯ ಕೆರೆ,\endnote{ ಎಕ 7 ಮವ 44 ನಡಗಲಪುರ 1510} ಮಂಡ್ಯ ತಾಮ್ರಶಾಸನದಲ್ಲಿ, ಹಾಳೆಹಳ್ಳಿಯ ಕಟ್ಟೊಬ್ಬೆಹಳ್ಳ,\index{ಕಟ್ಟೊಬ್ಬೆಹಳ್ಳ} ತಂಮಡಿಗಟ್ಟೆಯ ಕಾಲುವೆ, ಕಲ್ಲಹಳ್ಳಿಯ ಒಳಗೆರೆ, ತಂಡಸಹಳ್ಳಿಯ ಒಳಗೆರೆ, ಕಿರುಗೆರೆ, ಅತ್ತಣಿದ ಕಟ್ಟೆ, ಸಾತನೂರು ಹೊಸಕಟ್ಟೆಯ ಒಳಗೆರೆ, ಯಡಲಗೆರೆ(ಅದಲಗೆರೆ) ಕೆರೆ ಕಟ್ಟೆಗಳು,\endnote{ ಎಕ 7 ಮಂ 7 ಮಂಡ್ಯ 1516} ಕೆರಗೋಡು ಶಾಸನದ ಹೊಲ್ಲಿಗನಕಟ್ಟೆ(ಇಂದಿನ ಹುಣಗನಕಟ್ಟೆ),\endnote{ ಎಕ 7 ಮಂ 37 ಕೆರೆಗೋಡು 1520} ಉಲ್ಲೇಖವಿದೆ. ಈ ಹೊಲ್ಲಿಗನಕಟ್ಟೆಯ ಜೀರ್ಣೋದ್ಧಾರ ನಿರ್ವಹಣೆಗೆ \textbf{ಭೂದಸವಂದದ}\index{ಭೂದಸವಂದ} ಹಣವನ್ನು ದತ್ತಿ ಬಿಡಲಾಗಿದೆ. “ಇದನ್ನು ಯಾರು ಕಸಿದುಕೊಂಡರೂ ಅವರು ನಾಯ ಮಾಂಸವನ್ನು ತಿಂದಹಾಗೆ, ಕತ್ತೆ... ತಿಂದಹಾಗೆ” ಎಂದು ಹೇಳಿದೆ.\endnote{ ಎಕ 7 ಮಂ 38 ಕೆರಗೋಡು 15-16ನೇ ಶ.} ಮೇಲುಕೋಟೆಯ ಶಾಸನದಲ್ಲಿ ಭರತೆಪುರ ಕೆರೆ ಮತ್ತು ವಸಂತಪುರ ಕೆರೆಗಳ\index{ವಸಂತಪುರ ಕೆರೆ} ಉಲ್ಲೇಖವಿದೆ. ಶಾಸನೋಕ್ತ ವಸಂತರಾಯನು ಈ ಕೆರೆಗಳನ್ನು ನಿರ್ಮಿಸಿರಬಹುದು. ವಸಂತಪುರ ಎಂಬ ಊರು ಈಗಲೂ ಇದೆ.\endnote{ ಎಕ 6 ಪಾಂಪು 132 ಮೇಲುಕೋಟೆ 1530}

ಅಚ್ಯುತರಾಯ ಹಾಗೂ ನಂತರದ ಕಾಲದ ವಿಜಯನಗರದ ಶಾಸನಗಳಲ್ಲೂ ಅನೇಕ ಕೆರೆಗಳು ಶಾಸನೋಕ್ತವಾಗಿವೆ. ನೆಟ್ಟಕಲ್ಲು ಶಾಸನದಲ್ಲಿ ಮಾಯಣ್ಣನಪುರದ ಕೆರೆ,\endnote{ ಎಕ 7 ಮವ 86 ನೆಟ್ಟಕಲ್ಲು 1532} ಬ್ಯಾಲದಕೆರೆ ಶಾಸನದಲ್ಲಿ ಚಿಬ್ಬನೇಕಟ್ಟೆ, ಸಿಂದಘಟ್ಟ ಕೆರೆಗಳ ಉಲ್ಲೇಖವಿದೆ.\endnote{ ಎಕ 6 ಕೃಪೇ 99 ಬ್ಯಾಲದಕೆರೆ 1532} ಸದಾಶಿವರಾಯನ ಹೊನ್ನೇನಹಳ್ಳಿ ತಾಮ್ರ ಶಾಸನದಲ್ಲಿ ಹೊಡುಕೆಕಟ್ಟೆ ತಟಾಕ(ಕೆರೆ), ಮಲ್ಲಯ್ಯನಹಳ್ಳಿ ಕೆರೆ, ಮಂಚನಹಳ್ಳಿ ಕೆರೆ, ತಿಗುಳನ ಕೆರೆ,\index{ತಿಗುಳನ ಕೆರೆ} ದೇವಸಮುದ್ರ, ತೊಂಡೇಹಳ್ಳಿಯ ಮಲ್ಲೀದೇವಿಕಟ್ಟೆ, ವೆಂಕಟಾದ್ರಿಸಮುದ್ರ\index{ವೆಂಕಟಾದ್ರಿಸಮುದ್ರ} ತಟಾಕ ಎಂಬ \textbf{ಎಂಟು ಕೆರೆಗಳ} ಉಲ್ಲೇಖವಿದೆ.\endnote{ ಎಕ 6 ಕೃಪೇ 64 ಸಂತೇಬಾಚಹಳ್ಳಿ 1553} ಸಂತೇಬಾಚಹಳ್ಳಿ ಶಾಸನದಲ್ಲಿ ಲೋಕನಹಳ್ಳಿಯ ಕೆರೆಯ ಉಲ್ಲೇಖವಿದೆ.\endnote{ ಎಕ 7 ನಾಮಂ 107 ಹೊನ್ನೇನಹಳ್ಳಿ} ಬಿಂಡಿಗನವಿಲೆಯ ಹತ್ತಿರ ರಾಮಚಂದ್ರಾಪುರ ಅಗ್ರಹಾರದ ಕೆರೆಯ ದೊಡ್ಡ ತೂಬಿನ ಮೇಲೆ \textbf{‘ಈ ಕೆರೆ ರಾಮಚಂದ್ರಹೆಬ್ಬಾರುವನ ಧರ್ಮ’} ಎಂದು ಹೇಳಿದೆ.\endnote{ ಎಕ 7 ನಾಮಂ 47 ರಾಮಚಂದ್ರಾಪುರ 16ನೇ ಶ.} ನಾಗಮಂಗಲ ತಾಲ್ಲೂಕು ದೇವರಹಳ್ಳಿ ಶಾಸನದಲ್ಲಿ ಸಿಣಗಾಣಕಟ್ಟೆಯ ಕೆಳಗೆ ತಿರುಮಲದೇವರಿಗೆ ಹೊಲವನ್ನು ದತ್ತಿಯಾಗಿ ನೀಡಿದೆ.\endnote{ ಎಕ 7 ನಾಮಂ 145 ದೇವರಹಳ್ಳಿ 16ನೇ ಶ.} ಕೆಸ್ತೂರು ಶಾಸನದಲ್ಲಿ, ಕೆಸ್ತೂರು ಕೆರೆ ಮತ್ತು ವರದರಾಜನ ಕೆರೆಯನ್ನು ಉಲ್ಲೇಖಿಸಿದೆ.\endnote{ ಎಕ 7 ಮ 40 ಕೆಸ್ತೂರು 16ನೇ ಶ.}

\section*{ಮೈಸೂರು ಒಡೆಯರ ಕಾಲದ ಶಾಸನೋಕ್ತ ಕೆರೆಗಳು}

ನರಸರಾಜ ಒಡೆಯರ ಮಗ ಚಾಮರಾಜ ಒಡೆಯರ ಹೊನ್ನಲಗೆರೆ ದತ್ತಿಶಾಸನದಲ್ಲಿ, ಹೊನ್ನಲಗೆರೆಯ ಎಲ್ಲೆಗಳನ್ನು ಹೇಳುವಾಗ, ನವುಲೆಸೊಣ್ನನಕಟ್ಟೆ,\index{ನವುಲೆಸೊಣ್ನನಕಟ್ಟೆ} ಹಾಳುಬಸ್ತಿಕಟ್ಟೆ, ಗುರುಮನಕಟ್ಟೆ, ಭೀಮನಕೆರೆ, ಕೊತ್ತಿಬೈಚನಕಟ್ಟೆ,\index{ಕೊತ್ತಿಬೈಚನಕಟ್ಟೆ} ನೇರಲತಾಳಕಟ್ಟೆ, ಪ್ರಭುದೇವರಕಟ್ಟೆ,\index{ಪ್ರಭುದೇವರಕಟ್ಟೆ} ಸಿದ್ಧೇವಡೆಯರಕಟ್ಟೆ, ಚಿಟ್ಟಿಗವಡೆಯರ ಕಟ್ಟೆಗಳನ್ನು, ಹಾಗಲಹಳ್ಳಿಯ ಎಲ್ಲೆಗಳನ್ನು ಹೇಳುವಾಗ, ದಾಸಗವುಡನಕಟ್ಟೆಯನ್ನೂ ಉಲ್ಲೇಖಿಸಿದೆ.\endnote{ ಎಕ 7 ಮ 64 ಹೊನ್ನಲಗೆರೆ 1623}

ಚಿಕ್ಕದೇವರಾಜ ಒಡೆಯರ ಶ‍್ರೀರಂಗಪಟ್ಟಣ ತಾಮ್ರ ಶಾಸನದಲ್ಲಿ, ಬಳಗೊಳ ಸ್ಥಳದ ಅವ್ವೇರಹಳ್ಳಿಯ ಸೀಮಾಂತರಗಳನ್ನು ಹೇಳುವಾಗ, ಕೊರಕಲಹಳ್ಳಕ್ಕೆ ಕಟ್ಟಿರಬಹುದಾದ ಬಸರಿಕಟ್ಟೆಯ ಜೊತೆಗೆ, ಕುರುಬನಕಟ್ಟೆ, ಮೊರವನಕಟ್ಟೆಗಳ ಉಲ್ಲೇಖವಿದೆ.\endnote{ ಎಕ 6 ಶ‍್ರೀಪ 24 ಶ‍್ರೀರಂಗಪಟ್ಟಣ 1686} ಒಂದನೇ ಕೃಷ್ಣರಾಜ ಒಡೆಯರ ಹದಿನಾರು ಹಾಳೆಗಳ ತೊಣ್ಣೂರು ತಾಮ್ರ ಶಾಸನದಲ್ಲಿ ಯಾದವಪುರಿ ಹೋಬಳಿಗೆ ಸೇರಿದ 21 ಹಳ್ಳಿಗಳ ಎಲ್ಲೆಯನ್ನು ಹೇಳುವಾಗ ಅನೇಕ ಕೆರೆ ಕಟ್ಟೆಗಳನ್ನು ಮೇರೆಯಾಗಿ ಹೇಳಿದೆ. ಕಾಮನಕೆರೆ, ಕುಡಿನೀರುಕಟ್ಟೆ,\index{ಕುಡಿನೀರುಕಟ್ಟೆ} ಚೆಲುವದೇವಾಂಬುಧಿ\index{ಚೆಲುವದೇವಾಂಬುಧಿ} ಗ್ರಾಮದ ಈಶಾನ್ಯಕ್ಕೆ ಹಳೇಕೆರೆಯ ಮೂಡಲುಕೋಡಿ, ಬಂಡಮಾರನಹಳ್ಳಿ ಹನುಮನಕಟ್ಟೆ, ಗಂಡೇನಹಳ್ಳಿ ಮುದ್ದೇಗೌಡನಕಟ್ಟೆ, ಊಚನಹಳ್ಳಿಯ ಕೋಡಿಮಾಳಕೆರೆ, ಜಕ್ಕನಹಳ್ಳಿ ಕೆರೆಕೋಡಿ, ಹಕ್ಕಿಮಂಚನಹಳ್ಳಿ ಗಿರೇಗೌಡನಕಟ್ಟೆ, ಬೇಲೆಕೆರೆಗೆ ದಕ್ಷಿಣದ ದೇವರಗುಡಿಕಟ್ಟೆ, ಶೀಳುನೆರೆ ಗ್ರಾಮಕ್ಕೆ ವಾಯುವ್ಯ ರಿಕಂನರಾಜನಕಟ್ಟೆ, ಮೋದೂರಿಗೆ ಪಡುವಲು ರಾಮದೇವರಕಟ್ಟೆ, ನಾಡಬೋವನಹಳ್ಳಿಯ ನೈಋತ್ಯದಲ್ಲಿ ಬೇವಿನಮರದಕಟ್ಟೆ, ಚೌಡಯ್ಯನಹಳ್ಳಿಯ ವಾಯುವ್ಯಕ್ಕೆ ಹೊಸಹೊಳಲ ಕೆರೆ, ಚವುಡಯ್ಯನ ಹಳ್ಳಿಯ ಆಗ್ನೇಯಕ್ಕೆ ಲಕ್ಕಿಕಟ್ಟೆ,\index{ಲಕ್ಕಿಕಟ್ಟೆ} ನಾಗನಹಳ್ಳಿಗೆ ಈಶಾನ್ಯ ಕಂಡೇರಿಕಟ್ಟೆ,\index{ಕಂಡೇರಿಕಟ್ಟೆ} ಸಾದುಗೊಂಡನಹಳ್ಳಿಯ ದಕ್ಷಿಣದ ಕಟ್ಟೆ, ಪಶ್ಚಿಮದ ಕಟ್ಟೆ, ಮಂಚನಹಳ್ಳಿಗೆ ಈಶಾನ್ಯದ ಮಾರ್ಗದಕಟ್ಟೆ, ಚಿಕ್ಕನಹಳ್ಳಿಗೆ ನೈಋತ್ಯ ಮಂಚನಹಳ್ಳಿ ಕಟ್ಟೆ, ಕುಂದನಹಳ್ಳಿ ವಾಯುವ್ಯದ ಕಂಚಿನಕೆರೆ, ಕೊಮರನಹಳ್ಳಿ ಉತ್ತರ ಮಾವಿನಕೆರೆ, ಪಟ್ಟಣಗೆರೆ ತೆಂಕಲಾಗಿ ಗುಂಮಟೆ ಕಟ್ಟೆ, ಬಡಗಲಾಗಿ ಚಟ್ಟಣಕೆರೆ ಮುಂತಾದ ಕೆರೆಗಳ ಉಲ್ಲೇಖವಿದೆ.\endnote{ ಎಕ 6 ಪಾಂಪು 99 ತೊಣ್ಣೂರು 1722} ಮೇಲುಕೋಟೆ ತಾಮ್ರಶಾಸನದಲ್ಲಿ ನರಿಗಲ್ಲತೊರೆ,\index{ನರಿಗಲ್ಲತೊರೆ} ನೇರಲಕೆರೆ, ಹರಳುಕೆರೆ, (ಹರಳಹಳ್ಳಿಯಕೆರೆ) ಗಳನ್ನು ಉಲ್ಲೇಖಿಸಿದೆ.\endnote{ ಎಕ 6 ಪಾಂಪು 216 ಮೇಲುಕೋಟೆ 1725} ಈ ತೊರೆಗಳ ನೀರು ಈ ಕೆರೆಗಳಿಗೆ ಹರಿದುಬರುತ್ತಿತೆಂದು ಊಹಿಸಬಹುದು. ಹಿಂದಿನ ಕಾಲದಲ್ಲಿ ರೈತರು ಎಷ್ಟೊಂದು ಕೆರೆ ಕಟ್ಟೆಗಳನ್ನು ಕಟ್ಟಿಕೊಂಡು, ನೀರನ್ನು ಹಿಡಿದಿಟ್ಟು ಉಪಯೋಗಿಸಿಕೊಳ್ಳುತ್ತಿದರು, ಹಾಗೂ ಅವುಗಳನ್ನು ಸಂರಕ್ಷಿಸಿಕೊಂಡು ಬರುತ್ತಿದ್ದರೆಂದು ಇದರಿಂದ ತಿಳಿದುಬರುತ್ತದೆ.

\section*{ಕೆರೆಯ ತೂಬುಗಳು}

ಕೆರೆಗೆ ತೂಬುಗಳನ್ನು ಇಟ್ಟು ನೀರಾವರಿ ಬೇಸಾಯಕ್ಕೆ ಕಾಲುವೆಗಳ ಮೂಲಕ ಸರದಿಯಲ್ಲಿ ನೀರನ್ನು ಬಿಟ್ಟುಕೊಳ್ಳುತ್ತಿದ್ದರು. ಈ ಬಗ್ಗೆ ಸ್ಪಷ್ಟ ಉಲ್ಲೇಖಗಳು ಶಾಸನಗಳಲ್ಲಿ ದೊರೆಯುತ್ತದೆ. “ಹಿಂದೆ ಒಂದು ಮರದ ಬಿರುಡೆಯನ್ನು ಉಪಯೋಗಿಸಿ ತೂಬನ್ನು ಮುಚ್ಚಲಾಗುತ್ತಿತ್ತು. ಇಪ್ಪತ್ತನೆಯ ಶತಮಾನದಲ್ಲಿ ಮರದ ಬಿರುಡೆಯ ಜಾಗದಲ್ಲಿ ಉಕ್ಕಿನ ಬಾಗಿಲು ರೂಢಿಗೆ ಬಂದಿತು. ತೂಬು ಇರುವುದು ಏರಿಯ ಮೇಲು ಪಾರ್ಶ್ವದಲ್ಲಿ, ಅದಕ್ಕೆ ಹೋಗಲು ಕಿರಿದಾದ ಒಂದು ಕಾಲುಸೇತುವೆ ಇರುತ್ತಿತ್ತು. ಹಿಂದಿನ ಕಾಲದ ಹಳೆಯ ಕೆರೆಗಳಲ್ಲಿ ಈಸಿಕೊಂಡು ತೂಬಿಗೂ ಹೋಗುತ್ತಿದ್ದರು. ತೂಬಿಗೆ ಹಾಗೂ ಏರಿಯ ಕೆಳ ಪಾರ್ಶ್ವದಲ್ಲಿ ಇರುವ ಕಾಲುವೆಗೂ ನಡುವೆ ಏರಿಯ ಮುಖಾಂತರ ರಚಿಸಲಾದ ಒಂದು ಸುರಂಗ ಸಂಪರ್ಕ ಇರುತ್ತದೆ. ಹಳೆಯ ಕಟ್ಟೆಗಳಲ್ಲಿ ಸುರಂಗ ಕಲ್ಲುಕಟ್ಟಡದ್ದಾಗಿರುತ್ತಿತ್ತು” ಎಂಬ ಮಾಹಿತಿ ತೂಬುಗಳ ಬಗ್ಗೆ ದೊರೆಯುತ್ತದೆ.\endnote{ ದೀಕ್ಷಿತ್​ ಜಿ.ಎಸ್​., ಕರ್ನಾಟಕದಲ್ಲಿ ಕೆರೆ ನೀರಾವರಿ, ಪುಟ 15} ತೂಬಿನ ಕೆರೆಗೆ ತುಂಬಿನ ಕೆರೆ ಎಂದು ಹೇಳಲಾಗಿದೆ.\endnote{ ಎಕ 7 ನಾಮಂ 168 ಕಸಲಗೆರೆ 1190}ಕೆಲವು ಕೆರೆಗಳಿಗೆ ಎರಡು ಕಡೆ ತೂಬುಗಳಿರುತ್ತಿದ್ದವು, ಈಗಲೂ ಅಂತಹ ಕೆರೆಗಳು ಇವೆ. ತೂಬಿನ ಮೊದಲ ಗದ್ದೆಗೆ ಬಹಳ ಪ್ರಾಶಸ್ತ್ಯ ಇರುತ್ತಿತ್ತು, ಹಾಗೂ ಈ ಗದ್ದೆಗಳನ್ನು ಬೀಜವರಿ ಗದ್ದೆಯಾಗಿ ದತ್ತಿ ಬಿಡುತ್ತಿದ್ದರು.

“ಕೆರೆಯ ತೂಬಿನಿಂದ ಹೊರಟ ನೀರು ಕಾಲುವೆಗಳಲ್ಲಿ (ಬಾಯ್ಕಲ್​ ಅಥವಾ ಕಾಲ್​) ಹರಿದು ಗದ್ದೆಗಳನ್ನು ಸೇರುತ್ತಿತ್ತು.\endnote{ ಚಿದಾನಂದಮೂರ್ತಿ ಡಾ॥ ಎಂ., ಕನ್ನಡ ಶಾಸನಗಳ ಸಾಂಸ್ಕೃತಿಕ ಅಧ್ಯಯನ, ಪುಟ 367} ಕೆರೆಯ ನೀರು ತೂಬಿನಿಂದ ಹೊರಬರುವ ಜಾಗದಲ್ಲಿರುವ ಗದ್ದೆಗೆ ಹೆಚ್ಚಿನ ಮಹತ್ವ ಇತ್ತು. ಕಾರಣ ಈ ಗದ್ದೆಗೆ ನೀರಿನ ಕೊರತೆ ಬೀಳುತ್ತಿರಲಿಲ್ಲ, ಇದನ್ನು \textbf{‘ತೂಂಬಿನ ಗದ್ದೆ’}\index{ತೂಂಬಿನ ಗದ್ದೆ} ಎಂದು ಕರೆಯಲಾಗಿದೆ. ‘ಕಂಬದಹಳ್ಳಿಯ ಪಿರಿಯ ಕೆರೆಯ ತೂಂಬಿನಿಂ ಬಡಗಣ ಹಳ್ಳದಿಂ ತೆಂಕಲಿಗೆ ಕೌಂಗಿನ ತೋಟ’,\endnote{ ಎಕ 7 ನಾಮಂ 33 ಕಂಬದಹಳ್ಳಿ 1118-19} ‘ತುಂಬಿನ ಮೊದಲೇರಿಯಲು ಖಂಡುಗ ಗದ್ದೆ’,\endnote{ ಎಕ 7 ನಾಮಂ 29 ಕಂಬದಹಳ್ಳಿ 1174} “ತುಂಬಿನ ಮೊದಲಲು ಗೞ್ದೆ ಸಲಗೆ ಯೆರಡು”\endnote{ ಎಕ 6 ಕೃಪೇ 73 ಹಿರಿಕಳಲೆ 12ನೇ ಶ.}, \textbf{“ಬಾಯಿಕಾಲಠಾವಿನಲು ಎಮ್ಮ (ಗದ್ದೆ)”}\endnote{ ಎಕ 7 ನಾಮಂ 84 ಬೆಳ್ಳೂರು 1269}, ಕಲ್ಲತುಂಬಿನ ಮೊದಲಲು ಗದ್ದೆ\endnote{ ಎಕ 7 ನಾಮಂ 98 ಮುದಿಗೆರೆ 1139}, ‘ತುಂಬಿನ ಬಾಯಿಕಾಲಿಂ ಪಡುವಲು ಗದ್ದೆ, ಹೊಸಕೆರೆಯ ತುಂಬಿನ ವೋದಕುಳಿಯಿಂ ಮೂಡಲು ಕ್ಷೇತ್ರ’,\endnote{ ಎಕ 7 ನಾಮಂ 73 ಬೆಳ್ಳೂರು 1284} ಕಂಬೆಗೆರೆಯ ತೂಬಿನ ಮೊದಲಲಿ ಗದ್ದೆ, \endnote{ ಎಕ 7 ನಾಮಂ 90 ಬೆಳ್ಳೂರು 15-16ನೇ ಶ.} ‘ಪಿರಿಯಂಣವೊಡೆಯರ ಧಂರ್ಮ್ಮದ ಚೆಂನರಾಮಸಾಗರದ\index{ಚೆಂನರಾಮಸಾಗರ} ಸೀತಾರಾಮನ ತೂಬು’\index{ಸೀತಾರಾಮನ ತೂಬು}\endnote{ ಎಕ 6 ಪಾಂಪು 18 ಹರವು 14ನೇ ಶ.} ಎಂದು ಶಾಸನಗಳಲ್ಲಿ ಉಲ್ಲೇಖಿಸಿದ್ದು, ಇವೆಲ್ಲಾ ತೂಬಿನಿಂದ ನೀರು ಹೊರಡುವ ಜಾಗದಲ್ಲಿರುವ ಗದ್ದೆಗಳಾಗಿವೆ. ತೂಬಿನಿಂದ ನೀರು ಹೊರಬರುವ ಜಾಗವನ್ನು ‘ಬಾಯಿಕಾಲು’\index{ಬಾಯಿಕಾಲು} ಎಂದು ಕರೆದಿರುವುದನ್ನು ಗಮನಿಸಬಹುದು.

ಕೆರೆಯಿಂದ ಗದ್ದೆಗೆ ತೂಬನಿಟ್ಟು, ನೀರು ಹಾಯಿಸುವ ವಿಚಾರದಲ್ಲಿ ಬೈರಾಪುರದ ಶಾಸನವು ವಿಶೇಷವಾದ ಮಾಹಿತಿಯನ್ನು ನೀಡುತ್ತದೆ. ಭಯಿರಮೇಶ್ವರಪುರ ಅಗ್ರಹಾರದ ಮಹಾಜನಗಳು, ಸ್ಥಾನೀಕ ಬೊಮ್ಮಣ್ಣ, ಸಮಸ್ತ ಪ್ರಜೆಗಳು, ಹಿರಿಯ ಕೆರೆಯ ಕೆಳಗಣ, ಯೋಗಣ್ಣಗಳ ಕಟ್ಟೆಯ\index{ಯೋಗಣ್ಣಗಳ ಕಟ್ಟೆ} ಕೆಳಗೆ ಐದು ಖಂಡುಗ ಬೀಜವರಿ ಗದ್ದೆಯನ್ನು ಸರ್ವಮಾನ್ಯವಾಗಿ ಕ್ರಯಕ್ಕೆ ಕೊಡುತ್ತಾರೆ. ಈ ಗದ್ದೆಗೆ ಯೋಗಣ್ಣಗಳ ಕಟ್ಟೆಯ ನೀರು ಸಲುವುದೆಂದೂ, ಯೋಗಣ್ಣಗಳ ಕಟ್ಟೆಯ ನೀರು ಈ ಗದ್ದೆಗೆ ಸಾಲದೇ ಇದ್ದರೆ, \textbf{“ಹಿರಿಯ ಕೆರೆಯ ಬಯಲ ಓಪಾಧಿಯಲಿ ಈ ಐದು ಖಂಡುಗ ಗದ್ದೆಗೆ ಹಿರಿಯ ಕೆರೆಯ ಆರಣಿಯನು ಯಿಕ್ಕಿಕೊಟ್ಟು ನೀರ ಸರಥಿಯನು ಕೊಟ್ಟುಬಹೆವು ಈ ಮರಿಯಾದೆಯಲು ತಪ್ಪದೇ ನಡೆಸಿ ಬಹೆವು”} ಎಂದು ಕರಾರು ಮಾಡಿಕೊಳ್ಳುತ್ತಾರೆ.\endnote{ ಎಕ 6 ಕೃಪೇ 95 ಭೈರಾಪುರ 1312} ಅಂದರೆ ಹಿರಿಯ ಕೆರೆಗೆ ಇನ್ನೊಂದು ಕಡೆ ತೂಬನ್ನು ಇಟ್ಟು ಆ ತೂಬಿನ ಮೂಲಕ ಹಿರಿಯ ಕೆರೆಯ ಗದ್ದೆಗಳಿಗೆ ಯಾವರೀತಿ ನೀರನ್ನು ಬಿಡಲಾಗುತ್ತದೇ ಆ ಕೆರೆಗಳಿಗೂ ನೀರನ್ನು ಬಿಡಲು ಒಪ್ಪಿಕೊಳ್ಳುತ್ತಾರೆ. ಆರಣಿ ಎಂದರೆ ತೂಬು ಎಂಬ ಅರ್ಥ ಬರುತ್ತದೆಂದು ಹೇಳಬಹುದು.

\section*{ಕೆರೆಯ ಏರಿ}

ಸಾಮಾನ್ಯವಾಗಿ ಎಲ್ಲ ಕೆರೆಗಳ ಏರಿಗಳನ್ನು ಕಲ್ಲು ಮತ್ತು ಮಣ್ಣಿನಿಂದ ನಿರ್ಮಿಸಲಾಗುತ್ತಿತ್ತು. ಮಣ್ಣಿನ ಏರಿಯ ಒಳಭಾಗದ ಇಳಿಜಾರಿನ ಉದ್ದಕ್ಕೂ ದೊಡ್ಡ ದೊಡ್ಡ ಕಲ್ಲುಗಳನ್ನು ಜೋಡಿಸಿ ರಕ್ಷಣೆ ಒದಗಿಸಲಾಗುತ್ತದೆ. “ಕೆಳಪಾರ್ಶ್ವಕ್ಕೆ ಅಂದರೆ ಏರಿಯ ಹಿಂಬದಿಯ ಇಳಿಜಾರಿಗೆ ಭಾರೀಮಳೆ ಬಿದ್ದು ಮಣ್ಣು ಕೊಚ್ಚಿಹೋಗದಂತೆ ಹುಲ್ಲು ಗಿಡಗಂಟೆಗಳನ್ನು ಬೆಳೆಸಿ ಸಂರಕ್ಷಣೆ ನೀಡಲಾಗುತ್ತಿತ್ತು. ಕೆಲವು ಹಳೆಯ ಕೆರೆಗಳಲ್ಲಿ ಹಿಂಬದಿಯ ಇಳಿಜಾರಿನಲ್ಲೂ ಕಲ್ಲಿನಸೋಪಾನವೂ ಇರುವುದುಂಟು ಎಂದು ತಿಳಿದುಬರುತ್ತದೆ”.\endnote{ ದೀಕ್ಷಿತ್​ ಜಿ.ಎಸ್​., ಕರ್ನಾಟಕದಲ್ಲಿ ಕೆರೆ ನೀರಾವರಿ, ಪುಟ 15} ಶಾಸನಗಳಲ್ಲಿ ಕೆರೆಯ ಏರಿಗಳನ್ನು ಮೊದಲೇರಿ, ಪೆರಿಯೇರಿ, ಕೀಳೇರಿ ಎಂದು ಹೇಳಿದೆ. ಮೊದಲೇರಿ ಎಂದರೆ ಏರಿಯ ಆರಂಭದ ಭಾಗ, ಪೆರಿಯೇರಿ\index{ಪೆರಿಯೇರಿ} ಎಂದರೆ ದೊಡ್ಡ ಏರಿ, ಕೀಳೇರಿ\index{ಕೀಳೇರಿ} ಏರಿಯ ಕೆಳಗೆ ಅಥವಾ ಕೊನೆ ಎಂದು ಹೇಳಬಹುದು.

ಶಾಸನಗಳಲ್ಲಿ ಆಸನಹಾಳ ಕೆರೆಯ ದೊಡ್ಡೇರಿ,\endnote{ ಎಕ 6 ಪಾಂಪು 44 ಕನ್ನಂಬಾಡಿ 1114} ಹಿರಿಕಳಲೆಯ ದೇವರ ಮುಂದಣ ಕೆರೆಯ ಕೀಳೇರಿಯೊಳು,\endnote{ ಎಕ 6 ಕೃಪೇ 73 ಹಿರಿಕಳಲೆ 12ನೇ ಶ.} ಸಾಸಲಿನ ಕೆರೆಯ ಮೂಡಣ ಏರಿಯ ಉಲ್ಲೇಖವಿದೆ.\endnote{ ಎಕ 6 ಕೃಪೇ 59 ಸಾಸಲು 1121} ತೊಣ್ಣೂರಿನ ತಮಿಳು ಶಾಸನದಲ್ಲಿ ಪೆರಿಯೇರಿಯ ಕೆಳಗಿನ ಗದ್ದೆಯ ಉಲ್ಲೇಖವಿದೆ. ಬಹುಶಃ ಈ ಏರಿಯು ತೊಣ್ಣೂರು ಕೆರೆಯ ಏರಿಯಾಗಿರಬಹುದು.\endnote{ ಎಕ 6 ಪಾಂಪು 100 ತೊಣ್ಣೂರು 12-13ನೇ ಶ.} ಭೈರವಪುರದ ಹಿರಿಯ ಕೆರೆಯ ಮೊದಲೇರಿಯ ತೋಟ ಗದ್ದೆ,\endnote{ ಎಕ 6 ಕೃಪೇ 98 ಭೈರಾಪುರ 1267} ಈ ಉಲ್ಲೇಖಗಳನ್ನು ಗಮನಿಸಬಹುದು.

\section*{ಕೆರೆಯ ಕೋಡಿಗಳು}

‘ಕೆರೆಯ ಹೆಚ್ಚುವರಿ ನೀರು ಸಲೀಸಾಗಿ ಹರಿದು ಹೋಗುವ ಏರಿಯ ಭಾಗವೇ ಕೋಡಿ. ಆ ಹೆಚ್ಚುವರಿ ನೀರು ಕೆಳಕ್ಕೆ ಹರಿದು ಕಟ್ಟೆಯ ಕೆಳಗೆ ನದಿಯ ಪಾತ್ರಕ್ಕೆ ಸೇರುತ್ತಿತ್ತು’\endnote{ ದೀಕ್ಷಿತ್​, ಜಿ.ಎಸ್​., ಕರ್ನಾಟಕದಲ್ಲಿ ಕೆರೆ ನೀರಾವರಿ, ಪುಟ 17} ಕೆರೆ ತುಂಬಿದಾಗ ಹೆಚ್ಚಾದ ಕೆರೆಯನೀರು ಈ ಕೋಡಿಗಳಿಂದ ಹೊರಬಿದ್ದು, ಕೋಡಿಯಹಳ್ಳಗಳ ಮೂಲಕ ಹರಿದುಹೋಗಿ, ಕೆರೆಗಳಿಗೆ ಅಪಾಯ ಒದಗದಂತೆ ರಕ್ಷಿಸುತ್ತಿದ್ದವು. ಜೊತೆಗೆ ಈ ಕೋಡಿಯ ಹಳ್ಳಗಳು ಮುಂದಿನ ಕೆರೆಗಳಿಗೆ ಸಂಪರ್ಕಕಲ್ಪಿಸುತ್ತಿದ್ದವು ಅಥವಾ ನದಿಗಳನ್ನು ಅಥವಾ ಉಪನದಿಗಳನ್ನು ಸೇರುತ್ತಿದ್ದವು. ಕಸಲಗೆರೆ ಶಾಸನದಲ್ಲಿ \textbf{“ಕೇದಗೆಗೆರೆಯ ಪಡುವಣಕೋಡಿಗೆ ಬಂದ ಚಂಚರೀವಳ್ಳಂ\index{ಚಂಚರೀವಳ್ಳಂ} ಪಿಡಿದು”} ಎಂದು ಹೇಳಿದ್ದು, ಪಡುವಣ ಕೋಡಿಯು ನೀರು ಚಂಚರಿವಳ್ಳವನ್ನು ಸೇರುತ್ತಿತ್ತು ಎಂದು ಊಹಿಸಬಹುದು.\endnote{ ಎಕ 7 ನಾಮಂ 169 ಕಸಲಗೆರೆ 1142} “ಊರಿನ ಹಿರಿಯ ಕೆರೆಗೆ ಪಡುವಣ ಕೋಡಿಯುಳ್ಳ ಅರೆಯ ಕಯಿ ಬೆದ್ದಲು” ಎಂದು ಮಾಳಗೂರು ಶಾಸನದಲ್ಲಿದ್ದು,\endnote{ ಎಕ 6 ಕೃಪೇ 66 ಮಾಳಗೂರು 1117} ಸಾಮಾನ್ಯವಾಗಿ ಕೋಡಿಯ ಹಳ್ಳದ ಹತ್ತಿರ ಅರೆ ಅಂದರೆ ಬಂಡೆಗಳಿರುವುದನ್ನು ಸೂಚಿಸುತ್ತಿದೆ. ಈಗಲೂ ಈ ಊರಿನ ಕೆರೆಯಕೋಡಿಯ ಹಾಗೂ ಕೋಡಿಯಹಳ್ಳದ ಪಕ್ಕದಲ್ಲಿ ಉದ್ದಕ್ಕೂ ಅರೆ ಅಂದರೆ ಬಂಡೆಗಳಿವೆ. “ಆ ಕೆರೆಯ ಮೂಡಣ ಕೋಡಿಯಿಂ ಪರಿದ ಪಳ್ಳದಿಂ” ಎಂದು ಹೊಸಹೊಳಲು ಶಾಸನದಲ್ಲಿ ಹೇಳಿದೆ.\endnote{ ಎಕ 6 ಕೃಪೇ 3 ಹೊಸಹೊಳಲು 1118} ಕೋಡಿಯ ಹಳ್ಳದ ಪಕ್ಕದಲ್ಲೂ ಕೋಡಿಯಹಳ್ಳದ ಗದ್ದೆಗಳಿರುತ್ತಿದ್ದವು. ಕೋಡಿಯಲ್ಲಿದ್ದ ದೇವಾಲಯವನ್ನು ಕೋಡಿಯ ಮಾಧವದೇವರ ದೇವಾಲಯ ಎಂದೇ ಕರೆಯಲಾಗಿದೆ.\endnote{ ಎಕ 7 ನಾಮಂ 83 ಬೆಳ್ಳೂರು 1269} ಕೋಡಿಯಲ್ಲಿದ್ದ ಊರನ್ನು ಕೋಡಿಯ ಹಳ್ಳಿ ಎಂದು ಕರೆಯಲಾಗಿದೆ.\endnote{ ಎಕ 7 ನಾಮಂ 68 ದಡಗ 13ನೇ ಶ

ಎಕ 7 ಮಂ 29 ಬಸರಾಳು 1234

ಎಕ 7 ಮ 121 ದೊಡ್ಡಅರಸನಕೆರೆ 1342}

ದೊಡ್ಡ ದೊಡ್ಡ ಕೆರೆಗಳಿಗೆ ಸಾಮಾನ್ಯವಾಗಿ ಎರಡೂ ಕಡೆ ಕೋಡಿಗಳಿರುತ್ತಿದ್ದವು. \textbf{“ತಾಂ ಕಟ್ಟಿಸಿದ ಕಂನ್ನಗೆರೆಯ ಯೆರಡು ಕೋಡಿಯ ನೀರುವರಿಯ ಗದ್ದೆಯನು”}\index{ಕೋಡಿಯ ನೀರು} ಎಂದು ಒಂದು ಶಾಸನದಲ್ಲಿ ಹೇಳಿದೆ.\endnote{ ಎಕ 10 ಚರಾಪ 122 ಕಬ್ಬಳ್ಳಿ 1118} “ಮಾಚಸಮುದ್ರ ಕೆರೆಗೆ ಮೂಡಣಕೋಡಿ, ಪಡುವಣಕೋಡಿ”ಗಳಿತ್ತೆಂದು ಹೇಳಿದೆ.\endnote{ ಎಕ 71 ನಾಮಂ 81 ಬೆಳ್ಳೂರು 1224} ಈಗಲೂ ಅನೇಕ ದೊಡ್ಡ ಕೆರೆಗಳಲ್ಲಿ ಎರಡೂ ಕಡೆ ಕೋಡಿಗಳಿರುವುದನ್ನು ನೋಡಬಹುದು. ಕೆರೆಯ ಏರಿಗಳು ಯಾವದಿಕ್ಕಿನಿಂದ ಯಾವದಿಕ್ಕಿಗೆ ಇರುತ್ತಿದ್ದವು ಎಂಬುದನ್ನೂ ಇದರಿಂದ ಊಹಿಸಬಹುದು. ಸುಜ್ಜಲೂರು ಶಾಸನದಲ್ಲಿ, ‘ಹಿರಿಯಕೆರೆ, ಕಟ್ಟೆಕೆರೆ, ಕೊಳಿಗೆರೆ ಇವುಗಳ ಎರಡು ಕೋಡಿಗಳಿಂದೊಳಗುಳ್ಳ ಯೆಲ್ಲೆ’ ಎಂದು ಹೇಳಿದೆ. ಈ ಕೆರೆಗಳಿಗೆ ಎರಡೂ ಕಡೆ ಕೋಡಿಗಳಿದ್ದುದು ಇದರಿಂದ ತಿಳಿದುಬರುತ್ತದೆ.\endnote{ ಎಕ 7 ಮವ 139 ಸುಜ್ಜಲೂರು 1473}

ಸಾಸಲಿನ ಕೆರೆಯ ಪಡುವಣ ಕೋಡಿ, ಮೂಡಣ ಏರಿಯ ಗದ್ದೆಗಳ ಉಲ್ಲೇಖವಿದೆ.\endnote{ ಎಕ 6 ಕೃಪೇ 59 ಸಾಸಲು 1121} ಲಾಳನಕೆರೆ ಶಾಸನದಲ್ಲಿ ದಾವಂಣನಕೆರೆಯ ತೆಂಕಣಕೋಡಿ, ಬೀಚೆಯನಕೆರೆಯ ತೆಂಕಣಕೋಡಿ, ದೇವರಕೆರೆಯ ಬಡಗಣ ಕೋಡಿ ಎಂದು ಹೇಳಿದ್ದು, ಈ ಕೆರೆಗಳಿಗೆ ಎರಡೂ ಕಡೆ ಕೋಡಿಗಳಿದ್ದಿರಬಹುದೆಂದು ಊಹಿಸಬಹುದು.\endnote{ ಎಕ 7 ನಾಮಂ 61 ಲಾಳನಕೆರೆ 1138} ಸಾಸಲಿನ ಶಾಸನದಲ್ಲಿ ಆ ಕೆರೆಯ ಪಡುವಣಕೋಡಿ,\endnote{ ಎಕ 6 ಕೃಪೇ 59 ಸಾಸಲು 1121} ಬೋಗಾದಿ ಶಾಸನದಲ್ಲಿ ದೊಡ್ಡಿನಕೆರೆಯ ಬಡಗಣಕೋಡಿ, ಮೂಡಣ ಕೆರೆಯ ಬಡಗಣಕೋಡಿಗಳ ಉಲ್ಲೇಖವಿದೆ.\endnote{ ಎಕ 7 ನಾಮಂ 184 ಬೋಗಾದಿ 1144}

ಯಲ್ಲಾದಹಳ್ಳಿ ಶಾಸನದಲ್ಲಿ ದೇವರಕೆರೆಯ ಪಡುವಣಕೋಡಿ, ಹೆರಡೆಗೇತನ ಕಟ್ಟೆಯ ಬಡಗಣಕೋಡಿ, ಕಬ್ಬಿನಕೆರೆಯ ಮೂಡಣಕೋಡಿಗಳ ಉಲ್ಲೇಖವಿದೆ.\endnote{ ಎಕ 7 ನಾಮಂ 64 ಯಲ್ಲಾದಹಳ್ಳಿ 1145} ಹಟ್ಟಣ ಶಾಸನದಲ್ಲಿ ಬಳ್ಳೆಯಕೆರೆಯ ಕೋಡಿ, ನಗರಸಮುದ್ರದ\index{ನಗರಸಮುದ್ರ} ಬಡಗಣಕೋಡಿಗಳ ಉಲ್ಲೇಖವಿದೆ.\endnote{ ಎಕ 7 ನಾಮಂ 118 ಹಟ್ಟಣ 1178} ಅಳಿಸಂದ್ರ ಶಾಸನದಲ್ಲಿ ಹಡವಳತಿಯ ಕೆರೆಯ ಪಡುವಣಕೋಡಿ, ಮಲ್ಲನಕೆರೆಯ ತೆಂಕಣಕೋಡಿ,\endnote{ ಎಕ 7 ನಾಮಂ 72 ಅಳೀಸಂದ್ರ 1183} ಕಸಲಗೆರೆ ಶಾಸನದಲ್ಲಿ ಕಲಂಬಳ್ಳೆಯ ಕೆರೆಯ ಬಡಗಣಕೋಡಿ,\endnote{ ಎಕ 7 ನಾಮಂ 168 ಕಸಲಗೆರೆ 1190} ಜಾಗನಕೆರೆ ಶಾಸನದಲ್ಲಿ ಕೆರೆಯ ಬಡಗಣ ಕೋಡಿ,\endnote{ ಎಕ 6 ಕೃಪೇ 69 ಜಾಗನಕೆರೆ 1242} ಇವುಗಳ ಉಲ್ಲೇಖವನ್ನು ಕಾಣಬಹುದು. ಬೆಳ್ಳೂರು ಶಾಸನದಲ್ಲಿ ಮಾಚಸಮುದ್ರದ\index{ಮಾಚಸಮುದ್ರ} ಮೂಡಣಕೋಡಿಯಲ್ಲಿ ಕೆರೆಗೊಡಗಿಗಳನ್ನು ಬಿಡಲಾಯಿತೆಂದು ಹೇಳಿದೆ.\endnote{ ಎಕ 7 ನಾಮಂ 81 ಬೆಳ್ಳೂರು 1245} ವೈದ್ಯನಾಥಪುರ ಶಾಸನದಲ್ಲಿ ಹದಲಿಕೆರೆಯ ‘ಪಡುವಣ ಕೋಡಿಯಿಂ ಬಡಗಲು ಹಳ್ಳವ ದಾಟಿ’ ಎಂದು ಹೇಳಿದೆ.

ಪೆರುಮಾಳೆದೇವ ದಂಡನಾಯಕನು ಕಟ್ಟಿಸಿದ ಪೆರುಮಾಳೆ ಸಮುದ್ರ\index{ಪೆರುಮಾಳೆ ಸಮುದ್ರ} ಕೆರೆಯ ಹಿರಿಯವೊಡವಿಂದ ಪಡುವಲ, ಪಡುವಣ ಹಳೆಯ ಕೋಡಿಯಿಂದ ಮೂಡಲು ನಡುವೆ ಬೀಜವರಿಗದ್ದೆಯನ್ನು ಮಹಾಜನಗಳಿಗೆ ದತ್ತಿ ಬಿಡಲಾಯಿತು.\endnote{ ಎಕ 7 ನಾಮಂ 82 ಬೆಳ್ಳೂರು 1269} ಇಲ್ಲಿ ಒಡವು ಎಂದರೆ ಏರಿ ಎಂದು ಅರ್ಥೈಸಬಹುದು. ಹಳೆಯ ಕೋಡಿಯಿಂದ ಮುಂದಕ್ಕೆ ಏರಿಯನ್ನು ವಿಸ್ತರಿಸಿ ಹೊಸ ಕೋಡಿಯನ್ನು ನಿರ್ಮಿಸಲಾಗಿದೆ ಎಂದು ಇದರಿಂದ ತಿಳಿಯಬಹುದು. ಬಲಸಮುದ್ರ ಕೆರೆಯ\index{ಬಲಸಮುದ್ರ ಕೆರೆ} ತೆಂಕಣ ಕೋಡಿ,\endnote{ ಎಕ 7 ಮವ 143 ಕಲ್ಕುಣಿ 14 ನೇ ಶ.} ನಗುಲನಹಳ್ಳಿ ಕೆರೆಯ ಮೂಡಣ ಕೋಡಿ\endnote{ ಎಕ 6 ಶ‍್ರೀಪ 25 ಶ‍್ರೀರಂಗಪಟ್ಟಣ 1430} ಇವುಗಳ ಉಲ್ಲೇಖವಿದೆ.

ನೆಲಮನೆ ಶಾಸನದಲ್ಲಿ ಅಗ್ರಹಾರದ ಮೇರೆಗಳನ್ನು ಹೇಳುವಾಗ ರಂಗಸಮುದ್ರ ತೆಂಕಣ ಕೋಡಿಯ ಹಳ್ಳ, ರಂಗಸಮುದ್ರ ಬಡಗಣ ಕೋಡಿಯ ಹಳ್ಳದ ಉಲ್ಲೇಖವಿದ್ದು ಈ ಕೆರೆಗೆ ಎರಡೂ ಕಡೆ ದೊಡ್ಡ ಕೋಡಿಗಳಿದ್ದುದನ್ನು ಅವುಗಳ ನೀರು ಹಳ್ಳದಲ್ಲಿ ಹರಿದು ಹೋಗುತ್ತಿದ್ದುದನ್ನೂ ಇದು ಸೂಚಿಸುತ್ತದೆ.\endnote{ ಎಕ 6 ಪಾಂಪು 93 ನೆಲಮನೆ 1458} ವಡೆಯರ ಕಟ್ಟೆಯ ಬಡಗಣ ಕೋಡಿ,\endnote{ ಎಕ 6 ಪಾಂಪು 99 ತೊಣ್ಣೂರು 1722} ಈ ರೀತಿ ಕೋಡಿಯ ಅನೇಕ ಉಲ್ಲೇಖಗಳನ್ನು ಕೆರೆಯ ಪ್ರಸ್ತಾಪ ಬಂದಾಗ ಶಾಸನಗಳಲ್ಲಿ ಹೇಳಿದೆ.

ಇದರಿಂದ ಮುಖ್ಯವಾಗಿ ತಿಳಿದುಬರುವ ವಿಷಯವೆಂದರೆ, ಕೋಡಿಗಳಿರುವ ದಿಕ್ಕನ್ನು ನೋಡಿ, ಕೆರೆಯ ಏರಿಯು ಯಾವ ದಿಕ್ಕಿನಿಂದ ಯಾವ ದಿಕ್ಕಿಗೆ ಇರುತ್ತಿತ್ತು ಎಂಬುದನ್ನು ಗುರುತಿಸಬಹುದು. ಆಯಾ ಊರುಗಳಲ್ಲಿ ಹಳ್ಳಕೊಳ್ಳಗಳು ಯಾವ ದಿಕ್ಕಿನಿಂದ ಯಾವ ದಿಕ್ಕಿಗೆ ಹರಿಯುತ್ತಿದ್ದವು ಎಂಬುದನ್ನು ಗಮನಿಸಿ, ಕೆರೆಯ ಏರಿಗಳನ್ನೂ ಕೋಡಿಗಳನ್ನೂ ನಿರ್ಮಿಸಿರುವುದನ್ನು ಸ್ಥಳಪರಿಶೀಲನೆಯಿಂದ ತಿಳಿಯಬಹುದು.

\section*{ಕಪಿಲೆ/ಯಾತ(ಏತ)}

ಬಾವಿಗಳಿಂದ ನೀರನ್ನು ಮೇಲಕ್ಕೆತ್ತಿ ಭೂಮಿಯನ್ನು ನೀರಾವರಿ ಸೌಲಭ್ಯಕ್ಕೆ ಒಳಪಡಿಸುವ ಪದ್ಧತಿಗಳು ಹಿಂದಿನಿಂದಲೂ ಪ್ರಾಮುಖ್ಯತೆ ಪಡೆದಿದ್ದವು. ಕಪಿಲೆ, ಯಾತಮ್ ಈ ಎರಡು ವಿಧಗಳನ್ನು ವಿದ್ವಾಂಸರು ಗುರುತಿಸಿದ್ದಾರೆ.\endnote{ \engfoot{Shivanna Dr.K.S., The Agrarian System of Karnataka, pp.15}} ವಿಜಯನಗರ ಕಾಲದಲ್ಲಿ ಈ ಪದ್ಧತಿ ವ್ಯಾಪಕವಾಗಿತ್ತು. ಏತ ನೀರಾವರಿಯನ್ನು \textbf{“ಏತಗುಯ್ಯಲು”}\index{ಏತಗುಯ್ಯಲು} ಎಂದು ಕರೆಯಲಾಗಿದೆ.\endnote{ ಎಕ 6 ಶ‍್ರೀಪ 7 ಶ‍್ರೀರಂಗಪಟ್ಟಣ 1528} ಏತವೆಂದರೆ ಮಾನವಚಾಲಿತ ನೀರಾವರಿ ಬಾವಿ, ಕಪಿಲೆ ಎಂದರೆ ಎತ್ತುಗಳಿಂದ ಚಾಲಿತವಾದ ನೀರಾವರಿ ಬಾವಿಗಳು ಎಂದು ಹೇಳಲಾಗಿದೆ. ಎತ್ತುಗಳನ್ನು ಉಪಯೋಗಿಸಿ ಬಾವಿಯಿಂದ ನೀರನ್ನು ಮೇಲೆತ್ತುವ ಘಟಿಯಂತ್ರ ಇದ್ದಿತೆಂದು ಹಾಸನ ಜಿಲ್ಲೆಯ ಶಾಸನದಲ್ಲಿದೆ ಎಂದು ತಿಳಿದುಬರುತ್ತದೆ.\endnote{ \engfoot{Gururajachar Dr.S., Some aspects of social and economic life in Karnataka, pp.56-57}} ಕೃಷ್ಣರಾಜಪೇಟೆ ತಾಲ್ಲೂಕಿನ ಬ್ಯಾಲದ ಕೆರೆ ಶಾಸನದಲ್ಲಿ \textbf{“ಯಂತ್ರಕೂಪ\index{ಯಂತ್ರಕೂಪ} ಮಹಾಕುಲ್ಯಾ ಮುಖ್ಯಸರ್ವಾಧಿಪತ್ಯಕಂ”} ಎಂದು ಹೇಳಿದ್ದು, ಯಂತ್ರಕೂಪ ಎಂದರೆ ಯಂತ್ರಗಳಿಂದ ನೀರನ್ನು ಎತ್ತುವ ಬಾವಿಗಳೆಂದು ಹೇಳಬಹುದು.\endnote{ ಎಕ 6 ಕೃಪೇ 99 ಬ್ಯಾಲದಕೆರೆ 1508} “ಏತಗಳಿಂದ ಮತ್ತು ಘಟೀಯಂತ್ರಗಳಿಂದ ತೋಟಕ್ಕೆ ನೀರು ಹಾಯಿಸುತ್ತಿದ್ದರು. ಘಟೀಯಂತ್ರದ ವರ್ಣನೆಯು ಒಂದು ಶಾಸನದಲ್ಲಿ ಬಂದಿದೆ”ಎಂದು ತಿಳಿದುಬರುತ್ತದೆ.\endnote{ ಚಿದಾನಂದಮೂರ್ತಿ ಡಾ॥ ಎಂ., ಕನ್ನಡ ಶಾಸನಗಳ ಸಾಂಸ್ಕೃತಿಕ ಅಧ್ಯಯನ, ಪುಟ 366} ಅದೇ ರೀತಿ \textbf{“ಕಪಿಲೆ”\index{ಕಪಿಲೆ} }ಬಾವಿಗಳು ಕೂಡಾ ಪ್ರಮುಖ ನೀರಾವರಿ ಮೂಲಗಳಾಗಿದ್ದವು. \textbf{“ಗೂಡೆಗುಯ್ಯಲು”}\index{ಗೂಡೆಗುಯ್ಯಲು} ಎಂಬುದೂ ಒಂದು ರೀತಿಯ ನೀರಾವರಿ ಪದ್ಧತಿ. ಇದು ಬಾವಿಗಳಿಂದ ಅಥವಾ ಅದಕ್ಕಿಂತ ಹೆಚ್ಚಾಗಿ ಸಮತಟ್ಟಾದ ಪ್ರದೇಶದಲ್ಲಿರುವ ಹಳ್ಳಗಳಿಂದ ನೀರನ್ನು ಎತ್ತಿ ಭೂಮಿಗೆ ಹರಿಸುವ ಒಂದು ವಿಧಾನ. ಒಂದು ಅಗಲವಾದ ಬಾಯುಳ್ಳ ಗೂಡೆಯಂತಹ ಸಾಧನಕ್ಕೆ ಎರಡೂ ಕಡೆ ಹಗ್ಗಗಳನ್ನು ಕಟ್ಟಿ, ಆ ಕಡೆ ಒಬ್ಬರು, ಈ ಕಡೆ ಒಬ್ಬರು, ಹಳ್ಳದ ಅಥವಾ ಬಾವಿಯ ಎರಡೂ ಭಾಗಗಳಲ್ಲಿ ನಿಂತು, ಗೂಡೆಯನ್ನು ಸಮತೋಲನದಿಂದ ಸರಿದೂಗಿಸಿ ನೀರೊಳಗೆ ಮುಳುಗಿಸಿ ಮೇಲಕ್ಕೆ ಎತ್ತಿ ತಂದು ಆ ನೀರನ್ನು ಕಾಲುವೆಗೆ ಸುರಿಯುವುದು, ಮತ್ತೆ ಗೂಡೆಯು ನೀರಿಗೆ ಬಂದು ಮತ್ತೆ ನೀರನ್ನು ತುಂಬಿಕೊಂಡು ನೀರನ್ನು ಕಾಲುವೆಗೆ ಸುರಿಯುವುದು, ಇದೊಂದು ರೀತಿಯ ಸುಲಭ ವಿಧಾನ ಆದರೆ ಶಕ್ತಿ ಬೇಕು. ಆಡು ಮನೆ ಎಂಬುದು, ಹೊಲ ಗದ್ದೆಗಳಲ್ಲಿ ಪಶುಪಾಲನೆಗಾಗಿ ಕಟ್ಟಿದ ಮನೆಗಳೆಂದು ಹೇಳಬಹುದು. ಅದು ಇಂದಿನ “ಫಾರಂ ಹೌಸ್​”ಗೆ ಸಮಾನವಾದವು ಎಂದು ತಿಳಿಯಬಹುದು. “ಹೊಸವಳ್ಳಿಯ ಕೆರೆ ನೀರು ಯೇತಗದ್ದೆಗೆ\index{ಯೇತಗದ್ದೆ} ಸಲ್ಗು” ಎಂದು ಹೇಳಿದೆ.\endnote{ ಎಕ 7 ನಾಮಂ 7 ನಾಗಮಂಗಲ 1134} ಏತದ ಬಾವಿಗಳಲ್ಲಿ ನೀರಿಲ್ಲದಾಗ ಗದ್ದೆಗಳಿಗೆ ನೇರವಾಗಿ ಕೆರೆಯಿಂದ ನೀರು ಹರಿಸಿಕೊಳ್ಳುತಿದ್ದರೆಂದು ಇದರಿಂದ ತಿಳಿದುಬರುತ್ತದೆ.

\section*{ಅಣೆಕಟ್ಟುಗಳು/ಕಟ್ಟೆಗಳು/ಕಟ್ಟುಕಾಲುವೆಗಳು}

ನದಿಗಳಿಗೆ ಅಥವಾ ದೊಡ್ಡ ದೊಡ್ಡ ಹಳ್ಳಗಳಿಗೆ (ಉಪನದಿಗಳು) ಅಣೆ ಅಥವಾ ಅಣೆಕಟ್ಟುಗಳನ್ನು ನಿರ್ಮಿಸಿ, ಇವುಗಳಿಂದ ನಾಲೆಗಳನ್ನು ತೆಗೆದು ನೀರಾವರಿ ಬೇಸಾಯವನ್ನು ಮಾಡುತ್ತಿದ್ದರು. ಗಂಗರ ಕಾಲದಲ್ಲಿ ಇದನ್ನು ಸೇತುಬಂಧ\index{ಸೇತುಬಂಧ} ಎಂದೂ,\endnote{ \engfoot{Gururajachar, S., Agriculture, Economic and Social Life in Karnataka, pp.53}} ಹೊಯ್ಸಳರ ಕಾಲದಲ್ಲಿ ಕಟ್ಟೆ ಕಾಲುವೆಗಳೆಂದೂ, ವಿಜಯನಗರ ಕಾಲದಲ್ಲಿ ಇವುಗಳನ್ನು ಅಣೆ, ಅಚ್ಚುಕಟ್ಟುಗಳೆಂದೂ ಕರೆಯಲಾಗಿದೆ. ಜಿಲ್ಲೆಯ ಶಾಸನಗಳಲ್ಲಿ ಗಂಗರ ಕಾಲದಿಂದ ಹಿಡಿದು, ಮೈಸೂರು ಒಡೆಯರ ಕಾಲದವರೆಗೂ, ನದಿಗಳಿಗೆ ಹಳ್ಳಗಳಿಗೆ ಅಣೆಕಟ್ಟು\-ಗಳನ್ನು ನಿರ್ಮಿಸಿ, ಕಾಲುವೆಗಳ ಮೂಲಕ ನೀರನ್ನು ಹರಿಸಿ, ನೀರಾವರಿ ಬೇಸಾಯಕ್ಕೆ ಅನುಕೂಲ ಕಲ್ಪಿಸಿರುವ ಅನೇಕ ಉದಾ\-ಹರಣೆಗಳನ್ನು ನೋಡಬಹುದು.

\section*{ಗಂಗರ ಕಾಲದ ಅಣೆಕಟ್ಟುಗಳು ಅಥವಾ ಸೇತುಬಂಧಗಳು}

ಶಿವಮಾರನ ಹಳ್ಳೆಗೆರೆ ತಾಮ್ರಪಟಗಳಲ್ಲಿ ಕೆರೆಗೋಡಿನ ಉತ್ತರಪಾರ್ಶ್ವದಲ್ಲಿದ್ದ ಕಿಳಿನೀ ನದಿಗೆ ಸೇತುಬಂಧವನ್ನು ಮಾಡಲಾಯಿತೆಂದು ಹೇಳಿದೆ. ಶಾಸನೋಕ್ತ ಪಲ್ಲವ ಯುವರಾಜರೇ, ಕಿಳಿನೀನದಿಗೆ\index{ಕಿಳಿನೀನದಿ} ಅಣೆಕಟ್ಟೆಯನ್ನು ನಿರ್ಮಿಸಿ,ಅದಕ್ಕೆ ಪಲ್ಲವತಟಾಕವೆಂಬ\index{ಪಲ್ಲವತಟಾಕ} ಹೆಸರನ್ನು ಇಟ್ಟಿರಬಹುದು.\endnote{ ಎಕ 7 ಮಂ 35 ಹಳ್ಳೆಗೆರೆ 713} ಅಂದಿನ ಕಾಲದ ಸೇತುಬಂಧ ಎಂದರೆ ಇಂದಿನ ಬ್ರಿಡ್ಜ್​ ಕಮ್ ಬ್ಯಾರೇಜ್​ ಎಂದು ಹೇಳಬಹುದು. ಸೇತುಬಂಧವೆಂದರೆ ಅಣೆಕಟ್ಟು ಎಂದು ವಿದ್ವಾಂಸರು ಅಭಿಪ್ರಾಯಪಟ್ಟಿದ್ದಾರೆ.\endnote{ ಚಿದಾನಂದಮೂರ್ತಿ, ಡಾ॥ ಎಂ., ಕನ್ನಡ ಶಾಸನಗಳ ಸಾಂಸ್ಕೃತಿಕ ಅಧ್ಯಯನ, ಪುಟ 366} ಶಿವಮಾರನ ಮಗನಾದ ಮಾರಸಿಂಗ ಎರೆಯಪ್ಪನ ಗಂಜಾಮ್ ತಾಮ್ರ ಪಟಗಳಲ್ಲಿ, ‘ಪಡುವಾಯ್ಕಾವೇರಿಯ ಸೆಟ್ಟಿಕೆರೆ’ ಎಂದು ಹೇಳಿದೆ. ಈ ಸ್ಥಳದಲ್ಲಿ ಕಾವೇರಿ\index{ಕಾವೇರಿ} ನದಿಗೆ ಕಟ್ಟೆಯನ್ನು ಹಾಕಲಾಗಿತ್ತೇ ಎಂದು ಒಂದು ಊಹೆಯನ್ನು ಮಾಡಬಹುದು.\endnote{ ಎಕ 6 ಶ‍್ರೀಪ 66 ಗಂಜಾಮ್ 8ನೇ ಶ.} ಶ‍್ರೀರಂಗಪಟ್ಟಣದ ಕಾವೇರಿ ನದಿ ದಡದ ಗೌತಮ ಕ್ಷೇತ್ರದ ಬಳಿ, ತಲೆನೆರೆಯಲ ಕಟ್ಟೆಯನ್ನು ಕಟ್ಟುವುದಕ್ಕೆ ಶ‍್ರೀ ಕೇಸಿಗನಿಗೆ ಕೆಲವು ತೆರಿಗೆಗಳನ್ನು ದತ್ತಿಯಾಗಿ ಬಿಡಲಾಗಿದೆ. ಇದು ಕಾವೇರಿ ನದಿಗೆ ಅಡ್ಡಲಾಗಿ ಕಟ್ಟಿರುವ ಕಟ್ಟೆಯಾಗಿರಬಹುದೆಂದು ವಿದ್ವಾಂಸರು ಊಹಿಸಿದ್ದಾರೆ. ಶಾಸನದ ಕಾಲದಲ್ಲಿ ಕಾವೇರಿ ನದಿಯಪಾತ್ರವು, ಈಗ ಗದ್ದೆಯಲ್ಲಿರುವ ಈ ಶಾಸನದವರೆಗೂ ಇದ್ದಿತೆಂದು ಹೇಳಹುದು.\endnote{ ಎಕ 6 ಶ‍್ರೀಪ 85 ರಾಂಪುರ 904-905}

\section*{ಹೊಯ್ಸಳರ ಕಾಲದ ಅಣೆಕಟ್ಟುಗಳು ಅಥವಾ ಕಟ್ಟುಕಾಲುವೆಗಳು}

ಹೊಯ್ಸಳರ ಕಾಲದಲ್ಲಿ ಜಿಲ್ಲೆಯಲ್ಲಿ ಹರಿಯುವ ಕಾವೇರಿ ನದಿಗೆ ಅಣೆಕಟ್ಟೆಯನ್ನು ನಿರ್ಮಿಸಿ, ನಾಲೆಗಳ ಮೂಲಕ ನೀರು ಹರಿಸಿ ನೀರಾವರಿ ಬೇಸಾಯಕ್ಕೆ ಹೆಚ್ಚಿನ ಅನುಕೂಲ ಕಲ್ಪಿಸಿಕೊಡಲಾಗಿತ್ತು. ಹಾಸನ ತಾಲ್ಲೂಕು ಯಲಗುಂದ ಶಾಸನದಲ್ಲಿ ವಿಷ್ಣುವರ್ಧನನನ್ನು \textbf{‘ಕಾವೇರಿ ತೀರ\index{ಕಾವೇರಿ ತೀರ} ವನವಿಹಾರ ಮದ ಮದಾಳನುಂ’} ಎಂಬ ವಿಶೇಷ ಬಿರುದಿನಿಂದ ವರ್ಣಿಸಿದೆ. ಇದರಿಂದ ವಿಷ್ಣುವರ್ಧನನಿಗೆ ಕಾವೇರಿ ನದಿಯ ಬಗ್ಗೆ ಅಪಾರ ಪ್ರೀತಿ ಇತ್ತೆಂದು ಹೇಳಬಹುದು. ಗಂಗರು ಕಾವೇರಿ ನದಿಗೆ ನಿರ್ಮಿಸಿದ್ದ ಅಣೆಕಟ್ಟುಗಳನ್ನು, ಚೋಳರು ಗಂಗವಾಡಿಯನ್ನು ಆಕ್ರಮಿಸಿದ ನಂತರ ಒಡೆದು ಹಾಕಿದರೆಂದು ಪ್ರತೀತಿ.

ಶ‍್ರೀರಂಗಪಟ್ಟಣದಲ್ಲಿ ಚಂದ್ರವನದಬಳಿ ಕಾವೇರಿ ನದಿ ತೀರದ ಒಂದು ಪೊದೆಯಲ್ಲಿರುವ ಒಂದನೆಯ ನರಸಿಂಹನ ತ್ರುಟಿತ ಶಾಸನದಲ್ಲಿ ಹಿರಿಯ ಭಂಡಾರಿ.... ನಾಯಕ ಮತ್ತು ಗಂಗ(ರಾಜ) ಇವರ ಉಲ್ಲೇಖವಿದ್ದು, “....ಲ್ಗಟ್ಟ ಕಟ್ಟಿಸಿ ತಿದ್ದಿಸಿದರು” ಎಂಬ ಬರಹವಿದೆ. ಬಹುಶಃ ಇಲ್ಲಿ ಕಾವೇರಿ ನದಿಗೆ ಅಡ್ಡಲಾಗಿ ಪ್ರಾಚೀನ ಕಾಲದಿಂದ ಇದ್ದ ಕಟ್ಟೆಯನ್ನು ತಿದ್ದಿಸಿರಬಹುದು ಅಂದರೆ ಜೀರ್ಣೋದ್ಧಾರ ಮಾಡಿರಬಹುದು.\endnote{ ಎಕ 6 ಶ‍್ರೀಪ 55 ಶ‍್ರೀಪ 12ನೇ ಶ.} ಕಾರೇಪುರದ ಬಳಿ ಹರಿಯುವ ಕಾವೇರಿ ನದಿಯ ಬಂಡೆಯ ಮೇಲೆ \textbf{“ಬಳ್ಳೆಗೊಳಕ್ಕೆ\index{ಬಳ್ಳೆಗೊಳ} ಕಟ್ಟೇರಿನ ಮಡು, ಆಚಂದ್ರ ತಾರಾಂಬರ ನಿಲ್ವುದು”} ಎಂದು ಹೇಳಿದೆ. ಬಹುಶಃ ಹೊಯ್ಸಳರ ಕಾಲದಲ್ಲಿ ಬಳಗೊಳದ ಬಳಿ ಕಾವೇರಿ ನದಿಗೆ ಅಣೆಕಟ್ಟೆಯನ್ನು ಕಟ್ಟಿರುವುದನ್ನು ಇದು ಸೂಚಿಸುತ್ತದೆ.\endnote{ ಎಕ 6 ಶ‍್ರೀಪ 84 ಕಾರೇಪುರ 13-14ನೇ ಶ.}

ಎರಡನೇ ಬಲ್ಲಾಳನ ಕ್ರಿ.ಶ. 1214ರ ತಮಿಳು ಶಾಸನದಲ್ಲಿ “ಪಾಪ್ಪಲ್​ ಕಾಲಿ ವಯಲಿಲ್​” ಎಂಬ ಉಲ್ಲೇಖವಿದೆ. ಇದು ಹರಹಿನ ಕಾಲುವೆಯ ಬಯಲನ್ನು ಸೂಚಿಸುತ್ತದೆ. ಹರಹಿನ ಬಳಿ ಕಾವೇರಿ ನದಿಗೆ ಕಟ್ಟೆಯನ್ನು ಕಟ್ಟಿಸಿ ಹರಹಿನ ಕಾಲುವೆಯನ್ನು ತೋಡಿಸಿರಬಹುದು. ಅಥವಾ ಈ ಕಾಲಕ್ಕೂ ಮುಂಚೆ ಇಲ್ಲಿ ಕಟ್ಟೆ ಕಾಲುವೆಗಳಿದ್ದಿರಬಹುದು. ಇದೇ ಹರಹಿನ ಕಾಲುವೆಯನ್ನು ತಿದ್ದಲು ಅಂದರೆ ಜೀರ್ಣೋದ್ಧಾರ ಮಾಡಲು ಎರಡನೆಯ ನರಸಿಂಹನು ತೊಂಡನೂರ ಅಶೇಷ ಮಹಾಜನಗಳಿಗೆ ಕುರುವಂಕನಾಡ ಹೊಳೆಯ ಸುಂಕದಿಂದ\index{ಹೊಳೆಯ ಸುಂಕ} 64 ಗದ್ಯಾಣವನ್ನು ದತ್ತಿಯಾಗಿ ಬಿಡುತ್ತಾನೆ. ಹರಹಿನ ಕಾಲುವೆಯ ಪ್ರಾಚೀನತೆ ಹಾಗೂ ಅದನ್ನು ವರ್ಷಂಪ್ರತಿ ದುರಸ್ತಿಗೊಳಿಸುತ್ತಿದ್ದ ವಿಚಾರ ಈ ಶಾಸನದಿಂದ ತಿಳಿದುಬರುತ್ತದೆ.\endnote{ ಎಕ 6 ಪಾಂಪು 56 ತೊಣ್ಣೂರು 13ನೇ ಶ.}

ಪಾಂಡವಪುರ ತಾಲ್ಲೂಕು, ಸೀತಾಪುರ ಶಾಸನದಲ್ಲಿ ಹರಹಿನ ಬಯಲಿನಲ್ಲಿ ಹಳೆಯ ಕಟ್ಟೆ ಕಾಲುವೆಗಳು ಮತ್ತು ಹೊಸ ಕಾಲುವೆಗಳ ಉಲ್ಲೇಖವಿದ್ದು, ಇಲ್ಲಿ ಮೊದಲೇ ಕಟ್ಟೆ ಕಾಲುವೆಗಳು ಇದ್ದುದನ್ನೂ ಮತ್ತು ಹೊಸದಾಗಿ ಕಾಲುವೆಗಳನ್ನು ತೋಡಿರುವುದನ್ನು ಗಮನಿಸಬಹುದು.\endnote{ ಎಕ 6 ಪಾಂಪು 19 ಸೀತಾಪುರ 1467} ಹರಹಿನ ಕಾಲುವೆಯ ಕೆಳಗೆ ಶ‍್ರೀರಂಗಪುರ, ಸೀತಾಪುರ ಮತ್ತು ಹರಹುಗಳು ಇದ್ದವೆಂದು, ಈ ಊರುಗಳ ಕಾಲುವೆಯ ಕೆಳಗೆ ಗದ್ದೆಯನ್ನು ಶ‍್ರೀರಂಗಪಟ್ಟಣದ ಗಂಗಾಧರೇಶ್ವರ ದೇವರಿಗೆ ದತ್ತಿ ನೀಡಲಾಯಿತೆಂದು ತಿಳಿದುಬರುತ್ತದೆ.\endnote{ ಎಕ 6 ಶ‍್ರೀಪ 31 ಶ‍್ರೀರಂಗಪಟ್ಟಣ 1517}

ಮೂರನೆಯ ಬಲ್ಲಾಳನ ಕಾಲದಲ್ಲಿ, ಹರಿಹರಪುರ ಅಗ್ರಹಾರದ, ರಾಜಗುರು ಸರ್ವಜ್ಞ ವಿಷ್ಣು ಭಟ್ಟಯ್ಯನ ಮಕ್ಕಳು ಹರಿಹರಭಟ್ಟೋಪಾಧ್ಯಾಯರು ಮತ್ತು ಮಹಾಜನಗಳು ಸೇರಿ, ಹರಿಹರಪುರಕ್ಕೆ ಸಮೀಪ ಇರುವ ಬಂಡಿಹೊಳೆ ಗ್ರಾಮದ ಹತ್ತಿರ ಹರಿಯುವ ಹೇಮಾವತಿ ನದಿಗೆ, ಕಟ್ಟೆಯನ್ನು ಕಟ್ಟಿ, ಕಾಲುವೆಯನ್ನು ತೆಗೆದು ಈ ಅಗ್ರಹಾರದ ಜಮೀನುಗಳಿಗೆ ನೀರಾವರಿ ವ್ಯವಸ್ಥೆಯನ್ನು ಮಾಡುತ್ತಾರೆ. ವೀರಬಲ್ಲಾಳನು ತಾನೇ ಖುದ್ದಾಗಿ ಹರಿಹರಪುರದ ಕಟ್ಟೆಗೆ ಬಂದು, ಕಟ್ಟೆ ಕಾಲುವೆಗಳನ್ನು ಚಿತ್ತೈಸಿ, ಕಟ್ಟೆ ಕಾಲುವೆಗಳನ್ನು ವರ್ಷಂಪ್ರತಿ ಜೀರ್ಣೋದ್ಧಾರ ಮಾಡಿಸುವುದಕ್ಕೆ, ಬಂಡಿಹಳ್ಳಿ, ಕೂಡಲುಗುಪ್ಪೆ ಸ್ಥಳಗಳ ಹೆಜ್ಜುಂಕವನ್ನು ದತ್ತಿ ಬಿಡುತ್ತಾನೆ.\endnote{ ಎಕ 6 ಕೃಪೇ 11 ಹರಿಹರಪುರ 1322} ಇದೇ ಜಾಗದಲ್ಲಿ ಮೈಸೂರಿನ ಒಡೆಯರು ಹೊಸದಾಗಿ ಹೇಮಾವತಿ ನದಿಗೆ ಅಣೆಕಟ್ಟೆಯನ್ನು ನಿರ್ಮಿಸಿ ನಾಲೆಗಳನ್ನು ತೋಡಿಸಿದ್ದಾರೆ. ಕಟ್ಟೆಯಲ್ಲಿ ನೀರು ಇಲ್ಲದಿದ್ದಾಗ, ಮಹಾಜನರು ಕಟ್ಟಿಸಿದ್ದ ಹಳೆಯ ಕಟ್ಟೆಯ ಕುರುಹುಗಳು ಕಂಡುಬರುತ್ತವೆ.

\section*{ವಿಜಯನಗರ ಕಾಲದ ಅಣೆ ಅಚ್ಚುಕಟ್ಟುಗಳು, ಕಾಲುವೆಗಳು}

ಜಿಲ್ಲೆಯಲ್ಲಿರುವ ವಿಜಯನಗರ ಕಾಲದ ಶಾಸನಗಳಲ್ಲಿ ದತ್ತಿಯನ್ನು ಹೇಳುವಾಗ, ಅಣೆ ಅಚ್ಚುಕಟ್ಟು, ಕಾಡಾರಂಭ ನೀರಾರಂಭ, ಕಟ್ಟೆ, ಕಾಲುವೆ ಎಂಬ ಪಾರಿಭಾಷಿಕಗಳನ್ನು ಉಪಯೋಗಿಸಲಾಗಿದೆ. ಇದರಿಂದ ಶಾಸನೋಕ್ತವಾದ ಆ ಊರಿನಲ್ಲಿ ಅಥವಾ ಆ ಊರಿನ ಸೀಮೆಯಲ್ಲಿ ಹರಿಯುವ ನದಿಗೆ ಅಥವಾ ಹಳ್ಳಕ್ಕೆ ಅಣೆಗಳನ್ನು ಕಟ್ಟಿ, ಅಚ್ಚುಕಟ್ಟುಗಳನ್ನು ಅಂದರೆ ಕಾಲುವೆಯ ನೀರಿನ ಮೂಲಕ ನೀರಾವರಿಯಾಗುವ ಕ್ಷೇತ್ರಗಳನ್ನು ಕಲ್ಪಿಸಲಾಗಿತ್ತೆಂದು ಹೇಳಬಹುದು. ಎರಡನೇ ಹರಿಹರನ ಮಂತ್ರಿಯಾಗಿದ್ದ ಭಟ್ಟರ ಬಾಚೆಯಪ್ಪನು ಚವುಡಪ್ಪನ ಕಾಲುವೆಯನ್ನು\index{ಚವುಡಪ್ಪನ ಕಾಲುವೆ} ತೋಡಿಸಿದನೆಂದು ಅರುವನಹಳ್ಳಿ ಶಾಸನದಿಂದ ತಿಳಿದುಬರುತ್ತದೆ.\endnote{ ಎಕ 7 ಮ 87 ಅರುವನಹಳ್ಳಿ 1381}

ಪ್ರೌಢದೇವರಾಯನ ಶ‍್ರೀರಂಗಪಟ್ಟಣ ತಾಮ್ರಶಾಸನದಲ್ಲಿ ಚಂದಿಗಾಲು ಅಗ್ರಹಾರಕ್ಕೆ ಎಲ್ಲೆಯನ್ನು ಹೇಳುವಾಗ, ನಗುಲನ ಹಳ್ಳಿಯ ಈಶಾನ್ಯ, ಚಂದಿಗಾಲು ವಾಯುವ್ಯಕ್ಕೆ ಕಾವೇರಿ ಹೊಳೆಯ ಕಾಲುವೆ\index{ಕಾವೇರಿ ಹೊಳೆಯ ಕಾಲುವೆ} ಎಂದು ಹೇಳಿದೆ. ಬಹುಶಃ ಈ ಕಾಲುವೆಯು ಕಾವೇರಿ ನದಿಗೆ ಕಟ್ಟಿರುವ ಕಟ್ಟೆಯಿಂದ ಬರುತ್ತಿದ್ದ ನಾಲೆ ಎಂದು ಹೇಳಬಹುದು.\endnote{ ಎಕ 6 ಶ‍್ರೀಪ 25 ಶ‍್ರೀರಂಗಪಟ್ಟಣ 1430} ಇದೇ ಶಾಸನದಲ್ಲಿ ಬಂಡಿಪಾತಿಯ (ಬಂಡಿಯದಾರಿ) ಮೊಹಲೆಯ ಕಾಲುವೆ ಎಂದು ಹೇಳಿದೆ. ಕಾಲುವೆಯ ಪಕ್ಕದಲ್ಲೇ ಬಂಡಿಯ ರಸ್ತೆಯನ್ನು ಮಾಡಿರುವುದನ್ನು ಇದು ಸೂಚಿಸುತ್ತದೆ.

ಮಳವಳ್ಳಿ ತಾ. ಚಾಮಲಾಪುರದ ವೀರಚಿಕ್ಕ ಒಡೆಯನ ಶಾಸನದಲ್ಲಿ ಹೋರಿನಿದೇವ ಒಡೆಯರಿಗೆ ಪೂರ್ವದಲ್ಲಿ ದೇವರಾಯನು ಹುಲಿವಾನದ ಬಡಗಣಕೋಟೆಯ ಒಂದು ಕೆರೆಯನ್ನು ದತ್ತಿಯಾಗಿ ನೀಡಿದ್ದನೆಂದೂ, ಚಿಕ್ಕ ಒಡೆಯನು ಅದನ್ನು ಬಿಡಿಸಿ, ಅದಕ್ಕೆ ಬದಲಾಗಿ, ಚಾಮಲಾಪುರವನ್ನು ಸುಧರ್ಮಪುರವಾಗಿ ದಾನ ನೀಡುತ್ತಾನೆ. ಈ ಪುರದ ಸೀಮೆಯಲ್ಲಿ ಅಣೆ, ಅಚ್ಚುಕಟ್ಟು, ಕಟ್ಟೆ, ಕಾಲುವೆಗಳಿದ್ದವೆಂದೂ ಇವುಗಳ ಸರ್ವಸಾಮ್ಯವನ್ನೂ ದತ್ತಿ ನೀಡಲಾಯಿತೆಂದೂ ತಿಳಿದುಬರುತ್ತದೆ.\endnote{ ಎಕ 7 ಮಂ 42 ಚಾಮಲಾಪುರ 1452}

ನಾಗಮಂಗಲದ ಶಿಂಗಣ್ಣ ವೊಡೆಯರ ಮಕ್ಕಳು ದೇವರಾಜನು\index{ದೇವರಾಜ} ಕಾವೇರಿ ನದಿಗೆ ಹೊಸದಾಗಿ ಕಟ್ಟೆಯೊಂದನ್ನು ಕಟ್ಟಿಸಿ ಹರಹಿನವರೆಗೂ ಕಾಲುವೆಯನ್ನು ನಿರ್ಮಿಸುತ್ತಾನೆ. ಆಗ ಹರಹಿನ 72 ಮಹಾಜನಗಳು, ಹರಹಿನ ಸೀಮೆಯ ತಮ್ಮ ಗ್ರಾಮಗಳಿಗೂ ಕೂಡಾ ಈ ಕಾಲುವೆಯನ್ನು ತರುವುದಕ್ಕೆ ಶಿಂಗಣ್ಣ ಒಡೆಯನನ್ನು ಒಡಂಬಡಿಸುತ್ತಾರೆ. ಈ ಕಾಲುವೆಯನ್ನು ತಂದುದಕ್ಕಾಗಿ ಆ ಗ್ರಾಮಸೀಮೆಯ ದೇವಭಾಗ, ಬ್ರಹ್ಮಾದಾಯ ಮತ್ತು ಯಜಮಾನ ಭಾಗ ಈ ಮೂರು ಭಾಗಗಳನ್ನು ಸಾಧನವಾಗಿ ಕೊಡುತ್ತಾರೆ.\endnote{ ಎಕ 6 ಪಾಂಪು 19 ಸೀತಾಪುರ 1455, 1467} ನದಿಗೆ ಕಟ್ಟೆಯನ್ನು ಕಟ್ಟಿ ಕಾಲುವೆಯನ್ನು ತರಲು ಮಹಾಜನಗಳಾದಿಯಾಗಿ ಎಲ್ಲರೂ ಕೂಡಾ ಸಹಕರಿಸುತ್ತಿದ್ದರು, ಈ ಕಾರ್ಯಕ್ಕೆ ಖರ್ಚು ಮಾಡಿದ ಹಣಕ್ಕೆ ತಮ್ಮ ಭಾಗವನ್ನೂ ಕೊಡುತ್ತಿದ್ದರು ಎಂಬುದು ಇದರಿಂದ ತಿಳಿದುಬರುತ್ತದೆ. ಇದೇ ಶಾಸನದಲ್ಲಿ ಮೇರೆಗಳನ್ನು ಹೇಳುವಾಗ ಹಳೆಯ ಕಾಲುವೆ ಮತ್ತು ಹೊಸ ಕಾಲುವೆಗಳು ಮತ್ತು ಆ ಕಾಲುವೆಗಳ ಕೆಳಗಿದ್ದ ಗದ್ದೆ ಬೆದ್ದಲುಗಳ ಉಲ್ಲೇಖ ಬರುತ್ತದೆ. ಜೊತೆಗೆ ಇಲ್ಲಿ ಒಂದು ಹಳೆಯ ಕಟ್ಟೆ ಇದ್ದ ಉಲ್ಲೇಖವು ಬರುತ್ತದೆ. ಅಂದರೆ ದೇವರಾಜನು ಕಾವೇರಿ ನದಿಗೆ ಕಟ್ಟೆಯನ್ನು ಕಟ್ಟುವುದಕ್ಕೆ ಮೊದಲೇ ಅಲ್ಲಿ ಒಂದು ಕಟ್ಟೆ ಇದ್ದಿತೆಂದು ಹೇಳಬಹುದು. \textbf{“ಹೊಸಕಾಲುವೆಯ ದಾಸರ ಕಡಹಗೆ ಹೊರದಾರಿಯ ಹೇರೊಬ್ಬೆ, ಆಗ್ನೇಯ ಹಳೆಯ ಕಾಲುವೆಯ ಹತ್ತಿರ ವಟವೃಕ್ಷ, ಪಡುವಲು ಹಳೆಯ ಕಾಲುವೆಯ ವೊತ್ತುಕೊಂಡು, ಆಗ್ನೇಯದ ಹಳೆಯ ಕಾಲವೆಯೇ ಮೇರೆಯಾಗಿ ಹಳೆಯ ಕಟ್ಟೆಯ ಪರ್ಯಂತ, ಕಾವೇರಿಯ ಸಾಗರದ\index{ಕಾವೇರಿಯ ಸಾಗರ} ಪಡುವಳದಿಂ, ದುಪಗಟ್ಟದ ಕಟ್ಟೇರಿ ಮುಂದಣ ಕೋಡಿ, ಹೊಸ ಕಾಲುವೆಯ ಚದಿರಗಟ್ಟ”} ಮುಂತಾದ ಉಲ್ಲೇಖಗಳನ್ನು ನೋಡಿದರೆ, ಹರಹಿನ ಬಳಿ ಕಾವೇರಿ ನದಿಗೆ\index{ಕಾವೇರಿ ನದಿ} ನಿರ್ಮಿಸಿದ್ದ ಅಣೆಕಟ್ಟೆಯಿಂದ ಹೊರಟ ಹಳೆಯ ಕಾಲುವೆಗಳೂ ಇದ್ದು, ಅದರ ಜೊತೆಗೆ ಹೊಸದಾಗಿ ಕಟ್ಟೆಯನ್ನು ನಿರ್ಮಿಸಲಾಯಿತು ಅಥವಾ ಅದನ್ನೇ ಜೀರ್ಣೋದ್ಧಾರ ಮಾಡಲಾಯಿತು, ಅದರಿಂದ ಹೊಸದಾಗಿ ಕಾಲುವೆಗಳನ್ನು ತೆಗೆಯಲಾಯಿತೆಂಬುದು ತಿಳಿದುಬರುತ್ತದೆ.\endnote{ ಪೂರ್ವೋಕ್ತ} ಪೂರ್ವೋಕ್ತ ತಲೆನೆರೆಯಲ ಕಟ್ಟೆಯೇ ಈ ಹಳೆಯ ಅಣೆಕಟ್ಟಾಗಿರುವ ಸಾಧ್ಯತೆ ಇದೆ. ಅಥವಾ ಇದು ಮೇಲೆ ಸೂಚಿಸಿದ ಹೊಯ್ಸಳರ ಕಾಲದ ಶಾಸನಗಳಲ್ಲಿ ಉಲ್ಲೇಖವಾದ ಕಟ್ಟೆಯೇ ಆಗಿರಬಹುದು. ಹೊಯ್ಸಳರ ಕಾಲದ ಚಂದ್ರವನ ಶಾಸನಗಳಲ್ಲಿ ಈ ಹರಹಿನ ಕಾಲುವೆಯ ಉಲ್ಲೇಖವಿರುವುದು ಇಲ್ಲಿ ಹಳೆಯ ಅಣೆಕಟ್ಟು ಮತ್ತು ಕಾಲುವೆಗಳಿದ್ದುದನ್ನು ದೃಢಪಡಿಸುತ್ತದೆ.

ತಿಮ್ಮಣ್ಣ ದಂಡನಾಯಕನ ನೆಲಮನೆ\index{ನೆಲಮನೆ} ಶಾಸನದಲ್ಲಿ ಬಲ್ಲೇನಹಳ್ಳಿ ಅಗ್ರಹಾರ\index{ಬಲ್ಲೇನಹಳ್ಳಿ ಅಗ್ರಹಾರ} ಸೀಮೆಯ ಗಡಿಗಳನ್ನು ಹೇಳುವಾಗ, ಬಲ್ಲೇನಹಳ್ಳಿಯ ಮೂಡಲು ಲೋಕಪಾವನೆಯ ಸಾಗರವಿತ್ತೆಂದು\index{ಲೋಕಪಾವನೆಯ ಸಾಗರ} ಹೇಳಿದೆ. ಇದು ಲೋಕಪಾವನಿ ನದಿಗೆ ಅಡ್ಡಲಾಗಿ ನಿರ್ಮಿಸಿರುವ ಅಣೆಕಟ್ಟಾಗಿರಬಹುದು. ನಾಗನಾಗನ ಕಾಲುವೆಯಿಂದ ಹೊರಟ ಕಾಲುವೆ ಈ ಲೋಕಪಾವನೆ ಸಾಗರವನ್ನು ಸೇರುತ್ತಿತ್ತು ಎಂದು ಹೇಳಿದೆ. ನಾಗನಾಗನ ಕಟ್ಟೆಗೆ ಕಾವೇರಿ ನದಿಯ ಕಾಲುವೆಯನ್ನು ಹರಿಸಿ, ಅದರ ಹೆಚ್ಚುವರಿ ನೀರನ್ನು ಲೋಕಪಾವನೆಯ ಸಾಗರಕ್ಕೆ ಬಿಡಲಾಗುತ್ತಿತ್ತೆಂದು ಊಹಿಸಬಹುದು. ಅರಕೆರೆಯ ನರಸಿಂಹದೇವರಿಗೆ ಆ ಊರ ಮೊದಲಗಾಲುವೆಯ\index{ಮೊದಲಗಾಲುವೆ} ಬಳಿ ಗದ್ದೆಯನ್ನು ದತ್ತಿ ಬಿಡಲಾಗಿದೆ. \endnote{ ಎಕ 6 ಶ‍್ರೀಪ 110 ಅರಕೆರೆ 1512} ಅರಕೆರೆಯ ಸಮೀಪ ಇರುವ ಮಂಡ್ಯ ಕೊಪ್ಪಲಿನ ಬಳಿ ಕಾವೇರಿ ನದಿಗೆ ಅಡ್ಡಲಾಗಿ ರಾಮಸ್ವಾಮಿ ಅಣೆಕಟ್ಟೆಯನ್ನು ನಿರ್ಮಿಸಲಾಗಿದೆ. ಅದು ನೆಲಮಟ್ಟದಲ್ಲಿದ್ದು, ಒಂದು ಕಾಲುವೆಯು ಅರಕೆರೆಯ ಕಡೆಗೆ ಹೋಗುತ್ತದೆ. ಇದೇ ಅರಕೆರೆಯ ಮೊದಲಗಾಲುವೆ ಇರಬಹುದು. ಕೃಷ್ಣದೇವರಾಯನ ಮಂಡ್ಯ ತಾಮ್ರಶಾಸನದಲ್ಲಿ ತಮ್ಮಡಿಗಟ್ಟೆಯ ಕಾಲುವೆಯನ್ನು ಉಲ್ಲೇಖಿಸಿದೆ.\endnote{ ಎಕ 7 ಮಂ 7 ಮಂಡ್ಯ 1516}

ಶ‍್ರೀರಂಗಪಟ್ಟಣ ಸೀಮೆಯೊಳಗಣ, ಕಾವೇರಿ ಕಟ್ಟುಕಾಲುವೆಯೊಳಗಾದ\index{ಕಾವೇರಿ ಕಟ್ಟುಕಾಲುವೆ} ಬಲ್ಲಾಳಪುರ ಸ್ಥಳದ\index{ಬಲ್ಲಾಳಪುರ ಸ್ಥಳ} (ಬಲ್ಲೇನಹಳ್ಳಿ) ಕಾಲುವಳ್ಳಿಗಳ ಸೀಮೆಯನ್ನು (ಉಪಗ್ರಾಮಗಳನ್ನು), ಕಂಣಂಬಾಡಿ ಹೋಬಳಿಯ, ಮೊಳನಾಡ ಸ್ಥಳದ ಹೇಮಾವತಿ ಕಟ್ಟುಕಾಲುವೆ\-ಯೊಳಗಾದ\index{ಹೇಮಾವತಿ ಕಟ್ಟುಕಾಲುವೆ} ವರಾಹನ ಕಲ್ಲಹಳ್ಳಿ ಸ್ಥಳ ಮತ್ತು ಅದಕ್ಕೆ ಸೇರಿದ ಉಪಗ್ರಾಮಗಳ ಸೀಮೆಯನ್ನು ಯಾದವಗಿರಿಯ ಶ‍್ರೀ ನಾರಾಯಣದೇವರ ಪೂಜೆಗೆ, ರಾಮಾನುಜಕೂಟಕ್ಕೆ, ವೇದಾಂತಿ ರಾಮಾನುಜಜೀಯರ ಮಠಕ್ಕೆ ದತ್ತಿಯಾಗಿ ಬಿಡುತ್ತಾನೆ.\endnote{ ಎಕ 6 ಪಾಂಪು 129 ಮೇಲುಕೋಟೆ 1545} ಕಾವೇರಿ ಮತ್ತು ಹೇಮಾವತಿ ನದಿಗಳಿಗೆ ಇಲ್ಲಿ ಅಣೆಕಟ್ಟುಗಳನ್ನು ನಿರ್ಮಿಸಿ, ಕಾಲುವೆಗಳನ್ನು ನಿರ್ಮಿಸಿರಬಹುದೆಂದು ಹೇಳಬಹುದು. ಇಲ್ಲಿಂದ ಮುಂದೆಯೇ ಕೃಷ್ಣರಾಜಸಾಗರ ಜಲಾಶಯವನ್ನು ನಿರ್ಮಿಸಲಾಗಿದ್ದು, ಅದರ ನೀರು ವರಾಹನಾಥ ಕಲ್ಲಹಳ್ಳಿಯವರೆಗೂ ಬಂದು ನಿಲ್ಲುತ್ತದೆ. ಬಹುಶಃ ಇದೇ ಸ್ಥಳದಲ್ಲಿ ಚಿಕ್ಕದೇವರಾಜ ಒಡೆಯರು ಕಾವೇರಿ ನದಿಗೆ ಕಟ್ಟೆಯನ್ನು ಕಟ್ಟಿಸಿರಬಹುದೆಂದು, ಆ ನಂತರ ಟಿಪ್ಪೂಸುಲ್ತಾನ್​ ಕೂಡಾ ಕಾವೇರಿ ನದಿಗೆ ಇಲ್ಲಿ 70 ಅಡಿ ಎತ್ತರದ ಕಟ್ಟೆಯನ್ನು ನಿರ್ಮಿಸಿದ್ದನೆಂದೂ ಹೇಳಲಾಗಿದೆ.\endnote{ ಶಂಸ ಐತಾಳ, ಕೃಷ್ಣರಾಜಸಾಗರ, ಪುಟ 4-5} ಆದರೆ ವಿಜಯನಗರ ಕಾಲದಲ್ಲೇ ಕಾವೇರಿ ನದಿಗೆ ಇಲ್ಲಿ ಕಟ್ಟೆಯನ್ನು ಕಟ್ಟಿದ್ದ ವಿಚಾರ ಮೇಲ್ಕಂಡ ಶಾಸನದಿಂದ ತಿಳಿದುಬರುತ್ತದೆ. ಕಾಲುವೆಯ ನೀರನ್ನು ಸರದಿಯ ಮೂಲಕ ಹರಿಸಲಾಗುತ್ತಿದ್ದ ವಿಷಯ ಕನ್ನಂಬಾಡಿ ಶಾಸನದಿಂದ ತಿಳಿದುಬರುತ್ತದೆ.\endnote{ ಎಕ 6 ಪಾಂಪು 30 ಕನ್ನಂಬಾಡಿ 1553}

\section*{ಮೈಸೂರು ಅರಸರ ಕಾಲದ ಅಣೆ ಅಚ್ಚುಕಟ್ಟುಗಳು}

ದೇವರಾಜ ಒಡೆಯರ ಶಾಸನದಲ್ಲಿ ಬೆಳಕವಾಡಿ ಗ್ರಾಮದ ಉಪಗ್ರಾಮದ ಎಲ್ಲೆಗಳನ್ನು ಹೇಳುವಾಗ, ‘ಹೊಳೆ ಕಟ್ಟೆಹಳ್ಳದಿಂ ಮೂಡಲು’ ಎಂದು ಹೇಳಿದೆ. ಇಲ್ಲಿಗೆ ಸಮೀಪದಲ್ಲಿ ಹರಿಯುವ ಕಾವೇರಿನದಿಗೆ ಕಟ್ಟೆಯನ್ನು ಕಟ್ಟಲಾಗಿತ್ತೆಂದು ಊಹಿಸ\-ಬಹುದು.\endnote{ ಎಕ 7 ಮವ 98 ಬೆಳಕವಾಡಿ 1603} ಶಿಂಶದ ಬಳಿ ಒಂದು ಹಳೆಯ ಕಾಲದ ಅಣೆಕಟ್ಟಿದೆ.

ಅದೇ ರೀತಿ ಒಡೆಯರ ಕಾಲದ ಅರಕೆರೆ ಶಾಸನದಲ್ಲಿ, ಚಿಕ್ಕಸಿಂಗರಾಯರ ಕೆರೆ\index{ಚಿಕ್ಕಸಿಂಗರಾಯರ ಕೆರೆ} ಮತ್ತು ಕಾಲುವೆಗಳ ಪ್ರಸ್ತಾವಿದೆ.\endnote{ ಎಕ 6 ಶ‍್ರೀಪ 101 ಅರಕೆರೆ 17-18ನೇ ಶ.} 1792 ರಲ್ಲಿ ಕಾರ್ನ್​ವಾಲೀಸ್​ನು ಶ‍್ರೀರಂಗಪಟ್ಟಣದ ಪೂರ್ವಕ್ಕೆ ಒಂಭತ್ತು ಮೈಲಿ ದೂರದಲ್ಲಿ ಕಾವೇರಿಗೆ ಅಡ್ಡಲಾಗಿ ಕಟ್ಟಿದ ಅರಕೆರೆಯ ಕಲ್ಲು ಅಣೆಯನ್ನು ಕಂಡನೆಂದು ತಿಳದುಬರುತ್ತದೆ. ಬಹುಶಃ ಇದೇ ಅಣೆಕಟ್ಟೆಯ ಕಾಲುವೆಯನ್ನು ಮೇಲಿನ ಶಾಸನ ಉಲ್ಲೇಖಿಸಿರಬಹುದೆಂದು ತೋರುತ್ತದೆ.

ತೊಣ್ಣೂರು ತಾಮ್ರಶಾಸನದಲ್ಲಿ ಯಾದವಪುರಿ ಹೋಬಳಿಯ ಹಳ್ಳಿಗಳ ಎಲ್ಲೆಯನ್ನು ಹೇಳುವಾಗ, “ಬೊಪ್ಪನಹಳ್ಳಿಗೆ ದಕ್ಷಿಣ ತೊರೆಯ ಬಳಿಯಣ ಕಲ್ಲಣೆ ಮತ್ತು ದೊಡ್ಡ ಕಾಲುವೆಯ ಉಲ್ಲೇಖವಿದೆ. ಕಲ್ಲಣೆ ಅಂದರೆ ಹಳ್ಳಗಳಿಗೆ ಕಟ್ಟುತ್ತಿದ್ದ ಚೆಕ್​ ಡ್ಯಾಮ್ ಎಂದು ಹೇಳಬಹುದು.\endnote{ ಎಕ 6 ಪಾಂಪು 99 ತೊಣ್ಣೂರು 1722} ಬೊಪ್ಪನಹಳ್ಳಿಯು ಮೇಲುಕೋಟೆ ಕಣಿವೆಯಲ್ಲಿರುವ ಹಳ್ಳಿಯಾಗಿದೆ.

\section*{ಹಳ್ಳಕೊಳ್ಳಗಳು}
\index{ಹಳ್ಳಕೊಳ್ಳಗಳು}

ಊರಿನ ಸುತ್ತ ಮುತ್ತ ನದಿಗಳು ಮತ್ತು ಅನೇಕ ಹಳ್ಳಕೊಳ್ಳಗಳು ಹರಿಯುತ್ತಿದ್ದವು. ಸಾಮಾನ್ಯವಾಗಿ ಎಲ್ಲ ಹಳ್ಳಿಗಳೂ ಕೂಡಾ ನದಿಗಳು ಮತ್ತು ಇಂತಹ ದೊಡ್ಡ ಅಥವಾ ಸಣ್ಣ ಹಳ್ಳಗಳಿಗೆ ಸಮೀಪದಲ್ಲೇ ಇರುತ್ತಿದ್ದವು. ಇಂತಹ ಅನೇಕ ನದಿಗಳು ಮತ್ತು ಹಳ್ಳಗಳ ಹೆಸರುಗಳನ್ನು ಶಾಸನಗಳಲ್ಲಿ ಹೇಳಲಾಗಿದೆ.\endnote{ \engfoot{Gururajachar Dr.S.,Agriculture, Economic and Social Life in Karnataka pp 53}} ಈ ಹಳ್ಳಕೊಳ್ಳಗಳನ್ನು ದತ್ತಿಯ ಭೂಮಿಗೆ ಮೇರೆಯಾಗಿ ಹೇಳಿದೆ. ಈ ಹಳ್ಳಗಳು ಊರಿನ ಕೆರೆಗೆ ನೀರುಣಿಸುತ್ತಿದ್ದವು. ಅಥವಾ ಸಮೀಪದ ನದಿಯನ್ನು ಸೇರುತ್ತಿದ್ದವು. ಇದರಿಂದ ರೈತರು ಈ ಹಳ್ಳಗಳನ್ನು ಅತಿಕ್ರಮಿಸದೇ ಕಾಪಾಡಿಕೊಂಡು ಬರುತ್ತಿದ್ದರು. ಅನೇಕ ಕಡೆ ಈ ಹಳ್ಳಗಳಿಗೆ ಅಣೆಯನ್ನೂ ನಿರ್ಮಿಸಿದ್ದರು. ಕಳೆದ 40-50 ವರ್ಷಗಳ ಹಿಂದೆಯೂ ರೈತರು ಈ ಹಳ್ಳಗಳನ್ನು ಕಾಪಾಡಿಕೊಂಡು ಬರುತ್ತಿದ್ದರು. ಆ ಹಳ್ಳಗಳ ಹೆಸರನ್ನೂ ಹೇಳುತ್ತಿದ್ದರು. ಆದರೆ ಇಂದು ಈ ಹಳ್ಳಗಳು ಅತಿಕ್ರಮಣಕ್ಕೆ ಒಳಗಾಗಿ ಮುಚ್ಚಿಹೋಗಿವೆ ಅಥವಾ ಸಣ್ಣ ಚರಂಡಿಯ ರೂಪಕ್ಕೆ ಬಂದಿವೆ. ಈ ಹಳ್ಳಕೊಳ್ಳಗಳ ಹೆಸರೂ ಜನಗಳ ಮನಸ್ಸಿನಿಂದ ಮರೆಯಾಗಿವೆ. ಆದರೆ ಶಾಸನಗಳಲ್ಲಿ ಈ ರೀತಿಯ ಅನೇಕ ಹಳ್ಳ ಕೊಳ್ಳಗಳ ಹೆಸರುಗಳನ್ನು ನೀಡಿರುವುದು ವಿಶೇಷವಾಗಿದೆ. ಆದರೆ ಈ ಹಳ್ಳಗಳಲ್ಲಿ ಇಂದು ಎಷ್ಟು ಉಳಿದಿವೆ, ಮೂಲಹೆಸರಿನಲ್ಲಿಯೇ ಅವು ಉಳಿದಿವೆಯೇ, ಎಷ್ಟು ಹಳ್ಳಗಳು ಕಣ್ಮರೆಯಾಗಿವೆ ಎಂಬುದನ್ನು ಬೇರೆಯಾಗಿಯೇ ಅಧ್ಯಯನಕ್ಕೆ ಒಳಪಡಿಸಬಹುದು.

ಶಿವಮಾರನ ಹಳ್ಳೆಗೆರೆ ತಾಮ್ರಪಟಗಳಲ್ಲಿ, ಇರ್ಗ್ಗರೆ ನದಿ\index{ಇರ್ಗ್ಗರೆ ನದಿ} ಮತ್ತು ನೇಸರಪಳ್ಳವನ್ನು ಉಲ್ಲೇಖಿಸಿದೆ.\endnote{ ಎಕ 7 ಮಂ 35 ಹಳ್ಳೆಗೆರೆ 713} ಇರ್ಗ್ಗರೆ ನದಿಯು ಶಿಂಶಾ ನದಿ\index{ಶಿಂಶಾ ನದಿ} ಇರಬಹುದು. ಮಳವಳ್ಳಿ ತಾ. ಕಿರಗಸೂರು ಶಾಸನದಲ್ಲಿ ಶ‍್ರೀ ಕಾವೇರೀ ಮಹಾನದಿ ಪಶ್ಚಿಮವಾಹಿನಿ\index{ಪಶ್ಚಿಮವಾಹಿನಿ} ತೀರವಾದ ಗಜಾರಣ್ಯ ಕ್ಷೇತ್ರವಾದ ತಳಕಾಡು ಎಂದು ಹೇಳಿದ್ದು, ಕಾವೇರಿಯನ್ನು ಮಹಾನದಿ\index{ಮಹಾನದಿ} ಎಂದು ಕರೆಯಲಾಗಿದೆ.\endnote{ ಎಕ 7 ಮವ 102 ಕಿರಗಸೂರು 1440} ದಕ್ಷಿಣ ವಾರಣಾಸಿ ನಂಜನಗೂಡು ಕ್ಷೇತ್ರವು ಕಪಿಲಾ ಕೌಂಡಿಣ್ಯ ನದಿಗಳ ಸಂಗಮಕ್ಷೇತ್ರದಲ್ಲಿತ್ತೆಂದು ಹೇಳಿದೆ. ಆದರೆ ಈಗ ಕೌಂಡಿಣ್ಯ ನದಿಯು ಹೆಸರಿಗೆ ಮಾತ್ರ ನದಿಯಾಗಿದ್ದು ಸಣ್ಣ ಹಳ್ಳದಂತಿದೆ.\endnote{ ಎಕ 7 ಮವ 146 ಕಲ್ಕುಣಿ 1511} ಮೈಸೂರು ಒಡೆಯರ ಶಾಸನಗಳಲ್ಲಿ ಕಾವೇರಿಯನ್ನು “ಸಹ್ಯಜಾ” ಎಂದು ಕರೆಯಲಾಗಿದೆ, ಅಂದರೆ ಸಹ್ಯಾದ್ರಿ ಪರ್ವತದಿಂದ ಹುಟ್ಟುವ ನದಿ ಎಂದು ಅರ್ಥೈಸಬಹುದು.

ಪ್ರತಿಯೊಂದು ಊರಿನ ಪಕ್ಕದಲ್ಲಿ, ಆಸುಪಾಸಿನಲ್ಲಿ ಅನೇಕ ಹಳ್ಳಗಳು\index{ಹಳ್ಳಗಳು} ಅಥವಾ ತೊರೆಗಳು\index{ತೊರೆಗಳು} ಹರಿಯುತ್ತಿದ್ದವು \textbf{“ಹಿಂದಣ ಹಳ್ಳ ಮುಂದಣ ತೊರೆ ಸಲ್ಲುವ ಪರಿಯೆಂತು ಹೇಳಾ”} ಎಂದು ಅಕ್ಕಮಹಾದೇವಿಯು ತನ್ನ ವಚನದಲ್ಲಿ ಹೇಳಿದ್ದಾಳೆ. ನಾಗಮಂಗಲ ಶಾಸನದಲ್ಲಿ, ಮೊದಲಿಹಳ್ಳಿಯ ಕುಡಿಹಳ್ಳ, ಅರಸೆಟಿಯಹಳ್ಳಿಯ ಎಡೆಹಳ್ಳ, ತಡಿನ ಹಣೆಯ ಪಡುವಣಹಳ್ಳ, ಕೆಂತಟಿಯಹಳ್ಳ, ಶುಂಣ್ನಹರಳ ಹಳ್ಳ, ರಾಜನಹಳ್ಳ, ಮಂದರಿಗೆಯಹಳ್ಳ, ಚಂಚರೀಹಳ್ಳ ಎಂದು ಅನೇಕ ಹಳ್ಳಕೊಳ್ಳಗಳ ಹೆಸರನ್ನು ಹೇಳಿದೆ. ಆ ಹಳ್ಳಗಳನ್ನು ಈಗ ಗುರುತಿಸಲೂ ಸಾಧ್ಯವಿಲ್ಲ.\endnote{ ಎಕ 7 ನಾಮಂ 1 ನಾಗಮಂಗಲ 1171-73} ಸುಣ್ಣಹರಳಿನಹಳ್ಳದ ಪ್ರಸ್ತಾಪ ಲಾಳನಕೆರೆ ಶಾಸನದಲ್ಲೂ ಇದೆ.\endnote{ ಎಕ 7 ನಾಮಂ 61 ಲಾಳನಕೆರೆ 1138} ಕಸಲಗೆರೆ ಶಾಸನದಲ್ಲಿ ದತ್ತಿಯಮೇರೆಗಳನ್ನು ಹೇಳುವಾಗ, ಪಡುವಣ ಹರಳಹಳ್ಳ, ಬಂಡೆಹಳ್ಳ, ಉಪ್ಪುವಳ್ಳ, ಹಿರಿಯಹಳ್ಳಗಳ ಉಲ್ಲೇಖವಿದೆ.\endnote{ ಎಕ 7 ನಾಮಂ 168 ಕಸಲಗೆರೆ 1190}

ನೆಲಮನೆ ಶಾಸನದಲ್ಲಿ ಅಗ್ರಹಾರದ ಮೇರೆಗಳನ್ನು ಹೇಳುವಾಗ ರಂಗಸಮುದ್ರ ತೆಂಕಣ ಕೋಡಿಯ ಹಳ್ಳ,\index{ತೆಂಕಣ ಕೋಡಿಯ ಹಳ್ಳ} ರಂಗಸಮುದ್ರ ಬಡಗಣ ಕೋಡಿಯ ಹಳ್ಳ,\index{ಬಡಗಣ ಕೋಡಿಯ ಹಳ್ಳ} ಕೆಂಬರೆಹಳ್ಳ, ಕುಂಬಾರಗುಂಡಿಯ ಹಳ್ಳಗಳ ಉಲ್ಲೇಖವಿದ್ದು ಇವೆಲ್ಲಾ ಕಾವೇರಿ ಅಥವಾ ಲೋಕಪಾವನಿ ನದಿಗೆ ಸೇರುತ್ತಿದ್ದ ಹಳ್ಳಗಳಾಗಿದ್ದವೆಂದು ಊಹಿಸಬಹುದು.\endnote{ ಎಕ 6 ಪಾಂಪು 93 ನೆಲಮನೆ 1458} ಸೀತಾಪುರ ಶಾಸನದಲ್ಲಿ ಕಾವೇರಿ ನದಿಗೆ ಸೇರುತ್ತಿದ್ದ ಹಳ್ಳಗಳ ಪ್ರಸ್ತಾಪವಿದೆ.\endnote{ ಎಕ 6 ಪಾಂಪು 19 ಸೀತಾಪುರ 1467} ಸುಜ್ಜಲೂರು ತಾಮ್ರಶಾಸನದಲ್ಲಿ ಹೊನ್ನಹಳ್ಳ, ಪಪ್ಪರ ಹಳ್ಳಗಳ ಉಲ್ಲೇಖವಿದೆ.\endnote{ ಎಕ 7 ಮವ 139 ಸುಜ್ಜಲೂರು 1473} ನಡಗಲಪುರ ಶಾಸನದಲ್ಲಿ ಹೇರೊಬ್ಬೆ ಹಳ್ಳ ಮತ್ತು ಗೌಂಡಿಯ ಹಳ್ಳದ ಉಲ್ಲೇಖವಿದೆ. ಈ ಹಳ್ಳಗಳ ನಂತರ ಕೆರೆಯ ಒಳಗೆರೆಯ ಉಲ್ಲೇಖವಿರುವುದರಿಂದ ಈ ಹಳ್ಳಗಳು ಈ ಕೆರೆಗೆ ಕೂಡುತ್ತಿದ್ದವೆಂದು ಹೇಳಬಹುದು.\endnote{ ಎಕ 7 ಮವ 44 ನಡಗಲಪುರ 1510} ಕೃಷ್ಣದೇವರಾಯನ ಮಂಡ್ಯ ತಾಮ್ರಶಾಸನದಲ್ಲಿ, ಚಿಕ್ಕಮಂಟೆಯದ ಈಶಾನ್ಯದ ಹೆಬ್ಬಳ್ಳವನ್ನು ಉಲ್ಲೇಖಿಸಿದೆ. ಈ ಹಳ್ಳ ಈಗಲೂ ಹೆಬ್ಬಳ್ಳವೆಂಬ ಹೆಸರಿನಿಂದ ಹರಿಯುತ್ತಿದೆ. ಯಡಗೋಡಿ ಹಳ್ಳ, ಕಟ್ಟೊಬ್ಬೆ\-ಹಳ್ಳ, ಡೊಗರ ಹಳ್ಳಗಳ ಉಲ್ಲೇಖವಿದೆ.\endnote{ ಎಕ 7 ಮಂ 7 ಮಂಡ್ಯ 1516} ಅಚ್ಯುತರಾಯನ ಬ್ಯಾಲದಕೆರೆ ತಾಮ್ರಶಾಸನದಲ್ಲಿ ಮೇರೆಗಳನ್ನು ಹೇಳುವಾಗ, ಗಿರಿಯಿಂದ ಹರಿದು ಬರುತ್ತಿದ್ದ ತೊರೆ, ಅರೆಬೊಪ್ಪನಹಳ್ಳಿಯ ತೀರ್ಥವೆಂದು ಹೆಸರಾದ ತೊರೆ, ಅಯ್ಯಗೊಂಡನಪಲ್ಲಿಯ ಮಧ್ಯದಲ್ಲಿ ಹರಿದು ಹೋಗುವ ತೊರೆಗಳನ್ನು ಉಲ್ಲೇಖಿಸಿದೆ. ಸಿಂದಘಟ್ಟದಲ್ಲಿ ಸಾಗರಮಾರಿಕಾ ಎಂಬ ದೇವಾಲಯವಿದ್ದಿತಂತೆ. ಬಹುಶಃ ಇದು ಸಿಂದಘಟ್ಟದ ದೊಡ್ಡ ಕೆರೆಯ ದಡದಲ್ಲಿರುವ ಗ್ರಾಮದೇವತೆ ದೇವಾಲಯವಿರಬಹುದು.\endnote{ ಎಕ 6 ಕೃಪೇ 99 ಬ್ಯಾಲದಕೆರೆ 1532} ಹೊನ್ನೇನಹಳ್ಳಿ ತಾಮ್ರಶಾಸನದಲ್ಲಿ ಬಕನೇಕಲಗುಡ್ಡ ಹಳ್ಳ, ಕೆಂಬರೇನ ಹಳ್ಳ, ಬಾಳೆಹಳ್ಳ, ಅಪ್ಪಳಕ್ಕನಹಳ್ಳಗಳ ಉಲ್ಲೇಖವಿದೆ. ಇದೇ ಶಾಸನದಲ್ಲಿ \textbf{“ಕ್ಷುದ್ರ ಶೈವಾಲಿನೀ ತೀರಾದ್ದಕ್ಷಿಣಸ್ಯಾಂ ದಿಶಿಸ್ಥಿತಂ”} ಎಂದು ಹೇಳಿದ್ದು, ಇದನ್ನು ಚಿಕ್ಕದಾದ ಶೈವಾಲಿನಿ\index{ಶೈವಾಲಿನಿ} ನದಿಯ ದಕ್ಷಿಣ ತೀರ ಎಂದು ಅರ್ಥೈಸಬಹುದು. ಹೊನ್ನೇನಹಳ್ಳಿ ಬಳಿ ಹರಿಯುವ ಶಿಂಶಾನದಿಯೇ\index{ಶಿಂಶಾನದಿ} ಕ್ಷುದ್ರಶೈವಾಲಿನಿ ಆಗಿರಬಹುದು.\endnote{ ಎಕ 7 ನಾಮಂ 107 ಹೊನ್ನೇನಹಳ್ಳಿ 1545} ದೇವರಾಜ ಒಡೆಯನ ಶ‍್ರೀರಂಗಪಟ್ಟಣ ತಾಮ್ರಶಾಸನದಲ್ಲಿ ಬಳಗೊಳ ಸ್ಥಳದ ಅವ್ವೇರಹಳ್ಳಿ ಮೇರೆಗಳನ್ನು ಹೇಳುವಾಗ, ಮೊರದನಕಟ್ಟೆ ಹಳ್ಳ, ಕೊರಕಲ ಹಳ್ಳಗಳ ಉಲ್ಲೇಖವಿದೆ.\endnote{ ಎಕ 6 ಶ‍್ರೀಪ 24 ಶ‍್ರೀರಂಗಪಟ್ಟಣ 1686} ಇಮ್ಮಡಿ ಕೃಷ್ಣರಾಜರ ತೊಣ್ಣೂರು ತಾಮ್ರ ಶಾಸನದಲ್ಲಿ ಯಾದವಗಿರಿ ಹೋಬಳಿಯ ಹಳ್ಳಿಗಳ ಎಲ್ಲೆಗಳನ್ನು ಉಲ್ಲೇಖಿಸುವಾಗ, ಕರಿಕಲ್ಲ ಹಳ್ಳ, ಬೇಲೆಕೆರೆಗೆ ನೈರುತ್ಯದ ತೊರೆ, ಮೋದೂರಿಗೆ ಬಡಗಣಹಳ್ಳ, ತೊಳಸಿ ತಿಮ್ಮನಹಳ್ಳಗಳನ್ನು ಉಲ್ಲೇಖಿಸಿದೆ.\endnote{ ಎಕ 7 ಪಾಂಪು 99 ತೊಣ್ಣೂರು 1722} ಮೇಲುಕೋಟೆ ತಾಮ್ರಶಾಸನದಲ್ಲಿ ಹುಳ್ಳೇನಹಳ್ಳಿಗೆ ಎಲ್ಲೆಗಳನ್ನು ಹೇಳುವಾಗ ನರಿಗಲ್ಲತೊರೆ ಎಂಬ ಹಳ್ಳದ ಹೆಸರನ್ನು ಹೇಳಿದೆ.\endnote{ ಎಕ 6 ಪಾಂಪು 216 ಮೇಲುಕೋಟೆ 1725}

\vskip 5pt

\section*{ನದಿಗಳು/ಅಣೆಕಟ್ಟುಗಳು}

ಮಂಡ್ಯ ಜಿಲ್ಲೆಯ ಪ್ರಮುಖ ನದಿ ಕಾವೇರಿ. ಈ ನದಿಯ ಶಾಸನೋಕ್ತ ಉಲ್ಲೇಖಗಳನ್ನು ಮತ್ತು ಇದಕ್ಕೆ ಕಟ್ಟಿದ ಅಣೆಕಟ್ಟುಗಳ ವಿವರಗಳನ್ನು ಈ ಹಿಂದೆಯೇ ವಿವರಿಸಲಾಗಿದೆ. ಶಾಸನೋಕ್ತವಲ್ಲದ ಕೆಲವು ಅಣೆ ಅಚ್ಚುಕಟ್ಟುಗಳ ವಿವರಗಳನ್ನು ಕ್ಷೇತ್ರ ಕಾರ್ಯದ ಮೂಲಕ ಸಂಗ್ರಹಿಸಿ ನೀಡಲಾಗಿದೆ. ಕಾವೇರಿ ನದಿಗೆ ಅನೇಕ ಕಡೆ ಗಂಗರು, ಹೊಯ್ಸಳರು ಮತ್ತು ವಿಜಯನಗರ ಕಾಲದಲ್ಲಿ ಅಣೆಕಟ್ಟುಗಳನ್ನು ನಿರ್ಮಿಸಿದ ವಿಚಾರವನ್ನು ಈ ಹಿಂದೆಯೇ ಉಲ್ಲೇಖಿಸಲಾಗಿದೆ.

ಕಾವೇರಿ ನದಿಗೆ ತಿಮ್ಮಣ್ಣ ದಂಡನಾಯಕನ ತಮ್ಮ ದೇವರಾಜನು,\index{ದೇವರಾಜ} ಇಂದಿನ ಎಡಮುರಿಯ\index{ಎಡಮುರಿ} ಬಳಿ ಒಂದು ಅಣೆಕಟ್ಟೆಯನ್ನು ನಿರ್ಮಿಸಿದ್ದನು. ಈ ಅಣೆಕಟ್ಟಿನಿಂದ ದೇವರಾಜನು, ಹರಹಿನ ಮಹಾಜನಗಳು\index{ಹರಹಿನ ಮಹಾಜನಗಳು} ನೆರವಿನಿಂದ ಹರಹಿನವರೆಗೂ ನಾಲೆಯನ್ನು ತೆಗೆದುಕೊಂಡು ಹೋದ ವಿಚಾರ ಶಾಸನೋಕ್ತವಾಗಿದೆ. ಒಂದನೇ ನರಸಿಂಹನ ಕಾಲದ ಮೇಲುಕೋಟೆ ಶಾಸನದಲ್ಲೂ ಹರಹಿನ ನಾಲೆಯ ಬಯಲನ್ನು ‘ಪಾಪ್ಪಾಲಿ ಕಾವಲ್​\index{ಪಾಪ್ಪಾಲಿ ಕಾವಲ್​} ವಯಲಿಲ್​” ಎಂದು ಹೇಳಿದೆ. ಗಂಗರ ಕಾಲದಲ್ಲಿ ಕಾವೇರಿ ನದಿಗೆ ಅಡ್ಡಲಾಗಿ ನಿರ್ಮಿಸಿದ್ದ ಅಣೆಕಟ್ಟೆಯನ್ನು ಚೋಳರ ಆಕ್ರಮಣ ಕಾಲದಲ್ಲಿ ಒಡೆದುಹಾಕಿದ್ದರೆಂದು ತೋರುತ್ತದೆ. ವಿಜಯನಗರ ಕಾಲದಲ್ಲಿ ದೇವರಾಜನು ಇಲ್ಲಿ ಪುನಃ ಅಣೆಕಟ್ಟನ್ನು ನಿರ್ಮಿಸಿ, ಹರಹಿನವರೆಗೂ ನಾಲೆಯನ್ನು ತೆಗೆದುಕೊಂಡು ಹೋದನು. ಈ ಶಾಸನದಲ್ಲಿ ಹಳೆಯ ಕಾಲುವೆಗಳ ಉಲ್ಲೇಖವಿರುದನ್ನು ಈಗಾಗಲೇ ಉಲ್ಲೇಖಿಸಲಾಗಿದೆ. ದೇವರಾಜನ ಕಾಲದಲ್ಲಿ ಈ ಅಣೆಕಟ್ಟೆಯ ಉತ್ತರಕ್ಕೆ ಹೊರಡುವ ನಾಲೆಯು, ಸೀತಾಪುರ, ಹರವು, ಕ್ಯಾತನಹಳ್ಳಿ ಇವುಗಳನ್ನು ಸುತ್ತಿಕೊಂಡು ಹೋಗುತ್ತದೆ. ಇಂದಿಗೂ ಈ ಕಾಲುವೆಯಿಂದ ಸಾವಿರಾರು ಎಕರೆ ನೀರಾವರಿ ಆಗುತ್ತಿದೆ. ಈ ನಾಲೆಯನ್ನು ದೊಡ್ಡ ದೇವರಾಜ ನಾಲೆ\index{ದೊಡ್ಡ ದೇವರಾಜ ನಾಲೆ} ಎಂದು ಕರೆಯುತ್ತಾರೆಂದು ಸ್ಥಳೀಯ ಹಿರಿಯರು ತಿಳಿಸಿದರು. ದೊಡ್ಡ ದೇವರಾಜನೆಂದರೆ, ಈ ಅಣೆಕಟ್ಟೆಯನ್ನು ನಿರ್ಮಿಸಿದ ತಿಮ್ಮಣ್ಣ ದಂಡನಾಯಕನ ತಮ್ಮ ದೇವರಾಜನೇ ಆಗಿದ್ದಾನೆಂದು ಹೇಳಬಹುದು.

ಮೈಸೂರು ಒಡೆಯರ ಕಾಲದಲ್ಲಿ ಮಂಡ್ಯ ಜಿಲ್ಲೆಯ ನೀರಾವರಿಗೆ ಸಂಬಂಧಿಸಿದಂತೆ ಕಾವೇರಿ ನದಿಗೆ ಅನೇಕ ಅಣೆ ಅಚ್ಚುಕಟ್ಟುಗಳು ನಿರ್ಮಾಣವಾಗಿವೆ. ಇಂದಿನ ಕನ್ನಂಬಾಡಿ ಕಟ್ಟೆಯ ಮುಂದೆ, ಎಡಮುರಿಯ ಬಳಿ(ಕಾವೇರಿ ನದಿ ಎಡಕ್ಕೆ ತಿರುಗುತ್ತದೆ), ಇಲ್ಲಿ ವಿಜಯನಗರ ಕಾಲದಲ್ಲಿ ದೇವರಾಜನು ನಿರ್ಮಿಸಿದ್ದ ಅಣೆಕಟ್ಟು ಜೀರ್ಣವಾಗಿ ಒಡೆದುಹೋಗಿರಲು, ಚಿಕ್ಕದೇವರಾಜ ಒಡೆಯರು, ಪುನರ್​ ನಿರ್ಮಾಣ ಮಾಡಿ ಸುಮಾರು 36 ಅಡಿ ಎತ್ತರದ ಅಣೆಕಟ್ಟೆಯನ್ನು ನಿರ್ಮಿಸಿದ್ದರೆಂದೂ, ಅದು ಮತ್ತೆ ಪ್ರವಾಹದ ಕಾರಣ ಒಡೆದು ಹೋಯಿತೆಂದೂ ಹೇಳುತ್ತಾರೆ. ಈಗಲೂ ಈ ಅಣೆಕಟ್ಟು ಮಧ್ಯದಲ್ಲಿ ಒಡೆದುಹೋಗಿದೆ. ಎರಡೂ ಕಡೆಗೂ ಅಣೆಕಟ್ಟು ಉಳಿದಿದೆ. ಈಗ ಇದನ್ನು ಚಿಕ್ಕದೇವರಾಯ ಸಾಗರ ಎಂದು ಕರೆಯುತ್ತಾರೆ. ಇದರಿಂದ ದಕ್ಷಿಣದ ಕಡೆಗೆ ಹೊರಟಿರುವ ನಾಲೆಗೆ ಚಿಕ್ಕದೇವರಾಯಸಾಗರ ನಾಲೆ\index{ಚಿಕ್ಕದೇವರಾಯಸಾಗರ ನಾಲೆ} ಎಂದೂ ಹೇಳುತ್ತಾರೆ. ಇಂದಿಗೂ ಈ ನಾಲೆಯು ಬನ್ನೂರಿನವರೆಗೂ ಹೋಗುತ್ತದೆ. ಇದರಿಂದ ಸಾವಿರಾರು ಎಕರೆ ನೀರಾವರಿಯಾಗುತ್ತದೆ. ಉತ್ತರದ ಕಡೆಗೆ ಹೊರಟ ದೊಡ್ಡ ದೇವರಾಜ ನಾಲೆಯ ವಿಚಾರವನ್ನು ಮೇಲೆ ವಿವರಿಸಲಾಗಿದೆ. ಎಡಮುರಿ ಅಣೆಕಟ್ಟಿನ ಮೇಲೆ ನಿಂತು ಪಶ್ಚಿಮ ದಿಕಿಗೆ ನೋಡಿದರೆ, ಇಂದಿನ ಕೃಷ್ಣರಾಜಸಾಗರ ಅಣೆಕಟ್ಟು ಕಾಣುತ್ತದೆ.

ಶ‍್ರೀರಂಗಪಟ್ಟಣ ತಾಲ್ಲೂಕು ಬಲಮುರಿಯ\index{ಬಲಮುರಿ} ಬಳಿ, ಕಾವೇರಿ ನದಿಗೆ ನಿರ್ಮಿಸಿರುವ ಅಣೆಕಟ್ಟು ಚಿಕ್ಕದೇವರಾಜ ಒಡೆಯರ\index{ಚಿಕ್ಕದೇವರಾಜ ಒಡೆಯರು} ಕಾಲದಲ್ಲಿ ನಿರ್ಮಿತವಾಗಿದೆ. ಹಿಂದೆ ಇಲ್ಲಿ ನಿರ್ಮಿತವಾಗಿದ್ದ ಅಣೆಕಟ್ಟು ಜೀರ್ಣವಾಗಿರಲು ಹೊಸದಾಗಿ ಅದರ ಮೇಲೆ ಅಣೆಯನ್ನು ನಿರ್ಮಿಸಿರುವಂತೆ ಕಂಡು ಬರುತ್ತದೆ. ಈ ಅಣೆಕಟ್ಟೂ ಕೂಡಾ ಅರ್ಧಚಂದ್ರಾಕೃತಿಯಲ್ಲಿ ನಿರ್ಮಾಣವಾಗಿದೆ. ಇಲ್ಲಿಂದ ಹೊರಟ ವಿರಿಜಾ ನಾಲೆಯು\index{ವಿರಿಜಾ ನಾಲೆ} ಶ‍್ರೀರಂಗಪಟ್ಟಣ ತಾಲ್ಲೂಕಿನಲ್ಲಿ 53 ಕಿ.ಮೀ. ದೂರ ಹರಿದು ಶ‍್ರೀರಂಗಪಟ್ಟಣದವರೆಗೂ ಬರುತ್ತದೆ.

ಕಾವೇರಿ ನದಿಗೆ ರಂಗನತಿಟ್ಟು\index{ರಂಗನತಿಟ್ಟು} ಬಳಿ ಒಂದು ಒಡ್ಡನ್ನು ನಿರ್ಮಿಸಿದ್ದು, ಇಲ್ಲಿಂದ ಬಂಗಾರದೊಡ್ಡಿ ನಾಲೆಯು\index{ಬಂಗಾರದೊಡ್ಡಿ ನಾಲೆ} ಹೊರಡುತ್ತದೆ. ಇದು ಪಶ್ಚಿಮವಾಹಿನಿ, ಶ‍್ರೀರಂಗಪಟ್ಟಣ ಮೂಲಕ, ಕಾವೇರಿ ನದಿಯ ಮೇಲೆ ಹಾಯ್ದು, ಗಂಜಾಮ್ ಕಡೆಗೆ ಹೋಗಿ ಅಲ್ಲಿಂದ ಮುಂದೆ ಕಾವೇರಿ ನದಿಯನ್ನು ಸೇರುತ್ತದೆ. ಇದರಿಂದ ಸಾವಿರಾರು ಎಕರೆ ನೀರಾವರಿಯಾಗುತ್ತದೆ.

ಕಾವೇರಿ ನದಿಗೆ ಮಂಡ್ಯ ಕೊಪ್ಪಲು ಬಳಿ ನಿರ್ಮಿಸಿರುವ ರಾಮಸ್ವಾಮಿ ಅಣೆಕಟ್ಟಿನಿಂದ\index{ರಾಮಸ್ವಾಮಿ ಅಣೆಕಟ್ಟಿ} ಇಂದಿಗೂ ನೀರಾವರಿ ಆಗು\-ತ್ತಿದೆ. ನೆಲಮಟ್ಟದಲ್ಲಿ ಅರ್ಧಚಂದ್ರಾಕೃತಿಯಲ್ಲಿ ನಿರ್ಮಾಣವಾಗಿರುವ ಈ ಅಣೆಕಟ್ಟು ಪ್ರಾಚೀನ ನೀರಾವರಿ ತಂತ್ರಜ್ಞಾನಕ್ಕೆ ಒಂದು ಉದಾಹರಣೆ. ಈ ಅಣೆಕಟ್ಟೆಯಿಂದ ಮುಂದೆಯೇ ಕಾವೇರಿ ನದಿಯಲ್ಲಿ ಗೆಂಡೆಹೊಸಹಳ್ಳಿ\index{ಗೆಂಡೆಹೊಸಹಳ್ಳಿ} ಪಕ್ಷಿಧಾಮವಿದೆ, ಮಂಡ್ಯ ಕೊಪ್ಪಲಿನ ಬಳಿ ನದಿಗೆ ವಿಶಾಲವಾದ ಸೋಪಾನವಿದೆ. ನದಿಯ ಬಂಡೆಗಳ ಮೇಲೆ ಬಸವ, ಶಿವಲಿಂಗಗಳನ್ನು ಬಿಡಿಸಲಾಗಿದೆ. ಕೆಲವು ಸಣ್ಣ ಶಾಸನಗಳು ಇರುವಂತೆ ಕಂಡುಬರುತ್ತದೆ. ನದಿಯ ದಂಡೆಯಲ್ಲಿಯೆ ವಿಶಾಲವಾದ ಮರದ ತೋಪು, ತೋಪಿನಲ್ಲಿ ಹೊಯ್ಸಳರ ಕಾಲದ ನರಸಿಂಹ ದೇವಾಲಯ, ವಿಜಯನಗರ ಕಾಲದ ಭೈರವ ದೇವಾಲಯಗಳಿವೆ. ಇಲ್ಲಿಂದ ಪಶ್ಚಿಮಕ್ಕೆ, ಮಹದೇವಪುರ, ಮೈಸೂರಿನ ಕಡೆಗೆ ಒಂದು ನಾಲೆಯು ಹೊರಡುತ್ತದೆ. ಪೂರ್ವಕ್ಕೆ ಹೊರಡುವ ನಾಲೆಯು ಕರಿಘಟ್ಟವನ್ನು ಬಳಸಿ ಹೋಗಿ ಸಾವಿರಾರು ಎಕೆರೆ ಭೂಮಿಗೆ ನೀರನ್ನು ಒದಗಿಸುತ್ತದೆ.

ವೀರವೈಷ್ಣವಿ ನದಿಯು\index{ವೀರವೈಷ್ಣವಿ ನದಿ} ಮಂಡ್ಯ ಜಿಲ್ಲೆಯ ಮತ್ತೊಂದು ಸಣ್ಣ ನದಿ. ನಾಗಮಂಗಲ ತಾಲ್ಲೂಕಿನ ಹಸುವಿನ ಕಾವಲು ಕಾಡಿನ, ಹಂದಿಗುಡ್ಡದ ಬಳಿ ಮತ್ತು ಬೋರನ ಬೆಟ್ಟದ ಬಳಿ ಹುಟ್ಟುವ ವೀರವೈಷ್ಣವಿಯು ಉತ್ತರಾಭಿಮುಖವಾಗಿ ಹರಿಯುತ್ತದೆ. ಈ ನದಿಗೆ ನಾಗಮಂಗಲ ತಾಲ್ಲೂಕು, ಹೊನ್ನಾವರ ಮತ್ತು ಮಾಚನಾಯಕನಹಳ್ಳಿಯ ನಡುವೆ ಒಂದು ದೊಡ್ಡ ಕೆರೆಯನ್ನು\break ನಿರ್ಮಿಸಿದ್ದು, ಇದನ್ನು ಹಿರಿಕೆರೆ ಎನ್ನುತ್ತಾರೆ. ಈ ಕೆರೆಯು ಹೊಯ್ಸಳರ ಕಾಲದ ರಚನೆಯಾಗಿದೆ. ಇದರಿಂದ ಕೋಡಿಹಳ್ಳಿ, ಹೊನ್ನಾವಾರ, ಸುಬ್ಬರಾಯನಕೊಪ್ಪಲಿನವರೆಗೆ ಇರುವ ಜಮೀನಿಗೆ ನೀರಾವರಿಯಾಗುತ್ತಿದೆ. ಮುಂದೆ ಇದೇ ನದಿಗೆ\break ಬಿಂಡಿಗನವಿಲೆಯ ಬಳಿ ಕೆರೆಯನ್ನು ನಿರ್ಮಿಸಲಾಗಿದೆ. ಇದೂ ಕೂಡಾ ಹೊಯ್ಸಳರ ಕಾಲದಲ್ಲಿ ನಿರ್ಮಿತವಾದ ಕೆರೆ\-ಯಾಗಿದೆ. ಈ ಕೆರೆಯು ಭಾರೀ ಕೆರೆಯಾಗಿದ್ದು, ದಕ್ಷಿಣದ ಕೋಡಿಯು ಕಂಬದಹಳ್ಳಿ ಕಡೆಯೂ, ಉತ್ತರದ ಇನ್ನೊಂದು ಕೋಡಿಯ ಬಿಂಡಿಗನವಿಲೆಯ ಬಳಿಯೂ ಇದೆ. ಆದರೆ ಈ ಕೆರೆ ಇಂದು ಹೂಳು ತುಂಬಿದ್ದು ನೀರು ನಿಲ್ಲುವ ಪ್ರಮಾಣ ಕಡಿಮೆಯಾಗಿದೆ. ಕೆರೆ ಅತಿಕ್ರಮಣಕ್ಕೆ ಒಳಗಾಗಿದೆ. ಮುಂದೆ ಇದೇ ತಾಲ್ಲೂಕಿ ದಡಗದ ಬಳಿ, ವೀರವೈಷ್ಣವಿ ನದಿಗೆ ಭರತ ಬಾಹುಬಲಿ ದಂಡನಾಯಕರ ಕಾಲದಲ್ಲಿ ಮರಿಯಾನೆ ಸಮುದ್ರವೆಂಬ\index{ಮರಿಯಾನೆ ಸಮುದ್ರ} ಭಾರೀ ಕೆರೆಯನ್ನು ನಿರ್ಮಿಸಲಾಗಿದೆ.\endnote{ ಎಕ 7 ನಾಮಂ 68 ದಡಗ 12ನೇ ಶ.} ಈ ಶಾಸನದಲ್ಲಿ ಮರಿಯಾನೆ ಸಮುದ್ರದ ಬಯಲು ಎಂದೂ ಹೇಳಿದೆ. ಇಲ್ಲಿಂದ ಮುಂದಕ್ಕೆ ಹರಿಯುವ ಈ ನದಿಯು ಆರಣಿ ಮತ್ತು ಶ‍್ರೀರಂಗಪುರದ\index{ಶ‍್ರೀರಂಗಪುರ} ಬಳಿ ಹರಿದು, ಬೆಳ್ಳೂರಿನ ಪಕ್ಕದಲ್ಲೇ ಹರಿಯುತ್ತದೆ. ಈ ನದಿಗೆ ಬೆಳ್ಳೂರಿನ\index{ಬೆಳ್ಳೂರು} ಬಳಿ ಪೆರಮಾಳೆ ದೇವ ದಂಡನಾಯಕನು ಅಲ್ಲಾಳ ಸಮುದ್ರ\index{ಅಲ್ಲಾಳ ಸಮುದ್ರ} ಎಂಬ ಭಾರೀ ಕೆರೆಯನ್ನ ನಿರ್ಮಿಸಿದ ವಿಚಾರವನ್ನು ಈ ಹಿಂದೆ ಪ್ರಸ್ತಾಪಿಸಲಾಗಿದೆ. ಇದನ್ನು ದಾಸನ ಕೆರೆ ಎನ್ನುತ್ತಾರೆ. ಬೆಳ್ಳೂರಿನ ಬಳಿ ಈ ನದಿಗೆ ಕುಂಬಾರರ ಹಳ್ಳ ಎಂದೂ ಕರೆಯುತ್ತಾರೆ. ಮೊದಲು ಇದನ್ನು ಯಾವ ಹೆಸರಿನಿಂದ ಕರೆಯುತ್ತಿದ್ದರೋ ತಿಳಿದುಬರುವುದಿಲ್ಲ. ಪೆರಮಾಳೆ ದೇವ ದಂಡನಾಯಕನು ವೀರ ವೈಷ್ಣವನಾದುದರಿಂದ, ಇದಕ್ಕೆ ಅವನೇ ವೀರವೈಷ್ಣವಿ ನದಿ ಎಂಬ ಹೆಸರನ್ನು ನೀಡಿರುಬಹದು. ವೀರವೈಷ್ಣವಿಯು ಇಲ್ಲಿಂದ ಮುಂದೆ ಹರಿದು ಶಿಂಶಾ ನದಿಯನ್ನು ಸೇರುತ್ತದೆ.

ಲೋಕಪಾವನಿ\index{ಲೋಕಪಾವನಿ} ನದಿಯು ಮಂಡ್ಯ ಜಿಲ್ಲೆಯ ಇನ್ನೊಂದು ಪ್ರಾದೇಶಿಕ ನದಿ. ಇದು ನಾಗಮಂಗಲ ತಾಲ್ಲೂಕು, ಪಡುವಲ ಪಟ್ಟಣದ ಬಳಿ ಇರುವ, ಬಸವನಬೆಟ್ಟದಲ್ಲಿ ಹುಟ್ಟಿ ಪಶ್ಚಿಮಾಭಿಮುಖವಾಗಿ ಹರಿಯುತ್ತದೆ. ಬಸವನ ಬೆಟ್ಟದ\index{ಬಸವನ ಬೆಟ್ಟ} ಗುಹೆ\-ಯಲ್ಲಿಯೇ ರಾಮಾನುಜಾಚಾರ್ಯರು\index{ರಾಮಾನುಜಾಚಾರ್ಯ} ತಪಸ್ಸು ಮಾಡಿದರೆಂದು ಪ್ರತೀತಿ ಇದೆ.\endnote{ ಎಕ 7 ನಾಮಂ 18 ಮತ್ತು 19 ಪಡುವಲಪಟ್ಟಣ 19ನೇ ಶ.} ಈ ನದಿ ಅನೇಕ ಕೆರೆಗಳಿಗೆ ನೀರನ್ನು ಒದಗಿಸುತ್ತದೆ. ಇದಕ್ಕೆ ನಾಗಮಂಗಲ ತಾಲ್ಲೂಕು ಕರೀಕ್ಯಾತನಹಳ್ಳಿಯ ಸಮೀಪದ ಉಯ್ಯನಹಳ್ಳಿಯಲ್ಲಿ\index{ಉಯ್ಯನಹಳ್ಳಿ} ಒಂದು ಅಣೆಕಟ್ಟೆಯನ್ನು, ಸ್ವಾತಂತ್ರ್ಯಾ ನಂತರದ ಕಾಲದಲ್ಲಿ, ಕೆಂಗಲ್​ ಹನುಮಂತಯ್ಯನವರ ಹಾಗೂ ಎಸ್​. ನಿಜಲಿಂಗಪ್ಪನವರ ಕಾಲದಲ್ಲಿ ನಿರ್ಮಿಸಲಾಗಿದೆ. ಅಂದಿನ ಸರ್ಕಾರದಲ್ಲಿ ಮಂತ್ರಿಗಳಾಗಿದ್ದ ನಾಗಮಂಗಲ ತಾಲ್ಲೂಕಿನವರಾದ ಟಿ. ಮರಿಯಪ್ಪನವರು ಇದಕ್ಕೆ ಕಾರಣ ಕರ್ತರು. ನಾಗಮಂಗಲದಲ್ಲಿ ಬಹುಶಃ ಜಗದೇವರಾಯನ ಕಾಲದಲ್ಲಿ ನಿರ್ಮಿತವಾಗಿರುವ ಸೂಳೆಕೆರೆಗೆ ನೀರು ತುಂಬಿಸುವ ಉದ್ದೇಶದಿಂದ ಇದನ್ನು ನಿರ್ಮಿಸಿದ್ದು, ಉಯ್ಯನಹಳ್ಳಿ ಅಣೆಯಿಂದ ಸೂಳೆಕೆರೆಗೆ ನೀರು ಹರಿಯಲು ಪಿಕಪ್ ನಾಲೆಯನ್ನು ನಿರ್ಮಿಸಲಾಗಿದೆ. ಆದರೆ ಈಗ ಸೂಳೆ ಕೆರೆಗೆ\index{ಸೂಳೆ ಕೆರೆ} ಹೇಮಾವತಿ ನೀರೇ ಹರಿಯುತ್ತಿದೆ. ಉಯ್ಯನಹಳ್ಳಿ ಅಣೆಯು ಹೂಳು ತುಂಬಿದ್ದು ಪೂರ್ಣ ಸಾಮರ್ಥ್ಯ ಬಳಕೆಯಾಗುತ್ತಿಲ್ಲ. ಜೋರಾಗಿ ಮಳೆ ಬಂದು ಲೋಕಪಾವನಿ ತುಂಬಿ ಹರಿದರೆ, ಸೂಳೆಕೆರೆಗೆ ಹೋಗುವ ಪಿಕಪ್​ ನಾಲೆಯಲ್ಲಿ ಇಂದಿಗೂ ನೀರು ಹರಿಯುತ್ತದೆ. ಆದರೆ ಈಗ ಲೋಕಪಾವನಿಯು ಬತ್ತಿದ್ದು ಅದರಲ್ಲಿಯೂ ನೀರು ಹರಿಯುತ್ತಿಲ್ಲ. ಮುಂದೆ ಲೋಕಪಾವನಿ ನದಿಗೆ ಬೋಳೆನಹಳ್ಳಿ ಬಳಿ ಒಂದು ದೊಡ್ಡ ಕೆರೆಯನ್ನು ನಿರ್ಮಿಸಲಾಗಿದೆ. ಲೋಕಪಾವನಿ ನದಿಗೆ ಶ‍್ರೀರಂಗಪಟ್ಟಣ ತಾಲ್ಲೂಕು, ನೆಲಮನೆಯ ಬಳಿ ಒಂದು ದೊಡ್ಡ ಅಣೆಕಟ್ಟನ್ನು ವಿಜಯನಗರ ಕಾಲದಲ್ಲಿ ತಿಮ್ಮಣ್ಣ ದಂಡನಾಯಕನು ನಿರ್ಮಿಸಿದ್ದು, ಲೋಕಪಾವನೆ ಸಾಗರದ\index{ಲೋಕಪಾವನೆ ಸಾಗರ} ಉಲ್ಲೇಖ ನೆಲಮನೆ ಶಾಸನದಲ್ಲಿದೆ.\endnote{ ಎಕ 6 ಶ‍್ರೀಪ 93 ನೆಲಮನೆ 1458} ಈ ನದಿಯು ಕರಿಘಟ್ಟ ಬೆಟ್ಟದ ಕೆಳಗೆ ಕಾವೇರಿ ನದಿಯನ್ನು ಸೇರುತ್ತದೆ. ಇಲ್ಲಿಗೆ ಸಮೀಪದಲ್ಲಿಯೇ ಕಾವೇರಿ ನದಿ ದಡದಲ್ಲಿ ಪ್ರಸಿದ್ಧವಾದ ಕನಕನ ಬಂಡೆ ಇದೆ. ಕನಕದಾಸರು ಶ‍್ರೀರಂಗಪಟ್ಟಣಕ್ಕೆ ಬಂದಿದ್ದಾಗ, ಇಲ್ಲಿ ಕುಳಿತು ಧ್ಯಾನಾಸಕ್ತರಾಗಿದ್ದರೆಂದು ಸ್ಥಳೀಯ ಐತಿಹ್ಯವಿದೆ. ಇಲ್ಲಿಂದ ಅವರು ಬೇಲೂರಿಗೆ ಹೋದರೆಂದು ತಿಳಿದುಬರುತ್ತದೆ.

ಹೇಮಾವತಿ ನದಿ, ಮಂಡ್ಯ ಜಿಲ್ಲೆಯ ಇನ್ನೊಂದು ಜೀವನ ನದಿ. ಇದು ಹೊಳೆನರಸಿಪುರ ತಾಲ್ಲೂಕಿನ ಗೂಡೆ\break ಹೊಸಹಳ್ಳಿಯ ಬಳಿ, ಕೃಷ್ಣರಾಜಪೇಟೆ ತಾಲ್ಲೂಕನ್ನು ಪ್ರವೇಶಿಸಿ, ತಾಲ್ಲೂಕಿನ ಗಡಿಯ ಉದ್ದಕ್ಕೂ ಹರಿದು, ಮಾದಾಪುರ, ಮಂದಗೆರೆ, ಅಕ್ಕಿಹೆಬ್ಬಾಳು ಮುಖಾಂತರ ಹರಿದು, ಕನ್ನಂಬಾಡಿ ಕಟ್ಟೆಯ ಹಿನ್ನೀರಿನಲ್ಲಿರುವ, ಕೃಷ್ಣರಾಜಪೇಟೆ ತಾಲ್ಲೂಕು ಸಂಗಮದ ಬಳಿ ಕಾವೇರಿ ನದಿಯನ್ನು ಸೇರುತ್ತದೆ. ಹೆಚ್ಚು ಕಡಿಮೆ ನೇರವಾಗಿ ಕನ್ನಂಬಾಡಿ ಕಟ್ಟೆಯ ಒಳಕ್ಕೇ ಸೇರುತ್ತಿದೆ. ಕನ್ನಂಬಾಡಿಯ ಶಾಸನದಲ್ಲಿ, ಕಾವೇರಿ, ಹೇಮಾವತಿ ಕಟ್ಟು ಕಾಲುವೆಗಳ\index{ಹೇಮಾವತಿ ಕಟ್ಟು ಕಾಲುವೆ} ವಿಚಾರ ಉಲ್ಲೇಖವಾಗಿದ್ದು, ಅದನ್ನು ಈಗಾಗಲೇ ವಿವರಿಸಲಾಗಿದೆ. ಈ ನದಿಗೆ ಹೊಳೆನರಸಿಪುರ ತಾಲ್ಲೂಕಿನಲ್ಲಿ, ಹೊಳೆನರಸಿಪುರಕ್ಕೆ 7 ಕಿ.ಮೀ. ದೂರದಲ್ಲಿ ಶ‍್ರೀರಾಮದೇವರ ಕಟ್ಟೆ ಎಂಬ ಹೆಸರಿನ ಅಣೆಕಟ್ಟೆಯನ್ನು ನಿರ್ಮಿಸಲಾಗಿದೆ. ಇದನ್ನು ಕ್ರಿ.ಶ.1533ರಲ್ಲಿ ವಿಜಯನಗರ ಕಾಲದಲ್ಲಿ ನಿರ್ಮಿಸಲಾಗಿತ್ತೆಂದು, ಅದು ಪ್ರವಾಹದ ಕಾರಣ ಒಡೆದು ಜೀರ್ಣವಾಗಿರಲು, ಮೈಸೂರಿನ ಒಡೆಯರ ಕಾಲದಲ್ಲಿ ಕ್ರಿ.ಶ.1863ರಲ್ಲಿ ಈ ಕಟ್ಟೆಯಿಂದ ಮುಂದೆ ಹೊಸದಾಗಿ ಅಣೆಕಟ್ಟೆಯನ್ನು ನಿರ್ಮಿಸಲಾಯಿತೆಂದೂ ತಿಳಿದುಬರುತ್ತದೆ. ಇದನ್ನು ರಾಮದೇವರ ಅಣೆಕಟ್ಟು ಎನ್ನುತ್ತಾರೆ. ಇದರಿಂದ ಹೊರಡುವ ಉತ್ತರ ಭಾಗದ ನಾಲೆಯು 77 ಕಿ.ಮೀ. ಸಾಗುತ್ತದೆ. ಇದರಿಂದ, ಹಾಸನ ಜಿಲ್ಲೆಯ ಹೊಳೆನರಸಿಪುರ, ಅರಕಲಗೂಡು ಮತ್ತು ಚನ್ನರಾಯಪಟ್ಟಣ ತಾಲ್ಲೂಕಿನ ಪ್ರದೇಶವು ನೀರಾವರಿಗೆ ಒಳಪಟ್ಟಿದೆ. ದಕ್ಷಿಣದ ನಾಲೆಯು 20 ಕಿ.ಮೀ. ಉದ್ದ ಇದ್ದು, ಇದರಿಂದ ಕೃಷ್ಣರಾಜಪೇಟೆ ತಾಲ್ಲೂಕಿನ ಮಾದಾಪುರ, ಮಂದಗರೆಯವರೆಗಿನ ಪ್ರದೇಶವು ಇದರಿಂದ ನೀರಾವರಿಗೆ ಒಳಪಡುತ್ತದೆ.

ಮಾದಾಪುರದ\index{ಮಾದಾಪುರ} ಬಳಿ ಹೇಮಾವತಿ ನದಿಗೆ ಪ್ರಾಚೀನ ಕಾಲದಲ್ಲಿ ಒಂದು ಅಣೆಕಟ್ಟೆಯನ್ನು ನಿರ್ಮಿಸಲಾಗಿತ್ತು. ಇದನ್ನು ಗಾಣದ ಕಟ್ಟೆ\index{ಗಾಣದ ಕಟ್ಟೆ} ಎಂದು ಕರೆಯುತ್ತಿದ್ದರು. ಆದರೆ ಇದು ಪ್ರವಾಹದ ಕಾರಣ ಕೊಚ್ಚಿ ಹೋಯಿತು. ಇಲ್ಲಿಂದ ಹೊರಡುತ್ತಿದ್ದ\break ನಾಲೆಗೆ ಹಳೇ ನಾಲೆ ಎನ್ನುತ್ತಾರೆ. ಈ ನಾಲೆಯು ಈಗಲೂ ಇದೆ. ಈಗ ಶ‍್ರೀರಾಮದೇವರ ಅಣೆಕಟ್ಟೆಯಿಂದ\index{ಶ‍್ರೀರಾಮದೇವರ ಅಣೆಕಟ್ಟೆ} ಇಲ್ಲಿಯ\-ವರೆಗೆ ನಾಲೆ ಬರುತ್ತದೆ. ಮಾದಾಪುರದ ಆಂಜನೇಯ ದೇವಾಲಯದ ಮುಂದಿರುವ ರಣಧೀರ ಕಂಠೀರವ ನರಸಿಂಹರಾಜ ಒಡೆಯರ ಶಾಸನದಲ್ಲಿ ಈ ಅಣೆಕಟ್ಟಿನ ಉಲ್ಲೇಖವಿದೆ.\endnote{ ಎಕ 6 ಕೃಪೆ 45 ಮಾದಾಪುರ 17ನೇ ಶ.} ಮಾದಾಪುರವು ಹೊಯ್ಸಳರ ಕಾಲದಲ್ಲಿ ಒಂದು ಪಟ್ಟಣವಾಗಿತ್ತು. ಇಲ್ಲಿ \hbox{ತ್ರ್ಯಂಬಕೇಶ್ವರ}, ಮಹಲಿಂಗೇಶ್ವರ, ಭೈರವ, ಚೆನ್ನಿಗರಾಯ, ಆಂಜನೇಯ ದೇವಾಲಯಗಳು ಹೊಯ್ಸಳರ ಹಾಗೂ ವಿಜಯನಗರ ಕಾಲದ ರಚನೆಗಳು.

ಇಲ್ಲಿಂದ ಮುಂದೆ ಮಂದಗೆರೆಯ\index{ಮಂದಗೆರೆ} ಬಳಿ ಹೇಮಾವತಿ ನದಿಗೆ ಒಂದು ಅಣೆಕಟ್ಟೆಯನ್ನು ನಿರ್ಮಿಸಲಾಗಿದೆ. ಇದು ಚಿಕದೇವರಾಜ ಒಡೆಯರ ಕಾಲದಲ್ಲಿ ನಿರ್ಮಿತವಾದುದೆಂದು ಹೇಳುತ್ತಾರೆ. ಇದರ ಬಲ ದಂಡೆ ನಾಲೆ 27 ಮೈಲಿ ಉದ್ದವಿದೆ.\break ಮೊದಲಿಗೆ 15 ಮೈಲಿ ಉದ್ದ ಇದ್ದ ಈ ನಾಲೆಯನ್ನು 1873 ರಲ್ಲಿ 6 ಮೈಲಿ, 1879ರಲ್ಲಿ 4 ಮೈಲಿ ವಿಸ್ತರಿಸಲಾಯಿತೆಂದು ತಿಳಿದುಬರುತ್ತದೆ. ಈ ಅಣೆಕಟ್ಟು ನೇರವಾಗಿದೆ. ಈ ಅಣೆಕಟ್ಟೆಯಿಂದ ಕೃಷ್ಣರಾಜಪೇಟೆ ತಾಲ್ಲೂಕು ಮತ್ತು ಕೃಷ್ಣರಾಜನಗರ ತಾಲ್ಲೂಕಿನ ಕೆಲವು ಭಾಗ ನೀರಾವರಿಗೆ ಒಳಪಟ್ಟಿದೆ.

ಮಂದಗೆರೆಯಿಂದ ಮುಂದೆ ಹರಿಯುವ ನದಿಗೆ, ಬಂಡಿಹೊಳೆ\index{ಬಂಡಿಹೊಳೆ} ಬಳಿ ಮೈಸೂರು ಒಡೆಯರ ಕಾಲದಲ್ಲಿ ಒಂದು ಅಣೆಕಟ್ಟೆಯನ್ನು ನಿರ್ಮಿಸಲಾಗಿದೆ. ಕ್ರಿ.ಶ.1322ರಲ್ಲಿ ಹರಿಹರಪುರದ ಮಹಾಜನಗಳಿಂದ ನಿರ್ಮಿತವಾಗಿದ್ದ ಈ ಅಣೆಕಟ್ಟೆಯು ಜೀರ್ಣವಾಗಿರಲು, ಮೈಸೂರಿನ ಒಡೆಯರು ಅದನ್ನು ಪುನರ್​ ನಿರ್ಮಾಣ ಮಾಡಿದರು. ಅಣೆಕಟ್ಟೆಯಲ್ಲಿ ನೀರು ಕಡಿಮೆ ಆದಾಗ ಮಹಾಜನರು ನಿರ್ಮಿಸಿದ್ದ ಅಣೆಕಟ್ಟೆಯ ಅವಶೇಷಗಳು (ದೊಡ್ಡ ದೊಡ್ಡ ಕಲ್ಲುಗಳು) ಕಂಡು ಬರುತ್ತವೆ. ಇಲ್ಲಿಂದ ಹೊರಡುವ ನಾಲೆ 17ಮೈಲಿ ಉದ್ದ ಇದ್ದು, ಕೃಷ್ಣರಾಜಪೇಟೆ ತಾಲ್ಲೂಕಿನ ಲಕ್ಷ್ಮೀಪುರ, ಹರಿಹರಪುರ, ಅಕ್ಕಿಹೆಬ್ಬಾಳಿನವರೆಗೆ ಹೋಗುತ್ತದೆ. ಅಕ್ಕಿಹೆಬ್ಬಾಳಿನ ಪ್ರಾಚೀನ ಹೆಸರು ಹೆಬ್ಬೊಳಲು\index{ಹೆಬ್ಬೊಳಲು} ಎಂದು ಇದ್ದಿತು. ಅದಕ್ಕೆ ಅಕ್ಕೀಹೆಬ್ಬಾಳು\index{ಅಕ್ಕೀಹೆಬ್ಬಾಳು} ಎಂದು ಹೆಸರು ಬರಲು, ಈ ನಾಲೆಯ ಅಚ್ಚುಕಟ್ಟಿನಲ್ಲಿ ಭಾರೀ ಪ್ರಮಾಣದಲ್ಲಿ ಭತ್ತವನ್ನು ಬೆಳೆದು, ಅಕ್ಕಿಯನ್ನು ತಯಾರು ಮಾಡುತ್ತಿದ್ದುದೇ ಕಾರಣ.

ಅಕ್ಕಿಹೆಬ್ಬಾಳಿನಿಂದ ಮುಂದೆ ಈ ನದಿಯಲ್ಲಿ ಒಂದು ಸಣ್ಣ ದ್ವೀಪ ಇದೆ. ಇದನ್ನು ಹೊಸಪಟ್ಟಣ\index{ಹೊಸಪಟ್ಟಣ} ಎನ್ನುತ್ತಾರೆ. ಇಲ್ಲಿ ಒಂದು ಸಣ್ಣ ಅಣೆಕಟ್ಟೆಯನ್ನು ಈ ನದಿಗೆ ನಿರ್ಮಿಸಲಾಗಿದೆ. ಇದರಿಂದ 9 ಮೈಲಿ ಉದ್ದದ ನಾಲೆ ಹೊರಡುತ್ತದೆ. ಇದನ್ನು ಅಕ್ಕೀಹೆಬ್ಬಾಳು ನಾಲೆ ಎಂದು ಕರೆಯುತ್ತಾರೆ. ಮುಮ್ಮಡಿ ಬಲ್ಲಾಳನು ನಿರ್ಮಿಸಿದ್ದ ಹೊಸ ಪಟ್ಟಣ ಇದೇ ಆಗಿದ್ದೆ. ಹೊಸಪಟ್ಟಣಕ್ಕೆ ಅಗತ್ಯವಾದ ನೀರಿನ ಸರಬರಾಜಿಗೆ ಈ ಅಣೆಕಟ್ಟೆಯನ್ನು ನಿರ್ಮಿಸಿರಬಹುದು.

ಶಿಂಷಾ ನದಿಯು, ತುಮಕೂರು ಜಿಲ್ಲೆಯ ದೇವರಾಯನ ದುರ್ಗದಲ್ಲಿ ಹುಟ್ಟಿ, ಮಂಡ್ಯ ಜಿಲ್ಲೆಯಲ್ಲಿ ಹರಿಯುವ ಕಾವೇರಿಯ ಉಪನದಿ. ಇದನ್ನು ಅಚ್ಯುತರಾಯನ ಹೊನ್ನೇನಹಳ್ಳಿ ತಾಮ್ರಶಾಸನದಲ್ಲಿ ಕ್ಷುದ್ರಶೈವಾಲಿನಿ ಎಂದು ಕರೆಯಲಾಗಿದೆ. ಹೊನ್ನೇನಹಳ್ಳಿಯು\index{ಹೊನ್ನೇನಹಳ್ಳಿ} ಕ್ಷುದ್ರ ಶೈವಾಲಿನಿ\index{ಕ್ಷುದ್ರ ಶೈವಾಲಿನಿ} ನದಿಯ ದಕ್ಷಿಣ ತೀರದಲ್ಲಿತ್ತೆಂದು ಹೇಳಿದೆ.\endnote{ ಎಕ 7 ನಾಮಂ 107 ಹೊನ್ನೇನಹಳ್ಳಿ 1545} ನಾಗಮಂಗಲ ತಾಲ್ಲೂಕಿನಲ್ಲಿ ಹದ್ದಿನ\-ಕಲ್ಲು ಹನುಂತರಾಯನ ಬೆಟ್ಟ, ಹಟ್ಟಣ, ಕಾಳಿಂಗನಹಳ್ಳಿ ಹತ್ತಿರ ಹರಿದು, \hbox{ಯೆಡೆಯೂರಿನ} ಕಡೆಗೆ ತಿರುಗುತ್ತದೆ. \hbox{ಯೆಡೆಯೂರಿನ} ಸಮೀಪ ಮಾರ್ಕೋನಹಳ್ಳಿ\index{ಮಾರ್ಕೋನಹಳ್ಳಿ} ಬಳಿ ಇದಕ್ಕೆ ಒಂದು ಜಲಾಶಯವನ್ನು ನಿರ್ಮಿಸಲಾಗಿದೆ. ಇದು ಮೈಸೂರು ಒಡೆಯರ ಕಾಲದಲ್ಲಿ, ವಿಶ್ವೇಶ್ವರಯ್ಯನವರು ನಿರ್ಮಿಸಿದರೆಂದು ತಿಳಿದುಬರುತ್ತದೆ. ಮಾರ್ಕೋನಹಳ್ಳಿ ಜಲಾಶಯದ ಹೆಚ್ಚಿನ ನೀರು, ಮಂಗಳಾ ಜಲಾಶಯಕ್ಕೆ ಹೋಗುತ್ತದೆ. ಅಲ್ಲಿಂದ ಮುಂದಕ್ಕೆ ಹುಲಿಯೂರು ದುರ್ಗದ ಬೆಟ್ಟಗಳ ಬಳಿ ಹರಿದು, ಮತ್ತೆ ನಾಗಮಂಗಲ ತಾಲ್ಲೂಕನ್ನು ಪ್ರವೇಶಿಸುತ್ತದೆ. ನಾಗಮಂಗಲ ತಾಲ್ಲೂಕಿನ ಕೀಳಘಟ್ಟ, ದೊಡ್ಡಂಕನಹಳ್ಳಿ ಹತ್ತಿರ ಇದಕ್ಕೆ ಒಂದು ಅಣೆಯನ್ನು ನಿರ್ಮಿಸಲಾಗಿದೆ. ಅಲ್ಲಿಂದ ಮುಂದೆ ಹರಿದು ಮದ್ದೂರು ತಾಲ್ಲೂಕಿನ ವಾಯುವ್ಯದಲ್ಲಿ ಕಿರಂಗೂರು ಬಳಿ ಮಂಡ್ಯ ಜಿಲ್ಲೆಯನ್ನು ಪ್ರವೇಶಿಸುತ್ತದೆ. ಮದ್ದೂರು ಪಟ್ಟಣದ ಪಕ್ಕದಲ್ಲಿ ಇದು ಹರಿಯುವುದರಿಂದ ಇದನ್ನು ಮದ್ದೂರು ಹೊಳೆ\index{ಮದ್ದೂರು ಹೊಳೆ} ಎನ್ನುತ್ತಾರೆ. ಮದ್ದೂರಿಗೆ ಸಮೀಪದ, ಪ್ರಖ್ಯಾತ ವೈದ್ಯನಾಥಪುರದ ವೈದ್ಯನಾಥೇಶ್ವರ ದೇವಾಲಯದ ಪಕ್ಕದಲ್ಲೇ ಹರಿಯುತ್ತದೆ. ಈ ನದಿಗೆ\break ಇಗ್ಗಲೂರಿನ\index{ಇಗ್ಗಲೂರು} ಬಳಿ ಒಂದು ಅಣೆಕಟ್ಟೆಯನ್ನು ನಿರ್ಮಿಸಲಾಗಿದೆ. ಅಲ್ಲಿಂದ ಮುಂದೆ ತೊರೆಕಾಡನಹಳ್ಳಿ, ಮುಖಾಂತರ ಹರಿದು ಶಿಂಷಾ ಬಳಿ, ಗಗನಚುಕ್ಕಿ ಜಲಪಾತದ ಮುಂದೆ ಕಾವೇರಿ ನದಿಯನ್ನು ಸೇರುತ್ತದೆ.

\section*{ದೇವಾಲಯದ ಸರೋವರ ಕೊಳಗಳು}

ದೇವಾಲಯದ ಬಳಿ ದೇವರ ಪೂಜೆಗೆ ಅಗತ್ಯವಾದ ನೀರಿಗೋಸ್ಕರ ಕೊಳಗಳನ್ನು ನಿರ್ಮಿಸುತ್ತಿದ್ದು ಸಾಮಾನ್ಯ. ಇಂತಹ ಕೆಲವು ಕೊಳಗಳು ಶಾಸನೋಕ್ತವಾಗಿದ್ದು, ಇವುಗಳನ್ನು ಪುಷ್ಕರಣಿ, ಕಲ್ಯಾಣಿ, ಸರಸ್ಸು, ಸರೋವರಗಳೆಂದು ಕರೆಯಲಾಗಿದೆ. ಕಸಲಗೆರೆ ಶಾಸನದಲ್ಲಿ ಎಗಣನ ಕೊಳ, ಆಲದ ಕೊಳಗಳ ಉಲ್ಲೇಖವಿದೆ.\endnote{ ಎಕ 7 ನಾಮಂ 168 ಕಸಲಗೆರೆ 1190} ಮೇಲುಕೋಟೆಯಲ್ಲಿ ಬಹುಸಂಖ್ಯೆಯ ಕೊಳಗಳಿದ್ದರೂ ಚಿಕ್ಕಯ್ಯನ ಕೊಳ\index{ಚಿಕ್ಕಯ್ಯನ ಕೊಳ} ಮತ್ತು ಕಲ್ಯಾಣಿ\index{ಕಲ್ಯಾಣಿ} ಇವೆರಡೂ ಶಾಸನೋಕ್ತವಾಗಿವೆ. ಚಿಕ್ಕದೇವರಾಜ ಒಡೆಯರು, ಸುಜ್ಜಲೂರು ಶಾಸನದಲ್ಲಿ ಇಟ್ಟೀಯ ಕೊಳವೆಂದು ಹೇಳಿದ್ದು ಇಟ್ಟಿಗೆಯಿಂದ ನಿರ್ಮಿತವಾದ ಸೋಪಾನವಿದ್ದ ಕೊಳ ಇದಾಗಿರಬಹುದು.\endnote{ ಎಕ 7 ಮವ 139 ಸುಜ್ಜಲೂರು 1473} ಚಿಕ್ಕದೇವರಾಜ ಒಡೆಯರು, 1685ರಲ್ಲಿ ಲಕ್ಷ್ಮೀನರಸಿಂಹ ಪರಿಪಾಲಿತ ದುರ್ಗವಾದ ಮಳವಳ್ಳಿ ಕೋಟೆಯ ಆಗ್ನೇಯ ದಿಕ್ಕಿಗೆ ಬಹಳ ರಮ್ಯವಾದ ಸರೋವರವನ್ನು\index{ಸರೋವರ} ನಿರ್ಮಿಸಿದರೆಂದು, ಅದರ ಸುತ್ತಮುತ್ತ ಫಲಪುಷ್ಪಗಳನ್ನು ಬಿಡುವ ಮರಗಿಡಗಳನ್ನು ಬೆಳೆಸಿದರೆಂದು, ಮಳವಳ್ಳಿಯ ಜನಗಳ ಉಪಯೋಗಕ್ಕಾಗಿ ಇದನ್ನು ನಿರ್ಮಿಸಲಾಯಿತೆಂದು ತಿಳಿದುಬರುತ್ತದೆ.\endnote{ ಎಕ 7 ಮವ 2 ಮಳವಳ್ಳಿ 1685} ಶ‍್ರೀರಂಗಪಟ್ಟಣದ ತಿರುಮಲೆ ಆನಂದಾಂಪಿಳ್ಳೆ ಗೋವಿಂದರಾಜರ ಕುಮಾರ ತಿರುಮಲಾಚಾರ್ಯರು ಗೋವಿಂದರಾಜ ಪುಷ್ಕರಣಿ\index{ಪುಷ್ಕರಣಿ} ಎಂಬ ಸರಸ್ಸನ್ನು, ಗೋವಿಂದರಾಜೋದ್ಯಾನವನವನ್ನು ನಿರ್ಮಿಸಿದರು.\endnote{ ಎಕ 7 ಮಂ 2 ರಿಂದ 5ಮಂಡ್ಯ 1810} ಇದು ಮಂಡ್ಯದ ಲಕ್ಷ್ಮೀಜನಾರ್ದನ ದೇವಾಲಯಕ್ಕೆ ಸಮೀಪದಲ್ಲಿದೆ. ಪ್ರಾಣದೇವರ ದೇವಸ್ಥಾನಕ್ಕೆ ಮತ್ತು ಜನಗಳಿಗೆ ಉಪಯೋಗವಾಗಲೆಂದು ಮಂಡ್ಯ ತಾಲ್ಲೂಕು ಅಮೀಲ ಶ‍್ರೀನಿವಾಸರಾವು ಈ ಸರೋವರ ಮತ್ತು ನಂದನವನಗಳನ್ನು ನಿರ್ಮಿಸಿದನು.\endnote{ ಎಕ 7 ಮಂ 1 ಮಂಡ್ಯ 1847}

\begin{center}
***
\end{center}

\theendnotes

