
\chapter{ಮಂಡ್ಯ ಜಿಲ್ಲೆಯ ಹಿನ್ನೆಲೆ}

 ಮಂಡ್ಯ ಜಿಲ್ಲೆಯು ಕರ್ನಾಟಕ ರಾಜ್ಯದ ದಕ್ಷಿಣಭಾಗದಲ್ಲಿರುವ ಒಂದು ಜಿಲ್ಲೆಯಾಗಿದೆ. ಮೊದಲಿಗೆ ಇಂದಿನ ಮಂಡ್ಯ ಜಿಲ್ಲೆಯ ವ್ಯಾಪ್ತಿಗೆ ಬರುವ ಪ್ರದೇಶವು, ಹಳೆಯ ಮೈಸೂರು ಸಂಸ್ಥಾನದ\index{ಮೈಸೂರು ಸಂಸ್ಥಾನ} ಮೈಸೂರು ಜಿಲ್ಲೆಯ ಒಂದು ಭಾಗವಾಗಿತ್ತು. ಆಡಳಿತದ ಅನುಕೂಲಕ್ಕಾಗಿ 1939ರಲ್ಲಿ ಮೈಸೂರು ಜಿಲ್ಲೆಯನ್ನು ವಿಭಜಿಸಿ, ಇಂದಿನ \textbf{‘ಮಂಡ್ಯ ಜಿಲ್ಲೆ’ }\index{ಮಂಡ್ಯ ಜಿಲ್ಲೆ}ಯನ್ನು ರಚಿಸಲಾಯಿತು. ಮಂಡ್ಯ ಜಿಲ್ಲೆಯ ವಿಸ್ತೀರ್ಣ 4961 ಚ.ಕಿ.ಮೀ. ಗಳಾಗಿದ್ದು, ಮಂಡ್ಯ, ಮಳವಳ್ಳಿ ,\index{ಮಳವಳ್ಳಿ } ಮದ್ದೂರು,\index{ಮದ್ದೂರು} ನಾಗಮಂಗಲ,\index{ನಾಗಮಂಗಲ} ಕೃಷ್ಣರಾಜಪೇಟೆ,\index{ಕೃಷ್ಣರಾಜಪೇಟೆ} ಪಾಂಡವಪುರ\index{ಪಾಂಡವಪುರ} ಮತ್ತು ಶ‍್ರೀರಂಗಪಟ್ಟಣ\index{ಶ‍್ರೀರಂಗಪಟ್ಟಣ} ಎಂಬ ಏಳು ತಾಲ್ಲೂಕುಗಳನ್ನು ಹೊಂದಿದೆ. ಜಿಲ್ಲೆಯಲ್ಲಿ 31 ಹೋಬಳಿಗಳಿದ್ದು\index{ಹೋಬಳಿ}, 1365 ವಸತಿ ಗ್ರಾಮಗಳು 113 ನಿರ್ವಸತಿ (ಬೇಚಿರಾಕ್) ಗ್ರಾಮಗಳು, 652 ಉಪದಾಖಲೆ ಗ್ರಾಮಗಳಿವೆ. ಇದುವರೆಗೆ ದಾಖಲಾಗಿರುವಂತೆ, ಸುಮಾರು 320 ಗ್ರಾಮಗಳಲ್ಲಿ ಶಾಸನಗಳು ಕಂಡುಬರುತ್ತವೆ.

\section*{ಮಂಡ್ಯ ಜಿಲ್ಲೆಯ ಭೂಗೋಳ}

ಕಾವೇರಿ ನದಿಯ ಕಣಿವೆಯು ಪ್ರಾಗೈತಿಹಾಸಿಕ ಕಾಲದಿಂದಲೂ ಮಾನವನ ನೆಲೆಯಾಗಿದೆ. ಇಂದಿನ ಮಂಡ್ಯ ಜಿಲ್ಲೆಯ ಪ್ರದೇಶವು ಈ ಕಣಿವೆಯಲ್ಲಿದ್ದು, ಕಾವೇರಿ\index{ಕಾವೇರಿ}, ಹೇಮಾವತಿ\index{ಹೇಮಾವತಿ} ಮತ್ತು ಶಿಂಷಾ\index{ಶಿಂಷಾ} ನದಿಗಳ ಮಧ್ಯಗತವಾಗಿರುವುದರಿಂದ, ಜಲ ಸಮೃದ್ಧಿಯನ್ನೂ, ಫಲವತ್ತತೆಯನ್ನೂ ಹೊಂದಿದ್ದ ಭೂಮಿಯಾಗಿದೆ. ಇದರ ಜೊತೆಗೆ ವೀರವೈಷ್ಣವಿ ಮತ್ತು ಲೋಕಪಾವನಿ\index{ಲೋಕಪಾವನಿ}, ನಾರಾಯಣಗಿರಿ\index{ನಾರಾಯಣಗಿರಿ} ದುರ್ಗದ ತೊರೆ\index{ದುರ್ಗದ ತೊರೆ}, ಜಿಲ್ಲೆಯ ಪ್ರಮುಖ ಸಣ್ಣ ಉಪ ನದಿಗಳಾಗಿವೆ. ಇವು ಮಂಡ್ಯ ಜಿಲ್ಲೆಯಲ್ಲಿ ಉಗಮವಾಗಿ ಮಂಡ್ಯ ಜಿಲ್ಲೆಯಲ್ಲಿಯೇ ಹರಿದು ಶಿಂಷಾ, ಕಾವೇರಿ ಮತ್ತು ಹೇಮಾವತಿ ನದಿಗಳನ್ನು ಸೇರುತ್ತವೆ. ಜಿಲ್ಲೆಯಲ್ಲಿ ನೂರಾರು ದೊಡ್ಡ ತೊರೆಗಳು ಹರಿಯುತ್ತಿದ್ದು ಇವೆಲ್ಲಾ ಹೇಮಾವತಿ, ಕಾವೇರಿ, ಶಿಂಷಾ, ಲೋಕಪಾವನಿ ನದಿಯನ್ನು ಸೇರುವ ತೊರೆಗಳಾಗಿವೆ.

ಮಂಡ್ಯ ಜಿಲ್ಲೆಯು ಸಮತಟ್ಟು ಪ್ರದೇಶವಾದರೂ, ಅಲ್ಲಲ್ಲಿ ಬೆಟ್ಟಗುಡ್ಡಗಳೂ, ಅವುಗಳ ಕಣಿವೆ ಪ್ರದೇಶಗಳೂ ಇವೆ. ಮಳವಳ್ಳಿ ತಾಲ್ಲೂಕಿನಲ್ಲಿ ಭೀಮನಕಂಡಿಬೆಟ್ಟ\index{ಭೀಮನಕಂಡಿಬೆಟ್ಟ}, ಬಸವನಬೆಟ್ಟ\index{ಬಸವನಬೆಟ್ಟ}, ಕುಂದೂರುಬೆಟ್ಟ\index{ಕುಂದೂರುಬೆಟ್ಟ}, ಬೆಳಕವಾಡಿಬೆಟ್ಟ\index{ಬೆಳಕವಾಡಿಬೆಟ್ಟ}, ಒಡ್ಡಗಲ್ಲುರಂಗಸ್ವಾಮಿಬೆಟ್ಟ\index{ಒಡ್ಡಗಲ್ಲುರಂಗಸ್ವಾಮಿಬೆಟ್ಟ}, ಕಿರುಗಾವಲಿನ ಜೇನುಗುಡ್ಡ\index{ಜೇನುಗುಡ್ಡ}, ಮಂಡ್ಯ ಜಿಲ್ಲೆಯ ಗಡಿಯಲ್ಲಿರುವ ಕಬ್ಬಾಳುದುರ್ಗ\index{ಕಬ್ಬಾಳುದುರ್ಗ} ಬೆಟ್ಟದ ಸಾಲುಗಳಿವೆ. ಮದ್ದೂರು ತಾಲ್ಲೂಕಿನಲ್ಲಿ ಅರೆತಿಪ್ಪೂರು ಬೆಟ್ಟಗಳ ಸಾಲು ಮತ್ತು  ಹಿರಿದೇವಮ್ಮನ ಬೆಟ್ಟಗಳ ಸಾಲಿದೆ. ಪಾಂಡವಪುರ ತಾಲ್ಲೂಕಿನಲ್ಲಿ ಕುಂತಿಬೆಟ್ಟ\index{ಕುಂತಿಬೆಟ್ಟ}, ಬ್ಯಾಡರಹಳ್ಳಿ ಬೆಟ್ಟಗಳ\index{ಬ್ಯಾಡರಹಳ್ಳಿ ಬೆಟ್ಟ} ಸಾಲು, ಚಿನಕುರಳಿಬೆಟ್ಟ\index{ಚಿನಕುರಳಿಬೆಟ್ಟ}, ಬೇಬಿಬೆಟ್ಟ\index{ಬೇಬಿಬೆಟ್ಟ}, ಮೇಲುಕೋಟೆ ಬೆಟ್ಟ\index{ಮೇಲುಕೋಟೆ ಬೆಟ್ಟ} ಮತ್ತು ಮುದಿಬೆಟ್ಟದ\index{ಮುದಿಬೆಟ್ಟ} ಸಾಲುಗಳಿವೆ. ಮಂಡ್ಯ ತಾಲ್ಲೂಕಿನಲ್ಲಿ ಸಾತನೂರು ಬೆಟ್ಟದ\index{ಸಾತನೂರು ಬೆಟ್ಟ} ಸಾಲಿದೆ. ಕೃಷ್ಣರಾಜಪೇಟೆ ತಾಲ್ಲೂಕಿನಲ್ಲಿ ನಾರಾಯಣಗಿರಿದುರ್ಗದ ಬೆಟ್ಟಗಳ ಸಾಲು, ಗಜರಾಜಗಿರಿ\index{ಗಜರಾಜಗಿರಿ}, ಹೇಮಗಿರಿಯ\index{ಹೇಮಗಿರಿ} ಬೆಳ್ಳಿ ಬೆಟ್ಟದ\index{ಬೆಳ್ಳಿ ಬೆಟ್ಟ} ಸಾಲುಗಳಿವೆ. ನಾಗಮಂಗಲ ತಾಲ್ಲೂಕಿನಲ್ಲಿ, ಆದಿಚುಂಚನಗಿರಿ ಬೆಟ್ಟದ\index{ಆದಿಚುಂಚನಗಿರಿ} ಸಾಲು, ಹದ್ದಿನ ಕಲ್ಲು ಹನುಮಂತರಾಯನ ಬೆಟ್ಟ\index{ಹನುಮಂತರಾಯನ ಬೆಟ್ಟ}, ಕೋಟೆಬೆಟ್ಟ\index{ಕೋಟೆಬೆಟ್ಟ}, ಹಾಲತಿ ಬೆಟ್ಟ\index{ಹಾಲತಿ ಬೆಟ್ಟ}, ಬಸವನಬೆಟ್ಟ\index{ಬಸವನಬೆಟ್ಟ}, ಕೋಣನಕಲ್ಲು\index{ಕೋಣನಕಲ್ಲು}, ಮುದಿಬೆಟ್ಟದ(ಸಾತೇನಹಳ್ಳಿ) ಸಾಲುಗಳಿವೆ. ಬಸವನ ಕಲ್ಲು ಅಥವಾ ಬಸವನ ಬೆಟ್ಟದಲ್ಲಿ ಲೋಕಪಾವನಿ ನದಿಯು ಹುಟ್ಟುತ್ತದೆ. ಇಲ್ಲಿನ ಒಂದು ಗುಹೆಯಲ್ಲಿ ರಾಮಾನುಜಾಚಾರ್ಯರು\index{ರಾಮಾನುಜಾಚಾರ್ಯರು} ತಪಸ್ಸು ಮಾಡಿದ್ದರೆಂದು ಹೇಳುವ ಸುಣ್ಣದಲ್ಲಿ ಬರೆದಿರುವ ಅಪರೂಪದ ಶಾಸನವೂ ಇದೆ. ಕೃಷ್ಣರಾಜಪೇಟೆ ಮತ್ತು ನಾಗಮಂಗಲ ಗಡಿಯಲ್ಲಿರುವ ಬೆಟ್ಟದ ಹಳ್ಳಿಯಿಂದ ಒಂದು ಬೆಟ್ಟದ ಸಾಲು ಶ್ರವಣಬೆಳಗೊಳದ ಕಡೆಗೆ, ಇನ್ನೊಂದು ಬೆಟ್ಟದ ಸಾಲು ಚುಂಚನಗಿರಿ ಬೆಟ್ಟದ ಕಡೆಗೆ ಹೋಗಿದೆ. ಇವೆಲ್ಲಾ ಹೆಚ್ಚು ಎತ್ತರವಿಲ್ಲದ ಸುಮಾರು 500 ಮೀಟರ್ ಎತ್ತರದ ಒಳಗಿರುವ ಬೆಟ್ಟಗಳಾಗಿವೆ.

ಈ ಗಿರಿ ಶ್ರೇಣಿಗಳ ಪೈಕಿ ಕಬ್ಬಾಳು ದುರ್ಗವು 1069 ಮೀಟರ್ ಎತ್ತರ ಇದ್ದರೆ, ನಾರಾಯಣಗಿರಿದುರ್ಗ ಅಥವಾ ಕೈವಲ್ಯೇಶ್ವರ ಬೆಟ್ಟ\index{ಕೈವಲ್ಯೇಶ್ವರ ಬೆಟ್ಟ} ಮತ್ತು ಮೇಲುಕೋಟೆ ಬೆಟ್ಟಗಳು ಸುಮಾರು 1000 ಮೀಟರ್​ ಎತ್ತರದಲ್ಲಿದ್ದು, ಇವುಗಳ ಕೆಳಗೆ ಫಲವತ್ತಾದ ಭೂಮಿಯನ್ನು ಹೊಂದಿರುವ ಕಣಿವೆಗಳಿವೆ. ಈ ಕಣಿವೆ ಪ್ರದೇಶದಲ್ಲಿ ದಟ್ಟವಾದ ಕಾಡುಗಳಿದ್ದು, ಕ್ರಮೇಣ ಕುರುಚಲು ಗಿಡದ ಕಾಡುಗಳಾಗಿ ಪರಿವರ್ತಿತವಾಗಿದೆ. ಫಲವತ್ತಾದ ಭೂಮಿಯನ್ನು ವ್ಯವಸಾಯಕ್ಕೆ ಒಳಪಡಿಸಲಾಗಿದೆ. ಈ ಪ್ರದೇಶಗಳು ಹಿತಕರವಾದ ಹವೆಯನ್ನು ಹೊಂದಿವೆ. ಚಿನಕುರಳಿಯ ಕಣಿವೆಯು ಬಹಳ ದೊಡ್ಡದು ಫಲವತ್ತಾದುದು. ಈ ಕಣಿವೆಯಲ್ಲಿ ಅನೇಕ ಹಳ್ಳಿಗಳಿದ್ದು, ಕಣಿವೆ ಕೊಪ್ಪಲು ಎಂಬ ಹಳ್ಳಿಯೇ ಇದೆ. ಆದರೆ ಈಚೆಗೆ ಇಲ್ಲಿ ನೀರಿನ ಕೊರತೆ ಉಂಟಾಗಿದೆ.

\vskip -3pt

ಮಂಡ್ಯ ಜಿಲ್ಲೆಯಲ್ಲಿ ಸುಮಾರು 24,063 ಹೆಕ್ಟೇರ್​ ಮೀಸಲು ಅರಣ್ಯ ಇದೆ. ಜಿಲ್ಲೆಯಲ್ಲಿರುವ ಕಾಡುಗಳಲ್ಲಿ ಮಳವಳ್ಳಿ ತಾಲ್ಲೂಕಿನ ಬಸವನಬೆಟ್ಟದ ಕಾಡೇ ಅತ್ಯಂತ ದೊಡ್ಡದು. ಈ ಕಾಡು ಈಗ ಸಂರಕ್ಷಿತ ಅಭಯಾರಣ್ಯವಾಗಿದ್ದು ಇದನ್ನು ಕಾವೇರಿ ವನ್ಯಧಾಮ\index{ಕಾವೇರಿ ವನ್ಯಧಾಮ} ಎಂದು ಕರೆಯಲಾಗುತ್ತದೆ. ಇದರ ವಿಸ್ತೀರ್ಣ ಸು. 10,035 ಹೆಕ್ಟೇರ್​. ಇದು ನೆರೆಯ ಕನಕಪುರ ತಾಲ್ಲೂಕು, ಕೊಳ್ಳೇಗಾಲ ತಾಲ್ಲೂಕುಗಳವೆರೆಗೆ ವಿಸ್ತರಿಸಿದೆ. ಸು. 2623 ಹೆಕ್ಟೇರ್​ ವಿಸ್ತೀರ್ಣವಿರುವ ಮಳವಳ್ಳಿ ತಾಲ್ಲೂಕಿನ ದನಗೂರು\index{ದನಗೂರು} ಅರಣ್ಯ ಪ್ರದೇಶವೂ ಈ ಕಾಡಿಗೆ ಹೊಂದಿಕೊಂಡಿದೆ. ಈ ಕಾಡಿನ ಮಧ್ಯದಲ್ಲಿ ಹರಿಯುವ ಕಾವೇರಿ ನದಿಯಲ್ಲಿ ಮುತ್ತೆತ್ತಿ, ತೊರೆಕಾಡನಹಳ್ಳಿ, ಶಿಂಶಾ, ಮೊದಲಾದ ಪ್ರಸಿದ್ಧ ಊರುಗಳಿವೆ. ಈ ಕಾಡಿನಲ್ಲಿ ಆನೆಗಳು ಅಪಾರ ಸಂಖ್ಯೆಯಲ್ಲಿವೆ. ಜೊತೆಗೆ ಚಿರತೆ, ಕಾಡುಹಂದಿ, ಜಿಂಕೆ, ಕಾಡೆಮ್ಮೆ ಮೊದಲಾದ ಕಾಡುಪ್ರಾಣಿಗಳಿವೆ. ಹೆಬ್ಬಾವುಗಳೂ ಕಂಡು ಬರುತ್ತವೆ. ಕಾವೇರಿ ನದಿಯಲ್ಲಿ ಭೀಮೇಶ್ವರಿ ಜಲಧಾಮ ಇದ್ದು, ಇಲ್ಲಿ ಮುಷೀರ್​ ಎಂಬ ವಿಶೇಷ ಜಾತಿಯ ಬೃಹತ್​ ಮೀನುಗಳಿವೆ. ಅದನ್ನು ಬಿಟ್ಟರೆ ಮೇಲುಕೋಟೆ ಕಾಡೇ ಅತ್ಯಂತ ದೊಡ್ಡದು. ಇದನ್ನು ನಾರಾಯಣಗಿರಿದುರ್ಗದ ಕಾಡು ಎಂದೂ ಕರೆಯುತ್ತಾರೆ. ಈ ಕಾಡು ನಾಗಮಂಗಲ, ಕೃಷ್ಣರಾಜಪೇಟೆ, ಪಾಂಡವಪುರ ತಾಲ್ಲೂಕುಗಳಲ್ಲಿ ಹರಡಿದ್ದು, ಸು.4,402 ಕಿ.ಮೀ. ವಿಸ್ತಾರವಾಗಿದೆ. ಈ ಕಾಡಿನಲ್ಲಿ ಹಿಂದೆ, ಹುಲಿ, ಚಿರತೆಗಳೂ ಇದ್ದವು. ಈಗ ಹುಲಿಗಳಿಲ್ಲ. ಚಿರತೆಗಳು ಮಾತ್ರ ಕಂಡು ಬರುತ್ತವೆ. ಜೊತೆಗೆ, ಕಾಡುಹಂದಿ, ತೋಳ, ನರಿ, ಜಿಂಕೆ, ಮೊಲ ಮೊದಲಾದ ಪ್ರಾಣಿಗಳು ಕಂಡು ಬರುತ್ತವೆ. ಮೊದಲು ಈ ಕಾಡಿನಲ್ಲಿ ಹೆಬ್ಬಾವುಗಳಿದ್ದವು. ಈಗ ಹೆಬ್ಬಾವುಗಳು ಕಂಡು ಬರುತ್ತಿಲ್ಲ.

\vskip -10pt

\section*{ಪುರಾತತ್ತ್ವ\index{ಪುರಾತತ್ತ್ವ}}

ಬ್ರೂಸ್​ಫೂಟ್​\index{ಬ್ರೂಸ್​ಫೂಟ್​} ಎಂಬ ಸಂಶೋಧಕನು 19ನೇ ಶತಮಾನದ ಕೊನೆಯ ಭಾಗದಲ್ಲೇ ಮಂಡ್ಯ ಜಿಲ್ಲೆಯಲ್ಲಿ ಸಂಚರಿಸಿ ಅನೇಕ ಕಡೆಗಳಲ್ಲಿ ಪುರಾತತ್ವ ನೆಲೆಗಳನ್ನು ಪತ್ತೆ ಹಚ್ಚಿದ್ದನೆಂದೂ, ಅದರಲ್ಲಿ ಇಂದಿನ ಕುಂತಿಬೆಟ್ಟ\index{ಕುಂತಿಬೆಟ್ಟ} ಅಥವಾ ಫ್ರೆಂಚ್​ರಾಕ್ಸ್​ನಲ್ಲಿ\index{ಫ್ರೆಂಚ್​ರಾಕ್ಸ್​} ದೊರೆತ ನವಶಿಲಾಯುಗದ ಸಂಸ್ಕೃತಿಯು ಪ್ರಮುಖವೆಂದೂ ತಿಳಿದುಬರುತ್ತದೆ.\endnote{ ಕೃಷ್ಣಮೂರ್ತಿ,ಡಾ॥ ಎಂ.ಎಸ್​., ಅಧ್ಯಕ್ಷರ ಭಾಷಣ, ಮಂಡ್ಯ ಜಿಲ್ಲೆಯ ಇತಿಹಾಸ ಮತ್ತು ಪುರಾತತ್ವ, ಪುಟ 9-10} ಶ‍್ರೀರಂಗಪಟ್ಟಣ ತಾಲ್ಲೂಕಿನ ರಂಗನತಿಟ್ಟು\index{ರಂಗನತಿಟ್ಟು}, ಶ‍್ರೀರಂಗಪಟ್ಟಣ\index{ಶ‍್ರೀರಂಗಪಟ್ಟಣ} ಕೋಟೆಯ ಒಳಗೆ ರಂಗನಾಥದೇವಾಲಯದಿಂದ ಆಗ್ನೇಯದಲ್ಲಿ ಒಂದು ಫರ್ಲಾಂಗ್​ ದೂರದಲ್ಲಿರುವ ರಂಗನಾಥನಗರದಲ್ಲಿ\index{ರಂಗನಾಥನಗರ} ಕಂಡುಬಂದಿರುವ ಪ್ರಾಗೈತಿಹಾಸಿಕ ನೆಲೆ,\endnote{ ಮಹದೇವ, ಡಾ॥ ಸಿ., ಶ‍್ರೀರಂಗಪಟ್ಟಣದ ಪ್ರಾಗೈತಿಹಾಸಿಕ ನೆಲೆ, ಕರ್ನಾಟಕ ಪುರಾತತ್ವ, ಪುಟ 1-4} ಮಳವಳ್ಳಿ ತಾಲ್ಲೂಕು, ಹೊನಗನಹಳ್ಳಿಯ ಬೃಹತ್​ ಶಿಲಾಯುಗದ ಸಮಾಧಿಯ ಅವಶೇಷಗಳು,\endnote{ ಬಸವರಾಜು, ಡಾ॥ ಬಿ., ಹೊನಗನಹಳ್ಳಿಯಲ್ಲಿ ಪುರಾತತ್ವ ಶೋಧನೆ, ಮಂಡ್ಯ ಜಿಲ್ಲೆಯ ಇತಿಹಾಸ ಮತ್ತು ಪುರಾತತ್ವ, ಪುಟ 72-75} ಮಳವಳ್ಳಿ ತಾಲ್ಲೂಕಿನ ಹುಳ್ಳಂಬಳ್ಳಿ\index{ಹುಳ್ಳಂಬಳ್ಳಿ} ಮತ್ತು ಶೆಟ್ಟಿಹಳ್ಳಿಗಳಲ್ಲಿ\index{ಶೆಟ್ಟಿಹಳ್ಳಿ} ಕಂಡು ಬರುವ ನೂತನ ಶಿಲಾಯುಗದ\index{ಶಿಲಾಯುಗ} ಸಂಸ್ಕೃತಿಯ ಕುರುಹುಗಳು, ಬೆಳಕವಾಡಿ, ಹಲಗೂರು, ಕುಂತಿಬೆಟ್ಟ, ತೊಣ್ಣೂರುಗಳಲ್ಲಿ ಕಂಡುಬರುವ ಕಬ್ಬಿಣಯುಗ ಮತ್ತು ಬೃಹತ್​ ಶಿಲಾಯುಗದ ನೆಲೆಗಳು, ಮಂಡ್ಯ ಜಿಲ್ಲೆಯು ಪ್ರಾಚೀನ ಕಾಲದಿಂದಲೂ ಮಾನವನ ನೆಲೆವೀಡಾಗಿತ್ತೆಂಬುದನ್ನು ಖಚಿತಪಡಿಸುತ್ತವೆ.

\vskip -3pt

ಸ್ಥಳಪರಿವೀಕ್ಷಣೆಯ ಪ್ರಕಾರ ಕೃಷ್ಣರಾಜಪೇಟೆ ತಾಲ್ಲೂಕು ಸಂತೇಬಾಚಹಳ್ಳಿಯ ಬೆಣ್ಣೆಸಿದ್ದನಗುಡ್ಡದ\index{ಬೆಣ್ಣೆಸಿದ್ದನಗುಡ್ಡ} ಬಳಿ ಇರುವ \break ಶಿಲಾಯುಗದ ಅವಶೇಷಗಳು ಮತ್ತು ಪಡುವಲಪಟ್ಟಣದ ಹತ್ತಿರದ ಬಸವನ ಬೆಟ್ಟದಲ್ಲಿರುವ\index{ಬಸವನ ಬೆಟ್ಟ} ಪಾಂಡವರ ಗುಹೆಯ ಸುತ್ತಮುತ್ತಲೂ ಶಿಲಾಯುಗದ ಸಂಸ್ಕೃತಿಯ ಕುರುಹುಗಳು ಕಂಡು ಬಂದಿದ್ದು, ಈ ಬಗ್ಗೆ ಸಂಶೋಧನೆ ನಡೆಯಬೇಕಾಗಿದೆ. ಮಂಡ್ಯ ಜಿಲ್ಲೆಯ ಗಡಿಗೆ ಹೊಂದಿಕೊಂಡಿರುವ ಹಾಸನ ಜಿಲ್ಲೆ, ಹಿರಿಸಾವೆ ಹೋಬಳಿಯ, ಬೆಟ್ಟದಹಳ್ಳಿ \index{ಬೆಟ್ಟದಹಳ್ಳಿ } ಗ್ರಾಮದ ಬಳಿ ಇರುವ ಸಿಡಿಲುಕಲ್ಲು\index{ಸಿಡಿಲುಕಲ್ಲು} ಎಂಬ ಗುಡ್ಡ ಪ್ರದೇಶದಲ್ಲಿ, ಪಾಂಡವರ ಗುಡಿಗಳು, ಗುಹೆ ಇದ್ದು, ಇಲ್ಲಿ ಶಿಲಾಯುಗದ ಬೆಣಚುಕಲ್ಲಿನ ಆಯುಧಗಳು ಸಿಗುತ್ತವೆ. ಮಂಡ್ಯ ಜಿಲ್ಲೆಯ ಪಶ್ಚಿಮೋತ್ತರ ಗಡಿಯಿಂದ ಕೇವಲ 5 ಕಿ.ಮೀ ದೂರದಲ್ಲಿರುವ ಶ್ರವಣಬೆಳಗೊಳವು\index{ಶ್ರವಣಬೆಳಗೊಳ} ಕಟವಪ್ರ\index{ಕಟವಪ್ರ (ಗಿರಿ, ಶೈಲ)} ಎಂದು ಅನಾದಿಕಾಲದಿಂದ ಪ್ರಸಿದ್ಧವಾಗಿದ್ದು, ಇಲ್ಲಿಗೆ ಕ್ರಿ.ಪೂ.ದಲ್ಲಿಯೇ ಭದ್ರಬಾಹು\index{ಭದ್ರಬಾಹು} ಮನಿಗಳು, ಚಂದ್ರಗುಪ್ತ ಮೌರ್ಯ\index{ಚಂದ್ರಗುಪ್ತ ಮೌರ್ಯ} ಮತ್ತು ಅವರ ಜೊತೆ ಅನೇಕ ಜೈನಮುನಿಗಳು ಬಂದು ನೆಲೆಸಿದ್ದರೆಂಬುದು ಐತಿಹಾಸಿಕ ಪರಂಪರೆಯಿಂದ ತಿಳಿದುಬರುತ್ತದೆ. ಅದಕ್ಕೆ ಮುಂಚೆಯೇ ಶ್ರವಣಬೆಳಗೊಳವು ಶಿಲಾಯುಗದ ಕಾಲದಿಂದಲೂ ಮಾನವನ ನೆಲೆಯಾಗಿತ್ತೆಂದು ಹೇಳಬಹುದು.


\section*{ಸ್ಥಳಪುರಾಣ ಮತ್ತು ಸಾಂಪ್ರದಾಯಕ ದಾಖಲೆಗಳಲ್ಲಿ ಮಂಡ್ಯ ಹೆಸರಿನ ನಿಷ್ಪತ್ತಿ}

ಕೃತಯುಗದಲ್ಲಿ ಮಂಡ್ಯವನ್ನು ‘ವೇದಾರಣ್ಯ’\index{ವೇದಾರಣ್ಯ}, ‘ವಿಷ್ಣುಪುರ’\index{ವಿಷ್ಣುಪುರ} ಎಂದು ಕರೆಯಲಾಗುತ್ತಿತ್ತೆಂದು ಹೇಳಲಾಗಿದೆ. ಮಾಂಡವ್ಯ ಋಷಿಗಳು\index{ಮಾಂಡವ್ಯ ಋಷಿ} ಈ ಸ್ಥಳದಲ್ಲಿ ತಪಸ್ಸು ಮಾಡುತ್ತಿದ್ದರೆಂದೂ, ಪ್ರಾಣಿಗಳಿಗೂ ಕೂಡಾ ವೇದವನ್ನು ಉಪದೇಶ ಮಾಡುತ್ತಿದ್ದರೆಂದು, ಆದುದರಿಂದ ಈ ಪ್ರದೇಶವನ್ನು ವೇದಾರಣ್ಯವೆಂದೂ ಕರೆಯುತ್ತಿದ್ದರೆಂದು, ಮಾಂಡವ್ಯ ಋಷಿಗಳ ಕಾರಣದಿಂದಾಗಿಯೇ ಮಂಡ್ಯ ಎಂಬ ಹೆಸರೂ ಬಂದಿದೆ ಎಂಬುದು ಸ್ಥಳಪುರಾಣಗಳ ಪ್ರತೀತಿ. ಲಕ್ಷ್ಮೀಜನಾರ್ದನ\index{ಲಕ್ಷ್ಮೀಜನಾರ್ದನ} ದೇವರನ್ನು ಮಾಂಡವ್ಯ ಋಷಿಗಳೇ ಪ್ರತಿಷ್ಠಾಪಿಸಿದರೆಂದೂ ಹೇಳುತ್ತಾರೆ. ಲಕ್ಷ್ಮೀಜನಾರ್ದನ ದೇವಾಲಯದಲ್ಲಿರುವ\index{ಲಕ್ಷ್ಮೀಜನಾರ್ದನ ದೇವಾಲಯ} ಅಮ್ಮನವರಿಗೆ ವೇದವಲ್ಲಿ\index{ವೇದವಲ್ಲಿ} ಅಮ್ಮನವರೆಂಬ ಹೆಸರು ಇದ್ದು, ವೇದಾರಣ್ಯವೆಂಬ ಹೆಸರಿಗೆ ಪುಷ್ಟಿ ನೀಡುತ್ತದೆ. ಲಕ್ಷ್ಮೀಜನಾರ್ದನ ದೇವಾಲಯವು ವಿಜಯನಗರ ಕಾಲದ ರಚನೆಯಾಗಿದೆ. ಮಂಡ್ಯ\index{ಮಂಡ್ಯ} ನಗರದ ಪುರಾತನ ದೇವಾಲಯವೆಂದರೆ ಮಂಡ್ಯ ಕೆರೆಯ ಕೋಡಿ ಹಳ್ಳದ ದಂಡೆಯಲ್ಲಿರುವ ಸಕಲೇಶ್ವರ ಸ್ವಾಮಿ\index{ಸಕಲೇಶ್ವರ ಸ್ವಾಮಿ} ದೇವಾಲಯ ಇದನ್ನೂ ಒಬ್ಬ ಋಷಿಯು ಸ್ಥಾಪಿಸಿದನೆಂದು ಹೇಳುತ್ತಾರೆ. ಈ ದೇವಾಲಯ ಹಾಗೂ ಜೈನರ ಬೀದಿಯಲ್ಲಿರುವ ಜೈನಬಸದಿಯು ವಾಸ್ತುದೃಷ್ಟಿಯಿಂದ ಗಂಗರ ಕಾಲದ ರಚನೆಗಳು ಎಂದು ಕಂಡುಬರುತ್ತವೆ. ವೈದ್ಯನಾಥ ಪುರದ ವೈದ್ಯನಾಥೇಶ್ವರ ಮತ್ತು ಗುತ್ತಲಿನ ಅರ್ಕೇಶ್ವರ ದೇವಾಲಯಗಳು\index{ಅರ್ಕೇಶ್ವರ ದೇವಾಲಯ} ಬಹಳ ಪುರಾತನವಾದುವು.

ಸ್ಥಳಪುರಾಣದ ಪ್ರಕಾರ, ದ್ವಾಪರಯುಗದ ಅಂತ್ಯದಲ್ಲಿ ಇಂದ್ರವರ್ಮನೆಂಬ\index{ಇಂದ್ರವರ್ಮ} ರಾಜನು ಲಕ್ಷ್ಮೀಜನಾರ್ದನ ದೇವರನ್ನು ಹಾಗೂ ಈ ಸ್ಥಳದ ದೇವರುಗಳನ್ನು ಪೂಜಿಸಿ ಸೋಮವರ್ಮನೆಂಬ\index{ಸೋಮವರ್ಮ} ಮಗನನ್ನು ಪಡೆದನೆಂದು, ಸೋಮವರ್ಮನು ಈ ಊರಿಗೆ ಕೋಟೆಯನ್ನು ಕಟ್ಟಿ, ಅಗ್ರಹಾರವನ್ನು ನಿರ್ಮಿಸಿ ಅದಕ್ಕೆ ‘ಮಂಡೆವೇಮು’\index{ಮಂಡೆವೇಮು} ಎಂಬ ಹೆಸರಿಟ್ಟನೆಂದೂ, ಅದೇ ‘ಮಂಡ್ಯ’\index{ಮಂಡ್ಯ} ಎಂಬುದಾಗಿ ರೂಪಾಂತರವಾಯಿತೆಂದು ತಿಳಿದುಬರುತ್ತದೆ. ಈಗಲೂ ಸಂತಾನಕ್ಕಾಗಿ ಸ್ಥಳೀಯ ಜನರು ಲಕ್ಷ್ಮೀಜನಾರ್ದನ ದೇವರಿಗೆ ತೊಟ್ಟಿಲು ಕಟ್ಟಿ ಮಕ್ಕಳನ್ನು ಪಡೆಯಲು ಹರಕೆ ಹೊರುವ ಪದ್ಧತಿ ರೂಢಿಯಲ್ಲಿದೆ.

ರಾಮಾನುಜಾಚಾರ್ಯರ ನೇರ ಶಿಷ್ಯರಾದ ಕಿರಂಗೂರಿನ ಅನಂತಾಚಾರ್ಯರ ವಂಶದ ಹನ್ನೆರಡನೇ ತಲೆಮಾರಿನ\break ಗೋವಿಂದರಾಜರಿಗೆ ಕೃಷ್ಣದೇವರಾಯನು ಈ ಊರನ್ನು ದತ್ತಿ ಹಾಕಿಕೊಟ್ಟನೆಂದೂ, ಇಲ್ಲಿಗೆ ಅವರ ಜೊತೆಯಲ್ಲಿ ಬಂದು ನೆಲೆಸಿದ ಶ‍್ರೀವೈಷ್ಣವ ಬ್ರಾಹ್ಮಣರು ತಿರುಪತಿ ಬಳಿ ತಾವು ವಾಸಿಸುತ್ತಿದ್ದ ‘ಮಂಡೆಯಂ’\index{ಮಂಡೆಯಂ} ಎಂಬ ಅಗ್ರಹಾರದ ಹೆಸರನ್ನು ಇದಕ್ಕೂ ಇಟ್ಟರೆಂದೂ ತಿಳಿದುಬರುತ್ತದೆ.\endnote{ ಕೇಶವರಾವ್​ ಮನ್ನೇಕೋಟೆ, ಮಂಡ್ಯ ಜಿಲ್ಲೆ ದರ್ಶನ, ಪುಟ 4-5} ಮೇಲುಕೋಟೆಯ ಕಲ್ಯಾಣಿ ತೀರದ ಗಜೇಂದ್ರಮಂಟಪವನ್ನು\index{ಗಜೇಂದ್ರಮಂಟಪ} ನಿರ್ಮಿಸುವುದರಲ್ಲಿ ‘ಮಂಡ್ಯಂ ಪಾರ್ಥಸಾರಥಿ’\index{ಮಂಡ್ಯಂ ಪಾರ್ಥಸಾರಥಿ} ಕುಟುಂಬದವರೂ ತಮ್ಮ ಸೇವೆ ಸಲ್ಲಿಸಿದ್ದಾರೆ.\endnote{ ಎಕ 6 ಪಾಂಪು 192 ಮೇಲುಕೋಟೆ 19 ನೇ ಶ.} “ಮಂಡೆಯಂ ಶ‍್ರೀ ವೈಷ್ಣವ ವಂಶದವರು ಮೊದಲು ಕೆಳತಿರುಪತಿಯ ಸಮೀಪದಲ್ಲಿ ‘ಇಳೆಯ ಮಂಡೆಯಂ’\index{ಇಳೆಯ ಮಂಡೆಯಂ} ಅಗ್ರಹಾರದವರೆಂದು ಅಲ್ಲಿಂದ ಬಂದು ಮಂಡ್ಯ, ಮಂಡ್ಯ ಕೊಪ್ಪಲು\index{ಮಂಡ್ಯ ಕೊಪ್ಪಲು} ಮುಂತಾದ ಕಡೆ ನೆಲೆನಿಂತರೆಂಬುದೂ ಒಂದು ಪ್ರತೀತಿ”.\endnote{ ವೆಂಕಟಾಚಲಶಾಸ್ತ್ರೀ ಡಾ॥ ಟಿ.ವಿ., ಪ್ರಾಕ್ತನ, ಆರ್​.ನರಸಿಂಹಾಚಾರ್ಯ: ಜೀವನ ಮತ್ತು ಕಾರ್ಯ, ಪುಟ \engfoot{ix}} ಇಂದಿಗೂ ಲಕ್ಷ್ಮೀಜನಾರ್ದನ ದೇವಾಲಯದಲ್ಲಿ ಗೋವಿಂದರಾಜಗುರುಗಳ \index{ಗೋವಿಂದರಾಜಗುರು} ವಿಗ್ರಹವಿದ್ದು, ಪೂಜೆಯ ಕಾಲದಲ್ಲಿ ಅರ್ಪಿಸುವ ನೈವೇದ್ಯದ ಒಂದು ಭಾಗವನ್ನು ಗೋವಿಂದರಾಜ ಗುರುಗಳ ಭಾಗವೆಂದು ತೆಗೆದಿರಿಸುವುದು ರೂಢಿಯಲ್ಲಿದೆ ಎಂಬುದು ಅರ್ಚಕರು ನೀಡಿದ ಮಾಹಿತಿಯಿಂದ ತಿಳಿದುಬರುತ್ತದೆ.

“ಶ‍್ರೀರಂಗಪಟ್ಟಣ ಸಮೀಪದ ಕಿರಂಗೂರಿನ ಅನಂತಾಚಾರ್ಯರು\index{ಅನಂತಾಚಾರ್ಯರು} (ಆನಂದಾನ್​ಪುಳ್ಳೆ\index{ಆನಂದಾನ್​ ಪುಳ್ಳೆ}) (ಜನನ ಕ್ರಿ.ಶ.1053) ಶ‍್ರೀರಂಗಕ್ಕೆ\index{ಶ‍್ರೀರಂಗ} ಹೋಗಿ ಶ‍್ರೀರಾಮಾನುಜರಲ್ಲಿ ವಿದ್ಯಾಭ್ಯಾಸ ಮಾಡಿ ಅವರ 74 ಸಿಂಹಾಸನಾಧಿಪತಿಗಳಲ್ಲಿ ಒಬ್ಬರಾದರು. ರಾಮಾನುಜರ ಅಪ್ಪಣೆಯಂತೆ ಅವರು ಶ‍್ರೀನಿವಾಸನ ಅಂತರಂಗದ ಭಕ್ತರಾಗಿ ತಿರುಪತಿಯಲ್ಲಿ\index{ತಿರುಪತಿ} ನೆಲೆಸಿದರು. ಇವರ ವಂಶಸ್ಥರು ಅದ್ಯಾಪಿ ತಿರುಮಲೆ, ಕುನ್ನಂಪಾಕಂ, ಕುಣರಪಾಕಂ, ಪುರಿಶೈ ಮತ್ತು ಮಂಡಯಂ ಎಂಬ ಐದು ಅಗ್ರಹಾರ ಸಮುದಾಯಗಳಿಗೆ ಸೇರಿದವರು. ಅವರಿಂದ ಹತ್ತನೆಯ ತಲೆಮಾರಿನ ಕುನ್ನಪಾಕಂ ಶಾಖೆಗೆ ಸೇರಿದ ಶ‍್ರೀನಿವಾಸಾಚಾರ್ಯರೆಂಬ\index{ಶ‍್ರೀನಿವಾಸಾಚಾರ್ಯ} ಪಂಡಿತವರ್ಯರು ತಮ್ಮ ಗುರುಗಳೊಡನೆ ಮೈಸೂರು ಪ್ರಾಂತ್ಯಕ್ಕೆ ಬಂದು ಮೊದಲು ಮೇಲುಕೋಟೆಯ\index{ಮೇಲುಕೋಟೆ} ಯತಿರಾಜಮಠದಲ್ಲಿ\index{ಯತಿರಾಜಮಠ} ರಾಮಾನುಜಜೀಯರ್​\index{ರಾಮಾನುಜಜೀಯರ್} ಎಂಬ ಹೆಸರಿನ ಗುರುಗಳಾದರು. ಇವರಿಗೆ 1558 ರಲ್ಲಿ ವಿಜಯನಗರದ ಅರಸರು ಬೆಂಗಳೂರು ಜಿಲ್ಲೆ, ಚನ್ನಪಟ್ಟಣ ತಾಲ್ಲೂಕು ಮಾರ್ಚನಹಳ್ಳಿಯನ್ನು\index{ಮಾರ್ಚನಹಳ್ಳಿ} ದತ್ತಿಯಾಗಿ ಹಾಕಿಕೊಟ್ಟು, ಈ ಹಳ್ಳಿಗೆ ಶ‍್ರೀನಿವಾಸ ರಾಮಾನುಜಪುರಂ\index{ಶ‍್ರೀನಿವಾಸ ರಾಮಾನುಜಪುರಂ} ಎಂಬ ಹೆಸರಿಟ್ಟರು. ಈ ಸಮುದಾಯದವರು ಚನ್ನಪಟ್ಟಣದ ಹತ್ತಿರವಿರುವ ಮಳೂರಿನಲ್ಲಿಯೂ\index{ಮಳೂರು}, ಶ‍್ರೀರಂಗಪಟ್ಟಣ ತಾಲ್ಲೂಕಿನ ಮಹದೇವಪುರದ\index{ಮಹದೇವಪುರ} ಬಳಿ ಕಾವೇರಿ ನದಿ ದಂಡೆಯಲ್ಲಿರುವ ಮಂಡ್ಯದ ಕೊಪ್ಪಲಿನಲ್ಲಿಯೂ\index{ಮಂಡ್ಯದ ಕೊಪ್ಪಲು} ನೆಲೆಸಿದರು”.\endnote{ ಅನಂತರಂಗಾಚಾರ್​, ಎನ್​., ರಾ.ನರಸಿಂಹಾಚಾರ್ಯ, ಪುಟ 2-3} ಕವಿಚರಿತೆಯನ್ನು ರಚಿಸಿದ ರಾವ್​ಬಹದ್ದೂರ್​\index{ರಾವ್​ಬಹದ್ದೂರ್​} ನರಸಿಂಹಾಚಾರ್ಯರು\index{ನರಸಿಂಹಾಚಾರ್ಯ} ಈ ಮಂಡ್ಯ ಕೊಪ್ಪಲಿನವರು. ಮಂಡ್ಯ ಕೊಪ್ಪಲಿನ ಬಳಿ ಕಾವೇರಿ ನದಿ ದಂಡೆಯಲ್ಲಿ, ಹೊಯ್ಸಳರ ಕಾಲದ ನರಸಿಂಹ ದೇವಾಲಯವಿದೆ\index{ನರಸಿಂಹ ದೇವಾಲಯ}. ಈ ಮಂಡೆಯಂ ವಂಶದವರಿಂದಲೇ ಮಂಡ್ಯ ಎಂಬ ಹೆಸರು ಬಂದಿದೆ ಎಂದು ಹೇಳಲಾಗುತ್ತದೆ. ಈ ಮೂಲದ ಶ‍್ರೀ ವೈಷ್ಣವರನ್ನು ಮಂಡೆಯಂ ಅಯ್ಯಂಗಾರ್​\index{ಮಂಡೆಯಂ ಅಯ್ಯಂಗಾರ್} ಎಂದು ಕರೆಯುತ್ತಾರೆ.


\section*{ಶಾಸನಗಳಲ್ಲಿ ಮಂಡ್ಯ ಹೆಸರಿನ ನಿಷ್ಪತ್ತಿ}

ಮಂಡ್ಯ ಸ್ಥಳನಾಮದ ನಿಷ್ಪತ್ತಿಯ ಬಗ್ಗೆ ಜಿಲ್ಲೆಯ ಶಾಸನಗಳನ್ನು ಪರಿಶೀಲಿಸಿದಾಗ, ಕ್ರಿ.ಶ. 1242ರ ಹಂಪಾಪುರ\index{ಹಂಪಾಪುರ} ವೀರಗಲ್ಲಿನಲ್ಲಿ \textbf{‘ಮಂಡಗೌಡ’}\index{ಮಂಡಗೌಡ}ನೆಂಬ ಹೆಸರು ಕಂಡುಬರುತ್ತದೆ. ಈ ಮಂಡ ಎಂಬುದರಿಂದ, ಮಂಡ\textgreater ಮಂಡೆ\textgreater ಮಂಡೇವು\index{ಮಂಡೇವು}\textgreater ಮಂಡ್ಯ ಹೆಸರು ಬಂದಿರಬಹುದು ಎಂಬ ಊಹೆಯನ್ನು ಮಾಡಬಹುದು.\endnote{ ಎಕ 7 ಮಂ 23 ಹಂಪಾಪುರ 1242} ಕ್ರಿ.ಶ. 1276ರ ಹೊಯ್ಸಳರ ಮೂರನೆಯ ನರಸಿಂಹನ ಹೊಸಬೂದನೂರು ಶಾಸನದಲ್ಲಿ \textbf{“ಶ‍್ರೀಮದನಾದಿಯಗ್ರಹಾರಂ ಮಂಡೆಯದ ಮಹಾಜನಂಗಳು”\index{ಮಂಡೆಯದ ಮಹಾಜನ} ಮತ್ತು ''ಮಲ್ಲಿಕಾರ್ಜುನ ಪುರವಾದ\index{ಮಲ್ಲಿಕಾರ್ಜುನ ಪುರ} ಗುತ್ತಲ\index{ಗುತ್ತಲು} ಮಹಾಜನಂಗಳ”} ಉಲ್ಲೇಖವಿದೆ.\endnote{ ಎಕ 7 ಮಂ 56 ಹೊಸಬೂದನೂರು 1276} ಇದು ಮಂಡ್ಯ ಮತ್ತು ಅದರ ಪಕ್ಕದಲ್ಲೇ ಇರುವ ಗುತ್ತಲಿನ ಪ್ರಾಚೀನ ಶಾಸನೋಕ್ತ ಉಲ್ಲೇಖಗಳು. ಈ ಶಾಸನದಲ್ಲಿ ನಂಬಿಪಿಳ್ಳೆ, ಪುರುಷೋತ್ತಮದೇವ, ಎಂಬ ಶ‍್ರೀವೈಷ್ಣವರ\index{ಶ‍್ರೀವೈಷ್ಣವ} ಹೆಸರುಗಳ ಉಲ್ಲೇಖವಿದೆ. ಗುತ್ತಲು ಗ್ರಾಮವು ಯಾದವನಾರಾಯಣಪುರವಾದ\index{ಯಾದವನಾರಾಯಣಪುರ} ಗುತ್ತಲು\index{ಗುತ್ತಲು} ಎಂಬ ಶ‍್ರೀವೈಷ್ಣವ ಬ್ರಾಹ್ಮಣರ ಅಗ್ರಹಾರವಾಗಿತ್ತು. ಇದರಿಂದ ಕ್ರಿ.ಶ.1276ರ ಹೊತ್ತಿಗೇ ಮಂಡೆಯದ ಅಗ್ರಹಾರದಲ್ಲಿ ಶ‍್ರೀವೈಷ್ಣವರು ನೆಲೆಸಿದ್ದರೆಂದು ತಿಳಿದುಬರುತ್ತದೆ. ಗುತ್ತಲು ಇಂದು ಮಂಡ್ಯ ನಗರದಲ್ಲಿ ಸೇರಿಹೋಗಿದೆ. ಗುತ್ತಲು ಮಂಡ್ಯಕ್ಕಿಂತಲೂ ಪ್ರಾಚೀನವಾದ ಊರಾಗಿದ್ದು ಇಲ್ಲಿನ ಅರ್ಕೇಶ್ವರ ದೇವಾಲಯದಲ್ಲಿ ಗಂಗರ ಕಾಲದ ಶಾಸನಗಳು ದೊರಕಿವೆ.

ವಿಜಯನಗರದ ಪ್ರಖ್ಯಾತ ದೊರೆ ಕೃಷ್ಣದೇವರಾಯನು\index{ಕೃಷ್ಣದೇವರಾಯ} ಕ್ರಿ.ಶ.1516 ರಲ್ಲಿ ‘ಮಂಠೆಯ’\index{ಮಂಠೆಯ} ಗ್ರಾಮವನ್ನು\break “ಕೃಷ್ಣರಾಯಪುರ”\index{ಕೃಷ್ಣರಾಯಪುರ}ವೆಂಬ ಅಗ್ರಹಾರವನ್ನಾಗಿ ಮಾಡಿ ಶ‍್ರೀ ಗೋವಿಂದರಾಜಗುರುವಿಗೆ\index{ಗೋವಿಂದರಾಜಗುರು} ದತ್ತಿ ಹಾಕಿಕೊಟ್ಟನು.\endnote{ ಎಕ 7 ಮಂ 7 ಮಂಡ್ಯ 1516} ಮಂಠೆಯ ಅಗ್ರಹಾರಕ್ಕೆ ಕಲ್ಲಹಳ್ಳಿ,\index{ಕಲ್ಲಹಳ್ಳಿ } ಚಿಕ್ಕಮಂಠೆಯ\index{ಚಿಕ್ಕಮಂಠೆ}(ಚಿಕ್ಕಮಂಡ್ಯ\index{ಚಿಕ್ಕಮಂಡ್ಯ}) ಹೊಸಹಳ್ಳಿ\index{ಹೊಸಹಳ್ಳಿ} ಉಪಗ್ರಾಮಗಳಾಗಿದ್ದವು. \textbf{ಮಂಠೆಯವೇ ಇಂದಿನ ಮಂಡ್ಯ, ಚಿಕ್ಕಮಂಠೆಯವೇ ಇಂದಿನ ಚಿಕ್ಕಮಂಡ್ಯ.} ಕಲ್ಲಹಳ್ಳಿ\index{ಕಲ್ಲಹಳ್ಳಿ}, ಹೊಸಹಳ್ಳಿ\index{ಹೊಸಹಳ್ಳಿ}, ಇಂದಿಗೂ ಮಂಡ್ಯ ನಗರಕ್ಕೆ ಸೇರಿಹೋಗಿರುವ ಗ್ರಾಮಗಳೇ ಆಗಿ ಉಳಿದಿವೆ.

ಚಾಮರಾಜನಗರ ತಾಲ್ಲೂಕು ಯಳಂದೂರಿನ ಸು. ಕ್ರಿ.ಶ.1560ಕ್ಕೆ ಸೇರಿದ ಶಾಸನದಲ್ಲಿ “ಮಂಡ್ಯ\index{ಮಂಡ್ಯ}...ಗೋಪಣನ ಬೊರಹ” ಎಂದು ಇದೆ.\endnote{ ಎಕ 4 ಯಳಂದೂರು 162 ಮಾಂಬಳ್ಳಿ 1560} ಸು. 16-17ನೇ ಶತಮಾನದ ಯಳಂದೂರಿನ ಶಾಸನದಲ್ಲಿ “ಮಂಡ್ಯಂಣ ಶೆಟ್ಟಿ” ಎಂಬ ಹೆಸರಿದೆ.\endnote{ ಎಕ 4 ಯಳಂದೂರು 156 ಮಾಂಬಳ್ಳಿ 16-17ನೇ ಶ.} ಕ್ರಿ.ಶ.1663ರ ದೇವರಾಜ ಒಡೆಯರ\index{ದೇವರಾಜ ಒಡೆಯ} ತಲಕಾಡು\index{ತಲಕಾಡು} ತಾಮ್ರಶಾಸನದಲ್ಲಿ ಹಲ್ಲೆಗೆರೆ ಅಗ್ರಹಾರಕ್ಕೆ\index{ಹಲ್ಲೆಗೆರೆ ಅಗ್ರಹಾರ} “ಮಂಡೇವು”\index{ಮಂಡೇವು} ಗ್ರಾಮವನ್ನು ಸೀಮೆಯಾಗಿ ಹೇಳಿದೆ.\endnote{ ಎಕ 5 ತೀನಪು 218 ತಲಕಾಡು 1663} 40-50 ವರ್ಷಗಳ ಹಿಂದೆ ಈ ಭಾಗದ ಜನರು ಮಂಡ್ಯವನ್ನು ‘ಮಂಡೇವು’ ಎಂಬುದಾಗಿಯೇ ಉಚ್ಚರಿಸುತ್ತಿದರು. “ಇಲ್ಲೇ ಮಂಡೇವುಕೆ ಹೋಗಿದ್​ ಬರ್ತೀನಿ” ಎಂದು ಹೇಳುತ್ತಿದ್ದರು. ಇಂದಿಗೂ ಹಳೆಯ ತಲೆಮಾರಿನವರು ಮಂಡೇವು ಎಂದೇ ಉಚ್ಛರಿಸುತ್ತಾರೆ. ಮಂಡ್ಯ ತಾಲ್ಲೂಕಿನ\index{ಮಂಡ್ಯ ತಾಲ್ಲೂಕು} ಪ್ರಥಮ ಉಲ್ಲೇಖ ಕ್ರಿ.ಶ.1847ರ ಶಾಸನದಲ್ಲಿದೆ.\endnote{ ಎಕ 7 ಮಂ 1 ಮಂಡ್ಯ 1847}


\section*{ಪ್ರಾಕೃತಿಕವಾಗಿ ಮಂಡ್ಯ ಸ್ಥಳನಾಮ ನಿಷ್ಪತ್ತಿ}

ಮಂಡ್ಯ ಸ್ಥಳನಾಮವನ್ನು ಪ್ರಕೃತಿಜನ್ಯ ದೃಷ್ಟಿಯಿಂದ ವಿವೇಚಿಸಿದಲ್ಲಿ, ಕಲ್ಲುಮಂಟಿಗಳಿಂದ\index{ಕಲ್ಲುಮಂಟಿ} ಕೂಡಿದ ಜಾಗದಲ್ಲಿ ಕಟ್ಟಿರುವ ಊರೇ ಮಂಠೆಯ ಆಗಿರಬಹುದು. ಕೃಷ್ಣರಾಜಸಾಗರ ಜಲಾಶಯವು ನಿರ್ಮಾಣವಾಗಿ, ಅದರಿಂದ ಹೊರಟ ವಿಶ್ವೇಶ್ವರಯ್ಯ ನಾಲೆಯಿಂದ ಈ ಭಾಗವು ಸಮೃದ್ಧವಾಗುವುದಕ್ಕೆ ಮುಂಚೆ, ಮಂಡ್ಯ ಜಿಲ್ಲೆಯ ಬಹುತೇಕ ಭಾಗ ಬರಡುಭೂಮಿಯಾಗಿತ್ತು. ಮಂಡ್ಯ ನಗರವಿರುವ ಪ್ರದೇಶವು ಕಲ್ಲುಗಳಿಂದ ಕೂಡಿದ ದಿಣ್ಣೆಯ ಪ್ರದೇಶವಾಗಿತ್ತು. ಇಂತಹ ಪ್ರದೇಶವನ್ನು ಕಲ್ಲುಮಂಟಿ ಎನ್ನುತ್ತಾರೆ. “ಗಿರಿಯಲಲ್ಲದೆ ಕಲ್ಲುಮೊರಡಿಯೊಳಾಡುವುದೇ ನವಿಲು” ಎಂಬ ಅಕ್ಕನ ವಚನವನ್ನೂ, “ಮೊರಡಿಯೊಳ್​ ಮಾದಳ ಫಲಂ ಅರಸಿದಂತಾದುದು” ಎಂಬ ಜನ್ನನ ಕಾವ್ಯದ ಸಾಲನ್ನೂ ಗಮನಿಸಬಹುದು. ಮಂಠಿ, ಮಂಠೆ, ಮರಡಿ, ಮೊರಡಿ ಇವೆಲ್ಲಾ ಸಮಾನಾರ್ಥಕಗಳು. ಇದು ಈ ಪ್ರದೇಶದ ಭೂಗೋಳವನ್ನು ಸೂಚಿಸುತ್ತಿದ್ದು ಇದು ಕಲ್ಲು ಮಂಟಿಗಳಿಂದ ಆವೃತವಾದ ಊರಾಗಿದ್ದರಿಂದ ಇದಕ್ಕೆ ಮಂಠೆಯ ಎಂಬ ಹೆಸರು ಬಂದಿರುವುದು ಸ್ವಾಭಾವಿಕ. ಮೇಲೆ ಉಲ್ಲೇಖಿಸಿದ ಕೃಷ್ಣದೇವರಾಯನ ಮಂಡ್ಯ ತಾಮ್ರ ಶಾಸನದಲ್ಲೂ ಕೂಡಾ ಮಂಡ್ಯವನ್ನು ಮಂಠೆಯ ಎಂದೇ ಹೇಳಿದೆ. ಮಂಠೆದ, ಮಂಠೆಯ\index{ಮಂಠೆಯ} ಗ್ರಾಮಕ್ಕೆ ಎಲ್ಲೆಗಳನ್ನು ಹೇಳುವಾಗ ಚಿಕ್ಕಮಂಟೆಯಕ್ಕೆ ಈಶಾನ್ಯದ ಹೆಬ್ಬಳ್ಳದ ತಾಳತಿಟ್ಟು, ತಂಮಡಿಗಟ್ಟೆಯ ಕಾಲ್ವೆ ತಿಟ್ಟು, ಯಡಗೋಡಿಯ ಹಳ್ಳದ ತಿಟ್ಟು, ತೆಂಕಲು ಮೇಡು, ಪಡುವಲು ಸೀಪನಮರಡಿ, ಕಲ್ಲಹಳ್ಳಿಯ ತೆಂಕಣ ಓಕದಕಲ್ಲು, ಪಡುವಲು ಬೆಲದತಾಲ ಗುಡ್ಡದ ಕಲ್ಲುಗುಡ್ಡೆ, ಮೂಡಣ ಮೇಡು,\endnote{ ಎಕ 7 ಮಂ 7 ಮಂಡ್ಯ 1516} ಈ ರೀತಿಯಾಗಿ ಹೇಳಿದೆ. ತಿಟ್ಟು ಮೇಡು ಎಂಬುದು ಮಂಟಿ, ಮರಡಿಗೆ ಇನ್ನೊಂದು ಹೆಸರು.

ಈಗಲೂ ಬಸರಾಳಿನ ಸಮೀಪ ಬಳಪದಕಲ್ಲುಮಂಟಿ\index{ಬಳಪದಕಲ್ಲುಮಂಟಿ} ಎಂಬ ಹಳ್ಳಿ ಇದೆ. ಮಂಡ್ಯಕ್ಕೆ ಸಮೀಪದಲ್ಲಿ ಮರಡಿಪುರ\index{ಮರಡಿಪುರ}, ಕನಗನಮರಡಿ\index{ಕನಗನಮರಡಿ} ಮೊದಲಾದ ಊರುಗಳಿವೆ. ಜಿಲ್ಲೆಯ ಅನೇಕ ಹಳ್ಳಿಗಳ ಬಳಿ ಇರುವ ಕಲ್ಲುಬಂಡೆಗಳಿಂದ ಕೂಡಿದ ಸಣ್ಣ ಗುಡ್ಡಗಳಿರುವ ಜಾಗವನ್ನು, ಗುದ್ಲಿಕಲ್ಲುಮಂಠಿ, ಗೂಬೆಕಲ್ಲುಮಂಠಿ, ಕರಿಕಲ್ಲುಮಂಟಿ, ಬಿಳಿಕಲ್ಲುಮಂಠಿ ಎಂದು ಎಂದು ಬೇರೆ ಬೇರೆ ಹೆಸರುಗಳಿಂದ ಈಗಲೂ ಕರೆಯುತ್ತಾರೆ. ನೀರಾವರಿಗೆ ಒಳಪಡುವುದಕ್ಕೆ ಮುಂಚೆ ಈ ಭಾಗವು ಬರಗಾಲ ಪ್ರದೇಶವಾಗಿತ್ತು. ಈಗಲೂ ಕೆಲವು ಭಾಗವು ನೀರಿಲ್ಲದ ಬರಡು ಭೂಮಿಯಾಗಿದೆ. ಮಂಡ್ಯವನ್ನು ಯಾವಕಾರಣಕ್ಕೋ ಗೊತ್ತಿಲ್ಲ ‘ಕಾಗೆ ಮಂಡ್ಯ’ ಎನ್ನುತ್ತಿದ್ದರು. ಈ ಭಾಗದಲ್ಲಿ ಕಾಗೆಗಳು ವಿಪರೀತವಾಗಿದ್ದಂತೆ ತೋರುತ್ತದೆ.

ನಮ್ಮ ಹಳ್ಳಿಗಳ ಕಡೆಗೆ ಕಾವಿಧಾರಿಯಾಗಿ, ಹಣೆ ಹಾಗೂ ಮೈತುಂಬಾ ವಿಭೂತಿಯನ್ನು ಹಚ್ಚಿಕೊಂಡಿರುತ್ತಿದ್ದ, ವೀರಶೈವ ಜಂಗಮರು, ಅವರ ಹಿಂದೆ ಒಬ್ಬ ಶಿಷ್ಯನೂ ಕಾಲಜ್ಞಾನವನ್ನು ಹೇಳಿಕೊಂಡು ಬರುತ್ತಿದ್ದರು. ಇವರು ಮಂಡ್ಯ ಮಳವಳ್ಳಿ ಕಡೆಯಿಂದ ಬರುತ್ತಿದ್ದರೆಂದೂ ಆದುದರಿಂದ ಇವರನ್ನು ಮಂಠೇದಯ್ಯ\index{ಮಂಠೇದಯ್ಯ}- ಮಂಠೇದ ಅಯ್ಯನವರು\index{ಮಂಠೇದ ಅಯ್ಯ} ಎಂಬುದರ ಹ್ರಸ್ವರೂಪ- ಎಂದು ಕರೆಯುತ್ತಿದ್ದರು. ಮಂಡ್ಯ, ಮದ್ದೂರು, ಮಳವಳ್ಳಿಯ ಕಡೆ ಮೊದಲಿನಿಂದಲೂ ವೀರಶೈವಧರ್ಮ ಪ್ರಬಲವಾಗಿತ್ತು. ವೀರಶೈವಧರ್ಮಕ್ಕೆ ಹತ್ತಿರವಾದ ಶೈವ ಪರಂಪರೆಯ ಮಲೆಮಹದೇಶ್ವರ ಸಂಪ್ರದಾಯವೂ ವ್ಯಾಪಕವಾಗಿ ಇಲ್ಲಿ ಹರಡಿದೆ. ಜೊತೆಗೆ ರಾಚಪ್ಪಾಜೀ ಸಿದ್ಧಪ್ಪಾಜೀ ಪರಂಪರೆಯೂ ಈ ಭಾಗದಲ್ಲಿ ವಿಶೇಷವಾಗಿದೆ. ವೀರಶೈವ ಧರ್ಮದ ಸಿದ್ಧಪುರುಷರುಗಳು, ಪ್ರಚಾರಕರು ಮತ್ತು ಪುರೋಹಿತರಿಗೆ ಅಯ್ಯನವರು ಎಂಬ ಹೆಸರು. ಇವರು ತಮ್ಮ ಶಿಷ್ಯರೊಡಗೂಡಿ ಹಳ್ಳಿಹಳ್ಳಿಗಳಲ್ಲಿ ತಿರುಗಾಡುತ್ತಾ, ಕಾಲಜ್ಞಾನ, ಶಿವನ ಮಹಿಮೆ ಮೊದಲಾದವುಗಳನ್ನು ಹೇಳಿಕೊಂಡು ಧರ್ಮ ಪ್ರಚಾರ ಮಾಡುತ್ತಿದ್ದರು. ಇವರನ್ನು ಮಂಠೇದಯ್ಯನವರು ಎಂದು ಕರೆಯುವುದು ವಾಡಿಕೆ. ಮಂಠೆಯದ ಅಯ್ಯನವರು ಅಂದರೆ ಮಂಡ್ಯ ಮಳವಳ್ಳಿ ಕಡೆಯವರು ಎಂಬ ಕಲ್ಪನೆ ಮೊದಲಿನಿಂದಲೂ ಇದ್ದಿತು. ಮೇಲಿನ ವಿಶ್ಲೇಷಣೆಗಳಿಂದ ಮಂಡ್ಯದ ಹೆಸರಿನ ನಿಷ್ಪತ್ತಿಯನ್ನು ಮಂಟಿ\textgreater  ಮಂಠಿ\textgreater  ಮಂಠೆ\textgreater  ಮಂಠೆಯ\textgreater  ಮಂಡೆಯ\textgreater  ಮಂಡೇವು\textgreater ಮಂಡ್ಯ ಎಂದು ಅರ್ಥಪೂರ್ಣವಾಗಿ ನಿಷ್ಪತ್ತಿಗೊಳಿಸಬಹುದು. ಮಂಡೆಯಂ ಅಗ್ರಹಾರದಿಂದ ಬಂದು ನೆಲೆಸಿದವರಿಂದ ಮಂಡ್ಯ ಎಂಬ ಹೆಸರು ಬಂದಿದೆ ಎಂಬುದು ಸಂಸ್ಕೃತೀಕರಣದ ರೂಪವೇ ಹೊರತು ಮೂಲರೂಪವಾಗಿರಲಾರದು.


\section*{ಮಂಡ್ಯ ಜಿಲ್ಲೆಯ ರಚನೆಯ ಹಿನ್ನೆಲೆ}

ಕ್ರಿ.ಶ.1799 ರಲ್ಲಿ ನಡೆದ ನಾಲ್ಕನೆಯ ಮೈಸೂರು ಯುದ್ಧದಲ್ಲಿ ಟಿಪ್ಪೂ\index{ಟಿಪ್ಪೂ } ಪತನದ ನಂತರ ಕರ್ನಾಟಕ ರಾಜ್ಯವನ್ನು ನಾಲ್ಕು ಭಾಗಗಳಾಗಿ ವಿಂಗಡಿಸಲಾಯಿತು. ಉತ್ತರಭಾಗವನ್ನು ಮರಾಠರಿಗೆ, ಈಶಾನ್ಯ ಭಾಗವನ್ನು ಹೈದರಾಬಾದಿನ ನೈಜಾಮನಿಗೆ ನೀಡಲಾಯಿತು. ಕೆನರಾ ಕಡಲ ತೀರಪ್ರದೇಶ ಮತ್ತು ಕೊಡಗನ್ನು ಬ್ರಿಟಿಷರು ತಾವೇ ಇಟ್ಟುಕೊಂಡರು. ಶ‍್ರೀರಂಗಪಟ್ಟಣವನ್ನು\index{ಶ‍್ರೀರಂಗಪಟ್ಟಣ} ಕೇಂದ್ರವನ್ನಾಗಿ ಹೊಂದಿದ್ದ, ಮೈಸೂರು, ಬೆಂಗಳೂರು, ಕೋಲಾರ, ತುಮಕೂರು, ಹಾಸನ, ಕಡೂರು, ಶಿವಮೊಗ್ಗ ಮತ್ತು ಚಿತ್ರದುರ್ಗ ಈ ಎಂಟು ಜಿಲ್ಲೆಗಳ ಮೈಸೂರು ಸಂಸ್ಥಾನವನ್ನು\index{ಮೈಸೂರು ಸಂಸ್ಥಾನ} ಮೈಸೂರು ಒಡೆಯರಿಗೆ ನೀಡಲಾಯಿತು. ಸಂಸ್ಥಾನದ ರಾಜಧಾನಿಯನ್ನು 30.6.1799ರಲ್ಲಿ ಶ‍್ರೀರಂಗಪಟ್ಟಣದಿಂದ\index{ಶ‍್ರೀರಂಗಪಟ್ಟಣ} ಮೈಸೂರಿಗೆ ಸ್ಥಳಾಂತರಿಸಲಾಯಿತು. ಶ‍್ರೀರಂಗಪಟ್ಟಣ ಕೋಟೆಯನ್ನು ಕಂಪನಿಯವರು ಬ್ರಿಟೀಷ್​ ಸೇನೆಯ ವಶಕ್ಕೆ ನೀಡಿದರು. ಇದನ್ನು ಸಹಾಯಕ ಸೇನಾದಳದ ಕೇಂದ್ರವಾಗಿ ಮಾಡಲಾಯಿತು. ಆರ್ಥರ್​ ವೆಲ್ಲೆಸ್ಲಿಯು 1805ರವರೆಗೆ ಇಲ್ಲಿನ ಸೇನೆಯ ಅಧಿಪತಿಯಾಗಿದ್ದು, ದರಿಯಾದೌಲತ್​ನಲ್ಲಿ\index{ದರಿಯಾದೌಲತ್​} ವಾಸಮಾಡುತ್ತಿದ್ದನು. ಬುಕ್​ನಾನ್\index{ಬುಕ್​ನಾನ್}​ ವರದಿಯಂತೆ ಶ‍್ರೀರಂಗಪಟ್ಟಣದ ಜನಸಂಖ್ಯೆಯು 1,50,00 ಇದ್ದುದು ಅದರ ಪತನ ಹಾಗೂ ರಾಜಧಾನಿಯ ಸ್ಥಳಾಂತರದ ನಂತರ 32,000ಕ್ಕೆ ಇಳಿಯಿತೆಂದು ತಿಳಿದುಬರುತ್ತದೆ.

\newpage

ಟಿಪ್ಪು ಸುಲ್ತಾನನು\index{ಟಿಪ್ಪು ಸುಲ್ತಾನ} ತನ್ನ ರಾಜ್ಯಭಾರದ ಕಾಲದಲ್ಲಿ ಬ್ರಿಟಿಷರನ್ನು ಬಗ್ಗು ಬಡಿಯಬೇಕೆಂಬ ಉದ್ದೇಶದಿಂದ ಫ್ರೆಂಚರ ಸಖ್ಯವನ್ನು ಬೆಳೆಸಿದ್ದನಷ್ಟೆ. ಆಗ ಫ್ರೆಂಚರು ಅವನಿಗೆ ಸಹಾಯ ಮಾಡಲು ಒಂದು ಸೇನಾ ತುಕಡಿಯನ್ನು ಹಿರೋಡೆಯ\index{ಹಿರೋಡೆ} ಬಳಿಯ ಕುಂತಿ ಬೆಟ್ಟದ ಬಳಿ ಇರಿಸಿದ್ದರು. ಅದರಿಂದಾಗಿ ಕುಂತಿ ಬೆಟ್ಟಕ್ಕೆ ಫ್ರೆಂಚ್​ರಾಕ್ಸ್​ ಎಂಬ ಹೆಸರು ಬಂದಿತು. ಟಿಪ್ಪುವಿನ ಪತನಾ ನಂತರವೂ ಬ್ರಿಟಿಷರು, ಶ‍್ರೀರಂಗಪಟ್ಟಣದಲ್ಲಿ ಮತ್ತು ಫ್ರೆಂಚ್​ರಾಕ್ಸ್​ನಲ್ಲಿ\index{ಫ್ರೆಂಚ್​ರಾಕ್ಸ್} ಸೇನೆಯ ಒಂದೊಂದು ತುಕಡಿಯನ್ನು ಇರಿಸಿದ್ದರು. ಇದಕ್ಕೆ ಸಾಕ್ಷಿಯಾಗಿ ಫ್ರೆಂಚ್​ರಾಕ್ಸ್​ನಲ್ಲಿದ್ದ ಬ್ರಿಟಿಷ್​ ಸೇನಾ ರೆಜಿಮೆಂಟ್​ ಅಥವಾ ತುಕಡಿಯಲ್ಲಿದ್ದು ಮಡಿದ ಯೋಧರುಗಳ ಸಮಾಧಿಗಳು ಪಾಂಡವಪುರದಲ್ಲಿವೆ.\endnote{ ಎಕ 6 ಪಾಂಪು 1 ರಿಂದ 10 ಪಾಂಡವಪುರ} ಶ‍್ರೀರಂಗಪಟ್ಟಣದ ಪ್ರತಿಕೂಲ ವಾತಾವರಣವು ಸೇನೆಗೆ ಹೊಂದಿಕೆಯಾಗದಿರಲು, ಸೇನಾನೆಲೆಯನ್ನು ಬೆಂಗಳೂರು ದಂಡು\index{ಬೆಂಗಳೂರು ದಂಡು} ಪ್ರದೇಶಕ್ಕೆ ಸ್ಥಳಾಂತರಿಸಲಾಯಿತು.

1831 ರಿಂದ 1881ರವರೆಗೆ ಮೈಸೂರು ಸಂಸ್ಥಾನದಲ್ಲಿ ಕಮೀಷನರ್​ಗಳ ಆಡಳಿತ ನಡೆಯಿತು. 1881ರಲ್ಲಿ ಬ್ರಿಟಿಷರಿಗೂ, ಮೈಸೂರು ರಾಜಮನೆತನದವರಿಗೂ ಒಪ್ಪಂದವಾಗಿ (ರೆಂಡಿಷನ್​\index{ರೆಂಡಿಷನ್​}) ಮೈಸೂರು ಒಡೆಯರಿಗೆ ಪುನಃ ಸಂಸ್ಥಾನದ ಅಧಿಕಾರವನ್ನು ಹಸ್ತಾಂತರಿಸಲಾಯಿತು. ಈಗಿನ ಮಂಡ್ಯ ಜಿಲ್ಲೆಯ ಭಾಗವು 1862ರಲ್ಲಿ ಮೈಸೂರು ಮತ್ತು ಹಾಸನ ಜಿಲ್ಲೆಗಳ ಪ್ರದೇಶಗಳನ್ನು ಒಳಗೊಂಡ ಅಷ್ಟಗ್ರಾಮ ಫೌಜ್​ದಾರಿಯ\index{ಅಷ್ಟಗ್ರಾಮ}, ಅಷ್ಟಗ್ರಾಮ ಡಿವಿಜನ್​ನಲ್ಲಿ\index{ಅಷ್ಟಗ್ರಾಮ} ಸೇರಿತ್ತು. 1869ರಲ್ಲಿ ಮೈಸೂರು ಜಿಲ್ಲೆಯಲ್ಲಿ 14 ತಾಲ್ಲೂಕುಗಳಿದ್ದವು. ಅವುಗಳಲ್ಲಿ ಪಟ್ಟಣ ಅಷ್ಟಗ್ರಾಮ\index{ಅಷ್ಟಗ್ರಾಮ}, ತಲಕಾಡು, ಮಂಡ್ಯ, ಮದ್ದೂರು, ಮಳವಳ್ಳಿ, ತಾಲ್ಲೂಕುಗಳಲ್ಲಿ ಇಂದಿನ ಮಂಡ್ಯ ಜಿಲ್ಲೆಯ ಪ್ರದೇಶವು ಅಂತರ್ಗತವಾಗಿತ್ತು. 1882 ರಲ್ಲಿ ಹಾಸನ ಜಿಲ್ಲೆಯನ್ನು ರದ್ದುಪಡಿಸಿ, ಹಾಸನ ಜಿಲ್ಲೆಯ ಅರಕಲಗೂಡು, ಚನ್ನರಾಯಪಟ್ಟಣ, ನಾಗಮಂಗಲ, ಅತ್ತಿಕುಪ್ಪೆ (ಕೃಷ್ಣರಾಜಪೇಟೆ) ತಾಲ್ಲೂಕುಗಳನ್ನು ಮೈಸೂರು ಜಿಲ್ಲೆಗೆ ಸೇರಿಸಲಾಯಿತು. ಪಟ್ಟಣ ಅಷ್ಟಗ್ರಾಮದ ಹೆಸರನ್ನು ಶ‍್ರೀರಂಗಪಟ್ಟಣ\index{ಶ‍್ರೀರಂಗಪಟ್ಟಣ} ಎಂದು ಬದಲಾಯಿಸಲಾಯಿತು. ಶ‍್ರೀರಂಗಪಟ್ಟಣ ದ್ವೀಪವನ್ನು ಮೈಸೂರು ಜಿಲ್ಲೆಗೆ ಸೇರಿಸಲಾಯಿತು. ಕಿಕ್ಕೇರಿ ಹೋಬಳಿಗೆ\index{ಕಿಕ್ಕೇರಿ ಹೋಬಳಿ} ಚನ್ನರಾಯಪಟ್ಟಣ\index{ಚನ್ನರಾಯಪಟ್ಟಣ} ಮತ್ತು ಹೊಳೆನರಸಿಪುರ\index{ಹೊಳೆನರಸಿಪುರ} ತಾಲ್ಲೂಕುಗಳ ಕೆಲವು ಹಳ್ಳಿಗಳನ್ನು ವರ್ಗಾಯಿಸಲಾಯಿತು. ಮೇಲುಕೋಟೆ ಹೋಬಳಿಯನ್ನು\index{ಮೇಲುಕೋಟೆ ಹೋಬಳಿ} ಶ‍್ರೀರಂಗಪಟ್ಟಣ ತಾಲ್ಲೂಕಿಗೂ, ಸಂತೇಬಾಚಹಳ್ಳಿ ಹೋಬಳಿಯ\index{ಸಂತೇಬಾಚಹಳ್ಳಿ ಹೋಬಳಿ} ಕೆಲವು ಹಳ್ಳಿಗಳನ್ನು ನಾಗಮಂಗಲ ತಾಲ್ಲೂಕಿಗೂ ಸೇರಿಸಲಾಯಿತು. 1875ರಲ್ಲಿ ಮದ್ದೂರನ್ನು ಒಂದು ಉಪ ತಾಲ್ಲೂಕನ್ನಾಗಿ ಮಾಡಲಾಯಿತಾದರೂ, 1886ರಲ್ಲಿ ಅದನ್ನು ಮಂಡ್ಯ ತಾಲ್ಲೂಕಿನಲ್ಲಿ ವಿಲೀನ\break ಗೊಳಿಸಲಾಯಿತು. ಮತ್ತೆ 1931ರ ಮೇ 1 ರಂದು ಮದ್ದೂರು ತಾಲ್ಲೂಕನ್ನು ರಚಿಸಲಾಯಿತು.

1882ರಲ್ಲಿ ಮೈಸೂರು ಜಿಲ್ಲೆಯ ಪಟ್ಟಣ ಅಷ್ಟಗ್ರಾಮ ಡಿವಿಜನ್​ಗೆ ಸೇರಿದ್ದ, ಮೇಲ್ಕಂಡ ಐದು ತಾಲ್ಲೂಕುಗಳು ಮತ್ತು ಒಂದು ಸಬ್​ತಾಲ್ಲೂಕು ಇದ್ದ, ಫ್ರೆಂಚ್​ರಾಕ್ಸ್​(ಹಿರೋಡೆ) ಸಬ್​ಡಿವಿಜನ್​ನ್ನು ರಚಿಸಲಾಯಿತು. 1886 ರಲ್ಲಿ ಹಾಸನ ಜಿಲ್ಲೆಯನ್ನು ಪುನರ್ ರಚಿಸಲಾಯಿತು. ಚನ್ನರಾಯಪಟ್ಟಣ ತಾಲ್ಲೂಕನ್ನು ಹಾಸನ ಜಿಲ್ಲೆಗೆ ವರ್ಗಾಯಿಸಲಾಯಿತು. ನಾಗಮಂಗಲ ಮತ್ತು ಅತ್ತಿಕುಪ್ಪೆ ತಾಲ್ಲೂಕುಗಳನ್ನು ಮೈಸೂರಿನಲ್ಲಿ ಉಳಿಸಿಕೊಳ್ಳಲಾಯಿತು. 1889ರ ನಂತರ ಫ್ರೆಂಚ್​ರಾಕ್ಸ್​ ಉಪತಾಲ್ಲೂಕನ್ನು ರಚಿಸಲಾಯಿತು. 1923ರಲ್ಲಿ ಅದನ್ನು ರದ್ದುಗೊಳಿಸಿ, ಕ್ಯಾತನಹಳ್ಳಿ, ಪಾಂಡವಪುರ ಮತ್ತು ಕಸಬ ಹೋಬಳಿಗಳ ಜೊತೆಗೆ ಮೇಲುಕೋಟೆ ಉಪ ತಾಲ್ಲೂಕನ್ನೂ ಶ‍್ರೀರಂಗಪಟ್ಟಣ ತಾಲ್ಲೂಕಿಗೆ ವರ್ಗಾಯಿಸಲಾಯಿತು. 1928ರಲ್ಲಿ ಮಂಡ್ಯ, ಮಳವಳ್ಳಿ ಮತ್ತು ಮದ್ದೂರು ತಾಲ್ಲೂಕುಗಳನ್ನು ಒಳಗೊಂಡ ಮಂಡ್ಯ ಸಬ್​ಡಿವಿಜನ್​ನ್ನು ರಚಿಸಲಾಯಿತು. ಅತ್ತಿಗುಪ್ಪೆಗೆ 1930 ರಲ್ಲಿ ಕೃಷ್ಣರಾಜಪೇಟೆ ಎಂದು ನಾಮಕರಣ ಮಾಡಲಾಯಿತು.

1930ರಲ್ಲಿ ಮೈಸೂರು ಜಿಲ್ಲೆಯಲ್ಲಿ ನಂಜನಗೂಡು, ಮೈಸೂರು ಮತ್ತು ಫ್ರೆಂಚ್​ರಾಕ್ಸ್​ ಎಂಬ ಉಪವಿಭಾಗಗಳಿದ್ದು, ಒಟ್ಟು 13 ತಾಲ್ಲೂಕುಗಳು ಮತ್ತು ಒಂದು ಉಪ ತಾಲ್ಲೂಕು ಇದ್ದಿತು. ಅದರಲ್ಲಿ ಫ್ರೆಂಚ್​ರಾಕ್ಸ್​ ಉಪವಿಭಾಗಕ್ಕೆ ಸೇರಿದಂತೆ, ಫ್ರೆಂಚ್​ರಾಕ್ಸ್​, ಶ‍್ರೀರಂಗಪಟ್ಟಣ, ನಾಗಮಂಗಲ, ಕೃಷ್ಣರಾಜಪೇಟೆ ತಾಲ್ಲೂಕುಗಳು, 1928ರಲ್ಲಿ ರಚಿತವಾಗಿದ್ದ ಮಂಡ್ಯ ಉಪವಿಭಾಗದಲ್ಲಿ ಮಂಡ್ಯ, ಮಳವಳ್ಳಿ, ಮದ್ದೂರು ತಾಲ್ಲೂಕುಗಳೂ ಇದ್ದವು. ಆಡಳಿತದ ಅನುಕೂಲಕ್ಕೋಸ್ಕರ ಫ್ರೆಂಚ್​ರಾಕ್ಸ್​ ಸಬ್​ಡಿವಿಜನ್​\index{ಫ್ರೆಂಚ್​ರಾಕ್ಸ್​ ಸಬ್​ಡಿವಿಜನ್​} ಮತ್ತು ಮಂಡ್ಯ ಸಬ್​ಡಿವಿಜನ್​ಗಳನ್ನು\index{ಮಂಡ್ಯ ಸಬ್​ಡಿವಿಜನ್​}, ಮೈಸೂರು ಜಿಲ್ಲೆಯಿಂದ ಬೇರ್ಪಡಿಸಿ 1939ರಲ್ಲಿ ಮಂಡ್ಯ ಜಿಲ್ಲೆಯನ್ನು ರಚಿಸಲಾಯಿತು.

ಮೊದಲಿಗೆ ಫ್ರೆಂಚರ ಮತ್ತು ಟಿಪ್ಪು ಪತನಾ ನಂತರ ಬ್ರಿಟಿಷರ ಸೇನಾ ಠಾಣ್ಯವಿದ್ದ ಬೆಟ್ಟಗಳ (ಫ್ರೆಂಚ್​ರಾಕ್ಸ್​ ಅಂದರೆ ಕುಂತಿಬೆಟ್ಟ) ಸಮೀಪ ಇದ್ದ ಒಂದು ಹಳ್ಳಿ ಹಿರೋಡೆ. ಮೈಸೂರು ಬೆಂಗಳೂರು ರೈಲ್ವೆ ಮಾರ್ಗ ಸ್ಥಾಪನೆಯ ನಂತರ, ಹಿರೋಡೆ ಸಮೀಪ ಒಂದು ರೈಲ್ವೆ ಸ್ಟೇಷನ್​ ಸ್ಥಾಪನೆಯಾದ ಮೇಲೆ ಇದು ಪ್ರವರ್ಧಮಾನಕ್ಕೆ ಬಂದಿತು. ಕುಂತಿಬೆಟ್ಟದಲ್ಲಿ ಪಾಂಡವರು ನೆಲೆಸಿದ್ದರೆಂಬ ಐತಿಹ್ಯದ ಕಾರಣ, ಹಿರೋಡೆಯನ್ನು ಪಾಂಡವಪುರ ಎಂದು ಕರೆಯಲಾಯಿತು. ಪಾಂಡವಪುರ\index{ಪಾಂಡವಪುರ} ಅಂದರೆ ಹಿರೋಡೆಗೆ ಹೊಂದಿಕೊಂಡ ಹಾಗೆ ಚಿಕ್ಕಾಡೆ\index{ಚಿಕ್ಕಾಡೆ} ಎಂಬ ಒಂದು ಹಳ್ಳಿ ಈಗಲೂ ಇದೆ. ಇದರ ಪ್ರಾಚೀನ ರೂಪ ಚಿಕ್ಕ ವೋಡೆ\index{ಚಿಕ್ಕ ವೋಡೆ}. ವೋಡೆ ಎಂಬುದು ಧಾನ್ಯದ ಅಳತೆಯ ಒಂದು ಮಾನವಾಗಿದ್ದು, ಹಿರಿವೋಡೆ\index{ಹಿರಿವೋಡೆ}, ಚಿಕ್ಕವೋಡೆ ಇವೆರಡೂ ಧಾನ್ಯದ ಅಳತೆಯ ಅಥವಾ\break ಶೇಖರಣೆಯ ಮಾನಗಳಿಂದ ಬಂದ ಹೆಸರುಗಳಾಗಿವೆ.


\section*{ಮಂಡ್ಯ ಜಿಲ್ಲೆಗೆ ಸಂಬಂಧಿಸಿದ ಕೈಫಿಯತ್ತುಗಳು}

ಮಂಡ್ಯ ಜಿಲ್ಲೆಯ ಬಹಳ ಮುಖ್ಯವಾದ ಊರುಗಳಿಗೆ ಸಂಬಂಧಿಸಿದಂತೆ ಅನೇಕ ಕೈಫಿಯತ್ತುಗಳು\index{ಕೈಫಿಯತ್ತು} ದೊರಕಿವೆ. ಡಾ.ಎಂ.ಎಂ. ಕಲಬುರ್ಗಿಯವರು ಮಂಡ್ಯ ಜಿಲ್ಲೆಗೆ ಸಂಬಂಧಿಸಿದಂತಹ ಸ್ಥಳಪುರಾಣ, ಇತಿಹಾಸ ಮತ್ತು ಸಂಸ್ಕೃತಿಯ ವಿಚಾರಗಳನ್ನುಳ್ಳ ಈ ಕೆಳಕಂಡ ಕೈಫಿಯತ್ತುಗಳನ್ನು, ಮೆಕೆಂಝಿ ಸಂಪುಟಗಳಿಂದ ಸಂಗ್ರಹಿಸಿ ಕೊಟ್ಟಿದ್ದಾರೆ.\endnote{ ಕಲಬುರ್ಗಿ, ಎಂ.ಎಂ., ಕರ್ನಾಟಕದ ಕೈಫಿಯತ್ತುಗಳು, ಪ್ರಸ್ತಾವನೆ, ಪುಟ \engfoot{xvii – xxv}} ಅಂಕನಾಥಪುರದ ಅರ್ಕೇಶ್ವರಸ್ವಾಮಿ ಕೈಫಿಯತ್ತು, ಅಷ್ಟಗ್ರಾಮಗಳ ಕೈಫಿಯತ್ತು, ಅಕ್ಕಿಯೆಬ್ಬಾಳು ಕೈಫಿಯತ್ತು, ಅತ್ತಿಕುಪ್ಪೆ ಗ್ರಾಮದ ಕೈಫಿಯತ್ತು, ತಳಕಾಡು\break ಕೈಫಿಯತ್ತು-1, ತಳಕಾಡು ಕೈಫಿಯತ್ತು-2, ತೊಣ್ಣೂರು ರಾಕ್ಷಸಿ ಕೈಫಿಯತ್ತು. ನಾಗಮಂಗಲದ ಕೈಫಿಯತ್ತು, ಮಿರ್ಲೆ ಕೈಫಿಯತ್ತು,\break ಶ‍್ರೀ ಎಡತೊರೆಮಠದ ಕೈಫಿಯತ್ತು, ಸಾಲಿಗ್ರಾಮದ ಜೈನಬಸ್ತಿಯ ಕೈಫಿಯತ್ತು, ಹೈದರನ ಕೈಫಿಯತ್ತು-1, ಹೈದರನ\break ಕೈಫಿಯತ್ತು-2. ಈ ಕೈಫಿಯತ್ತುಗಳಲ್ಲಿ ಇಂದಿನ ಮಂಡ್ಯ ಜಿಲ್ಲೆಗೆ ಸಂಬಂಧಿಸಿದ ಅನೇಕ ಊರುಗಳ, ಸ್ಥಳಪುರಾಣ, ರಾಜಕೀಯ ಇತಿಹಾಸ, ದೇವರು, ಹಬ್ಬ ಹರಿದಿನ, ಜಾತ್ರೆ ಮೊದಲಾದ ಸಾಂಸ್ಕೃತಿಕ ವಿಷಯಗಳು ನಿರೂಪಿತವಾಗಿದ್ದು, ಇವುಗಳನ್ನು ಆಳವಾಗಿ ನಾನಾ ದೃಷ್ಟಿಕೋನಗಳಿಂದ ಅಧ್ಯಯನ ಮಾಡಿದಲ್ಲಿ, ಜಿಲ್ಲೆಗೆ ಸಂಬಂಧಿಸಿದಂತೆ ನೈಜವಾದ ಮತ್ತು ಗಟ್ಟಿಯಾದ ಅನೇಕ ವಿಷಯಗಳು ತಿಳಿದುಬರುತ್ತವೆ. ಇದು ಶಾಸನಗಳ ಅಧ್ಯಯನಕ್ಕೆ ಹೊರತಾದ ವಿಷಯವಾಗಿದೆ.

\section*{ಮಂಡ್ಯ ಜಿಲ್ಲೆಯಲ್ಲಿರುವ ಶಾಸನಗಳ ಪ್ರಕಟಣೆ ಹಾಗೂ ಅಧ್ಯಯನದ ಆರಂಭ}

ಶಾಸನಗಳ ಅಧ್ಯಯನದ ಇತಿಹಾಸದಲ್ಲಿ, ಶಾಸನಗಳ ಸಂಗ್ರಹಣೆ, ಪ್ರಕಟಣೆ, ವಿಶ್ಲೇಷಣೆ ಮತ್ತು ಅಧ್ಯಯನಗಳು, ಪ್ರಾದೇಶಿಕ ಮಿತಿಯಲ್ಲಿಯೇ ಆರಂಭವಾಯಿತೆನ್ನಬಹುದು. ಕರ್ನಾಟಕ ಶಾಸನ ಪಿತಾಮಹರೆನಿಸಿಕೊಂಡ ಬಿ.ಎಲ್​.ರೈಸ್​\index{ಬಿ.ಎಲ್​.ರೈಸ್​} ಅವರು ಅಂದಿನ ಹಳೆಯ ಮೈಸೂರು ಸಂಸ್ಥಾನದ ಆಡಳಿತ ಘಟಕಗಳಾಗಿದ್ದ ಜಿಲ್ಲೆಗಳನ್ನು ಆಧಾರವಾಗಿಟ್ಟುಕೊಂಡು, ಅಲ್ಲಿದ್ದ ಶಾಸನಗಳನ್ನು ಸಂಗ್ರಹಿಸಿ, ಜಿಲ್ಲಾವಾರು ಶಾಸನ ಸಂಪುಟಗಳನ್ನು ಪ್ರಕಟಿಸಿದರು. ಈ ಜಿಲ್ಲಾವಾರು ಶಾಸನ ಸಂಪುಟಗಳಿಗೆ, ಮುಖ್ಯವಾಗಿ ಆ ಜಿಲ್ಲೆಯ ಶಾಸನಗಳಲ್ಲಿ ಕಂಡುಬರುವ ರಾಜವಂಶದ ಇತಿಹಾಸವನ್ನು, ಸ್ಥೂಲವಾಗಿ ಸಾಂಸ್ಕೃತಿಕ ಇತಿಹಾಸಗಳನ್ನು ಒಳಗೊಂಡ ವಿಸ್ತಾರವಾದ ಪೀಠಿಕೆಯನ್ನು ಆಂಗ್ಲಭಾಷೆಯಲ್ಲಿ ರಚಿಸಿದರು. ಬಿ.ಎಲ್​.ರೈಸ್​ ಅವರು ಎಪಿಗ್ರಾಫಿಯಾ ಕರ್ನಾಟಿಕಾ ಸಂಪುಟಗಳನ್ನು ಸಂಪಾದಿಸಿ ಪ್ರಕಟಿಸುವ ಕಾಲದಲ್ಲಿ, ಮಂಡ್ಯ ಜಿಲ್ಲೆಯ ರಚನೆಯಾಗಿರಲಿಲ್ಲ. ಅದು ಮೈಸೂರು ಜಿಲ್ಲೆಯ ಒಂದು ಭಾಗವಾಗಿತ್ತು. ಬಿ.ಎಲ್​. ರೈಸ್​ ಅವರು ಸಂಪಾದಿಸಿದ ಎಪಿಗ್ರಾಫಿಯಾ ಕರ್ನಾಟಿಕಾ\index{ಎಪಿಗ್ರಾಫಿಯಾ ಕರ್ನಾಟಿಕಾ} ಸಂಪುಟ-3 ಭಾಗ 1, ಮತ್ತು ಸಂಪುಟ-4, ಭಾಗ 2 ರಲ್ಲಿ ಮೈಸೂರು ಜಿಲ್ಲೆಯ ಶಾಸನಗಳು ಪ್ರಕಟವಾದವು. 1894ರಲ್ಲಿ ಪ್ರಕಟವಾದ, ಸಂಪುಟ-3, ಭಾಗ-1 ರಲ್ಲಿ, ಮೈಸೂರು ಜಿಲ್ಲೆಯ ಪೂರ್ವದ ತಾಲ್ಲೂಕುಗಳಾದ ಮೈಸೂರು, ಶ‍್ರೀರಂಗಪಟ್ಟಣ, ಮಂಡ್ಯ, ಮಳವಳ್ಳಿ, ತಿರುಮಕೂಡಲು ನರಸೀಪುರ, ಹಾಗೂ ನಂಜನಗೂಡಿನ 803 ಶಾಸನಗಳಿವೆ. 1898ರಲ್ಲಿ ಪ್ರಕಟವಾದ ಸಂಪುಟ-4, ಭಾಗ-2 ರಲ್ಲಿ ಮೈಸೂರು ಜಿಲ್ಲೆಯ ಪಶ್ಚಿಮ ತಾಲ್ಲೂಕುಗಳಾದ ಚಾಮರಾಜನಗರ, ಎಳಂದೂರು, ಗುಂಡ್ಲುಪೇಟೆ, ಹೆಗ್ಗಡದೇವನಕೋಟೆ, ಹುಣಸೂರು, ಕೃಷ್ಣರಾಜಪೇಟೆ ಮತ್ತು ನಾಗಮಂಗಲ ತಾಲ್ಲೂಕುಗಳಲ್ಲಿದ್ದ 962 ಶಾಸನಗಳು ಸೇರಿವೆ.

ಮೈಸೂರು ವಿಶ್ವವಿದ್ಯಾನಿಲದಯ ಕುವೆಂಪು ಕನ್ನಡ ಅಧ್ಯಯನ ಸಂಸ್ಥೆಯು\index{ಕುವೆಂಪು ಕನ್ನಡ ಅಧ್ಯಯನ ಸಂಸ್ಥೆ}, ಶಾಸನ ತಜ್ಞರು ಮತ್ತು ವಿದ್ವನ್ಮಂಡಲಿಯ ನೆರವಿನಿಂದ, ರೈಸ್​ರವರು ಸಂಪಾದಿಸಿದ್ದ ಎಪಿಗ್ರಾಫಿಯಾ ಕರ್ನಾಟಿಕಾ ಸಂಪಟುಗಳ ಪರಿಷ್ಕೃತ ಸಂಪುಟಗಳ ಪ್ರಕಟಣಾ ಕಾರ್ಯವನ್ನು ಯಶಸ್ವಿಯಾಗಿ ಮಾಡುತ್ತಿದೆ. ರೈಸ್​ರವರ\index{ರೈಸ್} ಕಾಲದಲ್ಲಿದ್ದ ಜಿಲ್ಲೆಯ ಸ್ವರೂಪವನ್ನು, ಆಧುನಿಕ ಆಡಳಿತ ಕಾಲದಲ್ಲಿ ರಚಿತವಾದ ಜಿಲ್ಲೆಗಳಿಗೆ ಬದಲಾಯಿಸಿಕೊಂಡು, ಶಾಸನಗಳನ್ನು ಹಂಚಿಕೆ ಮಾಡಿ ಪರಿಷ್ಕೃತ ಸಂಪುಟಗಳನ್ನು ಪ್ರಕಟಿಸುತ್ತಿದೆ. ಮೈಸೂರು ಜಿಲ್ಲೆಯು ಮೈಸೂರು, ಮಂಡ್ಯ ಜಿಲ್ಲೆಗಳಾಗಿ ವಿಭಜನೆಯಾಗಿದ್ದರಿಂದ, ಮಂಡ್ಯ ಜಿಲ್ಲೆ ಮತ್ತು ಮೈಸೂರು ಜಿಲ್ಲೆಗೆ ಸಂಬಂಧಿಸಿದಂತೆ ಹೊಸ ಶಾಸನ ಸಂಪುಟಗಳನ್ನು ಪ್ರಕಟಿಸಿದೆ. ಮಂಡ್ಯ ಜಿಲ್ಲೆಗೆ ಸಂಬಂಧಿಸಿದಂತೆ 1977ರಲ್ಲಿ ಪ್ರಕಟವಾದ ಹೊಸ ಎಪಿಗ್ರಾಫಿಯಾ ಕರ್ನಾಟಿಕಾ\index{ಎಪಿಗ್ರಾಫಿಯಾ ಕರ್ನಾಟಿಕಾ} ಸಂಪುಟ 6 ರಲ್ಲಿ, ಕೃಷ್ಣರಾಜಪೇಟೆ-114, ಪಾಂಡವಪುರ-263 ಮತ್ತು ಶ‍್ರೀರಂಗಪಟ್ಟಣ-122 ಈ ಮೂರು ತಾಲ್ಲೂಕುಗಳ ಒಟ್ಟು 499 ಶಾಸನಗಳಿವೆ. 1979ರಲ್ಲಿ ಪ್ರಕಟವಾದ ಸಂಪುಟ-7 ರಲ್ಲಿ, ನಾಗಮಂಗಲ-185,\break ಮಂಡ್ಯ-87, ಮದ್ದೂರು-145 ಮತ್ತು ಮಳವಳ್ಳಿ-149 ಹೀಗೆ ನಾಲ್ಕು ತಾಲ್ಲೂಕುಗಳ 566 ಶಾಸನಗಳಿವೆ. ಈ ಸಂಪುಟಗಳಲ್ಲಿರುವ ಮಂಡ್ಯ ಜಿಲ್ಲೆಯ ಒಟ್ಟು ಶಾಸನಗಳ ಸಂಖ್ಯೆ ಒಟ್ಟು 1065. ಇದರ ಜೊತೆಗೆ ವರ್ಷೇ ವರ್ಷೆ ಪ್ರಕಟವಾಗುತ್ತಿದ್ದ ಮೈಸೂರು ಆರ್ಕಿಯೋಲಜಿಕಲ್​ ರಿಪೋರ್ಟ್‌ನಲ್ಲಿ\index{ಮೈಸೂರು ಆರ್ಕಿಯೋಲಜಿಕಲ್​ ರಿಪೋರ್ಟ್‌} (ಎಮ್.ಎ.ಆರ್​.) ಹೊಸ ಹೊಸ ಶಾಸನಗಳು ಪ್ರಕಟವಾಗುತ್ತಿದ್ದವು. ಅವುಗಳನ್ನು ಸಾಧ್ಯವಾದ ಮಟ್ಟಿಗೂ ಈ ಹೊಸ ಶಾಸನ ಸಂಪುಟಗಳಲ್ಲಿ ಸೇರಿಸಲಾಗಿದೆ.

ಪ್ರತಿಯೊಂದು ಶಾಸನ ಸಂಪಟಕ್ಕೆ ರೈಸ್​ರವರು ಬರೆದಿದ್ದ ಪೀಠಿಕೆಯನ್ನು ಪರಿಷ್ಕರಿಸಿ, ಹೊಸ ಅಂಶಗಳನ್ನು ಸೇರಿಸಿ, ಕನ್ನಡ ಮತ್ತು ಆಂಗ್ಲ ಭಾಷೆಗಳೆರಡರಲ್ಲೂ ಪೀಠಿಕೆಗಳನ್ನು ಬರೆಯಲಾಗಿದೆ. ಈ ಪೀಠಿಕೆಗಳು ಸ್ಥೂಲ ಸ್ವರೂಪದ್ದಾಗಿದ್ದು, ಹೆಚ್ಚಾಗಿ ರಾಜಕೀಯ ವಿಷಯಗಳನ್ನು ಒಳಗೊಂಡಿವೆ. ಸಾಂಸ್ಕೃತಿಕ ಅಂಶಗಳು ಗೌಣವಾಗಿವೆ. ಶಾಸನ ಸಂಪುಟದ ದೃಷ್ಟಿಯಿಂದ ಪೀಠಿಕೆಗಳು ಕೇವಲ 25-30 ಪುಟಕ್ಕೆ ಸೀಮಿತವಾಗಿವೆ. ಇವೇ ಮಂಡ್ಯ ಜಿಲ್ಲೆಯ ಶಾಸನ ಹಾಗೂ ಸಂಸ್ಕೃತಿಯ ಸಂಶೋಧನಾತ್ಮಕ ವಿಸ್ತೃತ ಅಧ್ಯಯನಕ್ಕೆ ಮೂಲ ಆಕರವಾಗಿವೆ.

ರಾಜವಂಶಗಳಿಗೆ ಸೇರಿದ ಶಾಸನಗಳನ್ನು ವಿಂಗಡಿಸಿದರೆ, ಗಂಗರು-23, ನೊಳಂಬ-1, ರಾಷ್ಟ್ರಕೂಟ-2, ಚೋಳ-7, ಹೊಯ್ಸಳ-248, ಪಾಂಡ್ಯ-2, ವಿಜಯನಗರ- ಸುಮಾರು 130, ಮೈಸೂರು ಒಡೆಯರು- ಸುಮಾರು 70, ಉಮ್ಮತ್ತೂರು-5, ಹದಿನಾಡು-1, ಚನ್ನಪಟ್ಟಣ-3, ಕಳಲೆ-1, ಇತರೆ-ಉಳಿದವು ಎಂದು ಶಾಸನ ತಜ್ಞರಾದ ಶ‍್ರೀ ಎಂ.ಎಚ್​. ನಾಗರಾಜರಾವ್​ ಅವರು ಲೆಕ್ಕಹಾಕಿದ್ದಾರೆ. ಉಳಿದಂತೆ ಈ ಶಾಸನಗಳ ಪೈಕಿ 28 ತಾಮ್ರಶಾಸನಗಳು, ದೇವಾಲಯದ ಲೋಹದ ವಸ್ತುಗಳ ಮೇಲೆ ಬರೆದಿರುವ ಶಾಸನಗಳನ್ನು ಹೊರತುಪಡಿಸಿದರೆ ಉಳಿದೆಲ್ಲವೂ ಶಿಲಾಶಾಸನಗಳು ಎಂದು ಅವರು ಹೇಳಿದ್ದಾರೆ. ಭಾಷೆ ಹಾಗೂ ಲಿಪಿಯ ಪ್ರಕಾರ ವಿಂಗಡಿಸಿದರೆ ಬಹುಪಾಲು ಶಾಸನಗಳು ಕನ್ನಡ ಲಿಪಿ ಮತ್ತು ಭಾಷೆಯಲ್ಲಿ ರಚಿತವಾಗಿದ್ದು, ಕೆಲವು ಶಾಸನಗಳು ಬೇರೆಬೇರೆ ಲಿಪಿ ಮತ್ತು ಭಾಷೆಗಳಲ್ಲಿರುವುದನ್ನು ವಿಂಗಡಿಸಿ ನೀಡಿದ್ದಾರೆ. ತಮಿಳು ಮತ್ತು ಗ್ರಂಥಲಿಪಿ-ಕನ್ನಡ\break ಭಾಷೆಯವು- 109, ಗ್ರಂಥಲಿಪಿ ಮತ್ತು ಸಂಸ್ಕೃತ ಭಾಷೆಯವು 1, ಗ್ರಂಥಲಿಪಿ ಮತ್ತು ಸಂಸ್ಕೃತ-ಕನ್ನಡ ಭಾಷೆಯವು 1, ಗ್ರಂಥ ಮತ್ತು ತಮಿಳು ಲಿಪಿ-ಸಂಸ್ಕೃತ-ತಮಿಳು ಭಾಷೆಯವು 1, ನಾಗರಿಲಿಪಿ ಮತ್ತು ಕನ್ನಡ ಭಾಷೆಯವು 4, ಕನ್ನಡ ಲಿಪಿ ಮತ್ತು ತಮಿಳು ಭಾಷೆಯವು 1, ನಾಗರಿಲಿಪಿ ಮತ್ತು ಹಿಂದೂಸ್ಥಾನಿ ಭಾಷೆಯವು 2. ಸಂಸ್ಕೃತ ಭಾಷೆಯವು 24, ಸಂಸ್ಕೃತ-ಕನ್ನಡ ಭಾಷೆಯವು 5, ತೆಲುಗು ಭಾಷೆಯವು 14, ಪರ್ಷಿಯನ್​ ಮತ್ತು ಅರ‌್ಯಾಬಿಕ್​ 17, ಇಂಗ್ಲಿಷ್​ 4, ಗುಜರಾಥಿ 1 ಉಳಿದವು ಕನ್ನಡ ಲಿಪಿ ಭಾಷೆಯವು ಎಂದು ಅವರು ಲೆಕ್ಕಹಾಕಿದ್ದಾರೆ.\endnote{ ನಾಗರಾಜರಾವ್​, ಎಂ.ಎಚ್​., ಮಂಡ್ಯ ಜಿಲ್ಲೆಯ ಶಾಸನಗಳ ಸಮೀಕ್ಷೆ, ಮಂಡ್ಯ ಜಿಲ್ಲೆಯ ಇತಿಹಾಸ ಮತ್ತು ಪುರಾತತ್ವ, ಪುಟ 16-17} ಮಂಡ್ಯ ಜಿಲ್ಲೆಯ 1333 ಗ್ರಾಮಗಳಲ್ಲಿ 319 ಗ್ರಾಮಗಳಲ್ಲಿ ಮಾತ್ರ ಶಾಸನಗಳು ಲಭ್ಯವಾಗಿದ್ದು, ಈಚೆಗೆ ಸುಮಾರು ಮೂವತ್ತು ಶಾಸನಗಳು ಹೊಸದಾಗಿ ಲಭ್ಯವಾಗಿವೆ ಎಂದು ಅವರು ಹೇಳಿದ್ದಾರೆ. ಮಂಡ್ಯ ಜಿಲ್ಲೆಯ ದೊಡ್ಡಗಾಡಿಗನಹಳ್ಳಿ, ಅಗ್ರಹಾರ ಬಾಚಹಳ್ಳಿ, ಶಿವಪುರ, ಬೂಕನಕೆರೆ, ರಾಮನಹಳ್ಳಿ, ಸೀಳನೆರೆ(3), ದೊಡ್ಡಗಾಡಿಗನಹಳ್ಳಿ, ಸಾದೊಳಲು, ಮಂಗಲ, ಬೋಗಾದಿ, ಹಿರೇಮರಳಿ(3), ಕೋಡಾಲ, ಕೆ.ಬೆಟ್ಟಹಳ್ಳಿ, ಊರುಗಳಲ್ಲಿ ಹೊಸ ಶಾಸನಗಳನ್ನು ಪತ್ತೆ ಹಚ್ಚಿ, ಅವರು ತಮ್ಮ ‘ಶಿವಶೋಧ’ ಕೃತಿಯಲ್ಲಿ ಪ್ರಕಟಿಸಿದ್ದಾರೆ. ಇತಿಹಾಸ ದರ್ಶನ ಸಂಪುಟದಲ್ಲಿ ಸೀಳನೆರೆ, ಮುತ್ತತ್ತಿಯ ತಾಮ್ರಶಾಸನಗಳನ್ನು ಕಂಡುಹಿಡಿದು ಪ್ರಕಟಿಸಲಾಗಿದೆ. ಅರ್ಚಕರಂಗಸ್ವಾಮಿಯವರು ತಮ್ಮ ಹುಟ್ಟಿದಹಳ್ಳಿ ಕೃತಿಯಲ್ಲಿ ಬಂಡಿಹೊಳೆಯ ಒಂದು ತಾಮ್ರಶಾಸನದ ಪಾಠವನ್ನು ನೀಡಿದ್ದಾರೆ. ಡಾ. ಎಮ್.ಜಿ. ಮಂಜುನಾಥ್​ ಅವರು ಬಂಡಿಹೊಳೆಯ ಗಂಗರ ಶ‍್ರೀಪುರುಷನ ಶಾಸನವನ್ನು ಕಂಡುಹಿಡಿದು ಹಂಪಿ ವಿ.ವಿ. ಶಾಸನ ಪತ್ರಿಕೆಯಲ್ಲಿ ಪ್ರಕಟಿಸಿದ್ದಾರೆ. ಮೈಸೂರು ಆರ್ಕಿಯಾಲಾಜಿಕಲ್​ ರಿಪೋರ್ಟ್(1916)ನಲ್ಲಿ ಅಗ್ರಹಾರಬಾಚಹಳ್ಳಿಯ ತಾಮ್ರಶಾಸನದ ಭಾಗಶಃ ಪಾಠವು ಪ್ರಕಟವಾಗಿದೆ. ಈ ಎಲ್ಲ ಶಾಸನಗಳನ್ನೂ ಈ ಕೃತಿ ರಚನೆಯಲ್ಲಿ ಬಳಸಿಕೊಳ್ಳಲಾಗಿದೆ.

\section*{ಮಂಡ್ಯ ಜಿಲ್ಲೆಯ ಬಗ್ಗೆ ಇದುವರೆಗಿನ ಪ್ರಮುಖ ಅಧ್ಯಯನಗಳು}

ಮಂಡ್ಯ ಜಿಲ್ಲೆಯ ಶಾಸನಗಳನ್ನು ಕುರಿತು ಇದುವರೆಗೆ ಅನೇಕ ಸಂಶೋಧನಾ ಕೃತಿಗಳು ಮತ್ತು ಲೇಖನಗಳು ಪ್ರಕಟವಾಗಿವೆ. ಕರ್ನಾಟಕದ ಇತಿಹಾಸವನ್ನು ಕುರಿತು ರಚಿತವಾಗಿರುವ, ಅದರಲ್ಲೂ ಪ್ರಮುಖವಾಗಿ ಶಾಸನಗಳನ್ನು ಆಧರಿಸಿ ರಚಿತವಾಗಿರುವ ಕೃತಿಗಳಲ್ಲಿ, ಮಂಡ್ಯ ಜಿಲ್ಲೆಯಲ್ಲಿರುವ ಪ್ರಮುಖವಾದ ಶಾಸನಗಳನ್ನು ಮಾತ್ರ ಉಲ್ಲೇಖಿಸಲಾಗಿದೆ. ಹೊಯ್ಸಳರ ಇತಿಹಾಸವನ್ನು ರಚಿಸಿದ ಕೊಯಿಲೋ ಮತ್ತು ಡಂಕನ್​ಡೆರೆಟ್​ ಅವರು ತಮ್ಮ ಕೃತಿಯಲ್ಲಿ ಮಂಡ್ಯ ಜಿಲ್ಲೆಯ ಅನೇಕ ಶಾಸನಗಳನ್ನು ಬಳಸಿ\break ಕೊಂಡಿದ್ದಾರೆ. ಗಂಗರ ಇತಿಹಾಸವನ್ನು ರಚಿಸಿರುವ ಡಾ.ಬಿ.ಶೇಕ್​ಅಲಿ, ಎಂ.ವಿ.ಕೃಷ್ಣರಾವ್​, ಡಾ.ದೇವರಕೊಂಡಾರೆಡ್ಡಿ, ಎಸ್​.ಶಿವಣ್ಣ, ಮೊದಲಾದವರು ತಮ್ಮ ಕೃತಿಗಳಲ್ಲಿ ಜಿಲ್ಲೆಯ ಗಂಗರ ಶಾಸನಗಳನ್ನು ಉಲ್ಲೇಖಿಸಿದ್ದಾರೆ. ವಿಜಯನಗರದ ಇತಿಹಾಸವನ್ನು ರಚಿಸಿರುವ ಪಿ.ಬಿ.ದೇಸಾಯಿ, ವಸುಂಧರಾ ಫಿಲಿಯೋಜಾ, ಮೊದಲಾದವರು, ಮೈಸೂರು ಒಡೆಯರ ಇತಿಹಾಸವನ್ನು ರಚಿಸಿರುವ ಡಾ.ಎ.ಸತ್ಯನಾರಾಯಣ ಅವರು, ಕರ್ನಾಟಕದ ಇತಿಹಾಸ ಗ್ರಂಥಗಳನ್ನು ರಚಿಸಿರುವ\break ಡಾ.ಸೂರ್ಯನಾಥಕಾಮತ್​, ಪ್ರೊ. ಎಂ.ವಿ. ಕೃಷ್ಣರಾವ್​, ನೆಲಮಂಗಲ ಲಕ್ಷ್ಮೀನಾರಾಯಣರಾವ್​, ಆರ್​.ಎಸ್​.ಪಂಚಮುಖಿ, ಬಾ.ರಾ.ಗೋಪಾಲ್​, ಪ್ರೊ. ನಂಜುಂಡಸ್ವಾಮಿ, ಡಾ.ಎಸ್​.ಎನ್​. ಶಿವರುದ್ರಸ್ವಾಮಿ, ಮೊದಲಾದ ಇತಿಹಾಸ ವಿದ್ವಾಂಸರು, ತಮ್ಮ ಕೃತಿಗಳಲ್ಲಿ ಮಂಡ್ಯ ಜಿಲ್ಲೆಯ ಪ್ರಮುಖ ಶಾಸನಗಳನ್ನು ಅಲ್ಲಲ್ಲಿ ಉಲ್ಲೇಖಿಸಿದ್ದಾರೆ.

ಶಾಸನಗಳ ಆಧಾರದ ಮೇಲೆ ಕರ್ನಾಟಕದ ಸಾಂಸ್ಕೃತಿಕ ಇತಿಹಾಸದ ಬಗ್ಗೆ ಕೃತಿಗಳನ್ನು, ಸಂಶೋಧನಾ ಪ್ರಬಂಧಗಳನ್ನು ರಚಿಸಿರುವ ವಿದ್ವಾಂಸರುಗಳು, ತಮ್ಮ ಕೃತಿಯ ಅಗತ್ಯಗಳಿಗೆ ತಕ್ಕಂತೆ ಮಂಡ್ಯ ಜಿಲ್ಲೆಯ ಕೆಲವು ಶಾಸನಗಳನ್ನು ಉಲ್ಲೇಖಿಸಿ ವಿಶ್ಲೇಷಿಸಿದ್ದಾರೆ. ಡಾ.ಎಂ. ಚಿದಾನಂದಮೂರ್ತಿಯವರ ‘ಕನ್ನಡ ಶಾಸನಗಳ ಸಾಂಸ್ಕೃತಿಕ ಅಧ್ಯಯನ’, ಡಾ.ಎಂ.ಎಂ. ಕಲಬುರ್ಗಿಯವರ ‘ಶಾಸನಗಳಲ್ಲಿ ಶಿವಶರಣರು’, ‘ಮಾರ್ಗ ಸಂಪುಟಗಳು’, ‘ಸಮಾಧಿ ಬಲಿದಾನ ವೀರಮರಣ ಸ್ಮಾರಕಗಳು’\break ಡಾ. ಆರ್​. ಶೇಷಶಾಸ್ತ್ರಿಯವರ ‘ಕರ್ನಾಟಕದ ವೀರಗಲ್ಲುಗಳು’, ಡಾ. ಚೆನ್ನಕ್ಕ ಎಲಿಗಾರರ ‘ಶಾಸನಗಳಲ್ಲಿ ಕರ್ನಾಟಕದ ಸ್ತ್ರೀ ಸಮಾಜ’, ಡಾ.ಬಿ.ಆರ್​. ಹಿರೇಮಠ್​ರವರ “ಶಾಸನಗಳಲ್ಲಿ ಕರ್ನಾಟಕದ ವರ್ತಕರು’, ಡಾ. ಎಸ್​.ಕೆ. ಕುಮಾರಸ್ವಾಮಿಯವರ ‘ಪ್ರಾಚೀನ ಕರ್ನಾಟಕದಲ್ಲಿ ಶಿಲ್ಪಾಚಾರಿಯರು’, ಎಂ.ವಿ. ಕೃಷ್ಣಪ್ಪನವರ \enginline{‘Social and Economic Conditions of Karnataka’, } ಡಾ.ಎಸ್​.ಗುರುರಾಜಾಚಾರ್ಯರ \enginline{“Some Aspects of Ecnomic and Social Life in Karnataka”} ಡಾ.ಎಂ.ಬಿ.ಪದ್ಮ ಅವರ \enginline{“Position of Woman in Medieval Karnataka”,} ಮುಂತಾದ ಸಂಶೋಧನಾ ಪ್ರಬಂಧಗಳಲ್ಲಿ, ಸಾಂಸ್ಕೃತಿಕ ಮತ್ತು ಐತಿಹಾಸಿಕ ಲೇಖನಗಳು, ಪ್ರಬಂಧಗಳು ಮತ್ತು ಕೃತಿಗಳಲ್ಲಿ ಧಾರ್ಮಿಕ, ಸಾಮಾಜಿಕ, ಆರ್ಥಿಕ ಮತ್ತು ಸಾಂಸ್ಕೃತಿಕ ವಿಷಯಕ್ಕೆ ಸಂಬಂಧಿಸಿದಂತೆ ಮಂಡ್ಯ ಜಿಲ್ಲೆಯ ಅನೇಕ ಪ್ರಮುಖ ಶಾಸನಗಳನ್ನ ಸಂದರ್ಭೋಚಿತವಾಗಿ ಉಲ್ಲೇಖಿಸಲಾಗಿದೆ.

ಡಾ. ಎಸ್​. ಶ‍್ರೀಕಂಠಶಾಸ್ತ್ರಿಯವರ\index{ಡಾ.ಎಸ್​. ಶ‍್ರೀಕಂಠಶಾಸ್ತ್ರಿ} ‘ಹೊಯ್ಸಳ ವಾಸ್ತುಶಿಲ್ಪ’, ಡಾ. ದೇವರಕೊಂಡಾರೆಡ್ಡಿಯವರ ‘ತಲಕಾಡಿನ ಗಂಗರ ದೇವಾಲಯಗಳು’ ಮತ್ತು ‘ಗಂಗರ ಶಿಲ್ಪಕಲೆ’, ಡಾ.ಎಚ್​.ಎಸ್​. ಗೋಪಾಲರಾವ್​ ಅವರ ‘ಶಾಸನಗಳ ಹಿನ್ನೆಲೆಯಲ್ಲಿ ಕಲ್ಯಾಣದ ಚಾಲುಕ್ಯರ ದೇವಾಲಯಗಳು’, ಡಾ. ವಸಂತಲಕ್ಷ್ಮಿಯವರ ‘ಹೊಯ್ಸಳ ಶಿಲ್ಪಕಲೆ’, ಮತ್ತು ‘ಕರ್ನಾಟಕದ ಶೈವಶಿಲ್ಪಗಳು’ ಡಾ.ಅ.ಲ.ನರಸಿಂಹನ್​ ಅವರ ‘ವಿಜಯನಗರ ಶಿಲ್ಪಕಲೆ’, ಡಾ. ಕೂ.ಸ.ಅಪರ್ಣ ಅವರ ‘ಸಂಗಮರ ಕಾಲದ ದೇವಾಲಯಗಳು’, ಡಾ. ಸತೀಶ್​ ಅವರ ‘ವಿಜಯನಗರ ಕಾಲದ ಶೈವ ದೇವಾಲಯಗಳು’, ಡಾ. ಶೋಭಾ\index{ಡಾ. ಶೋಭಾ} ಅವರ ‘ಮಂಡ್ಯ ಜಿಲ್ಲೆಯ ಹೊಯ್ಸಳ ದೇವಾಲಯಗಳು’ ಇನ್ನೂ ಮುಂತಾದ ಕೃತಿಗಳಲ್ಲಿ ಮಂಡ್ಯ ಜಿಲ್ಲೆಯ ಅನೇಕ ದೇವಾಲಯ, ಬಸದಿಗಳ ವಾಸ್ತು ಹಾಗೂ ಅದಕ್ಕೆ ಸಂಬಂಧಿಸಿದ ವಿವರಗಳನ್ನು, ಸಂದರ್ಭಕ್ಕೆ ತಕ್ಕಂತೆ ಉಲ್ಲೇಖಿಸಲಾಗಿದೆ. ಡಾ.ಎಸ್​.ರಂಗರಾಜು ಅವರ \enginline{“Hoysala Temples in Mandya and Tumkur Districts”} ಕೃತಿಯಲ್ಲಿ, ಮಂಡ್ಯ ಜಿಲ್ಲೆಯಲ್ಲಿ 26 ಹೊಯ್ಸಳ ದೇವಾಲಯಗಳನ್ನು ಪ್ರಧಾನವಾಗಿ ವಾಸ್ತುವಿನ ದೃಷ್ಟಿಯಿಂದ ವಿವೇಚಿಸಿದ್ದು, ಈ ದೇವಾಲಯಗಳಲ್ಲಿ ಕಂಡುಬರುವ ಶಾಸನಗಳ ಬಗ್ಗೆಯೂ ಟಿಪ್ಪಣಿಗಳನ್ನು ನೀಡಿದ್ದಾರೆ. ತೈಲೂರು ವೆಂಕಟಕೃಷ್ಣ ಅವರ “ಮಂಡ್ಯ ಜಿಲ್ಲೆಯ ದೇವಾಲಯಗಳು-ಒಂದು ಅವಲೋಕನ” ಕೃತಿಯಲ್ಲಿ, ಗಂಗರಕಾಲ ಬಸದಿಗಳೂ ಸೇರಿದಂತೆ ಗಂಗರ ಕಾಲದ ಸುಮಾರು 15, ಚೋಳರಕಾಲದ ಸುಮಾರು 8, ಹೊಯ್ಸಳಕಾಲದ ಸುಮಾರು 38, ವಿಜಯನಗರದ ಕಾಲ ಮತ್ತು ಮೈಸೂರು ಅರಸರ ಕಾಲಸ ಸುಮಾರು 30 ದೇವಾಲಯಗಳನ್ನು ವಾಸ್ತು ಮತ್ತು ಸ್ಥಳಪುರಾಣಗಳ ಹಿನ್ನೆಲೆಯಲ್ಲಿ ವಿವರಿಸಿದ್ದಾರೆ. ಕೆಲವು ದೇವಾಲಯಗಳು ಮತ್ತು ಮೂರ್ತಿಶಿಲ್ಪಗಳ ಅಪರೂಪದ ಛಾಯಾಚಿತ್ರಗಳನ್ನು ನೀಡಿದ್ದಾರೆ. ಆದರೆ ದೇವಾಲಯಗಳಿಗೆ ಸಂಬಂಧಿಸಿದ ಶಾಸನಗಳನ್ನು ಕೆಲವೆಡೆ ಉಲ್ಲೇಖಿಸಿದ್ದರೂ, ವಿಶೇಷವಾಗಿ ಶಾಸನಗಳ ವಿವರಗಳನ್ನು ನೀಡಿರುವುದಿಲ್ಲ. ‘ಮಂಡ್ಯ ಜಿಲ್ಲೆಯ ಹೊಯ್ಸಳ ದೇವಾಲಯಗಳು’ ಕೃತಿಯಲ್ಲಿ ಡಾ. ಶೋಭಾ ಅವರು, ಹೊಯ್ಸಳರ ಸಂಕ್ಷಿಪ್ತ ಇತಿಹಾಸವನ್ನು ನೀಡಿ, ಸುಮಾರು 38 ಹೊಯ್ಸಳ ದೇವಾಲಯಗಳನ್ನು ಗುರುತಿಸಿದ್ದಾರೆ. ಇದರ ಜೊತೆಗೆ ಕಂಬದಹಳ್ಳಿ, ಸೂರನಹಳ್ಳಿಯ ಬಸದಿಗಳನ್ನು ಅಧ್ಯಯನ ಮಾಡಿದ್ದಾರೆ. ದೇವಾಲಯದ ನಿರ್ಮಾಣ ಮತ್ತು ಇತಿಹಾಸಕ್ಕಿಂತ ಮುಖ್ಯವಾಗಿ, ದೇವಾಲಯದ ಮೂರ್ತಿ ಶಿಲ್ಪಗಳ ಉಡುಪು, ಆಭರಣ, ಕೇಶಾಲಂಕಾರಗಳನ್ನೂ, ಶಿಲ್ಪಕಲೆಯಲ್ಲಿ ಬಿಡಿಸಲಾಗಿರುವ ಪೀಠೋಪಕರಣಗಳು ಮತ್ತು ಗೃಹೋಪಕರಣಗಳು, ಸಂಗೀತ ವಾದ್ಯಗಳು, ಸಂಚಾರ ಸಾಧನಗಳು, ಆಯುಧಗಳು, ಸಸ್ಯ ಮತ್ತು ಪ್ರಾಣಿಗಳು ಹಾಗೂ ಜನಜೀವನ, ಇವುಗಳನ್ನು ಕುರಿತು ಅಧ್ಯಯನ ಮಾಡಿದ್ದಾರೆ. ಜಿಲ್ಲೆಯಲ್ಲಿರುವ ಶಾಸನೋಕ್ತ ಹೊಯ್ಸಳರ ಅನೇಕ ಸಣ್ಣಪುಟ್ಟ ದೇವಾಲಯಗಳನ್ನು ಇವರು ತಮ್ಮ ಅಧ್ಯಯನದಲ್ಲಿ ಸೇರ್ಪಡೆ ಮಾಡಿಕೊಂಡಿರುವುದಿಲ್ಲ.

ಡಾ. ಜಿ.ಎಸ್​.ದೀಕ್ಷಿತ್​ ರವರ \enginline{“Local Self Government in Medieval Karnataka”}, ಡಾ. ಎ.ವಿ. ವೆಂಕಟರತ್ನಂ ರವರ \enginline{“Local Government in the Vijayanagara Empire”} ಡಾ. ಎಂ.ಎಂ. ಕಲಬುರ್ಗಿಯವರು ಸಂಪಾದಿಸಿರುವ “ಪ್ರಾಚೀನ ಕರ್ನಾಟಕದ ಆಡಳಿತ ವಿಭಾಗಗಳು” ಮುಂತಾದ ಕೃತಿಗಳಲ್ಲಿ ಆಡಳಿತ ಪದ್ಧತಿಗೆ ಸಂಬಂಧಿಸಿದ ಮಂಡ್ಯ ಜಿಲ್ಲೆಯ ಶಾಸನಗಳನ್ನು ಉಲ್ಲೇಖಿಸಲಾಗಿದೆ. ಜಿ.ಎಸ್​.ದೀಕ್ಷಿತ್​, ಜಿ.ಆರ್​.ಕುಪ್ಪುಸ್ವಾಮಿ ಮತ್ತು ಎಸ್​.ಕೆ.ಮೋಹನ್​ರವರ “ಕರ್ನಾಟಕದಲ್ಲಿ ಕೆರೆ ನೀರಾವರಿ”, ರಾಜಾರಾಮ ಹೆಗ್ಗಡೆಯವರ ‘ ಕೆರೆ ನೀರಾವರಿ ನಿರ್ವಹಣೆ,ಚಾರಿತ್ರಿಕ ಅಧ್ಯಯನ’ ಡಾ. ಕೆ.ಎಸ್​. ಶಿವಣ್ಣ ಅವರ \enginline{‘The Agrarian Sysytem of Karnataka-1336-1761’} ಮೊದಲಾದ ಕೃತಿಗಳಲ್ಲಿ ಕೃಷಿಪದ್ಧತಿ ಮತ್ತು ನೀರಾವರಿಯ ಬಗ್ಗೆ ಮಂಡ್ಯ ಜಿಲ್ಲೆಯ ಪ್ರಮುಖ ಶಾಸನಗಳನ್ನು ಉಲ್ಲೇಖಿಸಿದ್ದಾರೆ. ಡಾ.ಎಸ್​.ನಾಗರಾಜು ಅವರ ಪ್ರೌಢದೇವರಾಯ ಮತ್ತು ಅವನ ಕಾಲ, ಡಾ. ರಾಧಾಪಟೇಲ್​ ಅವರ ‘ಹೊಯ್ಸಳರ ಮೂರನೆಯ ವೀರನರಸಿಂಹ ಮತ್ತು ಅವನ ಕಾಲ(ಆಂಗ್ಲಭಾಷೆಯ ಸಂಶೋಧನಾ ಪ್ರಬಂಧ)’ ಮುಂತಾದ ಇತಿಹಾಸ ಕೃತಿಗಳಲ್ಲಿ ಮಂಡ್ಯ ಜಿಲ್ಲೆಯ ಶಾಸನಗಳನ್ನು ಸಂದರ್ಭೋಚಿತವಾಗಿ ಉಲ್ಲೇಖಿಸಲಾಗಿದೆ.

ಮಂಡ್ಯ ಜಿಲ್ಲೆಯ ಸ್ಥಳನಾಮಗಳ ಬಗ್ಗೆ ಅಧ್ಯಯನ ಮಾಡಿರುವ ತೈಲೂರು ವೆಂಕಟಕೃಷ್ಣರವರು\index{ತೈಲೂರು ವೆಂಕಟಕೃಷ್ಣ} “ಮಂಡ್ಯ ಜಿಲ್ಲೆಯ ಸ್ಥಳನಾಮಗಳ ಅವಲೋಕನ- ಭಾಗ-1’ ಕೃತಿಯಲ್ಲಿ, ಜಿಲ್ಲೆಯ ಏಳು ತಾಲ್ಲೂಕುಗಳ, ಶಾಸನೋಕ್ತವಾದ, ಶಾಸನೋಕ್ತವಲ್ಲದ ಕೆಲವು ಪ್ರಮುಖ ಊರುಗಳ ಸ್ಥಳನಾಮಗಳನ್ನು ಸ್ಥಳಪುರಾಣ, ಭೌಗೋಳಿಕ ಮತ್ತು ಚಾರಿತ್ರಿಕ ಹಿನ್ನೆಲೆಗಳೊಡನೆ ವಿವೇಚಿಸಿದ್ದಾರೆ. ತಾಲ್ಲೂಕುವಾರು ಲೆಕ್ಕದಲ್ಲಿ ಸುಮಾರು ಮಂಡ್ಯ-81, ಮದ್ದೂರು-80, ಮಳವಳ್ಳಿ-83, ನಾಗಮಂಗಲ-101,\break ಕೃಷ್ಣರಾಜಪೇಟೆ-92, ಪಾಂಡವಪುರ-50 ಮತ್ತು ಶ‍್ರೀರಂಗಪಟ್ಟಣ-47 ಗ್ರಾಮ ನಾಮಗಳನ್ನು ವಿವೇಚಿಸಿದ್ದಾರೆ.

ಕರ್ನಾಟಕ ಇತಿಹಾಸ ಅಕಾಡೆಮಿಯಿಂದ ಪ್ರಕಟವಾಗಿರುವ ‘ಇತಿಹಾಸ ದರ್ಶನ’ದ ಸುಮಾರು 25 ಸಂಪುಟಗಳಲ್ಲಿ, ಮಂಡ್ಯ ಜಿಲ್ಲೆಯ ಪ್ರಾಗಿತಿಹಾಸ, ಶಾಸನ, ಸಂಸ್ಕೃತಿ, ಸ್ಥಳನಾಮ, ವಾಸ್ತು ವಿಚಾರಗಳಿಗೆ ಸಂಬಂಧಿಸಿದಂತೆ ಸುಮಾರು 20 ಕ್ಕೂ\break ಹೆಚ್ಚು ವಿದ್ವತ್​ ಲೇಖಗಳು ಪ್ರಕಟವಾಗಿವೆ. ವಿಶ್ವವಿದ್ಯಾನಿಲಯಗಳ ನಿಯತಕಾಲಿಕಗಳಲ್ಲಿ ಮಂಡ್ಯ ಜಿಲ್ಲೆಯ ಶಾಸನಗಳ ಅಧ್ಯಯನಕ್ಕೆ ಸಂಬಂಧಿಸಿದ ಲೇಖನಗಳು ಪ್ರಕಟವಾಗಿವೆ. ಕರ್ನಾಟಕ ಸರ್ಕಾರದ ಪ್ರಾಚ್ಯವಸ್ತು ಮತ್ತು ಸಂಗ್ರಹಾಲಯವು ಮಂಡ್ಯ ಜಿಲ್ಲೆಯ ಇತಿಹಾಸ ಮತ್ತು ಪುರಾತತ್ವದ ಬಗ್ಗೆ ವಿಚಾರ ಸಂಕಿರಣವನ್ನು ನಡೆಸಿ ಅಲ್ಲಿ ವಿದ್ವಾಂಸರಿಂದ ಮಂಡಿಸಲ್ಪಟ್ಟ ಪ್ರಬಂಧಗಳನ್ನು ‘ಮಂಡ್ಯ ಜಿಲ್ಲೆಯ ಇತಿಹಾಸ ಮತ್ತು ಪುರಾತತ್ವ’ ಎಂಬ ಕೃತಿಯನ್ನು ಪ್ರಕಟಿಸಿದೆ. ಇದರಲ್ಲಿ ಬಹುತೇಕ ಲೇಖನಗಳು ಜಿಲ್ಲೆಯ ಶಾಸನಗಳನ್ನು ಆಧರಿಸಿ ಬರೆದ ಲೇಖನಗಳಾಗಿವೆ. ಮಂಡ್ಯ ಜಿಲ್ಲೆಗೆ ಸಂಬಂಧಿಸಿದ ಎಪಿಗ್ರಾಫಿಯಾ ಕರ್ನಾಟಿಕಾ ಹಳೆಯ ಸಂಪುಟಗಳಲ್ಲಿ ಬಿ.ಎಲ್​.ರೈಸ್​ರವರು\index{ಬಿ.ಎಲ್​.ರೈಸ್​} ಬರೆದಿರುವ ಪೀಠಿಕೆಯಲ್ಲಿ, ಶಾಸನಗಳನ್ನು ರಾಜವಂಶದ ಇತಿಹಾಸದ ದೃಷ್ಟಿಯಿಂದ ಪ್ರಮುಖವಾಗಿ ವಿಶ್ಲೇಷಿಸಿದ್ದಾರೆ. ಅದೇ ರೀತಿ, ಮಂಡ್ಯ ಜಿಲ್ಲೆಗೆ ಸಂಬಂಧಿಸಿದಂತೆ ಎಪಿಗ್ರಾಫಿಯಾ ಕರ್ನಾಟಿಕಾ ಹೊಸ ಪರಿಷ್ಕೃತ ಸಂಪುಟಗಳಲ್ಲಿ (6 ಮತ್ತು 7) ಸಂಪಾದಕರು ಬರೆದಿರುವ ಪೀಠಿಕೆಯಲ್ಲಿ ಮಂಡ್ಯ ಜಿಲ್ಲೆಯ ಪ್ರಮುಖ ಶಾಸನಗಳಲ್ಲಿರುವ ರಾಜಕೀಯ ಇತಿಹಾಸವನ್ನು ಪ್ರಮುಖವಾಗಿ, ಧಾರ್ಮಿಕ, ಸಾಮಾಜಿಕ, ಆರ್ಥಿಕ ಮತ್ತು ಸಾಹಿತ್ಯಿಕ ವಿಷಯಗಳನ್ನು ಸ್ವಲ್ಪಮಟ್ಟಿಗೆ ವಿಶ್ಲೇಷಣೆಗೆ ಒಳಪಡಿಸಲಾಗಿದೆ. ಆದರೆ ಮಂಡ್ಯ ಜಿಲ್ಲೆಯ ಬಗ್ಗೆ, ರಾಜಕೀಯ, ಧಾರ್ಮಿಕ, ಸಾಂಸ್ಕೃತಿಕ, ಸಾಮಾಜಿಕ ಮತ್ತು ಆರ್ಥಿಕ ದೃಷ್ಠಿಕೋನಗಳಿಂದ, ಶಾಸನಗಳನ್ನು ಆಧರಿಸಿ, ಇದುವರೆಗೆ ವಿವರವಾದ ವಿಶ್ಲೇಷಣಾತ್ಮಕ ಅಧ್ಯಯನ ನಡೆಸಿರುವುದು ಕಂಡುಬರುವುದಿಲ್ಲ.

ಮಂಡ್ಯ ಜಿಲ್ಲೆ ಹಾಗೂ ಜಿಲ್ಲೆಯಲ್ಲಿರುವ ತಾಲ್ಲೂಕುಗಳು ಮತ್ತು ಕೆಲವು ಊರುಗಳ ಬಗ್ಗೆ, ಜಿಲ್ಲೆಯ ಆರ್ಥಿಕ, ಸಾಮಾಜಿ ಮತ್ತು ಸಾಂಸ್ಕೃತಿಕ ವಿಷಯಗಳ ಬಗ್ಗೆ, ಜಿಲ್ಲೆಯ ಶಾಸನ, ಸಂಸ್ಕೃತಿ, ಐತಿಹ್ಯ, ಜಾನಪದ ಮೊದಲಾದ ವಿಷಯಗಳ ಬಗ್ಗೆ ಅನೇಕ ಕೃತಿಗಳು ಮತ್ತು ಸಂಶೋಧನಾ ಪ್ರಬಂಧಗಳು ಪ್ರಕಟವಾಗಿದ್ದು ಅವುಗಳಲ್ಲಿ ಪ್ರಸ್ತಾಪಿತವಾಗಿರುವ ಅಂಶಗಳನ್ನು ಈ ಕೃತಿಯಲ್ಲಿ ಬಳಸಿಕೊಳ್ಳಲಾಗಿದೆ. ಕೆ.ಅನಂತರಾಮು ಅವರ ‘ಸಕ್ಕರೆಯ ಸೀಮೆ’\index{ಸಕ್ಕರೆಯ ಸೀಮೆ} ಮಂಡ್ಯ ಜಿಲ್ಲೆಯನ್ನು ಕುರಿತಂತೆ ರಚಿತವಾಗಿರುವ ಪ್ರವಾಸಿ ವಿಶ್ವಕೋಶದಂತಿದೆ. ಈ ಕೃತಿಯಲ್ಲಿ ಜಿಲ್ಲೆಯ ಎಲ್ಲ ತಾಲ್ಲೂಕುಗಳ ಪ್ರಮುಖ ಊರುಗಳ ಸ್ಥಳಪುರಾಣ, ಜಾನಪದ ಹಿನ್ನೆಲೆ, ದೇವಾಲಯಗಳು, ಶಾಸನಗಳು, ಜಾನಪದ ಸಂಸ್ಕೃತಿ, ಜನಜೀವನ, ಭಾಷೆ ಇವುಗಳೆಲ್ಲವನ್ನೂ ಕೂಡಾ ಕ್ಷೇತ್ರಕಾರ್ಯದ ಮೂಲಕವೇ ನೋಡಿ ದಾಖಲಿಸಲಾಗಿದೆ. ಆದರೆ ಈ ಕೃತಿಯಲ್ಲಿ ಶಾಸನಗಳ ಬಗ್ಗೆ ಉಲ್ಲೇಖವಿದೆಯೇ ಹೊರತು, ಅವುಗಳ ಆಳವಾದ ಅಧ್ಯಯನವಾಗಲೀ, ವಿಶ್ಲೇಷಣೆಯಾಗಲೀ ಕಂಡು ಬರುವುದಿಲ್ಲ. ಈ ಕೃತಿಯ ಉದ್ದೇಶವೂ ಅದಲ್ಲ. ಇದೊಂದು ಜಾನಪದೀಯ ಹಿನ್ನೆಲೆಯಲ್ಲಿ ರಚಿತವಾದ ಕೃತಿಯಾಗಿದೆ. ಡಾ. ಸಿ. ಮಹದೇವ ಅವರು ಸಂಪಾದಿಸಿರುವ ‘ತೊಣ್ಣೂರು’\index{ತೊಣ್ಣೂರು} ಮತ್ತು ‘ನಾಗಮಂಗಲ’\index{ನಾಗಮಂಗಲ} ಕೃತಿಗಳು ಜಿಲ್ಲೆಯ ಎರಡು ಪ್ರಮುಖ ಐತಿಹಾಸಿಕ ಸ್ಥಳಗಳ ಬಗ್ಗೆ ಶಾಸನಗಳು ಹಾಗೂ ಜಾನಪದೀಯ ಹಿನ್ನೆಲೆಯಲ್ಲಿ ರಚಿತವಾದ ಕೃತಿಗಳು. ಡಾ. ರಾಜೇಶ್ವರಿ ಗೌಡ ಅವರು ‘ಆದಿಚುಂಚನಗಿರಿ-ಒಂದು ಸಾಂಸ್ಕೃತಿಕ ಅಧ್ಯಯನ’ ಎಂಬ ಸಂಶೋಧನಾ ಪ್ರಬಂಧವು, ಜಿಲ್ಲೆಯ, ರಾಜ್ಯದ ಒಂದು ದೊಡ್ಡ ಗುರುಪೀಠದ ಬಗ್ಗೆ, ಪುರಾಣ, ಇತಿಹಾಸ, ಜಾನಪದ ಮತ್ತು ಸಾಂಸ್ಕೃತಿಕ ಹಿನ್ನೆಲೆಯಲ್ಲಿ ರಚಿತವಾದ ಪ್ರಬಂಧವಾಗಿದೆ. ಹ. ಕ. ರಾಜೇಗೌಡ ಅವರು ‘ಆದಿಚುಂಚನಗಿರಿ’\index{ಆದಿಚುಂಚನಗಿರಿ} ಪರಿಚಯಾತ್ಮಕ ಕೃತಿಯನ್ನು ರಚಿಸಿದ್ದಾರೆ. ಮಂಡ್ಯ ಜಿಲ್ಲೆಯ ಪ್ರಮುಖ ಧಾರ್ಮಿಸ್ಥಳವಾದ ಮೇಲುಕೋಟೆಯ\index{ಮೇಲುಕೋಟೆ} ಬಗ್ಗೆ ಅನೇಕ ಕೃತಿಗಳು ಪ್ರಕಟವಾಗಿವೆ. ಅನೇಕ ಗ್ರಂಥಗಳು ಮತ್ತು ಸ್ಮರಣ ಸಂಚಿಕೆಗಳಲ್ಲಿ ಮಂಡ್ಯ ಜಿಲ್ಲೆಯ ಬಗ್ಗೆ ಲೇಖನಗಳು ಪ್ರಕಟವಾಗಿವೆ. ಈ ಕೃತಿಗಳು ಮತ್ತು ಲೇಖನಗಳೆಲ್ಲವನ್ನೂ ಸಾಂದರ್ಭೋಚಿತವಾಗಿ ಅಲ್ಲಲ್ಲಿ ಬಳಸಿಕೊಳ್ಳಲಾಗಿದೆ.


\section*{ಕೃತಿಯ ಉದ್ದೇಶ ಮತ್ತು ವೈಶಿಷ್ಟ್ಯ}

ಆಧುನಿಕ ಆಡಳಿತ ಘಟಕಗಳಾದ ಜಿಲ್ಲೆಯ ಶಾಸನಗಳ ಸಮಗ್ರ ಅಧ್ಯಯನ ನಡೆಸುವ ಕಡೆಗೆ ಹೆಚ್ಚಿನ ಒಲವು ಇತ್ತೀಚೆಗೆ ಕಂಡುಬರುತ್ತಿದೆ. ಕರ್ನಾಟಕದ ಎಲ್ಲ ಜಿಲ್ಲೆಗಳಿಗೆ ಸಂಬಂಧಿಸಿದಂತೆ ಹೊಸದಾಗಿ ಹಾಗೂ ಪರಿಷ್ಕೃತಗೊಂಡ ಶಾಸನ ಸಂಪುಟಗಳೂ ಪ್ರಕಟಣೆಯಾಗುತ್ತಿರುವ ಹಿನ್ನೆಲೆಯಲ್ಲಿ, ಅವುಗಳನ್ನು ಆಧರಿಸಿ, ಅದಕ್ಕೆ ಪೂರಕವಾಗಿ ಜಿಲ್ಲೆಯ ಶಾಸನಗಳ ಬಗ್ಗೆ ಆಳವಾದ ವಿಶ್ಲೇಷಣೆ, ಅಧ್ಯಯನ ಅಗತ್ಯವಾಗಿದೆ. ಶಾಸನಗಳನ್ನು ವಿವರವಾಗಿ ಅಧ್ಯಯನ ಮಾಡುವುದರಿಂದ, ಅರ್ವಾಚೀನ ಕಾಲದಿಂದ ಕಳೆದ ಶತಮಾನದವರೆಗೆ, ಜಿಲ್ಲೆಯ ರಾಜಕೀಯ, ಸಾಂಸ್ಕೃತಿಕ ಇತಿಹಾಸವು ಕ್ರಮಬದ್ಧವಾಗಿ ಪ್ರಕಟಗೊಂಡು, ಒಟ್ಟಾರೆ ಜಿಲ್ಲೆಯ ಪ್ರದೇಶಗಳ ಪ್ರಾಚೀನತೆ, ಅದರ ಮಹತ್ವ ಹಾಗೂ ಐತಿಹಾಸಿಕ ಮತ್ತು ಸಾಂಸ್ಕೃತಿ ಹಿರಿಮೆಗಳನ್ನು ಮನಗಾಣಲು ಸಾಧ್ಯವಾಗುತ್ತದೆ. ಮಂಡ್ಯ ಜಿಲ್ಲೆಯಲ್ಲಿರುವ ಶಾಸನೋಕ್ತವಾದ ಪ್ರತಿಯೊಂದು ಊರುಗಳು, ದೇವಾಲಯಗಳು, ಬಸದಿಗಳ ಮತ್ತು ಸ್ಮಾರಕಗಳ ಬಗೆಗಿನ ಇತಿಹಾಸ ಮತ್ತು ಸಂಸ್ಕೃತಿಯ ತಿಳಿವಳಿಕೆ ಉಂಟಾಗುತ್ತದೆ. ಇದರ ಜೊತೆಗೆ ಇಂದಿನ ಮಂಡ್ಯ ಜಿಲ್ಲೆಯ ಪ್ರದೇಶದಲ್ಲಿಯೇ ಹುಟ್ಟಿ ಬೆಳೆದು, ಈ ಪ್ರದೇಶದ ಆರ್ಥಿಕ, ಸಾಮಾಜಿಕ ಮತ್ತು ಸಾಂಸ್ಕೃತಿಕ ಅಭಿವೃದ್ಧಿಗೆ ಕಾರಣರಾದ ಅನೇಕ ಮಹಾಮಂಡಲೇಶ್ವರರು, ಪಾಳೆಯಗಾರರು, ಮಹಾಪ್ರಧಾನರು, ದಂಡನಾಯಕರು, ಅಧಿಕಾರಿಗಳು, ಸ್ತ್ರೀಯರು, ಅವರ ವಂಶ ಹಾಗೂ ಸಾಧನೆಗಳು ಈ ಜಿಲ್ಲೆಯ ಶಾಸನಗಳಲ್ಲಿ ಮಾತ್ರ ಕಂಡು ಬರುತ್ತವೆ. ಜಿಲ್ಲೆಯ ಶಾಸನಗಳ ಅಧ್ಯಯನದಿಂದ ಇಂತಹ ಅಜ್ಞಾತ ವ್ಯಕ್ತಿಗಳ, ಸಣ್ಣ ಸಣ್ಣ ರಾಜಮನೆತನಗಳ, ಮಾಂಡಲಿಕರ ಇತಿಹಾಸದ ಮೇಲೆ ಹೊಸ ಬೆಳಕನ್ನು ಚೆಲ್ಲಬಹುದಾಗಿದೆ.

\textbf{ಮಂಡ್ಯ ಜಿಲ್ಲೆಯು ಐತಿಹಾಸಿಕವಾಗಿ ಮತ್ತು ಸಾಂಸ್ಕೃತಿಕವಾಗಿ ಅತ್ಯಂತ ಸಂಪದ್ಭರಿತ ಜಿಲ್ಲೆಯಾಗಿದೆ. ಸಾವಿರಕ್ಕೂ ಹೆಚ್ಚು ಶಾಸನಗಳು ಜಿಲ್ಲೆಯಲ್ಲಿವೆ. ಮಂಡ್ಯ ಜಿಲ್ಲೆಯಲ್ಲಿ ಇದುವರೆಗೆ ಲಭ್ಯವಾಗಿ ಪ್ರಕಟವಾಗಿರುವ ಎಲ್ಲ ಶಾಸನಗಳನ್ನೂ ಸಾಧ್ಯವಾದ ಮಟ್ಟಿಗೆ ವಿಶ್ಲೇಷಣೆಗೆ ಒಳಪಡಿಸಿ ಈ ಕೃತಿಯನ್ನು ರಚಿಸಲಾಗಿದೆ. ಅದರ ಜೊತೆಗೆ ಜಿಲ್ಲೆಯ ಸಾಮಾಜಿಕ ಮತ್ತು ಸಾಂಸ್ಕೃತಿ ವಿಚಾರಗಳನ್ನು ಉಲ್ಲೇಖಿಸಲಾಗಿದೆ. } ಆಧುನಿಕ ಆಡಳಿತ ಘಟಕವಾಗಿ ಮಂಡ್ಯ ಜಿಲ್ಲೆಯು ರೂಪುಗೊಳ್ಳುವುದಕ್ಕೆ ಮುನ್ನ, ಅಕ್ಕಪಕ್ಕದ ಜಿಲ್ಲೆಗಳ ಕೆಲವು ಪ್ರದೇಶಗಳಲ್ಲಿರುವ ಶಾಸನಗಳಲ್ಲಿ, ಮಂಡ್ಯ ಜಿಲ್ಲೆಗೆ ಸಂಬಂಧಿಸಿದ ಮಾಹಿತಿಗಳಿರುವುದರಿಂದ, ಮಂಡ್ಯ ಜಿಲ್ಲೆಯ ಗಡಿಗೆ ಹೊಂದಿಕೊಂಡಿರುವ ಅಕ್ಕಪಕ್ಕದ ಜಿಲ್ಲೆಗಳ ಶಾಸನಗಳನ್ನೂ ಕೂಡಾ ಈ ಕೃತಿಯ ಮೂಲ ಸಾಮಗ್ರಿಯನ್ನಾಗಿ ಬಳಸಿಕೊಳ್ಳಲಾಗಿದೆ. ಸಾಮಾಜಿಕ, ಧಾರ್ಮಿಕ ಮತ್ತು ಸಾಂಸ್ಕೃತಿಕ ವಿಚಾರಗಳ ವಿಶ್ಲೇಷಣೆಗೆ ಪೂರಕವಾಗಿ ರಾಜ್ಯದ ಅನೇಕ ಜಿಲ್ಲೆಗಳ ಸಂಬಂಧಿಸಿದ ಶಾಸನಗಳನ್ನೂ, ಈ ಬಗೆಗಿನ ಅಧ್ಯಯನಗಳನ್ನೂ ಕೂಡಾ ಸಂದರ್ಭೋಚಿತವಾಗಿ ಬಳಸಿಕೊಳ್ಳಲಾಗಿದೆ. ಜಿಲ್ಲೆಯ ಸುಮಾರು 200 ಕ್ಕೂ ಹೆಚ್ಚು ಹಳ್ಳಿಗಳಿಗೆ ಭೇಟಿ ನೀಡಿ ಶಾಸನಗಳು, ದೇವಾಲಯಗಳು ಮತ್ತೂ ಸ್ಮಾರಕಗಳನ್ನು ವೀಕ್ಷಣೆ ಮಾಡಿ ಸ್ಥಳ ಪರಿಶೀಲನೆ ನಡೆಸಲಾಗಿದೆ. ಮೇಲೆ ಉಲ್ಲೇಖಿಸಿದ ಕೃತಿಗಳ ಜೊತೆಗೆ, ಮಂಡ್ಯ ಜಿಲ್ಲೆಯ ಬಗ್ಗೆ ಲಭ್ಯವಾದ ಸುಮಾರು 25-30 ಕೃತಿಗಳನ್ನು ಪರಿಶೀಲಿಸಲಾಗಿದೆ.

\begin{center}
***
\end{center}

\theendnotes

