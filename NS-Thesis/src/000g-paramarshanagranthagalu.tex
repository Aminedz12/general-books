\chapter*{ಪರಾಮರ್ಶನ ಗ್ರಂಥಗಳು}
\rhead[]{ಪರಾಮರ್ಶನ ಗ್ರಂಥಗಳು\quad\small\thepage}

\begin{center}
\textbf{ಮೂಲ ಆಕರ ಗ್ರಂಥಗಳು, (ಪ್ರೈಮರಿ ಸೋರ್ಸಸ್​)}
\end{center}


\section*{ಶಾಸನ ಸಂಪುಟಗಳು:}

\noindent
ಎಪಿಗ್ರಾಫಿಯಾ ಕರ್ನಾಟಿಕಾ, ಸಂಪುಟ 6, ಕೃಷ್ಣರಾಜಪೇಟೆ, ಪಾಂಡವಪುರ, ಶ‍್ರೀರಂಗಪಟ್ಟಣ ತಾಲ್ಲೂಕುಗಳು 

\noindent
ಎಪಿಗ್ರಾಫಿಯಾ ಕರ್ನಾಟಿಕಾ, ಸಂಪುಟ 7, ನಾಗಮಂಗಲ, ಮಂಡ್ಯ, ಮದ್ದೂರು, ಮಳವಳ್ಳಿ ತಾಲ್ಲೂಕುಗಳು

\noindent
ಎಪಿಗ್ರಾಫಿಯಾ ಕರ್ನಾಟಿಕಾ, ಸಂಪುಟ 1 ರಿಂದ 15 ಹಾಗೂ 24, 

\noindent
ಪ್ರಕಾಶಕರು: ಕುವೆಂಪು ಕನ್ನಡ ಅಧ್ಯಯನ ಸಂಸ್ಥೆ, ಮೈಸೂರು ವಿಶ್ವವಿದ್ಯಾನಿಲಯ, ಮೈಸೂರು. (*ಎಕ), 1974 ರಿಂದ 2015ರವರೆಗೆ

\noindent
ಎಪಿಗ್ರಾಫಿಯಾ ಕರ್ನಾಟಿಕಾ, ಬಿ.ಎಲ್​.ರೈಸ್​, ಹಳೆಯ ಸಂಪುಟಗಳು (*ಇಸಿ)

\noindent
ಕನ್ನಡ ವಿಶ್ವವಿದ್ಯಾನಿಲಯದ ಶಾಸನ ಸಂಪುಟಗಳು, ಸಂ: ಡಾ. ದೇವರಕೊಂಡಾರೆಡ್ಡಿ. ಹಂಪಿ ಕನ್ನಡ ವಿಶ್ವವಿದ್ಯಾನಿಲಯ,\break ವಿದ್ಯಾರಣ್ಯ, ಹಂಪಿ

\noindent
ಆಂಧ್ರಪ್ರದೇಶದ ಕನ್ನಡ ಶಾಸನಗಳು, ಸಂಪುಟ 1 ಮತ್ತು 2, ಸಂ: ಡಾ. ಕೆ.ಆರ್​. ಗಣೇಶ್​, ಹಂಪಿ ಕನ್ನಡ ವಿಶ್ವವಿದ್ಯಾನಿಲಯ, ಹಂಪಿ,

\noindent
ರಾಷ್ಟ್ರಕೂಟರ ಶಾಸನಗಳು, ಸಂಪುಟ 1 ಮತ್ತು 2, ಸಂ: ಡಾ. ಬಿ.ಆರ್​. ಭಾರತಿ, ಕುವೆಂಪು ಕನ್ನಡ ಅಧ್ಯಯನ ಸಂಸ್ಥೆ, ಮೈಸೂರು. ವಿ.ವಿ., ಮೈಸೂರು

\noindent
ಕಲ್ಯಾಣ ಚಾಲುಕ್ಯರ ಶಾಸನಗಳು, ಸಂಪುಟ 1, ಸಂ. ಜೆ.ಎಮ್.ನಾಗಯ್ಯ, ಸಂ: ಕುವೆಂಪು ಕನ್ನಡ ಅಧ್ಯಯನ ಸಂಸ್ಥೆ, ಮೈಸೂರು,

\noindent
ಕರ್ನಾಟಕದ ಪರ್ಶಿಯನ್​, ಅರೇಬಿಕ್​ ಮತ್ತು ಉರ್ದು ಶಾಸನಗಳು, ಸಂ: ಸೀತಾರಾಮ ಜಾಗಿರ್​ದಾರ್​, ಯಾಸೀನ್​\break ಖದ್ದೂಸಿ,ಪ್ರಸಾರಾಂಗ, ಕನ್ನಡ ವಿಶ್ವವಿದ್ಯಾನಿಲಯ, ಹಂಪಿ, 2001

\noindent
ಮೈಸೂರು ಆರ್ಕಿಯೋಲಾಜಿಕಲ್​ ರಿಪೋರ್ಟ್ಸ್

\noindent
\textbf{*ಪರಿಷ್ಕೃತ ಸಂಪಟಗಳನ್ನು ಎ.ಕ. ಎಂಬ ಸಂಕೇತಾಕ್ಷರಗಳ ಮೂಲಕ, ಹಳೆಯ ಸಂಪುಟಗಳನ್ನು ಇಸಿ ಸಂಕೇತಾಕ್ಷರದ ಮೂಲಕ ಸೂಚಿಸಲಾಗಿದೆ.}


\section*{ಹಳಗನ್ನಡ ಕಾವ್ಯಗಳು}

\noindent
ಚಾವುಂಡರಾಯ ವಿರಚಿತ ಚಾವುಂಡರಾಯ ಪುರಾಣಂ(ತ್ರಿಷಷ್ಟಿ ಲಕ್ಷಣ ಮಹಾಪುರಾಣಂ) ಸಂ: ಡಾ. ಕಮಲಾ ಹಂಪನಾ, ಕೆ.ಆರ್​.ಶೇಷಗಿರಿ, ಕನ್ನಡ ಸಾಹಿತ್ಯ ಪರಿಷತ್ತು, ಬೆಂಗಳೂರು 1983

\noindent
ಜನ್ನ ಕವಿಯ ಅನಂತನಾಥ ಪುರಾಣಂ,

\noindent
ತಿಮ್ಮ ಕವಿ ವಿರಚಿತ ಪಶ್ಚಿಮ ರಂಗ ಕ್ಷೇತ್ರ ಮಾಹಾತ್ಮ್ಯಂ ಮತ್ತು ಚಿಕದೇವರಾಯ ವಂಶಾವಳಿ, ಸಂ: ಎಂ.ಪಿ. ಮಂಜಪ್ಪ ಶೆಟ್ಟಿ, ಕುವೆಂಪು ಕನ್ನಡ ಅಧ್ಯಯನ ಸಂಸ್ಥೆ, ಮೈಸೂರು ವಿಶ್ವವಿದ್ಯಾನಿಲಯ, ಮೈಸೂರು, 1978

\noindent
ತಿಮ್ಮ ಕವಿ ವಿರಚಿತ, ಯಾದವಗಿರಿ ಮಾಹಾತ್ಮ್ಯಂ, ಸಂ. ಎಂ.ಪಿ. ಮಂಜಪ್ಪ ಶೆಟ್ಟಿ, ಕುವೆಂಪು ಕನ್ನಡ ಅಧ್ಯಯನ ಸಂಸ್ಥೆ, ಮೈಸೂರು ವಿಶ್ವವಿದ್ಯಾನಿಲಯ, ಮೈಸೂರು, 1978

\noindent
ಗೋವಿಂದ ವೈದ್ಯ ಕೃತ ಕಂಠೀರವ ನರಸರಾಜ ವಿಜಯ, ಸಂ. ಆರ್​. ಶಾಮ ಶಾಸ್ತ್ರಿ, ಕನ್ನಡ ಅಧ್ಯಯನ ಸಂಸ್ಥೆ, ಮೈಸೂರು ವಿಶ್ವವಿದ್ಯಾನಿಲಯ, ಮೈಸೂರು, 1971

\noindent
ರುದ್ರಭಟ್ಟನ ಜಗನ್ನಾಥ ವಿಜಯಂ, ಗದ್ಯಾನುವಾದ: ಆಸ್ಥಾನ ವಿದ್ವಾನ್​ ಎಂ.ಆರ್​.ವರದಾಚಾರ್ಯ, ಕನ್ನಡ ಸಾಹಿತ್ಯ ಪರಿಷತ್ತು, 1999


\section*{ಪತ್ರಿಕೆಗಳು-ವಾರ್ಷಿಕ ಸಂಚಿಕೆಗಳು}

\noindent
ಮಾನವಿಕ ಕರ್ನಾಟಕ, ಬೆಳ್ಳಿ ಸಂಪುಟ, ಪ್ರಸಾರಾಂಗ, ಮೈಸೂರು, ವಿವಿ. 1994

\noindent
ಇತಿಹಾಸ ದರ್ಶನ, ಕರ್ನಾಟಕ ಇತಿಹಾಸ ಅಕಾಡೆಮಿಯ ವಾರ್ಷಿಕ ಸಂಚಿಕೆಗಳು, ಕರ್ನಾಟಕ ಇತಿಹಾಸ ಅಕಾಡೆಮಿ, ಬಿ.ಎಂ.ಶ‍್ರೀ. ಪ್ರತಿಷ್ಠಾನ, ನರಸಿಂಹರಾಜಾ ಕಾಲೋನಿ, ಬೆಂಗಳೂರು

\noindent
ಮಂಡ್ಯ ಜಿಲ್ಲೆ ಗ್ಯಾಸೆಟಿಯಾರ್​ (ಪರಿಷ್ಕೃತ ಆವೃತ್ತಿ), ಮುಖ್ಯ ಸಂಪಾದಕರು: ಟಿ.ಎ. ಪಾರ್ಥಸಾರಥಿ, ಕರ್ನಾಟಕ ಸರ್ಕಾರ, 2003

\noindent
ಮೈಸೂರು ಆರ್ಕಿಯಾಲೋಜಿಕಲ್​ ರಿಪೋರ್ಟ್ 1916


\section*{ಅನುಷಂಗಿಕ ಆಧಾರ ಗ್ರಂಥಗಳು (ಸೆಕೆಂಡರಿ ಸೋರ್ಸಸ್​)}

\begin{longtable}[l]{@{}>{\raggedright}p{4.7cm}cp{9.2cm}<{\raggedright}@{}}
ಅರ್ಚಕ ಬಿ. ರಂಗಸ್ವಾಮಿ & : & ಹುಟ್ಟಿದಹಳ್ಳಿ, ಕುವೆಂಪು ಕನ್ನಡ ಅಧ್ಯಯನ ಸಂಸ್ಥೆ, ಮೈಸೂರು, ವಿ.ವಿ. 2008\\
ಅನಂತರಂಗಾಚಾರ್​, ಡಾ. ಎನ್​. & : & ರಾ. ನರಸಿಂಹಾಚಾರ್ಯರ ಜೀವನ ಮತ್ತು ಕಾರ್ಯ, ಕನ್ನಡ ಸಾಹಿತ್ಯ ಪರಿಷತ್ತು, ಬೆಂಗಳೂರು, 1999\\
ಅನಂತರಾಮು, ಡಾ. ಕೆ. & : & ಸಕ್ಕರೆಯ ಸೀಮೆ, ಕನ್ನಡ ಪುಸ್ತಕ ಪ್ರಾಧಿಕಾರ, ಬೆಂಗಳೂರು, 2004\\
ಅಪರ್ಣ, ಡಾ. ಸಿ.ಎಸ್​. & : & ಸಂಗಮರ ಕಾಲದ ದೇವಾಲಯಗಳು, ಕರ್ನಾಟಕ ಇತಿಹಾಸ ಸಂಶೋಧನಾ ಮಂಡಲ, ಧಾರವಾಡ, 2008\\
ಅಪ್ಪಾಜಿಗೌಡ, ಅರ್ಜುನಪುರಿ, ಡಾ. & : & ಮದ್ದೂರು ತಾಲ್ಲೂಕು ದರ್ಶನ, ಮದ್ದೂರು ತಾಲ್ಲೂಕು ಕನ್ನಡ ಸಾಹಿತ್ಯ ಪರಿತಷತ್ತು, ಮದ್ದೂರು, 1999\\
ಇಮ್ಮಡಿ ಶಿವಬಸವ ಸ್ವಾಮಿಗಳು, ವಿದ್ವಾನ್​, & : & ಸಂಸ್ಕೃತ ಸಾಹಿತ್ಯಕ್ಕೆ ಕರ್ನಾಟಕದ ಕೊಡುಗೆ, ಸಂವಹನ, ಮೈಸೂರು, 1999\\[-15pt]
 & : & ಸಾಯಣ ಮಾಧವ ಪ್ರಣೀತ ಸರ್ವದರ್ಶನ ಸಂಗ್ರಹ,\newline ಜಗದ್ಗುರು ಶಿವರಾತ್ರೀಶ್ವರ ಗ್ರಂಥಮಾಲೆ, ಮೈಸೂರು, 2012\\
ಉಷಾ ರಾಣಿ, ಡಾ. ಎಚ್​.ಎಸ್​. & : & ಕ್ಷೇತ್ರಾಂತರಂಗ, ದುರ್ಗಾ ಪ್ರಕಾಶನ, ಮೈಸೂರು 2009\\
ಐತಾಳ, ಶಂ.ಸ. & : & ಕೃಷ್ಣರಾಜ ಸಾಗರ, ಐಬಿಎಚ್​. ಪ್ರಕಾಶನ, ಬೆಂಗಳೂರು, 1976\\
ಕಡಪಟ್ಟಿ, ಡಾ. ಎನ್​.ಎಮ್. & : & ಪ್ರಾಚೀನ ಕರ್ನಾಟಕದಲ್ಲಿ ಶೈವಧರ್ಮ, ಕರ್ನಾಟಕ ಇತಿಹಾಸ ಸಂಶೋಧನಾ ಮಂಡಲ, ಧಾರವಾಡ,2008\\
ಕಲಬುರ್ಗಿ, ಡಾ. ಎಂ.ಎಂ.  & : & ಐತಿಹಾಸಿಕ, ಕನ್ನಡ ಮತ್ತು ಸಂಸ್ಕೃತಿ ಇಲಾಖೆ, ಬೆಂಗಳೂರು, 1991\\
                 & : & ಕನ್ನಡ ಸಂಶೋಧನಾ ಶಾಸ್ತ್ರ, ಚೇತನ ಬುಕ್​ ಹೌಸ್​, ಮೈಸೂರು, 1995\\
                 & : & ಪ್ರಾಚೀನ ಕರ್ನಾಟಕದ ಆಡಳಿತ ವಿಭಾಗಗಳು (ಸಂ), ಪ್ರಸಾರಾಂಗ,\newline ಕನ್ನಡ ವಿವಿ., ವಿದ್ಯಾರಣ್ಯ, ಹಂಪಿ, 1999\\
                 & : & ಮಾರ್ಗ ಸಂಪುಟಗಳು, ಸಪ್ನ ಬುಕ್​ ಸ್ಟೋರ್ಸ್, ಬೆಂಗಳೂರು \\
                 & : & ಶಾಸನಗಳಲ್ಲಿ ಶಿವಶರಣರು, ವೀರಶೈವ ಅಧ್ಯಯನ ಸಂಸ್ಥೆ,\newline ಡಂಬಳ, ಗದಗ, 1978\\
                 & : & ಶಾಸನ ಸೂಕ್ತಿ ಸುಧಾರ್ಣವ, ಕರ್ನಾಟಕ ಸಾಹಿತ್ಯ ಅಕಾಡೆಮಿ, 1977\\
                 & : & ಸಮಾಧಿ, ಬಲಿದಾನ, ವೀರಮರಣ ಸ್ಮಾರಕಗಳು, ಐಬಿಎಚ್​. ಪ್ರಕಾಶನ, ಬೆಂಗಳೂರು, 1980\\
ಕಲೀಂ ಉಲ್ಲಾ ಮಹಮದ್ & ​: & ಶಾಸನ ಪರಿಚಯ, ಮಳವಳ್ಳಿ ತಾಲ್ಲೂಕು, ಮುಜೀಬ್ ಪ್ರಕಾಶನ, ನಾಗಮಂಗಲ\\
& : & ಶಾಸನ ಪರಿಚಯ, ನಾಗಮಂಗಲ ತಾಲ್ಲೂಕ, ಮುಜೀಬ್ ಪ್ರಕಾಶನ, ನಾಗಮಂಗಲ\\
& : & ಬೆಳ್ಳೂರಿನ ಶಾಸನಗಳು (ಫೋಲ್ಡರ್​)\\
ಕುಮಾರ ಸ್ವಾಮಿ, ಸಿ.ಎಸ್​. & : & ಕರ್ನಾಟಕದ ಜಾತ್ರೆಗಳು, ಕನ್ನಡ ವಿಭಾಗ, ಮೈಸೂರು ವಿ.ವಿ. ಸ್ನಾತಕೋತ್ತರ ಅಧ್ಯಯನ ಕೇಂದ್ರ, ಬಿ.ಆರ್​. ಪ್ರಾಜೆಕ್ಟ್​, 1988\\
ಕುಮಾರಸ್ವಾಮಿ, ಡಾ. ಎಸ್​.ಕೆ. & : & ಪ್ರಾಚೀನ ಕರ್ನಾಟಕದಲ್ಲಿ ಶಿಲ್ಪಾಚಾರಿಯರು,\newline ಕನ್ನಡ ಸಾಹಿತ್ಯ ಪರಿಷತ್ತು, 1996\\
ಕುಲಕರ್ಣಿ, ಡಾ. ಪಿ.ವಿ. & : & ತಮಿಳಿನಲ್ಲಿ ಕನ್ನಡದ ವೀರಶೈವ ಕೃತಿಗಳು, ಪ್ರಭಸ ಬಿಡುಗಡೆ, ಮಧುರೈ\\
ಕೂಡಲೂರು ವೆಂಕಟಪ್ಪ, ಡಾ. & : &  ಸ್ಥಳನಾಮ ವೀರರು, ಕನ್ನಡ ಸಾಹಿತ್ಯ ಪರಿಷತ್ತು, ಬೆಂಗಳೂರು 2007\\
ಕೃಷ್ಣಮೂರ್ತಿ, ಡಾ. ಪಿ.ವಿ. & : &  ಬಾಣರಸರ ಶಾಸನಗಳು-ಒಂದು ಅಧ್ಯಯನ, ಸಿವಿಜಿ ಪಬ್ಲಿಕೇಷನ್ಸ್​, ಬೆಂಗಳೂರು, 2009\\
                              & :  & ತಮಿಳು ನಾಡಿನ ಕನ್ನಡ ಶಾಸನಗಳು, ಕನ್ನಡ ಸಾಹಿತ್ಯ ಪರಿಷತ್ತು, ಬೆಂಗಳೂರು\\
               & : & ಶಾಸನ ಮಂಥನ, ಪ್ರಸಾರಾಂಗ, ತುಮಕೂರು ವಿ.ವಿ. ತುಮಕೂರು\\
 ಕೃಷ್ಣರಾವ್​ ಡಾ. ಎಂ.ವಿ.,\newline ಕೇಶವಭಟ್ಟ. ಟಿ. & : & ಕರ್ನಾಟಕ ಇತಿಹಾಸ ದರ್ಶನ, ಕರ್ನಾಟಕ ಸಹಕಾರಿ ಪ್ರಕಾಶನ ಮಂದಿರ, ಬೆಂಗಳೂರು, 1970\\
ಕೃಷ್ಣರಾವ್​ ಕಪಟರಾಳ & : & ಕರ್ನಾಟಕ ಸಂಸ್ಕೃತಿಯ ಸಂಶೋಧನೆ, ಉಷಾ ಸಾಹಿತ್ಯ ಮಾಲೆ,\newline ಮೈಸೂರು, 1970\\
ಕೃಷ್ಣ ಶರ್ಮ ಬೆಟಗೇರಿ & : & ಕರ್ನಾಟಕ ಜನಜೀವನ, ಸಾಗರ್​ ಪ್ರಕಾಶನ, ಬೆಂಗಳೂರು, 2005\\
ಕೆಂಪೇಗೌಡ ಶಿವಳ್ಳಿ & : & ಗ್ರಾಮ ದರ್ಶನ, ಜಿಲ್ಲಾ ಕನ್ನಡ ಸಾಹಿತ್ಯ ಪರಿಷತ್ತು, ಮಂಡ್ಯ 2010\\
ಕೇಶವರಾವ್​ ಮನ್ನೆಕೋಟೆ & : & ಮಂಡ್ಯ ಜಿಲ್ಲೆ ದರ್ಶನ, ಐಬಿಎಚ್​. ಪ್ರಕಾಶನ, ಬೆಂಗಳೂರು, 1978\\
ಗಣೇಶ್​, ಡಾ. ಕೆ.ಆರ್​. & : & ಆಂಧ್ರ ಪ್ರದೇಶದ ಕನ್ನಡ ಶಾಸನಗಳು, ಕನ್ನಡ ಸಾಹಿತ್ಯ ಪರಿಷತ್ತು, 1996\\
ಗೋಕುಲನಾಥ್​, ಡಾ. ಕೆ. & : &  ವ್ಯಾಸರಾಯರು, ಪ್ರಸಾರಾಂಗ, ಕನ್ನಡ ವಿ.ವಿ., ಹಂಪಿ. 2002\\
ಗೋಪಾಲ್​, ಡಾ. ಆರ್​. & : &  (ಸಂ) ಮಂಡ್ಯ ಜಿಲ್ಲೆಯ ಇತಿಹಾಸ ಮತ್ತು ಪುರಾತತ್ವ, ಪ್ರಾಚ್ಯವಸ್ತು ಮತ್ತು ಸಂಗ್ರಹಾಲಯಗಳ ನಿರ್ದೇಶನಾಲಯ, ಮೈಸೂರು, 2008\\
ಗೋಪಾಲ್​, ಡಾ. ಬಾ.ರಾ. & : & ಕರ್ನಾಟಕದಲ್ಲಿ ಶ‍್ರೀ ರಾಮಾನುಜಾಚಾರ್ಯರು, ಪ್ರಸಾರಾಂಗ, ಮೈಸೂರು ವಿ.ವಿ. 1984\\
ಗೋಪಾಲ ರಾವ್​, ಡಾ. ಎಚ್​.ಎಸ್​. & : & ಶಾಸನಗಳ ಹಿನ್ನೆಲೆಯಲ್ಲಿಕಲ್ಯಾಣದ ಚಾಲುಕ್ಯರ ದೇವಾಲಯಗಳು,\newline ಪ್ರಾಚ್ಯ ವಸ್ತು ಮತ್ತು ಸಂಗ್ರಹಾಲಯಗಳ ನಿರ್ದೇಶನಾಲಯ,\newline ಕರ್ನಾಟಕ ಸರ್ಕಾರ, ಮೈಸೂರು, 1993\\
& : & ಚೆಂಗಾಳ್ವರು, ಪ್ರಸಾರಾಂಗ, ಕನ್ನಡ ವಿಶ್ವವಿದ್ಯಾನಿಲಯ, ಹಂಪಿ, 1998\\
& : & ನಮ್ಮ ನಾಡು ಕರ್ನಾಟಕ, ನವಕರ್ನಾಟಕ ಪ್ರಕಾಶನ, ಬೆಂಗಳೂರು, 1997\\
ಗುರುರತ್ನ ಬಾಬು, ಡಾ. & : &  ತಲಕಾಡಿನ ಗಂಗರಸರ ಸಾಂಸ್ಕೃತಿಕ ಪರಂಪರೆ, ಸಂಜಯ ಪ್ರಕಾಶನ, ಮೈಸೂರು, 2007\\
ಚೆನ್ನಕ್ಕ ಎಲಿಗಾರ, ಡಾ. & : &  ಶಾಸನಗಳಲ್ಲಿ ಕರ್ನಾಟಕದ ಸ್ತ್ರೀ ಸಮಾಜ, ಪ್ರಸಾರಾಂಗ, ಕರ್ನಾಟಕ ವಿ.ವಿ. ಧಾರವಾಡ, 1990\\
& : & ಬಂಕಾಪುರ ಶೋಧನೆ, ಚನ್ನಗಂಗಾ ಪ್ರಕಾಶನ, ಧಾರವಾಡ, 1990\\
& : & ಮಾಸ್ತಿ ಕಲ್ಲು, ಪ್ರಸಾರಾಂಗ, ಕರ್ನಾಟಕ ವಿ.ವಿ., ಧಾರವಾಡ\\
ಚಿತ್ತಯ್ಯ ಪೂಜಾರ್​, ಡಾ. ಡಿ. & : &  ಕನ್ನಡ ಶಾಸನಾಧ್ಯಯನದ ಬಹುಮುಖಿ ಆಯಾಮಗಳು, ಪ್ರಸಾರಂಗ,\newline ಕನ್ನಡ ವಿ.ವಿ. ಹಂಪಿ. 2010\\
ಚಿದಾನಂದ ಮೂರ್ತಿ, ಡಾ. ಎಂ.  & : & (ಸಂ) ಅಧ್ಯಯನ, ಡಾ.ಶಂ.ಬಾ. ಜೋಷಿ, ಅಭಿನಂದನಾ ಗ್ರಂಥ, ಶಂ.ಬಾ.ಜೋಷಿ, ಸನ್ಮಾನ ಸಮಿತಿ, ಬೆಂಗಳೂರು, 1980\\
& : & ಕನ್ನಡ ಶಾಸನಗಳ ಸಾಂಸ್ಕೃತಿಕ ಅಧ್ಯಯನ, ಪ್ರಸಾರಾಂಗ, ಮೈಸೂರು. ವಿ.ವಿ. ಮೈಸೂರು, 1979\\
& : & ಪಾಂಡಿತ್ಯ ರಸ, ಪ್ರಸಾರಾಂಗ, ಕನ್ನಡ ವಿಶ್ವವಿದ್ಯಾನಿಲಯ, ಹಂಪಿ, 2000\\
& : & ವೀರಶೈವಧರ್ಮ ಮತ್ತು ಭಾರತೀಯ ಸಂಸ್ಕೃತಿ, ಮಿಂಚು ಪ್ರಕಾಶನ, ಬೆಂಗಳೂರು, 2000\\
& : &  ಸಂಶೋಧನೆ, ಕನ್ನಡ ಸಾಹಿತ್ಯ ಪರಿಷತ್ತು, ಬೆಂಗಳೂರು, 1984\\
& : & ಸಂಶೋಧನಾ ತರಂಗ, ಪ್ರಸಾರಾಂಗ, ಬೆಂಗಳೂರು ವಿಶ್ವವಿದ್ಯಾನಿಲಯ\\
& : & ಹೊಸತು ಹೊಸತು, ಕನ್ನಡ ಪುಸ್ತಕ ಪ್ರಾಧಿಕಾರ, ಬೆಂಗಳೂರು,\\
ಜವರೇಗೌಡ, ಡಾ. ದೇ. (ದೇಜಗೌ) & : & ಕನ್ನಡ ಸಂಸ್ಕೃತಿ, ಕನ್ನಡ ಸಾಹಿತ್ಯ ಪರಿಷತ್ತು, ಬೆಂಗಳೂರು, 2010\\
ಜವರೇಗೌಡ, ಡಾ. ದೇ.(ದೇಜಗೌ) (ಪ್ರ.ಸಂ.) ರಾಜೇಗೌಡ, ಹ.ಕ.(ಸಂ) & : & ಸಿರಿಯೊಡಲು, ೬೩ನೇ ಅಖಿಲ ಭಾರತ ಕನ್ನಡ ಸಾಹಿತ್ಯ ಸಮ್ಮೇಳನ ಸ್ಮರಣ ಸಂಚಿಕೆ, ಮಂಡ್ಯ, ಜಿಲ್ಲಾ ಕನ್ನಡ ಸಾಹಿತ್ಯ ಪರಿಷತ್ತು, ಮಂಡ್ಯ, 1994\\
& : & ಸುವರ್ಣ ಮಂಡ್ಯ, ಮಂಡ್ಯ ಜಿಲ್ಲೆಯ ಸ್ಥಾಪನೆಯ ೫೦ನೇ ವರ್ಷದ ಸ್ಮರಣ ಸಂಚಿಕೆ ಸುವರ್ಣ ಮಹೋತ್ಸವ ಸಮಿತಿ, ಮಂಡ್ಯ.\\
ಜಯಮ್ಮ ಕರಿಯಣ್ಣ, ಡಾ. & : & ನೊಳಂಬರ ಶಾಸನಗಳ ಸಾಂಸ್ಕೃತಿಕ ಅಧ್ಯಯನ, ಕರ್ನಾಟಕ ಇತಿಹಾಸ ಸಂಶೋಧನಾ ಮಂಡಲ ಧಾರವಾಡ, 2007-2008\\
& : & ಶಾಸನಗಳಲ್ಲಿ ದಾನ ದತ್ತಿಗಳು, ಕನ್ನಡ ಸಾಹಿತ್ಯ ಪರಿಷತ್ತು, 2010\\
ಜೈರಾಮ್, ಬೋರಾಪುರ & : &  ಆಚಾರ್ಯ ರಾಮಾನುಜ, ಕನ್ನಡ ಸಾಹಿತ್ಯ ಪರಿಷತ್ತು, 2010\\
ದಾಸೇಗೌಡ, ಡಾ. ಜಿ.ವಿ. & : &  ಮಂಡ್ಯ ಜಿಲ್ಲೆಯ ಜಾತ್ರೆಗಳು, ಕನ್ನಡ ಪುಸ್ತಕ ಪ್ರಾಧಿಕಾರ, ಬೆಂಗಳೂರು, 2000\\
ದೀಕ್ಷಿತ್​, ಡಾ. ಜಿ.ಎಸ್​. & : &  (ಸಂ) ಕರ್ನಾಟಕದಲ್ಲಿ ಕೆರೆ ನೀರಾವರಿ, ಪ್ರಸಾರಾಂಗ,\newline ಕನ್ನಡ ವಿ.ವಿ. ಹಂಪಿ, 2006\\
ದೀಕ್ಷಿತ್​, ಡಾ. ಜಿ.ಎಸ್​. ವಿಶ್ವೇಶ್ವರ, ಎಂ.ವಿ. & : &  (ಸಂ) ಸಂಗಮರ ಕಾಲದ ವಿಜಯನಗರ, ಬಿ.ಎಂ.ಶ‍್ರೀ. ಪ್ರತಿಷ್ಠಾನ, ಬೆಂಗಳೂರು 1988\\
ದೇವರಕೊಂಡಾ ರೆಡ್ಡಿ, ಡಾ. & : & ಕರ್ನಾಟಕ ಶಾಸನಗಳಲ್ಲಿ ಶಾಪಾಶಯ, ಕರ್ನಾಟಕ ಇತಿಹಾಸ ಸಂಶೋಧನಾ ಮಂಡಲ,ಧಾರವಾಡ, 2008\\
& : & ಗಂಗ ಶಿಲ್ಪಕಲೆ, ಕರ್ನಾಟಕ ಶಿಲ್ಪ ಕಲಾ ಅಕಾಡೆಮಿ, ಬೆಂಗಳೂರು, 2008\\
& : & ತಲಕಾಡಿನ ಗಂಗರ ದೇವಾಲಯಗಳು, ಕನ್ನಡ ಸಾಹಿತ್ಯ ಪರಿಷತ್ತು, ಬೆಂಗಳೂರು, 1989\\
& : & ಶ್ರವಣಬೆಳಗೊಳದ ಬಸದಿಗಳ ವಾಸ್ತು ಶಿಲ್ಪ, ಚಂದ್ರಗುಪ್ತ ಗ್ರಂಥಮಾಲೆ, ಜೈನಮಠ, ಶ್ರವಣಬೆಳಗೊಳ, 1993\\
& : & ಶಾಸನ ಅಧ್ಯಯನ, ಸಂಪುಟ 2, ಪ್ರಸಾರಾಂಗ, ಕನ್ನಡ ವಿ.ವಿ. ಹಂಪಿ, 2004\\
ದೇಸಾಯಿ, ಡಾ. ಪಿ.ಬಿ. & : & ವಿಜಯನಗರ ಸಾಮ್ರಾಜ್ಯ, ಕರ್ನಾಟಕ ಇತಿಹಾಸ ಸಂಶೋಧನಾ ಮಂಡಲ, ಧಾರವಾಡ, 2008\\
ದೇಶಪಾಂಡೆ, ಡಾ. ಎಲ್​.ಎಸ್​. & : &  ಕರ್ನಾಟಕದ ಸ್ಥಳ ನಾಮಗಳು-ಒಂದು ಅಧ್ಯಯನ, ಕರ್ನಾಟಕ ಇತಿಹಾಸ. ಸಂ.ಮಂ., ಧಾರವಾಡ, 2008\\
ನರಸಿಂಹಾಚಾರ್​, ಡಾ. ಪು.ತಿ. & : &  ಮೇಲುಕೋಟೆ, ಐಬಿಎಚ್​ ಪ್ರಕಾಶನ, ಬೆಂಗಳೂರು, 1988\\
ನರಸಿಂಹಾಚಾರ್ಯ, ಡಾ. ರಾವ್​ಬಹದ್ದೂರ್​ & : &  ಕರ್ನಾಟಕ ಕವಿಚರಿತೆ, ಸಂಪುಟ 1, 2 ಮತ್ತು 3,\newline ಕನ್ನಡ ಸಾಹಿತ್ಯ ಪರಿಷತ್ತು, 2005\\
              & : & ಶಾಸನ ಪದ್ಯ ಮಂಜರಿ, ಪರಿಷ್ಕರಣೆ: ಡಾ. ಎಂ. ಚಿದಾನಂದ ಮೂರ್ತಿ, ಪ್ರಸಾರಾಂಗ, ಬೆಂಗಳೂರು ವಿಶ್ವವಿದ್ಯಾನಿಲಯ, 1975\\
ನರಸಿಂಹಮೂರ್ತಿ, ಡಾ.ಎ.ವಿ. & : & ಕರ್ನಾಟಕ ನಾಣ್ಯ ಪರಂಪರೆ, ಕನ್ನಡ ಪುಸ್ತಕ ಪ್ರಾಧಿಕಾರ, ಬೆಂಗಳೂರು, 2003\\
& : & ಕರ್ನಾಟಕ ನಾಣ್ಯ ಶಾಸ್ತ್ರ ಅಧ್ಯಯನ, ತ.ವೆಂ. ಸ್ಮಾರಕ ಗ್ರಂಥಮಾಲೆ, ಮೈಸೂರು, 2010\\
ನಂಜೇಗೌಡ, ಹೆಚ್​. & : & ಕನ್ನಡ ಸಾಹಿತ್ಯ ಚರಿತ್ರೆ, ಚೇತನ ಪುಸ್ತಕಾಲಯ, ಮೈಸೂರು, 2000\\
ನಂದೀಮಠ, ಡಾ.ಶಿ.ಚೆ. & : &  ಕನ್ನಡ ನಾಡಿನ ಚರಿತ್ರೆ, ಭಾಗ-೨, ಕನ್ನಡ ನಾಡಿನ ಧರ್ಮಗಳು, ಕನ್ನಡ ಸಾಹಿತ್ಯ ಪರಿಷತ್ತು, 1975\\
ನಾಗರಾಜ್​, ಹು.ಭೀ. & : & ಹುಲ್ಲಹಳ್ಳಿ ದರ್ಶನ, ತೃಣಪುರಿ ಪ್ರಕಾಶನ, ಹುಲ್ಲ ಹಳ್ಳಿ, ನಂಜನಗೂಡು ತಾಲ್ಲೂಕು, 1988\\[2pt]
ನಾಗರಾಜ ಅಯ್ಯಂಗಾರ್​ ಸ್ಥಾನೀಕಂ & : &  ಮೇಲುಕೋಟೆ, ಸ್ಥಾನೀಕಂ ಪ್ರಕಾಶನ, ಮೇಲುಕೋಟೆ, 1970\\[2pt]
ನಾಗರಾಜು, ಡಾ. ಎಂ.ಎಚ್​. & : & ನಾಗಮಂಗಲ ತಾಲ್ಲೂಕಿನ ಮಹತ್ವದ ಸ್ಥಳಗಳು, ಪ್ರಸಾರಾಂಗ,\newline ಮೈಸೂರು. ವಿ.ವಿ. 2012\\[2pt]
ನಾಗರಾಜಯ್ಯ, ಡಾ. ಹಂ.ಪ. & : &  ಕರ್ನಾಟಕ ಮತ್ತು ಜೈನಧರ್ಮ, ಐ.ಬಿ.ಎಚ್​. ಪ್ರಕಾಶನ, ಬೆಂಗಳೂರು 1981\\[2pt]
& : & ಚಂದ್ರಕೊಡೆ, ಪ್ರಸಾರಾಂಗ, ಕನ್ನಡ ವಿಶ್ವವಿದ್ಯಾನಿಲಯ, ಹಂಪಿ, 1997\\[2pt]
& : & ಶಾಸನಗಳಲ್ಲಿ ಜೈನತೀರ್ಥಗಳು, ಪಂಡಿತರತ್ನಂ ಎ.ಶಾಂತಿರಾಜ ಶಾಸ್ತ್ರೀ ಟ್ರಸ್ಟ್​(ರಿ), ಬೆಂಗಳೂರು, 1988\\[2pt]
& : & ಯಾಪನೀಯ ಸಂಘ, ಪ್ರಸಾರಾಂಗ, ಕನ್ನಡ ವಿಶ್ವವಿದ್ಯಾನಿಲಯ,\newline ಹಂಪಿ, 1999\\[2pt]
& : & ಶಾಸನಗಳಲ್ಲಿ ಬಸದಿಗಳು, ರತ್ನತ್ರಯ ಪ್ರಕಾಶನ, ಕುವೆಂಪು ನಗರ, ಮೈಸೂರು, 1998\\[2pt]
& : & ಶಾಸನಗಳಲ್ಲಿ ಎರಡು ವಂಶಗಳು, ಕನ್ನಡ ಅಧ್ಯಯನ ವಿಭಾಗ,\newline ಮುಂಬಯಿ ವಿಶ್ವವಿದ್ಯಾನಿಲಯ, 1995\\[2pt]
ನಾಗರಾಜರಾವ್​, ಎಂ.ಎಚ್​. & : &  ಶಾಸನ ಸಂಪದ, ಪ್ರಸಾರಾಂಗ, ಕರ್ನಾಟಕ ಮುಕ್ತ ವಿ.ವಿ., ಮೈಸೂರು, 2011\\
& : & ಶಿವ ಶೋಧ, ರಚನಾ ಪ್ರಕಾಶನ, ಮೈಸೂರು, 2009\\
ನಾಗಯ್ಯ, ಡಾ. ಜೆ.ಎಂ. & : & ಆರನೆಯ ವಿಕ್ರಮಾದಿತ್ಯನ ಶಾಸನಗಳು- ಒಂದು ಅಧ್ಯಯನ, (ಆಡಳಿತಕ್ಕೆ ಸಂಬಂಧಿಸಿದಂತೆ) ವೀರಶೈವ ಅಧ್ಯಯನ ಅಕಾಡೆಮಿ, ಶ‍್ರೀ ನಾಗನೂರು ರುದ್ರಾಕ್ಷಿ ಮಠ, ಬೆಳಗಾವಿ, 1992\\
ನಾಗೇಶ್​, ಡಾ. ಎಚ್​.ಎಂ. & : &  ಮದ್ದೂರು- ಒಂದು ಸಾಂಸ್ಕೃತಿಕ ಅಧ್ಯಯನ, (ಅಪ್ರಕಟಿತ ಪಿಎಚ್​.ಡಿ. ಪ್ರಬಂಧ) ಮೈಸೂರು ವಿವಿ. 2008\\
ನಾಯಕ್​, ಡಾ.ಹಾ.ಮಾ., ವೆಂಕಟಾಚಲ ಶಾಸ್ತ್ರೀ, ಡಾ. ಟಿ.ವಿ.,(ಸಂ) & : &  ಕನ್ನಡ ಸಾಹಿತ್ಯ ಚರಿತ್ರೆ ಸಂಪುಟಗಳು, 1 ರಿಂದ 4 ಕನ್ನಡ ಅಧ್ಯಯನ ಸಂಸ್ಥೆ, ಮೈಸೂರು ವಿಶ್ವವಿದ್ಯಾನಿಲಯ, ಮೈಸೂರು.\\[-12pt]
& : & ಡಿವಿಜಿ ಕೃತಿ ಶ್ರೇಣಿ, ಸಂಪುಟ 4, ಕನ್ನಡ ಮತ್ತು ಸಂಸ್ಕೃತಿ ನಿರ್ದೇಶನಾಲಯ,\newline ಬೆಂಗಳೂರು, 1994\\
ನಾರಾಯಣ, ಮ.ಸಿ. & : &  ಗಂಗರಾಜ ಮಡು, ಶ‍್ರೀ ಮಂಜು ಪ್ರಕಾಶನ, ಮಳವಳ್ಳಿ,\newline ಮಂಡ್ಯ ಜಿಲ್ಲೆ, 2010\\
ನಾರಾಯಣ, ಕೆ.ವಿ. & : & (ಸಂ) ಸ್ಥಳನಾಮಗಳು, ಪರಿವರ್ತನೆ ಮತ್ತು ಪ್ರಭಾವ, ಪ್ರಸಾರಾಂಗ,\newline ಕನ್ನಡ ವಿವಿ. ಹಂಪಿ. 1999\\
ನೇಗಿನಹಾಳ್​, ಡಾ.ಎಂ.ಬಿ. & : &  ನೇಗಿನಹಾಳ ಪ್ರಬಂಧಗಳು, ಪ್ರಸಾರಾಂಗ, ಕನ್ನಡ ವಿ.ವಿ. ಹಂಪಿ, 1999\\
ಪರಮಶಿವಮೂರ್ತಿ, ಡಾ. ಡಿ.ವಿ. & : & ತುರುಗೊಳ್​ ಸಂಕಥನ, ಪ್ರಸಾರಾಂಗ, ಕನ್ನಡ ವಿ.ವಿ., ಹಂಪಿ, 2010\\
& : & ಪೆಣ್ಬುಯ್ಯಲ್​, ಪ್ರಸಾರಾಂಗ, ಕನ್ನಡ ವಿ.ವಿ., ಹಂಪಿ, 2010\\
& : & ಶಾಸನ ಶಿಲ್ಪ, ಪ್ರಸಾರಾಂಗ, ಕನ್ನಡ ವಿ.ವಿ. ಹಂಪಿ, 1999\\
& : & ಶಾಸನ ಅಧ್ಯಯನ, ಸಂಪುಟ 4( ಸಂಚಿಕೆ 1 ಮತ್ತು 2), ಪ್ರಸಾರಾಂಗ,\newline ಕನ್ನಡ ವಿವಿ. ಹಂಪಿ. 2007\\
ಪಂಚಮುಖಿ, ಡಾ. ಆರ್​.ಎಸ್​. ನೆಲಮಂಗಲ ಲಕ್ಷ್ಮೀನಾರಾಯಣ ರಾವ್​ & : &  ಕರ್ನಾಟಕದ ಇತಿಹಾಸ (ಇತಿಹಾಸಪೂರ್ವ ಕಾಲದಿಂದ 10ನೇ ಶತಮಾನದ ವರೆಗೆ) ಕರ್ನಾಟಕ ಇತಿಹಾಸ ಸಂಶೋಧನಾ ಮಂಡಲ, ಧಾರವಾಡ, 1967\\[-14pt]
& : & ಕರ್ನಾಟಕದ ಅರಸು ಮನೆತನಗಳು, ಕರ್ನಾಟಕ ಇತಿಹಾಸ ಸಂಶೋಧನಾ ಮಂಡಲ,  ಧಾರವಾಡ, 2008\\
ಪ್ರಸನ್ನ ಕುಮಾರ್​, ಡಾ. ಎಂ. & : &  ಮೈಸೂರಿನ ಇತಿಹಾಸದಲ್ಲಿ ದಾನ (1600-1881), ಅನು: ಡಾ. ಆರ್​.ಎಲ್​. ಅನಂತರಾಮಯ್ಯ ಸಮನ್ವಯ ಪ್ರಕಾಶನ, ಮೈಸೂರು, 2001\\
ಬೋರಾಪುರ ಜಯರಾಮ್ & : &  ಆಚಾರ್ಯ ರಾಮಾನುಜರು, ಕನ್ನಡ ಸಾಹಿತ್ಯ ಪರಿಷತ್ತು, ಬೆಂಗಳೂರು, 2010\\
ಬಸವರಾಜ ತಗರಪುರ & : & ಐತಿಹ್ಯ ದರ್ಶನ, ಸುನಿಲ್​ ಪ್ರಕಾಶನ,ಮೈಸೂರು, 2008\\
ಬಸವರಾಜು, ಡಾ. ಎಲ್​. & : & ತಿರುಮಲಾರ್ಯ ಮತ್ತು ಚಿಕದೇವರಾಜ ಒಡೆಯರು, ಪ್ರಸಾರಾಂಗ, ಕರ್ನಾಟಕ ವಿವಿ. ಧಾರವಾಡ, 1972\\
& : & ಬಸವ ವಚನಾಮೃತ, ಬಸವ ಸಮಿತಿ, ಬೆಂಗಳೂರು 1984\\
& : & ಬಸವಣ್ಣನವರ ಷಟ್​ಸ್ಥಲ ವಚನಗಳು,\\
ಭಟ್​ ಸೂರಿ, ಡಾ. ಕೆ.ಜಿ. & : & ಕರ್ನಾಟಕ ಶಾಸನ ಸಂಶೋಧನೆ, ಪ್ರಸಾರಾಂಗ, ಕನ್ನಡ ವಿ.ವಿ. ಹಂಪಿ, 2009\\
ಭಟ್ಟ, ಜಿ.ಎಸ್​., ಜೀನಹಳ್ಳಿ ಸಿದ್ಧಲಿಂಗಪ್ಪ, & : & ಮೈಸೂರು ದರ್ಶನ, ಜಿಲ್ಲಾ ಕನ್ನಡ ಸಾಹಿತ್ಯ ಪರಿಷತ್ತು, ಮೈಸೂರು, 2010\\
ಮಹದೇವ, ಡಾ. ಸಿ. & : &  ಕರ್ನಾಟಕ ಪುರಾತತ್ವ ಶೋಧ, ಶ‍್ರೀ ಲಕ್ಷ್ಮೀ ಪ್ರಕಾಶನ,\newline ಶ‍್ರೀರಂಗಪಟ್ಟಣ, 2001\\
& : &  (ಸಂ) ತೊಣ್ಣೂರು, ಪ್ರಸಾರಾಂಗ, ಕನ್ನಡ ವಿ.ವಿ., ಹಂಪಿ, 2009\\
& : & (ಸಂ) ನಾಗಮಂಗಲ, ಪ್ರಸಾರಾಂಗ, ಕನ್ನಡ ವಿ.ವಿ., ಹಂಪಿ, 2009\\
ಮಂಜು ಡಾ. ಎಚ್​. & : & ಮಂಡ್ಯ ತಾಲ್ಲೂಕಿನ ಪಾರಂಪರಿಕ ಕೃಷಿ ತಂತ್ರಜ್ಞಾನ,\newline (ಅಪ್ರಕಟಿತ ಪಿಎಚ್​.ಡಿ. ಪ್ರಬಂಧ), ಮೈಸೂರು ವಿ.ವಿ. 2010\\
ಮಂಜುನಾಥ್​, ಡಾ. ಎಂ.ಜಿ. & : &  ಶಾಸನ ಪರಿಶೋಧನೆ, ಶಾರದಾ ಮಂದಿರ, ಮೈಸೂರು 2008\\
& : & ಕರ್ನಾಟಕದ ಪ್ರಮುಖ ಶಾಸನಗಳು, ಪ್ರಸಾರಾಂಗ, ಬೆಂಗಳೂರು, ವಿ.ವಿ.\\
ಮಲ್ಲಪ್ಪ ಪಂಡಿತ & : &  ಮಳವಳ್ಳಿ ತಾಲ್ಲೂಕು ದರ್ಶನ, ಐ.ಬಿ.ಎಚ್​. ಪ್ರಕಾಶನ, ಬೆಂಗಳೂರು 1998\\
ಮಲ್ಲಾಪುರ, ಡಾ. ಬಿ.ವಿ. & : & ಪ್ರೌಢದೇವರಾಯ-ವಿಜಯ ಕಲ್ಯಾಣ, ಪ್ರಸಾರಾಂಗ,\newline ಕನ್ನಡ ವಿವಿ. ಹಂಪಿ, 2000\\
ಮಹಮ್ಮದ್​ ಅಬ್ಬಾಸ್​ ಷೂಸ್ತ್ರೀ & : & ಇಸ್ಲಾಂ ಸಂಸ್ಕೃತಿ, ಪ್ರಸಾರಾಂಗ, ಮೈಸೂರು ವಿ.ವಿ.\\
ಮುದ್ದಾಚಾರಿ ಡಾ. ಬಿ. & : & ಮೈಸೂರು ಮರಾಠಾ ಬಾಂಧವ್ಯ, ಪ್ರಸಾರಾಂಗ, ಮೈಸೂರು, ವಿ.ವಿ. \\
ಮುನಿರಾಜಪ್ಪ, ಡಾ. & : &  ಮಾಗಡಿ ಸೀಮೆ- ಇತಿಹಾಸ ಮತ್ತು ಸಂಸ್ಕೃತಿ, ಸ್ಫೂರ್ತಿ ಪ್ರಕಾಶನ, ಹೊಸಪಾಳ್ಯ, ಮಾಗಡಿ ತಾ.2005\\
ಮೋಹನ ಕೃಷ್ಣ ರೈ, ಡಾ. & : &  ವಸಾಹತು ಪೂರ್ವ ಕರ್ನಾಟಕ ನಗರ ಚರಿತ್ರೆ, ಪ್ರಸಾರಾಂಗ,\newline ಕನ್ನಡ ವಿ.ವಿ. ಹಂಪಿ. 2006\\
ಯಾಮುನಾಚಾರ್ಯ, ಪ್ರೊ. ಎಂ. & : & ಆಳ್ವಾರರುಗಳು, ಪ್ರಸಾರಾಂಗ, ಮೈಸೂರು. ವಿ.ವಿ. 1974\\
ರಮೇಶ್​, ಡಾ. ಕೆ.ವಿ. & : &  ಕರ್ನಾಟಕ ಶಾಸನ ಸಮೀಕ್ಷೆ, ಪ್ರಸಾರಾಂಗ, ಬೆಂಗಳೂರು, ವಿ.ವಿ. 1971\\
ರಹಮತ್​ ತರೀಕೆರೆ & : & ಕರ್ನಾಟಕದ ನಾಥಪಂಥ, ಪ್ರಸಾರಾಂಗ, ಕನ್ನಡ ವಿ.ವಿ. ಹಂಪಿ, 2009\\
ರಾಜರತ್ನಂ, ಜಿ.ಪಿ. & : & ಆಳ್ವಾರರುಗಳು, ಶಾಕ್ಯ ಸಾಹಿತ್ಯ ಮಂಟಪ, ಬೆಂಗಳೂರು\\
ರಾಜಶೇಖರಪ್ಪ, ಡಾ. ಬಿ. & : &  ದುರ್ಗದ ಶೋಧನೆ, ಕನ್ನಡ ಸಾಹಿತ್ಯ ಪರಿಷತ್ತು, ಬೆಂಗಳೂರು, 2001\\
ರಾಜಾರಾಮ ಹೆಗ್ಗಡೆ, & : &  ಕೆರೆ ನೀರಾವರಿ ನಿರ್ವಹಣೆ-ಚಾರಿತ್ರಿಕ ಅಧ್ಯಯನ, ಪ್ರಸಾರಾಂಗ,\newline ಕನ್ನಡ ವಿವಿ. ಹಂಪಿ. 2002\\
ರಾಜಾರಾಮ ಹೆಗ್ಗಡೆ,(ಸಂ) ಅಶೋಕ್​ ಶೆಟ್ಟರ್​ & : &  ಮಲೆ ಕರ್ನಾಟಕದ ಅರಸು ಮನೆತನಗಳು, ಪ್ರಸಾರಾಂಗ,\newline ಕನ್ನಡ ವಿ.ವಿ. ಹಂಪಿ,\\
ರಾಜೇಶ್ವರಿ ಗೌಡ, ಡಾ. ಕೆ. & : &  ಆದಿಚುಂಚನಗಿರಿ- ಒಂದು ಸಾಂಸ್ಕೃತಿಕ ಅಧ್ಯಯನ, ಆದಿಚುಂಚನಗಿರಿ ಮಹಾಸಂಸ್ಥಾನ ಮಠ, ಆದಿಚುಂಚನಗಿರಿ, ನಾಗಮಂಗಲ ತಾಲ್ಲೂಕು,\newline ಮಂಡ್ಯ ಜಿಲ್ಲೆ,\\
ರಾಜೇಗೌಡ, ಹ.ಕ. & : &  ಆದಿಚುಂಚನಗಿರಿ, ಐ.ಬಿ.ಎಚ್​. ಪ್ರಕಾಶನ, ಬೆಂಗಳೂರು 1988\\
ರಾಮಕೃಷ್ಣ ಶರ್ಮ, ಗಡಿಯಾರಂ, ಆಲಂಪೂರ್​ & : &  ವಿದ್ಯಾರಣ್ಯರು ಒಂದು ಚಾರಿತ್ರಿಕ ಅಧ್ಯಯನ, 2002 (ಅನುವಾದಕರು ಮತ್ತು ಪ್ರಕಾಶಕರು: ರಾಮಚಂದ್ರರಾವ್​ ಗುಮಾಸ್ತೆ)\\
ರಾಬರ್ಟ್ ಸೆವೆಲ್​ (ಅನು: ಸದಾನಂದ ಕನವಳ್ಳಿ) & : & ಮರೆತುಹೋದ ಮಹಾಸಾಮ್ರಾಜ್ಯ, ಪ್ರಸಾರಾಂಗ, ಕನ್ನಡ ವಿವಿ. ಹಂಪಿ\\
ರಾಮಚಂದ್ರರಾವ್​, ಪ್ರೊ. ಎಸ್​.ಕೆ. & : & ಚುಂಚನಕಟ್ಟೆ, ಸಾಲಿಗ್ರಾಮ, ಹನಸೋಗೆ, ಐಬಿಎಚ್​. ಪ್ರಕಾಶನ, ಬೆಂಗಳೂರು, 1983\\
ರಾಮರಾವ್​, ಆರ್​.ಎಸ್​. & : & ಸಮರ ಚಿತ್ರಗಳು, ಐ.ಬಿ.ಎಚ್​. ಪ್ರಕಾಶನ, ಬೆಂಗಳೂರು, 1983\\
ರಾಮಲಿಂಗಪ್ಪ, ಎಚ್​. & : & ನಾಗರಿಕತೆಗಳ ಸಮೀಕ್ಷೆ, ಪ್ರಸಾರಾಂಗ, ಬೆಂಗಳೂರು ವಿ.ವಿ. 1980\\
ರಾಮಶೇಷನ್​, ನೀ.ಕೃ. & : & ಕರ್ನಾಟಕದ ಕೋಟೆ ಕೊತ್ತಲುಗಳು, ಐ.ಬಿ.ಎಚ್​. ಪ್ರಕಾಶನ,\newline ಬೆಂಗಳೂರು, 1980\\
& : & ದಕ್ಷಿಣ ಭಾರತದ ಆಳ್ವಾರರು-ಒಂದು ಅವಲೋಕನ, ಪ್ರಸಾರಾಂಗ,\newline ಕನ್ನಡ ವಿ.ವಿ. ಹಂಪಿ, 2011\\
ರಿತ್ತಿ, ಡಾ. ಎಚ್​.ಎಸ್​. & : &  ಪ್ರಾಚೀನ ಕರ್ನಾಟಕದ ಆಡಳಿತ ಪರಿಭಾಷಾ ಕೋಶ, ಪ್ರಾಚ್ಯವಸ್ತು ಮತ್ತು ಸಂಗ್ರಹಾಲಯಗಳ ನಿರ್ದೇಶನಾಲಯ, ಕರ್ನಾಟಕ ಸರ್ಕಾರ,\newline ಮೈಸೂರು 2000\\
& : & ಕರ್ನಾಟಕ ಗ್ರಾಮ ಸೂಚಿ, ಕನ್ನಡ ಮತ್ತು ಸಂಸ್ಕೃತಿ ಇಲಾಖೆ, ಕರ್ನಾಟಕ ಸರ್ಕಾರ, 1985\\
ಲಕ್ಷ್ಮಣ ತೆಲಗಾವಿ, ಪ್ರೊ & : &  ವಿಜಯನಗರ ಕಾಲದ ರಾಮಾನುಜ ಕೂಟಗಳು, ಪ್ರಸಾರಾಂಗ,\newline ಕನ್ನಡ ವಿವಿ. ಹಂಪಿ, 2009\\
ಲಕ್ಷ್ಮೀನರಸಿಂಹ ಶಾಸ್ತ್ರೀ ಹುರಗಲವಾಡಿ & : &  ವಿಜಯನಗರ ಸಾಮ್ರಾಜ್ಯ ಸಂಸ್ಥಾಪಕ ಶ‍್ರೀ ವಿದ್ಯಾರಣ್ಯರು, ನವಭಾರತೀ ಪ್ರಕಾಶನ, ಮೈಸೂರು, 2007\\
ಲಕ್ಷ್ಮೀನಾರಾಯಣ, ಬಿ.ಎಸ್​. & : & ನೆನಪಿನ ನಕ್ಷತ್ರಗಳು (ಬೆಳ್ಳೂರಿನ ಚಿತ್ರಣ)\\
& : & ಅಲ್ಲುಂಟು ನಂಟು, (ಬೆಳ್ಳೂರಿನ ಹಬ್ಬಹರಿದಿನ, ಜಾತ್ರೆ ಉತ್ಸವ ಇತ್ಯಾದಿ) \\
ಲಕ್ಷ್ಮೀಪುರಂ ಶ‍್ರೀನಿವಾಸಾಚಾರ್ಯ & : &  ಹಿಂದೂ ದರ್ಶನ ಸಾರ, ಪ್ರಸಾರಾಂಗ, ಮೈಸೂರು ವಿ.ವಿ. 1985\\
ಲಿಂಗದೇವರು ಡಾ. ಹಳೇಮನೆ & : & ಧೀರ ಟಿಪುವಿನ ಲಾವಣಿಗಳು, ವಿಸ್ಮಯ ಪ್ರಕಾಶನ, ಮೈಸೂರು, 2009\\
ವಸಂತ ಲಕ್ಷ್ಮೀ, ಡಾ. ಕೆ. & : & ಕರ್ನಾಟಕದ ಶೈವ ಶಿಲ್ಪಗಳು, ಕರ್ನಾಟಕ ಇತಿಹಾಸ ಸಂಶೋಧನಾ ಮಂಡಲ, ಧಾರವಾಡ 2008\\
& : & ಹೊಯ್ಸಳ ವಾಸ್ತು ಶಿಲ್ಪ, ಕರ್ನಾಟಕ ಶಿಲ್ಪಕಲಾ ಅಕಾಡೆಮಿ, ಬೆಂಗಳೂರು, \\
& : & ಹಾಸನ ಜಿಲ್ಲೆಯ ಹೊಯ್ಸಳರ ಬಸದಿಗಳು, ಸುಮುಖ ಪ್ರಕಾಶನ,\newline ವಿದ್ಯಾರಣ್ಯ ನಗರ, ಬೆಂಗಳೂರು\\
ವಸು, ಡಾ. ಎಂ.ವಿ. & : & (ಸಂ) ದಕ್ಷಿಣ ಕರ್ನಾಟಕದ ಅರಸುಮನೆತನಗಳು, ಪ್ರಸಾರಾಂಗ,\newline ಕನ್ನಡ ವಿವಿ. ಹಂಪಿ 2001\\
ವಸುಂಧರಾ ಫಿಲಿಯೋಜಾ, ಡಾ. & : & ವಿಜಯನಗರ ಸಾಮ್ರಾಜ್ಯ ಸ್ಥಾಪನೆ, ಕನ್ನಡ ಸಾಹಿತ್ಯ ಪರಿಷತ್ತು,\newline ಬೆಂಗಳೂರು, 1980\\
& : &  ಧಾರವಾಡ ಜಿಲ್ಲೆಯ ಕಾಳಾಮುಖ ಮತ್ತು ಪಾಶುಪತ ದೇವಾಲಯಗಳು, ಅನುವಾದಕರು: ಡಾ. ಹನುಮಾಕ್ಷಿ ಗೋಗಿ, ಕರ್ನಾಟಕ ಅನುವಾದ ಸಾಹಿತ್ಯ ಅಕಾಡೆಮಿ, 2007\\
ವಿರೂಪಾಕ್ಷ ಕುಲಕರ್ಣಿ & : & (ಅನುವಾದ) ಕಾಪಾಲಿಕರು ಮತ್ತು ಕಾಳಾಮುಖರು-ಕಣ್ಮರೆಯಾದ ಎರಡು ಶೈವ ಸಂಪ್ರದಾಯಗಳು, ಇಂಗ್ಲಿಷ್​ ಮೂಲ: ಡೇವಿಡ್​.ಎನ್​.ಲಾರೆಂಜನ್​, ಕನ್ನಡ ಪುಸ್ತಕ ಪ್ರಾಧಿಕಾರ, ಬೆಂಗಳೂರು 2005\\
ವಿರೂಪಾಕ್ಷಿ ಡಾ. ಪೂಜಾರಹಳ್ಳಿ & : &  ಕರ್ನಾಟಕದಲ್ಲಿ ತಳವಾರಿಕೆ, ಪ್ರಸಾರಾಂಗ, ಕನ್ನಡ ವಿವಿ. ಹಂಪಿ, 2006\\
ವೆಂಕಟಕೃಷ್ಣ, ತೈಲೂರು & : & ಮಂಡ್ಯ ಜಿಲ್ಲೆಯ ದೇವಾಲಯಗಳು- ಒಂದು ಅವಲೋಕನ,\newline ಭಾನು ಪ್ರಕಾಶನ, ಮಂಡ್ಯ, 2010\\
& : & ಮಂಡ್ಯ ಜಿಲ್ಲೆಯ ಸಾಂಸ್ಕೃತಿಕ ಪರಂಪರೆ,\\
& : & ಮಂಡ್ಯ ಜಿಲ್ಲೆಯ ಸ್ಥಳನಾಮಗಳು, ಭಾಗ-1, ಭಾನು ಪ್ರಕಾಶನ ಮಂಡ್ಯ,\\
ವೆಂಕಟಾಚಲ ಶಾಸ್ತ್ರೀ, ಡಾ. ಟಿ.ವಿ. & : & ಕನ್ನಡ ಅಭಿಜಾತ ಸಾಹಿತ್ಯದ ಅಧ್ಯಯನದ ಅವಕಾಶಗಳು ಮತ್ತು ಆಹ್ವಾನಗಳು, ಪ್ರಸಾರಾಂಗ, ಕನ್ನಡ ವಿ.ವಿ. ಹಂಪಿ, 2009\\
& : & ಪ್ರಾಕ್ತನ, ರಾ. ನರಸಿಂಹಾಚಾರ್ಯರ ಲೇಖನಗಳು, ಕನ್ನಡ ಅಧ್ಯಯನ ಸಂಸ್ಥೆ, ಮೈಸೂರು ವಿವಿ.1986\\
ವೆಂಕಟೇಶ ಜೋಯಿಸ್​, ಡಾ. ಕೆ.ಜಿ. & : & ಕೆಳದಿ ಶಾಸನಗಳ ಸಾಂಸ್ಕೃತಿಕ ಅಧ್ಯಯನ\\
ವೇಣುಗೋಪಾಲರಾವ್​, ಡಾ.ಎ.ಎಸ್​. & : & ಕನ್ನಡ ಸಾಹಿತ್ಯದ ಭಾಗವತ ಕವಿಗಳು, ಪ್ರಸಾರಾಂಗ,\newline ಮೈಸೂರು ವಿ.ವಿ. 1983\\
ವೇಣುಗೋಪಾಲಾಚಾರ್ಯ & : & ಭಕ್ತಪುರಿ ತೊಂಡನೂರು ಮಹಾತ್ಮೆ, ಪ್ರ: ವೇಣುಗೋಪಾಲಾಚಾರ್ಯ,\newline ಜೈನ್​ ಕಾಲೋನಿ, ಮಂಡ್ಯ 1999\\
ಶಂಕರಯ್ಯ, ಡಾ. ಎಚ್​.ಕೆ. & : & ಮದ್ದೂರು ತಾಲ್ಲೂಕು ಸ್ಥಳನಾಮಗಳು (ಅಪ್ರಕಟಿತ ಪಿಎಚ್​.ಡಿ. ಪ್ರಬಂಧ) ಮೈಸೂರು ವಿವಿ. 2009\\
ಶತಾವಧಾನಿ ಡಾ. ಆರ್​. ಗಣೇಶ್​, & : & ವಿಭೂತಿ ಪುರುಷ ವಿದ್ಯಾರಣ್ಯ, ಸಾಹಿತ್ಯ ಪ್ರಕಾಶನ, ಹುಬ್ಬಳ್ಳಿ, 2011\\
ಶಾಂತಕುಮಾರಿ, ಡಾ.ಎಲ್​.ಎಸ್​. & : & ಕುಕನೂರು, ಪ್ರಸಾರಾಂಗ, ಕರ್ನಾಟಕ ವಿ.ವಿ. 1975\\
ಶಾಮರಾವ್​, ತ.ಸು. & : & ಶಿವಶರಣ ಕಥಾ ರತ್ನಕೋಶ, ತ.ವೆಂ. ಸ್ಮಾರಕ ಗ್ರಂಥಮಾಲೆ, ಮೈಸೂರು\\
ಶಿವಣ್ಣ, ಡಾ. ಕೆ.ಎಸ್​. & : &  ಕಳಲೆ ವೀರಶೈವ ಮನೆತನದ ದಳವಾಯಿಗಳು, ಜೆ.ಎಸ್​.ಎಸ್​. ಗ್ರಂಥಮಾಲೆ, ಮೈಸೂರು, 1994\\
ಶಿವಮಾದಪ್ಪ, ಕೆ.ಎಸ್​. & : & ಆತಗೂರು ಪ್ರಾಣಿದಯೆ, ಕೆ.ಎಸ್​. ಪಬ್ಲಿಕೇಷನ್ಸ್​, ಆತಗೂರು, ಮಂಡ್ಯ ಜಿಲ್ಲೆ\\
ಶಿವರುದ್ರಪ್ಪ, ಡಾ. ಜಿ.ಎಸ್​. & : &  (ಸಂ) ಶ್ರವಣಬೆಳಗೊಳ- ಒಂದು ಸಮೀಕ್ಷೆ, ಪ್ರಸಾರಾಂಗ,\newline ಬೆಂಗಳೂರು ವಿವಿ. 1983\\
ಶಿವರುದ್ರಸ್ವಾಮಿ, ಡಾ. ಎಸ್​.ಎನ್​. & : & ಕರ್ನಾಟಕದ ಪ್ರೌಢ ಇತಿಹಾಸ ಮತ್ತು ಸಂಸ್ಕೃತಿ, ಪೌರಸ್ತ್ಯ ಪ್ರಕಾಶನ, ತಿಪಟೂರು, 2000\\
ಶಿವಾನಂದ ವಿರಕ್ತಮಠ, ಡಾ. & : &  ಪ್ರೌಢದೇವರಾಯನ ಕಾಲದ ಕನ್ನಡ ಸಾಹಿತ್ಯ, ಕರ್ನಾಟಕ ವಿ.ವಿ.\newline ಧಾರವಾಡ, 1978\\
ಶೀಲಾಕಾಂತ ಪತ್ತಾರ್​, ಡಾ. & : &  ಬಾದಾಮಿ ಸಾಂಸ್ಕೃತಿಕ ಅಧ್ಯಯನ, ಪ್ರಸಾರಾಂಗ,\newline ಕನ್ನಡ ವಿ.ವಿ. ಹಂಪಿ, 2000\\
ಶೀಲಾಕುಮಾರಿ, ಡಾ.ಡಿ. & : &  ಸಂಸ್ಕೃತ ಸಾಹಿತ್ಯಕ್ಕೆ ಮಹಾಕವಿ ಷಡಕ್ಷರದೇವನ ಕೊಡುಗೆ,\newline ಕನ್ನಡ ಸಾಹಿತ್ಯ ಪರಿಷತ್ತು, 1995\\
ಶೆಟ್ಟರ್​, ಡಾ. ಷ. & : & ಸಾವಿಗೆ ಆಹ್ವಾನ, ಕನ್ನಡ ಪುಸ್ತಕ ಪ್ರಾಧಿಕಾರ, ಬೆಂಗಳೂರು, 2004\\
ಶೇಖ್​ ಅಲಿ,ಡಾ. ಬಿ. & : & (ಸಂ) ಕರ್ನಾಟಕ ಚರಿತ್ರೆ, ಸಂಪುಟ 1 ರಿಂದ 7, ಕನ್ನಡ ವಿಶ್ವವಿದ್ಯಾನಿಲಯ, ಹಂಪಿ, 1997\\
ಶೇಷ ಶಾಸ್ತ್ರೀ, ಡಾ ಆರ್​ & : & ಕರ್ನಾಟಕದ ವೀರಗಲ್ಲುಗಳು, ಕನ್ನಡ ಸಾಹಿತ್ಯ ಪರಿಷತ್ತು, 2004 ಶಾಸನ ಪರಿಚಯ, ಬೆಂಗಳೂರು ವಿಶ್ವವಿದ್ಯಾನಿಲಯ, ಬೆಂಗಳೂರು 1982\\
ಶೋಭಾ, ಡಾ. & : & ಮಂಡ್ಯ ಜಿಲ್ಲೆಯ ಹೊಯ್ಸಳ ದೇವಾಲಯಗಳು, ದೇಸಿ ಪ್ರಕಾಶನ, ಬೆಂಗಳೂರು, 2011\\
ಶ‍್ರೀಕಂಠ ಶಾಸ್ತ್ರಿ, ಡಾ. ಎಸ್​. & : & ಭಾರತೀಯ ಸಂಸ್ಕೃತಿ, ಕನ್ನಡ ಮತ್ತು ಸಂಸ್ಕೃತಿ ಇಲಾಖೆ, ಬೆಂಗಳೂರು, 2006\\
& : & ಹೊಯ್ಸಳ ವಾಸ್ತು ಶಿಲ್ಪ, ಪ್ರಸಾರಾಂಗ, ಮೈಸೂರು ವಿ.ವಿ. 1965\\
& : & ಪುರಾತತ್ವ ಶೋಧನೆ, ಪ್ರಸಾರಾಂಗ, ಮೈಸೂರು ವಿ.ವಿ. 1975\\
& : & ಸಂಶೋಧನಾ ಲೇಖನಗಳು, ಕಾಮಧೇನು ಪುಸ್ತಕ ಭವನ, ಬೆಂಗಳೂರು,\\
ಶ‍್ರೀನಿವಾಸ ಮೂರ್ತಿ, ಡಾ.ಎಲ್​.ಎಸ್​. & : &  ಪುರಾತತ್ವ ಪಿತಾಮಹ, ಬಿ.ಎಲ್​. ರೈಸ್​-ಜೀವನ ಸಾಧನೆ, ಕನ್ನಡ ಪುಸ್ತಕ ಪ್ರಾಧಿಕಾರ, 2011\\
ಶ‍್ರೀನಿವಾಸಯ್ಯ, ಹೊ.& : & ನಾಗಮಂಗಲ ತಾಲ್ಲೂಕು ದರ್ಶನ, ಐ.ಬಿ.ಎಚ್​. ಪ್ರಕಾಶನ,\newline ಬೆಂಗಳೂರು 1980\\
ಶ‍್ರೀವತ್ಸ ಕಲಬಾಗಲ್​ & : &  ಶ‍್ರೀವೈಷ್ಣವ ದಿವ್ಯ ದೇಶಗಳು, ಪ್ರ: ಶ‍್ರೀವತ್ಸ ಕಲಬಾಗಲ್​,\newline ಬೆಂಗಳೂರು, 1996\\
ಸತೀಶ್​ ಡಾ. ಕೆ. & : & ವಿಜಯನಗರ ಕಾಲದ ಶೈವದೇವಾಲಯಗಳು, ಕನ್ನಡ ಸಾಹಿತ್ಯ ಪರಿಷತ್ತು, ಬೆಂಗಳೂರು 2008\\
ಸಿ.ಪಿ.ಕೃಷ್ಣಕುಮಾರ್​, ಡಾ.& : & (ಸಂ) ಇಕ್ಷು ಕಾವೇರಿ, ಕನ್ನಡ ಸಾಹಿತ್ಯ ಸಮ್ಮೇಳನದ ನೆನಪಿನ ಸಂಚಿಕೆ, ಮಂಡ್ಯ, 1974\\
& : &  (ಅನುವಾದ) ಭಾರತೀಯ ಶಾಸನ ಶಾಸ್ತ್ರ, ಮೂಲ. ಜೆ.ಎಫ್​. ಫ್ಲೀಟ್​,\newline ಲಕ್ಷ್ಮೀ ಪ್ರಿಂಟಿಂಗ್​ ಪ್ರೆಸ್​, ಕಾಳಮ್ಮನ ಗುಡಿ ಬೀದಿ, ಮೈಸೂರು 2001\\
ಸೀತಾರಾಮಯ್ಯ, ಡಾ.ವಿ. & : & ತಿರುಮಲಾರ್ಯ, ಐ.ಬಿ.ಎಚ್​. ಪ್ರಕಾಶನ, ಬೆಂಗಳೂರು, 1980\\
ಸೀತಾರಾಮ ಜಾಗಿರ್​ದಾರ್​ & : & ಕಂಬದಹಳ್ಳಿ ಒಂದು ಜೈನ ಕೇಂದ್ರ- ಅತ್ತಿಮಬ್ಬೆ ಪ್ರಕಾಶನ, ಕುವೆಂಪುನಗರ, ಮೈಸೂರು 1999\\
ಸೂರ್ಯನಾಥ ಕಾಮತ್​, ಡಾ. ಯು. & : & ಕರ್ನಾಟಕದ ಸಂಕ್ಷಿಪ್ತ ಇತಿಹಾಸ, ಬಾಪ್ಕೋ ಪ್ರಕಾಶನ, ಬೆಂಗಳೂರು, 1975\\
& : & ಒಕ್ಕಲುತನ ಮತ್ತು ಒಕ್ಕಲಿಗರು, ಅರ್ಚನ ಪ್ರಕಾಶನ, ಬೆಂಗಳೂರು, 2006\\
ಸ್ವಾಮಿ, ಡಾ. ಬಿ.ಜಿ.ಎಲ್​. & : & ಶಾಸನಗಳಲ್ಲಿ ಗಿಡಮರಗಳು, ಪ್ರಸಾರಾಂಗ, ಬೆಂಗಳೂರು ವಿವಿ. 1975\\
ಸ್ವಾಮಿ ರಾಮಕೃಷ್ಣಾನಂದ & : &  ಶ‍್ರೀ ರಾಮಾನುಜ ಜೀವನ ಚರಿತ್ರೆ, ರಾಮಕೃಷ್ಣ ಆಶ್ರಮ, ಮೈಸೂರು, 2006\\
ಹನುಮಾಕ್ಷಿ ಗೋಗಿ, & : & ಕಲ್ಬುರ್ಗಿ ಜಿಲ್ಲೆಯ ಶಾಸನಗಳು, ಶಿವಚಂದ್ರ ಪ್ರಕಾಶನ, ಹುಬ್ಬಳ್ಳಿ,\\
& : & ಮುದೆನೂರು ಮತ್ತು ಯಡ್ರಾಮಿಯ ಶಾಸನಗಳು,\\
ಹಂದೂರ್​, ಡಾ.ಬಿ.ಆರ್​. & : & ಜೈನಪರಂಪರೆಗೆ ಬೆಳಗಾವಿ ಪ್ರಾದೇಶಿಕ ಕೊಡುಗೆ, ಪ್ರ: ಮಂಜುಳಾ ಅಡಿಕೆ, ಬೆಳಗಾವಿ, 2008\\
ಹರಿಶಂಕರ್​, ಹೆಚ್​.ಎಸ್​. & : & ತಿರುಮಲಾರ್ಯ, ತ.ವೆಂ.ಸ್ಮಾರಕ ಗ್ರಂಥಮಾಲೆ, ಮೈಸೂರು, 2002\\
ಹಿರೇಮಠ, ಡಾ.ಆರ್​.ಸಿ. & : & (ಸಂ) ಕರ್ನಾಟಕ ಮತ್ತು ಜೈನಸಂಸ್ಕೃತಿ, ಕರ್ನಾಟಕ ವಿ.ವಿ.ಧಾರವಾಡ, 1975\\
ಹಿರೇಮಠ, ಡಾ. ಆರ್​.ಸಿ. ಕಲಬುರ್ಗಿ,ಡಾ. ಎಂ.ಎಂ. & : &  (ಸಂ) ಕನ್ನಡ ಶಾಸನ ಸಂಪದ, ಕರ್ನಾಟಕ ವಿ.ವಿ ಧಾರವಾಡ, 1986\\
ಹಿರೇಮಠ. ಡಾ. ಬಿ.ಆರ್​. & : &  ಶಾಸನಗಳಲ್ಲಿ ಕರ್ನಾಟಕದ ವರ್ತಕರು, ಪ್ರಸಾರಾಂಗ, ಕರ್ನಾಟಕ ವಿ.ವಿ. ಧಾರವಾಡ, 1986\\
& : & ವೀರಗಲ್ಲುಗಳು, ಪ್ರಸಾರಾಂಗ, ಕರ್ನಾಟಕ ವಿ.ವಿ. ಧಾರವಾಡ, 1986\\
ಹಿರೇಮಠ, ಪ್ರೊ: ಎಸ್​.ಎಸ್​. & : &  ಲಾಕುಳ ದರ್ಶನ, ಕನ್ನಡ ಸಾಹಿತ್ಯ ಪರಿಷತ್ತು, ಬೆಂಗಳೂರು, 2006\\
ಹಿರೇಮಠ, ಡಾ. ಎಸ್​.ಎಮ್. & : & ಶಾಸನ ಪರಿಭಾಷೆ, ವಿದ್ಯಾನಿಧಿ ಪ್ರಕಾಶನ, ಗದಗ, 2005\\
& : & ಶಾಸನಾಧ್ಯಯನ, ವಿದ್ಯಾನಿಧಿ ಪ್ರಕಾಶನ, ಗದಗ, 1998\\
ಹೆರಂಜೆ ಕೃಷ್ಣಭಟ್ಟ, ಎಸ್​.ಡಿ.ಶೆಟ್ಟಿ, ಡಾ. & : &  ತುಳು ಕರ್ನಾಟಕದ ಅರಸು ಮನೆತನಗಳು, ಪ್ರಸಾರಾಂಗ, ಕನ್ನಡ ವಿ.ವಿ. ಹಂಪಿ.,
\end{longtable}

\newpage

\section*{ಆಂಗ್ಲ ಕೃತಿಗಳು:}

{\renewcommand{\arraystretch}{.9}
\begin{longtable}[l]{@{}>{\raggedright}p{4.5cm}cp{9.3cm}<{\raggedright}@{}}
Derette, J.D.M. & : & The Hoysalas, 1956\\
Dixith, Dr. G.S. & : & Local Self Government in Mediaeval Karnataka, Karnataka Universtity Dharwar, 1964\\
Gopal, Dr.R. & : &  Cultural Study of Hoysala Inscriptions, Directorate of Archaeology and Museums in Karnataka, Mysore, 2000\\
& : & Vijayanagara Inscriptions, Volume III, Edited by. Directorate of Archaeology and Museums, Government of Karnataka, Mysore, 1990\\
Gururajachar, Dr.S. & : & Some aspects of Economic and Social Life in Karnataka, Prasaranga, University of Mysore, 1974\\
Krishnamurthy, Dr.S. & : & The Nolambas, Prasaranga, University of Mysore, 1980\\
Naga Raju Dr. M. H. & : & Devaraya II and his times, Prasaranga, University of Mysore\\ 
Padma, Dr.M.B. & : & The Position of Women in Mediaeval KarnatakaPrasaranga, University of Mysore, 1993\\
Prahakar Apte, Prof.,\newline Sri. Ravindra R Kamath & : & Melukote through the ages, Academy of Sanskrit Research, Melukote, 1998\\
Radha Patel, Dr. M. & : & Life and times of Hoysala Narasimha III, Prasaranga, University of Mysore\\
Rangaraju, Dr. N.S, & : & Hoysala Temples in Mandya and Tumkur Districts, Prasaranga, University of Mysore, 1998\\
Rice B.L. & : & Mysore and Coorg from the Inscriptions, Archibald Constable \& Co. London, 1909\\
Sathya Narayana, Dr.A. & : & History of the Wodeyars of Mysore (1610-1748), Dirctorate of Archaeology and Museums, Mysore, 1996\\
Sheik Ali, Dr. B. & : & History of the Western Gangas, Director, Prasaranga, University of Mysore,\\
& : &  (Editor) The Hoysala Dynasty, Department of Hystory, University of Mysore, 1972\\
Shivanna, Dr. & : & Rashtrakuta Relations with the Gangas of Talkad, Prasaranga, University of Mysore, 1997\\
Shivanna, Dr. K.S. & : & The Agrarian System of Karnataka(1336-1761), Prasaranga, University of Mysore, 1983\\
Srikanta Sastri, Dr. S., & : & The Sources of Karnataka History, Vellala Publishing House, Bangalore, 2018\\
Tara Kashyap, Dr. & : & Pancalingeswara Temple, Govindanahally, Mandya Dist.- Monograph Sahayog, Bangalore 2005\\
Venkata Rathnam, Dr. A.V. & : & Local Government in the Vijayanagara Empire, Prasaranga, University of Mysore, 1972\\
\end{longtable}}\relax
