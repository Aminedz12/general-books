\chapter*{ಸಂಕೇತಾಕ್ಷರ ಸೂಚಿ}

\noindent
ಎಕ - ಎಪಿಗ್ರಾಫಿಯಾ ಕರ್ನಾಟಿಕ ( ಪರಿಷ್ಕೃತ ಸಂಪುಟಗಳು), ಕನ್ನಡ ಅಧ್ಯಯನ ಸಂಸ್ಥೆ, ಮೈಸೂರು ವಿವಿ. ಮೈಸೂರು

\noindent
ಇ.ಸಿ- ಎಪಿಗ್ರಾಫಿಯಾ ಕರ್ನಾಟಿಕ (ಹಳೆಯ ಸಂಪುಟಗಳು) ಬಿ.ಎಲ್​. ರೈಸ್​.

\noindent
ಕ.ವಿ.ವಿ.ಶಾ.ಸಂ. - ಕನ್ನಡ ವಿಶ್ವವಿದ್ಯಾನಿಲಯ ಶಾಸನ ಸಂಪುಟ

\noindent
\textbf{ಅಡಿ ಟಿಪ್ಪಣಿ ಶಾಸನಗಳ ಸಂಕೇತಾಕ್ಷರಗಳನ್ನು ಓದುವ ವಿಧಾನ:}

\noindent
ಎಪಿಗ್ರಾಫಿಯಾ ಕರ್ನಾಟಿಕ, ಸಂಪುಟ 6: ಕೃಪೇ=ಕೃಷ್ಣರಾಜಪೇಟೆ, ಪಾಂಪು=ಪಾಂಡವಪುರ, ಶ‍್ರೀಪ=ಶ‍್ರೀರಂಗಪಟ್ಟಣ ತಾಲ್ಲೂಕು

\noindent
ಎಪಿಗ್ರಾಫಿಕಯಾ ಕರ್ನಾಟಿಕ, ಸಂಪುಟ 7: ನಾಮಂ=ನಾಗಮಂಗಲ, ಮಂ=ಮಂಡ್ಯ, ಮ=ಮದ್ದೂರು, ಮವ=ಮಳವಳ್ಳಿ ತಾಲ್ಲೂಕು

\noindent
\textbf{ಎಕ 6 ಕೃಪೇ 39 ಗೋವಿಂದನಹಳ್ಳಿ 1236}

\noindent
ಎಕ 6= ಎಪಿಗ್ರಾಫಿಯಾ ಕರ್ನಾಟಿಕ (ಪರಿಷ್ಕೃತ), ಸಂಪುಟ 6, ಕೃಪೇ = ಕೃಷ್ಣರಾಜಪೇಟೆ ತಾಲ್ಲೂಕು, 39= 39ನೇ ಸಂಖ್ಯೆಯ ಶಾಸನ, ಗೋವಿಂದನಹಳ್ಳಿ, (ಗೋವಿಂದನಹಳ್ಳಿಯಲ್ಲಿ ಈ ಶಾಸನ ಇದೆ), ಈ ಶಾಸನದ ಕಾಲ ಕ್ರಿ.ಶ.1236

\noindent
\textbf{ಎಕ 7 ನಾಮಂ 130 ದೊಡ್ಡಜಟಕ 1179}

\noindent
ಎಕ 7= ಎಪಿಗ್ರಾಫಿಯಾ ಕರ್ನಾಟಿಕ (ಪರಿಷ್ಕೃತ) ಸಂಪುಟ 7, ನಾಮಂ= ನಾಗಮಂಗಲ ತಾಲ್ಲೂಕು, 130= 130ನೇ ಸಂಖ್ಯೆಯ ಶಾಸನ, ದೊಡ್ಡಜಟಕ (ದೊಡ್ಡ ಜಟಕ ಗ್ರಾಮದಲ್ಲಿ ಈ ಶಾಸನ ಇದೆ), 1179= ಈ ಶಾಸನದ ಕಾಲ ಕ್ರಿ.ಶ. 1179

\noindent
ಬೇರೆ ಸಂಪುಟಗಳಿಗೆ ಸಂಬಂಧಿಸಿದಂತೆ, ಸಂಪುಟಗಳ ಸಂಖ್ಯೆ, ತಾಲ್ಲೂಕಿನ ಹೆಸರನ್ನು ಒಂದು, ಎರಡು ಅಥವಾ ಮೂರು ಅಕ್ಷರ\-ಗಳಲ್ಲಿ ನೀಡಿದೆ

\noindent
ಉದಾ.ಗೆ: ಅರ-ಅರಸೀಕೆರೆ, ಚರಾಪ- ಚನ್ನರಾಯಪಟ್ಟಣ, ಬೇ-ಬೇಲೂರು, ಹೊನಪು-ಹೊಳೆನರಸಿಪುರ, ಅಗೂ-ಅರಕಲ\-ಗೂಡು ಶ್ರಬೆ-ಶ್ರವಣಬೆಳಗೊಳ, ಚಿಬೆ-ಚಿಕ್ಕಬೆಟ್ಟ, ದೊಬೆ-ದೊಡ್ಡಬೆಟ್ಟ, ಚಾನ-ಚಾಮರಾಜನಗರ, ಗುಂಪೆ-ಗುಂಡ್ಲುಪೇಟೆ,\break ಹು-ಹುಣಸೂರು, ತಿನಪು-ತಿರುಮಕೂಡಲು ನರಸಿಪುರ, ತುಮ-ತುಮಕೂರು, ಕೃನಾ-ಕೃಷ್ಣರಾಜನಗರ, ಚಿಮ-ಚಿಕ್ಕ\-ಮಗಳೂರು, ಕ-ಕಡೂರು

\noindent
ಕನ್ನಡ ಮತ್ತು ಆಂಗ್ಲ ಕೃತಿಗಳ ಹೆಸರು ಮತ್ತು ಪುಟ ಸಂಖ್ಯೆಗಳನ್ನು ಪೂರ್ತಿಯಾಗಿ ನೀಡಿದೆ

\noindent
\textbf{ರಕ್ಷಾ ಪುಟ (ಮುಖ ಪುಟ):} ಕಿಕ್ಕೇರಿ ಬ್ರಹ್ಮೇಶ್ವರ ದೇವಾಲಯದ ಶಿಲಾಬಾಲಿಕೆಯರು 


\hspace{2cm} ಬಸರಾಳು ಮಲ್ಲಿಕಾರ್ಜುನ ದೇವಾಲಯ

\noindent
\textbf{ಹಿಂಬದಿಯ ರಕ್ಷಾಪುಟ:} ಅಗ್ರಹಾರ ಬಾಚಹಳ್ಳಿಯ ಗರುಡ ಸ್ಥಂಭಗಳು 

\hspace{1.45cm} ಅಲ್ಲಿಯೇ ಇರುವ ವೀರಗಲ್ಲುಗಳ ಸಾಲು

