ಅ
ಅಂಕ
ಅಂಕ-ಕಾರ
ಅಂಕ-ಕಾರ-ದೇವರ
ಅಂಕ-ಕಾರ-ದೇವರಿ
ಅಂಕ-ಕಾರ-ದೇವ-ರಿಗೆ
ಅಂಕ-ಕಾರ-ನಾಗಿದ್ದ-ನೆಂದಯ
ಅಂಕ-ಕಾರ-ನೆಂದು
ಅಂಕ-ಕಾರ-ಸೇನಾ-ಧಿ-ಪತಿ
ಅಂಕ-ಗ-ಳನ್ನು
ಅಂಕ-ಗ-ಳಿಗೆ
ಅಂಕ-ಗಾವುಂಡ
ಅಂಕ-ಗೌಂಡನ
ಅಂಕಣ
ಅಂಕ-ಣದ
ಅಂಕ-ಣ-ವನ್ನು
ಅಂಕಣ್ನ
ಅಂಕದ
ಅಂಕ-ನ-ಹಳ್ಳಿ
ಅಂಕ-ನ-ಹಳ್ಳಿಯ
ಅಂಕ-ನ-ಹಳ್ಳಿ-ಯನ್ನು
ಅಂಕ-ನ-ಹಳ್ಳಿ-ಯಲ್ಲಿ
ಅಂಕ-ನಾಥ-ದೇವರ
ಅಂಕ-ನಾಥ-ಪುರದ
ಅಂಕಯ್ಯ
ಅಂಕಿ
ಅಂಕಿ-ಅಂಶ
ಅಂಕಿ-ಗಳು
ಅಂಕಿ-ತ-ಗ-ಳಾಗಿ-ರ-ಬಹು-ದೆಂದು
ಅಂಕಿ-ತ-ಮಾಡಿದ್ದಾ-ರೆಂದು
ಅಂಕಿ-ತ-ವಾಗಿ
ಅಂಕಿ-ಸೆಟ್ಟಿಯ
ಅಂಕುಶ-ರಾಯ
ಅಂಕುಶೇಂದ್ರನ
ಅಂಕುಶೇಂದ್ರನಿರ-ಬ-ಹುದು
ಅಂಕುಸ-ರಾಯ-ವೊಡೆ-ಯರ
ಅಂಕೆಯ
ಅಂಕೆಯ-ದಂಣ್ನಾಯ-ಕರ
ಅಂಕೆಯ-ನಾಯ-ಕನು
ಅಂಗಡಿ
ಅಂಗಡಿ-ಗ-ಳನ್ನು
ಅಂಗಡಿ-ಗ-ಳನ್ನೂ
ಅಂಗಡಿ-ಗ-ಳಲ್ಲಿ
ಅಂಗಡಿ-ಗಳಿ-ರುತ್ತಿದ್ದು
ಅಂಗಡಿ-ಗಳು
ಅಂಗಡಿ-ದೆರೆ
ಅಂಗಡಿ-ದೆರೆ-ಯನ್ನು
ಅಂಗಡಿ-ದೆಱೆ-ಯಿಂದ
ಅಂಗಡಿಯ
ಅಂಗಡಿ-ಯಿಂದಾ-ದಾಯ
ಅಂಗಡಿ-ಸಾಲು
ಅಂಗಡಿ-ಸೆಟ್ಟಿ-ಯ-ವ-ರೆಂದು
ಅಂಗ-ಬ-ದಿಯರ್
ಅಂಗ-ಭಾಗೆಯ
ಅಂಗ-ಭೋಗ
ಅಂಗ-ಭೋ-ಗಕ್ಕೆ
ಅಂಗ-ಭೋಗವೂ
ಅಂಗರ
ಅಂಗ-ರಂಗ
ಅಂಗ-ರಂಗ-ಭೋಗ
ಅಂಗ-ರಂಗ-ಭೋ-ಗಕ್ಕೆ
ಅಂಗರ-ಕ-ರಿಗೆ
ಅಂಗರಕ್ಕ
ಅಂಗರಕ್ಕ-ನೆಂದು
ಅಂಗ-ರಕ್ಷಕ
ಅಂಗ-ರಕ್ಷಕ-ನಾಗಿ
ಅಂಗ-ರಕ್ಷಕರು
ಅಂಗರಿ-ಕರು-ಕಾವ-ಲು-ಗಾರ
ಅಂಗಳ
ಅಂಗಳ-ದಲ್ಲಿ
ಅಂಗಳ-ವನ್ನು
ಅಂಗ-ವಾಗಿ
ಅಂಗ-ವಾ-ಗಿದ್ದ
ಅಂಗ-ವಾಗಿದ್ದರು
ಅಂಗಿ
ಅಂಗೀ-ಕರಿಸಿ
ಅಂಘರಕ-ರಿಗೆ
ಅಂಚಿತಮಂನಿಗೆ
ಅಂಚಿತಮಂನು
ಅಂಚಿ-ನಲ್ಲಿವೆ
ಅಂಚಿಲ್ಲಿ-ರುವ
ಅಂಚೆ
ಅಂಜಲೀ-ಬದ್ಧ-ವಾಗಿ
ಅಂಟಿ-ಸಿದುದು
ಅಂಡಲೆ-ದರಿ-ನರ-ಪಾ-ಳರ
ಅಂಡ-ಲೆಯುತ್ತಿದ್ದ
ಅಂಡ್
ಅಂಣ
ಅಂಣ-ಗಳ
ಅಂಣನ
ಅಂಣ-ನೆಂದೆನಿಸಿ
ಅಂಣಾಜೈಯ-ನ-ವರ
ಅಂಣಾಜೈಯ-ನ-ವರು
ಅಂಣಾಜೈಯ್ಯ
ಅಂಣೂರು
ಅಂಣ್ನಂಗೆ
ಅಂಣ್ನಯ್ಯನು
ಅಂತಃಕಲ-ಹದ
ಅಂತಃಪುರ-ದಲ್ಲಿ
ಅಂತಃಪುರಾಧ್ಯಕ್ಷ
ಅಂತಪ್ಪ
ಅಂತಪ್ಪ-ಯತಿ-ವ-ರಿಗೆ
ಅಂತರ
ಅಂತ-ರಂಗದ
ಅಂತರ-ದಲ್ಲಿ-ರುವು-ದ-ರಿಂದ
ಅಂತರ-ವಳ್ಳಿ
ಅಂತರ-ವಳ್ಳಿಗೆ
ಅಂತರ-ವಳ್ಳಿಯ
ಅಂತರ-ವಳ್ಳಿ-ಯಲ್ಲಿ
ಅಂತರ-ವಳ್ಳಿ-ವೃತ್ತಿಯ
ಅಂತರ-ವಿದೆ
ಅಂತರ-ಹಳ್ಳಿ
ಅಂತರಾಳ
ಅಂತರಿಕ
ಅಂತರ್ಗತ
ಅಂತರ್ಗತ-ವಾ-ಗಿತ್ತು
ಅಂತರ್ಗತ-ವಾಗಿತ್ತೆಂದು
ಅಂತರ್ಗತ-ವಾಗಿದ್ದವು
ಅಂತರ್ಗತ-ವಾಗಿದ್ದ-ವೆಂದು
ಅಂತಸ್ತು-ಗ-ಳನ್ನು
ಅಂತಹ
ಅಂತಹ-ವ-ರಲ್ಲಿ
ಅಂತಹ-ವರು
ಅಂತಿಂಥವ-ರಲ್ಲ
ಅಂತಿಮ
ಅಂತು
ಅಂತೂ
ಅಂತೆಂಬರ-ಗಂಡ
ಅಂತೆಯೇ
ಅಂತ್ಯ
ಅಂತ್ಯ-ಕಾಲದ
ಅಂತ್ಯ-ಕಾಲ-ದಲ್ಲಿ
ಅಂತ್ಯ-ಗೊಂಡಿತು
ಅಂತ್ಯ-ದಲ್ಲಿ
ಅಂತ್ಯ-ದೊಂದಿಗೆ
ಅಂತ್ಯ-ವಾಗುತ್ತದೆ
ಅಂತ್ಯ-ವಾ-ಗುತ್ತವೆ
ಅಂತ್ಯ-ವಾಗುವ
ಅಂದ
ಅಂದ-ಮಾತ್ರಕ್ಕೆ
ಅಂದ-ಮೇಲೆ
ಅಂದರೆ
ಅಂದ-ರೆ-ರಲ್ಲಿ
ಅಂದಾಜು
ಅಂದಿದ್ದ
ಅಂದಿನ
ಅಂದಿ-ನಿಂದ
ಅಂದೇ
ಅಂನ-ದಾನ
ಅಂನ್ಯಾಯ
ಅಂನ್ಯಾ-ಯನ
ಅಂಬರೀಶ
ಅಂಬಲಿ
ಅಂಬಾ-ದೇವಿ
ಅಂಬಾರಿಯ
ಅಂಬುಧಿ
ಅಂಬುರುಹ-ವನ
ಅಂಮನ-ಪುರದ
ಅಂಶ
ಅಂಶ-ಗಳ
ಅಂಶ-ಗ-ಳನ್ನು
ಅಂಶ-ಗ-ಳಾಗಿವೆ
ಅಂಶ-ಗಳಾವುವೂ
ಅಂಶ-ಗ-ಳಿಂದ
ಅಂಶ-ಗಳಿವೆ
ಅಂಶ-ಗಳು
ಅಂಶ-ವನ್ನು
ಅಂಶ-ವಾಗಿದೆ
ಅಂಶ-ವಾಗು-ವು-ದಿಲ್ಲ
ಅಂಶ-ವಿದೆ
ಅಂಶ-ವಿಲ್ಲ
ಅಂಶ-ವಿಲ್ಲದ
ಅಂಶವು
ಅಂಶವೂ
ಅಂಶೀ-ಭೂತ-ರಾದ
ಅಉ-ಬಳ-ದೇವ
ಅಕಜಾ-ಪುರವು
ಅಕಣ್ಡಿತ-ದೀಪಂ
ಅಕಬೆಯುಂ
ಅಕಬ್ಬೆ
ಅಕರ
ಅಕರ-ವಾಗಿ
ಅಕಲ್ಪ
ಅಕಳಂಕ
ಅಕಳಂಕ-ದೇವ
ಅಕಳಂಕ-ದೇವ-ರಿಂದ
ಅಕಳಂಕನ
ಅಕ-ಸಾಲೆ
ಅಕ-ಸಾಲೆ-ಗಳು
ಅಕಾಡೆಮಿ-ಯಿಂದ
ಅಕಾ-ರಾದಿ-ಯಾಗಿ
ಅಕಾಲ-ವರ್ಷ
ಅಕಾಲ-ವರ್ಷನು
ಅಕಾಲ-ವರ್ಷ-ನು-ಇಮ್ಮಡಿ
ಅಕ್ಕ
ಅಕ್ಕ-ಚೆನ್ನಿ-ಸೆಟ್ಟಿಯ
ಅಕ್ಕ-ಜಾ-ಪುರದ
ಅಕ್ಕ-ತಂಗಿಯೋ
ಅಕ್ಕನ
ಅಕ್ಕ-ಪಕ್ಕದ
ಅಕ್ಕ-ಪಕ್ಕ-ದಲ್ಲಿ
ಅಕ್ಕ-ಪಕ್ಕ-ದಲ್ಲಿದ್ದ
ಅಕ್ಕ-ಪಕ್ಕ-ದಲ್ಲಿದ್ದು
ಅಕ್ಕ-ಪಕ್ದ
ಅಕ್ಕಪ್ಕ-ದಲ್ಲಿ
ಅಕ್ಕ-ಮಹ-ದೇವಿಯು
ಅಕ್ಕಮಾ
ಅಕ್ಕಯ್ಯನ
ಅಕ್ಕ-ರ-ಸಾಕ್ಷಿ
ಅಕ್ಕ-ಲ-ಅಕ್ಕನ
ಅಕ್ಕ-ಲ-ಚೆನ್ನಿ-ಸೆಟ್ಟಿ
ಅಕ್ಕ-ಸಾಲಿ
ಅಕ್ಕ-ಸಾಲಿ-ಕೆಗೆ
ಅಕ್ಕ-ಸಾಲಿ-ಕೆ-ಯನ್ನು
ಅಕ್ಕ-ಸಾಲಿ-ಗರು
ಅಕ್ಕ-ಸಾಲೆ
ಅಕ್ಕ-ಸಾಲೆ-ಗಳು
ಅಕ್ಕ-ಸಾಲೆಗೆ
ಅಕ್ಕಾಳೆಯ
ಅಕ್ಕಿ
ಅಕ್ಕಿಗೆ
ಅಕ್ಕಿ-ಪಡಿ-ಯನ್ನು
ಅಕ್ಕಿಯ
ಅಕ್ಕಿ-ಯನ್ನು
ಅಕ್ಕಿ-ಯೆಬ್ಬಾಳು
ಅಕ್ಕಿ-ಹೆಬ್ಬಾ-ಳಿಗೆ
ಅಕ್ಕಿ-ಹೆಬ್ಬಾಳಿನ
ಅಕ್ಕಿ-ಹೆಬ್ಬಾಳು
ಅಕ್ಟೋ-ಬರ್
ಅಕ್ಷತೃ-ತೀಯ-ದಂದು
ಅಕ್ಷತೆ
ಅಕ್ಷ-ದಲು
ಅಕ್ಷಯ
ಅಕ್ಷರ
ಅಕ್ಷರ-ಗ-ಳನ್ನೂ
ಅಕ್ಷರ-ಗಳು
ಅಕ್ಷರ-ದಲು
ಅಕ್ಷರ-ದಲೂ
ಅಕ್ಷರ-ವನ್ನು
ಅಕ್ಷರಾ-ವೃತ್ತಿಯ
ಅಖಂಡ
ಅಖಂಡ-ಬಾ-ಗಿಲಿನ
ಅಖಂಡಿತ
ಅಖಂಡಿತ-ವಹ
ಅಖಂಡಿತ-ವಾದ
ಅಖಿಲ-ಭಾರತ
ಅಖಿಲಾಂಡ-ಕೋಟಿ
ಅಖಿಳ-ಗುಣ-ಧಾರೆ
ಅಗಡಿ
ಅಗಣ್ಯ
ಅಗಣ್ಯ-ಪುಣ್ಯವೇ
ಅಗತಿ-ಯಪ್ಪ
ಅಗತಿ-ಯಪ್ಪನ
ಅಗತಿ-ಯಪ್ಪ-ನಿಗೆ
ಅಗತ್ತಿ
ಅಗತ್ಯ
ಅಗತ್ಯಕ್ಕೆ
ಅಗತ್ಯ-ಗ-ಳನ್ನು
ಅಗತ್ಯ-ಗ-ಳಿಗೆ
ಅಗತ್ಯ-ವಾ-ಗಿತ್ತು
ಅಗತ್ಯ-ವಾಗಿದೆ
ಅಗತ್ಯ-ವಾದ
ಅಗತ್ಯ-ವಾ-ದು-ದನ್ನು
ಅಗತ್ಯ-ವಿಲ್ಲ-ವೆಂದು
ಅಗತ್ಯ-ವೆಂದು
ಅಗಮ್ಯ-ವಾಗಿತ್ತೆಂದು
ಅಗರ
ಅಗರ-ದಿಂದ
ಅಗರ-ದುರ್ಗ-ಅಗ್ರ-ಹಾರ
ಅಗರ-ಹಾರ-ಬಾಚ-ಹಳ್ಳಿ-ಯಲ್ಲಿ-ರುವ
ಅಗಲ
ಅಗಲ-ಗಳ
ಅಗಲ-ಗ-ಳನ್ನು
ಅಗಲ-ವಾದ
ಅಗಸ-ಗೊ-ರವ
ಅಗ-ಸರ-ಹಳ್ಳಿ
ಅಗಸ್ತ್ಯೇಶ್ವರ
ಅಗಾಧ
ಅಗೆದು
ಅಗೆಯುತ್ತಿದ್ದಾಗ
ಅಗ್ನಿಗ-ಳಂತೆ
ಅಗ್ನಿಪ್ರವೇಶ
ಅಗ್ಯ-ವಾದ
ಅಗ್ರ
ಅಗ್ರ-ಅ-ಹಾರ
ಅಗ್ರ-ಗಣಿ-ಯಾ-ಗಿದ್ದ
ಅಗ್ರ-ಗಣ್ಯ-ನೆನಿಸಿ
ಅಗ್ರ-ವಧು
ಅಗ್ರ-ವಧು-ಅಗ್ರ-ಸತಿ
ಅಗ್ರ-ಸತಿ
ಅಗ್ರ-ಸತಿ-ಯಾದ
ಅಗ್ರ-ಹಾರ
ಅಗ್ರ-ಹಾರಂ
ಅಗ್ರ-ಹಾರಕ್ಕೆ
ಅಗ್ರ-ಹಾರ-ಗಳ
ಅಗ್ರ-ಹಾರ-ಗಳಂತೆ
ಅಗ್ರ-ಹಾರ-ಗಳನ್ನಾಗಿ
ಅಗ್ರ-ಹಾರ-ಗ-ಳನ್ನು
ಅಗ್ರ-ಹಾರ-ಗ-ಳಲ್ಲಿ
ಅಗ್ರ-ಹಾರ-ಗಳಲ್ಲಿದ್ದ
ಅಗ್ರ-ಹಾರ-ಗ-ಳಾಗಿದ್ದ
ಅಗ್ರ-ಹಾರ-ಗ-ಳಿಂದ
ಅಗ್ರ-ಹಾರ-ಗಳಿಗೂ
ಅಗ್ರ-ಹಾರ-ಗ-ಳಿಗೆ
ಅಗ್ರ-ಹಾರ-ಗಳಿವೆ
ಅಗ್ರ-ಹಾರ-ಗಳು
ಅಗ್ರ-ಹಾರ-ಗಳೂ
ಅಗ್ರ-ಹಾರ-ಗಳೆಂದರೆ
ಅಗ್ರ-ಹಾರ-ಗ-ಳೆಂದು
ಅಗ್ರ-ಹಾರದ
ಅಗ್ರ-ಹಾರ-ದ-ಅತ್ತಿ-ಗುಪ್ಪೆ
ಅಗ್ರ-ಹಾರ-ದ-ಮಜ್ಜಿಗೆ-ಪುರದ
ಅಗ್ರ-ಹಾರ-ದಲ್ಲಿ
ಅಗ್ರ-ಹಾರ-ದಲ್ಲಿದ್ದ
ಅಗ್ರ-ಹಾರ-ದ-ವ-ನಾಗಿದ್ದ-ನೆಂಬುದು
ಅಗ್ರ-ಹಾರ-ದ-ವ-ರೆಂದು
ಅಗ್ರ-ಹಾರ-ದಾನ-ವನ್ನು
ಅಗ್ರ-ಹಾರ-ದಿಂದ
ಅಗ್ರ-ಹಾರದ್ವಯ
ಅಗ್ರ-ಹಾರದ್ವಯಂ
ಅಗ್ರ-ಹಾರದ್ವಯ-ವನ್ನಾಗಿ
ಅಗ್ರ-ಹಾರ-ನ-ವನ್ನು
ಅಗ್ರ-ಹಾರ-ಬಂಡಿ-ಹೊಳೆ
ಅಗ್ರ-ಹಾರ-ಬಾಚ-ಹಳ್ಳಿ
ಅಗ್ರ-ಹಾರ-ಬಾಚ-ಹಳ್ಳಿಯ
ಅಗ್ರ-ಹಾರ-ಬಾಚ-ಹಳ್ಳಿ-ಯಂತೆ
ಅಗ್ರ-ಹಾರ-ಬಾಚ-ಹಳ್ಳಿ-ಯನ್ನು
ಅಗ್ರ-ಹಾರ-ಬಾಚ-ಹಳ್ಳಿ-ಯಲ್ಲಿ
ಅಗ್ರ-ಹಾರ-ಬಾಚ-ಹಳ್ಳಿ-ಯಲ್ಲಿ-ರುವ
ಅಗ್ರ-ಹಾರ-ಬಾಚೆಯ-ಹಳ್ಳಿ
ಅಗ್ರ-ಹಾರ-ಬೆಳ-ಗಲಿ
ಅಗ್ರ-ಹಾರ-ಬೆಳ-ಗಲಿ-ಯಲ್ಲಿ
ಅಗ್ರ-ಹಾರ-ಬೆಳಗುಲಿ
ಅಗ್ರ-ಹಾರ-ಬೆಳುಗ-ಲಿಯ
ಅಗ್ರ-ಹಾರ-ಬೆಳುಗ-ಲಿಯಲ್ಲಿ
ಅಗ್ರ-ಹಾರ-ಯಾದ-ವ-ನಾ-ರಾಯಣ
ಅಗ್ರ-ಹಾರವ
ಅಗ್ರ-ಹಾರ-ವನ್ನಾಗಿ
ಅಗ್ರ-ಹಾರ-ವನ್ನಾಗಿಯೂ
ಅಗ್ರ-ಹಾರ-ವನ್ನಾಗಿಸಿ
ಅಗ್ರ-ಹಾರ-ವನ್ನು
ಅಗ್ರ-ಹಾರ-ವನ್ನೂ
ಅಗ್ರ-ಹಾರ-ವಾಗಲೀ
ಅಗ್ರ-ಹಾರ-ವಾಗಿ
ಅಗ್ರ-ಹಾರ-ವಾ-ಗಿತ್ತು
ಅಗ್ರ-ಹಾರ-ವಾಗಿತ್ತೆಂದು
ಅಗ್ರ-ಹಾರ-ವಾಗಿತ್ತೆಂಬ
ಅಗ್ರ-ಹಾರ-ವಾಗಿತ್ತೇ
ಅಗ್ರ-ಹಾರ-ವಾಗಿದೆ
ಅಗ್ರ-ಹಾರ-ವಾ-ಗಿದ್ದ
ಅಗ್ರ-ಹಾರ-ವಾಗಿದ್ದಿರ-ಬ-ಹುದು
ಅಗ್ರ-ಹಾರ-ವಾ-ಗಿದ್ದು
ಅಗ್ರ-ಹಾರ-ವಾಗಿದ್ದು-ದ-ರಿಂದ
ಅಗ್ರ-ಹಾರ-ವಾಗಿಯೂ
ಅಗ್ರ-ಹಾರ-ವಾಗಿ-ರ-ಬ-ಹುದು
ಅಗ್ರ-ಹಾರ-ವಾಗಿ-ರ-ಬಹು-ದೆಂದು
ಅಗ್ರ-ಹಾರ-ವಾಗಿ-ರಲೇ
ಅಗ್ರ-ಹಾರ-ವಾಗಿ-ರುವುದು
ಅಗ್ರ-ಹಾರ-ವಾಗು-ವು-ದಕ್ಕೆ
ಅಗ್ರ-ಹಾರ-ವಾಗು-ಹ-ದಕ್ಕೆ
ಅಗ್ರ-ಹಾರ-ವಾದ
ಅಗ್ರ-ಹಾರ-ವಾ-ದಂತೆ
ಅಗ್ರ-ಹಾರ-ವಾಯಿತು
ಅಗ್ರ-ಹಾರ-ವಿತ್ತೆಂದೂ
ಅಗ್ರ-ಹಾರ-ವಿಲ್ಲ
ಅಗ್ರ-ಹಾರವು
ಅಗ್ರ-ಹಾರವೂ
ಅಗ್ರ-ಹಾರ-ವೆಂದರೆ
ಅಗ್ರ-ಹಾರ-ವೆಂದು
ಅಗ್ರ-ಹಾರ-ವೆಂದೇ
ಅಗ್ರ-ಹಾರ-ವೆಂಬ
ಅಗ್ರ-ಹಾರವೋ
ಅಗ್ರ-ಹಾರೇ
ಅಗ್ರಾ-ಹರ-ವನ್ನು
ಅಗ್ರಾ-ಹಾರ
ಅಘ-ಲ-ಯದ
ಅಘ-ಹಾರಿ
ಅಚಲಾನಂದ
ಅಚಿನ್ತ್ಯ
ಅಚ್ಚ-ಕನ್ನಡಿಗ-ರಾದ
ಅಚ್ಚ-ರಿಯ
ಅಚ್ಚ-ರಿಯೇನಿಲ್ಲ
ಅಚ್ಚು-ಕಟ್ಟಿಗೆ
ಅಚ್ಚು-ಕಟ್ಟು
ಅಚ್ಚು-ಕಟ್ಟು-ಗ-ಳನ್ನು
ಅಚ್ಚು-ಕಟ್ಟು-ಗಳು
ಅಚ್ಚು-ಕಟ್ಟು-ಗ-ಳೆಂದೂ
ಅಚ್ಚುತ-ರಾಯ
ಅಚ್ಚೊತ್ತಿದ
ಅಚ್ಯತ-ರಾಯ-ವೀರಣ್ಣ
ಅಚ್ಯತೇಂದ್ರ
ಅಚ್ಯುತ
ಅಚ್ಯುತ-ದೇವ
ಅಚ್ಯುತ-ದೇವನ
ಅಚ್ಯುತ-ದೇವ-ನನ್ನು
ಅಚ್ಯುತ-ದೇವನು
ಅಚ್ಯುತ-ದೇವ-ಪುರ-ವಾದ
ಅಚ್ಯುತ-ದೇವ-ಮಹಾ-ರಾಯನು
ಅಚ್ಯುತ-ದೇವ-ರಾಯನ
ಅಚ್ಯುತನ
ಅಚ್ಯುತ-ಪುರ-ವೆಂಬ
ಅಚ್ಯುತ-ಮಹಾ-ರಾಯ-ರಿಗೆ
ಅಚ್ಯುತ-ರಾಯ
ಅಚ್ಯುತ-ರಾಯನ
ಅಚ್ಯುತ-ರಾಯ-ನನ್ನು
ಅಚ್ಯುತ-ರಾಯ-ನಿಗೆ
ಅಚ್ಯುತ-ರಾಯನು
ಅಚ್ಯುತ-ರಾಯ-ರಿಗೆ
ಅಚ್ಯುತ-ಸ-ಮುದ್ರ
ಅಚ್ಯುತಾಖ್ಯೋ
ಅಚ್ಯುತಿ-ಮಯ್ಯ
ಅಚ್ಯುತಿ-ಮಯ್ಯ-ಗಳು
ಅಚ್ಯುತೇಂದ್ರ
ಅಚ್ಯುತೇಂದ್ರನು
ಅಚ್ಯುತೇಂದ್ರ-ಪುರ-ವೆಂಬ
ಅಚ್ಯುತೇಂದ್ರ-ಮಹಾ-ರಾಯ
ಅಚ್ಯುತೇಂದ್ರ-ಮಹಾ-ರಾಯ-ಸ-ಮುದ್ರ
ಅಜಸೋ-ಯಪ್ಪನು
ಅಜಿತ-ನಾಥ-ಪುರಾಣ-ದಲ್ಲಿ
ಅಜಿತ-ಮುನಿ
ಅಜಿತ-ಸೇನ
ಅಜಿತ-ಸೇನನ
ಅಜಿತ-ಸೇನ-ಮುನಿ
ಅಜಿತ-ಸೇನ-ರೆಂದು
ಅಜಿತ-ಸೇನಾ-ಚಾರ್ಯರ
ಅಜಿತಾ-ದೇವಿ
ಅಜ್ಜ
ಅಜ್ಜ-ಊರು
ಅಜ್ಜನ
ಅಜ್ಜ-ನ-ಹಳ್ಳಿ
ಅಜ್ಜ-ನಾಯ-ಕ-ನ-ಹಳ್ಳಿ
ಅಜ್ಜ-ನಿಗೆ
ಅಜ್ಜ-ವೂ-ರನ್ನೇ
ಅಜ್ಜ-ವೂ-ರಾದ
ಅಜ್ಜ-ವೂರು
ಅಜ್ಜಿ
ಅಜ್ಞಾತ
ಅಜ್ಞಾ-ತನ
ಅಜ್ಞೆ
ಅಜ್ಞೆಯ
ಅಟಕೇಶ್ವರ
ಅಟ್ಟದೆಱೆ
ಅಟ್ಟಾ-ಡಿಸಿ-ಕೊಂಡು
ಅಟ್ಟಿ
ಅಟ್ಟಿ-ದನು
ಅಟ್ಟಿ-ದ-ನೆಂದು
ಅಟ್ಟಿ-ದ-ನೆಂದೂ
ಅಟ್ಟಿ-ದಾಗ
ಅಟ್ಟಿ-ಸಿ-ಕೊಂಡು
ಅಟ್ಟುಣ್ಣಲೀ-ಯದೆ
ಅಠ-ವಣ
ಅಠ-ವಣೆ
ಅಠವ-ಣೆಯ
ಅಠ್ಠ-ವಣೆಗೆ
ಅಡ
ಅಡಕೆ
ಅಡಕೆ-ಮರದ
ಅಡ-ಕೆಯ
ಅಡಕೆ-ಯ-ತೊಟ
ಅಡಕೆ-ಯ-ತೋಟ
ಅಡಗಿ-ದರೂ
ಅಡಗಿದ್ದನ್ನು
ಅಡಗಿ-ಸಲು
ಅಡಗಿಸಿ
ಅಡಗಿ-ಸಿದ
ಅಡಗಿಸಿ-ದ-ನೆಂದೂ
ಅಡಗಿ-ಸುವುದಕ್ಕಾಗಿ
ಅಡ-ತೆ-ರಿಗೆ
ಅಡ-ದೆರೆ
ಅಡಪ
ಅಡಳಿತ-ದಲ್ಲಿ
ಅಡಿ
ಅಡಿಕೆ
ಅಡಿಕೆ-ಮರದ
ಅಡಿಕೆಯ
ಅಡಿಕೆಯು
ಅಡಿ-ಗಲ್ಲು
ಅಡಿ-ಗಳ
ಅಡಿ-ಗಳು
ಅಡಿ-ಗೆರಗು-ವಂತೆ
ಅಡಿ-ಗೈ-ಮಾನ್
ಅಡಿ-ಯಲ್ಲಿ
ಅಡುಗಬ್ಬಿನ
ಅಡುವಿನ
ಅಡೆ-ಕಲ್ಲು
ಅಡೆ-ಕಲ್ಲು-ವಣ
ಅಡೆ-ತಡೆ
ಅಡೆ-ತಡೆ-ಗ-ಳನ್ನು
ಅಡೆ-ತಡೆ-ಗಳೂ
ಅಡೆ-ಪಯ್ಯ
ಅಡೆ-ಪಯ್ಯಂ
ಅಡೆಯಪ್ಪಯ್ಯ
ಅಡ್ಡ-ಲಾಗಿ
ಅಡ್ಡ-ಹೆಸ-ರನ್ನು
ಅಡ್ಡ-ಹೆ-ಸರು
ಅಡ್ಡಾ-ಯದ
ಅಡ್ಡಾ-ಯದದ
ಅಡ್ಡಾ-ಯಿ-ದದ
ಅಡ್ಡಾಯುಧಅಢಾಯುಧ
ಅಡ್ಡಾಯ್ದದ
ಅಡ್ಡಿ-ಪಡಿ-ಸಿದರು
ಅಡ್ಡಿ-ಯಿಲ್ಲ
ಅಣತಿ
ಅಣವ-ಸ-ಮುದ್ರದ
ಅಣಿಲ-ಹಳ್ಳಿ
ಅಣಿಲೇಶ್ವರ
ಅಣುಕ್ಕರ್ಅಣುಗ-ಜೀವಿ-ತ-ದ-ವರು
ಅಣುವ-ಸ-ಮುದ್ರ
ಅಣುವ-ಸ-ಮುದ್ರದ
ಅಣುವ-ಸ-ಮುದ್ರ-ದಲ್ಲಿ-ಇಂದಿನ
ಅಣೆ
ಅಣೆ-ಕಟ್ಟನ್ನು
ಅಣೆ-ಕಟ್ಟಾಗಿರ-ಬ-ಹುದು
ಅಣೆ-ಕಟ್ಟಾ-ಗಿ-ರುವ
ಅಣೆ-ಕಟ್ಟಿದೆ
ಅಣೆ-ಕಟ್ಟಿನ
ಅಣೆ-ಕಟ್ಟು
ಅಣೆ-ಕಟ್ಟು-ಗ-ಳನ್ನು
ಅಣೆ-ಕಟ್ಟು-ಗಳು
ಅಣೆ-ಕಟ್ಟು-ಗಳು-ಕಟ್ಟೆ-ಗಳು-ಕಟ್ಟು-ಕಾಲುವೆ-ಗಳು
ಅಣೆ-ಕಟ್ಟೆಯ
ಅಣೆ-ಕಟ್ಟೆ-ಯನ್ನು
ಅಣೆ-ಕಟ್ಟೆ-ಯಿಂದ
ಅಣೆ-ಗ-ಳನ್ನು
ಅಣೆ-ಯನ್ನು
ಅಣೆ-ಯನ್ನೂ
ಅಣ್ಡಾರ್
ಅಣ್ಣ
ಅಣ್ಣಂದಿರು
ಅಣ್ಣಂದಿರೂ
ಅಣ್ಣ-ತಮ್ಮಂದಿರ
ಅಣ್ಣ-ತಮ್ಮಂದಿ-ರಂತೆ
ಅಣ್ಣ-ತಮ್ಮಂದಿ-ರನ್ನು
ಅಣ್ಣ-ತಮ್ಮಂದಿ-ರಾ-ಗಿದ್ದು
ಅಣ್ಣ-ತಮ್ಮಂದಿ-ರೊಳಗೆ
ಅಣ್ಣ-ದೀಕ್ಷಿತ
ಅಣ್ಣನ
ಅಣ್ಣ-ನಂಕುರ
ಅಣ್ಣ-ನನ್ನು
ಅಣ್ಣ-ನ-ವರ
ಅಣ್ಣ-ನಾದ
ಅಣ್ಣ-ನಿ-ಗಿಂತ
ಅಣ್ಣನು
ಅಣ್ಣನೂ
ಅಣ್ಣ-ನೆಂದು
ಅಣ್ಣ-ನೆಂಬುದೂ
ಅಣ್ಣನೇ
ಅಣ್ಣನ್
ಅಣ್ಣಪ್ಪ
ಅಣ್ಣ-ಬೂಚಣ-ನಿಗೆ
ಅಣ್ಣಯ್ಯ-ನ-ವರ
ಅಣ್ಣಾ-ಮಲೆ
ಅಣ್ನಯ್ಯ
ಅಣ್ನಾ-ಮಲೆ
ಅಣ್ಪಿಳ್ಳೈ-ಅನಂದಾನ್ಪಿಳ್ಳೈ
ಅತನ
ಅತಿ-ಕುಪ್ಪೆ
ಅತಿ-ಕುಪ್ಪೆಯ
ಅತಿಕ್ರಮ-ಣಕ್ಕೆ
ಅತಿಕ್ರಮಿ-ಸದೇ
ಅತಿಥಿ-ಗಳ
ಅತಿಥಿ-ಗ-ಳಾಗಿ
ಅತಿ-ರಸ
ಅತಿ-ರಸ-ದ-ಹರಿ-ವಾಣ
ಅತಿ-ರಸ-ನೈ-ವೇದ್ಯಕ್ಕೆ
ಅತಿರಾತ್ರ
ಅತಿರಾತ್ರಿಗೆ
ಅತಿವಿಷಮ
ಅತಿಶಯ-ವಾಗಿ
ಅತಿಶಯ-ವಾದ
ಅತಿಶಯೋಕ್ತಿ
ಅತಿ-ಸಣ್ಣ
ಅತೀ
ಅತೀ-ತನಾ-ದನು
ಅತೀಪ್ರೇ-ಮದ
ಅತು-ವಾಸು-ವಿನ
ಅತ್ತಣಿದ
ಅತ್ತಿ-ಕುಪ್ಪೆ
ಅತ್ತಿ-ಕುಪ್ಪೆ-ಗ-ಳನ್ನು
ಅತ್ತಿ-ಕುಪ್ಪೆಯ
ಅತ್ತಿ-ಕುಪ್ಪೆಯು
ಅತ್ತಿ-ಕುಪ್ಪೆ-ಯೆಂಬ
ಅತ್ತಿ-ಗಾಲಾ
ಅತ್ತಿ-ಗುಪ್ಪೆ
ಅತ್ತಿ-ಗುಪ್ಪೆ-ಕೃಷ್ಣ-ರಾಜ-ಪೇಟೆ
ಅತ್ತಿ-ಗುಪ್ಪೆಗೆ
ಅತ್ತಿಗೆ
ಅತ್ತಿ-ಗೊಂಡ-ನ-ಹಳ್ಳಿ
ಅತ್ತಿ-ತಾಳಾ-ದತ್ತ-ಲ-ಪುರ-ವನ್ನು
ಅತ್ತಿಮಬ್ಬೆಗೆ
ಅತ್ತಿಮಬ್ಬೆಯ
ಅತ್ತಿಮಬ್ಬೆ-ಯನ್ನು
ಅತ್ತಿಯ
ಅತ್ತೀ-ಕುಪ್ಪೆ-ಯಲ್ಲಿ
ಅತ್ಯಂತ
ಅತ್ಯಮ-ನಾಯಕ
ಅತ್ಯಮ-ನಾಯ-ಕನ
ಅತ್ಯುತ್ತಮ
ಅತ್ಯುನ್ನತ
ಅತ್ರಿ-ಗೋತ್ರದ
ಅಥವ
ಅಥವಾ
ಅಥವ್
ಅಥಾವ
ಅದಕೆ
ಅದಕ್ಕಾಗಿ
ಅದಕ್ಕಾಗಿಯೇ
ಅದಕ್ಕಿಂತ
ಅದಕ್ಕೂ
ಅದಕ್ಕೆ
ಅದಕ್ಕೇ
ಅದ-ನಾಗ-ದೆಂದವಂ
ಅದನ್ನ
ಅದನ್ನ-ರಿತೇ
ಅದನ್ನು
ಅದನ್ನೂ
ಅದನ್ನೆಲ್ಲಾ
ಅದನ್ನೇ
ಅದರ
ಅದ-ರಂತೆ
ಅದ-ರಂತೆಯೇ
ಅದ-ರಲ್ಲಿ
ಅದ-ರಲ್ಲಿದ್ದ
ಅದ-ರಲ್ಲಿಯೂ
ಅದ-ರಲ್ಲಿ-ರುವ
ಅದ-ರಲ್ಲೂ
ಅದ-ರಿಂದ
ಅದ-ರಿಂದಲೇ
ಅದ-ರಿಂದಾಗಿ
ಅದ-ರಿಂದಾಗಿಯೆ
ಅದ-ರಿಂದಾಗಿಯೇ
ಅದರೆ
ಅದ-ರೊಡನೆ
ಅದರೊಳಕ್ಕೆ
ಅದರೊಳ-ಗಿ-ರುವ
ಅದ-ರೊಳಗೆ
ಅದಲ-ಗೆರೆ
ಅದಲ್ಲ
ಅದಾ-ಗುತ್ತಲೂ
ಅದಾದ-ನಂತರ
ಅದಿರದ
ಅದು
ಅದೂ
ಅದೂರನ
ಅದೆ
ಅದೆಲ್ಲಾ
ಅದೇ
ಅದೇ-ರೀತಿ
ಅದೇವುದೋ
ಅದೊಂದು
ಅದ್ದ-ಹಳ್ಳಿ
ಅದ್ದಿಯಾಪಳ್ತಿಯ
ಅದ್ದಿ-ಹಳ್ಳಿ
ಅದ್ಭುತ-ವಾಗಿದೆ
ಅದ್ಯ-ತನ
ಅದ್ಯಾಪಿ
ಅದ್ವಿ-ತೀಯ-ವಾದು-ದೆಂದರೆ
ಅದ್ವೈತ
ಅದ್ವೈತ-ವಿಶಿಷ್ಟಾದ್ವೈತ
ಅದ್ವೈತಿ-ಗಳು
ಅಧಮ
ಅಧಿಕ
ಅಧಿಕ-ಬಳ
ಅಧಿಕ-ವಾಗಿ
ಅಧಿ-ಕಾರ
ಅಧಿ-ಕಾರಕ್ಕಾಗಿ
ಅಧಿ-ಕಾರಕ್ಕೆ
ಅಧಿ-ಕಾರ-ಗ-ಳನ್ನು
ಅಧಿ-ಕಾರದ
ಅಧಿ-ಕಾರ-ದಲ್ಲಿ
ಅಧಿ-ಕಾರ-ದಲ್ಲಿದ್ದನು
ಅಧಿ-ಕಾರ-ದಲ್ಲಿದ್ದು
ಅಧಿ-ಕಾರ-ದಲ್ಲಿ-ರುವಾಗ
ಅಧಿ-ಕಾರ-ದಿಂದ
ಅಧಿ-ಕಾರ-ಪ-ದದ
ಅಧಿ-ಕಾರ-ವನ್ನು
ಅಧಿ-ಕಾರ-ವರ್ಗ
ಅಧಿ-ಕಾರ-ವರ್ಗಕ್ಕೆ
ಅಧಿ-ಕಾರ-ವರ್ಗ-ಗಳ
ಅಧಿ-ಕಾರ-ವರ್ಗ-ದ-ವರು
ಅಧಿ-ಕಾರ-ವಿತ್ತು
ಅಧಿ-ಕಾರ-ವಿತ್ತೆಂದು
ಅಧಿ-ಕಾರ-ವಿರ-ಲಿಲ್ಲ
ಅಧಿ-ಕಾರವು
ಅಧಿ-ಕಾರವೂ
ಅಧಿ-ಕಾರ-ವೆಲ್ಲಾ
ಅಧಿ-ಕಾರಸ್ಥರ
ಅಧಿ-ಕಾರಿ
ಅಧಿ-ಕಾರಿ-ಗಳ
ಅಧಿ-ಕಾರಿ-ಗ-ಳನ್ನು
ಅಧಿ-ಕಾರಿ-ಗ-ಳಲ್ಲಿ
ಅಧಿ-ಕಾರಿ-ಗಳಾ-ಗಲೀ
ಅಧಿ-ಕಾರಿ-ಗ-ಳಾಗಿ
ಅಧಿ-ಕಾರಿ-ಗ-ಳಾಗಿದ್ದ
ಅಧಿ-ಕಾರಿ-ಗ-ಳಾಗಿದ್ದರು
ಅಧಿ-ಕಾರಿ-ಗ-ಳಾಗಿದ್ದ-ರೆಂದು
ಅಧಿ-ಕಾರಿ-ಗ-ಳಾಗಿದ್ದಿರ-ಬ-ಹುದು
ಅಧಿ-ಕಾರಿ-ಗ-ಳಾಗಿದ್ದು
ಅಧಿ-ಕಾರಿ-ಗ-ಳಾದ
ಅಧಿ-ಕಾರಿ-ಗ-ಳಿಂದ
ಅಧಿ-ಕಾರಿ-ಗಳಿ-ಗಿ-ರುವು-ದ-ರಿಂದ
ಅಧಿ-ಕಾರಿ-ಗ-ಳಿಗೆ
ಅಧಿ-ಕಾರಿ-ಗ-ಳಿಗೆವೂ
ಅಧಿ-ಕಾರಿ-ಗಳಿದ್ದರು
ಅಧಿ-ಕಾರಿ-ಗಳಿದ್ದ-ರೆಂದು
ಅಧಿ-ಕಾರಿ-ಗಳಿ-ರ-ಬ-ಹುದು
ಅಧಿ-ಕಾರಿ-ಗಳಿ-ರುತ್ತಿದ್ದರು
ಅಧಿ-ಕಾರಿ-ಗಳು
ಅಧಿ-ಕಾರಿ-ಗಳೂ
ಅಧಿ-ಕಾರಿ-ಗ-ಳೆಂದು
ಅಧಿ-ಕಾರಿಗೆ
ಅಧಿ-ಕಾರಿಗೇ
ಅಧಿ-ಕಾರಿಯ
ಅಧಿ-ಕಾರಿ-ಯನ್ನು
ಅಧಿ-ಕಾರಿ-ಯನ್ನೂ
ಅಧಿ-ಕಾರಿ-ಯರ
ಅಧಿ-ಕಾರಿ-ಯಾಗಿ
ಅಧಿ-ಕಾರಿ-ಯಾ-ಗಿದ್ದ
ಅಧಿ-ಕಾರಿ-ಯಾಗಿದ್ದಂತೆ
ಅಧಿ-ಕಾರಿ-ಯಾಗಿದ್ದನು
ಅಧಿ-ಕಾರಿ-ಯಾಗಿದ್ದ-ನೆಂದು
ಅಧಿ-ಕಾರಿ-ಯಾಗಿದ್ದ-ನೆಂಬ
ಅಧಿ-ಕಾರಿ-ಯಾಗಿದ್ದರೂ
ಅಧಿ-ಕಾರಿ-ಯಾಗಿದ್ದಿರ-ಬಹು-ದಾದ
ಅಧಿ-ಕಾರಿ-ಯಾಗಿದ್ದಿರ-ಬ-ಹುದು
ಅಧಿ-ಕಾರಿ-ಯಾಗಿದ್ದಿರ-ಬಹು-ದೆಂದು
ಅಧಿ-ಕಾರಿ-ಯಾ-ಗಿದ್ದು
ಅಧಿ-ಕಾರಿ-ಯಾಗಿ-ರ-ಬ-ಹುದು
ಅಧಿ-ಕಾರಿ-ಯಾಗಿ-ರ-ಬೇಕು
ಅಧಿ-ಕಾರಿ-ಯಾಗಿ-ರುತ್ತಾನೆ
ಅಧಿ-ಕಾರಿ-ಯಾದ
ಅಧಿ-ಕಾರಿಯು
ಅಧಿ-ಕಾರಿಯೂ
ಅಧಿ-ಕಾರಿಯೇ
ಅಧಿ-ಕಾರಿ-ಯೊಬ್ಬ
ಅಧಿ-ಕಾರಿ-ಯೊಬ್ಬನ
ಅಧಿ-ಕಾರಿಯೋ
ಅಧಿಕೃತ
ಅಧಿಕೃತ-ಗೊ-ಳಿಸಿರ
ಅಧಿಕೃತ-ಗೊಳಿ-ಸುತ್ತವೆ
ಅಧಿಕ್ರಯ
ಅಧಿ-ದೈವ-ವಾಗಿ
ಅಧಿನ-ನಾ-ಗಿದ್ದ
ಅಧಿನ-ರಾಗಿ
ಅಧಿ-ಪತಿ
ಅಧಿ-ಪತಿ-ಗ-ಳಾಗಿದ್ದ
ಅಧಿ-ಪತಿ-ಗ-ಳಾಗಿದ್ದರು
ಅಧಿ-ಪತಿ-ಗ-ಳಾಗಿದ್ದ-ರೆಂದು
ಅಧಿ-ಪತಿ-ಯನ್ನಾಗಿ
ಅಧಿ-ಪತಿ-ಯಾಗಿದ್ದಂತೆ
ಅಧಿ-ಪತಿ-ಯಾಗಿದ್ದಿರ-ಬಹು-ದೆಂದು
ಅಧಿ-ಪತಿ-ಯಾ-ಗಿದ್ದು
ಅಧಿ-ಪತಿ-ಯಾದ-ನೆಂಬ
ಅಧಿ-ರಾಜ
ಅಧಿ-ರಾಜತ್ವದ
ಅಧಿ-ರಾಜತ್ವ-ವನ್ನು
ಅಧಿ-ರಾಜರು
ಅಧಿವಸಿತ-ನಾಗಿದ್ದ-ನೆಂದು
ಅಧಿವೇಶನ
ಅಧಿವೇಶನ-ದಲ್ಲಿ
ಅಧಿಷ್ಠಾನ
ಅಧೀನ
ಅಧೀನ-ತೆ-ಯನ್ನು
ಅಧೀನ-ದಲ್ಲಿ
ಅಧೀನ-ದಲ್ಲಿತ್ತು
ಅಧೀನ-ದಲ್ಲಿದ್ದ
ಅಧೀನ-ದಲ್ಲೇ
ಅಧೀನ-ನಾಗಿ
ಅಧೀನ-ರಾಗಿ
ಅಧೀನ-ರಾಗಿದ್ದ-ರೆಂದು
ಅಧೀನ-ವಾಗಿತ್ತೆಂದು
ಅಧ್ಯಕ್ಷ
ಅಧ್ಯಕ್ಷ-ತೆ-ಯಲ್ಲಿ
ಅಧ್ಯಯನ
ಅಧ್ಯಯ-ನಕ್ಕೆ
ಅಧ್ಯಯ-ನ-ಗ-ಳನ್ನೂ
ಅಧ್ಯಯ-ನ-ಗಳು
ಅಧ್ಯಯ-ನದ
ಅಧ್ಯಯ-ನ-ದಲ್ಲಿ
ಅಧ್ಯಯ-ನ-ದಿಂದ
ಅಧ್ಯಯ-ನ-ವನ್ನು
ಅಧ್ಯಯ-ನ-ವಾಗಲೀ
ಅಧ್ಯಯ-ನ-ಶೀ-ಲರು
ಅಧ್ಯಾತ್ಮಿ
ಅಧ್ಯಾಪ-ಕರ
ಅಧ್ಯಾಪಕ-ರಾ-ಗಿದ್ದ
ಅಧ್ಯಾಪಕ-ರಿಗೂ
ಅಧ್ಯಾ-ಪನ
ಅಧ್ಯಾಯ
ಅಧ್ಯಾಯ-ಗಳ
ಅಧ್ಯಾಯ-ಗ-ಳನ್ನು
ಅಧ್ಯಾ-ಯದ
ಅಧ್ಯಾಯ-ದಲ್ಲಿ
ಅಧ್ಯಾಯ-ವನ್ನು
ಅಧ್ಯಾಯ-ವಾರು
ಅಧ್ವರ್ಯು-ಗ-ಳಾದ
ಅನಂತ
ಅನಂತ-ಚಾರ್ಯ
ಅನಂತ-ನಾಥ-ಪುರಾಣ
ಅನಂತ-ನಾಥ-ಪುರಾಣ-ದಲ್ಲಿ
ಅನಂತ-ನಾಥ-ಪುರಾಣ-ದಿಂದ
ಅನಂತ-ಪದ್ಮ-ನಾಭ
ಅನಂತ-ಪುರ
ಅನಂತ-ಬಣಂಜು-ಧರ್ಮ
ಅನಂತಯ್ಯ-ನ-ವರ
ಅನಂತಯ್ಯ-ನಿಗೆ
ಅನಂತರ
ಅನಂತ-ರ-ಕಾಲದ
ಅನಂತ-ರ-ದಲ್ಲಿ
ಅನಂತ-ರವೂ
ಅನಂತ-ರಾಮು
ಅನಂತ-ವೀರ್ಯನ
ಅನಂತ-ವೀರ್ಯ-ರಾದ್ಧಾಂತ-ಪಾರಗ
ಅನಂತ-ಸೂರಿ
ಅನಂತಾ-ಚಾರ್ಯನ
ಅನಂತಾ-ಚಾರ್ಯರ
ಅನಂತಾ-ಚಾರ್ಯರು
ಅನಂತಾಳ್ವಾರರ
ಅನಂತಾಳ್ವಾರ್
ಅನಂತೋಜಿ
ಅನಂಥ-ನಾಥ-ಪುರಾಣ-ದಿಂದ
ಅನತಿ
ಅನತಿ-ದೂರ-ದಲ್ಲಿ-ರುವ
ಅನ-ವರತ
ಅನ-ವರ-ದ-ಖಿಳ-ಸುರಾ-ಸುರ-ನರ-ಪತಿ
ಅನಾಥ-ರಿಗೆ
ಅನಾದಿ
ಅನಾದಿ-ಅಗ್ರ-ಹಾರ
ಅನಾದಿ-ಕಾಲ-ದಿಂದ
ಅನಾದಿ-ಪಂಡಿತ
ಅನಾಯ-ಕತ್ವ
ಅನಾಹುತ-ವನ್ನು
ಅನಿರುದ್ಧ-ನಂತಹ
ಅನಿಸಿಕೆ
ಅನಿ-ಸುತ್ತದೆ
ಅನು-ಕರ-ಣೆ-ಯಾಗಿದೆ
ಅನು-ಕರ-ಣೆ-ಯಿಂದ
ಅನು-ಕರಿಸಿ
ಅನು-ಕರಿ-ಸಿ-ದ-ನೆಂದು
ಅನು-ಕರಿ-ಸಿವೆ
ಅನು-ಕೂಲ
ಅನು-ಕೂಲಕ್ಕಾಗಿ
ಅನು-ಕೂಲಕ್ಕೋಸ್ಕರ
ಅನು-ಕೂಲ-ಗ-ಳನ್ನು
ಅನು-ಕೂಲ-ತೆಯ
ಅನು-ಕೂಲ-ವಾಗು-ವ-ಹಾಗೆ
ಅನುಕ್ರಮ
ಅನು-ಗ-ಮನ-ವೆಂದು
ಅನು-ಗುಣ-ವಾಗಿ
ಅನುಜ
ಅನುಜ-ನಾದ
ಅನುನ-ಯದಿಂ
ಅನುಪಮ
ಅನುಪಮ-ತೇಜಂ
ಅನುಪಮ-ದಾನಿ
ಅನುಪಮ-ಮಹಿ-ಮಾಳಂಬಿನಿ
ಅನುಪಮ-ವಾದ
ಅನು-ಬಂಧ
ಅನು-ಬಂಧ-ಗ-ಳನ್ನಂತೂ
ಅನು-ಬಂಧ-ಗಳು
ಅನು-ಬಂಧ-ದಲ್ಲಿ
ಅನುಭವ
ಅನುಭವ-ವನ್ನು
ಅನುಭವಿಸ-ಬೇಕಾಗಿ
ಅನುಭವಿಸಬೇಕೆಂದೂ
ಅನುಭವಿ-ಸಲು
ಅನುಭವಿಸಿ-ಕೊಂಡು
ಅನುಭವಿಸು-ವಂತೆ
ಅನುಭವಿಸು-ವಂತೆಯೂ
ಅನುಭವಿ-ಸುವರು
ಅನುಭವಿ-ಸು-ವುದು
ಅನುಭವಿ-ಸುವುರು
ಅನು-ಮತ-ದಿಂದ
ಅನು-ಮತಿ
ಅನು-ಮತಿ-ನೀಡಿದ್ದ
ಅನು-ಮತಿಯ
ಅನು-ಮತಿ-ಯನ್ನು
ಅನು-ಮತಿ-ಯಿಂದ
ಅನು-ಮತಿಯೂ
ಅನು-ಮಾನ
ಅನು-ಮಾನ-ವಿಲ್ಲ
ಅನು-ಮಾನಾಸ್ಪದ-ವಾದ
ಅನುಯಾಯಿ
ಅನುಯಾಯಿ-ಗಳ
ಅನುಯಾಯಿ-ಗ-ಳನ್ನು
ಅನುಯಾಯಿ-ಗ-ಳಾಗಿದ್ದರು
ಅನುಯಾಯಿ-ಗ-ಳಾಗಿದ್ದರೆ
ಅನುಯಾಯಿ-ಗ-ಳಾಗಿದ್ದ-ರೆಂದು
ಅನುಯಾಯಿ-ಗ-ಳಾಗಿದ್ದಾ-ರೆಂದು
ಅನುಯಾಯಿ-ಗ-ಳಾಗಿದ್ದು
ಅನುಯಾಯಿ-ಗ-ಳಾದ
ಅನುಯಾಯಿ-ಗಳು
ಅನುಯಾಯಿ-ಗಳೂ
ಅನುಯಾಯಿ-ಗ-ಳೆಂದು
ಅನುಯಾಯಿ-ಗಳೊಡನೆ
ಅನುಯಾಯಿ-ಯ-ಗ-ಳಾಗಿದ್ದು
ಅನುಯಾಯಿ-ಯಾ-ಗಿದ್ದ
ಅನುಯಾಯಿ-ಯಾಗಿದ್ದನು
ಅನುಯಾಯಿ-ಯಾಗಿರ
ಅನುಯಾಯಿ-ಯಾಗಿ-ರ-ಬ-ಹುದು
ಅನುಯಾಯಿ-ಯಾದ
ಅನು-ಲಕ್ಷಿಸಿ
ಅನು-ವಂಶಿಕ-ವಾದ
ಅನು-ವಂಶೀಯ
ಅನು-ವರ-ದೊಳ್
ಅನು-ವಾದ
ಅನು-ವಾದ-ವಿದ್ದಂತಿದೆ
ಅನುಷ್ಠಾನ-ದಲ್ಲಿ
ಅನು-ಸಂಧಾನ
ಅನು-ಸರಿ-ಸ-ಬೇ-ಕಾದ
ಅನು-ಸರಿಸಿ
ಅನು-ಸರಿ-ಸಿ-ಕೊಂಡು
ಅನು-ಸರಿ-ಸಿದ
ಅನು-ಸರಿ-ಸಿ-ದಂತೆ
ಅನು-ಸರಿ-ಸಿ-ದನೇ
ಅನು-ಸರಿ-ಸಿ-ದ-ರೆಂದು
ಅನು-ಸರಿ-ಸಿ-ದ-ವರ
ಅನು-ಸರಿ-ಸಿದ್ದು
ಅನು-ಸರಿ-ಸಿ-ರ-ಬ-ಹುದು
ಅನು-ಸರಿ-ಸಿ-ರ-ಬಹು-ದೆಂದು
ಅನು-ಸರಿ-ಸು-ತಿದ್ದ
ಅನು-ಸರಿ-ಸುತ್ತಾ
ಅನು-ಸರಿ-ಸುತ್ತಿದ್ದ
ಅನು-ಸರಿ-ಸುತ್ತಿದ್ದರೂ
ಅನು-ಸರಿ-ಸುತ್ತಿದ್ದ-ರೆಂದು
ಅನು-ಸರಿ-ಸುತ್ತಿದ್ದ-ರೆಂಬುದು
ಅನು-ಸರಿ-ಸುತ್ತಿದ್ದ-ವರು
ಅನು-ಸರಿ-ಸುತ್ತಿದ್ದ-ವೆಂದು
ಅನು-ಸರಿ-ಸುತ್ತಿದ್ದು-ದಾಗಿ
ಅನು-ಸರಿ-ಸುವ
ಅನೆಕ
ಅನೇಕ
ಅನೇಕಅ
ಅನೇಕ-ಕಡೆ
ಅನೇಕ-ರಿಗೆ
ಅನೇ-ಕರು
ಅನೇಕ-ವರ್ಷ-ಗಳ
ಅನೇಕ-ವಿವೆ
ಅನೇಕ-ವೇಳೆ
ಅನೇ-ಕಾಂತ
ಅನ್ತೂ
ಅನ್ತೆನಿ-ಸಿರ್ದ್ದ
ಅನ್ನ
ಅನ್ನಂ
ಅನ್ನ-ಛತ್ರ
ಅನ್ನ-ದಾನ
ಅನ್ನ-ದಾನಕ್ಕೆ
ಅನ್ನ-ದಾನಕ್ಕೆಂದು
ಅನ್ನ-ದಾನ-ಪಲ್ಲಿ-ಯನ್ನು
ಅನ್ನ-ದಾನ-ಪಲ್ಲಿ-ಯಲ್ಲಿ
ಅನ್ನ-ದಾನ-ಪಳ್ಳಿ
ಅನ್ನ-ದಾನ-ಪಳ್ಳಿಯ
ಅನ್ನ-ದಾನ-ಪಳ್ಳಿ-ಯನ್ನು
ಅನ್ನ-ದಾನ-ಪಳ್ಳಿ-ಯಲ್ಲಿ
ಅನ್ನ-ದಾನ-ವಿನೋದ
ಅನ್ನ-ದಾನಿ-ಗೌಡ-ರು-ಗಳ
ಅನ್ನ-ಸತ್ರ
ಅನ್ನ-ಸತ್ರ-ವನ್ನು
ಅನ್ನ-ಸು-ವರ್ನ್ನ-ವುದಕಂ
ಅನ್ನು
ಅನ್ಯ
ಅನ್ಯಗ್ರಂಥ-ಗ-ಳಿಗೆ
ಅನ್ಯ-ಭಾಷೆ-ಯ-ಮಲೆ-ಯಾಳ
ಅನ್ಯ-ಮೂಲ-ಗ-ಳಿಂದ
ಅನ್ಯ-ರೂಪ-ಗಳು
ಅನ್ಯ-ಸಸ್ತ್ರಂ
ಅನ್ಯಾಯ
ಅನ್ಯಾಯ-ವೆಂಬ
ಅನ್ಯೋನ್ಯತೆ-ಯಿಂದ
ಅನ್ರಾಡು
ಅನ್ವಯ
ಅನ್ವ-ಯಕ್ಕೆ
ಅನ್ವಯದ
ಅನ್ವಯ-ವನ್ನು
ಅನ್ವಯ-ವಾಗ
ಅನ್ವಯ-ವಾಗುತ್ತ-ದೆಂದು
ಅನ್ವಯ-ವಾಗುತ್ತಿತ್ತು
ಅನ್ವಯ-ವಾಗು-ವು-ದಿಲ್ಲ
ಅನ್ವಯಾಗತ
ಅನ್ವಯಾಗತ-ವಾಗಿ
ಅನ್ವಯಾ-ಗದ
ಅನ್ವಯಾವ-ತಾರ-ವೆಂತೆಂದಡೆ
ಅನ್ವಯಿಸಿ
ಅನ್ವಯಿಸಿದ್ದಾರೆ
ಅನ್ವಯಿ-ಸುತ್ತದೆ
ಅನ್ವಯಿ-ಸುತ್ತ-ದೆಂದು
ಅನ್ವಯಿಸು-ವಂತೆ
ಅನ್ವೇಷಣೆ
ಅಪತಿಯ
ಅಪಭ್ರಂಶ
ಅಪಭ್ರಂಶ-ರೂಪ
ಅಪಭ್ರಂಶ-ವಾಗಿದೆ
ಅಪಭ್ರಂಶ-ವಾಗಿ-ರ-ಬ-ಹುದು
ಅಪಭ್ರಂಶ-ವಿರ-ಬ-ಹುದು
ಅಪಭ್ರಂಶ-ವಿರು-ವಂತೆ
ಅಪ-ಮಾನ-ಗಳು
ಅಪ-ರಾಜಿತ
ಅಪರಿಮಿತ-ದಾನ-ಸಾ-ರವೃಷ್ಟಿ
ಅಪ-ರೂಪ
ಅಪ-ರೂ-ಪಕ್ಕೆ
ಅಪ-ರೂಪದ
ಅಪ-ರೂಪ-ವಾಗಿ
ಅಪ-ರೂಪ-ವಾ-ಗಿದ್ದ
ಅಪ-ರೂಪ-ವಾದ
ಅಪರ್ಣ
ಅಪ-ವಾದಾತ್ಮಕ-ವಾಗಿ
ಅಪ-ಹರಣ
ಅಪ-ಹರ-ಣವೂ
ಅಪ-ಹರಿ-ಸಲು
ಅಪ-ಹರಿ-ಸಿ-ದ-ವನು
ಅಪಾಯ
ಅಪಾರ
ಅಪಾರಪ್ರ-ಮಾ-ಣದ
ಅಪಾರ-ವಾಗಿ
ಅಪಾರ-ವಾಗಿದೆ
ಅಪಾರ-ವಾದ
ಅಪಾರ-ಸೈನ್ಯ
ಅಪಿ
ಅಪೂರ್ಣ-ವಾಗಿ
ಅಪೂರ್ಣ-ವಾಗಿದೆ
ಅಪೂರ್ಬ್ಬಾಯ
ಅಪೂರ್ವ-ವಾದ
ಅಪೂರ್ವಾಯ
ಅಪೂರ್ವ್ವಾ-ಯ-ವೇನು
ಅಪೇಕ್ಷೆ-ಯಂತೆ
ಅಪ್ಪ
ಅಪ್ಪಣೆ
ಅಪ್ಪ-ಣೆಯ
ಅಪ್ಪ-ಣೆ-ಯಂತೆ
ಅಪ್ಪ-ಣೆ-ಯನ್ನು
ಅಪ್ಪಣೈ
ಅಪ್ಪಣ್ಣ
ಅಪ್ಪಣ್ಣ-ನಾಯ-ಕನು
ಅಪ್ಪಣ್ಣ-ಭೂಪ-ತಿಯು
ಅಪ್ಪಣ್ಣೈ
ಅಪ್ಪ-ದೇವ
ಅಪ್ಪ-ನಿಂದ
ಅಪ್ಪಯ್ಯ
ಅಪ್ಪಯ್ಯಂಗಾರಯ್ಯನ
ಅಪ್ಪಯ್ಯಂಗಾರರು
ಅಪ್ಪಯ್ಯಂಗಾರಿ-ಯ-ವರು
ಅಪ್ಪಯ್ಯನ
ಅಪ್ಪ-ಳಕ್ಕನ-ಹಳ್ಳ-ಗಳ
ಅಪ್ಪ-ಳಕ್ಕನ-ಹಳ್ಳಿ
ಅಪ್ಪಾಜಪ್ಪ-ಗಳ
ಅಪ್ಪಾಜಪ್ಪ-ಗಳು
ಅಪ್ಪಾ-ಜಯ್ಯ
ಅಪ್ಪಾ-ಜಯ್ಯನ
ಅಪ್ಪಾಜಿ
ಅಪ್ಪಾಜಿ-ಗಳ
ಅಪ್ಪಾಜಿ-ಪಂತ-ನಿಗೆ
ಅಪ್ಪಾಜಿ-ಪಂತಪಂತ-ನೆಂಬ
ಅಪ್ಪಿ-ಕೊಂಡರು
ಅಪ್ಪಿ-ಕೊಂಡು
ಅಪ್ಪಿದ
ಅಪ್ಪಿದ-ನೆಂದು
ಅಪ್ಪಿ-ದರು
ಅಪ್ಪಿ-ದು-ದನ್ನು
ಅಪ್ಪೆ-ನಾಯಕ
ಅಪ್ಪೆಯ
ಅಪ್ಪೆಯ-ನಾಯ-ಕನ
ಅಪ್ಪೇ-ಗೌಡರ
ಅಪ್ರತಿ-ಕ-ಮಲ್ಲ
ಅಪ್ರತಿಮ
ಅಪ್ರತಿ-ಮ-ತೇಜಂ
ಅಪ್ರತಿ-ಮ-ನಾ-ಗಿದ್ದು
ಅಪ್ರತಿ-ಮ-ವೀರ
ಅಪ್ರತಿ-ಮ-ವೀರ-ನರ-ಪತಿ
ಅಪ್ರಮೇಯ
ಅಪ್ರಮೇ-ಯನ
ಅಪ್ರಮೇಯ-ನನ್ನು
ಅಪ್ರಮೇ-ಯನು
ಅಪ್ಸರತ್ರೀ-ಯರು
ಅಫ್ತಬ್ಖಾನ್
ಅಬಲ-ವಾಡಿ
ಅಬಲ-ವಾಡಿ-ಯಲ್ಲಿ
ಅಬಲ-ವಾಡಿಯು
ಅಬ-ಸ-ಮುದ್ರ-ಅಹೋ-ಬಲ-ಸ-ಮುದ್ರ
ಅಬ್ದುಲ್
ಅಬ್ಬಗಂಜೂರು
ಅಬ್ಬ-ರಾಜ-ಗಳ
ಅಬ್ಬ-ರಾಜನ
ಅಬ್ಬ-ರಾಜು-ಗಳ
ಅಬ್ಬ-ರಾಜು-ವಿನ
ಅಬ್ಬಾಸ್ಗಾರ್ಡನ್
ಅಬ್ಬೂರಿ-ನಲ್ಲಿ
ಅಬ್ಬೂರು
ಅಬ್ರಾಹ್ಮಣ-ರನ್ನು
ಅಬ್ರಾಹ್ಮಣ-ರಿಗೆ
ಅಭಂಗ
ಅಭದ್ರತೆ
ಅಭಯ-ಹಸ್ತ-ಗಳನ್ನುಳ್ಳ
ಅಭ-ಯಾರಣ್ಯ-ವಾ-ಗಿದ್ದು
ಅಭಾಣೀನ್ಮೃದು-ಸಂದರ್ಭಂ
ಅಭಿ-ನಂದನಾ
ಅಭಿ-ನಂದಿ-ಸ-ಬೇಕು
ಅಭಿ-ನಂದಿಸಿ
ಅಭಿ-ನಂದಿ-ಸಿ-ದರು
ಅಭಿ-ನನ್ನೆಂದು
ಅಭಿನವ
ಅಭಿನವ-ಕುಲ-ಶೇಖರ-ರಾದ
ಅಭಿನವ-ಮದ-ನಾವ-ತಾರ
ಅಭಿನ್ನ-ರಾ-ಗಿದ್ದು
ಅಭಿನ್ನ-ರಿದ್ದು
ಅಭಿನ್ನ-ರಿರ-ಬ-ಹುದು
ಅಭಿನ್ನರು
ಅಭಿನ್ನ-ರೆಂದು
ಅಭಿನ್ನವೇ
ಅಭಿಪ್ರಾಯ
ಅಭಿಪ್ರಾ-ಯಕ್ಕೆ
ಅಭಿಪ್ರಾಯ-ಗ-ಳನ್ನು
ಅಭಿಪ್ರಾಯದ
ಅಭಿಪ್ರಾಯ-ಪಟಿದ್ದಾರೆ
ಅಭಿಪ್ರಾಯ-ಪಟ್ಟಿದ್ದಾರೆ
ಅಭಿಪ್ರಾಯ-ಪಟ್ಟಿ-ರುವುದು
ಅಭಿಪ್ರಾಯ-ಪಡ-ಲಾಗಿದೆ
ಅಭಿಪ್ರಾಯ-ಪಡುತ್ತಾರೆ
ಅಭಿಪ್ರಾಯ-ವನ್ನು
ಅಭಿಪ್ರಾಯ-ವನ್ನೇ
ಅಭಿಪ್ರಾಯ-ವಾಗಿದೆ
ಅಭಿಪ್ರಾಯ-ವಿದೆ
ಅಭಿಪ್ರಾಯವು
ಅಭಿಪ್ರಾಯವೂ
ಅಭಿ-ಮತ
ಅಭಿ-ಮತ-ದಂತೆ
ಅಭಿ-ಮನ್ಯುವು
ಅಭಿ-ಮಾನ
ಅಭಿ-ಮಾನ-ದಿಂದ
ಅಭಿ-ಮಾನಿ
ಅಭಿ-ಮಾನಿ-ಗ-ಳಾಗಿ-ರುವು-ದ-ರಿಂದಲೂ
ಅಭಿ-ಮಾನಿ-ಯಾಗಿದ್ದ-ನೆಂದು
ಅಭಿ-ಯೋಗ
ಅಭಿರುಚಿ
ಅಭಿ-ವೃದ್ಧಿ
ಅಭಿ-ವೃದ್ಧಿ-ಗಾಗಿ
ಅಭಿ-ವೃದ್ಧಿಗೆ
ಅಭಿ-ವೃದ್ಧಿ-ಪ-ಡಿಸಿ
ಅಭಿ-ವೃದ್ಧಿ-ಯಾಗ-ಬೇಕೆಂದು
ಅಭಿವೃದ್ಯರ್ಥ-ವಾಗಿ
ಅಭಿವ್ಯಕ್ತಿಗೆ
ಅಭಿಷಿಕ್ತ
ಅಭಿಷೇಕ
ಅಭಿಷೇ-ಕಕ್ಕೆ
ಅಭ್ಯಂತರ
ಅಭ್ಯಂತರ-ವಿಲ್ಲ
ಅಭ್ಯಾಗತೆ
ಅಭ್ಯು-ದಯ-ದಲ್ಲಿ
ಅಭ್ಯುದ-ಯಾರ್ಥ
ಅಮರ
ಅಮರಂಬೇಡು
ಅಮರಂಬೋದು
ಅಮರ-ಗಿರಿ-ತುಂಗ
ಅಮರ-ನಾಯ-ಕ-ತನಕ್ಕೆ
ಅಮರ-ನಾಯ-ಕ-ತನಕ್ಕೆ-ಸೇರಿತ್ತು
ಅಮರ-ನಾಯ-ಕ-ನಾಗಿ-ರುತ್ತಾನೆ
ಅಮರ-ನಾಯ-ಕರ
ಅಮರ-ನಾಯ-ಕ-ರಿಗೆ
ಅಮರ-ನಾಯ-ಕರು
ಅಮರ-ಪಡೆಯ
ಅಮರ-ಮಹಲೆ
ಅಮರ-ಮಾ-ಗಣಿ
ಅಮರ-ಮಾ-ಗಣಿಗೆ
ಅಮರ-ಮಾ-ಗಣಿ-ಯಾಗಿ
ಅಮರ-ಮಾ-ಗಣಿ-ಯಾ-ಗಿದ್ದ
ಅಮರ-ಮಾ-ಗಣೆಗೆ
ಅಮರ-ಮಾ-ಗಣೆ-ಯಾಗಿ
ಅಮರೇಂದ್ರ
ಅಮರೇಂದ್ರ-ಭ-ವನ-ವೆನಿಪ
ಅಮಲ-ಗುಣ
ಅಮಲ-ಶೀಲ-ವತಿ
ಅಮಲ್ದಾರ್
ಅಮಳ-ವೂರ್
ಅಮಾತ್ಯ
ಅಮಾತ್ಯನ
ಅಮಾತ್ಯ-ನಾ-ಗಿದ್ದ
ಅಮಾತ್ಯನೂ
ಅಮಾತ್ಯ-ಪದ
ಅಮಾತ್ಯರು
ಅಮಾನಿಸ
ಅಮಾವಾಸ್ಯೆಯ
ಅಮಿತ್ತ-ನೂರ
ಅಮಿತ್ತ-ನೂರಿನ
ಅಮಿಲ್
ಅಮಿಷೈ
ಅಮೀರ್
ಅಮೀರ್ಖಾನೆ
ಅಮೀಲ
ಅಮೀಲ್ದಾರ-ನನ್ನು
ಅಮುಕ್ತ
ಅಮುದ-ಸ-ಮುದ್ರ
ಅಮುಲ್ದಾರ್
ಅಮೃತ-ನಾಥ-ಪುರ-ವಾದ
ಅಮೃತ-ಪಡಿ
ಅಮೃತ-ಪಡಿ-ಗಾಗಿ
ಅಮೃತ-ಪಡಿಗೂ
ಅಮೃತ-ಪ-ಡಿಗೆ
ಅಮೃತ-ಪಡಿಯ
ಅಮೃತ-ಪಡಿ-ಯನೂ
ಅಮೃ-ತಪ್ಪನ
ಅಮೃತಬ್ಬೆಕಂತಿಗೆ
ಅಮೃ-ತಮ್ಮ-ನ-ವರ
ಅಮೃ-ತಾಂಬ
ಅಮೃ-ತಾಂಬಾ
ಅಮೃತಾ-ಪುರ
ಅಮೃತಿ
ಅಮೃ-ತಿಯ
ಅಮೃತೂರಿನ
ಅಮೃತೂರು
ಅಮೃತೇಶ್ವರ
ಅಮೃತೇಶ್ವರ-ದೇವರು
ಅಮೆರಿಕೆ-ಯಲ್ಲಿ-ರುವ
ಅಮೋಘ-ವರ್ಷ
ಅಮೋಘ-ವರ್ಷನ
ಅಮೋಘ-ವರ್ಷನು
ಅಮೋಘ-ವಾದ
ಅಮೋಘ-ವೃತ್ತಿ-ಯನ್ನು
ಅಮ್ಮ-ಗಳ್ಅಮ್ಮಾಳ್
ಅಮ್ಮಣ್ಯಮ್ಮ
ಅಮ್ಮನ
ಅಮ್ಮನ-ಪುರದ
ಅಮ್ಮನ-ವರ
ಅಮ್ಮನ-ವರ-ಗುಡಿ
ಅಮ್ಮನ-ವ-ರನ್ನು
ಅಮ್ಮನ-ವ-ರಿಗೂ
ಅಮ್ಮನ-ವ-ರಿಗೆ
ಅಮ್ಮನ-ವರು
ಅಮ್ಮನ-ವ-ರೆಂಬ
ಅಮ್ಮಮ್ಮ
ಅಮ್ರಿತ-ಪಡಿ
ಅಮ್ರಿತ-ಪ-ಡಿಗೆ
ಅಮ್ರಿತ-ರಾಶಿಯ
ಅಮ್ರುತ
ಅಮ್ರುತ-ಪಡಿ
ಅಮ್ರುತ-ಪ-ಡಿಗೆ
ಅಮ್ರುತಯ್ಯ
ಅಮ್ರುತುಪ
ಅಯಪ
ಅಯಿ-ರಮೆ-ನಾಯಕ
ಅಯಿ-ರಮೆ-ನಾಯ-ಕನು
ಅಯಿ-ವತ್ತಿಬ್ಬರು
ಅಯ್ಕಣಂ
ಅಯ್ಕ-ಣದ
ಅಯ್ಕ-ಣನ
ಅಯ್ಕ-ಣನನ್ನು
ಅಯ್ಕ-ಣನಿಗೆ
ಅಯ್ಕ-ಣನು
ಅಯ್ಕ-ಣನೆಂಬ
ಅಯ್ದಳವಿ
ಅಯ್ದು
ಅಯ್ದು-ವರಿ-ಸಕ್ಕೊಮ್ಮೆ
ಅಯ್ನೂರ್ವರ
ಅಯ್ನೂರ್ವ್ವರ-ದಂಡೆ
ಅಯ್ಯ
ಅಯ್ಯಂಗಾರರ
ಅಯ್ಯಂಗಾರ-ರಿಂದ
ಅಯ್ಯಂಗಾರ್
ಅಯ್ಯಂಗಾರ್ರಿಂದ
ಅಯ್ಯ-ಗೊಂಡ-ನ-ಪಲ್ಲಿ
ಅಯ್ಯ-ಗೊಂಡ-ನ-ಪಲ್ಲಿಯ
ಅಯ್ಯಣ
ಅಯ್ಯ-ದೇವ
ಅಯ್ಯನ
ಅಯ್ಯ-ನರ
ಅಯ್ಯ-ನ-ವರ
ಅಯ್ಯ-ನ-ವ-ರಿಗೆ
ಅಯ್ಯ-ನ-ವ-ರಿಗೆ-ಕೃಷ್ಣ-ದೇವ-ರಾಯ
ಅಯ್ಯ-ನ-ವರು
ಅಯ್ಯ-ನಿಗೆ
ಅಯ್ಯನು
ಅಯ್ಯಪ್ಪನ-ಹಳ್ಳಿ
ಅಯ್ಯ-ರ-ವೀರ
ಅಯ್ಯಾ
ಅಯ್ಯಾ-ದಕ್ಕ
ಅಯ್ಯಾ-ದಕ್ಕೆಯು
ಅಯ್ಯಾದ್ಯಕ್ಕನು
ಅಯ್ಯಾ-ಪೊಳಲ್
ಅಯ್ಯಾ-ಪೊೞಲ್ನ
ಅಯ್ಯಾ-ವ-ಳೆಯ
ಅಯ್ಯಾ-ವೊಳೆ
ಅಯ್ಯಾ-ವೊಳೆ-ಐ-ನೂರ್ವರು
ಅಯ್ಯಾ-ವೊಳೆಯ
ಅಯ್ಯಾ-ವೊಳೆ-ಯಲ್ಲಿ
ಅಯ್ಯಾ-ವೊಳೆ-ಯಲ್ಲಿದ್ದ
ಅಯ್ವರು
ಅರ-ಕನ-ಕೆರೆ
ಅರ-ಕಲಗೂಡು
ಅರ-ಕೆರ
ಅರ-ಕೆರೆ
ಅರ-ಕೆರೆಯ
ಅರ-ಕೆರೆ-ಯನ್ನು
ಅರ-ಕೆರೆ-ಯಲ್ಲಿ
ಅರ-ಕೆರೆ-ಯಲ್ಲಿದ್ದ
ಅರ-ಕೆರೆಯೂ
ಅರಕೆಲ್ಲ
ಅರ-ಕೇಸಿ
ಅರ-ಕೇಸಿಯ
ಅರ-ಕೇಸಿ-ಯ-ರಅ
ಅರ-ಕೇ-ಸಿಯು
ಅರ-ಣಿಯ
ಅರಣ್ಯ
ಅರಣ್ಯ-ವನ್ನು
ಅರ-ನ-ಕೆರೆಯ
ಅರಬ್ಬೀ
ಅರ-ಮನೆ
ಅರ-ಮ-ನೆಗೆ
ಅರ-ಮ-ನೆಯ
ಅರ-ಮನೆ-ಯಲ್ಲಿ
ಅರ-ಮನೆ-ಯಲ್ಲಿದ್ದ
ಅರ-ಮನೆ-ಯಲ್ಲಿನ
ಅರ-ಮನೆ-ಯಲ್ಲಿಯೂ
ಅರ-ಮನೆ-ಯಾನ್ತ
ಅರ-ಮನೆ-ಯಿಂದ
ಅರಮೆಯುಂ
ಅರಲು-ಕುಪ್ಪೆ
ಅರಳಿ
ಅರ-ಳಿ-ಕಟ್ಟೆ
ಅರ-ಳಿ-ಕಟ್ಟೆಯ
ಅರ-ಳಿ-ಮರ-ಗ-ಳಿಗೆ
ಅರ-ವತ್ತ-ನಾಲ್ಕು
ಅರ-ವತ್ತು
ಅರ-ವತ್ತು-ನಾಲ್ಕು
ಅರ-ವತ್ತೊಕ್ಕಲಿನ
ಅರ-ವನ-ಹಳ್ಳಿಯ
ಅರ-ವ-ಮನೆ-ಯಲ್ಲಿ
ಅರ-ವೀಟಿ
ಅರ-ವೀಟಿ-ರಂಗ-ರಾಜ
ಅರ-ವೀಡು
ಅರ-ವೆ-ಗಳ
ಅರಸ
ಅರ-ಸಂಕಸೂನೆ-ಗಾರ
ಅರ-ಸನ
ಅರ-ಸ-ನ-ಕೆರೆ
ಅರ-ಸ-ನ-ಕೆರೆಯ
ಅರ-ಸ-ನನ್ನಾಗಿ
ಅರ-ಸ-ನನ್ನು
ಅರ-ಸ-ನಲ್ಲಿ
ಅರ-ಸ-ನಾಗಿ
ಅರ-ಸ-ನಾ-ಗಿದ್ದ
ಅರ-ಸ-ನಾಗಿದ್ದಲ್ಲದೆ
ಅರ-ಸ-ನಾ-ಗಿದ್ದು
ಅರ-ಸ-ನಾಗಿ-ರ-ಬ-ಹುದು
ಅರ-ಸ-ನಾದ
ಅರ-ಸ-ನಿಂದ
ಅರ-ಸನು
ಅರ-ಸ-ನೆಂದು
ಅರ-ಸನೇ
ಅರ-ಸ-ನೊಬ್ಬ
ಅರ-ಸನೋ
ಅರ-ಸರ
ಅರ-ಸ-ರ-ಕಾಲ-ದ-ವರೆಗೂ
ಅರ-ಸ-ರನ್ನು
ಅರ-ಸ-ರಲ್ಲಿ
ಅರ-ಸ-ರಲ್ಲಿಯೇ
ಅರ-ಸ-ರಾದ
ಅರ-ಸ-ರಿಗೂ
ಅರ-ಸ-ರಿಗೆ
ಅರ-ಸರು
ಅರ-ಸ-ರು-ಗಳು
ಅರ-ಸ-ರು-ಮಣಲೇರ
ಅರ-ಸರೂ
ಅರ-ಸ-ರೆಂದು
ಅರ-ಸ-ರೆಲ್ಲರೂ
ಅರ-ಸರ್
ಅರ-ಸರ್ಮ್ಮಹಾ-ಸಾ-ಮನ್ತಾಧಿ-ಪತೀ
ಅರ-ಸಾ-ದಿತ್ಯ
ಅರಸಿ
ಅರ-ಸಿ-ಅಮ್ಮನ-ವರು
ಅರ-ಸಿ-ಕೆರ
ಅರ-ಸಿ-ಕೆರೆ
ಅರ-ಸಿ-ಕೆರೆಯ
ಅರ-ಸಿ-ಕೆರೆ-ಯಲ್ಲಿ
ಅರ-ಸಿಗ
ಅರ-ಸಿ-ದಂತಾ-ದುದು
ಅರ-ಸಿನ
ಅರ-ಸಿ-ನ-ಕೆರೆ
ಅರ-ಸಿ-ನ-ಕೆರೆಯ
ಅರ-ಸಿ-ನ-ಕೆರೆಯು
ಅರ-ಸಿ-ಯ-ಕರೆಯ
ಅರ-ಸಿ-ಯ-ಕೆರೆ
ಅರ-ಸಿ-ಯ-ಕೆರೆಯ
ಅರ-ಸಿ-ಯ-ಕೆರೆ-ಯನ್ನು
ಅರ-ಸೀ-ಕೆರೆ
ಅರ-ಸೀ-ಕೆರೆಯ
ಅರಸು
ಅರ-ಸು-ಗಂಡ
ಅರ-ಸು-ಗಳ
ಅರ-ಸು-ಗಳು
ಅರ-ಸು-ಮಕ್ಕಳು
ಅರ-ಸೆಟಿ-ಯ-ಹಳ್ಳಿಯ
ಅರ-ಸೊತ್ತಿಗೆಯ
ಅರ-ಸೊತ್ತಿಗೆ-ಯನ್ನು
ಅರಾಜ-ಕತೆ-ಯನ್ನು
ಅರಾ-ಮಿತ್ರ
ಅರಿ-ಎಂಬಂಬುಜ-ವನಮಂ
ಅರಿ-ಕನ-ಕಟ್ಟ
ಅರಿ-ಕನ-ಕಟ್ಟದ
ಅರಿ-ಕನ-ಕಟ್ಟ-ವನ್ನು
ಅರಿ-ಕನ-ಕಟ್ಟೆ
ಅರಿ-ಕುಂಟೆ
ಅರಿ-ಕುಠಾರ-ಪುರ-ದ-ವನು
ಅರಿ-ಕುಶ-ಕುಠಾರ
ಅರಿಕೆ
ಅರಿಗ-ಳೆನಿ-ಪದ-ಟ-ಮಲ್ಲರ
ಅರಿಗೋಧೂಮಘ-ರಟ್ಟ
ಅರಿತು-ಕೊಂಡು
ಅರಿ-ದಾನ್
ಅರಿ-ನೃಪರ
ಅರಿನ್ದಾ-ಅರ್ಜು-ನಾಮೆಂಟಿ-ರಾಜ-ಮನೇಕ
ಅರಿ-ಬಿರುದರ
ಅರಿ-ಬಿರುದ-ರ-ದಂಡ-ನಾಥ
ಅರಿ-ಯಪ್ಪನು
ಅರಿಯಮ್ಮ-ಸೆಟ್ಟಿ
ಅರಿ-ಯೂರಿನ
ಅರಿ-ರಾಯ-ದಟ್ಟ
ಅರಿ-ರಾಯ-ವಿಭಾಡ
ಅರಿ-ರೂಪ-ಸಿಂಗ
ಅರಿ-ವರ್ಮ
ಅರಿ-ವಾಗುತ್ತದೆ
ಅರಿ-ವಾಳ್ತಾಂಡ
ಅರಿ-ವಿಗೆ
ಅರಿ-ಸಿರ-ದಲು
ಅರಿ-ಸುಭಟರ-ಗಿರಿ
ಅರುಮುಳಿ
ಅರುಮುಳಿ-ದೇವ
ಅರುಮುಳಿ-ದೇವನ
ಅರುಮುಳಿ-ದೇವನು
ಅರು-ಮೋಳಿ-ದೇವ
ಅರುಳ-ಪಾಡೂ
ಅರುಳಾಳ-ನಾಥನ
ಅರುಳಿಯೋ-ಜಂಗೆ
ಅರುಳು-ಪಾಡು
ಅರುಳ್ನಾದ-ನಿಗೆ
ಅರುಳ್ಪಾಡು
ಅರುವ-ನ-ಹಳ್ಳಿ
ಅರುವ-ನ-ಹಳ್ಳಿಯ
ಅರುವ-ನ-ಹಳ್ಳಿ-ಯನ್ನು
ಅರುವ-ನ-ಹಳ್ಳಿ-ಯಲ್ಲಿ
ಅರುವ-ನ-ಹಳ್ಳಿ-ಯಲ್ಲಿದೆ
ಅರುವ-ನ-ಹಳ್ಳಿ-ಯಲ್ಲಿದ್ದು
ಅರುವ-ನ-ಹಳ್ಳಿಯೇ
ಅರುಹ-ನಳ್ಳಿ-ಯನ್ನು
ಅರು-ಹನ-ಹಳ್ಳಿ
ಅರು-ಹನ-ಹಳ್ಳಿಯ
ಅರು-ಹನ-ಹಳ್ಳಿ-ಯನ್ನು
ಅರು-ಹನ-ಹಳ್ಳಿ-ಯಲ್ಲಿ
ಅರು-ಹನ-ಹಳ್ಳಿ-ಯಲ್ಲಿ-ರುವ
ಅರು-ಹನ-ಹಳ್ಳಿ-ಯ-ವ-ರಿಗೂ
ಅರುಹಳಿ-ಅರು-ಹನ-ಹಳ್ಳಿ
ಅರೂಪ-ವಾಗಿತ್ತೆಂದು
ಅರೂಪ-ವಾ-ಗಿದ್ದ
ಅರೂಪ-ವಾಗಿ-ರ-ಲಾಗಿ
ಅರೆ
ಅರೆ-ಕೊಠಾರದ
ಅರೆ-ಕೊಠಾರ-ದಲ್ಲಿ
ಅರೆ-ತಿಪ್ಪೂರಿನ
ಅರೆ-ತಿಪ್ಪೂರು
ಅರೆ-ತಿಪ್ಪೂರೇ
ಅರೆ-ಬಂಡೆ-ಗಳು
ಅರೆ-ಬೊಪ್ಪ-ನ-ಹಳ್ಳಿ
ಅರೆ-ಬೊಪ್ಪ-ನ-ಹಳ್ಳಿಯ
ಅರೆಯ
ಅರೆ-ಯರ್
ಅರೆ-ಯ-ಹಳ್ಳಿ
ಅರೆ-ವಾಸಿ
ಅರೆ-ವಾಸಿ-ಯನ್ನು
ಅರೇ-ಬಿಕ್
ಅರ್ಕ-ಒಡೆಯ-ನಿಗೂ
ಅರ್ಕ-ಗುಪ್ತಿ-ಪುರ
ಅರ್ಕ-ಗುಪ್ತಿ-ಪುರ-ವಾದ
ಅರ್ಕ-ಗುಪ್ತಿ-ಪುರ-ವೆಂದು
ಅರ್ಕ-ನಾಥ
ಅರ್ಕಾವತಿ
ಅರ್ಕೆಶ್ವರ
ಅರ್ಕೇಶ್ವರ
ಅರ್ಕೇಶ್ವರನ
ಅರ್ಕೇಶ್ವರಸ್ವಾಮಿ
ಅರ್ಕ್ಕ-ನಾಥ
ಅರ್ಕ್ಕರ-ದುರ್ಕ್ಕೆ-ಯಿಂದ
ಅರ್ಕ್ಕೇಶ್ವರ
ಅರ್ಚಕ
ಅರ್ಚಕ-ನಾಗಿ-ರ-ಬ-ಹುದು
ಅರ್ಚ-ಕನೋ
ಅರ್ಚ-ಕರ
ಅರ್ಚಕ-ರಂಗಸ್ವಾಮಿ-ಯ-ವರು
ಅರ್ಚಕ-ರಾಗಿ
ಅರ್ಚಕ-ರಿ-ಗಾಗಿ
ಅರ್ಚಕ-ರಿಗೆ
ಅರ್ಚ-ಕರು
ಅರ್ಚನಾ
ಅರ್ಚನಾ-ವೃತ್ತಿ
ಅರ್ಚನಾ-ವೃತ್ತಿಗೆ
ಅರ್ಚನಾ-ವೃತ್ತಿ-ಯಾಗಿ
ಅರ್ಚ-ನೆಗೆ
ಅರ್ಚನೆ-ಯನ್ನು
ಅರ್ಜಿ
ಅರ್ಜಿ-ಗ-ಳನ್ನು
ಅರ್ಜುನ-ಹಳ್ಳಿ
ಅರ್ಜುನ್
ಅರ್ತಿ-ಗಳಿಗೂ
ಅರ್ಥ
ಅರ್ಥ-ಗಳೂ
ಅರ್ಥದ
ಅರ್ಥ-ದಲ್ಲಿ
ಅರ್ಥ-ಪೂರ್ಣ-ವಾಗಿ
ಅರ್ಥ-ವ-ನಿಕ್ಕಿ
ಅರ್ಥ-ವನ್ನು
ಅರ್ಥ-ವಲ್ಲ
ಅರ್ಥ-ವಾ-ಗದೆ
ಅರ್ಥ-ವಾಗುತ್ತದೆ
ಅರ್ಥ-ವಾಗುವ
ಅರ್ಥ-ವಾಗು-ವು-ದಿಲ್ಲ
ಅರ್ಥ-ವಿದೆ
ಅರ್ಥ-ವಿದೆಯೇ
ಅರ್ಥ-ವಿವೇಚ-ನೆ-ಯನ್ನು
ಅರ್ಥವು
ಅರ್ಥವೂ
ಅರ್ಥ-ವೆಂದು
ಅರ್ಥ-ವೆಂದೂ
ಅರ್ಥವೇ
ಅರ್ಥಸ್ಪಷ್ಟತೆ
ಅರ್ಥೈಸ-ಬ-ಹುದು
ಅರ್ಥೈಸ-ಬಹುದೇ
ಅರ್ಥೈಸ-ಲಾಗಿದ್ದರೂ
ಅರ್ಥೈಸಿ
ಅರ್ಥೈಸಿದ್ದಾರೆ
ಅರ್ಥೈಸಿದ್ದಾ-ರೆಂದು
ಅರ್ಥೈ-ಸು-ವುದು
ಅರ್ದ್ಧಾಂಗ
ಅರ್ಧ
ಅರ್ಧಕ್ಕೇ
ಅರ್ಧಗ್ರಾ-ವನ್ನು
ಅರ್ಧ-ಜಾ-ಮದ
ಅರ್ಧ-ಫಲ
ಅರ್ಧ-ಭಾಗ
ಅರ್ಧ-ಭಾಗದ
ಅರ್ಧ-ಭಾಗ-ವನ್ನು
ಅರ್ಧ-ಭಾಗ-ವನ್ನೂ
ಅರ್ಧ-ಭಾಗ-ವನ್ನೇ
ಅರ್ಧ-ಮಂಟಪ
ಅರ್ಧ-ಯೋಜನ
ಅರ್ಧ-ವನ್ನು
ಅರ್ಧ-ವೃತ್ತಿ-ಯನ್ನು
ಅರ್ಧಾಂಗ
ಅರ್ಪಿಸ-ಲಾಗಿದೆ
ಅರ್ಪಿಸಲಾಯಿ-ತೆಂದು
ಅರ್ಪಿಸಲಾಯಿ-ತೆಂದೂ
ಅರ್ಪಿ-ಸಲು
ಅರ್ಪಿ-ಸಿದ-ನೆಂದು
ಅರ್ಪಿಸಿದ್ದಾನೆ
ಅರ್ಪಿಸುತ್ತಾನೆ
ಅರ್ಪಿ-ಸುವ
ಅರ್ಪೊ-ಳೆಯ
ಅರ್ಪ್ಪೊಳೆ
ಅರ್ಯಾ-ಬಿಕ್
ಅರ್ವಾಚೀನ
ಅರ್ಹಂತರ
ಅರ್ಹ-ಗೇಹ
ಅರ್ಹ-ಗೇಹ-ಗ-ಳನ್ನು
ಅರ್ಹ-ಗೇಹ-ಗಳೆಂದರೆ
ಅರ್ಹ-ಗೇಹ-ವನ್ನು
ಅರ್ಹ-ಗೇಹವು
ಅರ್ಹ-ತೆಯ
ಅರ್ಹ-ತೆಯೂ
ಅರ್ಹ-ನಂದಿ
ಅರ್ಹ-ನಲ್ಲ-ವೆಂದು
ಅರ್ಹ-ಪೂಜೆಗೆ
ಅರ್ಹ-ರನ್ನು
ಅರ್ಹ-ರೆಂದು
ಅರ್ಹ-ವಾಗಿ-ದೆಯೇ
ಅರ್ಹ-ವಾಗಿ-ರು-ವಂತ-ಹದು
ಅರ್ಹ-ವಾಗಿವೆ
ಅರ್ಹ-ವಾದ
ಅಱ-ಸಿದ-ಡೆಉ
ಅಲಂಕರಿ-ಸಿದ
ಅಲಂಕರಿ-ಸಿದ-ನೆಂದು
ಅಲಂಕರಿ-ಸಿದ್ದ
ಅಲಂಕರಿ-ಸಿದ್ದನು
ಅಲಂಕರಿ-ಸಿದ್ದರು
ಅಲಂಕರಿ-ಸಿರು-ವು-ದನ್ನು
ಅಲಂಕರಿ-ಸುತ್ತಾನೆ
ಅಲಂಕರಿ-ಸುತ್ತಿದ್ದರು
ಅಲಂಕಾರ
ಅಲಂಕಾರಪ್ರಾಯ-ವಾಗಿ
ಅಲಂಕಾರ-ಶಾಸ್ತ್ರ
ಅಲಂಕಾ-ರಾರ್ಥ-ವಾಗಿ
ಅಲಂಕಾರಿ-ಕ-ವಾದ
ಅಲಂಕೃತ-ವಾಗಿದೆ
ಅಲಂಘ್ಯ
ಅಲ-ಗನ್ನು
ಅಲ-ನರ-ಸಿಂಹನ್
ಅಲ-ಸಿಂಗ-ರಾರ್ಯಸ್ಯ
ಅಲಾ
ಅಲಿ
ಅಲಿಂದಲು
ಅಲಿ-ಖಾನ್
ಅಲಿಯ
ಅಲಿ-ಯ-ವರು
ಅಲಿಯು
ಅಲಿ-ಹೈದ-ರನು
ಅಲೀ
ಅಲು-ಗನ್ನು
ಅಲುಗಾ-ಡದ
ಅಲುಗು-ಗಳ
ಅಲುಗು-ಗ-ಳನ್ನು
ಅಲ್
ಅಲ್ದೀನ್
ಅಲ್ಪ
ಅಲ್ಪ-ಕಾಲ
ಅಲ್ಪ-ಕಾಲ-ದ-ವರೆಗೆ
ಅಲ್ಪ-ಮತಿಯ
ಅಲ್ಪಳ್ಳಿಯೂ
ಅಲ್ಪ-ಸಂಖ್ಯೆ-ಯಲ್ಲಾ-ದರೂ
ಅಲ್ಪಸ್ವಲ್ಪ
ಅಲ್ಲ
ಅಲ್ಲ-ಗಳೆ-ದಿದ್ದು
ಅಲ್ಲದೆ
ಅಲ್ಲದೇ
ಅಲ್ಲಪ್ಪ
ಅಲ್ಲಪ್ಪ-ದಂಡ-ನಾಯ-ಕನು
ಅಲ್ಲಪ್ಪನ
ಅಲ್ಲಪ್ಪ-ನ-ಹಳ್ಳಿ
ಅಲ್ಲಪ್ಪ-ನಾಯ-ಕನು
ಅಲ್ಲಪ್ಪನು
ಅಲ್ಲ-ಮಪ್ರಭುವು
ಅಲ್ಲಲ್ಲಿ
ಅಲ್ಲಲ್ಲಿಯೇ
ಅಲ್ಲಲ್ಲೇ
ಅಲ್ಲ-ವೆಂದು
ಅಲ್ಲಾಂಬಾ
ಅಲ್ಲಾಳ
ಅಲ್ಲಾಳ-ದೇವ
ಅಲ್ಲಾಳ-ದೇವಂಗೆ
ಅಲ್ಲಾಳ-ದೇವನ
ಅಲ್ಲಾಳ-ನಾಥ
ಅಲ್ಲಾಳ-ನಾಥ-ದೇವರ
ಅಲ್ಲಾಳ-ನಾಥ-ದೇವ-ರಿಗೆ
ಅಲ್ಲಾಳ-ನಾಥನ
ಅಲ್ಲಾಳ-ನಾಥ-ವರ-ದ-ರಾಜ
ಅಲ್ಲಾಳ-ಪೆರು-ಮಾಳ
ಅಲ್ಲಾಳ-ಪೆರು-ಮಾಳ-ದೇವ-ರಿಗೆ-ವರ-ದ-ರಾಜ
ಅಲ್ಲಾಳ-ಪೆರು-ಮಾಳೆ
ಅಲ್ಲಾಳ-ಪೆರು-ಮಾಳ್
ಅಲ್ಲಾಳ-ಸ-ಮುದ್ರ
ಅಲ್ಲಾಳ-ಸ-ಮುದ್ರ-ವೆಂಬ
ಅಲ್ಲಾಳಿ
ಅಲ್ಲಾಳಿಯು
ಅಲ್ಲಾವುದ್ದೀನ್
ಅಲ್ಲಿ
ಅಲ್ಲಿಂ
ಅಲ್ಲಿಂದ
ಅಲ್ಲಿಂದಲೇ
ಅಲ್ಲಿಗೆ
ಅಲ್ಲಿ-ಡಲಾಯಿ-ತೆಂದು
ಅಲ್ಲಿದ್ದ
ಅಲ್ಲಿದ್ದ-ವರು
ಅಲ್ಲಿದ್ದು
ಅಲ್ಲಿನ
ಅಲ್ಲಿಯ
ಅಲ್ಲಿ-ಯ-ವರೆಗೆ
ಅಲ್ಲಿಯೂ
ಅಲ್ಲಿಯೇ
ಅಲ್ಲಿ-ರುವ
ಅಲ್ಲೂ
ಅಲ್ಲೇ
ಅಲ್ಲೋಲ-ಕಲ್ಲೋಲ
ಅಲ್ಲ್ಲೇ
ಅಳಂದ್ನವ-ರಾದ
ಅಳ-ಗಿಯ
ಅಳ-ಗಿಯ-ಮಣ-ವಾಳ-ರಿಗೆ
ಅಳ-ಗಿಹಿಯ
ಅಳಗುವಂಣ-ನ-ವರು
ಅಳಗುವಂಣನು
ಅಳಘಿಯ
ಅಳತೆ
ಅಳತೆ-ಗ-ಳನ್ನು
ಅಳತೆ-ಗಳು
ಅಳ-ತೆಗೆ
ಅಳ-ತೆಯ
ಅಳತೆ-ಯನ್ನು
ಅಳತೆ-ಯನ್ನೂ
ಅಳತೆ-ಯಲ್ಲಿ
ಅಳತೆ-ಯಾ-ಗಿದ್ದು
ಅಳ-ತೆಯು
ಅಳವ
ಅಳವ-ಡಿಸಿ
ಅಳವ-ಡಿಸಿ-ಕೊಂಡರೂ
ಅಳವ-ಡಿಸಿ-ಕೊಂಡು
ಅಳವ-ಡಿಸಿದೆ
ಅಳವ-ಡಿಸಿದ್ದೇನೆ
ಅಳವ-ಡಿಸಿ-ರುವ
ಅಳವ-ಡಿಸು-ವಂತೆ
ಅಳವಿ
ಅಳ-ಶಿಂಗಯ್ಯನ
ಅಳ-ಶಿಂಗರಯ್ಯಂಗಾರರು
ಅಳಹ
ಅಳಹಿಯ
ಅಳಹಿಯ-ಅಳಘಿಯ
ಅಳಹಿಯ-ಮಣ-ವಾಳ
ಅಳಹು
ಅಳಿ
ಅಳಿಉ
ಅಳಿದ
ಅಳಿ-ದಲ್ಲಿ
ಅಳಿ-ದ-ವನ್ನು
ಅಳಿ-ದ-ವರು
ಅಳಿ-ದಿಹ-ನೆಂದು
ಅಳಿ-ಪಿ-ದರೆ
ಅಳಿ-ಪಿ-ದ-ವರು
ಅಳಿಪು
ಅಳಿಬ
ಅಳಿ-ಬನ
ಅಳಿ-ಬನು
ಅಳಿ-ಬಳಿ
ಅಳಿಯ
ಅಳಿ-ಯಂದಿ-ರಾದ
ಅಳಿ-ಯನ
ಅಳಿ-ಯ-ನಾ-ಗಿದ್ದು
ಅಳಿ-ಯ-ನಾಗಿ-ರ-ಬ-ಹುದು
ಅಳಿ-ಯ-ನಾದ
ಅಳಿ-ಯ-ನಿಗೆ
ಅಳಿ-ಯನೂ
ಅಳಿ-ಯ-ನೆಂದು
ಅಳಿ-ಯನೇ
ಅಳಿ-ಯ-ರಾಮ-ರಾಯ
ಅಳಿ-ಯ-ರಾಮ-ರಾಯನ
ಅಳಿ-ಯ-ರಾಮ-ರಾಯನೂ
ಅಳಿ-ವಂನ್ಯಾಯ
ಅಳಿ-ವನ್ಯಾ-ಯ-ವೇನು
ಅಳಿವು
ಅಳಿ-ವು-ಅನ್ಯಾಯ-ಗೂಡಿ
ಅಳಿ-ಸಂದ್ರ
ಅಳಿ-ಸಂದ್ರದ
ಅಳಿ-ಸಂದ್ರ-ಶಾ-ಸನ-ದಲ್ಲಿ
ಅಳಿಸಿ
ಅಳಿ-ಸಿದ
ಅಳಿ-ಸಿದೆ
ಅಳಿ-ಸಿ-ಹೋ-ಗಲು
ಅಳಿ-ಸಿ-ಹೋಗಿ
ಅಳಿ-ಸಿ-ಹೋಗಿದೆ
ಅಳಿ-ಸಿ-ಹೋಗಿ-ರುವ
ಅಳಿ-ಸಿ-ಹೋಗಿವೆ
ಅಳೀ-ಸಂದ್ರ
ಅಳೀ-ಸಂದ್ರದ
ಅಳುಕು
ಅಳುಡೈಯಾನ್
ಅಳುಡೈ-ಯಾರ್
ಅಳುಡೈ-ಯಾರ್ಕುಮ್
ಅಳುಹು
ಅಳೆಗ-ಶಿಂಗರಯ್ಯಂಗಾರ-ರಿಂದ
ಅಳೆಗ-ಶಿಂಗರಯ್ಯಂಗಾರರು
ಅಳೆಯ-ಲಾಗಿದೆ
ಅಳೆ-ಯಲು
ಅಳೆಯುತ್ತಿದ್ದುದು
ಅವಕಾಶ
ಅವಕಾಶ-ವನ್ನು
ಅವಕಾಶ-ವಾಯಿ-ತೆಂದು
ಅವಕಾಶ-ವಿದೆ
ಅವಕಾಶ-ವಿದ್ದರೂ
ಅವಚ
ಅವ-ತಾರ
ಅವ-ತಾರ-ಗಳ
ಅವ-ತಾರದ
ಅವ-ತಾರ-ವೆಂದು
ಅವ-ತಾರ-ವೆಂಬು-ದನ್ನು
ಅವಧಿ
ಅವಧಿಗೆ
ಅವಧಿಯ
ಅವಧಿ-ಯಲ್ಲಿ
ಅವನ
ಅವನ-ತಿ-ಯತ್ತ
ಅವ-ನದ್ದೇ
ಅವ-ನನ್ನು
ಅವ-ನಾದ
ಅವ-ನಿಂದ
ಅವ-ನಿಂದಲೇ
ಅವ-ನಿ-ಗಾಗಿ
ಅವ-ನಿಗೂ
ಅವ-ನಿಗೆ
ಅವನಿ-ಪನೆನಗಿತ್ತಪ-ನೆಂದ-ವರಿ-ವರ-ವೊಲು-ಳಿದ
ಅವನು
ಅವನೇ
ಅವ-ನೊಬ್ಬ
ಅವನ್ನು
ಅವ-ಮಾನಿ-ಸುವ
ಅವರ
ಅವರಅ
ಅವ-ರದ್ದು
ಅವ-ರದ್ದೇ
ಅವ-ರನ್ನಿಕ್ಕಿಸಿ
ಅವ-ರನ್ನು
ಅವ-ರಲ್ಲಿ
ಅವ-ರಲ್ಲಿಗೆ
ಅವರ-ವರ
ಅವ-ರಿಂದ
ಅವ-ರಿಂದಲೇ
ಅವ-ರಿ-ಗಾಗಿ
ಅವ-ರಿ-ಗಾಗಿಯೇ
ಅವ-ರಿ-ಗಿಂತ
ಅವ-ರಿ-ಗಿದ್ದ
ಅವ-ರಿಗೆ
ಅವರಿಬ್ಬ-ರಿಗೂ
ಅವ-ರಿಬ್ಬರೂ
ಅವರಿವ-ರಂತೆ
ಅವರೀರ್ವ-ರಲ್ಲಿ
ಅವರು
ಅವರು-ಗಳ
ಅವರು-ಗಳು
ಅವರು-ಹೇಳಿದ್ದಾರೆ
ಅವರೂ
ಅವರೆ-ಗೆ-ರೆಯ
ಅವ-ರೆಲ್ಲರ
ಅವ-ರೆಲ್ಲ-ರನ್ನೂ
ಅವ-ರೆಲ್ಲರೂ
ಅವರೇ
ಅವ-ರೊಡನೆ
ಅವ-ರೊಳಗೆ
ಅವರೊಳಯ್ಗಣ್ಡುಗ-ಮಣ್ನಮ್ಮ
ಅವ-ಲಂಬಿಸಿ
ಅವ-ಲಂಬಿ-ಸಿತ್ತು
ಅವ-ಲಂಬಿಸಿದೆ
ಅವ-ಲಂಬಿಸಿದ್ದಿತು
ಅವ-ಲಂಬಿಸುತ್ತಿದ್ದರು
ಅವ-ಲೋಕನ
ಅವ-ಲೋಕಿ-ಸಿದಾಗ
ಅವಳ
ಅವ-ಳನ್ನು
ಅವಳಿ
ಅವ-ಳಿಗೆ
ಅವಳು
ಅವಳೇ
ಅವ-ಶೇಷ
ಅವ-ಶೇಷ-ಗ-ಳಿಂದ
ಅವ-ಶೇಷ-ಗಳಿವೆ
ಅವ-ಶೇಷ-ಗಳು
ಅವ-ಶೇಷ-ಗಳೂ
ಅವ-ಶೇಷ-ಗ-ಳೆಂದು
ಅವ-ಶೇಷ-ವಿದೆ
ಅವಶ್ಯಕ-ವಾದ
ಅವ-ಸರ
ಅವಸ-ರದ
ಅವ-ಸರ-ವನ್ನು
ಅವ-ಸರ-ವನ್ನು-ಸೇವೆ-ಯನ್ನು
ಅವಸಾನ-ಕಾಲ
ಅವ-ಸಾ-ನದ
ಅವಾಂತರ-ದಲ್ಲಿ
ಅವಾರ್ಯ-ವೀರ್ಯ-ನಾದ
ಅವಿದ್ಯೆ-ಯಿಂದ
ಅವಿನಾ-ಭಾವ
ಅವಿನೀ-ತನು
ಅವಿರ್ಭವಿಸಿ
ಅವು
ಅವು-ಗಳ
ಅವು-ಗ-ಳನ್ನು
ಅವು-ಗ-ಳಲ್ಲಿ
ಅವು-ಗ-ಳಲ್ಲೂ
ಅವು-ಗ-ಳಿಂದ
ಅವು-ಗ-ಳಿಗೆ
ಅವು-ಗಳು
ಅವು-ಬಳ-ರಾಜಯ್ಯ-ದೇವ
ಅವೆಲ್ಲಾ
ಅವೈದಿಕ
ಅವೈದಿಕ-ವಾಗಿದ್ದರೂ
ಅವ್ವಗಾ-ರಿಗೆ
ಅವ್ವಾ-ಜಮ್ಮ
ಅವ್ವೆ
ಅವ್ವೆಯ
ಅವ್ವೆ-ಯರ
ಅವ್ವೆ-ಯ-ರ-ಕೆರೆ
ಅವ್ವೆ-ಯ-ರಾಣೆ
ಅವ್ವೇರ-ಹಳ್ಳಿ
ಅವ್ವೇರ-ಹಳ್ಳಿಯ
ಅವ್ವೇರ-ಹಳ್ಳಿ-ಯನ್ನು
ಅಶನ
ಅಶರೀರ-ವಾಣಿ-ಯಾ-ಯಿತು
ಅಶುದ್ಧ-ವಾ-ದರೂ
ಅಶೇಷ
ಅಶೇಷ-ಮಹಾ-ಜನ-ಗಳು
ಅಶೇಷ-ಮಹಾ-ಜ-ನರು
ಅಶೇಷ-ರಾಜ್ಯ-ಭಾರ
ಅಶೋ-ಕನ
ಅಶ್ವ-ಪತಿ
ಅಶ್ವ-ಸೇನೆಯು
ಅಶ್ವಾರೋಹಿ
ಅಷವಂ
ಅಷ್ಟಗ್ರಾಮ
ಅಷ್ಟಗ್ರಾಮ-ಗಳ
ಅಷ್ಟಗ್ರಾಮದ
ಅಷ್ಟ-ದಿಕ್ಕು-ರಾಯ
ಅಷ್ಟದಿಕ್ಪಾಲ-ಕರ
ಅಷ್ಟ-ಭೋಗ
ಅಷ್ಟ-ಭೋಗ-ತೇಜಸ್ವಾಮ್ಯ
ಅಷ್ಟರ
ಅಷ್ಟ-ರಲ್ಲಿ
ಅಷ್ಟ-ವಿಧಾರ್ಚನೆಗಮೊಂದು
ಅಷ್ಟ-ವಿಧಾರ್ಚ-ನೆಗೆ
ಅಷ್ಟ-ವಿಧಾರ್ಚನೆ-ಯಲ್ಲಿ
ಅಷ್ಟಾಗಿ
ಅಷ್ಟಾ-ದರೂ
ಅಷ್ಟಾದಶಪ್ರಧಾನ-ರಲ್ಲಿ
ಅಷ್ಟು
ಅಷ್ಟು-ದೂರದ
ಅಷ್ಟೇ
ಅಷ್ಟೋಪ-ವಾಸಿ
ಅಸಂಖ್ಯಾತ
ಅಸಗ-ನೊಳಗಾದ
ಅಸ-ಗರ
ಅಸನ್ನ
ಅಸಮ
ಅಸಮ-ಭಟ್ಟನು
ಅಸಮ-ಭಟ್ಟ-ರಿಗೆ
ಅಸಮಾಧಾನ
ಅಸರಿಸ್ವಯಂಭು
ಅಸರು
ಅಸವಯ್ಯನು
ಅಸವಯ್ಯನುಂ
ಅಸಾ-ಮಾನ್ಯವೂ
ಅಸಿ-ವರ-ದಲಿ
ಅಸಿ-ವರ-ದಲ್ಲಿ
ಅಸೀನ-ವಾಗಿದೆ
ಅಸುಂಗೊಳ್ವನ್ನೆಗಂ
ಅಸುನೀಗಿ-ದಂತೆ
ಅಸೂಯೆ
ಅಸೆಂಬ್ಲಿ-ಎಂದೂ
ಅಸೇಷ
ಅಸೋಫನಿದ್ದನು
ಅಸೋಫಿ-ಗ-ಳಾಗಿ
ಅಸೋಫಿಗೆ
ಅಸೋಫ್
ಅಸ್ಕರ್
ಅಸ್ತಂಗ-ರಾದ
ಅಸ್ತ-ಮಾನ-ರಾದಾಗ
ಅಸ್ತ-ಮಾನ-ವಾದಾಗ
ಅಸ್ತಿತ್ವ
ಅಸ್ತಿತ್ವಕ್ಕೆ
ಅಸ್ತಿತ್ವ-ದಲ್ಲಿತ್ತು
ಅಸ್ತಿತ್ವ-ದಲ್ಲಿತ್ತೆಂದು
ಅಸ್ತಿತ್ವ-ದಲ್ಲಿತ್ತೆಂಬುದು
ಅಸ್ತಿತ್ವ-ದಲ್ಲಿದ್ದ
ಅಸ್ತಿತ್ವ-ದಲ್ಲಿದ್ದವು
ಅಸ್ತಿತ್ವ-ದಲ್ಲಿದ್ದು
ಅಸ್ತಿತ್ವ-ದಲ್ಲಿದ್ದುದು
ಅಸ್ತಿತ್ವ-ದಲ್ಲಿ-ರ-ಲಿಲ್ಲ-ವೆಂದು
ಅಸ್ತಿತ್ವ-ದಲ್ಲಿ-ರುವ
ಅಸ್ತಿತ್ವ-ದಲ್ಲಿವೆ
ಅಸ್ತಿತ್ವಲ್ಲಿದ್ದ
ಅಸ್ತಿತ್ವ-ವನ್ನು
ಅಸ್ತಿ-ಯನ್ನು
ಅಸ್ಥಿರ-ತೆ-ಯನ್ನು
ಅಸ್ಪಷ್ಟ
ಅಸ್ಪಷ್ಟ-ವಾಗಿದೆ
ಅಸ್ಪಷ್ಟ-ವಾಗಿ-ರುವ
ಅಸ್ಪಷ್ಟ-ವಾಗಿವೆ
ಅಸ್ಯ
ಅಹು-ಬಳ
ಅಹು-ಬಳ-ದೇವ-ರಾಜಯ್ಯ-ದೇವ
ಅಹು-ಬಳ-ರಾಜಯ್ಯ-ನಿಗೆ
ಅಹೇರಿನ-ಸುಂಕದ
ಅಹೊ-ಬಲ-ದೇವ-ರಾಜಯ್ಯನು
ಅಹೋ-ಬಲ
ಅಹೋ-ಬಲ-ದೇವ
ಅಹೋ-ಬಲ-ದೇವ-ಗಳ
ಅಹೋ-ಬಲ-ದೇವನ
ಅಹೋ-ಬಲ-ದೇವ-ರಾ-ಜಯ್ಯ
ಅಹೋಬ-ಲಯ್ಯ
ಅಹೋ-ಬಲಯ್ಯನ
ಅಹೋ-ಬಲ-ವಾಡಿ
ಅಹೋ-ಬಲವು
ಅಹೋ-ಬಳ
ಅಹೋ-ಬಳ-ಪುರ-ವೆಂಬ
ಅಹೋ-ಬಳ-ರಾಜನ
ಆ
ಆಂಗೀ-ರಸ
ಆಂಗ್ಲ
ಆಂಗ್ಲ-ಭಾಷೆ
ಆಂಗ್ಲ-ಭಾಷೆ-ಯಲ್ಲಿ
ಆಂಗ್ಲೋ
ಆಂಜನೇಯ
ಆಂಜನೇ-ಯ-ಚಲುವ-ನಾ-ರಾಯಣ
ಆಂಜನೇ-ಯನ
ಆಂಡಯ್ಯ-ನೆಂಬ
ಆಂಡಾನು
ಆಂಡಾನ್
ಆಂಡಾಳ್
ಆಂಡೈ-ಯಾಳ್
ಆಂದೋಲನ
ಆಂಧ್ರ
ಆಂಧ್ರದ
ಆಂಧ್ರ-ನಾಡಿ-ನಲ್ಲಿ
ಆಂಧ್ರಪ್ರ-ದೇಶದ
ಆಂಧ್ರ-ರಾಜ-ಮದ-ಗಜ-ಗ-ಳಿಗೆ
ಆಂಶಕ್ಕೆ
ಆಕರ-ಗಳು
ಆಕರ್ಷಕ
ಆಕರ್ಷಿಸು-ವುದೇ
ಆಕಲ್ಪನ್ಮಸ್ನುತೇತ್ರೈಕಾಂ
ಆಕಲ್ಯಾಂತಂ
ಆಕಸ್ಮಿಕ-ವಾಗಿ
ಆಕಾರ
ಆಕಾರಣ
ಆಕಾರ-ದಲ್ಲಿದ್ದು
ಆಕಿ
ಆಕೆಯ
ಆಕೆಯು
ಆಕ್ಟ್ರಾಯ್
ಆಕ್ರಮಣ
ಆಕ್ರಮ-ಣಕ್ಕೆ
ಆಕ್ರಮ-ಣ-ಗ-ಳನ್ನು
ಆಕ್ರಮ-ಣದ
ಆಕ್ರಮ-ಣ-ದಲ್ಲಿ
ಆಕ್ರಮ-ಣ-ನಡೆ-ಸಲು
ಆಕ್ರಮ-ಣ-ವನ್ನು
ಆಕ್ರಮ-ಣ-ವಾಗಿ-ರ-ಬ-ಹುದು
ಆಕ್ರಮಿತ-ವಾಗಿ
ಆಕ್ರಮಿಸಿ
ಆಕ್ರಮಿಸಿ-ಕೊಂಡನು
ಆಕ್ರಮಿಸಿ-ಕೊಂಡರು
ಆಕ್ರಮಿಸಿ-ಕೊಂಡು
ಆಕ್ರಮಿ-ಸಿತು
ಆಕ್ರಮಿ-ಸಿದ
ಆಕ್ರಮಿಸಿ-ದನು
ಆಕ್ರಮಿಸಿ-ದ-ಮೇಲೂ
ಆಕ್ರಮಿ-ಸಿದ್ದು
ಆಗ
ಆಗಂತುಕ
ಆಗ-ತಾನೆ
ಆಗದೇ
ಆಗ-ಬಹು-ದೇನೋ
ಆಗ-ಬೇಕು
ಆಗ-ಬೇಕೆಂದು
ಆಗ-ಮ-ಗಳ
ಆಗ-ಮನ
ಆಗ-ಮನಕ್ಕೆ
ಆಗ-ಮನ-ದಿಂದ
ಆಗಮಿಕ
ಆಗಮಿಸಿ
ಆಗಮಿಸಿ-ದ-ರೆಂದೂ
ಆಗಲಿ
ಆಗಲು
ಆಗಲೇ
ಆಗವ-ಹಾಳ
ಆಗಸ್ಟ್
ಆಗಸ್ಟ್ನಿಂದ
ಆಗಾಗ್ಗೆ
ಆಗಾಮಿ
ಆಗಾಮಿಕೆ
ಆಗಿ
ಆಗಿಂದಾಗ್ಗೆ
ಆಗಿತ್ತು
ಆಗಿತ್ತೆಂದು
ಆಗಿದೆ
ಆಗಿ-ದೆಯೇ
ಆಗಿದ್ದ
ಆಗಿದ್ದಂತೆ
ಆಗಿದ್ದನು
ಆಗಿದ್ದ-ನೆಂದು
ಆಗಿದ್ದನ್ನು
ಆಗಿದ್ದ-ರಿಂದ
ಆಗಿದ್ದರು
ಆಗಿದ್ದರೂ
ಆಗಿದ್ದ-ರೆಂದು
ಆಗಿದ್ದಲ್ಲಿ
ಆಗಿದ್ದಳು
ಆಗಿದ್ದ-ಳೆಂದು
ಆಗಿದ್ದಾಗ
ಆಗಿದ್ದಾನೆ
ಆಗಿದ್ದಾ-ನೆಂದು
ಆಗಿದ್ದಾರೆ
ಆಗಿದ್ದಾಳೆಂದು
ಆಗಿದ್ದಿತು
ಆಗಿದ್ದಿರ-ಬ-ಹುದು
ಆಗಿದ್ದಿರ-ಬಹು-ದೆಂದು
ಆಗಿದ್ದು
ಆಗಿನ
ಆಗಿನ್ನೂ
ಆಗಿ-ರಬ-ಬ-ಹುದು
ಆಗಿ-ರ-ಬಹದು
ಆಗಿ-ರ-ಬ-ಹುದು
ಆಗಿ-ರ-ಬಹು-ದೆಂದು
ಆಗಿ-ರಬೇಕೆಂದೂ
ಆಗಿ-ರ-ಲಿಲ್ಲ
ಆಗಿ-ರುತ್ತಾನೆ
ಆಗಿ-ರುತ್ತಾ-ನೆಂದು
ಆಗಿ-ರುತ್ತಾರೆ
ಆಗಿ-ರುತ್ತಿದ್ದರು
ಆಗಿ-ರುವ
ಆಗಿ-ರು-ವಂತೆ
ಆಗಿ-ರು-ವು-ದನ್ನು
ಆಗಿ-ರುವುದು
ಆಗಿಲ್ಲ
ಆಗಿವೆ
ಆಗಿ-ಹೋದ-ಮೇಲೆ
ಆಗಿ-ಹೋದರು
ಆಗಿ-ಹೋದ-ರೆಂದೂ
ಆಗುತ್ತದ
ಆಗುತ್ತದೆ
ಆಗುತ್ತಿತು
ಆಗುತ್ತಿಲ್ಲ
ಆಗು-ಮಾಡಿ-ಕೊಂಡು
ಆಗು-ಮಾಡಿ-ಸಿ-ಕೊಟ್ಟರು
ಆಗು-ವಷ್ಟು
ಆಗು-ವು-ದಿಲ್ಲ
ಆಗು-ವು-ದಿಲ್ಲ-ಎಂಬ
ಆಗ್ನೇಯ
ಆಗ್ನೇ-ಯಕ್ಕೆ
ಆಗ್ನೇ-ಯದ
ಆಗ್ನೇಯ-ದಲ್ಲಿ
ಆಗ್ರ-ಹಾರದ
ಆಘಾತ
ಆಚಂಣ
ಆಚಂದ್ರ
ಆಚಂದ್ರಾರ್ಕ-ವಾಗಿ
ಆಚಂದ್ರಾರ್ಕಸ್ಥಾಯಿ-ಯಾಗಿ
ಆಚಂದ್ರಾರ್ಕಸ್ಥಾಯಿಯಾಯಿ
ಆಚನ-ಹಳ್ಳಿ
ಆಚಮಂಗೆ
ಆಚಮಆಚಮ್ಮ
ಆಚ-ಮನ
ಆಚಮ-ನಿಗೇ
ಆಚ-ಮನು
ಆಚ-ರಣೆ-ಗಳ
ಆಚ-ರಣೆ-ಗ-ಳನ್ನು
ಆಚ-ರಣೆ-ಗ-ಳಲ್ಲಿ
ಆಚ-ರಣೆ-ಗಳು
ಆಚ-ರಣೆಗೆ
ಆಚ-ರಣೆಯ
ಆಚ-ರಣೆ-ಯನ್ನು
ಆಚ-ರಣೆ-ಯಲ್ಲಿ
ಆಚ-ರಾಜ
ಆಚರಿಸ-ಬೇಕು
ಆಚರಿಸಿ
ಆಚ-ರಿ-ಸಿದ
ಆಚರಿಸಿ-ರ-ಬ-ಹುದು
ಆಚರಿ-ಸುತ್ತಾ
ಆಚರಿ-ಸುತ್ತಾ-ರೆಂದು
ಆಚರಿಸುತ್ತಿದ್ದರು
ಆಚರಿಸು-ವಂತೆ
ಆಚ-ರಿ-ಸುವಾಗ
ಆಚಾಂಬಿಕೆ
ಆಚಾಂಬಿ-ಕೆಗೆ
ಆಚಾಂಬಿ-ಕೆಯ
ಆಚಾನ್
ಆಚಾರಿ
ಆಚಾರಿ-ಗಳ
ಆಚಾರಿ-ಗಳು
ಆಚಾರಿ-ಪಿಳ್ಳೆ
ಆಚಾರ್ಯ
ಆಚಾರ್ಯ-ನಾಗಿದ್ದಂತೆ
ಆಚಾರ್ಯ-ನಾದ
ಆಚಾರ್ಯ-ನೆಂದು
ಆಚಾರ್ಯ-ಪರಂಪ-ರೆಯ
ಆಚಾರ್ಯ-ಪರಂಪರೆ-ಯನ್ನು
ಆಚಾರ್ಯ-ಪುರುಷ-ರಾಗಿದ್ದಿರ-ಬ-ಹುದು
ಆಚಾರ್ಯ-ಪುರುಷ-ರಾದ
ಆಚಾರ್ಯ-ಪುರುಷರು
ಆಚಾರ್ಯ-ಪುರುಷ-ರು-ಗಳು
ಆಚಾರ್ಯರ
ಆಚಾರ್ಯ-ರಾ-ಗಿದ್ದರು
ಆಚಾರ್ಯ-ರಾದ
ಆಚಾರ್ಯ-ರಿಗೆ
ಆಚಾರ್ಯರು
ಆಚಾರ್ಯ-ರು-ಗಳ
ಆಚಾರ್ಯ-ರು-ಗಳು
ಆಚಾರ್ಯ-ರೆಂದು
ಆಚಾರ್ಯರೇ
ಆಚಾರ್ಯಾಯ
ಆಚಿ-ಕಬ್ಬೆ
ಆಚಿ-ಯಕ್ಕನು
ಆಚೀಚೆ
ಆಚೆ
ಆಚೆಗೆ
ಆಚೆಯೇ
ಆಚೋಜ
ಆಚ್ಚಾನ್
ಆಜ್ಞಾನುಸಾರ-ವಾಗಿ
ಆಜ್ಞಾಪಿ-ಸಿದ-ನೆಂದಿದೆ
ಆಜ್ಞಾಪಿ-ಸಿದ-ನೆಂದು
ಆಜ್ಞಾಪಿಸಿ-ದ-ವನು
ಆಜ್ಞಾಪಿ-ಸುವ
ಆಜ್ಞೆ
ಆಜ್ಞೆಯ
ಆಜ್ಞೆ-ಯಂತೆ
ಆಟು
ಆಠವ-ಣೆಯ
ಆಡಳಿ-ಗಾರ-ನನ್ನಾಗಿ-ರಾಜ್ಯ-ಪಾಲ
ಆಡಳಿ-ಗಾರ-ನಾದ
ಆಡಳಿತ
ಆಡಳಿ-ತಕ್ಕೆ
ಆಡಳಿತ-ಗಾರ-ರನ್ನು
ಆಡಳಿತ-ಗಾರರು
ಆಡಳಿ-ತದ
ಆಡಳಿತ-ದಲ್ಲಿ
ಆಡಳಿತ-ದಲ್ಲಿದ್ದ
ಆಡಳಿತ-ದಲ್ಲಿದ್ದ-ರೆಂದು
ಆಡಳಿತ-ದಲ್ಲೂ
ಆಡಳಿತ-ನ-ವನ್ನು
ಆಡಳಿತ-ವನ್ನು
ಆಡಳಿತ-ವನ್ನೂ
ಆಡಳಿತ-ವರ್ಗ-ದಿಂದ
ಆಡಳಿತ-ವರ್ಷ-ದಲ್ಲಿ
ಆಡಳಿತ-ವಿ-ಭಾಗ-ವಾಗಿತ್ತೆಂದು
ಆಡಳಿತ-ವೆಲ್ಲವೂ
ಆಡಳಿತವ್ಯವ-ಹಾರ-ಗಳು
ಆಡಳಿತ-ಸೂತ್ರ-ಗ-ಳನ್ನೂ
ಆಡಳಿತಾಧಿ-ಕಾರಿ
ಆಡಳಿತಾಧಿ-ಕಾರಿ-ಗಳ
ಆಡಳಿತಾಧಿ-ಕಾರಿ-ಗಳು
ಆಡಳಿತಾಧಿ-ಕಾರಿ-ಗಳೊಡನೆ
ಆಡಳಿತಾವಧಿ-ಯನ್ನು
ಆಡಳಿತಾವಧಿ-ಯಲ್ಲಿ
ಆಡಿನ
ಆಡು
ಆಡುಂಬೊಲ-ವಾದ
ಆಡು-ದೆರೆಯ
ಆಡು-ದೆಱೆ
ಆಡು-ಮನೆ
ಆಡು-ಮಾತೇ
ಆಣವಕ್ಕುರ-ನುಮ್
ಆಣೆ
ಆತ
ಆತಕೂರ
ಆತ-ಕೂರಿನ
ಆತಕೂರಿ-ನಲ್ಲಿ
ಆತಕೂರು
ಆತಕೂರು-ರಲ್ಲಿ
ಆತನ
ಆತ-ನನ್ನು
ಆತನನ್ವಯ
ಆತನಿಗಿರ-ಲಿಲ್ಲ
ಆತ-ನಿಗೆ
ಆತನು
ಆತನೇ
ಆತಿಶ್
ಆತಿಶ್ಖಾನ್
ಆತೂರು
ಆತ್ಕೂರು-ಆ-ತಕೂರು
ಆತ್ಮ
ಆತ್ಮ-ಜನೋ
ಆತ್ಮ-ಬಲಿ
ಆತ್ಮ-ಬಲಿ-ದಾನ
ಆತ್ಮ-ಬಲಿ-ದಾನದ
ಆತ್ಮ-ಬಲಿ-ದಾನವು
ಆತ್ಮ-ಬಲಿ-ದಾನ-ವೆಂದು
ಆತ್ಮ-ಬಲಿ-ಯನ್ನು
ಆತ್ಮ-ಭಕ್ತಿ-ಯಿಂದ
ಆತ್ಮ-ಮನೋ-ಹರೆ
ಆತ್ಮ-ವಧು
ಆತ್ಮ-ಶಕ್ತಿ
ಆತ್ಮ-ಹತ್ಯೆ
ಆತ್ಮ-ಹತ್ಯೆ-ಗ-ಳನ್ನು
ಆತ್ಮಾಗ್ರಜ
ಆತ್ಮಾಗ್ರಜ-ನನ್ನು
ಆತ್ಮಾರ್ಪಣೆ
ಆತ್ಮಾಹುತಿ
ಆತ್ಮೀಯ
ಆತ್ರೇಯ
ಆತ್ರೇಯ-ಗೋತ್ರದ
ಆತ್ರೇಯಸ
ಆಥವಾ
ಆದ
ಆದಂಣ್ಣ
ಆದ-ಕಾರಣ
ಆದಪ್ಪ
ಆದ-ಯಪ್ಪ
ಆದಯ್ಯ
ಆದರ-ದಿಂದ
ಆದರು
ಆದರೂ
ಆದರೆ
ಆದರ್ಶ
ಆದರ್ಶ-ಗ-ಳನ್ನು
ಆದರ್ಶ-ವಾಗಿದ್ದಿತು
ಆದಷ್ಟು
ಆದಾಯ
ಆದಾಯ-ಗ-ಳನ್ನು
ಆದಾಯದ
ಆದಾಯ-ದಲಿ
ಆದಾಯ-ದಲ್ಲಿ
ಆದಾಯ-ದಿಂದ
ಆದಾಯ-ವನ್ನು
ಆದಾಯ-ವನ್ನೆಲ್ಲಾ
ಆದಾಯ-ವಿ-ರುವ
ಆದಾ-ಯವು
ಆದಾಯ-ವುಳ್ಳ
ಆದಾವ-ನನ್ತ
ಆದಿ
ಆದಿಕ್ರೋಡಂ
ಆದಿ-ಗುಂಜ-ನರ-ಸಿಂಹ-ದೇವ-ರಿಗೆ
ಆದಿ-ಗುಂಜೆಯ
ಆದಿ-ಗೌಡ-ನ-ಕೆರೆ
ಆದಿ-ಚುಂಚನ-ಗಿರಿ
ಆದಿ-ಚುಂಚನ-ಗಿರಿ-ಒಂದು
ಆದಿ-ಚುಂಚನ-ಗಿರಿಯ
ಆದಿ-ಚುಂಚನ-ಗಿರಿಯು
ಆದಿತ್ಯ
ಆದಿತ್ಯ-ಗಾವುಂಡನು
ಆದಿತ್ಯನ
ಆದಿ-ದೇವ
ಆದಿ-ದೇವನ
ಆದಿ-ದೇವ-ರಿಗೆ
ಆದಿ-ನಾಥ
ಆದಿ-ಪುರಾಣ-ದಲ್ಲಿ
ಆದಿಬ್ಯಾದಿ
ಆದಿಭ್ಯಾದಿ
ಆದಿ-ಮಂಡಳ
ಆದಿಯ
ಆದಿ-ಯಮ
ಆದಿ-ಯ-ಮ-ನನ್ನು
ಆದಿ-ಯ-ಮನು
ಆದಿ-ಯ-ಮ-ನೋಡಿ-ದೋಟ
ಆದಿ-ರಂಗ-ವೆಂದು
ಆದಿ-ರುದ್ರ
ಆದಿಲ್
ಆದಿಲ್ನು
ಆದಿ-ವರಾಹ-ನಿಗೂ
ಆದಿ-ವಾರ
ಆದಿ-ಶೇಷನ
ಆದಿ-ಶೇಷ-ನಿದ್ದಾನೆ
ಆದಿ-ಶೇಷನು
ಆದಿ-ಸಿಂಗೆಯ
ಆದಿ-ಸಿಂಗೆಯ-ದಣ್ಣಾ-ಯ-ಕರುಮ
ಆದುದ-ರಿಂದ
ಆದುದ-ರಿಂದಲೇ
ಆದು-ದರಿ-ಮದ
ಆದು-ದಾಗಿದೆ
ಆದು-ರಿಂದ
ಆದೆಪ್ಪ
ಆದೆಪ್ಪನ
ಆದೆಮ್ಮ
ಆದೇಶ
ಆದೇಶಕ್ಕೆ
ಆದೇಶದ
ಆದೇಶ-ದಂತೆ
ಆದೇ-ಶ-ದಿಂದ
ಆದೇಶ-ವಾಗು-ವುದು
ಆದ್ದ-ರಿಂದ
ಆದ್ಯತೆ
ಆಧರಿಸಿ
ಆಧಾರ
ಆಧಾರ-ಗ-ಳನ್ನು
ಆಧಾರ-ಗ-ಳಲ್ಲಿ
ಆಧಾರ-ಗ-ಳಿಂದ
ಆಧಾರ-ಗಳಿಲ್ಲ
ಆಧಾರ-ಗಳಿವೆ
ಆಧಾರ-ಗಳು
ಆಧಾರ-ಗಳೂ
ಆಧಾರ-ಗಳೊಡೆನ
ಆಧಾ-ರದ
ಆಧಾರ-ದ-ಮೇಲೆ
ಆಧಾರ-ದಲ್ಲಿ
ಆಧಾರ-ದಿಂದ
ಆಧಾರ-ದಿಂದಲೂ
ಆಧಾರ-ವಾಗಿ
ಆಧಾರ-ವಾಗಿಟ್ಟು-ಕೊಂಡು
ಆಧಾರ-ವಾ-ಗಿದ್ದ
ಆಧಾರ-ವಿಲ್ಲ
ಆಧಾ-ರವೂ
ಆಧಾರ-ವೆಂದು
ಆಧಿ-ಕಾರಿ
ಆಧಿಕ್ಯ-ವನ್ನು
ಆಧಿ-ಪತ್ಯ-ದಲ್ಲಿ
ಆಧಿ-ಪತ್ಯ-ವನ್ನು
ಆಧುನಿಕ
ಆಧುನಿಕ-ವಾಗಿ
ಆನಂತರ
ಆನಂತ-ರದ
ಆನಂದ
ಆನಂದ-ವನ
ಆನಂದ-ಸಂವತ್ಸರಕ್ಕೆ
ಆನಂದಾಂಪಿಳ್ಳೆ
ಆನಂದಾನ್
ಆನಂದಾನ್ಪುಳ್ಳೆ
ಆನಂದಾನ್ಪುಳ್ಳೆ-ಯ-ವರ
ಆನಂದಾಳ್ವಾನ್
ಆನಂದೂರಿ-ನಲ್ಲಿ
ಆನ-ತರಾಗು-ವಂತೆ
ಆನೆ
ಆನೆ-ಕೆರೆ
ಆನೆ-ಗನ-ಕೆರಿ-ಆನೆ-ಕೆರೆ
ಆನೆ-ಗಳ
ಆನೆ-ಗ-ಳನ್ನು
ಆನೆ-ಗ-ಳಲ್ಲಿ
ಆನೆ-ಗಳು
ಆನೆ-ಗೊಂದಿ-ಯಲ್ಲಿ
ಆನೆ-ಬ-ಸದಿ
ಆನೆ-ಬ-ಸದಿಗೆ
ಆನೆ-ಬ-ಸದಿಯ
ಆನೆ-ಬ-ಸದಿಯು
ಆನೆ-ಮಂಡ-ಲಿಕರ
ಆನೆಯ
ಆನೆಯಂ
ಆನೆಯನು
ಆನೆ-ಯನ್ನು
ಆನೆಯ-ಸೇಸೆ
ಆನೆಯು
ಆನೆ-ಯೊಡನೆ
ಆನೆ-ವಾಳ
ಆನೆ-ವಾಳದ
ಆನೆ-ಸಲಗ
ಆನೆ-ಸಾ-ಸಲು
ಆನೆ-ಹಳ್ಳಿ
ಆನೆ-ಹಾಳು
ಆಪಸ್ತಂಭ
ಆಪಸ್ತಂಭ-ಸೂತ್ರದ
ಆಪಸ್ಥಂಬ
ಆಪ್ತ
ಆಪ್ತ-ಮಿತ್ರ
ಆಪ್ತರ
ಆಪ್ತ-ಸಹಾಯ-ಕರ
ಆಪ್ತೇಷ್ಟ-ರಲ್ಲಿ
ಆಪ್ತೇಷ್ಟರು
ಆಬಲ-ವಾಡಿಯ
ಆಬಲ-ವಾಡಿ-ಯಲ್ಲಿ
ಆಭಣ-ದನು-ಗುಣಂ
ಆಭರಣ
ಆಭರ-ಣ-ಗ-ಳನ್ನು
ಆಭರ-ಣ-ಗಳು
ಆಭಾಣೀ
ಆಭಾರಿ-ಯಾಗಿದ್ದೇನೆ
ಆಮೂಲಾಗ್ರ-ವಾಗಿ
ಆಮೆ-ಯನ್ನು
ಆಮೇಲೆ
ಆಮ್ನಾಯ
ಆಮ್ನಾಯ-ಆ-ವಳಿ
ಆಯ
ಆಯಂಗ-ಳಿಂದ
ಆಯಆಯತ
ಆಯ-ಕಟ್ಟಿನ
ಆಯ-ಗ-ಳನ್ನು
ಆಯ-ಗ-ಳಿಂದ
ಆಯ-ಗಾರರು
ಆಯ-ತಕ್ಕೆ
ಆಯ-ತದ
ಆಯದ
ಆಯ-ದಲ್ಲಿ
ಆಯ-ದಿಂದ
ಆಯ-ವತಿಬ್ಬರು
ಆಯ-ವನ್ನು
ಆಯಸ್ವಾಮ್ಯ
ಆಯಸ್ವಾಮ್ಯ-ದಿಂದ
ಆಯಸ್ಸು
ಆಯಾ
ಆಯಾ-ಮ-ಗ-ಳಿಗೆ
ಆಯಾಯ
ಆಯಿಘಂಡುಗ
ಆಯಿ-ತಕ್ಕೆ
ಆಯಿತು
ಆಯಿ-ತೆಂದು
ಆಯಿದು
ಆಯಿದು-ಮೊತ್ತದ
ಆಯಿರತ್ತು
ಆಯಿ-ವತಿ-ಬರು
ಆಯಿ-ವತಿ-ಬರೂ
ಆಯಿ-ವತಿಬ್ಬರ
ಆಯಿ-ವತಿಬ್ಬ-ರನ್ನು
ಆಯಿ-ವತಿಬ್ಬ-ರಲ್ಲಿ
ಆಯಿ-ವತಿಬ್ಬ-ರಿಗೆ
ಆಯಿ-ವತಿಬ್ಬರು
ಆಯಿ-ವತ್ತಿಬ್ಬರ
ಆಯಿ-ವತ್ತಿಬ್ಬ-ರಿಗೆ
ಆಯಿ-ವತ್ತಿಬ್ಬರು
ಆಯಿ-ವತ್ತಿಬ್ಬರೇ
ಆಯಿ-ವತ್ತು
ಆಯುಂ
ಆಯುಧ
ಆಯುಧ-ಗ-ಳನ್ನು
ಆಯುಧ-ಗಳು
ಆಯುರಾರೋಗ್ಯ
ಆಯ್ಕೆ
ಆಯ್ಕೆ-ಮಾಡಿ-ಕೊಳ್ಳುತ್ತಿದ್ದ-ರೆಂದು
ಆಯ್ಕೆ-ಯಾಗುತ್ತಿದ್ದರು
ಆಯ್ಕೆ-ಯಾದ-ವರು
ಆಯ್ಕೆ-ಯಾದ-ವ-ರೆಂದು
ಆಯ್ದು-ಕೊಂಡಿದೆ
ಆರಂಭ
ಆರಂಭ-ಗೊಂಡು
ಆರಂಭ-ಗೊಳ್ಳುತ್ತದೆ
ಆರಂಭದ
ಆರಂಭ-ದಲ್ಲಿ
ಆರಂಭ-ದಲ್ಲೇ
ಆರಂಭಲ್ಲನು
ಆರಂಭಲ್ಲ-ವನು
ಆರಂಭ-ವನು
ಆರಂಭ-ವಾಗಿ
ಆರಂಭ-ವಾಗಿದೆ
ಆರಂಭ-ವಾ-ಗಿದ್ದು
ಆರಂಭ-ವಾಗಿ-ರ-ಬ-ಹುದು
ಆರಂಭ-ವಾಗಿ-ರುವುದು
ಆರಂಭ-ವಾಗು-ತವೆ
ಆರಂಭ-ವಾಗುತ್ತದೆ
ಆರಂಭ-ವಾ-ಗುತ್ತವೆ
ಆರಂಭ-ವಾಗುವ
ಆರಂಭ-ವಾಗು-ವುದ-ರಿಂದ
ಆರಂಭ-ವಾ-ದಂತೆ
ಆರಂಭ-ವಾ-ದುದು
ಆರಂಭ-ವಾಯಿತು
ಆರಂಭ-ವಾಯಿ-ತೆಂದು
ಆರಂಭ-ವಾಯಿ-ತೆಂದೂ
ಆರಂಭ-ವಾಯಿ-ತೆನ್ನ-ಬ-ಹುದು
ಆರಂಭಿಸ-ಲಾಗಿದೆ
ಆರಂಭಿ-ಸಿ-ದಂತೆ
ಆರಂಭಿ-ಸಿ-ದನು
ಆರಂಭಿ-ಸಿದರು
ಆರಂಭಿ-ಸಿದ-ರೆಂದು
ಆರಂಭಿಸಿದ್ದನ್ನು
ಆರಂಭಿ-ಸುತ್ತಿ-ರುವ
ಆರಣ-ಪೂಜೆ
ಆರಣ-ಪೂಜೆ-ಗ-ಳಿಗೆ
ಆರಣ-ಪೂಜೆಗೆ
ಆರಣಿ
ಆರಣಿಯ
ಆರಣಿ-ಯನು
ಆರಣಿ-ಯಲ್ಲಿ
ಆರಣಿ-ಯಲ್ಲಿದ್ದ
ಆರಣಿಯು
ಆರಣಿ-ಯೂರ
ಆರಣಿ-ಸಯಸ್ಥಳದ
ಆರತಿ-ಯನ್ನು
ಆರ-ನೆಯ
ಆರ-ನೆಯ-ಬಾರಿ
ಆರನೇ
ಆರವೆ
ಆರಾತಿ-ನಾಯ-ಕ-ರ-ನಿಕ್ಕಿ
ಆರಾಧಕ-ನೆಂದು
ಆರಾಧಕ-ರಾಗಿದ್ದರು
ಆರಾಧಕ-ರಾದರು
ಆರಾಧಕ-ರೆಂದು
ಆರಾ-ಧನಾ
ಆರಾ-ಧನೆ-ಯನ್ನು
ಆರಾಧಿಸಿ
ಆರಾಧಿಸುತ್ತಿದ್ದ
ಆರಾಧಿ-ಸುವ
ಆರಾಧಿ-ಸು-ವುದು
ಆರಾಧ್ಯ
ಆರಾಧ್ಯ-ದೈವ
ಆರಾಧ್ಯೆ-ಯಾಗಿದ್ದುದು
ಆರಾರು
ಆರಾಸೆ
ಆರಿದ-ವಾಳಿ-ಕೆಯ
ಆರಿಸಿ-ಕೊಂಡ
ಆರಿಸಿ-ಕೊಂಡನು
ಆರಿಸಿ-ಕೊಂಡು
ಆರಿಸಿ-ಕೊಳ್ಳುತ್ತಿದ್ದರು
ಆರಿಸುತ್ತಿದ್ದಂತೆ
ಆರು
ಆರು-ಗುಳ
ಆರುಗ್ರಾಮ-ಗ-ಳನ್ನು
ಆರು-ತಲೆ-ಮಾರು-ಗಳ
ಆರು-ಪಟ್ಟಿ-ಕೆ-ಗಳುಳ್ಳ
ಆರು-ಬಾರಿ
ಆರು-ಸಾರಿ
ಆರು-ಹರಿ-ವಾಣ
ಆರೆಂಟು
ಆರೆಯರು
ಆರೈಕೆ
ಆರೊಬ್ಬರು
ಆರೋ-ಗಣೆ
ಆರೋ-ಗಣೆ-ಗ-ಳಿಗೆ
ಆರೋ-ಗಣೆಗೆ
ಆರೋಗ-ಣೆಯ
ಆರೋ-ಗಣೆ-ಯನ್ನು
ಆರೋಪಿಸ-ಲಾಗಿ-ರು-ವು-ದನ್ನು
ಆರೋಪಿಸಿ
ಆರೋಪಿ-ಸಿದ
ಆರೋಪಿ-ಸಿಲ್ಲ
ಆರೋಪಿ-ಸು-ವುದು
ಆರ್
ಆರ್ಎಸ್ಪಂಚ-ಮುಖಿ
ಆರ್ಕಾಟಿನ
ಆರ್ಕಾಟಿನ-ವ-ನಾದ
ಆರ್ಕಾಟ್
ಆರ್ಕಿಯಾಲಾಜಿಕಲ್
ಆರ್ಕಿಯಾಲಾಜಿಲ್
ಆರ್ಕಿಯೋಲಜಿಕಲ್
ಆರ್ತಧ್ವನಿ
ಆರ್ಥರ್
ಆರ್ಥಿಕ
ಆರ್ಥಿಕ-ವಾಗಿ
ಆರ್ದ್ರತೆ-ಯಿಂದ
ಆರ್ನರ-ಸಿಂಹಾ-ಚಾರ್ಯ
ಆರ್ಯ-ಮಂಡು-ನದ
ಆರ್ರಂಗಸ್ವಾಮಿ-ಯ-ವರು
ಆರ್ಶೇಷ-ಶಾಸ್ತ್ರಿ-ಯ-ವರು
ಆಱು-ಬಾರಿ
ಆಲಂಗಿಸಿ
ಆಲಂಪುರ-ದಲ್ಲಿ-ರುವ
ಆಲಂಬಾಡಿ
ಆಲ-ಕೆರೆ
ಆಲ-ಗಾವುಂಡ
ಆಲತಿ
ಆಲತ್ತೂ-ರನಿಱಿದು
ಆಲತ್ತೂರಿನ
ಆಲದ
ಆಲದ-ಮರ-ದೆಲೆ
ಆಲದ-ಹಳ್ಳಿ
ಆಲದ-ಹಳ್ಳಿಯ
ಆಲದ-ಹಳ್ಳಿ-ಯಲ್ಲಿ
ಆಲಪ್ಪ
ಆಲಯ-ಗ-ಳಿಗೆ
ಆಲ-ಯದ
ಆಲಯ-ದಲ್ಲಿ
ಆಲಯ-ಸುಂಕ
ಆಲಯ-ಸುಂಕ-ವೆಲ್ಲ-ವನೂ
ಆಲುಗೊಡು
ಆಲು-ಗೋಡನ್ನು
ಆಲು-ಗೋಡಿನ
ಆಲು-ಗೋಡು
ಆಲು-ಗೋಡು-ರಾಜ್ಯ
ಆಲೂರ
ಆಲೂರಿನ
ಆಲೂರಿನ-ವರ
ಆಲೂರಿನ-ವ-ರಿಗೂ
ಆಲೂರು
ಆಲೆ-ದೆರೆ
ಆಲೆ-ಮನೆ
ಆಲೆ-ಮ-ನೆಯ
ಆಲೆ-ವನೆ
ಆಲೇನ-ಹಳ್ಳಿ
ಆಲೇನ-ಹಳ್ಳಿಯ
ಆಲ್ಗೋಡು
ಆಲ್ಲಪ್ಪನ-ಹಳ್ಳಿ
ಆಳ
ಆಳತ್ತಿದ್ದ-ನೆಂದು
ಆಳದೇ
ಆಳ-ಮಾಡಿ-ದೊಡ್ಡ-ದೊಡ್ಡ
ಆಳಲು
ಆಳ-ವಾಗಿ
ಆಳ-ವಾಗಿ-ರುತ್ತಿದ್ದವು
ಆಳ-ವಾದ
ಆಳಹಿಯ
ಆಳಾನ್
ಆಳಿ-ಕೊಂಡು
ಆಳಿದ
ಆಳಿ-ದನು
ಆಳಿದ-ನೆಂದು
ಆಳಿದ-ನೆಂದೂ
ಆಳಿ-ದರು
ಆಳಿದ-ರೆಂದೂ
ಆಳಿಸಿ-ಹೋಗಿವೆ
ಆಳು-ಗಳು
ಆಳು-ಗೋಡೀ
ಆಳು-ಗೋಡು
ಆಳುಡೈಯಾನ್
ಆಳುತಿದ್ದನು
ಆಳು-ತಿದ್ದರು
ಆಳು-ತಿದ್ದಾಗ
ಆಳುತ್ತಾ
ಆಳುತ್ತಿದ್ದ
ಆಳುತ್ತಿದ್ದಂತೆ
ಆಳುತ್ತಿದ್ದನು
ಆಳುತ್ತಿದ್ದ-ನೆಂದಿದೆ
ಆಳುತ್ತಿದ್ದ-ನೆಂದು
ಆಳುತ್ತಿದ್ದ-ನೆಂದೂ
ಆಳುತ್ತಿದ್ದ-ನೆಂಬುದು
ಆಳುತ್ತಿದ್ದರು
ಆಳುತ್ತಿದ್ದರೂ
ಆಳುತ್ತಿದ್ದ-ರೆಂದು
ಆಳುತ್ತಿದ್ದ-ರೆಂದೂ
ಆಳುತ್ತಿದ್ದ-ರೆಂಬ
ಆಳುತ್ತಿದ್ದಳು
ಆಳುತ್ತಿದ್ದ-ಳೆಂದು
ಆಳುತ್ತಿದ್ದ-ವನು
ಆಳುತ್ತಿದ್ದ-ವ-ರಿಗೆ
ಆಳುತ್ತಿದ್ದ-ವ-ರೆಂದರೆ
ಆಳುತ್ತಿದ್ದಾಗ
ಆಳುತ್ತಿದ್ದಿರ
ಆಳುತ್ತಿದ್ದಿರ-ಬ-ಹುದು
ಆಳುತ್ತಿದ್ದು
ಆಳುತ್ತಿದ್ದು-ದನ್ನು
ಆಳುತ್ತಿದ್ದು-ದ-ರಿಂದ
ಆಳುತ್ತಿದ್ದುದು
ಆಳುತ್ತಿ-ರಲು
ಆಳುತ್ತಿರುತ್ತಾನೆ
ಆಳುವ
ಆಳುವ-ಖೇಡ
ಆಳು-ವ-ವನು
ಆಳೆತ್ತ-ರದ
ಆಳ್ತನ-ವನ್ನು
ಆಳ್ದನ
ಆಳ್ವಕೆ
ಆಳ್ವ-ಖೇಡ
ಆಳ್ವಾ-ನಂಗೈ
ಆಳ್ವಾನ್
ಆಳ್ವಾರರ
ಆಳ್ವಾ-ರರು
ಆಳ್ವಾ-ರರು-ಗಳ
ಆಳ್ವಾ-ರರು-ಗಳು
ಆಳ್ವಾರು
ಆಳ್ವಾರು-ಗ-ಳಲ್ಲಿ
ಆಳ್ವಾರ್
ಆಳ್ವಿಕೆ
ಆಳ್ವಿಕೆಗೆ
ಆಳ್ವಿಕೆಯ
ಆಳ್ವಿಕೆ-ಯನ್ನು
ಆಳ್ವಿಕೆ-ಯನ್ನೇ
ಆಳ್ವಿಕೆ-ಯಲ್ಲಿ
ಆಳ್ವಿಕೆ-ಯಲ್ಲೂ
ಆಳ್ವಿಕೆಯೇ
ಆಳ್ವಿ-ಭಟ್ಟ-ರು-ಗ-ಳಿಗೆ
ಆವಗಂ
ಆವಟ್ಟ
ಆವಣ-ದಲ್ಲಿ
ಆವ-ನೊಬ್ಬ
ಆವರ-ಣದ
ಆವರ-ಣ-ದಲ್ಲಿ
ಆವರ-ಣದಲ್ಲಿಯೇ
ಆವರ-ಣದಲ್ಲಿ-ರುವ
ಆವರ-ಣ-ದೊಳಗೆ
ಆವಳಿ
ಆವಾಸಸ್ಥಾನ-ವಾಗಿತ್ತೆಂದು
ಆವು-ಗೆಯ
ಆವು-ಗೆಯ-ದೇವರ
ಆವುದೂ
ಆವೃತ-ವಾದ
ಆಶಯ
ಆಶಯ-ವನ್ನು
ಆಶಯ-ವಾಗಿದೆ
ಆಶಾ-ದಾಯಕ
ಆಶೇಷ
ಆಶ್ಚರ್ಯಕರ
ಆಶ್ಚರ್ಯಕ-ರ-ವಾಗಿದೆ
ಆಶ್ಚರ್ಯ-ವಾಯಿತು
ಆಶ್ರಯ
ಆಶ್ರಯ-ದಲ್ಲಿ
ಆಶ್ರಯ-ದಲ್ಲಿದ್ದ-ನೆಂದೂ
ಆಶ್ರಯ-ದಲ್ಲಿ-ರುವ
ಆಶ್ರಯ-ದಾತ-ನಾಗಿದ್ದ-ನೆಂಬುದು
ಆಶ್ರಯ-ದಾತ-ನಾದ
ಆಶ್ರಯ-ವರ್ತಿ
ಆಶ್ರಯ-ವರ್ತಿ-ಯಾ-ಗಿದ್ದ
ಆಶ್ರಯ-ವರ್ತಿ-ಯಾಗಿದ್ದು-ಕೊಂಡು
ಆಶ್ರಯಿಸಿ
ಆಶ್ರಯಿಸಿ-ದ-ನೆಂದು
ಆಶ್ರಯಿ-ಸಿದ್ದು
ಆಶ್ರಿ-ತ-ಜನ-ಕಲ್ಪ-ವೃಕ್ಷ
ಆಶ್ರಿ-ತ-ನಾ-ಗಿದ್ದ
ಆಶ್ರಿ-ತ-ನಾ-ಗಿದ್ದು
ಆಶ್ವ-ದಳ-ದೊಂದಿಗೆ
ಆಶ್ವಲಾ-ಯನ
ಆಶ್ವಲಾ-ಯನ-ಸೂತ್ರ
ಆಶ್ವೀಜ
ಆಸಂದಿ
ಆಸಂದಿ-ನಾಡ
ಆಸಂಧಿ-ನಾಡ
ಆಸಂನ್ನ
ಆಸಕ್ತ-ನಾಗಿದ್ದನು
ಆಸಕ್ತನು
ಆಸಕ್ತರು
ಆಸಕ್ತಿ
ಆಸಕ್ತಿ-ದಾಯ-ಕ-ವಾಗಿ-ರುವು-ದ-ರಿಂದ
ಆಸಕ್ತಿ-ಯನ್ನು
ಆಸನ-ಹಾಳ
ಆಸನ-ಹಾಳ-ಕೆರೆಯ
ಆಸನ್ನ
ಆಸಸ್ತಂಭ-ಸೂತ್ರದ
ಆಸೀನಳಾ-ಗಿದ್ದು
ಆಸೀನ-ವಾದ
ಆಸುಪಾಸಿ-ನಲ್ಲಿ
ಆಸೆ
ಆಸೆ-ಮಾ-ಡುವ
ಆಸೇತು-ಮೇರು-ಪರ್ಯಂತಂ
ಆಸ್ತಿ
ಆಸ್ತಿ-ಗ-ಳನ್ನು
ಆಸ್ತಿ-ತೆ-ರಿಗೆ
ಆಸ್ತಿ-ಪಾಸ್ತಿ-ಗಳು
ಆಸ್ತಿಯ
ಆಸ್ತಿ-ಯ-ಕರ-ಗಳು
ಆಸ್ತಿ-ಯನ್ನು
ಆಸ್ತಿ-ಯಾ-ಗಿದ್ದು
ಆಸ್ತಿ-ಹಂಚಿಕೆ
ಆಸ್ಥಾನ
ಆಸ್ಥಾನ-ಕವಿ-ಯಾ-ಗಿದ್ದ
ಆಸ್ಥಾನಕ್ಕೆ
ಆಸ್ಥಾನ-ಜಗಜೆಟಿ
ಆಸ್ಥಾನದ
ಆಸ್ಥಾನ-ದಲ್ಲಿ
ಆಸ್ಥಾನ-ದಲ್ಲಿದ್ದ
ಆಸ್ಥಾನ-ದಲ್ಲಿದ್ದನು
ಆಸ್ಥಾನ-ದಲ್ಲಿದ್ದ-ನೆಂದು
ಆಸ್ಥಾನ-ದಲ್ಲಿದ್ದು
ಆಸ್ಥಾನ-ವನ್ನು
ಆಸ್ಥಾ-ಯಿಕಾ
ಆಸ್ಪದ
ಆಹವ-ಮಲ್ಲನ
ಆಹಾರ
ಆಹಾರಕ್ಕೆ
ಆಹಾರ-ದಾನ
ಆಹಾರ-ದಾನಕ್ಕಾಗಿ
ಆಹಾರ-ದಾನಕ್ಕೆ
ಆಹಾರ-ಧಾನ್ಯ-ಗಳ
ಆಹಾರ-ಧಾನ್ಯ-ವನ್ನು
ಆಹಾರ-ಪ-ದಾರ್ಥ-ಗಳ
ಆಹಾರ-ಪೂರೈಕೆ
ಆಹಾರ-ಮಂಡ-ಲ-ಭುಕ್ತಿ-ವಿಷಯ-ದೇಶ
ಆಹಾರಾಭಯ
ಆಹಾರಾಭ-ಯನುಂ
ಆಹಾರಾಭಯ-ಭೈಷಜ್ಯ-ಶಾಸ್ತ್ರ-ದಾನ
ಆಹಾರಾಭಯ-ಭೈಷಜ್ಯ-ಶಾಸ್ತ್ರ-ವಿನೋ-ದನುಂ
ಆಹೋ-ಬಲ-ವಾಡಿ-ಯಾಗಿ
ಆಹೋ-ಬಳ-ಪುರ-ವೆಂದು
ಆಹ್ವಾನಿ-ಸ-ಲಾ-ಯಿತು
ಆಹ್ವಾನಿಸುತ್ತಿದ್ದನು
ಇ
ಇಂಗಲ-ಗುಪ್ಪೆ
ಇಂಗಲ-ಗುಪ್ಪೆಯ
ಇಂಗ-ಲಿಕನ
ಇಂಗ-ಲಿಕ-ನ-ಕುಪ್ಪೆಯ
ಇಂಗ-ಲೀಕನ
ಇಂಗ-ಲೀಕ-ನ-ಕುಪ್ಪೆಯ
ಇಂಗಳ-ಗುಪ್ಪೆ
ಇಂಗಳ-ಗುಪ್ಪೆಯ
ಇಂಗಳೇಶ್ವರ
ಇಂಗಳೇಶ್ವರ-ಬಳಿಯ
ಇಂಗ್ಲಿಷ್
ಇಂಚು
ಇಂಡಿಯಾ-ದಲ್ಲಿ
ಇಂತಹ
ಇಂತಹದೇ
ಇಂತಿ-ವರ
ಇಂತಿ-ವರು-ಭಯಾನು-ಮತದಿಂ
ಇಂತಿಷ್ಟು
ಇಂತೀ
ಇಂಥ
ಇಂದಿಗೂ
ಇಂದಿನ
ಇಂದು
ಇಂದು-ಕೊಟ್ಟು
ಇಂದ್ರ-ನಂತೆ
ಇಂದ್ರ-ನನ್ನು
ಇಂದ್ರ-ನಾಗಿದ್ದಾ-ನೆಂದು
ಇಂದ್ರ-ನಿಗೆ
ಇಂದ್ರನು
ಇಂದ್ರ-ಪರ್ವ
ಇಂದ್ರ-ಪರ್ವ-ಗಳ
ಇಂದ್ರ-ಪರ್ವ-ಗ-ಳಿಗೆ
ಇಂದ್ರ-ಪರ್ವದ
ಇಂದ್ರ-ಪೂಜೆ
ಇಂದ್ರ-ಪೂಜೆಗೆ
ಇಂದ್ರಪ್ರಸ್ತ-ವನು
ಇಂದ್ರ-ರಾಜನ
ಇಂದ್ರ-ವರ್ಮ-ನೆಂಬ
ಇಂನೂರ
ಇಂಬು
ಇಂಬು-ಕೊಡುತ್ತವೆ
ಇಂಮಡಿ-ದೇವ
ಇಕ್ಕದೆ
ಇಕ್ಕಿ-ಕೊಟ್ಟು
ಇಕ್ಕಿದಂಥಾ
ಇಕ್ಕಿಸಿ
ಇಕ್ಕಿಸಿ-ದೆವು
ಇಕ್ಕುಳ
ಇಕ್ಕೇರಿ
ಇಕ್ಕೇರಿಯ
ಇಕ್ಕೋಯಿ-ಲಿನ
ಇಕ್ಕೋ-ಯಿಲ್
ಇಗಿಲ-ಗುಪ್ಪೆ-ಯಇಂಗಳೀ-ಕನ
ಇಗ್ಗ-ಲೂರು
ಇಚ್ಚಿ-ಸದೇ
ಇಚ್ಚೆ-ಯಂತೆ
ಇಜ್ಜಲ-ಘಟ್ಟ-ವೆಂಬ
ಇಜ್ಜ-ಲನ್ನು
ಇಟಗಿ
ಇಟ್ಟ
ಇಟ್ಟನು
ಇಟ್ಟ-ನೆಂದೂ
ಇಟ್ಟ-ರೆಂದೂ
ಇಟ್ಟಾಗ
ಇಟ್ಟಾಡಿ
ಇಟ್ಟಿಗೆ
ಇಟ್ಟಿಗೆ-ಗಂದ-ಗಳ್ದ
ಇಟ್ಟಿಗೆ-ಯಲ್ಲಿ
ಇಟ್ಟಿಗೆ-ಯಿಂದ
ಇಟ್ಟಿದ್ದ
ಇಟ್ಟಿದ್ದನು
ಇಟ್ಟಿದ್ದರು
ಇಟ್ಟಿದ್ದಾ-ನೆಂದು
ಇಟ್ಟಿದ್ದಾರೆ
ಇಟ್ಟಿರ-ಬ-ಹುದು
ಇಟ್ಟಿರ-ಬೇಕಾಗುತ್ತಿತ್ತು
ಇಟ್ಟಿ-ರುವ
ಇಟ್ಟಿ-ರುವುದು
ಇಟ್ಟೀಯ-ಕೊಳ-ವೆಂದು
ಇಟ್ಟು
ಇಟ್ಟು-ಕೊಂಡ-ನೆಂದು
ಇಟ್ಟು-ಕೊಂಡರು
ಇಟ್ಟು-ಕೊಂಡರೂ
ಇಟ್ಟು-ಕೊಂಡರೆ
ಇಟ್ಟು-ಕೊಂಡ-ರೆಂದು
ಇಟ್ಟು-ಕೊಂಡಿದ್ದ
ಇಟ್ಟು-ಕೊಂಡಿದ್ದ-ನೆಂದೂ
ಇಟ್ಟು-ಕೊಂಡಿದ್ದರು
ಇಟ್ಟು-ಕೊಂಡಿದ್ದ-ರೆಂದು
ಇಟ್ಟು-ಕೊಂಡಿದ್ದಾನೆ
ಇಟ್ಟು-ಕೊಂಡಿದ್ದೇನೆ
ಇಟ್ಟು-ಕೊಂಡಿರು-ವುದು
ಇಟ್ಟು-ಕೊಂಡು
ಇಟ್ಟು-ಕೊಂಡೆರೆ
ಇಟ್ಟು-ಕೊಳ್ಳ-ಬ-ಹುದು
ಇಟ್ಟು-ಕೊಳ್ಳು
ಇಟ್ಟು-ಕೊಳ್ಳುತ್ತಾರೆ
ಇಟ್ಟು-ಕೊಳ್ಳುತ್ತಿದ್ದ
ಇಟ್ಟು-ಕೊಳ್ಳುತ್ತಿದ್ದ-ನೆಂದು
ಇಟ್ಟು-ಕೊಳ್ಳುತ್ತಿದ್ದರು
ಇಟ್ಟು-ಕೊಳ್ಳುತ್ತಿದ್ದ-ರೆಂದು
ಇಟ್ಟು-ಕೊಳ್ಳುತ್ತಿದ್ದ-ರೆಂಬುದು
ಇಟ್ಟು-ಕೊಳ್ಳುತ್ತಿದ್ದು
ಇಟ್ಟು-ಕೊಳ್ಳು-ವ-ವರ
ಇಟ್ಟು-ಕೊಳ್ಳು-ವುದು
ಇಟ್ಟು-ಕೊಳ್ಳು-ವುದುನ್ನು
ಇಡ-ಗೂರು
ಇಡ-ಲಾಗಿದೆ
ಇಡ-ಲಾಗುತ್ತಿತ್ತು
ಇಡ-ಲಾ-ಯಿತು
ಇಡೀ
ಇಡುಗೂರ
ಇಡು-ಗೂರು
ಇಡುತುರೈ-ನಾಟ್ಟು
ಇಡುತ್ತಾನೆ
ಇಡುತ್ತಾರೆ
ಇಡುತ್ತಿದ್ದುದು
ಇಡುದುರೈ
ಇಡು-ಪಡಿ-ಯಾಗಿ
ಇಡುವ
ಇಡು-ವು-ದನ್ನು
ಇಡೆಯ-ನಾಡು
ಇಡೈತುರೈ
ಇಡೈತುರೈ-ನಾಡನ್ನು
ಇಡೈತುರೈ-ನಾಡನ್ನು-ಎಡ-ದೊರೆ-ನಾಡು
ಇಡೈತುರೈ-ನಾಡು-ಕಾವೇರಿ
ಇಡೈಮು-ನೂರು-ಇಡೈಕುನ್ದ-ನಾಡು
ಇತರ
ಇತರ-ರನ್ನು
ಇತರರು
ಇತರೆ
ಇತರೆ-ಉ-ಳಿದ-ವರು
ಇತಿ
ಇತಿ-ಹಾಸ
ಇತಿ-ಹಾಸ-ಕಾರರ
ಇತಿ-ಹಾಸ-ಕಾರರು
ಇತಿ-ಹಾಸಕ್ಕಿಂತ
ಇತಿ-ಹಾಸಕ್ಕೆ
ಇತಿ-ಹಾಸ-ಗ-ಳನ್ನು
ಇತಿ-ಹಾಸದ
ಇತಿ-ಹಾಸ-ದಲ್ಲಿ
ಇತಿ-ಹಾಸ-ದಿಂದ
ಇತಿ-ಹಾಸಪ್ರ-ಸಿದ್ಧ
ಇತಿ-ಹಾಸ-ವನ್ನು
ಇತಿ-ಹಾಸ-ವಿದ್ವಾಂಸರು
ಇತಿ-ಹಾ-ಸವು
ಇತಿ-ಹಾಸ-ವೆಂದರೆ
ಇತಿ-ಹಾಸವೇ
ಇತೀಚೆಗಿ-ವರೆಗೆ
ಇತ್ತ
ಇತ್ತಣ
ಇತ್ತಿರು-ಮರಮ್
ಇತ್ತಿರುಮುರ್ರಮ್
ಇತ್ತೀಚಿನ-ವರೆಗೂ
ಇತ್ತೀಚೆ
ಇತ್ತೀಚೆ-ಗಿನ-ವರೆಗೂ
ಇತ್ತೀಚೆಗೆ
ಇತ್ತು
ಇತ್ತೆ
ಇತ್ತೆಂದು
ಇತ್ತೆಂದೂ
ಇತ್ತೆಂಬುದು
ಇತ್ತೇ
ಇತ್ಯಾದಿ
ಇತ್ಯಾದಿ-ಗಳ
ಇತ್ಯಾದಿ-ಗ-ಳನ್ನು
ಇತ್ಯಾದಿ-ಗ-ಳಿಂದ
ಇತ್ಯಾದಿ-ಗ-ಳಿಗೆ
ಇತ್ಯಾದಿ-ಯಾಗಿ
ಇದ
ಇದಕ್ಕಾಗಿ
ಇದಕ್ಕಿಂತ
ಇದಕ್ಕಿದ್ದ
ಇದಕ್ಕೂ
ಇದಕ್ಕೆ
ಇದ-ದಿರ-ಬ-ಹುದು
ಇದ-ನ-ಳಿದುಣ್ಡೋನ್
ಇದ-ನ-ಳಿದೋಂ
ಇದನು
ಇದನ್ನು
ಇದನ್ನೂ
ಇದನ್ನೆಲ್ಲಾ
ಇದನ್ನೇ
ಇದರ
ಇದ-ರಲ್ಲಿ
ಇದ-ರಲ್ಲಿದೆ
ಇದ-ರಲ್ಲಿದ್ದು
ಇದ-ರಿಂದ
ಇದ-ರಿಂದಲೇ
ಇದ-ರಿಂದಾಗ
ಇದ-ರಿಂದಾಗಿ
ಇದ-ರಿಂದಾಗಿಯೇ
ಇದಲ್ಲದೆ
ಇದಾಗಿದೆ
ಇದಾ-ಗಿದ್ದು
ಇದಾಗಿ-ರ-ಬ-ಹುದು
ಇದಾಗಿವೆ
ಇದಾದ
ಇದಿ-ರಾದಾಗ
ಇದಿಷ್ಟೂ
ಇದು
ಇದು-ಳೆ-ಯನ್ನು
ಇದು-ವರೆ-ಗಿನ
ಇದು-ವರೆಗೆ
ಇದೂ
ಇದೆ
ಇದೆ-ಯೆಂದು
ಇದೆಯೇ
ಇದೇ
ಇದೊಂದು
ಇದೊಂದೇ
ಇದ್ದ
ಇದ್ದಂತಹ
ಇದ್ದಂತೆ
ಇದ್ದಂತೆಯೂ
ಇದ್ದಕ್ಕಿದ್ದ-ಹಾಗೆ
ಇದ್ದದ್ದೇ
ಇದ್ದ-ನಂತೆ
ಇದ್ದನು
ಇದ್ದ-ನೆಂದು
ಇದ್ದ-ನೆಂದೂ
ಇದ್ದ-ನೆಂಬುದು
ಇದ್ದ-ನೆನ್ನು-ವು-ದರ
ಇದ್ದ-ಮೇಲೆ
ಇದ್ದರು
ಇದ್ದರೂ
ಇದ್ದರೆ
ಇದ್ದ-ರೆಂದು
ಇದ್ದ-ರೆಂದೂ
ಇದ್ದ-ರೆಂಬ
ಇದ್ದಳು
ಇದ್ದ-ಳೆಂದು
ಇದ್ದವು
ಇದ್ದವೆ
ಇದ್ದ-ವೆಂದು
ಇದ್ದ-ಹಾಗೆ
ಇದ್ದಾಗ
ಇದ್ದಾರೆ
ಇದ್ದಿತು
ಇದ್ದಿ-ತೆಂದು
ಇದ್ದಿ-ತೆಂದೂ
ಇದ್ದಿತೆಂಬ
ಇದ್ದಿತೆಂಬುದು
ಇದ್ದಿತೇ
ಇದ್ದಿರ-ಬಹು-ದಾದ
ಇದ್ದಿರ-ಬ-ಹುದು
ಇದ್ದಿರ-ಬೇಕು
ಇದ್ದಿರುವ
ಇದ್ದೀಯ
ಇದ್ದು
ಇದ್ದು-ಕೊಂಡು
ಇದ್ದು-ದನ್ನು
ಇದ್ದು-ದನ್ನೂ
ಇದ್ದು-ದ-ರಿಂದ
ಇದ್ದುದು
ಇದ್ದುದೇ
ಇದ್ದೇ
ಇನನ
ಇನಾ-ಮಾಗಿ
ಇನಿಗೆ
ಇನ್ತೀ
ಇನ್ದರ
ಇನ್ನಿಬ್ಬರು
ಇನ್ನು
ಇನ್ನುರ
ಇನ್ನೂ
ಇನ್ನೂರ
ಇನ್ನೂ-ರ-ಹನ್ನೊಂದ
ಇನ್ನೂರು
ಇನ್ನೂ-ರೆಂಬತ್ತು
ಇನ್ನೆ-ರಡು
ಇನ್ನೊಂದನ್ನು
ಇನ್ನೊಂದ-ರಲ್ಲಿ
ಇನ್ನೊಂದಿಬ್ಬ-ರಿಗೆ
ಇನ್ನೊಂದಿಬ್ಬರು
ಇನ್ನೊಂದು
ಇನ್ನೊಬ್ಬ
ಇನ್ನೊಬ್ಬ-ರಿಗೆ
ಇನ್ನೊಬ್ಬಳು
ಇನ್ನೊಮ್ಮೆ
ಇಪತ್ತಿನ
ಇಪ್ಪತ್ತ-ನಾಲ್ಕು
ಇಪ್ಪತ್ತ-ನೆಯ
ಇಪ್ಪತ್ತಾರು
ಇಪ್ಪತ್ತು
ಇಪ್ಪತ್ತು-ಸಾವಿರ
ಇಪ್ಪತ್ತು-ಸಾವಿರದ
ಇಪ್ಪತ್ತೆಂಟು
ಇಪ್ಪತ್ತೆ-ರಡು
ಇಪ್ಪತ್ತೈದು
ಇಪ್ಪತ್ತೊಂದ-ನೆಯ
ಇಪ್ಪತ್ತೊಂದು
ಇಪ್ಪೊತ್ತಿನ
ಇಬನ್ಬತೂತ್
ಇಬ್ಬರ
ಇಬ್ಬ-ರನ್ನು
ಇಬ್ಬ-ರನ್ನೂ
ಇಬ್ಬ-ರಿಗೂ
ಇಬ್ಬ-ರಿಗೆ
ಇಬ್ಬರು
ಇಬ್ಬರೂ
ಇಬ್ರಾಹಿಂ
ಇಭಾಟು
ಇಮಾಮ್
ಇಮ್ಮಡಿ
ಇಮ್ಮಡಿ-ದೇವ-ನ-ದೇ-ರಾಯನ
ಇಮ್ಮಡಿ-ದೇವ-ರಾಯ
ಇಮ್ಮಡಿ-ದೇವ-ರಾಯ-ನನ್ನು
ಇಮ್ಮಡಿ-ಬಲ್ಲಾಳನ
ಇಮ್ಮಡಿ-ಬಲ್ಲಾಳ-ನಿಂದ
ಇಮ್ಮಡಿ-ಬೀರ
ಇಮ್ಮಡಿ-ಬುಕ್ಕ-ರಾಜ-ಪುರ-ವೆಂಬ
ಇಮ್ಮಡಿ-ಬೂತುಗನು
ಇಮ್ಮಡಿ-ಯಾ-ಯಿತು
ಇಮ್ಮಡಿ-ರಾಯ
ಇಮ್ಮಡಿ-ರಾ-ವುತ್ತ-ರಾಯ
ಇರಂಡು
ಇರಣ್ಡು-ಕರೈ
ಇರ-ಬಹದು
ಇರ-ಬಹು-ದಾ-ದರೂ
ಇರ-ಬ-ಹುದು
ಇರ-ಬಹುದೆ
ಇರ-ಬಹು-ದೆಂದು
ಇರ-ಬಹು-ದೆಂದೂ
ಇರ-ಬೇಕಾಗಿತ್ತೆಂಬು-ದನ್ನು
ಇರ-ಬೇಕಾಗಿತ್ತೆಂಬುದು
ಇರ-ಬೇಕು
ಇರಬೇಕೆಂದಷ್ಟೇ
ಇರಲಿ
ಇರ-ಲಿಲ್ಲ
ಇರ-ಲಿಲ್ಲ-ವೆಂದು
ಇರ-ಲಿಲ್ಲ-ವೆಂದೂ
ಇರ-ಲಿಲ್ಲ-ವೆಂಬುದು
ಇರಲು
ಇರಲೂ-ಬ-ಹುದು
ಇರಲೇ
ಇರಲೇ-ಬೇ-ಕಷ್ಟೆ
ಇರವಿ-ಕುಲ-ಮಾಣಿಕ್ಯ
ಇರಾಮನ್
ಇರಿದಂ
ಇರಿ-ದನು
ಇರಿ-ದನು-ಯುದ್ಧ-ಮಾಡಿ-ದನು
ಇರಿದ-ನೆಂದು
ಇರಿದು
ಇರಿದು-ಹೋರಾಡಿ
ಇರಿವಬೆಡಂಗ
ಇರಿ-ಸ-ಲಾ-ಗಿತ್ತು
ಇರಿಸ-ಲಾಗಿದೆ
ಇರಿಸಿ
ಇರಿಸಿ-ಕೊಂಡಿದ್ದ-ನೆಂದು
ಇರಿಸಿ-ಕೊಂಡು
ಇರಿಸಿ-ದರು
ಇರಿಸಿದ್ದ-ನೆಂದು
ಇರಿಸಿದ್ದರು
ಇರು
ಇರುಂಗೋ-ಳನ
ಇರುಂಗೋ-ಳನ-ಕೋಟೆ
ಇರುಂಗೋ-ಳನೂ
ಇರು-ಗಂಗಣ್ಣ
ಇರು-ಗ-ಕಾರ-ನೆಂಬು
ಇರು-ಗಪ್ಪನು
ಇರು-ತಿದ್ದ-ನೆಂದು
ಇರುತ್ತದೆ
ಇರುತ್ತವೆ
ಇರುತ್ತಾನೆ
ಇರುತ್ತಿತತ್ತೆಂದು
ಇರುತ್ತಿತ್ತು
ಇರುತ್ತಿತ್ತೆಂದು
ಇರುತ್ತಿದ್ದ
ಇರುತ್ತಿದ್ದನು
ಇರುತ್ತಿದ್ದ-ನೆಂದು
ಇರುತ್ತಿದ್ದರು
ಇರುತ್ತಿದ್ದ-ರೆಂದು
ಇರುತ್ತಿದ್ದವು
ಇರುತ್ತಿದ್ದುದು
ಇರುತ್ತೇವೆ
ಇರು-ಮುಡಿ-ಚೋಳ
ಇರುವ
ಇರು-ವಂತೆ
ಇರು-ವ-ವರು
ಇರು-ವು-ದಕ್ಕೆ
ಇರು-ವು-ದನ್ನು
ಇರು-ವು-ದನ್ನೂ
ಇರು-ವುದ-ರಿಂದ
ಇರು-ವುದ-ರಿಂದಲೂ
ಇರು-ವು-ದಿಲ್ಲ
ಇರು-ವುದು
ಇರು-ವುದುಂಟು
ಇರೈ-ಅಪ್ಪನ್
ಇರೈ-ವಾನ-ರೈ-ಯೂರಿನ
ಇರೈ-ವಾನ-ರೈ-ಯೂರಿಲ್
ಇರೈ-ವಾನ್
ಇರೋಜಿ
ಇರ್ಗ್ಗರೆ
ಇರ್ದ್ದ
ಇರ್ರಾಜೇಂದ್ರ
ಇರ್ರಾಮನ್
ಇರ್ರಾಮನ್ಪಿರಾನ್
ಇರ್ರಾಮಾ-ನುಜ
ಇಲಾಖೆ
ಇಲಾಖೆ-ಗಳ
ಇಲಾಖೆ-ಗ-ಳನ್ನು
ಇಲಾಖೆ-ಗ-ಳಿಗೆ
ಇಲಾಖೆಗೆ
ಇಲಾಖೆಯ
ಇಲಾಖೆ-ಯನ್ನು
ಇಲಾಖೆ-ಯಲ್ಲಿ
ಇಲಾಖೆಯು
ಇಲಿಂದ
ಇಲ್ಲ
ಇಲ್ಲದ
ಇಲ್ಲ-ದಿದ್ದರೂ
ಇಲ್ಲ-ದಿದ್ದರೆ
ಇಲ್ಲ-ದಿದ್ದಲ್ಲಿ
ಇಲ್ಲ-ದಿದ್ದಾಗ
ಇಲ್ಲ-ದಿ-ರಲು
ಇಲ್ಲ-ದಿರುವ
ಇಲ್ಲ-ದಿರು-ವಂಶ
ಇಲ್ಲ-ದಿ-ರು-ವು-ದನ್ನು
ಇಲ್ಲ-ದಿ-ರುವು-ದರ
ಇಲ್ಲ-ದಿ-ರುವುದು
ಇಲ್ಲ-ದಿಲ್ಲ
ಇಲ್ಲದೆ
ಇಲ್ಲದೇ
ಇಲ್ಲವೇ
ಇಲ್ಲವೋ
ಇಲ್ಲಾ
ಇಲ್ಲಾ-ದೆರೆ
ಇಲ್ಲಿ
ಇಲ್ಲಿಂದ
ಇಲ್ಲಿಂದಾಚೆಗೆ
ಇಲ್ಲಿಗೆ
ಇಲ್ಲಿದೆ
ಇಲ್ಲಿದ್ದ
ಇಲ್ಲಿದ್ದ-ನೆಂದು
ಇಲ್ಲಿದ್ದರು
ಇಲ್ಲಿದ್ದು
ಇಲ್ಲಿನ
ಇಲ್ಲಿಯ
ಇಲ್ಲಿಯೂ
ಇಲ್ಲಿಯೇ
ಇಲ್ಲಿ-ರುವ
ಇಲ್ಲಿ-ರು-ವಾಗಲೇ
ಇಲ್ಲಿಲ್ಲ
ಇಲ್ಲಿವೆ
ಇಲ್ಲೂ
ಇಲ್ಲೇ
ಇಲ್ಲೊಂದು
ಇಳಕಲ್
ಇಳಿ-ಜಾ-ರಿಗೆ
ಇಳಿಜಾರಿನ
ಇಳಿಜಾರಿ-ನಲ್ಲೂ
ಇಳಿದು-ಕೊಂಡಿದ್ದ
ಇಳಿ-ಮುಖ
ಇಳಿ-ಮುಖ-ವಾಗಿ
ಇಳಿ-ಯಿತು
ಇಳಿಯಿ-ತೆಂದು
ಇಳಿ-ಯುವ
ಇಳೆಯ
ಇಳೆಯಾಂಡ
ಇಳೈಯಭಿರಾನ್
ಇಳೈಯಭಿರಾನ್ಭಟ್ಟನ
ಇಳೈ-ಯಾಳ್ವಾನ್
ಇವತ್ತಿಗೂ
ಇವನ
ಇವ-ನದೇ
ಇವನ-ನಿಗೆ
ಇವ-ನನ್ನು
ಇವ-ನಿಂದ
ಇವ-ನಿಂದಲೇ
ಇವ-ನಿಗೂ
ಇವ-ನಿಗೆ
ಇವ-ನಿಗೇ
ಇವ-ನಿರ-ಬೇಕೆಂದು
ಇವನು
ಇವನು-ಮ-ರಿಯಾನೆ
ಇವನೂ
ಇವ-ನೆಂದು
ಇವನೇ
ಇವ-ನೊಬ್ಬ
ಇವರ
ಇವರ-ಗಳು
ಇವ-ರದು
ಇವರ-ದೊಂದು
ಇವರದ್ದೂ
ಇವ-ರನ್ನು
ಇವ-ರನ್ನೂ
ಇವ-ರನ್ನೆಲಾ
ಇವರನ್ನೆಲ್ಲಾ
ಇವ-ರನ್ನೇ
ಇವ-ರಲ್ಲಿ
ಇವ-ರಲ್ಲೂ
ಇವ-ರಲ್ಲೇ
ಇವರ-ವನ್ನು
ಇವ-ರಾದ-ಮೇಲೆ
ಇವರಾರೂ
ಇವ-ರಿಂದ
ಇವ-ರಿಗಿಲ್ಲ-ದಿ-ರುವುದು
ಇವ-ರಿಗೂ
ಇವ-ರಿಗೆ
ಇವ-ರಿಗೆಲ್ಲಾ
ಇವ-ರಿಗೇ
ಇವರಿಬ್ಬರ
ಇವರಿಬ್ಬ-ರನ್ನೂ
ಇವರಿಬ್ಬ-ರಲ್ಲಿ
ಇವರಿಬ್ಬ-ರಿಗೂ
ಇವರಿಬ್ಬ-ರಿಗೆ
ಇವರಿಬ್ಬರು
ಇವ-ರಿಬ್ಬರೂ
ಇವರು
ಇವರು-ಗಳ
ಇವರು-ಗ-ಳನ್ನು
ಇವರು-ಗಳನ್ನೊಳ-ಗೊಂಡ
ಇವರು-ಗ-ಳಿಗೆ
ಇವರು-ಗಳು
ಇವರು-ಗಳೂ
ಇವರೂ
ಇವ-ರೆಲ್ಲ-ರಿಗೂ
ಇವ-ರೆಲ್ಲರೂ
ಇವ-ರೆಲ್ಲಾ
ಇವರೇ
ಇವ-ರೊಡ-ಗೂಡಿ
ಇವ-ರೊಳಗಾದ
ಇವಳ
ಇವ-ಳನ್ನು
ಇವ-ಳಿಂದ
ಇವ-ಳಿಗೆ
ಇವಳು
ಇವು
ಇವು-ಗಳ
ಇವು-ಗಳನೂ
ಇವು-ಗ-ಳನ್ನ
ಇವು-ಗ-ಳನ್ನು
ಇವು-ಗಳಲ್ಲದೆ
ಇವು-ಗ-ಳಲ್ಲಿ
ಇವು-ಗ-ಳಲ್ಲೂ
ಇವು-ಗ-ಳಿಂದ
ಇವು-ಗಳಿ-ಗಾಗಿ
ಇವು-ಗಳಿಗೂ
ಇವು-ಗ-ಳಿಗೆ
ಇವು-ಗಳು
ಇವು-ಗಳೆಲ್ಲ-ವನ್ನೂ
ಇವು-ಗಳೆಲ್ಲವೂ
ಇವು-ಗಳೆಲ್ಲಾ
ಇವು-ಗಳೇ
ಇವೆ
ಇವೆ-ರಡಕ್ಕೂ
ಇವೆ-ರಡನ್ನೂ
ಇವೆ-ರಡರ
ಇವೆ-ರಡು
ಇವೆ-ರಡೂ
ಇವೆ-ರಡೇ
ಇವೆಲ್ಲ
ಇವೆಲ್ಲ-ವನ್ನೂ
ಇವೆಲ್ಲವೂ
ಇವೆಲ್ಲಾ
ಇವೇ
ಇಶಾ-ಮುದ್ರ
ಇಷ್ಟ-ದೈವ-ಲಿಂಗದ
ಇಷ್ಟ-ಲಿಂಗ-ಗಳ
ಇಷ್ಟಲ್ಲದೆ
ಇಷ್ಟಾ-ದರೂ
ಇಷ್ಟು
ಇಷ್ಟೂ
ಇಷ್ಟೊಂದು
ಇಸವಿ
ಇಸವಿಯ
ಇಸಿ
ಇಸ್ಲಾಂ
ಇಸ್ಲಾಂಧರ್ಮ
ಇಸ್ಲಾಂನ
ಇಹ
ಇಹಪ-ಗೆಯಾಂಡ
ಈ
ಈಕೆ
ಈಕೆಗೆ
ಈಕೆಯ
ಈಕೆಯು
ಈಕೆಯೂ
ಈಕೆಯೇ
ಈಗ
ಈಗಲೂ
ಈಗಾ-ಗಲೇ
ಈಗಿನ
ಈಗಿ-ನಂತೆ
ಈಗ್ಗೆ
ಈಚಿನ
ಈಚೆಗೆ
ಈಡೇ-ರಿದು-ದ-ರಿಂದ
ಈತ
ಈತನ
ಈತ-ನನ್ನು
ಈತ-ನಾಗಿದ್ದಾ-ನೆಂದು
ಈತ-ನಿಗೆ
ಈತನು
ಈತನೂ
ಈತನೇ
ಈತನೇ-ನಾ-ದರೂ
ಈತನೇ-ನಾನ-ದರೂ
ಈರಿದು
ಈರೀತಿ
ಈರೇ-ಗೌಡ
ಈರೋಡು
ಈವರೆಗೆ
ಈಶಾನ್ಯ
ಈಶಾನ್ಯಕ್ಕೆ
ಈಶಾನ್ಯದ
ಈಶಾನ್ಯ-ದಲ್ಲಿ
ಈಶ್ವರ
ಈಶ್ವರ-ದೇವ
ಈಶ್ವರ-ದೇವ-ಸಾ-ನದ
ಈಶ್ವರ-ದೇವಾ-ಲಯ
ಈಶ್ವರ-ದೇವಾ-ಲಯ-ಗಳಿವೆ
ಈಶ್ವರನ
ಈಶ್ವರ-ನನ್ನು
ಈಶ್ವರನೇ
ಈಶ್ವರ-ಪೆದ್ದಿ
ಈಶ್ವರ-ಭಕ್ತ-ನಾ-ದರೂ
ಈಶ್ವರಯ್ಯಈ-ಸರಯ್ಯ
ಈಶ್ವರಯ್ಯನ
ಈಶ್ವರಯ್ಯನೂ
ಈಶ್ವರ-ಸಂವತ್ಸರ-ದಲ್ಲಿ
ಈಶ್ವರಾಂಕ
ಈಶ್ವರಾರ್ಪಿತ-ವಾಗಿ
ಈಸರ-ಗಂಡನ
ಈಸರ-ಗಂಡ-ನಿಗೆ
ಈಸರ-ಗಂಡನು
ಈಸರಯ್ಯ-ಈಶ್ವರಯ್ಯ
ಈಸರಯ್ಯನ
ಈಸರಯ್ಯನೆಂಬ
ಈಸರಯ್ಯನೇ-ಈಶ್ವರಯ್ಯ
ಈಸಿ-ಕೊಂಡು
ಉಂಟಾಗಲು
ಉಂಟಾಗಿದೆ
ಉಂಟಾ-ಗಿದ್ದ
ಉಂಟಾ-ಗಿದ್ದು
ಉಂಟಾ-ಗಿ-ರುವ
ಉಂಟಾಗುತ್ತದೆ
ಉಂಟಾದ
ಉಂಟಾ-ದರೂ
ಉಂಟಾ-ದರೆ
ಉಂಟಾ-ದಾಗ
ಉಂಟಾ-ಯಿತು
ಉಂಟಾಯಿ-ತೆಂದು
ಉಂಟು
ಉಂಟು-ಮಾಡುತ್ತವೆ
ಉಂಡಿಗ-ನ-ಹಾಳು
ಉಂಡಿಗೆ
ಉಂಡಿಗೆಯ
ಉಂಡಿಗೆ-ಯಾಗಿ
ಉಂಡೆಗೆ
ಉಂಬಳಿ
ಉಂಬಳಿ-ಗ-ಳನ್ನು
ಉಂಬಳಿ-ಗ-ಳನ್ನೂ
ಉಂಬಳಿ-ಗಳಿ-ಗಾಗಿ
ಉಂಬಳಿ-ಗಳು
ಉಂಬಳಿ-ಗಳೂ
ಉಂಬಳಿಯ
ಉಂಬಳಿ-ಯನ್ನು
ಉಂಬಳಿ-ಯಾಗಿ
ಉಂಮರ-ಹಳ್ಳಿ
ಉಂಮರ-ಹಳ್ಳಿ-ಉಮ್ಮಡ-ಹಳ್ಳಿ
ಉಕ್ಕಿನ
ಉಕ್ತ-ನಾ-ಗಿದ್ದು
ಉಕ್ತ-ನಾಗಿ-ರುವ
ಉಕ್ತ-ನಾದ
ಉಕ್ತ-ರಾಗಿ-ರುವ
ಉಕ್ತ-ರಾದ
ಉಕ್ತ-ವಾಗಿದೆ
ಉಕ್ತ-ವಾಗಿ-ರುವ
ಉಕ್ತ-ವಾಗಿಲ್ಲ
ಉಕ್ತ-ವಾದ
ಉಕ್ತ-ವಾನ್ಮು
ಉಕ್ಥ್ಯಯಾಜಿ-ಯಾದ
ಉಗತ್ತಿ
ಉಗಮಕ್ಕೆ
ಉಗ-ಮದ
ಉಗಮ-ವಾಗಿ
ಉಗುರ-ಕೆರೆ
ಉಗುರು
ಉಗುರು-ಮೂ-ನೂರ್ವ್ವರು
ಉಗ್ರ-ಭಕ್ತರು
ಉಗ್ರಾಣ-ಪಾಲ-ಕಸ್ಟೋರ್ಕೀ-ಪರ್ಇ-ವರು-ಗಳ
ಉಚಿತ-ವಾಗಿದೆ
ಉಚ್ಚಂಗಿ
ಉಚ್ಚಂಗಿ-ಗ-ಳನ್ನು
ಉಚ್ಚರಿ-ಸುತ್ತಿ-ದರು
ಉಚ್ಛಂಗಿ
ಉಚ್ಛಂಗಿಯೋ
ಉಚ್ಛರಿಸುತ್ತಾರೆ
ಉಚ್ಛಾಟನೆ-ಗೊಂಡಂತೆ
ಉಚ್ಛಾರಣಾ
ಉಚ್ಛ್ರಾಯ
ಉಚ್ಛ್ರಾಯದ
ಉಜ್ಜ-ಯನಿ-ಯಲ್ಲಿ
ಉಜ್ಜೈನಿಯ
ಉಟ್ಟಿದ್ದಂತಹ
ಉಡ-ಹಳ್ಳಿಯ
ಉಡಿಯ-ಗಾಲ
ಉಡುಗೊರೆ
ಉಡುಗೊರೆ-ಗ-ಳನ್ನು
ಉಡುಗೊರೆ-ಯಾಗಿ
ಉಡುಪು
ಉಡುವ
ಉಡುವಂಕ-ನಾಡ
ಉಡೈಯ
ಉಡೈಯ-ಪಿಳ್ಳೆ
ಉಡೈಯ-ಪಿಳ್ಳೈಯು
ಉಡೈ-ಯರ್
ಉಡೈಯ-ವರ್
ಉಡೈ-ಯಾರ್
ಉಣ್ಣಿಚ್
ಉಣ್ಬೊ
ಉಣ್ಬೋದು
ಉತ್ಕರ್ಷಸ್ಥಿತಿ
ಉತ್ಕೃಷ್ಟ
ಉತ್ತಮ
ಉತ್ತಮ-ಚೋಳ
ಉತ್ತಮ-ನಂಬಿಯು
ಉತ್ತಮ-ಮಟ್ಟದ್ದು
ಉತ್ತಮ-ವಾದ
ಉತ್ತರ
ಉತ್ತರ-ಕರ್ನಾಟ-ಕ-ದ-ವರೋ
ಉತ್ತರ-ಕರ್ನಾಟದ
ಉತ್ತರಕ್ಕಿ-ರುವ
ಉತ್ತರಕ್ಕೂ
ಉತ್ತರಕ್ಕೆ
ಉತ್ತ-ರದ
ಉತ್ತರ-ದಲ್ಲಿ
ಉತ್ತರ-ದಿಂದ
ಉತ್ತರ-ದಿಕ್ಕಿನ
ಉತ್ತರ-ದಿಕ್ಕಿನಲ್ಲಿ-ರುವ
ಉತ್ತರ-ದೇಶ-ದಲ್ಲಿ
ಉತ್ತರ-ಪಾರ್ಶ್ವ-ದಲ್ಲಿದ್ದ
ಉತ್ತರ-ಭಾಗ-ಗ-ಳನ್ನು
ಉತ್ತರ-ಭಾಗ-ಗಳು
ಉತ್ತರ-ಭಾಗ-ದಲ್ಲಿ
ಉತ್ತರ-ಭಾಗ-ದಲ್ಲಿಯೂ
ಉತ್ತರ-ಭಾಗ-ವನ್ನು
ಉತ್ತರ-ಭಾರ-ತದ
ಉತ್ತರಾಧಿ-ಕಾರಕ್ಕೆ
ಉತ್ತರಾಧಿ-ಕಾರತ್ವವು
ಉತ್ತರಾಧಿ-ಕಾರಿ
ಉತ್ತರಾಧಿ-ಕಾರಿಯ
ಉತ್ತರಾಧಿ-ಕಾರಿ-ಯಾದ
ಉತ್ತರಾ-ಪಥ-ದಿಂದ
ಉತ್ತರಾ-ಫಲ್ಗುಣಿ
ಉತ್ತ-ರಾರ್ಧ-ದಲ್ಲಿ
ಉತ್ತ-ರಾರ್ಧ-ದಲ್ಲಿಯೇ
ಉತ್ತರಿ-ಸುತ್ತಾ
ಉತ್ತರೇ
ಉತ್ತುಂಗ
ಉತ್ತುಮ
ಉತ್ಪತ್ತಿ-ಯನ್ನು
ಉತ್ಪನ್ನ
ಉತ್ಪನ್ನದ
ಉತ್ಪನ್ನ-ದಲ್ಲಿ
ಉತ್ಪಾದಕ
ಉತ್ಪಾದ-ಕರು
ಉತ್ಪಾದ-ನ-ವಿನಿ-ಮಯ
ಉತ್ಪಾದನೆ
ಉತ್ಪಾದ-ನೆ-ವಿನಿ-ಮಯ
ಉತ್ಪ್ರೇಕ್ಷೆ
ಉತ್ಪ್ರೇಕ್ಷೆ-ಯಲ್ಲ
ಉತ್ಪ್ರೇಕ್ಷೆ-ಯಿಂದ
ಉತ್ಸವ
ಉತ್ಸ-ವಕ್ಕೆ
ಉತ್ಸವ-ಗಳಂ
ಉತ್ಸವ-ಗ-ಳನ್ನು
ಉತ್ಸವ-ಗ-ಳಲ್ಲಿ
ಉತ್ಸವ-ಗ-ಳಿಗೆ
ಉತ್ಸವ-ಗಳು
ಉತ್ಸವದ
ಉತ್ಸವ-ಮಂಟಪ
ಉತ್ಸವ-ಮೂರ್ತಿಯೂ
ಉತ್ಸ-ವರು
ಉತ್ಸವವು
ಉತ್ಸವ-ವೆಂದು
ಉತ್ಸಾಹ-ದಿಂದ
ಉತ್ಸಾಹಿಯೂ
ಉದಕ
ಉದಕೆ
ಉದಯ-ಕು-ಮಾರ
ಉದಯ-ಗಿರಿ
ಉದಯ-ಗಿರಿಯ
ಉದಯ-ಗಿರಿ-ಯಲ್ಲಿ
ಉದಯ-ಚಂದ್ರ
ಉದಯ-ಮಯ್ಯ
ಉದಯ-ವಾದ
ಉದಯ-ವಾಯಿ-ತೆಂದು
ಉದ-ಯಾದಿ-ಕಾಲತ್ರಯಾರಾ-ಧನೆ
ಉದ-ಯಾ-ದಿತ್ಯ
ಉದ-ಯಾ-ದಿತ್ಯನ
ಉದ-ಯಾ-ದಿತ್ಯರು
ಉದ-ಯಾ-ದಿತ್ಯರ್
ಉದಯಿ-ಸಿತು
ಉದಯಿ-ಸಿದ
ಉದಾಗೆ
ಉದಾತ್ತ-ವಾದ
ಉದಾರ
ಉದಾರ-ವಾಗಿ
ಉದಾರ-ವಾರಿ-ನಿಧಿ
ಉದಾರಿಯೂ
ಉದಾ-ಹಣೆ-ಗಳಿವೆ
ಉದಾಹ-ರಣೆ
ಉದಾಹ-ರಣೆ-ಗಳಿವೆ
ಉದಾಹ-ರಣೆ-ಗಳು
ಉದಾಹ-ರಣೆ-ಗಳೂ
ಉದಾಹ-ರಣೆಗೆ
ಉದಾಹ-ರಣೆ-ಯನ್ನಾಗಿ
ಉದಾಹ-ರಣೆ-ಯನ್ನು
ಉದಾಹ-ರಣೆ-ಯನ್ನೂ
ಉದಾಹ-ರಣೆ-ಯಾಗಿ
ಉದಾಹ-ರಣೆ-ಯಾಗಿದೆ
ಉದಾಹ-ರಣೆ-ಯಾ-ಗಿದ್ದು
ಉದಾಹ-ರಣೆ-ಯಿಂದಲೇ
ಉದಾಹ-ರಣೆಯೂ
ಉದಾ-ಹರಿ-ಸಿದ್ದಾರೆ
ಉದಿತೋದಿತ
ಉದಿ-ಸಿದ
ಉದೆ-ಯಕೊ-ಮಾರ-ನಾಯ-ಕನು
ಉದ್ಗರಿ-ಸು-ವಂತೆ
ಉದ್ಗಾರ
ಉದ್ಗಾರ-ವೆತ್ತಿದೆ
ಉದ್ಘಾಟ-ನೆಗೆ
ಉದ್ದ
ಉದ್ದಂಡ
ಉದ್ದಕ್ಕೂ
ಉದ್ದ-ಗ-ಲಕ್ಕೂ
ಉದ್ದ-ಗಲ-ಗ-ಳನ್ನು
ಉದ್ದಾಮ
ಉದ್ದಿಪಾಂಗಾಳ
ಉದ್ದೇಶ
ಉದ್ದೇಶ-ಗ-ಳಿಗೆ
ಉದ್ದೇಶ-ದಿಂದ
ಉದ್ದೇಶವೂ
ಉದ್ದೇಶಿ-ಸಿದ್ದ
ಉದ್ದೇಶಿ-ಸಿದ್ದು
ಉದ್ದೇಶಿ-ಸಿದ್ದೇನೆ
ಉದ್ಧೃತ-ವಾಗಿ-ರುವುದು
ಉದ್ಭವ
ಉದ್ಭವ-ನರ-ಸಿಂಹ-ಪುರ-ವಾದ
ಉದ್ಭವ-ನರ-ಸಿಂಹ-ಪುರ-ವೆಂಬ
ಉದ್ಭವ-ವಿಶ್ವ-ನಾಥ-ಪುರ-ಬಾಳಗಂಚಿ
ಉದ್ಭವ-ಸರ್ವಜ್ಞ
ಉದ್ಭವ-ಸರ್ವಜ್ಞ-ಪುರದ
ಉದ್ಭವ-ಸರ್ವಜ್ಞ-ಪುರ-ವಾದ
ಉದ್ಭವಿಸಲಾ-ರದು
ಉದ್ಯ-ಮ-ಗ-ಳಿಗೆ
ಉದ್ಯೋಗ
ಉದ್ಯೋಗ-ಗಳ
ಉದ್ಯೋಗ-ದಲ್ಲಿದ್ದ
ಉದ್ಯೋಗ-ಮಲ್ಲ-ನೆನಿ-ಸಿ-ದನು
ಉಧಾರಣ
ಉನ್
ಉನ್ನತ
ಉನ್ನತ-ಅಧಿ-ಕಾರ-ವರ್ಗದ
ಉನ್ನತ-ದರ್ಜೆಯ
ಉನ್ನತ-ದರ್ಜೆ-ಯಲ್ಲಿ
ಉನ್ನತ-ಮಟ್ಟದ
ಉನ್ನತ-ವಾದ
ಉನ್ನತಸ್ಥರದ
ಉನ್ನತಿ-ಗೇರಿ
ಉಪ
ಉಪ-ಕರ-ಣ-ಗ-ಳನ್ನು
ಉಪ-ಕರ-ಣ-ಗ-ಳನ್ನೂ
ಉಪ-ಕರ-ಣ-ಗಳು
ಉಪ-ಕಾರದ
ಉಪಕ್ರಮಿ-ಸಿರ-ಬಹು-ದೆಂದು
ಉಪಗ್ರಾಮ
ಉಪಗ್ರಾಮ-ಗಳ
ಉಪಗ್ರಾಮ-ಗಳನ್ನಾಗಿ
ಉಪಗ್ರಾಮ-ಗ-ಳನ್ನು
ಉಪಗ್ರಾಮ-ಗ-ಳಾಗಿ
ಉಪಗ್ರಾಮ-ಗ-ಳಾಗಿದ್ದವು
ಉಪಗ್ರಾಮ-ಗ-ಳಾಗಿದ್ದ-ವೆಂದು
ಉಪಗ್ರಾಮ-ಗ-ಳಾದ
ಉಪಗ್ರಾಮ-ಗಳು
ಉಪಗ್ರಾಮದ
ಉಪಗ್ರಾಮ-ವನ್ನು
ಉಪಗ್ರಾಮ-ವಾದ
ಉಪಟಳ
ಉಪಟಳವೂ
ಉಪ-ತಾಲ್ಲೂ-ಕನ್ನು
ಉಪದೇಶ
ಉಪದ್ರವ-ದಿಂದ
ಉಪ-ನದಿ-ಗ-ಳನ್ನು
ಉಪ-ನದಿ-ಗ-ಳಲ್ಲಿ
ಉಪ-ನದಿ-ಗ-ಳಾದ
ಉಪ-ನದಿ-ಗಳು
ಉಪ-ನ-ಯನ
ಉಪ-ನಾಡು-ಗ-ಳಾಗಿದ್ದ-ವೆಂದು
ಉಪ-ನಾಮ-ಗಳಿದ್ದವು
ಉಪ-ನಾಮ-ವನ್ನು
ಉಪ-ನಾಮ-ವಿರುತ್ತದೆ
ಉಪ-ನಾಮವು
ಉಪನ್ಯಾ-ಸ-ಕರು-ಗಳು
ಉಪಪಂಗಡ-ವಿದೆ
ಉಪಪಂಗಡ-ವಿದ್ದು
ಉಪ-ಭೋಗಕ್ಕಾಗಿ
ಉಪ-ಭೋಗಿ-ಸಲು
ಉಪ-ಯುಕ್ತ
ಉಪ-ಯೋಗ
ಉಪ-ಯೋಗಕ್ಕಾಗಿ
ಉಪ-ಯೋ-ಗಕ್ಕೆ
ಉಪ-ಯೋಗ-ವಾಗಲಿ
ಉಪ-ಯೋಗ-ವಾಗ-ಲೆಂದು
ಉಪ-ಯೋಗ-ವಾಗಿದೆ
ಉಪ-ಯೋಗ-ವಿಲ್ಲ
ಉಪ-ಯೋಗಿ-ಸ-ಬ-ಹುದು
ಉಪ-ಯೋಗಿ-ಸ-ಬೇಕೆಂದು
ಉಪ-ಯೋಗಿ-ಸ-ಲಾಗಿದೆ
ಉಪ-ಯೋಗಿಸಿ
ಉಪ-ಯೋಗಿ-ಸಿ-ಕೊಂಡು
ಉಪ-ಯೋಗಿ-ಸಿ-ಕೊಳ್ಳಲು
ಉಪ-ಯೋಗಿ-ಸಿ-ಕೊಳ್ಳುತ್ತಿ-ದರು
ಉಪ-ಯೋಗಿ-ಸಿ-ಕೊಳ್ಳು-ವಂತೆ
ಉಪ-ಯೋಗಿ-ಸಿ-ಕೊಳ್ಳು-ವು-ದಕ್ಕೆ
ಉಪ-ಯೋಗಿ-ಸಿ-ಕೊಳ್ಳು-ವು-ದನ್ನು
ಉಪ-ಯೋಗಿ-ಸಿದ್ದಾರೆ
ಉಪ-ಯೋಗಿ-ಸಿ-ರುವ
ಉಪ-ಯೋಗಿ-ಸಿ-ರುವು-ದ-ರಿಂದ
ಉಪ-ಯೋಗಿ-ಸಿ-ರುವುದು
ಉಪ-ಯೋಗಿಸು
ಉಪ-ಯೋಗಿ-ಸುತ್ತಿದ್ದ
ಉಪ-ಯೋಗಿ-ಸುತ್ತಿದ್ದರು
ಉಪ-ಯೋಗಿ-ಸುತ್ತಿದ್ದ-ರೆಂದು
ಉಪ-ಯೋಗಿ-ಸುತ್ತಿದ್ದುದೂ
ಉಪ-ಯೋಗಿ-ಸುವ
ಉಪ-ಯೋಗಿ-ಸು-ವ-ವರು
ಉಪ-ವಾಸ
ಉಪ-ವಿ-ಭಾಗ
ಉಪ-ವಿ-ಭಾ-ಗಕ್ಕೆ
ಉಪ-ವಿ-ಭಾಗ-ಗ-ಳನ್ನಾಗಿ
ಉಪ-ವಿ-ಭಾಗ-ಗಳಿದ್ದವು
ಉಪ-ವಿ-ಭಾಗ-ಗಳಿದ್ದು
ಉಪ-ವಿ-ಭಾಗ-ದಲ್ಲಿ
ಉಪವಿ-ಭಾಗ-ವಿದ್ದು
ಉಪ-ಶಾಂತ-ವಾಯಿತು
ಉಪಸ್ಥಿತಿ
ಉಪ-ಹಾರಕ್ಕೆ
ಉಪಾ-ದಾ-ನಾರ್ಥ-ವಾಗಿ
ಉಪಾಧ್ಯಾಯ-ನೆಂಬು-ವವ-ನಿಗೆ
ಉಪಾಧ್ಯಾಯ-ರಾಗಿ-ರ-ಬ-ಹುದು
ಉಪಾಧ್ಯಾ-ಯರು
ಉಪಾಧ್ಯಾ-ಯರು-ಗಳು
ಉಪಾ-ರಕೆ
ಉಪಾಸ-ಕಳಾ-ಗಿದ್ದಳು
ಉಪಾ-ಹಾರ
ಉಪ್ಪಾ-ರರ
ಉಪ್ಪಿನ
ಉಪ್ಪಿನ-ಕಾಯಿ
ಉಪ್ಪು
ಉಪ್ಪು-ಗ-ಳನ್ನು
ಉಪ್ಪು-ವಳ್ಳ
ಉಬ್ಬು
ಉಬ್ಬು-ಶಿಲ್ಪ-ಗ-ಳನ್ನು
ಉಬ್ಬು-ಶಿಲ್ಪ-ಗ-ಳಿಂದ
ಉಬ್ಬು-ಶಿಲ್ಪ-ಗಳಿವೆ
ಉಬ್ಬು-ಶಿಲ್ಪ-ವಾಗಿ-ರ-ಬ-ಹುದು
ಉಬ್ಬು-ಶಿಲ್ಪ-ವಿದೆ
ಉಬ್ಬು-ಶಿಲ್ಪ-ವಿ-ರುವ
ಉಬ್ಬು-ಶಿಲ್ಪ-ವಿ-ರುವುದು
ಉಭಯ
ಉಭಯ-ಕಾವೇರಿ
ಉಭಯ-ಕಾವೇರಿಯ
ಉಭಯತ್ರರೂ
ಉಭಯ-ದೇಶಿ
ಉಭಯ-ದೇಸಿ
ಉಭಯ-ನಾಚ್ಚಿ-ಯಾ-ರರು-ಗ-ಳಿಗೆ
ಉಭಯ-ನಾನಾ-ದೇಸಿ-ಗರು
ಉಭಯ-ಬಲ-ಸುಭಟ
ಉಭಯ-ಬಳ-ಸುಭಟ-ನು-ಮಪ್ಪ
ಉಭಯ-ಭಾಷಾ
ಉಭಯ-ರಾಯ
ಉಭಯ-ವೆಂದರೆ
ಉಭಯ-ವೇದಾಂತಾ-ಚಾರ್ಯ
ಉಭಯಾನು-ಮತ-ದಿಂದ
ಉಭಯಾನ್ವಯ
ಉಮೆ-ಯಕ್ಕನೆಂದೂ
ಉಮೆಯ-ನೊಡೆಯಂ
ಉಮೆಯಾಂಡೆಯು
ಉಮ್ಮಡ-ಹಳ್ಳಿ
ಉಮ್ಮತೂರ
ಉಮ್ಮತ್ತೂರ
ಉಮ್ಮತ್ತೂರನ್ನು
ಉಮ್ಮತ್ತೂರಿಗೆ
ಉಮ್ಮತ್ತೂರಿನ
ಉಮ್ಮತ್ತೂರಿನಿಂದ
ಉಮ್ಮತ್ತೂರಿನಿಂದಲೂ
ಉಮ್ಮತ್ತೂರು
ಉಮ್ಮತ್ತೂರು-ಗ-ಳನ್ನು
ಉಯ-ಕೊಂಡ
ಉಯ-ಕೊಂಡ-ಪಿಳ್ಳೆ
ಉಯಿಲು
ಉಯ್ಯ-ಕೊಂಡ-ಪಿಳ್ಳೆ
ಉಯ್ಯ-ಕೊಂಡಾ-ನುಮ್
ಉಯ್ಯಕೊಣ್ಡ-ಭಟ್ಟನ್
ಉಯ್ಯಕೊಣ್ದ
ಉಯ್ಯಲ-ನೇರು
ಉರದಿ-ದಿರಾನ್ತ
ಉರವ-ಣೆ-ಯನ್ನು
ಉರವ-ಣೆ-ಯಿಂದ
ಉರಿಯುವ
ಉರು
ಉರು-ಳಿಸಿ
ಉರು-ಸನ್ನು
ಉರು-ಸಾಲ
ಉರೆಗಂ
ಉರೋದ್ಗು-ಕರತೆ
ಉರ್ಕಣೆ
ಉರ್ದುವಿ-ನಲ್ಲಿ
ಉರ್ದುಶಾಸ-ದಿಂದ
ಉರ್ವೀ-ಧರ
ಉಲಗಮುಣ್ಢಾನ್
ಉಲ-ಗಾ-ಮುಂಡನ
ಉಲ್
ಉಲ್ಲೆಖ
ಉಲ್ಲೆಖ-ವಿದೆ
ಉಲ್ಲೆಖ-ವಿ-ರು-ವು-ದಿಲ್ಲ
ಉಲ್ಲೆಖ-ವಿಲ್ಲ
ಉಲ್ಲೆಖಿ-ಸಿದ
ಉಲ್ಲೆಖಿ-ಸುತ್ತದೆ
ಉಲ್ಲೇಖ
ಉಲ್ಲೇಖ-ಗ-ಳನ್ನು
ಉಲ್ಲೇಖ-ಗ-ಳನ್ನೂ
ಉಲ್ಲೇಖ-ಗ-ಳಲ್ಲಿ
ಉಲ್ಲೇಖ-ಗ-ಳಿಂದ
ಉಲ್ಲೇಖ-ಗಳಿದ್ದು
ಉಲ್ಲೇಖ-ಗಳಿವೆ
ಉಲ್ಲೇಖ-ಗಳು
ಉಲ್ಲೇಖ-ಗಳುಳ್ಳ
ಉಲ್ಲೇಖ-ಗಳೂ
ಉಲ್ಲೇಖ-ಗೊಂಡಿದೆ
ಉಲ್ಲೇಖ-ಗೊಂಡಿ-ರುವ
ಉಲ್ಲೇಖ-ಗೊಳ್ಳುವ
ಉಲ್ಲೇಖದ
ಉಲ್ಲೇಖ-ದಿಂದ
ಉಲ್ಲೇಖದೆ
ಉಲ್ಲೇಖ-ದೊಡನೆ
ಉಲ್ಲೇಖ-ನ-ಗಳಿವೆ
ಉಲ್ಲೇಖ-ನಾರ್ಹ
ಉಲ್ಲೇಖ-ವನ್ನು
ಉಲ್ಲೇಖ-ವಾಗಲೀ
ಉಲ್ಲೇಖ-ವಾಗಿದೆ
ಉಲ್ಲೇಖ-ವಾಗಿದ್ದರೂ
ಉಲ್ಲೇಖ-ವಾಗಿ-ರುತ್ತಾರೆ
ಉಲ್ಲೇಖ-ವಾಗಿ-ರುವ
ಉಲ್ಲೇಖ-ವಾಗಿ-ರು-ವು-ದನ್ನು
ಉಲ್ಲೇಖ-ವಾಗಿಲ್ಲ
ಉಲ್ಲೇಖ-ವಾಗಿವೆ
ಉಲ್ಲೇಖ-ವಾದ
ಉಲ್ಲೇಖ-ವಿದ
ಉಲ್ಲೇಖ-ವಿದೆ
ಉಲ್ಲೇಖ-ವಿದೆ-ಯೆಂದು
ಉಲ್ಲೇಖ-ವಿದೆ-ಯೆಂದೂ
ಉಲ್ಲೇಖ-ವಿದೆಯೇ
ಉಲ್ಲೇಖ-ವಿದ್ದಲ್ಲಿ
ಉಲ್ಲೇಖ-ವಿದ್ದು
ಉಲ್ಲೇಖ-ವಿರ-ಬಹು-ದೆಂದು
ಉಲ್ಲೇಖ-ವಿ-ರುವ
ಉಲ್ಲೇಖ-ವಿ-ರುವಿ-ದಿಲ್ಲ
ಉಲ್ಲೇಖ-ವಿ-ರುವು-ದ-ರಿಂದ
ಉಲ್ಲೇಖ-ವಿ-ರು-ವು-ದಿಲ್ಲ
ಉಲ್ಲೇಖ-ವಿ-ರುವುದು
ಉಲ್ಲೇಖ-ವಿಲ್ಲ
ಉಲ್ಲೇಖ-ವಿಲ್ಲದ
ಉಲ್ಲೇಖವು
ಉಲ್ಲೇಖ-ವುಳ್ಳ
ಉಲ್ಲೇಖವೂ
ಉಲ್ಲೇಖ-ವೆಂದು
ಉಲ್ಲೇಖವೇ
ಉಲ್ಲೇಖಾರ್ಹ
ಉಲ್ಲೇಖಾರ್ಹ-ವಾಗಿದೆ
ಉಲ್ಲೇಖಿತ
ಉಲ್ಲೇಖಿತ-ನಾಗಿದ್ದಾನೆ
ಉಲ್ಲೇಖಿತ-ನಾಗಿ-ರುವ
ಉಲ್ಲೇಖಿತ-ನಾದ
ಉಲ್ಲೇಖಿತ-ರಾಗಿದ್ದಾರೆ
ಉಲ್ಲೇಖಿತ-ರಾಗಿ-ರುವ
ಉಲ್ಲೇಖಿತ-ರಾದ
ಉಲ್ಲೇಖಿತ-ವಾಗಿದೆ
ಉಲ್ಲೇಖಿತ-ವಾಗಿವೆ
ಉಲ್ಲೇಖಿತ-ವಾದ
ಉಲ್ಲೇಖಿ-ಸದೇ
ಉಲ್ಲೇಖಿಸ-ಬಹದು
ಉಲ್ಲೇಖಿಸ-ಬ-ಹುದು
ಉಲ್ಲೇಖಿಸ-ಲಾಗಿದೆ
ಉಲ್ಲೇಖಿಸ-ಲಾ-ಗಿದ್ದು
ಉಲ್ಲೇಖಿಸ-ಲಾದ
ಉಲ್ಲೇಖಿಸಿ
ಉಲ್ಲೇಖಿಸಿದ
ಉಲ್ಲೇಖಿಸಿ-ದಂತೆ
ಉಲ್ಲೇಖಿಸಿದೆ
ಉಲ್ಲೇಖಿಸಿದ್ದರೂ
ಉಲ್ಲೇಖಿಸಿದ್ದಾರೆ
ಉಲ್ಲೇಖಿಸಿದ್ದು
ಉಲ್ಲೇಖಿಸಿದ್ದೇನೆ
ಉಲ್ಲೇಖಿಸಿ-ರ-ಬ-ಹುದು
ಉಲ್ಲೇಖಿಸಿ-ರ-ಬಹು-ದೆಂದು
ಉಲ್ಲೇಖಿಸಿ-ರುವ
ಉಲ್ಲೇಖಿಸಿ-ರು-ವು-ದಿಲ್ಲ
ಉಲ್ಲೇಖಿಸಿ-ರುವುದು
ಉಲ್ಲೇಖಿಸಿಲ್ಲ
ಉಲ್ಲೇಖಿಸಿವೆ
ಉಲ್ಲೇಖಿ-ಸುತ್ತದೆ
ಉಲ್ಲೇಖಿ-ಸುತ್ತವೆ
ಉಲ್ಲೇಖಿ-ಸುತ್ತಾ
ಉಲ್ಲೇಖಿ-ಸುತ್ತಾರೆ
ಉಲ್ಲೇಖಿ-ಸುವ
ಉಲ್ಲೇಖಿ-ಸುವಾಗ
ಉಳಿಂಜ-ಗೌಡ
ಉಳಿದ
ಉಳಿ-ದಂತೆ
ಉಳಿ-ದನು
ಉಳಿದ-ವರ
ಉಳಿ-ದವ-ರಲ್ಲಿ
ಉಳಿದ-ವ-ರಿಗೆ
ಉಳಿದ-ವರು
ಉಳಿದವು
ಉಳಿದಾಗ
ಉಳಿ-ದಿತ್ತು
ಉಳಿ-ದಿದೆ
ಉಳಿದಿದ್ದಂತೆ
ಉಳಿ-ದಿದ್ದು
ಉಳಿ-ದಿರುತ್ತಾನೆ
ಉಳಿ-ದಿರುವ
ಉಳಿ-ದಿಲ್ಲ
ಉಳಿ-ದಿವೆ
ಉಳಿದಿ-ವೆಯೇ
ಉಳಿದು
ಉಳಿದು-ಕೊಂಡಿದ್ದು
ಉಳಿದು-ಕೊಂಡು
ಉಳಿದು-ದನ್ನು
ಉಳಿದು-ದಲ್ಲದೆ
ಉಳಿ-ದುದು
ಉಳಿದು-ದೆಲ್ಲ
ಉಳಿದು-ದೆಲ್ಲವೂ
ಉಳಿದು-ಬಂದಿಲ್ಲ
ಉಳಿದು-ಬಿಟ್ಟ-ರಂತೆ
ಉಳಿದೆ-ರಡು
ಉಳಿ-ದೆಲ್ಲ
ಉಳಿ-ದೆಲ್ಲವೂ
ಉಳಿಯ
ಉಳಿ-ಯದೆ
ಉಳಿ-ಯಿತು
ಉಳಿಯುತ್ತದೆ
ಉಳಿಯುತ್ತಾರೆ
ಉಳಿಯು-ವಂತೆ
ಉಳಿ-ಸಲು
ಉಳಿಸಿ
ಉಳಿಸಿ-ಕೊಂಡನು
ಉಳಿಸಿ-ಕೊಂಡ-ನೆಂದು
ಉಳಿಸಿ-ಕೊಂಡಿತ್ತು
ಉಳಿಸಿ-ಕೊಂಡಿದೆ
ಉಳಿಸಿ-ಕೊಂಡಿವೆ
ಉಳಿಸಿ-ಕೊಂಡು
ಉಳಿಸಿ-ಕೊಳ್ಳ-ಲಾಗಿದೆ
ಉಳಿಸಿ-ಕೊಳ್ಳ-ಲಾ-ಯಿತು
ಉಳಿಸಿ-ಕೊಳ್ಳಲು
ಉಳಿಸುತ್ತಿದ್ದರು
ಉಳಿ-ಸುವುದಕ್ಕಾಗಿ
ಉಳುಮೆ
ಉಳುವ
ಉಳು-ವರ್ತಿ
ಉಳುವ-ವ-ರಿಗೆ
ಉಳ್ಳ
ಉಳ್ಳಂತಾ
ಉಸ್ತು-ವಾರಿ-ಯಲ್ಲಿದೆ
ಊಂಚ-ಹಳ್ಳಿ
ಊಚನ-ಹಳ್ಳಿಯ
ಊಟ
ಊಟದ
ಊಟ-ಮಾ-ಡುವ
ಊದುತ್ತಿದ್ದರು
ಊದುತ್ತಿ-ರುವ
ಊದುವ-ಕಾಳೆ
ಊರ
ಊರ-ಗಾವುಂಡ-ನಿರುತ್ತಿದ್ದನು
ಊರ-ಗಾವುಂಡನೇ
ಊರ-ಗಾವುಂಡ-ರನ್ನು
ಊರನ್ನಾಗಿ
ಊರನ್ನು
ಊರನ್ನೇ
ಊರ-ಮಧ್ಯೆ
ಊರ-ಮುಂದಣ
ಊರ-ಮುಂದೆ
ಊರ-ಳಿ-ವನ್ನು
ಊರ-ಳಿವಿ
ಊರ-ಳಿ-ವಿನ
ಊರ-ಳಿವಿ-ನಲ್ಲಿ
ಊರ-ಳಿವಿ-ನೊಳ್ಕಾದಿ
ಊರ-ಳಿ-ವಿಲ್ಲಿ
ಊರ-ಳಿವು
ಊರ-ಳಿವು-ಊರ
ಊರ-ಳಿವು-ಊ-ರಹುಯ್ಯಲು
ಊರವ-ರಿಂದ
ಊರ-ಸೇನ-ಬೋವ
ಊರಹುಯ್ಯಲು
ಊರಾಗಿ
ಊರಾ-ಗಿತ್ತು
ಊರಾಗಿದೆ
ಊರಾಗಿದ್ದ-ರಿಂದ
ಊರಾಗಿದ್ದರೂ
ಊರಾ-ಗಿದ್ದು
ಊರಾಗಿ-ರ-ಬ-ಹುದು
ಊರಾದ
ಊರಾಯಿ-ತೆಂದು
ಊರಿಂದ
ಊರಿಂದ-ಬಹ
ಊರಿಂದೂ-ರಿಗೆ
ಊರಿಗೆ
ಊರಿದೆ
ಊರಿದ್ದ
ಊರಿದ್ದು
ಊರಿನ
ಊರಿ-ನಲ್ಲಿ
ಊರಿ-ನಲ್ಲಿದೆ
ಊರಿ-ನಲ್ಲಿದ್ದ
ಊರಿ-ನಲ್ಲಿಯೂ
ಊರಿ-ನಲ್ಲಿ-ರುವ
ಊರಿ-ನಲ್ಲೇ
ಊರಿ-ನಲ್ಲೋ
ಊರಿನ-ವ-ನಾದ
ಊರಿನ-ವನೇ
ಊರಿನ-ವ-ರಾ-ಗಿದ್ದು
ಊರಿನ-ವ-ರಾಗಿ-ರ-ಬ-ಹುದು
ಊರಿನ-ವರು
ಊರಿನ-ವರೆಗೂ
ಊರಿನ-ವರೇ
ಊರಿ-ನಿಂದ
ಊರಿ-ನೊಳಗೆ
ಊರಿರ-ಬ-ಹುದು
ಊರು
ಊರುಂಬ
ಊರು-ಗಳ
ಊರು-ಗ-ಳನ್ನು
ಊರು-ಗ-ಳಲ್ಲಿ
ಊರು-ಗ-ಳಾಗಿದ್ದ-ವೆಂದು
ಊರು-ಗ-ಳಾಗಿ-ರ-ಬ-ಹುದು
ಊರು-ಗ-ಳಾಗಿವೆ
ಊರು-ಗ-ಳಾದ
ಊರು-ಗ-ಳಿಗೆ
ಊರು-ಗಳಿವೆ
ಊರು-ಗಳು
ಊರು-ಗಳೂ
ಊರು-ಗ-ಳೆಂದು
ಊರು-ಗಳೆಲ್ಲಾ
ಊರು-ಗಳೇ
ಊರೂ
ಊರೆ
ಊರೇ
ಊರೊಡ-ಯರೆಂದ
ಊರೊಡೆಯ
ಊರೊಡೆ-ಯನೇ
ಊರೊಡೆ-ಯರು
ಊರೊಳ-ಗಿನ
ಊರೊಳ-ಗಿ-ರುವ
ಊರೊಳಗೆ
ಊರ್ಗ್ಗೆೞಳವಿ-ಯನಿಕಿ
ಊಳಿಗ
ಊಳಿ-ಗಕ್ಕೆ
ಊಳಿ-ಗದ
ಊಳಿಗ-ದ-ವರು
ಊಳಿಗ-ಮಾನ್ಯ
ಊಳಿಗ-ಮಾನ್ಯದ
ಊಳಿಗ-ವನ್ನು
ಊಹಿಸ
ಊಹಿಸ-ಬಹುದ
ಊಹಿಸ-ಬಹು-ದಾದ
ಊಹಿಸ-ಬ-ಹುದು
ಊಹಿಸ-ಬಹು-ದು-ಜಿನ-ಗೃಹಮಂ
ಊಹಿಸ-ಬುದು
ಊಹಿಸ-ಲಾಗಿದೆ
ಊಹಿ-ಸಲು
ಊಹಿಸ-ಹುದು
ಊಹಿಸಿ
ಊಹಿಸಿದ್ದಾರೆ
ಊಹಿಸಿದ್ದಾ-ರೆಂದು
ಊಹಿ-ಸು-ವುದು
ಊಹೆ
ಊಹೆ-ಗ-ಳಲ್ಲೂ
ಊಹೆ-ಗಿಂತ
ಊಹೆ-ಯನ್ನು
ಊಹೆ-ಯಾಗಿದೆ
ಋಕ್
ಋಕ್ಶಾಖೆಗೆ
ಋಗೇದ
ಋಗ್ಯಜುರ್
ಋಗ್ವೇದದ
ಋಗ್ವೇದ-ದಲ್ಲಿ
ಋಗ್ವೇ-ದಾರ್ಣವ
ಋಚೋಧ್ಯೇತಾ
ಋಣಕ್ಕಾಗಿ
ಋಣಕ್ಕೆ
ಋಷಿ-ಗಳ
ಋಷಿ-ಗಳು
ಋಷಿ-ಗಳೇ
ಋಷಿ-ಯರ
ಋಷಿ-ಯರು
ಋಷಿಯು
ಋಷಿಸ-ಮು-ದಾಯ
ಎ
ಎಂ
ಎಂಎ
ಎಂಎಂ
ಎಂಎಂಕಲಬುರ್ಗಿ-ಯ-ವರು
ಎಂಎ-ಆರ್ನಲ್ಲಿ
ಎಂಎಚ್
ಎಂಎಚ್ನಾಗ-ರಾಜ-ರಾವ್
ಎಂಚಿ-ದಾನಂದ-ಮೂರ್ತಿ-ಗಳು
ಎಂಜಿನಿ-ಯರ್
ಎಂಟ-ನೆಯ
ಎಂಟು
ಎಂಟು-ಅಲು-ಗಿನ
ಎಂಟು-ನೂರು
ಎಂಟು-ಮೈಲಿ
ಎಂಟು-ವೃತ್ತಿ-ಯಲ್ಲಿ
ಎಂಣೆ
ಎಂತಹ
ಎಂತು
ಎಂದ
ಎಂದ-ಮೇಲೆ
ಎಂದರು
ಎಂದರೂ
ಎಂದರೆ
ಎಂದ-ರೇನು
ಎಂದಷ್ಟೇ
ಎಂದಾಗ
ಎಂದಾಗಿದೆ
ಎಂದಾಗಿ-ರ-ಬ-ಹುದು
ಎಂದಾ-ಗಿ-ರುವ
ಎಂದಾಗುತ್ತದೆ
ಎಂದಾ-ದರೆ
ಎಂದಿದೆ
ಎಂದಿದ್ದರು
ಎಂದಿದ್ದರೂ
ಎಂದಿದ್ದರೆ
ಎಂದಿದ್ದಿರ-ಬ-ಹುದು
ಎಂದಿದ್ದು
ಎಂದಿರ-ಬ-ಹುದು
ಎಂದಿರ-ಬೇಕು
ಎಂದಿ-ರುವು-ದ-ರಿಂದ
ಎಂದಿ-ರುವುದು
ಎಂದು
ಎಂದು-ಹೇ-ಳಿದೆ
ಎಂದೂ
ಎಂದೆ
ಎಂದೆಂದಿಗೂ
ಎಂದೆ-ನಿಸಿ-ಕೊಳ್ಳು
ಎಂದೆಲ್ಲಾ
ಎಂದೇ
ಎಂಫಿಲ್
ಎಂಬ
ಎಂಬಂತೆ
ಎಂಬತ್ತು
ಎಂಬಲ್ಲಿ
ಎಂಬಲ್ಲಿಗೆ
ಎಂಬವು
ಎಂಬಾತ-ನಿಗೆ
ಎಂಬಾರಯ್ಯ-ನ-ವರ
ಎಂಬಾರಯ್ಯ-ನ-ವರು
ಎಂಬಾರಯ್ಯನು
ಎಂಬಾರೈಯ-ನವ-ರಿಗೆ
ಎಂಬಾರ್ಗೋವಿಂದ
ಎಂಬಿ
ಎಂಬಿವು
ಎಂಬು
ಎಂಬು-ದಕ್ಕಾಗಿ
ಎಂಬು-ದಕ್ಕೂ
ಎಂಬು-ದಕ್ಕೆ
ಎಂಬು-ದನ್ನು
ಎಂಬು-ದನ್ನೂ
ಎಂಬು-ದರ
ಎಂಬು-ದ-ರಿಂದ
ಎಂಬು-ದಾಗಿ
ಎಂಬು-ದಾಗಿಯೂ
ಎಂಬು-ದಾ-ಗಿಯೇ
ಎಂಬುದು
ಎಂಬುದೂ
ಎಂಬುದೇ
ಎಂಬು-ವ-ನನ್ನು
ಎಂಬು-ವ-ನಿಂದ
ಎಂಬು-ವನು
ಎಂಬು-ವರು
ಎಂಬು-ವ-ವನ
ಎಂಬು-ವವ-ನನ್ನು
ಎಂಬು-ವವ-ನಿಂದ
ಎಂಬು-ವವ-ನಿಗೂ
ಎಂಬು-ವ-ವ-ನಿಗೆ
ಎಂಬು-ವ-ವನು
ಎಂಬು-ವ-ವನೇ
ಎಂಬು-ವ-ವರ
ಎಂಬು-ವ-ವರ-ನೇ-ಕರು
ಎಂಬು-ವ-ವ-ರನ್ನು
ಎಂಬು-ವ-ವ-ರಿಗೆ
ಎಂಬು-ವ-ವರು
ಎಂಬು-ವ-ವಳ
ಎಂಬು-ವ-ವಳು
ಎಂಬುವು
ಎಂಬೆರು-ಮಾನರು
ಎಂಭತ್ತು-ವರ-ಹವ
ಎಂಭತ್ತೆಂಟು
ಎಂಮ
ಎಂಮ-ದೂರ
ಎಂಮ-ದೂರ-ಹಳ್ಳಿಯ
ಎಂಮಾರ
ಎಂಮೆ
ಎಂಮೆ-ಬಸವೇಂದ್ರ
ಎಂವಿ
ಎಂವಿ-ಕೃಷ್ಣ-ರಾವ್
ಎಕ
ಎಕರೆ
ಎಕ-ವೀರ
ಎಕಾಎಕುವಾಘನ-ಮಾರಿತ್ಯಾ-ರಾಯಾಂಬಾಮುಲಾ
ಎಕ್ಕಟೆಯ
ಎಕ್ಕವ್ವೆಯರ
ಎಕ್ಕೆ-ಹಟ್ಟಿ-ಯ-ಕೆರೆ-ಗಳ
ಎಕ್ಕೋಟಿ
ಎಕ್ಕೋಟಿ-ಏಳ್ಕೋಟಿ
ಎಗಣನ
ಎಗರ-ಬಾ-ದನು
ಎಚ್ಎಸ್
ಎಚ್ಚ-ದೇವ
ಎಚ್ವಿ
ಎಟ್ಟು
ಎಡ
ಎಡಗಡೆ
ಎಡ-ಗಯ್ಯ
ಎಡ-ಗೆಯ್ಯ
ಎಡಗೈ
ಎಡಗೈ-ಎಡ-ಭಾಗ
ಎಡಗೈಯ
ಎಡಗೈಯ್ಯ
ಎಡಗೈಲಿ
ಎಡ-ಗೋಡೆಯ
ಎಡ-ತ-ಲೆಯ
ಎಡ-ತಲೆ-ಯ-ಹೆಡ-ತಲೆ
ಎಡ-ತೊರೆ-ಮಠದ
ಎಡ-ತೋಳಿ-ನಿಂದ
ಎಡ-ದರೆ
ಎಡ-ದರೆ-ಸಾಯಿರ
ಎಡ-ದೊರೆ
ಎಡ-ದೊರೆ-ನಾ-ಡಿಗೆ
ಎಡ-ದೊರೆ-ನಾಡು
ಎಡ-ಬಲ-ಗ-ಳಲ್ಲಿ
ಎಡ-ಬಲ-ದಲ್ಲಿ-ರುವ
ಎಡ-ಭಾ-ಗಕ್ಕೆ
ಎಡ-ಭಾಗದ
ಎಡ-ಭಾಗ-ದಲ್ಲಿ
ಎಡವಂಕ-ದಲ್ಲಿ
ಎಡವದ
ಎಡ-ವಾರಯ
ಎಡೂರು
ಎಡೆ
ಎಡೆ-ನಾಡ
ಎಡೆ-ಯೂರ
ಎಡೆ-ಹಳ್ಳ
ಎಣಿಸಲ್ಕೇಳ-ನೆಯ-ಬಾರಿ
ಎಣಿಸಲ್ಕೇ-ಳೆನೆ-ಬಾರಿ
ಎಣಿಸು
ಎಣಿಸು-ಮಗ
ಎಣ್ಣೆ
ಎಣ್ಣೆಗ
ಎಣ್ಣೆಗೆ
ಎಣ್ಣೆ-ನಾಡ
ಎಣ್ಣೆ-ಯನ್ನು
ಎಣ್ಣೆ-ಯನ್ನೂ
ಎತಿಗೆ
ಎತ್ತ-ಬೇ-ಕಾದ
ಎತ್ತರ
ಎತ್ತ-ರದ
ಎತ್ತರ-ದಲ್ಲಿದ್ದು
ಎತ್ತರ-ವಾದ
ಎತ್ತರ-ವಿದೆ
ಎತ್ತಲಾ-ಗುತ್ತಿತ್ತೆಂದು
ಎತ್ತಿ
ಎತ್ತಿ-ಕಟ್ಟಿದ
ಎತ್ತಿ-ಕೊಂಡು
ಎತ್ತಿ-ಕೊಟ್ಟಿದ್ದಾರೆ
ಎತ್ತಿ-ಕೊಳ್ಳ-ಲು-ವಸೂಲು
ಎತ್ತಿ-ಕೊಳ್ಳುವು-ದಾಗಿಯೂ
ಎತ್ತಿ-ತಂದು
ಎತ್ತಿ-ತನ್ದು
ಎತ್ತಿ-ದರು
ಎತ್ತಿನ
ಎತ್ತಿ-ನ-ಹೇಱಿಂಗೆ
ಎತ್ತಿಲ್ಲ
ಎತ್ತಿಸಿ
ಎತ್ತಿ-ಸಿದ
ಎತ್ತಿ-ಸಿ-ದಂತೆ
ಎತ್ತಿ-ಸಿ-ದ-ನೆಂದು
ಎತ್ತಿ-ಸಿ-ದ-ರೆಂದು
ಎತ್ತಿ-ಸುತ್ತಾನೆ
ಎತ್ತು
ಎತ್ತು-ಗಳ
ಎತ್ತು-ಗ-ಳನ್ನು
ಎತ್ತು-ಗ-ಳಿಂದ
ಎತ್ತುತ್ತಿದ್ದ
ಎತ್ತುವ
ಎದು-ರಾಗಿ
ಎದು-ರಾದ
ಎದು-ರಿಗಿನ
ಎದು-ರಿಗೆ
ಎದುರಿಸ
ಎದುರಿಸ-ಬೇಕಾಗಿ
ಎದುರಿಸ-ಬೇಕಾ-ಯಿತು
ಎದುರಿ-ಸಲು
ಎದುರಿಸಿ
ಎದುರಿಸಿ-ದಾಗ
ಎದುರಿಸುತ್ತಾನೆ
ಎದುರಿಸುತ್ತಿದ್ದಾಗ
ಎದು-ರಿ-ಸುವಂತ
ಎದುರಿ-ಸು-ವಂತೆ
ಎದುರು
ಎದೆಯ
ಎದೆ-ಯೋಜನು
ಎದ್ದನು
ಎದ್ದರು
ಎದ್ದಾಗ
ಎದ್ದಿದ್ದ
ಎದ್ದು
ಎದ್ದು-ಹೋ-ಗಲು
ಎನ-ಗರಿ-ದೆಂದು
ಎನಿ-ತಾನುಂ
ಎನಿಸಿ-ಕೊಂಡರು
ಎನಿಸಿ-ಕೊಳ್ಳುತ್ತಿದ್ದಳು
ಎನಿ-ಸಿತು
ಎನಿ-ಸುತ್ತದೆ
ಎನುಳ್ಳ
ಎನ್
ಎನ್ಎಸ್
ಎನ್ನ-ಬ-ಹುದು
ಎನ್ನಲು
ಎನ್ನುತ
ಎನ್ನುತ್ತಾರೆ
ಎನ್ನುತ್ತಿದ್ದರು
ಎನ್ನುವ
ಎನ್ನು-ವಂತಹ
ಎನ್ನು-ವಲ್ಲಿಯೂ
ಎನ್ನು-ವ-ವನ
ಎನ್ನು-ವ-ವ-ನಿಗೆ
ಎನ್ನು-ವ-ವನು
ಎನ್ನು-ವ-ವರು
ಎನ್ನು-ವಷ್ಟ-ರಲ್ಲಿ
ಎನ್ನು-ವು-ದಕ್ಕೆ
ಎನ್ನು-ವು-ದನ್ನು
ಎನ್ನು-ವುದು
ಎಪಿಗ್ರಾ-ಪಿಯಾ
ಎಪಿಗ್ರಾಫಿಯಾ
ಎಪ್ಪತ್ತಕ್ಕೆ
ಎಪ್ಪತ್ತನ್ನು
ಎಪ್ಪತ್ತು
ಎಪ್ಪತ್ತೆ-ರಡು
ಎಪ್ಪತ್ತೆ-ರೆಡು
ಎಬ್ಬೆ
ಎಬ್ಬೆ-ಎಮ್ಬೆ-ಎಮ್ಮೆ
ಎಬ್ಬೆ-ಬಸವನು
ಎಬ್ಬೆ-ಬಸವ-ನೆಂಬು-ವ-ವನು
ಎಬ್ಬೆಯ
ಎಮ್
ಎಮ್ಎ-ಆರ್
ಎಮ್ಎಸ್
ಎಮ್ಜಿ
ಎಮ್ಬೆ
ಎಮ್ಮ
ಎಮ್ಮ-ದೂರ
ಎಮ್ಮ-ದೂರ-ಹಳ್ಳಿ-ಯನ್ನು
ಎಮ್ಮಳ್ದಕ್ಕೆ
ಎಮ್ಮೆ
ಎಮ್ಮೆ-ಗಳ
ಎಮ್ಮೆ-ನ-ವರು
ಎಮ್ಮೆ-ಬಸವನ
ಎಮ್ಮೆಯ
ಎಮ್ಮೆ-ಯ-ಕೇತ-ನ-ಹಟ್ಟಿ-ಯನ್ನು
ಎಮ್ಮೆ-ಯರ
ಎಮ್ಮೆ-ಯಾಗಿದೆ
ಎಮ್ಮೆ-ಸಂದಿ-ಯನ್ನು
ಎರಕ
ಎರಕಬ್ಬೆಯ
ಎರ-ಗನ-ಹಳ್ಳಿ-ಯನು
ಎರ-ಗನ-ಹಳ್ಳಿ-ಯನ್ನು
ಎರಗಿ
ಎರಗಿತು
ಎರಗಿ-ದ-ವರು
ಎರಡ
ಎರಡಕೆ
ಎರಡಕ್ಕೂ
ಎರಡನೆ
ಎರಡ-ನೆಯ
ಎರಡ-ನೆ-ಯ-ದಾಗಿ
ಎರಡ-ನೆ-ಯ-ವನು
ಎರಡನೇ
ಎರಡ-ನೋಂತು
ಎರಡನ್ನೂ
ಎರಡರ
ಎರಡ-ರಲ್ಲಿಯೂ
ಎರಡ-ರಲ್ಲೂ
ಎರಡ-ರಿಂದ
ಎರಡ-ರು-ನೂರು
ಎರಡಾ-ದಂತೆ
ಎರಡು
ಎರಡು-ಕಟ್ಟೆ
ಎರಡು-ನಾಡು-ಗಳ
ಎರಡು-ನೂರು
ಎರಡು-ಬಾರಿ
ಎರಡು-ಮೂರು
ಎರಡು-ಸಲಗೆ
ಎರಡು-ಸಾವಿರ
ಎರಡೂ
ಎರಡೂ-ವರೆ
ಎರಡೂ-ವರೆ-ಪಾವು
ಎರಡೆ-ರಡು
ಎರದಿಂಮ-ರಾ-ಜಯ್ಯ
ಎರೆ-ಕಳಿಂಗ
ಎರೆ-ಗಂಗ
ಎರೆ-ಗಂಗನ
ಎರೆ-ಗಂಗ-ನದು
ಎರೆ-ಗಂಗ-ನಿಗೆ
ಎರೆ-ಗಂಗನು
ಎರೆ-ಗಂಗ-ನೆಂದು
ಎರೆ-ಗಾಂಕ
ಎರೆ-ಗಾಂಕನ
ಎರೆಗಿತ್ತೂರು
ಎರೆಗಿತ್ತೂರು-ಗಣ
ಎರೆದು
ಎರೆಯ
ಎರೆಯಂಗ
ಎರೆಯಂಗ-ದೇವ
ಎರೆಯಂಗ-ದೇವನ
ಎರೆಯಂಗನ
ಎರೆಯಂಗ-ನ-ಕಾಲ-ದಿಂದ
ಎರೆಯಂಗ-ನಿಗೆ
ಎರೆಯಂಗನು
ಎರೆಯಂಗ-ನೆಂಬ
ಎರೆಯಂಗನೇ
ಎರೆಯ-ಗಂಗನು
ಎರೆ-ಯಣ್ಣ
ಎರೆಯಣ್ಣನ
ಎರೆಯಣ್ಣನು
ಎರೆಯಣ್ಣನೂ
ಎರೆ-ಯಪ್ಪ
ಎರೆಯಪ್ಪನ
ಎರೆಯಪ್ಪ-ನನ್ನು
ಎರೆಯಪ್ಪನು
ಎರೆಯಪ್ಪ-ನೆಂಬ
ಎರೆಯಪ್ಪ-ನೊಂದಿಗೆ
ಎರೆಯಪ್ಪ-ನೊಡನೆ
ಎರೆಯಪ್ಪ-ರಸ
ಎರೆಯಪ್ಪ-ರ-ಸನ
ಎರೆಯಪ್ಪ-ರಸ-ನನ್ನು
ಎರೆಯಪ್ಪ-ರ-ಸನು
ಎರೆಯಪ್ಪ-ರಸ-ನೆಂಬ
ಎರೆಯಪ್ಪ-ರಸ-ರಿಗೆ
ಎರೆಯಪ್ಪ-ರ-ಸರು
ಎರೆಯ-ಮಂಗಲದ
ಎರೆಯಮ್ಮನ
ಎರೆಯಮ್ಮನು
ಎರೆಯಮ್ಮ-ನೆಂಬು-ವ-ವನು
ಎರೆವೆಸ
ಎರೆ-ಹಳ್ಳಿ
ಎಱೆಯಂಗ
ಎಲಮಣ್ಠೆ
ಎಲಿ-ಗಾರ
ಎಲಿ-ಗಾರರ
ಎಲಿ-ಗಾರರು
ಎಲೆ
ಎಲೆ-ಕೊಪ್ಪ
ಎಲೆ-ಕೊಪ್ಪದ
ಎಲೆ-ಚಾ-ಕನ-ಹಳ್ಳಿಯ
ಎಲೆ-ನಂಬಿ-ಯರ
ಎಲೆಯ
ಎಲೆ-ಯ-ಗುಳಿ
ಎಲ್
ಎಲ್ಲ
ಎಲ್ಲಕ್ಕಿಂತ
ಎಲ್ಲರ
ಎಲ್ಲ-ರನ್ನೂ
ಎಲ್ಲ-ರಿಂದಲೂ
ಎಲ್ಲ-ರಿಗೂ
ಎಲ್ಲರೂ
ಎಲ್ಲ-ವನೂ
ಎಲ್ಲ-ವನ್ನೂ
ಎಲ್ಲವೂ
ಎಲ್ಲಾ
ಎಲ್ಲಾ-ದರೂ
ಎಲ್ಲಿ
ಎಲ್ಲಿಂದ
ಎಲ್ಲಿಂದಲೋ
ಎಲ್ಲಿಗೋ
ಎಲ್ಲಿತ್ತು
ಎಲ್ಲಿದೆ
ಎಲ್ಲಿದ್ದರೆ
ಎಲ್ಲಿಯೂ
ಎಲ್ಲಿವೆ
ಎಲ್ಲೂ
ಎಲ್ಲೆ
ಎಲ್ಲೆ-ಗಳ
ಎಲ್ಲೆ-ಗ-ಳನ್ನು
ಎಲ್ಲೆ-ಗ-ಳಾಗಿದ್ದು
ಎಲ್ಲೆಡೆ
ಎಲ್ಲೆ-ಯನ್ನಾಗಿ
ಎಲ್ಲೆ-ಯನ್ನು
ಎಲ್ಲೆ-ಯನ್ನು-ಮೇರೆ
ಎಲ್ಲೆ-ಯಲ್ಲಿ
ಎಲ್ಲೆಯೇ
ಎಲ್ಲೋ
ಎಳಂದೂರು
ಎಳಗ
ಎಳಗ-ನೆಂಬ
ಎಳಮೆ
ಎಳ-ವಾರೆ
ಎಳ-ಸನ-ಕಟ್ಟವ
ಎಳೆಎಳೆ-ಯಾಗಿ
ಎಳೆಯ
ಎಳೆಯ-ನೆಂದು
ಎಳ್ಳುಂಟೆ
ಎಳ್ಳೂ
ಎವಿ
ಎಷ್ಟರ-ಮಟ್ಟಿಗೆ
ಎಷ್ಟು
ಎಷ್ಟು-ಗದ್ದೆ
ಎಷ್ಟು-ಬೇಕೋ
ಎಷ್ಟೆಂದು
ಎಷ್ಟೆಷ್ಟು
ಎಷ್ಟೊಂದು
ಎಷ್ಟೋ
ಎಷ್ಟೋ-ವೇಳೆ
ಎಸಗಿದ
ಎಸ-ಗೂರು
ಎಸುವ-ರಾ-ದಿತ್ಯ
ಎಸೆವೆ
ಎಸ್
ಎಸ್ಎಚ್
ಎಸ್ಕೆ
ಎಸ್ಕೆ-ಮೋ-ಹನ್ರ-ವರ
ಎಸ್ವು-ರಾ-ದಿತ್ಯನುಂ
ಎಸ್ಶಿವಣ್ಣ
ಏಕ-ಕಾಲಕ್ಕೆ
ಏಕ-ಕಾಲ-ದಲ್ಲಿ
ಏಕ-ಪಾರ್ಶ್ವ
ಏಕಭೊಗ
ಏಕ-ಭೋಗ
ಏಕ-ಭೋಗ-ದತ್ತಿಯ
ಏಕ-ಭೋಗ-ದತ್ತಿ-ಯಾಗಿ
ಏಕ-ವೀರ
ಏಕಸ್ಥ-ವಾಗಿ
ಏಕಾಂಗ-ವೀರ
ಏಕಾಂಗಿಯಾ-ದನು
ಏಕಾಂತದ
ಏಕಾಂತ-ವಾದಿ-ಯಾದ
ಏಕಾದ-ಶ-ಪಲ್ಲೀ
ಏಕಾದ-ಶಿಯ
ಏಕೀ-ಕರಣ
ಏಕೆಂದರೆ
ಏಕೈಕ
ಏಕೋಜಿಯು
ಏಗ-ಗವುಂಡನ
ಏಚ
ಏಚಣ್ಣ
ಏಚಣ್ಣ-ದಂಡ-ನಾಯಕ
ಏಚಣ್ಣನ
ಏಚಣ್ಣನು
ಏಚ-ದಂಡಾಧೀಶ
ಏಚ-ನಿಗೆ
ಏಚ-ನೆಂಬ
ಏಚಬ್ಬೆ
ಏಚಲ-ದೇವಿ
ಏಚಲ-ದೇವಿಯ
ಏಚಲ-ದೇವಿ-ಯರ
ಏಚಲ-ದೇವಿರು
ಏಚವ್ವೆ
ಏಚಿ-ಕಬ್ಬೆ
ಏಚಿ-ದಂಡಾಧಿಪ-ನನ್ನು
ಏಚಿ-ಮಯ್ಯ
ಏಚಿ-ಮಯ್ಯ-ದಂಡ-ನಾಯಕ
ಏಚಿ-ರಾಜ
ಏಚಿ-ರಾಜ-ದಂಡಾಧೀಶನ
ಏಚಿ-ರಾಜನ
ಏಚಿ-ರಾಜ-ನನ್ನು
ಏಚಿ-ರಾಜ-ನಿಗೆ
ಏಚಿ-ರಾಜನು
ಏಚಿ-ರಾಜನೂ
ಏಚಿ-ರಾಜ-ಹಿ-ರಿಯ
ಏಚೋ-ಜನ
ಏಟೂರಿ
ಏಟೂರು
ಏಟೂರು-ನಿ-ವಾಸಿ
ಏತ
ಏತಕ್ಕಾಗಿ
ಏತಕ್ಕೆ
ಏತ-ಗ-ಳಿಂದ
ಏತಗುಯ್ಯಲು
ಏತದ
ಏತ-ವೆಂದರೆ
ಏನನ್ನಾ-ದರೂ
ಏನಾ-ದರೂ
ಏನಿದ್ದರೂ
ಏನಿಲ್ಲ
ಏನು
ಏನುಂಟು
ಏನುಟ್ಟೆಂದು
ಏನು-ಬೇ-ಕಾದರೂ
ಏನುಳ
ಏನುಳ್ಳ
ಏನೂ
ಏನೆಂಬುದು
ಏನೇ
ಏಪ್ರಿಲ್
ಏರಲು
ಏರಿ
ಏರಿಕೆ
ಏರಿ-ಗ-ಳನ್ನು
ಏರಿ-ಗ-ಳನ್ನೂ
ಏರಿ-ಗಳು
ಏರಿದ
ಏರಿ-ದನು
ಏರಿ-ದ-ನೆಂದು
ಏರಿ-ದ-ರೆಂದು
ಏರಿದ್ದನು
ಏರಿದ್ದ-ರೆಂದು
ಏರಿದ್ದಾನೆ
ಏರಿಯ
ಏರಿ-ಯನ್ನು
ಏರಿ-ಯಾಗಿ-ರ-ಬ-ಹುದು
ಏರಿ-ಯಿಲ್
ಏರಿಯು
ಏರಿ-ರ-ಬ-ಹುದು
ಏರಿ-ರ-ಬಹು-ದೆಂದು
ಏರಿ-ರುವ
ಏರಿ-ರುವುದು
ಏರಿ-ಸಿರುವ
ಏರುತ್ತಿದ್ದರು
ಏರುತ್ತಿದ್ದ-ರೆಂದು
ಏರ್ಪಟ್ಟಾಗ
ಏರ್ಪಡಿ-ಸಲಾ-ಗಿದ್ದ
ಏರ್ಪಡಿ-ಸಲು
ಏರ್ಪ-ಡಿಸಿ
ಏರ್ಪಡಿ-ಸಿದ
ಏರ್ಪಡಿ-ಸಿ-ದಂತೆ
ಏರ್ಪಡಿ-ಸಿ-ದನು
ಏರ್ಪಡಿ-ಸಿ-ದ-ರೆಂದು
ಏರ್ಪಡಿ-ಸಿ-ರ-ಬ-ಹುದು
ಏರ್ಪಡಿ-ಸುವ
ಏರ್ಪಡಿ-ಸು-ವಲ್ಲಿ
ಏರ್ಪಡಿ-ಸು-ವುದು
ಏರ್ಪಾಡು
ಏಱಿಂಗಳಿಂಗೆ
ಏಱು
ಏಱುಂಗ-ಳಿಗೆ
ಏಳನೆ
ಏಳ-ನೆಯ
ಏಳನೇ
ಏಳನೇ-ಬಾರಿಗೆ
ಏಳರ-ಲಕ್ಕ-ಏ-ಳೂವ-ರೆ-ಲಕ್ಷ
ಏಳಳವಿ
ಏಳಾ-ಚಾರ್ಯ
ಏಳಾ-ಚಾರ್ಯ-ದಿವಾ-ಕರ-ಣಂದಿ
ಏಳಾ-ಚಾರ್ಯ-ನಿಗೇ
ಏಳಾ-ಚಾರ್ಯ-ನೆಂಬ
ಏಳಾ-ಚಾರ್ಯನೇ
ಏಳಾ-ಚಾರ್ಯರ
ಏಳಾ-ಚಾರ್ಯರೇ
ಏಳು
ಏಳುಂಮಲೆ
ಏಳು-ಕಲ್ಲಿನ
ಏಳು-ಕೋಟಿ
ಏಳು-ಕೋಟಿ-ರುದ್ರರು
ಏಳು-ತಲೆ-ಮಾರು-ಗಳು
ಏಳು-ತ-ಳಿಗೆ-ಯನ್ನು-ಳಿದ
ಏಳುತ್ತದೆ
ಏಳು-ದಿನ-ಗ-ಳಲ್ಲಿ
ಏಳು-ನೂರು
ಏಳು-ನೂ-ರುಮ್
ಏಳು-ಪುರ
ಏಳು-ಪುರದ
ಏಳು-ಪುರ-ವೆಂದು
ಏಳುಪ್ರಜೆ-ಲೆಂಕರು
ಏಳು-ಬತ್ತೆಟ್ಟು
ಏಳು-ಬಾರಿ
ಏಳು-ಬೀಳು-ಗಳು
ಏಳು-ಬೀಳು-ಗಳೇ
ಏಳು-ಭಾಗ
ಏಳು-ಮಲೆ
ಏಳು-ಸಾರಿ
ಏಳು-ಹಡೆ-ಯನ್ನೂ
ಏಳು-ಹಣದ
ಏಳು-ಹೆಡೆ-ಗಳ
ಏಳೂವ-ರೆ-ಲಕ್ಷ
ಏಳೂವ-ರೆ-ಲಕ್ಷ-ವನ್ನು
ಏಳೆಂಟು
ಏಳ್ಕೋಟಿ-ರುದ್ರರ
ಏವೊ-ಗಳ್ಪುದುಣ್ನ
ಏವೊ-ಗಳ್ವೆನುನ್ನ-ತಿಯಂ
ಏೞ-ನೆಯ
ಐಂದ್ರ-ಪರ್ವಕ್ಕೆ
ಐಂಬತ್ತಿ-ರು-ವರ್ನ್ನು
ಐಕ್ಯ-ತೆಯ
ಐಕ್ಯನಾ-ದನು
ಐಕ್ಯ-ವಾ-ದಂತೆ
ಐತ-ಪಾರ್ಯನ
ಐತಿ-ಹಾಸಿಕ
ಐತಿ-ಹಾಸಿಕ-ವಾಗಿ
ಐತಿಹ್ಯ
ಐತಿಹ್ಯ-ಗಳ
ಐತಿಹ್ಯದ
ಐತಿಹ್ಯ-ದಂತೆ
ಐತಿಹ್ಯ-ದಿಂದ
ಐತಿಹ್ಯ-ದಿಂದಲೂ
ಐದ-ನೆಯ
ಐದನೇ
ಐದು
ಐದು-ಆರು-ಏಳು
ಐದು-ಕೋಣೆ-ಗಳ
ಐದು-ಖಂಡುಗ
ಐದು-ಜನ
ಐದು-ಜನ-ರಿಗೆ
ಐದು-ನೂರು
ಐದು-ಹಳ್ಳಿ-ಗ-ಳನ್ನು
ಐನೂರು
ಐನೂರ್ವರ
ಐನೂರ್ವರು
ಐನ್
ಐನ್ನೂರರ್ವ್ವರು
ಐಮಂಗಳ
ಐಯ್ಯಂಗಾರ-ರೆಂದು
ಐವತಿಬ್ಬರ
ಐವತ್ತನಾಲ್ಕರ
ಐವತ್ತು
ಐವತ್ತೆ-ರಡು
ಐವತ್ತೊಕ್ಕಲು
ಐವರು
ಐಶ್ವರ್ಯ
ಐಶ್ವರ್ಯ-ವನ್ನು
ಐಹೊಳೆ-ಯಲ್ಲಿ
ಒಂಟಿ-ಕೊಪ್ಪಲ್
ಒಂಟೆ
ಒಂಡಂಬಟ್ಟು
ಒಂದಕಂ
ಒಂದಕೆ
ಒಂದಕ್ಕಿಂತ
ಒಂದಕ್ಕೆ
ಒಂದ-ನೆಯ
ಒಂದ-ನೆಯ-ಬಲ್ಲಾಳನ
ಒಂದನೇ
ಒಂದನ್ನು
ಒಂದರ
ಒಂದ-ರಲ್ಲಿ
ಒಂದರೆ
ಒಂದಾಗಿ
ಒಂದಾ-ಗಿತ್ತು
ಒಂದಾಗಿತ್ತೆಂದು
ಒಂದಾಗಿದೆ
ಒಂದಾ-ಗಿದ್ದು
ಒಂದಾಗಿ-ರುವು-ದ-ರಿಂದ
ಒಂದಾದ
ಒಂದಾ-ದರೆ
ಒಂದಿಬ್ಬರು
ಒಂದಿರ-ಬ-ಹುದು
ಒಂದಿ-ರವಿ-ನೊಳು
ಒಂದು
ಒಂದು-ಕಡೆ
ಒಂದು-ಕಡೆ-ಯೊಳು
ಒಂದು-ಗೂಡಿ-ಸಿ-ಕೊಂಡು
ಒಂದು-ನೂರು
ಒಂದುರಿ
ಒಂದು-ಸಲಗೆ
ಒಂದು-ಸಾವಿರ
ಒಂದೂ
ಒಂದೂ-ವರೆ
ಒಂದೆಂದು
ಒಂದೆಡೆ
ಒಂದೆ-ರಡು
ಒಂದೇ
ಒಂದೊಂದು
ಒಂದೊಂದೇ
ಒಂದೋ
ಒಂಬತ್ತ-ನೆಯ
ಒಂಬತ್ತು
ಒಂಭತ್ತು
ಒಂಭೈ-ನೂರ
ಒಂಭೈ-ನೂರ-ಮೂ-ವತ್ತೈದ-ನೆಯಕ್ರಿಶ
ಒಕಲುಮ-ಕಳು
ಒಕ್ಕಣಿಗೆ
ಒಕ್ಕಣೆ
ಒಕ್ಕಣೆ-ಗಿಂತ
ಒಕ್ಕ-ಣೆಯ
ಒಕ್ಕಣೆ-ಯನ್ನೇ
ಒಕ್ಕ-ಣೆಯು
ಒಕ್ಕ-ಣೆಯೂ
ಒಕ್ಕಲಲು
ಒಕ್ಕ-ಲಾ-ಗಿದ್ದು
ಒಕ್ಕ-ಲಾದ
ಒಕ್ಕಲಿಕ್ಕಿದ-ನೆಂದು
ಒಕ್ಕಲಿಗ
ಒಕ್ಕಲಿ-ಗರ
ಒಕ್ಕಲಿ-ಗ-ರಲ್ಲಿ
ಒಕ್ಕಲಿ-ಗ-ರಲ್ಲೂ
ಒಕ್ಕಲಿ-ಗರು
ಒಕ್ಕಲಿ-ಗರೇ
ಒಕ್ಕಲು
ಒಕ್ಕಲು-ಗಳ
ಒಕ್ಕಲು-ಗ-ಳಿಂದ
ಒಕ್ಕಲು-ಗ-ಳಿಗೆ
ಒಕ್ಕಲು-ಗಳು
ಒಕ್ಕಲು-ಗೂಡಿದ್ದ-ರೆಂದು
ಒಕ್ಕಲು-ತನ
ಒಕ್ಕಲು-ತ-ನದ
ಒಕ್ಕಲು-ದೆರೆ
ಒಕ್ಕಲು-ಮಕ್ಕಳು
ಒಕ್ಕಲು-ವಿ-ನಲ್ಲಿ
ಒಕ್ಕಲ್ಗೆ
ಒಕ್ಕುವ
ಒಕ್ಕೂಟ-ವನ್ನು
ಒಗೆ-ಯಲು
ಒಗೆಯುತ್ತಿದ್ದ
ಒಗೆ-ಯು-ವುದು
ಒಗೊಂಡಿತ್ತು
ಒಜಾಯಿತ
ಒಟ್ಟಾಗಿ
ಒಟ್ಟಾ-ಗಿಯೇ
ಒಟ್ಟಾರೆ
ಒಟ್ಟಾರೆ-ಯಾಗಿ
ಒಟ್ಟಿಗೆ
ಒಟ್ಟಿ-ನಲ್ಲಿ
ಒಟ್ಟು
ಒಟ್ಟು-ರೂಪ
ಒಡಂಬಟ್ಟು
ಒಡಂಬಡಿಸುತ್ತಾರೆ
ಒಡಕ್ಕು
ಒಡ-ಗೂಡಿ
ಒಡ-ಗೆರೆ
ಒಡ-ಗೆರೆ-ಮಲ್ಲಂ
ಒಡ-ಗೆರೆ-ಮಲ್ಲ-ಇದೂ
ಒಡ-ಗೆರೆ-ಮಲ್ಲನುಂ
ಒಡ-ನೆಯೇ
ಒಡಮೂಡು-ವಂತೆ
ಒಡವು
ಒಡವೆ-ಗ-ಳನ್ನು
ಒಡವೆ-ವಸ್ತು-ಗ-ಳನ್ನು
ಒಡ-ಹುಟ್ಟಿದ
ಒಡುಕ್ಕಿನ
ಒಡುಕ್ಕಿನ್
ಒಡೆ-ತನ
ಒಡೆ-ತನಕ್ಕೂ
ಒಡೆ-ತನಕ್ಕೆ
ಒಡೆ-ತ-ನದ
ಒಡೆ-ತನ-ವನ್ನು
ಒಡೆ-ತನವು
ಒಡೆ-ತನವೂ
ಒಡೆದು
ಒಡೆದು-ಹೋಗಿದ್ದ
ಒಡೆದು-ಹೋಗಿದ್ದು
ಒಡೆದು-ಹೋದ
ಒಡೆಯ
ಒಡೆಯ-ತನ
ಒಡೆ-ಯನ
ಒಡೆಯ-ನಂಬಿ
ಒಡೆಯ-ನಂಬಿ-ಯಾದ
ಒಡೆಯ-ನ-ಕಾರ್ಯಕೆ
ಒಡೆಯ-ನದು
ಒಡೆಯ-ನನ್ನು
ಒಡೆಯ-ನನ್ನೇ
ಒಡೆಯ-ನಾಗಿ
ಒಡೆಯ-ನಾ-ಗಿದ್ದ
ಒಡೆಯ-ನಾಗಿದ್ದ-ನೆಂದು
ಒಡೆಯ-ನಾದ
ಒಡೆಯ-ನಿಗೂ
ಒಡೆಯ-ನಿಗೆ
ಒಡೆಯ-ನಿದ್ದಂತೆ
ಒಡೆ-ಯನು
ಒಡೆಯ-ನು-ದೊಡ್ಡ-ದೇವ-ರಾಜ
ಒಡೆಯ-ನೂ-ಮಹಾಪ್ರಧಾನಿ
ಒಡೆಯ-ನೆಂದರೆ
ಒಡೆಯ-ನೆಂದು
ಒಡೆಯ-ನೆಂಬ
ಒಡೆಯ-ನೆಂಬು-ವ-ವನು
ಒಡೆ-ಯನೇ
ಒಡೆಯ-ನೊಬ್ಬ-ನಿಗೆ
ಒಡೆಯಪ್ಪಯ್ಯ
ಒಡೆ-ಯರ
ಒಡೆಯ-ರಅ
ಒಡೆಯ-ರ-ಕಟ್ಟೆ
ಒಡೆಯ-ರನ್ನು
ಒಡೆಯ-ರ-ವರ
ಒಡೆ-ಯರ-ಹಳ್ಳಿಯ
ಒಡೆಯ-ರಾಗಿ
ಒಡೆಯ-ರಾಗಿದ್ದರು
ಒಡೆಯ-ರಾಗಿದ್ದ-ರೆಂದು
ಒಡೆಯ-ರಾಗಿ-ರ-ಬಹು-ದೆಂದು
ಒಡೆಯ-ರಾದ
ಒಡೆಯ-ರಿಂದ
ಒಡೆಯ-ರಿಗೆ
ಒಡೆ-ಯರು
ಒಡೆಯ-ರು-ಗಳು
ಒಡೆ-ಯರೂ
ಒಡೆ-ಯರ್
ಒಡೆ-ಯಾರ
ಒಡೆಯುನು
ಒಡೆಯುರ
ಒಡೆಯುರು
ಒಡೆ-ಯುವ
ಒಡೆರ-ಯರು
ಒಡ್ಡ-ಗಲ್ಲು-ರಂಗಸ್ವಾಮಿ-ಬೆಟ್ಟ
ಒಡ್ಡು
ಒತ್ತಡ-ವನ್ನು
ಒತ್ತರಿಸಿ-ದ-ವನು
ಒತ್ತಿ
ಒತ್ತಿ-ಕೊಂಡು
ಒತ್ತುವ-ರಿ-ಯಾಗಿದೆ
ಒತ್ತೆ
ಒತ್ತೆ-ಇಟ್ಟಿದ್ದ
ಒತ್ತೆ-ಯಾಗಿ
ಒತ್ತೆ-ಯಾಗಿ-ರಿಸಿ-ಕೊಂಡಿದ್ದ-ನೆಂದೂ
ಒದಗ-ದಂತೆ
ಒದಗಿ-ದಾಗ
ಒದಗಿಸ-ಲಾಗುತ್ತದೆ
ಒದಗಿ-ಸಲು
ಒದಗಿ-ಸಿದರೆ
ಒದಗಿ-ಸುತ್ತದೆ
ಒದಗಿ-ಸುತ್ತವೆ
ಒದಗಿಸುತ್ತಿದ್ದ-ವರೇ
ಒದಗಿ-ಸುವ
ಒದಗಿ-ಸುವದೇ
ಒದಗಿಸು-ವು-ದನ್ನು
ಒದೆ-ಯೂರು
ಒನ್ದರ-ಣಿಯ
ಒಪ್ಪ
ಒಪ್ಪಂದ
ಒಪ್ಪಂದ-ಅ-ವನ್ನು
ಒಪ್ಪಂದಕ್ಕೆ
ಒಪ್ಪಂದ-ಗ-ಳನ್ನು
ಒಪ್ಪಂದದ
ಒಪ್ಪಂದ-ವನ್ನು
ಒಪ್ಪಂದ-ವಾಗಿ
ಒಪ್ಪಂದ-ವಾಗಿ-ರುವುದು
ಒಪ್ಪ-ತಕ್ಕದ್ದಾಗಿದೆ
ಒಪ್ಪ-ತಕ್ಕದ್ದೇ
ಒಪ್ಪತ್ತಿನ
ಒಪ್ಪದ
ಒಪ್ಪ-ಬ-ಹುದು
ಒಪ್ಪ-ವನ್ನು
ಒಪ್ಪ-ವಾ-ಗಿದ್ದು
ಒಪ್ಪ-ವಿದೆ
ಒಪ್ಪವೂ
ಒಪ್ಪ-ಸಹಿತ
ಒಪ್ಪಿ
ಒಪ್ಪಿ-ಕೊಂಡಂತಿದೆ
ಒಪ್ಪಿ-ಕೊಂಡರು
ಒಪ್ಪಿ-ಕೊಂಡಿ-ತೆಂದೂ
ಒಪ್ಪಿ-ಕೊಂಡು
ಒಪ್ಪಿ-ಕೊಳ್ಳದೆ
ಒಪ್ಪಿ-ಕೊಳ್ಳ-ಬೇಕಾ-ಯಿತು
ಒಪ್ಪಿ-ಕೊಳ್ಳುತ್ತಾನೆ
ಒಪ್ಪಿ-ಕೊಳ್ಳುತ್ತಾರೆ
ಒಪ್ಪಿಗೆ
ಒಪ್ಪಿ-ಗೆಗೆ
ಒಪ್ಪಿ-ಗೆ-ಯಂತೆ
ಒಪ್ಪಿ-ಗೆ-ಯನ್ನು
ಒಪ್ಪಿ-ತ-ವಾಗಿ
ಒಪ್ಪಿ-ದಂತೆಯೋ
ಒಪ್ಪಿ-ದರು
ಒಪ್ಪಿ-ದರೂ
ಒಪ್ಪಿ-ಸಲಾಯಿ-ತೆಂದು
ಒಪ್ಪಿಸಿ
ಒಪ್ಪಿ-ಸಿದ
ಒಪ್ಪಿ-ಸಿ-ದ-ನಂತೆ
ಒಪ್ಪಿ-ಸಿ-ದ-ನೆಂದು
ಒಪ್ಪಿ-ಸಿ-ದಾಗ
ಒಪ್ಪಿ-ಸು-ತಿದ್ದರು
ಒಪ್ಪಿ-ಸುವು-ದಾಗಿಯೂ
ಒಪ್ಪು-ತಾರೆ
ಒಪ್ಪುತ್ತಾನೆ
ಒಪ್ಪುತ್ತಾರೆ
ಒಪ್ಪುವ
ಒಪ್ಪು-ವುದು
ಒಬ್ಬ
ಒಬ್ಬನ
ಒಬ್ಬ-ನಂದಿನ್ತ-ವರಿ-ವರ-ಳವೆ
ಒಬ್ಬ-ನಾ-ಗಿದ್ದ
ಒಬ್ಬ-ನಾಗಿದ್ದನು
ಒಬ್ಬ-ನಾ-ಗಿದ್ದು
ಒಬ್ಬ-ನಾಗಿ-ರ-ಬ-ಹುದು
ಒಬ್ಬ-ನಾ-ಗಿ-ರುವ
ಒಬ್ಬ-ನಾದ
ಒಬ್ಬ-ನಿಗೆ
ಒಬ್ಬ-ನಿರ-ಬಹು-ದೆಂದು
ಒಬ್ಬನು
ಒಬ್ಬನೇ
ಒಬ್ಬರ
ಒಬ್ಬ-ರನ್ನೊಬ್ಬರು
ಒಬ್ಬ-ರಾ-ಗಿದ್ದ
ಒಬ್ಬ-ರಾ-ಗಿದ್ದರು
ಒಬ್ಬ-ರಾಗಿದ್ದ-ರೆಂದು
ಒಬ್ಬ-ರಾಗಿ-ರು-ವಂತೆ
ಒಬ್ಬ-ರಾಜ-ನಾಗಿ
ಒಬ್ಬ-ರಾದ
ಒಬ್ಬ-ರಾದರು
ಒಬ್ಬ-ರಿಂದ
ಒಬ್ಬ-ರಿ-ಗಿಂತ
ಒಬ್ಬ-ರಿಗೊಬ್ಬ-ರಿಗೆ
ಒಬ್ಬ-ರಿರ-ಬ-ಹುದು
ಒಬ್ಬರು
ಒಬ್ಬ-ರೆಂದು
ಒಬ್ಬರೇ
ಒಬ್ಬಳು
ಒಬ್ಬೆ
ಒಬ್ಬೊಬ್ಬ
ಒಮ-ಲೂರು
ಒಮ್ಮತ-ವಿಲ್ಲ
ಒಮ್ಮೆಲೇ
ಒಯ್ದರು
ಒಯ್ದಿದ್ದಾರೆ
ಒರಗಿ-ಸಲ್ಪಟ್ಟಿದೆ
ಒರಟೂರು
ಒರಿಸ್ಸಾ
ಒರಿಸ್ಸಾದ
ಒರ್ವರೋರ್ವಂ
ಒಲಘಾಂಡ
ಒಲ-ವನ್ನು
ಒಲವು
ಒಲು-ವರ
ಒಲ್ಲದ-ವರ
ಒಳ
ಒಳಂಗೆರೆ-ಗಳ
ಒಳ-ಕೇರಿಯ
ಒಳ-ಕೇರಿ-ಯಲ್ಲಿ
ಒಳಕ್ಕೆ
ಒಳಗಗಿ
ಒಳ-ಗಣ
ಒಳ-ಗಾಗಿ
ಒಳ-ಗಾಗಿ-ರುವುದು
ಒಳಗಾದ
ಒಳಗಾ-ದರು
ಒಳಗಾದ-ವ-ರಿಗೆ
ಒಳಗಿತ್ತೆಂದು
ಒಳಗಿದೆ
ಒಳ-ಗಿದ್ದ
ಒಳ-ಗಿ-ರುವ
ಒಳಗೂ
ಒಳಗೆ
ಒಳ-ಗೆರೆ
ಒಳಗೆ-ರೆಯ
ಒಳಗೆ-ರೆ-ಯಲ್ಲಿ
ಒಳಗೆ-ರೆ-ಯಲ್ಲಿದೆ
ಒಳಗೆ-ರೆ-ಯಿಂದಂ
ಒಳಗೇ
ಒಳ-ಗೊಂಡ
ಒಳ-ಗೊಂಡಂತೆ
ಒಳ-ಗೊಂಡಿತ್ತು
ಒಳ-ಗೊಂಡಿತ್ತೆಂದು
ಒಳ-ಗೊಂಡಿದೆ
ಒಳ-ಗೊಂಡಿದ್ದವು
ಒಳ-ಗೊಂಡಿವೆ
ಒಳ-ನಾಡಾಗಿ-ರ-ಬ-ಹುದು
ಒಳ-ನಾಡಿನ
ಒಳಪಂಗಡ-ವಿದೆ
ಒಳ-ಪಟ್ಟ
ಒಳ-ಪಟ್ಟರೂ
ಒಳ-ಪಟ್ಟಿತ್ತು
ಒಳ-ಪಟ್ಟಿತ್ತೆಂದು
ಒಳ-ಪಟ್ಟಿದ್ದ
ಒಳ-ಪಟ್ಟಿದ್ದಂತೆ
ಒಳ-ಪಡಿ
ಒಳ-ಪಡಿ-ಸ-ಬ-ಹುದು
ಒಳ-ಪಡಿ-ಸ-ಲಾಗಿದೆ
ಒಳ-ಪ-ಡಿಸಿ
ಒಳ-ಪಡಿ-ಸಿ-ಕೊಳ್ಳಲು
ಒಳ-ಪಡಿ-ಸುವ
ಒಳ-ಪಡು-ವು-ದಕ್ಕೆ
ಒಳಪ್ರಾ-ಕಾರದ
ಒಳ-ಬಾ-ಗಿಲ
ಒಳ-ಭಾಗದ
ಒಳ-ಭಾಗ-ದಲ್ಲಿದ್ದು
ಒಳ-ಭೇದ-ಗಳೂ
ಒಳ-ಮುಟ್ಟನ-ಹಳ್ಳಿ-ಗ-ಳನ್ನು
ಒಳಲು
ಒಳವಾ-ರನ್ನು
ಒಳ-ವಾರು
ಒಳ-ವಾರು-ಹೊರ-ವಾರು
ಒಳ-ವಾಱು
ಒಳಹೊಕ್ಕ-ನೆಂದು
ಒಳಹೊಕ್ಕವ-ರಲ್ಲಿ
ಒಳ-ಹೊಕ್ಕು
ಒಳ್ಳೆಯ
ಒಳ್ಳೆಯ-ದನ್ನು
ಒಳ್ಳೆಯ-ದಾಗುತ್ತದೆ
ಒಳ್ಳೆ-ಯದು
ಓಂ
ಓಕದ-ಕಲ್ಲು
ಓಜ
ಓಜಓವಜ
ಓಜಓವ-ಜರು
ಓಜ-ಕುಲ
ಓಜರು
ಓಡಾಡುತ್ತಿದ್ದ-ನಷ್ಟೆ
ಓಡಾಡುತ್ತಿದ್ದರು
ಓಡಿ
ಓಡಿ-ಸಲು
ಓಡಿಸಿ
ಓಡಿ-ಸಿ-ಕೊಂಡು
ಓಡಿ-ಸಿ-ದಂತೆ
ಓಡಿ-ಸಿ-ದ-ನೆಂದೂ
ಓಡಿ-ಸಿ-ದುದು
ಓಡಿ-ಸುವ-ವರ
ಓಡಿ-ಸುವ-ವರು
ಓಡಿ-ಹೋಗಿದ್ದು
ಓಡಿ-ಹೋ-ದದ್ದು
ಓಡಿ-ಹೋದ-ನಂತೆ
ಓಡಿ-ಹೋದನು
ಓಡಿ-ಹೋ-ಯಿತು
ಓದ-ಬೇಕಾಗಿದೆ
ಓದಿ
ಓದಿ-ಕೊಂಡರೆ
ಓದಿದೆ
ಓದಿದ್ದಾರೆ
ಓದಿದ್ದೆ
ಓದಿದ್ದೇನೆ
ಓದಿ-ನಿಂದ
ಓದುಗ-ರಿಗೆ
ಓದುತ್ತಿದ್ದ
ಓದುತ್ತಿದ್ದೆ
ಓದುತ್ತಿದ್ದೇನೆ
ಓದುವ
ಓದುವ-ವರು
ಓದು-ವುದ-ರಲ್ಲಿ
ಓಪಾಧಿ-ಯಲಿ
ಓಬಾಂಬಿಕೆ
ಓಬಾಂಬಿ-ಕೆಯ
ಓಬಾಂಬಿಕೆ-ಯಲ್ಲಿ
ಓಬಾಂಬಿಕೆ-ಯಿಂದ
ಓಬಾಂಬೆ-ಯರ
ಓರಂಗಲ್
ಓರಂಗಲ್ಲು
ಓರ-ಪಣ-ಪುರ
ಓರ-ಪಣ-ಪುರದ
ಓರು-ಗಲ್ಲು
ಓರೆಕೋರೆ-ಗ-ಳನ್ನು
ಓಲಗ-ಸಾಲೆ
ಓಲಗ-ಸಾಲೆ-ಯನ್ನು
ಓಲೆ
ಓಲ್ಡ್ಟೌನ್
ಓವ-ಜರು
ಓಷಧಿ-ಪತ್ಯುಪ-ಮಾಯಿ-ತ-ಗಂಡಸ್ತೋಷಣ-ರೂಪ-ಜಿತಾಸಮಕಾಂಡಃ
ಔದಾರ್ಯಕ್ಕೆ
ಔರಸ-ಪುತ್ರರು
ಕ
ಕಂ
ಕಂಗು
ಕಂಚ-ಗಾರ
ಕಂಚ-ಗಾರ-ಕುಳ
ಕಂಚ-ಗಾರ-ಗೊತ್ತಳಿ-ಗಳು
ಕಂಚಲ-ದೇವಿ
ಕಂಚ-ಹಡ-ವಳ
ಕಂಚಿ
ಕಂಚಿ-ಗ-ಹಳ್ಳಿಯ
ಕಂಚಿ-ಗುರಿ-ಯಪ್ಪನ-ಮೋಡಿದ
ಕಂಚಿಗೆ
ಕಂಚಿ-ಗೊಂಡ
ಕಂಚಿ-ಗೋ-ಜನ
ಕಂಚಿನ
ಕಂಚಿ-ನ-ಕೆರೆ
ಕಂಚಿ-ನ-ಬೋ-ರನ
ಕಂಚಿ-ಪಟ್ಟ-ಣ-ದತ್ತ
ಕಂಚಿ-ಮಠದ
ಕಂಚಿ-ಮಠ-ವೆಂಬ
ಕಂಚಿಯ
ಕಂಚಿ-ಯತ್ತ
ಕಂಚಿ-ಯನ್ನೇ
ಕಂಚಿ-ಯಲ್ಲಿ
ಕಂಚಿ-ಯಿಂದ
ಕಂಚಿ-ಯಿತ್ತ
ಕಂಚಿ-ರಾಯನ
ಕಂಚಿ-ಹಳ್ಳಿಯ
ಕಂಚೀ-ಪುರಕ್ಕೆ
ಕಂಚೀ-ಮಠದ
ಕಂಚು
ಕಂಚು-ಕುಲದ
ಕಂಚು-ಗ-ಹಳ್ಳಿ-ಗಳು
ಕಂಚು-ಗಾರ
ಕಂಚು-ಗಾರರ
ಕಂಚು-ಗಾರರು
ಕಂಟಕ-ಗ-ಳನ್ನು
ಕಂಟಿ-ಮಯ್ಯ
ಕಂಟಿ-ಮಯ್ಯಂ
ಕಂಟಿ-ಮಯ್ಯನ
ಕಂಟಿ-ಮಯ್ಯನು
ಕಂಟಿ-ಮಯ್ಯನೂ
ಕಂಠೀ-ರವ
ಕಂಠೀ-ರವ-ಗುಳಿಗೆ
ಕಂಠೀ-ರವನ
ಕಂಠೀ-ರವ-ನ-ರಸ-ನೃಪಾಂಬೋಧಿ
ಕಂಠೀ-ರವ-ನ-ರಸ-ರಾಜ
ಕಂಠೀ-ರವ-ನ-ರಸ-ರಾಜನು
ಕಂಠೀ-ರವನು
ಕಂಠೀ-ರವ-ನೆನಿಸಿ
ಕಂಠೀ-ರವ-ಮಮ್
ಕಂಠೀ-ರವರ
ಕಂಠೀ-ರವ-ರವ
ಕಂಠೀ-ರವಾ-ಕೃತಿಃ
ಕಂಠೀ-ರವಾಕ್ರುತಿಃ
ಕಂಠೀ-ರಾಯ
ಕಂಠೀ-ರಾಯ-ವರ-ಹ-ವನ್ನು
ಕಂಠೀ-ವರ
ಕಂಡ
ಕಂಡ-ದಿಂಡೆ
ಕಂಡದ್ದು
ಕಂಡನು
ಕಂಡ-ನೆಂದು
ಕಂಡ-ಬರುತ್ತದೆ
ಕಂಡ-ಬ-ರುತ್ತವೆ
ಕಂಡ-ರಣೆ
ಕಂಡ-ರಣೆ-ಕಾರರ
ಕಂಡ-ರಣೆ-ಕಾರರು
ಕಂಡ-ರಿಸ-ಲಾಗಿದೆ
ಕಂಡ-ರಿಸಲ್ಪಟ್ಟಿದೆ
ಕಂಡ-ರಿ-ಸಿದ
ಕಂಡ-ರಿಸಿದ್ದಕ್ಕೆ
ಕಂಡ-ರಿ-ಸಿದ್ದಾನೆ
ಕಂಡ-ರಿ-ಸಿದ್ದು
ಕಂಡ-ರಿ-ಸಿರ-ಬ-ಹುದು
ಕಂಡ-ರಿ-ಸಿರುವ
ಕಂಡ-ರಿ-ಸಿರುವ-ವರ
ಕಂಡ-ರಿ-ಸಿರುವುದು
ಕಂಡ-ರಿಸುತ್ತಿದ್ದರು
ಕಂಡ-ರಿ-ಸುವ
ಕಂಡರೂ
ಕಂಡಿತ್ತೆಂದು
ಕಂಡಿ-ರು-ವು-ದನ್ನು
ಕಂಡು
ಕಂಡುಂದಿದ್ದು
ಕಂಡು-ಕೊಂಡ
ಕಂಡುಗ
ಕಂಡು-ಬಂದ
ಕಂಡು-ಬಂದಾಗ
ಕಂಡು-ಬಂದಿದೆ
ಕಂಡು-ಬಂದಿದ್ದು
ಕಂಡು-ಬಂದಿರುವ
ಕಂಡು-ಬ-ರಲು
ಕಂಡು-ಬರುತ್ತದೆ
ಕಂಡು-ಬರುತ್ತ-ದೆಂದು
ಕಂಡು-ಬ-ರುತ್ತವೆ
ಕಂಡು-ಬ-ರುತ್ತವೆಯೇ
ಕಂಡು-ಬರುತ್ತಾರೆ
ಕಂಡು-ಬರುತ್ತಿದೆ
ಕಂಡು-ಬರುತ್ತಿದ್ದು
ಕಂಡು-ಬರುತ್ತಿಲ್ಲ
ಕಂಡು-ಬರುತ್ತೆ
ಕಂಡು-ಬ-ರುವ
ಕಂಡು-ಬ-ರು-ವು-ದಿಲ್ಲ
ಕಂಡು-ಬ-ರು-ವು-ದಿಲ್ಲ-ವೆಂದು
ಕಂಡು-ಹಿಡಿದು
ಕಂಡೆವು
ಕಂಡೇರಿ-ಕಟ್ಟೆ
ಕಂಣಂಬಾಡಿ
ಕಂಣಂಬಾ-ಡಿಯ
ಕಂಣ-ನೂರ
ಕಂಣ್ನಂಬಿ-ನಾ-ತನುಂ
ಕಂಣ್ನಯ
ಕಂಣ್ನಯ-ನಾಯ-ಕನು
ಕಂತಿಯ-ರಾಗಿ
ಕಂತಿ-ಯರು
ಕಂತಿ-ಯಾಗಿ-ರ-ಬಹು-ದೆಂದು
ಕಂದ
ಕಂದಂ
ಕಂದ-ಕುದ್ದಾಳ
ಕಂದ-ಪದ್ಯ
ಕಂದ-ಪದ್ಯ-ಗ-ಳಲ್ಲಿ
ಕಂದ-ಪದ್ಯ-ಗಳು
ಕಂದ-ಪದ್ಯ-ಗಳೂ
ಕಂದ-ಪದ್ಯ-ದಲ್ಲಿ
ಕಂದ-ಬರ
ಕಂದರ್ಪ
ಕಂದರ್ಪ-ದೀಕ್ಷಿತ
ಕಂದರ್ಪ-ದೇವರ
ಕಂದರ್ಪರ
ಕಂದರ್ಪ್ಪರ
ಕಂದಾ-ಚಾರ
ಕಂದಾ-ಚಾರದ
ಕಂದಾಡಿ
ಕಂದಾಡೈ-ದಾಸ-ರೆಂಬ
ಕಂದಾಯ
ಕಂದಾ-ಯಕ್ಕೆ
ಕಂದಾಯದ
ಕಂದಾಯ-ವನ್ನು
ಕಂದಾಯ-ವಾಗಿ
ಕಂದಾಯ-ವಿರ-ಬ-ಹುದು
ಕಂದ್ಯಾ
ಕಂಧರ
ಕಂಧರ-ನುಮಂ
ಕಂನಂಬಾಡಿ
ಕಂನಂಬಾ-ಡಿಯ
ಕಂನ-ಗೊಂಡೇಶ್ವರ
ಕಂನಡಿ
ಕಂನಡಿಗ
ಕಂನಡಿ-ಗ-ಮೊನೆ-ಯಾಳ್ತನಂಗೆಯ್ವ
ಕಂನಡಿ-ತೆ-ರಿಗೆ
ಕಂನಡಿ-ವಣ
ಕಂನಬಾಡಿ
ಕಂನಾರ-ದೇವ
ಕಂನಾರ-ದೇವ-ನೆಂದು
ಕಂನೆ-ಗೆರೆ
ಕಂನೆ-ಗೆರೆಯ
ಕಂನೆಯ
ಕಂನೆಯ-ನಾಯಕಂ
ಕಂನೆಯ-ನಾಯ-ಕನು
ಕಂನೆವಸ-ದಿಯಂ
ಕಂನ್ನ-ಗೆರೆಯ
ಕಂಪಂಣ
ಕಂಪಂಣ-ಗಳು
ಕಂಪಂಣ-ವೊಡೆ-ಯರ
ಕಂಪಣ
ಕಂಪ-ಣಕ್ಕೆ
ಕಂಪಣ-ಗಳೆಂಬ
ಕಂಪಣದ
ಕಂಪಣನ
ಕಂಪಣ್ಣ
ಕಂಪಣ್ಣನು
ಕಂಪನಿ-ಯಲ್ಲಿ
ಕಂಪನಿ-ಯ-ವರು
ಕಂಪನು
ಕಂಪ-ಮಂತ್ರಿ-ಯಿದ್ದ-ನೆಂದು
ಕಂಪ-ರಾಜನ
ಕಂಪ-ರಾಜನು
ಕಂಪ-ರಾಜ-ಪಟ್ಟ-ಣ-ಗೇರಿ-ಯಲ್ಲಿ
ಕಂಪಿಲ-ದೇವ-ನೊಡನೆ
ಕಂಪಿಲನ
ಕಂಪಿಲನು
ಕಂಪಿಲ-ನೊಡನೆ
ಕಂಪಿಲಿ-ದೇವನು
ಕಂಪಿ-ಲಿಯ
ಕಂಪೆಲನ
ಕಂಬ
ಕಂಬ-ಗಳ
ಕಂಬ-ಗಳಿದ್ದು
ಕಂಬ-ಗಳು
ಕಂಬ-ಗೈದು
ಕಂಬದ
ಕಂಬ-ದ-ಮೇಲಿ-ರುವ
ಕಂಬ-ದಲ್ಲಿ
ಕಂಬ-ದಳ್ಳಿ
ಕಂಬ-ದ-ಹಳ್ಳಿ
ಕಂಬ-ದ-ಹಳ್ಳಿ-ಗಳು
ಕಂಬ-ದ-ಹಳ್ಳಿಗೆ
ಕಂಬ-ದ-ಹಳ್ಳಿಯ
ಕಂಬ-ದ-ಹಳ್ಳಿ-ಯನ್ನು
ಕಂಬ-ದ-ಹಳ್ಳಿ-ಯಲ್ಲಿ
ಕಂಬ-ದ-ಹಳ್ಳಿ-ಯಲ್ಲಿ-ರುವ
ಕಂಬ-ದ-ಹಳ್ಳಿ-ಯಿಂದ
ಕಂಬ-ದ-ಹಳ್ಳಿಯು
ಕಂಬ-ದಿಂದ
ಕಂಬನ
ಕಂಬನು
ಕಂಬಯ್ಯ
ಕಂಬಯ್ಯ-ನನ್ನು
ಕಂಬಯ್ಯನು
ಕಂಬ-ರಾಜನು
ಕಂಬ-ರಿಂಗೆ
ಕಂಬ-ವನ್ನು
ಕಂಬ-ವನ್ನೇರಿ
ಕಂಬವು
ಕಂಬ-ವೊಂದಕ್ಕೆ
ಕಂಬಿ
ಕಂಬಿ-ಯನ್ನು
ಕಂಬೆ-ಗೆರೆಯ
ಕಂಬೇಶ್ವರ
ಕಂಭದ
ಕಂಭೇಶ್ವರ
ಕಂಮ-ಗಾರ
ಕಂಮಟೇಸ್ವರ
ಕಂಮಾರ
ಕಂಮಾರ-ಕಮ್ಮಾರ
ಕಂಮಾರ-ಪೆಂಮೋ-ಜನ
ಕಂಮಾಱ
ಕಂಮ್ಮ
ಕಂರ್ಣ್ನಿಕಾ
ಕಂಸರ
ಕಇ-ವಾರ
ಕಇ-ವಾರಕ
ಕಇ-ವಾರ-ಜ-ಗದ್ಧಳ
ಕಉಲೆ
ಕಕುದ್ಗಿರಿ
ಕಕುದ್ಗಿರಿ-ಯಲ್ಲಿ
ಕಕ್ಕಗ
ಕಗ್ಗಲೀ-ಪುರ
ಕಗ್ಗಲೀ-ಹಳ್ಳಿ
ಕಗ್ಗಲ್ಲ
ಕಗ್ಗೆರೆ
ಕಙ್ಗು
ಕಙ್ಗುಕ್ಸೇತ್ರಂ
ಕಚ
ಕಚೇರಿ
ಕಚೇರಿ-ಯಲ್ಲಿ
ಕಚೇರಿ-ಯಲ್ಲೇ
ಕಚ್ಚ-ವರ
ಕಚ್ಚ-ವರದ
ಕಚ್ಚಾಣ-ಗದ್ಯಾಣ-ವನ್ನು
ಕಚ್ಚೆಗ
ಕಚ್ಛ-ವರದ
ಕಚ್ಛಾಣ
ಕಜ್ಜಾಯ
ಕಞ್ಚಗಾಱಅ
ಕಞ್ಚಗಾಱ-ಕುಳ-ದಲ್ಲಿ
ಕಟ
ಕಟಕ
ಕಟಕ-ಕತ್ತಿ
ಕಟಕ-ತೋಟಿ-ಕಾರ
ಕಟ-ಕದ
ಕಟಕ-ದೆರೆ
ಕಟಕ-ದೊಡನೆ
ಕಟಕ-ಪಾತ್ರ
ಕಟಕ-ರಕ್ಷಣೆ
ಕಟ-ಕವು
ಕಟಕ-ಸೇಸೆ
ಕಟವಪ್ರ
ಕಟವಪ್ರ-ಗಿರಿ
ಕಟವಪ್ರ-ವೆಂದು
ಕಟವಪ್ರ-ಶೈಲ
ಕಟಿಸಿ-ದ-ನೆಂದು
ಕಟಿ-ಸುತ್ತಾಳೆ
ಕಟು
ಕಟು-ಗುತ್ತಗೆ
ಕಟು-ದಕೂ
ಕಟೆಯೇರಿ
ಕಟ್ಟ
ಕಟ್ಟಂಕಟ್ಟು-ವು-ದರ್ಕ್ಕೆ
ಕಟ್ಟ-ಕಂಮ-ಹದ
ಕಟ್ಟ-ಕಮ್ಮ-ಹದ
ಕಟ್ಟ-ಡದೊಳಕ್ಕೆ
ಕಟ್ಟ-ಡ-ವಿದೆ
ಕಟ್ಟ-ದೇವರ್
ಕಟ್ಟ-ಲಾಗಿತ್ತೆಂದು
ಕಟ್ಟ-ಲಾಗಿದೆ
ಕಟ್ಟ-ಲಾಗಿ-ದೆ-ಯೆಂದೂ
ಕಟ್ಟ-ಲಾ-ಯಿತು
ಕಟ್ಟ-ಲಾಯಿ-ತೆಂದು
ಕಟ್ಟಲು
ಕಟ್ಟ-ಲೆಯ
ಕಟ್ಟಲ್ಪಟ್ಟಿ-ತೆಂದು
ಕಟ್ಟಳೆ
ಕಟ್ಟ-ಳೆ-ಗಳ
ಕಟ್ಟ-ಳೆಗೆ
ಕಟ್ಟ-ಳೆಯ
ಕಟ್ಟ-ಳೆ-ಯನ್ನು
ಕಟ್ಟ-ವನ್ನು
ಕಟ್ಟಸಿ
ಕಟ್ಟಾಳು
ಕಟ್ಟಿ
ಕಟ್ಟಿ-ಕೊಂಡಿದ್ದ
ಕಟ್ಟಿ-ಕೊಂಡು
ಕಟ್ಟಿ-ಕೊಟ್ಟರು
ಕಟ್ಟಿ-ಕೊಟ್ಟಿದ್ದಾರೆ
ಕಟ್ಟಿ-ಕೊಡ
ಕಟ್ಟಿ-ಕೊಡ-ಬಹದ್ದು
ಕಟ್ಟಿ-ಕೊಡ-ಬ-ಹುದು
ಕಟ್ಟಿ-ಕೊಡ-ಲಾಗಿದೆ
ಕಟ್ಟಿದ
ಕಟ್ಟಿ-ದಂತೆ
ಕಟ್ಟಿ-ದ-ದಕ್ಕೆ
ಕಟ್ಟಿ-ದನು
ಕಟ್ಟಿ-ದ-ನೆಂದು
ಕಟ್ಟಿ-ದ-ರಂದು
ಕಟ್ಟಿ-ದರು
ಕಟ್ಟಿ-ದ-ಲಗಿ-ನಂತಿದ್ದ
ಕಟ್ಟಿ-ದ-ವ-ರನ್ನು
ಕಟ್ಟಿ-ದ-ವರ್ಗ್ಗೆ
ಕಟ್ಟಿ-ದಾಗ
ಕಟ್ಟಿ-ದಾಗ-ಮುತ್ತಿಗೆ
ಕಟ್ಟಿ-ದಿರ-ನಡೆದ-ಅ-ರಿಯ
ಕಟ್ಟಿ-ದು-ದಕ್ಕೆ
ಕಟ್ಟಿದ್ದ
ಕಟ್ಟಿದ್ದಕ್ಕಾಗಿ
ಕಟ್ಟಿದ್ದು
ಕಟ್ಟಿ-ನ-ಪಡಿ
ಕಟ್ಟಿ-ಬಿಡು-ವ-ನಾಯ-ಕರ
ಕಟ್ಟಿ-ಬಿಡು-ವ-ವರ-ಗಣ್ಡನು
ಕಟ್ಟಿರ
ಕಟ್ಟಿ-ರ-ಬಹು-ದಾದ
ಕಟ್ಟಿ-ರ-ಬ-ಹುದು
ಕಟ್ಟಿ-ರ-ಬಹು-ದೆಂದು
ಕಟ್ಟಿ-ರುವ
ಕಟ್ಟಿ-ರು-ವು-ದನ್ನು
ಕಟ್ಟಿ-ರು-ವು-ದಾಗಿ
ಕಟ್ಟಿ-ಸ-ಬೇಕೆಂದು
ಕಟ್ಟಿ-ಸ-ಲಾಗಿದೆ
ಕಟ್ಟಿ-ಸ-ಲಾ-ಯಿತು
ಕಟ್ಟಿ-ಸಲು
ಕಟ್ಟಿಸಿ
ಕಟ್ಟಿ-ಸಿ-ಕೊಡುತ್ತಾನೆ
ಕಟ್ಟಿ-ಸಿ-ಕೊಡು-ವಂತೆ
ಕಟ್ಟಿ-ಸಿದ
ಕಟ್ಟಿ-ಸಿದಂ
ಕಟ್ಟಿ-ಸಿ-ದಂತೆ
ಕಟ್ಟಿ-ಸಿ-ದನು
ಕಟ್ಟಿ-ಸಿ-ದ-ನುದ್ಯೋಗ
ಕಟ್ಟಿ-ಸಿ-ದ-ನೆಂದು
ಕಟ್ಟಿ-ಸಿ-ದ-ನೆಂದೂ
ಕಟ್ಟಿ-ಸಿ-ದರು
ಕಟ್ಟಿ-ಸಿ-ದರೆ
ಕಟ್ಟಿ-ಸಿ-ದ-ರೆಂದು
ಕಟ್ಟಿ-ಸಿ-ದ-ರೆೆಂದೂ
ಕಟ್ಟಿ-ಸಿ-ದಳು
ಕಟ್ಟಿ-ಸಿ-ದ-ಳೆಂದು
ಕಟ್ಟಿ-ಸಿ-ದಾಗ
ಕಟ್ಟಿ-ಸಿ-ದೆವು
ಕಟ್ಟಿ-ಸಿದ್ದ
ಕಟ್ಟಿ-ಸಿದ್ದನು
ಕಟ್ಟಿ-ಸಿದ್ದ-ನೆಂದು
ಕಟ್ಟಿ-ಸಿದ್ದಾನೆ
ಕಟ್ಟಿ-ಸಿದ್ದಾರೆ
ಕಟ್ಟಿ-ಸಿದ್ದಾಳೆ
ಕಟ್ಟಿ-ಸಿದ್ದು
ಕಟ್ಟಿ-ಸಿನು
ಕಟ್ಟಿ-ಸಿ-ರ-ಬಹು-ದಾದ
ಕಟ್ಟಿ-ಸಿ-ರ-ಬ-ಹುದು
ಕಟ್ಟಿ-ಸಿ-ರ-ಬಹು-ದೆಂದು
ಕಟ್ಟಿ-ಸಿ-ರ-ಬಹು-ದೆಂದೂ
ಕಟ್ಟಿ-ಸಿ-ರುತ್ತಾರೆ
ಕಟ್ಟಿ-ಸಿ-ರುವ
ಕಟ್ಟಿ-ಸಿ-ರು-ವಂತೆ
ಕಟ್ಟಿ-ಸಿ-ರುವುದು
ಕಟ್ಟಿಸು
ಕಟ್ಟಿ-ಸುತ್ತಾನೆ
ಕಟ್ಟಿ-ಸುತ್ತಾರೆ
ಕಟ್ಟಿ-ಸುತ್ತಾಳೆ
ಕಟ್ಟಿ-ಸುತ್ತಿದ್ದರು
ಕಟ್ಟಿ-ಸುತ್ತಿದ್ದ-ರೆಂದು
ಕಟ್ಟಿ-ಸುವ
ಕಟ್ಟಿ-ಸು-ವಂತೆ
ಕಟ್ಟಿ-ಸು-ವಾಗ
ಕಟ್ಟಿ-ಸು-ವು-ದರ
ಕಟ್ಟಿ-ಸು-ವುದು
ಕಟ್ಟಿ-ಹಾ-ಕಿರು-ವುದು
ಕಟ್ಟು
ಕಟ್ಟು-ಕಾಲುವೆ
ಕಟ್ಟು-ಕಾಲುವೆ-ಗಳ
ಕಟ್ಟು-ಕಾಲುವೆ-ಗ-ಳನ್ನು
ಕಟ್ಟು-ಕಾಲುವೆ-ಗಳು
ಕಟ್ಟು-ಕಾಲುವೆಯ
ಕಟ್ಟು-ಕಾಲುವೆ-ಯೊಳಗಾದ
ಕಟ್ಟು-ಕಾಲುವೆ-ಯೊಳ-ಗಿನ
ಕಟ್ಟು-ಕಾಲುವೆ-ಯೊಳಗೆ
ಕಟ್ಟು-ಗುತ್ತಗೆ
ಕಟ್ಟು-ಗುತ್ತ-ಗೆ-ಯನ್ನು
ಕಟ್ಟು-ಗುತ್ತ-ಗೆ-ಯಾಗಿ
ಕಟ್ಟು-ಗೂಳು
ಕಟ್ಟುತ್ತಿದ್ದ
ಕಟ್ಟುತ್ತಿದ್ದ-ನೆಂದು
ಕಟ್ಟುತ್ತಿದ್ದರು
ಕಟ್ಟುತ್ತಿದ್ದ-ವ-ರಿಗೆ
ಕಟ್ಟು-ಪಾಡನ್ನು
ಕಟ್ಟು-ಪಾಡು
ಕಟ್ಟು-ಪಾಡು-ಗ-ಳನ್ನು
ಕಟ್ಟು-ಮಾಡಿ
ಕಟ್ಟು-ಮಾಡುತ್ತಾನೆ
ಕಟ್ಟು-ಮಾಡುತ್ತಾಳೆ
ಕಟ್ಟುವ
ಕಟ್ಟು-ವಂತೆ
ಕಟ್ಟು-ವಂತೆಯೂ
ಕಟ್ಟು-ವಿಚ್ಚ
ಕಟ್ಟು-ವುದಕ್ಕಾಗಿ
ಕಟ್ಟು-ವು-ದಕ್ಕೆ
ಕಟ್ಟೆ
ಕಟ್ಟೆ-ಕಾಲುವೆ
ಕಟ್ಟೆ-ಕಾಲುವೆ-ಗ-ಳನ್ನು
ಕಟ್ಟೆ-ಕಾಲುವೆ-ಗಳು
ಕಟ್ಟೆ-ಕೆರೆ
ಕಟ್ಟೆ-ಕೇತ-ನ-ಹಳ್ಳಿ
ಕಟ್ಟೆ-ಗಳ
ಕಟ್ಟೆ-ಗ-ಳನ್ನು
ಕಟ್ಟೆ-ಗ-ಳನ್ನೇ
ಕಟ್ಟೆ-ಗ-ಳಲ್ಲಿ
ಕಟ್ಟೆ-ಗಳಿಗೂ
ಕಟ್ಟೆ-ಗಳಿದ್ದ
ಕಟ್ಟೆ-ಗಳು
ಕಟ್ಟೆಗೆ
ಕಟ್ಟೆಯ
ಕಟ್ಟೆಯಂ
ಕಟ್ಟೆ-ಯ-ಕೆಳಗೆ
ಕಟ್ಟೆ-ಯನ್ನು
ಕಟ್ಟೆ-ಯನ್ನು-ಬಸಿ
ಕಟ್ಟೆ-ಯಲ್ಲಿ
ಕಟ್ಟೆ-ಯಾಗಿದೆ
ಕಟ್ಟೆ-ಯಾ-ಗಿದ್ದು
ಕಟ್ಟೆ-ಯಾಗಿ-ರ-ಬಹು-ದೆಂದು
ಕಟ್ಟೆ-ಯಿಂದ
ಕಟ್ಟೆಯು
ಕಟ್ಟೆಯೇ
ಕಟ್ಟೆ-ಯೊಂದನ್ನು
ಕಟ್ಟೆ-ಹಳ್ಳದಿಂ
ಕಟ್ಟೇನ-ಹಳ್ಳಿ
ಕಟ್ಟೇರಿ
ಕಟ್ಟೇರಿನ
ಕಟ್ಟೇ-ಸೋಮ-ನ-ಹಳ್ಳಿ
ಕಟ್ಟೊಬ್ಬೆ-ಹಳ್ಳ
ಕಠಾ-ರದ
ಕಠಾರಿ
ಕಠಾ-ರಿಯ
ಕಠಾರಿ-ಯನ್ನು
ಕಠಾರಿ-ರಾಯ
ಕಠಾರಿ-ರಾಯ-ರಾದ
ಕಠಾರಿ-ಸಾ-ಳುವ
ಕಠಾರಿ-ಸು-ರಗಿ
ಕಠೋರ
ಕಡಂಬಿ-ಗೆಯ
ಕಡತ-ವನ್ನು
ಕಡಪಾ-ಜಿಲ್ಲೆಯ
ಕಡಬ
ಕಡ-ಬದ
ಕಡಬ-ದ-ಕೆರೆ
ಕಡಲ
ಕಡಲ-ವಾಗಿಲ
ಕಡಲ-ವಾಗಿಲು
ಕಡಲು
ಕಡಲು-ವಾಗಿಲು
ಕಡಲೆ
ಕಡವದ-ಕೊಳದ
ಕಡವಿ-ಗೆರೆ-ಗಳ
ಕಡವು
ಕಡವು-ಅಪ್ಪು
ಕಡ-ವೂರ
ಕಡ-ಹಗೆ
ಕಡಹು
ಕಡಹೆಮ್ಮೊ-ಗೆಯ
ಕಡಾಂಬಿ
ಕಡಾರಂನ್ನು
ಕಡಿ-ತಕ್ಕೇರಿತು
ಕಡಿ-ತಕ್ಕೇರಿದ
ಕಡಿದು
ಕಡಿದು-ಕೊಂಡ-ನೆಂದು
ಕಡಿದು-ಕೊಂಡರೂ
ಕಡಿದು-ಕೊಂಡು
ಕಡಿಮೆ
ಕಡಿಮೆ-ಯಾಗಿ
ಕಡಿಮೆ-ಯಾಗಿ-ರ-ಬ-ಹುದು
ಕಡಿಮೆ-ಯಾಗಿ-ರ-ಬಹು-ದೆಂದು
ಕಡಿಮೆ-ಯಾಗುತ್ತದೆ
ಕಡಿಮೆ-ಯಾ-ಗುತ್ತಾ
ಕಡಿಮೆ-ಯಾಗುತ್ತಿತ್ತು
ಕಡಿಮೆ-ಯಾದ
ಕಡಿಸಿ
ಕಡಿಸಿ-ಕೊಂಡು
ಕಡು-ಕಲಿ
ಕಡುಗ-ಲಿಯಮ್
ಕಡುನಿಷ್ಠೆ-ಯನ್ನು
ಕಡುಪಿಂ
ಕಡುಮು-ಳಿದು
ಕಡು-ವನ್ನು
ಕಡೂರು
ಕಡೆ
ಕಡೆ-ಗಣಿಸಿ
ಕಡೆ-ಗ-ಳಲ್ಲಿ
ಕಡೆ-ಗ-ಳಲ್ಲಿಯೇ
ಕಡೆ-ಗ-ಳಲ್ಲಿ-ರುವ
ಕಡೆ-ಗ-ಳಲ್ಲಿವೆ
ಕಡೆಗೂ
ಕಡೆಗೆ
ಕಡೆ-ಗೆ-ಹೋಗಿ
ಕಡೆಗೇ
ಕಡೆಗೋ
ಕಡೆದ
ಕಡೆ-ದ-ವ-ರನ್ನು
ಕಡೆ-ದ-ವರೂ
ಕಡೆ-ದಿರುವ
ಕಡೆ-ದಿ-ರುವುದು
ಕಡೆಯ
ಕಡೆ-ಯಲ್ಲಿ
ಕಡೆ-ಯ-ವ-ರನ್ನು
ಕಡೆ-ಯ-ವ-ರಾದ
ಕಡೆ-ಯ-ವರು
ಕಡೆ-ಯ-ವಳೇ
ಕಡೆ-ಯಿಂದ
ಕಡೆಯೂ
ಕಡೆ-ಯೊಳು
ಕಡ್ಡಾಯ
ಕಡ್ಡಾ-ಯದ
ಕಡ್ಡಾಯ-ವಾಗಿ
ಕಡ್ಡಾಯ-ವಾಗಿದ್ದಂತೆ
ಕಡ್ಡಾಯ-ವಾಗಿ-ರ-ಲಿಲ್ಲ
ಕಡ್ಡಾಯ-ವಾದರೆ
ಕಡ್ಲ-ವಾಗಿ-ಲಿನ
ಕಡ್ಲ-ವಾಗಿಲು
ಕಡ್ವಪ್ಪು
ಕಣ
ಕಣ-ಕಟ್ಟೆ
ಕಣ-ಗಳ
ಕಣದ
ಕಣ-ಸ-ಲಿಗೆ
ಕಣಿ-ಕಟ್ಟೆ
ಕಣಿ-ಕಟ್ಟೆಯ
ಕಣಿ-ಕಟ್ಟೆ-ಯಲ್ಲಿ
ಕಣಿ-ಗಿ-ಲೆಯ-ಕಟ್ಟೆ
ಕಣಿವೆ
ಕಣಿ-ವೆ-ಗಳಿವೆ
ಕಣಿ-ವೆಯ
ಕಣಿ-ವೆ-ಯನ್ನು
ಕಣಿ-ವೆ-ಯಲ್ಲಿ
ಕಣಿ-ವೆ-ಯಲ್ಲಿದ್ದು
ಕಣಿ-ವೆಯು
ಕಣ್ಗೊಳ-ಸಲು-ಕತಿ
ಕಣ್ಡುಗ
ಕಣ್ಡೆವೆನೆ
ಕಣ್ಣ
ಕಣ್ಣಂಬಾಡಿ
ಕಣ್ಣ-ನೆಂಬು-ವ-ವನು
ಕಣ್ಣನ್ನಿಟ್ಟೇ
ಕಣ್ಣ-ವಂಗಲ-ಕಣ್ಣಾ-ಗಾಲ
ಕಣ್ಣ-ವಂಗಲ-ವನು
ಕಣ್ಣಾ-ಡಿಸಿದ
ಕಣ್ಣಾ-ನೂರ
ಕಣ್ಣಾನೂ-ರನ್ನೂ
ಕಣ್ಣಾ-ನೂರಿ-ನಲ್ಲಿ
ಕಣ್ಣಾ-ನೂರಿ-ನಿಂದಲೇ
ಕಣ್ಣಾ-ನೂರೇ
ಕಣ್ಣಾರೆ-ಕಂಡು
ಕಣ್ಣಿಗೆ
ಕಣ್ಣಿನ
ಕಣ್ಣು-ಗಳಿಗೂ
ಕಣ್ಣೆ-ಗಾಲ-ದಲ್ಲಿ
ಕಣ್ಣೇ-ಗಾಲ-ದಲ್ಲಿ
ಕಣ್ಣೇಶ್ವರ
ಕಣ್ನ
ಕಣ್ನಂಬಾಡಿ
ಕಣ್ನಂಬಾ-ಡಿಯ
ಕಣ್ನಂಬಿ-ನೊಡೆಯ
ಕಣ್ಮರೆ-ಯಾಗಿದೆ
ಕಣ್ಮರೆ-ಯಾಗಿವೆ
ಕಣ್ವ
ಕಣ್ವ-ಪುರಿ
ಕಣ್ವೇಶ್ವರ
ಕಣ್ವೇಶ್ವರಕ್ಕೆ
ಕತ್ತರಿ-ಗಟ್ಟ
ಕತ್ತರಿ-ಗಟ್ಟದ
ಕತ್ತರಿ-ಗಟ್ಟ-ದಲ್ಲಿ
ಕತ್ತರಿ-ಗಟ್ಟ-ದಲ್ಲಿ-ರುವ
ಕತ್ತರಿಘಟರ್ಟ-ದಲ್ಲಿ
ಕತ್ತರಿ-ಘಟ್ಟ
ಕತ್ತರಿ-ಘಟ್ಟದ
ಕತ್ತರಿ-ಘಟ್ಟ-ದಲ್ಲಿ
ಕತ್ತರಿ-ಸಲು
ಕತ್ತರಿಸಿ
ಕತ್ತರಿಸಿ-ಕೊಂಡು
ಕತ್ತಿ
ಕತ್ತಿ-ಗು-ರಾಣಿ-ಗಳೊಂದಿಗೆ
ಕತ್ತೆ
ಕತ್ತೆಯ
ಕತ್ರಿ-ಗಟ್ಟದ
ಕಥನ-ಮನ್ತೆಂದಡೆ
ಕಥಾ-ಪರಂಪ-ರೆಯ
ಕಥಾ-ಮಣಿ
ಕಥಾ-ಸಾ-ಗರ-ವನ್ನು
ಕಥಾಸಾರ
ಕಥೆ
ಕಥೆ-ಗ-ಳನ್ನು
ಕಥೆ-ಗ-ಳಲ್ಲಿ
ಕಥೆ-ಗಳು
ಕಥೆಯ
ಕಥೆಯಂ
ಕಥೆ-ಯನ್ನು
ಕಥೆ-ಯನ್ನು-ಹೇ-ಳುವ
ಕಥೆ-ಯಲ್ಲಿ
ಕಥೆ-ಯಲ್ಲಿಯೂ
ಕಥೆಯಿಂ
ಕಥೆ-ಯಿಂದ
ಕಥೆ-ಯಿವಂ
ಕಥೆಯು
ಕಥೆಯೂ
ಕದಂಬ
ಕದಂಬರ
ಕದಂಬರು
ಕದಂಬ-ರೆಂದರೆ
ಕದಂಬೆ-ಹಳ್ಳಿ
ಕದಂಬೆ-ಹಳ್ಳಿಯ
ಕದಂವ
ಕದನ
ಕದನ-ಗ-ಳಲ್ಲಿ
ಕದನ-ಗಳು
ಕದನತ್ರಿಣೇತ್ತನುಂ
ಕದನತ್ರಿಣೇತ್ರನುಂ
ಕದನ-ದಲ್ಲಿ
ಕದನ-ದೊಳಾಂತು
ಕದನ-ದೊಳ್
ಕದನಪ್ರಚಂಡ
ಕದನ-ವಾಗಿ
ಕದನ-ವಾಗಿ-ರ-ಬ-ಹುದು
ಕದನವು
ಕದನೈಕ
ಕದನೈಕ-ಸೂದ್ರಕ
ಕದ-ಪ-ನಾಯಕ
ಕದ-ಪ-ನಾಯ-ಕರು
ಕದ-ಬಳ್ಳಿ
ಕದ-ಬಳ್ಳಿಯು
ಕದ-ಬಳ್ಳಿಯೂ
ಕದ-ಬ-ಹಳ್ಳಿ-ಯನ್ನು
ಕದಬೆ
ಕದ-ಬೆ-ಹಳ್ಳಿಯ
ಕದ-ರಪ್ಪ
ಕದ-ರಿಯೂರ
ಕದರೂರ
ಕದರೂ-ರಿ-ನಲ್ಲಿ
ಕದರೂರು
ಕದರೆ-ನಾಯ-ಕನು
ಕದ-ರೆಯ-ನಾಯಕ
ಕದ-ರೆಯ-ನಾಯ-ಕನು
ಕದಲಂಬಳ್ಳೆಯ
ಕದಲ-ಗೆರೆ
ಕದಲ-ಗೆರೆಯ
ಕದಲ-ಗೆರೆ-ಯಿಂದ
ಕದ-ಳ-ಗೆರೆಯ
ಕದ-ವಳ್ಳಿಯ
ಕದವಿ
ಕದವೆ-ಹಳ್ಳಿ
ಕದಿಯುತ್ತಿದ್ದರು
ಕದು-ಲೆಯ
ಕದ್ದ-ಲ-ಗೆರೆ
ಕದ್ದ-ಳಗೆರ-ಇಂದಿನ
ಕದ್ದ-ಳ-ಗೆರೆ
ಕದ್ದ-ಳ-ಗೆರೆ-ಕ-ದಲ-ಗೆರೆಯ
ಕದ್ದ-ಳ-ಗೆರೆಗೆ
ಕದ್ದ-ಳ-ಗೆರೆಯ
ಕದ್ದ-ಳ-ಗೆರೆ-ಯು-ಇಂದಿನ
ಕದ್ದ-ವನಾ-ದರೂ-ಬಾಯಿ-ತಪ್ಪು
ಕದ್ದು
ಕದ್ರಿ
ಕದ್ರಿ-ಮಠದ
ಕನ-ಕ-ಕರ್ಪ್ಪೂರ-ಧಾರಾಪ್ರವಾಹ
ಕನ-ಕ-ಗಟ್ಟ
ಕನ-ಕ-ಗಿರಿ
ಕನ-ಕ-ಗಿರಿಯ
ಕನ-ಕ-ಗಿರಿಯು
ಕನ-ಕ-ಚಂದ್ರ
ಕನ-ಕ-ಚಂದ್ರ-ದೇವ-ರು-ನಯ-ಕೀರ್ತಿ
ಕನ-ಕ-ಚಕ್ರ
ಕನ-ಕ-ಚಾ-ಮರ
ಕನ-ಕ-ಛತ್ರಂಗಳಂ
ಕನ-ಕ-ದಂಡಿಗೆ
ಕನ-ಕ-ದಾಸನ
ಕನ-ಕ-ದಾಸರ
ಕನ-ಕ-ದಾಸರು
ಕನ-ಕನ-ಘಟ್ಟ
ಕನ-ಕ-ಪುರ
ಕನ-ಕ-ಸೇನ
ಕನ-ಗನ-ಮರಡಿ
ಕನ-ಸಿ-ನಲ್ಲಿ
ಕನಿಷ್ಟ
ಕನಿ-ಸದ
ಕನೂರು
ಕನೂರ್
ಕನ್ತಿ-ಯರುಸ್ವರ್ಗಸ್ಥ-ರಾ-ಗಲು
ಕನ್ತಿ-ಯರ್ಗೆ
ಕನ್ದಾಡೈ
ಕನ್ನಂಬಾಡಿ
ಕನ್ನಂಬಾಡಿ-ಕಟ್ಟೆಯ
ಕನ್ನಂಬಾಡಿಗೆ
ಕನ್ನಂಬಾಡಿಯ
ಕನ್ನಂಬಾಡಿ-ಯನ್ನು
ಕನ್ನಂಬಾಡಿ-ಯಲ್ಲಿ
ಕನ್ನಂಬಾಡಿ-ಯಲ್ಲಿದ್ದ
ಕನ್ನಂಬಾಡಿ-ಯಲ್ಲಿ-ರುವ
ಕನ್ನಂಬಾಡಿ-ಯ-ವರೆಗೆ
ಕನ್ನಂಬಾಡಿಯು
ಕನ್ನಡ
ಕನ್ನಡದ
ಕನ್ನಡ-ದಲ್ಲಿ
ಕನ್ನಡ-ನಾಡನ್ನು
ಕನ್ನಡ-ನಾಡಿನ
ಕನ್ನಡ-ನಾಡಿ-ನಲ್ಲಿಯೇ
ಕನ್ನಡ-ನಾಡು
ಕನ್ನಡಲಿ
ಕನ್ನಡ-ವನ್ನು
ಕನ್ನಡಿಗ
ಕನ್ನಡಿಗ-ನಾ-ಗಿದ್ದ
ಕನ್ನಡಿ-ಗರ
ಕನ್ನಡಿಗ-ರಾದ
ಕನ್ನಡಿಯ
ಕನ್ನ-ಪಳ್ಳಿಯ
ಕನ್ನ-ಯನ-ಪಳ್ಳಿ-ಯಲ್ಲಿ
ಕನ್ನ-ಯನ-ಹಳ್ಳಿ-ಯ-ಕನ್ನಪ್ಪಳ್ಳಿ
ಕನ್ನಯ್ಯನ
ಕನ್ನರ
ಕನ್ನರ-ದೇವ
ಕನ್ನರ-ದೇವನು
ಕನ್ನ-ರನು
ಕನ್ನರ-ಪಾಡಿ
ಕನ್ನಲ್ಲಿ
ಕನ್ನ-ಸತ್ತಿ
ಕನ್ನಿ-ಕುಮ್ಬಿರಾನ್
ಕನ್ನಿ-ಕೆಯ
ಕನ್ನಿಕೇಶ್ವರ
ಕನ್ನೆ-ಗೆರೆ
ಕನ್ನೆ-ಗೆರೆ-ಗ-ಳನ್ನು
ಕನ್ನೆ-ಗೆರೆ-ಮಲ್ಲ
ಕನ್ನೆ-ಗೆರೆ-ಯನ್ನು
ಕನ್ನೆ-ಗೆರೆ-ಯಾಗಿ
ಕನ್ನೆಯನ
ಕನ್ನೆವಸ-ದಿಯಂ
ಕನ್ನೆವ-ಸದಿ-ಯನ್ನು
ಕನ್ನೆವೆ-ಸದಿ
ಕನ್ನೇಶ್ವರ
ಕನ್ಯ-ಕೆಯ-ರನೊಂದೆ
ಕನ್ಯಾ-ಕು-ಮಾರಿ
ಕನ್ಯಾ-ದಾನ
ಕನ್ಯಾ-ದಾನಂಗಳ-ನತ್ಯೋನ್ನದಿಂ
ಕನ್ಯಾ-ದಾನಕ್ಕೆ
ಕನ್ಯಾ-ದಾನ-ಗ-ಳನ್ನು
ಕನ್ಯಾ-ದಾನದ
ಕನ್ಯಾ-ದಾನವು
ಕನ್ಯೆ-ಯ-ರನೊಂದೇ
ಕಪನಿ-ಪತಯ್ಯ
ಕಪಿಲಾ
ಕಪಿಲಾ-ನದಿ-ಗಳ
ಕಪಿಲೆ
ಕಪಿ-ಲೆಯ
ಕಪಿಲೆ-ಯಾತಏತ
ಕಪಿ-ಳೆಯ
ಕಪಿಶೀರ್ಷ
ಕಪ್ಪ-ಕಾಣಿ-ಕೆ-ಗ-ಳನ್ನು
ಕಪ್ಪ-ವನ್ನು
ಕಪ್ಪು
ಕಬರಸ್ಥಾನಕ್ಕಾಗಿ
ಕಬರಸ್ಥಾನದ
ಕಬಾಹು
ಕಬಾಹು-ನಾಡಾ-ಳುವ
ಕಬಿನ-ಕೆರೆಯ
ಕಬೆದಮ್ಮನ
ಕಬ್ಬನ್ನು
ಕಬ್ಬಪ್ಪು
ಕಬ್ಬಪ್ಪು-ನಾಡ
ಕಬ್ಬಪ್ಪು-ವಿ-ನಲ್ಲಿ
ಕಬ್ಬರೆ
ಕಬ್ಬಲಿ-ಗರು
ಕಬ್ಬಳ್ಳಿಯ
ಕಬ್ಬಹ-ಲಿನ
ಕಬ್ಬಹಲು
ಕಬ್ಬಹು
ಕಬ್ಬಹು-ನಾಡಾ-ಳುವ
ಕಬ್ಬಹು-ನಾಡಿನ
ಕಬ್ಬಹು-ಸಾ-ಸಿರದ
ಕಬ್ಬಾರೆ
ಕಬ್ಬಾಳ
ಕಬ್ಬಾಳು
ಕಬ್ಬಾಳು-ದುರ್ಗ
ಕಬ್ಬಾಳು-ದುರ್ಗವು
ಕಬ್ಬಾಹು
ಕಬ್ಬಾಹು-ಕಬ್ಬಹು
ಕಬ್ಬಾಹು-ನಾಡನ್ನು
ಕಬ್ಬಾಹು-ನಾಡಾ-ಳುವ
ಕಬ್ಬಾಹು-ನಾಡಾಳ್ವರುಂ
ಕಬ್ಬಾಹು-ನಾಡಿನ
ಕಬ್ಬಾಹು-ನಾಡು
ಕಬ್ಬಾಹು-ನಾಡೇ
ಕಬ್ಬಾಹು-ಸಾ-ಸಿರದ
ಕಬ್ಬಿ-ಣದ
ಕಬ್ಬಿಣ-ಯುಗ
ಕಬ್ಬಿನ
ಕಬ್ಬಿನ-ಕೆರೆ
ಕಬ್ಬಿನ-ಕೆರೆಯ
ಕಬ್ಬಿನ-ಹಳ್ಳಿ
ಕಬ್ಬಿಲ-ಗೆರೆ
ಕಬ್ಬಿ-ಲರ
ಕಬ್ಬಿ-ಲರ-ಹದಿಕೆ
ಕಬ್ಬಿ-ಲರು
ಕಬ್ಬಿಲವ-ಡಿಕೆ
ಕಬ್ಬಿಲವ-ಡಿಕೆಯ
ಕಬ್ಬು
ಕಬ್ಬು-ನಾಡ
ಕಬ್ಬು-ನಾಡಾಗಿ-ರ-ಬ-ಹುದು
ಕಬ್ಬು-ನಾಡಿನ
ಕಬ್ಬು-ಲರ-ಮಗ್ಗ
ಕಬ್ಬು-ಹ-ನಾಡಿನ
ಕಬ್ಬು-ಹು-ನಾಡ
ಕಬ್ಬೆರೆ
ಕಮನೀಯಾ
ಕಮಲ
ಕಮಲಜಂ
ಕಮಲದ
ಕಮಲ-ನಾದ
ಕಮಲ-ಭದ್ರ
ಕಮಲಯಾ
ಕಮಲ-ಯಾ-ಭೂಮ್ಯೇ
ಕಮಲ-ವನ
ಕಮಳ-ರಾಜ
ಕಮಳಿನೀ
ಕಮಾನೂ
ಕಮೀ-ಷನರ್ಗಳ
ಕಮ್
ಕಮ್ಮ-ಗಾರ
ಕಮ್ಮಟ
ಕಮ್ಮಟದ
ಕಮ್ಮಟ-ದುರ್ಗ-ಗಳು
ಕಮ್ಮಟೇಶ್ವರ
ಕಮ್ಮಯೊಳಗಿ
ಕಮ್ಮ-ಹದ
ಕಮ್ಮಾರ
ಕಮ್ಮಾರರು
ಕಮ್ಮಾರಿ-ಕೆಗೆ
ಕಮ್ಮೆ-ಕುಲಕ್ಕೆ
ಕಮ್ಮೆ-ಕುಲದ
ಕಯಲು
ಕಯಿ
ಕಯಿ-ಕೊಂಡು
ಕಯಿ-ಸೆರೆ-ಯ-ನಿಕ್ಕಿದ
ಕಯ್ಕೊಂಡ
ಕಯ್ಯನ್ನು
ಕಯ್ಯಲು
ಕರ
ಕರಂಡ
ಕರಕ್ಕೆ
ಕರಗ
ಕರ-ಗ-ಳನ್ನು
ಕರ-ಗಳು
ಕರಗಿ-ಸಿ-ಕೊಳ್ಳುತ್ತಾ
ಕರ-ಗುಂದ
ಕರಗ್ರಾಮ-ಗ-ಳನ್ನು
ಕರಗ್ರಾಮ-ಗಳು
ಕರ-ಜಿತ-ಸುರ-ಭೂಜಃ
ಕರ-ಡಚ್ಚನ್ನು
ಕರ-ಡನ್ನು
ಕರ-ಡ-ಹಳ್ಳಿ
ಕರ-ಡ-ಹಳ್ಳಿ-ಯನ್ನು
ಕರ-ಡಿ-ಕೊಪ್ಪಲು
ಕರ-ಡಿಯ-ಹಳ್ಳಿ
ಕರಡು
ಕರಣ
ಕರ-ಣ-ಕರು
ಕರ-ಣ-ಗಳು
ಕರ-ಣದ
ಕರ-ಣ-ನಾಗಿದ್ದ-ನೆಂದು
ಕರ-ಣ-ನಾದ
ಕರ-ಣ-ನೆಂದರೆ
ಕರ-ಣ-ರಾ-ಗಿದ್ದು
ಕರ-ಣರು
ಕರ-ಣರೂ
ಕರಣಿ
ಕರ-ಣಿಕ
ಕರ-ಣಿ-ಕ-ರಾದ
ಕರ-ಣಿ-ಕರು
ಕರ-ಣಿ-ಕ-ರೆಂದೇ
ಕರ-ಣಿ-ಕ-ಸೇನ-ಬೋವ-ಕುಲ-ಕರಣಿ
ಕರ-ಣಿ-ಕ-ಸೇನ-ಬೋವ-ಕುಳ-ಕರಣಿ
ಕರ-ತಾಳ
ಕರ-ದಾಖಿಲ-ಭೂ-ಪಾಲಃ
ಕರ-ದಾಳ
ಕರ-ದಾ-ಳದ
ಕರ-ಬಸಾಣಿ
ಕರ-ಮಂಡ-ಲಸ್ವಾಮಿಯ
ಕರ-ಮೂರ್ತಿ-ಗಳ್ಗುಹಾವಾ-ಸಿಗಳ
ಕರಮೆ-ಸೆದಂ
ಕರಮೆ-ಸೆಯೆ
ಕರವ
ಕರ-ವೆಂದು
ಕರಸ್ತ-ಳದ
ಕರಸ್ಥಲ
ಕರಸ್ಥ-ಲದ
ಕರಸ್ಥಳದ
ಕರಾರು
ಕರಾರು-ವಕ್ಕಾಗಿ
ಕರಾ-ವಳಿ
ಕರಿ
ಕರಿ-ಅಯ್ಕ-ಣನ
ಕರಿ-ಅಯ್ಕ-ಣನೆಂಬ
ಕರಿ-ಎಮ್ಮಾಉರ
ಕರಿ-ಕಲ್ಲ-ಹಳ್ಳ
ಕರಿ-ಕಲ್ಲು-ಮಂಟಿ
ಕರಿ-ಗೌಡರು
ಕರಿ-ಘಟ್ಟ
ಕರಿ-ತುರಕಗ-ಪಟ್ಟ-ಸಾ-ಹಣಿ
ಕರಿ-ತುರಗ
ಕರಿ-ಪ-ಗವುಡ
ಕರಿ-ಮದ-ದಿಂದೊಕ್ಕಲಿಕ್ಕಿ
ಕರಿಯ
ಕರಿ-ಯ-ಅಯ್ಕ-ಣನೆಂದು
ಕರಿ-ಯ-ಜೀಯ-ನ-ಹಳ್ಳಿ
ಕರಿ-ಯದ್ರಮ್ಮ
ಕರಿ-ಯ-ನೆಚ್ಚಡೆ
ಕರಿ-ಯ-ಮಾರ-ಗ-ವುಡನು
ಕರಿ-ಯಯ್ಕಣ-ನೆಂದು
ಕರಿ-ಯ-ವೀರನ
ಕರಿ-ಯ-ವೊಡೆ-ಯ-ರಿಗೆ
ಕರಿಲ
ಕರೀ-ಘಟ್ಟ
ಕರೀ-ಘಟ್ಟದ
ಕರೀ-ಜೀರ-ಹಳ್ಳಿ
ಕರು-ಣದ
ಕರು-ಣಿಸು
ಕರು-ಣಿಸುತ್ತಾನೆ
ಕರು-ಣೈಕ
ಕರುತ್ತ-ವಾನ್ತೀ-ಸರ-ಗಣ್ಡಕ್ಕಳುಕಿ
ಕರೆ
ಕರೆ-ಕಂಠ-ಜೀಯನು
ಕರೆ-ಗೌಡನ
ಕರೆ-ತಂದು
ಕರೆ-ದರು
ಕರೆ-ದರೆ
ಕರೆ-ದಿದೆ
ಕರೆ-ದಿದ್ದರೆ
ಕರೆ-ದಿದ್ದಾನೆ
ಕರೆ-ದಿದ್ದಾರೆ
ಕರೆ-ದಿದ್ದು
ಕರೆ-ದಿರ-ಬ-ಹುದು
ಕರೆ-ದಿರ-ಬಹು-ದುಪ್ರಿಯ-ಸುತ
ಕರೆ-ದಿರು-ಬ-ಹುದು
ಕರೆ-ದಿರುವ
ಕರೆ-ದಿ-ರು-ವು-ದನ್ನು
ಕರೆ-ದಿ-ರುವು-ದ-ರಿಂದ
ಕರೆ-ದಿ-ರು-ವು-ದಿಲ್ಲ
ಕರೆ-ದಿ-ರುವುದು
ಕರೆ-ದಿಲ್ಲ
ಕರೆ-ದಿವೆ
ಕರೆ-ದಿವೆ-ಮೇಲ್ಕಂಡ
ಕರೆ-ದೀವ-ದಾನಿವುಂ
ಕರೆದು
ಕರೆ-ದು-ಕೊಂಡ
ಕರೆ-ದು-ಕೊಂಡರೇ
ಕರೆ-ದು-ಕೊಂಡಿದ್ದಾನೆ
ಕರೆ-ದು-ಕೊಂಡಿದ್ದಾರೆ
ಕರೆ-ದು-ಕೊಂಡಿರು
ಕರೆ-ದು-ಕೊಂಡಿರು-ವುದ-ರಿಂದ
ಕರೆ-ದು-ಕೊಂಡಿರು-ವುದು
ಕರೆ-ದು-ಕೊಂಡು
ಕರೆ-ದು-ಕೊಳ್ಳುತ್ತಾರೆ
ಕರೆ-ದು-ಕೊಳ್ಳುತ್ತಿದ್ದುದು
ಕರೆ-ದು-ಕೊಳ್ಳು-ವುದುಂಟು
ಕರೆದೇ
ಕರೆಯ
ಕರೆ-ಯ-ಕ-ಲಾಗಿದೆ
ಕರೆ-ಯ-ತೊಡಗಿ-ದರು
ಕರೆ-ಯತ್ತಿದ್ದ-ರೆಂದು
ಕರೆ-ಯ-ಬ-ಹುದು
ಕರೆ-ಯ-ಲಾಗತ್ತಿತ್ತು
ಕರೆ-ಯ-ಲಾಗಿ
ಕರೆ-ಯ-ಲಾಗಿದೆ
ಕರೆ-ಯ-ಲಾಗಿ-ದೆಯೇ
ಕರೆ-ಯ-ಲಾಗಿ-ದೆೆ
ಕರೆ-ಯ-ಲಾ-ಗಿದ್ದು
ಕರೆ-ಯ-ಲಾಗಿ-ರುವು-ದ-ರಿಂದ
ಕರೆ-ಯ-ಲಾಗು
ಕರೆ-ಯ-ಲಾಗುತ್ತದೆ
ಕರೆ-ಯ-ಲಾಗುತ್ತಿತು
ಕರೆ-ಯ-ಲಾಗುತ್ತಿ-ತೆಂದೂ
ಕರೆ-ಯ-ಲಾಗುತ್ತಿತ್ತು
ಕರೆ-ಯ-ಲಾಗುತ್ತಿತ್ತೆಂದು
ಕರೆ-ಯ-ಲಾಗುತ್ತಿತ್ತೆಂದೂ
ಕರೆ-ಯ-ಲಾಗುತ್ತಿದೆ
ಕರೆ-ಯ-ಲಾಗುತ್ತಿದ್ದಿತು
ಕರೆ-ಯ-ಲಾ-ಯಿತು
ಕರೆ-ಯ-ಲಾಯಿ-ತೆಂದು
ಕರೆ-ಯ-ಲುದುಂರೆಸೆ-ಯಲು
ಕರೆ-ಯಲ್ಪಟ್ಟಿದೆ
ಕರೆ-ಯಲ್ಪಟ್ಟುದೇ
ಕರೆ-ಯಲ್ಪಡುತ್ತಾರೆ
ಕರೆ-ಯಲ್ಪಡುತ್ತಿತ್ತೆಂದು
ಕರೆ-ಯಲ್ಪಡುತ್ತಿದ್ದ
ಕರೆ-ಯಲ್ಪ-ಡುವ
ಕರೆ-ಯಿಸಿ-ಕೊಂಡು
ಕರೆ-ಯು-ತಿದ್ದು
ಕರೆ-ಯುತ್ತಾರೆ
ಕರೆ-ಯುತ್ತಿದ್ದ-ನೆಂದು
ಕರೆ-ಯುತ್ತಿದ್ದರು
ಕರೆ-ಯುತ್ತಿದ್ದ-ರೆಂದು
ಕರೆ-ಯು-ಲಾಗಿದೆ
ಕರೆ-ಯುವ
ಕರೆ-ಯು-ವು-ದರ
ಕರೆ-ಯು-ವುದು
ಕರೆಸಿ
ಕರೆ-ಸಿ-ಕೊಳ್ಳುವ
ಕರೆ-ಸುತ್ತಿದ್ದರು
ಕರೈ
ಕರ್ಕ-ನನ್ನು
ಕರ್ಚ್ಚಿ
ಕರ್ಣನು
ಕರ್ಣವೃತ್ತಾಂತದ
ಕರ್ಣಾಟ
ಕರ್ಣಾಟಕ
ಕರ್ಣಾಟ-ಕದ
ಕರ್ಣಾಟ-ಕರ್ಣಾಟ-ಕ-ಕರ್ನಾಟ-ಕರ್ನಾಟಕ
ಕರ್ಣಾಟೇಶ್ವರ-ರಾಯ
ಕರ್ಣ್ನಾಟಧ-ರಾಮ-ರೋತ್ತಂಸಂ
ಕರ್ಣ್ನಾ-ವತಂಸ
ಕರ್ತ-ನಾದ
ಕರ್ತ-ನಾದ-ಅಧಿ-ಕಾರಿ
ಕರ್ತ-ರಾದ
ಕರ್ತರು
ಕರ್ತವ್ಯ
ಕರ್ತವ್ಯ-ಗ-ಳನ್ನು
ಕರ್ತವ್ಯ-ವಾ-ಗಿತ್ತು
ಕರ್ತೃ
ಕರ್ತೃವೂ
ಕರ್ನಲ್
ಕರ್ನಾಟ
ಕರ್ನಾಟಕ
ಕರ್ನಾಟ-ಕಕ್ಕೆ
ಕರ್ನಾಟ-ಕದ
ಕರ್ನಾಟ-ಕ-ದಲ್ಲಿ
ಕರ್ನಾಟ-ಕ-ದಲ್ಲಿತ್ತೆಂಬು-ದಕ್ಕೆ
ಕರ್ನಾಟ-ಕ-ದಲ್ಲಿದ್ದ
ಕರ್ನಾಟ-ಕ-ದಲ್ಲಿ-ರುವ
ಕರ್ನಾಟ-ಕ-ದಲ್ಲಿವೆ
ಕರ್ನಾಟ-ಕರ್ಣ್ನಾಟಕ
ಕರ್ನಾಟ-ಕ-ವನ್ನು
ಕರ್ನಾಟ-ಕವು
ಕರ್ನಾಟಕೇ
ಕರ್ನಾಟ-ಕೇಂದು-ವಾಗಿದ್ದ-ನೆಂದು
ಕರ್ನಾಟ-ದಲ್ಲಿ
ಕರ್ನಾಟ-ಲಕ್ಷ್ಮೀ
ಕರ್ನಾಟ-ವನ್ನೂ
ಕರ್ನಾಟಿಕಾ
ಕರ್ನಾಟಿ-ಕಾದ
ಕರ್ನಾಟಿಕ್
ಕರ್ನಾಟೇಶ್ವರ-ರಾಯ
ಕರ್ನೂಲ್
ಕರ್ನ್ನರುಂಮಪ
ಕರ್ನ್ನಾಟ
ಕರ್ನ್ನಾಟಕ
ಕರ್ನ್ನಾಟಾಂಧ್ರ
ಕರ್ನ್ನಿಕೇಶ್ವರ
ಕರ್ಪಿನೊಡನುಣ್ಮುವ
ಕರ್ಪುರ
ಕರ್ಪೂರ-ದಾರ-ತಿಯ
ಕರ್ಪ್ಪೂರ-ದಾರ-ತಿಯ
ಕರ್ಬಪ್ಪು-ಕಳ್ಬಪ್ಪು
ಕರ್ಮ-ಗ-ರಾಜ
ಕರ್ಮ-ಗ-ಳಲ್ಲಿ
ಕರ್ಮ-ಟೇಶ್ವರ
ಕರ್ಮ-ನಾಥ
ಕರ್ಮ-ವಿಪಾಕ
ಕರ್ಮಾಚ್ಯುತೇಂದ್ರಃ
ಕರ್ಮ್ಮಗ-ರಾಚ
ಕರ್ಮ್ಮಗ-ರಾಚನು
ಕರ್ಮ್ಮ-ಟೇಶ್ವರ
ಕರ್ಮ್ಮ-ಠೇಶ್ವರ
ಕರ್ಮ್ಮನ
ಕರ್ಮ್ಮ-ವಿಪಾಕ
ಕರ್ವ್ವು-ನದ
ಕಱಿ-ಗಟ್ಟಿದ
ಕಱಿಯ-ಮದು
ಕಲ
ಕಲಂಬಳ್ಳೆಯ
ಕಲಉರ-ಕಲಿ-ಯೂರು
ಕಲ-ಕುಣಿ
ಕಲಚುರಿ
ಕಲಚೂ-ರಿಗ-ಳ-ವರೆಗೆ
ಕಲಬುರ್ಗಿ
ಕಲಬುರ್ಗಿಯ
ಕಲಬುರ್ಗಿ-ಯಲ್ಲಿ
ಕಲಬುರ್ಗಿ-ಯ-ವರ
ಕಲಬುರ್ಗಿ-ಯ-ವರು
ಕಲವುರ
ಕಲ-ವೂರ
ಕಲ-ವೂ-ರಾ-ಗಿದ್ದು
ಕಲಶ
ಕಲಸೋ-ಗರದ
ಕಲಸ್ತ-ವಾಡಿ
ಕಲಸ್ತ-ವಾಡಿಯೇ
ಕಲಹ-ಗಳು
ಕಲ-ಹದ
ಕಲಹ-ದಲ್ಲಿ
ಕಲಹಳಿ-ಯನು
ಕಲಹ-ವಾಗಿ
ಕಲಾ-ಕೃತಿ-ಯಾ-ಗಿದ್ದು
ಕಲಾ-ನಿಧಿ-ರು-ದಾರಶ್ರೀ
ಕಲಿ
ಕಲಿ-ಕಣಿ
ಕಲಿ-ಕಣಿ-ನಾಡ
ಕಲಿ-ಕಾಲ-ಧರ್ಮ್ಮ-ರಾಜ
ಕಲಿ-ಕಾಳೇಸ್ಮಿನ್ಗಂಗ-ಮಂಡಲ
ಕಲಿ-ಗಳಂಕುಸ
ಕಲಿ-ಗಳನೆ
ಕಲಿ-ಗ-ಳನ್ನು
ಕಲಿತ
ಕಲಿ-ತ-ನದ
ಕಲಿತು
ಕಲಿ-ತು-ಕೊಂಡ
ಕಲಿ-ದೇವ
ಕಲಿ-ದೇವನ
ಕಲಿ-ದೇವ-ನ-ಹಳ್ಳಿ
ಕಲಿ-ದೇವರ
ಕಲಿ-ದೇವ-ರ-ಇಂದಿನ
ಕಲಿ-ದೇವ-ರಿಗೆ
ಕಲಿ-ದೇವರು
ಕಲಿ-ನಾಗಪ್ಪ
ಕಲಿ-ನಾಗಪ್ಪ-ನ-ವರ
ಕಲಿ-ನಾಗಪ್ಪನು
ಕಲಿ-ನಾಥ-ಪುರ
ಕಲಿ-ನಿತಿ
ಕಲಿ-ನೊಳಂಬಾದಿ
ಕಲಿ-ನೊಳಂಬಾದಿ-ರಾಜ
ಕಲಿ-ನೊಳಂಬಾದಿ-ರಾಜನ
ಕಲಿ-ನೊಳಂಬಾದಿ-ರಾಜ-ನಾ-ಗಿದ್ದು
ಕಲಿ-ನೊಳಂಬಾದಿ-ರಾಜ-ನೆಂದು
ಕಲಿ-ನೊಳಂಬಾಧಿ-ರಾಜನು
ಕಲಿಯ
ಕಲಿ-ಯಂಣನ
ಕಲಿ-ಯಣ್ಣ
ಕಲಿ-ಯಣ್ಣನ
ಕಲಿ-ಯಣ್ಣನು
ಕಲಿ-ಯ-ರ-ಗಂಡ
ಕಲಿ-ಯುಗ
ಕಲಿ-ಯುಗ-ಭೀಮ-ನೆಂದು
ಕಲಿ-ಯುಗ-ಭೀಮಾರ್ಹ-ಗೇಹಾದಿ
ಕಲಿ-ಯುಗ-ಭೀಮಾರ್ಹ-ಸತ್ಫೂಜೆ
ಕಲಿ-ಯುಗ-ಮಾತ್ತಂಡನುಂ
ಕಲಿ-ಯೂರಿನ
ಕಲಿ-ಯೂರು
ಕಲಿ-ರತ್ನ-ಪಾಲನ
ಕಲಿ-ರತ್ನ-ಪಾಲ-ನನ್ನು
ಕಲಿ-ರತ್ನ-ಪಾಲನು
ಕಲಿ-ರತ್ನ-ಪಾ-ಳನ
ಕಲಿ-ವರ್ಷ-ವನ್ನು
ಕಲಿ-ಸೆಟ್ಟಿಯ
ಕಲಿ-ಹೃದುವ
ಕಲು-ಕಣಿ
ಕಲು-ಕಣಿ-ಕುಣಿ
ಕಲು-ಕಣಿ-ನಾಡ
ಕಲು-ಕಣಿ-ನಾಡಾಳ್ವಂ
ಕಲು-ಕಣಿ-ನಾ-ಡಿಗೆ
ಕಲು-ಕಣಿ-ನಾಡಿನ
ಕಲು-ಕಣಿ-ನಾಡು
ಕಲು-ಕಣಿಯ
ಕಲು-ಕಣಿ-ಯೆಪ್ಪತ್ತಕ್ಕೆ
ಕಲುಕ-ಪಣ-ವೊಂದು
ಕಲು-ಕರೆ
ಕಲು-ಕುಣಿ
ಕಲು-ವೆ-ಸನ
ಕಲೆ-ಗಾರ-ರಿಗೆ-ಇ-ವರು
ಕಲೆ-ಯಲ್ಲಿ
ಕಲೋಕ್ವಿಯಂಗೆ
ಕಲ್ಕಣಿ
ಕಲ್ಕಣಿ-ನಾಡ
ಕಲ್ಕಣಿ-ನಾಡು
ಕಲ್ಕಣಿ-ನಾಡೊಳ-ಗಣ
ಕಲ್ಕರೆ-ಕಲ್ಕುಣಿ
ಕಲ್ಕಱೆ
ಕಲ್ಕುಟಿಕ
ಕಲ್ಕುಟಿ-ಕರು
ಕಲ್ಕುಟಿಗ
ಕಲ್ಕುಟಿಗರು
ಕಲ್ಕುಣಿ
ಕಲ್ಕುಣಿ-ಕಾಲು-ಕಣಿ
ಕಲ್ಕುಣಿ-ನಾಡಿನ
ಕಲ್ಕುಣಿಯ
ಕಲ್ಕುಣಿ-ಯಲ್ಲಿ
ಕಲ್ನಟ್ಟು
ಕಲ್ನಾಟಿನ್ದು
ಕಲ್ನಾಟು
ಕಲ್ನಾಟ್ಟಾಗಿ
ಕಲ್ನಾಟ್ಟು
ಕಲ್ನಾಟ್ಟು-ಗೊಟ್ಟರ್
ಕಲ್ನಾಡ
ಕಲ್ನಾಡಾಗಿ
ಕಲ್ನಾಡಾಗೆ
ಕಲ್ನಾಡಿನ
ಕಲ್ನಾಡಿನಿಂದ
ಕಲ್ನಾಡು
ಕಲ್ನೆಲೆ-ದೇವ
ಕಲ್ಪಕುಜಾಳಿ
ಕಲ್ಪಗಂ
ಕಲ್ಪಗಂಕೊಂಡಾಳ್
ಕಲ್ಪ-ತರು-ವೆನಿಸಿ
ಕಲ್ಪದ್ರು-ಮನ
ಕಲ್ಪನೆ
ಕಲ್ಪ-ನೆಯ
ಕಲ್ಪನೆ-ಯನ್ನು
ಕಲ್ಪನೆ-ಯಾಗಿದೆ
ಕಲ್ಪ-ಯತ್
ಕಲ್ಪಯಿತ್ವಾ
ಕಲ್ಪ-ವೃಕ್ಷ-ನು-ಮಪ್ಪ
ಕಲ್ಪಿ-ಸ-ಲಾ-ಗಿತ್ತು
ಕಲ್ಪಿಸ-ಲಾಗಿತ್ತೆಂದು
ಕಲ್ಪಿ-ಸ-ಲಾ-ಯಿತು
ಕಲ್ಪಿ-ಸಲು
ಕಲ್ಪಿಸಿ
ಕಲ್ಪಿಸಿ-ಕೊಂಡು
ಕಲ್ಪಿಸಿ-ಕೊಡುವ
ಕಲ್ಪಿ-ಸಿದ
ಕಲ್ಪಿಸಿ-ದರು
ಕಲ್ಪಿ-ಸು-ವು-ದಿಲ್ಲ
ಕಲ್ಮಣ-ಕರ್
ಕಲ್ಯ
ಕಲ್ಯ-ದಲ್ಲಿ
ಕಲ್ಯಾಣ
ಕಲ್ಯಾಣ-ಚಾಲುಕ್ಯ
ಕಲ್ಯಾಣ-ಚಾಲುಕ್ಯರ
ಕಲ್ಯಾ-ಣದ
ಕಲ್ಯಾಣ-ದಲ್ಲಿ
ಕಲ್ಯಾಣಮ್ಮ
ಕಲ್ಯಾಣ-ರಾಯ
ಕಲ್ಯಾಣ-ವನ್ನು
ಕಲ್ಯಾಣಾಭ್ಯು-ದಯ
ಕಲ್ಯಾಣಾಯ
ಕಲ್ಯಾಣಿ
ಕಲ್ಯಾಣಿಯ
ಕಲ್ಯಾಣಿ-ಯನ್ನು
ಕಲ್ಯಾಣಿ-ಯಲ್ಲಿ
ಕಲ್ಲ
ಕಲ್ಲಂ
ಕಲ್ಲಂಪೂಜಿ-ಸದುಣ್ಡ-ರಪ್ಪೊಡೆ
ಕಲ್ಲ-ಕಂಡು
ಕಲ್ಲ-ಕೆರೆಯ
ಕಲ್ಲ-ಕೆಲಸಕ್ಕೆ
ಕಲ್ಲ-ಕೋಟೆ-ಯನ್ನು
ಕಲ್ಲ-ಗಳೇ
ಕಲ್ಲ-ಗವುಂಡನು
ಕಲ್ಲ-ಗ-ವುಡನು
ಕಲ್ಲ-ಗಾಣ-ವನ್ನು
ಕಲ್ಲ-ಗುಂಡಿಯ-ಹಳ್ಳಿ-ಯನು
ಕಲ್ಲ-ಡು-ಪಿನ
ಕಲ್ಲಣೆ
ಕಲ್ಲ-ತುಂಬಿನ
ಕಲ್ಲ-ದೂಮ-ರವು
ಕಲ್ಲ-ದೇಗುಲ
ಕಲ್ಲ-ದೇಗುಲ-ವನ್ನು
ಕಲ್ಲ-ನಟ್ಟು
ಕಲ್ಲ-ನಿ-ರಿ-ಸಿದಳ್
ಕಲ್ಲ-ನಿರಿಸುತ್ತಾನೆ
ಕಲ್ಲ-ನಿರಿ-ಸುತ್ತಾಳೆ
ಕಲ್ಲ-ನೆಟ್ಟು
ಕಲ್ಲ-ನೆ-ಡಿಸಿ
ಕಲ್ಲನ್ನ-ಡಿಸಿ
ಕಲ್ಲನ್ನು
ಕಲ್ಲ-ಪಲ್ಲಕ್ಕಿ-ಯನ್ನು
ಕಲ್ಲ-ಫಲವ
ಕಲ್ಲ-ಬಾ-ಗಿಲ
ಕಲ್ಲಬ್ಬ-ರಸಿಯ
ಕಲ್ಲ-ಮಸೀದಿಯ
ಕಲ್ಲಯ್ಯ-ಕಾಳಯ್ಯ
ಕಲ್ಲವ್ವೆ
ಕಲ್ಲವ್ವೆಯಾ
ಕಲ್ಲ-ಹಳ್ಳಿ
ಕಲ್ಲ-ಹಳ್ಳಿಯ
ಕಲ್ಲ-ಹಳ್ಳಿ-ಯನ್ನು
ಕಲ್ಲ-ಹಳ್ಳಿ-ಯ-ವರಗೂ
ಕಲ್ಲ-ಹಳ್ಳಿಯೇ
ಕಲ್ಲ-ಹಳ್ಳಿ-ಶಾ-ಸನ-ದಲ್ಲಿ
ಕಲ್ಲಾಗಿ-ರ-ಬಹದು
ಕಲ್ಲಿಂದ
ಕಲ್ಲಿ-ದೇವನ
ಕಲ್ಲಿನ
ಕಲ್ಲಿ-ನಕ್ರಮ
ಕಲ್ಲಿ-ನ-ಬೆಟ್ಟ
ಕಲ್ಲಿ-ನ-ಮೇಲೆ
ಕಲ್ಲಿ-ನಲ
ಕಲ್ಲಿ-ನಲ್ಲಿ
ಕಲ್ಲಿ-ನ-ಸೋಪಾನವೂ
ಕಲ್ಲಿ-ನಿಂದ
ಕಲ್ಲೀ-ಗುಂಡಿ
ಕಲ್ಲು
ಕಲ್ಲು-ಕಂಬ-ಗಳು
ಕಲ್ಲು-ಕಟ್ಟ-ಡದ್ದಾಗಿ-ರುತ್ತಿತ್ತು
ಕಲ್ಲು-ಕುಟಿಗ
ಕಲ್ಲು-ಕುಟಿಗರು
ಕಲ್ಲು-ಗಳ
ಕಲ್ಲು-ಗ-ಳನ್ನು
ಕಲ್ಲು-ಗ-ಳಿಂದ
ಕಲ್ಲು-ಗ-ಳಿಗೆ
ಕಲ್ಲು-ಗಳು
ಕಲ್ಲು-ಗಳೇ
ಕಲ್ಲು-ಗಾಣ-ವನ್ನು
ಕಲ್ಲು-ಗುಂಡಿ-ಯನ್ನು
ಕಲ್ಲು-ಗುಂಡು-ಗ-ಳನ್ನು
ಕಲ್ಲು-ಗುಡ್ಡೆ
ಕಲ್ಲು-ಚಪ್ಪಡಿ-ಗ-ಳನ್ನು
ಕಲ್ಲು-ನೆಟ್ಟು
ಕಲ್ಲು-ನೆಲ-ಹಾ-ಸನ್ನು
ಕಲ್ಲು-ಬಂಡೆ-ಗ-ಳಿಂದ
ಕಲ್ಲು-ಬಂಡೆ-ಗಳು
ಕಲ್ಲು-ಮಂಟಿ
ಕಲ್ಲು-ಮಂಟಿ-ಗ-ಳಿಂದ
ಕಲ್ಲು-ಮಣ್ಣನ್ನು
ಕಲ್ಲು-ಮರ-ಡಿ-ಯೊಳಾ-ಡು-ವುದೇ
ಕಲ್ಲು-ಮಸೀ-ತಿಗೆ
ಕಲ್ಲು-ಮಸೀದಿಗೆ
ಕಲ್ಲು-ಮಸೀದಿ-ಯನ್ನು
ಕಲ್ಲು-ಮಸೀದಿ-ಯೊಂದನ್ನು
ಕಲ್ಲೆಯ-ನಾಯಕ
ಕಲ್ಲೇ-ದೇವ-ರ-ಪುರ
ಕಲ್ಲೇಶ್ವರ
ಕಲ್ಲೇ-ಹದ
ಕಲ್ಲೇಹವು
ಕಲ್ಲೋಲ-ಲೀಲಃ
ಕಲ್ವೆಸೆದ
ಕಳ
ಕಳಂಕ
ಕಳಕು
ಕಳ-ಕೊಠಾರ
ಕಳಚಿ
ಕಳ-ಚುರಿ
ಕಳ-ಚುರಿ-ಗಳಿಗೂ
ಕಳ-ಚುರಿ-ಗಳು
ಕಳ-ಚುರ್ಯರ
ಕಳ-ಚೂರ್ಯರು
ಕಳತ್ರ-ಕಳುತ್ರ
ಕಳದ-ಲದ
ಕಳದು
ಕಳ-ದೆಱೆ
ಕಳ-ನಾಗಿ
ಕಳನಿ
ಕಳ-ನಿ-ಗದ್ದೆ-ಯನ್ನು
ಕಳ-ನೇರಿ-ಯಿಳಿವ-ನಾಯ-ಕ-ರ-ಗಂಡರುಂ
ಕಳಭ್ರ
ಕಳ-ಮನೆ
ಕಳ-ಲದ
ಕಳಲೆ
ಕಳ-ಲೆ-ನಾಡ
ಕಳ-ಲೆ-ನಾಡಿನ
ಕಳ-ಲೆಯ
ಕಳಲ್ತೂರಿನ
ಕಳ-ಶಂಬರ
ಕಳ-ಶಂಬರಂ
ಕಳ-ಶನೂ
ಕಳ-ಶ-ವಾಡಿ-ಯಲ್ಲಿದ್ದ
ಕಳ-ಶಹಾ-ರ-ರಾದರು
ಕಳಸ
ಕಳ-ಸತ್ತು
ಕಳ-ಸ-ದಂತಿದ್ದನು
ಕಳ-ಸ-ನಿರ್ಬಾಣ-ವಾಗಿ
ಕಳ-ಸ-ನಿರ್ವಾಣ-ಗೆಯ್ಸಿ
ಕಳಸ್ತ-ವಾಡಿ
ಕಳ-ಹಂಸ
ಕಳಾ
ಕಳಾಭ್ಯಸ್ತ-ರಿಗೆ
ಕಳಿಂಗ
ಕಳಿ-ಕಟ್ಟೆ-ಯಲ್ಲಿ
ಕಳಿ-ಯೂರ
ಕಳುಕು
ಕಳುತ್ರ
ಕಳುಹಿ-ಸ-ಲಾ-ಗಿತ್ತು
ಕಳುಹಿಸ-ಲಾಗುತ್ತಿದೆ
ಕಳುಹಿಸಿ
ಕಳುಹಿಸಿ-ಕೊಟ್ಟ-ನೆಂದು
ಕಳುಹಿ-ಸಿದ
ಕಳುಹಿಸಿ-ದನು
ಕಳುಹಿಸಿ-ದ-ನೆಂದು
ಕಳುಹಿಸಿ-ದಾಗ
ಕಳುಹಿ-ಸಿದ್ದ
ಕಳುಹಿಸುತ್ತಾನೆ
ಕಳುಹಿಸುತ್ತಾರೆ
ಕಳುಹು
ಕಳೆದ
ಕಳೆ-ದಿರ-ಬ-ಹುದು
ಕಳೆದು
ಕಳೆದು-ಕೊಂಡ
ಕಳೆದು-ಕೊಂಡ-ವರು
ಕಳೆದು-ಕೊಂಡಿತ್ತೆಂದು
ಕಳೆದು-ಕೊಂಡಿರ-ಬ-ಹುದು
ಕಳೆದು-ಕೊಂಡು
ಕಳೆದು-ಕೊಳ್ಳಲು
ಕಳೆದು-ಕೊಳ್ಳುತ್ತಿತ್ತು
ಕಳ್
ಕಳ್ಬಪ್ಪ
ಕಳ್ಬಪ್ಪಿನಾ
ಕಳ್ಬಪ್ಪು
ಕಳ್ಳ-ತನ
ಕಳ್ಳನ-ಕೆರೆ
ಕಳ್ಳರ
ಕಳ್ಳ-ರನ್ನು
ಕಳ್ಳ-ರಿಂದ
ಕಳ್ಳರು
ಕಳ್ಳರೂ
ಕಳ್ಳ-ರೊಡನೆ
ಕಳ್ಳರ್ವಾಡಿ
ಕಳ್ಳರ್ವಾಡಿ-ಯ-ಇಂದಿನ
ಕಳ್ಳರ್ವಾಡಿಯು
ಕಳ್ಳೀ-ಪುರ-ವನ್ನು
ಕಳ್ವಪ್ಪಿನಾ
ಕಳ್ವಪ್ಪು
ಕಳ್ವಪ್ಪು-ನಾಡು
ಕವ
ಕವ-ಡಯ್ಯ
ಕವಡಿ-ಯಿಲ್ಲ
ಕವಡೆಗೆ
ಕವ-ಣಿಕೆ
ಕವನ-ಗಳಿದ್ದು
ಕವರೈ
ಕವರೈ-ಗ-ವರೈ
ಕವಿ
ಕವಿ-ಕಂಠ
ಕವಿ-ಕಂದರ್ಪ
ಕವಿ-ಕಂದರ್ಪನು
ಕವಿ-ಕಂದರ್ಪರ
ಕವಿ-ಕಂದರ್ಪ್ಪನ
ಕವಿ-ಕುಳ-ತಿಳಕ
ಕವಿ-ಕುಳ-ತಿಳಕಂ
ಕವಿ-ಗಳ
ಕವಿ-ಗ-ಳಾಗಿದ್ದಂತೆ
ಕವಿ-ಗ-ಳಾದ
ಕವಿ-ಗಳಿಗೂ
ಕವಿ-ಗ-ಳಿಗೆ
ಕವಿ-ಗಳು
ಕವಿ-ಗಳು-ಶಾ-ಸನ
ಕವಿಗೆ
ಕವಿ-ಚಕ್ರ-ವರ್ತಿ
ಕವಿ-ಚರಿತೆ-ಕಾರರು
ಕವಿ-ಚರಿತೆ-ಕಾರರೂ
ಕವಿ-ಚರಿತೆ-ಯನ್ನು
ಕವಿ-ಚರಿತ್ರೆ-ಕಾರರು
ಕವಿತಾ
ಕವಿತ್ವದ
ಕವಿತ್ವಾ
ಕವಿ-ದಿದ್ದ
ಕವಿನಾ
ಕವಿ-ಬುಧಾರ್ತಿಂ
ಕವಿಯ
ಕವಿ-ಯಲು
ಕವಿ-ಯಾಗಿದ್ದ-ನೆಂದು
ಕವಿ-ಯಾಗಿದ್ದ-ನೆ-ಮದು
ಕವಿ-ಯಾಗಿ-ರ-ಲಿಲ್ಲ-ವೆಂದು
ಕವಿಯು
ಕವಿಯೂ
ಕವಿಯೆ
ಕವಿ-ಯೆಂದು
ಕವಿ-ಯೆಂದೂ
ಕವಿ-ಲಿಖಿತ-ವೆಂದು
ಕವಿ-ಲೆ-ಪಟ್ಟ
ಕವಿ-ಳಾ-ಸದುತ್ತರ
ಕವುಂಗಿನ
ಕವುಂಗು
ಕಷ್ಟ
ಕಷ್ಟ-ವಾಗುತ್ತದೆ
ಕಷ್ಟ-ವಿತ್ತು
ಕಸಕಡ್ಡಿಗಳೂ
ಕಸಬ
ಕಸಬಾ
ಕಸಬಾ-ದಿಂದ
ಕಸಲಗೆರ
ಕಸಲಗೆರೆ
ಕಸಲಗೆ-ರೆಗೆ
ಕಸಲಗೆ-ರೆಯ
ಕಸಲಗೆ-ರೆಯೇ
ಕಸಳ-ಗೆರೆಯ
ಕಸವಯ್ಯ
ಕಸಿ-ದನು
ಕಸಿದು-ಕೊಂಡರೂ
ಕಸಿದು-ಕೊಂಡು
ಕಸು-ಕೊಂಡರೂ
ಕಸು-ಕೊಂಡ್ರು
ಕಸು-ಬಿನವ-ರನ್ನು
ಕಸುಬು
ಕಸ್ಯ
ಕಹಳೆ
ಕಹಳೆ-ಯನ್ನು
ಕಹಿನ
ಕಾಂಚಿಕಾಂಚನ
ಕಾಂಚಿ-ಗೊಂಡ
ಕಾಂಚಿ-ಪುರ-ವನ್ನು
ಕಾಂಚೀ-ಪುರದ
ಕಾಂಚೀ-ಪುರ-ದಲ್ಲಿ
ಕಾಂಜೀ-ವರದ
ಕಾಂಡಂ
ಕಾಂತಯ್ಯ
ಕಾಂತಯ್ಯನ
ಕಾಂತಯ್ಯ-ನ-ವರ
ಕಾಂತಯ್ಯ-ನಿರ-ಬ-ಹುದು
ಕಾಂತಾ
ಕಾಂತಾ-ಮಧಿ-ಕದ್ವಿಲಾಸ
ಕಾಂತಿ-ರಾಜಿಷ್ಣು
ಕಾಂತೆ
ಕಾಂತೈಯ್ಯ-ನ-ವರ
ಕಾಂದು
ಕಾಂಭೋಜಭೋಜ-ಕಾಲಿಂಗಕ-ರಹಾ-ತಾದಿ-ಪಾರ್ಥಿವೈಃ
ಕಾಕಡೆ
ಕಾಕ-ತೀಯ
ಕಾಕ-ತೀಯ-ರನ್ನು
ಕಾಕನ-ಹಳ್ಳಿ
ಕಾಕಿಣಿ
ಕಾಕಿಣಿ-ಕಾ-ಕಣಿ-ಕಾ-ಗಣಿ
ಕಾಕಿವ-ರಾಜ್ಯ
ಕಾಕುಸ್ಥ-ವರ್ಮನ
ಕಾಗಡಿ
ಕಾಗಡಿ-ಯಿಂದ
ಕಾಗಣಿ
ಕಾಗಣಿಯ
ಕಾಗಣಿ-ಯರ
ಕಾಗದ-ಪತ್ರ
ಕಾಗೆ
ಕಾಗೆ-ಗಳು
ಕಾಚವ್ವೆ
ಕಾಚಿ-ದೇವ
ಕಾಚಿ-ದೇವಂಗೆ
ಕಾಚಿ-ದೇವನ
ಕಾಚಿ-ದೇವನು
ಕಾಚಿ-ನಾಯಕ
ಕಾಚಿಯ
ಕಾಚೀ-ದೇವ
ಕಾಚೀ-ದೇವನ
ಕಾಚೀ-ದೇವ-ನನ್ನು
ಕಾಚೀ-ದೇವ-ನಿಗೆ
ಕಾಚೀ-ದೇವನು
ಕಾಚೀ-ದೇವ-ನೆಂದು
ಕಾಚೆಯ
ಕಾಚೆಯ-ನಾಯ-ಕನ್ನು
ಕಾಚೇನ-ಹಳ್ಳಿ
ಕಾಚೇನ-ಹಳ್ಳಿಯ
ಕಾಚೇನ-ಹಳ್ಳಿಯೇ
ಕಾಚೈಯ್ಯ
ಕಾಜನ
ಕಾಟ
ಕಾಟಿ-ಗೌಡನು
ಕಾಡಂಕ-ಪುರ
ಕಾಡಂಕಾಖ್ಯ-ಪುರ-ವನ್ನು
ಕಾಡಕ್ಕಿ
ಕಾಡಕ್ಕಿ-ಯನ್ನು
ಕಾಡನ್ನು
ಕಾಡಯ-ನಾಯ-ಕನು
ಕಾಡಯ್ಯ
ಕಾಡವ-ರಾಯ
ಕಾಡಾನೆ
ಕಾಡಾನೆ-ಗಳು
ಕಾಡಾ-ನೆಯು
ಕಾಡಾರಂಭ
ಕಾಡಿಗೆ
ಕಾಡಿನ
ಕಾಡಿನಲ್ಲಿ
ಕಾಡಿನಲ್ಲಿದ್ದ
ಕಾಡಿಯೂ-ರನ್ನು
ಕಾಡಿ-ಯೂರಿನತ್ತ
ಕಾಡಿಲ-ಗೌಂಡ-ನೊಡನೆ
ಕಾಡಿ-ಸಲು
ಕಾಡು
ಕಾಡು-ಕೊತ್ತನ-ಹಳ್ಳಿ
ಕಾಡು-ಕೊತ್ತನ-ಹಳ್ಳಿಯ
ಕಾಡು-ಕೊತ್ತ-ಹಳ್ಳಿ
ಕಾಡು-ಗ-ಳನ್ನು
ಕಾಡು-ಗ-ಳಲ್ಲಿ
ಕಾಡು-ಗ-ಳಾಗಿ
ಕಾಡು-ಗಳಿದ್ದು
ಕಾಡು-ಗಳು
ಕಾಡುತ್ತಿದ್ದಿಲ್ಲ
ಕಾಡುಪ್ರಾಣಿ-ಗಳ
ಕಾಡುಪ್ರಾಣಿ-ಗಳಿವೆ
ಕಾಡುಪ್ರಾಣಿ-ಗಳೊಡನೆ
ಕಾಡು-ಮೆಣಸಿಗೆ-ಕಾಡು-ಮೆಣಸ
ಕಾಡು-ವಿಟ್ಟಿಯ
ಕಾಡು-ವಿಟ್ಟಿಯನ್ನು
ಕಾಡು-ವಿಟ್ಟಿಯು
ಕಾಡು-ಹಂದಿ
ಕಾಡೆಮ್ಮೆ
ಕಾಡೇ
ಕಾಣಚಿ-ಯಾಗಿ-ಕಾ-ಣಿಕೆ
ಕಾಣ-ಬ-ರುವ
ಕಾಣ-ಬಹು
ಕಾಣ-ಬಹು-ದಾಗಿದೆ
ಕಾಣ-ಬ-ಹುದು
ಕಾಣ-ಲಿಲ್ಲ-ವಲ್ಲ
ಕಾಣಿ
ಕಾಣಿ-ಕಾರನ
ಕಾಣಿಕೆ
ಕಾಣಿ-ಕೆ-ಗ-ಳನ್ನು
ಕಾಣಿ-ಕೆಗೆ
ಕಾಣಿ-ಕೆಯ
ಕಾಣಿ-ಕೆ-ಯನ್ನು
ಕಾಣಿ-ಕೆ-ಯನ್ನೂ
ಕಾಣಿ-ಕೆ-ಯಲ್ಲಿ
ಕಾಣಿ-ಕೆ-ಯಾಗಿ
ಕಾಣಿ-ಯನ್ನು
ಕಾಣಿ-ಸ-ಕೊಂಡರೂ
ಕಾಣಿ-ಸ-ಕೊಂಡಾಗ
ಕಾಣಿ-ಸ-ಕೊಂಡಿದೆ
ಕಾಣಿ-ಸ-ಕೊಳ್ಳುತ್ತದೆ
ಕಾಣಿ-ಸ-ಕೊಳ್ಳುತ್ತವೆ
ಕಾಣಿ-ಸ-ಕೊಳ್ಳುತ್ತಾನೆ
ಕಾಣಿಸಿ
ಕಾಣಿ-ಸಿ-ಕೊಂಡ-ರೆಂದು
ಕಾಣಿ-ಸಿ-ಕೊಂಡವು
ಕಾಣಿ-ಸಿ-ಕೊಂಡಿತು
ಕಾಣಿ-ಸಿ-ಕೊಂಡಿದೆ
ಕಾಣಿ-ಸಿ-ಕೊಂಡಿದ್ದಾನೆ
ಕಾಣಿ-ಸಿ-ಕೊಂಡಿದ್ದಾರೆ
ಕಾಣಿ-ಸಿ-ಕೊಂಡಿದ್ದು
ಕಾಣಿ-ಸಿ-ಕೊಂಡಿರ
ಕಾಣಿ-ಸಿ-ಕೊಂಡಿಲ್ಲ
ಕಾಣಿ-ಸಿ-ಕೊಂಡಿವೆ
ಕಾಣಿ-ಸಿ-ಕೊಂಡು
ಕಾಣಿ-ಸಿ-ಕೊಳ್ಳ-ಲಾರಂಭಿ
ಕಾಣಿ-ಸಿ-ಕೊಳ್ಳಲು
ಕಾಣಿ-ಸಿ-ಕೊಳ್ಳುತ್ತದೆ
ಕಾಣಿ-ಸಿ-ಕೊಳ್ಳುತ್ತ-ದೆಂದು
ಕಾಣಿ-ಸಿ-ಕೊಳ್ಳುತ್ತವೆ
ಕಾಣಿ-ಸಿ-ಕೊಳ್ಳುತ್ತಾನೆ
ಕಾಣಿ-ಸಿ-ಕೊಳ್ಳುತ್ತಾರೆ
ಕಾಣಿ-ಸಿ-ಕೊಳ್ಳುವ
ಕಾಣಿ-ಸಿ-ಕೊಳ್ಳುವಾತ
ಕಾಣಿ-ಸಿ-ಕೊಳ್ಳು-ವು-ದಿಲ್ಲ
ಕಾಣಿ-ಸಿ-ಕೊಳ್ಳು-ವುದು
ಕಾಣಿ-ಸು-ವು-ದಿಲ್ಲ-ವೆಂದು
ಕಾಣುತ್ತದೆ
ಕಾಣುತ್ತವೆ
ಕಾಣುತ್ತಿದ್ದರು
ಕಾಣುತ್ತಿದ್ದವು
ಕಾಣುತ್ತಿದ್ದು-ದನ್ನು
ಕಾಣುತ್ತಿದ್ದುದು
ಕಾಣುತ್ತಿಲ್ಲ
ಕಾಣು-ವು-ದಿಲ್ಲ
ಕಾಣು-ವುದು
ಕಾಣುವುವು
ಕಾಣೂರ್ಗಣ
ಕಾಣೂರ್ಗಣದ
ಕಾಣೂರ್ಗ-ಣವು
ಕಾಣೂರ್ಗ್ಗಣ
ಕಾಣೂರ್ಗ್ಗಣದ
ಕಾಣೆಮೆ
ಕಾಣೆ-ಯಾಗಿ-ರುವ
ಕಾದ
ಕಾದಲ್ಲಿ
ಕಾದಲ್ಲೇ
ಕಾದಾಡಿ
ಕಾದಾಡಿದ
ಕಾದಾಡಿದ್ದಾರೆ
ಕಾದಿ
ಕಾದಿ-ಕೊಂದು
ಕಾದಿದಂ
ಕಾದಿ-ದ-ರೆಂದು
ಕಾದಿ-ಬಿದ್ದಾಗ
ಕಾದಿ-ರಕ್ಷಿಸಿ
ಕಾದಿ-ಸತ್ತೊಡೆ
ಕಾದು-ವಂದು
ಕಾದುವಲಿ
ಕಾದು-ವಲ್ಲಿ
ಕಾದುವು-ದೆಂದು
ಕಾನನ
ಕಾನನಂ
ಕಾನಿ-ಕೆರೆ
ಕಾನೀನ-ನೆನಿಸಿ
ಕಾನೂರ್ಗ್ಗಣದ
ಕಾನ್ಸ್ಟಾಂಟಿನೋಪಲ್ಗೆ
ಕಾಪಾಡ-ಬೇಕಾ-ಯಿತು
ಕಾಪಾ-ಡಲು
ಕಾಪಾಡಿ-ಕೊಂಡು
ಕಾಪಾಡಿ-ದ-ವನು
ಕಾಪಾ-ಲಿಕ
ಕಾಪಾ-ಲಿಕರ
ಕಾಪಾ-ಲಿಕ-ರದು
ಕಾಪಾ-ಲಿಕ-ರಿಂದ
ಕಾಪಾ-ಲಿಕರು
ಕಾಪು
ಕಾಫು-ರನು
ಕಾಬೈ-ಯನು
ಕಾಮ
ಕಾಮಂಣನ
ಕಾಮ-ಕೋಟಿ-ದೇವಿ
ಕಾಮ-ಗ-ವುಡನ
ಕಾಮ-ಗೆರೆ
ಕಾಮ-ಗೆರೆಯ
ಕಾಮ-ಗೆರೆ-ಯಲ್ಲಿದೆ
ಕಾಮ-ತಮ್ಮ
ಕಾಮತ್
ಕಾಮತ್ರ-ವರು
ಕಾಮ-ದೇವನ
ಕಾಮ-ಧೇನು
ಕಾಮ-ಧೇನುವ
ಕಾಮ-ನ-ಕೆರೆ
ಕಾಮ-ನ-ಹಳ್ಳಿ
ಕಾಮ-ನಾಯ-ಕ-ನ-ಹಳ್ಳಿ
ಕಾಮಪ್ಪ-ನಾಯ-ಕನು
ಕಾಮಪ್ಪ-ನಾಯ-ಕ-ನೆಂಬ
ಕಾಮಯ
ಕಾಮಯ್ಯ
ಕಾಮಯ್ಯನು
ಕಾಮ-ಲ-ದೇವಿ
ಕಾಮ-ಲಾ-ದೇವಿ
ಕಾಮ-ಲಾ-ದೇವಿ-ಯನ್ನು
ಕಾಮಾಂಬಿಕಾ
ಕಾಮಾಂಬೆ
ಕಾಮಿ-ಕಬ್ಬೆ
ಕಾಮಿ-ಗೆರೆ
ಕಾಮಿನಿ
ಕಾಮಿನೀ-ಕಾಮ-ದೇವನುಂ
ಕಾಮಿನೀನಾಂ
ಕಾಮಿ-ಯಕ್ಕ
ಕಾಮಿ-ಯಕ್ಕನ
ಕಾಮಿ-ಯಕ್ಕ-ಳನ್ನು
ಕಾಮೆ-ನಾಯ-ಕ-ನ-ಹಳ್ಳಿ
ಕಾಮೆಯ
ಕಾಮೆಯ-ದಂಡ-ನಾಯ-ಕನು
ಕಾಮೆಯ-ದಣ್ನಾಯ-ಕರ
ಕಾಮೆಯ-ನಾಯಕ
ಕಾಮೆಯ-ನಾಯ-ಕನ
ಕಾಮೆಯ-ನಾಯ-ಕ-ನ-ಹಳ್ಳಿ
ಕಾಮೆಯ-ನಾಯ-ಕನು
ಕಾಮೆಯ-ನಾಯ-ಕರ
ಕಾಮೆಯಪ್ಪನು
ಕಾಮೋ-ಜ-ಗಳು
ಕಾಮೋಜವೋಜ
ಕಾಯಕ-ದಲ್ಲಿದ್ದರು
ಕಾಯಕ-ದವ-ನಿರ-ಬ-ಹುದು
ಕಾಯಲು
ಕಾಯುತ್ತಾ
ಕಾಯುತ್ತಿದ್ದ
ಕಾಯ್ದಿಟ್ಟ
ಕಾಯ್ದು-ಕೊಳ್ಳಲು
ಕಾರ
ಕಾರ-ಕೂ-ನನೂ
ಕಾರ-ಕೂ-ನರು
ಕಾರಕ್ಕೆ
ಕಾರ-ಗನ-ಹಳ್ಳಿ
ಕಾರ-ಡಿ-ಕೆರೆ
ಕಾರಣ
ಕಾರ-ಣ-ಕಥೆ
ಕಾರ-ಣ-ಕರ್ತ-ರಾದ
ಕಾರ-ಣಕ್ಕಾಗಿ
ಕಾರ-ಣಕ್ಕಾಗಿಯೆ
ಕಾರ-ಣಕ್ಕಾಗಿಯೇ
ಕಾರ-ಣ-ಗ-ಳನ್ನು
ಕಾರ-ಣ-ಗ-ಳಿಂದ
ಕಾರ-ಣ-ಗಳಿ-ಗಾಗಿ
ಕಾರ-ಣ-ಗಳೇನು
ಕಾರ-ಣತ್ವಮಬಾಧ್ಯತ್ವಮುಪಾ-ಯತ್ವುಪೇಯತಾ
ಕಾರ-ಣ-ದಿಂದ
ಕಾರ-ಣ-ದಿಂದಲೇ
ಕಾರ-ಣ-ದಿಂದಾಗಿ
ಕಾರ-ಣ-ದಿಂದಾಗಿಯೇ
ಕಾರ-ಣ-ನಾದ-ನೆಂದೂ
ಕಾರ-ಣ-ರಲ್ಲ
ಕಾರ-ಣ-ರಾದ
ಕಾರ-ಣ-ವನ್ನು
ಕಾರ-ಣ-ವನ್ನೂ
ಕಾರ-ಣ-ವಲ್ಲ
ಕಾರ-ಣ-ವಾಗಿತ್ತೆಂಬುದು
ಕಾರ-ಣ-ವಾಗಿ-ರ-ಬ-ಹುದು
ಕಾರ-ಣ-ವಾ-ದಂತಿದೆ
ಕಾರ-ಣ-ವಾದರೆ
ಕಾರ-ಣ-ವಾಯಿತು
ಕಾರ-ಣ-ವಾಯಿ-ತೆಂದು
ಕಾರ-ಣ-ವಿರ-ಬ-ಹುದು
ಕಾರ-ಣ-ವಿಲ್ಲದೆ
ಕಾರ-ಣ-ವೆಂದು
ಕಾರ-ಣ-ವೇನು
ಕಾರ-ಣ-ವೇ-ನೆಂದು
ಕಾರಣ್ಯದ
ಕಾರ-ಬಯ-ಲಿನ
ಕಾರ-ಬ-ಯಲು
ಕಾರ-ಯಿತ್ವಾಜಜಾಖ್ಯಕಾಂ
ಕಾರ-ಸ-ವಾಡಿ
ಕಾರಾ-ಗೃಹ-ದಲ್ಲಿಟ್ಟು
ಕಾರಿ-ಕುಡಿ
ಕಾರಿ-ಮಂಗಲ-ನಾಡ
ಕಾರು
ಕಾರುಂಣ್ಯದ
ಕಾರುಂಣ್ಯ-ಸಿಷ್ಯರು
ಕಾರು-ಕನ-ಕೊಳ್ಳ
ಕಾರು-ಗನ-ಹಳ್ಳಿ
ಕಾರು-ಗ-ಹಳ್ಳಿಯ
ಕಾರುಣ್ಯ
ಕಾರುಣ್ಯಂಗೆಯ್ದು
ಕಾರುಣ್ಯದ
ಕಾರುಣ್ಯದಿಂ
ಕಾರುಣ್ಯ-ದಿಂದ
ಕಾರು-ಹಳ್ಳಿ
ಕಾರೆ-ಯದ
ಕಾರೇ-ಪುರದ
ಕಾರೈ-ಕುಡಿ
ಕಾರೈ-ಕುಡಿಯ
ಕಾರೈಕ್ಕುಡಿ
ಕಾರ್ತಿ-ವೀರ್ಯ
ಕಾರ್ತಿ-ವೀರ್ಯ-ನನ್ನು
ಕಾರ್ತಿ-ವೀರ್ಯನು
ಕಾರ್ತೀಕ
ಕಾರ್ನ್ವಾಲಿ-ಸನು
ಕಾರ್ಮನ
ಕಾರ್ಯ
ಕಾರ್ಯ-ಕರ್ತ
ಕಾರ್ಯ-ಕರ್ತನ
ಕಾರ್ಯ-ಕರ್ತ-ನಾದ
ಕಾರ್ಯ-ಕರ್ತನೂ
ಕಾರ್ಯ-ಕರ್ತ-ನೊಬ್ಬನು
ಕಾರ್ಯಕೆ
ಕಾರ್ಯ-ಕೆ-ಕರ್ತ
ಕಾರ್ಯ-ಕೆ-ಕರ್ತ-ನಾದ
ಕಾರ್ಯ-ಕೆ-ಕರ್ತ-ರನ್ನು
ಕಾರ್ಯ-ಕೆ-ಕರ್ತ-ರಾದ
ಕಾರ್ಯ-ಕೆ-ಕರ್ತರು
ಕಾರ್ಯ-ಕೆ-ಕರ್ತ-ಹೆ-ಸರು
ಕಾರ್ಯಕ್ಕಾಗಿ
ಕಾರ್ಯಕ್ಕೆ
ಕಾರ್ಯಕ್ರಮ
ಕಾರ್ಯಕ್ರಮ-ಗಳ
ಕಾರ್ಯಕ್ರಮ-ಗಳಲಿ
ಕಾರ್ಯಕ್ರಮ-ವಾಗಿದೆ
ಕಾರ್ಯಕ್ಷೇತ್ರ-ವನ್ನಾಗಿ
ಕಾರ್ಯ-ಗಳ
ಕಾರ್ಯ-ಗ-ಳನ್ನು
ಕಾರ್ಯ-ಗ-ಳಲ್ಲಿ
ಕಾರ್ಯ-ಗ-ಳಿಗೆ
ಕಾರ್ಯ-ಗಳು
ಕಾರ್ಯ-ಗಳೂ
ಕಾರ್ಯ-ಗಳೆ-ರಡನ್ನೂ
ಕಾರ್ಯದ
ಕಾರ್ಯ-ದಲ್ಲಿ
ಕಾರ್ಯ-ನಿಮಿತ್ತ
ಕಾರ್ಯ-ನಿರ್ವ-ಹಣೆಗೆ
ಕಾರ್ಯ-ನಿರ್ವ-ಹಣೆಯ
ಕಾರ್ಯ-ನಿರ್ವಹಿಸಿದ್ದಾರೆ
ಕಾರ್ಯ-ನಿರ್ವಹಿಸುತ್ತಿದ್ದ
ಕಾರ್ಯ-ನಿರ್ವಹಿಸುತ್ತಿದ್ದ-ರೆಂದು
ಕಾರ್ಯ-ಪೂರ್ಣ-ವಾಗಿ
ಕಾರ್ಯ-ಮಠದ
ಕಾರ್ಯ-ಮಠವು
ಕಾರ್ಯ-ವ-ನಿರ್ವಹಿಸುತ್ತಿದ್ದ-ರೆಂದು
ಕಾರ್ಯ-ವನ್ನು
ಕಾರ್ಯವು
ಕಾರ್ಯ-ಶಾಲಿ-ಗ-ಳಾಗಿದ್ದು
ಕಾರ್ಯಸ್ಥಾನ-ವನ್ನಾಗಿ
ಕಾರ್ಯಾ-ಲ-ಯದ
ಕಾರ್ಯಾವಧಿಯು
ಕಾಲ
ಕಾಲಂ
ಕಾಲಂಕರ್ಚ್ಚಿ
ಕಾಲಂಗಳಲಿ
ಕಾಲ-ಆಂಗ್ಲ-ಭಾಷೆಯ
ಕಾಲ-ಕಾಲಕ್ಕೆ
ಕಾಲಕ್ಕಂ
ಕಾಲಕ್ಕಾ-ಗಲೇ
ಕಾಲಕ್ಕಿಂತ
ಕಾಲಕ್ಕೂ
ಕಾಲಕ್ಕೆ
ಕಾಲಕ್ಕೇ
ಕಾಲಕ್ರಮ-ದಲ್ಲಿ
ಕಾಲಕ್ರಮೇಣ
ಕಾಲ-ಗ-ಳಲ್ಲಿ
ಕಾಲ-ಘಟ್ಟ-ದಲ್ಲಿ
ಕಾಲ-ಘಟ್ಟ-ದಿಂದ
ಕಾಲಜ್ಞಾನ
ಕಾಲಜ್ಞಾನ-ವನ್ನು
ಕಾಲದ
ಕಾಲ-ದಂದು
ಕಾಲ-ದ-ಮೂರ-ನೆಯ
ಕಾಲ-ದಲ
ಕಾಲ-ದಲಿ
ಕಾಲ-ದಲ್ಲಿ
ಕಾಲ-ದಲ್ಲಿದ್ದ
ಕಾಲ-ದಲ್ಲಿದ್ದನು
ಕಾಲ-ದಲ್ಲಿದ್ದ-ವ-ನೆಂದು
ಕಾಲ-ದಲ್ಲಿಯೂ
ಕಾಲ-ದಲ್ಲಿಯೇ
ಕಾಲ-ದಲ್ಲೂ
ಕಾಲ-ದಲ್ಲೇ
ಕಾಲ-ದ-ವ-ನಿರ-ಬ-ಹುದು
ಕಾಲ-ದ-ವನು
ಕಾಲ-ದ-ವ-ನೆಂದು
ಕಾಲ-ದ-ವರೆ-ಗಿನ
ಕಾಲ-ದ-ವರೆಗೂ
ಕಾಲ-ದ-ವರೆಗೆ
ಕಾಲ-ದ-ಸುಪ್ರ-ಸಿದ್ಧ
ಕಾಲ-ದಿಂದ
ಕಾಲ-ದಿಂದಲೂ
ಕಾಲ-ದಿಂದಲೇ
ಕಾಲ-ದೊಳ-ಗಾ-ಗಿಯೇ
ಕಾಲದ್ದಾ-ಗಿದ್ದು
ಕಾಲದ್ದಾಗಿ-ರುವು-ದ-ರಿಂದ
ಕಾಲದ್ದಿರ-ಬಹು-ದೆಂದು
ಕಾಲದ್ದಿರ-ಬಹುದೇ
ಕಾಲದ್ದು
ಕಾಲದ್ದೆಂದು
ಕಾಲದ್ಲಲಿ
ಕಾಲ-ನಿ-ರೂಪ-ಣೆ-ಯಲ್ಲಿ
ಕಾಲನ್ನು
ಕಾಲ-ಭೈ-ರವೇಶ್ವರ
ಕಾಲ-ಮಾನ-ಗ-ಳಲ್ಲಿ
ಕಾಲಮ್ರಿತ್ತು
ಕಾಲ-ರಾಜ
ಕಾಲ-ಳ-ದೇವಿಯ
ಕಾಲ-ಳೆ-ಗೊಟ್ಟ
ಕಾಲ-ಳೇಶ್ವರ
ಕಾಲ-ವನ್ನು
ಕಾಲ-ವ-ಶ-ದಿಂದ
ಕಾಲ-ವಾಗಿದೆ
ಕಾಲ-ವಾದ
ಕಾಲ-ವಾದ-ನೆಂದು
ಕಾಲ-ವಿರ-ಬ-ಹುದು
ಕಾಲವು
ಕಾಲವೂ
ಕಾಲವೆ
ಕಾಲ-ವೆಂದು
ಕಾಲ-ವೆ-ಯನ್ನು
ಕಾಲ-ವೆಯೇ
ಕಾಲವೇ
ಕಾಲ-ಶಕ್ತಿ
ಕಾಲಸ
ಕಾಲಾಂತರ-ದಲ್ಲಿ
ಕಾಲಾಂತರ-ದಿಂದಲೂ
ಕಾಲಾನು-ಕಾಲಕ್ಕೆ
ಕಾಲಾನುಕ್ರಮ-ವಾಗಿ
ಕಾಲಾಳು-ಗಳ
ಕಾಲಾವಧಿ
ಕಾಲಾವಧಿಯ
ಕಾಲಿ
ಕಾಲಿಟ್ಟುದು
ಕಾಲು-ಕಣಿಯ
ಕಾಲು-ಕುಣಿಯ
ಕಾಲು-ಕುಣಿ-ಯಲ್ಲಿ
ಕಾಲುನ-ಡಿಗೆ-ಯಲ್ಲಿ
ಕಾಲು-ಭಾಗ
ಕಾಲು-ಭಾಗ-ವನ್ನು
ಕಾಲುವಲಿ-ಗಳು
ಕಾಲು-ವಳಿ
ಕಾಲು-ವಳಿ-ಗ-ಳಾದ
ಕಾಲು-ವಳಿ-ಯಾಗಿ
ಕಾಲು-ವಳ್ಳಿ
ಕಾಲು-ವಳ್ಳಿಗ
ಕಾಲು-ವಳ್ಳಿ-ಗಳ
ಕಾಲು-ವಳ್ಳಿ-ಗ-ಳನ್ನು
ಕಾಲು-ವಳ್ಳಿ-ಗ-ಳನ್ನು-ಹೆ-ಸರಿ-ಸಿದೆ
ಕಾಲು-ವಳ್ಳಿ-ಗ-ಳಾದ
ಕಾಲು-ವಳ್ಳಿ-ಗಳು
ಕಾಲು-ವಳ್ಳಿಯ
ಕಾಲು-ವಳ್ಳಿ-ಯಾದ
ಕಾಲುವೆ
ಕಾಲುವೆ-ಗಳ
ಕಾಲುವೆ-ಗ-ಳನ್ನು
ಕಾಲುವೆ-ಗ-ಳಲ್ಲಿ
ಕಾಲುವೆ-ಗ-ಳಿಗೆ
ಕಾಲುವೆ-ಗಳಿದ್ದ-ವೆಂದೂ
ಕಾಲುವೆ-ಗಳಿದ್ದಿರ-ಬ-ಹುದು
ಕಾಲುವೆ-ಗಳಿದ್ದು-ದನ್ನು
ಕಾಲುವೆ-ಗಳು
ಕಾಲುವೆ-ಗಳೂ
ಕಾಲುವೆ-ಗ-ಳೆಂದೂ
ಕಾಲುವೆಗೂ
ಕಾಲುವೆಗೆ
ಕಾಲುವೆಯ
ಕಾಲುವೆ-ಯ-ಗ-ಳಿಂದ
ಕಾಲುವೆ-ಯನ್ನು
ಕಾಲುವೆ-ಯಿಂದ
ಕಾಲುವೆಯು
ಕಾಲುವೆ-ಯೊಳಗೆ
ಕಾಲು-ಸೇತುವೆ
ಕಾಲೇಜಿ-ನಲ್ಲಿ
ಕಾಲೇಜು
ಕಾಲೇ-ಹಳ್ಳಿ
ಕಾಲೈಯ-ನಾಯಕ
ಕಾಲೋಗ್ರ-ಗಣ
ಕಾಲ್
ಕಾಲ್ಗಾ-ಹಿನ
ಕಾಲ್ದ-ಳದ
ಕಾಲ್ಪನಿಕ
ಕಾಲ್ವೆ
ಕಾಳ-ಕೊತ್ತನ-ಹಳ್ಳಿ
ಕಾಳಗ
ಕಾಳ-ಗ-ದಲಿ
ಕಾಳ-ಗ-ದಲ್ಲಿ
ಕಾಳ-ಗ-ವಾಗಿ-ರ-ಬಹದು
ಕಾಳ-ಗಾವುಂಡನು
ಕಾಳ-ಗೌಡ
ಕಾಳ-ಗೌಡನ
ಕಾಳ-ಜೀಯ-ನಿಗೆ
ಕಾಳ-ಜೀಯ-ನು-ಕಾಳೋಜ
ಕಾಳನ
ಕಾಳನ್ರಿಪಾ-ಳನ
ಕಾಳ-ಬೋವ-ನ-ಹಳ್ಳಿ
ಕಾಳ-ಬೋವ-ನ-ಹಳ್ಳಿ-ಯನ್ನು
ಕಾಳಮ್ಮ-ನ-ವರ
ಕಾಳಯ್ಯ
ಕಾಳಯ್ಯಂ
ಕಾಳಯ್ಯ-ನನ್ನು
ಕಾಳಯ್ಯನು
ಕಾಳ-ರಾಜ-ನನ್ನು
ಕಾಳ-ಲ-ದೇವಿ
ಕಾಳ-ಲ-ದೇವಿ-ಯರ
ಕಾಳ-ಲೇಶ್ವರ
ಕಾಳ-ಲೇಶ್ವರ-ದೇವ
ಕಾಳ-ಲೇಶ್ವರ-ದೇವರ
ಕಾಳ-ಹಸ್ತಿ
ಕಾಳಾಂಚಿಯ
ಕಾಳಾಂತಕ-ನಲುತೆ
ಕಾಳಾ-ಚಾರಿ
ಕಾಳಾ-ಚಾರಿಯ
ಕಾಳಾ-ಮುಖ
ಕಾಳಾ-ಮುಖಕ್ಕೆ
ಕಾಳಾ-ಮುಖ-ಗಳು
ಕಾಳಾ-ಮುಖರ
ಕಾಳಾ-ಮುಖ-ರಿಗೂ
ಕಾಳಾ-ಮುಖ-ರಿಗೆ
ಕಾಳಾ-ಮುಖರು
ಕಾಳಾ-ಮುಖ-ರೆಂದೂ
ಕಾಳಾ-ಮುಖವು
ಕಾಳಾ-ಮುಖಾ-ಚಾರ್ಯರು
ಕಾಳಿ
ಕಾಳಿಂಗನ-ಹಳ್ಳಿ
ಕಾಳಿಂಗ-ರಾಮ-ನ-ಹಳ್ಳಿಯ
ಕಾಳಿಂಗ-ರಾಯ-ನೆಂಬು-ವ-ವನು
ಕಾಳಿ-ಗ-ರಾಮ-ನ-ಹಳ್ಳಿ
ಕಾಳಿ-ಗೆರೆ
ಕಾಳಿ-ಭಕ್ತನ
ಕಾಳಿಯಂ
ಕಾಳಿ-ಯಕ್ಕ
ಕಾಳಿಯು
ಕಾಳಿ-ಯೆಂಬ
ಕಾಳು-ಪಳ್ಳಿ
ಕಾಳು-ಪಳ್ಳಿ-ಗ-ಳನ್ನು
ಕಾಳು-ಪಳ್ಳಿ-ಗಳು
ಕಾಳೆಗ-ರಾಮ-ನ-ಹಳ್ಳಿ
ಕಾಳೆಗ-ರಾಮ-ನ-ಹಳ್ಳಿ-ಯನ್ನು
ಕಾಳೆಯ
ಕಾಳೆಯನ
ಕಾಳೆಯ-ನಾಯಕ
ಕಾಳೆಯ-ನಾಯ-ಕನ
ಕಾಳೆಯ-ನಾಯ-ಕ-ನಿಗೆ
ಕಾಳೆಯ-ನಾಯ-ಕನು
ಕಾಳೆಯ-ನಾಯ-ಕನೂ
ಕಾಳೇಯ
ಕಾಳ್ಳಭೋಜ
ಕಾವ
ಕಾವ-ಗಾವುಂಡನ
ಕಾವಡಿ
ಕಾವ-ಡಿ-ಯನ್ನು
ಕಾವಣ್ಣ
ಕಾವಣ್ಣನು
ಕಾವಣ್ಣ-ನೆಂಬು-ವ-ವನೂ
ಕಾವ-ನ-ಹಳ್ಳಿ
ಕಾವ-ನ-ಹಳ್ಳಿ-ಯನ್ನು
ಕಾವಪ್ಪ
ಕಾವ-ಬಾನ
ಕಾವ-ರಾಜ
ಕಾವ-ರಾಜ-ನಿಗೆ
ಕಾವಲಿ
ಕಾವ-ಲಿ-ಗಳ
ಕಾವ-ಲಿ-ದೆಱೆ
ಕಾವಲು
ಕಾವ-ಲು-ಗಾರ
ಕಾವ-ಲು-ಗಾರ-ನಿರ-ಬ-ಹುದು
ಕಾವ-ಲು-ಗಾರರು
ಕಾವಾಡಿ
ಕಾವಿ-ಧಾರಿ-ಯಾಗಿ
ಕಾವೇಟಿ-ರಂಗ
ಕಾವೇರಿ
ಕಾವೇರಿಗೆ
ಕಾವೇರಿ-ನದಿ
ಕಾವೇರಿ-ನದಿಗೆ
ಕಾವೇರಿಯ
ಕಾವೇರಿ-ಯನ್ನು
ಕಾವೇರಿ-ಯಿಂದ
ಕಾವೇರಿಯು
ಕಾವೇರೀ
ಕಾವ್ಯ
ಕಾವ್ಯ-ಗ-ಳಲ್ಲಿ
ಕಾವ್ಯ-ಗಳು
ಕಾವ್ಯದ
ಕಾವ್ಯ-ದಿಂದ
ಕಾವ್ಯ-ಮಯ-ವಾಗಿವೆ
ಕಾವ್ಯ-ವನ್ನು
ಕಾವ್ಯ-ವನ್ನೂ
ಕಾವ್ಯ-ಸೌಂದರ್ಯ-ದಿಂದ
ಕಾವ್ಯೇಷು
ಕಾಶಿ-ಯಲ್ಲಿ
ಕಾಶಿ-ಯಿಂದ
ಕಾಶಿ-ವಿಶ್ವ-ನಾಥ
ಕಾಶಿ-ವಿಶ್ವೇಶ್ವರ
ಕಾಶೀ
ಕಾಶೀ-ರಾವ್
ಕಾಶೀ-ರಾವ್ಗೆ
ಕಾಶ್ಮೀರ
ಕಾಶ್ಮೀರ-ದಿಂದ
ಕಾಶ್ಯಪ
ಕಾಶ್ಯಪ-ಗೋತ್ರ
ಕಾಶ್ಯಪ-ಗೋತ್ರದ
ಕಾಶ್ಯಪಾನ್ವಯಃ
ಕಾಶ್ಯಪೋ
ಕಾಷ್ಟ-ಕರ್ಮ್ಮ
ಕಿಂಚಿತ್
ಕಿಕ್ಕೆರಿ
ಕಿಕ್ಕೇರಮ್ಮನ
ಕಿಕ್ಕೇರಮ್ಮ-ನಾ-ಗಿದ್ದಾಳೆ
ಕಿಕ್ಕೇರಮ್ಮ-ನಾ-ಗಿದ್ದಾಳೆ-ಮಹಾ-ಲಕ್ಷ್ಮಿ
ಕಿಕ್ಕೇರಿ
ಕಿಕ್ಕೇರಿಕ್ಕೆ
ಕಿಕ್ಕೇರಿಗೆ
ಕಿಕ್ಕೇರಿನ್ನು
ಕಿಕ್ಕೇರಿ-ಪುರ-ದಲ್ಲಿ
ಕಿಕ್ಕೇರಿಯ
ಕಿಕ್ಕೇರಿ-ಯನ್ನು
ಕಿಕ್ಕೇರಿ-ಯ-ಪುರ-ದಲ್ಲಿ
ಕಿಕ್ಕೇರಿ-ಯಲ್ಲಿ
ಕಿಕ್ಕೇರಿ-ಯ-ವೀ-ಡಿಗೆ
ಕಿಕ್ಕೇರಿ-ಯಿಂದ
ಕಿಕ್ಕೇರಿಯು
ಕಿಗ್ಗದ
ಕಿಚ್ಚನ್ನು
ಕಿಚ್ಚು
ಕಿಚ್ಚು-ಹಾಯ್ದು
ಕಿಟ್ಟೆಲ್
ಕಿಟ್ಟೆಲ್ರ-ವರು
ಕಿಡ-ದಂತೆ
ಕಿಡಿಸಿದ
ಕಿಡಿಸಿ-ದವಂ
ಕಿತ್ತ-ಕನೆರೆ
ಕಿತ್ತನ-ಕೆರೆ
ಕಿತ್ತನ-ಕೆರೆ-ಯನ್ನು
ಕಿತ್ತಪ್ಪ
ಕಿತ್ತಪ್ಪ-ದಂಡ-ನಾಯ-ಕನು
ಕಿತ್ತವ-ಗಟ್ಟ
ಕಿತ್ತಿಪ್ಪ
ಕಿತ್ತಿಸಿ
ಕಿತ್ತು
ಕಿತ್ತು-ಕೊಂಡನು
ಕಿತ್ತು-ಕೊಂಡಿ-ದಿರ್ಚ್ಚಿದ
ಕಿತ್ತು-ಕೊಂಡು
ಕಿತ್ತೂ-ರನ್ನು
ಕಿಮೀ
ಕಿರಂಗೂ-ರಾಗಿ-ರ-ಬ-ಹುದು
ಕಿರಂಗೂ-ರಿಗೆ
ಕಿರಂಗೂರಿನ
ಕಿರಂಗೂರು
ಕಿರಂಗೂರೇ
ಕಿರಂಜಿ-ಪೆರು-ಮಾಳರು
ಕಿರಂಜಿ-ಪೆರು-ಮಾಳ್
ಕಿರ-ಗತೂರ
ಕಿರಗಸೂರು
ಕಿರ-ಗುಂದೂರ
ಕಿರಣ
ಕಿರಾತ
ಕಿರಾತ-ಕ-ರನ್ನು
ಕಿರಿ-ದಾದ
ಕಿರಿಯ
ಕಿರಿಯದು
ಕಿರಿಯ-ನಾದುದ-ರಿಂದ
ಕಿರಿಯ-ಪುತ್ರ
ಕಿರಿಯಯ್ಯ
ಕಿರಿಯ-ವಯಸ್ಸಿ-ನಲ್ಲೇ
ಕಿರಿಯ-ಸಮ-ಕಾಲೀನರೂ
ಕಿರಿಯೈಯ-ಗಳು
ಕಿರೀಟದ
ಕಿರೀಟ-ಮಗ್ರ್ಯಂ
ಕಿರು
ಕಿರು-ಕಾವ-ಲನ್ನು-ಸೇನೆ
ಕಿರು-ಕಾವಲು
ಕಿರು-ಕುಳ
ಕಿರು-ಕುಳವು
ಕಿರು-ಕೆರೆ
ಕಿರು-ಕೆರೆ-ಗ-ಳನ್ನು
ಕಿರು-ಕೆರೆ-ಗ-ಳನ್ನೂ
ಕಿರು-ಕೆರೆ-ಗಳು
ಕಿರು-ಕೆರೆಯ
ಕಿರು-ಕೆರೆ-ಯನ್ನೂ
ಕಿರು-ಕೆರೆ-ಯೊಳ-ಗಣ
ಕಿರು-ಕೊಣ್ಣಿನ್ದ
ಕಿರು-ಗಣಬ್ಬೆ
ಕಿರು-ಗತೂರ
ಕಿರು-ಗ-ವರ
ಕಿರು-ಗ-ವರೆ
ಕಿರು-ಗ-ವರೆಅ
ಕಿರು-ಗಾವ-ಲನ್ನು
ಕಿರು-ಗಾವ-ಲಾಗಿದೆ
ಕಿರು-ಗಾವ-ಲಿನ
ಕಿರು-ಗಾವ-ಲಿ-ನಲ್ಲಿ
ಕಿರು-ಗಾವಲಿರ-ಬ-ಹುದು
ಕಿರು-ಗಾವಲು
ಕಿರು-ಗುಂದೂರ
ಕಿರು-ಗೆರೆ
ಕಿರು-ದೆರೆ
ಕಿರು-ದೆಱೆ
ಕಿರು-ನ-ಗರ
ಕಿರು-ನಗ-ರವು
ಕಿರು-ನಗ-ರವೇ
ಕಿರು-ಬಳ್ಳಿ-ಯೂರು
ಕಿರು-ಭಾಗ
ಕಿರು-ವೆಳ್ನ-ಗರ
ಕಿರು-ವೆಳ್ನ-ಗರದ
ಕಿರು-ವೆಳ್ನ-ಗರ-ವನ್ನು
ಕಿರು-ವೆಳ್ನಗ-ರವು
ಕಿರು-ವೆಳ್ನ-ನ-ಗರ
ಕಿಳಕಂಬ-ರದಂ
ಕಿಳಲೆ
ಕಿಳಲೆ-ನಾಡ
ಕಿಳಲೆ-ನಾಡನ್ನು
ಕಿಳಲೆ-ನಾಡು
ಕಿಳಲೆ-ಸಹಸ್ರದ
ಕಿಳಲೈ
ಕಿಳಲೈ-ನಾಟ್ಟ
ಕಿಳಿನೀ
ಕಿಳಿನೀ-ನದಿಗೆ
ಕಿಳುವ-ನ-ಹಳ್ಳಿ
ಕಿಳುವ-ನ-ಹಳ್ಳಿ-ಕೆರೆ
ಕಿಳ್ಚುವೊಕ್ಕಂ
ಕಿವುಡನೂ
ಕಿಸು-ಕಾಡು
ಕಿಸುಕಾಡೆಪ್ಪತ್ತು
ಕೀರ್ತನೆ-ಯಲ್ಲಿ
ಕೀರ್ತಿ
ಕೀರ್ತಿ-ಅರ-ಸರ
ಕೀರ್ತಿ-ಗಳು
ಕೀರ್ತಿ-ತ-ನಾಗಿದ್ದಾನೆ
ಕೀರ್ತಿ-ತನೂ-ಭವ
ಕೀರ್ತಿ-ದೇವ
ಕೀರ್ತಿ-ದೇವನ
ಕೀರ್ತಿ-ದೇವ-ನಿಗೆ
ಕೀರ್ತಿ-ದೇವನು
ಕೀರ್ತಿ-ನನ್ದಾ-ಚಾರ್ಯ
ಕೀರ್ತಿ-ನಾ-ರಾಯ-ಣ-ದೇವರ
ಕೀರ್ತಿ-ನಾ-ರಾಯ-ಣ-ದೇ-ವ-ರಿಗೆ
ಕೀರ್ತಿ-ನಾ-ರಾಯ-ಣನ
ಕೀರ್ತಿ-ನಾ-ರಾಯ-ಣ-ರಾಯ
ಕೀರ್ತಿ-ಮಾನ್
ಕೀರ್ತಿಯ
ಕೀರ್ತಿ-ಯ-ರಸ
ಕೀರ್ತಿ-ಯ-ರ-ಸನ
ಕೀರ್ತಿ-ಯ-ರಸ-ನ-ನನ್ನು
ಕೀರ್ತಿ-ಯ-ರಸ-ನಿಗೆ
ಕೀರ್ತಿ-ಯ-ರಸರ
ಕೀರ್ತಿ-ಯ-ರಸು-ಗಳು
ಕೀರ್ತಿಯು
ಕೀರ್ತಿ-ರಾಜನ
ಕೀರ್ತಿ-ರಾಜು-ವಿಗೆ
ಕೀರ್ತಿರ್ಹರತಿ
ಕೀರ್ತಿ-ಲಕ್ಷ್ಮೀ
ಕೀರ್ತಿ-ವಂತ-ನಾಗಿದ್ದ-ನೆಂದು
ಕೀರ್ತಿ-ವಂತ-ನೆಂದೂ
ಕೀರ್ತಿ-ವಿಲಾಸಿ-ಯಾಗಿ
ಕೀರ್ತಿ-ಸ-ಮುದ್ರ
ಕೀರ್ತಿ-ಸೆಟ್ಟಿ
ಕೀರ್ತೌ
ಕೀರ್ತ್ತಿಗ
ಕೀರ್ತ್ತಿ-ದೇವ
ಕೀರ್ತ್ತಿ-ದೇವಂಗಳು
ಕೀರ್ತ್ತಿ-ರಾಜನ
ಕೀರ್ತ್ತ್ಯಾಶ್ಯಾಮಿಕಾ-ಪನು-ದೇಭುವಃ
ಕೀರ್ತ್ಯಂಗನ-ವಲ್ಲಭ
ಕೀರ್ತ್ಯಾವ-ತಾರ-ವೆನ್ತೆಂದಡೆ
ಕೀರ್ತ್ರಯಾಂ
ಕೀರ್ಮಾನಿ
ಕೀಲಕ
ಕೀಲಾರ
ಕೀಳಾಗಿ
ಕೀಳಿನಿ
ಕೀಳಿಸಿ
ಕೀಳು
ಕೀಳೇತಮಂ
ಕೀಳೇರಿ
ಕೀಳೇ-ರಿಯ
ಕೀಳೇರಿ-ಯೊಳು
ಕೀಳ್
ಕೀಳ್ಪಟ್ಟು
ಕೀಳ್ವಾಳ-ಗೋವಂ
ಕೀೞ್ಗುಂಟೆ
ಕುಂಕುಮಕ್ಕೆ
ಕುಂಚ-ಗನ-ಹಳ್ಳಿ
ಕುಂಚದ-ಹಳ್ಳಿ
ಕುಂಚನ-ಹಳ್ಳಿ-ಗಳ
ಕುಂಚನ-ಹಳ್ಳಿ-ಯನ್ನು
ಕುಂಚಪ್ಪವಿಲ್
ಕುಂಚಿ-ಕೊಂಡ
ಕುಂಚಿ-ಕೊಂಡ-ಭೂ-ಪಾಲ
ಕುಂಚಿ-ಗನ-ಹಳ್ಳಿ
ಕುಂಚಿ-ಗನ-ಹಳ್ಳಿ-ಬೇಚಿರಾಕ್
ಕುಂಚಿ-ಗರ
ಕುಂಚಿಟಿಗ-ರಲ್ಲೂ
ಕುಂಚಿಯ
ಕುಂಜರ
ಕುಂಟ
ಕುಂಟಂಬ-ದ-ವರು
ಕುಂಟುಂಬ-ಗ-ಳನ್ನು
ಕುಂಟೆ
ಕುಂಟೆ-ಗಳ
ಕುಂತಲ
ಕುಂತಲ-ಗಳ
ಕುಂತಲ-ಗಳು
ಕುಂತಲವು
ಕುಂತಲ-ವೆಂದೂ
ಕುಂತಲೇಂದ್ರ
ಕುಂತಳೋತ್ಕರದ
ಕುಂತಿ
ಕುಂತಿ-ಬೆಟ್ಟ
ಕುಂತಿ-ಬೆಟ್ಟದ
ಕುಂತಿ-ಬೆಟ್ಟ-ದಲ್ಲಿ
ಕುಂತೂ-ರನ್ನು
ಕುಂತೂರು
ಕುಂತೂರು-ಮಠದ
ಕುಂದ-ಕುಂದಾನ್ವಯದ
ಕುಂದ-ಕುಟುಮಳ
ಕುಂದ-ಘಟ್ಟ
ಕುಂದಣ
ಕುಂದ-ನ-ಹಳ್ಳಿ
ಕುಂದನ್ನಾಡಿನ
ಕುಂದನ್ನಾಡು
ಕುಂದನ್ನಾಡು-ಗಳಾದ್ದಿರ-ಬಹು-ದೆಂದು
ಕುಂದವೈ
ಕುಂದ-ಸತ್ತಿ
ಕುಂದಾಚಿ
ಕುಂದಾ-ಚಿಯು
ಕುಂದಾಚ್ಚಿ
ಕುಂದಾಚ್ಚಿ-ಯರ
ಕುಂದಾಚ್ಚಿಯು
ಕುಂದೂರ
ಕುಂದೂ-ರನ್ನು
ಕುಂದೂರಿನ
ಕುಂದೂರಿನಲ್ಲಿ
ಕುಂದೂರು
ಕುಂದೇಂದುಮಂದಾಕಿನೀ-ವಿ-ಶದ-ಯಶಂ
ಕುಂನಂ
ಕುಂನಿಯ
ಕುಂನೆಯ-ನಾಯಕ
ಕುಂನ್ದ
ಕುಂಪೆ-ನಾಡಾ-ಳುವ-ವ-ನನ್ನು
ಕುಂಪೆ-ನಾಡಾಳ್ವನು
ಕುಂಪೆ-ನಾಡು
ಕುಂಬ-ಗೆರೆ
ಕುಂಬಯ್ಯ
ಕುಂಬಯ್ಯ-ನನ್ನು
ಕುಂಬಯ್ಯ-ನಿಗೆ
ಕುಂಬಾರ
ಕುಂಬಾರ-ಕಟ್ಟೆ
ಕುಂಬಾರ-ಗುಂಡಯ್ಯನ
ಕುಂಬಾರ-ಗುಂಡಿಗಳ
ಕುಂಬಾರ-ಗುಂಡಿಯ
ಕುಂಬಾರ-ಗುಂಡಿ-ಯಲ್ಲಿ
ಕುಂಬಾರ-ಗುಂಡಿಯ-ಹಳ್ಳ
ಕುಂಬಾರ-ತೆ-ರಿಗೆ
ಕುಂಬಾ-ರರ
ಕುಂಬಾ-ರರು
ಕುಂಬೇನ-ಹಳ್ಳಿ-ಯಲ್ಲಿ
ಕುಂಬೇನ-ಹಳ್ಳಿಯು
ಕುಂಭ-ಕಾರ
ಕುಂಭಲಗ್ನ-ದಲ್ಲಿ
ಕುಂಭಸ್ಥಳಕ್ಕೆ
ಕುಂಭಸ್ಥಳ-ವನ್ನು
ಕುಂಭಸ್ಥಳವು
ಕುಂಭಾರ-ಗುಂಡಿ
ಕುಂಮಟ
ಕುಕ-ನೂರು
ಕುಕ್ಕ-ನೂರು
ಕುಕ್ಕುಟ
ಕುಕ್ಕುಟಾ-ಸನ
ಕುಚಪ್ಪವಿಲ್
ಕುಞ್ಚಪ್ಪವಿಲ್
ಕುಟುಂಬ
ಕುಟುಂಬಕ್ಕೆ
ಕುಟುಂಬ-ಗ-ಳಿಂದ
ಕುಟುಂಬ-ಗಳಿದ್ದು
ಕುಟುಂಬ-ಗಳಿವೆ
ಕುಟುಂಬ-ಗಳು
ಕುಟುಂಬದ
ಕುಟುಂಬ-ದಲ್ಲಿ
ಕುಟುಂಬ-ದ-ವ-ರಿಗೆ
ಕುಟುಂಬ-ದ-ವರು
ಕುಟುಂಬ-ದ-ವರೂ
ಕುಟುಂಬ-ನಾಮ-ವನ್ನು
ಕುಟುಂಬ-ವಿಲ್ಲ
ಕುಟುಂಬವೂ
ಕುಟುಂಬಿನಿ
ಕುಟ್ಟಾಡಿ
ಕುಡ
ಕುಡ-ಗಬಾಳ
ಕುಡ-ಲಿಲ್ಲ
ಕುಡಿ-ನೀರು
ಕುಡಿ-ನೀರು-ಕಟ್ಟೆ
ಕುಡಿ-ಯರ
ಕುಡಿ-ಯಲು
ಕುಡಿ-ಯುವ
ಕುಡಿ-ಹಳ್ಳ
ಕುಡುಗ-ಬಾಳು
ಕುಡುಗು-ಕೋಲಾಹಲ
ಕುಡುಗು-ನಾಡ
ಕುಡುಗು-ನಾಡನ್ನು
ಕುಡುಗು-ನಾಡು-ಗ-ಳನ್ನು
ಕುಡುಗು-ನಾಡೊಳ-ಗಣ
ಕುಡುಗು-ಬಾಳು
ಕುಡುಗೆ
ಕುಡೆ
ಕುಣರಪಾಕಂ
ಕುಣಿಂಗಲ
ಕುಣಿಂಗಲ-ವೂರ
ಕುಣಿಂಗಲ-ವೂರ-ಇಂದಿನ
ಕುಣಿಂಗಲ-ವೂರಿನ
ಕುಣಿಂಗಲ-ವೂರಿನಲ್ಲಿ
ಕುಣಿಂಗಲಾ-ಚಾರ
ಕುಣಿ-ಗಲ
ಕುಣಿ-ಗಲ-ಬೀದಿ
ಕುಣಿ-ಗಲು
ಕುಣಿ-ಗಲ್
ಕುಣಿ-ಗಲ್ನಲ್ಲಿ
ಕುಣಿ-ಗಲ್ನಾಡಿನ
ಕುಣಿ-ಗಲ್ಲು
ಕುಣಿ-ಗಿಲ
ಕುಣ್ರಾರ-ದೇವ
ಕುತನೀಯೆ
ಕುತುಬ್ಷಾ
ಕುತೂಹಲ-ಕರ
ಕುತೂಹಲ-ಕರ-ವಾಗಿದೆ
ಕುತೂಹಲ-ಕಾರಿಯೂ
ಕುತೂಹಲ-ಭರಿತ
ಕುತ್ತಾಲ
ಕುತ್ತಿ-ಕೊಂಡು-ಗುರು-ತಿಸಿ
ಕುತ್ತುವ
ಕುತ್ತೂರು
ಕುದಿಹೆರುಕುದೇರು
ಕುದುರೆ
ಕುದುರೆ-ಗಳ
ಕುದುರೆ-ಗ-ಳನ್ನು
ಕುದುರೆ-ಗ-ಳಿಗೆ
ಕುದುರೆ-ಗಳು
ಕುದುರೆ-ಗುಂಡಿ
ಕುದುರೆ-ಗುಂಡಿ-ಯನು
ಕುದುರೆ-ಗುಂಡಿ-ಯನ್ನು
ಕುದು-ರೆಯ
ಕುದುರೆ-ಯನ್ನು
ಕುದುರೆ-ಯ-ಸೇಸೆ
ಕುದುರೆ-ಯುಮಂ
ಕುದ್ದಾಳ
ಕುದ್ಧಾಳ
ಕುದ್ರುಗ್ದೇವಾರ್ತ್ಥ-ಸೇವಾಂಜಲಿಃ
ಕುನ್ದ-ಗಾಮುಣ್ಡರು
ಕುನ್ದನ್ನಾಡನ್ನು
ಕುನ್ದನ್ನಾಡಿನ
ಕುನ್ದನ್ನಾಡು
ಕುನ್ದ-ಸತ್ತಿ
ಕುನ್ದ-ಸತ್ತಿ-ಅರ-ಸನು
ಕುನ್ದಾಚ್ಚಿಯು
ಕುನ್ದು-ನಾಟ್ಟು
ಕುನ್ದು-ನಾಡಾಳ್ವ
ಕುನ್ದೂರು
ಕುನ್ನಂಪಾಕಂ
ಕುನ್ನಪಾಕಂ
ಕುನ್ನಲ-ಬೊಪ್ಪ
ಕುನ್ನಿಯ
ಕುಪಂಣ
ಕುಪೆ-ಗೌಡನ
ಕುಪ್ಪ-ಗವು-ಡಿ-ಯ-ಹಳ್ಳಿ
ಕುಪ್ಪಣ್ಣ
ಕುಪ್ಪಮ್ಮ
ಕುಪ್ಪಮ್ಮನು
ಕುಪ್ಪಾಲ್ಮಾ-ದ-ವರುಂ
ಕುಪ್ಪೆ
ಕುಪ್ಪೆ-ಮಂಚನ-ಹಳ್ಳಿ
ಕುಪ್ಪೆ-ಮದ್ದೂ-ರನ್ನು
ಕುಪ್ಪೆ-ಮದ್ದೂರು-ಗ-ಳಲ್ಲಿ
ಕುಪ್ಪೆಯ
ಕುಪ್ಪೆ-ಯನ್ನು
ಕುಪ್ಪೆ-ಯಲ್ಲಿ
ಕುಬೇರ-ಪುರ
ಕುಮಾಂಡೂರಾಚರ
ಕುಮಾರ
ಕುಮಾರ-ಗೋವಿಯಂಣ್ನನ
ಕುಮಾರ-ಣನ್ದಿ
ಕುಮಾರ-ನಾದ
ಕುಮಾರ-ನಿಗೆ
ಕುಮಾರನು
ಕುಮಾರ-ನೆಂದು
ಕುಮಾರ-ಮಾ-ಲಿಕೆ-ಯನ್ನು
ಕುಮಾರರ
ಕುಮಾರ-ರ-ಗಂಡ
ಕುಮಾರ-ರಾದ
ಕುಮಾರ-ರಾಮನ
ಕುಮಾರರು
ಕುಮಾರ-ವೃತ್ತಿ-ಯಿಂದ
ಕುಮಾರವ್ಯಾಸ
ಕುಮಾರ-ಸೇನ
ಕುಮಾರಸ್ವಾಮಿ
ಕುಮಾರಸ್ವಾಮಿ-ಯ-ವರ
ಕುಮಾರಸ್ವಾಮಿ-ಯ-ವರು
ಕುಮಾರ-ಹೆಗ್ಗಡೆ-ದೇವ
ಕುಮಾರಿ
ಕುಮಾರಿ-ಯಾದ
ಕುಮುದ-ಚಂದ್ರ
ಕುಮ್ಮಟ-ದುರ್ಗದ
ಕುರಂದರ
ಕುರ-ಣೇನ-ಹಳ್ಳಿ
ಕುರ-ವಂಕ-ನಾಡ
ಕುರಿ-ಗಳ
ಕುರಿತಂತೆ
ಕುರಿತು
ಕುರಿದಿಂಬರಿಸಿ
ಕುರಿ-ದೆರೆ
ಕುರುಂಕ-ನಾಡ
ಕುರುಂಬಿಡಿ
ಕುರು-ಕಳ-ವಂಪಡಿ-ಯಲ್ಲಿ
ಕುರುಕಿ
ಕುರುಕಿ-ಮಾಳೆಯರ
ಕುರು-ಕುಳ-ವಂಪಡಿಯ
ಕುರು-ಕುಳ-ವಂಪಡಿ-ಯಲ್ಲಿ
ಕುರು-ಕುಳ-ವ-ಳಂಪ-ಡಿಯು
ಕುರುಕ್ಕಿ
ಕುರುಕ್ಕಿ-ನಾಡ
ಕುರು-ಗೋಡು
ಕುರುಚಲು
ಕುರು-ಣೆಯನ
ಕುರು-ಣೆಯ-ನ-ಹಳ್ಳಿ-ಯನ್ನು
ಕುರು-ಣೆಯ-ನ-ಹಳ್ಳಿಯು
ಕುರುಣೇನ-ಹಳ್ಳಿ
ಕುರು-ದೆರೆ
ಕುರು-ದೆರೆ-ಗಳ
ಕುರು-ದೆಱೆ
ಕುರು-ನಂದ-ನರ
ಕುರುಬ
ಕುರುಬ-ನ-ಕಟ್ಟೆ
ಕುರು-ಬರ
ಕುರುಬ-ರ-ಕಾಳೇನ-ಹಳ್ಳಿ
ಕುರು-ಬರು
ಕುರುಬ-ಸೇಣಿ
ಕುರು-ಭೂಮಿ-ಯಲ್ಲಿ
ಕುರು-ಭೂಮಿ-ಯೊಳ್
ಕುರುವಂಕ
ಕುರುವಂಕದ
ಕುರುವಂಕ-ನಾಡ
ಕುರುವಂಕ-ನಾ-ಡಿಗೆ
ಕುರುವಂಕ-ನಾಡಿನ
ಕುರುವಂಕ-ನಾಡಿ-ನಲ್ಲಿ
ಕುರುವಂಕ-ನಾಡು
ಕುರುವಂದ
ಕುರುವಂದ-ಕುಲ
ಕುರುವಂದ-ಕುಲ-ಕ-ಮಲ-ಮಾರ್ತಾಂಡ
ಕುರುವಂದ-ಕುಲದ
ಕುರುವಂದ-ಕುಲ-ದ-ವರು
ಕುರುವಂದ-ಕುಲೈಕಭೂಷಣ-ನೆನಿಸಿದ
ಕುರುವಂದ-ವರ
ಕುರುವಂದಾನ್ವಯ
ಕುರುವಂದಾನ್ವಯರು
ಕುರುವಂದೇಶ್ವರ
ಕುರುವತ್ತಿಯ
ಕುರುವಮಕ
ಕುರುವೈ
ಕುರು-ಹಾಗಿ
ಕುರುಹು
ಕುರುಹು-ಗಳಿವೆ
ಕುರುಹು-ಗಳು
ಕುರುಹು-ಗಳೂ
ಕುರುಹೆಂದು
ಕುರ್ಕ್ಯಾಲ್
ಕುರ್ತ-ಕೋಟಿ
ಕುರ್ರಬೂಲಾಡು
ಕುರ್ವಂಕ
ಕುರ್ವಂಕ-ನಾಡಿನ
ಕುರ್ವಂಕಸ್ಥಳದ
ಕುರ್ವ್ವಂಕ-ನಾಡ
ಕುಱ-ಡುವ
ಕುಱಿದೆಱಿಗೆ
ಕುಱುಕಿ
ಕುಱುಕ್ಕಿ
ಕುಱುವಂಕ-ನಾಡ
ಕುಱುವಂದೇಶ್ವರ
ಕುಲ
ಕುಲಕ
ಕುಲಕಂ
ಕುಲ-ಕ-ಮಲ
ಕುಲ-ಕರ-ಣಿ-ಗಳ
ಕುಲ-ಕರ-ಣಿ-ಗಳು
ಕುಲ-ಕರ-ಣಿಯ
ಕುಲ-ಕರ್ಣಿ
ಕುಲಕೆ
ಕುಲಕ್ಕೂ
ಕುಲಕ್ಕೆ
ಕುಲಕ್ರಮ-ವೆಂತೆಂದಡೆ
ಕುಲಕ್ಷತ್ರಿಯ
ಕುಲ-ಗಳ
ಕುಲ-ಗ-ಳನ್ನು
ಕುಲ-ಗ-ಳಿಗೆ
ಕುಲ-ಗಳು
ಕುಲ-ಗಾಣಾ
ಕುಲ-ತಿಲಕ
ಕುಲ-ತಿಲಕ-ನಾಗಿದ್ದಾನೆ
ಕುಲ-ತಿಲಕ-ರಾದ
ಕುಲ-ತಿಲಕ-ರೆಂದು
ಕುಲ-ತೆಲ್ಲಿ-ಗರು
ಕುಲದ
ಕುಲ-ದಲ್ಲಿ
ಕುಲ-ದಲ್ಲಿದ್ದ
ಕುಲ-ದವ
ಕುಲ-ದ-ವ-ನಾಗಿ-ರ-ಬ-ಹುದು
ಕುಲ-ದ-ವನೇ
ಕುಲ-ದ-ವರ
ಕುಲ-ದ-ವರು
ಕುಲ-ದೀಪಕ
ಕುಲ-ದೀಪಕಂ
ಕುಲ-ದೀಪಕಃ
ಕುಲ-ದೀಪ-ಕ-ನಾದ
ಕುಲ-ದೀಪ-ಕ-ನುಮೆನಿಪ
ಕುಲ-ದೀಪ-ನುಮೆನಿಪ
ಕುಲ-ದೇವ-ತೆಯು
ಕುಲ-ದೇವ-ರಾದ
ಕುಲ-ದೈವ-ತಮೀಕ್ಷಿತುಂ
ಕುಲ-ದೈವವು
ಕುಲ-ಧ-ವಳ
ಕುಲ-ನಾಮ-ವೆಂದು
ಕುಲ-ನಾಯ-ಕಸ್ಯ
ಕುಲ-ವನ-ಹಳ್ಳ-ದಲ್ಲಿ
ಕುಲ-ವನ್ನು
ಕುಲ-ವಾ-ಗಿದ್ದು
ಕುಲ-ವಿರ-ಬಹು-ದೆಂದು
ಕುಲವು
ಕುಲ-ವೆಂದರೆ
ಕುಲ-ವೆಂದು
ಕುಲ-ವೆಂಬ
ಕುಲ-ಶೇಖರ
ಕುಲ-ಶೇಖರ-ದಾಸ
ಕುಲ-ಶೇಖರ-ದಾಸರ್
ಕುಲ-ಶೇಖರ-ನೆಂಬ
ಕುಲಸ್ತ್ರೀ
ಕುಲಾಂತಕ
ಕುಲಾ-ಚಾರ
ಕುಲಾನ್ವಯ
ಕುಲಾನ್ವಯದ
ಕುಲಾನ್ವಯ-ದ-ವರು
ಕುಲಾನ್ವಯ-ರಾದ
ಕುಲಾನ್ವಯರು
ಕುಲಾನ್ವಯರುಂ
ಕುಲೀನಾಂಶ್ಚ
ಕುಲುಕ-ಕುಲಕ
ಕುಲುಮೆ-ಯಲ್ಲಿ
ಕುಲು-ವನ-ಹಳ್ಳ-ದಲ್ಲಿ
ಕುಲೈಕ
ಕುಲೋತ್ತುಂಗ
ಕುಲೋತ್ತುಂಗ-ಚೋಳನ
ಕುಲೋತ್ತುಂಗ-ನಿಗೆ
ಕುಲೋದ್ಧರಣ
ಕುಲೋದ್ಭವನೂ
ಕುಳ
ಕುಳಂಜರ್
ಕುಳ-ಕರಣಿ
ಕುಳ-ಕರ-ಣಿ-ಕ-ರನ್ನು
ಕುಳ-ಕರ-ಣಿ-ಗಳ
ಕುಳ-ಕರ್ಣಿ
ಕುಳ-ತಿಳಕ
ಕುಳತ್ತೂರು
ಕುಳದ
ಕುಳ-ದಲ್ಲಿ
ಕುಳ-ದಲ್ಲೂ
ಕುಳ-ಪತಿ
ಕುಳಪ್ರ-ದೀಪ
ಕುಳವ
ಕುಳ-ವ-ಕಟ್ಟಿಸಿ
ಕುಳ-ವ-ಕಟ್ಟಿ-ಸಿ-ಹೊಸ-ದಾಗಿ
ಕುಳ-ವ-ಕ-ಡಿಸಿ
ಕುಳ-ವ-ಕಡ್ಸಿ
ಕುಳ-ವ-ಕಡ್ಸಿದ
ಕುಳ-ವ-ಕಳೆದು
ಕುಳ-ವನು
ಕುಳ-ವನ್ನು
ಕುಳ-ವಾಳರ
ಕುಳ-ವಾಳ-ರಿಗೆ
ಕುಳ-ಸಹಿತ
ಕುಳ-ಸುಂಕ-ವನ್ನು
ಕುಳಾಂತಕ
ಕುಳಾಂಬರ
ಕುಳಾಗ್ರಣಿ
ಕುಳಾಗ್ರಣಿಯು
ಕುಳಾ-ಳರು
ಕುಳಿ
ಕುಳಿ-ಗ-ಳಿಂದ
ಕುಳಿ-ಗಳು
ಕುಳಿ-ಗಳೇ
ಕುಳಿತ
ಕುಳಿ-ತರು
ಕುಳಿ-ತಿದ್ದ
ಕುಳಿ-ತಿದ್ದರೆ
ಕುಳಿ-ತಿದ್ದಾನೆ
ಕುಳಿ-ತಿದ್ದೆ
ಕುಳಿ-ತಿರುತ್ತಿದ್ದರು
ಕುಳಿ-ತಿ-ರುವ
ಕುಳಿತು
ಕುಳಿ-ತು-ಕೊಂಡು
ಕುಳಿಯ
ಕುಳಿ-ಯನ್ನು
ಕುಳಿ-ಯಿಂದ
ಕುಳು-ವನು
ಕುಳೈನ್ದಾನ್
ಕುಳ್ಳಾಗಿ-ರುವು-ದ-ರಿಂದ
ಕುಳ್ಳಿರ್ದ್ದು
ಕುವರ
ಕುವರ-ಲಕ್ಷ್ಮ
ಕುವ-ರಿಯರು
ಕುವಳಾಲಕೊ-ವಳಾಲ-ಪುರ-ವ-ರಾಧೀಶ್ವರ-ರೆಂದು
ಕುವಳಾಲ-ಪುರ-ವ-ರೇಶ್ವರ
ಕುವೆಂಪು
ಕುಶ
ಕುಶ-ಲ-ಕೆಲಸ-ಗಾರರ
ಕುಶ-ಲ-ತೆಯ
ಕುಶ-ಲಮೆಂದಿರ-ದೋಡಿ-ದನೊಂದೆ
ಕುಶ-ಲಮೆಂದೋಡಿ-ದನೊಂದೆ
ಕುಶೇಶೆ-ಯನ
ಕುಷ್ಟಗಿ
ಕುಸಳ-ತೆ-ಯುಳ್ಳ
ಕುಹು-ಯೋಗ-ವನ್ನು
ಕುಹೂಯೋಗ-ವನ್ನು
ಕುೞವಕೞದು
ಕುೞ-ವಾಳರ
ಕೂಂಡಿ
ಕೂಂಡಿ-ನಾಡ-ಕು-ಹುಂಡಿ
ಕೂಂಬಡಿ
ಕೂಗಾಡಿ
ಕೂಗಿ-ದನು
ಕೂಗುತ್ತಾರೆ
ಕೂಚಿ-ತಂದೆ
ಕೂಟ
ಕೂಟಂಗಳಿಂ
ಕೂಟಕ್ಕೆ
ಕೂಟ-ತೆಗೆ
ಕೂಟದ
ಕೂಟ-ದಲ್ಲಿ
ಕೂಟ-ದೊಳು
ಕೂಟ-ವನ್ನು
ಕೂಟ-ವನ್ನೂ
ಕೂಟವು
ಕೂಟ-ಶಾ-ಸನ-ವಿರ-ಬ-ಹುದು
ಕೂಟ-ಶಾ-ಸನ-ವೆನ್ನಲು
ಕೂಡ
ಕೂಡಲಿ
ಕೂಡ-ಲುಕುಪ್ಟೆ
ಕೂಡ-ಲು-ಕುಪ್ಪೆ
ಕೂಡ-ಲು-ಗುಪ್ಪೆ
ಕೂಡ-ಲೂರ
ಕೂಡ-ಲೂರ-ಕುಲ
ಕೂಡ-ಲೂರಿ-ನಲ್ಲಿ
ಕೂಡ-ಲೂರು
ಕೂಡಲೇ
ಕೂಡಾ
ಕೂಡಿ
ಕೂಡಿ-ಕೊಂಡು
ಕೂಡಿ-ತಪ್ಪು-ನಾಯ-ಕ-ರ-ಗಂಡ
ಕೂಡಿ-ತಪ್ಪುವ
ಕೂಡಿತ್ತು
ಕೂಡಿದ
ಕೂಡಿದೆ
ಕೂಡಿದ್ದ
ಕೂಡಿದ್ದು
ಕೂಡಿ-ರುವ
ಕೂಡಿವೆ
ಕೂಡಿ-ಸಂದರು
ಕೂಡಿಸಿ
ಕೂಡಿ-ಸುವ
ಕೂಡು
ಕೂಡು-ಒಕ್ಕಲಿ-ಗರು
ಕೂಡುತ್ತಿದ್ದ-ವೆಂದು
ಕೂಡುವ
ಕೂಡು-ವ-ನಾಯ-ಕರ
ಕೂಡು-ವಲ್ಲಿ
ಕೂಡೆ
ಕೂಡೆ-ಸೋಮೆಯ
ಕೂಡ್ಲಿ
ಕೂಡ್ಲು-ಕುಪ್ಪೆ
ಕೂತಾಂಡಿಯರ
ಕೂತೋಜನು
ಕೂತ್ತ-ಗಾವುಂಡ
ಕೂತ್ತಾಂಡಿ
ಕೂತ್ತಾಣ್ಡಿ
ಕೂತ್ತಾಣ್ಡೈ
ಕೂತ್ತಾನ್
ಕೂಮ್ಬಡಿ
ಕೂರ-ತೆಗೆಂಟುಮವೆಯ್ದೆ
ಕೂರಯ
ಕೂರಯ-ನಾಯ-ಕನ
ಕೂರಯ-ನಾಯ-ಕನು
ಕೂರಲಗ-ನೆಯ್ದೆ
ಕೂರಲ-ಗನ್ನು
ಕೂರ-ಲಗು
ಕೂರಿಗಿ-ಹಳ್ಳಿಯ
ಕೂರಿ-ಸಲು
ಕೂರಿಸಲ್ಪಟ್ಟನು
ಕೂರಿಸಿ
ಕೂರಿಸಿ-ಕೊಂಡು
ಕೂರಿಸಿ-ದನು
ಕೂರೆಯ-ನಾಯಕ
ಕೂರೆಯ-ನಾಯ-ಕನ
ಕೂರೆಯ-ನಾಯ-ಕನು
ಕೂರೆಯ-ನಾಯ-ಕರು
ಕೂರ್ಗಲ್ಲನ್ನು
ಕೂಲಿ-ಗೆರೆಯ
ಕೂಲಿಗ್ಗೆರೆ
ಕೂಲಿಗ್ಗೆರೆ-ಕೂಳ-ಗೆರೆ
ಕೂಲಿಗ್ಗೆರೆಯು
ಕೂಳಣ-ವಾಗಿ
ಕೂಳಿ-ಕಾಟ್ಟು
ಕೂಳು
ಕೂಸ
ಕೂಸಂ
ಕೂಸ-ಅ-ಪರ್ಣ
ಕೂಸ-ರಾಮೆಯ
ಕೂಸು-ಗ-ಳೆಂದು
ಕೃತಕ
ಕೃತಕ-ವಾಗಿ
ಕೃತಕ-ವೆಂದು
ಕೃತಜ್ಞಂ
ಕೃತಜ್ಞ-ನಾದ
ಕೃತ-ಯುಗ-ದಲ್ಲಿ
ಕೃತ-ವತಿ
ಕೃತಾರ್ಥ-ನಾದ-ನೆಂದು
ಕೃತಿ
ಕೃತಿ-ಗಳ
ಕೃತಿ-ಗ-ಳನ್ನು
ಕೃತಿ-ಗ-ಳನ್ನೂ
ಕೃತಿ-ಗ-ಳಲ್ಲಿ
ಕೃತಿ-ಗ-ಳಾದ
ಕೃತಿ-ಗ-ಳಿಂದ
ಕೃತಿ-ಗ-ಳಿಗೆ
ಕೃತಿ-ಗಳು
ಕೃತಿ-ಗಳೂ
ಕೃತಿನಾ
ಕೃತಿಯ
ಕೃತಿ-ಯನ್ನು
ಕೃತಿ-ಯಲ್ಲಾ-ಗಲೀ
ಕೃತಿ-ಯಲ್ಲಿ
ಕೃತಿ-ಯಾಗಿದೆ
ಕೃತಿ-ಯಿಂದ
ಕೃತಿಯು
ಕೃತಿ-ರಚನೆ-ಯನ್ನೇ
ಕೃತೀ
ಕೃತ್ತಿ-ಕಾರ್ಯ
ಕೃಪೆ-ಯನ್ನು
ಕೃಪೆ-ಯಿಂದ
ಕೃಪೇ
ಕೃಷ-ದೇವ-ರಾಯನ
ಕೃಷಿ
ಕೃಷಿಗೂ
ಕೃಷಿಗೆ
ಕೃಷಿ-ಪದ್ಧತಿ
ಕೃಷಿಪ್ರಧಾನ-ವಾದ
ಕೃಷಿ-ಭೂಮಿ
ಕೃಷಿ-ಯೇ-ತರ
ಕೃಷ್ಣ
ಕೃಷ್ಣ-ಕಂಧರ
ಕೃಷ್ಣ-ಕಂಧರ-ನನ್ನು
ಕೃಷ್ಣ-ಕಂಧರ-ನುಮಂ
ಕೃಷ್ಣ-ಕೊ-ಮಾರ
ಕೃಷ್ಣ-ದೇವ
ಕೃಷ್ಣ-ದೇವ-ಮಹಾ-ರಾಯ
ಕೃಷ್ಣ-ದೇವ-ರಾಯ
ಕೃಷ್ಣ-ದೇವ-ರಾಯನ
ಕೃಷ್ಣ-ದೇವ-ರಾಯ-ನನ್ನು
ಕೃಷ್ಣ-ದೇವ-ರಾಯ-ನಿಂದ
ಕೃಷ್ಣ-ದೇವ-ರಾಯ-ನಿಗೆ
ಕೃಷ್ಣ-ದೇವ-ರಾಯನು
ಕೃಷ್ಣ-ದೇವ-ರಾಯ-ನೆಂದು
ಕೃಷ್ಣ-ದೇವ-ರಾಯ-ನೆಂಬು-ವ-ವನು
ಕೃಷ್ಣ-ದೇವ-ರಾಯನೇ
ಕೃಷ್ಣ-ದೇವ-ರಾಯ-ಪಟ್ಟಣ
ಕೃಷ್ಣ-ದೇವ-ರಾಯ-ಪಟ್ಟ-ಣಕ್ಕೆ
ಕೃಷ್ಣ-ದೇವ-ರಾಯ-ಪುರ
ಕೃಷ್ಣ-ದೇವ-ವೊಡೆ-ಯರ
ಕೃಷ್ಣ-ದೇವಾ-ಲಯ
ಕೃಷ್ಣ-ದೇವಾ-ಲಯ-ಗ-ಳಲ್ಲಿ
ಕೃಷ್ಣ-ದೇವಾ-ಲಯ-ಗೋಪೀ-ನಾಥ
ಕೃಷ್ಣ-ದೇವಾ-ಲಯದ
ಕೃಷ್ಣ-ದೇವಾ-ಲಯ-ವಾದ
ಕೃಷ್ಣ-ದೇ-ವೊಡೆಯರ
ಕೃಷ್ಣನ
ಕೃಷ್ಣ-ನನ್ನೇ
ಕೃಷ್ಣ-ನಾಗು
ಕೃಷ್ಣ-ನಿ-ಗಿಂತಲೂ
ಕೃಷ್ಣ-ನಿಗೆ
ಕೃಷ್ಣನು
ಕೃಷ್ಣನೂ
ಕೃಷ್ಣನೇ
ಕೃಷ್ಣ-ಪಕ್ಷದ
ಕೃಷ್ಣಪ್ಪ
ಕೃಷ್ಣಪ್ಪ-ನ-ವರ
ಕೃಷ್ಣಪ್ಪ-ನಾಯ-ಕನ
ಕೃಷ್ಣಪ್ಪ-ನಾಯ-ಕ-ನಿಗೆ
ಕೃಷ್ಣಪ್ಪ-ನಾಯ-ಕನು
ಕೃಷ್ಣಪ್ಪ-ನಾಯ-ಕನೂ
ಕೃಷ್ಣಪ್ಪ-ನಾಯ-ಕ-ರಿಗೆ
ಕೃಷ್ಣ-ಭಟ್ಟನು
ಕೃಷ್ಣ-ಮಹಾ-ಧಿ-ರಾಜನ
ಕೃಷ್ಣ-ಮಾಚಾರ್ರಿ
ಕೃಷ್ಣ-ಮೂರ್ತಿ
ಕೃಷ್ಣ-ಮೂರ್ತಿ-ಯ-ವರು
ಕೃಷ್ಣಯ್ಯ
ಕೃಷ್ಣಯ್ಯ-ನ-ವರ
ಕೃಷ್ಣ-ರ-ವರು
ಕೃಷ್ಣ-ರಾಜ
ಕೃಷ್ಣ-ರಾಜಃ
ಕೃಷ್ಣ-ರಾಜ-ಒಡೆ-ಯನು
ಕೃಷ್ಣ-ರಾಜನ
ಕೃಷ್ಣ-ರಾಜ-ನ-ಗರ
ಕೃಷ್ಣ-ರಾಜ-ನನ್ನು
ಕೃಷ್ಣ-ರಾಜನು
ಕೃಷ್ಣ-ರಾಜ-ಪೇಟೆ
ಕೃಷ್ಣ-ರಾಜ-ಪೇಟೆ-ಗಳ
ಕೃಷ್ಣ-ರಾಜ-ಪೇಟೆಯ
ಕೃಷ್ಣ-ರಾಜ-ಮುಡಿ
ಕೃಷ್ಣ-ರಾಜರ
ಕೃಷ್ಣ-ರಾಜರು
ಕೃಷ್ಣ-ರಾಜ-ವಡರೈಯ
ಕೃಷ್ಣ-ರಾಜ-ವಡೆ-ಯರು-ಗಳ
ಕೃಷ್ಣ-ರಾಜ-ವಡೆ-ಯರೈಯ್ಯಾ-ನ-ವರು
ಕೃಷ್ಣ-ರಾಜ-ಸಾ-ಗರ
ಕೃಷ್ಣ-ರಾಜ-ಸಾ-ಗರಕ್ಕೇ
ಕೃಷ್ಣ-ರಾಜ-ಸಾ-ಗರದ
ಕೃಷ್ಣ-ರಾಜ-ಸಾ-ಗರ-ದೊಳಗೆ
ಕೃಷ್ಣ-ರಾಜು
ಕೃಷ್ಣ-ರಾಜೊಡೆಯರ
ಕೃಷ್ಣ-ರಾಜೊಡೆಯರು
ಕೃಷ್ಣ-ರಾಯ
ಕೃಷ್ಣ-ರಾಯ-ನಾಯಕ
ಕೃಷ್ಣ-ರಾಯ-ನಾಯ-ಕನು
ಕೃಷ್ಣ-ರಾಯ-ನಾಯ-ಕರು
ಕೃಷ್ಣ-ರಾಯನು
ಕೃಷ್ಣ-ರಾಯ-ಪುರ
ಕೃಷ್ಣ-ರಾಯ-ಪುರ-ಗಳೆಂಬ
ಕೃಷ್ಣ-ರಾಯ-ಪುರ-ವನ್ನಾಗಿ
ಕೃಷ್ಣ-ರಾಯ-ಪುರ-ವಾದ
ಕೃಷ್ಣ-ರಾಯ-ಪುರ-ವೆಂಬ
ಕೃಷ್ಣ-ರಾಯ-ಮಹಾ-ರಾಯನ
ಕೃಷ್ಣ-ರಾಯ-ರ-ಕೆರೆಯ
ಕೃಷ್ಣ-ರಾಯ-ಸ-ಮುದ್ರ-ವೆಂದು
ಕೃಷ್ಣ-ರಾ-ಯಸ್ಯ
ಕೃಷ್ಣ-ರಾಯೇ
ಕೃಷ್ಣ-ರಾವ್
ಕೃಷ್ಣ-ವರ್ಮ
ಕೃಷ್ಣ-ವರ್ಮ-ಮಹಾ-ಧಿ-ರಾಜನು
ಕೃಷ್ಣ-ವಾಸು-ದೇವ
ಕೃಷ್ಣ-ವಿಲಾಸ
ಕೃಷ್ಣ-ವಿಲಾಸದ
ಕೃಷ್ಣ-ವಿಳಾಸದ
ಕೃಷ್ಣ-ವೇಣಿ-ತೀರ-ದಲ್ಲಿ
ಕೃಷ್ಣ-ಸೂರಿ
ಕೃಷ್ಣಸ್ವಾಮಿ
ಕೃಷ್ಣಾ-ನದಿ
ಕೃಷ್ಣಾರ್ಪಣ
ಕೃಸಂ
ಕೆ
ಕೆಂಗಲ್ಕೊಪ್ಪ-ಲಿನ
ಕೆಂಚ-ಗಾರ
ಕೆಂಚ-ಗಾರ-ಸೆಟ್ಟಿ
ಕೆಂಚಪ-ನಾಯಕ
ಕೆಂಚಪ-ನಾಯ-ಕರು
ಕೆಂಚ-ವೀರನು
ಕೆಂಚೇನ-ಹಳ್ಳಿ-ಯನ್ನು
ಕೆಂಜೆಡೆದೊಂಗಲ
ಕೆಂತಟಿಯ-ಹಳ್ಳ
ಕೆಂದನ-ಹಾಳು
ಕೆಂದನ-ಹಾಳು-ಕೆನ್ನಾಳು
ಕೆಂದ್ರ-ವಾ-ಗಿತ್ತು
ಕೆಂದ್ರ-ವಾಗಿತ್ತೆಂದು
ಕೆಂಪ
ಕೆಂಪ-ಗವುಂಡನ
ಕೆಂಪ-ಗ-ವುಡನ
ಕೆಂಪ-ದೇವಯ್ಯ-ರಸ-ನಿಗೆ
ಕೆಂಪ-ನಂಜಮ್ಮಣ್ಣಿಗೆ
ಕೆಂಪ-ನಂಜಾಂಬಾ
ಕೆಂಪ-ನಂಜೇ-ದೇವ-ರಿಗೆ
ಕೆಂಪ-ನ-ಪುರ
ಕೆಂಪ-ಬಯಿರ-ರಸ
ಕೆಂಪಮ್ಮನ
ಕೆಂಪ-ಸೆಟ್ಟಿಯ
ಕೆಂಪಿನ-ಳಪ್ಪುವ
ಕೆಂಪು
ಕೆಂಪು-ಅರ್ಕ
ಕೆಂಪು-ಅರ್ಕ-ಒಡೆಯ-ನಿರ-ಬಹು-ದೆಂದು
ಕೆಂಪು-ಅರ್ಕ-ಒಡೆ-ಯನು
ಕೆಂಪುಗ್ರಾನೈಟ್
ಕೆಂಪು-ನಾಯಕ
ಕೆಂಪೇ-ಗೌಡ-ನ-ಕೊಪ್ಪಲು
ಕೆಂಬರೆ-ಹಳ್ಳ
ಕೆಂಬರೇನ-ಹಳ್ಳ
ಕೆಂಬಾ-ಳಿಗೆ
ಕೆಂಬಾವಿ
ಕೆಂಬಾವಿಯ
ಕೆಂಬೊಳ-ಲನ್ನು
ಕೆಂಬೊಳ-ಲಿಗೆ
ಕೆಂಬೊ-ಳಲು
ಕೆಅ-ನಂತ-ರಾಮು
ಕೆಆರ್
ಕೆಆರ್ಗಣೇಶ್
ಕೆಆರ್ನ-ಗರ
ಕೆಎಸ್ಶಿವಣ್ಣ
ಕೆಟ್ಟು-ಹೋಗಿ-ರಲು
ಕೆಡವಿ
ಕೆಡಿಸಿ
ಕೆಡಿಸಿದ
ಕೆಡಿ-ಸುವ-ವರು
ಕೆತ್ತನೆ
ಕೆತ್ತನೆ-ಕೆಲಸದ
ಕೆತ್ತ-ಲಾಗಿದೆ
ಕೆತ್ತಲಾ-ಗಿದ್ದು
ಕೆತ್ತಲ್ಪಟ್ಟಿದೆ
ಕೆತ್ತಿ
ಕೆತ್ತಿದ
ಕೆತ್ತಿ-ದ-ನೆಂದು
ಕೆತ್ತಿ-ದ-ವನು
ಕೆತ್ತಿ-ದ-ವರು
ಕೆತ್ತಿದ್ದಾನೆ
ಕೆತ್ತಿದ್ದಾ-ರೆಂಬುದು
ಕೆತ್ತಿ-ಯಪ್ಪಗೆ
ಕೆತ್ತಿ-ರುವ
ಕೆತ್ತಿ-ರು-ವು-ದನ್ನು
ಕೆತ್ತಿಸಿ
ಕೆತ್ತಿ-ಸಿದ
ಕೆತ್ತುತ್ತಿದ್ದ-ರೆಂದು
ಕೆತ್ತುತ್ತಿದ್ದ-ರೆಂಬುದು
ಕೆತ್ತುತ್ತಿದ್ದ-ವ-ರನ್ನು
ಕೆತ್ತುವುದು
ಕೆನರಾ
ಕೆನ್ನ
ಕೆನ್ನಾಳು
ಕೆಬೆಟ್ಟ-ಹಳ್ಳಿ
ಕೆಬ್ಬೆ-ಹಳ್ಳಿ
ಕೆಯ್ದಾರ್
ಕೆಯ್ಯಂಗಳ
ಕೆರಗೆ
ಕೆರ-ಗೋಡ
ಕೆರ-ಗೋಡಿ
ಕೆರ-ಗೋಡಿಗೆ
ಕೆರ-ಗೋಡು
ಕೆರ-ಯಲ್ಲಿ
ಕೆರ-ಯೆಲಿ
ಕೆರಾಜೇಶ್ವರಿ-ಗೌಡ
ಕೆರೆ
ಕೆರೆ-ಏ-ರಿಯ
ಕೆರೆ-ಕಟ್ಟುವ
ಕೆರೆ-ಕಟ್ಟೆ
ಕೆರೆ-ಕಟ್ಟೆ-ಗಳ
ಕೆರೆ-ಕಟ್ಟೆ-ಗ-ಳನ್ನು
ಕೆರೆ-ಕಟ್ಟೆ-ಗ-ಳಿಗೆ
ಕೆರೆ-ಕಟ್ಟೆ-ಗಳಿದ್ದವು
ಕೆರೆ-ಕಟ್ಟೆ-ಗಳು
ಕೆರೆ-ಕಟ್ಟೆ-ಯಿಂದ
ಕೆರೆ-ಕೋಡಿ
ಕೆರೆ-ಕೋಡಿ-ಕೆರ-ಗೋಡು
ಕೆರೆ-ಗಳ
ಕೆರೆ-ಗ-ಳನ್ನು
ಕೆರೆ-ಗ-ಳನ್ನೂ
ಕೆರೆ-ಗ-ಳಲ್ಲಿ
ಕೆರೆ-ಗಳಷ್ಟೇ
ಕೆರೆ-ಗ-ಳಾಗಿ-ರ-ಬ-ಹುದು
ಕೆರೆ-ಗ-ಳಾಗಿ-ವೆ-ಯೆಂದು
ಕೆರೆ-ಗ-ಳಿಂದ
ಕೆರೆ-ಗಳಿಗೂ
ಕೆರೆ-ಗ-ಳಿಗೆ
ಕೆರೆ-ಗಳಿದ್ದಂತೆ
ಕೆರೆ-ಗಳಿದ್ದು
ಕೆರೆ-ಗಳಿದ್ದುದು
ಕೆರೆ-ಗಳಿ-ರುತ್ತಿದ್ದವು
ಕೆರೆ-ಗಳು
ಕೆರೆ-ಗಳು-ನಿರ್ಮಾಣ-ವಾದರೂ
ಕೆರೆ-ಗಿಕ್ಕಿತು
ಕೆರೆ-ಗಿ-ಳಿದ
ಕೆರೆ-ಗುಳ್ಳ
ಕೆರೆಗೆ
ಕೆರೆ-ಗೊಡಂಗೆಯ
ಕೆರೆ-ಗೊಡಂಗೆ-ಯಾಗಿ
ಕೆರೆ-ಗೊಡಗಿ-ಗ-ಳನ್ನು
ಕೆರೆ-ಗೊಡಗಿ-ಯಾಗಿ
ಕೆರೆ-ಗೊಡಗೆ-ಗ-ಳನ್ನು
ಕೆರೆ-ಗೊಡಗೆಯ
ಕೆರೆ-ಗೊಡಗೆ-ಯಾಗಿ
ಕೆರೆ-ಗೋಡ
ಕೆರೆ-ಗೋಡ-ನಾಡ
ಕೆರೆ-ಗೋಡನ್ನೇ
ಕೆರೆ-ಗೋಡಿನ
ಕೆರೆ-ಗೋಡಿ-ನಾಡ
ಕೆರೆ-ಗೋಡಿ-ನಾಡಿನ
ಕೆರೆ-ಗೋಡು
ಕೆರೆ-ಗೋಡು-ನಾಡ
ಕೆರೆ-ದೇವಾ-ಲಯ
ಕೆರೆ-ದೇವಾಲ್ಯವ-ನ-ಳಿದ
ಕೆರೆಯ
ಕೆರೆಯಂ
ಕೆರೆ-ಯ-ಕೆಳಗೆ
ಕೆರೆ-ಯ-ಕೋಡಿಯ
ಕೆರೆ-ಯ-ನೀರು
ಕೆರೆ-ಯನ್ನು
ಕೆರೆ-ಯನ್ನೂ
ಕೆರೆ-ಯ-ಬಳಿ
ಕೆರೆ-ಯಲ್ಲಿ
ಕೆರೆ-ಯ-ವರೆಗೆ
ಕೆರೆ-ಯಾಗಿ
ಕೆರೆ-ಯಾ-ಗಿದ್ದ
ಕೆರೆ-ಯಾ-ಗಿದ್ದು
ಕೆರೆ-ಯಾಗಿ-ರ-ಬಹು
ಕೆರೆ-ಯಾಗಿ-ರ-ಬ-ಹುದು
ಕೆರೆ-ಯಾದವು
ಕೆರೆ-ಯಿಂದ
ಕೆರೆಯು
ಕೆರೆಯುಂ
ಕೆರೆ-ಯು-ಮಾರ-ವೆಯುಮ-ನ-ಳಿದು
ಕೆರೆಯೂ
ಕೆರೆಯೇ
ಕೆರೆ-ಯೊಳ-ಗಣ
ಕೆರೆ-ಯೊಳಗೆ
ಕೆರೆ-ಯೊಳ್ಕೂಡುವ
ಕೆರೆ-ಹಳ್ಳಿ
ಕೆರೆೆ
ಕೆಱೆ
ಕೆಱೆಗೆ
ಕೆಱೆ-ಗೊಡಂಗೆ
ಕೆಱೆ-ಗೊಡಂಗೆಯ
ಕೆಱೆಯ
ಕೆಲ-ಗೆರೆಯ
ಕೆಲ-ದಲ್ಲಿ
ಕೆಲ-ದೊಳು
ಕೆಲ-ಬಲ-ಗಳಲ್ಲಿದ್ದ
ಕೆಲ-ಮಟ್ಟಿಗೆ
ಕೆಲ-ವಕ್ಕೆ
ಕೆಲ-ವನ-ರನ್ನು
ಕೆಲ-ವನ್ನು
ಕೆಲ-ವರ
ಕೆಲ-ವರಂತೂ
ಕೆಲ-ವ-ರನ್ನು
ಕೆಲವ-ರಿಗೆ
ಕೆಲ-ವರು
ಕೆಲ-ವಾರು
ಕೆಲವು
ಕೆಲವು-ಕಾಲ
ಕೆಲವು-ಭಾಗ
ಕೆಲವೆಡೆ
ಕೆಲವೇ
ಕೆಲವೊಂದ-ರಲ್ಲಿ
ಕೆಲ-ವೊಂದು
ಕೆಲವೊಮ್ಮೆ
ಕೆಲಸ
ಕೆಲಸಕ್ಕಾಗಿ
ಕೆಲಸಕ್ಕೆ
ಕೆಲಸ-ಗಳ
ಕೆಲಸ-ಗ-ಳನ್ನು
ಕೆಲಸ-ಗ-ಳಿಂದಾಗಿ
ಕೆಲಸ-ಗ-ಳಿಗೆ
ಕೆಲಸ-ಗಳೆಲ್ಲ-ವನ್ನೂ
ಕೆಲಸ-ಗಾರರ
ಕೆಲಸ-ಗಾರ-ರನ್ನು
ಕೆಲಸ-ಗಾ-ರ-ರಿಗೆ
ಕೆಲಸದ
ಕೆಲಸ-ದಲಿ
ಕೆಲಸ-ದಲ್ಲಿ
ಕೆಲಸ-ವನ್ನು
ಕೆಲಸ-ವಾಗಿ-ರುವುದು
ಕೆಲಸವು
ಕೆಲಸವೂ
ಕೆಲಸಾದಿ-ಗ-ಳಲ್ಲಿ
ಕೆಲ್ಲಂಗೆರೆ
ಕೆಲ್ಲಂಗೆರೆಯ
ಕೆಲ್ಲಂಗೆರೆ-ಯನು
ಕೆಲ್ಲಂಗೆರೆ-ಯನ್ನು
ಕೆಲ್ಲಂಗೆರೆ-ಯಲ್ಲಿ
ಕೆಲ್ಲಂಗೆರೆಯು
ಕೆಲ್ಲ-ಬ-ಸದಿ
ಕೆಲ್ಲ-ಬ-ಸದಿಯ
ಕೆಲ್ಲ-ವತ್ತಿ
ಕೆಲ್ಸಕ್ಕೆ
ಕೆಳ
ಕೆಳ-ಕಂಡ
ಕೆಳ-ಕಂಡಂತೆ
ಕೆಳಕ್ಕೆ
ಕೆಳ-ಗಣ
ಕೆಳ-ಗಳ-ಮಣ್ನು
ಕೆಳ-ಗಿದ್ದ
ಕೆಳ-ಗಿನ
ಕೆಳ-ಗಿ-ನಂತಿದೆ
ಕೆಳ-ಗಿ-ನಂತಿದ್ದು
ಕೆಳ-ಗಿನಂತಿವೆ
ಕೆಳ-ಗಿ-ನಂತೆ
ಕೆಳ-ಗಿ-ರುವ
ಕೆಳ-ಗಿ-ಳಿಸಿ
ಕೆಳ-ಗಿ-ಳಿಸಿ-ದರು
ಕೆಳ-ಗಿ-ಳಿಸಿ-ದುದಕ್ಕಾಗಿ
ಕೆಳಗೆ
ಕೆಳ-ಗೆರೆ
ಕೆಳ-ಗೆ-ರೆಯ
ಕೆಳ-ಗೆ-ರೆಯೇ
ಕೆಳಗೇ
ಕೆಳ-ತಿರು-ಪತಿಯ
ಕೆಳ-ದರ್ಜೆ
ಕೆಳದಿ
ಕೆಳ-ದಿಯ
ಕೆಳ-ದಿ-ರಾಜರ
ಕೆಳ-ಪಾರ್ಶ್ವಕ್ಕೆ
ಕೆಳ-ಭಾಗ-ದಲ್ಲಿಯೇ
ಕೆಳ-ಭಾಗ-ದಲ್ಲಿ-ರುವ
ಕೆಳಲಿ
ಕೆಳ-ಲಿ-ನಾಡ
ಕೆಳಲೆ
ಕೆಳ-ಲೆ-ಕಿಳಲೆ
ಕೆಳ-ಲೆ-ನಾಡ
ಕೆಳ-ಲೆ-ನಾಡನ್ನು
ಕೆಳ-ಲೆ-ನಾಡಿನ
ಕೆಳ-ಲೆ-ನಾಡಿ-ನಲ್ಲಿದ್ದವು
ಕೆಳ-ಲೆ-ನಾಡು
ಕೆಳ-ಲೆಯ
ಕೆಳ-ಲೆ-ಯ-ನಾಡ
ಕೆಳ-ವಾಡಿ
ಕೆಳಸ್ಥರ-ದಲ್ಲಿ
ಕೆಳ-ಹಂತದ
ಕೆಳ-ಹಂತ-ದಲ್ಲಿ
ಕೆಳೆ
ಕೆಳೆ-ಯದೆ
ಕೆಳೆ-ಯಬ್ಬ-ರಸಿ
ಕೆಳೆ-ಯಬ್ಬ-ರಸಿಯ
ಕೆಳೆ-ಯಬ್ಬ-ರಸಿ-ಯನ್ನು
ಕೆಳೆ-ಯಬ್ಬ-ರಸಿಯು
ಕೆಳೆ-ಯಬ್ಬೆ-ಯ-ಹಳ್ಳಿ-ಯನ್ನು
ಕೆಳೆ-ಯಬ್ಬೆಯು
ಕೆಳೆ-ಯಲ
ಕೆವಿ
ಕೆವಿ-ರಮೇಶ್
ಕೆಸ-ವಿನ-ಕಟ್ಟೆ
ಕೆಸಿ
ಕೆಸೆವ
ಕೆಸ್ತೂರು
ಕೇಂದ್ರ
ಕೇಂದ್ರ-ಗಳ
ಕೇಂದ್ರ-ಗ-ಳಲ್ಲಿ
ಕೇಂದ್ರ-ಗ-ಳಲ್ಲೂ
ಕೇಂದ್ರ-ಗಳಾಗತೊಡಗಿ-ದವು
ಕೇಂದ್ರ-ಗ-ಳಾಗಿ
ಕೇಂದ್ರ-ಗ-ಳಾಗಿದ್ದವು
ಕೇಂದ್ರ-ಗ-ಳಾಗಿ-ರು-ವು-ದನ್ನು
ಕೇಂದ್ರ-ಗಳಿದ್ದವು
ಕೇಂದ್ರ-ಗಳು
ಕೇಂದ್ರ-ಗಳೂ
ಕೇಂದ್ರ-ವನ್ನಾಗಿ
ಕೇಂದ್ರ-ವನ್ನಾಗಿ-ರಿಸಿ-ಕೊಂಡು
ಕೇಂದ್ರ-ವನ್ನಾಗಿ-ಸಿ-ಕೊಂಡು
ಕೇಂದ್ರ-ವನ್ನು
ಕೇಂದ್ರ-ವಾಗಿ
ಕೇಂದ್ರ-ವಾ-ಗಿತ್ತು
ಕೇಂದ್ರ-ವಾಗಿತ್ತೆಂದು
ಕೇಂದ್ರ-ವಾಗಿದೆ
ಕೇಂದ್ರ-ವಾ-ಗಿದ್ದ
ಕೇಂದ್ರ-ವಾಗಿದ್ದ-ರಿಂದ
ಕೇಂದ್ರ-ವಾಗಿದ್ದವು
ಕೇಂದ್ರ-ವಾಗಿದ್ದಿರ-ಬ-ಹುದು
ಕೇಂದ್ರ-ವಾ-ಗಿದ್ದು
ಕೇಂದ್ರ-ವಾಗಿ-ರುವುದು
ಕೇಂದ್ರ-ವಾಗಿ-ಸಿ-ಕೊಂಡಿದ್ದ
ಕೇಂದ್ರ-ವಾದ
ಕೇಂದ್ರ-ವಾಯಿತು
ಕೇಂದ್ರ-ವಾಯಿ-ತೆಂದು
ಕೇಂದ್ರವೂ
ಕೇಂದ್ರ-ವೆಂದು
ಕೇಂದ್ರಸ್ಥಳ-ಗ-ಳಲ್ಲಿ
ಕೇಂದ್ರಸ್ಥಳ-ವನ್ನಾಗಿ
ಕೇಂದ್ರಸ್ಥಳ-ವಾಗಿ
ಕೇಂದ್ರಸ್ಥಾನ-ವನ್ನಾಗಿ-ಸಿ-ಕೊಂಡರು
ಕೇಂದ್ರೀ-ಕರಿಸಿ
ಕೇಂದ್ರೀಯ
ಕೇಂದ್ರೀ-ಯವೇ
ಕೇಣಿಯ
ಕೇತ
ಕೇತ-ಗ-ಉಡ
ಕೇತ-ಗವುಂಡನ
ಕೇತ-ಗವುಂಡನು
ಕೇತ-ಗವುಡು
ಕೇತ-ಗಾವುಂಡ
ಕೇತ-ಗೌಂಡ
ಕೇತ-ಚಮೂಪತಿ
ಕೇತ-ಚಮೂಪ-ತಿಯ
ಕೇತ-ಜೀಯ
ಕೇತಣ
ಕೇತ-ಣ-ವಾ-ಹಿನೀ
ಕೇತಣ್ಣ
ಕೇತಣ್ಣನ
ಕೇತಣ್ಣನು
ಕೇತ-ನ-ಕಟ್ಟ-ಗ-ಳನ್ನು
ಕೇತ-ನ-ಹಟ್ಟಿ
ಕೇತ-ನ-ಹಳ್ಳಿ
ಕೇತ-ನ-ಹಳ್ಳಿ-ಇಂದಿನ
ಕೇತ-ನ-ಹಳ್ಳಿ-ಯನ್ನು
ಕೇತನು
ಕೇತಪ್ಪ
ಕೇತಪ್ಪನ
ಕೇತ-ಮ-ಗೆರೆ
ಕೇತ-ಮಲ್ಲ
ಕೇತಮ್ಮ
ಕೇತಯ್ಯ
ಕೇತಯ್ಯಂಗಳ
ಕೇತಯ್ಯ-ದಂಡ-ನಾಯಕ
ಕೇತಯ್ಯನ
ಕೇತಯ್ಯನು
ಕೇತಯ್ಯನೂ
ಕೇತ-ಲ-ದೇವಿ
ಕೇತ-ಲ-ದೇವಿ-ಯನ್ನು
ಕೇತ-ಲೇಶ್ವರ
ಕೇತವ್ವೆ
ಕೇತಿ
ಕೇತಿ-ಗಾವುಂಡ
ಕೇತಿ-ಯಪ್ಪ
ಕೇತಿ-ಸೆಟ್ಟಿ
ಕೇತಿ-ಸೆಟ್ಟಿಯ
ಕೇತೆಮಾದೆ-ನಾಯ-ಕನು
ಕೇತೆ-ಮಾದೆಯ-ನಾಯಕ
ಕೇತೆ-ಮಾದೆಯ-ನಾಯ-ಕನು
ಕೇತೆಯ
ಕೇತೆಯ-ಕೇತ-ಚಮೂಪತಿ
ಕೇತೆಯ-ದಂಡ-ನಾಯ-ಕನು
ಕೇತೆಯ-ನಾಯಕ
ಕೇತೋಜ
ಕೇತೋಜನ
ಕೇತೋ-ಜರು
ಕೇತ್ತ-ಗೌಂಡನ
ಕೇತ್ರ
ಕೇದ-ಗೆಗೆ-ರೆ-ಗಳ
ಕೇದಗೆ-ಗೆರೆಯ
ಕೇದಾರ
ಕೇದಾರ-ಕೊಂಡೇಶ್ವರ
ಕೇದಾರ-ಕೊಂಡೇಶ್ವರದ
ಕೇದಾರ-ದಿಂದ
ಕೇರಳ
ಕೇರ-ಳದ
ಕೇರಳ-ವಡ್ಡಿಯ
ಕೇರಳಾಧಿ-ಪತಿ-ಯಾಗಿರ್ದೆ
ಕೇರಳಾ-ಪುರ-ವೆಂಬ
ಕೇರಳೆ
ಕೇರಳೇ
ಕೇರಳೇ-ನಾಡಿನ
ಕೇರ-ಹಳ್ಳಿಯ
ಕೇರಾಳ-ನಾಯಕ
ಕೇರಾಳ-ನಾಯ-ಕನು
ಕೇರಾಳ-ನಾಯ-ಕ-ನೆಂದಿದೆ
ಕೇರಾಳ-ಪುರವು
ಕೇರಿ
ಕೇರಿ-ಯಲ್ಲಿ
ಕೇರಿ-ಯೊಳ್
ಕೇಲವ
ಕೇಳಲು
ಕೇಳಿ
ಕೇಳಿ-ಕೊಂಡು
ಕೇಳಿದ
ಕೇಳಿ-ದ-ನೆಂದು
ಕೇಳಿ-ದಾಗ
ಕೇಳಿದೆ
ಕೇಳಿದ್ದೇನೆ
ಕೇಳಿ-ಪಡೆ-ದನು
ಕೇಳಿ-ಬ-ರು-ವು-ದಿಲ್ಲ-ವೆಂದೂ
ಕೇಳಿ-ರ-ಬ-ಹುದು
ಕೇಳುತ್ತಾನೆ
ಕೇಳುತ್ತಿದ್ದ
ಕೇಳುತ್ತಿದ್ದ-ನೆಂದು
ಕೇಳುತ್ತಿದ್ದೆ
ಕೇಳ್ದಿ-ದಿರು-ವಂದು
ಕೇಳ್ದಿರು-ವಂದು
ಕೇಳ್ದು
ಕೇವಲ
ಕೇವಲಿ-ಗಳ
ಕೇಶ-ಪರ್ಯಂತ
ಕೇಶವ
ಕೇಶವ-ಚೆನ್ನ-ಕೇಶವ-ದೇವಾ-ಲಯ
ಕೇಶವ-ದೀಕ್ಷಿತ-ರಿಂದ
ಕೇಶವ-ದೇವರ
ಕೇಶವ-ದೇವ-ರಿಗೆ
ಕೇಶವ-ದೇವರು
ಕೇಶವ-ದೇವಾ-ಲಯ
ಕೇಶವ-ದೇವಾ-ಲ-ಯದ
ಕೇಶವನ
ಕೇಶವ-ನ-ಮೂರ್ತಿ
ಕೇಶವ-ನಾಥ
ಕೇಶವ-ಭಕ್ತ
ಕೇಶವ-ಭಕ್ತ-ನಾದ
ಕೇಶವ-ಭಟ್ಟ
ಕೇಶವಾ-ಪುರ
ಕೇಶಾಲಂಕಾರ-ಗ-ಳನ್ನೂ
ಕೇಶಿ-ಯಣ್ಣ
ಕೇಶಿ-ಯಣ್ಣನು
ಕೇಶಿ-ರಾಜ
ಕೇಶಿ-ರಾಜನ
ಕೇಶಿ-ರಾಜನು
ಕೇಸರೀ
ಕೇಸವ-ನಾಯ-ಕ-ರು-ಗಳ
ಕೇಸವಯ್ಯ-ನಿಗೆ
ಕೇಸವ-ಸೆಟ್ಟಿ
ಕೇಸಿಗ
ಕೇಸಿಗ-ನಿಗೆ
ಕೇಸಿ-ಮಯ್ಯ
ಕೇಸಿ-ಯಣ್ಣನು
ಕೈ
ಕೈಂಕರ್ಯ
ಕೈಂಕರ್ಯಂ
ಕೈಂಕರ್ಯಕೆ
ಕೈಂಕರ್ಯಕ್ಕೆ
ಕೈಂಕರ್ಯ-ಗಳ
ಕೈಂಕರ್ಯ-ಗ-ಳನ್ನು
ಕೈಂಕರ್ಯ-ಗಳಿ-ಗಾಗಿ
ಕೈಂಕರ್ಯ-ಗ-ಳಿಗೆ
ಕೈಂಕರ್ಯ-ಗಳು
ಕೈಂಕರ್ಯದ
ಕೈಂಕರ್ಯ-ವನ್ನು
ಕೈಂಕರ್ಯ-ವಾಗಿ
ಕೈಂಕರ್ಯ್ಯ-ವನ್ನು
ಕೈಅದ್ದಿ
ಕೈಕಂರ್ಯ-ಗಳ
ಕೈಕಾಲು
ಕೈಕೆಳ-ಗಿನ
ಕೈಕೆಳಗೆ
ಕೈಕೊಂಡ
ಕೈಕೊಂಡ-ರಾರ್
ಕೈಕೊಂಡ-ರಾರ್ಚ್ಚೋಳನಂ
ಕೈಕೊಂಡು
ಕೈಕೊಳೆ
ಕೈಗಳಿವೆ
ಕೈಗಾಣ-ವನ್ನು
ಕೈಗಾರಿಕೆ
ಕೈಗಾರಿಕೆ-ಗಳ
ಕೈಗಿತ್ತ
ಕೈಗೆ
ಕೈಗೊಂಡ-ನ-ಪಲ್ಲಿ-ಕೈಗೋನ-ಹಳ್ಳಿ
ಕೈಗೊಂಡ-ನ-ಪಲ್ಲಿ-ಯನ್ನು
ಕೈಗೊಂಡ-ನ-ಪಲ್ಲಿಯು
ಕೈಗೊಂಡ-ನ-ಪಳ್ಳಿ
ಕೈಗೊಂಡ-ನೆಂದು
ಕೈಗೊಂಡಿದ್ದ
ಕೈಗೊಂಡಿರುತ್ತಿದ್ದ
ಕೈಗೊಂಡು
ಕೈಗೊಳ್ಳುತ್ತಿದ್ದ
ಕೈಗೊಳ್ಳುತ್ತಿದ್ದ-ರೆಂದು
ಕೈಗೊಳ್ಳುವ
ಕೈಗೋನ-ಹಳ್ಳಿ
ಕೈಗೋನ-ಹಳ್ಳಿ-ಗ-ಳಲ್ಲಿ
ಕೈಗೋನ-ಹಳ್ಳಿಯ
ಕೈಗೋನ-ಹಳ್ಳಿ-ಯ-ವರೆಗೂ
ಕೈತಪ್ಪಿ
ಕೈತಪ್ಪಿ-ಹೋಗಿದ್ದವು
ಕೈದಾ-ಳದ
ಕೈದೀ-ವಿಗೆಗೆ
ಕೈಪಿ-ಡಿಯ
ಕೈಪಿಡಿ-ಯಂತೆ
ಕೈಪಿ-ಡಿ-ಯಲ್ಲಿ
ಕೈಫಿ-ಯತ್ತು
ಕೈಫಿ-ಯತ್ತು-ಗ-ಳನ್ನು
ಕೈಫಿ-ಯತ್ತು-ಗ-ಳಲ್ಲಿ
ಕೈಫಿ-ಯತ್ತು-ಗಳು
ಕೈಬಿಟ್ಟರು
ಕೈಬಿಟ್ಟರೂ
ಕೈಬಿಟ್ಟಿದ್ದೇನೆ
ಕೈಬಿಟ್ಟು
ಕೈಬಿಟ್ಟು-ಹೋಗಿದ್ದ
ಕೈಬಿಟ್ಟು-ಹೋದವು
ಕೈಬಿಡ-ಬ-ಹುದು
ಕೈಬಿಡು-ವುದು
ಕೈಮುಗಿದು
ಕೈಮುಗಿ-ಯುತ್ತಿ-ದುದು-ರಲ್ಲಿ
ಕೈಯ
ಕೈಯಲಿ
ಕೈಯಲು
ಕೈಯಲ್ಲಿ
ಕೈಯಲ್ಲಿತ್ತು
ಕೈಯಲ್ಲಿದ್ದು-ದ-ರಿಂದ
ಕೈಯಲ್ಲಿಯೇ
ಕೈಯಲ್ಲೇ
ಕೈಯಿಂದ
ಕೈಯ್ಯಲ್ಲಿ
ಕೈಲಾದ
ಕೈಲಾಸ
ಕೈಲಾಸ-ದೇವ-ರಿಗೆ
ಕೈಲಾಸ-ನಾಥ
ಕೈಲಾಸಪ್ರಾಪ್ತ-ರಾಗುತ್ತಾರೆ
ಕೈಲಾಸ-ಮಯ-ಡೈ-ಯಾರ್
ಕೈಲಾಸ-ಮುಡೆ-ಯಾರ್
ಕೈಲಾಸ-ಮುಡೈ-ಯಾರ್
ಕೈಲಾಸಸ್ಥಾನ-ದಲ್ಲಿ
ಕೈಲಾಸೇಶ್ವರ
ಕೈಲಿ
ಕೈಲಿದ್ದ
ಕೈಲಿದ್ದು-ದನ್ನು
ಕೈಳಾಸ-ದಿನೊಸೆದೀ
ಕೈವಲ್ಯೇಶ್ವರ
ಕೈವಶ-ವಾಗ-ದಾ-ಯಿತು
ಕೈವಾಡಕ್ಕೆ
ಕೈವಾರ-ಕರ-ನಿರೋಧಕ
ಕೈವಾರ-ನಿಸಂಕ-ಮಲ್ಲ
ಕೈಶಿಕ-ಕೌಶಿಕ
ಕೈಸಾರ್ವ್ವಿನಂ
ಕೈಸಾಲೆ
ಕೈಸಾಲೆ-ಗಳ
ಕೈಸಾಲೆಯ
ಕೈಸಾಲೆ-ಯಲ್ಲಿ
ಕೈಸಾಲೆ-ಯಲ್ಲಿ-ರುವ
ಕೈಸೆರೆ-ಯಾ-ಗಿದ್ದ
ಕೈಸೇರಿ
ಕೈಹಾಕಿ
ಕೊ
ಕೊಂಕಣ
ಕೊಂಕ-ಣದ
ಕೊಂಕಣೇಶ್ವರ
ಕೊಂಗ
ಕೊಂಗಣಿ
ಕೊಂಗ-ಣಿ-ಕೆರೆ-ಯನ್ನು
ಕೊಂಗ-ಣಿ-ವರ್ಮ
ಕೊಂಗ-ನವೆ-ಯವ-ದಿಂದಂ
ಕೊಂಗ-ನಾಡ-ಕೊಂಗಾಳ್ನಾಡ
ಕೊಂಗ-ನಾಡನ್ನಾಳುತ್ತಿದ್ದಾಗ
ಕೊಂಗ-ನಾಡು
ಕೊಂಗ-ನಾಡೆಳ್ಪತ್ತುಂ
ಕೊಂಗ-ಪಟ್ಟ-ಣದ
ಕೊಂಗ-ಮಾರಿ
ಕೊಂಗ-ಯರ್
ಕೊಂಗರ
ಕೊಂಗ-ರ-ದಿಶಾ-ಪಟ್ಟ
ಕೊಂಗ-ರ-ನಡಗಿಸಿ
ಕೊಂಗ-ರನ್ನು
ಕೊಂಗ-ರಿಳಂಚಿಂಗರು
ಕೊಂಗಲ್ನಾಡಿನ
ಕೊಂಗಲ್ನಾಡೊಳ-ಗಣ
ಕೊಂಗಳ್ನಾ-ಡಿಗೆ
ಕೊಂಗಳ್ನಾಡಿನ
ಕೊಂಗಳ್ನಾಡು
ಕೊಂಗಳ್ನಾಡೆಂದು
ಕೊಂಗ-ಸೇನೆ-ಯನ್ನು
ಕೊಂಗಾಳೇಶ್ವರ
ಕೊಂಗಾಳೇಸ್ವರ
ಕೊಂಗಾಳ್ನಾಡ
ಕೊಂಗಾಳ್ನಾಡಿನ
ಕೊಂಗಾಳ್ನಾಡಿ-ನಲ್ಲಿ
ಕೊಂಗಾಳ್ನಾಡಿನಲ್ಲಿದ್ದ
ಕೊಂಗಾಳ್ನಾಡು
ಕೊಂಗಾಳ್ವ
ಕೊಂಗಾಳ್ವ-ದೇವ-ನಿಂದ
ಕೊಂಗಾಳ್ವ-ದೇವನು
ಕೊಂಗಾಳ್ವನು
ಕೊಂಗಾಳ್ವರ
ಕೊಂಗಾಳ್ವ-ರನ್ನು
ಕೊಂಗಾಳ್ವ-ರಾಜ-ಕು-ಮಾರಿ
ಕೊಂಗಾಳ್ವ-ರಿ-ಗಿದ್ದ
ಕೊಂಗಾಳ್ವರು
ಕೊಂಗಾಳ್ವ-ರು-ಚೆಂಗಾಳ್ವರ
ಕೊಂಗಾಳ್ವ-ರೊಡನೆ
ಕೊಂಗಾಳ್ವ-ಸಿದ್ಧಿ
ಕೊಂಗು
ಕೊಂಗು-ಕೊಂಡ
ಕೊಂಗುಣಿ
ಕೊಂಗು-ಣಿ-ಮುತ್ತ-ರಸ
ಕೊಂಗು-ಣಿ-ವರ್ಮ
ಕೊಂಗು-ಣಿ-ವರ್ಮನು
ಕೊಂಗು-ದೇಶ
ಕೊಂಗು-ದೇಶೈಕ
ಕೊಂಗು-ನಾಡನ್ನು
ಕೊಂಗು-ನಾಡಿನ
ಕೊಂಗು-ನಾಡಿನಿಂದ
ಕೊಂಗು-ನಾಡು
ಕೊಂಗು-ರಿಳಂಜೈ-ಗರ್
ಕೊಂಗು-ರಿಳಅಂಚಿಂಗ-ರುಮ್
ಕೊಂಚ-ನಿರಾಶ-ನಾದ
ಕೊಂಚ-ಭಾಗ
ಕೊಂಡ
ಕೊಂಡಂ
ಕೊಂಡ-ಕುಂದ
ಕೊಂಡ-ಕುಂದನ್ವಯದ
ಕೊಂಡ-ಕುಂದಾನ್ವದ
ಕೊಂಡ-ಕುಂದಾನ್ವಯ
ಕೊಂಡ-ಕುಂದಾನ್ವಯಂ
ಕೊಂಡ-ಕುಂದಾನ್ವಯದ
ಕೊಂಡ-ಗುಳಿ
ಕೊಂಡ-ದ-ಹಬ್ಬ
ಕೊಂಡ-ನಸಮ
ಕೊಂಡ-ನಿಂತು
ಕೊಂಡಯ್ಯ-ದೇವ
ಕೊಂಡ-ರಾ-ಜಯ-ದೇವ
ಕೊಂಡ-ರಾಜಯ್ಯ-ದೇವ
ಕೊಂಡ-ವನ್ನು
ಕೊಂಡ-ಹಾಗೆ
ಕೊಂಡಾ-ಡಿವೆ
ಕೊಂಡಾನ್
ಕೊಂಡಾಳ್
ಕೊಂಡಿದ್ದಾರೆ
ಕೊಂಡು
ಕೊಂಡೆ-ಯವ-ನಾಡಿದ
ಕೊಂಡೊಯ್ದು
ಕೊಂಡೊಯ್ಯುತ್ತಿದ್ದ
ಕೊಂಡೊಯ್ಯುತ್ತಿದ್ದಾರೆ
ಕೊಂಡೊಯ್ಯುವ
ಕೊಂತದ
ಕೊಂತಿ-ದೇವಿ-ಗಧಿಕಂ
ಕೊಂತಿಯ
ಕೊಂದ
ಕೊಂದ-ನಲ್ಲದೆ
ಕೊಂದನು
ಕೊಂದ-ನೆಂದು
ಕೊಂದರೆ
ಕೊಂದಿಕ್ಕಿ
ಕೊಂದಿಕ್ಕಿ-ದನೊಕ್ಕಿ-ಲಿಕ್ಕಿ
ಕೊಂದಿರುವ
ಕೊಂದು
ಕೊಂದು-ದಕ್ಕಾಗಿ
ಕೊಂದು-ದಕ್ಕಾಗಿಯೇ
ಕೊಂದು-ದಕ್ಕೆ
ಕೊಂದು-ಹಾಕಿದ
ಕೊಂದು-ಹಾಕಿ-ದಂತೆ
ಕೊಂಬಾಳೆ-ಯಲ್ಲಿ
ಕೊಂಬು
ಕೊಂಬು-ದುಮಾ-ತನ
ಕೊಂಮೆಯರ
ಕೊಂಮೆಯ-ರ-ಕೊಮ್ಮೆಯರು
ಕೊಂಮೇಶ್ವರ
ಕೊಕೊಳಗ
ಕೊಗಿ-ಯೂರು
ಕೊಙ್ಗು-ಕೊಂಡ
ಕೊಚನ್ನಬಸಪ್ಪ
ಕೊಚ್ಚಿ-ಹೋಗ-ದಂತೆ
ಕೊಟ
ಕೊಟಟ್ಟ
ಕೊಟೆ-ವಾಗಿ
ಕೊಟ್ಟ
ಕೊಟ್ಟಂತಹ
ಕೊಟ್ಟಂತೆ
ಕೊಟ್ಟ-ಗದ್ಯಾಣ
ಕೊಟ್ಟಗೆ
ಕೊಟ್ಟನು
ಕೊಟ್ಟ-ನೆಂದು
ಕೊಟ್ಟರ
ಕೊಟ್ಟ-ರದ
ಕೊಟ್ಟ-ರ-ವೆಗ್ಗಡೆ
ಕೊಟ್ಟರು
ಕೊಟ್ಟ-ರೆಂದು
ಕೊಟ್ಟ-ರೆಂದೂ
ಕೊಟ್ಟ-ಲಿಗೆ-ಎಂದು
ಕೊಟ್ಟಳು
ಕೊಟ್ಟಾಗ
ಕೊಟ್ಟಿಗೆ-ಗಳ
ಕೊಟ್ಟಿಗೆ-ಯಲ್ಲಿ-ರುವ
ಕೊಟ್ಟಿತು
ಕೊಟ್ಟಿತ್ತು
ಕೊಟ್ಟಿದೆ
ಕೊಟ್ಟಿದ್ದ
ಕೊಟ್ಟಿದ್ದ-ನೆಂದು
ಕೊಟ್ಟಿದ್ದ-ರಿಂದ
ಕೊಟ್ಟಿದ್ದರೋ
ಕೊಟ್ಟಿದ್ದಾನೆ
ಕೊಟ್ಟಿದ್ದಾರೆ
ಕೊಟ್ಟಿದ್ದೀರಾ
ಕೊಟ್ಟಿದ್ದು
ಕೊಟ್ಟಿದ್ದೇ
ಕೊಟ್ಟಿರ-ಬಹು-ದೆಂದು
ಕೊಟ್ಟಿರುತ್ತಾನೆ
ಕೊಟ್ಟಿರುತ್ತಾರೆ
ಕೊಟ್ಟಿ-ರುವ
ಕೊಟ್ಟಿ-ರುವಂತೆ
ಕೊಟ್ಟು
ಕೊಟ್ಟು-ಬಹೆವು
ಕೊಟ್ಟೊಡಲ್ಲಿ
ಕೊಟ್ಟೋನೆಱೆ-ಯಪ್ಪ
ಕೊಠಡಿ-ಗ-ಳನ್ನೂ
ಕೊಠಡಿ-ಯಲ್ಲಿ
ಕೊಠದು
ಕೊಠಾರ
ಕೊಠಾರಕ್ಕೆ
ಕೊಠಾರ-ದಲ್ಲಿ
ಕೊಡ-ಗನ್ನು
ಕೊಡ-ಗರು
ಕೊಡ-ಗ-ಹಳ್ಳಿ
ಕೊಡ-ಗ-ಹಳ್ಳಿಯ
ಕೊಡ-ಗಿ-ದೆರೆ
ಕೊಡ-ಗಿಯ
ಕೊಡ-ಗಿ-ಯಾಗಿ
ಕೊಡಗು
ಕೊಡಗೆ
ಕೊಡ-ಗೆ-ದೆರೆ
ಕೊಡ-ಗೆಯ
ಕೊಡ-ಗೆ-ಯನ್ನು
ಕೊಡ-ಗೆ-ಯಾಗಿ
ಕೊಡ-ಗೆ-ಹಳ್ಳಿ
ಕೊಡದೇ
ಕೊಡ-ಬ-ಹುದು
ಕೊಡ-ಬಾ-ರದು
ಕೊಡ-ಬೇ-ಕಾದ
ಕೊಡ-ಬೇಕು
ಕೊಡ-ಲಾ-ಗದ
ಕೊಡ-ಲಾಗಿದೆ
ಕೊಡ-ಲಾ-ಗಿದ್ದು
ಕೊಡ-ಲಾಗಿ-ರುತ್ತದೆ
ಕೊಡ-ಲಿಲ್ಲ
ಕೊಡಲು
ಕೊಡ-ಲೇ-ಬೇಕಾಗಿತ್ತು
ಕೊಡ-ವನ್ನು
ಕೊಡಿ
ಕೊಡಿಗಿ
ಕೊಡಿಗೆ
ಕೊಡಿ-ಯಾಲ
ಕೊಡಿ-ಸಿದ-ನೆಂದು
ಕೊಡಿ-ಸಿದ-ರೆಂದು
ಕೊಡು
ಕೊಡುಂಗೆ-ಯಾಗಿ
ಕೊಡು-ಉ-ದಕೆ
ಕೊಡುಗೆ
ಕೊಡು-ಗೆ-ಗಳ
ಕೊಡು-ಗೆ-ಗ-ಳನ್ನು
ಕೊಡು-ಗೆ-ಗಳು
ಕೊಡು-ಗೆಗೆ
ಕೊಡು-ಗೆ-ದೆರೆ-ಯನ್ನು
ಕೊಡು-ಗೆಯ
ಕೊಡು-ಗೆ-ಯನ್ನು
ಕೊಡು-ಗೆ-ಯನ್ನೂ
ಕೊಡು-ಗೆ-ಯಲ್ಲಿ
ಕೊಡು-ಗೆ-ಯಾಗಿ
ಕೊಡು-ಗೆ-ಹಳ್ಳಿ
ಕೊಡುತ್ತ-ದೆಂದು
ಕೊಡುತ್ತಾ
ಕೊಡುತ್ತಾನೆ
ಕೊಡುತ್ತಾರೆ
ಕೊಡುತ್ತಾಳೆ
ಕೊಡುತ್ತಿದ್ದ
ಕೊಡುತ್ತಿದ್ದನು
ಕೊಡುತ್ತಿದ್ದರು
ಕೊಡುತ್ತಿದ್ದುದೂ
ಕೊಡುತ್ತಿದ್ದೆ
ಕೊಡುತ್ತಿರ-ಲಿಲ್ಲ-ವೆಂದು
ಕೊಡುತ್ತಿವೆ
ಕೊಡುವ
ಕೊಡು-ವಂತೆ
ಕೊಡು-ವ-ವನೇ
ಕೊಡು-ವುದ-ರಲ್ಲಿ
ಕೊಡು-ವುದು
ಕೊಡೆ
ಕೊಡೆ-ಹಾಳ
ಕೊಡೆ-ಹಾಳದ
ಕೊಣ-ನೂರು
ಕೊಣೆ-ಹಳ್ಳಿ
ಕೊಣ್ಡಕುಂದಾನ್ವಯದ
ಕೊಣ್ಡನಾ
ಕೊಣ್ಡೋನ್
ಕೊತ್ತತ್ತಿ
ಕೊತ್ತತ್ತಿಯ
ಕೊತ್ತತ್ತಿ-ಯನ್ನು
ಕೊತ್ತನ-ಕೆರೆ
ಕೊತ್ತಲ-ವಾಡಿ
ಕೊತ್ತಲು-ಗ-ಳನ್ನು
ಕೊತ್ತಲು-ಗಳಿವೆ
ಕೊತ್ತಳಿ
ಕೊತ್ತಳಿ-ಗುತ್ತ
ಕೊತ್ತಳಿಗೆ
ಕೊತ್ತಳಿ-ಯ-ಉಲ್ಲೇಖ-ವಿದ್ದು
ಕೊತ್ತಳಿ-ಯನ್ನು
ಕೊತ್ತಳಿ-ಯನ್ನು-ಸಂಘ
ಕೊತ್ತಳಿ-ಯ-ವರು
ಕೊತ್ತಾ-ಗಾಲ
ಕೊತ್ತಿತ್ತಿ
ಕೊತ್ತಿ-ಬೈಚ-ನ-ಕಟ್ಟೆ
ಕೊತ್ತಿ-ವರ-ದ-ಹಳ್ಳಿ-ಯನ್ನು
ಕೊನಯ
ಕೊನೆ
ಕೊನೆ-ಗಾಣಿಸಿ-ಕೊಂಡಾಗ
ಕೊನೆ-ಗಾಣಿಸಿ-ಕೊಳ್ಳುತ್ತಿದ್ದರು
ಕೊನೆ-ಗಾಲ-ದಲ್ಲಿ
ಕೊನೆಗೆ
ಕೊನೆ-ಗೊಳಿ-ಸಿ-ದನು
ಕೊನೆಯ
ಕೊನೆ-ಯ-ಬಾರಿಗೆ
ಕೊನೆ-ಯಲ್ಲಿ
ಕೊನೆ-ಯ-ವರ್ಷದ
ಕೊನೆ-ಯಾಗು-ವು-ದನ್ನು
ಕೊನೆ-ಯಾದ
ಕೊನೆ-ಯು-ಸಿರೆ-ಳೆಯುತ್ತಾನೆ
ಕೊನೆರಿ
ಕೊನೆ-ರಿನ್ಮೈ
ಕೊನೆ-ರೀ-ದೇವ-ನೆಂದು
ಕೊನೇರಿ
ಕೊನೇರಿ-ದೇವ
ಕೊನೇರಿಮ್ಮೈಕೊಣ್ಡಾಣ್
ಕೊನ್ತದ
ಕೊನ್ದ-ಡಿಯ
ಕೊನ್ದ-ಡಿಯು
ಕೊನ್ದು
ಕೊನ್ನಾ-ಪುರ
ಕೊಪ-ಣಾದಿ-ತೀರ್ಥ
ಕೊಪ್ಪ
ಕೊಪ್ಪದ
ಕೊಪ್ಪ-ಲನ್ನೂ
ಕೊಪ್ಪ-ಲಿಗೆ
ಕೊಪ್ಪ-ಲಿನ
ಕೊಪ್ಪ-ಲಿ-ನಲ್ಲಿ
ಕೊಪ್ಪ-ಲಿನಲ್ಲಿಯೂ
ಕೊಪ್ಪ-ಲಿನ-ವರು
ಕೊಪ್ಪಲು
ಕೊಪ್ಪಳ
ಕೊಪ್ಪ-ಳ-ದಲ್ಲಿ
ಕೊಮರ-ನ-ಹಳ್ಳಿ
ಕೊಮಾರ
ಕೊಮಾರ-ತಿ-ಯನ್ನು
ಕೊಮಾರ-ನಾದ
ಕೊಮಾರರು
ಕೊಮಾರ-ಸೇನ
ಕೊಮ್ಬಾಳೆ-ಯೊಳ್
ಕೊಮ್ಮಣ್ಣ
ಕೊಮ್ಮಣ್ಣನು
ಕೊಮ್ಮ-ರಾಜ
ಕೊಮ್ಮ-ರಾಜಂ
ಕೊಮ್ಮ-ರಾಜನ
ಕೊಮ್ಮ-ರಾಜ-ನಿರ-ಬ-ಹುದು
ಕೊಮ್ಮ-ರಾಜನೇ
ಕೊಮ್ಮೆಯರ
ಕೊಮ್ಮೆಯರು
ಕೊಮ್ಮೇಶ್ವರ
ಕೊಯತೂ
ಕೊಯಲು
ಕೊಯಳ-ರ-ಸನು
ಕೊಯಿಲೋ
ಕೊಯ್ಯುವ
ಕೊಯ್ಲಾಳಿ-ಗಳು
ಕೊರಕ-ಗೌಡ
ಕೊರಕಲ-ಹಳ್ಳಕ್ಕೆ
ಕೊರಕಲ-ಹಳ್ಳ-ಗಳ
ಕೊರಟಿಯ-ಹಳ್ಳಿ-ಯನ್ನು
ಕೊರಟಿ-ಹಳ್ಳಿ
ಕೊರಟಿ-ಹಳ್ಳಿಯ
ಕೊರಟಿ-ಹಳ್ಳಿ-ಯನ್ನು
ಕೊರತೆ
ಕೊರಳ-ಹಾರದ
ಕೊರಿ-ಕೆಯ
ಕೊರೆದ
ಕೊರೆ-ದಿದ್ದಾರೆ
ಕೊರೆದು
ಕೊರೆಯಿಸಿ
ಕೊಲು-ವಲ್ಲಿ
ಕೊಲೆ
ಕೊಲ್ಲಲು
ಕೊಲ್ಲಾ-ಪುರ
ಕೊಲ್ಲಾ-ಪುರಕ್ಕೆ
ಕೊಲ್ಲಾ-ಪುರದ
ಕೊಲ್ಲಾ-ಪುರ-ದಲ್ಲಿ-ರು-ವುದು
ಕೊಲ್ಲಾ-ಪುರವು
ಕೊಲ್ಲಿ-ಪಲ್ಲವ
ಕೊಲ್ಲಿ-ಪಲ್ಲ-ವ-ನೊಳಂಬ-ನೆಂಬ
ಕೊಲ್ಲಿ-ಪೊಲ್ಲವ
ಕೊಲ್ಲಿ-ಯಮ್ಮೆಯಂಗಳ
ಕೊಲ್ಲಿ-ಯ-ರಸ
ಕೊಲ್ಲಿ-ಯ-ರ-ಸನು
ಕೊಲ್ಲಿ-ಯಲ್ಲಿ
ಕೊಲ್ಲಿ-ಸಿದ-ನೆಂದು
ಕೊಲ್ಲುತ್ತಾನೆ
ಕೊಲ್ಲುತ್ತಿ-ರುವ
ಕೊಲ್ಲುವ
ಕೊಳ
ಕೊಳ-ಕೂರಿನ
ಕೊಳ-ಕೂರಿನಲ್ಲಿದ್ದು
ಕೊಳಗ
ಕೊಳ-ಗಕ್ಕೆ
ಕೊಳ-ಗ-ಗಳ
ಕೊಳ-ಗ-ದಲಿ
ಕೊಳ-ಗ-ದಲು
ಕೊಳ-ಗ-ದಲೂ
ಕೊಳ-ಗ-ದಲ್ಲಿ
ಕೊಳ-ಗ-ಬಳ್ಳ-ಮಾನ
ಕೊಳ-ಗ-ಮೂರು
ಕೊಳ-ಗಳ
ಕೊಳ-ಗ-ಳನ್ನು
ಕೊಳ-ಗ-ಳನ್ನೂ
ಕೊಳ-ಗಳಿದ್ದರೂ
ಕೊಳ-ಗಳು
ಕೊಳ-ಗವು
ಕೊಳ-ಗ-ವೆರಡಕ್ಕಂ
ಕೊಳ-ಗ-ವೊಂದು
ಕೊಳ-ತೂರು-ಇಂದಿನ
ಕೊಳದ
ಕೊಳ-ದಿಂದ
ಕೊಳ-ಲಿಲ್ಲ
ಕೊಳ-ವನ್ನು
ಕೊಳವೂ
ಕೊಳಾ-ಲದ
ಕೊಳಿ-ಗೆರೆ
ಕೊಳು-ಗುಂದದ
ಕೊಳು-ವಲ್ಲಿ
ಕೊಳೆ-ಗೊಳು
ಕೊಳೆತು
ಕೊಳ್
ಕೊಳ್ತಿಲೆ
ಕೊಳ್ಳ
ಕೊಳ್ಳ-ಗಳ
ಕೊಳ್ಳದೆ
ಕೊಳ್ಳ-ಬೇಕು
ಕೊಳ್ಳಲು
ಕೊಳ್ಳಿ
ಕೊಳ್ಳಿ-ಅಯ್ಯ
ಕೊಳ್ಳಿ-ಅಯ್ಯನ
ಕೊಳ್ಳಿ-ಪಾಕೆ
ಕೊಳ್ಳಿ-ಮಯ್ಯ
ಕೊಳ್ಳಿ-ಮಯ್ಯನ
ಕೊಳ್ಳಿ-ಯಮ್ಮನ
ಕೊಳ್ಳಿ-ಯಮ್ಮೆಯಂಗಳ
ಕೊಳ್ಳಿ-ಯಮ್ಮೆಯ್ಯ
ಕೊಳ್ಳಿ-ಯಮ್ಮೆಯ್ಯಂಗಳ
ಕೊಳ್ಳಿ-ಯಮ್ಮೆಯ್ಯನ
ಕೊಳ್ಳುತ್ತಿದ್ದ
ಕೊಳ್ಳುತ್ತಿದ್ದರು
ಕೊಳ್ಳುವ
ಕೊಳ್ಳುವ-ವರು
ಕೊಳ್ಳೆ-ಗಾಲ
ಕೊಳ್ಳೇ-ಗಾಲ
ಕೊಳ್ಳೇ-ಗಾಲದ
ಕೊಳ್ಳೋಕೆ
ಕೊವರ-ಗೊಳತ್ತೊಂ
ಕೊವಳೆವೆಟ್ಟು
ಕೋಗಳಿ
ಕೋಗಳಿ-ನಾಡು
ಕೋಗಿಲಲಿ
ಕೋಟ-ಗಾರರ
ಕೋಟಿ-ಯುತ
ಕೋಟಿ-ವಾಳ
ಕೋಟೆ
ಕೋಟೆ-ಕುರ
ಕೋಟೆ-ಕೊತ್ತಲು
ಕೋಟೆ-ಕೊತ್ತಲು-ಗಳ
ಕೋಟೆ-ಕೊತ್ತಲು-ಗ-ಳನ್ನು
ಕೋಟೆ-ಗ-ಳನ್ನು
ಕೋಟೆಗೆ
ಕೋಟೆ-ಬೆಟ್ಟ
ಕೋಟೆ-ಬೆಟ್ಟದ
ಕೋಟೆಯ
ಕೋಟೆ-ಯನ್ನು
ಕೋಟೆ-ಯಪ್ಪಂಗೆ
ಕೋಟೆ-ಯ-ಬ-ಯಲ
ಕೋಟೆ-ಯಲ್ಲಿ
ಕೋಟೆ-ಯೆಂಬ
ಕೋಟೆ-ಯೊಳ-ಗಿ-ರುವ
ಕೋಟೆ-ಯೊಳಗೆ
ಕೋಟ್ಟ
ಕೋಡಾಲ
ಕೋಡಾ-ಲದ
ಕೋಡಾಲ-ವನ್ನು
ಕೋಡಾಲವು
ಕೋಡಿ
ಕೋಡಿ-ಗ-ಳನ್ನೂ
ಕೋಡಿ-ಗ-ಳಿಂದ
ಕೋಡಿ-ಗ-ಳಿಂದೊಳ-ಗುಳ್ಳ
ಕೋಡಿ-ಗಳಿದ್ದಿರ-ಬಹು-ದೆಂದು
ಕೋಡಿ-ಗಳಿದ್ದು-ದನ್ನು
ಕೋಡಿ-ಗಳಿದ್ದುದು
ಕೋಡಿ-ಗಳಿ-ರುತ್ತಿದ್ದವು
ಕೋಡಿ-ಗಳಿ-ರುವ
ಕೋಡಿ-ಗಳಿ-ರು-ವು-ದನ್ನು
ಕೋಡಿ-ಗಳು
ಕೋಡಿ-ನ-ಕೊಪ್ಪ
ಕೋಡಿ-ನ-ಕೊಪ್ಪ-ಕೋಡಿ-ಹಳ್ಳಿ
ಕೋಡಿ-ಪುರ
ಕೋಡಿ-ಮಾಳ-ಕೆರೆ
ಕೋಡಿಯ
ಕೋಡಿ-ಯನ್ನು
ಕೋಡಿ-ಯಲಿ
ಕೋಡಿ-ಯಲ್ಲಿ
ಕೋಡಿ-ಯಲ್ಲಿದ್ದ
ಕೋಡಿ-ಯ-ಹಳ್ಳ-ಗಳ
ಕೋಡಿ-ಯ-ಹಳ್ಳದ
ಕೋಡಿಯಿಂ
ಕೋಡಿ-ಯಿಂದ
ಕೋಡಿಯು
ಕೋಡಿಯುಂ
ಕೋಡಿ-ಯುಳ್ಳ
ಕೋಡಿ-ಹಳ್ಳಿ
ಕೋಡಿ-ಹಳ್ಳಿಯ
ಕೋಡಿ-ಹಳ್ಳಿ-ಯಲ್ಲಿದೆ
ಕೋಡಿ-ಹಳ್ಳಿ-ಯಲ್ಲಿದ್ದಿರ-ಬಹು-ದಾದ
ಕೋಡ್ಗಲ್ಲಿನ
ಕೋಣದ
ಕೋಣನ-ಕಲ್ಲು
ಕೋಣೆ
ಕೋಣೆಯ
ಕೋಣ್ಡೋಮ್ಮಹಾ-ಪಾ-ತಕನ್
ಕೋತನ-ಪುರ
ಕೋದಂಡ-ರಾಮ
ಕೋದಂಡ-ರಾಮಸ್ವಾಮಿ
ಕೋದಂಡ-ರಾಮಸ್ವಾಮಿಯ
ಕೋದೈಯಾಣ್ಡಾಳಮ್ಮೈ
ಕೋದೈಯಾಣ್ಡಾಳ್ಅಮ್ಮೈ
ಕೋನಾ-ಪುರ
ಕೋನೇಟಿ
ಕೋನೇರಿ
ಕೋನೇ-ರಿನ್ಮೈ
ಕೋನೇರಿಮ್ಮೈ
ಕೋನೇರಿಮ್ಮೈ-ಕೊಣ್ಡಾನ್
ಕೋಮಟಿ
ಕೋಮಟಿ-ಗಳ
ಕೋಮನ-ಹಳ್ಳಿ
ಕೋಮಲ-ಕಟ್ಟೆಯ
ಕೋಮಿ-ನ-ವರು
ಕೋಯಿರ್ರಮ್
ಕೋಯಿಲ್
ಕೋಯಿಲ್ನ
ಕೋರವಂಗಲ
ಕೋರವಂಗಲದ
ಕೋರಿಕೆ
ಕೋರಿ-ಕೆಯ
ಕೋರಿ-ದನು
ಕೋರುತ್ತಾನೆ
ಕೋರು-ವಾರು
ಕೋರೆ-ಗಾಲ
ಕೋರೆ-ಗಾಲದ
ಕೋಲ
ಕೋಲಾರ
ಕೋಲಾ-ಹ-ಲರುಂ
ಕೋಲೋಜ
ಕೋಲೋ-ಜನು
ಕೋಳಾಲ
ಕೋಳಾಲ-ಪುರದ
ಕೋಳಾಲ-ಪುರ-ಪರ-ಮೇಶ್ವರ
ಕೋಳಾಲ-ಬ-ಯಲು-ನಾಡುಮಂ
ಕೋಳಾಹಳ
ಕೋಳಿ-ಗಾಲ
ಕೋಳೋ-ಗಾಲ
ಕೋಳೋ-ಗಾಲ-ವನ್ನು
ಕೋವರುಂ
ಕೋವಿದುರುಂ
ಕೋಶ-ದಲ್ಲಿ
ಕೋಶವು
ಕೋಶಸ್ಯ
ಕೋಶಾಧಿ-ಕಾರಿ
ಕೋಸಲ
ಕೋಸಲೈ-ಪುರ
ಕೌಂಗಿನ
ಕೌಂಡಿಣ್ಯ
ಕೌಂಡಿನ್ಯ
ಕೌಂಡಿಲ್ಯ
ಕೌಟುಂಬಿಕ
ಕೌಡಲಿ
ಕೌಡಲಿ-ಇಂದಿನ
ಕೌಡ್ಲೆ
ಕೌಡ್ಲೆಯು
ಕೌಣ್ಡಿಲ್ಯ
ಕೌಲನ್ನು
ಕೌಲಿ-ಸೆಟ್ಟಿ
ಕೌಳರು
ಕೌಶಿಕ
ಕೌಶಿಕ-ಕುಳಾಂಬರ
ಕೌಶಿಕ-ಗೋತ್ರ-ಅ-ಪವಿತ್ರನೂ
ಕೌಶಿಕ-ಗೋತ್ರದ
ಕೌಶಿಕ-ಗೋತ್ರ-ದ-ಳೈ-ಪಿಳ್ಳೈ-ಯಾರ್
ಕೌಶಿಕ-ಪುರಾಣ
ಕೌಶಿಕ-ಪುರಾಣದ
ಕೌಶಿಕಾನ್ವಯ-ಸಿಂಧು
ಕೌಸಲ್ಯ
ಕೌಸಲ್ಯಾ
ಕ್ಕಂದ
ಕ್ಕಾಗಿ
ಕ್ಕೂ
ಕ್ಕೆ
ಕ್ಕೋವಿದಂ
ಕ್ಯಾಗಟ್ಟದ
ಕ್ಯಾತನ-ಹಳ್ಳಿ
ಕ್ಯಾತನ-ಹಳ್ಳಿಯ
ಕ್ಯಾತನ-ಹಳ್ಳಿ-ಯಲ್ಲಿ
ಕ್ಯಾತನ-ಹಳ್ಳಿ-ಯಲ್ಲಿದ್ದ
ಕ್ಯಾತನ-ಹಳ್ಳಿ-ಯಲ್ಲೂ
ಕ್ಯಾತುಂಗೆರೆ
ಕ್ಯಾಸ್ಟ್್ಸ-ಕೃತಿ-ಯಲ್ಲಿ
ಕ್ರಮ
ಕ್ರಮಕ್ಕಿಂತಲೂ
ಕ್ರಮ-ಗಳು
ಕ್ರಮ-ಬದ್ಧ-ವಾಗಿ
ಕ್ರಮ-ವನೆ-ಸೆದು
ಕ್ರಮ-ವಾಗಿ
ಕ್ರಮ-ವೆಂತೆಂದಡೆ
ಕ್ರಮ-ವೆಂತೆಂದರೆ
ಕ್ರಮ-ವೆನ್ತೆನ್ದೊಡೆ
ಕ್ರಮಿತ
ಕ್ರಮೇಣ
ಕ್ರಮೇಣ-ವಾಗಿ
ಕ್ರಯ
ಕ್ರಯಕ್ಕೆ
ಕ್ರಯದ
ಕ್ರಯ-ದಾನ
ಕ್ರಯ-ದಾನ-ವಾಗಿ
ಕ್ರಯದ್ರಬ್ಯಮಂ
ಕ್ರಯದ್ರವ್ಯ
ಕ್ರಯದ್ರವ್ಯ-ವಾಗಿ
ಕ್ರಯ-ಪತ್ರ
ಕ್ರಯ-ಪತ್ರದ
ಕ್ರಯ-ಪತ್ರ-ವನ್ನು
ಕ್ರಯ-ಮಾಡಿ-ಕೊಡುತ್ತಾರೆ
ಕ್ರಯ-ಲಕ್ಷ-ಣ-ಲಕ್ಷಿತ
ಕ್ರಯ-ವಾಗಿ
ಕ್ರಯ-ವಾಗಿ-ಕೊಂಡು-ಅ-ದನ್ನು
ಕ್ರಯ-ವಿಕ್ರಯ-ಗಳು
ಕ್ರಯ-ಶಾ-ಸನದ
ಕ್ರಯ-ಶಾ-ಸನ-ವನ್ನು
ಕ್ರಯ-ಶಾ-ಸನ-ವಾಗಿ
ಕ್ರಯ-ಸಾ-ಧನ-ವಾಗಿ
ಕ್ರಯ-ಹುಲು-ವೀಸ
ಕ್ರಾಂತಿಯ
ಕ್ರಾಂತಿಯು
ಕ್ರಾಸ್ಗೆ
ಕ್ರಿ
ಕ್ರಿಪೂ
ಕ್ರಿಪೂ-ದಲ್ಲಿಯೇ
ಕ್ರಿಪೂನೇ
ಕ್ರಿಪೂ-ರಲ್ಲಿ
ಕ್ರಿಯಾ
ಕ್ರಿಯಾ-ಪ-ದವೇ
ಕ್ರಿಯಾ-ಶಕ್ತಿ
ಕ್ರಿಯಾ-ಶಕ್ತಿಯ
ಕ್ರಿಯೆ-ಯನ್ನು
ಕ್ರಿಯೆ-ಯಲ್ಲಿ
ಕ್ರಿಯೆಯೇ
ಕ್ರಿರ
ಕ್ರಿಶ
ಕ್ರಿಶಕ್ಕಿಂತ
ಕ್ರಿಶಕ್ಕೂ
ಕ್ರಿಶಕ್ಕೆ
ಕ್ರಿಶ-ನೆಯ
ಕ್ರಿಶನೇ
ಕ್ರಿಶರ
ಕ್ರಿಶ-ರದ್ದೇ
ಕ್ರಿಶ-ರ-ನಂತರ
ಕ್ರಿಶ-ರ-ರಲ್ಲಿ
ಕ್ರಿಶ-ರಲ್ಲಿ
ಕ್ರಿಶ-ರಲ್ಲಿ-ಮಾದ-ರ-ಸನು
ಕ್ರಿಶ-ರಲ್ಲಿಯೂ
ಕ್ರಿಶ-ರಲ್ಲೂ
ಕ್ರಿಶ-ರಲ್ಲೇ
ಕ್ರಿಶ-ರ-ವರೆ-ಗಿನ
ಕ್ರಿಶ-ರ-ವರೆಗೂ
ಕ್ರಿಶ-ರ-ವರೆಗೆ
ಕ್ರಿಶ-ರಿಂದ
ಕ್ರಿಶ-ರಿಂದಲೇ
ಕ್ರಿಶ-ಸು-ಮಾರು
ಕ್ರಿಶ-ಹಿ-ಜರಿ
ಕ್ರಿಶ್ಚಿಯನ್
ಕ್ರಿಸ್ತ-ಪೂರ್ವದ
ಕ್ರಿಸ್ತಶಕ-ಪೂರ್ವ-ದಲ್ಲಿ
ಕ್ರೂರಪ್ರಾಣಿ-ಗ-ಳನ್ನು
ಕ್ರೋಧಾದಿ
ಕ್ಲಪ್ತವಿಷ್ಣ್ವೀಶಪೂಜಃ
ಕ್ವಚಿತ್ತಾಗಿ
ಕ್ವ್ಮಾಪತೇಃ
ಕ್ಷಣಿಕ-ವಾದುವು
ಕ್ಷತಮಱೆವಾತಂ
ಕ್ಷತ್ರ-ಚೂಡಾ-ಮಣಿ
ಕ್ಷತ್ರಿ
ಕ್ಷತ್ರಿಯ
ಕ್ಷತ್ರಿ-ಯ-ನಾದ
ಕ್ಷತ್ರಿ-ಯ-ರದ್ದು
ಕ್ಷತ್ರಿ-ಯ-ರಾದ
ಕ್ಷತ್ರಿ-ಯರು
ಕ್ಷತ್ರಿ-ಲಾಡರಿ
ಕ್ಷಮಾಧೀಶ
ಕ್ಷಮಾಧೀ-ಶತಾ
ಕ್ಷಮಿಸಿ
ಕ್ಷಮೆಯಂ
ಕ್ಷಾತ್ರ-ತೇಜಸ್ಸಿ-ನಿಂದ
ಕ್ಷಾಮ
ಕ್ಷಾಮಕ್ಷೋಭೆ
ಕ್ಷಾಮ-ಡಾಮ-ರಕೆ
ಕ್ಷಾಮ-ಢಾ-ಮರ
ಕ್ಷಾಮ-ಢಾ-ಮರ-ಗಳು
ಕ್ಷಾಮ-ಢಾ-ಮರದ
ಕ್ಷಾಮ-ವಾಗು
ಕ್ಷಾಮಿಸ್ಫು-ರನ್
ಕ್ಷಿತಿ-ಧರ್ತೇ-ನರ-ಪತೇರ್ಗು-ರವೇ
ಕ್ಷಿತಿ-ನಾಥ
ಕ್ಷಿತಿ-ಪಾಲ-ಕನು
ಕ್ಷಿತಿ-ಪಾಲ-ಮೌಳಿರ್ವ-ದಾನ್ಯ-ಮೂರ್ತಿಃ
ಕ್ಷಿತಿ-ಯೊಳ್ಗಭೀರಮಂ
ಕ್ಷಿತಿ-ವಿ-ನುತೆ
ಕ್ಷಿತೀಂದ್ರ
ಕ್ಷಿತೀಂದ್ರನ
ಕ್ಷಿತೀಂದ್ರನು
ಕ್ಷಿತೀಂದ್ರ-ನೆಂಬ
ಕ್ಷಿತೀಶ್ವರನು
ಕ್ಷಿಪ್ರ-ವಾಗಿ
ಕ್ಷೀರ-ಧಾರೆಯ
ಕ್ಷುದ್ರ
ಕ್ಷುದ್ರ-ಶೈವಾ-ಲಿನಿ
ಕ್ಷೇತ್ರ
ಕ್ಷೇತ್ರಂ
ಕ್ಷೇತ್ರಂಗಳಾಯ
ಕ್ಷೇತ್ರಂಗಳಾಯಂಗ-ಳಿಂದಲೂ
ಕ್ಷೇತ್ರ-ಕಾರ್ಯ
ಕ್ಷೇತ್ರ-ಕಾರ್ಯದ
ಕ್ಷೇತ್ರ-ಕಾರ್ಯ-ದಿಂದ
ಕ್ಷೇತ್ರಕ್ಕೆ
ಕ್ಷೇತ್ರ-ಗಳ
ಕ್ಷೇತ್ರ-ಗ-ಳನ್ನು
ಕ್ಷೇತ್ರ-ಗಳ-ಭೂಮಿಯ
ಕ್ಷೇತ್ರ-ಗ-ಳಾದ
ಕ್ಷೇತ್ರ-ಗಳು
ಕ್ಷೇತ್ರ-ಗಳು-ಮತ್ತು
ಕ್ಷೇತ್ರದ
ಕ್ಷೇತ್ರ-ದತ್ತ
ಕ್ಷೇತ್ರ-ದಲ್ಲಿ
ಕ್ಷೇತ್ರ-ದಲ್ಲಿ-ರುವ
ಕ್ಷೇತ್ರ-ದಲ್ಲೇ
ಕ್ಷೇತ್ರ-ದ-ವ-ನಲ್ಲ
ಕ್ಷೇತ್ರ-ದ-ವ-ನಲ್ಲ-ವಾದುದ-ರಿಂದ
ಕ್ಷೇತ್ರ-ಪಾಲ
ಕ್ಷೇತ್ರ-ಪಾಲಯ್ಯ
ಕ್ಷೇತ್ರ-ರಾಜ-ವೆನಿಸಿದ
ಕ್ಷೇತ್ರವ
ಕ್ಷೇತ್ರ-ವನ್ನು
ಕ್ಷೇತ್ರ-ವಾಗಿ
ಕ್ಷೇತ್ರ-ವಾ-ಗಿತ್ತು
ಕ್ಷೇತ್ರ-ವಾಗಿದೆ
ಕ್ಷೇತ್ರ-ವಾದ
ಕ್ಷೇತ್ರ-ವಾದರೂ
ಕ್ಷೇತ್ರ-ವಾಸಿ-ಗಳಪ್ಪ
ಕ್ಷೇತ್ರ-ವಿದೆ
ಕ್ಷೇತ್ರವು
ಕ್ಷೇತ್ರವೂ
ಕ್ಷೇತ್ರ-ವೆಂದು
ಕ್ಷೇತ್ರವೇ
ಕ್ಷೇತ್ರಾಯ
ಕ್ಷೋಣೀ-ವಧೂ-ಭೂಷಣೇ
ಕ್ಷೋಭೆ
ಕ್ಷೋಭೆ-ಗ-ಳನ್ನು
ಕ್ಷೋಭೆ-ಯನ್ನು
ಕ್ಷ್ಮಾಯಾಂರಾಜ್ಯ
ಕ್ಷ್ಮೀಕಾಂತ
ಕ್ಷ್ಮೀವರ-ನಾಗಿ
ಕೞಅ್ಬಪ್ಪು
ಕೞಅ್ಬಹು-ಕೞ್ಬಾಹು-ಕಬ್ಬಾಹು
ಕೞನಿ
ಕೞ್ಬಪು-ಗಿರಿ
ಕೞ್ಬಪ್ಪು
ಕೞ್ಬಪ್ಪು-ಗಿರಿ
ಕೞ್ಬಪ್ಪು-ತೀರ್ತ್ತ
ಖ
ಖಂ
ಖಂಡಸ್ಫುಟಿತ
ಖಂಡಿಕ
ಖಂಡಿ-ಕದ
ಖಂಡಿಸಿ
ಖಂಡಿಸಿದ
ಖಂಡೀಕ
ಖಂಡೀ-ಕದ
ಖಂಡುಗ
ಖಂಡುಗಕ್ಕೆ
ಖಂಡುಗ-ಗದ್ದೆ-ಯನ್ನು
ಖಂಡುಗ-ವನ್ನು
ಖಂಡೆಯ-ರಾಯ
ಖಂತಿ-ಕಾರ
ಖಗ-ರಾಜ-ನಿನೇ-ಕದೊಡನಿಂ
ಖಚಿತ
ಖಚಿತ-ಗುಂಪೇ
ಖಚಿತ-ಪಡಿ-ಸಿದ್ದಾರೆ
ಖಚಿತ-ಪಡಿ-ಸುತ್ತದೆ
ಖಚಿತ-ಪಡಿ-ಸುತ್ತವೆ
ಖಚಿತ-ಪಡಿ-ಸುವ
ಖಚಿತ-ಪಡಿ-ಸು-ವಲ್ಲಿ
ಖಚಿತ-ಪಡುತ್ತದೆ
ಖಚಿತ-ವಾಗಿ
ಖಚಿತ-ವಾಗಿಯೇ
ಖಚಿತ-ವಾಗುತ್ತದೆ
ಖಚಿತ-ವಾದ
ಖಚಿತ-ವಿಲ್ಲ
ಖಜಾನೆ
ಖಜಾ-ನೆಗೆ
ಖಜಾ-ನೆಯ
ಖಜಾನೆ-ಯಲ್ಲಿ
ಖಡಾಯ
ಖಡಿಲೆ-ಗೊಂಡು
ಖಡ್ಡಾಯ
ಖಣ್ಡನ
ಖಣ್ಡಿತ
ಖಣ್ಡೋಘನಃ
ಖದೀರ್
ಖಬರ
ಖಬರಸ್ಥಾನದ
ಖಬರಸ್ಥಾನ-ದಲ್ಲಿ-ರುವ
ಖರ
ಖರದೂಷಣ-ರನ್ನು
ಖರೀದಿ
ಖರೀದಿಸಿ
ಖರೀದಿ-ಸಿದ
ಖರೀದಿ-ಸುತ್ತಾರೆ
ಖರೀದಿ-ಸುತ್ತಿದೆ
ಖರ್ಚು
ಖರ್ಚು-ಗ-ಳಿಗೆ
ಖರ್ಚು-ಮಾಡಿ
ಖರ್ಚು-ಮಾಡಿದ
ಖರ್ಚು-ವೆಚ್ಚ-ಗಳ
ಖಲೀಫ-ನಾದ
ಖಳತ್ರಿಣೇತ್ರ
ಖಳ್ಗದಿಂ
ಖಾಣ
ಖಾತಿ-ಧರಿತಿತೆ
ಖಾದ್ರಿ
ಖಾನೆ
ಖಾನೆ-ಯೆಂದು
ಖಾನ್
ಖಾಯಂ
ಖಾಯಿಲೆ
ಖಾಲಿ
ಖಾಸ
ಖಾಸಗಿ
ಖಾಸ-ಬೊಕ್ಕ-ಸದ
ಖಾಸಾ
ಖಾಸಾ-ಬೊಕ್ಕ-ಸದ
ಖಾಸ್
ಖಿಡಿಜ್ಚಿಣ್ಡಿಜ್ಡಿ
ಖಿಲಜಿಯ
ಖಿಲ-ವಾಗಿ
ಖಿಲ-ವಾಗಿತ್ತೆಂದೂ
ಖಿಲ-ವಾ-ಗಿದ್ದ
ಖಿಲ-ವಾಗಿ-ರಲು
ಖಿಲ-ವಾಗಿ-ರುವ
ಖುದಾದಾದ್
ಖುದ್ದಾಗಿ
ಖುರಾನಿನ
ಖೆಡಿ-ಲೆಯ
ಖೈಬ-ರದ
ಖೈಬರ್
ಖೊಟ್ಟಿ-ಗನು
ಖ್ಯಾತ
ಖ್ಯಾತ-ಕರ್ನಾಟ-ಕದ
ಖ್ಯಾತ-ನಾಗಿದ್ದನು
ಖ್ಯಾತ-ನಾ-ಗಿದ್ದು
ಖ್ಯಾತ-ನಾದ-ನೆಂದು
ಖ್ಯಾತ-ಳಾಗಿದ್ದ-ಳೆಂದು
ಖ್ಯಾತಸ್ಯಾನಸ್ಯ
ಖ್ಯಾತಿ
ಖ್ಯಾತಿ-ಮುಪೇಯುಷೇ
ಖ್ಯಾತಿಯ
ಖ್ಯಾತಿ-ವೆತ್ತ
ಖ್ಯಾತೆ-ಯಾಗ
ಖ್ಯಾತೆ-ಯಾದ
ಖ್ವಾಜಾ
ಗ
ಗಂ
ಗಂಗ
ಗಂಗಂಣನು
ಗಂಗ-ಕುಳ-ಚಂದ್ರಂ
ಗಂಗ-ಗಾ-ಮುಂಡ
ಗಂಗ-ಚಮೂಪಂ
ಗಂಗ-ಡಿ-ಕಾರ
ಗಂಗ-ಡಿ-ಕಾರರ
ಗಂಗ-ಡಿ-ಕಾರ-ರಲ್ಲಿ
ಗಂಗ-ಡಿ-ಕಾರರು
ಗಂಗ-ಡಿ-ಕಾರರೇ
ಗಂಗಣ
ಗಂಗಣ್ಣ
ಗಂಗಣ್ಣನ
ಗಂಗಣ್ಣನು
ಗಂಗಣ್ನನ
ಗಂಗ-ದಂಡಾಧೀಶ
ಗಂಗ-ದಂಡಾಧೀಶನ
ಗಂಗ-ದಂಡಾಧೀಶ-ನನ್ನು
ಗಂಗ-ದಂಡಾಧೀಶನು
ಗಂಗ-ದಂಡೇಶ
ಗಂಗ-ದೇಶಾಧಿಪ
ಗಂಗ-ದೊರೆಯ
ಗಂಗ-ನನ್ನು
ಗಂಗ-ನ-ಹಳ್ಳಿ
ಗಂಗ-ನ-ಹಳ್ಳಿ-ಯಲ್ಲಿ-ರುವ
ಗಂಗ-ನಾ-ರಾಯಣ
ಗಂಗ-ನೃಪನು
ಗಂಗ-ನೊಳಂಬರ
ಗಂಗ-ಪಯ್ಯನ
ಗಂಗ-ಪಲ್ಲ-ವರ
ಗಂಗ-ಪೆರ್ಮಾ-ನಡಿ
ಗಂಗ-ಪೆರ್ಮಾನ-ಡಿ-ಯನ್ನು
ಗಂಗ-ಪೆರ್ಮಾನ-ಡಿಯು
ಗಂಗ-ಪೆರ್ಮ್ಮಾನ-ಡಿಯು
ಗಂಗಪ್ಪಯ್ಯ
ಗಂಗಪ್ರವಾ-ಹೋದಾರ
ಗಂಗ-ಮಂಡಲ
ಗಂಗ-ಮಂಡ-ಲ-ಗ-ಳಿಗೆ
ಗಂಗ-ಮಂಡ-ಲ-ವನ್ನು
ಗಂಗ-ಮಂಡ-ಲ-ವೆಂದು
ಗಂಗ-ಮಂಡ-ಲಾಧಿ-ಪತ್ಯ-ವನ್ನು
ಗಂಗ-ಮಂಡಳ
ಗಂಗ-ಮಂಡ-ಳ-ವನ್ನು
ಗಂಗ-ಮಂಡ-ಳೇಶ್ವರ
ಗಂಗ-ಮಹಾ-ದೇವಿ
ಗಂಗಯ್ಯ
ಗಂಗರ
ಗಂಗ-ರಅ
ಗಂಗ-ರ-ಕಾಲ
ಗಂಗ-ರ-ಕಾಲದ
ಗಂಗ-ರ-ಕಾಲ-ದಲ್ಲಿ
ಗಂಗ-ರ-ಮೇ-ಲಿನ
ಗಂಗ-ರಸ
ಗಂಗ-ರ-ಸ-ರಾಗಿ
ಗಂಗ-ರ-ಸ-ರಿಗೆ
ಗಂಗ-ರ-ಸರು
ಗಂಗ-ರಾಜ
ಗಂಗ-ರಾಜಂ
ಗಂಗ-ರಾಜ-ಕು-ಮಾರ
ಗಂಗ-ರಾಜನ
ಗಂಗ-ರಾಜ-ನ-ಕಾಲ-ದಲ್ಲಿಯೇ
ಗಂಗ-ರಾಜ-ನನ್ನು
ಗಂಗ-ರಾಜ-ನನ್ನೂ
ಗಂಗ-ರಾಜ-ನಾದ
ಗಂಗ-ರಾಜ-ನಿಂದ
ಗಂಗ-ರಾಜ-ನಿಗೆ
ಗಂಗ-ರಾಜ-ನಿರ-ಬ-ಹುದು
ಗಂಗ-ರಾಜ-ನಿರ-ಬಹು-ದೆಂಬ
ಗಂಗ-ರಾಜನು
ಗಂಗ-ರಾಜ-ನೆಂಬುದು
ಗಂಗ-ರಾಜ-ನೊಡನೆ
ಗಂಗ-ರಾಜನ್ನು
ಗಂಗ-ರಾಜರ
ಗಂಗ-ರಾಜ್ಯ
ಗಂಗ-ರಾಜ್ಯಕ್ಕೆ
ಗಂಗ-ರಾಜ್ಯ-ವನ್ನು
ಗಂಗ-ರಾಷ್ಟ್ರ-ಕೂಟರ
ಗಂಗ-ರಿಂದ
ಗಂಗ-ರಿಗೂ
ಗಂಗ-ರಿಗೆ
ಗಂಗರು
ಗಂಗ-ರೊಂದಿಗೆ
ಗಂಗ-ರೊಡನೆ
ಗಂಗ-ವಂಶದ
ಗಂಗ-ವಂಶ-ದ-ರೆೆಂದೂ
ಗಂಗ-ವಂಶ-ದ-ವ-ನೆಂದು
ಗಂಗ-ವಾಡಿ
ಗಂಗ-ವಾಡಿ-ಕಾರ
ಗಂಗ-ವಾಡಿಗೆ
ಗಂಗ-ವಾಡಿ-ತೊಂಭತ್ತ-ರು-ಸಾ-ಸಿರದ
ಗಂಗ-ವಾಡಿ-ನಾಮಂ
ಗಂಗ-ವಾಡಿಯ
ಗಂಗ-ವಾಡಿ-ಯನ್ನು
ಗಂಗ-ವಾಡಿ-ಯಲ್ಲಿ
ಗಂಗ-ವಾಡಿ-ಯಿಂದ
ಗಂಗ-ವಾಡಿಯು
ಗಂಗ-ವಾಡಿ-ಯೊಳಕ್ಕೆ
ಗಂಗ-ಸಂದ್ರ
ಗಂಗ-ಸ-ಮುದ್ರ
ಗಂಗ-ಸೇನಾ-ಪತಿಯ
ಗಂಗಾತ-ರಂಗೆ
ಗಂಗಾ-ದೇವಿ
ಗಂಗಾ-ಧರ
ಗಂಗಾ-ಧರ-ದೇವರ
ಗಂಗಾಧ-ರನ
ಗಂಗಾ-ಧರ-ನೆಂಬ
ಗಂಗಾ-ಧರ-ಪುರ
ಗಂಗಾ-ಧರ-ಪುರ-ವೆಂದು
ಗಂಗಾ-ಧರ-ಪುರ-ವೆಂಬ
ಗಂಗಾ-ಧರಯ್ಯನ
ಗಂಗಾ-ಧರಯ್ಯನು
ಗಂಗಾ-ಧರೇಶ್ವರ
ಗಂಗಾ-ಧರೇಶ್ವರನ
ಗಂಗಾ-ನದಿ
ಗಂಗಾ-ನ-ದಿಯ
ಗಂಗಾ-ನದೀ-ಸಂಗ-ಮ-ಸಹಚರ-ಮಾಸ್ಯೇಂದು
ಗಂಗಾನ್ವಯ
ಗಂಗಾನ್ವಯಕ್ಕೆ
ಗಂಗಾ-ವನಿ-ರಟ್ಟ
ಗಂಗಾ-ವನಿ-ರಟ್ಟ-ವಾಡಿ
ಗಂಗಾ-ಸ-ಮುದ್ರ
ಗಂಗಿ-ಗವುಂಡನ
ಗಂಗೆಯ
ಗಂಗೇ-ಗೌಡ
ಗಂಗೇಶ್ವರ
ಗಂಗೈ-ಕೊಂಡ
ಗಂಗೈ-ಕೊಂಡ-ಚೋಳ-ಪುರಕ್ಕೆ
ಗಂಜಾಂ
ಗಂಜಾಂನ
ಗಂಜಾಮ್
ಗಂಜಾಮ್ನಲ್ಲಿ-ರುವ
ಗಂಜಿ-ಗೆರೆಗೆ
ಗಂಜೀ
ಗಂಟು
ಗಂಟು-ಹಾಕಿ-ಕೊಂಡು
ಗಂಟೆ-ಯನ್ನು
ಗಂಡ
ಗಂಡಂದಿರು
ಗಂಡಃ
ಗಂಡ-ಗೂಳಿ
ಗಂಡನ
ಗಂಡ-ನಾದ
ಗಂಡ-ನಾ-ರಾಯಣ
ಗಂಡ-ನಾ-ರಾಯ-ಣರುಂ
ಗಂಡ-ನಾ-ರಾಯ-ಣ-ಸೆಟ್ಟಿ
ಗಂಡ-ನಾ-ರಾಯ-ಣ-ಸೆಟ್ಟಿಗೆ
ಗಂಡ-ನಾ-ರಾಯ-ಣ-ಸೆಟ್ಟಿಯ
ಗಂಡ-ನಾ-ರಾಯ-ಣ-ಸೆಟ್ಟಿ-ಯರು
ಗಂಡ-ನಾ-ರಾಯ-ನ-ಸೆಟ್ಟಿಯ
ಗಂಡನು
ಗಂಡ-ನೆಂದೂ
ಗಂಡ-ನೆನಿಸಿದ
ಗಂಡ-ಪೆಂಡಾರ
ಗಂಡ-ಪೆಂಡಾರ-ಗೊಂಡ-ನೆಂದು
ಗಂಡ-ಪೆಂಡಾರ-ಗೊಂಡಿದ್ದ-ನೆಂದು
ಗಂಡ-ಪೆಂಡಾರ-ವನ್ನು
ಗಂಡ-ಬೇರುಂಡ
ಗಂಡ-ಭೇರುಂಡ
ಗಂಡ-ಮಾರ್ತ್ತಾಂಡ
ಗಂಡರ
ಗಂಡ-ರ-ಗಂಡಂ
ಗಂಡ-ರ-ಗಂಡ-ಮುಂಡ
ಗಂಡ-ರ-ನಾಂಪೆ-ವೆಂದು
ಗಂಡ-ರಾಗಿದ್ದ-ರೆಂದು
ಗಂಡ-ರಾಜ
ಗಂಡರುಂ
ಗಂಡ-ವಿ-ಮುಕ್ತ
ಗಂಡ-ವಿ-ಮುಕ್ತ-ದೇವರು
ಗಂಡ-ಸೆಯ
ಗಂಡಾಂತರ-ದಿಂದ
ಗಂಡಾ-ಗುಂಡಿ-ಯಿಂದ
ಗಂಡಾನೆ
ಗಂಡು
ಗಂಡು-ಗಲಿ
ಗಂಡು-ಮಕ್ಕಳಾದ
ಗಂಡು-ಮಕ್ಕಳೂ
ಗಂಡೇನ-ಹಳ್ಳಿ
ಗಂದಕೆ
ಗಂದದ
ಗಂಧ
ಗಂಧಕೆ
ಗಂಧಕ್ಕೆ
ಗಂಧ-ಗೋಡಿ
ಗಂಧದ
ಗಂಧ-ನ-ಹಳ್ಳಿ
ಗಂಧ-ವನ್ನು
ಗಂಧ-ವಾರಣ
ಗಂಧ-ವಾರ-ಣ-ನು-ದಾರಂ
ಗಂಧ-ವಾರ-ಣ-ನೆಂದು
ಗಂಧ-ವಿ-ಮುಕ್ತ
ಗಂಧಿ-ಸೆಟ್ಟಿ
ಗಂಧಿ-ಸೆಟ್ಟಿಯ
ಗಂಭೀ-ರಪ್ಪ
ಗಂಭೀ-ರವೂ
ಗಉಂಡು-ಗಳು
ಗಉಡ-ನವ-ರಿಗೆ
ಗಉಡು
ಗಉಡು-ಕುಲ-ತಿಲಕರುಂ
ಗಉಡು-ಗ-ಳಿಗೆಊ
ಗಉಡು-ಗಳು
ಗಉತಯಗೆ
ಗಗದ್ಯಾಣ
ಗಗನ
ಗಗುಡಿಯ
ಗಚ್ಛ
ಗಚ್ಛಕ್ಕೆ
ಗಚ್ಛ-ಗಳು
ಗಚ್ಛದ
ಗಚ್ಛ-ವನ್ನು
ಗಛ-ಮದನ್ವಯಂ
ಗಜ
ಗಜಖೇಟಕ
ಗಜ-ಗಾರ
ಗಜ-ಗಾರ-ಕುಪ್ಪೆ
ಗಜ-ಗಾರ-ಕುಪ್ಪೆಯೇ
ಗಜ-ಗಾರ-ಗುಪ್ಪೆ
ಗಜ-ಗಾರ-ರೆಂದು
ಗಜ-ಪತಿ
ಗಜ-ಪತಿ-ಗಳು
ಗಜ-ಪತಿಯ
ಗಜ-ಪ-ತಿಯು
ಗಜ-ಬೇಂಟೆ-ಕಾರ
ಗಜಬೇಂಟೆಕಾಱ
ಗಜಮಸ್ತ-ಕದ
ಗಜಮಸ್ತ-ಕದ-ಮೇಲೇರಿ
ಗಜಮಸ್ತಕ-ದೊಳು
ಗಜ-ರಾಜ-ಗಿರಿ
ಗಜ-ರಾಜ-ಗಿರಿಯ
ಗಜವು
ಗಜವೈದ್ಯ-ವಿದ್ಯಾ
ಗಜ-ಸಿಂಹ
ಗಜ-ಸೇನೆ-ಯೊಂದಿಗೆ
ಗಜಾಂಕುಶ
ಗಜಾಖೇಟಕ-ಮತ್ಯುಗ್ರಂ
ಗಜಾರಣ್ಯ
ಗಜೇಂದ್ರ
ಗಜೇಂದ್ರ-ಮಂಟಪ-ವನ್ನು
ಗಜೇಂದ್ರೋತ್ಸವ-ವನ್ನೂ
ಗಜೌಘ
ಗಜೌಘ-ಗಂಡ-ಭೇರುಂಡೋ
ಗಟಿಕಾಸ್ತಾ-ನದ
ಗಟಿಕಾಸ್ಥಾನ
ಗಟ್ಟಿ-ಗೊಳ್ಳ-ಬೇಕಾಗಿದೆ
ಗಟ್ಟಿ-ಯಾದ
ಗಟ್ಟೇಶ್ವರ
ಗಡ
ಗಡದ
ಗಡದ-ಗಡ್ಡದ
ಗಡಿ
ಗಡಿ-ಕಾಳ-ಗಕ್ಕೆ
ಗಡಿ-ಗಂಬದ
ಗಡಿ-ಗ-ಳನ್ನು
ಗಡಿ-ಗ-ಳಲ್ಲಿ
ಗಡಿಗೆ
ಗಡಿಗೇ
ಗಡಿದ
ಗಡಿ-ದ-ತಿರು-ಮಲಾ-ಪುರ
ಗಡಿ-ಭಾಗ-ದಲ್ಲಿವೆ
ಗಡಿ-ಮೂ-ಡಲು
ಗಡಿಯ
ಗಡಿ-ಯಲ್ಲಿ
ಗಡಿ-ಯಲ್ಲಿದೆ
ಗಡಿ-ಯಲ್ಲಿದ್ದ
ಗಡಿ-ಯಲ್ಲಿ-ರುವ
ಗಡಿ-ಯಲ್ಲೇ
ಗಡಿ-ಯಾ-ಗಿದ್ದ
ಗಡಿ-ಯಾಗಿದ್ದಿರ-ಬ-ಹುದು
ಗಡಿ-ಯಿಂದ
ಗಡಿಯು
ಗಡಿ-ವಿ-ವಾದ
ಗಡಿ-ಸದ
ಗಡ್ಡದ
ಗಡ್ಡದ-ದಾ-ಡಿಯ
ಗಡ್ಡದ-ದಾ-ಡಿಯ-ಸೋಮೆಯ-ದಂಡ-ನಾಯಕ
ಗಣ
ಗಣಂಗಳ
ಗಣಂಗ-ಳಲ್ಲಿ
ಗಣಂಗ-ಳಿಂದ
ಗಣಂಗಳು
ಗಣ-ಕ-ಕುಲ
ಗಣ-ಕರು
ಗಣ-ಕು-ಮಾರ
ಗಣ-ಕು-ಮಾರ-ರಾದ
ಗಣ-ಗಚ್ಛ-ಗ-ಳನ್ನು
ಗಣ-ಗಳ
ಗಣ-ಗಳು
ಗಣದ
ಗಣ-ದಲ್ಲಿ
ಗಣ-ದ-ವರೇ
ಗಣ-ದೊಳು
ಗಣ-ಪತಿ
ಗಣ-ಪತಿಕ್ರಮಿ-ತರ
ಗಣ-ಪತಿ-ಯರ
ಗಣವು
ಗಣ-ಹಳ್ಳಿ
ಗಣ-ಹಳ್ಳಿ-ಯನ್ನು
ಗಣಾ-ಚಾರ
ಗಣಾ-ಚಾರಿ
ಗಣಾ-ಚಾರಿಕೆ
ಗಣಾ-ಚಾರಿ-ಕೆ-ಯನ್ನೂ
ಗಣಾ-ಚಾರ್ಯ
ಗಣಾಧಿ-ಪತಯೇ
ಗಣಾರಾ-ಧನೆ-ಯನ್ನು
ಗಣಿಕೆ-ಯ-ರಿದ್ದಂತೆ
ಗಣಿ-ಗ-ನೂರ
ಗಣೇಶ್
ಗಣ್ಟಮ್ಮನ
ಗಣ್ಡಪೆಣ್ಡಾರ
ಗಣ್ದಪೆಣ್ಡಾರ
ಗಣ್ಯ-ದೊರೆ
ಗತಿಸಿದ
ಗತಿಸಿ-ದ-ನೆಂದು
ಗತಿಸಿ-ರ-ಬ-ಹುದು
ಗತಿಸಿ-ರ-ಬಹು-ದೆಂಬುದು
ಗದಗ
ಗದಗು
ಗದಾ-ಧಾರಿ-ಯಾಗಿದ್ದಾನೆ
ಗದಿರದೆ
ಗದು-ಗಿನ
ಗದೆ
ಗದೆಯ
ಗದೆ-ಯನು
ಗದೆ-ಯನ್ನು
ಗದೆ-ಯಿಂದಲೂ
ಗದ್ದುಗೆ
ಗದ್ದೆ
ಗದ್ದೆ-ಗಳ
ಗದ್ದೆ-ಗ-ಳನ್ನು
ಗದ್ದೆ-ಗ-ಳನ್ನೂ
ಗದ್ದೆ-ಗ-ಳಲ್ಲಿ
ಗದ್ದೆ-ಗ-ಳಾಗಿವೆ
ಗದ್ದೆ-ಗಳಿಗೂ
ಗದ್ದೆ-ಗ-ಳಿಗೆ
ಗದ್ದೆ-ಗಳಿ-ರುತ್ತಿದ್ದವು
ಗದ್ದೆ-ಗಳು
ಗದ್ದೆಗೆ
ಗದ್ದೆ-ಗೆವೂ
ಗದ್ದೆ-ಬೆದ್ದ-ಲನ್ನು
ಗದ್ದೆ-ಬೆದ್ದಲು
ಗದ್ದೆ-ಬೆದ್ದ-ಲು-ಗ-ಳನ್ನು
ಗದ್ದೆ-ಭೂಮಿ
ಗದ್ದೆಯ
ಗದ್ದೆಯಂ
ಗದ್ದೆ-ಯ-ನನು
ಗದ್ದೆ-ಯನು
ಗದ್ದೆ-ಯನೂ
ಗದ್ದೆ-ಯನ್ನು
ಗದ್ದೆ-ಯನ್ನೂ
ಗದ್ದೆ-ಯಲ್ಲಿ
ಗದ್ದೆ-ಯಲ್ಲಿದ್ದು
ಗದ್ದೆ-ಯಲ್ಲಿ-ರುವ
ಗದ್ದೆ-ಯಾಗಿ
ಗದ್ದೆ-ಯಾ-ಗಿದ್ದು
ಗದ್ದೆ-ಯಾಗಿ-ರ-ಬ-ಹುದು
ಗದ್ದೆ-ಯಿಂದ
ಗದ್ದೆಯೂ
ಗದ್ದೆ-ಯೊಳಗೆ
ಗದ್ದೆಯ್ನನು
ಗದ್ಯ
ಗದ್ಯವೂ
ಗದ್ಯಾಣ
ಗದ್ಯಾಣಂ
ಗದ್ಯಾಣಕ್ಕೂ
ಗದ್ಯಾಣಕ್ಕೆ
ಗದ್ಯಾಣ-ಗ-ಳನ್ನು
ಗದ್ಯಾಣ-ಗಳಷ್ಟು
ಗದ್ಯಾಣ-ಗ-ಳಿಗೆ
ಗದ್ಯಾಣದ
ಗದ್ಯಾಣಮ್
ಗದ್ಯಾಣವ
ಗದ್ಯಾಣ-ವನ್ನು
ಗದ್ಯಾಣ-ವನ್ನೂ
ಗದ್ಯಾಣ-ವ-ರಹ
ಗದ್ಯಾಣ-ವರ-ಹ-ಗ-ಳನ್ನು
ಗದ್ಯಾಣ-ವ-ಱು-ವತ್ತನಾಲ್ಕಂ
ಗದ್ಯಾಣ-ವಾಗಿ-ರ-ಬ-ಹುದು
ಗದ್ಯಾಣ-ವಿಪ್ಪತ್ತೆಂಟು
ಗದ್ಯಾಣ-ವೆ-ರಡು
ಗದ್ಯಾಣ-ವೊಂದು
ಗನೂ
ಗನ್ನು
ಗಬಯಗಿದ
ಗಮನ
ಗಮನಕ್ಕೆ
ಗಮನ-ದಲ್ಲಿಟ್ಟು-ಕೊಂಡು
ಗಮನ-ವನ್ನು
ಗಮನ-ಸೆ-ಳೆಯುತ್ತದೆ
ಗಮನ-ಹರಿ-ದಿಲ್ಲ-ವೆಂದು
ಗಮ-ನಾರ್ಹ
ಗಮ-ನಾರ್ಹ-ವಾಗಿದೆ
ಗಮ-ನಾರ್ಹ-ವಾದ
ಗಮನಿ-ಸ-ತಕ್ಕದ್ದಾಗಿದೆ
ಗಮನಿ-ಸ-ತಕ್ಕದ್ದಾಗಿವೆ
ಗಮನಿ-ಸ-ಬಹದು
ಗಮನಿ-ಸ-ಬ-ಹುದು
ಗಮನಿ-ಸಬೇಕಾಗುತ್ತದೆ
ಗಮನಿ-ಸ-ಬೇ-ಕಾದ
ಗಮನಿ-ಸ-ಬೇ-ಕಾ-ದುದು
ಗಮನಿ-ಸ-ಬೇಕು
ಗಮನಿ-ಸ-ಲಾಗಿದೆ
ಗಮನಿಸಿ
ಗಮನಿ-ಸಿ-ದರೆ
ಗಮನಿ-ಸಿ-ದಾಗ
ಗಮನಿ-ಸಿ-ಬಹು
ಗಮನಿ-ಸಿ-ಬ-ಹುದು
ಗರ-ಡಿ-ಮನೆ
ಗರಡು-ನಲುಗು
ಗರುಜೆ
ಗರುಡ
ಗರುಡಂ
ಗರು-ಡ-ಗಂಬ-ಗಳು
ಗರು-ಡ-ಗಂಬದ
ಗರು-ಡ-ಗಂಬದೀಪಸ್ಥಂಭ-ವಿದೆ
ಗರು-ಡ-ಗಂಬ-ವನ್ನು
ಗರು-ಡ-ಗಂಬ-ವನ್ನು-ದೀಪ-ಮಾಲೆ
ಗರು-ಡ-ಗಲ್ಲು-ಗಳು-ಎಂದು
ಗರು-ಡಧ್ವಜ-ವನ್ನು
ಗರು-ಡನ
ಗರು-ಡ-ನಂತೆ
ಗರು-ಡ-ನ-ನಪ್ಪಿ
ಗರು-ಡ-ನ-ನಪ್ಪಿದ
ಗರು-ಡ-ನ-ನಪ್ಪಿದಂ
ಗರು-ಡ-ನನ್ನು
ಗರು-ಡ-ನ-ಲೆಯ-ದಂಬಾರಿ-ಯೊಗ್ಳಪಿ
ಗರು-ಡ-ನ-ಲೆಯ-ದ-ನೆಯ
ಗರು-ಡ-ನ-ಹಳ್ಳಿ
ಗರು-ಡ-ನ-ಹಳ್ಳಿ-ಯನ್ನು
ಗರು-ಡ-ನಾಗಿದ್ದಿರ-ಬ-ಹುದು
ಗರು-ಡ-ನಾದ
ಗರು-ಡ-ನಾ-ರಾಯ-ಣರುಂ
ಗರು-ಡ-ನಿಗೂ
ಗರು-ಡ-ನಿಗೆ
ಗರು-ಡನು
ಗರು-ಡನೆ
ಗರು-ಡ-ನೆ-ಸೆವಾ-ಕೃತಿ-ಗಳ್
ಗರು-ಡ-ನೊಡನೆ
ಗರು-ಡ-ಪೀಠ-ವನ್ನು
ಗರು-ಡರ
ಗರು-ಡ-ರಾಗಿ
ಗರು-ಡ-ರಾಗಿದ್ದ-ವರು
ಗರು-ಡ-ರಾದ
ಗರು-ಡ-ರಿಗೆ
ಗರು-ಡರು
ಗರು-ಡರೂ
ಗರು-ಡ-ಲೆಂಕ-ರಾಗಿ
ಗರು-ಡ-ವಾ-ಹನ-ವನ್ನು
ಗರು-ಡ-ಶಾ-ಸನದ
ಗರು-ಡೇಶ್ವರ
ಗರು-ಡೇಶ್ವರ-ನೆಂದು
ಗರು-ದಂಗೆ
ಗರ್ಗ-ಗೋತ್ರದ
ಗರ್ಗೇಶ್ವರ
ಗರ್ಜನೆಗಾದಮಳ್ಕಿ
ಗರ್ಬ್ಭ-ಸರ್ಬ್ಬಸ್ವಾಪ-ಹಾರ
ಗರ್ಭಗಹ-ದಲ್ಲಿ
ಗರ್ಭ-ಗುಡಿ
ಗರ್ಭ-ಗುಡಿ-ಗ-ಳಲ್ಲಿ
ಗರ್ಭ-ಗುಡಿಯ
ಗರ್ಭ-ಗುಡಿ-ಯಲ್ಲಿ
ಗರ್ಭ-ಗೃಹ
ಗರ್ಭ-ಗೃಹ-ಗ-ಳನ್ನು
ಗರ್ಭ-ಗೃಹ-ಗಳು
ಗರ್ಭ-ಗೃಹದ
ಗರ್ಭ-ಗೃಹ-ದಲ್ಲಿ
ಗರ್ಭ-ಗೃಹದ್ವಾರ-ವನ್ನು
ಗರ್ವನರ್
ಗರ್ವೋದ್ಧತ-ನಾದ
ಗಳ
ಗಳ-ಗರ್ಜ್ಜನೆಗಾದಮಳ್ಕಿ
ಗಳನ್ನು
ಗಳ-ಹನ
ಗಳಾಗಿ
ಗಳಾಗಿದ್ದ
ಗಳಾಗಿದ್ದರು
ಗಳಾಗಿ-ರ-ಬ-ಹುದು
ಗಳಾಗಿ-ರ-ಲಿಲ್ಲ
ಗಳಾದ
ಗಳಿಗೆ
ಗಳಿದ್ದರೂ
ಗಳಿದ್ದವು
ಗಳಿಸಿ-ಕೊಂಡಿದ್ದ
ಗಳಿಸಿ-ಕೊಂಡು
ಗಳಿಸಿ-ಕೊಟ್ಟಿದ್ದಕ್ಕಾಗಿಯೇ
ಗಳಿಸಿದ
ಗಳಿ-ಸಿ-ದಂತೆ
ಗಳಿಸಿ-ದ-ವ-ನಾಗಿ-ರ-ಬೇಕು
ಗಳಿಸಿದ್ದನು
ಗಳಿ-ಸಿದ್ದು
ಗಳಿ-ಸುತ್ತಿದ್ದೆ
ಗಳಿ-ಸುವುದ-ರಿಂದ
ಗಳು
ಗಳೆಂದು
ಗವರಾ-ಚಾರ್ಯನ
ಗವರಾ-ಚಾರ್ಯ-ನೆಂಬ
ಗವರೀಶ್ವರ
ಗವರೆ
ಗವರೆ-ಕುಲ-ತಿಲಕ
ಗವರೆ-ಗ-ಳಾಗಿದ್ದು
ಗವರೆ-ಗಳು
ಗವರೆ-ಗೌರಿ-ಕೊಳ್ಳದ
ಗವರೆ-ಸೆಟ್ಟಿ
ಗವರೆ-ಸೆಟ್ಟಿಯ
ಗವರೆ-ಸೆಟ್ಟಿ-ಯನ್ನು
ಗವರೆ-ಸೆಟ್ಟಿಯು
ಗವರೇಶ್ವರ
ಗವರೇಶ್ವರ-ಗೌರೇಶ್ವರ
ಗವರೇಶ್ವರ-ದೇವಾ-ಲಯ
ಗವರೇಶ್ವರ-ನೆಂದು
ಗವರೈ
ಗವಱಾ-ಚಾರ್ಯನ
ಗವಿ-ಮಠ
ಗವಿ-ಮುಂಡೂರಿನ
ಗವಿ-ರಂಗಾ-ಪುರ-ದಂತಹ
ಗವುಂಡನ
ಗವುಡ
ಗವು-ಡ-ಕುಲ
ಗವು-ಡ-ಗೆರೆ
ಗವು-ಡ-ಗೆರೆ-ಯಲ್ಲಿ
ಗವು-ಡ-ನಾ-ಗಿದ್ದ
ಗವು-ಡ-ನೊಡ-ಹುಟ್ಟಿದ
ಗವು-ಡರ
ಗವು-ಡ-ರನ್ನು
ಗವು-ಡ-ರಿಗೆ
ಗವು-ಡರು
ಗವು-ಡ-ರು-ಗಳು
ಗವುಡಿ
ಗವು-ಡಿ-ಕೆಗೆ
ಗವು-ಡಿ-ಕೆ-ಯನ್ನು
ಗವು-ಡಿ-ಗೆರೆ
ಗವು-ಡಿ-ಗೆರೆಯ
ಗವು-ಡಿ-ಗೆರೆಯು
ಗವು-ಡಿ-ತಂಮಂಗೆ
ಗವು-ಡಿ-ತಮ್ಮನ
ಗವುಡು
ಗವು-ಡು-ಗಳ
ಗವು-ಡು-ಗ-ಳಾದ
ಗವು-ಡು-ಗ-ಳಾದ-ಗಾವುಂಡರು
ಗವು-ಡು-ಗ-ಳಿಗೆ
ಗವು-ಡು-ಗಳು
ಗವು-ಡು-ಗಳೆಂದರೆ
ಗವು-ಡು-ಗೆರೆಯ
ಗವು-ಡು-ಗೆರೆ-ಯಲ್ಲಿ
ಗವು-ಡುಪ್ರಜೆ-ಗಳ
ಗವು-ಡುಪ್ರಜೆ-ಗ-ಳನ್ನು
ಗವು-ಡುಪ್ರಜೆ-ಗಳು
ಗವುಣ್ಡ
ಗವುದು
ಗಾಂ
ಗಾಂಗೇಯ
ಗಾಂಚ-ನೂ-ರನ್ನು
ಗಾಂಧರ್ವ-ವೇದ
ಗಾಂಧರ್ವ್ವಕೇಷು
ಗಾಂನ್ಧ-ವರಾನೆ
ಗಾಜ-ನೂರು
ಗಾಜ-ನೂರೇ
ಗಾಞ್ಚ-ನೂ-ರನ್ನು
ಗಾಡಿಗ-ಳ-ಮೇಲೆ
ಗಾಡಿಗೆ
ಗಾಡಿ-ಮೇ-ಲಿನ
ಗಾಣ
ಗಾಣಂ
ಗಾಣ-ಗರು
ಗಾಣ-ಗಳ
ಗಾಣ-ಗ-ಳನ್ನು
ಗಾಣ-ಗಳೂ
ಗಾಣದ
ಗಾಣ-ದಾಳು
ಗಾಣ-ದೆರೆ
ಗಾಣ-ದೆರೆ-ಗ-ಳನ್ನು
ಗಾಣ-ದೆರೆ-ಯನ್ನು
ಗಾಣ-ದೆಱೆ-ಯನು
ಗಾಣ-ಮುಮಂ
ಗಾಣ-ವನ್ನು
ಗಾಣ-ವನ್ನೂ
ಗಾಣ-ವನ್ನೇ
ಗಾಣ-ವಿದೆ
ಗಾಣ-ಸ-ಮುದ್ರದ
ಗಾಣಿಕೆ
ಗಾಣಿ-ಗನ-ಪುರ
ಗಾಣಿ-ಗನ-ಪುರ-ವೆಂಬ
ಗಾಣಿ-ಗರು
ಗಾತ್ರ
ಗಾತ್ರಂ
ಗಾತ್ರಃ
ಗಾತ್ರಿಗ
ಗಾದೆ
ಗಾದೆ-ಯನ್ನು
ಗಾಮಬ್ಬೆ-ಯನ್ನು
ಗಾಮ-ವನ್ನು
ಗಾಮುಂಡ
ಗಾಮುಂಡ-ಗಳು
ಗಾಮುಂಡನ
ಗಾಮುಂಡನು
ಗಾಮುಂಡರ
ಗಾಮುಂಡ-ರನ್ನು
ಗಾಮುಂಡ-ರಾಗಿರು
ಗಾಮುಂಡ-ರಿಗೆಲ್ಲ
ಗಾಮುಂಡರು
ಗಾಮುಂಡಸ್ವಾಮಿ
ಗಾಮುಂಡಸ್ವಾಮಿ-ಗಳ
ಗಾಮುಂಡಿಯ
ಗಾಮುಂಡಿಯ-ರೆಂದು
ಗಾಮುಣ್ಡ
ಗಾಮುಣ್ಡರು
ಗಾಮುಣ್ಡಸ್ವಾಮಿ-ಗಳ
ಗಾಮುಣ್ಡಸ್ವಾಮಿಯು
ಗಾಯಕ
ಗಾಯಿ-ಗೋಪಾಳ
ಗಾಯಿ-ಗೋವಳ
ಗಾರೆ-ಗ-ಳಿಂದ
ಗಾರೆಯ
ಗಾರೆ-ಯಿಂದ
ಗಾರ್ಗ್ಯ-ಗೋತ್ರದ
ಗಾರ್ಗ್ಯ-ಗೋತ್ರೋದ್ಭವಾಯ
ಗಾಳಿಯಂದೊಡಿಂ
ಗಾವುಂಡ
ಗಾವುಂಡ-ತನಕ್ಕೆ
ಗಾವುಂಡ-ದೇವ-ಗಾವುಂಡ-ಕಾವ-ಗಾವುಂಡ-ಚಾಮ-ಗಾವುಂಡ
ಗಾವುಂಡನ
ಗಾವುಂಡ-ನದು
ಗಾವುಂಡ-ನನ್ನೇ
ಗಾವುಂಡ-ನಿರುತ್ತಿದ್ದನು
ಗಾವುಂಡನು
ಗಾವುಂಡರ
ಗಾವುಂಡ-ರನ್ನು
ಗಾವುಂಡ-ರನ್ನೇ
ಗಾವುಂಡ-ರಲ್ಲಿ
ಗಾವುಂಡ-ರಾಗಿದ್ದ-ರೆಂದು
ಗಾವುಂಡ-ರಾಗಿ-ರು-ವು-ದಿಲ್ಲ
ಗಾವುಂಡ-ರಾದ
ಗಾವುಂಡ-ರಿಂದ
ಗಾವುಂಡ-ರಿ-ಗಿಂತ
ಗಾವುಂಡ-ರಿಗೆ
ಗಾವುಂಡ-ರಿರುತ್ತಿದ್ದರೂ
ಗಾವುಂಡರು
ಗಾವುಂಡ-ರು-ಗಳ
ಗಾವುಂಡ-ರು-ಗಳು
ಗಾವುಂಡರೂ
ಗಾವುಂಡರೇ
ಗಾವುಂಡಿ-ಯರ
ಗಾವುಂಡಿ-ಯರು
ಗಾವುಂಡಿ-ಯ-ರೆಂದು
ಗಾವುಂಡು-ಗಳ
ಗಾವುಂಡು-ಗಳು
ಗಾವುಡ
ಗಾವುಡಂರು
ಗಾವುಡ-ನಿಗೆ
ಗಾವುಡನು
ಗಾವುಣ್ಡ-ರಯ್ವತ್ತೊಕ್ಕಲು
ಗಾಹೆ
ಗಿಜ-ಹಳ್ಳಿ
ಗಿಜಿ-ಹಳ್ಳಿ
ಗಿಜೆ-ಹಳ್ಳಿಯ
ಗಿಡಗಂಟಿ-ಗಳು
ಗಿಡಗಂಟೆ-ಗ-ಳನ್ನು
ಗಿಡದ
ಗಿಡ-ಬೆಳೆದು
ಗಿಫ್ಟ್
ಗಿರಿ-ಜೆಗಂ
ಗಿರಿ-ಜೆಯ
ಗಿರಿ-ಜೇಶ್ವರ-ನೊಪ್ಪುವ
ಗಿರಿ-ದುರ್ಗ-ಮಲ್ಲ
ಗಿರಿ-ದುರ್ಗ-ವಾಗಿತ್ತೆಂಬುದು
ಗಿರಿ-ಯಣ್ಣ-ನಾಯಕ
ಗಿರಿ-ಯಣ್ಣ-ನಾಯ-ಕರ
ಗಿರಿ-ಯ-ಲಲ್ಲದೆ
ಗಿರಿ-ಯಿಂದ
ಗಿರಿಶ್ರೇಣಿ-ಗಳ
ಗಿರಿಶ್ರೇಣಿ-ಯಿದೆ
ಗಿರಿಸ್ಸು-ಗುಣೋ
ಗಿರೆ-ಗೌಡ-ನ-ಕಟ್ಟೆ
ಗಿರೇ-ಗೌಡ-ನ-ಕಟ್ಟೆ
ಗೀತ
ಗೀತ-ಗ-ಳನ್ನು
ಗೀತಿಕೆ-ಯನ್ನು
ಗೀತೆ-ಗ-ಳಲ್ಲಿ
ಗೀತೆ-ಗಳಿವೆ
ಗು
ಗುಂ
ಗುಂಡನ್ನು
ಗುಂಡಮ್ಮ
ಗುಂಡಲು-ಪೇಟೆ
ಗುಂಡಿ-ನಲ್ಲಿ
ಗುಂಡು-ಗಳಿದ್ದು
ಗುಂಡೇನ-ಹಳ್ಳಿ
ಗುಂಡ್ಲು-ಪೇಟೆ
ಗುಂಡ್ಲು-ಪೇಟೆ-ಯಲ್ಲಿ
ಗುಂಪನ್ನು
ಗುಂಪಿಗೆ
ಗುಂಪಿತ್ತು
ಗುಂಪಿನ
ಗುಂಪು
ಗುಂಪು-ಗ-ಳನ್ನು
ಗುಂಪು-ಗಳಿ-ವೆ-ಯೆಂದು
ಗುಂಪು-ಗುಂಪಾಗಿ
ಗುಂಬಜ್
ಗುಂಬಜ್ನ
ಗುಂಬಜ್ನಲ್ಲಿ
ಗುಂಬಜ್ನಲ್ಲಿ-ರುವ
ಗುಂಬದ್
ಗುಂಮಂಣನು
ಗುಂಮಟಂಣನು
ಗುಂಮಟಣ್ಣನ
ಗುಂಮಟ-ದೇವನು
ಗುಂಮಟೆ
ಗುಂಮನ-ಹಳ್ಳಿಯ
ಗುಂಮಳಾ-ಪುರದ
ಗುಗಂ
ಗುಜರಾ-ತಿನ
ಗುಜರಾಥಿ
ಗುಜ್ಜ
ಗುಜ್ಜಂ
ಗುಜ್ಜಂಸ್ವಸ್ತಿ
ಗುಜ್ಜ-ಯ-ನಾಯ್ಕನ
ಗುಜ್ಜ-ರ-ರೊಡನೆ
ಗುಜ್ಜಲೆ
ಗುಜ್ಜ-ಲೆಯ
ಗುಜ್ಜವ್ವೆ
ಗುಜ್ಜವ್ವೆ-ಕೆರೆ
ಗುಜ್ಜವ್ವೆ-ನಾಯ-ಕಿತ್ತಿಯ
ಗುಜ್ಜೆಯ
ಗುಜ್ಜೆಯ-ನಾಯ-ಕನು
ಗುಡಿ
ಗುಡಿ-ಗ-ಳನ್ನು
ಗುಡಿ-ಗ-ಳಲ್ಲಿ
ಗುಡಿ-ಗಳಿವೆ
ಗುಡಿ-ಗಳು
ಗುಡಿ-ಗಳೂ
ಗುಡಿಗೆ
ಗುಡಿಯ
ಗುಡಿ-ಯನ್ನು
ಗುಡಿ-ಯ-ಭಾ-ಗೆಗೆ
ಗುಡಿ-ಯ-ಮಧುಕೇಶ್ವರ
ಗುಡಿ-ಯಲ್ಲಿ
ಗುಡಿ-ಯಲ್ಲಿದೆ
ಗುಡಿ-ಯಲ್ಲಿ-ರುತ್ತವೆ
ಗುಡಿ-ಯಲ್ಲಿ-ರುವ
ಗುಡಿ-ಯಾಗಿ
ಗುಡಿ-ಯಾಗಿದೆ
ಗುಡಿ-ಯಾಗಿ-ರ-ಬ-ಹುದು
ಗುಡಿಯು
ಗುಡಿಯೂ
ಗುಡಿ-ಯೆಂದು
ಗುಡಿಯೇ
ಗುಡಿ-ವುತ್ತ
ಗುಡಿ-ವುತ್ತಂ
ಗುಡಿ-ಸ-ಲಿಗೆ
ಗುಡಿ-ಸಲು
ಗುಡೇನ-ಹಳ್ಳಿ
ಗುಡ್ಡ
ಗುಡ್ಡಂ
ಗುಡ್ಡ-ಗಳ
ಗುಡ್ಡ-ಗ-ಳಿಂದ
ಗುಡ್ಡ-ಗಳಿ-ರುವ
ಗುಡ್ಡ-ಗಳು
ಗುಡ್ಡದ
ಗುಡ್ಡ-ದ-ಮೇಲಿ-ರುವ
ಗುಡ್ಡಧ್ವಜ
ಗುಡ್ಡನಾ
ಗುಡ್ಡ-ನಾ-ಗಿದ್ದ
ಗುಡ್ಡ-ನಾ-ಗಿದ್ದನು
ಗುಡ್ಡ-ನಾ-ಗಿದ್ದು
ಗುಡ್ಡ-ನೆಂದು
ಗುಡ್ಡರ
ಗುಡ್ಡ-ರಾದ
ಗುಡ್ಡಿ-ಯಾಗಿದ್ದಳು
ಗುಡ್ಡು-ಗ-ಳಾದ
ಗುಡ್ಡು-ಗಳು
ಗುಡ್ಡೆ-ಹಳ್ಳಿ
ಗುಣ
ಗುಣಃ
ಗುಣ-ಗಣ-ದಿನಾ-ತನೆಣೆ-ಯಪ್ಪಂನಂ
ಗುಣ-ಗನೆ-ಯಿಂದ
ಗುಣ-ಗಾನ
ಗುಣ-ಗಾನ-ವನ್ನು
ಗುಣ-ಚಂದ್ರ
ಗುಣ-ದಭಿ
ಗುಣದಿಂ
ಗುಣ-ದಿನಾ-ದ-ನದಾವಂ
ಗುಣ-ದೊಳು
ಗುಣ-ಮಯ-ಸು-ಮಣಿಶ್ರೇಣಿನಾ
ಗುಣ-ವತಿ
ಗುಣ-ವನ್ನು
ಗುಣ-ವಿ-ಶೇಷ-ಗಳು
ಗುಣ-ಸಂಪಂನ್ನ
ಗುಣ-ಸಂಪತ್ತಿ-ಯನ್ನು
ಗುಣ-ಸಂಪದೇ
ಗುಣ-ಸಂಪನ್ನ
ಗುಣ-ಸಂಪನ್ನನುಂ
ಗುಣ-ಸಂಪನ್ನ-ರಪ್ಪ
ಗುಣ-ಸೇನ
ಗುಣಸ್ವ-ರೂಪ-ರಾದ
ಗುಣಸ್ವ-ರೂಪರುಂ
ಗುಣಾ-ಕರ
ಗುಣಾಲಂಕ್ರಿತ
ಗುಣಾ-ವಳಿ
ಗುಣೋತ್ಕರ
ಗುಣೋ-ದಗ್ರ
ಗುತ್ತ-ಕೊತ್ತಳಿ
ಗುತ್ತಗೆ
ಗುತ್ತ-ಗೆ-ಕಾರರು
ಗುತ್ತ-ಗೆ-ದಾರ-ರ-ರಾದ
ಗುತ್ತ-ಗೆ-ದಾ-ರರು
ಗುತ್ತ-ಗೆಯ
ಗುತ್ತ-ಗೆ-ಯನ್ನು
ಗುತ್ತ-ಗೆ-ಯಾಗಿ
ಗುತ್ತ-ಗೆ-ಯಿಂದ
ಗುತ್ತಲ
ಗುತ್ತ-ಲನ್ನು
ಗುತ್ತ-ಲನ್ನೂ
ಗುತ್ತ-ಲಲ್ಲಿ
ಗುತ್ತ-ಲಿನ
ಗುತ್ತ-ಲಿ-ನಲ್ಲಿ
ಗುತ್ತಲು
ಗುತ್ತಿಗೆ
ಗುತ್ತಿಗೆಗೆ
ಗುತ್ತಿಗೆಯ
ಗುತ್ತಿಗೆ-ಯನ್ನು
ಗುತ್ತಿಗೆಯು
ಗುತ್ತಿಯ
ಗುತ್ತಿಯ-ಗಂಗ
ಗುತ್ತಿಯ-ಗಂಗ-ನೆಂದು
ಗುತ್ತಿಯ-ನಾಯ-ಕನ
ಗುದಿ
ಗುದಿಗೆ
ಗುದಿ-ಯರ
ಗುದಿ-ಯರ-ಕುಲದ
ಗುದ್ಲಿ-ಕಲ್ಲು-ಮಂಠಿ
ಗುಬಿ-ಹಳ್ಳಿ
ಗುಬ್ಬಿ
ಗುಬ್ಬಿಯ
ಗುಬ್ಬಿ-ಹಳ್ಳಿ
ಗುಮ್ಮಟ-ದೇವನು
ಗುಮ್ಮಟನೇ
ಗುಮ್ಮಣ್ಣ
ಗುಮ್ಮನ
ಗುಮ್ಮನ-ವೃತ್ತಿ
ಗುಮ್ಮನ-ವೃತ್ತಿಯ
ಗುಮ್ಮನ-ವೃತ್ತಿಯೇ
ಗುಮ್ಮನ-ಹಳ್ಳಿ
ಗುಮ್ಮನ-ಹಳ್ಳಿಯ
ಗುಮ್ಮನ-ಹಳ್ಳಿ-ಯನ್ನು
ಗುಮ್ಮಳಾ-ಪುರಕ್ಕೆ
ಗುಮ್ಮಳಾ-ಪುರದ
ಗುರಿ-ಕಾರ
ಗುರಿ-ಕಾರ-ರಾಗಿ
ಗುರಿಯನ್ನಾಗಿ-ಸಿ-ಕೊಂಡು
ಗುರು
ಗುರು-ಕವಿಪ್ರಾಜ್ಞೈಃವೃತೇ
ಗುರು-ಕುಲ
ಗುರು-ಕುಲ-ವನ್ನು
ಗುರು-ಕುಳಕ್ರಮ
ಗುರು-ಕುಳ-ಮದೆಂನ್ತೆಂದಡೆ
ಗುರು-ಕುಳ-ವನ್ನು
ಗುರು-ಕುಳ-ವೆಂತೆಂದಡೆ
ಗುರು-ಗಳ
ಗುರು-ಗ-ಳನ್ನು
ಗುರು-ಗ-ಳಾಗಿದ್ದ
ಗುರು-ಗ-ಳಾಗಿದ್ದು
ಗುರು-ಗ-ಳಾದ
ಗುರು-ಗ-ಳಾದರು
ಗುರು-ಗ-ಳಿಗೆ
ಗುರು-ಗಳಿ-ರ-ಬ-ಹುದು
ಗುರು-ಗಳು
ಗುರು-ಗಳೂ
ಗುರು-ಗಳೊಡನೆ
ಗುರು-ಚಿತ್ತಂ
ಗುರು-ತಿದೆ
ಗುರು-ತಿಸ-ಬಹು-ದಾಗಿದೆ
ಗುರು-ತಿಸ-ಬ-ಹುದು
ಗುರು-ತಿಸ-ಬಹು-ದೆಂದು
ಗುರು-ತಿಸಬೇಕಾಗುತ್ತದೆ
ಗುರು-ತಿಸಲಾಗ-ದೆಂದು
ಗುರು-ತಿಸ-ಲಾಗಿದೆ
ಗುರು-ತಿಸ-ಲಾ-ಗಿದ್ದು
ಗುರು-ತಿ-ಸಲು
ಗುರು-ತಿ-ಸಲೂ
ಗುರು-ತಿಸಲ್ಲ
ಗುರು-ತಿಸಿ
ಗುರು-ತಿಸಿ-ಕೊಂಡಿದ್ದಾರೆ
ಗುರು-ತಿಸಿ-ಕೊಂಡಿದ್ದು
ಗುರು-ತಿ-ಸಿದೆ
ಗುರು-ತಿಸಿದ್ದಾರೆ
ಗುರು-ತಿ-ಸಿದ್ದು
ಗುರು-ತಿಸಿ-ರುವ
ಗುರು-ತಿಸಿ-ರು-ವು-ದಿಲ್ಲ
ಗುರು-ತಿಸಿ-ರುವುದು
ಗುರು-ತಿ-ಸುವ
ಗುರು-ತಿ-ಸುವಾಗ
ಗುರು-ತಿ-ಸು-ವುದು
ಗುರುತು
ಗುರು-ತು-ಗ-ಳನ್ನು
ಗುರು-ತು-ಗ-ಳಾದ
ಗುರು-ತು-ಗಳು
ಗುರು-ದೇವತಾ
ಗುರು-ಪರಂಪ-ರೆಗೆ
ಗುರು-ಪರಂಪ-ರೆಯ
ಗುರು-ಪರಂಪರೆ-ಯನ್ನು
ಗುರು-ಪರಂಪರೆ-ಯಲ್ಲಿ
ಗುರು-ಪರಂಪ-ರೆಯು
ಗುರು-ಪೀಠದ
ಗುರು-ಮನ-ಕಟ್ಟೆ
ಗುರು-ಮಹಾತ್ಮ
ಗುರು-ರಾಜ
ಗುರು-ರಾಜ-ರಾವ್
ಗುರು-ರಾಜಾ-ಚಾರ್
ಗುರು-ಲಿಂಗ-ಜಂಗಮ
ಗುರು-ಲಿಂಗ್ಗ
ಗುರುವ
ಗುರು-ವನ್ನು
ಗುರು-ವಾಗಿ
ಗುರು-ವಾಗಿದ್ದಂತೆ
ಗುರು-ವಾಗಿದ್ದನು
ಗುರು-ವಾಗಿದ್ದ-ನೆಂದೂ
ಗುರು-ವಾಗಿದ್ದುದು
ಗುರು-ವಾಗಿ-ರ-ಬ-ಹುದು
ಗುರು-ವಾದ
ಗುರು-ವಾರ
ಗುರು-ವಾರಕ್ಕೆ
ಗುರು-ವಿಗೆ
ಗುರು-ವಿನ
ಗುರುವೂ
ಗುರು-ವೆಂದು
ಗುರುವೋ
ಗುರ್ಜ್ಜರ
ಗುಲ-ಬರ್ಗಾ
ಗುಲಾಮ್
ಗುಲ್ಲಯ್ಯನು
ಗುಳ-ಯನ
ಗುಳಿ
ಗುಳಿಗಿ
ಗುಳಿಗೆ
ಗುಳಿಯ
ಗುಳಿ-ಯ-ಎ-ಲೆಯ
ಗುಳಿ-ಯ-ಕೆರೆ
ಗುಳೆ
ಗುಹಾ-ರೂಪದ
ಗುಹಾ-ಲಯ-ವನ್ನು
ಗುಹಾ-ವಾಸಿ-ಗ-ಳಾದ
ಗುಹೆ
ಗುಹೆ-ಯಲ್ಲಿ
ಗುಹೆ-ಯಿಂದ
ಗುಹೆ-ಯಿಂದ-ಮಣಂ
ಗೂಡಿ
ಗೂಡೆಗುಯ್ಯಲು
ಗೂಡೆ-ಯಂತಹ
ಗೂಡೆ-ಯನ್ನು
ಗೂಡೆಯು
ಗೂಢಾವ-ತಾರಾ-ಸಮಂ
ಗೂಬೆ-ಕಲ್ಲು-ಮಂಠಿ
ಗೂರ್ಜರ
ಗೂರ್ಜರ-ದೇಶದ
ಗೂರ್ಜ-ರರು
ಗೂಳಿ-ಗೌಡ
ಗೂಳೂ-ರನ್ನು
ಗೂಳೂರು
ಗೂಳೂರೇ
ಗೃದ್ಧ
ಗೃಹ
ಗೃಹಕ್ಷೇತ್ರ-ಗಳನು
ಗೃಹಕ್ಷೇತ್ರ-ಗ-ಳನ್ನು
ಗೃಹ-ನಿವೇಶನ-ಗ-ಳನ್ನು
ಗೃಹ-ಸ-ಮಾರಾಧ-ನೆಯ
ಗೃಹ-ಸೋಪಸ್ಕರ
ಗೃಹಸ್ಥೆ-ಯರು
ಗೃಹಾನ್
ಗೃಹೋಪ
ಗೃಹೋಪ-ಕರ-ಣ-ಗಳು
ಗೆ
ಗೆಜ-ಗಾರ-ಕುಪ್ಪೆ-ಯಾಗಿರ
ಗೆಜ್ಜ-ಗಾರ-ಗುಪ್ಪೆಯ
ಗೆಜ್ಜ-ಗಾರ-ಗುಪ್ಪೆ-ಯಲ್ಲಿಯೂ
ಗೆಡಿ-ಸದ
ಗೆದನ-ಕೆರೆ
ಗೆದೆ-ಗಾಂತು-ಕಂಬಳ
ಗೆದ್ದ
ಗೆದ್ದನು
ಗೆದ್ದ-ನೆಂದು
ಗೆದ್ದಾಗ
ಗೆದ್ದಿರ-ಬಹು-ದೆಂದು
ಗೆದ್ದು
ಗೆದ್ದು-ಕೊಂಡ-ನೆಂದು
ಗೆದ್ದು-ಕೊಂಡರು
ಗೆದ್ದು-ಕೊಟ್ಟ
ಗೆದ್ದು-ಕೊಟ್ಟದ್ದಕ್ಕಾಗಿ
ಗೆದ್ದು-ಕೊಟ್ಟನು
ಗೆದ್ದು-ಕೊಟ್ಟ-ನೆಂದು
ಗೆದ್ದು-ಕೊಟ್ಟಾಗ
ಗೆದ್ದುದು
ಗೆದ್ದು-ಬಂದ-ವ-ರಿ-ಗಾಗಿ
ಗೆಯಾದ-ವಗಂ
ಗೆಯ್ದಂ
ಗೆಯ್ದ-ವರ
ಗೆಯ್ದು
ಗೆಯ್ಯುತ್ತಿದ್ದು
ಗೆಯ್ಸಿದಂ
ಗೆಲವು
ಗೆಲಿದು
ಗೆಲುವು
ಗೆಲ್ದಡೆ
ಗೆಲ್ಲಲು
ಗೆಲ್ಲುತ್ತಾನೆ
ಗೆಲ್ಲುತ್ತಿದ್ದ-ನಂತೆ
ಗೇಟ್ನ
ಗೇಣಾಂಕ-ಚಕ್ರೇಶ್ವರ
ಗೇಣು
ಗೇಯದ್ಭಾಸಿ-ಗನೂನ
ಗೇರ-ಹಳ್ಳಿ
ಗೇಹ
ಗೇಹದ
ಗೇಹ-ವನ್ನು
ಗೈದ
ಗೈಯಲು
ಗೈಯುತ್ತಿದ್ದ
ಗೊಂಚ-ಲಿಗೆ
ಗೊಂಡಿದೆ-ಯೆಂದು
ಗೊಂದಲ-ವನ್ನು
ಗೊಂದಲ-ವಾಗು-ವುದು
ಗೊಂದಲ-ವಿದೆ
ಗೊಂಮಟೇಶ್ವರ
ಗೊಟ್ಟಿಯಕ್ಕಿಯ-ಹಳ್ಳಿ
ಗೊಟ್ಟಿಯಕ್ಕಿಯ-ಹಳ್ಳಿ-ಯನ್ನು
ಗೊಡಗಿ-ಕೊಡಗಿ
ಗೊಡಗಿ-ಯಾಗಿ
ಗೊಡಗೆ
ಗೊಡಮ್ಮ
ಗೊಡುಗೆ
ಗೊತ್ತಳಿ
ಗೊತ್ತಾಗುತ್ತದೆ
ಗೊತ್ತಾಗು-ವು-ದಿಲ್ಲ
ಗೊತ್ತಿ-ರಲೇ-ಬೇಕು
ಗೊತ್ತಿ-ರುವ
ಗೊತ್ತಿಲ್ಲ
ಗೊತ್ತು
ಗೊತ್ತು-ಪಡಿ-ಸ-ಲಾ-ಯಿತು
ಗೊತ್ತು-ಪ-ಡಿಸಿ
ಗೊತ್ತು-ಪಡಿ-ಸಿದ್ದ
ಗೊತ್ರದ
ಗೊಬ್ಬರಕ್ಕೆ
ಗೊಬ್ಬೂರ
ಗೊಮ್ಮಟ
ಗೊಮ್ಮಟ-ಜಿನಸ್ತುತಿ
ಗೊಮ್ಮಟ-ದೆವ-ನಿಗೆ
ಗೊಮ್ಮಟ-ದೇವರ
ಗೊಮ್ಮಟ-ದೇವರು
ಗೊಮ್ಮಟನ
ಗೊಮ್ಮಟ-ನ-ನನ್ನು
ಗೊಮ್ಮಟ-ನಲ್ತೆ
ಗೊಮ್ಮಟ-ನಿಗೆ
ಗೊಮ್ಮಟೇಶ್ವರ
ಗೊಮ್ಮಟೇಶ್ವರನ
ಗೊಮ್ಮಟೇಶ್ವರ-ನಿಗೆ
ಗೊರ-ಊರು
ಗೊರವ
ಗೊರವಂಕ
ಗೊರವನ
ಗೊರವನಾ
ಗೊರವನು
ಗೊರವರ
ಗೊರವ-ರ-ಕೆರೆ
ಗೊರವ-ರ-ಕೆರೆಯ
ಗೊರವ-ರ-ಗುಡಿಯ
ಗೊರವ-ರನ್ನು
ಗೊರವ-ರಾ-ಗಿದ್ದು
ಗೊರವರು
ಗೊರೂರು
ಗೊಲ್ದೀಯೆ
ಗೊಲ್ಲರ-ಚೆಟ್ಟನ-ಹಳ್ಳಿ
ಗೊಲ್ಲರ-ಚೆಟ್ಟನ-ಹಳ್ಳಿ-ಗಳ
ಗೊಲ್ಲರ-ಚೆಟ್ಟನ-ಹಳ್ಳಿಯ
ಗೊಲ್ಲರ-ಚೆಟ್ಟ-ಹಳ್ಳಿಯ
ಗೊಲ್ಲರ-ಹೊಸ-ಹಳ್ಳಿ
ಗೊಸಗೆಮಾಡೋ-ಡುವ-ಲೆಂಕ-ರ-ಗಂಡರುಂ
ಗೊಹೆಯ
ಗೊಹೆಯ-ಭಟ್ಟಾರಕ-ನೆಂಬ
ಗೊಹೆಯ-ಭಟ್ಟಾರಕ-ನೆಂಬುದು
ಗೊಹೆಯ-ಭಟ್ಟಾರ-ಕನ್
ಗೋ
ಗೋಂವಿಂದಯ್ಯ
ಗೋಂವಿದ-ರ-ನೆಂದೂ
ಗೋಗಂಬ-ವೆಂದು
ಗೋಗಲ್ಲು
ಗೋಗ್ರಹ-ಣ-ವಾಗಿ-ರ-ಬ-ಹುದು
ಗೋಜಲು-ಗಳಿವೆ
ಗೋಡೆ-ಕೆರೆಯ
ಗೋಡೆಗೆ
ಗೋಡೆಯ
ಗೋಡೆ-ಯನ್ನು
ಗೋಡೆಯ-ಮೇ-ಲಿದೆ
ಗೋಡೆಯ-ಮೇಲಿ-ರುವ
ಗೋಣಿ
ಗೋಣಿ-ಸೋಮ-ನ-ಹಳ್ಳಿ
ಗೋತ-ಮಯ್ಯ-ನಿಗೆ
ಗೋತ್ರ
ಗೋತ್ರಕ್ಕೆ
ಗೋತ್ರಕ್ಕೇ
ಗೋತ್ರ-ಗಣ-ಗಳ
ಗೋತ್ರ-ಗಳ
ಗೋತ್ರ-ಗ-ಳನ್ನು
ಗೋತ್ರ-ಗ-ಳನ್ನೇ
ಗೋತ್ರ-ಚಿಂತಾ-ಮಣಿ
ಗೋತ್ರದ
ಗೋತ್ರ-ದ-ವನು
ಗೋತ್ರ-ದ-ವರು
ಗೋತ್ರ-ಪವಿತ್ರ
ಗೋತ್ರ-ಪವಿತ್ರಂ
ಗೋತ್ರ-ಪವಿತ್ರನುಂ
ಗೋತ್ರ-ಪವಿತ್ರ-ನು-ಮಪ್ಪ
ಗೋತ್ರ-ಪವಿತ್ರ-ರೆಂದು
ಗೋತ್ರಪ್ರವೃತ
ಗೋತ್ರ-ವನ್ನು
ಗೋತ್ರ-ವನ್ನೇ
ಗೋತ್ರ-ಸೂತ್ರ-ಗ-ಳನ್ನು
ಗೋತ್ರ-ಸೂತ್ರ-ಪಿತೃಸ್ವಾಸ್ಥ್ಯ-ವೃತ್ತಿ
ಗೋತ್ರ-ಸೂತ್ರಾದಿ-ಗ-ಳನ್ನು
ಗೋತ್ರಾಯ
ಗೋತ್ರೋ-ದಯಃ
ಗೋದಾನ
ಗೋಪಣ್ಣ
ಗೋಪಯ್ಯ
ಗೋಪಯ್ಯನ
ಗೋಪಯ್ಯ-ನನ್ನು
ಗೋಪಾಲ
ಗೋಪಾಲ-ಕ-ರನ್ನು
ಗೋಪಾಲ-ಕೃಷ್ಣ
ಗೋಪಾಲ-ಕೃಷ್ಣ-ದೇವರ
ಗೋಪಾಲ-ಕೃಷ್ಣ-ದೇವಾ
ಗೋಪಾಲ-ಕೃಷ್ಣ-ದೇವಾ-ಲ-ಯದ
ಗೋಪಾಲ-ಕೃಷ್ಣ-ದೇವಾ-ಲಯ-ವನ್ನು
ಗೋಪಾಲ-ಕೃಷ್ಣ-ದೇವಾ-ಲಯ-ವಾಗಿ-ರುವ
ಗೋಪಾಲ-ಕೃಷ್ಣಸ್ವಾಮಿ
ಗೋಪಾಲ-ದೇವನ
ಗೋಪಾಲ-ದೇವರ
ಗೋಪಾಲ-ಪುರ
ಗೋಪಾಲ-ರಾಜ
ಗೋಪಾಲ-ರಾಯನ
ಗೋಪಾಲ-ರಾಯ-ರಂತಹ
ಗೋಪಾಲ-ರಾವ್
ಗೋಪಾಲಸ್ವಾಮಿ
ಗೋಪಾಲಸ್ವಾಮಿಯ
ಗೋಪಾಲಸ್ವಾಮಿ-ಯ-ವರ
ಗೋಪಾಲ್
ಗೋಪಾಳ
ಗೋಪಾಳ-ದೇವ
ಗೋಪಾಳ-ದೇವನ
ಗೋಪಾಳ-ದೇವನು
ಗೋಪಾಳ-ದೇವರ
ಗೋಪಾಳ-ದೇವರು
ಗೋಪಾಳ-ರಾಜನ
ಗೋಪಿ-ನಾಥ
ಗೋಪಿ-ನಾಥ-ದೇವರ
ಗೋಪಿ-ನಾಥ-ದೇವ-ರನ್ನು
ಗೋಪಿ-ನಾಥ-ದೇವ-ರಿಗೆ
ಗೋಪಿಯ
ಗೋಪಿಯ-ನಾಯಕ
ಗೋಪಿಯ-ನಾಯ-ಕನ
ಗೋಪೀ-ನಾಥ-ದೇವ-ರಿಗೆ
ಗೋಪುರ
ಗೋಪುರ-ಗ-ಳಿಗೆ
ಗೋಪುರ-ವತಿ
ಗೋಪುರ-ವನ್ನು
ಗೋಪುರವು
ಗೋಬ್ರಾಹ್ಮಣಪ್ರಿಯ
ಗೋಬ್ರಾಹ್ಮಣ-ಹಯಧೂಳಿಧೂ-ಸರ
ಗೋಮಠದ
ಗೋಮಹಿಷಿ-ಗಳ
ಗೋಮಹಿಷಿ-ಗ-ಳನ್ನು
ಗೋಮಾಂಶಕೆ
ಗೋಮಾತೆ
ಗೋಮಿನಿಗಂ
ಗೋಯರ
ಗೋಯ-ರನು
ಗೋಯಿಂದರ
ಗೋಯಿಂದ-ರನ
ಗೋಯಿಗ
ಗೋರಿ
ಗೋರಿಯ
ಗೋರಿ-ಯನ್ನು
ಗೋರಿ-ಯೊಂದರ
ಗೋಲೂರು
ಗೋಲ್ಕೊಂಡ-ದ-ವರ
ಗೋಳ-ಗ-ವುಡನು
ಗೋವಧೆ
ಗೋವ-ಬೆಟ್ಟದ
ಗೋವರ್ಧನ
ಗೋವರ್ಧನ-ಗಿರಿ-ಯಲ್ಲಿ
ಗೋವಿಂದ
ಗೋವಿಂದ-ಜೀಯ
ಗೋವಿಂದನ
ಗೋವಿಂದ-ನನ್ನು
ಗೋವಿಂದ-ನ-ಹಳ್ಳಿ
ಗೋವಿಂದ-ನ-ಹಳ್ಳಿಯ
ಗೋವಿಂದನು
ಗೋವಿಂದಪೈ
ಗೋವಿಂದ-ಪೈ-ರ-ವರ
ಗೋವಿಂದ-ಮಯ್ಯ
ಗೋವಿಂದ-ಮಯ್ಯನ
ಗೋವಿಂದ-ಮಯ್ಯ-ನೆಂಬ
ಗೋವಿಂದಯ್ಯ
ಗೋವಿಂದಯ್ಯ-ನ-ವರ
ಗೋವಿಂದಯ್ಯ-ನ-ವ-ರಿಗೆ
ಗೋವಿಂದಯ್ಯನು
ಗೋವಿಂದಯ್ಯ-ರಿಗೆ
ಗೋವಿಂದಯ್ಯಾಖ್ಯ
ಗೋವಿಂದರ
ಗೋವಿಂದ-ರ-ದೇವ
ಗೋವಿಂದ-ರ-ದೇವನ
ಗೋವಿಂದ-ರ-ದೇವ-ನಿಗೂ
ಗೋವಿಂದ-ರ-ದೇವ-ನಿಗೆ
ಗೋವಿಂದ-ರ-ದೇವನು
ಗೋವಿಂದ-ರನ
ಗೋವಿಂದ-ರ-ನಾಗಿ-ರ-ಬ-ಹುದು
ಗೋವಿಂದ-ರ-ನಿರ-ಬ-ಹುದು
ಗೋವಿಂದ-ರನು
ಗೋವಿಂದ-ರನೂ
ಗೋವಿಂದ-ರನೇ
ಗೋವಿಂದ-ರ-ರಕ್ಕಸ-ಗಂಗ
ಗೋವಿಂದ-ರ-ರಲ್ಲಿ
ಗೋವಿಂದ-ರಸ
ಗೋವಿಂದ-ರ-ಸನು
ಗೋವಿಂದ-ರಾಜ
ಗೋವಿಂದ-ರಾಜ-ಗು-ರವೇ
ಗೋವಿಂದ-ರಾಜ-ಗುರು-ಗಳ
ಗೋವಿಂದ-ರಾಜ-ಗುರು-ವಿಗೆ
ಗೋವಿಂದ-ರಾಜ-ಗುರು-ವಿನ
ಗೋವಿಂದ-ರಾಜನ
ಗೋವಿಂದ-ರಾಜ-ನನ್ನು
ಗೋವಿಂದ-ರಾಜಯ್ಯನ
ಗೋವಿಂದ-ರಾಜಯ್ಯನ-ವರ
ಗೋವಿಂದ-ರಾಜಯ್ಯನಿಗೆ
ಗೋವಿಂದ-ರಾಜರ
ಗೋವಿಂದ-ರಾಜ-ರಿಗೆ
ಗೋವಿಂದ-ರಾಜೋದ್ಯಾನ
ಗೋವಿಂದ-ರಾಜೋದ್ಯಾನ-ವನ-ವನ್ನು
ಗೋವಿಂದ-ವಾಡಿ-ಯನ್ನು
ಗೋವಿಂದಾರ್ಯನ
ಗೋವಿ-ದೇವಿ-ಯರು
ಗೋವಿ-ನಾಯಕ
ಗೋವು-ಗಳ
ಗೋವು-ಗ-ಳನ್ನು
ಗೋಸಂಪತ್ತನ್ನು
ಗೋಸನೆ
ಗೋಸ್ಕರ-ವಾಗಿಯೇ
ಗೋಹತ್ಯೆ
ಗೌಂಡ
ಗೌಂಡಿಕೆ
ಗೌಂಡಿಯ
ಗೌಂಡಿಯರು
ಗೌಡ
ಗೌಡ-ಗೆರೆ
ಗೌಡ-ಗೆರೆಯ
ಗೌಡ-ಗೆರೆ-ಯಲ್ಲಿ
ಗೌಡ-ಗೊಡುಗೆ
ಗೌಡರ
ಗೌಡ-ರನ್ನು
ಗೌಡ-ರಿಗೆ
ಗೌಡರು
ಗೌಡ-ರು-ಗಳು
ಗೌಡರೇ
ಗೌಡಿಕೆ
ಗೌಡಿ-ಕೆಗೆ
ಗೌಡಿಕೆಯ
ಗೌಡಿಕೆ-ಯನ್ನು
ಗೌಡಿಕೆಯು
ಗೌಡಿ-ತಿ-ಯರು
ಗೌಡಿನ
ಗೌಡಿಯ
ಗೌಡು
ಗೌಡು-ಗಳ
ಗೌಡು-ಗಳು
ಗೌಡು-ಗೆರೆಯ
ಗೌಡು-ಪಟ್ಟ-ಣಸ್ವಾಮಿ-ಗಳು
ಗೌಡುಪ್ರಜೆ-ಗಳು
ಗೌಣ-ವಾಗಿವೆ
ಗೌತಮ
ಗೌತಮಕ್ಷೇತ್ರ-ದಲ್ಲಿ
ಗೌತಮಕ್ಷೇತ್ರ-ವಾಸ
ಗೌತಮಕ್ಷೇತ್ರ-ವೆಂಬ
ಗೌತಮ-ಗಣ-ಧ-ರನ
ಗೌತಮ-ಗೋತ್ರದ
ಗೌತಮ-ತೀರ್ಥ-ವೆಂದು
ಗೌತಮಯ್ಯ-ನಿಗೆ
ಗೌತಮಿ
ಗೌತಮೇಶ್ವರ
ಗೌರವ
ಗೌರ-ವಕ್ಕೆ
ಗೌರವ-ಗಳಿದ್ದ-ವೆಂದು
ಗೌರವದ
ಗೌರವ-ದಿಂದ
ಗೌರವ-ಧನ
ಗೌರವ-ಧನ-ವನ್ನು
ಗೌರವ-ಪೂರ್ವ-ಕ-ವಾಗಿ
ಗೌರವ-ವನ್ನು
ಗೌರವ-ವಾಗಿದೆ
ಗೌರವವು
ಗೌರವ-ಸೂಚಕ
ಗೌರವ-ಸೂಚಕ-ವಾದ
ಗೌರವ-ಸೂಚಿ
ಗೌರವಾ-ದರ-ಗ-ಳಿಗೆ
ಗೌರವಾ-ದರ-ಗಳು
ಗೌರವಾನ್ವಿತ
ಗೌರವಾರ್ಹ-ವಾದ
ಗೌರವಿಸ-ಲಾಗಿದೆ
ಗೌರವಿಸುತ್ತಿದ್ದು-ದಕ್ಕೆ
ಗೌರಾಂಬಿಕಾ
ಗೌರಿ
ಗೌರಿ-ಕೊಳ್ಳ-ವೆಂದು-ಗ-ವರೆ
ಗೌರಿ-ಯ-ಹಳ್ಳಿ
ಗೌರೀಸ್ತನ-ಕ-ಳ-ಸರು-ಚಿವ್ಯಾಪಿವಾ-ಮಾರ್ಧದೇಹಂ
ಗೌರೇಶ್ವರ
ಗೌರ್ನರ್
ಗ್ದೇವಿ-ಯ-ರೊಳ್
ಗ್ಯಾನ-ಮಂಟಪ
ಗ್ರಂಥ
ಗ್ರಂಥ-ಕರ್ತೃ-ಗ-ಳಾದ
ಗ್ರಂಥ-ಗಳ
ಗ್ರಂಥ-ಗ-ಳನ್ನು
ಗ್ರಂಥ-ಗಳಲ್ಲಿ-ತಿ-ರುವಾಯ್ಮೋಳಿ
ಗ್ರಂಥ-ಗ-ಳಿಂದ
ಗ್ರಂಥ-ಗಳು
ಗ್ರಂಥದ
ಗ್ರಂಥ-ದಲ್ಲಿ
ಗ್ರಂಥ-ದಿಂದ
ಗ್ರಂಥ-ಲಿಪಿ
ಗ್ರಂಥ-ಲಿಪಿ-ಕನ್ನಡ
ಗ್ರಂಥ-ಲಿಪಿಯ
ಗ್ರಂಥ-ಲಿಪಿ-ಯಲ್ಲಿದೆ
ಗ್ರಂಥ-ಲಿಪಿ-ಯಲ್ಲಿ-ರುವು-ದ-ರಿಂದ
ಗ್ರಂಥ-ವನ್ನು
ಗ್ರಂಥ-ವಾಗಿ
ಗ್ರಂಥಾಕ್ಷರ-ದಲ್ಲಿ
ಗ್ರಂಥಾಸ್ಸಂತೋಷೌಯಪ್ರಭ-ವಂತ್ವಿಹ
ಗ್ರಹ-ಗ-ತಿ-ಗಳ
ಗ್ರಹ-ಸೋಪಸ್ಕರೈರ್ಯುಕ್ತಾನ್ಮೃದ್ವಾಸ್ತರಣ
ಗ್ರಹಿ-ಸ-ಬೇಕೋ
ಗ್ರಾಂಒ-ಯರ್
ಗ್ರಾಮ
ಗ್ರಾಮಂ
ಗ್ರಾಮಕ್ಕೆ
ಗ್ರಾಮಕ್ಕೆಸ್ಥಳ
ಗ್ರಾಮಕ್ಷೇತ್ರಾದಿ
ಗ್ರಾಮ-ಗದ್ಯಾಣ
ಗ್ರಾಮ-ಗಳ
ಗ್ರಾಮ-ಗ-ಳನ್ನು
ಗ್ರಾಮ-ಗ-ಳನ್ನೇ
ಗ್ರಾಮ-ಗ-ಳಲ್ಲಿ
ಗ್ರಾಮ-ಗಳಲ್ಲಿಯೂ
ಗ್ರಾಮ-ಗಳಲ್ಲಿ-ರುವ
ಗ್ರಾಮ-ಗ-ಳಾಗಿದ್ದವು
ಗ್ರಾಮ-ಗ-ಳಾಗಿದ್ದ-ವೆಂದು
ಗ್ರಾಮ-ಗ-ಳಾಗಿವೆ
ಗ್ರಾಮ-ಗ-ಳಿಂದ
ಗ್ರಾಮ-ಗಳಿಗೂ
ಗ್ರಾಮ-ಗ-ಳಿಗೆ
ಗ್ರಾಮ-ಗಳಿದ್ದವು
ಗ್ರಾಮ-ಗಳು
ಗ್ರಾಮ-ಗಳುಳ್ಳ
ಗ್ರಾಮ-ಗಳೂ
ಗ್ರಾಮ-ಗ-ಳೆಂದು
ಗ್ರಾಮ-ಗಳೆಲ್ಲ
ಗ್ರಾಮ-ಗಳೆಲ್ಲವೂ
ಗ್ರಾಮ-ಗಳೇ
ಗ್ರಾಮ-ಗಾ-ಮುಂಡ
ಗ್ರಾಮ-ಗೊಡಗೆ
ಗ್ರಾಮ-ಗೊಡಗೆಯ
ಗ್ರಾಮ-ಗೊಡಗೆ-ಯನ್ನು
ಗ್ರಾಮ-ಗೊಡುಗೆ-ಯಾಗಿ
ಗ್ರಾಮದ
ಗ್ರಾಮ-ದಲಿ
ಗ್ರಾಮ-ದಲ್ಲಿ
ಗ್ರಾಮ-ದಲ್ಲಿದ್ದ
ಗ್ರಾಮ-ದಲ್ಲಿ-ರುವ
ಗ್ರಾಮ-ದಲ್ಲೂ
ಗ್ರಾಮ-ದ-ಸೇನ-ಬೋವ
ಗ್ರಾಮ-ದಾನ
ಗ್ರಾಮ-ದಾನಬ್ರಹ್ಮ-ದೇಯ
ಗ್ರಾಮ-ದಿಂದ
ಗ್ರಾಮ-ದೇವ-ತಾ-ಪುರದ
ಗ್ರಾಮ-ದೇವತೆ
ಗ್ರಾಮ-ದೇವ-ತೆ-ಗಳ
ಗ್ರಾಮ-ದೇವ-ತೆಯ
ಗ್ರಾಮ-ದೇವ-ತೆ-ಯಾದ
ಗ್ರಾಮ-ದೇವ-ತೆಯು
ಗ್ರಾಮ-ನಾ-ಮ-ಗ-ಳನ್ನು
ಗ್ರಾಮ-ಪಂಚ-ಕ-ವೆಂದು
ಗ್ರಾಮಬ್ರಯ
ಗ್ರಾಮಬ್ರಯವ
ಗ್ರಾಮಬ್ರಯ-ವೆಂಬ
ಗ್ರಾಮ-ಮಟ್ಟ-ದಲ್ಲಿ
ಗ್ರಾಮ-ಮಧ್ಯದ
ಗ್ರಾಮವ
ಗ್ರಾಮ-ವನು
ಗ್ರಾಮ-ವನ್ನು
ಗ್ರಾಮ-ವನ್ನು-ಧನ-ಗೂರು
ಗ್ರಾಮ-ವನ್ನೇ
ಗ್ರಾಮ-ವಳಿ
ಗ್ರಾಮ-ವಾಗಿತು
ಗ್ರಾಮ-ವಾ-ಗಿತ್ತು
ಗ್ರಾಮ-ವಾಗಿತ್ತುಬ್ರಹ್ಮ-ದೇಯ
ಗ್ರಾಮ-ವಾಗಿತ್ತೆಂದು
ಗ್ರಾಮ-ವಾಗಿದೆ
ಗ್ರಾಮ-ವಾ-ಗಿದ್ದ
ಗ್ರಾಮ-ವಾ-ಗಿದ್ದು
ಗ್ರಾಮ-ವಾಗಿ-ರ-ಬ-ಹುದು
ಗ್ರಾಮ-ವಾದ
ಗ್ರಾಮ-ವಿತ್ತೆಂದು
ಗ್ರಾಮ-ವಿದೆ
ಗ್ರಾಮವು
ಗ್ರಾಮವೂ
ಗ್ರಾಮ-ವೆಂದರೆ
ಗ್ರಾಮ-ವೆಂದು
ಗ್ರಾಮವೇ
ಗ್ರಾಮ-ಸಂಖ್ಯಾ-ಧಾರಿತ
ಗ್ರಾಮ-ಸಭೆ-ಗಳು
ಗ್ರಾಮ-ಸಭೆಗೆ
ಗ್ರಾಮ-ಸಭೆಯ
ಗ್ರಾಮ-ಸೀಮೆ-ಗಳ
ಗ್ರಾಮ-ಸೀಮೆಗೂ
ಗ್ರಾಮ-ಸೀಮೆಯ
ಗ್ರಾಮಸ್ಯ
ಗ್ರಾಮಾ-ದಾಯ
ಗ್ರಾಮಾಧಿ
ಗ್ರಾಮಾಧಿ-ದೇವತೆ
ಗ್ರಾಮೀಣ
ಗ್ರಾಮೇ
ಗ್ರಾಹ್ಯರಾಹುಃ
ಗೞ್ದೆ
ಘಂಟಣ್ಣನ
ಘಂಟೆ-ಯನ್ನು
ಘಂಠಂಣ
ಘಟಕ
ಘಟಕ-ಗಳ
ಘಟಕ-ಗ-ಳಾಗಿದ್ದ
ಘಟಕ-ಗ-ಳಾಗಿದ್ದವು
ಘಟಕ-ಗ-ಳಾದ
ಘಟಕ-ಗಳಿದ್ದವು
ಘಟಕ-ಗಳು
ಘಟಕ-ವಾಗಿ
ಘಟನಾತ್ಮಕ-ವಾಗಿ-ರುವುದು
ಘಟನೆ
ಘಟನೆ-ಗಳ
ಘಟನೆ-ಗ-ಳನ್ನು
ಘಟನೆ-ಗ-ಳಿಂದ
ಘಟನೆ-ಗಳು
ಘಟನೆ-ಯನ್ನು
ಘಟನೆ-ಯಾಗಿದೆ
ಘಟನೆ-ಯಿಂದ
ಘಟ-ನೆಯು
ಘಟನೆ-ಯೊಂದನ್ನು
ಘಟ-ವೆಂಬ
ಘಟಾನುಘಟಿ-ಗಳು
ಘಟಿಕಾ
ಘಟಿಕಾಸ್ಥಾನ-ಗ-ಳನ್ನು
ಘಟಿಕಾಸ್ಥಾನದ
ಘಟಿಕಾಸ್ಥಾನ-ವೆಂದು
ಘಟಿತ
ಘಟಿಯಂತ್ರ
ಘಟಿಸಿ-ರ-ಬೇಕೆಂದು
ಘಟೀ-ಯಂತ್ರ-ಗ-ಳಿಂದ
ಘಟೀಯಂತ್ರದ
ಘಟೆಯಂ
ಘಟೆಯುಂ
ಘಟ್ಟ
ಘಟ್ಟದ
ಘಟ್ಟ-ವನ್ನು
ಘಟ್ಟವು
ಘಟ್ಟಿ
ಘಟ್ಟಿ-ವರಹ
ಘಣ್ಟಮ್ಮ
ಘಣ್ಟಮ್ಮನ
ಘಣ್ಟಮ್ಮ-ನಿಗೆ
ಘಣ್ಟಮ್ಮನು
ಘನ-ಗಿರಿ
ಘನ-ಗಿರಿಗೆ
ಘನ-ಘೋರ
ಘನ-ತರ
ಘನ-ತರ-ಕೂಟ-ಕೋಟಿ-ಯುತ
ಘನತೆ-ಗೌ-ರವ-ಗ-ಳನ್ನು
ಘನ-ತೆಯು
ಘನತೆ-ವೆತ್ತ
ಘನವೃತ್ತ
ಘನವೃತ್ತಸ್ತನ-ಹಾರ-ಶೂ-ರನು
ಘಮ್ಮ-ನಾಯಕ
ಘರ್ಜ-ನೆಗೆ
ಘರ್ಜಿ-ಸುತ್ತಾ
ಘರ್ಷಣೆ
ಘರ್ಷಣೆ-ಗ-ಳನ್ನು
ಘರ್ಷಣೆ-ಗ-ಳಾಗಿ-ರುವು-ದರ
ಘರ್ಷಣೆ-ಗಳು
ಘರ್ಷ-ಣೆಯ
ಘರ್ಷಣೆ-ಯನ್ನು
ಘರ್ಷಣೆ-ಯಲ್ಲಿ
ಘರ್ಷ-ಣೆಯೇ
ಘೃತ-ಪರ್ವತ
ಘೃತ-ಪರ್ವ-ತ-ದಾನ-ವನ್ನು
ಘೇಣಾಂಕ
ಘೋರ
ಘೋರ-ವಾದ
ಘೋಷಣೆ
ಘೋಷಿಸ-ಬೇಕಾ-ಯಿತು
ಘೋಷಿಸ-ಲಾಗಿದೆ
ಘೋಷಿ-ಸ-ಲಾ-ಯಿತು
ಘೋಷಿಸಿ
ಘೋಷಿಸಿ-ಕೊಂಡಿದ್ದನು
ಘೋಷಿಸಿ-ದರು
ಘೋಷಿಸಿ-ದ-ರೆಂದು
ಘೋಷಿಸುತ್ತಿದ್ದರು
ಚ
ಚಂಗ
ಚಂಗ-ಣವ್ವೆ
ಚಂಗ-ಭೂ-ಪನ
ಚಂಗ-ವಾಡಿ
ಚಂಗ-ವಾಡಿಯ
ಚಂಗ-ವಾಡಿ-ಯನ್ನು
ಚಂಗ-ವಾಡಿ-ಯಲ್ಲಿ
ಚಂಗಾಳ್ವನಂ
ಚಂಗಾಳ್ವರ
ಚಂಗಾಳ್ವ-ರದ್ದು
ಚಂಗಾಳ್ವರು
ಚಂಗಿ
ಚಂಗಿ-ಕುಲ
ಚಂಗಿ-ಕುಲ-ಕ-ಮಲ
ಚಂಗಿ-ಕುಲದ
ಚಂಗಿ-ಕುಳ
ಚಂಗಿ-ಕುಳ-ಕಮಳ
ಚಂಗಿ-ಕುಳ-ಕಮಳ-ಮಾರ್ತ್ರಂಡ-ನತುಳ
ಚಂಚರಿ-ವಳ್ಳ-ವನ್ನು
ಚಂಚರೀ-ವಳ್ಳಂ
ಚಂಚರೀ-ಹಳ್ಳ
ಚಂಡ-ಮಾರಿ-ದೇವ-ತೆಗೆ
ಚಂಡ-ಶಾ-ಸನನ
ಚಂಡಾಲ
ಚಂಡಿಕೇಶ್ವರರ
ಚಂಣನುಯ
ಚಂದ-ಗಾಲು
ಚಂದನ-ಸುಧಾ
ಚಂದಪ್ಪ-ವೊಡೆ-ಯನ
ಚಂದಯ್ಯ
ಚಂದಯ್ಯನು
ಚಂದಲ-ದೇವಿ
ಚಂದಲ-ದೇ-ವಿಗೆ
ಚಂದಲ-ದೇವಿಯ
ಚಂದಲ-ದೇವಿ-ಯನ್ನು
ಚಂದಲ-ದೇವಿ-ಯರು
ಚಂದಲ-ದೇವಿಯು
ಚಂದಲ-ದೇವಿಯೂ
ಚಂದ-ಹಳ್ಳಿ
ಚಂದ-ಹಳ್ಳಿಯ
ಚಂದ-ಹಳ್ಳಿ-ಯನ್ನು
ಚಂದಿಗ
ಚಂದಿ-ಗಾಲು
ಚಂದಿ-ಯಕ್ಕರ
ಚಂದ್ರ
ಚಂದ್ರಃ
ಚಂದ್ರ-ಗಿರಿ
ಚಂದ್ರ-ಗಿರಿ-ಗ-ಳನ್ನು
ಚಂದ್ರ-ಗುಪ್ತ
ಚಂದ್ರ-ಗುಪ್ತ-ನಲ್ಲ-ವೆಂದೂ
ಚಂದ್ರ-ಗುಪ್ತ-ನಿಗೆ
ಚಂದ್ರ-ಗುಪ್ತ-ನೆಂದೂ
ಚಂದ್ರ-ಗುಪ್ತ-ನೊಡನೆ
ಚಂದ್ರ-ಗುಪ್ತ-ಮೌರ್ಯನೂ
ಚಂದ್ರ-ಗುಪ್ತರ
ಚಂದ್ರ-ಗುಪ್ತ-ರನ್ನು
ಚಂದ್ರ-ಗುಪ್ತರು
ಚಂದ್ರ-ಧರಾ
ಚಂದ್ರ-ನಂತಹ
ಚಂದ್ರ-ನಂತಿದೆ
ಚಂದ್ರ-ನಂದಿ
ಚಂದ್ರ-ನಂದಿಗೆ
ಚಂದ್ರ-ನಂದಿಯ
ಚಂದ್ರ-ನನ್ನು
ಚಂದ್ರ-ನಾಥ
ಚಂದ್ರ-ನಾಥಸ್ವಾಮಿಯ
ಚಂದ್ರಪ್ರಭ
ಚಂದ್ರ-ಭೂಷಣ-ರಿಗೆ
ಚಂದ್ರಮಾ
ಚಂದ್ರ-ಮಾಶ್ಚಂದ್ರ-ಕೀರ್ತಿ-ಮಾನ್
ಚಂದ್ರ-ಮೌಳಿ
ಚಂದ್ರ-ಮೌಳಿ-ಯಣ್ಣ
ಚಂದ್ರ-ಮೌಳಿ-ಯಣ್ಣನ
ಚಂದ್ರ-ಮೌಳಿ-ಯಣ್ಣನು
ಚಂದ್ರ-ಮೌಳಿಯು
ಚಂದ್ರ-ಮೌಳೀಶ್ವರ
ಚಂದ್ರ-ಮೌಳೇಶ್ವರ
ಚಂದ್ರ-ರೂಪೋ
ಚಂದ್ರ-ವ-ನದ
ಚಂದ್ರ-ವ-ನದ-ಬಳಿ
ಚಂದ್ರ-ಶೇಖರ
ಚಂದ್ರಾ-ಕುರ-ಜೋತ್ಸ್ನಾ
ಚಂದ್ರಾ-ಭರಣ
ಚಂದ್ರೊಬ್ಬಲಬ್ಬೆ-ಯನ್ನು
ಚಂದ್ರ್ರಗುಪ್ರ
ಚಂನ-ಪಟ್ಟಣ
ಚಂನ-ರಾಜ
ಚಂಪೂ-ಕಾವ್ಯ-ವನ್ನಾಗಿ
ಚಂಪೂ-ಕಾವ್ಯ-ವನ್ನು
ಚಂಬಲ್ಲೀ-ಪುರ
ಚಂಬಿನ
ಚಂಮಟಿ
ಚಉಗಾವೆಯ
ಚಉಡೋ-ಜನ
ಚಕಿತ
ಚಕ್ಕಟೆಯ
ಚಕ್ಕೆರೆ
ಚಕ್ತ-ವರ್ತಿ
ಚಕ್ರ
ಚಕ್ರ-ಕೊಳ
ಚಕ್ರ-ಗಳ
ಚಕ್ರ-ಗೊಟ್ಟ
ಚಕ್ರ-ಗೊಟ್ಟಮು
ಚಕ್ರದ
ಚಕ್ರ-ವರ್ತಿ
ಚಕ್ರ-ವರ್ತಿ-ಗಳ
ಚಕ್ರ-ವರ್ತಿ-ಗ-ಳಿಗೆ
ಚಕ್ರ-ವರ್ತಿ-ಗಳು
ಚಕ್ರ-ವರ್ತಿಗೆ
ಚಕ್ರ-ವರ್ತಿ-ಭಟ್ಟೋಪಾಧ್ಯಾಯ-ರಿಗೆ
ಚಕ್ರ-ವರ್ತಿಯ
ಚಕ್ರ-ವರ್ತಿ-ಯಾ-ಗಿದ್ದ
ಚಕ್ರ-ವರ್ತಿ-ಯಾದ
ಚಕ್ರ-ವರ್ತಿ-ಯಿಂದ
ಚಕ್ರ-ವರ್ತಿಯು
ಚಕ್ರ-ವರ್ತಿಯೂ
ಚಕ್ರ-ವರ್ತ್ತೀ
ಚಕ್ರಾಧಿ-ಪತಿ-ಯಾದ
ಚಕ್ರೇಶ
ಚಕ್ರೇಶನ
ಚಕ್ರೇಶ್ವರ
ಚಕ್ರೇಶ್ವರನ
ಚಕ್ಷುಷೇ
ಚಟಮ-ಗೆರೆ
ಚಟಯ-ನಾಯ-ಕನ
ಚಟಾಕು
ಚಟುಕುಂ
ಚಟುವಟಿಕೆ-ಗಳ
ಚಟುವಟಿಕೆ-ಗ-ಳನ್ನು
ಚಟುವಟಿಕೆ-ಗ-ಳಲ್ಲಿ
ಚಟುವಟಿಕೆ-ಗ-ಳಿಗೆ
ಚಟುವಟಿಕೆ-ಗಳು
ಚಟ್ಟಂಗೆರೆ
ಚಟ್ಟಂಗೆರೆ-ಯನ್ನು
ಚಟ್ಟಣ-ಕೆರೆ
ಚಟ್ಟಣ-ಕೆರೆ-ಚಟ್ಟಂಗೆರೆ
ಚಟ್ಟಣ-ಕೆರೆ-ಚಟ್ಟಮೆರೆ
ಚಟ್ಟ-ದೇವ
ಚಟ್ಟ-ಪಯ್ಯ
ಚಟ್ಟಮ-ಗೆರೆ
ಚಟ್ಟಮ-ಗೆರೆಯ
ಚಟ್ಟಮ-ಗೆರೆ-ಯನ್ನು
ಚಟ್ಟಮ-ಗೆರೆ-ಯಲ್ಲಿ
ಚಟ್ಟಮ-ಗೆರೆ-ಯಾಗಿ-ರ-ಬ-ಹುದು
ಚಟ್ಟಯ
ಚಟ್ಟಯ-ಚಟ್ಟಮ-ಗೆರೆ
ಚಟ್ಟ-ಯವು
ಚಟ್ಟಯ್ಯನ-ಹಳ್ಳಿ
ಚಟ್ಟಲ-ದೇವಿ
ಚಟ್ಟಲೆ-ಕೂಡೆ
ಚಟ್ಟ-ಲೆಯುಂ
ಚಟ್ಟೇನ-ಹಳ್ಳಿ-ಯನ್ನು
ಚಟ್ಟೊಡೆಯ
ಚಟ್ಟೊಡೆ-ಯನು
ಚತುಃಶಾಲಾ
ಚತು-ರಂಗ-ಬಲ-ವನ್ನು
ಚತುರಙ್ಗ
ಚತುರ-ತರೋ-ದಾರ
ಚತುರಳಾ-ಗಿದ್ದು
ಚತುರ್ತ್ಥ
ಚತುರ್ತ್ಥ-ವಂಶರು
ಚತುರ್ತ್ಥ-ವಂಶ-ರೊಳು
ಚತುರ್ಥ
ಚತುರ್ಥ-ಕುಲ-ದ-ವರು
ಚತುರ್ಥ-ಗೋತ್ರ
ಚತುರ್ಥ-ಗೋತ್ರದ
ಚತುರ್ದ್ಧಶ-ವಿದ್ಯಾಸ್ಥಾನಾಧಿಗಮ-ವಿ-ಮಲ-ಮತಿಃ
ಚತುರ್ಭಾಷಾ
ಚತುರ್ಭುಜ
ಚತುರ್ಮುಖ
ಚತುರ್ವಿದ
ಚತುರ್ವಿಧ
ಚತುರ್ವೇದಿ
ಚತುರ್ವೇದಿ-ಮಂಗಲದ
ಚತುರ್ವೇದಿ-ಮಂಗಲ-ವಾದ
ಚತುರ್ವೇದಿ-ಮಂಗಲ-ವೆಂಬ
ಚತುರ್ವ್ವಿಧಾನೂನ-ದಾನ-ವಿನೋದಂ
ಚತುರ್ವ್ವೇದಿ
ಚತು-ಷಷ್ಟಿ-ಕಳಾ-ಕಳಿತ
ಚತುಷ್ಕಣ್ಡುಗ
ಚತುಷ್ಟ-ಯಕ್ಕಂ
ಚತುಷ್ಟಯ-ಗ-ಳಿಗೆ
ಚತು-ಸ-ಮಯ-ಸ-ಮುದ್ಧರಣಂ
ಚತು-ಸ-ಮುದ್ರಾಧಿ-ಪತಿ
ಚತು-ಸೀಮೆ-ಯೊಳ-ಗುಳ
ಚತುಸ್ಸ-ಮುದ್ರಾಧಿ-ಪತಿ
ಚತುಸ್ಸೀಮೆಗೆ
ಚತುಸ್ಸೀಮೆಯ
ಚತುಸ್ಸೀಮೆ-ಯನ್ನು
ಚತುಸ್ಸೀಮೆ-ಯಲಿ
ಚತುಸ್ಸೀಮೆ-ಯಾಗಿ
ಚತುಸ್ಸೀಮೆ-ಯೊಳಗಾದ
ಚತುಸ್ಸೀಮೆ-ಯೊಳಗೆ
ಚತ್ರ
ಚತ್ರ-ರಿಗೆ-ಬಹುಶಃ
ಚತ್ರಾಧಿ-ಕಾರಿ
ಚತ್ರಾಧಿ-ಕಾರಿ-ಯಾಗಿದ್ದನು
ಚದಿರ-ಗಟ್ಟ
ಚದುರಿ-ದರು
ಚದುರಿ-ಹೋ-ದಂತೆ
ಚದುರಿ-ಹೋದರು
ಚದು-ರುವಿ-ಕೆಯ
ಚನ್ದ-ಕಕೋ-ಜನ
ಚನ್ದಯ್ಯನು
ಚನ್ದ್ರ-ಗುಪ್ತ
ಚನ್ದ್ರ-ಗುಪ್ತ-ಮುನಿ-ಪತಿ
ಚನ್ನಕೆಶವ-ಪುರ-ವೆಂಬ
ಚನ್ನ-ಕೇಶವ
ಚನ್ನ-ಕೇಶವ-ದೇವ-ರಿಗೆ
ಚನ್ನ-ಕೇಶವ-ದೇವಾ-ಲಯ-ವನ್ನು
ಚನ್ನ-ಕೇಶವ-ದೇವಾ-ಲ-ಯವು
ಚನ್ನ-ಕೇಶವನ
ಚನ್ನ-ಕೇಶವ-ಪುರ-ವಾದ
ಚನ್ನ-ಕೇಶವ-ಪುರ-ವೆಂಬ
ಚನ್ನ-ಕೇಶವರ
ಚನ್ನಕೇಶ್ವರ
ಚನ್ನ-ದೇವಿಯು
ಚನ್ನನಂಜ-ರಾಜ-ನಿಗೇ
ಚನ್ನನಂಜ-ರಾಜ-ನೆಂದು
ಚನ್ನ-ಪಟ್ಟಣ
ಚನ್ನ-ಪಟ್ಟ-ಣದ
ಚನ್ನ-ಪಟ್ಟ-ಣ-ರಾಜ್ಯದ
ಚನ್ನ-ಪಟ್ಟ-ಣ-ರಾಜ್ಯ-ವನ್ನು
ಚನ್ನ-ಪಟ್ಟ-ಣ-ವನ್ನು
ಚನ್ನ-ಪಟ್ಟ-ಣಸ್ಥಳಕ್ಕೆ
ಚನ್ನಪ್ಪನ-ದೊಡ್ಡಿ-ಯಲ್ಲಿದೆ
ಚನ್ನಪ್ಪ-ನ-ವರು
ಚನ್ನಪ್ಪನು
ಚನ್ನ-ಬಸವ
ಚನ್ನಮ್ಮ
ಚನ್ನಯ್ಯ
ಚನ್ನಯ್ಯನ
ಚನ್ನಯ್ಯನು
ಚನ್ನ-ರಾಜ
ಚನ್ನ-ರಾಯ
ಚನ್ನ-ರಾಯ-ಪಟ್ಟಣ
ಚನ್ನ-ರಾಯ-ಪಟ್ಟ-ಣದ
ಚನ್ನಿ-ಗೌಡನ
ಚಪೂತ
ಚಪ್ಪಡಿ
ಚಪ್ಪರಿಸಿ
ಚಪ್ಪಲಿ-ಗಳ
ಚಬನ-ಹ-ಳಿಯ
ಚಮದ್ರ-ಮೌಳಿಯ
ಚಮೂ-ಧರ
ಚಮೂಪ
ಚಮೂಪತಿ
ಚಮೂಪ-ತಿಯ
ಚಮೂಪನ
ಚಮೂಪನು
ಚಮೂಪನೋ
ಚಮೂಪ-ರೆಂದೂ
ಚಮ್ಮ-ಕಾರ-ರಚರ್ಮ-ಕಾರರ
ಚಮ್ಮಾವು-ಗೆಯ
ಚರಂಡಿಯ
ಚರಣ
ಚರಣ-ಗ-ಳಿಗೆ
ಚರ-ಣದ
ಚರಣಾಕ್ಯನೆ-ನಲು
ಚರಣಾ-ರವಿಂದ
ಚರಪಿಗೆ
ಚರಮ-ಗೀತೆ-ಯಂತಿದೆ
ಚರಮ-ಗೀತೆ-ಯಂತಿದ್ದು
ಚರ-ಸುಂಕ
ಚರಾದಾಯ
ಚರಿತಂ
ಚರಿತಃ
ಚರಿತರೆಂ
ಚರಿತೆ
ಚರಿತೆ-ಯಲ್ಲಿ
ಚರಿತೆ-ಯಲ್ಲೂ
ಚರಿತ್ರೆ
ಚರಿತ್ರೆಯ
ಚರಿತ್ರೆ-ಯನ್ನು
ಚರಿತ್ರೆ-ಯಲ್ಲಿ
ಚರಿತ್ರೆ-ಯಿಂದ
ಚರಿತ್ರೆಯು
ಚರು-ಪನ್ನು
ಚರುಪಿಗೆ
ಚರುಪು
ಚರುಪು-ಕಟ್ಟಳೆ
ಚರ್ಚಾಸ್ಪದ-ವಾಗಿ-ರುವ
ಚರ್ಚಾಸ್ಪದ-ವಾದ
ಚರ್ಚಿಸುತ್ತಿದ್ದರು
ಚರ್ಚೆಯ
ಚರ್ಪನ್ನು
ಚರ್ಪಿಗೆ
ಚರ್ಪು
ಚರ್ಮದ-ವಸ್ತು
ಚರ್ವಣೆ
ಚಲಂ
ಚಲಕ-ದೇವ
ಚಲಕೆ-ಬಲು-ಗಂಡ
ಚಲದ
ಚಲದಂಕ-ರಾಮ
ಚಲದಂಕ-ರಾಮಂ
ಚಲದುತ್ತ-ರಂಗ
ಚಲನಚಿತ್ರ
ಚಲನವಲನ-ಗ-ಳನ್ನು
ಚಲನೆ
ಚಲಾಯಿ-ಸಲು
ಚಲಾ-ವಣೆಗೆ
ಚಲಾ-ವಣೆ-ಯಲ್ಲಿದ್ದ
ಚಲಾ-ವಣೆ-ಯಲ್ಲಿದ್ದರೂ
ಚಲಾ-ವಣೆ-ಯಲ್ಲಿದ್ದಿ-ತೆಂದೂ
ಚಲುಕ್ಯ
ಚಲುಕ್ಯರ
ಚಲುಕ್ಯರು
ಚಲುವ-ನಾ-ರಾಯಣ
ಚಲುವ-ನಾ-ರಾಯ-ಣನ
ಚಲು-ವನು
ಚಲುವ-ರಾಯಸ್ವಾಮಿ
ಚಲುವವ್ವೆಯ
ಚಲುವವ್ವೆಯರ
ಚಲುವಾಜಮ್ಮಣಿ-ಯ-ವರು
ಚಲ್ಲೆಶ್ವರದ
ಚಲ್ಲೇಶ್ವರ
ಚಳ-ವಳಿ-ಯಲ್ಲಿ
ಚಳಿಸೆ
ಚವರಂ
ಚವರ-ಬಂಬಾಳು
ಚವುಂಡಾಡಿ
ಚವುಂಡಾ-ಡಿಯ
ಚವುಂಡಾ-ಡಿಯೂ
ಚವು-ಗಾವುಂಡ-ಗ-ಳನ್ನು
ಚವುಗಾವು-ಗಳ
ಚವುಗಾವು-ಗಳು
ಚವುಗಾವೆ
ಚವುಡ-ಗಾವುಂಡನ
ಚವುಡಪ್ಪ
ಚವುಡಪ್ಪನ
ಚವು-ಡಯ್ಯ
ಚವು-ಡಯ್ಯನ
ಚವು-ಡಯ್ಯ-ನ-ಹಳ್ಳಿ
ಚವುಡಾ-ಚಾರಿಯ
ಚವುಡಿ-ಗೌಡ
ಚವುಡಿ-ತಮ್ಮಕ್ಕ
ಚವುಡೆ-ಗೊಂಡ-ನಿಗೆ
ಚವುಡೆ-ಗೌಡಂಗೆ
ಚವುಡೆ-ಗೌಡ-ನಿಗೆ
ಚವುಡೋಜ
ಚವುಡೋ-ಜನ
ಚವುತ್ತ-ರನು
ಚವ್ವಪ್ಪ
ಚಾಂದ್ರಾಯ-ಣ-ದೇವರು
ಚಾಕ-ಗಾವುಂಡನ
ಚಾಕಲೆ
ಚಾಕಲೆಯ
ಚಾಕಲೆ-ಯನ್ನು
ಚಾಕಳ-ಹಳ್ಳಿಯ
ಚಾಕೆನ-ಹಳ್ಳಿಯ
ಚಾಕೆಯ-ನ-ಹಳ್ಳಿ-ಗಳ
ಚಾಕೆಯ-ನ-ಹಳ್ಳಿ-ಯನ್ನು
ಚಾಕೇನ-ಹಳ್ಳಿ
ಚಾಕೇನ-ಹಳ್ಳಿಯ
ಚಾಗಿ
ಚಾಗಿ-ಪೆರ್ಮಾನ-ಡಿ-ಗಳ
ಚಾಣಕ್ಯನೆನಿಪ
ಚಾತುರ್ದಂತ-ಬಲಂ
ಚಾತುರ್ವರ್ಣದ
ಚಾತುರ್ವರ್ಣದಲ್ಲಿ
ಚಾತುರ್ವ್ವೈದ್ಯ
ಚಾನವೆಯು
ಚಾನವ್ವೆಯರು
ಚಾಪಿ
ಚಾಪೆ
ಚಾಮ
ಚಾಮಂಡ-ಹಳ್ಳಿ
ಚಾಮ-ಗವುಂಡನ
ಚಾಮ-ಗಾ-ಮುಂಡನು
ಚಾಮ-ಗಾವುಂಡನ
ಚಾಮ-ಗಾವುಂಡನು
ಚಾಮ-ಡ-ಹಳ್ಳಿ
ಚಾಮಣ್ಣನು
ಚಾಮ-ದೇವ-ನಿರ-ಬಹು-ದೆಂದು
ಚಾಮ-ನೃಪ-ನಿಗೆ
ಚಾಮ-ನೃಪ-ನು-ಬೋಳು-ಚಾಮ-ರಾಜ
ಚಾಮಪ್ಪನು
ಚಾಮಮ್ಮಣ್ಣಿ
ಚಾಮಯ್ಯ
ಚಾಮರ
ಚಾಮ-ರಸ
ಚಾಮ-ರ-ಸ-ಗೌಡನು
ಚಾಮ-ರ-ಸನು
ಚಾಮ-ರ-ಸ-ವೊಡೆ-ಯನು
ಚಾಮ-ರ-ಸ-ವೊಡೆ-ಯರ
ಚಾಮ-ರ-ಸ-ವೊಡೆ-ಯ-ರಿಗೆ
ಚಾಮ-ರ-ಸ-ವೊಡೆ-ಯರು
ಚಾಮ-ರ-ಸೊಡೆಯ-ರ-ವರ
ಚಾಮ-ರ-ಸೊಡೆ-ಯರೈಯ-ನ-ವರ
ಚಾಮ-ರಾಜ
ಚಾಮ-ರಾಜ-ಒಡೆ-ಯರು
ಚಾಮ-ರಾಜನ
ಚಾಮ-ರಾಜ-ನ-ಗರ
ಚಾಮ-ರಾಜ-ನ-ಗರದ
ಚಾಮ-ರಾಜ-ನ-ಗರ-ದಲ್ಲಿ
ಚಾಮ-ರಾಜ-ನನ್ನು-ಹತ್ತ-ನೆಯ
ಚಾಮ-ರಾಜ-ನಿಗೆ
ಚಾಮ-ರಾಜನು
ಚಾಮ-ರಾಜ-ನೆಂದು
ಚಾಮ-ರಾಜ-ನೆಂಬ
ಚಾಮ-ರಾಜನೇ
ಚಾಮ-ರಾಜ-ಪುರ-ಹಲುಕೂರು
ಚಾಮ-ರಾಜ-ವಡೆ-ಯರ್
ಚಾಮ-ರಾಜ-ವೊಡೇರ
ಚಾಮ-ರಾಜೇಂದ್ರ
ಚಾಮ-ರಾಜೊಡೆಯರ
ಚಾಮ-ರಾಜೊಡೆಯ-ರಿಗೆ
ಚಾಮ-ರಾಜೊಡೆಯರು
ಚಾಮ-ಲ-ದೇವಿ
ಚಾಮ-ಲಾ-ಪುರ
ಚಾಮ-ಲಾ-ಪುರದ
ಚಾಮ-ಲಾ-ಪುರ-ವನ್ನು
ಚಾಮ-ಲಾ-ಪುರ-ವೆಂಬ
ಚಾಮ-ಲೆಯು
ಚಾಮವ್ವೆ
ಚಾಮವ್ವೆಯ
ಚಾಮವ್ವೆ-ಯರು
ಚಾಮವ್ವೆಯು
ಚಾಮಾಂಬಿ-ಕೆಗೆ
ಚಾಮುಂಡನು
ಚಾಮುಂಡ-ರಾಯ
ಚಾಮುಂಡ-ರಾಯಂ
ಚಾಮುಂಡ-ರಾಯನ
ಚಾಮುಂಡ-ರಾಯ-ನಿಗೆ
ಚಾಮುಂಡ-ರಾಯ-ನಿರ-ಬ-ಹುದು
ಚಾಮುಂಡ-ರಾಯನು
ಚಾಮುಂಡ-ರಾಯ-ನೆಂಬ
ಚಾಮುಂಡ-ರಾಯ-ನೆಂಬು-ವ-ವನು
ಚಾಮುಂಡವ್ವೆ
ಚಾಮುಂಡೇಶ್ವರಿ
ಚಾಮುಂಡೇಶ್ವರಿಯ
ಚಾಮುಣ್ಡಯ್ಯನೂ
ಚಾಮುಣ್ಡ-ರಿಬ್ಬರೂ
ಚಾಮೇಂದ್ರ-ನಿಗೆ
ಚಾಯರುಂ
ಚಾರ-ಗೌಂಡ
ಚಾರಿತ್ರ
ಚಾರಿತ್ರನುಂ
ಚಾರಿತ್ರ-ಲಕ್ಷ್ಮೀ-ಕರ್ಣ್ನಪೂರಂ
ಚಾರಿತ್ರಿಕ
ಚಾರಿತ್ರ್ಯ
ಚಾರು
ಚಾರು-ಚರಿತ್ರ
ಚಾರು-ಪೊನ್ನೇರ
ಚಾರ್ಯ
ಚಾರ್ವಾಕ
ಚಾಲಿತ-ವಾದ
ಚಾಲುಕ್ಯ
ಚಾಲುಕ್ಯರ
ಚಾಲುಕ್ಯ-ರನ್ನು
ಚಾಲುಕ್ಯ-ರಿಗೆ
ಚಾಲುಕ್ಯರು
ಚಾಲ್ತಿ-ಯಲ್ಲಿತ್ತು
ಚಾಳಿಸಿ
ಚಾಳುಕ್ಯ
ಚಾಳುಕ್ಯ-ರನ್ನೂ
ಚಾಳುಕ್ಯರು
ಚಾಳುಕ್ಯ-ವಿಕ್ರಮ
ಚಾವ-ಗೌಂಡ
ಚಾವಡಿ
ಚಾವಡಿ-ಗ-ಳನ್ನಾಗಿ
ಚಾವಡಿಯ
ಚಾವಡಿ-ಯಲ್ಲಿ
ಚಾವಡಿ-ಯಾಗಿರ
ಚಾವಡಿಯು
ಚಾವಣ-ನೆಂಬ
ಚಾವಣಿಗೆ
ಚಾವಣಿ-ಯಲ್ಲಿ-ರುವ
ಚಾವಯ್ಯ
ಚಾವಲ-ದೇವಿ
ಚಾವಾಟ
ಚಾವುಂಡ
ಚಾವುಂಡ-ರಾಜ
ಚಾವುಂಡ-ರಾಜನ
ಚಾವುಂಡ-ರಾಜ-ನೆಂಬು-ವ-ವನು
ಚಾವುಂಡ-ರಾಯ-ಚಾ-ಮುಂಡ-ರಾಯ
ಚಾವುಂಡ-ರಾಯನ
ಚಾವುಂಡವ್ವೆ
ಚಾವುಂಡವ್ವೆ-ಯರ
ಚಾವುಣ್ಡ
ಚಾವುಣ್ಡನು
ಚಾವುಣ್ಡ-ನೆಂಬ
ಚಾವುಣ್ಡಬ್ಬ-ರಸಿ-ಯ-ಮ-ಗಳು
ಚಿಂಣ್ನಂ
ಚಿಂತಾ-ಮಣಿ
ಚಿಂತಾ-ಮಣಿ-ಯಲ್ಲಿ
ಚಿಂತಿಸಿ-ದ-ನೆಂದು
ಚಿಂನ-ವು-ಳಿ-ಯದೆ
ಚಿಂನ್ನ
ಚಿಂನ್ಮೂರ್ತ್ತಿ
ಚಿಂಮತೂರು
ಚಿಂಮತ್ತೂರು
ಚಿಕಂಣೈಯ
ಚಿಕಆ-ರಸ-ನಾಯ್ಕ-ರಿಗೆ
ಚಿಕ-ಒಡೆ-ಯನು
ಚಿಕ-ಕಂನೆಯ-ನ-ಹಳ್ಳಿ
ಚಿಕ-ಕೇತಯ
ಚಿಕ-ಗವುಂಡನು
ಚಿಕ-ಗವುಡ
ಚಿಕ-ದೇವ-ರಾಜ
ಚಿಕ-ದೇವ-ರಾಜನ
ಚಿಕ-ದೇವ-ರಾಜ-ನ-ವರೆಗೆ
ಚಿಕ-ದೇವ-ರಾಜನು
ಚಿಕ-ದೇವ-ರಾಜರ
ಚಿಕ-ದೇವ-ರಾಜ-ವಿಜಯ-ದಲ್ಲಿ
ಚಿಕ-ದೇವ-ರಾಯ-ನೃಪತೀ
ಚಿಕ-ದೇವ-ರಾಯ-ವಂಶಾ-ವಳಿ-ಯಲ್ಲೂ
ಚಿಕನ-ಹಳ್ಳಿ
ಚಿಕ-ಬಾಚೆಯ
ಚಿಕ-ಬಾಚೆಯನ
ಚಿಕ-ಮಾ-ದವ್ವೆ
ಚಿಕ-ಮಾಯಿ-ನಾಯ-ಕನ
ಚಿಕ-ರಾಜನ
ಚಿಕವಂಗಲ
ಚಿಕವಂಗಲ-ವಕ್ಕೆ
ಚಿಕವಂಗಲವು
ಚಿಕವಡೆಯ
ಚಿಕ-ವಡೆ-ಯರು
ಚಿಕ-ವೀರ-ಗವುಡ
ಚಿಕ-ಸಿಂಗ-ರಾಯಗೆ
ಚಿಕ-ಸಿದ್ಧಯ್ಯ-ಗವುಡ
ಚಿಕ-ಹೆಬ್ಬಾ-ಗಿಲು
ಚಿಕ್ಕ
ಚಿಕ್ಕಂದಿ-ನಲ್ಲಿ
ಚಿಕ್ಕ-ಅಠವ-ಣೆಯ
ಚಿಕ್ಕ-ಅಬ್ಬಾ-ಗಿಲಿನ
ಚಿಕ್ಕ-ಅಬ್ಬಾ-ಗಿಲು
ಚಿಕ್ಕ-ಅಮ್ಮ
ಚಿಕ್ಕ-ಅರ-ಸಿ-ನ-ಕೆರೆ
ಚಿಕ್ಕ-ಅರ-ಸಿ-ನ-ಕೆರೆಯ
ಚಿಕ್ಕ-ಅಲ್ಲಪ್ಪ
ಚಿಕ್ಕ-ಅಲ್ಲಪ್ಪ-ನಾಯಕ
ಚಿಕ್ಕ-ಅಲ್ಲಪ್ಪ-ನಾಯ-ಕನು
ಚಿಕ್ಕ-ಅಲ್ಲಪ್ಪ-ನಾಯ-ಕರ
ಚಿಕ್ಕ-ಕಂನೆಯ
ಚಿಕ್ಕ-ಕಂನೆಯ-ನ-ಹಳ್ಳಿ-ಯನ್ನು
ಚಿಕ್ಕ-ಕಂಪಣ್ಣನು
ಚಿಕ್ಕ-ಕಟ್ಟ-ಣ-ಗೆರೆ-ಗಳ
ಚಿಕ್ಕ-ಕಟ್ಟ-ಣ-ಗೆರೆಯ
ಚಿಕ್ಕ-ಕ-ಪಯ್ಯ-ನ-ವರು
ಚಿಕ್ಕ-ಕಳಲೆ
ಚಿಕ್ಕ-ಕೃಷ್ಣ-ರಾಜ
ಚಿಕ್ಕ-ಕೆರೆ-ಗ-ಳನ್ನು
ಚಿಕ್ಕ-ಕೇತಣ್ಣ
ಚಿಕ್ಕ-ಕೇತಯ
ಚಿಕ್ಕ-ಕೇತ-ಯನು
ಚಿಕ್ಕ-ಕೇತಯ್ಯ
ಚಿಕ್ಕ-ಕೇತಯ್ಯ-ದಂಡ-ನಾಯಕ
ಚಿಕ್ಕ-ಕೇತಯ್ಯನು
ಚಿಕ್ಕ-ಕೇತಯ್ಯನೇ
ಚಿಕ್ಕ-ಕೇತೆಯ
ಚಿಕ್ಕ-ಕೇತೆಯನು
ಚಿಕ್ಕ-ಕೇತೆಯ್ಯ-ನಾಯ-ಕನು
ಚಿಕ್ಕ-ಗಂಗ-ವಾಡಿ
ಚಿಕ್ಕ-ಗಂಗ-ವಾಡಿಯ
ಚಿಕ್ಕ-ಗಂಡಸಿ
ಚಿಕ್ಕ-ಗರು-ಡ-ನ-ಹಳ್ಳಿ
ಚಿಕ್ಕ-ಗರು-ಡ-ನ-ಹಳ್ಳಿಯ
ಚಿಕ್ಕ-ಗ-ವುಡನು
ಚಿಕ್ಕ-ಗೊಂಡನ
ಚಿಕ್ಕಗ್ರಾಮ-ಗ-ಳಿಗೆ
ಚಿಕ್ಕ-ಜಟಕ
ಚಿಕ್ಕ-ಜಟ್ಟಿಗ-ಹಳ್ಳಿ-ಇಂದಿನ
ಚಿಕ್ಕ-ಜೀಯನು
ಚಿಕ್ಕ-ಜೀಯ-ನೆಂಬು-ವ-ವನು
ಚಿಕ್ಕ-ತಿರುನಾಳ್
ಚಿಕ್ಕ-ದಾಗಿ
ಚಿಕ್ಕ-ದಾ-ಗಿದ್ದು
ಚಿಕ್ಕ-ದಾದ
ಚಿಕ್ಕ-ದೇವ
ಚಿಕ್ಕ-ದೇವ-ರಾಜ
ಚಿಕ್ಕ-ದೇವ-ರಾಜನ
ಚಿಕ್ಕ-ದೇವ-ರಾಜನು
ಚಿಕ್ಕ-ದೇವ-ರಾಜರ
ಚಿಕ್ಕ-ದೇವ-ರಾಜೇಂದ್ರ
ಚಿಕ್ಕ-ದೇವ-ರಾಯನ
ಚಿಕ್ಕ-ದೇವ-ರಾಯ-ನಿ-ಗಿಂತ
ಚಿಕ್ಕ-ದೇವೇಂದ್ರ
ಚಿಕ್ಕ-ದೇವೇಂದ್ರನು
ಚಿಕ್ಕ-ನ-ಹಳ್ಳಿ
ಚಿಕ್ಕ-ನ-ಹಳ್ಳಿಗೆ
ಚಿಕ್ಕ-ನಾಯ-ಕ-ನ-ಪುರ-ದಿಂದ
ಚಿಕ್ಕ-ನಾಯ-ಕ-ನ-ಹಳ್ಳಿ
ಚಿಕ್ಕ-ನಾಯ-ಕರು
ಚಿಕ್ಕಪ್ಪ
ಚಿಕ್ಕಪ್ಪಂದಿರು
ಚಿಕ್ಕಪ್ಪ-ಗೌಡನ
ಚಿಕ್ಕಪ್ಪನ
ಚಿಕ್ಕಪ್ಪ-ನ-ಹಳ್ಳಿ
ಚಿಕ್ಕಪ್ಪ-ನಿಂದ
ಚಿಕ್ಕಪ್ರ-ದೇಶದ
ಚಿಕ್ಕ-ಬಯಿಚಪ್ಪ
ಚಿಕ್ಕ-ಬಳ್ಳಾ-ಪುರ
ಚಿಕ್ಕ-ಬಳ್ಳಿ
ಚಿಕ್ಕ-ಬಾ-ಗಿಲ
ಚಿಕ್ಕ-ಬಾ-ಗಿಲು
ಚಿಕ್ಕ-ಬಾಚೆಯನು
ಚಿಕ್ಕ-ಬೆಟ್ಟಕ್ಕೆ
ಚಿಕ್ಕ-ಬೆಟ್ಟ-ಜಿನ-ಗುಡ್ಡ
ಚಿಕ್ಕ-ಬೆಟ್ಟದ
ಚಿಕ್ಕ-ಬೆಟ್ಟ-ದಲ್ಲಿ
ಚಿಕ್ಕ-ಬೆಟ್ಟ-ದಲ್ಲಿ-ರುವ
ಚಿಕ್ಕ-ಬೆಳೂರ
ಚಿಕ್ಕ-ಬೆಳೂ-ರನ್ನು
ಚಿಕ್ಕ-ಬೆಳೂರು
ಚಿಕ್ಕಬ್ಬೆ-ಹಳ್ಳಿ
ಚಿಕ್ಕಬ್ಬೆ-ಹಳ್ಳಿಗೆ
ಚಿಕ್ಕಬ್ಬೆ-ಹಳ್ಳಿ-ಯನ್ನು
ಚಿಕ್ಕಬ್ಬೆ-ಹಳ್ಳಿ-ಯು-ಚಿಕ್ಕ-ಬಳ್ಳಿ
ಚಿಕ್ಕ-ಮಂಟೆ-ಯಕ್ಕೆ
ಚಿಕ್ಕ-ಮಂಟೆಯ-ಚಿಕ್ಕ-ಮಂಡ್ಯ
ಚಿಕ್ಕ-ಮಂಟೆ-ಯದ
ಚಿಕ್ಕ-ಮಂಠೆ-ಯ-ಚಿಕ್ಕ-ಮಂಡ್ಯ
ಚಿಕ್ಕ-ಮಂಠೆ-ಯವೇ
ಚಿಕ್ಕ-ಮಂಡ್ಯ
ಚಿಕ್ಕ-ಮ-ಗಳೂರು
ಚಿಕ್ಕ-ಮರಲಿ
ಚಿಕ್ಕ-ಮರಳಿ
ಚಿಕ್ಕ-ಮಲ್ಲ-ನಾಯ-ಕನ
ಚಿಕ್ಕ-ಮಲ್ಲ-ನಿಗೆ
ಚಿಕ್ಕ-ಮಲ್ಲಯ್ಯ-ನಾಯ-ಕನ
ಚಿಕ್ಕ-ಮಲ್ಲೆ-ಯ-ನಾಯ-ಕನ
ಚಿಕ್ಕ-ಮಲ್ಲೆ-ಯ-ನಾಯ-ಕನು
ಚಿಕ್ಕ-ಮಲ್ಲೆ-ಯ-ನಾಯ-ಕ-ನೆಂಬ
ಚಿಕ್ಕ-ಮಳಲಿ
ಚಿಕ್ಕ-ಮಾಯಿ
ಚಿಕ್ಕ-ಯಗಟಿ
ಚಿಕ್ಕಯ್ಯನ
ಚಿಕ್ಕಯ್ಯನ-ಕೊಳ-ವೆಂದು
ಚಿಕ್ಕಯ್ಯನ-ವರು
ಚಿಕ್ಕ-ರಸ
ಚಿಕ್ಕ-ರ-ಸಿನ-ಕೆರೆಯ
ಚಿಕ್ಕ-ರಾಜ
ಚಿಕ್ಕ-ರಾಜನ
ಚಿಕ್ಕ-ರಾಮ-ರಾಜನು
ಚಿಕ್ಕ-ರಾಯ
ಚಿಕ್ಕ-ರಾಯನ
ಚಿಕ್ಕ-ರಾಯ-ನಿಗೆ
ಚಿಕ್ಕ-ರಾಯನು
ಚಿಕ್ಕ-ರಾಯನೂ
ಚಿಕ್ಕ-ರಾಯ-ಪಟ್ಟಣ
ಚಿಕ್ಕ-ರಾಯ-ಪಟ್ಟ-ಣ-ವನಾ-ಳುವ
ಚಿಕ್ಕ-ರಾಯ-ಪಟ್ಟ-ಣ-ವನ್ನಾಳುತ್ತಿದ್ದ-ನೆಂದು
ಚಿಕ್ಕ-ರಾಯ-ಪಟ್ಟವೇ
ಚಿಕ್ಕ-ರಾಯ-ಪುರ-ವೆಂಬ
ಚಿಕ್ಕ-ರಾಯಪ್ಪ-ನವ-ರಿಗೆ
ಚಿಕ್ಕ-ರಾಯರು
ಚಿಕ್ಕ-ರಾಯ-ಸಾ-ಗರ
ಚಿಕ್ಕ-ಲಿಂಗ-ನ-ಕೊಪ್ಪಲು
ಚಿಕ್ಕ-ವಡ್ಡ-ರ-ಗುಡಿ
ಚಿಕ್ಕ-ವನ-ಹಳ್ಳಿ
ಚಿಕ್ಕ-ವ-ನಾಗಿ-ರು-ವಾಗಲೇ
ಚಿಕ್ಕ-ವೊಡೆಯ
ಚಿಕ್ಕ-ವೊಡೆ-ಯ-ನೆಂಬ
ಚಿಕ್ಕ-ವೋಡೆ
ಚಿಕ್ಕ-ಸಾದಿಪ್ಪ
ಚಿಕ್ಕ-ಸಾದಿಪ್ಪ-ನಿಗೆ
ಚಿಕ್ಕ-ಸಾದಿಪ್ಪನು
ಚಿಕ್ಕ-ಸಾದಿ-ಯಪ್ಪ-ನಿಗೆ
ಚಿಕ್ಕ-ಸಾದಿ-ಯಪ್ಪನೂ
ಚಿಕ್ಕ-ಸಾಧಿ-ಯಪ್ಪ-ನವ-ರಿಗೆ
ಚಿಕ್ಕ-ಸಿಂಗ-ರಾಯ-ನಿಗೆ
ಚಿಕ್ಕ-ಸಿಂಗ-ರಾಯರ
ಚಿಕ್ಕ-ಹಡೆ-ವಳನ
ಚಿಕ್ಕ-ಹಡೆ-ವಳ್ಳ
ಚಿಕ್ಕ-ಹನ-ಸೋಗೆ
ಚಿಕ್ಕ-ಹರಿ-ಯಲೆ
ಚಿಕ್ಕ-ಹೊಸ-ಹಳ್ಳಿ
ಚಿಕ್ಕಾ-ಚಾರಿ
ಚಿಕ್ಕಾಡೆ
ಚಿಕ್ಕಿಯ
ಚಿಕ್ಕಿ-ಯರ
ಚಿಕ್ಕೆಯ-ನಾಯಕ
ಚಿಕ್ಕೆಯ-ನಾಯ-ಕನ
ಚಿಕ್ಕೆ-ಹಳ್ಳಿ-ಮಂಡ್ಯ
ಚಿಕ್ಕೇ-ಹಳ್ಳಿ
ಚಿಕ್ಕೊಡೆಯ-ನಿಗೆ
ಚಿಕ್ಕೊಡೆ-ಯರ
ಚಿಕ್ಕೊಲೆ
ಚಿಕ್ಕೋ-ಜನ-ಕಟ್ಟೆ
ಚಿಗುಡ-ಹಳ್ಳಿ
ಚಿಗುಡ-ಹಳ್ಳಿ-ತಿಗಡ-ಹಳ್ಳಿ
ಚಿಗುಲಿ-ಹಳ್ಳಿ
ಚಿಟ್ಟನ-ಪಲ್ಲಿ-ಚಿಟ್ನ-ಹಳ್ಳಿ
ಚಿಟ್ಟನ-ಹಳ್ಳಿ
ಚಿಟ್ಟನ-ಹಳ್ಳಿಯ
ಚಿಟ್ಟಿಗ-ವಡೆ-ಯರ
ಚಿಟ್ಟಿ-ಗವು-ಡಿಯ
ಚಿಣ್ಣ
ಚಿಣ್ಣ-ನನ್ನು
ಚಿಣ್ಣನು
ಚಿಣ್ಣಯ್ಯ
ಚಿಣ್ಣಯ್ಯನು
ಚಿಣ್ನಂ
ಚಿಣ್ನನು
ಚಿಣ್ನಯ
ಚಿಣ್ನಯ-ನಾಡ
ಚಿತಯಿಸಿ
ಚಿತೆಯೇರಿ
ಚಿತೆ-ಯೇ-ರುವ
ಚಿತ್ತಮಂ
ಚಿತ್ತ-ವಲ್ಲ-ಭೆ-ಯಾದ
ಚಿತ್ತಿರೈ
ಚಿತ್ತೈಸಿ
ಚಿತ್ರ-ಕರ್ಮ್ಮ
ಚಿತ್ರಕ-ಲಾಭಿಜ್ಞೇನ
ಚಿತ್ರ-ಕಾರರೂ
ಚಿತ್ರ-ಕೊಂಡ-ಮ-ನಾಯ-ಕರ
ಚಿತ್ರಣ
ಚಿತ್ರಣ-ವನ್ನು
ಚಿತ್ರ-ದುರ್ಗ
ಚಿತ್ರ-ದುರ್ಗ-ದಲ್ಲಿ
ಚಿತ್ರ-ದುರ್ಗವು
ಚಿತ್ರಭಾನು
ಚಿತ್ರಮ-ಕೊಂಡ-ನಾಯ-ಕನ
ಚಿತ್ರ-ಮಾಸ-ದಲು
ಚಿತ್ರ-ವನ್ನು
ಚಿತ್ರ-ವಿದೆ
ಚಿತ್ರಿರೈ
ಚಿದಾನಂದ
ಚಿದಾನಂದ-ಮಲ್ಲಿ-ಕಾರ್ಜುನನ
ಚಿದಾನಂದ-ಮಲ್ಲಿ-ಕಾರ್ಜುನನು
ಚಿದಾನಂದ-ಮೂರ್ತಿ-ಯ-ವರ
ಚಿದಾನಂದ-ಮೂರ್ತಿ-ಯ-ವರು
ಚಿದಾನನ್ದಂ
ಚಿನ-ಕುರಳಿ
ಚಿನ-ಕುರ-ಳಿಗೆ
ಚಿನ-ಕುರ-ಳಿ-ಬೆಟ್ಟ
ಚಿನ-ಕುರ-ಳಿಯ
ಚಿನ-ಕುರ-ಳಿ-ಯಲ್ಲಿ
ಚಿನ್ನದ
ಚಿನ್ನ-ದೇವ-ಚೋಡ-ಮಹಾ-ಅರ-ಸನು
ಚಿನ್ನನ
ಚಿನ್ನಪ್ಪ
ಚಿನ್ನ-ಬೆಳ್ಳಿ
ಚಿನ್ನ-ಬೆಳ್ಳಿಯ
ಚಿನ್ನಮ
ಚಿನ್ನ-ಮಗ್ಗ-ದೆರೆ
ಚಿನ್ನಾ-ದೇವಿ
ಚಿನ್ನಾ-ದೇವಿ-ಪುರ
ಚಿನ್ನಾ-ದೇವಿಯ
ಚಿನ್ನಾ-ದೇವಿಯು
ಚಿನ್ನಾ-ದೇವಿಯೂ
ಚಿನ್ಮಯ-ಭಟ್ಟ
ಚಿಪ್ಪಿಗ
ಚಿಬ್ಬನೇ-ಕಟ್ಟೆ
ಚಿಮತೂರ-ಕಲ್ಲ
ಚಿಮತೂರಬೆ-ಮತೂರ
ಚಿಮತೂರಿನ
ಚಿಮತ್ತೂರ-ಕಲ್ಲ
ಚಿಮತ್ತೂರ-ಕಲ್ಲು
ಚಿಮ್ಮತ್ತ-ನ-ಕಲ್ಲು
ಚಿಮ್ಮತ್ತ-ನೂರಿನ
ಚಿಮ್ಮತ್ತೂರ-ಕಲ್ಲ
ಚಿಮ್ಮತ್ತೂರು
ಚಿರಋಣಿ-ಯಾಗಿದ್ದೇನೆ
ಚಿರತೆ
ಚಿರತೆ-ಗಳು
ಚಿರತೆ-ಗಳೂ
ಚಿರಸ್ಥಾಯಿ-ಯಾಗಿ
ಚಿರುತೊ-ಣೆಯಾಂಡ
ಚಿಲ-ಕುರ್ಲಿ-ಚಿನ-ಕುರಳಿ
ಚಿಲು-ಕುರ್ಲಿಚಿನ್ನ-ಕುರಳಿ
ಚಿಲ್ಲರೆ
ಚಿಶ್ರೀ-ನಿ-ವಾಸ-ರಾಜು
ಚಿಹ್ನೆ-ಗ-ಳನ್ನು
ಚಿಹ್ನೆ-ಗಳು
ಚಿಹ್ನೆ-ಯನ್ನು
ಚಿಹ್ನೆ-ಯಾಗಿ
ಚಿಹ್ನೆಯಾದ
ಚೀಟಿ-ಗ-ಳನ್ನು
ಚೀಣ್ಯ
ಚೀರಾಮ-ಶೆಟ್ಟಿ
ಚುಂಚನ
ಚುಂಚನ-ಕೋಟೆ
ಚುಂಚನ-ಕೋಟೆಯ
ಚುಂಚನ-ಗಿರಿ
ಚುಂಚನ-ಗಿರಿಯ
ಚುಂಚನ-ಗಿರಿಯು
ಚುಂಚನ-ಭಯಿ-ರವ
ಚುಂಚನ-ಹಳ್ಳಿ
ಚುಂಚನ-ಹಳ್ಳಿಗೆ
ಚುಂಚನ-ಹಳ್ಳಿಯ
ಚುಂಚನ-ಹಳ್ಳಿ-ಯನ್ನು
ಚುಕ್ಕಿ
ಚುತರ್ವೇದಿ
ಚುರುಳ-ಮದು
ಚೂಡಮ
ಚೂಡಮ-ದೇವರ
ಚೂಡಾ-ಮಣಿ
ಚೂಡಾ-ರತ್ನ
ಚೂಡಿ-ಕುಡುತ-ನಾಚ್ಚಾರ್
ಚೂಡಿ-ಕುಡುತ್ತ
ಚೂರ್ಣೀ-ಕರಿ-ಸಿದ
ಚೆಂಗಣಿ-ಗಿಲ
ಚೆಂಗಪ್ಪ
ಚೆಂಗ-ವಾಡಿಯ
ಚೆಂಗ-ವಾಡಿ-ಯನ್ನು
ಚೆಂಗ-ವಾಡಿ-ಯಲ್ಲಿ
ಚೆಂಗಾಳ್ವ
ಚೆಂಗಾಳ್ವನಂ
ಚೆಂಗಾಳ್ವ-ನಾಗಿ-ರ-ಬಹು-ದೆಂದು
ಚೆಂಗಾಳ್ವನು
ಚೆಂಗಾಳ್ವರ
ಚೆಂಗಾಳ್ವ-ರನ್ನು
ಚೆಂಗಾಳ್ವ-ರಿಗೂ
ಚೆಂಗಾಳ್ವರು
ಚೆಂಗಿ-ಕುಲ-ದ-ವರೇ
ಚೆಂಗಿರಿ-ಯನ್ನು
ಚೆಂಗುಂಟೈ
ಚೆಂಡಾಡಿದ-ನೆಂದು
ಚೆಂದಾ-ಪುರ-ವೆಂಬ
ಚೆಂನಪ್ಪಾಜಿ
ಚೆಂನಯ್ಯನ
ಚೆಂನ-ರಾಮ-ಸಾ-ಗರದ
ಚೆಂನ-ಶಂಕರ
ಚೆಂನಿ
ಚೆಂನಿ-ಸೆಟ್ಟಿಯ
ಚೆಂನಿ-ಸೆಟ್ಟಿ-ಯರ
ಚೆಂಬಿ
ಚೆಂಬಿನ
ಚೆಂಬೊಂಗಳಂ
ಚೆಕ್
ಚೆಟ್ಟ-ಹಳ್ಳಿ
ಚೆಟ್ಟಿ-ಪುತ್ರರು
ಚೆಟ್ಟಿಯ-ರನ್ನು
ಚೆಟ್ಟಿ-ಸೆಟ್ಟಿ-ಯರ
ಚೆನ್ನ-ಕೇಶವ
ಚೆನ್ನ-ಕೇಶವ-ಚೆನ್ನಿಗ-ರಾಯ
ಚೆನ್ನ-ಕೇಶವ-ದೇವರ
ಚೆನ್ನ-ಕೇಶವ-ದೇವಾ-ಲ-ಯದ
ಚೆನ್ನ-ಕೇಶವನ
ಚೆನ್ನ-ಕೇಶವ-ನಾ-ರಾಯಣ
ಚೆನ್ನ-ಕೇಶವ-ನಿಗೆ
ಚೆನ್ನ-ಕೇಶವ-ಪದಾಂಭೋಜ-ಕಮಳಿನೀ-ಕಳ-ಹಂಸ-ಭಿನವಪ್ರಹ-ರಾಜ
ಚೆನ್ನ-ಕೇಶವ-ಪುರ-ವೆಂಬ
ಚೆನ್ನಕೇಶ್ವರ-ಚೆನ್ನ-ಕೇಶವ
ಚೆನ್ನಕ್ಕ
ಚೆನ್ನ-ದೀಕ್ಷಿತ-ನಿಗೆ
ಚೆನ್ನ-ದೇವ
ಚೆನ್ನ-ದೇವ-ಚೋಡ-ಮಹಾ-ಅರ-ಸನು
ಚೆನ್ನ-ದೇವ-ಚೋಡ-ಮಹಾ-ಅರಸು
ಚೆನ್ನ-ನಂಜ-ರಾಜ
ಚೆನ್ನಪ್ಪಾಜಿ
ಚೆನ್ನಮ್ಮನ್ನು
ಚೆನ್ನಯ್ಯ-ನೆಂಬು-ವ-ವನ
ಚೆನ್ನ-ರಸ
ಚೆನ್ನ-ರ-ಸನು
ಚೆನ್ನ-ರಾಜಯ್ಯನು
ಚೆನ್ನ-ರಾಯ-ಪಟ್ಟಣ
ಚೆನ್ನ-ರಾಯ್ಯ-ನವ-ರಿಗೆ
ಚೆನ್ನ-ವೀರಯ್ಯ-ಗೌಡ-ನ-ವರ
ಚೆನ್ನಾಗಿ
ಚೆನ್ನಾ-ಗಿದ್ದು
ಚೆನ್ನಾ-ದೇವಿ-ಪುರ-ವೆಂಬ
ಚೆನ್ನಿಗ-ರಾಯನ
ಚೆನ್ನಿಗ-ರಾಯಸ್ವಾಮಿ
ಚೆನ್ನಿ-ಸೆಟ್ಟಿ
ಚೆನ್ನಿ-ಸೆಟ್ಟಿ-ಯರ
ಚೆನ್ನಿ-ಸೆಟ್ಟಿ-ಯಿಂದಲೇ
ಚೆಬ್ರೋಲು
ಚೆರುಪಿಗೆ
ಚೆರುಪು
ಚೆಲಪಿಳೆ
ಚೆಲಪಿಳೆ-ರಾಯರ
ಚೆಲ-ಪಿಳ್ಳೆ-ರಾಯ
ಚೆಲುಪಿಳೆ-ರಾಯರ
ಚೆಲು-ಪಿಳ್ಳೆ-ದೇವರ
ಚೆಲು-ಪಿಳ್ಳೆ-ರಾಯರ
ಚೆಲುವ
ಚೆಲುವ-ದೇವಾಂಬುದಿ
ಚೆಲುವ-ದೇವಾಂಬುಧಿ
ಚೆಲುವ-ದೇವಾ-ಜ-ಮಾಂಬ
ಚೆಲುವ-ನಾ-ರಾಯಣ
ಚೆಲುವ-ನಾ-ರಾಯ-ಣನ
ಚೆಲುವ-ನಾ-ರಾಯ-ಣಸ್ವಾಮಿ
ಚೆಲುವ-ನಾ-ರಾಯ-ಣಸ್ವಾಮಿ-ಯ-ವರ
ಚೆಲುವ-ನಾ-ರಾಯಸ್ವಾಮಿ
ಚೆಲು-ವನು
ಚೆಲುವ-ಪಿಳ್ಳೆ
ಚೆಲುವ-ಪಿಳ್ಳೆ-ದೇವ-ರಿಗೆ
ಚೆಲುವ-ಪಿಳ್ಳೆಯ
ಚೆಲುವ-ಪಿಳ್ಳೆ-ರಾಯರ
ಚೆಲುವ-ಪಿಳ್ಳೆ-ರಾಯ-ರಿಗೆ
ಚೆಲುವ-ರಾಯಸ್ವಾಮಿ
ಚೆಲುವ-ರಾಯಸ್ವಾಮಿ-ಯ-ವರ
ಚೆಲುವವ್ವೆಯ
ಚೆಲುವವ್ವೆಯರ
ಚೆಲುವಾಂಬಾ
ಚೆಲುವಾಂಬಾ-ದೇವಿ-ಚೆಲುವಾ-ಜಮ್ಮಣ್ಣಿನೇ
ಚೆಲುವಾ-ಜ-ಮಾಂಬ
ಚೆಲುವಾ-ಜ-ಮಾಂಬ-ಳಿಗೆ
ಚೆಲುವಾ-ಜಮ್ಮ
ಚೆಲ್ಲ-ಪಿಳ್ಳೆ-ರಾಯ-ರಿಗೆ
ಚೆಲ್ಲಪ್ಪ
ಚೆಲ್ಲ-ಬಹು-ದಾಗಿದೆ
ಚೆಲ್ಲಾಪಿಲ್ಲಿ-ಯಾಗಿ
ಚೆಲ್ಲುತ್ತದೆ
ಚೆಲ್ಲೇಶ್ವರದ
ಚೆಲ್ವಡ-ರಾಯ-ನೆಂಬ
ಚೆಲ್ವ-ಪಿಳ್ಳೆ
ಚೆಲ್ವ-ಪಿಳ್ಳೆ-ರಾಯರ
ಚೆಲ್ವ-ಪುಳ್ಳೆ-ಯ-ವರು
ಚೆಲ್ವಾ-ಜ-ಮಾಂಬ
ಚೆಲ್ವಾಜ-ಮಾಂಬೆಯ
ಚೆಲ್ವು
ಚೆಲ್ವುಳ್ಳುದು
ಚೆಲ್ವೊಡ-ರಾಯ
ಚೆಳೆಯ-ಕಟ್ಟೆ
ಚೇತಸೇ
ಚೇತಿ
ಚೇತ್ಕಥಂ
ಚೇರ
ಚೇರ-ಮನ-ಹಳ್ಳಿ-ಯನ್ನು
ಚೇವ-ಗರ್ಸೇವಕ-ರು-ಗ-ಳನ್ನು
ಚೈತ್ಯಾಲಯ
ಚೈತ್ಯಾಲ-ಯ-ಗಳಂ
ಚೈತ್ಯಾ-ಲ-ಯದ
ಚೈತ್ಯಾಲ-ಯ-ವನ್ನು
ಚೈತ್ಯಾಲ-ಯ-ವೆಂದೂ
ಚೈತ್ಯಾಲ-ವನ್ನು
ಚೈತ್ರ
ಚೈತ್ರ-ಪವಿತ್ರ
ಚೊಕ-ಚಾರ್ಯ್ಯ-ಕಟ್ಟೆ
ಚೊಕ್ಕ-ಚಾರ್ಯ-ಕಟ್ಟೆ
ಚೊಕ್ಕ-ಜಿನಾ-ಲ-ಯಕ್ಕೆ
ಚೊಕ್ಕಣ್ಣ
ಚೊಕ್ಕಣ್ಣನ
ಚೊಕ್ಕ-ನಾಥನ
ಚೊಕ್ಕ-ಪೆರು-ಮಾಳ್
ಚೊಕ್ಕಾಣ್ಡೈ
ಚೊಟ್ಟನ-ಹಳ್ಳಿ
ಚೊಟ್ಟನ-ಹಳ್ಳಿಯ
ಚೊಟ್ಟನ-ಹಳ್ಳಿ-ಯಲ್ಲಿ
ಚೊತ್ತರಳಿ
ಚೊಳ-ನಿಗೆ
ಚೋಕರ-ಕನು-ಮೊ-ದಲಿ-ಯಣ್ಣನ
ಚೋಕಲ
ಚೋಕಲ-ದೇವಿ
ಚೋಕಲ-ದೇ-ವಿಗೆ
ಚೋಕವ್ವೆ
ಚೋಡ-ಮಹಾ-ಅರ-ಸನು
ಚೋಳ
ಚೋಳ-ಕೊಟ್ಟ
ಚೋಳ-ಗಂಗ
ಚೋಳ-ಗಉಂಡ
ಚೋಳ-ಗವುಂಡನು
ಚೋಳ-ಗವುಡ
ಚೋಳ-ಗಾವುಂಡನ
ಚೋಳ-ಗಾವುಂಡನು
ಚೋಳ-ಗೌಂಡನು
ಚೋಳ-ಚತುರ್ವೇದಿ
ಚೋಳ-ತುರು-ನಾಡನ್ನು
ಚೋಳ-ದೇಶ-ದಲ್ಲಿ
ಚೋಳ-ದೇ-ಶ-ದಿಂದ
ಚೋಳನ
ಚೋಳ-ನ-ಕೋಟೆ
ಚೋಳ-ನನ್ನು
ಚೋಳ-ನಾಡನ್ನಾಗಿ
ಚೋಳ-ನಾಡನ್ನು
ಚೋಳ-ನಿಗೆ
ಚೋಳನು
ಚೋಳನೆ
ಚೋಳ-ನೆಂಬ
ಚೋಳ-ಪಯ್ಯನ
ಚೋಳ-ಪರಾಂತಕ
ಚೋಳ-ಪರಾಂತ-ಕನ
ಚೋಳ-ಪುರದ
ಚೋಳಪ್ಪಯ್ಯನ
ಚೋಳ-ಬಲ-ವನ್ನು
ಚೋಳ-ಭೂಮಿಯ
ಚೋಳ-ಮಂಡ-ಲದ
ಚೋಳ-ಮಹಾ-ಅರಸ
ಚೋಳ-ಮಹಾ-ಅರ-ಸು-ಗಳ
ಚೋಳ-ಯನ-ಹಳ್ಳಿಯ
ಚೋಳರ
ಚೋಳ-ರ-ಕಾಲದ
ಚೋಳ-ರನ್ನು
ಚೋಳ-ರಾಜ-ನನ್ನು
ಚೋಳ-ರಾಜನೇ
ಚೋಳ-ರಾಜೇಂದ್ರನು
ಚೋಳ-ರಾಜ್ಯ
ಚೋಳ-ರಾಜ್ಯಕ್ಕೆ
ಚೋಳ-ರಾಜ್ಯದ
ಚೋಳ-ರಾಜ್ಯ-ದಲ್ಲಿ
ಚೋಳ-ರಾಯ
ಚೋಳ-ರಾಯಸ್ಥಾ-ಪನಾ-ಚಾರ್ಯ್ಯ
ಚೋಳ-ರಿಂದ
ಚೋಳ-ರಿಗೂ
ಚೋಳರು
ಚೋಳ-ರೊಡನೆ
ಚೋಳ-ಲಾಳಾದಿ-ಗಳು
ಚೋಳ-ಸಿಂಹಾಸ-ನದ
ಚೋಳ-ಸಿಂಹಾಸ-ನ-ದಲ್ಲಿ
ಚೋಳ-ಸೇನೆ-ಯನ್ನು
ಚೋಳಿ
ಚೋಳಿ-ಕ-ರನ್ನು
ಚೋಳಿಗ
ಚೋಳೆ-ಯನ-ಹಳ್ಳಿ
ಚೋಳೇಂದ್ರ
ಚೋಳೋರ್ವ್ವಿಯಂ
ಚೌಂಡಾಡಿ
ಚೌಕಟ್ಟನ್ನು
ಚೌಕಿ-ಮಠದ
ಚೌಗಾವೆಯ
ಚೌಡಪ್ಪನ
ಚೌಡಮ್ಮನ
ಚೌಡಯ್ಯ
ಚೌಡಯ್ಯ-ನ-ಹಳ್ಳಿಯ
ಚೌಡ-ಹಳ್ಳಿ-ಯಲ್ಲಿ-ರುವ
ಚೌಡೇಶ್ವರಿ
ಚೌತ
ಚೌತನ್ನು
ಚೌತ-ವನ್ನು
ಚ್ಚಂಗಿ
ಚ್ಚರಿ-ಸುವ-ರೆಲ್ಲರುಂ
ಚ್ಚಲ್ಲಕ್ಕಡ-ವಿದಾಗ
ಚ್ಛ್ರೋತ್ರಿಯಾನ್ವೇದ
ಚ್ಯಂನಪ್ಪನು
ಛಂದಸ್ಸಿನ
ಛಂದಸ್ಸು
ಛತ್ರ
ಛತ್ರ-ಗಳ
ಛತ್ರ-ಗ-ಳನ್ನು
ಛತ್ರ-ಛಾಯೆಯಿಂ
ಛತ್ರದ
ಛತ್ರ-ದಲ್ಲಿ
ಛತ್ರ-ವನ್ನು
ಛತ್ರ-ಶು-ವರ್ಣ-ಮತ್ಸ್ಯ-ಮ-ಕರಾ-ಕಾರಧ್ವಜೈಶ್ಚಿನ್ಹಿತ
ಛತ್ರಿ-ಯನ್ನು
ಛಾತ್ರಸ್ಯಾಪಿ
ಛಾಯಾಗ್ರಾಹಕ-ನಲ್ಲ
ಛಾಯಾಚಿತ್ರ-ಗ-ಳನ್ನು
ಛಾಯಾಚಿತ್ರ-ಗ-ಳಿಂದ
ಛಾಯೆ-ಯನ್ನು
ಛಾಯೆ-ಯಿಂದ
ಜಂಗ
ಜಂಗಮ
ಜಂಗ-ಮರು
ಜಂಗ-ಮ-ವರ್ಗಕ್ಕೆ
ಜಂಗ-ಮೊಡೆ-ಯರ
ಜಂಗುಳಿ
ಜಂಗುಳಿ-ಯ-ವರು
ಜಂಗ್ಗಮ
ಜಂಟಿ
ಜಂಟಿ-ಯಾಗಿ
ಜಂನಿಕೂಚಿ-ಗಳ
ಜಂನೈಯ್ಯಂಗಾರ್
ಜಂನ್ನೇಶ್ವರ
ಜಂಬು
ಜಂಬೂರು
ಜಕ-ಗೌಡನ
ಜಕವ್ವೆ
ಜಕ್ಕಂಣ
ಜಕ್ಕಣಬ್ಬೆ
ಜಕ್ಕಣಬ್ಬೆಯ
ಜಕ್ಕಣಬ್ಬೆ-ಯರ
ಜಕ್ಕಣ್ಣ-ನಾಯ-ಕ-ನಿಗೆ
ಜಕ್ಕನ-ಹಳ್ಳಿ
ಜಕ್ಕಯ್ಯ
ಜಕ್ಕಯ್ಯ-ನಾಯಕ
ಜಕ್ಕಲ-ದೇವಿ
ಜಕ್ಕಲೆ
ಜಕ್ಕಲೆಗೆ
ಜಕ್ಕವ್ವೆ
ಜಕ್ಕವ್ವೆಯೇ
ಜಕ್ಕಿ-ಕಟ್ಟೆ
ಜಕ್ಕಿಮವ್ವೆಯು
ಜಕ್ಕಿಯಕ್ಕ-ಜಕ್ಕಲೆ
ಜಕ್ಕಿ-ಸೆಟ್ಟಿ
ಜಕ್ಕಿ-ಸೆಟ್ಟಿಯ
ಜಕ್ಕಿ-ಸೆಟ್ಟಿಯು
ಜಗ
ಜಗ-ಕಾರಿಗ
ಜಗ-ಗೌಡನ
ಜಗತಿಕಟೆಯ
ಜಗ-ತಿಗೆ
ಜಗ-ತಿಯ
ಜಗ-ತಿಯ-ಮೇ-ಲಿದೆ
ಜಗ-ತಿಯ-ಮೇಲೆ
ಜಗತೀ-ಸಾಮ್ರಾಜ್ಯ-ದೀಕ್ಷಾಂ
ಜಗತೇಶ್ವರ
ಜಗತ್ತಿನ
ಜಗತ್ತು-ಗಳೆಂಬ
ಜಗದ-ಳೆಯ-ರೆನಿಸಿ
ಜಗದಾಳ-ಮೊನೆ-ಯೊಳು
ಜಗದು-ಗೋಪಾಳ
ಜಗದೇಕ
ಜಗದೇಕ-ನಾಯಕ
ಜಗದೇಕ-ಮಲ್ಲ
ಜಗದೇಕ-ಮಲ್ಲನು
ಜಗದೇಕ-ರಾಯ
ಜಗದೇಕ-ರಾಯನು
ಜಗದೇಕ-ವೀರನು
ಜಗದೇಕ-ವೊಡೆ-ಯನ
ಜಗ-ದೇವ
ಜಗ-ದೇವ-ಕಅ-ರಾಯ
ಜಗ-ದೇವನ
ಜಗ-ದೇವ-ರಾಯ
ಜಗ-ದೇವ-ರಾಯನ
ಜಗ-ದೇವ-ರಾಯ-ನಿಗೆ
ಜಗ-ದೇವ-ರಾಯನು
ಜಗ-ದೇವ-ರಾಯ-ನೆಂಬ
ಜಗದ್ಗುರು-ವಾಗಿದ್ದ-ನೆಂದು
ಜಗ-ನಾಯ-ಕನು
ಜಗನ್ನಾಥ-ವಿಜಯ-ವೆಂಬ
ಜಗನ್ಮೋ-ಹನ
ಜಗ-ಲಿಗೆ
ಜಗ-ಲಿಯ-ಮೇಲೂ
ಜಗಲ್ಲಲಾ-ಮಾಯಿತ
ಜಗಳ
ಜಗಳ-ಗ-ಳಲ್ಲಿ
ಜಗಳ-ದಲ್ಲಿ
ಜಗಳವು
ಜಗಳೂರು
ಜಗವೆಲ್ಲ
ಜಗುಲಿ
ಜಗ್ಗ-ಗ-ವುಡನ
ಜಟಕ
ಜಟಕಾ
ಜಟಾಜಾಳಕಂ
ಜಟಾ-ವರ್ಮ-ಸುಂದರ-ಪಾಂಡ್ಯನ
ಜಟ್ಟಿಗ-ವನ್ನು
ಜಡೆಯ
ಜಡೆ-ಯದ
ಜತ-ಗರ
ಜತೆ
ಜತ್ತ
ಜನ
ಜನಂಗಳು
ಜನ-ಉರ್ದ್ಧಿ
ಜನಕ
ಜನಕಂ
ಜನ-ಕನ
ಜನ-ಕಾತ್ಮಜೆಗಂ
ಜನ-ಗನ್ನು-ತನಾ
ಜನ-ಗಳ
ಜನ-ಗ-ಳಿಗೆ
ಜನ-ಗಳಿ-ಗೋಸ್ಕರ
ಜನ-ಗಳು
ಜನ-ಗಳೂ
ಜನ-ಜನಿತ-ವಾದ
ಜನ-ಜೀ-ವನ
ಜನ-ಜೀ-ವನಕ್ಕೆ
ಜನ-ಜೀ-ವನದ
ಜನ-ಜೀ-ವನವು
ಜನತಾಧಾ-ರನು-ದಾರ-ನನ್ಯ
ಜನತಾಪ್ರಿಯೇಣ
ಜನತೆ
ಜನ-ತೆಯ
ಜನತೆ-ಯನ್ನು
ಜನನ
ಜನ-ನದ
ಜನನ-ವಾಗಿದೆ
ಜನ-ನಾದ-ಪುರ-ಜ-ನಾರ್ದನ-ಪುರ
ಜನನಿ
ಜನನಿಯ
ಜನನೀ
ಜನ-ಪದ
ಜನ-ಪದ-ದಲ್ಲಿ
ಜನಪ್ರಿಯ
ಜನಪ್ರಿಯ-ವಾಯಿತು
ಜನ-ಬಳ-ಕೆಯ
ಜನ-ಬಳ-ಕೆ-ಯಲ್ಲಿತ್ತೆಂದು
ಜನ-ಮನ್ನಣೆ-ಗ-ಳನ್ನೂ
ಜನ-ಮಾನ-ಸ-ದಲ್ಲಿ
ಜನರ
ಜನ-ರಂತೆ
ಜನ-ರನ್ನು
ಜನರಲ್
ಜನರಲ್ಲಿ
ಜನ-ರಿಂದ
ಜನ-ರಿಗೆ
ಜನರು
ಜನರು-ಗಳು
ಜನರೂ
ಜನರೇ
ಜನ-ವರಿ
ಜನ-ವೃದ್ಧಿ
ಜನ-ಸಂಖ್ಯೆಯು
ಜನ-ಸ-ಮು-ದಾಯವೇ
ಜನ-ಸಮೂಹಕ್ಕೆ
ಜನ-ಸಾ-ಮಾನ್ಯ
ಜನ-ಸಾ-ಮಾನ್ಯರ
ಜನ-ಸಾ-ಮಾನ್ಯ-ರನ್ನು
ಜನ-ಸಾ-ಮಾನ್ಯ-ರಲ್ಲಿ
ಜನ-ಸಾ-ಮಾನ್ಯರು
ಜನ-ಸಾ-ಮಾನ್ಯ-ರು-ಗಳು
ಜನ-ಹಳ್ಳಿಯ
ಜನಾಂಗಕ್ಕೆ
ಜನಾಂಗ-ಗಳ
ಜನಾಂಗ-ಗಳು
ಜನಾಂಗದ
ಜನಾಂಗ-ದಲ್ಲಿ
ಜನಾಂಗದ-ವನು
ಜನಾಂಗದ-ವರ
ಜನಾಂಗದ-ವ-ರನ್ನು
ಜನಾಂಗದ-ವ-ರಿಗೆ
ಜನಾಂಗದ-ವರು
ಜನಾಂಗದ-ವ-ರೊಡನೆ
ಜನಾಂಗ-ವನ್ನು
ಜನಾಂಗ-ವಾಗಿದ್ದು-ದಲ್ಲದೆ
ಜನಾನು-ರಾಗಿ-ಯಾಗಿ-ರ-ಲಿಲ್ಲ
ಜನಾರಾಧ್ಯೆ
ಜನಾರ್ದನ
ಜನಾರ್ದನ-ದೇವರ
ಜನಿಸಿ
ಜನಿ-ಸಿದ
ಜನಿಸಿ-ದಂತೆ
ಜನಿಸಿ-ದ-ನಂತೆ
ಜನಿಸಿ-ದನು
ಜನಿಸಿ-ದ-ನೆಂದು
ಜನಿಸಿ-ದ-ನೆಂದೂ
ಜನಿಸಿ-ದರು
ಜನಿಸಿ-ದ-ರೆಂದು
ಜನಿಸಿ-ದ-ವರು
ಜನಿಸಿದ್ದರೂ
ಜನಿಸಿ-ಬಂದು
ಜನಿಸುತ್ತಾ-ನೆಂದು
ಜನಿ-ಸುತ್ತೇನೆ
ಜನೋ-ಪ-ಕಾರಿ
ಜನ್ನ
ಜನ್ನ-ಕವಿಯ
ಜನ್ನ-ಕವಿಯು
ಜನ್ನನ
ಜನ್ನಯ್ಯ-ದೀಕ್ಷಿತನ
ಜನ್ನೇಶ್ವರ
ಜನ್ನೈಯ್ಯಂಗಾರ್
ಜನ್ಮ
ಜನ್ಮಕ್ಷೇತ್ರವಂ
ಜನ್ಮಕ್ಷೇತ್ರ-ವನ್ನಾಗಿ
ಜನ್ಮತಃ
ಜನ್ಮ-ತಪಃ
ಜನ್ಮ-ನಕ್ಷತ್ರ-ವಾದ
ಜನ್ಮನೇ
ಜನ್ಮ-ಭೂಮಿ-ಯನ್ನಾಗಿ
ಜನ್ಮಸ್ಥಳವೋ
ಜನ್ಮೋತ್ಸವ-ದಂದು
ಜಪತ-ಪಹೋ-ಮ-ಗ-ಳನ್ನು
ಜಪಸ-ಮಾದಿ
ಜಪ-ಸಮಾಧಿ
ಜಪಸಮಾಧೀ
ಜಪ್ಯೋಪ-ಹಾರೇ-ನೋ-ಪತಿಷ್ಠೇತ್
ಜಮದಗ್ನಿ-ಗೋತ್ರದ
ಜಮರಂಣನು
ಜಮರಣ್ಣನು
ಜಮೀನ-ಗ-ಳಲ್ಲಿ
ಜಮೀ-ನಿಗೆ
ಜಮೀನಿನ
ಜಮೀನಿ-ನಲ್ಲಿ
ಜಮೀನು
ಜಮೀನು-ಗ-ಳಿಗೆ
ಜಮೀನು-ದಾರ
ಜಮ್ಮ-ಜಮ್ಮಾಂತರ-ದೊಳ್
ಜಯ
ಜಯಂತಿ
ಜಯಂತಿಯ
ಜಯ-ಗೊಂಡಾ-ಚಾರಿ
ಜಯ-ಜೀಯ-ವರ್ದ್ಧ-ನ-ಕರಂ
ಜಯತಿ
ಜಯತು
ಜಯತ್ಯಸೌ
ಜಯದ
ಜಯದುತ್ತ-ರಂಗ
ಜಯ-ನಾಮ
ಜಯ-ಪತ್ರದ
ಜಯಪ್ರಿಯ-ಗೊಳಿಸು-ವುದ-ರಲ್ಲಿ
ಜಯಮ್ಮ
ಜಯಮ್ಮನ
ಜಯ-ರೇಖೆ
ಜಯ-ರೇಖೆ-ಯನ್ನು
ಜಯ-ಲಕ್ಷ್ಮಿ-ಗಿತ್ತು
ಜಯ-ಲಕ್ಷ್ಮಿಗೆ
ಜಯ-ವನ್ನು
ಜಯಶಾಲಿ-ಗ-ಳಾಗಿ
ಜಯಶಾಲಿ-ಗ-ಳಾದ-ವರು
ಜಯಶಾಲಿ-ಯಾಗಿ
ಜಯ-ಶಾ-ಸನ
ಜಯಶೀಲ-ರಾಗಿ
ಜಯ-ಸಿಂಹನ
ಜಯ-ಸಿಂಹ-ನನ್ನು
ಜಯಸ್ಥಂಭ-ವ-ನೆತ್ತಿಸಿ
ಜಯಾಂಗ-ನಾ-ವಲ್ಲಭಂ
ಜಯಿಸಿ
ಜಯಿಸಿ-ಕೊಳ್ಳಿ
ಜಯಿ-ಸಿದ
ಜಯಿಸಿ-ದ-ನಂತೆ
ಜಯಿಸಿ-ದ-ನೆಂದು
ಜಯಿ-ಸಿದ್ದ
ಜಯಿಸೋ-ಜನು
ಜರಯ್ಯಂ
ಜರು-ಗುತ್ತಿದ್ದುದು
ಜಲ
ಜಲಕ್ರೀಡೆ-ಯಾಡುತ್ತಿದ್ದ-ನೆಂದು
ಜಲ-ದುರ್ಗ
ಜಲಧಾಮ
ಜಲ-ಮಾರ್ಗ
ಜಲಳ
ಜಲ-ಸಂಗ್ರಹ-ದಲ್ಲಿ
ಜಲಾಶಯ-ಗಳಿದ್ದವು
ಜಲಾಶಯ-ಗಳು
ಜಲಾಶ-ಯದ
ಜಲಾಶಯ-ದಲ್ಲಿ
ಜಲಾಶಯ-ವನ್ನು
ಜಲಾಶ-ಯವು
ಜಲಾಶ-ಯವೇ
ಜಲೆಳ
ಜಲ್ಮೋತ್ಸವ
ಜಲ್ಮೋತ್ಸವ-ದಲ್ಲಿ
ಜಲ್ಲ
ಜಲ್ಲೆ-ಯಲ್ಲಿ
ಜಳಚ-ರನಿಚ-ಯನಿಚಿತ
ಜಳಧಿ-ಯೊಳ್
ಜವ-ನವ್ವೆ
ಜವ-ನವ್ವೆ-ಯುಂಮವೆನಾರ-ತಿ-ಗಳ-ನೆತ್ತಿರ್ದರ್ತೊಳಪ
ಜವನಿಕೆ
ಜವನಿಕೆ-ನಾ-ರಾಯಣ
ಜವನಿಕೆ-ಯೊಡಲಿರ್ವ್ವ-ಲದ
ಜವನೊಡ-ನಾದಡಂ
ಜವನೋಜ
ಜವನೋ-ಜ-ನೆಂಬ
ಜವಳಿ
ಜವಳಿ-ಇದು
ಜವಳಿ-ಗೆಯ
ಜವಾದಿ
ಜವಾದಿ-ಕೋಳಾಹಳ
ಜವಾಬ್ದಾರಿ-ಯನ್ನು
ಜವಾಬ್ದಾರಿ-ಯುತ
ಜವಾಬ್ದಾರಿ-ಯುತ-ವಾದ
ಜಸದ-ನೆರ-ವನೆ-ಳ-ಗೆರೆ-ಗಾಂಕಂ
ಜಸ-ವತರ
ಜಸಹಿತ-ದೇವ
ಜಹಗೀ-ರಾಗಿರ
ಜಹಗೀ-ರಿಗೆ
ಜಹಗೀರು-ಗಳಿದ್ದಂತೆ
ಜಹಗೀರು-ಗಳು
ಜಹಾಂಗೀರ್
ಜಾಗ
ಜಾಗಕ್ಕೆ
ಜಾಗ-ಗ-ಳಲ್ಲಿ
ಜಾಗ-ದಲ್ಲಿ
ಜಾಗ-ದಲ್ಲಿದ್ದ
ಜಾಗ-ದಲ್ಲಿ-ರುವ
ಜಾಗ-ದಲ್ಲೂ
ಜಾಗ-ದಿಂದಲೇ
ಜಾಗ-ನ-ಕೆರೆ
ಜಾಗ-ನ-ಕೆರೆ-ಯಲ್ಲಿ
ಜಾಗ-ವನ್ನು
ಜಾಗ-ವಾ-ಗಿತ್ತು
ಜಾಗ-ವಿದ್ದು
ಜಾಗವು
ಜಾಗವೂ
ಜಾಗಿನ-ಕೆರೆ
ಜಾಗಿರ್ದಾರ್
ಜಾಡರು
ಜಾಣ್ಮೆಗೆ
ಜಾತಾ-ಯಸ್ಯ
ಜಾತಿ
ಜಾತಿ-ಗ-ಳಲ್ಲಿ
ಜಾತಿ-ಗ-ಳಿಗೆ
ಜಾತಿ-ಗಳು
ಜಾತಿ-ಗೂಟ
ಜಾತಿಗೆ
ಜಾತಿಗ್ರಾಮ
ಜಾತಿ-ತೆ-ರಿಗೆ
ಜಾತಿ-ಧರ್ಮ-ದಲಿ
ಜಾತಿಯ
ಜಾತಿ-ಯನ್ನು
ಜಾತಿ-ಯಲ್ಲಿ
ಜಾತಿ-ಯ-ವರು
ಜಾತಿ-ಯ-ವರೂ
ಜಾತಿ-ಯಾಗಿ-ರ-ಬ-ಹುದು
ಜಾತಿ-ಯಾದ
ಜಾತಿಯು
ಜಾತಿ-ವಾಚಕ-ವಾಗಿ
ಜಾತಿ-ಸೂಚಕ-ವಾಗಿದೆ
ಜಾತ್ಯಶ್ವದಿಂ
ಜಾತ್ರೆ
ಜಾತ್ರೆ-ಗಳು
ಜಾತ್ರೆಗೆ
ಜಾತ್ರೆಯ
ಜಾತ್ರೆ-ಯನ್ನು
ಜಾನ-ಪದ
ಜಾನ-ಪದೀಯ
ಜಾನು-ವಾರು-ಗ-ಳಿಗೆ
ಜಾನು-ವಾರು-ಗಳು
ಜಾರಿಗೆ
ಜಾರಿ-ಯಲ್ಲಿ
ಜಾರಿ-ಯಲ್ಲಿತ್ತು
ಜಾಲ-ಮರಾಳ
ಜಾವದ-ಪೂಜೆಗೆ
ಜಾವ-ಳಿಗೆ
ಜಾಸ್ತಿ
ಜಾಸ್ತಿ-ಯಾಗಿ
ಜಾಸ್ತಿ-ಯಾ-ಗಿದ್ದ
ಜಾಸ್ತಿ-ಯಾದಾಗ
ಜಾಹ್ನವಿ
ಜಿ
ಜಿಂಕೆ
ಜಿಂಕೆ-ಯಂತೆ
ಜಿಂಜಿಯ
ಜಿಆರ್ಕುಪ್ಪುಸ್ವಾಮಿ
ಜಿಎಸ್ದೀಕ್ಷಿತ್
ಜಿಎಸ್ದೀಕ್ಷಿತ್ರ-ವರು
ಜಿಗಿದು
ಜಿಜ್ಞಾಸೆ
ಜಿತ-ಪಾರ್ಶ್ವಂ
ಜಿತೇನ
ಜಿನ
ಜಿನ-ಗಂಧೋದಕ
ಜಿನ-ಗನ್ಧೋದಕ
ಜಿನ-ಗುಡ್ಡ
ಜಿನ-ಗುಡ್ಡದ
ಜಿನ-ಗೇಹ-ವನ್ನು
ಜಿನ-ಚಂದ್ರನು
ಜಿನ-ಚಂದ್ರ-ಪಂಡಿ-ತ-ನಿಗೆ
ಜಿನ-ಜ-ನನಿ-ಯೆನೆ
ಜಿನ-ದೇವಣ್ಣನು
ಜಿನ-ದೇವನ
ಜಿನ-ದೇವ-ನ-ನಜಿತ-ಸೇನ-ಮುನಿ-ಪ-ವರ
ಜಿನ-ದೊಣೆಲಕ್ಕ-ದೊ-ಣೆಯ
ಜಿನ-ಧರ್ಮಕ್ಕಾಧಾರಭೂ-ತೆಯುಂ
ಜಿನ-ಧರ್ಮ-ದಾ-ವಾಸ-ಮಾ-ದತ್ತ-ಮಳ
ಜಿನ-ಧರ್ಮಾಗ್ರಣಿ
ಜಿನ-ನಾಥ-ಪುರ
ಜಿನ-ನಾಥ-ಪುರದ
ಜಿನ-ನಾಥ-ಪುರ-ದಲ್ಲಿ
ಜಿನ-ನಾಥ-ಪುರ-ವನ್ನು
ಜಿನನೇ
ಜಿನ-ಪದ
ಜಿನ-ಪದ-ಸರ-ಸಿರುಹ
ಜಿನ-ಪಾದ-ಪಂಕಜ
ಜಿನ-ಪಾರ್ಶ್ವ-ದೇವರ
ಜಿನ-ಪಾರ್ಶ್ವ-ನಾಥ
ಜಿನ-ಪೂಜೆಗೆ
ಜಿನ-ಬಿಂಬ-ಗಳು
ಜಿನ-ಬಿಂಬ-ಗಳೂ
ಜಿನ-ಬಿಂಬದ
ಜಿನ-ಬಿಂಬವು
ಜಿನ-ಭಕ್ತ
ಜಿನ-ಭಕ್ತಿ-ಯನ್ನು
ಜಿನ-ಭ-ವ-ನದ
ಜಿನ-ಭ-ವನ-ವನ್ನು
ಜಿನ-ಮುಖ-ಚಂದ್ರ-ವಾಕ್ಚಂದ್ರಿಕಾಚಕೋರಂ
ಜಿನ-ಮುನಿ
ಜಿನ-ಮುನಿ-ಗಳ
ಜಿನ-ಮುನಿ-ಯಲ್ಲಿ
ಜಿನ-ರಾಜ-ರಾಜತ್ಪೂಜಾ-ಪುರಂದರಂ
ಜಿನ-ವಲ್ಲ-ಭನು
ಜಿನ-ಶಾ-ಸನ-ರಕ್ಷಾ-ಮಣಿ
ಜಿನ-ಸ-ಮಯ
ಜಿನ-ಸ-ಮಯ-ಸ-ಮುದ್ಧ-ರಣೆಯ
ಜಿನ-ಸ-ಮೆಯ
ಜಿನಾರ್ಚ-ನಲುಬ್ಧಂ
ಜಿನಾರ್ಚ-ನೆಗೆ
ಜಿನಾ-ಲಯ
ಜಿನಾ-ಲ-ಯಕ್ಕೆ
ಜಿನಾ-ಲಯ-ಗ-ಳನ್ನು
ಜಿನಾ-ಲಯದ
ಜಿನಾ-ಲಯ-ವನ್ನು
ಜಿನಾ-ಲಯ-ವನ್ನೂ
ಜಿನಾ-ಲಯ-ವಾಗಿದೆ
ಜಿನಾ-ಲಯ-ವೆಂದು
ಜಿನಾ-ಲಯ-ವೆಂಬ
ಜಿನಾಲೆ-ಯ-ವೆಂದು
ಜಿನುಗು-ಕೆರೆ
ಜಿಪದ-ಕಮಳೆ
ಜಿಫ್ರಿ
ಜಿಲೆಯ
ಜಿಲೆಯಲ್ಲಿದ್ದು
ಜಿಲ್ಲಾ
ಜಿಲ್ಲಾ-ವಾರು
ಜಿಲ್ಲೆ
ಜಿಲ್ಲೆ-ಗಳ
ಜಿಲ್ಲೆ-ಗ-ಳನ್ನು
ಜಿಲ್ಲೆ-ಗ-ಳಲ್ಲಿ
ಜಿಲ್ಲೆ-ಗಳಲ್ಲಿ-ರುವ
ಜಿಲ್ಲೆ-ಗ-ಳಾಗಿ
ಜಿಲ್ಲೆ-ಗ-ಳಿಗೆ
ಜಿಲ್ಲೆ-ಗಳಿದ್ದವು
ಜಿಲ್ಲೆಗೆ
ಜಿಲ್ಲೆಯ
ಜಿಲ್ಲೆ-ಯನ್ನು
ಜಿಲ್ಲೆ-ಯಲ್ಲಂತೂ
ಜಿಲ್ಲೆ-ಯಲ್ಲಿ
ಜಿಲ್ಲೆ-ಯಲ್ಲಿದೆ
ಜಿಲ್ಲೆ-ಯಲ್ಲಿದ್ದ
ಜಿಲ್ಲೆ-ಯಲ್ಲಿಯೂ
ಜಿಲ್ಲೆ-ಯಲ್ಲಿಯೇ
ಜಿಲ್ಲೆ-ಯಲ್ಲಿ-ರುವ
ಜಿಲ್ಲೆ-ಯಲ್ಲಿವೆ
ಜಿಲ್ಲೆ-ಯಲ್ಲೂ
ಜಿಲ್ಲೆ-ಯಲ್ಲೇ
ಜಿಲ್ಲೆ-ಯಾಗಿದೆ
ಜಿಲ್ಲೆ-ಯಿಂದ
ಜಿಲ್ಲೆಯು
ಜೀ
ಜೀಗುಂಡಿ
ಜೀಗುಂಡಿ-ಪಟ್ಟಣ
ಜೀಗುಂಡಿ-ಪಟ್ಟ-ಣ-ದಲ್ಲೂ
ಜೀಣೋದ್ಧಾರ
ಜೀಯ
ಜೀಯನ
ಜೀಯ-ನಾಗಿ-ರ-ಬ-ಹುದು
ಜೀಯ-ನಿಗೂ
ಜೀಯ-ನಿಗೆ
ಜೀಯ-ನಿದ್ದನು
ಜೀಯನು
ಜೀಯ-ನೆಂಬ
ಜೀಯ-ಪಾರ್ಯನ
ಜೀಯರ
ಜೀಯ-ರನೇ
ಜೀಯರು
ಜೀಯರ್
ಜೀಯರ್ಮಠ
ಜೀಯರ್ರವ-ರಿಗೆ
ಜೀಯಾತು
ಜೀಯಾತ್
ಜೀಯಾದಾಚ್ಚಂದ್ರ-ತಾರಕಂ
ಜೀಯಾದಾಚ್ಯಂದ್ರ-ತಾರಕಂ
ಜೀರ-ಹಳ್ಳಿ
ಜೀರ-ಹಳ್ಳಿ-ಗಳು-ಅಂಬಲ-ಜೀರ-ಹಳ್ಳಿ
ಜೀರಿಗೆ-ಯೊಕ್ಕಲಿಕ್ಕಿ
ಜೀರ್ಣ-ಗೊಂಡಿದ್ದಿರ-ಬಹು-ದೆಂದು
ಜೀರ್ಣ-ಜಿನಾ-ಲಯ
ಜೀರ್ಣ-ಜಿನಾ-ಲಯ-ಗ-ಳನ್ನು
ಜೀರ್ಣ-ವಾಗಿದೆ
ಜೀರ್ಣ-ವಾ-ಗಿದ್ದ
ಜೀರ್ಣ-ವಾಗಿದ್ದರೂ
ಜೀರ್ಣ-ವಾಗಿದ್ದಾಗ
ಜೀರ್ಣ-ವಾ-ಗಿದ್ದು
ಜೀರ್ಣ-ವಾಗಿ-ರಲು
ಜೀರ್ಣ-ವಾಗಿ-ರುವ
ಜೀರ್ಣ-ವಾಗಿವೆ
ಜೀರ್ಣ-ವಾದ
ಜೀರ್ಣ-ವಾದರೆ
ಜೀರ್ಣ-ವಾ-ದಲ್ಲಿ
ಜೀರ್ಣಾವಸ್ಥೆ-ಯಲ್ಲಿದೆ
ಜೀರ್ಣಾವಸ್ಥೆ-ಯಲ್ಲಿದ್ದು
ಜೀರ್ಣಾವಸ್ಥೆ-ಯಲ್ಲಿ-ರುವ
ಜೀರ್ಣೊದ್ಧಾರ
ಜೀರ್ಣೋ
ಜೀರ್ಣೋದ್ಧರ-ವಾಗಿದೆ
ಜೀರ್ಣೋದ್ಧಾರ
ಜೀರ್ಣೋದ್ಧಾರಂ
ಜೀರ್ಣೋದ್ಧಾರಕ
ಜೀರ್ಣೋದ್ಧಾರ-ಕ-ನೆಂದು
ಜೀರ್ಣೋದ್ಧಾರ-ಕ-ನೆನಿಸಿದ
ಜೀರ್ಣೋದ್ಧಾರಕ್ಕೆ
ಜೀರ್ಣೋದ್ಧಾರ-ಗ-ಳನ್ನು
ಜೀರ್ಣೋದ್ಧಾರ-ಗಳಿ-ಗಾಗಿ
ಜೀರ್ಣೋದ್ಧಾರ-ಗ-ಳಿಗೆ
ಜೀರ್ಣೋದ್ಧಾರ-ಗಳು
ಜೀರ್ಣೋದ್ಧಾರ-ಗೊಂಡವು
ಜೀರ್ಣೋದ್ಧಾರ-ಗೊಂಡಿದೆ
ಜೀರ್ಣೋದ್ಧಾರ-ಗೊಂಡಿವೆ
ಜೀರ್ಣೋದ್ಧಾರ-ಗೊಳಿಸ-ಲಾಗಿದೆ
ಜೀರ್ಣೋದ್ಧಾರ-ಗೊಳಿ-ಸಿ-ದಂತೆ
ಜೀರ್ಣೋದ್ಧಾರದ
ಜೀರ್ಣೋದ್ಧಾರ-ವನ್ನು
ಜೀರ್ಣೋದ್ಧಾರ-ವಾಗಿ
ಜೀರ್ಣೋದ್ಧಾರ-ವಾಗಿದೆ
ಜೀರ್ಣೋದ್ಧಾರ-ವಾಗಿದ್ದಂತೆ
ಜೀರ್ಣೋದ್ಧಾರ-ವಾ-ಗಿದ್ದು
ಜೀರ್ಣೋದ್ಧಾರ-ವಾಗಿ-ರ-ಬ-ಹುದು
ಜೀರ್ಣೋದ್ಧಾರ-ವಾಗಿ-ರ-ವಂತೆ
ಜೀರ್ಣೋದ್ಧಾರ-ವಾಗಿ-ರುವ
ಜೀರ್ಣೋದ್ಧಾರ-ವಾಗಿ-ರು-ವಂತೆ
ಜೀರ್ಣೋದ್ಧಾರ-ವಾಗಿವೆ
ಜೀರ್ಣೋದ್ಧಾರ-ವಾ-ದಲ್ಲಿ
ಜೀರ್ಣೋದ್ಧಾರ-ವಾದವು
ಜೀರ್ಣೋದ್ಧಾರ-ವಾಯಿ-ತೆಂದು
ಜೀರ್ಣೋದ್ಧಾರ-ವಿಜಯ-ನ-ಗರದ
ಜೀವಂತ-ವಾಗಿ
ಜೀವಂಧರ
ಜೀವಂಧರ-ನೆಂಬ
ಜೀವ-ಕಳೆ
ಜೀವತ-ವರ್ಗಕ್ಕೆ
ಜೀವದ
ಜೀವನ
ಜೀವನಕ್ಕೆ
ಜೀವನದ
ಜೀವನ-ದಲ್ಲಿ
ಜೀವನ-ವನ್ನು
ಜೀವನ-ವಾಗಿ-ರುತ್ತಿತ್ತು
ಜೀವನೋಪಾಯಕ್ಕಾಗಿ
ಜೀವನೋಪಾ-ಯಕ್ಕೆ
ಜೀವನೋಪಾಯಕ್ಕೆಂದು
ಜೀವ-ರಾಶಿ-ಯಲ್ಲಿ
ಜೀವ-ವನ್ನು
ಜೀವ-ಸಹಿತ
ಜೀವಾತ್ಮ
ಜೀವಾತ್ಮರು
ಜೀವಿ-ಕೆ-ಗಾಗಿ
ಜೀವಿ-ಗಳ
ಜೀವಿ-ಗ-ಳಾಗಿದ್ದರು
ಜೀವಿತ
ಜೀವಿ-ತಕ್ಕೂ
ಜೀವಿ-ತಕ್ಕೆ
ಜೀವಿ-ತದ
ಜೀವಿ-ತ-ದ-ವ-ರಿಗೆ
ಜೀವಿ-ತ-ನಿ-ಯೋಗ
ಜೀವಿ-ತ-ವನ್ನಾಗಿ
ಜೀವಿ-ತ-ವನ್ನು
ಜೀವಿ-ತ-ವರ್ಗಕ್ಕೆ
ಜೀವಿ-ತ-ವರ್ಗದ
ಜೀವಿ-ತ-ವರ್ಗ-ದ-ವರ
ಜೀವಿ-ತ-ವರ್ಗ-ದ-ವ-ರಿಗೆ
ಜೀವಿ-ತ-ವರ್ಗ-ದ-ವರು-ನಿ-ಬಂಧ-ಕಾರರು
ಜೀವಿ-ತ-ವರ್ಗ-ವಕ್ಕೆ
ಜೀವಿ-ತ-ವಾಗಿ
ಜೀವಿ-ತವೂ
ಜೀವಿ-ತ-ವೆಂದೂ
ಜೀವಿ-ತ-ವೊಳಗಾಯ್ತಕ್ಕೆ
ಜೀವಿ-ತ-ವೊಳಗಾಯ್ತಕ್ಕೆ-ಆಯ-ತಕ್ಕೆ
ಜೀವಿ-ತಾವಧಿಯ
ಜೀವಿಯೂ
ಜೀವಿ-ವೊಳಗಾಯ್ತಕ್ಕೆ
ಜೀವಿ-ಸಲು
ಜೀವಿ-ಸಿದ್ದನು
ಜೀವಿ-ಸಿದ್ದ-ನೆಂದು
ಜೀವಿ-ಸಿರುವ-ವರೆಗೆ
ಜೀವಿ-ಸುತ್ತಿದ್ದರು
ಜೀವಿ-ಸುತ್ತಿದ್ದು
ಜುಂಜಾ-ಪುರ
ಜುಮ್ಮಾ-ಮಸೀದಿ
ಜುಲೈ
ಜೂಟದಾ
ಜೂನ್
ಜೆಂನಿಗೆ
ಜೆಎಂನಾಗಯ್ಯ-ನ-ವರು
ಜೆಟ್ಟಿಗದ
ಜೆಟ್ಟಿಗ-ದಲ್ಲಿ
ಜೆಟ್ಟಿಗ-ವನ್ನು
ಜೆಡಲ
ಜೆಡೆಯ
ಜೆರೂ-ಸಲೆಮ್ನಲ್ಲಿ
ಜೇಡ-ಗೊತ್ತಳಿ
ಜೇಡ-ಗೊತ್ತಳಿಯ
ಜೇಡ-ದೆರೆಯ
ಜೇಡರ
ಜೇಡರ-ದಾಸಿ-ಮಯ್ಯನ
ಜೇಡರ-ಮೊಜ್ಜನ
ಜೇಡಿ-ಮಣ್ಣನ್ನು
ಜೇನು-ಗುಡ್ಡ
ಜೇನ್ನೇಬ್
ಜೇವ-ರಗಿ
ಜೈತಾಜಿ-ಕಾಟ್ಕರ್
ಜೈತಾಜಿಯ-ರನ್ನು
ಜೈತು-ಗಿಯ
ಜೈತೆಯನ
ಜೈದೇವ
ಜೈನ
ಜೈನ-ಕವಿ
ಜೈನ-ಕವಿ-ಗಳು
ಜೈನ-ಕವಿ-ಯೆಂದೂ
ಜೈನ-ಕುಟುಂಬ-ಗಳು
ಜೈನ-ಕೇಂದ್ರ
ಜೈನ-ಕೇಂದ್ರ-ಅ-ವಾಗಿದ್ದವು
ಜೈನ-ಕೇಂದ್ರ-ಗಳ
ಜೈನ-ಕೇಂದ್ರ-ಗ-ಳಾದವು
ಜೈನ-ಕೇಂದ್ರ-ಗಳು
ಜೈನ-ಕೇಂದ್ರ-ವಾ-ಗಿತ್ತು
ಜೈನ-ಕೇಂದ್ರ-ವಾ-ಗಿದ್ದ
ಜೈನ-ಕೇಂದ್ರ-ವಾಗಿದ್ದಿರ-ಬ-ಹುದು
ಜೈನ-ಕೇಂದ್ರ-ವಾ-ಗಿದ್ದು
ಜೈನ-ಕೇಂದ್ರ-ವಾಗಿ-ರ-ಬ-ಹುದು
ಜೈನ-ಕೇಂದ್ರ-ವಾದ
ಜೈನ-ಕೇದ್ರ-ವಾ-ಗಿತ್ತು
ಜೈನಕ್ಷೇತ್ರ-ವಾದ
ಜೈನ-ಗುರು-ಗಳು
ಜೈನ-ಗೇಹ-ಗಳಂತೆನೆತುಂ
ಜೈನ-ತೀರ್ಥ-ಗ-ಳಿಗೆ
ಜೈನ-ತೀರ್ಥ-ಗಳು
ಜೈನ-ತೀರ್ಥ-ಗ-ಳೆಂದು
ಜೈನ-ತೀರ್ಥ-ವಾ-ಗಿತ್ತು
ಜೈನ-ತೀರ್ಥ-ವಾ-ಗಿದ್ದ
ಜೈನ-ಧರ್ಮ
ಜೈನ-ಧರ್ಮಕ್ಕೂ
ಜೈನ-ಧರ್ಮಕ್ಕೆ
ಜೈನ-ಧರ್ಮದ
ಜೈನ-ಧರ್ಮ-ದತ್ತ
ಜೈನ-ಧರ್ಮ-ದಲ್ಲಿ
ಜೈನ-ಧರ್ಮ-ನಿರ್ಮಳಾಂಬರ
ಜೈನ-ಧರ್ಮ-ವನ್ನು
ಜೈನ-ಧರ್ಮವು
ಜೈನ-ಧರ್ಮವೂ
ಜೈನ-ಧರ್ಮಾವ-ಲಂಬಿ-ಗ-ಳಾಗಿದ್ದು
ಜೈನ-ಧರ್ಮಾವ-ಲಂಬಿ-ಗ-ಳಾದ
ಜೈನ-ಧರ್ಮೀಯ-ನಾದ
ಜೈನ-ಧರ್ಮ್ಮ-ನಿರ್ಮ್ಮಳಾಂಬ-ರಹಿ-ಮಕ-ರನುಂ
ಜೈನ-ನಾ-ಗಿದ್ದ
ಜೈನ-ನಾಗಿದ್ದ-ನೆಂದು
ಜೈನ-ಪುರಾಣ-ಗ-ಳಲ್ಲಿ
ಜೈನ-ಬ-ಸದಿ
ಜೈನ-ಬ-ಸದಿ-ಗ-ಳನ್ನು
ಜೈನ-ಬ-ಸದಿ-ಗ-ಳಿಗೆ
ಜೈನ-ಬ-ಸದಿಗೆ
ಜೈನ-ಬ-ಸದಿ-ಯಲ್ಲಿದ್ದ
ಜೈನ-ಬ-ಸದಿ-ಯಿದ್ದು
ಜೈನ-ಬ-ಸದಿಯು
ಜೈನ-ಬ-ಸದಿಯೇ
ಜೈನ-ಬಸ್ತಿಯ
ಜೈನ-ಮಂದಿ-ರ-ವನ್ನು
ಜೈನ-ಮತಾವ-ಲಂಬಿ-ಯಾಗಿದ್ದಳು
ಜೈನ-ಮುನಿ
ಜೈನ-ಮುನಿ-ಗಳು
ಜೈನ-ಯತಿ
ಜೈನ-ಯತಿ-ಗಳ
ಜೈನ-ಯತಿ-ಗ-ಳಾಗಿದ್ದು
ಜೈನ-ಯತಿ-ಗ-ಳಿಗೆ
ಜೈನ-ಯತಿ-ಗಳು
ಜೈನ-ಯತಿ-ಪರಂಪರೆ
ಜೈನ-ಯತಿ-ಪರಂಪರೆ-ಯನ್ನು
ಜೈನ-ಯತಿ-ಪರಂಪ-ರೆಯು
ಜೈನ-ಯತಿಯ
ಜೈನ-ಯತಿ-ಯನ್ನು
ಜೈನ-ಯತಿ-ಯಾ-ಗಿದ್ದ
ಜೈನ-ಯತಿ-ಯಾದ
ಜೈನ-ಯ-ತಿಯು
ಜೈನ-ಯತಿ-ಯೊಬ್ಬರು
ಜೈನರ
ಜೈನ-ರನ್ನು
ಜೈನ-ರಾ-ಗಿದ್ದುದು
ಜೈನರು
ಜೈನ-ರೊಡನೆ
ಜೈನ-ವೀರ-ಶೈವ
ಜೈನ-ವೈಷ್ಣವ
ಜೈನ-ಸಂಘ-ವನ್ನು
ಜೈನ-ಸನ್ಯಾ-ಸಿ-ಗಳು
ಜೈನ-ಸನ್ಯಾ-ಸಿಯು
ಜೈನುದ್ದೀನ್
ಜೈಮಿನಿ
ಜೊತಗೆ
ಜೊತೆ
ಜೊತೆ-ಗಿದ್ದು
ಜೊತೆ-ಗೂಡಿ
ಜೊತೆಗೆ
ಜೊತೆ-ಗೆ-ಹೋಗಿ
ಜೊತೆಗೇ
ಜೊತೆ-ಯಲ್ಲಿ
ಜೊತೆ-ಯಲ್ಲಿದ್ದ
ಜೊತೆ-ಯಲ್ಲಿಯೇ
ಜೊತೆ-ಯಲ್ಲೇ
ಜೊತೆ-ಯಾಗಿ
ಜೊತೆ-ಯಾ-ಗಿದ್ದು
ಜೊಮ್ಮಣ್ಣ
ಜೊಮ್ಮಯ್ಯ-ಗಳ
ಜೊಮ್ಮಿ-ಸೆಟ್ಟಿ
ಜೊರೆಗ
ಜೊಸೆಫ್
ಜೋಗಿಯ-ಕಟ್ಟೆ
ಜೋಗುಂಡಯ್ಯ
ಜೋಗುಣ್ಡಯ್ಯ-ನಿಗೆ
ಜೋಡಿ
ಜೋಡಿ-ಕರ-ಡ-ಹಳ್ಳಿ
ಜೋಡಿ-ಲಿಂಗ-ನ-ಗುಡಿ-ಗಳು
ಜೋಡಿ-ಲಿಂಗೇಶ್ವರ
ಜೋಡಿಸಿ
ಜೋಡಿ-ಹಣದ
ಜೋಡು
ಜೋಳದ
ಜೋಳ-ವಾಳಿ
ಜೋಳ-ವಾಳಿಯ
ಜೋಳ-ವಾಳಿ-ಯ-ವನಲ್ಲಲ್ಲ
ಜೋಳ-ವಾಳಿ-ಯೊಳಿರ್ಪನಾ
ಜೋಸೆಫ್
ಜೋಸೆಫ್ಸಿ-ಬಾಲ್
ಜ್ಞಾತಿ
ಜ್ಞಾನಕ್ಕೆ
ಜ್ಞಾನ-ದೇಹಾಯ
ಜ್ಞಾನ-ಮಂಟಪ
ಜ್ಞಾನಾಶ್ವತ್ಥವೆಂದ
ಜ್ಞಾನಿ-ಗಳೂ
ಜ್ಞಾಪ-ಕಾರ್ಥ
ಜ್ಞಾಪ-ಕಾರ್ಥ-ವಾಗಿ
ಜ್ಯೇಷ್ಠ-ಪತ್ನಿ
ಜ್ಯೋತಿ-ಶಾಸ್ತ್ರಾರ್ಥ
ಜ್ವರ
ಜ್ವರದ
ಝರಾ-ಧರಾ
ಝುಮ್ರಾ
ಟಂಕ-ಸಾಲೆ
ಟಂಕ-ಸಾಲೆಗೆ
ಟಂಕ-ಸಾಲೆ-ಯನ್ನು
ಟಂಕಿ-ಸಿದ
ಟನಿಟುರ
ಟಪ್ಪು-ಸುಲ್ತಾ-ನನು
ಟಿಪು
ಟಿಪು-ವಿನ
ಟಿಪ್ಪಣಿ-ಗ-ಳನ್ನು
ಟಿಪ್ಪಣಿ-ಯನ್ನು
ಟಿಪ್ಪಣಿ-ಯಲ್ಲಿ
ಟಿಪ್ಪ-ವಿನ
ಟಿಪ್ಪು
ಟಿಪ್ಪು-ವನ್ನು
ಟಿಪ್ಪು-ವಿಗ
ಟಿಪ್ಪು-ವಿನ
ಟಿಪ್ಪುವು
ಟಿಪ್ಪು-ಸುಲ್ತಾ-ನನು
ಟಿಪ್ಪು-ಸುಲ್ತಾನ್
ಟಿಪ್ಪೂ
ಟಿಪ್ಪೂ-ವಿನ
ಟಿಪ್ಪೂ-ಸುಲ್ತಾ-ನನ
ಟಿಪ್ಪೂ-ಸುಲ್ತಾ-ನನು
ಟಿಪ್ಪೂ-ಸುಲ್ತಾನ್
ಟಿವಿ
ಟೆಕ್ನಾಲಜೀಸ್ನ
ಟೇಕಲ್
ಟ್ರೆಜರಿ
ಟ್ರೈಬ್ಸ್
ಠಾಣೆಯ
ಠೋಕು
ಡ
ಡಂಕನ್
ಡಂಕನ್ಡೆರೆಟ್
ಡಂಕನ್ಡೆರೆಟ್ರ-ವರು
ಡಬ್ಬಿಯ
ಡಳ-ಸಾಮಿ
ಡಳೇಶ್ವರದ
ಡಾ
ಡಾಅಲ-ನರ-ಸಿಂಹನ್
ಡಾಎಂ
ಡಾಎಂಎಂ
ಡಾಎಂಬಿ-ಪದ್ಮ
ಡಾಎಚ್ಎಸ್
ಡಾಎನ್ಎಸ್
ಡಾಎಲ್
ಡಾಎವಿ-ನರ-ಸಿಂಹ-ಮೂರ್ತಿ-ಯ-ವರು
ಡಾಎ-ಸತ್ಯ-ನಾ-ರಾಯಣ
ಡಾಎಸ್
ಡಾಎಸ್ಎನ್
ಡಾಎಸ್ಎಲ್
ಡಾಎಸ್ಕೃಷ್ಣ-ಮೂರ್ತಿ-ಯ-ವರು
ಡಾಎಸ್ಕೆ-ಕು-ಮಾರಸ್ವಾಮಿ
ಡಾಎಸ್ಗುರು-ರಾಜಾ-ಚಾರ್ಯರ
ಡಾಎಸ್ನಾಗ-ರಾಜು
ಡಾಎಸ್ರಂಗ-ರಾಜು
ಡಾಕ-ರಸ
ಡಾಕ-ರ-ಸನ
ಡಾಕ-ರಸ-ನನ್ನು
ಡಾಕ-ರಸ-ನೆಂದು
ಡಾಕೆ-ಆರ್
ಡಾಚೆನ್ನಕ್ಕ
ಡಾಡಿವಿ-ಪರ-ಮ-ಶಿವ-ಮೂರ್ತಿ
ಡಾದೇವ-ರ-ಕೊಂಡಾ-ರೆಡ್ಡಿ
ಡಾದೇವ-ರ-ಕೊಂಡಾ-ರೆಡ್ಡಿ-ಯ-ವರು
ಡಾನಾಗ-ರಾಜ್
ಡಾಪಿವಿ
ಡಾಬಿ-ಆರ್
ಡಾಬಿಶೇಕ್ಅಲಿ
ಡಾರಂ
ಡಾರಾಧಾ-ಪಟೇಲ್
ಡಾಶೇಷ-ಶಾಸ್ತ್ರಿ-ಯ-ವರು
ಡಾಶೋಭಾ
ಡಾಶ್ರೀ-ನಿ-ವಾಸ
ಡಾಸಾಲೆ-ತೂರ್
ಡಾಸೂರ್ಯ-ನಾಥ-ಕಾ-ಮತ್
ಡಿಂಕ
ಡಿಎಚ್
ಡಿಕೆ
ಡಿಜಿಟಲ್
ಡಿಟಿ
ಡಿಟಿಪಿ
ಡಿಟಿಬಿ
ಡಿಡ-ಗದ
ಡಿಬಿ
ಡಿವಿ-ಜನ್
ಡಿವಿ-ಜನ್ಗೆ
ಡಿವಿ-ಜನ್ನಲ್ಲಿ
ಡಿಸೆಂಬರ್
ಡುಂಡುಂಕಾರ
ಡೆಂಕಣಿ-ಕೋಟೆ
ಡೆಕ್ಕನ್
ಡೆಪ್ಯುಟೇ-ಷನ್
ಡೆರೆಟ್
ಡೆರೆಟ್ರ-ವರು
ಡೊಗರ-ಹಳ್ಳ-ಗಳ
ಡೊಣೆ
ಡೊಣೆ-ಗಳು
ಡೊಣೆ-ಯಿದ್ದು
ಡೊಣೆಯೂ
ಡ್ಯಾಮ್
ಢಣಾ-ಯಕನ
ಣ್ನೆðಲನಂವಯ್ವತ್ತರಂ
ತ
ತಂಗಡಗಿ
ತಂಗಲು
ತಂಗಿ
ತಂಗಿದ್ದ
ತಂಗಿದ್ದು
ತಂಗಿಯ
ತಂಗಿ-ಯನ್ನೇನಾ
ತಂಗಿ-ರ-ಬಹು-ದೆಂದು
ತಂಗುದಾಂಣ
ತಂಜಾವೂ-ರನ್ನು
ತಂಜಾ-ವೂರು
ತಂಡ-ಗಳು
ತಂಡವ-ತಂದು
ತಂಡವೇ
ತಂಡಸ-ಹಳ್ಳಿಯ
ತಂಡಸೇ-ಹಳ್ಳಿ
ತಂಡ್ರಿ
ತಂತಂಮಂಗಮಂ
ತಂತಕ-ನಂತೆ-ಸಂಗ-ರದೊಳೋವದೆ
ತಂತ್ರ
ತಂತ್ರ-ಗಳ
ತಂತ್ರಜ್ಞ
ತಂತ್ರ-ವೆಗ್ಗಡೆ
ತಂತ್ರ-ವೆಗ್ಗಡೆ-ತನ-ವನ್ನು
ತಂತ್ರ-ವೆಗ್ಗಡೆ-ಯಾ-ಗಿದ್ದ
ತಂತ್ರ-ವೆಗ್ಗಡೆಯು
ತಂತ್ರಾದಿಷ್ಟಾ-ಯಕ
ತಂತ್ರಾಧಿಷ್ಟಾ-ಯಕ
ತಂತ್ರಾ-ಧಿಷ್ಠಾ-ಯಕ
ತಂದ
ತಂದಂತೆ
ತಂದ-ನೆಂದು
ತಂದ-ರೆಂದೂ
ತಂದ-ವರು
ತಂದಿದ್ದ
ತಂದಿದ್ದಾರೆ
ತಂದಿರ-ಬ-ಹುದು
ತಂದಿರಿ-ಸಿದ್ದಾರೆ
ತಂದು
ತಂದು-ಕೊಂಡು
ತಂದು-ದಕ್ಕಾಗಿ
ತಂದೆ
ತಂದೆ-ಗಳ
ತಂದೆಗೆ
ತಂದೆ-ತಾಯಿ-ಗಳು
ತಂದೆ-ತಾಯಿ-ಗ-ಳೆಂದು
ತಂದೆ-ತಾಯಿ-ಯನು
ತಂದೆಯ
ತಂದೆ-ಯ-ಕಾಲ-ದಲ್ಲೇ
ತಂದೆ-ಯ-ಗಂಧ-ವಾರಣ
ತಂದೆ-ಯನ್ನು
ತಂದೆ-ಯಾ-ಗಿದ್ದು
ತಂದೆ-ಯಾದ
ತಂದೆಯು
ತಂದೆಯೂ
ತಂದೆ-ಯೆಂದೂ
ತಂದೆ-ಯೊಲ್
ತಂನ
ತಂನ-ನುಜಾ-ತರ್ಬ್ಬೋಕಣಂ
ತಂನ-ವರ
ತಂನಾಮ್ನಾ-ಯದ
ತಂನೊಳ್
ತಂನ್ನ
ತಂಪಿನ
ತಂಬಿ
ತಂಬಿ-ಯಣ್ಣ
ತಂಬಿ-ಯರ
ತಂಬುಲಿಗ
ತಂಮ
ತಂಮಂಗೆ
ತಂಮಂನ
ತಂಮ-ಡಿ-ಗಟ್ಟೆಯ
ತಂಮ-ಡಿ-ಹಳ್ಳಿಯ
ತಂಮಯ
ತಂಮೆಯ-ದೇವ
ತಕ್ಕ
ತಕ್ಕಂತೆ
ತಕ್ಕ-ಮಟ್ಟಿಗೆ
ತಕ್ಕೋಲಂ
ತಕ್ಕೋಲಂನಲ್ಲಿ
ತಕ್ಕೋಲ-ದಲ್ಲಿ
ತಕ್ಕೋ-ಲದೊಲ್ಕಾದಿ
ತಕ್ಕೋ-ಲಮ್
ತಕ್ಷಣ
ತಕ್ಷಣ-ದಲ್ಲಿ
ತಕ್ಷಣ-ದಲ್ಲಿಯೇ
ತಗಚ-ಗೆರೆ
ತಗಚೆ-ಗೆರೆ-ಗ-ಳನ್ನು
ತಗಚೆ-ಗೆರೆ-ಗ-ಳನ್ನೇ
ತಗಚೆ-ಗೆರೆ-ಗಳಿದ್ದ-ವೆಂದು
ತಗಚೆ-ಗೆರೆಯ
ತಗಡಿನ
ತಗಡೂರ
ತಗಡೂರ-ರೊಳಗೆ
ತಗಡೂರಿ-ನಲ್ಲಿ
ತಗಡೂರು
ತಗ-ರಿಗಲ್
ತಗರೆ
ತಗಲುವ
ತಗುಲಿದ
ತಗುಳ್ದು
ತಗ್ಗ-ಲೂರು
ತಗ್ಗಿನ
ತಗ್ಗಿಳೂರು
ತಗ್ಗುಪ್ರ-ದೇಶ-ದಲ್ಲಿದ್ದು-ದ-ರಿಂದ
ತಜ್ಜ-ವನಿಕೆ-ಗೊಂಡ-ಗಂಡ
ತಜ್ಞ-ರಾದ
ತಜ್ಞರು
ತಜ್ಞರೂ
ತಜ್ಞ-ರೊಬ್ಬರು
ತಟದ
ತಟಾಕ
ತಟಾಕ-ಕೆರೆ
ತಟಾಕ-ಗಳ
ತಟಾಕತ್ರಯ-ಗ-ಳನ್ನು
ತಟಾಕತ್ರಯ-ವನ್ನು
ತಟಾಕಾಂತ
ತಟ್ಟ-ಹಳ್ಳದ
ತಟ್ಟಿದರು
ತಟ್ಟೆ-ಯನ್ನು
ತಟ್ಟೆ-ಯಲ್ಲಿ
ತಟ್ಟೆ-ವಳ್ಳವ
ತಟ್ಟೆ-ಹಳ್ಳದ
ತಟ್ಟೇ-ಕೆರೆ
ತಟ್ಟೇ-ಹಳ್ಳಿ
ತಡರೆ-ಬಲ್ಗಂಡನುಂ
ತಡಿನ
ತಡಿಮಾ-ಲಿಂಗಿ
ತಡಿಮಾ-ಲಿಂಗಿ-ಯಲ್ಲಿ
ತಡಿ-ಯಲ್ಲಿದ್ದ
ತಡಿ-ವಿಡಿದು
ತಡೆ
ತಡೆ-ಗಟ್ಟಲು
ತಡೆ-ದಿರ-ಬಹು-ದೆಂದು
ತಡೆದು
ತಡೆ-ದು-ಕೊಂಡು
ತಡೆ-ಯಲಾ-ಗದೆ
ತಡೆ-ಯುವ
ತತು-ಕಾಲೋಚಿತಕ್ರಯದ್ರಬ್ಯ
ತತೇಜೋ-ನಿಳಯಂ
ತತ್ಕಾಲೋಚಿತ
ತತ್ಕ್ರಮ-ಮ-ಮರ್ದೆಸೆ-ವಂತು
ತತ್ತ್ವವು
ತತ್ಪಾದ-ಪದ್ಮೋಪ-ಜೀವಿ-ಗಳೆನಿಸಿ-ದನ್ವ-ಯಾಗತ
ತತ್ಪುತ್ರಃ
ತತ್ಪೂರ್ವ-ದಲ್ಲಿ
ತತ್ರ
ತತ್ರಯಿ
ತತ್ರ-ಯಿಕ
ತತ್ವ
ತತ್ವ-ಗ-ಳನ್ನು
ತತ್ವ-ಗಳುಳ್ಳ
ತತ್ವಗ್ರಹಣ
ತತ್ವಜ್ಞಾನಿ
ತತ್ವಜ್ಞಾನ್ತ್ಸದಾ-ಚಾರ
ತತ್ವ-ದೊಳು
ತತ್ವೈ-ಕನಿಷ್ಠುರ
ತತ್ಸಾಹಸಾಭ್ಯು-ದಯಂ
ತಥ್ಯ-ವಿದೆ
ತದ-ನಂತ-ರದ
ತದನು
ತದನು-ಜನ್ಮಾ
ತದಪಿ
ತದಿದಂ
ತದೀ-ಯಾರಾ-ಧನವು
ತದೀ-ಯಾರಾ-ಧನೆ
ತದ್ಭವ
ತನಂಗಾಡಿ
ತನಂಗಾಡಿ-ಯಾದ
ತನಕ
ತನಕ್ಕೆ
ತನಗೆ
ತನ-ಗೊಪ್ಪಂಬೆತ್ತಮಾದ
ತನದ-ಕಯ್ಯ
ತನ-ದಟ್ಟಿಬಡಿವಂ
ತನಯ
ತನಯಃ
ತನ-ಯರ
ತನ-ಯಸ್ತಿರುಮ-ಯಾರ್ಯ್ಯೋ
ತನುಜ
ತನುಜ-ನಾದ
ತನುಜರು
ತನು-ಮಂಡ-ಳಮಂ
ತನೂಜ
ತನೂ-ಭವ
ತನೆಯಂ
ತನೆಯರು
ತನ್ನ
ತನ್ನನ್ನು
ತನ್ನಾರ್ಪ್ಪಿನಿಂ
ತನ್ನೂರ
ತನ್ನೂರಾದ
ತನ್ನೊಂದೆ
ತನ್ನೊಳಗೆ
ತಪಃಫಲ
ತಪದಿಂ
ತಪವಂ
ತಪಸಾ
ತಪಸಿಯ
ತಪಸಿಯ-ತಪಸಿ-ಹಳ್ಳಿ
ತಪಸೀ-ರಾಯನ
ತಪಸ್ಯರು
ತಪಸ್ವಿ
ತಪಸ್ಸನ್ನು
ತಪಸ್ಸಿ-ನಿಂದ
ತಪಸ್ಸು
ತಪಿ-ದ-ವರು
ತಪುವ-ರಾಯರ
ತಪೋ-ಧನ
ತಪೋ-ಧನ-ನಿಗೆ
ತಪೋಧ-ನರ
ತಪೋಧ-ನರು
ತಪೋಪ-ವಾಸ
ತಪ್ಪ
ತಪ್ಪ-ತಪ್ಪು-ವ-ರಗಾಳ್ವದ್ಧಱಿಪ್ಪುವಂ
ತಪ್ಪದೆ
ತಪ್ಪದೇ
ತಪ್ಪಾಗಿ
ತಪ್ಪಾಗಿ-ರ-ಬ-ಹುದು
ತಪ್ಪಾಗಿ-ರುವ
ತಪ್ಪಾಗಿ-ರು-ವು-ದನ್ನು
ತಪ್ಪಾಗು-ವುದು
ತಪ್ಪಿದ
ತಪ್ಪಿ-ದರೂ
ತಪ್ಪಿ-ದರೆ
ತಪ್ಪಿದ-ವರು
ತಪ್ಪಿದ್ದು
ತಪ್ಪಿಯಾ-ದರೂ
ತಪ್ಪಿ-ಹೋಗಿ
ತಪ್ಪು
ತಪ್ಪು-ಗ-ಳಾದಾಗ
ತಪ್ಪು-ಗಳಿ-ರ-ಬ-ಹುದು
ತಪ್ಪುವ
ತಪ್ಪು-ವ-ನಾಯ-ಕರ
ತಪ್ಪು-ವ-ರಾಯರ
ತಪ್ಪೆ-ತಪ್ಪುವಂ
ತಮಗಂಜಿ
ತಮಗೆ
ತಮಿಳಿ-ನಲ್ಲಿ
ತಮಿಳು
ತಮಿಳುಗ್ರಂಥ-ಲಿಪಿಯ
ತಮಿಳು-ದೇಶದ
ತಮಿಳು-ನಾಡನ್ನು
ತಮಿಳು-ನಾಡಿನ
ತಮಿಳು-ನಾಡಿ-ನಲ್ಲಿ
ತಮಿಳು-ನಾಡಿನಲ್ಲಿದ್ದನು
ತಮಿಳು-ನಾಡಿ-ನಲ್ಲೇ
ತಮಿಳು-ನಾಡಿನಿಂದ
ತಮಿಳು-ನಾಡು
ತಮಿಳು-ರೂಪ
ತಮಿಳು-ಶಾ-ಸನ-ದಲ್ಲಿ
ತಮುಉತ್ತಯ್ವರು
ತಮ್ಮ
ತಮ್ಮಂ
ತಮ್ಮಂದಿರ
ತಮ್ಮಂದಿರಿರ-ಬ-ಹುದು
ತಮ್ಮಂದಿರು
ತಮ್ಮಂದಿರು-ಗಳು
ತಮ್ಮಂದಿ-ರೆಂದು
ತಮ್ಮಂದಿರೋ
ತಮ್ಮಜ್ಜ
ತಮ್ಮಡಿ
ತಮ್ಮ-ಡಿ-ಗಟ್ಟೆಯ
ತಮ್ಮ-ಡಿ-ಗಳ
ತಮ್ಮ-ಡಿ-ಗ-ಳಾಗಿದ್ದಾರೆ
ತಮ್ಮ-ಡಿ-ಗಳು
ತಮ್ಮ-ಡಿ-ಗಳೇ
ತಮ್ಮ-ಡಿಯು
ತಮ್ಮ-ಡಿ-ಹಳ್ಳಿ
ತಮ್ಮ-ಡಿ-ಹಳ್ಳಿಯ
ತಮ್ಮಣ್ಣ
ತಮ್ಮ-ತೀರ್ವ್ವರ್ಗ್ಗೆ
ತಮ್ಮನ
ತಮ್ಮ-ನಂತೆ
ತಮ್ಮ-ನನ್ನೋ
ತಮ್ಮ-ನಾದ
ತಮ್ಮ-ನಿಗೆ
ತಮ್ಮ-ನಿದ್ದನು
ತಮ್ಮ-ನಿದ್ದ-ನೆಂದು
ತಮ್ಮ-ನಿರ-ಬ-ಹುದು
ತಮ್ಮ-ನೀರ್ವ್ವರ್ಗೆ
ತಮ್ಮನು
ತಮ್ಮನುಂ
ತಮ್ಮನೂ
ತಮ್ಮ-ನೆಂದು
ತಮ್ಮ-ನೆಂದೂ
ತಮ್ಮನ್ನ
ತಮ್ಮನ್ನು
ತಮ್ಮ-ಯಣ್ಣ
ತಮ್ಮಲ್ಲೇ
ತಮ್ಮವ್ವೆ
ತಮ್ಮೊಳಗೆ
ತಮ್ಮೋಜಿ
ತಯಾರಿ-ಸಲು
ತಯಾರಿಸಿ
ತಯಾ-ರಿ-ಸಿದ
ತಯಾರಿಸುತ್ತಿದ್ದ-ರೆಂದು
ತಯಾರಿಸುತ್ತಿದ್ದ-ವರು
ತಯಾರಿಸು-ವುದ-ರಲ್ಲಿ
ತಯಾರಿಸು-ವುದ-ರಲ್ಲಿಯೂ
ತಯಾರು
ತರ-ಗ-ತಿ-ಗ-ಳಲ್ಲಿ
ತರ-ಗತಿ-ಯಲ್ಲಿ
ತರ-ಣಿ-ಯೆಂಬ
ತರ-ಬಿನ-ಕೋಟೆ
ತರ-ಬೇಕೆಂದು
ತರ-ಲಾಗುತ್ತಿತ್ತು
ತರಲು
ತರವೇ
ತರಿದಿಕ್ಕಿ-ದನು
ತರಿದು-ಹಾಕಿದ
ತರಿದು-ಹೋರಾಡಿ
ತರಿಸಿ
ತರಿಸಿ-ಕೊಡುತ್ತಾರೆ
ತರು-ಣ-ದಲ್ಲಿಯೇ
ತರುತ್ತದೆ
ತರುತ್ತವೆ
ತರುತ್ತಿದ್ದ
ತರು-ನಂದನ-ವನ
ತರುವ
ತರುವಾಯ
ತರು-ವು-ದಕ್ಕೆ
ತರೈ
ತರ್ಕ
ತರ್ಕ್ಯನದ್ವಯಂ
ತರ್ದ-ವಾಡಿ
ತಲಕಾಡ
ತಲ-ಕಾಡನ್ನು
ತಲಕಾಡಾದ
ತಲ-ಕಾ-ಡಿಗೆ
ತಲ-ಕಾ-ಡಿನ
ತಲ-ಕಾ-ಡಿನಲ್ಲಿ
ತಲ-ಕಾ-ಡಿನಲ್ಲಿದ್ದ
ತಲ-ಕಾ-ಡಿನಲ್ಲಿದ್ದ-ನೆಂದು
ತಲ-ಕಾ-ಡಿನಲ್ಲಿದ್ದಾಗ
ತಲ-ಕಾ-ಡಿನ-ವರೆಗೂ
ತಲ-ಕಾ-ಡಿನಿಂದ
ತಲ-ಕಾಡು
ತಲ-ಕಾಡು-ಗೊಂಡ
ತಲಗ-ವಾಡಿ
ತಲಗಾಳು-ಗೌಡ
ತಲ-ವನ-ಪುರ
ತಲಾ
ತಲುಪುತ್ತಿತ್ತೆಂದು
ತಲೆ
ತಲೆ-ಕಡಿದು-ಕೊಂಡು
ತಲೆ-ಕ-ಡಿಸಿ-ಕೊಂಡು
ತಲೆ-ಕಾ-ಡಿನ
ತಲೆ-ಕೆಳಾ-ಗಿದ್ದು
ತಲೆ-ಗ-ಳಾದ
ತಲೆ-ಗಳಿ-ಯಿಸಿ
ತಲೆ-ಗಳಿ-ಯಿಸಿ-ಕೊಂಡು
ತಲೆಗೆ
ತಲೆ-ಗೊಂಡ-ನಿ-ರದೆ
ತಲೆ-ತಲಾಂತರ-ವಾಗಿ
ತಲೆ-ದೋರಿ-ತೆಂದು
ತಲೆ-ನೆ-ರೆಯಲ
ತಲೆ-ಬಾಗಿರ-ಲಿಲ್ಲ-ವೆಂದು
ತಲೆ-ಮಾರಿಗೆ
ತಲೆ-ಮಾರಿನ
ತಲೆ-ಮಾರಿ-ನ-ವರು
ತಲೆ-ಮಾರಿ-ನ-ವರೆಗ
ತಲೆ-ಮಾರು
ತಲೆ-ಮಾರು-ಗಳ
ತಲೆ-ಮಾರು-ಗ-ಳನ್ನು
ತಲೆಯ
ತಲೆಯಂ
ತಲೆ-ಯನ್ನು
ತಲೆ-ಯ-ಮಾಳೆಯ
ತಲೆ-ಯ-ಮೇಲೆ
ತಲೆ-ಯ-ವ-ರಾದ
ತಲೆ-ಯೆತ್ತಿ
ತಲೆ-ಯೆತ್ತಿದ
ತಲ್ಲಿ
ತಳಕಾಡ
ತಳಕಾಡಂ
ತಳಕಾಡ-ಅಧಿ-ಕಾರಿ
ತಳಕಾಡ-ನಾಡ
ತಳಕಾಡ-ನಾಡನ್ನು
ತಳಕಾಡ-ನಾಡಪ್ರಭು
ತಳ-ಕಾಡನ್ನು
ತಳಕಾಡಪ್ರಭು
ತಳಕಾಡಸ್ತ-ಳದ
ತಳಕಾಡಾದ
ತಳ-ಕಾ-ಡಿನ
ತಳ-ಕಾಡು
ತಳ-ಕಾಡು-ಗೊಂಡ-ನೆಂದು
ತಳ-ಕಾಡು-ಗೊಂಡ-ನೆಂಬ
ತಳ-ಕಾಡು-ನಾಡು
ತಳ-ಕಾಡು-ರೊದ್ದ
ತಳ-ದಿಂದಂ
ತಳ-ಪಾ-ದಿಯ
ತಳ-ಪಾದಿ-ಯಲ್ಲಿ
ತಳ-ಪಾ-ದಿಯಲ್ಲಿ-ರುವ
ತಳಪಾ-ಯದ
ತಳಮಳಗ-ನಾಗಿ
ತಳ-ರದೆ
ತಳ-ಲೂರು
ತಳ-ವನ-ಪುರ
ತಳ-ವನ-ಪುರ-ತಲ-ಕಾಡು
ತಳ-ವನ-ಪುರ-ತಲ-ಕಾಡು-ವಿಜಯಸ್ಕಾಂದ-ವಾರ-ದಲ್ಲಿದ್ದಾಗ-ರಾಜ-ಧಾನಿ
ತಳ-ವನ-ಪುರವೇ
ತಳ-ವಳ-ಗ-ನಾಗಿ
ತಳ-ವಾರ
ತಳ-ವಾರ-ರಿಗೆ
ತಳ-ವಾರ-ರಿದ್ದರು
ತಳ-ವಾರಿಕೆ
ತಳ-ವಾರಿಕೆ-ಗ-ಳಿಂದ
ತಳ-ವಾರಿ-ಕೆಗೆ
ತಳ-ವಾರಿ-ಕೆಯ
ತಳ-ವಾರಿಕೆ-ಯಿಂದ
ತಳ-ವೃತ್ತಿ-ಯನ್ನು
ತಳವ್ರಿತ್ತಿಯಂ
ತಳಾರ
ತಳಾ-ರರು
ತಳಾರಿ
ತಳಾರಿ-ಕೆಯು
ತಳಿಗೆ
ತಳಿಗೆಯ
ತಳಿಗ್ರಾಮದ
ತಳಿ-ಯಣ್ಣ
ತಳಿಯೂ-ರನ್ನು
ತಳುಕು
ತಳುತಿ-ರಿದು
ತಳುತೆಸೆವವೀಶ್ವರ
ತಳುವ
ತಳುವ-ಕುಳೈನ್ದಾನ್
ತಳೆಕಾಡಂ
ತಳೆಕಾಡ-ಸೀಮೆಯ
ತಳೆ-ಕಾಡು-ಪಟ್ಟಣ
ತಳೆ-ಗೊಟ್ಟ
ತಳೆ-ಗೊಟ್ಟೀಮಹಿ
ತಳೆದ
ತಳೆದ-ನೆಂದು
ತಳೆ-ದಿದ್ದರು
ತಳೆದು
ತಳ್ತ
ತಳ್ತಾರ-ವೆ-ಯನ್ನು
ತಳ್ತಿ-ಯಣ್ಣ
ತಳ್ತಿಯದ
ತಳ್ತಿ-ರಿದು
ತಳ್ತಿ-ರಿದು-ಗುರ್ಬ್ಬಿ
ತಳ್ತಿರಿವೆಡೆಗೊರ್ವ್ವ-ರಪ್ಪೊಡಮಿ-ದಿರ್ಚುವ
ತಳ್ತಿರ್ದ್ದ
ತಳ್ಳಿಯದ
ತಳ್ಳಿಹಾಕ
ತಳ್ಳುತ್ತಿದ್ದಾಗ
ತವಡ
ತವಡಿ
ತವಡಿ-ಗ-ಳನ್ನು
ತವರದ
ತವರ-ದ-ಮಾರಿ-ಸೆಟ್ಟಿಯೂ
ತವಸಿಯ
ತವಿ-ಸಿ-ದನು
ತಸ್ಯ
ತಸ್ಯಾ
ತಸ್ಯಾತ್ಮಜೋ
ತಸ್ಯಾಪಿ
ತಸ್ಯಾಪ್ಯಾಸೀದ
ತಸ್ಯಾಸ್ಮೀನ್ಮಹಿಷೀ
ತಸ್ಯಾಸ್ಯ
ತಾ
ತಾಂ
ತಾಂಜಂ
ತಾಂಡ-ವ-ಮೂರ್ತಿ
ತಾಂಡ-ವೇಶ್ವರ
ತಾಂಡ-ವೇಶ್ವರನ
ತಾಂಬೂಲ
ತಾಂಬೂಲ-ಸೇವೆಯ
ತಾಂಬ್ರ-ಶಾಸಂ
ತಾಂಬ್ರ-ಶಾ-ಸನ
ತಾಂಬ್ರ-ಶಾ-ಸನಂ
ತಾಂಬ್ರ-ಶಾ-ಸನೇ
ತಾಂಬ್ರ-ಸಾ-ಸನಂ
ತಾಗ-ಲಾರಂಭಿ-ಸಿತು
ತಾಗಲೊಡನೆ
ತಾಗಿ
ತಾಗಿತ್ತು
ತಾಗುವ-ಲಗು
ತಾಗೆ
ತಾಣ
ತಾಣ-ಗ-ಳಾಗಿವೆ
ತಾಣ-ದೀ-ವಿಗೆಗೆ
ತಾಣ-ದೀ-ವಿಗೆ-ಯನ್ನು
ತಾಣ-ಪನ-ಹಳ್ಳಿ
ತಾತ
ತಾತ-ಚಾರ್ಯ-ನೆಂದು
ತಾತ-ಪಾರ್ಯನು
ತಾತಯ್ಯನ
ತಾತಾ-ಚಾರಿ-ಯ-ರ-ಡಿಗೆ
ತಾತಾ-ಚಾರ್ಯ
ತಾತಾ-ಚಾರ್ಯ-ತಿರು-ಮಲೆ
ತಾತಾ-ಚಾರ್ಯನ
ತಾತಾ-ಚಾರ್ಯ-ನಿಂದ
ತಾತಾ-ಚಾರ್ಯ-ನಿಗೆ
ತಾತಾ-ಚಾರ್ಯ-ನಿರ-ಬ-ಹುದು
ತಾತಾ-ಚಾರ್ಯನು
ತಾತಾ-ಚಾರ್ಯ-ನೆಂಬ
ತಾತಾ-ಚಾರ್ಯರ
ತಾತಾ-ಚಾರ್ಯ-ರದ್ದು
ತಾತಾ-ಚಾರ್ಯ-ರನ್ನು
ತಾತಾ-ಚಾರ್ಯ-ರಿಗೆ
ತಾತಾ-ಚಾರ್ಯರು
ತಾತಾರ್ಯ
ತಾತ್ಕಾಲಿ-ಕ-ವಾಗಿ
ತಾನಿತ್ತು
ತಾನು
ತಾನುಂ
ತಾನೂ
ತಾನೆ
ತಾನೇ
ತಾಮೊನೆ-ಬೆನ್ನಬಾ-ರನೆ
ತಾಮ್ರ
ತಾಮ್ರದ
ತಾಮ್ರ-ಪಟ-ಗ-ಳಲ್ಲಿ
ತಾಮ್ರ-ಪಟ-ಗ-ಳಿಂದ
ತಾಮ್ರ-ಪಟ-ಗಳಿವೆ
ತಾಮ್ರ-ಪಟ-ಗಳು
ತಾಮ್ರ-ಪಟದ
ತಾಮ್ರ-ಪಟ-ದಲ್ಲಿ
ತಾಮ್ರ-ಪಟದಲ್ಲಿದೆ
ತಾಮ್ರ-ಶಾ-ಶಾ-ಸನ-ಗ-ಳನ್ನು
ತಾಮ್ರ-ಶಾಸ-ಗಳು
ತಾಮ್ರ-ಶಾ-ಸನ
ತಾಮ್ರ-ಶಾ-ಸನಂ
ತಾಮ್ರ-ಶಾ-ಸನಕ್ಕಾಗಿ
ತಾಮ್ರ-ಶಾ-ಸನ-ಗಳ
ತಾಮ್ರ-ಶಾ-ಸನ-ಗಳನ್ನಲ್ಲದೆ
ತಾಮ್ರ-ಶಾ-ಸನ-ಗ-ಳನ್ನು
ತಾಮ್ರ-ಶಾ-ಸನ-ಗ-ಳಲ್ಲಿ
ತಾಮ್ರ-ಶಾ-ಸನ-ಗ-ಳಲ್ಲೂ
ತಾಮ್ರ-ಶಾ-ಸನ-ಗಳು
ತಾಮ್ರ-ಶಾ-ಸನದ
ತಾಮ್ರ-ಶಾ-ಸನ-ದ-ಅಲ್ಲಿ
ತಾಮ್ರ-ಶಾ-ಸನ-ದಲ್ಲಿ
ತಾಮ್ರ-ಶಾ-ಸನ-ದಲ್ಲಿದೆ
ತಾಮ್ರ-ಶಾ-ಸನ-ದಲ್ಲಿದ್ದಯ
ತಾಮ್ರ-ಶಾ-ಸನ-ದಲ್ಲೂ
ತಾಮ್ರ-ಶಾ-ಸನ-ದಿಂದ
ತಾಮ್ರ-ಶಾ-ಸನ-ವನ್ನು
ತಾಮ್ರ-ಶಾ-ಸನ-ವನ್ನೂ
ತಾಮ್ರ-ಶಾ-ಸನ-ವಾ-ಗಿದ್ದು
ತಾಮ್ರ-ಶಾ-ಸನ-ವಾಗಿ-ರ-ಬ-ಹುದು
ತಾಮ್ರ-ಶಾ-ಸನ-ವಿದ್ದು
ತಾಮ್ರ-ಶಾ-ಸನವು
ತಾಮ್ರ-ಶಾ-ಸನವೇ
ತಾಮ್ರ-ಶಾ-ಸನಸ್ಥ-ವಾದ
ತಾಮ್ರ-ಶಾ-ಸನಾರ್ಥಂ
ತಾಮ್ರ-ಶಾ-ಸನ್
ತಾಮ್ರ-ಸಾ-ಧನ-ವನ್ನು
ತಾಯ
ತಾಯಣ್ಣನು
ತಾಯ-ಲೂರಿನ
ತಾಯ-ಲೂರಿನಲ್ಲಿ-ರುವ
ತಾಯ-ಲೂರು
ತಾಯಿ
ತಾಯಿ-ಗಳ
ತಾಯಿ-ಗಳು
ತಾಯಿ-ಗಳೇ
ತಾಯಿಗೆ
ತಾಯಿ-ಮಂಚಿ-ಯಕ್ಕನ
ತಾಯಿಯ
ತಾಯಿ-ಯನ್ನು
ತಾಯಿ-ಯರ
ತಾಯಿ-ಯಾಗಿ-ರ-ಬ-ಹುದು
ತಾಯಿ-ಯಾದ
ತಾಯಿ-ಯಾದರೂ
ತಾಯಿ-ಯೆಂದೇ
ತಾಯಿಯೋ
ತಾಯಿ-ಸಾಕು-ತಾಯಿ
ತಾಯೂರು
ತಾಯ್ಮುದ್ದ-ರಸಿ
ತಾರ
ತಾರಾ
ತಾರಾಂಬರ
ತಾರಾಂಬರಂ
ತಾರಾಂಬಿ-ಕೆಗೆ
ತಾರೀಖನ್ನು
ತಾರೀಖಿನ
ತಾರೀಖಿನಂದು
ತಾರೀಖಿನಂದೇ
ತಾರೀಖು
ತಾರೀಖೆಂದು
ತಾರ್ಕಿಕ
ತಾಲ-ದೇವನು
ತಾಲೂಕು
ತಾಲೂಕು-ಗಳು
ತಾಲ್ಲು-ಕಿನ
ತಾಲ್ಲೂ-ಕನ್ನಾಗಿ
ತಾಲ್ಲೂ-ಕನ್ನು
ತಾಲ್ಲೂ-ಕನ್ನೂ
ತಾಲ್ಲೂಕಾಗಿದೆ
ತಾಲ್ಲೂಕಾ-ಗಿದ್ದು
ತಾಲ್ಲೂಕಿ
ತಾಲ್ಲೂಕಿಗೂ
ತಾಲ್ಲೂ-ಕಿಗೆ
ತಾಲ್ಲೂಕಿಗೇ
ತಾಲ್ಲೂಕಿನ
ತಾಲ್ಲೂಕಿ-ನಅ
ತಾಲ್ಲೂಕಿ-ನಲ್ಲಿ
ತಾಲ್ಲೂಕಿ-ನಲ್ಲಿದೆ
ತಾಲ್ಲೂಕಿ-ನಲ್ಲಿದ್ದರೆ
ತಾಲ್ಲೂಕಿ-ನಲ್ಲಿದ್ದ-ವೆಂದು
ತಾಲ್ಲೂಕಿ-ನಲ್ಲಿರು
ತಾಲ್ಲೂಕಿ-ನಲ್ಲಿ-ರುವ
ತಾಲ್ಲೂಕಿ-ನಲ್ಲಿವೆ
ತಾಲ್ಲೂಕು
ತಾಲ್ಲೂಕು-ಗಳ
ತಾಲ್ಲೂಕು-ಗ-ಳನ್ನು
ತಾಲ್ಲೂಕು-ಗಳನ್ನೊಳ-ಗೊಂಡ
ತಾಲ್ಲೂಕು-ಗ-ಳಲ್ಲಿ
ತಾಲ್ಲೂಕು-ಗಳಲ್ಲಿದ್ದ
ತಾಲ್ಲೂಕು-ಗಳ-ವೆ-ರೆಗೆ
ತಾಲ್ಲೂಕು-ಗ-ಳಾಗಿ
ತಾಲ್ಲೂಕು-ಗ-ಳಾದ
ತಾಲ್ಲೂಕು-ಗಳಿದ್ದವು
ತಾಲ್ಲೂಕು-ಗಳು
ತಾಲ್ಲೂಕು-ಗಳೂ
ತಾಲ್ಲೂಕು-ಗಳೆಂಬ
ತಾಲ್ಲೂಕು-ವಾರು
ತಾಲ್ಲೂಕ್ನಲ್ಲಿ
ತಾಳ
ತಾಳ-ಗುಂದ
ತಾಳಗ್ರಾಹಿ
ತಾಳ-ತಿಟ್ಟು
ತಾಳೆ-ಯ-ಕೆರೆ
ತಾಳ್ತುವಾ-ಯರಾ-ಕೆರೆ
ತಾಳ್ದಿದಂ
ತಾಳ್ದಿದರ್ಜ್ಜಗ
ತಾವರೆ-ಕಟ್ಟೆಯ
ತಾವರೆ-ಕೆರೆಯ
ತಾವ-ರೆಯ
ತಾವ-ರೆಯ-ಕೆರೆಯ
ತಾವಾಗಿಯೇ
ತಾವು
ತಾವೂ
ತಾವೇ
ತಿಂಗಳ
ತಿಂಗ-ಳಲ್ಲಿ
ತಿಂಗಳ-ವರೆಗೆ
ತಿಂಗಳಿಂಗೆ
ತಿಂಗ-ಳಿಗೆ
ತಿಂಗಳಿ-ಗೊಮ್ಮೆ
ತಿಂಗಳಿ-ನಲ್ಲಿ
ತಿಂಗಳು
ತಿಂತ್ರಣೀಕ
ತಿಂತ್ರಿಣಿಕ-ಗಚ್ಛದ
ತಿಂತ್ರಿಣೀ
ತಿಂತ್ರಿಣೀಕ
ತಿಂತ್ರಿಣೀ-ಕ-ಗಚ್ಛ
ತಿಂತ್ರಿಣೀ-ಕ-ಗಚ್ಛದ
ತಿಂತ್ರೀಣಿಕ
ತಿಂದ-ಹಾಗೆ
ತಿಂಮಂಣ್ನ
ತಿಂಮಣ
ತಿಂಮಣ್ಣ
ತಿಂಮ-ನಾಯ-ಕನ
ತಿಂಮ-ನಿಗೆ
ತಿಂಮನು
ತಿಂಮಪ-ನಾಯಕ
ತಿಂಮಪೈಯ್ಯನವ
ತಿಂಮ-ರ-ಸರು
ತಿಂಮ-ರಾಜ-ಗಳ
ತಿಂಮವ್ವೆಯು
ತಿಂಮ-ಸ-ಮುದ್ರ-ವಾದ
ತಿಇರಿ
ತಿಕ್ಕಮ
ತಿಕ್ಕಮ-ನನ್ನು
ತಿಕ್ಕಾಟ
ತಿಗಡ-ಹಳ್ಳಿ
ತಿಗಳ
ತಿಗಳನು
ತಿಗಳರು
ತಿಗುಳನ
ತಿಗುಳ-ನ-ಕೆರೆ
ತಿಗುಳರ
ತಿಗುಳರು
ತಿಗುಳರೇ
ತಿಟ್ಟು
ತಿಥಿ
ತಿದ್ದಲು
ತಿದ್ದಿ
ತಿದ್ದಿ-ಕೊಟ್ಟ
ತಿದ್ದಿ-ಕೊಳ್ಳುತ್ತೇನೆ
ತಿದ್ದಿ-ದಂತೆ
ತಿದ್ದಿಸಿ
ತಿದ್ದಿ-ಸಿ-ದರು
ತಿದ್ದಿ-ಸಿ-ಪುನಃ
ತಿದ್ದಿ-ಸಿ-ರ-ಬ-ಹುದು
ತಿದ್ದು-ವಲ್ಲಿ
ತಿದ್ದು-ವಲ್ಲಿಗೆ
ತಿನ-ರಸಿ-ಪುರ
ತಿನ-ರಸೀ-ಪುರ
ತಿನ-ರಸೀ-ಪುರದ
ತಿನಿ-ರಾಯ
ತಿನಿ-ರಾಯಾಂಬಾಮುಲ
ತಿಪಟೂರು
ತಿಪ್ಪಣ್ಣ
ತಿಪ್ಪಣ್ಣ-ನಾಯ-ಕ-ನೆಂಬು-ನನೂ
ತಿಪ್ಪಣ್ಣ-ನಾಯ-ಕ-ನೆಂಬು-ವ-ವನು
ತಿಪ್ಪಣ್ಣ-ನಾಯ-ಕ-ರಿಗೆ
ತಿಪ್ಪಣ್ಣ-ನಾಯ-ಕರು
ತಿಪ್ಪನೆ
ತಿಪ್ಪಯ್ಯ
ತಿಪ್ಪಯ್ಯನ
ತಿಪ್ಪಯ್ಯನು
ತಿಪ್ಪ-ರ-ಸರು
ತಿಪ್ಪ-ರಸಾರ್ಯನ
ತಿಪ್ಪ-ರಸಾರ್ಯನ-ಪುತ್ರ
ತಿಪ್ಪಳಿ-ನಾಯ-ಕನು
ತಿಪ್ಪವ್ವೆ
ತಿಪ್ಪವ್ವೆಗೂ
ತಿಪ್ಪವ್ವೆಗೆ
ತಿಪ್ಪವ್ವೆ-ಯನ್ನು
ತಿಪ್ಪಾಜಿ
ತಿಪ್ಪಾಜಿಗೆ
ತಿಪ್ಪಾಜಿಯ
ತಿಪ್ಪೂರ
ತಿಪ್ಪೂ-ರನ್ನು
ತಿಪ್ಪೂ-ರಿಗೂ
ತಿಪ್ಪೂ-ರಿಗೆ
ತಿಪ್ಪೂರಿನ
ತಿಪ್ಪೂರಿನಲ್ಲಿದ್ದ
ತಿಪ್ಪೂರು
ತಿಪ್ಪೂರು-ಗಳ
ತಿಪ್ಪೂರೇ
ತಿಪ್ಪೆ-ಗಳ
ತಿಪ್ಪೆ-ಗ-ಳನ್ನು
ತಿಪ್ಪೆಯ
ತಿಪ್ಪೆಯ-ಗುಳ
ತಿಪ್ಪೆಯ-ಗೊಬ್ಬ-ರದ
ತಿಪ್ಪೆ-ಯೂರಿನ
ತಿಪ್ಪೆ-ಯೂರು
ತಿಪ್ಪೆರು-ವಳ್ಳಿಯ
ತಿಪ್ಪೆರೂ-ರನ್ನು
ತಿಪ್ಪೆರೂ-ರಿಗೆ
ತಿಪ್ಪೆರೂರಿನ
ತಿಪ್ಪೆರೂರು
ತಿಬ್ಬನ-ಹಳ್ಳಿ
ತಿಬ್ಬನ-ಹಳ್ಳಿಗೆ
ತಿಬ್ಬನ-ಹಳ್ಳಿಯ
ತಿಬ್ಬನ-ಹಳ್ಳಿ-ಯನ್ನು
ತಿಬ್ಬ-ಸೆಟ್ಟಿಯ
ತಿಬ್ಬ-ಸೆಟ್ಟಿ-ಯರ
ತಿಬ್ಬ-ಸೆಟ್ಟಿ-ಯ-ವರ
ತಿಬ್ಬ-ಸೆಟ್ಟಿಯು
ತಿಬ್ಬಾ-ದೇವಿ
ತಿಮ್ಮ
ತಿಮ್ಮ-ಅ-ರಸಯ್ಯ
ತಿಮ್ಮ-ಕವಿ
ತಿಮ್ಮ-ಕವಿಯ
ತಿಮ್ಮ-ಜಗ-ದೇವ-ರಾಯನ
ತಿಮ್ಮಣ್ಣ
ತಿಮ್ಮಣ್ಣ-ದಂಡ-ನಾಯ-ಕನ
ತಿಮ್ಮಣ್ಣ-ದಂಡ-ನಾಯ-ಕನು
ತಿಮ್ಮಣ್ಣ-ದಂಡೇಠೋ
ತಿಮ್ಮಣ್ಣನ
ತಿಮ್ಮಣ್ಣ-ನಾಯಕ
ತಿಮ್ಮಣ್ಣನು
ತಿಮ್ಮಣ್ಣನ್ನು
ತಿಮ್ಮನ
ತಿಮ್ಮ-ನ-ಹಳ್ಳ-ಗ-ಳನ್ನು
ತಿಮ್ಮ-ನಾಯ-ಕನ
ತಿಮ್ಮ-ನಾಯ-ಕನು
ತಿಮ್ಮನು
ತಿಮ್ಮಪ್ಪ
ತಿಮ್ಮಪ್ಪ-ಗಳ
ತಿಮ್ಮಪ್ಪನ
ತಿಮ್ಮಪ್ಪ-ನಾಯಕ
ತಿಮ್ಮಪ್ಪ-ನಾ-ಯ-ಕರು
ತಿಮ್ಮಪ್ಪ-ನಿಗೆ
ತಿಮ್ಮಪ್ಪಯ್ಯ
ತಿಮ್ಮ-ಬೋಯಿ
ತಿಮ್ಮಮ್ಮ
ತಿಮ್ಮ-ಯ-ದೇವ
ತಿಮ್ಮಯ್ಯ
ತಿಮ್ಮಯ್ಯ-ದೇವ
ತಿಮ್ಮಯ್ಯನ
ತಿಮ್ಮಯ್ಯನ್ನು
ತಿಮ್ಮ-ರಸ
ತಿಮ್ಮ-ರ-ಸನ
ತಿಮ್ಮ-ರ-ಸನು
ತಿಮ್ಮ-ರಸಯ್ಯನು
ತಿಮ್ಮ-ರಸರ
ತಿಮ್ಮ-ರ-ಸರು
ತಿಮ್ಮ-ರಾಜ
ತಿಮ್ಮ-ರಾಜನ
ತಿಮ್ಮ-ರಾಜ-ನಿಗೆ
ತಿಮ್ಮ-ರಾಜನು
ತಿಮ್ಮ-ರಾ-ಜಯ್ಯ
ತಿಮ್ಮ-ರಾಜು
ತಿಮ್ಮ-ಸ-ಮುದ್ರ
ತಿಮ್ಮಾಂಬ
ತಿಮ್ಮಾಂಬೆ-ಯರ
ತಿಮ್ಮಾರ್ಯ-ನೆಂಬ
ತಿರ-ಣದ
ತಿರಸ್ಕರಿ-ಸಿದ್ದಾರೆ
ತಿರಿಕಂಣ್ನದರ
ತಿರಿ-ನಂದನ-ವನ-ವನ್ನು
ತಿರಿ-ನಂದನ-ವನ್ನು
ತಿರಿ-ನಾಮದ
ತಿರಿನಾಳ
ತಿರಿನಾಳು
ತಿರಿ-ಮಂಣ
ತಿರಿಮಣ್ಣ
ತಿರಿ-ಮಣ್ಣು
ತಿರಿಯಮ್ಮನು
ತಿರಿವ-ರಂಗ
ತಿರಿವೆಸ
ತಿರಿವೆಸೆಕ
ತಿರು-ಕುಡಿ
ತಿರು-ಕುಲ-ದ-ವರು
ತಿರುಕೋಯಿ-ಲೂರ್
ತಿರು-ಗನ-ಹಳ್ಳಿ-ಯಲ್ಲಿ
ತಿರುಗಾಡುತ್ತಾ
ತಿರುಗಿ
ತಿರುಗಿ-ದರೆ
ತಿರುಗಿ-ಬಿದ್ದ
ತಿರುಗಿ-ಬಿದ್ದನು
ತಿರುಗಿ-ಬಿದ್ದ-ರೆಂದೂ
ತಿರುಗಿ-ಬಿದ್ದಿರುವ
ತಿರುಗಿ-ಬಿದ್ದು
ತಿರುಗುತ್ತಿದ್ದ
ತಿರುಗುವ
ತಿರುಚನಾ-ಪಳ್ಳಿ-ಯಿಂದ
ತಿರುಚಿರಾ-ಪಳ್ಳಿ
ತಿರುಣ-ನಾಯಕ
ತಿರುಣ-ನಾಯ-ಕರು
ತಿರುತಂಬುಲ
ತಿರು-ನಂದನ-ವನಕ್ಕೆ
ತಿರು-ನಂದನ-ವನ-ವನ್ನು
ತಿರುನಂದಾ
ತಿರು-ನಂದಾ-ದೀ-ಪಕ್ಕೆ
ತಿರುನಂದಾ-ದೀಪದ
ತಿರುನಂದಾ-ದೀಪ-ವನ್ನು
ತಿರುನಂದಾ-ವನ
ತಿರುನಂದಾ-ವನ-ದಲಿ
ತಿರುನಂದಾ-ವನ-ವನ್ನು
ತಿರು-ನಕ್ಷತ್ರ
ತಿರು-ನಕ್ಷತ್ರಕ್ಕೆ
ತಿರು-ನಕ್ಷತ್ರದ
ತಿರು-ನಕ್ಷತ್ರ-ದಲ್ಲಿ
ತಿರು-ನನ್ದವಿಳಕ್ಕಿಕ್ಕು
ತಿರು-ನನ್ದಾವಿಳಕ್ಕುಕ್ಕು
ತಿರು-ನರೈ-ಯೂರ್
ತಿರು-ನಾ-ರಾಯಣ
ತಿರು-ನಾ-ರಾಯ-ಣ-ದ-ಪುರದ
ತಿರು-ನಾ-ರಾಯ-ಣ-ದೇವರ
ತಿರು-ನಾ-ರಾಯ-ಣ-ದೇವ-ರಿಗೆ
ತಿರು-ನಾ-ರಾಯ-ಣನ್
ತಿರು-ನಾ-ರಾಯ-ಣ-ಪುರ
ತಿರು-ನಾ-ರಾಯ-ಣ-ಪುರಕ್ಕೆ
ತಿರು-ನಾ-ರಾಯ-ಣ-ಪುರದ
ತಿರು-ನಾ-ರಾಯ-ಣ-ಪುರ-ದಲ್ಲಿ
ತಿರು-ನಾ-ರಾಯ-ಣ-ಪುರ-ವಾಗಿ
ತಿರು-ನಾ-ರಾಯ-ಣ-ಪುರ-ವಾದ
ತಿರು-ನಾ-ರಾಯ-ಣ-ಪೆರು-ಮಾಳಿಗೆ
ತಿರುನಾಳ
ತಿರುನಾಳದ
ತಿರುನಾಳಿಗೆ
ತಿರುನಾಳಿನ
ತಿರುನಾಳಿನಲ್ಲಿ
ತಿರುನಾಳು
ತಿರುನಾಳ್
ತಿರುನಾಳ್ಗೆ
ತಿರು-ನೆತ್ತಿ
ತಿರುಪಂಣ್ಯಾರ
ತಿರುಪಂಣ್ಯಾರಕ್ಕೆ
ತಿರು-ಪಡಿ-ವಾಳ
ತಿರು-ಪತಿ
ತಿರು-ಪತಿಯ
ತಿರು-ಪತಿ-ಯಲ್ಲಿ
ತಿರುಪಿಕ್ಕೂ-ಡಮ್
ತಿರುಪ್ರತಿಷ್ಠೆ
ತಿರುಪ್ರತಿಷ್ಠೆಗೆ
ತಿರುಪ್ರತಿಷ್ಠೆ-ಯನ್ನು
ತಿರುಪ್ರತಿಷ್ಠೈ
ತಿರುಮಂಜನ
ತಿರುಮಂಜನಕ್ಕೆ
ತಿರು-ಮಂಟಪದ
ತಿರು-ಮಂಡ-ಪಮ್ನ್ನು
ತಿರುಮ-ಕೂ-ಡಲು
ತಿರುಮ-ಕೂಡ-ಲು-ನ-ರಸೀ-ಪುರ
ತಿರುಮ-ಕೂಡು
ತಿರು-ಮಣ್ಣನ್ನು
ತಿರುಮಣ್ಣಿ-ಗಾಗಿ
ತಿರುಮಣ್ಣಿಗೋಸ್ಕರ
ತಿರು-ಮಣ್ಣು
ತಿರು-ಮಲ
ತಿರು-ಮಲ-ಗಿರಿ
ತಿರು-ಮಲ-ಗಿರಿ-ನ-ಗರಿಯು
ತಿರು-ಮಲ-ಗಿರಿಯ
ತಿರು-ಮಲ-ತಾತಾ-ಚಾರಿಯ
ತಿರು-ಮಲ-ದೀಕ್ಷಿತನ
ತಿರು-ಮಲ-ದೇವ
ತಿರು-ಮಲ-ದೇವರ
ತಿರು-ಮಲ-ದೇವ-ರ-ನರ-ಸಿಂಹ
ತಿರು-ಮಲ-ದೇವ-ರಿಗೆ
ತಿರು-ಮಲ-ದೇವ-ರುತಪಸೀ-ರಾಯ
ತಿರು-ಮಲನ
ತಿರು-ಮಲ-ನನ್ನು
ತಿರು-ಮಲ-ನಾಥ
ತಿರು-ಮಲ-ನಾಥನ
ತಿರು-ಮಲ-ನಾಥನು
ತಿರು-ಮಲ-ನಾಯ-ಕನ
ತಿರು-ಮಲ-ನಿಂದ
ತಿರು-ಮಲನು
ತಿರು-ಮಲ-ನೆಂಬ
ತಿರು-ಮಲನೇ
ತಿರುಮ-ಲಮ್ಮ
ತಿರು-ಮಲ-ಯಾರ್ಯೋವ್ಯತಾನೀತ್ತಾಂಬ್ರ
ತಿರು-ಮಲಯ್ಯ
ತಿರು-ಮಲಯ್ಯಂಗಾರರ
ತಿರು-ಮಲಯ್ಯ-ನ-ವರು
ತಿರು-ಮಲಯ್ಯನೇ
ತಿರು-ಮಲ-ರಾಜ
ತಿರು-ಮಲ-ರಾಜನ
ತಿರು-ಮಲ-ರಾಜ-ನನ್ನು
ತಿರು-ಮಲ-ರಾಜ-ನಾಯ-ಕಗೆ
ತಿರು-ಮಲ-ರಾಜ-ನಿಗೂ
ತಿರು-ಮಲ-ರಾಜನು
ತಿರು-ಮಲ-ರಾಜ-ನೆಂದು
ತಿರು-ಮಲ-ರಾಜ-ಯ-ದೇವ
ತಿರು-ಮಲ-ರಾ-ಜಯ್ಯ
ತಿರು-ಮಲ-ರಾಜಯ್ಯ-ದೇವ
ತಿರು-ಮಲ-ರಾಜಯ್ಯನ
ತಿರು-ಮಲ-ರಾಜಯ್ಯ-ನ-ವರ
ತಿರು-ಮಲ-ರಾಜಯ್ಯ-ನ-ವರು
ತಿರು-ಮಲ-ರಾಜಯ್ಯನು
ತಿರು-ಮಲ-ರಾಜಯ್ಯ-ನೆಂಬ
ತಿರು-ಮಲ-ರಾಜಯ್ಯನೇ
ತಿರು-ಮಲ-ರಾಜರು
ತಿರು-ಮಲ-ರಾಜ-ರು-ಗಳು
ತಿರು-ಮಲ-ರಾಜು
ತಿರು-ಮಲ-ರಾಯ
ತಿರು-ಮಲ-ರಾಯನ
ತಿರು-ಮಲ-ರಾಯರ
ತಿರು-ಮಲ-ಸಾ-ಗರ
ತಿರು-ಮಲ-ಸಾ-ಗರ-ಛತ್ರದ
ತಿರು-ಮಲಾ-ಚಾರ್ಯ
ತಿರು-ಮಲಾ-ಚಾರ್ಯನ
ತಿರು-ಮಲಾ-ಚಾರ್ಯನು
ತಿರು-ಮಲಾ-ಚಾರ್ಯರ
ತಿರು-ಮಲಾ-ಚಾರ್ಯರು
ತಿರು-ಮಲಾರ್ಯ
ತಿರು-ಮಲಾರ್ಯನ
ತಿರು-ಮಲಾರ್ಯ-ನನ್ನು
ತಿರು-ಮಲಾರ್ಯ-ನಿಂದ
ತಿರು-ಮಲಾರ್ಯನು
ತಿರು-ಮಲಾರ್ಯ-ನೆಂದು
ತಿರು-ಮಲಾರ್ಯನೇ
ತಿರು-ಮಲೆ
ತಿರು-ಮಲೆ-ಗಳು
ತಿರು-ಮಲೆಗೆ
ತಿರು-ಮಲೆಯ
ತಿರು-ಮಲೆ-ಯಪ್ಪ
ತಿರು-ಮಲೆ-ಯಲ್ಲೇ
ತಿರು-ಮಲೆ-ಯ-ವ-ನಾ-ಗಿದ್ದು
ತಿರು-ಮಲೆ-ಯ-ವನೇ
ತಿರು-ಮಲೆ-ಯಾ-ಚಾರ್ಯ
ತಿರು-ಮಲೆ-ಯಾ-ಚಾರ್ಯನು
ತಿರು-ಮಲೆ-ಯಾ-ಚಾರ್ಯೇ-ಣದಂ
ತಿರು-ಮಲೆ-ಯಾ-ಚಾರ್ಯೇಣೇ-ದನ್ತಾಮ್ರ
ತಿರು-ಮಲೆ-ಯಾ-ಚಾರ್ಯ್ಯೇಣೇಮ
ತಿರು-ಮಲೆ-ಯಾರ್ಯನು
ತಿರು-ಮಲೆ-ಯಾರ್ಯರ
ತಿರು-ಮಲೆಯು
ತಿರು-ಮಲೈ-ಯಂಗಾರರ
ತಿರು-ಮಲೈಯ್ಯಂಗಾರರ
ತಿರುಮಳಾ-ಚಾರ್ಯ
ತಿರು-ಮಾಲೆ
ತಿರು-ಮಾಲೆಗೆ
ತಿರು-ಮಾಲೆ-ದಾನಕ್ಕೆ
ತಿರು-ಮಾ-ಳಿಗೆ-ಯಲ್ಲಿ
ತಿರು-ಮಾಳೆ
ತಿರುಮುರ್ರಮ್
ತಿರುಳು-ನಾಡು
ತಿರುವ-ಡಿ-ಗ-ಳನ್ನು
ತಿರುವಣ್ಣಾ-ಮಲೆ
ತಿರುವಣ್ಣಾ-ಮಲೆಗೂ
ತಿರುವಣ್ಣಾ-ಮಲೆ-ಯನ್ನೇ
ತಿರುವತ್ತಿ-ಯೂರಿನ
ತಿರುವತ್ತಿಯೂರು
ತಿರುವಧ್ಯಾನ
ತಿರುವಧ್ಯಾನಕ್ಕಾಗಿ
ತಿರುವ-ನಂತ-ಪುರದ
ತಿರುವ-ಮೃದಿಂಗೆ
ತಿರುವ-ಮೃದಿಙ್ಗೆ
ತಿರುವ-ರಂಗ
ತಿರುವ-ರಂಗ-ದಾಸ
ತಿರುವ-ರಂಗ-ದಾಸನ
ತಿರುವ-ರಂಗ-ದಾಸ-ನಿಗೂ
ತಿರುವ-ರಂಗ-ದಾಸ-ನಿಗೆ
ತಿರುವ-ರಂಗ-ದಾಸನು
ತಿರುವ-ರಂಗ-ನಾ-ರಾಯಣ
ತಿರುವ-ರಾಧ-ನೆಯ
ತಿರುವ-ರಾಧಾ-ನೆಗೆ
ತಿರುವ-ರುಂಗ
ತಿರುವ-ರುಂಗ-ದಾಸ
ತಿರುವ-ರುಂಗ-ದಾಸನ
ತಿರುವ-ಲಾರ್ಯ
ತಿರುವಲ್ಲಾಳ
ತಿರುವವ್ವೆ
ತಿರುವಾ-ಭರಣ
ತಿರುವಾ-ಭರ-ಣ-ದೇವರ
ತಿರುವಾಯ್
ತಿರುವಾಯ್ಮೋಳಿ
ತಿರುವಾಯ್ಮೋಳಿಯ
ತಿರುವಾಯ್ಮೋಳಿ-ಯನ್ನು
ತಿರುವಾರಾಧ-ನೆ-ಗಾಗಿ
ತಿರುವಾರಾಧ-ನೆಗೆ
ತಿರುವಾರಾಧ-ನೆಯ
ತಿರುವಿ
ತಿರುವಿಂದಳೂರ
ತಿರುವಿ-ಡಿಯಾಟಕೆ
ತಿರುವಿ-ಡಿಯಾಟಕ್ಕೆ
ತಿರುವಿ-ಡಿಯಾಟದ
ತಿರುವಿ-ಡಿಯಾಟ್ಟಕ್ಕೆ
ತಿರುವಿ-ಡಿ-ಯಾರ್ಥ-ವಾಗಿ
ತಿರುವಿ-ಡೆಯಾಟ್ಟಕ್ಕೆ
ತಿರುವಿ-ಡೈಯಾಟ್ಟಕ್ಕೆ
ತಿರುವಿ-ಡೈಯಾಟ್ಟ-ಗ-ಳಿಗೆ
ತಿರುವಿ-ಡೈಯಾಟ್ಟದ
ತಿರುವಿ-ಶಾಖಾ
ತಿರು-ವಿಷ್ಣು
ತಿರುವುರಲಿ
ತಿರು-ವೆಂಕಟ-ನಾಥ-ನಿಗೆ
ತಿರು-ವೆಂಕಟ-ನಾಯಕ
ತಿರು-ವೆಂಕಟಾದ್ರಿ
ತಿರುವೆಂಗಟಯ್ಯನ
ತಿರುವೆಂಗ-ಡಮ್ಮ
ತಿರುವೆಂಗಳ-ನಾಥ
ತಿರುವೆಂಗಳ-ನಾಥನ
ತಿರುವೆಡೆಯಾಟದ
ತಿರುವೇಂಕಟ-ನಾಯ-ಕನು
ತಿರುವೇಂಕಟಪ-ನಾಯ-ಕನು
ತಿರೆ
ತಿಲಕ
ತಿಲಕ-ನೆನಿಸಿದ
ತಿಲಕ-ರಂತೆ
ತಿಲಕ-ರಾದ
ತಿಲಕರು
ತಿಲಕ-ವೆಂಬ
ತಿಲಿ
ತಿಲಿ-ಕೂತ್ತಾಂಡಿ
ತಿಲೆ
ತಿಲೆ-ನಾಯಕ
ತಿಲೈಕೂತ್ತ
ತಿಲೈಕೂತ್ತನ್
ತಿಲೈಕೂತ್ತ-ವಿಣ್ನಘರ
ತಿಲೈಕೂತ್ತ-ವಿಣ್ನಘರ್
ತಿಲ್ಲೆಕೂತ್ತ
ತಿಲ್ಲೆಕೂತ್ತ-ವಿಣ್ಣಘರ್ಕಾರಿ-ಕುಡಿ
ತಿಲ್ಲೆಕೂತ್ತ-ವಿಣ್ನಘರಂ
ತಿಲ್ಲೈಕೂತ್ತನ್
ತಿಲ್ಲೈಕೂತ್ತ-ವಿಣ್ಣ-ಗರ್
ತಿಲ್ಲೈಕೂತ್ತವಿಣ್ಣಘರ್
ತಿಳಂಕಂಗೀ
ತಿಳಕ
ತಿಳಕಂಗೀ
ತಿಳಕದ
ತಿಳಕ-ನು-ಮಪ್ಪ
ತಿಳಖ
ತಿಳದು-ಬರುತ್ತದೆ
ತಿಳಿದ
ತಿಳಿದ-ಬರುತ್ತದೆ
ತಿಳಿದ-ವನೂ
ತಿಳಿದ-ವರು
ತಿಳಿದ-ವರೇ
ತಿಳಿ-ದಿದೆ
ತಿಳಿ-ದಿದೆ-ಯಾಗಿ
ತಿಳಿ-ದಿರುವ
ತಿಳಿ-ದಿ-ರುವುದು
ತಿಳಿದು
ತಿಳಿದು-ಕೊಂಡಿದ್ದರು
ತಿಳಿದು-ಕೊಳ್ಳ-ಬ-ಹುದು
ತಿಳಿದು-ಕೊಳ್ಳು-ವುದು
ತಿಳಿದು-ಬಂದರೆ
ತಿಳಿದು-ಬರುತ್ತದೆ
ತಿಳಿದು-ಬರುತ್ತದೆ-ಲಾಳ-ನ-ಕೆರೆಯ
ತಿಳಿದು-ಬ-ರುತ್ತವೆ
ತಿಳಿದು-ಬರುತ್ತೆ
ತಿಳಿದು-ಬರುದೆ
ತಿಳಿದು-ಬ-ರುವ
ತಿಳಿದು-ಬ-ರುವು-ದ-ರಿಂದ
ತಿಳಿದು-ಬ-ರು-ವು-ದಿಲ್ಲ
ತಿಳಿದು-ಬ-ರುವುದು
ತಿಳಿದು-ಬು-ರುವಿ-ದಿಲ್ಲ
ತಿಳಿದೇ
ತಿಳಿಯದ
ತಿಳಿ-ಯದು
ತಿಳಿ-ಯದೆಂದು
ತಿಳಿಯ-ಬ-ಹುದು
ತಿಳಿಯ-ಬಾ-ರದು
ತಿಳಿ-ಯಾ-ಗಿದ್ದು
ತಿಳಿ-ಯಾದ
ತಿಳಿ-ಯಿತು
ತಿಳಿಯುತ್ತದೆ
ತಿಳಿಯುತ್ತ-ದೆಂದು
ತಿಳಿಯುತ್ತವೆ
ತಿಳಿಯು-ವು-ದಿಲ್ಲ
ತಿಳಿ-ಯು-ವುದು
ತಿಳಿ-ವಳಿಕೆ
ತಿಳಿಸ-ಬೇಕು
ತಿಳಿಸ-ಲಾಗಿದೆ
ತಿಳಿಸಿ
ತಿಳಿಸಿ-ಕೊಡುತ್ತವೆ
ತಿಳಿಸಿದ
ತಿಳಿಸಿ-ದಂತೆ
ತಿಳಿಸಿ-ದನು
ತಿಳಿಸಿ-ದ-ನೆಂದೂ
ತಿಳಿ-ಸಿದೆ
ತಿಳಿಸಿ-ರು-ವಂತೆ
ತಿಳಿ-ಸುತ್ತದೆ
ತಿಳಿ-ಸುತ್ತ-ದೆಂದು
ತಿಳಿ-ಸುತ್ತವೆ
ತಿಳಿ-ಸುತ್ತಿದೆ
ತಿಳಿಸುತ್ತಿದ್ದರು
ತಿಳಿ-ಸುವ
ತಿಳಿ-ಸು-ವಂತೆ
ತಿಳಿ-ಸು-ವಲ್ಲಿ
ತಿಳ್ಳಯ್ಯ
ತಿವಡಿ-ಸೆಟ್ಟಿ
ತಿವಿದು
ತಿಷ್ಟೇಕ
ತೀತಳ-ಮರಿ-ವನ್ತು
ತೀರ
ತೀರದ
ತೀರ-ದಲ್ಲಿ
ತೀರ-ದಲ್ಲಿತ್ತು
ತೀರ-ದಲ್ಲಿದ್ದ
ತೀರ-ದಲ್ಲಿದ್ದು
ತೀರ-ದಲ್ಲಿ-ರುವ
ತೀರ-ದ-ವರೆಗೂ
ತೀರದು
ತೀರಪ್ರ-ದೇಶ
ತೀರ-ವಾಗಿ-ರ-ಬ-ಹುದು
ತೀರ-ವಾದ
ತೀರವು
ತೀರಾ
ತೀರಾದ್ದಕ್ಷಿಣಸ್ಯಾಂ
ತೀರಿ-ಕೊಂಡನು
ತೀರಿ-ಕೊಂಡ-ನೆಂದು
ತೀರಿ-ಕೊಂಡಾಗ
ತೀರಿ-ಕೊಂಡಿದ್ದನು
ತೀರಿ-ಕೊಂಡಿದ್ದ-ನೆಂದು
ತೀರಿ-ಕೊಂಡಿರ-ಬೇಕು
ತೀರಿ-ಕೊಳ್ಳಲು
ತೀರಿ-ಸಿ-ಕೊಳ್ಳುವ
ತೀರಿ-ಹೋ-ಯಿತು
ತೀರುವಳಿ
ತೀರ್ತ್ತ
ತೀರ್ತ್ಥಂ
ತೀರ್ತ್ಥ-ದೊಳು
ತೀರ್ತ್ಥವ
ತೀರ್ತ್ಥವಲ್ಲರೇ
ತೀರ್ತ್ಥಾವಗಾ-ಹನ
ತೀರ್ಥ
ತೀರ್ಥಂಕರ
ತೀರ್ಥಂಕ-ರರ
ತೀರ್ಥ-ಎಂದು
ತೀರ್ಥ-ಕಂಬ-ದ-ಹಳ್ಳಿ
ತೀರ್ಥಕ್ಕೆ
ತೀರ್ಥ-ಗ-ಳನ್ನೂ
ತೀರ್ಥ-ಗಳು
ತೀರ್ಥದ
ತೀರ್ಥ-ದಲ್ಲಿ
ತೀರ್ಥ-ದಲ್ಲಿದ್ದ
ತೀರ್ಥಪ್ರಸಾದ-ವನ್ನು
ತೀರ್ಥ-ಯಾತ್ರಾದಿ-ಗ-ಳಲ್ಲಿ
ತೀರ್ಥರ
ತೀರ್ಥ-ವನ್ನಾಗಿ
ತೀರ್ಥ-ವನ್ನು
ತೀರ್ಥ-ವಾಗಿತ್ತೆಂದು
ತೀರ್ಥ-ವಾ-ಗಿದ್ದು
ತೀರ್ಥ-ವಾಗಿ-ರ-ಬ-ಹುದು
ತೀರ್ಥವು
ತೀರ್ಥ-ವು-ಇಂದಿನ
ತೀರ್ಥ-ವೆಂದು
ತೀರ್ಥ-ವೆಂಬ
ತೀರ್ಮಾನ
ತೀರ್ಮಾನಕ್ಕೆ
ತೀರ್ಮಾನ-ದಂತೆ
ತೀರ್ಮಾನಿ-ಸ-ಬ-ಹುದು
ತೀರ್ಮಾ-ನಿ-ಸಿದ-ರೆಂದು
ತೀರ್ಮಾನಿ-ಸುತ್ತಾರೆ
ತೀವ್ರ
ತು
ತುಂಗಗವೇಂದ್ರ
ತುಂಗ-ಭದ್ರಾ
ತುಂಗ-ಭದ್ರಾ-ತೀರ
ತುಂಗ-ಭದ್ರಾ-ತೀರದ
ತುಂಗ-ಭದ್ರಾ-ತೀರ-ದಲ್ಲಿ
ತುಂಗ-ಭದ್ರಾ-ತೀರಲ್ಲಿದ್ದಾಗ
ತುಂಗ-ಭದ್ರಾ-ತೀರ್ಥ-ದಲ್ಲಿ
ತುಂಗ-ಭದ್ರಾ-ನದಿಗೆ
ತುಂಗ-ಭದ್ರೆಗೂ
ತುಂಗ-ಭದ್ರೆಯ
ತುಂಗ-ಭದ್ರೆ-ಯನ್ನು
ತುಂಡನುಂನತಂ
ತುಂಡು
ತುಂಡು-ಶಾ-ಸನ-ವಿದೆ
ತುಂಬ
ತುಂಬ-ದೇವ-ನ-ಹಳ್ಳಿಯ
ತುಂಬ-ಬಾರ-ದೆಂಬ
ತುಂಬಲ
ತುಂಬಾ
ತುಂಬಿ
ತುಂಬಿ-ಕೊಂಡು
ತುಂಬಿ-ದನು
ತುಂಬಿ-ದಾಗ
ತುಂಬಿದೆ
ತುಂಬಿದ್ದು
ತುಂಬಿನ
ತುಂಬಿ-ನ-ಕೆರೆ-ಗಳ
ತುಂಬಿ-ನ-ತೂಬಿನ
ತುಂಬಿ-ರ-ಬ-ಹುದು
ತುಂಬಿ-ರುತ್ತಿತ್ತು
ತುಂಬಿ-ಸುತ್ತಿತ್ತು
ತುಂಬಿ-ಸುತ್ತಿದ್ದವು
ತುಂಬಿ-ಹರಿ-ಯುತ್ತಿದ್ದ
ತುಂಬುವ
ತುಕಡಿ-ಯನ್ನು
ತುಕಡಿ-ಯಲ್ಲಿ
ತುಗಲಕ್
ತುಗವಿ
ತುಗಿ-ಲೂರ
ತುಗ್ಗಿ-ಲೂರ
ತುಗ್ಗಿ-ಲೂರಿ-ನಲ್ಲಿದ್ದ
ತುಗ್ಗಿ-ಲೂರು-ನುಗ್ಗಿ-ಲೂರು
ತುಟ್ಟಿಗೆ
ತುಡಿಕೆ
ತುಣಿ
ತುತ್ತಾಗಿ
ತುಪದ
ತುಪ್ಪ
ತುಪ್ಪದ
ತುಪ್ಪ-ದ-ದೀಪ
ತುಪ್ಪ-ದಲ್ಲಿ
ತುಪ್ಪ-ದೆರೆ
ತುಪ್ಪ-ವನ್ನು
ತುಪ್ಪ-ವನ್ನೂ
ತುಮ-ಕೂರಿನ
ತುಮಕೂರು
ತುರಂಗಮಂ
ತುರಂಗ-ಮಂಗಳಂ
ತುರಗ
ತುರಗ-ಕಳ
ತುರಗ-ಕಳ-ನಿ-ರಿದು
ತುರಗ-ಕಳ-ವ-ನಿ-ರಿದು
ತುರಗ-ಗ-ಳನ್ನು
ತುರಗ-ಗ-ಳನ್ನೆಲ್ಲಾ
ತುರಗಾರೂಢಾ
ತುರಲೋಭತು
ತುರು-ಕರ
ತುರುಕ-ರಿಗೆ
ತುರು-ಕರು
ತುರು-ಕಳ-ಗ-ನಿ-ರಿದು
ತುರು-ಕಳ-ವ-ನಿ-ರಿದು
ತುರುಕವ-ನಿ-ರಿದು
ತುರುಗ-ಕಳ-ನಿ-ರಿದು
ತುರುಗ-ಮಂಗಳಂ
ತುರು-ಗಳ
ತುರು-ಗಳಂ
ತುರು-ಗ-ಳನ್ನು
ತುರು-ಗ-ಳನ್ನೂ
ತುರು-ಗಳು
ತುರುಗಾಳಕ್ಕೆ
ತುರುಗಾಳಗ
ತುರುಗಾಳಗ-ದಲ್ಲಿ
ತುರುಗಾ-ಳದ
ತುರುಗೊಳೆಳ್ದು
ತುರುಗೊಳ್
ತುರುಗೋ-ಳನ್ನು
ತುರುಗೋಳಾ-ದರೂ
ತುರುಗೋಳಿನ
ತುರುಗೋಳಿ-ನಲ್ಲಿ
ತುರುಗೋಳು
ತುರುಗೋಳು-ಗಳ
ತುರುಗೋಳ್
ತುರು-ದೇವರು
ತುರು-ಪರಿವಿ-ನಲ್ಲಿ
ತುರು-ಪರಿವಿ-ನಲ್ಲಿ-ತುರುಗೊಳ್
ತುರು-ಪರಿವಿ-ನಲ್ಲಿ-ತುರುಗೋಳ್
ತುರುಪಾರಿವಿ
ತುರುಮುಣ್ಡಿ
ತುರುವ
ತುರುವಂ
ತುರುವ-ನಿಕ್ಕಿಸಿ
ತುರು-ವನ್ನು
ತುರುವ-ಳಿ-ವಿನಲಿ
ತುರುವ-ಳಿವು
ತುರುವೆ-ಕೆರೆ
ತುರುವೆ-ಕೆರೆ-ಯವ-ರೆಂದು
ತುರುಷ್ಕ
ತುರುಷ್ಕಂ
ತುರುಷ್ಕ-ತುರಗಾರೂಢ
ತುರುಷ್ಕನು
ತುರುಷ್ಕ-ಬಲ-ವನ್ನು
ತುರುಷ್ಕ-ಮುಸ್ಲಿಂ
ತುರುಷ್ಕರ
ತುರುಷ್ಕ-ರನ್ನು
ತುರುಷ್ಕ-ರಾಜ
ತುರುಹೋಹ
ತುರ್ಕಿಯ
ತುರ್ತು
ತುಲ-ಗಣ್ಡ
ತುಲಾ-ಪುರ-ಷಾದಿ
ತುಲಾ-ಪುರುಷಾದಿ
ತುಲಾ-ಸಂಕ್ರಮಣ
ತುಲುವೇಂದ್ರ-ನಾದ
ತುಳಸೀ-ಬೃಂದಾ-ವ-ನದ
ತುಳಿದಂ
ತುಳಿ-ವಂತೆ-ವೋಲ್
ತುಳುಕುತ್ತಿತು
ತುಳು-ನಾಡಿನ
ತುಳುವ
ತುಳುವ-ನರ-ಸಿಂಹ
ತುಳುವ-ರಾಜೇಂದ್ರ-ಪುರಂ
ತುಳುವಲ
ತುಳುವ-ಲ-ದೇವಿ
ತುಳುವ-ಲ-ದೇವಿ-ಯನ್ನೂ
ತುಳುವ-ಲ-ದೇವಿ-ಯರು
ತುಳುವ-ಲೇಶ್ವರ
ತುಳುವ-ವಂಶದ
ತುವ್ವಲೇಶ್ವರ
ತುವ್ವಲೇಶ್ವರ-ತು-ಳುವ-ಲೇಶ್ವರ
ತುಷ್ಟಾ-ಶೇಷದ್ವಿ-ಜನ್ಮನಃ
ತೂಂಬ-ನಿಕ್ಕಿಸಿ
ತೂಂಬ-ನಿ-ರಿ-ಸಿದ
ತೂಂಬಿನ
ತೂಂಬಿನಿಂ
ತೂಕ-ವನ್ನೂ
ತೂಬನಿಟ್ಟು
ತೂಬನ್ನಿಕ್ಕಿದ
ತೂಬನ್ನಿ-ಡಿಸಿ
ತೂಬನ್ನು
ತೂಬಿಗೂ
ತೂಬಿಗೆ
ತೂಬಿನ
ತೂಬಿನ-ಕೆರೆ
ತೂಬಿನಿಂದ
ತೂಬು
ತೂಬು-ಗಳ
ತೂಬು-ಗ-ಳನ್ನು
ತೂಬು-ಗಳಿ-ರುತ್ತಿದ್ದವು
ತೂಬು-ಗಳು
ತೂರ್ಯ
ತೂಲ್ದ-ವನ-ನೋ-ಡಿಸಿ
ತೃಪ್ತಿ
ತೆಂಕ
ತೆಂಕಕ್ಕೆ
ತೆಂಕಣ
ತೆಂಕ-ಣ-ಕೋಡಿ
ತೆಂಕ-ಣ-ತೂಂಬಿ-ನಿಂದ
ತೆಂಕ-ಣ-ಭಾಗ
ತೆಂಕ-ಣ-ಭಾಗದ
ತೆಂಕ-ಣ-ಭಾಗ-ದಲ್ಲಿ
ತೆಂಕ-ಣ-ಭಾಗ-ದಲ್ಲಿದ್ದ
ತೆಂಕ-ಣ-ಭಾಗ-ವನ್ನು
ತೆಂಕ-ಣ-ರಾಯ
ತೆಂಕ-ಣ-ಹಳ್ಳಿಯ
ತೆಂಕ-ಭಾಗ-ದಲ್ಲಿದ್ದ
ತೆಂಕ-ಲಂಕದ
ತೆಂಕ-ಲಾಗಿ
ತೆಂಕ-ಲಿಗೆ
ತೆಂಕಲು
ತೆಂಕ-ಳಣ
ತೆಂಗಿನ
ತೆಂಗಿನ-ಕಟ್ಟ
ತೆಂಗಿನ-ಕಟ್ಟ-ಇಂದಿನ
ತೆಂಗಿನ-ಕಟ್ಟದ
ತೆಂಗಿನ-ಕಟ್ಟ-ದಲ್ಲಿ
ತೆಂಗಿನ-ಕಟ್ಟ-ವನು
ತೆಂಗಿನ-ಕಟ್ಟ-ವನ್ನು
ತೆಂಗಿನ-ಘಟ್ಟ
ತೆಂಗಿನ-ಘಟ್ಟದ
ತೆಂಗಿನ-ಘಟ್ಟ-ವನ್ನು
ತೆಂಗು
ತೆಂದು
ತೆಂದೂ
ತೆಂಪಾಗೈ
ತೆಕೊಂಡದು
ತೆಕೊಳ-ಲಿಲಾ
ತೆಗಡರ-ಹಳ್ಳಿ
ತೆಗೆ-ದಿಡುತ್ತಾರೆ
ತೆಗೆ-ದಿದ್ದಾನೆ
ತೆಗೆ-ದಿದ್ದೇನೆ
ತೆಗೆ-ದಿರಿಸ-ಲಾಗುತ್ತದೆ
ತೆಗೆ-ದಿ-ರಿ-ಸಿದ್ದು
ತೆಗೆ-ದಿರಿ-ಸು-ವುದು
ತೆಗೆದು
ತೆಗೆ-ದು-ಕೊಂಡದ್ದನ್ನು
ತೆಗೆ-ದು-ಕೊಂಡನು
ತೆಗೆ-ದು-ಕೊಂಡ-ರೆಂದು
ತೆಗೆ-ದು-ಕೊಂಡಿತು
ತೆಗೆ-ದು-ಕೊಂಡಿದ್ದ
ತೆಗೆ-ದು-ಕೊಂಡಿದ್ದಂತೆ
ತೆಗೆ-ದು-ಕೊಂಡಿದ್ದನು
ತೆಗೆ-ದು-ಕೊಂಡಿದ್ದ-ನೆಂದೂ
ತೆಗೆ-ದು-ಕೊಂಡಿರು-ವುದು
ತೆಗೆ-ದು-ಕೊಂಡು
ತೆಗೆ-ದು-ಕೊಳ್ಳ
ತೆಗೆ-ದು-ಕೊಳ್ಳ-ಬ-ಹುದು
ತೆಗೆ-ದು-ಕೊಳ್ಳ-ಲಾ-ಯಿತು
ತೆಗೆ-ದು-ಕೊಳ್ಳುತ್ತಿದ್ದರು
ತೆಗೆ-ದು-ಕೊಳ್ಳುತ್ತಿದ್ದ-ರೆಂಬುದು
ತೆಗೆ-ದು-ಕೊಳ್ಳುವ
ತೆಗೆದೆ
ತೆಗೆ-ಯ-ಲಾಯಿತೆಂಬುದು
ತೆಗೆ-ಯುತ್ತಿದ್ದ
ತೆಗೆಸಿ
ತೆಗೆ-ಸು-ವಂತೆಯೂ
ತೆತ್ತ-ರೆಂದು
ತೆತ್ತಿಗ-ನೆನೆ-ವುದು
ತೆತ್ತು
ತೆತ್ತು-ಬ-ರುವ
ತೆನದಂಕ
ತೆನದಂಕ-ಕುಲ
ತೆನದಂಕರ
ತೆನದಂಕಾನ್ವಯ
ತೆನದಂಕಾನ್ವಯದ
ತೆನದ-ಕರ
ತೆನದ-ಕರ-ತೆ-ನದಂಕ-ಕುಲದ
ತೆನ-ದಕ್ಕ
ತೆನದಕ್ಕನ
ತೆನೆ
ತೆನ್ನ
ತೆಪ್ಪ
ತೆಪ್ಪ-ಕೊಳ
ತೆಪ್ಪ-ಕೊಳದ
ತೆಪ್ಪ-ಕೊಳ-ವನ್ನು
ತೆಪ್ಪಣ್ಣತೇ-ಪಣ್ಣ-ದೇವಣ್ಣ
ತೆಪ್ಪ-ತಿರುನಾಳ
ತೆಪ್ಪ-ತಿರುನಾಳಿಗೆ
ತೆಪ್ಪ-ತಿರುನಾಳಿನ
ತೆಪ್ಪ-ತಿರುನಾಳು
ತೆಪ್ಪ-ತಿರುನಾಳ್
ತೆಪ್ಪದ
ತೆಪ್ಪ-ದ-ನಾಗಣ್ಣನು
ತೆಪ್ಪೋತ್ಸವ
ತೆರ
ತೆರ-ಕಣಾಂಬಿ
ತೆರ-ಕಣಾಂಬಿಯ
ತೆರ-ಕಣಾಂಬಿ-ಯಲ್ಲಿ
ತೆರ-ಕಣಾಂಬಿ-ಯಲ್ಲಿದ್ದ
ತೆರ-ಕಣಾಂಬಿ-ಯಿಂದ
ತೆರ-ಕಣಾಂಬಿ-ಸೀಮೆಯ
ತೆರ-ಕಣಾಂಬೆ
ತೆರ-ಕಣಾಂಬೆಯ
ತೆರ-ಕಣಾಂಬೆ-ಯನ್ನು
ತೆರ-ಗಣ
ತೆರ-ಣೆನ-ಹಳ್ಳಿ
ತೆರ-ಣೆನ-ಹಳ್ಳಿ-ಯನ್ನು
ತೆರದ
ತೆರದಿಂ
ತೆರ-ದಿಂದವೆ
ತೆರ-ನಾಗಿದ್ದರೂ
ತೆರ-ನಾ-ಗಿದ್ದು
ತೆರ-ನಾದ
ತೆರ-ಪಿನ
ತೆರ-ಬಹು-ದಾ-ಗಿದ್ದು
ತೆರ-ಬೇಕಾಗಿತ್ತೆಂದು
ತೆರ-ಬೇಕಾಗಿದ್ದ
ತೆರ-ಬೇ-ಕಾದ
ತೆರಲು
ತೆರ-ಳಲು
ತೆರಳಿ
ತೆರ-ಳು-ವಾಗ
ತೆರಾಯಾಂಬಾಮುಲ
ತೆರಿಗೆ
ತೆರಿಗೆ-ಕಂದಾಯ
ತೆರಿಗೆ-ಗಳ
ತೆರಿಗೆ-ಗ-ಳನ್ನು
ತೆರಿಗೆ-ಗಳನ್ನು-ದತ್ತಿ-ಗ-ಳನ್ನು
ತೆರಿಗೆ-ಗ-ಳನ್ನೂ
ತೆರಿಗೆ-ಗ-ಳಲ್ಲಿ
ತೆರಿಗೆ-ಗ-ಳಾಗಿದ್ದ-ವೆಂದು
ತೆರಿಗೆ-ಗ-ಳಾಗಿದ್ದು
ತೆರಿಗೆ-ಗ-ಳಿಂದ
ತೆರಿಗೆ-ಗ-ಳಿಗೆ
ತೆರಿಗೆ-ಗಳಿ-ರ-ಬ-ಹುದು
ತೆರಿಗೆ-ಗಳು
ತೆರಿಗೆ-ಗಳು-ಸುಂಕ-ಗಳು
ತೆರಿಗೆ-ಗಳೂ
ತೆರಿಗೆ-ಗ-ಳೆಂದು
ತೆರಿಗೆ-ಗಳೋ
ತೆರಿ-ಗೆಗೆ
ತೆರಿಗೆಯ
ತೆರಿಗೆ-ಯನ್ನು
ತೆರಿಗೆ-ಯನ್ನೂ
ತೆರಿಗೆ-ಯನ್ನೇ
ತೆರಿಗೆ-ಯಲ್ಲಿ
ತೆರಿಗೆ-ಯಾ-ಗಿದ್ದಿರ-ಬ-ಹುದು
ತೆರಿಗೆ-ಯಾಗಿ-ರ-ಬ-ಹುದು
ತೆರಿಗೆ-ಯಾಗಿ-ರ-ಹ-ಬುದು
ತೆರಿಗೆ-ಯಿಂದ
ತೆರಿಗೆ-ಯಿರ-ಬ-ಹುದು
ತೆರಿಗೆಯು
ತೆರಿಗೆಯೂ
ತೆರಿಗೆ-ಯೆಂದೂ
ತೆರಿಗೆಯೇ
ತೆರಿಗೆಯೋ
ತೆರುತ
ತೆರುತ್ತ
ತೆರುತ್ತ-ಬಹರು
ತೆರುತ್ತಿದ್ದ
ತೆರುವ
ತೆರು-ವಂತೆ
ತೆರು-ವಂತೆಯೂ
ತೆರುವರು
ತೆರು-ವು-ದಾಗಿ
ತೆರು-ವು-ದಾಗಿಯೂ
ತೆರೆ
ತೆರೆದ
ತೆರೆ-ದ-ಜಾಗ-ದಲ್ಲಿ
ತೆರೆ-ದ-ಬಾ-ಗಿಲ
ತೆರೆ-ದ-ಮಂಟಪ
ತೆರೆ-ದ-ವರು
ತೆರೆ-ದು-ಕೊಟ್ಟನು
ತೆರೆಯ
ತೆರೆ-ಯನು
ತೆರೆ-ಯನ್ನು
ತೆರೆ-ಯನ್ನೂ
ತೆಱುವ
ತೆಱೆ
ತೆಱೆಯ
ತೆಲಗಾವಿ
ತೆಲಿಗ
ತೆಲುಂಗನ
ತೆಲುಂಗ-ರಾಯಸ್ಥಾ-ಪನಾ-ಚಾರ್ಯ
ತೆಲುಗು
ತೆಲುಗು-ಚೋಡ
ತೆಲುಗು-ಚೋಡರ
ತೆಲುಗು-ಚೋಳ-ರನ್ನು
ತೆಲುಗು-ಭಾಷೆ
ತೆಲುಗು-ಭಾಷೆಯ
ತೆಲುಗು-ಮೂಲದ
ತೆಲುಗು-ಮೂಲ-ವೆಂದು
ತೆಲುಗು-ಸೀಮೆ-ಯ-ವರು
ತೆಲ್ಲ
ತೆಲ್ಲ-ಕುಲ
ತೆಲ್ಲಿಗ
ತೆಲ್ಲಿ-ಗರ
ತೆಲ್ಲಿ-ಗ-ರಿಗೆ
ತೆಲ್ಲಿ-ಗರು
ತೆಲ್ಲಿ-ಗರೂ
ತೆಲ್ಲಿ-ಗ-ರೈ-ವತ್ತೊಕ್ಕಲು
ತೆಳ-ನೂರ
ತೆಳರ
ತೆಳರ-ಕುಲ-ತಿಲಕ
ತೆಳರ-ಕುಲದ
ತೆಳರ-ತೆಳ್ಳರ
ತೆಳ್ಳರ
ತೆಳ್ಳರ-ಕುಲದ
ತೆಳ್ಳಾರಮ್ಮ-ನೆಂಬ
ತೇಕಲ್ಲು
ತೇಗಿನ-ಹಳ್ಳಿ
ತೇಜ
ತೇಜ-ಸಾಮ್ಯ-ವಿಲ್ಲದೆ
ತೇಜಸ್ವಾಮ್ಯ-ದಲ್ಲಿ
ತೇಜಸ್ವಾಮ್ಯ-ವನು
ತೇಜಸ್ವಾಮ್ಯ-ವನ್ನೂ
ತೇಜೋ-ಮಯ-ರಾ-ಗಿದ್ದು
ತೇಜೋ-ಮೂರ್ತಿ
ತೇದಿ
ತೇದಿ-ಗ-ಳನ್ನು
ತೇದಿ-ಗಳಿವೆ
ತೇದಿ-ಗಳು
ತೇದಿಯ
ತೇದಿ-ಯನ್ನು
ತೇದಿ-ಯಲ್ಲಿ
ತೇದಿ-ಯಿಲ್ಲದ
ತೇದಿಯು
ತೇದಿ-ಯುಕ್ತ
ತೇದಿ-ಯುಳ್ಳ
ತೇದಿ-ರಹಿತ
ತೇದಿ-ರಹಿತ-ವಾಗಿದೆ
ತೇದಿ-ರಹಿತ-ವಾದ
ತೇನೇಮ
ತೇರಣ್ಯ
ತೇರ್ಗಡೆ-ಯಾಗಿ
ತೇರ್ಗಡೆ-ಯಾದೆ
ತೈರೂರ
ತೈರೂರಿನ
ತೈಲನ
ತೈಲನು
ತೈಲ-ಪನು
ತೈಲೂರಿನ
ತೈಲೂರು
ತೊಂಡ
ತೊಂಡ-ನೂರ
ತೊಂಡ-ನೂ-ರನ್ನು
ತೊಂಡ-ನೂರಾದ
ತೊಂಡ-ನೂ-ರಿಗೂ
ತೊಂಡ-ನೂ-ರಿಗೆ
ತೊಂಡ-ನೂರಿನ
ತೊಂಡ-ನೂರಿನಲ್ಲಿ
ತೊಂಡ-ನೂರಿನಲ್ಲಿದ್ದಾಗ
ತೊಂಡ-ನೂರು
ತೊಂಡ-ಮಂಡ-ಲದ
ತೊಂಡಾ-ಚಾರಿ
ತೊಂಡಾಳು
ತೊಂಡೆಯ-ಹಾಳ
ತೊಂಡೆ-ಹಳ್ಳ-ದಿಂದ
ತೊಂಡೇ-ಹಳ್ಳಿ
ತೊಂಡೇ-ಹಳ್ಳಿಯ
ತೊಂಡೈ-ಮಂಡ-ಲದ
ತೊಂಡೈ-ಮಂಡ-ಲಮ್
ತೊಂಣ-ನೂರಿ-ನ-ವರು
ತೊಂದರೆ
ತೊಂದರೆ-ಯಾದಾಗ
ತೊಂನೂರು-ತೊಂಣ-ನೂರು
ತೊಂಬ-ತಾರು
ತೊಂಬತ್ತರು-ಸಾವಿರ
ತೊಂಬತ್ತರು-ಸಾ-ಸಿರ
ತೊಂಬತ್ತರು-ಸಾ-ಸಿರದ
ತೊಂಬತ್ತರು-ಸಾ-ಸಿರಮಂ
ತೊಂಬತ್ತರು-ಸಾ-ಸಿರ-ಮನು
ತೊಂಬತ್ತರು-ಸಾ-ಸಿರ-ಮನೇಕ
ತೊಂಬತ್ತಾರು
ತೊಂಬತ್ತಾರು-ಸಾವಿರ
ತೊಂಬತ್ತಾರು-ಸಾವಿರದ
ತೊಂಬತ್ತಾರು-ಸಾವಿ-ರನ್ನು
ತೊಂಬತ್ತಾರು-ಸಾವಿರ-ವನ್ನು
ತೊಂಭತ್ತಾರು
ತೊಗಟರ
ತೊಗಟ-ವೀರ-ರೆಂದು
ತೊಗರ-ವಾಡಿ
ತೊಗರಿ
ತೊಗ-ರಿಯ
ತೊಟ-ವನ್ನು
ತೊಟ್ಟಿ
ತೊಟ್ಟಿ-ಮಂಟಪದ
ತೊಟ್ಟಿಯು
ತೊಟ್ಟಿಲು
ತೊಡಗಿ-ಕೊಂಡಿದ್ದನು
ತೊಡಗಿದ್ದನು
ತೊಡಗಿದ್ದ-ರೆಂಬು-ದನ್ನು
ತೊಡಗಿದ್ದವು
ತೊಡಗಿದ್ದು-ದನ್ನು
ತೊಡಗಿ-ಸಲು
ತೊಡಗಿಸಿ-ಕೊಂಡರು
ತೊಡಗಿಸಿ-ಕೊಂಡಿದ್ದಾನೆ
ತೊಡಗಿಸಿ-ಕೊಂಡಿರ-ಬ-ಹುದು
ತೊಡಗಿಸಿ-ಕೊಂಡಿರು-ವುದು
ತೊಡಗುತ್ತಿದ್ದ
ತೊಡರ್ದರ-ಡೊಂಕಿಯುಂ
ತೊಡರ್ದ್ದ-ರಂಕುಸ
ತೊಡಿಸಿ
ತೊಡೆಯ
ತೊಣಚಿ
ತೊಣ-ಚಿಯ
ತೊಣ್ಣೂ-ರನ್ನು
ತೊಣ್ಣೂ-ರಿಗೆ
ತೊಣ್ಣೂರಿನ
ತೊಣ್ಣೂರಿ-ನಲ್ಲಿ
ತೊಣ್ಣೂರಿನಲ್ಲಿ-ರುವ
ತೊಣ್ಣೂರಿ-ನಲ್ಲೇ
ತೊಣ್ಣೂರಿ-ನಿಂದ
ತೊಣ್ಣೂರು
ತೊಣ್ಣೂರು-ಗ-ಳಲ್ಲಿ
ತೊಣ್ಣೂರು-ಗ-ಳಿಗೆ
ತೊಣ್ಣೂರು-ತೊಂಡ-ನೂರು
ತೊಣ್ಣೈ-ಕೂಡು
ತೊತ್ತಕದ್ದವ
ತೊತ್ತಿನ
ತೊತ್ತು
ತೊತ್ತೇರಿ-ಗುಂಟು
ತೊನದ-ಕರ
ತೊರೆ
ತೊರೆ-ಕಾಡನ-ಹಳ್ಳಿ
ತೊರೆ-ಗ-ಳನ್ನು
ತೊರೆ-ಗ-ಳಾಗಿವೆ
ತೊರೆ-ಗಳು
ತೊರೆದು
ತೊರೆ-ನಾಡು
ತೊರೆ-ಬೊಮ್ಮನ-ಹಳ್ಳಿ
ತೊರೆ-ಮಗ್ಗ
ತೊರೆಯ
ತೊರೆ-ಯಂಣಯ್ಯ-ನ-ವರ
ತೊರೆ-ಯ-ಬಳಿಯ
ತೊಱಗ-ಲೆಯ
ತೊಲ-ಗಂಡ
ತೊಲ-ಗದ
ತೊಲೆಯ
ತೊಲೆಯ-ಮೇಲಿ-ರುವ
ತೊಳಂಚೆ
ತೊಳಂಚೆಯ
ತೊಳಂಚೆ-ಯನ್ನು
ತೊಳಂಚೆ-ಯಲ್ಲಿ
ತೊಳಂಚೆಯು
ತೊಳಲ್ದು
ತೊಳಸಿ
ತೊಳಸಿಯ
ತೊಳ-ಸಿಯು
ತೊಳೆದು
ತೊಳೆಪ
ತೊಳೆ-ಯಲು
ತೋಂಟದ
ತೋಟ
ತೋಟಕ
ತೋಟ-ಕದ
ತೋಟ-ಕಾ-ಚಾರ್ಯ-ರಪ್ಪ
ತೋಟಕೆ
ತೋಟಕ್ಕೆ
ತೋಟಕ್ಕೆ-ವೀಳ್ಯದೆಲೆ
ತೋಟ-ಗದ್ದೆ-ಗ-ಳನ್ನು
ತೋಟ-ಗಳ
ತೋಟ-ಗ-ಳನ್ನು
ತೋಟ-ಗ-ಳನ್ನೂ
ತೋಟ-ಗ-ಳಿಂದ
ತೋಟ-ತು-ಡಿಕೆ
ತೋಟದ
ತೋಟ-ದಯ್ಯ
ತೋಟ-ದಯ್ಯನು
ತೋಟ-ದಲ್ಲಿದ್ದು
ತೋಟ-ದಲ್ಲಿ-ರುವ
ತೋಟ-ನ-ವನ್ನು
ತೋಟನ್ನು
ತೋಟವ
ತೋಟ-ವನ್ನು
ತೋಟ-ವನ್ನೂ
ತೋಟ-ವೃತ್ತಿ
ತೋಟಸ್ಥಳ
ತೋಟಸ್ಥಳ-ಗಳನು
ತೋಟಿ
ತೋಟಿ-ಗರು
ತೋಟಿ-ಗರೈ-ನೂರ್ವರು
ತೋಡರ
ತೋಡಿರು-ವು-ದನ್ನು
ತೋಡಿಸಿದ
ತೋಡಿಸಿ-ದ-ನೆಂದು
ತೋಡಿಸಿ-ದಾಗ
ತೋಡಿಸಿದ್ದ-ನೆಂದೂ
ತೋಡಿ-ಸಿದ್ದಾರೆ
ತೋಡಿಸಿ-ರ-ಬ-ಹುದು
ತೋಡಿಸುತ್ತಾರೆ
ತೋಪನ್ನೂ
ತೋಪಿನ
ತೋಪಿ-ನಲ್ಲಿ
ತೋರಣ-ವನ್ನು
ತೋರಿ-ನಾಡ
ತೋರಿ-ನಾಡು
ತೋರಿಸ-ಬೇಕು
ತೋರಿ-ಸಲು
ತೋರಿ-ಸಲೆಂದು
ತೋರಿಸಿ-ಕೊಟ್ಟಿದ್ದಾರೆ
ತೋರಿ-ಸಿದರು
ತೋರಿ-ಸಿದರೆ
ತೋರಿ-ಸಿದ್ದಾರೆ
ತೋರಿಸು
ತೋರಿಸುತ್ತದೆ
ತೋರಿಸುತ್ತವೆ
ತೋರಿಸುತ್ತವೇನೋ
ತೋರಿಸುತ್ತಾರೆ
ತೋರಿ-ಸುತ್ತಿದೆ
ತೋರಿ-ಸುವ
ತೋರಿಸು-ವುದೇ
ತೋರು
ತೋರುತ್ತ
ತೋರುತ್ತದೆ
ತೋರುತ್ತ-ದೆ-ಎಂದು
ತೋರುತ್ತವೆ
ತೋರುತ್ತಿ-ರುವಂತಿದ್ದರೆ
ತೋರುತ್ತುದೆ
ತೋರ್ಪ್ಪ-ಸುರ-ತರು
ತೋಳ
ತೋಳಕೈ
ತೋಳ-ಬಿಂಕಮಂ
ತೋಳಿನ
ತೋಳು-ಕಯಿ-ಕೊಟ್ಟು
ತೌಳಿ-ಯಮ್ಮ
ತ್ತಂಗಜ
ತ್ತದೆ
ತ್ತಳಮೆನಿ-ಸಿತು
ತ್ತಾನೆ
ತ್ತಾರೆ
ತ್ತಿತ್ತು
ತ್ತಿದ್ದ
ತ್ತಿದ್ದರು
ತ್ತಿದ್ದ-ರೆಂದು
ತ್ತಿದ್ದುದು
ತ್ತಿರ್ಮಾಲೆಗೆ
ತ್ತುತ್ತಿರೆ
ತ್ತೆಂದು
ತ್ತೆಂಬುದು
ತ್ತೊಮ್ಭತ್ತಱು
ತ್ಯಂತ-ವಾಗಿ
ತ್ಯಕ್ಷುಂಣ-ಚರಿತ-ರಿ-ವರೆ
ತ್ಯಜಿ-ಸಿದ
ತ್ಯಾಗದ
ತ್ಯಾಗ-ದ-ಕೊಡು-ಗೆ-ಯಾಗಿ
ತ್ಯಾಗ-ನ-ಹಳ್ಳಿ
ತ್ಯಾಗ-ವಾಗಿ
ತ್ಯುಕ್ಷುಣ್ಣ
ತ್ರಾಸನಂ
ತ್ರಿಕಾಲ
ತ್ರಿಕೂಟ
ತ್ರಿಕೂಟ-ಜಿನಾ-ಲಯ-ವನ್ನು
ತ್ರಿಕೂಟಬ
ತ್ರಿಕೂಟ-ರತ್ನತ್ರಯ
ತ್ರಿಕೂಟ-ರತ್ನತ್ರಯದ
ತ್ರಿಕೂಟ-ರತ್ನತ್ರಯ-ಬ-ಸದಿಗೆ
ತ್ರಿಕೂಟ-ಲಕ್ಷ್ಮೀ-ನಾ-ರಾಯಣ
ತ್ರಿಕೂಟಾಚಲ
ತ್ರಿಕೂಟಾಚಲ-ದೇವಾ-ಲಯ-ವಾದ
ತ್ರಿಕೂಟಾಚಲ-ವಾಗಿದೆ
ತ್ರಿಕೂಟಾಚಲ-ವಾ-ಗಿದ್ದು
ತ್ರಿಕೂಟಾಚಲ-ವೆಂದು
ತ್ರಿಣೇತ್ರ
ತ್ರಿಣೇತ್ರನಂ
ತ್ರಿಣೇತ್ರ-ನೆಂದು
ತ್ರಿಪೂರುಷೈರೇವಂ
ತ್ರಿಭುವ-ಚಕ್ರ-ವರ್ತಿ
ತ್ರಿಭು-ವನ
ತ್ರಿಭು-ವನ-ಕಠಾರಿ-ರಾಯ
ತ್ರಿಭು-ವನ-ಕಠಾರಿ-ರಾಯನೂ
ತ್ರಿಭು-ವನ-ಗಂಡ
ತ್ರಿಭು-ವನ-ಚಕ್ರ-ವರ್ತಿ
ತ್ರಿಭು-ವನ-ತಿಳಕ
ತ್ರಿಭು-ವನ-ತೀರ್ಥದ
ತ್ರಿಭು-ವನ-ಮಲ್ಲ
ತ್ರಿಭು-ವನೀ-ರಾಯ
ತ್ರಿಮೂರ್ತಿ-ಗಳ
ತ್ರಿಯಂಬಕೇಶ್ವರ
ತ್ರಿವಿಷ್ಟಪಾವೇಷ್ಟಿತ
ತ್ರಿವೇದೀ
ತ್ರಿಷಷ್ಠಿ
ತ್ರುಟಿತ
ತ್ರುಟಿತ-ಭಾಗ
ತ್ರುಟಿತ-ವಲ್ಲದ
ತ್ರುಟಿತ-ವಾಗಿತ್ತೆಂದು
ತ್ರುಟಿತ-ವಾಗಿದೆ
ತ್ರುಟಿತ-ವಾ-ಗಿದ್ದು
ತ್ರುಟಿತ-ವಾಗಿ-ರುವ
ತ್ರುಟಿತ-ವಾಗಿ-ರುವು-ದ-ರಿಂದ
ತ್ರುಟಿತ-ವಾಗಿವೆ
ತ್ರುಟಿತ-ವಾದ
ತ್ರುಟಿತ-ಶಾ-ಸನವು
ತ್ರುಟಿದ
ತ್ರುಟಿ-ವಾಗಿದೆ
ತ್ರೈಲೋಕ್ಯ
ತ್ರೈಲೋಕ್ಯ-ರಂಜನ
ತ್ರೈವಿದ್ಯ
ತ್ರೈವಿದ್ಯ-ದೇವ
ತ್ರೈವಿದ್ಯ-ದೇವನು
ತ್ರೈವಿದ್ಯ-ದೇವರ
ತ್ರೈವಿದ್ಯ-ದೇವರು
ತ್ರೈವಿದ್ಯನ
ತ್ರೈವಿದ್ಯ-ರು-ವಾಸು-ಪೂಜ್ಯ
ತ್ರೈವಿದ್ಯಾ
ತ್ವಂ
ತ್ವರಿ-ತಗತಿ-ಯಲ್ಲಿ
ತ್ವಷ್ಟ
ತ್ವಷ್ಟಾ
ಥಾಣ
ದ
ದಂಗೆ
ದಂಗೆಯ
ದಂಗೆ-ಯನ್ನು
ದಂಡ
ದಂಡಂಗಳು
ದಂಡಗಿ
ದಂಡ-ಡ-ನಾಯ-ಕನ
ದಂಡದ
ದಂಡ-ದ-ಧಿಷ್ಠಾ-ಯಕ
ದಂಡ-ದ-ಧಿಷ್ಠಾ-ಯಕರು
ದಂಡ-ದೋಷ
ದಂಡ-ದೋಷದ
ದಂಡ-ದೋಷ-ವೆಂದು-ಕೊಳ-ಲಿಲ್ಲ
ದಂಡ-ನಾ-ತಾಂಬ-ರಾರ್ಕ್ಕಂ
ದಂಡ-ನಾಥ
ದಂಡ-ನಾಥನ
ದಂಡ-ನಾಥ-ನನ್ನು
ದಂಡ-ನಾಥನು
ದಂಡ-ನಾಥಾಧಿಪ
ದಂಡ-ನಾಥೋ
ದಂಡ-ನಾಯಕ
ದಂಡ-ನಾಯ-ಕಂಗೆ
ದಂಡ-ನಾಯ-ಕತ್ವ-ದಲ್ಲಿ
ದಂಡ-ನಾಯ-ಕನ
ದಂಡ-ನಾಯ-ಕ-ನದ್ದೇ
ದಂಡ-ನಾಯ-ಕ-ನನ್ನಾಗಿ
ದಂಡ-ನಾಯ-ಕ-ನನ್ನು
ದಂಡ-ನಾಯ-ಕ-ನಾಗಿ
ದಂಡ-ನಾಯ-ಕ-ನಾ-ಗಿದ್ದ
ದಂಡ-ನಾಯ-ಕ-ನಾಗಿದ್ದಂತೆ
ದಂಡ-ನಾಯ-ಕ-ನಾಗಿದ್ದನು
ದಂಡ-ನಾಯ-ಕ-ನಾಗಿದ್ದ-ನೆಂದು
ದಂಡ-ನಾಯ-ಕ-ನಾಗಿದ್ದ-ನೆಂಬುದು
ದಂಡ-ನಾಯ-ಕ-ನಾ-ಗಿದ್ದು
ದಂಡ-ನಾಯ-ಕ-ನಾಗಿ-ರ-ಬಹು-ದಾದ
ದಂಡ-ನಾಯ-ಕ-ನಾಗಿ-ರಲು
ದಂಡ-ನಾಯ-ಕ-ನಾಗಿ-ರುತ್ತಿದ್ದನು
ದಂಡ-ನಾಯ-ಕ-ನಾದ
ದಂಡ-ನಾಯ-ಕ-ನಿಕ್ಕ-ಯಣ್ಣನು
ದಂಡ-ನಾಯ-ಕ-ನಿ-ಗಿಂತ
ದಂಡ-ನಾಯ-ಕ-ನಿ-ಗಿದ್ದ
ದಂಡ-ನಾಯ-ಕ-ನಿಗೂ
ದಂಡ-ನಾಯ-ಕ-ನಿಗೆ
ದಂಡ-ನಾಯ-ಕನು
ದಂಡ-ನಾಯ-ಕನೂ
ದಂಡ-ನಾಯ-ಕ-ನೂ-ಸೋಮ-ದಂಡ-ನಾಯಕ
ದಂಡ-ನಾಯ-ಕ-ನೆಂದು
ದಂಡ-ನಾಯ-ಕ-ನೆಂಬ
ದಂಡ-ನಾಯ-ಕ-ನೆಂಬು-ವ-ವನು
ದಂಡ-ನಾಯ-ಕ-ನೆನಿ-ಸಿ-ದನು
ದಂಡ-ನಾಯ-ಕನೇ
ದಂಡ-ನಾಯ-ಕ-ನೊಬ್ಬ
ದಂಡ-ನಾಯ-ಕ-ನೊಬ್ಬನು
ದಂಡ-ನಾಯ-ಕರ
ದಂಡ-ನಾಯ-ಕ-ರನ್ನು
ದಂಡ-ನಾಯ-ಕ-ರಲ್ಲಿ
ದಂಡ-ನಾಯ-ಕ-ರಾ-ಗಲೀ
ದಂಡ-ನಾಯ-ಕ-ರಾಗಿ
ದಂಡ-ನಾಯ-ಕ-ರಾ-ಗಿದ್ದ
ದಂಡ-ನಾಯ-ಕ-ರಾಗಿದ್ದರು
ದಂಡ-ನಾಯ-ಕ-ರಾಗಿದ್ದ-ರೆಂದು
ದಂಡ-ನಾಯ-ಕ-ರಾಗಿದ್ದ-ವರೇ
ದಂಡ-ನಾಯ-ಕ-ರಾದ
ದಂಡ-ನಾಯ-ಕ-ರಿಂದ
ದಂಡ-ನಾಯ-ಕ-ರಿ-ಗಿಂತ
ದಂಡ-ನಾಯ-ಕ-ರಿಗೆ
ದಂಡ-ನಾಯ-ಕರು
ದಂಡ-ನಾಯ-ಕರುಂ
ದಂಡ-ನಾಯ-ಕ-ರು-ಗಳ
ದಂಡ-ನಾಯ-ಕ-ರು-ಗಳಂತ
ದಂಡ-ನಾಯ-ಕ-ರು-ಗ-ಳನ್ನು
ದಂಡ-ನಾಯ-ಕ-ರು-ಗ-ಳಾಗಿದ್ದ
ದಂಡ-ನಾಯ-ಕ-ರು-ಗಳಿ-ಗಿಂತ
ದಂಡ-ನಾಯ-ಕ-ರು-ಗ-ಳಿಗೆ
ದಂಡ-ನಾಯ-ಕ-ರು-ಗಳು
ದಂಡ-ನಾಯ-ಕ-ರು-ಗಳೂ
ದಂಡ-ನಾಯ-ಕ-ರು-ದಂಡಾಧೀಶರು
ದಂಡ-ನಾಯ-ಕ-ರು-ಮಂತ್ರಿ-ಗಳು
ದಂಡ-ನಾಯ-ಕರೂ
ದಂಡ-ನಾಯ-ಕ-ರೆಂದು
ದಂಡ-ನಾಯ-ಕ-ರೆಂಬ
ದಂಡ-ನಾಯ-ಕರೇ
ದಂಡ-ನಾಯ-ಕ-ವೀರಯ್ಯ
ದಂಡ-ನಾಯ-ಕಸು
ದಂಡ-ನಾಯ-ಕ-ಸು-ರಿಗೆ
ದಂಡ-ನಾಯ-ಕಿತಿ
ದಂಡ-ನಾಯ-ಕಿತ್ತಿ
ದಂಡ-ನಾಯ-ಕಿತ್ತಿಗೆ
ದಂಡ-ನಾಯ-ಕಿತ್ತಿಯ
ದಂಡ-ನಾಯ-ಕಿತ್ತಿ-ಯನ್ನು
ದಂಡ-ನಾಯ-ಕಿತ್ತಿ-ಯರ
ದಂಡ-ನಾಯ-ಕಿತ್ತಿ-ಯರು
ದಂಡ-ನಾಯ-ಕಿತ್ತಿಯು
ದಂಡ-ನಾಯ-ಕಿಯ
ದಂಡ-ನಾಯ-ಕಿಯು
ದಂಡ-ನಾಯನ
ದಂಡ-ನಾಯರು
ದಂಡ-ನಾಯು-ಕನ
ದಂಡ-ನಾಯು-ಕರ
ದಂಡನ್ನು
ದಂಡ-ಯಾತ್ರೆ
ದಂಡ-ಯಾತ್ರೆ-ಗಳ
ದಂಡ-ಯಾತ್ರೆ-ಗ-ಳಲ್ಲಿ
ದಂಡ-ಯಾತ್ರೆ-ಗ-ಳಿಂದ
ದಂಡ-ಯಾತ್ರೆ-ಗಳಿಗೂ
ದಂಡ-ಯಾತ್ರೆಯ
ದಂಡ-ಯಾತ್ರೆ-ಯನ್ನು
ದಂಡ-ಯಾತ್ರೆ-ಯಲ್ಲಿ
ದಂಡ-ರೂಪದ
ದಂಡವ
ದಂಡ-ವನ್ನು
ದಂಡ-ವನ್ನು-ಧರ್ಮ-ದಂಡ
ದಂಡಾಧಿಪ
ದಂಡಾಧಿಪ-ನದ್ದಲ್ಲ-ವೆಂದು
ದಂಡಾಧಿ-ಪರ
ದಂಡಾಧಿಪ-ರೊಳತಿಶಯಂ
ದಂಡಾಧೀಶ
ದಂಡಾಧೀಶಂ
ದಂಡಾಧೀಶ-ದಾ-ವಾನ-ಲನೂ
ದಂಡಾಧೀಶನ
ದಂಡಾಧೀಶ-ನಾ-ಗಿದ್ದು
ದಂಡಾಧೀಶನು
ದಂಡಾಧೀಶ-ನೆಂದು
ದಂಡಾಧೀಶರ
ದಂಡಾಧೀಶ-ರೆಂಬ
ದಂಡಾ-ಪುರ
ದಂಡಿಗ-ನ-ಹಳ್ಳಿ-ಯಲ್ಲಿ
ದಂಡಿಗೆ
ದಂಡಿಗೆತ್ತಿ
ದಂಡಿಗೆಯ
ದಂಡಿನ
ದಂಡಿನಭ್ಯಾಗತೆ
ದಂಡಿನ-ಹಳ್ಳಿ
ದಂಡಿ-ನೊಡನೆ
ದಂಡಿ-ಯಿಂದ
ದಂಡು
ದಂಡು-ಅಹೋ-ಬಲ-ದೇವನ
ದಂಡೆ
ದಂಡೆ-ಗಳಿಗೂ
ದಂಡೆತ್ತಿ
ದಂಡೆತ್ತಿ-ಬಂದನು
ದಂಡೆತ್ತಿ-ಬಂದು
ದಂಡೆತ್ತಿ-ಹೋಗಿ
ದಂಡೆತ್ತಿ-ಹೋದ
ದಂಡೆತ್ತಿ-ಹೋದರು
ದಂಡೆಯ
ದಂಡೆ-ಯ-ಗುಂಟ
ದಂಡೆ-ಯಲ್ಲಿ
ದಂಡೆ-ಯಲ್ಲಿದೆ
ದಂಡೆ-ಯಲ್ಲಿ-ರುವ
ದಂಡೇಶ
ದಂಡೇಶನ
ದಂಡೇಶನು
ದಂಡೇಶನೇ
ದಂಡೇ-ಷನೇಚಿ-ರಾಜಂ
ದಂಣಾ-ಯಕ
ದಂಣಾ-ಯಕರ
ದಂಣಾ-ಯಕ-ರಿಗೆ
ದಂಣಾ-ಯಕರು
ದಂಣಾಯ್ಕರ
ದಂಣಾಯ್ಕರು
ದಂಣ್ಣಾ-ಯಕರ
ದಂಣ್ನ-ಯಕರ
ದಂಣ್ನಾಯಕ
ದಂಣ್ನಾಯಕಂ
ದಂಣ್ನಾಯ-ಕ-ನನ್ನು
ದಂಣ್ನಾಯ-ಕನೂ
ದಂಣ್ನಾಯ-ಕರ
ದಂಣ್ನಾಯ-ಕ-ರಿಗೆ
ದಂಣ್ನಾಯ-ಕರು
ದಂಣ್ನಾಯ್ಕರ
ದಂಣ್ನು
ದಂತ-ಕಥೆ
ದಂಪತಿ
ದಂಪತಿ-ಗಳ
ದಂಪತಿ-ಗ-ಳಿಗೆ
ದಂಪತಿ-ಗಳು
ದಂಮಿ-ಗೆರೆ
ದಂಷ್ಟ್ರಾಗ್ರದೊಳಂ
ದಕ್ಷ
ದಕ್ಷ-ತೆ-ಗ-ಳಿಂದ
ದಕ್ಷನ
ದಕ್ಷಿಣ
ದಕ್ಷಿಣಕ್ಕಿ-ರುವ
ದಕ್ಷಿಣಕ್ಕೂ
ದಕ್ಷಿಣಕ್ಕೆ
ದಕ್ಷಿಣ-ಚಕ್ರ-ವರ್ತಿ
ದಕ್ಷಿಣದ
ದಕ್ಷಿಣ-ದ-ಕಡೆಗೆ
ದಕ್ಷಿಣ-ದಲ್ಲಿ
ದಕ್ಷಿಣ-ಭಾಗ-ದಲ್ಲಿ
ದಕ್ಷಿಣ-ಭಾಗ-ದಲ್ಲಿದ್ದ
ದಕ್ಷಿಣ-ಭಾಗ-ದಲ್ಲಿಯೂ
ದಕ್ಷಿಣ-ಭಾಗ-ದಲ್ಲಿ-ರುವ
ದಕ್ಷಿಣ-ಭಾರ-ತದ
ದಕ್ಷಿಣ-ಭಾರ-ತ-ದಲ್ಲಿ
ದಕ್ಷಿಣ-ಭಾರ-ತ-ದಲ್ಲೆಲ್ಲಾ
ದಕ್ಷಿಣ-ಭಾರ-ತ-ವನ್ನು
ದಕ್ಷಿಣ-ಭುಜಾ-ದಂಡ-ನೆನಿಸಿದ್ದ
ದಕ್ಷಿಣ-ವಾರ-ಣಸಿ
ದಕ್ಷಿಣ-ವಾರ-ಣಾಸಿ
ದಕ್ಷಿಣಾ
ದಕ್ಷಿಣಾ-ಪಥಕ್ಕೆ
ದಕ್ಷಿಣಾ-ಪಥದ
ದಕ್ಷಿಣಾ-ಮೂರ್ತಿ
ದಕ್ಷಿಣಾ-ಮೂರ್ತಿಯ
ದಗಂಡ-ಪೆಂಡಾರ
ದಟ್ಟ-ವಾಗಿದ್ದವು
ದಟ್ಟ-ವಾದ
ದಡಗ
ದಡ-ಗದ
ದಡಗ-ದ-ಡಿಗ-ನ-ಕೆರೆ
ದಡಗ-ದಲ್ಲಿ-ರುವ
ದಡಗ-ಳಲ್ಲೂ
ದಡದ
ದಡ-ದಲ್ಲಿ
ದಡ-ದಲ್ಲಿ-ರುವ
ದಡದ-ಹಳ್ಳಿ
ದಡಿಗ
ದಡಿಗಟ್ಟ-ವನ್ನೂ
ದಡಿಗದ
ದಡಿಗ-ದ-ಡಿಗ-ನ-ಕೆರೆ
ದಡಿಗ-ದೀ-ಡಿಗನ
ದಡಿಗನ
ದಡಿಗ-ನ-ಕೆರೆ
ದಡಿಗ-ನ-ಕೆರೆಗೆ
ದಡಿಗ-ನ-ಕೆರೆಯ
ದಡಿಗ-ವನ್ನು
ದಡಿಗ-ವಾಡಿ
ದಡಿಗ-ವಾಡಿ-ಯನ್ನು
ದಡಿಗ-ವಾಡಿ-ಯಲ್ಲಿ
ದಡಿಗ-ವಾಡಿಯುಂ
ದಡಿಗೇಶ್ವರ
ದಡಿ-ಘಟ್ಟ
ದಣಾ-ಯ-ಕರು
ದಣ್ಡ-ನಾಯ-ಕಂಗೆ
ದಣ್ಡ-ನಾಯಕ್ಕನ್
ದಣ್ಡೆ
ದಣ್ಣಾ-ಯಕ-ನ-ಪುರ
ದಣ್ಣಾಯ-ಕನು
ದಣ್ಣಾ-ಯಕರ
ದಣ್ಣಾ-ಯ-ಕರು
ದಣ್ನಾಯಕ
ದಣ್ನಾಯ-ಕನೂ
ದಣ್ನಾಯ-ಕರು
ದಣ್ನಾಯ-ಕಿತಿ
ದತ್ತಂ
ದತ್ತಕ
ದತ್ತ-ಕ-ಪಡೆ-ದಳು
ದತ್ತ-ಕ-ಪಡೆ-ದ-ಳೆಂದು
ದತ್ತ-ಯಾಗಿ
ದತ್ತಿ
ದತ್ತಿ-ಕೊಟ್ಟಿ-ರುವುದು
ದತ್ತಿ-ಗಳ
ದತ್ತಿ-ಗ-ಳನ್ನು
ದತ್ತಿ-ಗ-ಳನ್ನೂ
ದತ್ತಿ-ಗ-ಳಲ್ಲಿ
ದತ್ತಿ-ಗ-ಳಾಗಿವೆ
ದತ್ತಿ-ಗ-ಳಿಗೆ
ದತ್ತಿ-ಗಳು
ದತ್ತಿಗೂ
ದತ್ತಿಗೆ
ದತ್ತಿ-ನೀಡ-ಲಾಗಿದೆ
ದತ್ತಿ-ನೀ-ಡಿಕೆ
ದತ್ತಿ-ನೀಡುತ್ತಾನೆ
ದತ್ತಿ-ಪಡೆದು
ದತ್ತಿ-ಬಿಟಿ-ರು-ವಂತೆ
ದತ್ತಿ-ಬಿಟ್ಟ
ದತ್ತಿ-ಬಿಟ್ಟನು
ದತ್ತಿ-ಬಿಟ್ಟ-ನೆಂಉ
ದತ್ತಿ-ಬಿಟ್ಟ-ನೆಂದು
ದತ್ತಿ-ಬಿಟ್ಟ-ನೆಂದೂ
ದತ್ತಿ-ಬಿಟ್ಟರು
ದತ್ತಿ-ಬಿಟ್ಟ-ರೆಂದು
ದತ್ತಿ-ಬಿಟ್ಟ-ಳೆಂದು
ದತ್ತಿ-ಬಿಟ್ಟಾಗ
ದತ್ತಿ-ಬಿಟ್ಟಿ
ದತ್ತಿ-ಬಿಟ್ಟಿದೆ
ದತ್ತಿ-ಬಿಟ್ಟಿದ್ದ
ದತ್ತಿ-ಬಿಟ್ಟಿದ್ದಾನೆ
ದತ್ತಿ-ಬಿಟ್ಟಿದ್ದಾರೆ
ದತ್ತಿ-ಬಿಟ್ಟಿದ್ದಾಳೆ
ದತ್ತಿ-ಬಿಟ್ಟಿ-ರುತ್ತಾನೆ
ದತ್ತಿ-ಬಿಟ್ಟಿ-ರುವ
ದತ್ತಿ-ಬಿಟ್ಟಿ-ರುವುದು
ದತ್ತಿ-ಬಿಡ-ಲಾಗಿದೆ
ದತ್ತಿ-ಬಿಡ-ಲಾ-ಯಿತು
ದತ್ತಿ-ಬಿಡು-ತಾನೆ
ದತ್ತಿ-ಬಿಡುತ್ತಾನೆ
ದತ್ತಿ-ಬಿಡುತ್ತಾರೆ
ದತ್ತಿ-ಬಿಡುತ್ತಾಳೆ
ದತ್ತಿ-ಬಿಡುವ
ದತ್ತಿ-ಬಿಡು-ವುದು
ದತ್ತಿಯ
ದತ್ತಿ-ಯನ್ನು
ದತ್ತಿ-ಯ-ಮೇರೆ-ಗ-ಳನ್ನು
ದತ್ತಿ-ಯ-ಯಾಗಿ
ದತ್ತಿ-ಯಲ್ಲಿ
ದತ್ತಿ-ಯಾಗಿ
ದತ್ತಿ-ಯಾ-ಗಿತ್ತು
ದತ್ತಿ-ಯಾ-ಗಿದ್ದು
ದತ್ತಿ-ಯಾಗಿ-ಬಿಟ್ಟನು
ದತ್ತಿ-ಯಾಗಿ-ಬಿಟ್ಟ-ನೆಂದು
ದತ್ತಿ-ಯಾಗಿ-ರ-ಬ-ಹುದು
ದತ್ತಿಯು
ದತ್ತಿ-ವೈಷ್ಣವ
ದತ್ತಿ-ಶಾ-ಸನ
ದತ್ತಿ-ಶಾ-ಸನ-ಗ-ಳೆಂದು
ದತ್ತಿ-ಶಾ-ಸನ-ದಲ್ಲಿ
ದತ್ತಿ-ಹಾಕಿ
ದತ್ತಿ-ಹಾಕಿ-ಕೊಟ್ಟ-ನೆಂದು
ದತ್ತಿ-ಹಾಕಿ-ಕೊಟ್ಟಿರುತ್ತಾನೆ
ದತ್ತಿ-ಹಾಕಿ-ಕೊಡ-ಲಾ-ಗಿತ್ತು
ದತ್ತಿ-ಹಾಕಿ-ಕೊಡುತ್ತಾನೆ
ದತ್ತಿ-ಹಾಕಿ-ಕೊಡುತ್ತಾರೆ
ದತ್ತಿ-ಹಾಕಿ-ಕೊಡುತ್ತಾಳೆ
ದತ್ತು
ದತ್ತು-ಪುತ್ರನೇ
ದದತಾ
ದಧ್ಯಾಂನದ
ದಧ್ಯಾನಕ್ಕೆ
ದನ-ಕ-ರು-ಗಳ
ದನ-ಕರು-ಗ-ಳಿಗೆ
ದನ-ಗ-ಳನ್ನು
ದನ-ಗೂರಸ್ಥಳ-ದೊಳ-ಗಣ
ದನ-ಗೂರಿನ
ದನ-ಗೂರು
ದನದ
ದನು
ದನು-ಗೂರಸ್ಥಳದ
ದನು-ಗೂರು
ದಪು-ಯರ-ದಡೆ
ದಬಗ
ದಬ-ಗಾ-ವುಡ
ದಬ್ಬಾಳಿ-ಕೆ-ಯನ್ನು
ದಮ್ಮ-ಗವುಂಡನ
ದಮ್ಮಿ-ಸೆಟ್ಟಿಯ
ದಮ್ಮಿ-ಸೆಟ್ಟಿ-ಯರ
ದಮ್ಮೇಶ್ವರ
ದಯಂಣ
ದಯಂಣ-ದೇವಣ್ಣ
ದಯಣ್ಣ
ದಯ-ಪಾಲಿಸಿ
ದಯ-ಪಾಲಿಸಿದ್ದ
ದಯ-ಪಾಲಿಸುತ್ತಾನೆ
ದಯಾಂಬುಧಿ-ಸೋಮ
ದಯಾ-ಪಾಲ
ದಯಾ-ಪಾಲ-ದೇವ
ದಯಾ-ಪಾಲ-ದೇವ-ನಾಗಿದ್ದಾನೆ
ದಯಾ-ಪಾಲ-ದೇವ-ರಿಗೆ
ದಯಾ-ಪಾಲ-ಮುನಿ
ದಯೆ-ಗೆಯ್ದನಳ್ಕ-ರಿದು
ದಯೆ-ಗೆಯ್ಯೆನ್ದು
ದಯೆಯ
ದಯೆ-ಯನ್ನು
ದರ
ದರ-ಗಿರಿ-ತುಂಗ-ನಿಂದು-ಕು-ಮುದೋಜ್ವಳ-ಕೀರ್ತ್ತಿ
ದರವೇಸ್ಗಳು
ದರಸ-ಗುಪ್ಪೆ-ಯಾಗಿ-ರುವ
ದರಸಿ-ಕುಪ್ಪೆ
ದರಿದ್ರರ
ದರಿದ್ರರು
ದರಿ-ಯಾದೌಲತ್ನಲ್ಲಿ
ದರು-ಶನ
ದರೂ
ದರೋಡೆಯ
ದರ್ಗ
ದರ್ಗಕ್ಕೆ
ದರ್ಗಾ
ದರ್ಗಾಕ್ಕೆ
ದರ್ಗಾದ
ದರ್ಗಾ-ದಲ್ಲಿ
ದರ್ಗಾ-ವನ್ನಾಗಿಸಿ
ದರ್ಗಾವೇ
ದರ್ಜೆ-ಯಲ್ಲಿ
ದರ್ಪ್ಪದ-ಳನ
ದರ್ವೇಷ-ನಿಗೆ
ದರ್ಶನ
ದರ್ಶನಕ್ಕೆ
ದರ್ಶನದ
ದರ್ಶನ-ವನ್ನು
ದರ್ಶನಾರ್ಥ-ವಾಗಿ
ದಲ್ಲಿ
ದಲ್ಲಿ-ಅಳೀ-ಸಂದ್ರ
ದಲ್ಲಿದೆ
ದಲ್ಲಿದ್ದ
ದಲ್ಲಿ-ರುವ
ದಳ
ದಳದ
ದಳದ-ಳ-ವಾಗಿ
ದಳ-ದೆರೆ
ದಳ-ಪತಿ
ದಳ-ಪತಿ-ಗಳ
ದಳ-ಪತಿ-ಗ-ಳಲ್ಲಿ
ದಳ-ಪತಿ-ಗಳಲ್ಲೊಬ್ಬನು
ದಳ-ಪತಿ-ಗ-ಳಾದ
ದಳ-ಪತಿ-ಗಳು
ದಳ-ಪತಿ-ಯಾ-ಗಿದ್ದ
ದಳ-ಭಾರ-ಸಹಿತ
ದಳಮಿಳಿ
ದಳ-ವಾಯಿ
ದಳ-ವಾಯಿ-ಗಳ
ದಳ-ವಾಯಿ-ಗ-ಳಿಂದ
ದಳ-ವಾಯಿ-ಗಳು
ದಳ-ವಾಯಿ-ಯಾದ
ದಳ-ವಿಟ್ಟಿರಿದ
ದಳ-ವಿಟ್ಟಿರಿದಲ್ಲಿ
ದಳೇ
ದಳ್ಳಾಳಿ
ದಳ್ಳಾಳಿ-ಗಳ
ದವನ
ದವ-ರಲ್ಲಿ
ದವರು
ದವಳ-ವಾಯಿ
ದವಸ
ದವ-ಸಕೆ
ದವ-ಸ-ಗಳ
ದವ-ಸ-ಧಾನ್ಯ-ಗಳು
ದವ-ಸಾ-ದಾಯ
ದವ-ಸಿಗ
ದವ-ಸಿಗರ
ದವೆ
ದವೇಶ್ನನ್ನು
ದವ್ಸ
ದಶನೋತ್ಕಂಧರ
ದಶಮೀ
ದಶವಂದ
ದಶವಂದಮುಂ
ದಶವಂದ-ಮುಮಂ
ದಶವಂದ-ವನ್ನು
ದಶವಂದವು
ದಶ-ಸೂರಿ-ಗಳ-ಹತ್ತು-ಜನ
ದಶಾವ-ತಾರದ
ದಶೇ-ಕಾದ-ಶ-ವರ್ಷ-ದಲ್ಲಿ
ದಸವಂದದ
ದಸವಂದ-ವನ್ನು
ದಾಂನ್ಯ
ದಾಕ್ಷಿಂಣ್ಯನಿಧ
ದಾಕ್ಷಿಂಣ್ಯನಿಧಿ
ದಾಖಲಾ-ಗುತ್ತಾ
ದಾಖಲಾತಿ
ದಾಖಲಿಸ-ಲಾಗಿದೆ
ದಾಖಲಿ-ಸಿ-ದಂತೆ
ದಾಖಲಿ-ಸಿದೆ
ದಾಖಲಿ-ಸಿವೆ
ದಾಖಲು-ಪತ್ರ
ದಾಖಲೆ
ದಾಖಲೆ-ಗಳ
ದಾಖಲೆ-ಗ-ಳನ್ನು
ದಾಖಲೆ-ಗ-ಳಲ್ಲಿ
ದಾಖಲೆ-ಯಾಗಿದೆ
ದಾಖಲೆ-ಯಾ-ಗಿದ್ದು
ದಾಖಲೆ-ಯಿದೆ
ದಾಗಿದೆ
ದಾಟಲೂ
ದಾಟಿ
ದಾಡಿಯ
ದಾಡಿಯ-ಗಡ್ಡದ
ದಾಡಿಯ-ಸೋಮೆಯ
ದಾದಂ
ದಾದಾಜಿ
ದಾದಿ
ದಾದುರಿ
ದಾದೋಜಿ
ದಾನ
ದಾನಕ್ಕೆ
ದಾನ-ಗ-ಳನ್ನು
ದಾನ-ಗ-ಳನ್ನೂ
ದಾನ-ಗುಣ
ದಾನ-ಗುಣಾಶ್ರಯ
ದಾನದ
ದಾನ-ದಂನ-ಪುರದ
ದಾನ-ದತ್ತಿ-ಗ-ಳನ್ನು
ದಾನ-ದುನ್ನತಿ-ಯಿಂದ
ದಾನ-ದೊಳು
ದಾನ-ಧರ್ಮ
ದಾನ-ಧರ್ಮ-ಗಳದು
ದಾನ-ಧರ್ಮ-ಗ-ಳನ್ನು
ದಾನ-ಧರ್ಮ-ಗ-ಳಿಗೆ
ದಾನ-ಧರ್ಮದ
ದಾನ-ಧರ್ಮ-ವೆಂದು
ದಾನ-ಧರ್ಮ್ಮದ
ದಾನಪ್ರವ್ರುತ್ತಯೇ
ದಾನ-ಬಿಟ್ಟಿದ್ದಾನೆ
ದಾನ-ಮದ್ಭುತಂ
ದಾನ-ಮಾಡಿ-ದನು
ದಾನ-ಮಾನ್ಯ-ವಾಗಿ
ದಾನ-ವನ್ನು
ದಾನ-ವಾಗಿ
ದಾನ-ವಿನೋದೆ
ದಾನ-ಶಾ-ಸಣ
ದಾನ-ಶಾ-ಸನ-ವಿದೆ
ದಾನಶ್ರೇಯಾಂಸಂ
ದಾನ-ಸಾಲೆ-ಯನ್ನು
ದಾನಸ್ಯ
ದಾನಾದ
ದಾನಾಧ-ಮನ-ವಿಕ್ರೀತ-ಯೋಗ್ಯ
ದಾನಿ-ಗಳ
ದಾನಿಗೆ
ದಾನಿಯ
ದಾನಿಯು
ದಾಮ
ದಾಮ-ಣ-ನಂದಿ
ದಾಮಣ್ಣ
ದಾಮಣ್ಣನು
ದಾಮಣ್ಣ-ನೆಂದು
ದಾಮ-ನಂದಿ
ದಾಮ-ನಂದಿತ್ರೈ-ವಿದ್ಯ-ಮುನೀಶ್ವರ
ದಾಮ-ನಂದಿಯ
ದಾಮನೂ
ದಾಮ-ನೆಂಬ
ದಾಮನೆಯ್ದನೆ
ದಾಮ-ಪಯ್ಯ-ನನ್ನು
ದಾಮ-ಪಯ್ಯ-ನೆಂಬು-ವ-ನನ್ನು
ದಾಮ-ರಲೈಯಪೇಂದ್ರ
ದಾಮಾದ್
ದಾಮೋದರ
ದಾಮೋದ-ರ-ದೇವನ
ದಾಮೋದ-ರನ
ದಾಮೋದ-ರ-ನಾ-ಯ-ಕರು
ದಾಮೋದ-ರನು
ದಾಮೋದ-ರಯ್ಯನು
ದಾಯಾದ
ದಾಯಾದಿ
ದಾಯಾದಿ-ಗಳ
ದಾಯಾದಿ-ಗಳು
ದಾಯಾದಿ-ಯಾಗಿ-ರ-ಬ-ಹುದು
ದಾಯಾದ್ಯ
ದಾಯಾದ್ಯಕ್ಕೆ
ದಾಯಿಗ-ಬೇಂಟೆ-ಕಾರ
ದಾಯೋಜನ
ದಾಯೋಜನು
ದಾಯ್ಗರು
ದಾರಿ
ದಾರಿ-ಗಳ್ಳ-ರೊಡನೆ
ದಾರಿರ್ಯ-ವಿದ್ರಾ-ವಣಂ
ದಾರಿ-ಯಲ್ಲಿ
ದಾರಿ-ಯಲ್ಲಿಯೇ
ದಾರಿ-ಯಲ್ಲಿ-ರುವ
ದಾರುಸ್ಸಲ್ತ-ನತ್
ದಾರ್ಶನಿಕ
ದಾಳಿ
ದಾಳಿ-ಕಾರರು
ದಾಳಿ-ಗ-ಳಲ್ಲಿ
ದಾಳಿ-ಗಿ-ದಿರು
ದಾಳಿಗೆ
ದಾಳಿ-ಮಾಡಿ
ದಾಳಿಯ
ದಾಳಿ-ಯನ್ನು
ದಾಳಿ-ಯಲ್ಲಿ
ದಾಳಿ-ಯಿಂದ
ದಾವಂಣನ
ದಾವಂಣನ-ಕೆರೆಯ
ದಾವಯ್ಯನ
ದಾವಹವಿ
ದಾವಾನಳ
ದಾಶ-ರಾಜ-ನಲ್ಲಿಗೆ
ದಾಸ
ದಾಸ-ಗ-ವುಡನ
ದಾಸ-ಗವು-ಡ-ನ-ಕಟ್ಟೆ-ಯನ್ನೂ
ದಾಸನ
ದಾಸ-ನ-ದೊಡ್ಡಿ
ದಾಸ-ನ-ಪುರ
ದಾಸ-ನಾದ
ದಾಸನು
ದಾಸ-ನೂರು
ದಾಸ-ನೆಂಬು-ವ-ವನು
ದಾಸ-ಪ-ನಾಯ-ಕರ
ದಾಸಪ್ಪ-ನಾಯ-ಕನ
ದಾಸ-ಯ-ಹೆಗ್ಗಡೆ
ದಾಸಯ್ಯ
ದಾಸಯ್ಯ-ನೆಂದರೆ
ದಾಸರ
ದಾಸ-ರಿಗೆ
ದಾಸರು
ದಾಸ-ರೆಂಬು-ವರು
ದಾಸರ್
ದಾಸಾನು-ದಾಸ-ನೆಂದು
ದಾಸಿ
ದಾಸಿ-ಮಯ್ಯ
ದಾಸಿ-ಮಯ್ಯನ
ದಾಸಿ-ಮಯ್ಯ-ನ-ವರ
ದಾಸೋ-ಜನ
ದಾಸೋ-ಜನಗ್ರತ-ನೆಯಮ್ಮ-ಸಣಂ
ದಾಸೋಜ-ನಿಗೆ
ದಾಸೋಜ-ನೆಂಬು-ವ-ವನು
ದಾಸೋಹಕ್ಕೆ
ದಾಸ್ತಾನು
ದಿ
ದಿಂಡಕ
ದಿಂಡಿಕ
ದಿಂಡಿಕಗ
ದಿಂಡಿಗ
ದಿಂಡಿಗನ
ದಿಂಡಿಗ-ನ-ಕೆರೆಯ
ದಿಂಡಿಗ-ನಾಡಿನ
ದಿಂಡಿಗ-ನಾ-ಡಿಯರು
ದಿಂಡಿಗನು
ದಿಂಡಿಗ-ಮಹಾಪ್ರಭುವೇ
ದಿಂಡಿಗ-ರಾಜ
ದಿಂಡಿಗ-ರಾಜನು
ದಿಂಡಿಗರು
ದಿಂಡಿಗಲ್ಲಿನ
ದಿಂದ
ದಿಂಮ-ರಾಜಯ್ಯನು
ದಿಕ್ಕನ್ನು
ದಿಕ್ಕಿಗೆ
ದಿಕ್ಕಿನ
ದಿಕ್ಕಿನಲ್ಲಿ
ದಿಕ್ಕಿನಲ್ಲಿ-ರುವ
ದಿಕ್ಕಿನಿಂದ
ದಿಕ್ಕು
ದಿಕ್ಕು-ಗಳಿಗೂ
ದಿಕ್ಕೆಟ್ಟನು
ದಿಕ್ಚಕ್ರಂಗಳೊಳ್
ದಿಗಂಬರ
ದಿಗಂಬರ-ರಾವ್
ದಿಗ್ವಿಜಯ
ದಿಗ್ವಿಜ-ಯಕ್ಕೆ
ದಿಗ್ವಿಜಯ-ಗ-ಳನ್ನು
ದಿಗ್ವಿಜಯದ
ದಿಗ್ವಿಜಯಾರ್ಥ-ವಾಗಿ
ದಿಡಗ
ದಿಡ-ಗದ
ದಿಡಗ-ವಾಗಿದೆ
ದಿಡ-ಗವು
ದಿಡುಗ-ವನ್ನು
ದಿಣ್ಡಿಗ
ದಿಣ್ಡಿಗ-ಕೂಡ-ಲೂರು
ದಿಣ್ಡಿಗೋ
ದಿಣ್ಣೆ
ದಿಣ್ಣೆಯ
ದಿದು
ದಿನ
ದಿನಂಪ್ರತಿ
ದಿನಂಪ್ರತಿ-ನಡೆ-ಯುವ
ದಿನ-ಕರ-ರೆಂದು
ದಿನಕ್ಕೆ
ದಿನ-ಗಟ್ಟ-ಳೆಯ
ದಿನ-ಗಳ
ದಿನ-ಗ-ಳಿಗೆ
ದಿನ-ಗಳಿದ್ದ
ದಿನ-ಚರಿ-ಯಲ್ಲಿ
ದಿನ-ದಂದು
ದಿನ-ದಲಿ
ದಿನ-ದಲು
ದಿನ-ದಲ್ಲಿ
ದಿನ-ದಿ-ನದ
ದಿನ-ನಿತ್ಯದ
ದಿನ-ಬಳ-ಕೆಗೆ
ದಿನ-ಬಳ-ಕೆಯ
ದಿನ-ವನ್ನು
ದಿನವೂ
ದಿನವೇ
ದಿನ-ವೊಂದಕ್ಕೆ
ದಿನ-ಸಿಗಳ
ದಿನಾ
ದಿನಾಂಕ
ದಿನಾಂಕದ
ದಿನ್
ದಿಬ್ಯ
ದಿಬ್ಯಬ್ರತ
ದಿಬ್ಯಬ್ರತ-ಸಮಿ-ತಿಗೆ
ದಿಬ್ಯ-ಮುನಿಗಂ
ದಿಲ್ಲಿಗೆ
ದಿವಂಗತ
ದಿವಂಗತ-ನಾದಾಗ
ದಿವಸ
ದಿವಸ-ಗಳ
ದಿವಾ-ಕರಂ
ದಿವಾ-ಕರ-ಣಂದಿ
ದಿವಾ-ಕರ-ಣಂದಿಯ
ದಿವಾ-ಕರ-ನೆಂದು
ದಿವಾ-ಕರ-ನೆನಿಸಿದ
ದಿವಾನರ
ದಿವಾನ್
ದಿವಾರ-ಕರ-ಣಂದಿ
ದಿವಿಜ-ಲಲನೆ-ಯರು
ದಿವ್ಯ
ದಿವ್ಯಜ್ಞಾನ-ದಿಂದ
ದಿವ್ಯ-ತಿರು-ಮಾಲೆ
ದಿವ್ಯ-ದೇಶ-ಗ-ಳಲ್ಲಿ
ದಿವ್ಯ-ದೇಶ-ದಲ್ಲಿ
ದಿವ್ಯ-ದೇಶ-ವಾ-ಗಿದ್ದ
ದಿವ್ಯ-ದೇಶ-ವಾದ
ದಿವ್ಯ-ಮುನಿ-ವರ-ನೆಂದು
ದಿವ್ಯ-ಲಕ್ಷ್ಮೀ-ದೇವಿ-ಯರ
ದಿವ್ಯ-ಲೀಲಾ-ವಿಲಾಸಕ್ಕೆ
ದಿವ್ಯ-ವನ್ನು
ದಿವ್ಯ-ವಾ-ಹನ
ದಿವ್ಯವ್ರತ
ದಿವ್ಯಶ್ರೀ
ದಿವ್ಯಶ್ರೀ-ಪಾದ-ಪದ್ಮದ
ದಿವ್ಯೋತ್ಸವ-ವನ್ನೂ
ದಿಶಾ-ಪಟ್ಟನುಂ
ದಿಶಿಸ್ಥಿತಂ
ದಿಶೆ-ಯಲ್ಲಿ
ದಿಸೆ-ಯಲ್ಲಿ
ದೀಕ್ಷಾ
ದೀಕ್ಷಾ-ಗುರು-ವೆಂದು
ದೀಕ್ಷಿತ
ದೀಕ್ಷಿತಅ
ದೀಕ್ಷಿತಿ
ದೀಕ್ಷಿತ್
ದೀಕ್ಷಿತ್ರ-ವರು
ದೀಕ್ಷೆ
ದೀಕ್ಷೆಯ
ದೀಕ್ಷೆ-ಯನ್ನು
ದೀನ-ಜನ-ಕೋಟಿಗೆ
ದೀನಾ-ನಾಥ
ದೀಪ
ದೀಪಕ
ದೀಪಕಃ
ದೀಪಕೆ
ದೀಪಕೋ
ದೀಪಕ್ಕೆ
ದೀಪ-ಮಾಲೆ
ದೀಪ-ಮಾಲೆ-ಕಂಬ
ದೀಪ-ಮಾಲೆ-ಕಂಬ-ವನ್ನು
ದೀಪ-ಮಾಲೆಯ
ದೀಪ-ವನ್ನು
ದೀಪಾಂಕುರನಂ
ದೀಪಾಂಕುರರು
ದೀಪಾರಾ-ಧನೆ
ದೀಪಾರಾ-ಧನೆಗೆ
ದೀಪಿತ
ದೀಪೋತ್ಸವ
ದೀರ್ಘ
ದೀರ್ಘ-ಕಾಲ
ದೀರ್ಘ-ಕಾಲ-ದಿಂದ
ದೀರ್ಘ-ವಾಗಿ
ದೀರ್ಘ-ವಾದ
ದೀರ್ಘ-ಸು-ಮಂಗಲ
ದೀರ್ಘಾ-ಯುವಂ
ದೀರ್ಘಾಯು-ವಾಗ-ಲೆಂದು
ದೀವಿಗೆ
ದೀವಿಗೆಗೆ
ದೀವಿಗೆಯ
ದು
ದುಂಡು
ದುಂಡು-ವಿನ
ದುಂಡುವು
ದುಂಡುವೇ
ದುಂಡು-ಸ-ಮುದ್ರದಾ
ದುಂಡೇನ-ಹಳ್ಳಿಯ
ದುಃಖ-ದಿಂದ
ದುಇಪಹ-ರರಾಉತು
ದುಗದ
ದುಗ್ಗನ-ಹಳ್ಳಿಯ
ದುಗ್ಗ-ಮಾರನ
ದುಗ್ಗ-ಮಾರನು
ದುಗ್ಗಯ್ಯಂ
ದುಗ್ಗಲ-ದೇವಿ
ದುಗ್ಗಲೆ
ದುಗ್ಗಲೆ-ಯನ್ನು
ದುಗ್ಗವೆ
ದುಗ್ಗವ್ವೆ
ದುಗ್ಗವ್ವೆ-ಯ-ರಿಗೆ
ದುಡೋ-ಜನು
ದುದ್ದ
ದುದ್ದದ
ದುದ್ದ-ಮಲ್ಲ-ದೇವನ
ದುದ್ದ-ಮಲ್ಲ-ದೇವನು
ದುಪ-ಗಟ್ಟದ
ದುಬಾಸಿ
ದುಬಿ-ಗಾವುಂಡಿ
ದುಬಿ-ಗಾವುಂಡಿ-ಯನ್ನು
ದುಬಿ-ಗಾವುಂಡಿ-ಯರ
ದುಮ್ಮ-ಸ-ಮುದ್ರ
ದುಮ್ಮೆ-ತನಕ
ದುಮ್ಮೆಯ
ದುಮ್ಮೆಯ-ನಾಯಕ
ದುಮ್ಮೆಯ-ನಾಯ-ಕನ
ದುಮ್ಮೆಯ-ನಾಯ-ಕನು
ದುಮ್ಮೆಯ-ನಾಯ-ಕರು
ದುಮ್ಮೆ-ವರೆಗೆ
ದುರಂಧರ-ನೆಂದು
ದುರಂಧರ-ರಾದ
ದುರಂಧರೆ-ಯಾದ
ದುರಂಧರೋ
ದುರಸ್ತಿ
ದುರಸ್ತಿ-ಗೊಳಿಸುತ್ತಿದ್ದ
ದುರಸ್ತೆ
ದುರಾದೃಷ್ಟ-ಕರ
ದುರಾದೃಷ್ಟವಶಾತ್
ದುರಿತ-ದೂರಂ
ದುರಿತಧ್ವಂಸಿ-ವೇಷಾಯ
ದುರ್ಗ-ಗ-ಳನ್ನು
ದುರ್ಗದ
ದುರ್ಗ-ದೊಳಗೆ
ದುರ್ಗಮ-ನುರ-ವ-ಣೆಯಿಂ
ದುರ್ಗ-ವನಾ-ಳುವಲ್ಲಿ
ದುರ್ಗ-ವನ್ನು
ದುರ್ಗ-ವಾದ
ದುರ್ಗಾಧಿ-ಪತಿ
ದುರ್ಗಾಧಿ-ಪತಿ-ಗಳು
ದುರ್ಗಿ
ದುರ್ಜನ
ದುರ್ಬಲ
ದುರ್ಬಲ-ವಾ-ಗಿದ್ದ
ದುರ್ಬಲ-ವಾಯಿತು
ದುರ್ಮ್ಮಣ್ಣನು
ದುರ್ಯೋಧ-ನನು
ದುಷ್ಟಜ-ನದುರ್ಲಭ
ದುಷ್ಟ-ನಿರ್ಮೂಲ-ನೆ-ಗಾಗಿ
ದುಷ್ಟ-ಶಾರ್ದೂಲ-ಮರ್ದನಃ
ದುಷ್ಟ-ಸಿದ್ಧಾಂತಾ
ದುಸ್ಥಿತಿ-ಯಲ್ಲಿದೆ
ದುಸ್ಥಿತಿ-ಯಲ್ಲಿದ್ದಾಗ
ದುಸ್ಸಾಧ್ಯ-ವಾದ
ದೂರ
ದೂರದ
ದೂರ-ದಲ್ಲಿ
ದೂರ-ದಲ್ಲಿ-ರುವ
ದೂರ-ದಲ್ಲಿವೆ
ದೂರ-ದೂರ-ದಲ್ಲಿವೆ
ದೂರ-ವಾಣಿ
ದೂರ-ವಿ-ರುವ
ದೃಢಂ
ದೃಢ-ಪ-ಡಿಸಿ
ದೃಢ-ಪಡಿ-ಸುತ್ತದೆ
ದೃಢ-ಪಡಿ-ಸುತ್ತವೆ
ದೃಢಪಡುತ್ತದೆ
ದೃಷ್ಟಿಕೋನ-ಗ-ಳಿಂದ
ದೃಷ್ಟಿ-ಯಲ್ಲಿಟ್ಟು-ಕೊಂಡು
ದೃಷ್ಟಿ-ಯಿಂದ
ದೃಷ್ಟಿ-ಯಿಂದಲೂ
ದೃಷ್ಟಿ-ಯಿಂದಲೇ
ದೃಷ್ಟಿ-ಯಿಂದಲೋ
ದೆಂದು
ದೆಂದೂ
ದೆತ್ತಿದ
ದೆಲೆ-ಗೌಡ
ದೆವ-ರಾಜೊಡೆಯರ
ದೆಸೆ
ದೆಹ-ಲಿಯ
ದೇ
ದೇಕಣ್ಣ
ದೇಕಬ್ಬೆ
ದೇಕಬ್ಬೆಯ
ದೇಕಬ್ಬೆ-ಯನ್ನು
ದೇಕವೆ-ದಂಡ-ನಾಯ-ಕಿತಿಯ
ದೇಕವ್ವೆ
ದೇಕವ್ವೆಗೆ
ದೇಕಿ-ಸೆಟ್ಟಿ
ದೇಕೆಯ-ನಾಯಕ
ದೇಕೆಯ-ನಾಯ-ಕನ
ದೇಕೆಯ-ನಾಯ-ಕನು
ದೇಕೆಯ-ನಾಯ-ಕರ
ದೇಕೆಯ-ನಾಯ-ಕರು
ದೇಗಲು
ದೇಗಳಂ
ದೇಗುಲ-ಗ-ಳನ್ನು
ದೇಗುಲ-ಗಳೆನಿ-ತಾನುಂ
ದೇಗುಲ-ಗೌಂಡಿಯ
ದೇಗುಲ-ಗೌಣ್ಡಿ
ದೇಗುಲ-ಗೌಣ್ಡಿ-ಯನ್ನು
ದೇಗುಲದ
ದೇಗುಲಮಂ
ದೇಗುಲ-ಮ-ನೆತ್ತಿಸಿ
ದೇಗುಲ-ವನ್ನು
ದೇಚಲ
ದೇಚಲ-ನಾರಿ
ದೇಚಲೆ
ದೇಪಂಣೊಡೆ-ಯರ
ದೇಪಣ್ಣ
ದೇಪಯ-ನಾಮ-ಧೇಯೋ
ದೇಪಯಸ್ತು
ದೇಪಯ್ಯ
ದೇಪಯ್ಯನ
ದೇಪಯ್ಯ-ನನ್ನು
ದೇಪಯ್ಯನು
ದೇಮಲ-ದೇವಿ
ದೇಮಲ-ದೇವಿ-ದೇವ-ಲ-ದೇವಿ
ದೇಮಲ-ದೇವಿಯು
ದೇಮಲ-ಮಹಾ-ಸ-ಮುದ್ರ-ವೆಂಬ
ದೇಮಲಾ-ದೇವಿಯ
ದೇಮಲಾ-ದೇವಿ-ಯನ್ನು
ದೇಮಲಾ-ಪುರ-ವೆಂಬ
ದೇಮ-ಸ-ಮುದ್ರ
ದೇಮಾಂಬಿಕಾ
ದೇಮಾಂಬಿಕೆ
ದೇಮಾಂಬಿಕೆ-ಯರ
ದೇಮಾಂಬಿ-ಕೆಯು
ದೇಮಿ-ಕಬ್ಬೆ
ದೇಮಿ-ಕಬ್ಬೆ-ಯರು
ದೇಯ-ಮಿತಿ
ದೇಯಾತು
ದೇವ
ದೇವ-ಕರ್ಮಿ
ದೇವ-ಕಾರ್ಯ-ವನ್ನು
ದೇವಕಿ
ದೇವ-ಕಿಗೆ
ದೇವ-ಕೀರ್ತಿ-ಪಂಡಿತ
ದೇವ-ಕೀರ್ತಿ-ಪಂಡಿ-ತರ
ದೇವ-ಕು-ಮಾರ
ದೇವ-ಕೆರೆಗೆ
ದೇವ-ಕೆರೆ-ಯೆಂಬ
ದೇವಕ್ಷೇತ್ರ
ದೇವಕ್ಷೇತ್ರ-ದಲ್ಲಿ
ದೇವಕ್ಷೇತ್ರ-ವಾದ
ದೇವಕ್ಷೇತ್ರ-ವೆಂದರೆ
ದೇವ-ಗ-ರಾಣೆ
ದೇವ-ಗಿರಿ
ದೇವ-ಗಿರಿ-ಪಿ-ರಾಟ್ಟಿ
ದೇವ-ಗಿರಿಯ
ದೇವ-ಚಂದ್ರ
ದೇವ-ಚಂದ್ರನ
ದೇವ-ಚಂದ್ರನು
ದೇವ-ಚಂದ್ರ-ಪಂಡಿತ
ದೇವ-ಚಂದ್ರ-ಪಂಡಿ-ತರ
ದೇವ-ಜಾಂಬಾ
ದೇವ-ಡಿಗ-ಗೆ-ದೇವಾ-ಡಿಗ-ನಿಗೆ
ದೇವಣ
ದೇವಣಃ
ದೇವಣ್ಣ-ಗಳ
ದೇವಣ್ಣನ
ದೇವಣ್ಣನು
ದೇವತಾ
ದೇವ-ತಾ-ಕಾರ್ಯ-ವನ್ನು
ದೇವ-ತಾ-ಕಾರ್ಯ್ಯ
ದೇವ-ತಾ-ಗೃಹ-ಮಲ್ಲಿ-ಕಾರ್ಜುನ
ದೇವ-ತಾಗ್ರಾಮ
ದೇವ-ತಾಗ್ರಾಮಂ
ದೇವ-ತಾ-ಪೂಜೆಗೆ
ದೇವ-ತಾ-ಮಂದಿರ
ದೇವ-ತೆ-ಗಳ
ದೇವ-ತೆ-ಗ-ಳನ್ನು
ದೇವ-ತೆ-ಗ-ಳಾಗಿ-ರ-ಬ-ಹುದು
ದೇವ-ತೆ-ಗಳು
ದೇವ-ತೆಯ
ದೇವ-ತೆ-ಯನ್ನು
ದೇವ-ತೆ-ಯಾಗಿದ್ದಾನೆ
ದೇವ-ತೆ-ಯಾಗಿ-ರ-ಬ-ಹುದು
ದೇವ-ತೆ-ಯಾದ
ದೇವ-ತೆ-ಯಾದರೂ
ದೇವ-ತೆ-ಯಾದರೆ
ದೇವ-ದಂಡ-ನಾಯ-ಕ-ನಿಗೆ
ದೇವ-ದಾನ
ದೇವ-ದಾನಕ್ಕೆ
ದೇವ-ದಾನದ
ದೇವ-ದಾನ-ವನ್ನು
ದೇವ-ದಾನ-ವಾಗಿ
ದೇವ-ದಾ-ಯವು
ದೇವ-ದೇ-ವೋತ್ತಮ
ದೇವದ್ವಿಜ-ಬಂಧು-ಮಿತ್ರ-ವರ್ಗ್ಗಾಣಾಂ
ದೇವನ
ದೇವನಂ
ದೇವ-ನಂದಿ
ದೇವ-ನ-ಣು-ಗಿನರ್ಕ್ಕರಿನ
ದೇವ-ನ-ದಾನದ
ದೇವ-ನನ್ನು
ದೇವನಿಂ
ದೇವ-ನಿಂದ
ದೇವ-ನಿಗೆ
ದೇವ-ನಿಗೇ
ದೇವ-ನಿದ್ದ-ನೆಂದೂ
ದೇವ-ನೀಗೆ
ದೇವನು
ದೇವನೂ
ದೇವ-ನೂ-ರನ್ನು
ದೇವ-ನೂರು
ದೇವ-ನೆಂದು
ದೇವ-ನೆಂಬ
ದೇವನೇ
ದೇವ-ನೊಡನೆ
ದೇವನ್
ದೇವ-ಪರ್ವ
ದೇವ-ಪುರ-ದೊಳಗೆ
ದೇವ-ಪುರಿ
ದೇವ-ಪುರಿ-ಇಂದಿನ
ದೇವ-ಪೂಜಾ-ದಾನ-ಧರ್ಮ್ಮ
ದೇವ-ಪೂಜೆ
ದೇವ-ಪೆರು-ಮಾಳಿಗೆ
ದೇವಪ್ಪ
ದೇವಪ್ಪನ
ದೇವಪ್ಪ-ನಾಯಕ
ದೇವಪ್ಪ-ನಾಯ-ಕನ
ದೇವಪ್ಪ-ನಾಯ-ಕನು
ದೇವಪ್ಪ-ನಾ-ಯ-ಕರು
ದೇವಪ್ಪನು
ದೇವಪ್ಪಿಳ್ಳೆಗೂ
ದೇವಪ್ಪಿಳ್ಳೆಯ
ದೇವಪ್ಪಿಳ್ಳೆ-ಯರ
ದೇವಪ್ಪಿಳ್ಳೈ
ದೇವಬ್ರಾಹ್ಮಣ
ದೇವಬ್ರಾಹ್ಮಣ-ರಿಗೆ
ದೇವ-ಭಟ್ಟ-ರಿಗೆ
ದೇವ-ಭಾಗ
ದೇವ-ಭು-ವನೇ
ದೇವ-ಭೂಮಿ-ಯಾಗಿ
ದೇವ-ಭೋಗ-ವನ್ನು
ದೇವ-ಮಾಂಬ
ದೇವ-ಮಾಂಬ-ದೇವಾ-ಜಮ್ಮಣ್ಣಿ
ದೇವ-ಮಾನ್ಯ-ವನ್ನು
ದೇವ-ಮಾನ್ಯ-ವಾಗಿ
ದೇವಮ್ಮ
ದೇವಯ್ಯ
ದೇವಯ್ಯ-ಗಳ
ದೇವಯ್ಯನು
ದೇವರ
ದೇವ-ರಂಗ-ಭೋ-ಗಕ್ಕೆ
ದೇವ-ರ-ಇಂದಿನ
ದೇವ-ರ-ಕಟ್ಟೆಗೆ
ದೇವ-ರ-ಕೆರೆ
ದೇವ-ರ-ಕೆರೆಯ
ದೇವ-ರ-ಕೆರೆ-ಯನ್ನು
ದೇವ-ರ-ಕೊಂಡಾ
ದೇವ-ರ-ಕೊಂಡಾ-ರೆಡ್ಡಿ
ದೇವ-ರ-ಕೊಂಡಾ-ರೆಡ್ಡಿ-ಯ-ವರ
ದೇವ-ರ-ಕೊಂಡಾ-ರೆಡ್ಡಿ-ಯ-ವರು
ದೇವ-ರ-ಗುಡಿ-ಕಟ್ಟೆ
ದೇವ-ರ-ದತ್ತಿ
ದೇವ-ರ-ದರ್ಶನ
ದೇವ-ರನು
ದೇವ-ರನ್ನು
ದೇವ-ರ-ಪಡಿಯ
ದೇವ-ರಪ್ರತಿಷ್ಠೆಗೆ
ದೇವ-ರ-ಬಳಿ
ದೇವ-ರ-ಭಟ್ಟ-ನಿಗೆ
ದೇವ-ರ-ಭಟ್ಟ-ರಿಗೆ
ದೇವ-ರಸ
ದೇವ-ರ-ಸ-ಗವುಡ
ದೇವ-ರ-ಸನ
ದೇವ-ರ-ಸ-ನನ್ನು
ದೇವ-ರ-ಸನು
ದೇವ-ರ-ಸರ
ದೇವ-ರ-ಸ-ರಿಗೆ
ದೇವ-ರ-ಸರು
ದೇವ-ರ-ಹಳ್ಳಿ
ದೇವ-ರ-ಹಳ್ಳಿ-ಗಳು
ದೇವ-ರ-ಹಳ್ಳಿಯ
ದೇವ-ರ-ಹಳ್ಳಿ-ಯನ್ನು
ದೇವ-ರ-ಹಳ್ಳಿ-ಯಾಗಿದೆ
ದೇವ-ರಾಜ
ದೇವ-ರಾಜ-ದೇವ-ರಾಜ
ದೇವ-ರಾಜನ
ದೇವ-ರಾಜ-ನನ್ನು
ದೇವ-ರಾಜ-ನಿಗೆ
ದೇವ-ರಾಜನು
ದೇವ-ರಾಜ-ಪುರ
ದೇವ-ರಾಜ-ಪುರ-ವಾದ
ದೇವ-ರಾಜ-ಪುರ-ವೆಂದು
ದೇವ-ರಾಜ-ಪುರ-ವೆಂಬ
ದೇವ-ರಾಜ-ಭೂ-ಪಾಲನು
ದೇವ-ರಾಜ-ಮಹೀ-ಪಾಲ-ಕರು
ದೇವ-ರಾಜ-ಮಹೀ-ಪಾಲರು
ದೇವ-ರಾಜಮ್ಮಣ್ಣಿ-ಯ-ವರು
ದೇವ-ರಾ-ಜಯ್ಯ
ದೇವ-ರಾಜಯ್ಯ-ದೇವನ
ದೇವ-ರಾಜಯ್ಯನ
ದೇವ-ರಾಜಯ್ಯನು
ದೇವ-ರಾಜರ
ದೇವ-ರಾಜರು
ದೇವ-ರಾಜ-ವೊಡೆ-ಯನು
ದೇವ-ರಾಜೇಂದ್ರ
ದೇವ-ರಾಜೇಂದ್ರ-ನಿಗೆ
ದೇವ-ರಾಜೊಡೆಯನೂ
ದೇವ-ರಾಜೊಡೆಯರ
ದೇವ-ರಾಜೊಡೆಯರು
ದೇವ-ರಾಜ್ಯಂ
ದೇವ-ರಾದ
ದೇವ-ರಾಯ
ದೇವ-ರಾಯನ
ದೇವ-ರಾಯ-ನನ್ನು
ದೇವ-ರಾಯ-ನಿಂದ
ದೇವ-ರಾಯ-ನಿಗೆ
ದೇವ-ರಾಯನು
ದೇವ-ರಾಯ-ನೆಂದು-ಇಮ್ಮಡಿ
ದೇವ-ರಾಯ-ಪಟ್ಟಣ
ದೇವ-ರಾಯ-ಪಟ್ಟ-ಣ-ವೆಂಬ
ದೇವ-ರಾಯ-ಪುರ-ವಾದ
ದೇವ-ರಾಯಪ್ರೌಢ-ದೇವ-ರಾಯ
ದೇವ-ರಾಯ-ಮಹಾ-ರಾಯರ
ದೇವ-ರಾಯ-ವಟ್ಟ
ದೇವ-ರಿಗೆ
ದೇವ-ರಿರ-ಬ-ಹುದು
ದೇವರು
ದೇವ-ರು-ಗಳ
ದೇವ-ರು-ಗ-ಳನ್ನು
ದೇವ-ರು-ಗ-ಳನ್ನೂ
ದೇವ-ರು-ಗ-ಳಿಗೆ
ದೇವ-ರು-ಗಳು
ದೇವ-ರು-ಶುಭ-ಚಂದ್ರ
ದೇವ-ರುಶ್ರೀ
ದೇವ-ರೆಂದು
ದೇವ-ರೆಂದೇ
ದೇವ-ರೆಂಬ
ದೇವರೇ
ದೇವ-ರೊಡೆ-ಯರ
ದೇವ-ರೊಲಾ-ಸನ-ದಲಿರ್ದ್ದನು
ದೇವರ್
ದೇವರ್ವಲ್ಲ-ವನ್
ದೇವಲ
ದೇವ-ಲ-ಪುರ-ವಾಗಿ
ದೇವ-ಲ-ಪುರ-ವಾದ
ದೇವ-ಲ-ಪುರ-ವೆಂಬ
ದೇವ-ಲ-ಮಹಾ-ಸ-ಮುದ್ರ
ದೇವ-ಲಾ-ದೇವಿ
ದೇವ-ಲಾ-ಪ-ರದ
ದೇವ-ಲಾ-ಪುರ
ದೇವ-ಲಾ-ಪುರದ
ದೇವ-ಲಾ-ಪುರ-ದಲ್ಲಿ
ದೇವ-ಲಾ-ಪುರ-ವನ್ನು
ದೇವ-ಲಾ-ಪುರವು
ದೇವ-ಲಾ-ಪುರವೂ
ದೇವ-ಲಾ-ಪುರ-ವೆಂಬ
ದೇವ-ಲಾ-ಪುರವೇ
ದೇವ-ಲಾ-ಪುರಸ್ಥಳದ
ದೇವ-ಲೋ-ಕಕ್ಕೆ
ದೇವಲ್ಯನ್ಯೆತ್ತಿ
ದೇವ-ವೃಂದ
ದೇವ-ಶಿಖಾ-ಮಣಿ
ದೇವ-ಸತ್ತಿ
ದೇವ-ಸ-ಮುದ್ರ
ದೇವ-ಸಿದ್ಧಾಂತಿ-ಗರಂ
ದೇವ-ಸಿದ್ಧಾಂತಿ-ಗರು-ಕುಕ್ಕುಟಾ-ಸನ
ದೇವ-ಸೆಟ್ಟಿ-ಜೀಯರ
ದೇವಸ್ಥಾನ
ದೇವಸ್ಥಾನಕ್ಕೆ
ದೇವಸ್ಥಾನ-ಗ-ಳಲ್ಲಿ
ದೇವಸ್ಥಾನ-ಗ-ಳಿಗೆ
ದೇವಸ್ಥಾನ-ಗಳು
ದೇವಸ್ಥಾನದ
ದೇವಸ್ಥಾನ-ವನ್ನು
ದೇವಾ
ದೇವಾಂಗ
ದೇವಾಂಗ-ದ-ವರ
ದೇವಾಂಗ-ಮಗ್ಗ-ದ-ಶೆಟ್ಟರು
ದೇವಾಂಬಾ
ದೇವಾಂಬಾ-ದೇವಿ
ದೇವಾ-ಕಳಂಕಃ
ದೇವಾ-ಗಾರಮಂ
ದೇವಾ-ಜ-ಮಾಂಬ
ದೇವಾ-ಜಮ್ಮ
ದೇವಾ-ಜಮ್ಮಣ್ಣಿಯು
ದೇವಾ-ಜಮ್ಮ-ನ-ವರ
ದೇವಾ-ಡಿಗ
ದೇವಾ-ದಯ
ದೇವಾ-ದಾಯ
ದೇವಾ-ದೇಯ
ದೇವಾ-ದೇಯ-ವೆಂದೂ
ದೇವಾ-ಪುರ
ದೇವಾಪ್ರುಥುವೀ
ದೇವಾ-ರಾಧ್ಯರು
ದೇವಾ-ಲ-ಗಳಿ-ರುವುದು
ದೇವಾ-ಲ-ಪುರದ
ದೇವಾ-ಲಯ
ದೇವಾ-ಲಯಂ
ದೇವಾ-ಲಯ-ಅಂಕ-ನ-ಹಳ್ಳಿ
ದೇವಾ-ಲಯ-ಈಶ್ವರ
ದೇವಾ-ಲಯ-ಕಿಕ್ಕೇರಿ
ದೇವಾ-ಲ-ಯಕ್ಕೂ
ದೇವಾ-ಲ-ಯಕ್ಕೆ
ದೇವಾ-ಲಯ-ಗಳ
ದೇವಾ-ಲಯ-ಗ-ಳನ್ನು
ದೇವಾ-ಲಯ-ಗ-ಳನ್ನೂ
ದೇವಾ-ಲಯ-ಗ-ಳಲ್ಲಿ
ದೇವಾ-ಲಯ-ಗಳಲ್ಲಿದ್ದ
ದೇವಾ-ಲಯ-ಗಳಲ್ಲಿ-ರುವ
ದೇವಾ-ಲಯ-ಗ-ಳಾಗಿ
ದೇವಾ-ಲಯ-ಗ-ಳಾಗಿವೆ
ದೇವಾ-ಲಯ-ಗ-ಳಿಂದ
ದೇವಾ-ಲಯ-ಗ-ಳಿಂದಲ
ದೇವಾ-ಲಯ-ಗ-ಳಿಂದಲೂ
ದೇವಾ-ಲಯ-ಗಳಿಗೂ
ದೇವಾ-ಲಯ-ಗ-ಳಿಗೆ
ದೇವಾ-ಲಯ-ಗಳಿ-ಗೋಸ್ಕರ
ದೇವಾ-ಲಯ-ಗಳಿದ್ದವು
ದೇವಾ-ಲಯ-ಗಳಿದ್ದ-ವೆಂದು
ದೇವಾ-ಲಯ-ಗಳಿದ್ದು
ದೇವಾ-ಲಯ-ಗಳಿವೆ
ದೇವಾ-ಲಯ-ಗಳು
ದೇವಾ-ಲಯ-ಗಳು-ಒಂದು
ದೇವಾ-ಲಯ-ಗಳೂ
ದೇವಾ-ಲಯ-ಗ-ಳೆಂದು
ದೇವಾ-ಲಯ-ಗಳೆ-ರಡೂ
ದೇವಾ-ಲಯ-ಗಳೇ
ದೇವಾ-ಲ-ಯದ
ದೇವಾ-ಲಯ-ದಂತೇ
ದೇವಾ-ಲಯ-ದ-ಮುಂದೆ
ದೇವಾ-ಲಯ-ದಲಿ
ದೇವಾ-ಲಯ-ದಲ್ಲಿ
ದೇವಾ-ಲಯ-ದಲ್ಲಿದ್ದ
ದೇವಾ-ಲಯ-ದಲ್ಲಿಯೇ
ದೇವಾ-ಲಯ-ದಲ್ಲಿರು
ದೇವಾ-ಲಯ-ದಲ್ಲಿ-ರುವ
ದೇವಾ-ಲಯ-ದಲ್ಲಿವೆ
ದೇವಾ-ಲಯ-ದಲ್ಲೂ
ದೇವಾ-ಲಯ-ದಿಂದ
ದೇವಾ-ಲಯ-ದೊಳ-ಗಿ-ರುವ
ದೇವಾ-ಲಯ-ದೊಳಗೆ
ದೇವಾ-ಲಯ-ದೊಳಗೇ
ದೇವಾ-ಲ-ಯನ್ನು
ದೇವಾ-ಲ-ಯನ್ನೂ
ದೇವಾ-ಲಯ-ಯೋಗಾ
ದೇವಾ-ಲಯ-ವನ್ನು
ದೇವಾ-ಲಯ-ವನ್ನೂ
ದೇವಾ-ಲಯ-ವಾಗಲೀ
ದೇವಾ-ಲಯ-ವಾಗಿತ್ತೆಂದು
ದೇವಾ-ಲಯ-ವಾಗಿದೆ
ದೇವಾ-ಲಯ-ವಾ-ಗಿದ್ದು
ದೇವಾ-ಲಯ-ವಾಗಿರ
ದೇವಾ-ಲಯ-ವಾಗಿ-ರ-ಬ-ಹುದು
ದೇವಾ-ಲಯ-ವಾಗಿ-ರುವ
ದೇವಾ-ಲಯ-ವಾಗುತ್ತ-ದೆಂದು
ದೇವಾ-ಲಯ-ವಾದ
ದೇವಾ-ಲಯ-ವಿತ್ತು
ದೇವಾ-ಲಯ-ವಿದ
ದೇವಾ-ಲಯ-ವಿದೆ
ದೇವಾ-ಲಯ-ವಿದ್ದ
ದೇವಾ-ಲಯ-ವಿದ್ದಿತಂತೆ
ದೇವಾ-ಲಯ-ವಿದ್ದಿತು
ದೇವಾ-ಲಯ-ವಿದ್ದಿ-ತೆಂದು
ದೇವಾ-ಲಯ-ವಿದ್ದಿರ-ಬೇಕು
ದೇವಾ-ಲಯ-ವಿದ್ದು
ದೇವಾ-ಲಯ-ವಿ-ರ-ಬಹದು
ದೇವಾ-ಲಯ-ವಿರ-ಬ-ಹುದು
ದೇವಾ-ಲಯ-ವಿ-ರು-ವು-ದಿಲ್ಲ
ದೇವಾ-ಲ-ಯವು
ದೇವಾ-ಲಯ-ವು-ಹಲಗೆ-ಕಾರ-ನಾಥ
ದೇವಾ-ಲ-ಯವೂ
ದೇವಾ-ಲಯ-ವೆಂದರೆ
ದೇವಾ-ಲಯ-ವೆಂದು
ದೇವಾ-ಲಯ-ವೆಂಬ
ದೇವಾ-ಲ-ಯವೇ
ದೇವಾ-ಲ-ವನ್ನು
ದೇವಾ-ಲ-ವಾಗಿ-ರುವ
ದೇವಾ-ಲವೂ
ದೇವಾಲ್ಯಕ್ಕೆ
ದೇವಾಲ್ಯವ
ದೇವಾಲ್ಯವಂ
ದೇವಾಲ್ಯವನು
ದೇವಾಲ್ಯವ-ನೆತ್ತಿಸಿ
ದೇವಿ
ದೇವಿ-ಕೆರೆಯ
ದೇವಿಗೆ
ದೇವಿ-ಗೆರೆ
ದೇವಿ-ಗೆ-ರೆ-ಗಳ
ದೇವಿ-ಗೆ-ರೆೆ
ದೇವಿಯ
ದೇವಿ-ಯನ್ನು
ದೇವಿ-ಯರ
ದೇವಿ-ಯ-ರಿಗೆ
ದೇವಿ-ಯರು
ದೇವೀರಮ್ಮ
ದೇವೀರಮ್ಮಣ್ಣಿ
ದೇವೀರಮ್ಮನ
ದೇವೇಂದ್ರ
ದೇವೇ-ಗೌಡ
ದೇವೋಜ
ದೇಶ
ದೇಶ-ಕಾವ-ಲು-ಗಾರ-ರಿದ್ದರು
ದೇಶಕ್ಕೆ
ದೇಶ-ಗಳ
ದೇಶ-ಗ-ಳನ್ನು
ದೇಶ-ಗಳೊಡನೆ
ದೇಶದ
ದೇಶ-ದಲ್ಲಿ
ದೇಶ-ದಲ್ಲಿ-ರುವ
ದೇಶಪ್ರೇಮಿ
ದೇಶ-ರಾಜ್ಯ-ನಾಡು-ಮಂಡಲ
ದೇಶ-ವಳ್ಳಿ
ದೇಶ-ವಾಗಿತ್ತೆಂದು
ದೇಶ-ವೆಂದರೆ
ದೇಶಶ್ರೀ
ದೇಶಸ್ಥಂ
ದೇಶಸ್ಯ
ದೇಶ-ಹಳ್ಳಿ
ದೇಶಾಂತರ
ದೇಶಾಂತರಿ
ದೇಶಾಂತ-ರಿ-ಗ-ಳಿಗೆ
ದೇಶಾಂತ-ರಿ-ಗಳು
ದೇಶಾಂತ್ರ-ಮಠ-ವನ್ನು
ದೇಶಾಂತ್ರಿ
ದೇಶಾಂತ್ರಿ-ಮುದ್ರೆ-ಯನ್ನು
ದೇಶಾಖ್ಯೇ
ದೇಶಾಣಿ
ದೇಶಿ
ದೇಶಿ-ಕಾ-ಚಾರ್ಯರ
ದೇಶಿ-ಗ-ಳೆಂದು
ದೇಶಿ-ಮುಖ
ದೇಶಿಯ
ದೇಶಿ-ಯ-ಗಣ
ದೇಶಿ-ಯ-ಗಣದ
ದೇಶಿ-ಯಪ್ಪನ
ದೇಶಿ-ಯಪ್ಪನ-ಹಳ್ಳಿ
ದೇಶಿ-ಯ-ರನ್ನು
ದೇಶಿ-ಯ-ರಿಗೆ
ದೇಶಿ-ಯ-ರು-ಗಳು
ದೇಶೀ
ದೇಶೀ-ಕೇಂದ್ರ
ದೇಶೀಯ
ದೇಶೀ-ಯಕ್ಕೊಂಡ
ದೇಶೀ-ಯ-ಗಣ
ದೇಶೀ-ಯಾ-ಚಾರಿ
ದೇಶೀ-ವರ್ತಕರ
ದೇಶೀಸೀ
ದೇಶೇ
ದೇಶೇಶ್ವರ
ದೇಸಾ-ಭಾಗದ
ದೇಸಾಯಿ
ದೇಸಾಯಿ-ಯ-ವರು
ದೇಸಿ
ದೇಸಿಗ
ದೇಸಿ-ಗ-ಗಣ
ದೇಸಿ-ಗ-ಗಣದ
ದೇಸಿ-ಗ-ಣದ
ದೇಸಿ-ಗ-ರಲ್ಲೇ
ದೇಸಿ-ಗಳು
ದೇಸಿ-ಮಲೆ-ಯಾಳ
ದೇಸಿಯ
ದೇಸಿಯಂ
ದೇಸಿ-ಯಂಕ-ಕಾರ
ದೇಸಿ-ಯ-ಗಣದ
ದೇಸಿ-ಯಪ್ಪನ
ದೇಸಿ-ಯಪ್ಪನ-ಹಳ್ಳಿ
ದೇಸಿ-ಯರು
ದೇಸಿ-ಯಾ-ಭರಣ
ದೇಸಿ-ಯಾ-ಭರ-ಣ-ನೆಂಬ
ದೇಸಿಯಿಂ
ದೇಸಿ-ಯುಂವಿರ್ದ್ದು
ದೇಸೀ
ದೇಸೀ-ಗೌಡ
ದೇಸೀಯ
ದೇಸೀ-ಯ-ರಿಗೆ
ದೇಸೀ-ಯಾ-ಭರಣ
ದೇಹತ್ಯಾಗ
ದೇಹತ್ಯಾಗ-ಗ-ಳನ್ನು
ದೇಹ-ವನ್ನು
ದೈನಂದಿನ
ದೈವ
ದೈವ-ಕೃಪೆ-ಯಿಂದ
ದೈವ-ಗಳ
ದೈವತ್ವಕ್ಕೆ
ದೈವ-ದತ್ತ-ವಾಗಿ
ದೈವ-ದತ್ತ-ವಾದ
ದೈವ-ಭಕ್ತ-ರಾಗಿದ್ದರು
ದೊಂದೆ-ಮಾದಿ-ಹಳ್ಳಿ
ದೊಡಿ-ಯಮ್ಮನ
ದೊಡ್ಡ
ದೊಡ್ಡ-ಅಂಣಾಜೈಯ್ಯ-ನ-ವರು-ಪೆರಿ
ದೊಡ್ಡ-ಅಬ್ಬಾ-ಗಿಲು
ದೊಡ್ಡ-ಅರ-ಸಿ-ನ-ಕೆರೆ
ದೊಡ್ಡ-ಅರ-ಸಿ-ನ-ಕೆರೆಯ
ದೊಡ್ಡ-ಉಳು-ವರ್ತಿ
ದೊಡ್ಡ-ಕಟ್ಟ
ದೊಡ್ಡ-ಕಿ-ರಂಗೂರಿನ
ದೊಡ್ಡ-ಕೃಷ್ಣ-ರಾಜ
ದೊಡ್ಡ-ಕೃಷ್ಣ-ರಾಯರು
ದೊಡ್ಡ-ಕೆರೆ-ಗಳು
ದೊಡ್ಡ-ಕೆರೆ-ಯಾಗಿ-ರ-ಬ-ಹುದು
ದೊಡ್ಡಕ್ಯಾ-ತನ-ಹಳ್ಳಿ-ಯಲ್ಲೂ
ದೊಡ್ಡ-ಗದ್ದ-ವಳ್ಳಿ
ದೊಡ್ಡ-ಗದ್ದ-ವಳ್ಳಿಯ
ದೊಡ್ಡ-ಗರು-ಡ-ನ-ಹಳ್ಳಿ
ದೊಡ್ಡ-ಗಾ-ಡಿಗ-ನ-ಹಳ್ಳಿ
ದೊಡ್ಡ-ಗಾ-ಡಿಗ-ನ-ಹಳ್ಳಿಯ
ದೊಡ್ಡ-ಗಾ-ಡಿಗ-ನ-ಹಳ್ಳಿ-ಯಲ್ಲಿ
ದೊಡ್ಡ-ಜಟಕ
ದೊಡ್ಡ-ಜಟಕದ
ದೊಡ್ಡ-ಜಟಕಾ
ದೊಡ್ಡ-ತಮ್ಮಣ್ಣ
ದೊಡ್ಡ-ದಾ-ಗಿದ್ದು
ದೊಡ್ಡ-ದಾದ
ದೊಡ್ಡ-ದಾ-ಯಿತು
ದೊಡ್ಡದು
ದೊಡ್ಡ-ದೇವಯ್ಯನ
ದೊಡ್ಡ-ದೇವಯ್ಯ-ನ-ವರು
ದೊಡ್ಡ-ದೇವ-ರಾಜ
ದೊಡ್ಡ-ದೇವ-ರಾಜನ
ದೊಡ್ಡ-ದೇವ-ರಾಜನು
ದೊಡ್ಡ-ದೇವ-ರಾಜನೂ
ದೊಡ್ಡ-ದೇವ-ರಾಜ-ರಿಗೆ
ದೊಡ್ಡ-ದೇವ-ರಾಯರು
ದೊಡ್ಡ-ದೊಂದು
ದೊಡ್ಡ-ದೊಡ್ಡ
ದೊಡ್ಡದ್ಯಾಮ-ಗೌಡ-ನಿಗೆ
ದೊಡ್ಡ-ನಂಜಮ್ಮನ
ದೊಡ್ಡಪ್ಪ
ದೊಡ್ಡಪ್ಪಂದಿರು
ದೊಡ್ಡಪ್ಪ-ನೊಡನೆ
ದೊಡ್ಡಪ್ರ-ಮಾಣ-ದಲ್ಲಿ
ದೊಡ್ಡ-ಬಳ್ಳಾ-ಪುರ
ದೊಡ್ಡ-ಬೆಟ್ಟ
ದೊಡ್ಡ-ಬೆಟ್ಟದ
ದೊಡ್ಡ-ಮಸೀದಿ
ದೊಡ್ಡ-ಮುಲು-ಗೋಡು
ದೊಡ್ಡಮ್ಮ
ದೊಡ್ಡ-ಯಗಟಿ
ದೊಡ್ಡಯ್ಯ
ದೊಡ್ಡಯ್ಯನ
ದೊಡ್ಡಯ್ಯ-ನ-ವ-ರಿಗೆ
ದೊಡ್ಡಯ್ಯ-ನ-ಹಳ್ಳಿ
ದೊಡ್ಡಯ್ಯನು
ದೊಡ್ಡ-ರ-ಸಿನ-ಕೆರೆ
ದೊಡ್ಡ-ವಡ್ಡ-ಅರ-ಗುಡಿ
ದೊಡ್ಡ-ವನಯ
ದೊಡ್ಡ-ಹುಂಡಿ
ದೊಡ್ಡ-ಹೆಬ್ಬಾ-ಗಿಲು
ದೊಡ್ಡಾದಣ್ಣನ
ದೊಡ್ಡಾ-ಬಾಲ-ವು-ದೊಡ್ಡ
ದೊಡ್ಡಿ
ದೊಡ್ಡಿ-ಗಟ್ಟ
ದೊಡ್ಡಿ-ಘಟ್ಟ
ದೊಡ್ಡಿನ
ದೊಡ್ಡಿ-ನ-ಕೆರೆಯ
ದೊಡ್ಡಿ-ಯಮ್ಮ
ದೊಡ್ಡಿ-ಯಮ್ಮನ
ದೊಡ್ಡಿಯೇ
ದೊಡ್ಡೇರಿ
ದೊಡ್ಡೇರಿ-ಗೌಡರ
ದೊಡ್ಡೇ-ರಿಯ
ದೊಡ್ಡೈಯ
ದೊರ-ಕಿತ್ತು
ದೊರಕಿತ್ತೆಂದು
ದೊರಕಿದೆ
ದೊರಕಿದ್ದು
ದೊರ-ಕಿ-ರುವ
ದೊರ-ಕಿರು-ವು-ದಿಲ್ಲ
ದೊರ-ಕಿರು-ವುದು
ದೊರಕಿಲ್ಲ
ದೊರಕಿಲ್ಲ-ದಿ-ರುವುದು
ದೊರಕಿವೆ
ದೊರಕುತ್ತದೆ
ದೊರಕುತ್ತವೆ
ದೊರಕುವ
ದೊರ-ಕೊಳ್ಗೆ
ದೊರ-ತಿ-ರುವ
ದೊರಭಕ್ಕೆರೆ
ದೊರೆ
ದೊರೆ-ಗಳ
ದೊರೆ-ಗ-ಳಲ್ಲಿ
ದೊರೆ-ಗ-ಳಾಗಿ-ರ-ಬ-ಹುದು
ದೊರೆ-ಗ-ಳಾದ
ದೊರೆ-ಗಳು
ದೊರೆತ
ದೊರೆ-ತ-ನದಿ-ರ-ವನ್ನು
ದೊರೆ-ತವು
ದೊರೆ-ತಿದ್ದು
ದೊರೆ-ತಿ-ರುವ
ದೊರೆ-ತಿಲ್ಲ
ದೊರೆ-ತಿವೆ
ದೊರೆತು
ದೊರೆ-ಮಿ-ಗಿಲ್
ದೊರೆ-ಯ-ತೊಡಗಿತು
ದೊರೆ-ಯ-ದಿ-ರುವುದು
ದೊರೆ-ಯದೇ
ದೊರೆ-ಯ-ಲಾರಂಭಿ-ಸುತ್ತವೆ
ದೊರೆ-ಯ-ಲಿಲ್ಲ
ದೊರೆ-ಯವ
ದೊರೆ-ಯಾಗಿ-ರ-ಬ-ಹುದು
ದೊರೆ-ಯಾದುವು
ದೊರೆ-ಯಿತು
ದೊರೆ-ಯಿ-ತೆಂದು
ದೊರೆ-ಯಿ-ತೆಂದೂ
ದೊರೆ-ಯಿಲ್ಲ-ರೆಂಬ
ದೊರೆಯು
ದೊರೆ-ಯುತ್ತದೆ
ದೊರೆ-ಯುತ್ತವೆ
ದೊರೆ-ಯುತ್ತಿತ್ತು
ದೊರೆ-ಯುತ್ತಿತ್ತೆಂದು
ದೊರೆ-ಯುವ
ದೊರೆ-ಯು-ವು-ದಿಲ್ಲ
ದೊರೆ-ಯು-ವುದು
ದೊರೆ-ವರು
ದೋಣಿ-ಗ-ಳಲ್ಲಿ
ದೋರ
ದೋರನು
ದೋರ-ಸ-ಮುದ್ರ
ದೋರ-ಸ-ಮುದ್ರಕ್ಕೆ
ದೋರ-ಸ-ಮುದ್ರದ
ದೋರ-ಸ-ಮುದ್ರ-ದ-ದಲ್ಲಿದ್ದನು
ದೋರ-ಸ-ಮುದ್ರ-ದಲು
ದೋರ-ಸ-ಮುದ್ರ-ದಲ್ಲಿ
ದೋರ-ಸ-ಮುದ್ರ-ದಲ್ಲಿದ್ದ
ದೋರ-ಸ-ಮುದ್ರ-ದಲ್ಲಿದ್ದಾಗ
ದೋರ-ಸ-ಮುದ್ರ-ದಲ್ಲೇ
ದೋರ-ಸ-ಮುದ್ರ-ದಿಂದ
ದೋರ-ಸ-ಮುದ್ರ-ವನ್ನು
ದೋರ-ಸ-ಮುದ್ರಾಖ್ಯಾಂ
ದೋಷಕೆ
ದೋಷ-ದಲು
ದೋಸ್ತಂಭ-ದೊಳು
ದೌರ್ಜನ್ಯವೇ
ದೌರ್ಬಲ್ಯ-ಗಳ
ದ್ದನು
ದ್ದರೆಂದು
ದ್ದವರು
ದ್ದವು
ದ್ದಷ್ಟೇ
ದ್ದಾನೆ
ದ್ದುದರ
ದ್ದೋಷಾಣಾಂ
ದ್ಧರಣಂ
ದ್ಯಾವಣ್ಣನು
ದ್ಯಾವಣ್ಣಹೆಮ್ಮಾಡಿ-ಯಣ್ಣ
ದ್ಯಾವ-ರ-ಹಳ್ಳಿ
ದ್ಯಾವ-ರ-ಹಳ್ಳಿ-ಯಲ್ಲೂ
ದ್ಯಾವಾ
ದ್ಯಾವಾ-ಜಮ್ಮಣ್ಣಿ
ದ್ಯೋತಕ-ವಾಗಿದೆ
ದ್ರಮಿಳ
ದ್ರಮಿಳ-ಸಂಘದ
ದ್ರಮ್ಮ
ದ್ರಮ್ಮ-ವನ್ನು
ದ್ರವ್ಯ-ವನ್ನು
ದ್ರವ್ಯ-ವೆಂದು
ದ್ರಾವಿಡ
ದ್ರಾವಿಡ-ವೇದದ
ದ್ರಾವಿಡ-ಶೈಲಿ-ಯಲ್ಲಿದೆ
ದ್ರಾವಿಡಾನ್ವಯದ
ದ್ರಾವಿಡಾಮ್ನಾಯ
ದ್ರಾಹ್ಯಾ-ಯಣ
ದ್ರಾಹ್ಯಾ-ಯಣ-ಸೂತ್ರದ
ದ್ರುವನು
ದ್ರೋಹಕೆ
ದ್ರೋಹ-ಘ-ರಟ್ಟ
ದ್ರೋಹ-ಘ-ರಟ್ಟಂ
ದ್ರೋಹ-ಘ-ರಟ್ಟ-ನೆಂಬ
ದ್ರೋಹ-ಘ-ರಟ್ಟಾ-ಚಾರಿ
ದ್ರೋಹನಂ
ದ್ರೋಹಿ
ದ್ವಾದ-ಶ-ವರ್ಷ
ದ್ವಾದ-ಶಾವ-ತಾರ
ದ್ವಾಪರ-ಯುಗದ
ದ್ವಾರ
ದ್ವಾರ-ಕೆ-ಯಿಂದ
ದ್ವಾರದ
ದ್ವಾರ-ದಲ್ಲಿ-ರುವ
ದ್ವಾರ-ಪಕ್ಷದ
ದ್ವಾರ-ಪಾಲ-ಕರ
ದ್ವಾರ-ಪಾಲ-ಕರು
ದ್ವಾರ-ಪಾಲ-ಕರು-ಗಳ
ದ್ವಾರ-ಬಂಧಧ
ದ್ವಾರ-ಮಂಟಪ
ದ್ವಾರ-ವನ್ನು
ದ್ವಾರ-ವುಳ್ಳ
ದ್ವಾರ-ಸ-ಮುದ್ರದ
ದ್ವಾರ-ಸ-ಮುದ್ರ-ದಿಂದ
ದ್ವಾರಸ್ಥಿತ
ದ್ವಾರಾ-ವತಿ
ದ್ವಾರಾ-ವತಿ-ನಾಥನ
ದ್ವಾವೇತಾವಥ
ದ್ವಿಕಣ್ಡುಕ
ದ್ವಿಕಣ್ಡುಗ
ದ್ವಿಕೂಟ
ದ್ವಿಕೂಟಾಚಲ
ದ್ವಿಕೂಟಾಚಲ-ವಾ-ಗಿದ್ದು
ದ್ವಿಕೂಟಾಚಲ-ವಾದ
ದ್ವಿಗುಣ-ಮತ್ರಿ-ಗುಣಂಚತುರ್ಗಣಂ
ದ್ವಿಗುಣೀಕೃತ
ದ್ವಿಜ-ಗುರು-ಬುಧ-ಗೋತ್ರಾದಿ
ದ್ವಿಜರು
ದ್ವಿಜ-ವಂಶ-ತಿಲಕನೂ
ದ್ವಿಜಾತೀ-ನಾಮನ್ನ
ದ್ವಿತೀಯ
ದ್ವಿತೀಯ-ಲಕ್ಷ್ಮೀ
ದ್ವಿತೀಯ-ವಿಭವಂ
ದ್ವಿರುಕ್ತಿ-ಗಾಗಿ
ದ್ವೀಪ-ವನ್ನು
ದ್ವೀಪ-ವಾಗಿ-ರು-ವು-ದನ್ನು
ದ್ವೀಪ-ವಿದ್ದು
ದ್ವೇಷ-ವನ್ನು
ದ್ವೈತ
ದ್ವೈತ-ಪಂಥವು
ದ್ವೈತ-ಮತ
ದ್ವೈತ-ಮತ-ದಲ್ಲಿ
ದ್ವೈತಾದ್ವೈತ-ಮತ
ಧಂನೋಜಿ
ಧಂರ್ಮ್ಮದ
ಧಕ್ಕೆ-ಯಾದ
ಧಣುಗೂ-ರನ್ನು
ಧನ
ಧನಂ
ಧನಂಜಯ-ಪುರ-ವನ್ನಾಗಿ
ಧನಂಜಯ-ರಾಯ
ಧನಂಜಯ-ರಾಯ-ವೊಡೆ-ಯನು
ಧನ-ಗೂ-ರನ್ನು
ಧನ-ಗೂ-ರಿಗೆ
ಧನ-ಗೂರಿನ
ಧನ-ಗೂರಿ-ನಲ್ಲಿ
ಧನ-ಗೂರು
ಧನ-ಗೂರುಸ್ಥಳದ
ಧನ-ಮನಪ್ರಾಣಂಗಳೊಳು
ಧನ-ವೆಲ್ಲ
ಧನ-ಸಹಾಯ-ವನ್ನು
ಧನುಗೂ-ರನ್ನು
ಧನುಗೂರಿ-ನಲ್ಲಿ
ಧನು-ಗೂರು
ಧನುರು
ಧನುರ್
ಧನುರ್ನ್ನಾಮಾಂಕಿತ
ಧನುರ್ಮಾಸ
ಧನುರ್ಮಾಸದ
ಧನುರ್ವಿದ್ಯಾ-ಪರಿಣ-ತರುಂ
ಧನ್ನೋಜಿಯು
ಧನ್ಯ
ಧನ್ಯಂ
ಧನ್ಯತೆ
ಧಮಕ್ಕೆ
ಧಮ್ಮ-ಗವುಂಡನ
ಧರ-ಣಿ-ದೇವ-ತಾ-ರುದ್ರನುಂ
ಧರ-ಣೀತೇಃ
ಧರ-ಣೀ-ದೇವ
ಧರ-ಣೀ-ಪಾಳಕ-ನಪ್ಪ
ಧರ-ಣೀ-ವರಾಹ
ಧರಾ-ನಾಥಂಗೆ
ಧರಾಮ-ರೋತ್ತಂಸ-ನಾಗಿದ್ದ-ನೆಂದು
ಧರಾ-ರಾಜ್ಯ-ವಾಳುತ್ತಿದ್ದಾಗ
ಧರಿತ್ರಿ-ಯೊಳು-ಪೆ-ಸರ್ವ್ವಡೆದಂ
ಧರಿ-ಸದೇ
ಧರಿಸಿ
ಧರಿ-ಸಿದ
ಧರಿಸಿ-ದಂತೆ
ಧರಿಸಿ-ದನು
ಧರಿಸಿ-ದ-ನೆಂದು
ಧರಿ-ಸಿದ್ದ
ಧರಿಸಿದ್ದನು
ಧರಿಸಿದ್ದ-ನೆಂದು
ಧರಿಸಿದ್ದನ್ನು
ಧರಿಸಿದ್ದ-ರಿಂದ
ಧರಿಸಿದ್ದ-ರೆಂದೂ
ಧರಿಸಿದ್ದಾನೆ
ಧರಿಸಿ-ರ-ಬ-ಹುದು
ಧರಿಸಿ-ರುವ
ಧರಿಸುತ್ತಿದ್ದ
ಧರಿ-ಸುವ
ಧರಿ-ಸುವುದ-ರಿಂದ
ಧರೆ
ಧರೆ-ಕೂರ್ತ್ತುಕೀರ್ತ್ತಿಕುಂ
ಧರೆ-ಗೆಲ್ಲಂ
ಧರೆ-ತನ್ನಂ
ಧರೆ-ಪೊ-ಗಳೆ
ಧರೆ-ಮುಖ್ಯಾನ್ವಯ-ನಾದ
ಧರೆ-ಯನ್ನು
ಧರೆ-ಯೊಳ್
ಧರೊ-ಯೊಳ್
ಧರ್ಮ
ಧರ್ಮಂಗ-ಳನ್ನು
ಧರ್ಮಂಗ-ಳಿಗೆ
ಧರ್ಮಃ
ಧರ್ಮ-ಅ-ನಂತ
ಧರ್ಮ-ಕರ್ತ-ನಾದ
ಧರ್ಮ-ಕಾರ್ಯಕ್ಕೆ
ಧರ್ಮ-ಕಾರ್ಯ-ಗ-ಳನ್ನು
ಧರ್ಮ-ಕಾರ್ಯ-ಗ-ಳಲ್ಲಿ
ಧರ್ಮ-ಕಾರ್ಯ-ಗ-ಳಿಗೆ
ಧರ್ಮ-ಕಾರ್ಯ-ಗಳು
ಧರ್ಮ-ಕಾರ್ಯ-ವನ್ನು
ಧರ್ಮಕೆ
ಧರ್ಮ-ಕೆರೆ
ಧರ್ಮ-ಕೋ-ವಿದಃ
ಧರ್ಮಕ್ಕೆ
ಧರ್ಮ-ಗ-ಳಂತೆ
ಧರ್ಮ-ಗ-ಳನ್ನು
ಧರ್ಮ-ಗ-ಳನ್ನೂ
ಧರ್ಮ-ಗ-ಳಲ್ಲಿ
ಧರ್ಮ-ಗ-ಳಾದ
ಧರ್ಮ-ಗಳು
ಧರ್ಮ-ಗುರು-ಗಳೂ
ಧರ್ಮ-ಛತ್ರವೂ
ಧರ್ಮದ
ಧರ್ಮ-ದ-ತೋಪು
ಧರ್ಮ-ದ-ಮೇರು
ಧರ್ಮ-ದಲ್ಲಿ
ಧರ್ಮ-ದ-ವರ
ಧರ್ಮ-ದ-ವರು
ಧರ್ಮ-ದಿಂದ
ಧರ್ಮ-ದೊಡನೆ
ಧರ್ಮ-ದೊಳು
ಧರ್ಮ-ಪತ್ನಿ
ಧರ್ಮ-ಪತ್ನಿ-ಯರ
ಧರ್ಮ-ಪತ್ನಿ-ಯ-ರಾದ
ಧರ್ಮ-ಪತ್ನಿ-ಯರು
ಧರ್ಮ-ಪತ್ನಿ-ಯಾದ
ಧರ್ಮ-ಪ-ರಾಯ-ಣ-ರಾದ
ಧರ್ಮ-ಪುರಿ
ಧರ್ಮ-ಪುರೋಭಿ-ವೃದ್ಧಿ-ಯಾಗಿ
ಧರ್ಮಪ್ರ-ಯುಕ್ತೆ
ಧರ್ಮಪ್ರ-ವರ್ತಕನೂ
ಧರ್ಮಪ್ರ-ವರ್ತನೆ
ಧರ್ಮಪ್ರಸಂಗದ
ಧರ್ಮಪ್ರಸಂಗ-ವನ್ನು
ಧರ್ಮ-ಬುದ್ಧಿ
ಧರ್ಮ-ಬೊಜ್ಜ
ಧರ್ಮ-ಬೊಜ್ಜ-ವಿಷ್ಣು-ವರ್ಧನ
ಧರ್ಮ-ಬೋಧಕ-ರಾದ
ಧರ್ಮ-ಭೂಮಿ-ಯಾ-ಗಿದ್ದ
ಧರ್ಮಮಂ
ಧರ್ಮ-ಮನೋ
ಧರ್ಮ-ಮಹಾ-ಧಿ-ರಾಜ
ಧರ್ಮ-ಮಹಾ-ಧಿ-ರಾಜ-ರೆಂದು
ಧರ್ಮ-ಮಹಾ-ರಾಜಾಧಿ
ಧರ್ಮ-ಮಹಾ-ರಾಜಾಧಿ-ರಾಜ
ಧರ್ಮ-ಮಾರ್ಗ-ದತ್ತ
ಧರ್ಮ-ಯಾತ್ರೆ-ಯಲ್ಲಿ
ಧರ್ಮ-ರಕ್ಷಣೆ
ಧರ್ಮ-ರತ್ನ-ವೆಂದೂ
ಧರ್ಮ-ರಾಜ
ಧರ್ಮ-ರಾಸಿ
ಧರ್ಮಲ
ಧರ್ಮ-ಲಿಂಗ-ಭಟ್ಟ
ಧರ್ಮವ
ಧರ್ಮವಂ
ಧರ್ಮ-ವನು
ಧರ್ಮ-ವನ್ನು
ಧರ್ಮ-ವನ್ನೇ
ಧರ್ಮ-ವಾಗ-ಬೇಕೆಂದು
ಧರ್ಮ-ವಾಗ-ಲೆಂದು
ಧರ್ಮ-ವಾಗಿ
ಧರ್ಮ-ವಾ-ಗಿತ್ತು
ಧರ್ಮವು
ಧರ್ಮ-ವೆನಿಸಿ
ಧರ್ಮವೇ
ಧರ್ಮ-ಶಾಲಿ-ಗ-ಳಾದ
ಧರ್ಮ-ಶಾ-ಸನ-ಗ-ಳಾಗಿದ್ದು
ಧರ್ಮ-ಶಾ-ಸನ-ವನ್ನು
ಧರ್ಮ-ಶಾಸ್ತ್ರಕ್ಕನು-ಗುಣ-ವಾಗಿ
ಧರ್ಮಶ್ಚ
ಧರ್ಮ-ಸಂವಾದ-ದಲ್ಲಿ
ಧರ್ಮ-ಸತ್ರ-ಗ-ಳನ್ನು
ಧರ್ಮ-ಸಾ-ಧನ-ವಾಗಿ
ಧರ್ಮಾಗ್ರ-ಹಾರ-ವಾಗಿ
ಧರ್ಮಾಧಿ-ಕಾರಿ-ಯೆನಿಸಿ-ಕೊಂಡ-ವನು
ಧರ್ಮಾನುಯಾಯಿ-ಗ-ಳಾದ
ಧರ್ಮಾ-ಪುರ
ಧರ್ಮಾ-ಪುರ-ವನ್ನು
ಧರ್ಮಾ-ಪುರ-ವೆಂಬ
ಧರ್ಮಾಮೃತ-ದಲ್ಲಿ
ಧರ್ಮಾರ್ಥ-ವಾಗಿ
ಧರ್ಮಾ-ವರ
ಧರ್ಮೇಣ
ಧರ್ಮೇಶ್ವರ-ದೇವರೇ
ಧರ್ಮ್ಮಪ್ರತಿ-ಪಾಳಕ-ರು-ಮಪ್ಪ
ಧರ್ಮ್ಮ-ಬುದ್ಧಿಗಂ
ಧರ್ಮ್ಮಮಂ
ಧರ್ಮ್ಮ-ಮ-ಮನೋ
ಧರ್ಮ್ಮ-ವನು
ಧರ್ಮ್ಮ-ವೆ-ಗಳ
ಧರ್ಮ್ಮ-ಶೀಲಾಕ್ಕಮಾ-ಗರ್ಭ-ಶುಕ್ತಿ-ಮುಕ್ತಾ-ಫಲಾತ್ಮನಃ
ಧರ್ಮ್ಮೋ-ದಯ
ಧವಳಂಕ-ಭೀಮ
ಧವಳಾಂಕ-ಭೀಮ
ಧವಳಾ-ರದ
ಧಾತ್ರಿಯ
ಧಾನ್ಯ
ಧಾನ್ಯ-ಗ-ಳನ್ನು
ಧಾನ್ಯ-ಗಳು-ಎಣ್ಣೆ
ಧಾನ್ಯದ
ಧಾನ್ಯ-ವನ್ನು
ಧಾನ್ಯ-ಸಂಗ್ರಹಾ-ಲಯ-ಗಳು
ಧಾರಂ
ಧಾರಣ
ಧಾರ-ಪೂರ್ವ-ಕ-ವಾಗಿ
ಧಾರ-ವಾಡದ
ಧಾರಾ
ಧಾರಾ-ಧತ್ತ-ವಾಗಿ
ಧಾರಾ-ನ-ಗರ-ವನ್ನು
ಧಾರಾ-ಪುರ-ಗ-ಳನ್ನು
ಧಾರಾ-ಪುರ-ದಿಂದ
ಧಾರಾ-ಪೂರ್ಬ್ಬಕಂ
ಧಾರಾ-ಪೂರ್ವಕ
ಧಾರಾ-ಪೂರ್ವಕಂ
ಧಾರಾ-ಪೂರ್ವ-ಕ-ವಾಗಿ
ಧಾರಿ-ಗ-ಳಾದ
ಧಾರಿ-ಣಿಯು
ಧಾರೆ
ಧಾರೆ-ನೆಱದು
ಧಾರೆಯಂ
ಧಾರೆ-ಯನಾತ್ಮ
ಧಾರೆ-ಯ-ನೆರೆದು
ಧಾರೆ-ಯನೆರೆಸಿ
ಧಾರೆ-ಯೆರೆದು
ಧಾರೆ-ಯೆರೆದು-ಕೊಡುತ್ತಾರೆ
ಧಾರೆ-ಯೆರೆಸಿ
ಧಾರ್ಮಿಕ
ಧಾರ್ಮಿಕ-ವಾಗಿ
ಧಾರ್ಮಿಕಸ್ವ-ರೂಪದ
ಧಾರ್ಮಿಕಾಯ
ಧಾರ್ಮಿಸ್ಥಳ-ವಾದ
ಧಾವಿ-ಸಿದ
ಧೀಮಂತ
ಧೀಮಾನ್
ಧೀಮಾನ್ವೃತಿಮೇ-ಕಾಮಿಹಾಶ್ನುತೇ
ಧೀರ
ಧೀರರು
ಧೀರ-ಸೇನಾನಿ
ಧುರಂಧರಂಮ-ಮಾತ್ಯ
ಧುರದ
ಧುರೀಣ
ಧುರೀ-ಣಸ್ಯ
ಧುರ್ಮ್ಮಣ್ಣ
ಧೂಪ
ಧೂಪ-ದೀಪ
ಧೂಪ-ದೀಪ-ನೈ-ವೇದ್ಯಕ್ಕೆ
ಧೂಳೀಪಟ-ಮಾಡಿ
ಧೂಸರಾತ್ಮನಾಂ
ಧೂಸಱಿಂ
ಧೃತಪೀಡಿತ
ಧೃತ-ಸತ್ಯ-ವಾಕ್ಯಂ
ಧೃತಿಷೇಣ
ಧೇಯಂಗಗ್ಗದ
ಧೈರ್ಯ
ಧೈರ್ಯ-ಸುರ-ಗಾತ್ರ
ಧೋರ
ಧ್ಯಾನ
ಧ್ಯಾನ-ಧಾರಣ
ಧ್ಯಾನಾಸಿಃ
ಧ್ಯೇಯ
ಧ್ರುವ
ಧ್ರುವ-ಉಂಡಿಗೆಯ
ಧ್ರುವ-ಜಲ-ವಹಂತಾಗಿ
ಧ್ರುವನ
ಧ್ರುವನು
ಧ್ರುವ-ವುಂಡಿಗೆ-ಯನ್ನು
ಧ್ವಂಸ
ಧ್ವಂಸ-ಮಾಡಿ
ಧ್ವಂಸ-ಮಾಡಿ-ದನು
ಧ್ವಂಸ-ವಾಗಿವೆ
ಧ್ವಜ-ವನ್ನು
ಧ್ವಜಸ್ಥಂಬದ
ಧ್ವಜಸ್ಥಂಭ-ವನ್ನು
ಧ್ವಜಿನೀ-ಪತಿ
ಧ್ವಜಿನೀ-ಪತಿ-ಯಾದ
ಧ್ವನಿಯೂ
ಧ್ವಾಮಂ
ನ
ನಂಗಲಿ
ನಂಗ-ಲಿಯ
ನಂಗಲಿ-ಯನ್ನು
ನಂಗಿಲಿ
ನಂಗೈ
ನಂಗೈ-ಯಾಂಡಾಳ್
ನಂಗೈ-ಯಾರ್
ನಂಜನ-ಗೂಡಿನ
ನಂಜನ-ಗೂಡು
ನಂಜಪ್ಪ
ನಂಜಪ್ಪ-ದೇವರ
ನಂಜಮ್ಮ
ನಂಜಮ್ಮನು
ನಂಜಯ
ನಂಜಯ-ದೇವ-ರಿಗೆ
ನಂಜಯ್ಯನು
ನಂಜ-ರಾಜ
ನಂಜ-ರಾಜನ
ನಂಜ-ರಾಜನು
ನಂಜ-ರಾಜ-ನೆಂದೇ
ನಂಜ-ರಾಜನೇ
ನಂಜ-ರಾಜ-ಬಹ-ದೂರ್ವರೆಗೆ
ನಂಜ-ರಾ-ಜಯ್ಯ
ನಂಜ-ರಾಜಯ್ಯನ
ನಂಜ-ರಾಜಯ್ಯನು
ನಂಜ-ರಾಜ-ವೊಡೆ-ಯರ
ನಂಜ-ರಾಜ-ಸ-ಮುದ್ರ
ನಂಜ-ರಾಜ-ಸ-ಮುದ್ರ-ವಾದ
ನಂಜ-ರಾಜ-ಸ-ಮುದ್ರ-ವೆಂಬ
ನಂಜ-ರಾಜೈಯ್ಯ-ನ-ವರ
ನಂಜ-ರಾಜೊಡೆಯನ
ನಂಜ-ರಾಜೊಡೆಯರ
ನಂಜ-ರಾಯ
ನಂಜ-ರಾಯಜ
ನಂಜ-ರಾಯನು
ನಂಜ-ರಾಯ-ಪಟ್ಟ-ಣದ
ನಂಜ-ರಾ-ಯೊಡೆ-ಯನ
ನಂಜ-ರಾಯ್ಯನ
ನಂಜ-ರಾಯ್ಯ-ನನ್ನು
ನಂಜ-ವೊಡೇರಿಗೆ
ನಂಜಿನ
ನಂಜೀ-ನಾಥ
ನಂಜೀ-ನಾಥ-ನಿಗೆ
ನಂಜುಂಡ
ನಂಜುಂಡ-ಗಳು
ನಂಜುಂಡ-ದೇವ-ನೆಂಬ
ನಂಜುಂಡ-ಭಟ್ಟರ
ನಂಜುಂಡಯ್ಯ-ನ-ವರು
ನಂಜುಂಡಸ್ವಾಮಿ
ನಂಜುಂಡಾರಾಧ್ಯನು
ನಂಜುಂಡೇಶ್ವರ
ನಂಜುಂಡೇಶ್ವರನ
ನಂಜುಂಡೇಶ್ವರ-ನಿಗೆ
ನಂಜುಡೇಶ್ವರ
ನಂಜೆಯ
ನಂಜೇ-ಗವುಡ
ನಂಜೇ-ಹೆಬ್ಬಾ-ರುವ-ನಿಗೆ
ನಂಟ-ರಂಗ-ರ-ಕನುಂ
ನಂತ
ನಂತರ
ನಂತ-ರದ
ನಂತ-ರ-ದಲ್ಲಿ
ನಂತ-ರ-ವಷ್ಟೇ
ನಂತ-ರವೂ
ನಂತ-ರವೇ
ನಂದ-ಗಿರಿ
ನಂದ-ಗಿರಿ-ನಾಥ
ನಂದನ
ನಂದನಃ
ನಂದನ-ವನ
ನಂದನ-ವನ-ಗ-ಳನ್ನು
ನಂದನ-ವನ-ದಲ್ಲಿ-ರುವ
ನಂದನ-ವನ್ನು
ನಂದಾ-ದೀಪ
ನಂದಾ-ದೀ-ಪಕ್ಕೆ
ನಂದಾ-ದೀಪ್ತಿಗೆ
ನಂದಾ-ದೀ-ವಿಗೆ
ನಂದಾ-ದೀ-ವಿಗೆಗೆ
ನಂದಾ-ದೀ-ವಿಗೆ-ಗೋಸುಗಂ
ನಂದಾ-ದೀ-ವಿಗೆಯು
ನಂದಾ-ವನ
ನಂದಾವೆಳ-ಕಿಗೆ
ನಂದಿ
ನಂದಿ-ಕೋಟೂರು
ನಂದಿಧ್ವಜ
ನಂದಿ-ನಾಥ
ನಂದಿನ್ತ-ವರಿ-ವರ-ಳವೆ
ನಂದಿ-ಮಠ
ನಂದಿ-ಮಠರು
ನಂದಿಯ
ನಂದಿ-ಯನ್ನು
ನಂದಿ-ಯಾಲದ
ನಂದಿ-ಯೊಂದಿದ್ದು
ನಂದಿ-ವರ್ಮನ
ನಂದಿ-ವರ್ಮ-ನಿಂದ
ನಂದಿ-ವಾ-ಹನೋತ್ಸವ
ನಂದಿ-ಸಂಘ
ನಂದೀಧ್ವಜ-ವನ್ನು
ನಂದೀ-ನಾಥ
ನಂದ್ದಿ-ನಾಥ
ನಂದ್ಯಾಲ
ನಂದ್ಯಾಲದ
ನಂನಿಯ-ಮೇರು
ನಂನಿಯ-ಮೇರು-ವನ್ನು
ನಂಬ-ಲಾಗಿದೆ
ನಂಬಲು
ನಂಬಿ
ನಂಬಿಕೆ
ನಂಬಿ-ಕೆಗೆ
ನಂಬಿ-ಕೆ-ಯಿಂದ
ನಂಬಿ-ಗಳು
ನಂಬಿದ
ನಂಬಿ-ನಾಯ-ಕ-ನ-ಹಳ್ಳಿ
ನಂಬಿ-ನಾಯ-ಕ-ನ-ಹಳ್ಳಿಯ
ನಂಬಿ-ನಾ-ರಾಯಣ
ನಂಬಿ-ನಾ-ರಾಯ-ಣ-ನೆಂಬ
ನಂಬಿ-ಪಿಳ್ಳೆ
ನಂಬಿ-ಪಿಳ್ಳೆಯ
ನಂಬಿ-ಪಿಳ್ಳೆಯೂ
ನಂಬಿಯ
ನಂಬಿ-ಯರ
ನಂಬಿ-ಯರು
ನಂಬಿ-ಯ-ರು-ಗ-ಳಾದ
ನಂಬಿ-ಯ-ರು-ಗಳು
ನಂಬಿ-ಯಾ-ಚಾರಿ
ನಂಬಿ-ಯಾದ
ನಂಬಿ-ಯಾ-ರರು
ನಂಬಿ-ಯೋಜನ
ನಂಬು-ಗೆಯ
ನಕರ
ನಕರ-ಗಳ
ನಕರ-ಗಳು
ನಕರಮುಂ
ನಕರ-ಸೆಟ್ಟಿ
ನಕರಾ-ಚಾರಿ
ನಕರಾ-ಚಾರಿಯೂ
ನಕಲುಗ-ಳಂತೆ
ನಕ್ಷತ್ರ
ನಕ್ಷತ್ರದ
ನಕ್ಷತ್ರ-ದಲ್ಲಿ
ನಕ್ಷತ್ರಾ-ಕಾರದ
ನಕ್ಷೆ-ಯಲ್ಲಿ
ನಖರ
ನಖರಂಗಳು
ನಖರಕ್ಕೆ
ನಖರ-ಗಳೆಲ್ಲರೂ
ನಖರಮುಂ
ನಖ-ರರು
ನಖರಾ-ಚಾರಿ
ನಖರುಂಗಳುಂ
ನಖರೇಶ್ವರ
ನಗರ
ನಗರಕ್ಕೆ
ನಗರ-ಗ-ಳನ್ನು
ನಗರ-ಗ-ಳಲ್ಲಿ
ನಗರ-ಗ-ಳಾದವು
ನಗರ-ತರ
ನಗರದ
ನಗರ-ದಲ್ಲಿ
ನಗರ-ದಲ್ಲಿ-ರುವ
ನಗರ-ದಿಂದ
ನಗ-ರವಿ-ರುವ
ನಗ-ರವೂ
ನಗರ-ಸ-ಮುದ್ರ
ನಗರ-ಸ-ಮುದ್ರದ
ನಗರಿಯು
ನಗರೀ-ಕರಣ
ನಗರೀ-ಕರ-ಣಕ್ಕಿ-ರುವ
ನಗರೀಶ್ವರ
ನಗರುರ
ನಗರೂರ
ನಗರೂರಿನ
ನಗರೂರು
ನಗರೇರ
ನಗರೇಶ್ವರ
ನಗರೇಶ್ವರ-ದೇವ-ರಿಗೆ
ನಗಾ-ರಿಯ
ನಗಾರಿ-ಯನ್ನು
ನಗುಲನ
ನಗುಲನ-ಹಳ್ಳಿ
ನಗುಲನ-ಹಳ್ಳಿಯ
ನಗು-ವನ-ಹಳ್ಳಿ
ನಙ್ಗೈ-ಯಾರ್
ನಜರಾನಾ
ನಟಿ
ನಟಿ-ವೆಂಕಟಪ್ಪ-ನಾಯಕ
ನಟೀ
ನಟ್ಟ
ನಟ್ಟ-ಕಲ್ಲು
ನಡ-ಗಲ-ಪುರ
ನಡ-ಗಲ್ಪುರ
ನಡ-ಗಲ್ಪುರದ
ನಡದ
ನಡದು
ನಡ-ಯಿತು
ನಡವ
ನಡವಲ್ಲಿ-ಗೆ-ಪುರ-ವೆಂದು
ನಡವ-ಳಿ-ಕಾರ
ನಡ-ಸಿದಂ
ನಡ-ಸುವನ
ನಡಾ-ವಳಿ-ಕೆ-ಗ-ಳನ್ನು
ನಡಿ-ಗ-ರಗಿಸಿ
ನಡಿಗೆ-ರಗಿಸಿ
ನಡಿಸಿ
ನಡು-ನಾಡಿ-ನಲ್ಲಿ
ನಡುವಣ
ನಡುವಿನ
ನಡುವಿಲ್
ನಡುವೆ
ನಡುವೆಯೂ
ನಡು-ವೆಯೇ
ನಡೆದ
ನಡೆ-ದದ್ದು
ನಡೆ-ದಲ್ಲಿ
ನಡೆದವು
ನಡೆದಾಗ
ನಡೆದಿದೆ
ನಡೆದಿರ-ಬಹು-ದಾದ
ನಡೆದಿರ-ಬ-ಹುದು
ನಡೆದಿರ-ಬಹು-ದೆಂದು
ನಡೆದಿರ-ಬೇಕೆಂದು
ನಡೆದಿ-ರಲು
ನಡೆ-ದಿರುವ
ನಡೆ-ದಿ-ರು-ವಂತೆ
ನಡೆ-ದಿ-ರು-ವು-ದನ್ನು
ನಡೆ-ದಿ-ರುವುದು
ನಡೆದಿವೆ
ನಡೆದು
ನಡೆದು-ಕೊಂಡು
ನಡೆದು-ಕೊಳ್ಳುತ್ತಾರೆ
ನಡೆದು-ಕೊಳ್ಳು-ವು-ದನ್ನು
ನಡೆದು-ದರ
ನಡೆದುದು
ನಡೆದು-ಬಂದು
ನಡೆದು-ಬ-ರುವ-ಹಾಗೆ
ನಡೆದು-ಹೋಗ-ಬೇಕಾ-ಯಿತು
ನಡೆ-ಯದೇ
ನಡೆಯ-ಬೇಕಾಗಿದೆ
ನಡೆಯ-ಬೇಕೆಂದು
ನಡೆ-ಯಲಿ-ಎಂದು
ನಡೆ-ಯಲು
ನಡೆ-ಯವ
ನಡೆ-ಯಿತು
ನಡೆಯಿ-ತೆಂದು
ನಡೆಯು
ನಡೆ-ಯುತ
ನಡೆ-ಯುತಂ
ನಡೆಯುತ್ತದೆ
ನಡೆಯುತ್ತ-ದೆಂದು
ನಡೆಯುತ್ತಲೇ
ನಡೆಯುತ್ತವೆ
ನಡೆಯುತ್ತ-ವೆಂದೂ
ನಡೆಯುತ್ತಿತು
ನಡೆಯುತ್ತಿತ್ತು
ನಡೆಯುತ್ತಿತ್ತೆಂದು
ನಡೆಯುತ್ತಿತ್ತೆಂಬುದು
ನಡೆಯುತ್ತಿದ್ದ
ನಡೆಯುತ್ತಿದ್ದವು
ನಡೆಯುತ್ತಿದ್ದ-ವೆಂಬುದು
ನಡೆಯುತ್ತಿದ್ದು-ಅಲ್ಲಿಗೆ
ನಡೆಯುತ್ತಿದ್ದು-ದನ್ನು
ನಡೆಯುತ್ತಿದ್ದು-ದರ
ನಡೆಯುತ್ತಿದ್ದು-ದರಿಂದ
ನಡೆಯುತ್ತಿದ್ದುದು
ನಡೆಯುತ್ತಿ-ರುವಾ-ಗಲೇ
ನಡೆ-ಯುವ
ನಡೆಯು-ವಂತೆ
ನಡೆವ
ನಡೆವಂತಾಗಿ
ನಡೆ-ವಂದು
ನಡೆ-ವಲ್ಲಿ
ನಡೆವಲ್ಲಿ-ಗೆ-ಪುರ-ದಲ್ಲಿದ್ದ
ನಡೆ-ವುದು
ನಡೆಸ-ತೊಡಗಿ-ದರು
ನಡೆ-ಸತೊಡಗಿ-ದಳು
ನಡೆಸ-ಬೇ-ಕಾದ
ನಡೆಸ-ಬೇಕು
ನಡೆಸ-ಬೇಕೆಂದು
ನಡೆಸಲಾ-ಗದೇ
ನಡೆಸ-ಲಾಗಿದೆ
ನಡೆಸ-ಲಿಲ್ಲ
ನಡೆ-ಸಲು
ನಡೆಸಿ
ನಡೆಸಿ-ಕೊಂಡು
ನಡೆಸಿ-ಕೊಂಡು-ಬ-ರುವುದು
ನಡೆಸಿ-ಕೊಡ-ಬೇಕೆಂದು
ನಡೆಸಿ-ಕೊಡಲು
ನಡೆಸಿ-ಕೊಡು-ವಂತೆ
ನಡೆಸಿ-ಕೊಡು-ವಂತೆಯೂ
ನಡೆಸಿ-ಕೊಡು-ವರು
ನಡೆಸಿದ
ನಡೆಸಿ-ದನು
ನಡೆಸಿ-ದ-ನೆಂದು
ನಡೆಸಿ-ದ-ನೆಂದೂ
ನಡೆಸಿ-ದರು
ನಡೆಸಿ-ದರೂ
ನಡೆಸಿ-ದ-ರೆಂದು
ನಡೆಸಿ-ದಲ್ಲಿ
ನಡೆಸಿ-ದಾಗ
ನಡೆಸಿದ್ದಾ-ರೆಂದು
ನಡೆ-ಸಿದ್ದು
ನಡೆಸಿ-ರ-ಬಹು-ದೆಂದು
ನಡೆಸಿ-ರುವ
ನಡೆಸಿ-ರುವು
ನಡೆಸಿ-ರುವುದು
ನಡೆ-ಸುತ್ತ
ನಡೆ-ಸುತ್ತಾ
ನಡೆ-ಸುತ್ತಾರೆ
ನಡೆ-ಸುತ್ತಾಳೆ
ನಡೆ-ಸುತ್ತಿದ-ರೆಂಬುದು
ನಡೆ-ಸುತ್ತಿದ್ದ
ನಡೆ-ಸುತ್ತಿದ್ದಂತೆ
ನಡೆ-ಸುತ್ತಿದ್ದನು
ನಡೆ-ಸುತ್ತಿದ್ದ-ನೆಂದು
ನಡೆ-ಸುತ್ತಿದ್ದ-ನೆಂದೂ
ನಡೆ-ಸುತ್ತಿದ್ದ-ನೆಂಬುದು
ನಡೆ-ಸುತ್ತಿದ್ದರು
ನಡೆ-ಸುತ್ತಿದ್ದರೂ
ನಡೆಸುತ್ತಿದ್ದ-ರೆಂದು
ನಡೆ-ಸುತ್ತಿದ್ದ-ರೆಂಬುದು
ನಡೆ-ಸುತ್ತಿದ್ದರೇ
ನಡೆ-ಸುತ್ತಿದ್ದ-ವನ್ನೂ
ನಡೆ-ಸುತ್ತಿದ್ದ-ವ-ರನ್ನು
ನಡೆ-ಸುತ್ತಿದ್ದಾಗ
ನಡೆ-ಸುತ್ತಿದ್ದಿರ-ಬ-ಹುದು
ನಡೆ-ಸುತ್ತಿದ್ದು
ನಡೆ-ಸುತ್ತಿದ್ದು-ದಂತೂ
ನಡೆ-ಸುತ್ತಿದ್ದು-ದನ್ನು
ನಡೆ-ಸುತ್ತಿದ್ದು-ದ-ರಿಂದ
ನಡೆ-ಸುತ್ತಿರ-ಬಹು-ದೆಂದು
ನಡೆ-ಸುತ್ತಿರುತ್ತಾನೆ
ನಡೆ-ಸುತ್ತಿ-ರುವ-ವ-ರನ್ನು
ನಡೆ-ಸುತ್ತೇ-ವೆಂದು
ನಡೆ-ಸುವ
ನಡೆ-ಸುವಂಗೆ
ನಡೆಸು-ವಂತಹ
ನಡೆ-ಸು-ವಂತೆ
ನಡೆ-ಸುವ-ರ-ಸಂಖ್ಯಾತ
ನಡೆ-ಸುವ-ರೆಂದು
ನಡೆ-ಸುವ-ವ-ರಿಗೆ
ನಡೆ-ಸುವಾತ
ನಡೆಸು-ವು-ದಾಗಿ
ನಡೆ-ಸು-ವುದು
ನಣಕ್ಕನ್
ನದಿ
ನದಿ-ಗಳ
ನದಿ-ಗ-ಳನ್ನು
ನದಿ-ಗಳ-ಸಂಗ-ಮದ
ನದಿ-ಗ-ಳಾಗಿವೆ
ನದಿ-ಗ-ಳಿಗೆ
ನದಿ-ಗಳು
ನದಿಗೂ
ನದಿಗೆ
ನದಿ-ತೀರ-ದಲ್ಲಿ
ನದಿಯ
ನದಿ-ಯನ್ನು
ನದಿ-ಯನ್ನೂ
ನದಿ-ಯ-ಪಾತ್ರವು
ನದಿ-ಯ-ಮಡು
ನದಿ-ಯಲ್ಲಿ
ನದಿ-ಯ-ವರೆಗೂ
ನದಿ-ಯಾ-ಗಿದ್ದು
ನದಿ-ಯಾಚೆ
ನದಿ-ಯಿಂದ
ನದಿಯು
ನನಗೆ
ನನ್ದ-ನನೊಲವಿಂ
ನನ್ದಾ-ದೀ-ವಿಗೆಗೆ
ನನ್ನ
ನನ್ನಂತಹ
ನನ್ನನ್ನು
ನನ್ನಲ್ಲಿ
ನನ್ನಲ್ಲಿದೆ
ನನ್ನವ್ವೆ
ನನ್ನವ್ವೆಯ
ನನ್ನಿ
ನನ್ನಿ-ಕಂದರ್ಪ-ನೆಂಬು-ವ-ವನು
ನನ್ನಿಗ
ನನ್ನಿ-ನೊಳಂಬನು
ನನ್ನಿ-ಮಳ-ಲೂರಂ
ನನ್ನಿಯ
ನನ್ನಿ-ಯ-ಗಂಗ
ನನ್ನಿ-ಯ-ಮೇರು
ನನ್ನಿ-ಯ-ಸೇ-ಕರ
ನನ್ನು
ನಮ
ನಮಃ
ನಮಗೆ
ನಮಗೊಸೆದು
ನಮನ-ಗಳು
ನಮಸಿವಯ
ನಮಸಿವಾಯ
ನಮ-ಸೋಮಾರ್ದ್ಧ-ಧಾರಿಣೇ
ನಮಸ್ಕರಿಸಿ
ನಮಸ್ಕಾರ
ನಮಸ್ಕಾರ-ವನ್ನು
ನಮಸ್ತುಂಗ
ನಮಾಮಿ
ನಮಿಸಿ
ನಮೂ-ದಾಗಿ-ದೆ-ಯೆಂದು
ನಮೂದಾ-ಗಿ-ರುವ
ನಮೂ-ದಾಗಿ-ರು-ವಂತೆ
ನಮೂದಿ-ನಲ್ಲಿ
ನಮೂದಿ-ಸದ
ನಮೂದಿಸ-ಲಾಗಿದೆ
ನಮೂದಿ-ಸಿದೆ
ನಮೂದಿ-ಸಿದ್ದು
ನಮೂದಿ-ಸಿರ-ಬ-ಹುದು
ನಮೂದಿಸಿ-ರು-ವು-ದಿಲ್ಲ
ನಮೂದಿಸಿ-ರುವುದು
ನಮೂದಿ-ಸಿಲ್ಲ
ನಮೂದಿ-ಸುತ್ತಾ
ನಮ್ಮ
ನಮ್ಮ-ದೇನೂ
ನಮ್ಮಾಳ್ವರರ
ನಮ್ಮಾಳ್ವರ್
ನಮ್ಮಾಳ್ವಾರರ
ನಮ್ಮಾಳ್ವಾ-ರ-ರಿಗೆ
ನಮ್ಮಾಳ್ವಾರ್
ನಮ್ಮಾಳ್ವಾರ್ಗೆ
ನಯ-ಕೀರ್ತಿ
ನಯ-ಕೀರ್ತಿ-ದೇವ
ನಯ-ಕೀರ್ತಿಯ
ನಯಣದ
ನಯ-ಧೀರ
ನಯ-ಧೀರ-ರೊಡ
ನಯ-ಭದ್ರ
ನಯ-ಸೇನನ
ನಯಿ
ನಯಿ-ವೇದ್ಯಕೆ
ನಯೋಂನತೇಃ
ನರ-ಅಸೀ-ಪುರ
ನರಗ
ನರ-ಗಲು
ನರ-ಗುಂದ
ನರ-ಣ-ನಾ-ರಾಯಣ
ನರ-ಪತಿ
ನರ-ಪತಿ-ಬೆನ್ನೊಳಿರ್ದೊನಿ-ದಿರಾಂತುದು
ನರ-ಭಕ್ಷಕ
ನರ-ಮನ-ಕಟ್ಟೆ
ನರ-ರಾದ
ನರ-ಶೀ-ಪುರ
ನರಸ
ನರಸಂಣ-ನಾಯ್ಕರು
ನರ-ಸಣ್ಣ
ನರ-ಸಣ್ಣ-ನಾಯ-ಕರ
ನರ-ಸಣ್ಣ-ನಾಯ-ಕರು
ನರ-ಸನ
ನರಸ-ನಾಯಕ
ನರಸ-ನಾಯ-ಕನ
ನರಸ-ನಾಯ-ಕ-ನಿಗೆ
ನರಸ-ನಾಯ-ಕನು
ನರಸ-ನಿಗೆ
ನರ-ಸನು
ನರಸಯ್ಯ
ನರಸಯ್ಯ-ನ-ವರ
ನರಸಯ್ಯನು
ನರಸ-ರಾಜ
ನರಸ-ರಾಜನ
ನರಸ-ರಾಜನು
ನರಸ-ರಾಜ-ನೆಂದೇ
ನರಸ-ರಾಜರ
ನರಸ-ರಾಜರು
ನರಸ-ರಾಜೇಂದ್ರನ
ನರಸ-ರಾಜೊಡೆಯ
ನರಸ-ರಾಜೊಡೆಯರ
ನರಸ-ರಾಜೊಡೆಯ-ರಿಗೂ
ನರ-ಸಾಕ್ಷಿ-ಯಾಗಿ
ನರ-ಸಾಕ್ಷಿ-ಯಾಗಿದ್ದ-ರೆಂದು
ನರ-ಸಾಕ್ಷಿ-ಯಾಗಿ-ರುತ್ತಾರೆ
ನರಸಾ-ವನಿ-ಪಾಲಜ
ನರ-ಸಿಂಗ
ನರ-ಸಿಂಗಣ್ಣ-ಗ-ಳಿಗೆ
ನರ-ಸಿಂಗ-ದೇವ
ನರ-ಸಿಂಗ-ದೇವನು
ನರ-ಸಿಂಗ-ದೇವರು
ನರ-ಸಿಂಗನ
ನರ-ಸಿಂಗ-ನಾಯಕ
ನರ-ಸಿಂಗ-ನಾಯಕಂ
ನರ-ಸಿಂಗ-ನಾಯ-ಕನ
ನರ-ಸಿಂಗ-ನಾಯ-ಕನು
ನರ-ಸಿಂಗ-ಯ-ದೇವ
ನರ-ಸಿಂಗಯ್ಯ
ನರ-ಸಿಂಗಯ್ಯ-ದೇವ
ನರ-ಸಿಂಗಯ್ಯನ
ನರ-ಸಿಂಗಯ್ಯನೂ
ನರ-ಸಿಂಗ-ರಾಜ
ನರ-ಸಿಂಗ-ರಾಜ-ವೊಡೆ-ಯರ
ನರ-ಸಿಂಗ-ರಾಯ
ನರ-ಸಿಂಗ-ವರ್ಮ
ನರ-ಸಿಂಗ-ವರ್ಮನು
ನರ-ಸಿಂಹ
ನರ-ಸಿಂಹಕ್ಷೇತ್ರಕ್ಕೆ
ನರ-ಸಿಂಹ-ದೇವ
ನರ-ಸಿಂಹ-ದೇವನ
ನರ-ಸಿಂಹ-ದೇವರ
ನರ-ಸಿಂಹ-ದೇವ-ರಿಗೆ
ನರ-ಸಿಂಹ-ದೇವರು
ನರ-ಸಿಂಹನ
ನರ-ಸಿಂಹ-ನಂತೆಯೇ
ನರ-ಸಿಂಹ-ನ-ಕಾಲ-ದಲ್ಲಿ
ನರ-ಸಿಂಹ-ನ-ಕಾಲ-ದ-ವರೆಗೆ
ನರ-ಸಿಂಹ-ನದೇ
ನರ-ಸಿಂಹ-ನನ್ನು
ನರ-ಸಿಂಹ-ನನ್ನೂ
ನರ-ಸಿಂಹ-ನ-ರಪಾಳಂ
ನರ-ಸಿಂಹ-ನ-ರಸು-ಗೆಯ್ಯುತ್ತಿರ್ದ್ದಂ
ನರ-ಸಿಂಹ-ನಲ್ಲಿ
ನರ-ಸಿಂಹ-ನಾಯ-ಕನ
ನರ-ಸಿಂಹ-ನಿಂದ
ನರ-ಸಿಂಹ-ನಿಗೂ
ನರ-ಸಿಂಹ-ನಿಗೆ
ನರ-ಸಿಂಹನು
ನರ-ಸಿಂಹನೂ
ನರ-ಸಿಂಹ-ನೃ-ಪತಿ
ನರ-ಸಿಂಹ-ಪುರ-ವಾದ
ನರ-ಸಿಂಹ-ಪುರ-ವೆಂಬ
ನರ-ಸಿಂಹ-ಮಹಾ-ರಾಯರು
ನರ-ಸಿಂಹ-ಮೂರ್ತಿ-ಯ-ವರು
ನರ-ಸಿಂಹ-ರನ್ನು
ನರ-ಸಿಂಹ-ರಾಯನ
ನರ-ಸಿಂಹ-ರಾಯರು
ನರ-ಸಿಂಹ-ವರ್ಮನ
ನರ-ಸಿಂಹ-ವರ್ಮನು
ನರ-ಸಿಂಹ-ವರ್ಮನೂ
ನರ-ಸಿಂಹ-ವರ್ಮ್ಮ-ನೋಡಿದ
ನರ-ಸಿಂಹ-ಶಠಕೋಪ
ನರ-ಸಿಂಹ-ಸೂರಿಯು
ನರ-ಸಿಂಹಸ್ವಾಮಿ
ನರ-ಸಿಂಹಸ್ವಾಮಿಗೆ
ನರ-ಸಿಂಹಸ್ವಾಮಿಯ
ನರ-ಸಿಂಹಸ್ವಾಮಿ-ಯ-ವರ
ನರ-ಸಿಂಹಾ-ಚಾರ್ಯರು
ನರ-ಸಿಂಹೋರ್ವೀಶನಡ್ಡಾ-ಯದದ
ನರ-ಸಿಂಹೋರ್ವ್ವೀಶನ
ನರಸಿ-ಪುರದ
ನರಸೀ-ಪುರ
ನರಸೀ-ಪುರದ
ನರಸೀ-ಪುರವೇ
ನರಸೀ-ಪುರ-ಹೋ-ಬಳಿ
ನರಸೇಂದ್ರ-ನೆಂಬ
ನರ-ಹರಿ-ಭಟ್ಟ
ನರಿ
ನರಿ-ಗಲ್ಲ-ತೊರೆ
ನರಿ-ಹಳ್ಳಿ
ನರಿ-ಹಳ್ಳಿ-ಯಲ್ಲೂ
ನರೆ-ಗನ-ಹಳ್ಳಿ
ನರೇಂದ್ರ-ಕೀರ್ತಿ
ನರ್ತ-ಕಿಯರು
ನರ್ತ್ತಕೆ
ನಲ-ಗೌಡ
ನಲ-ವತ್ತು
ನಲ-ವತ್ತೊಂದು
ನಲಿ-ವಂತು
ನಲು-ಗನ-ಹಳ್ಳಿ
ನಲು-ಗನ-ಹಳ್ಳಿ-ಯನ್ನು
ನಲು-ವ-ಶಿಯ
ನಲೂರು
ನಲ್ಲ-ಖಡ್ಡಾಯ
ನಲ್ಲ-ತಂಬಿಯು
ನಲ್ಲ-ನಂಬಿ
ನಲ್ಲಿ
ನಲ್ಲೆಂಮ್ಮೆ
ನಲ್ಲೆತ್ತು
ನಳನ-ಹಳ್ಳಿಯ
ನಳನ-ಹು-ಷಾದಿ-ಗ-ಳಂತೆ
ನಳ-ನಾಮ-ಸಂವತ್ಸರದ
ನಳ-ಮಾರುಡು
ನಳ-ಸಂವತ್ಸರ
ನಳ-ಸಂವತ್ಸರದ
ನಳ-ಸಂವತ್ಸರವು
ನಳಿನ-ಕೆರೆ-ಯನ್ನು
ನವಖಂಡಮಾಗೆ
ನವಗ್ರಹ-ಗ-ಳೆಂದು
ನವ-ದಂಡ-ನಾಯ-ಕ-ರು-ಗ-ಳೆಂದು
ನವ-ದಣ್ಣಾ-ಯಕ-ರೆಂಬ
ನವ-ನಿಧಿ-ಕುಲ-ಪರ್ವ-ತದ
ನವರ
ನವರಂಗ
ನವ-ರಂಗಕ್ಕೆ
ನವರಂಗ-ಗ-ಳನ್ನು
ನವರಂಗ-ಗಳನ್ನೊಳ-ಗೊಂಡ
ನವರಂಗ-ಗ-ಳಿಂದ
ನವ-ರಂಗದ
ನವ-ರಂಗ-ದಲ್ಲಿ
ನವ-ರಂಗ-ದಲ್ಲಿ-ರುವ
ನವರಂಗ-ನ-ವನ್ನೊಳ-ಗೊಂಡ
ನವ-ರಂಗ-ಮಂಟಪ
ನವ-ರಂಗ-ಮಂಟಪದ
ನವರಂಗ-ಳನ್ನು
ನವರಂಗ-ಳಿಂದ
ನವರಂಗ-ವನ್ನು
ನವರಂಗ-ವಿದೆ
ನವ-ರತ್ನ
ನವ-ರತ್ನ-ಕಿರೀಟ-ವನ್ನು
ನವ-ರತ್ನ-ಗ-ಳೆಂದು
ನವರ-ರಂಗ
ನವ-ರಿಗೆ
ನವರು
ನವಲೆ-ನಾಡ
ನವಲೆ-ನಾಡು
ನವ-ಶಿಲಾ-ಯುಗದ
ನವಸಂಸ್ಕಾರ
ನವಾಬ
ನವಾಬ್
ನವಿಲ-ಹಳ್ಳವು
ನವಿಲು
ನವಿ-ಲೆಯ
ನವೀ-ಕರಿ-ಸ-ಲಾಗಿದೆ
ನವೀ-ಕರಿ-ಸಿ-ದನು
ನವುಲೆಸೊಣ್ನನ-ಕಟ್ಟೆ
ನವೆಂಬರ್
ನವೆಂಬರ್ಡಿಸೆಂಬರ್
ನವ್ಯ
ನವ್ಯ-ಚಾರಿತ್ರನುಂ
ನವ್ಯ-ಸೇಸೆ
ನವ್ವಾಬ್
ನಶ್ವರ
ನಾಂಟಲಾ
ನಾಇಂದರ
ನಾಕಣ
ನಾಕ-ನಿಪ್ಪ
ನಾಕಯ್ಯನು
ನಾಕ-ಹಳ್ಳಿ
ನಾಕು
ನಾಗ
ನಾಗಂಣ
ನಾಗಂಣ-ಗಳು
ನಾಗಂಣ-ವೊಡೆಯ
ನಾಗಂಣ-ವೊಡೆ-ಯನು
ನಾಗಂಣ್ಣ
ನಾಗಂಣ್ನನ
ನಾಗ-ಚಂದ್ರನ
ನಾಗ-ಡೊ-ಯನು
ನಾಗಣ
ನಾಗ-ಣ-ನಾಯ-ಕರು
ನಾಗಣ್ಣ
ನಾಗಣ್ಣನ
ನಾಗಣ್ಣ-ನ-ವರು
ನಾಗಣ್ಣನು
ನಾಗಣ್ಣನ್ನನು
ನಾಗ-ದೇವ
ನಾಗ-ದೇವನ
ನಾಗ-ದೇವನು
ನಾಗ-ದೇವ-ಭಟ್ಟ-ರಿಗೆ
ನಾಗ-ದೇವ-ರ-ಸರು
ನಾಗನ
ನಾಗ-ನ-ಹಳ್ಳಿ
ನಾಗ-ನ-ಹಳ್ಳಿಗೆ
ನಾಗ-ನಾ-ಗನ
ನಾಗ-ನಾ-ಗನ-ಕಟ್ಟೆ
ನಾಗ-ನಾಯ-ಕನ
ನಾಗ-ನಿಂದ
ನಾಗ-ನಿಂದಲೇ
ನಾಗನು
ನಾಗಪ್ಪ
ನಾಗಪ್ಪ-ಗ-ಉಡ
ನಾಗಪ್ಪನ
ನಾಗಪ್ಪ-ನನ್ನು
ನಾಗಪ್ಪ-ನಾ-ಗರಸ
ನಾಗ-ಭಟ್ಟ-ನಿಗೆ
ನಾಗ-ಭಟ್ಟನು
ನಾಗ-ಮಂಗಲ
ನಾಗ-ಮಂಗಲಕೆ
ನಾಗ-ಮಂಗಲಕ್ಕೆ
ನಾಗ-ಮಂಗಲಕ್ಕೆ-ರಾಜ್ಯಕ್ಕೆ
ನಾಗ-ಮಂಗಲದ
ನಾಗ-ಮಂಗಲ-ದಲು
ನಾಗ-ಮಂಗಲ-ದಲ್ಲಿ
ನಾಗ-ಮಂಗಲ-ದ-ವರೇ
ನಾಗ-ಮಂಗಲ-ದ-ವೀರ-ಭದ್ರ-ದೇವರ
ನಾಗ-ಮಂಗಲ-ದಿಂದ
ನಾಗ-ಮಂಗಲ-ದೊಡ್ಡ-ಕೆರೆ-ಗಳು
ನಾಗ-ಮಂಗಲ-ರಾಜ್ಯದ
ನಾಗ-ಮಂಗಲ-ವನ್ನು
ನಾಗ-ಮಂಗಲವು
ನಾಗ-ಮಂಗಲವೂ
ನಾಗ-ಮಂಗಲ-ವೆಂದು
ನಾಗ-ಮಂಗಲಶ್ರವ-ಣ-ಬೆಳಗೊಳ
ನಾಗ-ಮಂಗಲಸ್ಥಳದ
ನಾಗ-ಮಂಲ-ಗದ
ನಾಗ-ಮಯ್ಯ
ನಾಗ-ಮಯ್ಯ-ನಿಗೆ
ನಾಗ-ಮಯ್ಯನು
ನಾಗ-ಮರ್ವ
ನಾಗ-ಮಲ್ಲಿ-ದೇವ
ನಾಗ-ಮಾರ್ಜಿತ-ಮದಾತ್ತಿಂಮಕ್ಷಿತೀಂದ್ರಾತ್ಮಜಃ
ನಾಗ-ಮೈನ್ದ
ನಾಗಯ್ಯ
ನಾಗಯ್ಯ-ಗಳ
ನಾಗಯ್ಯನ
ನಾಗಯ್ಯ-ನನ್ನು
ನಾಗಯ್ಯ-ನ-ವರ
ನಾಗಯ್ಯ-ನ-ವರು
ನಾಗಯ್ಯ-ನ-ವರೂ
ನಾಗಯ್ಯ-ನಿಗೆ
ನಾಗಯ್ಯನು
ನಾಗಯ್ಯನೂ
ನಾಗಯ್ಯ-ನೆಂಬ
ನಾಗಯ್ಯ-ನೆಂಬು-ವ-ವನು
ನಾಗಯ್ಯನೇ
ನಾಗ-ರ-ಕಟ್ಟದ
ನಾಗ-ರ-ಕಲ್ಲು-ಗಳ
ನಾಗ-ರಖಂಡ
ನಾಗ-ರಖಂಡ-ನ-ವನ್ನು
ನಾಗ-ರ-ಘಟ್ಟದ
ನಾಗ-ರದ
ನಾಗ-ರದ-ಮೊಲೆ-ಗೋಡನ್ನು
ನಾಗ-ರಸ
ನಾಗ-ರ-ಸನ
ನಾಗ-ರ-ಸನು
ನಾಗ-ರಸರ
ನಾಗ-ರ-ಸರು
ನಾಗ-ರಸ-ರೆಂದು
ನಾಗ-ರ-ಹಾಳ
ನಾಗ-ರಾಜಯ್ಯ-ನ-ವರು
ನಾಗ-ರಾಜ-ರಾವ್
ನಾಗ-ರಾಜು
ನಾಗ-ರಾಸಿ
ನಾಗ-ರಿಕ-ತೆಯ
ನಾಗ-ರಿ-ಲಿಪಿ
ನಾಗ-ಲ-ದೇವಿ
ನಾಗ-ಲ-ದೇವಿ-ಯರ
ನಾಗ-ಲ-ದೇವಿ-ಯಿಂದ
ನಾಗ-ಲಾಂಬಿಕಾ
ನಾಗ-ಲಾಂಬಿ-ಕೆಯ
ನಾಗ-ಲಾಂಬಿ-ಕೆಯರು
ನಾಗ-ಲಾ-ದೇವಿ
ನಾಗ-ಲಾ-ದೇ-ವಿಗೆ
ನಾಗ-ಲಾ-ದೇವಿ-ಯಿಂದ
ನಾಗ-ಲಾ-ಪುರ
ನಾಗ-ಲಾ-ಪುರ-ವೆಂದು
ನಾಗ-ಲಾ-ಪುರ-ವೆಂಬ
ನಾಗ-ಲಿಂಗನ
ನಾಗ-ಲೂಟಿ
ನಾಗಲೆ
ನಾಗ-ವರ್ಮ
ನಾಗ-ವರ್ಮ್ಮಯ್ಯ
ನಾಗ-ವಲ್ಲಿ
ನಾಗ-ವಾಸದ
ನಾಗವ್ವೆಯ-ಕೆರೆ
ನಾಗವ್ವೆಯ-ಕೆರೆ-ಯನ್ನೂ
ನಾಗವ್ವೆಯೂ
ನಾಗ-ಶಕ್ತಿ
ನಾಗ-ಶರ್ಮ
ನಾಗಾಂಬಾ
ನಾಗಾಕ್ಷರ-ದಲ್ಲಿ
ನಾಗಾ-ಚಾರಿ
ನಾಗಾ-ದಯ
ನಾಗಾ-ಪಂಡಿ-ತರ
ನಾಗಾ-ಭಟ್ಟ-ರಿಗೆ
ನಾಗಾಯ-ಭಟ್ಟನ
ನಾಗಿ
ನಾಗಿ-ದೇವಣ್ಣನ
ನಾಗಿ-ದೇವಣ್ಣನು
ನಾಗಿದ್ದ
ನಾಗಿದ್ದಂತೆ
ನಾಗಿದ್ದನು
ನಾಗಿದ್ದ-ನೆಂದು
ನಾಗಿದ್ದಾನೆ
ನಾಗಿದ್ದು
ನಾಗಿದ್ದು-ದರ
ನಾಗಿ-ಯಕ್ಕನು
ನಾಗಿ-ಯಣ್ಣ
ನಾಗಿ-ಯಣ್ಣನು
ನಾಗಿ-ಯಣ್ಣ-ನೆಂಬು-ವ-ವನು
ನಾಗಿಯೂ
ನಾಗೀ-ದೇವ-ನನ್ನು
ನಾಗುತ್ತಾನೆ
ನಾಗುಳ
ನಾಗು-ಳದ
ನಾಗೂರು-ಗಳೂ
ನಾಗೆಯ
ನಾಗೆಯ-ನಾಯಕ
ನಾಗೆಯ-ನಾಯ-ಕನ
ನಾಗೆಯ-ನಾಯ-ಕರು
ನಾಗೇಣ
ನಾಗೇಶ್ವರ
ನಾಗೇಶ್ವರ-ದೇವ-ರಿಗೆ
ನಾಗೊಡೆ-ಯನ
ನಾಗೊಡೆಯ-ನಿಗೆ
ನಾಗೊಡೆ-ಯನು
ನಾಗೋ-ಹಳ್ಳಿ
ನಾಚ-ಗಾವುಂಡ
ನಾಚಾರಮ್ಮ
ನಾಚಾರಮ್ಮನು
ನಾಚಿ-ಯಾರಮ್ಮನ
ನಾಚಿ-ಯಾರಮ್ಮನಿಗೆ
ನಾಚಿ-ಯಾರಮ್ಮನು
ನಾಚಿ-ರಾಜೀಯ-ದಲ್ಲಿದೆ
ನಾಚ್ಚಾರಮ್ಮ-ನ-ವರ
ನಾಚ್ಚಾರ್
ನಾಚ್ಚಿ-ಯಾರ-ರಿಗೆ
ನಾಚ್ಚಿ-ಯಾರ್
ನಾಚ್ಚಿ-ಯಾರ್ಗಳ
ನಾಚ್ಚಿ-ಯಾರ್ಗೆ
ನಾಟಕೇಷು
ನಾಟನ-ಹಳ್ಳಿ
ನಾಟ್ಟು
ನಾಟ್ಟೋರ್ಗಳ್
ನಾಟ್ಯಾಂಗರುಂ
ನಾಡ
ನಾಡ-ಅನ್ನು
ನಾಡ-ಗವುಡ
ನಾಡ-ಗವು-ಡ-ಗಳ
ನಾಡ-ಗವು-ಡರು
ನಾಡ-ಗಾವುಂಡ-ನನ್ನು
ನಾಡ-ಗಾವುಂಡರು
ನಾಡ-ಗೌಡ-ರೆಂದು
ನಾಡ-ಗೌ-ಡಿಕೆ
ನಾಡನ್ನು
ನಾಡನ್ನೂ
ನಾಡನ್ನೇ
ನಾಡಪ್ರಭು
ನಾಡಪ್ರಭು-ಗಳು
ನಾಡ-ಬೋ-ಯನ-ಹಳ್ಳಿ
ನಾಡ-ಬೋ-ಯನ-ಹಳ್ಳಿ-ನಾಡ-ಬೋವ-ನ-ಹಳ್ಳಿ
ನಾಡ-ಬೋವ-ನ-ಹಳ್ಳಿ
ನಾಡ-ಬೋವ-ನ-ಹಳ್ಳಿಯ
ನಾಡ-ಮಂಡ-ಳಿಕ
ನಾಡ-ಮಂಡ-ಳೀ-ಕರು
ನಾಡ-ಮಾಣಿ-ಕ-ದೊಡಲೂ-ರನ್ನು
ನಾಡ-ಮಾಣಿ-ಕ-ದೊಡ-ಲೂರಿನ
ನಾಡ-ಮಾಣಿ-ಕ-ದೊಡ-ಲೂರಿನಲ್ಲಿ
ನಾಡ-ಮಾಣಿ-ಕ-ದೊಡ-ಲೂರಿನಲ್ಲಿದ್ದ
ನಾಡ-ಮಾಣಿ-ಕ-ದೊಡ-ಲೂರು
ನಾಡ-ರಸ-ರಾದ
ನಾಡ-ಸುಂಕ
ನಾಡ-ಸುಂಕ-ವನ್ನು
ನಾಡ-ಸೇನ-ಬೋವ-ನ-ಬ-ರಹ
ನಾಡಾ-ಗಿತ್ತು
ನಾಡಾಗಿತ್ತೆಂದು
ನಾಡಾಗಿದ್ದ
ನಾಡಾ-ಗಿದ್ದು
ನಾಡಾಗಿ-ರ-ಬ-ಹುದು
ನಾಡಾಗಿ-ರುವ
ನಾಡಾದು-ದೆಲ್ಲ-ವ-ಮನೇಕಚ್ಛತ್ರಂ
ನಾಡಾಳುತ್ತಿದ್ದ-ನೆಂದು
ನಾಡಾ-ಳುವ
ನಾಡಾ-ಳುವನ
ನಾಡಾಳ್ವ
ನಾಡಾಳ್ವಂ
ನಾಡಾಳ್ವನು
ನಾಡಾಳ್ವರು
ನಾಡಿ-ಗರು
ನಾಡಿ-ಗವುಡ
ನಾಡಿ-ಗವು-ಡ-ನ-ವರ
ನಾಡಿ-ಗವು-ಡ-ನ-ವರು
ನಾಡಿ-ಗವು-ಡರ
ನಾಡಿಗೂ
ನಾಡಿಗೆ
ನಾಡಿಗೇ
ನಾಡಿಗೌ-ನನ್ನು
ನಾಡಿತ್ತೆಂದು
ನಾಡಿನ
ನಾಡಿ-ನಲ್ಲಿ
ನಾಡಿ-ನಲ್ಲಿತ್ತು
ನಾಡಿ-ನಲ್ಲಿತ್ತೆಂದು
ನಾಡಿ-ನಲ್ಲಿದ್ದ
ನಾಡಿ-ನಲ್ಲಿದ್ದವು
ನಾಡಿ-ನಲ್ಲಿಯೇ
ನಾಡಿನ-ವರ
ನಾಡಿನ-ವ-ರಿಗೂ
ನಾಡಿನ-ವರು
ನಾಡಿನಿಂದ
ನಾಡಿನೊಳ-ಗಿ-ರುವ
ನಾಡಿನೊಳಗೇ
ನಾಡಿಯರು
ನಾಡು
ನಾಡು-ಕಟ್ಟು-ದೊಳ-ಗಣ
ನಾಡು-ಕಬ್ಬಪ್ಪು
ನಾಡು-ಕಲ್ಕಣಿ-ನಾಡು-ಕಲಿ-ಕಣಿ-ನಾಡು
ನಾಡು-ಗಳ
ನಾಡು-ಗ-ಳನ್ನು
ನಾಡು-ಗ-ಳಾಗಿ
ನಾಡು-ಗ-ಳಿಗೆ
ನಾಡು-ಗಳು
ನಾಡು-ಗ-ಳೆಂದು
ನಾಡು-ಗಳೆಂಬ
ನಾಡು-ನೀರ್ಗುಂದ
ನಾಡು-ಬಡ-ಗುಂದ
ನಾಡು-ಬಡಗು-ನಾಡು-ವಡ-ಗೆರೆ
ನಾಡು-ವಟ್ಟ-ವಾಗಿ
ನಾಡೂ
ನಾಡೆಂದರೆ
ನಾಡೆಂದು
ನಾಡೆ-ಹಳ್ಳಿ-ಗ-ಳನ್ನು
ನಾಡೇ
ನಾಡೊಳ-ಗಣ
ನಾಡೊಳಗಿ-ದೆಸೆದ
ನಾಡೊಳ-ಗಿನ
ನಾಣ್ಯ
ನಾಣ್ಯಕ್ಕೆ
ನಾಣ್ಯ-ಗಳ
ನಾಣ್ಯ-ಗ-ಳನ್ನು
ನಾಣ್ಯ-ಗಳಿದ್ದವು
ನಾಣ್ಯ-ಗಳು
ನಾಣ್ಯ-ವನ್ನು
ನಾಣ್ಯ-ವಾ-ಗಿದ್ದು
ನಾಣ್ಯವು
ನಾಣ್ಯ-ವೆಂದೂ
ನಾತನ-ಸುತ-ರ-ಗಣಿತ
ನಾಥ
ನಾಥ-ನನ್ನು
ನಾಥ-ನಿಂದ
ನಾಥ-ಪಂಥದ
ನಾಥ-ಪರಂಪರೆ
ನಾಥ-ಮುನಿ-ಗಳು
ನಾಥರ
ನಾಥ-ಸಂಪ್ರ-ದಾ-ಯವು
ನಾದ
ನಾದ-ರದಿಂ
ನಾನಲ
ನಾನಲ-ಕೆರೆ
ನಾನಲ-ಕೆರೆಯ
ನಾನಲ-ಕೆರೆ-ಯನ್ನು
ನಾನಲ-ಕೆರೆ-ಯಲ್ಲಿ
ನಾನಲ-ಕೆರೆ-ಯು-ಇಂದಿನ
ನಾನಲ-ಕೆರೆ-ಲಾಳ-ನ-ಕೆರೆ-ಯನ್ನು
ನಾನಲ-ಕೆಱೆಯ
ನಾನಾ
ನಾನಾ-ಉಭಯ-ದೇಶಿ
ನಾನಾ-ಗೋತ್ರ
ನಾನಾ-ಗೋತ್ರದ
ನಾನಾ-ಚಿತ್ರ-ಪತ್ರಂಗಳಿಂ
ನಾನಾ-ಜಾತಿ-ಯಾದ
ನಾನಾ-ದೇಶಿ
ನಾನಾ-ದೇಶಿ-ಗಳು
ನಾನಾ-ದೇಶಿ-ಯರು
ನಾನಾ-ದೇಸಿ
ನಾನಾ-ದೇಸಿ-ಗರು
ನಾನಾ-ದೇಸಿ-ಯ-ರನ್ನು
ನಾನಾ-ದೇಸಿ-ಯರು
ನಾನಾ-ದೇಸಿ-ಯಲ್ಲೂ
ನಾನಾ-ದೇಸಿ-ಯ-ವರು
ನಾನಾ-ದೇಸಿಯಿಂ
ನಾನಾ-ದೇಸಿ-ಯಿಂದ
ನಾನಾ-ದೇಸೀ
ನಾನಾ-ಬ-ಗೆಯ
ನಾನಾ-ವರ್ನ
ನಾನಾ-ಶಾಖೆಯ
ನಾನಾ-ಸೂತ್ರದ
ನಾನು
ನಾನೂ
ನಾನೂರು
ನಾನೋಜಿ
ನಾಮ
ನಾಮ-ಕರಣ
ನಾಮ-ಕರ-ಣ-ಮಾಡಿ
ನಾಮಕ್ಕೆ
ನಾಮ-ಗಳು
ನಾಮ-ಗಾ-ಣಿಕೆ
ನಾಮಗ್ರಾಮ
ನಾಮ-ತೀರ್ಥದ
ನಾಮದ
ನಾಮ-ದ-ಕಟ್ಟೆ
ನಾಮ-ದ-ತೊಟ್ಟಿ
ನಾಮ-ದ-ಯಾಂಕ
ನಾಮ-ಧರಿ-ಸು-ವುದು
ನಾಮ-ಧೇಯ
ನಾಮ-ಧೇಯ-ನಾದ
ನಾಮ-ವಿ-ಶೇಷ-ಣ-ದಿಂದ
ನಾಮವು
ನಾಮಸ್ಮ-ರಣೆ
ನಾಮಸ್ಮ-ರಣೆ-ಯನ್ನು
ನಾಮಾಯಂ
ನಾಮಾ-ವಳಿ
ನಾಮಾವ-ಳಿ-ಸಮಾಲಂಕೃ-ತರುಂ
ನಾಮಾವ-ಶೇಷ-ಗೊ-ಳಿಸಿ-ದ-ನೆಂದು
ನಾಮಾವ-ಶೇಷ-ವಾಗಿತ್ತೆಂದು
ನಾಮೆ
ನಾಮ್ನಾ
ನಾಯ
ನಾಯಂಕರ
ನಾಯಕ
ನಾಯಕಂ
ನಾಯ-ಕಂಕುರಿ-ದರಿದ-ರಿದು
ನಾಯ-ಕ-ಅರಸ
ನಾಯ-ಕ-ತನ
ನಾಯ-ಕ-ತನಕೆ
ನಾಯ-ಕ-ತನಕ್ಕೆ
ನಾಯ-ಕ-ತ-ನದ
ನಾಯ-ಕ-ತನ-ದಿಂದ
ನಾಯ-ಕ-ತನಮಂ
ನಾಯ-ಕ-ತನ-ವನ್ನು
ನಾಯ-ಕ-ತನ-ವೆಂಬ
ನಾಯ-ಕ-ತ-ವನ್ನು
ನಾಯ-ಕತ್ವ
ನಾಯ-ಕತ್ವಕ್ಕೆ
ನಾಯ-ಕ-ದೇವ
ನಾಯ-ಕ-ದೇವನ
ನಾಯ-ಕ-ದೇವ-ಪಿಳ್ಳೆ
ನಾಯ-ಕ-ದೇವರು
ನಾಯ-ಕ-ದೇವರ್
ನಾಯ-ಕ-ದೇವರ್ಗೂ
ನಾಯ-ಕನ
ನಾಯ-ಕ-ನನೂ
ನಾಯ-ಕ-ನನ್ನಾಗಿ
ನಾಯ-ಕ-ನನ್ನು
ನಾಯ-ಕ-ನ-ಹಳ್ಳಿ
ನಾಯ-ಕ-ನಾಗಿ
ನಾಯ-ಕ-ನಾ-ಗಿದ್ದ
ನಾಯ-ಕ-ನಾಗಿ-ರ-ಬ-ಹುದು
ನಾಯ-ಕ-ನಾದ
ನಾಯ-ಕ-ನಿಗೆ
ನಾಯ-ಕ-ನಿರ-ಬ-ಹುದು
ನಾಯ-ಕನು
ನಾಯ-ಕನೂ
ನಾಯ-ಕ-ನೆಂಬ
ನಾಯ-ಕ-ನೆಂಬು-ವ-ವನು
ನಾಯ-ಕ-ನೆತ್ತಿದ
ನಾಯ-ಕನೇ
ನಾಯ-ಕ-ಮಕ್ಕಳು
ನಾಯ-ಕರ
ನಾಯ-ಕ-ರ-ಗಂಡ
ನಾಯ-ಕ-ರನ್ನು
ನಾಯ-ಕ-ರಲ್ಲಿ
ನಾಯ-ಕ-ರಾ-ಗಿದ್ದರು
ನಾಯ-ಕ-ರಿಗೂ
ನಾಯ-ಕ-ರಿಗೆ
ನಾಯ-ಕ-ರಿಗೆಲ್ಲಾ
ನಾಯ-ಕ-ರಿದ್ದ
ನಾಯ-ಕ-ರಿದ್ದು
ನಾಯ-ಕರು
ನಾಯ-ಕ-ರು-ಗ-ಳಿಗೆ
ನಾಯ-ಕ-ರು-ಗಳು
ನಾಯ-ಕರೂ
ನಾಯ-ಕರ್ಗ್ಗೆ
ನಾಯ-ಕ-ವೃತ್ತಿಗೂ
ನಾಯ-ಕ-ವೃತ್ತಿಗೆ
ನಾಯ-ಕ-ಹೆಗ್ಗಡೆ
ನಾಯ-ಕಿತ್ತಿ
ನಾಯ-ಕಿತ್ತಿಗೆ
ನಾಯ-ಕಿತ್ತಿಯ
ನಾಯ-ಕಿತ್ತಿ-ಯನ್ನು
ನಾಯ-ಕಿತ್ತಿ-ಯರ
ನಾಯ-ಕಿತ್ತಿ-ಯ-ರನ್ನು
ನಾಯ-ಕಿತ್ತಿ-ಯರು
ನಾಯ-ಕಿತ್ತಿಯು
ನಾಯ-ಗಲ್ತೆ-ವರ
ನಾಯ-ನಾ-ರರ
ನಾಯ-ನಾರ್
ನಾಯನ್ಮಾರ
ನಾಯ-ಮಾಂಸ
ನಾಯಿ
ನಾಯಿಂದರ
ನಾಯಿಂದ-ರಿಗೆ
ನಾಯಿಂದರು
ನಾಯಿಂದ-ರು-ಗ-ಳಿಗೆ
ನಾಯಿಂದ-ರು-ನಾ-ವಿದ-ರು-ವಾಲಗ-ದ-ವರು
ನಾಯಿ-ಗಳ
ನಾಯಿ-ಗ-ಳನ್ನು
ನಾಯಿ-ಗಾಗಿ
ನಾಯಿಡು
ನಾಯಿ-ನ-ವನ-ಯಮ್
ನಾಯಿ-ಯನ್ನು
ನಾಯಿ-ಯನ್ನೂ
ನಾಯಿ-ಯನ್ನೇ
ನಾಯಿ-ಹಳ್ಳ
ನಾಯುಕ
ನಾಯ್ಕ
ನಾಯ್ಕ-ರ-ಸನು
ನಾಯ್ಕ-ರಸ-ರನ
ನಾಯ್ಕ-ರು-ಗಳು
ನಾರಣ-ದೇವನ
ನಾರಣ-ದೇವಯ್ಯ
ನಾರಣ-ದೇವರು
ನಾರಣ-ದೇವಿ
ನಾರಣ-ವೆಗ್ಗಡೆ
ನಾರಣ-ವೆಗ್ಗಡೆ-ಯಾಗಿ-ರು-ವಂತೆ
ನಾರಣ-ವೆಗ್ಗಡೆಯು
ನಾರಣ-ವೆಗ್ಗಡೆಯೇ
ನಾರಣ-ವೆರ್ಗಡೆ
ನಾರಣ-ವೆರ್ಗ್ಗಡೆ
ನಾರಣ-ವೆರ್ಗ್ಗಡೆ-ಯಯು
ನಾರಣಾಂಕ-ವಿದು
ನಾರ-ಣಾ-ಚಾರಿ
ನಾರಪ್ಪ-ರಾಜ
ನಾರಪ್ಪ-ರಾಜಯ್ಯನ
ನಾರಪ್ಪ-ರಾಜಯ್ಯನು
ನಾರಯ-ದೇವ
ನಾರಯ-ದೇವನು
ನಾರಯ್ಯ-ದೇವ
ನಾರಯ್ಯ-ದೇವನ
ನಾರ-ಸಿಂಗ
ನಾರ-ಸಿಂಗ-ಚತುರ್ವೇದಿ
ನಾರ-ಸಿಂಗಣ
ನಾರ-ಸಿಂಗ-ದೇವನ
ನಾರ-ಸಿಂಗ-ದೇವನು
ನಾರ-ಸಿಂಗ-ದೇವರು
ನಾರ-ಸಿಂಗನು
ನಾರ-ಸಿಂಗಯ್ಯ-ದೇವನ
ನಾರ-ಸಿಂಘ
ನಾರ-ಸಿಂಘ-ದೇವನ
ನಾರ-ಸಿಂಘ-ದೇವನು
ನಾರ-ಸಿಂಘ-ಯ-ದೇವ
ನಾರ-ಸಿಂಹ
ನಾರ-ಸಿಂಹ-ಚತುರ್ವೇದಿ
ನಾರ-ಸಿಂಹ-ಚತುರ್ವೇದಿ-ಮಂಗಲದ
ನಾರ-ಸಿಂಹ-ದೇವ
ನಾರ-ಸಿಂಹ-ದೇವನ
ನಾರ-ಸಿಂಹ-ದೇವ-ನಿಗೆ
ನಾರ-ಸಿಂಹ-ದೇವರ
ನಾರ-ಸಿಂಹ-ದೇವ-ರಸ
ನಾರ-ಸಿಂಹ-ದೇವ-ರ-ಸನು
ನಾರ-ಸಿಂಹ-ದೇವ-ರ-ಸ-ನೊಡನೆ
ನಾರ-ಸಿಂಹ-ದೇವ-ರ-ಸರ
ನಾರ-ಸಿಂಹ-ದೇವ-ರ-ಸರು
ನಾರ-ಸಿಂಹ-ದೇವ-ರಿಗೆ
ನಾರ-ಸಿಂಹ-ದೇವರು
ನಾರ-ಸಿಂಹ-ದೇವ-ರೆಂದೂ
ನಾರ-ಸಿಂಹನ
ನಾರ-ಸಿಂಹ-ನನ್ನು
ನಾರ-ಸಿಂಹ-ನಿಗೂ
ನಾರ-ಸಿಂಹ-ನಿಗೆ
ನಾರ-ಸಿಂಹನು
ನಾರ-ಸಿಂಹ-ಪಟ್ಟ-ಣದ
ನಾರ-ಸಿಂಹ-ಪಟ್ಟ-ಣ-ವಾ-ಗಿತ್ತು
ನಾರ-ಸಿಂಹ-ಪುರ-ವೆಂದು
ನಾರ-ಸಿಂಹ-ರಸ
ನಾರ-ಸಿಂಹ-ರಾಯನ
ನಾರ-ಸಿಂಹಸ್ವಾಮಿಗೆ
ನಾರಾಯಣ
ನಾರಾಯಣಃ
ನಾರಾಯ-ಣ-ಗಿರಿ
ನಾರಾಯ-ಣ-ಗಿರಿ-ದುರ್ಗ
ನಾರಾಯ-ಣ-ಗಿರಿ-ದುರ್ಗದ
ನಾರಾಯ-ಣ-ಗಿರೌ
ನಾರಾಯ-ಣದ
ನಾರಾಯ-ಣ-ದೇವರ
ನಾರಾಯ-ಣ-ದೇ-ವ-ರಿಗೆ
ನಾರಾಯ-ಣ-ದೇವರು
ನಾರಾಯ-ಣ-ದೇ-ವ-ರೆಂದು
ನಾರಾಯ-ಣ-ದೇವಾ-ಲಯ
ನಾರಾಯ-ಣ-ದೇವಾ-ಲಯ-ದಲ್ಲಿ-ರುವ
ನಾರಾಯ-ಣನ
ನಾರಾಯ-ಣ-ನಿಗೆ
ನಾರಾಯ-ಣನು
ನಾರಾಯ-ಣ-ನೆಂದು
ನಾರಾಯ-ಣ-ನೆಂದೂ
ನಾರಾಯ-ಣನ್
ನಾರಾಯ-ಣ-ಪರ್ವ-ತ-ವಪ್ಪ
ನಾರಾಯ-ಣ-ಪಾದ-ಪಙ್ಕಜ-ಯುಗೀ-ವನ್ಯಸ್ತವಿಪ್ಪಗ್ಭರಃ
ನಾರಾಯ-ಣ-ಪುರ
ನಾರಾಯ-ಣ-ಪುರ-ವನ್ನು
ನಾರಾಯ-ಣ-ಪೆರು-ಮಾಳ್
ನಾರಾಯ-ಣ-ಭಟ್ಟನ
ನಾರಾಯ-ಣಯ್ಯ
ನಾರಾಯ-ಣಯ್ಯನ
ನಾರಾಯ-ಣ-ರಾಯ-ರೆಂಬು-ವ-ವರ
ನಾರಾಯ-ಣ-ಶೈಲಕ್ಕೆ
ನಾರಾಯ-ಣ-ಸ-ಮುದ್ರ
ನಾರಾಯ-ಣಸ್ಯ
ನಾರಾಯ-ಣಸ್ವ-ರೂಪ-ರಾಗಿದ್ದರು
ನಾರಾಯ-ಣಸ್ವಾಮಿ
ನಾರಾಯ-ಣಸ್ವಾಮಿಯ
ನಾರಾಯ-ಣಸ್ವಾಮಿ-ಯರ
ನಾರಾಯ-ಣಾಂಬಿ-ಕೆಯರ
ನಾರಾಯ-ಣಾ-ಚಾರಿ
ನಾರಾಯಣೀ
ನಾರಿ
ನಾರಿಗೆ
ನಾರಿಯ
ನಾರಿ-ಯಪ್ಪ
ನಾರಿ-ವಾಳ-ವನು
ನಾರ್ತ್ಬ್ಯಾಂಕ್
ನಾರ್ತ್ಬ್ಯಾಂಕ್ನಲ್ಲಿ
ನಾಲಯ್ಯಗೆ
ನಾಲಯ್ಯ-ನಿಗೆ
ನಾಲೂರಿನ
ನಾಲೂರು
ನಾಲೆ
ನಾಲೆ-ಗ-ಳನ್ನು
ನಾಲೆಯ
ನಾಲೆ-ಯಿಂದ
ನಾಲೆಯು
ನಾಲ್ಕಂಡುಗಂ
ನಾಲ್ಕಕೆ
ನಾಲ್ಕನೆ
ನಾಲ್ಕ-ನೆಯ
ನಾಲ್ಕನೆ-ಯ-ವನು
ನಾಲ್ಕನೇ
ನಾಲ್ಕರ-ಲೊಂದು
ನಾಲ್ಕ-ರಲ್ಲಿ
ನಾಲ್ಕಾರು
ನಾಲ್ಕು
ನಾಲ್ಕು-ಗದ್ಯಾಣ-ವನ್ನು
ನಾಲ್ಕು-ಜನ
ನಾಲ್ಕು-ಪಡಿ
ನಾಲ್ಕು-ಪಣ
ನಾಲ್ಕು-ಪಾ-ವಿಗೆ
ನಾಲ್ಕು-ಬಾರಿ
ನಾಲ್ಕು-ವೃತ್ತಿ-ಗಳ
ನಾಲ್ಕು-ವೃತ್ತಿ-ಗ-ಳಿಗೆ
ನಾಲ್ಕು-ಹಣ-ವನ್ನು
ನಾಲ್ಕೂ
ನಾಲ್ಕೂ-ವರೆ
ನಾಲ್ದೆಸೆಗೆ
ನಾಲ್ದೆಸೆ-ಯಲ್ಲೂ
ನಾಲ್ಮಡಿ
ನಾಲ್ವತ್ತ-ರೊಳಗೆ
ನಾಲ್ವತ್ತು
ನಾಲ್ವತ್ತೊಕ್ಕಲು
ನಾಲ್ವ-ದಿಮ್ಬರು
ನಾಲ್ವರ
ನಾಲ್ವರು
ನಾಲ್ವರೂ
ನಾಳನ-ಕೆರೆ
ನಾಳಾ-ಪದೆ
ನಾಳೆ-ಯಿಲ್ಲೆಂದ
ನಾಳ್ಗಾವುಂಡರು
ನಾವಳಿಗಂ
ನಾವಿಂದರ
ನಾವಿಂದ-ರಿಗೆ
ನಾವಿದ
ನಾವಿದ್ದೇವೆ
ನಾವು
ನಾವೆ-ಲರು
ನಾವೇ
ನಾಶ
ನಾಶ-ಕೃತ್ಯ-ವನ್ನು
ನಾಶ-ಗೊಳಿ-ಸಿ-ದನು
ನಾಶ-ಪಡಿ-ಸಿ-ತೆಂದು
ನಾಶ-ಪಡಿ-ಸಿದ-ನೆಂದು
ನಾಶ-ಮಾಡಿ
ನಾಶ-ಮಾಡಿ-ದವ-ರೆಂಬುದು
ನಾಶ-ವಾಗಲು
ನಾಶ-ವಾಗಿ
ನಾಶ-ವಾಗಿದೆ
ನಾಶ-ವಾ-ಗಿದ್ದು
ನಾಶ-ವಾಗಿ-ರ-ಬ-ಹುದು
ನಾಶ-ವಾಗಿ-ರುವ
ನಾಶ-ವಾಗಿ-ರುವು-ದರ
ನಾಶ-ವಾಗಿವೆ
ನಾಸಿ
ನಾಸಿರ್ಜಂಗ್
ನಾಸಿರ್ಜಂಗ್ನ
ನಾೞ್ಪ್ರಭು
ನಿಂತಂತೆ
ನಿಂತರು
ನಿಂತ-ರು-ಎಂದು
ನಿಂತರೆ
ನಿಂತವು
ನಿಂತಾಗ
ನಿಂತಿದೆ
ನಿಂತಿದ್ದಾನೆ
ನಿಂತಿದ್ದಾರೆ
ನಿಂತಿರ-ಬ-ಹುದು
ನಿಂತಿ-ರುವ
ನಿಂತಿ-ರುವುದು
ನಿಂತಿವೆ
ನಿಂತು
ನಿಂತು-ಕೊಂಡು
ನಿಂತುವೇ
ನಿಂತು-ಹೋಗಿದೆ
ನಿಂತು-ಹೋಗಿದ್ದು
ನಿಂತು-ಹೋಗಿ-ರಲು
ನಿಂದ
ನಿಂದ-ರಿನ್ರಿಪಾ-ಳರ
ನಿಂದಲೇ
ನಿಂನಂತಾರೊಳ-ಪೊಕ್ಕು
ನಿಂನಂತಾರೊಳವೊಕ್ಕು
ನಿಂಬಕಾ-ಚಾರ್ಯನು
ನಿಂಬಾ-ಪುರದ
ನಿಕಟ-ವಾಗಿ-ರ-ಬಹು-ದೆಂದು
ನಿಕರು
ನಿಕಾಮಾತ್ತ
ನಿಕೇಶ್ವರದ
ನಿಕೇಶ್ವರ-ನಿಕ್ಕೀಶ್ವರ
ನಿಕ್ಕ-ರಸ
ನಿಕ್ಕ-ರಸನ
ನಿಕ್ಕ-ರಸರ
ನಿಕ್ಕ-ರಸರ್
ನಿಕ್ಕಿ-ಯಣ್ಣ
ನಿಕ್ಕಿ-ಯಣ್ಣನ
ನಿಕ್ಕಿ-ಯ-ರಸ
ನಿಕ್ಕಿ-ಯ-ರ-ಸನ
ನಿಕ್ಕಿ-ರಸ-ನಿಕ್ಕ-ರಸ
ನಿಕ್ಕೀಶ್ವರ
ನಿಕ್ಕೇಶ್ವರ
ನಿಕ್ಷೇ-ಪದ
ನಿಖರ
ನಿಖರ-ವಾಗಿ
ನಿಖಿಳ
ನಿಖಿಳ-ಲಕ್ಷ್ಮೀ
ನಿಖಿಳಾಂ
ನಿಗದಿ
ನಿಗದಿತ
ನಿಗದಿ-ಪಡಿ-ಸ-ಲಾ-ಗಿತ್ತು
ನಿಗದಿ-ಪಡಿ-ಸ-ಲಾಗಿದೆ
ನಿಗದಿ-ಪಡಿ-ಸ-ಲಾಗುತ್ತಿತ್ತು
ನಿಗದಿ-ಪಡಿ-ಸಲು
ನಿಗದಿ-ಪ-ಡಿಸಿ
ನಿಗದಿ-ಪಡಿ-ಸಿ-ಕೊಂಡಿದ್ದ
ನಿಗದಿ-ಪಡಿ-ಸಿದ
ನಿಗದಿ-ಪಡಿ-ಸಿದೆ
ನಿಗದಿ-ಪಡಿ-ಸಿದ್ದ
ನಿಗದಿ-ಪಡಿ-ಸಿದ್ದನ್ನು
ನಿಗದಿ-ಪಡಿ-ಸಿದ್ದಾರೆ
ನಿಗದಿ-ಪಡಿ-ಸಿ-ರುವುದು
ನಿಗದಿ-ಪಡಿಸು
ನಿಗದಿ-ಪಡಿ-ಸುತ್ತಾನೆ
ನಿಗದಿ-ಪಡಿ-ಸುತ್ತಿದ್ದು-ದ-ರಿಂದ
ನಿಗದಿ-ಪಡಿ-ಸುವ
ನಿಗದಿ-ಯಾಗಿ
ನಿಗದಿ-ಯಾ-ಗಿದ್ದ
ನಿಗದಿ-ಯಾಗಿ-ರುತ್ತಿತು
ನಿಗ-ಮಾರ್ಥ-ಚರಿತೆ
ನಿಗೆ
ನಿಗ್ರಹಿ-ಸಿ-ದನು
ನಿಜ
ನಿಜ-ಕಳತ್ರ
ನಿಜ-ಕೀರ್ತಿಯಂ
ನಿಜಕ್ಕೂ
ನಿಜ-ಗುರು-ಗ-ಳಾದ
ನಿಜ-ಗುರು-ವಾ-ಗಿದ್ದ
ನಿಜ-ನಾಮ-ವಾಗಿ-ರು-ವು-ದಿಲ್ಲ-ವೆಂದು
ನಿಜ-ಪತಿ-ಯ-ಭಿ-ವೃದ್ಧಿಗೆ
ನಿಜ-ಪಾ-ವನ
ನಿಜಪ್ರತಾಪ-ದಿಂದ
ನಿಜಪ್ರತಾಪಾದಧಿಗತ್ಯ
ನಿಜಪ್ರಧಾನ
ನಿಜಭ್ರಾತ್ತೃನಿಹಿತ
ನಿಜ-ಮನಪ್ರಿಯೆ
ನಿಜ-ರಾಜ-ಧಾನಿ
ನಿಜ-ರಾಧಾ-ನಿಮಧಿವ-ಸನ್
ನಿಜ-ರಾಮ
ನಿಜ-ವಂಶ-ಜರ್ಗ್ಗಮಿದು
ನಿಜ-ವಾದ
ನಿಜ-ವಾದು-ದೆಂದು
ನಿಜ-ವಿಜಯ
ನಿಜವೇ
ನಿಜ-ಸ-ಮಯ
ನಿಜಸ್ವ-ರೂಪ
ನಿಜಸ್ವಾಮಿ
ನಿಜಾಂ
ನಿಜಾಂಶಂ
ನಿಜಾಜ್ಞೆವೆ-ರಸಾದ-ರದಿಂ
ನಿಜಾ-ಮರು
ನಿಜಾಮ-ರೊಂದಿಗೆ
ನಿಟ್ಟೂರು
ನಿಡ-ದವೋಲು
ನಿಡಿದು
ನಿಡು-ಗಲ್ಲಿನ
ನಿಡುವುಟೆ-ಯನ್ನು
ನಿಡುವುಟೆಯು
ನಿಡು-ವೆತ್ತ-ಕೆರೆ-ಗ-ಳನ್ನು
ನಿಡು-ವೊಳಲಾಗಿರ-ಬ-ಹುದು
ನಿತ್ತರಿಸಲಾ-ರದೆ
ನಿತ್ಯ
ನಿತ್ಯಂ
ನಿತ್ಯ-ಕಟ್ಟ-ಣೆ-ಯಾಗಿ
ನಿತ್ಯ-ಕಟ್ಟಳೆ
ನಿತ್ಯ-ಕಟ್ಟ-ಳೆಗೆ
ನಿತ್ಯ-ಕಟ್ಟ-ಳೆ-ಯಾಗಿ
ನಿತ್ಯ-ಕೃತ್ಯ
ನಿತ್ಯ-ಕೃತ್ಯ-ಗ-ಳಿಗೆ
ನಿತ್ಯಕ್ರತು
ನಿತ್ಯ-ಗಟ್ಟಲೆ
ನಿತ್ಯ-ಗಟ್ಟ-ಳೆಗೆ
ನಿತ್ಯ-ದಲ್ಲಿ
ನಿತ್ಯ-ದಲ್ಲೂ
ನಿತ್ಯ-ಪಡಿ
ನಿತ್ಯ-ಪಡಿ-ತರ-ದೀಪಾರಾ-ಧನೆ
ನಿತ್ಯಪ್ರ-ವಾಸಿ
ನಿತ್ಯವೂ
ನಿತ್ಯ-ಸಂದಿಗೆ
ನಿತ್ಯ-ಸಂಧಿ
ನಿತ್ಯ-ಸಂಧಿಗೆ
ನಿತ್ಯ-ಸೇವೆಗೆ
ನಿತ್ಯಾತ್ಮ-ಶುಕ-ಯೋಗಿ
ನಿತ್ಯೋತ್ಸವ
ನಿತ್ಯೋತ್ಸಾಹ-ಗ-ಳಿಗೆ
ನಿದರ್ಶನ
ನಿದರ್ಶನ-ಗ-ಳನ್ನು
ನಿದರ್ಶನ-ವನ್ನು
ನಿದರ್ಶನ-ವಾಗಿದೆ
ನಿದಾ-ನಾರ್ಥಂ
ನಿಧನ-ನಾಗಿದ್ದ-ನೆಂದು
ನಿಧನ-ರಾದ-ರೆಂದು
ನಿಧನ-ಳಾದ
ನಿಧನ-ಳಾದ-ಳೆಂದು
ನಿಧನ-ವನ್ನು
ನಿಧನ-ಹೊಂದಿದ್ದು
ನಿಧನಾ-ನಂತರ
ನಿಧಾನಃ
ನಿಧಾನ-ವಾಗಿ
ನಿಧಿ-ನಿಕ್ಷೇಪ-ಸಹಿತ-ವಾಗಿ
ನಿಧಿಯ
ನಿನಗೆಂದು
ನಿನಗೇನು
ನಿನ್ನ
ನಿಪುಣ
ನಿಪುಣರು
ನಿಪ್ಪ
ನಿಬಂಧ
ನಿಬಂಧಂಗಳು
ನಿಬಂಧ-ಕಾರ-ಅರ
ನಿಬಂಧ-ಕಾರರ
ನಿಬಂಧ-ಕಾರ-ರಿಗೆ
ನಿಬಂಧ-ಕಾರರು
ನಿಬಂಧ-ಗಳು
ನಿಬಂಧದ
ನಿಬಂಧಧಿ
ನಿಬಂಧ-ವೆಂದೂ
ನಿಬಂಧಿ
ನಿಬಂಧಿಧ
ನಿಬಂಧಿ-ಯನೂ
ನಿಬಂಧಿ-ಯನ್ನು
ನಿಬಂಧಿ-ಯಾಗಿ
ನಿಬಿಡಂ
ನಿಮಗೆ
ನಿಮನ್ದ
ನಿಮಾಣ-ವಾಗಿತ್ತೆಂದು
ನಿಮಾನ
ನಿಮಿತ್ತ
ನಿಮಿತ್ತ-ದೊಳಾ-ಯದ
ನಿಮಿತ್ತ-ವಾಗಿ
ನಿಮಿತ್ತಾಯ
ನಿಮಿಸುತ್ತಿದ್ದರು
ನಿಮ್ಮ
ನಿಮ್ಮ-ಡಿ-ಯನ್ನು
ನಿಯ-ಗಾ-ಮುಂಡನ
ನಿಯತ-ಕಾಲಿ-ಕ-ಗ-ಳಲ್ಲಿ
ನಿಯಮ
ನಿಯಮ-ಗಳ
ನಿಯಮ-ದಂತೆ
ನಿಯಮದಿಂ
ನಿಯಮ-ನಪ್ಪಡಿ
ನಿಯಮ-ವನ್ನು
ನಿಯಮಾ-ವಳಿ-ಗ-ಳಲ್ಲಿ
ನಿಯಮಿ-ಸಿದ್ದರು
ನಿಯಾ-ಮದಿಂ
ನಿಯಾ-ಮದಿಂದ-ಶಾ-ಸನವ
ನಿಯಾಮ್ಯ-ಗ-ಳೆಂದು
ನಿಯುಕ್ತ-ರಾದರೆ
ನಿಯುಕ್ತ-ಳಾದಳು
ನಿಯುಕ್ತಿ-ಗೊಂಡ
ನಿಯುಕ್ತೆ
ನಿಯೋಗ
ನಿಯೋಗ-ಅಧಿ-ಕಾರಿ-ಗಳು
ನಿಯೋಗ-ಗಳ
ನಿಯೋಗ-ಗ-ಳಿಗೆ
ನಿಯೋಗ-ದಂತೆ
ನಿಯೋಗದಿಂ
ನಿಯೋಗ-ದಿಂದ
ನಿಯೋಗ-ದುರಂಧರ
ನಿಯೋಗ-ನ-ವನು
ನಿಯೋಗ-ವನ್ನು
ನಿಯೋಗ-ವೆಂದೂ
ನಿಯೋಗಾ-ಧಿ-ಪತಿ
ನಿಯೋಗಾ-ಧಿ-ಪತಿ-ಗಳ
ನಿಯೋಗಾ-ಧಿ-ಪತಿ-ಗಳು
ನಿಯೋಗಾ-ಧಿ-ಪತಿ-ಗಳೂ
ನಿಯೋಗಿ
ನಿರ
ನಿರಂತರ
ನಿರಂತರಂ
ನಿರಂತರ-ವಾಗಿ
ನಿರಂತರ-ವೆನ್ನಲು
ನಿರ-ತ-ನಾದ-ನೆಂದು
ನಿರ-ತ-ರಾದ
ನಿರ-ನಾಗಿದ್ದ-ನೆಂದು
ನಿರ-ವದ್ಯ
ನಿರ-ವಧಿಕ
ನಿರಾ-ಕುಳ-ದಿಂದ
ನಿರಾಡಂಬರ
ನಿರಾತಂಕ-ವಾಗಿ
ನಿರಾನ್ವಿಷ್ಣು
ನಿರಿ-ಸಿದರ್ಸ್ಸೋ-ಮಾನ್ವಯೋರ್ವ್ವೀಶ್ವರರ್
ನಿರಿ-ಸಿದಸ್ಸೋ-ಮಾನ್ವ-ಯಯೋರ್ವ್ವೀಶ್ವರರ್
ನಿರೀಕ್ಷಿಸುತ್ತಿದ್ದ
ನಿರೀಕ್ಷೆ
ನಿರು-ಪಾದಿಕ
ನಿರುಪಾಧಿಕ
ನಿರುಪಾಯ-ನಾಗಿ
ನಿರೂಪ
ನಿರೂಪ-ಗ-ಳನ್ನು
ನಿರೂಪಣೆ
ನಿರೂಪ-ಣೆ-ಗ-ಳನ್ನು
ನಿರೂಪದ
ನಿರೂಪ-ದಂತೆ
ನಿರೂಪ-ದಲಿ
ನಿರೂಪ-ದಿಂದ
ನಿರೂಪ-ವನ್ನು
ನಿರೂಪ-ವಿಡಿದು
ನಿರೂಪ-ಶಾ-ಸನ-ದಲ್ಲಿ
ನಿರೂಪಿತ
ನಿರೂಪಿತ-ವಾ-ಗಿದ್ದು
ನಿರೂಪಿಸ-ಲಾಗಿದೆ
ನಿರೂಪಿಸ-ಲಾ-ಗಿದ್ದು
ನಿರೂಪಿಸಿ
ನಿರೂಪಿಸಿದ್ದಾರೆ
ನಿರೂಪಿ-ಸುತ್ತವೆ
ನಿರೂಪಿ-ಸುವ
ನಿರ್ಗ-ಮನದ
ನಿರ್ಗ್ಗುಂದ
ನಿರ್ಜಿತ
ನಿರ್ಜಿತ್ಯ
ನಿರ್ಣಯ
ನಿರ್ಣಯ-ವನ್ನು
ನಿರ್ಣ-ಯವು
ನಿರ್ಣಯಿಸು-ವಂತೆಯೂ
ನಿರ್ದಿಷ್ಟಪ್ರ-ಮಾ-ಣದ
ನಿರ್ದೇಶ-ನಾ-ಲ-ಯವು
ನಿರ್ದೇಶಿ-ಸಲು
ನಿರ್ಧರಿತ-ವಾಗುತ್ತಿದ್ದವು
ನಿರ್ಧರಿಸಿ-ದನು
ನಿರ್ಧ-ರಿ-ಸಿದೆ
ನಿರ್ಧರಿಸಿದ್ದಾರೆ
ನಿರ್ಧರಿ-ಸುತ್ತಿ-ರ-ಬ-ಹುದು
ನಿರ್ನ್ನಯಿ-ಸುವರು
ನಿರ್ಬಾಣಿ-ದೇವರು
ನಿರ್ಬ್ಬಾಣಿ-ದೇವರು
ನಿರ್ಮತ್ಸರ
ನಿರ್ಮಾಣ
ನಿರ್ಮಾಣ-ಕಾರ್ಯ-ಗ-ಳನ್ನು
ನಿರ್ಮಾಣಕ್ಕೆ
ನಿರ್ಮಾಣ-ಗ-ಳನ್ನು
ನಿರ್ಮಾಣ-ಗಳು
ನಿರ್ಮಾಣ-ಗೊಂಡಂಥಾ
ನಿರ್ಮಾಣ-ಗೊಂಡು
ನಿರ್ಮಾಣದ
ನಿರ್ಮಾಣ-ದಂತಹ
ನಿರ್ಮಾಣ-ದತ್ತಿ
ನಿರ್ಮಾಣ-ದಲ್ಲಿ
ನಿರ್ಮಾಣ-ದಲ್ಲಿಯೂ
ನಿರ್ಮಾಣ-ವನ್ನು
ನಿರ್ಮಾಣ-ವಾಗಿ
ನಿರ್ಮಾಣ-ವಾ-ಗಿತ್ತು
ನಿರ್ಮಾಣ-ವಾಗಿತ್ತೆಂದು
ನಿರ್ಮಾಣ-ವಾಗಿದೆ
ನಿರ್ಮಾಣ-ವಾಗಿದ್ದರೂ
ನಿರ್ಮಾಣ-ವಾ-ಗಿದ್ದು
ನಿರ್ಮಾಣ-ವಾಗಿರ
ನಿರ್ಮಾಣ-ವಾಗಿ-ರ-ಬ-ಹುದು
ನಿರ್ಮಾಣ-ವಾಗಿ-ರ-ಬಹು-ದೆಂದು
ನಿರ್ಮಾಣ-ವಾಗಿ-ರ-ಬಹು-ದೆಂಬ
ನಿರ್ಮಾಣ-ವಾಗಿ-ರುವ
ನಿರ್ಮಾಣ-ವಾಗಿ-ರುವುದು
ನಿರ್ಮಾಣ-ವಾಗಿಲ್ಲ-ವೆಂದು
ನಿರ್ಮಾಣ-ವಾಗಿವೆ
ನಿರ್ಮಾಣ-ವಾದ
ನಿರ್ಮಾಣ-ವಾಯಿತು
ನಿರ್ಮಾಣ-ವಾಯಿ-ತೆಂದು
ನಿರ್ಮಾಣವೂ
ನಿರ್ಮಾಪಕ-ನೆನಿಸಿದ್ದಾನೆ
ನಿರ್ಮಿತ
ನಿರ್ಮಿತ-ಮಪ
ನಿರ್ಮಿತ-ವಾಗಿದೆ
ನಿರ್ಮಿತ-ವಾ-ಗಿದ್ದ
ನಿರ್ಮಿತ-ವಾ-ಗಿದ್ದು
ನಿರ್ಮಿತ-ವಾಗಿ-ಬಹು-ದಾದ
ನಿರ್ಮಿತ-ವಾಗಿ-ರ-ಬ-ಹುದು
ನಿರ್ಮಿತ-ವಾಗಿ-ರ-ಬಹು-ದೆಂದು
ನಿರ್ಮಿತ-ವಾಗಿ-ರವು
ನಿರ್ಮಿತ-ವಾಗಿ-ರುವ
ನಿರ್ಮಿತ-ವಾಗಿವೆ
ನಿರ್ಮಿತ-ವಾದ
ನಿರ್ಮಿತ-ವಾದವು
ನಿರ್ಮಿತ-ವಾಯಿ
ನಿರ್ಮಿ-ಸ-ಲಾ-ಗಿತ್ತು
ನಿರ್ಮಿಸ-ಲಾಗಿದೆ
ನಿರ್ಮಿಸ-ಲಾ-ಗಿದ್ದು
ನಿರ್ಮಿಸ-ಲಾಗುತ್ತಿತ್ತು
ನಿರ್ಮಿ-ಸ-ಲಾ-ಯಿತು
ನಿರ್ಮಿಸಲಾಯಿ-ತೆಂದು
ನಿರ್ಮಿ-ಸಲು
ನಿರ್ಮಿಸಲ್ಪಟ್ಟವು
ನಿರ್ಮಿಸಲ್ಪಟ್ಟಿದೆ
ನಿರ್ಮಿಸಿ
ನಿರ್ಮಿಸಿ-ಅ-ದಕ್ಕೆ
ನಿರ್ಮಿಸಿ-ಕೊಟ್ಟ-ರೆಂದು
ನಿರ್ಮಿಸಿ-ಕೊಟ್ಟಿರ-ಬ-ಹುದು
ನಿರ್ಮಿಸಿ-ಕೊಟ್ಟಿರ-ಬಹು-ದೆಂದು
ನಿರ್ಮಿಸಿ-ಕೊಡುತ್ತಾನೆ
ನಿರ್ಮಿಸಿ-ಕೊಳ್ಳು-ವಂತಿರ
ನಿರ್ಮಿ-ಸಿದ
ನಿರ್ಮಿಸಿ-ದಂತೆ
ನಿರ್ಮಿಸಿ-ದ-ನಷ್ಟೆ
ನಿರ್ಮಿಸಿ-ದನು
ನಿರ್ಮಿಸಿ-ದ-ನೆಂದು
ನಿರ್ಮಿಸಿ-ದರು
ನಿರ್ಮಿಸಿ-ದ-ರೆಂದು
ನಿರ್ಮಿಸಿ-ದಳು
ನಿರ್ಮಿಸಿ-ದ-ವ-ರನ್ನು
ನಿರ್ಮಿಸಿ-ದ-ವರು
ನಿರ್ಮಿಸಿ-ದಾಗ
ನಿರ್ಮಿ-ಸಿದೆ
ನಿರ್ಮಿ-ಸಿದ್ದ
ನಿರ್ಮಿಸಿದ್ದಕ್ಕೆ
ನಿರ್ಮಿಸಿದ್ದ-ನೆಂದೂ
ನಿರ್ಮಿಸಿದ್ದರು
ನಿರ್ಮಿಸಿದ್ದಾನೆ
ನಿರ್ಮಿಸಿದ್ದಾ-ನೆಂದು
ನಿರ್ಮಿಸಿದ್ದಾರೆ
ನಿರ್ಮಿಸಿದ್ದಾರೋ
ನಿರ್ಮಿಸಿದ್ದಾಳೆ
ನಿರ್ಮಿ-ಸಿರ
ನಿರ್ಮಿಸಿ-ರ-ಬಹದು
ನಿರ್ಮಿಸಿ-ರ-ಬ-ಹುದು
ನಿರ್ಮಿಸಿ-ರ-ಬಹು-ದೆಂದು
ನಿರ್ಮಿಸಿ-ರುತ್ತಾನೆ
ನಿರ್ಮಿಸಿ-ರುವ
ನಿರ್ಮಿಸಿ-ರು-ವಂತೆ
ನಿರ್ಮಿಸಿ-ರು-ವು-ದನ್ನು
ನಿರ್ಮಿಸಿ-ರುವು-ದ-ರಿಂದ
ನಿರ್ಮಿಸಿ-ರು-ವು-ದಾಗಿ
ನಿರ್ಮಿಸಿ-ರುವುದು
ನಿರ್ಮಿಸು
ನಿರ್ಮಿಸುತ್ತಾನೆ
ನಿರ್ಮಿಸುತ್ತಾರೆ
ನಿರ್ಮಿಸುತ್ತಾಳೆ
ನಿರ್ಮಿಸುತ್ತಿದ್ದ
ನಿರ್ಮಿಸುತ್ತಿದ್ದರು
ನಿರ್ಮಿಸುತ್ತಿದ್ದು
ನಿರ್ಮಿಸುತ್ತಿದ್ದುದು
ನಿರ್ಮಿಸುತ್ತಿ-ರುವುದು
ನಿರ್ಮಿ-ಸುವ
ನಿರ್ಮಿಸು-ವಂತೆಯೂ
ನಿರ್ಮಿಸು-ವ-ವ-ರಿಗೆ
ನಿರ್ಮಿಸು-ವು-ದರ
ನಿರ್ಮಿಸು-ವುದ-ರಲ್ಲಿ
ನಿರ್ಮಿಸು-ವುದು
ನಿರ್ಮೂಲ
ನಿರ್ಮೂಲನ
ನಿರ್ಮೂಲನೆ
ನಿರ್ಮೂ-ಳನ
ನಿರ್ಮ್ಮಡಿ-ಯನ್ನು
ನಿರ್ಮ್ಮಿಸಿ
ನಿರ್ಮ್ಮಿಸಿ-ದಂತಿರಂಜಿತಂ
ನಿರ್ವಹಣಾ
ನಿರ್ವ-ಹಣೆ
ನಿರ್ವ-ಹಣೆ-ಗಾಗಿ
ನಿರ್ವ-ಹಣೆಗೆ
ನಿರ್ವ-ಹಣೆ-ಚಾರಿತ್ರಿಕ
ನಿರ್ವ-ಹಣೆಯ
ನಿರ್ವ-ಹಣೆ-ಯನ್ನು
ನಿರ್ವ-ಹಣೆಯು
ನಿರ್ವಹಿ-ಸಲು
ನಿರ್ವಹಿಸಿ-ಕೊಂಡು
ನಿರ್ವಹಿಸಿದ
ನಿರ್ವಹಿಸು
ನಿರ್ವಹಿಸುತ್ತಿದ್ದರು
ನಿರ್ವಹಿಸುತ್ತಿದ್ದ-ರೆಂದು
ನಿರ್ವಹಿಸುತ್ತಿದ್ದು-ದನ್ನು
ನಿರ್ವಹಿಸುತ್ತಿದ್ದುದು
ನಿರ್ವಹಿ-ಸುವ
ನಿರ್ವಹಿಸು-ವ-ವರು
ನಿರ್ವಾಹಕ-ರಾದ
ನಿಲಮ್
ನಿಲವಂದ-ದೇವ-ರಿಗೆ
ನಿಲಿಸಿದ
ನಿಲಿಸಿದಂ
ನಿಲಿಸಿ-ದಕ್ಕೆ
ನಿಲಿಸಿ-ದನು
ನಿಲಿಸಿ-ಳೆಯಂ
ನಿಲಿ-ಸುವರು
ನಿಲೆವು
ನಿಲ್ಲಿ-ಸ-ದರು
ನಿಲ್ಲಿ-ಸದೆ
ನಿಲ್ಲಿ-ಸಲು
ನಿಲ್ಲಿಸಿ
ನಿಲ್ಲಿ-ಸಿದ
ನಿಲ್ಲಿ-ಸಿ-ದಂತೆ
ನಿಲ್ಲಿ-ಸಿ-ದನು
ನಿಲ್ಲಿ-ಸಿ-ದ-ನೆಂಬುದು
ನಿಲ್ಲಿ-ಸಿ-ದರು
ನಿಲ್ಲಿ-ಸಿ-ದ-ರೆಂದು
ನಿಲ್ಲಿ-ಸಿದ್ದ
ನಿಲ್ಲಿ-ಸಿದ್ದಕ್ಕಾಗಿ
ನಿಲ್ಲಿ-ಸಿದ್ದಾನೆ
ನಿಲ್ಲಿ-ಸಿದ್ದು
ನಿಲ್ಲಿ-ಸಿ-ರ-ಬ-ಹುದು
ನಿಲ್ಲಿ-ಸಿ-ರ-ಬಹು-ದೆಂದು
ನಿಲ್ಲಿ-ಸಿ-ರುವ
ನಿಲ್ಲಿ-ಸಿ-ರು-ವಂತೆ
ನಿಲ್ಲಿ-ಸಿ-ರುವುದು
ನಿಲ್ಲಿ-ಸುತ್ತಾನೆ
ನಿಲ್ಲಿ-ಸುತ್ತಾರೆ
ನಿಲ್ಲಿ-ಸುತ್ತಾಳೆ
ನಿಲ್ಲಿ-ಸುತ್ತಿದ್ದರು
ನಿಲ್ಲಿ-ಸುವ
ನಿಲ್ಲುತ್ತದೆ
ನಿಲ್ಲುವ
ನಿಲ್ಲು-ವಂತಹ
ನಿಲ್ವುದು
ನಿಲ್ಸಿದ
ನಿವನ
ನಿವಾರ-ಣೆ-ಗಾಗಿ
ನಿವಾರಿಸಿ-ಕೊಂಡು
ನಿವಾರಿ-ಸಿ-ದನು
ನಿವಾಸ
ನಿವಾಸಾಶ್ರಾಯಾಂ
ನಿವಾಸಿ
ನಿವಾಸಿ-ಗಳ
ನಿವಾಸಿ-ಗ-ಳಾಗಿದ್ದ
ನಿವಾಸಿ-ಗ-ಳಾದ
ನಿವಾಸಿ-ಯಾದ
ನಿವಾಸಿ-ಯೆಂದೂ
ನಿವೃತ್ತ-ನಾಗ-ಲಿಚ್ಚಿಸಿ
ನಿವೃತ್ತ-ರಾಗಿದ್ದಾರೆಂದಾಗ
ನಿವೃತ್ತಿ-ಗೊ-ಳಿಸಿ
ನಿವೇದನ
ನಿವೇದ್ಯ
ನಿವೇದ್ಯಕ್ಕೆ
ನಿವೇದ್ಯದ
ನಿವೇಶನ
ನಿವೇಶನ-ಗ-ಳನ್ನು
ನಿವೇಶನ-ವನ್ನು
ನಿವೇಶನವು
ನಿಶಂಕ-ಮಲ್ಲು
ನಿಶಿತಾಸಿಯ
ನಿಶಿದಿ
ನಿಶಿದಿ-ಗೆ-ಯನ್ನು
ನಿಶಿಧಿ
ನಿಶಿಧಿಯಂ
ನಿಶ್ಶಂಕ
ನಿಶ್ಶಂಕಪ್ರತಾಪ
ನಿಷಿಧಿ
ನಿಷಿಧಿ-ಗಲ್ಲು
ನಿಷಿಧಿ-ಶಾ-ಸನ-ಗಳ
ನಿಷಿಧಿ-ಶಾ-ಸನ-ಗಳು
ನಿಷೀಧಿಕಾ
ನಿಷ್ಕಂಟಕಂ
ನಿಷ್ಕಂಟಕ-ವನ್ನಾಗಿ
ನಿಷ್ಕಂಟಕ-ವಾದ
ನಿಷ್ಕಾಮೇಶ್ವರ
ನಿಷ್ಕ್ರಿ-ಯತೆ-ಯನ್ನು
ನಿಷ್ಟ
ನಿಷ್ಠ-ನಾಗಿ
ನಿಷ್ಠ-ನಾಗಿದ್ದರೂ
ನಿಷ್ಠ-ರಾಗಿ
ನಿಷ್ಠ-ರಾಗಿದ್ದ-ರೆಂಬು-ದನ್ನು
ನಿಷ್ಠ-ರಾಗಿದ್ದ-ರೆಂಬುದು
ನಿಷ್ಠ-ಸಾಮಂತರು
ನಿಷ್ಠಾ-ವಂತ-ರಾಗಿದ್ದ-ವರು
ನಿಷ್ಠಾ-ವಂತರೂ
ನಿಷ್ಠೆ-ಯಿಂದ
ನಿಷ್ಪತ್ತಿ
ನಿಷ್ಪತ್ತಿಯ
ನಿಷ್ಪತ್ತಿ-ಯನ್ನು
ನಿಷ್ಪತ್ತಿ-ಯಾಗಿ-ರು-ವಂತೆ
ನಿಷ್ಪನ್ನ-ಗೊಳಿ-ಸಲು
ನಿಷ್ಪನ್ನ-ವಾಗಿ-ದೆಯೇ
ನಿಷ್ಪ್ರ-ಯೋಜ-ಕ-ವಾದು-ದೆಂದು
ನಿಸದಿ
ನಿಸಿತಾಸಿಯ
ನಿಸಿದಿ
ನಿಸಿದಿ-ಕಲ್ಲು-ಗಳು
ನಿಸಿದಿ-ಗಲ್ಲನ್ನು
ನಿಸಿದಿ-ಗಲ್ಲಿದೆ
ನಿಸಿದಿ-ಗಲ್ಲು
ನಿಸಿದಿ-ಗಲ್ಲು-ಗಳು
ನಿಸಿದಿ-ಗಲ್ಲೊಂದು
ನಿಸಿದಿಗೆ
ನಿಸಿದಿ-ಗೆಯ
ನಿಸಿದಿ-ಗೆ-ಯನ್ನು
ನಿಸಿದಿಯಂ
ನಿಸಿಧಿ-ಗೆಯ
ನಿಸ್ಸಂಕ
ನಿಸ್ಸಂಕ-ರೆ-ನಿಪ್ಪ
ನಿಸ್ಸಂಕ-ರೆನಿ-ಸಿದ್ದರು
ನಿಸ್ಸೀಮ
ನಿಸ್ಸೀಮಂ
ನಿಹಿತಾ
ನೀ
ನೀಚೋಚ್ಛ
ನೀಡ-ದಿದ್ದರೂ
ನೀಡದೆ
ನೀಡದೇ
ನೀಡ-ಬಹು-ದಾ-ಗಿತ್ತು
ನೀಡ-ಬ-ಹುದು
ನೀಡ-ಬಾರ-ದೆಂದೂ
ನೀಡ-ಬೇಕಾಗಿತ್ತೆಂಬು-ದನ್ನು
ನೀಡ-ಬೇಕಾಗಿತ್ತೆಂಬುದು
ನೀಡ-ಬೇಕಾಗಿದ್ದ
ನೀಡ-ಬೇಕಾಗುತ್ತಿತ್ತು
ನೀಡ-ಬೇ-ಕಾದ
ನೀಡ-ಲಾ-ಗಿತ್ತು
ನೀಡ-ಲಾಗಿತ್ತೆಂದು
ನೀಡ-ಲಾಗಿದೆ
ನೀಡ-ಲಾಗಿ-ದೆ-ಯೆಂದೂ
ನೀಡ-ಲಾಗಿ-ದೆಯೇ
ನೀಡಲಾ-ಗಿದ್ದ
ನೀಡ-ಲಾ-ಗಿದ್ದು
ನೀಡಲಾ-ಗಿ-ರುವ
ನೀಡ-ಲಾಗುತ್ತದೆ
ನೀಡ-ಲಾಗುತ್ತಿತ್ತು
ನೀಡಲಾ-ಗುತ್ತಿತ್ತೆಂದು
ನೀಡಲಾಯಿ
ನೀಡ-ಲಾ-ಯಿತು
ನೀಡಲಾಯಿ-ತೆಂದು
ನೀಡಲಾಯಿ-ತೆಂದೂ
ನೀಡಲು
ನೀಡಲ್ಪಟ್ಟ
ನೀಡಲ್ಪಟ್ಟಿದೆ
ನೀಡಲ್ಪಡುತ್ತಿದ್ದ
ನೀಡಿ
ನೀಡಿ-ಕೊಂಡು
ನೀಡಿತ್ತೆಂದು
ನೀಡಿದ
ನೀಡಿ-ದಂತೆ
ನೀಡಿ-ದನ
ನೀಡಿ-ದ-ನಂತೆ
ನೀಡಿ-ದನು
ನೀಡಿ-ದ-ನೆಂದಿದೆ
ನೀಡಿ-ದ-ನೆಂದು
ನೀಡಿ-ದ-ನೆಂದೂ
ನೀಡಿ-ದ-ನೆಂಬ
ನೀಡಿ-ದ-ನೆಂಬುದು
ನೀಡಿ-ದರು
ನೀಡಿ-ದರೆ
ನೀಡಿ-ದ-ರೆಂದು
ನೀಡಿ-ದ-ರೆಂಬ
ನೀಡಿ-ದಳು
ನೀಡಿ-ದ-ವರು
ನೀಡಿ-ದಾಗ
ನೀಡಿದೆ
ನೀಡಿದ್ದ
ನೀಡಿದ್ದನು
ನೀಡಿದ್ದ-ನೆಂದು
ನೀಡಿದ್ದ-ನೆಂದೂ
ನೀಡಿದ್ದ-ನೆಂಬ
ನೀಡಿದ್ದರ
ನೀಡಿದ್ದರು
ನೀಡಿದ್ದ-ರೆಂದು
ನೀಡಿದ್ದಾನೆ
ನೀಡಿದ್ದಾ-ನೆಂದು
ನೀಡಿದ್ದಾನೆಂಬ
ನೀಡಿದ್ದಾರೆ
ನೀಡಿದ್ದಾ-ರೆಂದು
ನೀಡಿದ್ದಾಳೆ
ನೀಡಿದ್ದಿರ-ಬ-ಹುದು
ನೀಡಿದ್ದು
ನೀಡಿದ್ದೇನೆ
ನೀಡಿ-ರ-ಬಹು
ನೀಡಿ-ರ-ಬಹು-ದಾದ್ದನ್ನು
ನೀಡಿ-ರ-ಬ-ಹುದು
ನೀಡಿ-ರ-ಬಹು-ದೆಂದು
ನೀಡಿ-ರ-ಬಹುದೇ
ನೀಡಿ-ರುತ್ತಾನೆ
ನೀಡಿ-ರುತ್ತಾರೆ
ನೀಡಿ-ರುವ
ನೀಡಿ-ರು-ವಂತೆ
ನೀಡಿ-ರು-ವಂತೆಯೇ
ನೀಡಿ-ರು-ವು-ದನ್ನು
ನೀಡಿ-ರುವು-ದರ
ನೀಡಿ-ರು-ವು-ದ-ರಿಂದ
ನೀಡಿ-ರು-ವು-ದಿಲ್ಲ
ನೀಡಿ-ರುವುದು
ನೀಡಿ-ರುವುದೂ
ನೀಡಿಲ್ಲ
ನೀಡಿವೆ
ನೀಡುತ್ತದೆ
ನೀಡುತ್ತವೆ
ನೀಡುತ್ತಾ
ನೀಡುತ್ತಾನೆ
ನೀಡುತ್ತಾ-ನೆಂದು
ನೀಡುತ್ತಾರೆ
ನೀಡುತ್ತಾಳೆ
ನೀಡುತ್ತಿದ್ದ
ನೀಡುತ್ತಿದ್ದಂತೆ
ನೀಡುತ್ತಿದ್ದರು
ನೀಡುತ್ತಿದ್ದ-ರೆಂದು
ನೀಡುತ್ತಿದ್ದ-ರೆಂಬು-ದಕ್ಕೆ
ನೀಡುತ್ತಿದ್ದಾರೆ
ನೀಡುತ್ತಿದ್ದು
ನೀಡುತ್ತಿದ್ದುದು
ನೀಡುತ್ತಿರ-ಲಿಲ್ಲ-ವೆಂಬುದು
ನೀಡುವ
ನೀಡು-ವಂತೆ
ನೀಡು-ವಂತೆಯೂ
ನೀಡುವಾಗ
ನೀಡು-ವು-ದನ್ನು
ನೀಡು-ವು-ದರ
ನೀಡು-ವುದ-ರಲ್ಲಿ
ನೀಡು-ವು-ದಾಗಿ
ನೀಡುವು-ದಾಗಿಯೂ
ನೀಡು-ವುದು
ನೀತಿ
ನೀತಿಗೆ
ನೀತಿ-ಮಹಾ-ರಾಜ
ನೀತಿ-ಮಾರ್ಗ
ನೀತಿ-ಮಾರ್ಗನ
ನೀತಿ-ಮಾರ್ಗ-ನನ್ನು
ನೀತಿ-ಮಾರ್ಗ-ನಿಗೆ
ನೀತಿ-ಮಾರ್ಗನು
ನೀತಿ-ಮಾರ್ಗ-ನೆಂಬ
ನೀತಿ-ಮಾರ್ಗನೇ
ನೀತಿ-ಮಾರ್ಗ್ಗ
ನೀತಿ-ವಾಕ್ಯ
ನೀತಿ-ವಿ-ದರೂ
ನೀತಿ-ವಿಶಾ-ರದಃ
ನೀತಿ-ಶಾಸ್ತ್ರ
ನೀತಿ-ಶಾಸ್ತ್ರಸ್ಯ
ನೀನೂ
ನೀರ
ನೀರ-ಗುಂಡಿಯ
ನೀರ-ಗುಂದ
ನೀರನ್ನು
ನೀರ-ಮಡು-ಸೇವೆಗೆ
ನೀರಲ್ಲಿ
ನೀರ-ಸ-ರದಿ-ಯನ್ನು
ನೀರಾರಂಭ
ನೀರಾರಂಭಕ್ಕೆ
ನೀರಾ-ವರಿ
ನೀರಾ-ವ-ರಿಗೆ
ನೀರಾ-ವ-ರಿಯ
ನೀರಾ-ವರಿ-ಯನ್ನು
ನೀರಾ-ವರಿ-ಯಾಗುತ್ತಿದೆ
ನೀರಾ-ವರಿ-ಯಾಗುವ
ನೀರಾ-ವರಿ-ಯಿಂದ
ನೀರಾ-ವ-ರಿಯು
ನೀರಿಗೆ
ನೀರಿಗೋಸ್ಕರ
ನೀರಿದ್ದು
ನೀರಿನ
ನೀರಿನಲ್ಲಿ
ನೀರಿನಿಂದ
ನೀರಿರುತ್ತಿದ್ದು
ನೀರಿಲ್ಲದ
ನೀರು
ನೀರು-ಣಿಸುತ್ತಿದ್ದವು
ನೀರು-ಣಿ-ಸುವ
ನೀರು-ವ-ರಿಯ
ನೀರು-ವ-ರಿಯು
ನೀರು-ಸಂಧಿಗೆ
ನೀರು-ಹರಿ-ಸಲು
ನೀರೊತ್ತನ್ನು
ನೀರೊತ್ತು
ನೀರೊಳಗೆ
ನೀರ್ಗುಂದ
ನೀರ್ಗುಂದದ
ನೀರ್ಗ್ಗುಂದ
ನೀರ್ಗ್ಗುನ್ದೆಳಾ
ನೀರ್ನೆಲ-ವನ್ನು
ನೀರ್ಮಣ್ಣನ್ನು-ಗದ್ದೆ
ನೀರ್ವ್ವ-ರಿಯ
ನೀಲ-ಕಂಠ
ನೀಲ-ಕಂಠ-ನ-ಹಳ್ಳಿ-ಗ-ಳನ್ನು
ನೀಲ-ಕಂಠ-ನ-ಹಳ್ಳಿಯ
ನೀಲ-ಕಂಠಾ-ಚಾರ್ಯನ
ನೀಲ-ಗಳಂ
ನೀಲ-ಗಿರಿ
ನೀಲ-ಗಿರಿಯ
ನೀಲ-ಗಿರಿ-ಯಲ್ಲಿ
ನೀಲ-ಗಿರಿ-ಸಾಧಾರ
ನೀಲ-ಚೊಟ್ಟ
ನೀಲತ್ತ-ಹಳ್ಳಿ-ಯಾಗಿ
ನೀಲ-ಮಸೂದ
ನೀಲ-ಮಸೂದ್
ನೀಲಯ್ಯ
ನೀಲ-ಸ-ಮುದ್ರ
ನೀಲಾಂತ-ಪಳ್ಳಿ-ಯ-ವರು
ನೀಲಾಚಲ-ನೀಲ-ಗಿರಿ-ಯನ್ನು
ನೀಲಾಚಲ-ವನ್ನು
ನೀಲಾತ-ಹಳ್ಳಿ
ನೀಲಾದ್ರಿಗೆ
ನೀಳಾಚ-ಳಮಂ
ನೀಳಾದ್ರಿಯಂ
ನೀಳಾದ್ರೀ-ಪಡಿ-ಯ-ಘಟ್ಟಂ
ನೀವು
ನು
ನುಂಗಿ
ನುಂಗುವ
ನುಕ-ರಾಜ-ನಿಗೆ
ನುಗು-ನಾಡು
ನುಗ್ಗಿ
ನುಗ್ಗಿತು
ನುಗ್ಗಿ-ತೆಂದೂ
ನುಗ್ಗಿದ
ನುಗ್ಗಿ-ದನು
ನುಗ್ಗಿ-ಲೂರ
ನುಗ್ಗಿ-ಲೂರು
ನುಗ್ಗಿ-ಹಳ್ಳಿಯ
ನುಗ್ಗೆ-ಹಳ್ಳಿ
ನುಗ್ಗೆ-ಹಳ್ಳಿ-ಗಳ
ನುಗ್ಗೇ-ಹಳ್ಳಿಯ
ನುಡಿ
ನುಡಿದ
ನುಡಿ-ದಂನ್ತೆ
ನುಡಿ-ದಂನ್ತೆ-ಗಂಡನುಂ
ನುಡಿ-ದುದೇ
ನುಡಿ-ದು-ಮತ್ತೆನ್ನನುಂ
ನುಡಿಯ
ನುಡಿ-ಯದ
ನುಡಿ-ಯಾದ
ನುಡಿವ
ನುಣು-ಪಾದ
ನುತ-ಬಲ್ಲಾಳ-ಭೂ-ಪನ
ನುತೆ
ನುಪ-ಡದ
ನುಪ-ಮರೆ-ಸೆದ-ರವ-ರೊಳಗೆ
ನುಳಮ್ಬನುಂ
ನೂ
ನೂಂಕು
ನೂಂಕು-ಲ-ಮರೈಯಂ
ನೂಕಿ-ದನು
ನೂತನ
ನೂತನ-ವಾಗಿ
ನೂತ್ನ-ರತ್ನಮುಂ
ನೂತ್ನ-ರತ್ನ-ಮುಮಂ
ನೂರನ್ನು
ನೂರ-ಯಿ-ವತ್ತು
ನೂರಾರು
ನೂರು
ನೂರು-ಕುಳಿ
ನೂರೆಂಟು
ನೂರೊಂದು
ನೂರ್ಮಡಿ
ನೂರ್ಮ್ಮಡಿ
ನೂಱನ್ನು
ನೂಱು-ವರಹ
ನೂಲ
ನೂಲ-ಕಳ-ಹುಗೆ
ನೂಲಿನ
ನೂಲು
ನೂಳುಲು
ನೃತ್ಯ
ನೃತ್ಯ-ದಲು
ನೃಪಂ
ನೃಪಃ
ನೃಪತಿಯ
ನೃಪ-ತುಂಗನ
ನೃಪ-ತುಂಗನು
ನೃಪತೇ
ನೃಪತೇಃ
ನೃಪನ
ನೃಪ-ನಿಂದೇ-ವಣ್ನಿಪೆಂ
ನೃಪನು
ನೃಪ-ಭೂ-ಪನ
ನೃಪ-ಭೂಪ-ನೆಂದು
ನೃಪ-ರಾಜ್ಯ-ವಾರ್ದ್ಧಿ-ಸಂವರ್ದ್ಧನ
ನೃಪ-ರಿಂದೊಡ-ಗೂಡಿದ
ನೃಪಾಂಬೋಧಿ
ನೃಸಿಂಹ
ನೃಸಿಂಹಂ
ನೃಸಿಂಹನ
ನೃಸಿಂಹ-ಭೂಪ-ನೆ-ಳೆಯಂ
ನೃಸಿಂಹ-ಭೂಪ-ನೆ-ಳೆಯಂದೋಸ್ಥಂಭ-ದೊಲು
ನೃಸಿಂಹ-ರಾಟ್
ನೃಸಿಂಹ-ಸೂರಿ
ನೃಸಿಂಹಾರ್ಪಣ-ಬುಧ್ಯಾ
ನೆಂದು
ನೆಂಬ
ನೆಂಬು-ದಾಗಿ
ನೆಂಬು-ವ-ವನು
ನೆಗಪಿದ-ನಂದಿಂ
ನೆಗಳ್ದ
ನೆಗಳ್ದಂ
ನೆಗಳ್ದ-ತೆಂಕ-ಣ-ರಾಯ-ನೆನಲ್ಕೆ-ನಿಪ್ಪ
ನೆಗಳ್ದ-ರೊಳ್
ನೆಗಳ್ದಳೋ
ನೆಗಳ್ದಾಧಿ-ರಾಜ-ಪದ-ವಿಗೆ
ನೆಗೆದು
ನೆಚ್ಚಿ-ಕೊಂಡಿದ್ದನು
ನೆಚ್ಚಿನ
ನೆಟ್ಟ-ಕಲ್ಲು
ನೆಟ್ಟನು
ನೆಟ್ಟನೆ
ನೆಟ್ಟ-ಲಗು
ನೆಟ್ಟಿಲುವಂ
ನೆಟ್ಟು
ನೆಟ್ಟೂರು
ನೆಡ-ಲಾಗಿದೆ
ನೆಡಸಿ
ನೆಡಿಸಿ-ಕೊಟ್ಟು
ನೆಡಿಸಿ-ದ-ರೆಂದು
ನೆಡಿಸಿದ್ದ
ನೆಡಿಸುತ್ತಾನೆ
ನೆಡಿ-ಸುತ್ತಾಳೆ
ನೆಡಿ-ಸು-ವುದು
ನೆಡುಮಾಮಿಟಿ
ನೆಡ್ಸಿದ
ನೆತ್ತರು
ನೆತ್ತರು-ಕೊಡು-ಗೆ-ಯನ್ನು
ನೆತ್ತರು-ಗೊಡಗೆ-ಯಾಗಿ
ನೆತ್ತಿ
ನೆನ-ಪನ್ನು
ನೆನಪಿ-ಗಾಗಿ
ನೆನಪಿಗೆ
ನೆನ-ಪಿನ
ನೆನಪಿ-ನಲ್ಲಿ
ನೆನಪು
ನೆನೆಯ-ಬ-ಹುದು
ನೆನೆಯ-ಬೇಕು
ನೆನೆವ
ನೆಮ್ಮದಿ-ಯಿಂದ
ನೆಮ್ಮೆದಿ-ಯನ್ನು
ನೆಯ
ನೆಯ-ಭದ್ರ
ನೆಯ-ಭದ್ರ-ನಾಗಿ-ರ-ಬ-ಹುದು
ನೆಯಾಮ
ನೆಯ್ಗೆ
ನೆರನೆ-ರಪಿ
ನೆರಪಲ್
ನೆರಪಿ-ದಂಕದ
ನೆರ-ಳಾಗಿ
ನೆರಳು
ನೆರ-ವನ್ನು
ನೆರ-ವಾಗಿ
ನೆರ-ವಾಗಿದ್ದಾನೆ
ನೆರ-ವಾಗಿ-ರ-ಬ-ಹುದು
ನೆರ-ವಾಗಿರೆ
ನೆರ-ವಾ-ಗುತ್ತವೆ
ನೆರ-ವಾಗುತ್ತಿದ್ದನು
ನೆರ-ವಾಗು-ವು-ದಲ್ಲದೆ
ನೆರ-ವಾದನು
ನೆರ-ವಿಗೆ
ನೆರ-ವಿನ
ನೆರವಿ-ನಿಂದ
ನೆರವು
ನೆರವೇ-ರದೇ
ನೆರವೇ-ರಿ-ಸಲು
ನೆರವೇ-ರಿಸಿ
ನೆರವೇ-ರಿಸಿ-ಕೊಂಡು
ನೆರವೇ-ರಿ-ಸಿದ
ನೆರವೇ-ರಿಸುತ್ತಿದ್ದ
ನೆರವೇ-ರಿಸು-ವಂತೆಯೂ
ನೆರೆ-ದಿದ್ದರು
ನೆರೆದಿದ್ದ-ರೆಂದು
ನೆರೆದು
ನೆರೆ-ದೊರೆ
ನೆರೆಯ
ನೆರೆಯಿಸುತ್ತಿದ್ದರು
ನೆರೆಯೆ
ನೆರೆ-ಹೊ-ರೆಯ
ನೆಲ
ನೆಲಂ
ನೆಲನಂ
ನೆಲ-ಮಂಗಲ
ನೆಲ-ಮಟ್ಟ-ದಲ್ಲಿದ್ದು
ನೆಲ-ಮನೆ
ನೆಲ-ಮ-ನೆಯ
ನೆಲ-ಮೆಟ್ಟು
ನೆಲ-ವನ್ನು
ನೆಲ-ಸಮ-ವಾಗಿದೆ
ನೆಲ-ಸಮ-ವಾಗಿವೆ
ನೆಲ-ಹದ
ನೆಲ-ಹಾ-ಸಿನ
ನೆಲ-ಹಾ-ಸಿನ-ಕಲ್ಲಿನ
ನೆಲ-ಹಾ-ಸಿನಲ್ಲಿ
ನೆಲಾ-ಪುರ
ನೆಲು-ಮನೆ
ನೆಲು-ಮನೆ-ಯಲ್ಲಿ
ನೆಲೆ
ನೆಲೆ-ಗ-ಳನ್ನು
ನೆಲೆ-ಗ-ಳಾಗಿದ್ದವು
ನೆಲೆ-ಗಳು
ನೆಲೆ-ಗೊಂಡ
ನೆಲೆ-ಗೊ-ಳಿಸಿ-ದ-ರೆಂದು
ನೆಲೆ-ಗೊಳ್ಳು-ವಂತೆ
ನೆಲೆ-ನಿಂತ
ನೆಲೆ-ನಿಂತರು
ನೆಲೆ-ನಿಂತ-ರೆಂಬುದೂ
ನೆಲೆ-ನಿಂತು
ನೆಲೆ-ಬೀಡಂ
ನೆಲೆ-ಬೀಡನ್ನು
ನೆಲೆ-ಬೀಡಾಗಿ
ನೆಲೆ-ಬೀಡಾಗಿತ್ತು
ನೆಲೆ-ಬೀಡಾಗಿದ್ದ
ನೆಲೆ-ಬೀಡಾಗಿ-ರ-ಬ-ಹುದು
ನೆಲೆ-ಬೀ-ಡಿಗೆ
ನೆಲೆ-ಬೀಡಿ-ನಲ್ಲಿ
ನೆಲೆ-ಬೀಡಿನಲ್ಲಿದ್ದ-ನೆಂದು
ನೆಲೆ-ಬೀಡಿನಿಂದ
ನೆಲೆ-ಬೀಡು-ಗ-ಳನ್ನು
ನೆಲೆ-ಬೀಡು-ಗಳೂ
ನೆಲೆ-ಯಾಗಿ
ನೆಲೆ-ಯಾ-ಗಿತ್ತು
ನೆಲೆ-ಯಾಗಿತ್ತೆಂದು
ನೆಲೆ-ಯಾಗಿದೆ
ನೆಲೆ-ಯಾಗಿದ್ದಿತು
ನೆಲೆ-ಯಾ-ಗಿದ್ದು
ನೆಲೆ-ಯಾಗಿರ
ನೆಲೆ-ಯೂರಿತೆ
ನೆಲೆ-ಯೂರಿತೆಂದು
ನೆಲೆ-ವೀಡಾಗಿತ್ತೆಂದು
ನೆಲೆ-ವೀಡಾಗಿತ್ತೆಂಬು-ದನ್ನು
ನೆಲೆ-ವೀಡಾಗಿತ್ತೆಂಬುದು
ನೆಲೆ-ವೀಡಾ-ಗಿದ್ದ
ನೆಲೆ-ವೀಡಿ-ನಲ್ಲಿ
ನೆಲೆ-ವೀಡಿ-ನಲ್ಲಿದ್ದಾಗ
ನೆಲೆ-ವೀಡಿ-ನಿಂದ
ನೆಲೆ-ವೀಡಿ-ನೊಳ್ಸ-ಮುತ್ತುಂಗ
ನೆಲೆ-ವೀಡು-ಗ-ಳಲ್ಲಿ
ನೆಲೆ-ವೃತ್ತಿ-ಗಳ
ನೆಲೆ-ಸಲು
ನೆಲೆಸಿ
ನೆಲೆ-ಸಿದ
ನೆಲೆ-ಸಿ-ದ-ನೆಂದು
ನೆಲೆ-ಸಿ-ದರು
ನೆಲೆ-ಸಿ-ದ-ರೆಂದು
ನೆಲೆ-ಸಿ-ದ-ವ-ರಿಂದ
ನೆಲೆ-ಸಿದ್ದ
ನೆಲೆ-ಸಿದ್ದನು
ನೆಲೆ-ಸಿದ್ದ-ನೆಂದು
ನೆಲೆ-ಸಿದ್ದ-ರಿಂದ
ನೆಲೆ-ಸಿದ್ದರು
ನೆಲೆ-ಸಿದ್ದ-ರೆಂದು
ನೆಲೆ-ಸಿದ್ದ-ರೆಂದೂ
ನೆಲೆ-ಸಿದ್ದ-ರೆಂಬ
ನೆಲೆ-ಸಿದ್ದ-ರೆಂಬುದು
ನೆಲೆ-ಸಿದ್ದಾಗ
ನೆಲೆ-ಸಿದ್ದಾರೆ
ನೆಲೆ-ಸಿದ್ದು
ನೆಲೆ-ಸಿಪ್ಪ
ನೆಲೆ-ಸಿರ
ನೆಲೆ-ಸಿ-ರ-ಬ-ಹುದು
ನೆಲೆ-ಸಿ-ರುವ
ನೆಲೆ-ಸಿ-ರು-ವಂತೆ
ನೆಲೆ-ಸಿರ್ಪ್ಪ
ನೆಲೆ-ಸುವ
ನೆಲೆ-ಸುವಿಕೆ
ನೆಲ್ಲ
ನೆಲ್ಲ-ಕೂ-ಳಣ
ನೆಲ್ಲ-ಕೂ-ಳಣ-ವಾಗಿ
ನೆಲ್ಲ-ಕೂೞಣಲಾಗೊದೆನ್ದು
ನೆಲ್ಲನ್ನು
ನೆಲ್ಲು
ನೆಳಿ-ಲೂರು
ನೆವನಣ
ನೇ
ನೇಕಾರರ
ನೇಕಾರರು
ನೇಕ್ರಿಶ
ನೇಗಿ-ಲಿಗೆ
ನೇಡಿ-ಪುದೆಲೆ
ನೇಣಿ-ನಲ್ಲಿತ್ತವ-ರಿಗೆ
ನೇತಾರರ
ನೇತೃತ್ವ
ನೇತೃತ್ವ-ದಲ್ಲಿ
ನೇತೃತ್ವ-ವನ್ನು
ನೇತ್ರ
ನೇತ್ರಃ
ನೇತ್ರ-ನೆಂದುಮೀ
ನೇಮ
ನೇಮಕ
ನೇಮ-ಕ-ಮಾಡ-ಲಾಗುತ್ತಿತ್ತು
ನೇಮ-ಕ-ಮಾಡಿ-ದನು
ನೇಮ-ಕ-ವಾಗಿದ್ದರು
ನೇಮ-ಕ-ವಾದ
ನೇಮ-ಕ-ವಾದ-ವ-ರೆಂದು
ನೇಮ-ಕಾತಿ
ನೇಮ-ದಂಡೇಶ
ನೇಮ-ದಂಡೇಶನ
ನೇಮ-ದಂಡೇಸ-ದಿಕ್ಕುಂ
ನೇಮ-ಮಂತ್ರಿಯ
ನೇಮ-ಮಂತ್ರೀಶ-ಪುತ್ರಂ
ನೇಮ-ವೆರ್ಗಡೆ
ನೇಮ-ಸಂಪನ್ನ-ರು-ಮಪ್ಪ
ನೇಮ-ಹೆರ್ಗಡೆ
ನೇಮಿ-ಚಂದ್ರ
ನೇಮಿತ-ವಾದ
ನೇಮಿಸ-ಲಾಗುತ್ತಿತ್ತು
ನೇಮಿಸಲಾ-ಗುತ್ತಿತ್ತೆಂದು
ನೇಮಿ-ಸ-ಲಾ-ಯಿತು
ನೇಮಿಸಲಾಯಿ-ತೆಂದು
ನೇಮಿಸಲ್ಪಟ್ಟ
ನೇಮಿಸಲ್ಪಟ್ಟನು
ನೇಮಿಸಲ್ಪಡುತ್ತಿದ್ದ
ನೇಮಿಸಿ
ನೇಮಿಸಿ-ಕೊಂಡನು
ನೇಮಿ-ಸಿದ
ನೇಮಿಸಿ-ದಂತೆ
ನೇಮಿಸಿ-ದನು
ನೇಮಿಸಿ-ದ-ನೆಂದು
ನೇಮಿಸಿ-ದನೇ
ನೇಮಿಸಿ-ದರು
ನೇಮಿಸಿ-ದು-ದಂತೂ
ನೇಮಿ-ಸಿದ್ದ
ನೇಮಿಸಿದ್ದನು
ನೇಮಿಸಿದ್ದರೂ
ನೇಮಿಸಿದ್ದಳು
ನೇಮಿಸಿದ್ದಾನೆ
ನೇಮಿ-ಸಿದ್ದು
ನೇಮಿಸಿ-ರ-ಬೇಕು
ನೇಮಿಸಿ-ರುವು
ನೇಮಿಸುತ್ತಾನೆ
ನೇಮಿಸುತ್ತಿದ್ದನು
ನೇಮಿಸುತ್ತಿದ್ದರು
ನೇಮಿಸುತ್ತಿದ್ದ-ರೆಂದು
ನೇಮಿಸುತ್ತಿದ್ದುರು
ನೇಮಿ-ಸುವ
ನೇಮಿ-ಸೆಟ್ಟಿ
ನೇಮೀಶ್ವರ
ನೇಯ್ಗೆ
ನೇಯ್ಗೆಗೆ
ನೇರ
ನೇರ-ಲ-ಕಟ್ಟೆ
ನೇರ-ಲ-ಕೆರೆ
ನೇರ-ಲ-ಕೆರೆಯ
ನೇರ-ಲ-ತಾಳ-ಕಟ್ಟೆ
ನೇರ-ಲಿಗೆ
ನೇರಲು
ನೇರ-ಳ-ಕಟ್ಟೆಯ
ನೇರ-ಳ-ಕೆರೆಯು
ನೇರಳೆ
ನೇರ-ಳೆ-ಕೆರೆಯ
ನೇರ-ವಾಗಿ
ನೇರ-ವಾದ
ನೇರ-ಶಿಷ್ಯರು
ನೇಶ
ನೇಸ-ರಂಗಿದಿ-ರಾಗಿ
ನೇಸರ-ಪಳ್ಳ-ವನ್ನು
ನೈಋತ್ಯ
ನೈಋತ್ಯ-ದಲ್ಲಿ
ನೈಜ-ವಾದ
ನೈಜಾಮ-ನಿಗೆ
ನೈಜಾಮ್
ನೈಮಿತ್ತಿಕ
ನೈಯ-ಮದು
ನೈರುತ್ಯಕ್ಕೆ
ನೈರುತ್ಯದ
ನೈವೇದ್ಯ
ನೈವೇದ್ಯಕ್ಕೆ
ನೈವೇದ್ಯದ
ನೈವೇದ್ಯ-ವನ್ನು
ನೈವೇದ್ಯವೋ
ನೈಷ್ಠಿಕ
ನೈಸರ್ಗಿಕ
ನೊಣಂಬ-ವಾಡಿ
ನೊಣಂಬಿ-ಸೆಟ್ಟಿ
ನೊಣಂಬಿ-ಸೆಟ್ಟಿ-ಯನ್ನು
ನೊಬೆಯ
ನೊಳಂಬ
ನೊಳಂಬ-ಕುಲಾಂತಕ-ದೇವನ
ನೊಳಂಬನ
ನೊಳಂಬ-ನಿಗೆ
ನೊಳಂಬನು
ನೊಳಂಬರ
ನೊಳಂಬ-ರನ್ನು
ನೊಳಂಬ-ರ-ಸ-ರಾಗಿ-ರ-ಬ-ಹುದು
ನೊಳಂಬ-ರಾಜನ
ನೊಳಂಬ-ರಾಜಾನ್ವಯದ
ನೊಳಂಬ-ರಿಂದ
ನೊಳಂಬರು
ನೊಳಂಬರೂ
ನೊಳಂಬ-ರೊಡನೆ
ನೊಳಂಬ-ಳಿಗೆ
ನೊಳಂಬ-ವಾಡಿ
ನೊಳಂಬ-ವಾಡಿಯ
ನೊಳಂಬ-ವಾಡಿ-ಯನ್ನು
ನೊಳಂಬ-ವಾಡಿಯು
ನೊಳಂಬಾದಿ-ರಾಜ
ನೊಳಂಬಾದಿ-ರಾಜನು
ನೊಳಂಬಾದಿ-ರಾಜ-ನುಕ್ರಿಶ
ನೊಳಂಬಾಧಿ-ರಾಜನು
ನೊಳಂಬಾಧಿ-ರಾಜ-ರನ್ನು
ನೊಳಂಬಿ
ನೊಳಂಬಿ-ಸೆಟ್ಟಿ
ನೊಳಬಂನೂ
ನೊೞಂಬ
ನೋಂತು
ನೋಟ
ನೋಟದ
ನೋಟ-ನೆನಪು
ನೋಡದೇ
ನೋಡ-ಬ-ಹುದು
ನೋಡಲು
ನೋಡಿ
ನೋಡಿ-ಕೊಂಡು
ನೋಡಿ-ಕೊಳ್ಳಲು
ನೋಡಿ-ಕೊಳ್ಳುತ್ತಿದ್ದ
ನೋಡಿ-ಕೊಳ್ಳುತ್ತಿದ್ದರು
ನೋಡಿ-ಕೊಳ್ಳುತ್ತಿದ್ದ-ರೆಂದು
ನೋಡಿ-ಕೊಳ್ಳುತ್ತಿದ್ದ-ರೆಂಬುದು
ನೋಡಿ-ಕೊಳ್ಳುತ್ತಿದ್ದ-ವನೇ
ನೋಡಿ-ಕೊಳ್ಳುತ್ತಿದ್ದವು
ನೋಡಿ-ಕೊಳ್ಳುತ್ತಿದ್ದುದು
ನೋಡಿ-ಕೊಳ್ಳುವ
ನೋಡಿ-ಕೊಳ್ಳು-ವಂತೆ
ನೋಡಿ-ಕೊಳ್ಳುವ-ವರು
ನೋಡಿ-ಕೊಳ್ಳು-ವುದು
ನೋಡಿದ
ನೋಡಿ-ದಂತೆ
ನೋಡಿ-ದರೆ
ನೋಡಿ-ದಾಗ
ನೋಡಿ-ದಾಗಂ
ನೋಡಿಸಿ
ನೋಡು
ನೋಡು-ತಿರೆ
ನೋಡುತ್ತಾ
ನೋಡುವ
ನೋಡೆ
ನೋಯಾ-ಯಿಕ
ನೋರ್ಪ್ಪಡೆಲ್ಲರುಂ
ನೋಳ್ಪಡ
ನೋಳ್ಪಡೆ
ನೋವಿನ
ನೌಕರ
ನೌಕ-ರನ
ನೌಕ-ರರ
ನೌಕ-ರರು
ನೌಕರ-ವರ್ಗ-ದ-ವ-ನಾದ
ನೌಕಾ-ಸೇನೆಯ
ನ್ತಿವರ್ಸ್ಸಲೆ
ನ್ದೀಕ್ಷಿತಿ
ನ್ದೇವ-ತರು-ಕು-ಡುವ
ನ್ನು
ನ್ಮಾನ
ನ್ಮಾನದ
ನ್ಮಾನಿ-ಗಳೇ-ಕ-ವೀರ
ನ್ಯಾ
ನ್ಯಾಯ-ತೀರ್ಮಾನ
ನ್ಯಾಯ-ತೀರ್ಮಾನ-ವನ್ನು
ನ್ಯಾಯ-ಯತೀಂದ್ರ
ನ್ಯೂನಿಸ್ರು
ನ್ರಿಪಾಳಂ
ಪ
ಪಂಕರುಂ
ಪಂಕರು-ಹೋದರಂ
ಪಂಕ್ತಿಯ
ಪಂಗಡಕ್ಕೆ
ಪಂಗಡ-ಗಳಿದ್ದವು
ಪಂಗ-ಡದ-ವ-ರಾಗಿದ್ದರು
ಪಂಚ
ಪಂಚಂಪಲ್ಲಿ
ಪಂಚ-ಕಂಬಿ
ಪಂಚ-ಕಂಬಿ-ಮೇಳ
ಪಂಚ-ಕಾರುಕ
ಪಂಚ-ಕಾರು-ಕರು
ಪಂಚ-ಕೂಟ
ಪಂಚ-ಗೊಂಡ
ಪಂಚದ
ಪಂಚ-ನಾ-ರಾಯಣ
ಪಂಚ-ನೇತ್ರಧ್ವಜ
ಪಂಚ-ಪದ-ಮನುಚ್ಚಾರಿ-ಸುತ್ತಂ
ಪಂಚ-ಪರ್ವ
ಪಂಚಪ್ರ-ಕಾರ
ಪಂಚಪ್ರಧಾನ
ಪಂಚಪ್ರಧಾನರ
ಪಂಚಪ್ರಧಾನ-ರಲ್ಲಿ
ಪಂಚ-ಬ-ಸದಿ-ಗ-ಳನ್ನು
ಪಂಚ-ಬ-ಸದಿ-ಯೊಳಗೆ
ಪಂಚ-ಬಾಣ-ಕವಿಯ
ಪಂಚ-ಭಾಗ-ವತ
ಪಂಚಮ
ಪಂಚ-ಮಟ
ಪಂಚ-ಮಠ
ಪಂಚ-ಮ-ಠ-ಗಳ
ಪಂಚ-ಮಠ-ಗ-ಳನ್ನು
ಪಂಚ-ಮ-ಠ-ಗ-ಳಿಗೆ
ಪಂಚ-ಮ-ಠ-ಗಳು
ಪಂಚ-ಮ-ಠದ
ಪಂಚ-ಮ-ಠಸ್ಥಾನ-ಪತಿ
ಪಂಚ-ಮ-ಠಸ್ಥಾನ-ಪತಿ-ಗಳ
ಪಂಚ-ಮ-ಠಸ್ಥಾನ-ಪತಿ-ಗಳಪ್ಪ
ಪಂಚ-ಮ-ಠಸ್ಥಾನ-ಪತಿ-ಗಳು
ಪಂಚ-ಮ-ರೆಂದು
ಪಂಚ-ಮಹಾ-ಪಾ-ತಕ-ಗ-ಳಲ್ಲಿ
ಪಂಚ-ಮಹಾ-ಪಾ-ತಕ-ನಾಗುತ್ತಾ-ನೆಂದೂ
ಪಂಚ-ಮಹಾಪ್ರಧಾನ
ಪಂಚ-ಮಹಾಪ್ರಧಾನರ
ಪಂಚ-ಮಹಾಪ್ರಧಾನ-ರೆಂದರೆ
ಪಂಚ-ಮ-ಹಾ-ಶಬ್ದ
ಪಂಚ-ಮ-ಹಾ-ಶಬ್ದ-ಗ-ಳನ್ನು
ಪಂಚಮಿ
ಪಂಚ-ಮಿಯ
ಪಂಚ-ಮುಖ-ವಿಭಾಡ
ಪಂಚರ
ಪಂಚ-ರಲ್ಲಿ
ಪಂಚ-ರಿಗೆ
ಪಂಚರು
ಪಂಚ-ಲಿಂಗ
ಪಂಚ-ಲಿಂಗ-ಗ-ಳಿಗೆ
ಪಂಚ-ಲಿಂಗೇಶ್ವರ
ಪಂಚ-ಲೋ-ಹದ
ಪಂಚ-ವನ್
ಪಂಚ-ವನ್ಮಹಾ-ರಾಯ-ನೆಂಬ
ಪಂಚ-ವಮಾ-ರಾಯ-ನಾದ
ಪಂಚ-ವಿಧಾ-ಚಾರ-ನಿರ-ತ-ರು-ಮಪ್ಪ
ಪಂಚ-ವೃತ್ತ್ಯಾ-ಚಾರ್ಯ-ದಲ್ಲಿಯ
ಪಂಚ-ಶತ
ಪಂಚ-ಶತ-ವೀರ-ಶ-ಸನ
ಪಂಚ-ಸಂಸ್ಕಾರ
ಪಂಚ-ಸಂಸ್ಕಾರ-ದಲ್ಲಿ
ಪಂಚ-ಸಂಸ್ಕಾರ-ವನ್ನು
ಪಂಚ-ಸಂಸ್ಕಾರ-ವೆಂಬ
ಪಂಚ-ಸತ
ಪಂಚಾ-ನನಂ
ಪಂಚಾರತಿ
ಪಂಚಾರ-ತಿಯ
ಪಂಚಾಳ
ಪಂಚಾಳ-ದ-ವರು
ಪಂಚಾಳ-ದ-ವರು-ಪಾಂಚಾಳ-ದ-ವರು
ಪಂಚಾ-ಳರು
ಪಂಚಾ-ಶತತ್ರಿಂಶತಶ್ಚ
ಪಂಚಿಕೇಶ್ವರ
ಪಂಚಿಕೇಶ್ವರ-ಗ-ಳನ್ನು
ಪಂಚಿಕೇಶ್ವರದ
ಪಂಚಿಕೇಶ್ವರ-ದಲ್ಲಿ
ಪಂಜದ
ಪಂಡರಿ-ದೇವನು
ಪಂಡರೀ-ದೇವ
ಪಂಡರೀ-ದೇವನು
ಪಂಡಿ
ಪಂಡಿತ
ಪಂಡಿ-ತ-ದೇವರು
ಪಂಡಿ-ತನ
ಪಂಡಿ-ತ-ನಾಗಿದ್ದನು
ಪಂಡಿ-ತ-ನಾ-ಗಿದ್ದು
ಪಂಡಿ-ತ-ನಿಗೆ
ಪಂಡಿ-ತನು
ಪಂಡಿ-ತ-ನೆಂದು
ಪಂಡಿ-ತ-ನೆಂಬ
ಪಂಡಿ-ತ-ಮರ-ಣ-ವೆನ್ನ-ಲಾಗಿದೆ
ಪಂಡಿ-ತರ
ಪಂಡಿ-ತ-ರಿಗೂ
ಪಂಡಿ-ತ-ರಿಗೆ
ಪಂಡಿ-ತರು
ಪಂಡಿ-ತ-ರು-ಗಳ
ಪಂಡಿ-ತ-ರು-ಗಳು
ಪಂಡಿ-ತ-ವರ್ಯರು
ಪಂಡಿ-ತ-ಹಳ್ಳಿ
ಪಂಡಿ-ತೋಜ
ಪಂಡಿ-ತೋಜನ
ಪಂಡಿ-ತೋಜನು
ಪಂತನ
ಪಂತಳೆದಂ
ಪಂಥ
ಪಂಥಕ್ಕೆ
ಪಂಥದ
ಪಂಥ-ದಲ್ಲಿ
ಪಂಥ-ದವ-ರಲ್ಲಿ
ಪಂಥ-ವನ್ನಷ್ಟೇ
ಪಂಥ-ವಾಗಿ
ಪಂಥವು
ಪಂಥ-ವೊಂದು
ಪಂದಲ-ದೇವ
ಪಂದಲ-ದೇವನು
ಪಂದಲೆ-ಗ-ಳಿಗೆ
ಪಂದ-ಲೆಯಂ
ಪಂದೂರಕ್ಕಿಯ
ಪಂನಗವೈ-ನತೇಯ
ಪಂನಾಯ
ಪಂನಾಯವಂ
ಪಂನಾಯ-ವನ್ನು
ಪಂಪ
ಪಂಪನ
ಪಂಪ-ಭಾರತ
ಪಂಪ-ಭಾರ-ತ-ದಲ್ಲಿ
ಪಂಪ-ರಾಜ
ಪಂಪ-ರಾ-ಮಾ-ಯಣ
ಪಂಪಾಕ್ಷೇತ್ರದ
ಪಕ್ಕ
ಪಕ್ಕದ
ಪಕ್ಕ-ದಲ್ಲಿ
ಪಕ್ಕ-ದಲ್ಲಿದೆ
ಪಕ್ಕ-ದಲ್ಲಿದ್ದ
ಪಕ್ಕ-ದಲ್ಲಿಯೇ
ಪಕ್ಕ-ದಲ್ಲಿ-ರುವ
ಪಕ್ಕ-ದಲ್ಲಿವೆ
ಪಕ್ಕ-ದಲ್ಲೂ
ಪಕ್ಕ-ದಲ್ಲೇ
ಪಕ್ವಾಂನ್ನದ
ಪಕ್ವಾನ್ನದ
ಪಕ್ಷ
ಪಕ್ಷದ
ಪಕ್ಷ-ದಲ್ಲಿ
ಪಕ್ಷ-ದಲ್ಲಿ-ದೇವರ
ಪಕ್ಷ-ದ-ವ-ರನ್ನು
ಪಕ್ಷ-ಪಾತಿ-ಗ-ಳಾಗಿದ್ದರು
ಪಕ್ಷ-ಪಾತಿ-ಗ-ಳಾಗಿದ್ದರೂ
ಪಕ್ಷಯ
ಪಕ್ಷ-ಯ-ಕರ
ಪಕ್ಷ-ವಹಿಸಿ
ಪಕ್ಷಾರ್ಧ-ದಲ್ಲಿ
ಪಕ್ಷೋತ್ಸವ
ಪಕ್ಷೋಪ-ವಾಸಿ
ಪಕ್ಷೋಪ-ವಾಸಿ-ಗಳಳ್ಪಾ
ಪಕ್ಷೋಪ-ವಾಸಿ-ಗಳ್ಪಾ
ಪಗೋಡಾ
ಪಘವುಲೆಟಿ
ಪಙ್ಗು
ಪಙ್ಗುಮ್
ಪಚೆ-ಯಣ್ಣ
ಪಚ್ಚ-ಕರ್ಪೂರಕಸ್ತೂರಿ
ಪಚ್ಚಮಂದಾದುರಿ
ಪಚ್ಚೆ-ಕರ್ಪೂರ
ಪಞ್ಚ-ಗೊಂಡ
ಪಞ್ಚ-ಮಹಾ-ಪಾ-ತಕಂ
ಪಞ್ಚ-ಮಹಾ-ಪಾ-ತಕ-ರಪ್ಪೋರ್
ಪಟ-ಗ-ಳಲ್ಲಿ
ಪಟಿದ್ದಾರೆ
ಪಟುಪ್ರ-ಭಾವಃ
ಪಟು-ರರ್ಹತೋ
ಪಟೇಲ
ಪಟೇಲ-ರಿಗೆ
ಪಟೇಲ್
ಪಟ್ಟ
ಪಟ್ಟಂಗಟ್ಟಿದ
ಪಟ್ಟಂಗಟ್ಟಿ-ದ-ನೆಂದು
ಪಟ್ಟ-ಕಟ್ಟಿದ
ಪಟ್ಟ-ಕಟ್ಟಿ-ದನು
ಪಟ್ಟ-ಕಟ್ಟಿ-ದರು
ಪಟ್ಟ-ಕಟ್ಟಿ-ಸಿದ
ಪಟ್ಟಕ್ಕೆ
ಪಟ್ಟಕ್ಕೇರಿ
ಪಟ್ಟಕ್ಕೇರಿದ
ಪಟ್ಟಕ್ಕೇರಿ-ದರು
ಪಟ್ಟಡಿ
ಪಟ್ಟಣ
ಪಟ್ಟ-ಣ-ಕಟ್ಟಿ-ಸುವ
ಪಟ್ಟ-ಣಕ್ಕೆ
ಪಟ್ಟ-ಣ-ಗಳ
ಪಟ್ಟ-ಣ-ಗಳಂತಹ
ಪಟ್ಟ-ಣ-ಗಳನ್ನಾಗಿ
ಪಟ್ಟ-ಣ-ಗ-ಳನ್ನು
ಪಟ್ಟ-ಣ-ಗ-ಳಲ್ಲಿ
ಪಟ್ಟ-ಣ-ಗಳಲ್ಲಿದ್ದು-ಕೊಂಡು
ಪಟ್ಟ-ಣ-ಗ-ಳಾಗಿದ್ದ-ವೆಂದು
ಪಟ್ಟ-ಣ-ಗ-ಳಾಗಿ-ರ-ಬ-ಹುದು
ಪಟ್ಟ-ಣ-ಗ-ಳಿಂದ
ಪಟ್ಟ-ಣ-ಗ-ಳಿಗೆ
ಪಟ್ಟ-ಣ-ಗಳು
ಪಟ್ಟ-ಣ-ಗಾರ
ಪಟ್ಟ-ಣ-ಗೆರೆ
ಪಟ್ಟ-ಣ-ಗೆರೆ-ಯಲ್ಲಿ
ಪಟ್ಟ-ಣದ
ಪಟ್ಟ-ಣ-ದಂತಹ
ಪಟ್ಟ-ಣ-ದಮ್ಮನ
ಪಟ್ಟ-ಣ-ದಲ್ಲಿ
ಪಟ್ಟ-ಣ-ದಲ್ಲಿಯೂ
ಪಟ್ಟ-ಣ-ದ-ವ-ರಿಗೆ
ಪಟ್ಟ-ಣ-ದಿಂದ
ಪಟ್ಟ-ಣ-ಪತ್ತನ
ಪಟ್ಟ-ಣ-ಪುರ
ಪಟ್ಟ-ಣವಂ
ಪಟ್ಟ-ಣ-ವನ್ನಾಗಿ
ಪಟ್ಟ-ಣ-ವನ್ನು
ಪಟ್ಟ-ಣ-ವಾಗಿ
ಪಟ್ಟ-ಣ-ವಾ-ಗಿತ್ತು
ಪಟ್ಟ-ಣ-ವಾ-ಗಿದ್ದ
ಪಟ್ಟ-ಣ-ವಾಗಿದ್ದವು
ಪಟ್ಟ-ಣ-ವಾಗಿದ್ದಿರ-ಬಹು-ದೆಂದು
ಪಟ್ಟ-ಣವು
ಪಟ್ಟ-ಣವೂ
ಪಟ್ಟ-ಣ-ವೆಂದು
ಪಟ್ಟ-ಣ-ವೆನಿಸುತ್ತಿದ್ದಿತು
ಪಟ್ಟ-ಣ-ಸಾಮಿ-ಯೆಂಬ
ಪಟ್ಟ-ಣ-ಸೆಟ್ಟಿ
ಪಟ್ಟ-ಣ-ಸೆಟ್ಟಿ-ಗಳಲ್ಲೇ
ಪಟ್ಟ-ಣ-ಸೆಟ್ಟಿ-ಗ-ಳಿಗೆ
ಪಟ್ಟ-ಣ-ಸೆಟ್ಟಿ-ಗಳು
ಪಟ್ಟ-ಣಸ್ಥಳದ
ಪಟ್ಟ-ಣಸ್ವಾಮಿ
ಪಟ್ಟ-ಣಸ್ವಾಮಿ-ಗಳ
ಪಟ್ಟ-ಣಸ್ವಾಮಿ-ಗ-ಳಾಗಿ-ರುತ್ತಿದ್ದ-ರೆಂದು
ಪಟ್ಟ-ಣಸ್ವಾಮಿ-ಗಳಾಗುತ್ತಿದ್ದ-ರೆಂದು
ಪಟ್ಟ-ಣಸ್ವಾಮಿ-ಗ-ಳಾದ
ಪಟ್ಟ-ಣಸ್ವಾಮಿ-ಗ-ಳಿಗೆ
ಪಟ್ಟ-ಣಸ್ವಾಮಿ-ಗಳು
ಪಟ್ಟ-ಣಸ್ವಾಮಿ-ಗಿಂತ
ಪಟ್ಟ-ಣಸ್ವಾಮಿಗೂ
ಪಟ್ಟ-ಣಸ್ವಾಮಿ-ಪಟ್ಟ-ಣ-ಸೆಟ್ಟಿ-ಸೆಟ್ಟಿ-ವಟ್ಟ
ಪಟ್ಟ-ಣಸ್ವಾಮಿಯ
ಪಟ್ಟ-ಣಸ್ವಾಮಿ-ಯನ್ನು
ಪಟ್ಟ-ಣಸ್ವಾಮಿ-ಯಾ-ಗಿದ್ದ
ಪಟ್ಟ-ಣಸ್ವಾಮಿ-ಯಾ-ಗಿದ್ದು
ಪಟ್ಟ-ಣಸ್ವಾಮಿ-ಸೆಟ್ಟಿ
ಪಟ್ಟ-ಣಸ್ವಾಮಿ-ಹಳ್ಳಿಯ
ಪಟ್ಟ-ಣೀ-ಕರಣ
ಪಟ್ಟ-ಣೀ-ಕರ-ಣದ
ಪಟ್ಟ-ಣೀ-ಕರ-ಣ-ದಲ್ಲಿ
ಪಟ್ಟ-ಣೀ-ಕರ-ಣವು
ಪಟ್ಟ-ಣೋ-ಜನ
ಪಟ್ಟದ
ಪಟ್ಟ-ದ-ರಸಿ
ಪಟ್ಟ-ದ-ರಸಿ-ಯನ್ನು
ಪಟ್ಟ-ದ-ರಾಣಿ
ಪಟ್ಟ-ದ-ರಾಣಿಯ
ಪಟ್ಟ-ದಾ-ನೆ-ಯಂತೆ
ಪಟ್ಟ-ಬಂಧ
ಪಟ್ಟ-ಬಂಧೋತ್ಸವ
ಪಟ್ಟ-ಬದ್ಧ
ಪಟ್ಟಮಂ
ಪಟ್ಟ-ಮಹಾ-ದೇವಿ
ಪಟ್ಟ-ಮಹಾ-ದೇವಿ-ಗಿಂತ
ಪಟ್ಟ-ಮಹಾ-ದೇವಿಯ
ಪಟ್ಟ-ಮಹಿಷಿ
ಪಟ್ಟ-ಮುಮ್
ಪಟ್ಟ-ಯಙ್ಗನ್
ಪಟ್ಟ-ಯಾಂಗ-ನಿಗೆ
ಪಟ್ಟ-ಯೆ-ಲೆಯ
ಪಟ್ಟ-ಲದಮ್ಮಪ-ಡ-ಲದಮ್ಮ-ಶಕ್ತಿ-ದೇವತೆ
ಪಟ್ಟ-ವನು
ಪಟ್ಟ-ವನ್ನು
ಪಟ್ಟ-ವರ್ಧನರ
ಪಟ್ಟ-ವಾಗಿ
ಪಟ್ಟ-ವಾಯಿತು
ಪಟ್ಟ-ವಾಯಿ-ತೆಂದೂ
ಪಟ್ಟವೂ
ಪಟ್ಟ-ವೇರಿ-ದನು
ಪಟ್ಟ-ಸಾಲೆ-ಯನ್ನು
ಪಟ್ಟ-ಸಾ-ಹಣಿ
ಪಟ್ಟ-ಸಾ-ಹಣಿ-ಯಾಗಿ
ಪಟ್ಟ-ಸಾ-ಹಣಿಯು
ಪಟ್ಟ-ಸೋಮ-ನ-ಹಳ್ಳಿ
ಪಟ್ಟಸ್ವಾಮಿ
ಪಟ್ಟಾಭಿಷಿಕ್ತನಾ-ದನು
ಪಟ್ಟಾಭಿಷಿಕ್ತ-ನಾದ-ನೆಂದು
ಪಟ್ಟಾಭಿಷೇಕ
ಪಟ್ಟಾಭಿಷೇ-ಕದ
ಪಟ್ಟಾಭಿಷೇಕ-ವನ್ನು
ಪಟ್ಟಾಭಿಷೇಕ-ವನ್ನೇ
ಪಟ್ಟಾಭಿಷೇಕ-ವಾದ-ಕೂಡಲೇ
ಪಟ್ಟಾಭಿಷೇಕ-ವಾದಾಗ
ಪಟ್ಟಿ
ಪಟ್ಟಿ-ಕೆಯ
ಪಟ್ಟಿದ್ದಾರೆ
ಪಟ್ಟಿದ್ದಾ-ರೆಂದು
ಪಟ್ಟಿದ್ದಾರೆೆ
ಪಟ್ಟಿ-ಮಾಡಿ
ಪಟ್ಟಿ-ಮಾಡಿ-ಕೊಡ-ಲಾಗಿದೆ
ಪಟ್ಟಿ-ಮಾಡಿದೆ
ಪಟ್ಟಿ-ಮಾಡಿದ್ದಾರೆ
ಪಟ್ಟಿ-ಮಾಡುತ್ತದೆ
ಪಟ್ಟಿ-ಯನ್ನು
ಪಟ್ಟಿ-ಯನ್ನೂ
ಪಟ್ಟಿ-ಯಲ್ಲಿ
ಪಟ್ಟಿಯೇ
ಪಟ್ಟಿ-ರುವ
ಪಟ್ಟಿ-ರುವುದು
ಪಟ್ಟೆ
ಪಟ್ಟೆ-ಯಾಂಗ
ಪಟ್ಟೆ-ಯಾಂಗ-ನಿಗೆ
ಪಟ್ಟೆ-ಯಾಂಗ-ನೆಂಬು-ವವ-ನಿಗೆ
ಪಟ್ಟೆ-ಯೆ-ಲೆಯ
ಪಠ-ಣಕ್ಕೆ
ಪಠ-ಣವು
ಪಠಣೆ-ಯನ್ನು
ಪಠ-ನೆಯ
ಪಠಿ-ಸಲು
ಪಠಿಸಿ
ಪಠಿ-ಸುವ
ಪಡ-ದನು
ಪಡ-ಲಾಗಿದೆ
ಪಡಿ
ಪಡಿಗೆ
ಪಡಿ-ತರ
ಪಡಿ-ತರ-ದಲ್ಲಿ
ಪಡಿಯ
ಪಡಿ-ಯನ್ನು
ಪಡಿ-ಯಾರ್
ಪಡಿ-ಯಾರ್ದ್ದಕ್ಷಿಣ
ಪಡಿ-ಯಾರ್ದ್ದಕ್ಷಿಣ-ಚಕ್ರ-ವರ್ತ್ತಿ
ಪಡಿ-ಯಿಡು-ವಂತೆ
ಪಡಿ-ಯಿಪ್ಪಂತೆ
ಪಡಿಯೆ
ಪಡಿ-ಯೋರ್ವ್ವಳೆಣ್ನೆ
ಪಡಿರ
ಪಡಿ-ವೊಳ-ಗಾಗಿ
ಪಡಿ-ಸಿದ್ದಾಗಿಯೂ
ಪಡಿ-ಸಿದ್ದು
ಪಡಿ-ಹಾರ
ಪಡಿ-ಹಾರರು
ಪಡುತ್ತಾರೆ
ಪಡುಮೊಗ-ವಾಗಿ
ಪಡುವ
ಪಡು-ವಂತ-ಹ-ವರು
ಪಡುವಣ
ಪಡುವ-ಣ-ಕೋಡಿ
ಪಡುವ-ಣ-ಕೋಡಿ-ಗಳಿತ್ತೆಂದು
ಪಡುವ-ಣ-ಕೋಡಿಗೆ
ಪಡುವ-ಣ-ಹಳ್ಳ
ಪಡುವ-ನಾಡ
ಪಡುವಲ
ಪಡುವ-ಲ-ಪಟ್ಟ-ಣದ
ಪಡುವ-ಲಾಗಿ
ಪಡುವಲು
ಪಡುವಲ್
ಪಡುವ-ಳದಿಂ
ಪಡುವ-ಸಾ-ಗರ
ಪಡುವಾಯ್ಕಾವೇರಿಯ
ಪಡುವೆಣ್ಣೆ
ಪಡೆ
ಪಡೆ-ಗಳ
ಪಡೆ-ಗ-ಳಿಗೆ
ಪಡೆ-ಗಳು
ಪಡೆದ
ಪಡೆದಂ
ಪಡೆ-ದಂತೆ
ಪಡೆ-ದದ್ದು
ಪಡೆ-ದ-ನಂತೆ
ಪಡೆ-ದನು
ಪಡೆ-ದ-ನೆಂದು
ಪಡೆ-ದ-ನೆಂದೂ
ಪಡೆ-ದರು
ಪಡೆ-ದರೂ
ಪಡೆ-ದ-ರೆಂದು
ಪಡೆ-ದ-ಳೆಂದು
ಪಡೆ-ದ-ವನು
ಪಡೆ-ದ-ವರ
ಪಡೆ-ದ-ವ-ರನ್ನು
ಪಡೆ-ದ-ವ-ರಾ-ಗಿದ್ದರು
ಪಡೆ-ದ-ವ-ರಾಗಿ-ರ-ಬ-ಹುದು
ಪಡೆ-ದ-ವ-ರಾಗಿ-ರುತ್ತಿದ್ದರು
ಪಡೆ-ದ-ವರು
ಪಡೆ-ದ-ವರೂ
ಪಡೆ-ದ-ವ-ರೆಲ್ಲಾ
ಪಡೆ-ದಾಗ
ಪಡೆ-ದಿತ್ತು
ಪಡೆ-ದಿತ್ತೆಂಬು-ದನ್ನು
ಪಡೆ-ದಿತ್ತೆಂಬುದು
ಪಡೆ-ದಿದೆ
ಪಡೆ-ದಿದ್ದ
ಪಡೆ-ದಿದ್ದನು
ಪಡೆ-ದಿದ್ದರು
ಪಡೆ-ದಿದ್ದ-ರೆಂದು
ಪಡೆ-ದಿದ್ದಳು
ಪಡೆ-ದಿದ್ದ-ವರು
ಪಡೆ-ದಿದ್ದವು
ಪಡೆ-ದಿದ್ದಾನೆ
ಪಡೆ-ದಿದ್ದಾನೆಂಬುದು
ಪಡೆ-ದಿದ್ದಾರೆ
ಪಡೆ-ದಿದ್ದು
ಪಡೆ-ದಿರ-ಬ-ಹುದು
ಪಡೆ-ದಿರುತ್ತಾನೆ
ಪಡೆ-ದಿರುವ
ಪಡೆ-ದಿ-ರು-ವಂತೆ
ಪಡೆ-ದಿ-ರು-ವು-ದನ್ನು
ಪಡೆ-ದಿ-ರುವುದು
ಪಡೆ-ದಿವೆ
ಪಡೆದು
ಪಡೆ-ದು-ಕೊಂಡ
ಪಡೆ-ದು-ಕೊಂಡನು
ಪಡೆ-ದು-ಕೊಂಡಿದ್ದ-ರಿಂದಲೇ
ಪಡೆ-ದು-ಕೊಂಡಿದ್ದರು
ಪಡೆ-ದು-ಕೊಂಡು
ಪಡೆ-ದು-ಕೊಳ್ಳದೇ
ಪಡೆ-ದು-ಕೊಳ್ಳಲು
ಪಡೆ-ದು-ಕೊಳ್ಳುತ್ತಾನೆ
ಪಡೆ-ದು-ಕೊಳ್ಳುತ್ತಾರೆ
ಪಡೆ-ದುದು
ಪಡೆದೇ
ಪಡೆ-ದೋನ್
ಪಡೆದ್ದಿ-ತೆಂದೂ
ಪಡೆ-ಮಾತೇಂ
ಪಡೆ-ಮೆಚ್ಚೆ-ಗಂಡ
ಪಡೆಯ
ಪಡೆಯಂ
ಪಡೆ-ಯದೇ
ಪಡೆ-ಯನ್ನು
ಪಡೆ-ಯ-ಬೇಕಾಗುತ್ತಿತ್ತು
ಪಡೆ-ಯ-ಬೇ-ಕಾದರೆ
ಪಡೆ-ಯ-ಲಾಗಿದೆ
ಪಡೆ-ಯ-ಲಾ-ಗುತ್ತಿತ್ತೆಂದು
ಪಡೆ-ಯ-ಲಿಲ್ಲ
ಪಡೆ-ಯಲು
ಪಡೆ-ಯ-ವ-ರಾ-ಗಿದ್ದು
ಪಡೆ-ಯ-ವರು
ಪಡೆ-ಯಿತು
ಪಡೆಯು
ಪಡೆ-ಯುತ್ತಾನೆ
ಪಡೆ-ಯುತ್ತಾಳೆ
ಪಡೆ-ಯುತ್ತಿದ್ದ
ಪಡೆ-ಯುತ್ತಿದ್ದನು
ಪಡೆ-ಯುತ್ತಿದ್ದ-ರೆಂದು
ಪಡೆ-ಯುತ್ತಿದ್ದು-ದ-ರಿಂದ
ಪಡೆ-ಯುವ
ಪಡೆ-ಯು-ವಲ್ಲಿಯೇ
ಪಡೆ-ಯು-ವುದು
ಪಡೆ-ಯೆ-ಲೆಯ
ಪಡೆ-ವಳ
ಪಡೆ-ವಳರ
ಪಡೈಕ್ಕಣಕ್ಕನ್
ಪಣ
ಪಣಂ
ಪಣಂಮೂರು
ಪಣಕ್ಕೆ
ಪಣ-ಗಟ್ಟರು
ಪಣದ
ಪಣ-ಮೆ-ರಡು
ಪಣಮ್
ಪಣಮ್ಪಾ-ವಾಡೈ
ಪಣ-ವನ್ನು
ಪಣ-ವಾರು
ಪಣ-ವೆಂಟು
ಪಣ-ವೆ-ರಡು
ಪಣ-ವೆ-ರಡು-ಸೇಸೆ
ಪಣ-ವೇಳು
ಪಣ-ವೈದು
ಪಣ-ವೊಂದು
ಪಣ-ವೊಂಬತ್ತು
ಪಣಾಧಿಕಂ
ಪಣಿ
ಪಣ್ಡಿತ
ಪಣ್ಡಿ-ತರ್ವ್ವಸು-ಮತಿ-ಯೊಳ್
ಪಣ್ಡಿ-ತೋಜ-ನಿಗೆ
ಪಣ್ಯಂಗೆರೆ
ಪಣ್ಯಾ-ಗುಣಂ
ಪತನ
ಪತ-ನ-ಗೊಂಡ-ನಂತರ
ಪತ-ನದ
ಪತನಾ
ಪತನಾ-ನಂತರ
ಪತಾಕ
ಪತಿ
ಪತಿಗೆ
ಪತಿಬ್ರತೆಗೆ
ಪತಿ-ಭಕ್ತಂ
ಪತಿ-ಭಕ್ತಿಗೆ
ಪತಿ-ಭಕ್ತಿ-ಯಲ್ಲಿ
ಪತಿ-ಮೆಚ್ಚೆ-ಮೂ-ವರು
ಪತಿಯ
ಪತಿ-ಯರ
ಪತಿ-ಯರು
ಪತಿ-ಯಾದ
ಪತಿಯು
ಪತಿ-ಯೊಡ-ಗೂಡಿ
ಪತಿವ್ರತಾ-ಗುಣ-ದಿಂದ
ಪತಿ-ಹಿ-ತದೆ-ಯೊಳು
ಪತಿ-ಹಿತಬ್ರತೆ
ಪತಿ-ಹಿ-ತರೂ
ಪತ್ತಂಗಿ
ಪತ್ತನ-ದಲ್ಲಿ-ಹಟ್ಟಣ
ಪತ್ತನಸ್ವಾಮಿ
ಪತ್ತಿನ
ಪತ್ತಿಯ
ಪತ್ತಿರೆ
ಪತ್ತಿಸೆ
ಪತ್ತೆ
ಪತ್ತೆ-ಯಾಗಿದೆ
ಪತ್ತೆ-ಯಾ-ಗಿದ್ದು
ಪತ್ತೆ-ಯಾಗಿ-ರುವ
ಪತ್ತೆ-ಹಚ್ಚಿ
ಪತ್ತೊಂದಿ
ಪತ್ತೊಂದಿ-ಯಲ್ಲಿ
ಪತ್ತೊನ್ದಿ
ಪತ್ತೊನ್ದಿ-ಯನಿಕಿ
ಪತ್ತೊನ್ದಿ-ಯ-ನಿಕ್ಕಿ
ಪತ್ನಿ
ಪತ್ನಿಯ
ಪತ್ನಿ-ಯನ್ನು
ಪತ್ನಿ-ಯರ
ಪತ್ನಿ-ಯ-ರ-ದಾಗಿ-ರುತ್ತಿತ್ತೆಂದು
ಪತ್ನಿ-ಯ-ರನ್ನು
ಪತ್ನಿ-ಯ-ರಾದ
ಪತ್ನಿ-ಯ-ರಿಗೂ
ಪತ್ನಿ-ಯ-ರಿದ್ದಂತೆ
ಪತ್ನಿ-ಯರು
ಪತ್ನಿ-ಯರೂ
ಪತ್ನಿ-ಯಾಗಿ-ರ-ಬಹು-ದೆಂದು
ಪತ್ನಿ-ಯಾದ
ಪತ್ನಿ-ಯೊರಡನೆ
ಪತ್ಮಾ
ಪತ್ಮಾ-ವಸುಂಧರಾಭ್ಯಾಮಾಕಲ್ಪಂ
ಪತ್ಯ-ಬನ
ಪತ್ರ
ಪತ್ರ-ಕರ್ಮ್ಮ
ಪತ್ರ-ಗಳ
ಪತ್ರ-ಗ-ಳನ್ನು
ಪತ್ರ-ಗಳು
ಪತ್ರದ
ಪತ್ರ-ವನ್ನು
ಪತ್ರವ್ಯವ-ಹಾರ-ಗ-ಳನ್ನು
ಪತ್ರ-ಸಾ-ಸನವ
ಪಥ-ದಲ್ಲಿ
ಪದ
ಪದ-ಕಮಳ
ಪದ-ಕ-ವನ್ನು
ಪದ-ಕೋಶ-ಗಳು
ಪದಕ್ಕೂ
ಪದಕ್ಕೆ
ಪದ-ಗಳ
ಪದ-ಗ-ಳನ್ನು
ಪದ-ಗ-ಳಾಗಿ
ಪದ-ಗ-ಳಾಗಿದ್ದು
ಪದ-ಗಳಿಗೂ
ಪದ-ಗಳು
ಪದ-ಗಳುಈ
ಪದ-ಗಳೂ
ಪದದ
ಪದ-ದಲ್ಲಿ-ರುವ
ಪದ-ದಿಂದ
ಪದ-ಪದ್ಮಾ-ರಾಧಕ-ರು-ಮಪ್ಪ
ಪದಪಿಂ
ಪದಪ್ರ-ಯೋಗ
ಪದಪ್ರ-ಯೋಗ-ದಿಂದ
ಪದಪ್ರ-ಯೋಗ-ವಿದೆ
ಪದಪ್ರ-ಯೋಗ-ವಿದೆ-ಯೆಂದು
ಪದಮಂ
ಪದ-ಯುಗ್ಮಾಂಭೋಜ
ಪದ-ವನ್ನು
ಪದ-ವನ್ನೈದಿ-ದ-ನೆಂದೂ
ಪದ-ವಾಕ್ಯ
ಪದ-ವಾಕ್ಯಪ್ರಮಾಣಜ್ಞ
ಪದ-ವಾಗಿ-ರ-ಲಿಲ್ಲ
ಪದವಿ
ಪದವಿ-ಗ-ಳನ್ನು
ಪದವಿ-ಗ-ಳನ್ನೂ
ಪದವಿ-ಗಳು
ಪದ-ವಿಗೆ
ಪದವಿ-ಗೇರುತ್ತಿದ್ದ-ರೆಂದು
ಪದವಿದೌ
ಪದವಿಯ
ಪದವಿ-ಯನ್ನು
ಪದವಿ-ಯನ್ನೂ
ಪದವಿ-ಯಲ್ಲಿ
ಪದವಿ-ಯಾ-ಗಿತ್ತು
ಪದವಿ-ಯಾ-ಗಿದ್ದು
ಪದವಿ-ಯಿಂದ
ಪದವಿಯು
ಪದವಿ-ಯೆಂದು
ಪದವೀ
ಪದವು
ಪದವೂ
ಪದಾಂಬೋಜ
ಪದಾಕ್ರಾಂತ-ವಾದ
ಪದಾತಿ
ಪದಾದ್ವಿಪ್ರ-ಗಣೇ
ಪದಾಬ್ಜಂಗ-ಳ-ವರ್ಗ್ಗೆ
ಪದಾಬ್ಜ-ಗಳು
ಪದಾ-ರಾಧಕನುಂ
ಪದಾರ್ಥ
ಪದಾರ್ಥ-ಗಳ
ಪದಾರ್ಥ-ಗ-ಳನ್ನು
ಪದಾರ್ಥ-ಗಳಲ್ಲ
ಪದಾರ್ಥ-ಗಳು
ಪದಿ-ಗಳು
ಪದಿ-ನಾಲ್ಕು
ಪದಿ-ನಾಲ್ಕು-ನಾಡಿನ
ಪದು-ಮಂಣನ
ಪದುಮಣ್ಣ
ಪದುಮಣ್ಣನ
ಪದುಮಣ್ಣ-ನ-ವರ
ಪದುಮಣ್ಣನು
ಪದುಮಣ್ಣ-ಸೆಟ್ಟರ
ಪದುಮಣ್ಣ-ಸೆಟ್ಟಿಯ
ಪದುಮ-ನಾಭ-ನಾಯಕ
ಪದುಮ-ನಾಭ-ಪುರದ
ಪದುಮ-ನಾಯ-ಕರ
ಪದು-ಮಲ-ದೇವಿ
ಪದು-ಮಲ-ದೇವಿ-ಯರ
ಪದುಳಂ
ಪದು-ಳದಿಂ
ಪದೇ
ಪದೇ-ಪದೇ
ಪದೋನ್ನತಿ
ಪದ್ಧತಿ
ಪದ್ಧತಿ-ಗ-ಳನ್ನು
ಪದ್ಧತಿ-ಗಳು
ಪದ್ಧತಿಗೆ
ಪದ್ಧತಿಯ
ಪದ್ಧತಿ-ಯನ್ನು
ಪದ್ಧತಿ-ಯ-ವರು
ಪದ್ಧತಿ-ಯಿಂದ
ಪದ್ಧತಿಯು
ಪದ್ಧತಿಯೂ
ಪದ್ಧತಿಯೇ
ಪದ್ಮ
ಪದ್ಮ-ಕುಲಕ್ಕೆ
ಪದ್ಮ-ಕುಲದ
ಪದ್ಮಜ
ಪದ್ಮ-ನಂದಿ
ಪದ್ಮ-ನಂದಿ-ಯಾ-ಗಿದ್ದು
ಪದ್ಮ-ನಂದಿಯು
ಪದ್ಮ-ನನ್ದಿ
ಪದ್ಮ-ನನ್ದಿ-ಮುನಿಪ
ಪದ್ಮ-ನಾಭನ
ಪದ್ಮ-ನಾಭ-ಪಂಡಿತ
ಪದ್ಮ-ನಾಭ-ಪುರ
ಪದ್ಮ-ನಾಭ-ಪುರದ
ಪದ್ಮ-ನಾಭ-ಪುರ-ವಾದ
ಪದ್ಮ-ಪಾದಾ-ಚಾರ್ಯರು
ಪದ್ಮಪ್ರಭ
ಪದ್ಮಪ್ರಭ-ನೆಂದು
ಪದ್ಮ-ರ-ವರು
ಪದ್ಮ-ರಾ-ಜಯ್ಯ
ಪದ್ಮ-ರಾಸಿ
ಪದ್ಮ-ಲ-ದೇವಿ
ಪದ್ಮ-ಲ-ದೇವಿಯ
ಪದ್ಮ-ಲಾಖ್ಯಾ
ಪದ್ಮ-ಲಾ-ದೇವಿ-ಯನ್ನು
ಪದ್ಮಾ
ಪದ್ಮಾ-ವತಿ
ಪದ್ಮಾ-ವತಿ-ದೇವೀಲಬ್ಧ-ವರಪ್ರಸಾದಂ
ಪದ್ಮಾ-ಸನ-ದಲ್ಲಿ
ಪದ್ಮೋಪ-ಜೀವಿ-ಯಾಗಿ
ಪದ್ಯ
ಪದ್ಯ-ಗಳ
ಪದ್ಯ-ಗಳಂತೂ
ಪದ್ಯ-ಗ-ಳನ್ನು
ಪದ್ಯ-ಗಳನ್ನೊಳ-ಗೊಂಡ
ಪದ್ಯ-ಗ-ಳಲ್ಲಿ
ಪದ್ಯ-ಗಳಿವೆ
ಪದ್ಯ-ಗಳು
ಪದ್ಯ-ಗಳೂ
ಪದ್ಯ-ಗಳೆನಿಸಿ-ಕೊಂಡಿವೆ
ಪದ್ಯದ
ಪದ್ಯ-ದಲ್ಲಿ
ಪದ್ಯ-ದಿಂದ
ಪದ್ಯ-ರೂಪ-ದಲ್ಲಿ
ಪದ್ಯವ
ಪದ್ಯವು
ಪದ್ಯವೂ
ಪದ್ಯ-ವೊಂದು
ಪನಃ
ಪನರ್
ಪನೆ-ಕೊಳ
ಪನ್ನಗವೈ-ನತೇಯ-ನೆನಿಸಿದ್ದ
ಪನ್ನಗವೈರಿ
ಪನ್ನಗವೈ-ರಿಗೆ
ಪನ್ನಗಶಾಯಿ-ಯಾದ
ಪನ್ನಗಶಾಯೀ
ಪನ್ನರ್ವ್ವರ್ಗ್ಗಂ
ಪನ್ನಾಯ
ಪನ್ನಾಯ-ಪಂನಾಯ
ಪನ್ನಾಯ-ವನ್ನು
ಪನ್ನಾಯ-ವೆಂಬ
ಪನ್ನಾಸು
ಪನ್ನಿರ್ಚ್ಛಾ-ಸಿರ
ಪನ್ನಿರ್ಛಾ-ಸಿರ
ಪನ್ನಿರ್ವ್ವರ್ಗೆ
ಪನ್ನೀರು
ಪನ್ನೆರಡಕ್ಕೆ
ಪನ್ನೆರಡನ್ನು
ಪನ್ನೆ-ರಡರ
ಪನ್ನೆರಡ-ರಲ್ಲಿದ್ದ
ಪನ್ನೆ-ರಡು
ಪಪಣ
ಪಪ್ಪರ-ಹಳ್ಳ-ಗಳ
ಪಯಣ-ಬೆಳೆಸಿ
ಪಯೋಜ-ಭಾನು
ಪಯೋಧಿ
ಪರ
ಪರಂ
ಪರಂಪರಾಗತ
ಪರಂಪರಾನುಗತ-ವಾ-ಗಿತ್ತು
ಪರಂಪರೆ
ಪರಂಪರೆ-ಗಳ
ಪರಂಪ-ರೆಗೆ
ಪರಂಪ-ರೆಯ
ಪರಂಪರೆ-ಯನ್ನು
ಪರಂಪರೆ-ಯಲ್ಲಿ
ಪರಂಪರೆ-ಯ-ವ-ನಿರ-ಬಹು-ದೆಂದು
ಪರಂಪರೆ-ಯ-ವನು
ಪರಂಪರೆ-ಯ-ವನೇ
ಪರಂಪರೆ-ಯ-ವ-ರಾಗಿರ
ಪರಂಪರೆ-ಯ-ವ-ರಾಗಿ-ರ-ಬ-ಹುದು
ಪರಂಪರೆ-ಯ-ವರು
ಪರಂಪರೆ-ಯ-ವ-ರೆಂದು
ಪರಂಪರೆ-ಯ-ವರೇ
ಪರಂಪರೆ-ಯಾಗಿ
ಪರಂಪರೆ-ಯಿಂದ
ಪರಂಪ-ರೆಯು
ಪರಂಪ-ರೆಯೂ
ಪರಂಪ-ರೆಯೇ
ಪರ-ಕ-ಪಾಂಡ್ಯರು
ಪರ-ಕಾಲ
ಪರ-ಕಾಲ-ಮಠದ
ಪರ-ಕಾಲ-ಮಠವು
ಪರ-ಕಾಲ-ಯ-ತಿ-ಗಳು
ಪರ-ಕಾಲಸ್ವಾಮಿ-ಯ-ವರು
ಪರ-ಕಾಷ್ಠೆ-ಯನ್ನು
ಪರ-ಕೀಯರ
ಪರ-ಕೇ-ಸರಿ
ಪರ-ಟಿ-ಗಳು
ಪರ-ಡಿ-ಸೆಟ್ಟಿ
ಪರ-ಡಿ-ಸೆಟ್ಟಿಗು
ಪರದ
ಪರ-ದತ್ತಂ
ಪರದರ
ಪರದ-ರ-ಕುಲದ
ಪರದಾರ-ಸ-ಹೋದರಃ
ಪರ-ದೇಶ-ಪರಸ್ಥಳ-ವೆಂದು
ಪರ-ದೇಶಿ
ಪರ-ದೇಶಿ-ಗ-ಳೆಂದು
ಪರ-ದೇಶಿ-ಗ-ಳೆಂದೂ
ಪರ-ದೇಶಿ-ಗೌಡರು
ಪರ-ದೇಶಿ-ಯಪ್ಪನ
ಪರ-ದೇಶಿ-ಯಪ್ಪರ
ಪರ-ದೇಶಿ-ಯರ
ಪರ-ದೇಶಿಸಿ
ಪರ-ದೇಸಿ
ಪರ-ದೇಸಿ-ಯಪ್ಪ
ಪರ-ದೇಸಿ-ಯಪ್ಪನ
ಪರ-ದೇಸಿ-ಯಪ್ಪ-ನಿಗೆ
ಪರ-ದೇಸಿ-ಯಪ್ಪನು
ಪರ-ದೇಸಿ-ಯಪ್ಪ-ರಾದ
ಪರ-ದೇಸೀ-ಗೌಡ
ಪರ-ನಾರಿ
ಪರನಾರೀ-ದೂ-ರನುಂ
ಪರನಾರೀ-ಪುತ್ರನುಂ
ಪರನಾರೀ-ಸೋ-ದರ
ಪರ-ಪುರುಷಾರ್ತ್ತ-ಚರಿತನು
ಪರ-ಬಲ-ಭೀಮ
ಪರ-ಬಳ-ಕಕ್ಷ
ಪರ-ಬಳ-ಕೃ-ತಾಂತ
ಪರ-ಬಳ-ಜಳಧಿಬಡ-ಬಾನಳಂ
ಪರ-ಬಳ-ಭಕ್ಷಕ
ಪರ-ಭಾರೆ
ಪರಮ
ಪರ-ಮಂಡ-ಳಮಂ
ಪರ-ಮ-ಕಲ್ಯಾಣಾಭ್ಯು-ದಯ
ಪರ-ಮ-ಗುರು-ಗ-ಳಾಗಿದ್ದಂತಹ
ಪರ-ಮ-ಗೂಳ
ಪರ-ಮ-ಗೂಳನ
ಪರ-ಮ-ಗೂಳನು
ಪರ-ಮ-ಗೂಳನೂ
ಪರ-ಮ-ಗೂಳ-ನೆಂದು
ಪರ-ಮ-ಗೂಳ-ನೆಂದೂ
ಪರ-ಮ-ಗೂಳ-ಪತ್ನಿ
ಪರ-ಮ-ಜೈ-ನರೂ
ಪರ-ಮ-ತಭಂಜನ
ಪರ-ಮ-ತಸಹಿಷ್ಣತೆ
ಪರ-ಮ-ದೇವ-ತೆ-ಯಾಗಿ
ಪರ-ಮ-ದೇವನ
ಪರ-ಮ-ನೈಷ್ಠಿಕ
ಪರ-ಮ-ನೈಷ್ಠಿಕಾ
ಪರ-ಮ-ಪದ
ಪರ-ಮ-ಪದ-ವನೆ-ಯಿದಿ
ಪರ-ಮ-ಪದ-ವನ್ನೈದಿ-ಳೆಂದು
ಪರ-ಮಪ್ರಕಾಶ
ಪರ-ಮಪ್ರಕಾಶ-ಯೋಗೀಶ್ವರನ
ಪರ-ಮಪ್ರೇಮ-ದಿಂದ
ಪರ-ಮಬ್ಬೆ
ಪರ-ಮಬ್ಬೆಯು
ಪರ-ಮ-ಭಾಗ-ವತ
ಪರ-ಮ-ಭಾಗ-ವತ-ನಾ-ಗಿದ್ದು
ಪರ-ಮ-ಭಾಗ-ವತೋತ್ತಮೆ
ಪರ-ಮ-ಭಾಗ-ವತೋತ್ತಮೆ-ಯಾದ
ಪರ-ಮ-ವಿಶ್ವಾಸಿ
ಪರ-ಮ-ವೈದಿಕ
ಪರ-ಮಶ್ರೀ-ವೈಷ್ಣವ
ಪರ-ಮಶ್ರೀ-ವೈಷ್ಣ-ವ-ನಾದ
ಪರ-ಮ-ಹಂಸ
ಪರ-ಮಾತ್ಮನ
ಪರ-ಮಾತ್ಮ-ನನ್ನು
ಪರ-ಮಾತ್ಮ-ನಿಂದ
ಪರ-ಮಾತ್ಮನು
ಪರ-ಮಾರ
ಪರ-ಮಾರರ
ಪರ-ಮಾರ್ತ್ಥ
ಪರ-ಮಾರ್ತ್ಥಂ
ಪರ-ಮುದು
ಪರ-ಮೇಶ್ವರ
ಪರ-ಮೇಶ್ವರ-ರಿಗೆ
ಪರ-ಮೇಶ್ವರ-ವರ್ಮ-ನಿರ-ಬ-ಹುದು
ಪರ-ಮೋಚ್ಛ
ಪರ-ರಾಜ-ಭಯಂಕರ
ಪರ-ರಾಜ-ಭಯಂಕರಃ
ಪರ-ರಾಯ
ಪರ-ರಾಯ-ಭಯಂಕರಃ
ಪರ-ರಾಷ್ಟ್ರ
ಪರ-ಲೋಕ
ಪರ-ವಾಗಿ
ಪರ-ವಾಗಿಯೋ
ಪರ-ವಾಗಿಲ್ಲ
ಪರ-ವಾದಿ
ಪರ-ವಾದಿ-ಮಲ್ಲ
ಪರವೆಂಡಿರಣ್ಣ
ಪರವೆಣ್ಡಿರಣ್ನನೀ-ಸರಯ್ಯ
ಪರಸ-ದೇಸೀ-ಗೌಡ
ಪರ-ಸರಿ-ಸಿತ್ತು
ಪರ-ಸಲು
ಪರಸ್ಥಳ-ದ-ವ-ರಿಗೆ
ಪರಸ್ಥಳ-ದಿಂದ
ಪರಸ್ಪರ
ಪರಾಂ
ಪರಾಂಕುಶ
ಪರಾಂಕುಶ-ಜೀಯರು
ಪರಾಂಕುಶ-ಸ-ಮುದ್ರ-ಕೆರೆ
ಪರಾಂಗನಾ-ಪುತ್ರ
ಪರಾಂತ-ಕನ
ಪರಾಂತಕ-ನಿಗೆ
ಪರಾಕು
ಪರಾಕ್ರಮ
ಪರಾಕ್ರಮಕ್ಕೆ
ಪರಾಕ್ರಮ-ಗ-ಳಿಂದ
ಪರಾಕ್ರಮ-ದಿಂದ
ಪರಾಕ್ರಮ-ವನ್ನು
ಪರಾಕ್ರಮ-ವುಳ್ಳ
ಪರಾಕ್ರಮಿ-ಯಾದ
ಪರಾಜ-ಯ-ನ-ವರ
ಪರಾಜಯ್ಯ-ನ-ವರ
ಪರಾಜಿತ-ಗೊಂಡ
ಪರಾಧೀ-ನತೆ-ಯಿಂದ
ಪರಾಭವ-ಗೊ-ಳಿಸಿ
ಪರಾಭಿ-ದಾನ
ಪರಾಮರ್ಶನ
ಪರಾಮರ್ಶೆ
ಪರಾಯ-ಣನೂ
ಪರಾಯ-ಣ-ರಪ್ಪ
ಪರಾಯ-ಣರೂ
ಪರಾಯ-ಣಾನ್
ಪರಾರಿ-ಯಾದರು
ಪರಿಗ-ಣಿತ-ವಾಯಿತು
ಪರಿಗ-ಣಿಸ-ಬ-ಹುದು
ಪರಿಗ-ಣಿ-ಸ-ಲಾ-ಗಿತ್ತು
ಪರಿ-ಗಣಿಸಿ
ಪರಿ-ಗಣಿಸಿ-ದರೆ
ಪರಿ-ಗಣಿಸಿ-ರುತ್ತಾರೆ
ಪರಿ-ಗೆರೆ
ಪರಿಚಯ
ಪರಿಚ-ಯವೂ
ಪರಿಚಯಾತ್ಮಕ
ಪರಿ-ಚಾರಕ
ಪರಿ-ಚಾರ-ಕರ
ಪರಿ-ಚಾರ-ಕ-ರನ್ನು
ಪರಿ-ಚಾರ-ಕ-ರಿಗೆ
ಪರಿ-ಚಾರ-ಕರು
ಪರಿ-ಚಾರ-ಕ-ರೆಲ್ಲ
ಪರಿ-ಜನಕಂ
ಪರಿ-ಜನಕ್ಕೆ
ಪರಿ-ಜನ-ಪರಿವೃತ-ನಾಗಿ
ಪರಿಜೃಂಭಮಾಣೇ
ಪರಿಣತ
ಪರಿಣತ-ನಲ್ಲ
ಪರಿಣತ-ನಾಗಿದ್ದ-ನೆಂದು
ಪರಿಣ-ತನೂ
ಪರಿಣತ-ನೆಂದು
ಪರಿಣತ-ರಾಗಿದ್ದಂತೆ
ಪರಿಣತ-ರಾಗಿದ್ದರು
ಪರಿಣತ-ರಾದ
ಪರಿಣತ-ರಾದ-ವ-ರನ್ನು
ಪರಿಣತ-ರಿದ್ದ
ಪರಿಣತ-ರಿದ್ದರು
ಪರಿಣ-ತರು
ಪರಿಣತೆ-ಯನ್ನು
ಪರಿಣಮಿಸಿ
ಪರಿಣಮಿಸಿ-ದನು
ಪರಿಣಾಮ
ಪರಿಣಾಮ-ವಾಗಿ
ಪರಿತಾಪಹತ
ಪರಿದ
ಪರಿ-ಪಾಠ
ಪರಿ-ಪಾಠ-ವಿದೆ
ಪರಿ-ಪಾಲ-ಕ-ರಾದ
ಪರಿ-ಪಾಲಿತ
ಪರಿ-ಪಾಲಿತ-ವಾದ
ಪರಿ-ಪಾಲಿಸಲ್ಪಟ್ಟ
ಪರಿ-ಪಾಲಿಸು-ವಂತೆ
ಪರಿ-ಪೂರ್ಣ-ತೆ-ಯನ್ನು
ಪರಿಭವಿಪಂ
ಪರಿ-ಭಾಷೆಯ
ಪರಿ-ಭಾಷೆ-ಯಲ್ಲಿ
ಪರಿ-ಭಾಷೆ-ಯಾಗಿದೆ
ಪರಿ-ಭಾಷೆ-ಯಿಂದ
ಪರಿಮಳ
ಪರಿಯಂ
ಪರಿಯಂಕ
ಪರಿಯೆಂತು
ಪರಿ-ವಂತೆ
ಪರಿ-ವರ್ತನಕ್ಕೆ
ಪರಿ-ವರ್ತನ-ಗಳೆಂಬ
ಪರಿ-ವರ್ತನೆ
ಪರಿ-ವರ್ತನೆ-ಯಾಗ
ಪರಿ-ವರ್ತನೆ-ಯಾಗ-ಲಿಲ್ಲ
ಪರಿ-ವರ್ತನೆ-ಯಾಗಿ-ದೆ-ಯೆಂದು
ಪರಿ-ವರ್ತನೆ-ಯಾಗಿ-ರ-ಬ-ಹುದು
ಪರಿ-ವರ್ತನೆ-ಯಾ-ಗುತ್ತಾ
ಪರಿ-ವರ್ತನೆ-ಯಾ-ಗುತ್ತಿತ್ತೆಂದು
ಪರಿ-ವರ್ತನೆ-ಯಾದ
ಪರಿ-ವರ್ತನೆ-ಯಾದವು
ಪರಿ-ವರ್ತಿ-ತ-ರಾದ
ಪರಿ-ವರ್ತಿ-ತ-ವಾಗಿದೆ
ಪರಿ-ವರ್ತಿ-ತವಾ-ಗುತ್ತಿತ್ತೆಂದು
ಪರಿ-ವರ್ತಿ-ತ-ವಾಯಿ-ತೆಂದು
ಪರಿ-ವರ್ತಿ-ತ-ವಾಯಿ-ತೆಂಬುದು
ಪರಿ-ವರ್ತಿ-ಸಲಾಯಿ-ತೆಂದು
ಪರಿ-ವರ್ತಿಸಿ
ಪರಿ-ವರ್ತಿ-ಸಿ-ದ-ನೆಂದು
ಪರಿ-ವರ್ತಿ-ಸಿ-ದ-ನೆಂಬುದು
ಪರಿ-ವರ್ತಿ-ಸಿ-ದಾಗ
ಪರಿ-ವರ್ತಿಸು
ಪರಿ-ವರ್ತಿ-ಸು-ವು-ದಕ್ಕೆ
ಪರಿ-ವಾರ
ಪರಿ-ವಾರದ
ಪರಿ-ವಾರ-ದ-ವರ
ಪರಿ-ವಾರ-ದ-ವ-ರೊಂದಿಗೆ
ಪರಿ-ವಾರ-ದೇವ-ತೆ-ಗಳು
ಪರಿ-ವಾರ-ಪಾರಪಾರ್ಥಿತ
ಪರಿವೃಢಃ
ಪರಿವೇಷ-ವನ್ನು
ಪರಿವೇಷ್ಟಿತೇ
ಪರಿವ್ರಾಜ-ಕಾ-ಚಾರ್ಯ
ಪರಿಶಿಷ್ಟ-ಜಾತಿ
ಪರಿಶೀಲ-ನಾರ್ಹ
ಪರಿಶೀಲನೆ
ಪರಿಶೀಲ-ನೆಗೆ
ಪರಿಶೀಲನೆ-ಯಿಂದ
ಪರಿಶೀಲಿ-ಸದಾಗ
ಪರಿಶೀಲಿಸ-ಬ-ಹುದು
ಪರಿಶೀಲಿಸಬೇಕಾಗುತ್ತದೆ
ಪರಿಶೀಲಿಸ-ಬೇ-ಕಾದ
ಪರಿಶೀಲಿಸ-ಲಾಗಿದೆ
ಪರಿಶೀಲಿ-ಸಲು
ಪರಿಶೀಲಿಸಿ
ಪರಿಶೀಲಿಸಿ-ದಾಗ
ಪರಿಶೀಲಿ-ಸಿದ್ದು
ಪರಿಶೀಲಿಸುತ್ತಿದ್ದರು
ಪರಿಷಂಪ್ರತಿ
ಪರಿಷತ್ತಿನ
ಪರಿಷತ್ತಿ-ನಲ್ಲಿ
ಪರಿಷತ್ತಿನ-ವರು
ಪರಿಷೆ
ಪರಿಷೆ-ಯನ್ನು
ಪರಿಷ್ಕರಿ-ಸ-ಬೇಕು
ಪರಿಷ್ಕರಿ-ಸ-ಲಾಗುತ್ತಿತ್ತು
ಪರಿಷ್ಕರಿ-ಸಲು
ಪರಿಷ್ಕರಿಸಿ
ಪರಿಷ್ಕಾರ
ಪರಿಷ್ಕೃತ
ಪರಿಷ್ಕೃತ-ಗೊಂಡ
ಪರಿ-ಸರಕ್ಕೆ
ಪರಿಸ-ರದ
ಪರಿ-ಸರ-ದಲ್ಲಿ
ಪರಿಸ-ರದಲ್ಲಿ-ರುವ
ಪರಿಸ-ರದ್ದೆಂದು
ಪರಿಸ-ರವೇ
ಪರಿಸೆ
ಪರಿಸೇವ್ಯ-ಮಾನಃ
ಪರಿಸ್ಥಿತಿ
ಪರಿಸ್ಥಿತಿ-ಗಳ
ಪರಿಸ್ಥಿತಿ-ಯನ್ನು
ಪರಿಸ್ಥಿತಿ-ಯಲ್ಲಿ
ಪರಿ-ಹರಿಸಿ
ಪರಿ-ಹರಿ-ಸಿ-ಕೊಟ್ಟು
ಪರಿ-ಹರಿ-ಸಿ-ಕೊಡಲು
ಪರಿ-ಹರಿ-ಸಿ-ಕೊಡುತ್ತಾರೆ
ಪರಿ-ಹರಿ-ಸಿ-ಕೊಡು-ವಂತೆ
ಪರಿ-ಹರಿ-ಸಿ-ಕೊಡು-ವರು
ಪರಿ-ಹರಿ-ಸಿ-ಕೊಡು-ವು-ದಾಗಿ
ಪರಿ-ಹಾರ
ಪರಿ-ಹಾರ-ಮಾಗಿ
ಪರಿ-ಹಾರ-ವಾಗಿ
ಪರೀಕ್ಷಿಸಿ
ಪರೀಕ್ಷೆ-ಗ-ಳಲ್ಲಿ
ಪರೀಕ್ಷೆ-ಯಲ್ಲಿ
ಪರೀಕ್ಷ್ಯ
ಪರೀಧಾವಿ
ಪರುಪು
ಪರು-ಬೊಮ್ಮವ್ವೆ-ಯಚಳ
ಪರುವಿ
ಪರುಷೆ
ಪರುಷೆಯ
ಪರುಷೆ-ಯಲ್ಲಿ
ಪರೋಕ್ಷ
ಪರೋಕ್ಷ-ವಾಗಿ
ಪರೋಕ್ಷ-ವಿನ-ಯ-ವಾಗಿ
ಪರೋಪ-ಕಾರಿ-ಗ-ಳಾದ
ಪರ್ಗ್ಗೊಣ್ಣಿನ್ದ
ಪರ್ಯಂತ
ಪರ್ಯಾಯ
ಪರ್ಯಾಯ-ದಲು
ಪರ್ಯ್ಯಾಯ-ದಲು
ಪರ್ವ-ಗ-ಳನ್ನು
ಪರ್ವತ
ಪರ್ವ-ತ-ಕಂಬಿ-ಕಾವಡಿ
ಪರ್ವ-ತಾ-ವಳಿ-ಯೆಂದು
ಪರ್ವಿ-ಯಲ್ಲಿ
ಪರ್ಷಿ-ಯನ್
ಪರ್ಷೆ
ಪರ್ಷೆ-ಪರಿಸೆ
ಪಲಂಬರುಂ
ಪಲಕ್ಕಿ
ಪಲಗೈಯಾನುಮ್
ಪಲರ್ಪ್ಪೊ-ಗೞ್ದೆ-ನದಟಿಂ
ಪಲ-ವಾಳುಂ
ಪಲವುಂ
ಪಲ-ಸಿಗೆ
ಪಲಾ-ಯನ
ಪಲಾ-ಯನಂ
ಪಲಾ-ಯನ-ಮಾಡಿ
ಪಲ್ಲ
ಪಲ್ಲಕ್ಕಿ
ಪಲ್ಲಕ್ಕಿಯ
ಪಲ್ಲಕ್ಕಿ-ಯಂತಹ
ಪಲ್ಲಕ್ಕಿ-ಯನ್ನು
ಪಲ್ಲಕ್ಕಿ-ಯಲ್ಲಿ
ಪಲ್ಲ-ಟ-ವಾಗಿದೆ
ಪಲ್ಲ-ಪಂಡಿತ
ಪಲ್ಲ-ಪಂಡಿ-ತ-ರಿಗೆ
ಪಲ್ಲ-ಪೆರಿ-ಯೂರಿ-ನಲ್ಲಿ
ಪಲ್ಲ-ಪೆರಿ-ಯೂರ್ನಲ್ಲಿ
ಪಲ್ಲವ
ಪಲ್ಲ-ವ-ಕುಲದ
ಪಲ್ಲ-ವ-ತಟಾಕ
ಪಲ್ಲ-ವ-ತಟಾಕ-ವೆಂಬ
ಪಲ್ಲ-ವ-ತಟಾಕಾ
ಪಲ್ಲ-ವತ್ರಿಣೇತ್ರ
ಪಲ್ಲ-ವನ್ವಾಯ
ಪಲ್ಲ-ವ-ಮಲ್ಲ
ಪಲ್ಲ-ವರ
ಪಲ್ಲ-ವ-ರನ್ನು
ಪಲ್ಲ-ವ-ರನ್ನೇ
ಪಲ್ಲ-ವ-ರಾಜ-ವಂಶದ
ಪಲ್ಲ-ವ-ರಾಯ
ಪಲ್ಲ-ವ-ರಾಯ-ನನ್ನು
ಪಲ್ಲ-ವ-ರಾಯನು
ಪಲ್ಲ-ವ-ರಾಯ-ನೆಂಬ
ಪಲ್ಲ-ವರು
ಪಲ್ಲ-ವ-ರೆಂದು
ಪಲ್ಲ-ವ-ರೊಡನೆ
ಪಲ್ಲ-ವ-ವಂಶ-ದ-ವ-ರೆಂದು
ಪಲ್ಲ-ವಾ-ದಿತ್ಯ
ಪಲ್ಲ-ವಾಧಿ-ರಾಜ
ಪಲ್ಲ-ವಾಧಿ-ರಾಜನ
ಪಲ್ಲ-ವಾಧಿ-ರಾಜನು
ಪಲ್ಲ-ವಾಧಿ-ರಾಜನೂ
ಪಲ್ಲ-ವಾಧಿ-ರಾಜರ
ಪಲ್ಲ-ವೆಂದ್ರ-ನನ್ನು
ಪಳಂಚಿ
ಪಳಗಿದ-ವ-ರೆಂಬ
ಪಳ-ಗಿದ್ದ
ಪಳಗಿಸಿ
ಪಳಗಿ-ಸುವ-ವರ
ಪಳಗಿಸು-ವ-ವ-ರಿಗೆ
ಪಳಗಿಸು-ವುದ-ರಲ್ಲಿ
ಪಳಿಯುಲನು
ಪಳಿಯು-ಳನ
ಪಳ್ಳದಿಂ
ಪಳ್ಳಿ-ಗರ
ಪವಾಡ-ವನ್ನು
ಪವಿತ್ರ
ಪವಿತ್ರಕ್ಕೆ
ಪವಿತ್ರಕ್ಕೆ-ಅ್ಗ
ಪವಿತ್ರ-ಜಲ-ವನ್ನು
ಪವಿತ್ರ-ವಾದ
ಪವಿತ್ರಾಗ್ನಿ-ಯನ್ನು
ಪವಿತ್ರಾರೋ-ಹಣ
ಪವಿತ್ರಾ-ಲಯ-ಗಳ
ಪವಿತ್ರೀಕೃ-ತೋತ್ತಮಾಂಗನುಂ
ಪವಿತ್ರೀಕೃ-ತೋತ್ತಮಾಂಗರುಂ
ಪವಿತ್ರೀಕೃ-ತೋತ್ತಮಾಂಗೆಯುಂ
ಪವಿತ್ರೀಕ್ರಿ-ತೋತ್ತಮಾಂಗ
ಪವಿತ್ರೀಕ್ರಿ-ತೋತ್ತಮಾಂಗೆ-ಯರುಂ
ಪವಿತ್ರೆ
ಪಶು
ಪಶು-ಪತಿ
ಪಶು-ಪಾಲ-ನೆ-ಗಾಗಿ
ಪಶ್ಚಿಮ
ಪಶ್ಚಿಮಕ್ಕೂ
ಪಶ್ಚಿಮಕ್ಕೆ
ಪಶ್ಚಿಮದ
ಪಶ್ಚಿಮ-ದಿಕ್ಕಿನಲ್ಲಿ
ಪಶ್ಚಿಮ-ಭಾಗ
ಪಶ್ಚಿಮ-ರಂಗ
ಪಶ್ಚಿಮ-ರಂಗಕ್ಷೇತ್ರ-ದಲ್ಲಿ
ಪಶ್ಚಿಮ-ರಂಗದ
ಪಶ್ಚಿಮ-ರಂಗನ
ಪಶ್ಚಿಮ-ರಂಗ-ನಾಥ
ಪಶ್ಚಿಮ-ರಂಗ-ನಾಥಸ್ವಾಮಿ
ಪಶ್ಚಿಮ-ರಂಗ-ನಾಥಸ್ವಾಮಿಯ
ಪಶ್ಚಿಮ-ರಂಗ-ನಾಥಸ್ವಾಮಿ-ಯ-ವರ
ಪಶ್ಚಿಮ-ರಂಗ-ರಾಜ-ನ-ಗರೀ
ಪಶ್ಚಿಮ-ರಂಗೇ
ಪಶ್ಚಿಮ-ವಾ-ಹಿನಿ
ಪಶ್ಚಿಮ-ವಾ-ಹಿನಿ-ಯಲ್ಲಿ
ಪಶ್ಚಿಮ-ವಾ-ಹಿನಿ-ಯಾಗಿ
ಪಶ್ಚಿಮೋತ್ತರ-ದಲ್ಲಿ-ರುವ
ಪಶ್ಚಿಮೋತ್ತರ-ವಾಗಿ
ಪಷಟ್ಟು-ಪಟ್ಟಾಳಮ್ನ
ಪಸಯಿತ-ನಿಗೆ
ಪಸರಿಸಿ
ಪಸರಿ-ಸಿತು
ಪಸರಿ-ಸಿತ್ತು
ಪಸರಿ-ಸಿತ್ತೆಂದೂ
ಪಸರಿ-ಸಿದ್ದು
ಪಸಾಯತರು
ಪಸಾಯಿತ
ಪಸಾಯಿತ-ನಿಗೆ
ಪಸಾಯ್ತ
ಪಸುವುಂ
ಪಹ-ರದು
ಪಾಂಚ-ಜನ್ಯ
ಪಾಂಚನ-ನದ-ಪಾಂಚಾ-ಲರು
ಪಾಂಚ-ಮಠ
ಪಾಂಚರಾತ್ರ
ಪಾಂಚರಾತ್ರಾಗಮ-ದಲ್ಲಿ
ಪಾಂಚಾ-ಲದ
ಪಾಂಚಾಲ-ದೇವ
ಪಾಂಚಾಲ-ದೇವ-ನನ್ನು
ಪಾಂಚಾಳ-ದವ-ರಲ್ಲಿ
ಪಾಂಚಾಳ-ದ-ವರು
ಪಾಂಚಾ-ಳರ
ಪಾಂಚಾ-ಳಿಕೆ
ಪಾಂಡವ-ಪುರ
ಪಾಂಡವ-ಪುರದ
ಪಾಂಡವ-ಪುರ-ದಲ್ಲಿವೆ
ಪಾಂಡ-ವರ
ಪಾಂಡ-ವರ-ಗುಹೆಯ
ಪಾಂಡವ-ರಿಂದ
ಪಾಂಡ-ವರು
ಪಾಂಡಿತ್ಯ
ಪಾಂಡ್ಯ
ಪಾಂಡ್ಯ-ಕುಲ
ಪಾಂಡ್ಯನ
ಪಾಂಡ್ಯನು
ಪಾಂಡ್ಯ-ಪಾಡಿ
ಪಾಂಡ್ಯ-ಬಲ
ಪಾಂಡ್ಯರ
ಪಾಂಡ್ಯ-ರನ್ನು
ಪಾಂಡ್ಯ-ರಾಜ-ನೊಡನೆ
ಪಾಂಡ್ಯ-ರಾಜರ
ಪಾಂಡ್ಯ-ರಾಜ್ಯ
ಪಾಂಡ್ಯ-ರಾಜ್ಯ-ವನ್ನು
ಪಾಂಡ್ಯ-ರಾಯ
ಪಾಂಡ್ಯ-ರಾಯಪ್ರತಿಷ್ಟಾ-ಚಾರ್ಯ್ಯ
ಪಾಂಡ್ಯ-ರಿಂದ
ಪಾಂಡ್ಯ-ರಿಗೂ
ಪಾಂಡ್ಯರು
ಪಾಂಬಬ್ಬೆ
ಪಾಕ-ಶಾಲೆಯ
ಪಾಕ-ಶಾಲೆ-ಯನ್ನು
ಪಾಚ-ಯಪ್ಪ
ಪಾಚಿ-ಯಪ್ಪನ
ಪಾಚ್ಚ-ನನಂ
ಪಾಛಾ-ಬಾದಶಹಾ-ರ-ವರ
ಪಾಛಾ-ರ-ವರ
ಪಾಠ
ಪಾಠ-ದಿಂದ
ಪಾಠ-ವನ್ನು
ಪಾಠವು
ಪಾಠ-ಶಾಲೆ
ಪಾಠ-ಶಾಲೆ-ಗ-ಳನ್ನು
ಪಾಠಾಂತರ-ವಿದೆ
ಪಾಡಿ
ಪಾಡಿ-ಗಳೆತ್ತಿ
ಪಾಡು-ಗಾರ
ಪಾಡೆ
ಪಾತಕ-ದಲು
ಪಾತಕ್ದಲಿ
ಪಾತಾಳಾಂಕಣ
ಪಾತಾಳಾಂಕ-ಣದ
ಪಾತಾಳಾಂಕಣ-ವನ್ನು
ಪಾತ್ರ
ಪಾತ್ರಕ್ಕೆ
ಪಾತ್ರ-ಧಾರಿ-ಯನ್ನು
ಪಾತ್ರ-ನಾಗಿದ್ದಾನೆ
ಪಾತ್ರ-ಭೂತ-ರಾದ
ಪಾತ್ರ-ರಾಗಿದ್ದರು
ಪಾತ್ರ-ವನ್ನು
ಪಾತ್ರ-ವಹಿಸಿದ್ದ-ನೆಂಬುದು
ಪಾತ್ರ-ವಹಿಸಿ-ರುವುದು
ಪಾತ್ರವೂ
ಪಾತ್ರೆ
ಪಾತ್ರೆ-ಗ-ಳನ್ನು
ಪಾತ್ರೆ-ಗ-ಳಾಗಿ-ರ-ಬ-ಹುದು
ಪಾತ್ರೆ-ಪಡಗ-ಗಳು
ಪಾತ್ರೆ-ಯನ್ನು
ಪಾಥಕು-ಸು-ಮನಾ-ಥಜೀ
ಪಾಥೇಯ
ಪಾದ
ಪಾದ-ಕ-ಮಲ-ಗ-ಳಿಗೆ
ಪಾದಕ್ಕೆ
ಪಾದ-ಗಳ
ಪಾದ-ಗ-ಳನ್ನು
ಪಾದ-ಗಳೂ
ಪಾದ-ಚಾರ-ಕ-ನಾದ
ಪಾದ-ದ-ಬಳಿ
ಪಾದ-ಧೂಳಿ-ಯಿಂದ
ಪಾದ-ಪದ್ಮ-ಗ-ಳನ್ನು
ಪಾದ-ಪದ್ಮಾ-ರಾಧಕ
ಪಾದ-ಪದ್ಮಾ-ರಾಧಕ-ನಾಗಿದ್ದರೂ
ಪಾದ-ಪದ್ಮಾ-ರಾಧಕ-ರಾಗಿದ್ದರು
ಪಾದ-ಪದ್ಮಾ-ರಾಧಕರು
ಪಾದ-ಪದ್ಮಾ-ರಾಧಕ-ರು-ಮಪ್ಪ
ಪಾದ-ಪದ್ಮೋಪ
ಪಾದ-ಪದ್ಮೋಪ-ಜೀವಿ
ಪಾದ-ಪದ್ಮೋಪ-ಜೀವಿ-ಗ-ಳಾಗಿದ್ದು
ಪಾದ-ಪದ್ಮೋಪ-ಜೀವಿ-ಯಾಗಿ
ಪಾದ-ಪದ್ಮೋಪ-ಜೀವಿ-ಯಾ-ಗಿದ್ದ
ಪಾದ-ಪದ್ಮೋಪ-ಜೀವಿ-ಯಾಗಿದ್ದ-ನೆಂದು
ಪಾದ-ಪದ್ಮೋಪ-ಜೀವಿ-ಯೆಂದು
ಪಾದ-ಪೂಜೆ-ಯನ್ನು
ಪಾದ-ಪೂಜೆ-ಯಾಗಿ
ಪಾದ-ಮಾರ್ಗ
ಪಾದ-ವೃತ್ತಿಯ
ಪಾದ-ವೃತ್ತಿ-ಯನ್ನು
ಪಾದ-ಸೇವಕ-ನಾದ
ಪಾದ-ಸೇವಕ-ಳಾದ
ಪಾದ-ಸೇ-ವರ-ಕರೂ
ಪಾದ-ಸೇವೆ-ಯನ್ನು
ಪಾದಾದಿ
ಪಾದಾಬ್ಜಕೃಕಟಾಯಿತಚೇತಸಃ
ಪಾದಾ-ರವಿಂದ
ಪಾದಾ-ರಾಧಕ
ಪಾದಾ-ರಾಧಕನುಂ
ಪಾದಾ-ರಾಧಕ-ನು-ಮಪ್ಪ
ಪಾದಾ-ರಾಧಕೆಯೂ
ಪಾದಾರಾ-ಧನಾ
ಪಾದಾರ್ಚ-ನೆಗೆ
ಪಾದಾರ್ಚ್ಯನಕ್ಕಂ
ಪಾದೋದಕ
ಪಾಪ
ಪಾಪಕೆ
ಪಾಪಕ್ಷಯ
ಪಾಪಕ್ಷ-ಯದ್ವಾರಾ
ಪಾಪಣ್ಣ
ಪಾಪಣ್ಣನು
ಪಾಪ-ದಲಿ
ಪಾಪಮ್ಮ
ಪಾಪಯ್ಯನ-ಕೊಪ್ಪಲು
ಪಾಪರ್
ಪಾಪಾರ-ಪಟ್ಟಿ
ಪಾಪಾರ್ಪಟ್ಟಿ
ಪಾಪಾರ್ಪ್ಪಟ್ಟಿ
ಪಾಪ್ಪಲ್
ಪಾಪ್ಪಾಲಿ
ಪಾಮರನ್
ಪಾಯ-ಸದ
ಪಾಯಾತ್
ಪಾಯೆಸ್
ಪಾಯ್ವುದಂ
ಪಾರಂಗತ
ಪಾರಂಗ-ತ-ರಾದ
ಪಾರಂಗ-ತೆ-ಯನ್ನು
ಪಾರಂಪರಿಕ
ಪಾರಂಪರಿಕ-ವಾಗಿ
ಪಾರಗಃ
ಪಾರ-ಗನೂ
ಪಾರಗಾನ್
ಪಾರ-ಮಾರ್ಥಿಕ
ಪಾರಮ್ಯ-ವನ್ನು
ಪಾರರಾ
ಪಾರಾದ-ನೆಂದು
ಪಾರಾಯಣ
ಪಾರಾ-ವಾರ
ಪಾರಿಜಾತ
ಪಾರಿಜಾ-ತಸ್ಯ
ಪಾರಿಜಾ-ತಾಪ-ಹರ-ಣಮು
ಪಾರಿ-ಭಾಷಿಕ
ಪಾರಿ-ಭಾಷಿಕ-ಗ-ಳನ್ನು
ಪಾರಿ-ಭಾಷಿಕ-ಪದ
ಪಾರಿ-ಭಾಷಿಕ-ವಾಗಿ
ಪಾರಿಶ್ವ-ದೇವ
ಪಾರೀಷ-ದೇವರ
ಪಾರು-ಪತ್ತೆ-ಗಾರ
ಪಾರು-ಪತ್ತೇ-ಗಾರ್
ಪಾರ್ಥಸಾರಥಿ
ಪಾರ್ಥಿವಃ
ಪಾರ್ಥಿವಸ್ಯಾಸ್ಯ
ಪಾರ್ಥಿವೋ
ಪಾರ್ಬತಿ
ಪಾರ್ವತಿ
ಪಾರ್ವತಿ-ಯನ್ನು
ಪಾರ್ವತೀ
ಪಾರ್ವತೀ-ವಲ್ಲಭ
ಪಾರ್ವತ್ಯೋಸ್ತು
ಪಾರ್ವರುಂ
ಪಾರ್ಶ್ವ
ಪಾರ್ಶ್ವ-ಜಿನ-ಗೃಹಮಂ
ಪಾರ್ಶ್ವ-ಜಿನ-ಗೃಹ-ವನ್ನು
ಪಾರ್ಶ್ವ-ಜಿನ-ಭ-ವನ-ವನ್ನು
ಪಾರ್ಶ್ವ-ಜಿನಾ-ಲಯ
ಪಾರ್ಶ್ವ-ಜಿನೇಶ್ವರ
ಪಾರ್ಶ್ವ-ದಲ್ಲಿ
ಪಾರ್ಶ್ವ-ದಾನ
ಪಾರ್ಶ್ವ-ದಾನಸ್ಥಳ
ಪಾರ್ಶ್ವ-ದೇವ
ಪಾರ್ಶ್ವ-ದೇವಂ
ಪಾರ್ಶ್ವ-ದೇವನ
ಪಾರ್ಶ್ವ-ದೇವನು
ಪಾರ್ಶ್ವ-ದೇವರ
ಪಾರ್ಶ್ವನ
ಪಾರ್ಶ್ವ-ನಾಥ
ಪಾರ್ಶ್ವ-ನಾಥ-ದೇವರ
ಪಾರ್ಶ್ವ-ನಾಥ-ದೇವರು
ಪಾರ್ಶ್ವ-ನಾಥನ
ಪಾರ್ಶ್ವ-ಪಂಡಿತ
ಪಾರ್ಶ್ವ-ಪಂಡಿ-ತನೂ
ಪಾರ್ಶ್ವ-ಪುರ-ವನ್ನಾಗಿ
ಪಾಲಗ್ರ-ಹಾರವು
ಪಾಲನ್ನು
ಪಾಲ-ಯನ-ಖಿಳಾಂ
ಪಾಲ-ಯನ್
ಪಾಲ-ಹಳ್ಳಿ
ಪಾಲಿತ
ಪಾಲಿನ
ಪಾಲಿಸ-ಬೇಕೆಂದು
ಪಾಲಿ-ಸಲು
ಪಾಲಿಸಿ
ಪಾಲಿಸಿದ
ಪಾಲಿಸಿ-ದ-ವ-ರಿಗೆ
ಪಾಲಿಸಿದ್ದ
ಪಾಲಿಸಿದ್ದನು
ಪಾಲಿಸಿದ್ದ-ನೆಂದು
ಪಾಲಿಸಿದ್ದ-ನೆಂದೂ
ಪಾಲಿಸುತ್ತಾರೆ
ಪಾಲಿಸುತ್ತಿದ್ದರು
ಪಾಲಿಸುತ್ತಿದ್ದ-ವರು
ಪಾಲಿ-ಸುವರು
ಪಾಲಿ-ಸುವುದ-ರಿಂದಲೂ
ಪಾಲು
ಪಾಲು-ಗಾರಳು
ಪಾಲು-ದಾರಿ-ಕೆಯ
ಪಾಲು-ಪಾರಿಕತ್ತು-ಸಾಲ
ಪಾಲೂ
ಪಾಲ್ಗೊಂಡಿರ-ಬ-ಹುದು
ಪಾಲ್ಗೊಂಡು
ಪಾಲ್ಗೊಳ್ಳುತ್ತಿದ್ದರು
ಪಾಲ್ಯಂ
ಪಾಲ್ಯ-ಕೀರ್ತಿ
ಪಾಲ್ಯ-ಕೀರ್ತಿ-ಪಂಡಿತ
ಪಾಲ್ಯ-ಕೀರ್ತಿಯೇ
ಪಾಳಿ-ಸಿ-ದ-ನುರ್ವ್ವ-ರೆಯಂ
ಪಾಳು
ಪಾಳು-ಬ-ಸದಿ-ಯಲ್ಲಿ-ರುವ
ಪಾಳು-ಮಂಟಪ-ಗಳು
ಪಾಳು-ಮಲ್ಲೇಶ್ವರ
ಪಾಳೆ-ಗಾರ-ನಾಗಿ
ಪಾಳೆ-ಗಾರರು
ಪಾಳೆಯ-ಗಾರನ
ಪಾಳೆಯ-ಗಾರ-ನಾ-ಗಿದ್ದ
ಪಾಳೆಯ-ಗಾರ-ನಾ-ಗಿದ್ದ-ನೆಂದು
ಪಾಳೆಯ-ಗಾರ-ನಾದ
ಪಾಳೆಯ-ಗಾರರ
ಪಾಳೆಯ-ಗಾರ-ರನ್ನು
ಪಾಳೆಯ-ಗಾರರು
ಪಾಳೆಯ-ದ-ವರು
ಪಾಳೆಯ-ಪಟ್ಟನ್ನು
ಪಾಳೆಯ-ಪಟ್ಟೆಂದು
ಪಾಳೇ-ಗಾರ
ಪಾಳ್ಯ
ಪಾವಗಡ
ಪಾವಟೆ-ಯ-ವರು
ಪಾವತಿಯ
ಪಾವು
ಪಾಶ
ಪಾಶ-ಪತ
ಪಾಶ-ವನ್ನು
ಪಾಶಿ
ಪಾಶು-ಪತ
ಪಾಶು-ಪತ-ಗಳು
ಪಾಶು-ಪತದ
ಪಾಶು-ಪತ-ಧರ್ಮವು
ಪಾಶು-ಪತರ
ಪಾಶು-ಪತ-ರಿಗೂ
ಪಾಶು-ಪತರು
ಪಾಶು-ಪತವು
ಪಾಶು-ಪತ-ವೆಂಬ
ಪಾಶು-ಪತವ್ರತ-ವನ್ನು
ಪಾಶು-ಪಥ
ಪಿಂಗಾಣಿ-ಗ-ಳನ್ನು
ಪಿಂಛಾ-ಚಾರ್ಯ
ಪಿಂಛಾತ-ಪತ್ರಾನ್ವಿತಾ-ಸನ
ಪಿಂಡಾಂಡ-ದಾನ
ಪಿಂಡಾ-ದಾನ
ಪಿಂಡಾ-ದಾನ-ಗಳು
ಪಿಂಡಾ-ದಾನ-ವನ್ನು
ಪಿಂಡಾ-ದಾನ-ವಾಗಿ
ಪಿಂಡಾ-ದಾನ-ವೆಂಬ
ಪಿಎಚ್ಡಿ
ಪಿಎಚ್ಡಿಗೆ
ಪಿಎನ್
ಪಿಡಿದಂ
ಪಿಡಿದು
ಪಿತಾದಿ
ಪಿತಾಮಹ
ಪಿತಾಮ-ಹನ
ಪಿತಾಮಹ-ರಾದ
ಪಿತಾಮಹ-ರೆನಿಸಿ-ಕೊಂಡ
ಪಿತೃ-ಗ-ಳಿಗೆ
ಪಿತ್ರಾರ್ಜಿತ-ವಾಗಿ
ಪಿನಾಕಿನಿಯ
ಪಿಬಿ
ಪಿಬಿ-ದೇಸಾಯಿ
ಪಿಬಿ-ದೇಸಾಯಿ-ಯ-ವರು
ಪಿಯ
ಪಿರಾಟ್ಟಿ
ಪಿರಾನ್
ಪಿರಿಯ
ಪಿರಿಯಂಣ-ವೊಡೆ-ಯರ
ಪಿರಿಯಂಮಂಗಳು
ಪಿರಿಯ-ಆಳ್ವಿಕೆ
ಪಿರಿಯ-ಕಳ-ಲೆಯ
ಪಿರಿಯ-ಕಳಿ-ಲೆಯ
ಪಿರಿಯ-ಕೆರೆಯ
ಪಿರಿಯ-ದಂಡ-ನಾಯಕ
ಪಿರಿಯ-ಪಟ್ಟದ
ಪಿರಿಯ-ರಸಿ
ಪಿರಿಯ-ರ-ಸಿಗೆ
ಪಿರಿಯ-ರಸಿಯ
ಪಿರಿಯ-ರಸಿ-ಯಾ-ಗಿದ್ದ
ಪಿರಿಯು-ರಮೆಂಬಿವಂ
ಪಿರಿ-ಯೊಡೆ-ಯನ
ಪಿರಿ-ಯೊಡೆ-ಯ-ನನ್ನು
ಪಿರಿ-ಯೊಡೆ-ಯ-ನಿಗೂ
ಪಿರಿ-ಯೊಡೆ-ಯನು
ಪಿರಿ-ಯೊಡೆ-ಯನೂ
ಪಿರಿ-ಯೊಡೆ-ಯ-ನೆಂಬ
ಪಿರಿ-ಯೊಡೆ-ಯರು
ಪಿಳ್ಳೆ
ಪಿಳ್ಳೆ-ಯ-ಮ-ಗಳ
ಪಿಳ್ಳೆ-ಯಾಂಡರ
ಪಿಳ್ಳೆ-ಯಾಂಡರನ
ಪಿಳ್ಳೆ-ಯಾಂಡರ-ನಿಗೆ
ಪಿಳ್ಳೈ
ಪಿಳ್ಳೈ-ದೇವರ
ಪಿಳ್ಳೈ-ಯಾಂಡರನು
ಪಿಳ್ಳೈಯ್ಯಂಗಳ
ಪೀಠ
ಪೀಠ-ಗಳ
ಪೀಠದ
ಪೀಠ-ದಲ್ಲಿ
ಪೀಠ-ದಿಂದ
ಪೀಠಿಕೆ-ಗ-ಳನ್ನು
ಪೀಠಿಕೆ-ಗಳು
ಪೀಠಿಕೆ-ಯನ್ನು
ಪೀಠಿಕೆ-ಯಲ್ಲಿ
ಪೀಠೋಪ-ಕರ-ಣ-ಗಳು
ಪೀರ-ಜೀಯ-ನಿಗೆ
ಪೀರ್
ಪುಂಗ-ನೂರ
ಪುಂಡರಿಕ
ಪುಂಡರೀಕ
ಪುಂಡರೀಕ-ನಂಬಿ
ಪುಂಡಾಟಿಕೆ-ಯನ್ನು
ಪುಂಣ್ಯಕ್ರುತ
ಪುಂಣ್ಯ-ಚರಿತಂ
ಪುಂಣ್ಯ-ನಿಳಯಂಗಾ-ಚಂದ್ರ
ಪುಗಿರಿ-ನಾಡಿನ
ಪುಗಿರಿ-ನಾಡು-ಪೊ-ಗರ್ನಾಡು
ಪುಗಿಸಿ-ದಾರ್
ಪುಟಕ್ಕೆ
ಪುಟ-ಗಳಷ್ಟಾ-ಯಿತು
ಪುಟ-ಗಳಷ್ಟು
ಪುಟ-ಗಳಾಯಿತು
ಪುಟ-ದಲ್ಲಿ
ಪುಟ-ವನ್ನು
ಪುಟ್ಟ
ಪುಟ್ಟಂಣ
ಪುಟ್ಟಂಣ-ಗಳು
ಪುಟ್ಟ-ಗುರ-ವೆಂಬ
ಪುಟ್ಟ-ನ-ರಸಮ್ಮನು
ಪುಟ್ಟ-ನ-ರಸಿ
ಪುಟ್ಟ-ಶಿಂಗಮ್ಮನು
ಪುಟ್ಟಿದ-ರೇಚಲ-ದೇ-ವಿಗೆ
ಪುಡೋಲೋಣ್ಡಿಚೆಟ್ಟಿ-ಯಾರ್
ಪುಣಗು-ಕಾಪು
ಪುಣ-ಸ-ಮಯ್ಯ-ನನ್ನು
ಪುಣಸಿ-ಮಯ್ಯನು
ಪುಣಸೆ-ಪಟ್ಟಿ
ಪುಣಿ-ಗದ
ಪುಣಿಸ
ಪುಣಿಸ-ದಂಡ-ನಾಥನು
ಪುಣಿಸ-ಮಯ್ಯ
ಪುಣಿಸ-ಮಯ್ಯನ
ಪುಣಿಸ-ಮಯ್ಯನು
ಪುಣಿಸ-ಮಯ್ಯ-ನೆಂದು-ಹೇ-ಳಿದೆ
ಪುಣಿಸಮ್ಮ
ಪುಣಿಸಶ್ರೀ
ಪುಣು-ಗಿನ
ಪುಣುಗು-ಕಾಪು
ಪುಣ್ಯ
ಪುಣ್ಯ-ಕಾಲ-ದಲ್ಲಿ
ಪುಣ್ಯಕ್ಷೇತ್ರ-ಗ-ಳಿಂದ
ಪುಣ್ಯಕ್ಷೇತ್ರ-ಗಳು
ಪುಣ್ಯಕ್ಷೇತ್ರ-ದಲಿ
ಪುಣ್ಯಕ್ಷೇತ್ರ-ದಲ್ಲಿ
ಪುಣ್ಯ-ಜನ-ಧಾಮ
ಪುಣ್ಯ-ತಮೇ
ಪುಣ್ಯ-ದಿನ-ಗ-ಳಿಗೆ
ಪುಣ್ಯ-ದೇವ-ತೆಯೆ-ನಲೇಂ
ಪುಣ್ಯರುಂ
ಪುಣ್ಯ-ಲೋಕ
ಪುಣ್ಯ-ವತಿ
ಪುಣ್ಯ-ವಾಗ-ಬೇಕು-ಯೆಂದು
ಪುಣ್ಯ-ವಾಗ-ಬೇಕೆಂದು
ಪುಣ್ಯ-ವಾಗಲಿ
ಪುಣ್ಯ-ವಾಗುತ್ತದೆ
ಪುಣ್ಯಾರ್ಥ-ವಾಗಿ
ಪುಣ್ಯಾಹವೈಃ
ಪುಣ್ಯೇ
ಪುತ್ತದ-ಮೆಯ್ಯ
ಪುತ್ತೂರಿನ
ಪುತ್ತೂರಿ-ನಲ್ಲಿ
ಪುತ್ತೂರು
ಪುತ್ತ್ರಂ
ಪುತ್ರ
ಪುತ್ರಂ
ಪುತ್ರ-ನಾಗಿ-ರ-ಬ-ಹುದು
ಪುತ್ರ-ನಾದ
ಪುತ್ರನು
ಪುತ್ರ-ನೆಂದರೆ
ಪುತ್ರ-ನೆಂದು
ಪುತ್ರನೇ
ಪುತ್ರ-ಪರಂಪ-ರೆಯ
ಪುತ್ರ-ಪವುತ್ರ
ಪುತ್ರ-ಪೌತ್ರ
ಪುತ್ರ-ರತ್ನ-ರನ್ನೂ
ಪುತ್ರ-ರನ್ನು
ಪುತ್ರ-ರಲ್ಲಿ
ಪುತ್ರ-ರಾದ
ಪುತ್ರರು
ಪುತ್ರರೂ
ಪುತ್ರ-ರೆಂಬ
ಪುತ್ರರೋ
ಪುತ್ರ-ಸ-ಮಾನ-ನಾಗಿ
ಪುತ್ರ-ಸ-ಮಾನ-ನಾ-ಗಿದ್ದ
ಪುತ್ರಿ
ಪುತ್ರಿಕಂ
ಪುತ್ರಿಯ
ಪುತ್ರಿ-ಯ-ರನ್ನು
ಪುತ್ರಿ-ಯಾದ
ಪುತ್ರೋಚ್ಛಾಹ
ಪುತ್ರೋತ್ಸವ-ಮಾ-ಗಲ್
ಪುತ್ರೋತ್ಸವ-ವಾದಾಗ
ಪುತ್ರೋತ್ಸಾಹ
ಪುತ್ರೋತ್ಸಾಹ-ದಲ್ಲಿ
ಪುದುಂಗೊಳಿ-ಸುತ್ತುಂ
ಪುದು-ಚೇರಿ-ವರೆಗೆ
ಪುನ
ಪುನಃ
ಪುನ-ರಪಿ
ಪುನ-ರಾ-ವರ್ತನೆ-ಯಾ-ಗಿದ್ದು
ಪುನ-ರುಜ್ಜೀನವ
ಪುನ-ರುಜ್ಜೀ-ವನ-ಗೊ-ಳಿಸಿ-ರು-ವುದು
ಪುನ-ರುಜ್ಜೀ-ವನ-ಗೊಳಿಸು-ವಂತೆ
ಪುನ-ರುಜ್ಜೀವಿ-ಸಿರುವ
ಪುನ-ರುದ್ಧಾರ-ಮಾಡಿದ
ಪುನರ್
ಪುನರ್ದತ್ತ-ಯಾಗಿ
ಪುನರ್ದತ್ತಿ-ಯಾಗಿ
ಪುನರ್ದಾನ
ಪುನರ್ದ್ಧಾರಾ-ಪೂರ್ವಕಂ
ಪುನರ್ಧಾರಾ-ಪೂರ್ವ-ಕ-ವಾಗಿ
ಪುನರ್ಧಾರಾ-ಪೂರ್ವ್ವಕಂ
ಪುನರ್ನಿ-ಗದಿ-ಪ-ಡಿಸಿ
ಪುನರ್ನಿರ್ಮಾಣ
ಪುನರ್ನಿರ್ಮಾಣ-ಗೊಂಡಿದೆ
ಪುನರ್ನಿರ್ಮಿತ-ವಾಗಿದೆ
ಪುನರ್ನಿರ್ಮಿಸ-ಲಾಗಿದೆ
ಪುನರ್ನಿರ್ಮಿಸ-ಲಾ-ಗಿದ್ದು
ಪುನರ್ರಚನೆ-ಯಾಗಿ-ರ-ಬ-ಹುದು
ಪುನರ್ರಚಿಸ-ಬ-ಹುದು
ಪುನರ್ರಚಿ-ಸ-ಲಾ-ಯಿತು
ಪುನರ್ರಚಿಸಿ
ಪುನರ್ವಸು
ಪುನಸ್ಕಾರ-ಗ-ಳಿಗೆ
ಪುನಸ್ಕಾರ-ಗಳು
ಪುಮಾನೇಷಃ
ಪುಮಾನ್
ಪುರ
ಪುರಂದರ
ಪುರಃ
ಪುರ-ಕೊಳ-ಗ-ದಲೂ
ಪುರಕ್ಕೆ
ಪುರ-ಗ-ಳನ್ನು
ಪುರ-ಗ-ಳಲ್ಲಿ
ಪುರ-ಗಳು
ಪುರಗ್ರಾಮದ
ಪುರ-ಜನಕಂ
ಪುರ-ಣೋಕ್ತ
ಪುರದ
ಪುರ-ದ-ಕಟ್ಟೆ-ಬೇಚಿರಾಕ್
ಪುರ-ದ-ಮಾ-ಗಣಿಗೆ
ಪುರ-ದಲ್ಲಿ
ಪುರ-ದಲ್ಲಿದ್ದ
ಪುರ-ದಾ-ಚಾರಿ
ಪುರ-ದಾ-ಚಾರಿಯ
ಪುರ-ದಾನ-ವಾಗಿ
ಪುರ-ದೊಳ-ಗಿ-ರುವ
ಪುರ-ದೊಳಗೆ
ಪುರ-ದೊಳು
ಪುರ-ಧರ್ಮ-ಗಳು
ಪುರ-ಧರ್ಮ-ವನ್ನು
ಪುರ-ಧರ್ಮ-ವಾಗಿ
ಪುರ-ಪಟ್ಟ-ಣ-ವಾಗಿ
ಪುರವ
ಪುರ-ವನ್ನಾಗಿ
ಪುರ-ವನ್ನು
ಪುರ-ವ-ರಾಧೀಶ್ವರ
ಪುರ-ವರ್ಗ-ಗಳ
ಪುರ-ವರ್ಗ-ದಾನ
ಪುರ-ವರ್ಗ-ದಾನ-ಗ-ಳನ್ನು
ಪುರ-ವಾಗಿ
ಪುರ-ವಾ-ಗಿತ್ತು
ಪುರ-ವಾಗಿದ್ದ-ರಿಂದ
ಪುರ-ವಾದ
ಪುರ-ಷಾಶ್ಚ
ಪುರಸ್ಕರಿ-ಸಿರು-ವು-ದನ್ನು
ಪುರಾಣ
ಪುರಾಣಃ
ಪುರಾಣ-ಗ-ಳಲ್ಲಿ
ಪುರಾಣದ
ಪುರಾಣ-ವನ್ನು
ಪುರಾಣಾ-ನಾಮ
ಪುರಾಣೇ-ತಿ-ಹಾಸ-ಗಳು
ಪುರಾಣೇಷು
ಪುರಾ-ಣೋಕ್ತ
ಪುರಾ-ತತ್ತ್ವ
ಪುರಾ-ತತ್ವ
ಪುರಾ-ತತ್ವದ
ಪುರಾ-ತನ
ಪುರಾ-ತ-ನದ
ಪುರಾತ-ನರ
ಪುರಾ-ತನ-ರಾದ
ಪುರಾ-ತನ-ರೆಂಬ
ಪುರಾ-ತನ-ವಾದ
ಪುರಾ-ತನ-ವಾ-ದುದು
ಪುರಾಧಿ-ಪನ
ಪುರಾಧೀಶ್ವರಂ
ಪುರಾಯ-ತ-ತರಂ
ಪುರಾವೆ-ಗಳೂ
ಪುರಾವೆ-ಯನ್ನು
ಪುರಿಗೆರೆ
ಪುರಿಗೆ-ರೆನ್ನು
ಪುರಿಶೈ
ಪುರುಷ
ಪುರುಷನ
ಪುರುಷ-ನಾದ
ಪುರುಷಪ್ರಧಾನ-ವಾ-ದರೂ
ಪುರುಷ-ಮಾಣಿಕ್ಯ-ಸೆಟ್ಟಿಯು
ಪುರುಷ-ರಿಗೆ
ಪುರುಷರು
ಪುರುಷಾರ್ತ್ಥ
ಪುರುಷಾರ್ಥ
ಪುರುಷೋತ್ತಮ
ಪುರುಷೋತ್ತಮ-ದೇವ
ಪುರುಷೋತ್ತಮಯ್ಯನ
ಪುರೋಗ-ಮಿಯೂ
ಪುರೋಹಿತ
ಪುರೋಹಿತ-ರಿಗೆ
ಪುರೋಹಿ-ತರು
ಪುರೋಹಿತ-ಶಾಹಿ
ಪುರೋಹಿತ-ಶಾಹಿ-ಯಐ-ಹೊಳೆ
ಪುರ್ವ್ವಾಯ
ಪುಱ
ಪುಲಿಕ
ಪುಲಿಕ-ಗಚ್ಛದ
ಪುಲಿ-ಗಳ
ಪುಲಿ-ಗೆರೆ
ಪುಲಿಗೆ-ರೆನ್ನು
ಪುಲಿ-ಗೆರೆಯ
ಪುಲಿ-ಗೆರೆ-ಯಿಂದ
ಪುಲಿ-ಬ-ಸದಿ
ಪುಲಿ-ಯಣ್ಣ
ಪುಲಿ-ಯಣ್ಣನ
ಪುಲಿಯೂಟ
ಪುಲಿಯೂ-ಟಾಗಿ
ಪುಳಿಮೆಯ್ಯನ
ಪುಳಿ-ಯಬ್ಬೆ
ಪುಳುಗು-ಕಾಪು
ಪುಳುದಿ
ಪುಳ್ಳಿ-ಯಬ್ಬೆ
ಪುಳ್ಳುದು
ಪುಳ್ಳೆ
ಪುಳ್ಳೈ
ಪುಳ್ಳೈ-ಲೋಕಾ-ಚಾರ್ಯರ
ಪುವ-ಗಾಮ-ವನ್ನು
ಪುಷ್ಕರಣಿ
ಪುಷ್ಕರ-ಣಿಯೂ
ಪುಷ್ಟಿ
ಪುಷ್ಟಿ-ಯನ್ನು
ಪುಷ್ಟಿರ್ಜ್ವಯಶ್ಚ
ಪುಷ್ಪ
ಪುಷ್ಪ-ಕಮಂ
ಪುಷ್ಪ-ತಟ್ಟೆ-ಯನ್ನು
ಪುಷ್ಪ-ಮಾಲೆ-ಯನ್ನು
ಪುಷ್ಪವ್ರಿಷ್ಟಿ
ಪುಷ್ಪೋತ್ಗ-ಮನ
ಪುಸಿ-ವರ-ಗಂಡ
ಪುಸ್ತಕ
ಪುಸ್ತ-ಕಕ್ಕೆ
ಪುಸ್ತಕ-ಗಚ್ಚ
ಪುಸ್ತಕ-ಗಚ್ಚದ
ಪುಸ್ತಕ-ಗಚ್ಛದ
ಪುಸ್ತಕ-ಗ-ಳನ್ನು
ಪುಸ್ತ-ಕದ
ಪುಹುಲಿ-ಗೆರೆಯ
ಪೂಜಕ-ರಾಗಿದ್ದ-ರಿಂದ
ಪೂಜಾ
ಪೂಜಾ-ಕರ್ತ-ರಾಗಿದ್ದಾರೆ
ಪೂಜಾ-ಕಾರ್ಯ-ಗ-ಳಿಗೆ
ಪೂಜಾ-ಕೈಂಕರ್ಯ-ಗಳ
ಪೂಜಾ-ಕೈಂಕರ್ಯ-ಗ-ಳನ್ನು
ಪೂಜಾ-ಕೈಂಕರ್ಯ-ಗ-ಳಿಗೆ
ಪೂಜಾದಿ
ಪೂಜಾ-ದಿ-ಕಾರ್ಯ-ಗ-ಳಿಗೆ
ಪೂಜಾ-ದಿ-ಗಳು
ಪೂಜಾ-ಪದ್ಧತಿ-ಗ-ಳನ್ನು
ಪೂಜಾ-ಪರಿ-ಕರ
ಪೂಜಾ-ರತಂ
ಪೂಜಾ-ರರ
ಪೂಜಾ-ರ-ರಿಗೆ
ಪೂಜಾ-ರರು
ಪೂಜಾರಿ
ಪೂಜಾ-ರಿ-ಗಳು
ಪೂಜಾ-ರಿ-ತಮ್ಮಡಿ
ಪೂಜಾವ್ಯವಸ್ಥೆ
ಪೂಜಾವ್ಯವಸ್ಥೆ-ಗ-ಳನ್ನು
ಪೂಜಾಸ್ಥಳ-ಗಳ
ಪೂಜಿತ-ರಾಗಿದ್ದರು
ಪೂಜಿಸಿ
ಪೂಜಿಸಿ-ದನು
ಪೂಜಿ-ಸುತ್ತಾ
ಪೂಜಿ-ಸುತ್ತಾರೆ
ಪೂಜಿಸುತ್ತಿದ್ದ
ಪೂಜಿಸುತ್ತಿದ್ದನು
ಪೂಜಿಸುತ್ತಿದ್ದರು
ಪೂಜಿಸುತ್ತಿದ್ದ-ರೆಂದು
ಪೂಜಿಸುತ್ತಿದ್ದ-ರೆಂಬ
ಪೂಜಿಸುತ್ತಿದ್ದಿರ-ಬ-ಹುದು
ಪೂಜಿ-ಸುವ
ಪೂಜೆ
ಪೂಜೆ-ಗಾಗಿ
ಪೂಜೆಗೆ
ಪೂಜೆ-ಗೊಳ್ಳುತ್ತಿದೆ
ಪೂಜೆ-ಗೊಳ್ಳುತ್ತಿವೆ
ಪೂಜೆ-ಪುನಸ್ಕಾರ
ಪೂಜೆ-ಪುನಸ್ಕಾರ-ಗಳು
ಪೂಜೆಯ
ಪೂಜೆ-ಯನ್ನು
ಪೂಜೆಯು
ಪೂಜೋಪ-ಕರ-ಣ-ಗ-ಳನ್ನು
ಪೂಜ್ಯ
ಪೂಜ್ಯ-ತೆ-ಗಳು
ಪೂಜ್ಯ-ತೆಯ
ಪೂಜ್ಯ-ತೆ-ಯನ್ನು
ಪೂಜ್ಯ-ತೆ-ಯಿಂದ
ಪೂಜ್ಯ-ನಲ್ತೆ
ಪೂಜ್ಯ-ರಾದ-ವರ
ಪೂಜ್ಯ-ವಾದ
ಪೂನಾಡು-ಪುನ್ನಾಡು-ಗ-ಳನ್ನು
ಪೂರಕ
ಪೂರಕ-ವಾಗಿ
ಪೂರಯ್ಸಿ-ದನು
ಪೂರಿಗಾಲಿ
ಪೂರೈ-ಸಲು
ಪೂರೈಸಿ
ಪೂರೈ-ಸಿದ
ಪೂರೈಸಿ-ದರು
ಪೂರೈಸಿ-ರುತ್ತ-ರೆಂದು
ಪೂರೈಸುತ್ತಿದ್ದರು
ಪೂರೈ-ಸುವ
ಪೂರೈ-ಸುವುದಕ್ಕಾಗಿ
ಪೂರೈ-ಸುವು-ದಾಗಿಯೂ
ಪೂರ್ಣ
ಪೂರ್ಣತ್ರುಟಿತ-ವಾಗಿದೆ
ಪೂರ್ಣ-ನೆಂಬು-ವ-ವನ
ಪೂರ್ಣನೇ
ಪೂರ್ಣಯ್ಯ-ನ-ವರ
ಪೂರ್ಣಯ್ಯ-ನೆಂಬ
ಪೂರ್ಣ-ವಾಗಿ
ಪೂರ್ಣ-ವಾಗಿಯೂ
ಪೂರ್ತಿ
ಪೂರ್ತಿ-ಯಾಗಿ
ಪೂರ್ಬ್ಬಾಯ
ಪೂರ್ಬ್ಬಾಯ-ವಾಗಿ
ಪೂರ್ವ
ಪೂರ್ವ-ಕ-ವಾಗಿ
ಪೂರ್ವ-ಕವೇ-ಕಚ್ಛತ್ರಚ್ಛಾಯೆ
ಪೂರ್ವಕ್ಕೂ
ಪೂರ್ವಕ್ಕೆ
ಪೂರ್ವ-ಜರ
ಪೂರ್ವ-ಜರು
ಪೂರ್ವ-ತೀರ-ದಲ್ಲಿ
ಪೂರ್ವದ
ಪೂರ್ವ-ದಕ್ಷಿಣ-ಪಶ್ಚಿಮ
ಪೂರ್ವ-ದಲ್ಲಿ
ಪೂರ್ವ-ದಲ್ಲಿದ್ದ
ಪೂರ್ವ-ದಲ್ಲಿಯೇ
ಪೂರ್ವ-ದಿಂದಲೂ
ಪೂರ್ವ-ದಿಕ್ಕಿನ
ಪೂರ್ವ-ದಿಕ್ಕಿನಲ್ಲಿ
ಪೂರ್ವ-ಪದ-ವಾಗಿ
ಪೂರ್ವ-ಪಶ್ಚಿಮ
ಪೂರ್ವ-ಪಶ್ಚಿಮ-ಗ-ಳಲ್ಲಿ
ಪೂರ್ವ-ಪಶ್ಚಿಮ-ಸ-ಮದ್ರಾಧಿ-ಪತಿ
ಪೂರ್ವ-ಭಾಗದ
ಪೂರ್ವ-ಭಾಗ-ದಲ್ಲಿ
ಪೂರ್ವ-ಮರಿ-ಯಾದೆ-ಯಲು
ಪೂರ್ವ-ಮರ್ಯಾದೆಯ
ಪೂರ್ವ-ಶಾ-ಸನದ
ಪೂರ್ವಾಚಲ-ಮಾರ್ತಾಂಡ
ಪೂರ್ವಾನ್ವಯ
ಪೂರ್ವಾಯ
ಪೂರ್ವಾ-ಯಕ್ಕೆ
ಪೂರ್ವಾ-ಯದ
ಪೂರ್ವಾಯ-ವಾಗಿ
ಪೂರ್ವಾಯ-ವೆಂದರೆ
ಪೂರ್ವಾರ್ಧದ-ವರೆಗೆ
ಪೂರ್ವಾಶ್ರ-ಮದ
ಪೂರ್ವಿ-ಕರು
ಪೂರ್ವೋಕ್ತ
ಪೂರ್ವ್ವ-ದಲ್ಲಿ
ಪೂರ್ವ್ವ-ಮರ್ಯ್ಯಾದೆ-ಯಾಗಿ
ಪೂವಗಾಮವೇ
ಪೂವಗಾಮೆ-ಯನ್ನು
ಪೂವಿನ
ಪೃಥಕ್
ಪೃಥಿವೀ
ಪೃಥಿವೀಗೆ
ಪೃಥಿವೀ-ನೀರ್ಗುಂದ
ಪೃಥುವಿ
ಪೃಥುವೀ
ಪೃಥುವೀ-ಗಂಗನ
ಪೃಥುವೀ-ಗಾಮುಣ್ಡರು
ಪೃಥುವೀ-ಗಾವುಂಡರು
ಪೃಥುವೀ-ನೀರ್ಗುಂದ
ಪೃಥ್ವಿ
ಪೃಥ್ವೀ
ಪೃಥ್ವೀಂ
ಪೃಥ್ವೀ-ಗಂಗನ
ಪೃಥ್ವೀ-ಗಂಗ-ನಿದ್ದು
ಪೃಥ್ವೀ-ಗಂಗನು
ಪೃಥ್ವೀ-ನೀರ್ಗುಂದ
ಪೃಥ್ವೀ-ಪತಿ
ಪೃಥ್ವೀ-ಪತಿ-ಗಳು
ಪೃಥ್ವೀ-ಪ-ತಿಗೆ
ಪೃಥ್ವೀ-ಪತಿ-ಯಾಗಿದ್ದಾ-ನೆಂದು
ಪೃಥ್ವೀ-ಪ-ತಿಯು
ಪೃಥ್ವೀ-ರಾಜ್ಯಂಗೆ-ಯುತ್ತಿರೆ
ಪೃಥ್ವೀ-ರಾಜ್ಯಂಗೆಯ್ಯುತ್ತಿದ್ದನು
ಪೆಂ
ಪೆಂಗೆ-ನಾಯ-ಕನ
ಪೆಂಡಿರುಡೆಯುರ್ಚನ್ನು
ಪೆಂಡಿರುಡೆಯುರ್ಚಿ
ಪೆಂಡಿರುಡೆಯುರ್ಚಿ-ನಲ್ಲಿ
ಪೆಂಡಿರುಡೆಯುರ್ಚ್ಚು
ಪೆಂಡಿರುಡೆರ-ಳರ
ಪೆಂನಿ-ಪೆದ್ದಿ
ಪೆಂಪಿನ
ಪೆಂಪಿ-ನೊಳು
ಪೆಂಪು
ಪೆಂಪು-ವೆತ್ತ
ಪೆಂಮಾಳೆ-ಹಳ್ಳಿ-ಯಲ್ಲಿ
ಪೆಂಮಿಯಂಣ್ನನ
ಪೆಂಮೋ-ಜನೂ
ಪೆಂರ-ಗೂರು
ಪೆಗ್ಗಡೆ-ನಾಯ್ಕ
ಪೆಟ್ಯ
ಪೆಟ್ಯವ್
ಪೆಣ್ಪುಂಅಲ್ಹೆಣ್ಣು-ಸೆರೆ
ಪೆಣ್ಪು-ಯಲ್
ಪೆಣ್ಪುಯ್ಯ-ಲಿನ
ಪೆಣ್ಪುಯ್ಯ-ಲಿ-ನಲ್ಲಿ
ಪೆಣ್ಪುಯ್ಯಲು
ಪೆಣ್ಪುಯ್ಯಲ್
ಪೆಣ್ಬುಯ್ಯಲಿ-ನಲ್ಲಿ
ಪೆಣ್ಬುಯ್ಯಲು
ಪೆತ್ತ
ಪೆದ್ದ
ಪೆದ್ದ-ಗವು-ಡು-ಗಳ
ಪೆದ್ದಣ್ಣನು
ಪೆದ್ದಿ-ರಾಜುಗೆ
ಪೆದ್ದಿ-ರಾಜು-ವಿಗೆ
ಪೆದ್ದಿ-ರಾಜು-ವಿನ
ಪೆದ್ದಿ-ರಾಜುವು
ಪೆನು-ಗೊಂಡೆ
ಪೆನು-ಗೊಂಡೆಗೆ
ಪೆನು-ಗೊಂಡೆ-ದುರ್ಗ-ದಲ್ಲಿ
ಪೆನು-ಗೊಂಡೆಯ
ಪೆನು-ಗೊಂಡೆ-ಯಲ್ಲಿ
ಪೆನು-ಗೊಂಡೆ-ಯಲ್ಲಿದ್ದ
ಪೆನು-ಗೊಂಡೆ-ಯಲ್ಲಿದ್ದ-ನೆಂದು
ಪೆನು-ಗೊಂಡೆ-ಯ-ವರೆಗೆ
ಪೆನು-ಗೊಂಡೆ-ಯ-ವರೋ
ಪೆನು-ಗೊಂಡೆ-ಯಿಂದ
ಪೆನು-ಗೊಂಡೆ-ಯೆಂದೂ
ಪೆನು-ಗೊಂಡೆ-ಯೊಳು
ಪೆಮೋಜ
ಪೆಮ್ಪು
ಪೆಮ್ಮಣ್ಣ
ಪೆಮ್ಮಣ್ಣ-ನ-ವರು
ಪೆಮ್ಮಾಡಿ
ಪೆಮ್ಮಾಡಿಯು
ಪೆಮ್ಮಿ-ಯಣ್ಣನು
ಪೆಮ್ಮೋ-ಜನೂ
ಪೆರಂಗಡಿ
ಪೆರಂಗಡಿಯೇ
ಪೆರಂಗೂರಯ್ಯ-ನ-ವರ
ಪೆರಂಗೂರಯ್ಯ-ನ-ವ-ರಿಗೆ
ಪೆರಂಗೂರು
ಪೆರಗೆ
ಪೆರಮ-ಗಾವುಂಡನ
ಪೆರ-ಮಾನುಡನ್
ಪೆರ-ಮಾನ್
ಪೆರ-ಮಾಳ
ಪೆರ-ಮಾಳ-ದೇವ
ಪೆರ-ಮಾಳು
ಪೆರ-ಮಾಳೆ
ಪೆರ-ಮಾಳೆ-ದೇವ
ಪೆರ-ಮಾಳೆ-ದೇವನ
ಪೆರ-ಮಾಳೆ-ದೇವ-ನಿಗೆ
ಪೆರ-ಮಾಳೆ-ದೇವ-ನಿಗೇ
ಪೆರ-ಮಾಳೆ-ದೇವನು
ಪೆರ-ಮಾಳೆ-ದೇವನೇ
ಪೆರ-ಮಾಳೆಯು
ಪೆರ-ಮಾಳ್
ಪೆರರಾರುಂ
ಪೆರರಿ-ಯಣ್ಣನ್
ಪೆರಾಳ್ಕೆ
ಪೆರಾಳ್ಕೆಯೇ
ಪೆರಿಯ
ಪೆರಿ-ಯ-ಜೀಯರ
ಪೆರಿ-ಯ-ಜೀಯರ್
ಪೆರಿ-ಯ-ಜೀರ್
ಪೆರಿ-ಯ-ಪುರಾಣದ
ಪೆರಿ-ಯ-ಪೆರು-ಮಾಳ್ನ
ಪೆರಿ-ಯ-ಮನೈ
ಪೆರಿ-ಯ-ಮಲ-ನಂಬಿ
ಪೆರಿ-ಯಾಳ್ವಾನ್
ಪೆರಿ-ಯಾಳ್ವಿ
ಪೆರಿ-ಯೇರಿ
ಪೆರಿ-ಯೇ-ರಿಯ
ಪೆರಿ-ರಾಜನ
ಪೆರಿ-ರಾಜನು
ಪೆರಿ-ರಾಜರು
ಪೆರಿ-ರಾಜ-ರು-ಗಳು
ಪೆರಿ-ರಾಜು
ಪೆರುಂಕೋಟೆ-ರಾಜ್ಯದ
ಪೆರುಂದೇವಿ
ಪೆರುಂದೇವಿಯುಂ
ಪೆರುಂಬ-ದೂರಿ-ನಲ್ಲಿಯೂ
ಪೆರುಮ-ಗವುಂಡನ
ಪೆರು-ಮಾನ್
ಪೆರು-ಮಾಳ
ಪೆರು-ಮಾಳ-ದೇವನು
ಪೆರು-ಮಾಳ-ದೇವ-ರ-ಸ-ನಿಗೆ
ಪೆರು-ಮಾಳ-ರಸ
ಪೆರು-ಮಾಳಿಗೆ
ಪೆರು-ಮಾಳು
ಪೆರು-ಮಾಳು-ಸ-ಮುದ್ರ
ಪೆರು-ಮಾಳೆ
ಪೆರು-ಮಾಳೆ-ಚಮೂಪ-ತಿ-ಗಿಂತು
ಪೆರು-ಮಾಳೆ-ದೇವ
ಪೆರು-ಮಾಳೆ-ದೇವನ
ಪೆರು-ಮಾಳೆ-ದೇವ-ನಿ-ಗಿದ್ದ
ಪೆರು-ಮಾಳೆ-ದೇವ-ನಿಗೆ
ಪೆರು-ಮಾಳೆ-ದೇವನು
ಪೆರು-ಮಾಳೆ-ದೇವನೂ
ಪೆರು-ಮಾಳೆ-ದೇವನೇ
ಪೆರು-ಮಾಳೆ-ದೇವ-ರ-ಸನು
ಪೆರು-ಮಾಳೆ-ಪುರ-ವೆಂಬ
ಪೆರು-ಮಾಳೆಯ
ಪೆರು-ಮಾಳೆ-ಯ-ರಿಗೆ
ಪೆರು-ಮಾಳೆಯು
ಪೆರು-ಮಾಳೆಯೂ
ಪೆರು-ಮಾಳೆ-ಲಕ್ಷ್ಮೀ-ನಾ-ರಾಯಣ
ಪೆರು-ಮಾಳೈ
ಪೆರು-ಮಾಳ್
ಪೆರು-ಮಾಳ್ಆದಿ-ಮಾಧವ
ಪೆರು-ಮಾಳ್ಗೆ
ಪೆರು-ಮಾಳ್ಭಟ್ಟನು
ಪೆರುವಿಲೈ
ಪೆರ್ಗಡಿ
ಪೆರ್ಗಡೆ
ಪೆರ್ಗಡೆ-ಯಾಗಿ-ರುತ್ತಿದ್ದನು
ಪೆರ್ಗಡೆಯು
ಪೆರ್ಗಡೆ-ಯೊಳ್
ಪೆರ್ಗಡೆ-ಹೆಗ್ಗಡೆ
ಪೆರ್ಗ್ಗಡೆ
ಪೆರ್ಗ್ಗಡೆ-ಗಳ
ಪೆರ್ಗ್ಗಡೆ-ಗ-ಳನ್ನು
ಪೆರ್ಗ್ಗಡೆ-ಗಳು
ಪೆರ್ಗ್ಗಡೆಗೆ
ಪೆರ್ಗ್ಗಡೆ-ನಾಯಕ
ಪೆರ್ಗ್ಗಡೆ-ಯಾ-ಗಿದ್ದ
ಪೆರ್ಗ್ಗಡೆ-ಯಾ-ಗಿದ್ದು
ಪೆರ್ಗ್ಗಡೆಯು
ಪೆರ್ಗ್ಗಡೆಯೂ
ಪೆರ್ಗ್ಗಡೆಯೇ
ಪೆರ್ಗ್ಗಡೆ-ಹಿರಿ-ಯ-ಹೆಗ್ಗಡೆ-ಗಳು
ಪೆರ್ಗ್ಗೆರೆ
ಪೆರ್ಚ್ಚಿ-ಪನುಂ
ಪೆರ್ಚ್ಚುಗೆ
ಪೆರ್ಜುಂಕ
ಪೆರ್ದ್ದೊರೆ
ಪೆರ್ಬಾಣ-ನ-ಹಳ್ಳಿಯ
ಪೆರ್ಬಾಣ-ನ-ಹಳ್ಳಿ-ಯನ್ನು
ಪೆರ್ಬಾಣ-ನ-ಹಳ್ಳಿ-ಯಲ್ಲಿ
ಪೆರ್ಬ್ಬಣಿಗ-ಹಳ್ಳಿ
ಪೆರ್ಬ್ಬೞ
ಪೆರ್ಮಾಡಿ
ಪೆರ್ಮಾಡಿ-ದೇವನ
ಪೆರ್ಮಾಡಿ-ದೇವನು
ಪೆರ್ಮಾಡಿ-ರಾಯ
ಪೆರ್ಮಾ-ನಡಿ
ಪೆರ್ಮಾನ-ಡಿ-ಎರಡನೇ
ಪೆರ್ಮಾನ-ಡಿ-ಗಳ
ಪೆರ್ಮಾನ-ಡಿ-ಗಳು
ಪೆರ್ಮಾನ-ಡಿಯ
ಪೆರ್ಮಾನ-ಡಿ-ಯ-ಎರಡನೇ
ಪೆರ್ಮಾನ-ಡಿ-ಯನ್ನು
ಪೆರ್ಮಾನ-ಡಿಯು
ಪೆರ್ಮಾನ-ಡಿ-ಯೆಂಬ
ಪೆರ್ಮ್ಮನ-ಡಿಗಳ
ಪೆರ್ಮ್ಮಾ-ನಡಿ
ಪೆರ್ಮ್ಮಾನ-ಡಿ-ಗಳ್
ಪೆರ್ಮ್ಮಾನ-ಡಿಯ
ಪೆರ್ಮ್ಮೆ
ಪೆರ್ರಂಡಿ
ಪೆರ್ವ್ವ-ಯಲ
ಪೆರ್ವ್ವಳ್ಳಂ
ಪೆಸರಂ
ಪೆಸ-ರಾಗಿರ್ದ-ಪುದು
ಪೆಸರುಂ
ಪೆಸರ್ವೆತ್ತುದಂ
ಪೆಸಾಳಿ
ಪೆಸಾಳಿ-ಹನುಮ
ಪೇಟಿ-ರಾ-ಜಯ್ಯ
ಪೇಟಿ-ರಾಜಯ್ಯನು
ಪೇಟೆ
ಪೇಟೆ-ನಾಡ-ದೇಶದ
ಪೇಟೆ-ಯನ್ನಾಗಿ
ಪೇಠೆ-ಯಲಿ
ಪೇದ
ಪೇರಾಳ್ಕೆ
ಪೇರೂರಿ-ನಲ್ಲಿ
ಪೇರೂರಿ-ನಲ್ಲಿದ್ದ
ಪೇರೂರೊಳಿದ್ದು-ರಿನ್ನಾಕ-ನಿಪ್ಪ
ಪೇಳೆಂಬಿನಂ
ಪೇಳ್ದನ-ೞಅ್ತಯಿಂ
ಪೇಳ್ವೆ
ಪೈಕಿ
ಪೈಗಂಬರ್
ಪೈರು
ಪೊ
ಪೊಂನಣ್ಣ
ಪೊಂಮಿಗೆ
ಪೊಂಮು
ಪೊಕ್ಕು
ಪೊಕ್ಕುಮೆ
ಪೊಗಲ್ತೊ-ಣೆಯಾಂಡ-ನನ್ನು
ಪೊಗ-ಳಲು
ಪೊಗಳೆ
ಪೊಗಳ್ಗು
ಪೊಗಳ್ತೆಯೆಂ
ಪೊಗಳ್ದ-ಪುದು
ಪೊಗಳ್ವಿನಂ
ಪೊಟ್ಟಳಿ-ಸುವ
ಪೊಡರ್ಪ್ಪವೇ-ವೇಳ್ವುದೋ
ಪೊಡ-ವಿಗೆ
ಪೊಡವಿಯೊಳೆ
ಪೊದೆ-ಯಲ್ಲಿ-ರುವ
ಪೊನ್
ಪೊನ್ಕೆವಿ
ಪೊನ್ನ
ಪೊನ್ನ-ಕಬ್ಬೆಯ
ಪೊನ್ನ-ಗಾವುಣ್ಡ
ಪೊನ್ನಡಿ
ಪೊನ್ನದಿ
ಪೊನ್ನ-ಪಚ್ಚಮಂ
ಪೊನ್ನಪ್ಪ
ಪೊನ್ನ-ಲ-ದೇವಿ-ಯರ
ಪೊನ್ನಳ್ಳಿ
ಪೊನ್ನಳ್ಳಿ-ಯನ್ನು
ಪೊನ್ನು
ಪೊನ್ನೆತ್ತಿ-ಕೊಳೆ
ಪೊನ್ನೆತ್ತಿ-ಕೊಳ್ಳು-ವಂತೆ
ಪೊನ್ನ್ನು
ಪೊನ್ವಿಟ್ಟು
ಪೊಮ್ಮು
ಪೊಯ್
ಪೊಯ್ದ
ಪೊಯ್ದಿರಿದಂ
ಪೊಯ್ದು
ಪೊಯ್ಸಳ
ಪೊಯ್ಸಳ-ದೇವ
ಪೊಯ್ಸಳ-ದೇವ-ರಸ
ಪೊಯ್ಸಳ-ದೇವ-ರ-ಸರು
ಪೊಯ್ಸಳ-ದೇವ-ರಾಜ್ಯ-ದಲ್ಲಿ
ಪೊಯ್ಸಳ-ದೇವರು
ಪೊಯ್ಸ-ಳನ
ಪೊಯ್ಸಳ-ನ-ರಾಜ್ಯ
ಪೊಯ್ಸ-ಳನು
ಪೊಯ್ಸಳ-ನೆಂಬ
ಪೊಯ್ಸ-ಳನೇ
ಪೊಯ್ಸಳ-ರಾಜ್ಯ-ದಲ್ಲಿ
ಪೊಯ್ಸಳ-ಸೆಟ್ಟಿ
ಪೊಯ್ಸಳ-ಸೆಟ್ಟಿ-ಯ-ರಾದ
ಪೊರರ
ಪೊರಳ್ಚಿ
ಪೊರುಮಾಮಿಲ್ಲ
ಪೊರೆ-ದ-ನೆಂದೂ
ಪೊರ್ಪಲಗೈಯಾನುಮ್
ಪೊಲು-ವರಂ
ಪೊಳಲ
ಪೊಳಲನ್ನಾಗಿ
ಪೊಳಲ-ಸೆಟ್ಟಿಗೆ
ಪೊಳಲ-ಸೆಟ್ಟಿಯು
ಪೊಳಲು
ಪೊಳಲ್ಚೋ-ರನ
ಪೊಳಲ್ಚೋ-ರನು
ಪೊಳಲ್ನ
ಪೊಸ
ಪೊಸ-ತಾಗೆ
ಪೊಸ-ಪೂ-ವಿನ
ಪೊಸ್ತಕ
ಪೊಸ್ತಕ-ಗಚ್ಚದ
ಪೊಸ್ತಕ-ಗಚ್ಛದ
ಪೊಸ್ತಕ-ಗಚ್ಛ-ದ-ವ-ರಿಗಲ್ಲದೆ
ಪೊಸ್ಥಂ
ಪೊೞಲ-ಸೆಟ್ಟಿ
ಪೊೞಲ-ಸೆಟ್ಟಿಯು
ಪೋಗಿ
ಪೋಚಲ-ದೇವಿ-ಯ-ರರ್ತ್ಥಿ-ವಟ್ಟು
ಪೋಚಲ-ದೇವಿಯು
ಪೋಚವ್ವೆ-ಗಾಗಿ
ಪೋಚಾಂಬಿಕೆ
ಪೋಚಾಂಬಿ-ಕೆಯು
ಪೋಚಾಂಬಿಕೋ-ದರೋ-ದನ್ವತ್ಪಾ-ರಿಜಾತಂ
ಪೋಚಿ-ಕಬ್ಬೆ
ಪೋಚಿ-ಕಬ್ಬೆಯ
ಪೋಚಿ-ಕಬ್ಬೆ-ಯರ
ಪೋಚಿ-ಕಬ್ಬೆಯು
ಪೋಡಿನ
ಪೋಡು
ಪೋತ-ನಾಯಕ
ಪೋದಲ-ಶರ್ಮ
ಪೋಮನ್
ಪೋರಿಲಿಇಭದೆ
ಪೋರಿಲಿಭದೆ
ಪೋಲಾಂಡಲ-ಜೀಯ
ಪೋಲಾಂಡಲ-ಜೀಯ-ವೀತ-ರಾ-ಸಿಯು
ಪೋಲ್ತು
ಪೋಷಕ-ನಾಗಿ
ಪೋಷಕ-ರಲ್ಲಿ
ಪೋಷಕ-ವಾ-ಗಿದ್ದು
ಪೋಷಣ-ನಿರ್ಭರ-ಭೂನವಖಂಡಃ
ಪೋಷಿತ-ನಾದ
ಪೋಸಳ
ಪೋಸಳ-ದೇವ
ಪೋಸಳ-ದೇವರು
ಪೋಸಳ-ದೇವ-ರೆಂದರೆ
ಪೋಸ-ಳನು
ಪೌತ್ರ
ಪೌತ್ರ-ರಾದ
ಪೌತ್ರರು
ಪೌತ್ರರೂ
ಪೌರ-ಸೇವೆಯ
ಪೌರಾಣಿಕ
ಪೌರಾಣಿ-ಕ-ರಾಗಿದ್ದರು
ಪೌರಾಣಿ-ಕ-ರಾಗಿದ್ದ-ವರು
ಪೌರೋಹಿತ್ಯ-ತನ
ಪೌರೋಹಿತ್ಯ-ವನ್ನು
ಪೌಳಿ-ಗೋಡೆಯ
ಪ್ಪಿಲಿ-ಸೋಮ
ಪ್ಪುತ್ರ-ಮಿತ್ರಸ್ತೋಮಂ
ಪ್ಪೊನ್ನುಕ್ಕುಮಾಗ
ಪ್ಯಾಟೆ
ಪ್ಯಾರ-ಸಾಬಾದ್
ಪ್ರಉಡ-ದೇವ-ರಾಯ
ಪ್ರಕಟ-ಗೊಂಡು
ಪ್ರಕಟಗೊಳ್ಳು-ವಂತೆ
ಪ್ರಕಟಣಾ
ಪ್ರಕಟಣೆ
ಪ್ರಕಟಣೆ-ಗ-ಳನ್ನು
ಪ್ರಕಟಣೆ-ಗಳು
ಪ್ರಕಟಣೆಗೆ
ಪ್ರಕಟಣೆ-ಯಾಗಿ
ಪ್ರಕಟಣೆ-ಯಾಗುತ್ತಿ-ರುವ
ಪ್ರಕಟ-ವಾಗ-ಬೇಕಾಗಿದೆ
ಪ್ರಕಟ-ವಾಗಿದೆ
ಪ್ರಕಟ-ವಾ-ಗಿದ್ದು
ಪ್ರಕಟ-ವಾಗಿ-ರುವ
ಪ್ರಕಟ-ವಾಗಿವೆ
ಪ್ರಕಟ-ವಾಗುತ್ತಿದೆ
ಪ್ರಕಟ-ವಾಗುತ್ತಿದ್ದ
ಪ್ರಕಟ-ವಾಗುತ್ತಿದ್ದವು
ಪ್ರಕಟ-ವಾಗು-ವುದ-ರೊಂದಿಗೆ
ಪ್ರಕಟ-ವಾದ
ಪ್ರಕಟ-ವಾದವು
ಪ್ರಕಟಿಸ-ಬೇಕು
ಪ್ರಕಟಿಸ-ಲಾಗಿದೆ
ಪ್ರಕ-ಟಿಸಿ
ಪ್ರಕ-ಟಿಸಿ-ದರು
ಪ್ರಕಟಿ-ಸಿದೆ
ಪ್ರಕ-ಟಿಸಿದ್ದಾರೆ
ಪ್ರಕ-ಟಿಸಿ-ರುವ
ಪ್ರಕಟಿ-ಸುತ್ತಿದೆ
ಪ್ರಕಟಿ-ಸುತ್ತೇ-ವೆಂದು
ಪ್ರಕಟಿ-ಸುವ
ಪ್ರಕರ-ಣ-ಗಳು
ಪ್ರಕರ-ಣ-ದಿಂದ
ಪ್ರಕಾಂಡ
ಪ್ರಕಾರ
ಪ್ರಕಾರ-ಗ-ಳನ್ನು
ಪ್ರಕಾರ-ಗ-ಳಲ್ಲಿ
ಪ್ರಕಾರ-ಗಳು
ಪ್ರಕಾರದ
ಪ್ರಕಾರ-ವನ್ನು
ಪ್ರಕಾರವೂ
ಪ್ರಕಾರವೇ
ಪ್ರಕಾಶಂ
ಪ್ರಕಾಶ-ನದ
ಪ್ರಕಾಶ-ನದ-ವರ
ಪ್ರಕಾಶ-ನದಿಂದ
ಪ್ರಕೃತ
ಪ್ರಕೃತಃ
ಪ್ರಕೃತಯಃ
ಪ್ರಕೃತಿ
ಪ್ರಕೃತಿ-ಜನ್ಯ
ಪ್ರಕೃತೀನಾಂ
ಪ್ರಕೃಷ್ಟೋಭಯ-ವೇದಾಂತ
ಪ್ರಕ್ರಿ-ಯಯ
ಪ್ರಕ್ರಿಯೆ
ಪ್ರಕ್ರಿ-ಯೆ-ಯಲ್ಲಿ
ಪ್ರಕ್ರಿ-ಯೆಯು
ಪ್ರಕ್ಷಾಳಿತ
ಪ್ರಕ್ಷುಬ್ದ-ವಾಗಿ
ಪ್ರಖ್ಯಾತ
ಪ್ರಖ್ಯಾತಂ
ಪ್ರಖ್ಯಾತ-ನಾಗಿ
ಪ್ರಖ್ಯಾತ-ನಾದ
ಪ್ರಖ್ಯಾತ-ನಾದನು
ಪ್ರಖ್ಯಾತ-ರಾಗಿ
ಪ್ರಖ್ಯಾತ-ರಾ-ಗಿದ್ದು
ಪ್ರಖ್ಯಾತ-ವಾ-ಗಿತ್ತು
ಪ್ರಖ್ಯಾತ-ವಾಗಿದೆ
ಪ್ರಖ್ಯಾತ-ವಾಯಿತು
ಪ್ರಖ್ಯಾತಾ
ಪ್ರಖ್ಯಾತೌ
ಪ್ರಗತಿ
ಪ್ರಗತಿ-ಪರ
ಪ್ರಗಲ್ಬ-ನೆಂದು
ಪ್ರಚಂಡ
ಪ್ರಚಂಡ-ದಂಡ-ನಾಯಕ
ಪ್ರಚಂಡ-ದೇವ
ಪ್ರಚಂಡ-ಪುಂಡರಿಕ-ಮದ-ವೇದಂಡ-ರು-ಮಪ್ಪ
ಪ್ರಚಲಿತ-ದಲ್ಲಿತ್ತು
ಪ್ರಚಲಿತ-ದಲ್ಲಿದ್ದವು
ಪ್ರಚಲಿತ-ವಾ-ಗಿದ್ದ
ಪ್ರಚಲಿತ-ವಾಗಿ-ರುವುದ
ಪ್ರಚಲಿತ-ವಾಗಿ-ರುವು-ದ-ರಿಂದ
ಪ್ರಚಲಿತ-ವಾಗಿವೆ
ಪ್ರಚಲಿತ-ವಿತ್ತು
ಪ್ರಚಲಿತ-ವಿದ್ದ
ಪ್ರಚಾರ
ಪ್ರಚಾರ-ಕರು
ಪ್ರಚಾರಕ್ಕೆ
ಪ್ರಚಾರ-ದಲ್ಲಿದ್ದಿರ-ಬಹು-ದಲ್ಲದೆ
ಪ್ರಜಾಃ
ಪ್ರಜಾ-ಧರ್ಮ-ಪರಿ-ಪಾಲ-ನಾದಿ
ಪ್ರಜಾ-ಸಮೂಹ
ಪ್ರಜಾಹಿತ
ಪ್ರಜೆ
ಪ್ರಜೆ-ಗ-ಗೌಂಡ-ಗಳು
ಪ್ರಜೆ-ಗಳ
ಪ್ರಜೆ-ಗ-ಳನ್ನು
ಪ್ರಜೆ-ಗ-ಳಿಂದ
ಪ್ರಜೆ-ಗ-ಳಿಗೆ
ಪ್ರಜೆ-ಗಳು
ಪ್ರಜೆ-ಗ-ಳೆಂದು
ಪ್ರಜೆ-ಗಾವುಂಡ
ಪ್ರಜೆ-ಗಾವುಂಡ-ಗಳು
ಪ್ರಜೆ-ಗಾವುಂಡ-ನನ್ನಾಗಿ
ಪ್ರಜೆ-ಗಾವುಂಡರ
ಪ್ರಜೆ-ಗಾವುಂಡ-ರನ್ನು
ಪ್ರಜೆ-ಗಾವುಂಡ-ರನ್ನೇ
ಪ್ರಜೆ-ಗಾವುಂಡ-ರಿಗೆ
ಪ್ರಜೆ-ಗಾವುಂಡ-ರಿರ-ಬ-ಹುದು
ಪ್ರಜೆ-ಗಾವುಂಡರು
ಪ್ರಜೆ-ಗಾವುಂಡು-ಗಳು
ಪ್ರಜೆ-ಗೌಡಿನ
ಪ್ರಜೆ-ಗೌಡು-ಗಳು
ಪ್ರಜೆ-ನಾಯಕ
ಪ್ರಜೆ-ನಾಯ-ಕ-ರಿಗೆ
ಪ್ರಜೆ-ನಾಯ-ಕರು
ಪ್ರಜೆ-ನಾಯ್ಕ-ರಿಗೆ
ಪ್ರಜೆ-ಬೀಡಿನ-ವರು
ಪ್ರಜೆ-ಮೆಚ್ಚೆ-ಗಂಡ
ಪ್ರಜೆ-ಮೆಚ್ಚೆ-ಗಂಡ-ನು-ಮಪ್ಪ
ಪ್ರಜೆ-ಯನು
ಪ್ರಜೆ-ಯನೂ
ಪ್ರಜೆ-ಯಲ್ಲಿ
ಪ್ರಜೆಯು
ಪ್ರಜ್ಞೋಪಾಖ್ಯಾ
ಪ್ರಣೀತ-ವಾದ
ಪ್ರತಾನಂಗಳೊಳು
ಪ್ರತಾಪ
ಪ್ರತಾಪ-ಕಂಠೀ-ರವ
ಪ್ರತಾಪ-ಚಕ್ರ-ವರ್ತಿ
ಪ್ರತಾಪ-ದೇವ-ರಾಯ
ಪ್ರತಾಪ-ದೇವ-ರಾಯನ
ಪ್ರತಾಪ-ದೇವ-ರಾಯ-ನಿಗೆ
ಪ್ರತಾಪ-ದೇವ-ರಾಯ-ನೆಂಬ
ಪ್ರತಾಪ-ದೇವ-ರಾಯ-ಪುರ
ಪ್ರತಾಪ-ದೇವ-ರಾಯ-ಪುರ-ವೆಂಬ
ಪ್ರತಾಪ-ನಾರ-ಸಿಂಹ
ಪ್ರತಾಪ-ನಿಳಯಂ
ಪ್ರತಾಪ-ಪುರ-ವನ್ನು
ಪ್ರತಾಪ-ಮೆಂತೆಂದಡೆ
ಪ್ರತಾಪ-ವಂಶೇ
ಪ್ರತಾಪ-ವಾನ್
ಪ್ರತಾಪ-ವಿಜಯ-ಮದ-ನ-ಪುರ
ಪ್ರತಾಪ-ವೆಂತೆಂದಡೆ
ಪ್ರತಾಪ-ಸಮೇತರ್
ಪ್ರತಾಪ-ಹೊಯ್ಸಳ
ಪ್ರತಾಪಿ-ಗಳೂ
ಪ್ರತಿ
ಪ್ರತಿ-ಕೂಲ
ಪ್ರತಿಕ್ಷೇತ್ರ-ವಾಗಿ
ಪ್ರತಿಕ್ಷೇತ್ರ-ವಾಗಿ-ಬದ-ಲಾಗಿ
ಪ್ರತಿ-ಗ-ಳನ್ನು
ಪ್ರತಿಗ್ರಹ-ಗಳೆಂಬ
ಪ್ರತಿಗ್ರಹಿ
ಪ್ರತಿಗ್ರಹಿ-ಗಳ
ಪ್ರತಿಗ್ರಹಿ-ಯಾಗಿ
ಪ್ರತಿಜ್ಞೆ
ಪ್ರತಿಜ್ಞೆ-ಯನ್ನು
ಪ್ರತಿ-ದಿನ
ಪ್ರತಿ-ದಿನವೂ
ಪ್ರತಿ-ನಾಕ-ಮಲ್ಲ-ನೆಂಬ
ಪ್ರತಿ-ನಾಮ-ಕರಣ
ಪ್ರತಿ-ನಾಮ-ಧೇಯ
ಪ್ರತಿ-ನಾಮ-ಧೇಯ-ವನ್ನು
ಪ್ರತಿ-ನಾಮ-ಧೇಯ-ವಾದ
ಪ್ರತಿ-ನಾಮ-ಧೇಯ-ವಿಟ್ಟು
ಪ್ರತಿ-ನಾಮ-ಧೇಯ-ವಿತ್ತು
ಪ್ರತಿ-ನಾಮ-ಧೇಯ-ವುಳ್ಳ
ಪ್ರತಿ-ನಿತ್ಯ
ಪ್ರತಿ-ನಿಧಿ
ಪ್ರತಿ-ನಿಧಿ-ಗ-ಳನ್ನು
ಪ್ರತಿ-ನಿಧಿ-ಗ-ಳಿಗೆ
ಪ್ರತಿ-ನಿಧಿ-ಗಳು
ಪ್ರತಿ-ನಿಧಿಯ
ಪ್ರತಿ-ನಿಧಿ-ಯಾಗಿ
ಪ್ರತಿ-ನಿಧಿ-ಯಾದ
ಪ್ರತಿ-ನಿಧಿ-ಸಲು
ಪ್ರತಿ-ನಿಧಿ-ಸುತ್ತಾ
ಪ್ರತಿ-ಪದೆ
ಪ್ರತಿ-ಪನ್ನದಿ
ಪ್ರತಿ-ಪಾದಿ-ಸಲು
ಪ್ರತಿ-ಪಾಳಕ
ಪ್ರತಿ-ಪಾ-ಳನ
ಪ್ರತಿ-ಪಾ-ಳಿಸಿ
ಪ್ರತಿ-ಫಲ
ಪ್ರತಿ-ಫಲ-ವನ್ನು
ಪ್ರತಿ-ಬಂಧ-ಕ-ಗ-ಳಾದ
ಪ್ರತಿ-ಬದ್ಧ
ಪ್ರತಿ-ಬದ್ಧ-ವಾಗಿತ್ತೆಂದು
ಪ್ರತಿ-ಬಿಂಬ
ಪ್ರತಿ-ಬೋಧ-ದಿಂದ
ಪ್ರತಿ-ಭಾಗಿನ
ಪ್ರತಿ-ಭಾವಂತ-ರಾದ
ಪ್ರತಿ-ಭೆ-ಗ-ಳಾದ
ಪ್ರತಿ-ಮಾ-ಲಕ್ಷಣ
ಪ್ರತಿಮೆ
ಪ್ರತಿ-ಮೆ-ಗಳ
ಪ್ರತಿ-ಮೆ-ಗಳನು
ಪ್ರತಿ-ಮೆ-ಗ-ಳನ್ನು
ಪ್ರತಿ-ಮೆ-ಗಳಿವೆ
ಪ್ರತಿ-ಮೆ-ಗಳು
ಪ್ರತಿ-ಮೆಯ
ಪ್ರತಿ-ಮೆ-ಯನ್ನು
ಪ್ರತಿ-ಮೆ-ಯನ್ನೂ
ಪ್ರತಿ-ಯನ್ನು
ಪ್ರತಿ-ಯಮೆ-ಯನ್ನು
ಪ್ರತಿ-ಯಾಗಿ
ಪ್ರತಿಯು
ಪ್ರತಿ-ಯೊಂದ
ಪ್ರತಿ-ಯೊಂದು
ಪ್ರತಿ-ರೂಪ-ವೊಂದನ್ನು
ಪ್ರತಿ-ರೋಧದ
ಪ್ರತಿ-ರೋಧ-ವನ್ನು
ಪ್ರತಿ-ವರ್ಷ
ಪ್ರತಿ-ವರ್ಷವೂ
ಪ್ರತಿ-ವಾದಿ
ಪ್ರತಿ-ವಾದಿ-ಬುಧ
ಪ್ರತಿಷ್ಟ
ಪ್ರತಿಷ್ಟಾ-ಚಾರ್ಯ
ಪ್ರತಿಷ್ಟಾ-ಚಾರ್ಯ್ಯ
ಪ್ರತಿಷ್ಟಾಪಿ-ಸಿದುದೂ
ಪ್ರತಿಷ್ಠಾ-ಚಾರ್ಯ
ಪ್ರತಿಷ್ಠಾ-ಚಾರ್ಯರು
ಪ್ರತಿಷ್ಠಾ-ಚಾರ್ಯ್ಯ
ಪ್ರತಿಷ್ಠಾನ-ಪನೆ
ಪ್ರತಿಷ್ಠಾಪಕ-ರೆಂದು
ಪ್ರತಿಷ್ಠಾ-ಪನಾ
ಪ್ರತಿಷ್ಠಾ-ಪನಾ-ಚಾರ್ಯ
ಪ್ರತಿಷ್ಠಾ-ಪನಾ-ಚಾರ್ಯ-ರಾದ
ಪ್ರತಿಷ್ಠಾ-ಪನೆ
ಪ್ರತಿಷ್ಠಾ-ಪನೆಗೆ
ಪ್ರತಿಷ್ಠಾ-ಪನೆ-ಯಾಗಿ
ಪ್ರತಿಷ್ಠಾಪಿ
ಪ್ರತಿಷ್ಠಾಪಿ-ಸ-ಲಾಗಿದೆ
ಪ್ರತಿಷ್ಠಾಪಿ-ಸಲಾಯಿ-ತೆಂದು
ಪ್ರತಿಷ್ಠಾಪಿ-ಸಲಾಯಿ-ತೆಂದೂ
ಪ್ರತಿಷ್ಠಾಪಿಸಿ
ಪ್ರತಿಷ್ಠಾಪಿ-ಸಿದ
ಪ್ರತಿಷ್ಠಾಪಿ-ಸಿ-ದ-ನೆಂದು
ಪ್ರತಿಷ್ಠಾಪಿ-ಸಿ-ದ-ನೆಂದೂ
ಪ್ರತಿಷ್ಠಾಪಿ-ಸಿ-ದ-ರೆಂದು
ಪ್ರತಿಷ್ಠಾಪಿ-ಸಿ-ದ-ರೆಂದೂ
ಪ್ರತಿಷ್ಠಾಪಿ-ಸಿ-ದಳು
ಪ್ರತಿಷ್ಠಾಪಿ-ಸಿದ್ದಾನೆ
ಪ್ರತಿಷ್ಠಾಪಿ-ಸಿದ್ದಾರೆ
ಪ್ರತಿಷ್ಠಾಪಿ-ಸಿದ್ದಾಳೆ
ಪ್ರತಿಷ್ಠಾಪಿ-ಸಿದ್ದಾಳೆಂದು
ಪ್ರತಿಷ್ಠಾಪಿ-ಸುವ
ಪ್ರತಿಷ್ಠಿತರುಂ
ಪ್ರತಿಷ್ಠೆ
ಪ್ರತಿಷ್ಠೆ-ಮಾಡಿ
ಪ್ರತಿಷ್ಠೆಯ
ಪ್ರತಿಷ್ಠೆ-ಯನ್ನು
ಪ್ರತಿಷ್ಠೆ-ಯಾಗಿ
ಪ್ರತಿಷ್ಠೆ-ಯಾ-ಗಿದ್ದು
ಪ್ರತಿಷ್ಠೆ-ಯಾದ
ಪ್ರತಿಷ್ಠೆ-ಯಾ-ಯಿತು
ಪ್ರತಿಸ್ಪರ್ಧಿ-ಗ-ಳಾಗಿದ್ದ
ಪ್ರತೀಕ-ವಾಗಿ
ಪ್ರತೀತಿ
ಪ್ರತೀ-ತಿಯು
ಪ್ರತೀತಿ-ಯುಂಟು
ಪ್ರತ್ಯಕ್ಷ
ಪ್ರತ್ಯಕ್ಷ-ದರ್ಶಿ-ಯೊಬ್ಬ
ಪ್ರತ್ಯರ್ತ್ಥಿಕ್ಷಿತಿ-ಪಾಲ-ರತ್ನ-ಮಕುಟೀನೀ-ರಾಜಿ-ತಾಂಘ್ರಿಶ್ಚಿರಂ
ಪ್ರತ್ಯೇಕ
ಪ್ರತ್ಯೇಕ-ವಾಗಿ
ಪ್ರತ್ಯೇಕ-ವಾಗಿದ್ದ-ವೆಂದು
ಪ್ರತ್ಯೇಕ-ವಾಗಿ-ರುತ್ತಿತ್ತು
ಪ್ರತ್ಯೇಕ-ವಾಗಿ-ರುವ
ಪ್ರತ್ಯೇಕ-ವಾದ
ಪ್ರತ್ಯೇಕ-ಸಾಲ
ಪ್ರತ್ಯೇಕಿಸಿ
ಪ್ರಥಮ
ಪ್ರಥ-ಮತೋ
ಪ್ರಥಮ-ಶಿಷ್ಯರು
ಪ್ರದಕ್ಷಿಣ-ಪಥ
ಪ್ರದಕ್ಷಿಣಾ-ಪಥ
ಪ್ರದಕ್ಷಿಣಾ-ಪಥ-ವಾ-ಗಿದ್ದು
ಪ್ರದರ್ಶನ
ಪ್ರದರ್ಶಿಸಿ-ರುವು-ದ-ರಿಂದ
ಪ್ರದಾನ-ವಾಗಿದ್ದವು
ಪ್ರದಾನ-ವಾಯಿತು
ಪ್ರದೇಶ
ಪ್ರದೇಶಕ್ಕೆ
ಪ್ರದೇಶ-ಗಳ
ಪ್ರದೇಶ-ಗ-ಳನ್ನು
ಪ್ರದೇಶ-ಗ-ಳನ್ನೂ
ಪ್ರದೇಶ-ಗ-ಳಲ್ಲಿ
ಪ್ರದೇಶ-ಗಳಲ್ಲಿ-ರುವ
ಪ್ರದೇಶ-ಗ-ಳಿಗೆ
ಪ್ರದೇಶ-ಗಳು
ಪ್ರದೇಶ-ಗಳೂ
ಪ್ರದೇಶದ
ಪ್ರದೇಶ-ದಕ್ಕೆ
ಪ್ರದೇಶ-ದಲಿ
ಪ್ರದೇಶ-ದಲ್ಲಿ
ಪ್ರದೇಶ-ದಲ್ಲಿತ್ತು
ಪ್ರದೇಶ-ದಲ್ಲಿದ್ದ
ಪ್ರದೇಶ-ದಲ್ಲಿದ್ದು-ಕೊಂಡು
ಪ್ರದೇಶ-ದಲ್ಲಿಯೇ
ಪ್ರದೇಶ-ದಲ್ಲಿ-ರುವ
ಪ್ರದೇಶ-ದಲ್ಲಿ-ರುವು-ದ-ರಿಂದ
ಪ್ರದೇಶ-ದಲ್ಲೂ
ಪ್ರದೇಶ-ದ-ವನೋ
ಪ್ರದೇಶ-ದ-ವರು
ಪ್ರದೇಶ-ದ-ವರೇ
ಪ್ರದೇಶ-ದಿಂದ
ಪ್ರದೇಶ-ವನ್ನು
ಪ್ರದೇಶ-ವಾ-ಗಿತ್ತು
ಪ್ರದೇಶ-ವಾಗಿದೆ
ಪ್ರದೇಶ-ವಾಚಕ-ವಾಗಿ
ಪ್ರದೇಶ-ವಾ-ದರೂ
ಪ್ರದೇಶವು
ಪ್ರದೇಶವೂ
ಪ್ರದೇಶ-ವೆಂದೂ
ಪ್ರದೇಶವೇ
ಪ್ರಧಾನ
ಪ್ರಧಾನ-ನಾಗಿ
ಪ್ರಧಾನ-ನಾಗಿ-ಮಂತ್ರಿ-ಯಾಗಿ
ಪ್ರಧಾನ-ನಾಗಿ-ರುತ್ತಾನೆ
ಪ್ರಧಾನ-ಪಾತ್ರ
ಪ್ರಧಾನ-ಮಂತ್ರಿ
ಪ್ರಧಾನರು
ಪ್ರಧಾನರೇ
ಪ್ರಧಾನ-ವಾಗಿ
ಪ್ರಧಾನ-ವಾಗಿದೆ
ಪ್ರಧಾನ-ವಾ-ಗಿದ್ದು
ಪ್ರಧಾನಿ
ಪ್ರಧಾನಿ-ಗಳಾಗ-ಲಿಲ್ಲ
ಪ್ರಧಾನಿ-ಯಾ-ಗಿದ್ದ
ಪ್ರಧಾನಿ-ಯಾಗಿದ್ದರೂ
ಪ್ರನೀತ
ಪ್ರಪಂಚದ
ಪ್ರಪಂಚಾಂಚಿತ
ಪ್ರಬಂಧ
ಪ್ರಬಂಧಕ್ಕಿಂತ
ಪ್ರಬಂಧಕ್ಕೆ
ಪ್ರಬಂಧ-ಗಳ
ಪ್ರಬಂಧ-ಗ-ಳನ್ನು
ಪ್ರಬಂಧ-ಗ-ಳಲ್ಲಿ
ಪ್ರಬಂಧ-ಗಳು
ಪ್ರಬಂಧದ
ಪ್ರಬಂಧ-ದಲ್ಲಿ
ಪ್ರಬಂಧ-ದಲ್ಲಿದ್ದ
ಪ್ರಬಂಧ-ವನ್ನು
ಪ್ರಬಂಧ-ವಾಗಿದೆ
ಪ್ರಬಂಧವು
ಪ್ರಬಲ
ಪ್ರಬಲ-ಮ-ಭೂತ್ತುಷ್ಟಿಃ
ಪ್ರಬಲ-ವಾ-ಗಿತ್ತು
ಪ್ರಬಲ-ವಾಗಿತ್ತೆಂದು
ಪ್ರಬಲ-ವಾದ
ಪ್ರಬಲಿ-ಸಲು
ಪ್ರಬಲಿಸಿ
ಪ್ರಭತ್ವ-ವನ್ನು
ಪ್ರಭವ
ಪ್ರಭಾ-ಕರ
ಪ್ರಭಾ-ಕರದ
ಪ್ರಭಾ-ಕರ-ವೇದಾಧ್ಯಯ-ನದ
ಪ್ರಭಾ-ಕರ್
ಪ್ರಭಾ-ಚಂದ್ರ
ಪ್ರಭಾ-ಚಂದ್ರನ
ಪ್ರಭಾ-ಚಂದ್ರ-ನನ್ನು
ಪ್ರಭಾ-ಚಂದ್ರನು
ಪ್ರಭಾ-ಚಂದ್ರ-ಸಿದ್ಧಾಂತ
ಪ್ರಭಾ-ಚಂದ್ರ-ಸಿದ್ಧಾಂತ-ದೇವ-ರಿಗೆ
ಪ್ರಭಾಚನ್ದ್ರ
ಪ್ರಭಾತೇ
ಪ್ರಭಾವ
ಪ್ರಭಾ-ವಕ್ಕೆ
ಪ್ರಭಾವ-ಗ-ಳಾದವು
ಪ್ರಭಾವದ
ಪ್ರಭಾವ-ದಿಂದ
ಪ್ರಭಾವ-ದಿಂದಾಗಿ
ಪ್ರಭಾವ-ನೆನಿಸಿ
ಪ್ರಭಾ-ವನ್ನು
ಪ್ರಭಾ-ವಳಿ
ಪ್ರಭಾವ-ಳಿ-ಗ-ಳನ್ನು
ಪ್ರಭಾವ-ಳಿ-ಯಲ್ಲಿ
ಪ್ರಭಾವವು
ಪ್ರಭಾವವೇ
ಪ್ರಭಾವ-ಶಾಲಿ-ಗ-ಳಾಗಿದ್ದರು
ಪ್ರಭಾವ-ಶಾಲಿ-ಯಾಗಿ
ಪ್ರಭಾವ-ಶಾಲಿ-ಯಾ-ಗಿತ್ತು
ಪ್ರಭಾವಾವ-ತಾರಿತ
ಪ್ರಭಾವಿ
ಪ್ರಭಾವಿ-ಯಾಗಿ
ಪ್ರಭು
ಪ್ರಭು-ಗಳ
ಪ್ರಭು-ಗ-ಳನ್ನು
ಪ್ರಭು-ಗ-ಳಾಗಿ
ಪ್ರಭು-ಗ-ಳಾಗಿದ್ದ
ಪ್ರಭು-ಗಳು
ಪ್ರಭು-ಗವು-ಡ-ಗಳು
ಪ್ರಭು-ಗವು-ಡು-ಗಳ
ಪ್ರಭು-ಗವು-ಡು-ಗಳು
ಪ್ರಭು-ಗಾವುಂಡ
ಪ್ರಭು-ಗಾವುಂಡ-ಗಳ
ಪ್ರಭು-ಗಾವುಂಡ-ಗಳು
ಪ್ರಭು-ಗಾವುಂಡ-ನೆಂದು
ಪ್ರಭು-ಗಾವುಂಡರ
ಪ್ರಭು-ಗಾವುಂಡ-ರಿಗೆ
ಪ್ರಭು-ಗಾವುಂಡ-ರಿದ್ದರು
ಪ್ರಭು-ಗಾವುಂಡರು
ಪ್ರಭು-ಗಾವುಂಡ-ರು-ಗ-ಳಾಗಿದ್ದ-ರೆಂದು
ಪ್ರಭು-ಗಾವುಂಡ-ರುಪ್ರಜೆ-ಗಾವುಂಡರು
ಪ್ರಭು-ಗಾವುಂಡ-ರುಪ್ರಜೆ-ಗಾವುಂಡ-ರು-ಗಾವುಂಡರು
ಪ್ರಭು-ಗಾವುಂಡ-ರೆಂದು
ಪ್ರಭು-ಗಾವುಂಡುಗ-ಗಳು
ಪ್ರಭು-ಗಾವುಂಡು-ಗಳ
ಪ್ರಭು-ಗಾವುಂಡು-ಗಳು
ಪ್ರಭು-ಗಾ-ವುಡು-ಗಳು
ಪ್ರಭು-ಗೊಡಗೆ
ಪ್ರಭು-ತನಕ್ಕೆ
ಪ್ರಭುತ್ವ
ಪ್ರಭುತ್ವಕ್ಕೆ
ಪ್ರಭುತ್ವದ
ಪ್ರಭುತ್ವ-ವನ್ನು
ಪ್ರಭುತ್ವವು
ಪ್ರಭುತ್ವ-ಸಹಿತ
ಪ್ರಭುತ್ವ-ಸಹಿತಂ
ಪ್ರಭುತ್ವ-ಸಹಿತ-ವಾಗಿ
ಪ್ರಭು-ದೇವರ
ಪ್ರಭು-ದೇವ-ರ-ಕಟ್ಟೆ
ಪ್ರಭು-ಪೆರ್ಗ್ಗಡೆ
ಪ್ರಭು-ವರ್ಗ-ದ-ವರ
ಪ್ರಭು-ವರ್ಗ-ದ-ವ-ರಿದ್ದರು
ಪ್ರಭು-ವರ್ಗ-ದ-ವರು
ಪ್ರಭು-ವರ್ಗ-ದ-ವ-ರೆಂದು
ಪ್ರಭು-ವಾ-ಗಿದ್ದ
ಪ್ರಭು-ವಾಗಿದ್ದ-ನೆಂದು
ಪ್ರಭು-ವಾಗಿದ್ದ-ವ-ನನ್ನು
ಪ್ರಭು-ವಾ-ಗಿದ್ದು
ಪ್ರಭು-ವಾದ
ಪ್ರಭು-ವಿನ
ಪ್ರಭುವು
ಪ್ರಭು-ಶಕ್ತಿ
ಪ್ರಭು-ಶಕ್ತಿ-ಯನಾಂತ
ಪ್ರಭೇ-ದವೇ
ಪ್ರಭೇದ-ವೊಂದು
ಪ್ರಮಥಯ್ಯ
ಪ್ರಮಥಯ್ಯನ
ಪ್ರಮಾಣ-ಗ-ಳಿಂದ
ಪ್ರಮಾಣಜ್ಞ-ರಾದ
ಪ್ರಮಾ-ಣದ
ಪ್ರಮಾಣ-ದಲ್ಲಿ
ಪ್ರಮಾಣಾಂಜಲಿಃ
ಪ್ರಮಾಣೀಷು
ಪ್ರಮಾದೀಚ
ಪ್ರಮುಖ
ಪ್ರಮುಖಃ
ಪ್ರಮುಖ-ನಾ-ಗಿದ್ದ
ಪ್ರಮುಖ-ನಾಗಿದ್ದ-ನೆಂದು
ಪ್ರಮುಖ-ನಾ-ಗಿದ್ದು
ಪ್ರಮುಖ-ನೆಂದು
ಪ್ರಮುಖ-ಪಾತ್ರ
ಪ್ರಮುಖ-ಪಾತ್ರ-ವಹಿಸಿದ್ದ
ಪ್ರಮುಖ-ಮುಖ್ಯ
ಪ್ರಮುಖ-ಮುಖ್ಯರು
ಪ್ರಮುಖರ
ಪ್ರಮುಖ-ರನ್ನು
ಪ್ರಮುಖ-ರಾ-ಗಿದ್ದರು
ಪ್ರಮುಖ-ರಾಗಿದ್ದ-ರೆಂದು
ಪ್ರಮುಖ-ರಾ-ಗಿದ್ದ-ರೆಂಬುದು
ಪ್ರಮುಖ-ರಾ-ಗಿದ್ದು
ಪ್ರಮುಖ-ರಾದ-ವ-ರನ್ನು
ಪ್ರಮುಖ-ರಿಗೆ
ಪ್ರಮುಖರು
ಪ್ರಮುಖ-ವಾಗಿ
ಪ್ರಮುಖ-ವಾ-ಗಿತ್ತು
ಪ್ರಮುಖ-ವಾಗಿತ್ತೆಂದು
ಪ್ರಮುಖ-ವಾಗಿವೆ
ಪ್ರಮುಖ-ವಾಗುತ್ತದೆ
ಪ್ರಮುಖ-ವಾಗುತ್ತ-ದೆಂದು
ಪ್ರಮುಖ-ವಾದ
ಪ್ರಮುಖ-ವಾ-ದು-ದಾಗಿವೆ
ಪ್ರಮುಖ-ವಾ-ದುದು
ಪ್ರಮುಖ-ವಾದು-ವೆಂದರೆ
ಪ್ರಮುಖವೂ
ಪ್ರಮುಖ-ವೆಂದೂ
ಪ್ರಯತ್ನ
ಪ್ರಯತ್ನಕ್ಕೆ
ಪ್ರಯತ್ನ-ಗಳು
ಪ್ರಯತ್ನ-ದಲ್ಲೂ
ಪ್ರಯತ್ನ-ವನ್ನು
ಪ್ರಯತ್ನವು
ಪ್ರಯತ್ನಿಸ-ದಿ-ರುವುದು
ಪ್ರಯತ್ನಿಸಿ-ದರು
ಪ್ರಯತ್ನಿಸಿ-ದಾಗ
ಪ್ರಯತ್ನಿ-ಸಿದೆ
ಪ್ರಯತ್ನಿಸಿ-ರ-ಬ-ಹುದು
ಪ್ರಯತ್ನಿಸಿ-ರ-ಬಹು-ದೆಂದು
ಪ್ರಯತ್ನಿಸುತ್ತಿದ್ದನು
ಪ್ರಯತ್ನಿಸುತ್ತಿದ್ದರು
ಪ್ರಯಾಗ-ಪೆರು-ಮಾಳೆ
ಪ್ರಯಾಣ
ಪ್ರಯಾ-ಣದ
ಪ್ರಯಾಣ-ದಲ್ಲಿ
ಪ್ರಯಾಣಿಕ-ರಿಗೆ
ಪ್ರಯೋಕ್ತೃ-ಕುಶಲೋ
ಪ್ರಯೋಗ
ಪ್ರಯೋಗಕ್ಕೂ
ಪ್ರಯೋಗ-ಗ-ಳನ್ನು
ಪ್ರಯೋಗ-ಗಳು
ಪ್ರಯೋಗದ
ಪ್ರಯೋಗ-ವನ್ನು
ಪ್ರಯೋಗ-ವಾಗಿದೆ
ಪ್ರಯೋಗ-ವಾ-ಗಿದ್ದು
ಪ್ರಯೋಗ-ವಾಗಿ-ರುವ
ಪ್ರಯೋಗ-ವಾಗಿ-ರುವುದು
ಪ್ರಯೋಗ-ವಾಗಿವೆ
ಪ್ರಯೋಗ-ವಿದೆ
ಪ್ರಯೋಗ-ವಿದ್ದು
ಪ್ರಯೋಗವು
ಪ್ರಯೋಗಿ-ಸ-ಲಾಗಿದೆ
ಪ್ರಯೋಗಿ-ಸ-ಲಾದ
ಪ್ರಯೋಗಿಸಿ
ಪ್ರಯೋಗಿ-ಸಿ-ರುವ
ಪ್ರಯೋಗಿ-ಸಿ-ರು-ವು-ದಿಲ್ಲ
ಪ್ರಯೋಜ-ವಿಲ್ಲ
ಪ್ರವ-ಚನ
ಪ್ರವರ್ತಿ-ಸಲು
ಪ್ರವರ್ತಿ-ಸಿದ
ಪ್ರವರ್ತಿ-ಸಿದ್ದ
ಪ್ರವರ್ತಿ-ಸುತ್ತಿದ್ದ-ನೆಂದು
ಪ್ರವರ್ಧ-ಮಾನಕ್ಕೆ
ಪ್ರವರ್ಧ-ಮಾನ-ವಾಗಿ
ಪ್ರವಾದಿ
ಪ್ರವಾಸ
ಪ್ರವಾಸಿ
ಪ್ರವಾಸಿ-ಗ-ರಾದ
ಪ್ರವಾಸಿ-ಗ-ರಿಗೆ
ಪ್ರವಾಸಿ-ಗ-ಳಾಗಿ
ಪ್ರವಾಹ-ದಲ್ಲಿ
ಪ್ರವಾಹದಿಂ
ಪ್ರವಿಷ್ಟಕ್ಕೆ
ಪ್ರವಿಷ್ಟದ
ಪ್ರವಿಷ್ಠದ
ಪ್ರವೀಣ-ನಾಗಿದ್ದ-ನೆಂದು
ಪ್ರವೀಣ-ನಾದ
ಪ್ರವೀಣ-ರಾಗಿದ್ದರು
ಪ್ರವೀಣ-ರಾದ
ಪ್ರವು-ಡ-ದೇವ-ರಾಯ
ಪ್ರವು-ಡೆ-ಯರುಂ
ಪ್ರವು-ಢಪ್ರತಾಪ
ಪ್ರವೃತ್ತ-ವಾಗಿ
ಪ್ರವೇಶ
ಪ್ರವೇ-ಶಕ್ಕೆ
ಪ್ರವೇ-ಶ-ದಿಂದ
ಪ್ರವೇ-ಶದ್ವಾರ-ಗಳಿವೆ
ಪ್ರವೇ-ಶದ್ವಾರದ
ಪ್ರವೇ-ಶ-ವಾದ
ಪ್ರಶಸ್ತಿ
ಪ್ರಶಸ್ತಿಯ
ಪ್ರಶಸ್ತಿ-ಯನ್ನು
ಪ್ರಶಸ್ತಿ-ಶಾ-ಸನ
ಪ್ರಶಸ್ತಿ-ಶಾ-ಸನವು
ಪ್ರಶಸ್ತಿ-ಸಹಿತ
ಪ್ರಶಿದ್ಧಃ
ಪ್ರಶಿಷ್ಯ
ಪ್ರಶಿಷ್ಯರು
ಪ್ರಶ್ನಾರ್ಥಕ
ಪ್ರಶ್ನಾರ್ಹ-ವಾಗಿ
ಪ್ರಶ್ನೆ
ಪ್ರಶ್ನೆಗೆ
ಪ್ರಸಂಗ
ಪ್ರಸಂಗ-ಗಳಲ್ಲದೆ
ಪ್ರಸಂಗ-ದಲ್ಲಿ
ಪ್ರಸಂಗ-ವನ್ನು
ಪ್ರಸಂಗವು
ಪ್ರಸಕ್ತ
ಪ್ರಸಕ್ತಿ
ಪ್ರಸನ್ನ
ಪ್ರಸನ್ನ-ಕೇಶವ-ಪುರ
ಪ್ರಸನ್ನ-ಮಾಧವ
ಪ್ರಸನ್ನ-ಮಾಧವ-ಪುರ-ಸತ್ಯಾ-ಗಾಲ
ಪ್ರಸನ್ನ-ಮೂರ್ತಿ
ಪ್ರಸನ್ನ-ವೆಂಕಟ-ರಮಣಸ್ವಾಮಿ
ಪ್ರಸಸ್ತಿ
ಪ್ರಸಸ್ತಿ-ಯನ್ನು
ಪ್ರಸಾದ
ಪ್ರಸಾದಕ್ಕೆ
ಪ್ರಸಾದದ
ಪ್ರಸಾದ-ದಲ್ಲಿ
ಪ್ರಸಾದ-ನ-ಗಳು
ಪ್ರಸಾದ-ವನ್ನು
ಪ್ರಸಾದವು
ಪ್ರಸಾದವೂ
ಪ್ರಸಾದಾನು-ಭಾವಿ-ಗಳಪ್ಪ
ಪ್ರಸಾದಿತ
ಪ್ರಸಾರ
ಪ್ರಸಾರ-ಕ-ರಾದ
ಪ್ರಸಾರಕ್ಕೆ
ಪ್ರಸಾ-ರದ
ಪ್ರಸಾರ-ದಲ್ಲಿ
ಪ್ರಸಾರ-ವಾಯಿತು
ಪ್ರಸಿದ್ದ
ಪ್ರಸಿದ್ದ-ರಾ-ದ-ವರು
ಪ್ರಸಿದ್ಧ
ಪ್ರಸಿದ್ಧಃ
ಪ್ರಸಿದ್ಧ-ನಾಗಿ
ಪ್ರಸಿದ್ಧ-ನಾ-ಗಿದ್ದ
ಪ್ರಸಿದ್ಧ-ನಾಗಿದ್ದ-ನಂತೆ
ಪ್ರಸಿದ್ಧ-ನಾಗಿದ್ದನು
ಪ್ರಸಿದ್ಧ-ನಾದ
ಪ್ರಸಿದ್ಧ-ನಾದನು
ಪ್ರಸಿದ್ಧನು
ಪ್ರಸಿದ್ಧ-ರಾಗಿದ್ದರು
ಪ್ರಸಿದ್ಧ-ರಾಗಿದ್ದ-ರೆಂದು
ಪ್ರಸಿದ್ಧ-ರಾ-ಗಿದ್ದು
ಪ್ರಸಿದ್ಧ-ರಾದ
ಪ್ರಸಿದ್ಧ-ರಿದ್ದಂತೆ
ಪ್ರಸಿದ್ಧ-ವಾ-ಗಿತ್ತು
ಪ್ರಸಿದ್ಧ-ವಾಗಿತ್ತೆಂದು
ಪ್ರಸಿದ್ಧ-ವಾಗಿದೆ
ಪ್ರಸಿದ್ಧ-ವಾಗಿದ್ದವು
ಪ್ರಸಿದ್ಧ-ವಾ-ಗಿದ್ದು
ಪ್ರಸಿದ್ಧ-ವಾಗಿ-ರು-ವಂತೆ
ಪ್ರಸಿದ್ಧ-ವಾದ
ಪ್ರಸಿದ್ಧ-ವಾ-ದುದು
ಪ್ರಸಿದ್ಧಿಗೆ
ಪ್ರಸಿದ್ಧಿ-ಯನ್ನು
ಪ್ರಸಿದ್ಧಿ-ಯಾ-ಗಿತ್ತು
ಪ್ರಸಿದ್ಧಿ-ಯಾಗಿತ್ತೆಂದು
ಪ್ರಸಿದ್ಧಿ-ಯಾಗಿದೆ
ಪ್ರಸಿದ್ಧಿ-ಯಾದ
ಪ್ರಸೂ-ತರೂ
ಪ್ರಸ್ತಾಪ
ಪ್ರಸ್ತಾಪ-ಗ-ಳನ್ನು
ಪ್ರಸ್ತಾಪ-ವನ್ನು
ಪ್ರಸ್ತಾಪ-ವಾಗಿದೆ
ಪ್ರಸ್ತಾಪ-ವಾ-ಗಿದ್ದು
ಪ್ರಸ್ತಾಪ-ವಾಗಿಲ್ಲ
ಪ್ರಸ್ತಾಪ-ವಿದೆ
ಪ್ರಸ್ತಾಪ-ವಿದ್ದು
ಪ್ರಸ್ತಾಪ-ವಿ-ರುವ
ಪ್ರಸ್ತಾಪ-ವಿಲ್ಲ
ಪ್ರಸ್ತಾಪವು
ಪ್ರಸ್ತಾಪವೂ
ಪ್ರಸ್ತಾಪಿತ-ವಾಗಿ-ರುವ
ಪ್ರಸ್ತಾಪಿತ-ವಾಗಿವೆ
ಪ್ರಸ್ತಾಪಿಸ-ಲಾಗಿದೆ
ಪ್ರಸ್ತಾಪಿ-ಸಿದೆ
ಪ್ರಸ್ತಾಪಿ-ಸಿರುವ
ಪ್ರಸ್ತಾಪಿ-ಸುತ್ತದೆ
ಪ್ರಸ್ತಾಪಿ-ಸುತ್ತಾ
ಪ್ರಸ್ತಾಪಿ-ಸುವ
ಪ್ರಸ್ತಾವ
ಪ್ರಸ್ತಾವ-ನೆ-ಯಲ್ಲಿ
ಪ್ರಸ್ತಾವವೂ
ಪ್ರಸ್ತಾ-ವಿದೆ
ಪ್ರಸ್ತುತ
ಪ್ರಸ್ನನ್ನ
ಪ್ರಹಾರ
ಪ್ರಹುಡ-ದೇವ-ರಾಯ
ಪ್ರಹ್ಲಾದ
ಪ್ರಾಂಗಣ-ವಿದೆ
ಪ್ರಾಂತ-ಗಳ
ಪ್ರಾಂತ-ಗಳನ್ನು
ಪ್ರಾಂತ-ಗಳಾಗಿ
ಪ್ರಾಂತ-ಗಳಿಗೆ
ಪ್ರಾಂತ-ಗಳು
ಪ್ರಾಂತದ
ಪ್ರಾಂತ-ದಲ್ಲಿ
ಪ್ರಾಂತ-ದಿಂದ
ಪ್ರಾಂತ-ವನ್ನು
ಪ್ರಾಂತಾಧಿ-ಕಾರಿ-ಯಾಗಿ
ಪ್ರಾಂತಾಧಿ-ಕಾರಿ-ಯಾ-ಗಿದ್ದ
ಪ್ರಾಂತೀಯ
ಪ್ರಾಂತ್ಯ
ಪ್ರಾಂತ್ಯಕ್ಕೆ
ಪ್ರಾಂತ್ಯ-ಗಳ
ಪ್ರಾಂತ್ಯ-ಗ-ಳಲ್ಲಿ
ಪ್ರಾಂತ್ಯದ
ಪ್ರಾಂತ್ಯ-ದಲ್ಲಿ
ಪ್ರಾಂತ್ಯ-ದಿಂದ
ಪ್ರಾಂತ್ಯ-ವನ್ನಾಗಿ
ಪ್ರಾಕಾರ
ಪ್ರಾಕಾರದ
ಪ್ರಾಕಾರ-ದಲ್ಲಿ
ಪ್ರಾಕಾರ-ದಲ್ಲಿ-ರುವ
ಪ್ರಾಕು
ಪ್ರಾಕೃತ
ಪ್ರಾಕೃತಿ
ಪ್ರಾಕೃತಿಕ
ಪ್ರಾಗಿ-ತಿ-ಹಾಸ
ಪ್ರಾಗೈ-ತಿ-ಹಾಸಿಕ
ಪ್ರಾಚೀನ
ಪ್ರಾಚೀನ-ಕರ್ಮ
ಪ್ರಾಚೀನ-ಕಾಲಕ್ಕೆ
ಪ್ರಾಚೀನ-ಕಾಲ-ದಿಂದ
ಪ್ರಾಚೀ-ನತೆ
ಪ್ರಾಚೀನ-ತೆ-ಯನ್ನು
ಪ್ರಾಚೀನ-ಬ-ಸದಿ
ಪ್ರಾಚೀ-ನರು
ಪ್ರಾಚೀನ-ವಾ-ಗಿದ್ದು
ಪ್ರಾಚೀನ-ವಾದ
ಪ್ರಾಚೀನ-ವಾ-ದುದು
ಪ್ರಾಚೀನ-ವೆಂದು
ಪ್ರಾಚ್ಯ
ಪ್ರಾಚ್ಯ-ವಸ್ತು
ಪ್ರಾಜ್ಞೋಲಂಕಾರ-ಯಜ್ವಾ
ಪ್ರಾಣ
ಪ್ರಾಣ-ಘಾತ
ಪ್ರಾಣತ್ಯಾಗ
ಪ್ರಾಣದ
ಪ್ರಾಣ-ದೇವರ
ಪ್ರಾಣ-ನಾಥ
ಪ್ರಾಣ-ಲಿಂಗ
ಪ್ರಾಣ-ವನ್ನೂ
ಪ್ರಾಣಾಧಿ-ಕಾರಿ-ಗಳೂ
ಪ್ರಾಣಾರ್ಪಣೆ
ಪ್ರಾಣಿ
ಪ್ರಾಣಿ-ಗಳ
ಪ್ರಾಣಿ-ಗಳ-ಮೇಲೆ
ಪ್ರಾಣಿ-ಗಳಿಗೂ
ಪ್ರಾಣಿ-ಗ-ಳಿಗೆ
ಪ್ರಾಣಿ-ಗಳು
ಪ್ರಾಣಿ-ಗಳೊಡನೆ
ಪ್ರಾಣಿ-ಪಕ್ಷಿ-ಗಳು
ಪ್ರಾತಃ
ಪ್ರಾತಿ-ನಿಧಿ-ಕ-ವಾಗಿ
ಪ್ರಾತಿನಿಧ್ಯ
ಪ್ರಾದುದಭೂದ್ಗುಣಾಢ್ಯೋ-ನಾಮ್ನಾ
ಪ್ರಾದೇಶಿಕ
ಪ್ರಾಧಾನ್ಯ
ಪ್ರಾಧಾನ್ಯತೆ
ಪ್ರಾಧಾನ್ಯ-ತೆ-ಯನ್ನು
ಪ್ರಾಧಾನ್ಯ-ತೆ-ಯನ್ನೂ
ಪ್ರಾಧಾನ್ಯವು
ಪ್ರಾಧಿ-ಕಾರದ
ಪ್ರಾಪ್ತ
ಪ್ರಾಪ್ತ-ನಾಗಿ
ಪ್ರಾಪ್ತ-ನಾಗುತ್ತಾನೆ
ಪ್ರಾಪ್ತ-ನಾದ-ನೆಂದಿದೆ
ಪ್ರಾಪ್ತ-ನಾದ-ನೆಂದು
ಪ್ರಾಪ್ತ-ನಾದಾಗ
ಪ್ರಾಪ್ತ-ರಾಗುತ್ತಾರೆ
ಪ್ರಾಪ್ತ-ವನು
ಪ್ರಾಪ್ತ-ವಾಗಿ-ರು-ವು-ದನ್ನು
ಪ್ರಾಪ್ತ-ವಾದ
ಪ್ರಾಪ್ತ-ವಾಯಿತೋ
ಪ್ರಾಪ್ತಿಗೆ
ಪ್ರಾಪ್ತಿ-ಯಾಗಲಿ
ಪ್ರಾಪ್ತೈಃ
ಪ್ರಾಬಲ್ಯ
ಪ್ರಾಬಲ್ಯ-ದಿಂದಾಗಿ
ಪ್ರಾಬಲ್ಯ-ವಿದ್ದು
ಪ್ರಾಭವ
ಪ್ರಾಭಾವದಿಂ
ಪ್ರಾಮುಖ್ಯತೆ
ಪ್ರಾಮುಖ್ಯ-ತೆ-ಯನ್ನು
ಪ್ರಾಮುಖ್ಯ-ತೆ-ಯನ್ನೂ
ಪ್ರಾಮುಖ್ಯ-ತೆ-ಯಿಂದ
ಪ್ರಾಮುಖ್ಯ-ವಾಗಿ
ಪ್ರಾಯಶಃ
ಪ್ರಾಯಶ್ಚಿತ್ತ
ಪ್ರಾಯಶ್ಚಿತ್ತಕ್ಕಾಗಿ
ಪ್ರಾಯಶ್ಚಿತ್ತ-ಮವಾಂಘ್ರಿ-ವಾರಿ-ಜರ-ಜಃಸ್ನಾನಂ
ಪ್ರಾಯಶ್ಚಿತ್ತ-ವನ್ನು
ಪ್ರಾರಂಭ
ಪ್ರಾರಂಭ-ಗೊಂಡ
ಪ್ರಾರಂಭ-ಗೊಂಡಿತು
ಪ್ರಾರಂಭದ
ಪ್ರಾರಂಭ-ದಲ್ಲಿ
ಪ್ರಾರಂಭ-ವಾಗಿ
ಪ್ರಾರಂಭ-ವಾ-ಗಿದ್ದ
ಪ್ರಾರಂಭ-ವಾಗಿ-ರುವ
ಪ್ರಾರಂಭ-ವಾಗಿ-ರುವುದು
ಪ್ರಾರಂಭ-ವಾಗುತ್ತದೆ
ಪ್ರಾರಂಭ-ವಾಗುವ
ಪ್ರಾರಂಭ-ವಾಯಿತು
ಪ್ರಾರಂಭ-ವಾಯಿ-ತೆಂದು
ಪ್ರಾರಂಭಿಸಿ
ಪ್ರಾರಂಭಿಸಿ-ದ-ರೆಂದು
ಪ್ರಾರಂಭಿಸಿ-ದಾಗ
ಪ್ರಾರಂಭಿಸು
ಪ್ರಾರ್ಥನಾ
ಪ್ರಾರ್ಥನೆ
ಪ್ರಾರ್ಥನೆ-ಯಿಂದಲೇ
ಪ್ರಾರ್ಥಿ-ಸಲು
ಪ್ರಾರ್ಥಿಸಿ
ಪ್ರಾರ್ಥಿ-ಸುತ್ತೇನೆ
ಪ್ರಾಶಸ್ತ್ಯ
ಪ್ರಾಶಸ್ತ್ಯವು
ಪ್ರಾಶಸ್ತ್ಯ-ವುಳ್ಳ
ಪ್ರಾಸಂಗಿಕ-ವಾಗಿ
ಪ್ರಿಥುವಿಯ
ಪ್ರಿಥುವೀ
ಪ್ರಿಥ್ವೀ
ಪ್ರಿಯ-ತನ-ಯರು
ಪ್ರಿಯ-ತನ-ಯರೂ
ಪ್ರಿಯನೂ
ಪ್ರಿಯ-ಪುತ್ರಂ
ಪ್ರಿಯ-ಪುತ್ರ-ನೆಂದು
ಪ್ರಿಯ-ಪುತ್ರ-ರಾದ
ಪ್ರಿಯ-ಪುತ್ರ-ರೆನಿಸಿ-ಕೊಂಡ
ಪ್ರಿಯ-ರಾಗಿ
ಪ್ರಿಯರು
ಪ್ರಿಯ-ವಾದ
ಪ್ರಿಯ-ಶಿಷ್ಯ
ಪ್ರಿಯ-ಶಿಷ್ಯ-ರಾದ
ಪ್ರಿಯ-ಸುತ
ಪ್ರಿಯ-ಸುತ-ನೆಂದು
ಪ್ರಿಯ-ಸೇವಕ-ನಾದ
ಪ್ರಿಯ-ಸೇವಕ-ರಾದ
ಪ್ರೀಣನ
ಪ್ರೀತಿ
ಪ್ರೀತಿ-ಗೌ-ರವ-ಗ-ಳಿಂದ
ಪ್ರೀತಿ-ದಾನ-ವಾಗಿ
ಪ್ರೀತಿ-ಪಾತ್ರರು
ಪ್ರೀತಿಯ
ಪ್ರೀತಿ-ಯನ್ನು
ಪ್ರೀತಿ-ಯಿಂದ
ಪ್ರೀತ್ಯರ್ಥ-ವಾಗಿ
ಪ್ರುಥ್ವೀ
ಪ್ರುಥ್ವೀ-ರಾಜ್ಯಂಗೆ-ಯುತ್ತಿ-ರಲು
ಪ್ರುಥ್ವೀ-ವಲ್ಲಭ
ಪ್ರೇಮ
ಪ್ರೇಮಂ
ಪ್ರೇಮ-ಸಂಪದೇ
ಪ್ರೇಮಾ-ಲಯ-ಸುತ
ಪ್ರೇಯಸಿ-ವರ್ಗ-ದೊಳ್
ಪ್ರೇರಣೆ-ಯಂತೆ
ಪ್ರೇರಿಸಿ
ಪ್ರೇರೇ-ಪಣೆ
ಪ್ರೇರೇಪಿಸ-ದ-ವನು
ಪ್ರೇರೇಪಿಸಿ-ದ-ವನು
ಪ್ರೇರೇಪಿ-ಸಿರುತ್ತಾ-ರೆಂದು
ಪ್ರೊ
ಪ್ರೊಃ
ಪ್ರೊಫೆ-ಸರ್
ಪ್ರೋತ
ಪ್ರೋತ್ಸಾಹ
ಪ್ರೋತ್ಸಾಹಿಸ-ಬೇಕೆಂದು
ಪ್ರೋತ್ಸಾಹಿ-ಸಿದ
ಪ್ರೋತ್ಸಾಹಿ-ಸಿದರು
ಪ್ರೋಷ್ಠಿಲ
ಪ್ರೌಢ
ಪ್ರೌಢ-ದೇವ-ರಾಯ
ಪ್ರೌಢ-ದೇವ-ರಾಯನ
ಪ್ರೌಢ-ದೇವ-ರಾಯ-ನಲ್ಲಿ
ಪ್ರೌಢ-ದೇವ-ರಾಯನು
ಪ್ರೌಢ-ದೇವ-ರಾಯ-ಮಲ್ಲಿ-ಕಾರ್ಜುನ-ನ-ನಿಗೆ
ಪ್ರೌಢ-ದೇವೇಂದ್ರ-ನಿಗೆ
ಪ್ರೌಢಪ್ರತಾಪ
ಪ್ರೌಢಪ್ರಧಾನ
ಪ್ರೌಢ-ಭಾವ
ಪ್ರೌಢ-ರಾಯನ
ಪ್ರೌಢ-ರೇಖಾ
ಪ್ರೌಢ-ವಾ-ಗಿದ್ದು
ಪ್ರೌಢ-ಶಾಲೆ
ಪ್ರೌಢಿಮುಪೇಯುಷಂ
ಪ್ಲವ
ಫಕೀ-ರರು
ಫಕೀರಸ್ವಾಮಿ
ಫಣಿತಾಃ
ಫರಿಷ್ತಾನು
ಫರ್ಲಾಂಗ್
ಫಲ
ಫಲಂ
ಫಲ-ಗ-ಳನ್ನು
ಫಲನ್
ಫಲ-ಪುಷ್ಪ-ಗ-ಳನ್ನು
ಫಲಪ್ರದಂ
ಫಲಪ್ರದ-ವಾಗಿ
ಫಲ-ವತ್ತತೆ-ಯನ್ನೂ
ಫಲ-ವತ್ತಾದ
ಫಲ-ವತ್ತಾ-ದುದು
ಫಲ-ವೀವ
ಫಲ-ವೃಕ್ಷ-ಗ-ಳನ್ನು
ಫಲವೇ
ಫಲಾಕೃತೇಃ
ಫಲಾತಿಶಯಃ
ಫಳನೈಕ
ಫಸ-ಲಿನ
ಫಾತಿಮಾ
ಫಾದರ್
ಫಾದರ್ಹೆರಾಸ್
ಫಾರಂ
ಫಾಲ್ಗುಣ
ಫಿರೋಜ್
ಫಿರ್ಯಾದು
ಫಿಲಿ-ಯೋಜಾ
ಫೆಬ್ರ-ವರಿ
ಫೌಜ್ದಾರಿಯ
ಫೌಜ್ದಾರ್
ಫ್ರಾನ್ಸಿಸ್
ಫ್ರೆಂಚರ
ಫ್ರೆಂಚರು
ಫ್ರೆಂಚ್ರಾಕ್ಸ್
ಫ್ರೆಂಚ್ರಾಕ್ಸ್ನಲ್ಲಿ
ಫ್ರೆಂಚ್ರಾಕ್ಸ್ನಲ್ಲಿದ್ದ
ಫ್ರೆಂಚ್ರಾಕ್ಸ್ಹಿರೋಡೆ
ಫ್ಲೀಟ್
ಫ್ಲೀಟ್ರ-ವರ
ಬ
ಬಂಕ
ಬಂಕ-ನ-ಹಳ್ಳಿ
ಬಂಕ-ಪುರಂ
ಬಂಕಾ-ಪುರಕ್ಕೆ
ಬಂಕಾಪು-ರದ
ಬಂಕಾ-ಪುರ-ದಲ್ಲಿ
ಬಂಕಾ-ಪುರ-ದಲ್ಲಿದ್ದ
ಬಂಕಾಪು-ರದಿಂದ
ಬಂಕಾಪು-ರವೋ
ಬಂಕಿ-ನಾಡ
ಬಂಕಿ-ನಾಡನ್ನು
ಬಂಕಿ-ನಾಡು
ಬಂಕೆಯನ
ಬಂಕೆಯ-ನನ್ನು
ಬಂಕೆಯ-ನಿಗೆ
ಬಂಕೆಯನು
ಬಂಕೆಯುನು
ಬಂಕೇಶನ
ಬಂಕೇಶನು
ಬಂಕೇಶನೇ
ಬಂಗಲಿ
ಬಂಗ-ಲೆಯ
ಬಂಗ-ಲೆಯಲ್ಲಿ-ರುವ
ಬಂಗಾರ
ಬಂಗಾರದ
ಬಂಗಾಲ
ಬಂಗಾಳ-ದಲ್ಲಿ
ಬಂಙ್ಕೆಯನು
ಬಂಟ
ಬಂಟ-ಗೊಂತಮುನುಱಾ
ಬಂಟ-ರ-ಬಾ-ವನುಂ
ಬಂಟ-ರ-ಭಾವ
ಬಂಟ-ರ-ಭಾವನುಂ
ಬಂಡ-ಮಾರ-ನ-ಹಳ್ಳಿ
ಬಂಡ-ವಾಳ
ಬಂಡ-ಹಳ್ಳಿ
ಬಂಡಾಯ
ಬಂಡಿ
ಬಂಡಿ-ಗಳ
ಬಂಡಿ-ಗ-ಳನ್ನು
ಬಂಡಿ-ಗಳ-ವರ
ಬಂಡಿ-ಗ-ಳಿಗೆ
ಬಂಡಿ-ಗಳು
ಬಂಡಿ-ಪಾ-ತಿಯ
ಬಂಡಿ-ಬಸವ-ನಿಗೆ
ಬಂಡಿ-ಬಸವನು
ಬಂಡಿಯ
ಬಂಡಿ-ಯ-ದಾರಿ
ಬಂಡಿ-ಹಳ್ಳಿ
ಬಂಡಿ-ಹೊಳೆ
ಬಂಡಿ-ಹೊಳೆಯ
ಬಂಡೂರ
ಬಂಡೂರು
ಬಂಡೆ
ಬಂಡೆ-ಗ-ಳಿಂದ
ಬಂಡೆ-ಗಳಿ-ರು-ವು-ದನ್ನು
ಬಂಡೆ-ಗಳಿವೆ
ಬಂಡೆದ್ದಿದ್ದ
ಬಂಡೆದ್ದು
ಬಂಡೆಯ
ಬಂಡೆ-ಯನ್ನು
ಬಂಡೆ-ಯ-ಮೇಲೆ
ಬಂಡೆ-ಶಾ-ಸನ-ದಲ್ಲಿ
ಬಂಡೆ-ಹಳ್ಳ
ಬಂಣ
ಬಂಣಂಗಾಡಿ
ಬಂಣ-ಗಟ್ಟ
ಬಂಣ-ಗಟ್ಟಿ
ಬಂಣಿ-ಸುತ್ತ್ತುವಿರೆ
ಬಂತು
ಬಂದ
ಬಂದಂತೆ
ಬಂದಡಂ
ಬಂದ-ಡೆವೂ
ಬಂದ-ಣಿಕೆ
ಬಂದದ್ದು
ಬಂದ-ನಂತರ
ಬಂದ-ನಂತ-ರವೂ
ಬಂದ-ನಂತ-ರವೇ
ಬಂದನೀ
ಬಂದನು
ಬಂದ-ನೆಂದು
ಬಂದ-ನೆಂಬುದು
ಬಂದರು
ಬಂದರೂ
ಬಂದರೆ
ಬಂದ-ರೆಂದು
ಬಂದ-ರೆಂದೂ
ಬಂದ-ರೆಂಬ
ಬಂದಲಿ
ಬಂದಲ್ಲಿ
ಬಂದ-ವನಿದ್ದಡೂ
ಬಂದ-ವ-ನಿರ-ಬ-ಹುದು
ಬಂದ-ವನು
ಬಂದ-ವರ-ನೂರ-ವಕ್ಕ-ಲಾಗಿ
ಬಂದ-ವ-ರಾಗಿದ್ದ-ರಿಂದ
ಬಂದ-ವ-ರಾ-ಗಿದ್ದು
ಬಂದ-ವರಿ-ರ-ಬ-ಹುದು
ಬಂದ-ವರು
ಬಂದ-ವ-ರೆಂದು
ಬಂದ-ವರೇ
ಬಂದವು
ಬಂದ-ವು-ಗ-ಳಲ್ಲಿ
ಬಂದ-ವು-ಗ-ಳಾಗಿದ್ದಿರ-ಬೇಕೆಂದು
ಬಂದ-ಹಾಗೆ
ಬಂದಾಗ
ಬಂದಿತಂತೆ
ಬಂದಿತು
ಬಂದಿ-ತೆಂದು
ಬಂದಿ-ತೆಂದೂ
ಬಂದಿತ್ತು
ಬಂದಿತ್ತೆಂದು
ಬಂದಿತ್ತೆಂದೂ
ಬಂದಿತ್ತೆಂಬುದು
ಬಂದಿದೆ
ಬಂದಿದೆ-ಎಂದು
ಬಂದಿದೆಯೇ
ಬಂದಿದ್ದ
ಬಂದಿದ್ದನು
ಬಂದಿದ್ದ-ನೆಂದು
ಬಂದಿದ್ದ-ನೆಂದೂ
ಬಂದಿದ್ದರು
ಬಂದಿದ್ದರೂ
ಬಂದಿದ್ದ-ರೆಂದು
ಬಂದಿದ್ದ-ರೆಂಬ
ಬಂದಿದ್ದ-ಳೆಂದು
ಬಂದಿದ್ದ-ವರು
ಬಂದಿದ್ದಾಗ
ಬಂದಿದ್ದಾರೆ
ಬಂದಿದ್ದು
ಬಂದಿದ್ದೇ
ಬಂದಿ-ಯೋಜನ
ಬಂದಿ-ರ-ಬಹದು
ಬಂದಿರ-ಬಹು-ದಾದ
ಬಂದಿರ-ಬ-ಹುದು
ಬಂದಿರ-ಬಹು-ದೆಂದ
ಬಂದಿರ-ಬಹು-ದೆಂದು
ಬಂದಿರ-ಬೇಕು
ಬಂದಿರುತ್ತದೆ
ಬಂದಿರು-ಬ-ಹುದು
ಬಂದಿರುವ
ಬಂದಿರು-ವು-ದನ್ನು
ಬಂದಿರು-ವುದ-ರಿಂದ
ಬಂದಿ-ರುವುದು
ಬಂದಿರೆ
ಬಂದಿ-ಳಿದ-ನೆಂದು
ಬಂದಿವೆ
ಬಂದೀ
ಬಂದು
ಬಂದು-ದನ್ನು
ಬಂದು-ದರ
ಬಂದು-ದಲ್ಲ
ಬಂದುದು
ಬಂದುದೇ
ಬಂದು-ನೆಲೆ-ಸು-ವು-ದಕ್ಕೆ
ಬಂದೆ
ಬಂದೇ
ಬಂದೊಡನೆ
ಬಂದೊದಗಿದ
ಬಂಧ-ನ-ದಲ್ಲಿ-ರಿಸಿದ್ದ-ನಷ್ಟೆ
ಬಂಧಿ-ಯಾ-ಗಿದ್ದ
ಬಂಧಿ-ಯಾಗಿದ್ದಾಗ
ಬಂಧಿಸಿ
ಬಂಧಿಸಿ-ದಾಗ
ಬಂಧು-ಕಾರ
ಬಂಧು-ಗಳು
ಬಂಧು-ಜನಂಗಳು
ಬಂಧು-ಜನ-ಧ-ವಳ
ಬಂಧುಬಾಂಧವ-ರಿಗೆ
ಬಂಧುಬಾಂಧ-ವರು
ಬಂಧುರಿಮ-ಗುಣಿ
ಬಂಧುವುಮೆ-ನಿ-ಸಿದ
ಬಂನಿ-ಯೂರ
ಬಂನೂರು
ಬಂಮಚ
ಬಂಮಯನ
ಬಂಮಲ-ದೇವಿ
ಬಂಮೋ-ಜನ
ಬಇ-ಸಣಿಗೆ-ಯ-ನಿಕ್ಕಿ
ಬಕನೇ-ಕಲ-ಗುಡ್ಡ-ಹಳ್ಳ
ಬಕ-ರಿಪು
ಬಕಾಡೆ-ಹಳ್ಳಿ
ಬಕಾಡೇ-ಹಳ್ಳಿ
ಬಖೈರು-ಗ-ಳಲ್ಲಿ
ಬಗ-ಗಾವುಂಡ
ಬಗೆ
ಬಗೆ-ಗಿನ
ಬಗೆಗೂ
ಬಗೆಗೆ
ಬಗೆದು
ಬಗೆಯ
ಬಗೆ-ಯನ್ನು
ಬಗೆ-ಯಾಗಿದೆ
ಬಗೆಯು
ಬಗೆ-ಯೊಡೆ
ಬಗೆ-ಹರಿ-ಸಲು
ಬಗೆ-ಹರಿ-ಸುತ್ತಾನೆ
ಬಗೆ-ಹರಿ-ಸು-ವಾಗಲೂ
ಬಗ್ಗ-ವಳ್ಳಿ
ಬಗ್ಗ-ವಳ್ಳಿ-ಯನ್ನು
ಬಗ್ಗು
ಬಗ್ಗೆ
ಬಗ್ಗೆಯೂ
ಬಗ್ಗೆಯೇ
ಬಚಿ-ಗವುಂಡನ
ಬಜಗೂರಿನ
ಬಟ್ಟ-ಲಿನ
ಬಟ್ಟ-ಲಿನ-ಮೇಲೆ
ಬಟ್ಟಲು-ಗ-ಳನ್ನು
ಬಟ್ಟಾರ
ಬಟ್ಟೆ
ಬಟ್ಟೆ-ಗ-ಳನ್ನು
ಬಟ್ಟೆ-ಗ-ಳಿಗೆ
ಬಟ್ಟೆ-ಗಳು
ಬಟ್ಟೆಯ
ಬಟ್ಟೆ-ಯನ್ನು
ಬಡಗ
ಬಡಗಣ
ಬಡಗ-ಣ-ಕೋಟೆಯ
ಬಡಗ-ಣ-ಕೋಡಿ
ಬಡಗ-ಣ-ಕೋಡಿ-ಗಳ
ಬಡಗ-ಣ-ಹಳ್ಳ
ಬಡಗ-ರ-ನಾಡ
ಬಡಗ-ರ-ನಾಡು
ಬಡಗರೆ
ಬಡಗ-ರೆ-ನಾಡ
ಬಡಗ-ರೆ-ನಾಡೊಳ-ಗಣ
ಬಡಗಲ
ಬಡಗ-ಲಾಗಿ
ಬಡಗಲು
ಬಡಗಲ್
ಬಡಗಿ
ಬಡಗಿ-ಗಳು
ಬಡಗುಂಡ
ಬಡ-ಗುಂದ
ಬಡ-ಗುಂದ-ನಾಡ
ಬಡ-ಗುಂದ-ನಾಡನ್ನು
ಬಡ-ಗುಂದ-ನಾಡಿನ-ವ-ರಿಗೂ
ಬಡಗುಡ-ನಾಡ
ಬಡಗುಡ-ನಾಡು
ಬಡಗು-ನಾಡ
ಬಡಗು-ನಾಡು
ಬಡ-ಗೆರೆ
ಬಡ-ಗೆರೆ-ನಾಡ
ಬಡ-ಗೆರೆ-ನಾಡಾ-ಳುವ
ಬಡ-ಗೆರೆ-ನಾಡಿನ
ಬಡ-ಗೆರೆ-ನಾಡಿನ-ವ-ರೊಡನೆ
ಬಡ-ಗೆರೆ-ನಾಡು
ಬಡ-ಗೆರೆ-ನಾಡೊಳ-ಗಣ
ಬಡ-ಗೆರೆಯ
ಬಡವ-ನಾದೆ
ಬಡ-ವಾರ
ಬಡ-ವಾರ-ಕುಲದ
ಬಡ-ವಾರ-ಬಡು-ವಾರ
ಬಡಾಯಿಸುತ್ತಿದ್ದರು
ಬಡಿ-ಕೋಲ
ಬಡಿ-ಕೋಲ-ಭಟ್ಟ
ಬಡಿಯ-ಬೇಕೆಂಬ
ಬಡಿ-ಯಬ್ಬೆ
ಬಡಿ-ಯಬ್ಬೆಗೂ
ಬಡಿವ-ರ-ಸನು
ಬಡು-ವಾರ
ಬಡು-ವಾರ-ಬಡ-ವಾರ
ಬಡ್ಡಿ
ಬಡ್ಡಿಗೆ
ಬಡ್ಡಿಯ
ಬಡ್ಡಿ-ಯನ್ನು
ಬಡ್ಡಿ-ಯಲ್ಲಿ
ಬಡ್ಡಿ-ಯ-ಹಣ-ದಲ್ಲಿ
ಬಡ್ಡಿ-ಯಿಂದ
ಬಡ್ತಿ
ಬಡ್ತಿ-ಯನ್ನು
ಬಣಂಜ
ಬಣಂಜು
ಬಣಂಜು-ಧರ್ಮ
ಬಣಂಜು-ಧರ್ಮವು
ಬಣ-ಗ-ಳಲ್ಲಿ
ಬಣಜ-ವರ್ತನೆ
ಬಣಜಿಗ
ಬಣಜಿ-ಗರ
ಬಣಜಿ-ಗರು
ಬಣ್ಣ
ಬಣ್ಣಂಗಟ್ಟಿ-ಬನ್ನಂಗಾಡಿ
ಬಣ್ಣಂಗಟ್ಟಿಯು
ಬಣ್ಣ-ಬಳಿ-ದಿ-ರುವು-ದ-ರಿಂದ
ಬಣ್ಣಿ-ಗದೆ-ರೆ-ಹಳ್ಳಿ-ಯನ್ನು
ಬಣ್ಣಿಸ-ಲಾಗಿದೆ
ಬಣ್ಣಿ-ಸಿದೆ
ಬಣ್ಣಿ-ಸಿವೆ
ಬಣ್ಣಿ-ಸುತ್ತದೆ
ಬಣ್ಣಿ-ಸುತ್ತವೆ
ಬಣ್ಣಿಸೆ
ಬಣ್ನ-ಚಾರಿ
ಬಣ್ನಿಗ-ದರೆ-ಯನ್ನು
ಬಣ್ನಿ-ಗ-ದೆರೆ
ಬಣ್ನಿ-ಗದೆ-ರೆ-ಯನ್ನು
ಬಣ್ನಿ-ದರ-ಹಳ್ಳಿ
ಬಣ್ನಿ-ದರ-ಹಳ್ಳಿಯ
ಬಣ್ನಿಪು
ಬಣ್ನಿಸಲ್ಬಿಂಡಿ-ಗವಿ-ಲೆ-ಯೊಳಾ
ಬತ್ತದ
ಬತ್ತಿ
ಬತ್ತುತ್ತಿದೆ
ಬದನ-ಗುಪ್ಪೆ
ಬದರಿಕಾಶ್ರಮ
ಬದರಿ-ನಾ-ರಾಯ-ಣನ
ಬದ-ಲಾಗಿ
ಬದ-ಲಾ-ಯಿತು
ಬದಲಾಯಿ-ಸ-ಲಾ-ಯಿತು
ಬದಲಾಯಿಸಿ
ಬದಲಾಯಿಸಿ-ಕೊಂಡು
ಬದಲಾ-ವಣೆ
ಬದಲಾ-ವಣೆ-ಗ-ಳನ್ನು
ಬದಲಾ-ವಣೆ-ಗ-ಳಾಗಿ-ರುತ್ತವೆ
ಬದಲಾ-ವಣೆ-ಗಳೊಂದಿಗೆ
ಬದಲಾ-ವಣೆ-ಯನ್ನು
ಬದಲಾ-ವಣೆ-ಯಾದವು
ಬದಲಾ-ವಣೆ-ಯಿಂದ
ಬದಲಿ
ಬದ-ಲಿಗೆ
ಬದಲಿ-ಯಾಗಿ
ಬದಲಿ-ಸುತ್ತಾನೆ
ಬದಲು
ಬದಿ-ಗಿರಿಸಿ
ಬದಿಗೊತ್ತಿ
ಬದಿಗೊತ್ತಿ-ದನು
ಬದಿಯ
ಬದಿ-ಯಲ್ಲಿದ್ದು
ಬದಿ-ಯಲ್ಲಿ-ರುವ
ಬದುಕ-ಬೇಕು
ಬದುಕಿ-ದರು
ಬದುಕಿದ್ದ-ನೆಂದು
ಬದುಕಿದ್ದ-ನೆಂದೂ
ಬದುಕಿದ್ದ-ರೆಂದು
ಬದುಕಿದ್ದಾಗಲೇ
ಬದುಕಿದ್ದಿರ-ಬ-ಹುದು
ಬದುಕಿದ್ದು
ಬದುಕಿನ
ಬದು-ಕಿರು-ವಷ್ಟು
ಬದು-ಕಿರು-ವಾಗಲೇ
ಬದುಕುತ್ತಿದ್ದರು
ಬದ್ದೆಗ
ಬದ್ದೆ-ಗನ
ಬದ್ದೆ-ಗನು
ಬದ್ರಿ-ಹಾಳಿನ
ಬನ
ಬನದ
ಬನದ-ತೊಂಡ-ನೂರು
ಬನ-ವನ್ನು
ಬನ-ವಸೆ
ಬನ-ವಸೆ-ಕಾರರ
ಬನ-ವಸೆ-ಗ-ಳನ್ನು
ಬನ-ವಾಸಿ
ಬನ-ವಾಸಿ-ಪಟ್ಟ-ಣ-ದಲ್ಲಿ
ಬನ-ವಾಸಿಯ
ಬನ-ವಾಸಿ-ಯಲ್ಲಿ
ಬನ-ವಾಸಿ-ಯಿಂದ
ಬನ-ವಾಸೆ
ಬನ್ದಡೆ
ಬನ್ನ
ಬನ್ನಂಗಾಡಿ
ಬನ್ನ-ಹಳ್ಳಿ
ಬನ್ನ-ಹಳ್ಳಿಯ
ಬನ್ನಿ
ಬನ್ನಿ-ಯೂರ
ಬನ್ನೂರು
ಬಪ್ಪ
ಬಪ್ಪಡೆ
ಬಪ್ಪ-ದೇವಿ-ಯ-ರನ್ನು
ಬಪ್ಪೆಯಾಂಡನ
ಬಪ್ಪೆಯಾಂಡ-ನನ್ನು
ಬಪ್ಪೆ-ಯಾಂಡರ
ಬಪ್ಪೆ-ಯಾಂಡರ-ನಿಗೆ
ಬಪ್ಪೆ-ಯಾಂಡರಿಗೆ
ಬಬಳ್ಳ
ಬಬಿ-ನಾಡಾಳ್ವರು
ಬಬೆಯ-ನಾಡಾಂಕಿ-ಯಾದೀ
ಬಬ್ಬ
ಬಬ್ಬನು
ಬಬ್ಬ-ಯ-ನಾಯ-ಕನ
ಬಬ್ಬೀಶ್ವರ
ಬಬ್ಬೆಯ
ಬಬ್ಬೆಯ-ನಾಯ-ಕನ
ಬಬ್ಬೆಯ-ನಾಯ-ಕ-ನಿಗೆ
ಬಬ್ಬೆಯ-ನಾಯ-ಕನು
ಬಬ್ಬೆಯ-ನಾಯ-ಕ-ನೆಂಬ
ಬಭೈರಯ-ನಾಯಕ
ಬಮೋ-ಜನ
ಬಮ್ಮ
ಬಮ್ಮ-ಗವುಂಡನ
ಬಮ್ಮ-ಗ-ವುಡನ
ಬಮ್ಮಚ
ಬಮ್ಮ-ಚ-ನಧಿಕ-ಬಳಂ
ಬಮ್ಮಣ
ಬಮ್ಮಣ್ಣ
ಬಮ್ಮಣ್ಣ-ನ-ವೆಂಬು-ವ-ವನು
ಬಮ್ಮಣ್ಣನು
ಬಮ್ಮ-ನ-ಹಳ್ಳಿಯ
ಬಮ್ಮನು
ಬಮ್ಮ-ನೆಂಬ
ಬಮ್ಮ-ನೆಂಬು-ವ-ವನು
ಬಮ್ಮಲ
ಬಮ್ಮ-ಲ-ದೇವಿ
ಬಮ್ಮ-ಲ-ದೇವಿಯ
ಬಮ್ಮ-ಲ-ದೇವಿ-ಯನ್ನು
ಬಮ್ಮ-ಲ-ದೇವಿ-ಯರು
ಬಮ್ಮ-ಲ-ದೇವಿಯು
ಬಮ್ಮಲೆ
ಬಮ್ಮ-ಲೆಗೆ
ಬಮ್ಮ-ಲೆಯ
ಬಮ್ಮವ್ವೆ
ಬಮ್ಮವ್ವೆಗೆ
ಬಮ್ಮಿ-ಶೆಟ್ಟಿ
ಬಮ್ಮಿ-ಸೆಟ್ಟಿಯ
ಬಮ್ಮೋ-ಜಂಗೆ
ಬಮ್ಮೋ-ಜನುಂ
ಬಮ್ಮೋ-ಜನೇ
ಬಯಚಕ್ಕನೂ
ಬಯಲ
ಬಯ-ಲನ್ನು
ಬಯಲ-ಮಾರ್ತಾಂಡ
ಬಯಲಲಿ
ಬಯ-ಲಲ್ಲಿ
ಬಯಲ-ಹುಲಿ
ಬಯ-ಲಿನ
ಬಯ-ಲಿ-ನಲ್ಲಿ
ಬಯ-ಲಿನ-ವರೆಗೂ
ಬಯಲು
ಬಯಲು-ಗ-ಳನ್ನು
ಬಯಲು-ಗಳು
ಬಯಲುಮಂ
ಬಯಲ್ನಾಡ
ಬಯಲ್ನಾಡನಂ
ಬಯಲ್ನಾಡು
ಬಯಸುತ್ತಾರೆ
ಬಯ-ಸುವ
ಬಯಿಚಕ್ಕ
ಬಯಿಚಣ್ಣ
ಬಯಿಚಪ್ಪ
ಬಯಿಚೆಯ
ಬಯಿರ
ಬಯಿರ-ರಸ
ಬಯಿರ-ರಾಜ
ಬಯಿ-ರೆಯ
ಬಯಿ-ರೆಯ-ದಂಡ-ನಾಯ-ಕನ
ಬಯ್ಯಪ್ಪ-ನಾಯ-ಕರ
ಬರ-ಗಾಲ
ಬರಡು
ಬರಡು-ಭೂಮಿ-ಯಾ-ಗಿತ್ತು
ಬರ-ತೊಡಗಿ-ದರು
ಬರದ
ಬರದಂ
ಬರದ-ಕಲ್ಲ
ಬರದತ
ಬರದ-ಬರೆದ
ಬರದಾತ
ಬರ-ದಿದ್ದಲ್ಲಿ
ಬರದಿಹ
ಬರದು
ಬರದು-ದಕೆ
ಬರದೇ
ಬರ-ಬೇ-ಕಾದರೆ
ಬರ-ಬೇಕೆಂದು
ಬರಬೇಕೆಂದೂ
ಬರಮಣ್ಣ
ಬರ-ಲಾಗಿದೆ
ಬರ-ಲಾ-ಗುತ್ತಿತ್ತೆಂದು
ಬರ-ಲಾ-ಯಿತು
ಬರ-ಲಿಲ್ಲ
ಬರ-ಲಿಲ್ಲ-ವೆಂದು
ಬರಲು
ಬರ-ವಣಿಗೆ
ಬರ-ವಣಿಗೆಗೆ
ಬರ-ವಣಿಗೆಯ
ಬರ-ವಣಿಗೆ-ಯನ್ನು
ಬರ-ವಣಿಗೆ-ಯಲ್ಲಿ
ಬರಸಿ
ಬರಹ
ಬರಹಕ್ಕೆ
ಬರಹ-ಗ-ಳನ್ನು
ಬರಹ-ಗ-ಳಿಂದ
ಬರಹ-ಗ-ಳಿಗೆ
ಬರಹ-ಗಳು
ಬರಹ-ಗಾರ-ನಾಗಿದ್ದನು
ಬರಹ-ಗಾರ-ನಾಗಿದ್ದ-ನೆಂದು
ಬರಹ-ಗಾರ-ನಾದ
ಬರಹ-ಗಾರರ
ಬರಹ-ಗಾರ-ರನ್ನು
ಬರಹ-ಗಾರ-ರಾದ
ಬರಹ-ಗಾರರು
ಬರಹ-ಗಾರರೂ
ಬರಹ-ಗಾರ-ರೆಂದು
ಬರಹದ
ಬರಹ-ದಿಂದ
ಬರಹ-ವನ್ನು
ಬರಹ-ವನ್ನೇ
ಬರಹ-ವಿದೆ
ಬರಹ-ವಿದೆಈ
ಬರಹ-ವಿದ್ದು
ಬರಹ-ವಿ-ರುವ
ಬರಹ-ವಿಲ್ಲದ
ಬರಹ-ವಿಲ್ಲದೆ
ಬರಹ-ವುಳ್ಳ
ಬರಿಗೈಲಿ
ಬರೀದ-ಸಪ್ತಾಂಗ-ಹರಣ
ಬರುತಿದ್ದರು
ಬರು-ತಿದ್ದು
ಬರುತ್ತದೆ
ಬರುತ್ತ-ದೆಂದು
ಬರುತ್ತದೆೆ
ಬರುತ್ತವೆ
ಬರುತ್ತವೆಂದು
ಬರುತ್ತಾನೆ
ಬರುತ್ತಾರೆ
ಬರುತ್ತಿತ್ತು
ಬರುತ್ತಿತ್ತೆಂದು
ಬರುತ್ತಿದ್ದ
ಬರುತ್ತಿದ್ದ-ನೆಂಬುದು
ಬರುತ್ತಿದ್ದರು
ಬರುತ್ತಿದ್ದ-ರೆಂದು
ಬರುತ್ತಿದ್ದ-ರೆಂದೂ
ಬರುತ್ತಿದ್ದ-ರೆಂಬುದು
ಬರುತ್ತಿದ್ದಳು
ಬರುತ್ತಿದ್ದವು
ಬರುತ್ತಿದ್ದ-ವೆಂದು
ಬರುತ್ತಿದ್ದಾರೆ
ಬರುತ್ತಿದ್ದು-ದ-ರಿಂದ
ಬರುತ್ತಿದ್ದುದು
ಬರುತ್ತಿರ-ಲಿಲ್ಲ
ಬರುತ್ತಿ-ರುವ
ಬರುತ್ತಿ-ರುವಾಗ
ಬರುತ್ತಿಲ್ಲ
ಬರುತ್ತೀನಿ
ಬರುತ್ತೆ
ಬರುತ್ತೇ-ನೆಂದು
ಬರುವ
ಬರು-ವಂತೆ
ಬರು-ವಂತೆಯೂ
ಬರುವ-ವ-ನನ್ನು
ಬರುವಾಗ
ಬರು-ವು-ದಕ್ಕೆ
ಬರುವು-ದ-ರಿಂದ
ಬರು-ವು-ದಾಗಿಯೂ
ಬರು-ವು-ದಿಲ್ಲ
ಬರುವುದೇ
ಬರೂಲ್
ಬರೆದ
ಬರೆದಂ
ಬರೆದ-ನೆಂದು
ಬರೆದ-ನೆಂದೂ
ಬರೆದರು
ಬರೆದ-ವನು
ಬರೆದ-ವರು
ಬರೆ-ದಿದೆ
ಬರೆ-ದಿದ್ದ
ಬರೆ-ದಿದ್ದಕ್ಕಾಗಿ
ಬರೆ-ದಿದ್ದಾನೆ
ಬರೆದಿದ್ದಾ-ನೆಂದು
ಬರೆ-ದಿದ್ದಾನೆಂಬು-ದಕ್ಕೆ
ಬರೆ-ದಿದ್ದಾರೆ
ಬರೆ-ದಿದ್ದು
ಬರೆದಿರ-ಬ-ಹುದು
ಬರೆದಿರ-ಬಹು-ದೆಂದೂ
ಬರೆ-ದಿರು-ತಾನೋ
ಬರೆ-ದಿರುತ್ತಾನೆ
ಬರೆ-ದಿರುತ್ತಾ-ನೆಂದು
ಬರೆ-ದಿರುತ್ತಾ-ನೆಂದೂ
ಬರೆ-ದಿರುತ್ತಾರೆ
ಬರೆ-ದಿರುವ
ಬರೆ-ದಿ-ರು-ವಂತೆ
ಬರೆ-ದಿರು-ವವ
ಬರೆ-ದಿರು-ವ-ವ-ನನ್ನು
ಬರೆ-ದಿ-ರು-ವು-ದನ್ನು
ಬರೆ-ದಿ-ರುವುದು
ಬರೆದಿ-ವ-ವನೂ
ಬರೆದು
ಬರೆದು-ಕೊಂಡು
ಬರೆದು-ಕೊಡುತ್ತಿದ್ದರು
ಬರೆದು-ದಕ್ಕೆ
ಬರೆದುದು
ಬರೆಯದೇ
ಬರೆ-ಯಪ್ಪ
ಬರೆಯ-ಬಹು-ದೆಂದು
ಬರೆಯ-ಬಹುದೇ
ಬರೆಯ-ಬೇಕಾಗಿತ್ತೆ
ಬರೆಯ-ಲಾಗಿದೆ
ಬರೆ-ಯಲು
ಬರೆಯಿಸಿ
ಬರೆಯಿಸಿ-ಕೊಂಡಿದ್ದಾನೆ
ಬರೆಯಿಸಿ-ದ-ವನು
ಬರೆಯಿ-ಸಿದ್ದು
ಬರೆಯು
ಬರೆ-ಯುತ್ತಾ
ಬರೆಯುತ್ತಾನೆ
ಬರೆಯುತ್ತಿದ್ದರು
ಬರೆಯುತ್ತಿದ್ದರೆ
ಬರೆಯುತ್ತಿದ್ದು
ಬರೆಯುತ್ತಿದ್ದೇನೆ
ಬರೆಯುವ
ಬರೆಯು-ವ-ವ-ರಿಗೆ
ಬರೆಯು-ವ-ವರೂ
ಬರೆಯು-ವಷ್ಟು
ಬರೆಯು-ವಾಗ
ಬರೆಯು-ವುದು
ಬರೆಸಿ
ಬರೆಸಿ-ಕೊಂಡಿರ-ಬ-ಹುದು
ಬರೆಸಿ-ಕೊಂಡಿರು-ವಂತೆ
ಬರೆಸಿ-ಕೊಟ್ಟು
ಬರೆ-ಸಿದ
ಬರೆಸಿ-ದ-ನೆಂದು
ಬರೆಸಿ-ದರು
ಬರ್ತೀನಿ
ಬರ್ಮಯ್ಯ
ಬರ್ಮಯ್ಯನ
ಬರ್ಮಯ್ಯ-ನನ್ನು
ಬರ್ಮಯ್ಯ-ನಾಯ-ಕನ
ಬರ್ಮ್ಮಯ್ಯ
ಬರ್ಮ್ಮಯ್ಯನ
ಬರ್ಮ್ಮಯ್ಯನು
ಬಲ
ಬಲಂ
ಬಲಂಬ-ರಿಯ
ಬಲಂಬು-ತೀರ್ಥ-ದಲ್ಲಿ
ಬಲಕ್ಕಾ-ದಂತೆ
ಬಲ-ಗಯ್ಯ
ಬಲಗೈ
ಬಲ-ಗೈ-ಬಲ-ಭಾಗ
ಬಲ-ಗೈಯ
ಬಲ-ಗೈ-ಯ-ಸೇನಾ-ಧಿ-ಪತಿ
ಬಲ-ಗೈಯ್ಯ
ಬಲ-ಗೈಲಿ
ಬಲಙ್ಗಳ-ನಟ್ಟಿ-ಮುಟ್ಟಿ
ಬಲ-ತಾಯಿ
ಬಲ-ತಾಯಿ-ಯಾದ
ಬಲದ
ಬಲದ-ಕಯ್ಯ
ಬಲದಿಂ
ಬಲ-ದೇವಣ್ಣ
ಬಲ-ದೇವನು
ಬಲ-ಪಡಿ-ಸುವ
ಬಲ-ಭಾ-ಗಕ್ಕೆ
ಬಲ-ಭಾಗದ
ಬಲ-ಭಾಗ-ದಲ್ಲಿ-ರುವ
ಬಲ-ಮುರಿ
ಬಲ-ಮು-ರಿಯ
ಬಲ-ರನ್ನು
ಬಲ-ರಾಮ-ಕೃಷ್ಣ-ರಂತೆ
ಬಲ-ವಂಕ
ಬಲ-ವಂಕ-ದಲ್ಲಿ
ಬಲ-ವಂಕಪ್ಪ
ಬಲ-ವಂತ-ವಾಗಿ
ಬಲ-ವನ್ನು
ಬಲ-ವಾಗಿ
ಬಲ-ಸ-ಮುದ್ರ
ಬಲಾತ್ಕರ
ಬಲಾತ್ಕಾರ
ಬಲಿ
ಬಲಿ-ಗೆಯ್ದ-ರಂತೆ
ಬಲಿ-ದಾನ
ಬಲಿ-ದಾನಕ್ಕೆ
ಬಲಿ-ಪೀಠ
ಬಲಿ-ಯ-ಕೆರೆ
ಬಲಿ-ಯ-ಕೆರೆ-ಯನ್ನು
ಬಲಿ-ಯ-ಕೆರೆ-ಯಲ್ಲಿ
ಬಲಿ-ಶೆಲ್ವರ
ಬಲಿಷ್ಠ-ರಾದ
ಬಲೀಂದ್ರ
ಬಲು
ಬಲು-ಫೌಜು
ಬಲು-ಮನುಷ
ಬಲು-ಮನುಷ್ಯ
ಬಲು-ಹಿಂದ
ಬಲೆಯ-ನಾಯ-ಕರ
ಬಲೋ-ಜನು
ಬಲ್ಪಿ-ನಿಂದ-ದಟಿನಿಂ
ಬಲ್ಪುಳ್ಳುದು
ಬಲ್ಲ
ಬಲ್ಲಂ
ಬಲ್ಲ-ಗೌಡನ
ಬಲ್ಲಪ
ಬಲ್ಲಪಂ
ಬಲ್ಲ-ಪನು
ಬಲ್ಲ-ಪನೇ
ಬಲ್ಲಪ್ಪ
ಬಲ್ಲಪ್ಪ-ದಂಡ-ನಾಯ-ಕನ
ಬಲ್ಲಪ್ಪನು
ಬಲ್ಲಪ್ಪ-ಬಿಲ್ಲಪ್ಪ
ಬಲ್ಲ-ಯನ
ಬಲ್ಲ-ಯ-ನಾಯ-ಕನು
ಬಲ್ಲಯ್ಯ
ಬಲ್ಲಯ್ಯನ
ಬಲ್ಲಯ್ಯ-ನಯ್ಯನೀ
ಬಲ್ಲಯ್ಯನು
ಬಲ್ಲಯ್ಯನೇ
ಬಲ್ಲ-ವರು
ಬಲ್ಲಹ
ಬಲ್ಲಹಂ
ಬಲ್ಲ-ಹನ
ಬಲ್ಲ-ಹನು
ಬಲ್ಲಾನು
ಬಲ್ಲಾಳ
ಬಲ್ಲಾಳಈ
ಬಲ್ಲಾಳ-ಚತುರ್ವೇದಿ
ಬಲ್ಲಾಳ-ಜೀಯಂಗೆ
ಬಲ್ಲಾಳ-ಜೀಯನ
ಬಲ್ಲಾಳ-ಜೀಯ-ನಿಗೆ
ಬಲ್ಲಾಳ-ದಾಸರ
ಬಲ್ಲಾಳ-ದೇವ
ಬಲ್ಲಾಳ-ದೇವಂ
ಬಲ್ಲಾಳ-ದೇವನ
ಬಲ್ಲಾಳ-ದೇವ-ನತ್ಯಂತ-ವಾಗಿ
ಬಲ್ಲಾಳ-ದೇವ-ನಿಗೆ
ಬಲ್ಲಾಳ-ದೇವನು
ಬಲ್ಲಾಳ-ದೇವ-ನೊಡನೆ
ಬಲ್ಲಾಳ-ದೇವರ
ಬಲ್ಲಾಳ-ದೇವ-ರಸ
ಬಲ್ಲಾಳ-ದೇವ-ರ-ಸನು
ಬಲ್ಲಾಳ-ದೇವ-ರ-ಸ-ನು-ಎರಡ-ನೆಯ
ಬಲ್ಲಾಳ-ದೇವ-ರ-ಸರ
ಬಲ್ಲಾಳ-ದೇವ-ರ-ಸರು
ಬಲ್ಲಾಳ-ದೇವರು
ಬಲ್ಲಾಳನ
ಬಲ್ಲಾಳ-ನ-ದಲ್ಲಿ
ಬಲ್ಲಾಳ-ನ-ಪುರ-ವಾದ
ಬಲ್ಲಾಳ-ನಲ್ಲಿ
ಬಲ್ಲಾಳ-ನಲ್ಲಿದ್ದರು
ಬಲ್ಲಾಳ-ನ-ವರೆ-ಗಿನ
ಬಲ್ಲಾಳ-ನಾಗುತ್ತಾನೆ
ಬಲ್ಲಾಳ-ನಿಂದ
ಬಲ್ಲಾಳ-ನಿಗೆ
ಬಲ್ಲಾಳನು
ಬಲ್ಲಾಳನೂ
ಬಲ್ಲಾಳ-ನೆಂದು
ಬಲ್ಲಾಳನೇ
ಬಲ್ಲಾಳ-ಪುರ
ಬಲ್ಲಾಳ-ಪುರದ
ಬಲ್ಲಾಳ-ಪುರ-ದಲ್ಲಿ
ಬಲ್ಲಾಳ-ಪುರಸ್ಥಳ
ಬಲ್ಲಾಳ-ಭಟ್ಟ-ರಿಗೆ
ಬಲ್ಲಾಳ-ಭೂಪಾಳಂ
ಬಲ್ಲಾಳ-ಮಹೀ-ಕಾಂತನ
ಬಲ್ಲಾಳ-ಮಹೀ-ಪಾಲ
ಬಲ್ಲಾಳ-ಮಹೀ-ಪಾಲಯಂ
ಬಲ್ಲಾಳ-ರನ್ನೂ
ಬಲ್ಲಾಳ-ರಾಯನ
ಬಲ್ಲಾಳ-ರಾಯ-ನಿಗೆ
ಬಲ್ಲಾಳ-ರಾಯ್ಯ
ಬಲ್ಲಾಳು
ಬಲ್ಲಾಳು-ಗಳ
ಬಲ್ಲಾಳು-ಸೆಟ್ಟಿ
ಬಲ್ಲಾಳು-ಸೆಟ್ಟಿಯು
ಬಲ್ಲಾಳೇಶ್ವರ
ಬಲ್ಲಿ-ದ-ನಾಗಿ-ರುತ್ತಿದ್ದನು
ಬಲ್ಲೆ-ಕೆರೆ
ಬಲ್ಲೆಯ
ಬಲ್ಲೆಯ-ನಾಯಕ
ಬಲ್ಲೆಯ-ನಾಯ-ಕನ
ಬಲ್ಲೆಯ-ನಾಯ-ಕನು
ಬಲ್ಲೆಯ-ಬಲ್ಲಪ್ಪ
ಬಲ್ಲೇ-ಕೆರೆ
ಬಲ್ಲೇ-ಗೌಡ
ಬಲ್ಲೇನ-ಪಲ್ಲಿ
ಬಲ್ಲೇನ-ಹಳ್ಳಿ
ಬಲ್ಲೇನ-ಹಳ್ಳಿಯ
ಬಲ್ಲೇನ-ಹಳ್ಳಿ-ಯಾಗಿದೆ
ಬಲ್ಲೇಯ-ನಾಯ-ಕನ
ಬಳಕೆ
ಬಳ-ಕೆಗೆ
ಬಳ-ಕೆ-ಗೊಂಡಿವೆ
ಬಳ-ಕೆಯ
ಬಳ-ಕೆ-ಯಲ್ಲಿ
ಬಳ-ಕೆ-ಯಲ್ಲಿತ್ತು
ಬಳ-ಕೆ-ಯಲ್ಲಿ-ರ-ಲಿಲ್ಲ
ಬಳ-ಕೆ-ಯಾಗಿದೆ
ಬಳ-ಕೆ-ಯಾಗಿ-ರುವ
ಬಳ-ಕೆ-ಯಾಗಿ-ರು-ವುದ-ರಿಂದ
ಬಳ-ಕೆ-ಯಾಗಿ-ರು-ವುದು
ಬಳ-ಕೆ-ಯಾಗಿಲ್ಲ
ಬಳ-ಕೆ-ಯಾಗಿವೆ
ಬಳ-ಕೆ-ಯಾಗುವ
ಬಳ-ಕೆ-ಯಾದ
ಬಳ-ಗಟ್ಟ
ಬಳ-ಗಾ-ನೂರು
ಬಳ-ಗಾರ
ಬಳ-ಗಾರ-ಕುಲದ
ಬಳ-ಗುಂದಿಯ
ಬಳ-ಗುಳ
ಬಳ-ಗುಳಕ್ಕೆ
ಬಳ-ಗುಳದ
ಬಳ-ಗುಳ-ದಲ್ಲಿ
ಬಳ-ಗುಳ-ವನ್ನು
ಬಳ-ಗೊಳ
ಬಳ-ಗೊಳದ
ಬಳ-ಗೊಳ-ದಲ್ಲಿ
ಬಳ-ಗೊಳವು
ಬಳ-ಗೋಜ
ಬಳ-ಘಟ್ಟ-ಬಳಿಗ
ಬಳ-ಘಟ್ಟ-ವಾಗಿದೆ
ಬಳ-ಪದ-ಕಲ್ಲು-ಮಂಟಿ
ಬಳ-ಮಡು-ನಾಡಿ-ನಲ್ಲಿತ್ತೆಂದು
ಬಳ-ಮಡು-ನಾಡು
ಬಳ-ಮುರ್ಬ್ಬಿಗಗುರ್ವ್ವು
ಬಳ-ಯುತರ
ಬಳ-ಲಾಸುತ್ತಿತ್ತು
ಬಳಲ್ವ
ಬಳವ
ಬಳ-ಸ-ಲಾಗಿದೆ
ಬಳ-ಸ-ಲಾಗಿ-ದೆಯೇ
ಬಳ-ಸ-ಲಾಗುತ್ತಿದೆ
ಬಳಸಿ
ಬಳ-ಸಿ-ಕೊಂಡಿದ್ದಾರೆ
ಬಳ-ಸಿ-ಕೊಳ್ಳ-ಲಾಗಿದೆ
ಬಳ-ಸಿ-ಕೊಳ್ಳಲು
ಬಳ-ಸಿ-ಕೊಳ್ಳುತ್ತಿದ್ದ-ರೆಂದು
ಬಳ-ಸಿದೆ
ಬಳ-ಸಿದ್ದಾನೆ
ಬಳ-ಸಿದ್ದಾರೆ
ಬಳ-ಸಿದ್ದು
ಬಳ-ಸಿ-ರುವ
ಬಳ-ಸಿ-ರುವು-ದ-ರಿಂದ
ಬಳ-ಸಿಲ್ಲ
ಬಳ-ಸಿಲ್ಲ-ವೆಂಬು-ದನ್ನು
ಬಳ-ಸುತ್ತಾರೆ
ಬಳ-ಸುತ್ತಿದ್ದ
ಬಳ-ಸುತ್ತಿದ್ದು
ಬಳ-ಸುವ
ಬಳ-ಸುವ-ವ-ನನ್ನು
ಬಳಾತ್ಕರ
ಬಳಿ
ಬಳಿಕ
ಬಳಿಗ
ಬಳಿ-ಗ-ಗಟ್ಟದ
ಬಳಿ-ಗ-ಗಟ್ಟ-ವಾಗಿದೆ
ಬಳಿಯ
ಬಳಿ-ಯಣ
ಬಳಿ-ಯಲ್ಲೇ
ಬಳಿಯಿಂ
ಬಳಿಯೂ
ಬಳಿಯೆ
ಬಳಿಯೇ
ಬಳಿ-ಯೊಳು
ಬಳಿ-ವಳಿ
ಬಳಿ-ವಳಿಯ
ಬಳಿ-ವಳಿ-ಯಾಗಿ
ಬಳಿ-ಸಹಿತ
ಬಳು-ವಳಿ
ಬಳು-ವಳಿ-ಯಾ-ಗಲ್ಲ
ಬಳು-ವಳಿ-ಯಾಗಿ
ಬಳೆ-ಗಳ
ಬಳೆ-ಗಾರ
ಬಳೆಯಲಾ
ಬಳ್ಳ
ಬಳ್ಳ-ಅಕ್ಕಿ
ಬಳ್ಳಗ
ಬಳ್ಳ-ಗ-ಳಲ್ಲಿ
ಬಳ್ಳದ
ಬಳ್ಳ-ದಲ್ಲಿ
ಬಳ್ಳ-ಮೂರಕ್ಕಂ
ಬಳ್ಳ-ರೆ-ವಲ್ಲುರು
ಬಳ್ಳ-ವಳ್ಳಿ-ಯಲ್ಲಿ
ಬಳ್ಳ-ವೆರಡಕ್ಕಂ
ಬಳ್ಳ-ವೆ-ರಡು
ಬಳ್ಳ-ವೊಂದು
ಬಳ್ಳ-ಸ-ಲಿಗೆ
ಬಳ್ಳಾರಿ
ಬಳ್ಳಿ-ಗ-ಳನ್ನೂ
ಬಳ್ಳಿ-ಗಾವೆ
ಬಳ್ಳಿ-ಗಾವೆಯ
ಬಳ್ಳಿ-ಗಾವೆ-ಯನ್ನು
ಬಳ್ಳಿಗ್ರಾಮೆ-ಬಳ್ಳಿ-ಗಾ-ಮೆಯ
ಬಳ್ಳಿ-ಯ-ಕೆರೆ
ಬಳ್ಳೀ
ಬಳ್ಳೆಗೊಳ
ಬಳ್ಳೆಗೊಳಕ್ಕೆ
ಬಳ್ಳೆಗೊ-ಳದ
ಬಳ್ಳೆಯ-ಕೆರೆ
ಬಳ್ಳೆಯ-ಕೆರೆಯ
ಬಳ್ಳೇ-ಕೆರೆ
ಬವರ-ದಲ್ಲಿ
ಬವೆಯ
ಬಸದಿ
ಬಸದಿ-ಗಳ
ಬಸದಿ-ಗ-ಳನ್ನು
ಬಸದಿ-ಗಳಿಗೂ
ಬಸದಿ-ಗ-ಳಿಗೆ
ಬಸದಿ-ಗಳಿದ್ದ
ಬಸದಿ-ಗಳಿವೆ
ಬಸದಿ-ಗಳು
ಬಸದಿ-ಗಳೂ
ಬಸದಿ-ಗ-ಳೆಂದು
ಬಸದಿ-ಗಳೇ
ಬಸದಿಗೂ
ಬಸದಿಗೆ
ಬಸದಿ-ಗೆ-ಶಾ-ಸನ-ಬ-ಸದಿ
ಬಸದಿಯ
ಬಸದಿ-ಯನ್ನು
ಬಸದಿ-ಯನ್ನು-ಎ-ರಡು-ಕಟ್ಟೆ
ಬಸದಿ-ಯಲ್ಲಿ
ಬಸದಿ-ಯಲ್ಲಿದೆ
ಬಸದಿ-ಯಲ್ಲಿದ್ದ
ಬಸದಿ-ಯಲ್ಲಿ-ರುವ
ಬಸದಿ-ಯಲ್ಲಿಲ್ಲ
ಬಸದಿ-ಯಾಗಿತ್ತೆಂದು
ಬಸದಿ-ಯಾಗಿದ್ದು-ದ-ರಿಂದ
ಬಸದಿ-ಯಾಗಿ-ರ-ಬ-ಹುದು
ಬಸದಿಯು
ಬಸದಿಯೂ
ಬಸದಿಯೇ
ಬಸದಿ-ಯೊಳಗೆ
ಬಸದಿ-ಹಳ್ಳಿಯ
ಬಸರಾ-ಳನ್ನು
ಬಸರಾ-ಳಿಗೆ
ಬಸರಾ-ಳಿನ
ಬಸರಾ-ಳಿ-ನಲ್ಲಿ
ಬಸರಾ-ಳಿನಲ್ಲಿಯೂ
ಬಸರಾ-ಳಿನಲ್ಲಿ-ರುವ
ಬಸರಾಳು
ಬಸರಿ-ಕಟ್ಟೆಯ
ಬಸರಿ-ವಾಳದ
ಬಸರು-ವಾಣು
ಬಸವ
ಬಸವಂತ
ಬಸವಂತ-ಪಟ್ಟ-ಣ-ವನ್ನು
ಬಸವಂತ-ಪಟ್ಟ-ಣ-ವಾಗಿ-ರ-ಬ-ಹುದು
ಬಸವಂತ-ಬಸವ-ಪಟ್ಟಣ
ಬಸವ-ಕವಿಯ
ಬಸವ-ಗ-ವುಡನ
ಬಸವ-ಗಾವುಂಡ
ಬಸ-ವಟ್ಟಿಗೆ
ಬಸವ-ಣಯ್ಯ
ಬಸವಣ್ಣ
ಬಸವಣ್ಣನ
ಬಸವಣ್ಣ-ನ-ವರ
ಬಸವಣ್ಣ-ನ-ವರು
ಬಸವನ
ಬಸವ-ನ-ಕೋಟೆ
ಬಸವ-ನ-ಗುಡಿ
ಬಸವ-ನ-ಪುರ
ಬಸವ-ನ-ಪುರ-ದಲ್ಲಿ
ಬಸವ-ನ-ಬೆಟ್ಟ
ಬಸವ-ನ-ಬೆಟ್ಟದ
ಬಸವ-ನ-ಹಳ್ಳಿ
ಬಸವನು
ಬಸವನೂ
ಬಸವ-ನೆಂಬು-ದನ್ನು
ಬಸವ-ನೆಂಬು-ವ-ವನು
ಬಸವನ್ತ
ಬಸವ-ಪಟ್ಟಣ
ಬಸವ-ಪಟ್ಟ-ಣದ
ಬಸವ-ಪೂರ್ವದ
ಬಸವ-ಪೂರ್ವ-ಯುಗ
ಬಸ-ವಪ್ಪ
ಬಸವಪ್ಪ-ನಾಯ-ಕನು
ಬಸವ-ಭಕ್ತನ
ಬಸವ-ಭಕ್ತನು
ಬಸವ-ಭಕ್ತನೂ
ಬಸವ-ಮಾತ್ಯ
ಬಸವ-ಮಾತ್ಯನ
ಬಸವ-ಮಾತ್ಯನು
ಬಸವ-ಮಾತ್ಯ-ಬಸವ-ರಸ
ಬಸವ-ಯುಗ
ಬಸವಯ್ಯ
ಬಸವ-ರಸ
ಬಸವ-ರ-ಸನ
ಬಸವ-ರ-ಸನೂ
ಬಸವ-ರಸಯ್ಯ
ಬಸವ-ರಸರ
ಬಸವ-ರಾಜ
ಬಸವ-ರಾಜ-ದೇವರ
ಬಸವ-ರಾಜನ
ಬಸವ-ರಾಜಯ್ಯ-ದೇವ
ಬಸವ-ರಾಜು
ಬಸವ-ರಾಜ್
ಬಸವ-ರಾಜ್ಡಾ-ಡಿಟಿಬಿ
ಬಸವ-ಲಿಂಗ-ದೇವನು
ಬಸವಾ-ಚಾರಿಯು
ಬಸವಾ-ಪಟ್ಟ-ಣದ
ಬಸವಾ-ಪಟ್ಟ-ಣ-ವನ್ನು
ಬಸವಾ-ಪುರ
ಬಸವಿ-ಯಕ್ಕ
ಬಸವಿ-ಯಕ್ಕನು
ಬಸವೀ-ದೇವ
ಬಸವೀ-ದೇವನ
ಬಸವೇಶ್ವರ
ಬಸವೇಶ್ವರನ
ಬಸವೇಶ್ವರನು
ಬಸವೇಶ್ವರರ
ಬಸವೇಶ್ವರಿನಿ
ಬಸವೋತ್ತರ
ಬಸವೋತ್ತರ-ಯುಗ
ಬಸಿ-ಕಟ್ಟೆ
ಬಸಿ-ರಲ್ಲಿ
ಬಸುರ-ಬಂದ
ಬಸುರಲಿ
ಬಸುರ-ವಾಣ
ಬಸುರಿ-ನಲ್ಲಿ
ಬಸುರಿ-ವಾಣ-ದಲ್ಲಿ
ಬಸುರಿ-ವಾಳ
ಬಸುರಿ-ವಾಳದ
ಬಸುರಿ-ವಾಳ-ದಲ್ಲಿ
ಬಸುರಿ-ವಾಳ-ದೊಳು
ಬಸುರಿ-ವಾಳು
ಬಸು-ರುವಾಣು
ಬಸು-ರುವಾಣುಸ್ಥಳದ
ಬಸು-ರುವಾಳು
ಬಸ್ತಿ
ಬಸ್ತಿ-ಗದ್ದೆ
ಬಸ್ತಿ-ಪುರ
ಬಸ್ತಿಯ
ಬಸ್ತಿ-ಯನ್ನು
ಬಸ್ತಿ-ಯಲ್ಲಿ
ಬಸ್ತಿ-ಯಲ್ಲಿ-ನಾಡ-ಮಾಣಿ-ಕ-ದೊಡ-ಲೂರು
ಬಸ್ತಿ-ಯ-ಹಳ್ಳಿ
ಬಸ್ತಿ-ಹಳ್ಳಿ
ಬಸ್ತಿ-ಹಳ್ಳಿಯ
ಬಸ್ತಿ-ಹಳ್ಳಿ-ಯಲ್ಲಿ-ರುವ
ಬಸ್ತೀ-ಪುರ
ಬಸ್ತೀ-ಪುರದ
ಬಹ
ಬಹತ್ತರ
ಬಹ-ದೂ-ರನು
ಬಹದ್ದೂರ್
ಬಹ-ಮನಿ
ಬಹರಿಯ
ಬಹಲ್ಲಿ
ಬಹಳ
ಬಹ-ಳ-ವಾಗಿ
ಬಹ-ಳ-ವಾ-ಗಿತ್ತು
ಬಹಶಃ
ಬಹಾ-ದೂರ್
ಬಹಾದ್ದೂ-ರನು
ಬಹಾದ್ದೂರು
ಬಹಾದ್ದೂರ್
ಬಹಿತ್ರ
ಬಹಿತ್ರದ
ಬಹಿತ್ರ-ದಂತೆ
ಬಹಿತ್ರರು
ಬಹಿ-ರಂಗ-ವಾಗಿ
ಬಹಿರಿ
ಬಹಿಷ್ಕರಿ-ಸ-ಲಾಗುತ್ತಿತ್ತು
ಬಹು
ಬಹು-ಕಾಲ
ಬಹು-ಕುಲಾನ್ವಯ
ಬಹು-ತರ
ಬಹು-ತೆಕ
ಬಹು-ತೇಕ
ಬಹು-ದಾದ
ಬಹುದು
ಬಹು-ದೂರದ
ಬಹು-ದೆಂದು
ಬಹು-ದೆಂಬ-ದರ
ಬಹು-ದೊಡ್ಡ
ಬಹುಧಾ
ಬಹು-ಪಾಲು
ಬಹು-ಪುತೃತ್ರಿಯರಂ
ಬಹು-ಭಾಗ
ಬಹು-ಭಾಗ-ದಲ್ಲಿ
ಬಹು-ಮಟ್ಟಿಗೆ
ಬಹು-ಮನಿ
ಬಹು-ಮಾನ
ಬಹು-ಮಾನ-ಗಳೂ
ಬಹು-ಮಾನ-ವಾಗಿ
ಬಹು-ಮುಖ್ಯ
ಬಹು-ಮುಖ್ಯ-ವಾದ
ಬಹು-ರಾಜ್ಯ-ಕಾರ್ಯ್ಯಂ
ಬಹು-ರೇಖೆ-ಯಾಗಿ
ಬಹುಳ
ಬಹು-ವಾಗಿ
ಬಹುಶ
ಬಹುಶಃ
ಬಹು-ಸಂಖ್ಯಾ-ತ-ರಾಗಿದ್ದಾರೆ
ಬಹು-ಸಂಖ್ಯಾ-ತರು
ಬಹು-ಸಂಖ್ಯೆ
ಬಹು-ಸಂಖ್ಯೆಯ
ಬಹು-ಸಂಖ್ಯೆ-ಯಲ್ಲಿ
ಬಹು-ಸಂಖ್ಯೆ-ಯಲ್ಲಿದ್ದವು
ಬಹು-ಸಂಖ್ಯೆ-ಯಲ್ಲಿದ್ದು
ಬಹು-ಸಂತಾನಕ್ಷೇಮ-ಸರ್ವ್ವ
ಬಹೆನು
ಬಹೆವು
ಬಹ್ವೃಚ
ಬಾಂಧವ್ಯಕ್ಕೆ
ಬಾಂಧವ್ಯ-ವಿದ್ದಿತು
ಬಾಕಿ
ಬಾಗಣಬ್ಬೆ
ಬಾಗಣಬ್ಬೆಯ
ಬಾಗಣಬ್ಬೆಯು
ಬಾಗಳಿ
ಬಾಗ-ಸೆಟ್ಟಿ-ಹಳ್ಳಿ-ಗ-ಳನ್ನು
ಬಾಗಿ-ನಾಡೆಪ್ಪತ್ತು
ಬಾಗಿಲ
ಬಾಗಿ-ಲನ್ನು
ಬಾಗಿ-ಲಲ್ಲಿ
ಬಾಗಿಲಲ್ಲಿ-ರುವ
ಬಾಗಿಲಿಗೆ
ಬಾಗಿಲಿನ
ಬಾಗಿ-ಲಿ-ನಲ್ಲಿ
ಬಾಗಿಲಿ-ನಲ್ಲಿ-ರುವ
ಬಾಗಿಲು
ಬಾಗಿಲು-ವಾಡ
ಬಾಗಿಲು-ವಾಡ-ಗ-ಳನ್ನು
ಬಾಗಿಲು-ವಾಡದ
ಬಾಗಿಲು-ವಾಡ-ವನ್ನು
ಬಾಗಿವಾ-ಳನ್ನು
ಬಾಗಿ-ವಾಳು
ಬಾಗೆ
ಬಾಗೆ-ನಾಡು
ಬಾಗೆ-ನಾಡುನ್ನು
ಬಾಗೆ-ಸೆಟ್ಟಿ-ಹಳ್ಳಿ-ಗ-ಳನ್ನು
ಬಾಗೇ-ವಾಡಿ-ಯಲ್ಲಿ
ಬಾಚ-ಗವುಡ
ಬಾಚ-ಜೀಯಂಗೆ
ಬಾಚಣ್ಣನು
ಬಾಚನ-ಹಳ್ಳಿಯೇ
ಬಾಚನು-ದಾರ-ನುರ್ವಿ-ಯೊಳು
ಬಾಚ-ಪಟ್ಟ-ಣದ
ಬಾಚಪ್ಪ
ಬಾಚಪ್ಪನ
ಬಾಚಪ್ಪ-ನ-ಕೆರೆ
ಬಾಚಪ್ಪ-ನನ್ನು
ಬಾಚಪ್ಪನು
ಬಾಚಪ್ಪನೇ
ಬಾಚಹ-ಳಿಯ
ಬಾಚ-ಹಳ್ಳಿ
ಬಾಚ-ಹಳ್ಳಿಗೆ
ಬಾಚ-ಹಳ್ಳಿಯ
ಬಾಚ-ಹಳ್ಳಿ-ಯನ್ನು
ಬಾಚಿ-ಕಟ್ಟ
ಬಾಚಿಗ
ಬಾಚಿ-ಜೀಯ-ನಿಗೆ
ಬಾಚಿ-ಪಟ್ಟ-ಣ-ವನು
ಬಾಚಿ-ಪಟ್ಟ-ಣ-ವಾಗಿ-ರ-ಬ-ಹುದು
ಬಾಚಿ-ಯಪ್ಪ
ಬಾಚಿ-ಯಪ್ಪನ
ಬಾಚಿ-ಯಪ್ಪ-ನ-ವಿಗೆ
ಬಾಚಿ-ಯಪ್ಪ-ನಿಗೆ
ಬಾಚಿ-ಯಪ್ಪನು
ಬಾಚಿಯ-ಹಳ್ಳಿಯ
ಬಾಚಿ-ರಾಜನ
ಬಾಚಿ-ರಾಜನಂ
ಬಾಚಿ-ರಾಜನು
ಬಾಚಿ-ರಾಜ-ಪಟ್ಟ-ಣ-ವನ್ನು
ಬಾಚಿ-ರಾಜಾ-ಭಿ-ದಾನಃ
ಬಾಚಿ-ರಾಜಾಹ್ವಯ
ಬಾಚಿ-ಸೆಟ್ಟಿ-ಯರ
ಬಾಚಿಹ-ಳಿಯ
ಬಾಚಿ-ಹಳ್ಳಿಯ
ಬಾಚಿ-ಹಳ್ಳಿ-ಯನು
ಬಾಚಿ-ಹಳ್ಳಿ-ಯನ್ನು
ಬಾಚೆಯ
ಬಾಚೆಯಪ್ಪನು
ಬಾಚೆಯ-ಹಳ್ಳಿ
ಬಾಚೆಯ-ಹಳ್ಳಿಯ
ಬಾಚೆಯ-ಹಳ್ಳಿ-ಯನ್ನು
ಬಾಚೆಯ-ಹಳ್ಳಿ-ಯನ್ನೇ
ಬಾಚೆ-ಹಳ್ಳಿಯ
ಬಾಚೆ-ಹಳ್ಳಿಯು
ಬಾಜ-ದನ-ಮಲಕ
ಬಾಜ-ದನ-ಮಲುಕ
ಬಾಡಂಗಳ್ಗೆ-ರಗಿ-ದನ್ದೆ
ಬಾಡಸ-ರೊಕ್ಕಲು
ಬಾಡಿಗೆ-ಯನ್ನು
ಬಾಣ
ಬಾಣ-ಕುಲ-ಕ-ಲಾಕಲಃ
ಬಾಣ-ಕುಲದ
ಬಾಣದ
ಬಾಣ-ದಿಂದ
ಬಾಣಪ್ರ-ಯೋಗ
ಬಾಣರ
ಬಾಣ-ರ-ಮೇಲೆ
ಬಾಣ-ರ-ಸರ
ಬಾಣರು
ಬಾಣ-ವಂಶದ
ಬಾಣ-ವಂಶೋದ್ಭವ
ಬಾಣ-ವಂಶೋದ್ಭವ-ನಾದ
ಬಾಣ-ವನ್ನು
ಬಾಣ-ವಾಡಿ-ಯಲ್ಲಿ
ಬಾಣ-ಸಂದಾ-ಪುರ-ವನ್ನು
ಬಾಣಸಿ
ಬಾಣ-ಸಿಗ
ಬಾಣಾ-ವರ-ದಲ್ಲಿ
ಬಾತಾ-ದಂತೆ
ಬಾದಶಹ
ಬಾದಶಹ-ನನ್ನಾಗಿ
ಬಾದಶ-ಹರು
ಬಾದಶಹಾ
ಬಾದಶಾಹ
ಬಾದಾಮಿ
ಬಾದಾ-ಮಿಯ
ಬಾದಾ-ಮಿ-ಯನ್ನು
ಬಾದಾ-ಮಿಯು
ಬಾದ್ಶಾ
ಬಾಧೆ
ಬಾಧೆ-ಗ-ಳನ್ನು
ಬಾಧೆ-ಯನು
ಬಾಧೆ-ಯನ್ನು
ಬಾಧೆ-ಯನ್ನೂ
ಬಾಪ್ಪೆಂಬಿನಂ
ಬಾಬ
ಬಾಬ-ಚಾ-ಮುಂಡ-ರಾಯ
ಬಾಬ-ಚಾ-ಮುಂಡ-ರಾಯನ
ಬಾಬ-ಚಾವುಂಡ-ರಾಯ
ಬಾಬ-ಸಿಂಘೂ-ರಹರು
ಬಾಬಾ-ಡಿಯ
ಬಾಬು-ಸೆಟ್ಟಿಯು
ಬಾಬೂ
ಬಾಬೂ-ರಾಯನ
ಬಾಬೂ-ಸೆಟ್ಟಿ
ಬಾಬೂ-ಸೆಟ್ಟಿಯು
ಬಾಬೂ-ಸೆಟ್ಟಿಯೂ
ಬಾಬೆಯ
ಬಾಬೆಯ-ನಾಯಕ
ಬಾಮುಲ
ಬಾಯಲ
ಬಾಯಾನು-ಮತ-ದಿಂದ
ಬಾಯಿ
ಬಾಯಿ-ಕಾಲ-ಠಾವಿ-ನಲು
ಬಾಯಿ-ಕಾಲಿಂ
ಬಾಯಿ-ದೇವಿ
ಬಾಯುಳ್ಳ
ಬಾಯೊಳಕ್ಕೆ
ಬಾಯೊಳು
ಬಾಯ್ಕಲ್
ಬಾಯ್ಬೆಣ್ಣೆಗೆ
ಬಾರಂದರ
ಬಾರಂದೇಶ್ವರ
ಬಾರಕೂರು
ಬಾರಕೂರು-ಗ-ಳಿಂದ
ಬಾರಣಾಸಿ-ಯಮ-ವ-ನ-ಳಿದೋಂ
ಬಾರದು
ಬಾರದೆ
ಬಾರದೆ-ಯತಿ
ಬಾರಾ
ಬಾರಾ-ಗೋ-ಪಾಲ್
ಬಾರಾ-ಗೋ-ಪಾಲ್ರ-ವರು
ಬಾರಿ
ಬಾರಿ-ಕರು
ಬಾರಿಯೂ
ಬಾರಿ-ಯೊಗ್ಳಪ್ಪಿ
ಬಾಲ
ಬಾಲ-ಕ-ನಾಗಿ
ಬಾಲ-ಕ-ನಾಗಿದ್ದು-ದ-ರಿಂದ
ಬಾಲ-ಕರಿ-ಗಾಗಿ
ಬಾಲ-ಚಂದ್ರ
ಬಾಲ-ಚಂದ್ರ-ದೇವರ
ಬಾಲ-ಚಂದ್ರನ
ಬಾಲ-ಚಂದ್ರ-ನನ್ನು
ಬಾಲ-ಚಂದ್ರ-ಯತಿ
ಬಾಲ-ಚಂದ್ರರು
ಬಾಲ-ದೆರೆ
ಬಾಲದ್ಹಣ-ವನ್ನು
ಬಾಲ-ಪಣ
ಬಾಲ-ಪಣದ
ಬಾಲ-ಪಣ-ಬಾಲ-ವಣ
ಬಾಲ-ಪಣ-ವನೆತ್ತು-ವಲ್ಲಿ
ಬಾಲ-ವಣ
ಬಾಲ-ವಣ-ವನು
ಬಾಲ-ಶಿಕ್ಷೆ
ಬಾಲ-ಶಿಕ್ಷೆ-ಯನ್ನೂ
ಬಾಲ-ಸಿಕ್ಷೆ
ಬಾಲ-ಸಿಕ್ಷೆಯ
ಬಾಲಾತಪಂ
ಬಾಲೂರು
ಬಾಲ್ದಳಿ
ಬಾಲ್ದಳಿ-ಸೆಟ್ಟಿಯ
ಬಾಲ್ಯ-ದಲ್ಲಿ
ಬಾಲ್ಯವೆಲ್ಲ
ಬಾಳಗಂಚಿ
ಬಾಳ-ಗುಂಚಿ
ಬಾಳ-ಚಂದ್ರ
ಬಾಳ-ಚಂದ್ರ-ಕಂದರ್ಪನು
ಬಾಳ-ಚಂದ್ರ-ದೇವ
ಬಾಳ-ಚಂದ್ರ-ದೇವನು
ಬಾಳ-ಚಂದ್ರ-ದೇವರ
ಬಾಳ-ಚಂದ್ರ-ದೇವ-ರಿಗೆ
ಬಾಳ-ಚಂದ್ರ-ದೇವರು
ಬಾಳ-ಚಂದ್ರನು
ಬಾಳ-ಚಂದ್ರನೂ
ಬಾಳ-ಚಂದ್ರ-ಮುನೀಂದ್ರನು
ಬಾಳಚನ್ದ್ರ
ಬಾಳಚನ್ದ್ರ-ಮುನಿ-ಮುಖ್ಯ
ಬಾಳಿ
ಬಾಳಿ-ದ-ನೆಂದು
ಬಾಳಿ-ದ-ರೆಂದು
ಬಾಳು
ಬಾಳು-ಗುಂಚಿಯ
ಬಾಳೆ
ಬಾಳೆ-ಅತ್ತಿ-ಕುಪ್ಪೆ
ಬಾಳೆ-ಅತ್ತಿ-ಕುಪ್ಪೆಯ
ಬಾಳೆ-ಹಳ್ಳ
ಬಾಳೆ-ಹಳ್ಳಿ
ಬಾಳ್ಗಚ್ಚನ್ನು
ಬಾಳ್ಗಚ್ಚಾಗಿ
ಬಾಳ್ಗಳ್ಚನ್ನು
ಬಾಳ್ಗಳ್ಚಾಗಿ
ಬಾಳ್ಗಳ್ಚಿನ
ಬಾಳ್ಗಳ್ಚು
ಬಾಳ್ವೆವಂವೆತ್ತಂ
ಬಾವ
ಬಾವ-ಲಣ
ಬಾವ-ಲಣಯ
ಬಾವ-ಲ-ಯಣ-ವನ್ನು
ಬಾವಿ
ಬಾವಿ-ಗ-ಳನ್ನು
ಬಾವಿ-ಗ-ಳಿಂದ
ಬಾವಿ-ಗಳು
ಬಾವಿ-ಗ-ಳೆಂದು
ಬಾವಿತ್ತೆಯ-ದಕ್ಕ
ಬಾವಿಯ
ಬಾವಿಯಂ
ಬಾವಿ-ಯನ್ನು
ಬಾವಿ-ಯಿಂದ
ಬಾವಿ-ಸೆಟ್ಟಿ
ಬಾವಿ-ಸೆಟ್ಟಿಯ
ಬಾಷೆಯ
ಬಾಸ-ಣಯ್ಯಂ
ಬಾಸ-ಣಯ್ಯನ
ಬಾಸಣಯ್ಯನು
ಬಾಸ-ಣಯ್ಯ-ನೆಂಬು-ವ-ವನು
ಬಾಸಿಗ-ಮಿದೆನಿಸಿ
ಬಾಸಿ-ಮಯ್ಯನು-ಬಾಸ-ಣಯ್ಯ
ಬಾಸೆ
ಬಾಸೆಗೆ
ಬಾಸೆ-ಗೆ-ತಪ್ಪು-ವ-ಲೆಂಕ-ರ-ಗಂಡರುಂ
ಬಾಸೆ-ತಪ್ಪು-ವ-ವರ-ಗಣ್ಡನು
ಬಾಸೆಯ
ಬಾಸೆಯಂ
ಬಾಸೆ-ಯನ್ನು
ಬಾಹತ್ತರ
ಬಾಹತ್ತರ-ನಿ-ಯೋಗಾ-ಧಿ-ಪತಿ-ಪ-ದವಿ-ಗ-ಳನ್ನು
ಬಾಹುಃ
ಬಾಹುತು
ಬಾಹು-ಪೂರಕ-ದವೊಲ್
ಬಾಹು-ಬಲಿ
ಬಾಹು-ಬಲಿ-ಗಳು
ಬಾಹು-ಬ-ಲಿಯ
ಬಾಹು-ಬಲಿ-ಯರ
ಬಾಹು-ಸೌರ್ಯ್ಯಂ
ಬಾಹ್ಯ
ಬಿಂಕಂ
ಬಿಂಕಮ-ದೇವುದೋ
ಬಿಂಡಯ್ಯ
ಬಿಂಡಯ್ಯ-ನನ್ನು
ಬಿಂಡಯ್ಯನು
ಬಿಂಡಿಗ-ನವಿಯೂ
ಬಿಂಡಿಗ-ನವಿಲೆ
ಬಿಂಡಿಗ-ನವಿಲೆ-ಗಳು
ಬಿಂಡಿಗ-ನವಿ-ಲೆಗೆ
ಬಿಂಡಿಗ-ನವಿಲೆ-ತೀರ್ಥದ
ಬಿಂಡಿಗ-ನವಿ-ಲೆಯ
ಬಿಂಡಿಗ-ನವಿಲೆ-ಯಲ್ಲಿ
ಬಿಂಡಿಗ-ನವಿಲೆ-ಯ-ವರೆಗೂ
ಬಿಂಡಿಗ-ನವಿ-ಲೆಯು
ಬಿಂಡಿಗ-ನವಿಲೆ-ಯೊಳಗೆ
ಬಿಂಡಿ-ಗವಿಲೆ
ಬಿಂಡೇನ-ಹಳ್ಳಿ
ಬಿಂದಯ್ಯ
ಬಿಂದಾರ-ಪತಿ
ಬಿಂದಾರೊ-ಪತಿ
ಬಿಂನಂಹ
ಬಿಂನಹ
ಬಿಂಬ-ಗಳು
ಬಿಂಬದ
ಬಿಂಮ-ನಾಯಕ
ಬಿಆರ್
ಬಿಆರ್ಗೋ-ಪಾಲ್
ಬಿಆರ್ಹಿರೇ-ಮಠ್
ಬಿಎಟ್ಟೆವು
ಬಿಎಲ್
ಬಿಎಲ್ರೈಸ್
ಬಿಎಲ್ರೈಸ್ರ-ವರು
ಬಿಎಲ್ರೈಸ್ರ-ವರೂ
ಬಿಎ-ಸಾಲೆ-ತೂರ್
ಬಿಕ-ಸ-ಮುದ್ರದ
ಬಿಕ-ಸ-ಮುದ್ರ-ಬಿಕ್ಕ-ಸಂದ್ರ
ಬಿಕೆಯ-ನಾಯಕ
ಬಿಕೆಯ-ನಾಯ-ಕನು
ಬಿಕೆಯ-ನಾಯ-ಕನೂ
ಬಿಕೆಯ-ನಾಯ-ಕರು
ಬಿಕ್ಕಟ್ಟಾದ
ಬಿಕ್ಕ-ಸಂದ್ರ
ಬಿಕ್ಕ-ಸ-ಮುದ್ರ
ಬಿಗಡಾಯಿ-ಸಿರ-ಲಿಲ್ಲ-ವೆಂದು
ಬಿಚ್ಚಿದ್ದಾನೆ
ಬಿಚ್ಚು-ವುದು
ಬಿಜ-ಮಾಡಿಸಿ
ಬಿಜಯಂಗೆಯು-ವು-ದಕ್ಕೆ
ಬಿಜಯಂಗೆಯ್ಡು
ಬಿಜಯಂಗೆಯ್ದು
ಬಿಜಯಂಗೆಯ್ವ
ಬಿಜಯಂಗೈದು
ಬಿಜಯಂಗೈದು-ಯಿ-ಹಂತಾ
ಬಿಜಯಂಗೈ-ಯುತ್ತಿರ್ದು
ಬಿಜಯ-ಮಾಡಿ
ಬಿಜಯ-ಮಾಡಿ-ಸುತ್ತಿದ್ದ-ನೆಂದೂ
ಬಿಜಯ-ಮಾ-ಡುವ
ಬಿಜಯಿ
ಬಿಜಾ-ಪುರದ
ಬಿಜೆ-ಮಾಡಿ
ಬಿಜೆಯ-ಮಾಡಿದ್ದ
ಬಿಜ್ಜಲ-ದೇವಿ
ಬಿಜ್ಜಲ-ದೇವಿ-ಯರ
ಬಿಜ್ಜಲಾ-ಪುರ-ಹಾನು-ಗಲ್ಲು
ಬಿಜ್ಜಲೇಶ್ವರ-ಪುರ
ಬಿಜ್ಜಲೇಶ್ವರ-ಪುರ-ವಾದ
ಬಿಜ್ಜಲೇಶ್ವರ-ಪುರ-ವೆಂಬ
ಬಿಜ್ಜ-ಳನ
ಬಿಜ್ಜಳ-ನಿಗೆ
ಬಿಜ್ಜಳೇಶ್ವರ
ಬಿಜ್ಜೈ-ಯನು
ಬಿಜ್ಜೈಯ್ಯನು
ಬಿಟರು
ಬಿಟಿ-ರುವ
ಬಿಟ್ಟ
ಬಿಟ್ಟಂ
ಬಿಟ್ಟಂತೆ
ಬಿಟ್ಟ-ಗೊಂಡ-ನ-ಹಳ್ಳಿ-ಯನ್ನು
ಬಿಟ್ಟ-ದತ್ತಿ
ಬಿಟ್ಟದ್ದನ್ನು
ಬಿಟ್ಟದ್ದು
ಬಿಟ್ಟ-ನಾಯ-ಕ-ನ-ಹಳ್ಳಿ
ಬಿಟ್ಟನು
ಬಿಟ್ಟ-ನೆಂದು
ಬಿಟ್ಟ-ನೆಂದೂ
ಬಿಟ್ಟ-ನೆಂಬ
ಬಿಟ್ಟ-ನೆಂಬುದು
ಬಿಟ್ಟ-ಯನ
ಬಿಟ್ಟ-ರ-ಹಳ್ಳಿಯ
ಬಿಟ್ಟರು
ಬಿಟ್ಟರೆ
ಬಿಟ್ಟ-ರೆಂದು
ಬಿಟ್ಟ-ರೆಂದು-ಹೇ-ಳಿದೆ
ಬಿಟ್ಟ-ರೆಂದೂ
ಬಿಟ್ಟರ್
ಬಿಟ್ಟ-ಳೆಂದು
ಬಿಟ್ಟ-ಶಾ-ಸನ
ಬಿಟ್ಟಾಗ
ಬಿಟ್ಟಿ
ಬಿಟ್ಟಿ-ಗನು-ವಿಷ್ಣ
ಬಿಟ್ಟಿ-ಗವುಂಡ
ಬಿಟ್ಟಿ-ಗವುಡ
ಬಿಟ್ಟಿ-ಗ-ವುಡನು
ಬಿಟ್ಟಿ-ಗಾವುಂಡ
ಬಿಟ್ಟಿ-ಗಾ-ವುಡ
ಬಿಟ್ಟಿ-ಗಾ-ವುಡನ
ಬಿಟ್ಟಿ-ಗಾ-ವುಡನು
ಬಿಟ್ಟಿ-ದ-ದಾನೆ
ಬಿಟ್ಟಿ-ದೇವ
ಬಿಟ್ಟಿ-ದೇವನ
ಬಿಟ್ಟಿ-ದೇವ-ನನ್ನು
ಬಿಟ್ಟಿ-ದೇವ-ನಿಂದ
ಬಿಟ್ಟಿ-ದೇವನು
ಬಿಟ್ಟಿ-ದೇವ-ನೆಂದು
ಬಿಟ್ಟಿ-ದೇವರು
ಬಿಟ್ಟಿದ್ದ
ಬಿಟ್ಟಿದ್ದನು
ಬಿಟ್ಟಿದ್ದ-ನೆಂದು
ಬಿಟ್ಟಿದ್ದ-ನೆಂದೂ
ಬಿಟ್ಟಿದ್ದನ್ನು
ಬಿಟ್ಟಿದ್ದ-ರೆಂದೂ
ಬಿಟ್ಟಿದ್ದಾನೆ
ಬಿಟ್ಟಿದ್ದಾ-ನೆಂದು
ಬಿಟ್ಟಿದ್ದಾರೆ
ಬಿಟ್ಟಿದ್ದು
ಬಿಟ್ಟಿದ್ದುದು
ಬಿಟ್ಟಿದ್ದೇನೆ
ಬಿಟ್ಟಿನು
ಬಿಟ್ಟಿ-ಮಯ್ಯ
ಬಿಟ್ಟಿ-ಮಯ್ಯ-ಗಳ
ಬಿಟ್ಟಿ-ಮಯ್ಯನ
ಬಿಟ್ಟಿ-ಮಯ್ಯನು
ಬಿಟ್ಟಿ-ಮಯ್ಯನೂ
ಬಿಟ್ಟಿ-ಮಯ್ಯರು
ಬಿಟ್ಟಿಯ
ಬಿಟ್ಟಿ-ಯಣ್ಣ
ಬಿಟ್ಟಿ-ಯಣ್ಣನ
ಬಿಟ್ಟಿ-ಯಣ್ಣ-ನೆಂದೂ
ಬಿಟ್ಟಿ-ಯ-ದೇವನು
ಬಿಟ್ಟಿ-ಯ-ಬಂಡಿ
ಬಿಟ್ಟಿ-ಯ-ಭಂಡಿ
ಬಿಟ್ಟಿ-ಯಾಚ್ಚಾರಿಯ
ಬಿಟ್ಟಿಯು
ಬಿಟ್ಟಿ-ರ-ಬ-ಹುದು
ಬಿಟ್ಟಿ-ರ-ಬಹು-ದೆಂದು
ಬಿಟ್ಟಿರು
ಬಿಟ್ಟಿ-ರುತ್ತಾನೆ
ಬಿಟ್ಟಿ-ರುತ್ತಾರೆ
ಬಿಟ್ಟಿ-ರುವ
ಬಿಟ್ಟಿ-ರು-ವಂತೆ
ಬಿಟ್ಟಿ-ರು-ವಂತೆಯೂ
ಬಿಟ್ಟಿ-ರು-ವು-ದನ್ನು
ಬಿಟ್ಟಿ-ರು-ವುದ-ರಿಂದ
ಬಿಟ್ಟಿ-ರು-ವುದು
ಬಿಟ್ಟಿ-ರು-ವುದೂ
ಬಿಟ್ಟಿ-ಸೊಲ್ಲಗೆ
ಬಿಟ್ಟಿ-ಹೊಯ್ಸಳ-ದೇವನ
ಬಿಟ್ಟೀ-ದೇವ
ಬಿಟ್ಟೀ-ದೇವ-ನು-ಇಮ್ಮಡಿ-ಬಲ್ಲಾಳ
ಬಿಟ್ಟೀ-ದೇವ-ರ-ಸನ
ಬಿಟ್ಟು
ಬಿಟ್ಟು-ಕೊಟ್ಟ
ಬಿಟ್ಟು-ಕೊಟ್ಟಿದ್ದು
ಬಿಟ್ಟು-ಕೊಟ್ಟು
ಬಿಟ್ಟು-ಕೊಟ್ಟೆಉ
ಬಿಟ್ಟು-ಕೊಟ್ಟೆವು
ಬಿಟ್ಟು-ಕೊಡಲು
ಬಿಟ್ಟು-ಕೊಡುತ್ತಾನೆ
ಬಿಟ್ಟು-ಕೊಡುತ್ತಾರೆ
ಬಿಟ್ಟು-ಕೊಳ್ಳುತ್ತಿದ್ದರು
ಬಿಟ್ಟು-ಬಹೆವು
ಬಿಟ್ಟು-ಹೋಗಿ-ರು-ವಂತೆ
ಬಿಟ್ಟು-ಹೋಗಿವೆ
ಬಿಟ್ಟು-ಹೋ-ದಲ್ಲಿ
ಬಿಡ
ಬಿಡಕ್ಕ
ಬಿಡ-ದಂತೆ
ಬಿಡದೆ
ಬಿಡದೇ
ಬಿಡ-ಲಾ-ಗಿತ್ತು
ಬಿಡ-ಲಾಗಿತ್ತೆಂದು
ಬಿಡ-ಲಾಗಿತ್ತೆಂದೂ
ಬಿಡ-ಲಾಗಿದೆ
ಬಿಡ-ಲಾಗಿ-ದೆ-ಯೆಂದು
ಬಿಡ-ಲಾಗಿದ್ೆ
ಬಿಡ-ಲಾಗಿ-ರುತ್ತದೆ
ಬಿಡ-ಲಾಗುತ್ತದೆ
ಬಿಡ-ಲಾ-ಗುತ್ತದೇ
ಬಿಡ-ಲಾಗುತ್ತಿತು
ಬಿಡ-ಲಾಗುತ್ತಿತ್ತು
ಬಿಡ-ಲಾ-ಗುತ್ತಿತ್ತೆಂದು
ಬಿಡ-ಲಾಗುತ್ತಿದ್ದ
ಬಿಡ-ಲಾದೆ
ಬಿಡ-ಲಾ-ಯಿತು
ಬಿಡ-ಲಾಯಿ-ತೆಂದು
ಬಿಡ-ಲಾಯಿತೆಂಬ
ಬಿಡ-ಲಾಯಿತೆಂಬುದು
ಬಿಡ-ಲಿಲ್ಲ
ಬಿಡಲು
ಬಿಡಾರ
ಬಿಡಿ-ಸ-ಲಾಗಿದೆ
ಬಿಡಿ-ಸಲಾ-ಗಿ-ರುವ
ಬಿಡಿ-ಸಲು
ಬಿಡಿಸಿ
ಬಿಡಿ-ಸಿ-ಕೊಂಡು
ಬಿಡಿಸಿದ
ಬಿಡಿ-ಸಿ-ದನು
ಬಿಡಿ-ಸಿ-ದ-ನೆಂದು
ಬಿಡಿ-ಸಿ-ದರು
ಬಿಡಿ-ಸಿ-ದರೂ
ಬಿಡಿಸಿದೆ
ಬಿಡಿ-ಸಿದ್ದಾ-ನೆಂದು
ಬಿಡಿ-ಸಿ-ರ-ಬ-ಹುದು
ಬಿಡಿ-ಸಿ-ರುತ್ತಾನೆ
ಬಿಡಿ-ಸಿ-ರುತ್ತಾಳೆ
ಬಿಡಿ-ಸಿ-ರುವ
ಬಿಡಿ-ಸಿ-ರು-ವು-ದನ್ನು
ಬಿಡಿ-ಸಿ-ರುವುದು
ಬಿಡಿ-ಸುತ್ತಾನೆ
ಬಿಡಿ-ಸುತ್ತಾಳೆ
ಬಿಡಿ-ಸುತ್ತಿದ್ದ-ವರು
ಬಿಡು-ಗಡೆ
ಬಿಡುಗ್ರಾಮ
ಬಿಡು-ತಾರೆ
ಬಿಡುತ್ತಾನೆ
ಬಿಡುತ್ತಾರೆ
ಬಿಡುತ್ತಾಳೆ
ಬಿಡುತ್ತಿದ್ದ
ಬಿಡುತ್ತಿದ್ದನು
ಬಿಡುತ್ತಿದ್ದರು
ಬಿಡುತ್ತಿದ್ದ-ರೆಂಬುದು
ಬಿಡುತ್ತಿದ್ದು
ಬಿಡುತ್ತಿದ್ದುದು
ಬಿಡುವ
ಬಿಡು-ವಂತೆ
ಬಿಡು-ವಂತೆಯೂ
ಬಿಡುವಾಗ
ಬಿಡು-ವಿಲ್ಲದ
ಬಿಡು-ವುದ-ರಲ್ಲಿ
ಬಿಡು-ವು-ದಾಗಿಯೂ
ಬಿಡು-ವುದು
ಬಿಣ್ಡಯ
ಬಿಣ್ಡಯ್ಯ
ಬಿಣ್ಣಮ್ಮನ
ಬಿಣ್ನಾಂಡಿ-ಯನ್ನು
ಬಿಣ್ನಾಂಡಿಯು
ಬಿತ್ತ-ವಟ್ಟಕೆ
ಬಿತ್ತ-ವಟ್ಟ-ವನ್ನು
ಬಿತ್ತವಾಟ್ಟಕ್ಕೆ
ಬಿತ್ತಿ
ಬಿತ್ತು-ವಟ್ಟ
ಬಿತ್ತು-ವಟ್ಟಂ
ಬಿತ್ತು-ವಟ್ಟದ
ಬಿತ್ತು-ವಟ್ಟನ್ನು
ಬಿತ್ತು-ವಟ್ಟಮಂ
ಬಿತ್ತು-ವಟ್ಟ-ಮುಮಂ
ಬಿತ್ತು-ವಟ್ಟಮ್
ಬಿತ್ತು-ವಟ್ಟವಂ
ಬಿತ್ತು-ವಟ್ಟ-ವನ್ನೂ
ಬಿತ್ತು-ವಟ್ಟ-ವಾಗಿ
ಬಿತ್ತು-ವಟ್ಟಾಗಿ
ಬಿತ್ತುವಾಟಕ್ಕೆ
ಬಿತ್ತುವಾಟನ್ನು
ಬಿತ್ತುವಾಟ-ವನ್ನು
ಬಿದಿಗೆ
ಬಿದಿಯರ
ಬಿದಿರ-ಕೋಟೆ
ಬಿದಿರ-ಹಳ್ಳಿ
ಬಿದಿರು-ಕೋಟೆ
ಬಿದಿರು-ಕೋಟೆಯ
ಬಿದಿರು-ಕೋಟೆ-ಯನ್ನು
ಬಿದ್ದ
ಬಿದ್ದನು
ಬಿದ್ದ-ನೆಂದು
ಬಿದ್ದವ
ಬಿದ್ದಾಗ
ಬಿದ್ದಿದೆ
ಬಿದ್ದಿದ್ದ
ಬಿದ್ದಿದ್ದ-ನೆಂದು
ಬಿದ್ದಿದ್ದು
ಬಿದ್ದಿ-ರಲು
ಬಿದ್ದಿರುವ
ಬಿದ್ದಿ-ರು-ವು-ದನ್ನು
ಬಿದ್ದಿವೆ
ಬಿದ್ದು
ಬಿದ್ದು-ಹೋಗಿ
ಬಿದ್ದು-ಹೋಗಿದೆ
ಬಿದ್ದು-ಹೋ-ಗಿದ್ದ
ಬಿದ್ದು-ಹೋ-ಗಿದ್ದು
ಬಿದ್ದು-ಹೋಗಿ-ರುವ
ಬಿದ್ದು-ಹೋಗುವ
ಬಿನಕೋ-ಜನ
ಬಿನುಕೋ-ಜನ
ಬಿನುಗು
ಬಿನುಗು-ಜಾ-ತಿಗೆ
ಬಿನುಗು-ದೆರೆ
ಬಿನುಗು-ದೆರೆಯ
ಬಿನುಗು-ದೆಱೆ
ಬಿನುಗುಪ್ರಜೆ-ಯನ್ನು
ಬಿನ್ನಪಂ
ಬಿನ್ನಹ
ಬಿನ್ನಹಂಗೆಯೆ
ಬಿನ್ನ-ಹದ
ಬಿನ್ನಹ-ಮಾಡಿ
ಬಿನ್ನಹ-ಮಾಡಿ-ಕೊಂಡ
ಬಿನ್ನಹ-ವನ್ನು
ಬಿಮಿ-ಸೆಟ್ಟಿ
ಬಿಯಳಮ್ಮ
ಬಿಯಳಮ್ಮ-ನಿಗೆ
ಬಿಯಳಮ್ಮನು
ಬಿರಿದು
ಬಿರಿಯು-ವಂತೆ
ಬಿರಿಯೆ
ಬಿರುಕು
ಬಿರುಡೆಯ
ಬಿರುಡೆ-ಯನ್ನು
ಬಿರುದ
ಬಿರುದಂತೆಂಬರ
ಬಿರುದಂತೆಂಬರ-ಗಂಡ
ಬಿರು-ದನ್ನು
ಬಿರುದನ್ನು-ಹುದ್ದೆ-ಯನ್ನು
ಬಿರುದರ
ಬಿರುದ-ರ-ಗಂಡ
ಬಿರುದ-ರ-ಗೋವ
ಬಿರುದ-ರ-ಗೋವರುಂ
ಬಿರುದಾಂಕಿತ
ಬಿರುದಾಂಕಿತ-ನಾದ
ಬಿರು-ದಾ-ಗಿತ್ತು
ಬಿರು-ದಾಗಿದೆ
ಬಿರು-ದಾ-ಗಿದ್ದು
ಬಿರುದಾಗಿರ-ಬ-ಹುದು
ಬಿರುದಾದ
ಬಿರುದಾ-ವಳಿ
ಬಿರುದಾ-ವಳಿ-ಗಳ
ಬಿರುದಾ-ವಳಿ-ಗ-ಳನ್ನು
ಬಿರುದಾ-ವಳಿ-ಯನ್ನು
ಬಿರುದಿತ್ತ-ದೆಂದು
ಬಿರು-ದಿತ್ತು
ಬಿರುದಿತ್ತೆಂದು
ಬಿರುದಿತ್ತೆಂದೂ
ಬಿರುದಿತ್ತೆಂಬುದು
ಬಿರುದಿದೆ
ಬಿರುದಿದ್ದುದು
ಬಿರು-ದಿದ್ದುನ್ನು
ಬಿರುದಿನ
ಬಿರುದಿ-ನಂತೆ
ಬಿರುದಿರ-ಬ-ಹುದು
ಬಿರು-ದಿಲ್ಲ
ಬಿರುದು
ಬಿರುದು-ಗಳ
ಬಿರುದು-ಗ-ಳನ್ನು
ಬಿರುದು-ಗಳನ್ನುದ್ಧರಿಸಿ
ಬಿರುದು-ಗ-ಳನ್ನೂ
ಬಿರುದು-ಗ-ಳಲ್ಲಿ
ಬಿರುದು-ಗ-ಳಾಗಿದ್ದು
ಬಿರುದು-ಗಳಾವುವೂ
ಬಿರುದು-ಗ-ಳಿಂದ
ಬಿರುದು-ಗಳಿದ್ದಂತೆ
ಬಿರುದು-ಗಳಿದ್ದವು
ಬಿರುದು-ಗಳಿದ್ದು
ಬಿರುದು-ಗಳಿವೆ
ಬಿರುದು-ಗಳು
ಬಿರುದು-ಗಳೂ
ಬಿರುದು-ಗಳೇ
ಬಿರುದು-ಬಾವ-ಲಿ-ಗ-ಳನ್ನು
ಬಿರುದುಳ್ಳ
ಬಿರುದೂ
ಬಿರುದೆಂತೆಂಬರ
ಬಿರುದೆಂಬರ-ಗಂಡ
ಬಿರುದೇ
ಬಿರುದೈರ್ವಂದಿತತ್ಯಾ-ನಿತ್ಯ-ಮಭಿಷ್ಟತಃ
ಬಿಲ್ಬಲ್ಮೆಯಂ
ಬಿಲ್ಲಂಗೆರೆಯ
ಬಿಲ್ಲ-ಗೊಂಡ-ನ-ಹಳ್ಳಿ
ಬಿಲ್ಲನ್ನು
ಬಿಲ್ಲಪ್ಪ
ಬಿಲ್ಲ-ಬೆಳ-ಗುಂದ
ಬಿಲ್ಲ-ಬೆಳ-ಗುಂದಕ್ಕೆ
ಬಿಲ್ಲ-ಬೆಳ-ಗುಂದದ
ಬಿಲ್ಲ-ಬೆಳ-ಗುಂದ-ಹಳ್ಳಿ
ಬಿಲ್ಲ-ಮೂಲೂ-ನೂರ್ಪ್ಪಬ್ಬರು
ಬಿಲ್ಲಯ್ಯ-ನಿಗೆ
ಬಿಲ್ಲ-ರಾಮ-ನ-ಹಳ್ಳಿ
ಬಿಲ್ಲ-ವರ
ಬಿಲ್ಲ-ವರಿ-ರ-ಬೇಕು
ಬಿಲ್ಲೆ
ಬಿಲ್ವಿದ್ಯೆ-ಯಲ್ಲಿ
ಬಿಲ್ಹ-ಣನು
ಬಿಳಿ-ಕಲ್ಲು-ಮಂಠಿ
ಬಿಳಿ-ಕೆರೆ
ಬಿಳಿ-ಕೆರೆ-ಯನ್ನು
ಬಿಳಿ-ಗೆರೆ
ಬಿಳಿ-ನಾಮಕ್ಕೆ
ಬಿಳಿ-ಮಗ್ಗ-ದ-ವರ
ಬಿಳಿಯ
ಬಿಳಿಯಕ್ಕಿ-ಯನ್ನು
ಬಿಳಿಯಕ್ಕಿಯುಂ
ಬಿಳಿಯಕ್ಕಿಯೆಮ್ಬುದಿಲ್ಪ
ಬಿಳು-ಗಲಿ
ಬಿಷ್ಣುನೃ-ಪತಿ
ಬಿಸ
ಬಿಸ-ಗೂರ-ರ-ವರು
ಬಿಸಗೆ
ಬಿಸಾಡಿ-ಕೊಪ್ಪಲು
ಬಿಸಿ-ಯಾದ
ಬಿಸು-ಗೆಯ
ಬಿಸು-ಗೆಯ-ಕಳ
ಬೀಚ-ಗವುಂಡಿ
ಬೀಚ-ಜೀಯ
ಬೀಚನ-ಹಳ್ಳಿ
ಬೀಚನ-ಹಳ್ಳಿಯ
ಬೀಚನ-ಹಳ್ಳಿ-ಯನ್ನು
ಬೀಚವ್ವೆ
ಬೀಚವ್ವೆ-ಯನ್ನು
ಬೀಚಿ-ರಾಜ
ಬೀಚೆಯ
ಬೀಚೆಯ-ದಂಡ-ನಾಯ-ಕನ
ಬೀಚೆಯ-ನ-ಕೆರೆ
ಬೀಚೆಯ-ನ-ಕೆರೆಯ
ಬೀಚೇನ-ಹಳ್ಳಿ
ಬೀಜ-ಗಂಡುಗ
ಬೀಜ-ವರಿ
ಬೀಜ-ವರಿ-ಗದ್ದೆ
ಬೀಜ-ವರಿ-ಗದ್ದೆ-ಯನ್ನು
ಬೀಜ-ವರಿ-ಯನು
ಬೀಜ-ವರೀ
ಬೀಜವೊಂನಾಗಿ
ಬೀಜಾವಾಪ-ಮಾತ್ರಂ
ಬೀಜೋತ್ಪಾದ-ನೆ-ಗಾಗಿ
ಬೀಜೋತ್ಪಾದ-ನೆ-ಯನ್ನು
ಬೀಡಂ
ಬೀಡಾಗಿತ್ತು
ಬೀಡಿದಿ-ನಿಂದ
ಬೀಡಿನ
ಬೀಡಿ-ನಲ್ಲಿ
ಬೀಡಿನಲ್ಲಿದ್ದ
ಬೀಡಿನಲ್ಲಿದ್ದಾಗ
ಬೀಡಿನಿಂದ
ಬೀಡಿ-ನೊಳಗೆ
ಬೀಡು
ಬೀಡು-ಬಿಟ್ಟನು
ಬೀಡು-ಬಿಟ್ಟಲ್ಲಿ
ಬೀಡು-ಬಿಟ್ಟಿದ್ದ
ಬೀಡು-ಬಿಟ್ಟಿದ್ದ-ನೆಂದು
ಬೀಡು-ಬಿಟ್ಟಿದ್ದಾಗ
ಬೀಡು-ಬಿಟ್ಟಿ-ರ-ಬ-ಹುದು
ಬೀಡು-ಬಿಟ್ಟು
ಬೀಡು-ಬಿ-ಡಲು
ಬೀಡು-ಬಿಡುತ್ತದೆ
ಬೀದಿ
ಬೀದಿಗೆ
ಬೀದಿ-ಯಲ್ಲಿ
ಬೀದಿ-ಯಲ್ಲಿ-ರುವ
ಬೀದಿಯು
ಬೀದಿಯೇ
ಬೀಬಿ
ಬೀಮಣ್ಣನ
ಬೀಯಮ್ಮ
ಬೀರ
ಬೀರಂ
ಬೀರಂಮಲೆ-ಯಲ್ಲಿ
ಬೀರ-ಕಲನು
ಬೀರ-ಕಲ್ಲು
ಬೀರಕ್ಕ
ಬೀರಕ್ಕನ
ಬೀರಕ್ಕಾ-ಗರ-ಮಾಗಿ
ಬೀರ-ಗಲು
ಬೀರ-ಗಲ್ಲನ್ನಿರಿಸುತ್ತಾರೆ
ಬೀರ-ಗಲ್ಲನ್ನು
ಬೀರ-ಗಲ್ಲು
ಬೀರ-ಗವುಂಡನು
ಬೀರ-ಗವುಡ
ಬೀರ-ಗಾವುಂಡನ
ಬೀರ-ಗಾವುಂಡನು
ಬೀರ-ಗೌಡ
ಬೀರ-ಗೌಡನ
ಬೀರ-ನ-ಹಳ್ಳಿ
ಬೀರ-ಬಲ್ಲ-ವ-ರಾಗಿದ್ದರು
ಬೀರಮಂ
ಬೀರಯ್ಯ
ಬೀರಯ್ಯ-ನನ್ನು
ಬೀರಯ್ಯ-ನೆಂದು
ಬೀರರಂ
ಬೀರ-ಲಕ್ಷ್ಮಿ
ಬೀರ-ವಣಾ-ವೃತ್ತಿ
ಬೀರ-ವಣಾ-ವೃತ್ತಿಯು
ಬೀರ-ವಣಿಗ
ಬೀರ-ವನ್ನು
ಬೀರ-ಶಾ-ಸನ
ಬೀರ-ಸೆಟ್ಟಿ
ಬೀರಾ-ದೇವಿ
ಬೀರಾ-ದೇ-ವಿಗೆ
ಬೀರಾ-ದೇವಿಯ
ಬೀರಾ-ದೇವಿಯು
ಬೀರಾ-ದೇವಿಯೆ
ಬೀರಿ
ಬೀರಿತು
ಬೀರಿ-ದರು
ಬೀರಿದ್ದಂತೆ
ಬೀರಿ-ಸೆಟ್ಟಿ-ಹಳ್ಳಿ
ಬೀರುಗ
ಬೀರುಗೆ-ಹಳ್ಳಿಯ
ಬೀರುಗೆ-ಹಳ್ಳಿ-ಯನು
ಬೀರುಗೆ-ಹಳ್ಳಿ-ಯನ್ನು
ಬೀರುಗೆ-ಹಳ್ಳಿ-ಯಲ್ಲಿದ್ದ
ಬೀರುತ್ತಾ
ಬೀರು-ಬಳ್ಳಿ
ಬೀರು-ಬಳ್ಳಿ-ಯನು
ಬೀರು-ಬಳ್ಳಿ-ಯನ್ನು
ಬೀರು-ಬಳ್ಳಿ-ಯಿಂದ
ಬೀರುವ
ಬೀರು-ವಳ್ಳಿ
ಬೀರು-ಹಳ್ಳಿ
ಬೀರೆಯನ
ಬೀರೆಯ-ನಾಯಕ
ಬೀರೆಯ-ನಾಯ-ಕನ
ಬೀರೆಯ್ಯ
ಬೀಳಲು
ಬೀಳ-ವೃತ್ತಿ
ಬೀಳ-ವೃತ್ತಿ-ಯಿಂದ
ಬೀಳು-ಕೊಟ್ಟಾಗ
ಬೀಳುತ್ತಾನೆ
ಬೀಳುತ್ತಿರ-ಲಿಲ್ಲ
ಬೀಳ್ಕೊಡೆ
ಬೀವಿ-ಬೀಬಿ-ಬೇವಿ-ಬೇಬಿ
ಬೀವಿ-ಸೆಟ್ಟಿ
ಬೀವಿ-ಸೆಟ್ಟಿಯ
ಬೀಸುವ
ಬೀೞಅ್ಗುಂಡಿಕ್ಕಿದ
ಬೀೞ-ವೃತ್ತಿ
ಬೀೞ-ವೃತ್ತಿ-ಯಂತಹದೇ
ಬೀೞ-ವೃತ್ತಿಯು
ಬೀೞಾನು-ವೃತ್ತಿ-ಯಿಂದ
ಬುಕಂಣ
ಬುಕ್ಕ
ಬುಕ್ಕಂಣ
ಬುಕ್ಕಣ್ಣ
ಬುಕ್ಕಣ್ಣನ
ಬುಕ್ಕಣ್ಣನು
ಬುಕ್ಕಣ್ಣ-ವೊಡೆ-ಯರ
ಬುಕ್ಕನ
ಬುಕ್ಕ-ನನ್ನೇ
ಬುಕ್ಕನು
ಬುಕ್ಕ-ನೃ-ಪತಿ-ನೊಳಂದತಿ-ಶ-ಯದಿಂ
ಬುಕ್ಕನೇ
ಬುಕ್ಕ-ಮಹೀ-ಪಾಲ
ಬುಕ್ಕ-ಮಹೀ-ಪಾಲ-ನೆಂಬ-ಒಂದನೇ
ಬುಕ್ಕಮಾ
ಬುಕ್ಕ-ರಾಜನ
ಬುಕ್ಕ-ರಾಜನು
ಬುಕ್ಕ-ರಾಜ-ಪುರ
ಬುಕ್ಕ-ರಾಜ-ಪುರ-ವಾದ
ಬುಕ್ಕ-ರಾಜ-ಪುರ-ವೆಂಬ
ಬುಕ್ಕ-ರಾಜ-ರಾಯಾಬಾಹೂತ
ಬುಕ್ಕ-ರಾಯ
ಬುಕ್ಕ-ರಾಯ-ತನೂ-ಭವ
ಬುಕ್ಕ-ರಾಯನ
ಬುಕ್ಕ-ರಾಯ-ನನ್ನು
ಬುಕ್ಕ-ರಾಯ-ನಿಗೆ
ಬುಕ್ಕ-ರಾಯನು
ಬುಕ್ಕ-ರಾಯನೇ
ಬುಕ್ಕ-ರಾಯ-ಪುರ-ವಾದ
ಬುಕ್ಕ-ರಾಯರು
ಬುಕ್ಕ-ರಾಯ-ಸ-ಮುದ್ರ
ಬುಕ್ಕರು
ಬುಕ್ನಾನ್
ಬುಖನಾನ್
ಬುಜ-ಬಲಾ-ಚಾರಿ
ಬುಟ್ಟಿಹೆ-ಣೆಯು-ವ-ವರು
ಬುದ್ಧಿ-ಜೀವಿ-ಗಳ
ಬುದ್ಧಿ-ಯಿಂದ
ಬುದ್ಧಿಲ
ಬುಧ-ಜನ
ಬುಧ-ಸಂಪದಾಂ
ಬುಧೈ-ಕಕಲ್ಪಭೂ
ಬುರಹಾನ್ಅಲ್ದೀನ್
ಬುರಾನುದ್ದೀನ್
ಬುರಾನುದ್ದೀನ್ನ
ಬುರ್ಹಾನ್
ಬುಳ್ಳಪ್ಪ-ನಾಯ-ಕರ
ಬೂಕನ
ಬೂಕನ-ಕೆರೆ
ಬೂಕಿನ
ಬೂಕಿನ-ಕೆರೆ
ಬೂಕಿನ-ಕೆರೆಯು
ಬೂಕಿ-ಸೆಟ್ಟಿ
ಬೂಚಣ
ಬೂಚಲೆ
ಬೂಚಿ-ಯಣ್ಣ
ಬೂಚಿ-ರಾಜ
ಬೂಚಿ-ರಾಜನು
ಬೂತ-ಗನು
ಬೂತ-ರ-ಸರು
ಬೂತಾರ್ಯನು
ಬೂತುಗ
ಬೂತುಗನ
ಬೂತುಗ-ನನ್ನು
ಬೂತುಗ-ನ-ರಸಿ
ಬೂತುಗ-ನಿಗೆ
ಬೂತುಗನು
ಬೂತುಗನೇ
ಬೂತುಗ-ನೊಂದಿಗೆ
ಬೂತುಗರ
ಬೂತುಗ-ಸತ್ಯ-ವಾಕ್ಯ
ಬೂದ-ನೂರ
ಬೂದ-ನೂರಾದ
ಬೂದ-ನೂರಿನ
ಬೂದ-ನೂರಿನಲ್ಲಿ
ಬೂದ-ನೂರು
ಬೂದ-ನೂರು-ಹೊಸ
ಬೂವಣ್ನ
ಬೂವನ-ಹಳ್ಳಿ
ಬೂವನ-ಹಳ್ಳಿ-ಗಳ
ಬೂವನ-ಹಳ್ಳಿ-ಯನ್ನು
ಬೃಂದಾ-ವನ
ಬೃಂದಾ-ವನ-ದ-ಗಳು
ಬೃಹತ್
ಬೃಹತ್ತಾದ
ಬೃಹದಾ-ಕಾರದ
ಬೃಹನ್ಮಠವು
ಬೃಹಸ್ಪತಿ-ವಾರ
ಬೆಂಕಿ
ಬೆಂಕಿ-ನವಾಬ-ನೆಂದು
ಬೆಂಕೊಂಡು
ಬೆಂಗಳೂ-ರನ್ನು
ಬೆಂಗಳೂ-ರಿನ
ಬೆಂಗಳೂರು
ಬೆಂಗಿ
ಬೆಂಗಿರಿ
ಬೆಂಡರ-ವಾಡಿಯ
ಬೆಂಡರ-ವಾಡಿ-ಯಲ್ಲಿದೆ
ಬೆಂನಟ್ಟಿದಂ
ಬೆಂನೂರ
ಬೆಂನ್ನಂ
ಬೆಂಬಲ
ಬೆಂಬಲಕ್ಕೆ
ಬೆಂಬಲ-ದಿಂದ
ಬೆಂಬೆತ್ತಿ
ಬೆಂಬೆತ್ತಿ-ಹೋದನು
ಬೆಕನಾಟ
ಬೆಕ್ಕದ
ಬೆಗಿವಂದದ
ಬೆಗೆ-ಗವು-ಡ-ನೆಂದು
ಬೆಗೆವಂದಕ್ಕೆ
ಬೆಗೆವ-ಡದ
ಬೆಗೆ-ವಡೆದ
ಬೆಗೆ-ವನ್ದ
ಬೆಟಾ-ಲಿಯನ್ಗೆ
ಬೆಟ್ಟ
ಬೆಟ್ಟ-ಕೋಟೆ
ಬೆಟ್ಟ-ಕೋಟೆ-ಯ-ಹಳ್ಳಿ
ಬೆಟ್ಟಕ್ಕೆ
ಬೆಟ್ಟ-ಗಳ
ಬೆಟ್ಟ-ಗಳಅ
ಬೆಟ್ಟ-ಗ-ಳಿಗೆ
ಬೆಟ್ಟ-ಗಳಿದು
ಬೆಟ್ಟ-ಗಳಿವೆ
ಬೆಟ್ಟ-ಗಳು
ಬೆಟ್ಟ-ಗುಡ್ಡ-ಗಳೂ
ಬೆಟ್ಟದ
ಬೆಟ್ಟ-ದ-ಕೋಟೆ
ಬೆಟ್ಟ-ದ-ಕೋಟೆಗೆ
ಬೆಟ್ಟ-ದ-ಕೋಟೆ-ಯ-ಹಳ್ಳಿ
ಬೆಟ್ಟ-ದ-ಚಾಮ-ರಾಜ
ಬೆಟ್ಟ-ದ-ಚಾಮ-ರಾಜನು
ಬೆಟ್ಟ-ದ-ಪುರ
ಬೆಟ್ಟ-ದ-ಮೇಲಿ-ರುವ
ಬೆಟ್ಟ-ದ-ಮೇಲೆ
ಬೆಟ್ಟ-ದಲ್ಲಿ
ಬೆಟ್ಟ-ದಲ್ಲಿ-ರುವ
ಬೆಟ್ಟ-ದ-ಹಳ್ಳಿ
ಬೆಟ್ಟ-ದ-ಹಳ್ಳಿ-ಯಿಂದ
ಬೆಟ್ಟ-ದಿಂದ
ಬೆಟ್ಟ-ನಾಯ-ಕನ
ಬೆಟ್ಟ-ಮೆನೆ
ಬೆಟ್ಟಮ್ಮೇಲ್ಕಾಲಂ
ಬೆಟ್ಟಯ್ಯ
ಬೆಟ್ಟ-ವನ್ನು
ಬೆಟ್ಟ-ಹಳ್ಳಿ
ಬೆಟ್ಟ-ಹಳ್ಳಿ-ಯನ್ನು
ಬೆಟ್ಟು-ಕೋಟೆ
ಬೆಡಗಿಗೆ
ಬೆಡಗು
ಬೆಡಗು-ಗಳಿವೆ
ಬೆಡಗು-ಗಳು
ಬೆಡಿಗೆ
ಬೆಡುಂಗೊಳು
ಬೆಡು-ಗೂಳು
ಬೆಣಚು-ಕಲ್ಲಿನ
ಬೆಣ್ಣೆ-ಗೆರೆಯ
ಬೆಣ್ಣೆದೊಣೆ-ಯಲ್ಲಿ
ಬೆಣ್ಣೆ-ಸಿದ್ದ-ನ-ಗುಡ್ಡದ
ಬೆತ್ತಲೆ-ಯಾಗಿ
ಬೆತ್ತು
ಬೆದನ್ತೆ
ಬೆದರಿಸಿ
ಬೆದರೆ
ಬೆದಿ-ಕೆರೆ
ಬೆದ್ದಲ
ಬೆದ್ದ-ಲನ್ನು
ಬೆದ್ದ-ಲಿಗೆ
ಬೆದ್ದಲು
ಬೆದ್ದ-ಲು-ಗಳ
ಬೆದ್ದ-ಲು-ಗ-ಳನ್ನು
ಬೆದ್ದ-ಲು-ಗ-ಳನ್ನೂ
ಬೆದ್ದ-ಲು-ಗಳ-ವರೆಗೂ
ಬೆದ್ದ-ಲು-ಗ-ಳಿಗೆ
ಬೆದ್ದ-ಲು-ಗಳು
ಬೆದ್ದ-ಲು-ಗಳು-ಹೊಲ-ಗಳು
ಬೆದ್ದಲೆ
ಬೆದ್ದ-ಲೆ-ಗ-ಳನ್ನು
ಬೆದ್ದ-ಲೆ-ಯ-ನತಿ-ಬಳಂ
ಬೆದ್ದ-ಲೆ-ಯನ್ನು
ಬೆದ್ದವ್ವೆಯ-ಕೆರೆ
ಬೆನ-ಕನ-ಕೆರೆ
ಬೆನವಾ-ರವು
ಬೆನ್ನಂ
ಬೆನ್ನಚರ್ಮವೇ
ಬೆನ್ನಟ್ಟಿ
ಬೆನ್ನ-ಹಿಂದೆಯೇ
ಬೆನ್ನಾ-ವರದ
ಬೆನ್ನಿಗ-ಸೆಟ್ಟಿಯು
ಬೆನ್ನು
ಬೆನ್ನು-ಹತ್ತಿ
ಬೆಮತೂರ-ಕಲ್ಲ
ಬೆಮ್ಬಮ್ಪಾಳ್
ಬೆರಳಚ್ಚು
ಬೆರಳೆ-ಣಿಕೆ
ಬೆರಳೆ-ಣಿಕೆ-ಯಲ್ಲಿವೆ
ಬೆರಳೆ-ಣಿಕೆ-ಯಷ್ಟಿವೆ
ಬೆರಸು
ಬೆರೆತು
ಬೆರೆಸಿ
ಬೆರೆಸು
ಬೆರ್ರ-ಡಿಯಾನ್
ಬೆಲತೂರು
ಬೆಲತ್ತೂರು
ಬೆಲದ-ತಾಲ
ಬೆಲ-ವತ್ತ
ಬೆಲ-ವೊಂದು
ಬೆಲಹುರ-ಬೇ-ಲೂರು
ಬೆಲಹೂರ-ಬೇ-ಲೂರು-ಅಧಿ-ಕಾರಿಯು
ಬೆಲುಹೂರಲಿ
ಬೆಲೂರಿನ
ಬೆಲೂರು-ಬೆಳ್ಳೂರು
ಬೆಲೆ
ಬೆಲೆ-ಕೆರೆ
ಬೆಲೆ-ಕೆರೆ-ಇಂದಿನ
ಬೆಲೆ-ಯನ್ನು
ಬೆಲೆ-ಯುಳ್ಳ
ಬೆಲ್ಲ
ಬೆಲ್ಲೂರ-ಬೆಳ್ಳೂರು
ಬೆಲ್ಲೂರು
ಬೆಳ-ಕನ್ನು
ಬೆಳಕ-ವಾಡಿ
ಬೆಳಕ-ವಾಡಿಗೆ
ಬೆಳಕ-ವಾಡಿಯ
ಬೆಳಕ-ವಾಡಿ-ಯನ್ನು
ಬೆಳಕ-ವಾಡಿ-ಯಲ್ಲಿ
ಬೆಳಕ-ವಾಡಿ-ಯಲ್ಲೂ
ಬೆಳಕ-ವಾಡಿಯು
ಬೆಳ-ಕಿಗೆ
ಬೆಳಕು
ಬೆಳ-ಗಲಿ
ಬೆಳಗಾವಿ
ಬೆಳಗಾವಿಯ
ಬೆಳ-ಗಿನ-ಜಾವದ
ಬೆಳಗಿ-ಸಿದಳು
ಬೆಳ-ಗುಂದ
ಬೆಳ-ಗುಂಬ
ಬೆಳ-ಗುಂಬದ
ಬೆಳ-ಗುಳದ
ಬೆಳಗೊಳ
ಬೆಳಗೊಳದ
ಬೆಳಗೊಳ-ದ-ವ-ರಾಗಿದ್ದಾರೆ
ಬೆಳಗ್ಗೆ
ಬೆಳತೂರ
ಬೆಳತೂ-ರಿಗೆ
ಬೆಳತೂರಿನ
ಬೆಳತೂರಿನಲ್ಲಿ-ರುವ
ಬೆಳತೂರು
ಬೆಳ-ವಡಿ-ಯಲಿ
ಬೆಳ-ವಡಿ-ಯಲ್ಲಿ
ಬೆಳ-ವಡಿ-ಯಿಂದ
ಬೆಳ-ವಣಿಗೆಗೆ
ಬೆಳ-ವಣಿಗೆಯ
ಬೆಳ-ವಣಿಗೆ-ಯನ್ನು
ಬೆಳ-ವಣಿಗೆ-ಯಲ್ಲಿ
ಬೆಳ-ವಾಡಿ
ಬೆಳ-ವಾಡಿಯ
ಬೆಳ-ವಾಡಿ-ಯಲ್ಲಿ
ಬೆಳು-ಗಲಿ-ಯಲ್ಲಿದ್ದ
ಬೆಳು-ಪಿನೊಳಾತ್ಮ
ಬೆಳುವ-ಲ-ವೊಪ್ಪುವ
ಬೆಳುವೊ-ಲದ
ಬೆಳುಹು
ಬೆಳೂರ
ಬೆಳೆ
ಬೆಳೆ-ಗಳ
ಬೆಳೆದ
ಬೆಳೆ-ದಂತೆ
ಬೆಳೆ-ದನು
ಬೆಳೆ-ದ-ನೆಂದು
ಬೆಳೆ-ದವು
ಬೆಳೆ-ದಾಗ
ಬೆಳೆ-ದಿತ್ತು
ಬೆಳೆ-ದಿತ್ತೆಂದು
ಬೆಳೆ-ದಿ-ರುವುದು
ಬೆಳೆದು
ಬೆಳೆ-ದು-ಕೊಂಡು
ಬೆಳೆ-ದುದು
ಬೆಳೆಯ
ಬೆಳೆ-ಯ-ತೊಡಗಿತು
ಬೆಳೆ-ಯನ
ಬೆಳೆ-ಯ-ನ-ಹಳ್ಳಿ
ಬೆಳೆ-ಯ-ನ-ಹಳ್ಳಿಯ
ಬೆಳೆ-ಯಿತು
ಬೆಳೆ-ಯಿ-ತೆಂದು
ಬೆಳೆ-ಯುತ್ತಾ
ಬೆಳೆ-ಯುತ್ತಾರೆ
ಬೆಳೆ-ಯುತ್ತಿತು
ಬೆಳೆ-ಯುತ್ತಿದ್ದ
ಬೆಳೆ-ಯುತ್ತಿದ್ದ-ರೆಂದು
ಬೆಳೆ-ಯುವ
ಬೆಳೆಸಿ
ಬೆಳೆ-ಸಿ-ಕೊಂಡು
ಬೆಳೆ-ಸಿ-ಕೊಳ್ಳುತ್ತಾ
ಬೆಳೆ-ಸಿ-ಕೊಳ್ಳು-ವಂತೆ
ಬೆಳೆ-ಸಿತು
ಬೆಳೆ-ಸಿದ
ಬೆಳೆ-ಸಿ-ದನು
ಬೆಳೆ-ಸಿ-ದ-ರೆಂದು
ಬೆಳೆ-ಸಿ-ದಳು
ಬೆಳೆ-ಸಿದ್ದ-ನಷ್ಟೆ
ಬೆಳೆ-ಸಿದ್ದನೋ
ಬೆಳೆ-ಸಿದ್ದರು
ಬೆಳೆ-ಸಿದ್ದವು
ಬೆಳೆ-ಸುತ್ತಿದ್ದ-ರೆಂಬ
ಬೆಳೆ-ಸು-ವುದ-ರಲ್ಲಿ
ಬೆಳ್ಕೆರೆ
ಬೆಳ್ಗಲ್ಲಿಗೆ-ರೆಯ
ಬೆಳ್ಗುಂದದ
ಬೆಳ್ಗುಪ್ಪ
ಬೆಳ್ಗೊಳ
ಬೆಳ್ಗೊ-ಳದ
ಬೆಳ್ಗೊಳ-ದಲ್ಲಿ
ಬೆಳ್ಗೊಳ-ದೊಳ್ಜನ-ಮೆಲ್ಲಂ
ಬೆಳ್ಗೊಳ-ವಾಗುತ್ತದೆ
ಬೆಳ್ಗೊಳಾಧಿಪ-ತಿ-ಗಳಪ್ಪ
ಬೆಳ್ಪುಂ
ಬೆಳ್ಳ-ಹಳ್ಳಿಯ
ಬೆಳ್ಳಾಲೆ
ಬೆಳ್ಳಿ
ಬೆಳ್ಳಿ-ಕುಲೋದ್ಭವ
ಬೆಳ್ಳಿ-ಕೆರೆ
ಬೆಳ್ಳಿ-ಕೊಡ-ವನ್ನೂ
ಬೆಳ್ಳಿ-ಗ-ವುಡನ
ಬೆಳ್ಳಿ-ಗಿಂಡಿ-ಯನ್ನು
ಬೆಳ್ಳಿ-ಗುಂಬ
ಬೆಳ್ಳಿ-ತಟ್ಟೆಯ
ಬೆಳ್ಳಿ-ಬಟ್ಟ-ಲನ್ನು
ಬೆಳ್ಳಿ-ಬಟ್ಟಲು
ಬೆಳ್ಳಿ-ಬಟ್ಟಲು-ಗಳ
ಬೆಳ್ಳಿ-ಬೆಟ್ಟದ
ಬೆಳ್ಳಿ-ಮಾಣಿಯ
ಬೆಳ್ಳಿ-ಮುಲಾಮಿನ
ಬೆಳ್ಳಿಯ
ಬೆಳ್ಳಿ-ಯನ್ನು
ಬೆಳ್ಳಿ-ಯ-ಬೆಟ್ಟ-ದಂತಿ-ರುವ
ಬೆಳ್ಳಿ-ಯರ
ಬೆಳ್ಳಿ-ಯರು
ಬೆಳ್ಳೂರ
ಬೆಳ್ಳೂ-ರನ್ನು
ಬೆಳ್ಳೂ-ರನ್ನೇ
ಬೆಳ್ಳೂರಲಿ
ಬೆಳ್ಳೂ-ರಿಗೆ
ಬೆಳ್ಳೂರಿನ
ಬೆಳ್ಳೂರಿ-ನಲ್ಲಿ
ಬೆಳ್ಳೂರಿ-ನಲ್ಲೇ
ಬೆಳ್ಳೂರಿ-ನಿಂದ
ಬೆಳ್ಳೂರಿ-ನೊಳಗೆ
ಬೆಳ್ಳೂರು
ಬೆಳ್ಳೂರೊಳ-ಗಣ
ಬೆಳ್ಳೂ-ರೊಳು
ಬೆಳ್ಳೂ-ರೊಳ್
ಬೆಳ್ಳೇರ
ಬೆಳ್ಳೇರು
ಬೆಳ್ವಲ
ಬೆಳ್ವೊಲ
ಬೆಳ್ವೊ-ಲದ
ಬೆಳ್ವೊಲ-ನಾಡಿನ
ಬೆವಹ-ರ-ಪೂಜಾ-ಕೈಂಕರ್ಯ
ಬೆಸ-ಗರ-ಹಳ್ಳಿ
ಬೆಸ-ಗರ-ಹಳ್ಳಿಯ
ಬೆಸ-ಗರ-ಹಳ್ಳಿ-ಯನ್ನು
ಬೆಸಟೆಯ
ಬೆಸಣಿಪೆಸಾಣಿ
ಬೆಸದಿಂ
ಬೆಸ-ದಿಂದ
ಬೆಸ-ದೊಳು
ಬೆಸದೊಳೆ
ಬೆಸನಂ
ಬೆಸ-ನನ್ನು
ಬೆಸವಕ್ಕಳ
ಬೆಸ-ಸಲು
ಬೆಸಸಲ್ಕರ-ವರಿ-ಸದೆ
ಬೆಸಸಿ
ಬೆಸಸಿ-ದ-ನೆಂದು
ಬೆಸೆ-ಸಿ-ದನು
ಬೆಸ್ತರ
ಬೇಂಟೆ-ಕಾರ
ಬೇಕಾಗಿ
ಬೇಕಾಗಿತ್ತು
ಬೇಕಾಗಿತ್ತೆಂದು
ಬೇಕಾಗುತ್ತಿತ್ತು
ಬೇಕಾ-ಗುತ್ತಿತ್ತೆಂಬುದು
ಬೇಕಾದ
ಬೇಕಾದ-ವ-ರಾಗಿದ್ದ-ರೆಂದು
ಬೇಕಾದವು
ಬೇಕಾದಷ್ಟಿವೆ
ಬೇಕಾದಾಗ
ಬೇಕಾ-ದು-ದನ್ನು
ಬೇಕು
ಬೇಕೆಂದು
ಬೇಕೆಂಬುದು
ಬೇಗ
ಬೇಗಂ
ಬೇಗ-ಮಂಗಲ
ಬೇಗ-ಮಂಗಲ-ವಾಗಿ-ರ-ಬ-ಹುದು
ಬೇಗ-ಮಂಗಲ-ವೆಂಬ
ಬೇಗವ್ವೆ
ಬೇಗವ್ವೆಯ
ಬೇಚರಾಕ್
ಬೇಚಿರಾಕ್
ಬೇಟೆ
ಬೇಟೆ-ಗಾಗಿ
ಬೇಟೆಗೆ
ಬೇಟೆ-ಯನ್ನಾಡು-ವುದು
ಬೇಟೆ-ಯಲ್ಲಿ
ಬೇಟೆ-ಯಾಡುತ್ತಾ
ಬೇಟೆ-ಯಾಡು-ವುದ-ರಲ್ಲಿ
ಬೇಡದೆ
ಬೇಡಬೇಡ-ವೆಂದು
ಬೇಡರ
ಬೇಡ-ರನ್ನು
ಬೇಡರ-ಪಡೆ
ಬೇಡರ-ಪಡೆಯು
ಬೇಡರ-ಪಡೆಯೂ
ಬೇಡರ-ಹಳ್ಳಿ
ಬೇಡರ-ಹಳ್ಳಿ-ಯನ್ನು
ಬೇಡವ್ವೆ
ಬೇಡವ್ವೆಯ
ಬೇಡ-ಹರಳ್ಳಿ
ಬೇಡಿ
ಬೇಡಿಕೆ
ಬೇಡಿ-ಕೆಗೆ
ಬೇಡಿ-ಕೊಂಡು
ಬೇಡಿ-ಕೊಳ್ಳಿ-ಮೆನೆ
ಬೇಡಿ-ಕೊಳ್ಳೆನೆ
ಬೇಡಿ-ಕೊಳ್ಳೆನ್ದೊಡೆ
ಬೇಡಿಕೋ
ಬೇಡಿ-ಗನ-ಹಳ್ಳಿಯ
ಬೇಡಿ-ಗ-ಪಲ್ಲಿಯ
ಬೇಡಿಗೆ
ಬೇಡಿ-ಗೆ-ಒಂದು
ಬೇಡಿ-ಗೆಕೆ
ಬೇಡಿ-ಗೆ-ಗ-ಳನ್ನು
ಬೇಡಿ-ಗೆ-ಗಿಹ
ಬೇಡಿ-ದರ್ತ್ಥಿ-ಜನ-ಕನ್ನಿಚ್ಛ
ಬೇಡಿ-ಪಡೆದ
ಬೇಡಿ-ಪಡೆ-ದನು
ಬೇಡಿ-ಪಡೆದು
ಬೇಡು
ಬೇಡೆ
ಬೇಬಿ
ಬೇಬಿ-ಬೆಟ್ಟ
ಬೇಬಿ-ಬೆಟ್ಟದ
ಬೇರಂಬಾಡಿ
ಬೇರಾರು
ಬೇರಾವ
ಬೇರಿ
ಬೇರೂರಿ
ಬೇರೂರಿತ್ತೆಂದು
ಬೇರೆ
ಬೇರೆ-ಕಡೆ
ಬೇರೆ-ಕಡೆ-ಯಿಂದ
ಬೇರೆ-ಬೇರ-ಯಾ-ಗಿಯೇ
ಬೇರೆ-ಬೇರೆ
ಬೇರೆ-ಬೇರೆ-ಯಾಗಿ
ಬೇರೆ-ಬೇರೆ-ಯಾ-ಗಿಯೇ
ಬೇರೆ-ಬೇರೆಯೇ
ಬೇರೆಯ
ಬೇರೆ-ಯದೇ
ಬೇರೆ-ಯ-ವರ
ಬೇರೆ-ಯ-ವ-ರಿಗೆ
ಬೇರೆ-ಯಾಗಿ
ಬೇರೆ-ಯಾ-ಗಿಯೇ
ಬೇರೆ-ಯಾದ
ಬೇರೆಯೇ
ಬೇರೆಲ್ಲೋ
ಬೇರೆ-ವೇರೂರಿ
ಬೇರೊಂದು
ಬೇರ್ಪ-ಡಿಸಿ
ಬೇಲದ
ಬೇಲೂರ
ಬೇಲೂ-ರನ್ನು
ಬೇಲೂ-ರಿಗೂ
ಬೇಲೂ-ರಿಗೆ
ಬೇಲೂರಿನ
ಬೇಲೂರಿನಲ್ಲಿ
ಬೇಲೂರಿನಲ್ಲೇ
ಬೇಲೂರು
ಬೇಲೂರು-ಬೆಳ್ಳೂರು
ಬೇಲೆ-ಕೆರೆ
ಬೇಲೆ-ಕೆರೆಗೆ
ಬೇಲೆ-ಕೆರೆ-ಯನ್ನು
ಬೇಳ-ವಡಿಚ
ಬೇಳೆ
ಬೇಳೆ-ವೊಂದು
ಬೇಳ್ಪಡೆ-ಇಚ
ಬೇಳ್ವಡಿಚ
ಬೇಳ್ವನಂ
ಬೇವಿನ-ಕಲ್ಲಿನ
ಬೇವಿನ-ಕುಪ್ಪೆ
ಬೇವಿನ-ಕುಪ್ಪೆಯ
ಬೇವಿನ-ಮರದ
ಬೇವಿನ-ಮರ-ದ-ಕಟ್ಟೆ
ಬೇವು-ಕಲ್ಲು
ಬೇಸರ-ವಾಗುತ್ತದೆ
ಬೇಸರಿ-ಸಿ-ದವ-ರಲ್ಲ
ಬೇಸಾಯ
ಬೇಸಾ-ಯಕ್ಕೆ
ಬೇಸಾಯ-ವನ್ನು
ಬೇಹಾರಿ
ಬೇಹಾರಿ-ಅಧಿ-ಕಾರಿ-ರಾಜ-ವರ್ತಕ
ಬೇಹಾರಿ-ಗ-ಳಾಗಿದ್ದ
ಬೈಂಡ್
ಬೈಚ
ಬೈಚಂಣನು
ಬೈಚಣ್ಣ
ಬೈಚ-ದಂಡಾಧೀಶಂ
ಬೈಚ-ದಂಣಾ-ಯಕ
ಬೈಚ-ದಂಣಾ-ಯಕರ
ಬೈಚಪ್ಪ
ಬೈಚವ್ವೆಯ
ಬೈಚೆ-ದಂಡೇಶ-ನಿಗೆ
ಬೈಚೆಯ
ಬೈಚೆಯ-ಬೀಚೆಯ-ಬೈಚ
ಬೈತ್ರಂ
ಬೈರಮೇಶ್ವರ
ಬೈರವ-ಪುರ-ವೆಂಬ
ಬೈರಾ-ಪುರದ
ಬೊಂಬೆ-ಗಳ
ಬೊಂಬೆ-ಗ-ಳನ್ನು
ಬೊಂಮಂಣ
ಬೊಂಮ-ಗವುಡ
ಬೊಂಮ-ಣನ
ಬೊಂಮ-ದೇವ
ಬೊಂಮೋಜ-ನೊಳಗಾದ
ಬೊಕಬಿಲ್ಲಗಾಟಿಯ
ಬೊಕಸಕೆ
ಬೊಕ್ಕಸ
ಬೊಕ್ಕಸಕ್ಕೆ
ಬೊಕ್ಕ-ಸದ
ಬೊಜ್ಜೆ-ಗೆರೆಯ
ಬೊಟ್ಟೈಯ್ಯ
ಬೊದ್ದಿ-ಸೆಟ್ಟಿ
ಬೊಪ್ಪ
ಬೊಪ್ಪ-ಗೌಡನ
ಬೊಪ್ಪ-ಗೌಡ-ನ-ಪುರ
ಬೊಪ್ಪಣ್ಣ
ಬೊಪ್ಪಣ್ಣ-ಪಂಡಿ-ತನ
ಬೊಪ್ಪ-ದಂಡಾಧೀಶ
ಬೊಪ್ಪ-ದೇವ
ಬೊಪ್ಪ-ದೇವನ
ಬೊಪ್ಪ-ದೇವನು
ಬೊಪ್ಪ-ನ-ಹಳ್ಳಿ
ಬೊಪ್ಪ-ನ-ಹಳ್ಳಿಗೆ
ಬೊಪ್ಪನು
ಬೊಪ್ಪ-ನೆಂಬ
ಬೊಪ್ಪ-ಸಂದ್ರ
ಬೊಪ್ಪ-ಸ-ಮುದ್ರ
ಬೊಪ್ಪ-ಸ-ಮುದ್ರ-ಗಳು
ಬೊಪ್ಪ-ಸ-ಮುದ್ರದ
ಬೊಪ್ಪ-ಸ-ಮುದ್ರ-ದಲ್ಲಿ
ಬೊಪ್ಪ-ಸ-ಮುದ್ರ-ದಲ್ಲಿದೆ
ಬೊಪ್ಪ-ಸ-ಮುದ್ರ-ವನ್ನು
ಬೊಪ್ಪ-ಸ-ಮುದ್ರವು
ಬೊಪ್ಪಾ-ದೇವಿ-ಯ-ರನ್ನು
ಬೊಪ್ಪಾ-ದೇವಿ-ಯ-ರಿನ್ತೀ
ಬೊಪ್ಪಾ-ದೇವಿ-ಯೆಂಬ
ಬೊಮನ-ಹಳ್ಳಿ
ಬೊಮ್ಮ-ಗವುಂಡಿ
ಬೊಮ್ಮಣ್ಣ
ಬೊಮ್ಮಣ್ಣ-ಗಳಿಗೂ
ಬೊಮ್ಮಣ್ಣ-ಗ-ಳಿಗೆ
ಬೊಮ್ಮಣ್ಣ-ಗಳು
ಬೊಮ್ಮಣ್ಣ-ಗಳೇ
ಬೊಮ್ಮಣ್ಣನ
ಬೊಮ್ಮಣ್ಣ-ನನ್ನು
ಬೊಮ್ಮಣ್ಣ-ನಿಗೇ
ಬೊಮ್ಮಣ್ಣನು
ಬೊಮ್ಮಣ್ಣನೂ
ಬೊಮ್ಮಣ್ಣ-ನೆಂಬು-ವ-ವನು
ಬೊಮ್ಮನ-ಹಳ್ಳಿ
ಬೊಮ್ಮನ-ಹಳ್ಳಿ-ಗಳು
ಬೊಮ್ಮನ-ಹಳ್ಳಿಯ
ಬೊಮ್ಮನ-ಹಳ್ಳಿ-ಯನ್ನೂ
ಬೊಮ್ಮ-ನಾಯ-ಕ-ನ-ಹಳ್ಳಿ
ಬೊಮ್ಮ-ನಾಯ-ಕ-ನ-ಹಳ್ಳಿ-ಯನ್ನು
ಬೊಮ್ಮ-ನಾಯ-ಕ-ನ-ಹಳ್ಳಿ-ಯಲ್ಲೂ
ಬೊಮ್ಮ-ರ-ಸನ-ಕೊಪ್ಪಲು
ಬೊಮ್ಮವ್ವೆ
ಬೊಮ್ಮವ್ವೆಯು
ಬೊಮ್ಮಿಯ-ರ-ಕೆರೆಯ
ಬೊಮ್ಮಿ-ಸೆಟ್ಟಿ
ಬೊಮ್ಮೂರು
ಬೊಮ್ಮೆಯ-ನ-ಹಳ್ಳಿ-ಯನ್ನು
ಬೊಮ್ಮೇನ-ಹಳ್ಳಿ
ಬೊಯ್ಸಿ
ಬೊಯ್ಸಿ-ಕಟ್ಟೆ-ಯನ್ನು
ಬೊರಹ
ಬೊಲರ
ಬೊಲರ-ಕುಲ
ಬೊಲ್ಲನ
ಬೋಕಂಣ
ಬೋಕಂಣನು
ಬೋಕಣ
ಬೋಕಣಂ
ಬೋಕಣ್ಣ
ಬೋಕಣ್ಣನು
ಬೋಕಣ್ಣ-ರಲ್ಲದೆ
ಬೋಕ-ನಿಗೆ
ಬೋಕಬ್ಬೆಯ
ಬೋಕಿ
ಬೋಕಿ-ಮಯ್ಯನು
ಬೋಕಿ-ಸೆಟ್ಟಿ
ಬೋಕಿ-ಸೆಟ್ಟಿಯು
ಬೋಗ-ನ-ಹಳ್ಳಿ-ಯನ್ನು
ಬೋಗ-ವ-ದಿಯ
ಬೋಗಾದಿ
ಬೋಗಾದಿಯ
ಬೋಗಾಧಿ
ಬೋಗೆಯ
ಬೋಗೇ-ಗೌಡ
ಬೋಗೇ-ಗೌಡನು
ಬೋಗೈಯ
ಬೋಗೈಯ್ಯ
ಬೋಗೋ-ಗೌಡನ
ಬೋಟಕಾ-ಚಾರ್ಯನ
ಬೋಟಕಾ-ಚಾರ್ಯ-ಹೊನ್ನಾ-ಚಾರ್ಯ-ಹರೋಜ
ಬೋಟಕಾ-ಚಾರ್ಯ್ಯನ
ಬೋಧನಾ
ಬೋಧನೆ
ಬೋಧಾ-ಯನ
ಬೋಧಿಸುತ್ತಿದ್ದ
ಬೋಧಿಸುತ್ತಿದ್ದರು
ಬೋಯಿ-ಗನ
ಬೋಯೆ-ಗನು
ಬೋರ
ಬೋರಪ್ಪ
ಬೋರ-ಯನ-ಹಳ್ಳಿ
ಬೋರೇ-ಗೌಡ
ಬೋರೇ-ದೇವರ
ಬೋರೇ-ದೇವ-ರಿಗೆ-ಭೈ-ರವ-ದೇವರು
ಬೋಳ-ಚಾಮ-ರಾಜ
ಬೋಳ-ಚಾಮ-ರಾಜನ
ಬೋಳನ-ಕಟ್ಟೆ
ಬೋಳು-ವಾರು
ಬೋವ
ಬೋವಂಣನು
ಬೋವರು
ಬೋಸ್ರ-ವರ
ಬೌದ್ಧ-ದರ್ಮ-ವನ್ನು
ಬೌದ್ಧೈರ್ಯ್ಯೋ
ಬ್ಯಾಡ-ರ-ಹಳ್ಳಿ
ಬ್ಯಾರೇಜ್
ಬ್ಯಾಲ-ದ-ಕೆರೆ
ಬ್ರಣವಿಭೂಷಿತ
ಬ್ರಯ
ಬ್ರಹ್ಮ
ಬ್ರಹ್ಮ-ಕುಲ-ದೀಪ-ಕ-ನಪ್ಪ
ಬ್ರಹ್ಮಕ್ಷತ್ರಿಯ
ಬ್ರಹ್ಮ-ಚಾರಿ-ಗ-ಳಾಗಿದ್ದ
ಬ್ರಹ್ಮ-ಚಾರಿ-ಗಳು
ಬ್ರಹ್ಮಣ್ಯ
ಬ್ರಹ್ಮಣ್ಯ-ತೀರ್ಥರ
ಬ್ರಹ್ಮ-ತಂತ್ರ
ಬ್ರಹ್ಮ-ದೇಯ
ಬ್ರಹ್ಮ-ದೇ-ಯಕ್ಕೆ
ಬ್ರಹ್ಮ-ದೇಯ-ಗ-ಳಿಗೆ
ಬ್ರಹ್ಮ-ದೇಯ-ಗಳು
ಬ್ರಹ್ಮ-ದೇಯ-ದಲ್ಲಿ
ಬ್ರಹ್ಮ-ದೇಯ-ಮಿವ
ಬ್ರಹ್ಮ-ದೇಯ-ವನ್ನಾಗಿ
ಬ್ರಹ್ಮ-ದೇಯ-ವನ್ನು
ಬ್ರಹ್ಮ-ದೇಯ-ವಾಗಿ
ಬ್ರಹ್ಮ-ದೇಯ-ವಾಗಿದೆ
ಬ್ರಹ್ಮ-ದೇಯ-ವಾಗಿದ್ದಂತೆ
ಬ್ರಹ್ಮ-ದೇಯ-ವಿರ-ಬ-ಹುದು
ಬ್ರಹ್ಮ-ದೇಯವು
ಬ್ರಹ್ಮ-ದೇಯ-ವೆಂದರೆ
ಬ್ರಹ್ಮ-ದೇಯ-ವೆಂದು
ಬ್ರಹ್ಮ-ದೇವ
ಬ್ರಹ್ಮ-ದೇವರ
ಬ್ರಹ್ಮ-ದೇವ-ರಿಗೆ
ಬ್ರಹ್ಮ-ದೇವರು
ಬ್ರಹ್ಮ-ದೇವ-ರೆಂದು
ಬ್ರಹ್ಮ-ದೇಶದ
ಬ್ರಹ್ಮ-ಧೇಯ-ವಾಗಿ
ಬ್ರಹ್ಮ-ನಿಗೆ
ಬ್ರಹ್ಮ-ಪುರ-ವಾದ
ಬ್ರಹ್ಮ-ಪುರಿ
ಬ್ರಹ್ಮ-ಪುರಿ-ಗಳ
ಬ್ರಹ್ಮ-ಪುರಿ-ಗಳಿಗೂ
ಬ್ರಹ್ಮ-ಪು-ರಿಗೆ
ಬ್ರಹ್ಮ-ಪುರಿ-ಘಟಿಕಾಸ್ಥಾನ
ಬ್ರಹ್ಮ-ಪುರಿ-ಯನ್ನು
ಬ್ರಹ್ಮ-ರಾಶಿ
ಬ್ರಹ್ಮ-ರಾಸಿ
ಬ್ರಹ್ಮ-ರಾಸಿ-ಪಂಡಿ-ತ-ನಿಗೂ
ಬ್ರಹ್ಮ-ರಾ-ಸಿಯು
ಬ್ರಹ್ಮ-ವಿದ್ಯಾ-ಕೌ-ಮುದಿ
ಬ್ರಹ್ಮ-ವಿದ್ಯೆಗಾಸ್ಪದ
ಬ್ರಹ್ಮಾಂಡ
ಬ್ರಹ್ಮಾಂಡ-ನಾಯಕ
ಬ್ರಹ್ಮಾ-ದಾಯ
ಬ್ರಹ್ಮಾ-ದಾಯ-ಗಳ
ಬ್ರಹ್ಮಾ-ದಾಯ-ಗಳು
ಬ್ರಹ್ಮಾ-ದಾಯ-ವನ್ನು
ಬ್ರಹ್ಮೇಶನೇ
ಬ್ರಹ್ಮೇಶ್ವರ
ಬ್ರಹ್ಮೇಶ್ವರನ
ಬ್ರಹ್ಮೇ-ಸನು
ಬ್ರಾಹಣ-ರು-ಗ-ಳಿಗೆ
ಬ್ರಾಹ್ಮಣ
ಬ್ರಾಹ್ಮಣ-ಗಳ
ಬ್ರಾಹ್ಮಣ-ನಿಗೆ
ಬ್ರಾಹ್ಮಣನೂ
ಬ್ರಾಹ್ಮಣರ
ಬ್ರಾಹ್ಮಣ-ರ-ಅಯ್ಯರ್
ಬ್ರಾಹ್ಮಣ-ರನ್ನು
ಬ್ರಾಹ್ಮಣ-ರಲ್ಲಿ
ಬ್ರಾಹ್ಮಣ-ರಾಗಿದ್ದ-ರೆಂದು
ಬ್ರಾಹ್ಮಣ-ರಾದ-ರೆಂದು
ಬ್ರಾಹ್ಮಣ-ರಿಂದ
ಬ್ರಾಹ್ಮಣ-ರಿಂದಲೂ
ಬ್ರಾಹ್ಮಣ-ರಿ-ಗಾಗಿ
ಬ್ರಾಹ್ಮಣ-ರಿಗೂ
ಬ್ರಾಹ್ಮಣ-ರಿಗೆ
ಬ್ರಾಹ್ಮಣ-ರಿದ್ದು
ಬ್ರಾಹ್ಮಣ-ರಿ-ರುವ
ಬ್ರಾಹ್ಮಣ-ರಿಲ್ಲ
ಬ್ರಾಹ್ಮಣರು
ಬ್ರಾಹ್ಮಣ-ರು-ಗ-ಳಿಗೆ
ಬ್ರಾಹ್ಮಣ-ರು-ಗಳು
ಬ್ರಾಹ್ಮಣರೂ
ಬ್ರಾಹ್ಮಣ-ರೆಂದು
ಬ್ರಾಹ್ಮಣರೇ
ಬ್ರಾಹ್ಮಣಿ
ಬ್ರಾಹ್ಮಣಿ-ಕೆ-ಯಿಂದ
ಬ್ರಾಹ್ಮಣೊತ್ತಮ-ರಿಗೆ
ಬ್ರಾಹ್ಮಣೋತ್ತಮ-ರಿಗೆ
ಬ್ರಿಟಿಷರ
ಬ್ರಿಟಿಷ-ರನ್ನು
ಬ್ರಿಟಿಷ-ರಿಗೂ
ಬ್ರಿಟಿಷರು
ಬ್ರಿಟಿಷ್
ಬ್ರಿಟೀಷ್
ಬ್ರಿಡ್ಜ್
ಬ್ರೀಹಿ
ಬ್ರೂಸ್ಫೂಟ್
ಭಂಗಿಕರ
ಭಂಡಾರ
ಭಂಡಾರಕೆ
ಭಂಡಾರಕ್ಕೆ
ಭಂಡಾರ-ಗಳು
ಭಂಡಾರದ
ಭಂಡಾರ-ದಲ್ಲಿ
ಭಂಡಾರ-ದಿಂದ
ಭಂಡಾರದ್ರೋಹ
ಭಂಡಾರ-ಬ-ಸದಿಯ
ಭಂಡಾರ-ವನ್ನು
ಭಂಡಾರ-ವಿದ್ದು
ಭಂಡಾರ-ವೆನಿಪ
ಭಂಡಾರಿ
ಭಂಡಾರಿ-ಗ-ನಾಗಿದ್ದನು
ಭಂಡಾರಿ-ಗ-ಳಾಗಿದ್ದು
ಭಂಡಾರಿ-ಗಳು
ಭಂಡಾರಿ-ಗಳು-ಹಿರಿ-ಯ-ಭಂಡಾರಿ-ಮಾಣಿ-ಕ-ಭಂಡಾರಿ
ಭಂಡಾರಿ-ಗೌಂಡ
ಭಂಡಾರಿ-ಯಾ-ಗಿದ್ದ
ಭಂಡಾರಿ-ಯಾಗಿದ್ದನು
ಭಂಡಾರಿಯು
ಭಂಡಾರಿಯೂ
ಭಂಡಿ
ಭಂಡಿ-ಗಳ
ಭಂಡಿ-ಗ-ಳಿಗೆ
ಭಂಡಿಗೆ
ಭಂಡಿಯ
ಭಂಡಿ-ಯ-ಧರ್ಮಕೆ
ಭಂಡಿ-ಯನು
ಭಂಡಿ-ವಾಳ
ಭಂಢಾ-ರಿಯ
ಭಕ್ತ
ಭಕ್ತಗ್ರಾಮ-ದಲ್ಲಿ
ಭಕ್ತ-ನಾ-ಗಿದ್ದು
ಭಕ್ತನೂ
ಭಕ್ತರ
ಭಕ್ತ-ರಾಗಿ
ಭಕ್ತ-ರಿಂದ
ಭಕ್ತ-ರಿಗೆ
ಭಕ್ತರು
ಭಕ್ತರುಂ
ಭಕ್ತ-ವತ್ಸಲ
ಭಕ್ತಿ
ಭಕ್ತಿಗೆ
ಭಕ್ತಿಬ್ರಜಕ್ಕಾ-ಯುವ-ವನ
ಭಕ್ತಿ-ಯಿಂದ
ಭಕ್ತಿಯು
ಭಕ್ತಿ-ಯುಳ್ಳವ-ನಾಗಿದ್ದ-ನೆಂದು
ಭಕ್ತಿ-ಸೇವೆ-ಯಾಗಿ
ಭಕ್ತೆ
ಭಕ್ತೆ-ಯರ
ಭಕ್ಷಿ
ಭಗ-ವಂತ-ನೊಡನೆ
ಭಗ-ವತಸ್ಸೋಸ್ಯಪ್ರಸಾದೀಕ್ರಿತಃ
ಭಗೀರಥ
ಭಗ್ನ-ಗೊಂಡಿ-ರುವ
ಭಗ್ನ-ವಾಗಿದೆ
ಭಟ-ಭೀಮೆಯ-ನಾಯಕ
ಭಟಾರ
ಭಟಾರಕ
ಭಟಾರ-ಕ-ರ-ನೆಂಬ
ಭಟಾರ-ನೆಂಬ
ಭಟಾ-ರರ
ಭಟಾರ-ರನ್ನು
ಭಟಾರ-ರಿಗೆ
ಭಟಾ-ರರು
ಭಟ್ಟ
ಭಟ್ಟಂಗಿ
ಭಟ್ಟಂಗಿ-ಗ-ಳಾಗಿ
ಭಟ್ಟಂಗಿ-ಗ-ಳೆಂದು
ಭಟ್ಟ-ಗುತ್ತಿ-ಕೆಯ
ಭಟ್ಟ-ಗುತ್ತಿಗೆ
ಭಟ್ಟನು
ಭಟ್ಟ-ನೆಂಬ
ಭಟ್ಟನ್
ಭಟ್ಟಯ್ಯನ
ಭಟ್ಟಯ್ಯನು
ಭಟ್ಟರ
ಭಟ್ಟ-ರತ್ನಾ-ಕರ-ವಾದ
ಭಟ್ಟ-ರತ್ನಾ-ಕರ-ವೆಂಬ
ಭಟ್ಟ-ರನ್ನು
ಭಟ್ಟ-ರ-ಬಾಚಪ್ಪನ
ಭಟ್ಟ-ರ-ಬಾಚಪ್ಪ-ನ-ವರು
ಭಟ್ಟ-ರ-ಬಾಚಪ್ಪ-ರಲ್ಲದೆ
ಭಟ್ಟ-ರ-ಬಾಚಿ-ಯಪ್ಪ
ಭಟ್ಟ-ರ-ಬಾಚಿ-ಯಪ್ಪನ
ಭಟ್ಟ-ರ-ಬಾಚಿ-ಯಪ್ಪ-ನಿಗೆ
ಭಟ್ಟ-ರ-ಬಾಚಿ-ಯಪ್ಪನು
ಭಟ್ಟ-ರ-ಬಾಚಿ-ಯಪ್ಪನೂ
ಭಟ್ಟ-ರಿಗ
ಭಟ್ಟ-ರಿಗೆ
ಭಟ್ಟರು
ಭಟ್ಟರ್
ಭಟ್ಟ-ವೃತ್ತಿ
ಭಟ್ಟಾ-ಕಳಂಕ
ಭಟ್ಟಾರಕ
ಭಟ್ಟಾರಕ-ಕರು
ಭಟ್ಟಾರಕ-ದೇವನ
ಭಟ್ಟಾರ-ಕನ
ಭಟ್ಟಾರಕ-ನನ್ನು
ಭಟ್ಟಾರಕ-ನಾಗಿ-ರ-ಬ-ಹುದು
ಭಟ್ಟಾರ-ಕನು
ಭಟ್ಟಾರಕ-ನೆಂಬ
ಭಟ್ಟಾರ-ಕನ್
ಭಟ್ಟಾರ-ಕರ
ಭಟ್ಟಾರಕ-ರಾದ
ಭಟ್ಟಾರ-ಕರು
ಭಟ್ಟಾರಕ-ರು-ಅಭಯ-ನಂದಿ
ಭಟ್ಟಾರಕ-ರು-ಅರ್ಹ-ನಂದಿ
ಭಟ್ಟಾರಕ-ರೆಂದು
ಭಟ್ಟಾರ-ನೆಂಬ
ಭಟ್ಟೋಪಾಧ್ಯಾ-ಯನ
ಭಟ್ಟೋಪಾಧ್ಯಾ-ಯನಾಗಿದ್ದರೂ
ಭಟ್ಟೋಪಾಧ್ಯಾ-ಯನಿರ-ಬ-ಹುದು
ಭಟ್ಟೋಪಾಧ್ಯಾ-ಯನು
ಭಟ್ಟೋಪಾಧ್ಯಾ-ಯನೆಂದು
ಭಟ್ಟೋಪಾಧ್ಯಾ-ಯರು
ಭಟ್ಟೋಪಾಧ್ಯಾ-ರಿಗೆ
ಭತ್ತ
ಭತ್ತದ
ಭತ್ತ-ದಲ್ಲಿ
ಭತ್ತ-ವನ್ನು
ಭತ್ತ-ಸ-ಲಿಗೆ
ಭತ್ತಾಯ
ಭದ್ರ-ಕಾಳಮ್ಮ
ಭದ್ರ-ಕಾಳಿಯ
ಭದ್ರ-ಕಾಳಿ-ಯಣ್ಣ
ಭದ್ರ-ನ-ಕೊಪ್ಪಲು
ಭದ್ರ-ಪಡಿ-ಸಲು
ಭದ್ರ-ಪಡಿ-ಸಿರ-ಬ-ಹುದು
ಭದ್ರ-ಬಾಹು
ಭದ್ರ-ಬಾಹು-ಗಳಿದ್ದ-ರೆಂದು
ಭದ್ರ-ಬಾಹು-ಭಟಾ-ರರು
ಭದ್ರ-ಬಾಹು-ವಿ-ನೊಡನೆ
ಭದ್ರ-ಬಾಹುಸ್ವಾಮಿ-ಗಳು
ಭದ್ರ-ಬಾಹುಸ್ವಾಮಿಯು
ಭದ್ರ-ಭಾಹು
ಭದ್ರ-ವಾಹು
ಭಯಂಕರ
ಭಯಂಕರ-ನಾಗಿ
ಭಯ-ದಿಂದ
ಭಯಲೋಭದುರ್ಲ್ಲಭಂ
ಭಯಿರಮೇಶ್ವರ
ಭಯಿರಮೇಶ್ವರ-ಪುರ
ಭಯಿರಮೇಶ್ವರ-ಪುರದ
ಭಯಿ-ರವ-ನಿಗೆ-ಭೈ-ರವ-ನಿಗೆ
ಭಯಿರಾ-ಪುರದ
ಭರತ
ಭರ-ತ-ಖಂಡದ
ಭರ-ತ-ಖಂಡದಲ್ಲೆಲ್ಲಾ
ಭರ-ತ-ಚಮೂಪ-ತಿಯ
ಭರ-ತ-ಜೀಯ
ಭರ-ತ-ದಂಡ-ನಾಯ-ಕನು
ಭರ-ತನ
ಭರ-ತನೂ
ಭರ-ತನೇ
ಭರ-ತ-ಪುರ
ಭರ-ತ-ಮಯ್ಯ
ಭರ-ತ-ರನ್ನು
ಭರ-ತ-ರಾಜ
ಭರ-ತಿ-ಮಯ್ಯ
ಭರ-ತಿ-ಮಯ್ಯ-ಗಳು
ಭರ-ತಿ-ಮಯ್ಯನ
ಭರ-ತಿ-ಮಯ್ಯರ
ಭರ-ತಿ-ಮಯ್ಯರು
ಭರ-ತೆ-ಪುರ
ಭರ-ತೆಯ
ಭರ-ತೆಯ-ನಾಯಕ
ಭರ-ತೆಯ-ನಾಯಕಂ
ಭರ-ತೆಯ-ನಾಯ-ಕನು
ಭರ-ತೇಶ-ದಂಡ-ನಾಯ-ಕನ
ಭರ-ತೇಶರ
ಭರ-ತೇಶ್ವರ
ಭರದಿಂ
ಭರಿತಂ
ಭರ್ಜಿ
ಭರ್ತಿ-ಯಾಗಿತ್ತೆಂದು
ಭರ್ತಿ-ಯಾಗಿ-ರುತ್ತಿತ್ತು
ಭರ್ತಿ-ಯಾಗುತ್ತದೆ
ಭರ್ತಿ-ಯಾದಾಗ
ಭವತ್ಪ್ರತಾಪ
ಭವನ-ಗ-ಳನ್ನು
ಭವನ-ದಂತಿದ್ದ
ಭವನ-ದಲ್ಲಿ
ಭವನ-ವೆಂಬ
ಭವ-ಶರ್ಮನ
ಭವ್ಯ
ಭವ್ಯ-ಚಿಂತಾ-ಮಣಿ
ಭವ್ಯ-ಚೂಡಾ-ಮಣಿ
ಭವ್ಯ-ಜನ-ಸಂಕುಳ
ಭವ್ಯ-ವಾಗಿದೆ
ಭವ್ಯ-ವಾ-ಗಿದ್ದು
ಭವ್ಯ-ವಾದ
ಭವ್ಯವೂ
ಭವ್ಯೇಷ್ವ-ಭೀಷ್ಟಂ
ಭಸ್ಮ-ಧಾರಣೆ
ಭಾಂಡಾರಕ್ಕಲ್ಲ
ಭಾಗ
ಭಾಗಕ್ಕೆ
ಭಾಗ-ಗ-ಳನ್ನಾಗಿ
ಭಾಗ-ಗ-ಳನ್ನು
ಭಾಗ-ಗ-ಳಲ್ಲಿ
ಭಾಗ-ಗಳಲ್ಲಿದ್ದ
ಭಾಗ-ಗ-ಳಾಗಿ
ಭಾಗ-ಗ-ಳಾಗಿದ್ದ-ವೆಂದು
ಭಾಗ-ಗ-ಳಿಂದ
ಭಾಗ-ಗಳಿಗೂ
ಭಾಗ-ಗ-ಳಿಗೆ
ಭಾಗ-ಗಳು
ಭಾಗ-ಗಳೂ
ಭಾಗದ
ಭಾಗ-ದಲ್ಲಿ
ಭಾಗ-ದಲ್ಲಿದ್ದ
ಭಾಗ-ದಲ್ಲಿದ್ದ-ನೆಂದು
ಭಾಗ-ದಲ್ಲಿ-ರುವ
ಭಾಗ-ದಲ್ಲೇ
ಭಾಗ-ದ-ವ-ನಾ-ಗಿದ್ದು
ಭಾಗ-ದ-ವರು
ಭಾಗ-ದ-ವ-ಳಾಗಿ-ರ-ಬ-ಹುದು
ಭಾಗ-ದಿಂದ
ಭಾಗ-ಯನ್ನು
ಭಾಗ-ವತ
ಭಾಗ-ವತ-ಧರ್ಮದ
ಭಾಗ-ವತ-ಧರ್ಮವು
ಭಾಗ-ವತ-ಪಂಥ
ಭಾಗ-ವ-ತರು
ಭಾಗ-ವತೋತ್ತಮೆ
ಭಾಗ-ವತೋತ್ತಮೆ-ಯಾದ
ಭಾಗ-ವನ್ನು
ಭಾಗ-ವನ್ನೂ
ಭಾಗ-ವನ್ನೆಲ್ಲಾ
ಭಾಗ-ವಹಿ-ಸದೇ
ಭಾಗ-ವಹಿಸಿ
ಭಾಗ-ವಹಿಸಿದ
ಭಾಗ-ವಹಿಸಿ-ದರು
ಭಾಗ-ವಹಿಸಿದ್ದ
ಭಾಗ-ವಹಿಸಿದ್ದ-ನೆಂದು
ಭಾಗ-ವಹಿಸಿದ್ದ-ರೆಂದು
ಭಾಗ-ವಹಿಸಿದ್ದಾರೆ
ಭಾಗ-ವಹಿಸಿ-ರ-ಬ-ಹುದು
ಭಾಗ-ವಹಿಸಿ-ರುವುದು
ಭಾಗ-ವಹಿಸುತ್ತಿದ್ದ
ಭಾಗ-ವಹಿಸುತ್ತಿದ್ದ-ನೆಂಬು-ದನ್ನು
ಭಾಗ-ವಾಗಿ
ಭಾಗ-ವಾ-ಗಿತ್ತು
ಭಾಗ-ವಾಗಿತ್ತೆಂದು
ಭಾಗ-ವಾಗಿ-ರ-ಬ-ಹುದು
ಭಾಗ-ವಿದ್ದು
ಭಾಗವು
ಭಾಗವೂ
ಭಾಗ-ವೆಂದು
ಭಾಗವೇ
ಭಾಗಶಃ
ಭಾಗಿ-ಯಾಗಿದ್ದ-ರೆಂದು
ಭಾಗಿ-ಯಾಗಿದ್ದಾರೆ
ಭಾಗೆಗೆ
ಭಾಗೆಯ
ಭಾಗೆ-ಯನ್ನು
ಭಾಗೆಯಲು
ಭಾಗೆ-ಯಲ್ಲಿ
ಭಾಗೆಯು
ಭಾಗೆ-ಯೊಳಗೆ
ಭಾಗ್ಯ-ವಾಗಲಿ
ಭಾನು-ಕೀರ್ತಿ
ಭಾನು-ಕೀರ್ತಿ-ದೇವನು
ಭಾನು-ಕೀರ್ತಿ-ದೇವರ
ಭಾನು-ಕೀರ್ತಿಯ
ಭಾನು-ಕೀರ್ತಿ-ಸಿದ್ಧಾಂತ
ಭಾನುಕೀರ್ತ್ತಿ
ಭಾನು-ಚಂದ್ರ-ಸಿದ್ಧಾಂತ
ಭಾನು-ಮತಿ-ಯ-ವರು
ಭಾನು-ಮತಿ-ಯ-ವರೂ
ಭಾರ
ಭಾರತ
ಭಾರ-ತದ
ಭಾರ-ತ-ದಲ್ಲಿ
ಭಾರ-ತಾದಿ
ಭಾರ-ತಿಯು
ಭಾರ-ತೀ-ಪುರ
ಭಾರ-ತೀ-ಪುರದ
ಭಾರ-ತೀಯ
ಭಾರ-ತೀ-ಯರೂ
ಭಾರದ್ವಾಜ
ಭಾರದ್ವಾಜ-ಗೋತ್ರ
ಭಾರದ್ವಾಜ-ಗೋತ್ರದ
ಭಾರದ್ವಾಜ-ಗೋತ್ರ-ದ-ವನು
ಭಾರೀ
ಭಾರೀ-ಮಳೆ
ಭಾರ್ಯೆ
ಭಾಳ-ಚಂದ್ರ
ಭಾಳ-ಚಂದ್ರ-ದೇವನ
ಭಾವ
ಭಾವ-ನಾಮ
ಭಾವನೆ
ಭಾವ-ನೆ-ಗಿಂತ
ಭಾವ-ನೆ-ಯನ್ನು
ಭಾವ-ಮೈದ
ಭಾವ-ಮೈದುನ
ಭಾವ-ವಿದೆ
ಭಾವಾದ್ವೈತ
ಭಾವಾವೇಶ-ಗ-ಳನ್ನು
ಭಾವಿ-ಸ-ಬಹು-ದಾಗಿದೆ
ಭಾವಿ-ಸ-ಬ-ಹುದು
ಭಾವಿ-ಸಬೇಕಾಗುತ್ತದೆ
ಭಾವಿ-ಸಲು
ಭಾವಿಸಿ
ಭಾವಿ-ಸಿದ
ಭಾವಿ-ಸಿದ್ದಾರೆ
ಭಾವಿ-ಸಿ-ರುವಂತಿದೆ
ಭಾವಿ-ಸುತ್ತಾರೆ
ಭಾವಿ-ಸು-ವಂತೆ
ಭಾಷಣ-ದಲ್ಲಿ
ಭಾಷಾಪ್ರ-ಯೋಗ
ಭಾಷಿಕ
ಭಾಷೆ
ಭಾಷೆ-ಗಳ
ಭಾಷೆ-ಗ-ಳಲ್ಲಿ
ಭಾಷೆ-ಗಳಲ್ಲಿ-ರು-ವು-ದನ್ನು
ಭಾಷೆ-ಗಳೆರಡ-ರಲ್ಲೂ
ಭಾಷೆಗೆ
ಭಾಷೆ-ಗೆ-ತಪ್ಪುವ
ಭಾಷೆಯ
ಭಾಷೆ-ಯನ್ನು
ಭಾಷೆ-ಯನ್ನೇ
ಭಾಷೆ-ಯಲ್ಲಿ
ಭಾಷೆ-ಯಲ್ಲಿವೆ
ಭಾಷೆ-ಯವು
ಭಾಷೆ-ಯಿಂದ
ಭಾಷ್ಯ
ಭಾಷ್ಯ-ಕಾರರ
ಭಾಷ್ಯ-ಕಾರ-ರಾದ
ಭಾಷ್ಯ-ಕಾರ-ರಿಗೆ
ಭಾಷ್ಯ-ಕಾರರು
ಭಾಷ್ಯ-ಕಾರ-ರೆಂದೂ
ಭಾಸೆ-ಯನ್ನು
ಭಾಸ್ಕರ
ಭಾಸ್ವದ್ಬೃಹ
ಭಿತ್ತಿ
ಭಿತ್ತಿಯ
ಭಿತ್ತಿ-ಯನ್ನು
ಭಿನ್ನ
ಭಿನ್ನನು
ಭಿನ್ನ-ನೆಂದು
ಭಿನ್ನರು
ಭಿನ್ನ-ರೆಂದು
ಭಿನ್ನ-ವಾ-ಗಿತ್ತು
ಭಿನ್ನ-ವಾಗಿದೆ
ಭಿನ್ನ-ವಾದ
ಭಿನ್ನಾಭಿಪ್ರಾಯ-ಗಳು
ಭಿನ್ನಾಭಿಪ್ರಾಯ-ವನ್ನು
ಭಿಲ್ಲಮ-ನಿಗೂ
ಭಿಳಿ-ಗಿರಿ-ವೈಯ್ಯಂಗಾರರ
ಭೀಕರ-ತೆ-ಯನ್ನು
ಭೀಕರ-ವಾದ
ಭೀತ-ರಾಗಿ
ಭೀತಿ
ಭೀಮ
ಭೀಮ-ಗಾಮುಣ್ಡರು
ಭೀಮಣ್ಣ
ಭೀಮಣ್ಣನು
ಭೀಮ-ದೇವ
ಭೀಮ-ದೇವನ
ಭೀಮ-ನ-ಕಂಡಿ
ಭೀಮ-ನ-ಕಂಡಿ-ಬೆಟ್ಟ
ಭೀಮ-ನ-ಕೆರೆ
ಭೀಮ-ನ-ಕೆರೆಗೆ
ಭೀಮ-ನ-ಹಳ್ಳಿ
ಭೀಮ-ನ-ಹಳ್ಳಿಯ
ಭೀಮ-ನಾಥಂಗೆ
ಭೀಮನು
ಭೀಮ-ಪುರಾಣ-ದಲ್ಲಿ
ಭೀಮ-ರಾಯ
ಭೀಮ-ರಾಯನು
ಭೀಮಾರ್ಜು-ನರು
ಭೀಮೆಯ
ಭೀಮೆಯ-ನಾಯಕ
ಭೀಮೆಯ-ನಾಯ-ಕ-ನಾಗಿ-ರುವ
ಭೀಮೆಯ-ನಾಯ-ಕನು
ಭೀಮೆಯ-ನಾಯ-ಕ-ನೊಡನೆ
ಭೀಮೇಶ್ವರ
ಭೀಮೇಶ್ವರಿ
ಭೀರಂ
ಭೀಷಣಂ
ಭೀಷ್ಮ-ಪರ್ವ-ದಲ್ಲಿ
ಭುಕ್ತಿ
ಭುಜ-ಗಳ
ಭುಜದ-ಬ-ಲದೆ
ಭುಜ-ದೊಳ್
ಭುಜಪ್ರತಾಪದಿ
ಭುಜ-ಬಲ
ಭುಜ-ಬಲಪ್ರತಾಪ
ಭುಜ-ಬಲ-ರಾಯ-ನೆಂಬ
ಭುಜ-ಬಲ-ವೀರ-ಗಂಗ
ಭುಜ-ಬಲಿ
ಭುಜ-ಬಲಿ-ಚರಿತೆ-ಯೆಂಬ
ಭುಜ-ಬಳ
ಭುಜ-ಬಳಂ
ಭುಜ-ಬಳ-ವೀರ-ಗಂಗ
ಭುಜ-ಬಳಾವಷ್ಟಂಭ
ಭುಜವ-ನೇ-ರಲು
ಭುಜ-ವಿಜಯ
ಭುಜ-ಸಾಹ-ಸದಿಂ
ಭುಜಾ-ದಂಡ
ಭುಜಾ-ದಂಡ-ವೆನಿಸಿದ್ದ
ಭುವನ
ಭುವನ-ದೊಳಾಂತು
ಭುವನಮ
ಭುವನವೇ
ಭುವನಾ-ರತ್ನಂ
ಭುವನೇಶ್ವರಿ
ಭುವನೇಶ್ವರಿರ
ಭುವನೈಕ-ವೀರ-ನೆಂಬ
ಭುವನೈಕ-ವೀರನ್
ಭುವಾಂ
ಭುವಿ
ಭೂ
ಭೂಕಯಿ-ಲಾಸವೆನಿ-ಸುವ
ಭೂಕಾಮಿನಿ-ಯಿರ್ದ್ದಳಾ
ಭೂಕೈಲಾಸಕ್ಕೆ
ಭೂಗೋಳ
ಭೂಗೋಳ-ವನ್ನು
ಭೂಗೋಳ-ಶಾಸ್ತ್ರ-ವಾ-ಗಿದ್ದು
ಭೂಚಕ್ರ-ದೊಳ್ಗಂಡಪೆಂ
ಭೂಚಕ್ರ-ವ-ಲಯ
ಭೂತ-ಗ-ಣಕ್ಕೆ
ಭೂತ-ಪಿಸಾಚ-ಗಣಂಗಳುಂ
ಭೂತಾ-ನಾಮ
ಭೂತೆ-ರಿಗೆ
ಭೂದಸವಂದ
ಭೂದಸವಂದದ
ಭೂದಾನ
ಭೂದಾನಗ್ರಾಮ-ಧರ್ಮ-ಸಾ-ಧನ-ವಾಗಿ
ಭೂದಾನವು
ಭೂದೇವತಾ
ಭೂದೇವಿಯ
ಭೂದೇವಿ-ಯರ
ಭೂದೇವಿ-ಯೊರಡನೆ
ಭೂನ್ರಿಪಂ
ಭೂಪ
ಭೂಪತಿ
ಭೂಪತಿಃ
ಭೂಪ-ತಿಕ್ರಮಿ-ತರ
ಭೂಪ-ತಿಯ
ಭೂಪ-ತಿಯು
ಭೂಪನಾ
ಭೂಪ-ರಿಮಿತೇ
ಭೂಪಸ್ಥಾನ-ಚಿರಂಜಿತ
ಭೂಪಸ್ಥಾನ-ರಂಜಿತೇ
ಭೂಪಸ್ಯ
ಭೂಪಾಲ
ಭೂಪಾಲಂ
ಭೂಪಾಲ-ಚಿರ-ಪುಣ್ಯ
ಭೂಪಾಲನ
ಭೂಪಾಲನು
ಭೂಪಾಲರು
ಭೂಪಾಳ
ಭೂಬುಜಂ
ಭೂಭಾಗದ
ಭೂಭಾಗವು
ಭೂಭಾರ-ವನ್ನು
ಭೂಭುಜಂ
ಭೂಭು-ವನಂ
ಭೂಭೂಜಿ
ಭೂಭ್ಭ್ರುನಿಳಚಿಅು-ಮಣಿಪ್ರ-ದೀಪ-ಕಳ-ಸ-ನುಂಮಾ-ಗುತ್ತಿರ್ದ್ದಡೆ
ಭೂಭ್ರುನ್ನಿಳಯ
ಭೂಮಂಡ-ಳಮಂ
ಭೂಮಿ
ಭೂಮಿ-ಕಾಯೈ
ಭೂಮಿ-ಕಾರ
ಭೂಮಿ-ಕಾರ-ನಾದ
ಭೂಮಿ-ಗಳ
ಭೂಮಿ-ಗ-ಳನ್ನು
ಭೂಮಿ-ಗ-ಳಿಗೆ
ಭೂಮಿ-ಗಳು
ಭೂಮಿಗೆ
ಭೂಮಿ-ಚಿಅುನ್ನು
ಭೂಮಿ-ದಾನ
ಭೂಮಿ-ದಾನ-ವನ್ನೂ
ಭೂಮಿಪಃ
ಭೂಮಿ-ಪನ
ಭೂಮಿ-ಭಾಗ-ದೊ-ಳದನ್ಯ-ರದೇಕೆ
ಭೂಮಿಯ
ಭೂಮಿ-ಯಂತೆ
ಭೂಮಿ-ಯನ್ನು
ಭೂಮಿ-ಯನ್ನು-ಗದ್ದೆ
ಭೂಮಿ-ಯನ್ನೂ
ಭೂಮಿ-ಯ-ರೂಪ-ದಲ್ಲಿ
ಭೂಮಿ-ಯಲ್ಲಿ
ಭೂಮಿ-ಯಲ್ಲಿದ್ದ
ಭೂಮಿ-ಯಾ-ಗಿತ್ತು
ಭೂಮಿ-ಯಾಗಿತ್ತೆಂದು
ಭೂಮಿ-ಯಾಗಿದೆ
ಭೂಮಿ-ಯಾಗಿ-ರ-ಬ-ಹುದು
ಭೂಮಿ-ಯಿಂದ
ಭೂಮಿಯು
ಭೂಮಿಯೂ
ಭೂಮಿಯೇ
ಭೂಮ್ಯೇ
ಭೂರಿ
ಭೂಲೋಕ
ಭೂಲೋಕ-ವ-ಯಿಕುಂಠ
ಭೂವರಾಹ-ನಾಥ
ಭೂವಲ್ಲ-ಭ-ನಿಗೆ-ಬೂತುಗ
ಭೂವಿಕ್ರಮ-ನನ್ನು
ಭೂವಿಕ್ರಮನು
ಭೂವಿ-ವಾದ
ಭೂವಿಸ್ತಾರದ
ಭೂವೈಕುಂಠ
ಭೂವೈಜ್ಞಾನಿಕ
ಭೂವ್ಯವ-ಹಾರಕ್ಕೆ
ಭೂಶಿರ-ದ-ವರೆ-ಗಿನ
ಭೂಷಣ-ನಾಗಿದ್ದ-ನೆಂದು
ಭೂಷಣರು
ಭೂಷಾ
ಭೂಷಾಯ
ಭೂಷಿತಂ
ಭೂಸುರಾಣಾಂ
ಭೂಸ್ತುತ್ಯ
ಭೂಸ್ತುತ್ಯಂ
ಭೂಸ್ತುತ್ಯ-ನಾದ
ಭೂಸ್ವಾಧೀನ
ಭೂಸ್ವಾಧೀನಕ್ಕೆ
ಭೃಂಗಕುಂತಳೆ
ಭೃಂಗರುಂ
ಭೃಂಗಿ
ಭೃಂಗಿ-ನಾಥ
ಭೃಂಗೀ-ನಾಥ
ಭೃಂಗ್ಗಿ-ನಾಥ
ಭೆವಹ-ರಕ್ಕೆ
ಭೇಟಿ
ಭೇಟಿಗೆ
ಭೇಟಿ-ನೀಡಿದ್ದ-ನೆಂದು
ಭೇಟಿ-ನೀಡಿ-ರ-ಬ-ಹುದು
ಭೇಟಿ-ನೀಡಿ-ರುತ್ತಾನೆ
ಭೇಟಿ-ನೀಡಿ-ರು-ವಂತೆ
ಭೇಟಿ-ಯಾಗಿ-ರ-ಬ-ಹುದು
ಭೇಟಿ-ಯಾಗಿ-ರ-ಬಹು-ದೆಂದು
ಭೇಟಿ-ಯಾಗುತ್ತಾ-ರೆಂದರೆ
ಭೇದ
ಭೇದಂ
ಭೇದ-ಗಳು
ಭೇದ-ವಾದ
ಭೇದ-ವಿಲ್ಲ-ವೆಂದು
ಭೇದಾ-ಭೇದ
ಭೇದಿಸಿ
ಭೇದಿ-ಸಿದ
ಭೇಧ-ವನ್ನೆಣಿ-ಸದ
ಭೇರಿ
ಭೇರಿ-ಪಂಚ
ಭೇರುಂಡ-ವರ್ಗದ
ಭೇರುಂಡ-ವರ್ಗ-ವನ್ನು
ಭೈತ್ರ
ಭೈರ-ಕಂಬೆಯ
ಭೈರಪ್ಪ
ಭೈರಮೇಶ್ವರ
ಭೈರವ
ಭೈರವ-ದಂಣಾಯ-ಕಿತ್ತಿ-ಯರ
ಭೈರವ-ದೇವ
ಭೈರವ-ದೇವನ
ಭೈರವ-ದೇವರ
ಭೈರವ-ದೇವರು
ಭೈರ-ವನ
ಭೈರವ-ನ-ಮೂರ್ತಿಯು
ಭೈರವ-ನಿಗೆ
ಭೈರ-ವನು
ಭೈರ-ವನೇ
ಭೈರವ-ಪುರದ
ಭೈರವ-ಪುರ-ವೆಂಬ
ಭೈರವ-ಪೂಜೆಯು
ಭೈರವಪ್ರತಿಷ್ಠೆಯ
ಭೈರವ-ರಾಜನ
ಭೈರವಾ-ಪರು-ವೆಂಬ
ಭೈರವಾ-ಪುರ
ಭೈರವಾ-ಪುರ-ವಾದ
ಭೈರವಾ-ಪುರ-ವೆಂಬ
ಭೈರವೇಶ್ವರ
ಭೈರವೇಶ್ವರನ
ಭೈರವ್ವೆ
ಭೈರ-ಶೆಟ್ಟಿ
ಭೈರಾ-ಪುರ
ಭೈರಾ-ಪುರದ
ಭೈರಾ-ಪುರ-ದಲ್ಲಿ-ರುವ
ಭೈರಾ-ಪುರ-ವೆಂಬ
ಭೈರೆಯ-ನಾಯ-ಕನ
ಭೈರೇ-ಗೌಡ
ಭೈರೇ-ದೇವರ
ಭೈಷಜ್ಯ
ಭೊಗಯ್ಯ-ದೇವ
ಭೋಕ್ತೃ-ವಿನ
ಭೋಗ
ಭೋಗ-ದವ-ರಿಗೆ
ಭೋಗ-ನ-ಹಳ್ಳಿ
ಭೋಗ-ನಾಥ
ಭೋಗ-ನಿಧಿ
ಭೋಗ-ಭಾಗಿನಿ
ಭೋಗ-ಭಾಮಿನಿ
ಭೋಗಯ್ಯ
ಭೋಗಯ್ಯ-ದೇವ
ಭೋಗ-ರಾಜ
ಭೋಗ-ರಾಜ-ಭೂ-ಪಾಲನು
ಭೋಗ-ರಾಜ-ವರ-ತಲ್ಪಃ
ಭೋಗ-ವತಿ
ಭೋಗ-ವತಿ-ಯಲ್ಲಿ
ಭೋಗ-ವ-ದಿಯ
ಭೋಗ-ವ-ದಿಯ-ಬೋಗಾದಿ
ಭೋಗ-ವ-ಸದಿ-ಯೊಳು
ಭೋಗ-ಸೂದು
ಭೋಗ-ಸೆಟ್ಟಿ
ಭೋಗಾ-ನರ-ಸಿಂಹ
ಭೋಗಾನು-ಭಾವಿ
ಭೋಗೇ-ಗೌಡನ
ಭೋಗೇಶ್ವರ
ಭೋಗೈಯ್ಯ
ಭೋಜಃ
ಭೋಜನ
ಭೋಜನಕ್ಕಾಗಿ
ಭೋಜನಕ್ಕೆ
ಭೋಜ-ರರು
ಭೋಜ-ರಾಜ-ನಿಗೆ
ಭೌಗೋ-ಳಿಕ
ಭೌಗೋ-ಳಿಕ-ವಾಗಿ
ಭೌತಿಕ
ಭೌಮ-ವಾರ
ಭ್ಯಸ್ತರ್ಗ್ಗಂ
ಭ್ರುಂಗಿ-ನಾಥ
ಮ
ಮಂ
ಮಂಗಣ್ಣ
ಮಂಗಪ್ಪ
ಮಂಗಲ
ಮಂಗಲಕ್ಕೆ
ಮಂಗಲದ
ಮಂಗಲ-ದಲ್ಲಿ
ಮಂಗಲ-ದ-ವರ
ಮಂಗಲ-ಬಳ-ಗೊಳ
ಮಂಗಲಮ್
ಮಂಗಲ-ವಾದ
ಮಂಗಲವು
ಮಂಗಲ-ವೆಂದು
ಮಂಗಲ-ವೆಂಬ
ಮಂಗಲಾತ್ಮಿಕಾ
ಮಂಗಳ
ಮಂಗಳಂ
ಮಂಗಳಂತೇ
ಮಂಗಳ-ಮಹಾಶ್ರೀ
ಮಂಗಳ-ಮೂರ್ತ್ತಿ-ಗಂಗೆ
ಮಂಗಳ-ವಾಗಲಿ
ಮಂಗಳ-ವಾರ
ಮಂಗಳಾರ-ತಿಯ
ಮಂಗಳೂರು
ಮಂಗಳೇಶನು
ಮಂಚ-ಗಾವುಂಡನು
ಮಂಚ-ಗೌಂಡ
ಮಂಚ-ಗೌಂಡನ
ಮಂಚ-ಗೌಂಡನು
ಮಂಚ-ಗೌಡ
ಮಂಚ-ಜೀಯರ
ಮಂಚಣ್ಣ
ಮಂಚನ-ಹಳ್ಳಿ
ಮಂಚನ-ಹಳ್ಳಿ-ಕೆರೆ
ಮಂಚನ-ಹಳ್ಳಿಗೆ
ಮಂಚನ-ಹಳ್ಳಿ-ಯನ್ನು
ಮಂಚ-ನಾಯ-ಕನು
ಮಂಚನು
ಮಂಚಯ-ದಂಡ-ನಾಯಕ
ಮಂಚಯ್ಯನ
ಮಂಚಲಾ-ದೇವಿ
ಮಂಚಲಾ-ದೇವಿಯ
ಮಂಚಲೆ
ಮಂಚವ್ವೆ
ಮಂಚಿ
ಮಂಚಿ-ಗೌಡ
ಮಂಚಿ-ತಮ್ಮ
ಮಂಚಿ-ಬೀಡು
ಮಂಚೆ-ಗಾವುಂಡ
ಮಂಚೆ-ಗೌಡ
ಮಂಚೇ-ಗೌಂಡನ
ಮಂಚೇ-ಗೌಡನ
ಮಂಚೋಜ
ಮಂಜ-ಯಪ್ಪ
ಮಂಜಯ್ಯ
ಮಂಜಯ್ಯ-ನನ್ನು
ಮಂಜಯ್ಯನು
ಮಂಜರೀ
ಮಂಜು-ನಾಥ್
ಮಂಟಪ
ಮಂಟಪಕೆ
ಮಂಟ-ಪಕ್ಕೆ
ಮಂಟಪ-ಗ-ಳನ್ನು
ಮಂಟಪ-ಗ-ಳನ್ನೂ
ಮಂಟಪ-ಗ-ಳಾಗಿದ್ದು
ಮಂಟಪ-ಗಳಿತ್ತೆಂದು
ಮಂಟಪ-ಗಳು
ಮಂಟಪದ
ಮಂಟಪ-ದಂತಹ
ಮಂಟಪ-ದಲ್ಲಿ
ಮಂಟಪ-ದಲ್ಲಿದೆ
ಮಂಟಪ-ದಲ್ಲಿ-ರುವ
ಮಂಟಪದಿಂ
ಮಂಟಪ-ದಿಂದ
ಮಂಟಪ-ವನ್ನು
ಮಂಟಪ-ವನ್ನು-ರಂಗ-ಮಂಟಪ
ಮಂಟಪ-ವನ್ನೂ
ಮಂಟಪ-ವಾಗಿದೆ
ಮಂಟಪ-ವಿದೆ
ಮಂಟಪ-ವಿರ-ಬ-ಹುದು
ಮಂಟಪವು
ಮಂಟ-ವನ್ನು
ಮಂಟಿ
ಮಂಟಿ-ಗ-ಳಿಂದ
ಮಂಟಿಗೆ
ಮಂಠಿ
ಮಂಠೆ
ಮಂಠೆದ
ಮಂಠೆಯ
ಮಂಠೆ-ಯದ
ಮಂಠೆ-ಯ-ಮಂಡ್ಯ
ಮಂಠೆ-ಯವು
ಮಂಠೆ-ಯವೇ
ಮಂಠೇದ
ಮಂಠೇ-ದಯ್ಯ
ಮಂಠೇದಯ್ಯ-ನ-ವರು
ಮಂಡ
ಮಂಡ-ಗ-ವುಡನು
ಮಂಡ-ಗೌಡ-ನೆಂಬ
ಮಂಡ-ನಯಾಕ
ಮಂಡ-ನೆ-ಯಲ್ಲಿ
ಮಂಡ-ಮಂಡೆ-ಮಂಡೇವು-ಮಂಡ್ಯ
ಮಂಡಯಂ
ಮಂಡ-ರಿ-ವರ್ಮ-ರಾಜ
ಮಂಡಲ
ಮಂಡ-ಲಕ್ಕೂ
ಮಂಡ-ಲ-ಗ-ಳನ್ನಾಗಿ
ಮಂಡ-ಲ-ಗ-ಳಾಗಿ
ಮಂಡ-ಲ-ಗಳಿದ್ದವು
ಮಂಡ-ಲ-ನವೋ-ಲದಾ-ಚಂದ್ರಾರ್ಕ್ಕಂ
ಮಂಡ-ಲ-ವನ್ನು
ಮಂಡ-ಲ-ವನ್ನೂ
ಮಂಡ-ಲ-ವಿಷಯ-ದೇಶ-ನಾಡು-ಕಂಪಣ
ಮಂಡ-ಲಸ್ವಾಮಿ
ಮಂಡ-ಲಸ್ವಾಮಿಗೆ
ಮಂಡ-ಲಸ್ವಾಮಿಯ
ಮಂಡ-ಲಸ್ವಾಮಿ-ಯನ್ನು
ಮಂಡ-ಲಸ್ವಾಮಿ-ಯಾ-ಗಿದ್ದ
ಮಂಡ-ಲಸ್ವಾಮಿಯು
ಮಂಡ-ಲಾಧಿ-ಪತಿ-ಯನ್ನಾಗಿ
ಮಂಡ-ಲಾಧಿ-ಪತಿ-ಯಾ-ಗಿದ್ದ
ಮಂಡ-ಲಿಕ
ಮಂಡ-ಲಿಕರಂ
ಮಂಡ-ಲಿಕರು
ಮಂಡ-ಲಿ-ಯ-ವರು
ಮಂಡ-ಲೀಕ
ಮಂಡ-ಲೇಶ್ವರ
ಮಂಡ-ಲೇಶ್ವರ-ದೇವರು
ಮಂಡ-ಲೇಶ್ವರ-ರನ್ನು
ಮಂಡ-ಲೇಶ್ವರ-ರಾಗಿ
ಮಂಡ-ಲೇಶ್ವರರು
ಮಂಡ-ಳಿಕ
ಮಂಡ-ಳಿಕ-ಜೂಬು
ಮಂಡ-ಳಿಕ-ನಾಗಿ
ಮಂಡ-ಳಿಕ-ನಾದ
ಮಂಡ-ಳಿ-ಕರು
ಮಂಡ-ಳಿ-ಕಾ-ಚಾರಿ
ಮಂಡ-ಳೀಕ-ಜೂಬು
ಮಂಡ-ಳೀ-ಕರ
ಮಂಡ-ಳೀ-ಕರ-ಗಂಡ
ಮಂಡ-ಳೇಶ್ವರನ
ಮಂಡ-ಳೇಶ್ವರ-ನಾಗಿ
ಮಂಡ-ಳೇಶ್ವರಮಂ
ಮಂಡ-ಳೇಶ್ವರರು
ಮಂಡಸ್ವಾಮಿಗೆ
ಮಂಡಿತ
ಮಂಡಿಸ-ಲಾದ
ಮಂಡಿಸಲ್ಪಟ್ಟ
ಮಂಡಿಸಲ್ಪಟ್ಟಿತು
ಮಂಡಿಸು-ವಂತೆ
ಮಂಡೂರಿನ
ಮಂಡೆಯ
ಮಂಡೆಯಂ
ಮಂಡೆ-ಯದ
ಮಂಡೆವೇಮು
ಮಂಡೇವು
ಮಂಡೇವುಕೆ
ಮಂಡೇವುಕ್ಕೆ
ಮಂಡೇವು-ಮಂಡ್ಯ
ಮಂಡ್ಯ
ಮಂಡ್ಯಂ
ಮಂಡ್ಯಂಣ
ಮಂಡ್ಯ-ಕೊಪ್ಪ-ಲಿ-ನಲ್ಲಿ
ಮಂಡ್ಯಕ್ಕಿಂತಲೂ
ಮಂಡ್ಯಕ್ಕೆ
ಮಂಡ್ಯ-ಗೋ-ಪಣನ
ಮಂಡ್ಯ-ಜಿಲ್ಲೆ
ಮಂಡ್ಯ-ಜಿಲ್ಲೆಯ
ಮಂಡ್ಯ-ಜಿಲ್ಲೆ-ಯಲ್ಲಿ
ಮಂಡ್ಯ-ಜಿಲ್ಲೆಯು
ಮಂಡ್ಯದ
ಮಂಡ್ಯ-ದಲ್ಲಿ
ಮಂಡ್ಯ-ವನ್ನು
ಮಂಡ್ಯವು
ಮಂಡ್ಯವೂ
ಮಂಣ
ಮಂಣಿ-ದಲು
ಮಂತ-ಲಲಾ-ಮನೀ
ಮಂತೃ
ಮಂತ್ರ-ಗಳನ್ನೊಳ-ಗೊಂಡ
ಮಂತ್ರ-ಗಳು
ಮಂತ್ರಚಿನ್ತಾ-ಮಣಿ
ಮಂತ್ರ-ವನ್ನು
ಮಂತ್ರ-ವಾದಿ
ಮಂತ್ರ-ವಿದ್ಯಾ-ವಿಕಾಶಂ
ಮಂತ್ರಿ
ಮಂತ್ರಿ-ಗಳ
ಮಂತ್ರಿ-ಗ-ಳಲ್ಲಿ
ಮಂತ್ರಿ-ಗ-ಳಾಗಿದ್ದ
ಮಂತ್ರಿ-ಗ-ಳಾಗಿದ್ದರು
ಮಂತ್ರಿ-ಗ-ಳಾಗಿದ್ದ-ರೆಂದು
ಮಂತ್ರಿ-ಗ-ಳಾಗಿದ್ದಾಗ
ಮಂತ್ರಿ-ಗ-ಳಾಗಿದ್ದಿರ-ಬ-ಹುದು
ಮಂತ್ರಿ-ಗ-ಳಾಗಿದ್ದು
ಮಂತ್ರಿ-ಗ-ಳಾದ
ಮಂತ್ರಿ-ಗಳು
ಮಂತ್ರಿ-ಗಳೂ
ಮಂತ್ರಿ-ಗ-ಳೆಂದೂ
ಮಂತ್ರಿ-ಚೂಡಾ-ಮಣಿ
ಮಂತ್ರಿ-ಣಾವಭ-ವತಾಂ
ಮಂತ್ರಿಣೇ
ಮಂತ್ರಿ-ತಿಳಕಂ
ಮಂತ್ರಿ-ಪ-ದವಿ-ಯಲ್ಲಿದ್ದಿರ
ಮಂತ್ರಿ-ಪರಿಷತ್ತಿ-ನಲ್ಲಿ
ಮಂತ್ರಿಭಿಃ
ಮಂತ್ರಿ-ಮಂಡಲ
ಮಂತ್ರಿ-ಮಂಡ-ಲದ
ಮಂತ್ರಿ-ಮಾಣಿಕ್ಯ
ಮಂತ್ರಿ-ಮಾಣಿಕ್ಯಂ
ಮಂತ್ರಿ-ಮುಖ-ದರ್ಪಣ
ಮಂತ್ರಿಯ
ಮಂತ್ರಿ-ಯಾಗಿ
ಮಂತ್ರಿ-ಯಾ-ಗಿದ್ದ
ಮಂತ್ರಿ-ಯಾಗಿದ್ದಂತೆ
ಮಂತ್ರಿ-ಯಾಗಿದ್ದನು
ಮಂತ್ರಿ-ಯಾಗಿದ್ದ-ನೆಂದು-ಗೋವಿಂದಯ್ಯಾಖ್ಯ
ಮಂತ್ರಿ-ಯಾಗಿದ್ದು-ದರ
ಮಂತ್ರಿ-ಯಾಗಿ-ರ-ಬ-ಹುದು
ಮಂತ್ರಿ-ಯಾದ
ಮಂತ್ರಿ-ಯಾದಂ
ಮಂತ್ರಿ-ಯಾದ-ನೆಂದು
ಮಂತ್ರಿಯು
ಮಂತ್ರಿಯೂ
ಮಂತ್ರಿ-ಯೂ-ಥಾಗ್ರಣಿ
ಮಂತ್ರಿ-ಯೊಡನೆ
ಮಂತ್ರೀಶ
ಮಂತ್ರೀಶ್ವರ-ನಾ-ದಂತೆ
ಮಂದ-ಗೆರೆ
ಮಂದ-ಗೆರೆಯ
ಮಂದ-ರಿಗೆ-ಯ-ಹಳ್ಳ
ಮಂದಿ
ಮಂದಿರಂ
ಮಂದಿ-ರಲ್ಲಿದೆ
ಮಂದೆಯ
ಮಂನನ
ಮಂನಿತಿ
ಮಂನೆಯ
ಮಂನೆಯ-ಗಜ-ಪತಿ
ಮಂನೆಯ-ಜೂಬು
ಮಂನೆಯ-ನಾಯಕ
ಮಂನೆಯರು
ಮಂನೆ-ಯಾಳು-ತನ-ವನ್ನು
ಮಂಮಂ
ಮಂಶದ
ಮಉಲ್ಯದ
ಮಕರ
ಮಕರಂದ
ಮಕರಧ್ವಜ-ನೊಳ್ಸು-ಶಾಂತಿಯಂ
ಮಕರ-ಮಾಸದ
ಮಕರ-ರಾಜ್ಯ
ಮಕರ-ರಾಯ
ಮಕುಟ-ಮಂಡ-ಲಿಕರ
ಮಕ್ಕಳ
ಮಕ್ಕ-ಳನ್ನು
ಮಕ್ಕಳನ್ನೋ
ಮಕ್ಕಳಾ
ಮಕ್ಕಳಾ-ಗಿದ್ದ
ಮಕ್ಕಳಾ-ಗಿದ್ದ-ರೆಂದು
ಮಕ್ಕಳಾ-ಗಿದ್ದು
ಮಕ್ಕಳಾ-ಗಿರ-ಬ-ಹುದು
ಮಕ್ಕಳಾ-ಗುತ್ತಾರೆ
ಮಕ್ಕಳಾದ
ಮಕ್ಕಳಿಗೂ
ಮಕ್ಕಳಿಗೆ
ಮಕ್ಕಳಿಗೆ-ಹೆ-ಸರಿ-ಸಿದೆ
ಮಕ್ಕಳಿದ್ದರು
ಮಕ್ಕಳಿದ್ದ-ರೆಂದು
ಮಕ್ಕಳಿದ್ದ-ರೆಂದೂ
ಮಕ್ಕಳಿದ್ದು
ಮಕ್ಕಳಿದ್ದುದು
ಮಕ್ಕಳಿಬ್ಬರೂ
ಮಕ್ಕಳಿಲ್ಲದ
ಮಕ್ಕಳಿಲ್ಲದೇ
ಮಕ್ಕಳು
ಮಕ್ಕಳು-ಗಳಿಗೂ
ಮಕ್ಕಳು-ಚಿ-ಸೆಟ್ಟಿಯ
ಮಕ್ಕಳೂ
ಮಕ್ಕ-ಳೆಂದು
ಮಕ್ಕ-ಳೆಂದೂ
ಮಕ್ಕಳೆಂಬುದು
ಮಕ್ಕಳೇ
ಮಕ್ಕಳೊಡನೆ
ಮಕ್ಕಳ್
ಮಕ್ಕಾನ್
ಮಖ್ಖಳು
ಮಖ್ಯಸ್ಥ-ರಾಗಿದ್ದರು
ಮಗ
ಮಗಂ
ಮಗಅ
ಮಗಚಿ
ಮಗ-ಣಿ-ಗ-ವುಡನ
ಮಗ-ದೊಬ್ಬ
ಮಗನ
ಮಗ-ನನ್ನು
ಮಗ-ನನ್ನೂ
ಮಗ-ನಾಗಿ
ಮಗ-ನಾಗಿದ್ದನು
ಮಗ-ನಾಗಿದ್ದರೂ
ಮಗ-ನಾ-ಗಿದ್ದು
ಮಗ-ನಾಗಿ-ರ-ಬ-ಹುದು
ಮಗ-ನಾಗಿ-ರಲು
ಮಗ-ನಾಗಿ-ರುವ
ಮಗ-ನಾಗುತ್ತಾನೆ
ಮಗ-ನಾದ
ಮಗನಿಗ
ಮಗ-ನಿಗೆ
ಮಗನಿದ್ದ
ಮಗನಿದ್ದನು
ಮಗನಿದ್ದ-ನೆಂದು
ಮಗನಿದ್ದ-ನೆಂಬುದು
ಮಗ-ನಿದ್ದು
ಮಗ-ನಿರ-ಬ-ಹುದು
ಮಗ-ನಿರ-ಬಹು-ದೆಂದು
ಮಗ-ನಿರ-ಬೇಕು
ಮಗನೀಶ್ವರಯ್ಯ
ಮಗನು
ಮಗನೂ
ಮಗ-ನೆಂದರೆ
ಮಗ-ನೆಂದು
ಮಗ-ನೆಂದೂ
ಮಗ-ನೆಂಬುದು
ಮಗನೇ
ಮಗನೋ
ಮಗನ್
ಮಗರ
ಮಗ-ರನ
ಮಗ-ರ-ರಾಜ್ಯ
ಮಗ-ರ-ರಾಯ
ಮಗ-ರಾಧಿ-ರಾಯ
ಮಗಳ
ಮಗ-ಳನ್ನು
ಮಗ-ಳನ್ನೂ
ಮಗ-ಳಾಗಿದ್ದು
ಮಗ-ಳಾದ
ಮಗ-ಳಿಗೆ
ಮಗಳಿದ್ದ
ಮಗಳಿದ್ದಳು
ಮಗಳು
ಮಗಳು-ಶಾ-ಸನ
ಮಗಳೂ
ಮಗಳೊಬ್ಬಳು
ಮಗ-ವಂತ-ನಿಗೆ
ಮಗ-ಶಿಷ್ಯ-ನಾದ್ದ-ರಿಂದ
ಮಗು
ಮಗುಚಿ
ಮಗುರ್ಚಿ
ಮಗುರ್ದಡೆ-ರೆಪ್ಪುವ
ಮಗುಳ್ಚಿ
ಮಗ್ಗ
ಮಗ್ಗಕೆ
ಮಗ್ಗಕ್ಕೆ
ಮಗ್ಗಕ್ಕೆ-ಸಲ್ಲುವ
ಮಗ್ಗ-ಗಳ
ಮಗ್ಗ-ಗ-ಳನ್ನು
ಮಗ್ಗ-ಗಳು
ಮಗ್ಗ-ತೆರೆ
ಮಗ್ಗದ
ಮಗ್ಗ-ದಲ್ಲಿ
ಮಗ್ಗ-ದ-ವರು
ಮಗ್ಗ-ದೆರೆ
ಮಗ್ಗ-ದೆರೆಯ
ಮಗ್ಗ-ದೆರೆ-ಯನ್ನು
ಮಗ್ಗ-ದೆರೆ-ಯೊಳಗೆ
ಮಗ್ಗ-ದೆಱೆ
ಮಗ್ಗ-ದೆಱೆಯ
ಮಗ್ಗ-ನ-ಹಳ್ಳಿಯ
ಮಗ್ಗ-ನ-ಹಳ್ಳಿ-ಯನ್ನು
ಮಗ್ಗ-ವಣ
ಮಗ್ಗ-ವನ್ನು
ಮಗ್ಗ-ವನ್ನೇ
ಮಗ್ಗ-ಹಳ್ಳಿಯ
ಮಗ್ಗಾ-ಲಯ
ಮಗ್ನ-ನಾ-ಗಿದ್ದ
ಮಗ್ನ-ನಾದ
ಮಚ್ಚರಿಪ-ನಾಯ-ಕ-ರ-ಗಂಣ್ಡ
ಮಜ್ಜ-ನದ
ಮಜ್ಜಿಗೆ-ಪುರ-ಶಂಕ-ರ-ಪುರ
ಮಟ್ಟದ
ಮಟ್ಟಿಗೂ
ಮಟ್ಟಿಗೆ
ಮಟ್ಟಿ-ಯಮ್ಬಾಕ್ಕಮ್
ಮಠ
ಮಠಕೆ
ಮಠಕ್ಕೆ
ಮಠ-ಗಳ
ಮಠ-ಗ-ಳನ್ನು
ಮಠ-ಗ-ಳಿಗೆ
ಮಠ-ಗಳು
ಮಠ-ಗಳೂ
ಮಠ-ತೆ-ರಿಗೆ
ಮಠದ
ಮಠ-ದ-ಕೇರಿ
ಮಠ-ದಲ್ಲಿ
ಮಠ-ದಲ್ಲಿ-ರುವ
ಮಠ-ದಲ್ಲೇ
ಮಠ-ದ-ವ-ರಿ-ಗಾಗಿ
ಮಠ-ದ-ವ-ರಿಗೆ
ಮಠ-ದಿಂದ
ಮಠ-ಪತಿ
ಮಠ-ಪತಿ-ಗ-ಳಾದ
ಮಠ-ಪತಿ-ದಾಸ-ವೈಷ್ಣ-ವರ
ಮಠ-ಪತ್ತಿ
ಮಠ-ಮಾನ್ಯ-ಗ-ಳನ್ನು
ಮಠ-ಮಾನ್ಯ-ಗ-ಳಿಗೆ
ಮಠ-ವನು
ಮಠ-ವನ್ನು
ಮಠ-ವಾಗಿ
ಮಠ-ವಾಗಿದೆ
ಮಠ-ವಾ-ಗಿದ್ದು
ಮಠ-ವಾಗಿ-ರುವ
ಮಠ-ವಿದೆ
ಮಠವು
ಮಠವೂ
ಮಠವೇ
ಮಠಾಧಿ-ಪತಿ
ಮಠಾಧಿ-ಪತಿ-ಗ-ಳಿಗೆ
ಮಠಾಧಿ-ಪತಿ-ಯಾಗಿದ್ದನು
ಮಠಾಧಿ-ಪತಿಯೋ
ಮಠಾಧೀಶ-ರಿಗೆ
ಮಠಾರ್ಯ
ಮಡಕೆ-ಪಟ್ಟಣ
ಮಡಕೆ-ಹೊಸೂರು
ಮಡ-ವನ-ಕೋಡಿ
ಮಡವಯ
ಮಡಿ
ಮಡಿದ
ಮಡಿ-ದಂತೆ
ಮಡಿ-ದನು
ಮಡಿ-ದ-ನೆಂದು
ಮಡಿ-ದ-ನೆಂಬುದು
ಮಡಿ-ದರು
ಮಡಿ-ದ-ರೆಂದು
ಮಡಿ-ದ-ವನು
ಮಡಿ-ದ-ವರ
ಮಡಿ-ದ-ವ-ರಿ-ಗಾಗಿ
ಮಡಿ-ದಾಗ
ಮಡಿ-ದಿದ್ದು
ಮಡಿ-ದಿರ-ಬ-ಹುದು
ಮಡಿ-ದಿರ-ಬಹು-ದೆಂದು
ಮಡಿ-ದಿರ-ವುದು
ಮಡಿ-ದಿರುವ
ಮಡಿ-ದಿ-ರು-ವಂತೆ
ಮಡಿ-ದಿ-ರುವು-ದ-ರಿಂದ
ಮಡಿದು
ಮಡಿ-ಯನ-ಹಳ್ಳಿಯ
ಮಡಿ-ಯರು
ಮಡಿ-ಯಲು
ಮಡಿ-ಯುತ್ತಾನೆ
ಮಡಿ-ಯುತ್ತಾ-ನೆಂದಿದೆ
ಮಡಿ-ಯುತ್ತಾರೆ
ಮಡಿ-ಯುತ್ತಿದ್ದರು
ಮಡಿ-ಯುತ್ತಿದ್ದ-ರೆಂದು
ಮಡಿ-ಯುವ
ಮಡಿ-ಯು-ವುದು
ಮಡಿ-ವಳ್ಳ
ಮಡಿಸ
ಮಡು
ಮಡು-ಗಳು
ಮಡು-ವನ್ನು
ಮಡು-ವಿನ
ಮಡು-ವಿನ-ಕೋಡಿ
ಮಡು-ವಿನ-ಕೋಡಿಯ
ಮಡು-ವಿ-ನಲ್ಲಿ
ಮಡು-ಹಿನ
ಮಡೆಯ
ಮಣಂ
ಮಣಲ
ಮಣಲ-ಗದೆ-ಯಲು
ಮಣಲ-ಗದ್ದೆ
ಮಣಲ-ಯ-ರನ
ಮಣ-ಲಿಯ
ಮಣಲೆ
ಮಣಲೆ-ಅರ-ಸನು
ಮಣ-ಲೆಯ
ಮಣಲೆ-ಯರ
ಮಣಲೆ-ಯ-ರನ
ಮಣಲೆ-ಯ-ರನು
ಮಣಲೆ-ಯ-ರರು
ಮಣಲೆ-ಯ-ರ-ಸರ
ಮಣಲೆ-ಯ-ರ-ಸರಾ
ಮಣಲೆ-ಯಾರ-ನಿರ-ಬ-ಹುದು
ಮಣಲೆ-ಯಾರನು
ಮಣಲೆರ
ಮಣಲೆ-ರಙ್ಗೆ
ಮಣಲೆ-ರನ
ಮಣಲೆ-ರನು
ಮಣಲೇರ
ಮಣಲೇರನ
ಮಣಲೇರ-ನಿಗೆ
ಮಣಲೇರನು
ಮಣಲೇರ-ಮಣಾಲ-ರನ
ಮಣಲೇರ-ರನ್ನು
ಮಣಲೇರ-ರನ್ನು-ಮರು-ವರ್ಮ
ಮಣಲೇರರು
ಮಣಲೇಶ್ವರ
ಮಣಳೇಶ್ವರ
ಮಣ-ವಾಳ
ಮಣ-ವಾಳನ್
ಮಣ-ವಾಳಯ್ಯಗೆ
ಮಣ-ವಾಳಯ್ಯನೇ
ಮಣ-ವಾಳ್
ಮಣಾಲ-ರನ
ಮಣಾಲ-ರನನ್ನು
ಮಣಾಲ-ರನು
ಮಣಿ
ಮಣಿ-ಕರ್ಣಿಕಾ
ಮಣಿ-ಕೊನಖ-ಗಳ
ಮಣಿ-ಗಣ-ಖಚಿತ
ಮಣಿ-ನಾಗ-ಪುರ
ಮಣಿ-ನಾಗ-ಪುರ-ವ-ರಾಧೀಶ್ವರ
ಮಣಿ-ನಾಗ-ಪುರ-ವ-ರಾಧೀಶ್ವರನೂ
ಮಣಿ-ನಾಗ-ರಗ್ರಾಮ-ವೆಂದು
ಮಣಿ-ಪರುಪ್ಪುಗೆ
ಮಣಿಪ್ಪರುಪ್ಪು-ವಿಗೆ
ಮಣಿಪ್ರ-ದೀಪ
ಮಣಿ-ಮಯ-ಮು-ಕುರಂ
ಮಣಿಯ
ಮಣಿ-ಯ-ಮರಸ
ಮಣಿ-ಯ-ಮರ-ಸ-ನಕೆಅಱೇಲಿ
ಮಣಿ-ಯಮ್ಮನ
ಮಣಿ-ಯೂರು
ಮಣಿಹ
ಮಣೆ
ಮಣ್ಡಲಸ್ವಾಮಿ
ಮಣ್ಡ-ಳೀಕ-ಜೂಬು
ಮಣ್ಣನ್
ಮಣ್ಣನ್ನು
ಮಣ್ಣನ್ನು-ಗದ್ದೆ
ಮಣ್ಣನ್ನೂ-ಗದ್ದೆ
ಮಣ್ಣಿನ
ಮಣ್ಣಿ-ನಲ್ಲಿ
ಮಣ್ಣಿ-ನಿಂದ
ಮಣ್ಣು
ಮಣ್ಣು-ಗ-ಳನ್ನು
ಮಣ್ಣು-ಹಾಕು-ವು-ದಕ್ಕೆ
ಮಣ್ಣೆ
ಮಣ್ಣೆ-ಯನ್ನು
ಮಣ್ಣೆ-ಯಲ್ಲಿ
ಮಣ್ಣೆ-ಯಿಂದ
ಮಣ್ನ
ಮಣ್ನು
ಮಣ್ನುಂ
ಮತ
ಮತ-ಗ-ಳಲ್ಲಿ
ಮತದ
ಮತ-ದಲ್ಲಿ-ರುವ
ಮತ-ದ-ವರು
ಮತಪ್ರ-ಚಾರಕ್ಕಾಗಿ
ಮತ-ವನ್ನು
ಮತವೂ
ಮತ-ವೆಂದು
ಮತವೇ
ಮತಸ್ಥ-ರಿಗೆ
ಮತಾನು-ಯಾಯಿ-ಗಳು
ಮತಾವ-ಲಂಬಿ-ಗ-ಳಾಗಿದ್ದ-ರೆಂದು
ಮತಿ-ಶಾ-ಗರ
ಮತಿ-ಸಾ-ಗರ
ಮತಿ-ಸಾ-ಗರರ
ಮತೀಯ
ಮತು
ಮತ್ತ
ಮತ್ತಂ
ಮತ್ತ-ಭೃಂಗರುಂ
ಮತ್ತ-ಮಾತಂಗ
ಮತ್ತ-ಯರ
ಮತ್ತರು
ಮತ್ತಷ್ಟು
ಮತ್ತಿ-ಕೆರೆ-ಯನ್ನು
ಮತ್ತಿಗೆ
ಮತ್ತಿ-ದಾಗ
ಮತ್ತಿ-ಮರ-ಗ-ಳಿಂದ
ಮತ್ತಿಯ-ಕೆರೆ
ಮತ್ತಿಯರ
ಮತ್ತು
ಮತ್ತೂರ
ಮತ್ತೆ
ಮತ್ತೆ-ಗೆರೆ
ಮತ್ತೊಂದು
ಮತ್ತೊಬ್ಬ
ಮತ್ತೊಮ್ಮೆ
ಮಥ್ಚ-ಮತ್ಸ್ಯ
ಮದ-ಗಂಧೇಭ
ಮದ-ದಾನೆ-ಗಳೊಡನೆ
ಮದ-ದಾನೆ-ಯನ್ನು
ಮದ-ನ-ಪುರ
ಮದ-ನ-ವಿಲಾಸದ
ಮದ-ನಸ್ಸಿದ್ಧಾನ್ತಾಂಭೋ-ನಿಧಿಱ್ರಭಾಚನ್ದ್ರಃ
ಮದ-ನಿ-ಕೆಯರ
ಮದ-ವದುಗ್ರ-ವೈರಿ-ಮದ-ಮರ್ದ್ಧನ
ಮದ-ವದುದಗ್ರ
ಮದ-ವ-ಳಿಗೆ
ಮದ-ವ-ಳಿಗೆ-ಪತ್ನಿ
ಮದ-ವೇರಿದ
ಮದಾನ್ಧ
ಮದಿಮದ
ಮದಿಮ-ದ-ಮನೆ
ಮದಿಳತ್ತಿರ್
ಮದಿ-ಸಿದ
ಮದೀ-ನಾದ
ಮದು-ಗನ್ದೂರ
ಮದು-ರೆಯದ
ಮದುವ-ಳಿಗೆ
ಮದುವೆ
ಮದುವೆಗೆ
ಮದುವೆ-ದೆರೆ
ಮದುವೆ-ನಿಂದು
ಮದುವೆಯ
ಮದುವೆ-ಯನ್ನು
ಮದುವೆ-ಯ-ವೊಲೆ
ಮದುವೆ-ಯ-ಸುಂಕ
ಮದುವೆ-ಯಾಗಿದ್ದನು
ಮದುವೆ-ಯಾದ-ನೆಂದು
ಮದುವೆ-ಯೊಳೊಂದು
ಮದೂರು
ಮದೇ
ಮದ್ದಿಕ್ಕೆರೈ
ಮದ್ದಿನ-ಮನೆ-ಗಳು
ಮದ್ದಿ-ಯಕ್ಕ
ಮದ್ದಿ-ಯಕ್ಕನ
ಮದ್ದಿ-ಯಕ್ಕರ
ಮದ್ದೂರ
ಮದ್ದೂ-ರನ್ನು
ಮದ್ದೂ-ರಾದ
ಮದ್ದೂ-ರಿಗೆ
ಮದ್ದೂರಿನ
ಮದ್ದೂರಿನಲ್ಲಿ
ಮದ್ದೂರು
ಮದ್ರ-ವಾಡ
ಮದ್ರಾಸಿ-ನಲ್ಲಿದ್ದ
ಮದ್ರಾಸಿ-ನ-ವರೆಗೂ
ಮದ್ರಾಸ್
ಮಧುಕ-ರಾಸಿ
ಮಧು-ಕರಿ
ಮಧುಕೇಶ್ವರ
ಮಧುಕೇಶ್ವರ-ನಾಗಿ-ರು-ವು-ದನ್ನು
ಮಧುರ-ಮಂಡಲ
ಮಧುರ-ವಾಗಿದ್ದವು
ಮಧುರ-ವಾದ
ಮಧುರಾ
ಮಧುರೆ
ಮಧು-ರೆಯ
ಮಧುರೆ-ಯದ
ಮಧುರೆ-ಯನ್ನು
ಮಧುವಂಣ
ಮಧುಸೂ-ದನ
ಮಧುಸೂ-ದನ್
ಮಧುಸೂ-ದನ್ಗೆ
ಮಧುಸೂ-ಧನ
ಮಧುಸೂಧ-ನನ
ಮಧ್ಯ
ಮಧ್ಯ-ಕಾಲೀನ
ಮಧ್ಯ-ಗತ-ವಾಗಿ-ರುವು-ದ-ರಿಂದ
ಮಧ್ಯದ
ಮಧ್ಯ-ದಲು
ಮಧ್ಯ-ದ-ಲುಳ್ಳ
ಮಧ್ಯ-ದಲ್ಲಿ
ಮಧ್ಯ-ದಲ್ಲಿದ್ದ
ಮಧ್ಯ-ದಲ್ಲಿಯೇ
ಮಧ್ಯ-ದೇಶಕ್ಕೆ
ಮಧ್ಯ-ದೇಸ-ಮುದ್ದಂಡ-ವಿನಾಳ್ದು
ಮಧ್ಯ-ದೊಳ-ಗಣ
ಮಧ್ಯ-ದೊಳಗೆ
ಮಧ್ಯ-ಭಾಗ-ದಲ್ಲಿ
ಮಧ್ಯ-ಯುಗ
ಮಧ್ಯ-ರಂಗ-ವೆಂದು
ಮಧ್ಯ-ವರ್ತಿ-ಯಾದ
ಮಧ್ಯ-ಸು-ದರ್ಶನಾ-ಚಾರ್ಯ
ಮಧ್ಯ-ಸು-ದರ್ಶನಾ-ಚಾರ್ಯ-ನಾದ
ಮಧ್ಯಸ್ಥಿಕೆ
ಮಧ್ಯಾಹ್ನದ
ಮಧ್ಯೆ
ಮಧ್ವಾ-ಚಾರ್ಯರ
ಮಧ್ವಾ-ಚಾರ್ಯರು
ಮನ-ಕೋಜ
ಮನ-ಗಾಣ-ಲಿಲ್ಲ
ಮನ-ಗಾಣಲು
ಮನ-ಗೆಡಿ-ಸದೆ
ಮನ-ದನ್ನ-ನಪ್ಪ
ಮನ-ದಲ್ಲಿ
ಮನಮಂ
ಮನ-ಮೂರ್ತಿ
ಮನ-ಮೊಸೆದು
ಮನ-ವರಿಕೆ-ಯಾಗಿದೆ
ಮನ-ವರಿಕೆ-ಯಾಗು
ಮನ-ವಾಳ
ಮನ-ವಾಳನ್
ಮನ-ಸಾರೆ
ಮನ-ಸೆಳೆ-ಯುವ
ಮನಸ್ಸಿ-ನಿಂದ
ಮನಾ-ಲರ
ಮನಿ-ಗಳು
ಮನಿಚ್ಚನ್
ಮನಿ-ಷಿಗೆ
ಮನಿ-ಷಿಣಾಂಹ್ರೀ-ನಿ-ವಾಸ
ಮನಿ-ಷಿಯ
ಮನಿ-ಸಿಬಿ-ಟನು
ಮನೀಷಿ
ಮನು
ಮನು-ಚರಿತ
ಮನು-ಚರಿತರು
ಮನು-ಚಾರಿರ್ಯರೂ
ಮನು-ಧರ್ಮ-ಶಾಸ್ತ್ರ
ಮನುಪ್ರತಿಮಂ
ಮನುಬ್ರೋಲು
ಮನು-ಮಥಾಂತ-ಕರುಂ
ಮನು-ಮಾರ್ಗನು
ಮನು-ಮಾರ್ಗ-ರೆಂದು
ಮನು-ಮಾರ್ಗಾಗ್ರಣಿ-ಗಳು
ಮನು-ಮುನಿ-ಚರಿತನೂ
ಮನುಷ
ಮನುಷ್ಯ
ಮನುಷ್ಯ-ನಾಗಿ
ಮನೆ
ಮನೆ-ಗಳ
ಮನೆ-ಗ-ಳನ್ನು
ಮನೆ-ಗ-ಳನ್ನೂ
ಮನೆ-ಗ-ಳಲ್ಲಿ
ಮನೆ-ಗ-ಳಿಗೆ
ಮನೆ-ಗ-ಳಿವು
ಮನೆ-ಗಳಿವೆ
ಮನೆ-ಗಳು
ಮನೆ-ಗ-ಳೆಂದು
ಮನೆಗೆ
ಮನೆ-ತನ
ಮನೆ-ತನಕ್ಕೆ
ಮನೆ-ತನ-ಗಳ
ಮನೆ-ತನ-ಗಳು
ಮನೆ-ತ-ನದ
ಮನೆ-ತನ-ದಂತೆ
ಮನೆ-ತನ-ದಲ್ಲಿ
ಮನೆ-ತನ-ದ-ವ-ನಾ-ಗಿದ್ದು
ಮನೆ-ತನ-ದ-ವ-ನಾಗಿ-ರ-ಬ-ಹುದು
ಮನೆ-ತನ-ದ-ವ-ನೆಂದು
ಮನೆ-ತನ-ದ-ವರು
ಮನೆ-ತನ-ದ-ವ-ರೆಂದು
ಮನೆ-ತನ-ದ-ವರೇ
ಮನೆ-ತನ-ದ-ವಳೇ
ಮನೆ-ತನ-ದಿಂದ
ಮನೆ-ತನ-ವನ್ನು
ಮನೆ-ತನ-ವಾಗಿದೆ
ಮನೆ-ತನ-ವೆಂದು
ಮನೆ-ತನವೇ
ಮನೆ-ದೆರೆ
ಮನೆ-ದೆರೆ-ಯನ್ನೂ
ಮನೆ-ನದ-ವ-ರೆಂದು
ಮನೆ-ಮಗ
ಮನೆ-ಮಾತು
ಮನೆಯ
ಮನೆ-ಯನ್ನು
ಮನೆ-ಯನ್ನೂ
ಮನೆ-ಯ-ಬಲೆ
ಮನೆ-ಯ-ಮಗ
ಮನೆ-ಯಲ್ಲಿ
ಮನೆ-ಯಲ್ಲಿಯೇ
ಮನೆ-ಯಲ್ಲೇ
ಮನೆ-ಯವಂ
ಮನೆ-ಯ-ವ-ರಿಗೆ
ಮನೆ-ಯ-ವರು
ಮನೆ-ಯಿಂದ
ಮನೆಯು
ಮನೆಯೇ
ಮನೆ-ಯೊಂದನ್ನು
ಮನೆ-ಯೊಂದರ
ಮನೆರ್ದೋಡಿ-ಸುತಂ
ಮನೆ-ವಣ
ಮನೆ-ವಣ-ಯಿಲ್ಲ
ಮನೆ-ವಣ-ವಿಲ್ಲ
ಮನೆ-ವೆಗ್ಗಡೆ
ಮನೆ-ವೆರ್ಗ್ಗಡೆ
ಮನೆ-ಹಣ
ಮನೋ
ಮನೋಜಃ
ಮನೋ-ಜ-ಭಯಂಕರ
ಮನೋಜ್ಞ-ವಾಗಿವೆ
ಮನೋ-ನ-ಯನ
ಮನೋ-ನ-ಯನ-ವಲ್ಲಭೆ
ಮನೋ-ಭಾ-ವಕ್ಕೆ
ಮನೋ-ಭಾವನೆ
ಮನೋ-ಭಾವ-ನೆ-ಯನ್ನು
ಮನೋ-ಭಾವವು
ಮನೋ-ಭೀಷ್ಟ
ಮನೋ-ಭೀಷ್ಠ
ಮನೋ-ಮಿತ್ರ-ನಾಗಿದ್ದ-ನೆಂದು
ಮನೋ-ಮಿತ್ರ-ನಾದ
ಮನೋ-ಮುದ-ದಿ-ನೀವುತ್ತಿರ್ಕ್ಕೆ
ಮನೋ-ರಥಂಗಳಂ
ಮನೋ-ರಮೆ
ಮನೋ-ರಮೆ-ಪತ್ನಿ
ಮನೋ-ವಲ್ಲಭ
ಮನೋ-ವಲ್ಲ-ಭ-ನಾದ
ಮನೋ-ವಲ್ಲ-ಭೆ-ಯರು
ಮನೋ-ಹರತಾ
ಮನೋ-ಹರ-ವಾಗಿ
ಮನೋ-ಹರಿ
ಮನ್ತಣಿಪಿತ್ತು
ಮನ್ತ್ರಿ-ಚಾಮುಣ್ಡನ
ಮನ್ನಣೆ
ಮನ್ನ-ಸಿವೆ
ಮನ್ನಾ
ಮನ್ನಾ-ಮಾಡ-ಲಾಗಿದೆ
ಮನ್ನಾ-ಮಾ-ಡಲು
ಮನ್ನಾ-ಮಾ-ಡುವ
ಮನ್ನಾರು
ಮನ್ನಾ-ರು-ಕೃಷ್ಣಸ್ವಾಮಿ
ಮನ್ನೂರ-ರಲ್ಲಿ
ಮನ್ನೂರು
ಮನ್ನೆಯ
ಮನ್ನೆಯ-ರಿಗೆಲ್ಲಾ
ಮನ್ನೆಯ-ಸೂನು
ಮನ್ನೆ-ಯೊಳಗೆ
ಮನ್ಮಥ-ಪುಷ್ಕರ-ಣಿ-ಗ-ಳನ್ನು
ಮಮ್ಮಂ
ಮಯ
ಮಯನು
ಮಯಿದ-ಸೆಟ್ಟಿಯ
ಮಯಿ-ಭೋಗದ
ಮಯಿಲನ-ಹಳ್ಳಿ
ಮಯಿಲನ-ಹಳ್ಳಿಗೆ
ಮಯಿಲನ-ಹಳ್ಳಿಯ
ಮಯಿಲನ-ಹಳ್ಳಿ-ಯನ್ನು
ಮಯಿಲನ-ಹಳ್ಳಿ-ಯಲ್ಲಿ
ಮಯಿಸೂರ
ಮಯಿಸೂರು
ಮಯಿಸೆ-ನಾಡ
ಮಯೂರಾ-ಸನ
ಮಯ್ದುನ
ಮಯ್ದುನ-ನಾಗಿದ್ದ-ನೆಂದು
ಮಯ್ದುನನೂ
ಮಯ್ದುನ-ಹಳ್ಳಿಯ
ಮಯ್ಯಂಗಳು
ಮಯ್ವೆತ್ತ
ಮರ
ಮರ-ಕಾಡನ್ನು
ಮರ-ಕೋಜ
ಮರ-ಕೋ-ಜನು
ಮರ-ಗ-ಳಿಗೆ
ಮರ-ಗಳಿದ್ದ
ಮರ-ಗಳಿದ್ದವು
ಮರಗಿ-ಡ-ಗ-ಳನ್ನು
ಮರ-ಗೆಲಸ
ಮರ-ಗೆಲಸ-ವೆಲ್ಲಾ
ಮರಡಿ
ಮರ-ಡಿಗೆ
ಮರ-ಡಿ-ಪುರ
ಮರ-ಡಿ-ಪುರವೂ
ಮರ-ಡಿ-ಯೊಳ್
ಮರಣ
ಮರ-ಣ-ಕಾಲ-ದಲ್ಲಿ
ಮರ-ಣದ
ಮರ-ಣ-ವನ್ನಪ್ಪಿ-ದಾಗ
ಮರ-ಣ-ವನ್ನಪ್ಪುತ್ತಾರೆ
ಮರ-ಣ-ವನ್ನು
ಮರ-ಣ-ವನ್ನೇ
ಮರ-ಣ-ವೆಂದರೆ
ಮರ-ಣ-ಶಾ-ಸನ-ದಲ್ಲಿ
ಮರ-ಣ-ಹೊಂದಿದ
ಮರ-ಣ-ಹೊಂದಿ-ದಂತೆ
ಮರ-ಣ-ಹೊಂದಿ-ದನು
ಮರ-ಣ-ಹೊಂದಿ-ದ-ನೆಂದು
ಮರ-ಣ-ಹೊಂದಿ-ದ-ವರ
ಮರ-ಣ-ಹೊಂದಿದ್ದಾನೆ
ಮರ-ಣ-ಹೊಂದಿ-ನೆಂದು
ಮರಣಾ
ಮರ-ಣಾ-ನಂತರ
ಮರದ
ಮರದು-ರಾದ
ಮರ-ದೂರ
ಮರ-ದೂರ-ಮದ್ದೂರ
ಮರ-ದೂ-ರಾದ-ಮದ್ದೂರು
ಮರ-ದೂರು
ಮರಪ್ಪಂಗೆ
ಮರ-ಲ-ಹಳ್ಳಿ
ಮರ-ಲ-ಹಳ್ಳಿ-ಗ-ಳಲ್ಲಿ
ಮರ-ಲ-ಹಳ್ಳಿಯ
ಮರಲೆ
ಮರ-ಳಿ-ಕೆರೆ
ಮರ-ಳಿ-ನಿಂದ
ಮರ-ಳಿ-ಬಿಟ್ಟು
ಮರ-ಳಿ-ಸಲು
ಮರ-ಳಿಸಿ
ಮರ-ಳಿ-ಸಿದ
ಮರಳು
ಮರ-ಳು-ನೆಲ-ದಿಂದ
ಮರ-ಳೇಶ್ವರ
ಮರ-ವಿದ್ದ
ಮರ-ವೂರ
ಮರಸು
ಮರ-ಸೆಯ
ಮರ-ಹಳ್ಳಿ
ಮರಾಠರ
ಮರಾಠ-ರಿಗೆ
ಮರಾಠರು
ಮರಾಠಾ
ಮರಿ-ಗವುಡ
ಮರಿ-ದೇವ
ಮರಿ-ದೇವ-ರಾಜ
ಮರಿ-ದೇವ-ರಾಜ-ವಡೆಯ-ನೆಂಬ
ಮರಿ-ಯಣ್ಣ-ನ-ವರ
ಮರಿಯ-ನಾಯ-ಕನ
ಮರಿ-ಯಾದಿ
ಮರಿ-ಯಾದೆ
ಮರಿ-ಯಾದೆ-ಯಲು
ಮರಿಯಾನೆ
ಮರಿಯಾನೆ-ಕಿ-ರಿಯ
ಮರಿಯಾನೆ-ಗಳು
ಮರಿಯಾನೆಗೆ
ಮರಿಯಾನೆಯ
ಮರಿಯಾನೆಯೂ
ಮರಿಯಾನೆಯೇ
ಮರಿಯಾನೆ-ಸ-ಮುದ್ರ
ಮರಿಯಾನೆ-ಸ-ಮುದ್ರದ
ಮರಿಯಾ-ಯನೆ-ಯನ್ನು
ಮರಿ-ಸೆಟ್ಟಿಯು
ಮರು-ದ-ಗಾ-ಮುಂಡ
ಮರು-ದೇವಿ
ಮರು-ದೇವಿ-ಯರ
ಮರು-ಪರಿಶೀಲಿಸಬೇಕಾಗುತ್ತದೆ
ಮರು-ಪರಿಶೀಲಿ-ಸಿದ
ಮರುಳ
ಮರುಳ-ದೇವ
ಮರುಳನು
ಮರುಳ-ಸಿದ್ಧ
ಮರು-ವರ್ಮನ
ಮರು-ವರ್ಮ-ನಿಗೂ
ಮರು-ವರ್ಮನು
ಮರೆ
ಮರೆ-ತ-ರೆಂದೂ
ಮರೆತೂ
ಮರೆ-ಮಗ್ಗ
ಮರೆ-ಯದೆ
ಮರೆ-ಯಾಗಿವೆ
ಮರೆ-ವೊಕ್ಕಡೆ-ಕಾವ
ಮರ್ಕುಲಿ
ಮರ್ತ್ಯ-ಲೋಕದ
ಮರ್ದಿನಿ
ಮರ್ದಿಸಿ
ಮರ್ದ್ದಸ
ಮರ್ದ್ದಿಸಿ
ಮರ್ದ್ಧನ
ಮರ್ಯಾದೆ
ಮರ್ಯಾದೆ-ಉಂಬಳಿ-ಗವು-ಡು-ಗೊಡಗೆ
ಮರ್ಯಾದೆ-ಗಾಗಿ
ಮರ್ಯಾದೆಯ
ಮರ್ಯಾದೆ-ಯನ್ನು
ಮರ್ಯಾದೆ-ಯಲು
ಮರ್ಯಾದೆ-ಯಲ್ಲಿ
ಮರ್ಯಾದೆ-ಯಾಗಿ
ಮರ್ಯಾದೆಯು
ಮರ್ಯ್ಯಾದಿಗೆ
ಮರ್ರ-ವಿಗೆ
ಮರ್ರವ್ವ-ನಿಗೆ
ಮಱಿಯಾನೆ
ಮಱೆ-ಮಗ್ಗಕೆ
ಮಱೆವೊಕ್ಕರ-ಕಾವರುಂ
ಮಱೆವೊಕ್ಕರೆ
ಮಲ-ಗಿರು-ವ-ವನು
ಮಲ-ತಮ್ಮ
ಮಲ-ತಮ್ಮ-ನಾದ
ಮಲ-ತಮ್ಮ-ನಿದ್ದನು
ಮಲ-ತಮ್ಮ-ಮಲ್ಲಿ-ತಮ್ಮ
ಮಲದ್ವಿಷದ್ಬಲ-ಶಿಲಾಸ್ತಂಭಾವಲೀ
ಮಲ-ಧಾರಿ
ಮಲ-ಧಾರಿ-ದೇವ
ಮಲ-ಧಾರಿ-ದೇವನ
ಮಲ-ಧಾರಿ-ದೇವರ
ಮಲ-ಧಾರಿ-ದೇವರಂ
ಮಲ-ಧಾರಿ-ದೇವರು
ಮಲ-ಧಾರೀ-ದೇವ
ಮಲ-ನಾಯ-ಕ-ನ-ಹಳ್ಳಿ
ಮಲ-ಯಚ್ಚಿ
ಮಲಯಾ
ಮಲ-ಯಾಳ
ಮಲ-ಯಾ-ಳನ
ಮಲ-ಯಾ-ಳನ್
ಮಲ-ವರ-ನ-ಲಗು
ಮಲ-ಸ-ಹೋದ-ರರ
ಮಲ-ಸೆಟ್ಟಿ-ಯರ
ಮಲ-ಹ-ಗಳ್ಳಿಮಳ-ಗಳಲಿ
ಮಲಿ-ತಂಮ
ಮಲಿ-ದೇವ
ಮಲಿಯೂ-ರಾಗಿ-ರ-ಬ-ಹುದು
ಮಲಿ-ಯೂರಿನ
ಮಲಿ-ಯೂರು
ಮಲು-ಕಬ್ಬೆ-ಪುರ
ಮಲು-ಕಬ್ಬೆ-ಪುರ-ಇಂದಿನ
ಮಲು-ನಾಯ-ಕ-ನ-ಹಳ್ಳಿಯ
ಮಲು-ಭಾರ-ತಿಯ
ಮಲೆ
ಮಲೆ-ಗರು
ಮಲೆ-ನಾಡಿ-ನಲ್ಲಿ
ಮಲೆ-ನಾಡು
ಮಲೆ-ನಾಡು-ಗಳು
ಮಲೆ-ನಾಡೇಳು
ಮಲೆ-ಪ-ನಾಯಕ
ಮಲೆ-ಪ-ನಾಯ-ಕನು
ಮಲೆ-ಪರ-ಮಲ್ಲ
ಮಲೆ-ಮ-ಹದೇಶ್ವರ
ಮಲೆಯ
ಮಲೆ-ಯ-ಕಡೆಗೆ
ಮಲೆ-ಯ-ದಂಡಿಗೆತ್ತಿ
ಮಲೆ-ಯ-ನ-ಹಳ್ಳಿ-ಗಳು
ಮಲೆ-ಯ-ನಾಯ-ಕನ
ಮಲೆ-ಯ-ನಾಯ-ಕ-ನ-ಹಳ್ಳಿ
ಮಲೆ-ಯನು
ಮಲೆ-ಯಾಂಡನ್
ಮಲೆ-ಯಾಲ-ಗಮಿಡಿಪಿ
ಮಲೆ-ಯಾಳ
ಮಲೆ-ಯಾಳದ
ಮಲೆ-ಯಾಳನ
ಮಲೆ-ಯಾಳನ್
ಮಲೆ-ಯಾಳರಂ
ಮಲೆ-ಯಾಳಿ
ಮಲೆ-ಯೂ-ರಿಗೆ
ಮಲೆ-ರಾಜ-ರಾಜ
ಮಲೆ-ರಾಜ್ಯಕ್ಕೆ
ಮಲೆವ
ಮಲೈ
ಮಲೈ-ಮೇಲ್
ಮಲೈ-ಯಚ್ಚಿ
ಮಲೈ-ಯ-ಮಾನನ
ಮಲೈ-ಯ-ರನ್
ಮಲೈ-ಯರ-ಶನ್
ಮಲೈ-ಯಾಳನ್
ಮಲ್ಲ
ಮಲ್ಲಂ
ಮಲ್ಲ-ಗವುಡ
ಮಲ್ಲ-ಘಟ್ಟ
ಮಲ್ಲ-ಘಟ್ಟದ
ಮಲ್ಲ-ಘಟ್ಟ-ದಲ್ಲಿ
ಮಲ್ಲ-ಘಟ್ಟ-ವನ್ನೂ
ಮಲ್ಲ-ಜೀಯನ
ಮಲ್ಲ-ಜೀಯ-ನಿಗೆ
ಮಲ್ಲಣಃ
ಮಲ್ಲ-ಣಾ-ಚಾರ್ಯ
ಮಲ್ಲ-ಣಾ-ಚಾರ್ಯ-ವರ್ಯ
ಮಲ್ಲ-ಣಾರ್ಯನ
ಮಲ್ಲ-ನ-ಕೆರೆಯ
ಮಲ್ಲ-ನಾಯಕ
ಮಲ್ಲ-ನಾಯ-ಕನ
ಮಲ್ಲ-ನಾಯ-ಕ-ನ-ಹಳ್ಳಿ
ಮಲ್ಲನು
ಮಲ್ಲಪ್ಪ-ನ-ವರು
ಮಲ್ಲಪ್ಪ-ಸೆಟ್ಟಿ-ಯಾದ
ಮಲ್ಲಯ್ಯ
ಮಲ್ಲಯ್ಯನ
ಮಲ್ಲಯ್ಯ-ನ-ಹಳ್ಳಿ
ಮಲ್ಲಯ್ಯ-ನ-ಹಳ್ಳಿ-ಕೆರೆ
ಮಲ್ಲಯ್ಯ-ನಾಯಕ
ಮಲ್ಲಯ್ಯ-ನಿರ-ಬ-ಹುದು
ಮಲ್ಲಯ್ಯನು
ಮಲ್ಲಯ್ಯರು
ಮಲ್ಲ-ರಸ
ಮಲ್ಲ-ರ-ಸನು
ಮಲ್ಲ-ರ-ಸನೂ
ಮಲ್ಲ-ರಾಜ
ಮಲ್ಲ-ರಾಜ-ಗಳೆಂಬ
ಮಲ್ಲ-ರಾಜನ
ಮಲ್ಲ-ರಾಜ-ನೆಂಬ
ಮಲ್ಲ-ರಾಜ-ವೊಡ-ಯನೂ
ಮಲ್ಲ-ರಾಯ
ಮಲ್ಲ-ರಾಯ-ನೆಂಬ
ಮಲ್ಲ-ಸೆಟ್ಟಿಗೆ
ಮಲ್ಲ-ಸೆಟ್ಟಿ-ಯಾದ
ಮಲ್ಲಾಂಬಿಕೆ
ಮಲ್ಲಾಂಬಿ-ಕೆಗೆ
ಮಲ್ಲಾಂಬಿ-ಕೆಯ
ಮಲ್ಲಾ-ದೇವಿ-ಯಿಂದ
ಮಲ್ಲಾ-ಪುರ
ಮಲ್ಲಾರಿ-ಯ-ವರು
ಮಲ್ಲಿಕಾಜುನ-ಪುರ-ವಾದ
ಮಲ್ಲಿಕಾಫ-ರನು
ಮಲ್ಲಿ-ಕಾರ್ಜುನ
ಮಲ್ಲಿ-ಕಾರ್ಜುನ-ದೇವರ
ಮಲ್ಲಿ-ಕಾರ್ಜುನ-ದೇವ-ರಿಗೆ
ಮಲ್ಲಿ-ಕಾರ್ಜುನನ
ಮಲ್ಲಿ-ಕಾರ್ಜುನ-ನ-ವರೆಗೆ
ಮಲ್ಲಿ-ಕಾರ್ಜುನ-ನಿಗೆ
ಮಲ್ಲಿ-ಕಾರ್ಜುನನು
ಮಲ್ಲಿ-ಕಾರ್ಜುನ-ನೆಂಬ
ಮಲ್ಲಿ-ಕಾರ್ಜುನನ್ನು
ಮಲ್ಲಿ-ಕಾರ್ಜುನ-ಪುರ
ಮಲ್ಲಿ-ಕಾರ್ಜುನ-ಪುರ-ವಾದ
ಮಲ್ಲಿ-ಕಾರ್ಜುನ-ಪುರ-ವೆಂದೂ
ಮಲ್ಲಿ-ಕಾರ್ಜುನ-ರಾಯನು
ಮಲ್ಲಿ-ಕಾರ್ಜುನೋ
ಮಲ್ಲಿಕ್
ಮಲ್ಲಿಕ್ಕಾಫುರ-ನಿಂದ
ಮಲ್ಲಿಗೆ
ಮಲ್ಲಿಗೆ-ಗೆರೆ-ಯನ್ನು
ಮಲ್ಲಿಗೆ-ದು-ಡು-ಪಿನ
ಮಲ್ಲಿ-ಗೆರೆ
ಮಲ್ಲಿಗೆ-ರೆ-ಯೆಂಬ
ಮಲ್ಲಿ-ತಮ್ಮ
ಮಲ್ಲಿ-ತಮ್ಮನ
ಮಲ್ಲಿ-ತಮ್ಮನು
ಮಲ್ಲಿ-ದೇವ
ಮಲ್ಲಿ-ದೇವನ
ಮಲ್ಲಿ-ದೇವನು
ಮಲ್ಲಿ-ನ-ಗೆರೆ
ಮಲ್ಲಿ-ನಾಥ
ಮಲ್ಲಿ-ನಾಥ-ದೇವರ
ಮಲ್ಲಿ-ನಾಥನ
ಮಲ್ಲಿ-ನಾಥ-ನನ್ನು
ಮಲ್ಲಿ-ನಾಥ-ನಿಂದ
ಮಲ್ಲಿ-ನಾಥನು
ಮಲ್ಲಿ-ಯಣ
ಮಲ್ಲಿ-ಯಣ್ಣ
ಮಲ್ಲಿ-ಯಣ್ಣನ
ಮಲ್ಲಿ-ಯಣ್ಣ-ನನ್ನು
ಮಲ್ಲಿ-ಯಣ್ಣನು
ಮಲ್ಲಿ-ಶೆಟ್ಟಿ
ಮಲ್ಲಿಷೇಣ
ಮಲ್ಲಿಷೇ-ಣನ
ಮಲ್ಲಿಷೇಣ-ಮಲ-ಧಾರಿ-ದೇವ
ಮಲ್ಲಿ-ಸೆಟ್ಟಿಗೆ
ಮಲ್ಲಿ-ಸೆಟ್ಟಿ-ವಿಭು
ಮಲ್ಲೀ-ದೇವಿ-ಕಟ್ಟೆ
ಮಲ್ಲೆ
ಮಲ್ಲೆ-ನಾಯಕ
ಮಲ್ಲೆ-ನಾಯ-ಕನ
ಮಲ್ಲೆ-ನಾಯ-ಕ-ನಿಗೆ
ಮಲ್ಲೆ-ನಾಯ-ಕರು
ಮಲ್ಲೆಯ
ಮಲ್ಲೆ-ಯ-ನಾಯಕ
ಮಲ್ಲೆ-ಯ-ನಾಯ-ಕನ
ಮಲ್ಲೆ-ಯ-ನಾಯ-ಕ-ನಿಗೆ
ಮಲ್ಲೆ-ಯ-ನಾಯ-ಕನು
ಮಲ್ಲೆ-ಸಾಮಂತನು
ಮಲ್ಲೇ-ನ-ಹಳ್ಳಿ
ಮಲ್ಲೇ-ಪುರಂ
ಮಲ್ಲೇಶ್ವರ
ಮಲ್ಲೋಜ
ಮಲ್ಲೋಜಂ
ಮಲ್ಲೋಜ-ನೆಂಬ
ಮಳ-ಗಿಯರ
ಮಳಬ್ರಯ
ಮಳಭ್ರಯ
ಮಳಲನ-ಕೆರೆ
ಮಳಲ-ವಾಡಿಯ
ಮಳಲ-ವಾಡಿ-ಯಲ್ಲಿದ್ದು
ಮಳಲಿ-ಗನ
ಮಳಲಿಯ
ಮಳಲೂ-ರನ್ನು
ಮಳ-ಲೂರು
ಮಳಲ್ಪಾ-ಚಾರಿ
ಮಳವಡೆ
ಮಳ-ವಳ್ಳಿ
ಮಳ-ವಳ್ಳಿಯ
ಮಳ-ವಳ್ಳಿ-ಯನ್ನು
ಮಳ-ವಳ್ಳಿ-ಯಲ್ಲಿ
ಮಳ-ವಳ್ಳಿ-ಯಲ್ಲಿದ್ದಿರ-ಬ-ಹುದು
ಮಳ-ವಳ್ಳಿಯು
ಮಳಿಗೆ
ಮಳೂರಿ-ನಲ್ಲಿ
ಮಳೂರಿ-ನಲ್ಲಿಯೂ
ಮಳೂರು
ಮಳೂರು-ಪಟ್ಟಣ
ಮಳೆ-ಯ-ನೀರು
ಮಳೆ-ಹಳ್ಳಿಯ
ಮಳ್ಳಕವ್ವ
ಮವ
ಮಶ್ನುತೇತ್ರೈಕಾಂ
ಮಸಅಣಯ್ಯನು
ಮಸಣ
ಮಸಣ-ಜೀಯ
ಮಸಣ-ಜೀಯ-ರಿಗೆ
ಮಸ-ಣನು
ಮಸಣ-ಯನ
ಮಸಣಯ್ಯನ
ಮಸಣಯ್ಯ-ನನ್ನು
ಮಸಣಯ್ಯನಿಗೆ
ಮಸಣಯ್ಯ-ನೆಂಬು-ವವ-ನನ್ನು
ಮಸಣ-ಸ-ಮುದ್ರ
ಮಸಣ-ಸ-ಮುದ್ರ-ದಲು
ಮಸಣಿ
ಮಸಣಿ-ಕಮ್ಮ
ಮಸಣಿ-ತಂಮ
ಮಸಣಿ-ತಂಮನ
ಮಸಣಿ-ತಮ್ಮ
ಮಸಣಿ-ತಮ್ಮನೇ
ಮಸಣಿ-ತಮ್ಮರು
ಮಸಣೆ-ನಾಯಕ
ಮಸ-ಣೆಯ
ಮಸಣೈಯ
ಮಸಣೈ-ಯನ
ಮಸಣೈ-ಯನು
ಮಸಣೋಜ
ಮಸೀದಿ
ಮಸೀದಿ-ಗ-ಳನ್ನು
ಮಸೀದಿಗೆ
ಮಸೀದಿಯ
ಮಸೀದಿ-ಯನ್ನು
ಮಸೀದಿ-ಯಲ್ಲಿದೆ
ಮಸೀದಿಯು
ಮಸುಕಾ-ಗುತ್ತಾ
ಮಸುನಿ-ದೇಶ
ಮಸೂದ್
ಮಸೆದ
ಮಸ್ಜಿದ್
ಮಸ್ಜಿದ್ಇಅಕ್ಷಾ
ಮಸ್ಜಿದ್ಇ-ಅಲಾ
ಮಸ್ತಕ-ಶೂಲ
ಮಸ್ತಕ-ಸೂಲ
ಮಸ್ತಕಾಸೂಲ
ಮಹಂತಛೀ
ಮಹ-ಜನ-ಗಳ
ಮಹಡಿ
ಮಹತ್ತರ
ಮಹತ್ತರ-ವಾದವು
ಮಹತ್ವ
ಮಹತ್ವದ
ಮಹತ್ವದ್ದಾ-ಗಿತ್ತು
ಮಹತ್ವದ್ದಾಗಿದೆ
ಮಹತ್ವ-ಪೂರ್ಣ
ಮಹತ್ವ-ವನ್ನು
ಮಹತ್ವ-ವಾದ
ಮಹತ್ವಾಕಾಂಕ್ಷಿ-ಯಾದ
ಮಹತ್ವಾಕಾಂಕ್ಷಿಯೂ
ಮಹತ್ವಾಕಾಂಕ್ಷೆಗೆ
ಮಹ-ದೇವ
ಮಹ-ದೇವಣ್ಣ
ಮಹ-ದೇವಣ್ಣನ
ಮಹ-ದೇವಣ್ಣ-ನಿಂದ
ಮಹ-ದೇವಣ್ಣನು
ಮಹ-ದೇವಣ್ಣ-ನೆಂಬು-ವ-ವನು
ಮಹ-ದೇವಣ್ಣನೇ
ಮಹ-ದೇವ-ದಂಡ-ನಾಯ-ಕ-ನನ್ನು
ಮಹ-ದೇವನ
ಮಹ-ದೇವ-ನನ್ನು
ಮಹ-ದೇವ-ನಾಯಕ
ಮಹ-ದೇವ-ನಾಯ-ಕ-ನನ್ನು
ಮಹ-ದೇವ-ನಾಯ-ಕ-ನೆಂಬ
ಮಹ-ದೇವ-ನಿಗೆ
ಮಹ-ದೇವನು
ಮಹ-ದೇವ-ಪುರದ
ಮಹ-ದೇವ-ರಾಣೆ
ಮಹ-ದೇವ-ರಾಣೆಯಂ
ಮಹ-ದೇವ-ರಾಣೆಯಿಂ
ಮಹ-ದೇವರು
ಮಹ-ದೇವ-ವನೀ
ಮಹನೀಯ
ಮಹಪ್ರಧಾನ-ರೆಂದೂ
ಮಹ-ಮಂಡ-ಲೇಶ್ವರರು
ಮಹ-ಮದೀ-ಯರ
ಮಹ-ಮದೀ-ಯ-ರಿಗೆ
ಮಹ-ಮದ್
ಮಹ-ಮದ್ರಿಜಾ
ಮಹಮ್ಮ-ದರು
ಮಹಮ್ಮದ್
ಮಹ-ಲಿಂಗೇಶ್ವರ
ಮಹಲು
ಮಹಾ
ಮಹಾಂಗಮಂತ್ರ
ಮಹಾಂಡಲೇಶ್ವರ
ಮಹಾಂಡಳೇಶ್ವರರ
ಮಹಾಂತ-ಜೀಯು
ಮಹಾಂಭೋದಿ
ಮಹಾ-ಅಗ್ರ-ಹಾರದ
ಮಹಾ-ಅರಸ
ಮಹಾ-ಅರ-ಸನ
ಮಹಾ-ಅರ-ಸ-ನಿಗೂ
ಮಹಾ-ಅರ-ಸ-ನಿಗೆ
ಮಹಾ-ಅರ-ಸನು
ಮಹಾ-ಅರ-ಸರು
ಮಹಾ-ಅರಸು
ಮಹಾ-ಅರ-ಸು-ಕೊನೇಟಿ-ರಾಜ
ಮಹಾ-ಅರ-ಸು-ಗಳ
ಮಹಾ-ಅರ-ಸು-ಗಳ-ವರು
ಮಹಾ-ಅರ-ಸು-ಗಳು
ಮಹಾ-ಅರ-ಸು-ರಾಮ-ರಾಜಯ್ಯ-ದೇವ
ಮಹಾ-ಕರ್ಣಾಟ
ಮಹಾ-ಕವಿ
ಮಹಾ-ಕವಿಯ
ಮಹಾ-ಕವಿಯೂ
ಮಹಾ-ಕಾಲ
ಮಹಾ-ಕಾಳ
ಮಹಾ-ಕಾಳಿ
ಮಹಾ-ಕಾವ್ಯ-ಗಳ
ಮಹಾ-ಕುಲ್ಯಾ
ಮಹಾಕ್ರತು-ಗ-ಳನ್ನು
ಮಹಾ-ಗ-ಗನ
ಮಹಾ-ಗಣಂಗಳು
ಮಹಾ-ಗ-ಣಂಗ್ಗಳು
ಮಹಾ-ಗಣ-ಗಳು
ಮಹಾಗ್ರ-ಹಾರ
ಮಹಾಗ್ರ-ಹಾರಂ
ಮಹಾಗ್ರಾಮ-ವೆಂದು
ಮಹಾ-ಚಾರಿ
ಮಹಾ-ಜಂಗ-ಳಿಗೆ
ಮಹಾ-ಜನ
ಮಹಾ-ಜನಂಗಳಿಕ್ಕುವ
ಮಹಾ-ಜನಂಗಳಿಕ್ಕುವರು
ಮಹಾ-ಜನಂಗ-ಳಿಗೆ
ಮಹಾ-ಜನಂಗಳು
ಮಹಾ-ಜನಂಗಳೇ
ಮಹಾ-ಜನಂಗಳ್ಅ
ಮಹಾ-ಜನ-ಗಲು
ಮಹಾ-ಜನ-ಗಳ
ಮಹಾ-ಜನ-ಗಳದ್ದು
ಮಹಾ-ಜನ-ಗ-ಳಾದ
ಮಹಾ-ಜನ-ಗ-ಳಾದಿ-ಯಾಗಿ
ಮಹಾ-ಜನ-ಗ-ಳಿಂದ
ಮಹಾ-ಜನ-ಗ-ಳಿಗೆ
ಮಹಾ-ಜನ-ಗಳಿಗೇ
ಮಹಾ-ಜನ-ಗಳಿದ್ದರೂ
ಮಹಾ-ಜನ-ಗಳು
ಮಹಾ-ಜನ-ಗಳೆಂಬ
ಮಹಾ-ಜನ-ಗಳೇ
ಮಹಾ-ಜನ-ಗಳೋ-ಪಾದಿ-ಯಲ್ಲಿ
ಮಹಾ-ಜ-ನರ
ಮಹಾ-ಜನ-ರಂತೆ
ಮಹಾ-ಜನ-ರನ್ನು
ಮಹಾ-ಜನ-ರಷ್ಟೇ
ಮಹಾ-ಜನ-ರಿಂದ
ಮಹಾ-ಜನ-ರಿಗೂ
ಮಹಾ-ಜನ-ರಿಗೆ
ಮಹಾ-ಜ-ನರು
ಮಹಾ-ಜನ-ರು-ಗಳು
ಮಹಾ-ಜನ-ರುಬ್ರಾಹ್ಮಣರು
ಮಹಾ-ಜನ-ರೆಂದು
ಮಹಾ-ಜನ-ಸಂಸ್ಥೆ-ಯನ್ನು
ಮಹಾ-ಜನ-ಸಭೆಯು
ಮಹಾ-ತಟಾಕ-ವಾಗಿ-ರ-ಬ-ಹುದು
ಮಹಾ-ತಟಾ-ಕಾದಿ
ಮಹಾತ್ಮನೇ
ಮಹಾತ್ಮರ
ಮಹಾತ್ಮವು
ಮಹಾ-ದಂಡ-ನಾಯಕ
ಮಹಾ-ದಂಡಾಧಿ-ಕಾರಿ
ಮಹಾ-ದಾನ
ಮಹಾ-ದಾನದ
ಮಹಾ-ದಾನ-ದೊಳ್
ಮಹಾ-ದಾನ-ವನ್ನು
ಮಹಾ-ದೇವ
ಮಹಾ-ದೇವ-ದೇ-ವೋತ್ತಮ
ಮಹಾ-ದೇವನ
ಮಹಾ-ದೇವ-ನಿಗೆ
ಮಹಾ-ದೇವನು
ಮಹಾ-ದೇವ-ನೆಂಬ
ಮಹಾ-ದೇವ-ಭಟ್ಟ
ಮಹಾ-ದೇವ-ಭಟ್ಟನ
ಮಹಾ-ದೇವರ
ಮಹಾ-ದೇವ-ರ-ಇಂದಿನ
ಮಹಾ-ದೇವ-ರಿಗೆ
ಮಹಾ-ದೇವರ್
ಮಹಾ-ದೇ-ವರ್ಗ್ಗೆ
ಮಹಾ-ದೇವ-ಶಕ್ತಿ
ಮಹಾ-ದೇವಿ
ಮಹಾ-ದೇ-ವಿಗೆ
ಮಹಾ-ದೇವಿ-ಪಟ್ಟ-ಮಹಾ-ದೇವಿ
ಮಹಾ-ದೇವಿಯ
ಮಹಾ-ದೇವಿ-ಯರ
ಮಹಾ-ದೇವಿಯು
ಮಹಾ-ದೇ-ವೋತ್ತಮ
ಮಹಾ-ದೇಸಿ-ಗರ
ಮಹಾ-ದೇಸಿ-ಗರು
ಮಹಾದ್ವಾರ
ಮಹಾದ್ವಾರದ
ಮಹಾದ್ವಾರ-ವನ್ನು
ಮಹಾ-ಧಿ-ರಾಜನ
ಮಹಾ-ನದಿ
ಮಹಾ-ನಾಡು
ಮಹಾ-ನಾಯಂಕಾ-ಚಾರ್ಯ
ಮಹಾ-ನಾಯಕ
ಮಹಾ-ನಾಯ-ಕರು
ಮಹಾ-ನಾಯ-ಕಾ-ಚಾರ್ಯ
ಮಹಾ-ನಾಳ್ಪ್ರಭು
ಮಹಾ-ನು-ಭಾವನ
ಮಹಾ-ನು-ಭಾವನಿಂ
ಮಹಾ-ಪಸಾಯತ
ಮಹಾ-ಪಸಾಯಿತ
ಮಹಾ-ಪಸಾಯಿತರೂ
ಮಹಾ-ಪಸಾಯ್ತ
ಮಹಾ-ಪಸಾಯ್ತರು
ಮಹಾ-ಪಸಾಯ್ತ-ರು-ಪಸಾಯಿತರು
ಮಹಾ-ಪಸಾಯ್ತರೂ
ಮಹಾ-ಪಾ-ತಕ
ಮಹಾ-ಪುರಾಣ-ದಲ್ಲಿ
ಮಹಾ-ಪುರುಷನು
ಮಹಾಪ್ರಚಂಡ
ಮಹಾಪ್ರಚಂಡ-ದಂಡ-ನಾಯಕ
ಮಹಾಪ್ರಚಂಡ-ದಂಡ-ನಾಯ-ಕರ
ಮಹಾಪ್ರ-ದಾನ
ಮಹಾಪ್ರ-ಧನ
ಮಹಾಪ್ರಧಾನ
ಮಹಾಪ್ರಧಾನಂ
ಮಹಾಪ್ರಧಾನ-ದಂಡ-ನಾಯಕ
ಮಹಾಪ್ರಧಾನ-ದಂಡ-ನಾಯ-ಕನೇ
ಮಹಾಪ್ರಧಾನನ
ಮಹಾಪ್ರಧಾನ-ನಾಗಿ
ಮಹಾಪ್ರಧಾನ-ನಾ-ಗಿದ್ದ
ಮಹಾಪ್ರಧಾನ-ನಾಗಿದ್ದ-ನೆಂದು
ಮಹಾಪ್ರಧಾನ-ನಾದ-ನೆಂದು
ಮಹಾಪ್ರಧಾನನು
ಮಹಾಪ್ರಧಾನನೂ
ಮಹಾಪ್ರಧಾನರ
ಮಹಾಪ್ರಧಾನ-ರನ್ನು
ಮಹಾಪ್ರಧಾನ-ರಾಗಿ
ಮಹಾಪ್ರಧಾನ-ರಾಗಿದ್ದರೂ
ಮಹಾಪ್ರಧಾನ-ರಾದ
ಮಹಾಪ್ರಧಾನ-ರಿಗೆ
ಮಹಾಪ್ರಧಾನರು
ಮಹಾಪ್ರಧಾನ-ರೆಂದು
ಮಹಾಪ್ರಧಾನ-ರೆನಿಸಿ-ಕೊಂಡಿದ್ದ-ರೆಂದು
ಮಹಾಪ್ರಧಾನ-ರೆನಿಸಿದ್ದ-ರೆಂದು
ಮಹಾಪ್ರಧಾನರೇ
ಮಹಾಪ್ರಧಾನಿ
ಮಹಾಪ್ರಧಾನಿ-ಗಳ
ಮಹಾಪ್ರಧಾನೂ
ಮಹಾಪ್ರ-ಬಂಧ
ಮಹಾಪ್ರಭು
ಮಹಾಪ್ರಭು-ಗ-ಳಾಗಿ
ಮಹಾಪ್ರಭು-ಗಳು
ಮಹಾಪ್ರಭುಪ್ರಭು-ವಿಭು-ಗಳು
ಮಹಾಪ್ರಭು-ವಾಗಿದ್ದ-ನೆಂದು
ಮಹಾಪ್ರಭು-ವಾಗಿ-ರ-ಬ-ಹುದು
ಮಹಾಪ್ರಭು-ವಿನ
ಮಹಾಪ್ರಭುವು
ಮಹಾಪ್ರಭೋತ್ತ-ಮೂರ್ತಿ-ರನೇಕ
ಮಹಾಪ್ರಸಾದ-ವೆಂದು
ಮಹಾ-ಬಲಿ-ಪುರ
ಮಹಾ-ಬಲೇಶ್ವರ
ಮಹಾ-ಬಲೇಶ್ವರ-ನಿಗೆ
ಮಹಾ-ಬಳ-ರಾಯರ
ಮಹಾ-ಬಿರುದ
ಮಹಾ-ಭಕ್ತ-ನಾದ
ಮಹಾ-ಭಾರತ
ಮಹಾ-ಭಾರ-ತದ
ಮಹಾ-ಭಾರ-ತ-ದಲ್ಲಿ
ಮಹಾ-ಭಾರ-ತ-ವನ್ನು
ಮಹಾ-ಭೂತ
ಮಹಾ-ಭೈರ-ವನ
ಮಹಾ-ಮಂಡ-ಲಾ-ಚಾರ್ಯ
ಮಹಾ-ಮಂಡ-ಲಿಕ
ಮಹಾ-ಮಂಡ-ಲೇಶ್ವರ
ಮಹಾ-ಮಂಡ-ಲೇಶ್ವರಃ
ಮಹಾ-ಮಂಡ-ಲೇಶ್ವರ-ನನ್ನಾಗಿ
ಮಹಾ-ಮಂಡ-ಲೇಶ್ವರ-ನಾಗಿ
ಮಹಾ-ಮಂಡ-ಲೇಶ್ವರ-ನಾ-ಗಿದ್ದ
ಮಹಾ-ಮಂಡ-ಲೇಶ್ವರ-ನಾ-ಗಿದ್ದು
ಮಹಾ-ಮಂಡ-ಲೇಶ್ವರ-ನಾದ
ಮಹಾ-ಮಂಡ-ಲೇಶ್ವರ-ನಿರ-ಬ-ಹುದು
ಮಹಾ-ಮಂಡ-ಲೇಶ್ವರನು
ಮಹಾ-ಮಂಡ-ಲೇಶ್ವರ-ನೊಬ್ಬ
ಮಹಾ-ಮಂಡ-ಲೇಶ್ವರರ
ಮಹಾ-ಮಂಡ-ಲೇಶ್ವರ-ರನ್ನು
ಮಹಾ-ಮಂಡ-ಲೇಶ್ವರ-ರಲ್ಲಿ
ಮಹಾ-ಮಂಡ-ಲೇಶ್ವರ-ರಾದ
ಮಹಾ-ಮಂಡ-ಲೇಶ್ವರ-ರಿಂದ
ಮಹಾ-ಮಂಡ-ಲೇಶ್ವರರು
ಮಹಾ-ಮಂಡ-ಲೇಶ್ವರ-ರು-ಮಹಾ-ಸಾಮಂತರು
ಮಹಾ-ಮಂಡ-ಲೇಶ್ವರ-ರೆಂದು
ಮಹಾ-ಮಂಡ-ಲೇಶ್ವರ-ರೆಲ್ಲರೂ
ಮಹಾ-ಮಂಡ-ಲೇಶ್ವರರೇ
ಮಹಾ-ಮಂಡ-ಳಾ-ಚಾರ್ಯ
ಮಹಾ-ಮಂಡ-ಳಿಕ
ಮಹಾ-ಮಂಡ-ಳೇಶ್ವ-ನಾಗಿ
ಮಹಾ-ಮಂಡ-ಳೇಶ್ವರ
ಮಹಾ-ಮಂಡ-ಳೇಶ್ವರಂ
ಮಹಾ-ಮಂಡ-ಳೇಶ್ವರರ
ಮಹಾ-ಮಂಡ-ಳೇಶ್ವರ-ರಾ-ಗಿದ್ದ
ಮಹಾ-ಮಂಡ-ಳೇಶ್ವರ-ರಾ-ಗಿದ್ದ-ರಿಂದ
ಮಹಾ-ಮಂಡ-ಳೇಶ್ವರ-ರಾ-ಗಿದ್ದರು
ಮಹಾ-ಮಂಡ-ಳೇಶ್ವರ-ರಾ-ದರೂ
ಮಹಾ-ಮಂಡ-ಳೇಶ್ವರರು
ಮಹಾ-ಮಂಡ-ಳೇಶ್ವರ-ರು-ಮಂಡ-ಳೇಶ್ವರರ
ಮಹಾ-ಮಂತ್ರಿ
ಮಹಾ-ಮತು
ಮಹಾ-ಮತ್ಯ-ಪದ
ಮಹಾ-ಮ-ಮಂಡ-ಲೇಶ್ವರರು
ಮಹಾ-ಮ-ಹತ್ತಿಗೆ
ಮಹಾ-ಮ-ಹತ್ತಿನ
ಮಹಾ-ಮ-ಹತ್ತಿ-ನೊಳಗಾದ
ಮಹಾ-ಮಹಾತ್ಮರಾದ
ಮಹಾ-ಮಹಿಪಂ
ಮಹಾ-ಮಾತ್ಯ
ಮಹಾ-ಮಾತ್ಯನು
ಮಹಾ-ಮಾತ್ಯ-ಪ-ದವಿ
ಮಹಾ-ಮಾತ್ಯ-ಪ-ದವೀ
ಮಹಾ-ಮುನಿ
ಮಹಾ-ಮುನಿಗೆ
ಮಹಾ-ಮುನಿಯ
ಮಹಾ-ಯಶಾಃ
ಮಹಾ-ಯಾಗ
ಮಹಾ-ಯಾಗ-ವನ್ನು
ಮಹಾ-ರಾಜ
ಮಹಾ-ರಾಜನ
ಮಹಾ-ರಾಜ-ನಾದ
ಮಹಾ-ರಾಜ-ನಿಗೇ
ಮಹಾ-ರಾಜನು
ಮಹಾ-ರಾಜರ
ಮಹಾ-ರಾಜ-ರಿಗೆ
ಮಹಾ-ರಾಜರು
ಮಹಾ-ರಾಜಾ-ಧಿ-ರಾಜ
ಮಹಾ-ರಾಜಾ-ಧಿ-ರಾಜ-ನೆಂದು
ಮಹಾ-ರಾಜ್ಯ
ಮಹಾ-ರಾಜ್ಯಕ್ಕೆ
ಮಹಾ-ರಾಜ್ಯ-ದಲ್ಲಿ
ಮಹಾ-ರಾಣಿ
ಮಹಾ-ರಾಯ
ಮಹಾ-ರಾಯನ
ಮಹಾ-ರಾಯ-ನಿಂದ
ಮಹಾ-ರಾಯನು
ಮಹಾ-ರಾಯರ
ಮಹಾ-ರಾಯ-ರಿಗೆ
ಮಹಾ-ರಾಯರು
ಮಹಾ-ರಾಯರೂ
ಮಹಾ-ರಾಯೋ
ಮಹಾ-ರಾಷ್ಟ್ರ
ಮಹಾ-ರಾಷ್ಟ್ರ-ಕ-ಗ-ಳಾಗಿ
ಮಹಾ-ರುದ್ರ-ರಿಳ್ದು
ಮಹಾ-ರುದ್ರರು
ಮಹಾರ್ನ್ನವ
ಮಹಾ-ಲಕ್ಷ್ಮಾ್ಯಸ್ತಸ್ಯಾಃ
ಮಹಾ-ಲಕ್ಷ್ಮಿ
ಮಹಾ-ಲಕ್ಷ್ಮಿಯ
ಮಹಾ-ಲಿಂಗೇಶ್ವರ
ಮಹಾ-ವಡ್ಡ
ಮಹಾ-ವಡ್ಡವ್ಯವ-ಹಾರಿ
ಮಹಾ-ವಡ್ಡವ್ಯವ-ಹಾರಿ-ಗಳ
ಮಹಾ-ವಡ್ಡವ್ಯವ-ಹಾರಿ-ಗಳು
ಮಹಾ-ವಿದ್ವಾಂಸ-ರಾ-ಗಿದ್ದ-ರೆಂಬುದ-ರಲ್ಲಿ
ಮಹಾ-ವಿಭವದಿ
ಮಹಾ-ವೀರ
ಮಹಾ-ವೀರ-ನಿಂದಲೇ
ಮಹಾ-ವೀರನೇ
ಮಹಾವ್ಯವಸ್ಥೆ-ಯನ್ನು
ಮಹಾ-ಶಬ್ದವಂ
ಮಹಾ-ಶಾ-ಸನ-ವನ್ನು
ಮಹಾ-ಶಿಲಾಸ್ತಮ್ಬ
ಮಹಾಶ್ರೀ
ಮಹಾ-ಸತಿ
ಮಹಾ-ಸತಿಯ
ಮಹಾ-ಸತಿ-ಯರ
ಮಹಾ-ಸತಿ-ಯ-ರಾದ-ವರ
ಮಹಾ-ಸತಿ-ಯಾಗಿದೆ
ಮಹಾ-ಸತಿ-ಯಾದ
ಮಹಾ-ಸಭೆಯ
ಮಹಾ-ಸಭೆ-ಯ-ವರು
ಮಹಾ-ಸಭೆಯು
ಮಹಾ-ಸಮಾಂತಾಧಿ-ಪತಿ
ಮಹಾ-ಸ-ಮುದ್ರಮಾ
ಮಹಾ-ಸ-ಮುದ್ರ-ವೆಂದು
ಮಹಾ-ಸಾಂತಾಧಿ-ಪತಿ
ಮಹಾ-ಸಾಮಂತ
ಮಹಾ-ಸಾಮಂತ-ನನ್ನಾಗಿ
ಮಹಾ-ಸಾಮಂತ-ನಾಗಿ
ಮಹಾ-ಸಾಮಂತ-ನಾ-ಗಿದ್ದ
ಮಹಾ-ಸಾಮಂತ-ನಾಗಿದ್ದರೂ
ಮಹಾ-ಸಾಮಂತ-ನಾಗಿದ್ದುದು
ಮಹಾ-ಸಾಮಂತ-ನಾಗುತ್ತಿದ್ದನು
ಮಹಾ-ಸಾಮಂತ-ನೆಂದು
ಮಹಾ-ಸಾಮಂತನೋ
ಮಹಾ-ಸಾಮಂತರ
ಮಹಾ-ಸಾಮಂತ-ರನ್ನು
ಮಹಾ-ಸಾಮಂತ-ರಾಗಿ
ಮಹಾ-ಸಾಮಂತ-ರಾ-ಗಿದ್ದ
ಮಹಾ-ಸಾಮಂತ-ರಾಗಿದ್ದ-ರೆಂಬು-ದನ್ನ
ಮಹಾ-ಸಾಮಂತ-ರಾದ
ಮಹಾ-ಸಾಮಂತ-ರಿಗೆ
ಮಹಾ-ಸಾಮಂತರು
ಮಹಾ-ಸಾಮಂತಾಧಿ-ಪತಿ
ಮಹಾ-ಸಾಮಂತಾಧಿ-ಪತಿ-ಗಳು
ಮಹಾ-ಸಾಮ್ರಾಜ್ಯ-ವಾ-ಗಿತ್ತು
ಮಹಾ-ಸೇನ-ಪುರದ
ಮಹಾ-ಸೇನಾ
ಮಹಾ-ಸೇನಾ-ಸ-ಮುದ್ರ
ಮಹಾ-ಸೇವಾ-ಸ-ಮುದ್ರ
ಮಹಾ-ಸೋಪಾನ-ವನ್ನು
ಮಹಾಸ್ವಾಮಿ
ಮಹಾಸ್ವಾಮಿ-ಯ-ರಿಗೆ
ಮಹಾಸ್ವಾಮಿ-ಯ-ವರ
ಮಹಾಸ್ವಾಮಿ-ಯವ-ರಿಂದ
ಮಹಾಸ್ವಾಮಿ-ಯ-ವರು
ಮಹಾಸ್ವಾಮೀ
ಮಹಾ-ಹೋಸಲ
ಮಹಾ-ಹೋಸಲ-ನಾಡ
ಮಹಾ-ಹೋಸಲ-ನಾ-ಡಿಗೆ
ಮಹಾ-ಹೋಸಲ-ನಾಡು
ಮಹಿತ
ಮಹಿ-ಪನ
ಮಹಿಮೆ
ಮಹಿಮೆ-ಯನ್ನು
ಮಹಿಮೋಂನತಿಕ್ಕೆ
ಮಹಿಮ್ನಾ
ಮಹಿ-ಳೆಯರ
ಮಹಿಳೆ-ಯ-ರಿಗೆ
ಮಹಿಳೆ-ಯರು
ಮಹಿಳೆ-ಯರೂ
ಮಹಿಳೆ-ಯಾದ
ಮಹಿ-ಳೆಯು
ಮಹಿಶುರ
ಮಹಿ-ಶೂರ
ಮಹಿ-ಶೂರ-ನ-ಗರದ
ಮಹಿ-ಶೂರ-ನ-ಗರ-ದಲ್ಲಿ
ಮಹಿಷ
ಮಹಿಷ-ಮರ್ದಿನಿ
ಮಹಿಷಿ
ಮಹಿಷಿಕಾ
ಮಹಿಷಿ-ಗಳ
ಮಹಿಷಿ-ಯರು-ಒಟ್ಟು
ಮಹಿಷಿ-ಯಾದ
ಮಹಿಷೀ
ಮಹಿಸೂರ
ಮಹೀಂ
ಮಹೀಕ-ರನೂ
ಮಹೀ-ಪತಿ
ಮಹೀಪಾನಾಂ
ಮಹೀ-ಪಾಲ-ಕರು
ಮಹೀ-ಪಾಳಕಃ
ಮಹೀಪೇ
ಮಹೀಭು-ಜನು-ವಿಷ್ಣು-ವರ್ಧನ
ಮಹೀ-ಮಂಡ-ಳ-ವನ್ನು
ಮಹೀ-ವಲ್ಲಭ
ಮಹೀಶೂರ
ಮಹೀಶೂರ-ದಳ-ವಾಯಿ
ಮಹೂರ್ತ-ದಲ್ಲಿ
ಮಹೂರ್ತ-ವನ್ನೂ
ಮಹೇಂದ್ರ
ಮಹೇಂದ್ರ-ನನ್ನು
ಮಹೇಂದ್ರನು
ಮಹೇಂದ್ರನೇ
ಮಹೇಶ್ವರ-ನಿಗೆ
ಮಹೇಶ್ವರರು
ಮಹೇಶ್ವರಿ
ಮಹೋಗ್ರಾಜಿ-ಯೊಳಾಂತಿ-ದಿರ್ಚಿ-ದದಟಿಂ
ಮಹೋಗ್ರಾಜಿ-ಯೊಳಾಂತಿ-ದಿರ್ಚ್ಚಿ-ದದಟಿಂ
ಮಹೋತ್ತಮ-ನಾಗಿದ್ದನು
ಮಹೋತ್ಸವ
ಮಹೋತ್ಸವ-ದಲ್ಲಿ
ಮಹೋತ್ಸವ-ವ-ಅನ್ನು
ಮಹೋತ್ಸವ-ವಾಯಿ-ತೆಂದು
ಮಾ
ಮಾಂಡ-ಲಿಕ
ಮಾಂಡ-ಲಿಕ-ನಾಗಿ
ಮಾಂಡ-ಲಿಕ-ನಾ-ಗಿದ್ದ
ಮಾಂಡ-ಲಿಕ-ನಾಗಿದ್ದನು
ಮಾಂಡ-ಲಿಕ-ನಾದ
ಮಾಂಡ-ಲಿಕ-ನಿರ-ಬ-ಹುದು
ಮಾಂಡಲಿ-ಕನೂ
ಮಾಂಡ-ಲಿಕರ
ಮಾಂಡ-ಲಿಕ-ರನ್ನಾಗಿ
ಮಾಂಡ-ಲಿಕ-ರಾಗಿ
ಮಾಂಡ-ಲಿಕ-ರಾ-ಗಿದ್ದ
ಮಾಂಡ-ಲಿಕ-ರಾಗಿದ್ದ-ರೆಂದು
ಮಾಂಡ-ಲಿಕ-ರಾಗಿದ್ದ-ರೆಂಬುದು
ಮಾಂಡ-ಲಿಕ-ರಿಗೆ
ಮಾಂಡ-ಲಿಕರು
ಮಾಂಡ-ಲಿಕ-ರೊಡನೆ
ಮಾಂಡ-ಲೀಕ
ಮಾಂಡ-ಲೀಕ-ನಾಗಿ
ಮಾಂಡ-ಲೀಕ-ನಾ-ಗಿದ್ದ
ಮಾಂಡ-ಲೀಕ-ನಾದ
ಮಾಂಡಲೀ-ಕನೂ
ಮಾಂಡ-ಲೀಕ-ರಾಗಿ
ಮಾಂಡ-ಲೀಕ-ರಾ-ಗಿದ್ದ
ಮಾಂಡಲೀ-ಕರು
ಮಾಂಡವ್ಯ
ಮಾಂನ್ಯ
ಮಾಂನ್ಯ-ಗಳನು
ಮಾಂಬಳಿ
ಮಾಂಬಳ್ಳಿ
ಮಾಂಸ
ಮಾಂಸ-ವನ್ನು
ಮಾಕಣಬ್ಬೆ
ಮಾಕಣಬ್ಬೆ-ಯನ್ನು
ಮಾಕಣಬ್ಬೆ-ಯರ
ಮಾಕಲೆ
ಮಾಕಲೆ-ಯರ
ಮಾಕವ್ವೆ
ಮಾಕವ್ವೆ-ಯರ
ಮಾಕ-ಸ-ಮುದ್ರ-ವೆಂಬ
ಮಾಕುಂದ-ಮು-ಕುಂದ
ಮಾಕು-ಬಳ್ಳಿ-ಮಾಕ-ವಳ್ಳಿ
ಮಾಕೇಶ್ವರ
ಮಾಗಡಿ
ಮಾಗಡಿಯ
ಮಾಗಡಿ-ಯಿಂದ
ಮಾಗಣಿ
ಮಾಗಣಿಗೆ
ಮಾಗಣಿಯ
ಮಾಗಣಿ-ಯಾ-ಗಿತ್ತು
ಮಾಗಣಿ-ಯೊಳಗೆ
ಮಾಗಣೆಗೆ
ಮಾಗ-ಣೆಯ
ಮಾಗ-ನೂರು
ಮಾಗಲ
ಮಾಗಳಿ
ಮಾಗಿ-ಪುದು
ಮಾಗಿ-ಯ-ಮಹ-ದೇವ-ಸೆಟ್ಟಿ
ಮಾಗಿರ್ಪುದು
ಮಾಗುಂಡರಕಿಲ್ಲ
ಮಾಘ
ಮಾಘ-ಣಂದಿ
ಮಾಘ-ನಂದಿ
ಮಾಚ-ಗವುಡ
ಮಾಚ-ಗೌಂಡ
ಮಾಚಣ
ಮಾಚ-ಣನ
ಮಾಚಣ್ಣ
ಮಾಚಣ್ಣನ
ಮಾಚ-ದಂಡಾಧೀಶನು
ಮಾಚನ
ಮಾಚ-ನ-ಕಟ್ಟ
ಮಾಚ-ನ-ಕಟ್ಟಕ್ಕೆ
ಮಾಚ-ನ-ಕಟ್ಟದ
ಮಾಚ-ನ-ಕಟ್ಟ-ದಲ್ಲಿ
ಮಾಚ-ನ-ಕಟ್ಟ-ದೊಳಗೆ
ಮಾಚ-ನ-ಕಟ್ಟವು
ಮಾಚ-ನ-ಕಟ್ಟೆಯ
ಮಾಚ-ನ-ಹಳ್ಳಿ
ಮಾಚ-ನಾಯ-ಕ-ನಿಗೆ
ಮಾಚನು
ಮಾಚನೇ
ಮಾಚ-ಮಯ್ಯ
ಮಾಚ-ಮಯ್ಯನು
ಮಾಚ-ಯ-ನಾಯ-ಕನ
ಮಾಚಯ್ಯ
ಮಾಚಯ್ಯ-ದಂಡ-ನಾಯಕ
ಮಾಚಯ್ಯನ
ಮಾಚಯ್ಯ-ನನ್ನು
ಮಾಚಯ್ಯ-ನಿಗೆ
ಮಾಚಯ್ಯನು
ಮಾಚ-ಲಗಅಟ್ಟ
ಮಾಚ-ಲ-ಘಟ್ಟ
ಮಾಚ-ಲ-ಘಟ್ಟದ
ಮಾಚ-ಲ-ಘಟ್ಟವು
ಮಾಚ-ಲ-ರಾಣಿ
ಮಾಚಲೆ
ಮಾಚ-ಲೆ-ನಾರಿ
ಮಾಚ-ಲೆ-ಯರ
ಮಾಚ-ಳೇಶ್ವರ
ಮಾಚ-ವ-ಳಲು
ಮಾಚವ್ವೆ
ಮಾಚವ್ವೆ-ಯನ್ನೂ
ಮಾಚವ್ವೆ-ಯರ
ಮಾಚ-ಸಅ-ಮುದ್ರ
ಮಾಚ-ಸ-ಮುದ್ರ
ಮಾಚ-ಸ-ಮುದ್ರದ
ಮಾಚಿ-ಕಬ್ಬೆಯ
ಮಾಚಿಕೆ
ಮಾಚಿ-ಕೆಯ
ಮಾಚಿ-ಕೆಯು
ಮಾಚಿ-ಗ-ವುಡನ
ಮಾಚಿ-ಗವುಡಿ
ಮಾಚಿಗೆ-ಹಳ್ಳಿ-ಮಾಚ-ನ-ಹಳ್ಳಿ
ಮಾಚಿ-ದೇವ
ಮಾಚಿ-ದೇವನ
ಮಾಚಿ-ದೇವ-ನನ್ನು
ಮಾಚಿ-ದೇವನು
ಮಾಚಿ-ನಾಯ-ಕ-ನ-ಹಳ್ಳಿ
ಮಾಚಿ-ಮಯ್ಯ-ನಿಗೆ
ಮಾಚಿ-ಮಯ್ಯನು
ಮಾಚಿ-ಯಕ್ಕ
ಮಾಚಿ-ರಾಜ
ಮಾಚಿ-ರಾಜಂ
ಮಾಚಿ-ರಾ-ಜಂಗೆ
ಮಾಚಿ-ರಾಜನ
ಮಾಚಿ-ರಾಜ-ನನ್ನು
ಮಾಚಿ-ರಾಜನು
ಮಾಚಿ-ರಾಜನೂ
ಮಾಚಿ-ರಾಜ-ರಾಗಿದ್ದಾ-ರೆಂದು
ಮಾಚಿ-ರಾಜ-ರೆಲ್ಲರೂ
ಮಾಚಿ-ಸೆಟ್ಟಿಯು
ಮಾಚೀ-ದೇವ
ಮಾಚೀ-ದೇವನ
ಮಾಚೆ-ಗವುಡ
ಮಾಚೆ-ಗೌಡನ
ಮಾಚೆ-ದೇವ
ಮಾಚೆಯ
ಮಾಚೆಯ-ನಾಯಕ
ಮಾಚೆಯ-ನಾಯ-ಕನ
ಮಾಚೆಯ-ನಾಯ-ಕ-ನಿಗೆ
ಮಾಚೆಯ-ನಾಯ-ಕನು
ಮಾಚೆಯ-ನಾಯ-ಕ-ನೆಂಬ
ಮಾಚೆ-ಯನು
ಮಾಚೇಶ್ವರ
ಮಾಚೋಜ
ಮಾಚೋಜನ
ಮಾಚೋಜನು
ಮಾಚೋಜನುಂ
ಮಾಚೋಜನೂ
ಮಾಚೋಜನೇ
ಮಾಡ-ತಕ್ಕ
ಮಾಡ-ತೊಡಗಿ-ದರು
ಮಾಡತೊಡಗಿದ-ರೆಂದು
ಮಾಡದೇ
ಮಾಡ-ಬಹು-ದಾದ
ಮಾಡ-ಬ-ಹುದು
ಮಾಡ-ಬೇಕಾಗಿತ್ತು
ಮಾಡ-ಬೇಕಾಗಿದ್ದ
ಮಾಡ-ಬೇ-ಕಾದ
ಮಾಡ-ಬೇಕು
ಮಾಡ-ಬೇಕೆಂದು
ಮಾಡಲ
ಮಾಡಲಾ-ಗದು
ಮಾಡ-ಲಾ-ಗಿತ್ತು
ಮಾಡ-ಲಾಗಿದೆ
ಮಾಡ-ಲಾಗಿ-ದೆ-ಯೆಂದು
ಮಾಡ-ಲಾಗುತ್ತದೆ
ಮಾಡ-ಲಾಗುತ್ತಿತ್ತು
ಮಾಡಲಾ-ಗುತ್ತಿತ್ತೆಂದು
ಮಾಡ-ಲಾಗುತ್ತಿದೆ
ಮಾಡಲಾಯಿತಾ-ದರೂ
ಮಾಡ-ಲಾ-ಯಿತು
ಮಾಡಲಾಯಿ-ತೆಂದು
ಮಾಡಲಿ
ಮಾಡ-ಲಿಲ್ಲ
ಮಾಡಲು
ಮಾಡಲೋಸುಗ
ಮಾಡಲ್ಪಟ್ಟ
ಮಾಡವ
ಮಾಡಿ
ಮಾಡಿ-ಕೊಂಡ
ಮಾಡಿ-ಕೊಂಡನು
ಮಾಡಿ-ಕೊಂಡ-ನೆಂದು
ಮಾಡಿ-ಕೊಂಡರು
ಮಾಡಿ-ಕೊಂಡ-ರೆಂದು
ಮಾಡಿ-ಕೊಂಡಾಗ
ಮಾಡಿ-ಕೊಂಡಿತ್ತೆಂದು
ಮಾಡಿ-ಕೊಂಡಿದೆ
ಮಾಡಿ-ಕೊಂಡಿದ್ದ
ಮಾಡಿ-ಕೊಂಡಿದ್ದ-ನೆಂದು
ಮಾಡಿ-ಕೊಂಡಿದ್ದ-ವನು
ಮಾಡಿ-ಕೊಂಡಿದ್ದಾರೆ
ಮಾಡಿ-ಕೊಂಡಿರ-ತಕ್ಕ
ಮಾಡಿ-ಕೊಂಡಿರ-ಬ-ಹುದು
ಮಾಡಿ-ಕೊಂಡಿರುವ
ಮಾಡಿ-ಕೊಂಡಿರು-ವ-ವರ
ಮಾಡಿ-ಕೊಂಡಿ-ರು-ವು-ದಿಲ್ಲ
ಮಾಡಿ-ಕೊಂಡಿರು-ವುದು
ಮಾಡಿ-ಕೊಂಡು
ಮಾಡಿ-ಕೊಂಡೆ
ಮಾಡಿ-ಕೊಟ್ಟ
ಮಾಡಿ-ಕೊಟ್ಟ-ನೆಂದೂ
ಮಾಡಿ-ಕೊಟ್ಟ-ರೆಂದು
ಮಾಡಿ-ಕೊಟ್ಟಿತು
ಮಾಡಿ-ಕೊಟ್ಟಿದ್ದ
ಮಾಡಿ-ಕೊಟ್ಟಿದ್ದನು
ಮಾಡಿ-ಕೊಟ್ಟಿದ್ದಾರೆ
ಮಾಡಿ-ಕೊಟ್ಟಿರ-ಬ-ಹುದು
ಮಾಡಿ-ಕೊಟ್ಟು
ಮಾಡಿ-ಕೊಡ-ಲಾ-ಗಿತ್ತು
ಮಾಡಿ-ಕೊಡ-ಲಾಗಿದೆ
ಮಾಡಿ-ಕೊಡ-ಲಾಗಿ-ರುವುದು
ಮಾಡಿ-ಕೊಡ-ಲಾಗುತ್ತಿತ್ತು
ಮಾಡಿ-ಕೊಡುತ್ತಾನೆ
ಮಾಡಿ-ಕೊಡುತ್ತಾರೆ
ಮಾಡಿ-ಕೊಡುತ್ತಿದ್ದ-ರೆಂದು
ಮಾಡಿ-ಕೊಡುವ
ಮಾಡಿ-ಕೊಡೋಣ
ಮಾಡಿ-ಕೊಳ್ಳದೆ
ಮಾಡಿ-ಕೊಳ್ಳದೇ
ಮಾಡಿ-ಕೊಳ್ಳ-ಬ-ಹುದು
ಮಾಡಿ-ಕೊಳ್ಳಾಗಿದೆ
ಮಾಡಿ-ಕೊಳ್ಳು-ತಿದ್ದ
ಮಾಡಿ-ಕೊಳ್ಳುತ್ತಾನೆ
ಮಾಡಿ-ಕೊಳ್ಳುತ್ತಾರೆ
ಮಾಡಿ-ಕೊಳ್ಳುತ್ತಿದ್ದ
ಮಾಡಿ-ಕೊಳ್ಳುತ್ತಿದ್ದ-ರೆಂದು
ಮಾಡಿ-ಕೊಳ್ಳುತ್ತಿದ್ದ-ರೆಂದೂ
ಮಾಡಿ-ಕೊಳ್ಳುತ್ತಿದ್ದ-ವ-ರನ್ನು
ಮಾಡಿ-ಕೊಳ್ಳುವ-ವ-ರನ್ನು
ಮಾಡಿ-ಕೊಳ್ಳು-ವು-ದಕ್ಕೆ
ಮಾಡಿ-ಕೊಳ್ಳು-ವು-ದರ
ಮಾಡಿ-ಕೊಳ್ಳು-ವುದು
ಮಾಡಿದ
ಮಾಡಿದಂ
ಮಾಡಿ-ದಂತಹ
ಮಾಡಿ-ದಂತಾ
ಮಾಡಿ-ದಂತಿದೆ
ಮಾಡಿ-ದಂತೆ
ಮಾಡಿ-ದಂಥಾ
ಮಾಡಿ-ದಡೀ
ಮಾಡಿ-ದನು
ಮಾಡಿ-ದ-ನೆಂದಿದೆ
ಮಾಡಿ-ದ-ನೆಂದು
ಮಾಡಿ-ದ-ನೆಂದೂ
ಮಾಡಿ-ದರು
ಮಾಡಿ-ದ-ರೆಂದಿದೆ
ಮಾಡಿ-ದ-ರೆಂದು
ಮಾಡಿ-ದ-ರೆಂದೂ
ಮಾಡಿ-ದಲ್ಲಿ
ಮಾಡಿ-ದ-ಳೆಂದು
ಮಾಡಿ-ದ-ವನ
ಮಾಡಿ-ದ-ವನು
ಮಾಡಿ-ದ-ವ-ನೆಂದರೆ
ಮಾಡಿ-ದವ-ರಲ್ಲಿ
ಮಾಡಿ-ದ-ವ-ರಿಗೆ
ಮಾಡಿ-ದ-ವರು
ಮಾಡಿದಾ
ಮಾಡಿ-ದಾಗ
ಮಾಡಿ-ದುದಕ್ಕಾಗಿ
ಮಾಡಿ-ದು-ದನ್ನು
ಮಾಡಿ-ದು-ದರ
ಮಾಡಿ-ದು-ದ-ರಿಂದ
ಮಾಡಿ-ದು-ದಾಗಿ
ಮಾಡಿದೆ
ಮಾಡಿದ್ದ
ಮಾಡಿದ್ದಕ್ಕಾಗಿ
ಮಾಡಿದ್ದಕ್ಕೆ
ಮಾಡಿದ್ದ-ನೆಂದು
ಮಾಡಿದ್ದನ್ನು
ಮಾಡಿದ್ದನ್ನು-ಹೇ-ಳಿದೆ
ಮಾಡಿದ್ದರ
ಮಾಡಿದ್ದ-ರಿಂದ
ಮಾಡಿದ್ದರು
ಮಾಡಿದ್ದ-ರೆಂದು
ಮಾಡಿದ್ದಲಿ
ಮಾಡಿದ್ದ-ಳೆಂದು
ಮಾಡಿದ್ದಾಗಿ
ಮಾಡಿದ್ದಾನೆ
ಮಾಡಿದ್ದಾರ
ಮಾಡಿದ್ದಾರೆ
ಮಾಡಿದ್ದಾ-ರೆಂದು
ಮಾಡಿಯೇ
ಮಾಡಿರ
ಮಾಡಿ-ರ-ಬ-ಹುದು
ಮಾಡಿ-ರ-ಬಹು-ದೆಂದು
ಮಾಡಿ-ರುತ್ತಾನೆ
ಮಾಡಿ-ರುವ
ಮಾಡಿ-ರು-ವಂತೆ
ಮಾಡಿ-ರುವರು
ಮಾಡಿ-ರುವ-ರೆಂದು
ಮಾಡಿ-ರು-ವು-ದನ್ನು
ಮಾಡಿ-ರುವು-ದ-ರಿಂದಲೂ
ಮಾಡಿ-ರುವುದು
ಮಾಡಿ-ಸಲಾಯಿ-ತೆಂದು
ಮಾಡಿಸಿ
ಮಾಡಿ-ಸಿ-ಕೊ-ಟಿ-ರುವ
ಮಾಡಿ-ಸಿ-ಕೊಟ್ಟ
ಮಾಡಿ-ಸಿ-ಕೊಟ್ಟಂತೆ
ಮಾಡಿ-ಸಿ-ಕೊಟ್ಟನು
ಮಾಡಿ-ಸಿ-ಕೊಟ್ಟರು
ಮಾಡಿ-ಸಿ-ಕೊಟ್ಟ-ರೆಂದು
ಮಾಡಿ-ಸಿ-ಕೊಟ್ಟ-ವರು
ಮಾಡಿ-ಸಿ-ಕೊಟ್ಟಿದ್ದನ್ನು
ಮಾಡಿ-ಸಿ-ಕೊಟ್ಟಿದ್ದರೆ
ಮಾಡಿ-ಸಿ-ಕೊಟ್ಟಿದ್ದಾನೆ
ಮಾಡಿ-ಸಿ-ಕೊಟ್ಟಿದ್ದಾರೆ
ಮಾಡಿ-ಸಿ-ಕೊಟ್ಟಿದ್ದಾಳೆ
ಮಾಡಿ-ಸಿ-ಕೊಟ್ಟಿರುತ್ತಾರೆ
ಮಾಡಿ-ಸಿ-ಕೊಟ್ಟಿ-ರುವುದ-ರಿಂದ
ಮಾಡಿ-ಸಿ-ಕೊಟ್ಟಿ-ರುವುದು
ಮಾಡಿ-ಸಿ-ಕೊಡುತ್ತಾನೆ
ಮಾಡಿ-ಸಿ-ಕೊಡುತ್ತಾರೆ
ಮಾಡಿ-ಸಿ-ಕೊಡುತ್ತಾಳೆ
ಮಾಡಿ-ಸಿ-ಕೊಡುತ್ತಿದ್ದರು
ಮಾಡಿ-ಸಿದ
ಮಾಡಿ-ಸಿದಂ
ಮಾಡಿ-ಸಿ-ದಂತೆ
ಮಾಡಿ-ಸಿ-ದ-ನಿನ್ತೀ
ಮಾಡಿ-ಸಿ-ದನು
ಮಾಡಿ-ಸಿ-ದ-ನೆಂದಿದೆ
ಮಾಡಿ-ಸಿ-ದ-ನೆಂದು
ಮಾಡಿ-ಸಿ-ದರು
ಮಾಡಿ-ಸಿ-ದ-ರೆಂದು
ಮಾಡಿ-ಸಿ-ದ-ರೆಂದೂ
ಮಾಡಿ-ಸಿ-ದ-ರೆಂಬುದು
ಮಾಡಿ-ಸಿ-ದಳು
ಮಾಡಿ-ಸಿ-ದ-ಳೆಂದು
ಮಾಡಿ-ಸಿ-ದಾಗ
ಮಾಡಿ-ಸಿ-ದು-ದನ್ನು
ಮಾಡಿ-ಸಿ-ದು-ದರ
ಮಾಡಿ-ಸಿದೆ
ಮಾಡಿ-ಸಿದ್ದ
ಮಾಡಿ-ಸಿದ್ದ-ನಂತೆ
ಮಾಡಿ-ಸಿದ್ದನ್ನು
ಮಾಡಿ-ಸಿದ್ದಾನೆ
ಮಾಡಿ-ಸಿದ್ದಾರೆ
ಮಾಡಿ-ಸಿದ್ದಾ-ರೆಂದು
ಮಾಡಿ-ಸಿದ್ದಾಳೆ
ಮಾಡಿ-ಸಿರ
ಮಾಡಿ-ಸಿ-ರ-ಬ-ಹುದು
ಮಾಡಿ-ಸಿ-ರ-ಬಹು-ದೆಂದು
ಮಾಡಿ-ಸಿ-ರುವ
ಮಾಡಿ-ಸಿ-ರು-ವಂತೆ
ಮಾಡಿಸು
ಮಾಡಿ-ಸುತ್ತಾನೆ
ಮಾಡಿ-ಸುತ್ತಾರೆ
ಮಾಡಿ-ಸುತ್ತಾಳೆ
ಮಾಡಿ-ಸುತ್ತಿದ್ದ-ನೆಂದು
ಮಾಡಿ-ಸುತ್ತಿದ್ದ-ರೆಂದು
ಮಾಡಿ-ಸುತ್ತಿದ್ದಾಗ
ಮಾಡಿ-ಸುತ್ತೇನೆ
ಮಾಡಿ-ಸುವ
ಮಾಡಿ-ಸು-ವ-ವ-ರಿಗೆ
ಮಾಡಿ-ಸು-ವು-ದಕ್ಕೆ
ಮಾಡಿ-ಸು-ವುದ-ರಿಂದ
ಮಾಡಿ-ಸು-ವುದು
ಮಾಡು
ಮಾಡು-ಗೊಡಗಿ
ಮಾಡು-ಗೊಡಗೆ-ಯಾಗಿ
ಮಾಡು-ತಿದ್ದ-ರೆಂದು
ಮಾಡುತ್ತವೆ
ಮಾಡುತ್ತಾ
ಮಾಡುತ್ತಾನೆ
ಮಾಡುತ್ತಾರೆ
ಮಾಡುತ್ತಾ-ರೆಂದೂ
ಮಾಡುತ್ತಾಳೆ
ಮಾಡುತ್ತಿತ್ತು
ಮಾಡುತ್ತಿದ-ರೆಂದು
ಮಾಡುತ್ತಿದೆ
ಮಾಡುತ್ತಿದ್ದ
ಮಾಡುತ್ತಿದ್ದಂತೆ
ಮಾಡುತ್ತಿದ್ದನು
ಮಾಡುತ್ತಿದ್ದ-ನೆಂದು
ಮಾಡುತ್ತಿದ್ದ-ನೆಂದೂ
ಮಾಡುತ್ತಿದ್ದರು
ಮಾಡುತ್ತಿದ್ದ-ರೆಂದು
ಮಾಡುತ್ತಿದ್ದ-ರೆಂದೂ
ಮಾಡುತ್ತಿದ್ದ-ವರು
ಮಾಡುತ್ತಿದ್ದ-ವರೇ
ಮಾಡುತ್ತಿದ್ದವು
ಮಾಡುತ್ತಿದ್ದಾಗ
ಮಾಡುತ್ತಿದ್ದಿರ-ಬ-ಹುದು
ಮಾಡುತ್ತಿದ್ದು-ದಕ್ಕಾಗಿ
ಮಾಡುತ್ತಿದ್ದು-ದ-ರಿಂದ
ಮಾಡುತ್ತಿದ್ದುದು
ಮಾಡುತ್ತಿ-ರಲು
ಮಾಡುತ್ತಿ-ರುವ
ಮಾಡುತ್ತಿರ್ದ್ದಪ್ಪ-ರೆಂಬಂತೆರ-ದೊಳ್
ಮಾಡುತ್ತೀರಿ
ಮಾಡುತ್ತೇನೆ
ಮಾಡುತ್ತೇನೆಂದು
ಮಾಡುವ
ಮಾಡು-ವಂತಹ
ಮಾಡು-ವಂತಾಗಿ
ಮಾಡು-ವಂತೆ
ಮಾಡು-ವ-ರೆಂದು
ಮಾಡು-ವಲ್ಲಿ
ಮಾಡು-ವ-ವನ
ಮಾಡು-ವ-ವರ
ಮಾಡು-ವ-ವ-ರನ್ನು
ಮಾಡು-ವ-ವ-ರಿಗೆ
ಮಾಡು-ವ-ವರು
ಮಾಡು-ವ-ವರೂ
ಮಾಡು-ವಾಗ
ಮಾಡು-ವಾಗಲೂ
ಮಾಡು-ವು-ದಕ್ಕೆ
ಮಾಡು-ವು-ದರ
ಮಾಡು-ವುದ-ರಲ್ಲಿ
ಮಾಡು-ವು-ದ-ರಿಂದ
ಮಾಡು-ವುದ-ರೊಂದಿಗೆ
ಮಾಡು-ವುದಾ-ದರೆ
ಮಾಡು-ವುದು
ಮಾಡು-ವುದೂ
ಮಾಡ್ಸಿ
ಮಾಣಡ
ಮಾಣದ
ಮಾಣಿ
ಮಾಣಿಕ
ಮಾಣಿ-ಕ-ಭಂಡಾರಿ-ಗ-ಳಾಗಿ-ರುತ್ತಿದ್ದ-ರೆಂದು
ಮಾಣಿ-ಕ-ಭಂಡಾರಿ-ಗಳು
ಮಾಣಿ-ಕ-ಸೆಟ್ಟಿಯ
ಮಾಣಿ-ಕಾ-ಚಾರಿ
ಮಾಣಿಕ್ಯ
ಮಾಣಿಕ್ಯ-ಪೊಳಲ
ಮಾಣಿಕ್ಯ-ಪೊಳಲು
ಮಾಣಿಕ್ಯ-ಭಂಡಾರದ
ಮಾಣಿಕ್ಯ-ಭಂಡಾರಿ
ಮಾಣಿಕ್ಯ-ವೊಳಲ
ಮಾಣಿಕ್ಯ-ವೊಳಲು
ಮಾಣಿಕ್ಯ-ವೊಳಲೆಂಬ
ಮಾಣಿ-ಮಾಡೆಗೆ
ಮಾಣಿ-ಯೊಳ-ಗಣ
ಮಾತ-ನಾಡುತ್ತಿದ್ದರು
ಮಾತ-ನಾಡುವ
ಮಾತನ್ನು
ಮಾತಾ-ಗಿತ್ತು
ಮಾತು
ಮಾತೃ-ಕೆಯರ
ಮಾತೃಪ್ರೇಮ-ವನ್ನು
ಮಾತೃ-ಭಾಷೆಗೆ
ಮಾತೇಂ
ಮಾತ್ರ
ಮಾತ್ರಕ್ಕೆ
ಮಾದ-ಗವು-ಡಿಯ
ಮಾದ-ಗವು-ಡಿ-ಯನ್ನೂ
ಮಾದ-ಗವು-ಡಿ-ಯರ
ಮಾದ-ಜೀಯನ
ಮಾದಡಿ
ಮಾದಣ
ಮಾದಣ್ಣ
ಮಾದಣ್ಣ-ಗ-ಳಿಗೆ
ಮಾದಣ್ಣನ
ಮಾದಣ್ಣ-ನಿಗೆ
ಮಾದಣ್ಣನು
ಮಾದಪ್ಪ
ಮಾದಪ್ಪ-ದಂಡ-ನಾಯ-ಕನೂ
ಮಾದಪ್ಪ-ದಣ್ನಾಯ-ಕರ
ಮಾದಪ್ಪನು
ಮಾದರ-ಗವುಡಿ
ಮಾದ-ರ-ಸನ
ಮಾದ-ರ-ಸನು
ಮಾದ-ರಸ-ನೆಂಬ
ಮಾದರಿ
ಮಾದರಿ-ಗ-ಳಾಗಿದ್ದ-ವೆಂದು
ಮಾದರಿ-ಯಂತಿದೆ
ಮಾದರಿ-ಯಲ್ಲಿ
ಮಾದಲ-ಗೆರೆ
ಮಾದಲ-ದೇವಿ
ಮಾದಲ-ದೇವಿ-ಯರು
ಮಾದಲ-ಮಹ-ದೇವಿ-ಯರು
ಮಾದಲ-ಮಹಾ-ದೇವಿ
ಮಾದಲ-ಮಹಾ-ದೇವಿ-ಯನ್ನು
ಮಾದಲ-ಮಹಾ-ದೇವಿ-ಯರು
ಮಾದಲೇಶ್ವರ
ಮಾದಳ
ಮಾದವ್ವೆ
ಮಾದಾ-ಪುರ
ಮಾದಾ-ಪುರದ
ಮಾದಾ-ಪುರವೂ
ಮಾದಾರಿಕೆ
ಮಾದಿ
ಮಾದಿಗ
ಮಾದಿ-ಗ-ಉಡ
ಮಾದಿ-ಗರು
ಮಾದಿ-ಗ-ರುಳ
ಮಾದಿ-ಗ-ವುಡ
ಮಾದಿ-ಗ-ವುಡಂಗೆ
ಮಾದಿ-ಗ-ವುಡನ
ಮಾದಿ-ಗ-ವುಡ-ನಿಗೆ
ಮಾದಿ-ಗೌಡ
ಮಾದಿ-ದೇವ
ಮಾದಿ-ದೇವನ
ಮಾದಿ-ಯಕ್ಕ
ಮಾದಿ-ಯಕ್ಕ-ನೆಂಬ
ಮಾದಿ-ಯಣ್ಣ
ಮಾದಿ-ರಾಜ
ಮಾದಿ-ರಾಜನ
ಮಾದಿ-ರಾಜ-ನನ್ನು
ಮಾದಿ-ರಾಜನು
ಮಾದಿ-ರಾಜರ
ಮಾದಿ-ರಾಜ-ಹೆಗ್ಗಡೆಯು
ಮಾದಿ-ವೆಗ್ಗಡೆ
ಮಾದಿ-ವೆಗ್ಗಡೆಯು
ಮಾದಿ-ಹಳ್ಳಿ
ಮಾದಿ-ಹಳ್ಳಿಯ
ಮಾದಿ-ಹಳ್ಳಿ-ಯನ್ನು
ಮಾದಿ-ಹಳ್ಳಿ-ಯಲ್ಲಿ
ಮಾದಿ-ಹಳ್ಳಿಯು
ಮಾದೆ-ಗವುಂಡನು
ಮಾದೆ-ಗೌಡನ
ಮಾದೆಯ
ಮಾದೆಯ-ನಾಯಕ
ಮಾದೆಯ-ನಾಯ-ಕನ
ಮಾದೆಯ-ನಾಯ-ಕನು
ಮಾದೆ-ಯನೂ
ಮಾದೆ-ಹಳ್ಳಿ
ಮಾದೇ-ಗವುಂಡನು
ಮಾದೇ-ಗೌಡನ
ಮಾದೇವ
ಮಾದೇವಂ
ಮಾದೇವ-ನನ್ನು
ಮಾದೇವನ್
ಮಾದೇವಿ
ಮಾದೇವಿತ್ವಂ
ಮಾದೇವಿ-ಯ-ರಂತೆ
ಮಾದೇಶ್ವರ
ಮಾದ್ಯತ್
ಮಾದ್ಯತ್ತು
ಮಾಧಮ-ಶರ್ಮ-ನಿಗೆ
ಮಾಧವ
ಮಾಧವ-ಚಂದ್ರ
ಮಾಧವ-ಚಂದ್ರನ
ಮಾಧವ-ಚತುರ್ವೇದಿ
ಮಾಧವ-ಚೋಳ-ನ-ಹಳ್ಳಿಯ
ಮಾಧವ-ಚೋಳ-ಯನ-ಹಳ್ಳಿಯ
ಮಾಧವ-ಚೋಳ-ಯನ-ಹಳ್ಳಿಯೇ
ಮಾಧವ-ಜೀಯ-ನಿಗೆ
ಮಾಧವ-ಜೀಯ-ನಿದ್ದನು
ಮಾಧವ-ಜೀಯ-ರಿಗೆ
ಮಾಧವತ್ತಿ-ಮಾಧವ-ಶಕ್ತಿ
ಮಾಧವ-ದಂಡ-ನಾಯಕ
ಮಾಧವ-ದಂಡ-ನಾಯ-ಕನ
ಮಾಧವ-ದಂಡ-ನಾಯ-ಕ-ನಿಗೆ
ಮಾಧವ-ದಂಣಾ-ಯಕರುಂ
ಮಾಧವ-ದೇವರ
ಮಾಧವ-ದೇವ-ರಿಗೆ
ಮಾಧವ-ದೇವರು
ಮಾಧವ-ದೇವ-ರೊಳಗಾದ
ಮಾಧವನ
ಮಾಧವ-ನನ್ನು
ಮಾಧವ-ನಿಗೆ
ಮಾಧವನು
ಮಾಧವ-ನೆಂದೂ
ಮಾಧವ-ಪಟ್ಟ-ಣ-ವನ್ನಾಗಿ
ಮಾಧವ-ಪಟ್ಟ-ಣ-ವಾಗಿ
ಮಾಧವಪ್ಪೆರು-ಮಾಳುಕ್ಕು
ಮಾಧವರ
ಮಾಧವ-ರಾಯ
ಮಾಧವ-ರಿಗೂ
ಮಾಧವರು
ಮಾಧವಸ್ವಾಮಿ
ಮಾಧವಾಂಕ
ಮಾಧವಾ-ಚಾರ್ಯನ
ಮಾಧವಾ-ಚಾರ್ಯ-ನಿಗೆ
ಮಾಧವಾ-ಚಾರ್ಯರ
ಮಾಧವಾ-ಚಾರ್ಯ-ರಿಗೆ
ಮಾಧವಾ-ಚಾರ್ಯರು
ಮಾಧ್ಯಮ-ವನ್ನಾಗಿ
ಮಾಧ್ವ
ಮಾಧ್ವ-ಮ-ತಾನು-ಯಾಯಿ-ಗಳಿದ್ದ
ಮಾಧ್ವರ
ಮಾಧ್ವರು
ಮಾಧ್ವ-ಸಂಪ್ರ-ದಾಯ-ದ-ವರ
ಮಾನ
ಮಾನ-ಗ-ಳಿಂದ
ಮಾನ-ಗಳು
ಮಾನದ
ಮಾನ-ಭಂಗಕ್ಕೆ
ಮಾನ-ಮುನೀಂದ್ರರು
ಮಾನವ
ಮಾನ-ವ-ಚಾಲಿತ
ಮಾನ-ವ-ದುರ್ಗ-ವನ್ನು-ಮಾನವಿ
ಮಾನ-ವನ
ಮಾನ-ವ-ರೊಳು
ಮಾನ-ವ-ರೊಳ್
ಮಾನ-ವಾ-ಕಾರ
ಮಾನ-ವಾ-ಕಾರ-ವನ್ನು
ಮಾನ-ವಾ-ಗಿದ್ದು
ಮಾನ-ವೊಂದು
ಮಾನ-ಸ-ಗುರು-ಗಳು
ಮಾನ-ಸ-ರೂಪ-ವಾದುದೋ
ಮಾನಸ್ತಂಭ
ಮಾನಸ್ತಂಭದ
ಮಾನಸ್ತಂಭ-ವನ್ನು
ಮಾನಸ್ಥಂಬ
ಮಾನಸ್ಥಂಭ
ಮಾನಸ್ಥಂಭದ
ಮಾನಿಸ
ಮಾನಿ-ಸ-ಲೆಂಕರು
ಮಾನಿ-ಸ-ಲೆಂಕಿತಿ-ಯರು
ಮಾನಿ-ಸೆಟ್ಟಿಗೆ
ಮಾನ್ಯ
ಮಾನ್ಯಂ
ಮಾನ್ಯಕ್ಕೆ
ಮಾನ್ಯ-ಖೇಟ-ವನ್ನ
ಮಾನ್ಯ-ಗ-ಳನ್ನು
ಮಾನ್ಯ-ಗಳು
ಮಾನ್ಯ-ಗಳೂ
ಮಾನ್ಯಗ್ರಾಮ-ಗ-ಳಲ್ಲಿ
ಮಾನ್ಯತೆ
ಮಾನ್ಯ-ತೆಯೂ
ಮಾನ್ಯದ
ಮಾನ್ಯ-ದಲ್ಲಿ
ಮಾನ್ಯ-ಪುರ-ದಲ್ಲಿ
ಮಾನ್ಯವಂ
ಮಾನ್ಯ-ವನ್ನು
ಮಾನ್ಯ-ವಲ್ಲಿಂ
ಮಾನ್ಯ-ವಾಗಿ
ಮಾನ್ಯ-ವೆಲ್ಲಹ
ಮಾಬಲಣ್ಣನು
ಮಾಬ-ಲಯ್ಯ
ಮಾಬ-ಲಯ್ಯಂ
ಮಾಬಲಯ್ಯನ
ಮಾಬ-ಲಯ್ಯ-ನಿಗೆ
ಮಾಬಲಯ್ಯನು
ಮಾಬ-ಲಯ್ಯ-ನೆಂದೊಗೞದ-ರಾರ್
ಮಾಬಳಯ್ಯ
ಮಾಬಳ್ಳಿ
ಮಾಬ-ಹಳ್ಳಿ
ಮಾಮ-ರಿಯ
ಮಾಮಲೆ-ದಾರ
ಮಾಮಲೆ-ದಾರ್
ಮಾಮಲೇ-ದಾರ್
ಮಾಮಾರಿಯ
ಮಾಮೂಲಿ
ಮಾಯಕ-ಳಲ-ಯಿಕ
ಮಾಯಕ-ಳ-ಲಿಯಿಕ
ಮಾಯಣ
ಮಾಯಣನ
ಮಾಯಣ-ನ-ಪುರ
ಮಾಯಣ-ನೆಂಬ
ಮಾಯಣರ
ಮಾಯಣ್ಣ
ಮಾಯಣ್ಣನ
ಮಾಯಣ್ಣ-ನ-ಪುರದ
ಮಾಯಣ್ಣ-ನಿಗೆ
ಮಾಯಣ್ಣನು
ಮಾಯಣ್ಣನೂ
ಮಾಯಣ್ಣ-ನೆಂಬು-ವವ-ನಿಗೆ
ಮಾಯಪ್ಪ-ನಿಗೆ
ಮಾಯಪ್ಪ-ಹಳ್ಳಿ-ದೇಪ-ಸಾ-ಗರ
ಮಾಯಮ್ಮ
ಮಾಯಮ್ಮನ
ಮಾಯ-ಸಂದ್ರಕ್ಕೆ
ಮಾಯ-ಸಂದ್ರವೇ
ಮಾಯ-ಸ-ಮುದ್ರ
ಮಾಯ-ಸ-ಮುದ್ರ-ವಾಗಿದೆ
ಮಾಯಾ-ವಾದ
ಮಾಯಾ-ವಾದಿ
ಮಾಯಿ-ಗನ-ಕಟ್ಟೆ-ಗಳ
ಮಾಯಿ-ಗೌಂಡ
ಮಾಯಿ-ಗೌಡನ
ಮಾಯಿ-ತಂಮ
ಮಾಯಿ-ದೇವ
ಮಾಯಿ-ದೇವ-ರ-ಕೆರೆಯ
ಮಾಯಿ-ದೇವ-ರಿಗೆ
ಮಾಯಿ-ಲಂಗಿಯ
ಮಾಯಿ-ಲಂಗೆ
ಮಾಯಿ-ಲಂಗೆ-ಯಲ್ಲಿ
ಮಾಯಿ-ಲಿಂಗಿಗೆ
ಮಾಯಿ-ಲಿಂಗಿಯು
ಮಾಯಿ-ವೋ-ಜನ
ಮಾಯೀ-ದೇವ
ಮಾಯೀ-ದೇವರ
ಮಾರ
ಮಾರಃ
ಮಾರ-ಕಜ-ಭಕ್ತ
ಮಾರ-ಗದ್ಯಾಣ-ವೊಂದು
ಮಾರ-ಗವುಂಡನ
ಮಾರ-ಗವುಡ
ಮಾರ-ಗ-ವುಡಂಗೆ
ಮಾರ-ಗ-ವುಡನ
ಮಾರ-ಗ-ವುಡನೂ
ಮಾರ-ಗವು-ಡ-ನೆಂಬು-ವ-ವನು
ಮಾರ-ಗಾನ-ಹಳ್ಳಿ
ಮಾರ-ಗಾನ-ಹಳ್ಳಿ-ಯಲ್ಲಿದೆ
ಮಾರ-ಗಾ-ಮುಂಡ
ಮಾರ-ಗಾ-ಮುಂಡನ
ಮಾರ-ಗೂಳಿ
ಮಾರ-ಗೊಂಡ-ಗವುಡ
ಮಾರ-ಗೊಂಡ-ನ-ಹಳ್ಳಿ
ಮಾರ-ಗೊಂಡ-ನ-ಹಳ್ಳಿಗೆ
ಮಾರ-ಗೊಂಡ-ನ-ಹಳ್ಳಿ-ಯನ್ನು
ಮಾರ-ಗೊಂಡ-ಹಳ್ಳಿ
ಮಾರ-ಗೌಂಡ
ಮಾರ-ಗೌಂಡನ
ಮಾರ-ಗೌಡ
ಮಾರ-ಗೌಡನ
ಮಾರಣ್ಣ
ಮಾರಣ್ಣನು
ಮಾರ-ತಂಮನು
ಮಾರ-ತಮ್ಮನ
ಮಾರಥ
ಮಾರ-ದೇವ
ಮಾರ-ದೇವನ
ಮಾರ-ದೇವನು
ಮಾರ-ನಾಯಕ
ಮಾರ-ನಾಯ-ಕನ
ಮಾರ-ನಾಯ-ಕ-ನನ್ನೇ
ಮಾರ-ನಾಯ-ಕ-ನ-ಹಳ್ಳಿ
ಮಾರ-ನಾಯ-ಕ-ನಿಗೂ
ಮಾರ-ನಾಯ-ಕ-ನಿಗೆ
ಮಾರ-ನಾಯ-ಕನು
ಮಾರಪ್ಪ
ಮಾರಪ್ಪ-ಗೌಡ
ಮಾರಪ್ಪನೂ
ಮಾರ-ಭಕ್ತನ
ಮಾರ-ಮಯ್ಯನ
ಮಾರಮ್ಮನ
ಮಾರಯ್ಯ
ಮಾರಯ್ಯನ
ಮಾರಯ್ಯ-ನನ್ನು
ಮಾರಯ್ಯ-ನೆಂದು
ಮಾರಯ್ವೆ
ಮಾರಲ್ಪಡುತ್ತಿತ್ತು
ಮಾರವ್ವಿಗೆ
ಮಾರವ್ವೆ
ಮಾರವ್ವೆ-ಯನ್ನು
ಮಾರವ್ವೆಯ್ವೆ
ಮಾರ-ಶರ್ಮನ
ಮಾರ-ಸಿಂಗ
ಮಾರ-ಸಿಂಗ-ಗಾವುಂಡನು
ಮಾರ-ಸಿಂಹ
ಮಾರ-ಸಿಂಹ-ದೇವ
ಮಾರ-ಸಿಂಹ-ದೇವನ
ಮಾರ-ಸಿಂಹನ
ಮಾರ-ಸಿಂಹ-ನನ್ನು
ಮಾರ-ಸಿಂಹ-ನಲ್ಲಿ
ಮಾರ-ಸಿಂಹ-ನಲ್ಲೇ
ಮಾರ-ಸಿಂಹ-ನಿಗೆ
ಮಾರ-ಸಿಂಹನು
ಮಾರ-ಸಿಂಹನೂ
ಮಾರ-ಹಳ್ಳಿ
ಮಾರಾಂಡ
ಮಾರಾಂಡನೂ
ಮಾರಾಟ
ಮಾರಾಟಕ್ಕೆ
ಮಾರಾಟದ
ಮಾರಾಟ-ಮಾಡುತ್ತಾರೆ
ಮಾರಾಟ-ವಾದ
ಮಾರಾಯನ್
ಮಾರಿ
ಮಾರಿ-ಕೊಳ್ಳುತ್ತಾರೆ
ಮಾರಿ-ಕೊಳ್ಳುತ್ತಿದ್ದುದು
ಮಾರಿ-ಗುಡಿ
ಮಾರಿ-ಗುಡಿ-ಗಳು
ಮಾರಿ-ಗುಡಿಯ
ಮಾರಿ-ಗುಡಿ-ಯಾಗಿ-ರ-ಬ-ಹುದು
ಮಾರಿ-ಗುಡಿ-ಯಾಗಿ-ರುವ
ಮಾರಿತಿ
ಮಾರಿ-ದತ್ತ
ಮಾರಿ-ದರೆ
ಮಾರಿ-ರ-ಬ-ಹುದು
ಮಾರಿಲಿ
ಮಾರಿ-ಶೆಟ್ಟಿಯು
ಮಾರಿ-ಸೆಟ್ಟಿಯ
ಮಾರೀ-ಗುಡಿ-ಯನ್ನು
ಮಾರೀ-ಪೂಜೆ-ಯನ್ನು
ಮಾರು-ಗ-ನೆಂದು
ಮಾರು-ಗ-ನೆಂಬು-ವ-ವನು
ಮಾರು-ಗನೇ
ಮಾರು-ಗೋನ-ಹಳ್ಳಿ
ಮಾರುವ
ಮಾರು-ವ-ವರು
ಮಾರೂರ
ಮಾರೂ-ರು-ಗಳ
ಮಾರೂರು-ಹಳ್ಳಿ
ಮಾರೆಯ
ಮಾರೆಯ-ನಾಯಕ
ಮಾರೆಯ-ನಾಯ-ಕನ
ಮಾರೆಯ-ನಾಯ-ಕ-ನಿಗೂ
ಮಾರೆಯ-ನಾಯ-ಕ-ನಿಗೆ
ಮಾರೆಯ-ನಾಯ್ಕನ
ಮಾರೆಯ್ಯ-ನೆಂಬು-ವ-ವನೂ
ಮಾರೆ-ಹಳ್ಳಿ
ಮಾರೆ-ಹಳ್ಳಿಯ
ಮಾರೆ-ಹಳ್ಳಿ-ಯನ್ನು
ಮಾರೆ-ಹಳ್ಳಿ-ಯಲ್ಲಿ
ಮಾರೆ-ಹಳ್ಳಿಯು
ಮಾರೇನ-ಹಳ್ಳಿಯು
ಮಾರೇ-ಹಳ್ಳಿ
ಮಾರೇ-ಹಳ್ಳಿಗೆ
ಮಾರೇ-ಹಳ್ಳಿಯ
ಮಾರೇ-ಹಳ್ಳಿ-ಯನ್ನೂ
ಮಾರೇ-ಹಳ್ಳಿಯು
ಮಾರ್ಕ್ಕೊಂಡು
ಮಾರ್ಕ್ಕೋಲ-ಭೈ-ರವಂ
ಮಾರ್ಕ್ಸ್ವಿಲ್ಕ್್ಸ
ಮಾರ್ಗ
ಮಾರ್ಗ-ಗಳಿದ್ದವು
ಮಾರ್ಗ-ಗಳು
ಮಾರ್ಗ-ದ-ಕಟ್ಟೆ
ಮಾರ್ಗ-ದರ್ಶಕ-ರನ್ನಾಗಿ
ಮಾರ್ಗ-ದರ್ಶ-ಕರಾಗ
ಮಾರ್ಗ-ದರ್ಶಕ-ರಾದ
ಮಾರ್ಗ-ದರ್ಶನ
ಮಾರ್ಗ-ದರ್ಶನ-ದಲ್ಲಿ
ಮಾರ್ಗ-ವಾಗಿ
ಮಾರ್ಗವು
ಮಾರ್ಗವೂ
ಮಾರ್ಗೋನ-ಹಳ್ಳಿಯ
ಮಾರ್ಗ್ಗ-ಸಿರ
ಮಾರ್ಚನ-ಹಳ್ಳಿ-ಯನ್ನು
ಮಾರ್ಚ-ಹಳ್ಳಿ-ಯಲ್ಲೂ
ಮಾರ್ಚ್
ಮಾರ್ಚ್ರ
ಮಾರ್ತಾಂಡ
ಮಾರ್ತಾಂಡನುಂ
ಮಾರ್ತಾಂಡ-ನೆಂದು
ಮಾರ್ತಾಂಡ-ರಾದ
ಮಾರ್ತಾಂಡರು
ಮಾರ್ತಾಂಡರುಂ
ಮಾರ್ತಾಂಡ-ರೆಂದು
ಮಾರ್ತ್ತಾಂಡನುಂ
ಮಾರ್ಪಟ್ಟಿತು
ಮಾರ್ಪಟ್ಟಿತ್ತು
ಮಾರ್ಪಡಿ-ಸಬೇಕಾಗುತ್ತದೆ
ಮಾರ್ಪ-ಡಿಸಿ
ಮಾರ್ಪಡಿ-ಸಿ-ಕೊಂಡರು
ಮಾರ್ಪಾಡಾಗಿದ್ದವು
ಮಾರ್ಪಾಡಾಗಿ-ರುವುದು
ಮಾರ್ಪ್ಪಾಡಿ
ಮಾರ್ಪ್ಪೆನಾ-ನೆಂದೀ-ಗಳು
ಮಾರ್ವಾ-ದಂತೆ
ಮಾಲ-ಗಾರ-ನ-ಹಳ್ಳಿಯ
ಮಾಲನ-ಹಳ್ಳಿ
ಮಾಲಿಂಗಿಯ
ಮಾಲಿಕೆ
ಮಾಲಿತಮ
ಮಾಲೂರು
ಮಾಲೆ-ಗಾರ
ಮಾಲೆ-ಯ-ಹಳ್ಳಿ
ಮಾಲೆ-ಯ-ಹಳ್ಳಿ-ವೊಳಗಾದ
ಮಾಲ್ಯದ
ಮಾಳ-ಗುಂದ
ಮಾಳ-ಗುಂದ-ಮಾಳ-ಗೂರು
ಮಾಳ-ಗೂರಿನ
ಮಾಳ-ಗೂರಿ-ನಲ್ಲಿ
ಮಾಳ-ಗೂರು
ಮಾಳ-ಗೂರೇ
ಮಾಳವ
ಮಾಳ-ವ-ರಾಜ್ಯ
ಮಾಳವ್ವೆ
ಮಾಳವ್ವೆಯ
ಮಾಳವ್ವೆ-ಯ-ಕೆರೆ
ಮಾಳವ್ವೆ-ಯ-ಕೆರೆ-ಯನ್ನು
ಮಾಳಾನ-ಹಳ್ಳಿಯ
ಮಾಳಾನ-ಹಳ್ಳಿಯು
ಮಾಳಿಗೆ
ಮಾಳಿಗೆಯ
ಮಾಳಿಗೆ-ಯನ್ನು
ಮಾಳಿಗೆ-ಯಲ್ಲಿ
ಮಾಳಿಗೆ-ಯೂ-ರನ್ನಾಳುತ್ತಿದ್ದ
ಮಾಳಿಗೆ-ಯೂ-ರನ್ನು
ಮಾಳಿಗೆ-ಯೂರಿನ
ಮಾಳುಗಾಳ
ಮಾಳೆಯ
ಮಾಳೆಯ-ನ-ಹಳ್ಳಿ-ಯನ್ನು
ಮಾಳೆಯರ
ಮಾಳೇನ-ಹಳ್ಳಿ
ಮಾಳ್ಕಧೀಶಂ
ಮಾಳ್ಪ
ಮಾವ
ಮಾವಂ
ಮಾವಂದಿ-ರಾಗಿದ್ದ-ರೆಂದು
ಮಾವಂದಿ-ರೆಂದು
ಮಾವನ
ಮಾವ-ನಂಕ-ಕಾರ
ಮಾವ-ನಾಗುತ್ತಾನೆ
ಮಾವ-ನಾದ
ಮಾವಳ್ಳಿ
ಮಾವಿನ
ಮಾವಿನ-ಕೆರೆ
ಮಾವಿನ-ಕೆರೆ-ಯನ್ನು
ಮಾವಿನ-ಕೆರೆ-ಯನ್ನೂ
ಮಾವಿನ-ಬನ-ವನ್ನು
ಮಾವಿನ-ಮರ-ದಿಂದಂ
ಮಾವಿನ-ಹಳ್ಳವು
ಮಾವು-ತರಿರ-ಬ-ಹುದು
ಮಾಸ
ಮಾಸತಿ
ಮಾಸ-ತಿರು-ನಕ್ಷತ್ರ
ಮಾಸದ
ಮಾಸ-ದಲ್ಲಿ
ಮಾಸ-ಮದೇ-ಭಾವ-ಳಿಯಂ
ಮಾಸ-ವಾಡಿ
ಮಾಸ-ವೆಗ್ಗ-ಡಿಕೆಯ
ಮಾಸ-ವೆಗ್ಗಡೆ
ಮಾಸ-ವೆಗ್ಗಡೆ-ಗಳ
ಮಾಸ-ವೆಗ್ಗಡೆಗೆ
ಮಾಸ-ವೆಗ್ಗಡೆ-ತನ-ದಲ್ಲಿ
ಮಾಸ-ವೆಗ್ಗಡೆಯ
ಮಾಸೋತ್ಸವ
ಮಾಸೋಪ-ವಾಸಿ
ಮಾಸ್ತಮ್ಮನ
ಮಾಸ್ತಿ
ಮಾಸ್ತಿ-ಕಲ್ಲನ್ನು
ಮಾಸ್ತಿ-ಕಲ್ಲಿಗೆ
ಮಾಸ್ತಿ-ಕಲ್ಲಿ-ನಲ್ಲಿ
ಮಾಸ್ತಿ-ಕಲ್ಲಿ-ನಿಂದ
ಮಾಸ್ತಿ-ಕಲ್ಲು
ಮಾಸ್ತಿ-ಕಲ್ಲು-ಗಳ
ಮಾಸ್ತಿ-ಕಲ್ಲು-ಗ-ಳನ್ನು
ಮಾಸ್ತಿ-ಕಲ್ಲು-ಗಳಿವೆ
ಮಾಸ್ತಿ-ಕಲ್ಲು-ಗಳು
ಮಾಸ್ತಿ-ಗಲ್ಲು
ಮಾಸ್ತಿ-ಗಲ್ಲು-ಗಳೂ
ಮಾಸ್ತಿ-ಗುಡಿ-ಗಳಿ-ರುವು-ದುನ್ನು
ಮಾಸ್ತಿ-ಮರ-ಣದ
ಮಾಸ್ತಿ-ಯಾಗುವ
ಮಾಸ್ತಿ-ಸಂಪ್ರ-ದಾಯ-ವನ್ನು
ಮಾಹಾತ್ಮ್ಯಂ
ಮಾಹಾ-ಸಾ-ಮನ್ತ
ಮಾಹಿತಿ
ಮಾಹಿತಿ-ಗ-ಳನ್ನು
ಮಾಹಿತಿ-ಗಳಿ-ರುವುದ-ರಿಂದ
ಮಾಹಿತಿ-ಗಳು
ಮಾಹಿ-ತಿಯ
ಮಾಹಿತಿ-ಯನ್ನು
ಮಾಹಿತಿ-ಯಿಂದ
ಮಾಹೇಶ್ವರ
ಮಾಹೇಶ್ವರ-ನಾ-ಗಿದ್ದು
ಮಾಹೇಶ್ವರ-ರಿಗೆ
ಮಾಹೇಶ್ವರರು
ಮಾೞಿ-ಯಕ್ಕಾಚಾ-ರಿಗೆ
ಮಿಂಚ-ಗವುಂಡನ
ಮಿಂಚ-ಗವುಂಡನಾ
ಮಿಂಚಿ-ನಂತೆ
ಮಿಂಚು-ಗಾವುಂಡನ
ಮಿಂದಿತು
ಮಿಂದು
ಮಿಕ್ಕ
ಮಿಕ್ಕದ್ದನ್ನು
ಮಿಗಿ-ಲಾಗಿ
ಮಿಗಿಲಾ-ದುದು
ಮಿಗಿಲೆನಿಪಂ
ಮಿಗೆ
ಮಿತಿ-ಯಲ್ಲಿಯೇ
ಮಿತಿಯೇ
ಮಿತ್ರ-ನಂತಿದ್ದನು
ಮಿತ್ರ-ನಾಗಿದ್ದಿರ-ಬ-ಹುದು
ಮಿತ್ರ-ರಾಗಿದ್ದರು
ಮಿತ್ರ-ರಾಜ್ಯ-ವಾದ
ಮಿತ್ರ-ರಾದ
ಮಿತ್ರ-ರಿಗೆ
ಮಿನಾರು-ಗಳು
ಮಿರಾನ್
ಮಿರು-ಹನ-ಗಣ್ಯ
ಮಿರ್ಗ್ಗಿಡ-ಸಿರ್ಪ್ಪಿನಮಕ್ಕೆ-ನತ್ತ
ಮಿರ್ಲೆ
ಮಿರ್ಲೆ-ಶಾ-ಸನೋಕ್ತ-ನಾಗಿ
ಮಿಶ್ರ-ಲೋ-ಹದ
ಮಿಸುಪೆಸೆವ
ಮೀಟರ್
ಮೀನನ್ನೂ
ಮೀನು-ಗಳಿವೆ
ಮೀಮಾಂಸಾ-ಶಾಸ್ತ್ರ
ಮೀರದೇ
ಮೀರಿ
ಮೀರಿ-ಸಿ-ದನು
ಮೀರಿ-ಸಿದ್ದರು
ಮೀರ್
ಮೀರ್ಜೈನ್
ಮೀರ್ಜೈನ್ಉನ್
ಮೀರ್ಮಹ-ಮದ್
ಮೀಸರ-ಗಂಡ
ಮೀಸಲಾ-ಗಿಟ್ಟು
ಮೀಸ-ಲಾಗಿ-ಡ-ಲಾಗುತ್ತಿತ್ತು
ಮೀಸ-ಲಾಗಿ-ದೆ-ಯೆಂದರೆ
ಮೀಸಲಾ-ಗಿದ್ದ-ವೆಂದು
ಮೀಸಲಿಡುತ್ತಾನೆ
ಮೀಸಲಿ-ರಿ-ಸಿದ
ಮೀಸಲು
ಮುಂಗೊಳ
ಮುಂಚಿ-ನಿಂದಲೂ
ಮುಂಚೆ
ಮುಂಚೆಯೇ
ಮುಂಜಯ್ಯ
ಮುಂಜಯ್ಯನು
ಮುಂಜಿ-ಯನೂ
ಮುಂಡಿಗೈ
ಮುಂತಾಗಿ
ಮುಂತಾದ
ಮುಂತಾದ-ವ-ರನ್ನು
ಮುಂತಾದ-ವ-ರಿಗೆ
ಮುಂತಾದ-ವರು
ಮುಂತಾ-ದವು
ಮುಂತಾದ-ವು-ಗಳ
ಮುಂತಾದ-ವು-ಗ-ಳನ್ನು
ಮುಂತಾದ-ವು-ಗ-ಳಿಗೆ
ಮುಂತಾದೇನು
ಮುಂತಿ-ದಿರಾಂತ-ನಂತ-ರಿಪು
ಮುಂದಕ್ಕೆ
ಮುಂದಣ
ಮುಂದಾಗಿದ್ದರು
ಮುಂದಾ-ಗಿದ್ದು
ಮುಂದಾಗುತ್ತಿದ್ದರು
ಮುಂದಿಟ್ಟ-ಕೊಂಡು
ಮುಂದಿಟ್ಟು
ಮುಂದಿಟ್ಟು-ಕೊಂಡು
ಮುಂದಿಡುತ್ತಿದ್ದೇನೆ
ಮುಂದಿದೆ
ಮುಂದಿದ್ದ
ಮುಂದಿದ್ದರೆ
ಮುಂದಿದ್ದು
ಮುಂದಿನ
ಮುಂದಿ-ನಂತೆ
ಮುಂದಿನ-ಭಾಗದ
ಮುಂದಿನ-ವರ
ಮುಂದಿನ-ಹಂತ-ಗಳು
ಮುಂದಿರುವ
ಮುಂದಿರೆ
ಮುಂದಿವೆ
ಮುಂದುರಿ-ಯಿತು
ಮುಂದು-ವರಿದ
ಮುಂದು-ವರಿ-ದನು
ಮುಂದು-ವರಿ-ದರೂ
ಮುಂದು-ವರಿ-ದವು
ಮುಂದು-ವರಿ-ದಿತ್ತು
ಮುಂದು-ವರಿ-ದಿತ್ತೆಂದು
ಮುಂದು-ವರಿ-ದಿದೆ
ಮುಂದು-ವರಿ-ದಿದ್ದ-ನೆಂದು
ಮುಂದು-ವರಿ-ದಿದ್ದರೆ
ಮುಂದು-ವರಿ-ದಿದ್ದಾನೆ
ಮುಂದು-ವರಿ-ದಿದ್ದು
ಮುಂದು-ವರಿ-ದಿರ-ಬ-ಹುದು
ಮುಂದು-ವರಿ-ದಿರ-ಬಹು-ದೆಂದು
ಮುಂದು-ವರಿ-ದಿ-ರಬೇಕೆಂಬ
ಮುಂದು-ವರಿ-ದಿರುವ
ಮುಂದು-ವರಿ-ದಿ-ರು-ವು-ದನ್ನು
ಮುಂದು-ವರಿ-ದಿ-ರುವುದು
ಮುಂದುವ-ರಿದು
ಮುಂದು-ವ-ರಿಯಿ
ಮುಂದು-ವ-ರಿಯಿತು
ಮುಂದು-ವ-ರಿಯಿ-ತೆಂದು
ಮುಂದು-ವ-ರಿಯುತ್ತದೆ
ಮುಂದು-ವರಿ-ಸ-ಲಾಯಿ-ತೆಂದು
ಮುಂದು-ವರಿ-ಸಲು
ಮುಂದು-ವರಿಸಿ
ಮುಂದು-ವರಿ-ಸಿ-ಕೊಂಡು
ಮುಂದು-ವರಿ-ಸಿ-ಕೊಳ್ಳುತ್ತಾರೆ
ಮುಂದು-ವರಿ-ಸಿದ
ಮುಂದು-ವರಿ-ಸಿ-ದಂತೆ
ಮುಂದು-ವರಿ-ಸಿ-ದನು
ಮುಂದು-ವರಿ-ಸಿ-ದ-ನೆಂದು
ಮುಂದು-ವರಿ-ಸಿ-ದರು
ಮುಂದು-ವರಿ-ಸಿದೆ
ಮುಂದು-ವರಿ-ಸಿದ್ದ-ರೆಂದು
ಮುಂದು-ವರಿ-ಸಿದ್ದಾನೆ
ಮುಂದು-ವರಿ-ಸಿ-ರುವ
ಮುಂದು-ವರಿ-ಸುತ್ತಾನೆ
ಮುಂದು-ವರಿ-ಸುತ್ತಾರೆ
ಮುಂದು-ವರಿ-ಸುತ್ತೇನೆ
ಮುಂದು-ವ-ರಿ-ಸುವ
ಮುಂದು-ವರೆ-ದರು
ಮುಂದು-ವರೆ-ದಿದ್ದನ್ನು
ಮುಂದು-ವರೆ-ಸಿದ-ನೆಂಬುದು
ಮುಂದೆ
ಮುಂದೆಯೇ
ಮುಂದೊಡ್ಡಿ
ಮುಂನಂ
ಮುಂನಾದ
ಮುಂನೂರು
ಮುಂಬ-ರಿದು
ಮುಂಮರಿ-ದಂಡ
ಮುಂಮುರಿ-ದಂಡ
ಮುಕುಳಿ-ಕೆರೆಯ
ಮುಕ್ಕಾದ
ಮುಕ್ತ
ಮುಕ್ತ-ಗೊ-ಳಿಸಿ
ಮುಕ್ತ-ವಾದ
ಮುಕ್ತ-ಹಸ್ತ-ದಿಂದ
ಮುಕ್ತಾ
ಮುಕ್ತಾ-ಯ-ವಾಗಿದೆ
ಮುಕ್ತಾ-ಯ-ವಾಯಿತು
ಮುಕ್ತಿ
ಮುಕ್ತಿ-ವನಿತಾಸ್ತನ
ಮುಕ್ತ್ಯಾಂಗ-ನ-ವಲ್ಲಭೋ
ಮುಖ
ಮುಖಂಡ-ನಿಗೂ
ಮುಖಂಡ-ನಿರ-ಬಹು-ದೆಂದು
ಮುಖಂಡನು
ಮುಖಂಡರ
ಮುಖಂಡ-ರಾದ
ಮುಖ-ತಿಲಕ-ದಂತಿದ್ದ
ಮುಖ-ಮಂಟಪ
ಮುಖ-ಮಂಟಪ-ಗ-ಳನ್ನು
ಮುಖ-ಮಂಟಪ-ಗಳಿವೆ
ಮುಖ-ಮಂಟಪದ
ಮುಖ-ಮಂಟಪ-ದಲ್ಲಿ-ರುವ
ಮುಖ-ಮಂಟಪ-ವನ್ನು
ಮುಖ-ಮಂಟಪವು
ಮುಖ-ವನ್ನು
ಮುಖ-ವಾ-ಗಿದ್ದ
ಮುಖ-ಸುರ-ರತ್ನ-ದರ್ಪ್ಪಣಂ
ಮುಖಾಂತರ
ಮುಖಾಂತರ-ವಾಗಿ
ಮುಖ್ಕಾ-ದಲ್ಲಿ
ಮುಖ್ಯ
ಮುಖ್ಯ-ಕಾರ-ಣ-ವಾಗಿ-ರಬೇಕೆಂದೂ
ಮುಖ್ಯ-ಕಾಲುವೆಯ
ಮುಖ್ಯ-ಕೇಂದ್ರ-ಗ-ಳನ್ನು
ಮುಖ್ಯ-ಕೇಂದ್ರ-ವನ್ನಾಗಿ
ಮುಖ್ಯ-ಕೋಟೆ
ಮುಖ್ಯ-ಗರ್ಭ-ಗುಡಿ-ಯಲ್ಲಿ
ಮುಖ್ಯ-ನಾಗಿದ್ದ-ನೆಂದು
ಮುಖ್ಯ-ನಾಗಿದ್ದ-ನೆಂದೂ
ಮುಖ್ಯ-ನಾದ
ಮುಖ್ಯ-ಪಟ್ಟ-ಣ-ವಾಗಿತ್ತೆಂದು
ಮುಖ್ಯ-ಪಾಠ
ಮುಖ್ಯ-ಪಾತ್ರ
ಮುಖ್ಯ-ಪಾತ್ರ-ವನ್ನು
ಮುಖ್ಯ-ಮಂತ್ರಿಗೆ
ಮುಖ್ಯ-ಮಂತ್ರಿಯ
ಮುಖ್ಯ-ಮಪ್ಪ
ಮುಖ್ಯ-ರಪ್ಪ
ಮುಖ್ಯ-ರಾದ
ಮುಖ್ಯ-ಲಕ್ಷ-ಣ-ವಾ-ಗಿತ್ತು
ಮುಖ್ಯ-ವಪ್ಪ
ಮುಖ್ಯ-ವಾಗಿ
ಮುಖ್ಯ-ವಾ-ಗಿತ್ತು
ಮುಖ್ಯ-ವಾಗಿತ್ತೆಂಬು-ದನ್ನು
ಮುಖ್ಯ-ವಾಗಿದೆ
ಮುಖ್ಯ-ವಾ-ಗಿದ್ದು
ಮುಖ್ಯ-ವಾಗುತ್ತದೆ
ಮುಖ್ಯ-ವಾದ
ಮುಖ್ಯ-ವಾದ-ವು-ಗ-ಳನ್ನು
ಮುಖ್ಯ-ವಾ-ದು-ದನ್ನು
ಮುಖ್ಯ-ವಾ-ದುದು
ಮುಖ್ಯ-ವಾದು-ದೆಂದರೆ
ಮುಖ್ಯ-ವೆನಿ-ಸುತ್ತದೆ
ಮುಖ್ಯ-ಸರ್ವಾಧಿ-ಪತ್ಯಕಂ
ಮುಖ್ಯಸ್ಥ
ಮುಖ್ಯಸ್ಥನ
ಮುಖ್ಯಸ್ಥ-ನಾಗಿ
ಮುಖ್ಯಸ್ಥ-ನಾ-ಗಿದ್ದ
ಮುಖ್ಯಸ್ಥ-ನಾಗಿದ್ದನು
ಮುಖ್ಯಸ್ಥ-ನಾಗಿದ್ದ-ನೆಂದು
ಮುಖ್ಯಸ್ಥ-ನಾ-ಗಿದ್ದು
ಮುಖ್ಯಸ್ಥ-ನಾಗಿ-ರ-ಬ-ಹುದು
ಮುಖ್ಯಸ್ಥ-ನಾದ
ಮುಖ್ಯಸ್ಥ-ನೆಂದು
ಮುಖ್ಯಸ್ಥ-ರನ್ನಾಗಿ
ಮುಖ್ಯಸ್ಥ-ರನ್ನು
ಮುಖ್ಯಸ್ಥ-ರಲ್ಲಿ
ಮುಖ್ಯಸ್ಥ-ರಾಗಿ
ಮುಖ್ಯಸ್ಥ-ರಾ-ಗಿದ್ದ
ಮುಖ್ಯಸ್ಥ-ರಾಗಿದ್ದರು
ಮುಖ್ಯಸ್ಥ-ರಾಗಿದ್ದ-ರೆಂದು
ಮುಖ್ಯಸ್ಥ-ರಾಗಿದ್ದ-ರೆಂದೂ
ಮುಖ್ಯಸ್ಥ-ರಾಗಿ-ರ-ಬಹು-ದೆಂದು
ಮುಖ್ಯಸ್ಥ-ರಾಗಿರು
ಮುಖ್ಯಸ್ಥ-ರಾಗಿ-ರುತ್ತಿದ್ದರು
ಮುಖ್ಯಸ್ಥ-ರಾದ
ಮುಖ್ಯಸ್ಥರು
ಮುಖ್ಯಸ್ಥಳ
ಮುಖ್ಯಸ್ಥ-ಳ-ವಾಗಿ
ಮುಖ್ಯಸ್ಥ-ಳ-ವಾ-ಗಿತ್ತು
ಮುಖ್ಯಸ್ಥ-ಳ-ವಾಗಿತ್ತೆಂದು
ಮುಖ್ಯಸ್ಥ-ಳ-ವಾ-ಗಿದ್ದ
ಮುಖ್ಯಸ್ಥ-ಳ-ವಾಗಿದ್ದಿರ
ಮುಖ್ಯಸ್ಥ-ಳ-ವಾ-ಗಿದ್ದು
ಮುಖ್ಯಸ್ಥ-ಳ-ವಾದ
ಮುಖ್ಯಾಂಶ-ಗಳು
ಮುಖ್ಯಾಧಿ-ಕಾರಿ-ಗ-ಳಾಗಿ
ಮುಗಿದ
ಮುಗಿ-ದಿತ್ತು
ಮುಗಿ-ದಿದೆ
ಮುಗಿ-ದಿರು-ವಂತೆ
ಮುಗಿಯುತ್ತದೆ
ಮುಗಿಲ
ಮುಗಿಲ-ಕುಲ
ಮುಗಿಲ-ಕು-ಲಕ್ಕೂ
ಮುಗಿಲಿಗೂಮೋಡ
ಮುಗಿ-ಲಿಗೆ
ಮುಗುಳ್ನಗೆ-ಯೊಂದಿಗೆ
ಮುಘಲ್
ಮುಚಳಿನು
ಮುಚಿಸುತ್ತಾನೆ
ಮುಚ್ಚ-ಲಾಗುತ್ತಿತ್ತು
ಮುಚ್ಚಳ-ವನ್ನು
ಮುಚ್ಚಿದ್ದು
ಮುಚ್ಚಿ-ರುವ
ಮುಚ್ಚಿ-ಹೋಗಿದೆ
ಮುಚ್ಚಿ-ಹೋಗಿವೆ
ಮುಟ್ಟನ-ಹಳ್ಳಿ
ಮುಟ್ಟಲು
ಮುಟ್ಟಿದಂ
ಮುಟ್ಟಿದಂತೆ
ಮುಟ್ಟಿಯಾಮ್ಬಾಕ್ಕಮ್
ಮುಟ್ಟಿರು-ವು-ದನ್ನು
ಮುಟ್ಟು-ಪತ್ರ
ಮುಟ್ಣ-ಹಳ್ಳಿ
ಮುಡಿ-ಗುಂಡಂ
ಮುಡಿ-ಗೊಂಡ
ಮುಡಿ-ಗೊಂಡಂನ
ಮುಡಿ-ಗೊಂಡ-ವನ್ನು
ಮುಡಿಪಿ
ಮುಡಿ-ಪಿ-ದ-ನೆಂದು
ಮುಡಿ-ಪಿ-ದಾಗ
ಮುಡಿ-ಪಿ-ರಬೇಕೆಂದೂಆ
ಮುಡುಕು-ತೊರೆಯ
ಮುಡೆ-ಗೊಂಡು
ಮುಡೆ-ಯಾರ್
ಮುಡೈಯಾನ-ವನ್ನ
ಮುಡೈಯಾನ್
ಮುಡೈ-ಯಾರ್
ಮುಡೈ-ಯಾರ್ಗ-ವರೇಶ್ವರ
ಮುಡೈ-ಯಾರ್ಗೆ
ಮುತಹ-ಡೆಯ-ರಾಯನ
ಮುತು-ವರ್ಜಿ
ಮುತ್ತ
ಮುತ್ತತ್ತಿ
ಮುತ್ತತ್ತಿಯ
ಮುತ್ತಬ್ಬೆ
ಮುತ್ತಬ್ಬೆ-ತಾತ
ಮುತ್ತಬ್ಬೆಯು
ಮುತ್ತ-ರಸ
ಮುತ್ತ-ರ-ಸನು
ಮುತ್ತಲು
ಮುತ್ತಿ
ಮುತ್ತಿಗ
ಮುತ್ತಿಗೆ
ಮುತ್ತಿ-ಗೆ-ಯಲ್ಲಿ
ಮುತ್ತಿದ
ಮುತ್ತಿ-ದ-ನೆಂದು
ಮುತ್ತಿ-ದಲ್ಲಿ
ಮುತ್ತಿ-ದ-ವರು
ಮುತ್ತಿ-ದಾಗ
ಮುತ್ತಿದ್ದಲ್ಲಿ
ಮುತ್ತಿನ
ಮುತ್ತಿ-ನೆಕ್ಕಸ-ರದಂತುರ-ದೊಳ್
ಮುತ್ತೆ-ಗೆರೆ
ಮುತ್ತೆತ್ತಿ
ಮುತ್ತೆಯ
ಮುತ್ತೆ-ಯನು
ಮುತ್ತೇ-ಗೆರೆಯ
ಮುತ್ಯಾ-ತಾತ
ಮುತ್ಸಂದ್ರ-ಬೇಚಿರಾಕ್
ಮುದ
ಮುದ-ಗಂದೂರಿನ
ಮುದ-ಗಂದೂರಿನಲ್ಲಿ
ಮುದ-ಗಂದೂರು
ಮುದ-ಗನ್ದೂರು
ಮುದ-ಗಲ್
ಮುದ-ಗಾವುಂಡ
ಮುದ-ಗಾವುಂಡನ
ಮುದ-ಗುಂದೂರಿ-ನಲ್ಲಿ
ಮುದ-ಗುಂದೂರಿ-ನಲ್ಲಿ-ನಡೆದ
ಮುದ-ಗುಂದೂರು
ಮುದ-ಗೆರೆ
ಮುದ-ಜಾತಿ
ಮುದ-ಜಾತಿಗ್ರಾಮ
ಮುದದಿ
ಮುದದಿಂ
ಮುದ-ದಿಂದ
ಮುದ-ನೂರ
ಮುದ-ಬೋವ
ಮುದಮಂ
ಮುದ-ವಾಡಿ
ಮುದ-ವಿ-ಡಿಯ
ಮುದ-ವೊಳಲಿ-ನಲ್ಲಿ-ಮುಧೋಳ್
ಮುದ-ಸ-ಮುದ್ರ
ಮುದಿ-ಗೆರೆ
ಮುದಿ-ಗೆರೆಯ
ಮುದಿ-ಗೆರೆ-ಯನ್ನು
ಮುದಿ-ತ-ಮೂರ್ತಿರ್ಲೋಕ-ವಿಖ್ಯಾತ
ಮುದಿ-ಬೆಟ್ಟದ
ಮುದಿ-ಬೆಟ್ಟ-ದ-ಸಾತೇನ-ಹಳ್ಳಿ
ಮುದಿ-ಮಲೆ
ಮುದಿ-ಮಾರ-ನ-ಹಳ್ಳಿ
ಮುದು-ಕೊಂಗಣಿ
ಮುದು-ಗ-ನೂರು-ದುರ್ಗ
ಮುದು-ಗುಂದೂರು
ಮುದು-ಗುಂದೂರು-ಗಳ
ಮುದು-ಗುಪ್ಪೆ
ಮುದು-ಗುಪ್ಪೆಯ
ಮುದು-ಗುಪ್ಪೆಯು
ಮುದು-ಗೂರು
ಮುದು-ಗೆರೆ-ಯನ್ನು
ಮುದುಡಿ
ಮುದು-ಡಿಯ
ಮುದು-ರಾಚಯ್ಯ
ಮುದು-ರಾಚಯ್ಯ-ನನ್ನು
ಮುದು-ವೆಯ
ಮುದು-ವೊಳಲೊಳ್
ಮುದೇ
ಮುದೇ-ಗೌಡ-ನ-ಕಟ್ಟೆ
ಮುದೇ-ನ-ಹಳ್ಳಿ-ಯನ್ನು
ಮುದ್ದ-ಗೌಡ
ಮುದ್ದ-ಗೌಡನ
ಮುದ್ದ-ಣಾ-ಚಾರ್ಯ
ಮುದ್ದ-ಣಾ-ಚಾರ್ಯನ
ಮುದ್ದಣ್ಣನ
ಮುದ್ದಣ್ಣನು
ಮುದ್ದ-ನ-ಗೆರೆ
ಮುದ್ದಮ್ಮ-ನ-ವರು
ಮುದ್ದ-ರಸಿ
ಮುದ್ದ-ರ-ಸಿ-ಯರ
ಮುದ್ದಿ-ಯಕ್ಕರ
ಮುದ್ದು-ಕೃಷ್ಣಾ-ಜಮ್ಮಣ್ಣಿ
ಮುದ್ದು-ಲಿಂಗ-ಮಾಂಬಾ
ಮುದ್ದು-ಲಿಂಗಮ್ಮ-ನ-ವರು
ಮುದ್ದು-ಲಿಂಗಮ್ಮನು
ಮುದ್ದೆಯ
ಮುದ್ದೆಯ-ನಾಯ-ಕನು
ಮುದ್ದೇ-ಗೌಡ-ನ-ಕಟ್ಟೆ
ಮುದ್ದೇನ-ಹಳ್ಳಿ-ಯನ್ನು
ಮುದ್ರಣ
ಮುದ್ರ-ಣದ
ಮುದ್ರ-ಣಾ-ಲ-ಯದ
ಮುದ್ರಾಂಕಿತ
ಮುದ್ರಿಕಾ
ಮುದ್ರಿ-ಕೆಯ-ನೊಲ-ವಿನಿನೀ
ಮುದ್ರಿ-ಕೆ-ಯನ್ನು
ಮುದ್ರೆ
ಮುದ್ರೆಗೆ
ಮುದ್ರೆ-ಯನ್ನು
ಮುದ್ರೆ-ಯನ್ನೂ
ಮುನಿ
ಮುನಿ-ಗ-ಳಿಗೆ
ಮುನಿ-ಗಳು
ಮುನಿ-ಗೌತಮಸ್ಯ
ಮುನಿ-ಚಂದ್ರ
ಮುನಿ-ಚಂದ್ರ-ದೇವರ
ಮುನಿ-ಚಂದ್ರ-ದೇವ-ರಿಗೆ
ಮುನಿ-ನಾಥರೆ
ಮುನಿ-ನಾನೋ
ಮುನಿ-ಪತಿ
ಮುನಿ-ಪುಂಗ-ವರು
ಮುನಿ-ಭದ್ರ
ಮುನಿ-ಭದ್ರ-ಸಿದ್ಧಾಂತ-ದೇವ-ಮೇಘ-ಚಂದ್ರ
ಮುನಿ-ಮುಖ್ಯ
ಮುನಿಯ
ಮುನಿ-ಯನ್ನು
ಮುನಿ-ಯಿಂದ
ಮುನಿಯು
ಮುನಿ-ಯೊಬ್ಬನು
ಮುನಿ-ರಾಜಪ್ಪ-ನ-ವರು
ಮುನಿ-ವರ-ನೆಂದು
ಮುನಿಶ್ರೇಷ್ಠ-ನೆಂದು
ಮುನಿ-ಸಿ-ಕೊಂಡು
ಮುನೀಂದ್ರನು
ಮುನೀಂದ್ರ-ರನ್ನು
ಮುನೀಂದ್ರ-ರಿಗೆ
ಮುನೀಶ್ವರ-ನಿಂದ-ನೇ-ಕರುವಂ
ಮುನೂರ್ವ್ವರು
ಮುನ್ದೆ
ಮುನ್ನ
ಮುನ್ನಂರೂಢಿಯ
ಮುನ್ನಡೆಸುತ್ತಿದ್ದ-ರೆಂದು
ಮುನ್ನವೇ
ಮುನ್ನ-ಸಂದ
ಮುನ್ನಾದ
ಮುನ್ನುಗ್ಗಿ
ಮುನ್ನುರ್ವರು
ಮುನ್ನೂ-ರನ್ನು
ಮುನ್ನೂರು
ಮುನ್ನೂ-ರುಮ್
ಮುನ್ನೂರ್ವರು
ಮುನ್ನೂರ್ವರೇ
ಮುಪ್ಪಾಗ
ಮುಪ್ಪಿನ
ಮುಮ್ಮಡಿ
ಮುಮ್ಮಡಿ-ಚೋಳ
ಮುಮ್ಮಡಿ-ನರ-ಸಿಂಹ
ಮುಮ್ಮಡಿ-ಬಲ್ಲಾಳನ
ಮುಮ್ಮಡಿ-ಬಲ್ಲಾಳನು
ಮುಮ್ಮುರಿ
ಮುಮ್ಮುರಿ-ದಂಡ
ಮುರಾರಿ
ಮುರಾರಿ-ಮಲ್ಲಯ್ಯ
ಮುರಾರಿ-ರಾಯ-ಗೌಡ
ಮುರಿದ
ಮುರಿದು
ಮುರಿಯರು
ಮುರುಂಡಿ
ಮುರು-ಕನ-ಹಳ್ಳಿ
ಮುರು-ಕನ-ಹಳ್ಳಿಯ
ಮುರುಡಾನ-ಸೆಟ್ಟಿ
ಮುರು-ಯನ
ಮುರುಳಿ
ಮುರು-ವರ್ಮನ
ಮುರ್ಬ್ಬಿಗಗುರ್ವು
ಮುಱು-ಹಣ
ಮುಲಾಮಿನ
ಮುಲ್ಕ್
ಮುಲ್ಲಾ-ಗ-ಳನ್ನು
ಮುಳಬಾ-ಗಲ್
ಮುಳ-ಬಾ-ಗಿಲು
ಮುಳುಗಡೆ-ಯಾ-ಗಿದ್ದು
ಮುಳುಗಡೆ-ಯಾಗಿ-ರ-ಬ-ಹುದು
ಮುಳುಗಡೆ-ಯಾಗಿವೆ
ಮುಳುಗಡೆ-ಯಾ-ಗುತ್ತವೆ
ಮುಳುಗಡೆ-ಯಾದ
ಮುಳುಗಡೆ-ಯಾದವು
ಮುಳುಗಿ
ಮುಳು-ಗಿದ್ದ
ಮುಳು-ಗಿದ್ದು
ಮುಳುಗಿಸಿ
ಮುಳು-ಗುತ್ತವೆ
ಮುಳ್ಗುಂದ-ವನ್ನು
ಮುಳ್ಳ
ಮುಳ್ಳು
ಮುಳ್ಳೋಜ
ಮುವರು-ಮಿಳ್ದು
ಮುವರು-ರಾಯ-ರ-ಗಂಡ
ಮುಷೀರ್
ಮುಷ್ಕ-ರನ
ಮುಸಂದೂರ
ಮುಸಲ್ಮಾನ
ಮುಸಲ್ಮಾನರ
ಮುಸಲ್ಮಾನ-ರಿ-ಗಾಗಿ
ಮುಸಲ್ಮಾನ-ರಿಗೆ
ಮುಸಲ್ಮಾನರು
ಮುಸಲ್ಮಾನ್
ಮುಸುಕ
ಮುಸುಕ-ಮಾದೆ-ಗೊಂಡನ
ಮುಸುಕು
ಮುಸ್ತಫಾ-ಖಾನ್
ಮುಸ್ತೈದೆ
ಮುಸ್ತೈದೆ-ಗಳು-ದತ್ತಿ-ಗಳು
ಮುಸ್ಲಿಂ
ಮುಸ್ಲಿ-ಮರ
ಮುಸ್ಲಿಮ-ರಿದ್ದ-ರಂತೆ
ಮುಸ್ಲಿ-ಮರು
ಮುಸ್ಲಿ-ಮರೆನ್ನದೆ
ಮುಹೂರ್ತ-ದಿಂದ
ಮುೞ್ತಿಗೆಯ
ಮೂಕು-ತಿ-ಗಳುಂ
ಮೂಗನೂ
ಮೂಗನ್ನು
ಮೂಗರ
ಮೂಗರ-ನಾಡಾ-ಳುವ
ಮೂಗರ-ನಾಡು
ಮೂಗರ-ನಾಡು-ಮೂ-ಗೂರು-ಮೂರು-ನಾಡು
ಮೂಗರಿ-ವೋನ್
ಮೂಗೂ-ರನ್ನು
ಮೂಗೂರಿ-ನಲ್ಲಿ
ಮೂಗೂರಿ-ನಲ್ಲೂ
ಮೂಗೂರು
ಮೂಡ-ಕೆರೆ
ಮೂಡ-ಗೆರೆ
ಮೂಡಣ
ಮೂಡಣ-ಕೋಡಿ
ಮೂಡಣ-ಕೋಡಿ-ಗಳ
ಮೂಡಣ-ಕೋಡಿ-ಯಲ್ಲಿ
ಮೂಡ-ರಾಜ್ಯ
ಮೂಡ-ರಾಜ್ಯಕ್ಕೆ
ಮೂಡ-ರಾಜ್ಯದ
ಮೂಡಲು
ಮೂಡಲು-ಕೋಡಿ
ಮೂಡಲು-ವುಳ್ಳ
ಮೂಡಾಯ
ಮೂಡಿ
ಮೂಡಿ-ಗೆರೆ
ಮೂಡಿತು
ಮೂಡಿ-ಬಂದಿದೆ
ಮೂಡಿ-ಬಂದಿವೆ
ಮೂದಲಿ-ಸಿ-ಕೊಳ್ಳುತ್ತಿದ್ದ
ಮೂದಲಿ-ಸುತ್ತಿದ್ದರು
ಮೂದಲಿ-ಸು-ವುದು
ಮೂದ-ಲೆಗೆಯ್ವ
ಮೂದೇವ-ರಾದ-ಭೇದದ
ಮೂನೂರ
ಮೂನ್ರು
ಮೂನ್ಱು
ಮೂರಕ್ಕಂ
ಮೂರನೆ
ಮೂರ-ನೆಯ
ಮೂರನೆ-ಯ-ವನು
ಮೂರ-ನೆಯು
ಮೂರನೇ
ಮೂರರ
ಮೂರ-ರಲ್ಲಿ
ಮೂರು
ಮೂರು-ಕಣ್ಣಿ-ರುವು-ದ-ರಿಂದ
ಮೂರು-ಜನ
ಮೂರು-ನಾಡಾ-ದ-ವನು
ಮೂರು-ನಾಲ್ಕು
ಮೂರು-ನೂರು
ಮೂರು-ಪಡಿ
ಮೂರು-ಪಣ-ವನ್ನು
ಮೂರು-ಪಾ-ಲನ್ನು
ಮೂರು-ಪಾ-ಲಿ-ನಲ್ಲಿ
ಮೂರು-ಬಳ್ಳ-ಸ-ಲಿಗೆ
ಮೂರು-ಬಾರಿ
ಮೂರು-ಮಾನದ
ಮೂರು-ರಾಉ-ಯರ
ಮೂರು-ರಾಜನ
ಮೂರು-ರಾಯರ
ಮೂರು-ರಾಯ-ರ-ಗಂಡಾಂಕಃ
ಮೂರು-ರಾಯ-ರ-ಗಂಡಾಂಕೋ
ಮೂರು-ಲೋಕ-ಜ-ಗದಳಂ
ಮೂರು-ಲೋಕ-ಜ-ಗದಾಳಂ
ಮೂರು-ವೃತ್ತಿ-ಯನ್ನು
ಮೂರು-ವೇದ-ಗಳು
ಮೂರುಸ್ಥಳಂಗಳ
ಮೂರು-ಹಳ್ಳಿ-ಗ-ಳನ್ನೂ
ಮೂರು-ಹೊನ್ನನ್ನು
ಮೂರೂ
ಮೂರ್ಛಿತ-ವಾಗು-ವಂತೆ
ಮೂರ್ತಸ್ಯ
ಮೂರ್ತಿ
ಮೂರ್ತಿ-ಗಳ
ಮೂರ್ತಿ-ಗ-ಳನ್ನು
ಮೂರ್ತಿ-ಗಳಿವೆ
ಮೂರ್ತಿ-ಗಳು
ಮೂರ್ತಿ-ಗಳೂ
ಮೂರ್ತಿ-ಗಿಂತ
ಮೂರ್ತಿಯ
ಮೂರ್ತಿ-ಯನ್ನು
ಮೂರ್ತಿ-ಯ-ವರು
ಮೂರ್ತಿ-ಯ-ವರೂ
ಮೂರ್ತಿ-ಯಾಗಲೀ
ಮೂರ್ತಿಯು
ಮೂರ್ತಿಯೂ
ಮೂರ್ತಿಯೇ
ಮೂರ್ತಿ-ಯೊಳಲ್ಲದೆ
ಮೂರ್ತಿ-ಶಿಲ್ಪ-ಗಳ
ಮೂರ್ತಿ-ಶಿಲ್ಪ-ಗ-ಳಲ್ಲಿ
ಮೂರ್ತಿ-ಶಿಲ್ಪ-ಗ-ಳಾಗಿವೆ
ಮೂರ್ತಿ-ಶಿಲ್ಪ-ಗಳಿವೆ
ಮೂರ್ತಿ-ಶಿಲ್ಪ-ಗಳು
ಮೂರ್ತಿ-ಶಿಲ್ಪದ
ಮೂರ್ತಿ-ಶಿಲ್ಪ-ದಲ್ಲಿ
ಮೂರ್ತಿ-ಶಿಲ್ಪದ್ದು
ಮೂರ್ತಿ-ಶಿಲ್ಪವು
ಮೂರ್ತಿ-ಶಿಲ್ಪ-ವೆಂದು
ಮೂರ್ತೀ-ಗೊ-ಳಿಸಿ
ಮೂರ್ತ್ತಿ-ಗಳ್ಗುಹಾವಾ-ಸಿಗಳ
ಮೂರ್ಧ್ನೀಕೃತಾ-ಸನಂ
ಮೂಲ
ಮೂಲಂ
ಮೂಲಕ
ಮೂಲ-ಕಥೆ
ಮೂಲ-ಕ-ಥೆಯು
ಮೂಲ-ಕವೂ
ಮೂಲ-ಕವೇ
ಮೂಲ-ಗ-ಳಾಗಿದ್ದವು
ಮೂಲ-ಗ-ಳಿಂದ
ಮೂಲ-ಗಳು
ಮೂಲ-ಗುಣಸ್ತಥೋತ್ತರ
ಮೂಲಗ್ರಾಮ-ದಿಂದ
ಮೂಲತ
ಮೂಲತಃ
ಮೂಲ-ತತ್ವ-ಗಳು
ಮೂಲದ
ಮೂಲ-ದ-ವನೆನ್ನು-ವುದು
ಮೂಲ-ದಿಂದ
ಮೂಲ-ನೆಲೆ-ಯಿಂದ
ಮೂಲ-ಪದ
ಮೂಲ-ಪುರ-ಷನ
ಮೂಲ-ಪುರ-ಷ-ನೆಂದು
ಮೂಲ-ಪುರುಷ
ಮೂಲ-ಪುರುಷ-ನಾಗಿ-ರುವ
ಮೂಲ-ಪುರುಷ-ನೆಂದೂ
ಮೂಲ-ಪುರುಷರು
ಮೂಲಬ್ರಹ್ಮೇಶ್ವರ
ಮೂಲ-ಮೂರ್ತಿ-ಯನ್ನು
ಮೂಲ-ರಾಜರು
ಮೂಲ-ರೂಪ-ವಾಗಿ-ರಲಾ-ರದು
ಮೂಲ-ರೂಪ-ವಿರ-ಬ-ಹುದು
ಮೂಲ-ವನ್ನು
ಮೂಲ-ವಾಗಿದೆ
ಮೂಲ-ವಾಗಿವೆ
ಮೂಲ-ವಾದ
ಮೂಲ-ವಿರ-ಬ-ಹುದು
ಮೂಲ-ವೆನಿಪಗ್ಗದ
ಮೂಲ-ಸಂಘ
ಮೂಲ-ಸಂಘದ
ಮೂಲ-ಸಿದ್ಧಾಯ-ದಲ್ಲಿ
ಮೂಲ-ಸೆಲೆ
ಮೂಲಸ್ಥ
ಮೂಲಸ್ಥಳ-ಗಳೇನೇ
ಮೂಲಸ್ಥಾನ
ಮೂಲಸ್ಥಾನದ
ಮೂಲಸ್ಥಾನ-ದೇವರ
ಮೂಲಸ್ಥಾನ-ದೇ-ವ-ರಿಗೆ
ಮೂಲಸ್ಥಾನ-ದೇ-ವ-ರೆಂದು
ಮೂಲಸ್ಥಾನೇಶ್ವರ
ಮೂಲಸ್ವ-ರೂಪ-ದಲ್ಲಿ
ಮೂಲ-ಹೆ-ಸರಿ-ನಲ್ಲಿಯೇ
ಮೂಲಿಗ
ಮೂಲೆ
ಮೂಲೆ-ಗ-ಳಿಂದ
ಮೂಲೆಯ
ಮೂಲೆ-ಯಲ್ಲಿ
ಮೂಲೆ-ಯಲ್ಲೂ
ಮೂಲೆ-ಸಿಂಗೇಶ್ವರ
ಮೂಲೆ-ಸಿಂಗೇಶ್ವರ-ಸಿಂಧೇಶ್ವರ
ಮೂಳೇನ-ಹಳ್ಳಿ-ಯನ್ನು
ಮೂವಡಿ
ಮೂವಡಿ-ಚೋಳ
ಮೂವತು
ಮೂವತ್ತರ್ಛಾ-ಸಿರ
ಮೂವತ್ತಾರು
ಮೂವತ್ತು
ಮೂವತ್ತು-ಕೊಳಗ
ಮೂವತ್ತು-ಗದೊಳಮ್ಪುದು-ವಿನೊಳ
ಮೂವತ್ತೂರ
ಮೂವತ್ತೆ-ರಡು
ಮೂವರ
ಮೂವರಲ್ಲದೆ
ಮೂವರು
ಮೂವರುಂ
ಮೂವರು-ರಾಯರ
ಮೂವರೂ
ಮೂಷಕಸ್ತಥಾ
ಮೂಷವ
ಮೂಷಿಕ
ಮೂಷಿಕ-ಮೂಷಕ
ಮೃಗ-ತೀರ್ಥ-ದಲ್ಲಿ
ಮೃಗಯಾಂ
ಮೃಗೇಶ
ಮೃಗೇಶ-ವರ್ಮನ
ಮೃತದೇ-ಹದ
ಮೃತ-ನಾಗಿ
ಮೃತ-ನಾಗಿದ್ದನು
ಮೃತ-ನಾಗಿದ್ದ-ನೆಂದು
ಮೃತ-ನಾಗಿ-ರ-ಬ-ಹುದು
ಮೃತ-ನಾದ
ಮೃತ-ನಾದನು
ಮೃತ-ನಾದಾಗ
ಮೃತ-ಪಟ್ಟ
ಮೃತ-ಪಟ್ಟ-ನೆಂದು
ಮೃತ-ಪಟ್ಟಾಗ
ಮೃತ-ಪಟ್ಟಿದ್ದರು
ಮೃತ-ಪಟ್ಟಿದ್ದು
ಮೃತ-ಪಟ್ಟಿ-ರ-ಬ-ಹುದು
ಮೃತರ
ಮೃತ-ರಾ-ಗಿದ್ದ-ರೆಂಬುದು
ಮೃತ-ರಾದರೆ
ಮೃತ-ರಾದ-ರೆಂದು
ಮೃತ-ರಾದಾಗ
ಮೃತ-ವಾಗುತ್ತಿದ್ದಾಗ
ಮೃತ-ವಾದ
ಮೃತೇನಾಪಿ
ಮೃದು
ಮೃದು-ಪದ-ಮಿತಿ
ಮೃಷ್ಟಾಂನ-ದಾನ
ಮೆಂಟೆ-ಯದ
ಮೆಂಡೆಯ
ಮೆಂಡೆ-ಯದ
ಮೆಕೆಂಜಿ
ಮೆಕೆಂಜಿಯು
ಮೆಕೆಂಝಿ
ಮೆಕ್ಕಾದ
ಮೆಚ್ಚ-ದ-ವರು
ಮೆಚ್ಚ-ದೋ-ರಾರ್
ಮೆಚ್ಚಿ
ಮೆಚ್ಚಿದ
ಮೆಚ್ಚಿ-ದಲ್ಲಿ
ಮೆಚ್ಚಿದೆ
ಮೆಚ್ಚಿದೆಂ
ಮೆಚ್ಚಿದ್ದನ್ನು
ಮೆಚ್ಚಿಸಿ
ಮೆಚ್ಚುಗೆ
ಮೆಚ್ಚುಗೆ-ಯಾಗಿ
ಮೆಚ್ಚೆ
ಮೆಟ್ಟಿ
ಮೆಟ್ಟಿಲ
ಮೆಟ್ಟಿ-ಲು-ಗಳ
ಮೆಟ್ಟಿ-ಲು-ಗ-ಳನ್ನು
ಮೆಟ್ಟಿ-ಲು-ಗ-ಳಿಗೆ
ಮೆಣಸ
ಮೆಣಸದ
ಮೆಣಸನ್ನೂ
ಮೆಣಸು
ಮೆದೆ
ಮೆದೆ-ಮನೆ
ಮೆದೆಯ
ಮೆಯಿ-ಸಿರಿ-ವಟ್ಟ-ವನೆ
ಮೆಯೆ-ದೇವನ
ಮೆಯ್ಯೊ-ಳೆಯ್ದಿ
ಮೆಯ್ವ-ಗೆಯದ
ಮೆಯ್ವೆತ್ತ
ಮೆರ-ವಣಿಗೆ
ಮೆರ-ವಣಿಗೆಗೆ
ಮೆರ-ವಣಿಗೆ-ಯಲ್ಲಿ
ಮೆರೆ-ದನು
ಮೆರೆ-ದ-ವನು
ಮೆರೆದಿದ್ದಾನೆ
ಮೆರೆ-ದಿರುವ
ಮೆರೆದು
ಮೆರೆಮಿಂಡ
ಮೆರೆಯ
ಮೆರೆಯುವ
ಮೆರೆವ
ಮೆರೆ-ವಂತಿರೇರಿ
ಮೆರೆವೊಳ್ಳೆಮ್ಬ
ಮೆಲು-ಕೋಟೆ-ಯನ್ನು
ಮೆಲ್ಕಂಡ
ಮೆಲ್ಲಗೆ
ಮೆಲ್ವಾಹೆ-ಯೆಂಬೀನೆವದೊಳೆ
ಮೆಳಸು
ಮೆಳ-ಹಳ್ಳಿ
ಮೆಳೆ-ಯೂರ
ಮೆಳ್ಳ-ಹಳ್ಳಿ
ಮೇ
ಮೇಂಗಾ
ಮೇಗಾ-ಹಿನ
ಮೇಘ-ಚಂದ್ರ
ಮೇಘ-ಚಂದ್ರತ್ರೈ-ವಿದ್ಯ-ದೇವನ
ಮೇಘ-ಚಂದ್ರ-ಸಿದ್ಧಾಂತ
ಮೇಜರ್
ಮೇಟಿ-ಕುರಿಕೆ
ಮೇಡು
ಮೇದ
ಮೇದಿನಿ
ಮೇದಿನೀ
ಮೇದೂರ
ಮೇನಾ-ಗರ
ಮೇನಾ-ಪುರ
ಮೇನಾ-ಪುರದ
ಮೇರು
ಮೇರು-ಗಿರಯೋ
ಮೇರು-ಗಿರಿಯೋ
ಮೇರು-ಲಂಘಿಯಶೋ-ಭರಃ
ಮೇರು-ವಿನ
ಮೇರುವೆ
ಮೇರು-ವೆ-ನಿ-ಸಿದ
ಮೇರೆ-ಗ-ಳನ್ನು
ಮೇರೆ-ಗ-ಳಾಗಿ
ಮೇರೆ-ಗ-ಳಾಗಿದ್ದವು
ಮೇರೆಗೆ
ಮೇರೆ-ಯನ್ನು
ಮೇರೆ-ಯಾಗಿ
ಮೇರೆ-ಯಾಗಿತ್ತೆಂದು
ಮೇರೆ-ಯಾಗಿದ್ದವು
ಮೇರೆಯೂ
ಮೇಲಕ್ಕಂ
ಮೇಲಕ್ಕೆ
ಮೇಲಕ್ಕೆತ್ತಿ
ಮೇಲಣ
ಮೇಲಧಿ-ಕಾರಿ
ಮೇಲ-ಮಿಯಣ
ಮೇಲರಿ-ಮೆಯ
ಮೇಲಾಟಕ್ಕೆ
ಮೇಲಾ-ಯಿತು
ಮೇಲಾಳಿಕೆ
ಮೇಲಾಳ್ಕೆ
ಮೇಲಿಂದ
ಮೇಲಿಕ್ಕಿ
ಮೇಲಿದೆ
ಮೇಲಿದ್ದ
ಮೇಲಿದ್ದು
ಮೇಲಿನ
ಮೇಲಿನಂತೆ
ಮೇಲಿ-ನಂತೆಯೇ
ಮೇಲಿಪಿ-ಳತ್ತೂರು
ಮೇಲಿರು
ಮೇಲಿ-ರುವ
ಮೇಲಿರು-ವುದ-ರಿಂದ
ಮೇಲು
ಮೇಲು-ಕೋಟೆ
ಮೇಲು-ಕೋಟೆ-ಗ-ಳಲ್ಲಿ
ಮೇಲು-ಕೋಟೆ-ಗಳಷ್ಟೇ
ಮೇಲು-ಕೋಟೆ-ಗ-ಳಿಗೆ
ಮೇಲು-ಕೋಟೆಗೂ
ಮೇಲು-ಕೋಟೆಗೆ
ಮೇಲು-ಕೋಟೆಯ
ಮೇಲು-ಕೋಟೆ-ಯಂತಹ
ಮೇಲು-ಕೋಟೆ-ಯನ್ನು
ಮೇಲು-ಕೋಟೆ-ಯಲು
ಮೇಲು-ಕೋಟೆ-ಯಲ್ಲಷ್ಟೇ
ಮೇಲು-ಕೋಟೆ-ಯಲ್ಲಿ
ಮೇಲು-ಕೋಟೆ-ಯಲ್ಲಿದೆ
ಮೇಲು-ಕೋಟೆ-ಯಲ್ಲಿದ್ದ
ಮೇಲು-ಕೋಟೆ-ಯಲ್ಲಿದ್ದಿರ-ಬ-ಹುದು
ಮೇಲು-ಕೋಟೆ-ಯಲ್ಲೇ
ಮೇಲು-ಕೋಟೆ-ಯಿಂದ
ಮೇಲು-ಕೋಟೆಯು
ಮೇಲು-ಕೋಟೆಯೂ
ಮೇಲು-ಗೆಲಸ
ಮೇಲು-ಗೆಲಸ-ವನ್ನು
ಮೇಲುಗೈ
ಮೇಲು-ಗೊಟೆಯ
ಮೇಲು-ಗೋಟೆಯ
ಮೇಲು-ಗೋಟೆಯಲೂ
ಮೇಲು-ಭಾಗ-ದಲ್ಲಿ
ಮೇಲುಸ್ತು-ವಾರಿ-ಯಲ್ಲಿ
ಮೇಲೂ
ಮೇಲೆ
ಮೇಲೆತ್ತಿ
ಮೇಲೆತ್ತುವ
ಮೇಲೆಯೂ
ಮೇಲೆಯೇ
ಮೇಲೆ-ರಗಿ
ಮೇಲೆ-ರಗುತ್ತದೆ
ಮೇಲೆ-ವನ್ದು
ಮೇಲೇರಿ
ಮೇಲೇರಿದ
ಮೇಲೇರಿ-ರುವುದು
ಮೇಲೇರುತ್ತಿದ್ದರು
ಮೇಲ್
ಮೇಲ್ಕಂಡ
ಮೇಲ್ಕಂಡಂತೆ
ಮೇಲ್ಬಾಗ-ದಲ್ಲಿ
ಮೇಲ್ಮಟ್ಟದ
ಮೇಲ್ಮೆ-ಯನ್ನೇ
ಮೇಲ್ಮೆ-ಯಿಂದ
ಮೇಲ್ವಿ-ಚಾರ-ಕನ
ಮೇಲ್ವಿಚಾ-ರಣೆ
ಮೇಲ್ವಿ-ಚಾರ-ಣೆ-ಯಲ್ಲಿ
ಮೇಳ
ಮೇಳನಾ
ಮೇಳಯ್ಯನು
ಮೇಳಾ-ದೇವಿಯ
ಮೇಳಾ-ದೇವಿ-ಯರ
ಮೇಳಾ-ದೇವೀತಿ
ಮೇಳಾ-ಪುರ
ಮೇಳಾ-ಪುರ-ದಲ್ಲಿ
ಮೇಳಿ
ಮೇಳಿಯ
ಮೇಳೀ-ಸಾ-ಸಿರ್ವರು
ಮೇಳೇಶ್ವರ
ಮೇವುಂಡಿ
ಮೇವುಂಡಿ-ಯ-ವರು
ಮೇಷಪಾಷಾಣ
ಮೇಹರ-ನಾಥ-ಗುರು-ಬಾಬಾ
ಮೈತುಂಬಾ
ಮೈತ್ರಿ-ಯನ್ನು
ಮೈದಾನ-ದಲ್ಲಿ
ಮೈದಾನ-ವಿದ್ದು
ಮೈದುನ
ಮೈದುನ-ರಾದ
ಮೈನಾ
ಮೈಭೋಗ
ಮೈಭೋ-ಗಕ್ಕೆ
ಮೈಮರೆತ
ಮೈಮೆಟ್ಟಿ
ಮೈಲನ-ಹಳ್ಳಿ
ಮೈಲನ-ಹಳ್ಳಿ-ಯ-ಕೆರೆಯ
ಮೈಲಳ-ದೇವಿ
ಮೈಲಳ-ದೇವಿ-ಯನ್ನು
ಮೈಲಳ-ದೇವಿಯು
ಮೈಲಿ
ಮೈಲಿ-ಗಳ
ಮೈಸು-ನಾಡು-ಮೈಸೆ-ನಾಡು
ಮೈಸೂರ
ಮೈಸೂ-ರಿಗೆ
ಮೈಸೂರಿನ
ಮೈಸೂರಿ-ನಲ್ಲಿ
ಮೈಸೂರಿ-ನಲ್ಲಿ-ರುವ
ಮೈಸೂರಿ-ನಿಂದ
ಮೈಸೂರು
ಮೈಸೂರು-ತಲ-ಕಾಡು
ಮೈಸೂರ್
ಮೊ
ಮೊಗಚಿ
ಮೊಗ-ಮಾಡಿದೆ
ಮೊಗ-ವನ್ನು
ಮೊಗ-ವಾಡವೋ
ಮೊಜ್ಜನ
ಮೊಟ್ಟ-ಮೊ-ದಲ
ಮೊಟ್ಟ-ಮೊ-ದಲನೇ
ಮೊಟ್ಟ-ಮೊ-ದಲಿಗೆ
ಮೊಟ್ಟೆನವಿಲೆ-ಮಟ್ಟನೋಲೆ
ಮೊಟ್ಟೆ-ನವಿಲೆ-ಯನ್ನು
ಮೊಡನೊಡನೀ
ಮೊಡ-ವನ-ಕೋಡಿ
ಮೊಡ-ವನ-ಕೋಡಿಯ
ಮೊಡ-ವಿನ-ಕೋಡಿ
ಮೊಡ-ವಿನ-ಕೋಡಿಯ
ಮೊತ್ತ
ಮೊತ್ತಕ
ಮೊತ್ತದ
ಮೊತ್ತ-ದಾಳು-ಗಳ
ಮೊತ್ತ-ವನ್ನು
ಮೊತ್ತ-ಹಳ್ಳಿ
ಮೊತ್ತ-ಹಳ್ಳಿಯ
ಮೊತ್ತ-ಹಳ್ಳಿ-ಯಲ್ಲಿದ್ದು
ಮೊದ-ನೆಯ
ಮೊದ-ಮೊ-ದಲ
ಮೊದ-ಮೊ-ದಲು
ಮೊದಲ
ಮೊದಲಂ
ಮೊದಲ-ಗಾಲುವೆ
ಮೊದಲ-ಗಾಲುವೆಯ
ಮೊದಲನೆ
ಮೊದಲ-ನೆಯ
ಮೊದಲ-ನೆ-ಯ-ದಾಗಿ
ಮೊದಲ-ನೆ-ಯದು
ಮೊದಲ-ನೆ-ಯ-ವ-ನಾದ
ಮೊದಲ-ನೆ-ಯ-ವನು
ಮೊದಲನೇ
ಮೊದಲ-ಬಾರಿಗೆ
ಮೊದಲ-ಭಾಗ-ದಲ್ಲಿ
ಮೊದಲ-ಮಗ-ನಿಗೆ
ಮೊದಲಲಿ
ಮೊದಲಲು
ಮೊದಲ-ಹೊನ್ನಿ-ನಲ್ಲಿ
ಮೊದಲಾ-ಗನೇಕ
ಮೊದಲಾಗಿ
ಮೊದಲಾದ
ಮೊದಲಾದಂ
ಮೊದಲಾದ-ವನ್ನು
ಮೊದಲಾದ-ವರ
ಮೊದಲಾದ-ವ-ರನ್ನು
ಮೊದಲಾದ-ವ-ರಿಗೆ
ಮೊದಲಾದ-ವರು
ಮೊದಲಾದ-ವು-ಗ-ಳನ್ನು
ಮೊದಲಾದ-ವು-ಗ-ಳಲ್ಲಿ
ಮೊದಲಾದ-ವು-ಗಳಿದ್ದವು
ಮೊದಲಾ-ದವೂ
ಮೊದಲಾ-ದು-ದಕ್ಕೆ
ಮೊದ-ಲಾ-ಯಿತು
ಮೊದಲಿ
ಮೊದಲಿಂದಂ
ಮೊದಲಿಗ
ಮೊದಲಿ-ಗ-ನಲ್ಲ
ಮೊದಲಿ-ಗ-ರೆಂದು
ಮೊದಲಿಗೆ
ಮೊದಲಿನ
ಮೊದಲಿ-ನಿಂದಲೂ
ಮೊದಲಿ-ಯಣ್ಣನ
ಮೊದಲಿ-ಯಳ್ಳಿಯ
ಮೊದಲಿ-ಯಾರ್
ಮೊದಲಿ-ಯಾರ್ರವ-ರಿಗೆ
ಮೊದಲಿ-ಹಳ್ಳಿ
ಮೊದಲಿ-ಹಳ್ಳಿಯ
ಮೊದಲಿ-ಹಳ್ಳಿ-ಯ-ಕೆರೆ
ಮೊದಲಿ-ಹಳ್ಳಿ-ಯನ್ನು
ಮೊದಲು
ಮೊದಲುಸ್ವಾಂಮ್ಯದ
ಮೊದಲೇ
ಮೊದಲೇರಿ
ಮೊದಲೇ-ರಿಯ
ಮೊದಲೇ-ರಿ-ಯಲು
ಮೊದಲೇ-ರಿ-ಯಲ್ಲಿ
ಮೊದಲ್ಗೊಂಡು
ಮೊದಾಲ
ಮೊದೆ
ಮೊನೆ
ಮೊನೆ-ಮಟ್ಟರ-ಹಳ್ಳಿ
ಮೊನೆ-ಮುಟ್ಟರ-ಹಳ್ಳಿ
ಮೊನೆ-ಯಾಳು
ಮೊನೆ-ಯಾಳ್ತನ
ಮೊನೆ-ಯಾಳ್ತನಂಗೆಯ್ವ
ಮೊನೆ-ಯಾಳ್ತನಂಗೆಯ್ವ-ರಿಗೆ
ಮೊನೆ-ಯಾಳ್ತನಂಗೈವ-ರಿಗೆ
ಮೊನೆ-ಯಾಳ್ತನ-ವನ್ನು
ಮೊನೆ-ಯಾಳ್ವ
ಮೊನೆ-ಯಾಳ್ವರು
ಮೊನೆ-ಯೊಳು
ಮೊನೆ-ಯೊಳ್
ಮೊಮಮ್ಮಕ್ಕಳು
ಮೊಮ್ಮಕ್ಕಳಾದ
ಮೊಮ್ಮಕ್ಕಳು
ಮೊಮ್ಮಗ
ಮೊಮ್ಮಗ-ನಾಗಿ-ರ-ಬ-ಹುದು
ಮೊಮ್ಮಗ-ನಾದ
ಮೊಮ್ಮಗ-ನಿರ-ಬ-ಹುದು
ಮೊಮ್ಮಗ-ನೆಂದೂ
ಮೊಮ್ಮ-ಗನೇ
ಮೊಮ್ಮ-ಗಳು
ಮೊರಡೆ
ಮೊರದ-ನ-ಕಟ್ಟೆ-ಹಳ್ಳ
ಮೊರ-ವನ-ಕಟ್ಟೆ-ಗಳ
ಮೊರ-ಸರು
ಮೊರಸಾದಿ-ರಾಯರು
ಮೊರಸಾಧಿ-ರಾಯ
ಮೊರಸಾಧಿ-ರಾಯ-ರೆಂದು
ಮೊರಸು
ಮೊರಸು-ಕುಲ
ಮೊರಸು-ನಾಡು
ಮೊರಸು-ಮು-ಸುಕು
ಮೊಲ
ಮೊಲಿಗೆ
ಮೊಲೆ-ಗೋಡನ್ನು
ಮೊಲೆವಾಲ
ಮೊಳ-ನಾಡ
ಮೊಳ-ನಾಡಸ್ಥಳದ
ಮೊಸರು
ಮೊಸರುಕ್ರಯ
ಮೊಸರೋ-ಗರದ
ಮೊಸ-ಳೆಯ
ಮೊಹ-ಮದ್
ಮೊಹಮ್ಮದ್
ಮೊಹ-ಲೆಯ
ಮೋಕ್ಷ
ಮೋಕ್ಷ-ತಿಳಕ-ವೆಂಬ
ಮೋಕ್ಷಾಪೇಕ್ಷೆಯ
ಮೋಜಿಣಿಯ
ಮೋಜಿನಿ
ಮೋಡ-ಕು-ಲಕ್ಕೂ
ಮೋಡ-ಕುಲ-ದ-ವನು
ಮೋಡ-ಕುಲ-ಮೋಡೆಯ-ಕುಲಕ್ಕೆ
ಮೋಡ-ಕುಳ-ಕಮಳ
ಮೋಡ-ಹಳ್ಳಿ
ಮೋತಿ-ತಲಾಬ್
ಮೋದಿ-ಖಾನೆ
ಮೋದೀ-ಖಾನೆ
ಮೋದು-ನಾಡಿನ
ಮೋದು-ನಾಡುಕೇ
ಮೋದೂ-ರನ್ನು
ಮೋದೂ-ರಿಗೆ
ಮೋದೂರಿ-ನಲ್ಲಿ
ಮೋದೂರು
ಮೋದೂರು-ನಾಡಿ-ನಲ್ಲಿ
ಮೋದೂರು-ನಾಡು
ಮೋನಾಷ್ಠಾಣ
ಮೋಳಿ-ಯಲ್ಲಿ
ಮೋಸ
ಮೋಹ-ನತ-ರಂಗಿಣಿ
ಮೋಹ-ನತ-ರಂಗಿಣಿ-ಯಲ್ಲಿ
ಮೌಕ್ತಿಕ
ಮೌನ-ನಾಥ-ದಿಂದ
ಮೌನಾನುಷ್ಟಾಣಾ
ಮೌನಾನುಷ್ಟಾನ
ಮೌನಾನುಷ್ಠಾನ
ಮೌರ್ಯ
ಮೌರ್ಯ-ವಂಶದ
ಮೌಲಿ-ಕವು
ಮೌಲಿಮಾಲಾ-ಚರಣಾ-ರ-ವಿನ್ದ
ಮೌಲೂದಿ
ಮೌಲ್ಯ-ಗಳ
ಮೌಲ್ಯ-ಗ-ಳನ್ನು
ಮೌಲ್ಯ-ಗಳಿ-ಗಾಗಿ
ಮೌಲ್ಯಮಾ-ಪನ
ಮೌಲ್ವಿ
ಮ್ಯಾಳ
ಮ್ಯೂಸಿಯಂ
ಮ್ಯೂಸಿಯಂನಲ್ಲಿದೆ
ಮ್ಯೂಸಿಯಂನಲ್ಲಿವೆ
ಮ್ರಿಗಾಧಿ-ಪನೊಳಾ
ಮ್ರಿಡನ
ಮ್ಲೇಚ್ಛ
ಮ್ಲೇಚ್ಛ-ರು-ಗಳ
ಮೞ್ತಿ-ಕಾಲಂಗಳೊಳ್
ಯ
ಯಂ
ಯಂತ್ರಕೂಪ
ಯಂತ್ರ-ಗ-ಳಿಂದ
ಯಂತ್ರ-ಗಳು
ಯಃಕ್ಕುಶಲಃ
ಯಕ್ಷನಂ
ಯಕ್ಷ-ರಾಜ
ಯಕ್ಷ-ರಾಜನು
ಯಜನ
ಯಜ-ಮಾನ
ಯಜ-ಮಾನನ
ಯಜ-ಮಾನ-ನಾಗಿ
ಯಜ-ಮಾನ-ನಾದ
ಯಜ-ಮಾನ-ನೆಂದು
ಯಜ-ಮಾನ-ರಾಗಿ
ಯಜ-ಮಾನಿ-ಕೆ-ಯನ್ನು
ಯಜುರ್ವೇದ
ಯಜು-ಶಾಖೆಯ
ಯಜುಶ್ಶಾಖಾಧ್ಯಾಯಿ
ಯಜುಶ್ಶಾಖಾಧ್ಯಾಯಿ-ಗ-ಳಾದ
ಯಜುಶ್ಶಾಖೆಯ
ಯಜ್ಞ-ಗ-ಳನ್ನು
ಯಜ್ಞ-ಯಾಗಾದಿ-ಗ-ಳನ್ನು
ಯಡ-ಗೋಡಿಯ
ಯಡ-ಗೋಡಿ-ಹಳ್ಳ
ಯಡಲ-ಗೆರೆ-ಅ-ದಲ-ಗೆರೆ
ಯಡ-ವಣ್ಣೆ
ಯಡ-ಹಳ್ಳಿ
ಯತಿ-ಗಳ
ಯತಿ-ಗ-ಳಲ್ಲಿ
ಯತಿ-ಗ-ಳಿಗೆ
ಯತಿ-ಗಳು
ಯತಿ-ಗಿರಿ
ಯತಿಗೆ
ಯತಿ-ಪರಂಪರೆ
ಯತಿ-ಪರಂಪರೆ-ಯನ್ನು
ಯತಿ-ಪರಂಪರೆ-ಯಲ್ಲಿ
ಯತಿ-ಭಿಕ್ಷೆ-ಗಾಗಿ
ಯತಿಯ
ಯತಿ-ಯಾ-ಗಿದ್ದ
ಯತಿ-ಯಾ-ಗಿದ್ದು
ಯತಿ-ಯಾಗಿ-ರುವ
ಯತಿಯು
ಯತಿ-ರಾಜ
ಯತಿ-ರಾಜ-ಮಠ
ಯತಿ-ರಾಜ-ಮಠದ
ಯತಿ-ರಾಜ-ಮಠ-ದಲ್ಲಿ
ಯತಿ-ರಾಜ-ಮಠ-ದ-ವನ್ನು
ಯತಿ-ರಾಜ-ಮಠ-ವನು
ಯತಿ-ರಾಜ-ಮಠ-ವನ್ನು
ಯತಿ-ರಾಜ-ಮಠ-ವನ್ನೂ
ಯತಿ-ರಾಜ-ಮಠ-ವಿದೆ
ಯತಿ-ರಾಜ-ಮಠವು
ಯತಿ-ರಾಜರ
ಯತಿ-ರಾಜರು
ಯತಿ-ರಾಜ-ರೆಂದು
ಯತಿ-ರಾಜ-ಸಪ್ತ-ತಿ-ಯನ್ನು
ಯತಿ-ರಾಜ-ಸಪ್ತ-ಶತಿ-ಯನ್ನು
ಯತಿ-ರಾಜಸ್ಯ
ಯತಿ-ಹೊಯ-ದಟ್ಟ
ಯತೀಂದ್ರನ
ಯತೀಶ್ವರನು
ಯತ್ಕೀರ್ತಿರ್ನವ-ಕುಂದ
ಯತ್ನದಿಂ
ಯತ್ನ-ವನ್ನು
ಯತ್ನಿಸಿ
ಯತ್ನಿಸಿ-ದನು
ಯತ್ನಿಸಿದ್ದಾರೆ
ಯತ್ನಿಸುತ್ತಿದ್ದರು
ಯಥೇಚ್ಛ-ವಾಗಿ
ಯಥೇಷ್ಟಕಂ
ಯದು
ಯದು-ಗಿರಿ
ಯದು-ಗಿರಿಯ
ಯದು-ಗಿರಿ-ಯ-ಪ-ತಿಯೇ
ಯದು-ಗಿರಿ-ಯಲ್ಲಿ
ಯದು-ಗಿರಿ-ಶಿಖರಾ-ಭರಣಂ
ಯದು-ನೃಪಾಳ
ಯದು-ಪಾಟಲಿ
ಯದು-ರಾಜನ
ಯದು-ವಂಶ
ಯದು-ವಂಶದ
ಯದು-ವಂಶ-ದಲ್ಲಿ
ಯದು-ವಂಶ-ಮಹಾಂಭೋದಿ-ಚಂದ್ರ-ಮಾಶ್ಚಂದ್ರ-ಕೀರ್ತಿ-ಮಾನ್
ಯದು-ವಂಶ-ವರ್ಧ-ನ-ಕರಂ
ಯದು-ವಣ್ಣನು
ಯದು-ಶೈಲ-ದೀಪಂ
ಯನ್ನು
ಯಪ್ಪತ್ತಾರು
ಯಮ
ಯಮ-ನಿ-ಯಮ
ಯಮಯಾಂಡನ
ಯಮಿನಃ
ಯಮ್ಮ-ದೂರು
ಯರ
ಯರ-ಗಲ್
ಯರ-ಗುಂಜೆ-ಸೆಟ್ಟಿಯ
ಯರ-ಗುಜೆ-ಸೆಟ್ಟಿ-ಯರು
ಯರಹ-ಳಿಯ
ಯರ-ಹಳ್ಳ
ಯರ-ಹಳ್ಳಿ
ಯಲ-ಚಿ-ಮರ-ವಿದೆ-ಬ-ದರಿ-ವೃಕ್ಷ
ಯಲ-ಬುರ್ಗಿಯ-ಸಿಂಧ
ಯಲ-ವದ
ಯಲ-ವದ-ಪಲ್ಲಿ
ಯಲ-ವದ-ಪಳ್ಳಿ
ಯಲ-ವದ-ಹಳ್ಳಿ
ಯಲ-ವದ-ಹಳ್ಳಿ-ಗ-ಳನ್ನು
ಯಲ-ವದ-ಹಳ್ಳಿ-ಗ-ಳಿಂದ
ಯಲ-ಹಂಕ-ನಾಯ-ಕರು
ಯಲಾದ-ಹಳ್ಳಿ
ಯಲಿ-ವಾಲ
ಯಲು
ಯಲೆ-ಕೊಪ್ಪದ
ಯಲೆ-ಚಾ-ಕನ-ಹಳ್ಳಿಯ
ಯಲ್ಲಪ್ಪಯ್ಯ-ನೆಂಬು-ವ-ವನು
ಯಲ್ಲಯ್ಯ
ಯಲ್ಲಾದ-ಹಳ್ಳಿ
ಯಲ್ಲಾದ-ಹಳ್ಳಿಯ
ಯಲ್ಲಾ-ಪುರ
ಯಲ್ಲಿ
ಯಲ್ಲಿದ್ದು-ಕೊಂಡು
ಯಲ್ಲೂ
ಯಲ್ಲೇ
ಯಳಂದೂರಿನ
ಯಳವಂದೂರು
ಯವರ
ಯವರು
ಯಶಸ್ವಿನಿ
ಯಶಸ್ವಿ-ಯಾಗಿ
ಯಶಸ್ಸಾಗ-ಬೇಕೆಂದು
ಯಶಸ್ಸು
ಯಶಸ್ಸು-ಗ-ಳನ್ನು
ಯಶೋಧನ
ಯಶೋಧರ
ಯಸ್ಮನ್ರಂಜ-ಯತಿ
ಯಸ್ಮಿನ್
ಯಸ್ಯ
ಯಸ್ಯಾಚರ
ಯಾಂಡಆಂಡ
ಯಾಗ
ಯಾಗಿ
ಯಾಗಿದ್ದ
ಯಾಗಿದ್ದನು
ಯಾಗಿದ್ದ-ನೆಂದು
ಯಾಗಿ-ರ-ಬ-ಹುದು
ಯಾಗಿ-ರುತ್ತಿದ್ದನು
ಯಾಚಕಜ-ನಾಭಿವ್ರಿದ್ಧಿ
ಯಾಚನ-ಘಟ್ಟ
ಯಾಚನ-ಹಳ್ಳಿ
ಯಾಚ-ಮಾನ-ಹಳ್ಳಿ
ಯಾಚೈನ-ಹಳ್ಳಿ
ಯಾಜನ
ಯಾಜಿನೇ
ಯಾಡ
ಯಾತಕೂ-ರೊಳ್
ಯಾತಮ್
ಯಾತ್ರೆ
ಯಾತ್ರೆಗೆ
ಯಾತ್ರೆ-ಯನ್ನು
ಯಾದ
ಯಾದ-ಗಿರಿ
ಯಾದ-ಗಿರಿಯ
ಯಾದ-ನಾ-ರಾಯಣ
ಯಾದವ
ಯಾದ-ವ-ಕುಲಾಂಬುಧಿ
ಯಾದ-ವ-ಗರಿ
ಯಾದ-ವ-ಗಿರಿ
ಯಾದ-ವ-ಗಿರಿಗೆ
ಯಾದ-ವ-ಗಿರಿಯ
ಯಾದ-ವ-ಗಿರಿ-ಯನ್ನು
ಯಾದ-ವ-ಗಿರಿ-ಯಲ್ಲಿ
ಯಾದ-ವ-ಗಿರಿ-ಯಾದ
ಯಾದ-ವ-ಗಿರಿ-ಯಿಂದ
ಯಾದ-ವ-ನಾ-ರಾಯಣ
ಯಾದ-ವ-ನಾ-ರಾಯ-ಣ-ಪುರದ
ಯಾದ-ವ-ನಾ-ರಾಯ-ಣ-ಪುರ-ವಾದ
ಯಾದ-ವ-ನಾ-ರಾಯ-ಣ-ಪುರ-ವೆಂಬ
ಯಾದ-ವ-ಪುರ
ಯಾದ-ವ-ಪುರದ
ಯಾದ-ವ-ಪುರ-ದಲ್ಲಿ
ಯಾದ-ವ-ಪುರ-ದಲ್ಲಿದ್ದ-ನೆಂದು
ಯಾದ-ವ-ಪುರವ
ಯಾದ-ವ-ಪುರ-ವನ್ನು
ಯಾದ-ವ-ಪುರ-ವಾದ
ಯಾದ-ವ-ಪುರಿ
ಯಾದ-ವ-ಪುರಿ-ಮೇಲು-ಕೋಟೆ
ಯಾದ-ವ-ಪುರಿಯ
ಯಾದ-ವ-ರನ್ನು
ಯಾದ-ವ-ರಾಜ-ಧಾನಿ-ಯಾದ
ಯಾದ-ವ-ರಾಜ-ನಾದ
ಯಾದ-ವ-ರಾಜ್ಯ-ಲಕ್ಷ್ಮೀ
ಯಾದ-ವ-ಶೈಲ-ದಲ್ಲಿ
ಯಾದ-ವ-ಸ-ಮುದ್ರ
ಯಾದ-ವ-ಸ-ಮುದ್ರದ
ಯಾದ-ವಾಚಲ-ಪತಿ-ಯಾದ
ಯಾದ-ವಾಚಲ-ಪತೇಃ
ಯಾದ-ವಾದ್ರಿ
ಯಾದ-ವಾದ್ರಿ-ಪ-ತಿಯು
ಯಾದ-ವಾದ್ರಿಯ
ಯಾಪನಿಕ
ಯಾಪನೀಯ
ಯಾಮದ
ಯಾಮುನಾ-ಚಾರ್ಯರು
ಯಾರಾದ-ರೊಬ್ಬರ
ಯಾರಿಂದಲೂ
ಯಾರಿಗಾ-ದರೂ
ಯಾರಿದ್ದಾರೆ
ಯಾರು
ಯಾರೂ
ಯಾರೆಂಬು-ದನ್ನು
ಯಾರೆಂಬುದು
ಯಾರೋ
ಯಾರ್ಯಾರು
ಯಾಲಾದ-ಹಳ್ಳಿ
ಯಾವ
ಯಾವ-ಕಾರ-ಣಕ್ಕೋ
ಯಾವತ್ತೂ
ಯಾವ-ದಿಕ್ಕಿಗೆ
ಯಾವ-ದಿಕ್ಕಿನಿಂದ
ಯಾವದೇ
ಯಾವ-ಯಾವ
ಯಾವ-ರೀತಿ
ಯಾವ-ರೀತಿಯ
ಯಾವಾಗ
ಯಾವಾಗಲೂ
ಯಾವು-ದಕ್ಕೆ
ಯಾವು-ದನ್ನು
ಯಾವು-ದನ್ನೂ
ಯಾವು-ದಾದ
ಯಾವು-ದಾ-ದರೂ
ಯಾವು-ದಿದೆ
ಯಾವು-ದಿದ್ದರೂ
ಯಾವುದು
ಯಾವುದೇ
ಯಾವುದೋ
ಯಾವುವು
ಯಾವುವೂ
ಯಿ
ಯಿಂತ
ಯಿಂತ-ನಿಬ-ರಿಗೂ
ಯಿಂತಿ-ವರು-ಭಯ-ಮತದಿಂ
ಯಿಂದ
ಯಿಂದ-ವರದ
ಯಿಂದಾಳ್ದನಾ
ಯಿಂಮಡಿ
ಯಿಂಮಡಿ-ಯಾಗಿ
ಯಿಂಮಾನ-ರೆಯ-ಎ-ರಡೂ-ವರೆ-ಮಾನ
ಯಿಕ್ಕಿ-ಕೊಟ್ಟು
ಯಿಕ್ಕುತ
ಯಿಕ್ಕುತ-ಬ-ಹುದು
ಯಿಕ್ಕುಳ
ಯಿಕ್ಕು-ವರು
ಯಿಗ್ಗ-ಲೂರು
ಯಿತಿ-ಹಾಸಾದಿ
ಯಿದ
ಯಿದಕ್ಕೆ
ಯಿದ-ಱಿಂದಂ
ಯಿದೆ
ಯಿಪ್ಪತ್ತು
ಯಿಪ್ಪತ್ತು-ಸಾಸಿಸ-ರದ
ಯಿಪ್ಪತ್ತೈದು
ಯಿಬಳ
ಯಿರಪ-ವಿ-ರೂಪಾಕ್ಷ-ದೇವ-ರಿಗೆ
ಯಿರು-ಮನ-ಹಳ್ಳಿ
ಯಿಲ್ಲ-ದ-ವರ
ಯಿವರ
ಯೀ
ಯೀಗ
ಯೀಚಣ
ಯೀತಮ
ಯುಗ-ದಲಿ
ಯುಗಳ
ಯುತಾ-ನಾಮಾಭಿವಂದಿತಾಂ
ಯುತಾ-ನಾಮಾಭಿವಂದಿ-ತಾಂಹತ್ತು-ಸಾವಿರ
ಯುದ್ಧ
ಯುದ್ಧ-ಕಾಲ-ದಲ್ಲಿ
ಯುದ್ಧಕ್ಕೂ
ಯುದ್ಧಕ್ಕೆ
ಯುದ್ಧ-ಗ-ಳನ್ನು
ಯುದ್ಧ-ಗ-ಳನ್ನೂ
ಯುದ್ಧ-ಗ-ಳಲ್ಲಿ
ಯುದ್ಧ-ಗ-ಳಾದವು
ಯುದ್ಧ-ಗ-ಳಿಗೆ
ಯುದ್ಧ-ಗಳು
ಯುದ್ಧ-ಘೋಷಣೆ-ಯಾದಾಗ
ಯುದ್ಧದ
ಯುದ್ಧ-ದಲ್ಲಿ
ಯುದ್ಧ-ದಲ್ಲೇ
ಯುದ್ಧ-ಪಟು-ವಾಗುವ
ಯುದ್ಧಪ್ರ-ಭೇದ-ಗ-ಳಲ್ಲಿ
ಯುದ್ಧ-ಭೂಮಿ-ಗ-ಳಲ್ಲಿ
ಯುದ್ಧ-ಭೂಮಿ-ಯಲ್ಲೇ
ಯುದ್ಧ-ಮಾಡಿ
ಯುದ್ಧ-ಮಾಡಿದ
ಯುದ್ಧ-ಮಾಡಿ-ದನು
ಯುದ್ಧ-ಮಾಡಿ-ದು-ದಕ್ಕೆ
ಯುದ್ಧ-ಮಾಡಿ-ದು-ದನ್ನು
ಯುದ್ಧ-ಮಾಡುತ್ತಿದ್ದಾಗ
ಯುದ್ಧ-ರಂಗದ
ಯುದ್ಧ-ರಂಗ-ದಲ್ಲಿ
ಯುದ್ಧ-ವನ್ನು
ಯುದ್ಧ-ವ-ಮಾಡಿ
ಯುದ್ಧ-ವಾಗಿ-ರ-ಬ-ಹುದು
ಯುದ್ಧ-ವಾಗಿ-ರುತ್ತದೆ
ಯುದ್ಧ-ವಾಡುತ್ತಿ-ರುವ
ಯುದ್ಧ-ವಾದ
ಯುದ್ಧ-ವಾದಾಗ
ಯುದ್ಧ-ವಿರ-ಬ-ಹುದು
ಯುದ್ಧ-ವೀರ-ನಾಗಿ-ರ-ಬ-ಹುದು
ಯುದ್ಧ-ವೀರನೂ
ಯುದ್ಧವು
ಯುದ್ಧ-ವೆಂದೇ
ಯುದ್ಧವೇ
ಯುದ್ಧಾ-ನಂತ-ರ-ದಲ್ಲಿ
ಯುದ್ಧಾಯುದಮ
ಯುದ್ಧಾರ-ಚ-ರಣೆಯು
ಯುದ್ಧೋಪ-ಕರ-ಣ-ಗ-ಳನ್ನು
ಯುದ್ಧೋಪ-ಕರ-ಣ-ಗಳು
ಯುಧಿಷ್ಟಿರಾಭಿಷೇಕ
ಯುಧಿಷ್ಠಿರಾಭಿಷೇಕ
ಯುವ-ಕ-ನಾಗಿದ್ದಾಗಲೇ
ಯುವ-ಕ-ರಿಗೆ
ಯುವ-ರಾಜ
ಯುವ-ರಾಜ-ನನ್ನಾಗಿ
ಯುವ-ರಾಜ-ನನ್ನು
ಯುವ-ರಾಜ-ನಾಗಿ
ಯುವ-ರಾಜ-ನಾ-ಗಿದ್ದ
ಯುವ-ರಾಜ-ನಾಗಿದ್ದನು
ಯುವ-ರಾಜ-ನಾಗಿದ್ದ-ನೆಂದು
ಯುವ-ರಾಜ-ನಾಗಿದ್ದಾಗಲೇ
ಯುವ-ರಾಜ-ನಾಗಿದ್ದು-ಕೊಂಡು
ಯುವ-ರಾಜ-ನಾದ
ಯುವ-ರಾಜ-ನೆಂದು
ಯುವ-ರಾಜ-ನೆನಿಸಿದ
ಯುವ-ರಾಜ-ಭದ್ರ-ಬಾಹು
ಯುವ-ರಾಜ-ರನ್ನು
ಯುವ-ರಾಜ-ರಾದ
ಯುವ-ರಾಜರು
ಯುವ-ರಾಜರೂ
ಯುವ-ರಾಜರೇ
ಯುವ-ವೀರ-ರನ್ನು
ಯುಷ್ಮದ್
ಯೂಥರುಂ
ಯೆಂಕಟಪ್ಪ
ಯೆಂಣೆ-ನಾಡನ್ನು
ಯೆಂದೆಂದಿಂಗೇನು
ಯೆಂದೆಂದಿಗೂ
ಯೆಂಬೆರು-ಮಾನರ
ಯೆಡ-ತೊರೆ-ತಾಲ್ಲೂಕು
ಯೆಡ-ದೊರೆ
ಯೆಡ-ವಾಣೆ
ಯೆಡಿ-ಯೂರೇ
ಯೆಡೂರಿ
ಯೆಡೂರು
ಯೆಡೆ-ಯೂರು
ಯೆಣ್ಣೆ-ಎಣ್ಣೆ
ಯೆಣ್ನೆ
ಯೆತ್ತಕದ್ದವ
ಯೆಪ್ಪತ್ತಕ್ಕೆ
ಯೆಪ್ಪತ್ತು
ಯೆಮ್ಮೆ-ಗೆರೆ
ಯೆಮ್ಮೆ-ಗೆರೆಯ
ಯೆಮ್ಮೆಯ-ಕೇತನ
ಯೆಮ್ಮೆಯ-ಕೇತ-ನ-ಹಟ್ಟಿ
ಯೆರಡು
ಯೆರೆ-ಗಂಗ
ಯೆರೆ-ಭೂಮಿ-ಯನ್ನು
ಯೆರೆ-ಯಣ್ಣನು
ಯೆರೆ-ಯಣ್ಣನೂ
ಯೆಲೆ-ಕೊಪ್ಪ
ಯೆಲೆ-ಕೊಪ್ಪದ
ಯೆಲೆ-ಗ-ನೂರ
ಯೆಲೆಚಾ-ಕನ-ಹಳ್ಳಿಯ
ಯೆಲ್ಲಪ್ಪಯ್ಯನು
ಯೆಲ್ಲಾ
ಯೆಲ್ಲೆ
ಯೆಳಂದೂರು
ಯೆಹೂದಿ
ಯೇಕಾದ-ಶಿವ್ರತ-ನಿರತ
ಯೇಚಯ್ಯ
ಯೇಚಿ-ರಾಜ
ಯೇಡೂರಿ
ಯೇಡೂರು
ಯೇತ-ಗದ್ದೆಗೆ
ಯೇನ
ಯೇನೇನುಂಟಾದ
ಯೇರು
ಯೇರು-ಸುಂಕ
ಯೇಲಕ್ಕಿ
ಯೋ
ಯೋಗ
ಯೋಗ-ಗೌಡ
ಯೋಗಣ್ಣ-ಗಳ
ಯೋಗಣ್ಣ-ನಿಗೆ
ಯೋಗ-ನರ-ಸಿಂಹ
ಯೋಗ-ನರ-ಸಿಂಹ-ದೇವಾ-ಲಯ-ದಲ್ಲಿ
ಯೋಗ-ನರ-ಸಿಂಹಸ್ವಾಮಿ
ಯೋಗ-ಭಂಗಿ-ಯಲ್ಲಿ
ಯೋಗಾ-ನಂದ್
ಯೋಗಾ-ನರ-ಸಿಂಹ
ಯೋಗಾ-ನರ-ಸಿಂಹ-ದೇವ-ರಿಗೆ
ಯೋಗಾ-ನರ-ಸಿಂಹಸ್ವಾಮಿ
ಯೋಗಾ-ನರ-ಸಿಂಹಸ್ವಾಮಿಗೆ
ಯೋಗಿ
ಯೋಗಿಪ್ರ-ವರ
ಯೋಗಿಯ
ಯೋಗಿಯು
ಯೋಗೀ
ಯೋಗೀಂದ್ರರ
ಯೋಗೀಶ್ವರ-ತನಯ
ಯೋಗ್ಯ
ಯೋಗ್ಯ-ತೆಗೆ
ಯೋಗ್ಯ-ತೆಯ
ಯೋಗ್ಯ-ವಾದ
ಯೋದಂದ್ರಚೆಲ್ವರ
ಯೋಧರಿದ್ದ-ರೆಂದು
ಯೋಧ-ರು-ಗಳ
ಯೋಧ-ಸೈನಿಕ-ರಾದ
ಯೋಷಿಜ್ಜನ-ವಿ-ನುತೆ
ಯೋಸೌ-ಘಾತಿ
ಯೋಸ್ಯ
ಯ್ದೆಡೆ
ರ
ರಂಗ
ರಂಗಕ್ಷತೀಂದ್ರ
ರಂಗಕ್ಷಿತೀಂದ್ರ
ರಂಗಕ್ಷಿತೀಂದ್ರನ
ರಂಗ-ಗ-ವುಡನ
ರಂಗ-ಗ-ವುಡನು
ರಂಗ-ಗೌಡನು
ರಂಗಣ್ಣ-ನೆಂಬ
ರಂಗದ
ರಂಗ-ದಲ್ಲಿ
ರಂಗ-ಧಾಮಸ್ವಾಮಿಗೆ
ರಂಗನ
ರಂಗ-ನ-ಕೊಪ್ಪಲು
ರಂಗ-ನ-ತಿಟ್ಟು
ರಂಗ-ನಾಥ
ರಂಗ-ನಾಥ-ದೇವರ
ರಂಗ-ನಾಥ-ದೇವ-ರಿಗೆ
ರಂಗ-ನಾಥ-ದೇವರು
ರಂಗ-ನಾಥ-ದೇವಾ-ಲಯ-ದಲ್ಲಿ-ರುವ
ರಂಗ-ನಾಥ-ದೇವಾ-ಲಯ-ದಿಂದ
ರಂಗ-ನಾಥನ
ರಂಗ-ನಾಥ-ನ-ಗರ-ದಲ್ಲಿ
ರಂಗ-ನಾಥ-ನಿಗೆ
ರಂಗ-ನಾಥಸ್ವಾಮಿ
ರಂಗ-ನಾಯಕಿ
ರಂಗ-ನಾಯ-ಕಿಗೆ
ರಂಗ-ನಾಯಕೀ
ರಂಗ-ಪತಿ-ರಾಜಯ್ಯ-ನಿಗೆ
ರಂಗ-ಪತ್ತಿ
ರಂಗ-ಪಯ್ಯ
ರಂಗ-ಪಯ್ಯನು
ರಂಗ-ಪಯ್ಯನೂ
ರಂಗ-ಪುರಿ-ಯಲ್ಲಿ
ರಂಗಪ್ಪ-ನಾಯ-ಕನು
ರಂಗ-ಭೋಗ
ರಂಗ-ಭೋ-ಗಕ್ಕೆ
ರಂಗ-ಮಂಟಪ
ರಂಗ-ಮಂಟಪದ
ರಂಗ-ಮಂಟಪ-ದಲ್ಲಿ-ರುವ
ರಂಗ-ಮಂಟಪ-ವನ್ನು
ರಂಗ-ಮಂಟಪ-ವನ್ನೂ
ರಂಗ-ಮಠ
ರಂಗ-ಮಠದ
ರಂಗ-ಮಠ-ದಲ್ಲಿ
ರಂಗ-ಮಠ-ವನ್ನು
ರಂಗ-ಮಾಂಬ
ರಂಗ-ಮಾಂಬಾ
ರಂಗ-ಮಾಂಬಿಕೆ-ಯನ್ನು
ರಂಗ-ಮಾಂಬಿ-ಕೆಯರು
ರಂಗ-ಮಾಂಬಿ-ಕೆಯು
ರಂಗ-ಮಾಂಬೆ
ರಂಗ-ಮಾಂಬೆಯ
ರಂಗ-ಮಾಂಬೆ-ಯ-ವರು
ರಂಗ-ಮಾಂಬೆಯು
ರಂಗಮ್ಮ-ನ-ವರು
ರಂಗಮ್ಮನು
ರಂಗಯ್ಯ
ರಂಗಯ್ಯನ
ರಂಗಯ್ಯ-ನಾಯಕ
ರಂಗಯ್ಯ-ನಾಯ-ಕನು
ರಂಗ-ರಾಜಯ್ಯನ
ರಂಗ-ರಾಜು
ರಂಗ-ಶೆಟ್ಟಿ
ರಂಗ-ಸ-ಮುದ್ರ
ರಂಗ-ಸ-ಮುದ್ರ-ವನ್ನು
ರಂಗಸ್ವಾಮಿ
ರಂಗಸ್ವಾಮಿ-ಯವುರು
ರಂಗ-ಹಳ್ಳಿ
ರಂಗಾಂಬಯಾ
ರಂಗಾಂಬಾ
ರಂಗಾಂಬಿಕ
ರಂಗಾಂಬಿಕಾ
ರಂಗಾಂಬಿ-ಕೆಗೆ
ರಂಗಾಂಬಿ-ಕೆಯ
ರಂಗಾಂಬಿ-ಕೆಯು
ರಂಗಾಂಬಿ-ಕೆಯೂ
ರಂಗಾ-ಚಾರಿ
ರಂಗಾ-ಚಾರಿಯು
ರಂಗಾ-ಚಾರ್ರಿ
ರಂಗಾ-ಪುರ
ರಂಗಾ-ಪುರದ
ರಂಗೂಗೆ
ರಂಗೆ-ಯ-ನಾಯಕ
ರಂಗೆ-ಯ-ನಾಯ-ಕ-ಕೇತವ್ವೆ
ರಂಗೈಯ್ಯನುಂ
ರಂಜ-ಯನ್ನ-ಖಿಲಾಃ
ರಂಜಿತ-ವಾಗಿ
ರಂದು
ರಂಮ್ಯೇಗ್ರಹಾನ್ನಿರ್ಮಾಯ
ರಕ
ರಕ್ಕಸ-ಗಂಗ
ರಕ್ಕಸ-ಗಂಗನ
ರಕ್ಕಸ-ಗಂಗನು
ರಕ್ಕಸ-ಗಂಗ-ನೆಂಬ
ರಕ್ಕಸಗಿ
ರಕ್ತ-ಕೊಡು-ಗೆ-ಯನ್ನು
ರಕ್ತ-ಗೊಡುಗೆ
ರಕ್ತದ
ರಕ್ತಪಾತ-ವಾಗ-ಲಿಲ್ಲ
ರಕ್ತ-ಸಂಬಂಧಿ-ಗಳು
ರಕ್ತಾ-ಕೊಡಗಿ
ರಕ್ಷಕ
ರಕ್ಷಣಾ
ರಕ್ಷಣಾಂಗ
ರಕ್ಷಣಾಯ
ರಕ್ಷಣೀ-ಯಮ್
ರಕ್ಷಣೆ
ರಕ್ಷಣೆ-ಗಾಗಿ
ರಕ್ಷಣೆ-ಗಾ-ಗಿಯೇ
ರಕ್ಷಣೆಗೆ
ರಕ್ಷಣೆಯ
ರಕ್ಷಣೆ-ಯಲ್ಲಿ
ರಕ್ಷಣೆ-ಯಲ್ಲಿಟ್ಟು
ರಕ್ಷಾ-ಕರಃ
ರಕ್ಷಾ-ಪಾಳಕ-ರಾಗಿದ್ದರು
ರಕ್ಷಾಪಾಳ-ರಾಗಿದ್ದ-ರೆಂದು
ರಕ್ಷಿಪಂ
ರಕ್ಷಿಪ್ಪ
ರಕ್ಷಿಸಬೇಕೆಂದೂ
ರಕ್ಷಿ-ಸಲು
ರಕ್ಷಿಸಿ
ರಕ್ಷಿಸಿ-ಕೊಂಡು
ರಕ್ಷಿಸುತ್ತಿದ್ದವು
ರಕ್ಷಿ-ಸುವ
ರಗಳೆ-ಗ-ಳಲ್ಲಿ
ರಗಳೆ-ಯಲ್ಲಿ
ರಚನೆ
ರಚನೆ-ಗ-ಳನ್ನು
ರಚನೆ-ಗ-ಳಾಗಿವೆ
ರಚನೆ-ಗಳು
ರಚನೆ-ಗಾಗಿ
ರಚ-ನೆಗೆ
ರಚ-ನೆಯ
ರಚನೆ-ಯಂತೆ
ರಚನೆ-ಯನ್ನು
ರಚನೆ-ಯಲ್ಲಿ
ರಚನೆ-ಯಾಗಿ
ರಚನೆ-ಯಾಗಿದೆ
ರಚನೆ-ಯಾ-ಗಿದ್ದು
ರಚನೆ-ಯಾಗಿರ
ರಚನೆ-ಯಾಗಿ-ರ-ಬ-ಹುದು
ರಚನೆ-ಯಾಗಿ-ರ-ಬಹು-ದೆಂದು
ರಚನೆ-ಯಾಗಿ-ರ-ಲಿಲ್ಲ
ರಚನೆ-ಯಾಗಿ-ರು-ವಂತೆ
ರಚನೆ-ಯಾದ
ರಚನೆ-ಯಾದರೂ
ರಚನೆ-ಯಾದವು
ರಚ-ನೆಯೇ
ರಚಿತ-ವಾ-ಗಿದ್ದ
ರಚಿತ-ವಾ-ಗಿದ್ದು
ರಚಿತ-ವಾಗಿ-ರ-ಬ-ಹುದು
ರಚಿತ-ವಾಗಿ-ರುವ
ರಚಿತ-ವಾದ
ರಚಿಸ-ಬ-ಹುದು
ರಚಿಸ-ಲಾಗಿದೆ
ರಚಿಸ-ಲಾದ
ರಚಿ-ಸ-ಲಾ-ಯಿತು
ರಚಿಸಿ
ರಚಿಸಿ-ಕೊಂಡಿದ್ದ
ರಚಿಸಿ-ಕೊಂಡಿದ್ದಂತೆ
ರಚಿಸಿ-ಕೊಂಡಿದ್ದರು
ರಚಿಸಿ-ಕೊಂಡು
ರಚಿಸಿ-ಕೊಳ್ಳದ
ರಚಿಸಿದ
ರಚಿಸಿ-ದ-ನೆಂದೂ
ರಚಿಸಿ-ದರು
ರಚಿಸಿದ್ದಾನೆ
ರಚಿಸಿದ್ದಾ-ನೆಂದೂ
ರಚಿಸಿದ್ದಾರೆ
ರಚಿಸಿ-ರ-ಬಹು-ದಾದ
ರಚಿಸಿ-ರುತ್ತಾರೆ
ರಚಿಸಿ-ರುವ
ರಚಿಸಿ-ರು-ವಂತೆ
ರಚಿಸುತ್ತಿದ್ದ
ರಚಿಸುತ್ತಿದ್ದರು
ರಜತ-ಪ-ರಿಯಂಕ
ರಜತ-ಪರ್ಯಂಕ
ರಜತಾ-ಭರಣ
ರಜಯ್ಯ
ರಜಾಕ್
ರಟ್ಟ
ರಟ್ಟ-ಪಾಡಿ
ರಟ್ಟ-ಪಾ-ವಾಡಿ
ರಟ್ಟರ
ರಟ್ಟಿ-ಹಳ್ಳಿ-ಗಳು
ರಟ್ಟೆ
ರಠ-ಹಳ್ಳಿಯ
ರಣ
ರಣಕಲ-ಕೇತ
ರಣದುಲ್ಲಾಖಾ-ನನ
ರಣದುಲ್ಲಾ-ಖಾನ್
ರಣ-ದೊಳ್ಸಾ-ಸಿರ್ವ್ವರು
ರಣ-ಧೀರ
ರಣ-ಧೀರ-ಕಂಠೀ-ರವನ
ರಣ-ಧೀರ-ಕಂಠೀ-ರವನು
ರಣ-ಧೀರ-ಕಂಠೀ-ರವರ
ರಣಪಾರ
ರಣಪಾ-ರರ್
ರಣ-ಬೋವನು
ರಣ-ಮುಖ
ರಣ-ರಂಗ-ಕೇ-ಸರಿ
ರಣ-ರಂಗಕ್ಕೆ
ರಣ-ರಂಗದ
ರಣ-ರಂಗ-ದಲ್ಲಿ
ರಣ-ರಂಗ-ದಿಂದ
ರಣ-ರಂಗ-ಧೀರ-ನಾಯ್ಕ
ರಣ-ರಂಗ-ಧೀ-ರನುಂ
ರಣ-ರಂಗ-ವಾಗುತ್ತದೆ
ರಣ-ವಿಕ್ರಮಾರ್ಯ
ರಣಾವ-ಲೋಕ
ರಣಿತ-ಗವುಂಡ
ರಣಿತ-ಗವುಂಡನ
ರಣಿತ-ಗವುಂಡನು
ರಣಿತ-ಗವುಂಡ-ನುದ್ಭವಿಸಿ
ರಣಿ-ಭಾಟು
ರಣಿರಾಉ-ಪದ-ಭೋಗ
ರತಾನ್ಚ್ಛುಚೀನ್
ರತ್ನ
ರತ್ನ-ಕರ್ಮ್ಮ
ರತ್ನತ್ರಯ
ರತ್ನತ್ರಯ-ದಂತೆ
ರತ್ನತ್ರಯಾ-ಕರಂ
ರತ್ನತ್ರಯೆ-ಯರು
ರತ್ನ-ಧೇನು
ರತ್ನ-ಪಡಿ
ರತ್ನ-ಪಾಲ
ರತ್ನ-ಭಾರಣ
ರತ್ನ-ವೆಂದು
ರತ್ನ-ಸಿಂಹಾಸ-ನ-ದಿಂದ
ರತ್ನ-ಸಿಂಹಾಸ-ನಾರೂಢ-ನಾಗಿ
ರತ್ನ-ಸಿಂಹಾಸ-ನಾರೂಢ-ನಾಗಿದ್ದ-ನೆಂದು
ರತ್ನ-ಸಿಂಹಾಸ-ನಾರೂಢ-ರಾಗಿ
ರತ್ನ-ಸಿಂಹಾಸನೇ
ರತ್ನಾ-ಕರ
ರತ್ನಾ-ಭರಣ
ರತ್ನಾಯಿಗೆ
ರಥ
ರಥ-ಗಳು
ರಥ-ವನ್ನು
ರಥ-ಸಪ್ತ-ಮಿಯ
ರಥೋತ್ಸವ
ರಥೋತ್ಸ-ವಕ್ಕೆ
ರಥೋತ್ಸವ-ಗ-ಳಿಗೆ
ರಥೋತ್ಸವ-ಗಳು
ರಥೋತ್ಸವದ
ರದ್ದು
ರದ್ದು-ಗೊ-ಳಿಸಿ
ರದ್ದು-ಪ-ಡಿಸಿ
ರದ್ದು-ಮಾಡಿ
ರನ್ನ
ರನ್ನ-ಕವಿ
ರನ್ನ-ಕವಿಯು
ರನ್ನನ
ರನ್ನನು
ರನ್ನಯ್ಯ
ರಮಣೀ
ರಮಣೀಯ
ರಮಾ-ನಾಥಸ್ವಾಮಿ
ರಮಾರ-ಮಣಂ
ರಮಾ-ವಿಲಾಸದ
ರಮಾ-ವಿಳಾಸದ
ರಮ್ಯ
ರಮ್ಯಂ
ರಮ್ಯಃ
ರಮ್ಯ-ವಾಗಿದೆ
ರಮ್ಯ-ವಾದ
ರಮ್ಯೇ
ರಲು
ರಲ್ಲ-ವೆಂದೂ
ರಲ್ಲಾ-ಯಿತು
ರಲ್ಲಿ
ರಲ್ಲಿದ್ದ
ರಲ್ಲಿಯೂ
ರಲ್ಲೇ
ರವ
ರವಮಂ
ರವರ
ರವರು
ರವರೆ-ಗಿನ
ರವರೆಗೂ
ರವರೆಗೆ
ರವ-ಳ-ಬೋವ-ನಧೀಶನ
ರವ-ಳ-ಬೋವನು
ರವ-ಳ-ಬೋವನುಂ
ರವಿ
ರವಿಗ
ರವಿ-ಚಂದ್ರ-ಸಿದ್ಧಾಂತಿ-ಗಳು
ರವೆಗೆ
ರಸಂ
ರಸಾವಾತ್ರೇಯ
ರಸೋದ್ಘಾಟಕೇ
ರಸ್ತೆ
ರಸ್ತೆಯ
ರಸ್ತೆ-ಯನ್ನು
ರಸ್ತೆ-ಯಲ್ಲಿ-ರುವ
ರಹಕ್ಕೆ-ಇ-ತರೆ
ರಹ-ಗೌಡ
ರಹಸ್ಯ-ವನ್ನು
ರಹಿತ-ವಾಗಿ
ರಹಿತ-ವಾಗಿ-ರುವು-ದ-ರಿಂದ
ರಹಿತ-ವಾದ
ರಾ
ರಾಂಪುರ
ರಾಂಪುರದ
ರಾಕಂಣ್ನಂಗೆ
ರಾಕಾಸಮ್ಮನ
ರಾಕ್ಷಸಿ
ರಾಗಮುಣ-ಗಾ-ಮುಂಡ
ರಾಗಿ
ರಾಗಿದ್ದ-ರೆಂದು
ರಾಗಿ-ಹುಲ್ಲು
ರಾಘಣ್ಣ-ದೇವನ
ರಾಘವಾ-ಪುರ
ರಾಘವೇಂದ್ರಸ್ವಾಮಿ
ರಾಚಂದ್ರ-ದೇವರ
ರಾಚನ-ಹಳ್ಳಿ-ಯನ್ನು
ರಾಚಪ್ಪ
ರಾಚಪ್ಪಾಜೀ
ರಾಚ-ಮಲ್ಲ
ರಾಚ-ಮಲ್ಲಂ
ರಾಚ-ಮಲ್ಲನ
ರಾಚ-ಮಲ್ಲ-ನನ್ನು
ರಾಚ-ಮಲ್ಲ-ನೀತಿ-ಮಾರ್ಗ
ರಾಚ-ಮಲ್ಲನು
ರಾಚಯ್ಯ-ನಾಯ-ಕನ
ರಾಚ-ವಲ್ಲ
ರಾಚ-ವೂರು
ರಾಚೆಯಂ
ರಾಚೆಯ-ನಾಯಕ
ರಾಜ
ರಾಜ-ಒಡೆ-ಯನು
ರಾಜ-ಕಂಠೀ-ರವೇಂದ್ರ
ರಾಜ-ಕಾಂಅð-ವನ್ನು
ರಾಜ-ಕಾರ-ಣ-ದಲ್ಲೇ
ರಾಜ-ಕಾರ್ಯ
ರಾಜ-ಕಾರ್ಯಕ್ಕಾಗಿ
ರಾಜ-ಕಾರ್ಯದ
ರಾಜ-ಕೀಯ
ರಾಜ-ಕೀಯ-ಕಾರ-ಣ-ಗ-ಳಿಂದ
ರಾಜ-ಕೀಯ-ದಲ್ಲಿ
ರಾಜ-ಕೀಯ-ದಿಂದ
ರಾಜ-ಕೀಯ-ರಂಗ
ರಾಜ-ಕುಂಜರ
ರಾಜ-ಕು-ಮಾರ
ರಾಜ-ಕು-ಮಾರರ
ರಾಜ-ಕು-ಮಾರ-ರನ್ನು
ರಾಜ-ಕು-ಮಾರರೇ
ರಾಜ-ಕು-ಮಾರಿ
ರಾಜ-ಕೇ-ಸರಿ-ವರ್ಮ
ರಾಜ-ಗುರು
ರಾಜ-ಗುರು-ಗಳ
ರಾಜ-ಗುರು-ಗ-ಳನ್ನೇ
ರಾಜ-ಗುರು-ಗ-ಳಾಗಿದ್ದರು
ರಾಜ-ಗುರು-ಗ-ಳಾಗಿದ್ದು
ರಾಜ-ಗುರು-ಗಳಿದ್ದ-ರೆಂಬು-ದನ್ನು
ರಾಜ-ಗುರು-ಗಳು
ರಾಜ-ಗುರು-ಗ-ಳೆಂದು
ರಾಜ-ಗುರು-ವಾ-ಗಿದ್ದ
ರಾಜ-ಗುರು-ವಾಗಿದ್ದಾನೆ
ರಾಜ-ಗುರು-ವಿಗೆ
ರಾಜ-ಚಿಹ್ನೆ-ಗ-ಳನ್ನು
ರಾಜ-ಧನತ್ವಕ್ಕೆ
ರಾಜ-ಧರ್ಮ್ಮೇಣ
ರಾಜ-ಧಾನಿ
ರಾಜ-ಧಾನಿ-ಗ-ಳಾದ
ರಾಜ-ಧಾನಿ-ಗಳು
ರಾಜ-ಧಾನಿಗೆ
ರಾಜ-ಧಾನಿಯ
ರಾಜ-ಧಾನಿ-ಯನ್ನ
ರಾಜ-ಧಾನಿ-ಯನ್ನಾಗಿ
ರಾಜ-ಧಾನಿ-ಯನ್ನು
ರಾಜ-ಧಾನಿ-ಯಲ್ಲಿ
ರಾಜ-ಧಾನಿ-ಯ-ವರೆಗೆ
ರಾಜ-ಧಾನಿ-ಯಾ-ಗಿತ್ತು
ರಾಜ-ಧಾನಿ-ಯಾ-ಗಿದ್ದ
ರಾಜ-ಧಾನಿ-ಯಾ-ಗಿದ್ದ-ರಿಂದ
ರಾಜ-ಧಾನಿ-ಯಾ-ಗಿದ್ದರೂ
ರಾಜ-ಧಾನಿ-ಯಾ-ಗಿದ್ದು
ರಾಜ-ಧಾನಿ-ಯಾದ
ರಾಜ-ಧಾನಿ-ಯಾ-ಯಿತು
ರಾಜ-ಧಿ-ರಾಜ
ರಾಜನ
ರಾಜ-ನಂತಹ
ರಾಜ-ನನ
ರಾಜ-ನನ್ನು
ರಾಜ-ನ-ಸಂಬಂಧಿ-ಕರು
ರಾಜ-ನ-ಹಳ್ಳ
ರಾಜ-ನ-ಹಿ-ರಿಯ
ರಾಜ-ನ-ಹೆಸ-ರಿಲ್ಲ
ರಾಜ-ನ-ಹೆ-ಸರು
ರಾಜ-ನಾ-ಗಿದ್ದ
ರಾಜ-ನಾಗಿದ್ದನು
ರಾಜ-ನಾ-ಗಿದ್ದು
ರಾಜ-ನಾದ
ರಾಜ-ನಾದನು
ರಾಜ-ನಿಂದ
ರಾಜ-ನಿಗೆ
ರಾಜ-ನಿಗೇ
ರಾಜ-ನೀ-ತಿಯ
ರಾಜನು
ರಾಜನೂ
ರಾಜ-ನೃಪ
ರಾಜ-ನೆಂದು
ರಾಜ-ನೆನಿಸಿದ
ರಾಜನೇ
ರಾಜ-ನೊಡನೆ
ರಾಜನೋ
ರಾಜ-ಪಂಡಿ-ತ-ನಿಗೆ
ರಾಜ-ಪರ-ಮೇಶ್ವರ
ರಾಜ-ಪರ-ಮೇಶ್ವರಂಯಾದ-ವ-ಕುಲಾಂಬುದಿ
ರಾಜ-ಪರ-ಮೇಶ್ವರಃ
ರಾಜ-ಪರ-ಮೇಶ್ವರ-ನೆಂದು
ರಾಜ-ಪೂಜಿ-ತರೂ
ರಾಜ-ಪೂಜ್ಯ-ರಾಗಿದ್ದ-ರೆಂದು
ರಾಜಪ್ಪ
ರಾಜಪ್ರತಿ-ನಿಧಿಯ
ರಾಜ-ಬೆವ-ಹಾರಿ
ರಾಜ-ಬೆವ-ಹಾರಿ-ಯಾ-ಗಿದ್ದ
ರಾಜ-ಭಂಡಾರಕ್ಕೆ
ರಾಜ-ಭಟರೇ
ರಾಜ-ಮನೆ-ತನ
ರಾಜ-ಮನೆ-ತನ-ಗಳ
ರಾಜ-ಮನೆ-ತನ-ಗಳಿಗೂ
ರಾಜ-ಮನೆ-ತನ-ಗಳು
ರಾಜ-ಮನೆ-ತ-ನದ
ರಾಜ-ಮನೆ-ತನ-ದ-ವ-ರನ್ನು
ರಾಜ-ಮನೆ-ತನ-ದ-ವ-ರಿಗೂ
ರಾಜ-ಮನೆ-ತನ-ವನ್ನು
ರಾಜ-ಮನೆ-ತನ-ವಿರ-ಬ-ಹುದು
ರಾಜ-ಮನ್ನಣೆ
ರಾಜ-ಮಲ್ಲ
ರಾಜ-ಮಲ್ಲನ
ರಾಜ-ಮಲ್ಲ-ನನ್ನು
ರಾಜ-ಮಲ್ಲ-ನಿಗೆ
ರಾಜ-ಮಲ್ಲನು
ರಾಜ-ಮಲ್ಲರ
ರಾಜ-ಮಹಾ-ರ-ಜರ
ರಾಜ-ಮಹಾ-ರಾಜರು
ರಾಜ-ಮಹಾ-ರಾಜ-ರೆಂದಲ್ಲ
ರಾಜ-ಮಹೇಂದ್ರಿ-ಯನ್ನು
ರಾಜ-ಮಾಂನ್ಯ
ರಾಜ-ಮಾನ್ಯ
ರಾಜ-ಮಾರ್ತಾಂಡ-ನೆಂಬ
ರಾಜ-ಮುಡಿ
ರಾಜ-ಮುದ್ರೆಯ
ರಾಜಯ್ಯ
ರಾಜಯ್ಯನ
ರಾಜರ
ರಾಜ-ರಂತೆ
ರಾಜ-ರದು
ರಾಜ-ರನ್ನಾಗಿ
ರಾಜ-ರನ್ನು
ರಾಜ-ರಾಗಿದ್ದರು
ರಾಜ-ರಾಜ
ರಾಜ-ರಾಜ-ಚೋಳನ
ರಾಜ-ರಾಜ-ಚೋಳ-ನಿಗೆ
ರಾಜ-ರಾಜ-ಚೋಳನು
ರಾಜ-ರಾಜ-ಚೋಳ-ನೆಂದು
ರಾಜ-ರಾಜ-ದೇವನ
ರಾಜ-ರಾಜ-ದೇವನು
ರಾಜ-ರಾಜ-ನನ್ನು
ರಾಜ-ರಾಜ-ಪುರ
ರಾಜ-ರಾಜ-ಪುರದ
ರಾಜ-ರಾಜ-ಪುರ-ದಲ್ಲಿ-ರು-ವಾಗ
ರಾಜ-ರಾಜ-ಪುರ-ವಾದ
ರಾಜ-ರಾಜಶ್ರೀ
ರಾಜ-ರಾಜ-ಸಮಾಂಹತಿಃ
ರಾಜ-ರಾಜೇಶ್ವರ
ರಾಜ-ರಿಂದ
ರಾಜ-ರಿಗೆ
ರಾಜರು
ರಾಜ-ರು-ಗಳ
ರಾಜರೂ
ರಾಜ-ರೆಲ್ಲರಂ
ರಾಜರೇ
ರಾಜ-ವಂಶಕ್ಕೆ
ರಾಜ-ವಂಶ-ಗ-ಳಲ್ಲಿ
ರಾಜ-ವಂಶ-ಗ-ಳಾಗಿ
ರಾಜ-ವಂಶ-ಗ-ಳಿಗೆ
ರಾಜ-ವಂಶ-ಗಳು
ರಾಜ-ವಂಶದ
ರಾಜ-ವಂಶ-ವಾ-ಗಿತ್ತು
ರಾಜ-ವಡೇರ
ರಾಜ-ವರ್ತಕ
ರಾಜ-ವಿದ್ಯಾ-ಧರ
ರಾಜ-ವೆವ-ಹಾರಿ-ಯೊಬ್ಬ-ನನ್ನು
ರಾಜ-ವೊಡೆ-ಯನ
ರಾಜ-ವೊಡೆ-ಯರ
ರಾಜ-ವೊಡೆ-ಯ-ರಿಗೆ
ರಾಜ-ವೊಳಲ
ರಾಜ-ವೊಳಲಾಗಿರ-ಬ-ಹುದು
ರಾಜ-ಶಿಕ್ಷೆ
ರಾಜಶ್ರೀ
ರಾಜಶ್ರೇಷ್ಠಿ-ಗಳು
ರಾಜಶ್ರೇಷ್ಠಿ-ಗಳ್
ರಾಜ-ಸಭಾ-ಯೋಗ್ಯ-ನಾಗಿದ್ದನು
ರಾಜ-ಸಿಂಹಾಸ-ನ-ವನ್ನು
ರಾಜತ್ರೀ-ಯರನ್ನಾ-ಗಲೀ
ರಾಜಸ್ಯ
ರಾಜಾ
ರಾಜಾ-ದಿತ್ಯ
ರಾಜಾ-ದಿತ್ಯನ
ರಾಜಾ-ದಿತ್ಯ-ನನ್ನು
ರಾಜಾ-ದಿತ್ಯನು
ರಾಜಾ-ದಿತ್ಯನೂ
ರಾಜಾ-ದಿತ್ಯರು
ರಾಜಾ-ಧಿ-ರಾಜ
ರಾಜಾ-ಧಿ-ರಾಜಃ
ರಾಜಾ-ಧಿ-ರಾಜ-ಬಿರುದೋ
ರಾಜಾ-ಧಿ-ರಾಜ-ಯಿತ್ಯುಕ್ತೋ
ರಾಜಾ-ಧಿ-ರಾಜೇಂದ್ರ-ನೆಂದು
ರಾಜಾಧ್ಯಕ್ಷ
ರಾಜಾನ್ವಯ-ದೊಕ್ಕಲ
ರಾಜಾ-ರ-ಮಡು
ರಾಜಾ-ರಾಮ
ರಾಜಾ-ವಳಿ
ರಾಜಾಶ್ರಯ
ರಾಜಾಸ್ಥಾನ-ದಲ್ಲಿ
ರಾಜೇಂದ್ರ
ರಾಜೇಂದ್ರ-ಚೋಳ
ರಾಜೇಂದ್ರ-ಚೋಳನ
ರಾಜೇಂದ್ರ-ಚೋಳನು
ರಾಜೇಂದ್ರ-ಚೋಳನೇ
ರಾಜೇಂದ್ರಪ್ಪ
ರಾಜೇ-ಗೌಡ
ರಾಜೇ-ಗೌಡರು
ರಾಜೇನ್ದ್ರಚೋೞ
ರಾಜೇಶ
ರಾಜೇಶ್ವರಿ
ರಾಜೊಡೆಯರ
ರಾಜೋದ್ಯಾನ-ವನ-ವನ್ನು
ರಾಜ್ಞಃ
ರಾಜ್ಯ
ರಾಜ್ಯಂ
ರಾಜ್ಯಂಗೆಯು-ತಂಮಿ-ರಲು
ರಾಜ್ಯಂಗೆ-ಯುತ್ತ-ಮಿರೆ
ರಾಜ್ಯಂಗೆ-ಯುತ್ತಿರಲಿಕ್ಕಾಗಿ
ರಾಜ್ಯಂಗೆಯೆ
ರಾಜ್ಯಂಗೆಯ್ಯುತ್ತ-ಮಿರೆ
ರಾಜ್ಯಂಗೆಯ್ಯುತ್ತಿದ್ದನು
ರಾಜ್ಯಂಗೆಯ್ಯುತ್ತಿದ್ದ-ರೆಂದು
ರಾಜ್ಯಂಗೆಯ್ಯುತ್ತಿ-ರಲುದ
ರಾಜ್ಯಂಗೆಯ್ಯೆ
ರಾಜ್ಯಂಗೈ-ಉತ್ತಮಿ-ರಲು
ರಾಜ್ಯ-ಕಾರ್ಯಕ್ಕಾಗಿ
ರಾಜ್ಯ-ಕಾರ್ಯ-ವನ್ನು
ರಾಜ್ಯಕ್ಕೀತಂ
ರಾಜ್ಯಕ್ಕೆ
ರಾಜ್ಯ-ಗತಂ
ರಾಜ್ಯ-ಗಳ
ರಾಜ್ಯ-ಗಳನ್ನಾಗಿ
ರಾಜ್ಯ-ಗ-ಳನ್ನು
ರಾಜ್ಯ-ಗ-ಳಲ್ಲಿ
ರಾಜ್ಯ-ಗಳಲ್ಲಿಯೂ
ರಾಜ್ಯ-ಗ-ಳಾಗಿ
ರಾಜ್ಯ-ಗಳಿಗೂ
ರಾಜ್ಯ-ಗಳು
ರಾಜ್ಯ-ಚಿಹ್ನೆಯೂ
ರಾಜ್ಯ-ತಂತ್ರ-ಗಳೂ
ರಾಜ್ಯದ
ರಾಜ್ಯ-ದಲ್ಲಿ
ರಾಜ್ಯ-ದಲ್ಲಿತ್ತೆಂದು
ರಾಜ್ಯ-ದಿಂದ
ರಾಜ್ಯ-ದೇಶ-ಸೀಮೆ
ರಾಜ್ಯ-ದೊಳಗೆ
ರಾಜ್ಯ-ದೊಳ್
ರಾಜ್ಯ-ಪಾಲ-ನಾಗಿದ್ದನು
ರಾಜ್ಯ-ಪಾಲ-ರು-ಗಳು
ರಾಜ್ಯ-ಭರ
ರಾಜ್ಯ-ಭಾರ
ರಾಜ್ಯ-ಭಾರಕ್ಕೆ
ರಾಜ್ಯ-ಭಾರದ
ರಾಜ್ಯಭ್ರಷ್ಟ-ನನ್ನಾಗಿ
ರಾಜ್ಯಭ್ರಷ್ಟ-ನಾದ
ರಾಜ್ಯ-ಮಧ್ಯೇ
ರಾಜ್ಯ-ಲಕ್ಷ್ಮಿ
ರಾಜ್ಯ-ಲಕ್ಷ್ಮಿಯ
ರಾಜ್ಯ-ಲಕ್ಷ್ಮಿ-ಯನ್ನು
ರಾಜ್ಯ-ವನಾಳುತ್ತಮಿದ್ದ
ರಾಜ್ಯ-ವನ್ನಾಗಿ
ರಾಜ್ಯ-ವನ್ನು
ರಾಜ್ಯ-ವಾಳ-ತೊಡಗಿದ-ನೆಂದು
ರಾಜ್ಯ-ವಾಳದೇ
ರಾಜ್ಯ-ವಾಳ-ಲಿಲ್ಲ
ರಾಜ್ಯ-ವಾಳ-ಲಿಲ್ಲ-ವೆಂದು
ರಾಜ್ಯ-ವಾಳಲು
ರಾಜ್ಯ-ವಾಳಿದ
ರಾಜ್ಯ-ವಾಳಿ-ದನು
ರಾಜ್ಯ-ವಾಳಿ-ದ-ನೆಂದು
ರಾಜ್ಯ-ವಾಳಿ-ದ-ನೆಂದೂ
ರಾಜ್ಯ-ವಾಳಿ-ದ-ವನು
ರಾಜ್ಯ-ವಾಳುತ್ತಿದ್ದ
ರಾಜ್ಯ-ವಾಳುತ್ತಿದ್ದನು
ರಾಜ್ಯ-ವಾಳುತ್ತಿದ್ದ-ನೆಂದು
ರಾಜ್ಯ-ವಾಳುತ್ತಿದ್ದ-ರೆಂದು
ರಾಜ್ಯ-ವಾಳುತ್ತಿದ್ದಾಗ
ರಾಜ್ಯ-ವಿತ್ತು
ರಾಜ್ಯ-ವಿದ್ದಿತು
ರಾಜ್ಯ-ವಿಸ್ತ-ರಣೆ-ಯನ್ನು
ರಾಜ್ಯವು
ರಾಜ್ಯ-ವೆಂದು
ರಾಜ್ಯ-ವೆಂಬ
ರಾಜ್ಯಶ್ರೀ
ರಾಜ್ಯ-ಸಂವತ್ಸರ-ದಲ್ಲಿ
ರಾಜ್ಯ-ಸಮಾಗ-ಮಾಧ್ವ-ಪರಿಘಸ್ತೇ
ರಾಜ್ಯಸ್ಥಳಕ್ಕೆ
ರಾಜ್ಯಸ್ಥಾ-ಪನೆ
ರಾಜ್ಯಾಡಳಿತ
ರಾಜ್ಯಾಡಳಿ-ತಕ್ಕೆ
ರಾಜ್ಯಾಡಳಿತ-ವನ್ನು
ರಾಜ್ಯಾಡಳಿ-ತವು
ರಾಜ್ಯಾಧಿಪ
ರಾಜ್ಯಾಧಿ-ಪತಿ
ರಾಜ್ಯಾಧಿಪ-ತಿ-ಗಳು
ರಾಜ್ಯಾಧಿಪ-ತಿ-ಯಾ-ಗಿದ್ದ
ರಾಜ್ಯಾಧಿಪ-ತಿ-ಯಾಗಿದ್ದ-ನೆಂದು
ರಾಜ್ಯಾಪ-ಹಾರಕ್ಕೆ
ರಾಜ್ಯಾಭಿ-ವೃದ್ಧಿಗೆ
ರಾಜ್ಯಾಭಿಷಿಕ್ತ-ನಾದ-ನೆಂದು
ರಾಜ್ಯಾಭಿಷೇಕ
ರಾಜ್ಯಾಭ್ಯುದ
ರಾಜ್ಯಾಭ್ಯುದ-ಯಾರ್ತ್ಥ
ರಾಜ್ಯಾಭ್ಯುದ-ಯಾರ್ಥ-ವಾಗಿ
ರಾಜ್ಯಾರೋ-ಹ-ಣಕ್ಕೆ
ರಾಜ್ಯಾ-ವಾಳುತ್ತಿದ್ದನು
ರಾಡಿಯುಂ
ರಾಣಾ-ಜಗ-ದೇವ-ರಾಯ
ರಾಣಾ-ಜಗ-ದೇವ-ರಾಯನ
ರಾಣಾ-ಜಗ-ದೇವ-ರಾಯನು
ರಾಣಾ-ಪೆದ್ದ
ರಾಣಾ-ವಂಶ-ದ-ವರು
ರಾಣಿ
ರಾಣಿ-ಮುಖಜ್ಯೋತಿ
ರಾಣಿ-ಮೊಖಜ್ಯೋತಿ
ರಾಣಿಯ
ರಾಣಿ-ಯರ
ರಾಣಿ-ಯ-ರನ್ನು
ರಾಣಿ-ಯ-ರಿದ್ದರು
ರಾಣಿ-ಯ-ರಿದ್ದ-ರೆಂಬು-ದನ್ನು
ರಾಣಿ-ಯರು
ರಾಣಿ-ಯರೂ
ರಾಣಿ-ಯಾಗಿದ್ದಳು
ರಾಣಿ-ಯಾದ
ರಾಣಿಯು
ರಾಣಿ-ವಾಸ
ರಾಣೀ-ವಾಸ
ರಾಣುವೆಗೆ
ರಾಣೋಜಿ-ರಾವ್
ರಾತ್ರಿ
ರಾತ್ರಿ-ಯೊಳ್
ರಾದ
ರಾದ್ಧಾಂತ
ರಾಧನ-ಪುರ
ರಾಧಾ
ರಾಧಾ-ಪಟೇಲ್
ರಾಧೇಯ
ರಾಧೇಯ-ಕುಲ
ರಾಧೇಯನ
ರಾಮ
ರಾಮಂಣ
ರಾಮ-ಅರ-ಸಿ-ಕೆರೆಯ
ರಾಮ-ಕ-ಗಾವುಂಡನ
ರಾಮ-ಕು-ಮಾರ
ರಾಮ-ಕೃಷ್ಣ
ರಾಮ-ಕೃಷ್ಣ-ಗುರು
ರಾಮ-ಕೃಷ್ಣ-ಗುರು-ವಿಗೇ
ರಾಮ-ಕೃಷ್ಣ-ಗುರು-ವಿನ
ರಾಮ-ಕೃಷ್ಣ-ದೇವರ
ರಾಮ-ಕೃಷ್ಣ-ದೇವ-ರಿ-ಗಾಗಿ
ರಾಮ-ಕೃಷ್ಣ-ದೇವರು
ರಾಮ-ಕೃಷ್ಣಪ್ರಭು
ರಾಮ-ಕೃಷ್ಣ-ಭಿರಾಧ್ಯ
ರಾಮ-ಗ-ಉಡ
ರಾಮ-ಗವುಡ
ರಾಮ-ಗವು-ಡ-ನ-ಹಳ್ಳಿ
ರಾಮ-ಚಂದ್ರ
ರಾಮ-ಚಂದ್ರ-ದೇವರ
ರಾಮ-ಚಂದ್ರ-ದೇವ-ರಿಗೆ
ರಾಮ-ಚಂದ್ರ-ದೇವರು
ರಾಮ-ಚಂದ್ರ-ದೇವಾ-ಲಯ-ವಿದು
ರಾಮ-ಚಂದ್ರನ
ರಾಮ-ಚಂದ್ರಸ್ವಾಮಿಯ
ರಾಮ-ಚಂದ್ರಸ್ವಾಮಿ-ಯನ್ನು
ರಾಮ-ಚಂದ್ರಸ್ವಾಮಿವ್ಯಾಸ-ರಾಯರ
ರಾಮ-ಚಂದ್ರ-ಹೆಬ್ಬಾ-ರುವನ
ರಾಮ-ಚಂದ್ರಾ-ಪುರ
ರಾಮ-ಜೀಯ-ನಿಗೆ
ರಾಮ-ಜೀಯನು
ರಾಮ-ಣಾರ್ಯಸ್ಯ
ರಾಮಣ್ಣ
ರಾಮಣ್ಣನ
ರಾಮಣ್ಣನು
ರಾಮ-ತಂಮ
ರಾಮ-ತಮ್ಮ
ರಾಮ-ದೇವ
ರಾಮ-ದೇವನ
ರಾಮ-ದೇವನು
ರಾಮ-ದೇವ-ಮಹಾ-ರಾಯ
ರಾಮ-ದೇವ-ಮಹಾ-ರಾಯನ
ರಾಮ-ದೇವರ
ರಾಮ-ದೇವ-ರ-ಕಟ್ಟೆ
ರಾಮ-ದೇವ-ರಾಯ
ರಾಮ-ದೇವ-ರಾಯನು
ರಾಮ-ದೇವ-ರಿಗೆ
ರಾಮ-ದೇವಾ-ಲ-ಯದ
ರಾಮ-ದೇವಾ-ಲ-ಯವು
ರಾಮನ
ರಾಮ-ನ-ಗರ
ರಾಮ-ನ-ಬೆಂಕೊಂಡ-ಗಂಡ
ರಾಮ-ನ-ರಾಮ-ರಾ-ಜಯ್ಯ
ರಾಮ-ನ-ಹಳ್ಳಿ
ರಾಮ-ನಾಥ
ರಾಮ-ನಾಥ-ದೇವರ
ರಾಮ-ನಾಥ-ದೇವ-ರಿಗೆ
ರಾಮ-ನಾಥ-ದೇವಾ-ಲಯ
ರಾಮ-ನಾಥನ
ರಾಮ-ನಾಥ-ನಿಗೆ
ರಾಮ-ನಾಥ-ನೊಡನೆ
ರಾಮ-ನಾಥ-ಪುರದ
ರಾಮನು
ರಾಮನೇ
ರಾಮನ್ರಿಪ
ರಾಮ-ಪರಿ-ವಾರ-ವಾಗಿದೆ
ರಾಮ-ಪುರ
ರಾಮ-ಪುರದ
ರಾಮ-ಪುರವೇ
ರಾಮಪ್ಪ
ರಾಮಪ್ಪ-ನಾಯ-ಕನು
ರಾಮ-ಬಾಯಂಮ್ಮ-ಪುರ-ವಾದ
ರಾಮ-ಬಾ-ಯಮ್ಮನ
ರಾಮ-ಬಾ-ಯಮ್ಮ-ಪುರ-ವೆಂಬ
ರಾಮ-ಭಟಯ್ಯ
ರಾಮ-ಭಟ್ಟ
ರಾಮ-ಭದ್ರಾ-ದೇವಿ
ರಾಮ-ಮಾತ್ಯ
ರಾಮ-ಯ-ರಾಯ
ರಾಮ-ಯೋಗೀಶ್ವರ
ರಾಮಯ್ಯ
ರಾಮಯ್ಯ-ದೇವರ
ರಾಮ-ರ-ಸನ
ರಾಮ-ರಸ-ರ-ಮಗ
ರಾಮ-ರಾಜ
ರಾಮ-ರಾಜ-ಅಯ್ಯ
ರಾಮ-ರಾಜ-ಅ-ಳಿಯ
ರಾಮ-ರಾಜ-ತಿರು-ಮಲ-ರಾಜಯ್ಯ-ನ-ವರ
ರಾಮ-ರಾಜನ
ರಾಮ-ರಾಜ-ನ-ಸೇನಾನಿ
ರಾಮ-ರಾಜ-ನಾಯ-ಕ-ನಿಗೆ
ರಾಮ-ರಾಜ-ನಿಗೆ
ರಾಮ-ರಾಜ-ಯರ್ಸರು
ರಾಮ-ರಾ-ಜಯ್ಯ
ರಾಮ-ರಾಜಯ್ಯ-ದೇವ
ರಾಮ-ರಾಜಯ್ಯನ
ರಾಮ-ರಾಜಯ್ಯ-ನನ್ನು
ರಾಮ-ರಾಜಯ್ಯ-ನ-ವರ
ರಾಮ-ರಾಜಯ್ಯ-ನ-ವರು
ರಾಮ-ರಾಜಯ್ಯ-ನಿಗೆ
ರಾಮ-ರಾಜಯ್ಯನು
ರಾಮ-ರಾಜಯ್ಯನೂ
ರಾಮ-ರಾಜಯ್ಯ-ನೆಂದು
ರಾಮ-ರಾಜಯ್ಯನೇ
ರಾಮ-ರಾಜ-ವೊಡೆ-ಯರು
ರಾಮ-ರಾಜು
ರಾಮ-ರಾಯ
ರಾಮ-ರಾಯನ
ರಾಮ-ರಾಯ-ನಾಯ-ಕ-ನಿಗೆ
ರಾಮ-ರಾಯನು
ರಾಮ-ರಾಯರೇ
ರಾಮ-ಲಕ್ಷ್ಮಣ
ರಾಮ-ಲಕ್ಷ್ಮ-ಣ-ದೇವರ
ರಾಮ-ಲಕ್ಷ್ಮ-ಣ-ರಂತಿದ್ದು
ರಾಮ-ಲಕ್ಷ್ಮ-ಣ-ರಂತೆ
ರಾಮ-ಲಕ್ಷ್ಮ-ಣ-ರಿದ್ದಂತೆ
ರಾಮ-ಲಿಂಗಣ್ಣ-ಗಳ
ರಾಮ-ಲಿಂಗ-ದೇವ-ರಿಗೆ
ರಾಮ-ಲಿಂಗ-ದೇವರು
ರಾಮ-ಲಿಂಗೇಶ್ವರ
ರಾಮ-ಲಿಂಗೇಶ್ವರನು
ರಾಮ-ಸ-ಮುದ್ರ-ಗ-ಳನ್ನು
ರಾಮ-ಸೆಟ್ಟಿಯು
ರಾಮಸ್ವಾಮಿ
ರಾಮಾ
ರಾಮಾ-ಜ-ಯಪ್ಪ
ರಾಮಾ-ಜಯ್ಯ
ರಾಮಾ-ಜು-ಜರೇ
ರಾಮಾನು
ರಾಮಾ-ನುಜ
ರಾಮಾ-ನು-ಜ-ಕೂಟ
ರಾಮಾ-ನು-ಜ-ಕೂಟಕೆ
ರಾಮಾ-ನು-ಜ-ಕೂಟಕ್ಕೆ
ರಾಮಾ-ನು-ಜ-ಕೂಟ-ಗಳಿ-ರುತ್ತಿದ್ದವು
ರಾಮಾ-ನು-ಜ-ಕೂಟದ
ರಾಮಾ-ನು-ಜ-ಕೂಟ-ದಲ್ಲಿ
ರಾಮಾ-ನು-ಜ-ಕೂಟ-ವನ್ನು
ರಾಮಾ-ನು-ಜ-ಕೂಟವು
ರಾಮಾ-ನು-ಜಗೆ
ರಾಮಾ-ನು-ಜ-ಚಾರ್ಯನು
ರಾಮಾ-ನು-ಜ-ಜೀಯ
ರಾಮಾ-ನು-ಜ-ಜೀಯನ
ರಾಮಾ-ನು-ಜ-ಜೀಯ-ನಿಂದ
ರಾಮಾ-ನು-ಜ-ಜೀಯನು
ರಾಮಾ-ನು-ಜ-ಜೀಯನೂ
ರಾಮಾ-ನು-ಜ-ಜೀಯನೇ
ರಾಮಾ-ನು-ಜ-ಜೀಯರ
ರಾಮಾ-ನು-ಜ-ಜೀಯ-ರಿಗೆ
ರಾಮಾ-ನು-ಜ-ಜೀಯರ್
ರಾಮಾ-ನು-ಜನ
ರಾಮಾ-ನು-ಜನು
ರಾಮಾ-ನು-ಜ-ಪುರಂ
ರಾಮಾ-ನು-ಜ-ಮಠ
ರಾಮಾ-ನು-ಜ-ಮಠದ
ರಾಮಾ-ನು-ಜಯ್ಯ
ರಾಮಾ-ನು-ಜಯ್ಯ-ಗಳ
ರಾಮಾ-ನು-ಜಯ್ಯ-ನಿಗೆ
ರಾಮಾ-ನು-ಜಯ್ಯನೂ
ರಾಮಾ-ನು-ಜರ
ರಾಮಾ-ನು-ಜ-ರನ್ನು
ರಾಮಾ-ನು-ಜರು
ರಾಮಾ-ನು-ಜ-ರೆಂದೇ
ರಾಮಾ-ನು-ಜರೇ
ರಾಮಾ-ನು-ಜಸ್ವಾಮಿ-ಗಳು
ರಾಮಾ-ನು-ಜಾಂಘ್ರಿ
ರಾಮಾ-ನು-ಜಾ-ಚಾರ್ಯರ
ರಾಮಾ-ನು-ಜಾ-ಚಾರ್ಯ-ರನ್ನು
ರಾಮಾ-ನು-ಜಾ-ಚಾರ್ಯ-ರ-ರನ್ನು
ರಾಮಾ-ನು-ಜಾ-ಚಾರ್ಯ-ರಿಂದ
ರಾಮಾ-ನು-ಜಾ-ಚಾರ್ಯ-ರಿ-ಗಿಂತಲೂ
ರಾಮಾ-ನು-ಜಾ-ಚಾರ್ಯ-ರಿಗೆ
ರಾಮಾ-ನು-ಜಾ-ಚಾರ್ಯರು
ರಾಮಾ-ನು-ಜಾ-ಚಾರ್ಯರೂ
ರಾಮಾ-ನು-ಜಾ-ಚಾರ್ಯರೇ
ರಾಮಾ-ನು-ಜಾ-ಚಾರ್ಯ್ಯರು
ರಾಮಾ-ನು-ಜಾಯ-ನಮಃ
ರಾಮಾ-ನು-ಜಾರ್ಯನ
ರಾಮಾ-ನು-ಜೀಯಂಗಾರ್
ರಾಮಾ-ನು-ಜೈಯಂಗಾ-ರಿಗೆ
ರಾಮಾ-ನುರ
ರಾಮಾ-ಭಟನ
ರಾಮಾ-ಭಟ-ನಿಗೆ
ರಾಮಾ-ಭಟ-ನೆಂದು
ರಾಮಾ-ಭಟಯ್ಯ-ನವ-ರಿಗೆ
ರಾಮಾ-ಭಟಯ್ಯ-ನಿಂದ
ರಾಮಾ-ಭಟಯ್ಯ-ನಿಗೆ
ರಾಮಾ-ಭಟಯ್ಯನು
ರಾಮಾ-ಭಟ್ಟ
ರಾಮಾ-ಭಟ್ಟನು
ರಾಮಾ-ಭಟ್ಟಯ್ಯನ
ರಾಮಾ-ಭಟ್ಟಯ್ಯನ-ವ-ರಿಂದ
ರಾಮಾ-ಭಟ್ಟಯ್ಯನ-ವ-ರಿಗೂ
ರಾಮಾ-ಭಟ್ಟಯ್ಯನಿಗೆ
ರಾಮಾ-ಭಟ್ಟಯ್ಯನು
ರಾಮಾ-ಭಟ್ಟರ
ರಾಮಾ-ಯಣ
ರಾಮಾ-ಯಣಂ
ರಾಮಾ-ಯಣದ
ರಾಮಾ-ಯಣ-ಪೂರ್ವ್ವಕ
ರಾಮೆಯ
ರಾಮೆಯನ
ರಾಮೆಯ-ನಾಯ-ಕನು
ರಾಮೆಯ-ನಾಯ್ಕ
ರಾಮೆಯನು
ರಾಮೆಶ್ವರ
ರಾಮೇಶ್ವರ
ರಾಮೇಶ್ವರ-ಗ-ಳಿಂದ
ರಾಮೇಶ್ವರದ
ರಾಮೇಶ್ವರ-ದಲ್ಲಿ
ರಾಮೇಶ್ವರ-ದ-ವರೆಗೂ
ರಾಮೇಶ್ವರ-ದ-ವರೆಗೆ
ರಾಮೇಶ್ವರಾನ್ವಯೋದ್ಭೂತ
ರಾಮೈಯಂಗಾರರ
ರಾಮೈಯ್ಯಂಗಾರರ
ರಾಮೋಜ
ರಾಮೋ-ಜನ
ರಾಮೋಜ-ನಿಗೆ
ರಾಮೋ-ಜನೇ
ರಾಯ
ರಾಯಂಣ
ರಾಯ-ಕು-ಮಾರರ
ರಾಯ-ಣ-ನಾಯ-ಕನು
ರಾಯಣ್ಣ
ರಾಯಣ್ಣ-ದಂಡ-ನಾಥನ
ರಾಯಣ್ಣ-ನಾಯ-ಕನ
ರಾಯ-ತನಯ
ರಾಯ-ತಮ್ಮ
ರಾಯನ
ರಾಯ-ನಿಂದ
ರಾಯ-ನಿಗೆ
ರಾಯನು
ರಾಯ-ಪ-ನಾಯ-ಕನು
ರಾಯ-ಪ-ನಾಯ-ಕರ
ರಾಯ-ಪುರಂ
ರಾಯಪ್ಪ
ರಾಯಪ್ಪ-ನಾಯ-ಕ-ನಿಗೆ
ರಾಯ-ಭಾಟ
ರಾಯರ
ರಾಯ-ರ-ಕು-ಮಾರರ
ರಾಯ-ರ-ಗಂಡ
ರಾಯ-ರಾಜ-ಗುರು
ರಾಯ-ರಾಯ
ರಾಯ-ರಿಗೂ
ರಾಯ-ರಿಗೆ
ರಾಯರು
ರಾಯ-ರೊಡನೆ
ರಾಯ-ರೊಳು
ರಾಯ-ರೊಳ್
ರಾಯ-ಲೆಂಕಪ್ಪೃ-ತಿಯ
ರಾಯ-ವೊಡೆಯ
ರಾಯ-ವೊಡೆ-ಯರು
ರಾಯಸ
ರಾಯ-ಸದ
ರಾಯ-ಸ-ದ-ವನು
ರಾಯ-ಸ-ದ-ವರ
ರಾಯ-ಸ-ದ-ವ-ರಾದ
ರಾಯ-ಸ-ದ-ವ-ರಿಗೆ
ರಾಯ-ಸ-ದ-ವರು
ರಾಯ-ಸ-ಮುದ್ರ
ರಾಯ-ಸ-ಮುದ್ರದ
ರಾಯ-ಸ-ಮುದ್ರ-ವಾದ
ರಾಯ-ಸವು
ರಾಯ-ಸಸ್ವಾಮಿ
ರಾಯ-ಸೆಟ್ಟಿ-ಪುರ
ರಾಯ-ಸೆಟ್ಟಿ-ಪುರ-ವನ್ನು
ರಾಯ-ಸೆಟ್ಟಿ-ಪುರ-ವಾದ
ರಾಯಸ್ತ-ವರ್ತನೆಗೆ
ರಾಯಸ್ಥ
ರಾಯಾಂಬಾಮುಲ
ರಾಯೊಡೆ-ಯರ
ರಾವಂದೂ-ರಿಗೆ
ರಾವಂದೂರಿನ
ರಾವಂದೂರು
ರಾವಂದೂರು-ಗಳು
ರಾವಿಯ-ಹಾಳೆಯ-ಮಲ್ಲಿ-ನಾಥ-ಪುರ
ರಾವುತ
ರಾವು-ತರ
ರಾವು-ತ-ರಾಯ-ನು-ದಗ್ರದೊರ್ವ್ವಳಂ
ರಾವು-ತರು
ರಾವುತ್ತ-ರಾಯ
ರಾವುತ್ತ-ರಾಯಂ
ರಾವುಳ
ರಾವ್ಬಹದ್ದೂರ್
ರಾಶಿ
ರಾಶಿ-ಯಾಗಿ
ರಾಶಿಯೇ
ರಾಷ್ಟರ-ಕೂಟರು
ರಾಷ್ಟ್ರ
ರಾಷ್ಟ್ರಕ
ರಾಷ್ಟ್ರ-ಕೂಟ
ರಾಷ್ಟ್ರ-ಕೂಟರ
ರಾಷ್ಟ್ರ-ಕೂಟ-ರನ್ನು
ರಾಷ್ಟ್ರ-ಕೂಟ-ರಿಂದ
ರಾಷ್ಟ್ರ-ಕೂಟ-ರಿಗೂ
ರಾಷ್ಟ್ರ-ಕೂಟರು
ರಾಷ್ಟ್ರ-ಕೂಟರೇ
ರಾಷ್ಟ್ರ-ಕೂಟ-ರೊಡನೆ
ರಾಷ್ಟ್ರದ
ರಾಷ್ಟ್ರ-ವನ್ನು
ರಾಷ್ಟ್ರ-ವೆಂದು
ರಾಸಕ್ಕಲಿನ
ರಾಸಿ-ಮಾ-ಡಲು
ರಿಂದ
ರಿಂದಲೂ
ರಿಂದಲೇ
ರಿಕಂನ-ರಾಜ-ನ-ಕಟ್ಟೆ
ರಿಕಂರಾಜ-ನ-ಕಟ್ಟೆ
ರಿಗೆ
ರಿಜಿಮೆಂಟ್ನಲ್ಲಿ
ರಿಜಿಸ್ಟರ್
ರಿಣಕ್ಕೆ
ರಿತ್ತಿ-ಯ-ವರು
ರಿಪು-ರಾಮ-ಗಾಮುಣ್ಡರು
ರಿಪುಸ್ತೋಮ
ರಿಪುಸ್ತೋಮ-ಕರಿ
ರಿಪೋರ್ಟ್ನಲ್ಲಿ
ರಿಪೋರ್ಟ್ರಲ್ಲಿ
ರಿಯಾಯಿತಿ
ರಿಷಿ-ಯರ
ರೀಡರ್
ರೀತಿ
ರೀತಿ-ಗ-ಳಲ್ಲಿ
ರೀತಿಯ
ರೀತಿ-ಯ-ದಿರ-ಬ-ಹುದು
ರೀತಿ-ಯದು
ರೀತಿ-ಯನ್ನು
ರೀತಿ-ಯಲ್ಲಿ
ರೀತಿ-ಯಲ್ಲಿತ್ತು
ರೀತಿ-ಯಲ್ಲಿದ್ದ
ರೀತಿ-ಯಲ್ಲಿದ್ದನು
ರೀತಿ-ಯಲ್ಲಿಯೇ
ರೀತಿ-ಯಲ್ಲೇ
ರೀತಿ-ಯಾಗಿ
ರೀತಿ-ಯಾದ
ರುಕಮವ್ವೆ
ರುಕಯ್ಯಾ
ರುಕುಮವ್ವೆ
ರುಕ್
ರುಕ್ಮಾಂಗದ
ರುಕ್ಮಿಣಿ-ಯಂತಿದ್ದಳು
ರುಕ್ಶಾಖಾಧ್ಯಾಯಿ-ಗ-ಳಾದ
ರುಖಯ್ಯಾ-ಬೀಬಿಯ
ರುಗಧೀತಶ್ಚ
ರುಗ್ವೇದ
ರುದ್ರಣ್ಣ
ರುದ್ರಣ್ಣ-ನೆಂಬುದು
ರುದ್ರ-ದಂಡಾಧೀಶ
ರುದ್ರ-ದೇವಾತ್ಮಜಂ
ರುದ್ರ-ಭಟ್ಟ
ರುದ್ರ-ಭೂಮಿ-ಯಲ್ಲಿವೆ
ರುದ್ರ-ಮುನಿ
ರುದ್ರ-ಮುನಿ-ದೇವಾ-ರಾಧ್ಯರು
ರುದ್ರ-ಮುನಿಸ್ವಾಮಿ-ಗಳು
ರುದ್ರರ
ರುದ್ರರು
ರುದ್ರ-ಸ-ಮುದ್ರ
ರುದ್ರಾಕ್ಷಿಗೆ
ರುಧಿರೋದ್ಗಾರಿ
ರೂಕ
ರೂಕವು
ರೂಡಿ
ರೂಡಿಯ
ರೂಡಿ-ವಡೆದ
ರೂಢ-ಗೊಂಡು
ರೂಢಿ
ರೂಢಿಗೆ
ರೂಢಿಯ
ರೂಢಿ-ಯಲ್ಲಿತ್ತು
ರೂಢಿ-ಯಲ್ಲಿದೆ
ರೂಢಿ-ಯಲ್ಲಿದ್ದೆ
ರೂಢಿ-ಸಿ-ಕೊಂಡು
ರೂಪ
ರೂಪಂ
ರೂಪ-ಕಂದರ್ಪ್ಪನುಂ
ರೂಪಕ್ಕೆ
ರೂಪ-ಗ-ಳಾಗಿ-ರ-ಬ-ಹುದು
ರೂಪ-ಗಳು
ರೂಪ-ಗಳೇ
ರೂಪದ
ರೂಪ-ದಂತಿ-ರುವ
ರೂಪ-ದಲ್ಲಿ
ರೂಪ-ದಲ್ಲಿತ್ತೆಂದು
ರೂಪ-ದಲ್ಲಿಯೇ
ರೂಪ-ನಾ-ರಾಯಣ
ರೂಪ-ವಾಗಿದೆ
ರೂಪ-ವಾದ
ರೂಪ-ವಿರ-ಬ-ಹುದು
ರೂಪವು
ರೂಪ-ವುಳ್ಳ-ವನೂ
ರೂಪವೇ
ರೂಪವೋ
ರೂಪ-ಶಿವ
ರೂಪಾಂತರ
ರೂಪಾಂತರ-ವಾಯಿತು
ರೂಪಾಂತರ-ವಾಯಿ-ತೆಂದು
ರೂಪಾಯಿ
ರೂಪಿ-ನೊಳು
ರೂಪಿಸಲಾಯಿ-ತೆಂದು
ರೂಪಿಸಿ-ಕೊಂಡು
ರೂಪಿ-ಸಿದ
ರೂಪಿ-ಸಿದನು
ರೂಪು
ರೂಪು-ಗೊಂಡ
ರೂಪು-ಗೊಳ್ಳು-ವು-ದಕ್ಕೆ
ರೂಪ್ಯಕ
ರೂವಾರಿ
ರೂವಾರಿ-ಗಳ
ರೂವಾರಿ-ಗ-ಳನ್ನೂ
ರೂವಾರಿ-ಗಳು
ರೂವಾರಿ-ಗಳೆಂದರೂ
ರೂವಾರಿ-ಗ-ಳೆಂದು
ರೂವಾರಿ-ಗ-ಳೆಂದೂ
ರೂವಾ-ರಿಯ
ರೂವಾರಿ-ಯಾಗಿರ
ರೂವಾರಿ-ಯಾಗಿ-ರ-ಬ-ಹುದು
ರೂವಾರಿ-ಸುತ್ತಾನೆ
ರೂಹ
ರೂಹಾರ
ರೂಹಾರವ
ರೂಹಾರಿ
ರೂಹಾರಿ-ಗಳು
ರೂಹು
ರೆಂಡಿ-ಷನ್
ರೆಂದು
ರೆಂಬುದು
ರೆಜಿಮೆಂಟ್
ರೆಡ್ಡಿ
ರೆಡ್ಡಿ-ಯ-ವರು
ರೆಸಿಡೆಂಟ್
ರೇಕವ್ವೆ
ರೇಕವ್ವೆಯ
ರೇಕವ್ವೆಯು
ರೇಕಾ-ದೇವಿ
ರೇಕಾ-ದೇವಿಯು
ರೇಖಾ
ರೇಖಾ-ರೇಖೆ
ರೇಖಾ-ವಿಳಾಸ
ರೇಖಾ-ವಿಳಾಸ-ಎಂಬ
ರೇಖೆ
ರೇಖೆ-ಗ-ಳನ್ನು
ರೇಖೆಗೆ
ರೇಚಣ್ಣ
ರೇಚೆಯ
ರೇಣುಕಾ-ಚಾರ್ಯ
ರೇಮಟಿ-ವೆಂಕಟ-ನನ್ನು
ರೇಮೇ
ರೇವಂತ
ರೇವಕ
ರೇವಕ-ನಿಮ್ಮ-ಡಿಗೆ
ರೇವಕ-ನಿಮ್ಮ-ಡಿ-ಯನ್ನು
ರೇವಕ್ಕ
ರೇವಕ್ಕ-ನಿರ್ಮಡಿ-ಯನ್ನು
ರೇವ-ಣಯ್ಯ
ರೇವಣಾರಾಧ್ಯ
ರೇವಣಾರಾಧ್ಯರ
ರೇವಣ್ಣ
ರೇವಲಾಕಲ್ಪ-ವಲ್ಲಿ
ರೇವಲಾ-ದೇವಿಯ
ರೇವಲಾ-ದೇವಿ-ಯನ್ನು
ರೇವಲೇಶ್ವರ
ರೈಟ್ಹ್ಯಾಂಡ್
ರೈತ-ರನ್ನು
ರೈತ-ರಿಗೆ
ರೈತರು
ರೈಲ್ವೆ
ರೈಸ್
ರೈಸ್ರ-ವರ
ರೈಸ್ರ-ವರು
ರೊಂದು
ರೊಕ
ರೊಕ-ದಲಿ
ರೊಕ್ಕ
ರೊಕ್ಕಾ-ದಾನ
ರೊಕ್ಕಾ-ದಾನ-ಹಣ
ರೊಖ
ರೋಣ-ಗಲ್ಲು
ರೋಣ-ಶಾ-ಸನವು
ರೋಮಾಂಚ-ಕಾರಿ-ಯಾಗಿ
ರೋಹಿಣೀ
ರ್ತ್ತುಂಗ-ಪರಾಕ್ರಮಂ
ಲಂಕಪ್ಪ
ಲಂಕೆಯ-ವರೆಗೆ
ಲಂಬ
ಲಂಭ-ಹಸ್ತ-ಗಳಅ
ಲಕು-ಮನ
ಲಕು-ಮಯ್ಯ
ಲಕು-ಮಯ್ಯ-ಗಳ
ಲಕು-ಮಯ್ಯನ
ಲಕು-ಮಯ್ಯನು
ಲಕು-ಮಯ್ಯರು
ಲಕು-ಮಾ-ದೇವಿಯ
ಲಕುಲೀಶ
ಲಕುಲೀಶ-ನಿಂದ
ಲಕುಲೀಶ-ಪಂಥ-ವನ್ನಷ್ಟೇ
ಲಕುಲೀಶ-ಪಾಶು-ಪತ
ಲಕುಳೀಶ
ಲಕುಳೇಶ್ವರ
ಲಕೋಜ
ಲಕ್ಕ-ಜೀಯನ
ಲಕ್ಕ-ಜೀಯನು
ಲಕ್ಕ-ಜೀಯ-ರಿಗೆ
ಲಕ್ಕ-ಜೀಯರು
ಲಕ್ಕಣ್ಣ
ಲಕ್ಕಣ್ಣ-ದಂಡ-ನಾಯ-ಕನು
ಲಕ್ಕಣ್ಣ-ದಂಡ-ನಾಯ-ಕರ
ಲಕ್ಕಣ್ಣ-ದಂಡೇಶ
ಲಕ್ಕಣ್ಣ-ದಂಡೇಶನ
ಲಕ್ಕಣ್ಣ-ದಂಡೇಶ-ನನ್ನು
ಲಕ್ಕಣ್ಣ-ದಂಡೇಶನು
ಲಕ್ಕಣ್ಣ-ನನ್ನು
ಲಕ್ಕಣ್ಣ-ನಾಯ-ಕರ
ಲಕ್ಕಪ್ಪ
ಲಕ್ಕಪ್ಪ-ನ-ವರ
ಲಕ್ಕಮ್ಮ
ಲಕ್ಕಮ್ಮನ
ಲಕ್ಕಯ್ಯ
ಲಕ್ಕಯ್ಯನು
ಲಕ್ಕವ್ವೆ
ಲಕ್ಕಿ-ಕಟ್ಟೆ
ಲಕ್ಕಿದೊಣೆ-ಜಿನ-ದೊ-ಣೆಯ
ಲಕ್ಕಿಯೂರ
ಲಕ್ಕಿಯೂರು
ಲಕ್ಕುಂಡಿಯ-ತನಕ
ಲಕ್ಕೂರು
ಲಕ್ವ್ಮೀ-ನಾ-ರಾಯಣ
ಲಕ್ಷ
ಲಕ್ಷಣ
ಲಕ್ಷಮ್ಮಮ್ಮಣಿಯು
ಲಕ್ಷಿತ
ಲಕ್ಷಿಸ-ಬೇಕು
ಲಕ್ಷುಮಣ-ದಾಸ-ರಿಗೆ-ಲಕ್ಷ್ಮ-ಣ-ದಾಸ
ಲಕ್ಷುಮಿ
ಲಕ್ಷೋಪ-ಲಕ್ಷ
ಲಕ್ಷ್ಮ
ಲಕ್ಷ್ಮಣ
ಲಕ್ಷ್ಮ-ಣ-ದಾಸ
ಲಕ್ಷ್ಮ-ಣಯ್ಯ-ನ-ವರ
ಲಕ್ಷ್ಮ-ಣಯ್ಯನು
ಲಕ್ಷ್ಮ-ಣಾಧ್ವರಿ
ಲಕ್ಷ್ಮ-ಣಾಧ್ವರಿಯ
ಲಕ್ಷ್ಮನ
ಲಕ್ಷ್ಮನೂ
ಲಕ್ಷ್ಮಾಂಬಿ-ಕೆಯು
ಲಕ್ಷ್ಮಾ-ದೇವಿ
ಲಕ್ಷ್ಮಾ-ದೇವಿ-ಯರ
ಲಕ್ಷ್ಮಿ
ಲಕ್ಷ್ಮಿ-ನಾ-ರಾಯ-ಣ-ದೇವಾ-ಲಯ
ಲಕ್ಷ್ಮಿ-ಯಂತಿದ್ದಳು
ಲಕ್ಷ್ಮಿ-ಯಿ-ರುವ-ಳೆಂಬು-ದಕ್ಕೆ
ಲಕ್ಷ್ಮಿಯು
ಲಕ್ಷ್ಮಿ-ವೋಲ್
ಲಕ್ಷ್ಮೀ
ಲಕ್ಷ್ಮೀ-ಕಾಂತ
ಲಕ್ಷ್ಮೀ-ಕಾಂತ-ದೇವರ
ಲಕ್ಷ್ಮೀ-ಕಾಂತ-ದೇವ-ರೆಂದು
ಲಕ್ಷ್ಮೀ-ಕಾಂತ-ದೇವಾ-ಲ-ಯದ
ಲಕ್ಷ್ಮೀ-ಕಾಂತನ
ಲಕ್ಷ್ಮೀ-ಕಾಂತ-ಲಕ್ಷ್ಮೀ-ನಾ-ರಾಯಣ
ಲಕ್ಷ್ಮೀ-ಕಾಂತಸ್ವಾಮಿ
ಲಕ್ಷ್ಮೀ-ಕಾಂತಸ್ವಾಮಿಯ
ಲಕ್ಷ್ಮೀ-ಗುಣ-ಗಣಾಲಂಕೃತ
ಲಕ್ಷ್ಮೀ-ಜ-ನಾರ್ದನ
ಲಕ್ಷ್ಮೀ-ದೇವರ
ಲಕ್ಷ್ಮೀ-ದೇವ-ರಿಗೆ
ಲಕ್ಷ್ಮೀ-ದೇವಿ
ಲಕ್ಷ್ಮೀ-ದೇ-ವಿಗೆ
ಲಕ್ಷ್ಮೀ-ದೇವಿಯ
ಲಕ್ಷ್ಮೀ-ದೇವಿ-ಯನ್ನು
ಲಕ್ಷ್ಮೀ-ದೇವಿ-ಯರ
ಲಕ್ಷ್ಮೀ-ದೇವಿ-ಯ-ರಿಗೆ
ಲಕ್ಷ್ಮೀ-ದೇವಿಯೇ
ಲಕ್ಷ್ಮೀ-ಧರ
ಲಕ್ಷ್ಮೀ-ನರ-ಸಿಂಹ
ಲಕ್ಷ್ಮೀ-ನರ-ಸಿಂಹ-ದೇವರ
ಲಕ್ಷ್ಮೀ-ನರ-ಸಿಂಹ-ದೇವ-ರಿಗೆ
ಲಕ್ಷ್ಮೀ-ನರ-ಸಿಂಹನ
ಲಕ್ಷ್ಮೀ-ನರ-ಸಿಂಹಸ್ವಾಮಿಗೆ
ಲಕ್ಷ್ಮೀ-ನರ-ಸಿಂಹಸ್ವಾಮಿಯ
ಲಕ್ಷ್ಮೀ-ನ-ರಾಯಣ
ಲಕ್ಷ್ಮೀ-ನಾಥ
ಲಕ್ಷ್ಮೀ-ನಾಥ-ನಿಗೆ
ಲಕ್ಷ್ಮೀ-ನಾಥನು
ಲಕ್ಷ್ಮೀ-ನಾಥ-ನೆಂಬು-ವ-ವನು
ಲಕ್ಷ್ಮೀ-ನಾರ-ಸಿಂಹ
ಲಕ್ಷ್ಮೀ-ನಾ-ರಾಯಣ
ಲಕ್ಷ್ಮೀ-ನಾ-ರಾಯ-ಣ-ದಂಡ-ನಾಯಕ
ಲಕ್ಷ್ಮೀ-ನಾ-ರಾಯ-ಣ-ದೇವರ
ಲಕ್ಷ್ಮೀ-ನಾ-ರಾಯ-ಣ-ದೇವ-ರಿಗೆ
ಲಕ್ಷ್ಮೀ-ನಾ-ರಾಯ-ಣ-ದೇವರು
ಲಕ್ಷ್ಮೀ-ನಾ-ರಾಯ-ಣ-ದೇವಾ-ಲ-ಯಕ್ಕೆ
ಲಕ್ಷ್ಮೀ-ನಾ-ರಾಯ-ಣನ
ಲಕ್ಷ್ಮೀ-ನಾ-ರಾಯ-ಣನು
ಲಕ್ಷ್ಮೀ-ನಾ-ರಾಯ-ಣ-ಪೆರು-ಮಾಳ್
ಲಕ್ಷ್ಮೀ-ನಾ-ರಾಯ-ಣ-ಮೂತಿ-ಜ-ನಾರ್ದನ
ಲಕ್ಷ್ಮೀ-ನಾ-ರಾಯ-ಣ-ರಾವ್
ಲಕ್ಷ್ಮೀ-ಪತಿ
ಲಕ್ಷ್ಮೀ-ಪತಿಯ
ಲಕ್ಷ್ಮೀ-ಪತಿ-ಸೆಟ್ಟಿಯ
ಲಕ್ಷ್ಮೀ-ಪತಿ-ಸೆಟ್ಟಿಯು
ಲಕ್ಷ್ಮೀ-ಪುರ
ಲಕ್ಷ್ಮೀ-ಭೂ-ವರಾಹ-ನಾಥ
ಲಕ್ಷ್ಮೀ-ಮತಿ
ಲಕ್ಷ್ಮೀ-ಮತಿ-ದಂಡ-ನಾಯ-ಕಿತ್ತಿ
ಲಕ್ಷ್ಮೀ-ಮತಿ-ಯಿಂದ
ಲಕ್ಷ್ಮೀ-ಮುದೇ
ಲಕ್ಷ್ಮೀ-ರತ್ರಾವಿರಾಸೀನ್ನಿಖಿಳ-ಜ-ನನತಾ
ಲಕ್ಷ್ಮೀ-ಲಲಾ-ಮನು
ಲಕ್ಷ್ಮೀ-ವರಾಹ-ನಾಥಸ್ವಾಮಿ
ಲಕ್ಷ್ಮೀ-ವಿಲಾಸದ
ಲಕ್ಷ್ಮೀಶ
ಲಕ್ಷ್ಮೀ-ಸಾ-ಗರ
ಲಕ್ಷ್ಮೀ-ಸಾ-ಗರದ
ಲಕ್ಷ್ಮೀ-ಸಾ-ಗರ-ವನ್ನು
ಲಕ್ಷ್ಮೀ-ಸಾನ್ನಿಧ್ಯ-ವನ್ನು
ಲಕ್ಷ್ಮೀ-ಸೇನ
ಲಖಂಣ
ಲಖಂಣನ
ಲಖಂಣನು
ಲಖಂಣ-ವೊಡೆ-ಯರ
ಲಖಣ್ಣ-ವೊಡೆಯ
ಲಖಪ-ನಾಯ-ಕರ
ಲಖೆಯ-ನಾಯ-ಕ-ಗಂಗಾ-ದೇವಿ
ಲಖೋಜ
ಲಖ್ಖೆಯ
ಲಗುಡಿ-ಗಳ
ಲಗ್ಗೆ
ಲಚಂಣ
ಲಚ್ಚಣ್ಣ
ಲಚ್ಚಣ್ಣನು
ಲಚ್ಚಿಯ-ನಾಯಕ
ಲನಸ್ವಾಮಿ-ಯ-ವರು
ಲಬ್ದ-ಬಲ-ಪರಾಕ್ರಮ
ಲಬ್ಧನೂ
ಲಬ್ಧಾನೇಕ
ಲಭತೇ
ಲಭಿಸಿ-ರು-ವು-ದಿಲ್ಲ
ಲಭ್ಯ-ವಾಗ-ಬೇ-ಕಾದರೆ
ಲಭ್ಯ-ವಾಗಿ
ಲಭ್ಯ-ವಾ-ಗಿದ್ದು
ಲಭ್ಯ-ವಾಗಿವೆ
ಲಭ್ಯ-ವಾಗುತ್ತಿತ್ತು
ಲಭ್ಯ-ವಾಗುವ
ಲಭ್ಯ-ವಾದ
ಲಯ-ಕಾಳ
ಲಯ-ವಾಗಿದೆ
ಲಲನಾ-ಸಂಘ
ಲಲನೆ
ಲಲನೆ-ಯೆನಿಸಿ
ಲಲಾಟ-ದಲ್ಲಿ-ರುವ
ಲಲಾಮ
ಲಲಾಮಂ
ಲಲಿತ
ಲಲ್ಲ
ಲವ-ಕುಶ-ರಂತೆ
ಲವ-ರಿಗೆ
ಲಸಂ
ಲಸದ್ದೋರ್ದಣ್ಡ-ದೊಳ್ಸಂತೋಷಂ
ಲಸದ್ವಂಶ್ಯಂಗೆ
ಲಾಕುಳ
ಲಾಕುಳ-ಪಾಶು-ಪತ-ಕಾಳಾ-ಮುಖ
ಲಾಕುಳ-ರೆಂದೂ
ಲಾಕುಳ-ಶೈವರು
ಲಾಕುಳಾಗಮ-ದಲ್ಲಿ
ಲಾಕುಳಾಗಮವೇ
ಲಾಕುಳೀಸ್ವರ
ಲಾಕ್ಷಣ
ಲಾಕ್ಷಣ-ಗವಿ
ಲಾಕ್ಷಾ-ಗೃಹೋಪಾ-ಯಮುಂ
ಲಾಗಾಯ್ತಿ-ನಿಂದಲೂ
ಲಾಗಿದೆ
ಲಾಞ್ಚನಾಞ್ಚಿತ
ಲಾಡರಿ
ಲಾಡ್
ಲಾನ್ಸ್ನಾಯಕ್
ಲಾಭ
ಲಾಭ-ಪಡೆ-ಯಲು
ಲಾಭ-ವಿರ-ಲಿಲ್ಲ
ಲಾಭಾ-ದಾಯದ
ಲಾಯದ
ಲಾಲನ-ಕೆರೆ
ಲಾಳ-ನ-ಕೆರೆ
ಲಾಳ-ನ-ಕೆರೆಗೆ
ಲಾಳ-ನ-ಕೆರೆಯ
ಲಾಳ-ನ-ಕೆರೆ-ಯ-ವನೋ
ಲಾಳ-ಲನ-ಕೆರೆ
ಲಾವಣಿ-ಗ-ಳನ್ನು
ಲಾವಣಿ-ಗಳು
ಲಾವಣ್ಯ-ಸಿಂಧುವುಂ
ಲಿಂಗ
ಲಿಂಗಕ್ಕೆ
ಲಿಂಗ-ಗ-ವುಡನು
ಲಿಂಗ-ಣಾರ್ಯ-ನೆಂಬು-ವ-ವನು
ಲಿಂಗಣ್ಣ
ಲಿಂಗಣ್ಣನು
ಲಿಂಗಣ್ಣಯ್ಯನ
ಲಿಂಗಣ್ಣೊಡೆ-ಯನು
ಲಿಂಗದ
ಲಿಂಗ-ದ-ಬೀ-ರರು
ಲಿಂಗ-ದ-ವರರ್ಕರ-ಲಿಂಗದ
ಲಿಂಗ-ದೇವ-ರಿಗೆ
ಲಿಂಗ-ದೇವರು
ಲಿಂಗ-ಪಯ್ಯ
ಲಿಂಗ-ಪಯ್ಯನ
ಲಿಂಗ-ಪಯ್ಯನು
ಲಿಂಗಪ್ಪ
ಲಿಂಗಪ್ಪ-ಗವುಡ
ಲಿಂಗಪ್ಪ-ನಾಯ-ಕನ
ಲಿಂಗಪ್ಪಯ್ಯ
ಲಿಂಗಪ್ರತಿಷ್ಠೆಯಂ
ಲಿಂಗಪ್ರತಿಷ್ಠೆ-ಯನ್ನು
ಲಿಂಗ-ಮುದ್ರೆ
ಲಿಂಗಮೆ
ಲಿಂಗ-ಮೆ-ನಿ-ಸುವೆ
ಲಿಂಗ-ಮೆ-ನೆನಗಾಳ್ದನಾಪ್ತಬಾಂಧವ
ಲಿಂಗಯ್ಯ
ಲಿಂಗಯ್ಯ-ದೇವ
ಲಿಂಗಯ್ಯ-ನಿಂದ
ಲಿಂಗ-ವಂತ
ಲಿಂಗ-ವಂತ-ರೆಂಬ
ಲಿಂಗ-ವನ್ನು
ಲಿಂಗ-ವಿದೆ
ಲಿಂಗ-ವಿದ್ದರೆ
ಲಿಂಗ-ವಿ-ರುವ
ಲಿಂಗಾಂಬ
ಲಿಂಗಾಂಬಾ
ಲಿಂಗಾ-ಚಾರಿಯ
ಲಿಂಗಾ-ಚಾರ್ರಿ
ಲಿಂಗಾಜ-ಮಾಂಬಾ-ದೇವಿ
ಲಿಂಗಾ-ಜಮ್ಮಣ್ಣಿ
ಲಿಂಗಾ-ಜಮ್ಮಣ್ಣಿ-ಯ-ವರು
ಲಿಂಗಾ-ಪುರ
ಲಿಂಗಾ-ಭಟ್ಟರ
ಲಿಂಗಾಯಿತ
ಲಿಂಗಾಯಿ-ತರು
ಲಿಖಿತ
ಲಿಖಿತಂ
ಲಿಖಿತಂತ್ವಿದಂ
ಲಿಖಿತ-ಮೂಲ-ಗ-ಳಿಂದ
ಲಿಖಿತಮ್
ಲಿಖಿತ-ವಾಗಿವೆ
ಲಿಖಿತಾಃ
ಲಿಪಿ
ಲಿಪಿ-ಗಳು
ಲಿಪಿಯ
ಲಿಪಿ-ಯನ್ನು
ಲಿಪಿ-ಯಲ್ಲಿ
ಲಿಪಿ-ಯಲ್ಲಿದೆ
ಲಿಪಿ-ಯಲ್ಲಿದ್ದು
ಲಿಪಿ-ಯಲ್ಲಿ-ರುವ
ಲಿಪಿ-ಯಲ್ಲಿವೆ
ಲಿಪಿಯು
ಲಿಪಿ-ಸಂಸ್ಕೃತ-ತಮಿಳು
ಲಿಲ್ಲ
ಲಿಲ್ಲ-ವೆಂದು
ಲಿಲ್ಲೆಲ್
ಲೀಲೆ
ಲೀಲೆಯಿಂ
ಲೀಲೆ-ಯಿಂದ
ಲು
ಲುಪ್ತ-ವಾಗಿ-ರ-ಬ-ಹುದು
ಲೂಟಿ-ಹೊಡೆ-ಯಲು
ಲೆಂಕ
ಲೆಂಕಕ್ರಮ-ವನೆ-ಸೆದು
ಲೆಂಕ-ತನ-ವನ್ನು
ಲೆಂಕ-ತಿ-ಯರು
ಲೆಂಕ-ನಿಸ್ಸಂಕ-ನಾದ
ಲೆಂಕ-ಮಹಾ-ದೇವ
ಲೆಂಕರ
ಲೆಂಕ-ರ-ಗಂಡ-ರುಂವೆನಿಸಿದ
ಲೆಂಕ-ರ-ಪಡೆ-ಯನ್ನು
ಲೆಂಕ-ರಲ್ಲದ
ಲೆಂಕ-ರಾ-ಗದೆ
ಲೆಂಕ-ರಾ-ಗಲು
ಲೆಂಕ-ರಾಗಿ
ಲೆಂಕ-ರಾಗಿದ್ದರು
ಲೆಂಕ-ರಾಗಿದ್ದ-ರೆಂದು
ಲೆಂಕ-ರಾಗಿದ್ದ-ರೆಂಬ
ಲೆಂಕ-ರಾಗಿ-ರುತ್ತಿದ್ದರು
ಲೆಂಕ-ರಿಗೆ
ಲೆಂಕ-ರಿದ್ದು
ಲೆಂಕರು
ಲೆಂಕ-ಲೆಂಕ-ವಾಳಿ
ಲೆಂಕ-ವಾಳಿ
ಲೆಂಕ-ವಾಳಿ-ಗಳು
ಲೆಂಕ-ವಾಳಿ-ಗಳೆನಿಸಿ
ಲೆಂಕ-ವಾಳಿ-ತನವೂ
ಲೆಂಕ-ವಾಳಿಯ
ಲೆಂಕ-ವಾಳಿಯಂ
ಲೆಂಕ-ವಾಳಿ-ಯನ್ನು
ಲೆಂಕ-ವಾಳಿ-ಯಲ್ಲಿ
ಲೆಂಕ-ವಾಳಿ-ಯಿಂದ
ಲೆಂಕ-ವಾಳಿಯು
ಲೆಂಕಿತಿ-ಯರ
ಲೆಂಕಿತಿ-ಯರು
ಲೆಕ್ಕ
ಲೆಕ್ಕದ
ಲೆಕ್ಕ-ದಲಿ
ಲೆಕ್ಕ-ದಲು
ಲೆಕ್ಕ-ದಲೂ
ಲೆಕ್ಕ-ದಲ್ಲಿ
ಲೆಕ್ಕದೆ
ಲೆಕ್ಕ-ಪತ್ರ
ಲೆಕ್ಕ-ಪತ್ರ-ಗಳ
ಲೆಕ್ಕ-ಪತ್ರ-ಗ-ಳನ್ನು
ಲೆಕ್ಕ-ಪತ್ರ-ಗ-ಳನ್ನೂ
ಲೆಕ್ಕ-ಪತ್ರದ
ಲೆಕ್ಕ-ಪತ್ರ-ದಲ್ಲಿ
ಲೆಕ್ಕ-ಬ-ರೆಯುವ
ಲೆಕ್ಕ-ವನ್ನು
ಲೆಕ್ಕ-ಹಾಕ-ಲಾಗಿದೆ
ಲೆಕ್ಕ-ಹಾಕಿ
ಲೆಕ್ಕ-ಹಾಕಿ-ದರೆ
ಲೆಕ್ಕ-ಹಾಕಿದ್ದಾರೆ
ಲೆಕ್ಕಾ-ಚಾರ
ಲೆಕ್ಕಾ-ಚಾರ-ದಲ್ಲೇ
ಲೆಕ್ಕಾ-ಚಾರ-ವನ್ನು
ಲೆಕ್ಕಾ-ಚಾರ-ಹಾಕಿ
ಲೆಕ್ಕಾಧಿ-ಕಾರಿ-ಗ-ಳೆಂದು
ಲೆಕ್ಕಿ-ಸದೆ
ಲೆಡೆ-ಯೊಳ್
ಲೆವಿ
ಲೇಖಃ
ಲೇಖಕ
ಲೇಖಕಃ
ಲೇಖ-ಕನ
ಲೇಖ-ಕನು
ಲೇಖ-ಕ-ನೆಂದು
ಲೇಖ-ಕರು
ಲೇಖ-ಗಳು
ಲೇಖ-ನ-ಗ-ಳನ್ನು
ಲೇಖ-ನ-ಗಳಲ್ಲಾ-ಗಲೀ
ಲೇಖ-ನ-ಗ-ಳಾಗಿವೆ
ಲೇಖ-ನ-ಗಳು
ಲೇಖ-ನ-ಗಳೆಲ್ಲ-ವನ್ನೂ
ಲೇಖ-ನ-ದಲ್ಲಿ
ಲೇಖ-ನ-ವನ್ನು
ಲೇಖ-ಲಿಖಿತ
ಲೇಖಾ-ವಳ-ಯ-ವ-ಳೆಯಿತ
ಲೇವಾ-ದೇವಿ
ಲೊಕಿಯ-ಮಲ-ನ-ಕೆರೆ-ಗಳ
ಲೊಕ್ಕಣ-ಕವಿ
ಲೊಕ್ಕಾನೆ
ಲೊಕ್ಕಿ-ಗುಂಡಿಯ
ಲೊಕ್ಕಿ-ಗುಂಡಿ-ಯನ್ನು
ಲೊಕ್ಕಿ-ಗುಂಡಿ-ಯಲ್ಲಿ
ಲೊಕ್ಕಿಯ-ಹಳ್ಳಿ
ಲೋಕ
ಲೋಕಕ್ಕೆ
ಲೋಕ-ತಿಲಕ
ಲೋಕ-ತಿಲಕ-ಜಿನ-ಭ-ವನಕ್ಕೆ
ಲೋಕ-ತಿಲಕ-ಭ-ವನ-ವೆಂಬ
ಲೋಕ-ತಿಲಕ-ವೆಂಬ
ಲೋಕದ
ಲೋಕ-ದೊಳ್
ಲೋಕ-ನ-ಹಳ್ಳಿ
ಲೋಕ-ನ-ಹಳ್ಳಿಯ
ಲೋಕ-ಪಾ-ವನಿ
ಲೋಕ-ಪಾ-ವನೆ
ಲೋಕ-ಪಾ-ವನೆಗೆ
ಲೋಕ-ಪಾ-ವನೆಯ
ಲೋಕ-ಪಾ-ವನೆ-ಯ-ಸಾ-ಗರ
ಲೋಕಪ್ರ-ಸಿದ್ಧ-ನಾ-ಗಿದ್ದ
ಲೋಕ-ವಿದ್ಯಾ-ಧರ
ಲೋಕ-ವಿದ್ಯಾ-ಧ-ರನು
ಲೋಕಾಂತ-ಸಿಮ್ನಿ
ಲೋಕಾಂಬಿಕೆ
ಲೋಕಾ-ಚಾರ್ಯ-ರರ
ಲೋಕಾನಾಂಚ
ಲೋಕೋದ್ಭ-ವರು
ಲೋಕೋಪ-ಕಾರ-ದಲ್ಲಿ
ಲೋಚೆರ್ಲ
ಲೋಪದೊಷ-ಗಳಿ-ರ-ಬ-ಹುದು
ಲೋಭಿ-ರಾಯ
ಲೋಹ
ಲೋಹ-ಕರ್ಮ್ಮ
ಲೋಹ-ಗ-ಳಲ್ಲಿ
ಲೋಹದ
ಲೋಹ-ದ-ಕಂಚು-ಕೆಲಸ
ಲೋಹದ್ರಮ್ಮ
ಲೋಹದ್ರಮ್ಮ-ಗ-ಳನ್ನು
ಲೋಹ-ಮೂರ್ತಿ-ಗಳೂ
ಲೋಹ-ಶಿಲ್ಪ-ಗಳೂ
ಲೋಹಾರ್ಯ
ಲೋಹಿತ
ಲೋಹಿತ-ಕುಲ-ಶೇಖರ
ಲೋಹಿತ-ಗೋತ್ರದ
ಲೋಹಿತಾನ್ವಯ
ಲೋಹಿತಾಶ್ವ
ಲೌಕಿಕ
ಲೌಕಿಕ-ವನ್ನೇ
ಲ್ಗಟ್ಟ
ಲ್ದಾತಂ
ಲ್ದೊರೆ-ವಿತ್ತೀ
ಲ್ದೋದಿಪ
ಳಂತೆ
ಳಂಬಿತ
ಳೆಂದುಂ
ಳ್ಕೂಡಿ-ದ-ನೆಂಬೀ
ಳ್ತಿರಿದುದ
ವ
ವಂಕಣ-ಪಲ್ಲಿ
ವಂಗರು
ವಂಣ್ಪನಿ
ವಂತೆ
ವಂದಕಂ
ವಂದಿಗೆ
ವಂದಿಸಲ್ಪಡುತ್ತಿದ್ದನು
ವಂದಿಸಲ್ಪಡುತ್ತಿದ್ದ-ನೆಂದು
ವಂದಿ-ಸುವೆ-ನೆಂದು
ವಂದು
ವಂದ್ಯ-ನೊಳ್ಪುವೆ
ವಂಶ
ವಂಶಕ್ಕೂ
ವಂಶಕ್ಕೆ
ವಂಶ-ಗಳು
ವಂಶ-ಜ-ನಾ-ಗಿದ್ದು
ವಂಶ-ಜ-ನಾಗಿ-ರುವ
ವಂಶ-ಜ-ನಿರ-ಬಹು-ದೆಂದು
ವಂಶ-ಜನೇ
ವಂಶ-ಜ-ರನ್ನು
ವಂಶ-ಜ-ರಾದ
ವಂಶ-ಜರು
ವಂಶ-ಜ-ರೆಂದು
ವಂಶ-ಜ-ರೊಳು
ವಂಶದ
ವಂಶ-ದಲ್ಲಿ
ವಂಶ-ದ-ವ-ನಾಗಿದ್ದ-ನೆಂದು
ವಂಶ-ದ-ವ-ನಾಗಿದ್ದಾನೆ
ವಂಶ-ದ-ವ-ನಾಗಿ-ರ-ಬ-ಹುದು
ವಂಶ-ದ-ವ-ನಾಗಿ-ರುವ
ವಂಶ-ದ-ವ-ನಿರ-ಬ-ಹುದು
ವಂಶ-ದ-ವ-ನಿರ-ಬಹು-ದೆಂದು
ವಂಶ-ದ-ವನು
ವಂಶ-ದ-ವ-ನೆಂದು
ವಂಶ-ದ-ವನೇ
ವಂಶ-ದ-ವರ
ವಂಶ-ದ-ವ-ರನ್ನು
ವಂಶ-ದವ-ರಲ್ಲಿ
ವಂಶ-ದ-ವರ-ವರು
ವಂಶ-ದ-ವ-ರಾಗಿ-ರ-ಬ-ಹುದು
ವಂಶ-ದ-ವ-ರಾಗಿ-ರ-ಬಹು-ದೆಂದು
ವಂಶ-ದ-ವ-ರಾದ
ವಂಶ-ದ-ವ-ರಿಂದಲೇ
ವಂಶ-ದ-ವ-ರಿಗೂ
ವಂಶ-ದ-ವ-ರಿಗೆ
ವಂಶ-ದ-ವರಿ-ರ-ಬ-ಹುದು
ವಂಶ-ದ-ವರು
ವಂಶ-ದ-ವ-ರೆಂದು
ವಂಶ-ದ-ವ-ರೆಂಬ
ವಂಶ-ದ-ವ-ರೆಲ್ಲರೂ
ವಂಶ-ದ-ವರೇ
ವಂಶ-ದೊಡನೆ
ವಂಶ-ಪಾರಂಪರ್ಯ
ವಂಶ-ಪಾರಂಪರ್ಯ-ದಿಂದ
ವಂಶ-ಪಾರಂಪರ್ಯ-ವಾಗಿ
ವಂಶ-ಪಾರಂಪರ್ಯ-ವಾ-ಗಿತ್ತು
ವಂಶ-ಪಾರಂಪರ್ಯ-ವಾಗಿತ್ತೆಂದು
ವಂಶ-ಪಾರಂಪರ್ಯ-ವಾಗಿತ್ತೇ
ವಂಶ-ಪಾರಂಪರ್ಯ-ವಾ-ಗಿದ್ದ
ವಂಶ-ಪಾರಂಪರ್ಯ-ವಾಗಿದ್ದಂತೆ
ವಂಶ-ಪಾರಂಪರ್ಯ-ವಾಗಿದ್ದರೂ
ವಂಶ-ಪಾರಂಪರ್ಯ-ವಾದ
ವಂಶ-ಮೌಕ್ತಿಕ
ವಂಶ-ವನ್ನು
ವಂಶವು
ವಂಶ-ವೃಕ್ಷ
ವಂಶ-ವೃಕ್ಷ-ಗಳ
ವಂಶ-ವೃಕ್ಷ-ದಲ್ಲಿ
ವಂಶ-ವೃಕ್ಷ-ವನ್ನು
ವಂಶ-ವೆಂದರೆ
ವಂಶ-ವೆಂದೂ
ವಂಶವೇ
ವಂಶ-ವೊಂದು
ವಂಶಸ್ಥ
ವಂಶಸ್ಥ-ನಾದ
ವಂಶಸ್ಥ-ನಿರ-ಬ-ಹುದು
ವಂಶಸ್ಥ-ನಿರ-ಬಹು-ದೆಂದು
ವಂಶಸ್ಥ-ನೆಂದು
ವಂಶಸ್ಥನೋ
ವಂಶಸ್ಥರ
ವಂಶಸ್ಥ-ರನ್ನು
ವಂಶಸ್ಥ-ರಾ-ಗಿದ್ದ
ವಂಶಸ್ಥ-ರಾದ
ವಂಶಸ್ಥ-ರಿರ-ಬ-ಹುದು
ವಂಶಸ್ಥರು
ವಂಶಸ್ಥರೇ
ವಂಶಸ್ಥರೋ
ವಂಶಾಂಗ-ನೆಯರು
ವಂಶಾಂಬುಜ
ವಂಶಾವ-ತಾರ-ವೆಂತೆಂದೊಡೆ
ವಂಶಾ-ವಳಿ
ವಂಶಾ-ವಳಿ-ಗ-ಳನ್ನು
ವಂಶಾ-ವಳಿಯ
ವಂಶಾ-ವಳಿ-ಯನ್ನು
ವಂಶಾ-ವಳಿ-ಯನ್ನೂ
ವಂಶಾ-ವಳಿ-ಯಲ್ಲಿ
ವಂಶಾ-ವಳಿ-ಯಲ್ಲಿಯೂ
ವಂಶಾ-ವಳಿ-ಯಿಂದ
ವಂಶಾ-ವಳಿಯು
ವಂಶೋದ್ಭವ
ವಂಶೋದ್ಭವರು
ವಇಜ-ನಾಥ
ವಕ್ತೃ
ವಕ್ತೃಪ್ರಯೋಕ್ತೃ
ವಕ್ತ್ರಾಬ್ಜ
ವಕ್ರಗ್ರೀವ
ವಕ್ರೋಕ್ತಿ-ಯಲ್ಲಿ
ವಕ್ಷಪ್ರ-ದೇಶಮಂ
ವಕ್ಷಸ್ಥಲ
ವಚಃ
ವಚಃಸುಂದರೀ
ವಚನ-ಕಾರ
ವಚನ-ಕಾರ-ರಾ-ಗಿದ್ದ
ವಚನ-ಕಾರರೂ
ವಚನ-ಗ-ಳನ್ನು
ವಚನ-ಗಳು
ವಚನ-ದಂತೆ
ವಚನ-ದಲ್ಲಿ
ವಚನ-ವನ್ನೂ
ವಚನ-ಶತ-ಸಹಸ್ರ
ವಚೋ
ವಜ್ಜಲ-ದೇವ
ವಜ್ರದ
ವಜ್ರದಂಡನುಂ
ವಜ್ರದ-ಪುಡಿ-ಯನ್ನು
ವಜ್ರ-ನಂದಿ
ವಜ್ರಪಂಜರ
ವಜ್ರಪಂಜರರುಂ
ವಜ್ರಪ್ರಾ-ಕಾರನೂ
ವಜ್ರಬೈ-ಸಣಿಗೆ-ಯ-ನಿಕ್ಕಿ
ವಜ್ರಮಕುಟಿ
ವಜ್ರಾಂಗಿ-ಯನ್ನು
ವಜ್ರಾ-ಸನ-ದಲ್ಲಿ
ವಟವಾಪಿ
ವಟ-ವೃಕ್ಷ
ವಟ್ಟ
ವಟ್ಟ-ಪರಿಖಾಯ
ವಡ-ಕರೈ-ನಾಡಿನ
ವಡಕ್ಕರೈ
ವಡ-ಗಲೈ-ಬಡಗ
ವಡ-ಗೆರೆ
ವಡ-ಗೆರೆ-ನಾಡು
ವಡ-ಮಯ್ಯ
ವಡುಗ-ಪಿಳ್ಳೆಯು
ವಡುಗ-ಪಿಳ್ಳೈ
ವಡುಗ-ವೇಳೆ-ಕಾರ
ವಡುಗಿ-ಯಣ್ಣನ
ವಡೆ-ಯರ
ವಡೆ-ಯರ-ಕಟ್ಟೆ
ವಡೆ-ಯರು
ವಡೆ-ಯರೈಯ್ಯ-ನ-ವರ
ವಡೆಯಾಂಡ
ವಡೇರ
ವಡೇರಯ್ಯ-ನ-ವರ
ವಡೈ-ಯರೈಯ-ನ-ವರು
ವಡ್ಡ-ಗಲ್ಲು
ವಡ್ಡಬ್ಯವ-ಹಾರಿ
ವಡ್ಡರ
ವಡ್ಡ-ರಾವು-ಳ-ವನ್ನು
ವಡ್ಡವ್ಯವ-ಹಾರಿ
ವಡ್ಡವ್ಯವ-ಹಾರಿ-ಗಳಲ್ಲೇ
ವಡ್ಡವ್ಯವ-ಹಾರಿ-ಗಳೂ
ವಡ್ಡವ್ಯವ-ಹಾರಿ-ಮಹಾ-ವಡ್ಡವ್ಯವ-ಹಾರಿ
ವಡ್ಡಾರಾ-ಧನೆ-ಯಲ್ಲಿ
ವಡ್ಡಿ
ವಡ್ಡಿ-ಯಾಲ್
ವಡ್ನೆನಗ
ವಡ್ರ-ಬಿಳಿ-ಕೆರೆ
ವಣ
ವಣಿ-ಕರು
ವಣಿಕೋಜ
ವಣ್ಣಾ-ಕರುಂ
ವಣ್ಣಾ-ಕರು-ವರ್ತಕರು
ವತ್ಸರಗ್ರಾಸ-ಸಂಪೂರ್ಣಾನ್
ವದನ-ವನ-ದೇವ-ತೆಯೆ
ವದಾನ್ಯತಃ
ವದ್ದೆಗ
ವದ್ದೆಗ-ನೆಂದು
ವದ್ಯಂ
ವಧಿಸಿ
ವಧೂ
ವಧೂಃ
ವನ
ವನ-ಜಾ-ತಾಯತ
ವನ-ಜಾ-ತಾಯ-ತ-ನೇತ್ರ-ಪುಂಣ್ಯಕ್ರುತ
ವನ-ಮಧ್ಯ-ದೇಶೇ
ವನ-ಮಾಲೆಗೆ
ವನ-ಲಕ್ಷ್ಮಿಯೆ
ವನ-ವಾಸಿ
ವನ-ವೇಲಿ
ವನಾಂತರ-ದಲ್ಲಿ
ವನಿತಾ-ತಿಳಕ
ವನಿತಾ-ದೂರಂ
ವನಿ-ತೆಯರೊ
ವನಿ-ವಾರ್ದ್ಧಿಸುಧಾ-ಕರ
ವನ್ನಾಗಿ
ವನ್ನಿ-ಯಾರ್
ವನ್ನು
ವನ್ನೈ-ದಿದಾಗ
ವನ್ಯಧಾಮ
ವಪಿ-ಸಿದ
ವಪಿಸ್ತ
ವಬಲ್ಲಾಳ
ವಯಲಿಲ್
ವಯ-ಲಿಲ್ಲ
ವಯಲು
ವಯಸ್ಸಿನ
ವಯಸ್ಸಿ-ನಲ್ಲಿ
ವಯಸ್ಸಿನ-ವ-ರಾಗಿ-ರುವ
ವಯಿಕುಂಠ-ವರ್ಧನ-ಕೃತಾ
ವಯಿಜ-ನಾಥ
ವಯಿಜಾಂಡ
ವಯೋವೃದ್ಧ-ನಾಗಿ-ರ-ಬ-ಹುದು
ವರಂತ-ರನ್
ವರಂತ-ರುಮ್
ವರ-ಕೀರ್ತ್ತಿಯಂ
ವರ-ಕೌಶಿಕ
ವರ-ಗಹ-ಳನ್ನು
ವರ-ಗುಡ್ಡ
ವರ-ತಲ್ಪಃ
ವರದ
ವರ-ದಣ್ಣ
ವರ-ದಣ್ಣ-ನಾಯ-ಕನು
ವರ-ದ-ಪಾ-ಚಾರ್ಯ
ವರ-ದಪ್ಪ
ವರ-ದಪ್ಪನ
ವರ-ದಪ್ಪಾ-ಚಾರ್ಯ
ವರ-ದಯ್ಯ
ವರ-ದ-ರಾಜ
ವರ-ದ-ರಾಜ-ದೇವರ
ವರ-ದ-ರಾಜ-ನ-ಕೆರೆ-ಯನ್ನು
ವರ-ದ-ರಾಜ-ಪುರ-ವಾದ
ವರ-ದ-ರಾಜ-ಪುರ-ವೆಂಬ
ವರ-ದ-ರಾಜ-ಪೆರು-ಮಾಳ್
ವರ-ದ-ರಾ-ಜಯ್ಯ
ವರ-ದ-ರಾಜಯ್ಯ-ನವ-ರಿಗೆ
ವರ-ದ-ರಾಜಯ್ಯ-ನಿಗೆ
ವರ-ದ-ರಾಜಯ್ಯ-ನೆಂದು
ವರ-ದ-ರಾಜಯ್ಯ-ನೆಂಬ
ವರ-ದ-ರಾಜ-ಸ-ಮುದ್ರ-ವೆಂಬ
ವರ-ದ-ರಾಜಸ್ವಾಮಿ
ವರ-ದ-ರಾಜಸ್ವಾಮಿಗೆ
ವರ-ದ-ರಾಜಸ್ವಾಮಿ-ಯ-ವರ
ವರ-ದಾ-ಚಾರ್ಯ
ವರ-ದಾ-ಚಾರ್ಯನ
ವರ-ದಾ-ಚಾರ್ಯ್ಯ-ವರ್ಯ್ಯಾಸ್ಯ
ವರ-ದಾಪಾಚರ್ಯ
ವರದಿ
ವರ-ದಿ-ಗಳ
ವರ-ದಿ-ಗಳಲ್ಲಿಯೂ
ವರ-ದಿಯ
ವರ-ದಿ-ಯಂತೆ
ವರ-ದಿ-ಯಲ್ಲಿ
ವರ-ದೆಯ
ವರ-ದೆಯ-ನಾಯಕ
ವರ-ದೆಯ-ನಾಯ-ಕನು
ವರ-ದೆಯ-ನಾಯ-ಕನೂ
ವರ-ದೆಯ-ನಾಯುಕ-ನೆಂದೂ
ವರನ
ವರ-ನೊಳು
ವರನ್ತರ-ಪೆರು-ಮಾನ್
ವರಪ್ರಸಾದ
ವರಪ್ರಸಾದ-ನೆಂದು
ವರ-ಭುಜ
ವರ-ಮಂತ್ರ-ಶಕ್ತಿ-ಯುತ-ನಿಂದ್ರಗೆಂತು
ವರ-ಮಂತ್ರಿ-ವಲ್ಲಭ
ವರ-ಮಾನ
ವರಹ
ವರ-ಹ-ಅ-ವನ್ನು
ವರ-ಹಕ್ಕೆ
ವರ-ಹ-ಗದ್ಯಾಣ
ವರ-ಹ-ಗದ್ಯಾಣ-ಗ-ಳನ್ನು
ವರ-ಹ-ಗಳ
ವರ-ಹ-ಗ-ಳನ್ನು
ವರ-ಹ-ಗ-ಳಿಗೆ
ವರ-ಹ-ಗುಳಿಗೆ
ವರ-ಹಗೆ
ವರ-ಹದ
ವರ-ಹ-ದಂತೆ
ವರ-ಹ-ನಲ್ಲಿ-ನಿಲಿಸಿ
ವರ-ಹನು
ವರ-ಹನೂ
ವರ-ಹ-ವನು
ವರ-ಹ-ವನ್ನು
ವರ-ಹ-ವೆಂದು
ವರ-ಹಸ್ವಾಮಿಯ
ವರ-ಹಾ-ನಾಥ
ವರ-ಹಾ-ನಾಥ-ಕಲ್ಲ-ಹಳ್ಳಿಗೆ
ವರ-ಹಾ-ವನ್ನು
ವರ-ಹೀ-ಳನ-ಹಳ್ಳಿ
ವರಾಪಸ್ತಂಭ-ಸೂತ್ರಾಯ
ವರಾಹ
ವರಾಹ-ದೀಕ್ಷಿತ
ವರಾಹ-ನ-ಕಲ್ಲ-ಹಳ್ಳಿ
ವರಾಹ-ನಾಥ
ವರಾಹ-ನಾಥ-ಕಲ್ಲ-ಹಳ್ಳಿ
ವರಾಹ-ನಾಥ-ಕಲ್ಲ-ಹಳ್ಳಿಯ
ವರಾಹ-ನಾಥನ
ವರಾಹ-ನಾಥಸ್ವಾಮಿ
ವರಾಹ-ನಾ-ರಾಯಣ
ವರಾಹ-ಮುದ್ರೆಯ
ವರಾಹ-ಮೂರ್ತಿಯ
ವರಾಹ-ಮೂರ್ತಿ-ಯನ್ನು
ವರಾಹ-ಮೂರ್ತಿಯು
ವರಾಹಸ್ತುತಿ
ವರಾಹಸ್ತುತಿ-ಯಿಂದಲೇ
ವರಿಷ
ವರಿ-ಷ-ದನ್ದಿಗೆ
ವರಿ-ಷ-ಮೇೞಳವಿ
ವರಿಷ್ಠ
ವರಿಸ
ವರಿ-ಸ-ದನ್ದು
ವರಿ-ಸ-ನಿ-ಬದ್ಧ-ವಾಗಿ
ವರಿ-ಸ-ನಿ-ಬನ್ದ-ವಾಗಿ
ವರಿ-ಸ-ನಿ-ಬನ್ಧ-ವಾಗಿ
ವರಿಸಿ
ವರಿ-ಸಿದ್ದ
ವರಿ-ಸಿದ್ದನು
ವರಿ-ಸಿದ್ದು
ವರುಷ
ವರು-ಷಕ್ಕೆ
ವರು-ಷ-ವೊಂದಕ್ಕೆ
ವರುಸ
ವರೆ-ಗಿದ್ದು
ವರೆ-ಗಿನ
ವರೆಗೂ
ವರೆಗೆ
ವರ್ಗ
ವರ್ಗಕ್ಕೆ
ವರ್ಗದ
ವರ್ಗ-ದ-ವ-ನಾಗಿದ್ದನು
ವರ್ಗ-ದ-ವ-ರಿಗೆ
ವರ್ಗ-ದ-ವರು
ವರ್ಗ-ದ-ವರೂ
ವರ್ಗ-ದ-ವ-ರೆಂದು
ವರ್ಗ-ವನ್ನಾಗಿ
ವರ್ಗ-ವಾಗಿ
ವರ್ಗ-ವೆಂಬ
ವರ್ಗಾಯಿ-ಸ-ಲಾ-ಯಿತು
ವರ್ಗಾಯಿಸಿ
ವರ್ಗಾಯಿಸಿ-ದ-ನೆಂದು
ವರ್ಗಾಯಿಸಿ-ರ-ಬಹು-ದೆಂದು
ವರ್ಗೀ-ಕರಣ
ವರ್ಚಸ್ಸನ್ನು
ವರ್ಣದ
ವರ್ಣದ-ವ-ರಾದ
ವರ್ಣನಾ
ವರ್ಣನೆ
ವರ್ಣನೆಈ
ವರ್ಣನೆ-ಗ-ಳನ್ನು
ವರ್ಣನೆ-ಗಳು
ವರ್ಣ-ನೆಯ
ವರ್ಣನೆ-ಯನ್ನು
ವರ್ಣನೆ-ಯನ್ನೂ
ವರ್ಣನೆ-ಯಲ್ಲಿ
ವರ್ಣನೆ-ಯಾಗಲೀ
ವರ್ಣನೆ-ಯಾಗಿದೆ
ವರ್ಣನೆ-ಯಿಂದ
ವರ್ಣನೆ-ಯಿಂದಲೇ
ವರ್ಣನೆ-ಯಿದೆ
ವರ್ಣನೆ-ಯಿದ್ದು
ವರ್ಣ-ನೆಯು
ವರ್ಣ-ನೆಯೇ
ವರ್ಣನೆ-ಯೊಂದಿಗೆ
ವರ್ಣಿತ-ವಾಗಿದೆ
ವರ್ಣಿತ-ವಾಗಿವೆ
ವರ್ಣಿಸ-ಲಾಗಿದೆ
ವರ್ಣಿಸ-ಲಾಗಿ-ದೆಯೇ
ವರ್ಣಿಸಲ್ಪಟ್ಟ
ವರ್ಣಿಸಿ
ವರ್ಣಿ-ಸಿದೆ
ವರ್ಣಿಸಿದ್ದಾನೆ
ವರ್ಣಿ-ಸಿದ್ದು
ವರ್ಣಿಸಿ-ರುವು-ದ-ರಿಂದ
ವರ್ಣಿಸಿ-ರುವುದು
ವರ್ಣಿ-ಸಿವೆ
ವರ್ಣಿ-ಸುತ್ತದೆ
ವರ್ಣಿ-ಸುತ್ತವೆ
ವರ್ಣಿ-ಸುವ
ವರ್ಣಿ-ಸುವಾಗ
ವರ್ತಕ
ವರ್ತಕನ
ವರ್ತಕ-ನಾಗಿ-ದರೂ
ವರ್ತಕ-ನಾದ-ವ-ನಿಗೆ
ವರ್ತಕನು
ವರ್ತಕಪ್ರ-ಕಾರ-ದ-ವರು
ವರ್ತಕರ
ವರ್ತಕ-ರನ್ನು
ವರ್ತಕ-ರನ್ನೂ
ವರ್ತಕ-ರಲ್ಲ
ವರ್ತಕ-ರಲ್ಲಿ
ವರ್ತಕ-ರಾ-ಗಿದ್ದಾರೆ
ವರ್ತಕ-ರಿಗೂ
ವರ್ತಕ-ರಿಗೆ
ವರ್ತಕ-ರಿಬ್ಬರೂ
ವರ್ತಕರು
ವರ್ತಕ-ರೆಂದು
ವರ್ತಕರೇ
ವರ್ತಕ-ವೃತ್ತಿ-ಯನ್ನು
ವರ್ತಕ-ಸಂಘದ
ವರ್ತಕ-ಸ-ಮು-ದಾಯ-ದ-ವ-ರಾಗಿ
ವರ್ತ-ಗರ
ವರ್ತನೆ
ವರ್ತನೆಗೆ
ವರ್ತ-ಮಾನ
ವರ್ತಿ-ಸು-ವ-ವನು
ವರ್ತ್ತನೆ
ವರ್ತ್ತ-ಮಾನ-ರಾಯ
ವರ್ದಿ-ಬೇಗ್
ವರ್ಧ-ಮಾನ
ವರ್ಧ-ಮಾನಸ್ವಾಮಿ
ವರ್ಧ-ಮಾನಾಪ-ದಾನಃ
ವರ್ಧಿ-ಸಿತ್ತು
ವರ್ಮ-ನನ್ನು
ವರ್ಷ
ವರ್ಷಂಪ್ರತಿ
ವರ್ಷಕ್ಕಾಗು-ವಷ್ಟು
ವರ್ಷಕ್ಕಿಂತ
ವರ್ಷಕ್ಕೂ
ವರ್ಷಕ್ಕೆ
ವರ್ಷಕ್ಕೊಮ್ಮೆ
ವರ್ಷ-ಗಳ
ವರ್ಷ-ಗ-ಳಲ್ಲಿ
ವರ್ಷ-ಗ-ಳಿಂದ
ವರ್ಷ-ಗಳು
ವರ್ಷದ
ವರ್ಷ-ದಂತೆ
ವರ್ಷ-ದಂದು
ವರ್ಷ-ದಲ್ಲಿ
ವರ್ಷ-ದಲ್ಲಿಯೇ
ವರ್ಷದ್ದು
ವರ್ಷ-ನಿ-ಬಂಧಿ-ಯಾಗಿ
ವರ್ಷ-ವನ್ನು
ವರ್ಷ-ವರ್ಷವೂ
ವರ್ಷ-ವಾಗಿ-ಬಿಡುತ್ತದೆ
ವರ್ಷವೂ
ವರ್ಷ-ವೆಂದು
ವರ್ಷವೇ
ವರ್ಷೆ
ವರ್ಷೇ
ವಲಯ-ಗಳು
ವಲಸೆ
ವಲ-ಸೆಯ
ವಲಸೆ-ಹೋಗ-ದಂತೆ
ವಲ್ಲಭಂ
ವಲ್ಲ-ಭ-ನಾಗಿ
ವಲ್ಲ-ಭನು
ವಲ್ಲ-ಭಾ-ಚಾರ್ಯರು
ವಲ್ಲಭೆ
ವಲ್ಲಾಳ-ದಾಸರು
ವಳ-ಗೆರೆ
ವಳ-ಗೆರೆ-ಹಳ್ಳಿ
ವಳ-ನಾಡಿ-ನಲ್ಲಿದ್ದ
ವಳ-ಬಾ-ಗಿಲ
ವಳ-ಭೀ-ಪುರ-ವ-ರೇಶ್ವರ
ವಳ-ಭೀ-ಪುರೇಶ್ವರ
ವಳಾಯ
ವಳಿತ
ವಳಿ-ತಕ್ಕೆ
ವಳಿ-ತ-ಗ-ಳೆಂದು
ವಳಿ-ತದ
ವಳೈ-ಅಣ್ಣನ್
ವಳೈ-ಕುಳ-ಮಾನ
ವಳೈ-ಕುಳ-ವಾದ
ವವನು
ವವರೂ
ವಶಕ್ಕೆ
ವಶಕ್ಕೇ
ವಶ-ದಲ್ಲಿ
ವಶ-ದಲ್ಲಿತ್ತು
ವಶ-ಪಡಿ-ಸ-ಕೊಂಡ
ವಶ-ಪಡಿ-ಸಿ-ಕೊಂಡ
ವಶ-ಪಡಿ-ಸಿ-ಕೊಂಡದ್ದು
ವಶ-ಪಡಿ-ಸಿ-ಕೊಂಡನು
ವಶ-ಪಡಿ-ಸಿ-ಕೊಂಡ-ನೆಂದು
ವಶ-ಪಡಿ-ಸಿ-ಕೊಂಡರು
ವಶ-ಪಡಿ-ಸಿ-ಕೊಂಡಿದ್ದನು
ವಶ-ಪಡಿ-ಸಿ-ಕೊಂಡಿದ್ದ-ನೆಂದು
ವಶ-ಪಡಿ-ಸಿ-ಕೊಂಡಿದ್ದ-ನೆಂಬುದು
ವಶ-ಪಡಿ-ಸಿ-ಕೊಂಡಿದ್ದ-ರಿಂದ
ವಶ-ಪಡಿ-ಸಿ-ಕೊಂಡಿದ್ದು
ವಶ-ಪಡಿ-ಸಿ-ಕೊಂಡಿರ-ಬ-ಹುದು
ವಶ-ಪಡಿ-ಸಿ-ಕೊಂಡು
ವಶ-ಪಡಿ-ಸಿ-ಕೊಳ್ಳಲು
ವಶ-ಪಡಿ-ಸಿ-ಕೊಳ್ಳು-ವುದ-ರಲ್ಲಿ
ವಶ-ವಾಗಿ-ರ-ಬ-ಹುದು
ವಶ-ವಾಗಿ-ರ-ಲಿಲ್ಲ
ವಶ-ವಾದವು
ವಸಂತ
ವಸಂತ-ಗೋ-ಪಾಲ
ವಸಂತ-ಪುರ
ವಸಂತ-ಪುರ-ವೆಂಬ
ವಸಂತ-ಮಾಸ-ದಲ್ಲಿ
ವಸಂತ-ರಾಯನು
ವಸಂತ-ಲಕ್ಷ್ಮಿ
ವಸಂತ-ಲಕ್ಷ್ಮಿ-ಯ-ವರ
ವಸಂತೋತ್ಸವ
ವಸಂತೋತ್ಸವದ
ವಸಕ್ಕೆ
ವಸತಿ
ವಸತಿ-ಗಾಗಿ
ವಸಿಷ್ಠ-ಗೋತ್ರೋದ್ಭ-ವನೂ
ವಸುಂಧರಾ
ವಸುಂಧರಾ-ಫಿಲಿ-ಯೋಜಾ
ವಸುಂಧರಾಭ್ಯಾಮಾಕಲ್ಪಂ
ವಸುಂಧರೆ-ಯನ್ನು
ವಸು-ಮತಿ-ಯೊಳು
ವಸೂಲಿ
ವಸೂಲು
ವಸೂಲು-ಮಾಡಿ
ವಸ್ತು
ವಸ್ತು-ಗಳ
ವಸ್ತು-ಗ-ಳಾಗಿವೆ
ವಸ್ತು-ಗಳಿ-ಗಾಗಿ
ವಸ್ತುವಂ
ವಸ್ತು-ವನು
ವಸ್ತು-ವನ್ನಾ-ಗುಳ್ಳ
ವಸ್ತು-ವನ್ನು
ವಸ್ತು-ವಾ-ಹನ
ವಸ್ತು-ವಿನ-ಧಾನ್ಯದ
ವಸ್ತು-ವಿನಿ-ಮಯದ
ವಸ್ತು-ವಿಸ್ತಾ-ರನುಂ
ವಸ್ತು-ಸಂಗ್ರಹಾ-ಲಯ-ದಲ್ಲಿದೆ
ವಸ್ತ್ರ
ವಸ್ತ್ರಕ್ಕೆ
ವಸ್ತ್ರ-ಗಳ
ವಸ್ತ್ರ-ಭಂಡಾರ
ವಸ್ತ್ರ-ವನ್ನು
ವಸ್ತ್ರವೂ
ವಸ್ತ್ರಾ-ಭಂಡಾರಕ್ಕೆ
ವಸ್ಮತಿ-ಯೊಳು
ವಹನ್
ವಹಿವಾಟು-ಗಳ
ವಹಿಸ-ಕೊಟ್ಟ-ರೆಂದು
ವಹಿ-ಸ-ಲಾ-ಯಿತು
ವಹಿಸಿ
ವಹಿಸಿ-ಕೊಂಡ
ವಹಿಸಿ-ಕೊಂಡನು
ವಹಿಸಿ-ಕೊಂಡ-ಮೇಲೆ
ವಹಿಸಿ-ಕೊಂಡಿದ್ದ-ನೆಂದು
ವಹಿಸಿ-ಕೊಂಡಿರ-ಬ-ಹುದು
ವಹಿಸಿ-ಕೊಂಡಿರ-ಬಹು-ದೆಂದು
ವಹಿಸಿ-ಕೊಂಡು
ವಹಿಸಿ-ಕೊಟ್ಟಿದ್ದ
ವಹಿಸಿದ
ವಹಿಸಿದ್ದ
ವಹಿಸಿದ್ದನು
ವಹಿಸಿದ್ದ-ನೆಂದು
ವಹಿಸಿದ್ದಾರೆ
ವಹಿಸಿ-ರ-ಬಹು-ದೆಂದು
ವಹಿಸಿ-ರು-ವಂತೆ
ವಹಿಸಿ-ರುವುದು
ವಹಿ-ಸುತ್ತಾಳೆ
ವಹಿಸುತ್ತಿದ್ದ
ವಹಿಸುತ್ತಿದ್ದರು
ವಹಿಸುತ್ತಿದ್ದ-ರೆಂದು
ವಹಿ-ಸುವಂತಾಗ-ಬೇಕು
ವಾ
ವಾಂಡಿವಾಷ್ನಲ್ಲಿ
ವಾಂತಿಭ್ರಾಂತಿ
ವಾಕ್ಯ-ಗಳನ್ನೊಳ-ಗೊಂಡ
ವಾಕ್ಯ-ಗಳು
ವಾಕ್ಯ-ವಿದೆ
ವಾಕ್ಯವು
ವಾಕ್ಯವೂ
ವಾಗಿ
ವಾಗಿತ್ತು
ವಾಗಿತ್ತೆಂದು
ವಾಗಿದೆ
ವಾಗಿದ್ದಂತೆ
ವಾಗಿದ್ದರೂ
ವಾಗಿದ್ದವು
ವಾಗಿಯೂ
ವಾಗಿವೆ
ವಾಗೀಶ್ವರ
ವಾಗೀಶ್ವರ-ದೇವಾ-ಲಯ
ವಾಗೀಶ್ವರ-ಮಂಗಲ
ವಾಗೀಶ್ವರ-ಮಂಗಲದ
ವಾಗೀಶ್ವರ-ಮಂಗಲ-ವೆಂಬ
ವಾಗುತ್ತದೆ
ವಾಚಕ
ವಾಚಕ-ಗ-ಳಿಂದ
ವಾಚಕ-ದಿಂದ
ವಾಚನ-ಮಾಡಿಸಿ
ವಾಚನ-ವನ್ನು
ವಾಚ್ಯಾರ್ಥ-ವಾಗಿ
ವಾಜರ
ವಾಜ-ಸನೇಯ
ವಾಜಿ
ವಾಜಿ-ಕುಲ
ವಾಜಿ-ಕುಲಕ್ಕೆ
ವಾಜಿ-ಕುಲ-ತಿಲಕ-ನಾ-ಗಿದ್ದು
ವಾಜಿ-ಕುಲ-ತಿಲಕ-ನಾದ
ವಾಜಿ-ಕುಲದ
ವಾಜಿ-ಕುಲವು
ವಾಜಿ-ಕುಳ-ತಿಳಕ
ವಾಜಿ-ಮಂಗಲ-ವೆಂಬ
ವಾಜಿ-ವಂಶ-ದ-ವರೇ
ವಾಜಿ-ವಂಶೋತ್ತಮ
ವಾಡಕ್ಕೆ-ಘಟ್ಟ-ಹೊಡಾ-ಘಟ್ಟ
ವಾಡಿ
ವಾಡಿಕೆ
ವಾಡಿ-ಪಾಡಿ
ವಾಡುಕ್ಕೆ-ಘಟ್ಟ
ವಾಣ-ಸತ್ತಿ
ವಾಣಿಜ್ಯ
ವಾಣಿಜ್ಯ-ದಲ್ಲಿ
ವಾತಾಪಿ-ಯನ್ನು
ವಾತಾ-ವರ-ಣವು
ವಾತಾ-ವರ-ಣವೇ
ವಾತ್ಸಲ್ಯ-ವನ್ನು
ವಾದ
ವಾದಕ್ಕೆ
ವಾದ-ಗಳಿವೆ
ವಾದ-ಪರಾ-ಶೇಷಕ್ಷಿತಿ-ವಾಸಿ-ಮನಿ-ಷಿಣೇ
ವಾದ-ಮಾಡಿ
ವಾದ-ವನ್ನು
ವಾದ-ವನ್ನೆಬ್ಬಿ-ಸುತ್ತ-ದೆಂದೂ
ವಾದ-ವಿ-ವಾದ-ಗ-ಳನ್ನು
ವಾದ-ವಿ-ವಾದ-ಗಳು
ವಾದವು
ವಾದಿ
ವಾದಿ-ಕೇ-ಸರಿ
ವಾದಿ-ಕೋಲಾಹಲ
ವಾದಿ-ರಾಜ
ವಾದಿ-ರಾಜ-ಚಾಳುಕ್ಯ
ವಾದಿ-ರಾಜ-ದೇವನ
ವಾದಿ-ರಾಜನು
ವಾದೀಭ-ಕಂಠೀ-ರವ
ವಾದೀಭ-ಸಿಂಹ
ವಾದೀಭ-ಸಿಂಹ-ಸೂರಿಯ
ವಾದ್ಯ-ಗಳು
ವಾದ್ಯ-ವಿ-ಶೇಷ-ಗಳಿ-ರ-ಬ-ಹುದು
ವಾದ್ಯ-ವಿ-ಶೇಷವೋ
ವಾನ-ಮಾಲೆ
ವಾನ-ಮಾಲೈ
ವಾನ-ವನ್
ವಾನ-ವನ್ಮಹಾ-ದೇವಿ
ವಾನ-ವನ್ಮಾ-ದೇವಿ
ವಾಪಸ್
ವಾಮಾಚಾರ
ವಾಮಾಚಾರ-ಗಳು
ವಾಯುವ್ಯ
ವಾಯುವ್ಯಕ್ಕೆ
ವಾಯುವ್ಯದ
ವಾಯುವ್ಯ-ದಲ್ಲಿ
ವಾರ-ಗಳು
ವಾರ-ಣಾಸಿ
ವಾರ-ಣಾಸಿಗೆ
ವಾರ-ಣಾಸಿಯ
ವಾರ-ಣಾಸಿ-ಯಲಿ
ವಾರ-ಣಾಸಿ-ಯಲು
ವಾರದ
ವಾರದ್
ವಾರ-ಸು-ದಾರ-ರನ್ನು
ವಾರ-ಸು-ದಾರ-ರಿಲ್ಲದ
ವಾರ-ಸು-ದಾರ-ರಿಲ್ಲದೇ
ವಾರ-ಸು-ದಾರಿ-ಕೆಯು
ವಾರಿಚ-ಯವು
ವಾರಿಜಭವ
ವಾರಿಜವೈರಿ-ಯೊಳ್ಪ-ಡದು
ವಾರಿಜೋದ್ಭವ
ವಾರಿಧಿ-ಯೊಳ್ಮನೋ-ರ-ತೆಯಂ
ವಾರ್ತೆ-ಯನ್ನು
ವಾರ್ದ್ಧಿ-ವರ್ಧನ
ವಾರ್ಷಿಕ
ವಾಲಗ
ವಾಲಗ-ದವ-ರಿಗೆ
ವಾಲಗ-ದ-ವರು
ವಾಲಗ-ವನ್ನು
ವಾಲ-ಮುತ್ತು
ವಾಸಂತಿಕಾ
ವಾಸಕ್ಕೆ
ವಾಸನಾ
ವಾಸ-ಮದೇ-ಭಾವ-ಳಿಯಂ
ವಾಸ-ಮಾಡಿ-ಕೊಂಡು
ವಾಸ-ಮಾಡುತ್ತಿದ್ದನು
ವಾಸವ
ವಾಸ-ವನ
ವಾಸ-ವ-ನಿಗೆ
ವಾಸ-ವಾ-ಗಿದ್ದ
ವಾಸ-ವಾಗಿದ್ದರೇ
ವಾಸಿ-ಗಳಪ್ಪ
ವಾಸಿ-ಯಾಗಿ
ವಾಸಿ-ಸದ್ದ
ವಾಸಿ-ಸುತ್ತಿದ್ದ
ವಾಸಿ-ಸುತ್ತಿದ್ದರು
ವಾಸಿ-ಸುತ್ತಿದ್ದ-ರೆಂದು
ವಾಸಿ-ಸುತ್ತಿದ್ದ-ರೆಂದೂ
ವಾಸಿ-ಸುತ್ತಿದ್ದ-ರೆದೂ
ವಾಸಿ-ಸುತ್ತಿದ್ದಾ-ರೆಂದು
ವಾಸಿ-ಸುವ
ವಾಸು
ವಾಸುಗಿ
ವಾಸು-ದೇವ
ವಾಸು-ದೇವ-ಕೃಷ್ಣರ
ವಾಸು-ದೇವನ
ವಾಸು-ದೇವ-ನಿ-ಗಾಗಿ
ವಾಸು-ದೇವನು
ವಾಸು-ಪೂಜ್ಯ
ವಾಸು-ಪೂಜ್ಯ-ದೇವರ
ವಾಸು-ಪೂಜ್ಯರ
ವಾಸು-ವಿನ
ವಾಸ್ತವ-ವಾಗಿ
ವಾಸ್ತವ್ಯ
ವಾಸ್ತು
ವಾಸ್ತು-ದೃಷ್ಟಿ-ಯಿಂದ
ವಾಸ್ತು-ರಚನೆ-ಗ-ಳಲ್ಲಿ
ವಾಸ್ತು-ರಚನೆ-ಯಾ-ಗಿದ್ದು
ವಾಸ್ತು-ರಚನೆ-ಯಿಂದ
ವಾಸ್ತು-ವನ್ನು
ವಾಸ್ತು-ವರ್ಣನೆ-ಯನ್ನು
ವಾಸ್ತು-ವಿನ
ವಾಸ್ತು-ಶಿಲ್ಪ
ವಾಸ್ತು-ಶಿಲ್ಪಕ್ಕೆ
ವಾಸ್ತು-ಶಿಲ್ಪ-ದಲ್ಲಿ
ವಾಸ್ತು-ಶಿಲ್ಪ-ವನ್ನು
ವಾಹನ
ವಾಹನ-ಗಳು
ವಾಹನ-ಮಂಟಪ
ವಾಹನ-ವಸ್ತು-ಗಳು
ವಾಹನ-ವಾದ
ವಿ
ವಿಂಗಡಿ-ಸದೇ
ವಿಂಗಡಿ-ಸ-ಬ-ಹುದು
ವಿಂಗಡಿ-ಸ-ಬಹು-ದೆಂದು
ವಿಂಗಡಿ-ಸ-ಲಾಗಿತ್ತೆಂದು
ವಿಂಗಡಿ-ಸ-ಲಾಗಿತ್ತೆಂದೂ
ವಿಂಗಡಿ-ಸ-ಲಾಗಿದೆ
ವಿಂಗಡಿ-ಸ-ಲಾಗುತ್ತಿತ್ತು
ವಿಂಗಡಿ-ಸ-ಲಾ-ಯಿತು
ವಿಂಗಡಿ-ಸಲ್ಪಟ್ಟಿತ್ತು
ವಿಂಗ-ಡಿಸಿ
ವಿಂಗಡಿ-ಸಿ-ಕೊಳ್ಳ-ಬೇಕು
ವಿಂಗಡಿ-ಸಿ-ದರೆ
ವಿಂಗಡಿ-ಸಿದ್ದರು
ವಿಂಗಡಿ-ಸುತ್ತಾರೆ
ವಿಂಧ್ಯಪ್ರಾಂತ-ಗಳ
ವಿಕಲ್ಪ
ವಿಕಾರಿ
ವಿಕಾಸಶೀಲ-ವಾದ
ವಿಕೃತಿಭಿಃ
ವಿಕ್ರಮ
ವಿಕ್ರಮ-ಗಂಗ
ವಿಕ್ರಮನ
ವಿಕ್ರಮ-ನನ್ನು
ವಿಕ್ರಮ-ಯುತರೂ
ವಿಕ್ರಮ-ರಾಯ
ವಿಕ್ರಮ-ರಾಯ-ನೆಂಬು-ವ-ವನು
ವಿಕ್ರಮ-ರಾಯ-ವಿಗಡ
ವಿಕ್ರಮಾಂಕ-ದೇವ-ಚರಿತ-ದಲ್ಲಿ
ವಿಕ್ರಮಾ-ದಿತ್ಯ
ವಿಕ್ರಮಾ-ದಿತ್ಯನ
ವಿಕ್ರಮಾ-ದಿತ್ಯ-ನಿಂದ
ವಿಕ್ರಮಾ-ದಿತ್ಯ-ನಿಗೆ
ವಿಕ್ರಮಾ-ದಿತ್ಯನು
ವಿಕ್ರಮಾರ್ಜಿತ-ವಾಗಿ
ವಿಕ್ರಮಾರ್ಜಿತ-ವಾದ
ವಿಕ್ರಮಾರ್ಜುನ
ವಿಖ್ಯಾತ
ವಿಖ್ಯಾತಗ್ರಾಮಂ
ವಿಖ್ಯಾತಾ
ವಿಖ್ಯಾತೋ
ವಿಗತ
ವಿಗ-ಹ-ವನ್ನು
ವಿಗ್ರಗಹಳಿವೆ
ವಿಗ್ರ-ವಿದ್ದು
ವಿಗ್ರಹ
ವಿಗ್ರಹ-ಗಳ
ವಿಗ್ರಹ-ಗ-ಳನ್ನು
ವಿಗ್ರಹ-ಗಳಿ-ರುವುದು
ವಿಗ್ರಹ-ಗಳಿವೆ
ವಿಗ್ರಹ-ಗಳು
ವಿಗ್ರಹ-ಗಳೂ
ವಿಗ್ರಹದ
ವಿಗ್ರಹ-ದಿಂದ
ವಿಗ್ರಹ-ವನ್ನು
ವಿಗ್ರಹ-ವಿದೆ
ವಿಗ್ರಹ-ವಿದ್ದು
ವಿಗ್ರಹ-ವಿನೋದ
ವಿಗ್ರಹವು
ವಿಗ್ರಹವೂ
ವಿಗ್ರಹ-ವೆಂದು
ವಿಗ್ರಹಸ್ಥಾ-ಪನೆ-ಯಂತಹ
ವಿಘಟನ
ವಿಘ್ನೇಶ್ವರ
ವಿಚಳಿತಂ
ವಿಚಾರ
ವಿಚಾರಕ್ಕೆ
ವಿಚಾರ-ಗಳ
ವಿಚಾರ-ಗ-ಳನ್ನು
ವಿಚಾರ-ಗಳನ್ನುಳ್ಳ
ವಿಚಾರ-ಗಳಾ-ಗಲೀ
ವಿಚಾರ-ಗ-ಳಾಗಿವೆ
ವಿಚಾರ-ಗ-ಳಿಗೆ
ವಿಚಾರ-ಗಳು
ವಿಚಾರ-ಗಳೂ
ವಿಚಾರ-ಗ-ವಾಗಲೀ
ವಿಚಾರ-ಚಾ-ವಡಿ
ವಿಚಾರ-ಣೆಯ
ವಿಚಾರದ
ವಿಚಾರ-ದಲ್ಲಿ
ವಿಚಾರ-ವನ್ನು
ವಿಚಾರ-ವಾಗಲೀ
ವಿಚಾರ-ವಾಗಿದೆ
ವಿಚಾರ-ವಿದೆ
ವಿಚಾ-ರವು
ವಿಚಾ-ರವೂ
ವಿಚಾ-ರವೇ
ವಿಚಾ-ರಾರ್ಹ
ವಿಚಾ-ರಾರ್ಹ-ವಾಗುತ್ತದೆ
ವಿಚಾರಿ-ಸಲು
ವಿಚಾ-ರಿ-ಸಿದಾಗ
ವಿಚಿತ್ರ-ವಾದ
ವಿಜಗೀಷು-ವೃತ್ತಿಯಿಂ
ವಿಜಯ
ವಿಜಯ-ಕೀರ್ತಿ
ವಿಜಯ-ಕೀರ್ತಿ-ದೇವರ
ವಿಜಯ-ಕೀರ್ತಿಯ
ವಿಜಯಕ್ಕಾಗಿ
ವಿಜಯ-ಗಳ
ವಿಜಯ-ಗ-ಳನ್ನು
ವಿಜಯ-ಗ-ಳಾಗಿದ್ದು
ವಿಜಯತೇ
ವಿಜಯದ
ವಿಜಯ-ದಂತಹ
ವಿಜಯ-ದಲ್ಲಿ
ವಿಜಯ-ದೇವ-ರಾಯ
ವಿಜಯ-ದೇವ-ರಾಯನ
ವಿಜಯ-ದೇವ-ರಾಯ-ನೆಂಬ
ವಿಜಯ-ನ-ಗದ
ವಿಜಯ-ನ-ಗರ
ವಿಜಯ-ನ-ಗರಕ್ಕೆ
ವಿಜಯ-ನ-ಗರದ
ವಿಜಯ-ನ-ಗರ-ದಿಂದ
ವಿಜಯ-ನ-ಗರ-ದೊರೆ
ವಿಜಯ-ನ-ಗರ-ವಾದ
ವಿಜಯ-ನಗ-ರವು
ವಿಜಯ-ನ-ಗರಿ-ಯಲ್ಲಿ
ವಿಜಯ-ನ-ಗರೋತ್ತರ
ವಿಜಯ-ನರ-ಸಿಂಹ
ವಿಜಯ-ನಾರ-ಸಿಂಹ
ವಿಜಯ-ನಾರ-ಸಿಂಹ-ದೇವ-ರಿಗೆ
ವಿಜಯ-ನಾರ-ಸಿಂಹನ
ವಿಜಯ-ನಾರ-ಸಿಂಹನು
ವಿಜಯ-ನಾ-ರಾಯಣ
ವಿಜಯ-ನಾ-ರಾಯ-ಣ-ದೇವರ
ವಿಜಯ-ಪಾಂಡ್ಯ-ನುಚ್ಚಂಗಿ
ವಿಜಯ-ಪುರ
ವಿಜಯ-ಪುರದ
ವಿಜಯ-ಪುರ-ದಲ್ಲಿ-ರುವ
ವಿಜಯ-ಬುಕ್ಕ
ವಿಜಯ-ಬುಕ್ಕ-ರಾಯನ
ವಿಜಯ-ಭೂ-ಪತಿ
ವಿಜಯ-ಯಾತ್ರೆಯ
ವಿಜಯ-ರಘು-ನಾಥ
ವಿಜಯ-ರಾಜ-ಧಾನಿ
ವಿಜಯ-ರಾಯ
ವಿಜಯ-ವನ್ನು
ವಿಜಯ-ವನ್ನೇ
ವಿಜಯ-ವರ್ದ್ಧ-ನನೀ
ವಿಜಯ-ವಿ-ರೂಪಾಕ್ಷ
ವಿಜಯ-ವೆಂದು
ವಿಜಯವೇ
ವಿಜಯ-ಶಾಲಿ-ಯಾಗಿ
ವಿಜಯಶ್ರೀ
ವಿಜಯ-ಸಂವತ್ಸರದ
ವಿಜಯ-ಸ-ಮುದ್ರ-ವೆನಿಸಿದ
ವಿಜಯ-ಸೋಮ-ನಾಥ-ಪುರ
ವಿಜಯಸ್ಕನ್ದಾ-ವಾರ
ವಿಜಯಸ್ಕಾಂದ-ವಾರ-ವಾದ
ವಿಜಯಸ್ತಂಭ-ವನ್ನು
ವಿಜ-ಯಾ-ದಿತ್ಯ
ವಿಜ-ಯಾ-ದಿತ್ಯ-ನಾ-ಗಿದ್ದು
ವಿಜ-ಯಾ-ದಿತ್ಯ-ನಿಗೆ
ವಿಜ-ಯಾ-ದಿತ್ಯನು
ವಿಜಯೀಂದ್ರ
ವಿಜಯೋತ್ತುಂಗ
ವಿಜಯೋತ್ಸವ
ವಿಜಯೋತ್ಸವ-ವನ್ನಾಚರಿ-ಸಲು
ವಿಜಯೋತ್ಸವ-ವನ್ನು
ವಿಜಾ-ಪುರ
ವಿಜಿಗೀಷು
ವಿಜೃಂಭಿ-ಸಿತು
ವಿಜೆಯ-ನರ-ಸಿಂಹಂ
ವಿಜೆಯ-ರಾಯ
ವಿಜ್ಞಪ್ತಿ
ವಿಜ್ಞಾ-ಪನೆ
ವಿಜ್ಞಾಪ-ನೆಯ
ವಿಜ್ಞಾ-ಪನೆ-ಯನ್ನು
ವಿಟ್ಟ
ವಿಟ್ಟಣ್ಣನು
ವಿಟ್ಟಾರ್
ವಿಟ್ಟಿ-ದೇವನು
ವಿಟ್ಟಿಯಣ್ಣ
ವಿಟ್ಟಿಯಣ್ಣನು
ವಿಠಂಣ-ಗಳ
ವಿಠಂಣ್ಣ
ವಿಠಂಣ್ಣ-ಗಳ
ವಿಠಂಣ್ನ-ಗಳ
ವಿಠಣ್ಣ-ನಿಗೆ
ವಿಠಣ್ಣನ್ನು
ವಿಠಣ್ಣ-ಹೆಗ್ಗಡೆಯು
ವಿಠಣ್ನಂಗೆ
ವಿಠಲಪ್ರಭು-ಗಳು
ವಿಠಲೇಶ್ವರ
ವಿಣ್ಣ-ಗರ
ವಿಣ್ಣ-ಗರ್
ವಿಣ್ಣಯಾಂಡನ
ವಿಣ್ಣ-ಯಾಂಡರ
ವಿಣ್ಣ-ಯಾಂಡರನ
ವಿಣ್ನ-ಗರತ್ತಾಳ್ವಾರ್
ವಿಣ್ನಘರಂ
ವಿಣ್ನಘರ್
ವಿತರಿಸಿ
ವಿತ್ತಿ
ವಿತ್ತಿ-ರುಂದ
ವಿತ್ತಿ-ರುಂದ-ವಿರ್ರಿ-ರುಂದ
ವಿದಿತ
ವಿದಿಶಾ-ದಲ್ಲಿ
ವಿದು-ಷಸ್ಸತ್ಕುಲೋತ್ಪಂನಾಂತ್ಸಾತ್ವಿ-ಕಾ-ನನ-ಸೂ-ಯಕಾನ್
ವಿದು-ಷೋ-ಮುದೇ
ವಿದೂಷಕ-ಹಾಸ್ಯ-ಗಾರ-ರಾಗಿದ್ದ-ರೆಂದು
ವಿದೆ
ವಿದೇಶ
ವಿದೇಶಾಂಗ
ವಿದೇಶಾಂಗಕ್ಕೆ
ವಿದೇಶಿ-ಯರೂ
ವಿದ್ಯಾ
ವಿದ್ಯಾ-ಕೆಂದ್ರ-ಗ-ಳನ್ನು
ವಿದ್ಯಾ-ಕೇಂದ್ರ-ಗಳು
ವಿದ್ಯಾ-ದಾನ-ಗ-ಳಿಂದ
ವಿದ್ಯಾ-ದಾನ-ದಲ್ಲಿ
ವಿದ್ಯಾ-ಧರ
ವಿದ್ಯಾ-ಧ-ರನು
ವಿದ್ಯಾ-ಧ-ರನೂ
ವಿದ್ಯಾ-ನ-ಗರ-ದಿಂದ
ವಿದ್ಯಾ-ನ-ಗರಿ
ವಿದ್ಯಾ-ನ-ಗರಿಯ
ವಿದ್ಯಾ-ನ-ಗರಿ-ಯಿಂದ
ವಿದ್ಯಾ-ನ-ಗರ್ಯ್ಯಾಂ
ವಿದ್ಯಾ-ನಿಧಿ
ವಿದ್ಯಾಭ್ಯಾಸ
ವಿದ್ಯಾಭ್ಯಾಸಕ್ಕಾಗಿ
ವಿದ್ಯಾಭ್ಯಾಸಕ್ಕೆ
ವಿದ್ಯಾಭ್ಯಾಸ-ವನ್ನು
ವಿದ್ಯಾಭ್ಯಾಸವು
ವಿದ್ಯಾ-ರಣ್ಯರು
ವಿದ್ಯಾ-ರಾಜು
ವಿದ್ಯಾರ್ತ್ತಿಗುತ್ಸಾ-ಹದಿತ್ತಂ
ವಿದ್ಯಾರ್ಥಿ-ಗಳ
ವಿದ್ಯಾರ್ಥಿ-ಗ-ಳಿಗೆ
ವಿದ್ಯಾರ್ಥಿ-ಗಳು
ವಿದ್ಯಾರ್ಥಿಯೂ
ವಿದ್ಯಾರ್ಹತೆ-ಗ-ಳನ್ನು
ವಿದ್ಯಾ-ಲಕ್ಷ್ಮೀಪ್ರಧಾನ
ವಿದ್ಯಾ-ವಂತರು
ವಿದ್ಯಾ-ವಿಶಾ-ರದ-ರಪ್ಪ
ವಿದ್ಯಾ-ವಿ-ಶೇಷ
ವಿದ್ಯಾ-ವಿ-ಶೇಷ-ಗ-ಳನ್ನು
ವಿದ್ಯಾ-ವಿ-ಶೇಷ-ತೆ-ಗ-ಳನ್ನು
ವಿದ್ಯಾ-ಶಿಷ್ಯಾ-ಸನಾಧೀಶ್ವರ-ವಿದ್ಯಾ-ಸಿಂಹಾಸ-ನಾಧೀಶ್ವರ
ವಿದ್ಯಾ-ಸಿಂಹಾಸ-ನಾಧೀಶ್ವರ
ವಿದ್ಯೆ-ಯಲ್ಲಿ
ವಿದ್ರಾ-ವಣಂ
ವಿದ್ವ-ಜನ-ಪೋಷ-ಕನೂ
ವಿದ್ವ-ಜನ-ವಿ-ಪದ-ಳನ
ವಿದ್ವತ್
ವಿದ್ವತ್ಕವೀಂದ್ರಃ
ವಿದ್ವತ್ತಾ-ಗಲೀ
ವಿದ್ವತ್ತಿನ
ವಿದ್ವತ್ತು
ವಿದ್ವನ್
ವಿದ್ವನ್ಮಂಡ-ಲಿಯ
ವಿದ್ವಾಂಸ
ವಿದ್ವಾಂಸ-ನಾಗಿದ್ದ-ನೆಂದು
ವಿದ್ವಾಂಸ-ನಾದ
ವಿದ್ವಾಂಸ-ನಿಗೆ
ವಿದ್ವಾಂಸರ
ವಿದ್ವಾಂಸ-ರನ್ನು
ವಿದ್ವಾಂಸ-ರ-ಮತ
ವಿದ್ವಾಂಸ-ರಲ್ಲಿ
ವಿದ್ವಾಂಸ-ರಾ-ಗಿದ್ದ
ವಿದ್ವಾಂಸ-ರಾ-ಗಿದ್ದರು
ವಿದ್ವಾಂಸ-ರಾದ
ವಿದ್ವಾಂಸ-ರಿಂದ
ವಿದ್ವಾಂಸ-ರಿಗೆ
ವಿದ್ವಾಂಸರು
ವಿದ್ವಾಂಸ-ರು-ಗಳು
ವಿದ್ವಾನ್
ವಿದ್ವಿಷ್ಟ
ವಿಧ
ವಿಧ-ಗ-ಳನ್ನು
ವಿಧ-ಗಳೋ
ವಿಧದ
ವಿಧ-ವಾಗಿದ್ದಿತು
ವಿಧ-ವಾದ
ವಿಧ-ವೆ-ಯಾಗಿದ್ದರೂ
ವಿಧಾನ
ವಿಧಾನ-ಗ-ಳನ್ನು
ವಿಧಾನ-ದಿಂದ
ವಿಧಾನ-ವನ್ನು
ವಿಧಿ
ವಿಧಿ-ಗಳ
ವಿಧಿ-ಗ-ಳನ್ನು
ವಿಧಿಗೆ
ವಿಧಿ-ಯಿಂದ
ವಿಧಿ-ಸದೇ
ವಿಧಿ-ಸ-ಲಾಗಿದೆ
ವಿಧಿ-ಸ-ಲಾಗುತ್ತಿತು
ವಿಧಿ-ಸ-ಲಾಗುತ್ತಿತ್ತು
ವಿಧಿ-ಸಲಾ-ಗುತ್ತಿತ್ತೆಂದು
ವಿಧಿ-ಸಲ್ಪಡುತ್ತಿದ್ದ
ವಿಧಿಸಿ
ವಿಧಿ-ಸಿದೆ
ವಿಧಿ-ಸಿದ್ದು
ವಿಧಿ-ಸಿ-ರುವ
ವಿಧಿ-ಸಿ-ರುವುದು
ವಿಧಿ-ಸಿ-ಲಾಗಿ-ರುವುದು
ವಿಧಿ-ಸು-ತಿದ್ದ
ವಿಧಿ-ಸು-ತಿದ್ದರು
ವಿಧಿ-ಸುತ್ತಾರೆ
ವಿಧಿ-ಸುತ್ತಾಳೆ
ವಿಧಿ-ಸುತ್ತಿದ್ದ
ವಿಧಿ-ಸುತ್ತಿದ್ದರು
ವಿಧಿ-ಸುತ್ತಿದ್ದ-ರೆಂದು
ವಿಧಿ-ಸುತ್ತಿದ್ದಿರ-ಬ-ಹುದು
ವಿಧಿ-ಸುತ್ತಿದ್ದು
ವಿಧಿ-ಸುತ್ತಿದ್ದುದು
ವಿಧಿ-ಸುವ
ವಿಧಿ-ಸು-ವುದು
ವಿಧೇಯ-ರಾಗಿ
ವಿಧೋರಲಗ
ವಿನಂತಿ
ವಿನಂತಿ-ಯ-ಮೇರೆಗೆ
ವಿನಂತಿ-ಸಿ-ಕೊಂಡೆ
ವಿನಯ
ವಿನ-ಯ-ದಾ-ಗಾರ-ಮಾ-ದತ್ತು
ವಿನ-ಯ-ನನ್ದಿ-ದೇವರ
ವಿನ-ಯ-ನನ್ದಿ-ಮುನಿ
ವಿನ-ಯ-ಪರಂ
ವಿನ-ಯ-ಪೂರ್ವ-ಕ-ವಾಗಿ
ವಿನ-ಯ-ವಾಗಿ
ವಿನ-ಯ-ವಿಭೂಷಿ-ತನೂ
ವಿನ-ಯ-ಶಾ-ಲಿನಿ-ಯಾದ
ವಿನ-ಯ-ಶೀ-ಲರು
ವಿನ-ಯಸ್ಯೇವ
ವಿನ-ಯಾ-ದಿತ್ಯ
ವಿನ-ಯಾ-ದಿತ್ಯನ
ವಿನ-ಯಾ-ದಿತ್ಯ-ನನ್ನು
ವಿನ-ಯಾ-ದಿತ್ಯ-ನಾಗಿದ್ದಾನೆ
ವಿನ-ಯಾ-ದಿತ್ಯನು
ವಿನ-ಯಾ-ದಿತ್ಯನೂ
ವಿನ-ಯಾ-ದಿತ್ಯ-ನೆಂದು
ವಿನ-ಯಾ-ದಿತ್ಯನೇ
ವಿನಾ
ವಿನಾ-ಯತಿ-ಯನ್ನು
ವಿನಾ-ಶದ
ವಿನಿ-ಮಯ
ವಿನಿ-ಮಯವ್ಯಾಪಾರ-ವೃತ್ತಿ-ಯ-ವರು
ವಿನಿ-ಮಯೋಚಿತಂ
ವಿನಿ-ಯಮ
ವಿನಿ-ಯೋಗದ
ವಿನಿರ್ಜ್ಜಿತಾ-ಘಟಕುಟೀ
ವಿನುತ
ವಿನುತಪ್ರಾಭವ-ಕೀರ್ತ್ತಿ-ರಾಜನ
ವಿನೇಜತ್ಸಾಮ್ರ
ವಿನೇಯ-ವಿಳಾಸಂ
ವಿನೋದದಿಂ
ವಿನೋದ-ದಿಂದ
ವಿನೋದನೂ
ವಿನೋದ-ರುಂಮಪ್ಪ
ವಿನೋದಿ
ವಿನೋದಿಂದಾಳೆ
ವಿನೋದೆ-ಯ-ರು-ಮಪ್ಪ
ವಿನೋದೆಯುಂ
ವಿನ್ಯಾಸ
ವಿಪತ್ತು
ವಿಪರೀತ-ವಾಗಿದ್ದಂತೆ
ವಿಪಾಕ
ವಿಪುಲ-ವಾಗಿ
ವಿಪುಲ-ಸಂಪತ್ತನ್ನು
ವಿಪ್ರ-ಕುಲ-ತಿಲಕನೂ
ವಿಪ್ರರ
ವಿಪ್ರರು
ವಿಪ್ರಶ್ರೇಷ್ಠ-ನಾ-ಗಿದ್ದು
ವಿಪ್ರೋತ್ತಮ-ನಿಗೆ
ವಿಪ್ಲವ
ವಿಫಲ-ಗೊ-ಳಿಸಿ
ವಿಫಲ-ವಾಯಿತು
ವಿಬುಧ
ವಿಬುಧಂ
ವಿಬುಧಜ
ವಿಬುಧ-ಜ-ನ-ಫಳಪ್ರ-ದಾಯಕಂ
ವಿಬುಧಪ್ರಸನ್ನನುಂ
ವಿಭ-ಜನೆ
ವಿಭ-ಜನೆ-ಯಾಗಿದ್ದ-ರಿಂದ
ವಿಭ-ಜನೆ-ಯಾಗಿ-ರು-ವಂತೆ
ವಿಭಜಿತ-ವಾ-ಗಿದ್ದು
ವಿಭಜಿಸ-ಲಾಗಿ
ವಿಭಜಿ-ಸ-ಲಾ-ಗಿತ್ತು
ವಿಭಜಿಸ-ಲಾಗುತ್ತಿತ್ತು
ವಿಭಜಿಸಿ
ವಿಭಜಿ-ಸಿತ್ತು
ವಿಭಜಿಸಿ-ದನು
ವಿಭಜಿ-ಸಿದ್ದು
ವಿಭ-ವನಿ-ವಹ-ನಿ-ದಾನಸ್ಯ
ವಿಭವಪ್ರ-ಭಾವ-ತೆ-ಯಿಂದಂ
ವಿಭಾಗ
ವಿಭಾ-ಗಕ್ಕೆ
ವಿಭಾಗ-ಗಳ
ವಿಭಾಗ-ಗ-ಳನ್ನಾಗಿ
ವಿಭಾಗ-ಗ-ಳನ್ನು
ವಿಭಾಗ-ಗ-ಳಲ್ಲಿ
ವಿಭಾಗ-ಗ-ಳಾಗಿ
ವಿಭಾಗ-ಗ-ಳಾಗಿದ್ದವು
ವಿಭಾಗ-ಗಳಿಗೂ
ವಿಭಾಗ-ಗ-ಳಿಗೆ
ವಿಭಾಗ-ಗಳಿದ್ದವು
ವಿಭಾಗ-ಗಳು
ವಿಭಾಗ-ಗಳೂ
ವಿಭಾಗ-ಗ-ಳೆಂದು
ವಿಭಾಗ-ಗಳೆನ್ನ-ಬ-ಹುದು
ವಿಭಾಗ-ಗಳೇ
ವಿಭಾಗದ
ವಿಭಾಗ-ದಲ್ಲಿ
ವಿಭಾಗ-ಮೈಸೂರು
ವಿಭಾಗ-ವನ್ನು
ವಿಭಾಗ-ವಾ-ಗಿತ್ತು
ವಿಭಾಗ-ವಾಗಿತ್ತೆಂದು
ವಿಭಾಗ-ವಾ-ಗಿದ್ದ
ವಿಭಾಗ-ವಾಗಿದ್ದಿರ-ಬ-ಹುದು
ವಿಭಾಗ-ವಾ-ಗಿದ್ದು
ವಿಭಾಗವು
ವಿಭಾಗವೂ
ವಿಭಾಗ-ವೆಂದು
ವಿಭಾಗವೇ
ವಿಭಾಗಿಸ-ಬ-ಹುದು
ವಿಭಾಗಿ-ಸ-ಲಾ-ಗಿತ್ತು
ವಿಭಾಗಿಸ-ಲಾಗಿದೆ
ವಿಭಾಗಿಸಿ
ವಿಭಾಗಿಸಿ-ಕೊಂಡು
ವಿಭಾಗಿಸಿ-ಕೊಳ್ಳ-ಬ-ಹುದು
ವಿಭಾಡ-ರೆ-ನಿ-ಸಿದ
ವಿಭಾದ
ವಿಭಿನ್ನ-ವಾಗಿದೆ
ವಿಭಿನ್ನ-ವಾಗಿವೆ
ವಿಭು
ವಿಭು-ಗಳು
ವಿಭುದ
ವಿಭು-ದೇವ-ರಾಜನಂ
ವಿಭು-ದೇಶಂ
ವಿಭುಪ್ರಭು
ವಿಭು-ಬಲ್ಲಯ್ಯ-ನಾಯಕ
ವಿಭು-ಹೆರ್ಮಾಡಿ
ವಿಭೂತಿ
ವಿಭೂತಿ-ಕುಪ್ಪೆ
ವಿಭೂತಿ-ಯನ್ನು
ವಿಭೋತೇರಚ್ಯುತ
ವಿಭ್ಬಾಡ-ನರು
ವಿಭ್ಬಾಡ-ನರುಂ
ವಿಭ್ರಮಾ
ವಿಮಲ
ವಿಮಲ-ನಾಥ
ವಿಮಲ-ನಾಥನ
ವಿಮಲ-ವೇದಾಂತಾಂಅð
ವಿಮಳ-ಗಂಗಾನ್ವಯ
ವಿಮಳಚನ್ದ್ರಾಚಾಯ
ವಿಮಾನ
ವಿಮಾನ-ವನ್ನು
ವಿಯಷ-ವನ್ನು
ವಿರಕ್ತ
ವಿರಕ್ತನ
ವಿರಕ್ತರ
ವಿರಕ್ತ-ರನ್ನು
ವಿರಕ್ತ-ರಲ್ಲಿ
ವಿರಕ್ತ-ರಿಗೆ
ವಿರಚಿತ
ವಿರಚಿಸಿದಂ
ವಿರಣೆ-ಯನ್ನು
ವಿರ-ಭದ್ರ-ದುರ್ಗಸ್ಥಳದ
ವಿರಳ
ವಿರಳ-ವೆಂದು
ವಿರವ-ಗಳಿವೆ
ವಿರಾಜ-ಮಾನ
ವಿರಾಜ-ಮಾನಂತಂತ್ರ-ರಕ್ಷಾ-ಮಣಿ
ವಿರಾಜಿತ
ವಿರಾಜಿತ-ನಾಗಿದ್ದ-ನೆಂದು
ವಿರಾಜಿತ-ವಾಗಿ-ರುವ
ವಿರಾಜಿರಾಂಬರಂ
ವಿರಾಟನ-ಪೊಳಲು
ವಿರು-ದಯ-ರಾಯ
ವಿರುದ್ಧ
ವಿರುದ್ಧದ
ವಿರುದ್ಧ-ವಾಗಿಯೋ
ವಿರುದ್ಧ-ವಾಗಿ-ರಲು
ವಿರುದ್ಧವೇ
ವಿರುಪಂಣ
ವಿರು-ಪಣ್ಣ
ವಿರು-ಪಣ್ಣ-ನ-ವರ
ವಿರು-ಪಣ್ಣ-ನಾಯ-ಕನು
ವಿರು-ಪನ-ಪುರ
ವಿರು-ಪನ-ಪುರ-ವಿರು-ಪಾ-ಪುರ
ವಿರು-ಪಯ್ಯ
ವಿರುಪ-ರಾಜ
ವಿರುಪಾಕ್ಷ-ದೇವ
ವಿರುಪಾಕ್ಷ-ದೇವ-ಅಣ್ಣನು
ವಿರೂಪ-ವಾಗಿದೆ
ವಿರೂಪಾಕ್ಷ
ವಿರೂಪಾಕ್ಷ-ದಲಿ
ವಿರೂಪಾಕ್ಷ-ದೇವ
ವಿರೂಪಾಕ್ಷ-ದೇವನ
ವಿರೂಪಾಕ್ಷ-ದೇವರ
ವಿರೂಪಾಕ್ಷನ
ವಿರೂಪಾಕ್ಷ-ನನ್ನು
ವಿರೂಪಾಕ್ಷ-ನಿಗೆ
ವಿರೂಪಾಕ್ಷನು
ವಿರೂಪಾಕ್ಷ-ನೆಂಬ
ವಿರೂಪಾಕ್ಷ-ಪುರ
ವಿರೂಪಾಕ್ಷ-ಪುರ-ವೆಂದು
ವಿರೂಪಾಕ್ಷಯ್ಯ
ವಿರೂಪಾಕ್ಷಿ-ಪುರ-ಗ-ಳಲ್ಲಿ
ವಿರೋಧ
ವಿರೋಧಿ
ವಿರೋಧಿ-ಗ-ಳಾಗಿದ್ದುದೂ
ವಿರೋಧಿ-ಗ-ಳಾಗಿ-ರ-ಲಿಲ್ಲ
ವಿರೋಧಿ-ಸಂವತ್ಸರದ
ವಿರೋಧಿಸಿ
ವಿರೋಧಿ-ಸಿ-ಚಿವೋನ್ಮತ್ತೇಭ
ವಿರ್ರಿ-ರುಂದ
ವಿರ್ರಿ-ರುಂದ-ಪೆರು-ಮಾಳೆ
ವಿರ್ರಿರುದಂದ
ವಿಲಸ-ಗೆರೆ
ವಿಲಸ-ಗೆರೆಗೆ
ವಿಲ-ಸತ್
ವಿಲಸಿತ
ವಿಲಸ್
ವಿಲಾಸ-ದರ್ಪಣ-ದಂತೆ
ವಿಲೀನಗೊಳಿ-ಸ-ಲಾ-ಯಿತು
ವಿಲೇ-ವಾರಿ
ವಿಳಂದೆ
ವಿಳ-ಸತ್
ವಿಳ-ಸದ್ಬಲ್ಲಾಳ-ದೇವಾ-ವನೀ-ಪತಿಗೀ
ವಿಳಾಸಂ
ವಿಳು-ಕಾಟು
ವಿವರ
ವಿವರಃ
ವಿವರ-ಗಳ
ವಿವರ-ಗ-ಳನ್ನು
ವಿವರ-ಗ-ಳನ್ನೂ
ವಿವರ-ಗಳಾ-ಗಲೀ
ವಿವರ-ಗಳಿದ್ದು
ವಿವರ-ಗಳಿಲ್ಲ
ವಿವರ-ಗಳಿಲ್ಲದೇ
ವಿವರ-ಗಳಿವೆ
ವಿವರ-ಗಳು
ವಿವರ-ಗಳೂ
ವಿವರ-ಗಳೇ-ನಿಲ್ಲ
ವಿವ-ರಣೆ
ವಿವರ-ಣೆ-ಯನ್ನು
ವಿವರ-ಣೆ-ಯಿಂದ
ವಿವರ-ವನ್ನು
ವಿವರ-ವಾಗಿ
ವಿವರ-ವಾ-ಗಿದ್ದು
ವಿವರ-ವಾಗಿಯೇ
ವಿವರ-ವಾದ
ವಿವರ-ವಿದೆ
ವಿವ-ರವೂ
ವಿವರಿ-ಸ-ಬ-ಹುದು
ವಿವರಿ-ಸ-ಲಾಗಿದೆ
ವಿವರಿಸಿ
ವಿವರಿ-ಸಿದ
ವಿವರಿ-ಸಿದೆ
ವಿವರಿ-ಸಿದ್ದಾರೆ
ವಿವರಿ-ಸುತ್ತದೆ
ವಿವ-ರಿ-ಸುವ
ವಿವ-ರಿ-ಸುವುದಕ್ಕಿಂತ
ವಿವಾದ
ವಿವಾಹ
ವಿವಾಹಂ
ವಿವಾಹ-ಕಾಲ-ದಲ್ಲಿ
ವಿವಾಹಕ್ಕೆ
ವಿವಾ-ಹದ
ವಿವಾಹ-ಮಾಡಿ-ಕೊಟ್ಟು
ವಿವಾಹ-ವಾ-ಗಿದ್ದ
ವಿವಾಹ-ವಾದ
ವಿವಾಹ-ವಾದನು
ವಿವಿಧ
ವಿವಿಧ-ಕಡೆ-ಗ-ಳಿಗೆ
ವಿವಿಧ-ಬ-ಗೆಯ
ವಿವಿಧ-ಮುಖ-ಗ-ಳನ್ನು
ವಿವೃಣ್ವತೇ
ವಿವೇಚ-ನೆಯ
ವಿವೇಚಿಸ
ವಿವೇಚಿಸದ್ದಾರೆ
ವಿವೇಚಿಸ-ಬ-ಹುದು
ವಿವೇಚಿಸ-ಲಾಗಿದೆ
ವಿವೇಚಿಸಿ-ದಲ್ಲಿ
ವಿವೇಚಿ-ಸಿದಾಗ
ವಿವೇಚಿ-ಸಿದ್ದಾರೆ
ವಿವೇಚಿ-ಸಿದ್ದು
ವಿವೇಚಿ-ಸಿರುವ
ವಿಶದ
ವಿಶಾಖ
ವಿಶಾ-ರದ
ವಿಶಾಲ
ವಿಶಾಲ-ಮುದ್ರಿ
ವಿಶಾಲ-ವಾಗಿ
ವಿಶಾಲ-ವಾದ
ವಿಶಾಳ
ವಿಶಾಳ-ಶ-ಶೋಭಾ-ಯ-ಮಾನ
ವಿಶಾಳ-ಶಿರ
ವಿಶಿವಾನಂದ್
ವಿಶಿಷ್ಟ
ವಿಶಿಷ್ಟ-ತೆ-ಗ-ಳಿಂದೊಡ-ಗೂಡಿದ
ವಿಶಿಷ್ಟ-ದೊಳಗೆ
ವಿಶಿಷ್ಟ-ಪದ್ಧತಿ
ವಿಶಿಷ್ಟ-ರೀತಿ-ಯಲ್ಲಿ
ವಿಶಿಷ್ಟ-ವಾಗಿ
ವಿಶಿಷ್ಟ-ವಾದ
ವಿಶಿಷ್ಟ-ಶಾ-ಸನ-ವಾಗಿದೆ
ವಿಶಿಷ್ಟಾ-ಚಾರ್ಯ-ವೇಷಾಯ
ವಿಶಿಷ್ಟಾದ್ವೈತ
ವಿಶಿಷ್ಟಾದ್ವೈತ-ದಂತೆ
ವಿಶಿಷ್ಟಾದ್ವೈತವು
ವಿಶಿಷ್ಟಾದ್ವೈತ-ವೆಂದರೆ
ವಿಶಿಷ್ಟಾದ್ವೈತಿ-ಗಳ
ವಿಶಿಷ್ಟಾದ್ವೈತಿ-ಗಳು
ವಿಶಿಷ್ಯಾರ್ಥಂ
ವಿಶುದ್ಧ
ವಿಶೇಷ
ವಿಶೇಷಣ
ವಿಶೇಷ-ಣ-ಗ-ಳನ್ನು
ವಿಶೇಷ-ಣ-ಗ-ಳನ್ನೂ
ವಿಶೇಷ-ಣ-ಗ-ಳಿಂದ
ವಿಶೇಷ-ಣ-ಗ-ಳಿಂದಲೇ
ವಿಶೇಷ-ಣ-ಗಳಿಲ್ಲ
ವಿಶೇಷ-ಣ-ಗಳು
ವಿಶೇಷ-ಣ-ಗಳೂ
ವಿಶೇಷ-ಣ-ದಿಂದ
ವಿಶೇಷ-ಣ-ವನ್ನು
ವಿಶೇಷ-ಣ-ವಿಲ್ಲ
ವಿಶೇಷ-ಣವು
ವಿಶೇಷ-ಣವೂ
ವಿಶೇಷ-ಣವೇ
ವಿಶೇಷತಃ
ವಿಶೇಷದ
ವಿಶೇಷ-ದ-ವರು
ವಿಶೇಷದಿಂ
ವಿಶೇಷ-ವಾಗಿ
ವಿಶೇಷ-ವಾಗಿದೆ
ವಿಶೇಷ-ವಾ-ಗಿದ್ದು
ವಿಶೇಷ-ವಾದ
ವಿಶೇಷ-ವೆಂದರೆ
ವಿಶೇಷಾರ್ಥ-ದಲ್ಲಿ
ವಿಶೇಷೋತ್ಸವ
ವಿಶ್ರಮಿಸಿ-ಕೊಳ್ಳಲು
ವಿಶ್ರಾಂತಿ
ವಿಶ್ಲೇಶಿ-ಸಿದ್ದಾರೆ
ವಿಶ್ಲೇಷಣಾತ್ಮಕ
ವಿಶ್ಲೇಷಣೆ
ವಿಶ್ಲೇಷಣೆ-ಗ-ಳಿಂದ
ವಿಶ್ಲೇಷಣೆಗೆ
ವಿಶ್ಲೇಷಣೆ-ಯನ್ನೂ
ವಿಶ್ಲೇಷಣೆ-ಯಾಗಲೀ
ವಿಶ್ಲೇಷಣೆ-ಯಿಂದ
ವಿಶ್ಲೇಷಿಸ-ಬ-ಹುದು
ವಿಶ್ಲೇಷಿಸ-ಲಾಗಿದೆ
ವಿಶ್ಲೇಷಿಸಹ-ಬ-ಹುದು
ವಿಶ್ಲೇಷಿ-ಸಿದ್ದಾರೆ
ವಿಶ್ವ-ಕರ್ಮ
ವಿಶ್ವ-ಕರ್ಮ-ಕುಲದ
ವಿಶ್ವ-ಕರ್ಮ-ಣಾ-ಚಾರ್ಯನ
ವಿಶ್ವ-ಕರ್ಮದ
ವಿಶ್ವ-ಕರ್ಮನ
ವಿಶ್ವ-ಕರ್ಮರು
ವಿಶ್ವ-ಕರ್ಮಾ-ಚಾರ್ಯ
ವಿಶ್ವ-ಕರ್ಮಾ-ಚಾರ್ಯನು
ವಿಶ್ವ-ಕರ್ಮ್ಮಾ-ಚಾರ್ಯ್ಯೇಣೇದಂ
ವಿಶ್ವಕೋಶ-ದಂತಿದೆ
ವಿಶ್ವಜ್ಞ
ವಿಶ್ವಣ್ಣ
ವಿಶ್ವಣ್ಣ-ನಿಗೆ
ವಿಶ್ವ-ನಾಥ
ವಿಶ್ವ-ನಾಥ-ದೇವ-ರಿಗೆ
ವಿಶ್ವ-ನಾಥ-ದೇವಾ-ಲಯ-ವೊಂದೇ
ವಿಶ್ವ-ನಾಥ-ನೆಂಬು-ವ-ವನು
ವಿಶ್ವ-ನಾಥ-ಪುರ-ವಾದ
ವಿಶ್ವಬ್ರಾಹ್ಮಣ-ರೆಂದು
ವಿಶ್ವ-ಭೂಪಾಳ-ಕರ್
ವಿಶ್ವ-ವಿದ್ಯಾ-ನಿ-ಲದಯ
ವಿಶ್ವ-ವಿದ್ಯಾ-ನಿ-ಲ-ಯಕ್ಕೆ
ವಿಶ್ವ-ವಿದ್ಯಾ-ನಿ-ಲಯ-ಗಳ
ವಿಶ್ವ-ವಿದ್ಯಾ-ನಿ-ಲಯದ
ವಿಶ್ವ-ವಿದ್ಯಾ-ನಿ-ಲಯ-ದಲ್ಲಿ
ವಿಶ್ವ-ವಿದ್ಯಾ-ನಿ-ಲಯ-ದಿಂದ
ವಿಶ್ವ-ಸಂಗಳ
ವಿಶ್ವ-ಸಣ್ಣ
ವಿಶ್ವಾ-ಮಿತ್ರ
ವಿಶ್ವಾ-ವನಿ
ವಿಶ್ವಾಸಾರ್ಹವೂ
ವಿಶ್ವಾಸಾ-ವಾಸ-ವೇಶ್ಮನಃ
ವಿಶ್ವಾಸಿಕ
ವಿಶ್ವಾಸಿ-ಕ-ರಾಗಿ
ವಿಶ್ವೇಶ್ವರ
ವಿಶ್ವೇಶ್ವರ-ದೇವರ
ವಿಶ್ವೇಶ್ವರ-ದೇಶೇಶ್ವರ
ವಿಶ್ವೇಶ್ವರನ
ವಿಶ್ವೇಶ್ವರನು
ವಿಶ್ವೇಶ್ವರಯ್ಯ
ವಿಷಯ
ವಿಷಯ-ಕೆರೆ-ಗೋಡು-ನಾಡು
ವಿಷ-ಯಕ್ಕೆ
ವಿಷಯ-ಗಳ
ವಿಷಯ-ಗಳನ್ನಾಗಿ
ವಿಷಯ-ಗ-ಳನ್ನು
ವಿಷಯ-ಗ-ಳಾಗಿ
ವಿಷಯ-ಗ-ಳಾದ
ವಿಷಯ-ಗ-ಳಿಗೆ
ವಿಷಯ-ಗಳಿದ್ದ-ವೆಂದು
ವಿಷಯ-ಗಳಿವೆ
ವಿಷಯ-ಗಳು
ವಿಷಯ-ಗಳೂ
ವಿಷಯ-ಗಳೇ
ವಿಷಯ-ಗಳೊಂದಿಗೆ
ವಿಷಯದ
ವಿಷಯ-ದಲ್ಲಿ
ವಿಷಯ-ದಲ್ಲಿದ್ದ
ವಿಷಯ-ದಲ್ಲಿಯೇ
ವಿಷಯ-ದಲ್ಲಿ-ರುವ
ವಿಷಯ-ದಲ್ಲೂ
ವಿಷಯ-ವನ್ನಾಗಿ
ವಿಷಯ-ವನ್ನು
ವಿಷಯ-ವನ್ನೊಳ-ಗೊಂಡ
ವಿಷಯ-ವಾಗಲೀ
ವಿಷಯ-ವಾಗಿದೆ
ವಿಷಯ-ವಾಗಿ-ರ-ಬ-ಹುದು
ವಿಷಯ-ವಾಗುತ್ತ-ದೆಂದು
ವಿಷಯ-ವಿದೆ
ವಿಷಯವು
ವಿಷಯ-ವೆಂದರೆ
ವಿಷಯಾ-ಧಾರಿತ
ವಿಷಯಾಧೀಶ-ರನ್ನು
ವಿಷಯೋ
ವಿಷುವ-ಯನ
ವಿಷ್ಟ-ಪತ್ರಯ
ವಿಷ್ಣ-ವರ್ಧನನ
ವಿಷ್ಣ-ವರ್ಧನ-ನಿಗೆ
ವಿಷ್ಣ-ವಿಗೂ
ವಿಷ್ಣ-ವಿನ
ವಿಷ್ಣವುರ್ಧ-ನನ
ವಿಷ್ಣು
ವಿಷ್ಣು-ಗೋತ್ರದ
ವಿಷ್ಣು-ಚಮೂಪತಿ
ವಿಷ್ಣು-ದಂಡಾಧೀಶ
ವಿಷ್ಣು-ದಂಡಾಧೀಶ-ನನ್ನು
ವಿಷ್ಣು-ದಂಡಾಧೀಶನು
ವಿಷ್ಣು-ದಂಡಾಧೀಶನುಂ
ವಿಷ್ಣು-ದಂಡಾಧೀಶನೇ
ವಿಷ್ಣು-ದಂಡಾಧೀಶರುಃ
ವಿಷ್ಣು-ದೇವ
ವಿಷ್ಣು-ದೇವ-ನಿಗೆ
ವಿಷ್ಣು-ದೇವಾ-ಲಯ-ಗಳು
ವಿಷ್ಣು-ಪುರ
ವಿಷ್ಣು-ಪುರಾ-ಣಕ್ಕೆ
ವಿಷ್ಣು-ಪುರಾಣದ
ವಿಷ್ಣು-ಪೂಜೆ-ಯನ್ನು
ವಿಷ್ಣು-ಭಕ್ತಿ
ವಿಷ್ಣು-ಭಟ್ಟಂಗಳ-ಭಟ್ಟಯ್ಯ-ಗಳ
ವಿಷ್ಣು-ಭಟ್ಟನ
ವಿಷ್ಣು-ಭಟ್ಟಯ್ಯಂಗಳ
ವಿಷ್ಣು-ಭಟ್ಟಯ್ಯನ
ವಿಷ್ಣು-ಭಟ್ಟಯ್ಯನಿಗೆ
ವಿಷ್ಣು-ಭೂ-ಪನ
ವಿಷ್ಣು-ಭೂಪ-ನೊಳ್
ವಿಷ್ಣು-ರಾಯ
ವಿಷ್ಣು-ರಾಯ-ಮಹಾ-ರಾಯ-ನೆಂದರೆ
ವಿಷ್ಣು-ವನ್ನು
ವಿಷ್ಣು-ವರ್ಧನ
ವಿಷ್ಣು-ವರ್ಧನ-ದೇವ
ವಿಷ್ಣು-ವರ್ಧನ-ದೇವ-ರುದುಷ್ಟನಿಗ್ರಹ
ವಿಷ್ಣು-ವರ್ಧನನ
ವಿಷ್ಣು-ವರ್ಧನ-ನಂಥ
ವಿಷ್ಣು-ವರ್ಧನ-ನಿಂದ
ವಿಷ್ಣು-ವರ್ಧನ-ನಿಗೆ
ವಿಷ್ಣು-ವರ್ಧನನು
ವಿಷ್ಣು-ವರ್ಧನ-ನೆಂಬ
ವಿಷ್ಣು-ವರ್ಧನನೇ
ವಿಷ್ಣು-ವರ್ಧನನ್ನು
ವಿಷ್ಣು-ವರ್ಧನ-ಬೊಜ್ಜ
ವಿಷ್ಣು-ವರ್ಧನಿನ-ಗಾಗಿ
ವಿಷ್ಣು-ವರ್ಧನು
ವಿಷ್ಣು-ವರ್ಧ್ಧನ
ವಿಷ್ಣು-ವಿಗೆ
ವಿಷ್ಣು-ವಿಗ್ರಹ
ವಿಷ್ಣು-ವಿನ
ವಿಷ್ಣುವು
ವಿಷ್ಣು-ವೃದ್ಧೋ
ವಿಷ್ಣುವೇ
ವಿಸುದ್ಧ
ವಿಸ್ತ-ರಣೆ
ವಿಸ್ತ-ರಣೆ-ಗಳು
ವಿಸ್ತ-ರಣೆಗೆ
ವಿಸ್ತ-ರಣೆಯ
ವಿಸ್ತ-ರಣೆ-ಯನ್ನು
ವಿಸ್ತ-ರಣೆ-ಯಲ್ಲಿಯೂ
ವಿಸ್ತ-ರಣೆ-ಯಾಗಿದೆ
ವಿಸ್ತ-ರಣೆ-ಯಾಗಿರು
ವಿಸ್ತ-ರಣೆ-ಯಾಗಿ-ರುವ
ವಿಸ್ತ-ರಣೆ-ಯಾಗಿ-ರು-ವಂತೆ
ವಿಸ್ತ-ರಣೆಯು
ವಿಸ್ತರಿ-ಸದಾದ
ವಿಸ್ತರಿಸ-ಬ-ಹುದು
ವಿಸ್ತರಿಸಲ್ಪಟ್ಟ
ವಿಸ್ತರಿಸಿ
ವಿಸ್ತರಿಸಿತ್ತೆಂದು
ವಿಸ್ತರಿಸಿ-ದನು
ವಿಸ್ತರಿಸಿ-ದ-ನೆಂದೂ
ವಿಸ್ತರಿಸಿ-ದಾಗ
ವಿಸ್ತ-ರಿ-ಸಿದೆ
ವಿಸ್ತರಿಸಿ-ರ-ಬಹು-ದೆಂದು
ವಿಸ್ತರಿಸುತ್ತಾನೆ
ವಿಸ್ತ-ರಿ-ಸುವಾಗ
ವಿಸ್ತಾರ
ವಿಸ್ತಾರದ
ವಿಸ್ತಾರ-ವನ್ನು
ವಿಸ್ತಾರ-ವಾಗಿ
ವಿಸ್ತಾರ-ವಾ-ಗಿತ್ತು
ವಿಸ್ತಾರ-ವಾಗಿದೆ
ವಿಸ್ತಾರ-ವಾ-ಗಿದ್ದ
ವಿಸ್ತಾರ-ವಾದ
ವಿಸ್ತಾರ-ವಾ-ದಂತೆ
ವಿಸ್ತಾರ-ವಾ-ದಂತೆಲ್ಲಾ
ವಿಸ್ತೀರ್ಣ
ವಿಸ್ತೀರ್ಣ-ವನ್ನು
ವಿಸ್ತೀರ್ಣ-ವಿ-ರುವ
ವಿಸ್ತೃತ
ವಿಸ್ವ-ಕರ್ಮ
ವಿಸ್ಸಂಣ್ನಂಗಳು
ವಿಹಂಗ
ವಿಹಿತ
ವಿಹಿತಿ
ವೀಕ್ಷಣೆ
ವೀಡು
ವೀತರಾಗ
ವೀತ-ರಾಸಿಗೆ
ವೀತಾಯುಧವ್ರತ-ಧಾರಿ-ಯಾಗಿ
ವೀನಪಾಂಡಿಯನ್
ವೀನರ-ಸಿಂಹನ
ವೀಬಲ್ಲಾಳ
ವೀರ
ವೀರಂ
ವೀರಂಣ-ಗೌಡನು
ವೀರಂಣ-ನಾಯ-ಕರು
ವೀರ-ಅಚ್ಯುತ-ರಾಯ
ವೀರ-ಅಚ್ಯುತ-ರಾಯನ
ವೀರ-ಕಲ್ಲನ್ನು
ವೀರ-ಕಲ್ಲು
ವೀರ-ಕೀರ್ತ್ತಿಲ-ತಾಂಕುರಮಂ
ವೀರ-ಕೃಷ್ಣ-ರಾಯ-ಮಹಾ-ರಾಯ
ವೀರ-ಕೆಕ್ಕಾಯಿ
ವೀರ-ಕೇತೆಯ
ವೀರ-ಕೊಂಗಾಳ್ವ
ವೀರ-ಕೊಂಗಾಳ್ವ-ದೇವನು
ವೀರ-ಗಂಗ
ವೀರ-ಗಂಗ-ಪೆರ್ಮಾನ-ಡಿಯು
ವೀರ-ಗಜ-ಬೇಂಟೆ-ಕಾರ
ವೀರ-ಗಲ್ಲನ್ನು
ವೀರ-ಗಲ್ಲಾಗಿದೆ
ವೀರ-ಗಲ್ಲಾ-ಗಿದ್ದು
ವೀರ-ಗಲ್ಲಿಗೆ
ವೀರ-ಗಲ್ಲಿದೆ
ವೀರ-ಗಲ್ಲಿನ
ವೀರ-ಗಲ್ಲಿ-ನಲ್ಲಿ
ವೀರ-ಗಲ್ಲಿ-ನಲ್ಲಿದೆ
ವೀರ-ಗಲ್ಲಿ-ನಲ್ಲೂ
ವೀರ-ಗಲ್ಲಿ-ನಿಂದ
ವೀರ-ಗಲ್ಲು
ವೀರ-ಗಲ್ಲು-ಗಳ
ವೀರ-ಗಲ್ಲು-ಗ-ಳನ್ನು
ವೀರ-ಗಲ್ಲು-ಗಳಲ್ಲ
ವೀರ-ಗಲ್ಲು-ಗ-ಳಲ್ಲಿ
ವೀರ-ಗಲ್ಲು-ಗ-ಳಿಂದ
ವೀರ-ಗಲ್ಲು-ಗ-ಳಿಗೆ
ವೀರ-ಗಲ್ಲು-ಗಳಿ-ರುತ್ತವೆ
ವೀರ-ಗಲ್ಲು-ಗಳಿವೆ
ವೀರ-ಗಲ್ಲು-ಗಳು
ವೀರ-ಗಲ್ಲು-ಗಳೂ
ವೀರ-ಗಲ್ಲು-ಗ-ಳೆಂದು
ವೀರ-ಗಲ್ಲು-ಗ-ಳೆಂದೂ
ವೀರ-ಗಲ್ಲು-ಗಳೇ
ವೀರ-ಗಲ್ಲು-ಮಾಸ್ತಿ-ಕಲ್ಲು-ವೇಳೆ-ವಾಳಿ-ನಿ-ಸಿದಿ-ಗಲ್ಲು
ವೀರ-ಗಲ್ಲು-ಶಾ-ಸನ
ವೀರ-ಗಲ್ಲು-ಶಾ-ಸನ-ಗಳಿವೆ
ವೀರ-ಗಲ್ಲು-ಶಾ-ಸನ-ಗಳು
ವೀರ-ಗಲ್ಲು-ಶಾ-ಸನ-ದಿಂದ
ವೀರ-ಗಲ್ಲೂ
ವೀರ-ಗಲ್ಲೆಂದು
ವೀರ-ಗವುಂಡ
ವೀರ-ಗುಡಿ-ಗಳು
ವೀರ-ಗುಡಿಯ
ವೀರಗ್ರಾಣಿ-ಯಾಗಿದ್ದ-ನಂತೆ
ವೀರ-ಘಟನಗೆ-ಳನ್ನು
ವೀರ-ಚಿಕ-ರಾಯನು
ವೀರ-ಚಿಕ-ವೊಡ-ಯರ
ವೀರ-ಚಿಕ-ವೊಡೆ-ಯರ
ವೀರ-ಚಿಕ್ಕ
ವೀರ-ಚಿಕ್ಕ-ಕೇತಯ್ಯ
ವೀರ-ಚಿಕ್ಕ-ಕೇತೆಯ
ವೀರ-ಚಿಕ್ಕ-ಕೇತೆಯ್ಯನ
ವೀರ-ಚಿಕ್ಕ-ರಾಯ
ವೀರ-ಚಿಕ್ಕ-ರಾಯ-ನಿ-ಗಿಂತ
ವೀರ-ಚಿಕ್ಕ-ರಾಯನು
ವೀರ-ಚಿಕ್ಕ-ರಾಯನೇ
ವೀರ-ಚೋಳ
ವೀರ-ಜೀ-ವನ-ವನ್ನು
ವೀರ-ಣಂದಿ
ವೀರಣಃ
ವೀರ-ಣ-ನಾಯ-ಕರು
ವೀರ-ಣಾ-ಚಾರ್ಯ
ವೀರ-ಣಾ-ಚಾರ್ಯನ
ವೀರ-ಣಾ-ಚಾರ್ಯ-ನಿಗೆ
ವೀರ-ಣಾ-ಚಾರ್ಯ-ಸೂನುಃ
ವೀರ-ಣಾ-ಚಾರ್ಯ್ಯ
ವೀರ-ಣಾ-ಚಾರ್ಯ್ಯೋ
ವೀರ-ಣಾ-ಚಾರ್ಯ್ಯೋವ್ಯ
ವೀರಣ್ಣ
ವೀರಣ್ಣ-ಗೌಡ
ವೀರಣ್ಣನ
ವೀರಣ್ಣ-ನಾಯ-ಕ-ನಿಗೆ
ವೀರಣ್ಣ-ನಾಯ-ಕನು
ವೀರಣ್ಣ-ನಾಯ-ಕ-ನೆಂಬು-ವ-ವನು
ವೀರಣ್ಣೊಡೆಯ-ಗುರು-ಶಾಂತ-ವೀರಯ್ಯ-ನನ್ನು
ವೀರತಃ
ವೀರತ್ವ-ದಿಂದ
ವೀರತ್ವ-ವನ್ನು
ವೀರ-ದೇವನ
ವೀರ-ದೇವ-ನ-ಪುರ
ವೀರ-ದೇವ-ನ-ಹಳ್ಳಿ
ವೀರ-ದೇವ-ನ-ಹಳ್ಳಿಯ
ವೀರ-ದೇವನು
ವೀರ-ದೇವ-ರಾಯನ
ವೀರನ
ವೀರ-ನಂಜ-ರಾಜ
ವೀರ-ನಂಜ-ರಾಜೊಡೆಯರ
ವೀರ-ನಂಜ-ರಾಯ
ವೀರ-ನಂಜ-ರಾಯನ
ವೀರ-ನನ್ನು
ವೀರ-ನರ-ಪತಿ
ವೀರ-ನರ-ಸಿಂಹ
ವೀರ-ನರ-ಸಿಂಹನ
ವೀರ-ನರ-ಸಿಂಹ-ನಿಂದ
ವೀರ-ನರ-ಸಿಂಹನು
ವೀರ-ನರ-ಸಿಂಹ-ಪುರ-ವಾದ
ವೀರ-ನರ-ಸಿಂಹ-ರಾಯರ
ವೀರ-ನರ-ಸಿಂಹೇಂದ್ರ-ಪುರ
ವೀರ-ನರ-ಸಿಂಹೇಂದ್ರ-ಪುರ-ವಾದ
ವೀರ-ನರ-ಸಿಂಹೇಂದ್ರ-ಪುರ-ವೆಂಬ
ವೀರ-ನ-ಹಳ್ಳಿ
ವೀರ-ನಾ-ಗಿದ್ದು
ವೀರ-ನಾಗಿ-ರ-ಬ-ಹುದು
ವೀರ-ನಾಯಕ
ವೀರ-ನಾರ-ಸಿಂಹ
ವೀರ-ನಾರ-ಸಿಂಹಂಗೆ
ವೀರ-ನಾರ-ಸಿಂಹಗೆ
ವೀರ-ನಾರ-ಸಿಂಹ-ದೇವನ
ವೀರ-ನಾರ-ಸಿಂಹ-ದೇವನು
ವೀರ-ನಾರ-ಸಿಂಹ-ದೇವರ
ವೀರ-ನಾರ-ಸಿಂಹ-ದೇವ-ರ-ಸರ
ವೀರ-ನಾರ-ಸಿಂಹ-ದೇವ-ರ-ಸರು
ವೀರ-ನಾರ-ಸಿಂಹನ
ವೀರ-ನಾರ-ಸಿಂಹ-ನನ್ನು
ವೀರ-ನಾರ-ಸಿಂಹನು
ವೀರ-ನಾರ-ಸಿಂಹ-ಪುರ
ವೀರ-ನಾರ-ಸಿಂಹ-ಪುರ-ವಾದ
ವೀರ-ನಾ-ರಾಯಣ
ವೀರ-ನಾ-ರಾಯ-ಣದ
ವೀರ-ನಾ-ರಾಯ-ಣ-ದೇವ-ನೆಂಬ
ವೀರ-ನಾ-ರಾಯ-ಣ-ದೇವರ
ವೀರ-ನಾ-ರಾಯ-ಣ-ದೇ-ವ-ರಿಗೆ
ವೀರ-ನಿಗೆ
ವೀರ-ನಿದ್ದು
ವೀರ-ನೀರ-ಬ-ಹುದು
ವೀರನು
ವೀರನೂ
ವೀರ-ನೃ-ಸಿಂಹ
ವೀರ-ನೆಂದು
ವೀರ-ನೊಬ್ಬ-ನಿಗೆ
ವೀರ-ನೊಬ್ಬನು
ವೀರ-ಪಟ್ಟ
ವೀರ-ಪಟ್ಟಮಂ
ವೀರ-ಪಟ್ಟ-ವನ್ನು
ವೀರ-ಪಣ
ವೀರ-ಪನ
ವೀರ-ಪರಂಪ-ರೆಯ
ವೀರ-ಪಾಂಡ್ಯ
ವೀರ-ಪಾಂಡ್ಯನ
ವೀರ-ಪಾಂಡ್ಯ-ನನ್ನು
ವೀರ-ಪಿಳ್ಳನ
ವೀರ-ಪುತ್ರರ
ವೀರ-ಪುರುಷ-ನಾಗಿದ್ದನು
ವೀರ-ಪೆರ್ಮಾಡಿ
ವೀರ-ಪೆರ್ಮಾಡಿ-ದೇವನ
ವೀರಪ್ಪ
ವೀರಪ್ಪ-ಮಂತ್ರಿ
ವೀರಪ್ಪ-ವೊಡ-ಯರ
ವೀರಪ್ರತಾಪ
ವೀರಪ್ರತಾಪ-ದೇವ-ರಾಯನು
ವೀರ-ಬಂಕೆಯನ
ವೀರ-ಬಂಕೆಯನು
ವೀರ-ಬಣಂಜು
ವೀರ-ಬಣಂಜು-ಗಳು
ವೀರ-ಬಣಂಜು-ಧರ್ಮ
ವೀರ-ಬಮ್ಮಯ್ಯ
ವೀರ-ಬಮ್ಮಯ್ಯನ
ವೀರ-ಬಲ್ಲಾಳ
ವೀರ-ಬಲ್ಲಾಳ-ದೇವನ
ವೀರ-ಬಲ್ಲಾಳ-ದೇವ-ನಿಗೆ
ವೀರ-ಬಲ್ಲಾಳ-ದೇವನು
ವೀರ-ಬಲ್ಲಾಳ-ದೇವರ
ವೀರ-ಬಲ್ಲಾಳ-ದೇವ-ರ-ಸರ
ವೀರ-ಬಲ್ಲಾಳ-ದೇವ-ರ-ಸರು
ವೀರ-ಬಲ್ಲಾಳನ
ವೀರ-ಬಲ್ಲಾಳ-ನನ್ನು
ವೀರ-ಬಲ್ಲಾಳ-ನಲ್ಲಿ
ವೀರ-ಬಲ್ಲಾಳ-ನಿಗೂ
ವೀರ-ಬಲ್ಲಾಳ-ನಿಗೆ
ವೀರ-ಬಲ್ಲಾಳನು
ವೀರ-ಬಲ್ಲಾಳ-ಪುರ-ವನ್ನು
ವೀರ-ಬಲ್ಲಾಳ-ಪುರ-ವನ್ನೂ
ವೀರ-ಬಲ್ಲಾಳ-ರಾಯ
ವೀರ-ಬಲ್ಲಾಳು
ವೀರ-ಬಳಂಜು
ವೀರ-ಬಳಂಜು-ಧರ್ಮ
ವೀರ-ಬಳಂಜು-ಧರ್ಮಕ್ಕೆ
ವೀರ-ಬುಕ-ರಾಜ
ವೀರ-ಬುಕ್ಕ
ವೀರ-ಬುಕ್ಕಣ್ಣ
ವೀರ-ಬುಕ್ಕಣ್ಣೊಡೆ-ಯರ
ವೀರ-ಭಕ್ತನ
ವೀರ-ಭಟರಂ
ವೀರ-ಭಟಲಲಾಟ-ಪಟ್ಟಂ
ವೀರ-ಭಟಾ-ವಳಿ
ವೀರ-ಭದ್ರ
ವೀರ-ಭದ್ರ-ದುರ್ಗ
ವೀರ-ಭದ್ರ-ದುರ್ಗದ
ವೀರ-ಭದ್ರ-ದೇವರ
ವೀರ-ಭದ್ರ-ದೇವ-ರಿಗೆ
ವೀರ-ಭದ್ರ-ದೇವರು
ವೀರ-ಭದ್ರ-ದೇವಾ-ಲ-ಯದ
ವೀರ-ಭದ್ರ-ದೇವಾ-ಲಯ-ವನ್ನು
ವೀರ-ಭದ್ರನ
ವೀರ-ಭದ್ರ-ವೇಷ-ವನ್ನು
ವೀರ-ಭದ್ರಸ್ವಾಮಿ
ವೀರ-ಭದ್ರೇಶ್ವರ
ವೀರ-ಭುಜ-ಕಂದೈ-ಯರ
ವೀರ-ಭುಜಕ್ಕನ್ದೈ-ಯರ್
ವೀರಮಂ
ವೀರ-ಮಂಗಪ್ಪ
ವೀರ-ಮಕ್ಕಳು
ವೀರ-ಮಯ್ದುನ
ವೀರ-ಮರಣ
ವೀರ-ಮರ-ಣ-ಗಳ
ವೀರ-ಮರ-ಣ-ದಿಂದ
ವೀರ-ಮರ-ಣಸ್ಮಾರ-ಕ-ಗಳು
ವೀರ-ಮಲ್ಲಯ್ಯ
ವೀರ-ಮಲ್ಲಯ್ಯ-ನೆಂಬು-ವ-ವನು
ವೀರ-ಮಸ-ಣನು
ವೀರ-ಮಾಸ್ತಿ-ಕೆಂಪಮ್ಮನ
ವೀರಯ್ಯ
ವೀರಯ್ಯ-ಗಳು
ವೀರಯ್ಯ-ಗ-ವುಡನು
ವೀರಯ್ಯ-ದಂಡ-ನಾಯ-ಕನ
ವೀರಯ್ಯ-ದಂಡ-ನಾಯ-ಕನು
ವೀರಯ್ಯ-ನನ್ನು
ವೀರಯ್ಯನು
ವೀರರ
ವೀರ-ರ-ಗುಡಿ
ವೀರ-ರ-ಗುಡಿ-ಗಳ
ವೀರ-ರ-ಗುಡಿ-ಗಳು
ವೀರ-ರ-ಗುಡಿ-ಯನ್ನು
ವೀರ-ರ-ಗುಡಿ-ಯಲ್ಲಿರು
ವೀರ-ರ-ಗುಡಿಯು
ವೀರ-ರನ್ನಾ-ಗಲೀ
ವೀರ-ರಸ
ವೀರ-ರ-ಸರು
ವೀರ-ರಾಗಿದ್ದರು
ವೀರ-ರಾ-ಗಿದ್ದು
ವೀರ-ರಾಗಿ-ರುವು-ದ-ರಿಂದ
ವೀರ-ರಾಜನ
ವೀರ-ರಾಜ-ನಿಗೆ
ವೀರ-ರಾಜಯ್ಯನ
ವೀರ-ರಾಜೇಂದ್ರ
ವೀರ-ರಾಜೇಂದ್ರನ
ವೀರ-ರಾಜೇಂದ್ರ-ಹೊಯ್ಸಳ
ವೀರ-ರಾಜೈಯ್ಯ-ನ-ವರ
ವೀರ-ರಾಮ-ದೇವ
ವೀರ-ರಾಮ-ದೇವ-ರಾಯನ
ವೀರ-ರಾಮ-ದೇವ-ರಾಯ-ನಿಂದ
ವೀರ-ರಾಮ-ನಾಥನು
ವೀರ-ರಿಗೆ
ವೀರ-ರಿಗೆ-ಅ-ವರ
ವೀರರು
ವೀರ-ರುಮ್
ವೀರರೂ
ವೀರರೇ
ವೀರರ್ಕರು
ವೀರ-ಲಕ್ಷ್ಮೀ-ಭು-ಜಂಗ
ವೀರ-ಲಕ್ಷ್ಯಂಗ-ನೆಯ-ರಪ್ಪ
ವೀರ-ವಣ
ವೀರ-ವನ್ನು
ವೀರ-ವನ್ನೇ
ವೀರ-ವರೆಯೆತ್ತುವ
ವೀರ-ವರ್ತಕ
ವೀರ-ವಲ್ಲಾ-ಳನ್
ವೀರ-ವಾರಿಧಿ
ವೀರ-ವಿಜಯ-ರಾಯ
ವೀರ-ವಿಜಯ-ರಾಯನೇ
ವೀರ-ವಿ-ರೂಪಾಕ್ಷ-ನನ್ನು
ವೀರ-ವಿ-ರೂಪಾಕ್ಷನು
ವೀರ-ವಿ-ರೂಪಾಕ್ಷ-ಬಲ್ಲಾಳ
ವೀರ-ವಿಷ್ಣು-ವರ್ಧನ
ವೀರ-ವಿಷ್ಣು-ವರ್ಧನ-ದೇವ
ವೀರ-ವೆಂದೊಡೀ
ವೀರ-ವೈಷ್ಣವಿ
ವೀರ-ವೊಡೆ-ಯ-ನಿಗೆ
ವೀರವ್ರತ-ದಿಂದ
ವೀರ-ಶಾ-ಸನ
ವೀರ-ಶಾ-ಸನ-ಗಳು
ವೀರ-ಶಾ-ಸನ-ವನ್ನು
ವೀರ-ಶಾ-ಸನ-ವೆಂದರೆ
ವೀರ-ಶಾ-ಸನೋಕ್ತ-ವಾಗಿವೆ
ವೀರ-ಶೆಟ್ಟಿ-ಯುಮ್
ವೀರ-ಶೆಟ್ಟಿ-ಹಳ್ಳಿ
ವೀರ-ಶೈವ
ವೀರ-ಶೈವ-ಕೇಂದ್ರ-ವಾಗಿ
ವೀರ-ಶೈ-ವಕ್ಕೆ
ವೀರ-ಶೈವ-ಗುರು
ವೀರ-ಶೈವದ
ವೀರ-ಶೈವ-ದಲ್ಲಿ
ವೀರ-ಶೈವ-ಧರ್ಮ
ವೀರ-ಶೈವ-ಧರ್ಮಕ್ಕೂ
ವೀರ-ಶೈವ-ಧರ್ಮಕ್ಕೆ
ವೀರ-ಶೈವ-ಧರ್ಮದ
ವೀರ-ಶೈವ-ಧರ್ಮ-ದಲ್ಲಿ
ವೀರ-ಶೈವ-ಧರ್ಮ-ದ-ವ-ರೆಂಬುದ-ರಲ್ಲಿ
ವೀರ-ಶೈವ-ಧರ್ಮ-ವನ್ನು
ವೀರ-ಶೈವ-ಧರ್ಮವು
ವೀರ-ಶೈವ-ಧರ್ಮ-ವೆಂಬ
ವೀರ-ಶೈವ-ನಾಗಿ-ರ-ಬ-ಹುದು
ವೀರ-ಶೈವ-ಪೀಠ
ವೀರ-ಶೈವ-ಮತ-ಗಳ
ವೀರ-ಶೈವ-ಮತದ
ವೀರ-ಶೈವರ
ವೀರ-ಶೈವ-ರನ್ನೂ
ವೀರ-ಶೈವ-ರ-ವಾಗಿದ್ದ-ರಿಂದ
ವೀರ-ಶೈವ-ರಾಗಿ
ವೀರ-ಶೈವ-ರಿ-ಗಿದ್ದ
ವೀರ-ಶೈವ-ರಿಗೂ
ವೀರ-ಶೈವ-ರಿಗೆ
ವೀರ-ಶೈವರು
ವೀರ-ಶೈವರೂ
ವೀರ-ಶೈವ-ರೆಂದು
ವೀರ-ಶೈವರೇ
ವೀರ-ಶೈವ-ವಾಗಿದ್ದ-ರಿಂದ
ವೀರ-ಶೈವವು
ವೀರ-ಶೈವ-ಶಾ-ಸನ-ಗ-ಳಲ್ಲಿ
ವೀರ-ಶೈವ-ಶಾ-ಸನ-ಗಳು
ವೀರ-ಶೈ-ವಾಗ-ಮಜ್ಞ
ವೀರ-ಶೈವಾಮೃತ
ವೀರ-ಶೈವೀ-ಕರಣ
ವೀರ-ಶೋಳ
ವೀರಶ್ರೀ
ವೀರಶ್ರೀ-ದೇವನು
ವೀರಶ್ರೀ-ನಾರ-ಸಿಂಹೇಂದ್ರ-ಪುರ-ವಾದ
ವೀರಶ್ರೀಶ್ವರ
ವೀರಶ್ರೀಶ್ವರ-ದೇವ
ವೀರ-ಸಂಗ-ಮೇಶ್ವರ-ರಾಯ
ವೀರ-ಸ-ಮುದ್ರ-ವೆಂಬ
ವೀರ-ಸಾಮಂತ
ವೀರ-ಸಾ-ಸನ
ವೀರ-ಸಿಂಹಾಸನ
ವೀರ-ಸಿದ್ಧಿ-ವೆ-ರಸು
ವೀರ-ಸಿರಿ
ವೀರ-ಸೆಸೆ
ವೀರ-ಸೇವುಣರ
ವೀರ-ಸೇಸೆ
ವೀರ-ಸೋಮ-ನಾಥ-ಪುರದ
ವೀರ-ಸೋಮೆಯ-ನಾಯಕ
ವೀರ-ಸೋಮೇಶ್ವರ
ವೀರ-ಸೋಮೇಶ್ವರ-ದೇವನ
ವೀರ-ಸೋಮೇಶ್ವರ-ದೇವನು
ವೀರ-ಸೋಮೇಶ್ವರನ
ವೀರ-ಸೋಮೇಶ್ವರನು
ವೀರಸ್ತೋಮ
ವೀರಸ್ಥಂಭ-ಗ-ಳನ್ನು
ವೀರಸ್ಥಂಭ-ವೆಂದು
ವೀರಸ್ವರ್ಗ
ವೀರಸ್ವರ್ಗಸ್ಥ
ವೀರಸ್ವರ್ಗಸ್ಥ-ನಾಗುತ್ತಾನೆ
ವೀರಸ್ವರ್ಗ್ಗಮಂ
ವೀರ-ಹ-ನು-ಮಪ್ಪ
ವೀರ-ಹರಿ-ಯಪ್ಪ-ವೊಡೆ-ಯರು
ವೀರ-ಹರಿ-ಹರ
ವೀರ-ಹರಿ-ಹರ-ರಾಯನ
ವೀರ-ಹರಿ-ಹರ-ವೊಡೆ-ಯರ
ವೀರ-ಹರಿ-ಹರೇಶ್ವರ
ವೀರ-ಹರ್ಯಣ
ವೀರ-ಹರ್ಯಣನ
ವೀರಾಂಬಿ-ಕೆಯರ
ವೀರಾಂಬುಧಿ
ವೀರಾಗ್ರಣಿ
ವೀರಾ-ಚಾರಕ್ಕೆ
ವೀರಾ-ಚಾರ-ದ-ವರು
ವೀರಾ-ಚಾರ-ವನ್ನು
ವೀರಾ-ಚಾರಿ
ವೀರಾವೇಶ
ವೀರಾವೇಶ-ದಿಂದ
ವೀರಿ-ಶೆಟ್ಟಿ-ಹಳ್ಳಿ
ವೀರಿ-ಶೆಟ್ಟಿ-ಹಳ್ಳಿ-ಯನ್ನು
ವೀರು
ವೀರೇಶ್ವರ
ವೀರೊಡೆಯ-ನಿಗೆ
ವೀರೋ
ವೀರ್ಯ್ಯಸ್ತಚ್ಛಿಷ್ಯೋ
ವೀರ್ರ-ರುಂದ
ವೀರ್ರಿ-ರುಂದ
ವೀಳೆ-ಯಕ್ಕೆ
ವೀಳೆಯ-ವನ್ನು
ವೀಳ್ಯದೆ-ಲೆಯ
ವೀಳ್ಯದೆ-ಲೆ-ಯನ್ನು
ವೀಳ್ಳೆ-ಯದೆ-ಲೆ-ಗ-ಳನ್ನು
ವೀಸ
ವೀಸಿಗೆಯ
ವುಂಡಿಗೆ
ವುಂಡಿಗೆಯ
ವುಂಡಿಗೆ-ಯನ್ನು
ವುಂಬಳಿ-ಯಾಗಿ
ವುದನ್ನು
ವುದ-ರಲ್ಲಿ
ವುದ-ರಲ್ಲಿಯೇ
ವುದು-ಕ-ಸರಿ-ದೇವನ
ವುಪು
ವುಪ್ಪರ-ವಟ್ಟದ
ವುಳಿ-ಯದೆ
ವುಳ್ಳ
ವೂ
ವೂರ
ವೂರೊತ್ತು
ವೃಂದಾ-ವನ
ವೃಂದಾ-ವನಕ್ಕೆ
ವೃಂದಾ-ವ-ನದ
ವೃಂದಾ-ವನ-ದಲಿ
ವೃಂದಾವ-ನನ್ನು
ವೃಂದಾ-ವನ-ಮಠದ
ವೃಂದಾ-ವನವ
ವೃಂದಾ-ವನ-ವಿದೆ
ವೃಕ್ಷ-ಸಂಬಂಧಿ
ವೃಕ್ಷ-ಸಂಬಂಧಿ-ಯಾಗಿದೆ
ವೃತಿ-ಗ-ಳನ್ನು
ವೃತಿಯಂ
ವೃತಿ-ಯನ್ನು
ವೃತ್ತ-ಗ-ಳನ್ನು
ವೃತ್ತ-ಗ-ಳನ್ನೂ
ವೃತ್ತ-ಗ-ಳಿಂದ
ವೃತ್ತ-ಗಳು
ವೃತ್ತ-ಗಳೂ
ವೃತ್ತ-ದಲ್ಲಿ
ವೃತ್ತವು
ವೃತ್ತಾ-ಕಾರದ
ವೃತ್ತಿ
ವೃತ್ತಿಂ
ವೃತ್ತಿ-ಗಳ
ವೃತ್ತಿ-ಗಳನ್ನಾಗಿ
ವೃತ್ತಿ-ಗ-ಳನ್ನು
ವೃತ್ತಿ-ಗ-ಳನ್ನೂ
ವೃತ್ತಿ-ಗ-ಳಲ್ಲಿ
ವೃತ್ತಿ-ಗಳಾಗು-ವಂತೆ
ವೃತ್ತಿ-ಗ-ಳಿಗೆ
ವೃತ್ತಿ-ಗಳು
ವೃತ್ತಿಗೂ
ವೃತ್ತಿಗೆ
ವೃತ್ತಿ-ಗೌ-ರವ
ವೃತ್ತಿ-ತೆ-ರಿಗೆ
ವೃತ್ತಿ-ಧನ
ವೃತ್ತಿನಾ
ವೃತ್ತಿ-ನಾ-ಮ-ದಿಂದ
ವೃತ್ತಿ-ನಿರ-ತರು
ವೃತ್ತಿಪ್ರಾಪ್ತಿ
ವೃತ್ತಿ-ಮೇ-ಕಾಮೆಹಾಶ್ನುತೇ
ವೃತ್ತಿಯ
ವೃತ್ತಿ-ಯಂತೆ
ವೃತ್ತಿ-ಯ-ನಾಯಕ
ವೃತ್ತಿ-ಯ-ನಾಯ-ಕ-ನಾಗಿದ್ದನು
ವೃತ್ತಿ-ಯ-ನಾಯ-ಕ-ನಾಗಿದ್ದ-ನೆಂದು
ವೃತ್ತಿ-ಯನ್ನಾಗಿ
ವೃತ್ತಿ-ಯನ್ನಾಗಿ-ಸಿ-ಕೊಂಡು
ವೃತ್ತಿ-ಯನ್ನು
ವೃತ್ತಿ-ಯ-ಮೇಲೆ
ವೃತ್ತಿ-ಯಲ್ಲಿ
ವೃತ್ತಿ-ಯ-ವರು
ವೃತ್ತಿ-ಯವ್ರಿತ್ತಿಯ
ವೃತ್ತಿ-ಯ-ಹುಲ್ಲವಂಗಲ
ವೃತ್ತಿ-ಯಾಗಿ
ವೃತ್ತಿ-ಯಾ-ಗುಳ್ಳ
ವೃತ್ತಿ-ಯಿಂದ
ವೃತ್ತಿಯು
ವೃತ್ತಿಯೂ
ವೃತ್ತಿ-ರೇ-ಕಾಮ-ವಾಪ್ತ-ವಾನ್
ವೃತ್ತಿ-ವಂತರ
ವೃತ್ತಿ-ವಂತ-ರಾದ
ವೃತ್ತಿ-ವಂತರು
ವೃತ್ತಿ-ವಂತ-ರು-ಗಳ
ವೃತ್ತಿ-ವಿ-ಶೇಷ-ಣ-ದಿಂದಲೇ
ವೃತ್ತಿ-ವೇ-ತನ-ದಿಂದ
ವೃದ್ಧರು
ವೃದ್ಧಿ
ವೃದ್ಧಿ-ಗಂತವಾ-ಗುತ್ತ-ಲಿ-ರುವ
ವೃದ್ಧಿ-ಸಿ-ಕೊಂಡನು
ವೃದ್ಧ್ಯರ್ತ್ಥ-ವಾಗಿ
ವೃದ್ಧ್ಯರ್ಥ-ವಾಗಿ
ವೃಷಭ-ನಾಥ
ವೆಂಕಟ
ವೆಂಕಟ-ಕೃಷ್ಣ
ವೆಂಕಟ-ಕೃಷ್ಣ-ರ-ವರು
ವೆಂಕಟನ
ವೆಂಕಟ-ನನ್ನು
ವೆಂಕಟ-ನಿಗೆ
ವೆಂಕಟ-ಪತಯ್ಯ
ವೆಂಕಟ-ಪತಿ
ವೆಂಕಟ-ಪ-ತಿಗೆ
ವೆಂಕಟ-ಪತಿ-ಮಹಾ-ರಾಯ
ವೆಂಕಟ-ಪತಿ-ಮಹಾ-ರಾಯನ
ವೆಂಕಟ-ಪತಿ-ಮಹಾ-ರಾಯರ
ವೆಂಕಟ-ಪತಿ-ಯಾರನ
ವೆಂಕಟ-ಪ-ತಿಯು
ವೆಂಕಟ-ಪ-ತಿಯೇ
ವೆಂಕಟ-ಪತಿ-ರಾಯ
ವೆಂಕಟ-ಪತಿ-ರಾಯ-ದೇವ
ವೆಂಕಟ-ಪತಿ-ರಾಯನ
ವೆಂಕಟ-ಪತಿ-ರಾಯನು
ವೆಂಕಟ-ಪತಿ-ರಾಯರ
ವೆಂಕಟಪ್ಪ-ನಾಯ-ಕನು
ವೆಂಕಟಪ್ಪನು
ವೆಂಕಟ-ರತ್ನಂ
ವೆಂಕಟ-ರತ್ನಮ್
ವೆಂಕಟ-ರಮಣ
ವೆಂಕಟ-ರಮಣನ
ವೆಂಕಟ-ರಮಣಯ್ಯ-ನ-ವರು
ವೆಂಕಟ-ರಮಣಶ್ರೀ-ನಿ-ವಾಸ
ವೆಂಕಟ-ರಮಣಸ್ವಾಮಿ
ವೆಂಕಟ-ರಮಣಸ್ವಾಮಿಯ
ವೆಂಕಟ-ರಾವ್
ವೆಂಕಟ-ಲಕ್ಷ್ಮಮ್ಮನು
ವೆಂಕಟ-ವರ-ದಾ-ಚಾರ್ಯ
ವೆಂಕಟ-ವರ-ದಾ-ಚಾರ್ಯ-ನಿಗೆ
ವೆಂಕಟ-ವರ-ದಾರ್ಯಾಯ
ವೆಂಕಟಾಚಲ-ಶಾಸ್ತ್ರಿ-ಗಳ
ವೆಂಕಟಾ-ಚಾರ್ಯರ
ವೆಂಕಟಾದ್ರಿ
ವೆಂಕಟಾದ್ರಿಗೆ
ವೆಂಕಟಾದ್ರಿ-ನಾಯಕ
ವೆಂಕಟಾದ್ರಿ-ನಾಯ-ಕ-ನಿಂದ
ವೆಂಕಟಾದ್ರಿ-ನಾಯ-ಕ-ನಿಗೆ
ವೆಂಕಟಾದ್ರಿ-ನಾಯ-ಕನು
ವೆಂಕಟಾದ್ರಿ-ನಾಯ-ಕನೂ
ವೆಂಕಟಾದ್ರಿಯ
ವೆಂಕಟಾದ್ರಿಯು
ವೆಂಕಟಾದ್ರಿ-ಸ-ಮುದ್ರ
ವೆಂಕಟಾದ್ರಿ-ಸ-ಮುದ್ರ-ವಾದ
ವೆಂಕಟಾದ್ರಿ-ಸ-ಮುದ್ರ-ವೆಂಬ
ವೆಂಕಟಾದ್ರೀಶನ
ವೆಂಕಟಾದ್ರೀಶ-ನಾಯ-ಕಸ್ಯ
ವೆಂಕಟಾರ್ಯ
ವೆಂಕಟೇಶ
ವೆಂಕಟೇಶ-ಭಟ್ಟರು
ವೆಂಕಟೇಶ್
ವೆಂಗಟ-ಪತಯ್ಯ
ವೆಂಗಟ-ಪನು
ವೆಂಗಟಪ್ಪನು
ವೆಂಗಟ-ರಮಣಾ-ಚಾರ್ರಿಯು
ವೆಂಗಳ-ರಾಜಯ್ಯನು
ವೆಂಗಿ-ಮಂಡಲ
ವೆಂಗೇನ-ಹಳ್ಳಿ-ಗ-ಳನ್ನು
ವೆಂಗೇನ-ಹಳ್ಳಿ-ಯನ್ನೂ
ವೆಂಗೇನ-ಹಳ್ಳಿಯು
ವೆಂಜಿ-ಮಲೈ
ವೆಂಟೇಶ್
ವೆಂದರೆ
ವೆಂದು
ವೆಂಬ
ವೆಚ್ಚಕ್ಕಾಗಿ
ವೆಚ್ಚಕ್ಕೆ
ವೆಚ್ಚ-ವನ್ನು
ವೆಟ್ಟದುಳ್
ವೆಣ್ಣೈಕೂತ್ತ
ವೆಣ್ಣೈಕೂತ್ತ-ಪಿಳ್ಳೆ
ವೆಣ್ಣೈಕೂತ್ತ-ಭಟ್ಟನ್
ವೆಣ್ಣೈಕೂತ್ತ-ಭಟ್ಟರ್
ವೆಲುಗೊಡ
ವೆಲ್ಲೂರು-ಬೆಳ್ಳೂರು
ವೆಲ್ಲೆಸ್ಲಿಗೆ
ವೆಲ್ಲೆಸ್ಲಿಯು
ವೆಲ್ಲೆಸ್ಲಿ-ಯೊಂದಿ
ವೇಂಕಟ-ನಿಗೆ
ವೇಂಕಟಾದ್ರಿ-ನಾಯ-ಕನು
ವೇಂಕಟಾದ್ರೀಶ
ವೇಂಟೆ
ವೇಂಟೆಯ
ವೇಂಟೆ-ಯದ
ವೇಂಟೆ-ಯ-ದಲ್ಲಿ
ವೇಂಟೆ-ಯ-ದೊಳಗೆ
ವೇಂಠಕ
ವೇಂಠೆ
ವೇಂಠೆಯ
ವೇಂಠೆ-ಯಕ್ಕೆ
ವೇಂಠೆ-ಯದ
ವೇಂಠೆ-ಯ-ದಲ್ಲಿ
ವೇಂಠೆ-ಯ-ಮಾ-ಗಣಿ-ವಳಿತ
ವೇಗ-ಮಂಗಲ-ಇಂದಿನ
ವೇಗ-ವಾಗಿ
ವೇಣು-ಗೋ-ಪಾಲ
ವೇಣು-ಗೋ-ಪಾಲನ
ವೇಣುಗ್ರಾಮ-ದಲ್ಲಿದ್ದ-ನೆಂದು
ವೇತನ
ವೇತನ-ಗ-ಳನ್ನು
ವೇದ
ವೇದಂತಿ
ವೇದ-ಗ-ಳಲ್ಲಿ
ವೇದ-ಗಳು
ವೇದ-ಗುರುವೂ
ವೇದತ್ರಯ-ಬೋ-ಧನಾ
ವೇದ-ಪಾಠ-ಶಾಲೆ-ಯನ್ನು
ವೇದ-ಪಾ-ರಂಗ-ತರು
ವೇದ-ಪಾರಾಂಗತ-ರಾದ
ವೇದ-ಪುಷ್ಕರಣಿ
ವೇದ-ಪುಷ್ಕರ-ಣಿ-ಯನ್ನು
ವೇದ-ಮಾರ್ಗ
ವೇದ-ಮಾರ್ಗಪ್ರತಿಷ್ಠಾ-ಚಾರ್ಯ
ವೇದ-ಯುಗ-ದಲ್ಲಿ
ವೇದ-ವನ್ನು
ವೇದ-ವಲ್ಲಿ
ವೇದ-ವೇದಾಂಗ
ವೇದ-ವೇದಿನೇ
ವೇದ-ಶಾಸ್ತ್ರ
ವೇದ-ಶಾಸ್ತ್ರಾರ್ಥ
ವೇದ-ಸಂಪನ್ನ-ರಾದ
ವೇದಾಂತ
ವೇದಾಂತದ
ವೇದಾಂತಾ-ಚಾರ್ಯ
ವೇದಾಂತಾ-ಚಾರ್ಯ-ರಾದ
ವೇದಾಂತಿ
ವೇದಾದ್ರಿ
ವೇದಾಧ್ಯಯನ
ವೇದಾಧ್ಯಾಯಿ-ಗ-ಳಾದ
ವೇದಾರಣ್ಯ
ವೇದಾರಣ್ಯ-ವೆಂದೂ
ವೇದಾರಣ್ಯ-ವೆಂಬ
ವೇದಿ-ಕೆಯ
ವೇದೋಪಾಧ್ಯಾಯ-ರಾದ
ವೇಲಾ-ಕಾರೇಶ್ವರ
ವೇಳ
ವೇಳಗೆ
ವೇಳ-ಬೇಳ
ವೇಳಯ
ವೇಳರು
ವೇಳ-ವಡಿ
ವೇಳ-ವಳ
ವೇಳ-ವಳಿ-ಯಾಗಿ
ವೇಳ-ವಾಳಿ
ವೇಳ-ವಾಳಿ-ತನವೂ
ವೇಳ-ವಾಳಿಯ
ವೇಳ-ವಾಳಿ-ಲೆಂಕ-ವಾಳಿ-ಗರು-ಡರು
ವೇಳಾ-ಪುರಂ
ವೇಳೆ
ವೇಳೆ-ಕಾರ
ವೇಳೆ-ಕಾರೇಶ್ವರ
ವೇಳೆ-ಗಾ-ಗಲೇ
ವೇಳೆ-ಗಾಲೇ
ವೇಳೆಗೆ
ವೇಳೆ-ಗೊಣ್ಡು
ವೇಳೆಯ
ವೇಳೆ-ಯದ
ವೇಳೆ-ಯ-ದಲ್ಲಿ
ವೇಳೆ-ಯಲ್ಲಿ
ವೇಳೆ-ವಾಳಿ
ವೇಳೆ-ವಾಳಿ-ಗಳ
ವೇಳೆ-ವಾಳಿ-ಗ-ಳಿಗೆ
ವೇಳೆ-ವಾಳಿ-ಗಳು
ವೇಳೆ-ವಾಳಿಯ
ವೇಳೆ-ವಾಳಿ-ಯವಂ
ವೇಳೆ-ವಾಳಿ-ಯಾಗಿ
ವೇಳೆ-ವಾಳಿ-ಯಾ-ಗಿದ್ದ
ವೇಳೆ-ವಾಳಿ-ಯೊಳಿರ್ಪನಾ
ವೇಳೈ-ಕಾರೀಶ್ವರ
ವೇಶ್ಯೆ-ಯರು
ವೈ
ವೈಕುಂಠ
ವೈಕುಂಠಕ್ಷೇತ್ರವು
ವೈಕುಂಠದ
ವೈಕುಂಠ-ನಾಥನ
ವೈಕುಂಠ-ವರ್ಧನ
ವೈಕುಂಠ-ವಾಸಿ-ಯಾಗಿದ್ದ-ಳೆಂದು
ವೈಕುಂಠೇ
ವೈಚೋ-ಜಂಗೆಯ್ದ
ವೈಜದ್ಯ-ನಾಥ
ವೈಜ-ನಾಥ
ವೈಜ-ನಾಥಂಗೊಲವಿಂ
ವೈಜ-ನಾಥ-ದೇವರ
ವೈಜ-ನಾಥ-ದೇವ-ರಿಗೆ
ವೈಜ-ನಾಥ-ದೇವರು
ವೈಜ-ನಾಥನೀ
ವೈಜ-ನಾಥ-ಪುರ
ವೈಜ-ನಾಥ-ಪುರದ
ವೈಜಾಂಡ-ರಾದ
ವೈಜ್ಞಾನಿ-ಕ-ವಾಗಿಲ್ಲ
ವೈಜ್ಯ-ನಾಥ
ವೈಜ್ಯ-ನಾಥ-ದೇವ-ರಿಗೆ
ವೈಣಿಕಶ್ರೇಣಿ
ವೈದಿಕ
ವೈದಿಕ-ಧರ್ಮ-ವನ್ನು
ವೈದಿಕ-ಧರ್ಮವು
ವೈದಿಕ-ನಾದ
ವೈದಿಕರ
ವೈದಿಕ-ರಾ-ಗಿದ್ದು
ವೈದಿಕ-ರಿಗೆ
ವೈದಿಕರು
ವೈದಿಕರೂ
ವೈದ್ಯ-ನಾಥ
ವೈದ್ಯ-ನಾಥ-ದೇವ-ರಿಗೆ
ವೈದ್ಯ-ನಾಥನ
ವೈದ್ಯ-ನಾಥ-ನಿಗೆ
ವೈದ್ಯ-ನಾಥ-ಪುರ
ವೈದ್ಯ-ನಾಥ-ಪುರಕ್ಕೆ
ವೈದ್ಯ-ನಾಥ-ಪುರದ
ವೈದ್ಯ-ನಾಥ-ಪುರ-ದಲ್ಲಿ-ರುವ
ವೈದ್ಯ-ನಾಥ-ಮುಡೆ-ಯಾರ್
ವೈದ್ಯನು
ವೈಭೋ-ಗಕ್ಕೆ
ವೈಭೋಗ-ವುಳ್ಳವ
ವೈಮನಸ್ಯ
ವೈಯಕ್ತಿಕ
ವೈರತ್ವ-ವಿದ್ದಾಗ
ವೈರತ್ವವು
ವೈರ-ಮುಡಿ
ವೈರ-ಮುಡಿಯ
ವೈರವು
ವೈರಿಗಳ
ವೈರಿ-ಗ-ಳನ್ನು
ವೈರಿ-ದಿಕ್ಕುಂಜ-ರರುಂ
ವೈರಿ-ಮಂಡ-ಳಿಕ
ವೈರಿ-ಮದ-ಮರ್ದ್ಧನ
ವೈರಿ-ರಾಜರ
ವೈರಿಸಂಹಾರ
ವೈರಿಸಂಹಾರ-ವನ್ನು
ವೈರಿ-ಸಮೂಹ-ಮಿಲ್ಲಿ
ವೈರಿ-ಸಾಮಂತ
ವೈರಿ-ಸಾಮಂತ-ರೆಂಬ
ವೈರಿ-ಸೇನೆಯು
ವೈವಾ
ವೈವಾಗೆ
ವೈವಾಹಿ
ವೈವಾ-ಹಿಕ
ವೈವಿಧ್ಯತೆ-ಗ-ಳನ್ನು
ವೈವಿಧ್ಯ-ಮಯ
ವೈವಿಧ್ಯ-ಮಯ-ವಾಗಿ
ವೈಶಾಖ
ವೈಶಾಖ-ಮಾಸ-ದಲು
ವೈಶಾಖೋತ್ಸವ
ವೈಶಿಷ್ಟ್ಯ
ವೈಶ್ಯ
ವೈಶ್ಯಾನ್ವಯ
ವೈಷಮ್ಯ
ವೈಷಮ್ಯ-ವನ್ನು
ವೈಷ್ಣ
ವೈಷ್ಣವ
ವೈಷ್ಣ-ವ-ಕೇಂದ್ರ
ವೈಷ್ಣ-ವ-ಕೇಂದ್ರ-ಗಳಾಗ-ದವು
ವೈಷ್ಣ-ವ-ಕೇಂದ್ರ-ಗಳು
ವೈಷ್ಣ-ವ-ಕೇಂದ್ರ-ವನ್ನಾಗಿ
ವೈಷ್ಣ-ವ-ಕೇಂದ್ರ-ವಾಗಿ
ವೈಷ್ಣ-ವ-ಕೇಂದ್ರ-ವಾ-ಗಿತ್ತು
ವೈಷ್ಣ-ವಕ್ಷೇತ್ರ-ವಾಗಿದೆ
ವೈಷ್ಣ-ವಕ್ಷೇತ್ರ-ವಾ-ಗಿದ್ದು
ವೈಷ್ಣ-ವಕ್ಷೇತ್ರ-ವಾಯಿ-ತೆಂದು
ವೈಷ್ಣ-ವ-ಗುರು
ವೈಷ್ಣ-ವ-ದಾಸರ
ವೈಷ್ಣ-ವ-ದೇವ-ರಿಗೆ
ವೈಷ್ಣ-ವ-ದೇವಾ-ಲ-ಗಳ
ವೈಷ್ಣ-ವ-ದೇವಾ-ಲಯ-ಗಳ
ವೈಷ್ಣ-ವ-ದೇವಾ-ಲಯ-ಗ-ಳಿಗೆ
ವೈಷ್ಣ-ವ-ದೇವಾ-ಲಯ-ಗಳು
ವೈಷ್ಣ-ವ-ದೇವಾ-ಲಯ-ವಿತ್ತು
ವೈಷ್ಣ-ವ-ಧರ್ಮ
ವೈಷ್ಣ-ವ-ಧರ್ಮದ
ವೈಷ್ಣ-ವ-ಧರ್ಮ-ದಲ್ಲಿ
ವೈಷ್ಣ-ವ-ಧರ್ಮ-ವನ್ನು
ವೈಷ್ಣ-ವ-ಧರ್ಮವು
ವೈಷ್ಣ-ವ-ಧರ್ಮ-ವೆಂಬುದು
ವೈಷ್ಣ-ವ-ಧರ್ಮಾನುಯಾಯಿ-ಗ-ಳಾದ
ವೈಷ್ಣ-ವ-ನಾಡು-ಗ-ಳಲ್ಲಿ
ವೈಷ್ಣ-ವ-ಪರಂಪ-ರೆಯ
ವೈಷ್ಣ-ವಪ್ರಿಯನ್
ವೈಷ್ಣ-ವಬ್ರಾಹ್ಮಣರು
ವೈಷ್ಣ-ವ-ಭಕ್ತ
ವೈಷ್ಣ-ವ-ಭಕ್ತರು
ವೈಷ್ಣ-ವ-ಮಠವೇ
ವೈಷ್ಣ-ವ-ಮತ
ವೈಷ್ಣ-ವ-ಮಹಾ-ಜನ
ವೈಷ್ಣ-ವ-ಮಹಾ-ಜನ-ಗಳು
ವೈಷ್ಣ-ವ-ಮಹಾ-ಜ-ನರ
ವೈಷ್ಣ-ವ-ಯತಿ
ವೈಷ್ಣ-ವರ
ವೈಷ್ಣ-ವ-ರನ್ನು
ವೈಷ್ಣ-ವ-ರಲ್ಲಿ
ವೈಷ್ಣ-ವ-ರಾ-ಗಿದ್ದರು
ವೈಷ್ಣ-ವ-ರಿ-ಗಾಗಿಯೆ
ವೈಷ್ಣ-ವ-ರಿಗೆ
ವೈಷ್ಣ-ವರು
ವೈಷ್ಣ-ವ-ರು-ಗ-ಳಲ್ಲಿ
ವೈಷ್ಣ-ವರೂ
ವೈಷ್ಣ-ವ-ಸಂಪ್ರ-ದಾಯದ
ವೈಷ್ಣ-ವಾಗ್ರ
ವೈಷ್ಣ-ವಾನ್
ವೈಷ್ಣ-ವಾನ್ಸ್ತಾಂಶ್ಚ
ವೈಷ್ಣ-ವೇಭ್ಯೋನ್ಯ-ವೇದ-ಯತ್
ವೈಸಿ
ವೈಸ್ರಾಯ್
ವೊಂದಕೆ
ವೊಂದು
ವೊಂದು-ಗದ್ಯಾಣ
ವೊಂದೆ
ವೊಂದೆತ್ತಿನ
ವೊಂದೆತ್ತು
ವೊಕ್ಕಲು
ವೊಕ್ಕುಳ
ವೊಗೆಯ
ವೊಗೆಯ-ಸ-ಮುದ್ರ
ವೊಡ-ವನ್ನು
ವೊಡವು
ವೊಡವು-ಗ-ಳನ್ನು
ವೊಡವುಟ್ಟಿದ
ವೊಡವೆ
ವೊಡ-ಸಂದ
ವೊಡ-ಸಂದರು
ವೊಡೆ
ವೊಡೆ-ತನ
ವೊಡೆ-ಯ-ಅಪ್ಪಣ್ಣ
ವೊಡೆ-ಯ-ಗೌಡನು
ವೊಡೆ-ಯ-ನಿಗೆ
ವೊಡೆ-ಯರ
ವೊಡೆ-ಯರ-ಕೂಡೆ
ವೊಡೆ-ಯ-ರಿಗೆ
ವೊಡೆ-ಯರು
ವೊಡೆ-ಯಾರ
ವೊಡೇರ
ವೊತ್ತು-ಕೊಂಡು
ವೊತ್ತೆ
ವೊಪ
ವೊಪ್ಪ
ವೊಪ್ಪ-ಣದಿ
ವೊಪ್ಪ-ಣಾದಿ
ವೊಪ್ಪದ
ವೊಪ್ಪ-ವನ್ನು
ವೊಪ್ಪವೂ
ವೊಪ್ಪಿತ-ವಾಗಿ
ವೊಪ್ಪಿಸಿ
ವೊಮ್ಮಯ್ಯಮ್ಮ
ವೊಮ್ಮವ್ವೆ
ವೊಮ್ಮಾ-ಯಮ್ಮ
ವೊಳಗಾದ
ವೊಳಗೆ
ವೊಳಗೆ-ರೆಯ
ವೊಳ-ವಾರು
ವೊಳ-ವಾರು-ವಾಗಿ
ವೊಹಳವ
ವೋಜ-ಮಂಗಲ
ವೋಡೆ
ವೋಣ-ಮಯ್ಯನ
ವೋಣ-ಮಯ್ಯ-ನೆಂದು
ವೋತ್ತಮ
ವೋದ-ಕುಳಿಯಿಂ
ವೋಲಗಿಸುತ್ತಿದ್ದನು
ವ್ಮೆಚ್ಚಿ
ವ್ಯಕ್ತ-ಗೊಳಿ-ಸಿದ್ದಾರೆ
ವ್ಯಕ್ತ-ಪಡಿ-ಸಿದ್ದರೂ
ವ್ಯಕ್ತ-ಪಡಿ-ಸಿದ್ದಾರೆ
ವ್ಯಕ್ತ-ಪಡಿ-ಸಿರುವ-ರಲ್ಲದೆ
ವ್ಯಕ್ತ-ಪಡಿ-ಸುತ್ತದೆ
ವ್ಯಕ್ತ-ವಾಗಿ-ರುವ
ವ್ಯಕ್ತ-ವಾಗುತ್ತದೆ
ವ್ಯಕ್ತ-ವಾಗುತ್ತ-ದೆಂದು
ವ್ಯಕ್ತ-ವಾಗು-ವುದು
ವ್ಯಕ್ತಿ
ವ್ಯಕ್ತಿ-ಗಳ
ವ್ಯಕ್ತಿ-ಗ-ಳಿಂದ
ವ್ಯಕ್ತಿ-ಗ-ಳಿಗೆ
ವ್ಯಕ್ತಿ-ಗಳು
ವ್ಯಕ್ತಿ-ಗಿಂತಲೂ
ವ್ಯಕ್ತಿಗೆ
ವ್ಯಕ್ತಿತ್ವ
ವ್ಯಕ್ತಿತ್ವದ
ವ್ಯಕ್ತಿ-ನಾಮ
ವ್ಯಕ್ತಿ-ನಾಮ-ವಾಗಿದೆ
ವ್ಯಕ್ತಿಯ
ವ್ಯಕ್ತಿ-ಯನ್ನು
ವ್ಯಕ್ತಿ-ಯಾಗಿ-ರ-ಬ-ಹುದು
ವ್ಯಕ್ತಿಯು
ವ್ಯಕ್ತಿ-ಯೊಬ್ಬ-ನಿಗೆ
ವ್ಯಕ್ತಿ-ಯೊಬ್ಬ-ರಿಂದ
ವ್ಯತಾನೀತ್ತಾಂಬ್ರ-ಶಾ-ಸನ
ವ್ಯತ್ಯಯ-ವಿಲ್ಲದೆ
ವ್ಯತ್ಯಾಸ
ವ್ಯತ್ಯಾಸ-ಗಳಿ-ರ-ಲಿಲ್ಲ
ವ್ಯತ್ಯಾಸ-ಗಳೊಂದಿಗೆ
ವ್ಯತ್ಯಾಸ-ಗಳೊಡನೆ
ವ್ಯತ್ಯಾಸ-ವನ್ನು
ವ್ಯತ್ಯಾಸ-ವೆನ್ನ-ಬ-ಹುದು
ವ್ಯತ್ಯಾಸ-ವೇನೂ
ವ್ಯತ್ಯಾಸವೋ
ವ್ಯಯ-ಮಾಡಿ
ವ್ಯಯಿಸಿ
ವ್ಯಲಿ-ಖತ್ತಾಮ್ರ-ಶಾ-ಸನಂ
ವ್ಯಲಿಖಿತ್ತಾಮ್ರ
ವ್ಯವ-ಸಾಯ
ವ್ಯವ-ಸಾ-ಯಕ್ಕೆ
ವ್ಯವಸ್ಥಾಪಕ-ರಾದ
ವ್ಯವಸ್ಥಿತಂ
ವ್ಯವಸ್ಥಿತ-ವಾಗಿ
ವ್ಯವಸ್ಥೆ
ವ್ಯವಸ್ಥೆ-ಗ-ಳನ್ನು
ವ್ಯವಸ್ಥೆ-ಗಾಗಿ
ವ್ಯವಸ್ಥೆಗೂ
ವ್ಯವಸ್ಥೆಗೆ
ವ್ಯವಸ್ಥೆ-ಗೊ-ಳಿಸಿ
ವ್ಯವಸ್ಥೆಯ
ವ್ಯವಸ್ಥೆ-ಯನ್ನು
ವ್ಯವಸ್ಥೆ-ಯಲ್ಲಿ
ವ್ಯವಸ್ಥೆ-ಯಲ್ಲೂ
ವ್ಯವಸ್ಥೆ-ಯಿರ-ಲಿಲ್ಲ
ವ್ಯವಸ್ಥೆಯು
ವ್ಯವಸ್ಥೆಯೂ
ವ್ಯವಸ್ಥೆಯೇ
ವ್ಯವ-ಹಾರ
ವ್ಯವ-ಹಾರಕ್ಕೆ
ವ್ಯವ-ಹಾರ-ಗ-ಳನ್ನು
ವ್ಯವ-ಹಾರ-ಗ-ಳಲ್ಲಿ
ವ್ಯವ-ಹಾರ-ಗಳಲ್ಲೇ
ವ್ಯವ-ಹಾರ-ಗಳಷ್ಟೇ
ವ್ಯವ-ಹಾರ-ಗಳಿ-ಗಾಗಿ
ವ್ಯವ-ಹಾರ-ಗ-ಳಿಗೆ
ವ್ಯವ-ಹಾರ-ಗಳು
ವ್ಯವ-ಹಾರ-ದಿಂದ
ವ್ಯವ-ಹಾರ-ಪೂಜೆ-ಪುನಸ್ಕಾರ-ದತ್ತಿ-ಯಾಗಿ
ವ್ಯವ-ಹಾರ-ವನ್ನು
ವ್ಯವ-ಹಾರಿ
ವ್ಯವ-ಹಾರಿ-ಗಳು
ವ್ಯಾಕರಣ
ವ್ಯಾಕರ-ಣಕ್ಕೆ
ವ್ಯಾಕರ-ಣದ
ವ್ಯಾಖ್ಯಾತಾಖಿಲ
ವ್ಯಾಖ್ಯಾನಿ-ಸಿದ್ದಾರೆ
ವ್ಯಾಖ್ಯಾಪಟೀಯಸೇ
ವ್ಯಾಖ್ಯೋಪನ್ಯಾ-ಸಧಾಟೀ
ವ್ಯಾಪಕ-ವಾಗಿ
ವ್ಯಾಪಕ-ವಾ-ಗಿತ್ತು
ವ್ಯಾಪಾರ
ವ್ಯಾಪಾರಕ್ಕಾಗಿ
ವ್ಯಾಪಾರಕ್ಕಿಂತ
ವ್ಯಾಪಾರ-ಗ-ಳನ್ನು
ವ್ಯಾಪಾ-ರದ
ವ್ಯಾಪಾರ-ವನ್ನೇ
ವ್ಯಾಪಾ-ರವೂ
ವ್ಯಾಪಾರ-ವೃತ್ತಿ-ಯನ್ನು
ವ್ಯಾಪಾರಿ
ವ್ಯಾಪಾರಿ-ಗಳ
ವ್ಯಾಪಾರಿ-ಗ-ಳನ್ನೂ
ವ್ಯಾಪಾರಿ-ಗ-ಳಾಗಿದ್ದರು
ವ್ಯಾಪಾರಿ-ಗ-ಳಾಗಿದ್ದಾರೆ
ವ್ಯಾಪಾರಿ-ಗ-ಳಾಗಿದ್ದು
ವ್ಯಾಪಾರಿ-ಗ-ಳಿಂದ
ವ್ಯಾಪಾರಿ-ಗಳಿ-ವ-ರೆಂದು
ವ್ಯಾಪಾರಿ-ಗಳು
ವ್ಯಾಪಾರಿ-ಗಳೂ
ವ್ಯಾಪಾರಿ-ಗಳೇ
ವ್ಯಾಪಾರಿ-ಮಾರ್ಗ-ವನ್ನು
ವ್ಯಾಪಾ-ರಿಯ
ವ್ಯಾಪಾರಿ-ಯಾ-ಗಿದ್ದು
ವ್ಯಾಪಾರಿ-ಯಾದ
ವ್ಯಾಪಾ-ರಿಯು
ವ್ಯಾಪಾರಿ-ಯೊಬ್ಬ
ವ್ಯಾಪಾರಿ-ವರ್ಗ-ದ-ವ-ರಿಗೂ
ವ್ಯಾಪಾರಿ-ಶಾಹೀಪ್ರಭುತ್ವವು
ವ್ಯಾಪಿ-ಸಿತ್ತು
ವ್ಯಾಪಿಸಿತ್ತೆಂದು
ವ್ಯಾಪಿ-ಸಿದ್ದ
ವ್ಯಾಪ್ತಿ
ವ್ಯಾಪ್ತಿಗೆ
ವ್ಯಾಪ್ತಿ-ಯನ್ನು
ವ್ಯಾಪ್ತಿ-ಯಲ್ಲಿ
ವ್ಯಾಪ್ತಿ-ಯಲ್ಲಿದ್ದ
ವ್ಯಾಪ್ತಿ-ಯೊಳಗೆ
ವ್ಯಾವ-ಹಾರಿಕ
ವ್ಯಾಸ-ತೀರ್ಥ
ವ್ಯಾಸ-ತೀರ್ಥ-ರಿಗೆ
ವ್ಯಾಸ-ತೀರ್ಥರು
ವ್ಯಾಸ-ನ-ತೋಳು
ವ್ಯಾಸ-ರಾಜ
ವ್ಯಾಸ-ರಾಜರು
ವ್ಯಾಸ-ರಾಜರೇ
ವ್ಯಾಸ-ರಾಯರ
ವ್ಯಾಸ-ರಾಯ-ರಿಗೆ
ವ್ಯಾಸ-ರಾಯರು
ವ್ಯಾಸ-ರಾಯಸ್ವಾಮಿ
ವ್ಯಾಸ-ರಾಯಸ್ವಾಮಿಗೆ
ವ್ಯೋಮ-ಗಂಗಾತ-ರಂಗ
ವ್ರಣೋಪಲಬ್ದ
ವ್ರತ
ವ್ರತದ
ವ್ರತ-ದಿಂದ
ವ್ರತ-ದೀಕ್ಷಿತ
ವ್ರತ-ವನ್ನು
ವ್ರತಿ
ವ್ರತ್ತಿಯ
ವ್ರಿತ್ತಿ
ವ್ರಿತ್ತಿಯ
ವ್ರಿತ್ತಿ-ಯ-ನಾಯಕ
ವ್ರಿತ್ರಿ
ವೞ್ದರೆ
ವೞ್ದರೆ-ಯನ್ಯಮ್ಮೂರೊಳೆ
ಶ
ಶಂಕ
ಶಂಕ-ಚಕ್ರದ
ಶಂಕರ
ಶಂಕ-ರ-ದಾಸಿ-ಮಯ್ಯ
ಶಂಕ-ರ-ದಾಸಿ-ಮಯ್ಯನು
ಶಂಕ-ರ-ನ-ಹಳ್ಳಿ-ಯನ್ನು
ಶಂಕ-ರ-ನಾಯ-ಕನೇ
ಶಂಕ-ರ-ನಾ-ರಾಯಣ
ಶಂಕ-ರ-ನಾ-ರಾಯ-ಣ-ದೇವರ
ಶಂಕ-ರ-ಪುರ
ಶಂಕ-ರ-ಪುರ-ಮಜ್ಜಿಗೆ-ಪುರ
ಶಂಕ-ರ-ಮಂಗಳಂತೆ
ಶಂಕ-ರ-ಮಂಗಳಂತೇ
ಶಂಕ-ರ-ಮೂರ್ತಿ-ಯ-ವ-ರಿಗೆ
ಶಂಕ-ರಯ್ಯನು
ಶಂಕ-ರ-ರಸ
ಶಂಕ-ರ-ರ-ಸರು
ಶಂಕ-ರ-ರಸ-ಸಂಕರ-ರಸರ
ಶಂಕ-ರ-ಹಳ್ಳಿ
ಶಂಕ-ರಾ-ಚಾರ್ಯರ
ಶಂಕ-ರೇಶ್ವರ
ಶಂಖ
ಶಂಖ-ಚಕ್ರ
ಶಂಖ-ಚಕ್ರ-ಗದೆ
ಶಂಖ-ಚಕ್ರದ
ಶಂಖ-ಚಕ್ರ-ಧಾರಿ-ಯಾದ
ಶಂಘಂ
ಶಂಘ-ಮದು
ಶಂತನುವು
ಶಂಭವ-ರಾಯನ
ಶಂಭು
ಶಂಭು-ದೇವ
ಶಂಭು-ದೇವನ
ಶಂಭು-ದೇವ-ನಿಗೆ
ಶಂಭು-ದೇವ-ರನ್ನು
ಶಂಭು-ದೇವರು
ಶಂಭು-ದೇವರ್
ಶಂಭು-ಲಿಂಗ
ಶಂಭು-ಲಿಂಗೇಶ್ವರ
ಶಂಭು-ವನ್ನು
ಶಂಭು-ವ-ರಾಯರು
ಶಂಭೂನ-ಹಳ್ಳಿ
ಶಂಭೂನ-ಹಳ್ಳಿಯ
ಶಂಭೂನ-ಹಳ್ಳಿ-ಯಲ್ಲಿ
ಶಕ
ಶಕಳ
ಶಕ-ವರುಷ
ಶಕ-ವರ್ಷ
ಶಕ-ವರ್ಷದ
ಶಕ-ವರ್ಷ-ವನ್ನು
ಶಕ-ವರ್ಷವು
ಶಕ್ತತ್ರ-ಯ-ಸ-ಮನ್ವಿತಂ
ಶಕ್ತಿ
ಶಕ್ತಿ-ಗಳ
ಶಕ್ತಿ-ದೇವ-ತೆಯ
ಶಕ್ತಿ-ಪರಿಷೆಗೂ
ಶಕ್ತಿ-ಪರಿಷೆಯೇ
ಶಕ್ತಿಯ
ಶಕ್ತಿ-ಯನ್ನು
ಶಕ್ತಿ-ಶಾಲಿ-ಯಾಗಿ
ಶಕ್ತಿ-ಸಾ-ಮರ್ಥ್ಯ
ಶಖಳಶ್ರೀ-ರಾಜ್ಯ
ಶಠಕೋ-ಪಾರ್ಯನ
ಶಠಗೋಪ-ಜೀಯರ
ಶಠಗೋಪ-ಮುನಿ-ವರರ
ಶಠಗೋಪ-ರನ್ನು
ಶಠಾ-ರಿಯ
ಶಣಬ
ಶಣಬ-ದಲ್ಲಿ-ರುವ
ಶತ-ಪತ್ರ-ಸಹಸ್ರ-ಕಿರಣ
ಶತ-ಭಾಷಾ
ಶತ-ಮರ್ಷಣಶಠ-ಮರ್ಷಣ-ಗೋತ್ರದ
ಶತ-ಮಾನ
ಶತ-ಮಾನಕ್ಕೆ
ಶತ-ಮಾನ-ಗ-ಳಲ್ಲಿ
ಶತ-ಮಾನ-ಗ-ಳಿಂದಲೇ
ಶತ-ಮಾನದ
ಶತ-ಮಾನ-ದಲ್ಲಿ
ಶತ-ಮಾನ-ದಲ್ಲಿದ್ದ
ಶತ-ಮಾನ-ದಲ್ಲಿಯೇ
ಶತ-ಮಾನ-ದ-ವರೆ-ಗಿನ
ಶತ-ಮಾನ-ದ-ವರೆಗೆ
ಶತ-ಮಾನ-ದಿಂದ
ಶತ-ಮಾನ-ದಿಂದಲೇ
ಶತ-ಮಾನದ್ದೆಂದು
ಶತ್ಯಾಯ-ಲೋಕ
ಶತ್ರು
ಶತ್ರು-ಗಳ
ಶತ್ರು-ಗ-ಳನ್ನು
ಶತ್ರು-ಗ-ಳಿಗೆ
ಶತ್ರು-ಗಳು
ಶತ್ರು-ಗಳೊಡನೆ
ಶತ್ರು-ರಾಜ-ರನ್ನು
ಶತ್ರು-ರಾಜ-ರು-ಗ-ಳಿಗೆ
ಶತ್ರು-ವಿನ
ಶತ್ರು-ವಿ-ನಿಂದ
ಶತ್ರು-ಸೇನೆ
ಶತ್ರು-ಸೇನೆ-ಯನ್ನು
ಶತ್ರು-ಸೇನೆ-ಯ-ವರು
ಶದಿಂದ
ಶನಿ-ವಾರ
ಶನಿ-ವಾರ-ಸಿದ್ಧಿ
ಶಬದ್
ಶಬ್ದ
ಶಬ್ದಕ್ಕೂ
ಶಬ್ದಕ್ಕೆ
ಶಬ್ದ-ಗಳ
ಶಬ್ದ-ಗ-ಳನ್ನು
ಶಬ್ದ-ಗ-ಳಿಗೆ
ಶಬ್ದ-ಗಳು
ಶಬ್ದದ
ಶಬ್ದ-ದಿಂದ
ಶಬ್ದ-ದಿಂದಲೇ
ಶಬ್ದನ್ನು
ಶಬ್ದ-ಮಣಿ-ದರ್ಪಣದ
ಶಬ್ದ-ರೂಪ-ಗಳಿ-ರ-ಬ-ಹುದು
ಶಬ್ದ-ವನ್ನು
ಶಬ್ದವು
ಶಮಾದಿ-ಗುಣ
ಶಯನೋತ್ಸವ
ಶರಣ
ಶರಣನ
ಶರಣ-ಪಂಥದ
ಶರಣ-ಪಂಥವು
ಶರಣರ
ಶರಣರು
ಶರಣ-ರು-ಗಳು
ಶರಣಾಗತ
ಶರಣಾಗತ-ನಾ-ಗಲು
ಶರಣಾಗತ-ವಜ್ರಪಂಜರ
ಶರಣಾಗತ-ವಜ್ರಪಂಜರಂ
ಶರಣು
ಶರಣೆ-ಯರು
ಶರದ-ಮಳಚನ್ದ್ರ
ಶರಧಿಗಂಭೀರ-ನೆಂದು
ಶರಭ
ಶರಸಂದೋಹ-ಮನನ್ಯ-ಸೈನ್ಯ-ದೊಡಲೊಳ್
ಶರಾಗತಮಂದಾರಃ
ಶರೀರ-ಗಳು
ಶರೀರ-ವನ್ನು
ಶರ್ಮನ
ಶಲ-ವಾಗಿ
ಶಶಕ-ಪುರದ
ಶಶಧ-ರನ
ಶಶ-ಪುರದ
ಶಶಿವಂಶ-ತಿಲಕ
ಶಷ್ಟಬ್ರಂಹ್ಮ
ಶಸ್ತ್ರಾಸ್ತ್ರ
ಶಹಾ
ಶಹಾನ
ಶಾಂತ
ಶಾಂತ-ಕು-ಮಾರಿ
ಶಾಂತ-ಕೊಂಗ-ದೇವ-ನೆಮಬ
ಶಾಂತ-ಕೊಂಗ-ದೇವರ
ಶಾಂತ-ದೇವರು
ಶಾಂತ-ನೆಂಬ
ಶಾಂತಯ್ಯ
ಶಾಂತರು
ಶಾಂತ-ಲ-ದೇವಿ
ಶಾಂತ-ಲ-ದೇವಿ-ಯರ
ಶಾಂತ-ಲಿಂಗ-ದೇಶಿ-ಕ-ನೆಂಬ
ಶಾಂತಲೆ
ಶಾಂತ-ಲೆ-ಗಿಂತ
ಶಾಂತ-ಲೆಯ
ಶಾಂತ-ಲೆ-ಯರ
ಶಾಂತ-ಲೆಯು
ಶಾಂತಿ
ಶಾಂತಿಗ್ರಾಮ
ಶಾಂತಿಗ್ರಾಮದ
ಶಾಂತಿ-ದೇವ
ಶಾಂತಿ-ನಾಥ
ಶಾಂತಿ-ನಾಥ-ದೇವರ
ಶಾಂತಿ-ನಾಥ-ದೇವ-ರಿಗೆ
ಶಾಂತಿ-ನಾಥನ
ಶಾಂತಿ-ನಾಥ-ನಿ-ಗಿಂತ
ಶಾಂತಿ-ನಾಥನು
ಶಾಂತಿ-ನಾಥ-ನೆಂಬ
ಶಾಂತಿ-ನಾಥ-ಪಂಡಿತ
ಶಾಂತಿ-ನಾಥ-ರಿಬ್ಬರೂ
ಶಾಂತಿ-ರಥಾಂಕುರಃ
ಶಾಂತೀಶ್ವರ
ಶಾಕಟಾ-ಯನ
ಶಾಕೇಭ್ರೇಷು
ಶಾಖಾ
ಶಾಖೆ
ಶಾಖೆ-ಗಳ
ಶಾಖೆ-ಗ-ಳಲ್ಲಿ
ಶಾಖೆ-ಗಳಲ್ಲೊಂದಾದ
ಶಾಖೆ-ಗಳು
ಶಾಖೆ-ಗಳೂ
ಶಾಖೆಗೆ
ಶಾಖೆಯ
ಶಾಖೆ-ಯನ್ನು
ಶಾಖೆ-ಯ-ವ-ನಿರ-ಬ-ಹುದು
ಶಾಖೆ-ಯ-ವರ
ಶಾಖೆ-ಯಾ-ದರೂ
ಶಾಖೆಯು
ಶಾಖೋಪ-ಶಾಖೆ-ಗ-ಳನ್ನು
ಶಾತಯ್ಯ
ಶಾತ-ವಾ-ಹನರ
ಶಾತ-ವಾ-ಹನರು
ಶಾನ-ದಲ್ಲೂ
ಶಾನ-ಬೋವರ
ಶಾನ-ಸದ
ಶಾನಸ-ದಲ್ಲಿ
ಶಾನು-ಭಾಗ
ಶಾನು-ಭೋಗ
ಶಾನೋಕ್ತ-ರಾದ
ಶಾಪ-ವನ್ನು
ಶಾಪಾಶಯ
ಶಾಪಾಶ-ಯ-ಗಳು
ಶಾಪಾಶ-ಯದ
ಶಾಪಾಶ-ಯ-ದಲ್ಲಿ
ಶಾಪಾಶ-ಯ-ವನ್ನು
ಶಾಪಾಶ-ಯ-ವಿದ್ದು
ಶಾಪಾಶ-ಯವೂ
ಶಾಮ-ಸುಂದರ
ಶಾಮ-ಸುಂದರ-ರಾಯರ
ಶಾಮ-ಸುಂದರ-ರಾಯರು
ಶಾಮ-ಸುಂದರ-ರಾವ್
ಶಾರಾ-ದೇವಿ
ಶಾರೀರಕಸ್ಸಾಪ್ಯಮಿಪ
ಶಾರ್ದೂಲ
ಶಾರ್ವರಿ
ಶಾಲಾ
ಶಾಲಾ-ಕಟ್ಟ-ಡದ
ಶಾಲಿ-ವಾ-ಹನ
ಶಾಲೀ-ವಾ-ಹನ
ಶಾಲೆ-ಗ-ಳನ್ನು
ಶಾಲೆ-ಗ-ಳನ್ನೂ
ಶಾಲೆಗೆ
ಶಾಲೆಯ
ಶಾಲೆ-ಯಂತಿದೆ
ಶಾವಂತ
ಶಾಶ್ವತ
ಶಾಶ್ವತಂ
ಶಾಶ್ವತ-ಗೊಳಿ-ಸಿದ್ದಾರೆ
ಶಾಶ್ವತ-ವಾಗಿ
ಶಾಶ್ವತ-ವಾದ
ಶಾಶ್ವ-ತಾತ್ಮಃ
ಶಾಸ-ಕರ
ಶಾಸ-ಗ-ಳಲ್ಲಿ
ಶಾಸತಿ
ಶಾಸ-ದಲ್ಲಿ
ಶಾಸ-ದಿಂದಿ
ಶಾಸನ
ಶಾಸನಂ
ಶಾಸನ-ಕವಿ
ಶಾಸನ-ಕವಿ-ಗಳು
ಶಾಸನ-ಕಾರ
ಶಾಸನ-ಕಾರನು
ಶಾಸನ-ಕಾರರು
ಶಾಸನಕ್ಕೂ
ಶಾಸನಕ್ಕೆ
ಶಾಸನ-ಗಳ
ಶಾಸನ-ಗಳಂತೆ
ಶಾಸನ-ಗ-ಳನ್ನ
ಶಾಸನ-ಗ-ಳನ್ನು
ಶಾಸನ-ಗ-ಳನ್ನೂ
ಶಾಸನ-ಗ-ಳನ್ನೇ
ಶಾಸನ-ಗಳಲಿ
ಶಾಸನ-ಗಳಲ್ಲಂತೂ
ಶಾಸನ-ಗ-ಳಲ್ಲಿ
ಶಾಸನ-ಗಳಲ್ಲಿದೆ
ಶಾಸನ-ಗಳಲ್ಲಿದ್ದು
ಶಾಸನ-ಗಳಲ್ಲಿಯೂ
ಶಾಸನ-ಗಳಲ್ಲಿ-ರುವ
ಶಾಸನ-ಗಳಲ್ಲಿವೆ
ಶಾಸನ-ಗ-ಳಲ್ಲೂ
ಶಾಸನ-ಗಳಲ್ಲೇ
ಶಾಸನ-ಗ-ಳಾಗಿದ್ದು
ಶಾಸನ-ಗ-ಳಾಗಿ-ರುವುದು
ಶಾಸನ-ಗ-ಳಾಗಿವೆ
ಶಾಸನ-ಗಳಾವುವೂ
ಶಾಸನ-ಗ-ಳಿಂದ
ಶಾಸನ-ಗ-ಳಿಂದಲೂ
ಶಾಸನ-ಗಳಿಗೂ
ಶಾಸನ-ಗ-ಳಿಗೆ
ಶಾಸನ-ಗಳಿದ್ದು
ಶಾಸನ-ಗಳಿ-ರುವ
ಶಾಸನ-ಗಳಿ-ರುವುದ-ರಿಂದ
ಶಾಸನ-ಗಳಿಲ್ಲ
ಶಾಸನ-ಗಳಿವೆ
ಶಾಸನ-ಗಳು
ಶಾಸನ-ಗಳು-ಒಂದು
ಶಾಸನ-ಗಳುನ್ನು
ಶಾಸನ-ಗಳೂ
ಶಾಸನ-ಗಳೆಲ್ಲವೂ
ಶಾಸನ-ಗಳೆಲ್ಲಾ
ಶಾಸನ-ಗಳೇ
ಶಾಸನ-ತಜ್ಞರು
ಶಾಸನದ
ಶಾಸನ-ದ-ಲಕ್ಕ-ರವ
ಶಾಸನ-ದಲಿ
ಶಾಸನ-ದ-ಲಿದೆ
ಶಾಸನ-ದಲ್ಲಂತೂ
ಶಾಸನ-ದಲ್ಲಿ
ಶಾಸನ-ದಲ್ಲಿ-ದಲ್ಲಿ
ಶಾಸನ-ದಲ್ಲಿದೆ
ಶಾಸನ-ದಲ್ಲಿ-ದೆೆ
ಶಾಸನ-ದಲ್ಲಿದ್ದು
ಶಾಸನ-ದಲ್ಲಿಯೂ
ಶಾಸನ-ದಲ್ಲಿಯೇ
ಶಾಸನ-ದಲ್ಲಿರು
ಶಾಸನ-ದಲ್ಲಿ-ರುವ
ಶಾಸನ-ದಲ್ಲಿ-ರು-ವಂತೆ
ಶಾಸನ-ದಲ್ಲಿ-ರು-ವುದು
ಶಾಸನ-ದಲ್ಲಿವೆ
ಶಾಸನ-ದಲ್ಲೂ
ಶಾಸನ-ದಲ್ಲೂ-ಕೃಪೇ
ಶಾಸನ-ದಲ್ಲೇ
ಶಾಸನ-ದ-ವರೆಗೂ
ಶಾಸನ-ದ-ವರೆಗೆ
ಶಾಸನ-ದಿಂದ
ಶಾಸನ-ದಿದ
ಶಾಸನ-ದೇವಿ-ಯ-ರಿದ್ದಂತೆ
ಶಾಸನ-ಧಾರ-ಗ-ಳಿಂದ
ಶಾಸನ-ಧಾರ-ಗಳು
ಶಾಸನ-ನ-ಗ-ಳನ್ನು
ಶಾಸನ-ಪದ್ಯ-ಗಳು
ಶಾಸನ-ಪದ್ಯದ
ಶಾಸನ-ಪಾಠ-ವನ್ನು
ಶಾಸನ-ಪಾಠವೇ
ಶಾಸನ-ಬ-ರಹ-ಗಳು
ಶಾಸನಮಂ
ಶಾಸನ-ಮಮ್ಲಾನ
ಶಾಸನ-ಮುರು-ಕವಿ
ಶಾಸನಮ್
ಶಾಸನ-ರಹಿತ-ವಾದ
ಶಾಸನ-ಲೇಖ-ಕರ
ಶಾಸ-ನಲ್ಲಿ
ಶಾಸನವ
ಶಾಸನವಂ
ಶಾಸನ-ವ-ನ-ವನ್ನು
ಶಾಸನ-ವನ್ನು
ಶಾಸನ-ವನ್ನೂ
ಶಾಸನ-ವಾಗಿದೆ
ಶಾಸನ-ವಾ-ಗಿದ್ದು
ಶಾಸನ-ವಾಗಿ-ರ-ಬ-ಹುದು
ಶಾಸನ-ವಾಗಿ-ರ-ಬಹು-ದೆಂದು
ಶಾಸನ-ವಾಚಕ-ಚಕ್ರ-ವರ್ತಿ
ಶಾಸನ-ವಾದ
ಶಾಸನ-ವಾದರೆ
ಶಾಸನ-ವಿದ
ಶಾಸನ-ವಿದೆ
ಶಾಸನ-ವಿದ್ದು
ಶಾಸನ-ವಿ-ರುವ
ಶಾಸನ-ವಿಲ್ಲ
ಶಾಸನವು
ಶಾಸನ-ವು-ನರ-ಸಿಂಹ-ನನ್ನು
ಶಾಸನ-ವು-ಬಹಳ
ಶಾಸನ-ವು-ಲಾ-ವಣ್ಯ
ಶಾಸನ-ವು-ವೀರ-ಬಲ್ಲಾಳ
ಶಾಸನವೂ
ಶಾಸನ-ವೆಂದರೆ
ಶಾಸನ-ವೆಂದು
ಶಾಸನವೇ
ಶಾಸನ-ವೊಂದ-ರಲ್ಲಿ
ಶಾಸನ-ವೊಂದು
ಶಾಸನ-ಶಾಸ್ತ್ರ
ಶಾಸನಸ್ತ
ಶಾಸನಸ್ತ-ವಾದ
ಶಾಸನಸ್ಥ
ಶಾಸನಸ್ಥ-ವಹ
ಶಾಸನಸ್ಥ-ವಾಗಿ
ಶಾಸನಸ್ಥ-ವಾದ
ಶಾಸನಾ-ಚಾರ್ಯ
ಶಾಸನಾ-ಚಾರ್ಯ-ಧರ್ಮೇಣ
ಶಾಸನಾ-ಚಾರ್ಯರ
ಶಾಸನಾ-ಚಾರ್ಯರು
ಶಾಸನಾದ್ವೀರ-ಣಾತ್ಮಜಃ
ಶಾಸನಾಧಾರ-ಗಳಿಲ್ಲ
ಶಾಸನಾಧಾರ-ಗಳಿವೆ
ಶಾಸನಾಧಾರ-ಗಳು
ಶಾಸನಾಧಾ-ರವಿದ್ದರೆ
ಶಾಸನಾಧಾ-ರವಿ-ರುವ
ಶಾಸನಾಧಾ-ರ-ವಿಲ್ಲ
ಶಾಸನಾನ್ಮಲ್ಲ-ಣಾತ್ಮಜಃ
ಶಾಸನೇನ
ಶಾಸನೋಕ್ತ
ಶಾಸನೋಕ್ತ-ನಾಗಿದ್ದಾ-ನೆಂದು
ಶಾಸನೋಕ್ತ-ನಾ-ಗಿದ್ದು
ಶಾಸನೋಕ್ತ-ನಾದ
ಶಾಸನೋಕ್ತ-ರಾಗಿದ್ದಾರೆ
ಶಾಸನೋಕ್ತ-ರಾಗುವ
ಶಾಸನೋಕ್ತ-ರಾದ
ಶಾಸನೋಕ್ತ-ವಲ್ಲದ
ಶಾಸನೋಕ್ತ-ವಲ್ಲ-ದಿದ್ದರೂ
ಶಾಸನೋಕ್ತ-ವಾಗಿ
ಶಾಸನೋಕ್ತ-ವಾಗಿದೆ
ಶಾಸನೋಕ್ತ-ವಾ-ಗಿದ್ದು
ಶಾಸನೋಕ್ತ-ವಾಗಿ-ರುವ
ಶಾಸನೋಕ್ತ-ವಾಗಿಲ್ಲ
ಶಾಸನೋಕ್ತ-ವಾಗಿವೆ
ಶಾಸನೋಕ್ತ-ವಾದ
ಶಾಸನೋನಕ್ತ-ವಾಗಿಲ್ಲ
ಶಾಸನೋಲ್ಲೇಖ-ವನ್ನು
ಶಾಸನ್ದ-ದಲ್ಲಿ
ಶಾಸ-ವನವು
ಶಾಸ-ವನವೇ
ಶಾಸ-ವನು
ಶಾಸ-ವನ್ನು
ಶಾಸ-ವಿದೆ
ಶಾಸ-ವಿದ್ದು
ಶಾಸ-ಸವು
ಶಾಸೋಕ್ತ-ವಾಗಿವೆ
ಶಾಸ್ತ್ರ
ಶಾಸ್ತ್ರ-ಗಳ
ಶಾಸ್ತ್ರ-ಗ-ಳನ್ನು
ಶಾಸ್ತ್ರದ
ಶಾಸ್ತ್ರ-ದಲ್ಲಿ
ಶಾಸ್ತ್ರ-ದಾನ
ಶಾಸ್ತ್ರ-ಬದ್ಧ-ವಾಗಿ
ಶಾಸ್ತ್ರ-ವನ್ನು
ಶಾಸ್ತ್ರ-ವಿಧಿ-ಗ-ಳನ್ನು
ಶಾಸ್ತ್ರ-ವೇದಿನೇ
ಶಾಸ್ತ್ರಾಗಮ
ಶಾಸ್ತ್ರಾಭ್ಯಾಸ-ಗ-ಳನ್ನು
ಶಾಸ್ತ್ರಾಯ
ಶಾಸ್ತ್ರಾರ್ಥ
ಶಾಸ್ತ್ರಾರ್ಥ-ವನ್ನು
ಶಾಸ್ತ್ರಿ-ಗಳು
ಶಾಸ್ತ್ರೇಷು
ಶಿಂಗಂಣ-ಗಳ
ಶಿಂಗಣ್ಣ
ಶಿಂಗನ
ಶಿಂಗ-ಪೆರು-ಮಾಳ್
ಶಿಂಗಪ್ಪ-ನಾಯಕ
ಶಿಂಗಪ್ಪ-ನಾಯ-ಕನು
ಶಿಂಗಪ್ಪ-ನಾಯ-ಕರು
ಶಿಂಗಪ್ಪೆರು-ಮಾಳ್
ಶಿಂಗಪ್ಪೆರು-ಮಾಳ್ಗೆ
ಶಿಂಗ-ಮಾರ-ನ-ಹಳ್ಳಿ
ಶಿಂಗ-ಮಾರ-ನ-ಹಳ್ಳಿ-ಯನ್ನು
ಶಿಂಗಯ್ಯನ
ಶಿಂಗರಯ್ಯಂಗಾರರ
ಶಿಂಗರೈಂಗಾರ್
ಶಿಂಗರೈಯಂಗಾರರ
ಶಿಂಗರೈಯಂಗಾರ-ರಿಂದ
ಶಿಂಗರೈಯಂಗಾರರು
ಶಿಂಗರೈಯಂಗಾರ್
ಶಿಂಗರೈಯ್ಯಂಗಾರ
ಶಿಂಗರೈಯ್ಯಂಗಾರರ
ಶಿಂಗರೈಯ್ಯಂಗಾರರು
ಶಿಂಗಾರ-ಕೊಳದ
ಶಿಂಗಾರಮ್ಮ
ಶಿಂಗಾ-ರಾರ್ಯನ
ಶಿಂಗಿರೀ
ಶಿಂಗ್ಯಪ್ಪೆರು-ಮಾಳಯ್ಯ
ಶಿಂಗ್ಯಪ್ಪೆರು-ಮಾಳಯ್ಯಗೆ
ಶಿಂಗ್ರೈಯಂಗಾರ್
ಶಿಂಗ್ಲಾ-ಚಾರ್ಯರು
ಶಿಂಘಣ
ಶಿಂಶದ
ಶಿಂಶಾ
ಶಿಂಶಾ-ನ-ದಿಯ
ಶಿಂಶಾ-ನ-ದಿಯೇ
ಶಿಂಷಾ
ಶಿಂಹಾಸ್ವ-ನದ
ಶಿಂಹಾಸ್ವನಾರೂಢ-ರಾದ
ಶಿಕಾಮಿನಿ
ಶಿಕಾರಿ-ಪುರದ
ಶಿಕ್ಷಣ
ಶಿಕ್ಷ-ಣದ
ಶಿಕ್ಷಾ
ಶಿಕ್ಷೆ
ಶಿಖರ
ಶಿಖರಕ್ಕೆ
ಶಿಖ-ರದ
ಶಿಖರ-ದ-ಮೇಲೆ
ಶಿಖೆ-ಯನ್ನು
ಶಿಚೆ
ಶಿಚೆ-ನಂದಿ-ಮಠ್
ಶಿಡಲು
ಶಿಡ್ಲು-ಬಸವೇಶ್ವರ
ಶಿತ-ಕರ-ಗಂಡ
ಶಿಥಿಲಬೆಂಕೊಂಬರುಂ
ಶಿಥಿಲ-ವಾಗಿ
ಶಿಥಿಲ-ವಾ-ಗಿದ್ದ
ಶಿಥಿಲ-ವಾಗಿ-ರುವ
ಶಿಥಿಲಾವಸ್ಥೆ-ಯಲ್ಲಿದೆ
ಶಿದ್ಧ-ಬಸವನ
ಶಿಧಾಂತ
ಶಿಬಿರ-ದಲ್ಲಿ
ಶಿರಚ್ಛೇದ
ಶಿರಪ್ರಧಾನ
ಶಿರ-ಶಾ-ಸನ-ವನ್ನು
ಶಿರಶ್ಚುಂಬಿ
ಶಿರಸಾ
ಶಿರಸ್ತೆ-ದಾರ್
ಶಿರಸ್ತೇ-ದಾರ್
ಶಿರಸ್ಸನ್ನು
ಶಿರಿಮಕ್ಕನ-ಹಳ್ಳಿ
ಶಿರೋಗ್ರಮಂ
ಶಿರೋ-ಮಣಿ
ಶಿರೋ-ಮಣಿ-ಯಂತಿದ್ದ
ಶಿರೋ-ಮಣಿ-ಯಂತಿಪ್ಪ
ಶಿರೋ-ಮಣಿ-ಯಂತೆ
ಶಿಲಾ
ಶಿಲಾ-ಕರ್ಮ್ಮ
ಶಿಲಾದ
ಶಿಲಾಪ್ರತಿ-ಮೆ-ಗಳು
ಶಿಲಾಪ್ರತಿಷ್ಠೆ
ಶಿಲಾ-ಮಂಟಪ-ವನ್ನು
ಶಿಲಾ-ಮಯ
ಶಿಲಾ-ಮೂರ್ತಿ
ಶಿಲಾ-ಮೂರ್ತಿ-ಯನ್ನು
ಶಿಲಾ-ಯುಗದ
ಶಿಲಾ-ಲೇಖ-ವಿದೆ
ಶಿಲಾ-ಶಾ-ಸನ
ಶಿಲಾ-ಶಾ-ಸನ-ಗ-ಳನ್ನು
ಶಿಲಾ-ಶಾ-ಸನ-ಗ-ಳನ್ನೂ
ಶಿಲಾ-ಶಾ-ಸನ-ಗ-ಳಲ್ಲಿ
ಶಿಲಾ-ಶಾ-ಸನ-ಗಳು
ಶಿಲಾ-ಶಾ-ಸನದ
ಶಿಲಾ-ಶಾ-ಸನ-ದಲ್ಲಿ
ಶಿಲಾ-ಶಾ-ಸನ-ದಲ್ಲೂ
ಶಿಲಾ-ಶಾ-ಸನವ
ಶಿಲಾ-ಶಾ-ಸನ-ವನು
ಶಿಲಾ-ಶಾ-ಸನ-ವನ್ನು
ಶಿಲಾ-ಶಾ-ಸನವು
ಶಿಲಾ-ಶಾ-ಸನವೇ
ಶಿಲಾ-ಶಾಸ-ವನೇ
ಶಿಲಾ-ಶಾಸ-ವನ್ನು
ಶಿಲಾ-ಸಾ-ಧನ
ಶಿಲಾಸ್ತಂಭ-ವನ್ನು
ಶಿಲಾಸ್ತಂಭೋ
ಶಿಲಾಸ್ಥಂಭ-ವನ್ನು
ಶಿಲೆ-ಗ-ಳನ್ನು
ಶಿಲೆ-ಗ-ಳಾದ
ಶಿಲೆ-ಗ-ಳಿಗೆ
ಶಿಲೆ-ಗಳು
ಶಿಲೆ-ಯಲ್ಲಿ
ಶಿಲೆ-ಯಲ್ಲಿ-ರುವ
ಶಿಲೆ-ಯಿಂದ
ಶಿಲೆಯೆ
ಶಿಲ್ಪ
ಶಿಲ್ಪ-ಕಲೆ
ಶಿಲ್ಪ-ಕಲೆಯ
ಶಿಲ್ಪ-ಕಲೆ-ಯಲ್ಲಿ
ಶಿಲ್ಪ-ಕಲೆ-ಯಿಂದ
ಶಿಲ್ಪಕ್ಕಾಗಿ
ಶಿಲ್ಪ-ಗಳ
ಶಿಲ್ಪ-ಗ-ಳನ್ನು
ಶಿಲ್ಪ-ಗ-ಳಲ್ಲಿ
ಶಿಲ್ಪ-ಗ-ಳಿಂದ
ಶಿಲ್ಪ-ಗ-ಳಿಗೆ
ಶಿಲ್ಪ-ಗಳಿವೆ
ಶಿಲ್ಪ-ಗಳು
ಶಿಲ್ಪ-ಗಾರಿಕೆ
ಶಿಲ್ಪ-ಗಾರಿಕೆ-ಯನ್ನು
ಶಿಲ್ಪದ
ಶಿಲ್ಪ-ದಲ್ಲಿ
ಶಿಲ್ಪ-ದಿಂದ
ಶಿಲ್ಪ-ಮಾತ್ರ
ಶಿಲ್ಪ-ವನ್ನು
ಶಿಲ್ಪ-ವನ್ನೂ
ಶಿಲ್ಪ-ವಾದ
ಶಿಲ್ಪ-ವಿದೆ
ಶಿಲ್ಪ-ವಿ-ರುವ
ಶಿಲ್ಪವು
ಶಿಲ್ಪ-ವುಳ್ಳ
ಶಿಲ್ಪ-ಸಹಿತ-ವಾದ
ಶಿಲ್ಪಾ-ಚಾರಿ-ಯರು
ಶಿಲ್ಪಿ
ಶಿಲ್ಪಿ-ಗಳ
ಶಿಲ್ಪಿ-ಗಳದು
ಶಿಲ್ಪಿ-ಗ-ಳಾಗಿ
ಶಿಲ್ಪಿ-ಗ-ಳಾದ
ಶಿಲ್ಪಿ-ಗಳು
ಶಿಲ್ಪಿ-ಗಳು-ರೂ-ವಾರಿ-ಗಳು-ಕುಶ-ಲ-ಕರ್ಮಿ-ಗಳು
ಶಿಲ್ಪಿ-ಗಳೂ
ಶಿಲ್ಪಿ-ಗಳೇ
ಶಿಲ್ಪಿಗೆ
ಶಿಲ್ಪಿಯ
ಶಿಲ್ಪಿ-ಯಾ-ಗಿದ್ದು
ಶಿಲ್ಪಿ-ಯಾಗಿರ
ಶಿಲ್ಪಿ-ಯಾಗಿ-ರದೇ
ಶಿಲ್ಪಿ-ಯಾಗಿ-ರ-ಬ-ಹುದು
ಶಿಲ್ಪಿ-ಯಾಗಿ-ರ-ಬಹು-ದೆಂದು
ಶಿಲ್ಪಿ-ಯಾಗಿ-ರಲೂ
ಶಿಲ್ಪಿಯೂ
ಶಿವ
ಶಿವಃ
ಶಿವ-ಕಥಾಳಾಪೆ
ಶಿವ-ಕೆರೆ
ಶಿವ-ಗಂಗೆಗೂ
ಶಿವ-ಗಂಗೆಗೆ
ಶಿವ-ಗಂಗೆಯ
ಶಿವ-ಗಿರಿಯ
ಶಿವ-ಗೀತೆ-ಗ-ಳನ್ನು
ಶಿವ-ತತ್ತ್ವ
ಶಿವ-ತತ್ವ
ಶಿವ-ದೇವ
ಶಿವ-ದೇವನು
ಶಿವ-ದೇವಾ-ಲಯ
ಶಿವ-ದೇವಾ-ಲಯ-ಗ-ಳನ್ನು
ಶಿವದ್ರೋಹಿ
ಶಿವ-ಧರ್ಮ
ಶಿವನ
ಶಿವ-ನಂಜಯ್ಯ-ನಿಗೆ
ಶಿವ-ನಂಜಯ್ಯ-ನೆಂಬ
ಶಿವ-ನನ್ನು
ಶಿವ-ನ-ಸ-ಮುದ್ರ
ಶಿವ-ನ-ಸ-ಮುದ್ರ-ಗ-ಳನ್ನು
ಶಿವ-ನ-ಸ-ಮುದ್ರದ
ಶಿವ-ನ-ಸ-ಮುದ್ರ-ದಲ್ಲಿ
ಶಿವ-ನ-ಸ-ಮುದ್ರ-ದಲ್ಲಿ-ರುವ
ಶಿವ-ನ-ಸ-ಮುದ್ರ-ದಿಂದ
ಶಿವ-ನ-ಸ-ಮುದ್ರ-ದಿಂದಲೂ
ಶಿವ-ನ-ಸ-ಮುದ್ರವು
ಶಿವ-ನಾಥ-ನೆನಿಸಿ
ಶಿವ-ನಾಥನೇ
ಶಿವ-ನಿಗೂ
ಶಿವ-ನಿಗೆ
ಶಿವನು
ಶಿವ-ನೆಂದು
ಶಿವ-ಪಾದ
ಶಿವ-ಪಾದೋ-ದಕೆ
ಶಿವ-ಪಾರ್ವತಿ
ಶಿವ-ಪುರ
ಶಿವ-ಪುರ-ಗ-ಳನ್ನು
ಶಿವ-ಪುರ-ಗಳು
ಶಿವ-ಪುರದ
ಶಿವ-ಪುರ-ದಲ್ಲಿ
ಶಿವ-ಪುರ-ದೊಳ-ಗಣ
ಶಿವ-ಪುರ-ವನ್ನಾಗಿ
ಶಿವ-ಪುರ-ವನ್ನು
ಶಿವ-ಪುರ-ವಾಗಿ
ಶಿವ-ಪುರವು
ಶಿವ-ಪುರಿ-ಯಾಗಿ
ಶಿವಬ್ರಾಹ್ಮಣ
ಶಿವಬ್ರಾಹ್ಮಣ-ನೆಂದು
ಶಿವ-ಭಕ್ತ
ಶಿವ-ಭಕ್ತರ
ಶಿವ-ಭಕ್ತ-ರನ್ನು
ಶಿವ-ಭಕ್ತ-ರಿಗೂ
ಶಿವ-ಭಕ್ತ-ರಿಗೆ
ಶಿವ-ಭಕ್ತ-ರಿಗೆ-ಶರಣ-ರಿಗೆ
ಶಿವ-ಭಕ್ತರು
ಶಿವ-ಭಕ್ತಿಯ
ಶಿವ-ಭಟಾಧ್ಯಕ್ಷ-ರಾದ
ಶಿವ-ಮಾರ
ಶಿವ-ಮಾರನ
ಶಿವ-ಮಾರ-ನನ್ನು
ಶಿವ-ಮಾರ-ನಿಗೆ
ಶಿವ-ಮಾರನು
ಶಿವ-ಮಾರ-ಶರ್ಮನು
ಶಿವ-ಮಾರ-ಸಿಂಹ
ಶಿವ-ಮಾರಸ್ಯ
ಶಿವ-ಮೊಗ್ಗ
ಶಿವಯ
ಶಿವ-ಯೋಗಿ
ಶಿವ-ರ-ಮಂಡ್ಯ-ತಾಲ್ಲೂಕಿನ
ಶಿವ-ರ-ಶರಣರ
ಶಿವ-ರಾಚಯ್ಯ
ಶಿವ-ರಾಜ
ಶಿವ-ರಾಜನು
ಶಿವ-ರಾತ್ರಿಯ
ಶಿವ-ರಾತ್ರಿ-ಯಂದು
ಶಿವ-ರಾಮ-ಪಂಡಿ-ತರು
ಶಿವ-ರಾಯ
ಶಿವ-ರುದ್ರಸ್ವಾಮಿ
ಶಿವ-ಲಿಂಗ
ಶಿವ-ಲಿಂಗ-ವನ್ನು
ಶಿವ-ಲಿಂಗ-ವಿದೆ
ಶಿವ-ಲಿಂಗವು
ಶಿವ-ಲೆಂಕ
ಶಿವಳ್ಳಿ-ಯಲ್ಲಿ
ಶಿವ-ಶರಣ
ಶಿವ-ಶರಣನ
ಶಿವ-ಶರಣ-ನಾಗಿ-ರ-ಬ-ಹುದು
ಶಿವ-ಶರಣರ
ಶಿವ-ಶರಣ-ರನ್ನು
ಶಿವ-ಶರಣ-ರಲ್ಲಿ
ಶಿವ-ಶರಣ-ರಿಗೆ
ಶಿವ-ಶರಣರು
ಶಿವ-ಶೋಧ
ಶಿವ-ಸನ್ನಿಧಿ-ಯಲ್ಲಿ
ಶಿವ-ಸ-ಮಯ
ಶಿವ-ಸಾಯುಜ್ಯ-ವನ್ನೈದಿ-ದ-ನೆಂದೂ
ಶಿವಸ್ತುತಿ
ಶಿವಸ್ತುತಿಯ
ಶಿವಾ-ಚಾರ
ಶಿವಾ-ಚಾರದ
ಶಿವಾ-ಚಾರ-ದ-ವ-ರೆಂದೇ
ಶಿವಾಜಿಯ
ಶಿವಾದ್ವೈತ
ಶಿವಾರ
ಶಿವಾರಾ
ಶಿವಾ-ಲ-ಯಕ್ಕೆ
ಶಿವಾ-ಲಯ-ಗ-ಳನ್ನು
ಶಿವಾ-ಲಯದ
ಶಿವಾ-ಲಯ-ವನ್ನು
ಶಿವಾ-ಲಯ-ವಿದೆ
ಶಿವಾ-ಲಯವು
ಶಿವೇ-ತರ
ಶಿವೋಜಿ
ಶಿವೋಪಸಕ-ರಾ-ಗಿದ್ದು
ಶಿಶಿ-ಲದ
ಶಿಶುವಧೆ
ಶಿಶು-ಹತ್ಯೆ
ಶಿಷ್ಟಪ್ರತಿ-ಪಾ-ಳನ
ಶಿಷ್ಟಪ್ರಿಯ
ಶಿಷ್ಟಪ್ರಿಯ-ನಾದ
ಶಿಷ್ಯ
ಶಿಷ್ಯಃ
ಶಿಷ್ಯ-ಜ-ನರ
ಶಿಷ್ಯನ
ಶಿಷ್ಯ-ನಾ-ಗದೆ
ಶಿಷ್ಯ-ನಾಗಿ
ಶಿಷ್ಯ-ನಾ-ಗಿದ್ದ
ಶಿಷ್ಯ-ನಾಗಿದ್ದ-ನು-ನಯ-ಕೀರ್ತಿಯ
ಶಿಷ್ಯ-ನಾಗಿ-ರ-ಬ-ಹುದು
ಶಿಷ್ಯ-ನಾದ
ಶಿಷ್ಯ-ನಿರ-ಬಹು-ದೆಂದು
ಶಿಷ್ಯನೂ
ಶಿಷ್ಯ-ನೆಂದು
ಶಿಷ್ಯ-ನೆಂದೂ
ಶಿಷ್ಯ-ನೆಂಬುದು
ಶಿಷ್ಯನೇ
ಶಿಷ್ಯನೋ
ಶಿಷ್ಯ-ಪರಂಪ-ರೆಯ
ಶಿಷ್ಯ-ಪರಂಪರೆ-ಯಾಗಿ
ಶಿಷ್ಯ-ರನ್ನು
ಶಿಷ್ಯ-ರ-ಮಯ್ಯೊಳ್
ಶಿಷ್ಯ-ರಾ-ಗಿದ್ದ
ಶಿಷ್ಯ-ರಾಗಿದ್ದರು
ಶಿಷ್ಯ-ರಾಗಿದ್ದರೂ
ಶಿಷ್ಯ-ರಾ-ಗಿಯೇ
ಶಿಷ್ಯ-ರಾದ
ಶಿಷ್ಯ-ರಿಗೆ
ಶಿಷ್ಯರು
ಶಿಷ್ಯ-ರು-ಗಳ
ಶಿಷ್ಯ-ರು-ಗಳು
ಶಿಷ್ಯರೂ
ಶಿಷ್ಯ-ರೊಡ-ಗೂಡಿ
ಶಿಷ್ಯ-ವೃತ್ತಿಯು
ಶಿಷ್ಯಿತಿ-ಯ-ರಪ್ಪ
ಶಿಷ್ಯಿ-ತಿಯ-ರಾದ
ಶಿಷ್ಯಿತ್ತಿ-ಯ-ರಪ್ಪ
ಶಿಷ್ಯೆ
ಶಿಹ್ವ
ಶೀ
ಶೀಘ್ರ-ಲಿಪಿ
ಶೀಘ್ರವೇ
ಶೀಮೆ-ಗಳ
ಶೀರಂಗ-ದೇವ
ಶೀರಂಗ-ಪಟ್ಟಣ
ಶೀರ್ಯ-ಪೇಟೆ
ಶೀರ್ಯ್ಯ
ಶೀಲ-ಗುಣ
ಶೀಲ-ಗುಣೌಘ
ಶೀಲ-ಸಂಬಂಧಿ-ಯಾದ
ಶೀಳುನೆರೆ
ಶೀವೈಷ್ಣವ
ಶು
ಶುಂಕ
ಶುಂಣ್ನ-ಹರಳ
ಶುಕ
ಶುಕ-ಚರಿತ-ನೆಂದಿದೆ
ಶುಕ್ರ-ವಾರ
ಶುಕ್ಲ-ಪಕ್ಷದ
ಶುಚಿ-ಯಾಗಿ-ಡಲು
ಶುದ್ಧ
ಶುದ್ಧ-ಶೈವ
ಶುದ್ಧ-ಶೈವರ
ಶುದ್ಧಾದ್ವೈತ
ಶುದ್ಧೀ-ಕರಣ
ಶುದ್ಧೀ-ಕರಿ-ಸಲು
ಶುದ್ಧೋಭಯಾನ್ವಯ
ಶುಭ
ಶುಭ-ಕಷಾವಲಿ
ಶುಭ-ಕಾರ್ಯ-ಗ-ಳನ್ನು
ಶುಭ-ಚಂದ್ರ
ಶುಭ-ಚಂದ್ರನ
ಶುಭ-ಚಂದ್ರ-ಸಿದ್ಧಾಂತ
ಶುಭ-ದೀ-ಯಾರ-ಭ-ವತ್ಸದಾ
ಶುಭ-ಮಂಗಳ
ಶುಭ-ಯಸಿ
ಶುಭ-ಲಕ್ಷ-ಣ-ವುಳ್ಳ
ಶುಭಾವತೈಃ
ಶುಭೈಃ
ಶುರುಳು-ಮದು
ಶುರು-ವಾಗುತ್ತದೆ
ಶುಲ್ಕ-ಗಳು
ಶುಷ್ಕಾಸ್ತುರಷ್ಕಾ
ಶೂದ್ರ
ಶೂದ್ರಕಂ
ಶೂದ್ರ-ಕ-ನೆಂದು
ಶೂದ್ರ-ಕುಲದ
ಶೂದ್ರ-ಕುಲ-ದ-ವ-ರಾಗಿದ್ದ-ರೆಂದು
ಶೂದ್ರ-ಗಣದ
ಶೂದ್ರಪ್ರಜೆ
ಶೂದ್ರ-ರನ್ನು
ಶೂದ್ರ-ರಿಗೆ
ಶೂದ್ರರು
ಶೂದ್ರ-ವಾಡ-ವಾ-ಗಿದ್ದ
ಶೂರ-ತೆಯಂ
ಶೂರನು
ಶೂರ-ಯತಾ
ಶೂರರು
ಶೂಲ
ಶೃಂಗದ
ಶೃಂಗಾರ
ಶೃಂಗಾರ-ಹಾರ
ಶೃಂಗೇರಿ
ಶೃಂಗೇರಿಗೆ
ಶೃಂಗೇರಿಯ
ಶೃಂಗೇರಿ-ಯಲ್ಲಿ
ಶೆಟ್ಟರ್
ಶೆಟ್ಟಿ
ಶೆಟ್ಟಿ-ಹಳ್ಳಿ
ಶೆಟ್ಟಿ-ಹಳ್ಳಿ-ಗ-ಳಲ್ಲಿ
ಶೆಟ್ಳೂರು
ಶೆಪ್ಪುಕೆಣ್ಡಿ
ಶೆಪ್ಪು-ಮಣಿಯೈ
ಶೆಯ್ವಿಚ್ಚ
ಶೆಯ್ವಿತ್ತಾನ್
ಶೇಂಗಣಿ-ವಂಶದ
ಶೇಖರಃ
ಶೇಖರ-ಣೆ-ಯಾಗಿ
ಶೇಖರ-ಣೆ-ಯಾಗುತ್ತಿತ್ತು
ಶೇಖರ-ಮಣಿ
ಶೇಖರೆ-ಯಾದ
ಶೇಖರೆಯುಂಮಪ್ಪ
ಶೇಖ್
ಶೇಖ್ದಾರ್
ಶೇತೇ
ಶೇಲೆಯ-ಪುರ-ಸೇಲಂದ
ಶೇವೆ
ಶೇಷ
ಶೇಷಕ್ಷಿತಿ-ಪತಿ
ಶೇಷ-ಧರ್ಮ
ಶೇಷ-ವಾದ
ಶೇಷ-ಶಾಸ್ತ್ರಿ-ಯ-ವರ
ಶೇಷ-ಶಾಸ್ತ್ರಿ-ಯ-ವರು
ಶೇಷಾನ್ವಯ
ಶೇಷಾಯ
ಶೇಷಾ-ಶೇಷಾ-ನನಶ್ರೀ
ಶೇಸಶಜಗಜ್ಜನ
ಶೈಕ್ಷಣಿಕ
ಶೈಲ
ಶೈಲ-ದಲ್ಲಿ
ಶೈಲಿಗೆ
ಶೈಲಿಯ
ಶೈಲಿ-ಯನ್ನು
ಶೈಲಿ-ಯಲ್ಲಿ
ಶೈಲಿ-ಯಲ್ಲಿದೆ
ಶೈಲಿ-ಯಲ್ಲಿದ್ದು
ಶೈಲಿ-ಯಲ್ಲಿ-ರುವ
ಶೈಲಿ-ಯಲ್ಲಿವೆ
ಶೈಲೇಂದ್ರ
ಶೈವ
ಶೈವ-ಕೇಂದ್ರ
ಶೈವ-ಕೇಂದ್ರ-ವಾಗಿ
ಶೈವ-ಕೇಂದ್ರ-ವಾಗಿತ್ತೆಂದು
ಶೈವ-ಕೇಂದ್ರ-ವಾ-ಗಿದ್ದ
ಶೈವ-ಕೇಂದ್ರ-ವಾದ
ಶೈವ-ಕೇಂದ್ರವೂ
ಶೈವಕ್ಷೇತ್ರ-ಗ-ಳಲ್ಲಿ
ಶೈವಕ್ಷೇತ್ರ-ವಾಗಿ-ರ-ಬ-ಹುದು
ಶೈವಕ್ಷೇತ್ರ-ವಾದ
ಶೈವ-ಗುರು-ಗ-ಳನ್ನು
ಶೈವ-ದಿಂದ
ಶೈವ-ದೇವಾ-ಲಯ
ಶೈವ-ದೇವಾ-ಲಯ-ಗಳ
ಶೈವ-ದೇವಾ-ಲಯ-ಗ-ಳನ್ನು
ಶೈವ-ದೇವಾ-ಲಯ-ಗ-ಳಲ್ಲಿ
ಶೈವ-ದೇವಾ-ಲಯ-ಗಳಿತ್ತೆಂದು
ಶೈವ-ದೇವಾ-ಲಯ-ಗಳು
ಶೈವ-ದೇವಾ-ಲ-ಯದ
ಶೈವ-ದೇವಾ-ಲಯ-ವನ್ನು
ಶೈವ-ದೇವಾ-ಲಯ-ವಾಗಿರ
ಶೈವ-ದೇವಾ-ಲಯ-ವೆಂದರೆ
ಶೈವದ್ವಾರ-ಪಾಲ-ಕರು
ಶೈವ-ಧರ್ಮ
ಶೈವ-ಧರ್ಮಕ್ಕೆ
ಶೈವ-ಧರ್ಮದ
ಶೈವ-ಧರ್ಮ-ದಲ್ಲಿ
ಶೈವ-ಧರ್ಮ-ದಷ್ಟೇ
ಶೈವ-ಧರ್ಮ-ವನ್ನು
ಶೈವ-ಧರ್ಮ-ವಾದ
ಶೈವ-ಧರ್ಮವು
ಶೈವ-ಧರ್ಮಾನುಯಾಯಿ-ಗಳ
ಶೈವ-ಧರ್ಮಾನುಯಾಯಿ-ಗಳಾಗಿದ್ದರು
ಶೈವ-ಧರ್ಮಾನುಯಾಯಿ-ಗಳಾದ
ಶೈವ-ಧರ್ಮಾವ-ಲಂಬಿ-ಗ-ಳಾಗಿದ್ದ-ರೆಂದು
ಶೈವ-ನಾ-ಗಿದ್ದು
ಶೈವ-ನಾ-ದರೂ
ಶೈವ-ನೊಬ್ಬನ
ಶೈವ-ಪಂಗಡಕ್ಕೆ
ಶೈವ-ಪಂಥಕ್ಕೆ
ಶೈವ-ಪಂಥ-ಗಳ
ಶೈವ-ಪಂಥದ
ಶೈವ-ಪಂಥ-ವೊಂದರ
ಶೈವ-ಪದ್ಧತಿಯ
ಶೈವ-ಪರಂಪರೆಂಅ
ಶೈವ-ಪರಂಪ-ರೆಗೆ
ಶೈವ-ಪರಂಪ-ರೆಯು
ಶೈವಬ್ರಾಹ್ಮಣ
ಶೈವಬ್ರಾಹ್ಮಣ-ರಿ-ರು-ವಂತೆ
ಶೈವಬ್ರಾಹ್ಮಣರು
ಶೈವ-ಭಕ್ತರ
ಶೈವ-ಮಠದ
ಶೈವ-ಮಠ-ವಿತ್ತೆಂದು
ಶೈವ-ಮಠ-ವಿದ್ದು
ಶೈವ-ಮತ
ಶೈವ-ಮ-ತಕ್ಕೆ
ಶೈವ-ಮತ-ದವ-ರಾ-ಗಿದ್ದು
ಶೈವ-ಮತ-ದಿಂದ
ಶೈವ-ಮ-ತವು
ಶೈವ-ಮತಸ್ಥರು
ಶೈವ-ಮತಾವ-ಲಂಬಿ-ಗ-ಳಾಗಿದ್ದರು
ಶೈವ-ಯತಿ
ಶೈವ-ಯತಿ-ಗಳ
ಶೈವ-ಯತಿ-ಗ-ಳಿಗೆ
ಶೈವ-ಯತಿ-ಗಳು
ಶೈವ-ಯತಿಯ
ಶೈವ-ಯ-ತಿಯು
ಶೈವರ
ಶೈವ-ರನ್ನು
ಶೈವ-ರಾ-ಗಿದ್ದ
ಶೈವ-ರಾಗಿದ್ದರು
ಶೈವ-ರಿಗೂ
ಶೈವರು
ಶೈವರೂ
ಶೈವರೇ
ಶೈವರ್ಧ-ಮದ
ಶೈವ-ಳಾದ
ಶೈವ-ಳೆಂದೂ
ಶೈವ-ಶಾ-ಸನ
ಶೈವ-ಶಿಲ್ಪ-ಗಳು
ಶೈವ-ಸಂಸ್ಥೆ-ಗ-ಳಿಗೆ
ಶೈವ-ಸಮುಪಾ-ಸತೆ
ಶೈವಸ್ಥಾನ-ವನ್ನು
ಶೈವಾ-ಚಾರ್ಯರ
ಶೈವಾಧ್ವೈತಿ
ಶೈವಾ-ಲಿನಿ
ಶೈವಾ-ಲಿನೀ
ಶೈವೋಪಾಸಕ-ರೆಂದು
ಶೋತೆ
ಶೋಧ
ಶೋಧಿಸಲ್ಪಟ್ಟ
ಶೋಧಿಸಲ್ಪಟ್ಟಿವೆ
ಶೋಬಾರ್ಥ-ವಾಗಿಯೂ
ಶೋಭನ
ಶೋಭಾ
ಶೋಭಾ-ಕರ
ಶೋಭಾ-ಯ-ಮಾನ
ಶೋಭಾ-ಯ-ಮಾನ-ವಾದ
ಶೋಭಿಸು-ವಂತೆ
ಶೋಭೆ-ತ-ರುವ
ಶೋಳ-ಪಟ್ಟಣ
ಶೋಳ-ಪುರದ
ಶೋವನ
ಶೌಚ
ಶೌಚ-ಮಣಲೆ-ಯರುಂ
ಶೌರ್ಯ
ಶೌರ್ಯಕ್ಕೆ
ಶೌರ್ಯದಿಂ
ಶೌರ್ಯ-ದಿಂದ
ಶೌರ್ಯ-ದಿಮ
ಶೌರ್ಯ-ವನ್ನು
ಶೌರ್ಯಾಟೋಪ-ದೊಳು
ಶೌರ್ಯಾ-ಭರಣ
ಶೌರ್ಯ್ಯದಿಂ
ಶ್ಯಾನು-ಭೋಗ
ಶ್ಯಾಮಲಾ-ರತ್ನ-ಕು-ಮಾರಿ
ಶ್ಯಾಮಲಾ-ರತ್ನ-ಕು-ಮಾರಿ-ಯ-ವರು
ಶ್ಯಾವೆ
ಶ್ರಣ-ವನ-ಹಳ್ಳಿಯೇ
ಶ್ರತಿ-ಯೊಳ-ಗಣ
ಶ್ರದ್ಧೆ
ಶ್ರದ್ಧೆ-ಯನ್ನು
ಶ್ರಮಿ-ಸಿದ
ಶ್ರಮಿ-ಸಿರ-ಬ-ಹುದು
ಶ್ರಮಿಸುತ್ತಿದ್ದ-ನೆಂದು
ಶ್ರವಣ
ಶ್ರವ-ಣ-ಕಾಲ-ದಲ್ಲಿ
ಶ್ರವ-ಣ-ನ-ಹಳ್ಳಿ
ಶ್ರವ-ಣ-ಬೆಳಗೊಳ
ಶ್ರವ-ಣ-ಬೆಳಗೊಳಕ್ಕೆ
ಶ್ರವ-ಣ-ಬೆಳಗೊಳ-ಗಳು
ಶ್ರವ-ಣ-ಬೆಳಗೊಳದ
ಶ್ರವ-ಣ-ಬೆಳಗೊಳ-ದಂತೆಯೇ
ಶ್ರವ-ಣ-ಬೆಳಗೊಳ-ದಲ್ಲಿ
ಶ್ರವ-ಣ-ಬೆಳಗೊಳ-ದಲ್ಲಿ-ರುವ
ಶ್ರವ-ಣ-ಬೆಳಗೊಳ-ದಷ್ಟೇ
ಶ್ರವ-ಣ-ಬೆಳಗೊಳ-ವನ್ನು
ಶ್ರವ-ಣ-ಬೆಳಗೊಳವು
ಶ್ರವ-ಣ-ಬೆಳಗೊಳವೇ
ಶ್ರವ-ಣ-ಬೆಳಗೊಳ-ಶಾ-ಸನ-ದಲ್ಲಿ
ಶ್ರವ-ಣ-ಬೆಳ್ಗೊ-ಳದ
ಶ್ರವ-ಣ-ಸಂಘ
ಶ್ರಾವಕ-ನೊಬ್ಬನು
ಶ್ರಾವ-ಕರೂ
ಶ್ರಾವಣ
ಶ್ರಾವಣ-ಶನಿ-ವಾರ-ಗ-ಳಲ್ಲಿ
ಶ್ರಾವಿ-ಕೆಯರು
ಶ್ರಿ
ಶ್ರಿತಿಯ
ಶ್ರಿಮದ-ನಾದಿ
ಶ್ರಿಮ-ದಾನದಿ
ಶ್ರಿಮದ್ರಾಜ-ಗುರು
ಶ್ರಿಮನ್
ಶ್ರಿಮನ್ಮಹಾ
ಶ್ರಿಮನ್ಮಹಾಪ್ರಧಾನ
ಶ್ರಿಮಾನ್
ಶ್ರಿಯಂ
ಶ್ರಿವೈಷ್ಣ-ವ-ರಿಗೆ
ಶ್ರೀ
ಶ್ರೀಕಂಠ-ದೇವ
ಶ್ರೀಕಂಠ-ನಾಥ-ನೆಂದೂ
ಶ್ರೀಕಂಠ-ಶಾಸ್ತ್ರಿ
ಶ್ರೀಕಂಠ-ಶಾಸ್ತ್ರಿ-ಗಳ
ಶ್ರೀಕಂಠ-ಶಾಸ್ತ್ರಿ-ಗಳು
ಶ್ರೀಕಂಠ-ಶಾಸ್ತ್ರಿ-ಯ-ವರ
ಶ್ರೀಕಂಠ-ಶಾಸ್ತ್ರಿ-ಯ-ವರೂ
ಶ್ರೀಕಂಠ-ಶಾಸ್ತ್ರೀ
ಶ್ರೀಕಂಠೇಶ್ವರ
ಶ್ರೀಕರಣ
ಶ್ರೀಕರಣಂ
ಶ್ರೀಕರ-ಣಂಗಳು
ಶ್ರೀಕರ-ಣದ
ಶ್ರೀಕರ-ಣ-ದ-ಧಿಷ್ಟ-ಯಕ
ಶ್ರೀಕರ-ಣ-ದ-ಹೆಗ್ಗಡೆ
ಶ್ರೀಕರ-ಣಪ್ರ-ಮುಖ
ಶ್ರೀಕರ-ಣರ
ಶ್ರೀಕರ-ಣರು
ಶ್ರೀಕರ-ಣಾಗ್ರ-ಗಣ್ಯ
ಶ್ರೀಕರ-ಣಾಗ್ರಗಣ್ಯನೂ
ಶ್ರೀಕರ-ಣಾಗ್ರ-ಗಣ್ಯರು
ಶ್ರೀಕರ-ಣಾಧಿ-ಕಾರಿ
ಶ್ರೀಕರ-ಣಾಧಿ-ಪತಿ
ಶ್ರೀಕರ-ಣಿಕ
ಶ್ರೀಕಾಂತಯ್ಯನೇ
ಶ್ರೀಕಾಂತ್
ಶ್ರೀಕಾ-ಕುಳಂನ
ಶ್ರೀಕಾರ್ಯ
ಶ್ರೀಕಾರ್ಯಕ್ಕೆ
ಶ್ರೀಕಾರ್ಯ-ಗ-ಳನ್ನು
ಶ್ರೀಕಾರ್ಯ-ವನ್ನು
ಶ್ರೀಕಾರ್ಯ್ಯಮ್
ಶ್ರೀಕೃಷ್ಣ
ಶ್ರೀಕೃಷ್ಣ-ರಾಯ
ಶ್ರೀಕೇತ-ದಂಡಾಧಿಪಂ
ಶ್ರೀಕೈಲಾಸ-ಮುಡೆ-ಯಾರ್ಕ್ಕು
ಶ್ರೀಕೋವಿ-ರಾಜ
ಶ್ರೀಕೌಶಿಕ
ಶ್ರೀಗಂಧ
ಶ್ರೀಗಂಧದ
ಶ್ರೀಗಿರಿ
ಶ್ರೀಗೂ-ರನ-ಮಠ
ಶ್ರೀಚಾಮುಣ್ಡ-ರಾಜಂ
ಶ್ರೀಚಿಕವಂಗಲ-ದಲ್ಲಿ
ಶ್ರೀಜಯಂತಿ
ಶ್ರೀತೀರ್ಥ-ತಟಾಕ-ದಿಂದ
ಶ್ರೀದೇವಿ
ಶ್ರೀಧರ
ಶ್ರೀಧರಯ್ಯನ
ಶ್ರೀಧರಾ-ಚಾರ್ಯರು
ಶ್ರೀನಾರ-ಸಿಂಹ
ಶ್ರೀನಾರ-ಸಿಂಹ-ದೇವರು
ಶ್ರೀನಾರ-ಸಿಂಹ-ಪುರದ
ಶ್ರೀನಾ-ರಾಯಣ
ಶ್ರೀನಾ-ರಾಯ-ಣ-ದೇವರ
ಶ್ರೀನಾ-ರಾಯ-ಣ-ದೇವರು
ಶ್ರೀನಿಜ-ಬೋಧಪ್ರಭು-ಗಳ
ಶ್ರೀನಿ-ವಾರ-ಸೂಕ್ತಯೇ
ಶ್ರೀನಿ-ವಾಸ
ಶ್ರೀನಿ-ವಾಸಕ್ಷೇತ್ರದ
ಶ್ರೀನಿ-ವಾಸಕ್ಷೇತ್ರವೂ
ಶ್ರೀನಿ-ವಾಸ-ದೇವಾ-ಲಯ-ವನ್ನು
ಶ್ರೀನಿ-ವಾಸಧ್ವರಿ-ಯನ್ನು
ಶ್ರೀನಿ-ವಾಸನ
ಶ್ರೀನಿ-ವಾಸ-ನಿಗೆ
ಶ್ರೀನಿ-ವಾಸ-ಮೂರ್ತಿ
ಶ್ರೀನಿ-ವಾಸಯ್ಯನ-ವ-ರಿಂದ
ಶ್ರೀನಿ-ವಾಸ-ರಾಯರ
ಶ್ರೀನಿ-ವಾಸ-ರಾವು
ಶ್ರೀನಿ-ವಾಸಾಂಘ್ರಿ
ಶ್ರೀನಿ-ವಾಸಾ-ಚಾರಿ
ಶ್ರೀನಿ-ವಾಸಾ-ಚಾರಿಯ
ಶ್ರೀನಿ-ವಾಸಾ-ಚಾರ್ಯ-ರಿಗೆ
ಶ್ರೀನಿ-ವಾಸಾ-ಚಾರ್ಯ-ರೆಂಬ
ಶ್ರೀನಿ-ವಾಸಾಧ್ವ-ರಿಗೆ
ಶ್ರೀನಿ-ವಾಸಾಧ್ವರಿ-ಯನ್ನು
ಶ್ರೀನಿ-ವಾಸಾಧ್ವ-ರಿಯು
ಶ್ರೀನಿ-ವಾಸಾಧ್ವರೀಂದ್ರಾಯ
ಶ್ರೀನಿ-ವಾಸಾರ್ಯನ
ಶ್ರೀನಿ-ವಾಸ್ರ-ವರು
ಶ್ರೀನೀಲ-ಕಂಠ
ಶ್ರೀನೊಳಂಬ
ಶ್ರೀಪತೇರ್ವಾಮ-ನೇತ್ರ-ವಂಶಾಬ್ಧಿ
ಶ್ರೀಪದ-ಪುರ್ರ
ಶ್ರೀಪರ-ದೇಶಿ
ಶ್ರೀಪರು-ಷನು
ಶ್ರೀಪರ್ವ-ತಕ್ಕೆ
ಶ್ರೀಪಾದ
ಶ್ರೀಪಾದಕೆ
ಶ್ರೀಪಾ-ದಕ್ಕೆ
ಶ್ರೀಪಾದ-ಗಳ
ಶ್ರೀಪಾದದ
ಶ್ರೀಪಾದ-ವನ್ನು
ಶ್ರೀಪಾದ-ಸೇವ-ಕನು
ಶ್ರೀಪಾಲ
ಶ್ರೀಪಾಲ-ದೇವ
ಶ್ರೀಪಾಳ
ಶ್ರೀಪಾಳತ್ರೈ-ವಿದ್ಯ-ದೇವ-ರೆಂದು
ಶ್ರೀಪುರ
ಶ್ರೀಪುರದ
ಶ್ರೀಪುರ-ದಲ್ಲಿ
ಶ್ರೀಪುರ-ವಾಗಿ-ರ-ಬಹು-ದೆಂದು
ಶ್ರೀಪುರ-ವೆಂದು
ಶ್ರೀಪುರ-ಷನ
ಶ್ರೀಪುರುಷ
ಶ್ರೀಪುರುಷನ
ಶ್ರೀಪುರುಷ-ನನ್ನು
ಶ್ರೀಪುರುಷ-ನಿಗೆ
ಶ್ರೀಪುರುಷನು
ಶ್ರೀಪೃಥ್ವೀ
ಶ್ರೀಪೆರು-ಮಾಳೆ-ದೇವ-ದಂಣಾ-ಯಕರ
ಶ್ರೀಪ್ರಭಾ-ಚಂದ್ರ-ಸಿದ್ಧಾಂತ-ದೇವರ
ಶ್ರೀಬಣ್ಡಾರತ್ತಿಲೊಡುಕ್ಕಿನ
ಶ್ರೀಬಾಚಿ-ರಾಜಾಹ್ವಯಂ
ಶ್ರೀಬಾ-ಣದ
ಶ್ರೀಬಾಸ
ಶ್ರೀಬಾಸಣ್ಣನ
ಶ್ರೀಭಂಡಾರಕ್ಕೆ
ಶ್ರೀಭಂಡಾರದ
ಶ್ರೀಭಂಡಾರ-ದಿಂದ
ಶ್ರೀಭಂಡಾರ-ಮುದ್ರೆ
ಶ್ರೀಭಂಡಾರ-ವಿತ್ತು
ಶ್ರೀಭಂಡಾರವೂ
ಶ್ರೀಭಾಷ್ಯ
ಶ್ರೀಭಾಷ್ಯ-ಕಾರರು
ಶ್ರೀಭಾಷ್ಯಕ್ಕೆ
ಶ್ರೀಭಾಷ್ಯ-ವನ್ನು
ಶ್ರೀಭೂಮಿ
ಶ್ರೀಮಂಟಪ-ವನ್ನು
ಶ್ರೀಮಂತ
ಶ್ರೀಮಂತ-ನಿದ್ದಿರ-ಬೇಕು
ಶ್ರೀಮಂತಿಕೆ-ಗಾಗಿ
ಶ್ರೀಮಂನಯ-ಕೀರ್ತಿ
ಶ್ರೀಮಂನ್
ಶ್ರೀಮಠ-ವನ್ನು
ಶ್ರೀಮಠವು
ಶ್ರೀಮತಿ
ಶ್ರೀಮತು
ಶ್ರೀಮತೇ
ಶ್ರೀಮತ್
ಶ್ರೀಮತ್ಕೆಲ್ಲಂಗೆರೆಯ
ಶ್ರೀಮತ್ತಿರು-ಮಲಾಭಿಖ್ಯ-ದೀಕ್ಷಿತೇಂದ್ರಾತ್ಮ
ಶ್ರೀಮತ್ಪರಮ
ಶ್ರೀಮತ್ಪಶ್ಚಿಮ-ರಂಗ-ನಾಥ
ಶ್ರೀಮತ್ಪಶ್ಚಿಮ-ರಂಗ-ನಾಥ-ಮಹಿಷೀ
ಶ್ರೀಮತ್ಪಶ್ಚಿಮ-ರಂಗ-ಪಟ್ಟ-ಣ-ವರೇ
ಶ್ರೀಮತ್ಪಿ-ರಿಯ-ರಸಿ
ಶ್ರೀಮತ್ಪೆರ್ಗ್ಗಡೆ
ಶ್ರೀಮತ್ಪೋ-ಸಳ-ದೇವರು
ಶ್ರೀಮತ್ಸದ್ಭರ್ಯಕ್ಷೇತ್ರ
ಶ್ರೀಮತ್ಸರ್ವ-ನಮಸ್ಯದ
ಶ್ರೀಮತ್ಸಾಮಿ
ಶ್ರೀಮದ-ಕಳಂಕ
ಶ್ರೀಮದ-ಕಳಂಕಾನ್ವಯ
ಶ್ರೀಮದ-ಜಿತ-ಸೇನ
ಶ್ರೀಮದ-ನಾದಿ
ಶ್ರೀಮದ-ನಾದಿ-ಮಹಾಸ್ವಾಮಿಸ್ಥಾನಂ
ಶ್ರೀಮದ-ನಾದಿ-ಯಗ್ರ-ಹಾರಂ
ಶ್ರೀಮದ-ಶೇಷ
ಶ್ರೀಮದ-ಶೇಷ-ಮಹಾ-ಜನ-ಗ-ಳಿಂದ
ಶ್ರೀಮ-ದಾದಿ-ಚುಂಚನ-ಗಿರಿ-ನಿ-ಲಯ
ಶ್ರೀಮದುದ್ಭವ
ಶ್ರೀಮದುಭಯ
ಶ್ರೀಮದೇ-ಕಾಂತ
ಶ್ರೀಮದೇ-ಳಾ-ಚಾರ್ಯ್ಯರ
ಶ್ರೀಮದ್
ಶ್ರೀಮದ್ಅಗ್ರ-ಹಾರ
ಶ್ರೀಮದ್ಭಾಳೇಂದು
ಶ್ರೀಮದ್ರಾಜ-ಗುರು
ಶ್ರೀಮದ್ರಾಜಾ-ಧಿ-ರಾಜ
ಶ್ರೀಮದ್ರಾ-ರಾಜಾ-ಧಿ-ರಾಜ
ಶ್ರೀಮದ್ವೀರ
ಶ್ರೀಮನು
ಶ್ರೀಮನು-ಮಯ
ಶ್ರೀಮನು-ಮಯ-ವಿಸ್ವ-ಕರ್ಮ
ಶ್ರೀಮನು-ಮಹಾ
ಶ್ರೀಮನು-ಮಹಾಪ್ರಧಾನ
ಶ್ರೀಮನು-ಮಹಾ-ವಡ್ಡವ್ಯವ-ಹಾರಿ
ಶ್ರೀಮನು-ಮಹಾ-ಸಾಮಂತ
ಶ್ರೀಮನು-ಮಹಾ-ಸಾಮಂತರುಂ
ಶ್ರೀಮನ್
ಶ್ರೀಮನ್ನಯ-ಕೀರ್ತಿ
ಶ್ರೀಮನ್ನಯ-ಕೀರ್ತಿ-ಸಿದ್ಧಾಂತ
ಶ್ರೀಮನ್ನೊಳಂಬ
ಶ್ರೀಮನ್ಮಣಲ-ಯ-ರನ
ಶ್ರೀಮನ್ಮಹಾ
ಶ್ರೀಮನ್ಮಹಾ-ನಾಯಂಕಾ
ಶ್ರೀಮನ್ಮಹಾ-ನಾಯಕ
ಶ್ರೀಮನ್ಮಹಾ-ನಾಯ-ಕಾ-ಚಾರ್ಯ
ಶ್ರೀಮನ್ಮಹಾ-ನಾಳ್ಪ್ರಭು
ಶ್ರೀಮನ್ಮಹಾ-ಪಸಾಯಿತ
ಶ್ರೀಮನ್ಮಹಾ-ಪಸಾಯ್ತ
ಶ್ರೀಮನ್ಮಹಾಪ್ರಧಾನ
ಶ್ರೀಮನ್ಮಹಾಪ್ರಧಾನಂ
ಶ್ರೀಮನ್ಮಹಾಪ್ರಧಾನ-ನೆಂದು
ಶ್ರೀಮನ್ಮಹಾಪ್ರಧಾನಿ
ಶ್ರೀಮನ್ಮಹಾಪ್ರಧಾನೆ-ರೆಂದು
ಶ್ರೀಮನ್ಮಹಾಪ್ರಭು
ಶ್ರೀಮನ್ಮಹಾ-ಮಂಡ-ಲೇಶ್ವರ
ಶ್ರೀಮನ್ಮಹಾ-ಮಂಡ-ಲೇಶ್ವರಂ
ಶ್ರೀಮನ್ಮಹಾ-ಮಂಡ-ಲೇಶ್ವರ-ನೆಂದೇ
ಶ್ರೀಮನ್ಮಹಾ-ಮಂಡ-ಳೇಶ್ವರ
ಶ್ರೀಮನ್ಮಹಾ-ಮಹಿಮ-ನಪ್ಪ
ಶ್ರೀಮನ್ಮಹಾ-ರಾಜಾ-ಧಿ-ರಾಜ
ಶ್ರೀಮನ್ಮಹಾ-ರಾಯ
ಶ್ರೀಮನ್ಮಹಾ-ವಡ್ಡಬ್ಯವ-ಹಾರಿ
ಶ್ರೀಮನ್ಮಹಾ-ವೀರ-ರಾಜೇಂದ್ರ
ಶ್ರೀಮನ್ಮಹಾ-ಸಾಮಂತ
ಶ್ರೀಮನ್ಮಹಾ-ಸಾಮಂತನ
ಶ್ರೀಮನ್ಮಹಾ-ಸಾಮಂತ-ರಾದ
ಶ್ರೀಮನ್ಮಹಾ-ಸಾಮಂತಾಧಿ-ಪತಿ
ಶ್ರೀಮನ್ಮು-ಗಿಲ-ಕುಲ-ಕಮಳ
ಶ್ರೀಮನ್ಮುನಿ-ಚಂದ್ರ
ಶ್ರೀಮಪ್ರತಿಷ್ಟ-ವೀರಪ್ರಾಜ್ಯ-ರಾಜ್ಯ
ಶ್ರೀಮಲ್ಲಿ-ಕಾರ್ಜುನ-ಮಹಾ-ರಾಯರ
ಶ್ರೀಮಲ್ಲಿ-ಕಾರ್ಜುನೋ
ಶ್ರೀಮಾಚಿ-ಸೆಟ್ಟಿ
ಶ್ರೀಮಾಧವ
ಶ್ರೀಮಾರ-ಮಯ್ಯ
ಶ್ರೀಮುಖ
ಶ್ರೀಮೂಲ-ಗಣ
ಶ್ರೀಯ-ಗಾಮುಣ್ಡರು
ಶ್ರೀಯಜು-ಶಾಖಾಧ್ಯಾಯಿನೇ
ಶ್ರೀಯಾ
ಶ್ರೀಯಾದ-ವ-ಗಿರಿ-ಯಾದ
ಶ್ರೀಯಾದ-ವ-ನಾ-ರಾಯಣ
ಶ್ರೀಯುಂ
ಶ್ರೀಯುಳ್ಳಿನ
ಶ್ರೀಯುವಂ
ಶ್ರೀರಂಗ
ಶ್ರೀರಂಗಂ
ಶ್ರೀರಂಗ-ಐ-ದನೇ
ಶ್ರೀರಂಗಕ್ಕೆ
ಶ್ರೀರಂಗ-ಗಳ
ಶ್ರೀರಂಗದ
ಶ್ರೀರಂಗ-ದಲ್ಲಿ
ಶ್ರೀರಂಗ-ದಾಸ-ನೆಂದಿದ್ದು
ಶ್ರೀರಂಗ-ದಿಂದ
ಶ್ರೀರಂಗ-ದೇವ
ಶ್ರೀರಂಗ-ದೇವನ
ಶ್ರೀರಂಗ-ದೇವ-ರಾಯ
ಶ್ರೀರಂಗ-ದೇವ-ರಾಯನು
ಶ್ರೀರಂಗ-ಧಾಮಸ್ವಾಮಿಯ
ಶ್ರೀರಂಗನ
ಶ್ರೀರಂಗ-ನಾಥ
ಶ್ರೀರಂಗ-ನಾಥ-ದೇವರ
ಶ್ರೀರಂಗ-ನಾಥ-ದೇವ-ರಿಗೆ
ಶ್ರೀರಂಗ-ನಾಥನ
ಶ್ರೀರಂಗ-ನಾಯಕಿ
ಶ್ರೀರಂಗ-ನಾಯ-ಕಿ-ದೇವಿ-ಯರ
ಶ್ರೀರಂಗನು
ಶ್ರೀರಂಗ-ನೆಂದು
ಶ್ರೀರಂಗ-ಪಟಕ್ಕೆ
ಶ್ರೀರಂಗ-ಪಟ್ಟಣ
ಶ್ರೀರಂಗ-ಪಟ್ಟ-ಣಕ್ಕೆ
ಶ್ರೀರಂಗ-ಪಟ್ಟ-ಣ-ಗ-ಳನ್ನು
ಶ್ರೀರಂಗ-ಪಟ್ಟ-ಣ-ಗಳು
ಶ್ರೀರಂಗ-ಪಟ್ಟ-ಣದ
ಶ್ರೀರಂಗ-ಪಟ್ಟ-ಣ-ದಲು
ಶ್ರೀರಂಗ-ಪಟ್ಟ-ಣ-ದಲ್ಲಿ
ಶ್ರೀರಂಗ-ಪಟ್ಟ-ಣ-ದಲ್ಲಿದ್ದ
ಶ್ರೀರಂಗ-ಪಟ್ಟ-ಣ-ದಲ್ಲಿದ್ದ-ರೆಂದೂ
ಶ್ರೀರಂಗ-ಪಟ್ಟ-ಣ-ದಲ್ಲಿದ್ದಾಗ
ಶ್ರೀರಂಗ-ಪಟ್ಟ-ಣ-ದಿಂದ
ಶ್ರೀರಂಗ-ಪಟ್ಟ-ಣ-ವನ್ನು
ಶ್ರೀರಂಗ-ಪಟ್ಟ-ಣ-ವನ್ನೂ
ಶ್ರೀರಂಗ-ಪಟ್ಟ-ಣವು
ಶ್ರೀರಂಗ-ಪಟ್ಟ-ಣ-ವೆಂಬ
ಶ್ರೀರಂಗ-ಪಟ್ಟ-ಣ-ಸೀಮೆಯ
ಶ್ರೀರಂಗ-ಪಟ್ಟ-ಣಸ್ಥಳದ
ಶ್ರೀರಂಗ-ಪಟ್ಟ-ಣಾಭಿಧೇ
ಶ್ರೀರಂಗ-ಪಟ್ಟಣೇ
ಶ್ರೀರಂಗ-ಪುದ
ಶ್ರೀರಂಗ-ಪುರ
ಶ್ರೀರಂಗ-ಪುರದ
ಶ್ರೀರಂಗ-ಪುರ-ದಲ್ಲಿದ್ದು
ಶ್ರೀರಂಗ-ಪುರ-ದ-ವ-ರಿಗೆ
ಶ್ರೀರಂಗ-ಪುರ-ದ-ವರು
ಶ್ರೀರಂಗ-ಪುರ-ದಶ್ರೀ-ರಂಗ-ಪಟ್ಟಣ
ಶ್ರೀರಂಗ-ಪುರ-ವಾದ
ಶ್ರೀರಂಗ-ಪುರ-ವೆಂದಾಗಿ-ರ-ಬ-ಹುದು
ಶ್ರೀರಂಗ-ಮಂಟಪ-ವನ್ನು
ಶ್ರೀರಂಗ-ಮಂಟಪೇ
ಶ್ರೀರಂಗ-ಮಠದ
ಶ್ರೀರಂಗ-ಮಹಾ-ರಾಯರು
ಶ್ರೀರಂಗಮ್
ಶ್ರೀರಂಗ-ರಾಜ
ಶ್ರೀರಂಗ-ರಾಜ-ದೇವ
ಶ್ರೀರಂಗ-ರಾಜನ
ಶ್ರೀರಂಗ-ರಾಜನು
ಶ್ರೀರಂಗ-ರಾಜ-ಭಟ್ಟಃ
ಶ್ರೀರಂಗ-ರಾಜ-ಭಟ್ಟನ
ಶ್ರೀರಂಗ-ರಾಯ-ದೇವರು
ಶ್ರೀರಂಗ-ರಾಯನ
ಶ್ರೀರಂಗ-ರಾಯನು
ಶ್ರೀರಂಗ-ರಾಯ-ಮಹಾ-ರಾಯರು
ಶ್ರೀರಂಗ-ರಾಯರ
ಶ್ರೀರಂಗಾರ್ಯನ
ಶ್ರೀರಂಗೇ
ಶ್ರೀರರ್ದ್ಧನಾರೀನಟೇಶ್ವರಃ
ಶ್ರೀರಾಜ್ಯ-ವೆಂಬ
ಶ್ರೀರಾಮ
ಶ್ರೀರಾಮ-ಕೃಷ್ಣ-ಗುರು-ವಿನ
ಶ್ರೀರಾಮ-ಕೃಷ್ಣ-ದೇವರ
ಶ್ರೀರಾಮ-ಚಂದ್ರ-ದೇವ-ರಿಗೆ
ಶ್ರೀರಾಮ-ಚಂದ್ರ-ದೇವ-ರಿಗೆ-ಹೊಸ-ಹಳ್ಳಿಯ
ಶ್ರೀರಾಮ-ನವಮಿ
ಶ್ರೀರಾಮ-ನವಮಿ-ಯಂದು
ಶ್ರೀರಾಮ-ನಾಥ
ಶ್ರೀರಾಮ-ನಾಥ-ದೇವರ
ಶ್ರೀರಾಮ-ಭಟ್ಟನ್
ಶ್ರೀರಾಮ-ರಾಜನು
ಶ್ರೀರಾಮ-ಸೀತಾ-ದೇವ-ರಿಗೆ
ಶ್ರೀರಾಮ-ಸೀತಾ-ಪುರದ
ಶ್ರೀರಾಮ-ಸೀತಾ-ಪುರ-ವೆಂಬ
ಶ್ರೀರಾಮಸ್ವಾಮಿಯು
ಶ್ರೀರಾಮಾಖ್ಯಸ್ಯ
ಶ್ರೀರಾಮಾ-ನುಜ
ಶ್ರೀರಾಮಾ-ನುಜ-ರಲ್ಲಿ
ಶ್ರೀರಾಮಾ-ನು-ಜಾಂಘ್ರಿ
ಶ್ರೀರಾಮಾ-ಭಟ್ಟನ
ಶ್ರೀರಾ-ಮಾ-ಯಣ
ಶ್ರೀರಾಮೇಶ್ವರ
ಶ್ರೀಲಕುಮಿ
ಶ್ರೀವತ್ಸ
ಶ್ರೀವಧು
ಶ್ರೀವರ್ಧ-ದೇವ
ಶ್ರೀವಲ್ಲ-ಭ-ನೆಂಬ
ಶ್ರೀವಿಕ್ರಮನ
ಶ್ರೀವಿಜಯ
ಶ್ರೀವಿಜಯನು
ಶ್ರೀವಿಜಯ-ಹೇಮ-ಸೇನನ
ಶ್ರೀವಿನ-ಯಾ-ದಿತ್ಯ-ಪೊಯ್ಸಳ-ನೆ-ರೆಯಂಗ
ಶ್ರೀವಿಷ್ಣು-ಭೂ-ಪಾಳಕಂ
ಶ್ರೀವಿಷ್ಣು-ವರ್ಧನ-ದೇವರ
ಶ್ರೀವೀರಪ್ರತಾಪ
ಶ್ರೀವೀರ-ರಾಮ-ದೇವ-ರಾಯರು
ಶ್ರೀವುರ
ಶ್ರೀವುರ-ದಲ್ಲಿ
ಶ್ರೀವುರ-ಮಂಗಲದ
ಶ್ರೀವುರ-ಮಂಗಲಶ್ರೀ-ರಂಗ-ಪಟ್ಟಣ
ಶ್ರೀವೈಷ್ಣವ
ಶ್ರೀವೈಷ್ಣ-ವ-ಕೇಂದ್ರ-ವಾಗಿ
ಶ್ರೀವೈಷ್ಣ-ವಕ್ಷೇತ್ರ
ಶ್ರೀವೈಷ್ಣ-ವಕ್ಷೇತ್ರ-ಗ-ಳಾದ
ಶ್ರೀವೈಷ್ಣ-ವಕ್ಷೇತ್ರ-ವಾದ
ಶ್ರೀವೈಷ್ಣ-ವ-ದೀಕ್ಷೆ
ಶ್ರೀವೈಷ್ಣ-ವ-ದೇವಾ-ಲಯ-ಗಳು
ಶ್ರೀವೈಷ್ಣ-ವ-ಧರ್ಮ
ಶ್ರೀವೈಷ್ಣ-ವ-ಧರ್ಮದ
ಶ್ರೀವೈಷ್ಣ-ವ-ಧರ್ಮವೂ
ಶ್ರೀವೈಷ್ಣ-ವ-ನಾದ
ಶ್ರೀವೈಷ್ಣ-ವನೂ
ಶ್ರೀವೈಷ್ಣ-ವಪ್ರಿಯನ್
ಶ್ರೀವೈಷ್ಣ-ವ-ಮಠ-ವಿತ್ತೆಂದು
ಶ್ರೀವೈಷ್ಣ-ವ-ಮತ
ಶ್ರೀವೈಷ್ಣ-ವರ
ಶ್ರೀವೈಷ್ಣ-ವ-ರನ್ನು
ಶ್ರೀವೈಷ್ಣ-ವ-ರಲ್ಲಿ
ಶ್ರೀವೈಷ್ಣ-ವ-ರಾಣೆ-ಯನ್ನು
ಶ್ರೀವೈಷ್ಣ-ವ-ರಾದ
ಶ್ರೀವೈಷ್ಣ-ವ-ರಿ-ಗಾಗಿ
ಶ್ರೀವೈಷ್ಣ-ವ-ರಿಗೆ
ಶ್ರೀವೈಷ್ಣ-ವ-ರಿದ್ದರು
ಶ್ರೀವೈಷ್ಣ-ವ-ರಿದ್ದ-ರೆಂದೂ
ಶ್ರೀವೈಷ್ಣ-ವ-ರಿರ-ಬ-ಹುದು
ಶ್ರೀವೈಷ್ಣ-ವರು
ಶ್ರೀವೈಷ್ಣ-ವ-ರು-ಗಳ
ಶ್ರೀವೈಷ್ಣ-ವ-ರು-ಗ-ಳಲ್ಲಿ
ಶ್ರೀವೈಷ್ಣ-ವ-ರು-ಗ-ಳಿಂದ
ಶ್ರೀವೈಷ್ಣ-ವ-ರು-ಗ-ಳಿಗೆ
ಶ್ರೀವೈಷ್ಣ-ವರೇ
ಶ್ರೀವೈಷ್ಣ-ವ-ರೇ-ಹೆ-ಸರಿ-ಸಿದೆ
ಶ್ರೀವೈಷ್ಣ-ವಸ್ಥಳ-ವಾದ
ಶ್ರೀವೊಪ್ಪ-ಣದಿ-ಯಾಗ್ರ-ಹಾರಂ
ಶ್ರೀವೊಪ್ಪ-ಣಾದಿ
ಶ್ರೀಶಂ
ಶ್ರೀಶೈಲ
ಶ್ರೀಶೈಲಕ್ಕೆ
ಶ್ರೀಶೈಲದ
ಶ್ರೀಶೈಲ-ದಲ್ಲಿ
ಶ್ರೀಶೈಲ-ದಿಂದ
ಶ್ರೀಶೈಲ-ದೊಂದಿಗೆ
ಶ್ರೀಶೈಲ-ಪೂರ್ಣ
ಶ್ರೀಶೈಲ-ಪೂರ್ಣರು
ಶ್ರೀಶೈಲ-ಯಾತ್ರೆ
ಶ್ರೀಶೈಲ-ವಂಶಕ್ಕೆ
ಶ್ರೀಶೈಲಾರ್ಯ
ಶ್ರೀಶೈಲಾರ್ಯರ
ಶ್ರೀಶೈಲಾರ್ಯರು
ಶ್ರೀಶ್ರೀಶ್ರೀಶ್ರೀಶ್ರೀ
ಶ್ರೀಹಸ್ತದ
ಶ್ರೀಹಸ್ತ-ದಲು
ಶ್ರೀಹಸ್ತ-ದೊಪ್ಪ-ವನ್ನು
ಶ್ರೀಹೋಸಲ-ನಾಡಿನ
ಶ್ರೀಹ್ರೀಧೃತಿದ್ಧಾರ್ಯತಾಂ
ಶ್ರುತ
ಶ್ರುತಂ
ಶ್ರುತ-ಕೀರ್ತಿ
ಶ್ರುತ-ಕೀರ್ತಿ-ದೇವರು
ಶ್ರುತ-ತ-ಕೇವಲಿ
ಶ್ರುತ-ಧ-ರರು
ಶ್ರುತಿ-ಕೀರ್ತಿ-ಪಂಡಿತ
ಶ್ರುತಿ-ಗಿರಿ-ಯಲ್ಲಿ
ಶ್ರುತಿಯ
ಶ್ರುತಿ-ಯನ್ನು
ಶ್ರುತಿಯು
ಶ್ರುತಿಶ್ರೋತ್ರಿ-ಯೂರು
ಶ್ರೇಣಿ
ಶ್ರೇಣಿ-ಗ-ಳಾಗಿದ್ದು
ಶ್ರೇಣಿಯ
ಶ್ರೇಣೀಕೃತ
ಶ್ರೇಣೀ-ಮದ-ವಾರಣ
ಶ್ರೇಯ-ವಾಗ-ಬೇಕೆಂದು
ಶ್ರೇಯಾಂಶ
ಶ್ರೇಷ್ಠ
ಶ್ರೇಷ್ಠ-ನಾಗಿದ್ದನು
ಶ್ರೇಷ್ಠ-ನಾದ
ಶ್ರೇಷ್ಠ-ನಾದ-ವನು
ಶ್ರೇಷ್ಠ-ರಾದ
ಶ್ರೇಷ್ಠ-ವಾದು-ದೆಂದು
ಶ್ರೈಶೈಲಕ್ಕೆ
ಶ್ರೋತ್ರೀಯ
ಶ್ರೋತ್ರೀಯ-ವಾಗಿ
ಶ್ರೋತ್ರೀ-ಯಸ್ಯ
ಶ್ರೌತ
ಶ್ಲೇಷೆ-ಯಿಂದ
ಶ್ಲೋಕ-ಗಳು
ಶ್ಲೋಕದ
ಶ್ಲೋಕ-ದಲ್ಲಿ
ಶ್ಲೋಕ-ವಿದೆ
ಶ್ಲೋಕವು
ಶ್ಲೋಕವೂ
ಶ್ಲೋಕಾನ್
ಶ್ವರ
ಶ್ವಾಣ-ಪೊತ್ತರ
ಷ
ಷಟ್ಕರ್ಮ-ಗ-ಳನ್ನು
ಷಟ್ದರು-ಷನ
ಷಟ್ಸ್ವಪಿ
ಷಡಕ್ಷ-ರಿಯ
ಷಡಕ್ಷ-ರಿಯು
ಷಣ್ಣ-ವತಿ
ಷಣ್ಣ-ವತಿ-ಸಹಸ್ರ
ಷಣ್ಣ-ವತಿ-ಸಹಸ್ರ-ವಿಷಯ
ಷಣ್ಮುಖ
ಷಣ್ಮುಖ-ಪಂಡಿ-ತರ
ಷಫ್ಕತ್
ಷರತ್ತಿನ
ಷರತ್ತು
ಷರತ್ತು-ಗಳೊಡನೆ
ಷಷ್ಟ
ಷಷ್ಠಿ-ಯಂದು
ಷಹಬಾಜ್
ಷಹಾ
ಷೇಕದ
ಷೇಕ್
ಷೋಡಶ
ಷೋಡಶೋಪ-ಚಾರ
ಸ
ಸಂ
ಸಂಕ-ಜೀಯ
ಸಂಕಡಿಸಂನಾಹ
ಸಂಕಮ-ದೇವನ
ಸಂಕ-ಮನು
ಸಂಕರ
ಸಂಕರ-ಗೌಡನ
ಸಂಕ-ರಪ್ಪ
ಸಂಕ-ರಾಸಿ
ಸಂಕರು-ಷ-ಣನ
ಸಂಕಲನ
ಸಂಕಲ್ಪಿಸಿ-ದನು
ಸಂಕ-ಹಳ್ಳಿ
ಸಂಕಿಯರ
ಸಂಕಿಯ-ರ-ಕುಲ-ತಿಲಕ
ಸಂಕಿರಣ-ದಲ್ಲಿ
ಸಂಕಿರಣ-ವನ್ನು
ಸಂಕೀ-ದೇವನ
ಸಂಕೀರ್ಣ
ಸಂಕೀರ್ಣ-ವಾಗಿ-ರು-ವಂತೆ
ಸಂಕೇತ-ವಾಗಿ
ಸಂಕೇತಾಕ್ಷರ-ಗ-ಳನ್ನು
ಸಂಕೇತಾಕ್ಷರ-ಗ-ಳಲ್ಲಿ
ಸಂಕೇತಾಕ್ಷರ-ಗಳು
ಸಂಕೋಚ-ದಾಯಿ
ಸಂಕ್ರಮಣ
ಸಂಕ್ರಾಂತಿಯ
ಸಂಕ್ಷಿಪ್ತ
ಸಂಕ್ಷಿಪ್ತ-ಗೊಳಿ-ಸಲು
ಸಂಕ್ಷಿಪ್ತ-ಗೊ-ಳಿಸಿ
ಸಂಕ್ಷಿಪ್ತ-ಗೊ-ಳಿಸಿ-ದರೆ
ಸಂಕ್ಷಿಪ್ತ-ತೆ-ಯಲ್ಲಿ
ಸಂಕ್ಷಿಪ್ತ-ವಾಗಿ
ಸಂಕ್ಷಿಸುತ್ತಾರೆ
ಸಂಕ್ಷೇಪ
ಸಂಕ್ಷೇಪ-ಗೊ-ಳಿಸಿ
ಸಂಕ್ಷೇಪ-ವಾಗಿ
ಸಂಕ್ಷೇಪಿಸ-ಬಹು-ದಾ-ಗಿತ್ತು
ಸಂಕ್ಷೇಪಿಸಿದ್ದೇನೆ
ಸಂಕ್ಷೇಪಿ-ಸುವ
ಸಂಕ್ಷೇಪಿ-ಸು-ವುದು
ಸಂಖರ
ಸಂಖರ-ವಚ್ಚ-ಪಣಿ-ಕರು
ಸಂಖ್ಯಾ
ಸಂಖ್ಯಾ-ನುಕ್ರಮಾನುಗಾಃ
ಸಂಖ್ಯೆ
ಸಂಖ್ಯೆ-ಗ-ಳನ್ನು
ಸಂಖ್ಯೆ-ಗಳು
ಸಂಖ್ಯೆ-ಯನ್ನು
ಸಂಖ್ಯೆ-ಯಲಿ
ಸಂಖ್ಯೆ-ಯಲ್ಲಿ
ಸಂಖ್ಯೆ-ಯಲ್ಲಿದ್ದರೂ
ಸಂಖ್ಯೆ-ಯಲ್ಲಿದ್ದ-ರೆಂಬುದು
ಸಂಖ್ಯೆ-ಯಲ್ಲಿದ್ದು
ಸಂಖ್ಯೆ-ಯಲ್ಲಿವೆ
ಸಂಖ್ಯೆಯೂ
ಸಂಖ್ಯೆಯೇ
ಸಂಗ
ಸಂಗಂ
ಸಂಗಡ
ಸಂಗತಿ
ಸಂಗ-ತಿ-ಯಾಗಿದೆ
ಸಂಗ-ತಿ-ಯಾಗುತ್ತದೆ
ಸಂಗ-ತಿಯೇ
ಸಂಗ-ನ-ಬಸವನ
ಸಂಗಮ
ಸಂಗ-ಮ-ಕು-ಮಾರ-ನಾದ
ಸಂಗ-ಮಕ್ಷೇತ್ರ-ದಲ್ಲಿತ್ತೆಂದು
ಸಂಗ-ಮದ
ಸಂಗ-ಮನ
ಸಂಗ-ಮ-ನಿಂದ
ಸಂಗ-ಮ-ನಿಗೆ
ಸಂಗ-ಮರ
ಸಂಗ-ಮ-ರಾಯ
ಸಂಗ-ಮ-ವಂಶದ
ಸಂಗ-ಮ-ವಂಶಾ-ವಳಿ-ಯನ್ನು
ಸಂಗ-ಮ-ಸೋ-ದರರು
ಸಂಗ-ಮಾಲೆ
ಸಂಗ-ಮೇಶ್ವರ
ಸಂಗ-ಮೇಶ್ವರ-ಪುರ-ವಾದ
ಸಂಗ-ಮೇಶ್ವರ-ಪುರ-ವೆಂಬ
ಸಂಗ-ಮೇಶ್ವರ-ರಾಯ
ಸಂಗ-ರಕೆ
ಸಂಗ-ರ-ಮೇರು-ಕೇತ-ರಥಿನೀ-ಪತಿ-ಗೀಗೆ
ಸಂಗಾ-ಪುರ
ಸಂಗಾ-ಪುರದ
ಸಂಗೀತ
ಸಂಗೀತ-ದಲ್ಲಿ
ಸಂಗೀನ್
ಸಂಗ್ರಹ
ಸಂಗ್ರಹಕ್ಕೆ
ಸಂಗ್ರಹಣೆ
ಸಂಗ್ರಹ-ದಲ್ಲಿ
ಸಂಗ್ರಹ-ವಾಗಿ
ಸಂಗ್ರಹ-ವಾಗುತ್ತದೆ
ಸಂಗ್ರಹ-ವಾಗುತ್ತಿದ್ದ
ಸಂಗ್ರಹಾ-ಲ-ಯವು
ಸಂಗ್ರಹಿ-ಸ-ಬಹದು
ಸಂಗ್ರಹಿ-ಸ-ಬ-ಹುದು
ಸಂಗ್ರಹಿಸಿ
ಸಂಗ್ರಹಿ-ಸಿ-ಕೊಟ್ಟಿದ್ದಾರೆ
ಸಂಗ್ರಹಿ-ಸಿ-ದರು
ಸಂಗ್ರಹಿ-ಸಿದ್ದಾರೆ
ಸಂಗ್ರಹಿ-ಸಿದ್ದೇನೆ
ಸಂಗ್ರಹಿ-ಸುತ್ತಿದ್ದರು
ಸಂಗ್ರಹಿ-ಸುವ
ಸಂಗ್ರಾಮ
ಸಂಗ್ರಾಮ-ಭೀಮ-ಯೆಂಬ
ಸಂಗ್ರಾಮ-ರಂಗ
ಸಂಗ್ರಾಮ-ರಾಮ
ಸಂಗ್ರಾಮ-ಸಹಸ್ರ-ಬಾಹು
ಸಂಘ
ಸಂಘಕ್ಕೆ
ಸಂಘ-ಗಳ
ಸಂಘ-ಗ-ಳನ್ನು
ಸಂಘ-ಗ-ಳಲ್ಲಿ
ಸಂಘ-ಗ-ಳಾಗಿಯೂ
ಸಂಘ-ಗಳು
ಸಂಘ-ಟಿ-ಸಲು
ಸಂಘ-ಟಿಸಿ
ಸಂಘಟ್ಟ
ಸಂಘ-ಡಿಸ್ಟ್ರಿಕ್ಟ್
ಸಂಘದ
ಸಂಘ-ದಲ್ಲಿ
ಸಂಘ-ದ-ವರು
ಸಂಘ-ವನ್ನು
ಸಂಘ-ವನ್ನೂ
ಸಂಘ-ವಾಗಲೀ
ಸಂಘ-ವಾಗಿ-ರ-ಬ-ಹುದು
ಸಂಘವೂ
ಸಂಘ-ವೆಂದೂ
ಸಂಘ-ವೆಂದೇ
ಸಂಘ-ಸಂಸ್ಥೆ-ಗ-ಳಿಗೆ
ಸಂಚರಿಸಿ
ಸಂಚರಿಸಿದ್ದರು
ಸಂಚಾರ
ಸಂಚಾರ-ಗಳಿಗೂ
ಸಂಚಿಕೆ-ಗ-ಳಲ್ಲಿ
ಸಂಚಿಗ
ಸಂಚಿತ
ಸಂಚಿಯ
ಸಂಜಾತಂ
ಸಂಜಾತ-ನಾ-ಗಿದ್ದು
ಸಂಜಾತ-ನಾದ
ಸಂಜೆ
ಸಂಜೆ-ಮಠ
ಸಂಜ್ಞಿಕೇ
ಸಂಡೂರು
ಸಂಣ-ನರ-ಕೋಟೆಯ
ಸಂತತಂ
ಸಂತತಿ
ಸಂತತಿ-ಗಳ
ಸಂತ-ತಿಯ
ಸಂತತಿ-ಯಲ್ಲಿ
ಸಂತತಿ-ಯ-ವನೋ
ಸಂತನು
ಸಂತರು
ಸಂತರ್ಪ-ಣೆಯ
ಸಂತಾನ
ಸಂತಾನಕ್ಕಾಗಿ
ಸಂತಾನ-ಗಾಮಿ
ಸಂತಾನ-ವಾಗಿ
ಸಂತೆ
ಸಂತೆ-ಗ-ಳಲ್ಲಿ
ಸಂತೆಯ
ಸಂತೆ-ಯ-ಕರ
ಸಂತೆ-ಯ-ಕರದ
ಸಂತೆ-ಯನ್ನು
ಸಂತೆ-ಯಾ-ಗಿತ್ತು
ಸಂತೆ-ಯಿಂದ
ಸಂತೆಯು
ಸಂತೆ-ಶಿವರ
ಸಂತೇ-ಬಾಚ-ಹಳ್ಳಿ
ಸಂತೇ-ಬಾಚ-ಹಳ್ಳಿಯ
ಸಂತೇ-ಬಾಚ-ಹಳ್ಳಿ-ಯನ್ನು
ಸಂತೇ-ಬಾಚ-ಹಳ್ಳಿ-ಯಲ್ಲಿ
ಸಂತೇ-ಬಾಚ-ಹಳ್ಳಿ-ಯಲ್ಲಿದೆ
ಸಂತೇ-ಬಾಚ-ಹಳ್ಳಿ-ಯಲ್ಲಿ-ರುವ
ಸಂತೇ-ಬಾಚ-ಹಳ್ಳಿ-ಯ-ವನು
ಸಂತೋಷದಿಂ
ಸಂತೋಷ-ದಿಂದ
ಸಂತೋಷ-ಪಟ್ಟ-ನೆಂದು
ಸಂತೋಷ-ವಾಗು-ವಂತೆ
ಸಂತೋಷ-ವಾಯಿತು
ಸಂತ್ತಿಷ್ಟ
ಸಂತ್ರಾಸಿ-ನೃ-ಪಾ-ಪದಃ
ಸಂಥೆಯ
ಸಂಥೆ-ಶಾ-ಸನ
ಸಂದ
ಸಂದನು
ಸಂದ-ಬಳಿಕ್ಕ
ಸಂದರು
ಸಂದರ್ಭಂ
ಸಂದರ್ಭಕ್ಕೆ
ಸಂದರ್ಭ-ಗ-ಳಲ್ಲಿ
ಸಂದರ್ಭ-ಗ-ಳಲ್ಲಿಯೂ
ಸಂದರ್ಭ-ದಲಿ
ಸಂದರ್ಭ-ದಲ್ಲಿ
ಸಂದರ್ಭ-ದಲ್ಲೂ
ಸಂದರ್ಭ-ದಲ್ಲೇ
ಸಂದರ್ಭೋಚಿತ-ವಾಗಿ
ಸಂದರ್ಶಿ-ಸಿರ
ಸಂದ-ಳಾದಿವಂ
ಸಂದಳು
ಸಂದಾಯ
ಸಂದಾಯ-ವಾಯಿ-ತೆಂದು
ಸಂದಿ
ಸಂದಿದೆ
ಸಂದಿ-ರ-ಬ-ಹುದು
ಸಂದಿ-ರುವು-ದ-ರಿಂದ
ಸಂದು
ಸಂದೇಶ
ಸಂದ್ರ
ಸಂಧಾನ
ಸಂಧಿ
ಸಂಧಿಗೆ
ಸಂಧಿ-ವಿಗ್ರಹಿ
ಸಂಧಿಸಿ
ಸಂನಾಹ-ಮಾವ-ನಂಕಕಾಱ
ಸಂನಿಧಿ-ಯಲಿ
ಸಂನುತೆ
ಸಂನ್ನುತ
ಸಂನ್ಮಾರ್ಗ್ಗ
ಸಂಪಂನರುಂ
ಸಂಪಂನರು-ಮಪ್ಪ
ಸಂಪಟಕ್ಕೆ
ಸಂಪಟು-ಗಳ
ಸಂಪತ್
ಸಂಪತ್ಕರ
ಸಂಪತ್ಕರ-ನಾ-ರಾಯಣ
ಸಂಪತ್ಕರ-ನಾ-ರಾಯ-ಣ-ದೇವರು
ಸಂಪತ್ಕು-ಮಾರ
ಸಂಪತ್ಕು-ಮಾರ-ನಾದ
ಸಂಪತ್ಕು-ಮಾರರ
ಸಂಪತ್ತಿಗೆ
ಸಂಪತ್ತು
ಸಂಪತ್ಸಾ-ರಸ್ವತ
ಸಂಪದಂ
ಸಂಪ-ದದಿಂ
ಸಂಪದ್ಭರಿತ
ಸಂಪದ್ಭರಿತ-ವಾಗಿ
ಸಂಪದ್ಭರಿತ-ವಾಗಿ-ದೆಯೋ
ಸಂಪ-ನನ್ನನುಂ
ಸಂಪನ್ನ
ಸಂಪನ್ನಂ
ಸಂಪನ್ನ-ನಪ್ಪ
ಸಂಪನ್ನ-ರಾದ
ಸಂಪನ್ನ-ರು-ಮಪ್ಪ
ಸಂಪನ್ನಾನ್
ಸಂಪನ್ನೆ
ಸಂಪನ್ಮೂಲ-ಗಳಿ-ರುವ
ಸಂಪರ್ಕ
ಸಂಪರ್ಕ-ಕಲ್ಪಿಸುತ್ತಿದ್ದವು
ಸಂಪಾಕ-ದರು
ಸಂಪಾದ-ಕರು
ಸಂಪಾದ-ಕರೂ
ಸಂಪಾದ-ನೆ-ಯಿಂದ
ಸಂಪಾದಿಸಿ
ಸಂಪಾದಿ-ಸಿದ
ಸಂಪಾದಿ-ಸಿದ್ದ
ಸಂಪಾದಿಸಿ-ರುವ
ಸಂಪುಟ
ಸಂಪುಟ-ಗಳ
ಸಂಪುಟ-ಗ-ಳನ್ನು
ಸಂಪುಟ-ಗ-ಳಲ್ಲಿ
ಸಂಪುಟ-ಗಳಲ್ಲಿ-ರುವ
ಸಂಪುಟ-ಗ-ಳಿಂದ
ಸಂಪುಟ-ಗ-ಳಿಗೆ
ಸಂಪುಟ-ಗಳು
ಸಂಪುಟ-ಗಳೂ
ಸಂಪುಟದ
ಸಂಪುಟ-ದಲ್ಲಿ
ಸಂಪೂಜಿತೆ
ಸಂಪೂರ್ಣ
ಸಂಪೂರ್ಣ-ವಾಗಿ
ಸಂಪ್ಪಂನ್ನ
ಸಂಪ್ರತಿ
ಸಂಪ್ರ-ದಾಯ
ಸಂಪ್ರ-ದಾಯದ
ಸಂಪ್ರ-ದಾಯ-ದಲ್ಲಿ
ಸಂಪ್ರ-ದಾಯ-ದ-ವ-ರಾದ್ದ-ರಿಂದ
ಸಂಪ್ರ-ದಾಯ-ದ-ವರೇ
ಸಂಪ್ರ-ದಾಯ-ವನ್ನು
ಸಂಪ್ರ-ದಾ-ಯವು
ಸಂಪ್ರದಾ-ಯವೂ
ಸಂಪ್ರ-ದಾಯಸ್ಥ-ನಾದ
ಸಂಪ್ರಾಪ್ಯ
ಸಂಪ್ರೀತ-ನಾದ
ಸಂಬಂಧ
ಸಂಬಂಧ-ಗಳ
ಸಂಬಂಧ-ಗ-ಳನ್ನು
ಸಂಬಂಧ-ಗಳಿ
ಸಂಬಂಧ-ಗಳು
ಸಂಬಂಧದ
ಸಂಬಂಧ-ದಲ್ಲಿ
ಸಂಬಂಧ-ದಿಂದ
ಸಂಬಂಧ-ಪಟ್ಟ
ಸಂಬಂಧ-ಪಟ್ಟಂತೆ
ಸಂಬಂಧ-ಪಟ್ಟವ
ಸಂಬಂಧ-ಪಟ್ಟ-ವ-ರಾಗಿದ್ದಾರೆ
ಸಂಬಂಧ-ಪಟ್ಟಿ-ರ-ಬೇಕು
ಸಂಬಂಧ-ಪಟ್ಟು-ದಾಗಿ
ಸಂಬಂಧ-ವನ್ನು
ಸಂಬಂಧ-ವನ್ನೂ
ಸಂಬಂಧ-ವಾಗಿ
ಸಂಬಂಧ-ವಾದ
ಸಂಬಂಧ-ವಿದೆ
ಸಂಬಂಧ-ವಿದೆಯೇ
ಸಂಬಂಧ-ವಿರ-ಬ-ಹುದು
ಸಂಬಂಧ-ವಿಲ್ಲ
ಸಂಬಂಧ-ವಿಲ್ಲ-ವೆಂದು
ಸಂಬಂಧ-ವೇನು
ಸಂಬಂಧ-ವೇ-ನೆಂದು
ಸಂಬಂಧಿ
ಸಂಬಂಧಿ-ಗ-ಳಾದ
ಸಂಬಂಧಿ-ದಂತೆ
ಸಂಬಂಧಿ-ಯಾದ
ಸಂಬಂಧಿ-ಸ-ಸಿದ
ಸಂಬಂಧಿ-ಸಿದ
ಸಂಬಂಧಿ-ಸಿ-ದಂತಹ
ಸಂಬಂಧಿ-ಸಿ-ದಂತೆ
ಸಂಬಂಧಿ-ಸಿ-ದ-ವರ
ಸಂಬಂಧಿ-ಸಿದೆ
ಸಂಬಂಧಿ-ಸಿದ್ದಾಗಿದೆ
ಸಂಬಂಧಿ-ಸಿದ್ದಿರ-ಬ-ಹುದು
ಸಂಬಂಧಿ-ಸಿದ್ದು
ಸಂಬಂಧಿ-ಸಿ-ರ-ಬಹು-ದೆಂದು
ಸಂಬಂಮನ
ಸಂಬಳ
ಸಂಬಳ-ಕೊಡ-ಲಾ-ಗದೆ
ಸಂಬಳಪ್ರಾಪ್ತಿ
ಸಂಬಳ-ವನ್ನು
ಸಂಬಾರ
ಸಂಬುವ-ಗ-ಉಡ
ಸಂಬುವ-ಗವುಡ
ಸಂಬೋಧ-ನೆಗೆ
ಸಂಬೋಧಿಸ-ಲಾಗಿದೆ
ಸಂಬೋಧಿಸಿ
ಸಂಬೋಧಿಸಿ-ರುವ
ಸಂಬೋಧಿ-ಸಿಲ್ಲ
ಸಂಬೋಧಿ-ಸಿವೆ
ಸಂಭವ
ಸಂಭವ-ನೀಯ-ವಲ್ಲ
ಸಂಭವ-ರಾಯ
ಸಂಭವವೇ
ಸಂಭವಿಸಲಿ-ರುವ
ಸಂಭವಿ-ಸಿತು
ಸಂಭವಿ-ಸಿದ
ಸಂಭವಿ-ಸಿದವು
ಸಂಭವಿ-ಸಿದುದು
ಸಂಭಾಜಿಯು
ಸಂಭಾವನೆ
ಸಂಭಾವ-ನೆ-ಯಲ್ಲಿ
ಸಂಭು
ಸಂಭು-ದೇವ
ಸಂಭು-ದೇವ-ನೆಂದು
ಸಂಭು-ದೇವ-ರಿಗೆ
ಸಂಭು-ರಾಯ
ಸಂಭು-ವ-ಗವುಡ
ಸಂಭು-ವ-ರಾಯ
ಸಂಭೂತದ
ಸಂಮಂಧ
ಸಂಮಟಿ-ಭಾಗ
ಸಂಮಟಿ-ಭಾಗ-ಭೂ-ಪತಿ
ಸಂಮಿತಿ-ಯನ್ನು
ಸಂಯುಕ್ತೋ
ಸಂಯುತಾನ್
ಸಂರಕ್ಷಣೆ
ಸಂರಕ್ಷಿ-ಕೊಂಡು
ಸಂರಕ್ಷಿತ
ಸಂರಕ್ಷಿತ-ವಾಗಿದೆ
ಸಂರಕ್ಷಿಸ-ಲಾಗಿದೆ
ಸಂರಕ್ಷಿಸಿ
ಸಂರಕ್ಷಿಸಿ-ಕೊಂಡು
ಸಂರಕ್ಷಿಸುತ್ತಿದ್ದರು
ಸಂಲಗ್ನತೆ
ಸಂವ
ಸಂವ-ಛ-ರಾಗಿ
ಸಂವತ್ಸರ
ಸಂವತ್ಸರದ
ಸಂವತ್ಸರ-ದಲ್ಲಿ
ಸಂವತ್ಸರ-ವನ್ನು
ಸಂವತ್ಸರೋತ್ಸವ
ಸಂವತ್ಸ-ವರ
ಸಂವರಿಸಿ
ಸಂವರ್ದ್ಧತಾಂ
ಸಂವ-ಸನ್
ಸಂವಾಹಿ-ತಾಂಘ್ರಿಃ
ಸಂಶಂಕರ
ಸಂಶೇವಿತಃ
ಸಂಶೋಧ-ಕನು
ಸಂಶೋಧ-ಕರು
ಸಂಶೋಧನಾ
ಸಂಶೋಧ-ನಾತ್ಮಕ
ಸಂಶೋಧನೆ
ಸಂಶೋಧ-ನೆಯ
ಸಂಸಂ
ಸಂಸಾರಿ-ಗ-ಳಾಗಿ
ಸಂಸ್ಕಾರ-ದಲ್ಲಿ
ಸಂಸ್ಕಾರ-ವನ್ನು
ಸಂಸ್ಕೃತ
ಸಂಸ್ಕೃತ-ಕನ್ನಡ
ಸಂಸ್ಕೃ-ತದ
ಸಂಸ್ಕೃತ-ಶಾ-ಸನ
ಸಂಸ್ಕೃತಿ
ಸಂಸ್ಕೃ-ತಿಗೆ
ಸಂಸ್ಕೃತಿಯ
ಸಂಸ್ಕೃತಿ-ಯನ್ನು
ಸಂಸ್ಕೃತಿ-ಯಲ್ಲಿ
ಸಂಸ್ಕೃತಿ-ಯಿಂದಲೂ
ಸಂಸ್ಕೃ-ತಿಯು
ಸಂಸ್ಕೃತೀ-ಕರಣ
ಸಂಸ್ಕೃತೀ-ಕರ-ಣದ
ಸಂಸ್ಥಂ
ಸಂಸ್ಥತೋ-ನೃಪಃ
ಸಂಸ್ಥಾನ
ಸಂಸ್ಥಾನಂ
ಸಂಸ್ಥಾನಕ್ಕೆ
ಸಂಸ್ಥಾನದ
ಸಂಸ್ಥಾನ-ದಲ್ಲಿ
ಸಂಸ್ಥಾನ-ದಲ್ಲಿದ್ದ
ಸಂಸ್ಥಾನ-ದಲ್ಲಿ-ರುವ
ಸಂಸ್ಥಾನ-ವನ್ನು
ಸಂಸ್ಥಾಪಕ-ರಾದ
ಸಂಸ್ಥಾಪಿತ-ರಾದರು
ಸಂಸ್ಥೆಯ
ಸಂಸ್ಥೆ-ಯಲ್ಲಿ
ಸಂಸ್ಥೆಯಲ್ಲಿ-ರುವ
ಸಂಸ್ಥೆ-ಯಾ-ಗಿತ್ತು
ಸಂಸ್ಥೆ-ಯಾದ
ಸಂಸ್ಥೆ-ಯಿಂದ
ಸಂಸ್ಥೆಯು
ಸಂಸ್ಫು-ರನ್
ಸಂಹರಿ-ಸಿದ
ಸಂಹರಿ-ಸಿ-ದ-ನೆಂದು
ಸಂಹರಿ-ಸಿ-ದ-ನೆಂದೂ
ಸಂಹರಿ-ಸು-ವಲ್ಲಿ
ಸಅಂಕರ
ಸಕ
ಸಕಲ
ಸಕಲ-ಚಂದ್ರನ
ಸಕಲ-ಧರ್ಮ-ಗಳಂ
ಸಕಲ-ಮುದ-ಜಾತಿ
ಸಕಲ-ರಾಜ್ಯಾಧಿಪ-ತಿ-ಗ-ಳಾದ
ಸಕಲ-ವನ್ನೂ
ಸಕಲ-ವಿದ್ಯಾ-ನಿಧಿ
ಸಕಲ-ವಿದ್ಯಾ-ವಿಶಾ-ರದ-ರಾದ
ಸಕಲ-ವಿಧ
ಸಕಲ-ಶಾಸ್ತ್ರ
ಸಕಲಸ್ವಾಮ್ಯ-ವನ್ನು
ಸಕಲಸ್ವಾಮ್ಯವೂ
ಸಕಲೇಶ್ವರ
ಸಕಲ್ಪ-ಭೂಮಿ-ಜನಿತೋ
ಸಕಳ
ಸಕಳ-ಕಳಾ-ವಿಧಾನ-ಪದ್ಮಾ-ಸನ
ಸಕಳ-ಚಂದ್ರ
ಸಕಳ-ಚಂದ್ರ-ದೇವರು
ಸಕಳ-ಧರ್ಮ್ಮೋದ್ಧಾರಕ
ಸಕಳ-ಮುನಿ-ಜನ
ಸಕಳಾ-ಚಾರಿ
ಸಕಳಿ-ಗವು-ಡ-ನನ್ನು
ಸಕಳಿ-ಗ-ವುಡನು
ಸಕುಟುಂಬ-ಸಮೇತಂ
ಸಕುಟುಂಬ-ಸಮೇತ-ನಾಗಿ
ಸಕ್ಕ-ರಪ್ಪ
ಸಕ್ಕರೆ
ಸಕ್ಕರೆ-ಪಟ್ಟ-ಣ-ದಿಂದ
ಸಕ್ಕರೆಯ
ಸಕ್ಕರೆ-ಶೆಟ್ಟಿಯು
ಸಕ್ಕಿ-ಯಾಗೆ
ಸಕ್ರಿ-ಯ-ವಾಗಿ
ಸಖ್ಯ-ವನ್ನು
ಸಗಣಿ
ಸಗಣಿ-ಯನ್ನು
ಸಗಣಿ-ಯಿಂದ
ಸಗರ
ಸಗರ-ಕುಲ
ಸಗರ-ಕುಲ-ತಿಲಕ-ನೆಂಬ
ಸಗರ-ಕುಲದ
ಸಗರತ್ರಿಣೇತ್ರ
ಸಗರ-ವಂಶ
ಸಗರ-ವಂಶ-ಜರು
ಸಗರ-ವಂಶದ
ಸಗರ-ವಂಶ-ದ-ವ-ರಾ-ಗಿದ್ದು
ಸಗರ-ವಂಶ-ದ-ವರು
ಸಚಿವ
ಸಚಿವಂ
ಸಚಿವ-ನಾಗಿ
ಸಚಿವ-ನಾ-ಗಿದ್ದ
ಸಚಿವ-ನಿಗೆ
ಸಚಿ-ವನೂ
ಸಚಿವ-ಮಂತ್ರಿ
ಸಚಿ-ವರು
ಸಚಿವಾಧೀಶ್ವರ
ಸಚಿವೋಭ-ವತ್
ಸಚ್ಚರಿತರು
ಸಜನ
ಸಜ್ಜನ
ಸಜ್ಜನ-ರಾದ
ಸಜ್ಜ-ನರು
ಸಜ್ಜನಾ
ಸಜ್ಜನಾ-ಮೋದ
ಸಜ್ಜು-ಗೊ-ಳಿಸಿ
ಸಡಿಲಿಸಿ
ಸಣಂಬದ
ಸಣಂಬ-ವನು
ಸಣಬ
ಸಣ-ಬದ
ಸಣ-ಬಿನ-ಹಳ್ಳಿ
ಸಣ-ಬಿಮುಕ-ಳಿಯ
ಸಣ-ಬಿಮು-ಕುಳಿಯ
ಸಣ್ಣ
ಸಣ್ಣ-ಕಟ್ಟೆ-ಯಿದ್ದು
ಸಣ್ಣ-ಕೆರೆ-ಗ-ಳಲ್ಲಿ
ಸಣ್ಣ-ಕೆರೆ-ಗಳು
ಸಣ್ಣ-ಕೆರೆಯೂ
ಸಣ್ಣ-ಗುಡಿ-ಯಲ್ಲಿ
ಸಣ್ಣ-ಗುಡ್ಡ-ವನ್ನು
ಸಣ್ಣ-ತಮ್ಮಣ್ಣ
ಸಣ್ಣ-ತಿರು-ಮಾಲೆ
ಸಣ್ಣ-ತೆ-ರಿಗೆ-ಗ-ಳೆಂದೂ
ಸಣ್ಣ-ದೇವಾ-ಲಯ
ಸಣ್ಣ-ಪುಟ್ಟ
ಸಣ್ಣ-ವಿಗ್ರಹ-ವಿರುತ್ತದೆ
ಸಣ್ಣ-ಸಣ್ಣ
ಸಣ್ಣೇನ-ಹಳ್ಳಿ
ಸಣ್ನೆ
ಸಣ್ನೆ-ನಾಡನ್ನು
ಸಣ್ನೆ-ನಾ-ಡಿಗೆ
ಸಣ್ನೆ-ನಾಡು
ಸತತ
ಸತತಂ
ಸತ-ತ-ವಾಗಿ
ಸತಾಂ
ಸತಾದೃಗ್ಗುಣ
ಸತಿ
ಸತಿಯೂ
ಸತಿ-ಯೂರು
ಸತಿ-ಹೋಗಿದ್ದು
ಸತೀ-ರತ್ನ-ಳೆಂದು
ಸತೀಶ್
ಸತು-ಕವಿ
ಸತ್ಕರಿ-ಸುತ್ತಿದ್ದಾಗ
ಸತ್ಕವೀಶ್ವರಂ
ಸತ್ಕವೀಶ್ವರ-ನೆಂದು
ಸತ್ಕಾರ್ಯ-ನಿರ-ತ-ವಾಗು-ವಂತೆ
ಸತ್ಕುಲ
ಸತ್ಕ್ಷಾನ್ತಿಯಂ
ಸತ್ತ
ಸತ್ತದ್ದು
ಸತ್ತನು
ಸತ್ತ-ನೆಂದ
ಸತ್ತ-ನೆಂದಿದೆ
ಸತ್ತ-ನೆಂದು
ಸತ್ತ-ರೆಂದು
ಸತ್ತ-ವರ
ಸತ್ತ-ವೀ-ರರು
ಸತ್ತಾಗ
ಸತ್ತಿ-ಗನ-ಹಳ್ಳದ
ಸತ್ತಿ-ಗನ-ಹಳ್ಳ-ದಲ್ಲಿ
ಸತ್ತಿ-ದೇವರು
ಸತ್ತಿದ್ದಾನೆ
ಸತ್ತಿದ್ದಾನೆಂದು
ಸತ್ತಿ-ರ-ಬ-ಹುದು
ಸತ್ತು
ಸತ್ತು-ದಕ್ಕೆ
ಸತ್ತು-ಪಡೆದ
ಸತ್ತ್ಯ-ದಲಿ
ಸತ್ಯ
ಸತ್ಯಕೆ
ಸತ್ಯದ
ಸತ್ಯ-ದಂತೆ
ಸತ್ಯ-ನಾ-ರಾಯಣ
ಸತ್ಯ-ಭಾಮಾ
ಸತ್ಯ-ಭಾಮೆ-ಯಲ್ಲಿ
ಸತ್ಯ-ಮಂಗಲ
ಸತ್ಯ-ರಾ-ಧೇಯ
ಸತ್ಯ-ರಾ-ಧೇಯನುಂ
ಸತ್ಯ-ವಾಕ್ಯ
ಸತ್ಯ-ವಾಕ್ಯನ
ಸತ್ಯ-ವಾಕ್ಯನು
ಸತ್ಯ-ವಾಕ್ಯ-ಪೆರ್ಮಾನ-ಡಿಯ
ಸತ್ಯ-ವಾಕ್ಯ-ಪೆರ್ಮಾನ-ಡಿ-ಯ-ಇಮ್ಮಡಿ
ಸತ್ಯ-ವಾಕ್ಯ-ಪೆರ್ಮಾನ-ಡಿಯು
ಸತ್ಯ-ವಾದುವು
ಸತ್ಯ-ಸೌಚಾ-ಚಾರ
ಸತ್ಯಾ
ಸತ್ಯಾ-ಮೃತಶ್ರೀ
ಸತ್ಯಾ-ಯಾಸಕ್ತ-ಮತೀಂ
ಸತ್ರ
ಸತ್ರ-ಗ-ಳನ್ನು
ಸತ್ರ-ಗ-ಳನ್ನೂ
ಸತ್ರದ
ಸತ್ರ-ದಲ್ಲಿ
ಸತ್ರಾಧಿ-ಕಾರಿ-ಯಾಗಿದ್ದ-ನೆಂದು
ಸತ್ವ
ಸತ್ವ-ಶಾಲಿ-ಗಳಿ-ರ-ಬೇಕು
ಸದರಿ
ಸದಸಿ
ಸದ-ಸಿ-ವ-ರಾಯರು
ಸದಸ್ಯರ
ಸದಸ್ಯ-ರನ್ನು
ಸದಸ್ಯ-ರಾ-ಗಿದ್ದ
ಸದಸ್ಯರು
ಸದಸ್ಯರೇ
ಸದಾ
ಸದಾ-ಕಾಲ
ಸದಾ-ಚಾರಿ
ಸದಾ-ಶಿವ-ದೇವ
ಸದಾ-ಶಿವ-ದೇವ-ರಾಯನ
ಸದಾ-ಶಿವ-ದೇವ-ರಾಯ-ನಿಂದ
ಸದಾ-ಶಿವ-ನಿಗೆ
ಸದಾ-ಶಿವ-ಮಹಾ-ರಾಯ
ಸದಾ-ಶಿವ-ಮಹಾ-ರಾಯಕ್ಷಮಾ-ನಾಯಕಃ
ಸದಾ-ಶಿವ-ಮಹಾ-ರಾಯನ
ಸದಾ-ಶಿವ-ಮಹಾ-ರಾಯನು
ಸದಾ-ಶಿವ-ಮಹಾ-ರಾಯರು
ಸದಾ-ಶಿವರ
ಸದಾ-ಶಿವ-ರಾಯ
ಸದಾ-ಶಿವ-ರಾಯನ
ಸದಾ-ಶಿವ-ರಾಯ-ನನ್ನು
ಸದಾ-ಶಿವ-ರಾಯ-ನಿಗೆ
ಸದಾ-ಶಿವ-ರಾಯನು
ಸದಾ-ಶಿವ-ರಾಯ-ನೆಂದೂ
ಸದಾ-ಶಿವ-ರಾಯ-ರಿಗೆ
ಸದಾ-ಶಿವ-ರಾಯ-ಶಾ-ಸನೇನ
ಸದಿಯು
ಸದು-ಗುಣ-ಸಮೇತ
ಸದು-ಗುಣಿ
ಸದು-ಬುಧರ್ಗ್ಗ-ಮಾತ್ಯ-ರೊಳಧಿಕಂ
ಸದೃಶ್ಯ-ವಾದ
ಸದೆಬಡಿದು
ಸದ್ಗತಿ
ಸದ್ಗುಣ
ಸದ್ಗುಣ-ದೊಳಧಿಕ-ತೇಜಂ
ಸದ್ಗುರು
ಸದ್ಗೃಹಸ್ಥರು
ಸದ್ದಾನ-ದೀಕ್ಷಂ
ಸದ್ಯಕ್ಕೆ
ಸದ್ಯೋಜಾ-ತನ
ಸದ್ಯೋಜಾತ-ನೆಂದರೆ
ಸದ್ವಸ್ತು
ಸದ್ವೀರ
ಸಧ್ಯಾಯ
ಸನಾಥ
ಸನು-ಮಂತ್ರಿ-ಗಳೆನಿಸಿ
ಸನ್ತ-ತಿಯಂ
ಸನ್ತೊನ-ಮಾಳೆ
ಸನ್ದಿ
ಸನ್ನಿದಿಗೆ-ರಾಮಾ-ನು-ಜರ
ಸನ್ನಿಧಾನ
ಸನ್ನಿಧಾನ-ದಲ್ಲಿ
ಸನ್ನಿಧಿ
ಸನ್ನಿಧಿ-ಗರುಡ
ಸನ್ನಿಧಿ-ಗ-ಳಲ್ಲಿ
ಸನ್ನಿಧಿ-ಗಳಿದ್ದವು
ಸನ್ನಿಧಿಗೆ
ಸನ್ನಿಧಿಯ
ಸನ್ನಿಧಿ-ಯನ್ನು
ಸನ್ನಿಧಿ-ಯಲಿ
ಸನ್ನಿಧಿ-ಯಲ್ಲಿ
ಸನ್ನಿಧಿ-ಯಲ್ಲಿದ್ದು
ಸನ್ನಿಧಿ-ಯಲ್ಲಿ-ಯೂ-ದೇವಾ-ಲಯ-ಗಳು
ಸನ್ನಿಧಿ-ಯಿಂದ
ಸನ್ನಿವೇಶ-ಗ-ಳಿಗೆ
ಸನ್ನಿವೇಶ-ದಲ್ಲಿ
ಸನ್ನುತ
ಸನ್ನೆ-ಯನ್ನು
ಸನ್ಮಥ-ದಿಂದ
ಸನ್ಮಾನ
ಸನ್ಮಾನ್ಯ
ಸನ್ಮಾರ್ಗ
ಸನ್ಯ-ಸನ
ಸನ್ಯ-ಸನಂ
ಸನ್ಯ-ಸನಂಗೆಯ್ದು
ಸನ್ಯ-ಸನ-ದಿಂದ
ಸನ್ಯ-ಸನಮಂ
ಸನ್ಯಾ-ಸನ-ವನ್ನು
ಸನ್ಯಾ-ಸಿ-ಗಳು
ಸನ್ಯಾ-ಸಿ-ಪುರ
ಸಪ್ತ
ಸಪ್ತ-ಕೋಟಿ
ಸಪ್ತ-ತಿಯ
ಸಪ್ತ-ತಿ-ಯನ್ನು
ಸಪ್ತಮ
ಸಪ್ತ-ಮ-ಭಾಗೆಯ
ಸಪ್ತ-ಮ-ಭಾಗೆಯನು
ಸಪ್ತ-ಮ-ಭಾಗೆಯಲಿ
ಸಪ್ತ-ಮ-ಭಾಗೆಯಲು
ಸಪ್ತ-ಮಾತೃ-ಕೆಯರ
ಸಪ್ತ-ಮಾತೃ-ಕೆಯರು
ಸಪ್ತ-ಸಾ-ಗರ
ಸಪ್ತಾಂಗ-ಲಕ್ಷ್ಮೀ
ಸಪ್ತಾಷ್ಟ
ಸಪ್ಪೆ
ಸಪ್ಪೆಯ
ಸಪ್ಪೆ-ಯ-ದೇವ-ನೆಂಬು-ವ-ವನಿದ್ದಾನೆ
ಸಫಲ-ನಾ-ದನು
ಸಬಂಧ-ವೇ-ನೆಂದು
ಸಬಳಂ
ಸಬ್ಡಿ-ವಿ-ಜನ್
ಸಬ್ಡಿ-ವಿ-ಜನ್ಗ-ಳನ್ನು
ಸಬ್ಡಿ-ವಿ-ಜನ್ನ್ನು
ಸಬ್ತಾಲ್ಲೂಕು
ಸಬ್ಬ-ಗೊಡುಗೆ-ಯಾಗಿ
ಸಬ್ಬವ-ಮಪ್ಪ
ಸಬ್ಮಿಟ್
ಸಭಾ
ಸಭಾ-ಪತಿ
ಸಭಾ-ಪತಿಃ
ಸಭಾ-ಪತಿ-ಗ-ಳಾಗಿ
ಸಭಾ-ಪತಿಯ
ಸಭಾ-ಪತಿ-ಯ-ವ-ರಾ-ಗಿದ್ದ
ಸಭಾ-ಪತಿ-ಯ-ವ-ರಾದ
ಸಭಾ-ಮಂಟಪ
ಸಭಾ-ಮಂಟಪ-ದಿಂದ
ಸಭಾಸ್ಸು-ಧರ್ಮಾ-ಮಿವಾಧ್ಯಾಸ್ತೇ
ಸಭೆ
ಸಭೆ-ಗಳು
ಸಭೆಗೆ
ಸಭೆಯ
ಸಭೆ-ಯನ್ನು
ಸಭೆ-ಯಲ್ಲಿ
ಸಭೆ-ಯ-ವರು
ಸಭೆ-ಯಾ-ಗಿತ್ತು
ಸಭೆ-ಯಾಗಿ-ರ-ಬ-ಹುದು
ಸಭೆಯು
ಸಭೆ-ಯೋಜ
ಸಭೆ-ಯೋಜನ
ಸಭೆ-ಯೋಜ-ನನ್ನು
ಸಭೆ-ಯೋಜನು
ಸಭೆ-ಸೇರಿ
ಸಮ
ಸಮಂ
ಸಮಂತ-ಭದ್ರ
ಸಮಂತ-ಭದ್ರರ
ಸಮಂತ-ಭದ್ರರು
ಸಮಂತಾ
ಸಮ-ಕಾಲಿನ
ಸಮ-ಕಾಲೀನ
ಸಮ-ಕಾಲೀನ-ನಾ-ಗಿದ್ದ
ಸಮ-ಕಾಲೀನ-ನಾಗಿದ್ದ-ನೆಂದು
ಸಮ-ಕಾಲೀನ-ನಾ-ಗಿದ್ದು
ಸಮ-ಕಾಲೀನ-ನೆಂದು
ಸಮ-ಕಾಲೀನ-ರಾಗಿ
ಸಮ-ಕಾಲೀನರು
ಸಮ-ಕಾಲೀನ-ರೆಂದು
ಸಮ-ಕಾಲೀನ-ವಾಗಿ-ರ-ಬ-ಹುದು
ಸಮ-ಕಾಲೀನ-ವಾದ
ಸಮ-ಕಾಲೀನವೂ
ಸಮಕ್ಷಮ
ಸಮ-ಗಾರ
ಸಮಗ್ರ
ಸಮಗ್ರ-ಚಿತ್ರವು
ಸಮಗ್ರ-ಬ-ಲದ
ಸಮಗ್ರ-ಬಲ-ನಿ-ಲಯೇ
ಸಮಗ್ರ-ಬಲ-ವನ್ನು
ಸಮಗ್ರ-ವಾಗಿ
ಸಮಚಿತ್ತ-ದಿಂದ
ಸಮತಟ್ಟಾದ
ಸಮತಟ್ಟು
ಸಮತೋಲನ-ದಿಂದ
ಸಮಧಿಗತ
ಸಮ-ನಾ-ಗಿತ್ತು
ಸಮ-ನಾದ
ಸಮ-ನೆಂಬೊಂದು
ಸಮನ್ವಯ
ಸಮನ್ವಯ-ತೆ-ಯನ್ನು
ಸಮನ್ವಿತ
ಸಮಪಸ್ತೇಪ್ಸಿ-ತಾರ್ತ್ಥ-ಲಾಭಾಯ
ಸಮಯ
ಸಮಯಕ್ಕಾಗಿ
ಸಮ-ಯಕ್ಕೂ
ಸಮ-ಯಕ್ಕೆ
ಸಮಯ-ಗ-ಳಲ್ಲಿಯೂ
ಸಮಯ-ಗಳು
ಸಮಯ-ಜೈನ-ಧರ್ಮ
ಸಮಯ-ದಲ್ಲಿ
ಸಮಯ-ದಲ್ಲೂ
ಸಮಯ-ದ-ವ-ರಿಗೂ
ಸಮಯ-ದ-ವರು
ಸಮಯದ್ರೋಹನಂ
ಸಮಯ-ವನ್ನು
ಸಮಯ-ವೆಂದೂ
ಸಮಯಾವಕಾಶ
ಸಮರ
ಸಮರ-ಗ-ಳಲ್ಲಿ
ಸಮರ-ಧಾರ-ಧರಂ
ಸಮರ-ಧುರೀಣ-ರಾಗಿ
ಸಮರ-ಮುಖ-ಲ-ಸದ್
ಸಮರ-ಸ-ಗೊಂಡವು
ಸಮರಾಧಿ-ತತ್ರಿ-ವರ್ಗ್ಗ
ಸಮರ್ಥ-ನಲ್ಲದ
ಸಮರ್ಥನೀಯ-ವಾಗಿ-ರು-ವು-ದನ್ನು
ಸಮರ್ಥನೆ
ಸಮರ್ಥನೆ-ಯನ್ನು
ಸಮರ್ಥ-ವಾಗಿ
ಸಮರ್ಥಿ-ಸಿದ್ದಾರೆ
ಸಮರ್ಥಿ-ಸುತ್ತದೆ
ಸಮರ್ಥಿ-ಸುತ್ತವೆ
ಸಮರ್ಥಿ-ಸುತ್ತ-ವೆಂದು
ಸಮರ್ಥಿ-ಸುವಂತಿದೆ
ಸಮರ್ಪಕ-ವಾಗಿ
ಸಮರ್ಪಕ-ವಾಗಿದೆ
ಸಮರ್ಪಕ-ವಾಗುತ್ತದೆ
ಸಮರ್ಪಿಸ-ಲಾಗಿದೆ
ಸಮರ್ಪಿಸಿ
ಸಮರ್ಪಿ-ಸಿದ
ಸಮರ್ಪಿಸಿ-ದನು
ಸಮರ್ಪಿಸಿ-ದಾಗ
ಸಮರ್ಪಿಸಿ-ದೆ-ವಾಗ
ಸಮರ್ಪಿಸುತ್ತಾನೆ
ಸಮರ್ಪಿಸುತ್ತಾರೆ
ಸಮರ್ಪಿ-ಸುತ್ತಾಳೆ
ಸಮರ್ಪಿ-ಸುವ
ಸಮ-ವೆಂದು
ಸಮಸ್ತ
ಸಮಸ್ತ-ಗವು-ಡು-ಗಳು
ಸಮಸ್ತ-ಗುಣ-ಸಂಪನ್ನ
ಸಮಸ್ತ-ಗುಣ-ಸಂಪನ್ನ-ರು-ಮಪ್ಪ
ಸಮಸ್ತ-ಭಾಗ್ಯೈಃ
ಸಮಸ್ತ-ರಾಜ್ಯ-ಭಾರ
ಸಮಸ್ತರು
ಸಮಸ್ಯೆ-ಗ-ಳಾಗಿ
ಸಮಸ್ಯೆ-ಯನ್ನು
ಸಮಾಗ್ರ-ಗಣ್ಯರುಂ
ಸಮಾಜ
ಸಮಾಜಃ
ಸಮಾಜಕ್ಕೆ
ಸಮಾಜದ
ಸಮಾಜ-ದಲ್ಲಿ
ಸಮಾಜವು
ಸಮಾಜ-ಶಾಸ್ತ್ರ-ವನ್ನು
ಸಮಾಜಶ್ಚಾಮ-ರಾಜೇಂದ್ರ
ಸಮಾಜ-ಸೇವೆ-ಯಲ್ಲಿಯೂ
ಸಮಾಧಾನ
ಸಮಾಧಿ
ಸಮಾಧಿ-ಗಳ
ಸಮಾಧಿ-ಗಳಿದ್ದು
ಸಮಾಧಿ-ಗಳು
ಸಮಾಧಿ-ಗಳುಳ್ಳ
ಸಮಾಧಿ-ಗುಹೆ
ಸಮಾಧಿ-ಮರಣ
ಸಮಾಧಿ-ಮರ-ಣದ
ಸಮಾಧಿ-ಮರ-ಣ-ವನ್ನಪ್ಪಿ-ದನು
ಸಮಾಧಿ-ಮರ-ಣ-ವನ್ನು
ಸಮಾಧಿಯ
ಸಮಾಧಿ-ಯನ್ನು
ಸಮಾಧಿಯೆ
ಸಮಾಧಿ-ಯೆಂದು
ಸಮಾಧಿಸ್ಥ-ರಾಗಿ-ರ-ಬ-ಹುದು
ಸಮಾನ
ಸಮಾನತೆ
ಸಮಾನ-ನಾದ
ಸಮಾನ-ರಾಗಿಯೂ
ಸಮಾನ-ರಾದ
ಸಮಾನರು
ಸಮಾನ-ರೂಪ
ಸಮಾನ-ಳಾಗಿದ್ದ-ಳೆಂದು
ಸಮಾನ-ವಾಗಿ
ಸಮಾನ-ವಾಗಿಯೂ
ಸಮಾನ-ವಾದ
ಸಮಾನ-ವಾದವು
ಸಮಾನ-ವಾ-ದುದೇ
ಸಮಾನಾಂತರ
ಸಮಾ-ನಾರ್ಥಕ-ಗಳು
ಸಮಾ-ನಾರ್ಥಕ-ವಾದ
ಸಮಾನೆ
ಸಮಾನೆ-ಯ-ರಪ್ಪ
ಸಮಾ-ನೆಯುಂ
ಸಮಾಪ್ತಿಗೊಳಿ-ಸಿ-ದನು
ಸಮಾ-ಯಯೌ
ಸಮಾರಂಭ-ಗಳ
ಸಮಾರಂಭ-ಗ-ಳನ್ನು
ಸಮಾರಂಭ-ಗಳು
ಸಮಾರಾ-ಧನೆ
ಸಮಾರಾಧ-ನೆಗೆ
ಸಮಾರಾ-ಧನೆ-ಯನ್ನು
ಸಮಾರಾಧಿತಾ-ಶೇಷೈರ್ಭೂ-ಸುರ-ಪುಂಗವಾ
ಸಮಾರೋಪ
ಸಮಾರ್ಚ-ನಾಮ
ಸಮಾಲೋಚಕ
ಸಮಾವೇಶ-ಗೊ-ಳಿಸಿ
ಸಮಾವೇಶ-ವಾ-ಗಿತ್ತು
ಸಮಾಶ್ರಿತ
ಸಮಾ-ಹೂಯ
ಸಮಿತಿ
ಸಮಿ-ತಿಗೆ
ಸಮಿ-ತಿಯು
ಸಮೀ-ಕರಣ
ಸಮೀ-ಕರಿ-ಸ-ಲಾಗಿದೆ
ಸಮೀ-ಕರಿ-ಸ-ಲಾ-ಯಿತು
ಸಮೀಕ್ಷೆ-ಯನ್ನು
ಸಮೀ-ದಲ್ಲಿದೆ
ಸಮೀಪ
ಸಮೀ-ಪಕ್ಕೆ
ಸಮೀ-ಪದ
ಸಮೀಪ-ದಲ್ಲಿ
ಸಮೀಪ-ದಲ್ಲಿದೆ
ಸಮೀಪ-ದಲ್ಲಿದ್ದ
ಸಮೀಪ-ದಲ್ಲಿದ್ದು
ಸಮೀಪ-ದಲ್ಲಿಯೇ
ಸಮೀಪ-ದಲ್ಲಿ-ರುವ
ಸಮೀಪ-ದಲ್ಲೇ
ಸಮೀಪ-ವಾದ
ಸಮೀಪ-ವಿ-ರುವ
ಸಮೀಪಿ-ಸಿದಾಗ
ಸಮುಂನ
ಸಮುಚ್ಛಯ-ದಲ್ಲಿ
ಸಮುತ್ತುಂಗ
ಸಮು-ದಾಯಕ್ಕೆ
ಸಮು-ದಾಯ-ಗ-ಳಿಗೆ
ಸಮು-ದಾಯದ
ಸಮು-ದಾಯ-ದ-ವ-ನಿರ-ಬ-ಹುದು
ಸಮು-ದಾಯ-ದ-ವ-ರಾ-ಗಿದ್ದು
ಸಮು-ದಾಯ-ದ-ವರು
ಸಮುದ್ಧರಣ
ಸಮುದ್ಧರಣಂ
ಸಮುದ್ಧರ-ಣನುಂ
ಸಮುದ್ಧರ-ಣನೂ
ಸಮುದ್ಧರ-ಣರು
ಸಮುದ್ಧರ-ಣ-ವನ್ನು
ಸಮುದ್ರ
ಸಮುದ್ರದ
ಸಮುದ್ರ-ಪಾಂಡ್ಯ
ಸಮುದ್ರ-ವನ್ನು
ಸಮುದ್ರ-ವಾದ
ಸಮುದ್ರವು
ಸಮುದ್ರ-ವೆಂದು
ಸಮುದ್ರ-ವೆಂಬ
ಸಮುದ್ರ-ವೊಳಗಾದ
ಸಮುದ್ರಾಧಿ-ಪತಿ
ಸಮುದ್ರಾಧೀಶ್ವರ
ಸಮೂಹ
ಸಮೂಹಕ್ಕೆ
ಸಮೂಹದ
ಸಮೂಹ-ವನ್ನು
ಸಮೂಹವು
ಸಮೂಹವೂ
ಸಮೂಹವೇ
ಸಮೃದ್ಧ
ಸಮೃದ್ಧ-ವಾಗಿತ್ತೆಂದು
ಸಮೃದ್ಧ-ವಾಗಿದೆ
ಸಮೃದ್ಧ-ವಾ-ಗಿದ್ದ
ಸಮೃದ್ಧ-ವಾಗು-ವು-ದಕ್ಕೆ
ಸಮೃದ್ಧ-ವಾದ
ಸಮೃದ್ಧಿ-ಯನ್ನು
ಸಮೃದ್ಧಿ-ಯನ್ನೂ
ಸಮೃದ್ಧಿ-ಯಾಗಿ-ದೆಯೋ
ಸಮೃದ್ಧಿ-ಯಾಗಿದ್ದಿ-ತೆಂದು
ಸಮೇತ
ಸಮೇತಂ
ಸಮೇತ-ನಾಗಿ
ಸಮೇತ-ರಾದ
ಸಮೇತ-ವಾಗಿ
ಸಮ್ಮಟಿ-ಭಾಗ-ಭೂ-ಪತಿ
ಸಮ್ಮಸ್ತ
ಸಮ್ಮುಖದ
ಸಮ್ಮುಖ-ದಲ್ಲಿ
ಸಮ್ಮುಖ-ದಲ್ಲಿಯೇ
ಸಮ್ಯಕ್ತ್ವ
ಸಮ್ಯಕ್ತ್ವ-ಚೂಡಾ-ಮಣಿ
ಸಯಿಗೋಲಪಾರ್ತನುಂ
ಸಯ್ಯದ್
ಸರ-ಕನ್ನು
ಸರ-ಕಿಗೆ
ಸರ-ಕಿದೆ
ಸರ-ಕು-ಗಳ
ಸರ-ಕು-ಗ-ಳಿಗೆ
ಸರ-ಗೂರ
ಸರ-ಗೂರಿನ
ಸರ-ಗೂರು
ಸರ-ಣಾಗತ
ಸರ-ಥಿ-ಯನು
ಸರದಭ್ರಕಾನ್ತಿಯಂ
ಸರ-ದಾರ-ರಾದ
ಸರದಿ
ಸರ-ದಿಯ
ಸರದಿ-ಯಲ್ಲಿ
ಸರ-ಪಳಿ-ಗಳ
ಸರ-ಬ-ರಾಜು
ಸರ-ಮಾಲೆ
ಸರ-ಲನ್ನು
ಸರಳ
ಸರ-ಳ-ರೂಪ
ಸರ-ಳ-ವಾಗಿ
ಸರ-ಸತ-ರೇಣ
ಸರಸ್ವತಿ
ಸರಸ್ವತಿ-ಗಣ-ದಾಸಿ
ಸರಸ್ವತೀ
ಸರಸ್ಸನ್ನು
ಸರಸ್ಸನ್ನೂ
ಸರಸ್ಸು
ಸರಹದ್ದಿನ
ಸರಾ-ಗ-ವಾಗಿ
ಸರಿ
ಸರಿ-ಗಟ್ಟುವ
ಸರಿ-ಗಟ್ಟುವ-ದೇಗುಲ-ಮೊಳವೆ
ಸರಿ-ದೂಗಿಸಿ
ಸರಿ-ಪಡಿ-ಸಲು
ಸರಿ-ಪ-ಡಿಸಿ
ಸರಿ-ಮಕ್ಕನ-ಹಳ್ಳಿ
ಸರಿ-ಯಲ್ಲ
ಸರಿ-ಯಲ್ಲ-ವೆಂದು
ಸರಿ-ಯಾಗಿ
ಸರಿ-ಯಾಗಿದೆ
ಸರಿ-ಯಾ-ಗಿಯೇ
ಸರಿಯೆ
ಸರಿಯೇ
ಸರಿ-ಸ-ಮಾನ-ರಲ್ಲ-ವೆಂದು
ಸರಿ-ಸ-ಮಾನ-ರಾಗಿ
ಸರಿ-ಸ-ಮಾನಳು
ಸರಿ-ಸ-ಮಾನ-ವಾದ
ಸರಿಸಿ
ಸರಿ-ಹೊಂದಿಸಿ
ಸರಿ-ಹೊಂದುತ್ತದೆ
ಸರಿ-ಹೊಂದುತ್ತ-ದೆಂದು
ಸರಿ-ಹೊಂದುವ
ಸರಿ-ಹೋಗುತ್ತದೆ
ಸರೀರ
ಸರೀ-ರದ
ಸರೀರ-ಸಂಪತ್ತಿಗೆ
ಸರು-ಲೋಕಪ್ರಾಪ್ತ-ನಾಗುತ್ತಾರೆ
ಸರು-ವಯ್ಯ-ಸೆಟ್ಟಿ-ಯ-ರನ್ನು
ಸರೋ-ವರ
ಸರೋ-ವರ-ಗ-ಳೆಂದು
ಸರೋ-ವರದ
ಸರೋ-ವರ-ದಲ್ಲಿ
ಸರೋ-ವರ-ವನ್ನು
ಸರ್
ಸರ್ಕಾರಕ್ಕೆ
ಸರ್ಕಾರದ
ಸರ್ಕಾರ-ದಲ್ಲಿ
ಸರ್ಕಾರಿ
ಸರ್ಕಾರೆ
ಸರ್ದಾರ್
ಸರ್ಬ್ಬ-ದೇವ
ಸರ್ಬ್ಬ-ದೇವ-ನುತೆ
ಸರ್ಬ್ಬ-ಬಾಧಾ-ಪರಿ-ಹಾರ-ವಾಗಿ
ಸರ್ಭಾಂಗ
ಸರ್ವ
ಸರ್ವಂ
ಸರ್ವಜ್ಞ
ಸರ್ವಜ್ಞ-ದೇವ-ಪುರ-ದತಿ
ಸರ್ವಜ್ಞ-ನ-ಪರಿಮಿತ-ದಾನ-ವಿನೋದ-ಶೀಳ
ಸರ್ವಜ್ಞನೂ
ಸರ್ವಜ್ಞ-ಪದುಮ-ನಾಭ-ಪುರದ
ಸರ್ವಜ್ಞ-ಪುರ
ಸರ್ವಜ್ಞ-ಪುರ-ವೆಂಬ
ಸರ್ವಜ್ಞ-ವಿಷ್ಣು
ಸರ್ವಜ್ಞ-ವಿಷ್ಣು-ಭಟ್ಟಯ್ಯ
ಸರ್ವಜ್ಞ-ವಿಷ್ಣು-ಭಟ್ಟಯ್ಯನ
ಸರ್ವಜ್ಞ-ವಿಷ್ಣು-ಭಟ್ಟಯ್ಯನು
ಸರ್ವಜ್ಞ-ವೀರ-ನರ-ಸಿಂಹ-ಪುರ-ವಾದ
ಸರ್ವ-ತಂತ್ರಸ್ವ-ತಂತ್ರ-ಳಾಗಿ
ಸರ್ವ-ದರ್ಶನ
ಸರ್ವ-ಧರ್ಮ
ಸರ್ವ-ನಮಸ್ಯ
ಸರ್ವ-ನಮಸ್ಯದ
ಸರ್ವ-ನಮಸ್ಯ-ವಾಗಿ
ಸರ್ವ-ನಮಸ್ಯ-ವಾದ
ಸರ್ವ-ಪರಿ-ಹಾರ-ವಾಗಿ
ಸರ್ವ-ಬಾದಾ
ಸರ್ವ-ಬಾಧಾ
ಸರ್ವ-ಬಾಧಾ-ಪರಿ-ಹಾರ
ಸರ್ವ-ಬಾಧಾ-ಪರಿ-ಹಾರ-ವಾಗಿ
ಸರ್ವ-ಬಾಧಾ-ವಿ-ರಹಿತ
ಸರ್ವ-ಭೂ-ತಾನು-ಕಂಪಿನಃ
ಸರ್ವ-ಭೂ-ತಾನು-ಕಂಪಿಯೂ
ಸರ್ವ-ಮಾನ್ಯ
ಸರ್ವ-ಮಾನ್ಯದ
ಸರ್ವ-ಮಾನ್ಯ-ವನ್ನು
ಸರ್ವ-ಮಾನ್ಯ-ವಾಗಿ
ಸರ್ವ-ಮಾನ್ಯ-ವಾಗಿ-ತೆ-ರಿಗೆ-ರಹಿತ-ವಾಗಿ
ಸರ್ವ-ಮಾನ್ಯ-ವಾದ
ಸರ್ವ-ಮೆನಿ-ಸುವ
ಸರ್ವ-ವಿದ್ಯಾ-ವಿಚಕ್ಷ-ಣನು
ಸರ್ವ-ವಿದ್ಯಾ-ಸುವೈಚಕ್ಷಣ್ಯಂ
ಸರ್ವ-ವಿಭೂಷಣೋ
ಸರ್ವ-ಶಾಸ್ತ್ರ
ಸರ್ವ-ಸಮ್ಮತ-ವಾಗಿ
ಸರ್ವ-ಸಾಮ್ಯ-ಗ-ಳನ್ನು
ಸರ್ವ-ಸಾಮ್ಯ-ವನ್ನು
ಸರ್ವ-ಸಾಮ್ಯ-ವನ್ನೂ
ಸರ್ವ-ಸಾಮ್ಯ-ವಾಗಿ
ಸರ್ವ-ಸಾಮ್ಯ-ಸ-ಮನ್ವಿತ-ವಾಗಿ
ಸರ್ವಸ್ವ-ವನ್ನೂ
ಸರ್ವಸ್ವಾಧೀನ-ವಾದರೆ
ಸರ್ವಸ್ವಾಮ್ಯ-ವನ್ನು
ಸರ್ವಾಧಿ-ಕಾರಿ
ಸರ್ವಾಧಿ-ಕಾರಿ-ಗಳು
ಸರ್ವಾಧಿ-ಕಾರಿ-ಗಳೂ
ಸರ್ವಾಧಿ-ಕಾರಿಯ
ಸರ್ವಾಧಿ-ಕಾರಿ-ಯಾಗಿ
ಸರ್ವಾಧಿ-ಕಾರಿ-ಯಾ-ಗಿದ್ದ
ಸರ್ವಾಧಿ-ಕಾರಿ-ಯಾಗಿಯೂ
ಸರ್ವಾಧಿ-ಕಾರಿ-ಯಾ-ದನು
ಸರ್ವಾಧಿ-ಕಾರಿ-ಯಾ-ದರೂ
ಸರ್ವಾಧಿ-ಕಾರಿಯೂ
ಸರ್ವಾಧ್ಯಕ್ಷ-ನಾಗಿದ್ದ-ನೆಂದು
ಸರ್ವಾಧ್ಯಕ್ಷನೂ
ಸರ್ವಾಧ್ಯಕ್ಷ-ನೆಂದು
ಸರ್ವಾಯ-ಸುದ್ಧ-ವಾಗಿ
ಸರ್ವೈಕ-ಮತ್ಯ-ವಾಗಿ
ಸರ್ವೋತ್ತಮ
ಸರ್ವೋರ್ವೀಶ-ನತಃ
ಸರ್ವೋರ್ವೀಶ್ವರಾ-ರ-ವರೋಧ-ವಿನಯ
ಸರ್ವ್ವಕಲಾಧಾರ-ಭೂತ
ಸರ್ವ್ವ-ಧರ್ಮ್ಮ-ರಹಸ್ಯಸ್ಯ
ಸರ್ವ್ವಧಾರ್ಯಾಹ್ವಯೇ
ಸರ್ವ್ವ-ಬಾಧೆ
ಸರ್ವ್ವ-ಮಾಂನ್ಯ-ವಾಗಿ
ಸರ್ವ್ವ-ಮಾನ್ಯ-ವಾಗಿ
ಸರ್ವ್ವಸ್ಥಾನ-ಸಮುಚ್ಚಯೆ
ಸರ್ವ್ವೇಕ-ಮತ್ಯ-ವಾಗಿ
ಸಲ
ಸಲಕ-ರಾಜು
ಸಲಗ-ಗ-ಳನ್ನು
ಸಲಗೆ
ಸಲಗೆ-ಯನ್ನೂ
ಸಲ-ವುದು
ಸಲಹೆ-ಯಂತೆ
ಸಲಾಕೆ-ಯನ್ನು
ಸಲಾಕೆ-ಯಿಂದ
ಸಲಾಕೆ-ಯಿಂದಲೇ
ಸಲಾಕೆ-ಯೊಂದನ್ನು
ಸಲಿಗೆ
ಸಲಿಗೆ-ಮೂರು
ಸಲಿಲ
ಸಲಿಸಿ
ಸಲಿಸಿ-ಕೊಂಡು
ಸಲಿ-ಸುತ್ತ-ಮಿರೆ
ಸಲೀಸಾಗಿ
ಸಲುದು
ಸಲುವ
ಸಲು-ವಂತಾಗಿ
ಸಲು-ವಂಥಾ
ಸಲು-ವ-ಳಿಯ
ಸಲು-ವಾಗಿ
ಸಲುವು
ಸಲು-ವುದು
ಸಲು-ವು-ದೆಂದೂ
ಸಲೂ
ಸಲೆ
ಸಲೆ-ದೇವಕ್ಷೇತ್ರ-ದೊಳ್ಬಿಂಡಿಗ-ನವಿಲೆ-ಯೊಳಿರ್ಪ್ಪತ್ತು
ಸಲ್ಗು
ಸಲ್ಯದ-ಚಲ್ಯ
ಸಲ್ಲದೆ-ನಲು
ಸಲ್ಲ-ಬೇ-ಕಾದ
ಸಲ್ಲ-ಬೇಕೆಂದು
ಸಲ್ಲಲು
ಸಲ್ಲಿ-ಕೆ-ಯಾಗುತ್ತಿದ್ದರೂ
ಸಲ್ಲಿ-ಸ-ಬೇಕಾಗುತ್ತಿತ್ತು
ಸಲ್ಲಿ-ಸ-ಬೇಕೆಂದು
ಸಲ್ಲಿ-ಸಬೇಕೆಂಬ
ಸಲ್ಲಿಸಿ
ಸಲ್ಲಿ-ಸಿದ
ಸಲ್ಲಿ-ಸಿ-ದನು
ಸಲ್ಲಿ-ಸಿ-ದಾಗ
ಸಲ್ಲಿ-ಸಿದೆ
ಸಲ್ಲಿ-ಸಿದ್ದಾರೆ
ಸಲ್ಲಿ-ಸಿದ್ದಾ-ರೆಂದು
ಸಲ್ಲಿ-ಸಿ-ರ-ಬಹು-ದೆಂದು
ಸಲ್ಲಿ-ಸಿ-ರುತ್ತಾ-ರೆಂದು
ಸಲ್ಲಿ-ಸುತ್ತಿದ್ದ
ಸಲ್ಲಿ-ಸುತ್ತಿದ್ದರು
ಸಲ್ಲಿ-ಸುತ್ತಿದ್ದ-ರೆಂದು
ಸಲ್ಲಿ-ಸುತ್ತಿದ್ದ-ವ-ರಿಗೆ
ಸಲ್ಲಿ-ಸುವ
ಸಲ್ಲಿ-ಸುವ-ವ-ರಿಗೆ
ಸಲ್ಲಿ-ಸುವಾಗ
ಸಲ್ಲುತ್ತದೆ
ಸಲ್ಲುತ್ತ-ದೆಂದು
ಸಲ್ಲುತ್ತಿತ್ತೆಂದು
ಸಲ್ಲುತ್ತಿದ್ದ
ಸಲ್ಲುವ
ಸಲ್ಲು-ವಂತೆ
ಸಲ್ಲು-ವು-ದಿಲ್ಲ-ವೆಂದು
ಸಲ್ಲು-ವುದು
ಸಲ್ಲು-ವು-ದೆಂದು
ಸಲ್ಲು-ವುವು
ಸಲ್ಲೇಖನ
ಸಲ್ಲೇಖ-ನವ್ರತ
ಸಲ್ಲೇಖ-ನವ್ರತ-ವೆಂದು
ಸಲ್ಲೇಖನಾ
ಸಲ್ವ
ಸಲ್ವಿನಂ
ಸಲ್ವುದು
ಸಲ್ವು-ದೆಂದು
ಸಳ
ಸಳನ
ಸಳ-ನನ್ನು
ಸಳ-ನಿಗೆ
ಸಳನು
ಸಳ-ನೆಂದು
ಸಳ-ನೆಂಬ
ಸಳ-ಪಯ್ಯ
ಸಳ-ಪಯ್ಯ-ನೆಂಬ
ಸಳಿ-ಗವುಡ
ಸವಣಪ್ಪನ
ಸವಣಪ್ಪನಶ್ರವ-ಣಪ್ಪ
ಸವತಿ-ಗಂಧ-ವಾರಣ
ಸವಿಲಾಸ-ಮಾಸ
ಸವೆ-ದಿದೆ
ಸವೆದಿ-ರಲು
ಸವೆಸ-ಬೇಕು
ಸವೆಸು
ಸಶ್ಯಾಲ-ಪುರ
ಸಶ್ಯಾಲ-ಪುರದ
ಸಶ್ಯಾಲ-ಪುರ-ವನ್ನುಯ
ಸಸನ
ಸಸನ-ಶಾ-ಸನ
ಸಸಿ-ಯಾಲದ
ಸಸಿ-ಯಾಲ-ದ-ಪುರ
ಸಸಿ-ಯಾಲ-ದ-ಪುರ-ವನ್ನು
ಸಸಿ-ಯಾಲ-ಪುರ
ಸಸಿ-ಯಾಲ-ಪುರಕ್ಕೆ
ಸಸಿರರ್ಬ್ಬಲ್ಲ-ವರೆಮ್ಮೞ್ದಕ್ಕೆ
ಸಸ್ಯ
ಸಸ್ಯಾಧಿ-ಪ-ತಿಯೇ
ಸಹ
ಸಹಅ-ಗ-ಮನ-ವಾಗುತ್ತಾಳೆ
ಸಹ-ಕರಿ-ಸಿದರು
ಸಹ-ಕರಿ-ಸಿದ್ದಾರೆ
ಸಹ-ಕರಿ-ಸುತ್ತಿದ್ದರು
ಸಹ-ಕರಿ-ಸುತ್ತಿದ್ದ-ರೆಂದು
ಸಹ-ಕರಿ-ಸುತ್ತಿದ್ದ-ರೆಂಬ
ಸಹ-ಕಾರ
ಸಹ-ಕಾರ-ದಿಂದ
ಸಹ-ಕಾರಿ-ಯಾಗಿದ್ದನು
ಸಹ-ಕಾರಿ-ಯಾದ-ವನು
ಸಹ-ಗ-ಮನ
ಸಹ-ಗ-ಮನ-ವನ್ನು
ಸಹ-ಗ-ಮನ-ವಾಗಿದೆ
ಸಹ-ಗ-ಮನ-ವೆಂಬ
ಸಹ-ಗ-ಮನ-ವೆನಿಸಿ-ದರೆ
ಸಹಜ
ಸಹಜ-ವಾಗಿ
ಸಹಜ-ವಾಗಿದೆ
ಸಹ-ದೇವ
ಸಹ-ವಾಗಿ
ಸಹಸ್ರ
ಸಹಸ್ರ-ಕಿಳಲೆ-ನಾಡು-ಕೆಳ-ಲೆ-ನಾಡು
ಸಹಸ್ರ-ಗಳು
ಸಹಸ್ರ-ಗಾಥಾ
ಸಹಸ್ರ-ದೊಳಗೆ
ಸಹಸ್ರ-ಧಾರೆ
ಸಹಸ್ರ-ನಾ-ಮ-ಗ-ಳನ್ನು
ಸಹಸ್ರ-ಫಳ
ಸಹಸ್ರ-ಬಾಹು
ಸಹಸ್ರ-ವಿಷಯ
ಸಹಸ್ರಾರು
ಸಹಾನು-ಭೂತಿ-ಯಿಂದ
ಸಹಾಯ
ಸಹಾ-ಯಕ
ಸಹಾಯ-ಕ-ನಾಗಿ
ಸಹಾಯ-ಕ-ನಾಗಿದ್ದನು
ಸಹಾಯ-ಕ-ರಾಗಿ
ಸಹಾಯ-ಕ-ರಾಗಿದ್ದ-ರೆಂದು
ಸಹಾಯ-ಕ-ರಾ-ಗಿದ್ದು
ಸಹಾ-ಯಕ್ಕೆ
ಸಹಾಯ-ದಿಂದ
ಸಹಿತ
ಸಹಿತಂ
ಸಹಿತಃ
ಸಹಿತ-ರಪ್ಪ
ಸಹಿತ-ವಾಗಿ
ಸಹಿತ-ವಾದ
ಸಹಿತಾಃ
ಸಹಿರಣ್ಯೋದಕ
ಸಹಿರಣ್ಯೋದಕ-ಪೂರ್ವ-ಕ-ವಾಗಿ
ಸಹಿರಣ್ಯೋದಕ-ವಾಗಿ
ಸಹೃದ-ಯರು
ಸಹೋದರ
ಸಹೋದ-ರ-ನಾಗಿ-ರ-ಬ-ಹುದು
ಸಹೋದ-ರ-ನಾದ
ಸಹೋದ-ರರ
ಸಹೋದ-ರ-ರಾ-ಗಿದ್ದ-ರೆಂಬುದು
ಸಹೋದ-ರ-ರಾಗಿ-ರ-ಬ-ಹುದು
ಸಹೋದ-ರ-ರಾದ
ಸಹೋದ-ರ-ರಿಗೂ
ಸಹೋದ-ರ-ರಿಗೆ
ಸಹೋದ-ರರು
ಸಹೋದ-ರರೂ
ಸಹೋದ-ರಿಗೆ
ಸಹ್ಯಜಾ
ಸಹ್ಯಜಾ-ತೀರೇ
ಸಹ್ಯಜಾ-ನದಿ-ಯ-ಕಾವೇರಿ
ಸಹ್ಯಾದ್ರಿ-ಯಿಂದ
ಸಾ
ಸಾಂಕೇತಿ-ಸುತ್ತಿದೆ
ಸಾಂಖ್ಯ
ಸಾಂಗತ್ಯ-ದಲ್ಲಿ
ಸಾಂಗ-ವಾಗಿ
ಸಾಂತತ್ಯ
ಸಾಂತಮ-ಹಂತ
ಸಾಂತಮ-ಹಂತಂ
ಸಾಂತಮ-ಹಂತನು
ಸಾಂತಿ-ಯಕ್ಕ
ಸಾಂದರ್ಭಿಕ-ವಾಗಿ
ಸಾಂದರ್ಭೋಚಿತ-ವಾಗಿ
ಸಾಂಪ್ಪನ-ಹಳ್ಳಿ
ಸಾಂಪ್ರ-ದಾಯಕ
ಸಾಂಪ್ರ-ದಾಯ-ಕ-ವಾಗಿ
ಸಾಂಪ್ರದಾ-ಯಿಕ
ಸಾಂಪ್ರದಾ-ಯಿಕ-ವಾಗಿ
ಸಾಂಬಮ್ಮ
ಸಾಂಬ್ರಾಜ್ಯಂಗಯಿಉತ
ಸಾಂಬ್ರಾಜ್ಯಂಗೈ-ಯುತ್ತಿ-ರಲು
ಸಾಂಸ್ಕೃತಿ
ಸಾಂಸ್ಕೃತಿಕ
ಸಾಂಸ್ಕೃತಿ-ಕ-ವಾಗಿ
ಸಾಕಮ್ಮ-ಈಶ್ವರ-ದೇವಾ-ಲಯ-ದಲ್ಲಿ
ಸಾಕಲ್ಯ-ವಾಗಿ
ಸಾಕಷ್ಟಿದೆ
ಸಾಕಷ್ಟು
ಸಾಕಾರ
ಸಾಕು
ಸಾಕು-ತಾಯಿ
ಸಾಕುತ್ತಿದ್ದರು
ಸಾಕುತ್ತಿದ್ದರೆ
ಸಾಕೆನೆ-ಸಿದಾಗ
ಸಾಕ್ಷಾತ್
ಸಾಕ್ಷಿ
ಸಾಕ್ಷಿ-ಗಳ
ಸಾಕ್ಷಿ-ಗ-ಳಾಗಿ
ಸಾಕ್ಷಿ-ಗ-ಳಾಗಿದ್ದ-ರೆಂದು
ಸಾಕ್ಷಿ-ಗ-ಳಾಗಿದ್ದಾರೆ
ಸಾಕ್ಷಿ-ಗ-ಳಾಗಿ-ರುತ್ತಾರೆ
ಸಾಕ್ಷಿ-ಗಳೂ
ಸಾಕ್ಷಿಣಃ
ಸಾಕ್ಷಿ-ದಾರ-ರಾಗಿ-ರುತ್ತಾರೆ
ಸಾಕ್ಷಿ-ಯಾಗಿ
ಸಾಕ್ಷಿ-ಯಾಗಿದೆ
ಸಾಕ್ಷಿ-ಯಾಗಿದ್ದ-ನೆಂದು
ಸಾಕ್ಷಿ-ಯಾಗಿದ್ದ-ರೆಂದು
ಸಾಕ್ಷಿ-ಯಾಗಿದ್ದಾನೆ
ಸಾಕ್ಷಿ-ಯಾಗಿದ್ದಾರೆ
ಸಾಕ್ಷಿ-ಯಾಗಿ-ರಲು
ಸಾಕ್ಷಿ-ಯಾಗಿ-ರುತ್ತಾನೆ
ಸಾಕ್ಷಿ-ಯಾಗಿ-ರುತ್ತಾರೆ
ಸಾಕ್ಷಿ-ಯಾಗಿ-ರು-ವು-ದನ್ನು
ಸಾಕ್ಷಿ-ಯಾಗಿ-ರುವುದು
ಸಾಕ್ಷಿ-ಯಾಗಿವೆ
ಸಾಗತ-ವಳ್ಳಿಯ
ಸಾಗರ
ಸಾಗರಕ್ಕೆ
ಸಾಗರದ
ಸಾಗರ-ನಂದಿ
ಸಾಗರ-ಮಾರಿಕಾ
ಸಾಗರ-ವನ್ನು
ಸಾಗರ-ವಿತ್ತೆಂದು
ಸಾಗರೋತ್ತರ
ಸಾಗಾ-ಣಿಕೆ
ಸಾಗಿದ್ದನ್ನು
ಸಾಗಿಸಿ
ಸಾಗಿ-ಸಿದ
ಸಾಗಿಸುತ್ತಿದ್ದ
ಸಾಗಿಸುತ್ತಿದ್ದನು
ಸಾಗಿ-ಸುವಾಗ
ಸಾಗಿ-ಹೋಗಿ-ರುವು-ದ-ರಿಂದ
ಸಾಗುತ್ತಿದ್ದ-ನೆಂದು
ಸಾಡತಿ-ಕಾತಿ
ಸಾಣೆ-ಹಳ್ಳಿ
ಸಾಣೆ-ಹಳ್ಳಿಯ
ಸಾತ-ನೂರು
ಸಾತ-ನೂರು-ಗಳು
ಸಾತಿಗ್ರಾಮ
ಸಾತಿ-ಸೆಟ್ಟಿ
ಸಾತೇ-ಗೌಡನ
ಸಾದನ್ನು-ಶೈವರು
ಸಾದಿಪ್ಪ
ಸಾದಿ-ಯಪ್ಪ
ಸಾದಿ-ಯಪ್ಪನ
ಸಾದಿ-ಯಪ್ಪ-ನಿಗೆ
ಸಾದಿ-ಯಪ್ಪನು
ಸಾದಿಸಿ
ಸಾದಿಸಿ-ದ-ರಾರ್ಪ್ಪಾಂಡೇಶನಂ
ಸಾದು
ಸಾದು-ಗದ್ಯಾಣ-ವೊಂದು
ಸಾದು-ಗೊಂಡ-ನ-ಹಳ್ಳಿ
ಸಾದು-ಗೊಂಡ-ನ-ಹಳ್ಳಿಯ
ಸಾದು-ಪುರ
ಸಾದೊ-ಳಲು
ಸಾದ್ವಾದಾಧಾರ-ಭೂತ-ರಪ್ಪ
ಸಾಧನ
ಸಾಧನಕ್ಕೆ
ಸಾಧನ-ಗಳು
ಸಾಧನ-ವಾಗಿ
ಸಾಧನೆ
ಸಾಧನೆ-ಗಳ
ಸಾಧನೆ-ಗ-ಳನ್ನು
ಸಾಧನೆ-ಗ-ಳಲ್ಲಿ
ಸಾಧನೆ-ಗಳು
ಸಾಧನೆ-ಗಳೇ-ನೆಂದು
ಸಾಧನೆಯ
ಸಾಧರ್ಮಿ
ಸಾಧರ್ಮಿ-ಗ-ಳಾದ
ಸಾಧರ್ಮಿ-ಗಳು
ಸಾಧಾರ
ಸಾಧಾರಣ
ಸಾಧಾರ-ಣ-ವಾ-ಗಿದ್ದು
ಸಾಧಾರ-ಣ-ವಾದ
ಸಾಧಾರ-ಣ-ಶೈ-ಲಿಯ
ಸಾಧಾರ-ವಾಗಿ
ಸಾಧಾರ-ವಾದ
ಸಾಧಿಪ-ನ-ವರ
ಸಾಧಿ-ಸಲು
ಸಾಧಿಸಿ
ಸಾಧಿಸಿ-ಕೊಟ್ಟ-ನೆಂದು
ಸಾಧಿ-ಸಿದ
ಸಾಧಿಸಿ-ದ-ರಾರ್ಪ್ಪಾಂಡ್ಯೇಶನಂ
ಸಾಧಿಸಿ-ದರು
ಸಾಧಿ-ಸಿವೆ
ಸಾಧು-ಸಂತರು-ಗಳ
ಸಾಧ್ಯ
ಸಾಧ್ಯತೆ
ಸಾಧ್ಯ-ತೆ-ಗಳಿವೆ
ಸಾಧ್ಯ-ತೆ-ಗಳು
ಸಾಧ್ಯ-ತೆಯೂ
ಸಾಧ್ಯ-ವಾಗ-ದವು
ಸಾಧ್ಯ-ವಾಗದೇ
ಸಾಧ್ಯ-ವಾಗುತ್ತದೆ
ಸಾಧ್ಯ-ವಾದ
ಸಾಧ್ಯ-ವಿಲ್ಲ
ಸಾಧ್ಯ-ವಿಲ್ಲದ
ಸಾಧ್ಯವೇ
ಸಾನ್ನಿಧ್ಯ-ದಲು
ಸಾನ್ನಿಧ್ಯ-ರಾದ
ಸಾಮಂತ
ಸಾಮಂತಂ
ಸಾಮಂತ-ದೇವ
ಸಾಮಂತನ
ಸಾಮಂತ-ನನ್ನಾಗಿ
ಸಾಮಂತ-ನಾಗಿ
ಸಾಮಂತ-ನಾ-ಗಿದ್ದ
ಸಾಮಂತ-ನಾಗಿದ್ದನು
ಸಾಮಂತ-ನಾಗಿದ್ದ-ನೆಂದು
ಸಾಮಂತ-ನಾ-ಗಿದ್ದು
ಸಾಮಂತ-ನಾಗಿ-ರ-ಬ-ಹುದು
ಸಾಮಂತ-ನಾದ
ಸಾಮಂತ-ನಾದರೂ
ಸಾಮಂತ-ನಿರ-ಬ-ಹುದು
ಸಾಮಂತನು
ಸಾಮಂತ-ನೆಂದರೆ
ಸಾಮಂತ-ನೆಂದು
ಸಾಮಂತ-ನೆಂದೂ
ಸಾಮಂತ-ನೆನಿಸಿದ
ಸಾಮಂತನೋ
ಸಾಮಂತ-ಪ-ದವಿ-ಗೇರಿ-ರುವುದು
ಸಾಮಂತ-ಪ-ದವಿ-ಯನ್ನು
ಸಾಮಂತರ
ಸಾಮಂತ-ರ-ಗಂಡ
ಸಾಮಂತ-ರನ್ನಾಗಿ
ಸಾಮಂತ-ರನ್ನು
ಸಾಮಂತ-ರಾಗಿ
ಸಾಮಂತ-ರಾ-ಗಿದ್ದ
ಸಾಮಂತ-ರಾಗಿದ್ದರು
ಸಾಮಂತ-ರಾಗಿದ್ದ-ರೆಂದು
ಸಾಮಂತ-ರಾಗಿದ್ದ-ವ-ರಿಗೂ
ಸಾಮಂತ-ರಾ-ಗಿದ್ದು
ಸಾಮಂತ-ರಾಗಿ-ರ-ಬ-ಹುದು
ಸಾಮಂತ-ರಾಜ-ನೆಂದೂ
ಸಾಮಂತ-ರಾದ
ಸಾಮಂತ-ರಿಗೆ
ಸಾಮಂತರು
ಸಾಮಂತ-ರು-ಗ-ಳನ್ನು
ಸಾಮಂತ-ರು-ಗಳು
ಸಾಮಂತರೂ
ಸಾಮಂತ-ರೆಲ್ಲರಂ
ಸಾಮಂತ-ರೊಡ-ಗೂಡಿದ
ಸಾಮಂತರೋ
ಸಾಮಂತ-ಸೋಮನ
ಸಾಮಂತ-ಸೋಮನು
ಸಾಮಂತಾಧಿ-ಪತಿ
ಸಾಮಂತಾಧಿ-ಪತಿ-ಯಾಗಿದ್ದ-ನೆಂದು
ಸಾಮಗ್ರಿಯನ್ನಾಗಿ
ಸಾಮನ್ತ
ಸಾಮನ್ತ-ದೆಱೆ
ಸಾಮನ್ತ-ಬಬ್ಬ
ಸಾಮನ್ತಾಧಿ-ಪತಿ
ಸಾಮಮ-ತರು
ಸಾಮರಸ್ಯ-ವನ್ನು
ಸಾಮರ್ಥ್ಯ
ಸಾಮರ್ಥ್ಯ-ಗಳಿ-ಗನು-ಗುಣ-ವಾಗಿ
ಸಾಮರ್ಥ್ಯ-ದಿಂದ
ಸಾಮ-ವೇದ
ಸಾಮ-ವೇದ-ಗ-ಳಲ್ಲಿ
ಸಾಮ-ವೇದದ
ಸಾಮಾಜಿ
ಸಾಮಾಜಿಕ
ಸಾಮಾಜಿ-ಕ-ವಾಗಿ
ಸಾಮಾಜಿ-ವಾಗಿ
ಸಾಮಾಜ್ಯದ
ಸಾಮಾ-ನನ್ನು
ಸಾಮಾನು-ಗ-ಳಿಗೆ
ಸಾಮಾನ್ಯ
ಸಾಮಾನ್ಯ-ವಾಗಿ
ಸಾಮಾನ್ಯ-ವಾ-ಗಿತ್ತು
ಸಾಮಾನ್ಯ-ವಾಗಿದೆ
ಸಾಮಾನ್ಯ-ವಾದ
ಸಾಮಿ-ದೇವ
ಸಾಮಿ-ಶೆಟ್ಟಿ
ಸಾಮಿ-ಸಂಕಡಿ
ಸಾಮ್ಯ
ಸಾಮ್ಯಕ್ಕೆ
ಸಾಮ್ಯತೆ
ಸಾಮ್ಯ-ದಿಂದ
ಸಾಮ್ಯ-ವನು
ಸಾಮ್ಯ-ವಿಲ್ಲ
ಸಾಮ್ರಾಜ್ಯ
ಸಾಮ್ರಾಜ್ಯಂಗೈಯುತ್ತಿ-ರುವಲ್ಲಿ
ಸಾಮ್ರಾಜ್ಯಕ್ಕೆ
ಸಾಮ್ರಾಜ್ಯ-ಗ-ಳಲ್ಲಿ
ಸಾಮ್ರಾಜ್ಯ-ಗ-ಳಿಗೆ
ಸಾಮ್ರಾಜ್ಯದ
ಸಾಮ್ರಾಜ್ಯ-ದಲ್ಲಿ
ಸಾಮ್ರಾಜ್ಯ-ದಲ್ಲಿದ್ದ
ಸಾಮ್ರಾಜ್ಯ-ದಿಂದ
ಸಾಮ್ರಾಜ್ಯ-ಮಮ್
ಸಾಮ್ರಾಜ್ಯ-ರಮಾ-ಮಣಿಯ
ಸಾಮ್ರಾಜ್ಯ-ವನ್ನಾಗಿ
ಸಾಮ್ರಾಜ್ಯ-ವನ್ನು
ಸಾಮ್ರಾಜ್ಯ-ವನ್ನೂ
ಸಾಮ್ರಾಜ್ಯವು
ಸಾಮ್ರಾಜ್ಯ-ಶಾಹಿಯ
ಸಾಮ್ರಾಟ-ರಲ್ಲಿಯೇ
ಸಾಮ್ರಾಟ-ರಾದ
ಸಾಯಣ
ಸಾಯ-ಣನು
ಸಾಯಣಾ-ಚಾರ್ಯ
ಸಾಯಣಾ-ಚಾರ್ಯನ
ಸಾಯಣಾ-ಚಾರ್ಯ-ನಿಗೆ
ಸಾಯಣಾರ್ಯ
ಸಾಯಣ್ಣ
ಸಾಯಣ್ಣನು
ಸಾಯಣ್ಣನೂ
ಸಾಯ-ಬೇಕು
ಸಾಯಲೇ
ಸಾಯಿರ
ಸಾಯಿರವೂ
ಸಾಯಿ-ಸಿದರು
ಸಾಯುಜ್ಯ-ವಾಗುತ್ತದೆ
ಸಾಯುತ್ತಾನೆ
ಸಾಯುತ್ತಿದ್ದರು
ಸಾಯುತ್ತಿದ್ದ-ರೆಂದು
ಸಾಯು-ವು-ದಾಗಿ
ಸಾರಂ
ಸಾರಂಗಪಾಣಿ
ಸಾರಂಗಪಾಣಿಯ
ಸಾರಂಗಿ
ಸಾರ-ವನ್ನು
ಸಾರಸ್ವತಾಢ್ಯಃ
ಸಾರಾಂಶ-ವನ್ನು
ಸಾರಿ
ಸಾರಿಗೆ
ಸಾರಿದ
ಸಾರಿ-ದ-ರೆಂದು
ಸಾರಿ-ದ-ವನು
ಸಾರಿ-ಸಿದ
ಸಾರೆಯ್ಕ
ಸಾರ್ತಂದುದು
ಸಾರ್ಥವಾಹ
ಸಾರ್ದ್ದುದಾ
ಸಾರ್ದ್ಧ-ಬಂಧಾದಿ-ಕರ್ಮಸ್ವಕಂ
ಸಾರ್ಧಮಿದಂ
ಸಾರ್ವ-ಜನಿಕ
ಸಾರ್ವ-ಜನಿಕ-ವಾಗಿ
ಸಾರ್ವಭೌಮ
ಸಾರ್ವ್ವರೀ
ಸಾಲ
ಸಾಲ-ಗಳೋ
ಸಾಲ-ಗಾ-ಮೆಯ
ಸಾಲ-ಗಾವುಂಡ
ಸಾಲ-ಗಾವುಂಡನು
ಸಾಲಗ್ರಾಮ-ದಲ್ಲಿ
ಸಾಲದೆ
ಸಾಲದೇ
ಸಾಲನ್ನೂ
ಸಾಲ-ಮಂನೆಯ
ಸಾಲಾಗಿ
ಸಾಲಾರ್
ಸಾಲಿಕೆ
ಸಾಲಿಗ್ರಾಮ
ಸಾಲಿಗ್ರಾಮದ
ಸಾಲಿಗ್ರಾಮ-ವನ್ನು
ಸಾಲಿಗ್ರಾಮ-ವಾಗಿದೆ
ಸಾಲಿದೆ
ಸಾಲಿನ
ಸಾಲಿ-ನಲ್ಲಿ
ಸಾಲು
ಸಾಲು-ಗಳ
ಸಾಲು-ಗ-ಳಲ್ಲಿ
ಸಾಲು-ಗ-ಳಿಗೆ
ಸಾಲು-ಗಳಿವೆ
ಸಾಲು-ಗಳು
ಸಾಲೂ
ಸಾಲೂರ
ಸಾಲೂ-ರು-ಮಠದ
ಸಾಲೆ
ಸಾಲೊ-ಮನ್
ಸಾಳ
ಸಾಳುವ
ಸಾಳುವ-ಗಜ-ಸಿಂಹ
ಸಾಳುವ-ತಿಕ್ಕ-ಮನ
ಸಾಳುವನ
ಸಾಳುವ-ನರ-ಸಿಂಗನ
ಸಾಳುವ-ನರ-ಸಿಂಗನು
ಸಾಳ್ವ
ಸಾವಂತ
ಸಾವಂತ-ಗೌಡ
ಸಾವಂತನ
ಸಾವಂತನು
ಸಾವಂತ-ಬ-ಸದಿಯ
ಸಾವಕಾಶ-ವಾಗಿ
ಸಾವಗ್ರಿಗಳ
ಸಾವನ್ತ
ಸಾವನ್ತನ
ಸಾವನ್ನಪ್ಪಿದ
ಸಾವನ್ನು
ಸಾವಿನ
ಸಾವಿ-ಮಲೆ
ಸಾವಿ-ಯಣ್ಣ
ಸಾವಿ-ಯಬ್ಬೆಯು
ಸಾವಿಯಬ್ಬೇಶ್ವರ
ಸಾವಿಯಬ್ಬೇಶ್ವರಕ್ಕೆ
ಸಾವಿರ
ಸಾವಿರ-ಕಾಲ-ದಿಂದ
ಸಾವಿರ-ಕೊಳಗ
ಸಾವಿರಕ್ಕೂ
ಸಾವಿರಕ್ಕೆ
ಸಾವಿರದ
ಸಾವಿರ-ವನ್ನು
ಸಾವಿರಾರು
ಸಾವು
ಸಾವು-ಕ-ಗವುಡ
ಸಾವೆ-ಗಿರಿ-ಯ-ವರೆಗೆ
ಸಾವೆ-ಯ-ಹಳ್ಳಿಯ
ಸಾವೆ-ಹಳ್ಳಿಯ
ಸಾವೆ-ಹಳ್ಳಿ-ಯನ್ನು
ಸಾವೋಜ
ಸಾಶನಮ
ಸಾಶನಮಂ
ಸಾಸನ
ಸಾಸನಂ
ಸಾಸನ-ಗಲ್ಲ
ಸಾಸನದ
ಸಾಸನಮಂ
ಸಾಸನವ
ಸಾಸನ-ವನ್ನು
ಸಾಸ-ಲಿನ
ಸಾಸ-ಲಿನಲ್ಲಿದ್ದ
ಸಾಸಲು
ಸಾಸಿರ
ಸಾಸಿರ-ಕಬ್ಬಹು
ಸಾಸಿರ-ಕಳ್ವಪ್ಪು
ಸಾಸಿರಕ್ಕೆ
ಸಾಸಿರದ
ಸಾಸಿರ-ದಲ್ಲಿ
ಸಾಸಿರ-ದೊಳಗೆ
ಸಾಸಿರ-ದೊಳು
ಸಾಸಿರ್ವರು
ಸಾಸ್ತ್ರವಿ-ನೋ-ದನುಂ
ಸಾಸ್ವತ
ಸಾಹಣಿರು
ಸಾಹಳ್ಳಿ
ಸಾಹಸ
ಸಾಹಸಕ್ಕಾಗಿ
ಸಾಹಸ-ಗ-ಳನ್ನು
ಸಾಹಸ-ಗ-ಳಿಂದ
ಸಾಹಸ-ಭೀಮ
ಸಾಹಸ-ಭೀಮ-ವಿಜಯ-ಗಳಂತಹ
ಸಾಹಸ-ಮೆಂತೆಂದಡೆ
ಸಾಹಸ-ವನ್ನು
ಸಾಹಸಿ-ಯಾದ
ಸಾಹಸೋತ್ತುಂಗಂ
ಸಾಹಿತಿ
ಸಾಹಿತಿ-ಗಳು
ಸಾಹಿತ್ಯ
ಸಾಹಿತ್ಯಕ್ಕೆ
ಸಾಹಿತ್ಯದ
ಸಾಹಿತ್ಯ-ದಲ್ಲಿ
ಸಾಹಿತ್ಯ-ನತ್ಯಂತ
ಸಾಹಿತ್ಯ-ಶಾ-ಸನ
ಸಾಹಿತ್ಯಿಕ
ಸಾಹಿರ್
ಸಾಹೇಬ್
ಸಿ
ಸಿಂ
ಸಿಂಗ-ಟ-ಗೆರೆಯ
ಸಿಂಗಡಿ
ಸಿಂಗಣ
ಸಿಂಗ-ಣಾಖ್ಯ
ಸಿಂಗಣ್ಣ
ಸಿಂಗದಂ
ಸಿಂಗ-ನ-ಪಳ್ಳಿ
ಸಿಂಗ-ನ-ಹಳ್ಳಿ
ಸಿಂಗ-ಪೆರು-ಮಾಳಿಗೆ
ಸಿಂಗ-ಪೆರು-ಮಾಳೆ
ಸಿಂಗ-ಪೆರು-ಮಾಳ್
ಸಿಂಗ-ಪೆರು-ಮಾಳ್ಗೆ
ಸಿಂಗ-ಪೆರು-ಮಾಳ್ಯೋಗಾ-ನರ-ಸಿಂಹ
ಸಿಂಗಪ್ಪ
ಸಿಂಗಪ್ಪ-ನಾಯ-ಕನು
ಸಿಂಗಪ್ಪೆರು-ಮಾಳ್
ಸಿಂಗ-ಮಲೆ
ಸಿಂಗಮ್ಮ
ಸಿಂಗಮ್ಮ-ನಿಗೆ
ಸಿಂಗಮ್ಮ-ಶಿಂಗಾರಮ್ಮ
ಸಿಂಗಯ್ಯ
ಸಿಂಗಯ್ಯನ
ಸಿಂಗಯ್ಯ-ನಿಗೆ
ಸಿಂಗಯ್ಯನು
ಸಿಂಗರ
ಸಿಂಗ-ರಾಜಯ್ಯ-ನೆಂದೂ
ಸಿಂಗ-ರಾರ್ಯ-ನನ್ನು
ಸಿಂಗ-ರಿಸಿ-ಕೊಳ್ಳುತ್ತಿದ್ದರು
ಸಿಂಗ-ರೈಯ್ಯಂಗಾರರ
ಸಿಂಗ-ರೈಯ್ಯಂಗಾರ್
ಸಿಂಗಲೆ
ಸಿಂಗ-ಳ-ದೇವ
ಸಿಂಗಾಡಿ-ದೇವನ
ಸಿಂಗಾಡಿ-ದೇವ-ನೆಂದು
ಸಿಂಗಾಡಿ-ದೇವ-ನೆಂಬ
ಸಿಂಗಾಡಿ-ನಾಯಕ
ಸಿಂಗಾ-ನಲ್ಲೂರು
ಸಿಂಗಾ-ರಾರ್ಯನ
ಸಿಂಗಾ-ರಾರ್ಯ-ನಿಗೆ
ಸಿಂಗಾ-ರಾರ್ಯನು
ಸಿಂಗಾ-ರಾರ್ಯ-ರಿಂದ
ಸಿಂಗಾ-ರಾರ್ಯ್ಯಸ್ಯ
ಸಿಂಗೆಯ
ಸಿಂಗೆಯ-ದಂಡ-ನಾಯ-ಕನ
ಸಿಂಗೆಯ-ದಂಡ-ನಾಯ-ಕನು
ಸಿಂಗೆಯ-ದಂಡ-ನಾಯ-ಕನೂ
ಸಿಂಗೆಯ-ದಂಡ-ನಾಯ-ಕ-ರು-ಗಳು
ಸಿಂಗೆಯ-ದಂಣ್ನಾಯ-ಕರು
ಸಿಂಗೆಯ-ನಾಯಕ
ಸಿಂಗೆಯ-ನಿಗೆ
ಸಿಂಗೇಶ್ವರ
ಸಿಂಘ-ಣನ
ಸಿಂಡ-ಗಾವುಂಡನ
ಸಿಂದ-ಗೆರೆ
ಸಿಂದ-ಗೆರೆಯ
ಸಿಂದ-ಘಟ್ಟ
ಸಿಂದ-ಘಟ್ಟಕ್ಕೆ
ಸಿಂದ-ಘಟ್ಟದ
ಸಿಂದ-ಘಟ್ಟ-ದಲ್ಲಿ
ಸಿಂದ-ಘಟ್ಟ-ದಿಂದ
ಸಿಂದ-ಘಟ್ಟ-ವನ್ನು
ಸಿಂದ-ಘಟ್ಟವು
ಸಿಂದ-ಘಟ್ಟ-ವೆಂದು
ಸಿಂದ-ಘಟ್ಟ-ಸೀಮೆಯ
ಸಿಂದಣ-ಸೆಟ್ಟಿ
ಸಿಂದು-ಘಟ್ಟ
ಸಿಂಧ
ಸಿಂಧ-ಗಟ್ಟ
ಸಿಂಧಗಿ
ಸಿಂಧ-ಗಿ-ರಿ-ಸಿಂಧ-ಗೆರೆ
ಸಿಂಧ-ಗೆರೆ
ಸಿಂಧ-ಗೆರೆಯ
ಸಿಂಧ-ಗೆರೆ-ಯನ್ನು
ಸಿಂಧ-ಗೋವಿಂದ
ಸಿಂಧ-ಘಟ್ಟ
ಸಿಂಧ-ಘಟ್ಟದ
ಸಿಂಧ-ಘಟ್ಟ-ದಲ್ಲಿ
ಸಿಂಧ-ಘಟ್ಟವು
ಸಿಂಧ-ಘಟ್ಟವೇ
ಸಿಂಧ-ಘಟ್ಟಸ್ಯ
ಸಿಂಧರ
ಸಿಂಧ-ರ-ಸರು
ಸಿಂಧು
ಸಿಂಧುಃ
ಸಿಂಧು-ಗೋವಿಂದ
ಸಿಂಧು-ರದ
ಸಿಂಧು-ರ-ರಾಜ-ಗಭೀರಧೀಃ
ಸಿಂಧು-ರಾಜನ
ಸಿಂಧೆಯ
ಸಿಂಧೆಯ-ನಾಯಕ
ಸಿಂಧೆಯ-ನಾಯ-ಕನ
ಸಿಂಧೆಯ-ನಾಯ-ಕ-ನಿಗೆ
ಸಿಂಧೆಯ-ನಾಯ-ಕನು
ಸಿಂಧೇಶ್ವರ
ಸಿಂಧೇಶ್ವರ-ನನ್ನು
ಸಿಂಫ್ತ್
ಸಿಂಫ್ತ್ಗ-ಳಿಗೆ
ಸಿಂಫ್ತ್ದಾರ-ರಿದ್ದು
ಸಿಂಹ
ಸಿಂಹಕ್ಕೆ
ಸಿಂಹ-ಗಳ
ಸಿಂಹ-ದಂತೆ
ಸಿಂಹ-ನಂದಿ
ಸಿಂಹ-ನಂದಿ-ಮುನಿ
ಸಿಂಹ-ನಂದಿಯ
ಸಿಂಹ-ನಂದಿಯು
ಸಿಂಹ-ನೆನಿಸಿದ್ದ-ವನೂ
ಸಿಂಹ-ಪರಿಷೆಗೂ
ಸಿಂಹ-ಪರಿಷೆಯೇ
ಸಿಂಹ-ಪಾಲು
ಸಿಂಹ-ಪೆರು-ಮಾಳ್
ಸಿಂಹ-ಪೋತ
ಸಿಂಹ-ಪೋತ-ಕಲಿ-ನೊಳಂಬಾದಿ-ರಾಜನು
ಸಿಂಹ-ಪೋತನು
ಸಿಂಹಪ್ರಾಯ-ನೆಂದು
ಸಿಂಹ-ಳ-ದೇವಿ
ಸಿಂಹ-ಳ-ದೇವಿಯ
ಸಿಂಹ-ಳ-ದೇವಿ-ಯರ
ಸಿಂಹಾಸಕ್ಕೆ
ಸಿಂಹಾ-ಸನ
ಸಿಂಹಾ-ಸನಕೆ
ಸಿಂಹಾ-ಸನಕ್ಕಾಗಿ
ಸಿಂಹಾ-ಸನಕ್ಕಿದ್ದ
ಸಿಂಹಾ-ಸನಕ್ಕೆ
ಸಿಂಹಾಸ-ನದ
ಸಿಂಹಾ-ಸನ-ದಲ್ಲಿ
ಸಿಂಹಾ-ಸನ-ದಿಂದ
ಸಿಂಹಾಸ-ನದು
ಸಿಂಹಾ-ಸನ-ವನ್ನು
ಸಿಂಹಾ-ಸನ-ವನ್ನೇರಿ-ದನು
ಸಿಂಹಾ-ಸನ-ವನ್ನೇ-ರಿದ್ದ-ರೆಂಬು-ದನ್ನು
ಸಿಂಹಾ-ಸನ-ವನ್ನೇ-ರಿದ್ದು
ಸಿಂಹಾ-ಸನವು
ಸಿಂಹಾ-ಸನಸ್ಥ-ರಾಗಿದ್ದರು
ಸಿಂಹಾ-ಸನಸ್ಥಿತ-ರಪ್ಪ
ಸಿಂಹಾ-ಸನಾಧಿಪ-ತಿ-ಗ-ಳಲ್ಲಿ
ಸಿಂಹಾ-ಸನಾಧೀಶ್ವರ
ಸಿಂಹಾ-ಸನಾಧೀಶ್ವರ-ರಾಗಿ
ಸಿಂಹಾ-ಸನಾರೂಢ-ರಾಗಿ
ಸಿಂಹಾ-ಸನಾರೂಢ-ರಾದ
ಸಿಂಹಾ-ಸನಾರೋ-ಹಣ
ಸಿಂಹಾ-ಸನಾರೋ-ಹಣದಿ
ಸಿಂಹಾ-ಸನಾಸೀನ-ನಾಗಿ
ಸಿಂಹಾಸನೇ
ಸಿಂಹಾ-ಸನೋಚಿತ
ಸಿಂಹಾಸಾನೋರ-ಹ-ಣಕ್ಕೆ
ಸಿಎಸ್
ಸಿಕ್ಕಿ
ಸಿಕ್ಕಿತು
ಸಿಕ್ಕಿದೆ
ಸಿಕ್ಕಿ-ರುವ
ಸಿಕ್ಕಿಲ್ಲ
ಸಿಕ್ಕಿವೆ
ಸಿಕ್ಷಾ-ಗುರು
ಸಿಗದೆ
ಸಿಗುತ್ತದೆ
ಸಿಗುತ್ತವೆ
ಸಿಗುತ್ತಿದ್ದ
ಸಿಗುತ್ತಿಲ್ಲ
ಸಿಗುವ
ಸಿಗು-ವಷ್ಟು
ಸಿಗು-ವು-ದಿಲ್ಲ
ಸಿಗು-ವುದು
ಸಿಗು-ವುದುಂಟು
ಸಿಡಿದೇ-ಳಲು
ಸಿಡಿಲಂತೆ
ಸಿಡಿಲ-ನಂತೆ
ಸಿಡಿಲು-ಕಲ್ಲು
ಸಿಡುಬುರೋ-ಗಕ್ಕೆ
ಸಿಣ-ಗಾಣ-ಕಟ್ಟೆಯ
ಸಿತ-ಕರ-ಗಂಡ
ಸಿತ-ಗರ
ಸಿದನು
ಸಿದವು
ಸಿದಾಯ
ಸಿದಾಯ-ವನು
ಸಿದೋ-ಜನು
ಸಿದ್ದಂಣ
ಸಿದ್ದಯ್ಯ-ದೇವರು
ಸಿದ್ದಾನ್ತದ
ಸಿದ್ದೇಶ್ವರ
ಸಿದ್ಧ-ತೆ-ಗಾಗಿ
ಸಿದ್ಧನ
ಸಿದ್ಧ-ನಂಜೇಶನ
ಸಿದ್ಧ-ನಾಥ
ಸಿದ್ಧ-ನಾಥ-ದೇವನ
ಸಿದ್ಧ-ನಾಥ-ದೇವ-ರಿಗೆ
ಸಿದ್ಧ-ನಾಥ-ದೇವರು
ಸಿದ್ಧ-ನಾ-ದನು
ಸಿದ್ಧ-ಪಡಿ-ಸ-ಬೇಕೆಂದು
ಸಿದ್ಧ-ಪಡಿ-ಸ-ಲಾಗಿದೆ
ಸಿದ್ಧ-ಪ-ಡಿಸಿ
ಸಿದ್ಧ-ಪಡಿ-ಸಿ-ಕೊಡು-ವುದ-ರಲ್ಲಿ
ಸಿದ್ಧ-ಪಡಿ-ಸಿದ
ಸಿದ್ಧ-ಪಡಿ-ಸಿದ್ದಕ್ಕಾಗಿ
ಸಿದ್ಧ-ಪಡಿ-ಸುತ್ತಿದ್ದರು
ಸಿದ್ಧ-ಪಡಿ-ಸುವ-ವ-ನೆಂದೂ
ಸಿದ್ಧ-ಪೀಠಾಧಿ-ಪತಿ-ಯಾದನು
ಸಿದ್ಧ-ಪುರುಷ-ರು-ಗಳು
ಸಿದ್ಧಪ್ಪಾಜೀ
ಸಿದ್ಧ-ಯೋಗಿ
ಸಿದ್ಧಯ್ಯ
ಸಿದ್ಧಯ್ಯ-ಗವುಡ
ಸಿದ್ಧ-ರಾಗಿ-ರುತ್ತಿದ್ದರು
ಸಿದ್ಧ-ರಾಮನ
ಸಿದ್ಧರ್ದಪ್ಪಣ್ಣ
ಸಿದ್ಧ-ಲ-ದೇವಿಯ
ಸಿದ್ಧ-ಲಿಂಗ
ಸಿದ್ಧ-ಲಿಂಗಯ್ಯ
ಸಿದ್ಧ-ವಾಗಿ
ಸಿದ್ಧ-ವಾಗಿ-ರುತ್ತಿತ್ತು
ಸಿದ್ಧ-ವಾದ
ಸಿದ್ಧ-ಹಸ್ತ-ನೆಂಬುದು
ಸಿದ್ಧ-ಹಸ್ತ-ರಾಗಿದ್ದ-ರೆಂದು
ಸಿದ್ಧಾಂತ
ಸಿದ್ಧಾಂತಕ್ಕೆ
ಸಿದ್ಧಾಂತ-ಗಳ
ಸಿದ್ಧಾಂತ-ಗ-ಳನ್ನು
ಸಿದ್ಧಾಂತ-ಗ-ಳಲ್ಲಿ
ಸಿದ್ಧಾಂತದ
ಸಿದ್ಧಾಂತ-ದೇವನು
ಸಿದ್ಧಾಂತ-ದೇವರ
ಸಿದ್ಧಾಂತ-ದೇ-ವ-ರಿಗೆ
ಸಿದ್ಧಾಂತ-ದೇವರು
ಸಿದ್ಧಾಂತನ
ಸಿದ್ಧಾಂತ-ಪಂಥ
ಸಿದ್ಧಾಂತಿ
ಸಿದ್ಧಾಂತಿ-ಗಳು
ಸಿದ್ಧಾಂತಿ-ಗಳು-ಅಷ್ಟೋಪ-ವಾಸಿ
ಸಿದ್ಧಾಂತಿ-ಗಳು-ದೇವ-ಚಂದ್ರ
ಸಿದ್ಧಾನ್ತ-ದೇವರ
ಸಿದ್ಧಾ-ಪುರ
ಸಿದ್ಧಾಯ
ಸಿದ್ಧಾ-ಯಕೆ
ಸಿದ್ಧಾ-ಯದ
ಸಿದ್ಧಾಯ-ದಲ್ಲಿ
ಸಿದ್ಧಾ-ಯದಿಂ
ಸಿದ್ಧಾಯ-ದಿಂದ
ಸಿದ್ಧಾಯ-ದೊಳಗೆ
ಸಿದ್ಧಾ-ಯಮ್
ಸಿದ್ಧಾಯ-ವನು
ಸಿದ್ಧಾಯ-ವನ್ನು
ಸಿದ್ಧಾ-ಯವು
ಸಿದ್ಧಾಯ-ವೆಂದು
ಸಿದ್ಧಾರ್ಥ
ಸಿದ್ಧಿ-ಗ-ಳನ್ನು
ಸಿದ್ಧಿಯಂ
ಸಿದ್ಧಿ-ಯನ್ನು
ಸಿದ್ಧಿ-ಸಿ-ರಲಾ-ರದು
ಸಿದ್ಧೇ-ವಡೆ-ಯರ-ಕಟ್ಟೆ
ಸಿದ್ಧೇಶ್ವರ
ಸಿದ್ಧೇಶ್ವರರ
ಸಿದ್ಧೇಶ್ವರ-ರಂತೆ
ಸಿದ್ಯರ್ಥತಃ
ಸಿದ್ಯರ್ಥ-ವಾಗಿ
ಸಿಧ-ಯನ
ಸಿಧಾಯ
ಸಿನ್ನಿ-ವರ
ಸಿಪಾಯಿ-ಗಳು
ಸಿಬ್ಬನ-ಕಟ್ಟೆಯ
ಸಿಬ್ಬಾಲ್
ಸಿಮಹ-ದೇವ
ಸಿಮಾ-ಸಂಬಂಧಿ-ಯಾದ
ಸಿಮೆಂಟ್
ಸಿರಂಗನು
ಸಿರ-ಕು-ಬಳ್ಳಿ
ಸಿರ-ಗುಪ್ಪಿ
ಸಿರ-ಗು-ವಾರುವ
ಸಿರ-ನಡು
ಸಿರ-ವ-ಗುಂದದ
ಸಿರ-ವಗುನ್ದದ
ಸಿರಿ
ಸಿರಿಗ
ಸಿರಿಯ
ಸಿರಿ-ಯ-ಕಲ-ಸತ್ತ-ಪಾಡಿ
ಸಿರಿ-ಯ-ಕಲ-ಸತ್ತು-ಪಾಡಿ
ಸಿರಿ-ಯ-ಕಲ-ಸತ್ತು-ಪಾಡಿ-ಯಾದ
ಸಿರಿ-ಯ-ಗ-ವುಡನ
ಸಿರಿ-ಯಣ್ಣ-ನನ್ನು
ಸಿರಿ-ರಂಗ-ದಾಸ
ಸಿರಿ-ರಂಗ-ನಾಯಕ
ಸಿರಿ-ರಂಗ-ನಾಯ-ಕನ
ಸಿರಿ-ರಂಗ-ಪುರ
ಸಿರಿ-ರಂಗ-ಪುರದ
ಸಿರಿ-ವನ-ಹಳ್ಳಿಯ
ಸಿರು-ಮಯನ್
ಸಿರೋಂಬುಜಮಂ
ಸಿಲೆ
ಸಿಲೋನ್
ಸಿವಡಿ-ಮರ್ಯಾದೆಯ
ಸಿವ-ದೇವಂ
ಸಿವನ
ಸಿವ-ನಂಜಯ್ಯ-ನಿಗೆ
ಸಿವ-ನಂಜಯ್ಯನು
ಸಿವನೆ-ನಾಯ-ಕನ
ಸಿವ-ನೆಯ-ನಾಯಕ
ಸಿವ-ಪಾದ-ಸೇಖರೆ
ಸಿವ-ಪುರದ
ಸಿವ-ಪುರವ
ಸಿವ-ಪುರ-ವನ್ನಾಗಿ
ಸಿವ-ಮಯ್ಯ-ಗವುಡ
ಸಿವ-ಯೋಗಿ
ಸಿವರ-ಪುರ
ಸಿವ-ರಮ್ಯ-ಗೇಹ-ವನ್ನು-ಶಿವಾ-ಲಯ
ಸಿವ-ರಾಯ-ಕೆರೆ
ಸಿವ-ಸ-ಮೆಯ
ಸಿವಾ-ಚಾರ
ಸಿವಾ-ಚಾರ-ಸಂಪಂನರು-ಮಪ
ಸಿವಾ-ಲಯ
ಸಿವಾ-ಲ-ಯಕ್ಕೆ
ಸಿವಾ-ಲಯ-ಗ-ಳನ್ನು
ಸಿವಾ-ಲ-ಯದ
ಸಿವಾಲ್ಯ
ಸಿವಾಲ್ಯಂಗಳ-ನತ್ಯುಂನದಿಂ
ಸಿವೆ-ಯ-ನಾಯಕ
ಸಿವೋಜಿ-ನಾಯಕ
ಸಿವೋಜಿ-ನಾಯ-ಕನು
ಸಿಷ್ಯ
ಸಿಸು
ಸಿಸ್ಯರು
ಸಿಸ್ಯಿ
ಸೀಕ್ಷಾ-ಗುರು
ಸೀಗೆ
ಸೀಗೆ-ಯ-ನಾಡು
ಸೀತಾ
ಸೀತಾಂಬಿಕಾ
ಸೀತಾ-ಪುರ
ಸೀತಾ-ಪುರದ
ಸೀತಾ-ಪುರ-ವಾಗಿ-ರ-ಬ-ಹುದು
ಸೀತಾ-ಪುರ-ವಾದ
ಸೀತಾ-ಪುರ-ವೆಂಬ
ಸೀತಾ-ಯಂಮ-ನ-ವರ
ಸೀತಾ-ಯಂಮ-ವನರ
ಸೀತಾ-ಯಮ್ಮ-ನ-ವರ
ಸೀತಾ-ರಾಮ
ಸೀತಾ-ರಾಮ-ಜಾಗಿರ್ದಾರ್
ಸೀತಾ-ರಾಮನ
ಸೀತಾ-ಲಕ್ಷ್ಮ-ಣ-ಸೇವಿತ-ರಾದ
ಸೀತೆ
ಸೀತೆ-ಯಂತಿದ್ದಳು
ಸೀತೈ-ಯಾಂಡಾಳ್
ಸೀತೈಯಾಣ್ಡಾಳ್
ಸೀತೋಜನ
ಸೀದಾ
ಸೀದಾಧ್ಯರ್ಮಾನ-ತೋ-ಹರ
ಸೀನಣ್ಣನು
ಸೀಪನ-ಮರಡಿ
ಸೀಮಾ
ಸೀಮಾಂತರ-ಗ-ಳನ್ನು
ಸೀಮಾಂತರ-ದಲ್ಲಿ
ಸೀಮಾಂತ-ವರ್ತಿನ
ಸೀಮಾ-ಧಿ-ಕಾರಿ-ಗ-ಳಿಗೆ
ಸೀಮಾ-ಧಿ-ಕಾರಿ-ಗಳು
ಸೀಮಾ-ಧಿ-ಕಾರಿ-ಗಳೂ
ಸೀಮಾ-ರೇಖೆ-ಗಳೂ
ಸೀಮಾ-ವಿ-ವಾದ-ದಲ್ಲಿ
ಸೀಮಾ-ಸಂಬಂಧದ
ಸೀಮಾ-ಸಂಬಂಧ-ವಾದ
ಸೀಮಾ-ಸಂಬಂಧಿ
ಸೀಮಾ-ಸ-ಮನ್ವಿತ-ವಾಗಿ
ಸೀಮಾ-ಸಹಿತ-ವಾಗಿ
ಸೀಮಿತ-ಗೊ-ಳಿಸಿ
ಸೀಮಿತತೆ
ಸೀಮಿತ-ರಾಗಿ-ರುವ
ಸೀಮಿತ-ವಾಗ-ಬೇಕಾ-ಯಿತು
ಸೀಮಿತ-ವಾಗಿ
ಸೀಮಿತ-ವಾಗಿದೆ
ಸೀಮಿತ-ವಾಗಿದ್ದ-ರೆಂದು
ಸೀಮಿತ-ವಾಗಿವೆ
ಸೀಮಿತವಾ-ಗುತ್ತಾ
ಸೀಮೆ
ಸೀಮೆ-ಗ-ಳನ್ನು
ಸೀಮೆ-ಗ-ಳಿಗೆ
ಸೀಮೆ-ಗಳು
ಸೀಮೆಗೆ
ಸೀಮೆ-ನಾಡು
ಸೀಮೆ-ಮೈಸೂರು
ಸೀಮೆಯ
ಸೀಮೆ-ಯನು
ಸೀಮೆ-ಯನ್ನಾಗಿ
ಸೀಮೆ-ಯನ್ನು
ಸೀಮೆ-ಯನ್ನು-ಇಂದಿನ
ಸೀಮೆ-ಯಲ್ಲಿ
ಸೀಮೆ-ಯಲ್ಲಿದ್ದ
ಸೀಮೆ-ಯ-ವ-ನಿರ-ಬಹು-ದೆಂದು
ಸೀಮೆ-ಯ-ವರೆ-ಗಿನ
ಸೀಮೆ-ಯಾಗಿ
ಸೀಮೆ-ಯಾ-ಗಿತ್ತು
ಸೀಮೆ-ಯಾ-ಗಿದ್ದ
ಸೀಮೆ-ಯಾಗಿದ್ದವು
ಸೀಮೆಯಿಂ
ಸೀಮೆ-ಯಿಂದೊಳಗೆ
ಸೀಮೆಯು
ಸೀಮೆಯೇ
ಸೀಮೆ-ಯೊಳ-ಗಣ
ಸೀಮೆ-ಯೊಳ-ಗಿನ
ಸೀಮೆ-ಯೊಳಗೆ
ಸೀಮೆ-ಸಹಿತ
ಸೀಮೆಸ್ಥಳ
ಸೀಯ-ಕನು
ಸೀರೆ
ಸೀರೆ-ಯನ್ನು
ಸೀರೇ-ಹಳ್ಳಿ
ಸೀರ್ಯ
ಸೀರ್ಯದ
ಸೀಲ್
ಸೀಳನೆರೆ
ಸೀಳಿದ-ನೆಂದು
ಸೀಶ್ರೀ
ಸು
ಸುಂಕ
ಸುಂಕಃ
ಸುಂಕ-ಆಯ-ಇ-ತರ
ಸುಂಕಕ್ಕೆ
ಸುಂಕ-ಗದ್ಯಾಣ-ವೈದು
ಸುಂಕ-ಗಳ
ಸುಂಕ-ಗ-ಳನ್ನು
ಸುಂಕ-ಗ-ಳನ್ನೂ
ಸುಂಕ-ಗ-ಳನ್ನೇ
ಸುಂಕ-ಗ-ಳಲ್ಲಿ
ಸುಂಕ-ಗಳು
ಸುಂಕ-ತೆ-ರಿಗೆ
ಸುಂಕ-ತೆ-ರಿಗೆ-ಯನ್ನು
ಸುಂಕ-ತೊಂಡ-ನೂರು
ಸುಂಕದ
ಸುಂಕ-ದಲ್ಲಿ
ಸುಂಕ-ದ-ವರು
ಸುಂಕ-ದ-ವರೂ
ಸುಂಕ-ದ-ಹೆಗ್ಗಡೆ
ಸುಂಕ-ದ-ಹೆಗ್ಗಡೆ-ಗಳ
ಸುಂಕ-ದ-ಹೆಗ್ಗಡೆ-ಗಳು
ಸುಂಕ-ದಿಂದ
ಸುಂಕ-ದೊಳಗೆ
ಸುಂಕ-ರೂಪದ
ಸುಂಕ-ವನು
ಸುಂಕ-ವನ್ನು
ಸುಂಕ-ವನ್ನೂ
ಸುಂಕ-ವನ್ನೇ
ಸುಂಕ-ವಿತ್ತೆಂದು
ಸುಂಕ-ವಿಲ್ಲ-ದ-ವರು
ಸುಂಕವು
ಸುಂಕ-ವೆಂಬ
ಸುಂಕವೇ
ಸುಂಕಾ-ತೊಂಡ-ನೂರಿನ
ಸುಂಕಾ-ತೊಂಡ-ನೂರಿನಲ್ಲಿ
ಸುಂಕಾ-ತೊಂಡ-ನೂರು
ಸುಂಕಾಧಿ-ಕಾರಿ-ಯಾ-ಗಿದ್ದ
ಸುಂಖವ
ಸುಂಙ್ಕಕ್ಕೆ
ಸುಂದರ
ಸುಂದರ-ಪಾಂಡ್ಯನ
ಸುಂದರ-ಪಾಂಡ್ಯ-ನಿಗೆ
ಸುಂದರ-ವಾಗಿ
ಸುಂದರ-ವಾಗಿದೆ
ಸುಂದರ-ವಾ-ಗಿದ್ದು
ಸುಂದರ-ವಾಗಿವೆ
ಸುಂದರ-ವಾದ
ಸುಂದರವೂ
ಸುಂದರೀ
ಸುಂದಿಂ
ಸುಂದೇ-ಹಳ್ಳಿ
ಸುಕ
ಸುಕ-ದರೆ
ಸುಕದಿಂ
ಸುಕ-ದೋರ
ಸುಕ-ದೋರ-ಸುಗ-ಧರೆ
ಸುಕ-ದೋರಾ-ಇಂದಿನ
ಸುಕ-ಧರೆ
ಸುಕ-ಧರೆ-ಯಲ್ಲಿ
ಸುಕ-ನಾಸಿ
ಸುಕಪ್ರಾಪ್ತ-ನೆಂದು
ಸುಕ-ವಿ-ಜನ
ಸುಕ-ವೀಂದ್ರಲಪ-ವನ
ಸುಕ-ಸಂಕಥಾ-ವಿನೋದದಿಂ
ಸುಕು-ಮಾರ
ಸುಕು-ಮಾರ-ಚರಿತೆ-ಯಲ್ಲಿ
ಸುಕ್ಕು-ಧರೆ
ಸುಕ್ಕು-ಧ-ರೆಯ
ಸುಕ್ಕು-ಧರೆ-ಯಲ್ಲಿ
ಸುಕ್ಕು-ಧರೆ-ಯಲ್ಲಿ-ಇಂದಿನ
ಸುಖದಿಂ
ಸುಖ-ದಿಂದ
ಸುಖದಿಂದಿರೆ
ಸುಖ-ದಿನ-ರಸು-ಗೆಯ್ಯುತ್ತ-ಮಿರೆ
ಸುಖ-ದೊರೆ
ಸುಖ-ದೊರೆ-ಯನ್ನು
ಸುಖ-ಧರೆ-ಯಲ್ಲಿ
ಸುಖ-ನಾಸಿ
ಸುಖ-ನಾಸಿಕ
ಸುಖ-ನಾಸಿ-ಗ-ಳಿಂದ
ಸುಖ-ನಾಸಿಯ
ಸುಖ-ನಾಸಿ-ಯನ್ನು
ಸುಖ-ನಾಸಿ-ಯಲ್ಲಿ
ಸುಖ-ನಾಸಿ-ಯಲ್ಲಿ-ರುವ
ಸುಖ-ನಿ-ವಾಸ
ಸುಖ-ರಾಜ್ಯಂಗೈಯುತ್ತಿದ್ದರು
ಸುಖ-ರಾಜ್ಯ-ಗೆಯುತ್ತಿದ್ದ-ರೆಂದು
ಸುಖ-ರೂಪ-ದೇವರು
ಸುಖಸಂಕಥಾ
ಸುಖಸಂಪದಮಂ
ಸುಖೀಭವ
ಸುಗಣೋ
ಸುಗ-ತಸ್ಯ
ಸುಗ-ಧರೆ
ಸುಗ-ಧ-ರೆಯು
ಸುಗಮ-ವಾಗಿ
ಸುಗುಣೋ-ಧಿ-ಮಾನ್
ಸುಗ್ಗ-ಗವುಂಡನ
ಸುಗ್ಗ-ಗೌಂಡ
ಸುಗ್ಗಲ-ದೇವಿ
ಸುಚರಿತ್ರೆ
ಸುಚಿಸು-ಧನ್ವ
ಸುಜ್ಜ-ಲೂರಿನ
ಸುಜ್ಜ-ಲೂರಿನಲ್ಲಿದ್ದ
ಸುಜ್ಜ-ಲೂರು
ಸುಟ್ಟು-ಕೊಂಡರು
ಸುಟ್ಟು-ಕೊಳ್ಳುವಿಕೆ-ಯನ್ನು
ಸುಡೆ
ಸುಣ್ಣ-ದಲ್ಲಿ
ಸುಣ್ಣ-ಬಣ್ಣ
ಸುಣ್ಣ-ಬಣ್ಣ-ಗ-ಳಿಗೆ
ಸುಣ್ಣ-ಬಣ್ಣ-ವನ್ನು
ಸುಣ್ಣ-ಹರ-ಳಿನ-ಹಳ್ಳದ
ಸುತ
ಸುತಂ
ಸುತರು
ಸುತೆ
ಸುತ್ತ
ಸುತ್ತಣ
ಸುತ್ತ-ಮುತ್ತ
ಸುತ್ತ-ಮುತ್ತಲ
ಸುತ್ತ-ಮುತ್ತ-ಲಿನ
ಸುತ್ತ-ಮುತ್ತಲೂ
ಸುತ್ತ-ಲಿನ
ಸುತ್ತಲೂ
ಸುತ್ತ-ಳಿ-ಯದೆ
ಸುತ್ತಾ-ಲಯ
ಸುತ್ತಾ-ಲ-ಯದ
ಸುತ್ತಾ-ಲಯ-ವನ್ನು
ಸುತ್ತಿ
ಸುತ್ತಿ-ಕೊಂಡ
ಸುತ್ತಿನ
ಸುತ್ತಿರ್ದ್ದ
ಸುತ್ತು-ವರೆ-ದನು
ಸುತ್ತೂರು
ಸುದ
ಸುದತ್ತಾ-ಚಾರ್ಯ-ನೆಂದೇ
ಸುದತ್ತಾ-ಚಾರ್ಯರ
ಸುದರ್ಶನ
ಸುದರ್ಶನಾ-ಚಾರ್ಯ-ನಾದ
ಸುದ್ದಿ-ಯನ್ನು
ಸುಧರ್ಮ
ಸುಧರ್ಮ-ಪುರ
ಸುಧರ್ಮ-ಪುರ-ಗ-ಳನ್ನು
ಸುಧರ್ಮ-ಪುರ-ವಾಗಿ
ಸುಧರ್ಮ-ಪುರ-ವಾದ
ಸುಧಾಂಶುರಿವ
ಸುಧಾ-ಕರಂ
ಸುಧಾಕ-ರರುಂ
ಸುಧಾನಿಧೇಃ
ಸುಧಾರಿತ
ಸುಧಾರ್ಣವದ
ಸುಧಾರ್ಣವ-ವನ್ನು
ಸುಧಿಯೇ
ಸುಧೀಯಾಂದಿವಃ
ಸುಧೀವಕ್ತ್ರ
ಸುಧೃಡ-ವಾದ
ಸುನಾರ-ಖಾನೆ
ಸುನಾರ್
ಸುನಾರ್ಖಾನೆ
ಸುನಾರ್ಖಾನೆ-ಯಒಡವೆ
ಸುನೆ
ಸುಪತ್ರ
ಸುಪರ್ಣ-ವಿಶ್ವಜ್ಞ
ಸುಪರ್ದಿಗೆ
ಸುಪುತ್ರ
ಸುಪುತ್ರಂ
ಸುಪುತ್ರ-ನಪ್ಪ
ಸುಪುತ್ರ-ರಾದ
ಸುಪ್ರತಿಷ್ಠಿತ
ಸುಪ್ರತಿಷ್ಠೆಯಂ
ಸುಪ್ರತೀಕ
ಸುಪ್ರತೀಕ-ಗಜದ
ಸುಪ್ರ-ಸನ್ನ-ನಾಗ-ಬೇಕೆಂದು
ಸುಪ್ರ-ಸನ್ನಾಂಬಿಕಾ
ಸುಪ್ರ-ಸಿದ್ಧ
ಸುಪ್ರ-ಸಿದ್ಧ-ನಾದ
ಸುಪ್ರ-ಸಿದ್ಧ-ವಾದ
ಸುಬೇ-ದಾರ್
ಸುಬ್ಬಮ್ಮ
ಸುಬ್ಬಯ್ಯನ
ಸುಬ್ಬಯ್ಯನ-ವರ
ಸುಬ್ಬ-ರಾಯ
ಸುಬ್ಬ-ರಾಯ-ಕೊಪ್ಪ-ಲಿನ
ಸುಬ್ಬ-ರಾಯನ
ಸುಬ್ಬ-ರಾಯ-ನ-ಕೊಪ್ಪಲು
ಸುಬ್ಬ-ರಾಯನು
ಸುಬ್ಬಾ-ಪಂಡಿ-ತನು
ಸುಬ್ರಹ್ಮಣ್ಯ
ಸುಬ್ರಹ್ಮಣ್ಯನ
ಸುಬ್ರಹ್ಮಣ್ಯ-ನನ್ನು
ಸುಬ್ರಹ್ಮಣ್ಯ-ನೆಂಬ
ಸುಭಟರ
ಸುಭಟರಟ್ಟೆ-ಗಳಾ-ಡಲು
ಸುಭಟ-ರನ್ನು
ಸುಭಟ-ರಾ-ದಿತ್ಯನುಂ
ಸುಭದ್ರ-ವಾಗಿದೆ
ಸುಭದ್ರವೂ
ಸುಭದ್ರಸ್ಥಿತಿ-ಯಲ್ಲಿದ್ದು
ಸುಭಾಶ್
ಸುಮನಾ
ಸುಮಾ-ರಾಗಿ
ಸುಮಾ-ರಾದ
ಸುಮಾರಿಗೆ
ಸುಮಾರಿ-ನಲ್ಲಿ
ಸುಮಾರು
ಸುಮಿತ್ರೆ-ಯ-ರಂತೆ
ಸುಮ್ಮನೆ
ಸುರ
ಸುರಂಗ
ಸುರಂಗದ
ಸುರ-ಕರಿಯ
ಸುರ-ಕುಜದ
ಸುರಕ್ಷಿತ-ವಾಗಿ
ಸುರ-ಗ-ಣಿಕೆಯ
ಸುರ-ಗ-ಣಿಕೆ-ಯರ
ಸುರ-ಗ-ಣಿಕೆ-ಯ-ರಿಂಗೆ
ಸುರಗಿ
ಸುರ-ಗಿಯ
ಸುರ-ತರುಸ್ಪರ್ಧಾ-ಳುವಿಶ್ರಾ-ಣನಃ
ಸುರತ್ರಾಣ
ಸುರ-ದುಂದುಭಿ-ಗಳೆ-ಸೆಯೆ
ಸುರ-ಪದಮಂ
ಸುರ-ಪುರ
ಸುರ-ರಾಜ-ಪೂಜ್ಯ
ಸುರ-ಲೋಕ
ಸುರ-ಲೋಕಂ
ಸುರ-ಲೋಕಪ್ರಾಪ್ತ
ಸುರ-ಲೋಕಪ್ರಾಪ್ತ-ನಾಗುತ್ತಾನೆ
ಸುರ-ಲೋಕಪ್ರಾಪ್ತ-ನಾ-ದನು
ಸುರ-ಲೋಕಪ್ರಾಪ್ತ-ನಾದ-ನೆಂದು
ಸುರ-ಲೋಕ-ವಿಳಾಸ
ಸುರ-ಸ-ರಲ್ಲೀಲ
ಸುರ-ಹೊನ್ನೆ
ಸುರಾಂಗನಾ
ಸುರಾ-ಸುರ-ಯುದ್ಧ
ಸುರಿಗೆ
ಸುರಿಗೆ-ಕಾರ
ಸುರಿಗೆ-ನಾಗಯ್ಯನ
ಸುರಿಗೆಯ
ಸುರಿಗೆ-ಯನ್ನು
ಸುರಿಗೆ-ಯಿಂದ
ಸುರಿಗೆ-ವಿಡಿವ
ಸುರಿ-ತಾಣ-ಭೂ-ಪರಂ
ಸುರಿದ
ಸುರಿ-ದಾನು
ಸುರಿ-ಯು-ವುದು
ಸುರುಚಿರ
ಸುರೇಂದ್ರ-ತೀರ್ಥ
ಸುರ್ಣಾ-ದಾಯ-ವನೂ
ಸುಲಭ
ಸುಲಭ-ವಾಗಿ
ಸುಲಿಗೆ
ಸುಲ್ತಾನ
ಸುಲ್ತಾ-ನನ
ಸುಲ್ತಾ-ನನು
ಸುಲ್ತಾ-ನರ
ಸುಲ್ತಾನ್
ಸುಳ್ಳರಿ-ಗುಂಟು
ಸುಳ್ಳು
ಸುವರ್ಣ
ಸುವರ್ಣ-ಗರುಡ
ಸುವರ್ಣ-ದಾನ-ಶೂರ
ಸುವರ್ಣ-ಮಂಟಪ
ಸುವರ್ಣ-ಯುಗ-ವೆಂದು
ಸುವರ್ಣಾ-ದಾಯ
ಸುವರ್ಣಾ-ದಾಯ-ವನ್ನು
ಸುವರ್ಣಾಯ-ದಲ್ಲಿ
ಸುವರ್ಣ್ನ-ದಾನ
ಸುವರ್ನಾದಾಯ
ಸುವರ್ನ್ನ
ಸುವರ್ನ್ನಾಯ
ಸುವ್ಯವಸ್ಥೆಗೆ
ಸುಶೀಲೆ
ಸುಸಂಸ್ಕೃತ
ಸುಸಜ್ಜಿತ-ಗೊಳಿ-ಸಲು
ಸುಸೂತ್ರ-ವಾಗಿ
ಸುಸ್ಥಿತಿ-ಯಲ್ಲಿದೆ
ಸುಸ್ಥಿತಿ-ಯಲ್ಲಿದ್ದು
ಸುಸ್ಥಿರ-ತೆ-ವೆತ್ತುದಾ-ಚಂದ್ರಾರ್ಕ್ಕಂ
ಸುಹೃತ್ವ
ಸುಹೃತ್ವಕ್ತ್ರಾಬ್ಜ
ಸೂಕ್ತ
ಸೂಕ್ತ-ವಲ್ಲ-ವೆಂದು
ಸೂಕ್ತ-ವಾಗಿದೆ
ಸೂಕ್ತ-ವಾಗಿ-ದೆ-ಯೆಂದು
ಸೂಕ್ತ-ವಾಗುತ್ತದೆ
ಸೂಕ್ತ-ವಾದ
ಸೂಕ್ತ-ವಾದು-ದಲ್ಲ
ಸೂಕ್ತಿ
ಸೂಕ್ತಿ-ಕುಶಲ
ಸೂಕ್ತಿ-ಸುಧಾರ್ಣವ
ಸೂಕ್ತಿ-ಸುಧಾರ್ಣವದ
ಸೂಕ್ರ-ವಾಗಿದೆ
ಸೂಕ್ಷ್ಮ-ವಾಗಿ
ಸೂಗುರು
ಸೂಗೂರಿನ
ಸೂಚಕ
ಸೂಚಕ-ಗ-ಳಾಗಿವೆ
ಸೂಚನೆ
ಸೂಚನೆ-ಗ-ಳನ್ನು
ಸೂಚನೆ-ಗಳಿವೆ
ಸೂಚನೆ-ಗಳು
ಸೂಚನೆ-ಗಾಗಿ
ಸೂಚನೆ-ಯಾಗಿದೆ
ಸೂಚಿ
ಸೂಚಿ-ಕಬ್ಬೆ
ಸೂಚಿ-ಗ-ಳಾದರೆ
ಸೂಚಿತ
ಸೂಚಿ-ತ-ವಾಗಿ-ರುವ
ಸೂಚಿ-ತ-ವಾಗುತ್ತಿತ್ತು
ಸೂಚಿ-ಸ-ಬ-ಹುದು
ಸೂಚಿ-ಸ-ಲಾಗಿದೆ
ಸೂಚಿ-ಸ-ಲಾಗುತ್ತದೆ
ಸೂಚಿ-ಸಲು
ಸೂಚಿ-ಸಿ-ದಂತೆ
ಸೂಚಿ-ಸಿದರು
ಸೂಚಿ-ಸಿದರೆ
ಸೂಚಿ-ಸಿದೆ
ಸೂಚಿ-ಸಿರುವ
ಸೂಚಿ-ಸುತ್ತದ
ಸೂಚಿ-ಸುತ್ತದೆ
ಸೂಚಿ-ಸುತ್ತ-ದೆಂದು
ಸೂಚಿ-ಸುತ್ತವೆ
ಸೂಚಿ-ಸುತ್ತ-ವೆಂದು
ಸೂಚಿ-ಸುತ್ತ-ವೆ-ಯೆಂದು
ಸೂಚಿ-ಸುತ್ತಾನೆ
ಸೂಚಿ-ಸುತ್ತಿದೆ
ಸೂಚಿ-ಸುತ್ತಿದ್ದವು
ಸೂಚಿ-ಸುತ್ತಿದ್ದು
ಸೂಚಿ-ಸುತ್ತಿ-ರ-ಬ-ಹುದು
ಸೂಚಿ-ಸುತ್ತಿ-ರು-ವಂತೆ
ಸೂಚಿ-ಸುವ
ಸೂಚಿ-ಸು-ವು-ದಿಲ್ಲ
ಸೂಚ್ಯ-ವಾಗಿ
ಸೂತ
ಸೂತ-ಕುಲ
ಸೂತ-ಕುಲದ
ಸೂತರು
ಸೂತ್ರ
ಸೂತ್ರ-ಗಳ
ಸೂತ್ರ-ಗ-ಳನ್ನು
ಸೂತ್ರ-ಗ-ಳಲ್ಲಿ
ಸೂತ್ರ-ಗಳು
ಸೂತ್ರ-ಗುತ್ತಗೆ
ಸೂತ್ರ-ಗುತ್ತ-ಗೆ-ಯಾಗಿ
ಸೂತ್ರ-ಗುತ್ತಿಗೆ
ಸೂತ್ರದ
ಸೂತ್ರ-ದ-ವರು
ಸೂತ್ರ-ಧಾರಿ
ಸೂತ್ರ-ರೂಪ-ದಲ್ಲಿ
ಸೂತ್ರ-ವನ್ನು
ಸೂತ್ರಿಣೇ
ಸೂದ್ರಕ
ಸೂನವೇ
ಸೂನು
ಸೂನುಃ
ಸೂನೃತೋಕ್ತಯೇ
ಸೂರ-ಜೀಯನ
ಸೂರನ-ಹಳ್ಳಿ
ಸೂರನ-ಹಳ್ಳಿಯ
ಸೂರನ-ಹಳ್ಳಿಯಂ
ಸೂರನ-ಹಳ್ಳಿ-ಯನ್ನು
ಸೂರನ-ಹಳ್ಳಿ-ಯನ್ನು-ಮೆಚ್ಚುಗೆ-ಯಾಗಿ
ಸೂರನ-ಹಳ್ಳಿಯು
ಸೂರಪ್ಪನ
ಸೂರಸ್ತ-ಗಣದ
ಸೂರಸ್ಥ
ಸೂರಸ್ಥ-ಗಣದ
ಸೂರಾ-ಚಾರಿ
ಸೂರಾ-ಚಾರಿಯ
ಸೂರಾ-ಚಾರ್ಯನ
ಸೂರಿಣಾ
ಸೂರಿ-ಯ-ಭಟ್ಟ
ಸೂರೆ
ಸೂರೆ-ಗೊಂಡಳ್
ಸೂರೆ-ಗೊಂಡು
ಸೂರೆ-ಮಾಡಿ-ದನು
ಸೂರೆ-ಮಾಡಿ-ದ-ರೆಂದು
ಸೂರೋಜ
ಸೂರ್ಯ
ಸೂರ್ಯಗ್ರ-ಹಣದ
ಸೂರ್ಯಗ್ರ-ಹಣ-ದಂದು
ಸೂರ್ಯ-ದೇವಾ-ಲಯ-ದಲ್ಲಿದೆ
ಸೂರ್ಯನ
ಸೂರ್ಯ-ನಾಥ
ಸೂರ್ಯ-ನಾಥ-ಕಾ-ಮತ್
ಸೂರ್ಯಪ್ರತಿಷ್ಠೆ
ಸೂರ್ಯಪ್ರತಿಷ್ಠೆ-ಯನ್ನು
ಸೂರ್ಯಾ-ಭರಣ
ಸೂರ್ಯ್ಯಪ್ರತಿ-ಟೆಯಂ
ಸೂಳೆ
ಸೂಳೆ-ದೆಱೆ
ಸೂಳೆ-ಯ-ರಿಗೆ
ಸೂಳ್ನೆರೆದ
ಸೆಕೆಯು
ಸೆಜ್ಜೆ
ಸೆಜ್ಜೆೆ-ಮನೆ-ಯಲ್ಲಿ
ಸೆಟಿ-ಯ-ಕೆರೆ
ಸೆಟ್ಟರ
ಸೆಟ್ಟಿ
ಸೆಟ್ಟಿ-ಕಾರ
ಸೆಟ್ಟಿ-ಕೆರೆ
ಸೆಟ್ಟಿ-ಗ-ಗ-ಳಿಗೆ
ಸೆಟ್ಟಿ-ಗ-ಭೀಷ್ಟ
ಸೆಟ್ಟಿ-ಗಳ
ಸೆಟ್ಟಿ-ಗ-ಳಿಗೆ
ಸೆಟ್ಟಿ-ಗಳು
ಸೆಟ್ಟಿ-ಗವುಂಡ
ಸೆಟ್ಟಿ-ಗ-ವುಡನ
ಸೆಟ್ಟಿ-ಗುತ್ತ
ಸೆಟ್ಟಿ-ಗುತ್ತ-ಗೆ-ಯಾಗಿ
ಸೆಟ್ಟಿಗೆ
ಸೆಟ್ಟಿ-ಗೆರೆ
ಸೆಟ್ಟಿ-ಗೆ-ರೆಯು
ಸೆಟ್ಟಿ-ತಿಗೂ
ಸೆಟ್ಟಿ-ತಿ-ಯನ್ನು
ಸೆಟ್ಟಿ-ತಿ-ಯರು
ಸೆಟ್ಟಿ-ತಿಯ-ರೆಂದು
ಸೆಟ್ಟಿ-ಪಟ್ಟ-ವನ್ನು
ಸೆಟ್ಟಿ-ಪಟ್ಟವು
ಸೆಟ್ಟಿ-ಪುರ
ಸೆಟ್ಟಿ-ಬಣಂಜಿಗ
ಸೆಟ್ಟಿಯ
ಸೆಟ್ಟಿ-ಯಣ್ಣ
ಸೆಟ್ಟಿ-ಯನ್ನು
ಸೆಟ್ಟಿ-ಯ-ಪಟ್ಟ
ಸೆಟ್ಟಿ-ಯರ
ಸೆಟ್ಟಿ-ಯ-ರ-ಳಿಯ
ಸೆಟ್ಟಿ-ಯರು
ಸೆಟ್ಟಿ-ಯ-ಹಳ್ಳಿಯ
ಸೆಟ್ಟಿ-ಯ-ಹಳ್ಳಿ-ಯ-ಕೆರೆ
ಸೆಟ್ಟಿ-ಯಾದ
ಸೆಟ್ಟಿಯು
ಸೆಟ್ಟಿ-ವಅಟ್ಟವಂ
ಸೆಟ್ಟಿ-ವಟ್ಟ
ಸೆಟ್ಟಿ-ವಟ್ಟವಂ
ಸೆಟ್ಟಿ-ವಟ್ಟ-ವನ್ನು
ಸೆಟ್ಟಿ-ವಟ್ಟವು
ಸೆಟ್ಟಿ-ವಟ್ವಂ
ಸೆಟ್ಟಿ-ಹಳ್ಳಿ
ಸೆಟ್ಟಿ-ಹಳ್ಳಿಯ
ಸೆಟ್ಟಿ-ಹಳ್ಳಿ-ಯನ್ನು
ಸೆಣಸಿ
ಸೆಣಸೆ
ಸೆಣೆಯ-ಕಟ್ಟೆ
ಸೆಣೆ-ಯು-ವುದು
ಸೆತ್ತೆ
ಸೆಪ್ಟೆಂಬರ್
ಸೆರಗು-ವಾರ್ದಪೊರಿನ್ನಿರ-ನೆಂದು
ಸೆರೆ-ಯಲ್ಲಿಟ್ಟನು
ಸೆರೆ-ಯಲ್ಲಿಟ್ಟಿದ್ದನು
ಸೆರೆ-ಯಲ್ಲಿಟ್ಟು
ಸೆರೆ-ಯಲ್ಲಿ-ಡಿಸಿ-ದ-ನೆಂದು
ಸೆರೆ-ಯಲ್ಲಿದ್ದ
ಸೆರೆ-ಯಾಗಿ-ರ-ಬೇಕೆಂದು
ಸೆರೆ-ಹಿಡಿದ
ಸೆರೆ-ಹಿಡಿದು
ಸೆರೆ-ಹಿ-ಡಿಯು-ವಿಕೆಗೂ
ಸೆಲ-ವರ
ಸೆಲ-ವಾಗಿ
ಸೆಲೆ
ಸೆಲೆ-ವಾಗಿ
ಸೆಲ್ಲಬೆ-ಯರ
ಸೆಳೆ-ದಿರುವ
ಸೆಳೆದು
ಸೆಳೆದು-ಕೊಂಡ-ನೆಂದು
ಸೆಳೆದೊಯ್ಯುತ್ತಿದ್ದರು
ಸೆಳೆಯುತ್ತದೆ
ಸೆಳೆಯುತ್ತಿದ್ದಾನೆ
ಸೆಳೆ-ಯುವ
ಸೆಳ್ಳೆ-ಕೆರೆ
ಸೆವಲ-ರ-ವರ
ಸೇಂದ್ರಕ
ಸೇಂದ್ರಕ-ವಂಶದ
ಸೇಉಣರ
ಸೇಉಣ-ಸೈನ್ಯ
ಸೇಡು
ಸೇಣಿ
ಸೇತು-ಪತಿ
ಸೇತು-ಬಂಧ
ಸೇತು-ಬಂಧ-ಗಳು
ಸೇತು-ಬಂಧ-ಮಾಡಿ
ಸೇತು-ಬಂಧ-ವನ್ನು
ಸೇತು-ಬಂಧ-ವೆಂದರೆ
ಸೇತು-ವರಂ
ಸೇತು-ವಿನ
ಸೇತುವೆ
ಸೇತುವೆ-ಗ-ಳನ್ನು
ಸೇತುವೆಯ
ಸೇತುವೆ-ಯನ್ನು
ಸೇತುವೆ-ಯೊಂದನ್ನು
ಸೇನ
ಸೇನ-ಬೊವ
ಸೇನ-ಬೋಗ
ಸೇನ-ಬೋವ
ಸೇನ-ಬೋವ-ಅನಿದ್ದನು
ಸೇನ-ಬೋವನ
ಸೇನ-ಬೋವ-ನನ್ನು
ಸೇನ-ಬೋವ-ನನ್ನೇ
ಸೇನ-ಬೋವ-ನ-ಹಳ್ಳಿ
ಸೇನ-ಬೋವ-ನ-ಹಳ್ಳಿಯ
ಸೇನ-ಬೋವ-ನಾ-ಗಿದ್ದು
ಸೇನ-ಬೋವ-ನಾಗಿ-ರ-ಬ-ಹುದು
ಸೇನ-ಬೋವ-ನಿರ-ಬ-ಹುದು
ಸೇನ-ಬೋವ-ನಿರುತ್ತಿದ್ದನು
ಸೇನ-ಬೋವನು
ಸೇನ-ಬೋವನೂ
ಸೇನ-ಬೋವ-ನೊಳಗಾದ
ಸೇನ-ಬೋವರ
ಸೇನ-ಬೋವ-ರಂತಹ
ಸೇನ-ಬೋವ-ರನ್ನು
ಸೇನ-ಬೋವ-ರ-ವರೆ-ಗಿನ
ಸೇನ-ಬೋವ-ರಾ-ಗಲೀ
ಸೇನ-ಬೋವ-ರಿದ್ದ-ರೆಂದು
ಸೇನ-ಬೋವರು
ಸೇನ-ಬೋವ-ರೆಂದು
ಸೇನ-ಬೋವರೇ
ಸೇನ-ಬೋವಿ-ಕೆಯ
ಸೇನ-ಭೋಗ
ಸೇನ-ಭೋವ
ಸೇನಯ
ಸೇನ-ವಾರ
ಸೇನಾ
ಸೇನಾ-ಠಾಣ್ಯ-ವಿದ್ದ
ಸೇನಾ-ತುಕಡಿ-ಯನ್ನು
ಸೇನಾ-ತುಕಡಿ-ರೆಜಿಮೆಂಟ್
ಸೇನಾ-ದ-ಳದ
ಸೇನಾ-ಧಿ-ಕಾರಿ-ಗಳ
ಸೇನಾ-ಧಿ-ಕಾರಿ-ಗಳೂ
ಸೇನಾ-ಧಿ-ಕಾರಿ-ಯಾಗಿದ್ದ-ನೆಂದು
ಸೇನಾ-ಧಿ-ಕಾರಿ-ಯೆಂದೇ
ಸೇನಾ-ಧಿ-ಪತಿ
ಸೇನಾ-ಧಿ-ಪತಿ-ಆಗಿದ್ದನು
ಸೇನಾ-ಧಿ-ಪತಿ-ಗಳ
ಸೇನಾ-ಧಿ-ಪತಿ-ಗ-ಳನ್ನು
ಸೇನಾ-ಧಿ-ಪತಿ-ಗ-ಳಲ್ಲಿ
ಸೇನಾ-ಧಿ-ಪತಿ-ಗಳಿ-ರುತ್ತಿದ್ದ-ರೆಂದು
ಸೇನಾ-ಧಿ-ಪತಿ-ಗಳು
ಸೇನಾ-ಧಿ-ಪತಿಯ
ಸೇನಾ-ಧಿ-ಪತಿ-ಯಾ-ಗಿದ್ದ
ಸೇನಾ-ಧಿ-ಪತಿ-ಯಾ-ಗಿದ್ದನು
ಸೇನಾ-ಧಿ-ಪತಿ-ಯಾಗಿಯೂ
ಸೇನಾ-ಧಿ-ಪತಿಯೂ
ಸೇನಾ-ಧಿ-ಪತಿ-ಯೆಂದು
ಸೇನಾ-ಧಿ-ಪತಿಯೋ
ಸೇನಾ-ನ-ನಾಯ-ಕರು
ಸೇನಾ-ನಾಥ
ಸೇನಾ-ನಾ-ಯಕ
ಸೇನಾ-ನಾ-ಯಕ-ತನ
ಸೇನಾ-ನಾ-ಯಕ-ನಪ್ಪ
ಸೇನಾ-ನಾಯ-ಕ-ನಾಗಿ
ಸೇನಾ-ನಾಯ-ಕ-ನಾ-ಗಿದ್ದ
ಸೇನಾ-ನಾಯ-ಕ-ನಾಗಿದ್ದ-ನೆಂದು
ಸೇನಾ-ನಾಯ-ಕ-ನಾ-ಗಿದ್ದು
ಸೇನಾ-ನಾ-ಯಕ-ನಾದ
ಸೇನಾ-ನಾ-ಯಕ-ನೆಂದು
ಸೇನಾ-ನಾ-ಯಕ-ನೆನಿಸಿ-ದ-ಅನು
ಸೇನಾ-ನಾ-ಯಕರ
ಸೇನಾ-ನಾ-ಯಕ-ರಪ್ಪ
ಸೇನಾ-ನಾಯ-ಕ-ರಾ-ಗಿದ್ದರು
ಸೇನಾ-ನಾ-ಯಕ-ರಾದ-ರೆಂದು
ಸೇನಾ-ನಾ-ಯ-ಕರು
ಸೇನಾ-ನಾ-ಯ-ಕರುಂ
ಸೇನಾ-ನಾ-ಯಕ-ರು-ಗ-ಳಾಗಿದ್ದರು
ಸೇನಾ-ನಾ-ಯಕ-ರು-ಬಲ-ಗೈಯ
ಸೇನಾ-ನಾ-ಯಕ-ರೆಂದೇ
ಸೇನಾನಿ
ಸೇನಾ-ನಿ-ಗ-ಳಲ್ಲಿ
ಸೇನಾ-ನಿ-ಗ-ಳಾಗಿದ್ದ
ಸೇನಾ-ನಿ-ಯಾ-ಗಿದ್ದ
ಸೇನಾ-ನೆಲೆ
ಸೇನಾ-ನೆಲೆ-ಯನ್ನು
ಸೇನಾ-ಪಡೆ
ಸೇನಾ-ಪಡೆಗೆ
ಸೇನಾ-ಪಡೆಯ
ಸೇನಾ-ಪಡೆ-ಯನ್ನು
ಸೇನಾ-ಪಡೆ-ಯಾ-ಗಿದ್ದಿರ
ಸೇನಾ-ಪಡೆಯು
ಸೇನಾ-ಪಡೆ-ಯೆಂದೂ
ಸೇನಾ-ಪತಿ
ಸೇನಾ-ಪತಿ-ಗಳಾ-ಗಲೀ
ಸೇನಾ-ಪತಿ-ಗ-ಳಿಗೆ
ಸೇನಾ-ಪತಿ-ಗಳು-ಸೇನಾ-ಧಿ-ಪತಿ-ಗಳು-ಚಮೂಪರು
ಸೇನಾ-ಪತಿ-ಯಾಗಿ-ರ-ಬ-ಹುದು
ಸೇನಾ-ಪ-ತಿಯು
ಸೇನಾ-ಬಲ
ಸೇನಾ-ಬಲಕ್ಕೆ
ಸೇನಾ-ಬ-ಲದ
ಸೇನಾ-ವೀರ-ರಾಗಿದ್ದರು
ಸೇನಾ-ವೀ-ರರು
ಸೇನು-ಬೋವ
ಸೇನು-ಬೋವನು
ಸೇನು-ಬೋವ-ರಿದ್ದ-ರೆಂಬುದು
ಸೇನು-ಬೋವರೂ
ಸೇನು-ಬೋವಿ-ಕೆಗೆ
ಸೇನು-ಬೋವಿ-ಕೆಯ
ಸೇನೆ
ಸೇನೆ-ಗ-ಳಿಗೆ
ಸೇನೆಗೆ
ಸೇನೆಯ
ಸೇನೆ-ಯನ್ನು
ಸೇನೆ-ಯ-ಬೀಡಿನ
ಸೇನೆ-ಯಲ್ಲಿ
ಸೇನೆ-ಯಲ್ಲಿದ್ದ
ಸೇನೆ-ಯಲ್ಲಿದ್ದ-ರೆಂದು
ಸೇನೆ-ಯ-ವರು
ಸೇನೆಯಿಂ
ಸೇನೆ-ಯಿಂದ
ಸೇನೆಯು
ಸೇನೆ-ಯೊಂದಿಗೆ
ಸೇನೆ-ಯೊಡ-ಗೂಡಿ
ಸೇನೆ-ಯೊಡನೆ
ಸೇರನ-ಹಳ್ಳಿ
ಸೇರ-ಬೇಕಾಗಿದ್ದ
ಸೇರ-ಬೇ-ಕಾದ
ಸೇರಿ
ಸೇರಿ-ಕೊಂಡ
ಸೇರಿ-ಕೊಂಡಂತಿದೆ
ಸೇರಿ-ಕೊಂಡವು
ಸೇರಿ-ಕೊಂಡು
ಸೇರಿ-ತೆಂದು
ಸೇರಿತ್ತು
ಸೇರಿತ್ತೆಂದು
ಸೇರಿತ್ತೇ
ಸೇರಿದ
ಸೇರಿ-ದಂತೆ
ಸೇರಿ-ದಂತೆಯೂ
ಸೇರಿ-ದ-ಎಂಟ-ನೆಯ
ಸೇರಿ-ದರು
ಸೇರಿ-ದ-ರೆಂದು
ಸೇರಿ-ದ-ಳೆಂದು
ಸೇರಿ-ದವ
ಸೇರಿ-ದ-ವ-ನಲ್ಲ-ವೆಂದೂ
ಸೇರಿ-ದ-ವ-ನಾ-ಗಿದ್ದು
ಸೇರಿ-ದ-ವ-ನಾಗಿ-ರ-ಬ-ಹುದು
ಸೇರಿ-ದ-ವ-ನಿರ-ಬ-ಹುದು
ಸೇರಿ-ದ-ವನು
ಸೇರಿ-ದ-ವ-ನೆಂದು
ಸೇರಿ-ದ-ವ-ನೆಂದೂ
ಸೇರಿ-ದ-ವ-ರಾಗಿದ್ದರು
ಸೇರಿ-ದ-ವ-ರಾಗಿದ್ದಾರೆ
ಸೇರಿ-ದ-ವ-ರಾ-ಗಿದ್ದು
ಸೇರಿ-ದ-ವ-ರಾಗಿ-ರ-ಬ-ಹುದು
ಸೇರಿ-ದ-ವ-ರಾಗಿ-ರ-ಬಹು-ದೆಂದೂ
ಸೇರಿ-ದ-ವ-ರಾಗಿ-ರುತ್ತಾರೆ
ಸೇರಿ-ದ-ವರು
ಸೇರಿ-ದ-ವ-ರು-ಇಂದಿನ
ಸೇರಿ-ದ-ವ-ರೆಂದು
ಸೇರಿ-ದ-ವ-ರೆಂದೂ
ಸೇರಿ-ದ-ವ-ರೆಂಬ
ಸೇರಿ-ದ-ವ-ಳೆಂದು
ಸೇರಿ-ದ-ವಾ-ಗಿದ್ದು
ಸೇರಿ-ದಾಗ
ಸೇರಿದೆ
ಸೇರಿದ್ದ
ಸೇರಿದ್ದನು
ಸೇರಿದ್ದ-ರಿಂದ
ಸೇರಿದ್ದ-ರೆಂದು
ಸೇರಿದ್ದ-ರೆಂಬ
ಸೇರಿದ್ದವು
ಸೇರಿದ್ದ-ವೆಂದು
ಸೇರಿದ್ದ-ವೆಂದೂ
ಸೇರಿದ್ದಾನೆ
ಸೇರಿದ್ದಾರೆ
ಸೇರಿದ್ದಾ-ರೆಂದು
ಸೇರಿದ್ದು
ಸೇರಿದ್ದೆಂದು
ಸೇರಿ-ರ-ಬಹು-ದಾದ
ಸೇರಿ-ರ-ಬ-ಹುದು
ಸೇರಿ-ರ-ಬಹುದೆ
ಸೇರಿ-ರ-ಬಹು-ದೆಂದು
ಸೇರಿ-ರುತ್ತವೆ
ಸೇರಿ-ರುವ
ಸೇರಿ-ರುವು-ದ-ರಿಂದ
ಸೇರಿಲ್ಲ
ಸೇರಿಲ್ಲ-ವೆಂದಾ-ಯಿತು
ಸೇರಿವೆ
ಸೇರಿ-ಸ-ಲಾ-ಗಿತ್ತು
ಸೇರಿ-ಸ-ಲಾಗಿದೆ
ಸೇರಿ-ಸ-ಲಾ-ಯಿತು
ಸೇರಿಸಿ
ಸೇರಿ-ಸಿ-ಕೊಂಡ-ನೆಂದು
ಸೇರಿ-ಸಿ-ಕೊಂಡರು
ಸೇರಿ-ಸಿ-ಕೊಂಡಿದ್ದಾರೆ
ಸೇರಿ-ಸಿ-ಕೊಳ್ಳುತ್ತಿ-ದುದು
ಸೇರಿ-ಸಿ-ಕೊಳ್ಳುವ
ಸೇರಿ-ಸಿದ
ಸೇರಿ-ಸಿ-ದನು
ಸೇರಿ-ಸಿ-ದ-ನೆಂದು
ಸೇರಿ-ಸಿ-ದ-ರೆಂದು
ಸೇರಿ-ಸಿದೆ
ಸೇರಿ-ಸಿದ್ದಾರೆ
ಸೇರಿ-ಸಿ-ರ-ಬ-ಹುದು
ಸೇರಿ-ಸಿ-ರ-ಬಹು-ದೆಂದು
ಸೇರಿ-ಸಿ-ರುವ
ಸೇರಿ-ಸುತ್ತಾರೆ
ಸೇರಿ-ಸು-ವಂತೆ
ಸೇರಿ-ಸುವಾಗ
ಸೇರಿ-ಸು-ವುದು
ಸೇರಿ-ಹೋಗಿದೆ
ಸೇರಿ-ಹೋಗಿ-ರುವ
ಸೇರು
ಸೇರು-ತಿದ್ದ
ಸೇರುತ್ತದೆ
ಸೇರುತ್ತ-ದೆಂದು
ಸೇರುತ್ತವೆ
ಸೇರುತ್ತಾರೆ
ಸೇರುತ್ತಿತ್ತು
ಸೇರುತ್ತಿತ್ತೆಂದು
ಸೇರುತ್ತಿದ್ದ
ಸೇರುತ್ತಿದ್ದವು
ಸೇರುತ್ತಿದ್ದು-ದನ್ನು
ಸೇರುವ
ಸೇರು-ವು-ದಿಲ್ಲ
ಸೇರ್ಪಡೆ
ಸೇರ್ಪಡೆ-ಯಾ-ಗಿತ್ತು
ಸೇರ್ಪಡೆ-ಯಾಗಿದ್ದ-ವೆಂದು
ಸೇವಂತ-ನ-ಹಳ್ಳಿ
ಸೇವಕ
ಸೇವಕ-ನಾ-ಗಿದ್ದ
ಸೇವಕ-ರಾಗಿ
ಸೇವಕ-ರಿಗೆ
ಸೇವ-ಕರೂ
ಸೇವಾ
ಸೇವಾ-ಕಾರ್ಯ-ದಲ್ಲಿ
ಸೇವಾರ್ಥ
ಸೇವಾರ್ಥ-ದಾ-ರರು
ಸೇವಾರ್ಥ-ವಾಗಿ
ಸೇವಿತ-ನಾಗಿದ್ದ-ನೆಂದು
ಸೇವಿಸಲ್ಪಡುತ್ತಿದ್ದಾ-ನೆಂದು
ಸೇವುಣ
ಸೇವುಣ-ದ-ಳದ
ಸೇವು-ಣನ
ಸೇವುಣರ
ಸೇವುಣ-ರನ್ನು
ಸೇವುಣ-ರಾಯ-ದರ್ಪದ-ಳನ
ಸೇವುಣ-ರಿಗೂ
ಸೇವುಣ-ರಿಗೆ
ಸೇವುಣರು
ಸೇವುಣ-ರೊಡನೆ
ಸೇವುಣ-ಸೇನೆ-ಯನ್ನು
ಸೇವುಣ-ಸೈನ್ಯ
ಸೇವುಣ-ಸೈನ್ಯ-ವನ್ನು
ಸೇವುಣಾಧಿಪ
ಸೇವುಳ-ರಾಯ
ಸೇವೆ
ಸೇವೆ-ಎಂದು
ಸೇವೆ-ಗಳ
ಸೇವೆ-ಗ-ಳನ್ನು
ಸೇವೆ-ಗಳು
ಸೇವೆ-ಗಾಗಿ
ಸೇವೆಗೆ
ಸೇವೆಯ
ಸೇವೆ-ಯನು
ಸೇವೆ-ಯನ್ನು
ಸೇವೆ-ಯಲ್ಲಿ
ಸೇವೆ-ಯಿಂದ
ಸೇವೆ-ಸಲ್ಲಿಸಿ
ಸೇವೆ-ಸಲ್ಲಿ-ಸುತ್ತಿದ್ದರೂ
ಸೇವ್ಯ-ನೆಂದು
ಸೇಸೆ
ಸೇಸೆಗೆ
ಸೇಸೆ-ಯ-ನಿಕ್ಕಿ
ಸೇಸೆ-ಯನಿಕ್ಕು-ವಂತೆ
ಸೇಸೆ-ಯನು
ಸೇಸೆ-ಯೆನ್ದು
ಸೈಗೊಟ್ಟ
ಸೈಗೋಲ-ಮಾತ್ಯ
ಸೈದ್ಧಾಂತಿಕ
ಸೈದ್ಧಾಂತಿ-ಕರ
ಸೈದ್ಧಾಂತಿ-ಕರು
ಸೈನಿ-ಕರ
ಸೈನಿ-ಕರಂ
ಸೈನಿಕ-ರನ್ನು
ಸೈನಿಕ-ರಲ್ಲಿ
ಸೈನಿಕ-ರಿಂದ
ಸೈನಿಕ-ರಿಗೆ
ಸೈನಿ-ಕರು
ಸೈನಿ-ಕರು-ಗಳು
ಸೈನ್ಯ
ಸೈನ್ಯಕ್ಕೆ
ಸೈನ್ಯದ
ಸೈನ್ಯ-ದೊಡನೆ
ಸೈನ್ಯ-ಪಡೆ
ಸೈನ್ಯ-ವನ್ನು
ಸೈನ್ಯ-ಸಮೇತ-ನಾಗಿ
ಸೈನ್ಯಾಧಿ-ಕಾರಿ-ಗ-ಳಾಗಿದ್ದರು
ಸೈನ್ಯಾನೀ-ಕಮಂ
ಸೈಯದ್
ಸೊಂಡೆ-ಕೊಪ್ಪ
ಸೊಂನಾಕೋತ್ಸತಿ
ಸೊಂನ್ನಾ-ದೇವಿ-ಯರ
ಸೊಗಯಿ-ಪನೆನೆಸುಂ
ಸೊಗಯಿ-ಸುವಂ
ಸೊಗಸಾಗಿದೆ
ಸೊಡರ-ಗದ್ಯಾಣ
ಸೊದರ-ಳಿಯಂದಿ-ರಾದ
ಸೊನ್ನ-ಲಿಗೆ
ಸೊನ್ನಾದಾಯ
ಸೊನ್ನಾ-ದೇವಿ
ಸೊನ್ನಾ-ದೇವಿಯ
ಸೊಪ್ಪು
ಸೊಪ್ಪೆಯ
ಸೊಪ್ಪೆ-ಯಾರ್ಯ
ಸೊಮೆಯ-ದಂಡ-ನಾಯಕ
ಸೊಮೇಶ್ವರ-ದೇವ-ನೊಡನೆ
ಸೊರಟೂರು
ಸೊರ-ದಂತೆ
ಸೊರಬ
ಸೊಲಗೆ
ಸೊಲಿ-ಸಿ-ದನು
ಸೊಲ್ಲಗೆ
ಸೊಲ್ಲಗೆಯ
ಸೊಲ್ಲಗೆಯೇ
ಸೊಲ್ಲಗೆ-ಸ-ಲಿಗೆ
ಸೊಸಿ-ಯಪ್ಪ
ಸೊಸಿ-ಯಪ್ಪನ
ಸೊಸಿ-ಯಪ್ಪ-ನಾಯಕ
ಸೊಸಿ-ಯಪ್ಪ-ನಾಯ-ಕ-ರಿಗೆ
ಸೊಸಿ-ಯಪ್ಪನು
ಸೊಸೆ
ಸೊಸೆ-ವೂ-ರನ್ನು
ಸೊಸೆ-ವೂರಿ-ನಿಂದ
ಸೋತ
ಸೋತ-ನೆಂದು
ಸೋತ-ವರ
ಸೋತು
ಸೋದರ
ಸೋದರನ
ಸೋದರ-ನಾಗಿ-ರ-ಬ-ಹುದು
ಸೋದರ-ನಾದ
ಸೋದರ-ಮಾವ
ಸೋದರರ
ಸೋದರರು
ಸೋದರ-ರೆಂದು
ಸೋದರ-ಳಿಯ
ಸೋದರ-ಳಿಯಂದಿ-ರಾದ
ಸೋದರ-ಳಿಯಂದಿರು
ಸೋದರ-ಳಿಯ-ನಾದ
ಸೋದರಿ
ಸೋದರಿಯ
ಸೋದರಿ-ಯರು
ಸೋದೆ
ಸೋದೆ-ವೆ-ಸನ
ಸೋದೆ-ಸುಣ್ಣ
ಸೋಧೆ-ವೆ-ಸನಂ
ಸೋನಾರ್
ಸೋನೆ-ಬೋವ
ಸೋಪಸ್ಕರ-ಗ-ಳನ್ನು
ಸೋಪಾ-ದಲ್ಲಿ
ಸೋಪಾ-ನದ
ಸೋಪಾನ-ವನ್ನು
ಸೋಪಾನ-ವಿದ್ದ
ಸೋಭಿಸೆ
ಸೋಮ
ಸೋಮಂ
ಸೋಮಂಗ-ಳಿಯಂ
ಸೋಮಕ
ಸೋಮ-ಕುಳಪ್ರ-ಸಿದ್ಧಿ
ಸೋಮ-ಚಂದ್ರರ
ಸೋಮಣ್ಣ
ಸೋಮಣ್ಣ-ದಂಡ-ನಾಯ-ಕರು
ಸೋಮ-ದಂಡ-ನಾಥನ
ಸೋಮ-ದಂಡ-ನಾಯ-ಕನ
ಸೋಮ-ದಂಡ-ನಾಯ-ಕ-ನನ್ನು
ಸೋಮ-ದಂಡ-ನಾಯ-ಕ-ನಿಗೆ
ಸೋಮ-ದಂಡ-ನಾಯ-ಕನು
ಸೋಮ-ದಂಡಾಧಿ-ಪನ
ಸೋಮ-ದೇವನ
ಸೋಮನ
ಸೋಮ-ನನ್ನು
ಸೋಮ-ನ-ಳಿಯಂ
ಸೋಮ-ನ-ವರೆಗೆ
ಸೋಮ-ನ-ಹಳ್ಳಿ
ಸೋಮ-ನ-ಹಳ್ಳಿಯ
ಸೋಮ-ನ-ಹಳ್ಳಿ-ಯನ್ನು
ಸೋಮ-ನ-ಹಳ್ಳಿಯು
ಸೋಮ-ನಾಥ
ಸೋಮ-ನಾಥ-ದೇವರ
ಸೋಮ-ನಾಥ-ದೇವರು
ಸೋಮ-ನಾಥ-ನನ್ನು
ಸೋಮ-ನಾಥ-ನೆಂಬು-ವ-ವನು
ಸೋಮ-ನಾಥ-ಪುರ
ಸೋಮ-ನಾಥ-ಪುರದ
ಸೋಮ-ನಾಥ-ಪುರ-ದಲ್ಲಿ
ಸೋಮ-ನಾಥ-ಪುರ-ದಲ್ಲಿರು
ಸೋಮ-ನಾಥ-ಪುರ-ವನ್ನಾಗಿ
ಸೋಮ-ನಾಥ-ಪುರ-ವಾದ
ಸೋಮ-ನಾಥ-ಪುರ-ವೆಂಬ
ಸೋಮ-ನಿನಿ-ಸಿ-ದನು
ಸೋಮನು
ಸೋಮ-ನೃಪನ
ಸೋಮ-ನೊಡನೆ
ಸೋಮಯ
ಸೋಮ-ಯ-ನಾಯ-ಕ-ನೊಳಗಾದ
ಸೋಮಯೆ
ಸೋಮಯ್ಯ
ಸೋಮಯ್ಯಂಗಳು
ಸೋಮಯ್ಯ-ಗಳು
ಸೋಮಯ್ಯ-ದೇವನು
ಸೋಮಯ್ಯ-ದೇವರ
ಸೋಮಯ್ಯ-ದೇವ-ರಿಗೆ
ಸೋಮಯ್ಯನ
ಸೋಮಯ್ಯನು
ಸೋಮಯ್ಯ-ನೆಂಬ
ಸೋಮಯ್ಯ-ನೆಂಬು-ವ-ವನು
ಸೋಮ-ರಾಸಿ-ಜೀಯನು
ಸೋಮ-ಲಾಖ್ಯಾ
ಸೋಮ-ಲಿಂಗೇಶ್ವರ
ಸೋಮ-ವಂಶಾಧೀಶ್ವರ-ನೆಂದು
ಸೋಮ-ವರ್ಮನು
ಸೋಮ-ವರ್ಮ-ನೆಂಬ
ಸೋಮ-ವಾರ-ದಲು
ಸೋಮವ್ವ
ಸೋಮವ್ವೆ
ಸೋಮವ್ವೆ-ದೇ-ವಿಗೆ
ಸೋಮವ್ವೆ-ದೇವಿಯ
ಸೋಮ-ಶೆಟ್ಟಿ
ಸೋಮ-ಶೆಟ್ಟಿಯ
ಸೋಮ-ಶೇಖರ
ಸೋಮ-ಸೆಟ್ಟಿ
ಸೋಮಿ-ಗೌಡನ
ಸೋಮಿ-ದೇವ
ಸೋಮಿ-ಯಕ್ಕನ-ಣು-ಗಿನ
ಸೋಮಿ-ಯಕ್ಕ-ನನ್ನು
ಸೋಮಿ-ಯಕ್ಕ-ಸೋವಿಯಕ್ಕನ
ಸೋಮಿ-ಸೆಟ್ಟಿ
ಸೋಮಿ-ಸೆಟ್ಟಿಯು
ಸೋಮೆಯ
ಸೋಮೆಯಜ
ಸೋಮೆಯ-ದಂಡ-ನಾಯ-ಕನ
ಸೋಮೆಯ-ದಂಡ-ನಾಯ-ಕನು
ಸೋಮೆಯ-ದಂಡ-ನಾಯ-ಕ-ಮಲ್ಲಿ-ದೇವ
ಸೋಮೆಯ-ದಂಡ-ನಾಯ-ಕರ
ಸೋಮೆಯನ
ಸೋಮೆಯ-ನಾಯಕ
ಸೋಮೆಯ-ನಾಯ-ಕನ
ಸೋಮೆಯ-ನಾಯ-ಕನು
ಸೋಮೆಯ-ನಾ-ಯ-ಕರು
ಸೋಮೆಶ್ವರ
ಸೋಮೇಶ್ವರ
ಸೋಮೇಶ್ವರಕ್ಕೆ
ಸೋಮೇಶ್ವರ-ದೇವ-ರ-ಸರು
ಸೋಮೇಶ್ವರ-ದೇವ-ರಿಗೆ
ಸೋಮೇಶ್ವರ-ದೇವಾ-ಲಯ
ಸೋಮೇಶ್ವರನ
ಸೋಮೇಶ್ವರ-ನನ್ನು
ಸೋಮೇಶ್ವರ-ನಿಗೆ
ಸೋಮೇಶ್ವರನು
ಸೋಮೇಶ್ವರನೇ
ಸೋಮೇಷಂ
ಸೋಯಂ
ಸೋಯಕ್ಕ
ಸೋಯಕ್ಕಂಗೆ
ಸೋಯಕ್ಕ-ನಿಗೂ
ಸೋಯಿ-ಸೆಟ್ಟಿ
ಸೋಯೆಂದು
ಸೋರ-ದಂತೆ
ಸೋಲನ್ನು
ಸೋಲಾ-ಯಿತು
ಸೋಲಿ-ನಿಂದ
ಸೋಲಿಸಲಾ-ಗದ
ಸೋಲಿಸ-ಲಾಗಿ
ಸೋಲಿ-ಸಲು
ಸೋಲಿಸಲ್ಪಟ್ಟ
ಸೋಲಿಸಿ
ಸೋಲಿಸಿ-ಕೊಂದು
ಸೋಲಿ-ಸಿದ
ಸೋಲಿಸಿ-ದಂತೆ
ಸೋಲಿಸಿ-ದನು
ಸೋಲಿಸಿ-ದ-ನೆಂದು
ಸೋಲಿಸಿ-ದರು
ಸೋಲಿಸಿ-ದಾಗ
ಸೋಲಿಸಿದ್ದಕ್ಕಾಗಿಯೂ
ಸೋಲಿಸಿದ್ದ-ನೆಂದು
ಸೋಲಿಸಿದ್ದಲ್ಲದೆ
ಸೋಲಿಸಿ-ರ-ಬ-ಹುದು
ಸೋಲುಂಟಾಯಿ-ತೆಂದು
ಸೋಲೂರು
ಸೋವಂಣ
ಸೋವಂಣನೂ
ಸೋವ-ಗೌಡ-ನೊಳಗಾದ
ಸೋವಣ
ಸೋವಣ್ಣ
ಸೋವಣ್ಣನೂ
ಸೋವ-ನಾಥ-ಪಂಡಿ-ತ-ನಿದ್ದನು
ಸೋವ-ನಾಥ-ಪಂಡಿ-ತ-ರಿಗೆ
ಸೋವ-ರಸ
ಸೋವ-ರಾಸಿ
ಸೋವಲಾ-ದೇವಿಯ
ಸೋವಿ-ದೇವ
ಸೋವಿ-ದೇವ-ಘ-ಟೆಯಂ
ಸೋವಿ-ದೇವ-ನಿಗೂ
ಸೋವಿ-ದೇವ-ನೊಡನೆ
ಸೋವಿಯ
ಸೋವಿ-ಸೆಟ್ಟಿ
ಸೋವಿ-ಸೆಟ್ಟಿಗೆ
ಸೋವಿ-ಸೆಟ್ಟಿಯ
ಸೋವಿ-ಸೆಟ್ಟಿ-ಯನ್ನೂ
ಸೋವಿ-ಸೆಟ್ಟಿಯು
ಸೋವೆಯ
ಸೋವೆಯ-ದಂಡ-ನಾಯ-ಕ-ಸೋಮೆಯ
ಸೋವೆಯ-ನಾಯಕ
ಸೋವೆಯ-ನಾಯಕಂ
ಸೋವೆಯ-ನಾಯ-ಕನ
ಸೋಸಲಿ
ಸೋಸಲಿ-ಯ-ಇಂದಿನ
ಸೋಸಲೆ
ಸೋಸಲೆಯ
ಸೋಸಲೆ-ಯಲ್ಲಿ
ಸೋಸ-ಲೆಯೇ
ಸೌಂದತ್ತಿಯ
ಸೌಖ್ಯನೆಸೆವಂ
ಸೌಖ್ಯಾಯ
ಸೌಚ
ಸೌಜನ-ಬಾಂಧವ
ಸೌಭಾಗ್ಯ
ಸೌಮ್ಯ
ಸೌಮ್ಯ-ಕೇಶವ
ಸೌಮ್ಯ-ಕೇಶವ-ದೇವಾ-ಲ-ಯದ
ಸೌಮ್ಯ-ಕೇಶವ-ದೇವಾ-ಲಯ-ವನ್ನು
ಸೌಮ್ಯ-ಕೇಶವ-ದೇವಾ-ಲ-ಯವು
ಸೌಮ್ಯ-ಕೇಶವ-ಮೂರ್ತಿ
ಸೌಮ್ಯ-ಕೇಶವಸ್ವಾಮಿಯ
ಸೌಮ್ಯ-ಜಾಮಾತೃ
ಸೌಮ್ಯ-ರಾಜ
ಸೌಮ್ಯಸ್ವ-ರೂಪದ
ಸೌರಾಷ್ಟ್ರ-ಗ-ಳನ್ನು
ಸೌರಾಷ್ಟ್ರ-ದಿಂದ
ಸೌಲಭ್ಯ
ಸೌಲಭ್ಯಕ್ಕೆ
ಸೌಲಭ್ಯ-ಗ-ಳನ್ನು
ಸೌವಿ-ದಲ್ಲ-ಪದಂ
ಸೌಹಾರ್ದ-ವನ್ನು
ಸ್ಕಂಧಕಂ
ಸ್ಟೇಷನ್
ಸ್ತಂಭ-ಗಳು
ಸ್ತತ್ಸೂನುರ್ಜಿತ
ಸ್ತನ-ಹಾರ
ಸ್ತಳ
ಸ್ತಳದ
ಸ್ತಾನ-ಪತಿ
ಸ್ತಾನಿಕ
ಸ್ತಿತಿ-ಸಾರಂ
ಸ್ತೀರೋ-ಮಣಿ
ಸ್ತುತಿ
ಸ್ತುತಿ-ಗಳು
ಸ್ತುತಿಗೆ
ಸ್ತುತಿ-ತಸ-ಲಾಗಿದೆ
ಸ್ತುತಿಯ
ಸ್ತುತಿ-ಯ-ನಂತರ
ಸ್ತುತಿ-ಯನ್ನು
ಸ್ತುತಿ-ಯಿಂದ
ಸ್ತುತಿ-ಯಿಂದಲೂ
ಸ್ತುತಿ-ಯಿಂದಲೇ
ಸ್ತುತಿ-ಯೊಂದಿಗೆ
ಸ್ತುತಿ-ಸ-ಲಾಗಿದೆ
ಸ್ತುತಿ-ಸಿದು
ಸ್ತುತಿ-ಸಿದೆ
ಸ್ತುತಿ-ಸಿದ್ದಾರೆ
ಸ್ತುತಿ-ಸಿದ್ದು
ಸ್ತುತಿ-ಸಿರುವ
ಸ್ತುತಿ-ಸುತ್ತದೆ
ಸ್ತುತ್ಯ-ನಾಗಿದ್ದಾನೆ
ಸ್ತುಳಕಪಿಳ
ಸ್ತೋಮೋಧ್ಯಾಮಾಭಿ-ರಾಮಂ
ಸ್ತ್ರೀ
ಸ್ತ್ರೀಯರ
ಸ್ತ್ರೀಯರನ್ನಾ-ಗಲೀ
ಸ್ತ್ರೀಯ-ರನ್ನು
ಸ್ತ್ರೀಯ-ರಿಗೆ
ಸ್ತ್ರೀಯರು
ಸ್ತ್ರೀಯರೂ
ಸ್ತ್ರೀಯರೇ
ಸ್ತ್ರೀಸಮಾಜ
ಸ್ತ್ರೀಸಮಾಜದ
ಸ್ಥಂಬದ
ಸ್ಥಂಬನು
ಸ್ಥಂಭ
ಸ್ಥಂಭ-ಗಳ
ಸ್ಥಂಭದ
ಸ್ಥಂಭ-ನನ್ನು-ರಣಾವ-ಲೋಕ
ಸ್ಥಂಭನು
ಸ್ಥಂಭ-ವನ್ನು
ಸ್ಥಂಭ-ವಿದ್ದು
ಸ್ಥಂಭ-ಶಾ-ಸನದ
ಸ್ಥಂಭೀತ-ನನ್ನಾಗಿ
ಸ್ಥರ-ಗಳ
ಸ್ಥರ-ಗ-ಳಲ್ಲಿ
ಸ್ಥರ-ದಲ್ಲಿದ್ದು-ದ-ರಿಂದ
ಸ್ಥಳ
ಸ್ಥಳಂಗಳ
ಸ್ಥಳಕ್ಕೂ
ಸ್ಥಳಕ್ಕೆ
ಸ್ಥಳ-ಗಳ
ಸ್ಥಳ-ಗ-ಳನ್ನು
ಸ್ಥಳ-ಗ-ಳನ್ನೇ
ಸ್ಥಳ-ಗ-ಳಲ್ಲಿ
ಸ್ಥಳ-ಗಳಲ್ಲಿಯೂ
ಸ್ಥಳ-ಗಳಲ್ಲೇ
ಸ್ಥಳ-ಗ-ಳಿಂದ
ಸ್ಥಳ-ಗ-ಳಿಗೆ
ಸ್ಥಳ-ಗಳು
ಸ್ಥಳ-ಗ-ಳೆಂದು
ಸ್ಥಳ-ಗಳೆಂಬ
ಸ್ಥಳದ
ಸ್ಥಳ-ದಲು
ಸ್ಥಳ-ದಲೂ
ಸ್ಥಳ-ದಲ್ಲಿ
ಸ್ಥಳ-ದಲ್ಲಿದ್ದವು
ಸ್ಥಳ-ದಲ್ಲಿದ್ದು
ಸ್ಥಳ-ದಲ್ಲಿಯೇ
ಸ್ಥಳ-ದಲ್ಲಿ-ರುವ
ಸ್ಥಳ-ದಲ್ಲೇ
ಸ್ಥಳ-ದಲ್ಲೋ
ಸ್ಥಳ-ದಿಂದ
ಸ್ಥಳ-ದೊಳ-ಗಣ
ಸ್ಥಳ-ನಾಮ
ಸ್ಥಳ-ನಾಮ-ಗಳ
ಸ್ಥಳ-ನಾಮ-ಗ-ಳನ್ನು
ಸ್ಥಳ-ನಾಮ-ಗಳು
ಸ್ಥಳ-ನಾಮದ
ಸ್ಥಳ-ನಾಮ-ವನ್ನು
ಸ್ಥಳ-ನಾಮ-ವಾಗಿದೆ
ಸ್ಥಳ-ಪರಿ-ವೀಕ್ಷ-ಣೆಯ
ಸ್ಥಳ-ಪರಿಶೀಲನೆ-ಯಿಂದ
ಸ್ಥಳ-ಪುರಾಣ
ಸ್ಥಳ-ಪುರಾಣ-ಗಳ
ಸ್ಥಳ-ಪುರಾಣದ
ಸ್ಥಳ-ಪುರಾಣ-ದಿಂದ
ಸ್ಥಳ-ವನ್ನು
ಸ್ಥಳ-ವಾಗಿ
ಸ್ಥಳ-ವಾ-ಗಿತ್ತು
ಸ್ಥಳ-ವಾಗಿದೆ
ಸ್ಥಳ-ವಾ-ಗಿದ್ದ
ಸ್ಥಳ-ವಾ-ಗಿದ್ದು
ಸ್ಥಳ-ವಾಯಿತು
ಸ್ಥಳವು
ಸ್ಥಳ-ವು-ಇಂದಿನ
ಸ್ಥಳವೂ
ಸ್ಥಳ-ವೆಂದು
ಸ್ಥಳ-ವೆಂದೂ
ಸ್ಥಳ-ಸುಂಕ
ಸ್ಥಳಾಂತ-ರದ
ಸ್ಥಳಾಂತರ-ನಾದ-ನೆಂದು
ಸ್ಥಳಾಂತರ-ವಾ-ಗಿತ್ತು
ಸ್ಥಳಾಂತರಿಸ-ಲಾಗಿದೆ
ಸ್ಥಳಾಂತರಿ-ಸ-ಲಾ-ಯಿತು
ಸ್ಥಳಾಂತರಿಸಿ-ದ-ನೆಂದು
ಸ್ಥಳೀಯ
ಸ್ಥಳೀ-ಯರ
ಸ್ಥಳೀಯ-ರಾದ
ಸ್ಥಳೀ-ಯರು
ಸ್ಥಳೀ-ಯರೇ
ಸ್ಥಳೀಯ-ವಾಗಿ
ಸ್ಥಳೀ-ಯವೇ
ಸ್ಥಳೇ
ಸ್ಥಾನ
ಸ್ಥಾನಂ
ಸ್ಥಾನಕ್ಕೆ
ಸ್ಥಾನಕ್ಕೊಡೆ-ಯನು
ಸ್ಥಾನ-ಗಳ
ಸ್ಥಾನ-ಗ-ಳಿಗೆ
ಸ್ಥಾನ-ಗಳಿದ್ದುದು
ಸ್ಥಾನ-ಗಳು
ಸ್ಥಾನ-ಗಳೂ
ಸ್ಥಾನದ
ಸ್ಥಾನ-ದಲ್ಲಿ
ಸ್ಥಾನ-ದಲ್ಲಿದ್ದರು
ಸ್ಥಾನ-ದಲ್ಲಿದ್ದ-ರೆಂಬುದು
ಸ್ಥಾನ-ಪಡೆ-ದಿದ್ದರು
ಸ್ಥಾನ-ಪಡೆದು
ಸ್ಥಾನ-ಪಡೆ-ಯು-ವ-ವಳು
ಸ್ಥಾನ-ಪತಿ
ಸ್ಥಾನ-ಪತಿ-ಗಳ
ಸ್ಥಾನ-ಪತಿ-ಗ-ಳನ್ನು
ಸ್ಥಾನ-ಪತಿ-ಗಳಲ್ಲ
ಸ್ಥಾನ-ಪತಿ-ಗ-ಳಲ್ಲಿ
ಸ್ಥಾನ-ಪತಿ-ಗ-ಳಾಗಿ
ಸ್ಥಾನ-ಪತಿ-ಗ-ಳಾಗಿದ್ದ-ರೆಂದು
ಸ್ಥಾನ-ಪತಿ-ಗ-ಳಾಗಿದ್ದ-ರೆಂಬು-ದನ್ನು
ಸ್ಥಾನ-ಪತಿ-ಗ-ಳಾಗಿದ್ದು
ಸ್ಥಾನ-ಪತಿ-ಗ-ಳಾಗಿಯೂ
ಸ್ಥಾನ-ಪತಿ-ಗಳಿಗೂ
ಸ್ಥಾನ-ಪತಿ-ಗ-ಳಿಗೆ
ಸ್ಥಾನ-ಪತಿ-ಗಳಿದ್ದ-ರೆಂದು
ಸ್ಥಾನ-ಪತಿ-ಗಳು
ಸ್ಥಾನ-ಪ-ತಿಗೆ
ಸ್ಥಾನ-ಪತಿಯ
ಸ್ಥಾನ-ಪತಿ-ಯನ್ನಾಗಿ
ಸ್ಥಾನ-ಪತಿ-ಯಾಗಿ
ಸ್ಥಾನ-ಪತಿ-ಯಾಗಿ-ಗ-ಳಾಗಿದ್ದ-ರೆಂದು
ಸ್ಥಾನ-ಪತಿ-ಯಾ-ಗಿದ್ದ
ಸ್ಥಾನ-ಪತಿ-ಯಾಗಿದ್ದನು
ಸ್ಥಾನ-ಪತಿ-ಯಾಗಿದ್ದ-ನೆಂದು
ಸ್ಥಾನ-ಪತಿ-ಯಾಗಿದ್ದ-ನೆಂಬ
ಸ್ಥಾನ-ಪತಿ-ಯಾಗಿದ್ದ-ವರ
ಸ್ಥಾನ-ಪತಿ-ಯಾಗಿದ್ದಿರ-ಬ-ಹುದು
ಸ್ಥಾನ-ಪತಿ-ಯಾಗಿದ್ದಿರ-ಬಹು-ದೆಂದು
ಸ್ಥಾನ-ಪತಿ-ಯಾ-ಗಿದ್ದು
ಸ್ಥಾನ-ಪತಿ-ಯಾಗಿ-ರ-ಬಹು-ದಾದ
ಸ್ಥಾನ-ಪತಿ-ಯಾಗಿ-ರ-ಬ-ಹುದು
ಸ್ಥಾನ-ಪತಿ-ಯಾಗಿ-ರ-ಬಹು-ದೆಂದು
ಸ್ಥಾನ-ಪತಿ-ಯಾದ
ಸ್ಥಾನ-ಪತಿ-ಯಾದನು
ಸ್ಥಾನ-ಪತಿಯೂ
ಸ್ಥಾನಪ್ರಾಪ್ತಿಗೆ
ಸ್ಥಾನಪ್ರಾಪ್ರಿಗೂ-ದೇವಾ-ಲ-ಯಕ್ಕೆ
ಸ್ಥಾನ-ಮನಾಳ್ದೊರು
ಸ್ಥಾನ-ಮನಾಳ್ವ
ಸ್ಥಾನ-ಮಾನ
ಸ್ಥಾನ-ಮಾನ-ಗಳು
ಸ್ಥಾನ-ಮಾನ-ವನ್ನು
ಸ್ಥಾನ-ಮಾನ-ವನ್ನೋ
ಸ್ಥಾನ-ಮಾನ-ವಿತ್ತು
ಸ್ಥಾನ-ಮಾನವು
ಸ್ಥಾನ-ವನ್ನು
ಸ್ಥಾನ-ವಾಗಿ-ರ-ಬೇಕು
ಸ್ಥಾನ-ವಾದ
ಸ್ಥಾನ-ವಾಸಿಗೆ
ಸ್ಥಾನ-ವಿತ್ತು
ಸ್ಥಾನ-ವಿದೆ-ಯೆಂದು
ಸ್ಥಾನವು
ಸ್ಥಾನ-ವೃತ್ತಿಗೆ
ಸ್ಥಾನ-ವೆಂದು
ಸ್ಥಾನಿಕ
ಸ್ಥಾನಿ-ಕ-ತನ
ಸ್ಥಾನಿ-ಕ-ನಾಗಿ
ಸ್ಥಾನಿ-ಕ-ನಾಗಿದ್ದನು
ಸ್ಥಾನಿ-ಕ-ನಾಗಿದ್ದ-ನೆಂದು
ಸ್ಥಾನಿ-ಕ-ನಿಗೆ
ಸ್ಥಾನಿ-ಕ-ರಾಗಿದ್ದ-ರೆಂದು
ಸ್ಥಾನಿ-ಕರು
ಸ್ಥಾನೀಕ
ಸ್ಥಾನೀಕ-ತನ
ಸ್ಥಾನೀಕ-ತನ-ವನ್ನು
ಸ್ಥಾನೀಕ-ನನ್ನಾಗಿ
ಸ್ಥಾನೀಕ-ನಾಗಿದ್ದನು
ಸ್ಥಾನೀ-ಕನು
ಸ್ಥಾನೀಕ-ರನ್ನಾಗಿ
ಸ್ಥಾನೀಕ-ರಾದ
ಸ್ಥಾನೀ-ಕರು
ಸ್ಥಾಪಕ
ಸ್ಥಾಪಕ-ನೆಂದು
ಸ್ಥಾಪಕ-ರಾದ
ಸ್ಥಾಪ-ಕರು
ಸ್ಥಾಪನಾ-ಚಾರ್ಯ
ಸ್ಥಾಪನಾ-ಚಾರ್ಯ-ರಾದ
ಸ್ಥಾಪನೆ
ಸ್ಥಾಪನೆಗೆ
ಸ್ಥಾಪನೆಯ
ಸ್ಥಾಪನೆ-ಯಲ್ಲಿ
ಸ್ಥಾಪನೆ-ಯಾಗಿದ್ದು-ದ-ರಿಂದ
ಸ್ಥಾಪನೆ-ಯಾಗಿ-ರ-ಬ-ಹುದು
ಸ್ಥಾಪನೆ-ಯಾದ
ಸ್ಥಾಪಿತ-ಗೊಂಡ
ಸ್ಥಾಪಿತ-ವಾಗಿ
ಸ್ಥಾಪಿತ-ವಾ-ಗಿತ್ತು
ಸ್ಥಾಪಿತ-ವಾಗಿ-ರ-ಬಹು-ದಾದ
ಸ್ಥಾಪಿತ-ವಾದವು
ಸ್ಥಾಪಿಸ-ಲಾಗಿದೆ
ಸ್ಥಾಪಿಸ-ಲಾ-ಗಿದ್ದು
ಸ್ಥಾಪಿಸಲಾಯಿ-ತೆಂದು
ಸ್ಥಾಪಿ-ಸಲು
ಸ್ಥಾಪಿಸಲ್ಪಟ್ಟಿವೆ
ಸ್ಥಾಪಿಸಿ
ಸ್ಥಾಪಿ-ಸಿದ
ಸ್ಥಾಪಿಸಿ-ದನು
ಸ್ಥಾಪಿಸಿ-ದ-ನೆಂದು
ಸ್ಥಾಪಿಸಿ-ದ-ನೆಂದೂ
ಸ್ಥಾಪಿಸಿ-ದ-ರೆಂದು
ಸ್ಥಾಪಿಸಿದ್ದಾನೆ
ಸ್ಥಾಪಿಸಿ-ರ-ಬ-ಹುದು
ಸ್ಥಾಪಿಸಿ-ರುವ
ಸ್ಥಾಪಿಸುತ್ತಿದ್ದರು
ಸ್ಥಾಪಿ-ಸು-ವಲ್ಲಿ
ಸ್ಥಾಪ್ಯಂತೇ
ಸ್ಥಾಪ್ಯ-ತತ್ರೈವ
ಸ್ಥಾಫನಾ-ಚಾರ್ಯ
ಸ್ಥಾವರ
ಸ್ಥಾವರ-ವೆಂಬ
ಸ್ಥಾವರ-ಸುಂಕ
ಸ್ಥಾವರ-ಸುಂಕಕೆ
ಸ್ಥಾವರ-ಸುಂಕ-ವೆಂದು
ಸ್ಥಿತಿ
ಸ್ಥಿತಿ-ಗ-ತಿ-ಗಳ
ಸ್ಥಿತಿ-ಯನ್ನು
ಸ್ಥಿತಿ-ಯಲ್ಲಿದೆ
ಸ್ಥಿತಿ-ಯಿಂದ
ಸ್ಥಿತ್ಯಂತ-ರದ
ಸ್ಥಿರ
ಸ್ಥಿರಂ
ಸ್ಥಿರ-ಜೀವಿ-ಗಳಾದ-ರೆಂದು
ಸ್ಥಿರ-ತಾಟಂಕ-ವತ್ಯ-ಭೂತ್
ಸ್ಥಿರ-ನಾ-ರಾಯಣಂ
ಸ್ಥಿರ-ಪಡಿ-ಸತೊಡಗಿ-ದನು
ಸ್ಥಿರ-ವಾಗಿ
ಸ್ಥಿರ-ವಾದ
ಸ್ಥಿರ-ವಾಯಿ-ತೆಂದು
ಸ್ಥಿರ-ವೈಭವಸ್ತಸ್ಯ
ಸ್ಥೂಲ
ಸ್ಥೂಲ-ವಾಗಿ
ಸ್ಥೈರ್ಯಮಂದರಂ
ಸ್ನಾನ
ಸ್ನಾನ-ಪಾ-ನಾದಿ-ಗ-ಳಿಗೆ
ಸ್ನಾನ-ಮಾಡಿ
ಸ್ನಾನ-ಮಾಡುತ್ತಿದ್ದ
ಸ್ನಿಗ್ಧಂ
ಸ್ನೇಹ
ಸ್ನೇಹ-ವನ್ನು
ಸ್ನೇಹಿ-ತನೂ
ಸ್ಪತಿ
ಸ್ಪದ
ಸ್ಪದ-ದಿಂದಿರೆ
ಸ್ಪಪ್ಟ-ವಾಗಿ
ಸ್ಪಷ್ಟ
ಸ್ಪಷ್ಟ-ಚಿತ್ರಣ-ವನ್ನು
ಸ್ಪಷ್ಟತೆ
ಸ್ಪಷ್ಟ-ಪಡಿ-ಸುತ್ತದೆ
ಸ್ಪಷ್ಟ-ಪಡಿ-ಸುತ್ತ-ದೆಂದು
ಸ್ಪಷ್ಟ-ವಾಗಿ
ಸ್ಪಷ್ಟ-ವಾಗಿದೆ
ಸ್ಪಷ್ಟ-ವಾಗಿ-ರು-ವು-ದಿಲ್ಲ
ಸ್ಪಷ್ಟ-ವಾಗಿಲ್ಲ
ಸ್ಪಷ್ಟ-ವಾಗುತ್ತದೆ
ಸ್ಪಷ್ಟ-ವಾಗು-ವುದು
ಸ್ಪಷ್ಟ-ವಿ-ರು-ವು-ದಿಲ್ಲ
ಸ್ಪಷ್ಟ-ವಿಲ್ಲ
ಸ್ಫಾರಪ್ರತಾಪ
ಸ್ಮರಣ
ಸ್ಮರ-ಣಾರ್ಥ
ಸ್ಮರ-ಣಾರ್ಥ-ವಾಗಿ
ಸ್ಮರಣೆಗೆ
ಸ್ಮರಿಸ-ಬ-ಹುದು
ಸ್ಮರಿ-ಸುವ
ಸ್ಮಶಾ-ನದ
ಸ್ಮಶಾನ-ದಲ್ಲಿ-ರುವ
ಸ್ಮಾರಕ
ಸ್ಮಾರ-ಕ-ಗ-ಳನ್ನು
ಸ್ಮಾರ-ಕ-ಗಳಿ-ರುವ
ಸ್ಮಾರ-ಕ-ಗಳು
ಸ್ಮಾರ-ಕ-ಗ-ಳೆಂದು
ಸ್ಮಾರ-ಕ-ಗಳೇ
ಸ್ಮಾರ-ಕ-ವನ್ನು
ಸ್ಮಾರ-ಕ-ವಾಗಿ
ಸ್ಮಾರ-ಕ-ವಾಗಿಯೂ
ಸ್ಮಾರ-ಕ-ಶಾ-ಸನ
ಸ್ಮಾರ-ಕ-ಶಿಲೆ-ಗಳು
ಸ್ಮಾರ್ತ
ಸ್ಮಾರ್ತಬ್ರಾಹ್ಮಣನು
ಸ್ಮಾರ್ತಬ್ರಾಹ್ಮಣ-ರಲ್ಲಿ
ಸ್ಮಾರ್ತಬ್ರಾಹ್ಮಣ-ರಿಗೆ
ಸ್ಮಾರ್ತ-ಭಾಗ-ವತ
ಸ್ಮಾರ್ತ-ಭಾಗ-ವತರೇ
ಸ್ಮಾರ್ತ-ಸಂಪ್ರ-ದಾಯದ
ಸ್ಮಾರ್ತ-ಸಂಪ್ರ-ದಾಯ-ದ-ವರೇ
ಸ್ಮೃತಿ
ಸ್ಮೃತ್ಯುಕ್ತಾ-ಚಾರ
ಸ್ಯಮ್ಯಕ್ತ್ವ
ಸ್ಯಾಂಕಿ
ಸ್ಯಾಸನವ
ಸ್ರೇಯಪ್ರಾಪ್ತ
ಸ್ರೋತಸ್ವಿನೀ
ಸ್ವಂತ
ಸ್ವಇಚ್ಚೆ-ಯಿಂದ
ಸ್ವಇಚ್ಛೆ-ಯಿಂದ
ಸ್ವಕೀಯ
ಸ್ವಕೀಯ-ಕರ್ನಾಟ-ಕಕ
ಸ್ವಕೀಯೈ-ಕಾದ-ಶ-ಪಲ್ಲಿ
ಸ್ವಖಡ್ಗೈಕ
ಸ್ವಗ್ರಾಮಿ
ಸ್ವಚ್ಛ-ವಾಗಿ-ರಬೇಕೆಂಬ
ಸ್ವಜನಂ
ಸ್ವಜನ-ಗೋತ್ರ
ಸ್ವತಂತ್ರ
ಸ್ವತಂತ್ರ-ನಾಗ-ಬೇಕೆಂದು
ಸ್ವತಂತ್ರ-ನಾ-ಗಲು
ಸ್ವತಂತ್ರ-ನಾಗಿ
ಸ್ವತಂತ್ರ-ನಾದ
ಸ್ವತಂತ್ರ-ರಾಗಿ
ಸ್ವತಂತ್ರ-ರಾಜ-ನಂತೆ
ಸ್ವತಂತ್ರ-ವಾಗಿ
ಸ್ವತಂತ್ರ-ವಾದ
ಸ್ವತಂತ್ರ-ವಾದುವು-ಗಳಲ್ಲ
ಸ್ವತಃ
ಸ್ವದತ್ತಂ
ಸ್ವದೇಶಸ್ವಸ್ಥಳ
ಸ್ವಧರ್ಮ
ಸ್ವಧರ್ಮ-ದಿಂದ
ಸ್ವಪ-ರನ
ಸ್ವಪರ-ನಾಗಿ-ರ-ಬಹು-ದೆಂದು
ಸ್ವಪ್ನ-ದಲ್ಲಿ
ಸ್ವಭಾನು
ಸ್ವಯಂ
ಸ್ವಯಂಕೃತ-ವಹ
ಸ್ವಯಂಕೃತ್ವ
ಸ್ವಯಂಪಾಕಿ
ಸ್ವಯಂಭು
ಸ್ವಯಂಭು-ದೇವರ
ಸ್ವಯಂಭು-ದೇವ-ರಿಗೆ
ಸ್ವಯಂಭು-ನಾಥ
ಸ್ವಯಂಭು-ವೇಶ್ವರ
ಸ್ವಯಂಭೂ
ಸ್ವಯಂಭೂಃ
ಸ್ವಯಂಭೂ-ನಾಥ-ನಿಗೆ
ಸ್ವರ-ವನ-ಹಳ್ಳಿ
ಸ್ವರೂಪ
ಸ್ವರೂಪದ
ಸ್ವರೂಪದ್ದಾ-ಗಿದ್ದು
ಸ್ವರೂಪದ್ದಾಗಿವೆ
ಸ್ವರೂಪ-ವನ್ನು
ಸ್ವರ್ಗ-ಮರ್ತ್ಯಪಾ-ತಾಳ
ಸ್ವರ್ಗ-ಮೇಱಿದೊಡೆ
ಸ್ವರ್ಗ-ಲೋಕ
ಸ್ವರ್ಗ-ಲೋಕ-ಸುಕಪ್ರಾಪ್ತ-ನೆಂದು
ಸ್ವರ್ಗವ
ಸ್ವರ್ಗಸ್ಥ-ನಾಗಿ
ಸ್ವರ್ಗಸ್ಥ-ನಾಗುತ್ತಾನೆ
ಸ್ವರ್ಗಸ್ಥ-ನಾದ
ಸ್ವರ್ಗಸ್ಥ-ನಾದ-ನೆಂದು
ಸ್ವರ್ಗಸ್ಥ-ನಾದಲಿ
ಸ್ವರ್ಗಸ್ಥ-ನಾದಾಗ
ಸ್ವರ್ಗಸ್ಥರಾಗುತ್ತಾರೆ
ಸ್ವರ್ಗಸ್ಥ-ರಾದರು
ಸ್ವರ್ಗಸ್ಥ-ರಾದ-ರೆಂದು
ಸ್ವರ್ಗಸ್ಥ-ರಾದಾಗ
ಸ್ವರ್ಗಸ್ಥಳಾದ-ಳೆಂದು
ಸ್ವರ್ಗ್ಗಕ್ಕೆ
ಸ್ವರ್ಣಕಿರೀಟ
ಸ್ವರ್ಣಕಿರೀಟ-ವನ್ನು
ಸ್ವರ್ಣ-ಯುಗವು
ಸ್ವಲ್ಪ
ಸ್ವಲ್ಪ-ಕಾಲ
ಸ್ವಲ್ಪ-ಭಾಗ-ವನ್ನು
ಸ್ವಲ್ಪ-ಮಟ್ಟಿಗೆ
ಸ್ವಲ್ಪ-ಮಟ್ಟಿನ
ಸ್ವಲ್ಪವೂ
ಸ್ವಷ್ಟ-ವಾಗಿ
ಸ್ವಸಂತೋಷಕ್ಕೆ
ಸ್ವಸ್ತಾನೇಕ
ಸ್ವಸ್ತಿ
ಸ್ವಸ್ತಿ-ಪುರ-ವ-ರಾಧೀಶ್ವರ
ಸ್ವಸ್ತಿ-ಯ-ನ-ವರತ
ಸ್ವಸ್ತಿ-ಳಕೈಃ
ಸ್ವಸ್ತಿಶ್ರೀ
ಸ್ವಸ್ತಿಶ್ರೀ-ಯುತ
ಸ್ವಸ್ತ್ಯ-ನ-ವರತ
ಸ್ವಸ್ಥಿರ
ಸ್ವಸ್ಥಿ-ರದ
ಸ್ವಸ್ವಾಮಿನಂ
ಸ್ವಹಸ್ತದ
ಸ್ವಹಸ್ತ-ದಿಂದ
ಸ್ವಾಂಮ್ಯ
ಸ್ವಾಗ-ತಿ-ಸು-ವುದು
ಸ್ವಾಚಾರ್ಯಾಯ
ಸ್ವಾತಂತ್ರ್ಯ
ಸ್ವಾತಂತ್ರ್ಯ-ವನ್ನು
ಸ್ವಾತಂತ್ರ್ಯವೂ
ಸ್ವಾತಿ
ಸ್ವಾಧಿನ-ಪಡಿ-ಸ-ಕೊಂಡು
ಸ್ವಾಧೀ-ನನ-ಯಸಂಪದಃ
ಸ್ವಾಧ್ಯಾಯ
ಸ್ವಾಪರ
ಸ್ವಾಭಾವಿಕ
ಸ್ವಾಭಾವಿ-ಕ-ವಾದ
ಸ್ವಾಮಿ
ಸ್ವಾಮಿ-ಕಾರ್ಯ
ಸ್ವಾಮಿ-ಕಾರ್ಯ-ಧುರೀಣ-ನಾ-ಗಿದ್ದ
ಸ್ವಾಮಿ-ಗಳ
ಸ್ವಾಮಿ-ಗ-ಳನ್ನು
ಸ್ವಾಮಿ-ಗಳ-ವರು
ಸ್ವಾಮಿ-ಗ-ಳಾಗಿದ್ದ
ಸ್ವಾಮಿ-ಗ-ಳಿಗೆ
ಸ್ವಾಮಿ-ಗಳಿ-ರ-ಬ-ಹುದು
ಸ್ವಾಮಿ-ಗಳು
ಸ್ವಾಮಿಗೆ
ಸ್ವಾಮಿದ್ರೋಹ-ರ-ಗಂಣ್ಡನುಂ
ಸ್ವಾಮಿದ್ರೋಹ-ರ-ಗಣ್ಡನುಂ
ಸ್ವಾಮಿ-ಭಕ್ತಿಗಂ
ಸ್ವಾಮಿ-ಭಕ್ತಿಗೆ
ಸ್ವಾಮಿ-ಭೃತ್ಯಂ
ಸ್ವಾಮಿಯ
ಸ್ವಾಮಿ-ಯಂಗ-ಸನ್ನಾಹ
ಸ್ವಾಮಿ-ಯನ್ನು
ಸ್ವಾಮಿ-ಯರ
ಸ್ವಾಮಿ-ಯ-ವರ
ಸ್ವಾಮಿ-ಯಾದ
ಸ್ವಾಮಿಯು
ಸ್ವಾಮಿ-ವಂಚ-ಕರ-ಗಂಡ
ಸ್ವಾಮ್ಯ
ಸ್ವಾಮ್ಯ-ವಂತರು
ಸ್ವಾಮ್ಯ-ವನು
ಸ್ವಾಮ್ಯ-ವನ್ನೂ
ಸ್ವಾರಸ್ಯಕ-ರ-ವಾಗಿವೆ
ಸ್ವಾರಾಜ-ರಾಜ-ಮಾನಶ್ರೀ
ಸ್ವೀಕರಿ-ಸದೇ
ಸ್ವೀಕರಿ-ಸ-ಬೇಕು
ಸ್ವೀಕರಿ-ಸಲು
ಸ್ವೀಕರಿಸಿ
ಸ್ವೀಕರಿ-ಸಿ-ಕೊಂಡ
ಸ್ವೀಕರಿ-ಸಿದ
ಸ್ವೀಕರಿ-ಸಿ-ದನು
ಸ್ವೀಕರಿ-ಸಿದ್ದ-ನೆಂದು
ಸ್ವೀಕರಿ-ಸಿದ್ದರೆ
ಸ್ವೀಕರಿ-ಸಿದ್ದ-ರೆಂದು
ಸ್ವೀಕರಿ-ಸಿದ್ದ-ರೆಂಬುದು
ಸ್ವೀಕರಿ-ಸಿದ್ದಾರೆ
ಸ್ವೀಕರಿ-ಸಿ-ರ-ಬ-ಹುದು
ಸ್ವೀಕರಿ-ಸುತ್ತಾನೆ
ಸ್ವೀಕಾರ-ಸಾರೋ-ದಯ
ಸ್ವೀಯಪ್ರತಾಪೋ-ದಯೌ
ಸ್ವೋರ-ನಾಡಿನ
ಸ್ವೋರೆ-ನಾಡಿ-ನಲ್ಲಿ
ಸ್ವೋರೆ-ನಾಡು
ಸ್ಷಷ್ಟ-ವಾಗಿ
ಸ್ಷಷ್ಟ-ವಾಗುತ್ತದೆ
ಹ
ಹಂಗಾಮು
ಹಂಗಿ-ಸು-ವುದು
ಹಂಗು
ಹಂಚಿಕೆ
ಹಂಚಿಕೆ-ಗ-ಳಿಗೆ
ಹಂಚಿಕೆ-ಮಾಡಿ-ಕೊಂಡು
ಹಂಚಿಕೆ-ಯನ್ನು
ಹಂಚಿಕೆ-ಯಾಗಿದೆ
ಹಂಚಿಕೆ-ಯಾ-ದಂತೆ
ಹಂಚಿ-ಕೊಂಡು
ಹಂಚಿ-ಪುರ
ಹಂಚಿಯ
ಹಂಜ-ಮಾನ
ಹಂಜ-ಮಾನ-ಗಳು
ಹಂಣೆಚೌ-ಕನ-ಹಳ್ಳಿ-ಅಣ್ಣೆಚಾ-ಕನ-ಹಳ್ಳಿ
ಹಂತ
ಹಂತಕ್ಕೆ
ಹಂತ-ಗಳ
ಹಂತ-ಗ-ಳನ್ನು
ಹಂತದ
ಹಂತ-ದಲ್ಲಿ
ಹಂತ-ದ-ವರೆಗೆ
ಹಂತ-ದಿಂದ
ಹಂತ-ವಾದರೆ
ಹಂತ-ವೆಂದು
ಹಂತ-ಹಂತ-ವಾಗಿ
ಹಂದಿ
ಹಂದಿ-ಬೇಟೆ-ಯಲ್ಲಿ
ಹಂದಿ-ಯ-ನಿ-ರಿದು
ಹಂದಿ-ಯನ್ನು
ಹಂದೂರ್
ಹಂದೇ-ಹಳ್ಳಿ
ಹಂನೆ-ರಡು
ಹಂನೆ-ರಡು-ವರ್ಷ
ಹಂಪ
ಹಂಪನಾ
ಹಂಪ-ನಾಗ-ರಾಜಯ್ಯ-ನ-ವರು
ಹಂಪ-ರಾಜರ
ಹಂಪಾ-ಪುರ
ಹಂಪಾ-ಪುರದ
ಹಂಪಿ
ಹಂಪಿಗೆ
ಹಂಪೆಯ
ಹಂಪೆ-ಯನ್ನು
ಹಂಪೆ-ಯಲ್ಲಿ
ಹಂಪೆಯಲ್ಲಿಯೇ
ಹಂಪೆಯೇ
ಹಂಬಲ-ವನ್ನು
ಹಂಸ-ಗಳಿದ್ದು
ಹಂಸಾ-ನನ್ದ-ಕರ
ಹಕ
ಹಕುವು-ದಕೆ
ಹಕ್ಕನ್ನು
ಹಕ್ಕನ್ನೂ
ಹಕ್ಕಿನ
ಹಕ್ಕಿ-ಮಂಚನ-ಹಳ್ಳಿ
ಹಕ್ಕಿಯೂ
ಹಕ್ಕೀ-ಮಂಚನ-ಹಳ್ಳಿ
ಹಕ್ಕು
ಹಕ್ಕು-ದಾರ-ನೆಂದು
ಹಕ್ಕುಸ್ಥಾಪಿಸಿ-ದ-ರೆಂದೂ
ಹಗಮ-ಗೆರೆಯ
ಹಗಮ-ಗೆರೆ-ಯ-ಆ-ದಾಯದ
ಹಗವಮ-ಗೆರೆ
ಹಗವಮ-ಗೆರೆಗೆ
ಹಗವಮ-ಗೆರೆ-ಯನ್ನು
ಹಗೆ-ತನ-ವಿರ-ಲಿಲ್ಲ
ಹಗ್ಗ-ಗ-ಳನ್ನು
ಹಚ್ಚಿ-ಕೊಂಡಿರುತ್ತಿದ್ದ
ಹಚ್ಚಿದ್ದ-ನೆಂದೂ
ಹಜರಿ
ಹಜಾರ-ಗಳಿವೆ
ಹಜಾರ-ವಿದೆ
ಹಜಾ-ರವು
ಹಟ್ಟಣ
ಹಟ್ಟಣದ
ಹಟ್ಟಣ-ದಲ್ಲಿ
ಹಟ್ಟಣ-ದಲ್ಲಿ-ಪಟ್ಟಣ
ಹಟ್ಟಣ-ವನ್ನು
ಹಟ್ಟಿ
ಹಟ್ಟಿ-ಗ-ಳನ್ನೂ
ಹಟ್ಟಿ-ಗಾಲ-ಗಕ್ಕೆಕ
ಹಟ್ಟಿ-ಗಾಳಗ-ದಲ್ಲಿ
ಹಟ್ಟಿ-ಗಾಳೆಗ
ಹಟ್ಟಿ-ಗಾಳೆ-ಗಕ್ಕೆ
ಹಟ್ಟಿ-ಗಾಳೆಗ-ದಲ್ಲಿ
ಹಟ್ಟಿ-ಗಾಳೆಗ-ದೊಳ್
ಹಟ್ಟಿ-ಬಿರೆ
ಹಟ್ಟಿಯ
ಹಟ್ಟಿ-ಯನ್ನು
ಹಟ್ಣದ
ಹಠ-ಹರಣ
ಹಡ-ಗಲಿ
ಹಡ-ಗಿನ
ಹಡಗು
ಹಡಗು-ಗಳು
ಹಡದ
ಹಡದಕ್ಷೇತ್ರದ
ಹಡದು
ಹಡ-ಪದ
ಹಡಪ-ವಳ
ಹಡಪ-ವಳ-ರಾಗಿದ್ದರೂ
ಹಡ-ಬಳ
ಹಡ-ವನ-ಹಳ್ಳಿ
ಹಡವಪಳ
ಹಡ-ವಳ
ಹಡ-ವಳ-ತಿಯ
ಹಡ-ವಳದ
ಹಡ-ವಳ-ದೇವ
ಹಡ-ವಳ-ನಾ-ಗಿದ್ದ
ಹಡ-ವಳ-ರ-ಪಡೆ-ವಳ
ಹಡ-ವಳರು
ಹಡ-ವಳಿ-ತಿಯ
ಹಡ-ವಳಿ-ತಿಯರು
ಹಡುವಂಗಲ
ಹಡುವಳ
ಹಡುವ-ಳ-ಕೆರೆ
ಹಡುವ-ಳದ
ಹಡುವ-ಳರ
ಹಡುವ-ಳರು
ಹಡುವ-ಳರೇ
ಹಡುವ-ಳ-ಹಡೆ-ವಳ-ಹಡ-ಪದ
ಹಡು-ವಳಿ-ತಿಯ
ಹಡೆದು-ಪಡೆದು
ಹಡೆಪ-ವಳ
ಹಡೆ-ವಳ
ಹಡೆ-ವಳನ
ಹಣ
ಹಣ-ಇ-ನಾಮು
ಹಣ-ಕಾ-ಸನ್ನು
ಹಣ-ಕಾ-ಸಿನ
ಹಣ-ಕಾಸು
ಹಣಕ್ಕೆ
ಹಣ-ಗ-ಳನ್ನು
ಹಣ-ಗೋಳ
ಹಣದ
ಹಣ-ದ-ರೂಪ-ದಲ್ಲಿ
ಹಣ-ದ-ರೂಪ-ದಲ್ಲೇ
ಹಣ-ದ-ಲೆಕ್ಕ-ದಲ್ಲಿ
ಹಣ-ದಿಂದ
ಹಣ-ಪಡೆದು
ಹಣ-ವನು
ಹಣ-ವನ್ನು
ಹಣ-ವಿನ
ಹಣ-ವೀವೆ
ಹಣವು
ಹಣ-ವೆ-ರಡು
ಹಣು-ಗನ-ಕೆರೆ
ಹಣು-ಗನ-ಕೆರೆಯ
ಹಣೆ
ಹಣೆಗೆ
ಹಣೆಯ
ಹಣೆ-ಯ-ಮೇಲೆ
ಹಣೆ-ಯಲ್ಲಿ
ಹಣ್ಣೆಯ
ಹಣ್ನೆಯ
ಹತ-ನಾಗಿ-ರುವುದು
ಹತನಾ-ದನು
ಹತ-ನಾದಾಗ
ಹತಾಶ-ನಾಗಿ
ಹತಾಶೆ-ಗೊಂಡನು
ಹತು
ಹತೋಟಿ-ಯನ್ನು
ಹತೋಟಿ-ಯಲ್ಲಿದ್ದರು
ಹತ್ತನು
ಹತ್ತ-ನೆಯ
ಹತ್ತನೇ
ಹತ್ತಾರು
ಹತ್ತಿಕ್ಕಲು
ಹತ್ತಿಕ್ಕಿ
ಹತ್ತಿಕ್ಕಿ-ದನು
ಹತ್ತಿಕ್ಕು-ವಂತೆ
ಹತ್ತಿ-ಯ-ಕಟ್ಟೆ
ಹತ್ತಿರ
ಹತ್ತಿ-ರಕ್ಕೆ
ಹತ್ತಿ-ರದ
ಹತ್ತಿ-ರ-ದಲ್ಲಿ
ಹತ್ತಿ-ರ-ದಲ್ಲಿಯೇ
ಹತ್ತಿ-ರ-ದಿಂದಲೇ
ಹತ್ತಿ-ರ-ವಾದ
ಹತ್ತಿ-ರ-ವಾಯಿ-ತೆಂದು
ಹತ್ತಿ-ರ-ವಿ-ರುವ
ಹತ್ತು
ಹತ್ತು-ಅಡಿ
ಹತ್ತು-ಗದ್ಯಾಣ
ಹತ್ತು-ದಿವಸ
ಹತ್ತು-ವೃತ್ತಿಯ
ಹತ್ತು-ಸಲಗೆ
ಹತ್ತು-ಸಾವಿರ
ಹತ್ತು-ಹೊಂನ
ಹತ್ತು-ಹೊನ್ನನ್ನು
ಹತ್ತೊಂಬತ್ತ-ನೆಯ
ಹತ್ಯೆ-ಗಳೊಡನೆ
ಹದ-ಗೆಟ್ಟಿದ್ದ-ರಿಂದ
ಹದ-ರಹ-ಳಿ-ವಿನ
ಹದಲಿ-ಕೆರೆ
ಹದಲಿ-ಕೆರೆಯ
ಹದಿಕೆ
ಹದಿಕೆ-ಗಳ
ಹದಿಕೆ-ಗ-ಳನ್ನು
ಹದಿಕೆಯ
ಹದಿಕೆ-ಯನೂ
ಹದಿನಯ್ದನಿಳಿಹಿ-ಕೊಂಡು-ಳಿ-ಯಿತ್ತನಾ
ಹದಿ-ನಾಡ
ಹದಿ-ನಾಡನ್ನು
ಹದಿನಾ-ಡ-ಸೀಮೆಯ
ಹದಿ-ನಾ-ಡಿಗೆ
ಹದಿನಾ-ಡಿನ
ಹದಿ-ನಾಡು
ಹದಿನಾ-ರ-ನೆಯ
ಹದಿನಾರು
ಹದಿ-ನಾಲ್ಕ-ನೆಯ
ಹದಿ-ನಾಲ್ಕು
ಹದಿ-ನಾಲ್ಕು-ಮಂದಿ
ಹದಿ-ನಾಲ್ಕು-ಹ-ದಿನಾ-ಡು-ನಾಡು
ಹದಿನೆಂಟ-ನೆಯ
ಹದಿನೆಂಟು
ಹದಿನೆಂಟು-ನಾಡ
ಹದಿನೆಂಟು-ಪಟ್ಟಣ
ಹದಿನೈ-ಗುಳ
ಹದಿನೈ-ದನು
ಹದಿನೈದ-ನೆಯ
ಹದಿನೈದು
ಹದಿ-ಮೂರ-ನೆಯ
ಹದಿ-ಮೂರು
ಹನ-ಸೋಗೆ
ಹನ-ಸೋಗೆಯ
ಹನಿ-ಗಳು
ಹನಿಯ
ಹನುಂತನು
ಹನುಮ
ಹನುಮಂತ
ಹನುಮಂತ-ಮುದ್ರೆ
ಹನುಮಂತ-ರಾಯಸ್ವಾಮಿಗೆ
ಹನುಮಂತಸ್ವಾಮಿ-ಯನ್ನು
ಹನುಮಂತೇಶ್ವರ
ಹನು-ಮದ್ಗರುಡ
ಹನು-ಮನ
ಹನು-ಮ-ನ-ಕಟ್ಟೆ
ಹನು-ಮನೇ
ಹನೆ-ಮಠ
ಹನೆ-ಮಠದ
ಹನ್ನೆರಡ
ಹನ್ನೆರಡಕ್ಕಂ
ಹನ್ನೆರಡ-ನೆಯ
ಹನ್ನೆರಡನೇ
ಹನ್ನೆ-ರಡು
ಹನ್ನೊಂದ-ನೆಯ
ಹನ್ನೊಂದು
ಹನ್ಮನೆನೀ
ಹಪ್ಪ-ಳಿಗೆ-ಯನ್ನು
ಹಬೀ-ಬನ
ಹಬೀ-ಬನು
ಹಬೀಬುಲ್ಲಾ
ಹಬ್ಬ
ಹಬ್ಬ-ಗ-ಳನ್ನು
ಹಬ್ಬ-ಗ-ಳಲ್ಲಿ
ಹಬ್ಬ-ದಲ್ಲಿ
ಹಬ್ಬಿ
ಹಬ್ಬಿಸಿ
ಹಮೀದ್
ಹಯಗ್ರೀವ-ದೇವರ
ಹಯಪ್ರತ-ತಿಯಂ
ಹಯವ-ದನ-ರಾಯರು
ಹಯವ-ದನ-ರಾಯರೂ
ಹಯವ-ದನ-ರಾವ್
ಹಯವಸ
ಹಯಾರೂಢ
ಹಯಾರೂ-ಢ-ನಾದ-ನೆಂದು
ಹರ-ಈಶ್ವರ
ಹರಕೆ
ಹರ-ಗನ-ಹಳ್ಳಿ
ಹರ-ಡ-ಲಾರಂಭಿ-ಸಿತ್ತೆನ್ನು-ವು-ದಕ್ಕೆ
ಹರಡಿ
ಹರ-ಡಿತ್ತು
ಹರ-ಡಿತ್ತೆಂದು
ಹರ-ಡಿತ್ತೆಂಬುದು
ಹರ-ಡಿದೆ
ಹರ-ಡಿದ್ದ
ಹರ-ಡಿದ್ದು
ಹರ-ಡಿ-ರಬೇಕೆಂಬು-ದಕ್ಕೆ
ಹರ-ಡಿ-ರುವ
ಹರಡು-ವಿಕೆ
ಹರತಿ
ಹರ-ತಿ-ಸಿರಿ-ಯಲ್ಲಿ
ಹರದ
ಹರದ-ಗಾವುಂಡನು
ಹರದ-ನ-ಹಳ್ಳಿ
ಹರದ-ನ-ಹಳ್ಳಿ-ಬೇಚಿರಾಕ್
ಹರದ-ನ-ಹಳ್ಳಿಯ
ಹರದ-ನ-ಹಳ್ಳಿ-ಯನ್ನು
ಹರದ-ಸ-ಮುದ್ರ-ವೆಂಬ
ಹರದೇಶ್ವರ
ಹರ-ನಿಗೂ
ಹರ-ಪನ-ಹಳ್ಳಿಯ
ಹರ-ಪಾಲ
ಹರಮ
ಹರ-ಳ-ಹಳ್ಳ
ಹರ-ಳ-ಹಳ್ಳಿ-ಯ-ಕೆರೆ
ಹರ-ಳು-ಕೆರೆ
ಹರ-ಳು-ಹಳ್ಳಿ
ಹರ-ವಾಳ-ಗ-ಳಲ್ಲಿವೆ
ಹರವು
ಹರ-ಸಲು
ಹರಸಿ
ಹರಹ-ಗೌಡ
ಹರಹ-ಗೌಡನ
ಹರ-ಹಿನ
ಹರ-ಹಿನಲು
ಹರ-ಹಿನ-ವರೆಗೂ
ಹರಹು-ಗಳು
ಹರಿ
ಹರಿ-ಕಾರ
ಹರಿ-ಕಾರ್ಭಕ್ಷಿ-ಯಾ-ಗಿದ್ದ
ಹರಿ-ಕೆರೆ-ಗ-ಳನ್ನು
ಹರಿ-ಗಲ
ಹರಿ-ಗಿಲ
ಹರಿ-ಗೋ-ಲಿನ
ಹರಿ-ಜನ-ಕೇರಿಯ
ಹರಿಣ
ಹರಿತ
ಹರಿ-ತ-ವಾದ
ಹರಿತ್ಸ
ಹರಿ-ದಾಸ
ಹರಿ-ದಿನ
ಹರಿದು
ಹರಿ-ದು-ಬರುತ್ತಿ-ತೆಂದು
ಹರಿ-ದು-ಬರುತ್ತಿತ್ತೆಂದು
ಹರಿ-ದು-ಬರುತ್ತಿದ್ದ
ಹರಿ-ದು-ಹೋಗಿ
ಹರಿ-ದು-ಹೋಗುವ
ಹರಿ-ದೇವ
ಹರಿ-ನೀಲ
ಹರಿ-ಪಾಲನ
ಹರಿ-ಪಾಳಯ್ಯ
ಹರಿ-ಭಕ್ತಿ-ಸು-ಧಾನಿಧಿಃ
ಹರಿ-ಮಲ್ಲ-ವ-ರಿಗ-ವಜ್ರದಣ್ಡ
ಹರಿ-ಯಂಣನೆನಿ-ಸಿ-ದನು
ಹರಿ-ಯ-ಕೆರೆ-ಯಲ್ಲಿ
ಹರಿ-ಯಣ್ಣ
ಹರಿ-ಯಣ್ಣನ
ಹರಿ-ಯಣ್ಣನು
ಹರಿ-ಯ-ನಂದಿ
ಹರಿ-ಯಪ್ಪನು
ಹರಿ-ಯಪ್ಪ-ರಸ
ಹರಿ-ಯಪ್ಪ-ರ-ಸರ
ಹರಿ-ಯಬ್ಬೆ
ಹರಿ-ಯಲ
ಹರಿ-ಯಲು
ಹರಿ-ಯಲೆ
ಹರಿ-ಯಲೆಗೂ
ಹರಿ-ಯ-ಲೆಯ
ಹರಿ-ಯಲೆ-ಯರ
ಹರಿ-ಯಲೆರ
ಹರಿ-ಯಾ-ಲಮ್ಮ
ಹರಿ-ಯಾ-ಲಮ್ಮನ
ಹರಿ-ಯಿತು
ಹರಿಯು
ಹರಿ-ಯುತ್ತದೆ
ಹರಿ-ಯುತ್ತವೆ
ಹರಿ-ಯುತ್ತಿದೆ
ಹರಿ-ಯುತ್ತಿದ್ದ
ಹರಿ-ಯುತ್ತಿದ್ದವು
ಹರಿ-ಯುತ್ತಿದ್ದು
ಹರಿ-ಯುವ
ಹರಿ-ಯು-ವಂತೆ
ಹರಿ-ಯು-ವಾಗ
ಹರಿ-ಯು-ವು-ದಿಲ್ಲ
ಹರಿ-ವರ್ಮನ
ಹರಿ-ವಾಣ
ಹರಿ-ವಾಣ-ದಲ್ಲಿ
ಹರಿ-ವಾಣ-ವನ್ನು
ಹರಿ-ಸ-ಲಾಗುತ್ತಿದ್ದ
ಹರಿಸಿ
ಹರಿ-ಸಿ-ಕೊಳ್ಳುತಿದ್ದ-ರೆಂದು
ಹರಿ-ಸುವ
ಹರಿ-ಹ-ಪುರದ
ಹರಿ-ಹರ
ಹರಿ-ಹರಂ
ಹರಿ-ಹರ-ದಂಡ-ನಾಯಕ
ಹರಿ-ಹರ-ದಂಡ-ನಾಯ-ಕನ
ಹರಿ-ಹರ-ದಂಡ-ನಾಯ-ಕ-ನಿಗೆ
ಹರಿ-ಹರ-ದಂಡ-ನಾಯ-ಕನು
ಹರಿ-ಹರ-ದಂಡಾಯ-ಕನು
ಹರಿ-ಹರ-ದೇವ
ಹರಿ-ಹರ-ದೇವನು
ಹರಿ-ಹರ-ದೇವಾ-ಲವು
ಹರಿ-ಹರ-ಧರ-ಣೀ-ಪಾಲಕ
ಹರಿ-ಹ-ರನ
ಹರಿ-ಹರ-ನನ್ನು
ಹರಿ-ಹರ-ನಾ-ಗಿದ್ದು
ಹರಿ-ಹರ-ನಿಗೆ
ಹರಿ-ಹ-ರನು
ಹರಿ-ಹರ-ನೃಪ-ನ-ನುಜಾತಂ
ಹರಿ-ಹರ-ನೆಂಬ
ಹರಿ-ಹ-ರನೇ
ಹರಿ-ಹರ-ಪಟ್ಟಣ
ಹರಿ-ಹರ-ಪಟ್ಟ-ಣ-ದಲ್ಲಿ
ಹರಿ-ಹರ-ಪಟ್ಟ-ವರ್ಧನರ
ಹರಿ-ಹರ-ಪುರ
ಹರಿ-ಹರ-ಪುರಕ್ಕೆ
ಹರಿ-ಹರ-ಪುರ-ಗಳು
ಹರಿ-ಹರ-ಪುರದ
ಹರಿ-ಹರ-ಪುರ-ದಲ್ಲಿ
ಹರಿ-ಹರ-ಪುರ-ದಲ್ಲಿದ್ದ
ಹರಿ-ಹರ-ಪುರ-ವನ್ನು
ಹರಿ-ಹರ-ಪುರವು
ಹರಿ-ಹರ-ಪುರ-ವೆಂಬ
ಹರಿ-ಹರಬ್ರಹ್ಮಾದಿ-ಗಳೇ
ಹರಿ-ಹರ-ಭಟ್ಟನು
ಹರಿ-ಹರ-ಭಟ್ಟ-ನೆಂಬು-ವವ-ನಿಗೆ
ಹರಿ-ಹರ-ಭಟ್ಟೋಪಾಧ್ಯಾ-ಯನ
ಹರಿ-ಹರ-ಭಟ್ಟೋಪಾಧ್ಯಾ-ಯನೆಂದು
ಹರಿ-ಹರ-ಭಟ್ಟೋಪಾಧ್ಯಾಯ-ರಿಗೆ
ಹರಿ-ಹರ-ಭಟ್ಟೋಪಾಧ್ಯಾ-ಯರು
ಹರಿ-ಹರ-ಭಟ್ಟೋಪಾಧ್ಯಾ-ಯರೇ
ಹರಿ-ಹರ-ಭಟ್ಟೋಪಾಧ್ಯಾ-ರಿಗೆ
ಹರಿ-ಹರ-ಭಾಗ-ವತ-ಪಂಥದ
ಹರಿ-ಹರ-ಮಹಾ-ರಾಯನ
ಹರಿ-ಹರ-ಮಹಾ-ರಾಯನು
ಹರಿ-ಹರ-ಮಹಾ-ರಾಯರ
ಹರಿ-ಹರ-ಮೂರ್ತಿ
ಹರಿ-ಹರ-ರಲ್ಲಿ
ಹರಿ-ಹರ-ರಾಯ
ಹರಿ-ಹರ-ರಾಯನ
ಹರಿ-ಹರೇಶ್ವರ
ಹರುವ-ನ-ಹಳ್ಳಿಯ
ಹರುವು
ಹರೂರು
ಹರೆದು
ಹರೆಯ
ಹರೋಜ
ಹರೋಜ-ನನ್ನು
ಹರೋಜನು
ಹರ್ಮ್ಮ್ಯ-ಕುಲಕ್ಕೆ
ಹರ್ಮ್ಮ್ಯ-ಮಾಣಿಕ್ಯ
ಹರ್ಮ್ಯ-ಕುಲ
ಹರ್ಮ್ಯ-ಕುಲ-ದವ-ರಾಗಿದ್ದರು
ಹರ್ಮ್ಯ-ಮಾಣಿಕ್ಯ
ಹರ್ಯಣ
ಹರ್ಯಣನ
ಹರ್ಯಣ-ನನ್ನು
ಹರ್ಯಣ-ನಿಂದಾಗಿ
ಹರ್ಯಣನು
ಹರ್ಯ-ಣಾತ್ಮಜಃ
ಹರ್ಯಣೋ
ಹಲಕೂ-ರನ್ನು
ಹಲಕೂರು
ಹಲಗೂರ
ಹಲಗೂ-ರನ್ನು
ಹಲಗೂರಿನ
ಹಲಗೂರಿ-ನಲ್ಲಿ
ಹಲ-ಗೂರು
ಹಲಗೆ-ಕಾರ-ನಾಥ
ಹಲಗೆ-ಕಾರ-ನಾಥಂಗೆ
ಹಲಗೆ-ಗಳ
ಹಲ-ಗೆಯ
ಹಲಗೆ-ಯಿಂದ
ಹಲಗೆ-ಹೊಸ-ಹಳ್ಳಿ
ಹಲ-ನಾಡೊಳ-ಗಳ
ಹಲ-ಬರ
ಹಲರ
ಹಲರು
ಹಲವ-ರನ್ನು
ಹಲ-ವರು
ಹಲ-ವಾರು
ಹಲ-ವಾರು-ಕೊಳ
ಹಲ-ವಿವೆ
ಹಲವು
ಹಲವು-ಮಾ-ರಾದಿ
ಹಲ-ಸನ-ಹಳ್ಳಿ
ಹಲ-ಸನ-ಹಳ್ಳಿ-ಯನ್ನು
ಹಲಸ-ಹಳ್ಳಿ
ಹಲಸಿ
ಹಲ-ಸಿಗೆ
ಹಲಸಿ-ತಾಳ-ಹಳ್ಳಿಯ
ಹಲಸಿ-ನ-ತಾಳ
ಹಲಸಿ-ನ-ಹಳ್ಳಿ
ಹಲುಗೂರ
ಹಲುಗೂರಿನ
ಹಲ್ಮಿಡಿ
ಹಲ್ಲೆ-ಗೆರೆ
ಹಲ್ಲೆ-ಗೆರೆ-ಅಗ್ರ-ಹಾರ
ಹಳ-ಗನ್ನಡ
ಹಳಿ-ಕಾರ
ಹಳಿಕಾಱ
ಹಳಿ-ವನ
ಹಳುವು
ಹಳೆಯ
ಹಳೆಯ-ದಾ-ಗಿದ್ದು
ಹಳೆಯ-ಬಿಡು
ಹಳೆಯ-ಬೀಡ
ಹಳೆಯ-ಬೀ-ಡಿಗೆ
ಹಳೆಯ-ಬೀಡಿನ
ಹಳೆಯ-ಬೀಡು
ಹಳೆಯ-ಬೀಡೆಂದು
ಹಳೆಯ-ಬೂದ-ನೂರಿನ
ಹಳೆಯ-ಬೆಳ್ಗೊಳವೇ
ಹಳೆ-ಯೂರಿನ
ಹಳೇ-ಕೆರೆಯ
ಹಳೇ-ಬೀಡನ್ನು
ಹಳೇ-ಬೀಡಿನ
ಹಳೇ-ಬೀಡು
ಹಳೇ-ಬೂದ-ನೂರಿನ
ಹಳೇ-ಬೂದ-ನೂರಿನಲ್ಲಿದೆ
ಹಳೇ-ಬೂದ-ನೂರು
ಹಳೇ-ಮನೆ
ಹಳೇ-ಮಾದಾ-ಪುರವು
ಹಳ್ಳ
ಹಳ್ಳ-ಕೆರೆ-ಇಂದಿನ
ಹಳ್ಳ-ಕೊಳ್ಳ-ಗಳ
ಹಳ್ಳ-ಕೊಳ್ಳ-ಗ-ಳನ್ನು
ಹಳ್ಳ-ಕೊಳ್ಳ-ಗ-ಳನ್ನೂ
ಹಳ್ಳ-ಕೊಳ್ಳ-ಗಳು
ಹಳ್ಳಕ್ಕೆ
ಹಳ್ಳ-ಗಳ
ಹಳ್ಳ-ಗ-ಳನ್ನು
ಹಳ್ಳ-ಗ-ಳಲ್ಲಿ
ಹಳ್ಳ-ಗ-ಳಾಗಿದ್ದ-ವೆಂದು
ಹಳ್ಳ-ಗ-ಳಿಂದ
ಹಳ್ಳ-ಗ-ಳಿಗೆ
ಹಳ್ಳ-ಗಳು
ಹಳ್ಳದ
ಹಳ್ಳ-ದಂತಿದೆ
ಹಳ್ಳ-ದ-ಬೀಡಿನಲು
ಹಳ್ಳ-ದ-ಬೀಡು
ಹಳ್ಳ-ದಲ್ಲಿ
ಹಳ್ಳ-ದ-ಸಾ-ಗರ
ಹಳ್ಳದಿಂ
ಹಳ್ಳ-ದಿಂದ
ಹಳ್ಳ-ಬೀಡಾಗಿ-ರ-ಬ-ಹುದು
ಹಳ್ಳ-ಬೀಡಿ-ನಲ್ಲಿ
ಹಳ್ಳವ
ಹಳ್ಳ-ವಾಗಿ-ರುವ
ಹಳ್ಳ-ವಾದುವಂ
ಹಳ್ಳವು
ಹಳ್ಳ-ವೂರ
ಹಳ್ಳ-ವೂರು
ಹಳ್ಳಿ
ಹಳ್ಳಿ-ಕಾರ
ಹಳ್ಳಿ-ಕೆರೆ
ಹಳ್ಳಿ-ಕೆರೆ-ಗಳ
ಹಳ್ಳಿ-ಕೆರೆ-ಮಂಡ್ಯ
ಹಳ್ಳಿ-ಕೊಪ್ಪದ
ಹಳ್ಳಿ-ಗಲ
ಹಳ್ಳಿ-ಗಳ
ಹಳ್ಳಿ-ಗ-ಳನ್ನು
ಹಳ್ಳಿ-ಗ-ಳನ್ನೂ
ಹಳ್ಳಿ-ಗ-ಳಲ್ಲಿ
ಹಳ್ಳಿ-ಗಳಲ್ಲಿದ್ದ
ಹಳ್ಳಿ-ಗಳ-ಹೆ-ಸರಿ-ಸಿದೆ
ಹಳ್ಳಿ-ಗ-ಳಾಗಿ-ರ-ಬ-ಹುದು
ಹಳ್ಳಿ-ಗ-ಳಾಗಿ-ರ-ಬಹು-ದೆಂದು
ಹಳ್ಳಿ-ಗ-ಳಾಗಿವೆ
ಹಳ್ಳಿ-ಗ-ಳಿಂದ
ಹಳ್ಳಿ-ಗಳಿಗೂ
ಹಳ್ಳಿ-ಗ-ಳಿಗೆ
ಹಳ್ಳಿ-ಗಳಿದ್ದು
ಹಳ್ಳಿ-ಗಳಿವೆ
ಹಳ್ಳಿ-ಗಳು
ಹಳ್ಳಿ-ಗಳೂ
ಹಳ್ಳಿ-ಗಳೆಲ್ಲಾ
ಹಳ್ಳಿ-ಗಳೊಡ-ಗೂಡಿದ್ದ
ಹಳ್ಳಿಗೂ
ಹಳ್ಳಿಗೆ
ಹಳ್ಳಿ-ಗೊಡಗೆಯ
ಹಳ್ಳಿಯ
ಹಳ್ಳಿ-ಯನ್ನು
ಹಳ್ಳಿ-ಯನ್ನು-ಹೊನ್ನೇ-ನ-ಹಳ್ಳಿ
ಹಳ್ಳಿ-ಯನ್ನೂ
ಹಳ್ಳಿ-ಯನ್ನೇ
ಹಳ್ಳಿ-ಯಲ್ಲಿ
ಹಳ್ಳಿ-ಯಲ್ಲೂ
ಹಳ್ಳಿ-ಯಲ್ಲೇ
ಹಳ್ಳಿ-ಯ-ವರು
ಹಳ್ಳಿ-ಯಾ-ಗಿತ್ತು
ಹಳ್ಳಿ-ಯಾಗಿದೆ
ಹಳ್ಳಿ-ಯಿದ್ದು
ಹಳ್ಳಿಯು
ಹಳ್ಳಿಯೇ
ಹಳ್ಳಿ-ಸೀಮೆ-ಯಾಗಿ
ಹಳ್ಳಿ-ಹಳ್ಳಿ-ಗ-ಳಲ್ಲಿ
ಹಳ್ಳೆ-ಗೆರೆ
ಹಳ್ಳೆ-ಗೆರೆ-ಯನ್ನು
ಹವ-ಣಿಕೆ-ಯಲ್ಲಿದ್ದನು
ಹವಣಿಸಿ
ಹವಣಿಸಿ-ದನು
ಹವಣಿಸಿ-ದಾಗ
ಹವದೆಡೆಗಾ-ಗಳುಂ
ಹವಾ-ಲಿಗೆ
ಹವಾಲಿಸಿ
ಹವಾಲಿಸಿ-ಕೊಟ್ಟು
ಹವಾಲಿಸಿ-ಕೊಡುತ್ತಾನೆ
ಹವಾಲಿಸಿ-ಕೊಡುತ್ತಾರೆ
ಹವೆ-ಯನ್ನು
ಹವ್ಯಾಸವೂ
ಹಸಗಾವಿ
ಹಸಿತ
ಹಸಿಯಪ್ಪಂಗೆ
ಹಸು
ಹಸು-ಗಳ
ಹಸು-ಗ-ಳನ್ನು
ಹಸೆ-ಮಣೆ-ಯಲ್ಲಿ
ಹಸೆಯ
ಹಸೆ-ಯೊಳು
ಹಸೆ-ಯೊಳ್
ಹಸ್ತ
ಹಸ್ತಅ
ಹಸ್ತ-ಕು-ಸಳ-ತೆ-ಯೊಳು
ಹಸ್ತ-ಕೌಸಲ್ಯಂ
ಹಸ್ತ-ಗಳ
ಹಸ್ತ-ದಲ್ಲಿ-ರುವ
ಹಸ್ತ-ದಿಂದ
ಹಸ್ತಾಂತ-ರಿ-ಸ-ಲಾ-ಯಿತು
ಹಸ್ತಾಕ್ಷರ-ಗಳ
ಹಸ್ತಾಕ್ಷರ-ವಿದೆ
ಹಸ್ತಾಕ್ಷ-ರ-ವಿದ್ದು
ಹಸ್ತಾಕ್ಷ-ರವಿ-ರ-ಬ-ಹುದು
ಹಸ್ತಾಕ್ಷ-ರವೇ
ಹಸ್ತಿ-ಶೈಲೇಂದ್ರ-ಮಹಾತ್ಮೆ-ಯನ್ನು
ಹಸ್ತೇನ
ಹಾಕ-ಬ-ಹುದು
ಹಾಕ-ಲಾಗಿತ್ತೇ
ಹಾಕ-ಲಾಗಿದೆ
ಹಾಕಲಾಯಿ-ತೆಂದು
ಹಾಕಲು
ಹಾಕಸ-ಲಾಗಿದೆ
ಹಾಕಿ
ಹಾಕಿ-ಕೊಂಡಿದ್ದು
ಹಾಕಿ-ಕೊಂಡು
ಹಾಕಿ-ಕೊಟಿದ್ದಾರೆ
ಹಾಕಿ-ಕೊಟ್ಟ
ಹಾಕಿ-ಕೊಟ್ಟಂತೆ
ಹಾಕಿ-ಕೊಟ್ಟನು
ಹಾಕಿ-ಕೊಟ್ಟ-ನೆಂದು
ಹಾಕಿ-ಕೊಟ್ಟ-ನೆಂದೂ
ಹಾಕಿ-ಕೊಟ್ಟರು
ಹಾಕಿ-ಕೊಟ್ಟ-ರೆಂದು
ಹಾಕಿ-ಕೊಟ್ಟ-ಳೆಂದು
ಹಾಕಿ-ಕೊಟ್ಟ-ಹಾಗೆ
ಹಾಕಿ-ಕೊಟ್ಟಿದ್ದ
ಹಾಕಿ-ಕೊಟ್ಟಿದ್ದ-ನೆಂದು
ಹಾಕಿ-ಕೊಟ್ಟಿದ್ದಾನೆ
ಹಾಕಿ-ಕೊಟ್ಟಿರ-ಬಹು-ದೆಂದು
ಹಾಕಿ-ಕೊಟ್ಟಿ-ರುವ
ಹಾಕಿ-ಕೊಟ್ಟಿ-ರುವುದು
ಹಾಕಿ-ಕೊಟ್ಟು
ಹಾಕಿ-ಕೊಡ-ಲಾಗಿದೆ
ಹಾಕಿ-ಕೊಡ-ಲಾಗುತ್ತಿತ್ತು
ಹಾಕಿ-ಕೊಡುತ್ತಾನೆ
ಹಾಕಿ-ಕೊಡುತ್ತಾರೆ
ಹಾಕಿ-ಕೊಡುತ್ತಾಳೆ
ಹಾಕಿ-ಕೊಡುತ್ತಿದ್ದ
ಹಾಕಿ-ಕೊಳ್ಳೂವ
ಹಾಕಿದ
ಹಾಕಿ-ದನು
ಹಾಕಿ-ದರೂ
ಹಾಕಿ-ದಾಗ
ಹಾಕಿದೆ
ಹಾಕಿದ್ದ
ಹಾಕಿದ್ದಾನೆ
ಹಾಕಿದ್ದಾರೆ
ಹಾಕಿದ್ದು
ಹಾಕಿ-ರ-ಬಹು-ದಾದ
ಹಾಕಿ-ರುತ್ತಾರೆ
ಹಾಕಿ-ರುವ
ಹಾಕಿ-ರು-ವುದು
ಹಾಕಿ-ಸ-ಲಾಗಿದೆ
ಹಾಕಿ-ಸ-ಲಾಗಿ-ರುತ್ತದೆ
ಹಾಕಿಸಿ
ಹಾಕಿ-ಸಿ-ಕೊಂಡ-ರೆಂದು
ಹಾಕಿ-ಸಿ-ಕೊಂಡು
ಹಾಕಿ-ಸಿ-ಕೊಟ್ಟ
ಹಾಕಿ-ಸಿ-ಕೊಡುತ್ತಾರೆ
ಹಾಕಿ-ಸಿ-ಕೊಳ್ಳು-ವುದು
ಹಾಕಿ-ಸಿದ
ಹಾಕಿ-ಸಿ-ದನ
ಹಾಕಿ-ಸಿ-ದನು
ಹಾಕಿ-ಸಿ-ದ-ರೆಂದು
ಹಾಕಿ-ಸಿದ್ದಾನೆ
ಹಾಕಿ-ಸಿದ್ದಾರೆ
ಹಾಕಿ-ಸಿದ್ದಾಳೆ
ಹಾಕಿ-ಸಿದ್ದು
ಹಾಕಿ-ಸಿ-ರ-ಬ-ಹುದು
ಹಾಕಿ-ಸಿ-ರ-ಬಹು-ದೆಂದು
ಹಾಕಿ-ಸಿ-ರುವ
ಹಾಕಿ-ಸಿ-ರುವ-ವನು
ಹಾಕಿ-ಸಿ-ರುವುದು
ಹಾಕಿ-ಸುತ್ತಾನೆ
ಹಾಕಿ-ಸುತ್ತಾರೆ
ಹಾಕಿ-ಸುತ್ತಾಳೆ
ಹಾಕಿ-ಸುತ್ತಿದ್ದರು
ಹಾಕಿ-ಸುವ
ಹಾಕಿ-ಸು-ವಂತೆ
ಹಾಕುತ್ತಾರೆ
ಹಾಕುತ್ತಿದ್ದ
ಹಾಕುತ್ತಿದ್ದಿರ-ಬ-ಹುದು
ಹಾಕುತ್ತಿದ್ದು-ದನ್ನು
ಹಾಕುತ್ತಿದ್ದುದು
ಹಾಕುತ್ತಿರುತ್ತೇನೆ
ಹಾಕುವ
ಹಾಗ
ಹಾಗಂ
ಹಾಗಕ್ಕೆ
ಹಾಗದ
ಹಾಗ-ಲ-ಹಳ್ಳಿ
ಹಾಗ-ಲ-ಹಳ್ಳಿಯ
ಹಾಗ-ಲ-ಹಳ್ಳಿ-ಯನ್ನು
ಹಾಗ-ಲ-ಹಳ್ಳಿ-ಯಲ್ಲಿ
ಹಾಗ-ಲ-ಹಳ್ಳಿಯು
ಹಾಗ-ವನ್ನು
ಹಾಗ-ವೀಸ-ಪಣ
ಹಾಗ-ವೆ-ರಡು
ಹಾಗ-ವೊಂದು
ಹಾಗ-ವೊಂದೆಱ
ಹಾಗಾ
ಹಾಗಾಗಿ
ಹಾಗಾ-ಗಿಯೇ
ಹಾಗಿದ್ದಲ್ಲಿ
ಹಾಗು
ಹಾಗೂ
ಹಾಗೆ
ಹಾಗೆಯೇ
ಹಾಗೇ
ಹಾಜ-ನಂಬಿ
ಹಾಜ-ನಂಬಿಯು
ಹಾಜರಾಗ-ಬೇಕೆಂದು
ಹಾಜ-ರಾದೆ
ಹಾಜ-ರಿದ್ದ-ನೆಂದು
ಹಾಜರಿದ್ದ-ರೆಂದು
ಹಾಜರಿದ್ದು-ದರ
ಹಾಡನ್ನು
ಹಾಡಿ
ಹಾಡಿ-ಮಂಡಲ
ಹಾಡಿ-ಹೊ-ಗಳಿವೆ
ಹಾಡಿ-ಹೊ-ಗಳುತ್ತದೆ
ಹಾಡು-ಗಾರ
ಹಾಡುತ್ತಿದ್ದ
ಹಾಡುತ್ತಿದ್ದರು
ಹಾಡುತ್ತಿದ್ದ-ರೆಂದು
ಹಾಡುವ
ಹಾಡುವ-ನ-ಕೆರೆ
ಹಾಡುವ-ವರ
ಹಾಡು-ವುದು
ಹಾಡ್ಲೆಯ-ಪುರ
ಹಾಡ್ಲೆಯ-ಪುರದ
ಹಾಥಿದ-ರವಾಜ
ಹಾದ-ನೂರು
ಹಾದರ-ಕುಂಟು
ಹಾದರ-ವಾಗಿಲ
ಹಾದರ-ವಾಗಿ-ಲನ್ನು
ಹಾದರ-ವಾಗಿ-ಲಿ-ನಲ್ಲಿ
ಹಾದರ-ವಾಗಿಲು
ಹಾದರಿ-ವಾಗಿಲ
ಹಾದಿಯಲಿ
ಹಾದಿ-ಯಲ್ಲಿ
ಹಾದಿ-ಯಲ್ಲಿದ್ದ
ಹಾದಿ-ಯಲ್ಲಿ-ರುವ
ಹಾದಿರ-ವಾಗಿ-ಲನ್ನು
ಹಾದಿರ-ವಾಗಿಲು
ಹಾದು-ಹೋಗುತ್ತಿತ್ತೆಂದು
ಹಾನುಂಗಲಯ್ನೂರು-ಗ-ಳನ್ನು
ಹಾನುಂಗಲ್ಲು
ಹಾನುಂಗಲ್ಲು-ಗೊಂಡ
ಹಾನು-ಗಲ್ಲಿನ
ಹಾಯಿ-ಸಲು
ಹಾಯಿಸುತ್ತಿದ್ದರು
ಹಾಯಿ-ಸುವ
ಹಾಯ್ದು
ಹಾರಪ್ಪಂಗಳ
ಹಾರಿಗೆ
ಹಾರುವ
ಹಾರುವ-ಗೆರೆ
ಹಾರುವ-ರೆಂದು
ಹಾರುವಳ್ಳಿ-ಯನ್ನು
ಹಾರುವಳ್ಳಿಯು
ಹಾರುವ-ಹಳ್ಳಿ
ಹಾರುವ-ಹಳ್ಳಿ-ಹಾರೋ-ಹಳ್ಳಿ
ಹಾರ್ಯ-ಗವುಡ
ಹಾರ್ಯ-ಗವು-ಡರ
ಹಾಲತಿ
ಹಾಲ-ತಿಯ
ಹಾಲತಿ-ಯಲ್ಲಿ
ಹಾಲಾಳು
ಹಾಲಿ-ಮೊತ್ತದ
ಹಾಲಿಯ-ಮೊತ್ತದ
ಹಾಲು-ಗಂಗ-ಕೆರೆ
ಹಾಲು-ಗಂಗ-ಕೆರೆಗೆ
ಹಾಲು-ಗಂಗ-ಕೆರೆ-ಯನ್ನು
ಹಾಲೋ-ಜನ
ಹಾಲೋಜ-ಬಮ್ಮೋಜ-ಮಾಚೋಜ
ಹಾಲ್ತಿ
ಹಾಲ್ತಿ-ಯ-ಆಲತಿ
ಹಾಳ-ಬಸ್ತಿ-ಕಟ್ಟೆ
ಹಾಳ-ಹಾಳು-ಇಂದಿನ
ಹಾಳಾ-ಗಿದ್ದ
ಹಾಳಾಗಿದ್ದಾಗ
ಹಾಳಾಗಿ-ರಲು
ಹಾಳಾ-ಯಿತು
ಹಾಳು
ಹಾಳು-ಗೆಡ-ವಿದ್ದ
ಹಾಳು-ಬಸ್ತಿ-ಕಟ್ಟೆ
ಹಾಳು-ಬಿದ್ದಿದೆ
ಹಾಳು-ಮನೆ-ಯೊಂದ-ರಲ್ಲಿ
ಹಾಳೆ-ಗಳ
ಹಾಳೆಯ
ಹಾಳೆ-ಹಳ್ಳಿ
ಹಾಳೆ-ಹಳ್ಳಿಯ
ಹಾವಿನ
ಹಾಸನ
ಹಾಸ-ನ-ಜಿಲ್ಲೆ
ಹಾಸ-ನ-ಸೀಮೆಯ
ಹಾಹನ-ವಾಡಿಯ
ಹಾಹನ-ವಾಡಿ-ಯ-ಹನಿಯಂಬಾಡಿ
ಹಿ
ಹಿಂಡಂ
ಹಿಂಡನ್ನು
ಹಿಂಡು
ಹಿಂತೆಗೆ-ದ-ನೆಂದು
ಹಿಂದಕ್ಕೆ
ಹಿಂದಣ
ಹಿಂದಿಕ್ಕಿ
ಹಿಂದಿದ್ದ
ಹಿಂದಿನ
ಹಿಂದಿನ-ದಿರ-ಬ-ಹುದು
ಹಿಂದಿನ-ವನು
ಹಿಂದಿನ-ವರೆಗೂ
ಹಿಂದಿ-ನಿಂದಲೂ
ಹಿಂದಿ-ನಿಂದಲೇ
ಹಿಂದಿರು-ಗಬೇ-ಕಾಯಿ-ತೆಂದು
ಹಿಂದಿರುಗಿ
ಹಿಂದಿರು-ಗಿ-ದನು
ಹಿಂದಿರು-ಗಿ-ದರು
ಹಿಂದಿರು-ಗಿ-ದ-ರೆಂದೂ
ಹಿಂದಿರು-ಗಿ-ರ-ಬ-ಹುದು
ಹಿಂದಿರು-ಗಿಸಿ
ಹಿಂದಿರು-ಗಿ-ಸಿ-ದರು
ಹಿಂದಿರು-ಗು-ತಾರೆ
ಹಿಂದಿರು-ಗುತ್ತಿದ್ದ
ಹಿಂದಿರು-ಗುವ
ಹಿಂದಿರುತ್ತಿರುಗುತ್ತಿವೆ
ಹಿಂದಿರುವ
ಹಿಂದುಮುಂದಾಗಿದೆ
ಹಿಂದುಮುಂದಾ-ಯಿತು
ಹಿಂದೂ
ಹಿಂದೂ-ಗ-ಳಾಗಿದ್ದ-ರೆಂದು
ಹಿಂದೂ-ಗಳು
ಹಿಂದೂ-ಧರ್ಮಕ್ಕೆ
ಹಿಂದೂ-ಪುರ
ಹಿಂದೂ-ಮುಸ್ಲಿಂ
ಹಿಂದೂ-ರಾಯ
ಹಿಂದೂ-ರಾಯ-ಸುರತ್ರಾಣ
ಹಿಂದೂಸ್ಥಾನ-ದಲ್ಲಿ
ಹಿಂದೂಸ್ಥಾನಿ
ಹಿಂದೆ
ಹಿಂದೆ-ಗೆದು
ಹಿಂದೆಯೂ
ಹಿಂದೆಯೇ
ಹಿಂದೆ-ಹಬ್ಬಿ
ಹಿಂಬ-ದಿಯ
ಹಿಂಭಾಗ-ದಲ್ಲಿ
ಹಿಂಭಾಗ-ದಲ್ಲಿಯೇ
ಹಿಂಮ್ಯೇಳ
ಹಿಂಸಾತ್ಮಕ-ವಾದ
ಹಿಜರಿ
ಹಿಜಾಜ್
ಹಿಡಿತ-ದಿಂದ
ಹಿಡಿದ
ಹಿಡಿದಂತೆ
ಹಿಡಿದನು
ಹಿಡಿದ-ನೆಂದೂ
ಹಿಡಿದರೆ
ಹಿಡಿದಾಗ
ಹಿಡಿದಿಟ್ಟು
ಹಿಡಿದಿಡುತ್ತಿದ್ದರು
ಹಿಡಿದಿ-ರುವ
ಹಿಡಿದಿ-ರುವ-ವರ
ಹಿಡಿದಿ-ರುವು-ದ-ರಿಂದ
ಹಿಡಿದು
ಹಿಡಿದು-ಕೊಂಡಿದ್ದರು
ಹಿಡಿದು-ದಕ್ಕೆ
ಹಿಡಿ-ಯದೇ
ಹಿಡಿ-ಯಿತು
ಹಿಡಿಯುತ್ತಾನೆ
ಹಿಡಿಯುತ್ತಿದ್ದರು
ಹಿಡಿಯುವ
ಹಿಡಿವ
ಹಿಡಿಸಿ
ಹಿಡಿ-ಸಿತು
ಹಿತ-ಕರ-ವಾದ
ಹಿತ-ದೃಷ್ಟಿ-ಯಿಂದಲೋ
ಹಿತ-ರಕ್ಷಣೆ-ಗಾಗಿಯೋ
ಹಿತ-ವನ್ನೇ
ಹಿತವೇ
ಹಿತೇ
ಹಿತ್ತಪೊತ್ತರ
ಹಿತ್ತ-ಲು-ಗಳ
ಹಿತ್ತಾಳೆ
ಹಿತ್ತಾಳೆಯ
ಹಿದು-ವನ-ಕೆರೆಯ
ಹಿನಾಂಶು
ಹಿನ್ನೀರಿನ
ಹಿನ್ನೀರಿನಲ್ಲಿ
ಹಿನ್ನೆಯ-ಲೆ-ಯಲ್ಲಿ
ಹಿನ್ನೆಲೆ
ಹಿನ್ನೆಲೆ-ಗಳೊಡನೆ
ಹಿನ್ನೆಲೆಯ
ಹಿನ್ನೆಲೆ-ಯನ್ನು
ಹಿನ್ನೆಲೆ-ಯಲ್ಲಿ
ಹಿನ್ನೆಲೆ-ಯಲ್ಲಿಯೇ
ಹಿನ್ನೆಲೆ-ಯಾ-ಗಿಟ್ಟು-ಕೊಂಡು
ಹಿಮದಿಂ
ಹಿಮವದ್
ಹಿಮಾಚ-ಳ-ತ-ನುಜಾತೆ
ಹಿಮಾ-ಲ-ಯದ
ಹಿಮ್ಮೆಟ್ಟಿಸಿ
ಹಿರಣ್ಣಯ್ಯನ
ಹಿರಣ್ಣಯ್ಯ-ನ-ವರ
ಹಿರಣ್ಯ-ಗರ್ಭ
ಹಿರ-ಯೂರ
ಹಿರಿ
ಹಿರಿ-ಕಳಲೆ
ಹಿರಿ-ಕಳ-ಲೆಯ
ಹಿರಿ-ಕೆರೆ
ಹಿರಿ-ಕೆರೆ-ಯನ್ನು
ಹಿರಿ-ಕೊಂಡ-ರಾಜ
ಹಿರಿ-ತನ-ದಲ್ಲಿ
ಹಿರಿ-ತನ-ದಿಂದ
ಹಿರಿ-ದಾದ
ಹಿರಿದು
ಹಿರಿ-ದೇವ
ಹಿರಿ-ಮಂಡ-ಳಿಕ-ಮಾನ
ಹಿರಿ-ಮೆ-ಗ-ಳನ್ನು
ಹಿರಿಯ
ಹಿರಿ-ಯ-ಅಗ್ರ-ಹಾರ
ಹಿರಿ-ಯ-ಅಡವೆ-ಹಿರೋಡೆ
ಹಿರಿ-ಯ-ಅರ-ಸನ
ಹಿರಿ-ಯ-ಕಂನೆಯ-ನ-ಹಳ್ಳಿ
ಹಿರಿ-ಯ-ಕಟ್ಟ-ಣ-ಗೆರೆ
ಹಿರಿ-ಯ-ಕಳ-ಲೆಯ
ಹಿರಿ-ಯ-ಕೆರೆ
ಹಿರಿ-ಯ-ಕೆರೆ-ಗ-ಳನ್ನು
ಹಿರಿ-ಯ-ಕೆರೆಗೆ
ಹಿರಿ-ಯ-ಕೆರೆಯ
ಹಿರಿ-ಯ-ಕೆರೆ-ಯ-ಕೆಳಗೆ
ಹಿರಿ-ಯ-ಕೆರೆ-ಯನ್ನು
ಹಿರಿ-ಯ-ಚಾಮ-ರಸ
ಹಿರಿ-ಯ-ಜೀಯ
ಹಿರಿ-ಯ-ಜೆಟ್ಟಿಗ
ಹಿರಿ-ಯ-ಜೆಟ್ಟಿಗ-ವನ್ನು
ಹಿರಿ-ಯಣ್ಣ
ಹಿರಿ-ಯ-ತಮ್ಮನ
ಹಿರಿ-ಯ-ದಂಡ-ನಾಯಕ
ಹಿರಿ-ಯ-ದಂಡ-ನಾಯಕಂ
ಹಿರಿ-ಯ-ದಂಡ-ನಾಯ-ಕ-ನಾಗಿದ್ದನು
ಹಿರಿ-ಯ-ದೇವ
ಹಿರಿ-ಯ-ದೇವಂ
ಹಿರಿ-ಯ-ದೇವನು
ಹಿರಿ-ಯ-ನನ್ನು
ಹಿರಿ-ಯ-ನೀರ-ಗುಂದ
ಹಿರಿ-ಯಪ್ಪನ
ಹಿರಿ-ಯಪ್ಪನಿಂದ
ಹಿರಿ-ಯಪ್ಪನು
ಹಿರಿ-ಯಪ್ರಧಾನ
ಹಿರಿ-ಯ-ಬಯಿಚಪ್ಪ
ಹಿರಿ-ಯ-ಬಲ್ಲಾಳ
ಹಿರಿ-ಯ-ಬೆಟ್ಟದ
ಹಿರಿ-ಯ-ಬೆಟ್ಟ-ದ-ಚಾಮ-ರಾಜನು
ಹಿರಿ-ಯ-ಭಂಡಾರದ
ಹಿರಿ-ಯ-ಭಂಡಾರಿ
ಹಿರಿ-ಯ-ಭಂಡಾರಿ-ಯಾಗಿ
ಹಿರಿ-ಯ-ಭೇರುಂಡನ
ಹಿರಿ-ಯ-ಮಗ
ಹಿರಿ-ಯ-ಮರಳಿ
ಹಿರಿ-ಯ-ಮರ-ಳಿ-ಇಂದಿನ
ಹಿರಿ-ಯ-ಮ-ರಿಯಾನೆ
ಹಿರಿ-ಯ-ಮಾಚ
ಹಿರಿ-ಯರ
ಹಿರಿ-ಯ-ರಸ
ಹಿರಿ-ಯ-ರ-ಸ-ಕೆರೆ
ಹಿರಿ-ಯ-ರ-ಸ-ನ-ಕೆರೆ
ಹಿರಿ-ಯ-ರ-ಸ-ನ-ಕೆರೆಯ
ಹಿರಿ-ಯ-ರ-ಸಿನ
ಹಿರಿ-ಯ-ರ-ಸುತ-ನ-ವನ್ನು
ಹಿರಿ-ಯ-ರಾದ
ಹಿರಿ-ಯ-ರಾದ-ವರು
ಹಿರಿ-ಯರು
ಹಿರಿ-ಯ-ವೊಡವಿಂದ
ಹಿರಿ-ಯ-ಹಡ-ವಳ
ಹಿರಿ-ಯ-ಹಡೆ-ವಳ
ಹಿರಿ-ಯ-ಹಳ್ಳ-ಗಳ
ಹಿರಿ-ಯ-ಹಳ್ಳಿ-ಗಳೇ
ಹಿರಿ-ಯ-ಹೆಗ್ಗಡೆ
ಹಿರಿ-ಯೀರೇ-ಗೌಡ
ಹಿರಿ-ಯೂರ
ಹಿರಿ-ಯೂರಿನ
ಹಿರಿ-ಯೂರು
ಹಿರಿ-ವೋಡೆ
ಹಿರಿ-ಸಾವೆ
ಹಿರೀ-ಕಳಲೆ
ಹಿರೆ-ಕೊಲೆ
ಹಿರೆಜಂತಕಲ್
ಹಿರೆ-ಮರ-ಳಿಯ
ಹಿರೆ-ಮರ-ಳಿ-ಯಲ್ಲಿ
ಹಿರೇಜಟ್ಟಿಗ
ಹಿರೇ-ಬೆಟ್ಟದ
ಹಿರೇ-ಮ-ಗಳೂರು
ಹಿರೇ-ಮಠ್ರ-ವರ
ಹಿರೇ-ಮರಳಿ
ಹಿರೇ-ಮರ-ಳಿಯ
ಹಿರೋಡೆ
ಹಿರೋಡೆಗೆ
ಹಿರೋಡೆಯ
ಹಿರೋಡೆ-ಯನ್ನು
ಹಿಲಿಯೋಡೆ-ರ-ಸನು
ಹಿಳ-ಪಲ್ಲಿ
ಹಿಳ-ಪಲ್ಲಿ-ಹಿರಳ-ಹಳ್ಳಿ
ಹಿಳ್ಳ-ಹಳ್ಳಿ
ಹೀಗಾಗಿ
ಹೀಗಿದೆ
ಹೀಗಿದ್ದಾಗ
ಹೀಗಿವೆ
ಹೀಗೂ
ಹೀಗೆ
ಹೀಗೆಲ್ಲಾ
ಹೀರಿ-ಕೊಂಡು
ಹುಂಗೇನ-ಹಳ್ಳಿ
ಹುಂಚ
ಹುಂಚದ
ಹುಚ್ಚನ-ಹಳ್ಳಿ
ಹುಚ್ಚ-ಮಾರುಡು
ಹುಚ್ಚಮ್ಮ
ಹುಜೂರ್
ಹುಜೂರ್ನಾಯಕ
ಹುಟಿದ
ಹುಟು-ವಳಿ
ಹುಟ್ಟಲು
ಹುಟ್ಟಿ
ಹುಟ್ಟಿಗೆ
ಹುಟ್ಟಿ-ತನು
ಹುಟ್ಟಿದ
ಹುಟ್ಟಿ-ದನು
ಹುಟ್ಟಿ-ದ-ವನು
ಹುಟ್ಟಿ-ದ-ಹಳ್ಳಿ
ಹುಟ್ಟಿ-ದಾಗ
ಹುಟ್ಟಿದೆ
ಹುಟ್ಟಿ-ಬೆಳೆ-ದಿರ-ಬಹು-ದಾದ
ಹುಟ್ಟಿ-ಬೆಳೆದು
ಹುಟ್ಟಿ-ರ-ಬ-ಹುದು
ಹುಟ್ಟಿವೆ
ಹುಟ್ಟುತ್ತದೆ
ಹುಟ್ಟುತ್ತಾ-ನೆಂದು
ಹುಟ್ಟುವ
ಹುಟ್ಟು-ವಂತೆ
ಹುಟ್ಟು-ವಳಿ
ಹುಟ್ಟುವ-ಳಿ-ಗ-ಳನ್ನು
ಹುಟ್ಟುವ-ಳಿಯ
ಹುಟ್ಟುವ-ಳಿ-ಯಿಂದ
ಹುಟ್ಟುವ-ಳಿ-ಯುಳ್ಳ
ಹುಟ್ಟುವ-ಳಿಯೇ
ಹುಡು-ಕಾಟ
ಹುಡುಕಾಡಿ-ದಾಗ
ಹುಡುಕ್ಕಾರ
ಹುಣ-ಗನ-ಕಟ್ಟೆ
ಹುಣಸ-ನಾಯಕ
ಹುಣಸೂರು
ಹುಣ-ಸೆಯ
ಹುಣಸೇಶ್ವರ
ಹುತಾತ್ಮ
ಹುತಾತ್ಮ-ನಾ-ದದ್ದು
ಹುತಾತ್ಮ-ನಾದ-ನೆಂದು
ಹುತಾತ್ಮ-ನಾ-ದು-ದನ್ನು
ಹುತಾತ್ಮರ
ಹುತ್ತ-ನೇ-ರಲು
ಹುತ್ತ-ವನ್ನು
ಹುದವ
ಹುದ್ದೆ
ಹುದ್ದೆ-ಗಳ
ಹುದ್ದೆ-ಗ-ಳನ್ನು
ಹುದ್ದೆ-ಗ-ಳಲ್ಲಿ
ಹುದ್ದೆ-ಗ-ಳಿಗೆ
ಹುದ್ದೆ-ಗಳಿದ್ದುದು
ಹುದ್ದೆ-ಗಳಿಲ್ಲ
ಹುದ್ದೆ-ಗಳು
ಹುದ್ದೆ-ಗ-ಳೆಂದು
ಹುದ್ದೆಗೂ
ಹುದ್ದೆಗೆ
ಹುದ್ದೆ-ಗೇ-ರಿದ್ದು
ಹುದ್ದೆ-ಗೇರಿ-ರುವುದು
ಹುದ್ದೆಯ
ಹುದ್ದೆ-ಯಂತೆ
ಹುದ್ದೆ-ಯನ್ನು
ಹುದ್ದೆ-ಯನ್ನೂ
ಹುದ್ದೆ-ಯನ್ನೋ
ಹುದ್ದೆ-ಯಲ್ಲ
ಹುದ್ದೆ-ಯಲ್ಲಿ
ಹುದ್ದೆ-ಯಲ್ಲಿದ್ದ-ನೆಂದು
ಹುದ್ದೆ-ಯಲ್ಲಿದ್ದು
ಹುದ್ದೆ-ಯಾ-ಗಿತ್ತು
ಹುದ್ದೆ-ಯಾಗಿತ್ತೆಂದು
ಹುದ್ದೆ-ಯಾ-ಗಿದ್ದು
ಹುದ್ದೆ-ಯಾಗಿ-ರ-ಬ-ಹುದು
ಹುದ್ದೆ-ಯಾಗಿ-ರ-ಬಹು-ದು-ಮಹಾಪ್ರಧಾನ
ಹುದ್ದೆ-ಯಾದ
ಹುದ್ದೆ-ಯಿಂದ
ಹುದ್ದೆಯು
ಹುದ್ದೆಯೂ
ಹುದ್ದೆ-ಯೆಂದು
ಹುದ್ದೆಯೇ
ಹುನಮಂತಯ್ಯನ
ಹುಬ್ಬನ-ಹಳ್ಳಿ
ಹುಬ್ಬನ-ಹಳ್ಳಿಯ
ಹುಬ್ಬನ-ಹಳ್ಳಿ-ಯಲ್ಲಿ
ಹುಯಿ-ಸಿದ
ಹುಯ್ಯ-ಲಾಗುತ್ತದೆ
ಹುಯ್ಯಲಾಯಿ-ತೆಂದು
ಹುಯ್ಯ-ಲಿಗೆ
ಹುಯ್ಯಲಿ-ನಲ್ಲಿ
ಹುಯ್ಯಿಸಿ
ಹುರ-ಗಲ-ವಾಡಿ
ಹುರಳಿ
ಹುರ-ಳಿಯ
ಹುರಿದುಂಬಿ-ಸಿದರು
ಹುರು-ಗಲ-ವಾಡಿ
ಹುಲಗೂರ
ಹುಲ-ಗೂರು
ಹುಲಿ
ಹುಲಿ-ಕಲ್ಲ
ಹುಲಿ-ಕಲ್ಲು
ಹುಲಿ-ಕೆರೆ
ಹುಲಿ-ಗಳ
ಹುಲಿ-ಗಳಿದ್ದ
ಹುಲಿ-ಗಳಿಲ್ಲ
ಹುಲಿ-ಗೆರೆ
ಹುಲಿ-ಗೆರೆ-ದೇವರು
ಹುಲಿ-ಗೆರೆಯ
ಹುಲಿ-ಗೆರೆ-ಯಾಲೊಕ್ಕಿ-ಗುಂಡಿ
ಹುಲಿ-ನ-ವನ-ಇಂದಿನ
ಹುಲಿ-ಬ-ಸದಿ
ಹುಲಿ-ಬ-ಸದಿಗೆ
ಹುಲಿ-ಬೇಟೆಯ
ಹುಲಿ-ಮುಖದ
ಹುಲಿ-ಮುಖ-ವ-ನಿಕ್ಕಿ-ಸಿ-ದನು
ಹುಲಿ-ಮೊಗ-ವಾಡ-ವನ್ನು
ಹುಲಿಯ
ಹುಲಿ-ಯ-ಗೌಡ
ಹುಲಿ-ಯ-ಜಂಗುಳಿ
ಹುಲಿ-ಯ-ಜಂಗುಳಿಯ
ಹುಲಿ-ಯ-ನನ್ನು
ಹುಲಿ-ಯನ್ನು
ಹುಲಿಯು
ಹುಲಿ-ಯೂರು-ದುರ್ಗ
ಹುಲಿ-ಯೊಂದು
ಹುಲಿ-ಯೊಡನೆ
ಹುಲಿ-ರಾಯ
ಹುಲಿ-ವನ
ಹುಲಿ-ವಾನ
ಹುಲಿ-ವಾನದ
ಹುಲಿ-ವಾನ-ದಲ್ಲಿ
ಹುಲಿ-ವಾನ-ವನ್ನು
ಹುಲಿ-ವೀಸ
ಹುಲ್ಲಂಬಳ್ಳಿಯ
ಹುಲ್ಲಂಬಳ್ಳಿ-ಯಲ್ಲಿ-ರುವ
ಹುಲ್ಲ-ಕೊಪ್ಪಲು
ಹುಲ್ಲನ್ನು
ಹುಲ್ಲವಂಗಲ
ಹುಲ್ಲವಂಗಲದ
ಹುಲ್ಲವಂಗಲ-ವನ್ನು
ಹುಲ್ಲವಂಗಲ-ವಾಯಿ-ತೆಂದು
ಹುಲ್ಲವಂಗಲ-ವೆಂದು
ಹುಲ್ಲ-ಹಳ್ಳಿ
ಹುಲ್ಲಿಗೆ
ಹುಲ್ಲಿ-ಗೆ-ರೆ-ಪುರ
ಹುಲ್ಲಿ-ಗೆ-ರೆ-ಪುರದ
ಹುಲ್ಲಿ-ಗೆ-ರೆ-ಪುರ-ದಲ್ಲಿ
ಹುಲ್ಲಿನ
ಹುಲ್ಲು
ಹುಲ್ಲು-ಗ-ಳನ್ನು
ಹುಲ್ಲು-ಮೆದೆ
ಹುಲ್ಲು-ಹಣ
ಹುಲ್ಲೇ-ಗಾಲ
ಹುಲ್ಲೇ-ಗಾಲದ
ಹುಲ್ಲೇ-ಗಾಲ-ವನ್ನು
ಹುಳ್ಳ
ಹುಳ್ಳಂಬಳ್ಳಿ
ಹುಳ್ಳಂಬಳ್ಳಿ-ಗ-ಳಲ್ಲಿ
ಹುಳ್ಳಂಬಳ್ಳಿಯ
ಹುಳ್ಳ-ಗಾವುಂಡನ
ಹುಳ್ಳ-ಚಮೂಪ
ಹುಳ್ಳ-ಚಮೂಪನ
ಹುಳ್ಳ-ಚಮೂಪನು
ಹುಳ್ಳನ
ಹುಳ್ಳನು
ಹುಳ್ಳನೂ
ಹುಳ್ಳ-ಮಯ್ಯನೂ
ಹುಳ್ಳಯ್ಯ
ಹುಳ್ಳ-ರಾ-ಜಂಗೆ
ಹುಳ್ಳ-ರಾಜನು
ಹುಳ್ಳೆಯ
ಹುಳ್ಳೆಯ-ನ-ಹಳ್ಳಿ-ಯನು
ಹುಳ್ಳೆಯ-ನಾಯ-ಕನು
ಹುಳ್ಳೆಯ-ಹಳ್ಳಿ
ಹುಳ್ಳೆಯ-ಹಳ್ಳಿ-ಯಲ್ಲಿ
ಹುಳ್ಳೇನ-ಹಳ್ಳಿ
ಹುಳ್ಳೇನ-ಹಳ್ಳಿಗೆ
ಹುಳ್ಳೋ-ಹಳ್ಳಿ-ಹುಳ್ಳೇನ-ಹಳ್ಳಿ
ಹುಸಕೂರು
ಹುಸಗೂರ
ಹುಸೇನ್
ಹುಸೈನ್
ಹುಸ್ಕೂರಿನ
ಹುಸ್ಕೂರಿ-ನಲ್ಲಿ
ಹುಸ್ಕೂರಿನಲ್ಲಿದೆ
ಹುಸ್ಕೂರಿನಲ್ಲಿ-ರುವ
ಹುಸ್ಕೂರು
ಹೂಡಿ
ಹೂಡಿದ
ಹೂಡಿ-ದಾಗ
ಹೂಡಿದ್ದನು
ಹೂಡುತ್ತಿದ್ದ
ಹೂಡುವ
ಹೂಡುವ-ವ-ರಿಗೆ
ಹೂಣ-ರಾಜ-ನಾದ
ಹೂಣರು
ಹೂತು-ಹೋಗಿದ್ದ
ಹೂಯ
ಹೂರದ-ಹಳ್ಳಿ-ಯನ್ನು
ಹೂಲಿ
ಹೂಲಿ-ಕೆರೆಯ
ಹೂಲಿ-ಯ-ಕೆರೆ
ಹೂಳು
ಹೂಳು-ತುಂಬಿದೆ
ಹೂಳು-ಮಣ್ಣು
ಹೂಳು-ವಂತೆ
ಹೂವಿನ
ಹೂವಿನ-ಬಾಗೆ-ಯಲ್ಲಿ
ಹೂವಿ-ನಿಂದ
ಹೃತ್ಕ-ಮಲ
ಹೃದಯ-ಶಲ್ಯ
ಹೃದಯಸ್ಥಂಗಳ್
ಹೃದಯಸ್ಥ-ವಾಗಿದ್ದವು
ಹೃದುವ-ನ-ಕೆರೆಗೂ
ಹೃದುವ-ನ-ಕೆರೆಯ
ಹೃದುವ-ನ-ಕೆರೆ-ಯಲ್ಲಿ
ಹೃಷ್ಟಃ
ಹೆಂಗ-ಸರು
ಹೆಂಗ-ಸಿನ
ಹೆಂಗೊಲೆ
ಹೆಂಗೊಲೆ-ಯಲ್ಲಿ
ಹೆಂಚು-ಗ-ಳನ್ನು
ಹೆಂಡತಿ
ಹೆಂಡ-ತಿಯ
ಹೆಂಡತಿ-ಯ-ರಾದ
ಹೆಂಡತಿ-ಯರು
ಹೆಂಡತಿ-ಯ-ರೆಂದು
ಹೆಂಡತಿ-ಯ-ರೊಡನೆ
ಹೆಂಡತಿ-ಯಾದ
ಹೆಂಡ-ತಿಯು
ಹೆಂಡದ
ಹೆಂಡಿರ
ಹೆಂಡಿ-ರನ್ನು
ಹೆಂಡಿರು
ಹೆಂಣ್ನು-ಮ-ಗ-ಳಿಗೆ
ಹೆಂಣ್ನು-ಮ-ಗಳು-ಯಿದ್ದಡೂ
ಹೆಂದಡೆ
ಹೆಂಬೆಟ್ಟ
ಹೆಂಮ
ಹೆಂಮನ
ಹೆಂಮ-ನ-ಗೌಡನ
ಹೆಂಮ-ನ-ಹಳ್ಳಿ
ಹೆಂಮ-ನಾಜಿ-ಗುಂಮನಂ
ಹೆಂಮನೂ
ಹೆಂಮ-ನೆಂಬು-ವ-ವನ
ಹೆಂಮಯ್ಯಂಗಳಿ-ಯನ್
ಹೆಂಮ-ಹೆಂಮಯ್ಯ
ಹೆಂಮೆಯ
ಹೆಂಮೇಶ್ವರ-ದೇವರು
ಹೆಕ್ಟೇರ್
ಹೆಗಲೆಣೆ-ಯಾಗಿ
ಹೆಗ್ಗಡ-ದೇವ-ನ-ಕೋಟೆ
ಹೆಗ್ಗ-ಡಿಕೆ-ಯಲಿ
ಹೆಗ್ಗ-ಡಿಕೆ-ಯಲ್ಲಿ
ಹೆಗ್ಗಡಿ-ಗಳು
ಹೆಗ್ಗಡಿ-ತಿ-ಯರು
ಹೆಗ್ಗಡೆ
ಹೆಗ್ಗಡೆ-ಗಳ
ಹೆಗ್ಗಡೆ-ಗ-ಳನ್ನು
ಹೆಗ್ಗಡೆ-ಗ-ಳಿಗೆ
ಹೆಗ್ಗಡೆ-ಗಳಿತ್ತು
ಹೆಗ್ಗಡೆ-ಗಳಿದ್ದರು
ಹೆಗ್ಗಡೆ-ಗಳಿದ್ದ-ರೆಂದು
ಹೆಗ್ಗಡೆ-ಗಳು
ಹೆಗ್ಗಡೆ-ಗಳೂ
ಹೆಗ್ಗಡೆ-ಗ-ಳೆಂದೂ
ಹೆಗ್ಗಡೆ-ಗಳೆಲ್ಲರೂ
ಹೆಗ್ಗಡೆಗೆ
ಹೆಗ್ಗಡೆ-ತನ-ದಲ್ಲಿ
ಹೆಗ್ಗಡೆ-ತಿಲೆ-ನಾಯಕ
ಹೆಗ್ಗಡೆ-ದೇವ
ಹೆಗ್ಗಡೆ-ದೇವನ
ಹೆಗ್ಗಡೆ-ಪೆರ್ಗ್ಗಡೆ-ಪೆರಾಳ್ಕೆ
ಹೆಗ್ಗಡೆ-ಮೇ-ಲಾಳಿಕೆ
ಹೆಗ್ಗಡೆಯ
ಹೆಗ್ಗಡೆ-ಯಂತಹ
ಹೆಗ್ಗಡೆ-ಯರ
ಹೆಗ್ಗಡೆ-ಯರು
ಹೆಗ್ಗಡೆ-ಯ-ವರ
ಹೆಗ್ಗಡೆ-ಯ-ವರೆಗೆ
ಹೆಗ್ಗಡೆ-ಯಾ-ಗಿದ್ದ
ಹೆಗ್ಗಡೆ-ಯಾ-ಗಿದ್ದ-ನೆಂದು
ಹೆಗ್ಗಡೆ-ಯಾಗಿ-ರ-ಬ-ಹುದು
ಹೆಗ್ಗಡೆಯು
ಹೆಗ್ಗಪ್ಪ
ಹೆಗ್ಗಪ್ಪ-ಗಳು
ಹೆಗ್ಗೆಡಯೇ
ಹೆಗ್ಗೆಡೆ-ಗಳು
ಹೆಚಾಗಿ
ಹೆಚ್ಚಳ-ವನ್ನು
ಹೆಚ್ಚಾಗಿ
ಹೆಚ್ಚಾ-ಗಿತ್ತು
ಹೆಚ್ಚಾಗಿತ್ತೆಂದು
ಹೆಚ್ಚಾಗಿದೆ
ಹೆಚ್ಚಾ-ಗಿದ್ದು
ಹೆಚ್ಚಾಗಿದ್ದು-ಕೊಂಡು
ಹೆಚ್ಚಾಗಿ-ರುವ
ಹೆಚ್ಚಾಗಿ-ರುವು-ದ-ರಿಂದ
ಹೆಚ್ಚಾಗಿ-ರುವುದು
ಹೆಚ್ಚಾಗಿವೆ
ಹೆಚ್ಚಾ-ಗುತ್ತಾ
ಹೆಚ್ಚಾದ
ಹೆಚ್ಚಾದಂತೆಲ್ಲಾ
ಹೆಚ್ಚಾ-ಯಿತು
ಹೆಚ್ಚಿದೆ
ಹೆಚ್ಚಿನ
ಹೆಚ್ಚಿನ-ದಾ-ಗಿತ್ತು
ಹೆಚ್ಚಿನಪ್ರ-ಮಾಣ-ದಲ್ಲಿ
ಹೆಚ್ಚಿನ-ವರು
ಹೆಚ್ಚಿಸಿ
ಹೆಚ್ಚು
ಹೆಚ್ಚು-ಕಡಿಮೆ
ಹೆಚ್ಚು-ಗಟ್ಟಲೆ
ಹೆಚ್ಚುತ್ತಾ
ಹೆಚ್ಚು-ವರಿ
ಹೆಚ್ಚು-ವರಿ-ಯಾಗಿ
ಹೆಚ್ಛಾಗಿದ್ದವು
ಹೆಜ್ಜಾಜಿ
ಹೆಜ್ಜುಂಕ
ಹೆಜ್ಜುಂಕ-ವನ್ನು
ಹೆಜ್ಜುಂಕವು
ಹೆಡ-ತ-ಲೆಯ
ಹೆಡೆ-ಗಳಿವೆ
ಹೆಣಗಾಡಿ
ಹೆಣ್ಣನ್ನು
ಹೆಣ್ಣಾನೆ-ಗ-ಳನ್ನು
ಹೆಣ್ಣಿನ
ಹೆಣ್ಣು
ಹೆಣ್ಣು-ಗಳ
ಹೆಣ್ಣು-ಗ-ಳನ್ನು
ಹೆಣ್ಣು-ಮಕ್ಕಳ
ಹೆಣ್ಣು-ಮಕ್ಕ-ಳನ್ನು
ಹೆಣ್ಣು-ಮಕ್ಕಳಾದ
ಹೆಣ್ಣು-ಮಕ್ಕಳಿಗೂ
ಹೆಣ್ಣು-ಮಕ್ಕಳಿಗೆ
ಹೆಣ್ಣು-ಮಕ್ಕಳಿಗೇ
ಹೆಣ್ಣು-ಮಕ್ಕಳಿದ್ದರು
ಹೆಣ್ಣು-ಮಕ್ಕಳು
ಹೆಣ್ಣು-ಮಕ್ಕಳೂ
ಹೆಣ್ಣು-ಸೆರೆ
ಹೆತ್ತಗೋನ-ಹಳ್ಳಿ
ಹೆದರಿ
ಹೆದರಿ-ಕೊಂಡು
ಹೆದ್ದ-ರೆ-ವರೆಗಂ
ಹೆದ್ದಾರಿಯ
ಹೆದ್ದೊರೆ-ಯಾದಿ-ಯಾಗಿ
ಹೆಬಾ-ಗಿಲ
ಹೆಬಾಡ-ಗಿಕಯ್ಯನು
ಹೆಬಾ-ರುವ
ಹೆಬ್ಬಕ-ವಾಡಿ-ಯನ್ನು
ಹೆಬ್ಬಟ್ಟದ
ಹೆಬ್ಬಟ್ಟವು
ಹೆಬ್ಬಟ್ಟು
ಹೆಬ್ಬಟ್ಟು-ನಾಡ
ಹೆಬ್ಬನಿ
ಹೆಬ್ಬಳ್ಳದ
ಹೆಬ್ಬಳ್ಳ-ವನ್ನು
ಹೆಬ್ಬಳ್ಳವು
ಹೆಬ್ಬಳ್ಳ-ವೆಂಬ
ಹೆಬ್ಬ-ಹೊ-ಲೆಯ
ಹೆಬ್ಬಾಗಿ-ಲನ್ನು
ಹೆಬ್ಬಾ-ಗಿಲಿ
ಹೆಬ್ಬಾಗಿ-ಲಿ-ನಲ್ಲಿ
ಹೆಬ್ಬಾ-ಗಿಲು
ಹೆಬ್ಬಾ-ಗಿಲು-ಗಳ
ಹೆಬ್ಬಾ-ರನು
ಹೆಬ್ಬಾ-ರನೂ
ಹೆಬ್ಬಾ-ರರ
ಹೆಬ್ಬಾ-ರುವ
ಹೆಬ್ಬಾ-ರುವರ
ಹೆಬ್ಬಾಳ
ಹೆಬ್ಬಾಳು
ಹೆಬ್ಬಾವು
ಹೆಬ್ಬಾವು-ಗಳಿದ್ದವು
ಹೆಬ್ಬಾವು-ಗಳೂ
ಹೆಬ್ಬಿ-ದರ-ವಾಡಿಯ
ಹೆಬ್ಬಿ-ದಿರ
ಹೆಬ್ಬಿ-ದಿರ-ವಾಡಿಯ
ಹೆಬ್ಬಿ-ದಿರ-ವಾಡಿ-ಯಲ್ಲಿ
ಹೆಬ್ಬಿ-ದಿರೂರ್ವಾಡಿ
ಹೆಬ್ಬಿ-ದಿರೂರ್ವಾಡಿ-ಯಲಿ
ಹೆಬ್ಬಿ-ದಿರೂರ್ವಾಡಿ-ಯಲ್ಲಿ
ಹೆಬ್ಬಿ-ದಿರೂರ್ವಾಡಿ-ಯಲ್ಲಿ-ಇದೇ
ಹೆಬ್ಬಿ-ದಿರೂರ್ವಾಡಿ-ಯಾಗಿರ
ಹೆಬ್ಬಿ-ದಿರೂರ್ವಾಡಿಯೇ
ಹೆಬ್ಬಿ-ದಿರೂರ್ವ್ವಾಡಿಯ
ಹೆಬ್ಬೆಟ್ಟದ
ಹೆಬ್ಬೆಟ್ಟು-ನಾಡು
ಹೆಬ್ಬೊಳಲ
ಹೆಬ್ಬೊಳಲ-ಇಂದಿನ
ಹೆಬ್ಬೊ-ಳಲು
ಹೆಮೆಯ-ದಂಡ-ನಾಥ
ಹೆಮ್ಮಣ್ಣ
ಹೆಮ್ಮಣ್ಣ-ನಾಯ-ಕನ
ಹೆಮ್ಮನ-ಹಳ್ಳಿ
ಹೆಮ್ಮನ-ಹಳ್ಳಿಯ
ಹೆಮ್ಮ-ನಿಂದ
ಹೆಮ್ಮಯ್ಯ
ಹೆಮ್ಮಯ್ಯನ
ಹೆಮ್ಮಯ್ಯ-ನನ್ನು
ಹೆಮ್ಮಯ್ಯ-ನೆಂದಿದೆ
ಹೆಮ್ಮರ-ಗಾಲ
ಹೆಮ್ಮವ್ವೆ
ಹೆಮ್ಮವ್ವೆಯ
ಹೆಮ್ಮಾಡಿ-ಯಣ್ಣನು
ಹೆಮ್ಮೆಪಡ-ತಕ್ಕಂತಹ
ಹೆಮ್ಮೆಯ
ಹೆಮ್ಮೆಯ-ನಾಯಕಂ
ಹೆಮ್ಮೆಯ-ನಾಯ-ಕ-ನನ್ನು
ಹೆಮ್ಮೊ-ದಲ
ಹೆರಡೆಗೇ-ತನ
ಹೆರಾಸ್
ಹೆರುಳ-ಹಳ್ಳಿ
ಹೆರ್ಗ್ಗಡೆ
ಹೆರ್ಜ್ಜುಂಕ
ಹೆರ್ಮ್ಮಾಡಿ
ಹೆರ್ಮ್ಮಾಡಿ-ದೇವ-ನೆಂಬು-ವವ-ನಿಗೆ
ಹೆಲೋಜಹಾಲೋಜ-ನೆಂಬು-ವ-ವನು
ಹೆಸರ
ಹೆಸರಂ
ಹೆಸರ-ನೇನೋ
ಹೆಸರನ್ನಿಟ್ಟ-ನೆಂದು
ಹೆಸರನ್ನಿಟ್ಟು
ಹೆಸರನ್ನಿಟ್ಟು-ಕೊಳ್ಳು-ವುದು
ಹೆಸ-ರನ್ನು
ಹೆಸ-ರನ್ನೂ
ಹೆಸ-ರನ್ನೇ
ಹೆಸರಲಿ
ಹೆಸ-ರಲು
ಹೆಸ-ರಲ್ಲ
ಹೆಸರಾಂತ
ಹೆಸ-ರಾಗಿ
ಹೆಸರಾ-ಗಿತ್ತು
ಹೆಸ-ರಾಗಿದ್ದನು
ಹೆಸ-ರಾಗಿದ್ದಿರ-ಬಹು-ದೆಂದು
ಹೆಸ-ರಾ-ಗಿದ್ದು
ಹೆಸ-ರಾಗಿ-ರ-ಬ-ಹುದು
ಹೆಸರಾದ
ಹೆಸರಾ-ಯಿತು
ಹೆಸ-ರಿಂದ
ಹೆಸ-ರಿಗೆ
ಹೆಸರಿಟ್ಟನು
ಹೆಸರಿಟ್ಟ-ನೆಂದೂ
ಹೆಸರಿಟ್ಟರು
ಹೆಸರಿಟ್ಟಿರ-ಬ-ಹುದು
ಹೆಸರಿಟ್ಟು
ಹೆಸರಿಟ್ಟು-ಕೊಂಡು
ಹೆಸರಿಟ್ಟು-ಕೊಳ್ಳುತ್ತಿದ್ದರು
ಹೆಸರಿಟ್ಟು-ಕೊಳ್ಳುತ್ತಿದ್ದರೋ
ಹೆಸರಿ-ಡ-ಲಾ-ಗಿತ್ತು
ಹೆಸರಿ-ಡು-ವು-ದಕ್ಕೆ
ಹೆಸರಿತ್ತು
ಹೆಸರಿತ್ತೆಂದು
ಹೆಸರಿದೆ
ಹೆಸರಿದ್ದರೂ
ಹೆಸ-ರಿದ್ದು
ಹೆಸರಿನ
ಹೆಸರಿ-ನಂತೆ
ಹೆಸರಿ-ನಲಿ
ಹೆಸರಿ-ನಲ್ಲಿ
ಹೆಸರಿ-ನಲ್ಲಿಯೇ
ಹೆಸರಿ-ನಲ್ಲೇ
ಹೆಸರಿ-ನಿಂದ
ಹೆಸರಿ-ನಿಂದಲೇ
ಹೆಸರಿ-ನಿಂದಲ್ಲ
ಹೆಸರಿ-ರ-ಬ-ಹುದು
ಹೆಸ-ರಿ-ರುವ
ಹೆಸ-ರಿಲ್ಲ
ಹೆಸರಿ-ಸ-ಬಹು-ದಾದ
ಹೆಸರಿ-ಸ-ಬ-ಹುದು
ಹೆಸರಿ-ಸ-ಲಾಗಿದೆ
ಹೆಸರಿಸಿ
ಹೆಸರಿ-ಸಿದೆ
ಹೆಸರಿ-ಸಿದ್ದಾರೆ
ಹೆಸರಿ-ಸಿದ್ದು
ಹೆಸರಿ-ಸಿಲ್ಲ
ಹೆಸರಿ-ಸಿವೆ
ಹೆಸರಿ-ಸುತ್ತದೆ
ಹೆಸ-ರಿ-ಸುವ
ಹೆಸರು
ಹೆಸರುಂ
ಹೆಸರು-ಗಳ
ಹೆಸರು-ಗಳನ್ನಿಟ್ಟು-ಕೊಳ್ಳು-ವುದು
ಹೆಸರು-ಗ-ಳನ್ನು
ಹೆಸರು-ಗ-ಳನ್ನೂ
ಹೆಸರು-ಗ-ಳಲ್ಲಿ
ಹೆಸರು-ಗಳಲ್ಲಿ-ರುವ
ಹೆಸರು-ಗ-ಳಲ್ಲೂ
ಹೆಸರು-ಗ-ಳಾಗಿವೆ
ಹೆಸರು-ಗ-ಳಿಂದ
ಹೆಸರು-ಗಳಿದ್ದು
ಹೆಸರು-ಗಳಿವೆ
ಹೆಸರು-ಗ-ಳಿಸಿ-ರ-ಬ-ಹುದು
ಹೆಸರು-ಗಳು
ಹೆಸರು-ಗಳೂ
ಹೆಸರುನ್ನು
ಹೆಸರು-ಬಂದಿದೆ
ಹೆಸರುಳ್ಳ
ಹೆಸರು-ವಾಸಿ-ಯಾಗಿದ್ದಿರ-ಬ-ಹುದು
ಹೆಸರು-ವಾಸಿ-ಯಾ-ದನು
ಹೆಸರೂ
ಹೆಸ-ರೆಂದು
ಹೆಸರೇ
ಹೇಗಾಯಿತೆಂಬು-ದನ್ನು
ಹೇಗಿದ್ದರೂ
ಹೇಗೆ
ಹೇಗೋ
ಹೇಮ-ಕರ್ಮ್ಮ
ಹೇಮ-ಕೂಟ-ವಾಸಿ
ಹೇಮ-ಗಿರಿ
ಹೇಮ-ಗಿರಿಗೆ
ಹೇಮ-ಗಿರಿಯ
ಹೇಮದ
ಹೇಮ-ನನ್ದಿ-ಮುನಿ
ಹೇಮ-ಸೇನ
ಹೇಮಾಂಬಿಕೆ-ಯಿಂದ
ಹೇಮಾದ್ರಿಯೇ
ಹೇಮಾವತಿ
ಹೇಮಾವ-ತಿಯ
ಹೇಮಾಶ್ವ-ದಾನ
ಹೇಮೇಶ್ವರ
ಹೇಮೇಸ್ವರ-ದೇವರ
ಹೇರಿ-ಕೊಂಡು
ಹೇರಿಗೆ
ಹೇರಿನ
ಹೇರಿನಷ್ಟು
ಹೇರು-ಹೊರೆ-ಗ-ಳಿಗೆ
ಹೇರೊಬ್ಬೆ
ಹೇಱಿಗೆ
ಹೇಱಿನ
ಹೇಲೋ-ಜನ
ಹೇಳ
ಹೇಳದೆ
ಹೇಳ-ಬಹದು
ಹೇಳ-ಬಹುದ
ಹೇಳ-ಬಹು-ದಾಗಿದೆ
ಹೇಳ-ಬಹು-ದಾದ
ಹೇಳ-ಬ-ಹುದು
ಹೇಳ-ಬಹು-ದುವೀ
ಹೇಳ-ಬಹು-ದೆಂದು
ಹೇಳ-ಬಹು-ದೆಂದೂ
ಹೇಳ-ಬೇಕಾಗುತ್ತದೆ
ಹೇಳ-ಬೇಕೆಂದರೆ
ಹೇಳ-ಲಾಗಿದೆ
ಹೇಳ-ಲಾಗಿ-ರುತ್ತದೆ
ಹೇಳ-ಲಾಗುತ್ತದೆ
ಹೇಳ-ಲಾಗುತ್ತಿತ್ತು
ಹೇಳ-ಲಾಗು-ವು-ದಿಲ್ಲ
ಹೇಳ-ಲಾದ
ಹೇಳಲು
ಹೇಳ-ವುದು
ಹೇಳ-ಹುದು
ಹೇಳಾ
ಹೇಳಿ
ಹೇಳಿ-ಕಳುಹಿಸಿ-ದ-ನಂತೆ
ಹೇಳಿ-ಕಳುಹಿಸಿ-ರ-ಬ-ಹುದು
ಹೇಳಿಕೆ
ಹೇಳಿ-ಕೆ-ಗ-ಳನ್ನು
ಹೇಳಿ-ಕೆ-ಗ-ಳನ್ನೂ
ಹೇಳಿ-ಕೆ-ಗಳು
ಹೇಳಿ-ಕೆ-ಯನ್ನು
ಹೇಳಿ-ಕೆ-ಯನ್ನೂ
ಹೇಳಿ-ಕೊಂಡಿದ್ದರೂ
ಹೇಳಿ-ಕೊಂಡಿದ್ದಾನೆ
ಹೇಳಿ-ಕೊಂಡಿದ್ದಾರೆ
ಹೇಳಿ-ಕೊಂಡಿದ್ದಾರೆಂದು
ಹೇಳಿ-ಕೊಂಡಿದ್ದು
ಹೇಳಿ-ಕೊಂಡಿರ-ಬ-ಹುದು
ಹೇಳಿ-ಕೊಂಡಿರು-ವುದ-ರಿಂದ
ಹೇಳಿ-ಕೊಂಡಿಲ್ಲ
ಹೇಳಿ-ಕೊಂಡು
ಹೇಳಿ-ಕೊಳ್ಳುತ್ತಾ
ಹೇಳಿ-ಕೊಳ್ಳುತ್ತಾರೆ
ಹೇಳಿ-ಕೊಳ್ಳುತ್ತಿದ್ದರು
ಹೇಳಿ-ಕೊಳ್ಳು-ವು-ದನ್ನು
ಹೇಳಿದ
ಹೇಳಿ-ದನು
ಹೇಳಿ-ದರು
ಹೇಳಿ-ದರೂ
ಹೇಳಿ-ದರ್ತಿಯಿಂ
ಹೇಳಿ-ದಾಗ
ಹೇಳಿ-ದಾರೆ
ಹೇಳಿ-ದು-ದನ್ನು
ಹೇಳಿದೆ
ಹೇಳಿ-ದೆಯೇ
ಹೇಳಿದ್
ಹೇಳಿದ್ದ
ಹೇಳಿದ್ದರು
ಹೇಳಿದ್ದರೂ
ಹೇಳಿದ್ದರೆ
ಹೇಳಿದ್ದಾನೆ
ಹೇಳಿದ್ದಾ-ನೆಂದು
ಹೇಳಿದ್ದಾರೆ
ಹೇಳಿದ್ದಾ-ರೆಂದು
ಹೇಳಿದ್ದಾ-ರೆಂದೂ
ಹೇಳಿದ್ದಾರೋ
ಹೇಳಿದ್ದಾರ್
ಹೇಳಿದ್ದಾಳೆ
ಹೇಳಿದ್ದು
ಹೇಳಿದ್ದೇನೆ
ಹೇಳಿ-ರ-ಬ-ಹುದು
ಹೇಳಿ-ರುವ
ಹೇಳಿ-ರು-ವಂತೆ
ಹೇಳಿ-ರು-ವಂತೆಯೇ
ಹೇಳಿ-ರು-ವು-ದಕ್ಕೆ
ಹೇಳಿ-ರು-ವು-ದನ್ನು
ಹೇಳಿ-ರುವು-ದರ
ಹೇಳಿ-ರು-ವು-ದ-ರಿಂದ
ಹೇಳಿ-ರುವು-ದಿರಂದ
ಹೇಳಿ-ರು-ವು-ದಿಲ್ಲ
ಹೇಳಿ-ರುವುದು
ಹೇಳಿ-ರುವುದೇ
ಹೇಳಿ-ರುವುದೋ
ಹೇಳಿಲ್ಲ
ಹೇಳಿವೆ
ಹೇಳುತಿದ್ದ-ರೆಂದು
ಹೇಳುತ್ತದೆ
ಹೇಳುತ್ತದೆ-ಯಾ-ದರೂ
ಹೇಳುತ್ತವೆ
ಹೇಳುತ್ತಾ
ಹೇಳುತ್ತಾನೆ
ಹೇಳುತ್ತಾರೆ
ಹೇಳುತ್ತಾಳೆ
ಹೇಳುತ್ತಿದ್ದರು
ಹೇಳುತ್ತಿದ್ದರೂ
ಹೇಳುತ್ತಿದ್ದ-ರೆಂದು
ಹೇಳುತ್ತಿದ್ದಿರ-ಬ-ಹುದು
ಹೇಳುತ್ತಿದ್ದುದು
ಹೇಳುತ್ತಿ-ರುವಂತಿದೆ
ಹೇಳುತ್ತಿ-ರುವಷ್ಟ-ರಲ್ಲಿ
ಹೇಳುತ್ತಿಲ್ಲ-ವೆಂದು
ಹೇಳುತ್ತಿವೆಯೇ
ಹೇಳು-ಬ-ಹುದು
ಹೇಳುವ
ಹೇಳು-ವಂತೆ
ಹೇಳುವರು
ಹೇಳುವಲ್ಲಿ
ಹೇಳುವಾಗ
ಹೇಳುವುದಕ್ಕಿಂತ
ಹೇಳು-ವು-ದನ್ನು
ಹೇಳುವು-ದಲ್ಲದೆ
ಹೇಳು-ವು-ದಿಲ್ಲ
ಹೇಳು-ವುದು
ಹೇಳ್ದ-ನರ್ತಿಯಿಂ
ಹೈದ-ರನ
ಹೈದ-ರನು
ಹೈದರಾಬಾ-ದಿನ
ಹೈದರಾಬಾದ್
ಹೈದರಾಲಿ
ಹೈದರಾ-ಲಿಯ
ಹೈದರಾ-ಲಿಯು
ಹೈದರ್
ಹೈದರ್ಅಲಿ
ಹೈದರ್ಅಲಿ-ಖಾನ್
ಹೈದರ್ಅಲಿಯ
ಹೈದರ್ಅಲಿಯು
ಹೈದರ್ನ
ಹೈದರ್ನನ್ನು
ಹೈದರ್ನು
ಹೈದರ್ನೊಂದಿಗೆ
ಹೈಹಯ
ಹೈಹ-ಯರ
ಹೊಂಕುಂದದ
ಹೊಂಗ-ನೂರು
ಹೊಂಗಿ-ಗಾಲುವೆ
ಹೊಂಗೆ
ಹೊಂಗೆ-ತಾಳ
ಹೊಂಡ-ಗ-ಳಲ್ಲಿ
ಹೊಂಡರ-ಬಾಳು
ಹೊಂಡವೂ
ಹೊಂದಲ-ಗೆರೆ
ಹೊಂದಲ-ಗೆರೆಯ
ಹೊಂದಿ
ಹೊಂದಿ-ಕೆ-ಯಾಗ-ದಿ-ರಲು
ಹೊಂದಿ-ಕೊಂಡ
ಹೊಂದಿ-ಕೊಂಡಂತೆ
ಹೊಂದಿ-ಕೊಂಡಂತೆಯೇ
ಹೊಂದಿ-ಕೊಂಡ-ಹಾಗೆ
ಹೊಂದಿ-ಕೊಂಡ-ಹಾಗೇ
ಹೊಂದಿ-ಕೊಂಡಿತ್ತು
ಹೊಂದಿ-ಕೊಂಡಿದೆ
ಹೊಂದಿ-ಕೊಂಡಿದ್ದರೆ
ಹೊಂದಿ-ಕೊಂಡಿದ್ದು
ಹೊಂದಿ-ಕೊಂಡಿರುವ
ಹೊಂದಿ-ಕೊಂಡಿವೆ
ಹೊಂದಿ-ಕೊಳ್ಳು-ವಂತೆ
ಹೊಂದಿತು
ಹೊಂದಿತ್ತು
ಹೊಂದಿತ್ತೆಂದು
ಹೊಂದಿದ
ಹೊಂದಿ-ದನು
ಹೊಂದಿ-ದ-ನೆಂದು
ಹೊಂದಿ-ದ-ನೆಂದೂ
ಹೊಂದಿ-ದರು
ಹೊಂದಿ-ದ-ರೆಂದೂ
ಹೊಂದಿ-ದ-ವರ
ಹೊಂದಿ-ದ-ವರಿ-ರ-ಬೇಕು
ಹೊಂದಿ-ದ-ವರು
ಹೊಂದಿ-ದಾಗ
ಹೊಂದಿದೆ
ಹೊಂದಿದ್ದ
ಹೊಂದಿದ್ದನು
ಹೊಂದಿದ್ದ-ನೆಂದು
ಹೊಂದಿದ್ದನ್ನು
ಹೊಂದಿದ್ದ-ರಿಂದ
ಹೊಂದಿದ್ದ-ರಿಂದಲೇ
ಹೊಂದಿದ್ದರು
ಹೊಂದಿದ್ದರೂ
ಹೊಂದಿದ್ದರೆ
ಹೊಂದಿದ್ದ-ರೆಂದು
ಹೊಂದಿದ್ದ-ವರು
ಹೊಂದಿದ್ದವು
ಹೊಂದಿದ್ದು
ಹೊಂದಿದ್ದುದೇ
ಹೊಂದಿ-ರ-ಬ-ಹುದು
ಹೊಂದಿ-ರ-ಬಹುದೇ
ಹೊಂದಿ-ರ-ಬೇಕು
ಹೊಂದಿ-ರುತ್ತಿದ್ದ
ಹೊಂದಿ-ರುತ್ತಿದ್ದರು
ಹೊಂದಿ-ರುತ್ತಿದ್ದರೆ
ಹೊಂದಿ-ರುತ್ತಿದ್ದ-ರೆಂದು
ಹೊಂದಿ-ರುವ
ಹೊಂದಿ-ರು-ವಂತೆ
ಹೊಂದಿ-ರುವ-ವರು
ಹೊಂದಿ-ರು-ವು-ದನ್ನು
ಹೊಂದಿಲ್ಲದ
ಹೊಂದಿವೆ
ಹೊಂದುತ್ತದೆ
ಹೊಂದುತ್ತಾ
ಹೊಂದುತ್ತಾನೆ
ಹೊಂದುತ್ತಿದ್ದ
ಹೊಂದುತ್ತಿ-ರುವ
ಹೊಂದುವ
ಹೊಂದೊಡರ್ಕ್ಕಣ್ಗೊಳೆ
ಹೊಂನ-ಕ-ಹಳ್ಳಿ
ಹೊಂನನು
ಹೊಂನ-ಯನ-ಹಳ್ಳಿ
ಹೊಂನ-ಯನ-ಹಳ್ಳಿ-ಯನ್ನು
ಹೊಂನಯ್ಯ
ಹೊಂನಯ್ಯ-ನ-ಹಳ್ಳಿ-ಯನ್ನು
ಹೊಂನಲ್ಲದೆ
ಹೊಂನವ್ವೆ
ಹೊಂನಿನ
ಹೊಂನಿ-ಸೆಟ್ಟಿ
ಹೊಂನಿ-ಸೆಟ್ಟಿ-ಯರು
ಹೊಂನಿ-ಸೆಟ್ಟಿಯು
ಹೊಂನು
ಹೊಂನೆಯ
ಹೊಂನೇ-ಹಳ್ಳಿ
ಹೊಂನೊಳಗೆ
ಹೊಂಪುರ
ಹೊಂಬಳಿ
ಹೊಂಬಳಿ-ಗಳು
ಹೊಂಬಳು
ಹೊಇದ
ಹೊಇ-ಸಣ
ಹೊಇ-ಸಳ-ಮಂಡಲೇ
ಹೊಈಸ್ಸಳ-ಸೆಟ್ಟಿ-ವಟ್ಟ
ಹೊಕ್ಕು
ಹೊಗರ-ನಾ-ಡಿಗೆ
ಹೊಗರ್ನಾಡಿನ
ಹೊಗರ್ನಾಡಿನಲ್ಲಿ-ರುವ
ಹೊಗರ್ನ್ನಾಡಿನ
ಹೊಗರ್ನ್ನಾಡು
ಹೊಗಳ-ಲಾಗಿದೆ
ಹೊಗ-ಳಲು
ಹೊಗ-ಳಲುತೀ
ಹೊಗಳಿ-ಕೆಗೆ
ಹೊಗಳಿ-ಕೆ-ಯಲ್ಲಿ
ಹೊಗಳಿ-ಕೊಂಡಿದ್ದಾನೆ
ಹೊಗಳಿದೆ
ಹೊಗಳಿದ್ದಾನೆ
ಹೊಗಳಿದ್ದು
ಹೊಗಳಿ-ರುವುದ-ರಿಂದ
ಹೊಗಳುತ್ತವೆ
ಹೊಗಳುತ್ತಿತ್ತೆಂದು
ಹೊಗಳು-ಭಟ್ಟ-ರಲ್ಲ
ಹೊಗಳು-ಭಟ್ಟ-ರಾಗಿದ್ದ-ರೆಂದು
ಹೊಗೆ
ಹೊಗೆ-ದರೆ-ಯಿಂದ
ಹೊಗೆ-ದೆರೆ
ಹೊಗೆ-ದೆರೆ-ಯಿಂದ
ಹೊಗೆ-ಸೊಪ್ಪಿನ
ಹೊಗೆ-ಹಣ
ಹೊಟ-ಗ-ವುಡನು
ಹೊಡಾ-ಘಟ್ಟ-ಗ-ಳನ್ನು
ಹೊಡುಕೆ-ಕಟ್ಟೆ
ಹೊಡುಕೇ-ಕಟ್ಟ-ಹೊಡೆ-ಘಟ್ಟ
ಹೊಡೆ-ತಕ್ಕೆ
ಹೊಡೆ-ದಟ್ಟಿ-ದನು
ಹೊಡೆದಾಟಕ್ಕೆ
ಹೊಡೆದಾಡಿ-ಕೊಂಡು
ಹೊಡೆದು
ಹೊಡೆದೋ-ಡಿಸಿ
ಹೊಡೆದೋಡಿ-ಸು-ವಲ್ಲಿ
ಹೊಡೆಸಿ
ಹೊಣ-ಕನ-ಹಳ್ಳಿ
ಹೊಣ-ಕನ-ಹಳ್ಳಿಯೋ
ಹೊಣೆ
ಹೊಣೆ-ಕಾರರು
ಹೊಣೆ-ಗಾರಿಕೆ
ಹೊಣೆ-ಯನ್ನು
ಹೊತ್ತಿಗಂತೂ
ಹೊತ್ತಿ-ಗಾ-ಗಲೇ
ಹೊತ್ತಿಗೆ
ಹೊತ್ತಿಗೇ
ಹೊತ್ತಿದ್ದ
ಹೊತ್ತಿನ
ಹೊತ್ತು
ಹೊಥ್ತರೆನ-ವಲ್ಲಸು
ಹೊದ
ಹೊದಕೆ
ಹೊದ-ಕೆ-ಯೆಂಬ
ಹೊದಿಕೆ
ಹೊದಿಸಿ
ಹೊದಿ-ಸಿದ
ಹೊದ್ದಕೆ
ಹೊನ-ಗನ-ಹಳ್ಳಿ
ಹೊನ-ಗನ-ಹಳ್ಳಿಯ
ಹೊನ-ಗನ-ಹಳ್ಳಿಯೋ
ಹೊನ-ಗಾನ-ಹಳ್ಳಿ
ಹೊನ-ಗುಂಟಾ
ಹೊನ-ಗುಂದ
ಹೊನ್
ಹೊನ್ನಂ
ಹೊನ್ನ-ಕಳ-ಸ-ಗ-ಳನ್ನು
ಹೊನ್ನ-ಕಳ-ಸ-ವನ್ನು
ಹೊನ್ನ-ಗೊಂಡ-ನ-ಹಳ್ಳಿಯ
ಹೊನ್ನ-ಗೌಡನು
ಹೊನ್ನನ್ನು
ಹೊನ್ನನ್ನೂ
ಹೊನ್ನ-ಮಾಂಬ
ಹೊನ್ನ-ಮಾಂಬೆಯ
ಹೊನ್ನಮ್ಮ
ಹೊನ್ನಮ್ಮನ
ಹೊನ್ನ-ಯನ-ಹಳ್ಳಿಯು
ಹೊನ್ನಯ್ಯ
ಹೊನ್ನಯ್ಯನ
ಹೊನ್ನಯ್ಯ-ನನ್ನು
ಹೊನ್ನಯ್ಯ-ನಿಂದಲೇ
ಹೊನ್ನಯ್ಯನು
ಹೊನ್ನಲಗಿ
ಹೊನ್ನಲಗಿ-ಸೆಟ್ಟಿ
ಹೊನ್ನಲ-ಗೆರೆ
ಹೊನ್ನಲ-ಗೆರೆಗೆ
ಹೊನ್ನಲ-ಗೆರೆಯ
ಹೊನ್ನಲ-ಗೆರೆ-ಯನ್ನು
ಹೊನ್ನಲ-ಗೆರೆಯು
ಹೊನ್ನಲ-ಗೆರೆಯೇ
ಹೊನ್ನವ್ವೆ
ಹೊನ್ನವ್ವೆಗೆ
ಹೊನ್ನ-ಸೆಟ್ಟಿ
ಹೊನ್ನ-ಸೆಟ್ಟಿಯ
ಹೊನ್ನಹಲಗಿ
ಹೊನ್ನಹಲಗಿ-ಸೆಟ್ಟಿ
ಹೊನ್ನಹಲಗೆ
ಹೊನ್ನಹಲಗೆಯ
ಹೊನ್ನ-ಹಳ್ಳ
ಹೊನ್ನಾಂಬುದಿ
ಹೊನ್ನಾಂಬುಧಿ
ಹೊನ್ನಾ-ಚಾರ್ಯನ
ಹೊನ್ನಾ-ಚಾರ್ಯ್ಯನ
ಹೊನ್ನಾಜಿ
ಹೊನ್ನಾ-ವರ
ಹೊನ್ನಾ-ವರದ
ಹೊನ್ನಾ-ವರ-ದಲ್ಲಿ
ಹೊನ್ನಾ-ವರ-ದಲ್ಲಿದ್ದ
ಹೊನ್ನಾ-ವಾರ
ಹೊನ್ನಾ-ವಾರದ
ಹೊನ್ನಿಗೆ
ಹೊನ್ನಿ-ನಲ್ಲಿ
ಹೊನ್ನಿ-ನೊಳಗೆ
ಹೊನ್ನಿರ-ಬ-ಹುದು
ಹೊನ್ನಿ-ಸೆಟ್ಟಿ
ಹೊನ್ನು
ಹೊನ್ನು-ಗ-ಳನ್ನು
ಹೊನ್ನು-ಗ-ಳಿಗೆ
ಹೊನ್ನು-ಡಿಗೆ
ಹೊನ್ನು-ಡಿಗೆ-ಯೂ-ಹೊನ್ನು-ಡಿಕೆ-ಶಾ-ಸನೋಕ್ತ-ವಾಗಿದೆ
ಹೊನ್ನೂ-ರನ್ನು
ಹೊನ್ನೂರಿನ
ಹೊನ್ನೆಯ
ಹೊನ್ನೆಯ-ನ-ಹಳ್ಳಿ
ಹೊನ್ನೇ
ಹೊನ್ನೇ-ಗೌಡ
ಹೊನ್ನೇ-ಗೌಡನು
ಹೊನ್ನೇ-ನ-ಹಳ್ಳಿ
ಹೊನ್ನೇ-ನ-ಹಳ್ಳಿಯ
ಹೊನ್ನೇ-ನ-ಹಳ್ಳಿ-ಯನ್ನು
ಹೊನ್ನೇ-ನ-ಹಳ್ಳಿ-ಯಲ್ಲಿ
ಹೊನ್ನೇ-ನ-ಹಳ್ಳಿ-ಯಲ್ಲಿ-ರುವ
ಹೊನ್ನೇ-ನ-ಹಳ್ಳಿಯು
ಹೊನ್ನೊಳಗೆ
ಹೊಮ್ಮ
ಹೊಮ್ಮದ
ಹೊಯ-ಸಳ
ಹೊಯಿಕು
ಹೊಯಿದು
ಹೊಯಿಶ-ಣದಾಶಿ
ಹೊಯಿ-ಸಣ
ಹೊಯಿ-ಸಣ-ದೇಶದ
ಹೊಯಿ-ಸಣ-ನಾಡು
ಹೊಯಿ-ಸಣ-ರಾಜ್ಯದ
ಹೊಯಿಸ-ಣಾಭಿಧೇ
ಹೊಯಿಸಲ-ನಾಡಿನ
ಹೊಯಿ-ಸಳ
ಹೊಯಿ-ಸಳ-ದೇವರು
ಹೊಯಿ-ಸಳ-ರಾಜ್ಯದ
ಹೊಯಿ-ಸಳ-ರಾಜ್ಯ-ಲಕ್ಷ್ಮೀಪ್ರಾ-ಕಾರ
ಹೊಯಿ-ಸಳ-ಲೆಂಕ
ಹೊಯಿ-ಸಳೇಶ್ವರ
ಹೊಯಿಸುತ್ತಾನೆ
ಹೊಯ್
ಹೊಯ್ದಡೆ
ಹೊಯ್ದು
ಹೊಯ್ಯು-ವು-ದರ
ಹೊಯ್ಸಣ
ಹೊಯ್ಸಣ-ದೇವ
ಹೊಯ್ಸಣ-ದೇಶದ
ಹೊಯ್ಸಣ-ನಾಡು
ಹೊಯ್ಸಣ-ರಾಯ
ಹೊಯ್ಸಣಾಖ್ಯಸ್ಯ
ಹೊಯ್ಸಲ-ನಾಡ
ಹೊಯ್ಸಲ-ನಾಡು
ಹೊಯ್ಸಲಾಹ್ವಯ-ವತ
ಹೊಯ್ಸಳ
ಹೊಯ್ಸಳಂ
ಹೊಯ್ಸಳ-ಕರ್ನಾಟಕ
ಹೊಯ್ಸಳ-ಕಾಲದ
ಹೊಯ್ಸಳಖ್ಯಾತರಂ
ಹೊಯ್ಸಳ-ದೇವನ
ಹೊಯ್ಸಳ-ದೇವನು
ಹೊಯ್ಸಳ-ದೇವರ
ಹೊಯ್ಸಳ-ದೇವರು
ಹೊಯ್ಸಳ-ದೇವಾ-ಲಯ-ಗಳ
ಹೊಯ್ಸಳ-ದೇಶ
ಹೊಯ್ಸಳ-ದೇಶದ
ಹೊಯ್ಸಳ-ದೇಶ-ವನ್ನು
ಹೊಯ್ಸಳ-ದೇಶೇತ್ವಸ್ಮಿನ್
ಹೊಯ್ಸ-ಳನ
ಹೊಯ್ಸಳ-ನಾಡ
ಹೊಯ್ಸಳ-ನಾಡಾಗಿ
ಹೊಯ್ಸಳ-ನಾ-ಡಿಗೆ
ಹೊಯ್ಸಳ-ನಾಡಿನ
ಹೊಯ್ಸಳ-ನಾಡು
ಹೊಯ್ಸ-ಳನು
ಹೊಯ್ಸಳ-ಮಹಾ-ಸಾಮಂನ್ತ
ಹೊಯ್ಸಳ-ಮಹೀಶ
ಹೊಯ್ಸಳ-ಮಹೀಶ-ರಾಜ್ಯ
ಹೊಯ್ಸ-ಳರ
ಹೊಯ್ಸಳ-ರ-ಕಾಲ-ದಿಂದ
ಹೊಯ್ಸಳ-ರನ್ನು
ಹೊಯ್ಸಳ-ರ-ರಾಜ್ಯಕ್ಕೆ
ಹೊಯ್ಸಳ-ರಲ್ಲಿಯೇ
ಹೊಯ್ಸಳ-ರ-ವರೆಗೆ
ಹೊಯ್ಸಳ-ರಾಜನೇ
ಹೊಯ್ಸಳ-ರಾಜ್ಯ
ಹೊಯ್ಸಳ-ರಾಜ್ಯದ
ಹೊಯ್ಸಳ-ರಾಜ್ಯ-ದಲ್ಲಿ
ಹೊಯ್ಸಳ-ರಾಜ್ಯ-ಪ-ಯೋಜ-ಭಾನು
ಹೊಯ್ಸಳ-ರಾಜ್ಯ-ವನ್ನು
ಹೊಯ್ಸಳ-ರಾಜ್ಯಾಧಿ-ಪತಿ
ಹೊಯ್ಸಳ-ರಿಂದ
ಹೊಯ್ಸಳ-ರಿಗೂ
ಹೊಯ್ಸಳ-ರಿಗೆ
ಹೊಯ್ಸ-ಳರು
ಹೊಯ್ಸಳ-ರೆಂದು
ಹೊಯ್ಸ-ಳರೇ
ಹೊಯ್ಸಳ-ಲೆಂಕ-ನಿಸ್ಸಂಕರುಂ
ಹೊಯ್ಸಳ-ವಂಶದ
ಹೊಯ್ಸಳ-ಸಣ್ನೆ-ನಾಡಾಳ್ವ
ಹೊಯ್ಸಳ-ಸ-ಮುದ್ರ-ಗಳ
ಹೊಯ್ಸಳ-ಸಾಮ್ರಾಜ್ಯ
ಹೊಯ್ಸಳ-ಸಾಮ್ರಾಜ್ಯದ
ಹೊಯ್ಸಳ-ಸೆಟಿ
ಹೊಯ್ಸಳ-ಸೆಟ್ಟಿ
ಹೊಯ್ಸಳ-ಸೆಟ್ಟಿ-ಗೆಅ
ಹೊಯ್ಸಳ-ಸೆಟ್ಟಿಯು
ಹೊಯ್ಸಳ-ಸೆಟ್ಟಿ-ಯೆಂಬ
ಹೊಯ್ಸಳ-ಸೆಟ್ಟಿ-ವಟ್ಟ-ವು-ಪಟ್ಟ
ಹೊಯ್ಸಳಾ-ಚಾರಿಯ
ಹೊಯ್ಸಳಾ-ಚಾರಿಯು
ಹೊಯ್ಸಳಾನ್ವಯ
ಹೊಯ್ಸಳೇಶ್ವರ
ಹೊಯ್ಸಳೇಸ್ವರ
ಹೊಯ್ಸಿಳ
ಹೊಯ್ಸೆಯ
ಹೊಯ್ಸೆಯ-ನಾಯ-ಕನ
ಹೊಯ್ಸೆಯ-ನಾಯ-ಕನು
ಹೊಯ್ಸೊ-ಳಲು
ಹೊರ
ಹೊರಕ್ಕೆ
ಹೊರ-ಗಡೆಗೂ
ಹೊರ-ಗಣ
ಹೊರ-ಗಿಟ್ಟು
ಹೊರ-ಗಿನ
ಹೊರ-ಗಿನ-ವರು
ಹೊರ-ಗಿ-ನಿಂದ
ಹೊರಗು
ಹೊರ-ಗುತ್ತಿಗೆ-ಯಾಗಿ
ಹೊರಗೆ
ಹೊರ-ಗೋಡೆಯ
ಹೊರ-ಗೋಡೆ-ಯನ್ನು
ಹೊರಟ
ಹೊರ-ಟಂತೆ
ಹೊರ-ಟನು
ಹೊರ-ಟರು
ಹೊರ-ಟಾಗ
ಹೊರ-ಟಿದೆ
ಹೊರ-ಟಿದ್ದು
ಹೊರ-ಟಿ-ರುವ
ಹೊರ-ಟಿ-ರುವು
ಹೊರ-ಟಿ-ರುವುದು
ಹೊರ-ಟಿವೆ
ಹೊರಟು
ಹೊರ-ಡಿಸ-ಲಾಗಿದೆ
ಹೊರ-ಡಿಸಿ
ಹೊರ-ಡಿಸಿದ
ಹೊರ-ಡಿಸಿ-ದ-ನೆಂದೂ
ಹೊರ-ಡಿಸಿದ್ದರೂ
ಹೊರ-ಡಿಸಿ-ರ-ಬ-ಹುದು
ಹೊರ-ಡಿಸಿ-ರ-ಬಹು-ದೆಂದು
ಹೊರ-ಡಿಸಿ-ರುವ
ಹೊರ-ಡಿಸುತ್ತಿದ್ದ
ಹೊರ-ಡಿ-ಸುವ
ಹೊರ-ಡುತ್ತದೆ
ಹೊರ-ಡುತ್ತಾನೆ
ಹೊರ-ಡುತ್ತಿದ್ದರು
ಹೊರ-ಡುವ
ಹೊರ-ಡು-ವಂತೆಯೂ
ಹೊರ-ಡುವಾಗ
ಹೊರ-ತಂದ
ಹೊರ-ತಂದಿದ್ದಾರೆ
ಹೊರ-ತಾಗಿ
ಹೊರ-ತಾದ
ಹೊರತು
ಹೊರ-ತು-ಪ-ಡಿಸಿ
ಹೊರ-ತು-ಪಡಿ-ಸಿ-ದರೆ
ಹೊರ-ದಾರಿಯ
ಹೊರ-ದೂ-ಡಲು
ಹೊರ-ದೂಡು-ವಲ್ಲಿ
ಹೊರ-ನಾಡಿನ
ಹೊರ-ಪೌಳಿ
ಹೊರಪ್ರಾ-ಕಾರ-ಗ-ಳಿಂದ
ಹೊರ-ಬಂದರೆ
ಹೊರ-ಬ-ರುವ
ಹೊರ-ಬಿದ್ದು
ಹೊರ-ಬೇಕೇ
ಹೊರ-ಭಾಗ-ದಲ್ಲಿ
ಹೊರ-ಭಿತ್ತಿಯ
ಹೊರ-ಮಾಳ
ಹೊರ-ಲೂ-ಬಾ-ರದು
ಹೊರ-ವ-ಲಯ-ದಲ್ಲಿ
ಹೊರ-ವಲೆ-ನಾಡಿನ
ಹೊರ-ವಾರು
ಹೊರ-ವಾಱು
ಹೊರ-ವೃತ್ತಿಯ
ಹೊರ-ಹೋಗು-ವಂತೆ
ಹೊರಾಡಿ
ಹೊರಾದಾಯ
ಹೊರಿ-ಸಲೂ
ಹೊರುವ
ಹೊಱಗು
ಹೊಲ
ಹೊಲ-ಕುಪ್ಪೆ
ಹೊಲ-ಕುಪ್ಪೆಯ
ಹೊಲ-ಗದ್ದೆ
ಹೊಲ-ಗದ್ದೆ-ಗ-ಳನ್ನು
ಹೊಲ-ಗದ್ದೆ-ಗ-ಳಲ್ಲಿ
ಹೊಲ-ಗನ-ಹಳ್ಳಿ
ಹೊಲ-ಗನ-ಹಳ್ಳಿ-ಯನ್ನು
ಹೊಲ-ಗಾಹು
ಹೊಲ-ಗೆರೆಯ
ಹೊಲತ್ತಿ-ಇಂದಿನ
ಹೊಲತ್ತಿ-ಹಾಲತಿ-ಯಲ್ಲಿದ್ದ
ಹೊಲದ
ಹೊಲ-ದಲು
ಹೊಲ-ದಲ್ಲಿ
ಹೊಲ-ದಲ್ಲಿದೆ
ಹೊಲ-ದಲ್ಲಿದ್ದು
ಹೊಲ-ದಲ್ಲಿ-ರುವ
ಹೊಲ-ದೊಳಗೆ
ಹೊಲ-ವನ್ನು
ಹೊಲಿಯ-ಜಂಗುಲಿ-ಹುಲಿಯ
ಹೊಲೆ
ಹೊಲೆ-ಗಟ್ಟೆ-ಗಳು
ಹೊಲೆ-ಗೆರೆ
ಹೊಲೆ-ಗೆರೆಯ
ಹೊಲೆ-ಗೇರಿ
ಹೊಲೆ-ಗೇರಿಯ
ಹೊಲೆ-ದೆರೆಯ
ಹೊಲೆ-ದೆಱೆ
ಹೊಲೆ-ಮಗ್ಗ
ಹೊಲೆ-ಮಾದಿ-ಗರ
ಹೊಲೆಯ
ಹೊಲೆ-ಯರ
ಹೊಲೆ-ಯ-ರಸ
ಹೊಲೆ-ಯ-ರ-ಸ-ಕ-ವನ
ಹೊಲೆ-ಯ-ರಿಗೆ
ಹೊಲೆ-ಯರು
ಹೊಲೆ-ಸುಂಕ
ಹೊಲ್ಲಿ-ಗನ-ಕಟ್ಟೆ-ಇಂದಿನ
ಹೊಲ್ಲಿ-ಗನ-ಕಟ್ಟೆಯ
ಹೊಳಲ-ಕೆರೆಗೆ
ಹೊಳಲ-ಕೆರೆಯ
ಹೊಳಲ-ಕೆರೆ-ಯಲ್ಲಿ
ಹೊಳಲ-ಕೆರೆ-ಯ-ವರೆಗೆ
ಹೊಳಲ-ಗುಂದ
ಹೊಳಲ-ಯದ
ಹೊಳಲ-ಯ-ನಾಡ
ಹೊಳಲ-ಯ-ನಾಡು-ಗಳು
ಹೊಳಲಿನ
ಹೊಳಲಿಯ
ಹೊಳಲು
ಹೊಳಲೆ-ಯನ
ಹೊಳೆ
ಹೊಳೆಗೆ
ಹೊಳೆ-ನ-ರಸಿ-ಪುರ
ಹೊಳೆ-ನ-ರಸೀ-ಪುರ
ಹೊಳೆಯ
ಹೊಳೆ-ಯ-ಸುಂಕ
ಹೊಳೆಯು
ಹೊಸ
ಹೊಸ-ಒಳಲ
ಹೊಸ-ಕಟ್ಟೆಯ
ಹೊಸ-ಕನ್ನಂಬಾಡಿ
ಹೊಸ-ಕಾಲುವೆಯ
ಹೊಸ-ಕೆರಯೂ
ಹೊಸ-ಕೆರೆ
ಹೊಸ-ಕೆರೆ-ಗಳ
ಹೊಸ-ಕೆರೆಗೆ
ಹೊಸ-ಕೆರೆಯ
ಹೊಸ-ಕೆರೆ-ಯನ್ನೂ
ಹೊಸ-ಕೆರೆಯೂ
ಹೊಸ-ಕೊಳ್ಳಗ್ಗಿರಿ
ಹೊಸ-ಕೋಟೆ
ಹೊಸ-ಕೋಟೆಯ
ಹೊಸ-ಕೋಟೆ-ಯಲ್ಲಿ
ಹೊಸ-ಕೋಟೆಯು
ಹೊಸಗ್ರಾಮದ
ಹೊಸ-ಜಾಗ-ವನ್ನು
ಹೊಸ-ಣ-ಮಂಡ-ಲ-ಧೃತಃರಾಜಶ್ರೀ
ಹೊಸ-ದಾಗಿ
ಹೊಸ-ದುರ್ಗ
ಹೊಸ-ನಾಡು
ಹೊಸ-ಪಟ್ಟಣ
ಹೊಸ-ಪಟ್ಟ-ಣದ
ಹೊಸ-ಪಟ್ಟ-ಣ-ದಲ್ಲಿ
ಹೊಸ-ಪಟ್ಟ-ಣ-ದಿಂದ
ಹೊಸ-ಪಟ್ಟ-ಣ-ವನ್ನು
ಹೊಸ-ಪಟ್ಟ-ಣ-ವನ್ನೇ
ಹೊಸ-ಪಟ್ಟ-ಣ-ವಾಗಿ-ರುವ
ಹೊಸ-ಪಟ್ಟ-ಣ-ವೆಂದಾ-ಯಿತು
ಹೊಸ-ಪಟ್ಟ-ಣ-ವೆಂಬ
ಹೊಸ-ಪಟ್ಟ-ಣವೇ
ಹೊಸ-ಪುರ
ಹೊಸ-ಬಿರುದರ
ಹೊಸ-ಬೂದ-ನೂರಿನ
ಹೊಸ-ಬೂದ-ನೂರು
ಹೊಸ-ಮಲೆ
ಹೊಸ-ಲ-ಹೊಳಲು
ಹೊಸ-ವಳ್ಳಿಯ
ಹೊಸ-ವಾಡದ
ಹೊಸ-ವೀಡು
ಹೊಸ-ವೃತ್ತಿ-ಯಾಗಿ
ಹೊಸ-ವೊಳಲ
ಹೊಸ-ಹಳ್ಳಿ
ಹೊಸ-ಹಳ್ಳಿಗ
ಹೊಸ-ಹಳ್ಳಿ-ಪುರ
ಹೊಸ-ಹಳ್ಳಿಯ
ಹೊಸ-ಹಳ್ಳಿ-ಯನ್ನು
ಹೊಸ-ಹೊಳಲ
ಹೊಸ-ಹೊಳಲಿಗೆ
ಹೊಸ-ಹೊಳಲಿನ
ಹೊಸ-ಹೊಳ-ಲಿ-ನಲ್ಲಿ
ಹೊಸ-ಹೊಳಲಿನಿಂದ
ಹೊಸ-ಹೊಳಲು
ಹೊಸ-ಹೊಳಲೇ
ಹೊಸ-ಹೊಸ
ಹೊಸುರು
ಹೊಸೂರು
ಹೊಸ್ತಾಗಿ
ಹೋಗದೆ
ಹೋಗ-ಬ-ಹುದು
ಹೋಗ-ಬೇಕಾ-ಯಿತು
ಹೋಗ-ಬೇಕು
ಹೋಗ-ಬೇಕೆಂದು
ಹೋಗಬೇಕೆಂಬ
ಹೋಗಲಾ-ಗುತ್ತಿತ್ತೆಂದು
ಹೋಗಲಾಡಿ-ಸಲು
ಹೋಗಲಾ-ಡಿಸಿ-ದ-ರೆಂಬುದು
ಹೋಗಲು
ಹೋಗಲು-ಳ-ವರು
ಹೋಗಳಿದೆ
ಹೋಗಿ
ಹೋಗಿತ್ತೆಂದು
ಹೋಗಿದೆ
ಹೋಗಿದ್
ಹೋಗಿದ್ದ
ಹೋಗಿದ್ದನು
ಹೋಗಿದ್ದ-ನೆಂದು
ಹೋಗಿದ್ದ-ವೆಂದು
ಹೋಗಿದ್ದಾಗ
ಹೋಗಿದ್ದು
ಹೋಗಿ-ಬಂದ-ನೆಂಬುದು
ಹೋಗಿ-ರ-ಬ-ಹುದು
ಹೋಗಿ-ರ-ಬಹು-ದೆಂದು
ಹೋಗಿ-ರಲು
ಹೋಗಿ-ರುವ
ಹೋಗಿವೆ
ಹೋಗು
ಹೋಗುತ್ತದೆ
ಹೋಗುತ್ತ-ದೆಂದು
ಹೋಗುತ್ತದೋ
ಹೋಗುತ್ತಾನೆ
ಹೋಗುತ್ತಾರೆ
ಹೋಗುತ್ತಾರೆಂಬ
ಹೋಗುತ್ತಿತತ್ತು
ಹೋಗುತ್ತಿತ್ತು
ಹೋಗುತ್ತಿತ್ತೆಂದು
ಹೋಗುತ್ತಿದ್ದ
ಹೋಗುತ್ತಿದ್ದರು
ಹೋಗುತ್ತಿದ್ದರೆ
ಹೋಗುತ್ತಿದ್ದ-ರೆಂಉ
ಹೋಗುತ್ತಿದ್ದಾಗ
ಹೋಗುತ್ತಿದ್ದಾರೆ
ಹೋಗುತ್ತಿದ್ದು-ದನ್ನೂ
ಹೋಗುವ
ಹೋಗು-ವಂತೆ
ಹೋಗು-ವಂತೆಯೂ
ಹೋಗು-ವಾಗ
ಹೋಗು-ವುದಕ್ಕಾಗಿ
ಹೋಗು-ವು-ದಾಗಿ
ಹೋಗು-ವುದು
ಹೋತನ-ಡ-ಕೆಯ
ಹೋತ್ತ-ಮನಂ
ಹೋದ
ಹೋದನು
ಹೋದ-ನೆಂದು
ಹೋದ-ನೆಂಬ
ಹೋದರು
ಹೋದರೂ
ಹೋದರೆ
ಹೋದ-ರೆಂದು
ಹೋದಾಗ
ಹೋಬಲಾ
ಹೋಬಳಿ
ಹೋಬಳಿ-ಗಳ
ಹೋಬಳಿ-ಗಳು
ಹೋಬಳಿಗೂ
ಹೋಬಳಿಗೆ
ಹೋಬಳಿಯ
ಹೋಬಳಿ-ಯಂತಹ
ಹೋಬಳಿ-ಯನ್ನು
ಹೋಬಳಿ-ಯಲ್ಲಿ
ಹೋಬಳಿ-ಯಲ್ಲಿದೆ
ಹೋಬಳಿ-ಯಲ್ಲಿದ್ದರೆ
ಹೋಬಳಿ-ಯಾಗಿ
ಹೋಮ
ಹೋಯಿತಂತೆ
ಹೋಯಿತು
ಹೋರಾಟ
ಹೋರಾಟಕ್ಕೆ
ಹೋರಾಟ-ಗಳ
ಹೋರಾಟ-ಗ-ಳನ್ನು
ಹೋರಾಟ-ಗ-ಳಲ್ಲಿ
ಹೋರಾಟ-ಗ-ಳಾದ
ಹೋರಾಟ-ಗ-ಳಾದಾಗ
ಹೋರಾಟ-ಗಳು
ಹೋರಾಟ-ಗಾರ-ರಾ-ಗಿದ್ದ
ಹೋರಾಟದ
ಹೋರಾಟ-ದಲ್ಲಿ
ಹೋರಾಟ-ನಡೆಸಿ
ಹೋರಾಟಲ್ಲಿ
ಹೋರಾಟ-ವನ್ನು
ಹೋರಾಟ-ವಾಗಿ-ರುತ್ತದೆ
ಹೋರಾಟ-ವಿರ-ಬ-ಹುದು
ಹೋರಾಟ-ವೊಂದ-ರಲ್ಲಿ
ಹೋರಾ-ಡದೆ
ಹೋರಾ-ಡಲು
ಹೋರಾಡಿ
ಹೋರಾಡಿದ
ಹೋರಾಡಿ-ದಂತೆ
ಹೋರಾಡಿ-ದ-ನೆಂದು
ಹೋರಾಡಿ-ದ-ನೆಂದೂ
ಹೋರಾಡಿ-ದರು
ಹೋರಾಡಿ-ದರೆ
ಹೋರಾಡಿ-ದ-ವ-ರಲ್ಲಿ
ಹೋರಾಡಿ-ದಾಗ
ಹೋರಾಡಿ-ದು-ದನ್ನು
ಹೋರಾಡಿ-ರ-ಬ-ಹುದು
ಹೋರಾಡಿ-ರುವುದು
ಹೋರಾಡುತ್ತಾ
ಹೋರಾಡುತ್ತಿದ್ದ
ಹೋರಾಡುತ್ತಿದ್ದರು
ಹೋರಾಡುತ್ತಿದ್ದ-ರೆಂದು
ಹೋರಾಡುತ್ತಿದ್ದಾನೆ
ಹೋರಾಡುತ್ತಿದ್ದಾರೆ
ಹೋರಾಡುತ್ತಿ-ರುವಾಗ
ಹೋರಾಡು-ವಲ್ಲಿ
ಹೋರಾ-ಡುವಾಗ
ಹೋರಾಡು-ವುದು
ಹೋರಿಗೆ
ಹೋರಿನಿ
ಹೋರಿನಿ-ದೇವ
ಹೋರಿನಿ-ದೇವ-ನಿಗೆ
ಹೋರಿನೀದೇ-ವೊಡೆ-ರಿಗೆ
ಹೋರಿಸೆ
ಹೋರ್ಡಿ
ಹೋರ್ಷಣಾಹ್ವಯ
ಹೋಲಿಸ-ಬಹು-ದೆಂದು
ಹೋಲಿಸ-ಲಾಗಿದೆ
ಹೋಲಿಸಿ-ಕೊಂಡಿದ್ದಾನೆ
ಹೋಲಿ-ಸಿರು-ವುದ-ರಲ್ಲಿ
ಹೋಲಿಸಿ-ರುವುದು
ಹೋಲುತ್ತದೆ
ಹೋಲುತ್ತ-ದೆಂದು
ಹೋಲುತ್ತ-ದೆಂದೂ
ಹೋಲುತ್ತವೆ
ಹೋಲುತ್ತಿದ್ದು
ಹೋಲ್ಸೇಲ್
ಹೋಸಣ
ಹೋಸಣ-ದೇಶದ
ಹೋಸಣಾಖ್ಯ
ಹೋಸಲ-ನಾಡ
ಹೋಸಲ-ನಾಡಿನ
ಹೋಸಲ-ನಾಡು
ಹೋಸಲ-ನಾಡು-ಹೊಯ್ಸಳ
ಹೋಹರು
ಹೋಹಲಿ
ಹೌಸ್ಗೆ
ಹ್ಯಾರಿಸ್
ಹ್ರಸ್ವ
ಹ್ರಸ್ವ-ರೂಪ
ಹ್ವೈಸಣ
ಹ್ಾಕಿಸು-ವು-ದರ
ೃದು-ಸಂದರ್ಭಂ
್ಲ
ೞೋಡವ-ನಾಯಕ
ೞೋಡವ-ನಾಯ-ಕರು
