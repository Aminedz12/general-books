\sethyphenation{kannada}{
ಅಂಕ
ಅಂಕ-ಗ-ಳಿಗೆ
ಅಂಕ-ಗಳು
ಅಂಕಿ-ತ-ನಾ-ಮ-ವನ್ನೇ
ಅಂಕು-ರ-ವಾ-ಗಿ-ದ್ದವು
ಅಂಕುಶ
ಅಂಕು-ಶ-ವಿ-ಲ್ಲದೇ
ಅಂಖಿ
ಅಂಖಿಯ
ಅಂಗ-ಗ-ಳಲ್ಲೂ
ಅಂಗ-ಲಾ-ಚಿದ
ಅಂಗಳ
ಅಂಗ-ಸಂ-ಸ್ಥೆ-ಯಾ-ಗಿದ್ದ
ಅಂಗಿಯ
ಅಂಗೀ-ಕ-ರಿಸಿ
ಅಂಗೀ-ಕ-ರಿ-ಸಿದ
ಅಂಗೀ-ಕ-ರಿ-ಸು-ವುದು
ಅಂಚನ್ನೆ
ಅಂಚಿ-ನಲ್ಲಿ
ಅಂಚು
ಅಂಚೆ
ಅಂಜಿ
ಅಂಟದೆ
ಅಂಟಿಯೂ
ಅಂತ
ಅಂತಃ-ಕ-ರಣ
ಅಂತಃ-ಕ-ರ-ಣದ
ಅಂತರ
ಅಂತ-ರಂ-ಗದ
ಅಂತ-ರಂ-ಗ-ದಲ್ಲಿ
ಅಂತ-ರಂ-ಗ-ದೊ-ಳಗೆ
ಅಂತ-ರಂ-ಗ-ವನ್ನು
ಅಂತ-ರ-ವಿ-ದ್ಯಾ-ರ್ಥಿ-ನಿ-ಲಯ
ಅಂತ-ರಾಳ
ಅಂತ-ರಾ-ಳ-ದ-ಲೊಂದು
ಅಂತರ್
ಅಂತ-ರ್ಬಾಹ್ಯ
ಅಂತ-ವ-ರಲ್ಲಿ
ಅಂತ-ಸ್ತು-ಗ-ಳೊಂ-ದಿಗೆ
ಅಂತ-ಸ್ಸಂ-ಬಂ-ಧದ
ಅಂತಹ
ಅಂತ-ಹದ್ದು
ಅಂತ-ಹ-ವ-ರಲ್ಲಿ
ಅಂತ-ಹ-ವರು
ಅಂತ-ಹ-ವು-ಗ-ಳಲ್ಲಿ
ಅಂತ-ಹುದು
ಅಂತಾ-ದರೆ
ಅಂತಿಮ
ಅಂತಿ-ಮ-ವಾಗಿ
ಅಂತು
ಅಂತೂ
ಅಂತೆಯೆ
ಅಂತೆಯೇ
ಅಂತೆ-ವಾ-ಸಿ-ಗಳ
ಅಂತೆ-ವಾ-ಸಿ-ಗಳು
ಅಂತೇ-ವಾ-ಸಿ-ಯಾ-ಗುವ
ಅಂತೇ-ವಾ-ಸಿ-ಯಾ-ದದ್ದು
ಅಂತ್ಯ-ಕಾ-ಲ-ದಲ್ಲಿ
ಅಂತ್ಯ-ಗೊಂ-ಡಿತು
ಅಂತ್ಯ-ದಲ್ಲಿ
ಅಂತ್ಯಾ-ಕ್ಷರಿ
ಅಂಥ
ಅಂಥ-ದ-ರಲ್ಲಿ
ಅಂಥದ್ದು
ಅಂಥ-ವರ
ಅಂಥ-ವ-ರನ್ನು
ಅಂಥ-ವ-ರಲ್ಲಿ
ಅಂಥ-ವ-ರಿಗೆ
ಅಂಥ-ವರು
ಅಂಥಹ
ಅಂದ
ಅಂದರೆ
ಅಂದವೋ
ಅಂದಾಗ
ಅಂದಿ-ದ್ದರು
ಅಂದಿನ
ಅಂದಿ-ನಿಂದ
ಅಂದಿ-ನಿಂ-ದಲೇ
ಅಂದು
ಅಂದೂ
ಅಂದೇ
ಅಂದ್ರು
ಅಂದ್ರೆ
ಅಂಧ-ಕಾ-ರ-ವನ್ನು
ಅಂಧೇ
ಅಂಬೋಣ
ಅಂಶ
ಅಂಶ-ಗಳ
ಅಂಶ-ಗಳು
ಅಂಶ-ವನ್ನು
ಅಕ-ರ-ಗ್ರಂ-ಥ-ಗ-ಳನ್ನು
ಅಕ-ಶೇ-ರು-ಕ-ಗಳು
ಅಕ-ಸ್ಮಾತ್
ಅಕಾ-ರ-ಣ-ವಾ-ಗಿಯೇ
ಅಕಾ-ರ್ಯ-ಗ-ಳಿಗೆ
ಅಕಾ-ಲಿ-ಕ-ವಾಗಿ
ಅಕೃ-ತ್ರಿಮ
ಅಕೌಂ-ಟೆಂಟ್
ಅಕ್ಕ-ತಂ-ಗಿ-ಯರು
ಅಕ್ಕನ
ಅಕ್ಕ-ನಿಗೆ
ಅಕ್ಕಿ
ಅಕ್ಕ-ತಂಗಿ
ಅಕ್ರಮ
ಅಕ್ರ-ಮ-ಗ-ಳನ್ನು
ಅಕ್ಷರ
ಅಕ್ಷ-ರ-ಗ-ಳಿಗೆ
ಅಕ್ಷ-ರದ
ಅಕ್ಷ-ರಶಃ
ಅಕ್ಷ-ರಾ-ಭ್ಯಾ-ಸ-ದಿಂದ
ಅಕ್ಷ-ರಾ-ವೃ-ತ್ತಿ-ಯಿಂದ
ಅಖಂಡ
ಅಖಿ-ಲ-ಭಾ-ರತ
ಅಗ-ಣಿ-ತರು
ಅಗತ್ಯ
ಅಗ-ತ್ಯಕ್ಕೆ
ಅಗ-ತ್ಯ-ಗ-ನು-ಗು-ಣ-ವಾಗಿ
ಅಗ-ತ್ಯ-ತೆ-ಗ-ಳನ್ನು
ಅಗ-ತ್ಯ-ವನ್ನು
ಅಗ-ತ್ಯ-ವಾದ
ಅಗ-ತ್ಯ-ವಾ-ದ-ದ್ದನ್ನು
ಅಗ-ತ್ಯ-ವೆಂದು
ಅಗ-ತ್ಯ-ವೆ-ನಿ-ಸು-ವಷ್ಟು
ಅಗ-ಮಿಸಿ
ಅಗ-ರ-ಬತ್ತಿ
ಅಗ-ರ-ಬ-ತ್ತಿಯ
ಅಗರು
ಅಗ-ಲ-ವನ್ನು
ಅಗಲೇ
ಅಗಾಧ
ಅಗಾ-ಧ-ವಾ-ಗಿದೆ
ಅಗು-ಹೋ-ಗು-ಗ-ಳಿಗೆ
ಅಗೋ-ಚರ
ಅಗೌ-ರವ
ಅಗ್ಗದ
ಅಗ್ಗೆರೆ
ಅಗ್ಗೆ-ರೆಗೆ
ಅಗ್ಗೆ-ರೆ-ಮ-ಣ್ಣಿ-ಕೊ-ಪ್ಪಯ
ಅಗ್ಗೆ-ರೆ-ಮ-ನೆ-ತ-ನದ
ಅಗ್ಗೆ-ರೆಯ
ಅಗ್ಗೆ-ರೆ-ಯಲ್ಲಿ
ಅಗ್ಗೇರಿ
ಅಗ್ಗೇರೆ
ಅಗ್ಗೇ-ರೆಗೆ
ಅಗ್ಗೇ-ರೆಯ
ಅಗ್ಗೇ-ರೆ-ಯಲ್ಲಿ
ಅಗ್ನಿ-ಹೋತ್ರಿ
ಅಗ್ರ-ಗ-ಣ್ಯರು
ಅಗ್ರ-ಪಂ-ಕ್ತಿ-ಯಲ್ಲಿ
ಅಗ್ರ-ಮಾ-ನ್ಯ-ರಾದ
ಅಗ್ರ-ಶ್ರೇ-ಣಿಯ
ಅಗ್ರ-ಹಾ-ರದ
ಅಗ್ರ-ಹಾ-ರ-ದ-ಲ್ಲಿ-ರುವ
ಅಗ್ರೇ-ಸ-ರರು
ಅಘಾ-ದ-ವಾದ
ಅಚಲ
ಅಚ-ಲ-ವಾಗಿ
ಅಚ-ಲ-ವಾ-ಗಿತ್ತು
ಅಚಾ-ತು-ರ್ಯ-ದಿಂದ
ಅಚ್ಚರಿ
ಅಚ್ಚ-ರಿ-ಯಲ್ಲ
ಅಚ್ಚ-ಳಿ-ಯದೆ
ಅಚ್ಚು
ಅಚ್ಚು-ಕ-ಟ್ಟಾಗಿ
ಅಚ್ಚು-ಕಾ-ಟ್ಟಾಗಿ
ಅಚ್ಚು-ಮೆ-ಚ್ಚಿನ
ಅಚ್ಚು-ಮೆಚ್ಚು
ಅಜಪಾ
ಅಜ-ಪಾದ
ಅಜ-ಪಾ-ಹೆ-ಸರೇ
ಅಜ-ಮಾಸು
ಅಜಾ-ಮಿ-ತ್ವಾಯ
ಅಜ್ಜ
ಅಜ್ಜನ
ಅಜ್ಜ-ನ-ಮನೆ
ಅಜ್ಞಾನ
ಅಜ್ಞಾ-ನ-ರೂಪ
ಅಜ್ಞಾ-ನಿ-ಹೇಗೆ
ಅಟ್ಟಿದ್ದ
ಅಡ-ಗಿದ್ದು
ಅಡಿ
ಅಡಿ-ಕೆ-ತೋಟ
ಅಡಿ-ಗರ
ಅಡಿ-ಗರು
ಅಡಿಗೆ
ಅಡಿ-ಗೆಯ
ಅಡಿ-ಗೆ-ಯ-ವ-ನಿಗೆ
ಅಡಿ-ಪಾಯ
ಅಡಿ-ಪಾ-ಯ-ವನ್ನು
ಅಡುಗೆ
ಅಡು-ಗೆ-ಯನ್ನು
ಅಡೆ-ತ-ಡೆ-ಗ-ಳನ್ನು
ಅಡೆ-ತ-ಡೆ-ಗಳು
ಅಡೆ-ತ-ಡೆ-ಯಿ-ಲ್ಲದೇ
ಅಡ್ಡ-ಹೆ-ಸ-ರು-ಗಳು
ಅಡ್ಡಿ-ಯಾಗಿ
ಅಡ್ಮಿ-ಷನ್
ಅಣ-ತಿ-ಯಂತೆ
ಅಣಿ-ಗೊ-ಳಿ-ಸಿ-ದರು
ಅಣಿ-ಯಾ-ಗು-ತ್ತಿ-ದ್ದೇನೆ
ಅಣು-ಕ-ಣ-ಗ-ಳಿಂದ
ಅಣ್ಣ
ಅಣ್ಣಂ-ದಿರು
ಅಣ್ಣಂ-ದಿ-ರೆಲ್ಲ
ಅಣ್ಣ-ತ-ಮ್ಮಂ-ದಿ-ರಂತೆ
ಅಣ್ಣ-ತ-ಮ್ಮಂ-ದಿರು
ಅಣ್ಣನ
ಅಣ್ಣ-ನನ್ನು
ಅಣ್ಣ-ನ-ವರ
ಅಣ್ಣ-ನ-ವ-ರಿಗೆ
ಅಣ್ಣನಾ
ಅಣ್ಣ-ನಾ-ದರೂ
ಅಣ್ಣ-ನಾ-ದ-ವನು
ಅಣ್ಣ-ನಿಗೂ
ಅಣ್ಣ-ನಿಗೆ
ಅಣ್ಣಯ್ಯ
ಅಣ್ಣ-ಯ್ಯ-ನಾದೆ
ಅಣ್ಣಾ-ಜೀ-ರಾವ್
ಅಣ್ಣೆರಿ
ಅಣ್ಣ-ತಮ್ಮ
ಅತಿ
ಅತಿ-ಕ್ಲಿಷ್ಟ
ಅತಿ-ಗಳ
ಅತಿಥಿ
ಅತಿ-ಥಿ-ಗ-ಳಿಂದ
ಅತಿ-ಥಿ-ಗ-ಳಿಗೆ
ಅತಿ-ಥಿ-ಧ-ರ್ಮ-ವನ್ನು
ಅತಿ-ಥಿ-ಯಜ್ಞ
ಅತಿ-ಥಿ-ಯಾಗಿ
ಅತಿ-ಥಿ-ಯಾ-ಗಿದ್ದ
ಅತಿ-ಥಿ-ಸ-ತ್ಕಾ-ರ-ವನ್ನೂ
ಅತಿ-ಯಾ-ಗಿತ್ತು
ಅತಿ-ಯಾದ
ಅತಿ-ಯಾಯ್ತು
ಅತಿ-ವಿ-ರ-ಳರು
ಅತಿ-ಶ-ಯ-ವಲ್ಲ
ಅತಿ-ಶ-ಯ-ವ-ಲ್ಲದ
ಅತಿ-ಶ-ಯ-ವಾದ
ಅತಿ-ಶ-ಯವೂ
ಅತಿ-ಶ-ಯೋಕ್ತಿ
ಅತಿ-ಶ-ಯೋ-ಕ್ತಿ-ಯಲ್ಲ
ಅತಿ-ಶ-ಯೋ-ಕ್ತಿ-ಯಾ-ಗ-ಲಾ-ರದು
ಅತಿ-ಶ-ಯೋ-ಕ್ತಿಯೇ
ಅತೀ
ಅತೀವ
ಅತ್ತಿಂ-ದಿತ್ತ
ಅತ್ತಿಗೆ
ಅತ್ತಿ-ಗೆಯ
ಅತ್ತಿ-ಗೆ-ಯ-ರಿಗೆ
ಅತ್ತಿ-ಗೆ-ಯ-ವರ
ಅತ್ತಿ-ಗೆ-ಯೆಂ-ದರೆ
ಅತ್ತಿತ್ತ
ಅತ್ಮೀ-ಯತೆ
ಅತ್ಯಂತ
ಅತ್ಯಂ-ತ-ಪ್ರೀ-ತಿ-ಪಾತ್ರ
ಅತ್ಯ-ಗತ್ಯ
ಅತ್ಯ-ಗ-ತ್ಯ-ವಾಗಿ
ಅತ್ಯ-ದ್ಭು-ತ-ಗ-ಳಲ್ಲಿ
ಅತ್ಯ-ಧಿಕ
ಅತ್ಯಾ-ದ-ರ-ಗ-ಳಿಂದ
ಅತ್ಯಾ-ನಂ-ದ-ವಾ-ಯಿತು
ಅತ್ಯು-ತ್ತಮ
ಅತ್ಯು-ತ್ತ-ಮ-ವಾದ
ಅತ್ಯು-ತ್ಸಾ-ಹ-ದಿಂದ
ಅತ್ರಾರ್ಥಾ
ಅತ್ರಿ
ಅಥವಾ
ಅದ-ಕ್ಕಾಗಿ
ಅದ-ಕ್ಕಾ-ಗಿಯೇ
ಅದ-ಕ್ಕಾ-ಗೊಂದು
ಅದ-ಕ್ಕಿಂತ
ಅದ-ಕ್ಕಿಂ-ತಲೂ
ಅದ-ಕ್ಕಿದು
ಅದಕ್ಕೂ
ಅದಕ್ಕೆ
ಅದ-ಕ್ಕೆಂದೇ
ಅದ-ಕ್ಷ-ರಾ-ಗದೇ
ಅದ-ಕ್ಷರು
ಅದನ್ನ
ಅದ-ನ್ನ-ನು-ಸ-ರಿ-ಸಿ-ದರೆ
ಅದ-ನ್ನ-ರಿ-ಯದೇ
ಅದ-ನ್ನ-ವ-ರಿಗೆ
ಅದನ್ನು
ಅದನ್ನೂ
ಅದ-ನ್ನೆಲ್ಲ
ಅದನ್ನೇ
ಅದಮಿ
ಅದರ
ಅದ-ರಂತೆ
ಅದ-ರಂ-ತೆಯೇ
ಅದ-ರದ್ದೆ
ಅದ-ರಲ್ಲಿ
ಅದ-ರ-ಲ್ಲಿಯೂ
ಅದ-ರ-ಲ್ಲಿಯೇ
ಅದ-ರಲ್ಲು
ಅದ-ರಲ್ಲೂ
ಅದ-ರಲ್ಲೇ
ಅದ-ರ-ಲ್ಲೊಂದು
ಅದ-ರಿಂದ
ಅದರೂ
ಅದರೆ
ಅದಲು
ಅದಾ-ಗಲೇ
ಅದೀ-ರ್ಘ-ವಾಗಿ
ಅದು
ಅದುಮಿ
ಅದು-ಇದು
ಅದೂ
ಅದೃ-ಶ್ಯ-ನಾ-ದ-ನಂತೆ
ಅದೃಷ್ಟ
ಅದೃ-ಷ್ಟ-ದಂತೆ
ಅದೃ-ಷ್ಟವೂ
ಅದೃ-ಷ್ಟ-ವೆಂದೂ
ಅದೃ-ಷ್ಟ-ವೆಂದೇ
ಅದೃ-ಷ್ಟ-ಶಾಲಿ
ಅದೆಂ-ದರೆ
ಅದೆಷ್ಟು
ಅದೆಷ್ಟೇ
ಅದೆಷ್ಟೋ
ಅದೇ
ಅದೇನೂ
ಅದೇ-ನೆಂ-ದರೆ
ಅದೇ-ನೆಂ-ಬುದು
ಅದೇನೇ
ಅದೇನೋ
ಅದೊಂದು
ಅದೊಂದೇ
ಅದ್ದ-ರಿಂದ
ಅದ್ಭುತ
ಅದ್ಭು-ತ-ವಾದ
ಅದ್ಭು-ತವೂ
ಅದ್ಯಾ-ವು-ದಕ್ಕೂ
ಅದ್ವಿ-ತೀಯ
ಅದ್ವಿ-ತೀ-ಯ-ವಾ-ಗಿದೆ
ಅದ್ವೈತ
ಅದ್ವೈ-ತ-ವೇ-ದಾಂತ
ಅದ್ವೈ-ತ-ವೇ-ದಾಂ-ತದ
ಅದ್ವೈ-ತ-ವೇ-ದಾಂ-ತ-ವನ್ನು
ಅಧ-ರಕ್ಕೆ
ಅಧಾ-ರ-ವಿ-ಲ್ಲದೇ
ಅಧಿಕ
ಅಧಿ-ಕ-ವಾ-ಯಿತು
ಅಧಿ-ಕಾ-ದಾಯ
ಅಧಿ-ಕಾರ
ಅಧಿ-ಕಾ-ರಿ-ಗ-ಳನ್ನು
ಅಧಿ-ಕಾ-ರಿ-ಗ-ಳಿಗೆ
ಅಧಿ-ಕಾ-ರಿ-ಗಳು
ಅಧಿ-ಕಾ-ರಿ-ಯಲ್ಲ
ಅಧಿ-ಕೃತ
ಅಧಿ-ಗ-ತ-ತತ್ವಃ
ಅಧಿ-ಗ-ತತ್ವಃ
ಅಧೀತ
ಅಧೀ-ತ-ಕಾ-ವ್ಯ-ವ್ಯಾ-ಕ-ರ-ಣ-ಕೋ-ಶಾ-ದಿ-ಮಾನ್
ಅಧ್ಯಕ್ಷ
ಅಧ್ಯ-ಕ್ಷತೆ
ಅಧ್ಯ-ಕ್ಷ-ತೆ-ಯನ್ನು
ಅಧ್ಯ-ಕ್ಷ-ರದು
ಅಧ್ಯ-ಕ್ಷ-ರ-ನ್ನಾಗಿ
ಅಧ್ಯ-ಕ್ಷ-ರಾ-ಗ-ದಿ-ದ್ದರೂ
ಅಧ್ಯ-ಕ್ಷ-ರಾಗಿ
ಅಧ್ಯ-ಕ್ಷ-ರಾ-ಗಿ-ದ್ದ-ವರು
ಅಧ್ಯ-ಕ್ಷ-ರಿಗೆ
ಅಧ್ಯ-ಕ್ಷರು
ಅಧ್ಯ-ಕ್ಷರೇ
ಅಧ್ಯ-ಕ್ಷೀಯ
ಅಧ್ಯ-ನ-ಮಾ-ಡ-ಬೇ-ಕೆಂಬ
ಅಧ್ಯ-ಯ-ದಲ್ಲಿ
ಅಧ್ಯ-ಯನ
ಅಧ್ಯ-ಯ-ನ-ಅ-ಧ್ಯಾ-ಪನ
ಅಧ್ಯ-ಯ-ನ-ಕ್ಕಾಗಿ
ಅಧ್ಯ-ಯ-ನ-ಕ್ಕಾ-ಗಿಯೇ
ಅಧ್ಯ-ಯ-ನಕ್ಕೂ
ಅಧ್ಯ-ಯ-ನಕ್ಕೆ
ಅಧ್ಯ-ಯ-ನದ
ಅಧ್ಯ-ಯ-ನ-ದಲ್ಲಿ
ಅಧ್ಯ-ಯ-ನ-ಪ್ರೇ-ರಣೆ
ಅಧ್ಯ-ಯ-ನ-ಮಾ-ಡು-ತ್ತಿದ್ದ
ಅಧ್ಯ-ಯ-ನ-ಮಾ-ಡು-ವಾ-ಗಲೇ
ಅಧ್ಯ-ಯ-ನ-ವನ್ನು
ಅಧ್ಯ-ಯ-ನ-ವನ್ನೂ
ಅಧ್ಯ-ಯ-ನ-ವಾ-ಗಿತ್ತು
ಅಧ್ಯ-ಯ-ನ-ವಿ-ಲ್ಲದೆ
ಅಧ್ಯ-ಯ-ನವೂ
ಅಧ್ಯ-ಯ-ನ-ಶೀಲ
ಅಧ್ಯ-ಯ-ನ-ಶೀ-ಲ-ತೆ-ಯಿಂದ
ಅಧ್ಯ-ಯ-ನ-ಶೀ-ಲ-ರಾದ
ಅಧ್ಯ-ಯ-ನ-ಶೀ-ಲರು
ಅಧ್ಯ-ಯ-ನಾ-ಕಾಂ-ಕ್ಷಿ-ಗ-ಳಾಗಿ
ಅಧ್ಯ-ಯ-ನಾ-ಧ್ಯಾ-ಪ-ನ-ಗ-ಳನ್ನು
ಅಧ್ಯ-ಯ-ನಾ-ರ್ಥಿ-ಯಾ-ಗಿ-ದ್ದಾರೆ
ಅಧ್ಯ-ಯ-ನ-ಅ-ಧ್ಯಾ-ಪನ
ಅಧ್ಯ-ಯ-ನ-ಅ-ಧ್ಯಾ-ಪ-ನಕ್ಕೆ
ಅಧ್ಯ-ಯ-ನ-ಅ-ಧ್ಯಾ-ಪ-ನ-ಗ-ಳಿಂದ
ಅಧ್ಯ-ಯ-ನ-ಅ-ಧ್ಯಾ-ಪ-ನದ
ಅಧ್ಯ-ಯ-ನ-ಅ-ಧ್ಯಾ-ಪ-ನ-ವೆಂಬ
ಅಧ್ಯಾ-ಪಕ
ಅಧ್ಯಾ-ಪ-ಕ-ನಾಗಿ
ಅಧ್ಯಾ-ಪ-ಕರ
ಅಧ್ಯಾ-ಪ-ಕ-ರನ್ನು
ಅಧ್ಯಾ-ಪ-ಕ-ರಲ್ಲಿ
ಅಧ್ಯಾ-ಪ-ಕ-ರ-ಹಾಗೂ
ಅಧ್ಯಾ-ಪ-ಕ-ರಾಗಿ
ಅಧ್ಯಾ-ಪ-ಕ-ರಾ-ಗಿ-ದ್ದರು
ಅಧ್ಯಾ-ಪ-ಕ-ರಾ-ಗಿ-ದ್ದಾರೆ
ಅಧ್ಯಾ-ಪ-ಕ-ರಾ-ಗಿದ್ದು
ಅಧ್ಯಾ-ಪ-ಕ-ರಾದ
ಅಧ್ಯಾ-ಪ-ಕ-ರಿ-ರ-ಬ-ಹುದು
ಅಧ್ಯಾ-ಪ-ಕರು
ಅಧ್ಯಾ-ಪ-ಕ-ರೆಂ-ಬುದು
ಅಧ್ಯಾ-ಪ-ಕರೇ
ಅಧ್ಯಾ-ಪ-ಕ-ವೃ-ತ್ತಿಗೆ
ಅಧ್ಯಾ-ಪನ
ಅಧ್ಯಾ-ಪ-ನ-ಕ್ಕೆಂದೇ
ಅಧ್ಯಾ-ಪ-ನಕ್ಕೇ
ಅಧ್ಯಾ-ಪ-ನ-ಕ್ರ-ಮದ
ಅಧ್ಯಾ-ಪ-ನದ
ಅಧ್ಯಾ-ಪ-ನ-ದ-ಲ್ಲಿದ್ದೂ
ಅಧ್ಯಾ-ಪ-ನ-ದ-ಲ್ಲಿಯೂ
ಅಧ್ಯಾ-ಪ-ನ-ದಲ್ಲೂ
ಅಧ್ಯಾ-ಪ-ನ-ದಿಂದ
ಅಧ್ಯಾ-ಪ-ನ-ಮಾಡಿ
ಅಧ್ಯಾ-ಪ-ನ-ರತಿ
ಅಧ್ಯಾ-ಪ-ನ-ವೃ-ತ್ತಿಯ
ಅಧ್ಯಾ-ಪ-ನ-ವೃ-ತ್ತಿ-ಯನ್ನು
ಅಧ್ಯಾ-ಪ-ನವೇ
ಅಧ್ಯಾಯ
ಅಧ್ಯಾ-ಯ-ಗ-ಳುಳ್ಳ
ಅಧ್ಯಾ-ಯದ
ಅಧ್ಯೇ-ಯ-ವಿ-ಷ-ಯ-ದಲ್ಲಿ
ಅನಂತ
ಅನಂ-ತನ
ಅನಂ-ತರ
ಅನಂ-ತ-ರದ
ಅನಂ-ತ-ರ-ವಂತೂ
ಅನಂ-ತ-ರವೂ
ಅನಂ-ದ-ರ-ವರ
ಅನ-ತಿ-ದೂ-ರ-ದ-ಲ್ಲಿತ್ತು
ಅನ-ಧಿ-ಕಾ-ರಿ-ಗ-ಳಿಗೆ
ಅನ-ಧಿ-ಕಾ-ರಿ-ಯಾ-ಗಿಯೇ
ಅನ-ಧಿ-ಕೃತ
ಅನ-ಧೀ-ತ-ತ-ರ್ಕ-ಶಾಸ್ತ್ರಃ
ಅನ-ಧ್ಯ-ಯನ
ಅನನ್ಯ
ಅನ-ಭ್ಯಾಸೇ
ಅನ-ರ್ಘ್ಯ-ರ-ತ್ನ-ವಾ-ಗಿ-ದ್ದಾರೆ
ಅನರ್ಥ
ಅನ-ಲ್ಪಾ-ಕ್ಷ-ರ-ಗ-ಳಿಂದ
ಅನ-ವ-ದ್ಯ-ವಿ-ದ್ಯಾ-ಮ-ಣಿ-ಹಾ-ರ-ಭೂ-ಷಿ-ತ-ರಾಗಿ
ಅನ-ವ-ರ-ತವೂ
ಅನಾ-ಥ-ರನ್ನು
ಅನಾ-ದರ್ಶ
ಅನಾದಿ
ಅನಾ-ನು-ಕೂ-ಲ-ದಿಂ-ದಾಗಿ
ಅನಾ-ಮಿ-ಕದ
ಅನಾ-ವ-ರಣ
ಅನಾ-ವ-ರ-ಣ-ಗೊಂಡ
ಅನಿ-ತರ
ಅನಿ-ಯ-ತ-ವಾ-ಗಿ-ಯಾ-ದರೂ
ಅನಿ-ರೀ-ಕ್ಷಿ-ತ-ವಾಗಿ
ಅನಿ-ವಾರ್ಯ
ಅನಿ-ವಾ-ರ್ಯತೆ
ಅನಿ-ವಾ-ರ್ಯ-ತೆಯ
ಅನಿ-ವಾ-ರ್ಯ-ತೆ-ಯನ್ನು
ಅನಿ-ವಾ-ರ್ಯ-ತೆ-ಯಲ್ಲಿ
ಅನಿ-ವಾ-ರ್ಯ-ವಾಗಿ
ಅನಿ-ವಾ-ರ್ಯ-ವಾ-ಗಿತ್ತು
ಅನಿ-ವಾ-ರ್ಯ-ವಾ-ಗಿದೆ
ಅನಿ-ವಾ-ರ್ಯವೂ
ಅನಿ-ಸಿ-ಕ-ಗ-ಳಿಗೆ
ಅನಿ-ಸಿಕೆ
ಅನಿ-ಸಿ-ಕೆಗೆ
ಅನಿ-ಸಿ-ಕೆ-ಯಂತೆ
ಅನಿ-ಸು-ತ್ತಿದೆ
ಅನು-ಕ-ರ-ಣೀಯ
ಅನು-ಕೂಲ
ಅನು-ಕೂ-ಲಕ್ಕೆ
ಅನು-ಕೂ-ಲ-ಗ-ಳಿವೆ
ಅನು-ಕೂ-ಲ-ವಾ-ಗದೇ
ಅನು-ಕೂ-ಲ-ವಾ-ಗು-ವಂತೆ
ಅನು-ಕೂ-ಲ-ವಾದ
ಅನು-ಗ-ಮ್ಯರು
ಅನು-ಗು-ಣ-ವಾಗಿ
ಅನು-ಗು-ಣ-ವಾ-ಗಿ-ದ್ದಾಳೆ
ಅನು-ಗ್ರಹ
ಅನು-ಗ್ರ-ಹವೇ
ಅನು-ಗ್ರ-ಹಿ-ಸಲಿ
ಅನು-ತ್ತೀ-ರ್ಣ-ಗೊ-ಳಿ-ಸಿದ್ದ
ಅನು-ತ್ತೀ-ರ್ಣ-ಗೊ-ಳಿ-ಸಿ-ದ್ದಾ-ನೆಂಬ
ಅನು-ತ್ತೀ-ರ್ಣತೆ
ಅನು-ತ್ತೀ-ರ್ಣ-ನಾ-ಗಿ-ದ್ದೇನೋ
ಅನು-ತ್ತೀ-ರ್ಣ-ನಾ-ದದ್ದು
ಅನು-ತ್ತೀ-ರ್ಣ-ನಾದೆ
ಅನು-ತ್ತೀ-ರ್ಣ-ರ-ನ್ನಾ-ಗಿ-ಸು-ತ್ತಿ-ದ್ದರು
ಅನು-ತ್ತೀ-ರ್ಣ-ರಾ-ಗು-ವುದು
ಅನು-ತ್ತೀ-ರ್ಣ-ರಾದ
ಅನು-ದಾನ
ಅನು-ದಾ-ನಿತ
ಅನು-ಪಮ
ಅನು-ಪ-ಮ-ವಾ-ದದ್ದು
ಅನು-ಪ-ಸ್ಥಿ-ತಿ-ಯನ್ನು
ಅನು-ಭವ
ಅನು-ಭ-ವಕ್ಕೆ
ಅನು-ಭ-ವ-ಗ-ಳನ್ನು
ಅನು-ಭ-ವ-ಗ-ಳೆಂ-ದರೆ
ಅನು-ಭ-ವದ
ಅನು-ಭ-ವ-ದಂತೆ
ಅನು-ಭ-ವ-ದಿಂದ
ಅನು-ಭ-ವ-ವನ್ನು
ಅನು-ಭ-ವ-ವಾ-ಗಿ-ರುವ
ಅನು-ಭ-ವವೂ
ಅನು-ಭ-ವವೇ
ಅನು-ಭವಿ
ಅನು-ಭ-ವಿ-ಸ-ಬೇ-ಕಿತ್ತು
ಅನು-ಭ-ವಿ-ಸ-ಬೇಕು
ಅನು-ಭ-ವಿ-ಸಿದ
ಅನು-ಭ-ವಿ-ಸಿ-ದ-ವರು
ಅನು-ಭ-ವಿ-ಸಿ-ದಿರಿ
ಅನು-ಭ-ವಿ-ಸಿದ್ದು
ಅನು-ಮಿತಿ
ಅನು-ಯಾ-ಯಿಗೆ
ಅನು-ಯಾ-ಯಿ-ಯಾ-ದರೂ
ಅನು-ರ-ಣಿ-ಸು-ತ್ತಿದೆ
ಅನು-ರೂ-ಪ-ವಾದ
ಅನು-ವಾಕ
ಅನು-ವಾ-ದದ
ಅನುವು
ಅನು-ಷ್ಠಾ-ನಕ್ಕೆ
ಅನು-ಷ್ಠಾ-ನ-ಪ-ರ-ರಾದ
ಅನು-ಸ-ರ-ಣೀಯ
ಅನು-ಸ-ರಿ-ಸ-ದಿ-ದ್ದರೂ
ಅನು-ಸ-ರಿ-ಸ-ಬೇ-ಕಾದ
ಅನು-ಸ-ರಿ-ಸಲು
ಅನು-ಸ-ರಿಸಿ
ಅನು-ಸ-ರಿ-ಸಿ-ದರೆ
ಅನು-ಸ-ರಿ-ಸಿ-ದ್ದಾನೆ
ಅನು-ಸ-ರಿ-ಸಿ-ದ್ದೇನೆ
ಅನು-ಸ-ರಿ-ಸು-ತ್ತಿ-ದ್ದಾರೆ
ಅನು-ಸ-ರಿ-ಸು-ತ್ತಿ-ದ್ದೇನೆ
ಅನು-ಸ-ರಿ-ಸುವ
ಅನು-ಸ-ರಿ-ಸು-ವಲ್ಲಿ
ಅನು-ಸ್ಯೂ-ತ-ವಾಗಿ
ಅನೇಕ
ಅನೇ-ಕ-ಕಾ-ರ-ಣ-ಗ-ಳಿವೆ
ಅನೇ-ಕ-ಬಾರಿ
ಅನೇ-ಕರ
ಅನೇ-ಕ-ರನ್ನು
ಅನೇ-ಕ-ರಲ್ಲಿ
ಅನೇ-ಕ-ರ-ಲ್ಲಿದೆ
ಅನೇ-ಕ-ರಿಗೆ
ಅನೇ-ಕ-ರೀ-ತಿ-ಯಲ್ಲಿ
ಅನೇ-ಕರು
ಅನೈ-ಚ್ಛಿ-ಕ-ವಾಗಿ
ಅನ್ಧ-ಕಾ-ರ-ನಿ-ರೋ-ಧಿ-ತ್ವಾತ್
ಅನ್ನ
ಅನ್ನ-ಕ್ಕಾಗಿ
ಅನ್ನ-ಗಳ
ಅನ್ನ-ದಾ-ತರೂ
ಅನ್ನ-ದಾತಾ
ಅನ್ನ-ದಾನ
ಅನ್ನ-ದಾ-ನ-ಗ-ಳನ್ನು
ಅನ್ನ-ಧಾ-ನದ
ಅನ್ನ-ಪಾ-ನಾದಿ
ಅನ್ನಾ-ರ್ಥಿನಂ
ಅನ್ನಿ-ಸಲೇ
ಅನ್ನಿ-ಸು-ತ್ತದೆ
ಅನ್ನುವ
ಅನ್ಯ
ಅನ್ಯಥಾ
ಅನ್ಯ-ಭಾಷೆ
ಅನ್ಯ-ರಿಗೆ
ಅನ್ಯಾ-ದೃ-ಶ-ವಾ-ದುದು
ಅನ್ಯಾ-ಯಕ್ಕೆ
ಅನ್ಯಾ-ಯದ
ಅನ್ಯಾ-ಯ-ವನ್ನು
ಅನ್ಯೋ-ನ್ಯ-ವಾ-ಗಿಯೇ
ಅನ್ವಯ
ಅನ್ವ-ಯ-ಗಳ
ಅನ್ವ-ಯ-ಮಾ-ಡುವ
ಅನ್ವ-ಯ-ವಾ-ಗು-ತ್ತದೆ
ಅನ್ವ-ಯಾ-ನು-ಸಾರ
ಅನ್ವ-ಯಿಕ
ಅನ್ವ-ಯಿಸಿ
ಅನ್ವ-ಯಿ-ಸಿ-ರು-ವುದು
ಅನ್ವ-ಯಿ-ಸು-ತ್ತದೆ
ಅನ್ವ-ಯಿ-ಸು-ತ್ತ-ದೆಯೇ
ಅನ್ವ-ಯಿ-ಸು-ವು-ದಿಲ್ಲ
ಅನ್ವ-ಯಿ-ಸು-ವುದು
ಅನ್ವ-ರ್ಥ-ದಂ-ತಿ-ರುವ
ಅನ್ವ-ರ್ಥ-ನಾ-ಮ-ವ-ನ್ನಾ-ಗಿ-ಸಿ-ಕೊಂ-ಡಿ-ದ್ದಾರೆ
ಅನ್ವ-ರ್ಥ-ರಾ-ಗಿ-ರು-ವರೋ
ಅನ್ವ-ರ್ಥ-ವಾ-ಗಿ-ಸಿ-ಕೊಂಡ
ಅನ್ವಿ-ತ-ವಾದ
ಅನ್ವೇ-ಷಣೆ
ಅನ್ವೇ-ಷ-ಣೆಗೆ
ಅಪ-ಚಾರ
ಅಪ-ದ್ಧ-ವಾ-ಗ-ಬಾ-ರ-ದೆಂದು
ಅಪ-ಧ್ಭಾಂ-ದ-ವ-ರಾಗಿ
ಅಪ-ಬ-ಳ-ಕೆ-ಯನ್ನು
ಅಪ-ಭ್ರಂ-ಶವೇ
ಅಪ-ಮಾನ
ಅಪ-ರಂಜಿ
ಅಪ-ರ-ಸಂ-ಬ-ಧಿ-ಸ್ಮಾ-ರಮ್
ಅಪ-ರಾ-ಧ-ವಾ-ದೀತು
ಅಪ-ರಾ-ವ-ತಾ-ರ-ದಂ-ತಿ-ರುವ
ಅಪ-ರಾ-ವ-ತಾ-ರ-ವಾದ
ಅಪ-ರಿ-ಚಿತ
ಅಪ-ರಿ-ಹಾ-ರ್ಯ-ವಾ-ಯಿತು
ಅಪ-ರೂಪ
ಅಪ-ರೂ-ಪ-ಕ್ಕೆಂ-ಬಂತೆ
ಅಪ-ರೂ-ಪದ
ಅಪ-ರೂ-ಪ-ವಾಗಿ
ಅಪ-ರೂ-ಪವೂ
ಅಪ-ವಾದ
ಅಪ-ವಾ-ದವೋ
ಅಪ-ಹಾ-ಸ್ಯದ
ಅಪಾಯ
ಅಪಾ-ಯ-ವಿ-ಲ್ಲದ
ಅಪಾರ
ಅಪಾ-ರ-ವಾದ
ಅಪಾ-ರ-ವಾ-ದದ್ದು
ಅಪು-ತ್ರ-ರಾ-ಗಿಯೂ
ಅಪೂರ್ಣ
ಅಪೂ-ರ್ಣ-ತೆ-ಯಿಂದ
ಅಪೂರ್ವ
ಅಪೇ-ಕ್ಷಿತ
ಅಪೇ-ಕ್ಷಿ-ಸದ
ಅಪೇ-ಕ್ಷಿ-ಸದೇ
ಅಪೇ-ಕ್ಷಿಸಿ
ಅಪೇ-ಕ್ಷಿ-ಸಿದ
ಅಪೇ-ಕ್ಷಿ-ಸು-ತ್ತದೆ
ಅಪೇ-ಕ್ಷಿ-ಸು-ವಂ-ಥ-ದ್ದಾ-ದ್ದ-ರಿಂದ
ಅಪೇ-ಕ್ಷೆ-ಯಾ-ಗಿದೆ
ಅಪ್ಪ
ಅಪ್ಪಟ
ಅಪ್ಪು-ಗೆಯ
ಅಪ್ಪು-ವಿ-ಕೆ-ಯಿಂದ
ಅಪ್ರ-ತಿಮ
ಅಪ್ರ-ತಿ-ಮ-ಮತಿ
ಅಫಾ-ರ-ವಾಧ
ಅಬಾ-ಲ-ವೃ-ದ್ಧ-ರಿಗೂ
ಅಬಿ-ನಾ-ಭಾವ
ಅಬಿ-ವಂ-ದನ
ಅಭ-ಯ-ವಿ-ದ್ದ-ಹಾಗೆ
ಅಭಾ-ವ-ದಿಂ-ದಲೋ
ಅಭಾವೇ
ಅಭಿ-ಜಾತ
ಅಭಿ-ಜ್ಞಾ-ಪ-ಕ-ವಾಗಿ
ಅಭಿ-ನಂ-ದ-ನ-ಗ್ರಂ-ಥ-ದಲ್ಲಿ
ಅಭಿ-ನಂ-ದನಾ
ಅಭಿ-ನಂ-ದ-ನಾರ್ಹ
ಅಭಿ-ನಂ-ದ-ನೀ-ಯರು
ಅಭಿ-ನಂ-ದ-ನೆ-ಗಳು
ಅಭಿ-ನಂ-ದಾನಾ
ಅಭಿ-ನಂ-ದಿ-ಸು-ತ್ತೇನೆ
ಅಭಿ-ನಯ
ಅಭಿ-ಪ್ರಾಯ
ಅಭಿ-ಪ್ರಾ-ಯ-ಗಳು
ಅಭಿ-ಪ್ರಾ-ಯ-ಪ-ಡು-ತ್ತಾನೆ
ಅಭಿ-ಪ್ರಾ-ಯ-ವನ್ನೇ
ಅಭಿ-ಮ-ತ-ವನ್ನು
ಅಭಿ-ಮಾನ
ಅಭಿ-ಮಾ-ನ-ವಿ-ರು-ವು-ದ-ಲ್ಲದೇ
ಅಭಿ-ಮಾ-ನಾ-ಸ್ಪದ
ಅಭಿ-ಮಾನಿ
ಅಭಿ-ಮಾ-ನಿ-ಗಳು
ಅಭಿ-ಮಾ-ನಿ-ಗ-ಳೆಲ್ಲ
ಅಭಿ-ಮಾ-ನಿ-ಗ-ಳೊ-ಡ-ಗೂಡಿ
ಅಭಿ-ಯು-ಕ್ತರು
ಅಭಿ-ಲಾ-ಷೆ-ಯನ್ನು
ಅಭಿ-ವಂ-ದನ
ಅಭಿ-ವಂ-ದ-ನ-ಕಾರ್ಯ
ಅಭಿ-ವಂ-ದ-ನ-ಗ್ರಂಥ
ಅಭಿ-ವಂ-ದ-ನ-ಪತ್ರ
ಅಭಿ-ವಂ-ದ-ನ-ಸ-ಮಿತಿ
ಅಭಿ-ವಂ-ದನಾ
ಅಭಿ-ವಂ-ದ-ನಾ-ಸ-ಮಾ-ರಂ-ಭದ
ಅಭಿ-ವಂ-ದ-ನೆ-ಗ-ಳನ್ನು
ಅಭಿ-ವಂ-ದ-ನೆಯ
ಅಭಿ-ವಂ-ದ-ನೆ-ಯನ್ನು
ಅಭಿ-ವಂ-ದಿ-ಸಲು
ಅಭಿ-ವಂ-ದಿ-ಸು-ತ್ತಿ-ದ್ದಾರೆ
ಅಭಿ-ವಂ-ದಿ-ಸುವ
ಅಭಿ-ವಂ-ದ್ಯ-ರಾ-ದ-ವರು
ಅಭಿ-ವೃ-ದ್ಧಿ-ಗಾಗಿ
ಅಭಿ-ವೃ-ದ್ಧಿಗೆ
ಅಭಿ-ವೃ-ದ್ಧಿಯ
ಅಭಿ-ವೃ-ಧ್ಧಿಗೆ
ಅಭಿ-ವ್ಯ-ಕ್ತ-ರೂಪ
ಅಭಿ-ವ್ಯ-ಕ್ತಿಯ
ಅಭ್ಯ-ರ್ಥಿ-ಗ-ಳಿಗೆ
ಅಭ್ಯ-ರ್ಥಿ-ಯಾಗಿ
ಅಭ್ಯ-ರ್ಥಿ-ಯಾ-ಗಿಯೇ
ಅಭ್ಯಾ-ಗ-ತರ
ಅಭ್ಯಾ-ಗ-ತ-ರನ್ನು
ಅಭ್ಯಾ-ಗ-ತರು
ಅಭ್ಯಾಸ
ಅಭ್ಯಾ-ಸದ
ಅಭ್ಯಾ-ಸ-ದಿಂದ
ಅಭ್ಯಾ-ಸ-ಮಾ-ಡುವ
ಅಭ್ಯಾ-ಸವೂ
ಅಭ್ಯಾ-ಸವೇ
ಅಭ್ಯಾ-ಸವೇ
ಅಭ್ಯು-ದ-ಯ-ವನ್ನು
ಅಮ-ರ-ಕೋ-ಶ-ವಿ-ರಲಿ
ಅಮಾ-ಯ-ಕ-ರನ್ನು
ಅಮಿ-ತಾ-ನಂ-ದ-ವನ್ನು
ಅಮಿ-ತ್ರ-ವೈರಿ
ಅಮೂಲ್ಯ
ಅಮೂ-ಲ್ಯ-ವಾದ
ಅಮೂ-ಲ್ಯ-ಸೇವೆ
ಅಮೃ-ತೋ-ತ್ಸವ
ಅಮೆ-ರಿ-ಕಾದ
ಅಮೇ-ಜಾ-ನ್-ನ
ಅಮೇ-ರಿ-ಕಾದ
ಅಮ್ಮ
ಅಮ್ಮನ
ಅಮ್ಮ-ನನ್ನು
ಅಯ-ಸ್ಕಾಂ-ತಕ್ಕೆ
ಅಯುತಂ
ಅಯು-ರ್ವೇ-ದದ
ಅಯೋ-ಜಿ-ಸಿದೆ
ಅಯೋ-ಧ್ಯೆಯ
ಅಯ್ಕೆ-ಮಾ-ಡಿ-ಕೊ-ಳ್ಳು-ವಂತೆ
ಅಯ್ಯಂ-ಗಾರ
ಅಯ್ಯಂ-ಗಾ-ರ್ಯ-ರ-ವರ
ಅಯ್ಯೋ-ಉ-ಪ-ನ್ಯಾಸ
ಅರ-ಗಿಸಿ
ಅರ-ಗು-ತ್ತಿ-ರ-ಲಿಲ್ಲ
ಅರತಿ
ಅರ-ಬಿಂದೋ
ಅರ-ಳಲಿ
ಅರಳಿ
ಅರ-ಳಿದ
ಅರ-ಳಿದೆ
ಅರ-ಳಿ-ಸಿ-ಕೊಂ-ಡಿರಿ
ಅರ-ಳಿ-ಸುವ
ಅರ-ವತ್ತು
ಅರ-ವಿಂದ
ಅರ-ಸಲು
ಅರಸಿ
ಅರ-ಸಿ-ಕೊಂಡು
ಅರ-ಸಿ-ಬಂದ
ಅರಿಕೆ
ಅರಿ-ಗ-ಳಾದ
ಅರಿತ
ಅರಿ-ತಂತೆ
ಅರಿ-ತ-ವ-ರಾಗಿ
ಅರಿ-ತ-ವ-ರಿಗೇ
ಅರಿ-ತ-ವ-ರಿಲ್ಲ
ಅರಿ-ತಿ-ದ್ದರೆ
ಅರಿತು
ಅರಿ-ತು-ಕೊ-ಳ್ಳಲು
ಅರಿ-ತು-ಕೊ-ಳ್ಳು-ವಂತೆ
ಅರಿ-ಯದೆ
ಅರಿ-ವಾ-ಗದೆ
ಅರಿ-ವಾ-ಯಿತು
ಅರಿ-ವಿಗೆ
ಅರಿ-ವಿದೆ
ಅರಿ-ವಿದ್ದ
ಅರಿ-ವಿದ್ದು
ಅರಿ-ವಿನ
ಅರಿ-ವಿ-ನಲ್ಲಿ
ಅರಿ-ವಿ-ನಾಚೆ
ಅರಿ-ವಿಲ್ಲ
ಅರಿ-ವಿ-ಲ್ಲ-ದಂ-ತೆಯೇ
ಅರಿವು
ಅರಿ-ವೆಗೆ
ಅರಿವೇ
ಅರಿ-ಶಿ-ನ-ಗೋಡು
ಅರಿ-ಷ-ಡ್ವೈ-ರಿ-ಗಳ
ಅರುಕ್
ಅರು-ಹ-ಲ-ಸಾ-ಧ್ಯ-ವಾದ
ಅರೆ-ಕಾ-ಲಿಕಾ
ಅರ್
ಅರ್ಚಿ-ಸ-ಲಾ-ಗು-ತ್ತ-ದೆಂದೂ
ಅರ್ಜಿ
ಅರ್ಜಿ-ಕೊ-ಡಲು
ಅರ್ಜಿ-ಸಿ-ದ್ಧ-ಪ-ಡಿಸಿ
ಅರ್ಜುನ
ಅರ್ಜೆಂಟಾ
ಅರ್ಥ
ಅರ್ಥ-ಗ-ರ್ಭಿ-ತ-ವಾ-ದವು
ಅರ್ಥ-ಗ-ಳಿ-ಗೋ-ಸ್ಕರ
ಅರ್ಥ-ದಲ್ಲಿ
ಅರ್ಥ-ಧಾ-ರಿ-ಯಾಗಿ
ಅರ್ಥ-ನೀ-ತಿಯ
ಅರ್ಥ-ಪೂರ್ಣ
ಅರ್ಥ-ಪೂ-ರ್ಣ-ಗೊ-ಳಿ-ಸಿ-ಕೊ-ಳ್ಳ-ಲೆಂದು
ಅರ್ಥ-ಪೂ-ರ್ಣ-ವಾಗಿ
ಅರ್ಥ-ಮಾ-ಡಿ-ಕೊಂ-ಡ-ವರು
ಅರ್ಥ-ಮಾ-ಡಿ-ಕೊಂಡು
ಅರ್ಥ-ಮಾ-ಡಿ-ಕೊ-ಳ್ಳದ
ಅರ್ಥ-ಮಾ-ಡಿ-ಕೊ-ಳ್ಳು-ತ್ತಾರೆ
ಅರ್ಥ-ಮಾ-ಡಿ-ಕೊ-ಳ್ಳುವ
ಅರ್ಥ-ಮಾ-ಡಿ-ಸು-ತ್ತಿ-ದ್ದರು
ಅರ್ಥ-ಮಾ-ಡಿ-ಸುವ
ಅರ್ಥ-ಮಾ-ಡು-ವುದು
ಅರ್ಥ-ರೂಪ
ಅರ್ಥ-ವನ್ನು
ಅರ್ಥ-ವನ್ನೂ
ಅರ್ಥ-ವಾ-ಗದ
ಅರ್ಥ-ವಾ-ಗ-ದಿ-ರು-ವುದು
ಅರ್ಥ-ವಾ-ಗದು
ಅರ್ಥ-ವಾ-ಗದೆ
ಅರ್ಥ-ವಾ-ಗದೇ
ಅರ್ಥ-ವಾ-ಗಿ-ಲ್ಲ-ವೆಂ-ದರೆ
ಅರ್ಥ-ವಾ-ಗು-ತ್ತದೆ
ಅರ್ಥ-ವಾ-ಗುವ
ಅರ್ಥ-ವಾ-ಗು-ವಂತೆ
ಅರ್ಥ-ವಿ-ರು-ವುದು
ಅರ್ಥ-ವಿ-ಲ್ಲದೇ
ಅರ್ಥವೇ
ಅರ್ಥ-ವೇ-ನೆಂ-ದರೆ
ಅರ್ಥ-ವೈ-ಭವ
ಅರ್ಥ-ಶಾಸ್ತ್ರ
ಅರ್ಥ-ಶಾ-ಸ್ತ್ರ-ಗ-ಳಲ್ಲಿ
ಅರ್ಥ-ಶಾ-ಸ್ತ್ರದ
ಅರ್ಥ-ಶಾ-ಸ್ತ್ರ-ದ-ಲ್ಲಿನ
ಅರ್ಥ-ಶಾ-ಸ್ತ್ರ-ವನ್ನು
ಅರ್ಥ-ಶೌಚ
ಅರ್ಥ-ಶೌ-ಚವು
ಅರ್ಥ-ಶೌ-ಚ-ವೆಂ-ದರೆ
ಅರ್ಥ-ಸಂ-ಪಾ-ದನೆ
ಅರ್ಥ-ಸ-ಹಿ-ತ-ವಾಗಿ
ಅರ್ಥಾದಿ
ಅರ್ಥೆ
ಅರ್ಥೈ-ಸ-ಬ-ಹುದು
ಅರ್ಥೈ-ಸ-ಲಾ-ಗದ
ಅರ್ಥೈಸಿ
ಅರ್ಥೈ-ಸಿ-ಕೊ-ಳ್ಳಲು
ಅರ್ಥೈ-ಸಿ-ಕೊ-ಳ್ಳುವ
ಅರ್ಥೈ-ಸುವ
ಅರ್ಥೈ-ಸು-ವುದು
ಅರ್ಧ
ಅರ್ಧಾಂ-ಗಿ-ಯಾಗಿ
ಅರ್ಪಿಸಿ
ಅರ್ಪಿ-ಸು-ತ್ತಿ-ದ್ದೇನೆ
ಅರ್ಪಿ-ಸು-ತ್ತೇನೆ
ಅರ್ವಾ-ಚೀನ
ಅರ್ಹತೆ
ಅರ್ಹ-ತೆ-ಯನ್ನು
ಅರ್ಹ-ತೆ-ಯಿಂದ
ಅರ್ಹ-ತೆ-ಯಿ-ರ-ಲಿಲ್ಲ
ಅರ್ಹ-ರಾ-ಗಿ-ದ್ದಾರೆ
ಅರ್ಹರು
ಅರ್ಹ-ವಾ-ಗಿದೆ
ಅಲಂ-ಕ-ರಿ-ಸಿದ
ಅಲಂ-ಕ-ರಿ-ಸಿ-ದಂ-ತಾ-ಯಿತು
ಅಲಂ-ಕಾರ
ಅಲಂ-ಕಾ-ರ-ವಿ-ರಲಿ
ಅಲಂ-ಕಾ-ರ-ಶಾಸ್ತ್ರ
ಅಲಂ-ಕಾ-ರ-ಶಾ-ಸ್ತ್ರ-ವನ್ನು
ಅಲಂ-ಕೃ-ತ-ವಾದ
ಅಲ-ಭ್ಧ-ಮೀ-ಹೇ-ದ್ಧ-ರ್ಮೆಣ
ಅಲ-ಭ್ಧ-ವಾ-ದ-ದ್ದನ್ನು
ಅಲ-ರ್ಜಿ-ಯಾ-ಗಿ-ರು-ತ್ತ-ದೆ-ಕಾ-ರಣ
ಅಲೆ-ಗ-ಳನ್ನು
ಅಲ್ಪ
ಅಲ್ಪ-ಕಾ-ಲ-ದಲ್ಲೆ
ಅಲ್ಪಾ-ದರ
ಅಲ್ಪಾ-ರ್ಥ-ದಿಂ-ದಲೂ
ಅಲ್ಲ
ಅಲ್ಲದ
ಅಲ್ಲದೆ
ಅಲ್ಲದೇ
ಅಲ್ಲ-ನಮ್ಮ
ಅಲ್ಲವೇ
ಅಲ್ಲವೇ
ಅಲ್ಲ-ವೇನೋ
ಅಲ್ಲಿ
ಅಲ್ಲಿಂದ
ಅಲ್ಲಿಗೆ
ಅಲ್ಲಿಗೇ
ಅಲ್ಲಿ-ದ್ದ-ದ್ದೊಂದು
ಅಲ್ಲಿನ
ಅಲ್ಲಿಯ
ಅಲ್ಲಿ-ಯ-ವ-ರೆಗೆ
ಅಲ್ಲಿಯೂ
ಅಲ್ಲಿಯೇ
ಅಲ್ಲಿ-ರುವ
ಅಲ್ಲೆ
ಅಲ್ಲೆಲ್ಲ
ಅಲ್ಲೆಲ್ಲಾ
ಅಲ್ಲೇ
ಅಲ್ಲೇನಾ
ಅಲ್ಲೊಂದು
ಅಳ-ತೆ-ಗೋ-ಲಾಗಿ
ಅಳಲು
ಅಳ-ವ-ಡಿ-ಸಿ-ಕೊಂ-ಡಿ-ದ್ದೇನೆ
ಅಳ-ವ-ಡಿ-ಸಿ-ಕೊಂಡು
ಅಳ-ವ-ಡಿ-ಸಿ-ಕೊ-ಳ್ಳಲು
ಅಳ-ವಾದ
ಅಳಿದೂ
ಅಳಿಯ
ಅಳಿ-ಯ-ದು-ಳಿ-ಯ-ಲೆಂ-ದಿಗೂ
ಅಳಿಲು
ಅಳಿ-ಲು-ಪ್ರ-ಯ-ತ್ನ-ಮಾ-ತ್ರ-ದಿಂದ
ಅಳಿವ
ಅಳಿ-ಸ-ಲಾ-ಗದ್ದು
ಅಳುಕೂ
ಅಳೆ-ಯ-ಲಾ-ಗದ
ಅವ-ಕಾಶ
ಅವ-ಕಾ-ಶ-ದಿಂದ
ಅವ-ಕಾ-ಶ-ವನ್ನು
ಅವ-ಕಾ-ಶ-ವಿತ್ತು
ಅವ-ಕಾ-ಶ-ವಿ-ರ-ಲಿಲ್ಲ
ಅವ-ಕಾ-ಶ-ವಿ-ರುವ
ಅವ-ಕಾ-ಶ-ವಿಲ್ಲ
ಅವ-ಕಾ-ಶವು
ಅವ-ಕಾ-ಶವೇ
ಅವ-ಗಾ-ಹನೆ
ಅವ-ಘ-ಡ-ಗ-ಳನ್ನು
ಅವ-ತ-ರಿ-ಸು-ವುದು
ಅವ-ತಾ-ರವೋ
ಅವ-ಧಾ-ನಿ-ಗಳ
ಅವ-ಧಾ-ನಿ-ಗಳು
ಅವ-ಧಾ-ನಿ-ಯ-ವರು
ಅವಧಿ
ಅವ-ಧಿ-ಗ-ಳಿಗೆ
ಅವ-ಧಿಗೆ
ಅವ-ಧಿಯ
ಅವ-ಧಿ-ಯಲ್ಲಿ
ಅವ-ಧೂತ
ಅವನ
ಅವ-ನನ್ನು
ಅವ-ನಲ್ಲಿ
ಅವ-ನ-ಲ್ಲಿದೆ
ಅವ-ನಿಂದ
ಅವ-ನಿ-ಗಿಂತ
ಅವ-ನಿಗೂ
ಅವ-ನಿಗೆ
ಅವನು
ಅವನೇ
ಅವ-ನ್ನೆಲ್ಲ
ಅವ-ಭೃತ
ಅವ-ಭೃ-ಥರ
ಅವ-ಭೃ-ಥ-ರಿಗೆ
ಅವ-ಭೃ-ಥರು
ಅವ-ಮಾ-ನ-ಕರ
ಅವ-ಮಾ-ನ-ವಾ-ಗ-ದಂತೆ
ಅವ-ಮಾ-ನ-ವಾ-ಗು-ತ್ತದೆ
ಅವರ
ಅವ-ರಂ-ತಹ
ಅವ-ರಂ-ತ-ಹ-ವರೇ
ಅವ-ರಂ-ಥ-ವ-ರನ್ನು
ಅವ-ರತ್ತ
ಅವ-ರ-ದಾ-ಗಿತ್ತು
ಅವ-ರದು
ಅವ-ರದೂ
ಅವ-ರದೇ
ಅವ-ರ-ದ್ದಲ್ಲ
ಅವ-ರದ್ದು
ಅವ-ರನ್ನು
ಅವ-ರ-ನ್ನೆಲ್ಲ
ಅವ-ರಲ್ಲಿ
ಅವ-ರ-ಲ್ಲಿಗೆ
ಅವ-ರ-ಲ್ಲಿತ್ತು
ಅವ-ರ-ಲ್ಲಿದೆ
ಅವ-ರ-ಲ್ಲಿನ
ಅವ-ರ-ಲ್ಲಿಯೂ
ಅವ-ರ-ಲ್ಲಿಯೇ
ಅವ-ರ-ಲ್ಲಿ-ರುವ
ಅವ-ರ-ವರ
ಅವ-ರಾಗಿ
ಅವ-ರಾ-ದೃಷ್ಟ
ಅವ-ರಿಂದ
ಅವ-ರಿಂ-ದಲೇ
ಅವ-ರಿ-ಗಂತೂ
ಅವ-ರಿ-ಗಾಗಿ
ಅವ-ರಿ-ಗಿ-ರುವ
ಅವ-ರಿಗೂ
ಅವ-ರಿಗೆ
ಅವ-ರಿ-ಗೆಲ್ಲ
ಅವ-ರಿ-ಗೆಲ್ಲಾ
ಅವ-ರಿಗೇ
ಅವ-ರಿ-ವ-ರಲ್ಲಿ
ಅವ-ರಿ-ವ-ರಿ-ಗಿಂತ
ಅವ-ರಿ-ವರು
ಅವರು
ಅವರೂ
ಅವ-ರೆಂದೂ
ಅವ-ರೆಲ್ಲ
ಅವ-ರೆ-ಲ್ಲರ
ಅವ-ರೆ-ಲ್ಲ-ರನ್ನೂ
ಅವ-ರೆ-ಲ್ಲ-ರಿಗೆ
ಅವ-ರೆ-ಲ್ಲ-ರೊಂ-ದಿಗೆ
ಅವರೇ
ಅವ-ರೇನು
ಅವ-ರೊಂ-ದಿಗೆ
ಅವ-ರೊ-ಡನೆ
ಅವ-ರೊ-ಡ-ನೆಯ
ಅವ-ರೊ-ಳ-ಗಿನ
ಅವರೋ
ಅವ-ರ್ಣ-ನೀಯ
ಅವ-ಲಂ-ಬನೆ
ಅವ-ಲಂ-ಬಿ-ಸಿ-ದರೂ
ಅವ-ಲಂ-ಬಿ-ಸಿ-ದ-ವ-ರೆ-ಲ್ಲರೂ
ಅವ-ಲಂ-ಬಿ-ಸಿದೆ
ಅವ-ಲಂ-ಬಿ-ಸು-ತ್ತದೆ
ಅವ-ಲಂ-ಬಿ-ಸುವ
ಅವ-ಲಂ-ಬಿ-ಸು-ವುದು
ಅವ-ಲೋ-ಕನ
ಅವಳ
ಅವ-ಳನ್ನು
ಅವ-ಳಿಗೆ
ಅವ-ಶ್ಯ-ಕತೆ
ಅವ-ಶ್ಯ-ಕ-ತೆ-ಯಿದೆ
ಅವ-ಸರ
ಅವ-ಸ-ರದ
ಅವ-ಸ-ರ-ವಲ್ಲ
ಅವ-ಸ-ರೋ-ಚಿ-ತ-ವಾದ
ಅವಸ್ಥೆ
ಅವ-ಸ್ಥೆ-ಗ-ಳಿಗೆ
ಅವ-ಸ್ಥೆ-ಯಾದ
ಅವಿ-ಚಾರ
ಅವಿ-ನಾ-ಭಾವ
ಅವಿ-ನಾ-ಭಾ-ವ-ದಿಂದ
ಅವಿ-ಭಕ್ತ
ಅವಿ-ರತ
ಅವಿ-ರ-ತ-ವಾಗಿ
ಅವಿ-ರ-ತ-ವಾದ
ಅವಿ-ರೋಧಿ
ಅವಿ-ವೇಕಿ
ಅವಿ-ಸ್ಮ-ರ-ಣೀಯ
ಅವು
ಅವು-ಗಳ
ಅವು-ಗ-ಳನ್ನು
ಅವು-ಗ-ಳಲ್ಲಿ
ಅವು-ಗ-ಳಿಂದ
ಅವು-ಗಳು
ಅವು-ಗಳೇ
ಅವೆ-ಲ್ಲವೂ
ಅವೈ-ಜ್ಞಾ-ನಿಕ
ಅವ್ಯಾ-ಜ-ವಾ-ದುದು
ಅವ್ಯಾ-ಹತ
ಅಶ-ಕ್ತಿ-ವಿ-ಶೇಷ
ಅಶ-ನ-ವ್ಯ-ವ-ಸ್ಥೆ-ಯನ್ನು
ಅಶ-ನಾ-ದಿ-ಗಳ
ಅಶ-ನ-ವ-ಸ-ನ-ಗ-ಳೊಂ-ದಿಗೆ
ಅಶ-ನ-ವ-ಸ-ನಾದಿ
ಅಶ-ನ-ವ-ಸ-ನಾ-ದಿ-ಗ-ಳನ್ನು
ಅಶ-ನ-ವ-ಸ-ನಾ-ದಿ-ವ್ಯ-ವ-ಸ್ಥೆ-ಗ-ಳೆ-ಲ್ಲ-ವನ್ನೂ
ಅಶಾಂತಿ
ಅಶಾಂ-ತಿಯ
ಅಶಾ-ಕ-ಭುಕ್
ಅಶೇ-ಷ-ಕಾ-ಯ-ಪ್ರ-ಸೃ-ತಾ-ನ-ಶೇ-ಷಾನ್
ಅಶೋಕಾ
ಅಶೌಚ
ಅಶ್ಲೀ-ಲ-ವನ್ನು
ಅಷ್ಟ-ಗು-ಣ-ಗ-ಳನ್ನು
ಅಷ್ಟರ
ಅಷ್ಟ-ರ-ಮ-ಟ್ಟಿಗೆ
ಅಷ್ಟ-ರ-ಲ್ಲಾ-ಗಲೇ
ಅಷ್ಟ-ರಲ್ಲಿ
ಅಷ್ಟಾಂಗ
ಅಷ್ಟಾಂ-ಗ-ಗಳ
ಅಷ್ಟಾಂ-ಗ-ಹೃ-ದಯ
ಅಷ್ಟಾಂ-ಗ-ಹೃ-ದ-ಯದ
ಅಷ್ಟಾಂ-ಗ-ಹೃ-ದ-ಯ-ದಂ-ತಹ
ಅಷ್ಟಾಂ-ಗ-ಹೃ-ದ-ಯವು
ಅಷ್ಟಾಗಿ
ಅಷ್ಟಾ-ದರೂ
ಅಷ್ಟಿ-ಷ್ಟಾ-ದರೂ
ಅಷ್ಟು
ಅಷ್ಟೂ
ಅಷ್ಟೆ
ಅಷ್ಟೇ
ಅಷ್ಟೇನೂ
ಅಷ್ಟೊಂದು
ಅಷ್ಟೊ-ತ್ತಿ-ಗಾ-ಗಲೆ
ಅಷ್ಟೋ-ತ್ತ-ರ-ಶತ
ಅಷ್ಠ-ಬಂಧ
ಅಸಂಖ್ಯ
ಅಸಂ-ಖ್ಯ-ರಿ-ದ್ದಾರೆ
ಅಸಂ-ಖ್ಯಾತ
ಅಸಂ-ಖ್ಯಾತಂ
ಅಸಂ-ಖ್ಯಾ-ತರು
ಅಸಂ-ಖ್ಯಾ-ತ-ವಿ-ದ್ಯಾ-ರ್ಥಿ-ಗ-ಳಿಗೆ
ಅಸಂ-ದಿ-ಗ್ಧ-ವಾಗಿ
ಅಸಂ-ಭ-ವ-ವಾ-ಗಿತ್ತು
ಅಸ-ತ್ಪ್ರ-ತಿ-ಗ್ರಹ
ಅಸ-ದಳ
ಅಸ-ದೃ-ಶ-ವಾದ
ಅಸ-ಮ-ರ್ಥ-ರಾ-ಗಿ-ರು-ತ್ತಾರೆ
ಅಸ-ಮಾ-ಧಾನ
ಅಸ-ಮಾ-ಧಾ-ನಕ್ಕೆ
ಅಸ-ಮಾ-ಧಾ-ನ-ವನ್ನು
ಅಸ-ಮ್ಮ-ತಿ-ಸಿ-ದ್ದೇನೆ
ಅಸವು
ಅಸ-ಹಾ-ಯ-ಕ-ನಾಗಿ
ಅಸ-ಹಾ-ಯ-ರಾ-ಗಿ-ಬಿ-ಟ್ಟರು
ಅಸಾ-ಧಾ-ರಣ
ಅಸಾ-ಧ್ಯ-ವಾ-ಗು-ವ-ದರ
ಅಸಾ-ಮಾ-ನ್ಯರು
ಅಸೀಮ
ಅಸೀ-ಮ-ವಾದ
ಅಸೂಯೆ
ಅಸೆ
ಅಹಂ-ಕಾರ
ಅಹಂ-ಕಾ-ರವೋ
ಅಹಂ-ಕಾ-ರಿ-ಯೆಂ-ದು-ಕೊಂ-ಡರೆ
ಅಹ-ರ-ಹ-ರ್ದಾನ
ಅಹಿ
ಅಹೃ-ತೆ-ಗ-ಳೇನೂ
ಆ
ಆಂ
ಆಂಗಿ-ರಸ
ಆಂಗ್ಲ
ಆಂಗ್ಲ-ಪಾ-ಠ-ವನ್ನು
ಆಂಗ್ಲ-ಭಾ-ಷಾ-ಸಂ-ವ-ಹನ
ಆಂಗ್ಲ-ಭಾ-ಷೆ-ಗ-ಳಲ್ಲಿ
ಆಂಗ್ಲ-ಭಾ-ಷೆಯ
ಆಂಗ್ಲ-ಭಾ-ಷೆ-ಯಲ್ಲಿ
ಆಂಗ್ಲ-ವಿ-ದ್ಯಾ-ರ್ಜ-ನಾ-ವಿ-ಧಾನ
ಆಂಗ್ಲ-ಸಂ-ಸ್ಕೃ-ತ-ಕ-ನ್ನಡ
ಆಂಡ್
ಆಂತ-ರಿಕ
ಆಂತ-ರಿ-ಕ-ವಾಗಿ
ಆಂದೋ-ಲ-ನದ
ಆಂದೋ-ಲ-ನ-ದಲ್ಲಿ
ಆಂದೋ-ಲ-ನ-ದಿಂದ
ಆಂದೋ-ಲ-ನ-ಪ-ತ್ರಿ-ಕೆಯ
ಆಕ-ರ-ವಾ-ಗಿ-ರುವ
ಆಕ-ರ್ಷಕ
ಆಕ-ರ್ಷ-ಕ-ವಾ-ಗು-ವಂತೆ
ಆಕ-ರ್ಷ-ಕ-ವಾ-ದವು
ಆಕ-ರ್ಷಣೆ
ಆಕ-ರ್ಷ-ಣೆ-ಯನ್ನು
ಆಕ-ರ್ಷಿತ
ಆಕ-ರ್ಷಿ-ತ-ರಾಗಿ
ಆಕ-ರ್ಷಿ-ಸ-ಲಾ-ರಂ-ಭಿ-ಸಿತು
ಆಕ-ರ್ಷಿ-ಸಿದ
ಆಕ-ರ್ಷಿ-ಸಿದೆ
ಆಕ-ರ್ಷಿ-ಸಿ-ದ್ದರು
ಆಕ-ರ್ಷಿ-ಸು-ತ್ತಿತ್ತು
ಆಕ-ರ್ಷಿ-ಸುವ
ಆಕ-ಸ್ಮಿ-ಕ-ವಾಗಿ
ಆಕಾಂಕ್ಷಾ
ಆಕಾಂ-ಕ್ಷಿ-ಗ-ಳಲ್ಲಿ
ಆಕಾಂ-ಕ್ಷಿ-ಗ-ಳಾ-ಗಿ-ದ್ದರೂ
ಆಕಾ-ರ-ದಲ್ಲಿ
ಆಕಾ-ರ-ವಲ್ಲ
ಆಕಾ-ಶ-ವಾ-ಣಿಯ
ಆಕೂತ
ಆಕೃತಿ
ಆಕೃ-ತಿ-ಯಲ್ಲಿ
ಆಕೆ-ಯನ್ನು
ಆಗ
ಆಗ-ತಾನೆ
ಆಗ-ತಾನೇ
ಆಗದ
ಆಗ-ದಂತೆ
ಆಗದು
ಆಗ-ಬಾ-ರ-ದೆಂದು
ಆಗ-ಬೇ-ಕಾ-ಗಿತ್ತು
ಆಗಮ
ಆಗ-ಮ-ನ-ವಾ-ಯಿತು
ಆಗ-ಮಿಸಿ
ಆಗ-ಮಿ-ಸಿದೆ
ಆಗ-ಮಿ-ಸಿದ್ದು
ಆಗ-ಮಿ-ಸಿ-ರುವ
ಆಗ-ಮಿ-ಸು-ತ್ತಿ-ರು-ವ-ದನ್ನು
ಆಗ-ಮಿ-ಸುವ
ಆಗ-ಲಾ-ರದು
ಆಗಲಿ
ಆಗಲೇ
ಆಗ-ಸ-ದಲ್ಲಿ
ಆಗಾಗ
ಆಗಾಗ್ಗೆ
ಆಗಾಮಿ
ಆಗಿ
ಆಗಿತ್ತು
ಆಗಿದೆ
ಆಗಿದ್ದ
ಆಗಿ-ದ್ದರು
ಆಗಿ-ದ್ದರೂ
ಆಗಿ-ದ್ದರೆ
ಆಗಿ-ದ್ದ-ರೆಂ-ದರೆ
ಆಗಿ-ದ್ದಾ-ನೆಂ-ದರೆ
ಆಗಿ-ದ್ದಾರೆ
ಆಗಿದ್ದು
ಆಗಿದ್ದು
ಆಗಿದ್ದೆ
ಆಗಿನ
ಆಗಿ-ಬಿ-ಟ್ಟಿದೆ
ಆಗಿ-ಬಿ-ಡು-ತ್ತಾರೆ
ಆಗಿ-ಬಿ-ಡು-ತ್ತೇನೆ
ಆಗಿ-ರ-ಬ-ಹುದು
ಆಗಿ-ರ-ಬೇ-ಕೆಂದು
ಆಗಿ-ರ-ಲಿಲ್ಲ
ಆಗಿ-ರು-ತ್ತಿತ್ತು
ಆಗಿ-ರುವ
ಆಗಿ-ರು-ವಂತೆ
ಆಗಿ-ರು-ವ-ವರು
ಆಗಿ-ರು-ವು-ದ-ರಿಂದ
ಆಗು
ಆಗು-ತ್ತವೆ
ಆಗು-ತ್ತಿತ್ತು
ಆಗು-ತ್ತಿದೆ
ಆಗುವ
ಆಗು-ವು-ದಿಲ್ಲ
ಆಗು-ಹೋ-ಗು-ಗ-ಳನ್ನು
ಆಗು-ಹೋ-ಗು-ಗ-ಳಲ್ಲಿ
ಆಗೆಲ್ಲ
ಆಗ್ರ-ಹಕ್ಕೆ
ಆಗ್ರ-ಹದ
ಆಗ್ರ-ಹ-ದಂತೆ
ಆಗ್ರ-ಹ-ಪೂ-ರ್ವ-ಕ-ವಾಗಿ
ಆಗ್ರ-ಹಿ-ಸು-ತ್ತಿತ್ತು
ಆಘಾ-ತ-ಕ್ಕೊ-ಳ-ಗಾದ
ಆಘಾ-ಧ-ವಾದ
ಆಚ-ರ-ಣೆ-ಯನ್ನು
ಆಚ-ರ-ಣೆ-ಯಿಂದ
ಆಚಾರ
ಆಚಾ-ರ-ದಿಂದ
ಆಚಾ-ರ-ಪ-ರರು
ಆಚಾರ್ಯ
ಆಚಾ-ರ್ಯರ
ಆಚಾ-ರ್ಯ-ರನ್ನು
ಆಚಾ-ರ್ಯ-ರಾ-ಗಿ-ದ್ದರು
ಆಚಾ-ರ್ಯ-ರಿಗೆ
ಆಚಾ-ರ್ಯರು
ಆಚಾ-ರ್ಯ-ರೆ-ನಿ-ಸಿ-ಕೊಂ-ಡ-ವರು
ಆಚೆ
ಆಚ್ಛಾದ್ಯ
ಆಜ್ಞೆ
ಆಟ
ಆಟ-ದಲ್ಲಿ
ಆಟ-ವಾ-ಡಲು
ಆಟ-ವಾ-ಡು-ತ್ತಿ-ದ್ದಾಗ
ಆಟ-ವಾ-ಡು-ವುದು
ಆಟೋ-ರಿ-ಕ್ಷಾ-ವನ್ನು
ಆಟ-ಹು-ಡು-ಗಾ-ಟ-ಗ-ಳನ್ನು
ಆಡಂ-ಬ-ರದ
ಆಡದೇ
ಆಡ-ಳಿತ
ಆಡಿದ್ದ
ಆಡು
ಆಡುವ
ಆಡೂ-ಕ-ಟ್ಟೆ-ಯಲ್ಲಿ
ಆಢ್ಯರೂ
ಆಣೆಗೆ
ಆಣೆ-ಯನ್ನು
ಆತ
ಆತಂಕ
ಆತನ
ಆತ-ನಿಗೆ
ಆತನು
ಆತಿಥ್ಯ
ಆತಿ-ಥ್ಯ-ಗ-ಳನ್ನು
ಆತಿ-ಥ್ಯ-ದಲ್ಲಿ
ಆತಿ-ಥ್ಯ-ವನ್ನು
ಆತ್ಮಕ್ಕೆ
ಆತ್ಮ-ಗು-ಣ-ಗ-ಳನ್ನು
ಆತ್ಮ-ಗು-ಣ-ವನ್ನು
ಆತ್ಮ-ತೃಪಿ
ಆತ್ಮ-ತೃ-ಪ್ತಿ-ಯನ್ನು
ಆತ್ಮ-ಬೆ-ಳಕು
ಆತ್ಮ-ವನ್ನು
ಆತ್ಮಾ-ಭಿ-ಮಾ-ನಿ-ಯಾದ
ಆತ್ಮೀಯ
ಆತ್ಮೀ-ಯತೆ
ಆತ್ಮೀ-ಯ-ತೆಯ
ಆತ್ಮೀ-ಯ-ತೆ-ಯನ್ನು
ಆತ್ಮೀ-ಯ-ತೆಯೇ
ಆತ್ಮೀ-ಯರ
ಆತ್ಮೀ-ಯ-ರಾ-ಗಿ-ರು-ವ-ವರು
ಆತ್ಮೀ-ಯ-ರಾದ
ಆತ್ಮೀ-ಯ-ವಾಗಿ
ಆತ್ಮೋ-ದ್ಧಾರ
ಆತ್ಮೋ-ನ್ನ-ತಿ-ಯೆ-ಡೆಗೆ
ಆತ್ಯಂ-ತಿ-ಕ-ವಾದ
ಆದ
ಆದಕ್ಕೆ
ಆದದ್ದು
ಆದರ
ಆದ-ರಂತೆ
ಆದ-ರ-ಗ-ಳಿಂದ
ಆದ-ರ-ಪೂ-ರ್ವ-ಕ-ವಾಗಿ
ಆದ-ರಾ-ತಿಥ್ಯ
ಆದ-ರಾ-ತಿ-ಥ್ಯ-ವನ್ನು
ಆದ-ರಿಂದ
ಆದರು
ಆದರೂ
ಆದರೆ
ಆದರ್ಶ
ಆದ-ರ್ಶ-ಗು-ಣ-ಗಳು
ಆದ-ರ್ಶ-ದಂ-ಪತೀ
ಆದ-ರ್ಶ-ಪ್ರಾಯ
ಆದ-ರ್ಶ-ವಾ-ಗಿ-ಟ್ಟು-ಕೊಂಡು
ಆದ-ರ್ಶ-ವಿ-ಲ್ಲವೇ
ಆದ-ರ-ಆ-ತಿಥ್ಯ
ಆದಷ್ಟು
ಆದಾಯ
ಆದಾ-ಯದ
ಆದಿ
ಆದಿ-ಚುಂ-ಚ-ನ-ಗಿ-ರಿಯ
ಆದಿ-ದೇ-ವ-ನಾದ
ಆದಿ-ಯಲ್ಲಿ
ಆದಿರಿ
ಆದಿ-ವಾ-ಸಿ-ಗ-ಳಿಂದ
ಆದಿ-ವಾ-ಸಿ-ಗಳು
ಆದಿ-ಶಂ-ಕ-ರರು
ಆದು-ದ-ರಿಂದ
ಆದೇಶ
ಆದೇ-ಶದ
ಆದೇ-ಶಿ-ಸಿ-ದರು
ಆದೇ-ಶಿ-ಸಿ-ದ್ದ-ರಂತೆ
ಆದ್ದ-ರಿಂದ
ಆದ್ದ-ರಿಂ-ದಲೇ
ಆದ್ಯ
ಆದ್ಯಂ-ತ-ವಾಗಿ
ಆಧ-ರಿಸಿ
ಆಧ-ರಿ-ಸಿದೆ
ಆಧಾರ
ಆಧಾ-ರದ
ಆಧಾ-ರ-ವಾ-ದರೆ
ಆಧಾ-ರಿತ
ಆಧು-ನಿಕ
ಆಧೇಯ
ಆಧ್ಯ-ಯ-ನ-ಶೀ-ಲತೆ
ಆಧ್ಯಾ-ತ್ಮಿಕ
ಆನಂ-ತರ
ಆನಂದ
ಆನಂ-ದದ
ಆನಂ-ದ-ದಿಂದ
ಆನಂ-ದ-ವನ್ನೇ
ಆನಂ-ದ-ವಾ-ಗಿಯೂ
ಆನಂ-ದ-ವಾ-ಗಿ-ರಲಿ
ಆನಂ-ದವು
ಆನಂ-ದಿ-ಸಿ-ದಿರಿ
ಆನಂ-ದಿ-ಸುತ್ತಾ
ಆನು-ಕೂ-ಲ್ಯ-ವಿದೆ
ಆನ್ನು-ವು-ದೇನೋ
ಆಪ-ದ್ಗತಂ
ಆಪ-ದ್ಬಾಂ-ಧ-ವ-ರಾದ
ಆಪ-ಸ್ತಂ-ಭರು
ಆಪ್ತ
ಆಪ್ತರು
ಆಪ್ತಿ
ಆಪ್ತಿ-ಯನ್ನು
ಆಪ್ತಿಯೇ
ಆಪ್ಯಾ-ಯನ
ಆಭಾರಿ
ಆಭಿ-ಮಾನಿ
ಆಮಂ-ತ್ರಣ
ಆಮೇಲೆ
ಆಯಸ್ಸು
ಆಯಾ
ಆಯಾಮ
ಆಯಾ-ಮ-ಗಳು
ಆಯಿತು
ಆಯು-ಕ್ತ-ರನ್ನು
ಆಯು-ಕ್ತರು
ಆಯು-ರಾ-ರೋ-ಗ್ಯ-ಭಾ-ಗ್ಯ-ವನ್ನು
ಆಯು-ರಾ-ರೋ-ಗ್ಯ-ವನ್ನ
ಆಯು-ರಾ-ರೋ-ಗ್ಯ-ವನ್ನು
ಆಯು-ರಾ-ರೋ-ಗ್ಯ-ಸಂ-ಪ-ದಾ-ದಿ-ಸ-ಕ-ಲ-ಭಾ-ಗ್ಯ-ಗ-ಳನ್ನು
ಆಯು-ರಾ-ರೋ-ಗ್ಯೈ-ಶ್ವ-ರ್ಯ-ಗ-ಳಿಂದ
ಆಯು-ರ್ವೇದ
ಆಯು-ರ್ವೇ-ದಕ್ಕೂ
ಆಯು-ರ್ವೇ-ದಕ್ಕೆ
ಆಯು-ರ್ವೇ-ದದ
ಆಯು-ರ್ವೇ-ದ-ದಲ್ಲಿ
ಆಯು-ರ್ವೇ-ದ-ದಲ್ಲೂ
ಆಯು-ರ್ವೇ-ದ-ವನ್ನು
ಆಯು-ರ್ವೇ-ದವೂ
ಆಯು-ರ್ವೇ-ದ-ಶಾ-ಸ್ತ್ರ-ವನ್ನು
ಆಯು-ರ್ವೇ-ದಾ-ಧ್ಯ-ಯನ
ಆಯು-ರ್ವೇ-ದಾ-ಧ್ಯ-ಯ-ನ-ಕ್ಕೇಂದೇ
ಆಯು-ರ್ವೇ-ದಾ-ಧ್ಯ-ಯ-ನವೂ
ಆಯು-ರ್ವೇ-ದಿಕ್
ಆಯುಷ್ಯ
ಆಯು-ಷ್ಯ-ಆ-ನಂದ
ಆಯು-ಷ್ಯ-ವನ್ನು
ಆಯು-ಷ್ಯ-ವಿ-ನಿ-ಯೋ-ಗ-ವಾಗಿ
ಆಯೋ-ಜ-ಕ-ರಾದ
ಆಯೋ-ಜ-ಕ-ರಿಗೆ
ಆಯೋ-ಜ-ಕರು
ಆಯೋ-ಜ-ನೆ-ಗ-ಳಲ್ಲಿ
ಆಯೋ-ಜ-ನೆ-ಯಲ್ಲಿ
ಆಯೋ-ಜ-ನೆ-ಯಾ-ಗಿದೆ
ಆಯೋ-ಜಿಸಿ
ಆಯೋ-ಜಿ-ಸಿದ
ಆಯೋ-ಜಿ-ಸಿ-ದ್ದಾರೆ
ಆಯೋ-ಜಿ-ಸು-ವ-ದರ
ಆಯ್ಕೆ
ಆಯ್ಕೆ-ಮಾಡಿ
ಆಯ್ಕೆ-ಯಾ-ಗ-ಬ-ಹುದು
ಆಯ್ದು-ಕೊಂ-ಡಿ-ರು-ವುದು
ಆಯ್ದು-ಕೊಂಡು
ಆಯ್ದು-ಕೊ-ಳ್ಳ-ಬೇ-ಕೆಂಬ
ಆರಂ-ಭ-ಗೊಂಡ
ಆರಂ-ಭ-ಗೊಂ-ಡದ್ದು
ಆರಂ-ಭ-ಗೊಂಡು
ಆರಂ-ಭದ
ಆರಂ-ಭ-ದಲ್ಲಿ
ಆರಂ-ಭ-ಯಜ್ಞ
ಆರಂ-ಭ-ಯ-ಜ್ಞಾಃ
ಆರಂ-ಭ-ವಾ-ಗಿತ್ತು
ಆರಂ-ಭ-ವಾ-ಗು-ವುದು
ಆರಂ-ಭ-ವಾದ
ಆರಂ-ಭ-ವಾ-ದದ್ದೇ
ಆರಂ-ಭ-ವಾ-ದಾಗ
ಆರಂ-ಭ-ವಾ-ಯಿ-ತು-ಮತ್ತು
ಆರಂ-ಭಿಸಿ
ಆರಂ-ಭಿ-ಸಿ-ದರು
ಆರಂ-ಭಿ-ಸಿದೆ
ಆರಂ-ಭಿ-ಸಿ-ದೆನು
ಆರಂ-ಭಿ-ಸು-ವ-ದ-ಕ್ಕಿಂತ
ಆರಂ-ಭಿ-ಸು-ವು-ದ-ಕ್ಕಿಂ-ತಲೂ
ಆರ-ಭ್ಯಾ-ನಾ-ಮಿ-ಕಾ-ಮ-ಧ್ಯಾ-ತ್ಪ್ರ-ದ-ಕ್ಷಿ-ಣ-ಮ-ನು-ಕ್ರ-ಮಮ್
ಆರಾ-ಧಿ-ಸಿ-ಕೊಂಡು
ಆರಾಧ್ಯ
ಆರಾ-ಧ್ಯ-ದೇ-ವ-ನಾದ
ಆರಿ-ಸಿ-ಕೊಂಡು
ಆರು
ಆರೋಗ್ಯ
ಆರೋ-ಗ್ಯ-ಗ-ಳನ್ನು
ಆರೋ-ಗ್ಯದ
ಆರೋ-ಗ್ಯ-ದಲ್ಲಿ
ಆರೋ-ಗ್ಯ-ವಂ-ತ-ರಾಗಿ
ಆರೋ-ಗ್ಯ-ವಂ-ತ-ರಾ-ಗಿಯೇ
ಆರೋ-ಗ್ಯ-ವನ್ನು
ಆರೋ-ಗ್ಯ-ವಾ-ಗಿಯೇ
ಆರೋ-ಗ್ಯ-ಶಾ-ಸ್ತ್ರ-ದಲ್ಲಿ
ಆರೋಪಿ
ಆರೋ-ಪಿ-ಸು-ತ್ತೇವೆ
ಆರೋ-ಹಣ
ಆರ್
ಆರ್ಥಿಕ
ಆರ್ಥಿ-ಕತೆ
ಆರ್ಥಿ-ಕ-ವಾಗಿ
ಆರ್ಥಿ-ಕ-ವಾ-ಗಿಯೂ
ಆಲದ
ಆಲ-ಯ-ವನ್ನು
ಆಲಸ್ಯಂ
ಆಲ-ಸ್ಯ-ವನ್ನು
ಆಲೋ-ಚಿ-ಸಿ-ದ-ರಂತೆ
ಆಲೋ-ಚಿ-ಸಿ-ದೆವು
ಆಲ್ಸೋ
ಆಳ
ಆಳ-ದಿ-ಳಿವ
ಆಳ-ವನ್ನು
ಆಳ-ವಾಗಿ
ಆಳ-ವಾ-ಗಿತ್ತು
ಆಳ-ವಾ-ಗಿದೆ
ಆಳ-ವಾದ
ಆಳ-ವಾ-ದುದು
ಆಳ-ವೊಂ-ದಕ್ಕೆ
ಆಳಾಗಿ
ಆಳ್ವಾರ್
ಆಳ್ವಾಸ್
ಆವ-ರ-ಣ-ದ-ಲ್ಲಿವೆ
ಆವ-ರಿ-ಸಿ-ಬಿ-ಟ್ಟಿ-ರು-ತ್ತದೆ
ಆವ-ರ್ತಿ-ಸು-ವುದೇ
ಆವಾಗ
ಆವಿ-ಷ್ಕಾ-ರದ
ಆವೃ-ತ್ತಿಯೇ
ಆಶಯ
ಆಶ-ಯಕ್ಕೆ
ಆಶ-ಯ-ದಂತೆ
ಆಶಿ-ರ್ವಾ-ದ-ದಿಂದ
ಆಶಿ-ಸು-ತ್ತದೆ
ಆಶೀ-ರ್ವ-ಚನ
ಆಶೀ-ರ್ವ-ಚ-ನವೂ
ಆಶೀ-ರ್ವ-ದಿ-ಸಲಿ
ಆಶೀ-ರ್ವ-ದಿಸಿ
ಆಶೀ-ರ್ವ-ದಿ-ಸಿ-ದ್ದಾರೆ
ಆಶೀ-ರ್ವಾದ
ಆಶು
ಆಶ್ಚರ್ಯ
ಆಶ್ಚ-ರ್ಯ-ಕ-ರ-ವೇನೂ
ಆಶ್ಚ-ರ್ಯ-ವಾ-ಯಿತು
ಆಶ್ಚ-ರ್ಯ-ವಿಲ್ಲ
ಆಶ್ರಮ
ಆಶ್ರಯ
ಆಶ್ರ-ಯದ
ಆಶ್ರ-ಯ-ದಲ್ಲಿ
ಆಶ್ರ-ಯ-ವ-ನ್ನಿ-ತ್ತ-ವ-ರಾದ
ಆಶ್ರ-ಯ-ವನ್ನು
ಆಶ್ರ-ಯ-ವಾಗಿ
ಆಶ್ರ-ಯ-ವಿ-ತ್ತಿದೆ
ಆಶ್ರ-ಯ-ವೆಲ್ಲಿ
ಆಶ್ರ-ಯಿಸಿ
ಆಶ್ರ-ಯಿ-ಸು-ತ್ತಾರೆ
ಆಶ್ರ-ಯಿ-ಸು-ತ್ತಿ-ದ್ದಾರೆ
ಆಶ್ರ-ಯಿ-ಸು-ವು-ದನ್ನು
ಆಷಾ-ಢ-ಭೂ-ತಿ-ತ-ನಕ್ಕೆ
ಆಷ್ಟು
ಆಸ-ಕ್ತ-ರಾ-ಗಿ-ದ್ದರು
ಆಸ-ಕ್ತರು
ಆಸಕ್ತಿ
ಆಸ-ಕ್ತಿ-ಬ-ರು-ವಂತೆ
ಆಸ-ಕ್ತಿ-ಬೆ-ಳೆ-ಸು-ತ್ತಿ-ದ್ದರು
ಆಸ-ಕ್ತಿ-ಯನ್ನು
ಆಸ-ಕ್ತಿ-ಯಿಂದ
ಆಸ-ಕ್ತಿ-ಯಿಂ-ದಲೇ
ಆಸ-ಕ್ತಿ-ಯಿ-ರುವ
ಆಸ-ನ-ಗ-ಳನ್ನು
ಆಸ-ರೆ-ಯ-ನ್ನೊ-ದ-ಗಿ-ಸಿದೆ
ಆಸ-ರೆ-ಯಾ-ದುದು
ಆಸ-ರೆಯೇ
ಆಸ-ರೆ-ಆ-ಶ್ರಯ
ಆಸು-ರ-ವಾ-ದೀ-ತು-ಎಂದು
ಆಸೆ
ಆಸೆ-ಯ-ನ್ನಿ-ಟ್ಟು-ಕೊಂಡು
ಆಸೆ-ಯನ್ನು
ಆಸ್ತಿ
ಆಸ್ತಿ-ಕ-ರಾಗಿ
ಆಸ್ತಿ-ಕ್ಯ-ಬುದ್ಧಿ
ಆಸ್ತಿ-ಯಾ-ಗಿ-ದ್ದಾನೆ
ಆಸ್ವಾ-ದ-ನೀ-ಯ-ವಾ-ಗು-ವು-ದೆಂಬ
ಆಸ್ವಾ-ದಿ-ಸಿದ
ಆಸ್ವಾ-ಧ-ಕ-ನಾ-ದ-ವ-ರಲ್ಲಿ
ಆಹಾರ
ಆಹಾ-ರ-ಕ್ರ-ಮ-ವನ್ನೂ
ಆಹಾ-ರವೂ
ಆಹ್ವಾ-ನ-ವೀ-ಯು-ತ್ತಾರೆ
ಆಹ್ವಾ-ನಿ-ತ-ರಾಗಿ
ಇಂಗಿ
ಇಂಗಿ-ತ-ವೇ-ನೆಂ-ದರೆ
ಇಂಗಿ-ಸಿ-ಕೊಂಡು
ಇಂಗ್ಲಿಷ್
ಇಂಗ್ಲೀಶ್
ಇಂಗ್ಲೀ-ಷನ್ನು
ಇಂಗ್ಲೀ-ಷಿನ
ಇಂಗ್ಲೀ-ಷಿ-ನಲ್ಲಿ
ಇಂಗ್ಲೀ-ಷಿ-ನ-ಲ್ಲಿಯೂ
ಇಂಗ್ಲೀಷ್
ಇಂಟರ್ವ್ಯೂ
ಇಂಟ-ರ್ವ್ಯೂಗೆ
ಇಂತ
ಇಂತ-ವ-ರನ್ನು
ಇಂತಹ
ಇಂತ-ಹ-ವರ
ಇಂತ-ಹ-ವರು
ಇಂತ-ಹುದೇ
ಇಂತಿದೆ
ಇಂತು
ಇಂಥ
ಇಂಥ-ವರ
ಇಂಥ-ವ-ರನ್ನು
ಇಂಥಹ
ಇಂದಿಗೂ
ಇಂದಿನ
ಇಂದಿ-ನ-ವ-ರೆಗೂ
ಇಂದೀಗೂ
ಇಂದು
ಇಂದೂ
ಇಂದೇ
ಇಂದ್ರ-ಕೆ-ಫೆ-ಯಲ್ಲಿ
ಇಂದ್ರ-ಭ-ವ-ನ-ದಲ್ಲಿ
ಇಂದ್ರ-ವಿ-ಹಾರ
ಇಂದ್ರಿ-ಯ-ಗಳ
ಇಕ್ಕ-ಟ್ಟಾ-ಯಿತು
ಇಚ್ಛಿ-ಸು-ತ್ತೇನೆ
ಇಚ್ಛೆ
ಇಚ್ಛೆಯ
ಇಚ್ಛೆ-ಯನ್ನು
ಇಚ್ಛೆ-ಯಿಂ-ದಲೇ
ಇಟಗಿ
ಇಟ್ಟಿಗೆ
ಇಟ್ಟಿ-ದ್ದ-ರಂತೆ
ಇಟ್ಟಿರಿ
ಇಟ್ಟಿ-ರು-ತ್ತಾನೆ
ಇಟ್ಟಿ-ರು-ವ-ವ-ರಿಗೂ
ಇಟ್ಟು
ಇಟ್ಟು-ಕೊಂ-ಡ-ವ-ನಾ-ಗಿದ್ದು
ಇಟ್ಟು-ಕೊಂ-ಡಿ-ದ್ದಾಳೆ
ಇಟ್ಟು-ಕೊಂ-ಡಿಲ್ಲ
ಇಟ್ಟು-ಕೊಂಡು
ಇಟ್ಟು-ಕೊ-ಳ್ಳದೇ
ಇಡ-ಲಾ-ಗಿದೆ
ಇಡಲು
ಇಡಿ
ಇಡೀ
ಇಡೀ-ಮ-ನೆ-ತ-ನವೇ
ಇಡು-ವದು
ಇತರ
ಇತ-ರರ
ಇತ-ರ-ರನ್ನು
ಇತ-ರ-ರಿಗೆ
ಇತ-ರರು
ಇತರೆ
ಇತಿ
ಇತಿ-ಹಾ-ಸ-ಗ-ಳಲ್ಲಿ
ಇತಿ-ಹಾ-ಸ-ದಲ್ಲಿ
ಇತಿ-ಹಾ-ಸ-ದಲ್ಲೇ
ಇತಿ-ಹಾ-ಸ-ವನ್ನು
ಇತೋ-ಪ್ಯ-ತಿ-ಶ-ಯ-ವಾಗಿ
ಇತೋ-ಪ್ಯ-ತಿ-ಶ-ವ-ನ್ನುಂ-ಟು-ಮಾ-ಡಲಿ
ಇತ್ತ
ಇತ್ತಂತೆ
ಇತ್ತಿಂ-ದತ್ತ
ಇತ್ತೀ-ಚಿಗೆ
ಇತ್ತೀ-ಚಿನ
ಇತ್ತೀ-ಚೆಗೆ
ಇತ್ತು
ಇತ್ತೆಂದು
ಇತ್ಯಾದಿ
ಇತ್ಯಾ-ದಿ-ಗಳು
ಇತ್ಯಾ-ದಿ-ಯನ್ನು
ಇದಂತೂ
ಇದ-ಕ್ಕ-ನು-ಗು-ಣ-ವಾಗಿ
ಇದ-ಕ್ಕಾಗಿ
ಇದ-ಕ್ಕಾ-ಗಿಯೇ
ಇದಕ್ಕೂ
ಇದಕ್ಕೆ
ಇದ-ಕ್ಕೆಲ್ಲ
ಇದ-ಕ್ಕೆಲ್ಲಾ
ಇದ-ನ್ನ-ನು-ಸ-ರಿಸಿ
ಇದನ್ನು
ಇದನ್ನೆ
ಇದ-ನ್ನೆಲ್ಲ
ಇದ-ನ್ನೆಲ್ಲಾ
ಇದನ್ನೇ
ಇದರ
ಇದ-ರಂತೆ
ಇದ-ರಲ್ಲಿ
ಇದ-ರ-ಲ್ಲಿ-ರುವ
ಇದ-ರಲ್ಲೇ
ಇದ-ರಿಂದ
ಇದ-ರೆ-ಲ್ಲ-ದರ
ಇದ-ರೊಂ-ದಿಗೆ
ಇದ-ಲ್ಲದೆ
ಇದ-ಲ್ಲದೇ
ಇದಾ-ಗಿತ್ತು
ಇದಾ-ರಲ್ಲ
ಇದಾ-ವು-ದಕ್ಕೂ
ಇದೀಗ
ಇದೀ-ಗ-ತಾನೆ
ಇದು
ಇದು-ವ-ರೆಗೂ
ಇದೂ
ಇದೆ
ಇದೆಂಥ
ಇದೆಯೋ
ಇದೆ-ಲ್ಲ-ದರ
ಇದೇ
ಇದೇ-ನೆಂದು
ಇದೊಂದು
ಇದೊಂದೇ
ಇದ್ದ
ಇದ್ದಂ-ತಿ-ರ-ಲಿಲ್ಲ
ಇದ್ದಂತೆ
ಇದ್ದ-ಕೊಂಡು
ಇದ್ದ-ಕ್ಕಿ-ದ್ದಂತೆ
ಇದ್ದ-ದ್ದನ್ನು
ಇದ್ದದ್ದು
ಇದ್ದರು
ಇದ್ದರೂ
ಇದ್ದರೆ
ಇದ್ದ-ರೆಂದು
ಇದ್ದ-ರೇನು
ಇದ್ದ-ವರು
ಇದ್ದಾನೆ
ಇದ್ದಾರೆ
ಇದ್ದಾ-ರೆಂದು
ಇದ್ದಿದ್ದ
ಇದ್ದಿ-ದ್ದ-ರಿಂದ
ಇದ್ದು
ಇದ್ದೂ
ಇದ್ದೆ
ಇದ್ದೆವು
ಇದ್ದೇ
ಇದ್ದೇನೆ
ಇದ್ದೇವೆ
ಇದ್ರಿ-ಎಂದು
ಇನ್ನ
ಇನ್ನಷ್ಟು
ಇನ್ನಾವ
ಇನ್ನಾ-ವುದೇ
ಇನ್ನಾ-ವುದೋ
ಇನ್ನಿ-ತರ
ಇನ್ನಿ-ಲ್ಲದ
ಇನ್ನು
ಇನ್ನೂ
ಇನ್ನೆ-ರಡು
ಇನ್ನೆಲ್ಲೇ
ಇನ್ನೊಂದು
ಇನ್ನೊಂ-ದೆಡೆ
ಇನ್ನೊಬ್ಬ
ಇನ್ನೊ-ಬ್ಬ-ರಿಗೆ
ಇನ್ನೊ-ಬ್ಬರು
ಇನ್ನೊಮ್ಮೆ
ಇನ್ಮುಂ-ದೇನು
ಇನ್ಯಾ-ವುದೋ
ಇಪ್ಪ-ತ್ತೇ-ಳನೆ
ಇಬ್ಬ-ರಿಗೂ
ಇಬ್ಬ-ರಿ-ರಲಿ
ಇಬ್ಬರು
ಇಬ್ಬರೂ
ಇಮ್ಮ-ಡಿ-ಯಾ-ಯಿತು
ಇರದ
ಇರ-ಬ-ಹುದು
ಇರ-ಬ-ಹು-ದೇನೋ
ಇರ-ಬೇ-ಕಾ-ದರೆ
ಇರ-ಬೇಕು
ಇರ-ಲಾರ
ಇರ-ಲಾ-ರದು
ಇರಲಿ
ಇರ-ಲಿಲ್ಲ
ಇರ-ಲಿ-ಲ್ಲ-ವೆಂ-ದಲ್ಲ
ಇರಲೀ
ಇರಲು
ಇರ-ಲೆಂದು
ಇರ-ಲೆಂಬ
ಇರಲೇ
ಇರಿಸಿ
ಇರಿ-ಸಿ-ಕೊಂ-ಡ-ವರು
ಇರಿ-ಸಿ-ಕೊಂ-ಡಿ-ದ್ದೇನೆ
ಇರಿ-ಸಿ-ಕೊಂಡು
ಇರು-ತ್ತ-ದಲ್ಲ
ಇರು-ತ್ತದೆ
ಇರು-ತ್ತ-ದೆಂದು
ಇರು-ತ್ತ-ವಷ್ಟೆ
ಇರು-ತ್ತಾರೆ
ಇರು-ತ್ತಾಳೆ
ಇರು-ತ್ತಿತ್ತು
ಇರು-ತ್ತಿ-ದ್ದರು
ಇರು-ತ್ತಿ-ದ್ದ-ವರು
ಇರು-ತ್ತಿ-ದ್ದವು
ಇರು-ತ್ತಿ-ರ-ಲಿಲ್ಲ
ಇರುವ
ಇರು-ವಂ-ಥ-ವ-ರನ್ನು
ಇರು-ವಲ್ಲಿ
ಇರು-ವ-ವರು
ಇರು-ವಾಗ
ಇರು-ವುದು
ಇರು-ವು-ದು-ಇಂಥ
ಇರು-ವೆಗೆ
ಇಲಾ-ಖೆಗೆ
ಇಲಾ-ಖೆಯ
ಇಲಾ-ಖೆ-ಯಲ್ಲಿ
ಇಲ್ಲ
ಇಲ್ಲದ
ಇಲ್ಲ-ದಂ-ತಹ
ಇಲ್ಲ-ದಂ-ತಾ-ಯಿತು
ಇಲ್ಲ-ದಿ-ದ್ದರೆ
ಇಲ್ಲದೇ
ಇಲ್ಲ-ವಾದ
ಇಲ್ಲ-ವಾ-ಯಿತು
ಇಲ್ಲವೆ
ಇಲ್ಲ-ವೆಂದ
ಇಲ್ಲ-ವೆ-ನ್ನದೆ
ಇಲ್ಲ-ವೆ-ನ್ನದೇ
ಇಲ್ಲವೇ
ಇಲ್ಲವೇ
ಇಲ್ಲ-ವೇನೋ
ಇಲ್ಲವೋ
ಇಲ್ಲಾ
ಇಲ್ಲಿ
ಇಲ್ಲಿಗೆ
ಇಲ್ಲಿ-ದ್ದರೂ
ಇಲ್ಲಿದ್ದು
ಇಲ್ಲಿನ
ಇಲ್ಲಿ-ಯ-ವ-ರಿಗೂ
ಇಲ್ಲಿಯೆ
ಇಳಿ-ದರೆ
ಇಳಿದು
ಇಳಿ-ಬಿಟ್ಟು
ಇಳಿ-ಯು-ವ-ವರೇ
ಇಳಿ-ಸಿ-ಕೊಂಡು
ಇಳಿ-ಸಿ-ಕೊ-ಳ್ಳುವ
ಇಳಿ-ಸು-ವಾಗ
ಇಳೆ
ಇವ
ಇವ-ತ್ತಿಗೂ
ಇವ-ತ್ತಿನ
ಇವ-ತ್ತಿ-ನ-ವ-ರೆಗೂ
ಇವತ್ತು
ಇವನ
ಇವ-ನದು
ಇವ-ನನ್ನು
ಇವ-ನಲ್ಲಿ
ಇವ-ನಿಂದ
ಇವ-ನಿಂ-ದಾ-ಗಿದೆ
ಇವ-ನಿಗೂ
ಇವ-ನಿಗೆ
ಇವನು
ಇವ-ನ್ನೆಲ್ಲ
ಇವರ
ಇವ-ರ-ದಾ-ಗಿ-ರು-ತ್ತಿತ್ತು
ಇವ-ರದು
ಇವ-ರದೇ
ಇವ-ರದ್ದೆ
ಇವ-ರನ್ನು
ಇವ-ರಲ್ಲಿ
ಇವ-ರ-ಲ್ಲಿಲ್ಲ
ಇವ-ರಿಂದ
ಇವ-ರಿಗೆ
ಇವ-ರಿಗೇ
ಇವ-ರಿ-ದ್ದೆಡೆ
ಇವ-ರಿ-ಬ್ಬರೂ
ಇವ-ರೀ-ರ್ವರ
ಇವ-ರೀ-ರ್ವ-ರಿಗೂ
ಇವರು
ಇವ-ರು-ಗಳೂ
ಇವರೂ
ಇವ-ರೆ-ದು-ರಿಗೆ
ಇವ-ರೆ-ಲ್ಲರ
ಇವ-ರೆ-ಲ್ಲರೂ
ಇವರೇ
ಇವ-ರೊಂ-ದಿಗೆ
ಇವ-ರೋರ್ವ
ಇವಾ-ವವೂ
ಇವಾ-ವು-ದನ್ನೂ
ಇವು
ಇವು-ಗಳ
ಇವು-ಗ-ಳನ್ನು
ಇವು-ಗ-ಳ-ನ್ನೆಲ್ಲ
ಇವು-ಗ-ಳಲ್ಲಿ
ಇವು-ಗ-ಳಾ-ವುವೂ
ಇವು-ಗ-ಳಿಂದ
ಇವು-ಗ-ಳಿಗೆ
ಇವು-ಗಳು
ಇವು-ಗ-ಳೆಲ್ಲ
ಇವು-ಗ-ಳೆ-ಲ್ಲ-ದರ
ಇವು-ಗ-ಳೊಂ-ದಿಗೆ
ಇವೆ
ಇವೆ-ರ-ಡನ್ನು
ಇವೆ-ರ-ಡಿ-ದ್ದರೆ
ಇವೆ-ರಡು
ಇವೆ-ರಡೂ
ಇವೆಲ್ಲ
ಇವೆ-ಲ್ಲಕ್ಕೂ
ಇವೆ-ಲ್ಲ-ವನ್ನೂ
ಇವೆ-ಲ್ಲ-ವು-ಗಳ
ಇವೆ-ಲ್ಲವೂ
ಇವೆಲ್ಲಾ
ಇವೇ
ಇಷ್ಟ
ಇಷ್ಟಕ್ಕೂ
ಇಷ್ಟಕ್ಕೆ
ಇಷ್ಟಕ್ಕೇ
ಇಷ್ಟ-ಪ-ಟ್ಟ-ವ-ರಲ್ಲ
ಇಷ್ಟ-ಪ-ಡದ
ಇಷ್ಟ-ಪ-ಡುವ
ಇಷ್ಟ-ರಿಂದ
ಇಷ್ಟರು
ಇಷ್ಟ-ಲ್ಲದೇ
ಇಷ್ಟ-ವಾ-ಗ-ಲಾ-ರವು
ಇಷ್ಟ-ವಾ-ದಲ್ಲಿ
ಇಷ್ಟ-ವಿ-ರ-ಲಿಲ್ಲ
ಇಷ್ಟಾ-ಗಿಯೂ
ಇಷ್ಟಾ-ದರೂ
ಇಷ್ಟಾರ್ಥ
ಇಷ್ಟಾ-ರ್ಥ-ಸಿದ್ಧಿ
ಇಷ್ಟು
ಇಷ್ಟೆ
ಇಷ್ಟೆಲ್ಲ
ಇಷ್ಟೆಲ್ಲಾ
ಇಷ್ಟೇ
ಇಷ್ಟೊಂದು
ಇಸವಿ
ಇಸ-ವಿಯ
ಇಸ-ವಿ-ಯಲ್ಲಿ
ಇಹ-ಲೋಕ
ಈ
ಈಎ-ಸ್ವೆಂ-ಕ-ಣ್ಣಾ-ಚಾ-ರ್ಯರೇ
ಈಕ್ಷಿಸಿ
ಈಗ
ಈಗಲೂ
ಈಗಾ-ಗಲೇ
ಈಗಿನ
ಈಗಿ-ರುವ
ಈಗಿಲ್ಲ
ಈಗೀಗ
ಈಚೆ
ಈಜಾಡಿ
ಈಜಾ-ಡಿ-ದರೆ
ಈಜು
ಈಜು-ತ್ತಿದ್ದ
ಈತ
ಈರಿತಃ
ಈರ್ವರ
ಈಶ-ವ-ರ-ದಾ-ಚಾ-ರ್ಯರು
ಈಶ್ವ-ರನು
ಉಂಚಳ್ಳಿ
ಉಂಟಾ-ಗದೇ
ಉಂಟಾಗಿ
ಉಂಟಾ-ಗು-ತ್ತದೆ
ಉಂಟಾ-ಗು-ತ್ತ-ದೆಂದು
ಉಂಟಾ-ಗು-ತ್ತವೆ
ಉಂಟಾ-ಗುವ
ಉಂಟಾ-ಗು-ವು-ದನ್ನು
ಉಂಟಾ-ಗು-ವು-ದಿಲ್ಲ
ಉಂಟಾದ
ಉಂಟಾ-ದಾಗ
ಉಂಟು
ಉಂಟು-ಮಾ-ಡಿದೆ
ಉಂಟು-ಮಾ-ಡಿ-ರುವ
ಉಂಟು-ಮಾ-ಡು-ತ್ತದೆ
ಉಂಟು-ಮಾ-ಡು-ತ್ತಿದ್ದ
ಉಂಟು-ಮಾ-ಡುವ
ಉಂಡು-ಉ-ಟ್ಟು-ಕೊ-ಳ್ಳ್ಳಲು
ಉಕ
ಉಕ್ತಿ
ಉಕ್ತಿ-ಯೊಂದು
ಉಚಿ-ತ-ಕಾ-ಲ-ದಲ್ಲಿ
ಉಚಿ-ತ-ವ-ಲ್ಲವೆ
ಉಚಿ-ತ-ವಾಗಿ
ಉಚಿ-ತ-ವಾ-ಗಿಯೇ
ಉಚಿ-ತ-ವಾದ
ಉಚಿ-ತ-ವೆ-ನಿ-ಸು-ತ್ತದೆ
ಉಚ್ಚ-ರಿ-ಸು-ತ್ತಿ-ರು-ವು-ದಷ್ಟೇ
ಉಚ್ಚ-ರಿ-ಸು-ತ್ತೀರಿ
ಉಚ್ಚ-ರಿ-ಸುವ
ಉಚ್ಚಾರ
ಉಚ್ಚಾ-ರಣಾ
ಉಚ್ಚಾ-ರ-ಣಾಂ-ಗದ
ಉಚ್ಚಾ-ರ-ಣೆಯ
ಉಚ್ಚಾ-ರ-ಣೆ-ಯಿಂದ
ಉಚ್ಚೈಃ
ಉಚ್ಚೈ-ರ್ವಾ-ಚಿಕ
ಉಜ್ಜಾ-ಡಿ-ದರೆ
ಉಜ್ವ-ಲ-ವಾ-ಗು-ವಂತೆ
ಉಟ್ಟು
ಉಡುಪಿ
ಉಡು-ಪಿನ
ಉಡು-ಪಿ-ಯಲ್ಲಿ
ಉಣ-ಬ-ಡಿ-ಸಿ-ದ್ದಾರೆ
ಉಣ-ಬ-ಡಿ-ಸು-ತ್ತಿ-ದ್ದರು
ಉಣ-ಬ-ಡಿ-ಸುವ
ಉಣ್ಣಿ-ಸುವ
ಉತ್ಕ-ಟ-ವಾಗಿ
ಉತ್ಕ-ಟೇಚ್ಛೆ
ಉತ್ಕ-ರ್ಷಕ್ಕೆ
ಉತ್ಕೃಷ್ಟ
ಉತ್ಕೃ-ಷ್ಟ-ವಾ-ದುದು
ಉತ್ತಮ
ಉತ್ತ-ಮ-ರಾದ
ಉತ್ತ-ಮ-ವಾಗಿ
ಉತ್ತ-ಮ-ವಾ-ಗಿತ್ತು
ಉತ್ತ-ಮ-ವಾ-ಗಿದೆ
ಉತ್ತ-ಮ-ವಾ-ಗ್ಮಿ-ಗಳು
ಉತ್ತ-ಮ-ವಾದ
ಉತ್ತ-ಮ-ಶ್ರೇ-ಣಿ-ಯಲ್ಲಿ
ಉತ್ತ-ಮ-ಸ್ಥಾ-ನ-ದ-ಲ್ಲಿ-ದ್ದಾರೆ
ಉತ್ತ-ಮಾಂ-ಕ-ವನ್ನು
ಉತ್ತ-ಮಾ-ಧಿ-ಕಾ-ರಿ-ಗ-ಳಿಗೆ
ಉತ್ತ-ಮಾ-ಧಿ-ಕಾ-ರಿ-ಯಾಗಿ
ಉತ್ತರ
ಉತ್ತ-ರ-ಕ-ನ್ನಡ
ಉತ್ತ-ರ-ಕ-ನ್ನ-ಡ-ಜಿಲ್ಲೆ
ಉತ್ತ-ರ-ಕ-ನ್ನ-ಡ-ಜಿ-ಲ್ಲೆಯ
ಉತ್ತ-ರ-ಕ-ನ್ನ-ಡದ
ಉತ್ತ-ರ-ಕ-ನ್ನ-ಡ-ದಿಂದ
ಉತ್ತ-ರ-ಕ-ನ್ನ-ಡ-ಮೈ-ಸೂ-ರಿನ
ಉತ್ತ-ರ-ಜೀ-ವ-ನ-ವನ್ನು
ಉತ್ತ-ರ-ದಿಂದ
ಉತ್ತ-ರ-ವನ್ನು
ಉತ್ತ-ರ-ವ-ಯ-ಸ್ಸಿ-ನಲ್ಲಿ
ಉತ್ತ-ರ-ವಲ್ಲ
ಉತ್ತ-ರವು
ಉತ್ತ-ರವೂ
ಉತ್ತ-ರಾ-ಭಿ-ಮು-ಖ-ವಾಗಿ
ಉತ್ತ-ರಿ-ಸಲು
ಉತ್ತಿ-ರ್ಣ-ಗೊ-ಳಿ-ಸು-ವುದು
ಉತ್ತಿ-ರ್ಣ-ನಾ-ದ್ದ-ರಿಂದ
ಉತ್ತೀ-ರ್ಣ-ನಾ-ಗಲು
ಉತ್ತೀ-ರ್ಣ-ನಾಗಿ
ಉತ್ತೀ-ರ್ಣ-ನಾ-ಗಿದ್ದೆ
ಉತ್ತೀ-ರ್ಣ-ನಾ-ಗಿ-ದ್ದೆ-ನಷ್ಟೆ
ಉತ್ತೀ-ರ್ಣ-ನಾದ
ಉತ್ತೀ-ರ್ಣ-ನಾದೆ
ಉತ್ತೀ-ರ್ಣ-ರಾ-ದಿರಿ
ಉತ್ಸಾಹ
ಉತ್ಸಾ-ಹ-ಗ-ಳಿಗೆ
ಉತ್ಸಾ-ಹದ
ಉತ್ಸಾ-ಹ-ದಲ್ಲಿ
ಉತ್ಸಾ-ಹ-ದಿಂದ
ಉತ್ಸು-ಕ-ತೆ-ಯಿಂ-ದಲೇ
ಉತ್ಸು-ಕ-ರಾ-ಗಿ-ದ್ದರೂ
ಉದ-ಕ-ಶಾಂತಿ
ಉದ-ಯ-ವಾ-ಗಿದೆ
ಉದ-ಯ-ವಾ-ದದು
ಉದ-ಯ-ವಾ-ಯಿತು
ಉದರ
ಉದ-ರಂ-ಭ-ರ-ಣ-ಕ್ಕಾ-ಗಿಯೇ
ಉದ-ರಕ್ಕೆ
ಉದ-ರವು
ಉದಾ
ಉದಾ-ತ್ತ-ಬ-ಗೆ-ಯಿದು
ಉದಾರ
ಉದಾ-ರ-ತೆಯ
ಉದಾ-ರ-ವಾಗಿ
ಉದಾ-ರಿ-ಗ-ಳಾದ
ಉದಾ-ರಿ-ಗಳು
ಉದಾ-ಹ-ರಣೆ
ಉದಾ-ಹ-ರ-ಣೆ-ಗಳ
ಉದಾ-ಹ-ರ-ಣೆ-ಗ-ಳಿವೆ
ಉದಾ-ಹ-ರ-ಣೆ-ಗಳು
ಉದಾ-ಹ-ರ-ಣೆ-ಗಳೂ
ಉದಾ-ಹ-ರ-ಣೆಗೆ
ಉದಾ-ಹ-ರ-ಣೆ-ಯನ್ನು
ಉದಾ-ಹ-ರ-ಣೆ-ಯಾಗಿ
ಉದಾ-ಹ-ರ-ಣೆ-ಯಿ-ರ-ಬ-ಹುದು
ಉದಾ-ಹ-ರ-ಣೆಯೇ
ಉದುರಿ
ಉದ್ಗ-ರಿ-ಸಿಯೇ
ಉದ್ಗ್ರಂ-ಥ-ವನ್ನು
ಉದ್ಘಾ-ಟಿಸಿ
ಉದ್ದ-ನೆಯ
ಉದ್ದ-ವಾ-ಗಿದ್ದು
ಉದ್ದಾಮ
ಉದ್ದೇಶ
ಉದ್ದೇ-ಶ-ವನ್ನು
ಉದ್ದೇ-ಶ-ವನ್ನೂ
ಉದ್ದೇ-ಶ-ವಾ-ಗಲೀ
ಉದ್ದೇ-ಶ-ವು-ಳ್ಳ-ವ-ರಾ-ಗಿ-ದ್ದಾರೆ
ಉದ್ದೇ-ಶವೂ
ಉದ್ದೇ-ಶ-ವೇ-ನೆಂ-ದರೆ
ಉದ್ದೇ-ಶಿಸಿ
ಉದ್ಧಾ-ರ-ವನ್ನೂ
ಉದ್ಭವ
ಉದ್ಭ-ವ-ವಾದ
ಉದ್ಭ-ವ-ವಾ-ದವು
ಉದ್ಭ-ವಿ-ಸಿದ
ಉದ್ಯತಃ
ಉದ್ಯ-ತ-ರಾ-ಗಿ-ದ್ದರು
ಉದ್ಯ-ತ-ರಾದ
ಉದ್ಯ-ತ-ರಾ-ದ-ವರು
ಉದ್ಯ-ನ-ತ-ನಾ-ಗಿ-ರು-ತ್ತಾನೆ
ಉದ್ಯ-ಮಿ-ಗ-ಳ-ಲ್ಲ್ಲೊ-ಬ್ಬ-ನಾ-ಗಿ-ದ್ದಾನೆ
ಉದ್ಯೊ-ಗ-ಕ್ಕಾಗಿ
ಉದ್ಯೋಗ
ಉದ್ಯೋ-ಗ-ಕ್ಕಾಗಿ
ಉದ್ಯೋ-ಗಕ್ಕೆ
ಉದ್ಯೋ-ಗ-ದಿಂದ
ಉದ್ಯೋ-ಗ-ನಿ-ರ-ತ-ನಾ-ದಾಗ
ಉದ್ಯೋ-ಗ-ವ-ನ್ನ-ರಸಿ
ಉದ್ಯೋ-ಗ-ವನ್ನು
ಉದ್ಯೋ-ಗ-ವೇನೋ
ಉದ್ಯೋ-ಗಾ-ವ-ಕಾ-ಶ-ಗಳು
ಉದ್ಯೋ-ಗಿ-ಗ-ಳಾ-ಗಿದ್ದು
ಉದ್ಯೋ-ಗಿ-ಯಾಗಿ
ಉದ್ಯೋ-ಗಿ-ಯಾ-ಗಿದ್ದ
ಉದ್ಯೋ-ಗಿ-ಯಾದೆ
ಉದ್ರಿ-ಕ್ತ-ನಾ-ಗದ
ಉನ್ನತ
ಉನ್ನ-ತ-ಶಿ-ಕ್ಷಣ
ಉನ್ನ-ತಿ-ಗಾಗಿ
ಉನ್ನ-ತಿಗೆ
ಉಪ-ಕ-ರಿ-ಸ-ಬೇ-ಕೆಂಬ
ಉಪ-ಕ-ರಿಸಿ
ಉಪ-ಕ-ರಿ-ಸಿ-ದ್ದಾನೆ
ಉಪ-ಕ-ರಿ-ಸಿ-ದ್ದಾರೆ
ಉಪ-ಕ-ರಿ-ಸುವ
ಉಪ-ಕ-ರಿ-ಸು-ವುದು
ಉಪ-ಕ-ರಿ-ಸು-ವು-ದೆಂ-ದರೆ
ಉಪ-ಕಾರ
ಉಪ-ಕಾ-ರ-ಕ-ಗ-ಳಾ-ದು-ದ-ರಿಂದ
ಉಪ-ಕಾ-ರ-ಗ-ಳನ್ನು
ಉಪ-ಕಾ-ರ-ವಾ-ಗಿದೆ
ಉಪ-ಕೃ-ತ-ರಾ-ಗಿ-ದ್ದಾರೆ
ಉಪ-ಕೃ-ತ-ರಾ-ಗಿ-ದ್ದಾರೋ
ಉಪ-ಕೃ-ತ-ರಾ-ಗು-ತ್ತಾರೆ
ಉಪ-ಕೃ-ತ-ರಾದ
ಉಪ-ಕೃ-ತ-ರಾ-ದ-ವರು
ಉಪ-ಕೃ-ತರು
ಉಪ-ಚ-ರಿ-ಸದೇ
ಉಪ-ಚ-ರಿ-ಸಲು
ಉಪ-ಚ-ರಿಸಿ
ಉಪ-ಚ-ರಿ-ಸಿ-ದರು
ಉಪ-ಚ-ರಿ-ಸಿ-ದರೂ
ಉಪ-ಚ-ರಿ-ಸು-ತ್ತಿ-ದ್ದರು
ಉಪ-ಚ-ರಿ-ಸು-ವಲ್ಲಿ
ಉಪ-ಚಾರ
ಉಪ-ಚಾ-ರ-ವಿ-ಲ್ಲದೇ
ಉಪ-ದಿ-ಷ್ಟ-ವಾ-ಗಿ-ರ-ಬೇ-ಕೆಂ-ದರು
ಉಪ-ದೇಶ
ಉಪ-ದೇ-ಶ-ವಿ-ಲ್ಲದೆ
ಉಪ-ದೇ-ಶಿ-ಸುತ್ತಾ
ಉಪ-ದೇ-ಶಿ-ಸು-ತ್ತಾನೆ
ಉಪ-ದ್ರ-ವ-ವನ್ನು
ಉಪ-ನ-ಯನ
ಉಪ-ನ-ಯ-ನ-ಸಂ-ಸ್ಕಾ-ರ-ವನ್ನು
ಉಪ-ನಿ-ರ್ದೇ-ಶಕ
ಉಪ-ನಿ-ಷತ್
ಉಪ-ನಿ-ಷ-ತ್ತನ್ನು
ಉಪ-ನಿ-ಷತ್ತು
ಉಪ-ನೀ-ತ-ನಾಗಿ
ಉಪ-ನ್ಯಾಸ
ಉಪ-ನ್ಯಾ-ಸಕ
ಉಪ-ನ್ಯಾ-ಸ-ಕ-ನಾಗಿ
ಉಪ-ನ್ಯಾ-ಸ-ಕರ
ಉಪ-ನ್ಯಾ-ಸ-ಕ-ರಲ್ಲಿ
ಉಪ-ನ್ಯಾ-ಸ-ಕ-ರಾಗಿ
ಉಪ-ನ್ಯಾ-ಸ-ಕ-ರಾ-ಗಿ-ದ್ದ-ವರು
ಉಪ-ನ್ಯಾ-ಸ-ಕ-ರಾ-ಗಿಯೂ
ಉಪ-ನ್ಯಾ-ಸ-ಕ-ರಿ-ಗಾಗಿ
ಉಪ-ನ್ಯಾ-ಸ-ಕ-ರಿಗೂ
ಉಪ-ನ್ಯಾ-ಸ-ಕ-ರಿಗೆ
ಉಪ-ನ್ಯಾ-ಸ-ಕರು
ಉಪ-ನ್ಯಾ-ಸ-ಗಳ
ಉಪ-ನ್ಯಾ-ಸ-ಗ-ಳನ್ನು
ಉಪ-ನ್ಯಾ-ಸ-ಗಳು
ಉಪ-ನ್ಯಾ-ಸಾತ್
ಉಪ-ಮು-ಖ್ಯ-ಮಂತ್ರಿ
ಉಪ-ಮು-ಖ್ಯ-ಮಂ-ತ್ರಿ-ಯನ್ನು
ಉಪಮೆ
ಉಪ-ಯುಕ್ತ
ಉಪ-ಯು-ಕ್ತ-ವಾ-ದ-ವು-ಗಳು
ಉಪ-ಯೋ-ಗ-ವಾ-ಗುವ
ಉಪ-ಯೋ-ಗಿಸ
ಉಪ-ಯೋ-ಗಿ-ಸಲು
ಉಪ-ಯೋ-ಗಿಸಿ
ಉಪ-ಯೋ-ಗಿ-ಸು-ತ್ತಿ-ದ್ದೆವು
ಉಪ-ವಾಸ
ಉಪ-ಸ್ಥಾ-ಪನಂ
ಉಪ-ಸ್ಥಿತಿ
ಉಪ-ಹಾ-ರದ
ಉಪಾಂಶು
ಉಪಾಂ-ಶು-ವಿ-ಗಿಂತ
ಉಪಾಂ-ಶು-ವೆ-ನಿ-ಸು-ವುದು
ಉಪಾ-ದೇಯ
ಉಪಾ-ಧಿ-ಗ-ಳಾ-ವುವೂ
ಉಪಾ-ಧಿ-ವಂ-ತರೂ
ಉಪಾ-ಧ್ಯಾಯ
ಉಪಾ-ಧ್ಯಾ-ಯರ
ಉಪಾ-ಧ್ಯಾ-ಯ-ರನ್ನು
ಉಪಾ-ಧ್ಯಾ-ಯ-ರಾ-ಗಿ-ದ್ದರು
ಉಪಾ-ಧ್ಯಾ-ಯ-ರಾದ
ಉಪಾ-ಧ್ಯಾ-ಯ-ರಿಗೂ
ಉಪಾ-ಧ್ಯಾ-ಯ-ವರ್ಗ
ಉಪಾ-ಧ್ಯಾ-ಯ-ವೃಂ-ದ-ದಲ್ಲಿ
ಉಪಾ-ಧ್ಯಾ-ಯಾಶ್ಚ
ಉಪಾ-ನ್ಯಾಸ
ಉಪಾ-ಯ-ಗಳ
ಉಪಾ-ಯ-ಗ-ಳನ್ನೂ
ಉಪಾ-ಯ-ವನ್ನು
ಉಪಾ-ಯ-ವ-ನ್ನೊ-ಳ-ಗೊಂ-ಡಂ-ತಹ
ಉಪಾ-ಯ-ವೊಂ-ದನ್ನು
ಉಪಾ-ಸ-ಕರೇ
ಉಪಾ-ಸ-ನೆಯ
ಉಪಾ-ಹಾರ
ಉಪಾ-ಹಾ-ರ-ವನ್ನು
ಉಪ್ಪಿ-ಟ್ಟಿನ
ಉಭ-ಯ-ಜೀ-ವಿ-ಗಳು
ಉಭ-ಯ-ಥಾಪಿ
ಉಮಾ-ಕಾಂತ
ಉಮಾ-ಕಾಂ-ತ-ನಿಗೂ
ಉಮಾ-ಕಾಂ-ತನೂ
ಉಮಾ-ಕಾಂ-ತ-ಭ-ಟ್ಟರು
ಉಮೇಶ
ಉಮ್ಮ-ಚಗಿ
ಉಮ್ಮ-ಚ-ಗಿಯ
ಉರಿ
ಉರಿಸಿ
ಉಲಿಯೆ
ಉಲಿವ
ಉಲ್ಬ-ಣ-ವಾ-ಗು-ತ್ತದೆ
ಉಲ್ಲೇ-ಖ-ವಿ-ಲ್ಲದೇ
ಉಲ್ಲೇ-ಖಿ-ಸ-ಬ-ಹು-ದಾ-ಗಿದೆ
ಉಲ್ಲೇ-ಖಿ-ಸಲು
ಉಲ್ಲೇ-ಖಿ-ಸಿ-ದ್ದಾರೆ
ಉಲ್ಲೇ-ಖಿ-ಸಿ-ರು-ವ-ದ-ರಿಂದ
ಉಲ್ಲೇ-ಖಿ-ಸು-ವುದು
ಉಳಿದ
ಉಳಿ-ದಂತೆ
ಉಳಿ-ದ-ದ್ದೆ-ಲ್ಲ-ವನ್ನೂ
ಉಳಿ-ದ-ವರ
ಉಳಿ-ದಿದೆ
ಉಳಿ-ದಿ-ದ್ದಾರೆ
ಉಳಿ-ದಿ-ದ್ದಾರೆ
ಉಳಿದು
ಉಳಿದೂ
ಉಳಿದೆ
ಉಳಿ-ಪೆ-ಟ್ಟಿ-ನಿಂದ
ಉಳಿ-ಯ-ಬೇ-ಕೆ-ನ್ನುವ
ಉಳಿ-ಯಲು
ಉಳಿ-ಯ-ಲೊಂದು
ಉಳಿ-ಯಿತು
ಉಳಿವ
ಉಳಿ-ವಿ-ಗಾಗಿ
ಉಳಿ-ವು-ಬೆ-ಳ-ವ-ಣಿಗೆ
ಉಳಿ-ಸಿ-ಕೊಂ-ಡಿದ್ದೆ
ಉಳಿ-ಸಿ-ಕೊಂ-ಡಿ-ರಿ-ವುದೂ
ಉಳಿ-ಸಿ-ಕೊಂಡು
ಉಳಿ-ಸಿದ
ಉಳಿ-ಸು-ವುದು
ಉಳ್ಳ-ವ-ರಾ-ಗಿ-ದ್ದರೂ
ಉಳ್ಳ-ವರು
ಉಸು-ರಿ-ದ್ದೆಲ್ಲ
ಉಸ್ತು-ವಾರಿ
ಉಸ್ತು-ವಾ-ರಿ-ಯನ್ನು
ಊಟ
ಊಟ-ಕ್ಕಾಗಿ
ಊಟಕ್ಕೆ
ಊಟ-ಗ-ಳಿಂದ
ಊಟದ
ಊಟ-ದಲ್ಲಿ
ಊಟ-ದ-ಲ್ಲಿಯೂ
ಊಟ-ಮಾ-ಡಿಟ್ಟು
ಊಟ-ಮಾ-ಡಿ-ದ-ವನು
ಊಟ-ಮಾ-ಡಿ-ಸಿ-ದರು
ಊಟ-ಮಾ-ಡು-ತ್ತಿದ್ದ
ಊಟ-ವನ್ನು
ಊಟ-ವಾ-ಗಿದ್ದೂ
ಊಟೋ-ಪ-ಚಾ-ರ-ಗಳ
ಊಟೋ-ಪ-ಚಾ-ರದ
ಊರನ್ನು
ಊರಲ್ಲ
ಊರಲ್ಲೇ
ಊರಾದ
ಊರಿಂದ
ಊರಿಂ-ದಾ-ಗಲಿ
ಊರಿಗೆ
ಊರಿಗೇ
ಊರಿನ
ಊರಿ-ನ-ಕ-ಡೆಯ
ಊರಿ-ನಲ್ಲಿ
ಊರಿ-ನಲ್ಲೇ
ಊರಿ-ನ-ವರ
ಊರಿ-ನ-ವ-ರಾ-ದರೂ
ಊರಿ-ನ-ವ-ರಿಗೆ
ಊರಿ-ನ-ವರು
ಊರಿ-ನಿಂದ
ಊರು
ಊರು-ಗ-ಳಲ್ಲಿ
ಊರೂರು
ಊರ್ಜಿ-ತ-ಗೊ-ಳಿ-ಸುವ
ಊಹಿ-ಸ-ಬ-ಹುದು
ಊಹಿ-ಸ-ಬೇಕು
ಊಹಿ-ಸಿ-ಕೊ-ಳ್ಳ-ಬ-ಹು-ದಾ-ಗಿದೆ
ಊಹಿ-ಸಿಯೂ
ಊಹೆ
ಋಗ್
ಋಣ
ಋಣ-ದಲ್ಲಿ
ಋಣ-ಭಾ-ರವೇ
ಋಣಿ-ಗ-ಳಾ-ಗಿ-ದ್ದೇವೆ
ಋಣಿ-ಗಳು
ಋಣಿ-ಯಾ-ಗಿ-ದ್ದೇನೆ
ಋಷಿ
ಋಷಿ-ಗಳು
ಋಷಿ-ವಾ-ಕ್ಯ-ದೊ-ಡನೆ
ಎ
ಎಂ
ಎಂಎ
ಎಂಎ-ವಿ-ದ್ವತ್
ಎಂಟು
ಎಂಟೂ
ಎಂಡ್
ಎಂಡ್ರ್ಯೂ
ಎಂತ
ಎಂತಕ್
ಎಂತದ್ದೇ
ಎಂತ-ಲ್ಲವೇ
ಎಂತಹ
ಎಂತ-ಹದ್ದು
ಎಂತು
ಎಂಥ
ಎಂಥ-ದ್ದಾ-ಗಿ-ತ್ತೆಂ-ದರೆ
ಎಂಥ-ವರ
ಎಂಥ-ವ-ರಿಗೂ
ಎಂದ
ಎಂದಂತೆ
ಎಂದ-ಮೇಲೆ
ಎಂದರು
ಎಂದರೂ
ಎಂದರೆ
ಎಂದ-ರ್ಥ-ವ-ಲ್ಲ-ಆಗ
ಎಂದಲ್ಲ
ಎಂದಾ-ಕ್ಷಣ
ಎಂದಾಗ
ಎಂದಾ-ದರೂ
ಎಂದಾ-ದರೆ
ಎಂದಾ-ದಲ್ಲಿ
ಎಂದಾ-ಯಿತು
ಎಂದಿಗೂ
ಎಂದಿದೆ
ಎಂದಿದ್ದ
ಎಂದಿ-ದ್ದ-ನಂತೆ
ಎಂದಿ-ದ್ದರು
ಎಂದಿ-ದ್ದಾರೆ
ಎಂದಿ-ನಂತೆ
ಎಂದಿ-ರು-ತ್ತಾರೆ
ಎಂದಿ-ರು-ವು-ದ-ರಿಂದ
ಎಂದು
ಎಂದು-ಕೊಂ-ಡಿ-ದ್ದೇನೆ
ಎಂದೂ
ಎಂದೆ
ಎಂದೆಂ-ದಿಗೂ
ಎಂದೆಲ್ಲಾ
ಎಂದೇ
ಎಂಬ
ಎಂಬಂ-ತಹ
ಎಂಬಂ-ತಾ-ಗಿದೆ
ಎಂಬಂ-ತಿ-ರುವ
ಎಂಬಂತೆ
ಎಂಬಂ-ತೆಯೂ
ಎಂಬಲ್ಲಿ
ಎಂಬ-ಲ್ಲಿಗೆ
ಎಂಬ-ವರು
ಎಂಬು-ದಕ್ಕೆ
ಎಂಬು-ದನ್ನು
ಎಂಬು-ದನ್ನೂ
ಎಂಬು-ದನ್ನೇ
ಎಂಬು-ದ-ರಲ್ಲಿ
ಎಂಬು-ದಾಗಿ
ಎಂಬು-ದಾ-ಗಿದೆ
ಎಂಬು-ದಾ-ಗಿಯೇ
ಎಂಬುದು
ಎಂಬುದೇ
ಎಂಬು-ದೊಂದು
ಎಂಬು-ವ-ವ-ರಿ-ದ್ದಾರೆ
ಎಂಬು-ವುದು
ಎಆ-ರ್ನಾ-ಗ-ಭೂ-ಷಣ
ಎಎಂ
ಎಗ್ಗಿ-ಲ್ಲದ
ಎಚ್
ಎಚ್ಚತ್ತ
ಎಚ್ಚರ
ಎಚ್ಚ-ರಿ-ಕೆಯ
ಎಚ್ಚ-ರಿ-ಕೆ-ಯನ್ನೂ
ಎಚ್ವಿ-ನಾ-ಗ-ರಾ-ಜ-ರಾವ್
ಎಡ-ಪಾ-ರ್ಶ್ವ-ದಲ್ಲಿ
ಎಡರು
ಎಡ-ವು-ತ್ತೇವೆ
ಎಡ್ವ-ರ್ಡ್
ಎಣಿಕೆ
ಎಣಿ-ಸ-ದಿ-ದ್ದರೆ
ಎಣಿ-ಸಲೇ
ಎಣಿ-ಸು-ವಾಗ
ಎಣ್ಣೆ
ಎತ್ತ-ರಕ್ಕೆ
ಎತ್ತ-ರದ
ಎತ್ತ-ಲಾ-ಗದ
ಎತ್ತಲು
ಎತ್ತಿ
ಎದು-ರಾ-ಗಿತ್ತು
ಎದು-ರಾ-ಗು-ತ್ತದೆ
ಎದು-ರಾ-ಗು-ತ್ತಿ-ದ್ದವು
ಎದು-ರಾದ
ಎದು-ರಾ-ದರೂ
ಎದು-ರಾ-ಳಿ-ಗಳು
ಎದು-ರಿಗೆ
ಎದು-ರಿ-ನಲ್ಲಿ
ಎದು-ರಿ-ನ-ಲ್ಲಿ-ರುವ
ಎದು-ರಿ-ಸ-ಬೇ-ಕಾ-ಯಿತು
ಎದು-ರಿ-ಸ-ಲೇ-ಬೇಕು
ಎದು-ರಿಸಿ
ಎದು-ರಿ-ಸಿ-ದ-ವರು
ಎದು-ರಿ-ಸಿದ್ದೆ
ಎದು-ರಿ-ಸಿ-ದ್ದೇನೆ
ಎದು-ರಿ-ಸಿ-ರು-ವುದೂ
ಎದು-ರಿ-ಸು-ತ್ತಿದ್ದ
ಎದು-ರಿ-ಸುವ
ಎದುರು
ಎದೆ
ಎದೆ-ಗುಂ-ದದೆ
ಎದ್ದು
ಎನಿ-ಸಿ-ಕೊಂ-ಡನು
ಎನಿ-ಸಿ-ಕೊಂ-ಡ-ವ-ರಾಗಿ
ಎನಿ-ಸಿ-ಕೊಂ-ಡಿ-ದ್ದಾರೆ
ಎನಿ-ಸಿ-ಕೊಂಡು
ಎನಿ-ಸಿ-ದರೂ
ಎನಿ-ಸಿದ್ದು
ಎನಿ-ಸು-ತ್ತದೆ
ಎನಿ-ಸು-ತ್ತಿತ್ತು
ಎನೋ
ಎನ್
ಎನ್-ಎಸ್
ಎನ್-ಎ-ಸ್ರಾ-ಮ-ಭ-ದ್ರಾ-ಚಾ-ರ್ಯರ
ಎನ್-ಎ-ಸ್ರಾ-ಮ-ಭ-ದ್ರಾ-ಚಾ-ರ್ಯ-ರಿಂದ
ಎನ್-ಎ-ಸ್ರಾ-ಮ-ಭ-ದ್ರಾ-ಚಾ-ರ್ಯರು
ಎನ್ನ-ಬ-ಹು-ದಾದ
ಎನ್ನ-ಬ-ಹು-ದಿ-ತ್ತೇನೋ
ಎನ್ನ-ಬ-ಹುದು
ಎನ್ನ-ಲಾ-ಗದು
ಎನ್ನ-ಲಿಲ್ಲ
ಎನ್ನಲು
ಎನ್ನಿ-ಸು-ತ್ತಿತ್ತು
ಎನ್ನು-ತ್ತಾರೆ
ಎನ್ನು-ತ್ತೇವೆ
ಎನ್ನುವ
ಎನ್ನು-ವಷ್ಟು
ಎನ್ನು-ವು-ದಂತೂ
ಎನ್ನು-ವು-ದಕ್ಕೆ
ಎನ್ನು-ವು-ದನ್ನು
ಎನ್ನು-ವುದು
ಎನ್ನು-ವುದೂ
ಎನ್ನು-ವುದೇ
ಎನ್ಡೇಂ-ಜ-ರ್ಡ್
ಎಪ್
ಎಪ್ಲ-ರ್ರ-ವರು
ಎಮ್
ಎಮ್-ಅ-ರ್ಹೆ-ಗಡೆ
ಎಮ್ಎ
ಎಮ್-ಐಟಿ
ಎರ-ಡನೆ
ಎರ-ಡ-ನೆಯ
ಎರ-ಡ-ನೆ-ಯದು
ಎರ-ಡ-ನೆ-ಯ-ವ-ರಾದ
ಎರ-ಡನೇ
ಎರ-ಡನ್ನೂ
ಎರ-ಡ-ರ-ಲ್ಲಿಯೂ
ಎರಡು
ಎರಡು
ಎರಡೂ
ಎರಡೇ
ಎರೆದ
ಎರೆ-ದಿ-ದ್ದಾನೆ
ಎರೆ-ದಿ-ದ್ದಾರೆ
ಎಲೆ-ಕ್ಟ್ರಿ-ಕಲ್
ಎಲೆ-ಗ-ಳನ್ನು
ಎಲೆ-ಮ-ರೆಯ
ಎಲ್
ಎಲ್ಲ
ಎಲ್ಲ-ಕ್ಕಿಂತ
ಎಲ್ಲ-ದಕ್ಕೂ
ಎಲ್ಲ-ದರ
ಎಲ್ಲ-ದ-ರಲ್ಲೂ
ಎಲ್ಲರ
ಎಲ್ಲ-ರನ್ನು
ಎಲ್ಲ-ರನ್ನೂ
ಎಲ್ಲ-ರಲ್ಲೂ
ಎಲ್ಲ-ರಿಗೂ
ಎಲ್ಲರೂ
ಎಲ್ಲ-ರೊಂ-ದಿಗೂ
ಎಲ್ಲ-ರೊಂ-ದಿಗೆ
ಎಲ್ಲ-ವನ್ನು
ಎಲ್ಲ-ವನ್ನೂ
ಎಲ್ಲವೂ
ಎಲ್ಲಾ
ಎಲ್ಲಾ-ದರೂ
ಎಲ್ಲಿ
ಎಲ್ಲಿಂ-ದಾ-ಗಲಿ
ಎಲ್ಲಿಗೂ
ಎಲ್ಲಿಗೆ
ಎಲ್ಲಿದೆ
ಎಲ್ಲಿ-ದ್ದರೂ
ಎಲ್ಲಿಯೂ
ಎಲ್ಲಿಯೇ
ಎಲ್ಲಿಯೋ
ಎಲ್ಲಿ-ಲ್ಲದ
ಎಲ್ಲೂ
ಎಲ್ಲೆಡೆ
ಎಲ್ಲೆ-ಯನ್ನು
ಎಲ್ಲೆಲ್ಲೋ
ಎಲ್ಲೇ
ಎಲ್ಲೋ
ಎಳ-ವೆಯ
ಎಳೆದು
ಎಳೆಯ
ಎಳೆ-ವ-ಯ-ಸ್ಸಿ-ನ-ವ-ರನ್ನು
ಎವ-ಯಿ-ಕ್ಕದೇ
ಎಷ್ಟಿ-ತ್ತೆಂ-ದರೆ
ಎಷ್ಟು
ಎಷ್ಟೂ
ಎಷ್ಟೆಂ-ದರೆ
ಎಷ್ಟೆಷ್ಟೋ
ಎಷ್ಟೇ
ಎಷ್ಟೊ
ಎಷ್ಟೊಂದು
ಎಷ್ಟೋ
ಎಸೆ-ದರು
ಎಸೆದೆ
ಎಸ್
ಎಸ್-ಎ-ಸ್-ಎಲ್ಸಿ
ಎಸ್-ಎ-ಸ್-ಎ-ಲ್ಸಿಯ
ಎಸ್ಸೆ-ಸ್ಸೆ-ಲ್ಸಿ-ಯನ್ನು
ಏಕ
ಏಕಃ
ಏಕ-ಕಾ-ಲ-ದಲ್ಲಿ
ಏಕತೆ
ಏಕತ್ರ
ಏಕ-ಪಾ-ತ್ರಾ-ಭಿ-ನಯ
ಏಕ-ಲ-ವ್ಯ-ನಂತೆ
ಏಕ-ವ-ಚ-ನ-ದಲ್ಲಿ
ಏಕ-ಸಂ-ಬಂ-ಧಿ-ಜ್ಞಾ-ನಮ್
ಏಕಾಂ-ಗಿ-ಯಾಗಿ
ಏಕಾ-ಗ್ರ-ತೆ-ಯಿಂದ
ಏಕಾ-ದಶೀ
ಏಕೆ
ಏಕೆಂ-ದರೆ
ಏಕೈಕ
ಏತ-ನ್ಮಧ್ಯೆ
ಏನ-ನಿ-ಸಿತ್ತೋ
ಏನ-ನ್ನಾ-ದರು
ಏನನ್ನು
ಏನನ್ನೂ
ಏನನ್ನೇ
ಏನಾ-ಗ-ಬೇಕು
ಏನಾ-ದರೂ
ಏನಾ-ಯಿ-ತೆಂದು
ಏನಿದು
ಏನು
ಏನೂ
ಏನೆಂ-ದಿರಿ
ಏನೆಂದು
ಏನೆಂ-ಬುದು
ಏನೇ
ಏನೇ-ನನ್ನು
ಏನೇ-ನಿದೆ
ಏನೇನೂ
ಏನೋ
ಏಪ್ರಿ-ಲ್ನಲ್ಲಿ
ಏರಿಯೇ
ಏರು-ಧ್ವ-ನಿ-ಯಲ್ಲಿ
ಏರು-ಪೇ-ರಾ-ಗಿ-ದ್ದರೂ
ಏರ್ಪ-ಟ್ಟಿತು
ಏರ್ಪ-ಟ್ಟಿ-ರು-ವುದು
ಏರ್ಪ-ಡಿಸಿ
ಏರ್ಪ-ಡಿ-ಸಿತ್ತು
ಏರ್ಪ-ಡಿ-ಸಿದ
ಏರ್ಪ-ಡಿ-ಸಿ-ದಾಗ
ಏರ್ಪ-ಡಿ-ಸಿ-ದ್ದರು
ಏರ್ಪ-ಡಿ-ಸು-ತ್ತಿ-ದ್ದೆವು
ಏರ್ಪ-ಡಿ-ಸುವ
ಏರ್ಪ-ಡಿ-ಸು-ವು-ದ-ರಲ್ಲಿ
ಏರ್ಪಾ-ಡಾ-ಗಿತ್ತು
ಏರ್ಪಾ-ಡಾ-ಯಿತು
ಏಳ-ನೆಯ
ಏಳಿ-ಗೆ-ಯನ್ನು
ಏಳು
ಏಳೆಂಟು
ಏಳ್ಗೆಗೆ
ಏಳ್ಗೆ-ಯನ್ನು
ಏವ
ಏಷ್ಯಾ
ಐಇಸಿ
ಐಚ್ಛಿ-ಕ-ವಾಗಿ
ಐತಾ-ಳರು
ಐತಿ-ಹಾ-ಸಿಕ
ಐದಾರು
ಐದು
ಐಯ್ಯಂ-ಗಾ-ರ್ಯರ
ಐಶ್ವರ್ಯ
ಐಷಾ-ರಾಮಿ
ಐಸ್ಕ್ರೀಮ್
ಒಂಚೂರು
ಒಂಟಿ-ಯಾ-ಗಿ-ರು-ವಷ್ಟೂ
ಒಂಥರ
ಒಂದನ್ನು
ಒಂದ-ರ್ಥ-ದಲ್ಲಿ
ಒಂದಲ್ಲ
ಒಂದಲ್ಲಾ
ಒಂದಾ-ಗಿತ್ತು
ಒಂದಾ-ದರೆ
ಒಂದಿ-ಬ್ಬರು
ಒಂದಿ-ಲ್ಲೊಂದು
ಒಂದಿಷ್ಟು
ಒಂದು
ಒಂದು
ಒಂದು-ಕ್ಷ-ಣವೂ
ಒಂದು-ವಾರ
ಒಂದು-ವೇಳೆ
ಒಂದೂ
ಒಂದೂ-ವರೆ
ಒಂದೆ
ಒಂದೆ-ಡೆ-ಯಾ-ದರೆ
ಒಂದೆ-ರ-ಡನ್ನು
ಒಂದೆ-ರಡು
ಒಂದೇ
ಒಂದೇ-ಎಂ-ಬುದು
ಒಂದೇ-ಎ-ರಡೂ
ಒಂದೊಂದು
ಒಂದೊಮ್ಮೆ
ಒಂದೋ
ಒಂಬ-ತ್ತ-ನೆಯ
ಒಂಬತ್ತು
ಒಂಭ-ತ್ತನೆ
ಒಕ್ಕೊ-ರ-ಲಿಂದ
ಒಕ್ಕೊ-ರ-ಲಿ-ನಿಂದ
ಒಗ್ಗಿತು
ಒಗ್ಗು-ತ್ತಿ-ರ-ಲಿಲ್ಲ
ಒಗ್ಗೂಡಿ
ಒಟ್ಟಾಗಿ
ಒಟ್ಟಾರೆ
ಒಟ್ಟಾ-ರೆ-ಯಾಗಿ
ಒಟ್ಟಿಗೆ
ಒಟ್ಟಿ-ನಲ್ಲಿ
ಒಟ್ಟು
ಒಟ್ಟೊ-ಟ್ಟಿಗೆ
ಒಡ-ನಾಟ
ಒಡ-ನಾ-ಟ-ಗಳು
ಒಡ-ನಾ-ಟದ
ಒಡ-ನಾ-ಟ-ದಲ್ಲಿ
ಒಡ-ನಾ-ಟ-ವಿದೆ
ಒಡ-ನಾ-ಟವೇ
ಒಡ-ನಾ-ಡಿ-ಗ-ಳಾ-ಗಿ-ದ್ದರು
ಒಡ-ನಾ-ಡಿ-ಗಳು
ಒಡ-ಮೂ-ಡಿತು
ಒಡ-ಹು-ಟ್ಟಿ-ದ-ವರು
ಒತ್ತಡ
ಒತ್ತ-ಡವೇ
ಒತ್ತ-ಡ-ಹಾಕಿ
ಒತ್ತನ್ನು
ಒತ್ತ-ರಿಸಿ
ಒತ್ತಾಯ
ಒತ್ತಾ-ಯಕ್ಕೆ
ಒತ್ತಾಸೆ
ಒತ್ತಾ-ಸೆಗೆ
ಒತ್ತಾ-ಸೆಯ
ಒತ್ತಾ-ಸೆ-ಯನ್ನು
ಒತ್ತು
ಒತ್ತು-ಕೊಟ್ಟು
ಒತ್ತು-ನೀ-ಡ-ಲಾ-ಗಿದೆ
ಒದ-ಗಲಿ
ಒದಗಿ
ಒದ-ಗಿತು
ಒದ-ಗಿತ್ತು
ಒದ-ಗಿ-ದಾಗ
ಒದ-ಗಿ-ಬಂದ
ಒದ-ಗಿ-ಸ-ಬ-ಹುದೇ
ಒದ-ಗಿಸಿ
ಒದ-ಗಿ-ಸಿ-ಕೊ-ಟ್ಟಿ-ದ್ದಾರೆ
ಒದ-ಗಿ-ಸಿತು
ಒದ-ಗಿ-ಸಿದ
ಒದ-ಗಿ-ಸಿ-ದ್ದಾರೆ
ಒದ-ಗಿ-ಸು-ತ್ತಿ-ದ್ದ-ವರು
ಒದ್ದೆ-ಯಾ-ದವು
ಒಪ್ಪದ
ಒಪ್ಪ-ಲಾ-ರಂ-ಭಿ-ಸಿ-ದ್ದಾರೆ
ಒಪ್ಪ-ಲಿ-ಲ್ಲ-ಮುಂದೆ
ಒಪ್ಪಲೇ
ಒಪ್ಪಿ
ಒಪ್ಪಿ-ಕೊಂಡ
ಒಪ್ಪಿ-ಕೊಂ-ಡರು
ಒಪ್ಪಿ-ಕೊಂ-ಡಿದೆ
ಒಪ್ಪಿ-ಕೊಂಡು
ಒಪ್ಪಿ-ಕೊಂಡೆ
ಒಪ್ಪಿ-ಕೊ-ಳ್ಳು-ವು-ದಿಲ್ಲ
ಒಪ್ಪಿಗೆ
ಒಪ್ಪಿದ
ಒಪ್ಪಿ-ದರು
ಒಪ್ಪಿದೆ
ಒಪ್ಪಿಸಿ
ಒಪ್ಪಿ-ಸು-ತ್ತಿ-ದ್ದ-ರೆಂ-ಬುದು
ಒಪ್ಪಿ-ಸುವ
ಒಪ್ಪಿ-ಸು-ವುದು
ಒಪ್ಪು-ತ್ತೀರಾ
ಒಪ್ಪುವ
ಒಪ್ಪು-ವರು
ಒಪ್ಪು-ವರೋ
ಒಬ್ಬ
ಒಬ್ಬ-ನಾ-ಗಿ-ದ್ಧೇನೆ
ಒಬ್ಬ-ನಾ-ಗಿ-ಬಿ-ಟ್ಟಿದ್ದೆ
ಒಬ್ಬನು
ಒಬ್ಬ-ನೆಂದು
ಒಬ್ಬ-ರಾದ
ಒಬ್ಬ-ರಿಗೆ
ಒಬ್ಬ-ರಿ-ರಲಿ
ಒಬ್ಬರು
ಒಬ್ಬಿಬ್ಬ
ಒಬ್ಬೊ-ಬ್ಬರು
ಒಮ್ಮೆ
ಒಮ್ಮೆ-ಯಂತೂ
ಒಮ್ಮೆ-ಯಾ-ದರೂ
ಒಮ್ಮೆಲೆ
ಒಯ್ಯುವ
ಒಯ್ಯು-ವ-ವ-ನನ್ನು
ಒರೆಸಿ
ಒರೆ-ಸಿ-ಹಾ-ಕಿತ್ತು
ಒಲವ
ಒಲ-ವಿನ
ಒಲಿ-ದಿದೆ
ಒಲಿ-ದಿ-ದ್ದಾ-ನೆಂದು
ಒಳ
ಒಳ-ಗಾ-ಗ-ದ-ವರು
ಒಳ-ಗಾ-ಗು-ವ-ವ-ರಲ್ಲ
ಒಳಗೆ
ಒಳ-ಗೊಂ-ಡಿದೆ
ಒಳಿ-ತನ್ನು
ಒಳಿ-ತನ್ನೆ
ಒಳಿ-ತಿ-ಗಾ-ಗಿಯೂ
ಒಳಿ-ತಿಗೆ
ಒಳಿ-ತಿ-ನಲ್ಲಿ
ಒಳ್ಳೆ
ಒಳ್ಳೆಯ
ಒಳ್ಳೆ-ಯದು
ಒಳ್ಳೆ-ಯದೇ
ಒಳ-ಹೊ-ರ-ಗಿನ
ಓ
ಓಡ-ನಾಟ
ಓಡಾ-ಡು-ತ್ತಿ-ದ್ದು-ದುಂಟು
ಓದದ
ಓದದೇ
ಓದನ್ನು
ಓದ-ಬ-ರುವ
ಓದ-ಬೇಕು
ಓದ-ಬೇ-ಕೆಂಬ
ಓದಲು
ಓದಿ
ಓದಿ-ಕೊಂ-ಡಿದ್ದ
ಓದಿದ
ಓದಿ-ದರು
ಓದಿ-ದರೆ
ಓದಿ-ದ-ವನು
ಓದಿ-ದ-ವ-ರಲ್ಲಿ
ಓದಿ-ದ-ವರು
ಓದಿ-ದಷ್ಟು
ಓದಿದೆ
ಓದಿದ್ದ
ಓದಿ-ದ್ದ-ರಿಂದ
ಓದಿ-ದ್ದಾನೆ
ಓದಿ-ದ್ದೀರಿ
ಓದಿದ್ದು
ಓದಿದ್ದೆ
ಓದಿನ
ಓದಿ-ರ-ಲಿಲ್ಲ
ಓದಿ-ರು-ತ್ತಾರೆ
ಓದಿ-ರು-ವರು
ಓದಿ-ರುವೆ
ಓದಿ-ರು-ವೆಯಾ
ಓದಿ-ಸಿ-ದರು
ಓದಿ-ಸಿ-ದ-ವರು
ಓದಿ-ಸಿ-ದ್ದಾನೆ
ಓದಿಸು
ಓದಿ-ಸ್ತೀನಿ
ಓದು
ಓದು-ತ್ತ-ಲಿದ್ದೆ
ಓದು-ತ್ತಿದ್ದ
ಓದು-ತ್ತಿ-ದ್ದ-ವರು
ಓದು-ತ್ತಿ-ದ್ದಾಗ
ಓದು-ತ್ತಿ-ದ್ದಾ-ಗಲೇ
ಓದು-ತ್ತಿದ್ದು
ಓದು-ತ್ತಿದ್ದೆ
ಓದು-ತ್ತಿ-ದ್ದೇನೆ
ಓದು-ತ್ತಿ-ರುವ
ಓದುವ
ಓದು-ವಂ-ತಾ-ಯಿತು
ಓದು-ವಂ-ತೆಯೂ
ಓದು-ವ-ವ-ರಿ-ಗಂತೂ
ಓದು-ವ-ವ-ರಿಗೆ
ಓದು-ವಾಗ
ಓದು-ವಾ-ಗಲೂ
ಓದು-ವು-ದಕ್ಕೆ
ಓದು-ವು-ದ-ರಲ್ಲೂ
ಓದೋ-ದಕ್ಕೆ
ಓರ್ವ
ಓಲೈ-ಸುವ
ಔತಣ
ಔತ-ಣದ
ಔತ್ಸುಕ್ಯ
ಔತ್ಸು-ಕ್ಯ-ಮೋ-ಹಾ-ರ-ತಿ-ದಾನ್
ಔದಾರ್ಯ
ಔದಾ-ರ್ಯ-ದಿಂದ
ಔದಾ-ರ್ಯಾದಿ
ಔದ್ಯೋ-ಗಿ-ಕ-ವಾಗಿ
ಔನ್ನ-ತ್ಯ-ವನ್ನು
ಔಷ-ಧವು
ಔಷಧಿ
ಔಷ-ಧಿ-ಗ-ಳಿಗೆ
ಕಂಗಳ
ಕಂಗಾ-ಲಾದ
ಕಂಗಾ-ಲಾ-ದರು
ಕಂಗೊ-ಳಿ-ಸಿದ
ಕಂಗೊ-ಳಿ-ಸು-ತ್ತಿವೆ
ಕಂಗೊ-ಳಿ-ಸು-ವಂ-ತಹ
ಕಂಗೋ-ಳಿ-ಸು-ತ್ತಿತ್ತು
ಕಂಠ-ಸಿ-ರಿ-ಯಿಂದ
ಕಂಠ-ಸ್ಥ-ವಾ-ಗಿ-ದ್ದವು
ಕಂಠ-ಸ್ಥ-ವಾ-ಗಿ-ಬಿ-ಡು-ತ್ತಿತ್ತು
ಕಂಡ
ಕಂಡಂ-ತಾ-ಯಿತು
ಕಂಡಂತೆ
ಕಂಡ-ತ-ಕ್ಷಣ
ಕಂಡದ್ದು
ಕಂಡರು
ಕಂಡರೆ
ಕಂಡಲ್ಲಿ
ಕಂಡಳು
ಕಂಡ-ವನು
ಕಂಡ-ವರು
ಕಂಡಾಗ
ಕಂಡಿತ್ತು
ಕಂಡಿದ್ದ
ಕಂಡಿ-ದ್ದಾರೆ
ಕಂಡಿದ್ದು
ಕಂಡಿ-ದ್ದೇನೆ
ಕಂಡಿ-ದ್ದೇವೆ
ಕಂಡಿ-ದ್ಧೇನೆ
ಕಂಡಿ-ರುವ
ಕಂಡು
ಕಂಡು-ಕೊಂ-ಡರು
ಕಂಡು-ಕೊಂ-ಡ-ವನು
ಕಂಡು-ಕೊಂ-ಡ-ವರು
ಕಂಡು-ಕೊಂ-ಡಿ-ದ್ದೇನೆ
ಕಂಡು-ಕೊಂಡು
ಕಂಡು-ಕೊ-ಳ್ಳು-ತ್ತಿ-ರು-ವರು
ಕಂಡು-ಬ-ರು-ತ್ತವೆ
ಕಂಡೂತಿ
ಕಂಡೆ
ಕಂಡೊ-ಡನೆ
ಕಂದಾಯ
ಕಂಪ್ಯೂ-ಟ-ರ-ನಂತೆ
ಕಂಪ್ಯೂ-ಟ-ರ್ನ-ಲ್ಲಿಯೇ
ಕಂಬ-ನಿ-ಯಿತ್ತು
ಕಃ
ಕಗ್ಗದ
ಕಗ್ಗ-ಲ್ಲಿನ
ಕಚೇರಿ
ಕಚ್ಚಾ
ಕಚ್ಛೆ-ಪಂಚೆ
ಕಛೇ-ರಿಯ
ಕಛೇ-ರಿ-ಯತ್ತ
ಕಛೇ-ರಿ-ಯಲ್ಲಿ
ಕಛೇ-ರಿ-ಯಿಂದ
ಕಟು-ವಾಗಿ
ಕಟು-ಸ-ತ್ಯ-ವನ್ನು
ಕಟ್ಟಡ
ಕಟ್ಟ-ಡ-ಗಳ
ಕಟ್ಟ-ಡ-ದ-ಲ್ಲಿ-ರ-ಬ-ಹುದು
ಕಟ್ಟ-ಡ-ವನ್ನು
ಕಟ್ಟಲು
ಕಟ್ಟಾ
ಕಟ್ಟಿ
ಕಟ್ಟಿ-ಕೊಂಡು
ಕಟ್ಟಿ-ಕೊ-ಳ್ಳಲು
ಕಟ್ಟಿಟ್ಟ
ಕಟ್ಟಿದ
ಕಟ್ಟಿನ
ಕಟ್ಟು
ಕಟ್ಟು-ಬಿದ್ದು
ಕಟ್ಟು-ವಂತೆ
ಕಟ್ಟೆ-ಯಿ-ಲ್ಲದ
ಕಠಿಣ
ಕಠಿ-ಣ-ವೆ-ನ್ನುವ
ಕಠೋ-ರ-ವೆ-ನಿಸಿ
ಕಠೋ-ರಾಣಿ
ಕಡಕ್
ಕಡ-ತೋಕಾ
ಕಡ-ಬಿನ
ಕಡಲೆ
ಕಡ-ಲೆ-ಯಂತೆ
ಕಡ-ಲೆ-ಯನ್ನು
ಕಡಿದು
ಕಡಿಮೆ
ಕಡಿ-ಮೆ-ಯಾ-ಗಿ-ತ್ತಾ-ದರೂ
ಕಡಿ-ಮೆ-ಯಾ-ದರೆ
ಕಡಿ-ಮೆಯೇ
ಕಡಿ-ಯು-ವಂ-ತಿ-ರು-ತ್ತದೆ
ಕಡುಬು
ಕಡೆ
ಕಡೆ-ಗ-ಣಿ-ಸದೇ
ಕಡೆ-ಗ-ಣಿಸಿ
ಕಡೆ-ಗ-ಳಲ್ಲಿ
ಕಡೆಗೆ
ಕಡೆ-ಯಲ್ಲಿ
ಕಡೆ-ಯಿಂದ
ಕಡೆ-ಯುವ
ಕಡೆ-ಲೆ-ಯಂ-ತಿ-ರುವ
ಕಡ್ಡಾ-ಯ-ಗೊ-ಳಿ-ಸಿ-ದ್ದಾರೆ
ಕಡ್ಡಾ-ಯ-ವಾಗಿ
ಕಣ್ಣ
ಕಣ್ಣಂ-ಚಿ-ನಲ್ಲಿ
ಕಣ್ಣ-ಮುಚ್ಚಿ
ಕಣ್ಣಲ್ಲಿ
ಕಣ್ಣಾರೆ
ಕಣ್ಣಿಗೆ
ಕಣ್ಣಿನ
ಕಣ್ಣೀ-ರನ್ನು
ಕಣ್ಣು
ಕಣ್ಣು-ಗಳ
ಕಣ್ಣು-ಗಳೂ
ಕಣ್ಮ-ರೆ-ಯಾ-ಗಿ-ಸು-ತ್ತಿದೆ
ಕಣ್ಮುಚ್ಚಿ
ಕತೆ
ಕತ್ತ-ರಿ-ಯಲ್ಲಿ
ಕಥಾ-ಸ-ರ-ಸ್ವತೀ
ಕಥೆ
ಕಥೆ-ಗ-ಳನ್ನು
ಕಥೆ-ಗಳು
ಕಥೆ-ಗ-ಳೆಂ-ದರೆ
ಕಥೆ-ಯನ್ನು
ಕಥೆ-ಯಲ್ಲ
ಕಥೆ-ಯೊಂ-ದನ್ನು
ಕದಾ-ಚ-ನೇತಿ
ಕನ-ಸಿನ
ಕನಸು
ಕನಿಷ್ಠ
ಕನಿ-ಷ್ಠಿ-ಕಾದಿ
ಕನ್ನಡ
ಕನ್ನ-ಡ-ದಷ್ಟೇ
ಕನ್ನ-ಡ-ಭಾಷಾ
ಕನ್ನ-ಡಾ-ನು-ವಾ-ದದ
ಕನ್ನ-ಡಿ-ಯಂ-ತಿವೆ
ಕನ್ಯೆಯು
ಕಪ್ಪು
ಕಬ-ಳಿ-ಸಲು
ಕಬ-ಳಿ-ಸುವ
ಕಬ್ಬಿ-ಣದ
ಕಬ್ಬಿ-ಣ-ವಾದೆ
ಕರ-ಗ-ತ-ವಾ-ಗಿದೆ
ಕರ-ಗ-ಲಿಲ್ಲ
ಕರಗಿ
ಕರ-ಡು-ಪ್ರತಿ
ಕರ-ತ-ಲ-ಕ-ವಾ-ಗಿತ್ತು
ಕರ-ತ-ಲಾ-ಮ-ಲ-ಕ-ವಾ-ಗಿತ್ತು
ಕರ-ಪ-ತ್ರ-ಗ-ಳನ್ನು
ಕರವ
ಕರ-ಸಂ-ಗ್ರ-ಹ-ಣೆಯ
ಕರಾಂ-ಗು-ಲಿಃ
ಕರಾ-ರು-ವಾ-ಕ್ಕಾಗಿ
ಕರಿದ
ಕರು-ಣಾ-ಕ-ರಾ-ನಂದ
ಕರು-ಣಾಳು
ಕರು-ಣಿ-ಸಲಿ
ಕರು-ಣಿ-ಸಿ-ದ್ದಾರೆ
ಕರು-ಣಿ-ಸಿಲಿ
ಕರೆ-ಕ-ಳು-ಹಿ-ಸಿ-ದರೂ
ಕರೆ-ತಂದ
ಕರೆ-ತಂ-ದರು
ಕರೆ-ತಂದು
ಕರೆ-ತ-ರಲು
ಕರೆ-ದರು
ಕರೆ-ದಿದ್ದ
ಕರೆದು
ಕರೆ-ದು-ಕೊಂಡು
ಕರೆ-ದು-ಕೊ-ಡು-ಹೋಗಿ
ಕರೆ-ದೊ-ಯ್ದರು
ಕರೆ-ದೊಯ್ದು
ಕರೆ-ದೊ-ಯ್ಯ-ಲಾ-ಯಿತು
ಕರೆ-ಮಾಡಿ
ಕರೆ-ಯ-ಲಾ-ಯಿತು
ಕರೆ-ಯಲು
ಕರೆ-ಯ-ಲ್ಪ-ಡು-ತ್ತದೆ
ಕರೆ-ಯಿತು
ಕರೆ-ಯು-ತಿದ್ದ
ಕರೆ-ಯು-ತ್ತಾನೆ
ಕರೆ-ಯು-ತ್ತಾರೆ
ಕರೆ-ಯು-ತ್ತಾರೋ
ಕರೆ-ಯು-ತ್ತಿ-ದ್ದರು
ಕರೆ-ಯು-ತ್ತಿ-ದ್ದು-ದ-ರಿಂದ
ಕರೆ-ಯು-ತ್ತಿ-ರುವ
ಕರೆ-ಯುವ
ಕರೆ-ಯು-ವುದು
ಕರೆಸಿ
ಕರೆ-ಸಿ-ಕೊಂಡು
ಕರೆ-ಸಿ-ಕೊಂಡೆ
ಕರೆ-ಸಿ-ಕೊ-ಳ್ಳು-ತ್ತಿದ್ದ
ಕರೆ-ಸು-ವುದು
ಕರೋತಿ
ಕರ್ಕಷ
ಕರ್ಣ-ಕು-ಹ-ರ-ದಲ್ಲಿ
ಕರ್ಣಾ-ಕ-ರ್ಣಿ-ಯಾಗಿ
ಕರ್ಣಾ-ನಂ-ದದ
ಕರ್ಣಾ-ನಂ-ದ-ವಾ-ಗು-ತ್ತಿತ್ತು
ಕರ್ತವ್ಯ
ಕರ್ತವ್ಯಂ
ಕರ್ತ-ವ್ಯ-ಕ-ರ್ಮ-ನಿ-ಷ್ಠರು
ಕರ್ತ-ವ್ಯ-ದಲ್ಲಿ
ಕರ್ತ-ವ್ಯ-ಪ್ರ-ಜ್ಞೆ-ಯಿಂದ
ಕರ್ತ-ವ್ಯ-ವನ್ನು
ಕರ್ತ-ವ್ಯ-ವಾಗಿ
ಕರ್ತ-ವ್ಯ-ವಾ-ಗಿದೆ
ಕರ್ತ-ವ್ಯ-ವೆಂದು
ಕರ್ತ-ವ್ಯವೇ
ಕರ್ದಮ
ಕರ್ದ-ಮೇಷು
ಕರ್ನಾ-ಟಕ
ಕರ್ನಾ-ಟ-ಕಕ್ಕೂ
ಕರ್ನಾ-ಟ-ಕದ
ಕರ್ನಾ-ಟ-ಕ-ದಲ್ಲಿ
ಕರ್ನಾ-ಟ-ಕ-ಸಂ-ಸ್ಕೃ-ತ-ವಿ-ಶ್ವ-ವಿ-ದ್ಯಾ-ಲಯ
ಕರ್ಮ
ಕರ್ಮ-ಕ್ಷೇ-ತ್ರ-ವಾದ
ಕರ್ಮ-ಗಳ
ಕರ್ಮಣಾ
ಕರ್ಮದ
ಕರ್ಮ-ದಿಂದ
ಕರ್ಮ-ನಿ-ಷ್ಠ-ರಿಂದ
ಕರ್ಮ-ಫ-ಲ-ಗ-ಳಲ್ಲಿ
ಕರ್ಮ-ಫ-ಲದ
ಕರ್ಮ-ಫ-ಲವೇ
ಕರ್ಮ-ಫ-ಲ-ಸಿ-ದ್ಧಿಗೆ
ಕರ್ಮ-ಮಾರ್ಗ
ಕರ್ಮ-ಸಿ-ದ್ಧಿ-ರ್ವ್ಯ-ವ-ಸ್ಥಿತಾ
ಕರ್ಮಾ-ನು-ಷ್ಠಾ-ತೃ-ಗಳು
ಕಲಂ-ಕ-ರ-ಹಿ-ತ-ವಾಗಿ
ಕಲ-ಕಿ-ದರೆ
ಕಲ-ಗಾರ
ಕಲ-ಗಾ-ರರ
ಕಲ-ಗಾ-ರರು
ಕಲ-ಗಾರು
ಕಲಾ
ಕಲಾಂ
ಕಲಾ-ತ್ಮ-ಕ-ವಾದ
ಕಲಾಪ
ಕಲಾ-ವಿ-ದರ
ಕಲಾ-ವಿ-ದ-ರ-ನ್ನೊ-ಳ-ಗೊಂಡ
ಕಲಾ-ವಿ-ಧ-ನಾ-ಗಿದ್ದ
ಕಲಿ-ಕಾ-ಲ-ವ-ಶ-ದಿಂ-ದಾಗಿ
ಕಲಿ-ಕೋ-ತ್ಸಾ-ಹಿ-ಗಳ
ಕಲಿತ
ಕಲಿ-ತದ್ದು
ಕಲಿ-ತರು
ಕಲಿ-ತ-ವರು
ಕಲಿ-ತಿದ್ದ
ಕಲಿ-ತಿ-ದ್ದೇನೆ
ಕಲಿತು
ಕಲಿ-ತು-ಕೊಂಡು
ಕಲಿತೆ
ಕಲಿ-ತೆವು
ಕಲಿ-ಮ-ಲ-ಪ್ರ-ಧ್ವಂಸಿ
ಕಲಿಯ
ಕಲಿ-ಯ-ದಿ-ದ್ದರೆ
ಕಲಿ-ಯ-ಬೇ-ಕಾ-ಗಿ-ರುವ
ಕಲಿ-ಯ-ಬೇ-ಕಾ-ದದ್ದು
ಕಲಿ-ಯ-ಬೇ-ಕಿತ್ತು
ಕಲಿ-ಯ-ಬೇಕು
ಕಲಿ-ಯ-ಬೇ-ಕೆಂದು
ಕಲಿ-ಯಲು
ಕಲಿ-ಯ-ಲೊಂದು
ಕಲಿ-ಯು-ಗ-ದ-ಲ್ಲಿನ
ಕಲಿ-ಯು-ಗ-ದಲ್ಲೆ
ಕಲಿ-ಯು-ತ್ತಲೇ
ಕಲಿ-ಯುತ್ತಾ
ಕಲಿ-ಯುವ
ಕಲಿ-ಯು-ವು-ದಾ-ದರೂ
ಕಲಿವ
ಕಲಿ-ಸಲು
ಕಲಿಸಿ
ಕಲಿ-ಸಿ-ಕೊ-ಟ್ಟರು
ಕಲಿ-ಸಿ-ಕೊ-ಟ್ಟಿ-ದ್ದ-ಲ್ಲದೇ
ಕಲಿ-ಸಿ-ಕೊ-ಟ್ಟಿ-ದ್ದಾರೆ
ಕಲಿ-ಸಿದ
ಕಲಿ-ಸಿ-ದ್ದಾರೆ
ಕಲಿ-ಸಿದ್ದು
ಕಲಿ-ಸಿ-ದ್ದೇ-ನೆಯೋ
ಕಲಿ-ಸುತ್ತಾ
ಕಲಿ-ಸು-ತ್ತಿ-ದ್ದರು
ಕಲಿ-ಸು-ತ್ತಿ-ದ್ದಾರೆ
ಕಲಿ-ಸುವ
ಕಲಿ-ಸು-ವಲ್ಲಿ
ಕಲು-ಷಿತ
ಕಲೆ
ಕಲೆ-ಗಳ
ಕಲೆಗೆ
ಕಲೆ-ಯನ್ನು
ಕಲೌ
ಕಲ್ಪ-ಗ-ಳಲ್ಲೋ
ಕಲ್ಪ-ನೆ-ಯನ್ನೇ
ಕಲ್ಪ-ನೆಯೇ
ಕಲ್ಪ-ವೃಕ್ಷ
ಕಲ್ಪ-ವೃ-ಕ್ಷ-ದಂ-ತಿ-ದ್ದಾ-ರೆಂ-ದರೆ
ಕಲ್ಪ-ವೃ-ಕ್ಷವೇ
ಕಲ್ಪಿಸಿ
ಕಲ್ಪಿ-ಸಿ-ಕೊ-ಟ್ಟರು
ಕಲ್ಪಿ-ಸಿ-ಕೊ-ಟ್ಟಿ-ದ್ದೇನೆ
ಕಲ್ಪಿ-ಸಿತು
ಕಲ್ಪಿ-ಸಿ-ದರು
ಕಲ್ಪಿ-ಸಿ-ದ-ವರು
ಕಲ್ಪಿ-ಸಿ-ದ್ದರು
ಕಲ್ಪೋಕ್ತ
ಕಲ್ಮನೆ
ಕಲ್ಯಾಣ
ಕಲ್ಯಾ-ಣ-ವಾ-ಗು-ತ್ತದೆ
ಕಲ್ಯಾ-ಣೀ-ವಾ-ಗಿ-ತ್ಯೇ-ತಾನಿ
ಕಲ್ಲಿನ
ಕಲ್ಲು
ಕಳ-ಕಳಿ
ಕಳ-ಕ-ಳಿ-ಯುಳ್ಳ
ಕಳ-ಕ-ಳಿಯೇ
ಕಳ-ಕೊಂಡ
ಕಳ-ಚಿ-ಬಿದ್ದು
ಕಳ-ವಳ
ಕಳ-ಶ-ವಿ-ಟ್ಟಂತೆ
ಕಳಿ-ಸಿ-ಕೊ-ಟ್ಟರು
ಕಳಿ-ಸಿದ್ದು
ಕಳಿ-ಸಿ-ಬಿ-ಟ್ಟರೆ
ಕಳಿ-ಸುವ
ಕಳು-ಹಿ-ಸಿದ
ಕಳು-ಹಿ-ಸಿ-ದರು
ಕಳು-ಹಿ-ಸಿ-ದ-ವ-ರಲ್ಲ
ಕಳು-ಹಿ-ಸು-ತ್ತಿ-ದ್ದರು
ಕಳೆ
ಕಳೆ-ಗುಂ-ದಿದ
ಕಳೆದ
ಕಳೆ-ದದ್ದು
ಕಳೆ-ದದ್ದೇ
ಕಳೆ-ದಿ-ತ್ತಷ್ಟೇ
ಕಳೆ-ದಿತ್ತು
ಕಳೆ-ದಿದ್ದು
ಕಳೆದು
ಕಳೆ-ದು-ಕೊ-ಳ್ಳು-ತ್ತಾನೆ
ಕಳೆ-ದುದೇ
ಕಳೆ-ಯನ್ನು
ಕಳೆ-ಯಿತು
ಕಳೆ-ಯಿ-ದೆ-ಯಲ್ಲಾ
ಕಳೆ-ಯುತ್ತಾ
ಕಳೆ-ಯು-ತ್ತಿತ್ತು
ಕಳೆ-ಯು-ತ್ತಿದ್ದ
ಕಳೆ-ಯು-ತ್ತಿ-ದ್ದರು
ಕಳೆ-ಯು-ವಂತೆ
ಕವ-ನದ
ಕವ-ಲ-ಕೊಪ್ಪ
ಕವ-ಲ-ಕೊ-ಪ್ಪದ
ಕವ-ಲ-ಕೊ-ಪ್ಪ-ದಲ್ಲಿ
ಕವ-ಲು-ಗಳ
ಕವ-ಲೊಂದು
ಕವಾ-ಟ-ವನ್ನು
ಕವಿ
ಕವಿ-ಕು-ಲ-ಗು-ರು-ವಿನ
ಕವಿ-ಗ-ಳಿಗೆ
ಕವಿ-ಯೊ-ಬ್ಬರ
ಕವಿ-ವಾಣಿ
ಕಶೇ-ರು-ಕ-ಗ-ಳಲ್ಲಿ
ಕಶೇ-ರು-ಕ-ಗಳು
ಕಷಿ-ತ-ಜ್ಞರು
ಕಷ್ಟ
ಕಷ್ಟ-ಕ-ರ-ವಾಗಿ
ಕಷ್ಟ-ಕ-ರ-ವಾ-ದರೂ
ಕಷ್ಟ-ಕಾ-ಲ-ದಲ್ಲಿ
ಕಷ್ಟಕ್ಕೆ
ಕಷ್ಟ-ಕ್ಕೆಲ್ಲ
ಕಷ್ಟ-ಗ-ಳನ್ನು
ಕಷ್ಟ-ಗ-ಳೆಂಬ
ಕಷ್ಟದ
ಕಷ್ಟ-ದಲ್ಲಿ
ಕಷ್ಟ-ದಿಂದ
ಕಷ್ಟ-ಪಟ್ಟು
ಕಷ್ಟ-ಪ-ಡು-ತ್ತಿದ್ರೋ
ಕಷ್ಟ-ವನ್ನು
ಕಷ್ಟ-ವಾ-ಗಿ-ರುವ
ಕಷ್ಟ-ವಾ-ಗು-ತ್ತಿತ್ತು
ಕಷ್ಟ-ವೆಂ-ದಾಗ
ಕಷ್ಟ-ವೆ-ಷ್ಟಿ-ದ್ದರೂ
ಕಷ್ಟ-ವೇನು
ಕಷ್ಟ-ಸಾ-ಧ್ಯವೇ
ಕಸಿಗೆ
ಕಸ್ಯ-ಚಿ-ದಾ-ತ್ಮ-ಸಂಸ್ಥಾ
ಕಾಂಕ್ರೀಟ್
ಕಾಂಡ-ದಲ್ಲಿ
ಕಾಂಡ-ದಿಂದ
ಕಾಂತಿ
ಕಾಂತಿ-ಯುಕ್ತ
ಕಾಕ-ತಾ-ಳೀಯ
ಕಾಗ-ದ-ಪ-ತ್ರ-ಗ-ಳನ್ನು
ಕಾಗೇರಿ
ಕಾಠ-ಮಂಡು
ಕಾಠಿಣ್ಯ
ಕಾಡಿತು
ಕಾಡಿನ
ಕಾಡಿ-ನಲ್ಲಿ
ಕಾಡು-ಗಳ
ಕಾಡು-ತ್ತಿತ್ತು
ಕಾಡು-ತ್ತಿ-ದ್ದವು
ಕಾಣ-ದಿ-ರ-ಬ-ಹುದು
ಕಾಣದೆ
ಕಾಣ-ದ್ದಕ್ಕೆ
ಕಾಣ-ಬ-ಹು-ದಾ-ಗಿದೆ
ಕಾಣ-ಬ-ಹುದು
ಕಾಣ-ಲಾ-ರರು
ಕಾಣಲು
ಕಾಣ-ಸಿ-ಗು-ತ್ತವೆ
ಕಾಣ-ಸಿ-ಗು-ತ್ತಾರೆ
ಕಾಣಿ-ಸಿ-ಕೊ-ಳ್ಳದ
ಕಾಣಿ-ಸಿ-ಕೊ-ಳ್ಳ-ಲೇ-ಬೇಕು
ಕಾಣಿ-ಸಿ-ಕೊ-ಳ್ಳು-ತ್ತಿವೆ
ಕಾಣುತ್ತ
ಕಾಣು-ತ್ತದೆ
ಕಾಣು-ತ್ತಾನೆ
ಕಾಣು-ತ್ತಾರೆ
ಕಾಣು-ತ್ತಿತ್ತು
ಕಾಣು-ತ್ತಿ-ದ್ದರು
ಕಾಣು-ತ್ತಿ-ದ್ದಾರೆ
ಕಾಣು-ತ್ತೇವೆ
ಕಾಣುವ
ಕಾಣು-ವಂತೆ
ಕಾಣು-ವ-ವ-ಳಾ-ಗಿದ್ದು
ಕಾಣು-ವು-ದಿಲ್ಲ
ಕಾಣು-ವುದು
ಕಾದಂ-ಬ-ರಿ-ಗ-ಳಲ್ಲಿ
ಕಾದಿತ್ತು
ಕಾದು-ಕೊ-ಳ್ಳುವ
ಕಾನ-ಸೂರು
ಕಾನೂ-ನಾ-ತ್ಮಕ
ಕಾನೂ-ನಾ-ತ್ಮ-ಕ-ವಾಗಿ
ಕಾನೂನು
ಕಾನ್ಫೆ-ರೆ-ನ್ಸ್
ಕಾಪಾ-ಡಲಿ
ಕಾಪಾ-ಡ-ಲೆಂದು
ಕಾಪಾ-ಡಿದ
ಕಾಪಾ-ಡು-ವು-ದಾ-ದರೆ
ಕಾಮ-ನೆ-ಗಳ
ಕಾಮಾ-ತ್ಮತಾ
ಕಾಮಾದಿ
ಕಾಮ್ಯ
ಕಾಮ್ಯೋ
ಕಾಯ
ಕಾಯಕ
ಕಾಯ-ಕ-ವನ್ನು
ಕಾಯ-ಕ-ವಾ-ಗದೆ
ಕಾಯಾ
ಕಾಯಿಕ
ಕಾಯಿಲೆ
ಕಾಯಿ-ಲೆ-ಗ-ಳಿಗೆ
ಕಾಯಿ-ಲೆಗೆ
ಕಾಯು-ತ್ತಿ-ರು-ತ್ತಾರೆ
ಕಾಯೇನ
ಕಾಯ್ದ
ಕಾಯ್ದಿ-ರಿ-ಸಿದ
ಕಾರಃ
ಕಾರಕ
ಕಾರಣ
ಕಾರಣಂ
ಕಾರ-ಣ-ಎಂಬ
ಕಾರ-ಣ-ಕ್ಕಾಗಿ
ಕಾರ-ಣಕ್ಕೂ
ಕಾರ-ಣಕ್ಕೆ
ಕಾರ-ಣ-ಗ-ಳಿಂದ
ಕಾರ-ಣ-ಗ-ಳಿಂ-ದಾ-ಗು-ತ್ತದೆ
ಕಾರ-ಣ-ದಿಂದ
ಕಾರ-ಣ-ದಿಂ-ದಾಗಿ
ಕಾರ-ಣ-ನಾ-ಗಿ-ದ್ದಾನೆ
ಕಾರ-ಣ-ನಾ-ಗು-ತ್ತಿ-ದ್ದನು
ಕಾರ-ಣ-ರೆಂ-ದರೂ
ಕಾರ-ಣ-ವನ್ನು
ಕಾರ-ಣ-ವನ್ನೇ
ಕಾರ-ಣ-ವಾಗಿ
ಕಾರ-ಣ-ವಾ-ಗಿದೆ
ಕಾರ-ಣ-ವಾ-ಗು-ತ್ತದೆ
ಕಾರ-ಣ-ವಾದ
ಕಾರ-ಣ-ವಾ-ದರೂ
ಕಾರ-ಣ-ವಾ-ಯಿತು
ಕಾರ-ಣ-ವಿತ್ತು
ಕಾರ-ಣ-ವಿದೆ
ಕಾರ-ಣ-ವೆಂದು
ಕಾರ-ಣ-ವೆಂಬ
ಕಾರ-ಣ-ವೆಂ-ಬು-ದನ್ನು
ಕಾರ-ಣಾಂ-ತ-ರ-ಗ-ಳಿಂದ
ಕಾರ-ಣೀ-ಕ-ರ್ತ-ರಾದ
ಕಾರ-ಣೀ-ಭೂ-ತ-ವಾ-ಗಿದೆ
ಕಾರ-ಣೀ-ಭೂ-ತ-ವಾದ
ಕಾರ-ಯೇತ್
ಕಾರ-ವನ್ನು
ಕಾರ-ವಾರ
ಕಾರ-ವಾ-ರದ
ಕಾರಿ-ಕಾ-ವಲೀ
ಕಾರಿ-ನತ್ತ
ಕಾರಿ-ನಲ್ಲಿ
ಕಾರಿಯೇ
ಕಾರೋ
ಕಾರ್ಖಾ-ನೆಯ
ಕಾರ್ಪ-ಣ್ಯ-ಗ-ಳನ್ನು
ಕಾರ್ಮಿ-ಕ-ರಿಂದ
ಕಾರ್ಯ
ಕಾರ್ಯಂ
ಕಾರ್ಯ-ಕ-ರ್ತ-ರಾಗಿ
ಕಾರ್ಯ-ಕ-ರ್ತರೇ
ಕಾರ್ಯ-ಕ-ಲಾ-ಪ-ಗ-ಳಲ್ಲಿ
ಕಾರ್ಯ-ಕೈಂ-ಕ-ರ್ಯ-ಗ-ಳಲ್ಲಿ
ಕಾರ್ಯಕ್ಕೂ
ಕಾರ್ಯಕ್ಕೆ
ಕಾರ್ಯ-ಕ್ರಮ
ಕಾರ್ಯ-ಕ್ರ-ಮಕ್ಕೆ
ಕಾರ್ಯ-ಕ್ರ-ಮ-ಕ್ಕೆ-ಆ-ಗ-ಮಿ-ಸಿ-ದ-ಮ-ಠಾ-ಧಿ-ಪ-ತಿ-ಗಳು
ಕಾರ್ಯ-ಕ್ರ-ಮ-ಗಳ
ಕಾರ್ಯ-ಕ್ರ-ಮ-ಗ-ಳನ್ನು
ಕಾರ್ಯ-ಕ್ರ-ಮ-ಗ-ಳಲ್ಲಿ
ಕಾರ್ಯ-ಕ್ರ-ಮ-ಗ-ಳಿಗೆ
ಕಾರ್ಯ-ಕ್ರ-ಮ-ಗಳು
ಕಾರ್ಯ-ಕ್ರ-ಮದ
ಕಾರ್ಯ-ಕ್ರ-ಮ-ದಲ್ಲಿ
ಕಾರ್ಯ-ಕ್ರ-ಮ-ವನ್ನು
ಕಾರ್ಯ-ಕ್ರ-ಮ-ವನ್ನೇ
ಕಾರ್ಯ-ಕ್ರ-ಮ-ವಾ-ದರೂ
ಕಾರ್ಯ-ಕ್ರ-ಮ-ವಾ-ದ್ದ-ರಿಂದ
ಕಾರ್ಯ-ಕ್ಷೇ-ತ್ರ-ದ-ಲ್ಲಿ-ದ್ದು-ಕೊಂಡೇ
ಕಾರ್ಯ-ಗ-ತ-ಗೊ-ಳಿ-ಸಿ-ದ-ವನು
ಕಾರ್ಯ-ಗ-ಳನ್ನು
ಕಾರ್ಯ-ಗ-ಳಲ್ಲಿ
ಕಾರ್ಯ-ಗ-ಳಾಗಿ
ಕಾರ್ಯ-ಗಳು
ಕಾರ್ಯ-ಗೈ-ಯ್ಯುತ್ತಾ
ಕಾರ್ಯ-ಚ-ಟು-ವ-ಟಿ-ಕೆ-ಗ-ಳಲ್ಲಿ
ಕಾರ್ಯ-ತಂತ್ರ
ಕಾರ್ಯ-ತ-ತ್ಪ-ರತೆ
ಕಾರ್ಯದ
ಕಾರ್ಯ-ದrರ್ಶೀ
ಕಾರ್ಯ-ದರ್ಶಿ
ಕಾರ್ಯ-ದ-ರ್ಶಿ-ಗ-ಳಾ-ಗಿದ್ದು
ಕಾರ್ಯ-ದ-ರ್ಶಿ-ಗ-ಳಾದ
ಕಾರ್ಯ-ದ-ರ್ಶಿ-ಯಾಗಿ
ಕಾರ್ಯ-ದ-ರ್ಶಿ-ಯಾ-ಗಿದ್ದ
ಕಾರ್ಯ-ದ-ರ್ಶಿ-ಯಾ-ಗಿದ್ದೆ
ಕಾರ್ಯ-ದರ್ಶೀ
ಕಾರ್ಯ-ದಲ್ಲಿ
ಕಾರ್ಯ-ದಲ್ಲೇ
ಕಾರ್ಯ-ದಿಂದ
ಕಾರ್ಯ-ದಿಂ-ದಲೂ
ಕಾರ್ಯ-ನಿ-ಮಿ-ತ್ತ-ವಾಗಿ
ಕಾರ್ಯ-ನಿ-ರ್ವ-ಹಿಸಿ
ಕಾರ್ಯ-ನಿ-ರ್ವ-ಹಿ-ಸಿ-ದ-ವರು
ಕಾರ್ಯ-ನಿ-ರ್ವ-ಹಿ-ಸಿ-ದುದು
ಕಾರ್ಯ-ನಿ-ರ್ವ-ಹಿ-ಸು-ತ್ತಿತ್ತು
ಕಾರ್ಯ-ನಿ-ರ್ವ-ಹಿ-ಸು-ತ್ತಿ-ದ್ದರು
ಕಾರ್ಯ-ನಿ-ರ್ವ-ಹಿ-ಸು-ತ್ತಿ-ದ್ದಾಗ
ಕಾರ್ಯ-ನಿ-ರ್ವ-ಹಿ-ಸು-ವಲ್ಲಿ
ಕಾರ್ಯ-ಪ-ರತೆ
ಕಾರ್ಯ-ಭಾ-ರ-ಗ-ಳಿಗೆ
ಕಾರ್ಯ-ಭಾ-ರದ
ಕಾರ್ಯ-ರೂ-ಪಕ್ಕೆ
ಕಾರ್ಯ-ವನ್ನು
ಕಾರ್ಯ-ವಿ-ರಲಿ
ಕಾರ್ಯವೇ
ಕಾರ್ಯ-ವೈ-ಖ-ರಿ-ಯನ್ನು
ಕಾರ್ಯ-ಶಾ-ಲೆ-ಯನ್ನು
ಕಾರ್ಯ-ಸಾ-ಧು-ವಾ-ಗ-ಬ-ಹು-ದಾದ
ಕಾರ್ಯಾ
ಕಾರ್ಯಾಂ-ತ-ರ-ದಿಂದ
ಕಾರ್ಯಾಂತೇ
ಕಾರ್ಯಾ-ಗಾ-ರ-ಗ-ಳಲ್ಲಿ
ಕಾರ್ಯಾ-ಗಾ-ರ-ವನ್ನು
ಕಾರ್ಯಾ-ರಂ-ಭಿ-ಸಿ-ದ್ದರು
ಕಾರ್ಯೋ-ನ್ಮು-ಖ-ರಾ-ದರು
ಕಾಲ
ಕಾಲ-ಕರ್ಮ
ಕಾಲಕ್ಕೆ
ಕಾಲ-ಕ್ರ-ಮ-ದಲ್ಲಿ
ಕಾಲ-ಕ್ರ-ಮಿ-ಸಿ-ದಂತೆ
ಕಾಲ-ಗ-ಣ-ನೆಗೆ
ಕಾಲ-ಘ-ಟ್ಟ-ದಲ್ಲಿ
ಕಾಲದ
ಕಾಲ-ದಂತೆ
ಕಾಲ-ದಲ್ಲಿ
ಕಾಲ-ದ-ಲ್ಲಿಯೂ
ಕಾಲ-ದ-ಲ್ಲಿಯೇ
ಕಾಲ-ದಲ್ಲೂ
ಕಾಲ-ದಿಂದ
ಕಾಲ-ದಿಂ-ದಲೂ
ಕಾಲ-ಧ-ರ್ಮ-ಕ್ಕ-ನು-ಗು-ಣ-ವಾಗಿ
ಕಾಲ-ಭೇ-ಧ-ವಿ-ಲ್ಲದೇ
ಕಾಲ-ಮಾ-ತ್ರ-ದಿಂದ
ಕಾಲ-ಮಿ-ತಿ-ಗ-ನು-ಗು-ಣ-ವಾಗಿ
ಕಾಲ-ಮಿ-ತಿ-ಯಲ್ಲಿ
ಕಾಲ-ಮೇಲೆ
ಕಾಲ-ವದು
ಕಾಲ-ವನ್ನು
ಕಾಲ-ವಿದು
ಕಾಲ-ವೆಂದು
ಕಾಲ-ವ್ಯಯ
ಕಾಲ-ಹ-ರಣ
ಕಾಲ-ಹಾ-ಕಿದ್ದೆ
ಕಾಲಾ-ನಂ-ತರ
ಕಾಲಿಗೆ
ಕಾಲಿ-ಟ್ಟಂ-ತಹ
ಕಾಲು
ಕಾಲುವೆ
ಕಾಲೇ
ಕಾಲೇ-ಜನ್ನು
ಕಾಲೇ-ಜಿಗೂ
ಕಾಲೇ-ಜಿಗೆ
ಕಾಲೇ-ಜಿನ
ಕಾಲೇ-ಜಿ-ನಲ್ಲಿ
ಕಾಲೇ-ಜಿ-ನ-ಲ್ಲಿಯೂ
ಕಾಲೇ-ಜಿ-ನ-ಲ್ಲಿಯೇ
ಕಾಲೇ-ಜಿ-ನಿಂದ
ಕಾಲೇಜು
ಕಾಲೇ-ಜು-ಗ-ಳಲ್ಲಿ
ಕಾಲೇ-ಜು-ಗ-ಳಿಗೆ
ಕಾಲೇಜ್
ಕಾಲೋ
ಕಾಲ್ಕಿ-ತ್ತರು
ಕಾಲ್ನ-ಡಿ-ಗೆ-ಯಲ್ಲೆ
ಕಾಳಜಿ
ಕಾಳ-ಜಿಗೆ
ಕಾಳ-ಜಿಯೂ
ಕಾವಿಗೆ
ಕಾವಿ-ಧಾ-ರಿಯು
ಕಾವ್ಯ
ಕಾವ್ಯ-ಚಿತ್ರ
ಕಾವ್ಯ-ತ-ರ-ಗ-ತಿಯ
ಕಾವ್ಯದ
ಕಾವ್ಯ-ದೊ-ಲವು
ಕಾವ್ಯ-ಪ-ರೀ-ಕ್ಷೆ-ಯಲ್ಲಿ
ಕಾವ್ಯ-ರ-ಸ-ದೌ-ತ-ಣ-ದಂತೆ
ಕಾವ್ಯ-ವನ್ನು
ಕಾವ್ಯ-ವನ್ನೂ
ಕಾವ್ಯ-ಶಾ-ಸ್ತ್ರ-ವಿ-ನೋ-ದೇನ
ಕಾವ್ಯಾಂತ
ಕಾವ್ಯಾ-ದಿ-ಗ-ಳನ್ನು
ಕಾವ್ಯಾ-ನು-ಶಾ-ಸ-ನ-ದಲ್ಲಿ
ಕಿಂಚಿತ್
ಕಿಂಚಿತ್ತೂ
ಕಿಗ್ಗ
ಕಿತ್ತು
ಕಿತ್ತೋ-ಡಿ-ಸಿದ
ಕಿಮೀ
ಕಿಮೀಗೆ
ಕಿರಿ-ಜೀ-ವದ
ಕಿರಿಯ
ಕಿರಿ-ಯ-ನಾ-ಗಿರೆ
ಕಿರಿ-ಯ-ರಿಗೆ
ಕಿರಿ-ಯ-ರೆ-ಲ್ಲರ
ಕಿರಿ-ಸೊ-ಸೆ-ಯಾಗಿ
ಕಿರು
ಕಿರು-ಪ-ರಿ-ಚ-ಯವೂ
ಕಿರು-ಬೆ-ರಳ
ಕಿಲೋ-ಮೀ-ಟರ್
ಕಿವಿ
ಕಿವಿಗೂ
ಕಿವಿಗೆ
ಕಿವಿಯ
ಕಿವಿ-ಯಲ್ಲಿ
ಕೀಟಲೇ
ಕೀರ್ತ-ನೆ-ಯನ್ನು
ಕೀರ್ತಿ
ಕೀರ್ತಿ-ಗ-ಳನ್ನು
ಕೀರ್ತಿ-ಗ-ಳಿ-ಸು-ತ್ತಾರೆ
ಕೀರ್ತಿ-ಯನ್ನು
ಕೀರ್ತಿ-ಯನ್ನೂ
ಕೀರ್ತಿಯೂ
ಕೀರ್ತಿ-ಶೇಷ
ಕೀರ್ತಿ-ಶೇ-ಷ-ಮ-ಹಾ-ಬ-ಲೇ-ಶ್ವರ
ಕೀಳಲು
ಕುಂಡ
ಕುಂಡ-ಪ್ಪ-ನೆಂದೇ
ಕುಂದಿದ
ಕುಂದಿ-ದಾಗ
ಕುಂಭಾ-ಭಿ-ಷೇ-ಕ-ವಾಗಿ
ಕುಗ್ಗಿದೆ
ಕುಗ್ಗುವ
ಕುಗ್ರಾಮ
ಕುಟುಂಬ
ಕುಟುಂ-ಬ-	ವಿ-ಭ-ಕ್ತ-ವಾ-ಗಿತ್ತು
ಕುಟುಂ-ಬಕ್ಕೂ
ಕುಟುಂ-ಬಕ್ಕೆ
ಕುಟುಂ-ಬ-ಗಳ
ಕುಟುಂ-ಬದ
ಕುಟುಂ-ಬ-ದಲ್ಲಿ
ಕುಟುಂ-ಬ-ದ-ವ-ರ-ನ್ನೆಲ್ಲ
ಕುಟುಂ-ಬ-ದ-ವ-ರಿಗೆ
ಕುಟುಂ-ಬ-ದ-ವರು
ಕುಟುಂ-ಬ-ದ-ವ-ರೆಲ್ಲ
ಕುಟುಂ-ಬ-ದ-ವರೇ
ಕುಟುಂ-ಬ-ದಿಂದ
ಕುಟುಂ-ಬ-ದೊಂ-ದಿ-ಗಿನ
ಕುಟುಂ-ಬ-ದೊಂ-ದಿಗೆ
ಕುಟುಂ-ಬ-ವನ್ನು
ಕುಟುಂ-ಬ-ವ-ರ್ಗಕ್ಕೂ
ಕುಟುಂ-ಬ-ವಾ-ದ್ದ-ರಿಂದ
ಕುಡಿ-ಗಳು
ಕುಡಿ-ದಿ-ದ್ದನ್ನೂ
ಕುಡಿದು
ಕುಡಿದೆ
ಕುಡಿಯೇ
ಕುಡಿ-ಸು-ವರು
ಕುತೂ-ಹಲ
ಕುತೂ-ಹ-ಲ-ದಿಂದ
ಕುತ್ಸಿ-ತ-ವೆಂದು
ಕುದಿ
ಕುದಿ-ಯಿತು
ಕುಮ-ಟದ
ಕುಮಟಾ
ಕುಮಾರ
ಕುರಿತ
ಕುರಿ-ತಾಗಿ
ಕುರಿ-ತಾದ
ಕುರಿತು
ಕುರುಡ
ಕುರುಡು
ಕುರ್ಯಾಚ್ಚ
ಕುಲ
ಕುಲಂ
ಕುಲ-ಪ-ರಂ-ಪ-ರೆಯ
ಕುಲ-ಪ-ರಂ-ಪ-ರೆ-ಯನ್ನು
ಕುಲ-ಪ-ರಂ-ಪ-ರೆ-ಯಾದ
ಕುಲ-ವಾ-ಗಲೀ
ಕುಲ-ಶೀ-ಲಾದಿ
ಕುಲ-ಸ-ಚಿ-ವರು
ಕುಲಾ-ರ್ಣ-ವ-ತಂತ್ರ
ಕುಲೀ-ನರು
ಕುಳಿ-ತರೆ
ಕುಳಿ-ತಿ-ದ್ದರು
ಕುಳಿತು
ಕುಳಿ-ತು-ಕೊ-ಳ್ಳು-ತ್ತದೆ
ಕುಳಿ-ತು-ಕೋ-sರುಕ್
ಕುಳ್ಳಿ-ರಿ-ಸಿ-ಕೊಂ-ಡರು
ಕುಳ್ಳಿ-ರಿ-ಸಿ-ಕೊಂಡು
ಕುಶ-ಲ-ನಾ-ಗಿ-ದ್ದರು
ಕುಸಿದು
ಕುಸು-ಮ-ಗಳು
ಕುಸು-ಮ-ದಂತೆ
ಕುಸು-ಮಾ-ದಪಿ
ಕುಸು-ಮಿಸಿ
ಕುಸ್ತಿ
ಕುಹ-ಕ-ವನ್ನು
ಕೂಗಿ-ಕೊಂಡ
ಕೂಗಿ-ತಂತೆ
ಕೂಗಿತು
ಕೂಗುತ್ತ
ಕೂಡ
ಕೂಡಲೇ
ಕೂಡಾ
ಕೂಡಿದ
ಕೂಡಿದ್ದು
ಕೂಡಿ-ರಲು
ಕೂಡಿ-ರ-ಲೆಂದು
ಕೂಡು-ಕು-ಟುಂ-ಬದ
ಕೂಡು-ಕು-ಟುಂ-ಬ-ದಲ್ಲಿ
ಕೂತು
ಕೂದ-ಲು-ಗ-ಳೆಲ್ಲ
ಕೂರ-ದಿ-ದ್ದರೂ
ಕೂರಿ-ಸಲು
ಕೂರಿಸಿ
ಕೂಲಿ-ನಾಲಿ
ಕೂಸು
ಕೃತ-ಕತೆ
ಕೃತ-ಕ-ವಾ-ಗಿ-ರದ
ಕೃತ-ಕೃ-ತ್ಯತೆ
ಕೃತ-ಜ್ಙ-ತೆ-ಗ-ಳನ್ನು
ಕೃತಜ್ಞ
ಕೃತ-ಜ್ಞತಾ
ಕೃತ-ಜ್ಞತೆ
ಕೃತ-ಜ್ಞ-ತೆ-ಗಳು
ಕೃತ-ಜ್ಞ-ತೆಗೆ
ಕೃತ-ಜ್ಞ-ತೆಯ
ಕೃತ-ಜ್ಞ-ತೆ-ಯನ್ನು
ಕೃತ-ಜ್ಞ-ತೆ-ಯಿಂದ
ಕೃತ-ಜ್ಞ-ನಾ-ಗಿ-ದ್ದೇನೆ
ಕೃತ-ಜ್ಞ-ರಾ-ಗಿ-ದ್ದೇವೆ
ಕೃತ-ಯುಗೇ
ಕೃತಾಪಿ
ಕೃತಾ-ರ್ಥ-ನಾ-ದೆ-ನೆಂಬ
ಕೃತಾ-ರ್ಥ-ರಾ-ಗಿ-ದ್ದಾರೆ
ಕೃತಿ-ಯಲ್ಲಿ
ಕೃದಂತ
ಕೃಪ-ಣ-ತೆ-ಯನ್ನು
ಕೃಪೆ
ಕೃಪೆಗೆ
ಕೃಶದಿ
ಕೃಷಿ
ಕೃಷಿಕ
ಕೃಷಿ-ಕಾರ್ಯ
ಕೃಷಿ-ಕಾ-ರ್ಯ-ಗ-ಳಲ್ಲಿ
ಕೃಷಿ-ಕ್ಷೇತ್ರ
ಕೃಷಿ-ಯನ್ನು
ಕೃಷಿ-ಯಲ್ಲಿ
ಕೃಷಿ-ವಸ್ತು
ಕೃಷ್ಟಿ
ಕೃಷ್ಣ
ಕೃಷ್ಣನ
ಕೃಷ್ಣ-ನಂತೆ
ಕೃಷ್ಣ-ಮಂ-ಜ-ಭ-ಟ್ಟರ
ಕೃಷ್ಣ-ಮಾ-ಚಾ-ರ್ಯರ
ಕೃಷ್ಣ-ಯ-ಜು-ರ್ವೇದ
ಕೃಷ್ಣ-ಸ್ವಾ-ಮಿ-ಯ-ವರ
ಕೃಷ್ಣ-ಸ್ವಾ-ಮಿ-ಯ-ವ-ರಿಗೆ
ಕೃಷ್ಣಾ-ನಂದ
ಕೆ
ಕೆಂಪ-ರಾ-ಜು-ರ-ವರು
ಕೆಎ-ನ್ವ-ರ-ದ-ರಾಜ
ಕೆಎಸ್
ಕೆಎ-ಸ್ವ-ರದಾ
ಕೆಚ್ಚನ್ನು
ಕೆಚ್ಚೆ-ದೆಯ
ಕೆಜಿ
ಕೆಟ್ಟ-ದ್ದ-ರಿಂದ
ಕೆಡದೆ
ಕೆಡಿಸಿ
ಕೆಣ-ಕಿ-ದಾಗ
ಕೆತ್ತಿ
ಕೆತ್ತಿಸಿ
ಕೆದಕಿ
ಕೆಪೆ-ಇಂ-ದ್ರ-ಕೆಪೆ
ಕೆರ-ಳಿತು
ಕೆರ-ಳಿ-ಸಿತು
ಕೆರ-ಳಿ-ಸಿತ್ತು
ಕೆರೆಕೈ
ಕೆರೆಯ
ಕೆರೇಕೈ
ಕೆಲ
ಕೆಲ-ಕಾಲ
ಕೆಲ-ಕಾ-ಲ-ವಾ-ದರೂ
ಕೆಲ-ದಿ-ನ-ಗ-ಳನ್ನು
ಕೆಲ-ವಮ್ಮೆ
ಕೆಲ-ವರ
ಕೆಲ-ವ-ರಾ-ದರೆ
ಕೆಲ-ವ-ರಿಂದ
ಕೆಲ-ವ-ರಿಗೆ
ಕೆಲ-ವರು
ಕೆಲ-ವಷ್ಟು
ಕೆಲ-ವಾರು
ಕೆಲವು
ಕೆಲ-ವೆಡೆ
ಕೆಲವೇ
ಕೆಲ-ವೊಂದು
ಕೆಲ-ವೊಮ್ಮೆ
ಕೆಲಸ
ಕೆಲ-ಸ-ಕ್ಕಾಗಿ
ಕೆಲ-ಸಕ್ಕೆ
ಕೆಲ-ಸ-ಗ-ಳನ್ನು
ಕೆಲ-ಸ-ಗ-ಳಲ್ಲಿ
ಕೆಲ-ಸದ
ಕೆಲ-ಸ-ದಾ-ಳಿಂದ
ಕೆಲ-ಸ-ಮಾ-ಡು-ವಾಗ
ಕೆಲ-ಸ-ವ-ನ್ನಾ-ರಂ-ಭಿ-ಸಿ-ದ್ದರು
ಕೆಲ-ಸ-ವನ್ನು
ಕೆಲ-ಸ-ವಾ-ಗಿತ್ತು
ಕೆಲ-ಸ-ಕಾ-ರ್ಯ-ಗ-ಳಿ-ಗಾಗಿ
ಕೆಲೆವು
ಕೆಳ-ಗಿನ
ಕೆಳಗೆ
ಕೆಳದಿ
ಕೆಳ-ಭಾಗ
ಕೆಳ-ಭಾ-ಗ-ದಲ್ಲಿ
ಕೆವಿ-ಸಂ-ಪ-ತ್ಕು-ಮಾರ್
ಕೆಸರು
ಕೆಹೆ-ಮ್ಮ-ನ-ಹ-ಳ್ಳಿಯ
ಕೆಹೆ-ಮ್ಮ-ನ-ಹ-ಳ್ಳಿ-ಯ-ಲ್ಲಿನ
ಕೇಂದ್ರ
ಕೇಂದ್ರಕ್ಕೆ
ಕೇಂದ್ರ-ಗ-ಳಲ್ಲಿ
ಕೇಂದ್ರವೇ
ಕೇಂದ್ರೀಯ
ಕೇತ-ಕೀ-ಗಂ-ಧ-ಮಾ-ಘ್ರಾಯ
ಕೇಳದೇ
ಕೇಳ-ಬ-ಹು-ದೆಂದು
ಕೇಳ-ಬೇ-ಕಾ-ಗಿ-ರ-ಲಿಲ್ಲ
ಕೇಳ-ಬೇಕು
ಕೇಳ-ಲಾ-ಗದ
ಕೇಳಲು
ಕೇಳಿ
ಕೇಳಿ-ಕೊಂಡು
ಕೇಳಿ-ಕೊ-ಳ್ಳು-ತ್ತೇನೆ
ಕೇಳಿದ
ಕೇಳಿ-ದ-ರಂತೆ
ಕೇಳಿ-ದರು
ಕೇಳಿ-ದರೂ
ಕೇಳಿ-ದರೆ
ಕೇಳಿ-ದ-ವರ
ಕೇಳಿ-ದಾಗ
ಕೇಳಿ-ದ್ದ-ಕ್ಕಿಂತ
ಕೇಳಿ-ದ್ದರು
ಕೇಳಿ-ದ್ದಾನೆ
ಕೇಳಿದ್ದೆ
ಕೇಳಿದ್ದೇ
ಕೇಳಿ-ದ್ದೇನೆ
ಕೇಳಿ-ದ್ದೇವೆ
ಕೇಳಿ-ಬ-ರು-ತ್ತಾರೋ
ಕೇಳಿ-ಬ-ರುವ
ಕೇಳಿಯೇ
ಕೇಳಿ-ಸ-ದ-ವ-ರಂತೆ
ಕೇಳಿ-ಸಿ-ಕೊಂಡ
ಕೇಳಿ-ಸಿ-ಕೊಂಡೆ
ಕೇಳಿ-ಸಿ-ಕೊ-ಳ್ಳಲು
ಕೇಳೀ
ಕೇಳು
ಕೇಳು-ಗ-ನಿಗೆ
ಕೇಳು-ತ್ತಲೇ
ಕೇಳು-ತ್ತಾರೆ
ಕೇಳು-ತ್ತಿ-ದ್ದರು
ಕೇಳು-ತ್ತಿ-ದ್ದರೆ
ಕೇಳು-ತ್ತಿ-ದ್ದಾರೆ
ಕೇಳು-ತ್ತಿ-ದ್ದೆವು
ಕೇಳುವ
ಕೇಳು-ವ-ವ-ರಿಗೂ
ಕೇಳು-ವಾಗ
ಕೇಳು-ವು-ದಿಲ್ಲ
ಕೇಳು-ವುದು
ಕೇವಲ
ಕೇವಲಂ
ಕೈ
ಕೈಂಕ-ರ್ಯ-ದಲ್ಲಿ
ಕೈಗ-ನ್ನಡಿ
ಕೈಗೂ-ಡುವ
ಕೈಗೆ
ಕೈಗೆ-ತ್ತಿ-ಕೊಂಡ
ಕೈಗೊಂಡ
ಕೈಗೊ-ಳ್ಳ-ಬೇಕು
ಕೈಪಿ-ಡಿ-ಯಾ-ಗು-ವು-ದ-ರಲ್ಲಿ
ಕೈಬಿ-ಡ-ಲಾ-ಗಿದೆ
ಕೈಬೀಸಿ
ಕೈಮೀ-ರು-ತ್ತಿ-ರು-ವು-ದನ್ನು
ಕೈಮು-ಗಿದು
ಕೈಯ
ಕೈಯನ್ನು
ಕೈಯಲ್ಲಿ
ಕೈಯಿ
ಕೈಲಾದ
ಕೈಸನ್ನೆ
ಕೈಸ-ನ್ನೆ-ಯಿಂ-ದಲೇ
ಕೊಂಚ
ಕೊಂಡ-ವ-ರಿ-ಗೇನು
ಕೊಂಡಿಯೂ
ಕೊಂಡು
ಕೊಂಡು-ಕೊ-ಳ್ಳುವ
ಕೊಂಡೆವು
ಕೊಟ್ಟ
ಕೊಟ್ಟರೆ
ಕೊಟ್ಟ-ವ-ರನ್ನು
ಕೊಟ್ಟಿ-ದೆ-ಯೆಂದು
ಕೊಟ್ಟಿ-ದ್ದಾರೆ
ಕೊಟ್ಟಿದ್ದು
ಕೊಟ್ಟಿ-ರು-ವು-ದನ್ನು
ಕೊಟ್ಟು
ಕೊಟ್ಟೆ
ಕೊಠಡಿ
ಕೊಠ-ಡಿ-ಗಳು
ಕೊಠ-ಡಿಗೆ
ಕೊಠ-ಡಿಯ
ಕೊಠ-ಡಿ-ಯನ್ನು
ಕೊಠ-ಡಿ-ಯಲ್ಲಿ
ಕೊಠ-ಡಿ-ಯ-ಲ್ಲಿ-ಅದು
ಕೊಠ-ಡಿ-ಯ-ಲ್ಲಿ-ರುವ
ಕೊಠ-ಡಿ-ಯಾ-ಗಿತ್ತು
ಕೊಠ-ಡಿ-ಯಿದು
ಕೊಡ
ಕೊಡ-ದಿ-ದ್ದಾಗ
ಕೊಡ-ಬೇ-ಕೆಂದು
ಕೊಡ-ಬೇ-ಕೆಂ-ಬುದು
ಕೊಡ-ಮಾ-ಡಿದ್ದ
ಕೊಡ-ಲಿಲ್ಲ
ಕೊಡಿ-ಗೆ-ಗಳ
ಕೊಡಿಸಿ
ಕೊಡಿ-ಸು-ತ್ತಾರೆ
ಕೊಡಿ-ಸುವ
ಕೊಡುಗೆ
ಕೊಡು-ಗೆ-ಯಾಗಿ
ಕೊಡು-ಗೆಯೂ
ಕೊಡು-ತ್ತಾರೆ
ಕೊಡು-ತ್ತಿ-ದ್ದರು
ಕೊಡು-ತ್ತಿ-ದ್ದುದು
ಕೊಡು-ತ್ತೇನೆ
ಕೊಡುವ
ಕೊಡು-ವಂತೆ
ಕೊಡ್ತಾರೆ
ಕೊನೆಗೆ
ಕೊನೆಯ
ಕೊನೆ-ಯ-ದಾ-ಗಿ-ರುವ
ಕೊನೆ-ಯಲ್ಲಿ
ಕೊನೆ-ಯ-ವರು
ಕೊರ-ಗನ್ನು
ಕೊರ-ಗಿದ್ದೇ
ಕೊರಗು
ಕೊರತೆ
ಕೊರ-ತೆ-ಯನ್ನು
ಕೊರ-ತೆ-ಯಿಂದ
ಕೊರ-ತೆ-ಯಿ-ರು-ತ್ತ-ದೆಂ-ಬುದು
ಕೊರ-ತೆಯೇ
ಕೊರ್ಲಕೈ
ಕೊರ್ಲ-ಕೈಗೆ
ಕೊಲ-ಸಿರ್ಸಿ
ಕೊಳದ
ಕೊಳೆಯ
ಕೊಳೆ-ಯು-ತ್ತಿ-ರುವ
ಕೊಳ್ಳ
ಕೊಳ್ಳ-ಬೇ-ಡಿ-ಎಂದೂ
ಕೊಳ್ಳು-ತ್ತಿದ್ದೆ
ಕೋ
ಕೋsರುಕ್
ಕೋಟಿ
ಕೋಟಿಂ
ಕೋಟಿ-ಗಳು
ಕೋಟಿ-ತೀ-ರ್ಥ-ದಲ್ಲಿ
ಕೋಟಿಯ
ಕೋಟೆ-ಮನೆ
ಕೋಠ-ಡಿಗೆ
ಕೋಠ-ಡಿ-ಯೊ-ಳಕ್ಕೆ
ಕೋಣೆ-ಯಲ್ಲಿ
ಕೋಣೆ-ಯ-ಲ್ಲಿದ್ದೆ
ಕೋಪ
ಕೋಪಂ
ಕೋಪ-ತಾ-ಪಾ-ದಿ-ಗಳು
ಕೋರಿಕೆ
ಕೋರು-ತ್ತಿದ್ದ
ಕೋರ್ಸಿಗೆ
ಕೋಲಾ-ಹ-ಲ-ವೆ-ಬ್ಬಿ-ಸಿತು
ಕೋಲು
ಕೋಶ
ಕೋಶ-ಗ-ಳನ್ನು
ಕೌಟುಂ-ಬಿಕ
ಕೌತುಕ
ಕೌಮು-ದಿಯ
ಕೌಮುದೀ
ಕೌಶಲ
ಕೌಶ-ಲಕ್ಕೆ
ಕೌಶ-ಲ-ದಿಂದ
ಕೌಶ-ಲ್ಯಕ್ಕೆ
ಕೌಶ-ಲ
ಕ್ಕೂ
ಕ್ಕೆ
ಕ್ಕೆಸೇ-ರಿ-ಸಿ-ದರು
ಕ್ಯಾತ-ನಳ್ಳಿ
ಕ್ರಮ
ಕ್ರಮ-ಗಳ
ಕ್ರಮದ
ಕ್ರಮ-ದಲ್ಲಿ
ಕ್ರಮ-ಬ-ದ್ಧ-ವಾ-ಗಿದೆ
ಕ್ರಮ-ವಂತೂ
ಕ್ರಮ-ವಾಗಿ
ಕ್ರಾಫ-ರ್ಡ-ಭ-ವ-ನ-ದಲ್ಲಿ
ಕ್ರಿಯಾ
ಕ್ರಿಯಾ-ಪ್ರ-ವೃ-ತ್ತಿ-ಯಿದೆ
ಕ್ರಿಯಾ-ರ-ಹಿ-ತ-ವಾಗಿ
ಕ್ರಿಯಾ-ವಾನ್
ಕ್ರಿಯಾ-ಶೀ-ಲ-ರಾ-ಗಿ-ದ್ದಾಗ
ಕ್ರಿಯಾ-ಶೀ-ಲರು
ಕ್ರಿಯಾ-ಸಿ-ದ್ಧಿಃ
ಕ್ರಿಯಾ-ಹೀ-ನ-ವಾ-ದಾಗ
ಕ್ರಿಯೆ
ಕ್ರಿಯೆ-ಯಿಂದ
ಕ್ರಿಶ
ಕ್ರಿಶ್ಚಿ-ಯನ್
ಕ್ರೀಯಾ-ಯಾಂ
ಕ್ರೈಸ್ತ
ಕ್ರೋಢೀ-ಕ-ರಿಸಿ
ಕ್ರೋಧ
ಕ್ಲಾಸು
ಕ್ಲಾಸ್ಮು-ಗಿದ
ಕ್ಲಿನಿಕ್
ಕ್ಲಿಷ್ಟ
ಕ್ಲಿಷ್ಟ-ಕ-ರ-ವಾದ
ಕ್ಲಿಷ್ಟತೆ
ಕ್ಲೇಶ
ಕ್ಷಣ-ಗಳು
ಕ್ಷಣ-ಗಳೇ
ಕ್ಷಣ-ದಲ್ಲಿ
ಕ್ಷಣ-ವದು
ಕ್ಷಣ-ವಿದು
ಕ್ಷಣವೂ
ಕ್ಷಣ-ವೆಂದು
ಕ್ಷಣ-ಸ್ಪು-ರತೆ
ಕ್ಷಣಾ-ರ್ಧ-ದಲ್ಲಿ
ಕ್ಷಣಿಕ
ಕ್ಷಣಿ-ಕ-ವಾ-ಗು-ತ್ತಿ-ರು-ವು-ದನ್ನು
ಕ್ಷತ್ರಸ್ಯ
ಕ್ಷಮ-ತೆಗೆ
ಕ್ಷಮ-ತೆ-ಯು-ಳ್ಳದ್ದೇ
ಕ್ಷಮಾದಿ
ಕ್ಷಮಿಸಿ
ಕ್ಷಮೆ-ಯಾ-ಚಿ-ಸು-ತ್ತೇನೆ
ಕ್ಷಮೆಯೇ
ಕ್ಷೀಣಿಸಿ
ಕ್ಷೀಣಿ-ಸಿದ್ದು
ಕ್ಷೀಯಂತೇ
ಕ್ಷುತ್ತ್ರಂ
ಕ್ಷೇತ್ರಕ್ಕೂ
ಕ್ಷೇತ್ರಕ್ಕೆ
ಕ್ಷೇತ್ರ-ಗ-ಳ-ಲ್ಲಿನ
ಕ್ಷೇತ್ರದ
ಕ್ಷೇತ್ರ-ದಲ್ಲಿ
ಕ್ಷೇತ್ರ-ದ-ಲ್ಲಿದ್ದ
ಕ್ಷೇತ್ರ-ದ-ಲ್ಲಿನ
ಕ್ಷೇತ್ರ-ದಲ್ಲೇ
ಕ್ಷೇತ್ರ-ದಿಂ-ದಲೇ
ಖಂಡಿತ
ಖಂಡಿ-ತ-ವಾಗಿ
ಖಂಡಿ-ತ-ವಾದಿ
ಖಂಡಿ-ತ-ವಾ-ದಿ-ಗಳು
ಖಂಡಿತಾ
ಖಚಿ-ಡಿe
ಖಚಿ-ತ-ವಾಗಿ
ಖರ-ಪ್ರ-ಮು-ಖ-ರಿ-ಗಿಂತ
ಖರೀದಿ
ಖರೀ-ದಿ-ಸಿದ್ದ
ಖರ್ಚಾ-ಗುತ್ತಾ
ಖರ್ಚು
ಖರ್ಚು-ಗ-ಳೆಲ್ಲ
ಖಲು
ಖಾಯಿ-ಲೆ-ಗಳು
ಖಾಲಿ
ಖಾಲಿ-ಜಾ-ಗ-ವನ್ನು
ಖಾಲಿ-ಯಾಗಿ
ಖಾಲಿ-ಯಾ-ಗಿತ್ತು
ಖಾಸಗಿ
ಖಾಸ-ಗಿ-ಯಲ್ಲಿ
ಖಾಸಗೀ
ಖೀರನ್ನ
ಖುದ್ದು
ಖುಶಿ-ಯಿಂದ
ಖುಷಿ-ಪ-ಡು-ವ-ವರೆ
ಖುಷಿ-ಯಿಂದ
ಖೇದ
ಖೇದ-ವಿದೆ
ಖ್ಯಾತ-ನಾ-ಮ-ರಾ-ಗ-ಬ-ಹು-ದಿತ್ತು
ಖ್ಯಾತ-ರಾ-ಗಿ-ರುವ
ಖ್ಯಾತಿ-ಯನ್ನೋ
ಗಂಗಣ್ಣ
ಗಂಗ-ಣ್ಣನ
ಗಂಗ-ಣ್ಣ-ನದು
ಗಂಗ-ಣ್ಣ-ನ-ದ್ದಲ್ಲ
ಗಂಗ-ಣ್ಣ-ನನ್ನು
ಗಂಗ-ಣ್ಣ-ನಲ್ಲಿ
ಗಂಗ-ಣ್ಣ-ನ-ವ-ರನ್ನು
ಗಂಗ-ಣ್ಣ-ನ-ವ-ರಿಗೂ
ಗಂಗ-ಣ್ಣ-ನ-ವ-ರಿಗೆ
ಗಂಗ-ಣ್ಣ-ನ-ವರು
ಗಂಗ-ಣ್ಣ-ನ-ವರೇ
ಗಂಗ-ಣ್ಣ-ನ-ವ-ರೊ-ಟ್ಟಿಗೇ
ಗಂಗ-ಣ್ಣ-ನಾ-ದರು
ಗಂಗ-ಣ್ಣ-ನಿಂದ
ಗಂಗ-ಣ್ಣ-ನಿಗೂ
ಗಂಗ-ಣ್ಣ-ನಿಗೆ
ಗಂಗ-ಣ್ಣ-ನಿ-ದ್ದ-ಲ್ಲಿಗೆ
ಗಂಗ-ಣ್ಣನೂ
ಗಂಗ-ಣ್ಣನೆ
ಗಂಗ-ಣ್ಣನೇ
ಗಂಗ-ಣ್ಣ-ನೊಂ-ದಿ-ಗಿನ
ಗಂಗ-ಧ-ರ-ನಿಗೆ
ಗಂಗಾ-ಧರ
ಗಂಗಾ-ಧರಃ
ಗಂಗಾ-ಧ-ರಣ್ಣ
ಗಂಗಾ-ಧ-ರನ
ಗಂಗಾ-ಧ-ರ-ನದು
ಗಂಗಾ-ಧ-ರ-ನನ್ನು
ಗಂಗಾ-ಧ-ರ-ನನ್ನೂ
ಗಂಗಾ-ಧ-ರ-ನಿಗೆ
ಗಂಗಾ-ಧ-ರನೆ
ಗಂಗಾ-ಧ-ರನೇ
ಗಂಗಾ-ಧ-ರ-ನೊ-ಲು-ಮೆಗೆ
ಗಂಗಾ-ಧ-ರ-ಭ-ಟ್ಟ-ದಂ-ಪತೀ
ಗಂಗಾ-ಧ-ರ-ಭ-ಟ್ಟರ
ಗಂಗಾ-ಧ-ರ-ಭ-ಟ್ಟ-ರಲ್ಲಿ
ಗಂಗಾ-ಧ-ರ-ಭ-ಟ್ಟ-ರಿ-ಗಿಂತ
ಗಂಗಾ-ಧ-ರ-ಭ-ಟ್ಟ-ರಿಗೂ
ಗಂಗಾ-ಧ-ರ-ಭ-ಟ್ಟ-ರಿಗೆ
ಗಂಗಾ-ಧ-ರ-ಭ-ಟ್ಟರು
ಗಂಗಾ-ಧ-ರ-ಭ-ಟ್ಟರೂ
ಗಂಗಾ-ಧ-ರ-ಭ-ಟ್ಟ-ರೆಂ-ದರೆ
ಗಂಗಾ-ಧ-ರ-ಭ-ಟ್ಟ-ರೊಂ-ದಿ-ಗಿನ
ಗಂಗಾ-ಧ-ರ-ಭ-ಟ್ಟ-ರೊ-ಡ-ನಾ-ಟದ
ಗಂಗಾ-ಧ-ರ-ರಾದ
ಗಂಗಾ-ಧ-ರರು
ಗಂಗಾ-ಧ-ರ-ವಿ-ಭ-ಟ್ಟರ
ಗಂಗಾ-ಧ-ರೇಂದ್ರ
ಗಂಗಾ-ಧರೋ
ಗಂಗಾ-ಧಾರ
ಗಂಗಾ-ಪ್ರ-ವಾ-ಹ-ದಂತೆ
ಗಂಗೂ-ಬಾಯಿ
ಗಂಗೆ
ಗಂಗೆ-ಯನ್ನು
ಗಂಗೊ-ಳ್ಳಿ-ಯ-ವರ
ಗಂಗೊ-ಳ್ಳಿ-ಯ-ವ-ರಿಗೆ
ಗಂಗೋತ್ರಿ
ಗಂಗೋ-ತ್ರಿ-ಯಲ್ಲಿ
ಗಂಗ್ಯಾ
ಗಂಜಿ
ಗಂಟಲು
ಗಂಟೆ
ಗಂಟೆ-ಗ-ಟ್ಟಲೆ
ಗಂಟೆ-ಗಳ
ಗಂಟೆಗೆ
ಗಂಟೆಯ
ಗಂಟೆ-ಯ-ವ-ರೆಗೂ
ಗಂಡು
ಗಂಡು-ಮ-ಕ್ಕ-ಳಲ್ಲಿ
ಗಂಡು-ಮ-ಕ್ಕಳು
ಗಂಧದ
ಗಂಧ-ದ-ಗುಡಿ
ಗಂಧ-ವಿ-ಲ್ಲ-ದ-ವರು
ಗಂಧ-ವೇ-ದಿಯೊ
ಗಂಭೀರ
ಗಂಭೀ-ರ-ವಾಗಿ
ಗಂಭೀ-ರ-ವಾಣಿ
ಗಚ್ಛತಿ
ಗಟ್ಟಿ-ಯಾ-ಗುತ್ತ
ಗಟ್ಟಿ-ಯಾದ
ಗಟ್ಟಿ-ಯಾ-ದರೂ
ಗಡಿ-ಬಿ-ಡಿ-ಯ-ಲ್ಲಿ-ದ್ದೇನೆ
ಗಣಕ್ಕೆ
ಗಣನಾ
ಗಣ-ನೆ-ಯಿ-ಲ್ಲದ
ಗಣಪ
ಗಣ-ಪತಿ
ಗಣ-ಪ-ಭ-ಟ್ಟರ
ಗಣ-ಪ-ಭ-ಟ್ಟರು
ಗಣಿ
ಗಣಿ-ತ-ಸ್ಕಂಧ
ಗಣಿ-ಸು-ವುದು
ಗಣು
ಗಣೇಶ
ಗಣೇ-ಶ-ಭ-ಟ್ಟರ
ಗಣ್ಯ
ಗಣ್ಯ-ರಿ-ಗ್ಯಾ-ರಿಗೂ
ಗಣ್ಯರು
ಗಣ್ಯರೂ
ಗತಿ
ಗತಿ-ಯಾ-ಯಿತು
ಗತಿ-ಯಿ-ಲ್ಲದೇ
ಗತ್ತು
ಗತ್ಯಂ-ತ-ರ-ವಿಲ್ಲ
ಗದಾ-ಧ-ರೀಯ
ಗದ್ದೆ
ಗದ್ದೆಯ
ಗದ್ಯ-ವಿ-ರಲಿ
ಗನಾ
ಗಮನ
ಗಮ-ನ-ಕೊಟ್ಟು
ಗಮ-ನಕ್ಕೆ
ಗಮ-ನ-ವನ್ನು
ಗಮ-ನಾರ್ಹ
ಗಮ-ನಿ-ಸ-ಬ-ಹು-ದಾ-ಗಿದೆ
ಗಮ-ನಿ-ಸ-ಲೇ-ಬೇ-ಕಾದ
ಗಮ-ನಿಸಿ
ಗಮ-ನಿ-ಸಿದ
ಗಮ-ನಿ-ಸಿ-ದಂತೆ
ಗಮ-ನಿ-ಸಿ-ದರೆ
ಗಮ-ನಿ-ಸಿ-ದಾಗ
ಗಮ-ನಿ-ಸು-ತ್ತಿ-ದ್ದಾರೆ
ಗಮ-ನಿ-ಸೋಣ
ಗರಡಿ
ಗರ-ಡಿ-ಯಲ್ಲಿ
ಗರ್ಭ-ಗುಡಿ
ಗಳ-ಲ್ಲಿಯೂ
ಗಳಿ-ಸ-ಬ-ಹು-ದಿತ್ತು
ಗಳಿಸಿ
ಗಳಿ-ಸಿ-ಕೊಂ-ಡಿ-ದ್ದರು
ಗಳಿ-ಸಿ-ಕೊ-ಳ್ಳ-ಬ-ಹುದು
ಗಳಿ-ಸಿ-ಕೊ-ಳ್ಳ-ಬೇ-ಕೆಂ-ಬುದೇ
ಗಳಿ-ಸಿದ್ದ
ಗಳಿ-ಸಿ-ದ್ದ-ರಿಂದ
ಗಳಿ-ಸಿದ್ದೆ
ಗಳಿ-ಸು-ತ್ತಿದ್ದ
ಗಳಿ-ಸು-ತ್ತಿ-ದ್ದರು
ಗಳಿ-ಸು-ವಂತೆ
ಗವ-ಯ-ಮೃ-ಗ-ಗಳು
ಗಹ-ಗ-ಹಿಸಿ
ಗಾಂಭೀ-ರ್ಯ-ವಿದೆ
ಗಾಂಭೀ-ರ್ಯ-ವುಳ್ಳ
ಗಾಢ-ವಾ-ಗಿತ್ತು
ಗಾತ್ರ
ಗಾದಾ-ಧರೀ
ಗಾಯ-ಗೊಂಡು
ಗಾಯ-ತ್ರಿಯ
ಗಾಯದ
ಗಾಯ-ವಾ-ಯಿತು
ಗಾರ್ಹಸ್ಥ್ಯ
ಗಾಳಿ-ಮನೆ
ಗಾಳಿ-ಮ-ನೆ-ಯ-ವ-ರೆ-ಲ್ಲರ
ಗಾಳಿ-ಮ-ನೆ-ಅ-ಗ್ಗೆ-ರೆಯ
ಗಾವುದ
ಗಿಡ-ಮ-ರ-ಗಳ
ಗಿಡ-ಗಳ
ಗಿರೀಶ
ಗೀಳು
ಗು
ಗುಂಡಿ
ಗುಂಡಿ-ಗಿಂತ
ಗುಂಡಿಗೆ
ಗುಂಡಿಯ
ಗುಂಡಿ-ಯಲ್ಲಿ
ಗುಂಡಿಯೇ
ಗುಂಪನ್ನು
ಗುಂಪಿನ
ಗುಂಪಿ-ನ-ಲ್ಲಿದ್ದ
ಗುಂಪು
ಗುಂಪೊಂದು
ಗುಂಯ್ಗು-ಡು-ತ್ತಿದೆ
ಗುಕಾ-ರ-ಸ್ತ್ವ-ನ್ಧ-ಕಾ-ರ-ಸ್ಯಾತ್
ಗುಜ-ರಾ-ತಿನ
ಗುಟ್ಟನ್ನು
ಗುಡಿ
ಗುಡಿಯ
ಗುಡಿ-ಯನ್ನು
ಗುಡಿ-ಯಲ್ಲಿ
ಗುಡಿಯು
ಗುಡ್ಡ-ವೊಂ-ದರ
ಗುಣ
ಗುಣ-ಗ-ಣ-ಗ-ಳಿಂದ
ಗುಣ-ಗ-ಣನೇ
ಗುಣ-ಗ-ಳನ್ನು
ಗುಣ-ಗ-ಳಲ್ಲಿ
ಗುಣ-ಗ-ಳಿಂದ
ಗುಣ-ಗ-ಳಿಗೆ
ಗುಣ-ಗಳು
ಗುಣ-ಗ-ಳುಳ್ಳ
ಗುಣ-ಗಾನ
ಗುಣ-ಗಾ-ನ-ಗಳ
ಗುಣ-ದೋ-ಷ-ಗಳೇ
ಗುಣ-ಪ-ಡಿ-ಸ-ಬ-ಹು-ದಾ-ಗಿದೆ
ಗುಣ-ಪ್ರ-ಕಾ-ಶನ
ಗುಣ-ಭೂ-ಯಿ-ಷ್ಠರು
ಗುಣ-ಮು-ಖ-ರಾ-ಗು-ತ್ತಿ-ಲ್ಲ-ಎಂ-ದರೆ
ಗುಣ-ವ-ನ್ನಾಗಿ
ಗುಣ-ವನ್ನು
ಗುಣ-ವ-ಲ್ಲವೇ
ಗುಣ-ವಾ-ಗಿದೆ
ಗುಣವೇ
ಗುಣಾನ್
ಗುಣಾ-ವಿ-ರ್ಭಾ-ವ-ವನ್ನು
ಗುಣಿ-ನಿ-ಷ್ಠ-ಗು-ಣಾ-ಭಿ-ಧಾನಂ
ಗುಣೈಕ
ಗುಪ್ತ-ದಾ-ನದ
ಗುರ-ಗಳ
ಗುರವೇ
ಗುರವೋ
ಗುರಿ
ಗುರಿ-ತ-ಲು-ಪಲು
ಗುರಿ-ಮು-ಟ್ಟಿ-ದಾಗ
ಗುರಿ-ಯಾ-ಗಲಿ
ಗುರಿಯು
ಗುರು
ಗುರುಃ
ಗುರುಃ
ಗುರು-ಋ-ಣ-ವನ್ನು
ಗುರು-ಕುಲ
ಗುರು-ಕು-ಲದ
ಗುರು-ಕು-ಲ-ದಂತೆ
ಗುರು-ಕು-ಲ-ದಲ್ಲಿ
ಗುರು-ಕೃತ್ಯ
ಗುರು-ಕೃಪೆ
ಗುರು-ಗಳ
ಗುರು-ಗ-ಳನ್ನು
ಗುರು-ಗ-ಳಲ್ಲಿ
ಗುರು-ಗ-ಳಾಗಿ
ಗುರು-ಗ-ಳಾ-ಗಿ-ದ್ದಾರೆ
ಗುರು-ಗ-ಳಾದ
ಗುರು-ಗ-ಳಿಂದ
ಗುರು-ಗ-ಳಿಗೆ
ಗುರು-ಗ-ಳಿಗೇ
ಗುರು-ಗ-ಳಿ-ದ್ದಾರೆ
ಗುರು-ಗಳು
ಗುರು-ಗಳೂ
ಗುರು-ಗ-ಳೆಂ-ದರೆ
ಗುರು-ಗ-ಳೊಂ-ದಿಗೆ
ಗುರು-ಗ-ಳೊ-ಡನೆ
ಗುರು-ಗುಣ
ಗುರು-ಚ-ರ-ಣಕ್ಕೆ
ಗುರು-ಚ-ರ-ಣ-ಗ-ಳಿಗೆ
ಗುರು-ಚ-ರ-ಣಾ-ರ-ವಿಂ-ದ-ಗ-ಳಿಗೆ
ಗುರು-ಜ-ನ-ರನ್ನು
ಗುರು-ಜ-ನ-ರಾ-ಗಲು
ಗುರು-ಜ-ನ-ರಿಗೆ
ಗುರು-ಜ-ನ-ರೊಂ-ದಿಗೆ
ಗುರು-ಜ್ಞಾನ
ಗುರು-ತ-ರ-ವಾ-ದದ್ದು
ಗುರು-ತಿ-ಸಲು
ಗುರು-ತಿಸಿ
ಗುರು-ತಿ-ಸಿ-ದ್ದಾರೆ
ಗುರು-ತಿ-ಸಿ-ದ್ದೇನೆ
ಗುರು-ತಿ-ಸು-ತ್ತೇನೆ
ಗುರು-ತಿ-ಸು-ವಲ್ಲಿ
ಗುರುತ್ವ
ಗುರು-ದೆವ
ಗುರು-ದೇ-ವಾತ್
ಗುರು-ಪ್ರ-ಸಾದ
ಗುರು-ಪ್ರ-ಸಾ-ದರ
ಗುರು-ಭಕ್ತಿ
ಗುರು-ಭ-ಕ್ತಿ-ಗಿಂ-ತಿಲೂ
ಗುರು-ಭ-ಕ್ತಿಯ
ಗುರುಭ್ಯೋ
ಗುರು-ಮು-ಖೇನ
ಗುರು-ಮೂರ್ತಿ
ಗುರು-ರಿ-ತ್ಯ-ಭೀ-ಧೀ-ಯತೇ
ಗುರು-ರೇವ
ಗುರು-ರ್ದೆವೋ
ಗುರು-ರ್ದೇವಿ
ಗುರು-ರ್ಬ್ರಹ್ಮಾ
ಗುರು-ರ್ಮಾತಾ
ಗುರು-ರ್ವಿ-ಷ್ಣುಃ
ಗುರು-ವಂ-ದನಾ
ಗುರು-ವಂ-ದನೆ
ಗುರು-ವಂ-ದ-ನೆಯೇ
ಗುರು-ವನ್ನು
ಗುರು-ವ-ರ್ಯರು
ಗುರು-ವಾ-ಗಿ-ರು-ತ್ತಾನೆ
ಗುರು-ವಾ-ದ-ವನು
ಗುರು-ವಾ-ರ-ಗಳ
ಗುರು-ವಾ-ರ-ದಿಂದ
ಗುರು-ವಿಗೆ
ಗುರು-ವಿ-ದ್ದರೆ
ಗುರು-ವಿನ
ಗುರು-ವಿ-ನಿಂದ
ಗುರು-ವಿ-ನಿಂ-ದಲೇ
ಗುರು-ವಿ-ನೊ-ಡನೆ
ಗುರುವು
ಗುರು-ವೃಂ-ದಕ್ಕೆ
ಗುರು-ವೆಂ-ದರೆ
ಗುರು-ವೆಂದು
ಗುರು-ವೆಂದೇ
ಗುರು-ಶಾಂತ
ಗುರು-ಶಾಂ-ತ-ಸ್ವಾ-ಮಿ-ಗಳ
ಗುರು-ಶಿ-ಷ್ಯರ
ಗುರು-ಶ್ರೇ-ಷ್ಠ-ರಲ್ಲಿ
ಗುರು-ಸ-ನ್ನಿ-ಧಾ-ನ-ದಲ್ಲಿ
ಗುರು-ಸಾ-ನ್ನಿ-ಧ್ಯ-ವಿ-ಲ್ಲದೆ
ಗುರು-ಶಿಷ್ಯ
ಗುರು-ಶಿ-ಷ್ಯರ
ಗುರು-ಶಿ-ಷ್ಯ-ಸಂ-ಬಂ-ಧಕ್ಕೂ
ಗುರು-ಹಿ-ರಿ-ಯರ
ಗುರೂ-ಪ-ದಿ-ಷ್ಟವೂ
ಗುರೋ-ರ್ಹಿತಂ
ಗುರ್ವ-ನು-ಗ್ರ-ಹ-ದಿಂ-ದಲೇ
ಗುಲ-ಗಂಜಿ
ಗುಳಿ-ಗೆ-ಗ-ಳನ್ನು
ಗುಹ್ಯಂ
ಗೂಗಂ-ಗಾ-ಧರ
ಗೂಡಿ-ನೊ-ಳಗೆ
ಗೃಹ-ಕಾ-ರ್ಯ-ಗ-ಳಲ್ಲಿ
ಗೃಹ-ವಾಗಿ
ಗೃಹ-ಸ್ಥರ
ಗೃಹ-ಸ್ಥ-ರಾದ
ಗೃಹ-ಸ್ಥಾ-ಶ್ರ-ಮದ
ಗೃಹಿ-ಣಿ-ಯ-ರಾ-ಗಿ-ದ್ದಾರೆ
ಗೃಹಿ-ಣಿ-ಯಾದ
ಗೃಹಿಣೀ
ಗೃಹೇ
ಗೃಹ್ಯ-ಪ್ರ-ಯೋ-ಗ-ವನ್ನು
ಗೆದ್ದ-ವ-ರಲ್ಲ
ಗೆದ್ದ-ವರು
ಗೆದ್ದಿ-ದ್ದ-ರಿಂದ
ಗೆದ್ದಿ-ದ್ದೇನೆ
ಗೆದ್ದಿ-ರು-ವು-ದಲ್ಲ
ಗೆಲ್ಲ-ಬ-ಲ್ಲದು
ಗೆಲ್ಲು-ವ-ವರು
ಗೆಲ್ಲು-ವುದು
ಗೆಳೆಯ
ಗೇಟನ್ನು
ಗೊಂದ-ಲ-ವನ್ನು
ಗೊಂದ-ಲ-ವಿಲ್ಲ
ಗೊಂಬೆ
ಗೊತ್ತಾ-ಗದ
ಗೊತ್ತಾ-ಗ-ಬೇ-ಕಾದ
ಗೊತ್ತಾ-ಗಿದ್ದು
ಗೊತ್ತಾ-ದರೆ
ಗೊತ್ತಾ-ಯಿತು
ಗೊತ್ತಾ-ಯಿ-ತು-ಯಾ-ಕೆಂ-ದರೆ
ಗೊತ್ತಿತ್ತು
ಗೊತ್ತಿ-ರ-ಲಿಲ್ಲ
ಗೊತ್ತಿ-ರು-ತ್ತದೆ
ಗೊತ್ತಿ-ರು-ವು-ದನ್ನು
ಗೊತ್ತಿಲ್ಲ
ಗೊತ್ತಿ-ಲ್ಲದ
ಗೊತ್ತಿ-ಲ್ಲ-ದಿ-ರು-ವಿಕೆ
ಗೊತ್ತಿ-ಲ್ಲದೇ
ಗೊತ್ತು
ಗೊತ್ತೇ
ಗೊದ್ದ-ಲ-ಬೀ-ಳು-ವಿಗೆ
ಗೊರಿಲ
ಗೊಳಿ-ಸಿದ
ಗೊಳಿ-ಸು-ತ್ತದೆ
ಗೋಕರ್ಣ
ಗೋಕ-ರ್ಣದ
ಗೋಕ-ರ್ಣ-ದತ್ತ
ಗೋಕ-ರ್ಣ-ದಲ್ಲಿ
ಗೋಕ-ರ್ಣ-ದಿಂದ
ಗೋಕ-ರ್ಣ-ಮಂ-ಡ-ಲ-ದ-ಲ್ಲಿಯೇ
ಗೋಚ-ರ-ವಾ-ಗು-ವು-ದ-ಲ್ಲದೆ
ಗೋಚ-ರ-ವಾದ
ಗೋಚ-ರಿ-ಸ-ಲಾ-ರಂ-ಭಿ-ಸಿ-ದಾಗ
ಗೋಚ-ರಿ-ಸಿದ್ದೇ
ಗೋಚ-ರಿ-ಸು-ತ್ತದೆ
ಗೋಚ-ರಿ-ಸುವ
ಗೋಚ-ರಿ-ಸು-ವರೋ
ಗೋಜಿಗೆ
ಗೋತ್ರೋ-ತ್ಪ-ನ್ನರು
ಗೋಪ್ಯ-ತೆ-ಗ-ಳಿಂದ
ಗೋಮಯ
ಗೋಳನ್ನು
ಗೋಳಿಕೈ
ಗೋವ-ತ್ಸ-ಗಳ
ಗೋವಿಂ-ದ-ರಾವ್
ಗೋಶಾಲೆ
ಗೋಷ್ಠಿ
ಗೋಷ್ಠಿ-ಗ-ಳಲ್ಲಿ
ಗೋಷ್ಠೇ
ಗೌತ-ಮ-ನಂತೂ
ಗೌತ-ಮರು
ಗೌರವ
ಗೌರ-ವಕ್ಕೆ
ಗೌರ-ವ-ಗ-ಳಿಸಿ
ಗೌರ-ವ-ಗಳು
ಗೌರ-ವದ
ಗೌರ-ವ-ದಿಂದ
ಗೌರ-ವ-ಪೂ-ರ್ವಕ
ಗೌರ-ವ-ಭಾ-ವ-ವನ್ನು
ಗೌರ-ವ-ವನ್ನು
ಗೌರ-ವ-ವಿದೆ
ಗೌರ-ವ-ವೊಂದೇ
ಗೌರ-ವ-ಸಂ-ಭಾ-ವ-ನೆ-ಯನ್ನು
ಗೌರ-ವ-ಸ-ಲ-ಹಾ-ಗಾ-ರರು
ಗೌರ-ವಾ-ದ-ರಕ್ಕೆ
ಗೌರ-ವಾ-ದ-ರ-ಗ-ಳನ್ನು
ಗೌರ-ವಾ-ದ-ರ-ಗಳು
ಗೌರ-ವಾ-ಧ್ಯ-ಕ್ಷರ
ಗೌರ-ವಾ-ಧ್ಯ-ಕ್ಷ-ರ-ನ್ನಾಗಿ
ಗೌರ-ವಾ-ನ್ವಿ-ತ-ರಾ-ಗಿದ್ದು
ಗೌರ-ವಾ-ರ್ಹ-ರಾ-ಗಿ-ದ್ದ-ವರು
ಗೌರ-ವಿಸಿ
ಗೌರ-ವಿ-ಸಿದ್ದು
ಗೌರ-ವಿ-ಸು-ತ್ತಿ-ರು-ವುದು
ಗೌರ-ವ-ಭಕ್ತಿ
ಗ್ಯಾರೆ-ರ-ವರು
ಗ್ರಂಥ
ಗ್ರಂಥ-ಕಾ-ರ-ರಂತೆ
ಗ್ರಂಥಕ್ಕೆ
ಗ್ರಂಥ-ಗಳ
ಗ್ರಂಥ-ಗ-ಳನ್ನು
ಗ್ರಂಥ-ಗ-ಳಲ್ಲಿ
ಗ್ರಂಥ-ಗಳು
ಗ್ರಂಥ-ಗ-ಳೆಲ್ಲಾ
ಗ್ರಂಥದ
ಗ್ರಂಥ-ದ-ಮೂ-ಲ-ವನ್ನು
ಗ್ರಂಥ-ದಲ್ಲಿ
ಗ್ರಂಥ-ದ-ಲ್ಲಿದೆ
ಗ್ರಂಥ-ವನ್ನು
ಗ್ರಂಥ-ವನ್ನೂ
ಗ್ರಂಥ-ವಾಗಿ
ಗ್ರಂಥ-ವಾ-ದ್ದ-ರಿಂದ
ಗ್ರಂಥ-ವಿದು
ಗ್ರಂಥವು
ಗ್ರಂಥ-ಸಂ-ಪಾ-ದನ
ಗ್ರಂಥ-ಸಾ-ತ್ಪಾ-ಠ-ಮಾ-ಡು-ತ್ತಿ-ದ್ದರು
ಗ್ರಂಥ-ಹಾ-ರವು
ಗ್ರಂಥಾ-ಧ್ಯ-ಯನ
ಗ್ರಂಥಾ-ರೂ-ಢವೇ
ಗ್ರಂಥಾ-ಲಯ
ಗ್ರಂಥಾ-ವ-ಲೋ-ಕ-ನ-ದಿಂದ
ಗ್ರಂಥಿ-ಯನ್ನು
ಗ್ರನ್ಥ-ಹಾ-ರವು
ಗ್ರಹಣ
ಗ್ರಹ-ಣ-ಯೋ-ಗ್ಯ-ತೆಗೆ
ಗ್ರಹಿಕೆ
ಗ್ರಹಿ-ಸ-ಬ-ಹುದು
ಗ್ರಹಿ-ಸ-ಲಾ-ರ-ದ-ವನ
ಗ್ರಹಿಸಿ
ಗ್ರಹ-ನ-ಕ್ಷ-ತ್ರ-ಗಳ
ಗ್ರಾಮ
ಗ್ರಾಮ-ಗ-ಳಿಗೆ
ಗ್ರಾಮದ
ಗ್ರಾಮದ
ಗ್ರಾಮ-ದಲಿ
ಗ್ರಾಮ-ದಲ್ಲಿ
ಗ್ರಾಮ-ದ-ಲ್ಲಿನ
ಗ್ರಾಮ-ದವ
ಗ್ರಾಮ-ದ-ವರು
ಗ್ರಾಮೀಣ
ಗ್ರಾಹಕ
ಗ್ರೇಡ್
ಘಂಟೆ-ಯಾ-ಯಿತು
ಘಟನೆ
ಘಟ-ನೆ-ಗ-ಳನ್ನು
ಘಟ-ನೆ-ಗಳು
ಘಟ-ನೆಯ
ಘಟ-ನೆ-ಯನ್ನು
ಘಟಿ-ಸು-ತ್ತಿದ್ದ
ಘಟ್ಟದ
ಘಣಿ
ಘನ-ಗಂ-ಭೀ-ರ-ಮು-ದ್ರೆ-ಯಿಂದ
ಘನತೆ
ಘನ-ತೆ-ಯನ್ನು
ಘಮ-ಘಮ
ಘೋಷ
ಘೋಷಣೆ
ಚ
ಚಂದ-ನೈ-ರ್ಮೃ-ತ್ತಿ-ಕಯಾ
ಚಂದ್ರ-ಗು-ತ್ತಿ-ರಸ್ತೆ
ಚಂದ್ರ-ಗು-ಪ್ತನ
ಚಂದ್ರ-ಶೇ-ಖ-ರ-ರ-ವರು
ಚಂಪೂ-ರಾ-ಮಾ-ಯಣ
ಚಕ್ರ-ದಿಂದ
ಚಕ್ರ-ಪಾಣಿ
ಚಕ್ಷು-ರು-ನ್ಮೀ-ಲಿತಂ
ಚಟು-ಚ-ಟಿ-ಕೆ-ಯಿಂದ
ಚಟು-ವ-ಟಿ-ಕೆ-ಗಳ
ಚಟು-ವ-ಟಿ-ಕೆ-ಗ-ಳನ್ನು
ಚಟು-ವ-ಟಿ-ಕೆ-ಗ-ಳಲ್ಲಿ
ಚಟು-ವ-ಟಿ-ಕೆ-ಗ-ಳಲ್ಲೂ
ಚಟು-ವ-ಟಿ-ಕೆ-ಗ-ಳಿಗೆ
ಚಟು-ವ-ಟಿ-ಕೆ-ಗಳು
ಚಟು-ವ-ಟಿ-ಕೆ-ಯಲ್ಲಿ
ಚಟು-ವ-ಟಿ-ಕೆ-ಯ-ಲ್ಲಿಯೂ
ಚಟು-ವ-ಟಿ-ಕೆ-ಯೊಂ-ದಿಗೆ
ಚಡ-ಪ-ಡಿ-ಸು-ತ್ತಿ-ದ್ದರು
ಚತು-ರ-ತೆಯೂ
ಚತು-ರರು
ಚತು-ರ್ದ-ಶ-ಲ-ಕ್ಷಣೀ
ಚತು-ರ್ಮು-ಖನೆ
ಚತು-ರ್ವಿಧ
ಚಪಲ
ಚಪ್ಪ-ಲಿ-ಯಿ-ಲ್ಲದೆ
ಚಪ್ಪಾಳೆ
ಚಮ-ತ್ಕಾರ
ಚಯ-ಮಾ-ಸ್ಕಂ-ದತಿ
ಚರಕ
ಚರ-ಕ-ಸಂ-ಹಿತಾ
ಚರ-ಕ-ಸಂ-ಹಿತೆ
ಚರ-ಕ-ಸಂ-ಹಿ-ತೆ-ಯನ್ನು
ಚರ-ಕ-ಶು-ಶ್ರುತ
ಚರ-ಣಕೆ
ಚರ-ಣಾ-ರ-ವಿಂ-ದಾ-ಭ್ಯಾಂ
ಚರಿ-ತಾ-ಮೃ-ತ-ವನ್ನು
ಚರಿ-ತ್ರೆ-ಯನ್ನು
ಚರಿ-ತ್ರೆ-ಯಲ್ಲಿ
ಚರ್ಚಾ
ಚರ್ಚಾ-ಕೂ-ಟ-ಗ-ಳಲ್ಲಿ
ಚರ್ಚಾ-ಕೂ-ಟ-ಗ-ಳಿಗೆ
ಚರ್ಚಾ-ಪ-ಟು-ವಾ-ಗಿ-ದ್ದ-ರೆಂದು
ಚರ್ಚಾ-ಸ್ಪರ್ಧೆ
ಚರ್ಚಾ-ಸ್ಪ-ರ್ಧೆ-ಗ-ಳಲ್ಲಿ
ಚರ್ಚಾ-ಸ್ಪ-ರ್ಧೆ-ಗಳು
ಚರ್ಚಾ-ಸ್ಪ-ರ್ಧೆಗೆ
ಚರ್ಚಿ-ಸಲು
ಚರ್ಚಿಸಿ
ಚರ್ಚಿ-ಸಿ-ದೆವು
ಚರ್ಚಿ-ಸಿ-ದ್ದಾರೆ
ಚರ್ಚಿ-ಸಿ-ರ-ಬೇಕು
ಚರ್ಚಿ-ಸು-ತ್ತಿ-ದ್ದರು
ಚರ್ಚಿ-ಸು-ತ್ತಿ-ದ್ದೆವು
ಚರ್ಚೆ
ಚರ್ಚೆ-ಗಾಗಿ
ಚರ್ಚೆಯ
ಚರ್ಚೆ-ಯಲ್ಲಿ
ಚರ್ಚೆ-ಲೇ-ಖ-ನ-ಭಾ-ಷಣ
ಚರ್ಚೇ-ಯಲಿ
ಚರ್ಛಾ
ಚಲ-ನ-ಶೀ-ಲ-ವಾ-ಗಿ-ರ-ಬೇಕು
ಚಲನೆ
ಚಲ-ನೆಯ
ಚಲಿ-ಸ-ಲಾ-ರದೋ
ಚಲೊವಾ
ಚಳಿ
ಚಳು-ವಳಿ
ಚಹಾ
ಚಾಕ-ಚ-ಕ್ಯತೆ
ಚಾಕ-ಚ-ಕ್ಯ-ತೆ-ಯನ್ನು
ಚಾಚಿದ
ಚಾಚು-ತ್ತಿದ್ದ
ಚಾಚು-ತ್ತಿ-ದ್ದ-ವರು
ಚಾಚು-ವುದು
ಚಾಚೂ
ಚಾಣ-ಕ್ಯನ
ಚಾಣ-ಕ್ಯ-ನಂತೆ
ಚಾಣಾ-ಕ್ಷ-ತ-ನ-ದಿಂದ
ಚಾಣಾಕ್ಷ
ಚಾತು-ರ್ಮಾಸ್ಯ
ಚಾಪೆ
ಚಾಪೆಯ
ಚಾಮ-ರಾ-ಜ-ಪುರಂ
ಚಾಮ-ರಾ-ಜ-ಪೇಟೆ
ಚಾಮುಂ-ಡಿಯ
ಚಾಮುಂ-ಡೇ-ಶ್ವರಿ
ಚಾರಿ-ತ್ರಿಕ
ಚಾರ್ಟ-ರ್ಡ್
ಚಾರ್ಯರು
ಚಾಲನೆ
ಚಾಲ್ಸ್
ಚಿ
ಚಿಂತನ
ಚಿಂತ-ನ-ಗಳ
ಚಿಂತನಾ
ಚಿಂತನೆ
ಚಿಂತ-ನೆ-ಗ-ಳನ್ನು
ಚಿಂತ-ನೆ-ಗ-ಳಿಂ-ದಾಗಿ
ಚಿಂತ-ನೆಯ
ಚಿಂತ-ನೆ-ಯಲ್ಲಿ
ಚಿಂತ-ನೆ-ಯಾ-ಗಿತ್ತು
ಚಿಂತ-ನೆ-ಯಿಂದ
ಚಿಂತಿ-ಸ-ಬೇಡಿ
ಚಿಂತೆ
ಚಿಂತೆ-ಗ-ಳನ್ನು
ಚಿಂತೆ-ಯಾ-ಗಿತ್ತು
ಚಿಂತೆ-ಯಿಂದ
ಚಿಂತೆ-ಯೊಂ-ದಿತ್ತು
ಚಿಂಪಾಂ-ಜಿ-ಗ-ಳಿಂದ
ಚಿಕಿ-ತ್ಸಾ-ಲ-ಯ-ದಲ್ಲೇ
ಚಿಕಿತ್ಸೆ
ಚಿಕಿ-ತ್ಸೆಗೆ
ಚಿಕಿ-ತ್ಸೆ-ಗೆಂದು
ಚಿಕ್ಕ
ಚಿಕ್ಕಂ-ದಿ-ನಿಂ-ದಲೇ
ಚಿಕ್ಕ-ದೊಂದು
ಚಿಕ್ಕಪ್ಪ
ಚಿಕ್ಕ-ಪ್ಪನ
ಚಿಕ್ಕ-ಪ್ಪ-ನ-ವ-ರಾದ
ಚಿಕ್ಕ-ಪ್ಪ-ನಾ-ಗ-ಬೇಕು
ಚಿಕ್ಕ-ಮ-ಕ್ಕ-ಳಿಗೆ
ಚಿಕ್ಕ-ಮಾ-ರ್ಕೆ-ಟಿಗೆ
ಚಿಕ್ಕಮ್ಮ
ಚಿಕ್ಕ-ಮ್ಮನ
ಚಿಕ್ಕ-ವ-ನಾದ
ಚಿಕ್ಕ-ವ-ನಾ-ದರು
ಚಿಕ್ಕ-ವನು
ಚಿಕ್ಕ-ವ-ರಾ-ದರೂ
ಚಿಕ್ಕ-ವರು
ಚಿಕ್ಕ-ಚೊಕ್ಕ
ಚಿಗುರು
ಚಿತಾ-ವ-ಣೆ-ಯೊಂ-ದಿಗೆ
ಚಿತ್ತ
ಚಿತ್ತಂ
ಚಿತ್ತ-ದಿಂದ
ಚಿತ್ತ-ಭಿ-ತ್ತಿಯ
ಚಿತ್ತ-ವನ್ನು
ಚಿತ್ತಸ್ಯ
ಚಿತ್ತಾ-ಕ-ರ್ಷಕ
ಚಿತ್ತಾ-ಪ-ಹಾ-ರ-ಕಾಃ
ಚಿತ್ತಿ
ಚಿತ್ತೇ
ಚಿತ್ರ-ಗಳ
ಚಿತ್ರ-ವಿ-ರ-ಲಿಲ್ಲ
ಚಿತ್ರಿ-ತ-ವಾ-ಗು-ತ್ತದೆ
ಚಿತ್ರಿ-ಸುವ
ಚಿದ್ಯಾ-ರ್ಥಿ-ಗಳ
ಚಿನ್ನ
ಚಿನ್ನದ
ಚಿರ-ಮು-ದ್ರೆ-ಯಾಗಿ
ಚಿರ-ಸ್ಥಾ-ಯಿ-ಗೊ-ಳಿ-ಸಿ-ದಂ-ತ-ವರು
ಚಿರಾ-ಯು-ವಾ-ಗ-ಲೆಂ-ಬುದೆ
ಚಿಲುಮೆ
ಚಿಲು-ಮೆಯ
ಚಿಲು-ಮೆ-ಯಾ-ಗಿ-ರುವ
ಚಿಸೌ
ಚುರು-ಕಾಗಿ
ಚುರು-ಕು-ಮ-ತಿ-ಯಾದ
ಚೆಂದ-ವೇನು
ಚೆನ್ನಾಗಿ
ಚೆನ್ನಾ-ಗಿಯೂ
ಚೆನ್ನೇ-ನ-ಹಳ್ಳಿ
ಚೇತ-ರಿ-ಸಿ-ಕೊ-ಳ್ಳು-ವಂತೆ
ಚೇತೋ-ಹಾರಿ
ಚೇತೋ-ಹಾ-ರಿ-ಯಾದ
ಚೇದ್ವ-ಯಸಿ
ಚೈತನ್ಯ
ಚೈನಿ-ನಿಂದ
ಚೈವೇ-ಹಾ-ಸ್ತ್ಯ-ಕಾ-ಮತಾ
ಚೊಕ್ಕ-ಮ-ನ-ಸ್ಸಿ-ನ-ವರು
ಚೋಪ-ನೀ-ತಾಚ
ಚ್ಯುತಿ-ಯ-ನ್ನುಂ-ಟು-ಮಾ-ಡಿಲ್ಲ
ಛತ್ರ-ದಂತೆ
ಛತ್ರ-ದಲ್ಲಿ
ಛಲದ
ಛಳಿ-ಗಾ-ಲ-ದಲ್ಲಿ
ಛಾತ್ರ
ಛಾತ್ರ-ನಿ-ವಹ
ಛಾತ್ರ-ನಿ-ವ-ಹವೇ
ಛಾತ್ರ-ಮಾ-ನ-ಸ-ಕ-ಮ-ಲ-ವನ್ನು
ಛಾತ್ರರ
ಛಾತ್ರ-ವಾ-ತ್ಸಲ್ಯ
ಛಾತ್ರ-ವೃತ್ತಿ
ಛಾತ್ರ-ಸಂ-ಪತ್ತು
ಛಾತ್ರಾ-ವಾ-ಸ-ವಾಗಿ
ಛಾತ್ರೋ-ನ್ನ-ತಿ-ಕಾ-ರಕ
ಛಾಪನ್ನ
ಛಾಪನ್ನು
ಛಾಯಾ-ಚಿತ್ರ
ಛಾಯಾ-ಸಂ-ಭ್ರಮ
ಜ
ಜಂಬ
ಜಂಬ-ಕೊ-ಚ್ಚಿ-ಕೊಂಡು
ಜಂಭ-ವಂ-ತ-ರಲ್ಲ
ಜಗಕೆ
ಜಗ-ತ್ತನ್ನು
ಜಗ-ತ್ತಿನ
ಜಗ-ತ್ತಿ-ನಲ್ಲಿ
ಜಗ-ತ್ಸ-ರ್ವಮ್
ಜಗದ
ಜಗ-ದೀ-ಶ-ತ-ರ್ಕಾ-ಲಂ-ಕಾ-ರರ
ಜಗ-ದ್ಗುರು
ಜಗ-ದ್ಗು-ರು-ಗಳ
ಜಗ-ದ್ಗು-ರು-ಗ-ಳಾದ
ಜಗ-ನ್ಮಾ-ತೆ-ಯಾದ
ಜಗವ
ಜಗ್ಗ-ದಿ-ದ್ದಾಗ
ಜಘಾನ
ಜಟಿಲ
ಜಟಿ-ಲತೆ
ಜತೆ-ಯಲ್ಲಿ
ಜತೆಯೇ
ಜನ
ಜನ
ಜನ-ಜಂ-ಗುಳಿ
ಜನ-ಜೀ-ವ-ನಕೆ
ಜನನ
ಜನ-ಪ್ರಿ-ಯ-ರಾ-ಗಿ-ದ್ದ-ವರು
ಜನ-ಮಾ-ನ-ಸ-ದಲ್ಲಿ
ಜನರ
ಜನ-ರನ್ನು
ಜನ-ರಲ್ಲಿ
ಜನ-ರಿ-ಗ-ಲ್ಲದೇ
ಜನ-ರಿಗೂ
ಜನ-ರಿಗೆ
ಜನ-ರಿ-ಗೇನೋ
ಜನರು
ಜನರೇ
ಜನ-ರೊ-ಡನೆ
ಜನ-ವರಿ
ಜನ-ಸಾ-ಮಾ-ನ್ಯ-ರಿಗೂ
ಜನ-ಸಾ-ಮಾ-ನ್ಯರು
ಜನಾಂ-ಗ-ದ-ವ-ರನ್ನು
ಜನಾ-ನು-ರಾ-ಗಿ-ಗಳು
ಜನಿತಾ
ಜನಿ-ಸಿದ
ಜನಿ-ಸಿ-ದ-ವ-ನಾ-ದರೂ
ಜನಿ-ಸಿದ್ದು
ಜನು-ಮದ
ಜನ್ಮ
ಜನ್ಮ-ಗ-ಳಲ್ಲಿ
ಜನ್ಮ-ಜಾತ
ಜನ್ಮ-ಜಾ-ತ-ನಾಗಿ
ಜನ್ಮ-ಜಾ-ತ-ವಾ-ಗಿಯೇ
ಜನ್ಮತ
ಜನ್ಮತಃ
ಜನ್ಮದ
ಜನ್ಮ-ದಲ್ಲಿ
ಜನ್ಮ-ದಿಂ-ದಲೇ
ಜನ್ಮ-ಪ-ಡೆದು
ಜನ್ಮ-ಪಾಪ
ಜನ್ಮಶಃ
ಜನ್ಮ-ಹೇ-ತವಃ
ಜಪ
ಜಪಃ
ಜಪ-ಕಾ-ಲ-ದಲ್ಲಿ
ಜಪ-ಕಾ-ಲ-ದಲ್ಲೇ
ಜಪ-ಕಾಲೇ
ಜಪ-ಕ್ಕಿಂ-ತಲೂ
ಜಪ-ಗ-ಣ-ನೆಗೆ
ಜಪ-ಗ-ಣ-ನೆ-ಯನ್ನು
ಜಪ-ಗ-ಳಲ್ಲಿ
ಜಪ-ಗಳು
ಜಪ-ತಾಂ
ಜಪತಿ
ಜಪದ
ಜಪ-ದ-ಕಟ್ಟೆ
ಜಪ-ದ-ಕ-ಟ್ಟೆ-ಮ-ಠದ
ಜಪ-ದಲ್ಲಿ
ಜಪ-ದಿಂದ
ಜಪ-ಮಾ-ಡು-ವ-ವನ
ಜಪ-ಮಾ-ಡು-ವಾ-ಗಿನ
ಜಪ-ಮಾಲೆ
ಜಪ-ಮಾ-ಲೆ-ಯಲ್ಲಿ
ಜಪ-ಯಜ್ಞ
ಜಪ-ಯ-ಜ್ಞದ
ಜಪ-ಯ-ಜ್ಞಸ್ಯ
ಜಪ-ಯಜ್ಞಾ
ಜಪ-ಯ-ಜ್ಞೋಽಸ್ಮಿ
ಜಪ-ವನ್ನು
ಜಪ-ವಾಗಿ
ಜಪವು
ಜಪ-ವೆಂದೂ
ಜಪವೇ
ಜಪ-ಸಂ-ಖ್ಯಾಂ
ಜಪ-ಸ-ರ-ದಲ್ಲಿ
ಜಪ-ಸ್ಥಾ-ನವೂ
ಜಪ-ಸ್ಸ್ಯಾ-ದ-ಕ್ಷ-ರಾ-ವೃ-ತ್ತಿಃ
ಜಪಾ-ಚ-ರ-ಣೆಯ
ಜಪಾ-ರ್ಚ-ನಾ-ದಿಕಂ
ಜಪಿ-ಸ-ಬೇಕು
ಜಪಿ-ಸುವ
ಜಪಿ-ಸು-ವುದು
ಜಪೇತ್
ಜಪೇತ್ತಂ
ಜಪೇ-ದ್ದ-ಶಸು
ಜಪೇ-ನ್ಮೌ-ಕ್ತಿ-ಕ-ಪಂ-ಕ್ತಿ-ವತ್
ಜಪ್ಯಾತ್
ಜಮೀ-ನನ್ನು
ಜಯ
ಜಯಂತಿ
ಜಯಂ-ತಿ-ಯೊ-ಡನೆ
ಜಯ-ಭೇರಿ
ಜಯ-ಭೇ-ರಿ-ಯನ್ನು
ಜಯ-ಲ-ಕ್ಷ್ಮಿ-ಯ-ವರು
ಜಯಶ್ರೀ
ಜಯಿ-ಸು-ವುದೇ
ಜರು-ಗಿ-ದರೂ
ಜರ್ಮ-ನಿಯ
ಜಲ
ಜಲವ
ಜಲಾತ್
ಜಲೇಷು
ಜಲ್ಪ
ಜಲ್ಪಕ್ಕೆ
ಜವ-ರ-ಶೆಟ್ಟಿ
ಜವಾ-ಬ್ದಾರಿ
ಜವಾ-ಬ್ದಾ-ರಿ-ಗ-ಳನ್ನು
ಜವಾ-ಬ್ದಾ-ರಿ-ಯನ್ನು
ಜವಾ-ಬ್ದಾ-ರಿ-ಯ-ನ್ನೆಲ್ಲ
ಜವಾ-ಬ್ದಾ-ರಿ-ಯಲ್ಲೇ
ಜವಾ-ಬ್ದಾ-ರಿ-ಯುತ
ಜಹಾತಿ
ಜಾಂಬ-ವಂ-ತ-ನಂತೆ
ಜಾಂಬ-ವಂ-ತರು
ಜಾಗ-ಗ-ಳಿಗೆ
ಜಾಗಟೆ
ಜಾಗ-ದಲ್ಲಿ
ಜಾಗೃತ
ಜಾಗೃ-ತ-ವಾ-ಗಿ-ರು-ತ್ತದೆ
ಜಾಗೃ-ತಿ-ಯನ್ನು
ಜಾಗ್ರತ
ಜಾಗ್ರ-ತ-ರಾದ
ಜಾಗ್ರತೆ
ಜಾಗ್ರ-ತೆ-ವ-ಹಿ-ಸು-ತ್ತಿ-ದ್ದರು
ಜಾಡನ್ನು
ಜಾಣ್ಮೆ
ಜಾತಿ
ಜಾತಿಯ
ಜಾತಿ-ಮ-ತ-ಗ-ಳಲ್ಲಿ
ಜಾತೇನ
ಜಾತೋ
ಜಾತ್ರೆ-ಯಂದು
ಜಾನ್
ಜಾಪಕ
ಜಾಮೀ-ನಿನ
ಜಾಯತೇ
ಜಾಯ-ಮಾನ
ಜಾರದ
ಜಾರಿ-ಗೊ-ಳಿ-ಸುವ
ಜಾರಿ-ಸು-ತ್ತಿ-ರು-ತ್ತವೆ
ಜಾರು-ವು-ದ-ನ-ರಿ-ತರೂ
ಜಾಲ-ಇದು
ಜಾಲ-ದಿಂದ
ಜಾವ
ಜಾಸ್ತಿನೇ
ಜಾಹಿ-ರಾತು
ಜಾಹೀ-ರಾ-ತನ್ನು
ಜಾಹೀ-ರಾತು
ಜಿ
ಜಿಎ-ನ್-ಅ-ನಂ-ತ-ವ-ರ್ಧನ
ಜಿಜ್ಞಾ-ಸು-ಗ-ಳಾಗಿ
ಜಿಜ್ಞಾ-ಸು-ಗ-ಳಿಗೆ
ಜಿಜ್ಞಾ-ಸು-ಗಳು
ಜಿಜ್ಞಾ-ಸೆಯ
ಜಿಜ್ಞಾ-ಸೆ-ಯನ್ನು
ಜಿಟಿ-ನಾ-ರಾ-ಯಣ
ಜಿನು-ಗು-ತ್ತದೆ
ಜಿಪಿ-ಎಸ್
ಜಿಮಂ-ಜು-ನಾಥ
ಜಿಮಂ-ಜು-ನಾ-ಥರ
ಜಿಮಂ-ಜು-ನಾ-ಥರು
ಜಿಯ-ವರ
ಜಿಲ್ಲಾ
ಜಿಲ್ಲೆಯ
ಜಿಲ್ಲೆ-ಯ-ವ-ರಾದ
ಜಿಲ್ಲೆ-ಯಿಂದ
ಜಿಹ್ವೆ-ಯಲ್ಲಿ
ಜೀರ್ಣ-ವಾ-ಗಿತ್ತು
ಜೀರ್ಣಿ-ಸಿ-ಕೊಂಡು
ಜೀರ್ಣೋ-ದ್ಧಾ-ರ-ಗೊಂಡು
ಜೀವ
ಜೀವನ
ಜೀವ-ನಕ್ಕೆ
ಜೀವ-ನ-ಕ್ಕೊಂದು
ಜೀವ-ನ-ಗ-ಮ-ನದ
ಜೀವ-ನದ
ಜೀವ-ನ-ದಲ್ಲಿ
ಜೀವ-ನ-ದ-ಲ್ಲಿಯೂ
ಜೀವ-ನ-ದಿಂದ
ಜೀವ-ನ-ದು-ದ್ದಕ್ಕೂ
ಜೀವ-ನ-ಯಾನ
ಜೀವ-ನ-ವನ್ನು
ಜೀವ-ನ-ವನ್ನೇ
ಜೀವ-ನ-ವಾ-ಗಿದೆ
ಜೀವ-ನವು
ಜೀವ-ನವೇ
ಜೀವ-ನ-ಶಾಸ್ತ್ರ
ಜೀವ-ನ-ಸಾ-ಗಿ-ಸುತ್ತಾ
ಜೀವ-ನ-ಸುಮ
ಜೀವ-ನ-ಸು-ಮ-ವನ್ನು
ಜೀವ-ನಾ-ಡಿ-ಗ-ಳ್ಳ-ಲ್ಲೊ-ಬ್ಬ-ನಾ-ಗಿ-ರು-ವುದು
ಜೀವ-ನಾ-ಡಿಯೇ
ಜೀವ-ನಾ-ನು-ಭ-ವ-ವಾ-ದಂತೆ
ಜೀವನ್
ಜೀವ-ರ-ಲ್ಲಿಯೂ
ಜೀವ-ವಿ-ರುವ
ಜೀವವೂ
ಜೀವ-ವೈ-ಜ್ಞಾ-ನಿಕ
ಜೀವಾ-ಳ-ವಾಗಿ
ಜೀವಿ
ಜೀವಿ-ಕೆಗೆ
ಜೀವಿ-ಕೆ-ಯನ್ನು
ಜೀವಿ-ಕೆಯು
ಜೀವಿ-ಗಳ
ಜೀವಿ-ಗ-ಳಲ್ಲಿ
ಜೀವಿ-ಗ-ಳಾಗಿ
ಜೀವಿ-ಗ-ಳಿಗೂ
ಜೀವಿ-ಗಳು
ಜೀವಿ-ತದ
ಜೀವಿ-ಯನ್ನು
ಜೀವಿಸಿ
ಜೀವೋ
ಜುಟ್ಟು
ಜುಲೈ
ಜೂನ್
ಜೃಂಭಣಂ
ಜೇಡಿ-ಮಣ್ಣು
ಜೈನ-ಮು-ನಿ-ಗಳು
ಜೈರಾಮ
ಜೈವಿ-ಕ-ವ-ಸ್ತು-ಗ-ಳಿ-ದ್ದರೂ
ಜೊತೆ
ಜೊತೆ-ಗಿದ್ದ
ಜೊತೆ-ಗಿನ
ಜೊತೆಗೂ
ಜೊತೆ-ಗೂಡಿ
ಜೊತೆಗೆ
ಜೊತೆಗೇ
ಜೊತೆ-ಜೊ-ತೆಗೆ
ಜೊತೆ-ಜೊ-ತೆ-ಯಲ್ಲಿ
ಜೊತೆ-ಜೊ-ತೆ-ಯಾಗಿ
ಜೊತೆ-ಯಲ್ಲಿ
ಜೊತೆ-ಯ-ಲ್ಲಿದ್ದೆ
ಜೊತೆ-ಯಾಗಿ
ಜೋಡ-ಣೆ-ಯನ್ನು
ಜೋಡಿ-ಯಾಗಿ
ಜೋಡಿಸಿ
ಜೋಡಿ-ಸಿ-ದಂತೆ
ಜೋಡಿ-ಸಿದೆ
ಜೋತು
ಜೋಯ್ಸ
ಜೋರಾಗಿ
ಜೋಳಿ-ಗೆ-ಯನ್ನು
ಜ್ಙಾನ-ಭೀ-ಕ್ಷು-ಗ-ಳಾಗಿ
ಜ್ಙಾನ-ವನ್ನು
ಜ್ಞಾನ
ಜ್ಞಾನ-ಕ್ಕಾಗಿ
ಜ್ಞಾನಕ್ಕೆ
ಜ್ಞಾನ-ಗಂ-ಗಾ-ಧರ
ಜ್ಞಾನ-ಗಂ-ಗೆ-ಯನ್ನೇ
ಜ್ಞಾನ-ಗ-ಙ್ಗಾ-ಧರ
ಜ್ಞಾನ-ಗ-ಳನ್ನು
ಜ್ಞಾನ-ಗ-ಳಿ-ಕೆ-ಯಲ್ಲಿ
ಜ್ಞಾನದ
ಜ್ಞಾನ-ದಾ-ತರೂ
ಜ್ಞಾನ-ದಾ-ಹ-ದಿಂ-ದಾ-ಗಿಯೇ
ಜ್ಞಾನ-ದಾ-ಹ-ವನ್ನು
ಜ್ಞಾನ-ದಿಂದ
ಜ್ಞಾನ-ದೀ-ವಿ-ಗೆ-ಯಿಂದ
ಜ್ಞಾನ-ದೇ-ಗು-ಲ-ವಿದು
ಜ್ಞಾನ-ಪ-ಣ್ಯ-ನಾ-ಗ-ಲಿಲ್ಲ
ಜ್ಞಾನ-ಪಿ-ಪಾ-ಸು-ಗ-ಳಾದ
ಜ್ಞಾನ-ಬುತ್ತಿ
ಜ್ಞಾನ-ಬೋ-ಧ-ನೆಯ
ಜ್ಞಾನ-ಭಿ-ಕ್ಷೆ-ಗಾಗಿ
ಜ್ಞಾನ-ಮಾ-ರ್ಗಶ್ಚ
ಜ್ಞಾನ-ವಂತೂ
ಜ್ಞಾನ-ವನ್ನು
ಜ್ಞಾನ-ವನ್ನೂ
ಜ್ಞಾನ-ವಿ-ರದೆ
ಜ್ಞಾನ-ವಿ-ರಹ
ಜ್ಞಾನ-ವಿ-ರು-ವು-ದನ್ನು
ಜ್ಞಾನವು
ಜ್ಞಾನ-ವು-ಳ್ಳ-ವ-ರಂತು
ಜ್ಞಾನವೂ
ಜ್ಞಾನ-ವೆಂಬ
ಜ್ಞಾನ-ಶಂ-ಕರ
ಜ್ಞಾನಾಂ-ಜ-ನ-ಶ-ಲಾ-ಕಯಾ
ಜ್ಞಾನಾ-ದ್ಧ್ಯಾನಂ
ಜ್ಞಾನಾಯ
ಜ್ಞಾನಾ-ರ್ಜ-ನೆ-ಗಾಗಿ
ಜ್ಞಾನಾ-ರ್ಜ-ನೆಗೆ
ಜ್ಞಾನಾ-ರ್ಜ-ನೆ-ಗೆಂದು
ಜ್ಞಾನಾ-ರ್ಜ-ನೆ-ಗೋ-ಸ್ಕರ
ಜ್ಞಾನಿ-ಗ-ಳಿಂದ
ಜ್ಞಾನಿ-ಗಳು
ಜ್ಞಾಪಕ
ಜ್ಞಾಪ-ಕಕ್ಕೆ
ಜ್ಞಾಪ-ಸಿ-ಕೊ-ಳ್ಳು-ತ್ತದೆ
ಜ್ಯೋತಿಷ
ಜ್ಯೋತಿ-ಷ-ವಿ-ದ್ವ-ನ್ಮ-ಧ್ಯ-ಮಾ-ವನ್ನು
ಜ್ಯೋತಿ-ಷ-ಶಾ-ಸ್ತ್ರಕ್ಕೆ
ಜ್ಯೌತಿಷ
ಜ್ವರ-ವಿ-ದೆ-ಯೆಂದು
ಜ್ವಲಂತ
ಜ್ವಲಿತಂ
ಜ್ಷಾಪ-ಕ-ವಾ-ಗ-ಲಾ-ರಂ-ಬಿ-ಸಿದೆ
ಝರಿ-ತೊ-ರೆ-ಯಾಗಿ
ಟಾಕೀ-ಸಿನ
ಟಿಕೆ
ಟಿವಿ
ಟೊಂಕ
ಟ್ರಂಕಿನ
ಟ್ರಾವ-ಲ್ಸ್
ಟ್ರೆಡಿ-ಶನ್
ಠಾಣೆಗೆ
ಠಿಕಾಣಿ
ಡಂಬಾ-ಚಾ-ರ-ವನ್ನು
ಡಾ
ಡಾಎಸ್ಜಿ
ಡಾರ್ವಿನ್
ಡಾರ್ವಿನ್ನ
ಡಾರ್ವಿ-ನ್ನಿ-ಗಿಂತ
ಡಾರ್ವಿ-ನ್ವಾದ
ಡಿ
ಡಿಬ-ನು-ಮಯ್ಯ
ಡಿಸೆಂ-ಬ-ರ್ನಲ್ಲಿ
ಡೆಮೋ
ಡೋಂಗಿ-ತ-ನದ
ತಂಗಿ
ತಂಗಿ-ದ್ದೆವು
ತಂಗಿ-ಯರ
ತಂಗಿ-ಯ-ರನ್ನು
ತಂಗಿ-ಯರು
ತಂಗಿ-ಯ-ರೆಲ್ಲಾ
ತಂಗಿಯೂ
ತಂಗು-ತ್ತಿ-ದ್ದು-ದುಂಟು
ತಂಡ
ತಂತು
ತಂತ್ರ-ಗ್ರಂ-ಥ-ಗ-ಳಲ್ಲಿ
ತಂತ್ರ-ಗ್ರಂ-ಥ-ಗಳು
ತಂತ್ರ-ಶಾಸ್ತ್ರ
ತಂತ್ರ-ಶಾ-ಸ್ತ್ರದ
ತಂತ್ರ-ಸಾರ
ತಂತ್ರಾ-ಗ-ಮ-ಗ-ಳಲ್ಲೋ
ತಂತ್ರೋಕ್ತ
ತಂದ
ತಂದರು
ತಂದಿ-ಕೊಂಡು
ತಂದಿ-ಟ್ಟನು
ತಂದಿತು
ತಂದಿತ್ತು
ತಂದಿದೆ
ತಂದಿ-ದ್ದಾ-ರೆಂದು
ತಂದಿದ್ದೆ
ತಂದಿ-ರುವ
ತಂದಿವೆ
ತಂದು
ತಂದು-ಕೊಂಡು
ತಂದು-ಕೊಟ್ಟ
ತಂದು-ಕೊ-ಟ್ಟಿದೆ
ತಂದು-ಕೊ-ಟ್ಟಿದ್ದು
ತಂದು-ಕೊ-ಳ್ಳು-ವುದು
ತಂದೆ
ತಂದೆಗೆ
ತಂದೆ-ತಾ-ಯಿ-ಗ-ಳಾ-ಗಿ-ದ್ದಾರೆ
ತಂದೆಯ
ತಂದೆ-ಯಲ್ಲಿ
ತಂದೆ-ಯ-ವರ
ತಂದೆ-ಯ-ವ-ರನ್ನು
ತಂದೆ-ಯ-ವ-ರಾದ
ತಂದೆ-ಯ-ವರು
ತಂದೆ-ಯ-ವರೂ
ತಂದೆ-ಯ-ವ-ರೊಂ-ದಿಗೂ
ತಂದೆ-ಯಾ-ಗಿ-ರು-ವ-ವರು
ತಂದೆಯೂ
ತಂದೆ-ಯೊಂ-ದಿಗೆ
ತಂದೆ-ತಾಯಿ
ತಂದೆ-ತಾ-ಯಿ-ಗಳ
ತಂದೆ-ತಾ-ಯಿ-ಗ-ಳಾಗಿ
ತಂದೆ-ತಾ-ಯಿಗೆ
ತಂಪ-ತೊ-ರೆದ
ತಕ್ಕ
ತಕ್ಕಂ-ತಹ
ತಕ್ಕಂತೆ
ತಕ್ಕ-ಡಿ-ಯಲ್ಲಿ
ತಕ್ಕ-ನಾದ
ತಕ್ಕ-ಪಾ-ಠ-ವನ್ನು
ತಕ್ಕ-ವಳೇ
ತಕ್ಕು-ದಾದ
ತಕ್ಕು-ದಾ-ದದ್ದು
ತಕ್ಷಣ
ತಕ್ಷ-ಣಕ್ಕೆ
ತಕ್ಷ-ಣವೇ
ತಜ್ಞ-ನಾ-ಗಿ-ರು-ತ್ತಿದ್ದ
ತಜ್ಞ-ನಾದ
ತಟ-ಸ್ಥ-ನಾ-ಗ-ಲಿಲ್ಲ
ತಟ್ಟಿ-ದರು
ತಟ್ಟಿ-ಸರ
ತಟ್ಟೀ-ಸ-ರ-ದಲ್ಲಿ
ತಟ್ಟು-ವುದು
ತಟ್ಟೆ
ತಟ್ಟೆ-ಗ-ಳನ್ನು
ತಟ್ಟೆ-ಬ-ಟ್ಟ-ಲು-ಯನ್ನು
ತಡ
ತಡ-ಮಾ-ಡದೆ
ತಡ-ವಾಗಿ
ತಡ-ವಾ-ಗಿ-ಯಾ-ದರೂ
ತಡೆ
ತಡೆ-ಗಟ್ಟಿ
ತಡೆದ
ತಡೆ-ಯ-ಲಾ-ಗದೇ
ತಡೆ-ಯುವ
ತಡೆ-ಯು-ವು-ದಾಗಿ
ತಣಿ-ಯ-ಲಿಲ್ಲ
ತಣ್ಣೀ-ರು-ಬ-ಟ್ಟೆಯೇ
ತತ್
ತತ್ಕ್ಷಣ
ತತ್ಕ್ಷ-ಣ-ದಲ್ಲೇ
ತತ್ತ್ವ-ದೊ-ಡ-ಗೂಡೆ
ತತ್ತ್ವ-ಶಾಸ್ತ್ರ
ತತ್ತ್ವ-ಶಾ-ಸ್ತ್ರದ
ತತ್ಪ-ರ-ನಾ-ಗಿ-ದ್ದೇನೆ
ತತ್ರ
ತತ್ವ
ತತ್ವ-ಗಳು
ತತ್ವ-ವನ್ನು
ತತ್ಸಂ-ಬಂಧಿ
ತಥಾ
ತಥಾ-ರಾಮೇ
ತದ-ನಂ-ತರ
ತದ-ಫಲಂ
ತದೇ-ಕ-ಚಿ-ತ್ತ-ದಿಂದ
ತದೇವ
ತದ್ಧಿತ
ತದ್ಧಿ-ತಾಂತ
ತದ್ಭ-ವೇ-ದೇವಂ
ತದ್ವಿ-ರು-ದ್ಧ-ವಾದ
ತನಕ
ತನ-ಗಾಗಿ
ತನಗೆ
ತನಗೇ
ತನು
ತನು-ಮ-ನ-ಧನ
ತನ್ನ
ತನ್ನದೇ
ತನ್ನನ್ನು
ತನ್ನ-ಯೋ-ಚನಾ
ತನ್ನಲ್ಲಿ
ತನ್ನಾ-ರ್ಜಿತ
ತನ್ನೊಂ-ದಿಗೆ
ತನ್ಮಿ-ತ್ರ-ಮಾ-ಪದಿ
ತನ್ಮು-ಖೇನ
ತನ್ಮೂ-ಲಕ
ತಪ-ಸ್ಸಿನ
ತಪಾ-ಸಣೆ
ತಪ್ಪದೇ
ತಪ್ಪನ್ನು
ತಪ್ಪ-ಬ-ಹುದು
ತಪ್ಪ-ಲಿನ
ತಪ್ಪ-ಲ್ಲ-ವೆಂ-ದೆ-ನಿ-ಸಿ-ರ-ಬೇಕು
ತಪ್ಪಾ-ಗ-ಲಾ-ರದು
ತಪ್ಪಾ-ಗಿ-ದ್ದರೂ
ತಪ್ಪಿ-ನಿಂದ
ತಪ್ಪಿಲ್ಲ
ತಪ್ಪಿ-ಸಿ-ಕೊ-ಳ್ಳು-ವುದು
ತಪ್ಪಿ-ಸಿ-ದ್ದಾರೆ
ತಪ್ಪು
ತಪ್ಪು-ಗಳ
ತಪ್ಪು-ಗ-ಳನ್ನು
ತಪ್ಪು-ಗ-ಳ-ನ್ನೆಲ್ಲಾ
ತಪ್ಪು-ಗ-ಳಿಗೆ
ತಪ್ಪು-ಗಳು
ತಮ-ಗಾಗಿ
ತಮಗೂ
ತಮಗೆ
ತಮ-ಗೊ-ದ-ಗಿತು
ತಮ-ಗೊ-ಲಿ-ಯಿತು
ತಮನ್ನು
ತಮಸಿ
ತಮಾಷೆ
ತಮಾ-ಷೆಯ
ತಮ್ಮ
ತಮ್ಮತ್ತ
ತಮ್ಮ-ದಾ-ಗಿ-ಸಿ-ಕೊಂ-ಡಿರಿ
ತಮ್ಮದು
ತಮ್ಮನಾ
ತಮ್ಮನ್ನು
ತಮ್ಮ-ಯ್ಯಾ-ವ-ಧಾ-ನಿ-ಗಳ
ತಮ್ಮಲ್ಲಿ
ತಮ್ಮ-ಲ್ಲಿಯೇ
ತಮ್ಮಲ್ಲೇ
ತಮ್ಮ-ವರು
ತಮ್ಮಿಂ-ದಾ-ಗುವ
ತಮ್ಮೆ-ಡೆಗೆ
ತಮ್ಮೊಂ-ದಿಗೇ
ತಮ್ಮೊ-ಳ-ಗಿದೆ
ತಮ್ಮೊ-ಳ-ಗಿನ
ತಯಾ-ರಾ-ಗಿದೆ
ತಯಾ-ರಾ-ಗು-ತ್ತಿದ್ದ
ತಯಾರಿ
ತಯಾ-ರಿ-ಮಾ-ಡು-ತ್ತಿ-ದ್ದೆವು
ತಯಾ-ರಿ-ಸ-ಬೇ-ಕಾ-ಗಿತ್ತು
ತಯಾ-ರಿ-ಸಲು
ತಯಾ-ರಿ-ಸಿ-ಕೊಟ್ಟೆ
ತಯಾ-ರಿ-ಸಿ-ತ್ತಾರೆ
ತಯಾರು
ತರಂ-ಗ-ಗ-ಳನ್ನು
ತರ-ಕಾರಿ
ತರ-ಗತಿ
ತರ-ಗ-ತಿ-ಗಳ
ತರ-ಗ-ತಿ-ಗ-ಳಿಗೆ
ತರ-ಗ-ತಿ-ಗಳು
ತರ-ಗ-ತಿಗೆ
ತರ-ಗ-ತಿಯ
ತರ-ಗ-ತಿ-ಯನ್ನು
ತರ-ಗ-ತಿ-ಯಲ್ಲಿ
ತರ-ಗ-ತಿಯೂ
ತರ-ಗೆ-ಲೆಯು
ತರ-ಬೇ-ತಿ-ಗೊ-ಳಿ-ಸಿ-ದ್ದಾರೆ
ತರಹ
ತರಾ-ತು-ರಿ-ಯ-ಲ್ಲಿ-ದ್ದಾರೆ
ತರಿ-ಕ್ಷೇ-ತ್ರ-ವನ್ನು
ತರಿ-ಕ್ಷೇ-ತ್ರ-ವೆಂದು
ತರಿ-ಸು-ತ್ತದೆ
ತರುಣ
ತರು-ಣ್ಯವೂ
ತರು-ತ್ತದೆ
ತರುವ
ತರು-ವರು
ತರ್ಕ
ತರ್ಕ-ದಿಂದ
ತರ್ಕ-ಪ್ರಾ-ಧ್ಯಾ-ಪ-ಕ-ರಾ-ಗಿ-ರುವ
ತರ್ಕ-ಬದ್ಧ
ತರ್ಕ-ಬ-ದ್ಧ-ವಾದ
ತರ್ಕ-ವಿ-ಷ-ಯದ
ತರ್ಕ-ವ್ಯಾ-ಕ-ರ-ಣಾದಿ
ತರ್ಕ-ವ್ಯಾ-ಕ-ರ-ಣಾನಿ
ತರ್ಕ-ಶಾಸ್ತ್ರ
ತರ್ಕ-ಶಾ-ಸ್ತ್ರಕ್ಕೆ
ತರ್ಕ-ಶಾ-ಸ್ತ್ರದ
ತರ್ಕ-ಶಾ-ಸ್ತ್ರ-ದಂ-ತಹ
ತರ್ಕ-ಶಾ-ಸ್ತ್ರ-ದಲ್ಲಿ
ತರ್ಕ-ಶಾ-ಸ್ತ್ರ-ವನ್ನು
ತರ್ಕ-ಶಾ-ಸ್ತ್ರ-ವಿ-ರಲಿ
ತರ್ಕ-ಸಂ-ಗ್ರಹ
ತರ್ಕ-ಸಂ-ಗ್ರ-ಹದ
ತರ್ಕ-ಸಂ-ಗ್ರ-ಹ-ದಲ್ಲಿ
ತರ್ಕ-ಸಂ-ಗ್ರ-ಹ-ವನ್ನು
ತರ್ಕ-ಸಂ-ಗ್ರ-ಹವು
ತರ್ಕಾ-ಧ್ಯ-ಯ-ನ-ಕ್ಕಾಗಿ
ತರ್ಜ-ನಿಯ
ತರ್ಜನೀ
ತಲ-ಸ್ಪ-ರ್ಶಿ-ಯಾಗಿ
ತಲು-ಪಲು
ತಲು-ಪಿ-ದರೆ
ತಲು-ಪಿ-ದೆ-ಯೆಂ-ದರೆ
ತಲು-ಪಿಲ್ಲ
ತಲು-ಪಿ-ಸುವ
ತಲು-ಪು-ತ್ತಿ-ದ್ದಾರೆ
ತಲು-ಪು-ವಂ-ತಾ-ಯಿತು
ತಲೆ-ಕೆ-ಡಿ-ಸಿ-ಕೊ-ಳ್ಳದ
ತಲೆ-ಕೊ-ಟ್ಟಿರಿ
ತಲೆಗೆ
ತಲೆ-ತ-ಗ್ಗಿ-ಸಿದೆ
ತಲೆ-ದೂ-ಗು-ವಂತೆ
ತಲೆ-ದೋ-ರಿತು
ತಲೆ-ಬಾಗಿ
ತಲೆ-ಯಲ್ಲಿ
ತಳ-ದಲ್ಲಿ
ತಳ-ದ-ಲ್ಲಿ-ರುವ
ತಳ-ಸ್ಪ-ರ್ಶಿ-ಯಾಗಿ
ತಳ-ಹ-ದಿಯ
ತಳಿ-ರು-ತೋ-ರಣ
ತಳ್ಳಿದೆ
ತಳ್ಳಿ-ಬಿ-ಟ್ಟಿತು
ತಳ್ಳಿ-ಬಿ-ಡು-ತ್ತೇವೆ
ತಳ್ಳು-ತ್ತದೆ
ತವಕ
ತವ-ರ-ಲ್ಲವೇ
ತವರು
ತವ-ರೂರು
ತಸ್ಮಾ-ಜ್ಜಪ
ತಸ್ಮಾ-ದ-ಭಿ-ಯು-ಕ್ತಾ-ನಾಂ
ತಸ್ಮಾ-ನ್ನಿತ್ಯಂ
ತಸ್ಮಿ-ನ್ಪ್ರಾಪ್ತೇ
ತಸ್ಮೈ
ತಸ್ಮೈಶ್ರೀ
ತಾಂತ್ರಿ-ಕ-ವಾಗಿ
ತಾಂಬೂ-ಲ-ಪೂ-ರಿ-ತ-ವ-ದ-ನ-ರಾಗಿ
ತಾಣ
ತಾತ್ಕಾ-ಲಿಕ
ತಾತ್ಕಾ-ಲಿ-ಕ-ವಾಗಿ
ತಾತ್ಪರ್ಯ
ತಾನು
ತಾನೆ
ತಾನೆ
ತಾನೇ
ತಾನೇನೂ
ತಾನೊಬ್ಬ
ತಾಪ
ತಾಪ-ದಿಂದ
ತಾಯಿ
ತಾಯಿ-ಗಿಂ-ತಲೂ
ತಾಯಿ-ನ-ವ-ರ-ಲ್ಲಿತ್ತು
ತಾಯಿಯ
ತಾಯಿ-ಯಿಂದ
ತಾಯಿ-ತಂ-ದೆ-ಯರು
ತಾರ-ತಮ್ಯ
ತಾರ-ತ-ಮ್ಯ-ಮಾಡಿ
ತಾರ-ತ-ಮ್ಯ-ವಿ-ರ-ಲಿಲ್ಲ
ತಾರ್ಕಿಕ
ತಾರ್ಕಿ-ಕ-ವಾದ
ತಾಲೂ-ಕಿನ
ತಾಲೂ-ಕಿ-ನ-ವನು
ತಾಲೂ-ಕಿ-ನ-ವರೇ
ತಾಲೂ-ಕಿ-ನಿಂದ
ತಾಲ್ಲೂ-ಕಿನ
ತಾಲ್ಲೂಕು
ತಾಲ್ಲೂ-ಕು-ಗಳ
ತಾಳ
ತಾಳ-ಮ-ದ್ದಲೆ
ತಾಳ-ಮ-ದ್ದ-ಲೆ-ಯಲ್ಲೂ
ತಾಳ-ಮ-ದ್ದ-ಳೆಯ
ತಾಳಿ
ತಾಳ್ಮೆ
ತಾಳ್ಮೆ-ಗ-ಳೆ-ರಡೂ
ತಾವ-ಚ್ಛಾ-ಸ್ತ್ರಾಣಿ
ತಾವು
ತಾವೂ
ತಾವೇ
ತಾವೊ-ಬ್ಬರೇ
ತಿಂಗಳ
ತಿಂಗ-ಳ-ಕಾಲ
ತಿಂಗ-ಳಲ್ಲಿ
ತಿಂಗ-ಳಿಂದ
ತಿಂಗ-ಳಿ-ನಲ್ಲಿ
ತಿಂಗ-ಳಿ-ನಲ್ಲೂ
ತಿಂಗ-ಳಿ-ನಿಂದ
ತಿಂಗಳು
ತಿಂಗ-ಳು-ಗ-ಳ-ದ್ದಾ-ಗಿ-ದ್ದರೂ
ತಿಂಗ-ಳು-ಗಳು
ತಿಂಡಿ
ತಿಂಡಿ-ಗಾಗಿ
ತಿಂಡಿಗೆ
ತಿಂಡಿ-ಯನ್ನು
ತಿಂಡಿಯೂ
ತಿಂದ
ತಿಂದರೆ
ತಿಂದು
ತಿಙಂತ
ತಿತೀ-ರ್ಷು-ರ್ದು-ಸ್ತರಂ
ತಿದ್ದಿ
ತಿದ್ದಿ-ಕೊಂಡು
ತಿದ್ದಿ-ಕೊ-ಡು-ವುದು
ತಿದ್ದಿ-ಕೊ-ಳ್ಳು-ತ್ತಿ-ದ್ದರು
ತಿದ್ದುವ
ತಿನ್ನದೇ
ತಿನ್ನು-ವು-ದಿಲ್ಲ
ತಿಭು-ವನ
ತಿಮಿಂ-ಗಿಲ
ತಿಮಿ-ರಾಂ-ಧಸ್ಯ
ತಿರ-ಸ್ಕ-ರಿ-ಲಾ-ಗದೇ
ತಿರ-ಸ್ಕ-ರಿಸಿ
ತಿರ-ಸ್ಕ-ರಿ-ಸಿದ್ದ
ತಿರ-ಸ್ಕ-ರಿ-ಸುವ
ತಿರು-ಗಾಟ
ತಿರು-ಗಾ-ಡುತ್ತಾ
ತಿರುಗಿ
ತಿರು-ಗಿದೆ
ತಿರು-ಗು-ತ್ತಿ-ದ್ದಾನೆ
ತಿರು-ಗು-ತ್ತಿ-ದ್ದೆವು
ತಿರು-ಗು-ತ್ತಿ-ರು-ವ-ವರು
ತಿರು-ಗು-ತ್ತಿ-ರುವೆ
ತಿರು-ಪತಿ
ತಿರು-ಳನ್ನು
ತಿರು-ವಿ-ನ-ಲ್ಲಿ-ರುವ
ತಿರೋ-ಹಿ-ತ-ವಾ-ಗದೇ
ತಿರ್ಯ-ಕ್ಕೃತ್ವಾ
ತಿಳಿ
ತಿಳಿದ
ತಿಳಿ-ದಂತೆ
ತಿಳಿ-ದ-ವರ
ತಿಳಿ-ದಿ-ದ್ದರೂ
ತಿಳಿ-ದಿದ್ದೆ
ತಿಳಿ-ದಿ-ದ್ದೇನೆ
ತಿಳಿ-ದಿ-ರು-ತ್ತದೆ
ತಿಳಿ-ದಿ-ರುವ
ತಿಳಿ-ದಿ-ರು-ವಂತೆ
ತಿಳಿ-ದಿ-ರು-ವುದು
ತಿಳಿದು
ತಿಳಿ-ದು-ಕೊಂಡು
ತಿಳಿ-ದು-ಕೊಂಡೆ
ತಿಳಿ-ದು-ಕೊ-ಳ್ಳಲು
ತಿಳಿ-ದು-ಕೊ-ಳ್ಳು-ತ್ತಿ-ದ್ದರು
ತಿಳಿ-ದು-ಕೊ-ಳ್ಳುವ
ತಿಳಿದೇ
ತಿಳಿದೋ
ತಿಳಿ-ದ್ದಿತ್ತು
ತಿಳಿ-ಯದ
ತಿಳಿ-ಯ-ದಂತೆ
ತಿಳಿ-ಯದೆ
ತಿಳಿ-ಯ-ದೆಂದು
ತಿಳಿ-ಯ-ದೆಯೋ
ತಿಳಿ-ಯದೇ
ತಿಳಿ-ಯ-ಬೇಕು
ತಿಳಿ-ಯ-ಲಾ-ರೆವು
ತಿಳಿ-ಯಲು
ತಿಳಿ-ಯಾ-ಗಿ-ರು-ತ್ತದೆ
ತಿಳಿ-ಯಿತು
ತಿಳಿ-ಯು-ತ್ತದೆ
ತಿಳಿ-ಯು-ತ್ತಿ-ರ-ಲಿಲ್ಲ
ತಿಳಿ-ಯು-ತ್ತೇವೆ
ತಿಳಿ-ಯುವ
ತಿಳಿ-ಯು-ವಂತೆ
ತಿಳಿ-ಯು-ವು-ದಿಲ್ಲ
ತಿಳಿ-ವ-ಳಿಕೆ
ತಿಳಿವು
ತಿಳಿ-ಸ-ಲಾ-ಯಿತು
ತಿಳಿ-ಸಾರು
ತಿಳಿಸಿ
ತಿಳಿ-ಸಿ-ಕೊಟ್ಟು
ತಿಳಿ-ಸಿ-ಕೊ-ಡು-ತ್ತಿ-ದ್ದರು
ತಿಳಿ-ಸಿ-ಕೊ-ಡು-ವುದು
ತಿಳಿ-ಸಿದ
ತಿಳಿ-ಸಿ-ದಂತೆ
ತಿಳಿ-ಸಿ-ದರು
ತಿಳಿ-ಸಿದೆ
ತಿಳಿ-ಸಿದ್ದ
ತಿಳಿ-ಸಿ-ದ್ದರು
ತಿಳಿ-ಸಿ-ದ್ದಾನೆ
ತಿಳಿ-ಸು-ತ್ತದೆ
ತಿಳಿ-ಸು-ತ್ತವೆ
ತಿಳಿ-ಸುತ್ತಾ
ತಿಳಿ-ಸು-ತ್ತಾರೆ
ತಿಳಿ-ಸುವ
ತಿಳಿ-ಸು-ವಲ್ಲಿ
ತಿಳಿ-ಸು-ವು-ದ-ಲ್ಲದೇ
ತಿಳಿ-ಸು-ವುದು
ತಿಳಿ-ಹಾಸ್ಯ
ತಿಳಿ-ಹಾ-ಸ್ಯ-ಮಿ-ಶ್ರಿ-ತ-ವಾದ
ತಿಳಿ-ಹೇ-ಳುವ
ತಿಳಿ-ಹೇ-ಳು-ವಂತೆ
ತಿಳು-ವ-ಳಿಕೆ
ತಿಳು-ವ-ಳಿ-ಕೆಗೆ
ತಿಳು-ವ-ಳಿ-ಕೆ-ಯನ್ನು
ತಿಸ್ರ
ತೀಕ್ಷ್ಣ
ತೀಕ್ಷ್ಣ-ಗೊ-ಳಿ-ಸು-ವಲ್ಲಿ
ತೀರಿ-ಸ-ಲಂತೂ
ತೀರಿ-ಸ-ಲಾ-ರದ
ತೀರಿ-ಸಿತು
ತೀರಿ-ಸುತ್ತಾ
ತೀರಿ-ಸು-ವುದೇ
ತೀರ್ಥ-ರೂ-ಪರು
ತೀರ್ಥ-ರೂ-ಪರೇ
ತೀರ್ಥ-ರೂ-ಪು-ಗ-ಳಾದ
ತೀರ್ಮಾನ
ತೀರ್ಮಾ-ನಕ್ಕೆ
ತೀರ್ಮಾ-ನ-ವಾ-ಗಿತ್ತು
ತೀರ್ಮಾ-ನಿ-ಸು-ತ್ತಿ-ದ್ದರು
ತೀರ್ಮಾ-ನಿ-ಸು-ತ್ತಿ-ದ್ದೆವು
ತೀವ್ರ-ಗೊಂ-ಡಿತು
ತೀವ್ರ-ತ-ರ-ವಾದ
ತೀವ್ರ-ತೆ-ಯಿಂದ
ತೀವ್ರ-ವಾಗಿ
ತು
ತುಂಟ
ತುಂಬ
ತುಂಬಲು
ತುಂಬಾ
ತುಂಬಿ
ತುಂಬಿ-ಕೊಂಡು
ತುಂಬಿ-ಕೊ-ಳ್ಳ-ಬ-ಹುದು
ತುಂಬಿ-ಕೊ-ಳ್ಳು-ತ್ತಿತ್ತು
ತುಂಬಿದ
ತುಂಬಿ-ದಂ-ತಾ-ಯಿತು
ತುಂಬಿ-ದರು
ತುಂಬಿ-ದ್ದರೂ
ತುಂಬಿ-ದ್ದಾರೆ
ತುಂಬಿ-ರಲಿ
ತುಂಬಿ-ರು-ತ್ತದೆ
ತುಂಬಿ-ರು-ತ್ತಿತ್ತು
ತುಂಬಿ-ರುವ
ತುಂಬಿ-ಸಿ-ಕೊಂಡು
ತುಂಬಿ-ಸು-ವು-ದ-ರಲ್ಲೇ
ತುಂಬಿ-ಹೋ-ಯಿತು
ತುಂಬು
ತುಂಬು-ತ್ತಾನೆ
ತುಂಬು-ವುದು
ತುಟಿ-ಗ-ಳ-ಲು-ಗು-ತ್ತಿ-ದ್ದರೂ
ತುಡಿವ
ತುತ್ತಿ-ಗಾಗಿ
ತುಮ-ಕೂ-ರಿಗೆ
ತುಮ-ಕೂರು
ತುರು-ವೇ-ಕೆರೆ
ತುರ್ಜಮೆ
ತುಲ-ನಾ-ತ್ಮ-ಕ-ವಾಗಿ
ತುಳಸಿ
ತುಳಿ-ಯುವ
ತುಳು-ಕು-ವು-ದಿಲ್ಲ
ತೂಗಿ
ತೂಗಿ-ಸಿ-ಕೊಂಡು
ತೂಗುವ
ತೃಣಾನಿ
ತೃತೀಯ
ತೃಪ್ತ-ನಾ-ಗ-ಲಿಲ್ಲ
ತೃಪ್ತ-ರಾ-ದರು
ತೃಪ್ತಿ
ತೃಪ್ತಿ-ಕಾ-ಣು-ವ-ವರು
ತೃಪ್ತಿ-ಪ-ಟ್ಟಿದ್ದು
ತೃಪ್ತಿಯ
ತೃಪ್ತಿ-ಯನ್ನು
ತೃಪ್ತಿ-ಯಿಂದ
ತೃಪ್ತಿ-ಹೊಂ-ದು-ವುದು
ತೃಷೆ-ಯನ್ನು
ತೆಗೆದು
ತೆಗೆ-ದು-ಕೊಂಡು
ತೆಗೆ-ದು-ಕೊ-ಳ್ಳ-ಬೇಕು
ತೆಗೆ-ಯು-ವುದು
ತೆತ್ತು
ತೆಪ್ಪಾ-ರಿನ
ತೆರ-ದಲ್ಲಿ
ತೆರ-ನಾಗಿ
ತೆರ-ನಾ-ಗಿ-ರು-ತ್ತದೆ
ತೆರ-ಪಿನ
ತೆರ-ಳ-ಬೇ-ಕೆಂದು
ತೆರ-ಳಲು
ತೆರಳಿ
ತೆರ-ಳಿ-ದರು
ತೆರ-ಳಿ-ದಾಗ
ತೆರ-ಳಿದೆ
ತೆರ-ಳಿ-ದೆವು
ತೆರ-ಳಿದ್ದ
ತೆರ-ಳು-ತ್ತಿದ್ದ
ತೆರ-ಳುವ
ತೆರ-ವಾ-ಯಿತು
ತೆರೆ-ದಿ-ಡುವ
ತೆರೆದು
ತೆರೆ-ಯ-ಬ-ಹು-ದಿತ್ತು
ತೆರೆ-ಳಿದ್ದ
ತೆರೆ-ಸು-ತ್ತಾರೆ
ತೆವ-ಲಿ-ನಿಂದ
ತೇ
ತೇದು-ಕೊಂ-ಡಂತೆ
ತೇದು-ಕೊಂ-ಡಿ-ದ್ದಾರೆ
ತೇಭ್ಯೋ
ತೇರ್ಗ-ಡೆ-ಯಾ-ಗಿ-ರ-ಲಿಲ್ಲ
ತೈತ್ತಿ-ರೀಯ
ತೊಂದರೆ
ತೊಂದ-ರೆ-ಕಾರಿ
ತೊಂದ-ರೆ-ಗ-ಳನ್ನು
ತೊಂದ-ರೆ-ಗ-ಳಿಗೂ
ತೊಂದ-ರೆ-ಗ-ಳಿಗೆ
ತೊಂದ-ರೆಗೆ
ತೊಂದ-ರೆ-ಗೊ-ಳ-ಗಾದೆ
ತೊಂದ-ರೆ-ಯನ್ನು
ತೊಂದ-ರೆ-ಯಾ-ಗಿ-ದ್ದರೂ
ತೊಂದ-ರೆ-ಯಾ-ಗು-ತ್ತಿ-ರ-ಲಿಲ್ಲ
ತೊಗಟೆ
ತೊಗಲು
ತೊಡ-ಕಾ-ಗ-ದಂತೆ
ತೊಡ-ಕಾ-ಗಿ-ದ್ದಿಲ್ಲ
ತೊಡ-ಕಿ-ಲ್ಲದೆ
ತೊಡಕು
ತೊಡಗಿ
ತೊಡ-ಗಿ-ಕೊಂಡು
ತೊಡ-ಗಿ-ಕೊ-ಳ್ಳಲು
ತೊಡ-ಗಿ-ಕೊ-ಳ್ಳು-ವ-ವರ
ತೊಡ-ಗಿದ
ತೊಡ-ಗಿ-ದರು
ತೊಡ-ಗಿ-ದ-ವರು
ತೊಡ-ಗಿ-ದ್ದ-ವರು
ತೊಡ-ಗಿಸಿ
ತೊಡ-ಗಿ-ಸಿ-ಕೊಂ-ಡ-ವರು
ತೊಡ-ಗಿ-ಸಿ-ಕೊಂ-ಡಿ-ದ್ದರು
ತೊಡ-ಗಿ-ಸಿ-ಕೊಂ-ಡಿ-ದ್ದಾನೆ
ತೊಡ-ಗಿ-ಸಿ-ಕೊಂ-ಡಿದ್ದು
ತೊಡ-ಗಿ-ಸಿ-ಕೊಂಡು
ತೊಡ-ಗಿ-ಸಿ-ಕೊ-ಳ್ಳದೆ
ತೊಡ-ಗಿ-ಸಿ-ಕೊ-ಳ್ಳು-ವಂತೆ
ತೊಡ-ಗಿ-ಸಿತು
ತೊಡ-ಗಿ-ಸಿ-ದರು
ತೊಡ-ಗಿ-ಸು-ತ್ತಾನೆ
ತೊಡ-ಗು-ತ್ತಿ-ರು-ವುದು
ತೊಡ-ಗುವ
ತೊಡ-ರು-ಗಳು
ತೊಣ-ಚಿ-ಕೊ-ಪ್ಪಲು
ತೊರೆದು
ತೊರೆ-ಯಲ್ಲಿ
ತೊಳ-ಲಾಟ
ತೊಳಿತಾ
ತೊಳೆದು
ತೊಳೆ-ಯ-ಬೇ-ಕಾ-ಗಿದ್ದ
ತೊಳೆ-ಯ-ಲಾ-ರಂ-ಭಿ-ಸಿ-ದರು
ತೊಳೆ-ಯಲು
ತೊಳೆ-ಯು-ತ್ತಿ-ದ್ದರು
ತೋಚದೆ
ತೋಟ
ತೋಟದ
ತೋಟ-ದಲ್ಲಿ
ತೋಟ-ತು-ಡಿಕೆ
ತೋಡಿ-ಕೊಂಡ
ತೋಡಿ-ಕೊಂ-ಡಿ-ದ್ದೇನೆ
ತೋಡಿ-ಕೊಂಡೆ
ತೋರಲಿ
ತೋರಿ-ಕೆಯ
ತೋರಿ-ಕೊಟ್ಟ
ತೋರಿದ
ತೋರಿ-ದಿರಿ
ತೋರಿಸಿ
ತೋರಿ-ಸಿ-ಕೊ-ಟ್ಟ-ವರು
ತೋರಿ-ಸಿ-ಕೊ-ಡು-ತ್ತಿತ್ತು
ತೋರಿ-ಸಿ-ದರು
ತೋರಿ-ಸಿ-ದರೆ
ತೋರಿ-ಸಿ-ದ್ದಾರೆ
ತೋರಿ-ಸಿ-ರು-ತ್ತಾರೆ
ತೋರಿ-ಸು-ತ್ತದೆ
ತೋರಿ-ಸುವ
ತೋರಿ-ಸು-ವ-ದಾರಿ
ತೋರಿ-ಸು-ವು-ದರ
ತೋರಿ-ಸು-ವುದು
ತೋರು
ತೋರು-ತ್ತಿತ್ತು
ತೋರುವ
ತೋರ್ಪ-ಡದೇ
ತೋರ್ಪ-ಡಿ-ಸು-ತ್ತದೆ
ತ್ಯಜಿ-ಸಿದ್ದು
ತ್ಯಾಗ-ಮಾ-ಡು-ತ್ತಾನೆ
ತ್ರಾಸ-ದಾ-ಯ-ಕ-ವಾ-ಗಿಯೇ
ತ್ರಿಕ-ರ-ಣ-ಪೂ-ರ್ಣ-ವಾಗಿ
ತ್ರಿಗು-ಣ-ಗಳ
ತ್ರಿಭು-ವನ
ತ್ರಿವಿ-ಧ-ಶಿ-ಷ್ಯ-ಬು-ದ್ಧಿ-ಹಿ-ತಮ್
ತ್ರಿವೇಣಿ
ತ್ರೈಲೋಕ್ಯಂ
ದ
ದಂಡಂ
ದಂಡ-ನೆಯ
ದಂಡ-ನೆ-ಯಿ-ಲ್ಲದೆ
ದಂಡವು
ದಂಡ-ಕೋಲು
ದಂಪ-ತಿ-ಗಳ
ದಂಪ-ತಿ-ಗ-ಳಿಗೆ
ದಂಪ-ತಿ-ಗ-ಳೀ-ರ್ವ-ರನ್ನು
ದಂಪ-ತಿ-ಗಳು
ದಂಪ-ತಿ-ಗ-ಳೊಂ-ದಿಗೆ
ದಂಪತೀ
ದಂಪ-ತೀ-ಯ-ದ್ದಾ-ಗಲಿ
ದಕ್ಷತೆ
ದಕ್ಷ-ತೆ-ಯನ್ನು
ದಕ್ಷಿಣ
ದಕ್ಷಿ-ಣ-ಕ-ನ್ನಡ
ದಕ್ಷಿ-ಣೇನ
ದಟ್ಟ-ಅ-ರಿವು
ದಡ್ಡ
ದತ್ತ
ದತ್ತ-ಪೀ-ಠದ
ದತ್ತ-ಭ-ಟ್ಟ-ರಲ್ಲಿ
ದತ್ತ-ಭ-ಟ್ಟರು
ದತ್ತಾ-ತ್ರೇಯ
ದದಾತಿ
ದಧಿ-ಯಾ-ವ-ಕಮ್
ದನಿ
ದನಿ-ಗೂ-ಡಿಸಿ
ದಬ್ಬಿ
ದಯ-ಪಾಲಿ
ದಯ-ಪಾ-ಲಿ-ಸ-ಲೆಂದು
ದಯ-ಪಾ-ಲಿಸಿ
ದರ್ಭೈಸ್ತು
ದರ್ಶನ
ದರ್ಶ-ನ-ಗ-ಳನ್ನು
ದರ್ಶ-ನ-ಗಳು
ದರ್ಶ-ನ-ವನ್ನು
ದರ್ಶ-ನ-ವಾ-ಗಿತ್ತು
ದರ್ಶ-ನ-ವಾ-ದರೆ
ದರ್ಶ-ನ-ವಾ-ಯಿತು
ದರ್ಶ-ನ-ಶಾ-ಸ್ತ್ರ-ಗಳ
ದಳ-ವಾಯಿ
ದಳ್ಳೆ
ದಶ-ಕ-ಗಳ
ದಶ-ಕ-ಗಳು
ದಶ-ಗುಣಂ
ದಶ-ಗುಣಃ
ದಶ-ಗು-ಣ-ವು-ಳ್ಳದ್ದು
ದಶ-ಮಾ-ನೋ-ತ್ಸ-ವ-ದಲ್ಲಿ
ದಶ-ಲಕ್ಷ
ದಶ-ಸ-ಹ-ಸ್ರೇಷು
ದಶಾ-ವ-ತಾರ
ದಶಾ-ವ-ತಾ-ರ-ವನ್ನು
ದಹ-ತ್ಯಾಶು
ದಾಕ್ಷಿ-ಣ್ಯಕ್ಕೆ
ದಾಖ-ಲಾ-ಗಿಯೇ
ದಾಖ-ಲಾ-ತಿ-ಗ-ಳನ್ನು
ದಾಖ-ಲಾ-ದಾ-ಗಲೇ
ದಾಖ-ಲಿ-ಸ-ಲಾ-ಗಿದೆ
ದಾಖ-ಲಿ-ಸಲು
ದಾಖ-ಲಿ-ಸಿದ
ದಾಖಲೆ
ದಾಖ-ಲೆ-ಗಳ
ದಾಖ-ಲೆ-ಗ-ಳನ್ನು
ದಾಖ-ಲೆ-ಗ-ಳಿ-ಲ್ಲದ
ದಾಖ-ಲೆಯ
ದಾಖ-ಲೆ-ಯಲ್ಲಿ
ದಾಗಿ-ಸಿ-ಕೊಂ-ಡಿರಿ
ದಾಟಲು
ದಾಟಿ
ದಾಟಿದ
ದಾನ
ದಾನಂ
ದಾನಕ್ಕೆ
ದಾನ-ಗ-ಳೆ-ನಿ-ಸಿ-ಕೊಂ-ಡಿವೆ
ದಾನ-ದಲ್ಲೂ
ದಾನಾಯ
ದಾನೀ
ದಾಯಾ-ದರು
ದಾಯಾ-ದಿ-ಗ-ಳೊ-ಡನೆ
ದಾರ-ದಲ್ಲಿ
ದಾರಿ
ದಾರಿ-ಕಾ-ಣದೆ
ದಾರಿ-ದೀ-ಪ-ವಾ-ಗಿತ್ತು
ದಾರಿದ್ರ್ಯ
ದಾರಿ-ಯನ್ನು
ದಾರ್ಶ-ನಿಕ
ದಾಸ-ಶಿ-ರೋ-ಮಣಿ
ದಾಸೋಹ
ದಾಸೋ-ಹದ
ದಾಸೋ-ಹ-ದಂತೆ
ದಾಹ
ದಾಹ-ತೀ-ರುವ
ದಿ
ದಿಗಂತ
ದಿಗಿ-ಲಾ-ಯಿತು
ದಿಗಿಲು
ದಿಟ್ಟ
ದಿಟ್ಟಿಸಿ
ದಿನ
ದಿನ-ಕರಿ
ದಿನ-ಕರೀ
ದಿನ-ಕ-ರೀಯ
ದಿನ-ಕ-ರೀ-ಯ-ವನ್ನು
ದಿನಕ್ಕೆ
ದಿನ-ಕ್ಕೊಂ-ದ-ರಂತೆ
ದಿನ-ಗಳ
ದಿನ-ಗ-ಳನ್ನ
ದಿನ-ಗ-ಳನ್ನು
ದಿನ-ಗ-ಳಲ್ಲಿ
ದಿನ-ಗ-ಳ-ವ-ರೆಗೆ
ದಿನ-ಗ-ಳವು
ದಿನ-ಗ-ಳಾ-ದರೆ
ದಿನ-ಗಳು
ದಿನ-ಗಳೂ
ದಿನ-ಗ-ಳೆ-ರ-ಡ-ರಲ್ಲೂ
ದಿನದ
ದಿನ-ದಲ್ಲಿ
ದಿನ-ದಿಂದ
ದಿನ-ಪ-ತ್ರಿಕೆ
ದಿನ-ಪ-ತ್ರಿ-ಕೆ-ಗಾಗಿ
ದಿನ-ಪ-ತ್ರಿ-ಕೆಯ
ದಿನ-ಪ-ತ್ರಿ-ಕೆ-ಯಾದ
ದಿನ-ಮಾ-ತ್ರ-ವಿತ್ತು
ದಿನ-ಮಾ-ನ-ದಲ್ಲಿ
ದಿನವೂ
ದಿನವೇ
ದಿನಾಂಕ
ದಿನಾಂ-ಕ-ದಂದೇ
ದಿನಾಲೂ
ದಿನೇ
ದಿವಂ-ಗತ
ದಿವಂ-ಗ-ತ-ರಾ-ದ-ವರು
ದಿವ-ಸ-ಕ್ಕೊಮ್ಮೆ
ದಿವ್ಯ
ದಿವ್ಯ-ದೀ-ಪ್ತಿ-ಗೆಂತ
ದಿವ್ಯ-ಸ್ವ-ರೂ-ಪದ
ದಿವ್ಯಾ-ನು-ಗ್ರ-ಹ-ವಿ-ರುವ
ದಿಶೆ-ಯಲ್ಲಿ
ದೀಕ್ಷಿ-ತರು
ದೀಪವೇ
ದೀಪಿಕಾ
ದೀರ್ಘ
ದೀರ್ಘ-ಕಾಲ
ದೀರ್ಘ-ವಾದ
ದೀವ-ಟಿ-ಗೆಗೆ
ದೀವ್ಯಾತ್ತು
ದುಖ-ದಿಂದ
ದುಃಖ
ದುಗು-ಡ-ವೆಲ್ಲ
ದುಗ್ಧಾನ್ನಂ
ದುಡಿ-ದಿ-ದ್ದಾರೆ
ದುಡಿಮೆ
ದುಡಿ-ಯು-ತ್ತಿದ್ದ
ದುಡಿ-ಯು-ತ್ತಿ-ದ್ದಾರೆ
ದುಡಿ-ಸಿ-ಕೊ-ಳ್ಳು-ವುದು
ದುಡ್ಡನ್ನು
ದುಡ್ಡು
ದುರಂತ
ದುರ-ಭಿ-ಮಾನ
ದುರ-ಭಿ-ಮಾ-ನ-ಗ-ಳಿಂದ
ದುರ-ಭಿ-ಮಾ-ನ-ದಿಂದ
ದುರ-ಸ್ತಿಯೇ
ದುರಾ-ದೃಷ್ಟ
ದುರು-ದ್ದೇ-ಶ-ವಿ-ಲ್ಲದೆ
ದುರ್ಲಭ
ದುರ್ಲಭಃ
ದುರ್ಲ-ಭಮ್
ದುರ್ಲ-ಭವೇ
ದುರ್ಲ-ಭವೋ
ದುರ್ಲ-ಭಾ-ಸ್ಸನ್ತಿ
ದುರ್ಲಭೋ
ದುರ್ಲ-ಭೋಯಂ
ದುರ್ವಿಧಿ
ದುಷ್ಟ
ದುಷ್ಟ-ಶಕ್ತಿ
ದೂರ
ದೂರ-ದಲ್ಲಿ
ದೂರ-ದ-ಲ್ಲಿದ್ದ
ದೂರ-ದ-ಲ್ಲಿ-ರುವ
ದೂರ-ದ-ಲ್ಲಿ-ರು-ವ-ವ-ರಿಗೂ
ದೂರ-ದಿಂದ
ದೂರ-ದಿಂ-ದಲೇ
ದೂರ-ದೃಷ್ಟಿ
ದೂರ-ದೃ-ಷ್ಟಿಯ
ದೂರ-ನಾ-ಗಿದ್ದೆ
ದೂರ-ಮಾ-ಡುವ
ದೂರ-ವಾ-ಗಿತ್ತು
ದೂರ-ವಾ-ಗಿ-ದ್ದೀವಿ
ದೂರ-ವಾ-ಗಿ-ರು-ತ್ತಾರೆ
ದೂರ-ವಾ-ಗಿ-ಸಿ-ಕೊಂಡು
ದೂರ-ವಾಣಿ
ದೂರ-ವಾ-ಣಿಯ
ದೂರ-ವಾ-ದ-ದ್ದಾ-ಗ-ಲಾ-ರದು
ದೂರ-ವಿಟ್ಟು
ದೂರ-ವಿ-ಡು-ತ್ತಾನೆ
ದೂರ-ವಿದ್ದ
ದೂರ-ಶಿ-ಕ್ಷಣ
ದೂರಾ-ಗು-ವುದೋ
ದೂರಾದ
ದೂರಿ
ದೂರಿ-ಡ-ಬೇಕು
ದೂರು
ದೂಷ-ಣ-ವಲ್ಲ
ದೃಢ-ಪ-ಡಿ-ಸು-ತ್ತದೆ
ದೃಢ-ಬೆ-ಸುಗೆ
ದೃಢ-ವಾಗಿ
ದೃಢ-ವಾ-ಗುತ್ತಾ
ದೃಢ-ವಾದ
ದೃಢ-ಸಂ-ಕಲ್ಪ
ದೃಶ್ಯ-ಗಳು
ದೃಶ್ಯತೇ
ದೃಷ್ಟಾಂ-ತ-ಗ-ಳಿವೆ
ದೃಷ್ಟಾಂ-ತ-ವಾ-ಗಿ-ದ್ದಾರೆ
ದೃಷ್ಟಾಂ-ತೋ-ಪ-ಯೋಗ
ದೃಷ್ಟಿ
ದೃಷ್ಟಿ-ಕೋ-ಣ-ದಿಂದ
ದೃಷ್ಟಿ-ಕೋನ
ದೃಷ್ಟಿ-ಯಲ್ಲಿ
ದೃಷ್ಟಿ-ಯಿಂದ
ದೃಷ್ಟಿ-ಯಿಂ-ದಲೇ
ದೃಷ್ಟಿ-ಯಿಂ-ದಲೋ
ದೃಷ್ಟಿ-ಯು-ಳ್ಳ-ವ-ರಾ-ಗಿ-ದ್ದರು
ದೃಷ್ಟಿಯೇ
ದೃಷ್ಠಿ-ಕೋ-ನ-ದಿಂದ
ದೆಸೆ-ಯಿಂದ
ದೆಸೆ-ಯಿಂ-ದಲೇ
ದೆಹಲಿ
ದೇ
ದೇವ
ದೇವ-ಗಂಗೆ
ದೇವ-ಣ್ಣ-ರಸ್ತೆ
ದೇವತಾ
ದೇವ-ತೆ-ಗ-ಳಿಗೆ
ದೇವ-ತೆಯೇ
ದೇವ-ಪೂಜಾ
ದೇವರ
ದೇವ-ರಲ್ಲಿ
ದೇವರು
ದೇವ-ವಾಣಿ
ದೇವ-ಸ್ಥಾನ
ದೇವ-ಸ್ಥಾ-ನದ
ದೇವ-ಹಿತಂ
ದೇವಾ-ಲಯ
ದೇವಾ-ಲ-ಯಕ್ಕೆ
ದೇವಾ-ಲ-ಯಕ್ಕೇ
ದೇವಾ-ಲ-ಯದ
ದೇವಾ-ಲ-ಯ-ವನ್ನು
ದೇವಾ-ಲ-ಯವು
ದೇವಾ-ಲಯೇ
ದೇವಿ-ಯ-ವರ
ದೇವಿ-ಸರ
ದೇವೇಶಿ
ದೇವೋ-ಭವ
ದೇಶ
ದೇಶ-ಗ-ಳಿಂದ
ದೇಶದ
ದೇಶ-ಪ್ರ-ಜ್ಞೆ-ಯಿ-ಲ್ಲದೇ
ದೇಶ-ವಿ-ದೇ-ಶ-ಗ-ಳಿಂದ
ದೇಶ-ವಿ-ದೇ-ಶ-ಗ-ಳಿಗೂ
ದೇಶೀ-ಕೇಂದ್ರ
ದೇಶೀ-ಯರು
ದೇಶ-ವಿ-ದೇ-ಶ-ಗಳ
ದೇಹ
ದೇಹದ
ದೇಹ-ದಿಂದ
ದೇಹ-ಮಾ-ಶ್ರಿತಃ
ದೈನಂ-ದಿನ
ದೈವ
ದೈವ-ಕಾ-ರ-ಣ-ದಿಂ-ದಲೋ
ದೈವ-ನಿ-ರ್ಮಿತ
ದೈವ-ಮ-ಭಿ-ವ್ಯಕ್ತಂ
ದೈವ-ಮಾ-ತ್ರ-ದಿಂದ
ದೈವ-ವಿ-ದ್ದಲ್ಲಿ
ದೈವವು
ದೈವ-ವೆಂ-ದರೆ
ದೈವವೇ
ದೈವ-ಶಾಸ್ತ್ರ
ದೈವೇ
ದೈವೇ-ಚ್ಛೆ-ಯಿಂದ
ದೊಡ್ಡ
ದೊಡ್ಡ-ದಲ್ಲ
ದೊಡ್ಡ-ದಾಗಿ
ದೊಡ್ಡ-ದಾ-ದಷ್ಟೂ
ದೊಡ್ಡ-ದಿದೆ
ದೊಡ್ಡದು
ದೊಡ್ಡಪ್ಪ
ದೊಡ್ಡ-ಪ್ಪನ
ದೊಡ್ಡ-ಪ್ಪ-ನ-ಮಗ
ದೊಡ್ಡ-ಪ್ಪ-ನ-ವರು
ದೊಡ್ಡಮ್ಮ
ದೊಡ್ಡ-ವರು
ದೊಡ್ಡ-ಸಂ-ಸ್ಕಾರ
ದೊಡ್ದ-ಪ್ಪನ
ದೊರ-ಕಿತು
ದೊರ-ಕಿ-ದವು
ದೊರ-ಕಿ-ದ್ದ-ರಿಂದ
ದೊರ-ಕಿ-ರು-ವು-ದ-ರಿಂದ
ದೊರ-ಕಿ-ಸಿ-ಕೊಂ-ಡಿ-ದ್ದ-ರಲ್ಲಿ
ದೊರ-ಕಿ-ಸಿ-ಕೊ-ಟ್ಟಿ-ದ್ದರು
ದೊರ-ಕು-ತ್ತದೆ
ದೊರ-ಕು-ತ್ತಲೇ
ದೊರ-ಕು-ತ್ತಿ-ದ್ದ-ವಲ್ಲ
ದೊರ-ಕು-ವು-ದೆಂರೆ
ದೊರೆತ
ದೊರೆ-ತಂ-ತಾ-ಗು-ತ್ತದೆ
ದೊರೆ-ತ-ದ್ದ-ರಿಂದ
ದೊರೆ-ತದ್ದು
ದೊರೆ-ತಿತ್ತು
ದೊರೆ-ತಿದ್ದು
ದೊರೆತು
ದೊರೆ-ಯ-ದಿ-ದ್ದಾಗ
ದೊರೆ-ಯದು
ದೊರೆ-ಯ-ಲಿಲ್ಲ
ದೊರೆ-ಯಿತು
ದೊರೆ-ಯು-ತ್ತದೆ
ದೊರೆ-ಯು-ತ್ತ-ದೆಯೇ
ದೊರೆ-ಯು-ತ್ತಿತ್ತು
ದೊರೆ-ಯು-ತ್ತಿ-ರ-ಲಿಲ್ಲ
ದೊರೆ-ಯು-ತ್ತಿ-ರುವ
ದೊರೆ-ಯು-ವಂ-ತ-ಹ-ದ್ದಲ್ಲ
ದೋಷ
ದೋಷ-ಗ-ಳನ್ನು
ದೋಷ-ಗ-ಳಿ-ರ-ಲೇ-ಬೇಕು
ದೋಷ-ಗಳು
ದೋಷ-ದೌ-ರ್ಬ-ಲ್ಯ-ಗಳು
ದೋಷ-ವಾಗಿ
ದೌರ್ಬ-ಲ್ಯ-ದಿಂದ
ದೌರ್ಭಾಗ್ಯ
ದ್ಯೋತ-ಕ-ವಾದ
ದ್ರವ್ಯ
ದ್ರವ್ಯ-ಯಜ್ಞ
ದ್ರವ್ಯ-ಶುದ್ಧಿ
ದ್ರುತಂ
ದ್ವನಿ-ಪೆ-ಟ್ಟಿಗೆ
ದ್ವಾಪರೇ
ದ್ವಿಗು-ಣ-ಗೊ-ಳಿಸು
ದ್ವಿಜಾ-ತಯಃ
ದ್ವಿತೀಯ
ದ್ವೇಷ
ದ್ವೇಷ-ಗಳು
ಧಕ್ಕೆ
ಧನ
ಧನಂ
ಧನದ
ಧನ-ಸ-ಹಾ-ಯ-ಮಾ-ಡು-ತ್ತೇನೆ
ಧನಾ-ತ್ಮಕ
ಧನಾ-ರ್ಜ-ನೆ-ಯಲ್ಲಿ
ಧನ್ಯ
ಧನ್ಯತಾ
ಧನ್ಯ-ತೆ-ಗ-ಳನ್ನು
ಧನ್ಯ-ತೆ-ಯ-ನ್ನೊ-ದ-ಗಿ-ಸಿದ
ಧನ್ಯ-ತೆಯೂ
ಧನ್ಯರು
ಧನ್ಯ-ವಾಗಿ
ಧನ್ಯ-ವಾ-ಗಿದೆ
ಧನ್ಯ-ವಾ-ದ-ಗ-ಳನ್ನು
ಧನ್ಯಾ
ಧನ್ಯಾ-ತ್ಮ-ರು-ಪು-ಣ್ಯಾ-ತ್ಮರೂ
ಧನ್ವಂ-ತರಿ
ಧನ್ವಂ-ತ-ರಿಯು
ಧಯೂ-ಳಿ-ಗಕೆ
ಧರ-ಣಿ-ಯೊ-ಡಲು
ಧರಿಸಿ
ಧರಿ-ಸಿ-ಕೊಂಡ
ಧರಿ-ಸಿದ
ಧರಿ-ಸಿ-ದ-ವರು
ಧರಿ-ಸಿದ್ದು
ಧರ್ಮ
ಧರ್ಮ-ಗಳ
ಧರ್ಮ-ಗಳೂ
ಧರ್ಮ-ಜ್ಞರು
ಧರ್ಮ-ಜ್ಞಾ-ನ-ವನ್ನು
ಧರ್ಮದ
ಧರ್ಮ-ದೊ-ಲವು
ಧರ್ಮ-ಪತ್ನಿ
ಧರ್ಮ-ಪ-ತ್ನಿ-ಅ-ಕ್ಕಯ್ಯ
ಧರ್ಮ-ಪ-ತ್ನಿಯ
ಧರ್ಮ-ಪ-ತ್ನಿಯೂ
ಧರ್ಮ-ಪಾ-ಲನೆ
ಧರ್ಮ-ಪಾ-ಲ-ನೆ-ಯಲ್ಲಿ
ಧರ್ಮ-ವನ್ನು
ಧರ್ಮ-ಶಾಸ್ತ್ರ
ಧರ್ಮ-ಶಾ-ಸ್ತ್ರಕ್ಕೆ
ಧರ್ಮ-ಶಾ-ಸ್ತ್ರ-ಗ-ಳಲ್ಲಿ
ಧರ್ಮ-ಶಾ-ಸ್ತ್ರ-ಗಳು
ಧರ್ಮ-ಶಾ-ಸ್ತ್ರದ
ಧರ್ಮ-ಶಾ-ಸ್ತ್ರ-ದಲ್ಲಿ
ಧರ್ಮ-ಶಾ-ಸ್ತ್ರ-ವಿ-ಭಾಗ
ಧರ್ಮ-ಶಾ-ಸ್ತ್ರ-ವೆಂ-ದರೆ
ಧರ್ಮ-ಶಾ-ಸ್ತ್ರ-ವೆಂದು
ಧರ್ಮ-ಶಾ-ಸ್ತ್ರಾ-ದಿ-ಗ-ಳಲ್ಲಿ
ಧರ್ಮ-ಸ್ಥಳ
ಧರ್ಮ-ಸ್ಥ-ಳದ
ಧರ್ಮಾ-ತೀ-ತ-ವಾ-ದದ್ದು
ಧರ್ಮಾ-ಧಿ-ಕಾರಿ
ಧರ್ಮಾ-ನು-ಷ್ಠಾನ
ಧರ್ಮಾ-ನು-ಷ್ಠಾ-ನ-ಗ-ಳಿಗೆ
ಧರ್ಮಾ-ನು-ಷ್ಠಾ-ನ-ದಿಂದ
ಧರ್ಮಾ-ನು-ಷ್ಠಾ-ನವೇ
ಧರ್ಮಾ-ನು-ಸಾ-ರ-ವಾಗಿ
ಧರ್ಮಾ-ರ್ಥ-ಮೋ-ಕ್ಷೋ-ಪ-ದೇಷ್ಟಾ
ಧರ್ಮೋ-ತ್ಥಾನ
ಧಾಂಗುಡಿ
ಧಾತು-ವನ್ನೂ
ಧಾತು-ವಿ-ನಿಂದ
ಧಾತು-ವಿ-ನಿಂ-ದಲೂ
ಧಾನ
ಧಾನ್ಯೈರ್ನ
ಧಾರಣ
ಧಾರ-ಣ-ಶಕ್ತಿ
ಧಾರ-ಣಾ-ಶ-ಕ್ತಿ-ಯಿಂದ
ಧಾರಾಳ
ಧಾರೆ
ಧಾರೆಯ
ಧಾರೆ-ಯೆ-ರೆದ
ಧಾರೆ-ಯೆ-ರೆ-ದರು
ಧಾರೆ-ಯೆ-ರೆ-ಯು-ವಂ-ತಾ-ಗಲಿ
ಧಾರೇ-ಶ್ವರ
ಧಾರ್ಮಿಕ
ಧಾರ್ಮಿ-ಕ-ತೆಗೆ
ಧಾರ್ಮಿ-ಕ-ತೆಯ
ಧಾರ್ಮಿ-ಕ-ತೆ-ಯನ್ನೋ
ಧಾವಿಸಿ
ಧೀ
ಧೀಖ-ನನ
ಧೀಮ-ತಾಮ್
ಧೀಮಾನ್
ಧೀರ
ಧೀರ-ಪ್ರ-ತಿಮೆ
ಧೀರಾಃ
ಧೀಶಕ್ತಿ
ಧುತ್ತನೆ
ಧುಮು-ಕಿ-ಹುದು
ಧುರಿ
ಧೂಳಿ-ನಿಂದ
ಧೃಡ-ವಾಗಿ
ಧೃಢತೆ
ಧೃತಿ
ಧೃತಿ-ಗೆ-ಡದೇ
ಧೃತಿ-ಯನ್ನು
ಧೃತಿ-ವಿ-ಭ್ರಂಶ
ಧೈರ್ಯ
ಧೈರ್ಯ-ವನ್ನು
ಧೈರ್ಯವೇ
ಧೈರ್ಯ-ಸ-ಮ-ಸ್ಯೆ-ಯನ್ನು
ಧ್ಯಾನ-ಜಯ
ಧ್ಯಾನಿಸಿ
ಧ್ಯಾನಿ-ಸು-ವುದು
ಧ್ವನಿ
ಧ್ವನಿ-ಯನ್ನು
ನಂಜ-ನ-ಗೂ-ಡಿಗೆ
ನಂಜ-ನ-ಗೂ-ಡಿನ
ನಂಜ-ನ-ಗೂಡು
ನಂಜ-ನೂ-ಗುಡು
ನಂಜುಂ-ಡೇ-ಗೌಡ್ರು
ನಂಟಿದೆ
ನಂಟಿ-ರಿ-ವುದು
ನಂಟು
ನಂಟೇ
ನಂತರ
ನಂತ-ರದ
ನಂತೆ
ನಂದತಿ
ನಂದ-ತ್ಯೇವ
ನಂದಾ-ದೀ-ಪ-ವಾ-ಗಿ-ರ-ಲೆಂದು
ನಂಬ-ದಾದೆ
ನಂಬ-ರಿನ
ನಂಬರ್
ನಂಬ-ಲಾ-ಗದ
ನಂಬ-ಲಾ-ರರು
ನಂಬಲೇ
ನಂಬಿ
ನಂಬಿ-ಕೆಗೆ
ನಂಬಿ-ದ-ವ-ರಲ್ಲ
ನಂಬಿ-ದ-ವರು
ನಂಬಿದ್ದ
ನಂಬು-ತ್ತೇನೆ
ನಂಬು-ವ-ವರು
ನಃ
ನಕಾ-ರಕ್ಕೆ
ನಕಾ-ರ-ಭಾ-ವ-ವನ್ನೂ
ನಕಾ-ರಾ-ತ್ಮ-ಕ-ವಾಗಿ
ನಕ್ಕರು
ನಕ್ಕು
ನಗಣ್ಯ
ನಗ-ರಕ್ಕೆ
ನಗ-ರದ
ನಗ-ರ-ವನ್ನು
ನಗ-ರ-ವಾ-ಸಿ-ಗ-ಳಾ-ದರೂ
ನಗು
ನಗುತ್ತ
ನಗು-ಬ-ರು-ತ್ತಿದೆ
ನಗು-ಮೊ-ಗದ
ನಗು-ವುದು
ನಟಿಸಿ
ನಡ-ವ-ಳಿ-ಕೆ-ಗಳ
ನಡಿ-ಸು-ವು-ದರ
ನಡು-ವಿನ
ನಡುವೆ
ನಡು-ವೆಯೂ
ನಡೆ
ನಡೆ-ಗೆಂತೋ
ನಡೆದ
ನಡೆ-ದದ್ದು
ನಡೆ-ದರೆ
ನಡೆ-ದ-ವ-ರಲ್ಲಿ
ನಡೆ-ದಾ-ಡು-ತ್ತಾರೋ
ನಡೆ-ದಿತ್ತು
ನಡೆ-ದಿದೆ
ನಡೆ-ದಿದ್ದ
ನಡೆ-ದಿದ್ದು
ನಡೆ-ದಿ-ರು-ವುದು
ನಡೆ-ದಿ-ಲ್ಲ-ವೆಂಬ
ನಡೆ-ದಿವೆ
ನಡೆ-ದು-ಕೊಂ-ಡ-ವ-ರಲ್ಲ
ನಡೆ-ದು-ಕೊಂ-ಡಿರಿ
ನಡೆ-ದು-ಕೊಂಡು
ನಡೆ-ದು-ಬಂದ
ನಡೆದೇ
ನಡೆ-ನು-ಡಿ-ಗಳು
ನಡೆ-ಯ-ದಂತೆ
ನಡೆ-ಯ-ದಿ-ದ್ದರೂ
ನಡೆ-ಯ-ಬೇ-ಕಾದ
ನಡೆ-ಯ-ಬೇ-ಕೆಂಬ
ನಡೆ-ಯಲಿ
ನಡೆ-ಯಲ್ಲಿ
ನಡೆ-ಯಿತು
ನಡೆ-ಯು-ತ್ತದೆ
ನಡೆ-ಯು-ತ್ತಲೇ
ನಡೆ-ಯು-ತ್ತಿ-ತ್ತಂತೆ
ನಡೆ-ಯು-ತ್ತಿತ್ತು
ನಡೆ-ಯು-ತ್ತಿ-ತ್ತು-ಅ-ದ-ರಂತೆ
ನಡೆ-ಯು-ತ್ತಿದ್ದ
ನಡೆ-ಯು-ತ್ತಿ-ದ್ದವು
ನಡೆ-ಯು-ತ್ತಿ-ದ್ದಿ-ದ್ದನ್ನು
ನಡೆ-ಯು-ತ್ತಿದ್ದು
ನಡೆ-ಯು-ತ್ತಿ-ರ-ಲಿಲ್ಲ
ನಡೆ-ಯು-ತ್ತಿ-ರು-ವುದು
ನಡೆ-ಯುವ
ನಡೆ-ಯು-ವಲ್ಲಿ
ನಡೆ-ವ-ಳಿ-ಕೆ-ಗಿಂತ
ನಡೆ-ಸ-ತೊ-ಡ-ಗಿ-ದರು
ನಡೆ-ಸ-ಬ-ಹು-ದಾ-ಗಿದೆ
ನಡೆ-ಸ-ಬ-ಹು-ದೆಂ-ಬುದು
ನಡೆ-ಸ-ಲಾ-ಗಿತ್ತು
ನಡೆ-ಸಲು
ನಡೆಸಿ
ನಡೆ-ಸಿ-ಕೊಂಡು
ನಡೆ-ಸಿ-ಕೊಟ್ಟ
ನಡೆ-ಸಿ-ಕೊ-ಡುತ್ತ
ನಡೆ-ಸಿದ
ನಡೆ-ಸಿ-ದ-ವರೂ
ನಡೆ-ಸಿ-ದೆವು
ನಡೆ-ಸಿ-ದ್ದಾನೆ
ನಡೆ-ಸಿ-ದ್ದಾರೆ
ನಡೆ-ಸಿ-ದ್ದುದು
ನಡೆ-ಸಿ-ರು-ವುದೇ
ನಡೆ-ಸುತ್ತ
ನಡೆ-ಸುತ್ತಾ
ನಡೆ-ಸು-ತ್ತಿದೆ
ನಡೆ-ಸು-ತ್ತಿದ್ದ
ನಡೆ-ಸು-ತ್ತಿ-ದ್ದರು
ನಡೆ-ಸು-ತ್ತಿ-ದ್ದ-ರೇನೋ
ನಡೆ-ಸು-ತ್ತಿ-ದ್ದಾರೆ
ನಡೆ-ಸು-ತ್ತಿ-ರು-ವುದು
ನಡೆ-ಸುವ
ನಡೆ-ಸು-ವಂಥ
ನಡೆ-ಸು-ವರು
ನಡೆ-ಸು-ವು-ದ-ರಿಂದ
ನತ
ನದಿ-ಗ-ಳನ್ನು
ನದಿಯ
ನದ್ಯಾಂ
ನನ-ಗಂತೂ
ನನ-ಗ-ನಿ-ಸು-ತ್ತಿದೆ
ನನ-ಗಾದ
ನನ-ಗಾ-ಯಿತು
ನನ-ಗಿಂತ
ನನ-ಗಿಂ-ತಲೂ
ನನ-ಗಿದೆ
ನನ-ಗಿನ್ನೂ
ನನ-ಗುಂ-ಟಾದ
ನನಗೂ
ನನಗೆ
ನನಗೇ
ನನ-ಗೊಂದು
ನನಗೋ
ನನ-ಪಿವೆ
ನನಲ್ಲಿ
ನನ-ಸಾ-ದದ್ದು
ನನ್ನ
ನನ್ನಂ-ತಹ
ನನ್ನಂ-ತ-ಹ-ವರು
ನನ್ನಂತೆ
ನನ್ನಂ-ತೆಯೇ
ನನ್ನಣ್ಣ
ನನ್ನತ್ತ
ನನ್ನ-ದಲ್ಲ
ನನ್ನ-ದಾ-ಗಿತ್ತು
ನನ್ನದು
ನನ್ನ-ದೊಂದು
ನನ್ನನ್ನು
ನನ್ನನ್ನೇ
ನನ್ನಲ್ಲಿ
ನನ್ನ-ಲ್ಲಿಗೆ
ನನ್ನ-ಲ್ಲಿನ
ನನ್ನಿಂದ
ನನ್ನಿ-ಷ್ಟ-ದಂತೆ
ನನ್ನೀ-ಕ-ವನ
ನನ್ನೆ-ರಡು
ನನ್ನೊಂ-ದಿಗೆ
ನನ್ನೊಂ-ದಿಗೇ
ನನ್ನೊ-ಡನೆ
ನನ್ನೊ-ಬ್ಬನ
ನನ್ನೊ-ಬ್ಬ-ನನ್ನು
ನಮಃ
ನಮ-ಗ-ನಾದ
ನಮ-ಗ-ರಿ-ಯ-ದಂತೆ
ನಮ-ಗ-ರಿ-ವಿ-ಲ್ಲ-ದೇನೇ
ನಮ-ಗಷ್ಟೆ
ನಮ-ಗಾಗಿ
ನಮ-ಗಾ-ದರೋ
ನಮ-ಗಾ-ಯಿತು
ನಮ-ಗಿ-ಬ್ಬ-ರಿಗೂ
ನಮಗೆ
ನಮ-ಗೆಲ್ಲ
ನಮ-ಗೆ-ಲ್ಲ-ರಿಗೂ
ನಮ-ಗೆಲ್ಲಾ
ನಮಗೇ
ನಮ-ಗೊಂದು
ನಮ-ಗೋ-ಸ್ಕರ
ನಮ-ನ-ವನ್ನು
ನಮ-ಸ್ಕ-ರಿ-ಸುತ್ತಾ
ನಮ-ಸ್ಕಾರ
ನಮ-ಸ್ಕಾ-ರ-ಗ-ಳೊಂ-ದಿಗೆ
ನಮಿಪೆ
ನಮಿಸಿ
ನಮಿ-ಸು-ತ್ತೇನೆ
ನಮೂ-ದಿಸಿ
ನಮೂ-ದಿ-ಸಿ-ದ್ದಾರೆ
ನಮೂ-ನೆ-ಯನ್ನು
ನಮೆ-ಲ್ಲರ
ನಮೋ
ನಮ್ಮ
ನಮ್ಮಂಥ
ನಮ್ಮಣ್ಣ
ನಮ್ಮ-ದಾ-ಯಿತು
ನಮ್ಮದು
ನಮ್ಮ-ನೆಗೆ
ನಮ್ಮ-ನೆಗೇ
ನಮ್ಮನ್ನ
ನಮ್ಮನ್ನು
ನಮ್ಮನ್ನೂ
ನಮ್ಮ-ನ್ನೆಲ್ಲ
ನಮ್ಮ-ನ್ನೆಲ್ಲಾ
ನಮ್ಮ-ಭ-ಟ್ಟರು
ನಮ್ಮ-ಲ್ಲರ
ನಮ್ಮಲ್ಲಿ
ನಮ್ಮ-ಲ್ಲಿನ
ನಮ್ಮ-ಲ್ಲಿ-ಯ-ಉ-ತ್ತ-ರ-ಕ-ನ್ನ-ಡದ
ನಮ್ಮ-ಲ್ಲೆ-ಲ್ಲ-ರಲ್ಲೂ
ನಮ್ಮ-ವರ
ನಮ್ಮ-ವರೇ
ನಮ್ಮ-ಹೆಮ್ಮೆ
ನಮ್ಮಿಂದ
ನಮ್ಮಿ-ಬ್ಬರ
ನಮ್ಮಿ-ಬ್ಬ-ರನ್ನು
ನಮ್ಮಿ-ಬ್ಬ-ರಲ್ಲಿ
ನಮ್ಮಿ-ಬ್ಬ-ರ-ಲ್ಲಿ-ರು-ವುದು
ನಮ್ಮಿ-ಬ್ಬ-ರೊ-ಳಗೆ
ನಮ್ಮೀ-ರ್ವರ
ನಮ್ಮೆ-ರಡು
ನಮ್ಮೆ-ಲರ
ನಮ್ಮೆಲ್ಲ
ನಮ್ಮೆ-ಲ್ಲರ
ನಮ್ಮೆ-ಲ್ಲ-ರಿಗೂ
ನಮ್ಮೆ-ಲ್ಲ-ರೊಂ-ದಿಗೆ
ನಮ್ಮೊಂ-ದಿ-ಗಿ-ದ್ದಾರೆ
ನಮ್ಮೊಂ-ದಿಗೆ
ನಮ್ಮೊಂ-ದಿಗೇ
ನಮ್ಮೊ-ಡನೆ
ನಮ್ಮೊ-ಡ-ನೆಯೇ
ನಮ್ಮೊ-ಳ-ಗಿದೆ
ನಯ
ನಯ-ವಾಗಿ
ನರ
ನರ-ಕೋ-ಶ-ಗಳ
ನರ-ತಂ-ತು-ಗಳ
ನರ-ತಜ್ಞ
ನರ-ಮಂ-ಡ-ಲದ
ನರ-ಸಖ
ನರ-ಸಿಂಹ
ನರ-ಸಿಂ-ಹನ್
ನಲ-ವತ್ತು
ನಲಿ-ದಾ-ಡು-ತ್ತ-ಲಿದೆ
ನಲಿ-ಯಲಿ
ನಲಿ-ವು-ಗ-ಳನ್ನು
ನಲು-ಮೊ-ಗವ
ನಲ್ಮೆಯ
ನಲ್ಲಿ
ನಳ-ನ-ಳಿ-ಸು-ತಿಹ
ನವ-ಗ್ರಹ
ನವ-ಚೈ-ತ-ನ್ಯ-ವನ್ನು
ನವ-ದಂ-ಪ-ತಿ-ಗಳು
ನವ-ನ-ವೋ-ನ್ಮೇ-ಷ-ಶಾ-ಲಿನೀ
ನವ-ಮಾ-ನ-ವನ
ನವ-ಮಾ-ನ-ವ-ನನ್ನು
ನವ-ಯು-ಗದಿ
ನವ-ರ-ಸ-ಗಳೂ
ನವ-ರ-ಸ-ಭ-ರಿ-ತ-ವಾದ
ನವ-ರಾತ್ರಿ
ನವಿ-ರು-ಹಾಸ್ಯ
ನವಿ-ಲು-ಗೆ-ರೆಯ
ನವೀನ
ನವೀ-ನ-ನ್ಯಾಯ
ನವೀ-ನ-ನ್ಯಾ-ಯ-ವಿ-ದ್ವ-ದು-ತ್ತ-ಮಾ-ಪ-ರೀ-ಕ್ಷೆ-ಯ-ಲ್ಲಿಯೂ
ನವೀ-ನ-ನ್ಯಾ-ಯ-ಶಾಸ್ತ್ರ
ನವೀ-ನ-ನ್ಯಾ-ಯ-ಶಾ-ಸ್ತ್ರ-ಗ್ರಂ-ಥ-ಗ-ಳನ್ನು
ನವೀ-ನ-ನ್ಯಾ-ಯ-ಶಾ-ಸ್ತ್ರದ
ನವೆಂ-ಬರ್
ನವೋ-ದ-ಯ-ವಿ-ದ್ಯಾ-ಲಯ
ನವೋಸ್ತು
ನವ್ಯ-ತೆ-ಯನ್ನು
ನವ್ಯ-ನ್ಯಾಯ
ನವ್ಯ-ನ್ಯಾ-ಯದ
ನಷ್ಟ
ನಷ್ಟ-ಪ್ರಾ-ಯ-ವಾ-ಗು-ತ್ತಿವೆ
ನಹೀ
ನಾ
ನಾಂದಿ-ಯಾ-ಗಿದೆ
ನಾಂದಿ-ಯಾ-ಯಿತು
ನಾಕ್ಷ-ತೈ-ರ್ಹ-ಸ್ತ-ಪ-ರ್ವೈರ್ವಾ
ನಾಗ-ರಾಜ
ನಾಗ-ರೀ-ಕ-ರಿಂದ
ನಾಚಿ
ನಾಚಿ-ಕೆ-ಯಾ-ಗು-ತ್ತದೆ
ನಾಟಕ
ನಾಟ-ಕದ
ನಾಟ-ಕ-ದಲ್ಲಿ
ನಾಟ-ಕ-ವನ್ನು
ನಾಟಿ
ನಾಟು-ವಂತೆ
ನಾಡಿ
ನಾಡಿಗೆ
ನಾಡಿನ
ನಾಡಿ-ನಲ್ಲಿ
ನಾಣ್ಣುಡಿ
ನಾಥ-ರ-ನ್ನಾ-ಗಿ-ಸಿ-ದ-ವರು
ನಾನಾ
ನಾನಾಗ
ನಾನಾ-ಮೂ-ಲೆ-ಯಿಂದ
ನಾನಿದ್ದ
ನಾನಿ-ದ್ದೇನೆ
ನಾನಿಲ್ಲಿ
ನಾನು
ನಾನೂ
ನಾನೆಂದೂ
ನಾನೇ
ನಾನೇನು
ನಾನೊಂದು
ನಾನೊಬ್ಬ
ನಾನೊ-ಬ್ಬಳೆ
ನಾಮ
ನಾಮ-ಕ-ರಣ
ನಾಮದ
ನಾಮ-ಧೇಯ
ನಾಮ-ಸಾ-ರ್ಥಕ್ಯ
ನಾಯಕ
ನಾಯ-ಕತ್ವ
ನಾಯ-ಕ-ತ್ವ-ಗುಣ
ನಾರಾ-ಯಣ
ನಾರಾ-ಯ-ಣ-ಭ-ಟ್ಟ-ರಲ್ಲಿ
ನಾರ್ಹಂತಿ
ನಾಲಿ-ಗಾರ
ನಾಲಿ-ಗಾರು
ನಾಲ್ಕನೆ
ನಾಲ್ಕ-ನೆಯ
ನಾಲ್ಕ-ರಿಂದ
ನಾಲ್ಕಾ-ಗಿಸು
ನಾಲ್ಕಾರು
ನಾಲ್ಕು
ನಾಲ್ಕು-ಬಾರಿ
ನಾಲ್ಕೇ
ನಾಲ್ಕೈದು
ನಾಲ್ವ-ರಲಿ
ನಾಲ್ವರು
ನಾಲ್ವರೂ
ನಾಳೆ
ನಾಳೆ-ಯಿಂ-ದಲೇ
ನಾಳೆಯೇ
ನಾವಂತೂ
ನಾವ-ಲ್ಲದೇ
ನಾವಾ-ಡುವ
ನಾವಿ-ಬ್ಬರು
ನಾವಿ-ಬ್ಬರೂ
ನಾವು
ನಾವು-ಗಳು
ನಾವೂ
ನಾವೆಲ್ಲ
ನಾವೆ-ಲ್ಲರೂ
ನಾವೆಲ್ಲಾ
ನಾವೇ
ನಾಶ-ಪ-ಡಿ-ಸುವ
ನಾಶ-ವಾ-ಗದು
ನಾಸ್ತಿ
ನಿಂಗೆ
ನಿಂತ
ನಿಂತರು
ನಿಂತ-ರೆಂ-ದರೆ
ನಿಂತ-ವರು
ನಿಂತಿತು
ನಿಂತಿ-ದ್ದೇವೆ
ನಿಂತು
ನಿಂದಿ-ಸುತ್ತ
ನಿಃಶು-ಲ್ಕ-ವಾಗಿ
ನಿಃಸ್ಪೃಹ
ನಿಃಸ್ವಾ-ರ್ಥ-ದಿಂದ
ನಿಕ-ಟ-ವಾ-ಗಲು
ನಿಕ-ಟ-ವಾದ
ನಿಖ-ರ-ವಾದ
ನಿಗ-ದಿತ
ನಿಗ-ದಿ-ತ-ವಾ-ಗಿದ್ದು
ನಿಗೂ-ಹತಿ
ನಿಗ್ರ-ಹದ
ನಿಘಂ-ಟು-ವಿ-ನಲ್ಲೂ
ನಿಜ
ನಿಜ-ಕ್ಕಾ-ದರೂ
ನಿಜಕ್ಕೂ
ನಿಜದಿ
ನಿಜ-ವಾ-ಗಿಯೂ
ನಿಜ-ವಾದ
ನಿಜ-ಸ್ವ-ರೂ-ಪ-ವನ್ನು
ನಿಜಾ-ರ್ಥದ
ನಿಜಾ-ರ್ಥ-ದಲ್ಲಿ
ನಿಟ್ಟಿ-ನಲ್ಲಿ
ನಿಟ್ಟು-ಪ-ವಾ-ಸವೇ
ನಿಟ್ಟು-ಸಿರು
ನಿತ್ಯ
ನಿತ್ಯ-ಕರ್ಮ
ನಿತ್ಯ-ಜೀ-ವ-ನ-ವನ್ನು
ನಿತ್ಯದ
ನಿತ್ಯ-ದ-ಟ್ಟ-ದ-ಸವೊ
ನಿತ್ಯ-ಮಾ-ರಾ-ಧ-ಯೇ-ದ್ಗು-ರುಮ್
ನಿತ್ಯವೂ
ನಿತ್ಯ-ನಿ-ರಂ-ತ-ರ-ವಾಗಿ
ನಿದ-ರ್ಶನ
ನಿದ-ರ್ಶ-ನ-ರಾ-ದರು
ನಿದ-ರ್ಶ-ನ-ವಿ-ತ್ತಲ್ಲ
ನಿದ-ರ್ಶ-ನವೇ
ನಿದ್ದೆ-ಕೆಟ್ಟು
ನಿದ್ರಾಂ
ನಿದ್ರಿ-ಸು-ತ್ತಿ-ದ್ದರು
ನಿದ್ರೆಗೂ
ನಿಧ-ನಾ-ನಂ-ತರ
ನಿಧಾನ
ನಿಧಾ-ನ-ವಾಗಿ
ನಿಧಿ-ಯಾ-ಗಲಿ
ನಿಧೀ-ಯ-ತಾಂ
ನಿನ-ಗಿದೆ
ನಿನ್ನ
ನಿನ್ನಣ್ಣ
ನಿನ್ನನು
ನಿನ್ನಿಂದ
ನಿಬಂಧ
ನಿಬಂ-ಧನೆ
ನಿಭಾ-ಯಿ-ಸ-ಲ್ಪ-ಡು-ತ್ತಿ-ದ್ದವು
ನಿಭಾ-ಯಿ-ಸು-ತ್ತಿದ್ದೆ
ನಿಭಾ-ಯಿ-ಸು-ತ್ತಿ-ದ್ದೆವು
ನಿಭಾ-ಯಿ-ಸುವ
ನಿಮಗೆ
ನಿಮ-ಗೇನು
ನಿಮ-ಗೊಂದು
ನಿಮಿತ್ತ
ನಿಮಿ-ತ್ತ-ವಾಗಿ
ನಿಮಿ-ಷ-ಗಳ
ನಿಮ್ಮ
ನಿಮ್ಮ-ಎ-ಲ್ಲಾ-ಕ್ಷೇಮ
ನಿಮ್ಮ-ಕಾ-ರ್ಯಕ್ಕೆ
ನಿಮ್ಮ-ವರು
ನಿಯಂ-ತ್ರಣ
ನಿಯಂ-ತ್ರ-ಣ-ವಿ-ಲ್ಲ-ದಿ-ದ್ದರೆ
ನಿಯಂ-ತ್ರಿಸಿ
ನಿಯಂ-ತ್ರಿ-ಸು-ತ್ತದೆ
ನಿಯತ
ನಿಯ-ತತೆ
ನಿಯ-ತ-ತೆಯ
ನಿಯ-ತ-ವಾಗಿ
ನಿಯ-ತ-ವೆಂದು
ನಿಯ-ತ-ಸಾ-ಹ-ಚರ್ಯ
ನಿಯ-ತ-ಸಾ-ಹ-ಚ-ರ್ಯಮ್
ನಿಯಮ
ನಿಯ-ಮದ
ನಿಯ-ಮ-ದಂತೆ
ನಿಯ-ಮ-ವಿದೆ
ನಿಯ-ಮವೇ
ನಿಯ-ಮಾ-ನು-ಸಾರ
ನಿಯ-ಮಾ-ವ-ಳಿಯ
ನಿಯ-ಮಿ-ತವೂ
ನಿಯ-ಮಿ-ಸಿ-ದೆಯೋ
ನಿಯಮೋ
ನಿಯು-ಕ್ತ-ನಾ-ಗಿದ್ದ
ನಿಯು-ಕ್ತ-ನಾ-ಗಿದ್ದೆ
ನಿಯು-ಕ್ತ-ನಾ-ಗು-ವ-ವ-ರೆಗೂ
ನಿಯು-ಕ್ತ-ನಾದ
ನಿಯು-ಕ್ತ-ನಾ-ದಾಗ
ನಿಯು-ಕ್ತ-ನಾ-ದ್ದ-ರಿಂದ
ನಿಯು-ಕ್ತ-ರಾ-ದೆವು
ನಿಯು-ಕ್ತಿಯ
ನಿರಂ-ಜನ
ನಿರಂ-ಜ-ನ-ವಾ-ನಳ್ಳಿ
ನಿರಂ-ತರ
ನಿರಂ-ತ-ರ-ವಾಗಿ
ನಿರತ
ನಿರ-ತ-ರಾಗಿ
ನಿರ-ತರು
ನಿರ-ತ-ವಾ-ಗು-ವುದು
ನಿರ-ಪ-ರಾ-ಧಿ-ಗ-ಳಿಗೆ
ನಿರ-ಪೇ-ಕ್ಷಿ-ತ-ನಾಗಿ
ನಿರ-ರ್ಗ-ಳ-ವಾಗಿ
ನಿರಾ-ಕ-ರಿ-ಸಿದ
ನಿರಾ-ಕ-ರಿ-ಸು-ವು-ದಿಲ್ಲ
ನಿರಾ-ಡಂ-ಬ-ರ-ವಾದ
ನಿರಾ-ತಂ-ಕ-ವಾಗಿ
ನಿರಾ-ಯಾ-ಸ-ವಾಗಿ
ನಿರಾ-ಳ-ತೆ-ಯ-ನ್ನುಂ-ಟು-ಮಾ-ಡಿತ್ತು
ನಿರಾ-ಶ-ರಾಗಿ
ನಿರಾ-ಶ್ರಿ-ತರ
ನಿರಾ-ಶ್ರಿ-ತ-ರಂತೆ
ನಿರಾ-ಸಕ್ತಿ
ನಿರಾ-ಸೆ-ಯಿಂದ
ನಿರೀ-ಕ್ಷಿ-ತವೇ
ನಿರೀ-ಕ್ಷಿ-ಸು-ವಂ-ತಿ-ರ-ಲಿಲ್ಲ
ನಿರೀ-ಕ್ಷೆಯೇ
ನಿರೀ-ಕ್ಷೆ-ಪ-ರೀ-ಕ್ಷೆ-ಅ-ಪೇ-ಕ್ಷಾ-ರ-ಹಿ-ತ-ರಾಗಿ
ನಿರುಕ್ತ
ನಿರು-ಪ-ದ್ರ-ವ-ವನ್ನೂ
ನಿರೂಪ
ನಿರೂ-ಪ-ಣೆ-ಯ-ಲ್ಲಿನ
ನಿರೂ-ಪಿ-ಸ-ಬೇಕು
ನಿರೂ-ಪಿಸಿ
ನಿರೂ-ಪಿ-ಸಿದೆ
ನಿರೂ-ಪಿ-ಸಿ-ದ್ದಾರೆ
ನಿರೂ-ಪಿ-ಸು-ತ್ತದೆ
ನಿರೂ-ಪಿ-ಸುವ
ನಿರೂ-ಪಿ-ಸು-ವಾಗ
ನಿರ್ಣಯ
ನಿರ್ಣ-ಯ-ಗಳು
ನಿರ್ಣ-ಯ-ದಿಂದ
ನಿರ್ಣ-ಯ-ವಾ-ಯಿತು
ನಿರ್ಣ-ಯಿಸಿ
ನಿರ್ಣ-ಯಿ-ಸಿದೆ
ನಿರ್ಣ-ಯಿ-ಸುವ
ನಿರ್ದಾ-ಕ್ಷಿ-ಣ್ಯ-ವಾಗಿ
ನಿರ್ದಿಷ್ಟ
ನಿರ್ದೇ-ಶ-ಕ-ರಾಗಿ
ನಿರ್ದೇ-ಶ-ನ-ದಂತೆ
ನಿರ್ದೇ-ಶ-ನ-ದಲ್ಲಿ
ನಿರ್ದೇ-ಶಿ-ಸು-ತ್ತದೆ
ನಿರ್ಧ-ರಿಸಿ
ನಿರ್ಧಾರ
ನಿರ್ಧಾ-ರ-ದಿಂದ
ನಿರ್ಬಂ-ಧ-ವಿ-ರ-ಲಿಲ್ಲ
ನಿರ್ಭಯ
ನಿರ್ಭೀ-ತ-ರಾಗಿ
ನಿರ್ಮ-ತ್ಸ-ರಿ-ಗ-ಳಾದ
ನಿರ್ಮಾ-ಣ-ಗೊಂಡ
ನಿರ್ಮಾ-ಣ-ವಾ-ದಾಗ
ನಿರ್ಮಾ-ಣ-ವಾ-ಯಿತು
ನಿರ್ಮಿ-ಸ-ಬ-ಹುದು
ನಿರ್ಮಿ-ಸಲು
ನಿರ್ಮಿಸಿ
ನಿರ್ಮಿ-ಸಿದ
ನಿರ್ಮಿ-ಸಿದ್ದು
ನಿರ್ಮಿ-ಸು-ತ್ತಾನೆ
ನಿರ್ಮಿ-ಸು-ತ್ತಿ-ದ್ದರು
ನಿರ್ಲಕ್ಷ್ಯ
ನಿರ್ಲಿ-ಪ್ತ-ನಾ-ಗಿದ್ದ
ನಿರ್ವಂ-ಚ-ನೆ-ಯಿಂದ
ನಿರ್ವ-ಚನ
ನಿರ್ವ-ಹಣೆ
ನಿರ್ವ-ಹ-ಣೆ-ಗಷ್ಟೆ
ನಿರ್ವ-ಹ-ಣೆ-ಗಾಗಿ
ನಿರ್ವ-ಹ-ಣೆಯ
ನಿರ್ವ-ಹ-ಣೆ-ಯನ್ನು
ನಿರ್ವ-ಹ-ಣೆ-ಯಲ್ಲಿ
ನಿರ್ವ-ಹಿ-ಸ-ಬೇ-ಕಾ-ಯಿತು
ನಿರ್ವ-ಹಿಸಿ
ನಿರ್ವ-ಹಿ-ಸಿದ
ನಿರ್ವ-ಹಿ-ಸಿ-ದರು
ನಿರ್ವ-ಹಿ-ಸಿ-ದ್ದಾರೆ
ನಿರ್ವ-ಹಿ-ಸುತ್ತ
ನಿರ್ವ-ಹಿ-ಸು-ತ್ತಲೇ
ನಿರ್ವ-ಹಿ-ಸು-ತ್ತಿದ್ದ
ನಿರ್ವ-ಹಿ-ಸು-ತ್ತಿ-ದ್ದರು
ನಿರ್ವ-ಹಿ-ಸು-ತ್ತಿ-ರು-ವಾಗ
ನಿರ್ವ-ಹಿ-ಸು-ತ್ತಿ-ರು-ವು-ದನ್ನು
ನಿರ್ವ-ಹಿ-ಸು-ತ್ತಿ-ರು-ವು-ದ-ರಿಂದ
ನಿರ್ವ-ಹಿ-ಸುವ
ನಿರ್ವ-ಹಿ-ಸು-ವಾ-ಗಲೂ
ನಿರ್ವ-ಹಿ-ಸು-ವು-ದ-ರೊಂ-ದಿಗೆ
ನಿರ್ವ-ಹಿ-ಸು-ವುದು
ನಿರ್ವಾ-ಹಕ
ನಿರ್ವ್ಯಾಜ
ನಿಲ-ಯ-ದಲ್ಲಿ
ನಿಲ-ಯ-ವನ್ನು
ನಿಲು-ಕ-ದಂ-ತಿ-ರುವ
ನಿಲು-ಕುವ
ನಿಲು-ವನ್ನು
ನಿಲು-ವಿಗೆ
ನಿಲು-ವಿದೆ
ನಿಲುವು
ನಿಲು-ವು-ಗಳು
ನಿಲ್ದಾ-ಣಕ್ಕೂ
ನಿಲ್ಲ-ಬ-ಲ್ಲರು
ನಿಲ್ಲ-ಬೇ-ಕಾ-ಗು-ತ್ತದೆ
ನಿಲ್ಲಿ-ಸ-ಲಿಲ್ಲ
ನಿಲ್ಲಿ-ಸಿದೆ
ನಿಲ್ಲಿ-ಸಿ-ದ್ದಾರೆ
ನಿಲ್ಲು-ತ್ತ-ದೆಯೋ
ನಿಲ್ಲು-ತ್ತಾರೆ
ನಿಲ್ಲು-ತ್ತಿ-ದ್ದು-ದನ್ನು
ನಿಲ್ಲುವ
ನಿಲ್ಲು-ವಂ-ತಾ-ಯಿತು
ನಿಲ್ಲು-ವುದೂ
ನಿವಾ-ರ-ಣೆಗೆ
ನಿವಾ-ರಿ-ಸಿ-ಕೊ-ಳ್ಳಲು
ನಿವಾ-ರಿ-ಸಿ-ಕೊ-ಳ್ಳು-ತ್ತಿ-ದ್ದರು
ನಿವಾ-ರಿ-ಸಿ-ದ್ದನ್ನು
ನಿವಾ-ರಿ-ಸುವ
ನಿವಾ-ರಿ-ಸು-ವಲ್ಲಿ
ನಿವಾ-ಸಿ-ಗಳು
ನಿವಾ-ಸಿ-ಯಾದ
ನಿವೃತ್ತ
ನಿವೃ-ತ್ತ-ಜೀ-ವನ
ನಿವೃ-ತ್ತ-ನಾ-ಗು-ತ್ತಿ-ದ್ದಾನೆ
ನಿವೃ-ತ್ತ-ನಾ-ಗು-ತ್ತಿದ್ದು
ನಿವೃ-ತ್ತ-ನಾ-ಗು-ತ್ತಿ-ದ್ದೇನೆ
ನಿವೃ-ತ್ತ-ರಲ್ಲ
ನಿವೃ-ತ್ತ-ರಾ-ಗಿ-ದ್ದಾರೆ
ನಿವೃ-ತ್ತ-ರಾ-ಗು-ತ್ತಿ-ದ್ದಾರೆ
ನಿವೃ-ತ್ತ-ರಾ-ಗು-ತ್ತಿ-ರುವ
ನಿವೃ-ತ್ತ-ರಾ-ಗು-ತ್ತಿ-ರು-ವು-ದ-ರಿಂದ
ನಿವೃ-ತ್ತ-ರಾ-ಗು-ತ್ತಿ-ರು-ವುದು
ನಿವೃ-ತ್ತ-ರಾ-ಗುವ
ನಿವೃ-ತ್ತ-ರಾ-ಗು-ವ-ವ-ರಲ್ಲ
ನಿವೃ-ತ್ತ-ರಾ-ಗು-ವುದು
ನಿವೃ-ತ್ತ-ರಾದ
ನಿವೃ-ತ್ತ-ರಾ-ದರು
ನಿವೃ-ತ್ತ-ರಾ-ದರೂ
ನಿವೃತ್ತಿ
ನಿವೃ-ತ್ತಿಯ
ನಿವೃ-ತ್ತಿ-ಯನ್ನು
ನಿವೃ-ತ್ತಿ-ಯಷ್ಟೆ
ನಿವೃ-ತ್ತಿ-ಯಾ-ಗು-ತ್ತಿ-ದ್ದಾರೆ
ನಿವೃ-ತ್ತಿ-ಯಾ-ಗು-ತ್ತಿ-ದ್ದಾ-ರೆಯೇ
ನಿವೃ-ತ್ತಿ-ಯಿ-ರು-ವು-ದಿಲ್ಲ
ನಿವೃ-ತ್ತಿ-ಯಿಲ್ಲ
ನಿವೃ-ತ್ತಿಯೇ
ನಿವೃ-ತ್ತಿ-ಹೊಂ-ದು-ತ್ತಿ-ರುವ
ನಿವೃ-ತ್ತ-ನ್ಯಾ-ಯ-ಪ್ರಾ-ಧ್ಯಾ-ಪಕ
ನಿವೇ-ದಿ-ಸಿ-ದಾಗ
ನಿವೇ-ದಿ-ಸಿದೆ
ನಿವೇ-ಶ-ನ-ವನ್ನು
ನಿವ್ರ್ಯಾ-ಜ-ದಿಂದ
ನಿಶಿ-ತ-ಗೊ-ಳಿಸು
ನಿಶ್ಚ-ಯಿ-ಸ-ಬೇ-ಕಾ-ಗಿದೆ
ನಿಶ್ಚ-ಯಿಸಿ
ನಿಶ್ಚ-ಯಿ-ಸಿ-ದಾಗ
ನಿಶ್ಚ-ಲ-ತೆಯ
ನಿಶ್ಚಿಂತೆ
ನಿಶ್ಚಿತ
ನಿಶ್ಚಿ-ತ-ವಾ-ಯಿತು
ನಿಷೇ-ಧಿ-ಸಿದ್ದು
ನಿಷೇ-ಧಿ-ಸು-ತ್ತದೆ
ನಿಷ್ಕಂ-ಟ-ಕ-ವಾಗಿ
ನಿಷ್ಕ-ಲ್ಮಷ
ನಿಷ್ಕಾಮ
ನಿಷ್ಕಾ-ಮ-ಕ-ರ್ಮದ
ನಿಷ್ಕಾ-ಮ-ಕ-ರ್ಮ-ಯೋ-ಗಿ-ಯಾಗಿ
ನಿಷ್ಕಾ-ಮ-ಕ-ರ್ಮ-ಯೋಗೀ
ನಿಷ್ಕಾಮಾ
ನಿಷ್ಠ-ರಾ-ದಿರಿ
ನಿಷ್ಠೀ-ವನಂ
ನಿಷ್ಠು-ರ-ವಾಗಿ
ನಿಷ್ಠೆ
ನಿಷ್ಠೆಗೆ
ನಿಷ್ಠೆಯ
ನಿಷ್ಠೆ-ಯಲ್ಲಿ
ನಿಷ್ಠೆ-ಯಿಂದ
ನಿಷ್ಠೆಯೇ
ನಿಷ್ಣಾ-ತ-ರಾದ
ನಿಷ್ಪ್ರ-ಯೋ-ಜ-ಕಾಃ
ನಿಸ್ವಾರ್ಥ
ನಿಸ್ವಾ-ರ್ಥ-ದಿಂದ
ನಿಸ್ವಾ-ರ್ಥ-ವಾಗಿ
ನಿಸ್ಸಂ-ಕೋ-ಚ-ವಾಗಿ
ನೀ
ನೀಗಿ-ಸಿ-ಕೊಂ-ಡಿ-ದ್ದಾರೆ
ನೀಗಿ-ಸಿ-ಕೊ-ಳ್ಳು-ತ್ತಿ-ದ್ದೆವು
ನೀಚಾಂ-ಗ-ಸ್ಪ-ರ್ಶನಂ
ನೀಡ-ಬಲ್ಲ
ನೀಡ-ಬ-ಹು-ದಿತ್ತು
ನೀಡ-ಬೇ-ಕೆಂದು
ನೀಡ-ಲಾ-ಗಿತ್ತು
ನೀಡ-ಲಾ-ಯಿತು
ನೀಡ-ಲಾ-ರದು
ನೀಡಲಿ
ನೀಡ-ಲಿಲ್ಲ
ನೀಡ-ಲಿ-ಲ್ಲ-ವೇನೋ
ನೀಡಲು
ನೀಡ-ಲೆಂದು
ನೀಡಿ
ನೀಡಿತು
ನೀಡಿದ
ನೀಡಿ-ದರು
ನೀಡಿ-ದರೂ
ನೀಡಿ-ದರೆ
ನೀಡಿ-ದ-ವರು
ನೀಡಿದೆ
ನೀಡಿ-ದ್ದಕ್ಕೆ
ನೀಡಿ-ದ್ದನ್ನ
ನೀಡಿ-ದ್ದ-ರಿಂದ
ನೀಡಿ-ದ್ದರು
ನೀಡಿ-ದ್ದಾರೆ
ನೀಡಿ-ದ್ದಾ-ರೆಂದು
ನೀಡಿದ್ದು
ನೀಡಿದ್ದೆ
ನೀಡಿ-ರುವ
ನೀಡಿ-ರು-ವರು
ನೀಡಿ-ರು-ವ-ವ-ರನ್ನು
ನೀಡಿ-ರು-ವುದು
ನೀಡು-ತ್ತದೆ
ನೀಡು-ತ್ತ-ದೆಯೋ
ನೀಡು-ತ್ತ-ಲಿ-ರುವ
ನೀಡು-ತ್ತವೆ
ನೀಡು-ತ್ತಾರೆ
ನೀಡು-ತ್ತಿದ್ದ
ನೀಡು-ತ್ತಿ-ದ್ದರು
ನೀಡು-ತ್ತಿ-ದ್ದರೆ
ನೀಡು-ತ್ತಿ-ದ್ದಾಗ
ನೀಡು-ತ್ತಿ-ದ್ದಾರೆ
ನೀಡು-ತ್ತಿದ್ದೆ
ನೀಡು-ತ್ತಿ-ದ್ದೆವು
ನೀಡುವ
ನೀಡು-ವಂತೆ
ನೀಡು-ವ-ದ-ರಿಂದ
ನೀಡು-ವು-ದಾ-ಗಿತ್ತು
ನೀಡು-ವುದು
ನೀತಿ-ಗಳು
ನೀನು
ನೀನೆ
ನೀರನ್ನು
ನೀರಸ
ನೀರಾದೆ
ನೀರಿ-ನಂತೆ
ನೀರಿ-ನಲ್ಲಿ
ನೀರಿ-ನಾ-ಳ-ದಲ್ಲಿ
ನೀರು
ನೀರೆ-ರೆದು
ನೀರೇ
ನೀರೋ-ಗ-ವನ್ನೂ
ನೀಲ್ಗ-ಡಲು
ನೀವು
ನೀವೆಂ-ತಕ್
ನುಗ್ಗಿ
ನುಡಿ
ನುಡಿ-ಗ-ಟ್ಟನ್ನು
ನುಡಿ-ಗಳು
ನುಡಿಗೆ
ನುಡಿದ
ನುಡಿ-ದರು
ನುಡಿ-ನ-ಮನ
ನುಡಿಯ
ನುಡಿ-ಯಲ್ಲ
ನುಡಿ-ಯಲ್ಲಿ
ನುಡಿ-ಯೊಂದೇ
ನುರಿತ
ನೂತನ
ನೂತ-ನ-ವಾಗಿ
ನೂರ
ನೂರಕ್ಕೆ
ನೂರಡಿ
ನೂರಾ
ನೂರಾರು
ನೂರು
ನೂರ್ಕಾಲ
ನೃಣಾಂ
ನೆಂಟ-ನಾ-ಗಿ-ರ-ಬೇಕು
ನೆಂಟರು
ನೆಚ್ಚಿನ
ನೆಟ್ಟ
ನೆಡೆ-ಯಿತು
ನೆನ-ಪನ್ನು
ನೆನ-ಪಾ-ಗು-ವುದು
ನೆನ-ಪಾ-ದುದು
ನೆನ-ಪಿ-ಗಾಗಿ
ನೆನ-ಪಿಗೆ
ನೆನ-ಪಿದೆ
ನೆನ-ಪಿನ
ನೆನ-ಪಿ-ನಂ-ಗ-ಳ-ದಲ್ಲಿ
ನೆನ-ಪಿ-ನಲ್ಲಿ
ನೆನ-ಪಿ-ನ-ಲ್ಲೊಂದು
ನೆನ-ಪಿ-ನಾ-ಳ-ದಿಂದ
ನೆನ-ಪಿ-ರು-ವಂತೆ
ನೆನ-ಪಿಲ್ಲ
ನೆನ-ಪಿ-ಸಿ-ಕೊಂಡು
ನೆನ-ಪಿ-ಸಿ-ಕೊ-ಳ್ಳ-ಬೇ-ಕಾದ
ನೆನ-ಪಿ-ಸಿ-ಕೊ-ಳ್ಳ-ಲೇ-ಬೇ-ಕು-ಊ-ಟ-ವಿ-ಲ್ಲದ
ನೆನ-ಪಿ-ಸಿ-ಕೊಳ್ಳಿ
ನೆನ-ಪಿ-ಸಿ-ಕೊ-ಳ್ಳು-ತ್ತಾರೆ
ನೆನ-ಪಿ-ಸಿ-ಕೊ-ಳ್ಳು-ತ್ತಿ-ರು-ತ್ತೇನೆ
ನೆನ-ಪಿ-ಸಿ-ಕೊ-ಳ್ಳು-ತ್ತೇನೆ
ನೆನ-ಪಿ-ಸಿ-ಕೊ-ಳ್ಳುವ
ನೆನ-ಪಿ-ಸಿದ
ನೆನಪು
ನೆನ-ಪು-ಗ-ಳನ್ನು
ನೆನ-ಸಿ-ಕೊಂ-ಡರೆ
ನೆನ-ಸಿ-ಕೊ-ಳ್ಳು-ತ್ತಿ-ದ್ದಾರೆ
ನೆನಾ-ಪಾ-ಗು-ತ್ತಿ-ದ್ದಿದ್ದು
ನೆನೆ-ದಂ-ತಾ-ಗು-ವುದೋ
ನೆನೆ-ದರೆ
ನೆನೆ-ದು-ಕೊ-ಳ್ಳು-ತ್ತಲೇ
ನೆನೆ-ಸಿ-ಕೊಂ-ಡರೆ
ನೆಪ-ದಲ್ಲಿ
ನೆಪ-ವಾ-ಗಿ-ಸಿ-ಕೊಂಡು
ನೆಪವು
ನೆಮ್ಮದಿ
ನೆಮ್ಮ-ದಿ-ಗ-ಳನ್ನು
ನೆಮ್ಮ-ದಿಯ
ನೆಮ್ಮ-ದಿ-ಯನ್ನು
ನೆಮ್ಮ-ದಿ-ಯಿಂದ
ನೆರ-ವನ್ನೂ
ನೆರ-ವಾ-ಗು-ತ್ತದೆ
ನೆರ-ವಾ-ಗು-ತ್ತಿ-ದ್ದರು
ನೆರ-ವಾ-ಗು-ತ್ತಿ-ದ್ದ-ವರು
ನೆರ-ವಿಗೆ
ನೆರವು
ನೆರ-ವೇ-ರಿತು
ನೆರ-ವೇ-ರಿ-ಸ-ಬೇ-ಕೆಂದು
ನೆರ-ವೇ-ರಿ-ಸಿದ
ನೆರ-ವೇ-ರಿ-ಸುತ್ತ
ನೆರ-ವೇ-ರು-ತ್ತಿದೆ
ನೆಲ
ನೆಲ-ಗಟ್ಟು
ನೆಲದ
ನೆಲ-ಸಿದ
ನೆಲೆ
ನೆಲೆ-ಗ-ಟ್ಟು-ಗ-ಳನ್ನು
ನೆಲೆ-ಗೊಂ-ಡಿ-ರುವ
ನೆಲೆ-ಗೊ-ಳಿಸು
ನೆಲೆ-ನಿ-ಲ್ಲಲು
ನೆಲೆ-ಮಾವು
ನೆಲೆಯ
ನೆಲೆ-ಯನ್ನು
ನೆಲೆ-ಯನ್ನೂ
ನೆಲೆ-ಯಲ್ಲಿ
ನೆಲೆ-ಯಿಂದ
ನೆಲೆ-ವೀಡು
ನೆಲೆ-ಸಲು
ನೆಲೆಸಿ
ನೆಲೆ-ಸಿ-ದ-ವರು
ನೆಲೆ-ಸಿ-ದ್ದರು
ನೆಲೆ-ಸು-ತ್ತಿ-ದ್ದೆವು
ನೇ
ನೇತ-ರಾಣಿ
ನೇಪಾ-ಳದ
ನೇಮ-ಕ-ಗೊಂ-ಡರು
ನೇಮ-ಕ-ಗೊಂಡು
ನೇಮ-ಕಾತಿ
ನೇಮಿ-ಸಿ-ಕೊಂ-ಡಿ-ರುವ
ನೇರ
ನೇರಕ್ಕೆ
ನೇರ-ವಾಗಿ
ನೇರ್ಪು-ಗೊ-ಳಿ-ಸಿ-ಕೊಂ-ಡೆವು
ನೈಜ
ನೈಜ-ಮಾ-ಸ್ತ-ರ-ರಾಗಿ
ನೈತಿಕ
ನೈತಿ-ಕ-ಜ-ವಾ-ಬ್ದಾರಿ
ನೈಮಿ-ತ್ತಿಕ
ನೈಯಾ-ಯಿಕ
ನೈವ
ನೈವಾ-ಕೃ-ತಿಃ
ನೊಂದರು
ನೊಗಕ್ಕೆ
ನೊಟ್ಟು
ನೋಂದಿತ
ನೋಟವೇ
ನೋಡ-ದಂತೆ
ನೋಡದೆ
ನೋಡ-ಬೇ-ಕ-ಲ್ಲವೇ
ನೋಡಲು
ನೋಡಲೇ
ನೋಡಿ
ನೋಡಿ-ಕೊಂಡ
ನೋಡಿ-ಕೊಂ-ಡರು
ನೋಡಿ-ಕೊ-ಳ್ಳ-ಬೇ-ಕಿತ್ತು
ನೋಡಿ-ಕೊಳ್ಳಿ
ನೋಡಿ-ಕೊ-ಳ್ಳು-ತ್ತಾನೆ
ನೋಡಿ-ಕೊ-ಳ್ಳು-ತ್ತಿದ್ದ
ನೋಡಿ-ಕೊ-ಳ್ಳು-ತ್ತಿ-ದ್ದರು
ನೋಡಿ-ಕೊ-ಳ್ಳು-ತ್ತಿ-ದ್ದಾರೆ
ನೋಡಿ-ಕೊ-ಳ್ಳು-ತ್ತೇನೆ
ನೋಡಿದ
ನೋಡಿ-ದಂತೆ
ನೋಡಿ-ದರು
ನೋಡಿ-ದರೆ
ನೋಡಿ-ದರೆ
ನೋಡಿ-ದಾಗ
ನೋಡಿ-ದಾ-ಗ-ಲೆಲ್ಲ
ನೋಡಿ-ದ್ದೆ-ನಾ-ದರೂ
ನೋಡಿ-ದ್ದೇನೆ
ನೋಡಿ-ದ್ದೇವೆ
ನೋಡಿ-ಯಾ-ಯಿತು
ನೋಡಿಯೂ
ನೋಡಿಯೇ
ನೋಡಿ-ರ-ಲಿಲ್ಲ
ನೋಡಿಲ್ಲ
ನೋಡು-ತ್ತಲೇ
ನೋಡುತ್ತಾ
ನೋಡು-ತ್ತಿ-ದ್ದಾರೆ
ನೋಡು-ತ್ತೇವೆ
ನೋಡುವ
ನೋಡೋಣ
ನೋಡೋ-ಣ-ವೆಂದು
ನೋಪ-ಕ-ರಣೇ
ನೋವನ್ನು
ನೋವ-ನ್ನು-ಕೊ-ಟ್ಟರೂ
ನೋವಾ-ಗ-ದಂತೆ
ನೋವಿನ
ನೋವು
ನೋವುಂ-ಟು-ಮಾ-ಡುವ
ನೌಕ-ರಿಯ
ನೌಕೆ-ಯನ್ನು
ನ್ಯಾಯ
ನ್ಯಾಯ-ಕ್ಕಾಗಿ
ನ್ಯಾಯ-ಗಂ-ಗಾ-ಧರ
ನ್ಯಾಯ-ಗಂ-ಗಾ-ಧ-ರ-ಸ್ಸದಾ
ನ್ಯಾಯ-ಗ್ರಂ-ಥ-ಗ-ಳನ್ನು
ನ್ಯಾಯ-ನಿ-ಷ್ಠು-ರತೆ
ನ್ಯಾಯ-ಬ-ಲವ
ನ್ಯಾಯ-ಮಂ-ಜರೀ
ನ್ಯಾಯ-ಮೂ-ರ್ತಿ-ಗಳ
ನ್ಯಾಯ-ವನ್ನು
ನ್ಯಾಯ-ವಾ-ದಿ-ಗ-ಳು-ವ-ಕೀ-ಲರು
ನ್ಯಾಯ-ವಿ-ದ್ಯಾ-ವಿ-ಶಾ-ರದಃ
ನ್ಯಾಯ-ಶಾಸ್ತ್ರ
ನ್ಯಾಯ-ಶಾ-ಸ್ತ್ರಕ್ಕೆ
ನ್ಯಾಯ-ಶಾ-ಸ್ತ್ರದ
ನ್ಯಾಯ-ಶಾ-ಸ್ತ್ರ-ದಲ್ಲಿ
ನ್ಯಾಯ-ಶಾ-ಸ್ತ್ರ-ವನ್ನು
ನ್ಯಾಯ-ಶಾ-ಸ್ತ್ರ-ವನ್ನೇ
ನ್ಯಾಯ-ಶಾ-ಸ್ತ್ರ-ವಿ-ದ್ಯಾ-ರ್ಥಿ-ವೃಂದ
ನ್ಯಾಯ-ಶಾ-ಸ್ತ್ರ-ವಿ-ದ್ಯಾ-ರ್ಥಿ-ಸ-ಮೂ-ಹದ
ನ್ಯಾಯ-ಶಾ-ಸ್ತ್ರ-ವಿ-ಭಾ-ಗದ
ನ್ಯಾಯ-ಶಾ-ಸ್ತ್ರವು
ನ್ಯಾಯ-ಶಾ-ಸ್ತ್ರಾ-ಧ್ಯ-ಯ-ನ-ವನ್ನು
ನ್ಯಾಯ-ಸಿ-ದ್ಧಾಂ-ತ-ವನ್ನು
ನ್ಯಾಯಾದಿ
ನ್ಯಾಯಾ-ಲ-ಯದ
ನ್ಯಾಯಾ-ಲ-ಯ-ವನ್ನು
ನ್ಯಾಯೇ-ತ-ರ-ಶಾ-ಸ್ತ್ರ-ಗಳ
ನ್ಯಾಯ್ಯಾತ್
ನ್ಯಾಸದ
ನ್ಯೂ
ಪ
ಪಂಕ್ತಿ-ಯನ್ನು
ಪಂಕ್ತಿ-ಯಲ್ಲಿ
ಪಂಚ-ಕದ
ಪಂಚ-ತಂತ್ರ
ಪಂಚ-ತಂ-ತ್ರದ
ಪಂಚ-ಲ-ಕ್ಷ-ಣಿ-ಯ-ಲ್ಲೊಂದು
ಪಂಚ-ಲ-ಕ್ಷಣೀ
ಪಂಚಾ-ಯ-ತನ
ಪಂಚಾ-ಯ-ತಿ-ಯಿ-ರಲಿ
ಪಂಚೆ
ಪಂಚೆ-ಯ-ನ್ನು-ಲುಂಗಿ
ಪಂಚೈತೇ
ಪಂಡಿತ
ಪಂಡಿ-ತರ
ಪಂಡಿ-ತ-ರತ್ನಂ
ಪಂಡಿ-ತ-ರಾ-ಗಿ-ದ್ದರೂ
ಪಂಡಿ-ತ-ರಾದ
ಪಂಡಿ-ತ-ರಾ-ದ-ವ-ರಿಗೆ
ಪಂಡಿ-ತ-ರಿಗೆ
ಪಂಡಿ-ತರು
ಪಂಡಿ-ತಾಃ
ಪಂಡಿ-ತೋ-ತ್ತ-ಮ-ರಾದ
ಪಂಡಿ-ತ-ಪಾ-ಮರ
ಪಂಥ
ಪಂಥ-ಗಳೂ
ಪಂಥದ
ಪಕ್ಕದ
ಪಕ್ಕ-ದ-ಲ್ಲಿಯೇ
ಪಕ್ಕ-ದಲ್ಲೆ
ಪಕ್ಕ-ದಲ್ಲೇ
ಪಕ್ಷ
ಪಕ್ಷ-ಧ-ರ್ಮ-ತಾ-ಜ್ಞಾ-ನ-ದಿಂ-ದಲೇ
ಪಕ್ಷ-ಪಾ-ತಿ-ಗ-ಳಾದ
ಪಕ್ಷಿ-ಗಳು
ಪಕ್ಷಿ-ಯ-ರೂ-ಪ-ವನ್ನು
ಪಕ್ಷಿ-ಯಲ್ಲ
ಪಕ್ಷಿಯು
ಪಕ್ಷಿ-ರೂ-ಪ-ದ-ಲ್ಲಿ-ರುವ
ಪಕ್ಷೇ
ಪಟು-ತ್ವ-ವಲ್ಲ
ಪಟ್ಟ-ವ-ರಲ್ಲ
ಪಟ್ಟಾ-ಭಿ-ಷೇ-ಕ-ದಲ್ಲಿ
ಪಟ್ಟಿ-ದ್ದರೆ
ಪಟ್ಟಿ-ಯಲ್ಲಿ
ಪಠಿ-ಸಿದೆ
ಪಠಿ-ಸು-ವ-ವ-ರಿಗೂ
ಪಠಿ-ಸು-ವುದು
ಪಠ್ಯ-ಗಳು
ಪಠ್ಯ-ಗ್ರಂಥ
ಪಠ್ಯ-ದ-ಲ್ಲಿ-ಲ್ಲ-ದಿ-ದ್ದರೂ
ಪಠ್ಯ-ನಿ-ರ್ಮಾ-ಣ-ದಲ್ಲಿ
ಪಠ್ಯ-ಪು-ಸ್ತ-ಕದ
ಪಠ್ಯ-ಪು-ಸ್ತ-ಕ-ರ-ಚ-ನಾ-ಸ-ಮಿತಿ
ಪಠ್ಯ-ರ-ಚನಾ
ಪಠ್ಯೇ-ತರ
ಪಡ-ದಿ-ದ್ದರೆ
ಪಡದೇ
ಪಡಿ-ಯ-ಚ್ಚನ್ನು
ಪಡಿ-ಸು-ತ್ತಿ-ದ್ದರು
ಪಡು-ತ್ತಲೇ
ಪಡು-ತ್ತಾನೆ
ಪಡು-ತ್ತಾರೆ
ಪಡು-ತ್ತಿದ್ದ
ಪಡು-ತ್ತಿ-ರುವ
ಪಡು-ವಂ-ತ-ಹ-ದೇನೂ
ಪಡೆದ
ಪಡೆ-ದಂ-ತಹ
ಪಡೆ-ದ-ದ್ದನ್ನು
ಪಡೆ-ದ-ನಂತೆ
ಪಡೆ-ದರು
ಪಡೆ-ದರೂ
ಪಡೆ-ದ-ವ-ರಾ-ಗಿ-ದ್ದಾರೆ
ಪಡೆ-ದ-ವ-ರಿಲ್ಲ
ಪಡೆ-ದ-ವರು
ಪಡೆ-ದ-ವರೇ
ಪಡೆ-ದಾಗ
ಪಡೆ-ದಿ-ದ್ದ-ರಿಂದ
ಪಡೆ-ದಿ-ದ್ದ-ಲ್ಲದೇ
ಪಡೆ-ದಿ-ದ್ದಾನೆ
ಪಡೆ-ದಿ-ದ್ದಾರೆ
ಪಡೆ-ದಿದ್ದೆ
ಪಡೆ-ದಿ-ದ್ದೇನೆ
ಪಡೆ-ದಿ-ರ-ಲಿಲ್ಲ
ಪಡೆ-ದಿರಿ
ಪಡೆದು
ಪಡೆ-ದು-ಕೊಂ-ಡರು
ಪಡೆ-ದು-ಕೊಂ-ಡ-ವರು
ಪಡೆ-ದು-ಕೊಂ-ಡಾಗ
ಪಡೆ-ದು-ಕೊಂಡೆ
ಪಡೆ-ದು-ಕೊ-ಳ್ಳಲಿ
ಪಡೆದೆ
ಪಡೆಯ
ಪಡೆ-ಯ-ದಂತೆ
ಪಡೆ-ಯದೇ
ಪಡೆ-ಯನ್ನು
ಪಡೆ-ಯ-ಬೇ-ಕೆಂ-ಬುದು
ಪಡೆ-ಯ-ಲಾ-ಗ-ಲಿ-ಲ್ಲ-ವಲ್ಲ
ಪಡೆ-ಯಲಿ
ಪಡೆ-ಯಲು
ಪಡೆ-ಯಿತು
ಪಡೆ-ಯುತ್ತ
ಪಡೆ-ಯು-ತ್ತಾನೆ
ಪಡೆ-ಯು-ತ್ತಾರೆ
ಪಡೆ-ಯು-ತ್ತಿದ್ದ
ಪಡೆ-ಯು-ತ್ತಿ-ದ್ದರು
ಪಡೆ-ಯು-ವಂತೆ
ಪಡೆ-ಯು-ವಲ್ಲಿ
ಪಡೆ-ಯು-ವುದೇ
ಪಡೆವ
ಪತಿಗೆ
ಪತಿ-ವ್ರತೆ
ಪತ್ತೆ
ಪತ್ತೆ-ಮಾಡಿ
ಪತ್ನಿ
ಪತ್ನಿಯ
ಪತ್ನಿ-ಯ-ವರೂ
ಪತ್ನಿಯು
ಪತ್ನಿಯೂ
ಪತ್ನೀ
ಪತ್ರ
ಪತ್ರ-ಕ-ರ್ತ-ರನ್ನು
ಪತ್ರ-ಕ-ರ್ತರು
ಪತ್ರಕ್ಕೆ
ಪತ್ರ-ದಂ-ತಿತ್ತು
ಪತ್ರ-ಮಾ-ಧ್ಯ-ಮ-ವೊಂದೇ
ಪತ್ರಿ-ಕಾ-ಕ-ರ್ತರು
ಪತ್ರಿಕೆ
ಪತ್ರಿ-ಕೆ-ಗ-ಳಲ್ಲಿ
ಪತ್ರಿ-ಕೆ-ಗಾಗಿ
ಪತ್ರಿ-ಕೆಗೆ
ಪತ್ರಿ-ಕೆಯ
ಪತ್ರಿ-ಕೆ-ಯನ್ನು
ಪತ್ರಿ-ಕೆ-ಯಲ್ಲಿ
ಪತ್ರಿ-ಕೋ-ದ್ಯಮ
ಪಥಃ
ಪಥ-ದಲ್ಲಿ
ಪಥ್ಯ-ವೆ-ನಿ-ಸ-ದಿ-ರ-ಬ-ಹುದು
ಪದ
ಪದಂ
ಪದಕ
ಪದ-ಕ-ಗ-ಳನ್ನು
ಪದಕ್ಕೆ
ಪದ-ಗಳ
ಪದ-ಗ-ಳನ್ನು
ಪದ-ಗಳು
ಪದ-ಗಳೇ
ಪದ-ಜಾಲ
ಪದ-ತ-ಲ-ದಲ್ಲಿ
ಪದದ
ಪದ-ಪುಂ-ಜ-ಗಳು
ಪದ-ಬ್ರಹ್ಮ
ಪದ-ವನ್ನು
ಪದವಿ
ಪದ-ವಿ-ಗ-ಳಿಗೆ
ಪದ-ವಿ-ಪೂರ್ವ
ಪದ-ವಿಯ
ಪದ-ವಿ-ಯನ್ನು
ಪದ-ವಿ-ಯನ್ನೂ
ಪದ-ವಿ-ಯಲ್ಲೂ
ಪದ-ವಿ-ಯಿಂದ
ಪದವೀ
ಪದ-ವೀ-ಧರ
ಪದ-ವೀ-ಧ-ರ-ನಾ-ಗಿದ್ದು
ಪದ-ವೀ-ಧ-ರ-ರಾಗಿ
ಪದ-ವೀ-ಧ-ರರು
ಪದಾ-ಧಿ-ಕಾ-ರಿ-ಗ-ಳ-ನ್ನಾಗಿ
ಪದೆ-ದು-ಕೊ-ಳ್ಳು-ತ್ತಿ-ದ್ದರು
ಪದೇ
ಪದ್ದ-ತಿಗೆ
ಪದ್ದ-ತಿಯ
ಪದ್ದ-ತಿ-ಯಲ್ಲಿ
ಪದ್ಧತಿ
ಪದ್ಧ-ತಿ-ಗ-ಳನ್ನು
ಪದ್ಧ-ತಿಗೂ
ಪದ್ಧ-ತಿ-ಯಲ್ಲಿ
ಪದ್ಧ-ತಿ-ಯಾ-ಗಿತ್ತು
ಪದ್ಮ-ಪ-ತ್ರ-ಮಿ-ವಾಂ-ಭಸಾ
ಪದ್ಮ-ಪ-ತ್ರ-ಮಿ-ವಾಂ-ಭಸಿ
ಪದ್ಯ
ಪದ್ಯ-ಗ-ಳನ್ನು
ಪದ್ಯ-ದೋಘ
ಪದ್ಯ-ವಿ-ರಲಿ
ಪಯಃ
ಪಯೋ
ಪಯೋ-ಗಿ-ಸ-ಬ-ಹುದ
ಪರ
ಪರಂ
ಪರಂತು
ಪರಂ-ಪರೆ
ಪರಂ-ಪ-ರೆಯ
ಪರಂ-ಪ-ರೆ-ಯನ್ನು
ಪರಂ-ಪ-ರೆ-ಯಲ್ಲಿ
ಪರ-ದಾಟ
ಪರ-ದೇ-ಶಿ-ಗ-ಳಾಗಿ
ಪರ-ದೋಷ
ಪರ-ದೋ-ಷ-ಗ-ಳನ್ನು
ಪರ-ದೋ-ಷ-ಗ್ರ-ಹ-ಣ-ವಾ-ದರೆ
ಪರ-ಪ್ರ-ಸ್ತುತಿ
ಪರಮ
ಪರ-ಮ-ಧ-ರ್ಮ-ವಾದ
ಪರ-ಮ-ಪು-ನೀತೆ
ಪರ-ಮ-ವಾ-ಪ್ಸ್ಯಥ
ಪರ-ಮ-ಸಾ-ರ್ಥ-ಕ್ಯದ
ಪರ-ಮಾತ್ಮ
ಪರ-ಮಾ-ತ್ಮನ
ಪರ-ಮಾ-ತ್ಮ-ನನ್ನು
ಪರ-ಮಾ-ತ್ಮ-ನಲ್ಲಿ
ಪರ-ಮಾ-ತ್ಮ-ನಲ್ಲೂ
ಪರ-ಮಾ-ನಂದ
ಪರ-ಮಾ-ನಂ-ದ-ವಾ-ಯಿತು
ಪರ-ಮೇ-ಶ್ವರ
ಪರ-ಮೋ-ದ್ದೇಶ
ಪರ-ರನ್ನು
ಪರ-ರಿಂದ
ಪರ-ರಿಗೆ
ಪರ-ವಾಗಿ
ಪರ-ಶು-ರಾ-ಮ-ರಂತೆ
ಪರ-ಸ್ಪರ
ಪರ-ಸ್ಪರಂ
ಪರ-ಸ್ಪ-ರರ
ಪರ-ಸ್ಪ-ರರು
ಪರಾಂ-ಕುಶ
ಪರಾ-ಮರ್ಶ
ಪರಾ-ಮ-ರ್ಶಿ-ಸಿ-ದಾಗ
ಪರಿ
ಪರಿ-ಕ-ಲ್ಪ-ನೆ-ಯಿಂ-ದಲೇ
ಪರಿಕ್ಷಾ
ಪರಿ-ಗ-ಣ-ಸ-ಬೇ-ಕಾ-ದದ್ದು
ಪರಿ-ಗ-ಣಿ-ಸದೇ
ಪರಿ-ಗ-ಣಿಸಿ
ಪರಿ-ಗ-ಣಿ-ಸಿ-ದರೆ
ಪರಿ-ಗ-ಣಿ-ಸು-ವುದೇ
ಪರಿ-ಚಯ
ಪರಿ-ಚ-ಯಕ್ಕೆ
ಪರಿ-ಚ-ಯದ
ಪರಿ-ಚ-ಯ-ದ-ವರು
ಪರಿ-ಚ-ಯ-ದಿಂದ
ಪರಿ-ಚ-ಯ-ಮಾ-ಡಿ-ಕೊ-ಳ್ಳು-ವಾಗ
ಪರಿ-ಚ-ಯ-ವನ್ನೂ
ಪರಿ-ಚ-ಯ-ವಾ-ಗ-ದಿ-ರು-ವಾಗ
ಪರಿ-ಚ-ಯ-ವಾ-ಗಲೀ
ಪರಿ-ಚ-ಯ-ವಾ-ಗಿದ್ದು
ಪರಿ-ಚ-ಯ-ವಾದ
ಪರಿ-ಚ-ಯ-ವಾ-ದದ್ದು
ಪರಿ-ಚ-ಯ-ವಾ-ದರು
ಪರಿ-ಚ-ಯ-ವಾ-ದಾಗ
ಪರಿ-ಚ-ಯ-ವಾ-ಯಿತು
ಪರಿ-ಚ-ಯ-ವಿ-ರ-ಲಿಲ್ಲ
ಪರಿ-ಚ-ಯ-ವಿ-ಲ್ಲ-ದ-ವರೇ
ಪರಿ-ಚ-ಯವು
ಪರಿ-ಚ-ಯವೂ
ಪರಿ-ಚ-ಯವೇ
ಪರಿ-ಚ-ಯಾ-ತ್ಮಕ
ಪರಿ-ಚ-ಯಿಸಿ
ಪರಿ-ಚ-ಯಿ-ಸಿ-ಕೊಂಡು
ಪರಿ-ಚ-ಯಿ-ಸಿ-ದರು
ಪರಿ-ಚ-ಯಿ-ಸಿ-ದು-ದನ್ನೂ
ಪರಿ-ಚಾ-ರ-ಯಜ್ಞ
ಪರಿ-ಚಾ-ರ-ಯ-ಜ್ಞಾಃ
ಪರಿ-ಚಿತ
ಪರಿ-ಚಿ-ತ-ನಾ-ಗಿ-ದ್ದ-ರಿಂದ
ಪರಿ-ಚಿ-ತ-ರಾಗಿ
ಪರಿ-ಚಿ-ತ-ರೆಂ-ಬುದು
ಪರಿ-ಚಿ-ತರೇ
ಪರಿ-ಚಿ-ತ-ವಾ-ಗಿ-ರುವ
ಪರಿ-ಣಾಮ
ಪರಿ-ಣಾ-ಮ-ವಾಗಿ
ಪರಿ-ಣಾ-ಮ-ವಾ-ಯಿ-ತೆಂ-ದರೆ
ಪರಿ-ತ-ಪಿ-ಸಿದ
ಪರಿಧಿ
ಪರಿ-ಧಿ-ಯನ್ನು
ಪರಿ-ಧಿ-ಯಲ್ಲಿ
ಪರಿ-ಪಾ-ಠ-ವನ್ನು
ಪರಿ-ಪಾ-ಠ-ವಿದೆ
ಪರಿ-ಪಾ-ಲಿ-ಸದ
ಪರಿ-ಪೂರ್ಣ
ಪರಿ-ಪೂ-ರ್ಣ-ತೆಯ
ಪರಿ-ಪೂ-ರ್ಣ-ತೆ-ಯೆ-ಡೆಗೆ
ಪರಿ-ಭಾ-ಷೆ-ಯಲ್ಲಿ
ಪರಿ-ಮಳ
ಪರಿ-ಮ-ಳ-ವನ್ನು
ಪರಿ-ಮ-ಳಿ-ಸು-ತ್ತಿ-ರು-ವುದು
ಪರಿಯ
ಪರಿ-ಯಿಂ-ದಾಗಿ
ಪರಿ-ಯೇನು
ಪರಿ-ರ-ಕ್ಷ-ಣಾಯ
ಪರಿ-ವ-ರ್ತನೆ
ಪರಿ-ವ-ರ್ತಿ-ತ-ವಾ-ಯಿತು
ಪರಿ-ವ-ರ್ತಿನಿ
ಪರಿ-ವ-ರ್ತಿ-ಸು-ತ್ತದೆ
ಪರಿ-ವೀ-ಕ್ಷ-ಣೆ-ಯಲ್ಲಿ
ಪರಿವೇ
ಪರಿ-ಶಿ-ಲಿಸಿ
ಪರಿ-ಶೀ-ಲಿ-ಸಲು
ಪರಿ-ಶೀ-ಲಿಸಿ
ಪರಿ-ಶುದ್ಧ
ಪರಿ-ಶು-ದ್ಧವೂ
ಪರಿ-ಶು-ದ್ಧಾಂ-ತ-ರಂ-ಗ-ದ-ವರು
ಪರಿ-ಶ್ರ-ಮಿಸಿ
ಪರಿ-ಷ-ತ್ತಿನ
ಪರಿ-ಷ-ತ್ತಿ-ನಲ್ಲಿ
ಪರಿ-ಷತ್ತು
ಪರಿ-ಷತ್ನ
ಪರಿ-ಷ್ಕ-ರಿ-ಸಿ-ಕೊಂ-ಡಿರಿ
ಪರಿ-ಷ್ಕಾ-ರಕ್ಕೆ
ಪರಿ-ಷ್ಕಾ-ರ-ವನ್ನು
ಪರಿ-ಸರ
ಪರಿ-ಸ-ರ-ಗ-ಳಲ್ಲಿ
ಪರಿ-ಸ-ರ-ದಲ್ಲಿ
ಪರಿ-ಸ್ಥಿತಿ
ಪರಿ-ಸ್ಥಿ-ತಿ-ಗ-ನು-ಗು-ಣ-ವಾಗಿ
ಪರಿ-ಸ್ಥಿ-ತಿ-ಯನ್ನು
ಪರಿ-ಸ್ಥಿ-ತಿ-ಯಲ್ಲಿ
ಪರಿ-ಸ್ಥಿ-ತಿ-ಯ-ಲ್ಲಿದೆ
ಪರಿ-ಹ-ರಿ-ಸಲು
ಪರಿ-ಹ-ರಿಸಿ
ಪರಿ-ಹ-ರಿ-ಸಿ-ಕೊಂಡು
ಪರಿ-ಹ-ರಿ-ಸಿ-ಕೊ-ಳ್ಳಲು
ಪರಿ-ಹ-ರಿ-ಸಿ-ಕೊ-ಳ್ಳು-ತ್ತಿ-ದ್ದರು
ಪರಿ-ಹ-ರಿ-ಸಿ-ದೆ-ಯೆಂ-ಬುದು
ಪರಿ-ಹ-ರಿ-ಸು-ತ್ತಿ-ದ್ದರು
ಪರಿ-ಹ-ರಿ-ಸುವ
ಪರಿ-ಹಾರ
ಪರಿ-ಹಾ-ರ-ಕ್ಕಾಗಿ
ಪರಿ-ಹಾ-ರಕ್ಕೆ
ಪರಿ-ಹಾ-ರ-ವನ್ನೂ
ಪರಿ-ಹಾ-ರ-ವಿ-ರು-ತ್ತದೆ
ಪರಿ-ಹಾ-ವನ್ನು
ಪರಿ-ಹಾ-ಸ-ದಲ್ಲಿ
ಪರಿ-ಹಾ-ಸ-ಪ್ರಿ-ಯರೂ
ಪರಿ-ಹಾ-ಸ-ವಾಗಿ
ಪರೀ-ಕ್ಷ-ಕತ್ವ
ಪರೀ-ಕ್ಷ-ಕ-ತ್ವ-ವನ್ನು
ಪರೀ-ಕ್ಷ-ಕ-ನಾಗಿ
ಪರೀಕ್ಷಾ
ಪರೀ-ಕ್ಷಿತ್
ಪರೀ-ಕ್ಷಿ-ಸ-ಬೇಕು
ಪರೀ-ಕ್ಷಿ-ಸ-ಲಾಗಿ
ಪರೀ-ಕ್ಷಿಸಿ
ಪರೀಕ್ಷೆ
ಪರೀ-ಕ್ಷೆ-ಗಳ
ಪರೀ-ಕ್ಷೆ-ಗ-ಳನ್ನು
ಪರೀ-ಕ್ಷೆ-ಗ-ಳಲ್ಲಿ
ಪರೀ-ಕ್ಷೆ-ಗಾಗಿ
ಪರೀ-ಕ್ಷೆಗೆ
ಪರೀ-ಕ್ಷೆ-ಗೆಂದು
ಪರೀ-ಕ್ಷೆಯ
ಪರೀ-ಕ್ಷೆ-ಯನ್ನು
ಪರೀ-ಕ್ಷೆ-ಯಲ್ಲಿ
ಪರೀ-ಕ್ಷೋ-ಪ-ಯೋ-ಗಿ-ಯಾಗಿ
ಪರೇ-ಷಾಂ
ಪರೋ-ಕ್ಷ-ದಲ್ಲಿ
ಪರೋ-ಕ್ಷ-ವಾಗಿ
ಪರೋ-ದ್ಧಾ-ರ-ದಿಂದ
ಪರೋ-ಪ-ಕಾರ
ಪರೋ-ಪ-ಕಾ-ರಾ-ರ್ಥ-ಮಿದಂ
ಪರೋ-ಪ-ಕಾರಿ
ಪರ್ಯಂತ
ಪರ್ಯ-ವ-ಸಾ-ನ-ವಾದ
ಪರ್ಯಾಯ
ಪರ್ಯಾ-ಯ-ದಂತೆ
ಪರ್ವತೇ
ಪರ್ವ-ದಿಂ-ದಾ-ರಂ-ಭಿಸಿ
ಪರ್ವಸು
ಪರ-ವಿ-ರೋ-ಧ-ವಾಗಿ
ಪಳ-ಗಿದ
ಪಳೆ-ಯು-ಳಿ-ಕೆ-ಗಳು
ಪಶು
ಪಶ್ಚಿಮ
ಪಶ್ಯತಿ
ಪಸ-ರಿ-ಸಲೇ
ಪಸ-ರಿಸಿ
ಪಸ-ರಿ-ಸು-ತ್ತಿ-ರಲಿ
ಪಾಂಡಿ-ಚೇ-ರಿಗೆ
ಪಾಂಡಿ-ಚೇ-ರಿಯ
ಪಾಂಡಿ-ಚೇ-ರಿ-ಯ-ನ್ನೆಲ್ಲಾ
ಪಾಂಡಿತ್ಯ
ಪಾಂಡಿ-ತ್ಯಕ್ಕೆ
ಪಾಂಡಿ-ತ್ಯದ
ಪಾಂಡಿ-ತ್ಯ-ಪೂ-ರ್ಣ-ವಾದ
ಪಾಂಡಿ-ತ್ಯ-ವನ್ನು
ಪಾಂಡಿ-ತ್ಯ-ವ-ನ್ನೆಂದೂ
ಪಾಂಡಿ-ತ್ಯ-ವಿ-ರ-ಬ-ಹುದು
ಪಾಕ-ಶಾ-ಲೆಯ
ಪಾಕ್ಷಿ-ಕ-ಗೋ-ಷ್ಠಿ-ಯ-ಯನ್ನು
ಪಾಚಿ
ಪಾಟ
ಪಾಟವ
ಪಾಟೀ-ಲರು
ಪಾಟೀಲ್
ಪಾಠ
ಪಾಠ-ಪ್ರ-ವ-ಚನ
ಪಾಠ-ಪ್ರ-ವ-ಚ-ನ-ಗ-ಳಿಗೆ
ಪಾಠ-ಕ-ನಾಗಿ
ಪಾಠ-ಕ್ಕಾಗಿ
ಪಾಠಕ್ಕೂ
ಪಾಠಕ್ಕೆ
ಪಾಠ-ಕ್ಕೆಂದು
ಪಾಠ-ಕ್ರ-ಮ-ದಲ್ಲಿ
ಪಾಠ-ಗ-ಳನ್ನು
ಪಾಠದ
ಪಾಠ-ದಲ್ಲಿ
ಪಾಠ-ದಿಂದ
ಪಾಠನ
ಪಾಠ-ಪ್ರ-ವ-ಚನ
ಪಾಠ-ಪ್ರ-ವ-ಚ-ನ-ಗಳ
ಪಾಠ-ಪ್ರ-ವ-ಚ-ನ-ಗ-ಳನ್ನು
ಪಾಠ-ಪ್ರ-ವ-ಚ-ನ-ಗ-ಳಲ್ಲಿ
ಪಾಠ-ಪ್ರ-ವ-ಚ-ನದ
ಪಾಠ-ಪ್ರಾ-ರಂ-ಭಿ-ಸಿ-ದರು
ಪಾಠ-ಮಾ-ಡದೆ
ಪಾಠ-ಮಾ-ಡದೇ
ಪಾಠ-ಮಾ-ಡ-ಬ-ಲ್ಲ-ವರು
ಪಾಠ-ಮಾ-ಡ-ಬೇ-ಕೆಂಬ
ಪಾಠ-ಮಾ-ಡ-ಲಾ-ರಂ-ಭಿ-ಸಿದೆ
ಪಾಠ-ಮಾ-ಡಲು
ಪಾಠ-ಮಾಡಿ
ಪಾಠ-ಮಾ-ಡಿದ
ಪಾಠ-ಮಾ-ಡಿ-ದರು
ಪಾಠ-ಮಾ-ಡಿ-ದ-ವರು
ಪಾಠ-ಮಾ-ಡಿದೆ
ಪಾಠ-ಮಾ-ಡಿ-ದ್ದಾರೆ
ಪಾಠ-ಮಾ-ಡಿ-ರು-ತ್ತಾರೆ
ಪಾಠ-ಮಾ-ಡಿ-ಸ-ತೊ-ಡ-ಗಿ-ದರು
ಪಾಠ-ಮಾ-ಡಿಸಿ
ಪಾಠ-ಮಾ-ಡಿ-ಸಿ-ಕೊಂ-ಡ-ವನೇ
ಪಾಠ-ಮಾ-ಡಿ-ಸಿ-ಕೊಂಡು
ಪಾಠ-ಮಾ-ಡಿ-ಸಿ-ಕೊ-ಳ್ಳು-ತ್ತಿದ್ದ
ಪಾಠ-ಮಾ-ಡಿ-ಸಿ-ಕೊ-ಳ್ಳು-ತ್ತಿ-ದ್ದರು
ಪಾಠ-ಮಾ-ಡು-ತ್ತಿ-ದ್ದರು
ಪಾಠ-ಮಾ-ಡು-ತ್ತಿ-ದ್ದರೂ
ಪಾಠ-ಮಾ-ಡುವ
ಪಾಠ-ಮಾ-ಡು-ವಾಗ
ಪಾಠ-ಮಾ-ಡು-ವಾ-ಗಲೂ
ಪಾಠ-ಮಾ-ಡು-ವು-ದ-ರಲ್ಲಿ
ಪಾಠ-ಮಾ-ಡು-ವುದು
ಪಾಠವ
ಪಾಠ-ವನ್ನು
ಪಾಠ-ವನ್ನೂ
ಪಾಠ-ವನ್ನೇ
ಪಾಠ-ವಾ-ಗ-ಬ-ಲ್ಲದು
ಪಾಠ-ವಾ-ಗಿದೆ
ಪಾಠ-ವಾ-ದರೆ
ಪಾಠ-ವಾ-ಯಿತು
ಪಾಠ-ವಿ-ಷ-ಯ-ಗಳು
ಪಾಠವು
ಪಾಠವೇ
ಪಾಠ-ಶಾಲಾ
ಪಾಠ-ಶಾಲೆ
ಪಾಠ-ಶಾ-ಲೆ-ಗಳ
ಪಾಠ-ಶಾ-ಲೆ-ಗ-ಳಲ್ಲಿ
ಪಾಠ-ಶಾ-ಲೆ-ಗ-ಳಿಗೆ
ಪಾಠ-ಶಾ-ಲೆಗೆ
ಪಾಠ-ಶಾ-ಲೆಗೇ
ಪಾಠ-ಶಾ-ಲೆಯ
ಪಾಠ-ಶಾ-ಲೆ-ಯನ್ನು
ಪಾಠ-ಶಾ-ಲೆ-ಯನ್ನೂ
ಪಾಠ-ಶಾ-ಲೆ-ಯಲ್ಲಿ
ಪಾಠ-ಶಾ-ಲೆ-ಯ-ಲ್ಲಿಯೂ
ಪಾಠ-ಶಾ-ಲೆ-ಯ-ಲ್ಲಿಯೇ
ಪಾಠ-ಶಾ-ಲೆ-ಯಲ್ಲೂ
ಪಾಠ-ಶಾ-ಲೆ-ಯಿಂದ
ಪಾಠ-ಶಾ-ಲೆಯು
ಪಾಠ್ಯ-ಗ್ರಂ-ಥ-ವನ್ನು
ಪಾಠ್ಯ-ಪು-ಸ್ತಕ
ಪಾಠ್ಯ-ಪು-ಸ್ತ-ಕ-ಗಳ
ಪಾಠ್ಯ-ವಾ-ಗಿತ್ತು
ಪಾಠ್ಯ-ವಿ-ಷ-ಯ-ಗ-ಳೆ-ಲ್ಲ-ವನ್ನೂ
ಪಾಠ್ಯ-ವಿ-ಷ-ಯದ
ಪಾಠ್ಯೇ-ತರ
ಪಾಠ-ಪ್ರ-ವ-ಚನ
ಪಾಠ-ಪ್ರ-ವ-ಚ-ನ-ಗ-ಳನ್ನು
ಪಾಠ-ಪ್ರ-ವ-ಚ-ನ-ಗಳು
ಪಾಠ-ಪ್ರ-ವ-ಚ-ನ-ವನ್ನು
ಪಾಠ-ಪ್ರ-ವ-ಚ-ನ-ವನ್ನೇ
ಪಾಠ-ಪ್ರ-ವ-ಚ-ನವೂ
ಪಾಡಾ-ಗಿತ್ತು
ಪಾಡಿಗೆ
ಪಾಡಿತ್ಯ
ಪಾಣಿ-ನಿಯ
ಪಾತಾ-ಳ-ಗಂ-ಗೆಯೂ
ಪಾತ್ರ
ಪಾತ್ರ-ಧಾ-ರಿ-ಗಳು
ಪಾತ್ರ-ರಾಗಿ
ಪಾತ್ರ-ರಾ-ಗಿ-ದ್ದಾರೆ
ಪಾತ್ರ-ರಾದ
ಪಾತ್ರ-ವನ್ನ
ಪಾತ್ರ-ವನ್ನು
ಪಾತ್ರ-ವ-ಹಿ-ಸಿ-ದ್ದಾರೆ
ಪಾತ್ರ-ವಾಗಿ
ಪಾತ್ರ-ವಾ-ಯಿತು
ಪಾತ್ರವೂ
ಪಾತ್ರೆಯು
ಪಾದ-ಸ್ಪ-ರ್ಶ-ಜ-ನ್ಯ-ಸು-ಖಾ-ದಿಭಿ
ಪಾದಾ-ರ್ಪಣೆ
ಪಾನ-ಮಾಡಿ
ಪಾಪ
ಪಾಪ-ನಾ-ಶಕಃ
ಪಾಪಾ-ನ್ನಿ-ವಾ-ರ-ಯತಿ
ಪಾಯಸ
ಪಾಯಿಂ-ಟು-ಗ-ಳನ್ನು
ಪಾರಂ-ಗ-ತರು
ಪಾರಂ-ಪ-ರಿಕ
ಪಾರ-ದ-ರ್ಶ-ಕತೆ
ಪಾರ-ವುಂಟೆ
ಪಾರವೇ
ಪಾರಾ-ಯಣ
ಪಾರಾ-ಯಿತು
ಪಾರಿ-ತೋ-ಷಕ
ಪಾರಿ-ತೋ-ಷ-ಕ-ಗಳು
ಪಾರಿ-ತೋ-ಷ-ಕವು
ಪಾರ್ವ-ತಿ-ಯೊಂ-ದಿಗೆ
ಪಾಲ-ಕರು
ಪಾಲನೆ
ಪಾಲಾ-ಗು-ತ್ತಿತ್ತು
ಪಾಲಾ-ಯಿ-ತ-ಲ್ಲವೆ
ಪಾಲಿಗೆ
ಪಾಲಿ-ಗೊಂದು
ಪಾಲಿ-ಟೆ-ಕ್ನಿಕ್
ಪಾಲಿನ
ಪಾಲು
ಪಾಲೂ
ಪಾಲ್ಗೊಂ-ಡಿದ್ದು
ಪಾಲ್ಗೊ-ಳ್ಳ-ಬೇ-ಕಾ-ಯಿತು
ಪಾಲ್ಗೊ-ಳ್ಳು-ತ್ತಿ-ರುವ
ಪಾಲ್ಗೊ-ಳ್ಳುವ
ಪಾವ-ನ-ಗೊ-ಳಿ-ಸಿ-ಕೊ-ಳ್ಳಲಿ
ಪಾಶ್ಚಾ-ತ್ಯರ
ಪಾಸಾ-ಗಿದ್ದ
ಪಾಸು-ಮಾ-ಡಿ-ಕೊಂ-ಡೆವು
ಪಿಎಚ್
ಪಿಜಿ-ಭಟ್ಟ
ಪಿತರ
ಪಿತ-ರ-ಸ್ತಾ-ಸಾಂ
ಪಿತಾ
ಪಿತಾ-ಮಹ
ಪಿತಾ-ಮ-ಹರು
ಪಿತಾ-ಮ-ಹಿಯ
ಪಿತೃ-ಋ-ಣ-ದಲ್ಲಿ
ಪಿಪಾ-ಸು-ಘ-ಳಗಿ
ಪಿಯು
ಪಿಯುಸಿ
ಪಿಯು-ಸಿಗೂ
ಪಿಯು-ಸಿ-ಯಲ್ಲಿ
ಪೀಠಾ-ಧಿ-ಪ-ತಿ-ಗ-ಳಾ-ಗಿ-ರುವ
ಪೀಠಾ-ಧ್ಯ-ಕ್ಷರು
ಪೀಠಿಕೆ
ಪೀಠಿ-ಕೆಯ
ಪೀಡಿ-ತ-ನಾದ
ಪೀಳಿಗೆ
ಪೀಳಿ-ಗೆ-ಯನ್ನು
ಪೀಳಿ-ಗೆ-ಯಲ್ಲಿ
ಪೀಳಿ-ಗೆ-ಯಲ್ಲೂ
ಪುಂಖಾ-ನು-ಪುಂ-ಖ-ವಾಗಿ
ಪುಂಡರ
ಪುಂಡಾ-ಟಿಕೆ
ಪುಟ-ಗಳ
ಪುಟ-ದಲ್ಲಿ
ಪುಟ್ಟ
ಪುಟ್ಟ-ನ-ಮನೆ
ಪುಣ್ಯ
ಪುಣ್ಯ-ತ-ಮ-ಕಾ-ರ್ಯ-ವದು
ಪುಣ್ಯ-ವಂ-ತರು
ಪುಣ್ಯ-ವಿ-ಶೇಷ
ಪುಣ್ಯಾ-ತಿ-ಶಯ
ಪುಣ್ಯಾ-ತಿ-ಶ-ಯ-ವ-ಲ್ಲದೇ
ಪುಣ್ಯಾ-ರಣ್ಯೇ
ಪುಣ್ಯೇ
ಪುತ್ರ
ಪುತ್ರ-ನಂತೆ
ಪುತ್ರ-ನಾದ
ಪುತ್ರ-ರನ್ನು
ಪುತ್ರ-ರಾದ
ಪುತ್ರ-ರಿ-ಬ್ಬರು
ಪುತ್ರರು
ಪುತ್ರ-ವಂ-ತರು
ಪುತ್ರಿ-ಯ-ರನ್ನು
ಪುತ್ರಿ-ಯಾದ
ಪುತ್ರೋ-ತ್ಸ-ವ-ವಾ-ಯಿತು
ಪುನಃ
ಪುನಃ-ಗಂ-ಗಾ-ಧ-ರ-ಭ-ಟ್ಟರ
ಪುನ-ರಾ-ವ-ರ್ತಿ-ಸು-ವು-ದ-ರಿಂದ
ಪುರ-ಸ್ಕಾರ
ಪುರ-ಸ್ಕಾ-ರ-ಗ-ಳನ್ನು
ಪುರ-ಸ್ಕಾ-ರ-ಗಳು
ಪುರಾ-ಣ-ಗ-ಳಲ್ಲೋ
ಪುರಾ-ಣ-ಪ್ರಜ್ಞೆ
ಪುರಾ-ಣ-ಪ್ರ-ಸಿ-ದ್ಧ-ವೃ-ತ್ತಾಂ-ತ-ವಾ-ಗಿದೆ
ಪುರಾ-ತ-ನ-ವಾ-ದುದು
ಪುರಾವೆ
ಪುರುಷ
ಪುರುಷಃ
ಪುರು-ಷ-ಕಾ-ರದ
ಪುರು-ಷ-ಕಾರೇ
ಪುರು-ಷನ
ಪುರು-ಷಸ್ಯ
ಪುರು-ಷಾರ್ಥ
ಪುರೊ-ಹಿತ
ಪುರೋ-ಹಿತ
ಪುರೋ-ಹಿ-ತ-ರಾಗಿ
ಪುರೋ-ಹಿ-ತ-ರಾ-ದರೆ
ಪುರೋ-ಹಿ-ತರು
ಪುರೋ-ಹಿ-ತರೂ
ಪುರೋ-ಹಿ-ತ-ರೆಂದು
ಪುರೋ-ಹಿ-ತ-ರೆಂಬ
ಪುರೋ-ಹಿ-ತ-ವಾ-ದವು
ಪುಲ-ಕಿ-ತ-ನಾ-ದದ್ದು
ಪುಳ-ಕ-ವಾ-ದರೂ
ಪುಷ್ಟೀ-ಕ-ರಿ-ಸು-ತ್ತದೆ
ಪುಷ್ಪಕ
ಪುಷ್ಪ-ಕೈಃ
ಪುಷ್ಪ-ಗ-ಳಿಂದ
ಪುಷ್ಪವು
ಪುಸ್ತಕ
ಪುಸ್ತ-ಕ-ಗ-ಳನ್ನು
ಪುಸ್ತ-ಕ-ಗಳು
ಪುಸ್ತ-ಕ-ವನ್ನು
ಪುಸ್ತ-ಕ-ವಾಗಿ
ಪುಸ್ತ-ಕ-ವಾ-ದೀತು
ಪೂಜ-ನೀ-ಯರು
ಪೂಜಾ
ಪೂಜಾ-ವಿ-ಧಿ-ಗ-ಳನ್ನು
ಪೂಜಿ-ಸು-ತ್ತೇವೆ
ಪೂಜೆ-ಯನ್ನು
ಪೂಜೆ-ಯಲ್ಲಿ
ಪೂಜ್ಯ
ಪೂಜ್ಯತೇ
ಪೂಜ್ಯ-ಪೂ-ಜಾ-ವ್ಯ-ತಿ-ಕ್ರ-ಮ-ವಾ-ಗ-ದಂತೆ
ಪೂಜ್ಯ-ಪೂ-ಜೆ-ಯಿಂದ
ಪೂಜ್ಯ-ಭಾವ
ಪೂಜ್ಯರ
ಪೂಜ್ಯ-ರಾ-ದ-ಗಂ-ಗಾ-ಧರ
ಪೂಜ್ಯರು
ಪೂಜ್ಯ-ವಾದ
ಪೂರ-ಕ-ವಾ-ಗಿ-ರ-ಬೇ-ಕಾ-ಗು-ತ್ತದೆ
ಪೂರ-ಕ-ವಾದ
ಪೂರೈ-ಸಲು
ಪೂರೈಸಿ
ಪೂರೈ-ಸಿ-ಕೊ-ಳ್ಳಲು
ಪೂರೈ-ಸಿ-ಕೊ-ಳ್ಳು-ವ-ವ-ರೆಗೆ
ಪೂರೈ-ಸಿದ
ಪೂರೈ-ಸಿ-ದರು
ಪೂರೈ-ಸಿ-ದ-ಳೆಂ-ಬುದು
ಪೂರೈ-ಸಿ-ದ-ವರು
ಪೂರೈ-ಸಿ-ದ್ದ-ರಿಂದ
ಪೂರೈ-ಸು-ವಾಗ
ಪೂರ್ಣ
ಪೂರ್ಣ-ಗೊಂ-ಡಿ-ದ್ದವು
ಪೂರ್ಣ-ಗೊ-ಳಿ-ಸಲಿ
ಪೂರ್ಣ-ಗೊ-ಳಿ-ಸಿದ್ದೆ
ಪೂರ್ಣ-ಗೊ-ಳಿ-ಸು-ತ್ತಿದ್ದ
ಪೂರ್ಣ-ಗೊ-ಳ್ಳು-ವು-ದ-ರೊಂ-ದಿಗೆ
ಪೂರ್ಣ-ಜೀ-ವ-ನ-ವನ್ನು
ಪೂರ್ಣ-ತೆ-ಯನ್ನು
ಪೂರ್ಣ-ಪಾಠ
ಪೂರ್ಣ-ಯ್ಯನ
ಪೂರ್ಣ-ವಧಿ
ಪೂರ್ಣ-ವಾಗಿ
ಪೂರ್ಣ-ವಾದ
ಪೂರ್ಣಾ-ನು-ಗ್ರ-ಹದ
ಪೂರ್ಣಾ-ಯು-ಷ್ಯ-ವನ್ನು
ಪೂರ್ತಿ
ಪೂರ್ತಿಗೆ
ಪೂರ್ವ
ಪೂರ್ವ-ಕ-ವಾಗಿ
ಪೂರ್ವ-ಕ-ವಾದ
ಪೂರ್ವ-ಕಾ-ಲ-ದಿಂ-ದಲೂ
ಪೂರ್ವ-ಜ-ನ್ಮದ
ಪೂರ್ವ-ಜರು
ಪೂರ್ವ-ತ-ಪಸಾ
ಪೂರ್ವ-ದಲ್ಲಿ
ಪೂರ್ವ-ಪೀ-ಠಿ-ಕೆ-ಗ-ಳಿಂದ
ಪೂರ್ವ-ಭಾ-ಗದ
ಪೂರ್ವ-ಭಾ-ಗ-ದಲ್ಲಿ
ಪೂರ್ವ-ಮೀ-ಮಾಂ-ಸಾ-ಶಾ-ಸ್ತ್ರ-ವನ್ನು
ಪೂರ್ವ-ವ-ಯ-ಸ್ಸಿ-ನಲ್ಲಿ
ಪೂರ್ವ-ವಿ-ದ್ಯಾರ್ಥೀ
ಪೂರ್ವಾ-ಗ್ರಹ
ಪೂರ್ವಾ-ಪರ
ಪೂರ್ವಾ-ಭಿ-ಮು-ಖ-ವಾಗಿ
ಪೂರ್ವಾ-ಶ್ರ-ಮ-ದಲ್ಲಿ
ಪೂರ್ವಿ-ಕರ
ಪೂರ್ವಿ-ಕರು
ಪೃಥಿ-ವಿ-ಗಿಲ್ಲ
ಪೆಟ್ಟಾ-ಯಿತು
ಪೆಟ್ಟಿಗೆ
ಪೆಟ್ಟಿ-ಗೆ-ಯನ್ನು
ಪೆಟ್ಟು
ಪೆನ್ನು
ಪೈಕಿ
ಪೊಣಿಸಿ
ಪೋಣಿಸಿ
ಪೋಲೀ-ಸರ
ಪೋಲೀಸ್
ಪೋಷ-ಣೆ-ಯಲ್ಲಿ
ಪೋಷಿಣೀ
ಪೋಷಿ-ಣೀ-ಸ-ಭೆಯ
ಪೋಷಿ-ಸಿದ
ಪೋಸ್ಟ್
ಪೌಢಿಮೆ
ಪೌರಾ-ಣಿಕ
ಪೌರುಷಂ
ಪೌರೋ-ಹಿತ್ಯ
ಪೌರೋ-ಹಿ-ತ್ಯಕ್ಕೆ
ಪೌರೋ-ಹಿ-ತ್ಯ-ಗ-ಳಿಂದ
ಪೌರೋ-ಹಿ-ತ್ಯ-ವನ್ನು
ಪೌರೋ-ಹಿ-ತ್ಯವೂ
ಪೌರ್ವ-ದೇ-ಹಿ-ಕಮ್
ಪೌರ್ವಾ-ಪ-ರ್ಯದ
ಪೌಷ್ಟಿ-ಕ-ತೆಯ
ಪ್ಯೂನ್ಗ-ಳನ್ನು
ಪ್ಯೂನ್ಗ-ಳಿಗೆ
ಪ್ರಕ-ಟ-ಗೊಂ-ಡಿವೆ
ಪ್ರಕ-ಟ-ಣೆ-ಯಲ್ಲಿ
ಪ್ರಕ-ಟ-ನೆಗೆ
ಪ್ರಕ-ಟ-ವಾ-ಗಿತ್ತು
ಪ್ರಕ-ಟ-ವಾ-ಗಿ-ದೆ-ಯೆಂ-ದರೆ
ಪ್ರಕ-ಟ-ವಾ-ಗು-ತ್ತದೆ
ಪ್ರಕ-ಟ-ವಾ-ಗುವ
ಪ್ರಕ-ಟ-ವಾ-ದಾಗ
ಪ್ರಕ-ಟಿ-ಸಿ-ದ್ದಾರೆ
ಪ್ರಕ-ಟಿ-ಸು-ತ್ತಿ-ರು-ವುದು
ಪ್ರಕ-ಟಿ-ಸು-ವು-ದ-ರೊಂ-ದಿಗೆ
ಪ್ರಕ-ಟೀ-ಕ-ರೋತಿ
ಪ್ರಕ-ರಣ
ಪ್ರಕ-ರ-ಣ-ದಿಂದ
ಪ್ರಕ-ರ-ಣ-ವನ್ನು
ಪ್ರಕಾಂಡ
ಪ್ರಕಾರ
ಪ್ರಕಾ-ರ-ಗ-ಳಲ್ಲಿ
ಪ್ರಕಾ-ರ-ಗ-ಳ-ಲ್ಲಿಯೂ
ಪ್ರಕಾ-ರ-ಗಳು
ಪ್ರಕಾ-ರದ
ಪ್ರಕಾ-ಶಕ್ಕೆ
ಪ್ರಕಾ-ಶನ
ಪ್ರಕಾ-ಶ-ವನ್ನು
ಪ್ರಕಾ-ಶಿ-ಸು-ತ್ತಿ-ರಲಿ
ಪ್ರಕೃತ
ಪ್ರಕೃತಿ
ಪ್ರಕೃ-ತಿಯ
ಪ್ರಕೃ-ತಿ-ಯಲ್ಲಿ
ಪ್ರಕೃ-ತೋ-ಪ-ಯೋ-ಗಿ-ಯಾದ
ಪ್ರಕ್ರಿಯೆ
ಪ್ರಕ್ರಿ-ಯೆ-ಗ-ಳ-ನ್ನೆಲ್ಲಾ
ಪ್ರಕ್ರಿ-ಯೆಯ
ಪ್ರಕ್ರಿ-ಯೆ-ಯನ್ನು
ಪ್ರಕ್ರಿ-ಯೆ-ಯಲ್ಲೂ
ಪ್ರಖ-ರ-ತೆಗೂ
ಪ್ರಖ-ರ-ವಾ-ಗು-ತ್ತದೆ
ಪ್ರಖ-ರ-ವಾದ
ಪ್ರಖ್ಯಾತ
ಪ್ರಖ್ಯಾತಿ
ಪ್ರಚ-ಲಿ-ತ-ವಾ-ಗಿದ್ದ
ಪ್ರಚಾ-ರ-ಕರು
ಪ್ರಚಾ-ರಕ್ಕೆ
ಪ್ರಚಾ-ರ-ಪ್ರಿ-ಯ-ರಲ್ಲ
ಪ್ರಚೋ-ದನೆ
ಪ್ರಚೋ-ದಿ-ಸು-ತ್ತದೆ
ಪ್ರಜ-ನನ
ಪ್ರಜೆ-ಗ-ಳನ್ನು
ಪ್ರಜೆ-ಗ-ಳಿಗೆ
ಪ್ರಜೆ-ಗ-ಳಿ-ರುವ
ಪ್ರಜೆ-ಗಳು
ಪ್ರಜ್ಞಾ
ಪ್ರಜ್ಞಾ-ಜ್ಯೋ-ತಿಯ
ಪ್ರಜ್ಞಾ-ಪ-ರಾಧ
ಪ್ರಜ್ಞೆ
ಪ್ರಜ್ಞೆ-ಯದು
ಪ್ರಜ್ಞೆ-ಯಿ-ಲ್ಲದೇ
ಪ್ರಣಾ-ಮ-ಗ-ಳನ್ನು
ಪ್ರಣಾ-ಮ-ಗಳು
ಪ್ರತ-ನ್ಯಂತೇ
ಪ್ರತಿ
ಪ್ರತಿ-ಕ್ರಿ-ಯಿ-ಸುವ
ಪ್ರತಿ-ಕ್ರಿಯೆ
ಪ್ರತಿ-ತಿಂ-ಗಳು
ಪ್ರತಿ-ತ್ರ-ಯೋ-ದ-ಶಿಯ
ಪ್ರತಿ-ದಿ-ನದ
ಪ್ರತಿ-ದಿ-ನವೂ
ಪ್ರತಿ-ನಿತ್ಯ
ಪ್ರತಿ-ನಿ-ಧಿ-ಯಾಗಿ
ಪ್ರತಿ-ನಿ-ಧಿಸಿ
ಪ್ರತಿ-ನಿ-ಧಿ-ಸು-ತ್ತವೆ
ಪ್ರತಿ-ನಿ-ಧಿ-ಸು-ತ್ತಿದ್ದ
ಪ್ರತಿ-ಪಾ-ದನ
ಪ್ರತಿ-ಪಾ-ದ-ನೆ-ಯಲ್ಲಿ
ಪ್ರತಿ-ಪಾ-ದಿ-ತ-ವಾದ
ಪ್ರತಿ-ಪಾ-ದಿ-ಸಿದ
ಪ್ರತಿ-ಪಾ-ದಿ-ಸಿದ್ದು
ಪ್ರತಿ-ಪಾ-ದಿ-ಸುವ
ಪ್ರತಿ-ಪಾ-ದಿ-ಸು-ವ-ವರು
ಪ್ರತಿ-ಪ್ರ-ದೋ-ಷದ
ಪ್ರತಿ-ಫ-ಲದ
ಪ್ರತಿ-ಫ-ಲಾ-ಪೇಕ್ಷೆ
ಪ್ರತಿ-ಫ-ಲಾ-ಪೇ-ಕ್ಷೆ-ಯಿ-ಲ್ಲದ
ಪ್ರತಿ-ಫ-ಲಿ-ತ-ವಾ-ಗಿದೆ
ಪ್ರತಿ-ಬಾರಿ
ಪ್ರತಿ-ಭ-ಟನೆ
ಪ್ರತಿ-ಭ-ಟಿ-ಸ-ಬೇಕು
ಪ್ರತಿ-ಭ-ಟಿ-ಸಲು
ಪ್ರತಿ-ಭ-ಟಿ-ಸಿದ
ಪ್ರತಿ-ಭ-ಟಿ-ಸಿದೆ
ಪ್ರತಿ-ಭ-ಟಿ-ಸು-ತ್ತೇವೆ
ಪ್ರತಿ-ಭ-ಟಿ-ಸು-ವ-ವರು
ಪ್ರತಿ-ಭ-ಟಿ-ಸು-ವುದು
ಪ್ರತಿಭಾ
ಪ್ರತಿ-ಭಾ-ಕೌ-ಶ-ಲ್ಯ-ದಿಂ-ದಲೇ
ಪ್ರತಿ-ಭಾ-ನ್ವಿ-ತನೂ
ಪ್ರತಿ-ಭಾ-ಪ್ರ-ಕಾ-ಶಕ್ಕೆ
ಪ್ರತಿ-ಭಾ-ಪ್ರ-ಕಾ-ಶ-ವೆಂಬ
ಪ್ರತಿ-ಭಾ-ರತ್ನ
ಪ್ರತಿ-ಭಾ-ವಂ-ತ-ರ-ನ್ನೆಲ್ಲ
ಪ್ರತಿ-ಭಾ-ವಂ-ತ-ರಾದ
ಪ್ರತಿ-ಭಾ-ವಂ-ತರು
ಪ್ರತಿ-ಭಾ-ಶಾಲಿ
ಪ್ರತಿ-ಭಾ-ಸಂ-ಪನ್ನ
ಪ್ರತಿ-ಭಾ-ಸಂ-ಪ-ನ್ನ-ನಾ-ಗಿದ್ದ
ಪ್ರತಿ-ಭಾ-ಸಂ-ಪ-ನ್ನ-ರಾದ
ಪ್ರತಿ-ಭಾ-ಸಂ-ಪ-ನ್ನರು
ಪ್ರತಿಭೆ
ಪ್ರತಿ-ಭೆಗೆ
ಪ್ರತಿ-ಭೆಯ
ಪ್ರತಿ-ಭೆ-ಯನ್ನು
ಪ್ರತಿ-ಭೆ-ಯಾದ
ಪ್ರತಿ-ಭೆ-ಯು-ಳ್ಳ-ವರು
ಪ್ರತಿ-ಭೆ-ಯೆಂ-ದರೆ
ಪ್ರತಿ-ಯಾಗಿ
ಪ್ರತಿ-ಯೊಂ-ದನ್ನು
ಪ್ರತಿ-ಯೊಂ-ದ-ರಲ್ಲೂ
ಪ್ರತಿ-ಯೊಂದು
ಪ್ರತಿ-ಯೊಬ್ಬ
ಪ್ರತಿ-ಯೊ-ಬ್ಬನೂ
ಪ್ರತಿ-ಯೊ-ಬ್ಬ-ರಿಗೂ
ಪ್ರತಿ-ಯೊ-ಬ್ಬರು
ಪ್ರತಿ-ಯೋ-ಗಿ-ತಾ-ದಲ್ಲಿ
ಪ್ರತಿ-ಯೋ-ಗಿ-ತೆ-ಯನ್ನು
ಪ್ರತಿ-ವರ್ಷ
ಪ್ರತಿ-ವ-ರ್ಷದ
ಪ್ರತಿ-ವಾ-ದಿಯು
ಪ್ರತಿ-ಷ್ಠಾ-ಪ-ಯಿ-ತವ್ಯ
ಪ್ರತಿ-ಷ್ಠಾ-ಪಿ-ಸಲು
ಪ್ರತಿ-ಷ್ಠಾ-ಪಿಸೇ
ಪ್ರತಿ-ಷ್ಠೆಗೆ
ಪ್ರತಿ-ಷ್ಠೆ-ಯಲ್ಲ
ಪ್ರತಿ-ಷ್ಠೆ-ಕುಂ-ಭಾ-ಭಿ-ಷೇ-ಕಕ್ಕೆ
ಪ್ರತಿ-ಷ್ಥೆ-ಮಾ-ಡ-ಬೇಕು
ಪ್ರತಿ-ಸ್ಪಂ-ದಿ-ಸುವ
ಪ್ರತೀ
ಪ್ರತ್ಯ-ಕ್ಷ-ದಲ್ಲೆ
ಪ್ರತ್ಯ-ಕ್ಷ-ವಾಗಿ
ಪ್ರತ್ಯಾ-ಚ-ಕ್ಷೀತ
ಪ್ರತ್ಯು-ತ್ಪನ್ನ
ಪ್ರತ್ಯು-ತ್ಪ-ನ್ನ-ಮ-ತಿ-ತ್ವವೂ
ಪ್ರತ್ಯು-ಪ-ಕಾ-ರ-ವನ್ನು
ಪ್ರತ್ಯೇಕ
ಪ್ರತ್ಯೇ-ಕ-ವಾಗಿ
ಪ್ರಥಮ
ಪ್ರಥ-ಮ-ಬಾ-ರಿಗೆ
ಪ್ರಥ-ಮ-ಸ್ಥಾ-ನ-ವನ್ನು
ಪ್ರಥಮಾ
ಪ್ರಥ-ಮ-ಕಾವ್ಯ
ಪ್ರದ-ಕ್ಷಿ-ಣ-ಕ್ರ-ಮ-ದಲ್ಲಿ
ಪ್ರದ-ರ್ಶಕ
ಪ್ರದ-ರ್ಶನ
ಪ್ರದ-ರ್ಶ-ನ-ಕ್ಕಿಂತ
ಪ್ರದ-ರ್ಶ-ನದ
ಪ್ರದ-ರ್ಶಿ-ಸ-ಬ-ಹು-ದಿತ್ತು
ಪ್ರದ-ರ್ಶಿ-ಸ-ಲಾ-ಗಿತ್ತು
ಪ್ರದ-ರ್ಶಿ-ಸು-ವುದು
ಪ್ರದಾನ
ಪ್ರದೇಶ
ಪ್ರದೇ-ಶ-ಗಳ
ಪ್ರದೇ-ಶ-ಗ-ಳಿಂದ
ಪ್ರದೋಷ
ಪ್ರದೋ-ಷ-ಸಂ-ಘಕ್ಕೆ
ಪ್ರದೋ-ಷ-ಸಂ-ಘದ
ಪ್ರದೋ-ಷೋ-ಪ-ನ್ಯಾ-ಸ-ಮಾ-ಲೆ-ಯನ್ನು
ಪ್ರಧಾನ
ಪ್ರಧಾ-ನ-ವಾಗಿ
ಪ್ರಧಾ-ನ-ವಾ-ಗಿತ್ತು
ಪ್ರಧಾ-ನ-ವಾ-ದದ್ದು
ಪ್ರಧಾ-ನ-ವಾ-ದ-ದ್ದೆಂದು
ಪ್ರಧಾ-ನ-ವಾ-ದು-ದ-ರಿಂದ
ಪ್ರಧಾ-ನ-ವಾ-ದುದು
ಪ್ರಪಂಚ
ಪ್ರಪಂ-ಚದ
ಪ್ರಪಂ-ಚ-ದಲ್ಲಿ
ಪ್ರಪಂ-ಚ-ವನ್ನೇ
ಪ್ರಪ-ರಿ-ತಾಪ
ಪ್ರಪಾ-ತಕ್ಕೆ
ಪ್ರಪ್ರ-ಥ-ಮ-ವಾಗಿ
ಪ್ರಬಂಧ
ಪ್ರಬ-ಲ-ರನ್ನು
ಪ್ರಬ-ಲ-ವೆಂದು
ಪ್ರಬು-ದ್ಧ-ತೆ-ಗಳು
ಪ್ರಬೋ-ಧ-ಕ-ಶ್ಚಿ-ರಂ-ಜೀ-ಯಾತ್
ಪ್ರಭ
ಪ್ರಭಂಧ
ಪ್ರಭಾವ
ಪ್ರಭಾ-ವ-ದಿಂದ
ಪ್ರಭಾ-ವ-ಲ-ಯದ
ಪ್ರಭಾ-ವ-ಲ-ಯ-ವು-ಳ್ಳ-ವನು
ಪ್ರಭಾ-ವ-ವನ್ನು
ಪ್ರಭೆ
ಪ್ರಭೇದ
ಪ್ರಭೇ-ದ-ಗ-ಳನ್ನು
ಪ್ರಭೇ-ದ-ಗಳು
ಪ್ರಭೇ-ದದ
ಪ್ರಮಾಣ
ಪ್ರಮುಖ
ಪ್ರಮು-ಖವೂ
ಪ್ರಮೋದಾ
ಪ್ರಯ-ಚ್ಛತಿ
ಪ್ರಯತ್ನ
ಪ್ರಯ-ತ್ನಕ್ಕೆ
ಪ್ರಯ-ತ್ನ-ಗ-ಳಿಗೆ
ಪ್ರಯ-ತ್ನ-ಗಳು
ಪ್ರಯ-ತ್ನ-ದಲ್ಲಿ
ಪ್ರಯ-ತ್ನ-ದಿಂದ
ಪ್ರಯ-ತ್ನ-ದೊಂ-ದಿಗೆ
ಪ್ರಯ-ತ್ನ-ಪಟ್ಟು
ಪ್ರಯ-ತ್ನ-ಪ-ಡು-ತ್ತಿ-ದ್ದೇನೆ
ಪ್ರಯ-ತ್ನ-ಪೂ-ರ್ವ-ಕ-ವಾಗಿ
ಪ್ರಯ-ತ್ನ-ಮಾ-ತ್ರ-ದಿಂ-ದಲೇ
ಪ್ರಯ-ತ್ನ-ವನ್ನು
ಪ್ರಯ-ತ್ನ-ವಷ್ಟೆ
ಪ್ರಯ-ತ್ನ-ವಾಗಿ
ಪ್ರಯ-ತ್ನ-ವಿ-ಲ್ಲ-ದಿ-ದ್ದರೆ
ಪ್ರಯ-ತ್ನವು
ಪ್ರಯ-ತ್ನಿ-ಸಿ-ದರೂ
ಪ್ರಯ-ತ್ನಿ-ಸಿ-ದ್ದ-ರಿಂದ
ಪ್ರಯ-ತ್ನೇನ
ಪ್ರಯಾಣ
ಪ್ರಯಾ-ಣದ
ಪ್ರಯಾ-ಣಿಸಿ
ಪ್ರಯೋಗ
ಪ್ರಯೋ-ಗ-ವನ್ನು
ಪ್ರಯೋ-ಗಿ-ಸಿದ
ಪ್ರಯೋ-ಜನ
ಪ್ರಯೋ-ಜ-ನ-ವನ್ನು
ಪ್ರಯೋ-ಜ-ನ-ವನ್ನೂ
ಪ್ರಯೋ-ಜ-ನ-ವಿಲ್ಲ
ಪ್ರವ-ಚನ
ಪ್ರವ-ಚ-ನಕ್ಕೂ
ಪ್ರವ-ಚ-ನಕ್ಕೆ
ಪ್ರವ-ಚ-ನ-ಗ-ಳನ್ನು
ಪ್ರವ-ಚ-ನ-ಗ-ಳನ್ನೂ
ಪ್ರವ-ಚ-ನ-ಗಳು
ಪ್ರವ-ಚ-ನದ
ಪ್ರವ-ಚ-ನ-ವನ್ನು
ಪ್ರವ-ದಂತಿ
ಪ್ರವ-ಹಿ-ಸಿ-ಹುದು
ಪ್ರವ-ಹಿ-ಸುವ
ಪ್ರವಾಸ
ಪ್ರವಾ-ಸಕ್ಕೆ
ಪ್ರವಾಹ
ಪ್ರವಾ-ಹ-ವಾಗಿ
ಪ್ರವಿ-ಚ-ಲಂತಿ
ಪ್ರವೀಣ
ಪ್ರವೀ-ಣ-ರಲ್ಲ
ಪ್ರವೃ-ತ್ತ-ರಾ-ಗಿ-ರು-ವ-ವರೇ
ಪ್ರವೃ-ತ್ತ-ರಾ-ದರು
ಪ್ರವೃತ್ತಿ
ಪ್ರವೃ-ತ್ತಿಗೆ
ಪ್ರವೃ-ತ್ತಿ-ಯಿಂ-ದಲ್ಲ
ಪ್ರವೇಶ
ಪ್ರವೇ-ಶಕ್ಕೆ
ಪ್ರವೇ-ಶ-ಪ-ಡೆ-ಯಲು
ಪ್ರವೇ-ಶ-ವನ್ನು
ಪ್ರವೇ-ಶ-ವಾ-ಗು-ವ-ವ-ರೆಗೆ
ಪ್ರವೇ-ಶ-ವಾ-ಯಿತು
ಪ್ರವೇ-ಶಾ-ತಿ-ಗಾಗಿ
ಪ್ರವೇ-ಶಾ-ವಧಿ
ಪ್ರವೇ-ಶಿ-ಸ-ಬೇ-ಕಾ-ದಾಗ
ಪ್ರವೇ-ಶಿಸಿ
ಪ್ರವೇ-ಶಿ-ಸಿದ
ಪ್ರವೇ-ಶಿ-ಸಿ-ದಾಗ
ಪ್ರವೇ-ಶಿ-ಸಿದೆ
ಪ್ರವೇ-ಶಿ-ಸು-ತ್ತಿ-ದ್ದಂತೇ
ಪ್ರಶಂ-ಸಾರ್ಹ
ಪ್ರಶಂ-ಸಿ-ದ-ವರು
ಪ್ರಶಂ-ಸೆ-ಗ-ಳನ್ನು
ಪ್ರಶಂ-ಸೆಗೆ
ಪ್ರಶಸ್ತ
ಪ್ರಶ-ಸ್ತ-ವಾ-ಗಿತ್ತು
ಪ್ರಶ-ಸ್ತವೂ
ಪ್ರಶಸ್ತಾ
ಪ್ರಶಸ್ತಿ
ಪ್ರಶಾಂತ
ಪ್ರಶಾಂ-ತ-ಭಾ-ವ-ದಿಂದ
ಪ್ರಶಾಂತಿ
ಪ್ರಶಾಂತ್
ಪ್ರಶ್ನಾ-ನು-ಪ್ರ-ಶ್ನೆ-ಗ-ಳನ್ನು
ಪ್ರಶ್ನಿ-ಸಲು
ಪ್ರಶ್ನಿಸಿ
ಪ್ರಶ್ನಿ-ಸಿದ
ಪ್ರಶ್ನಿ-ಸಿ-ದರು
ಪ್ರಶ್ನೆ
ಪ್ರಶ್ನೆ-ಗ-ಳನ್ನು
ಪ್ರಶ್ನೆ-ಗ-ಳಿಗೆ
ಪ್ರಶ್ನೆ-ಗಳು
ಪ್ರಶ್ನೆಗೆ
ಪ್ರಶ್ನೆಯ
ಪ್ರಶ್ನೆ-ಯನ್ನು
ಪ್ರಶ್ನೋ-ತ್ತ-ರ-ದಲ್ಲಿ
ಪ್ರಸಂಗ
ಪ್ರಸಂ-ಗ-ಗ-ಳನ್ನು
ಪ್ರಸಂ-ಗ-ವಾ-ದ್ದ-ರಿಂದ
ಪ್ರಸವ
ಪ್ರಸ-ವಿಸಿ
ಪ್ರಸಾ-ರ-ಗೊ-ಳಿ-ಸು-ತ್ತಿ-ದ್ದೆವು
ಪ್ರಸಾ-ರ-ದಲ್ಲಿ
ಪ್ರಸಾ-ರ-ಪ್ರ-ಚಾ-ರ-ವಾ-ಗು-ತ್ತ-ದಲ್ಲ
ಪ್ರಸಿ-ದ್ದ-ರಾದ
ಪ್ರಸಿದ್ಧ
ಪ್ರಸಿ-ದ್ಧ-ವಾ-ಗಿದೆ
ಪ್ರಸಿ-ದ್ಧ-ವಾದ
ಪ್ರಸಿ-ದ್ಧ-ವಿ-ರುವ
ಪ್ರಸಿದ್ಧಿ
ಪ್ರಸಿ-ದ್ಧಿ-ಗ-ಳೆಲ್ಲ
ಪ್ರಸಿ-ದ್ಧಿ-ಯನ್ನು
ಪ್ರಸಿ-ದ್ಧಿ-ಯಿದೆ
ಪ್ರಸಿ-ಧ-ವಾ-ಗಿದ್ದ
ಪ್ರಸ್ತಾ-ಪಿ-ಸುತ್ತಾ
ಪ್ರಸ್ತಾ-ಪಿ-ಸು-ತ್ತೇನೆ
ಪ್ರಸ್ತಾ-ವ-ನೆ-ಯನ್ನು
ಪ್ರಸ್ತುತ
ಪ್ರಾಂತ್ಯ-ವನ್ನು
ಪ್ರಾಂಶು-ಪಾಲ
ಪ್ರಾಂಶು-ಪಾ-ಲರ
ಪ್ರಾಂಶು-ಪಾ-ಲ-ರಿಗೂ
ಪ್ರಾಂಶು-ಪಾ-ಲರು
ಪ್ರಾಂಶು-ಪಾ-ಲರೇ
ಪ್ರಾಂಶೌ
ಪ್ರಾಕಾ-ರದ
ಪ್ರಾಕೃ-ತಿಕ
ಪ್ರಾಚಾ-ರ್ಯರು
ಪ್ರಾಚೀನ
ಪ್ರಾಚೀ-ನ-ತೆಗೆ
ಪ್ರಾಚೀ-ನ-ನ್ಯಾಯ
ಪ್ರಾಚೀ-ನ-ನ್ಯಾ-ಯ-ಶಾ-ಸ್ತ್ರದ
ಪ್ರಾಚೀ-ನ-ವಾದ
ಪ್ರಾಚೀ-ನ-ಶಾ-ಸ್ತ್ರ-ಪ-ರಂ-ಪ-ರೆಯ
ಪ್ರಾಚೀ-ನ-ಶಾ-ಸ್ತ್ರ-ಪ-ರಂ-ಪ-ರೆಯು
ಪ್ರಾಚೀ-ನ-ನ-ವೀನ
ಪ್ರಾಚ್ಯ-ವಿ-ದ್ಯಾ-ಸಂ-ಶೋ-ಧ-ನಾ-ಲಯ
ಪ್ರಾಜ್ಞ-ರಾದ
ಪ್ರಾಜ್ಞರು
ಪ್ರಾಣ-ವನ್ನು
ಪ್ರಾಣ-ವು-ಳಿ-ಯುವ
ಪ್ರಾಣಾ-ಪಾ-ನ-ಯೋ-ರ್ಗ್ರಂ-ಥಿ-ರ-ಜ-ಪೇ-ತ್ಯ-ಭಿ-ಧೀ-ಯತೇ
ಪ್ರಾಣಿ
ಪ್ರಾಣಿ-ಗಳ
ಪ್ರಾಣಿ-ನಾಂ
ಪ್ರಾಣಿ-ವಿ-ಜ್ಞಾನ
ಪ್ರಾಣಿ-ಶಾಸ್ತ್ರ
ಪ್ರಾತಃ-ಕಾ-ಲ-ದಲ್ಲಿ
ಪ್ರಾತಃ-ಕಾ-ಲ-ದ-ಲ್ಲೆದ್ದು
ಪ್ರಾತಃ-ಪೂ-ಜೆ-ಯಿಂ-ದಲೇ
ಪ್ರಾಥ-ಮಿಕ
ಪ್ರಾಧಾನ್ಯ
ಪ್ರಾಧ್ಯಾ-ಪಕ
ಪ್ರಾಧ್ಯಾ-ಪ-ಕ-ನಾಗಿ
ಪ್ರಾಧ್ಯಾ-ಪ-ಕರ
ಪ್ರಾಧ್ಯಾ-ಪ-ಕ-ರಾಗಿ
ಪ್ರಾಧ್ಯಾ-ಪ-ಕ-ರಾ-ಗಿದ್ದ
ಪ್ರಾಧ್ಯಾ-ಪ-ಕ-ರಾ-ಗಿ-ದ್ದರು
ಪ್ರಾಧ್ಯಾ-ಪ-ಕ-ರಾ-ಗಿ-ದ್ದರೂ
ಪ್ರಾಧ್ಯಾ-ಪ-ಕ-ರಾ-ಗಿ-ರುವ
ಪ್ರಾಧ್ಯಾ-ಪ-ಕ-ರಾದ
ಪ್ರಾಧ್ಯಾ-ಪ-ಕ-ರಾ-ದಿರಿ
ಪ್ರಾಧ್ಯಾ-ಪ-ಕರು
ಪ್ರಾಧ್ಯಾ-ಪ-ಕ-ರೆಂದು
ಪ್ರಾಧ್ಯಾ-ಪ-ಕ-ರೆಲ್ಲ
ಪ್ರಾಧ್ಯಾ-ಪ-ಕಿ-ಯಾಗಿ
ಪ್ರಾಪ್ತ
ಪ್ರಾಪ್ತ-ವಾ-ಗಿದೆ
ಪ್ರಾಪ್ತ-ವಾ-ಗಿ-ಲ್ಲ-ದಿ-ದ್ದರೂ
ಪ್ರಾಪ್ತ-ವಾ-ಗು-ತ್ತಿತ್ತು
ಪ್ರಾಪ್ತ-ವಾ-ದದ್ದು
ಪ್ರಾಪ್ತಿ-ಯಾ-ಗ-ಬ-ಹುದು
ಪ್ರಾಪ್ತಿ-ಯಾಗಿ
ಪ್ರಾಮಾ-ಣಿಕ
ಪ್ರಾಮಾ-ಣಿ-ಕತೆ
ಪ್ರಾಮಾ-ಣಿ-ಕ-ತೆಗೆ
ಪ್ರಾಮಾ-ಣಿ-ಕ-ವಾಗಿ
ಪ್ರಾಮಾ-ಣಿ-ಕ-ವಾದ
ಪ್ರಾಮು-ಖ್ಯವೂ
ಪ್ರಾಯ-ದಲ್ಲಿ
ಪ್ರಾಯೋ-ಗಿಕ
ಪ್ರಾಯೋ-ಗಿ-ಕ-ವಾಗಿ
ಪ್ರಾರಂಭ
ಪ್ರಾರಂ-ಭದ
ಪ್ರಾರಂ-ಭ-ವಾಗಿ
ಪ್ರಾರಂ-ಭ-ವಾ-ಗಿತ್ತು
ಪ್ರಾರಂ-ಭ-ವಾ-ಗಿದ್ದ
ಪ್ರಾರಂ-ಭ-ವಾ-ಗು-ತ್ತಿತ್ತು
ಪ್ರಾರಂ-ಭ-ವಾದ
ಪ್ರಾರಂ-ಭ-ವಾ-ದಾಗ
ಪ್ರಾರಂ-ಭ-ವಾ-ಯಿತು
ಪ್ರಾರಂ-ಭಿಕ
ಪ್ರಾರಂ-ಭಿಸಿ
ಪ್ರಾರಂ-ಭಿ-ಸಿತ್ತು
ಪ್ರಾರಂ-ಭಿ-ಸಿ-ದರು
ಪ್ರಾರಂ-ಭಿ-ಸಿ-ದೆವು
ಪ್ರಾರಂ-ಭಿ-ಸಿ-ದ್ದಾರೆ
ಪ್ರಾರಂ-ಭಿ-ಸಿದ್ದು
ಪ್ರಾರಂ-ಭಿಸು
ಪ್ರಾರಂ-ಭಿ-ಸು-ತ್ತಿ-ದ್ದರು
ಪ್ರಾರಂ-ಭಿ-ಸುವ
ಪ್ರಾರಭ್ಧ
ಪ್ರಾರ-ಭ್ಧ-ವೆಂಬ
ಪ್ರಾರ್ಥನಾ
ಪ್ರಾರ್ಥನೆ
ಪ್ರಾರ್ಥಿಸಿ
ಪ್ರಾರ್ಥಿ-ಸಿ-ಕೊಂಡೆ
ಪ್ರಾರ್ಥಿ-ಸು-ತ್ತೇನೆ
ಪ್ರಾರ್ಥಿ-ಸು-ವು-ದೇ-ನೆಂ-ದರೆ
ಪ್ರಾರ್ಥಿ-ಸುವೆ
ಪ್ರಾರ್ಥಿ-ಸೋಣ
ಪ್ರಾವಿ-ಣ್ಯ-ತೆ-ಯಿಂ-ದಾಗಿ
ಪ್ರಾವೀ-ಣ್ಯ-ಪ-ಡೆದ
ಪ್ರಾವೀ-ಣ್ಯ-ವನ್ನು
ಪ್ರಾಶಸ್ತ್ಯ
ಪ್ರಾಸ-ಬ-ದ್ಧ-ವಾದ
ಪ್ರಾಸಾ-ದದ
ಪ್ರಾಸಾ-ದ-ದಲ್ಲಿ
ಪ್ರಾಹು-ರ-ನಂತಂ
ಪ್ರಿಂಟ-ರ್ಸ್
ಪ್ರಿಯ
ಪ್ರಿಯಂ
ಪ್ರಿಯರೂ
ಪ್ರಿಯ-ವಾದ
ಪ್ರಿಯ-ವಾ-ದದ್ದು
ಪ್ರೀತಿ
ಪ್ರೀತಿ-ಗಳು
ಪ್ರೀತಿ-ಪಾ-ತ್ರರು
ಪ್ರೀತಿಯ
ಪ್ರೀತಿ-ಯನ್ನು
ಪ್ರೀತಿ-ಯಿಂದ
ಪ್ರೀತಿ-ಯಿಂ-ದಲೇ
ಪ್ರೀತಿ-ಯೆಂ-ದರೆ
ಪ್ರೀತಿಯೋ
ಪ್ರೀತಿ-ಸುವ
ಪ್ರೀತ್ಯಾ-ದ-ರ-ಗ-ಳಿಂದ
ಪ್ರೀತ್ಯಾ-ದ-ರ-ಗಳೂ
ಪ್ರೆಮ
ಪ್ರೇಕ್ಷ-ಕ-ನಾ-ಗಿ-ರು-ತ್ತಿದ್ದ
ಪ್ರೇಕ್ಷ-ಕರ
ಪ್ರೇಮ
ಪ್ರೇರಣೆ
ಪ್ರೇರ-ಣೆ-ಯನ್ನು
ಪ್ರೇರ-ಣೆ-ಯಾ-ಗಿ-ದ್ದರು
ಪ್ರೇರ-ಣೆ-ಯಾ-ಗಿ-ದ್ದಾರೆ
ಪ್ರೇರ-ಣೆ-ಯಾ-ದದ್ದು
ಪ್ರೇರ-ಣೆ-ಯಿಂದ
ಪ್ರೇರ-ಣೆಯೂ
ಪ್ರೇರಿ-ತ-ರಾದ
ಪ್ರೇರೆ-ಣೆ-ಯಂತೆ
ಪ್ರೇರೇ-ಪಣೆ
ಪ್ರೇರೇ-ಪಿಸಿ
ಪ್ರೇರೇ-ಪಿ-ಸಿ-ದ್ದೇನೆ
ಪ್ರೇರೇ-ಪಿ-ಸು-ತ್ತಿ-ದ್ದರು
ಪ್ರೈವೇ-ಟಾಗಿ
ಪ್ರೊ
ಪ್ರೊಕೆ-ವಿ-ಅ-ರ್ಕ-ನಾಥ
ಪ್ರೊಕ್ತೋ
ಪ್ರೋಕ್ತೋ
ಪ್ರೋತ್ಸಾಹ
ಪ್ರೋತ್ಸಾ-ಹ-ದಿಂ-ದಲೇ
ಪ್ರೋತ್ಸಾ-ಹಿಸಿ
ಪ್ರೋತ್ಸಾ-ಹಿ-ಸು-ತ್ತಿ-ದ್ದರು
ಪ್ರೋತ್ಸಾ-ಹಿ-ಸುವ
ಪ್ರೋತ್ಸಾ-ಹಿ-ಸು-ವುದು
ಪ್ರೌಢ
ಪ್ರೌಢ-ಪಾ-ಠ-ವಾದ
ಪ್ರೌಢ-ವಾಗಿ
ಪ್ರೌಢ-ಶಾಲಾ
ಪ್ರೌಢ-ಶಾಲೆ
ಪ್ರೌಢ-ಶಾ-ಲೆಗೆ
ಪ್ರೌಢ-ಶಾ-ಲೆ-ಯಲ್ಲಿ
ಪ್ರೌಢ-ಶಿ-ಕ್ಷಣ
ಪ್ರೌಢ-ಶಿ-ಕ್ಷ-ಣದ
ಪ್ರೌಢ-ಶಿ-ಕ್ಷ-ಣ-ವನ್ನು
ಪ್ರೌಢ-ಶಿ-ಕ್ಷ-ಣಾ-ನಂ-ತರ
ಪ್ರೌಢಿಮೆ
ಪ್ರತಿ-ಭೆಯ
ಫಲ
ಫಲಂ
ಫಲಂತಿ
ಫಲ-ಕಾ-ರಿ-ಯಾ-ಗ-ಲಿಲ್ಲ
ಫಲ-ಕ್ಕಾಗಿ
ಫಲಕ್ಕೆ
ಫಲತಿ
ಫಲದ
ಫಲ-ದಿಂದ
ಫಲ-ನಿ-ಶ್ಚ-ಯ-ದಲ್ಲಿ
ಫಲ-ಭೋ-ಗಕ್ಕೆ
ಫಲ-ರೂಪ
ಫಲ-ರೂ-ಪ-ದಲ್ಲಿ
ಫಲ-ವನ್ನು
ಫಲ-ವಾಗಿ
ಫಲವೇ
ಫಲ-ಸಿ-ದ್ದಿ-ಯನ್ನು
ಫಲ-ಸಿದ್ಧಿ
ಫಲ-ಸಿ-ದ್ಧಿ-ಯನ್ನು
ಫಲ-ಸಿ-ದ್ಧಿ-ಯಾ-ಗು-ತ್ತ-ದೆ-ಯೆಂದು
ಫಲಾ-ಪೇ-ಕ್ಷೆ-ಯಿ-ಲ್ಲದೇ
ಫಲಿ-ತಾಂಶ
ಫಲಿ-ತಾಂ-ಶವೇ
ಫಲಿಸಿ
ಫಲಿ-ಸು-ತ್ತವೆ
ಫಸ-ಲನ್ನು
ಫ಼ಸ್ಟ್
ಫ಼ೆಬ್ರ-ವರಿ
ಫಾರಿನ್
ಫೀಸ್
ಫೆಬ್ರ-ವರಿ
ಫೆಬ್ರ-ವ-ರಿ-ವ-ರೆಗೆ
ಫೆಬ್ರು-ವರಿ
ಫೋಟೋ
ಫೋಷಿಣೀ
ಫ್ರಾಯ್ಡ್
ಬಂಗಾಳ
ಬಂತು
ಬಂದ
ಬಂದಂ-ತಹ
ಬಂದಂತೆ
ಬಂದ-ಕೂ-ಡಲೇ
ಬಂದ-ದ್ದನ್ನು
ಬಂದದ್ದು
ಬಂದದ್ದೇ
ಬಂದನು
ಬಂದರು
ಬಂದರೂ
ಬಂದರೆ
ಬಂದ-ರೆಂ-ದರೆ
ಬಂದ-ವನು
ಬಂದ-ವರ
ಬಂದ-ವ-ರಿಗೆ
ಬಂದ-ವ-ರಿ-ಗೆಲ್ಲ
ಬಂದ-ವರು
ಬಂದ-ವ-ರೆಲ್ಲ
ಬಂದವು
ಬಂದಾಗ
ಬಂದಾ-ಗ-ಲೆಲ್ಲ
ಬಂದಿ
ಬಂದಿತ್ತು
ಬಂದಿದೆ
ಬಂದಿ-ದ್ದನ್ನು
ಬಂದಿ-ದ್ದರು
ಬಂದಿ-ದ್ದರೂ
ಬಂದಿ-ದ್ದಳು
ಬಂದಿ-ದ್ದಾನೆ
ಬಂದಿ-ದ್ದಾರೆ
ಬಂದಿ-ದ್ದಾಳೆ
ಬಂದಿ-ದ್ದಿದೆ
ಬಂದಿದ್ದು
ಬಂದಿದ್ದೇ
ಬಂದಿ-ದ್ದೇನೆ
ಬಂದಿ-ರ-ಬ-ಹುದು
ಬಂದಿ-ರ-ಬೇಕು
ಬಂದಿ-ರ-ಲೂ-ಬ-ಹುದು
ಬಂದಿರಿ
ಬಂದಿ-ರು-ತ್ತಾರೆ
ಬಂದಿ-ರು-ವುದು
ಬಂದಿ-ಳಿದ
ಬಂದಿ-ಳಿ-ದಾಗಾ
ಬಂದಿ-ಳಿ-ದಿದ್ದೆ
ಬಂದಿ-ಳಿ-ದೆನು
ಬಂದು
ಬಂದು-ಬಿಟ್ಟ
ಬಂದು-ಬಿ-ಡು-ತ್ತಿ-ದ್ದರು
ಬಂದುವು
ಬಂದು-ಹೋ-ಗು-ತ್ತಿದ್ದೆ
ಬಂದೆ
ಬಂದೊ-ಡನೆ
ಬಂದೊ-ದ-ಗಿದ
ಬಂದೊ-ದ-ಗಿದೆ
ಬಂಧ
ಬಂಧಿಸಿ
ಬಂಧು
ಬಂಧು-ಗ-ಳಲ್ಲೂ
ಬಂಧು-ವ-ರ್ಗದ
ಬಂಧು-ವ-ರ್ಗ-ದ-ವರು
ಬಂಧು-ವಿ-ನಂತೆ
ಬಗೆ-ಗಿನ
ಬಗೆಗೂ
ಬಗೆಗೆ
ಬಗೆದೆ
ಬಗೆಯ
ಬಗೆ-ಯಲ್ಲಿ
ಬಗೆ-ಯು-ತ್ತೇನೆ
ಬಗೆವ
ಬಗೆ-ಹ-ರಿ-ಸುತ್ತಾ
ಬಗೆ-ಹ-ರಿ-ಸುವ
ಬಗ್ಗೆ
ಬಗ್ಗೆ-ಯಂತೂ
ಬಗ್ಗೆಯೂ
ಬಟ್ಟ-ಲಿ-ನಲ್ಲಿ
ಬಟ್ಟೆ-ಗೊ-ಡ್ಡಿದ
ಬಟ್ಟೆ-ಯಿ-ಲ್ಲ-ಚಾ-ಪೆಯ
ಬಡ-ತ-ನದ
ಬಡ-ವ-ಬ-ಲ್ಲಿದ
ಬಡಾ-ವ-ಣೆ-ಯ-ಲ್ಲಿ-ರುವ
ಬಡಿತ
ಬಡಿಸಿ
ಬಡಿ-ಸುವ
ಬಣ್ಣ-ಕ-ಟ್ಟಲು
ಬದ-ಲಾಗಿ
ಬದ-ಲಾ-ಗು-ತ್ತದೆ
ಬದ-ಲಾ-ದರೆ
ಬದ-ಲಾ-ದವು
ಬದ-ಲಾ-ಯಿ-ಸಿದ್ದ
ಬದ-ಲಾ-ವ-ಣೆ-ಗಳ
ಬದ-ಲಾ-ವ-ಣೆ-ಗ-ಳಾ-ಗಿವೆ
ಬದ-ಲಾ-ವ-ಣೆ-ಯಾಗಿ
ಬದಲು
ಬದಿ-ಯಲ್ಲಿ
ಬದು-ಕನ್ನ
ಬದು-ಕನ್ನು
ಬದು-ಕಾ-ಗಿತ್ತು
ಬದುಕಿ
ಬದು-ಕಿ-ಗಾಗಿ
ಬದು-ಕಿಗೆ
ಬದು-ಕಿ-ತ್ತಿ-ಲ್ಲ-ವೆಂ-ಬುದು
ಬದು-ಕಿ-ದ-ವರು
ಬದು-ಕಿನ
ಬದು-ಕಿ-ನಲ್ಲಿ
ಬದುಕು
ಬದು-ಕು-ಕ-ಟ್ಟಿ-ಕೊಂಡ
ಬದು-ಕು-ತ್ತಿ-ದ್ದೆವು
ಬದು-ಕು-ತ್ತಿ-ರು-ವುದು
ಬದ್ಧ-ವಾಗಿ
ಬದ್ಧ-ವಾದ
ಬನು-ಮಯ್ಯ
ಬನು-ಮ-ಯ್ಯ-ನ-ವರ
ಬನ್ನಿ
ಬಯಕೆ
ಬಯ-ಕೆಯ
ಬಯ-ಕೆ-ಯನ್ನು
ಬಯ-ಕೆ-ಯಿಂದ
ಬಯ-ಲಿ-ಗೆ-ಳೆ-ಯಲು
ಬಯ-ಸದೇ
ಬಯ-ಸ-ಬೇ-ಕೆಂದು
ಬಯ-ಸ-ಲಿಲ್ಲ
ಬಯಸಿ
ಬಯ-ಸಿದ
ಬಯ-ಸಿ-ದರೆ
ಬಯ-ಸಿ-ದ-ವ-ರಲ್ಲ
ಬಯ-ಸಿ-ದ-ವರು
ಬಯ-ಸಿದ್ದು
ಬಯ-ಸು-ತ್ತಿ-ದ್ದರು
ಬಯ-ಸು-ತ್ತೇನೆ
ಬಯ-ಸುವ
ಬಯ-ಸು-ವಾಗ
ಬಯು-ಸುವ
ಬರ-ದಂತೆ
ಬರ-ದ-ವರು
ಬರದೆ
ಬರ-ಬ-ಹು-ದಾದ
ಬರ-ಬ-ಹುದು
ಬರ-ಬೇ-ಕಾದ
ಬರ-ಬೇ-ಕಿತ್ತೇ
ಬರ-ಬೇ-ಕೆಂಬ
ಬರ-ಮಾ-ಡಿ-ಕೊಂ-ಡರು
ಬರ-ಮಾ-ಡಿ-ಕೊ-ಳ್ಳು-ತ್ತಿ-ದ್ದರು
ಬರ-ಲಾ-ರಂ-ಭಿ-ಸಿ-ದರು
ಬರ-ಲಾ-ರಂ-ಭಿ-ಸಿ-ದವು
ಬರ-ಲಿಲ್ಲ
ಬರಲು
ಬರಲೇ
ಬರ-ವ-ಣಿಗೆ
ಬರ-ವ-ಣಿ-ಗೆಗೂ
ಬರ-ವ-ಣಿ-ಗೆಗೆ
ಬರ-ವ-ಣಿ-ಗೆ-ಯಿಂದ
ಬರ-ವಿ-ರ-ಲಿಲ್ಲ
ಬರ-ವು-ದ-ರೊ-ಳಗೆ
ಬರಹ
ಬರಿಗೈ
ಬರಿ-ಗೈ-ನಲ್ಲಿ
ಬರಿ-ದಿ-ದ್ದಾ-ಗಲೂ
ಬರಿದೆ
ಬರೀ
ಬರು-ತ್ತದೆ
ಬರು-ತ್ತವೆ
ಬರು-ತ್ತಾರೆ
ಬರು-ತ್ತಿತ್ತು
ಬರು-ತ್ತಿದ್ದ
ಬರು-ತ್ತಿ-ದ್ದಂ-ತೆ-ಯಂತೂ
ಬರು-ತ್ತಿ-ದ್ದರು
ಬರು-ತ್ತಿ-ದ್ದ-ವರು
ಬರು-ತ್ತಿ-ದ್ದವು
ಬರು-ತ್ತಿ-ದ್ದಾರೆ
ಬರು-ತ್ತಿ-ದ್ದಾ-ರೆಂ-ದರೆ
ಬರು-ತ್ತಿ-ದ್ದು-ದನ್ನು
ಬರು-ತ್ತಿ-ದ್ದು-ದ-ರಿಂದ
ಬರು-ತ್ತಿ-ದ್ದುದು
ಬರು-ತ್ತಿ-ದ್ದೆವು
ಬರು-ತ್ತಿ-ರುವ
ಬರುವ
ಬರು-ವಂ-ತದ್ದು
ಬರು-ವಂ-ತಾ-ಯಿತು
ಬರು-ವ-ವರ
ಬರು-ವ-ವ-ರಿಂದ
ಬರು-ವ-ವರು
ಬರು-ವ-ವರೂ
ಬರು-ವ-ವ-ರೆಲ್ಲ
ಬರು-ವ-ವರೇ
ಬರು-ವ-ವರೋ
ಬರು-ವಾಗ
ಬರು-ವಾ-ಗಲೇ
ಬರು-ವು-ದಿ-ದ್ದರೂ
ಬರು-ವು-ದಿಲ್ಲ
ಬರು-ವು-ದಿ-ಲ್ಲ-ಅ-ವು-ಗಳ
ಬರು-ವುದು
ಬರುವೆ
ಬರೆದ
ಬರೆ-ದರು
ಬರೆ-ದರೆ
ಬರೆ-ದ-ವರು
ಬರೆ-ದಿದೆ
ಬರೆ-ದಿ-ದ್ದರೆ
ಬರೆ-ದಿ-ದ್ದಾರೆ
ಬರೆ-ದಿ-ದ್ದೇನೆ
ಬರೆ-ದಿ-ರುವ
ಬರೆದು
ಬರೆ-ದು-ಕೊ-ಡು-ವು-ದರ
ಬರೆ-ದು-ಕೊ-ಳ್ಳಲು
ಬರೆ-ಯ-ತೊ-ಡ-ಗಿ-ದರು
ಬರೆ-ಯ-ದಿ-ದ್ದರೆ
ಬರೆ-ಯ-ಬ-ಹುದು
ಬರೆ-ಯ-ಲಾ-ಗಿದೆ
ಬರೆ-ಯಲು
ಬರೆ-ಯ-ಲ್ಪ-ಡಲಿ
ಬರೆ-ಯು-ತ್ತಿ-ದ್ದರೆ
ಬರೆ-ಯು-ತ್ತಿ-ದ್ದೆವು
ಬರೆ-ಯು-ತ್ತಿ-ದ್ದೇನೆ
ಬರೆ-ಯುವ
ಬರೆ-ಯು-ವಷ್ಟು
ಬರೆ-ಯು-ವಾಗ
ಬರೆ-ಯು-ವು-ದಾ-ದರೆ
ಬರೆ-ಯು-ವುದು
ಬರೆ-ಯು-ವು-ದೆಂ-ದರೆ
ಬರೆ-ವ-ಣಿಗೆ
ಬರೆ-ವ-ಣಿ-ಗೆ-ಗಳ
ಬರೆ-ಸಿ-ಕೊಡಿ
ಬರೆ-ಸಿದ
ಬರೆ-ಸುವ
ಬರೆ-ಸು-ವಷ್ಟು
ಬರ್ತೀನಿ
ಬಲ
ಬಲ-ಗೈಯ
ಬಲ-ಗೈ-ಯಲ್ಲಿ
ಬಲ-ದಿಂದ
ಬಲ-ಪಾ-ರ್ಶ್ವದ
ಬಲ-ಭಾ-ಗದ
ಬಲ-ವಂ-ತ-ವಾಗಿ
ಬಲ-ವಾದ
ಬಲಿ-ಯಾಗಿ
ಬಲು
ಬಲೆ-ಯಲ್ಲಿ
ಬಲ್ಲ
ಬಲ್ಲ-ವ-ನ-ಲ್ಲ-ವಾ-ದರೂ
ಬಲ್ಲ-ವ-ರಲ್ಲ
ಬಲ್ಲ-ವ-ರಾಗಿ
ಬಲ್ಲ-ವ-ರಾ-ಗಿ-ದ್ದರು
ಬಲ್ಲ-ವ-ರಿ-ರು-ತ್ತಾರೆ
ಬಲ್ಲ-ವರು
ಬಲ್ಲ-ವ-ರೆಂದು
ಬಲ್ಲಾಳ
ಬಲ್ಲೆ
ಬಳಕೆ
ಬಳ-ಕೆಯ
ಬಳ-ಕೆ-ಯಾ-ಗಿದ್ದು
ಬಳ-ಕೆ-ಯಾ-ಗು-ತ್ತಲೇ
ಬಳ-ಕೆ-ಯಾ-ಗು-ತ್ತಿ-ರು-ವು-ದನ್ನು
ಬಳ-ಗ-ದಲ್ಲಿ
ಬಳ-ಗವೂ
ಬಳ-ಲಿ-ದರೂ
ಬಳಸಿ
ಬಳ-ಸಿ-ಕೊ-ಳ್ಳುವ
ಬಳ-ಸು-ತ್ತಿ-ರ-ಲಿಲ್ಲ
ಬಳ-ಸುವ
ಬಳ-ಸು-ವ-ವ-ರಲ್ಲೂ
ಬಳಿ
ಬಳಿಕ
ಬಳಿಗೆ
ಬಳಿಗೇ
ಬಳಿ-ಯಲ್ಲೂ
ಬಳಿ-ಯಿದ್ದ
ಬಳಿ-ಯಿ-ರುವ
ಬಳಿಯೇ
ಬಸ-ವ-ಳಿದ
ಬಸ್
ಬಸ್ನಲ್ಲಿ
ಬಸ್ಸು
ಬಸ್ನಲ್ಲಿ
ಬಹಳ
ಬಹ-ಳ-ವಾಗಿ
ಬಹ-ಳ-ವಾ-ಗಿಯೇ
ಬಹ-ಳ-ವಿದೆ
ಬಹ-ಳವೇ
ಬಹ-ಳ-ಷ್ಟನ್ನು
ಬಹ-ಳಷ್ಟು
ಬಹ-ಳ-ಸ್ವಾ-ರ-ಸ್ಯ-ಪೂ-ರ್ಣ-ವಾ-ಗಿ-ಸು-ತ್ತಿ-ದ್ದರು
ಬಹವಃ
ಬಹ-ವ-ಸ್ಸನ್ತಿ
ಬಹಿ-ರಂಗ
ಬಹಿ-ರಂ-ಗ-ಗೊ-ಳಿ-ಸಿ-ದ್ದಿದೆ
ಬಹಿ-ರಂ-ಗ-ವಾಗಿ
ಬಹು
ಬಹು-ಕಾ-ಲ-ದ-ವ-ರೆಗೆ
ಬಹು-ತೇಕ
ಬಹು-ತೇ-ಕರ
ಬಹು-ತೇ-ಕರು
ಬಹು-ತೇ-ಕ-ವಾಗಿ
ಬಹು-ದಾ-ಗಿದೆ
ಬಹು-ದೂರ
ಬಹು-ದೂ-ರವೇ
ಬಹು-ದೂರಾ
ಬಹು-ಧನ
ಬಹು-ಪಾಲು
ಬಹು-ಭಾಷಾ
ಬಹು-ಮಾನ
ಬಹು-ಮಾ-ನ-ಗ-ಳನ್ನು
ಬಹು-ಮಾ-ನ-ಗ-ಳಲ್ಲಿ
ಬಹು-ಮಾ-ನ-ಗಳು
ಬಹು-ಮಾ-ನ-ವನ್ನು
ಬಹು-ಮಾ-ನ-ವಿ-ರು-ವು-ದ-ರಿಂದ
ಬಹು-ಮುಖ
ಬಹು-ಮುಖಿ
ಬಹು-ಮು-ಖ್ಯ-ವಾ-ದದ್ದು
ಬಹು-ವಾಗಿ
ಬಹು-ವಾ-ಗಿದೆ
ಬಹುಶ
ಬಹುಶಃ
ಬಹು-ಶ್ರುತ
ಬಹು-ಶ್ರು-ತ-ವಿ-ದ್ವಾಂ-ಸ-ರಾಗಿ
ಬಹೂ-ಪ-ಕೃ-ತ-ರಾ-ಗಿ-ದ್ದಾರೆ
ಬಾ
ಬಾಂಧ-ವರ
ಬಾಂಧ-ವಾಃ
ಬಾಂಧವ್ಯ
ಬಾಂಧ-ವ್ಯ-ಗಳು
ಬಾಂಧ-ವ್ಯದ
ಬಾಂಧ-ವ್ಯ-ವನ್ನು
ಬಾಗಿಲು
ಬಾಗುತ
ಬಾಡಿಗೆ
ಬಾಡಿ-ಗೆ-ಗಿದ್ದ
ಬಾಡಿ-ಗೆಗೆ
ಬಾಡಿತು
ಬಾಧಿ-ಸುವ
ಬಾಯಿ
ಬಾಯಿಂದ
ಬಾಯಿ-ಯಲ್ಲಿ
ಬಾರ
ಬಾರದ
ಬಾರ-ದಂತೆ
ಬಾರ-ದಿ-ದ್ದರೂ
ಬಾರದು
ಬಾರದೆ
ಬಾರದೇ
ಬಾರಿ
ಬಾರಿಗೆ
ಬಾರಿ-ಸು-ತ್ತದೆ
ಬಾಲ
ಬಾಲಃ
ಬಾಲ-ಪಾಠ
ಬಾಲ-ಪಾ-ಠ-ಗಳು
ಬಾಲಾ-ನಾಂ
ಬಾಲ್ಯ
ಬಾಲ್ಯದ
ಬಾಲ್ಯ-ದಲ್ಲಿ
ಬಾಲ್ಯ-ದಿಂದ
ಬಾಲ್ಯ-ದಿಂ-ದಲೂ
ಬಾಳನ್ನು
ಬಾಳ-ಲಿಲ್ಲ
ಬಾಳಿ
ಬಾಳಿನ
ಬಾಳಿ-ನಲ್ಲಿ
ಬಾಳಿ-ಬೆ-ಳ-ಗಿದ
ಬಾಳು
ಬಾಳು-ತ್ತವೆ
ಬಾವ
ಬಾವಿ-ಯಲ್ಲಿ
ಬಾಹಿ-ರ-ವೆಂದು
ಬಾಹ್ಯ-ವಾಗಿ
ಬಿಂದು-ವಿಗೆ
ಬಿಂಬಿ-ಸಿ-ರು-ವುದು
ಬಿಎ
ಬಿಎಡ್
ಬಿಎಸ್
ಬಿಎಸ್ಸಿ
ಬಿಕಾಂ
ಬಿಕಾಮ್
ಬಿಗು-ವಿಲ್ಲ
ಬಿಚ್ಚಿ-ಕೊಂಡು
ಬಿಚ್ಚಿ-ಡುವ
ಬಿಜಿ-ನೆಸ್ಸೇ
ಬಿಟ್ಟ
ಬಿಟ್ಟರು
ಬಿಟ್ಟರೆ
ಬಿಟ್ಟ-ವನು
ಬಿಟ್ಟ-ವ-ರಲ್ಲ
ಬಿಟ್ಟಿದ್ದು
ಬಿಟ್ಟಿದ್ದೆ
ಬಿಟ್ಟಿ-ರದ
ಬಿಟ್ಟಿ-ರು-ತ್ತೇವೆ
ಬಿಟ್ಟಿ-ರು-ವು-ದಿಲ್ಲಾ
ಬಿಟ್ಟು
ಬಿಟ್ಟು-ಕೊ-ಟ್ಟ-ವ-ರಲ್ಲ
ಬಿಟ್ಟು-ಕೊಟ್ಟು
ಬಿಟ್ಟೆ
ಬಿಡದ
ಬಿಡ-ಲಿಲ್ಲ
ಬಿಡ-ಲೇ-ಕೂ-ಡದು
ಬಿಡಾರ
ಬಿಡಿ
ಬಿಡಿ-ಗಾಸು
ಬಿಡಿ-ಬಿ-ಡಿ-ಯಾಗಿ
ಬಿಡಿ-ಸ-ಬೇ-ಕೆಂಬ
ಬಿಡಿಸಿ
ಬಿಡು
ಬಿಡು-ಗಡೆ
ಬಿಡು-ಗ-ಡೆ-ಯಾ-ಗುವ
ಬಿಡು-ತ್ತಿತ್ತು
ಬಿಡು-ತ್ತಿ-ರ-ಲಿಲ್ಲ
ಬಿಡು-ತ್ತೇವೆ
ಬಿಡು-ವಂ-ತಾ-ಯಿತು
ಬಿಡು-ವಾ-ದಾ-ಗ-ಲೆಲ್ಲ
ಬಿಡು-ವಿನ
ಬಿತ್ತ-ರದಿ
ಬಿತ್ತ-ರಿ-ಸಿ-ದ್ದಾರೆ
ಬಿತ್ತು
ಬಿದ್ದ
ಬಿದ್ದ-ವ-ರಲ್ಲ
ಬಿದ್ದಿ-ರು-ತ್ತೇ-ವೆಯೋ
ಬಿದ್ದಿ-ರುವ
ಬಿದ್ದಿಲ್ಲ
ಬಿದ್ದು
ಬಿದ್ರ-ಕಾನ
ಬಿದ್ರ-ಕಾನ್
ಬಿದ್ರ-ಕಾ-ನ್ನ-ಲ್ಲಿದ್ದ
ಬಿದ್ರ್ರ-ಕಾನ
ಬಿಬಿಎಂ
ಬಿರು-ಗಾಳಿ
ಬಿರು-ದನ್ನೇ
ಬಿಳಿಯ
ಬಿಸಿ-ಲಿ-ನಿಂದ
ಬಿಸಿಲು
ಬೀಜ
ಬೀಜವೇ
ಬೀದಿ-ಗ-ಳಲ್ಲಿ
ಬೀರಿದ
ಬೀರಿವೆ
ಬೀರುತ್ತ
ಬೀರು-ತ್ತಿತ್ತು
ಬೀಳಗಿ
ಬೀಳಲು
ಬೀಳು-ವಂತೆ
ಬೀಳು-ವ-ವರು
ಬೀಳ್ಕೊ-ಡಲು
ಬೀಳ್ಕೊ-ಡುಗೆ
ಬೀಸಿ
ಬುತ್ತಿ
ಬುದ್ಧ-ನಂತ
ಬುದ್ಧಿ
ಬುದ್ಧಿ-ಗ-ಳಿಂದ
ಬುದ್ಧಿಗೆ
ಬುದ್ಧಿ-ಜೀ-ವಿ-ಗಳು
ಬುದ್ಧಿ-ಪ್ರ-ಚೋ-ದ-ಕ-ವಾದ
ಬುದ್ಧಿ-ಬ-ಲಕೆ
ಬುದ್ಧಿಯ
ಬುದ್ಧಿ-ಯ-ನ್ನಿಟ್ಟು
ಬುದ್ಧಿ-ಯ-ವ-ರಾ-ಗಿಯೇ
ಬುದ್ಧಿ-ಯಿಂದ
ಬುದ್ಧಿ-ಯೊ-ಡಲು
ಬುದ್ಧಿ-ವಂತ
ಬುದ್ಧಿ-ಶ-ಕ್ತಿ-ಗ-ನು-ಗು-ಣ-ವಾಗಿ
ಬುಧ
ಬುಧ-ವಾರ
ಬುಧ್ದಿ-ಮತ್ತೆ
ಬೃಹದ್
ಬೃಹ-ಸ್ಪತಿ
ಬೃಹ-ಸ್ಪ-ತಿಯೇ
ಬೆಂಗ-ಳೂ-ರಿಗೆ
ಬೆಂಗ-ಳೂ-ರಿನ
ಬೆಂಗ-ಳೂ-ರಿ-ನಲ್ಲಿ
ಬೆಂಗ-ಳೂರು
ಬೆಂಬಲ
ಬೆಂಬ-ಲಕ್ಕೆ
ಬೆಂಬ-ಲ-ವಾಗಿ
ಬೆಂಬ-ಲವೂ
ಬೆಟ್ಟ-ಗು-ಡ್ಡ-ಗ-ಳಿಂದ
ಬೆಟ್ಟ-ಗುಡ್ಡ
ಬೆತ್ತ-ಗೇ-ರಿಗೆ
ಬೆತ್ತ-ಗೇ-ರಿಯ
ಬೆದ-ರಿ-ಕೆಯೂ
ಬೆನ್ನ-ಮೇಲೆ
ಬೆನ್ನೆ-ಲು-ಬಾಗಿ
ಬೆರ-ಗಾ-ಗಲಿ
ಬೆರಳು
ಬೆರ-ಳು-ಗ-ಳನ್ನು
ಬೆರ-ಳು-ಗ-ಳಿಂ-ದಲೇ
ಬೆರ-ಳೆ-ಣಿ-ಕೆಯ
ಬೆರೆ-ತಿ-ರ-ಬೇಕು
ಬೆರೆ-ಯು-ವುದು
ಬೆಲೆ
ಬೆಲ್ಲ
ಬೆಳ-ಕನ್ನು
ಬೆಳ-ಕಾ-ಗಲಿ
ಬೆಳ-ಕಾ-ಯಿತು
ಬೆಳ-ಕಿಗೆ
ಬೆಳಕು
ಬೆಳ-ಗಲಿ
ಬೆಳ-ಗಿ-ದರೆ
ಬೆಳ-ಗಿನ
ಬೆಳ-ಗಿಸಿ
ಬೆಳ-ಗಿ-ಸಿ-ಕೊಂ-ಡಿರಿ
ಬೆಳ-ಗಿ-ಸಿ-ದಿರಿ
ಬೆಳ-ಗು-ವಂತ
ಬೆಳ-ಗ್ಗಿನ
ಬೆಳಗ್ಗೆ
ಬೆಳ-ವ-ಣಿ-ಗೆಯ
ಬೆಳಿಗ್ಗೆ
ಬೆಳೆ-ದಿದೆ
ಬೆಳೆ-ದಿ-ದ್ದಾರೆ
ಬೆಳೆದು
ಬೆಳೆ-ಯ-ಬೇ-ಕೆಂದು
ಬೆಳೆ-ಯಲಿ
ಬೆಳೆ-ಯಲು
ಬೆಳೆ-ಯು-ತ್ತಿತ್ತು
ಬೆಳೆ-ಯು-ತ್ತಿ-ದ್ದಾನೆ
ಬೆಳೆ-ಯು-ತ್ತಿದ್ದೆ
ಬೆಳೆ-ವ-ಣಿ-ಗೆಯ
ಬೆಳೆ-ಸಲು
ಬೆಳೆಸಿ
ಬೆಳೆ-ಸಿ-ಕೊಂ-ಡ-ವರು
ಬೆಳೆ-ಸಿ-ಕೊಂ-ಡಿದ್ದು
ಬೆಳೆ-ಸಿ-ಕೊಂಡೆ
ಬೆಳೆ-ಸಿ-ಕೊ-ಳ್ಳಲು
ಬೆಳೆ-ಸಿತು
ಬೆಳೆ-ಸಿದ
ಬೆಳೆ-ಸಿ-ದಿರಿ
ಬೆಳೆ-ಸು-ತ್ತಿ-ದ್ದಾರೆ
ಬೆಳೆ-ಸು-ತ್ತಿವೆ
ಬೆಳ್ಳಾವೆ
ಬೆಳ್ಳಿ
ಬೆವ-ರು-ತ್ತಿದ್ದ
ಬೇಕನಾ
ಬೇಕಾ-ಗಿತ್ತು
ಬೇಕಾ-ಗಿದೆ
ಬೇಕಾ-ಗಿ-ರು-ವಂ-ತಹ
ಬೇಕಾ-ಗು-ತ್ತಿತ್ತು
ಬೇಕಾದ
ಬೇಕಾ-ದರೂ
ಬೇಕಾ-ದರೆ
ಬೇಕಾ-ದಷ್ಟು
ಬೇಕಾ-ದು-ದನ್ನು
ಬೇಕಾ-ದು-ದನ್ನೇ
ಬೇಕಿದೆ
ಬೇಕು
ಬೇಕು
ಬೇಕೆಂದು
ಬೇಕೆಂಬ
ಬೇಕೆ-ನ್ನು-ತ್ತದೆ
ಬೇಕೇ
ಬೇಕೋ
ಬೇಗ
ಬೇಗು-ದಿ-ಯಲ್ಲಿ
ಬೇಗೆ-ಗ-ಳನ್ನು
ಬೇಟಿ-ಯಾದ
ಬೇಡ
ಬೇಡಲು
ಬೇಡಿ
ಬೇಡಿ-ಕೆ-ಗ-ಳನ್ನು
ಬೇಡಿ-ಕೆ-ಯನ್ನು
ಬೇಡಿ-ಕೆ-ಯಿ-ಟ್ಟರು
ಬೇಡಿ-ಕೊ-ಳ್ಳು-ತ್ತೇವೆ
ಬೇರು
ಬೇರೆ
ಬೇರೆ-ಗ್ರಂ-ಥ-ಗ-ಳನ್ನು
ಬೇರೆಡೆ
ಬೇರೆ-ಡೆಗೆ
ಬೇರೆ-ಯದೇ
ಬೇರೆ-ಯ-ವರ
ಬೇರೆ-ಯ-ವ-ರನ್ನು
ಬೇರೆ-ಯ-ವ-ರಾ-ದರೆ
ಬೇರೆ-ಯ-ವ-ರಿಗೂ
ಬೇರೆ-ಯ-ವ-ರಿಗೆ
ಬೇರೆ-ಯ-ವರು
ಬೇರೆ-ಯಾ-ಗಿ-ದ್ದರೂ
ಬೇರೆ-ಯಾ-ದರೂ
ಬೇರೆ
ಬೇರೆ-ಬೇರೆ
ಬೇರೊಂದು
ಬೇರೊ-ಬ್ಬ-ನಿಲ್ಲ
ಬೇರ್ಪ-ಟ್ಟಿತು
ಬೇರ್ಪ-ಡಿ-ಸದೆ
ಬೇರ್ಪ-ಡಿ-ಸ-ಲಾ-ಯಿತು
ಬೇಲೂರು
ಬೇಸರ
ಬೇಸ-ರ-ಮಾ-ಡಿ-ಕೊಂಡು
ಬೇಸ-ರ-ವಿಲ್ಲ
ಬೇಸ-ರ-ವಿ-ಲ್ಲದ
ಬೇಸ-ರ-ವಿ-ಲ್ಲ-ದ-ಪಾಠ
ಬೇಸ-ರ-ವಿ-ಲ್ಲದೇ
ಬೇಸ-ರಿ-ಸು-ತ್ತಿ-ರ-ಲಿಲ್ಲ
ಬೇಸಾಯ
ಬೇಸಾ-ಯಕ್ಕೆ
ಬೇಸಾ-ಯದ
ಬೇಸಾ-ಯ-ದಿಂದ
ಬೇಸಿಗೆ
ಬೇಸಿ-ಗೆಯ
ಬೇಸಿ-ಗೆ-ಯಲ್ಲಿ
ಬೈಂದ್ಯಾ
ಬೈಠ-ಕ್ಕು-ಗ-ಳಲ್ಲಿ
ಬೊಧ-ನೆಗೆ
ಬೊಧಾ-ಯ-ನರು
ಬೋಧಕ
ಬೋಧ-ಕ-ನಾದೆ
ಬೋಧನ
ಬೋಧ-ನ-ಶೈಲಿ
ಬೋಧನಾ
ಬೋಧ-ನಾ-ನು-ಭ-ವ-ವಿ-ಲ್ಲದ
ಬೋಧ-ನಾ-ಶೈ-ಲಿಯು
ಬೋಧನೆ
ಬೋಧ-ನೆಗೆ
ಬೋಧ-ನೆ-ಯನ್ನು
ಬೋಧ-ನೆ-ಯಲ್ಲಿ
ಬೋಧ-ನೆ-ಯ-ಲ್ಲಿಯೂ
ಬೋಧ-ನೆ-ಯಿಂದ
ಬೋಧಿ-ವೃ-ಕ್ಷದ
ಬೋಧಿ-ಸದೇ
ಬೋಧಿ-ಸಲು
ಬೋಧಿ-ಸಿದ
ಬೋಧಿ-ಸಿ-ದ್ದಾನೆ
ಬೋಧಿ-ಸಿ-ದ್ದಾರೆ
ಬೋಧಿ-ಸುತ್ತಾ
ಬೋಧಿ-ಸು-ತ್ತಿದ್ದ
ಬೋಧಿ-ಸು-ತ್ತಿ-ದ್ದರು
ಬೋಧಿ-ಸುವ
ಬೋಧಿ-ಸು-ವಾಗ
ಬೋಧಿ-ಸು-ವು-ದರ
ಬೋಧಿ-ಸು-ವುದು
ಬೋರ-ಲಾ-ಗಿ-ಸಲು
ಬೌದ್ಧಿಕ
ಬ್ಯಾಗು
ಬ್ರ
ಬ್ರಹ್ಮ
ಬ್ರಹ್ಮ-ಚ-ರ್ಯ-ಕ್ರ-ಮ-ದಿಂ-ದಲೂ
ಬ್ರಹ್ಮ-ಚ-ರ್ಯಾ-ಶ್ರಮ
ಬ್ರಹ್ಮ-ಚಾ-ರಿ-ಗ-ಳಾಗಿ
ಬ್ರಹ್ಮ-ವಿ-ದ್ಯೋ-ಪ-ನಿ-ಷತ್
ಬ್ರಹ್ಮಾಂ-ಡ-ಗಳ
ಬ್ರಾಹ್ಮಣ
ಭಂಗ-ಬ-ರ-ದಂತೆ
ಭಂಡಾರ
ಭಂಡಾ-ರದ
ಭಂಡಾ-ರ-ದಿಂದ
ಭಂಡಾ-ರ-ವಾದ
ಭಂಢಾ-ರದ
ಭಕ್ತಿಗೆ
ಭಕ್ತಿಯ
ಭಕ್ತಿ-ಯು-ಳ್ಳ-ವರು
ಭಕ್ಷಿತಂ
ಭಕ್ಷಿತಾ
ಭಕ್ಷ್ಯಂ
ಭಗ-ವಂತ
ಭಗ-ವಂ-ತನ
ಭಗ-ವಂ-ತ-ನನ್ನು
ಭಗ-ವಂ-ತ-ನಲ್ಲಿ
ಭಗ-ವಂ-ತ-ನಿಗೆ
ಭಗ-ವಂ-ತನು
ಭಗ-ವ-ದ್ಗೀತೆ
ಭಗ-ವ-ದ್ಧಿ-ತಕ್ಕೆ
ಭಗ-ವಾನ್
ಭಗೀ-ರ-ಥನ
ಭಗೀ-ರ-ಥ-ಯ-ತ್ನ-ಬ-ಲದ
ಭಟ್
ಭಟ್ಟ
ಭಟ್ಟ-ದಂ-ಪತೀ
ಭಟ್ಟ-ಪಾ-ದರ
ಭಟ್ಟರ
ಭಟ್ಟ-ರಂ-ತಹ
ಭಟ್ಟ-ರಂ-ತೆಯೇ
ಭಟ್ಟ-ರಂಥ
ಭಟ್ಟ-ರಂ-ಥಹ
ಭಟ್ಟ-ರದು
ಭಟ್ಟ-ರದ್ದು
ಭಟ್ಟ-ರದ್ದೇ
ಭಟ್ಟ-ರನ್ನು
ಭಟ್ಟ-ರಲ್ಲಿ
ಭಟ್ಟ-ರ-ಲ್ಲಿಯೂ
ಭಟ್ಟ-ರ-ವರು
ಭಟ್ಟ-ರಷ್ಟೇ
ಭಟ್ಟ-ರಾಗಿ
ಭಟ್ಟ-ರಾ-ಗಿ-ರು-ತ್ತಾರೆ
ಭಟ್ಟ-ರಿಂದ
ಭಟ್ಟ-ರಿಂ-ದಲೂ
ಭಟ್ಟ-ರಿ-ಗಿದೆ
ಭಟ್ಟ-ರಿಗೂ
ಭಟ್ಟ-ರಿಗೆ
ಭಟ್ಟ-ರಿಗೇ
ಭಟ್ಟರು
ಭಟ್ಟರೂ
ಭಟ್ಟರೇ
ಭಟ್ಟ-ರೊಂ-ದಿಗೆ
ಭಟ್ಟ-ರೊ-ಡನೆ
ಭಟ್ಟ-ರೊ-ಳ-ಗಿನ
ಭಟ್ಟಾ-ಚಾ-ರ್ಯರ
ಭಟ್ಟಾ-ಚಾ-ರ್ಯ-ರಿಗೆ
ಭಟ್ಟಾ-ಚಾ-ರ್ಯರು
ಭಟ್ಟಿ
ಭಟ್ರೆ
ಭತ್ತದ
ಭತ್ತ-ದ-ಗದ್ದೆ
ಭತ್ತ-ದ-ಗ-ದ್ದೆಯ
ಭದ್ರಂ
ಭದ್ರಾಣಿ
ಭಯ
ಭಯ-ತ್ರಾ-ತ-ರಾ-ಗಿಯೂ
ಭಯ-ತ್ರಾತಾ
ಭಯದ
ಭಯ-ಭೀ-ತ-ನಾದೆ
ಭಯ-ಭೀ-ತ-ರಾ-ಗಿ-ದ್ದರು
ಭಯ-ವನ್ನು
ಭಯ-ವಿತ್ತು
ಭಯ-ವಿ-ರ-ಲಿಲ್ಲ
ಭರತ
ಭರ-ತ-ಮಾತೆ
ಭರ-ತಾಂ
ಭರ-ವ-ಸೆ-ಯಿಂದ
ಭರ್ತಿ
ಭರ್ತಿ-ಯಾ-ಗಿ-ದ್ದರು
ಭರ್ತಿ-ಯಾ-ಗಿ-ರು-ತ್ತಿತ್ತು
ಭರ್ತೃ-ಹ-ರಿಯ
ಭವಂತಿ
ಭವಂತು
ಭವತ
ಭವತಿ
ಭವ-ದ-ಲೇಕೋ
ಭವ-ನ-ಕ-ಟ್ಟಲು
ಭವ-ನ-ವ-ನ್ನಾ-ದರೂ
ಭವಿ-ತ-ವ್ಯಕ್ಕೆ
ಭವಿ-ತ-ವ್ಯದ
ಭವಿ-ಷ್ಯದ
ಭವಿ-ಷ್ಯ-ದ-ಲ್ಲಿನ
ಭವಿ-ಷ್ಯ-ವನ್ನು
ಭವಿ-ಷ್ಯ-ವಿ-ಲ್ಲದ
ಭವೇತ್
ಭವೇ-ತ್ಪ್ರಿಯೇ
ಭವೇ-ನ್ನರಃ
ಭವ್ಯ-ಜೀ-ವ-ನದ
ಭವ್ಯ-ಪ-ರಂ-ಪ-ರೆ-ಯಲ್ಲಿ
ಭವ್ಯ-ವಾಗಿ
ಭವ್ಯ-ವಾ-ಗಿ-ದ್ದರೂ
ಭವ್ಯ-ಸ-ನ್ನಿ-ವೇ-ಶ-ದಲ್ಲಿ
ಭವ್ಯಾ
ಭಸ್ಮ-ನಾಂ-ಜ-ನಮ್
ಭಹುಷ
ಭಾಂದವ್ಯ
ಭಾಂಧವ್ಯ
ಭಾಗ-ಗ-ಳನ್ನು
ಭಾಗ-ಗ-ಳಲ್ಲಿ
ಭಾಗ-ಗ-ಳಾಗಿ
ಭಾಗದ
ಭಾಗ-ದಿಂದ
ಭಾಗ-ವ-ತ-ವನ್ನು
ಭಾಗ-ವನ್ನ
ಭಾಗ-ವ-ನ್ನಾ-ದರೂ
ಭಾಗ-ವ-ಹಿ-ಸ-ಬೇ-ಕಾ-ಯಿತು
ಭಾಗ-ವ-ಹಿ-ಸಲು
ಭಾಗ-ವ-ಹಿಸಿ
ಭಾಗ-ವ-ಹಿ-ಸಿದ
ಭಾಗ-ವ-ಹಿ-ಸಿ-ದ್ದ-ರಿಂದ
ಭಾಗ-ವ-ಹಿ-ಸಿ-ದ್ದಾರೆ
ಭಾಗ-ವ-ಹಿ-ಸಿದ್ದೆ
ಭಾಗ-ವ-ಹಿ-ಸು-ತ್ತಿದೆ
ಭಾಗ-ವ-ಹಿ-ಸು-ತ್ತಿ-ದ್ದರು
ಭಾಗ-ವ-ಹಿ-ಸು-ತ್ತಿದ್ದೆ
ಭಾಗ-ವ-ಹಿ-ಸು-ತ್ತಿ-ದ್ದೆವು
ಭಾಗ-ವ-ಹಿ-ಸು-ತ್ತೇನೆ
ಭಾಗ-ವ-ಹಿ-ಸುವ
ಭಾಗ-ವ-ಹಿ-ಸು-ವಂ-ತಾ-ದು-ದೊಂದು
ಭಾಗ-ವ-ಹಿ-ಸು-ವು-ದಿ-ಲ್ಲ-ವೆಂ-ದಾಗ
ಭಾಗವು
ಭಾಗಿ-ಯಾ-ಗಿ-ದ್ದಾನೆ
ಭಾಗಿ-ಯಾ-ಗು-ವ-ದನ್ನು
ಭಾಗಿ-ಯಾದ
ಭಾಗಿ-ಯಾ-ದದ್ದು
ಭಾಗ್ಯ
ಭಾಗ್ಯ-ಗ-ಳನ್ನು
ಭಾಗ್ಯ-ಗಳು
ಭಾಗ್ಯ-ದಿಂದ
ಭಾಗ್ಯ-ವೆಂದು
ಭಾಗ್ಯಾ-ದಿ-ಗ-ಳನ್ನು
ಭಾಗ್ಯಾನಿ
ಭಾಜ-ನ-ರಾದ
ಭಾನು-ವಾರ
ಭಾನು-ವಾ-ರ				
ಭಾನು-ವಾ-ರದ
ಭಾನು-ವಾ-ರ-ದಂದು
ಭಾರತ
ಭಾರ-ತದ
ಭಾರ-ತ-ವ-ರ್ಷ-ದಲ್ಲಿ
ಭಾರ-ತ-ವ-ರ್ಷ-ದಲ್ಲೇ
ಭಾರ-ತೀಯ
ಭಾರ-ದಲ್ಲಿ
ಭಾರ-ದ್ವಾಜ
ಭಾರ-ವನ್ನು
ಭಾರ-ವಿಯ
ಭಾರ-ವೆ-ನಿ-ಸಿತು
ಭಾರಿ-ಸಿದ
ಭಾವ
ಭಾವ-ಕರೂ
ಭಾವ-ಗಳು
ಭಾವ-ತೀ-ವ್ರ-ತೆಗೆ
ಭಾವ-ದಿಂದ
ಭಾವ-ನಾ-ತ್ಮಕ
ಭಾವ-ನಾ-ತ್ಮ-ಕ-ವೂ-ಬೌ-ದ್ಧಿ-ಕವೂ
ಭಾವನೆ
ಭಾವ-ನೆ-ಯನ್ನು
ಭಾವ-ನೆ-ಯಿಂದ
ಭಾವ-ಪೂ-ರ್ಣವೂ
ಭಾವ-ಯಂತ
ಭಾವ-ವನ್ನು
ಭಾವ-ವನ್ನೇ
ಭಾವ-ವಿದು
ಭಾವ-ಸಂ-ಸ್ಕಾ-ರ-ಸಾ-ಧ-ಕಮ್
ಭಾವಿ-ಸದೇ
ಭಾವಿ-ಸಲಿ
ಭಾವಿಸಿ
ಭಾವಿ-ಸಿ-ದರೂ
ಭಾವಿ-ಸಿದೆ
ಭಾವಿ-ಸು-ತ್ತೇನೆ
ಭಾವಿ-ಸುವ
ಭಾವೀ
ಭಾವು-ಕ-ತ-ನ-ವನ್ನು
ಭಾವು-ಕ-ತೆಗೆ
ಭಾವು-ಕರು
ಭಾಷಣ
ಭಾಷ-ಣ-ಕ-ಲೆ-ಯನ್ನು
ಭಾಷ-ಣ-ಕಾರ
ಭಾಷ-ಣ-ಕಾ-ರರು
ಭಾಷ-ಣಕ್ಕೆ
ಭಾಷ-ಣ-ಗಳು
ಭಾಷ-ಣದ
ಭಾಷ-ಣ-ದಲ್ಲಿ
ಭಾಷ-ಣ-ಮಾ-ಡಿ-ಸಿ-ದ್ದರು
ಭಾಷ-ಣ-ವನ್ನು
ಭಾಷ-ಣ-ವಿ-ಷಯ
ಭಾಷ-ಣ-ಸ್ಪ-ರ್ಧೆಗೆ
ಭಾಷಾ
ಭಾಷಾಂ-ತ-ರಿ-ಸಿ-ಕೊ-ಡು-ವು-ದ-ರೊಂ-ದಿಗೆ
ಭಾಷಾ-ಪ್ರ-ಭು-ತ್ತ್ವ-ವಿದೆ
ಭಾಷಾ-ಶಿ-ಕ್ಷ-ಕರ
ಭಾಷೆ
ಭಾಷೆ-ಗಳ
ಭಾಷೆ-ಗ-ಳಲ್ಲಿ
ಭಾಷೆಗೂ
ಭಾಷೆಗೆ
ಭಾಷೆಯ
ಭಾಷೆ-ಯನ್ನು
ಭಾಷೆ-ಯ-ಲ್ಲಿಯೂ
ಭಾಷೆ-ಯಲ್ಲೂ
ಭಾಷೆಯೂ
ಭಾಷ್ಯಂ
ಭಾಷ್ಯಂ-ರ-ವರ
ಭಾಸ-ವಾ-ಗ-ಬ-ಹು-ದಾದ
ಭಾಸ-ವಾ-ಗಿತ್ತು
ಭಾಸ್ಕ-ರರು
ಭಿಕ್ಷು-ಗ-ಳಾಗಿ
ಭಿಕ್ಷೆ
ಭಿನ್ನ-ವಾದ
ಭಿನ್ನಾ-ಭಿ-ಪ್ರಾ-ಯದ
ಭಿನ್ನಾ-ಭಿ-ಪ್ರಾ-ಯ-ವನ್ನು
ಭೀತಿ
ಭೀತಿ-ಯನ್ನು
ಭುಂಜೀತ
ಭುವ-ನೇ-ಶ್ವರಿ
ಭೂ
ಭೂಗಂ-ಗೆಯೂ
ಭೂಗ-ರ್ಭ-ದಲ್ಲಿ
ಭೂತ-ಪೂರ್ವ
ಭೂತ-ವನ್ನು
ಭೂಪ್ರ-ದೇಶ
ಭೂಮಿ-ಕೆ-ಯಾ-ಗು-ತ್ತದೆ
ಭೂಮಿಗೆ
ಭೂಮಿಯ
ಭೂಮಿ-ರು-ದಕಂ
ಭೂಯಾತ್
ಭೂಷಣ
ಭೂಷ-ಣವೇ
ಭೃಗು
ಭೇಟಿ
ಭೇಟಿ-ಮಾ-ಡಿದ
ಭೇಟಿಯ
ಭೇಟಿ-ಯಲ್ಲೇ
ಭೇಟಿ-ಯಾ-ಗಲು
ಭೇಟಿ-ಯಾಗಿ
ಭೇಟಿ-ಯಾ-ಗಿ-ದ್ದೇನೆ
ಭೇಟಿ-ಯಾ-ಗಿ-ರ-ಲಿಲ್ಲ
ಭೇಟಿ-ಯಾ-ಗು-ತ್ತಿದ್ದೆ
ಭೇಟಿ-ಯಾ-ದಾಗ
ಭೇಟಿ-ಯಾ-ದಾ-ಗ-ಲೆಲ್ಲ
ಭೇಟಿ-ಯಾದೆ
ಭೇಟಿ-ಯಾದ್ದು
ಭೇದ-ಭಾವ
ಭೇದ-ವಿ-ಲ್ಲದೇ
ಭೇದ-ಭಾ-ವವೂ
ಭೊಜನ
ಭೋಗಾದಿ
ಭೋಜನ
ಭೋಜ-ನಕ್ಕೆ
ಭೋಜ-ನದ
ಭೋಜ-ನ-ವ್ಯ-ವ-ಸ್ಥೆ-ಯಾ-ಗ-ದಿ-ದ್ದಾಗ
ಭೋಧಿ-ಸುವ
ಭೌತಿ-ಕ-ದೇ-ಹದ
ಭ್ರಂಶ
ಭ್ರಮಾ
ಭ್ರಮಿ-ಸುವ
ಭ್ರಮೆ
ಭ್ರಮೆ-ಯನ್ನು
ಮಂಕು-ತಿಮ್ಮ
ಮಂಗ-ಗ-ಳಲ್ಲಿ
ಮಂಗ-ನಿಂದ
ಮಂಗ-ಲ-ಮಾ-ತು-ಗ-ಳನ್ನು
ಮಂಗ-ಲಮ್
ಮಂಗ-ಳ-ವಾರ
ಮಂಗ-ಳೂ-ರಿ-ನಲ್ಲಿ
ಮಂಗ-ಳೂರು
ಮಂಚದ
ಮಂಜ
ಮಂಜ-ಗುಣಿ
ಮಂಜು
ಮಂಜು-ನಾಥ
ಮಂಜು-ನಾ-ಥರು
ಮಂಜು-ನಾ-ಥೇ-ಶ್ವರ
ಮಂಟಪ
ಮಂಡನೆ
ಮಂಡ-ನೆ-ಯನ್ನು
ಮಂಡ-ಲಈ
ಮಂಡ-ಲದ
ಮಂಡ-ಲ-ವಿ-ಲ್ಲದ
ಮಂಡಳಿ
ಮಂಡಿ-ಕಲ್
ಮಂಡಿಸಿ
ಮಂಡಿ-ಸಿದ
ಮಂತ್ರ
ಮಂತ್ರ-ಕೌ-ಮು-ದಿ-ಯಲ್ಲಿ
ಮಂತ್ರ-ಗಳ
ಮಂತ್ರ-ಗ-ಳಿಗೆ
ಮಂತ್ರ-ಜ-ಪ-ವಾ-ಗಿ-ಬಿ-ಡು-ವು-ದನ್ನು
ಮಂತ್ರ-ತ್ವವೂ
ಮಂತ್ರದ
ಮಂತ್ರ-ದ-ಲ್ಲಿಯೂ
ಮಂತ್ರ-ಮು-ಗ್ಧ-ರ-ನ್ನಾ-ಗಿ-ಸುವ
ಮಂತ್ರ-ವಾ-ಗಲೀ
ಮಂತ್ರ-ಶಕ್ತಿ
ಮಂತ್ರಾ-ಭ್ಯಾ-ಸ-ವಾ-ಗಿ-ರ-ಬೇಕು
ಮಂತ್ರಾ-ಭ್ಯಾ-ಸವು
ಮಂತ್ರಾ-ರ್ಥ-ಗ-ತ-ಮಾ-ನಸಃ
ಮಂತ್ರಾ-ರ್ಥ-ದಲ್ಲಿ
ಮಂತ್ರಾ-ಲ-ಯ-ದಡಿ
ಮಂತ್ರಿ-ಗಳು
ಮಂತ್ರೋ-ಚ್ಚಾ-ರಣೆ
ಮಂತ್ರೋ-ಽಯ-ಮಿತಿ
ಮಂದವೂ
ಮಂದ-ಸ್ಮಿ-ತ-ರಾಗಿ
ಮಂದ-ಹಾಸ
ಮಂದಾ-ಧಿ-ಕಾ-ರಿ-ಗ-ಳಿಗೆ
ಮಂದಾ-ಧಿ-ಕಾ-ರಿಯೂ
ಮಂದಿ
ಮಂದಿರ
ಮಂದಿ-ರ-ವಾದ
ಮಕ್ಕಳ
ಮಕ್ಕ-ಳಂತೆ
ಮಕ್ಕ-ಳನ್ನು
ಮಕ್ಕ-ಳಲ್ಲಿ
ಮಕ್ಕ-ಳಲ್ಲೇ
ಮಕ್ಕ-ಳಾದ
ಮಕ್ಕ-ಳಿಂದ
ಮಕ್ಕ-ಳಿ-ಗಿಂತ
ಮಕ್ಕ-ಳಿ-ಗಿಂ-ತಲೂ
ಮಕ್ಕ-ಳಿಗೆ
ಮಕ್ಕ-ಳಿಗೇ
ಮಕ್ಕ-ಳಿ-ದ್ದಾರೆ
ಮಕ್ಕಳು
ಮಕ್ಕ-ಳೆಲ್ಲ
ಮಕ್ಕಳೇ
ಮಗ
ಮಗನ
ಮಗ-ನನ್ನೂ
ಮಗ-ನಾದ
ಮಗ-ನಾ-ದ್ದ-ರಿಂದ
ಮಗನೂ
ಮಜಲು
ಮಜ-ಲು-ಗ-ಳಲ್ಲೂ
ಮಟ್ಟ-ಕ್ಕಿ-ಳಿದು
ಮಟ್ಟಕ್ಕೆ
ಮಟ್ಟ-ದಲ್ಲಿ
ಮಟ್ಟಿಗೆ
ಮಠ
ಮಠಕ್ಕೆ
ಮಠದ
ಮಠ-ದಂ-ತಿತ್ತು
ಮಠ-ದಲ್ಲಿ
ಮಠ-ಮಾ-ನ್ಯ-ಗಳ
ಮಡದಿ
ಮಡ-ದಿ-ಯೊಂ-ದಿಗೆ
ಮಡಿಲ
ಮಡಿ-ಲಿ-ನಲ್ಲಿ
ಮಡಿ-ಲ್ಲ-ಲ್ಲಿ-ಟ್ಟು-ಕೊಂ-ಡಿ-ರು-ವುದು
ಮಣಿ
ಮಣಿ-ಗ-ಳನ್ನು
ಮಣಿ-ಗ-ಳಿ-ರು-ತ್ತವೆ
ಮಣಿ-ಗಳು
ಮಣಿದು
ಮಣಿ-ಪಾಲ್
ಮಣೆ
ಮಣ್ಣಿ-ಕೊಪ್ಪ
ಮಣ್ಣಿ-ಕೊ-ಪ್ಪಕ್ಕೆ
ಮಣ್ಣಿ-ಕೊ-ಪ್ಪದ
ಮಣ್ಣಿ-ಕೊ-ಪ್ಪ-ದಲ್ಲಿ
ಮಣ್ಣೀ-ಕೊಪ್ಪ
ಮಣ್ಣು
ಮತ
ಮತಃ
ಮತ-ಗಳೂ
ಮತ-ಭೇದ
ಮತ-ಭೇ-ದ-ಗಳ
ಮತಾ
ಮತಿ
ಮತಿ-ಗಳು
ಮತಿ-ಭೇ-ದ-ಗಳ
ಮತಿ-ಭೇ-ದ-ಗ-ಳನ್ನು
ಮತ್ತ-ವರ
ಮತ್ತಷ್ಟು
ಮತ್ತಾರು
ಮತ್ತಾವ
ಮತ್ತಿ-ನ್ನೇನು
ಮತ್ತು
ಮತ್ತೂ
ಮತ್ತೆ
ಮತ್ತೆಗೆ
ಮತ್ತೆ-ಮತ್ತೆ
ಮತ್ತೆ-ರಡು
ಮತ್ತೆಲ್ಲೂ
ಮತ್ತೇನು
ಮತ್ತೊಂದು
ಮತ್ತೊಂ-ದೆಡೆ
ಮತ್ತೊಬ್ಬ
ಮತ್ತೊ-ಬ್ಬರ
ಮತ್ತೊ-ಬ್ಬ-ರಿಗೆ
ಮತ್ತೊಮ್ಮೆ
ಮತ್ಸ-ರದ
ಮತ್ಸ್ಯ-ಗಳು
ಮದ-ಲ್ಲೆಯ
ಮದು-ವೆಯ
ಮದು-ವೆ-ಯಾಗಿ
ಮದು-ವೆಯೂ
ಮಧುರ
ಮಧು-ರ-ಮ-ಧು-ವಾ-ಗಿತ್ತು
ಮಧು-ರ-ವಾ-ಗಿಯೂ
ಮಧು-ರ-ಸೇ-ವ-ನೆ-ಯಲ್ಲಿ
ಮಧ್ಯ
ಮಧ್ಯಂ-ತ-ರದ
ಮಧ್ಯ-ಗ-ತಿ-ಯಲ್ಲಿ
ಮಧ್ಯ-ಗ-ತಿ-ಯಿಂದ
ಮಧ್ಯದ
ಮಧ್ಯ-ದಲ್ಲಿ
ಮಧ್ಯಮ
ಮಧ್ಯಮಾ
ಮಧ್ಯ-ಮಾ-ರ್ಗಕ್ಕೆ
ಮಧ್ಯಾಹ್ನ
ಮಧ್ಯಾ-ಹ್ನ-ದ-ವ-ರೆಗೆ
ಮಧ್ಯೆ
ಮನ
ಮನಃ
ಮನ-ಗಂಡ
ಮನ-ಗಂ-ಡಿದ್ದ
ಮನ-ಗಂ-ಡಿ-ದ್ದೇನೆ
ಮನ-ಗಂಡು
ಮನ-ಗೆದ್ದ
ಮನ-ಗೆದ್ದು
ಮನ-ದ-ಟ್ಟಾ-ಗದೆ
ಮನ-ದ-ಟ್ಟಾ-ಗಿದೆ
ಮನ-ದ-ಟ್ಟಾ-ಗು-ತ್ತದೆ
ಮನ-ದ-ಟ್ಟಾ-ಗುವ
ಮನ-ದ-ಟ್ಟಾ-ಗು-ವಂತೆ
ಮನ-ದಲ್ಲಿ
ಮನ-ದಾ-ಳದ
ಮನದಿ
ಮನ-ದುಂ-ಬುವ
ಮನನ
ಮನ-ನ-ವೀಯೆ
ಮನ-ನೊಂದು
ಮನ-ಮು-ಟ್ಟು-ವಂತೆ
ಮನ-ವ-ರಿಕೆ
ಮನ-ಶ್ಶಾಂತಿ
ಮನ-ಸ-ಲ್ಲಿ-ರುವ
ಮನಸಾ
ಮನ-ಸೂ-ರೆ-ಗೊಂಡ
ಮನ-ಸೂ-ರೆ-ಗೊ-ಳ್ಳುವ
ಮನ-ಸ್ಥಿ-ತಿಗೆ
ಮನ-ಸ್ಥಿ-ತಿ-ಯಿಂದ
ಮನ-ಸ್ಸನ್ನು
ಮನ-ಸ್ಸಿ-ಗಾ-ನಂ-ದವೂ
ಮನ-ಸ್ಸಿಗೆ
ಮನ-ಸ್ಸಿನ
ಮನ-ಸ್ಸಿ-ನಲ್ಲಿ
ಮನ-ಸ್ಸಿ-ನ-ಲ್ಲಿ-ಟ್ಟು-ಕೊಂಡು
ಮನ-ಸ್ಸಿ-ನ-ವರು
ಮನ-ಸ್ಸಿ-ನಾ-ಳ-ದಲ್ಲಿ
ಮನ-ಸ್ಸಿ-ನಿಂದ
ಮನ-ಸ್ಸಿ-ರ-ಲಿಲ್ಲ
ಮನಸ್ಸು
ಮನ-ಸ್ಸು-ಎಂ-ಬ-ಲ್ಲಿಗೆ
ಮನ-ಸ್ಸು-ಳ್ಳ-ವ-ನಾ-ಗಿ-ರ-ಬೇ-ಕೆ-ನ್ನು-ತ್ತದೆ
ಮನಸ್ಸೇ
ಮನೀ-ಷಿ-ಗ-ಳಾದ
ಮನೀ-ಷಿ-ಗಳೂ
ಮನೀ-ಷಿಣಃ
ಮನು
ಮನು-ರ್ಯಾ-ವನ್ನ
ಮನು-ವಿಗೂ
ಮನು-ವಿನ
ಮನುವು
ಮನುಷ್ಯ
ಮನು-ಷ್ಯನ
ಮನು-ಷ್ಯ-ನಾಗಿ
ಮನು-ಷ್ಯನು
ಮನು-ಷ್ಯ-ಮಾ-ತ್ರ-ನಾದ
ಮನು-ಷ್ಯ-ರಲ್ಲಿ
ಮನು-ಷ್ಯಾ-ಣಾಂ
ಮನೆ
ಮನೆ-ಗ-ಳಿಗೆ
ಮನೆಗೂ
ಮನೆಗೆ
ಮನೆಗೇ
ಮನೆ-ತನ
ಮನೆ-ತ-ನ-ಕ್ಕಿದೆ
ಮನೆ-ತ-ನಕ್ಕೆ
ಮನೆ-ತ-ನ-ಗಳೂ
ಮನೆ-ತ-ನದ
ಮನೆ-ತ-ನ-ದ-ವರು
ಮನೆ-ತುಂಬ
ಮನೆ-ಪಾ-ಠ-ವಂತು
ಮನೆ-ಬಿಟ್ಟು
ಮನೆಯ
ಮನೆ-ಯಂತೆ
ಮನೆ-ಯದ್ದೇ
ಮನೆ-ಯನ್ನು
ಮನೆ-ಯನ್ನೂ
ಮನೆ-ಯಲ್ಲಿ
ಮನೆ-ಯ-ಲ್ಲಿಯೂ
ಮನೆ-ಯ-ಲ್ಲಿಯೇ
ಮನೆ-ಯ-ಲ್ಲಿ-ರಿ-ಸಿ-ಕೊಂಡು
ಮನೆ-ಯಲ್ಲೆ
ಮನೆ-ಯಲ್ಲೇ
ಮನೆ-ಯ-ವರ
ಮನೆ-ಯ-ವ-ರೆಲ್ಲಾ
ಮನೆ-ಯಾದ
ಮನೆ-ಯಿಂದ
ಮನೆಯು
ಮನೆ-ಯೆಂ-ದರೆ
ಮನೆಯೇ
ಮನೆ-ಯೊಂ-ದ-ರಲ್ಲಿ
ಮನೆ-ಯೊಂ-ದಿಗೆ
ಮನೆ-ಯೊಂದು
ಮನೆ-ಯೊ-ಳಗೇ
ಮನೆ-ಸಂ-ಸಾ-ರ-ದಲ್ಲೂ
ಮನೊ-ಬು-ದ್ಧಿ-ಭೂ-ಮಿ-ಕೆ-ಗ-ಳಲ್ಲಿ
ಮನೋ
ಮನೋ-ಧರ್ಮ
ಮನೋ-ಧೋ-ರ-ಣೆ-ಯಲ್ಲಿ
ಮನೋ-ನು-ಕೂ-ಲೆ-ಯಾದ
ಮನೋ-ಬ-ಲ-ವನ್ನು
ಮನೋ-ಬೇ-ನೆ-ಗ-ಳಾಗಿ
ಮನೋ-ಭಾವ
ಮನೋ-ಭಾ-ವದ
ಮನೋ-ಭಾ-ವ-ದ-ವರೇ
ಮನೋ-ಭಾ-ವ-ವನ್ನು
ಮನೋ-ಭೂ-ಮಿ-ಕೆ-ಯಲ್ಲಿ
ಮನೋ-ಭೂ-ಮಿ-ಕೆ-ಯೆಲ್ಲಿ
ಮನೋ-ಮಾ-ಲಿ-ನ್ಯ-ವನ್ನು
ಮನೋ-ಮಾ-ಲಿ-ನ್ಯ-ವಾ-ಗ-ಬ-ಹುದು
ಮನೋ-ಮು-ಕು-ಲ-ವನ್ನು
ಮನೋ-ವಾಂ-ಛಿ-ತ-ವನ್ನು
ಮನೋ-ವಾ-ಕ್ಕಾ-ಯ-ಕ-ರ್ಮ-ಭಿಃ
ಮನೋ-ವಿ-ಕಾ-ಸ-ರೂ-ಪ-ವಾದ
ಮನೋ-ವಿ-ಜ್ಞಾ-ನದ
ಮನೋ-ಸ್ಥಿ-ತಿ-ಯನ್ನು
ಮನ್ನಣೆ
ಮನ್ನ-ಣೆಗೆ
ಮನ್ನಾ-ರಾ-ಯ-ಣನ
ಮನ್ಮ-ಮ-ಹಾ-ರಾಜ
ಮಮತೆ
ಮಯಮ್
ಮಯೂರ
ಮರಕ್ಕೆ
ಮರ-ಗಳ
ಮರದ
ಮರ-ಳ-ದೆಂ-ದಿ-ಗೆಂ-ದಿಗೂ
ಮರಳಿ
ಮರ-ಳಿ-ದ್ದರು
ಮರ-ಳಿ-ಸು-ವಲ್ಲಿ
ಮರ-ವಾ-ಗು-ವು-ದ-ರಿಂದ
ಮರ-ಸೊ-ಬಗು
ಮರಿ-ಮ-ಲ್ಲಪ್ಪ
ಮರಿ-ಮ-ಲ್ಲ-ಪ್ಪ-ನ-ವರ
ಮರು
ಮರು-ಜೀವ
ಮರು-ದಿ-ನವೇ
ಮರು-ಪ-ರೀ-ಕ್ಷೆ-ಯನ್ನು
ಮರು-ಪ-ರೀ-ಕ್ಷೆ-ಯಲ್ಲಿ
ಮರು-ಪ್ರಶ್ನೆ
ಮರು-ಮಾ-ತಿ-ಲ್ಲದೆ
ಮರು-ಮೌ-ಲ್ಯ-ಮಾ-ಪ-ನದ
ಮರು-ಳಾ-ಗ-ದ-ವ-ರಿಲ್ಲ
ಮರು-ಳಾ-ರಾಧ್ಯ
ಮರು-ವರ್ಷ
ಮರೆ-ತಂ-ತಿದೆ
ಮರೆ-ತ-ವ-ರಲ್ಲ
ಮರೆ-ತಿ-ದ್ದಾರೆ
ಮರೆ-ತಿಲ್ಲ
ಮರೆತು
ಮರೆ-ಯದ
ಮರೆ-ಯ-ದಾ-ಗಿದೆ
ಮರೆ-ಯ-ಬಾ-ರದ
ಮರೆ-ಯ-ಲಾ-ಗದ
ಮರೆ-ಯ-ಲಾ-ಗದು
ಮರೆ-ಯ-ಲಾ-ಗದ್ದು
ಮರೆ-ಯ-ಲಾ-ಗು-ವು-ದಿಲ್ಲ
ಮರೆ-ಯ-ಲಾರ
ಮರೆ-ಯ-ಲಾ-ರದ
ಮರೆ-ಯ-ಲಾರೆ
ಮರೆ-ಯ-ಲಾ-ರೆ-ಉ-ತ್ತ-ರ-ಕ-ನ್ನಡ
ಮರೆ-ಯಲು
ಮರೆ-ಯ-ಲುಂಟೆ
ಮರೆ-ಯಾದ
ಮರೆ-ಯು-ತ್ತಿ-ರ-ಲಿಲ್ಲ
ಮರೆ-ಯು-ವಂ-ತಿಲ್ಲ
ಮರೆ-ಯು-ವುದು
ಮರೆ-ಯು-ವುದೇ
ಮರ್ಮ-ವೇನು
ಮರ್ಯಾದಾ
ಮಲಗಿ
ಮಲ-ಗಿದ್ದ
ಮಲ-ಗಿ-ರುವ
ಮಲಿ-ನಾಂ-ಬ-ರ-ಕೇ-ಶಾ-ದಿ-ಮು-ಖ-ದೌ-ರ್ಗಂ-ಧ್ಯ-ಸಂ-ಯುತಃ
ಮಲೆ-ನಾ-ಡಿ-ಗ-ರನ್ನು
ಮಲೆ-ನಾ-ಡಿನ
ಮಲೆ-ನಾ-ಡಿ-ನಲ್ಲಿ
ಮಲ್ಲೇ-ಶ್ವ-ರಂ-ನ-ಲ್ಲಿದ್ದ
ಮಳೆ
ಮಸ್ತ-ಕ-ವಾ-ಗಿ-ರು-ವು-ದನ್ನು
ಮಸ್ತಾದ
ಮಹ-ಡಿಯ
ಮಹ-ಡಿ-ಯಲ್ಲಿ
ಮಹ-ಡಿ-ಯ-ಲ್ಲಿ-ದ್ದರು
ಮಹ-ತಾಂ
ಮಹ-ತಾ-ಮೇ-ಕ-ರೂ-ಪತಾ
ಮಹ-ತ್ತ-ಮ-ವಾ-ದದ್ದು
ಮಹ-ತ್ತ-ರ-ವಾದ
ಮಹ-ತ್ತಾದ
ಮಹತ್ವ
ಮಹ-ತ್ವದ
ಮಹ-ತ್ವ-ದ್ದಾ-ಗಿದೆ
ಮಹ-ದೌ-ದಾ-ರ್ಯದ
ಮಹ-ನೀ-ಯರ
ಮಹ-ನೀ-ಯ-ರಾದ
ಮಹ-ನೀ-ಯ-ರಿಗೂ
ಮಹ-ನೀ-ಯರು
ಮಹರ್ಷಿ
ಮಹಾ
ಮಹಾ-ಕ-ವಿ-ಗಳೂ
ಮಹಾ-ಕ-ವಿಯ
ಮಹಾ-ಕಾರ್ಯ
ಮಹಾ-ಕಾ-ವ್ಯ-ದಲ್ಲಿ
ಮಹಾ-ತ್ಮರು
ಮಹಾ-ತ್ಮಾ-ಗಾಂಧಿ
ಮಹಾ-ನ-ಗ-ರ-ಗ-ಳಲ್ಲಿ
ಮಹಾನ್
ಮಹಾ-ಪಂ-ಡಿ-ತರು
ಮಹಾ-ಪಾ-ಠ-ಶಾಲಾ
ಮಹಾ-ಪಾ-ಠ-ಶಾಲೆ
ಮಹಾ-ಪಾ-ಠ-ಶಾ-ಲೆಗೆ
ಮಹಾ-ಪಾ-ಠ-ಶಾ-ಲೆಯ
ಮಹಾ-ಪಾ-ಠ-ಶಾ-ಲೆ-ಯಲ್ಲಿ
ಮಹಾ-ಪಾ-ಠ-ಶಾ-ಲೆ-ಯಲ್ಲೇ
ಮಹಾ-ಪಾ-ಠ-ಶಾ-ಲೆ-ಯಾ-ಗಿದೆ
ಮಹಾ-ಪಾ-ಠ-ಶಾ-ಲೆ-ಯಿಂದ
ಮಹಾ-ಪ್ರ-ಮಾದ
ಮಹಾ-ಬ-ಲ-ದ-ರ್ಶ-ನ-ಕ್ಕಾಗಿ
ಮಹಾ-ಬ-ಲೇ-ಶ್ವರ
ಮಹಾ-ಬ-ಲೇ-ಶ್ವ-ರ-ಣ್ಣನ
ಮಹಾ-ಬ-ಲೇ-ಶ್ವ-ರ-ಶರ್ಮಾ
ಮಹಾ-ಭಾ-ರತ
ಮಹಾ-ಭಾ-ರ-ತ-ಗ-ಳಲ್ಲಿ
ಮಹಾ-ಭಾ-ರ-ತದ
ಮಹಾ-ಮ-ಹೋ-ಪಾ-ಧ್ಯಾಯ
ಮಹಾ-ಯಜ್ಞ
ಮಹಾ-ರಾಜ
ಮಹಾ-ರಾ-ಜರ
ಮಹಾ-ರಾ-ಜ-ರಿಂದ
ಮಹಾ-ರಾ-ಜ-ರಿಗೆ
ಮಹಾ-ರಾ-ಜರು
ಮಹಾ-ರಾ-ಜ-ವಂ-ಶಕ್ಕೆ
ಮಹಾ-ರಾ-ಜ-ಸಂ-ಸ್ಕೃತ
ಮಹಾ-ರಾ-ಜ-ಸಂ-ಸ್ಕೃ-ತ-ಪಾ-ಠ-ಶಾಲೆ
ಮಹಾ-ರಾಜ್ಞೀ
ಮಹಾ-ರಾ-ಣಿ-ಯ-ವರು
ಮಹಾ-ಲಿಂ-ಗೇ-ಶ್ವರ
ಮಹಾ-ವಿ-ದ್ಯಾ-ಲಯ
ಮಹಾ-ವಿ-ದ್ಯಾ-ಲ-ಯದ
ಮಹಾ-ಸಂ-ಸ್ಥಾ-ನದ
ಮಹಾ-ಸ್ವಾ-ಮಿ-ಗಳು
ಮಹೀ-ಸು-ರ-ಪು-ರಿಯ
ಮಹೀ-ಸು-ರ-ಪುರೀ
ಮಹೇ-ಶ್ವರಃ
ಮಹೋ-ತ್ಸ-ವದ
ಮಹೋ-ನ್ನತ
ಮಾಂಸ
ಮಾಘ-ಕೃ-ಷ್ಣ
ಮಾಘ-ಮಾಸ
ಮಾಟು-ತ್ತಿ-ದ್ದೆವು
ಮಾಡ
ಮಾಡ-ಕೊಟ್ಟ
ಮಾಡ-ತೊ-ಡ-ಗಿದ
ಮಾಡದ
ಮಾಡ-ದಂತೆ
ಮಾಡ-ದಿ-ದ್ದರೂ
ಮಾಡ-ದಿ-ರು-ವುದು
ಮಾಡದೆ
ಮಾಡದೇ
ಮಾಡ-ಬ-ಲ್ಲರು
ಮಾಡ-ಬ-ಹು-ದಲ್ಲಾ
ಮಾಡ-ಬ-ಹು-ದಾ-ಗಿತ್ತು
ಮಾಡ-ಬಾ-ರದು
ಮಾಡ-ಬೇ-ಕಾದ
ಮಾಡ-ಬೇ-ಕಾದ್ದು
ಮಾಡ-ಬೇ-ಕಿತ್ತು
ಮಾಡ-ಬೇಕು
ಮಾಡ-ಬೇಕು
ಮಾಡ-ಬೇ-ಕೆಂ-ದರೆ
ಮಾಡ-ಬೇ-ಕೆಂದು
ಮಾಡ-ಬೇ-ಕೆಂಬ
ಮಾಡ-ಬೇ-ಕೆಂ-ಬುದು
ಮಾಡ-ಬೇ-ಕೆಂ-ಬುದೇ
ಮಾಡ-ಲಾ-ಗ-ಲಿಲ್ಲ
ಮಾಡ-ಲಾಗಿ
ಮಾಡ-ಲಾ-ಗಿತ್ತು
ಮಾಡ-ಲಾ-ಯಿತು
ಮಾಡಲು
ಮಾಡ-ಲೆಂದು
ಮಾಡಲೇ
ಮಾಡ-ಲೋ-ಸುಗ
ಮಾಡ-ಹೊ-ರ-ಟಿದೆ
ಮಾಡಿ
ಮಾಡಿ-ಕೊಂಡ
ಮಾಡಿ-ಕೊಂ-ಡ-ವರು
ಮಾಡಿ-ಕೊಂ-ಡ-ವರೇ
ಮಾಡಿ-ಕೊಂ-ಡಿತು
ಮಾಡಿ-ಕೊಂಡು
ಮಾಡಿ-ಕೊಂಡೆ
ಮಾಡಿ-ಕೊಂ-ಡೆನು
ಮಾಡಿ-ಕೊಟ್ಟ
ಮಾಡಿ-ಕೊ-ಟ್ಟರು
ಮಾಡಿ-ಕೊ-ಡಲಿ
ಮಾಡಿ-ಕೊಡಿ
ಮಾಡಿ-ಕೊ-ಡು-ವು-ದ-ರಿಂದ
ಮಾಡಿ-ಕೊ-ಡು-ವು-ದ-ಲ್ಲದೇ
ಮಾಡಿ-ಕೊ-ಡು-ವುದೇ
ಮಾಡಿ-ಕೊ-ಳ್ಳದೇ
ಮಾಡಿ-ಕೊ-ಳ್ಳ-ಬೇ-ಕೆಂ-ಬುದು
ಮಾಡಿ-ಕೊ-ಳ್ಳಲು
ಮಾಡಿ-ಕೊ-ಳ್ಳು-ತ್ತಲೇ
ಮಾಡಿ-ಕೊ-ಳ್ಳು-ತ್ತಾರೆ
ಮಾಡಿ-ಕೊ-ಳ್ಳುವ
ಮಾಡಿ-ಕೊ-ಳ್ಳು-ವ-ವ-ರಲ್ಲ
ಮಾಡಿ-ಕೊ-ಳ್ಳು-ವ-ವ-ರೆಗೂ
ಮಾಡಿ-ಕೊ-ಳ್ಳು-ವುದು
ಮಾಡಿಕೋ
ಮಾಡಿ-ಟ್ಟಿದ್ದ
ಮಾಡಿ-ಟ್ಟು-ಕೊಂಡು
ಮಾಡಿತು
ಮಾಡಿದ
ಮಾಡಿ-ದಂ-ಥ-ವರು
ಮಾಡಿ-ದರು
ಮಾಡಿ-ದರೂ
ಮಾಡಿ-ದರೆ
ಮಾಡಿ-ದ-ವನ
ಮಾಡಿ-ದ-ವನು
ಮಾಡಿ-ದ-ವ-ರಲ್ಲ
ಮಾಡಿ-ದ-ವ-ರಾ-ಗಿ-ದ್ದಾರೆ
ಮಾಡಿ-ದ-ವರು
ಮಾಡಿ-ದ-ವರೇ
ಮಾಡಿ-ದಾಗ
ಮಾಡಿ-ದುದು
ಮಾಡಿದೆ
ಮಾಡಿ-ದೆವು
ಮಾಡಿದ್ದ
ಮಾಡಿ-ದ್ದ-ಕ್ಕಿಂತ
ಮಾಡಿ-ದ್ದಕ್ಕೆ
ಮಾಡಿ-ದ್ದನ್ನೂ
ಮಾಡಿ-ದ್ದರು
ಮಾಡಿ-ದ್ದರೆ
ಮಾಡಿ-ದ್ದ-ವರು
ಮಾಡಿ-ದ್ದಾನೆ
ಮಾಡಿ-ದ್ದಾರೆ
ಮಾಡಿ-ದ್ದೀರಿ
ಮಾಡಿದ್ದು
ಮಾಡಿದ್ದೆ
ಮಾಡಿ-ದ್ದೆವು
ಮಾಡಿ-ದ್ದೇನೆ
ಮಾಡಿ-ಯಾ-ದರೂ
ಮಾಡಿಯೂ
ಮಾಡಿ-ರು-ತ್ತಾರೆ
ಮಾಡಿ-ರು-ವುದು
ಮಾಡಿ-ರು-ವುದೇ
ಮಾಡಿಲ್ಲ
ಮಾಡಿ-ಸಲು
ಮಾಡಿಸಿ
ಮಾಡಿ-ಸಿ-ಕೊ-ಳ್ಳಲು
ಮಾಡಿ-ಸಿ-ಕೊ-ಳ್ಳು-ವು-ದಕ್ಕೆ
ಮಾಡಿ-ಸಿತು
ಮಾಡಿ-ಸಿ-ದರು
ಮಾಡಿ-ಸಿದೆ
ಮಾಡಿ-ಸಿ-ದ್ದರು
ಮಾಡಿ-ಸಿ-ದ್ದ-ಲ್ಲದೇ
ಮಾಡಿ-ಸಿ-ದ್ದೇನೆ
ಮಾಡಿ-ಸುತ್ತ
ಮಾಡಿ-ಸು-ತ್ತಿ-ದ್ದರು
ಮಾಡಿ-ಸು-ತ್ತಿ-ದ್ದ-ವರು
ಮಾಡಿ-ಸುವ
ಮಾಡಿ-ಸು-ವುದು
ಮಾಡು
ಮಾಡುತ್ತ
ಮಾಡು-ತ್ತದೆ
ಮಾಡು-ತ್ತ-ಲಿ-ರುವ
ಮಾಡುತ್ತಾ
ಮಾಡು-ತ್ತಾನೆ
ಮಾಡು-ತ್ತಾರೆ
ಮಾಡು-ತ್ತಿದೆ
ಮಾಡು-ತ್ತಿದ್ದ
ಮಾಡು-ತ್ತಿ-ದ್ದ-ದನ್ನು
ಮಾಡು-ತ್ತಿ-ದ್ದರು
ಮಾಡು-ತ್ತಿ-ದ್ದಾಗ
ಮಾಡು-ತ್ತಿ-ದ್ದಾರೆ
ಮಾಡು-ತ್ತಿ-ದ್ದೀರಿ
ಮಾಡು-ತ್ತಿ-ದ್ದುದು
ಮಾಡು-ತ್ತಿ-ದ್ದೇವೆ
ಮಾಡು-ತ್ತಿ-ರು-ತ್ತದೆ
ಮಾಡು-ತ್ತಿ-ರು-ತ್ತಾರೆ
ಮಾಡು-ತ್ತಿ-ರುವ
ಮಾಡು-ತ್ತಿ-ರು-ವ-ವರು
ಮಾಡು-ತ್ತೇನೆ
ಮಾಡು-ತ್ತೇ-ನೆಂದು
ಮಾಡು-ತ್ತೇವೆ
ಮಾಡುವ
ಮಾಡು-ವಂ-ತಾ-ದರೆ
ಮಾಡು-ವಂ-ತಾ-ಯಿತು
ಮಾಡು-ವಂತೆ
ಮಾಡು-ವರು
ಮಾಡು-ವ-ಲ್ಲಿಯೂ
ಮಾಡು-ವ-ವ-ನನ್ನು
ಮಾಡು-ವ-ವ-ನಿಗೆ
ಮಾಡು-ವ-ವ-ರಿಗೆ
ಮಾಡು-ವ-ವ-ರಿ-ದ್ದಾರೆ
ಮಾಡು-ವ-ವರು
ಮಾಡು-ವು-ದಕ್ಕೆ
ಮಾಡು-ವು-ದರ
ಮಾಡು-ವು-ದ-ರಲ್ಲಿ
ಮಾಡು-ವು-ದ-ರಲ್ಲೇ
ಮಾಡು-ವು-ದ-ರಿಂದ
ಮಾಡು-ವು-ದ-ಲ್ಲದೇ
ಮಾಡು-ವು-ದಷ್ಟೆ
ಮಾಡು-ವು-ದಷ್ಟೇ
ಮಾಡು-ವು-ದಿಲ್ಲ
ಮಾಡು-ವುದು
ಮಾಡು-ವು-ದೆಂದು
ಮಾಡು-ವುದೇ
ಮಾಣ-ವ-ಕನೇ
ಮಾಣ-ವ-ಕ-ನೊಬ್ಬ
ಮಾಣಿ-ಗ-ಳಿಗೆ
ಮಾತ
ಮಾತ-ಡ-ಬ-ಲ್ಲರು
ಮಾತ-ನಾ-ಡದ
ಮಾತ-ನಾ-ಡದೇ
ಮಾತ-ನಾ-ಡ-ಬ-ಲ್ಲರು
ಮಾತ-ನಾ-ಡರು
ಮಾತ-ನಾ-ಡಲು
ಮಾತ-ನಾಡಿ
ಮಾತ-ನಾ-ಡಿ-ಕೊಂ-ಡಿ-ದ್ದುಂಟು
ಮಾತ-ನಾ-ಡಿ-ಕೊಂ-ಡಿ-ದ್ದೇವೆ
ಮಾತ-ನಾ-ಡಿ-ಕೊ-ಳ್ಳು-ತ್ತಿ-ದ್ದೆವು
ಮಾತ-ನಾ-ಡಿದೆ
ಮಾತ-ನಾ-ಡಿ-ಸಿ-ದರು
ಮಾತ-ನಾ-ಡಿ-ಸಿ-ದಾಗ
ಮಾತ-ನಾ-ಡಿ-ಸಿ-ದುದು
ಮಾತ-ನಾ-ಡು-ತ್ತಾ-ರೆಂದು
ಮಾತ-ನಾ-ಡುವ
ಮಾತ-ನಾ-ಡು-ವ-ವ-ರಲ್ಲ
ಮಾತ-ನಾ-ಡು-ವಾಗ
ಮಾತ-ನಾ-ಡು-ವು-ದಿಲ್ಲ
ಮಾತ-ನ್ನಾ-ಡು-ತ್ತಿ-ದ್ದರೆ
ಮಾತನ್ನು
ಮಾತಲ್ಲ
ಮಾತಲ್ಲಿ
ಮಾತಾ-ಡು-ತ್ತಿ-ದ್ದೆವು
ಮಾತಾ-ಪಿ-ತೃ-ಗ-ಳಂತೆ
ಮಾತಾ-ಪಿ-ತೃ-ಗಳು
ಮಾತಿ-ಗ-ನು-ಗು-ಣ-ವಾಗಿ
ಮಾತಿಗೆ
ಮಾತಿನ
ಮಾತಿ-ನಂ-ತಿ-ರು-ತ್ತದೆ
ಮಾತಿ-ನಂತೆ
ಮಾತಿ-ನಲ್ಲಿ
ಮಾತಿ-ನಿಂತೆ
ಮಾತಿಲ್ಲ
ಮಾತು
ಮಾತು-ಗ-ಳನ್ನು
ಮಾತು-ಗ-ಳಲ್ಲಿ
ಮಾತು-ಗ-ಳಿಂದ
ಮಾತು-ಗಳು
ಮಾತು-ಗಾ-ರನೂ
ಮಾತೂ
ಮಾತೃ
ಮಾತೃ-ಭಾ-ಷೆ-ಯಲ್ಲಿ
ಮಾತೃ-ಭಾ-ಷೆ-ಯಲ್ಲೋ
ಮಾತೃ-ವಾ-ತ್ಸ-ಲ್ಯ-ದಿಂದ
ಮಾತೃ-ಸ್ವ-ರೂ-ಪಿಣಿ
ಮಾತೃ-ಹೃ-ದಯಿ
ಮಾತೆ
ಮಾತೇ
ಮಾತೇ-ಯಿಲ್ಲ
ಮಾತೊಂ-ದನ್ನು
ಮಾತೊಂದು
ಮಾತ್ರ
ಮಾತ್ರಕ್ಕೆ
ಮಾತ್ರ-ದಿಂದ
ಮಾತ್ರ-ನಲ್ಲ
ಮಾತ್ರ-ನಾ-ದ-ಮೇಲೆ
ಮಾತ್ರ-ವಲ್ಲ
ಮಾತ್ರ-ವ-ಲ್ಲದೆ
ಮಾತ್ರ-ವ-ಲ್ಲದೇ
ಮಾತ್ರ-ವಾ-ಗಿ-ರದೇ
ಮಾದರಿ
ಮಾದ-ರಿ-ಯನ್ನು
ಮಾಧ್ಯಮ
ಮಾಧ್ಯ-ಮಿಕ
ಮಾನ
ಮಾನ-ದಂಡ
ಮಾನ-ನೀ-ಯ-ರಾಗಿ
ಮಾನವ
ಮಾನ-ವನ
ಮಾನ-ವ-ನಿಗೆ
ಮಾನ-ವನು
ಮಾನ-ವ-ನ್ನು-ಗೌ-ರ-ವ-ವನ್ನು
ಮಾನ-ವ-ಸ-ಹ-ಜ-ವಾದ
ಮಾನಸ
ಮಾನ-ಸ-ಗಂ-ಗೋತ್ರಿ
ಮಾನ-ಸಿಕ
ಮಾನ-ಸಿ-ಕ-ವಾ-ಗಲೀ
ಮಾನ-ಸಿ-ಕ-ವಾ-ಗಿಯೂ
ಮಾನಸೇ
ಮಾನಾ-ದಿ-ಗ-ಳನ್ನೋ
ಮಾನ್ಯ
ಮಾನ್ಯ-ರಾ-ಗಿದ್ದು
ಮಾನ್ಯರು
ಮಾನ್ಯ-ವಾ-ಗಿತ್ತು
ಮಾನ-ಮ-ರ್ಯಾದೆ
ಮಾಯ-ವಾ-ಗು-ವಂತೆ
ಮಾರ-ಣಾಂ-ತಿ-ಕ-ವಾಗಿ
ಮಾರ-ನೆಯ
ಮಾರು
ಮಾರುತಿ
ಮಾರು-ತಿಯ
ಮಾರ್ಕೆ-ಟ್ಟಿನ
ಮಾರ್ಗ
ಮಾರ್ಗ-ಗ-ಳಿವೆ
ಮಾರ್ಗ-ದ-ರ್ಶ-ಕ-ನಾಗಿ
ಮಾರ್ಗ-ದ-ರ್ಶ-ಕ-ರ-ನ್ನಾಗಿ
ಮಾರ್ಗ-ದ-ರ್ಶ-ಕ-ವಾ-ಗಿ-ರು-ತ್ತದೆ
ಮಾರ್ಗ-ದ-ರ್ಶನ
ಮಾರ್ಗ-ದ-ರ್ಶ-ನ-ಕ್ಕಾಗಿ
ಮಾರ್ಗ-ದ-ರ್ಶ-ನಕ್ಕೆ
ಮಾರ್ಗ-ದ-ರ್ಶ-ನ-ಗಳ
ಮಾರ್ಗ-ದ-ರ್ಶ-ನ-ಗಳು
ಮಾರ್ಗ-ದ-ರ್ಶ-ನ-ದಂತೆ
ಮಾರ್ಗ-ದ-ರ್ಶ-ನ-ದಲ್ಲಿ
ಮಾರ್ಗ-ದ-ರ್ಶ-ನ-ದಲ್ಲೇ
ಮಾರ್ಗ-ದ-ರ್ಶ-ನ-ವನ್ನು
ಮಾರ್ಗ-ದ-ರ್ಶ-ನ-ವಿತ್ತು
ಮಾರ್ಗ-ದಲ್ಲಿ
ಮಾರ್ಗ-ನಿ-ರ್ದೇ-ಶನ
ಮಾರ್ಗ-ವನ್ನು
ಮಾರ್ಗವೇ
ಮಾರ್ಚ್
ಮಾರ್ಟಿನ್
ಮಾರ್ಧ-ನಿ-ಸು-ತ್ತಿತ್ತು
ಮಾರ್ಪ-ಡು-ಗ-ಳನ್ನು
ಮಾರ್ಪಾ-ಡಾ-ಗುತ್ತ
ಮಾರ್ಪಾ-ಡು-ಗಳು
ಮಾಲಾ
ಮಾಲೀ-ಕ-ರಾದ
ಮಾಲೀ-ಕ-ರಿಂದ
ಮಾವ
ಮಾವಂ-ದಿ-ರೊಂ-ದಿಗೆ
ಮಾವ-ನಾದ
ಮಾವಾ
ಮಾವಿನ
ಮಾವಿ-ನ-ಹ-ಣ್ಣಿನ
ಮಾಸಿಲ್ಲ
ಮಾಹಾ-ಶಿ-ವ-ರಾತ್ರಿ
ಮಾಹಿತಿ
ಮಾಹಿ-ತಿ-ಗ-ಳನ್ನು
ಮಾಹೆ-ಯಲ್ಲಿ
ಮಿಂಚಿ-ಮೆ-ರೆದ
ಮಿಂಚುವ
ಮಿಂದೆದ್ದು
ಮಿಕ್ಕಿ-ದ್ದರೆ
ಮಿಗಿ-ಲಾದ
ಮಿತ-ಭುಕ್
ಮಿತಾ-ಹಾರ
ಮಿತ್ರ
ಮಿತ್ರತ್ವ
ಮಿತ್ರನ
ಮಿತ್ರ-ನೆಂದು
ಮಿತ್ರ-ಭಾ-ವವೇ
ಮಿತ್ರ-ರಾಗಿ
ಮಿತ್ರ-ರಾದ
ಮಿತ್ರ-ರಿಗೂ
ಮಿತ್ರರು
ಮಿತ್ರ-ರೆಲ್ಲ
ಮಿದು-ಳಿ-ನ-ಲ್ಲಿದೆ
ಮಿದುಳು
ಮಿನುಗಿ
ಮಿಲಿಯ
ಮಿಳಿ-ತ-ವಾ-ಗಿ-ರು-ವುದು
ಮಿಷ-ನ-ರಿ-ಗಳ
ಮಿಷನ್
ಮೀಟಿ-ದರೆ
ಮೀಟು-ಗೋ-ಲಿ-ನಂತೆ
ಮೀಮಾಂಸಾ
ಮೀಮಾಂ-ಸಾ-ವಿ-ಭಾ-ಗಾ-ಧ್ಯ-ಕ್ಷರು
ಮೀಮಾಂ-ಸಾ-ಶಾಸ್ತ್ರ
ಮೀಮಾಂ-ಸಾ-ಶಾ-ಸ್ತ್ರದ
ಮೀಮಾಂ-ಸಾ-ಶಾ-ಸ್ತ್ರ-ವನ್ನು
ಮೀರದ
ಮೀರ-ಬಾ-ರದು
ಮೀರಿ
ಮೀರಿದ
ಮೀರು-ವಂ-ತಿಲ್ಲ
ಮೀಸ-ಲಾ-ಗಿ-ರಿ-ಸಿ-ದ-ವರು
ಮೀಸ-ಲಾ-ಗಿ-ರು-ವುದು
ಮೀಸ-ಲಾದ
ಮೀಸ-ಲಾ-ದದ್ದು
ಮೀಸ-ಲಿ-ಟ್ಟ-ವ-ರಲ್ಲ
ಮೀಸ-ಲಿಟ್ಟು
ಮೀಸೆ
ಮುಂಚಿ-ತ-ವಾಗಿ
ಮುಂಚಿ-ನಿಂ-ದಲೂ
ಮುಂಜಾ-ವಿ-ನಲ್ಲೆ
ಮುಂಡೂ-ಸರ
ಮುಂತಾಗಿ
ಮುಂತಾದ
ಮುಂತಾ-ದ-ವನ್ನು
ಮುಂತಾ-ದ-ವ-ರಾ-ದರೆ
ಮುಂತಾ-ದ-ವರು
ಮುಂತಾ-ದ-ವು-ಗಳ
ಮುಂತಾ-ದ-ವು-ಗ-ಳನ್ನು
ಮುಂದಾಗಿ
ಮುಂದಾ-ಗಿದ್ದ
ಮುಂದಾದ
ಮುಂದಾ-ದರು
ಮುಂದಾ-ಳ-ತ್ವ-ವ-ಹಿಸಿ
ಮುಂದಿ-ಟ್ಟಾಗ
ಮುಂದಿಟ್ಟು
ಮುಂದಿನ
ಮುಂದಿ-ರುವ
ಮುಂದು
ಮುಂದು-ವ-ರಿದ
ಮುಂದು-ವ-ರಿ-ದಂತೆ
ಮುಂದು-ವ-ರಿ-ದ-ವರು
ಮುಂದು-ವ-ರಿ-ದಿದೆ
ಮುಂದು-ವ-ರಿದು
ಮುಂದು-ವ-ರಿ-ದು-ಕೊಂಡು
ಮುಂದು-ವ-ರಿ-ದುದು
ಮುಂದು-ವ-ರಿದೇ
ಮುಂದು-ವ-ರಿ-ಯಲಿ
ಮುಂದು-ವ-ರಿ-ಯಿತು
ಮುಂದು-ವ-ರಿ-ಯು-ತ್ತಿ-ರು-ವಲ್ಲಿ
ಮುಂದು-ವ-ರಿ-ಯು-ವುದು
ಮುಂದು-ವ-ರಿ-ಸ-ಬೇ-ಕಾದ
ಮುಂದು-ವ-ರಿ-ಸ-ಬೇ-ಕಾ-ದ್ದ-ರಿಂದ
ಮುಂದು-ವ-ರಿ-ಸಲು
ಮುಂದು-ವ-ರಿ-ಸಿದ
ಮುಂದು-ವ-ರಿ-ಸಿದ್ದೆ
ಮುಂದು-ವ-ರಿ-ಸುವ
ಮುಂದು-ವ-ರಿ-ಸು-ವಂ-ತಾ-ಯಿತು
ಮುಂದು-ವ-ರಿ-ಸು-ವುದು
ಮುಂದು-ವ-ರೆದ
ಮುಂದು-ವ-ರೆ-ದಿತ್ತು
ಮುಂದು-ವ-ರೆ-ದಿದೆ
ಮುಂದು-ವ-ರೆದು
ಮುಂದು-ವ-ರೆದೆ
ಮುಂದು-ವ-ರೆ-ಯು-ತ್ತಲೇ
ಮುಂದು-ವ-ರೆ-ಯು-ವು-ದೆಂದು
ಮುಂದು-ವ-ರೆ-ಸ-ಲಾ-ಗ-ಲಿಲ್ಲ
ಮುಂದು-ವ-ರೆ-ಸಲು
ಮುಂದು-ವ-ರೆ-ಸಿದ
ಮುಂದು-ವ-ರೆ-ಸಿ-ದೆವು
ಮುಂದು-ವ-ರೆ-ಸಿ-ದ್ದೇನೆ
ಮುಂದು-ವ-ರೆ-ಸುವ
ಮುಂದೆ
ಮುಂದೆಯೂ
ಮುಂದೆಯೇ
ಮುಂದೇನು
ಮುಂದೇನು
ಮುಂಬೈಯ
ಮುಂಭಾ-ಗ-ದಲ್ಲಿ
ಮುಕು-ಟ-ವನ್ನು
ಮುಕ್ಕೋಟಿ
ಮುಕ್ತ
ಮುಕ್ತ-ವಾಗಿ
ಮುಕ್ತ-ವಾ-ಗಿ-ರು-ತ್ತಿತ್ತು
ಮುಕ್ತಾ-ಫ-ಲೈ-ರ್ವಿ-ದ್ರು-ಮೇಣ
ಮುಕ್ತಾ-ವ-ಕಾ-ಶ-ವಿ-ರ-ಲಿಲ್ಲ
ಮುಕ್ತಾ-ವಲೀ
ಮುಖ
ಮುಖಂ-ಡರು
ಮುಖದಿ
ಮುಖ-ದಿಂದ
ಮುಖ-ಪು-ಟದ
ಮುಖ-ಬಿಂಬ
ಮುಖ-ಭಾ-ವಾ-ಭಿ-ನಯ
ಮುಖ-ಮಾ-ಡೋಣ
ಮುಖಾಂ-ತರ
ಮುಖೇನ
ಮುಖ್ಯ
ಮುಖ್ಯ-ರಲ್ಲಿ
ಮುಖ್ಯ-ವಾಗಿ
ಮುಖ್ಯ-ವಾ-ಗಿದೆ
ಮುಖ್ಯ-ವಾ-ದದ್ದು
ಮುಖ್ಯಸ್ಥ
ಮುಖ್ಯ-ಸ್ಥರು
ಮುಖ್ಯಾ-ಧ್ಯಾ-ಪ-ಕ-ರಾಗಿ
ಮುಖ್ಯೋ-ಪಾ-ಧ್ಯಾ-ಯ-ನ-ನ್ನಾಗಿ
ಮುಖ್ಯೋ-ಪಾ-ಧ್ಯಾ-ಯ-ನಾಗಿ
ಮುಖ್ಯೋ-ಪಾ-ಧ್ಯಾ-ಯ-ನೆಂದು
ಮುಖ್ಯೋ-ಪಾ-ಧ್ಯಾ-ಯರ
ಮುಖ್ಯೋ-ಪಾ-ಧ್ಯಾ-ಯ-ರಾಗಿ
ಮುಖ್ಯೋ-ಪಾ-ಧ್ಯಾ-ಯ-ರಾ-ಗಿದ್ದ
ಮುಖ್ಯೋ-ಪಾ-ಧ್ಯಾ-ಯ-ರಾ-ಗಿ-ದ್ದರು
ಮುಖ್ಯೋ-ಪಾ-ಧ್ಯಾ-ಯರೂ
ಮುಗ-ಯಿತೇ
ಮುಗಿದ
ಮುಗಿ-ದ-ಮೇಲೆ
ಮುಗಿದು
ಮುಗಿ-ದು-ಹೋ-ಯಿತೇ
ಮುಗಿದೆ
ಮುಗಿ-ಯಿತು
ಮುಗಿ-ಯಿತೇ
ಮುಗಿ-ಸ-ಲಾ-ರ-ದಷ್ಟು
ಮುಗಿಸಿ
ಮುಗಿ-ಸಿ-ಕೊ-ಡುವ
ಮುಗಿ-ಸಿದ
ಮುಗಿ-ಸಿ-ದ-ವ-ರಿ-ರುವ
ಮುಗಿ-ಸಿದೆ
ಮುಗಿ-ಸಿ-ದ್ದಳು
ಮುಗಿ-ಸಿ-ದ್ದ-ಳು-ಅ-ವ-ಳನ್ನು
ಮುಗಿ-ಸಿದ್ದೆ
ಮುಗಿ-ಸಿ-ಯೇ-ಬಿ-ಟ್ಟರು
ಮುಗಿ-ಸು-ತ್ತೇನೆ
ಮುಗಿ-ಸುವ
ಮುಗ್ಧ
ಮುಗ್ಧ-ಮ-ನ-ಸ್ಸಿ-ನಲ್ಲಿ
ಮುಗ್ಧ-ರಲ್ಲ
ಮುಗ್ಧ-ರಾ-ಗುವ
ಮುಚು-ಕುಂಟೆ
ಮುಚ್ಚಿ
ಮುಚ್ಚಿ-ಡು-ತ್ತಾನೆ
ಮುಜು-ಗ-ರ-ವನ್ನು
ಮುಟ್ಟದ
ಮುಟ್ಟು-ತ್ತಾರೆ
ಮುಟ್ಟು-ವಂತೆ
ಮುಡಿ-ಪಾ-ಗಿ-ಟ್ಟ-ವರು
ಮುಡು-ಪಾ-ಗಿಟ್ಟ
ಮುತ್ತಿ-ಕೊಂಡೆ
ಮುತ್ಸ-ದ್ಧಿ-ತ-ನ-ದಿಂದ
ಮುದ
ಮುದ-ಗೊ-ಳ್ಳು-ತ್ತದೆ
ಮುದ-ದಲಿ
ಮುದ್ರಣ
ಮುದ್ರ-ಣಾ-ಲ-ಯಕ್ಕೆ
ಮುದ್ರ-ಣಾ-ಲ-ಯ-ವೊಂ-ದನ್ನು
ಮುದ್ರ-ಣೋ-ದ್ಯ-ಮ-ದಲ್ಲಿ
ಮುದ್ರ-ಣೋ-ದ್ಯ-ಮಿ-ಯಾದ
ಮುದ್ರಾ-ರಾ-ಕ್ಷಸ
ಮುನಿ-ಗಳು
ಮುನಿ-ಗಳೂ
ಮುನಿ-ಯಲ್ಲ
ಮುನ್ನ
ಮುನ್ನ-ಡೆ-ದರು
ಮುನ್ನ-ಡೆ-ಯು-ವಂತೆ
ಮುನ್ನ-ಡೆ-ಯು-ವಲ್ಲಿ
ಮುನ್ನ-ಡೆ-ಸಲು
ಮುನ್ನ-ಡೆ-ಸಿ-ಕೊಂಡು
ಮುನ್ನ-ಡೆ-ಸಿ-ದರು
ಮುನ್ನ-ಡೆ-ಸಿ-ದ-ವರು
ಮುನ್ನೆ-ಲೆ-ಗ-ಳಲ್ಲಿ
ಮುನ್ನೋ-ಟ-ದಿಂದ
ಮುರ-ಳೀ-ಧರ
ಮುರಿ-ಯುವ
ಮುಳು-ಗಿದೆ
ಮುಳು-ಗುವ
ಮುಹೂ-ರ್ತ-ದಲ್ಲಿ
ಮೂಕ
ಮೂಗು
ಮೂಡ-ಬ-ಹುದು
ಮೂಡ-ಬಿ-ದಿರೆ
ಮೂಡಿತು
ಮೂಡಿದ
ಮೂಡಿ-ದ್ದ-ಸಂ-ದೇ-ಹ-ವನ್ನು
ಮೂಡಿದ್ದು
ಮೂಡಿ-ಸಿ-ದ-ವರು
ಮೂಡಿ-ಸು-ತ್ತಿ-ದ್ದರು
ಮೂಡು-ತ್ತಿದೆ
ಮೂಢ
ಮೂರನೆ
ಮೂರ-ನೆಯ
ಮೂರ-ನೆ-ಯದು
ಮೂರ-ನೆ-ಯ-ವರು
ಮೂರು
ಮೂರು-ಸಲ
ಮೂರೂ
ಮೂರ್ಖ
ಮೂರ್ಖಾಃ
ಮೂರ್ತ-ರೂಪ
ಮೂರ್ತಿ
ಮೂರ್ತಿಗೆ
ಮೂರ್ತಿಯ
ಮೂರ್ತಿ-ಯನ್ನು
ಮೂರ್ತಿ-ವೆ-ತ್ತಂ-ತಿ-ರುವ
ಮೂಲ
ಮೂಲಂ
ಮೂಲಕ
ಮೂಲ-ಕ-ವಾಗಿ
ಮೂಲ-ಕ-ವಾ-ಗಿಯೇ
ಮೂಲ-ಕವೇ
ಮೂಲ-ಕಾ-ರ-ಣ-ವನ್ನು
ಮೂಲಕ್ಕೆ
ಮೂಲ-ಗ್ರಂ-ಥ-ಗ-ಳನ್ನು
ಮೂಲ-ಗ್ರಂ-ಥ-ವನ್ನು
ಮೂಲತಃ
ಮೂಲದ
ಮೂಲದ್ದು
ಮೂಲ-ನಿ-ವಾಸಿ
ಮೂಲ-ಪ-ರ್ಯಂತಂ
ಮೂಲ-ಪ-ರ್ವ-ದ-ವ-ರೆಗೆ
ಮೂಲ-ಭೂತ
ಮೂಲ-ಭೂ-ತ-ವಾ-ದಕ್ಕೋ
ಮೂಲ-ಮಂ-ತ್ರ-ವಿ-ದ್ದ-ಹಾಗೆ
ಮೂಲ-ಮ-ನೆ-ಯಿಂದ
ಮೂಲ-ವನ್ನು
ಮೂಲ-ವಾ-ಗಿದೆ
ಮೂಲ-ವಾದ
ಮೂಲ-ವಿದು
ಮೂಲ-ಸ್ಥಾ-ನ-ವನ್ನು
ಮೂಲ-ಸ್ಥಾ-ನ-ವಾದ
ಮೂಳೆ-ಗಳ
ಮೂವತ್ತು
ಮೂವ-ರಿಗೂ
ಮೂವರು
ಮೃತ
ಮೃತ್ಯುಂ-ಜಯ
ಮೃದು
ಮೃದೂನಿ
ಮೃದ್ವಾ-ರಿ-ಶು-ಚಿಃ
ಮೆಕ್ಲಿನೋ
ಮೆಚ್ಚಿ-ದರು
ಮೆಚ್ಚಿ-ನ-ದಾ-ಗಿತ್ತು
ಮೆಟ್ಟಿ-ಲ-ನ್ನೇ-ರಿತು
ಮೆಟ್ರಿಕ್
ಮೆನೆಗೆ
ಮೆಮೋ-ರಿ-ಯಲ್
ಮೆರಗು
ಮೆಲಕು
ಮೆಲುಕು
ಮೆಲೆ
ಮೆಸ್
ಮೇಧಾ-ವಿ-ಗಳು
ಮೇಧೆ-ಯನ್ನು
ಮೇರೆಗೆ
ಮೇಲಂತೂ
ಮೇಲಕ್ಕೆ
ಮೇಲಾ-ಟ-ಗಳು
ಮೇಲಾದ
ಮೇಲಿದೆ
ಮೇಲಿದ್ದ
ಮೇಲಿನ
ಮೇಲಿ-ರುವ
ಮೇಲು-ಕೋ-ಟೆಯ
ಮೇಲು-ಕೀಳು
ಮೇಲೆ
ಮೇಲೆ-ತ್ತಿದ
ಮೇಲೆ-ತ್ತಿದ್ದು
ಮೇಲೆ-ತ್ತುವ
ಮೇಲೆಯೇ
ಮೇಲೆ-ಕೆ-ಳಗೆ
ಮೇಲ್ಚಾ-ವ-ಣಿಯು
ಮೇಲ್ನೋ-ಟಕ್ಕೆ
ಮೇಳ-ವಿಸೆ
ಮೇಳೈಸಿ
ಮೇಳೈ-ಸಿದ
ಮೇಳೈ-ಸಿ-ರು-ತ್ತಿತ್ತು
ಮೈ
ಮೈಕ-ಟ್ಟಿನ
ಮೈತ್ರಿ
ಮೈತ್ರಿ-ಯನ್ನು
ಮೈಬ-ಣ್ಣದ
ಮೈಮನ
ಮೈಮೇ-ಲಿನ
ಮೈಯಲ್ಲಿ
ಮೈಲಿ-ಗ-ಲ್ಲನ್ನು
ಮೈಸೂ-ರನ್ನು
ಮೈಸೂ-ರಿಗೂ
ಮೈಸೂ-ರಿಗೆ
ಮೈಸೂ-ರಿನ
ಮೈಸೂ-ರಿ-ನತ್ತ
ಮೈಸೂ-ರಿ-ನ-ಲ್ಲಂತೂ
ಮೈಸೂ-ರಿ-ನಲ್ಲಿ
ಮೈಸೂ-ರಿ-ನ-ಲ್ಲಿಯೂ
ಮೈಸೂ-ರಿ-ನ-ಲ್ಲಿಯೇ
ಮೈಸೂ-ರಿ-ನ-ಲ್ಲಿ-ರುವ
ಮೈಸೂ-ರಿ-ನ-ಲ್ಲಿ-ಲ್ಲ-ದಿ-ದ್ದರೂ
ಮೈಸೂ-ರಿ-ನಲ್ಲೆ
ಮೈಸೂ-ರಿ-ನಲ್ಲೇ
ಮೈಸೂ-ರಿ-ನಿಂದ
ಮೈಸೂರು				
ಮೈಸೂ-ರೆಂ-ದರೆ
ಮೈಸೂ-ರೆಂದು
ಮೈಸೂರೇ
ಮೊಂಡು-ತ-ನ-ದಿಂದ
ಮೊಕ್ಕಾ-ಮಿ-ನಲ್ಲಿ
ಮೊಗೆ-ದಷ್ಟೂ
ಮೊಟ-ಕು-ಗೊ-ಳಿಸಿ
ಮೊಟ್ಟ
ಮೊಟ್ಟ-ಮೊ-ದಲ
ಮೊಟ್ಟ-ಮೊ-ದ-ಲನೆ
ಮೊಟ್ಟ-ಮೊ-ದ-ಲಿಗೆ
ಮೊದ-ಮೊ-ದಲು
ಮೊದಲ
ಮೊದ-ಲನೆ
ಮೊದ-ಲ-ನೆಯ
ಮೊದ-ಲ-ನೆ-ಯ-ದ-ರಲ್ಲಿ
ಮೊದ-ಲ-ನೆ-ಯ-ದಾಗಿ
ಮೊದ-ಲ-ನೆ-ಯದು
ಮೊದ-ಲನೇ
ಮೊದ-ಲ-ಬಾರಿ
ಮೊದ-ಲಾದ
ಮೊದ-ಲಾ-ದ-ವರೂ
ಮೊದ-ಲಾ-ದ-ವು-ಗ-ಳನ್ನು
ಮೊದ-ಲಾ-ದ-ವು-ಗ-ಳಿಂದ
ಮೊದ-ಲಾ-ಯಿತು
ಮೊದ-ಲಿ-ಗ-ನಾಗಿ
ಮೊದ-ಲಿ-ಗ-ರೆಂದು
ಮೊದ-ಲಿಗೆ
ಮೊದ-ಲಿ-ನಂ-ತಾ-ಗಿದ್ದ
ಮೊದ-ಲಿ-ನಂ-ತಿ-ರದೆ
ಮೊದ-ಲಿ-ನಿಂ-ದಲೂ
ಮೊದಲು
ಮೊದಲೂ
ಮೊದಲೇ
ಮೊನ-ಚಾಗಿ
ಮೊಮ್ಮ-ಕ್ಕಳು
ಮೊಳ-ಕೆ-ಯೊ-ಡೆ-ಯುವ
ಮೊಳ-ಗು-ವಿ-ಕೆಯೇ
ಮೋಕ್ಷ-ಯೋಃ
ಮೋಡಿ
ಮೋಡಿಗೆ
ಮೋಹ
ಮೋಹಾ-ದಿ-ಗ-ಳನ್ನೂ
ಮೋಹಾ-ದು-ಡು-ಪೇ-ನಾಸ್ಮಿ
ಮೌನ
ಮೌನೀಶ
ಮೌಲಿಕ
ಮೌಲ್ಯ
ಮೌಲ್ಯ-ಗ-ಳನ್ನು
ಮೌಲ್ಯ-ಮಾ-ಪ-ಕ-ನಾಗಿ
ಮೌಲ್ಯ-ಮಾ-ಪ-ಕ-ರಾಗಿ
ಮೌಲ್ಯ-ಮಾ-ಪನ
ಮೌಲ್ಯ-ಮಾ-ಪ-ನ-ಕ್ಕಾಗಿ
ಮೌಲ್ಯ-ಮಾ-ಪ-ನಕ್ಕೆ
ಮೌಲ್ಯ-ಮಾ-ಪ-ನ-ವನ್ನು
ಮೌಲ್ಯ-ವಿ-ರುವ
ಯಂತ್ರ-ವಿ-ದ್ದಂತೆ
ಯಃ
ಯಕ್ಷ-ಗಾನ
ಯಕ್ಷ-ಗಾ-ನದ
ಯಚ್ಚ
ಯಜು-ರ್ವೇದ
ಯಜು-ರ್ವೇ-ದದ
ಯಜ್ಜಪ್ತಂ
ಯಜ್ಞ
ಯಜ್ಞ-ತ-ತ್ವ-ವನ್ನು
ಯಜ್ಞ-ದಲ್ಲಿ
ಯಜ್ಞ-ದೃ-ಷ್ಟಿಗೆ
ಯಜ್ಞ-ದೃ-ಷ್ಟಿ-ಯಿಂ-ದಲೂ
ಯಜ್ಞ-ಪತಿ
ಯಜ್ಞ-ಪ-ತಿ-ಭ-ಟ್ಟರು
ಯಜ್ಞ-ಪ್ರ-ಕ್ರಿ-ಯ-ಗ-ಳಲ್ಲಿ
ಯಜ್ಞಾ-ಜ್ಜಪ್ಯಂ
ಯಜ್ಞಾ-ನಾಂ
ಯಜ್ಞ-ಯಾ-ಗಾ-ದಿ-ಗ-ಳಲ್ಲಿ
ಯಡಿ-ಯೂ-ರ-ಪ್ಪ-ನ-ವ-ರನ್ನು
ಯಡಿ-ಯೂ-ರ-ಪ್ಪ-ನ-ವರು
ಯತ್
ಯತ್ನ
ಯತ್ನಿ-ಸು-ವುದು
ಯತ್ಸ್ನಾನಂ
ಯಥಾ
ಯಥಾರ್ಥ
ಯಥಾ-ರ್ಥ-ಜ್ಞಾನ
ಯಥಾ-ವ-ತ್ತಾಗಿ
ಯಥೇ-ಚ್ಛ-ವಾಗಿ
ಯಥೈವ
ಯಥೌ-ಷಧಂ
ಯದಾಯು
ಯದು-ತ್ತರೇ
ಯದ್ಯ-ದ್ಯ-ತ್ರೋ-ಪ-ಪ-ದ್ಯತೇ
ಯದ್ಯಪಿ
ಯನ್ನು
ಯಲ-ಹಂಕ
ಯಲ-ಹಂ-ಕದ
ಯಲ್ಲಾ-ಪುರ
ಯಲ್ಲಿ
ಯವಾ-ಗಲೂ
ಯವುದೇ
ಯಶ-ಪ-ರ್ವ-ತ-ವನ್ನು
ಯಶ-ಸ್ವಿ-ಗೊ-ಳಿ-ಸುವ
ಯಶ-ಸ್ವಿ-ಯಾ-ಗಲು
ಯಶ-ಸ್ವಿ-ಯಾಗಿ
ಯಶ-ಸ್ವಿ-ಯಾ-ಗು-ತ್ತಾ-ರೆಂ-ದರೆ
ಯಶ-ಸ್ವಿ-ಯಾ-ಗು-ತ್ತಿದ್ದ
ಯಶಸ್ವೀ
ಯಶ-ಸ್ಸನ್ನೂ
ಯಶ-ಸ್ಸಿಗೆ
ಯಶ-ಸ್ಸಿ-ನಲ್ಲಿ
ಯಶಸ್ಸು
ಯಶ-ಸ್ಸು-ಗ-ಳಿ-ಸಿ-ದ್ದಾರೆ
ಯಶೋ-ಮು-ಖಿ-ಗ-ಳಾ-ಗು-ವಂತೆ
ಯಶ್ಚ
ಯಸ್ತು
ಯಸ್ಯೋ-ಭಯಂ
ಯಾ
ಯಾಕೆ
ಯಾಕೆಂ-ದರೆ
ಯಾಚನೆ
ಯಾಚ-ನೆ
ಯಾಜ್ಞ-ವ-ಲ್ಕ್ಯರು
ಯಾತಿ
ಯಾರನ್ನು
ಯಾರನ್ನೂ
ಯಾರಲ್ಲಿ
ಯಾರ-ಲ್ಲಿಯೂ
ಯಾರಲ್ಲೂ
ಯಾರಾ-ದರೂ
ಯಾರಾ-ದ-ರೊ-ಬ್ಬರು
ಯಾರಿಂ-ದಲೂ
ಯಾರಿ-ಗಾ-ದರೂ
ಯಾರಿಗೂ
ಯಾರಿಗೆ
ಯಾರಿ-ದ್ದಾರೆ
ಯಾರಿ-ರ-ಬ-ಹುದು
ಯಾರಿಲ್ಲ
ಯಾರು
ಯಾರು
ಯಾರೂ
ಯಾರೇ
ಯಾರೊ-ಬ್ಬರೂ
ಯಾರೋ
ಯಾರ್ಯಾ-ರಿಂ-ದಲೋ
ಯಾವ
ಯಾವತ್ತೂ
ಯಾವತ್ತೊ
ಯಾವಾಗ
ಯಾವಾ-ಗ-ಬೇ-ಕಾ-ದರೂ
ಯಾವಾ-ಗಲು
ಯಾವಾ-ಗಲೂ
ಯಾವು-ದಕ್ಕೂ
ಯಾವು-ದನ್ನೂ
ಯಾವು-ದಾ-ದರೂ
ಯಾವು-ದಾ-ದ-ರೇ-ನು-ಎ-ನ್ನುತ್ತಾ
ಯಾವುದು
ಯಾವುದು
ಯಾವುದೆ
ಯಾವು-ದೆಂದು
ಯಾವುದೇ
ಯಾವುದೋ
ಯಾವುಧೇ
ಯಾವೊಬ್ಬ
ಯಾವೊ-ಬ್ಬನೂ
ಯುಕ್ತ-ವಾದ
ಯುಕ್ತಿ
ಯುಗೇ
ಯುತರು
ಯುದ್ಧ
ಯುವ-ಕರ
ಯುವ-ಕ-ರಿಗೆ
ಯುವ-ಸಂ-ನ್ಯಾ-ಸಿ-ಗಳು
ಯೂಟ್ಯೂಬ್
ಯೆಂದು
ಯೇನ
ಯೊ
ಯೊಗ್ಯ-ತೆ-ಗಳೂ
ಯೊಗ್ಯ-ತೆಗೆ
ಯೋ
ಯೋಗ
ಯೋಗಕ್ಕೂ
ಯೋಗ-ಧ್ಯಾ-ನ-ಗ-ಳಂತೆ
ಯೋಗ-ನ-ಗ-ರಿ-ಯಾದ
ಯೋಗ-ಭಾಷ್ಯ
ಯೋಗ-ವಾ-ಸಿ-ಷ್ಠ-ಕಾ-ರರು
ಯೋಗಶ್ಚ
ಯೋಗಾ
ಯೋಗಾ-ಧ್ಯಾ-ಪಕ
ಯೋಗಾ-ಭ್ಯಾ-ಸ-ಕ್ಕಾಗಿ
ಯೋಗಾ-ಯೋಗ
ಯೋಗಾ-ಸನ
ಯೋಗ್ಯ
ಯೋಗ್ಯ-ತಾ-ಸಂ-ಪ-ನ್ನರು
ಯೋಗ್ಯ-ತೆ-ಯನ್ನು
ಯೋಗ್ಯ-ವಾದ
ಯೋಚನೆ
ಯೋಚ-ನೆ-ಗಳು
ಯೋಚಿ-ಸದೇ
ಯೋಚಿಸಿ
ಯೋಜನಾ
ಯೋಜ-ನಾ-ಸ-ಹಾ-ಯ-ಕರು
ಯೋಜನೆ
ಯೋಜ-ನೆಯ
ಯೋಜ-ನೆ-ಯನ್ನು
ಯೋಜ-ನೆ-ಯಿಂದ
ಯೋಜ-ನೆ-ಯೊಂ-ದನ್ನು
ಯೋಜ-ಯತೇ
ಯೋಜಿಸಿ
ಯೋಜಿ-ಸಿದೆ
ಯೋಜಿ-ಸಿದ್ದೆ
ಯೋಪೂ-ರ್ವ-ವೈ-ದ್ಯಾಯ
ರ
ರಂಗ-ರಾವ್
ರಂಗ-ರಾ-ವ್ರ-ವರ
ರಂಗ-ರಾ-ವ್ರ-ವರು
ರಂಗ-ವೇ-ರಿ-ದ್ದರೆ
ರಂಜ-ನೀ-ಯವೂ
ರಂದು
ರಕ್ತ-ಗ-ತ-ವಾಗಿ
ರಕ್ತ-ಗ-ತ-ವಾ-ಗಿಯೇ
ರಕ್ತ-ಸಂ-ಬಂ-ಧ-ದಷ್ಟೆ
ರಕ್ತ-ಸಂ-ಬಂ-ಧಿ-ಗಳೂ
ರಕ್ಷಣೆ
ರಕ್ಷ-ಣೆ-ಗಾಗಿ
ರಕ್ಷ-ಣೆಗೆ
ರಕ್ಷ-ಣೆ-ಯಲ್ಲಿ
ರಕ್ಷಿ-ಸು-ತ್ತಿದ್ದ
ರಕ್ಷಿ-ಸುವ
ರಕ್ಷಿ-ಸು-ವಲ್ಲಿ
ರಘು-ವಂಶ
ರಚನಾ
ರಚ-ನಾ-ಸ-ಮಿ-ತಿ-ಯಲ್ಲಿ
ರಚನೆ
ರಚ-ನೆ-ಯ-ಲ್ಲಿಯೂ
ರಚಿ-ಸ-ಬೇ-ಕಿದೆ
ರಚಿ-ಸಿ-ಕೊಂಡು
ರಚಿ-ಸಿ-ದರು
ರಚಿ-ಸಿ-ರು-ತ್ತಾರೆ
ರಚಿ-ಸು-ವಾಗ
ರಜಾ
ರಜಾ-ದಲ್ಲಿ
ರಜೆಯ
ರಜೆ-ಯನ್ನು
ರಜೆ-ಯಲ್ಲಿ
ರಜೋ
ರಜೋ-ಗು-ಣದ
ರತ್ನಕ್ಕ
ರತ್ನ-ಮ-ಣಿ-ಯಿಂದ
ರತ್ನಾ-ಕ-ರೋಽಯಂ
ರತ್ನಾ-ವತಿ
ರತ್ನಾ-ವ-ತಿ-ಯ-ವ-ರಿ-ದ್ದರು
ರತ್ನಾ-ವತೀ
ರಥವು
ರಮ-ಣರು
ರಮಣಿ
ರಮ-ಣೀಯ
ರಮಾ-ನಂದ
ರಮಾ-ವಿ-ಲಾಸ
ರಮಿ-ಸಿ-ದಿರಿ
ರಮೇಶ
ರಮ್ಯ
ರಮ್ಯ-ನೋಟ
ರಲ್ಲಿ
ರಲ್ಲಿಯೇ
ರವರ
ರವರು
ರವರೇ
ರಷ್ಟಿದ್ದ
ರಸ
ರಸಕ್ಕೆ
ರಸ-ದ-ರ್ಶನ
ರಸ-ನಿ-ಮಿ-ಷ-ಗಳು
ರಸ-ಭ-ರಿ-ತ-ವಾಗಿ
ರಸ-ಮ-ಯ-ಸ-ಮ-ಯ-ವನ್ನು
ರಸಾ-ಯನ
ರಸಾ-ಯ-ನದ
ರಸಿ-ಕ-ರಿಗೆ
ರಸಿ-ಕರು
ರಸ್ತೆಯ
ರಸ್ತೆ-ಯಲ್ಲಿ
ರಸ್ತೆ-ಯ-ಲ್ಲಿದ್ದ
ರಸ್ತೆ-ಯ-ಲ್ಲಿನ
ರಸ್ತೆ-ಯ-ಲ್ಲಿ-ರುವ
ರಹ-ಸ್ಯ-ಗ-ಳನ್ನು
ರಹಿತ
ರಾಗ
ರಾಗದ
ರಾಗ-ವ-ನ್ನು-ವುದು
ರಾಗ-ವಾಗಿ
ರಾಗವು
ರಾಗ-ವೆಂಬ
ರಾಗವೇ
ರಾಗಾ-ದಿ-ಗ-ಳನ್ನು
ರಾಗಾ-ದಿ-ಗಳು
ರಾಗಾ-ದಿ-ರೋ-ಗಾನ್
ರಾಘವ
ರಾಘ-ವ-ಕೆ-ಎಲ್
ರಾಘ-ವೇಂ-ದ್ರ-ಭ-ವ-ನ-ದಲ್ಲಿ
ರಾಘು
ರಾಚ-ಯ್ಯ-ನ-ವರ
ರಾಚೋಟಿ
ರಾಜ
ರಾಜ-ಕಾ-ರ-ಣಿ-ಯೊ-ಬ್ಬರು
ರಾಜ-ಕಾರ್ಯ
ರಾಜ-ಕೀಯ
ರಾಜ-ತಂತ್ರ
ರಾಜ-ಮಾತಾ
ರಾಜ-ಮಾತೆ
ರಾಜ-ಶಾ-ಸ-ನ-ವನ್ನು
ರಾಜಾ
ರಾಜಾ-ಭಿ-ಷೇಕ
ರಾಜಿ-ಯಾ-ಗದೇ
ರಾಜೀ-ವ-ಗಾಂಧೀ
ರಾಜ್ಯ
ರಾಜ್ಯದ
ರಾಜ್ಯ-ಪು-ರಾ-ತತ್ತ್ವ
ರಾಜ್ಯ-ಮ-ಟ್ಟದ
ರಾಜ್ಯ-ಮ-ಟ್ಟ-ದಲ್ಲಿ
ರಾಜ್ಯ-ಸ್ತ-ರ-ದಲ್ಲಿ
ರಾಜ್ಯ-ಸ್ಥ-ರೀಯ
ರಾಡಿ
ರಾತ್ರಿ
ರಾತ್ರಿ-ಕಾ-ಲೇ-ಜಿಗೆ
ರಾತ್ರಿಯ
ರಾಮ-ಕೃಷ್ಣ
ರಾಮ-ಕೃ-ಷ್ಣಾ-ಶ್ರ-ಮದ
ರಾಮ-ಚಂದ್ರ
ರಾಮ-ಚಂ-ದ್ರ-ಶಾ-ಸ್ತ್ರಿ-ಗಳು
ರಾಮನ
ರಾಮ-ಭ-ದ್ರಾ-ಚಾ-ರ್ಯರ
ರಾಮ-ಭ-ದ್ರಾ-ಚಾ-ರ್ಯ-ರಲ್ಲಿ
ರಾಮ-ಭ-ದ್ರಾ-ಚಾ-ರ್ಯರು
ರಾಮ-ಭ-ದ್ರಾ-ಚಾ-ರ್ಯರೇ
ರಾಮ-ರಾಜ್ಯ
ರಾಮ-ರಾ-ಜ್ಯದ
ರಾಮ-ರಾ-ಯರು
ರಾಮ-ರಾ-ಯರೇ
ರಾಮ-ರಾವ್
ರಾಮ-ಶಾ-ಸ್ತ್ರೀ-ಗಳ
ರಾಮ-ಶಾ-ಸ್ತ್ರೀ-ಯ-ವರ
ರಾಮ-ಸ್ವಾಮಿ
ರಾಮಾ-ಯಣ
ರಾಮಾ-ಯ-ಣದ
ರಾಮಾ-ಯ-ಣ-ದಲ್ಲಿ
ರಾಮಾ-ಯ-ಣ-ಭಾ-ರತ
ರಾಮಾ-ಯ-ಣ-ಮ-ಹಾ-ಭಾ-ರತ
ರಾಮಾ-ಯ-ಣ-ಮ-ಹಾ-ಭಾ-ರ-ತ-ಗಳ
ರಾಯರು
ರಾರಾ-ಜಿ-ಸು-ವಂ-ತಾ-ಗಿದೆ
ರಾವ್
ರಾಷ್ಟ್ರ
ರಾಷ್ಟ್ರದ
ರಾಷ್ಟ್ರ-ಪ್ರ-ಶ-ಸ್ತಿ-ವಿ-ಜೇ-ತ-ರಾದ
ರಾಷ್ಟ್ರ-ಮ-ಟ್ಟದ
ರಾಷ್ಟ್ರ-ಮ-ಟ್ಟ-ದಲ್ಲಿ
ರಾಷ್ಟ್ರ-ಮ-ಟ್ಟ-ದ-ಲ್ಲಿಯೂ
ರಾಷ್ಟ್ರ-ಸ್ತ-ರದ
ರಿ
ರಿಂದ
ರಿಂದಲೂ
ರಿಕ್ತ
ರಿಕ್ತ-ವಾದ
ರಿಕ್ತ-ವಾ-ಯಿತು
ರಿಕ್ತ-ಹ-ಸ್ತ-ನಾ-ಗಿದ್ದ
ರೀತಿ
ರೀತಿಯ
ರೀತಿ-ಯನ್ನು
ರೀತಿ-ಯಲ್ಲಿ
ರೀತಿ-ಯ-ಲ್ಲಿಯೂ
ರೀತಿ-ಯಾದ
ರೀತಿಯೇ
ರೀತೀಯ
ರು
ರುಕಾ-ರ-ಸ್ತ-ನ್ನಿ-ರೋ-ಧಕಃ
ರುಚಿ
ರುಚಿ-ಯನ್ನು
ರುಚಿ-ಯಾದ
ರುಚಿಯೂ
ರುದ್ರಾ-ಕ್ಷಿ-ಗ-ಳನ್ನು
ರುದ್ರಾ-ಕ್ಷೈಃ
ರೂಂ
ರೂಗ-ಳನ್ನು
ರೂಢಿ
ರೂಢಿ-ಯೊ-ಳ-ಗು-ತ್ತ-ಮರು
ರೂಢಿ-ಸಿ-ಕೊಂ-ಡ-ವರು
ರೂಢಿ-ಸಿ-ಕೊಂಡೆ
ರೂಢಿ-ಸಿ-ಕೊ-ಳ್ಳ-ಲೇ-ಬೇ-ಕಾದ
ರೂಪ-ದ-ಲ್ಲಾ-ಗಲೀ
ರೂಪ-ದಲ್ಲಿ
ರೂಪ-ದ-ಲ್ಲಿದ್ದು
ರೂಪ-ವಾಗಿ
ರೂಪವೇ
ರೂಪಾಯಿ
ರೂಪಾ-ಯಿ-ಗಳ
ರೂಪಿ-ಸ-ಬ-ಲ್ಲ-ವರು
ರೂಪಿ-ಸಲು
ರೂಪಿ-ಸಿ-ಕೊಂ-ಡೆವು
ರೂಪಿ-ಸಿದ
ರೂಪಿ-ಸಿ-ದರು
ರೂಪಿ-ಸುವ
ರೂಪಿ-ಸು-ವಲ್ಲಿ
ರೂಪು-ಗೊಂ-ಡಿತು
ರೂಪು-ಗೊ-ಳಿಸಿ
ರೂಪು-ರೇಷೆ
ರೂಮಿಗೆ
ರೂಮಿ-ನಲ್ಲಿ
ರೂಮಿ-ನ-ಲ್ಲಿ-ದ್ದರು
ರೂಮಿ-ನ-ಲ್ಲಿ-ದ್ದು-ಕೊಂಡು
ರೂಮು
ರೂವಾರಿ
ರೆಜಿ-ಸ್ಟ್ರಾರ್
ರೆನ್
ರೇಗಿ-ಸುತ್ತಾ
ರೇರ್
ರೇವತಿ
ರೇವ-ತಿ-ಯ-ವರ
ರೇವ-ತಿ-ಯ-ವರು
ರೇವತೀ
ರೇವ-ತೀ-ವಿ-ಘ್ನೇ-ಶ್ವ-ತೀ-ರ್ಥ-ರಿಂದ
ರೋಗ-ಕಾ-ರ-ಣ-ಗ-ಳನ್ನು
ರೋಗ-ಗಳ
ರೋಗ-ಗ-ಳನ್ನು
ರೋಗ-ಗ-ಳ-ನ್ನೆಲ್ಲಾ
ರೋಗ-ಗ-ಳಿಗೆ
ರೋಗ-ಗಳು
ರೋಗ-ನಿ-ವಾ-ರಣಾ
ರೋಗ-ರು-ಜಿ-ನ-ಗಳ
ರೋಗ-ಲ-ಕ್ಷ-ಣ-ಗಳು
ರೋಗವೂ
ರೋಗ-ವೆಂಬ
ರೋಗಿ
ರೋಗಿ-ಗಳು
ರೋಗಿಯ
ರೋಚಕ
ರೋಡ್
ರೋಮಾಂಚ
ರೋಮಾಂ-ಚ-ನ-ಗೊ-ಳಿ-ಸುವ
ರೋಮಾಂ-ಚ-ವನ್ನೂ
ಲ
ಲಂಬೋ-ದರ
ಲಕ್ಷ
ಲಕ್ಷ-ಣ-ದಲ್ಲಿ
ಲಕ್ಷ-ಣಮ್
ಲಕ್ಷ-ಣ-ವನ್ನು
ಲಕ್ಷ-ಣವು
ಲಕ್ಷ-ಣವೇ
ಲಕ್ಷ-ಮು-ದೀ-ರಿ-ತಮ್
ಲಕ್ಷಾಂ-ತರ
ಲಕ್ಷಿಸಿ
ಲಕ್ಷ್ಮೀ-ನಾ-ರಾ-ಯಣ
ಲಕ್ಷ್ಮೀ-ಸಂ-ಗ್ರ-ಹ-ವ್ಯ-ಸ-ನಾ-ಸ-ಕ್ತ-ರಲ್ಲ
ಲಕ್ಷ್ಯ-ದ-ಲ್ಲಿಟ್ಟು
ಲಗು-ಬ-ಗೆ-ಯಿಂದ
ಲಗ್ಗೆ-ಯಿ-ಡು-ತ್ತಿ-ದ್ದರು
ಲಭಿ-ಸಿತು
ಲಭಿ-ಸಿದ
ಲಭಿ-ಸಿ-ದ್ದನ್ನು
ಲಭಿ-ಸು-ವುದೋ
ಲಭ್ಯ
ಲಭ್ಯ-ವಾ-ಗು-ತ್ತದೆ
ಲಭ್ಯ-ವಾದ
ಲಭ್ಯ-ವಾ-ಯಿತು
ಲಭ್ಯ-ವಿ-ರು-ತ್ತಿತ್ತು
ಲಯ
ಲಲಿ-ತಾ-ಸ-ಹ-ಸ್ರ-ನಾಮ
ಲವ-ಲೇಶ
ಲವ-ಲೇ-ಶವೂ
ಲಾಭ
ಲಾಯರ್
ಲಿಂಗಣ್ಣ
ಲಿಂಗ-ವನ್ನು
ಲೀಲಾ-ವತಿ
ಲೀಲಾ-ವತೀ
ಲೆಕ್ಕದ
ಲೆಕ್ಕ-ಮಾಡಿ
ಲೆಕ್ಕ-ವನ್ನು
ಲೆಕ್ಕ-ವಿ-ಡುವ
ಲೆಕ್ಕ-ವಿ-ಡು-ವಾಗ
ಲೆಕ್ಕಾ-ಚಾರ
ಲೆಕ್ಕಿ-ಸದೇ
ಲೆಕ್ಖ-ಪ-ರಿ-ಶೋ-ಧ-ಕರು
ಲೇಔ-ಟ್ನಲ್ಲಿ
ಲೇಖ-ಕನೂ
ಲೇಖ-ಕರ
ಲೇಖ-ಕ-ರಾಗಿ
ಲೇಖನ
ಲೇಖ-ನ-ಗಳ
ಲೇಖ-ನ-ಗ-ಳನ್ನು
ಲೇಖ-ನ-ಗ-ಳ-ನ್ನೊ-ಳ-ಗೊಂಡ
ಲೇಖ-ನ-ಗಳು
ಲೇಖ-ನದ
ಲೇಖ-ನ-ವನ್ನು
ಲೇಖ-ನ-ವನ್ನೇ
ಲೇಖ-ನ-ಸು-ಮ-ನ-ಸ-ಗ-ಳಿಂದ
ಲೇಖ-ನಿಗೆ
ಲೇಖ-ನಿ-ಯಿಂದ
ಲೇಸೆಂಬ
ಲೈಫ್
ಲೋಕ
ಲೋಕಕ್ಕೆ
ಲೋಕದ
ಲೋಕ-ದ-ಲ್ಲಾ-ದರೋ
ಲೋಕ-ದಲ್ಲಿ
ಲೋಕ-ಪಾ-ವ-ನೆ-ಯಾ-ಗಿದ್ದು
ಲೋಕ-ವನ್ನು
ಲೋಕ-ಹಿ-ತ-ದಲ್ಲಿ
ಲೋಕ-ಹಿ-ತ-ವನ್ನು
ಲೋಕಾ-ರ್ಪಣೆ
ಲೋಟ
ಲೋಟ-ದಷ್ಟು
ಲೋಟ-ದೊಂ-ದಿಗೆ
ಲೋಪ-ವನ್ನೂ
ಲೋಪ-ವಾ-ಗದ
ಲೋಭ
ಲೌಕಿಕ
ಲೌಕಿ-ಕ-ಜ್ಞಾನ
ಲೌಕಿ-ಕ-ದೃ-ಷ್ಟಾಂತ
ಲೌಕಿ-ಕ-ದೃ-ಷ್ಟಾಂ-ತದ
ಲೌಕಿ-ಕ-ವಾಗಿ
ಲೌಕಿ-ಕ-ವಿ-ಷ-ಯದ
ವಂಚಿ-ತ-ನಾ-ಗು-ವುದು
ವಂಚಿ-ತ-ನಾದೆ
ವಂದ-ನಾ-ರ್ಪಣೆ
ವಂದ-ನಾ-ರ್ಪ-ಣೆಗೆ
ವಂದ-ನಾ-ರ್ಪ-ಣೆ-ಯನ್ನು
ವಂದ-ನೆ-ಗ-ಳನ್ನು
ವಂದ-ನೆ-ಗಳು
ವಂಶ
ವಂಶೋ-ನ್ನ-ತಿ-ಯನ್ನು
ವಂಸಂ-ತ-ಗ-ಳನ್ನು
ವಕೀ-ಲ-ರು
ವಕ್ತಾ
ವಕ್ರ-ತೆ-ಯನ್ನು
ವಚನ
ವಚ-ನ-ದಂತೆ
ವಜ್ರ-ಗಾರೆ
ವಜ್ರಾ-ದಪಿ
ವಡ್ಡನ್ನು
ವಡ್ಡಿನ
ವತಿ-ಯಿಂದ
ವದ-ನ-ವನ್ನು
ವದಾನ್ಯ
ವದೇ-ದ್ವಾಕ್ಯಂ
ವಧೂ-ವನ್ನು
ವನ-ಸಿರಿ
ವನ್ನು
ವಯ-ಕ್ತಿಕ
ವಯ-ಸ್ಸಿಗೆ
ವಯ-ಸ್ಸಿ-ನಲ್ಲಿ
ವಯ-ಸ್ಸಿ-ನಲ್ಲೂ
ವಯ-ಸ್ಸಿ-ನಿಂ-ದಲೇ
ವಯಸ್ಸು
ವಯೋ-ಧರ್ಮ
ವಯೋ-ಮಾನ
ವಯೋ-ಮಾ-ನ-ದ-ವನೆ
ವರ-ದಾ-ಚಾ-ರ್ಯ-ರಲ್ಲೂ
ವರ-ದಾ-ನ-ವಾ-ಯಿತು
ವರದಿ
ವರ-ದಿ-ಯೊಂ-ದಿಗೆ
ವರಮ್
ವರಿಸಿ
ವರಿ-ಸಿ-ದಂತೆ
ವರಿ-ಸಿ-ದುದು
ವರುಷ
ವರೂ
ವರೆ-ಗಿನ
ವರೆಗೂ
ವರೆಗೆ
ವರೊಂ-ದಿಗೆ
ವರ್ಗದ
ವರ್ಗ-ವಾಗಿ
ವರ್ಗಾ-ಯಿಸಿ
ವರ್ಣ
ವರ್ಣ-ಕ್ರ-ಮ-ದಲ್ಲಿ
ವರ್ಣ-ನೆ-ಯನ್ನು
ವರ್ಣ-ಪ್ರಾಸ
ವರ್ತ-ನೆ-ಗಳು
ವರ್ತ-ಮಾ-ನ-ವನ್ನು
ವರ್ತಿ-ಸಿ-ದರು
ವರ್ತಿ-ಸಿ-ದ್ದ-ರಿಂ-ದಲೇ
ವರ್ತಿ-ಸುವ
ವರ್ಧಿ-ಸಿತು
ವರ್ಧಿ-ಸಿದ
ವರ್ಧಿ-ಸು-ತ್ತಿತ್ತು
ವರ್ಷ
ವರ್ಷಕ್ಕೆ
ವರ್ಷ-ಕ್ಕೊಮ್ಮೆ
ವರ್ಷ-ಗಳ
ವರ್ಷ-ಗ-ಳದು
ವರ್ಷ-ಗ-ಳಲ್ಲಿ
ವರ್ಷ-ಗ-ಳ-ವ-ರೆಗೆ
ವರ್ಷ-ಗ-ಳಿಂದ
ವರ್ಷ-ಗಳು
ವರ್ಷದ
ವರ್ಷ-ದಲ್ಲಿ
ವರ್ಷ-ದವ
ವರ್ಷ-ದ-ವ-ನಾ-ಗು-ವ-ವ-ರೆಗೆ
ವಲ-ಯ-ಗ-ಳಲ್ಲಿ
ವಲ-ಯ-ದಲ್ಲಿ
ವಶೀ-ಲಿ-ಯಿಂದ
ವಸಂತ
ವಸತಿ
ವಸ-ತಿ-ಗ-ಳನ್ನು
ವಸ-ತಿ-ಗೃ-ಹಕ್ಕೆ
ವಸ-ತಿ-ಯನ್ನು
ವಸ-ತಿ-ಯಿಂದ
ವಸು-ಮ-ತಿ-ಯಲ್ಲಿ
ವಸ್ತು
ವಸ್ತು-ಗ-ಳತ್ತ
ವಸ್ತು-ಗ-ಳನ್ನು
ವಸ್ತು-ಗ-ಳ-ನ್ನೆಲ್ಲ
ವಸ್ತು-ಗಳು
ವಸ್ತು-ವನ್ನು
ವಸ್ತು-ವಾ-ಗಿ-ರು-ವದು
ವಸ್ತು-ವಿ-ನಿಂದ
ವಸ್ತು-ಸ್ಥಿತಿ
ವಸ್ತೂ-ಪ-ಲ-ಬ್ಧಯೇ
ವಹಿ-ಸ-ಬೇ-ಕಾದ್ದು
ವಹಿ-ಸ-ಲಾ-ಯಿತು
ವಹಿ-ಸಿ-ಕೊಂಡು
ವಹಿ-ಸಿ-ಕೊ-ಟ್ಟಾಗ
ವಹಿ-ಸಿ-ದರು
ವಹಿ-ಸಿ-ದರೂ
ವಹಿ-ಸಿ-ದವ
ವಹಿ-ಸಿ-ದ್ದರು
ವಹಿ-ಸಿ-ದ್ದಾನೆ
ವಹ್ನಿ-ಮಾನ್
ವಾ
ವಾಕ್
ವಾಕ್ಕನ್ನು
ವಾಕ್ಪ-ಟುತ್ವ
ವಾಕ್ಪ-ಟು-ತ್ವ-ಗ-ಳಿಂದ
ವಾಕ್ಪು-ಷ್ಪ-ವನ್ನು
ವಾಕ್ಪ್ರ-ತಿ-ಯೋ-ಗಿತಾ
ವಾಕ್ಪ್ರ-ತಿ-ಯೋ-ಗಿ-ತಾ-ಸ್ಪ-ರ್ಧೆ-ಗ-ಳಲ್ಲೂ
ವಾಕ್ಯ
ವಾಕ್ಯ-ಗಳು
ವಾಕ್ಯ-ದಂತೆ
ವಾಕ್ಯಾರ್ಥ
ವಾಕ್ಯಾ-ರ್ಥ-ಕಲೆ
ವಾಕ್ಪ್ರ-ತಿ-ಯೋ-ಗಿತಾ
ವಾಗ್ಝ-ರಿ-ಯಿಂದ
ವಾಗ್ಭಟ
ವಾಗ್ಭಟಃ
ವಾಗ್ಭ-ಟನ
ವಾಗ್ಭ-ಟ-ನಾಮಾ
ವಾಗ್ಭ-ಟನು
ವಾಗ್ಭ-ಟಾ-ಚಾ-ರ್ಯರು
ವಾಗ್ಭ-ಟಾ-ಚಾ-ರ್ಯರೂ
ವಾಗ್ಮಿ
ವಾಗ್ಮಿ-ಗ-ಳಾ-ದರು
ವಾಗ್ಮಿ-ಗಳು
ವಾಗ್ವಾ-ದ-ಜ-ಗ-ಳ-ಗಳು
ವಾಗ್ವಿ-ಲಾಸ
ವಾಗ್ವೈ-ಖರಿ
ವಾಙ್ಮ-ಯಕ್ಕೆ
ವಾಚಂ
ವಾಚ-ಸ್ತ-ಥಾ-ಕ್ರೀಯಾ
ವಾಚಾ
ವಾಚಿ
ವಾಚಿಕ
ವಾಚ್ಯ-ಕ್ಕಿಂತ
ವಾಚ್ಯ-ವಾ-ಗಿದೆ
ವಾಟ್ಸಪ್
ವಾಣಿ
ವಾಣಿಜ್ಯ
ವಾಣಿ-ಜ್ಯ-ಪ-ದ-ವಿ-ಯನ್ನು
ವಾಣಿ-ಯನ್ನು
ವಾಣಿ-ವಿ-ಲಾಸ
ವಾತಾ-ವ-ರಣ
ವಾತಾ-ವ-ರ-ಣ-ದಲ್ಲಿ
ವಾತಾ-ವ-ರ-ಣ-ದಿಂದ
ವಾತ್ಸಲ್ಯ
ವಾತ್ಸ-ಲ್ಯ-ದಿಂದ
ವಾತ್ಸ-ಲ್ಯ-ಭ-ರಿತ
ವಾತ್ಸ-ಲ್ಯ-ವನ್ನೂ
ವಾತ್ಸ-ಲ್ಯ-ವಿತ್ತು
ವಾದ
ವಾದಕ್ಕೆ
ವಾದದ
ವಾದ-ವನ್ನು
ವಾದ-ವಿ-ವಾ-ದ-ಗಳ
ವಾದ-ಸ್ಥ-ಲ-ದಲ್ಲಿ
ವಾದಿಯು
ವಾನರ
ವಾನ-ರ-ಮಾ-ನ-ವರ
ವಾನಳ್ಳಿ
ವಾಪಿ
ವಾಮ-ನ-ನಂ-ತಿದ್ದ
ವಾಮ-ನ-ಮೂರ್ತಿ
ವಾಮ-ನಾ-ವ-ತಾ-ರದ
ವಾಯು-ಸೇನಾ
ವಾರ
ವಾರ-ಕ್ಕೊಂದು
ವಾರ-ಕ್ಕೊಮ್ಮೆ
ವಾರದ
ವಾರ-ದಲ್ಲಿ
ವಾರ-ನ್ನ-ಕ್ಕಾಗಿ
ವಾರಾನ್ನ
ವಾರಾ-ನ್ನ-ಕ್ಕಾಗಿ
ವಾರಾ-ನ್ನಕ್ಕೆ
ವಾರಾ-ನ್ನದ
ವಾರಾ-ನ್ನ-ದಿಂದ
ವಾರಾ-ನ್ನ-ವನ್ನು
ವಾರ್ಧಕ್ಯೇ
ವಾರ್ಷಿಕ
ವಾರ್ಷಿ-ಕ-ವಾಗಿ
ವಾರ್ಷಿ-ಕೋ-ತ್ಸ-ವ-ದಲ್ಲಿ
ವಾರ್ಷಿ-ಕೋ-ತ್ಸ-ವವು
ವಾಲಿತು
ವಾಲಿ-ರು-ವುದು
ವಾಲು-ವುದು
ವಾಲ್ಮೀಕಿ
ವಾಸ
ವಾಸದ
ವಾಸನೆ
ವಾಸ-ನೆಯು
ವಾಸ-ಮಾಡಿ
ವಾಸ-ವಾ-ಗಲಿ
ವಾಸ-ವಾ-ಗಿದ್ದ
ವಾಸ-ವಾ-ಗಿ-ದ್ದರು
ವಾಸ-ವಾ-ಗಿ-ರುವ
ವಾಸ-ವ್ಯ-ವಸ್ಥೆ
ವಾಸ-ವ್ಯ-ವ-ಸ್ಥೆ-ಯನ್ನು
ವಾಸಸಾ
ವಾಸ-ಸ್ಥಾನ
ವಾಸಿ-ಯಾ-ಗಿತ್ತು
ವಾಸಿ-ಸ-ತೊ-ಡ-ಗಿದ
ವಾಸಿ-ಸ-ತೊ-ಡ-ಗಿದೆ
ವಾಸಿ-ಸ-ಲಾ-ರಂ-ಭಿ-ಸಿ-ದರು
ವಾಸಿ-ಸು-ತ್ತಿದ್ದ
ವಾಸಿ-ಸು-ತ್ತಿ-ದ್ದರು
ವಾಸಿ-ಸು-ತ್ತಿ-ದ್ದೆವು
ವಾಸಿ-ಸುವ
ವಾಸಿ-ಸು-ವುದು
ವಾಸು
ವಾಸ್ತ-ವ-ತೆಯ
ವಾಸ್ತ-ವ-ವಾಗಿ
ವಾಸ್ತ-ವ-ವಾ-ದು-ದಲ್ಲ
ವಾಸ್ತ-ವಾ-ನು-ಭವ
ವಾಸ್ತ-ವಿಕ
ವಾಸ್ತವ್ಯ
ವಾಹ-ನ-ಗಳ
ವಿ
ವಿಂಗ-ಡಿ-ಸ-ಬ-ಹುದು
ವಿಕಲ್ಪ
ವಿಕ-ಸಿತ
ವಿಕಾಸ
ವಿಕಾ-ಸದ
ವಿಕಾ-ಸ-ದೆ-ಡೆಗೆ
ವಿಕಾ-ಸ-ವನ್ನು
ವಿಕಾ-ಸ-ವಾ-ಗದು
ವಿಕಾ-ಸ-ವಾಗಿ
ವಿಕಾ-ಸ-ವಾ-ದಕ್ಕೆ
ವಿಕಾ-ಸ-ವಾ-ದದ
ವಿಕಾ-ಸ-ವಾ-ದ-ವನ್ನು
ವಿಕಾ-ಸ-ವಾ-ದವು
ವಿಕಾ-ಸ-ವಾ-ದವೂ
ವಿಕೃತ
ವಿಕೃ-ತರೂ
ವಿಕೃ-ತಿ-ಗಳ
ವಿಗಂ-ಗಾ-ಧರ
ವಿಗ್ನೇ-ಶ್ವರ
ವಿಗ್ರ-ಹ-ಗ-ಳನ್ನು
ವಿಗ್ರ-ಹ-ಗಳು
ವಿಘ್ನೇ-ಶ-ಭ-ಟ್ಟರು
ವಿಘ್ನೇ-ಶ್ವರ
ವಿಘ್ನೇ-ಶ್ವ-ರ-ಭಟ್ಟ
ವಿಘ್ನೇ-ಶ್ವ-ರ-ಭ-ಟ್ಟ-ರನ್ನು
ವಿಘ್ನೇ-ಶ್ವ-ರ-ಭ-ಟ್ಟರು
ವಿಚಂ-ದ್ರ-ಶೇ-ಖರ
ವಿಚಾರ
ವಿಚಾ-ರ-ಗಳ
ವಿಚಾ-ರ-ಗ-ಳನ್ನು
ವಿಚಾ-ರ-ಗಳು
ವಿಚಾ-ರ-ಗೋ-ಷ್ಠಿ-ಗಳು
ವಿಚಾ-ರ-ದಲ್ಲಿ
ವಿಚಾ-ರ-ವನ್ನು
ವಿಚಾ-ರ-ವಾ-ಗಿದೆ
ವಿಚಾ-ರವು
ವಿಚಾ-ರವೇ
ವಿಚಾ-ರ-ಸ-ರಣಿ
ವಿಚಾ-ರಿ-ಸಿ-ದರು
ವಿಚಾ-ರಿಸು
ವಿಚ್ಚೇದಃ
ವಿಜಯ
ವಿಜ-ಯಕ್ಕೆ
ವಿಜ-ಯ-ಣ್ಣನ
ವಿಜ-ಯ-ಲಕ್ಷ್ಮಿ
ವಿಜ-ಯ-ವೈ-ಜ-ಯಂ-ತಿಯ
ವಿಜ-ಯ-ವೈ-ಜ-ಯಂ-ತಿ-ಯನ್ನು
ವಿಜೇತ
ವಿಜ್ಞಾನ
ವಿಜ್ಞಾ-ನದ
ವಿಜ್ಞಾ-ನವೂ
ವಿಜ್ಞಾ-ನಿ-ಗಳು
ವಿಜ್ಞಾ-ಪಿ-ಸಿ-ದ್ದೇನೆ
ವಿಟಿ
ವಿಣ್ಮೂ-ತ್ರೋ-ತ್ಸ-ರ್ಗ-ಶಂ-ಕಾ-ದಿ-ಯುಕ್ತಃ
ವಿತ-ರಿ-ಸ-ಬಾ-ರ-ದೆಂದು
ವಿತ-ರಿಸಿ
ವಿತ-ರಿ-ಸಿ-ದಿರಿ
ವಿತ-ರಿ-ಸುವ
ವಿತ್ತಿ-ಯಿಂದ
ವಿದಿ-ತ-ವಾ-ಗು-ತ್ತದೆ
ವಿದಿ-ತ-ವಾದ
ವಿದು-ರನ
ವಿದೇ-ಶ-ಗ-ಳಲ್ಲಿ
ವಿದೇ-ಶದ
ವಿದೇ-ಶ-ದಲ್ಲಿ
ವಿದೇಶಿ
ವಿದೇ-ಶಿ-ಗ-ರಿಗೆ
ವಿದೇ-ಶಿ-ಗರೂ
ವಿದೇ-ಶೀ-ಯ-ರಿಗೂ
ವಿದೇ-ಶೀ-ಯರು
ವಿದೇ-ಶೀ-ಯರೂ
ವಿದ್ಯ-ಮಾ-ನ-ಗಳ
ವಿದ್ಯ-ಮಾ-ನ-ಗಳು
ವಿದ್ಯ-ರ್ಥಿ-ಗಳ
ವಿದ್ಯಾ
ವಿದ್ಯಾಂ
ವಿದ್ಯಾ-ಕ-ಡ-ಲೆಂದು
ವಿದ್ಯಾ-ಕೇಂ-ದ್ರಕ್ಕೆ
ವಿದ್ಯಾ-ಕೇಂ-ದ್ರವೇ
ವಿದ್ಯಾ-ಕ್ಷೇ-ತ್ರ-ದ-ಲ್ಲಿಯೂ
ವಿದ್ಯಾ-ಗ-ಣ-ಪ-ತಿಯ
ವಿದ್ಯಾ-ಗ-ಣ-ಪ-ತಿ-ಯನ್ನು
ವಿದ್ಯಾ-ಗು-ರು-ಗಳ
ವಿದ್ಯಾ-ಗು-ರು-ಗ-ಳಾದ
ವಿದ್ಯಾ-ಗು-ರು-ಗ-ಳಿಗೆ
ವಿದ್ಯಾ-ಗು-ರು-ಗಳು
ವಿದ್ಯಾ-ಜಲ
ವಿದ್ಯಾ-ತೀ-ರ್ಥ-ರಲ್ಲಿ
ವಿದ್ಯಾ-ತೀ-ರ್ಥ-ವನ್ನು
ವಿದ್ಯಾ-ದಾನ
ವಿದ್ಯಾ-ದಾ-ನದ
ವಿದ್ಯಾ-ದಾ-ನ-ಧಾರೆ
ವಿದ್ಯಾ-ದಾ-ನ-ಮಾಡಿ
ವಿದ್ಯಾ-ದಾ-ನ-ಮಾ-ಡಿ-ದ-ವರು
ವಿದ್ಯಾ-ದಾ-ನ
ವಿದ್ಯಾನ್
ವಿದ್ಯಾ-ಪ-ಕ್ಷ-ಪಾ-ತಿ-ಗ-ಳಾ-ಗಿ-ದ್ದರು
ವಿದ್ಯಾಪಿ
ವಿದ್ಯಾ-ಪ್ರ-ವಾ-ಹ-ದಿಂದ
ವಿದ್ಯಾ-ಪ್ರೇ-ಮಿ-ಯಾದ
ವಿದ್ಯಾ-ಬೋ-ಧ-ನೆ-ಯಲ್ಲಿ
ವಿದ್ಯಾ-ಭ್ಯಾಸ
ವಿದ್ಯಾ-ಭ್ಯಾ-ಸ-ಕ್ಕಾಗಿ
ವಿದ್ಯಾ-ಭ್ಯಾ-ಸಕ್ಕೂ
ವಿದ್ಯಾ-ಭ್ಯಾ-ಸಕ್ಕೆ
ವಿದ್ಯಾ-ಭ್ಯಾ-ಸದ
ವಿದ್ಯಾ-ಭ್ಯಾ-ಸ-ದಲ್ಲಿ
ವಿದ್ಯಾ-ಭ್ಯಾ-ಸ-ವನ್ನು
ವಿದ್ಯಾ-ರಣ್ಯ
ವಿದ್ಯಾ-ರ್ಜನೆ
ವಿದ್ಯಾ-ರ್ಜ-ನೆ-ಗಾಗಿ
ವಿದ್ಯಾ-ರ್ಜ-ನೆಗೆ
ವಿದ್ಯಾ-ರ್ಜ-ನೆಯ
ವಿದ್ಯಾ-ರ್ಥ-ಗ-ಳಿಗೂ
ವಿದ್ಯಾರ್ಥಿ
ವಿದ್ಯಾ-ರ್ಥಿ-ಗಳ
ವಿದ್ಯಾ-ರ್ಥಿ-ಗ-ಳಂತೂ
ವಿದ್ಯಾ-ರ್ಥಿ-ಗ-ಳಂತೆ
ವಿದ್ಯಾ-ರ್ಥಿ-ಗ-ಳ-ನ್ನ-ಲ್ಲದೇ
ವಿದ್ಯಾ-ರ್ಥಿ-ಗ-ಳನ್ನು
ವಿದ್ಯಾ-ರ್ಥಿ-ಗ-ಳನ್ನೂ
ವಿದ್ಯಾ-ರ್ಥಿ-ಗ-ಳ-ಲ್ಲದೇ
ವಿದ್ಯಾ-ರ್ಥಿ-ಗ-ಳಲ್ಲಿ
ವಿದ್ಯಾ-ರ್ಥಿ-ಗ-ಳ-ಲ್ಲಿ-ರು-ತ್ತದೆ
ವಿದ್ಯಾ-ರ್ಥಿ-ಗ-ಳಷ್ಟೇ
ವಿದ್ಯಾ-ರ್ಥಿ-ಗ-ಳಾದ
ವಿದ್ಯಾ-ರ್ಥಿ-ಗ-ಳಿಂದ
ವಿದ್ಯಾ-ರ್ಥಿ-ಗ-ಳಿಂ-ದಲೇ
ವಿದ್ಯಾ-ರ್ಥಿ-ಗ-ಳಿ-ಗಾಗಿ
ವಿದ್ಯಾ-ರ್ಥಿ-ಗ-ಳಿಗೂ
ವಿದ್ಯಾ-ರ್ಥಿ-ಗ-ಳಿಗೆ
ವಿದ್ಯಾ-ರ್ಥಿ-ಗ-ಳಿ-ಬ್ಬರು
ವಿದ್ಯಾ-ರ್ಥಿ-ಗಳು
ವಿದ್ಯಾ-ರ್ಥಿ-ಗಳೂ
ವಿದ್ಯಾ-ರ್ಥಿ-ಗ-ಳೆಲ್ಲ
ವಿದ್ಯಾ-ರ್ಥಿ-ಗ-ಳೆಲ್ಲಾ
ವಿದ್ಯಾ-ರ್ಥಿ-ಗಳೇ
ವಿದ್ಯಾ-ರ್ಥಿ-ಗ-ಳೊಂ-ದಿಗೆ
ವಿದ್ಯಾ-ರ್ಥಿಗೆ
ವಿದ್ಯಾ-ರ್ಥಿ-ಜೀ-ವನ
ವಿದ್ಯಾ-ರ್ಥಿ-ಜೀ-ವ-ನ-ವನ್ನು
ವಿದ್ಯಾ-ರ್ಥಿ-ದೆ-ಸೆ-ಯಲ್ಲಿ
ವಿದ್ಯಾ-ರ್ಥಿ-ದೆ-ಸೆ-ಯಲ್ಲೇ
ವಿದ್ಯಾ-ರ್ಥಿನಿ
ವಿದ್ಯಾ-ರ್ಥಿ-ನಿಗೆ
ವಿದ್ಯಾ-ರ್ಥಿ-ನಿ-ಯೊ-ಬ್ಬಳು
ವಿದ್ಯಾ-ರ್ಥಿ-ನಿ-ಲಯ
ವಿದ್ಯಾ-ರ್ಥಿ-ನಿ-ಲ-ಯಕ್ಕೆ
ವಿದ್ಯಾ-ರ್ಥಿ-ನಿ-ಲ-ಯ-ಗ-ಳಲ್ಲಿ
ವಿದ್ಯಾ-ರ್ಥಿ-ನಿ-ಲ-ಯದ
ವಿದ್ಯಾ-ರ್ಥಿ-ನಿ-ಲ-ಯ-ದಲ್ಲಿ
ವಿದ್ಯಾ-ರ್ಥಿ-ನಿ-ಲ-ಯ-ವನ್ನು
ವಿದ್ಯಾ-ರ್ಥಿಯ
ವಿದ್ಯಾ-ರ್ಥಿ-ಯನ್ನು
ವಿದ್ಯಾ-ರ್ಥಿ-ಯ-ವರು
ವಿದ್ಯಾ-ರ್ಥಿ-ಯಾಗಿ
ವಿದ್ಯಾ-ರ್ಥಿ-ಯಾ-ಗಿದ್ದ
ವಿದ್ಯಾ-ರ್ಥಿ-ಯಾ-ಗಿ-ದ್ದರು
ವಿದ್ಯಾ-ರ್ಥಿ-ಯಾ-ಗಿ-ದ್ದಾಗ
ವಿದ್ಯಾ-ರ್ಥಿ-ಯಾ-ಗಿದ್ದೆ
ವಿದ್ಯಾ-ರ್ಥಿ-ಯಾ-ಗುವ
ವಿದ್ಯಾ-ರ್ಥಿ-ಯಾದ
ವಿದ್ಯಾ-ರ್ಥಿ-ಯಾದೆ
ವಿದ್ಯಾ-ರ್ಥಿಯೂ
ವಿದ್ಯಾ-ರ್ಥಿ-ಯೆಂಬ
ವಿದ್ಯಾ-ರ್ಥಿಯೇ
ವಿದ್ಯಾ-ರ್ಥಿ-ಯೊಬ್ಬ
ವಿದ್ಯಾ-ರ್ಥಿ-ಯೊ-ಬ್ಬನು
ವಿದ್ಯಾ-ರ್ಥಿ-ವರ್ಗ
ವಿದ್ಯಾ-ರ್ಥಿ-ವೃಂ-ದಕ್ಕೆ
ವಿದ್ಯಾ-ರ್ಥಿ-ವೃಂ-ದ-ದ-ರ್ಶ-ನ-ಕಾಂ-ಕ್ಷಿ-ಗ-ಳೆಂ-ಬುದು
ವಿದ್ಯಾ-ರ್ಥಿ-ವೇ-ತ-ನಕ್ಕೆ
ವಿದ್ಯಾ-ರ್ಥಿ-ವೇ-ತ-ನವೂ
ವಿದ್ಯಾ-ರ್ಥಿ-ಸಂಘ
ವಿದ್ಯಾ-ರ್ಥಿ-ಸ-ಮೂಹ
ವಿದ್ಯಾ-ರ್ಹ-ತೆಯ
ವಿದ್ಯಾ-ರ್ಹ-ತೆ-ಯಿಂದ
ವಿದ್ಯಾ-ಲಯ
ವಿದ್ಯಾ-ಲ-ಯಕ್ಕೆ
ವಿದ್ಯಾ-ಲ-ಯದ
ವಿದ್ಯಾ-ಲ-ಯ-ದಲ್ಲಿ
ವಿದ್ಯಾ-ಲ-ಯ-ದಲ್ಲೆ
ವಿದ್ಯಾ-ಲ-ಯ-ದಿಂದ
ವಿದ್ಯಾ-ಲ-ಯ-ವ-ನ್ನಾಗಿ
ವಿದ್ಯಾ-ವಂ-ತ-ನಾದ
ವಿದ್ಯಾ-ವಂ-ತ-ರಾ-ಗ-ಬೇ-ಕೆಂಬ
ವಿದ್ಯಾ-ವಂ-ತ-ರಾಗಿ
ವಿದ್ಯಾ-ವಂ-ತರು
ವಿದ್ಯಾ-ವಿ-ತ-ರಣೆ
ವಿದ್ಯಾ-ವಿ-ನ-ಯ-ಸಂ-ಪ-ನ್ನರೂ
ವಿದ್ಯಾ-ಶಾ-ಖೆ-ಗ-ಳಲ್ಲಿ
ವಿದ್ಯಾ-ಶಾ-ಲೆ-ಯನ್ನು
ವಿದ್ಯಾ-ಸಂಸ್ಥೆ
ವಿದ್ಯಾ-ಸಂ-ಸ್ಥೆಯ
ವಿದ್ಯಾ-ಸಾ-ರ್ವ-ಭೌ-ಮ-ರಾದ
ವಿದ್ಯಾ-ಅ-ರ್ಥ-ದಾ-ನ-ಗ-ಳಿಂದ
ವಿದ್ಯುತ್
ವಿದ್ಯೆ
ವಿದ್ಯೆ-ಗ-ಳನ್ನು
ವಿದ್ಯೆ-ಗಳೂ
ವಿದ್ಯೆಗೆ
ವಿದ್ಯೆಯ
ವಿದ್ಯೆ-ಯನ್ನು
ವಿದ್ಯೆ-ಯನ್ನೂ
ವಿದ್ಯೆ-ಯಾ-ಗಲೀ
ವಿದ್ಯೆ-ಯಿಂದ
ವಿದ್ಯೆಯು
ವಿದ್ಯೋ-ಪ-ದೇ-ಶ-ಮಾಡಿ
ವಿದ್ಯೋ-ಪಾ-ಸಕ
ವಿದ್ವ-ಜ್ಜ-ನ-ಮಾ-ನ್ಯವೂ
ವಿದ್ವ-ಜ್ಜ-ನರ
ವಿದ್ವ-ಜ್ಜ-ನ-ರ-ನ್ನಾಗಿ
ವಿದ್ವ-ಜ್ಜ-ನರು
ವಿದ್ವತ್
ವಿದ್ವ-ತ್ತನ್ನು
ವಿದ್ವ-ತ್ತ-ರ-ಗ-ತಿಯ
ವಿದ್ವ-ತ್ತ-ರ-ಗ-ತಿ-ಯಲ್ಲಿ
ವಿದ್ವ-ತ್ತಿಗೆ
ವಿದ್ವ-ತ್ತಿನ
ವಿದ್ವತ್ತು
ವಿದ್ವ-ತ್ಪ-ಠ್ಯ-ಗಳ
ವಿದ್ವ-ತ್ಪೂರ್ಣ
ವಿದ್ವ-ತ್ಪ್ರೌ-ಢಿ-ಮೆ-ಯನ್ನು
ವಿದ್ವ-ತ್ಸ-ಮೂ-ಹ-ದಲ್ಲಿ
ವಿದ್ವ-ತ್ಸ-ಮ್ಮಾ-ನ-ಸ-ಮಾ-ರೋ-ಹ-ಗ-ಳಲ್ಲಿ
ವಿದ್ವ-ದು-ತ್ತಮಾ
ವಿದ್ವ-ದ್ಗೋಷ್ಠಿ
ವಿದ್ವ-ದ್ಗೋ-ಷ್ಠಿ-ಗ-ಳಲ್ಲಿ
ವಿದ್ವ-ದ್ಗೋಷ್ಠೀ
ವಿದ್ವ-ದ್ವ-ಲಯ
ವಿದ್ವ-ನ್ಮ-ಣಿ-ಗ-ಳಾದ
ವಿದ್ವ-ನ್ಮ-ಣಿ-ಯಾಗಿ
ವಿದ್ವ-ನ್ಮ-ಣಿ-ಯಾದ
ವಿದ್ವ-ನ್ಮ-ಧ್ಯಮಾ
ವಿದ್ವ-ನ್ಮಾ-ನ್ಯರು
ವಿದ್ವ-ನ್ಮಿ-ತ್ರರ
ವಿದ್ವಾಂ-ಸನು
ವಿದ್ವಾಂ-ಸನೂ
ವಿದ್ವಾಂ-ಸ-ನೆಂದು
ವಿದ್ವಾಂ-ಸರ
ವಿದ್ವಾಂ-ಸ-ರ-ನ್ನಾಗಿ
ವಿದ್ವಾಂ-ಸ-ರನ್ನು
ವಿದ್ವಾಂ-ಸ-ರಲ್ಲಿ
ವಿದ್ವಾಂ-ಸ-ರಲ್ಲು
ವಿದ್ವಾಂ-ಸ-ರ-ಲ್ಲೊಬ್ಬ
ವಿದ್ವಾಂ-ಸ-ರಾಗಿ
ವಿದ್ವಾಂ-ಸ-ರಾ-ಗಿ-ರುವ
ವಿದ್ವಾಂ-ಸ-ರಾದ
ವಿದ್ವಾಂ-ಸ-ರಾ-ದಿರಿ
ವಿದ್ವಾಂ-ಸ-ರಿಂದ
ವಿದ್ವಾಂ-ಸ-ರಿಗೂ
ವಿದ್ವಾಂ-ಸ-ರಿಗೆ
ವಿದ್ವಾಂ-ಸರು
ವಿದ್ವಾಂ-ಸ-ರು-ಗಳ
ವಿದ್ವಾಂ-ಸ-ರು-ಗಳು
ವಿದ್ವಾಂ-ಸರೂ
ವಿದ್ವಾಂ-ಸ-ರೆಂದೂ
ವಿದ್ವಾನ
ವಿದ್ವಾ-ನಪಿ
ವಿದ್ವಾನ್
ವಿದ್ವಾ-ನ್ಗಂ-ಗಾ-ಧರ
ವಿಧ-ಗ-ಳಿವೆ
ವಿಧದ
ವಿಧ-ದಲ್ಲಿ
ವಿಧ-ದಲ್ಲೂ
ವಿಧಾನ
ವಿಧಾ-ನ-ಸೌ-ಧಕ್ಕೆ
ವಿಧಾ-ನ-ಸೌ-ಧ-ದಲ್ಲಿ
ವಿಧಾ-ಯಕ
ವಿಧಿಯ
ವಿಧಿ-ವ-ದ್ಗು-ರೂ-ಪ-ದಿ-ಷ್ಟ-ವೇ-ದಾ-ವಿ-ರು-ದ್ಧ-ಮಂ-ತ್ರಾ-ಭ್ಯಾಸಃ
ವಿಧಿ-ವಿ-ಧಾ-ನ-ಗಳ
ವಿಧಿ-ವಿ-ಧಾ-ನ-ಗ-ಳನ್ನು
ವಿಧಿ-ವಿ-ಲಾ-ಸವೇ
ವಿಧಿ-ಸಿ-ರು-ವ-ವೆಂದೂ
ವಿಧಿ-ಸು-ತ್ತಾರೆ
ವಿಧಿ-ಸು-ವಂತೆ
ವಿಧೇಯ
ವಿಧ್ಯಾ-ಭಾಸ
ವಿಧ್ಯಾ-ಭ್ಯಾ-ಸದ
ವಿಧ್ಯಾ-ರ್ಥಿ-ಗ-ಳನ್ನು
ವಿಧ್ವಾಂ-ಸ-ರಾದ
ವಿನಂ-ತಿ-ಸಿ-ಕೊಂಡ
ವಿನಂ-ತಿ-ಸಿ-ಕೊ-ಳ್ಳು-ತ್ತೇನೆ
ವಿನಂ-ತಿ-ಸಿ-ದರು
ವಿನಂ-ತಿ-ಸಿದೆ
ವಿನಃ
ವಿನಯ
ವಿನ-ಯದ
ವಿನ-ಯ-ಪೂ-ರ್ವ-ಕ-ವಾಗಿ
ವಿನ-ಯ-ವಂ-ತಿಕೆ
ವಿನಾ
ವಿನಾ-ಯಕ
ವಿನಾ-ಯ-ಕನ
ವಿನಾ-ಯ-ಕ-ನಿಂದ
ವಿನಾ-ಶಕಃ
ವಿನಾ-ಶ-ದತ್ತ
ವಿನಿ-ಯೋ-ಗ-ವನ್ನು
ವಿನೋ-ದ-ಕಮ್
ವಿಪ-ರ್ಯಾಸ
ವಿಪ-ರ್ಯಾ-ಸವೇ
ವಿಪಿ
ವಿಫ-ಲ-ಸ-ಫಲ
ವಿಭಟ್ಟ
ವಿಭಾಗ
ವಿಭಾ-ಗ-ಗಳ
ವಿಭಾ-ಗ-ಗ-ಳನ್ನು
ವಿಭಾ-ಗ-ಗಳು
ವಿಭಾ-ಗದ
ವಿಭಾ-ಗ-ದಲ್ಲಿ
ವಿಭಾ-ಗ-ದಲ್ಲೂ
ವಿಭಾ-ಗ-ಮು-ಖ್ಯರು
ವಿಭಾ-ಗ-ವಿ-ರಲಿ
ವಿಭಾ-ಗಾ-ಧ್ಯ-ಕ್ಷ-ರಾ-ಗಿ-ದ್ದ-ವರು
ವಿಭಾ-ಗಾ-ಧ್ಯ-ಕ್ಷರು
ವಿಭಾ-ಗಿ-ಸ-ಬ-ಹು-ದಾ-ಗಿದೆ
ವಿಭಾ-ಗಿ-ಸು-ತ್ತಾರೆ
ವಿಭಾ-ಗೀಯ
ವಿಭಿನ್ನ
ವಿಭಿ-ನ್ನ-ರೀ-ತಿಯ
ವಿಭಿ-ನ್ನ-ವಾ-ಗಿ-ರು-ತ್ತದೆ
ವಿಭೂ-ಷಿ-ತ-ರಾದ
ವಿಭ್ರಂಶ
ವಿಮ-ರ್ಶನ
ವಿಮ-ರ್ಶಾ-ತ್ಮಕ
ವಿಮರ್ಶೆ
ವಿಮ-ರ್ಶೆಯ
ವಿಮಾನ
ವಿರ-ಲಾ-ಸ್ಸಂತಿ
ವಿರಳ
ವಿರ-ಳ-ರಲ್ಲೂ
ವಿರ-ಳ-ರೆಂದೇ
ವಿರ-ಳ-ವೆಂದೇ
ವಿರ-ಳವೇ
ವಿರ-ಳಾ-ತಿ-ವಿ-ರಳ
ವಿರ-ಳಾ-ವ-ಕಾ-ಶ-ವಿದ್ದ
ವಿರಾ-ಘ-ವ-ಕೆ-ಎಲ್
ವಿರಾ-ಜಿ-ಸು-ತ್ತಿ-ರು-ವುದು
ವಿರಾಮ
ವಿರಾ-ಮ-ಕೊ-ಡು-ತ್ತೇವೆ
ವಿರಾ-ಮದ
ವಿರುದ್ಧ
ವಿರು-ದ್ಧ-ತೆ-ಯನ್ನೂ
ವಿರು-ದ್ಧ-ವಾ-ದ-ಗ-ಳನ್ನು
ವಿರು-ದ್ಧ-ವಾ-ದ-ಗಳು
ವಿರು-ದ್ಧ-ವಾ-ದುದು
ವಿರು-ದ್ಧಾರ್ಥ
ವಿರೋಧ
ವಿರೋ-ಧದ
ವಿರೋ-ಧ-ವನ್ನು
ವಿರೋ-ಧಾ-ಭಿ-ಪ್ರಾ-ಯ-ಗ-ಳನ್ನು
ವಿರೋ-ಧಿ-ಯಾ-ಗದೇ
ವಿರೋ-ಧಿ-ಯಾದ
ವಿರೋ-ಧಿ-ಸಲು
ವಿರೋ-ಧಿಸಿ
ವಿರೋ-ಧಿ-ಸಿ-ದರು
ವಿರೋ-ಧಿ-ಸಿದ್ದೆ
ವಿರೋ-ಧಿ-ಸುವ
ವಿರೋ-ಧಿ-ಸು-ವುದು
ವಿಲಂಬಂ
ವಿಲಾಸ
ವಿಳಂಬ
ವಿಳಾಸ
ವಿಳಾ-ಸ-ವಾ-ಗಲಿ
ವಿವರ
ವಿವ-ರಣೆ
ವಿವ-ರ-ಣೆ-ಯನ್ನು
ವಿವ-ರಿ-ಸ-ಬೇ-ಕೆಂ-ದರು
ವಿವ-ರಿ-ಸಲು
ವಿವ-ರಿಸಿ
ವಿವ-ರಿ-ಸಿ-ದರು
ವಿವ-ರಿ-ಸಿದೆ
ವಿವ-ರಿ-ಸಿ-ದ್ದಾರೆ
ವಿವ-ರಿ-ಸಿದ್ದೆ
ವಿವ-ರಿ-ಸಿ-ಯೇನು
ವಿವ-ರಿ-ಸಿ-ರು-ವುದು
ವಿವ-ರಿ-ಸು-ತ್ತಿ-ದ್ದರು
ವಿವ-ರಿ-ಸು-ವು-ದಕ್ಕೆ
ವಿವ-ರಿ-ಸು-ವು-ದರ
ವಿವ-ರ್ಜ-ಯೇತ್
ವಿವಾ-ದ-ಕ್ಕೆ-ಡೆ-ಯಾ-ಗ-ದಂತೆ
ವಿವಾ-ದ-ದಿಂದ
ವಿವಾ-ದಾ-ತೀತ
ವಿವಾಹ
ವಿವಾ-ಹ-ಮ-ಹೋ-ತ್ಸವ
ವಿವಾ-ಹ-ಮೂ-ಲದ
ವಿವಾ-ಹ-ವಾಗಿ
ವಿವಾ-ಹ-ವಾ-ಗು-ವ-ವ-ರೆಗೂ
ವಿವಾ-ಹಿ-ತ-ರಾಗಿ
ವಿವಾ-ಹೋ-ತ್ತರ
ವಿವಾ-ಹ-ಕು-ಟುಂಬ
ವಿವಿಧ
ವಿವಿ-ಧ-ವಾಗಿ
ವಿವಿ-ಧ-ಶಾ-ಸ್ತ್ರ-ಗಳ
ವಿವಿರ
ವಿವಿ-ರಿಸಿ
ವಿವಿ-ರಿ-ಸುವ
ವಿವೇ-ಕಿ-ಗ-ಳಾ-ಗಿ-ದ್ದರು
ವಿವೇ-ಚನಾ
ವಿವೇ-ಚನೆ
ವಿವೇ-ಚ-ನೆ-ಯನ್ನು
ವಿವೇ-ಚ-ನೆ-ಯಲ್ಲಿ
ವಿವೇ-ಚ-ನೆ-ಯ-ಲ್ಲಿನ
ವಿವೇ-ಚಿ-ಸಿ-ದ-ವ-ರಿಗೆ
ವಿವೇ-ಚಿ-ಸು-ತ್ತಾರೆ
ವಿಶಃ
ವಿಶ-ದೀ-ಕ-ರಿ-ಸಿ-ರು-ತ್ತಾರೆ
ವಿಶಾ-ರ-ದರು
ವಿಶಾಲ
ವಿಶಾ-ಲ-ಗುಣ
ವಿಶಾ-ಲ-ವಾದ
ವಿಶಿಷ್ಟ
ವಿಶಿ-ಷ್ಟ-ಜ್ಞಾ-ನಾಂ-ತ-ರ-ಬೇಕು
ವಿಶಿ-ಷ್ಟರೂ
ವಿಶಿ-ಷ್ಟ-ವಾದ
ವಿಶಿ-ಷ್ಟ-ವ್ಯಕ್ತಿ
ವಿಶಿ-ಷ್ಟಾ-ಧ್ಯಾಯ
ವಿಶೆಷ
ವಿಶೇಷ
ವಿಶೇ-ಷ-ಗುಣ
ವಿಶೇ-ಷ-ಗು-ಣ-ವನ್ನು
ವಿಶೇ-ಷಣ
ವಿಶೇ-ಷತೆ
ವಿಶೇ-ಷ-ಯುಕ್ತಾ
ವಿಶೇ-ಷ-ವಾಗಿ
ವಿಶೇ-ಷ-ವಾದ
ವಿಶೇ-ಷವೇ
ವಿಶೇ-ಷ-ವೇ-ನಲ್ಲ
ವಿಶೇ-ಷಾ-ಧ್ಯ-ಯನ
ವಿಶೇ-ಷ್ಯ-ಭಾ-ವ-ದಲ್ಲಿ
ವಿಶ್ರಾಂತ
ವಿಶ್ರಾಂ-ತ-ಜೀ-ವನ
ವಿಶ್ರಾಂ-ತ-ರಾ-ಗಿ-ದ್ದರು
ವಿಶ್ರು-ತರೂ
ವಿಶ್ಲೇ-ಷಣಾ
ವಿಶ್ಲೇ-ಷಣೆ
ವಿಶ್ಲೇ-ಷಿ-ಸು-ತ್ತಿ-ದ್ದರು
ವಿಶ್ಲೇ-ಷಿ-ಸುವ
ವಿಶ್ವ
ವಿಶ್ವ-ಕೋಶ
ವಿಶ್ವಕ್ಕೆ
ವಿಶ್ವದ
ವಿಶ್ವ-ದಲ್ಲೇ
ವಿಶ್ವ-ನಾಥ
ವಿಶ್ವ-ಮಾ-ನವ
ವಿಶ್ವ-ರೂ-ಪದಿ
ವಿಶ್ವ-ವಿ-ದ್ಯಾ-ನಿ-ಲಯ
ವಿಶ್ವ-ವಿ-ದ್ಯಾ-ನಿ-ಲ-ಯಕ್ಕೆ
ವಿಶ್ವ-ವಿ-ದ್ಯಾ-ನಿ-ಲ-ಯದ
ವಿಶ್ವ-ವಿ-ದ್ಯಾ-ನಿ-ಲ-ಯ-ದಲ್ಲಿ
ವಿಶ್ವ-ವಿ-ದ್ಯಾ-ನಿ-ಲ-ಯ-ದಿಂದ
ವಿಶ್ವ-ವಿ-ದ್ಯಾ-ನಿ-ಲ-ಯ-ವನ್ನು
ವಿಶ್ವ-ವಿ-ದ್ಯಾ-ನಿ-ಲ-ಯವು
ವಿಶ್ವ-ವಿ-ದ್ಯಾ-ಲಯ
ವಿಶ್ವ-ವಿ-ದ್ಯಾ-ಲ-ಯದ
ವಿಶ್ವ-ವಿ-ದ್ಯಾ-ಲ-ಯ-ದಿಂದ
ವಿಶ್ವಾ-ಮಿತ್ರ
ವಿಶ್ವಾ-ಮಿ-ತ್ರಾ-ಹಿ-ಪ-ಶುಷು
ವಿಶ್ವಾಸ
ವಿಶ್ವಾ-ಸಕ್ಕೆ
ವಿಶ್ವಾ-ಸ-ಗಳ
ವಿಶ್ವಾ-ಸ-ದಿಂದ
ವಿಶ್ವಾ-ಸ-ದೃ-ಢ-ತೆ-ಯನ್ನೂ
ವಿಶ್ವಾ-ಸ-ವಿತ್ತು
ವಿಶ್ವಾ-ಸ-ವಿದೆ
ವಿಶ್ವಾ-ಸವೇ
ವಿಶ್ವಾ-ಸಿ-ಕ-ನಾ-ಗಿದ್ದ
ವಿಶ್ವಾ-ಸಿ-ಗಳು
ವಿಶ್ವೇ-ಶ್ವರ
ವಿಶ್ವೇ-ಶ್ವ-ರ-ದೀ-ಕ್ಷಿ-ತ-ರಲ್ಲಿ
ವಿಶ್ವೇ-ಶ್ವ-ರ-ದೀ-ಕ್ಷಿ-ತ-ರಲ್ಲೂ
ವಿಷಂ
ವಿಷಮ
ವಿಷಯ
ವಿಷ-ಯ-ಕ-ವಾಗಿ
ವಿಷ-ಯಕ್ಕೆ
ವಿಷ-ಯ-ಗಳ
ವಿಷ-ಯ-ಗ-ಳನ್ನು
ವಿಷ-ಯ-ಗ-ಳಲ್ಲಿ
ವಿಷ-ಯ-ಗ-ಳ-ಲ್ಲಿಯೂ
ವಿಷ-ಯ-ಗ-ಳಲ್ಲೂ
ವಿಷ-ಯ-ಗ-ಳಿಂ-ದಲೇ
ವಿಷ-ಯ-ಗ-ಳಿಗೆ
ವಿಷ-ಯ-ಗ-ಳಿಲ್ಲ
ವಿಷ-ಯ-ಗಳು
ವಿಷ-ಯ-ಗಳೂ
ವಿಷ-ಯ-ಗ್ರ-ಹಿ-ಕೆಯೇ
ವಿಷ-ಯ-ಜ್ಞಾನ
ವಿಷ-ಯದ
ವಿಷ-ಯ-ದ-ಲ್ಲಾ-ದರೂ
ವಿಷ-ಯ-ದಲ್ಲಿ
ವಿಷ-ಯ-ದಲ್ಲೂ
ವಿಷ-ಯ-ದಲ್ಲೇ
ವಿಷ-ಯ-ದಿ-ನ-ಕ-ರೀದ
ವಿಷ-ಯ-ಮಂ-ಡನೆ
ವಿಷ-ಯ-ವನ್ನ
ವಿಷ-ಯ-ವನ್ನು
ವಿಷ-ಯ-ವನ್ನೂ
ವಿಷ-ಯ-ವನ್ನೊ
ವಿಷ-ಯ-ವಾ-ಗಿತ್ತು
ವಿಷ-ಯ-ವಾ-ಗಿದೆ
ವಿಷ-ಯ-ವಾ-ಗಿದ್ದ
ವಿಷ-ಯ-ವಾ-ಗಿ-ರಲಿ
ವಿಷ-ಯ-ವಾ-ದರು
ವಿಷ-ಯ-ವಿ-ದ್ದರೂ
ವಿಷ-ಯವೂ
ವಿಷ-ಯ-ವೆಂ-ದರೆ
ವಿಷ-ಯವೇ
ವಿಷ-ಯಾ-ಧಾ-ರಿತ
ವಿಷ-ಯಾ-ನು-ಗು-ಣ-ವಾಗಿ
ವಿಷ-ಯಾನ್
ವಿಷ-ವನ್ನು
ವಿಷ-ವು-ಣಿ-ಸುವ
ವಿಷಾ-ದದ
ವಿಷಾ-ದ-ನೀಯ
ವಿಷಾ-ದಿ-ಸು-ತ್ತಿ-ದ್ದರು
ವಿಷೇಶ
ವಿಷೇ-ಶ-ವಾಗಿ
ವಿಷ್ಣು-ಷ-ಟ್ಪ-ದಿಯ
ವಿಷ್ಣು-ಸ-ಹ-ಸ್ರ-ನಾಮ
ವಿಸ್ತ-ರ-ಣೆಯ
ವಿಸ್ತ-ರಿ-ಸ-ಬೇ-ಕೆಂಬ
ವಿಸ್ತಾ-ರ-ರೂ-ಪದ
ವಿಸ್ತಾ-ರ-ವನ್ನು
ವಿಸ್ತಾ-ರ-ವನ್ನೇ
ವಿಸ್ತಾ-ರ-ವಾದ
ವಿಸ್ತಾ-ರ-ವಾ-ದುದು
ವಿಸ್ತೃತ
ವಿಸ್ಮಿ-ತ-ರ-ನ್ನಾಗಿ
ವಿಹಿ-ತ-ವಾ-ಗಿ-ರ-ಬೇಕು
ವಿಹಿ-ತವೂ
ವಿಹಿ-ತವೇ
ವಿ-ಹಕ್ಕಿ
ವೀರ-ರ-ಸವೂ
ವೀರೇಂದ್ರ
ವೀರೋ
ವೃಂದ
ವೃಂದಕ್ಕೂ
ವೃಂದ-ಗಳು
ವೃಕ್ಷ-ಗ-ಳಂತೆ
ವೃಕ್ಷಾಃ
ವೃತ್ತ
ವೃತ್ತಾಂತ
ವೃತ್ತಾಂ-ತದ
ವೃತ್ತಾಂ-ತ-ವನ್ನು
ವೃತ್ತಾಂ-ತ-ವಿಲ್ಲ
ವೃತ್ತಿ
ವೃತ್ತಿ-ಗ-ಳನ್ನು
ವೃತ್ತಿ-ಗ-ಳಿವೆ
ವೃತ್ತಿ-ಗಳು
ವೃತ್ತಿಗೆ
ವೃತ್ತಿ-ಜೀ-ವ-ನ-ದಿಂದ
ವೃತ್ತಿ-ಜೀ-ವ-ನ-ವನ್ನು
ವೃತ್ತಿ-ಜೀ-ವ-ನ-ವೆ-ರ-ಡ-ರಲ್ಲೂ
ವೃತ್ತಿ-ನಿ-ವೃ-ತ್ತ-ರಾದ
ವೃತ್ತಿಯ
ವೃತ್ತಿ-ಯನ್ನು
ವೃತ್ತಿ-ಯಲ್ಲಿ
ವೃತ್ತಿ-ಯಲ್ಲೇ
ವೃತ್ತಿ-ಯಿಂದ
ವೃತ್ತಿಯೇ
ವೃಥಾ
ವೃದ್ಧಾ-ಪ್ಯ-ದಲ್ಲಿ
ವೃದ್ಧಿ-ಸಿ-ಕೊಂ-ಡಿರಿ
ವೆಂಕ-ಟ-ರ-ಮಣ
ವೆಂಕ-ಟ-ರ-ಮ-ಣ-ಸ್ವಾಮಿ
ವೆಂಕ-ಟಾ-ಚಾರ್ಯ
ವೆಂಕ-ಣ್ಣಾ-ಚಾ-ರ್ಯರ
ವೆಂಕ-ಣ್ಣಾ-ಚಾ-ರ್ಯ-ರಂ-ತಹ
ವೆಂಕ-ಣ್ಣಾ-ಚಾ-ರ್ಯ-ರನ್ನು
ವೆಂಕ-ಣ್ಣಾ-ಚಾ-ರ್ಯರು
ವೆಚ್ಚ
ವೆದಾಂ-ತಾ-ದಿ-ದ-ರ್ಶ-ನ-ಗಳು
ವೇ
ವೇಗವೂ
ವೇಟ್
ವೇತನ
ವೇತ-ನ-ವೆಲ್ಲ
ವೇತ-ನಾ-ನು-ದಾ-ನವು
ವೇದ
ವೇದಕ್ಕೆ
ವೇದ-ಗಳ
ವೇದ-ಘೋಷ
ವೇದ-ಘೋ-ಷದ
ವೇದದ
ವೇದ-ದೊಂ-ದಿಗೆ
ವೇದ-ಭಾ-ಷ್ಯ-ಭೂ-ಮಿ-ಕೆ-ಯಲ್ಲಿ
ವೇದ-ಮಂತ್ರ
ವೇದ-ಮೂರ್ತಿ
ವೇದ-ವನ್ನು
ವೇದ-ವ್ಯಾಸ
ವೇದ-ಶಾಸ್ತ್ರ
ವೇದ-ಶಾ-ಸ್ತ್ರ-ಗ-ಳನ್ನು
ವೇದ-ಶಾ-ಸ್ತ್ರ-ಪೋ-ಷಿಣೀ
ವೇದ-ಸಂ-ವಾ-ದಿ-ಯಾದ
ವೇದಾಂತ
ವೇದಾಂ-ತದ
ವೇದಾಂ-ತ-ವನ್ನು
ವೇದಾಃ
ವೇದಾ-ದ್ಯ-ಯ-ನ-ಕ್ಕಾಗಿ
ವೇದಾ-ಧಿ-ಗಮಃ
ವೇದಾ-ಧ್ಯ-ಯನ
ವೇದಾ-ಧ್ಯ-ಯ-ನಕ್ಕೆ
ವೇದಾ-ಧ್ಯ-ಯ-ನ-ಕ್ಕೆಂದು
ವೇದಾ-ಧ್ಯ-ಯ-ನದ
ವೇದಾ-ಧ್ಯ-ಯ-ನ-ವನ್ನು
ವೇದಾ-ವತಿ
ವೇದಾ-ವತೀ
ವೇದಿಕೆ
ವೇದಿ-ಕೆಯ
ವೇದಿ-ಕೆ-ಯಲ್ಲೂ
ವೇದೋಕ್ತ
ವೇದ್ಯ-ವಾದ
ವೇದ-ಶಾ-ಸ್ತ್ರ-ಗ-ಳನ್ನು
ವೇಧ-ಶಾ-ಸ್ತ್ರ-ಗಳ
ವೇಬ್ರಶ್ರೀ
ವೇಳಗೆ
ವೇಳೆ
ವೇಳೆ-ಗಾ-ಗಲೆ
ವೇಳೆಗೆ
ವೇಳೆ-ಯಲ್ಲಿ
ವೇಷ-ಭೂ-ಷ-ಣ-ಗಳ
ವೈ
ವೈಕ-ಲ್ಯ-ವಿಲ್ಲ
ವೈಜ-ಯಂತಿ
ವೈಜ್ಞಾ-ನಿಕ
ವೈಜ್ಞಾ-ನಿ-ಕ-ತೆ-ಯನ್ನು
ವೈದಿಕ
ವೈದಿ-ಕ-ರಾಗಿ
ವೈದಿ-ಕ-ರಾ-ಗಿ-ದ್ದ-ವರು
ವೈದಿ-ಕ-ರಿ-ಗಾ-ರಿಗೂ
ವೈದಿ-ಕರು
ವೈದಿ-ಕ-ರೆ-ನಿ-ಸಿ-ಕೊಂ-ಡಿ-ದ್ದರು
ವೈದೀಕ
ವೈದು-ಷ್ಯಕ್ಕೆ
ವೈದ್ಯ
ವೈದ್ಯ-ಕೀಯ
ವೈದ್ಯನ
ವೈದ್ಯ-ನನ್ನು
ವೈದ್ಯ-ನಿಗೆ
ವೈದ್ಯ-ನೆಂಬ
ವೈದ್ಯನೇ
ವೈದ್ಯರ
ವೈದ್ಯ-ರಲ್ಲಿ
ವೈದ್ಯ-ರಾ-ದರೂ
ವೈದ್ಯ-ರಿಗೆ
ವೈದ್ಯರು
ವೈದ್ಯರೂ
ವೈದ್ಯ-ವೃತ್ತಿ
ವೈದ್ಯಾಶ್ಚ
ವೈದ್ಯೋ
ವೈಭ-ವೀ-ಕ-ರಿ-ಸು-ತ್ತಿ-ರು-ವು-ದನ್ನು
ವೈಯ-ಕ್ತಿಕ
ವೈಯ-ಕ್ತಿ-ಕ-ವಾಗಿ
ವೈಯ-ಕ್ತಿ-ಕ-ವಾ-ಗಿಯೂ
ವೈವಾ-ಹಿಕ
ವೈವಿ-ಧ್ಯಕ್ಕೆ
ವೈವಿ-ಧ್ಯ-ಮಯ
ವೈಶ-ಮ್ಯ-ದಿಂದ
ವೈಶಾಲ್ಯ
ವೈಶಾ-ಲ್ಯದ
ವೈಶಿಷ್ಟ್ಯ
ವೈಶಿ-ಷ್ಟ್ಯ-ಶ್ರೇ-ಣಿ-ಯನ್ನು
ವೈಶಿಷ್ಠ್ಯ
ವೈಶೇ-ಷಿ-ಕ-ದ-ರ್ಶ-ನದ
ವ್ಯಂಗ್ಯ-ಪೂ-ರಿ-ತ-ವಾದ
ವ್ಯಕ್ತ-ಗೊ-ಳಿ-ಸಲು
ವ್ಯಕ್ತ-ಗೊ-ಳಿ-ಸಿ-ದ್ದಿದೆ
ವ್ಯಕ್ತ-ಪ-ಡಿ-ಸಿ-ದಾಗ
ವ್ಯಕ್ತ-ಪ-ಡಿ-ಸಿ-ದ್ದೇ-ನಷ್ಟೆ
ವ್ಯಕ್ತ-ಪ-ಡಿ-ಸು-ತ್ತಿ-ದ್ದು-ದನ್ನು
ವ್ಯಕ್ತ-ರೂ-ಪವೇ
ವ್ಯಕ್ತಾ-ಯಾಂ
ವ್ಯಕ್ತಿ
ವ್ಯಕ್ತಿ-ಗತ
ವ್ಯಕ್ತಿ-ಗಳ
ವ್ಯಕ್ತಿ-ಗ-ಳನ್ನು
ವ್ಯಕ್ತಿ-ಗ-ಳಲ್ಲಿ
ವ್ಯಕ್ತಿ-ಗ-ಳಿಗೆ
ವ್ಯಕ್ತಿಗೆ
ವ್ಯಕ್ತಿತ್ವ
ವ್ಯಕ್ತಿ-ತ್ವಕ್ಕೆ
ವ್ಯಕ್ತಿ-ತ್ವ-ಗ-ಳಿಂದ
ವ್ಯಕ್ತಿ-ತ್ವದ
ವ್ಯಕ್ತಿ-ತ್ವ-ದಿಂದ
ವ್ಯಕ್ತಿ-ತ್ವ-ವನ್ನು
ವ್ಯಕ್ತಿ-ತ್ವವೇ
ವ್ಯಕ್ತಿ-ತ್ವ-ಹೊಂ-ದಿದ
ವ್ಯಕ್ತಿಯ
ವ್ಯಕ್ತಿ-ಯಲ್ಲಿ
ವ್ಯಕ್ತಿ-ಯಾಗಿ
ವ್ಯಕ್ತಿ-ಯಾ-ಗಿ-ದ್ದಾನೆ
ವ್ಯಕ್ತಿ-ಯಾ-ಗಿ-ದ್ದಾರೆ
ವ್ಯಕ್ತಿ-ಯಾ-ದರೂ
ವ್ಯಕ್ತಿ-ಯೊಬ್ಬ
ವ್ಯಕ್ತಿ-ಯೋರ್ವ
ವ್ಯಗ್ರ-ವಾದ
ವ್ಯತ್ಯಾ-ಸ-ಮಾ-ಡುತ್ತಾ
ವ್ಯತ್ಯಾ-ಸ-ವಾ-ಗಿ-ಬಿ-ಡು-ವು-ದನ್ನು
ವ್ಯಪ್ತಿ-ಯಲ್ಲಿ
ವ್ಯಯಿಸಿ
ವ್ಯಯಿ-ಸುವ
ವ್ಯರ್ಥ-ವೆಂದು
ವ್ಯವ-ಧಾನ
ವ್ಯವ-ವ-ಹಾರ
ವ್ಯವ-ಸ್ಥಾ-ಪ-ಕರು
ವ್ಯವ-ಸ್ಥಿತ
ವ್ಯವಸ್ಥೆ
ವ್ಯವ-ಸ್ಥೆ-ಗ-ಳನ್ನೂ
ವ್ಯವ-ಸ್ಥೆ-ಗೊ-ಳಿ-ಸಿ-ದ್ದಾರೆ
ವ್ಯವ-ಸ್ಥೆಯ
ವ್ಯವ-ಸ್ಥೆ-ಯನ್ನು
ವ್ಯವ-ಸ್ಥೆ-ಯನ್ನೂ
ವ್ಯವ-ಸ್ಥೆ-ಯನ್ನೇ
ವ್ಯವ-ಸ್ಥೆ-ಯಾ-ಗ-ದಿ-ದ್ದಾಗ
ವ್ಯವ-ಸ್ಥೆ-ಯಾ-ಯಿತು
ವ್ಯವ-ಸ್ಥೆ-ಯಾ-ಯಿತೊ
ವ್ಯವ-ಸ್ಥೆ-ಯಿಂದ
ವ್ಯವ-ಸ್ಥೆ-ಯಿಂ-ದಲೇ
ವ್ಯವ-ಸ್ಥೆ-ಯಿತ್ತು
ವ್ಯವ-ಸ್ಥೆಯೇ
ವ್ಯವ-ಹ-ರಿ-ಸ-ಬೇ-ಕೆಂದು
ವ್ಯವ-ಹ-ರಿ-ಸುವ
ವ್ಯವ-ಹ-ರಿ-ಸು-ವ-ವ-ರಲ್ಲ
ವ್ಯವ-ಹ-ರಿ-ಸು-ವ-ವರು
ವ್ಯವ-ಹಾರ
ವ್ಯವ-ಹಾ-ರಕ್ಕೆ
ವ್ಯವ-ಹಾ-ರ-ಗ-ಳಲ್ಲಿ
ವ್ಯವ-ಹಾ-ರ-ಗ-ಳಿಗೂ
ವ್ಯವ-ಹಾ-ರ-ದಲ್ಲಿ
ವ್ಯವ-ಹಾ-ರ-ದಿಂದ
ವ್ಯಷೇಮ
ವ್ಯಸ್ತ-ಚಿ-ತ್ತ-ರಾ-ಗದೇ
ವ್ಯಸ್ತ-ರಾಗಿ
ವ್ಯಸ್ತ-ರಾ-ಗಿ-ದ್ದರು
ವ್ಯಾಕ-ರಣ
ವ್ಯಾಕ-ರ-ಣ-ಗಳ
ವ್ಯಾಕ-ರ-ಣದ
ವ್ಯಾಕ-ರ-ಣ-ವನ್ನು
ವ್ಯಾಕ-ರ-ಣ-ವಿ-ರಲಿ
ವ್ಯಾಕ-ರ-ಣಾದಿ
ವ್ಯಾಖ್ಯಾನ
ವ್ಯಾಖ್ಯಾ-ನದ
ವ್ಯಾಖ್ಯಾ-ನ-ವನ್ನು
ವ್ಯಾಖ್ಯಾ-ನಿ-ಸಲು
ವ್ಯಾಖ್ಯಾ-ನಿ-ಸುತ್ತಾ
ವ್ಯಾಜ್ಯ
ವ್ಯಾಧಿ-ಗಳ
ವ್ಯಾಪ-ಕ-ವಾಗಿ
ವ್ಯಾಪಾ-ರದ
ವ್ಯಾಪ್ತಿ
ವ್ಯಾಪ್ತಿಃ
ವ್ಯಾಪ್ತಿಗೆ
ವ್ಯಾಪ್ತಿ-ಗ್ರಂ-ಥ-ಗಳ
ವ್ಯಾಪ್ತಿ-ಜ್ಞಾನ
ವ್ಯಾಪ್ತಿಯ
ವ್ಯಾಪ್ತಿ-ಯಲ್ಲಿ
ವ್ಯಾಪ್ತಿ-ಯಾ-ಗಿದೆ
ವ್ಯಾಮೋ-ಹ-ಕ್ಕೊ-ಳ-ಗಾದ
ವ್ಯಾಯಾಮ
ವ್ಯಾಯಾ-ಮ-ದಿಂದ
ವ್ಯಾವ-ಹಾ-ರಿಕ
ವ್ಯಾವ-ಹಾ-ರಿ-ಕ-ವಾಗಿ
ವ್ಯಾವ-ಹಾ-ರಿ-ಕ-ವಾ-ಗಿಯೂ
ವ್ಯಾಸಂಗ
ವ್ಯಾಸಂ-ಗ-ಕ್ಕಾಗಿ
ವ್ಯಾಸಂ-ಗ-ಮಾ-ಡುವ
ವ್ಯಾಸಂ-ಗ-ವನ್ನು
ವ್ಯಾಸರು
ವ್ಯಾಸ-ವಾ-ಲ್ಮೀ-ಕಿ-ಯರು
ವ್ಯುತ್ಪತ್ತಿ
ವ್ಯುತ್ಪಾ-ದ-ಕ-ವಾ-ಗಿ-ರು-ವು-ದಿಲ್ಲ
ವ್ರತ
ವ್ರತ-ದಂತೆ
ಶಂಕಯಾ
ಶಂಕರ
ಶಂಕ-ರ-ಮಠ
ಶಂಕ-ರ-ಮ-ಠದ
ಶಂಕ-ರರ
ಶಂಕ-ರರು
ಶಂಕ-ರ-ವಿ-ಲಾಸ
ಶಂಕ-ರಾ-ಚಾ-ರ್ಯರ
ಶಂಕ-ರಾ-ಚಾ-ರ್ಯರು
ಶಂಕರ್
ಶಂಕಾ
ಶಂಕೆ
ಶಂಖ-ಗ-ಳನ್ನು
ಶಕ್ತಿ
ಶಕ್ತಿಃ
ಶಕ್ತಿಯ
ಶಕ್ತಿ-ಯನ್ನು
ಶಕ್ತಿ-ಯನ್ನೇ
ಶಕ್ತಿ-ಯಾಗಿ
ಶಕ್ತಿ-ಯಿಂದ
ಶಕ್ತಿ-ಯಿಂ-ದಲೂ
ಶಕ್ತಿ-ಯಿದೆ
ಶಕ್ತಿ-ಯಿ-ರುವ
ಶಕ್ತಿಯು
ಶಕ್ತಿ-ವಿ-ಶಿ-ಷ್ಟಾ-ದ್ವೈ-ತ-ವೇ-ದಾಂ-ತದ
ಶಕ್ತಿ-ವಿ-ಶೇ-ಷ-ವನ್ನು
ಶತ-ಮಾ-ನ-ಗಳ
ಶತ-ಮಾ-ನೋ-ತ್ಸವ
ಶತ-ಮಾ-ನೋ-ತ್ಸ-ವದ
ಶತ-ಶಾ-ರದಃ
ಶತಾಬ್ದಿ
ಶನಿ-ವಾರ
ಶಬ್ದ
ಶಬ್ದಕ್ಕೆ
ಶಬ್ದ-ಗ-ಳನ್ನು
ಶಬ್ದ-ಗ-ಳಲ್ಲಿ
ಶಬ್ದ-ಗ-ಳೊಂ-ದಿಗೆ
ಶಬ್ದ-ಝರಿ
ಶಬ್ದ-ತ-ರಂ-ಗ-ಗಳು
ಶಬ್ದದ
ಶಬ್ದ-ಭಾಗ
ಶಬ್ದ-ರೂಪ
ಶಬ್ದವು
ಶಬ್ದವೇ
ಶಬ್ದ-ವೇ-ದಿಯೊ
ಶಬ್ದ-ಶ-ಕ್ತಿ-ಪ್ರ-ಕಾ-ಶಿ-ಕಾದ
ಶಬ್ದಾ-ರ್ಥ-ವಾ-ಗಿ-ದ್ದರೂ
ಶಮ-ನ-ವಾಗಿ
ಶಮ-ದ-ಮ-ತಿ-ತಿ-ಕ್ಷಾ-ಉ-ಪ-ರತಿ
ಶಯನ
ಶಯ-ನ-ವಾಯ್ತು
ಶರ-ಣಾ-ಗ-ಲೇ-ಬೇಕು
ಶರಣು
ಶರೀ-ರದ
ಶರೀ-ರಮ್
ಶರೀ-ರವು
ಶರೀ-ರಾ-ಕೃ-ತಿ-ಯಾ-ಗಲಿ
ಶರ್ಮಾ
ಶಲಾ-ಕಾ-ಸ್ಪ-ರ್ಧೆಗೆ
ಶಲ್ಯ-ವನ್ನು
ಶಸ್ತಾ-ನ್ನಾಶೀ
ಶಸ್ತ್ರ
ಶಸ್ತ್ರ-ಕ್ರಿ-ಯೆ-ಗಾಗಿ
ಶಾಂಕ-ರ-ದ-ರ್ಶ-ನ-ದಲ್ಲಿ
ಶಾಂಡಿ-ಲ್ಯೋ-ಪ-ನಿ-ಷ-ತ್ತಿ-ನಲ್ಲಿ
ಶಾಂತಲ
ಶಾಂತ-ವಾಗಿ
ಶಾಂತಿ
ಶಾಂತಿ-ದೂತ
ಶಾಕಂ
ಶಾಖಾ
ಶಾಖಾ-ಧ್ಯಾ-ಯಿ-ಗ-ಳಾದ
ಶಾಖೆ
ಶಾಖೆ-ಗಳು
ಶಾಬ್ಧ-ಬೋಧ
ಶಾರದಾ
ಶಾರ-ದಾಂಬೆ
ಶಾರ-ದಾಂ-ಬೆಯ
ಶಾರ-ದಾ-ವಿ-ಲಾಸ
ಶಾರ-ದೆಯ
ಶಾರ-ದೇ-ನಿನ್ನ
ಶಾರೀ-ರಿ-ಕ-ವಾ-ಗಲಿ
ಶಾಲಾ
ಶಾಲಾ-ಕಾ-ಲೇ-ಜು-ಗಳ
ಶಾಲು-ಹೊ-ದಿಸಿ
ಶಾಲೆ
ಶಾಲೆ-ಗಳ
ಶಾಲೆಗೆ
ಶಾಲೆಯ
ಶಾಲೆ-ಯನ್ನು
ಶಾಲೆ-ಯ-ಲ್ಲಷ್ಟೇ
ಶಾಲೆ-ಯಲ್ಲಿ
ಶಾಲೆ-ಯಲ್ಲೂ
ಶಾಲೆ-ಯೊಂ-ದ-ರಲ್ಲಿ
ಶಾಶ್ವತಃ
ಶಾಶ್ವ-ತ-ದತ್ತ
ಶಾಶ್ವ-ತ-ಸ್ಯೋ-ಪ-ಲ-ಬ್ಧಿಃ
ಶಾಸ-ನ-ತಜ್ಞ
ಶಾಸ್ತ್ರ
ಶಾಸ್ತ್ರಕ್ಕೆ
ಶಾಸ್ತ್ರ-ಗಂ-ಧ-ವಿ-ಲ್ಲದ
ಶಾಸ್ತ್ರ-ಗಳ
ಶಾಸ್ತ್ರ-ಗ-ಳನ್ನು
ಶಾಸ್ತ್ರ-ಗ-ಳನ್ನೂ
ಶಾಸ್ತ್ರ-ಗ-ಳಲ್ಲಿ
ಶಾಸ್ತ್ರ-ಗ-ಳಿಗೆ
ಶಾಸ್ತ್ರ-ಗಳು
ಶಾಸ್ತ್ರ-ಗ-ಳೆಂಬ
ಶಾಸ್ತ್ರ-ಗು-ರು-ಗ-ಳನ್ನು
ಶಾಸ್ತ್ರ-ಗ್ರಂ-ಥ-ಗಳ
ಶಾಸ್ತ್ರ-ಗ್ರಂ-ಥ-ವನ್ನು
ಶಾಸ್ತ್ರ-ಚಿಂ-ತ-ನೆ-ಯಲ್ಲಿ
ಶಾಸ್ತ್ರ-ಚಿಂ-ತ-ನೆ-ಯಿ-ಲ್ಲ-ದಿ-ದ್ದರೆ
ಶಾಸ್ತ್ರ-ಜ್ಞ-ರಾಗಿ
ಶಾಸ್ತ್ರ-ಜ್ಞರು
ಶಾಸ್ತ್ರ-ಜ್ಞ-ರೂ-ವೈ-ದಿ-ಕರೂ
ಶಾಸ್ತ್ರ-ಜ್ಞಾ-ನ-ಕ್ಕಾಗಿ
ಶಾಸ್ತ್ರದ
ಶಾಸ್ತ್ರ-ದಲ್ಲಿ
ಶಾಸ್ತ್ರ-ದೊ-ಲವ
ಶಾಸ್ತ್ರ-ಪಂ-ಕ್ತಿ-ಗಳ
ಶಾಸ್ತ್ರ-ಪಂ-ಡಿ-ತ-ರ-ನ್ನಾಗಿ
ಶಾಸ್ತ್ರ-ಪ-ರಂ-ಪ-ರೆಯ
ಶಾಸ್ತ್ರ-ಪ-ರಂ-ಪ-ರೆ-ಯಿಂದ
ಶಾಸ್ತ್ರ-ಪ-ರಂ-ಪ-ರೆ-ಸಂ-ಸ್ಕೃ-ತ-ವಿ-ದ್ಯಾ-ಭ್ಯಾಸ
ಶಾಸ್ತ್ರ-ಪಾಂ-ಡಿತ್ಯ
ಶಾಸ್ತ್ರ-ಪಾ-ಠ-ಕ್ಕಾಗಿ
ಶಾಸ್ತ್ರ-ಪಾ-ಠಕ್ಕೆ
ಶಾಸ್ತ್ರ-ಪಾ-ಠ-ವನ್ನು
ಶಾಸ್ತ್ರ-ಪಾ-ಠ-ವನ್ನೂ
ಶಾಸ್ತ್ರ-ಪೂ-ತ-ವಾದ
ಶಾಸ್ತ್ರ-ಪ್ರ-ವ-ರ್ತನ
ಶಾಸ್ತ್ರ-ಪ್ರ-ಸ್ತಾ-ವ-ವನ್ನು
ಶಾಸ್ತ್ರ-ಬ-ಲ್ಲ-ವರು
ಶಾಸ್ತ್ರ-ಬೋ-ಧ-ನೆಯ
ಶಾಸ್ತ್ರ-ಭೇ-ದ-ವಿ-ಲ್ಲದೇ
ಶಾಸ್ತ್ರ-ಮ-ಥನ
ಶಾಸ್ತ್ರ-ಮ-ಧು-ವನ್ನು
ಶಾಸ್ತ್ರಮ್
ಶಾಸ್ತ್ರ-ವನ್ನು
ಶಾಸ್ತ್ರ-ವಾ-ಕ್ಯ-ಗಳು
ಶಾಸ್ತ್ರ-ವಿ-ಷ-ಯ-ವೆಂ-ದರೆ
ಶಾಸ್ತ್ರ-ವೆಂ-ದರೆ
ಶಾಸ್ತ್ರ-ವೇ-ತ್ತ-ರಾಗಿ
ಶಾಸ್ತ್ರ-ಸಂ-ಕು-ಲ-ದಲ್ಲಿ
ಶಾಸ್ತ್ರ-ಸಂ-ವಾ-ದ-ದಿಂದ
ಶಾಸ್ತ್ರ-ಸಂ-ಸ್ಕಾ-ರ-ದಿಂದ
ಶಾಸ್ತ್ರಾಂ-ತರ
ಶಾಸ್ತ್ರಾ-ಧ್ಯ-ಯನ
ಶಾಸ್ತ್ರಾ-ಧ್ಯ-ಯ-ನಕ್ಕೆ
ಶಾಸ್ತ್ರಾ-ಧ್ಯ-ಯ-ನದ
ಶಾಸ್ತ್ರಾ-ಧ್ಯ-ಯ-ನ-ವನ್ನು
ಶಾಸ್ತ್ರಾ-ಧ್ಯಾ-ಪ-ನ-ದಲ್ಲಿ
ಶಾಸ್ತ್ರಾ-ನ್ಯ-ದೀ-ತ್ಯಾಪಿ
ಶಾಸ್ತ್ರಾರ್ಥ
ಶಾಸ್ತ್ರಿ-ಗಳ
ಶಾಸ್ತ್ರಿ-ಗ-ಳಲ್ಲಿ
ಶಾಸ್ತ್ರಿ-ಗಳು
ಶಾಸ್ತ್ರಿ-ಯ-ವರ
ಶಾಸ್ತ್ರೀ
ಶಾಸ್ತ್ರೀಯ
ಶಾಸ್ತ್ರೀ-ಯ-ವಾಗಿ
ಶಾಸ್ತ್ರೀ-ಯ-ವಾದ
ಶಾಸ್ತ್ರೀ-ಯ-ವಿ-ಷ-ಯ-ವನ್ನೂ
ಶಿಕ್ಷಕ
ಶಿಕ್ಷ-ಕನ
ಶಿಕ್ಷ-ಕ-ನಾದ
ಶಿಕ್ಷ-ಕ-ನಾದೆ
ಶಿಕ್ಷ-ಕ-ನಿ-ಗಲ್ಲ
ಶಿಕ್ಷ-ಕ-ನಿಗೂ
ಶಿಕ್ಷ-ಕ-ನೆ-ನಿ-ಸು-ತ್ತಾನೆ
ಶಿಕ್ಷ-ಕನೇ
ಶಿಕ್ಷ-ಕರ
ಶಿಕ್ಷ-ಕ-ರನ್ನು
ಶಿಕ್ಷ-ಕ-ರಲ್ಲಿ
ಶಿಕ್ಷ-ಕ-ರಾಗಿ
ಶಿಕ್ಷ-ಕ-ರಾ-ಗಿ-ದ್ದರು
ಶಿಕ್ಷ-ಕ-ರಾದ
ಶಿಕ್ಷ-ಕ-ರಿಗೂ
ಶಿಕ್ಷ-ಕರು
ಶಿಕ್ಷ-ಕ-ರೊ-ಬ್ಬರು
ಶಿಕ್ಷ-ಕಾ-ಣಾಂ
ಶಿಕ್ಷಣ
ಶಿಕ್ಷ-ಣ-ಕ್ಕಾಗಿ
ಶಿಕ್ಷ-ಣಕ್ಕೂ
ಶಿಕ್ಷ-ಣಕ್ಕೆ
ಶಿಕ್ಷ-ಣ-ಗ-ಳೆ-ರ-ಡನ್ನೂ
ಶಿಕ್ಷ-ಣದ
ಶಿಕ್ಷ-ಣ-ದಲ್ಲಿ
ಶಿಕ್ಷ-ಣ-ದಲ್ಲೂ
ಶಿಕ್ಷ-ಣ-ದಿಂದ
ಶಿಕ್ಷ-ಣ-ವನ್ನು
ಶಿಕ್ಷ-ಣ-ವಿ-ಲ್ಲ-ವೆಂಬ
ಶಿಕ್ಷ-ಣವು
ಶಿಕ್ಷ-ರದ್ದು
ಶಿಕ್ಷಾ
ಶಿಕ್ಷಾ-ಣಾ-ರ್ಥಿ-ಗ-ಳಿಂದ
ಶಿಕ್ಷಿ-ತ-ರಾದ
ಶಿಖೆಯ
ಶಿಖೆ-ಯನ್ನು
ಶಿಥಿ-ಲ-ಗೊ-ಳಿ-ಸುವ
ಶಿಥಿ-ಲ-ವಾ-ಗದ
ಶಿಬಿ-ರ-ಗ-ಳಿಗೆ
ಶಿರವ
ಶಿರಸಿ
ಶಿರ-ಸಿಯ
ಶಿರ-ಸಿ-ಯಿಂದ
ಶಿರ-ಸ್ಸಿ-ನಲ್ಲಿ
ಶಿರೋ-ಲಂ-ಕಾ-ರ-ವಾ-ದರೆ
ಶಿರ್ಷ-ಯ-ರಿಗೆ
ಶಿಲಾ-ಶಾ-ಸ-ನ-ವಿದೆ
ಶಿಲೆ
ಶಿಲೆ-ಗ-ಳನ್ನು
ಶಿಲೆ-ಯನ್ನು
ಶಿಲೆ-ಯೊಂದು
ಶಿಲ್ಪ-ಗಳು
ಶಿಲ್ಪದ
ಶಿಲ್ಪ-ದಲ್ಲಿ
ಶಿಲ್ಪ-ವ-ನ್ನಾಗಿ
ಶಿಲ್ಪಿ
ಶಿವ
ಶಿವ-ಪಾ-ರ್ವತೀ
ಶಿವಂ
ಶಿವ-ಕು-ಮಾರ
ಶಿವ-ಕು-ಮಾ-ರ-ಸ್ವಾ-ಮಿ-ಗಳು
ಶಿವಣ್ಣ
ಶಿವ-ನಾದ
ಶಿವ-ನಿ-ಗೆ-ಹಾಗೆ
ಶಿವ-ಮೊಗ್ಗ
ಶಿವ-ರಾಂ-ಪೇ-ಟೆಯ
ಶಿವ-ರಾತ್ರಿ
ಶಿವ-ರಾಮ
ಶಿವ-ಸ-ನ್ನಿಧೌ
ಶಿವ-ಸ-ಹ-ಸ್ರ-ನಾಮ
ಶಿವ-ಸ್ಥಾ-ನ-ದಲ್ಲಿ
ಶಿವಾ-ಚಾರ್ಯ
ಶಿವೆ-ಯಾದ
ಶಿಶು-ಪಾ-ಲ-ವಧ
ಶಿಶು-ವಿಗೆ
ಶಿಷ್ಟ-ಪ-ರಂ-ಪ-ರೆ-ಯನ್ನು
ಶಿಷ್ಟರು
ಶಿಷ್ಯ
ಶಿಷ್ಯಂ-ದಿರೇ
ಶಿಷ್ಯ-ಕೋಟಿ
ಶಿಷ್ಯ-ಕೋ-ಟಿಯ
ಶಿಷ್ಯ-ಗ-ಣದ
ಶಿಷ್ಯ-ಗ-ಣ-ವನ್ನು
ಶಿಷ್ಯ-ಚಿ-ತ್ತಾ-ಪ-ಹಾ-ರಕಃ
ಶಿಷ್ಯ-ಚಿ-ತ್ತಾ-ಪ-ಹಾ-ರ-ಕಾಃ
ಶಿಷ್ಯ-ತ್ವ-ವನ್ನು
ಶಿಷ್ಯನ
ಶಿಷ್ಯ-ನಂತೆ
ಶಿಷ್ಯ-ನನ್ನು
ಶಿಷ್ಯ-ನನ್ನೂ
ಶಿಷ್ಯ-ನಲ್ಲ
ಶಿಷ್ಯ-ನಾಗಿ
ಶಿಷ್ಯ-ನಾ-ಗಿದ್ದ
ಶಿಷ್ಯ-ನಾ-ಗಿ-ರು-ವುದು
ಶಿಷ್ಯ-ನಾ-ದ-ವನು
ಶಿಷ್ಯ-ನಿ-ಗಿಂತ
ಶಿಷ್ಯರ
ಶಿಷ್ಯ-ರನ್ನು
ಶಿಷ್ಯ-ರ-ಲ್ಲಿದೆ
ಶಿಷ್ಯ-ರಾ-ಗಿ-ರು-ವ-ವರು
ಶಿಷ್ಯ-ರಾದ
ಶಿಷ್ಯ-ರಿ-ಗಾಗಿ
ಶಿಷ್ಯ-ರಿ-ಗಿ-ರುವ
ಶಿಷ್ಯ-ರಿಗೂ
ಶಿಷ್ಯ-ರಿಗೆ
ಶಿಷ್ಯ-ರಿ-ಗೆಲ್ಲ
ಶಿಷ್ಯರು
ಶಿಷ್ಯ-ರು-ಗಳ
ಶಿಷ್ಯ-ರು-ಗ-ಳಿಗೆ
ಶಿಷ್ಯ-ರು-ಗಳು
ಶಿಷ್ಯ-ರೆಂದು
ಶಿಷ್ಯ-ರೆ-ಲ್ಲರೂ
ಶಿಷ್ಯ-ರೊಂ-ದಿಗೆ
ಶಿಷ್ಯ-ರೊ-ಡ-ಗೂಡಿ
ಶಿಷ್ಯ-ವ-ತ್ಸ-ಲ-ರಾದ
ಶಿಷ್ಯ-ವ-ರ್ಗಕ್ಕೆ
ಶಿಷ್ಯ-ವ-ರ್ಗ-ದಲ್ಲಿ
ಶಿಷ್ಯ-ವ-ರ್ಗ-ದ-ವರು
ಶಿಷ್ಯ-ವಾ-ತ್ಸಲ್ಯ
ಶಿಷ್ಯ-ವಾ-ತ್ಸ-ಲ್ಯಕ್ಕೆ
ಶಿಷ್ಯ-ವಾ-ತ್ಸ-ಲ್ಯದ
ಶಿಷ್ಯ-ವಿ-ತ್ತ-ವನ್ನು
ಶಿಷ್ಯ-ವಿ-ತ್ತಾ-ಪ-ಹಾ-ರ-ಕಾಃ
ಶಿಷ್ಯ-ವೃಂದ
ಶಿಷ್ಯ-ವೃಂ-ದಕ್ಕೆ
ಶಿಷ್ಯ-ವೃಂ-ದದ
ಶಿಷ್ಯ-ವೃಂ-ದ-ವನ್ನು
ಶಿಷ್ಯ-ವೃಂ-ದವು
ಶಿಷ್ಯ-ವೃಂ-ದ-ವೆಲ್ಲ
ಶಿಷ್ಯ-ವೃ-ತ್ತಿ-ಪ-ಡೆದ
ಶಿಷ್ಯ-ಹಿತ
ಶಿಷ್ಯ-ಹಿ-ತ-ಕ್ಕಾಗಿ
ಶಿಷ್ಯ-ಹಿ-ತ-ರ-ಕ್ಷ-ಣೆ-ಯಲ್ಲಿ
ಶಿಷ್ಯ-ಹಿ-ತಾಯ
ಶಿಷ್ಯೆ-ಯಾಗಿ
ಶಿಷ್ಯೇ-ಣಾಪಿ
ಶಿಸ್ತು
ಶೀಘ್ರ-ಕೋಪ
ಶೀಘ್ರ-ದಲ್ಲಿ
ಶೀಘ್ರ-ವಾಗಿ
ಶೀಬಳಿ
ಶೀಲಂ
ಶೀಲ-ವಂ-ತರು
ಶೀಲ-ವಾ-ಗಲೀ
ಶುಕ್ರ-ವಾರ
ಶುಕ್ಲ
ಶುಚಿ
ಶುಚಿಃ
ಶುಚಿ-ಯಲ್ಲ
ಶುಚಿಯೋ
ಶುದ್ಧ
ಶುದ್ಧಯೇ
ಶುದ್ಧ-ವಾಗಿ
ಶುದ್ಧಿ-ಯನ್ನು
ಶುದ್ಧಿ-ಯಿಂದ
ಶುದ್ಧಿ-ಯೊಂ-ದಿಗೆ
ಶುಧ್ದ
ಶುಪ್ತಿ-ಸಂ-ಸ್ಥಿತಾ
ಶುಭ
ಶುಭಂ
ಶುಭ-ನಾ-ಮಾಂ-ಕಿ-ತ-ವಾಗಿ
ಶುಭ-ಶಿ-ರೋ-ನಾ-ಮೆಯ
ಶುಭ-ಹಾ-ರೈ-ಕೆ-ಗಳು
ಶುಭಾ-ವ-ಸರ
ಶುರು-ವಾದ
ಶುಲ್ಕ-ದೊಂ-ದಿಗೆ
ಶುಲ್ಕಾ-ದಿ-ಗ-ಳಿಗೆ
ಶುಶ್ರೂ-ಷೆ-ಯ-ಲ್ಲಿದ್ದ
ಶುಷ್ಕ
ಶೂದ್ರಾಶ್ಚ
ಶೂನ್ಯ
ಶೃಂಖ-ಲೆ-ಯಿಂದ
ಶೃಂಗ-ದಿಂದ
ಶೃಂಗೇರಿ
ಶೃಂಗೇ-ರಿ-ಯನ್ನು
ಶೇಕಡಾ
ಶೇಷ್ಠ
ಶೈಕ್ಷ-ಣಿಕ
ಶೈಕ್ಷ-ಣಿ-ಕ-ಸಂ-ಸ್ಥೆ-ಗಳ
ಶೈಕ್ಷ-ಣಿ-ಕ-ಸಾ-ಮಾ-ಜಿಕ
ಶೈಲ-ಕ್ಕ-ನ-ವರೂ
ಶೈಲ-ಕ್ಕ-ನಿಗೆ
ಶೈಲ-ಕ್ಕನೂ
ಶೈಲಜಾ
ಶೈಲ-ಜಾ-ಧ-ರ್ಮ-ಪ-ತ್ನಿ-ಯಿಂದ
ಶೈಲ-ಜಾ-ರ-ವರ
ಶೈಲ-ಜಾ-ಳೊಂ-ದಿಗೆ
ಶೈಲ-ಜೆಯು
ಶೈಲಾ-ರ-ವರು
ಶೈಲಿ
ಶೈಲಿಗೆ
ಶೈಲಿ-ಯಂತೂ
ಶೈಲಿಯು
ಶೈವಾ-ಗ-ಮದ
ಶೋಧ
ಶೋಧ-ಗಳು
ಶೋಧಿ-ಸಲು
ಶೋಭಂತೇ
ಶೋಭಿ-ತ-ವಾದ
ಶೋಭಿ-ಸು-ತ್ತಿವೆ
ಶೌಚ
ಶೌಚ-ಕ್ಕಿಂ-ತಲೂ
ಶೌಚ-ಗ-ಳನ್ನು
ಶೌಚ-ಗ-ಳ-ಲ್ಲೆಲ್ಲ
ಶೌಚ-ವನ್ನು
ಶೌಚಾ-ನಾ-ಮ-ರ್ಥ-ಶೌಚಂ
ಶ್ಯಾಮ-ಸುಂ-ದರ
ಶ್ರದ್ಧಾ-ಭ-ಕ್ತಿ-ಗಳೇ
ಶ್ರದ್ಧೆ
ಶ್ರದ್ಧೆ-ಯಿಂದ
ಶ್ರಮ
ಶ್ರಮ-ದಿಂದ
ಶ್ರಮ-ಪ-ಟ್ಟಿ-ದ್ದಾನೆ
ಶ್ರಮ-ವನ್ನು
ಶ್ರಮಿಸಿ
ಶ್ರಮಿ-ಸಿ-ದ್ದಾರೆ
ಶ್ರಮಿ-ಸು-ತ್ತಿ-ದ್ದಾರೆ
ಶ್ರವಣ
ಶ್ರಿಂಗೇರಿ
ಶ್ರಿಮ-ನ್ಮ-ಹಾ-ರಾ-ಜಾ-ಸಂ-ಸ್ಕೃತ
ಶ್ರೀ
ಶ್ರೀಉ
ಶ್ರೀಕಂ-ಠ-ದ-ತ್ತ-ಹ-ಸ್ತಾ-ವ-ಲಂ-ಬ-ಪ್ರ-ಮೋ-ದ-ಯಾ-ರ್ಯಯಾ
ಶ್ರೀಕೃಷ್ಣ
ಶ್ರೀಕೃ-ಷ್ಣ-ನಿಗೆ
ಶ್ರೀಗಂ-ಗಾ-ಧರ
ಶ್ರೀಗ-ಳಿಗೂ
ಶ್ರೀಗು-ರು-ಪ್ರ-ಸಾ-ದ-ವರು
ಶ್ರೀಧರ
ಶ್ರೀಧ-ರಣ್ಣ
ಶ್ರೀಧ-ರ-ಣ್ಣನ
ಶ್ರೀಧ-ರ-ಭ-ಟ್ಟರು
ಶ್ರೀಧರ್
ಶ್ರೀನಾ-ಥಾ-ಚಾ-ರ್ಯರು
ಶ್ರೀನಿ-ವಾ-ಸ-ಮೂ-ರ್ತಿ-ಯ-ವರ
ಶ್ರೀನಿ-ವಾ-ಸ-ಮೂ-ರ್ತಿ-ಯ-ವರು
ಶ್ರೀಪತಿ
ಶ್ರೀಪಾದ
ಶ್ರೀಮಂ-ಜು-ನಾಥ
ಶ್ರೀಮ-ಠದ
ಶ್ರೀಮತಿ
ಶ್ರೀಮ-ತಿ-ಯ-ವ-ರನ್ನು
ಶ್ರೀಮ-ತಿ-ಯ-ವ-ರಾದ
ಶ್ರೀಮ-ತಿ-ಯ-ವರೂ
ಶ್ರೀಮತೀ
ಶ್ರೀಮ-ದ-ಭಿ-ನವ
ಶ್ರೀಮ-ದು-ದ-ಯ-ನಾ-ಚಾ-ರ್ಯರ
ಶ್ರೀಮದ್
ಶ್ರೀಮ-ನ್ಮ-ಹಾ-ರಾಜ
ಶ್ರೀಮ-ನ್ಮ-ಹಾ-ರಾ-ಜರು
ಶ್ರೀಮ-ನ್ಮ-ಹಾ-ರಾ-ಜ-ಸಂ-ಸ್ಕೃತ
ಶ್ರೀಮ-ನ್ಮ-ಹಾ-ರಾ-ಜ-ಸಂ-ಸ್ಕೃ-ತ-ಮ-ಹಾ-ಪಾ-ಠ-ಶಾಲಾ
ಶ್ರೀಮ-ನ್ಮ-ಹಾ-ರಾಜಾ
ಶ್ರೀಮ-ನ್ಮ-ಹಾ-ರಾಜ
ಶ್ರೀಮಾನ್
ಶ್ರೀಯುತ
ಶ್ರೀಯು-ತರ
ಶ್ರೀಯು-ತ-ರನ್ನ
ಶ್ರೀಯು-ತ-ರಲ್ಲಿ
ಶ್ರೀಯು-ತರು
ಶ್ರೀರಾಮ
ಶ್ರೀರಾ-ಮ-ಮಿಶ್ರ
ಶ್ರೀರಾ-ರಾ-ಸಂ-ಸ್ಕೃತ
ಶ್ರೀವಾ-ಣೀ-ವಿ-ದ್ಯಾ-ಕೇಂದ್ರ
ಶ್ರೀವಿ-ದ್ಯಾ-ಗ-ಣ-ಪತಿ
ಶ್ರೀವೆಂ-ಕ-ಟ-ರ-ಮ-ಣ-ಹೆ-ಗಡೆ
ಶ್ರೀಶಂ-ಕ-ರ-ವಿ-ಲಾಸ
ಶ್ರೀಶ್ರೀ
ಶ್ರೀಸೂಕ್ತ
ಶ್ರುತ-ಪ-ಡಿ-ಸು-ತ್ತದೆ
ಶ್ರೇಣಿ-ಯಲ್ಲಿ
ಶ್ರೇಣಿ-ಯಲ್ಲೂ
ಶ್ರೇಯ
ಶ್ರೇಯ-ಕಾಂ-ಕ್ಷಿ-ಗ-ಳಾಗಿ
ಶ್ರೇಯಸೇ
ಶ್ರೇಯ-ಸ್ಕ-ರವು
ಶ್ರೇಯಸ್ಸು
ಶ್ರೇಷ್ಟ-ವಾ-ದದ್ದು
ಶ್ರೇಷ್ಠ
ಶ್ರೇಷ್ಠಃ
ಶ್ರೇಷ್ಠಃ-ಎಂಬ
ಶ್ರೇಷ್ಠ-ತೆ-ಯನ್ನು
ಶ್ರೇಷ್ಠ-ರಾದ
ಶ್ರೇಷ್ಠ-ರೆಂದು
ಶ್ರೇಷ್ಠ-ವಿ-ದ್ಯಾ-ಸಂ-ಸ್ಥೆ-ಗ-ಳಲ್ಲಿ
ಶ್ರೇಷ್ಠ-ವಿ-ದ್ವಾಂ-ಸ-ರಾಗಿ
ಶ್ರೇಷ್ಠಾಃ
ಶ್ರೋತೃ-ಗ-ಳಿಂದ
ಶ್ರೋತೃ-ಗ-ಳಿಗೆ
ಶ್ರೌತ-ಸ್ಮಾ-ರ್ಥ-ಕ-ರ್ಮಾ-ನು-ಷ್ಠಾನ
ಶ್ಲಾಘ-ನೀಯ
ಶ್ಲಾಘ-ನೀ-ಯ-ವಾ-ದದ್ದು
ಶ್ಲಿಷ್ಟಾ
ಶ್ಲೋಕ
ಶ್ಲೋಕ-ಗ-ಳನ್ನು
ಶ್ಲೋಕದ
ಶ್ಲೋಕ-ದಲ್ಲಿ
ಶ್ಲೋಕ-ದಲ್ಲೇ
ಶ್ಲೋಕ-ದಿಂದ
ಶ್ಲೋಕ-ವ-ನ್ನಂತೂ
ಶ್ಲೋಕ-ವನ್ನು
ಶ್ಲೋಕ-ವಾ-ಗಿದೆ
ಶ್ಲೋಕ-ವಾ-ಗಿ-ದ್ದರೂ
ಶ್ಲೋಕವು
ಶ್ವಾ-ನಾಯಿ
ಷಟ್ಪ-ದಾಃ
ಷಟ್ಪ-ದಾ-ರ್ಥ-ಚಿಂ-ತೆ-ಯನ್ನು
ಷಡಂ-ಗ-ಗ-ಳ-ಲ್ಲೊಂ-ದಾದ
ಷೀಲ್ಡ್ನ್ನು
ಷೋಡ-ಶೀಮ್
ಸ
ಸಂಕ-ಟಕೆ
ಸಂಕ-ರ-ಗದ್ದೆ
ಸಂಕಲ್ಪ
ಸಂಕ-ಲ್ಪ-ದಿಂದ
ಸಂಕ-ಲ್ಪ-ವಾ-ಗಿತ್ತು
ಸಂಕ-ಲ್ಪಿ-ಸಿದ್ದೆ
ಸಂಕ-ಷ್ಟ-ಗ-ಳಿಗೂ
ಸಂಕ-ಷ್ಟ-ಗ-ಳಿಗೆ
ಸಂಕ-ಷ್ಟದ
ಸಂಕ-ಷ್ಟ-ದ-ಲ್ಲಿ-ರು-ತ್ತಾರೆ
ಸಂಕೀರ್ಣ
ಸಂಕೇ-ತ-ವನ್ನು
ಸಂಕೋ-ಚ-ದಿಂದ
ಸಂಕೋ-ಚ-ವಿ-ಲ್ಲದೇ
ಸಂಕ್ರಾಂ-ತಿ-ರ-ನ್ಯಸ್ಯ
ಸಂಖ್ಯಾ-ಶಾ-ಸ್ತ್ರ-ದಲ್ಲಿ
ಸಂಖ್ಯೆ
ಸಂಖ್ಯೆಗೆ
ಸಂಖ್ಯೆಯ
ಸಂಖ್ಯೆ-ಯಿಂದ
ಸಂಖ್ಯೆಯೂ
ಸಂಖ್ಯೆ-ಯೆಂದೂ
ಸಂಗ
ಸಂಗಡ
ಸಂಗ-ಡವೇ
ಸಂಗತಿ
ಸಂಗ-ತಿ-ಯಲ್ಲ
ಸಂಗ-ತಿ-ಯಾ-ಗಿದೆ
ಸಂಗ-ಮ-ದಂತೆ
ಸಂಗ-ಮಿ-ಸಿದ
ಸಂಗ-ರಾ-ಗ-ವ್ಯ-ಪೇತಂ
ಸಂಗಾ-ತಿ-ಯಾದ
ಸಂಗೀತ
ಸಂಗೀ-ತ-ಕ-ಲೆ-ಗಳ
ಸಂಗ್ರ-ಹ-ಕ್ಕಾಗಿ
ಸಂಗ್ರ-ಹ-ದಲ್ಲಿ
ಸಂಗ್ರ-ಹ-ವಾ-ಯಿತು
ಸಂಗ್ರ-ಹಿ-ಸ-ಲ್ಪಟ್ಟ
ಸಂಗ್ರ-ಹಿ-ಸಿ-ರು-ವು-ದನ್ನು
ಸಂಗ್ರ-ಹಿ-ಸುವ
ಸಂಗ್ರಾಹ್ಯ
ಸಂಘ
ಸಂಘಂ-ದಿಂದ
ಸಂಘಕ್ಕೆ
ಸಂಘ-ಟ-ಕರು
ಸಂಘ-ಟನಾ
ಸಂಘ-ಟ-ನೆ-ಗ-ಳಿಗೆ
ಸಂಘ-ಟ-ನೆಗೂ
ಸಂಘ-ಟ-ನೆಯ
ಸಂಘದ
ಸಂಘ-ದಲ್ಲಿ
ಸಂಘ-ವನ್ನು
ಸಂಘವು
ಸಂಘ-ಸಂ-ಸ್ಥೆ-ಗ-ಳಲ್ಲಿ
ಸಂಘ-ಸಂ-ಸ್ಥೆ-ಗ-ಳಲ್ಲಿ
ಸಂಘ-ಸಂ-ಸ್ಥೆ-ಗಳು
ಸಂಚ-ಯ-ವಾ-ಗು-ತ್ತ-ದೆಂದು
ಸಂಚ-ರಿ-ಸಿದ್ದು
ಸಂಚ-ರಿ-ಸು-ತ್ತದೆ
ಸಂಚಿ-ಕೆ-ಗ-ಳಲ್ಲಿ
ಸಂಚಿತ
ಸಂಚಿ-ತಾನಿ
ಸಂಜೆ
ಸಂಜೆಯ
ಸಂತಃ
ಸಂತತ
ಸಂತ-ತಿಗೆ
ಸಂತರು
ಸಂತ-ಶೂಲ
ಸಂತಸ
ಸಂತ-ಸ-ಗೊಂ-ಡರು
ಸಂತ-ಸ-ಗೊ-ಳಿ-ಸ-ಲಿ-ಲ್ಲ-ವೇನೋ
ಸಂತ-ಸದ
ಸಂತ-ಸ-ದಿಂದ
ಸಂತ-ಸ-ಪ-ಟ್ಟಿದ್ದು
ಸಂತ-ಸ-ವಿದೆ
ಸಂತಾನ
ಸಂತಾ-ನದ
ಸಂತಾ-ನ-ವಾಗಿ
ಸಂತಿ
ಸಂತು-ಷ್ಟ-ಗೊಂಡ
ಸಂತು-ಷ್ಟ-ನಾ-ಗಿ-ದ್ದೇನೆ
ಸಂತು-ಷ್ಟ-ರಾದ
ಸಂತೃ-ಪ್ತ-ರಾದ
ಸಂತೃ-ಪ್ತ-ರಾ-ದರು
ಸಂತೃಪ್ತಿ
ಸಂತೆ
ಸಂತೋಷ
ಸಂತೋ-ಷಕ್ಕೆ
ಸಂತೋ-ಷ-ಗೊ-ಳ್ಳು-ತ್ತದೆ
ಸಂತೋ-ಷದ
ಸಂತೋ-ಷ-ದಲ್ಲಿ
ಸಂತೋ-ಷ-ದಾ-ಯ-ಕ-ವೇನೂ
ಸಂತೋ-ಷ-ದಿಂದ
ಸಂತೋ-ಷ-ದಿಂ-ದಲೇ
ಸಂತೋ-ಷ-ವನ್ನು
ಸಂತೋ-ಷ-ವಾ-ಗು-ತ್ತದೆ
ಸಂತೋ-ಷ-ವುಂಟು
ಸಂತೋ-ಷವೇ
ಸಂತೋ-ಷಿತೋ
ಸಂದ
ಸಂದರೂ
ಸಂದರ್ಭ
ಸಂದ-ರ್ಭಕ್ಕೆ
ಸಂದ-ರ್ಭ-ಗ-ಳಲ್ಲಿ
ಸಂದ-ರ್ಭ-ಗ-ಳಿಗೂ
ಸಂದ-ರ್ಭ-ಗ-ಳಿವೆ
ಸಂದ-ರ್ಭ-ಗಳು
ಸಂದ-ರ್ಭದ
ಸಂದ-ರ್ಭ-ದಲ್ಲಿ
ಸಂದ-ರ್ಭ-ದ-ಲ್ಲಿಯೂ
ಸಂದ-ರ್ಭ-ದಲ್ಲೂ
ಸಂದ-ರ್ಭ-ವದು
ಸಂದ-ರ್ಭ-ವಿ-ರು-ತ್ತಿತ್ತು
ಸಂದ-ರ್ಭವೂ
ಸಂದ-ರ್ಭೋ-ಚಿತ
ಸಂದ-ರ್ಶನ
ಸಂದ-ರ್ಶಿ-ಸಲು
ಸಂದಿಗ್ಧ
ಸಂದಿ-ಗ್ಧ-ಗಳ
ಸಂದಿ-ಗ್ಧ-ಗಳು
ಸಂದಿ-ಗ್ಧ-ದಲ್ಲಿ
ಸಂದಿದೆ
ಸಂದಿದ್ದು
ಸಂದು-ಹೋ-ಗಿತ್ತು
ಸಂದೇಹ
ಸಂದೇ-ಹ-ಗ-ಳನ್ನು
ಸಂದೇ-ಹ-ಗಳು
ಸಂದೇ-ಹ-ವನ್ನು
ಸಂದೇ-ಹ-ವಿಲ್ಲ
ಸಂದೇ-ಹ-ವೇ-ನೆಂ-ದರೆ
ಸಂಧ-ರ್ಭ-ದಲ್ಲಿ
ಸಂಧಿ
ಸಂಧಿ-ಸ-ಮಾ-ಸ-ಗಳ
ಸಂಧ್ಯಾ-ಕಾ-ಲ-ದಲ್ಲಿ
ಸಂಧ್ಯಾ-ನ-ಮನ
ಸಂಧ್ಯಾ-ವಂ-ದನೆ
ಸಂಪ-ತ್ತನ್ನೋ
ಸಂಪ-ತ್ತಿಗೆ
ಸಂಪ-ತ್ತು-ಗ-ಳನ್ನು
ಸಂಪ-ದ್ಭ-ರಿ-ತ-ವಾದ
ಸಂಪನ್ನ
ಸಂಪ-ನ್ನ-ಗೊ-ಳಿ-ಸಿ-ಕೊಂ-ಡಿ-ರು-ತ್ತಾನೆ
ಸಂಪ-ನ್ನ-ನಲ್ಲ
ಸಂಪ-ನ್ನ-ರಾಗಿ
ಸಂಪ-ನ್ನ-ರಾದ
ಸಂಪ-ನ್ನರು
ಸಂಪ-ನ್ನ-ವಾ-ಗ-ಲಿದೆ
ಸಂಪ-ನ್ನ-ವಾ-ಯಿತು
ಸಂಪ-ನ್ಮೂಲ
ಸಂಪ-ನ್ಮೂ-ಲ-ವ್ಯ-ಕ್ತಿ-ಯಾಗಿ
ಸಂಪರ್ಕ
ಸಂಪ-ರ್ಕ-ದ-ಲ್ಲಿ-ದ್ದರು
ಸಂಪ-ರ್ಕ-ಮಾ-ಧ್ಯ-ಮ-ವೆಂ-ದರೆ
ಸಂಪ-ರ್ಕ-ವನ್ನು
ಸಂಪ-ರ್ಕಿ-ಸಲಿ
ಸಂಪ-ರ್ಕಿಸಿ
ಸಂಪ-ರ್ಕಿ-ಸಿ-ದರೆ
ಸಂಪ-ರ್ಕಿ-ಸು-ವಲ್ಲಿ
ಸಂಪಾ-ದಕ
ಸಂಪಾ-ದ-ಕ-ನಾಗಿ
ಸಂಪಾ-ದ-ಕ-ರಾಗಿ
ಸಂಪಾ-ದ-ಕ-ರು-ಗ-ಳಿಗೆ
ಸಂಪಾ-ದನೆ
ಸಂಪಾ-ದ-ನೆಯ
ಸಂಪಾ-ದಿ-ಸ-ಬೇಕು
ಸಂಪಾ-ದಿಸಿ
ಸಂಪಾ-ದಿ-ಸಿ-ಕೊಂ-ಡಿ-ದ್ದಾರೆ
ಸಂಪಾ-ದಿ-ಸಿ-ಕೊಂ-ಡಿದ್ದು
ಸಂಪಾ-ದಿ-ಸಿ-ಕೊ-ಳ್ಳ-ಬ-ಹು-ದಿತ್ತು
ಸಂಪಾ-ದಿ-ಸಿ-ದರು
ಸಂಪಾ-ದಿ-ಸಿದ್ದ
ಸಂಪಾ-ದಿ-ಸಿ-ರು-ವು-ದಕ್ಕೆ
ಸಂಪಾ-ದಿ-ಸುವ
ಸಂಪಾ-ಧಿ-ಸ-ಬ-ಹು-ದಿತ್ತು
ಸಂಪೂರ್ಣ
ಸಂಪೂ-ರ್ಣ-ಗ್ರಂ-ಥದ
ಸಂಪೂ-ರ್ಣ-ವಾಗಿ
ಸಂಪೂ-ರ್ಣ-ಸ-ಹ-ಕಾ-ರ-ವ-ನ್ನಿತ್ತು
ಸಂಪ್ರ-ದಾ-ಯಕ್ಕೆ
ಸಂಪ್ರ-ದಾ-ಯ-ಗ-ಳನ್ನು
ಸಂಪ್ರ-ದಾ-ಯ-ದಂತೆ
ಸಂಪ್ರ-ದಾ-ಯ-ದಲ್ಲಿ
ಸಂಪ್ರಾ-ರ್ಥ-ನೆ-ಯನ್ನು
ಸಂಬಂಧ
ಸಂಬಂ-ಧದ
ಸಂಬಂ-ಧ-ದಲ್ಲಿ
ಸಂಬಂ-ಧ-ದಿಂದ
ಸಂಬಂ-ಧ-ಪಟ್ಟ
ಸಂಬಂ-ಧ-ಪ-ಟ್ಟಂತೆ
ಸಂಬಂ-ಧ-ಪ-ಟ್ಟಿ-ದ್ದವು
ಸಂಬಂ-ಧ-ಪ-ಟ್ಟಿದ್ದು
ಸಂಬಂ-ಧ-ಪ-ಡದ
ಸಂಬಂ-ಧ-ವನ್ನು
ಸಂಬಂ-ಧ-ವಾಗಿ
ಸಂಬಂ-ಧವೂ
ಸಂಬಂಧಿ
ಸಂಬಂ-ಧಿ-ಕ-ರಲ್ಲಿ
ಸಂಬಂ-ಧಿ-ಕರು
ಸಂಬಂ-ಧಿ-ಗ-ಳಾದ
ಸಂಬಂ-ಧಿ-ಗಳು
ಸಂಬಂ-ಧಿ-ಯಾದ
ಸಂಬಂ-ಧಿ-ಯೊ-ಬ್ಬರು
ಸಂಬಂ-ಧಿ-ಸಿದ
ಸಂಬಂ-ಧಿ-ಸಿ-ದಂತೆ
ಸಂಬಂ-ಧಿ-ಸಿಯೇ
ಸಂಬಳ
ಸಂಬ-ಳ-ದಲ್ಲಿ
ಸಂಬೊ-ಧಿಸಿ
ಸಂಬೋ-ಧಿ-ಸಲು
ಸಂಬೋ-ಧಿಸಿ
ಸಂಬೋ-ಧಿ-ಸು-ತ್ತೇವೆ
ಸಂಭಂ-ಧ-ವನ್ನು
ಸಂಭಂ-ಧವೇ
ಸಂಭಂ-ಧಿ-ಕ-ರಾದ
ಸಂಭಂ-ಧಿ-ಸಿದ
ಸಂಭ-ವಿ-ಸಿದ
ಸಂಭ-ವಿ-ಸಿ-ವೆ-ಯೆಂದೇ
ಸಂಭಾ-ಳಿ-ಸಿದ
ಸಂಭಾ-ವನೆ
ಸಂಭಾ-ವಿತ
ಸಂಭಾವ್ಯ
ಸಂಭಾ-ಷ-ಣೆಯು
ಸಂಭೋ-ದಿ-ಸು-ತ್ತಿ-ದ್ದೆನೆ
ಸಂಭ್ರಮ
ಸಂಭ್ರ-ಮಿ-ಸಿದೆ
ಸಂಯೋ-ಗದ
ಸಂಯೋ-ಗ-ದಿಂದ
ಸಂಯೋ-ಜನೆ
ಸಂಯೋ-ಜ-ನೆ-ಯಲ್ಲಿ
ಸಂಯೋ-ಜ-ನೆ-ಯುಳ್ಳ
ಸಂರ-ಕ್ಷ-ಣೆಗೂ
ಸಂವ-ತ್ಸರ
ಸಂವ-ರ್ಧಿನಿ
ಸಂವ-ಹ-ನ-ಗ-ಳೆಲ್ಲಾ
ಸಂವ-ಹ-ನ-ಮಾ-ಧ್ಯ-ಮ-ಗಳ
ಸಂವಾದ
ಸಂವಾ-ದಿ-ಯಾಗಿ
ಸಂವಾ-ದಿ-ಯಾ-ಗಿ-ರ-ಬೇಕು
ಸಂಶಯ
ಸಂಶ-ಯ-ವಿಲ್ಲ
ಸಂಶ-ಯ-ವಿ-ಲ್ಲಈ
ಸಂಶ-ಯ-ವಿ-ಲ್ಲ-ವಷ್ಟೇ
ಸಂಶ-ಯವೂ
ಸಂಶ-ವಿಲ್ಲ
ಸಂಶೋ-ಧನ
ಸಂಶೋ-ಧನಾ
ಸಂಶೋ-ಧ-ನಾ-ತ್ಮಕ
ಸಂಶೋ-ಧ-ನೆ-ಗಳ
ಸಂಶೋ-ಧ-ನೆ-ಗೈದು
ಸಂಸತ್
ಸಂಸಾ-ಧನ
ಸಂಸಾರ
ಸಂಸಾ-ರದ
ಸಂಸಾ-ರ-ದಲ್ಲೂ
ಸಂಸಾ-ರ-ವದು
ಸಂಸಾ-ರಿ-ಯಾಗಿ
ಸಂಸಾರೇ
ಸಂಸೃ-ತ-ದಲ್ಲಿ
ಸಂಸ್ಕಾರ
ಸಂಸ್ಕಾ-ರ-ಗ-ಳನ್ನು
ಸಂಸ್ಕಾ-ರ-ಗ-ಳಲ್ಲಿ
ಸಂಸ್ಕಾ-ರ-ಗ-ಳಿಂದ
ಸಂಸ್ಕಾ-ರದ
ಸಂಸ್ಕಾ-ರ-ದಲ್ಲಿ
ಸಂಸ್ಕಾ-ರ-ವಂತ
ಸಂಸ್ಕಾ-ರ-ವಂ-ತ
ಸಂಸ್ಕಾ-ರ-ವಿ-ಶೇ-ಷ-ದಂತೆ
ಸಂಸ್ಕೃತ
ಸಂಸ್ಕೃ-ತ-ಕ-ಲಿ-ಯುವ
ಸಂಸ್ಕೃ-ತ-ಕಾ-ಲೇ-ಜಿಗೆ
ಸಂಸ್ಕೃ-ತ-ಕಾ-ಲೇ-ಜಿ-ನಲ್ಲಿ
ಸಂಸ್ಕೃ-ತ-ಕಾ-ಲೇ-ಜಿ-ನಿಂದ
ಸಂಸ್ಕೃ-ತ-ಕಾ-ಲೇಜ್
ಸಂಸ್ಕೃ-ತಕ್ಕೂ
ಸಂಸ್ಕೃ-ತಕ್ಕೆ
ಸಂಸ್ಕೃ-ತ-ಜ್ಞರ
ಸಂಸ್ಕೃ-ತದ
ಸಂಸ್ಕೃ-ತ-ದಲ್ಲಿ
ಸಂಸ್ಕೃ-ತ-ದ-ಲ್ಲಿದೆ
ಸಂಸ್ಕೃ-ತ-ದಲ್ಲೇ
ಸಂಸ್ಕೃ-ತ-ದಲ್ಲೋ
ಸಂಸ್ಕೃ-ತ-ದಿಂದ
ಸಂಸ್ಕೃ-ತ-ದೆ-ಡೆಗೆ
ಸಂಸ್ಕೃ-ತ-ಪ-ದ-ಗಳು
ಸಂಸ್ಕೃ-ತ-ಪಾಠ
ಸಂಸ್ಕೃ-ತ-ಪಾ-ಠ-ಶಾ-ಲೆ-ಯಲ್ಲಿ
ಸಂಸ್ಕೃ-ತ-ಪಾ-ಠ-ಶಾ-ಲೆ-ಯೊಂ-ದರ
ಸಂಸ್ಕೃ-ತ-ಭಾ-ಷೆಯ
ಸಂಸ್ಕೃ-ತ-ಭಾ-ಷೆ-ಯನ್ನು
ಸಂಸ್ಕೃ-ತ-ಮಾ-ಧ್ಯ-ಮ-ದಲ್ಲಿ
ಸಂಸ್ಕೃ-ತ-ವನ್ನು
ಸಂಸ್ಕೃ-ತ-ವಾ-ಣಿ-ಜ್ಯಮ್
ಸಂಸ್ಕೃ-ತ-ವಿ-ದ್ವ-ಲ್ಲೋ-ಕಕ್ಕೆ
ಸಂಸ್ಕೃ-ತ-ವಿ-ಶ್ವ-ವಿ-ದ್ಯಾ-ಲ-ಯ-ಗ-ಳಲ್ಲೂ
ಸಂಸ್ಕೃ-ತ-ವೆಂ-ದರೆ
ಸಂಸ್ಕೃ-ತ-ವೊಂ-ದ-ರಲ್ಲೇ
ಸಂಸ್ಕೃ-ತ-ಶಾಸ್ತ್ರ
ಸಂಸ್ಕೃ-ತ-ಸಾ-ಹಿತ್ಯ
ಸಂಸ್ಕೃ-ತ-ಸಾ-ಹಿ-ತ್ಯದ
ಸಂಸ್ಕೃ-ತಾ-ಧ್ಯ-ಯನ
ಸಂಸ್ಕೃ-ತಾ-ಧ್ಯ-ಯ-ನ-ಕ್ಕಾಗಿ
ಸಂಸ್ಕೃ-ತಾ-ಧ್ಯ-ಯ-ನಕ್ಕೆ
ಸಂಸ್ಕೃ-ತಾ-ಧ್ಯ-ಯ-ನದ
ಸಂಸ್ಕೃ-ತಾ-ಧ್ಯ-ಯ-ನ-ದತ್ತ
ಸಂಸ್ಕೃ-ತಾ-ಧ್ಯ-ಯ-ನ-ದಿಂದ
ಸಂಸ್ಕೃ-ತಾ-ಧ್ಯ-ಯ-ನ-ವನ್ನು
ಸಂಸ್ಕೃ-ತಾ-ಧ್ಯೇ-ತೃ-ಗಳ
ಸಂಸ್ಕೃ-ತಾ-ಧ್ಯೇ-ತೃ-ಗ-ಳಿಗೆ
ಸಂಸ್ಕೃ-ತಾ-ಭಿ-ಜ್ಞ-ತೆಯ
ಸಂಸ್ಕೃ-ತಾ-ಭಿ-ಮಾ-ನ-ಗಳು
ಸಂಸ್ಕೃ-ತಾ-ಭ್ಯಾ-ಸ-ಕ್ಕಾಗಿ
ಸಂಸ್ಕೃತಿ
ಸಂಸ್ಕೃ-ತಿಗೆ
ಸಂಸ್ಕೃ-ತಿಯ
ಸಂಸ್ಕೃ-ತೋ-ತ್ಸ-ವ-ದಲ್ಲಿ
ಸಂಸ್ಕೃ-ತ
ಸಂಸ್ಕೃ-ತ-ಶಾಸ್ತ್ರ
ಸಂಸ್ಥಾ-ನದ
ಸಂಸ್ಥಾ-ಪ-ಕ-ರಾದ
ಸಂಸ್ಥೆ
ಸಂಸ್ಥೆ-ಗ-ಳಲ್ಲಿ
ಸಂಸ್ಥೆಗೆ
ಸಂಸ್ಥೆಯ
ಸಂಸ್ಥೆ-ಯಲ್ಲಿ
ಸಂಸ್ಥೆ-ಯಾಗಿ
ಸಂಹಿ-ತೆ-ಗಳ
ಸಂಹಿ-ತೆ-ಯನ್ನು
ಸಂಹಿ-ತೆ-ಯ-ಲ್ಲಿನ
ಸಂಹಿ-ತೆಯು
ಸಂಹೃತ್ಯ
ಸಃ
ಸಕಲ
ಸಕ-ಲ-ಭೂ-ತ-ಗ-ಳಿಗೂ
ಸಕ-ಲ-ರನ್ನೂ
ಸಕ-ಲ-ವನ್ನು
ಸಕ-ಲ-ವಿ-ದ್ಯಾ-ರ್ಥಿ-ಗಳೂ
ಸಕ-ಲ-ವಿಧ
ಸಕ-ಲ-ವಿ-ಧ-ದಲ್ಲಿ
ಸಕ-ಲವೂ
ಸಕ-ಲ-ವ್ಯ-ವಸ್ಥೆ
ಸಕ-ಲ-ಶಾ-ಸ್ತ್ರ-ಗ-ಳನ್ನು
ಸಕಾರ
ಸಕಾರಂ
ಸಕಾ-ಲ-ದಲ್ಲಿ
ಸಕ್ಕರೆ
ಸಕ್ಕ-ರೆ-ಯಂಥ
ಸಕ್ರಿ-ಯ-ವಾಗಿ
ಸಕ್ರಿ-ಯ-ವಾ-ಯಿತು
ಸಖ್ಯ
ಸಗಣಿ
ಸಙ್ಘೇ-ಶಕ್ತಿ
ಸಚ-ರಾ-ಚ-ರಮ್
ಸಚಿ-ವ-ರಲ್ಲಿ
ಸಚಿ-ವ-ರಾದ
ಸಚಿ-ವ-ರಿಗೂ
ಸಚಿ-ವರು
ಸಚಿ-ವಾ-ಲ-ಯವು
ಸಚ್ಚಿ-ದಾ-ನಂದ
ಸಚ್ಚಿ-ದಾ-ನಂ-ದಾ-ಶ್ರಮ
ಸಚ್ಛಿ-ಷ್ಯ-ಸ-ಮೂ-ಹ-ವನ್ನು
ಸಜ್ಜನ
ಸಜ್ಜ-ನ-ಪ್ರಿಯ
ಸಜ್ಜ-ನರ
ಸಜ್ಜ-ನರು
ಸಜ್ಜ-ನಿ-ಕೆಯ
ಸಜ್ಜು-ಗೊಂಡ
ಸಜ್ಜು-ಗೊ-ಳಿ-ಸಿದ
ಸಜ್ಜು-ಗೊ-ಳಿ-ಸು-ತ್ತಿದ್ದೆ
ಸಜ್ಜು-ಗೊ-ಳಿ-ಸುವ
ಸಡ್ಡು
ಸಣ್ಣ
ಸಣ್ಣ-ಗದ್ದೆ
ಸತತ
ಸತತಂ
ಸತ-ತಮ್
ಸತ-ತ-ವಾಗಿ
ಸತ-ತವೂ
ಸತ-ನಾ-ನು-ಷ-ಕ್ತಾನ್
ಸತೀ-ರ್ಥ-ರಾಗಿ
ಸತೀ-ರ್ಥ-ರಾ-ದದ್ದು
ಸತೀಶ
ಸತೋ-ಽಽಗಾರೇ
ಸತ್ಕಾರ
ಸತ್ಕಾ-ರ-ದಿಂದ
ಸತ್ತ
ಸತ್ಯ
ಸತ್ಯಕ್ಕೆ
ಸತ್ಯ-ನಾ-ರಾ-ಯಣ
ಸತ್ಯ-ನಾ-ರಾ-ಯ-ಣರು
ಸತ್ಯ-ನಿ-ಷ್ಠರು
ಸತ್ಯ-ವ-ತಮ್ಮ
ಸತ್ಯ-ವಾ-ಗಿದೆ
ಸತ್ಯ-ವಾ-ದುದು
ಸತ್ಯ-ವಾ-ಯಿತು
ಸತ್ಯವೇ
ಸತ್ವ
ಸತ್ವೇ
ಸತ್ಸಂಗ
ಸತ್ಸ-ಮಯ
ಸತ್ಪ್ರ-ತಿ-ಪ-ಕ್ಷದ
ಸತ್ಪ್ರ-ತಿ-ಪ-ಕ್ಷ-ಸ್ಥ-ಲ-ದಲ್ಲಿ
ಸತ್-ಚಾ-ರಿ-ತ್ರ್ಯ-ಗ-ಳನ್ನು
ಸದ-ಭಿ-ಮಾ-ನ-ಗಳು
ಸದ-ವ-ಕಾಶ
ಸದ-ವ-ಕಾ-ಶ-ದಿಂದ
ಸದ-ಸ್ಯ-ನಾಗಿ
ಸದ-ಸ್ಯ-ರನ್ನು
ಸದ-ಸ್ಯ-ರ-ನ್ನೆಲ್ಲ
ಸದ-ಸ್ಯ-ರಲ್ಲಿ
ಸದ-ಸ್ಯ-ರಾಗಿ
ಸದ-ಸ್ಯ-ರಾದ
ಸದ-ಸ್ಯ-ರಿಗೆ
ಸದ-ಸ್ಯರು
ಸದಾ
ಸದಾ-ಚಾರ
ಸದಾ-ವೃ-ದ್ದಿ-ಯಾ-ಗು-ತ್ತಲೇ
ಸದಾ-ಶಿವ
ಸದಾ-ಶಿ-ವನ
ಸದಾ-ಶಿ-ವ-ನನ್ನು
ಸದಾ-ಶಿ-ವ-ನಿಗೆ
ಸದಾ-ಶಿ-ವ-ನೊಂ-ದಿಗೆ
ಸದಾ-ಸುಖ
ಸದು-ಕ್ತಿಗೆ
ಸದೃ-ಢ-ವೇ-ನಲ್ಲ
ಸದ್ಗುಣ
ಸದ್ಗು-ಣ-ಗಳ
ಸದ್ಗು-ಣ-ಗ-ಳನ್ನು
ಸದ್ಗು-ಣ-ಗ-ಳಿಂದ
ಸದ್ಗು-ಣ-ಯುತ
ಸದ್ಗು-ಣಿ-ಗ-ಳನ್ನು
ಸದ್ಗು-ರು-ವಿ-ನಲ್ಲೂ
ಸದ್ಗೃ-ಹ-ಸ್ಥ-ರನ್ನು
ಸದ್ಗೃ-ಹಿಣಿ
ಸದ್ಗ್ರಂ-ಥಾ-ಧ್ಯ-ಯನ
ಸದ್ಭಾ-ವ-ದಿಂದ
ಸದ್ಭಾ-ವ-ನೆ-ಯಿಂದ
ಸದ್ಯ
ಸದ್ವಿ-ಚಾ-ರ-ವನ್ನು
ಸದ್ವಿದ್ಯಾ
ಸದ್ವೈ-ದಿ-ಕ-ರಾಗಿ
ಸದ್ವೈ-ದ್ಯರ
ಸಧಾ-ಕಾಲ
ಸಧೃ-ಢ-ವಾ-ಗಿದೆ
ಸನಾ-ತನ
ಸನಾ-ತ-ನ-ಧ-ರ್ಮದ
ಸನಿ-ಹ-ದಿಂದ
ಸನ್ನ-ಡತೆ
ಸನ್ನಿ-ಧಾ-ನಕ್ಕೆ
ಸನ್ನಿ-ಧಿ-ಯಲ್ಲಿ
ಸನ್ನಿ-ವೇಶ
ಸನ್ನಿ-ವೇ-ಶ-ಗ-ಳಲ್ಲಿ
ಸನ್ನಿ-ವೇ-ಶ-ಗಳು
ಸನ್ನಿ-ವೇ-ಶ-ದಲ್ಲಿ
ಸನ್ನಿ-ವೇ-ಶ-ವನ್ನು
ಸನ್ನಿ-ವೇ-ಶವು
ಸನ್ನಿ-ವೇ-ಶ-ವೊಂದು
ಸನ್ಮಂ-ಗ-ಲಾನಿ
ಸನ್ಮ-ತಿಯ
ಸನ್ಮಾ-ನ-ವನ್ನೂ
ಸನ್ಮಾ-ನಿಸಿ
ಸನ್ಮಾ-ರ್ಗ-ದ-ರ್ಶ-ಕರು
ಸನ್ಮಿ-ತ್ರನ
ಸನ್ಮಿ-ತ್ರ-ನಾಗಿ
ಸನ್ಮಿ-ತ್ರ-ನಾದ
ಸನ್ಮಿ-ತ್ರ-ಲ-ಕ್ಷ-ಣ-ಮಿದಂ
ಸನ್ಯಾ-ಸ-ಭಿ-ಕ್ಷೆ-ಯನ್ನು
ಸನ್ಯಾ-ಸಿ-ಗಳೂ
ಸಪ-ತ್ನೀ-ಕ-ರಾಗಿ
ಸಫ-ಲ-ನಾ-ಗಿ-ದ್ದಾನೋ
ಸಫ-ಲ-ರಾ-ಗು-ತ್ತಿ-ದ್ದರು
ಸಫ-ಲ-ರಾ-ದರೂ
ಸಫ-ಲ-ವೆಂದು
ಸಬ-ಲರು
ಸಭಾ
ಸಭಾ-ಕಾ-ರ್ಯ-ಕ್ರಮ
ಸಭಾ-ಕಾ-ರ್ಯ-ಕ್ರ-ಮ-ವನ್ನು
ಸಭಾದ
ಸಭಾ-ಸಿ-ದ್ಧತೆ
ಸಭಿ-ಕ-ರಿಗೆ
ಸಭಿ-ಕರೇ
ಸಭೆ
ಸಭೆ
ಸಭೆ-ಗಳ
ಸಭೆ-ಗ-ಳನ್ನು
ಸಭೆ-ಗ-ಳಲ್ಲಿ
ಸಭೆ-ಗಳು
ಸಭೆಗೆ
ಸಭೆ-ಯನ್ನು
ಸಭೆ-ಯಲ್ಲಿ
ಸಭ್ಯ
ಸಮಃ
ಸಮ-ಕಾ-ಲೀನ
ಸಮ-ಕ್ರಿಯಂ
ಸಮ-ಗ್ರ-ವಾಗಿ
ಸಮ-ದ-ರ್ಶಿನಃ
ಸಮ-ನಾಗಿ
ಸಮ-ನ್ವ-ಯ-ವನ್ನು
ಸಮ-ನ್ವ-ಯ-ವಾ-ಗುವ
ಸಮಯ
ಸಮ-ಯ-ಕ್ಕಿಂತ
ಸಮ-ಯಕ್ಕೆ
ಸಮ-ಯ-ಗಳ
ಸಮ-ಯ-ಗ-ಳಲ್ಲೂ
ಸಮ-ಯ-ದ-ಲ್ಲಾ-ಗಲೀ
ಸಮ-ಯ-ದಲ್ಲಿ
ಸಮ-ಯ-ದ-ಲ್ಲಿ-ಯಾ-ದರೂ
ಸಮ-ಯ-ದಲ್ಲೂ
ಸಮ-ಯ-ದಲ್ಲೇ
ಸಮ-ಯ-ಪಾ-ಲನೆ
ಸಮ-ಯ-ಪ್ರಜ್ಞೆ
ಸಮ-ಯ-ವಂತೂ
ಸಮ-ಯ-ವನ್ನು
ಸಮ-ಯ-ವನ್ನೂ
ಸಮ-ಯ-ವಿ-ರದೆ
ಸಮ-ಯ-ವೆಲ್ಲಾ
ಸಮ-ಯೋ-ಚಿತ
ಸಮ-ಯೋ-ಚಿ-ತ-ವಾಗಿ
ಸಮ-ಯೋ-ಚಿ-ತ-ವಾದ
ಸಮ-ರ-ಸ-ವನ್ನು
ಸಮ-ರ್ಥ-ನಾ-ಗಿ-ದ್ದೇನೆ
ಸಮ-ರ್ಥ-ನಾ-ಗು-ವಂತೆ
ಸಮ-ರ್ಥರು
ಸಮ-ರ್ಥ-ವಾಗಿ
ಸಮ-ರ್ಥ-ವಾದ
ಸಮ-ರ್ಥಿ-ಸಿ-ಕೊಂಡೆ
ಸಮ-ರ್ಥಿ-ಸುವ
ಸಮ-ರ್ಪ-ಕ-ವಾಗಿ
ಸಮ-ರ್ಪಣೆ
ಸಮ-ರ್ಪಿತ
ಸಮ-ರ್ಪಿ-ತ-ವಾ-ಗು-ತ್ತಿದೆ
ಸಮ-ರ್ಪಿ-ತವೀ
ಸಮ-ರ್ಪಿ-ಸ-ಲ್ಪ-ಟ್ಟಿದೆ
ಸಮ-ರ್ಪಿ-ಸು-ತ್ತಿ-ದ್ದೇನೆ
ಸಮ-ರ್ಪಿ-ಸು-ತ್ತಿ-ರುವ
ಸಮ-ರ್ಪಿ-ಸೋಣ
ಸಮ-ವ-ಯ-ಸ್ಕರು
ಸಮ-ಷ್ಟಿ-ಯಲ್ಲಿ
ಸಮಸ್ತ
ಸಮ-ಸ್ತ-ಸ-ನ್ಮಂ-ಗ-ಳಾನಿ
ಸಮ-ಸ್ಥಿ-ತಿ-ಯ-ಲ್ಲಿಯೇ
ಸಮಸ್ಯಾ
ಸಮಸ್ಯೆ
ಸಮ-ಸ್ಯೆ-ಗ-ಳನ್ನು
ಸಮ-ಸ್ಯೆಗೆ
ಸಮ-ಸ್ಯೆ-ಯನ್ನು
ಸಮ-ಸ್ಯೆ-ಯಿಂದ
ಸಮಾ-ಖ್ಯಾನಂ
ಸಮಾ-ಖ್ಯಾ-ನವೇ
ಸಮಾಜ
ಸಮಾ-ಜಕ್ಕೂ
ಸಮಾ-ಜಕ್ಕೆ
ಸಮಾ-ಜ-ಕ್ಕೆಲ್ಲ
ಸಮಾ-ಜ-ಕ್ಕೊಂದು
ಸಮಾ-ಜದ
ಸಮಾ-ಜ-ದಲ್ಲಿ
ಸಮಾ-ಜ-ದಿಂದ
ಸಮಾ-ಜ-ಮು-ಖಿ-ಯಾ-ಗಿದ್ದ
ಸಮಾ-ಜ-ಮು-ಖಿಯೂ
ಸಮಾ-ಜ-ಹಿ-ತ-ವನ್ನು
ಸಮಾ-ಧಾ-ನ-ಕರ
ಸಮಾ-ಧಾ-ನ-ಕ-ರ-ವಾ-ಗಿ-ರ-ಲಿಲ್ಲ
ಸಮಾ-ಧಾ-ನ-ಪ-ಟ್ಟು-ಕೊ-ಳ್ಳು-ತ್ತೇವೆ
ಸಮಾ-ಧಾ-ನ-ವ-ನ್ನುಂ-ಟು-ಮಾ-ಡು-ತ್ತಿತ್ತು
ಸಮಾ-ಧಾ-ನ-ವನ್ನೂ
ಸಮಾ-ಧಾ-ನ-ವಿ-ಲ್ಲ-ವಾ-ಗು-ತ್ತದೆ
ಸಮಾನ
ಸಮಾ-ನ-ಕಾ-ರ್ಯ-ಕ್ಷೇ-ತ್ರ-ದಲ್ಲಿ
ಸಮಾ-ನ-ವಾಗಿ
ಸಮಾ-ನ-ವಾ-ಗಿದೆ
ಸಮಾ-ನ-ವೆಂದು
ಸಮಾ-ಪ್ತ-ವಾ-ಗಿತ್ತು
ಸಮಾ-ಪ್ತಿ-ಗೊ-ಳಿ-ಸು-ತ್ತಿ-ದ್ದೇನೆ
ಸಮಾ-ರಂಭ
ಸಮಾ-ರಂ-ಭ-ಗ-ಳನ್ನು
ಸಮಾ-ರಂ-ಭ-ಗಳು
ಸಮಾ-ರಂ-ಭದ
ಸಮಾಸ
ಸಮಿತಿ
ಸಮಿ-ತಿಗೆ
ಸಮಿ-ತಿಯ
ಸಮಿ-ತಿ-ಯನ್ನು
ಸಮಿ-ತಿ-ಯ-ಲ್ಲಾ-ಗಲಿ
ಸಮಿ-ತಿ-ಯಲ್ಲಿ
ಸಮಿ-ತಿ-ಯಿಂದ
ಸಮೀಪ
ಸಮೀ-ಪದ
ಸಮೀ-ಪ-ದಲ್ಲಿ
ಸಮೀ-ಪ-ದ-ಲ್ಲಿದ್ದ
ಸಮೀ-ಪ-ವಿ-ರುವ
ಸಮೀ-ಪವೆ
ಸಮೀ-ಪ-ಸ್ಥ-ರಿಗೂ
ಸಮು-ದ್ಧ-ರ-ಣ-ಗೊ-ಳಿ-ಸುವ
ಸಮುದ್ರ
ಸಮು-ದ್ರ-ತ-ರ-ಣ-ದಲ್ಲಿ
ಸಮು-ದ್ರದ
ಸಮು-ದ್ರ-ದಂತೆ
ಸಮು-ದ್ರ-ದಲ್ಲಿ
ಸಮು-ನ್ನ-ತಿಮ್
ಸಮೂಹ
ಸಮೂ-ಹ-ದಿಂದ
ಸಮೂ-ಹ-ವನ್ನು
ಸಮೃ-ಧ್ಧಿ-ಯನ್ನು
ಸಮೇತ
ಸಮ್ಮ-ತಿ-ಸಿ-ದರು
ಸಮ್ಮಾನ
ಸಮ್ಮು-ಖ-ದಲ್ಲಿ
ಸಮ್ಮೇ-ಳ-ನ-ದಲ್ಲಿ
ಸಮ್ಯ-ಗಂ-ಗು-ಲಿ-ಪ-ರ್ವ-ಭಿಃ
ಸಯಾ-ಜಿ-ರಾವ್
ಸಯ್ಯಾಜಿ
ಸಯ್ಯಾ-ಜಿ-ರಾವ್
ಸರಂ-ಜಾ-ಮು-ಗಳ
ಸರ-ಕಾ-ರಕ್ಕೆ
ಸರ-ಕಾ-ರದ
ಸರ-ಕಾರಿ
ಸರ-ಕಾರೀ
ಸರ-ಕಾ-ರೀಯ
ಸರ-ಣಿ-ಯನ್ನು
ಸರ-ದಾ-ರ-ನಾ-ಗಿದ್ದ
ಸರ-ದಿ-ಯಲ್ಲಿ
ಸರ-ದಿ-ಯಿಂದ
ಸರ-ಮಾ-ಲೆ-ಗ-ಳನ್ನು
ಸರಳ
ಸರ-ಳತೆ
ಸರ-ಳ-ತೆ-ಯಿಂದ
ಸರ-ಳ-ವಾಗಿ
ಸರ-ಳ-ಸ-ಜ್ಜ-ನಿ-ಕೆಯ
ಸರ-ಳ-ಸತ್ಯ
ಸರ-ಳ-ಸ್ವ-ಭಾ-ವದ
ಸರ-ಸ-ವಾಗಿ
ಸರ-ಸ-ವಾ-ಗಿಯೂ
ಸರ-ಸ-ವಾ-ಡುವ
ಸರ-ಸಿ-ಗಳು
ಸರ-ಸ್ವ-ತಿಯೆ
ಸರ-ಸ್ವತೀ
ಸರ-ಸ್ವ-ತೀ-ಪ್ರಾ-ಸಾ-ದ-ದಲ್ಲಿ
ಸರ-ಹದ್ದು
ಸರಾ-ಗ-ವಾಗಿ
ಸರಿ
ಸರಿ-ದಾ-ರಿಗೆ
ಸರಿ-ದೂ-ಗಿ-ಸ-ಲಾ-ಗದೆ
ಸರಿ-ದೊರೆ
ಸರಿ-ಪ-ಡಿ-ಸ-ಬ-ಹುದು
ಸರಿ-ಪ-ಡಿ-ಸಲು
ಸರಿ-ಪ-ಡಿ-ಸುವ
ಸರಿ-ಯಾಗಿ
ಸರಿ-ಯಾದ
ಸರಿ-ಯಾ-ದುದು
ಸರಿ-ಯೆ-ನಿ-ಸಿ-ದ್ದನ್ನು
ಸರಿ-ಸ-ಮ-ನಾಗಿ
ಸರಿ-ಸ-ಮ-ನಾ-ಗಿ-ರ-ಲಿಲ್ಲ
ಸರಿ-ಹೊಂ-ದದೆ
ಸರೀ-ಸೃ-ಪ-ಗಳು
ಸರ್ಕಾ-ರದ
ಸರ್ಕಾ-ರವು
ಸರ್ಕಾರಿ
ಸರ್ಕಾ-ರಿ-ಮ-ಹಾ-ರಾಜ
ಸರ್ಕಾ-ರಿ-ಸೇ-ವೆ-ಯಿಂದ
ಸರ್ಪ
ಸರ್ವಂ
ಸರ್ವ-ಕಾ-ರೀಯ
ಸರ್ವ-ಕಾ-ರ್ಯ-ನಿ-ರ್ವಾ-ಹ-ಕರು
ಸರ್ವ-ಕಾ-ಲ-ದಲ್ಲೂ
ಸರ್ವ-ಗು-ಣ-ಸಂ-ಪ-ನ್ನರೇ
ಸರ್ವ-ಗ್ರಾ-ಹಿ-ಯಾದಿ
ಸರ್ವ-ಜ್ಞನ
ಸರ್ವ-ಜ್ಞರೇ
ಸರ್ವತಃ
ಸರ್ವ-ತೋ-ಮುಖ
ಸರ್ವತ್ರ
ಸರ್ವಥಾ
ಸರ್ವದಾ
ಸರ್ವ-ಪ್ರಾ-ವೀಣ್ಯ
ಸರ್ವ-ಮ-ಪ-ವಿತ್ರಂ
ಸರ್ವ-ಯ-ಜ್ಞ-ಕ್ರಿ-ಯಾಶ್ಚ
ಸರ್ವರ
ಸರ್ವ-ರಿಗೂ
ಸರ್ವ-ವನ್ನು
ಸರ್ವ-ವಿ-ದಿತ
ಸರ್ವ-ಶಾಸ್ತ್ರ
ಸರ್ವ-ಶಾ-ಸ್ತ್ರ-ಪ-ರಿ-ಚ-ಯ-ವನ್ನು
ಸರ್ವ-ಸ್ವ-ವನ್ನೂ
ಸರ್ವಾಂ-ಗೀಣ
ಸರ್ವೆ-ಷಾ-ಮೇವ
ಸರ್ವೇ
ಸರ್ವೇ-ಶ್ವ-ರರೇ
ಸರ್ವ್
ಸಲ
ಸಲಹಾ
ಸಲ-ಹಿ-ಸಿ-ದಾಗ
ಸಲಹೆ
ಸಲ-ಹೆ-ಗ-ಳಿಂದ
ಸಲ-ಹೆ-ಗಳು
ಸಲ-ಹೆ-ಗಾಗಿ
ಸಲ-ಹೆ-ಗಾ-ರ-ನ-ನ್ನಾಗಿ
ಸಲ-ಹೆ-ಗಾ-ರ-ರಾಗಿ
ಸಲ-ಹೆಯ
ಸಲ-ಹೆ-ಯಂತೆ
ಸಲ-ಹೆ-ಯನ್ನು
ಸಲೀ-ಸಾಗಿ
ಸಲು-ಗೆ-ಯಿಂದ
ಸಲು-ವಾಗಿ
ಸಲುಹಿ
ಸಲ್ಪ
ಸಲ್ಲ-ಬೇಕು
ಸಲ್ಲಿ-ಸ-ಬ-ಹು-ದಾ-ದದ್ದು
ಸಲ್ಲಿ-ಸ-ಬೇ-ಕಾ-ದದ್ದು
ಸಲ್ಲಿಸಿ
ಸಲ್ಲಿ-ಸಿದ
ಸಲ್ಲಿ-ಸಿ-ದರೆ
ಸಲ್ಲಿ-ಸಿ-ದ-ವರು
ಸಲ್ಲಿ-ಸಿದ್ದ
ಸಲ್ಲಿ-ಸಿ-ದ್ದರು
ಸಲ್ಲಿ-ಸಿ-ದ್ದಾರೆ
ಸಲ್ಲಿ-ಸು-ತ್ತಿ-ದ್ದಾರೆ
ಸಲ್ಲಿ-ಸು-ತ್ತಿ-ದ್ದು-ಎ-ಲ್ಲ-ರಿಗೂ
ಸಲ್ಲಿ-ಸು-ತ್ತಿ-ರುವ
ಸಲ್ಲಿ-ಸು-ತ್ತಿ-ರು-ವುದು
ಸಲ್ಲಿ-ಸು-ತ್ತೇನೆ
ಸಲ್ಲಿ-ಸುವೆ
ಸಲ್ಲು-ತ್ತದೆ
ಸವಾ-ಲನ್ನು
ಸವಾ-ಲನ್ನೂ
ಸವಾ-ಲಿನ
ಸವಾಲೇ
ಸವಿ
ಸವಿ-ದಿ-ದ್ದಿದೆ
ಸವಿದು
ಸವಿದೆ
ಸವಿ-ಯಲು
ಸವಿ-ಸ-ಮ-ಯ-ವನ್ನು
ಸವ್ಯ-ಸಾ-ಚಿ-ಯಾ-ಗಿ-ದ್ದಾರೆ
ಸಶಿ-ಖ-ನನ್ನು
ಸಸ್ತ-ನಿ-ಗಳು
ಸಸ್ಯ-ಗಳ
ಸಹ
ಸಹ-ಕ-ರಿಸಿ
ಸಹ-ಕ-ರಿ-ಸಿದ
ಸಹ-ಕ-ರಿ-ಸಿ-ದರು
ಸಹ-ಕ-ರಿ-ಸಿ-ದ-ವರು
ಸಹ-ಕ-ರಿ-ಸಿ-ದಷ್ಟು
ಸಹ-ಕ-ರಿ-ಸಿ-ದೆ-ನ-ಮ್ಮ-ಕು-ಟುಂಬ
ಸಹ-ಕ-ರಿ-ಸಿ-ದ್ದಾರೆ
ಸಹ-ಕಾರ
ಸಹ-ಕಾ-ರ-ದಿಂದ
ಸಹ-ಕಾ-ರ-ವಾ-ಯಿತು
ಸಹ-ಕಾ-ರವೂ
ಸಹ-ಕಾ-ರಿ-ಯಾ-ಗಿದೆ
ಸಹ-ಕಾ-ರಿ-ಯಾ-ಗಿ-ದ್ದಾರೆ
ಸಹ-ಕಾ-ರಿ-ಯಾ-ಗು-ತ್ತ-ದೆಂಬ
ಸಹ-ಕಾ-ರಿ-ಯಾ-ಗು-ತ್ತವೆ
ಸಹಜ
ಸಹ-ಜತೆ
ಸಹ-ಜ-ನ-ಡಿಗೆ
ಸಹ-ಜ-ಬ-ಡಿತ
ಸಹ-ಜ-ವಾಗಿ
ಸಹ-ಜ-ವಾ-ಗಿಯೇ
ಸಹ-ನೀ-ಯ-ವಾಗಿ
ಸಹ-ಪಾ-ಠಿ-ಗ-ಳಾ-ಗಿ-ಯಷ್ಟೇ
ಸಹ-ಪಾ-ಠಿ-ಗ-ಳಿಗೂ
ಸಹ-ಪಾ-ಠಿ-ಗಳು
ಸಹ-ಪಾ-ಠಿ-ಯೊ-ಬ್ಬ-ನನ್ನು
ಸಹ-ಭಾ-ಗಿ-ಯಾಗಿ
ಸಹ-ಭಾ-ಗಿ-ಯಾದ
ಸಹ-ವರ್ತಿ
ಸಹ-ವ-ರ್ತಿ-ಗ-ಳಾ-ದೆವು
ಸಹ-ವಾಸ
ಸಹ-ವಾ-ಸಿ-ಗ-ಳಾದ
ಸಹ-ವಾ-ಸಿ-ಗಳು
ಸಹಸ್ರ
ಸಹ-ಸ್ರ-ಗುಣ
ಸಹ-ಸ್ರ-ನಾ-ಮ-ಗಳ
ಸಹ-ಸ್ರ-ರಲ್ಲಿ
ಸಹ-ಸ್ರಾರು
ಸಹಾ-ಧ್ಯಾ-ಪ-ಕರ
ಸಹಾ-ಧ್ಯಾ-ಯಿ-ಗ-ಳನ್ನು
ಸಹಾಯ
ಸಹಾ-ಯಕ
ಸಹಾ-ಯ-ಕ-ರಾಗಿ
ಸಹಾ-ಯ-ಕ-ರಾ-ಗಿದ್ದ
ಸಹಾ-ಯ-ಕರು
ಸಹಾ-ಯ-ಕ-ಳಾ-ಗಿ-ದ್ದಾಳೆ
ಸಹಾ-ಯ-ಕ-ವಾ-ಗು-ತ್ತದೆ
ಸಹಾ-ಯಕ್ಕೆ
ಸಹಾ-ಯದ
ಸಹಾ-ಯ-ದಿಂದ
ಸಹಾ-ಯ-ಮಾಡಿ
ಸಹಾ-ಯ-ಮಾ-ಡಿ-ದ್ದ-ರಿಂದ
ಸಹಾ-ಯ-ಮಾ-ಡುವ
ಸಹಾ-ಯ-ವಾ-ದಂ-ತಾ-ಗು-ತ್ತದೆ
ಸಹಾ-ಯ-ಹಸ್ತ
ಸಹಾ-ಯ-ಹ-ಸ್ತ-ವನ್ನು
ಸಹಾ-ಯಾಕ್ಕೆ
ಸಹಾ-ಯ-ಸ-ವ-ಲ-ತ್ತು-ಗ-ಳನ್ನು
ಸಹಾ-ಯ-ಸ-ಹ-ಕಾರ
ಸಹಾ-ಯ-ಸ-ಹ-ಕಾ-ರ-ಮಾ-ರ್ಗ-ದ-ರ್ಶನ
ಸಹಿತ
ಸಹಿ-ಸ-ದಂತೆ
ಸಹಿ-ಸದೇ
ಸಹಿ-ಸ-ಲಿಲ್ಲ
ಸಹಿಸಿ
ಸಹಿ-ಸು-ವು-ದಿಲ್ಲ
ಸಹೃ-ದ-ಯತೆ
ಸಹೃ-ದ-ಯ-ತೆ-ಯಿಂದ
ಸಹೃ-ದ-ಯ-ರನ್ನು
ಸಹೃ-ದ-ಯ-ರಾ-ಗಿ-ರು-ತ್ತಿ-ದ್ದ-ರೆಂದು
ಸಹೃ-ದ-ಯ-ರಾದ
ಸಹೃ-ದ-ಯರು
ಸಹೃ-ದಯಿ
ಸಹೃ-ದ-ಯಿ-ಯಾ-ದ್ದ-ರಿಂದ
ಸಹೊ-ದ-ರಿ-ಯನ್ನು
ಸಹೋ-ದರ
ಸಹೋ-ದ-ರ-ರಾ-ಗಿದ್ದ
ಸಹೋ-ದ-ರರು
ಸಹೋ-ದ-ರಿಯ
ಸಹೋ-ದ-ರಿ-ಯರು
ಸಹೋ-ದ್ಯೋಗಿ
ಸಹೋ-ದ್ಯೋ-ಗಿ-ಗ-ಳಾಗಿ
ಸಹೋ-ದ್ಯೋ-ಗಿ-ಗ-ಳಾ-ದೆವು
ಸಹೋ-ದ್ಯೋ-ಗಿ-ಗಳು
ಸಹೋ-ದ್ಯೋ-ಗಿ-ಯಾಗಿ
ಸಹೋ-ದ್ಯೋ-ಗಿ-ಯಾದ
ಸಹೋ-ಧರಿ
ಸಹ್ಯತೆ
ಸಹ್ಯ-ವೆ-ನಿ-ಸದು
ಸಾ
ಸಾಂಕೇ-ತಿ-ಕ-ವಾಗಿ
ಸಾಂಖ್ಯ
ಸಾಂಖ್ಯ-ದ-ರ್ಶನ
ಸಾಂಖ್ಯ-ಯೋಗ
ಸಾಂಖ್ಯ-ಯೋ-ಗ-ಗಳ
ಸಾಂಗ-ವಾಗಿ
ಸಾಂಘಿಕ
ಸಾಂತ್ವನ
ಸಾಂತ್ವ-ನ-ಗೊಂಡು
ಸಾಂದ-ರ್ಭಿಕ
ಸಾಂಪ್ರ-ದಾ-ಯಿಕ
ಸಾಂಬ-ಮೂ-ರ್ತಿ-ಯ-ವ-ರಲ್ಲಿ
ಸಾಂಬಾರ್
ಸಾಂಸಾ-ರಿ-ಕ-ವಾಗಿ
ಸಾಂಸ್ಕೃ-ತಿಕ
ಸಾಕ-ಲ್ಯದ
ಸಾಕ-ಲ್ಯವು
ಸಾಕ-ಷ್ಟಿವೆ
ಸಾಕಷ್ಟು
ಸಾಕಾರ
ಸಾಕಾ-ರ-ಗೊಂ-ಡಿದ್ದು
ಸಾಕು
ಸಾಕು
ಸಾಕ್ಷ-ಚಿ-ತ್ರ-ವೊಂ-ದನ್ನು
ಸಾಕ್ಷ-ವಾ-ಗಿ-ಸುವ
ಸಾಕ್ಷಾತ್
ಸಾಕ್ಷಾ-ತ್ಕ-ರಿ-ಸಿ-ದೆ-ಯೆಂ-ದರೆ
ಸಾಕ್ಷಾ-ತ್ಕಾ-ರ-ಗೊಂ-ಡಿದೆ
ಸಾಕ್ಷಿ
ಸಾಕ್ಷಿ-ಭಾಗ್ಯ
ಸಾಕ್ಷಿ-ಯಾಗಿ
ಸಾಕ್ಷಿ-ಯಾ-ಗಿದೆ
ಸಾಕ್ಷಿ-ಯಿ-ದ್ದರೆ
ಸಾಕ್ಷ್ಯ
ಸಾಕ್ಷ್ಯ-ಚಿ-ತ್ರ-ದಲ್ಲಿ
ಸಾಗರ
ಸಾಗ-ರದ
ಸಾಗ-ರ-ದಂ-ತಿ-ರುವ
ಸಾಗ-ರ-ದಷ್ಟು
ಸಾಗ-ರ-ದಾ-ಚೆಗೂ
ಸಾಗ-ರ-ದಿಂದ
ಸಾಗ-ರಮ್
ಸಾಗಲು
ಸಾಗಿ
ಸಾಗಿತು
ಸಾಗಿತ್ತು
ಸಾಗಿ-ತ್ತು	
ಸಾಗಿದ
ಸಾಗಿ-ಬಂ-ದಿದೆ
ಸಾಗಿ-ರುವ
ಸಾಗಿವೆ
ಸಾಗಿ-ಸಿದ
ಸಾಗಿ-ಸಿ-ದ-ವರು
ಸಾಗಿಸು
ಸಾಗಿ-ಸು-ತ್ತಿ-ದ್ದರು
ಸಾಗಿ-ಸು-ತ್ತಿ-ದ್ದಾರೆ
ಸಾಗಿ-ಸು-ತ್ತಿದ್ದು
ಸಾಗಿ-ಸು-ತ್ತಿ-ರುವ
ಸಾಗಿ-ಸು-ತ್ತಿ-ರು-ವಾಗ
ಸಾಗಿ-ಸು-ವಲ್ಲಿ
ಸಾಗು-ತ್ತದೆ
ಸಾಗು-ತ್ತಿ-ದ್ದವು
ಸಾಗು-ತ್ತಿವೆ
ಸಾಟಿ
ಸಾಟಿ-ಯೇನು
ಸಾತ-ತ್ಯ-ದಿಂದ
ಸಾತ್ತ್ವಿಕ
ಸಾತ್ತ್ವಿ-ಕ-ರಲ್ಲಿ
ಸಾತ್ವಿಕ
ಸಾದಾ-ರಣ
ಸಾಧ-ಕನ
ಸಾಧ-ಕ-ನಿಗೆ
ಸಾಧ-ನ-ಗ-ಳಿ-ಲ್ಲದ
ಸಾಧನಾ
ಸಾಧ-ನೆ-ಗ-ಳಿಗೆ
ಸಾಧ-ನೆಗೆ
ಸಾಧ-ನೆಯ
ಸಾಧ-ನೆ-ಯಾ-ಗು-ವಲ್ಲಿ
ಸಾಧಾರ
ಸಾಧಾ-ರಣ
ಸಾಧಾ-ರ-ಣ-ವಾಗಿ
ಸಾಧಾ-ರ-ಣ-ವಾ-ಗಿತ್ತು
ಸಾಧಿಸ
ಸಾಧಿ-ಸಿ-ದರು
ಸಾಧಿ-ಸು-ತ್ತಲೆ
ಸಾಧಿ-ಸುವ
ಸಾಧು
ಸಾಧು-ವೃತ್ತಃ
ಸಾಧು-ವೃ-ತ್ತದ
ಸಾಧು-ವೃ-ತ್ತ-ನಾ-ದಲ್ಲಿ
ಸಾಧ್ಯ
ಸಾಧ್ಯ-ಗದು
ಸಾಧ್ಯ-ತೆ-ಯಿ-ರ-ಲಿಲ್ಲ
ಸಾಧ್ಯ-ವಾ-ಗ-ದಿ-ದ್ದರೂ
ಸಾಧ್ಯ-ವಾ-ಗ-ಲಾ-ರದು
ಸಾಧ್ಯ-ವಾ-ಗಿದೆ
ಸಾಧ್ಯ-ವಾ-ಗಿ-ಸುವ
ಸಾಧ್ಯ-ವಾ-ಗು-ತ್ತಿತ್ತು
ಸಾಧ್ಯ-ವಾ-ಗು-ತ್ತಿದೆ
ಸಾಧ್ಯ-ವಾ-ಗು-ತ್ತಿಲ್ಲ
ಸಾಧ್ಯ-ವಾ-ದದ್ದು
ಸಾಧ್ಯ-ವಾ-ದಷ್ಟು
ಸಾಧ್ಯ-ವಾ-ದಾಗ
ಸಾಧ್ಯ-ವಾ-ದೀತು
ಸಾಧ್ಯ-ವಿದೆ
ಸಾಧ್ಯ-ವಿ-ರ-ಲಿಲ್ಲ
ಸಾಧ್ಯ-ವಿಲ್ಲ
ಸಾಧ್ಯವೂ
ಸಾಧ್ಯವೇ
ಸಾಧ್ಯವೇ
ಸಾಧ್ಯ-ಸಾ-ಧಕ
ಸಾಧ್ಯ-ಸಾ-ಧ-ಕ-ತ್ವೇನ
ಸಾಧ್ಯಾ-ಭಾ-ವ-ದಲ್ಲಿ
ಸಾಧ್ಯಾ-ಭಾ-ವ-ಸಾ-ಧಕ
ಸಾಧ್ಯಾ-ಭಾ-ವ-ಸಾ-ಧ-ಕ-ಹೇ-ತುಂ
ಸಾಧ್ವಿ-ಯರು
ಸಾನ್ನಿ-ಧ್ಯ-ದಲ್ಲಿ
ಸಾಮಥ್ರ್ಯ
ಸಾಮ-ಥ್ರ್ಯಕ್ಕೆ
ಸಾಮ-ಥ್ರ್ಯ-ವಿತ್ತು
ಸಾಮ-ಥ್ರ್ಯ-ವು-ಳ್ಳ-ವರು
ಸಾಮರ್ಥ್ಯ
ಸಾಮ-ರ್ಥ್ಯದ
ಸಾಮ-ರ್ಥ್ಯ-ದಿಂ-ದಲೇ
ಸಾಮ-ರ್ಥ್ಯ-ವನ್ನು
ಸಾಮ-ರ್ಥ್ಯ-ವಿ-ದೆಯೋ
ಸಾಮ-ರ್ಥ್ಯ-ವಿ-ರ-ಬೆ-ಕಂತೆ
ಸಾಮ-ರ್ಥ್ಯ-ವಿ-ರು-ತ್ತದೋ
ಸಾಮ-ರ್ಥ್ಯ-ವಿ-ರುವ
ಸಾಮ-ರ್ಥ್ಯವು
ಸಾಮ-ರ್ಥ್ಯ-ವುಳ್ಳ
ಸಾಮ-ರ್ಥ್ಯ-ವೆಂ-ತ-ದ್ದೆಂ-ದರೆ
ಸಾಮ-ವೇ-ದ-ದಲ್ಲಿ
ಸಾಮಾ-ಜಿಕ
ಸಾಮಾ-ಜಿ-ಕರ
ಸಾಮಾ-ಜಿ-ಕ-ವಾಗಿ
ಸಾಮಾ-ನ-ನ್ನೆಲ್ಲ
ಸಾಮಾನ್ಯ
ಸಾಮಾ-ನ್ಯ-ಜ-ನರು
ಸಾಮಾ-ನ್ಯ-ರಂ-ತಿ-ರು-ವುದು
ಸಾಮಾ-ನ್ಯ-ವಾಗಿ
ಸಾಮಾ-ನ್ಯ-ವಾ-ಗಿತ್ತು
ಸಾಮಾ-ನ್ಯ-ವಾ-ಯಿತು
ಸಾಮಾ-ನ್ಯವೇ
ಸಾಮಿ-ಪ್ಯ-ದಿಂದ
ಸಾಮೀಪ್ಯ
ಸಾಮ್ಯ
ಸಾಮ್ಯತೆ
ಸಾಮ್ಯದ
ಸಾಯಂ-ಕಾಲ
ಸಾಯ-ಣಾ-ಚಾ-ರ್ಯರು
ಸಾಯದೇ
ಸಾಯುವ
ಸಾರ-ಭೂ-ತ-ವಾದ
ಸಾರ-ವ-ತ್ತಾದ
ಸಾರ-ವನ್ನೂ
ಸಾರ-ಸ್ವತ
ಸಾರಾ-ಸ-ಗ-ಟಾಗಿ
ಸಾರಿ
ಸಾರಿ-ಯಾ-ದರೂ
ಸಾರಿ-ಸಾರಿ
ಸಾರು-ತ್ತಾನೆ
ಸಾರು-ತ್ತಾರೆ
ಸಾರುವ
ಸಾರ್ಥಕ
ಸಾರ್ಥ-ಕತೆ
ಸಾರ್ಥ-ಕ-ತೆ-ಯನ್ನು
ಸಾರ್ಥ-ಕ-ಥೆ-ಯನ್ನು
ಸಾರ್ಥ-ಕ-ವಾ-ಗಿದೆ
ಸಾರ್ಥ-ಕ-ವಾ-ಯಿತು
ಸಾರ್ಥ-ಕ-ವೆಂಬ
ಸಾರ್ಥಕ್ಯ
ಸಾರ್ಥ-ಕ್ಯದ
ಸಾರ್ಥ-ಕ್ಯ-ವೆಂದೂ
ಸಾರ್ವ-ಕಾ-ಲಿ-ಕ-ವಾಗಿ
ಸಾರ್ವ-ಜ-ನಿಕ
ಸಾರ್ವ-ಜ-ನಿ-ಕರ
ಸಾರ್ವ-ಜ-ನಿ-ಕ-ರನ್ನೂ
ಸಾರ್ವ-ಜ-ನಿ-ಕ-ರಿಗೂ
ಸಾರ್ವ-ಜ-ನಿ-ಕ-ರಿಗೆ
ಸಾರ್ವ-ದೇ-ಶಿ-ಕ-ವಾಗಿ
ಸಾಲ
ಸಾಲ-ದೇನೋ
ಸಾಲ-ವನ್ನು
ಸಾಲಿನ
ಸಾಲು
ಸಾಲು-ಗ-ಳನ್ನು
ಸಾಲು-ಗಳು
ಸಾವ-ರಿ-ಸಿ-ಕೊಂಡು
ಸಾವಿ-ತ್ರ-ಮ್ಮ-ನ-ವರ
ಸಾವಿರ
ಸಾವಿ-ರಕ್ಕೂ
ಸಾವಿ-ರಾರು
ಸಾವಿ-ರೊಂದು
ಸಾಸ್ತಾ-ನದ
ಸಾಹ-ಚರ್ಯ
ಸಾಹ-ಚ-ರ್ಯದ
ಸಾಹ-ಚ-ರ್ಯವೇ
ಸಾಹ-ಸ-ವನ್ನು
ಸಾಹ-ಸವೇ
ಸಾಹಸಿ
ಸಾಹಿ-ತ-ಕೃ-ಷಿಯು
ಸಾಹಿ-ತಿ-ಗಳು
ಸಾಹಿತ್ಯ
ಸಾಹಿ-ತ್ಯ-ಗಳ
ಸಾಹಿ-ತ್ಯ-ಗ-ಳಲ್ಲಿ
ಸಾಹಿ-ತ್ಯದ
ಸಾಹಿ-ತ್ಯ-ಲೋ-ಕಕ್ಕೆ
ಸಾಹಿ-ತ್ಯಿಕ
ಸಾಹು-ಕಾ-ರ್ಹುಂಡಿ
ಸಿ
ಸಿಂಧು
ಸಿಂಧೂ
ಸಿಂಧೂರ
ಸಿಂಹಕ್ಕೆ
ಸಿಎ
ಸಿಕಂ-ದ-ರಾ-ಬಾ-ದ್ನಲ್ಲಿ
ಸಿಕ್ಕ
ಸಿಕ್ಕಂತೆ
ಸಿಕ್ಕರು
ಸಿಕ್ಕರೆ
ಸಿಕ್ಕಾ-ಗ-ಲೆಲ್ಲಾ
ಸಿಕ್ಕಿ
ಸಿಕ್ಕಿ-ತೆಂದೆ
ಸಿಕ್ಕಿತ್ತು
ಸಿಕ್ಕಿದ
ಸಿಕ್ಕಿ-ದೆ-ಯೆ-ನ್ನ-ಬ-ಹುದು
ಸಿಕ್ಕಿದ್ದು
ಸಿಕ್ಕಿ-ರು-ವುದು
ಸಿಕ್ಕು-ತ್ತಿ-ರ-ಲಿಲ್ಲ
ಸಿಗದ
ಸಿಗ-ದಿ-ದ್ದಾಗ
ಸಿಗದೆ
ಸಿಗದೇ
ಸಿಗ-ಲಾ-ರದು
ಸಿಗಲಿ
ಸಿಗ-ಲಿ-ಲ್ಲ-ವೆಂಬ
ಸಿಗಲು
ಸಿಗು-ತ್ತದೆ
ಸಿಗು-ತ್ತಾರೆ
ಸಿಗುತ್ತೋ
ಸಿಗು-ವಂ-ತಿ-ರ-ಲಿಲ್ಲ
ಸಿಗು-ವ-ವ-ರೆಗೆ
ಸಿಗು-ವು-ದಿಲ್ಲ
ಸಿಗು-ವುದು
ಸಿಗ್ಮಂಡ್
ಸಿಟಿಯ
ಸಿಡಿ
ಸಿಡಿ-ದೆದ್ದೆ
ಸಿಡಿ-ಮಿ-ಡಿ-ಗೊಂ-ಡರು
ಸಿದಾ-ಪುರ
ಸಿದ್ದ-ತೆ-ಯನ್ನು
ಸಿದ್ದ-ನೊಬ್ಬ
ಸಿದ್ದ-ರಿ-ದ್ದರು
ಸಿದ್ದಾ-ಪರ
ಸಿದ್ದಾ-ಪುರ
ಸಿದ್ದಾ-ಪು-ರ	
ಸಿದ್ದಾ-ಪು-ರ-ತಾ-ಲೂ-ಕಿನ
ಸಿದ್ದಾ-ಪು-ರದ
ಸಿದ್ದಾ-ಪು-ರ-ದಲ್ಲಿ
ಸಿದ್ದಾ-ಪು-ರ-ದಿಂದ
ಸಿದ್ದಾ-ಪು-ರ-ಸಿ-ರ-ಸಿ-ಸಾ-ಗ-ರ-ಯ-ಲ್ಲಾ-ಪು-ರದ
ಸಿದ್ದಾ-ರ್ಥ-ನ-ಗ-ರದ
ಸಿದ್ಧ
ಸಿದ್ಧ-ಗೊ-ಳ್ಳು-ವ-ದ-ರಿಂದ
ಸಿದ್ಧತೆ
ಸಿದ್ಧ-ತೆ-ಗ-ಳನ್ನು
ಸಿದ್ಧ-ತೆ-ಗ-ಳೆಲ್ಲ
ಸಿದ್ಧ-ತೆ-ಯಲ್ಲಿ
ಸಿದ್ಧ-ನಾದೆ
ಸಿದ್ಧ-ಪ-ಡಿ-ಸ-ಬೇ-ಕಾ-ಗಿ-ರು-ವು-ದ-ರಿಂದ
ಸಿದ್ಧ-ಪ-ಡಿಸಿ
ಸಿದ್ಧ-ಪ-ಡಿ-ಸಿ-ಕೊ-ಳ್ಳಲು
ಸಿದ್ಧ-ಪ-ಡಿ-ಸಿ-ದೆವು
ಸಿದ್ಧ-ರಾ-ಮ-ಯ್ಯ-ನ-ವರು
ಸಿದ್ಧ-ರಿ-ರ-ಲಿಲ್ಲ
ಸಿದ್ಧ-ಲಿಂ-ಗೆ-ಶ್ವರ
ಸಿದ್ಧ-ವಾ-ಗ-ಬೆ-ಕಾ-ದುದು
ಸಿದ್ಧ-ವಾ-ಗಿ-ರು-ತ್ತದೆ
ಸಿದ್ಧ-ವಾ-ಗಿ-ರು-ತ್ತಿ-ದ್ದರು
ಸಿದ್ಧ-ವಾದ
ಸಿದ್ಧ-ವಾ-ಯಿತು
ಸಿದ್ಧಾಂತ
ಸಿದ್ಧಾಂ-ತಕ್ಕೆ
ಸಿದ್ಧಾಂ-ತ-ಗ-ಳನ್ನು
ಸಿದ್ಧಾಂ-ತ-ಗಳು
ಸಿದ್ಧಾಂ-ತದ
ಸಿದ್ಧಾಂ-ತ-ಲ-ಕ್ಷಣ
ಸಿದ್ಧಾಂ-ತ-ವನ್ನು
ಸಿದ್ಧಾಂ-ತ-ವಿತ್ತು
ಸಿದ್ಧಾಂ-ತಾ-ಧಾ-ರಿತ
ಸಿದ್ಧಾ-ಪುರ
ಸಿದ್ಧಾ-ಪು-ರ-ದಲ್ಲಿ
ಸಿದ್ಧಾ-ರ್ಥ-ಲೇ-ಔಟ್
ಸಿದ್ಧಿಗೆ
ಸಿದ್ಧಿ-ಯಾ-ಗದು
ಸಿದ್ಧಿ-ಯಾ-ಗು-ತ್ತದೆ
ಸಿದ್ಧಿ-ವಿ-ನಾ-ಯಕ
ಸಿದ್ಧಿ-ಸ-ಲಾ-ರದು
ಸಿದ್ಧಿ-ಸಿದೆ
ಸಿದ್ಧಿ-ಸಿದ್ದು
ಸಿದ್ಧಿ-ಸು-ತ್ತದೆ
ಸಿನೇಮಾ
ಸಿಬ್ಬಂದಿ
ಸಿಬ್ಬಂ-ದಿ-ವ-ರ್ಗ-ದ-ವರು
ಸಿರಸಿ
ಸಿರ್ಸಿ
ಸಿರ್ಸಿಗೆ
ಸಿರ್ಸಿ-ಸಿ-ದ್ದಾ-ಪು-ರದ
ಸಿಲುಕಿ
ಸಿಹಿ-ತಿಂ-ಡಿ-ಮಾ-ಡಿ-ದಾ-ಗ-ಲೆಲ್ಲಾ
ಸಿಹಿ-ಯನ್ನು
ಸಿಹಿಯೂ
ಸಿಹಿ-ಯೂಟ
ಸೀಟು
ಸೀತಾ-ಪ-ರಿ-ತ್ಯಾಗ
ಸೀತಾ-ರಾಮ
ಸೀದಾ
ಸೀಮಿ-ತ-ರಾ-ಗಿದ್ದ
ಸೀಮಿ-ತ-ವಾ-ಗದೇ
ಸೀಮಿ-ತ-ವಾ-ಗಿ-ರದೇ
ಸೀಮಿ-ತ-ವಾ-ಗಿ-ರ-ಲಿಲ್ಲ
ಸೀಮಿ-ತ-ವಾ-ಗಿ-ರು-ತ್ತಿತ್ತು
ಸೀಮಿ-ತ-ವಾ-ದ-ದ್ದಲ್ಲ
ಸೀಮಿ-ತ-ವಾ-ದು-ದಲ್ಲ
ಸೀಳಿ
ಸುಂದರ
ಸುಂದ-ರ-ಮೂರ್ತಿ
ಸುಂದ-ರ-ಮೂ-ರ್ತಿ-ಸಾ-ವಿ-ತ್ರಮ್ಮ
ಸುಂದ-ರ-ವಾಗಿ
ಸುಂದ-ರ-ವಾ-ಗಿ-ದೆ-ಎಂಬ
ಸುಂದ-ರ-ವಾದ
ಸುಂದ-ರ-ಶಿ-ಲ್ಪ-ವ-ನ್ನಾ-ಗಿ-ಮಾ-ಡು-ವಲ್ಲಿ
ಸುಂದ-ರಾ-ಕೃತಿ
ಸುಕೃ-ತವೇ
ಸುಖ
ಸುಖ-ಕ-ರ-ವಾ-ಗಿ-ರಲಿ
ಸುಖ-ಬೋ-ಧಾಯ
ಸುಖ-ಮ-ಯ-ವಾ-ಗಿ-ರಲಿ
ಸುಖ-ಮ-ಯ-ವಾ-ಗಿ-ರ-ಲೆಂದು
ಸುಖ-ಮ-ಯ-ವಾ-ಗಿ-ರ-ಲೆಂ-ಬುದೇ
ಸುಖ-ಮ-ಯ-ವಾ-ಗಿ-ರು-ವಂತೆ
ಸುಖ-ವನ್ನು
ಸುಖ-ಶಾಂತಿ
ಸುಖಾಂತ್ಯ
ಸುಖೇ
ಸುಖ-ದುಃ-ಖ-ಗ-ಳಲ್ಲಿ
ಸುಖ-ದುಃ-ಖ-ಗ-ಳೆ-ರ-ಡ-ರಲ್ಲು
ಸುಗಂ-ಧವು
ಸುಗು-ಣಿಯ
ಸುಜ್ಞಾ-ನ-ರೂಪ
ಸುಡು-ಗಾಡು
ಸುತ-ರಾಂ
ಸುತ್ತ
ಸುತ್ತ-ಲಿನ
ಸುತ್ತಲೂ
ಸುತ್ತಾಡಿ
ಸುತ್ತಿ-ಕೊಂಡು
ಸುತ್ತೂರು
ಸುದೀರ್ಘ
ಸುದೀ-ರ್ಘ-ಕಾಲ
ಸುದೀ-ರ್ಘ-ಕಾ-ಲದ
ಸುದೀ-ರ್ಘ-ವಾಗಿ
ಸುದೀ-ರ್ಘ-ವಾದ
ಸುದೀ-ರ್ಘಾ-ನು-ವಿ-ತ್ತಿ-ಯಾಗಿ
ಸುದೃಢ
ಸುದೈ-ವವೇ
ಸುದ್ದಿ
ಸುದ್ದಿ-ಗ-ಳನ್ನು
ಸುಧರ್ಮಾ
ಸುಧ-ರ್ಮಾಗೆ
ಸುಧ-ರ್ಮಾದ
ಸುಧಾ-ರ-ಣೆ-ಗ-ಳನ್ನು
ಸುಧಾ-ರಿ-ಸಿ-ಕೊ-ಳ್ಳಲು
ಸುಧೀ-ಮ-ಣಿ-ಗಳು
ಸುಪು-ತ್ರರೇ
ಸುಪುತ್ರಿ
ಸುಪ್ತ
ಸುಪ್ತ-ಮ-ನ-ಸ್ಸಿ-ನೊ-ಳಗೆ
ಸುಪ್ತ-ವಾ-ಗಿದ್ದ
ಸುಪ್ರ-ಸಿದ್ಧ
ಸುಬ್ಬಣ್ಣ
ಸುಬ್ಬ-ಣ್ಣ-ನೆಂದು
ಸುಬ್ಬು
ಸುಬ್ರ-ಹ್ಮಣ್ಯ
ಸುಬ್ರ-ಹ್ಮ-ಣ್ಯ-ಭ-ಟ್ಟ-ರಲ್ಲಿ
ಸುಬ್ರಾಯ
ಸುಭಾ-ಷಿತ
ಸುಭಾ-ಷಿ-ತ-ಕಾ-ರನು
ಸುಭಾ-ಷಿ-ತ-ಕಾ-ರರು
ಸುಭಾ-ಷಿ-ತದ
ಸುಭಾ-ಷಿ-ತ-ದಂತೆ
ಸುಭಾ-ಷಿ-ತವು
ಸುಮ-ಗ-ಳಿಂದ
ಸುಮ-ತಿಗೆ
ಸುಮ-ತಿಯ
ಸುಮ-ತಿ-ಯಿಂದ
ಸುಮ-ಧುರ
ಸುಮಾರು
ಸುಮ್ಮ-ನಿ-ರುವ
ಸುಮ್ಮ-ನಿ-ರು-ವುದೇ
ಸುಮ್ಮನೇ
ಸುರಕ್ಷೆ
ಸುರ-ಸಾ-ಪ-ತ್ರಿ-ಕೆಯ
ಸುರಿವ
ಸುರು-ಳಿಯ
ಸುರು-ಳಿ-ಯೆಲ್ಲ
ಸುರೇಶ
ಸುರೇಶ್
ಸುಲಭ
ಸುಲ-ಭದ
ಸುಲ-ಭ-ವಲ್ಲ
ಸುಲ-ಭ-ವಾಗಿ
ಸುಲ-ಲಿತ
ಸುಲ-ಲಿ-ತ-ವಾದ
ಸುಳಿ-ಯ-ಗೊ-ಡದೇ
ಸುಳ್ಳು
ಸುವಿ-ಖ್ಯಾ-ತ-ರಾದ
ಸುವ್ಯ-ವ-ಸ್ಥೆ-ಗೊ-ಳಿಸಿ
ಸುಶ್ರಾ-ವ್ಯ-ವಾಗಿ
ಸುಶ್ರು-ತನ
ಸುಶ್ರು-ತ-ಸಂ-ಹಿತಾ
ಸುಶ್ರುತೋ
ಸುಸಂ-ದರ್ಭ
ಸುಸಂ-ಸ್ಕೃತ
ಸುಸಂ-ಸ್ಕೃ-ತ-ರಾ-ದು-ದ-ರಿಂದ
ಸುಸಂ-ಸ್ಕೃ-ತ-ವ್ಯ-ಕ್ತಿ-ಯಾಗಿ
ಸುಸ್ತು
ಸುಸ್ತೋ
ಸುಹೃಚ್ಚ
ಸುಹೃ-ದರು
ಸೂಕ್ತ
ಸೂಕ್ತ-ಗ-ಳನ್ನು
ಸೂಕ್ತ-ವಾದ
ಸೂಕ್ತಿ
ಸೂಕ್ತ-ರು-ದ್ರ-ಚ-ಮ-ಕಾ-ದಿ-ಗಳ
ಸೂಕ್ಷ್ಮ
ಸೂಕ್ಷ್ಮಾ
ಸೂಚ-ಕ-ವೆಂದು
ಸೂಚನೆ
ಸೂಚ-ನೆ-ಗ-ಳನ್ನು
ಸೂಚ-ನೆ-ಯನ್ನು
ಸೂಚ-ನೆ-ಯಾ-ಯಿತು
ಸೂಚಿ-ಸ-ಲಾ-ಗು-ತ್ತಿತ್ತು
ಸೂಚಿ-ಸಿದ
ಸೂಚಿ-ಸಿ-ದ-ರಂತೆ
ಸೂಚಿ-ಸಿ-ದರು
ಸೂಚಿ-ಸಿ-ದ-ವನೂ
ಸೂಚಿ-ಸಿ-ದ್ದರು
ಸೂಚಿ-ಸು-ತ್ತದೆ
ಸೂಜಿ-ಗ-ಲ್ಲಿ-ನಂತೆ
ಸೂತ್ರ
ಸೂತ್ರ-ಗಳ
ಸೂತ್ರ-ಧಾರಿ
ಸೂತ್ರ-ಧಾ-ರಿಕೆ
ಸೂತ್ರ-ರೂ-ಪ-ದಲ್ಲಿ
ಸೂತ್ರ-ಸ್ಥಾನ
ಸೂತ್ರ-ಸ್ಥಾ-ನದ
ಸೂತ್ರ-ಸ್ಥಾ-ನವು
ಸೂತ್ರಿ-ತಾಃ
ಸೂರಿ
ಸೂರಿ-ನಲ್ಲಿ
ಸೂರು
ಸೂರೆ-ಗೊಂ-ಡವು
ಸೂರೆ-ಗೊ-ಳ್ಳು-ವ-ವರು
ಸೂರ್ಯನ
ಸೂರ್ಯ-ನಾ-ರಾ-ಯಣ
ಸೂರ್ಯೋ-ದ-ಯ-ವಾ-ದಾಗ
ಸೃಷ್ಟಿ-ಶೀಲ
ಸೃಷ್ಟಿಸಿ
ಸೃಷ್ಟಿ-ಸಿತು
ಸೃಷ್ಟಿ-ಸಿ-ದರು
ಸೃಷ್ಟಿ-ಸ್ಥಿ-ತಿ-ಲ-ಯ-ಗಳ
ಸೆರೆ
ಸೆರೆ-ಸಿಕ್ಕ
ಸೆರೆ-ಹಿ-ಡಿ-ಯ-ಲಾ-ಗಿದೆ
ಸೆಲೆ
ಸೆಲೆಗೆ
ಸೆಳೆತ
ಸೆಳೆ-ತಕ್ಕೆ
ಸೆಳೆ-ದಿತ್ತು
ಸೆಳೆ-ದಿರಿ
ಸೆಳೆ-ದಿವೆ
ಸೆಳೆದು
ಸೆಳೆ-ಯುತ್ತಾ
ಸೆಶನ್
ಸೇತು-ಬಂಧ
ಸೇತು-ವೆಯ
ಸೇತು-ವೆ-ಯ-ನ್ನಾ-ಗಿ-ಸಿ-ಕೊ-ಳ್ಳು-ತ್ತಾರೆ
ಸೇತು-ವೆ-ಯಾ-ಗಲು
ಸೇರ-ಬ-ಹು-ದೆಂಬ
ಸೇರ-ಬೇಕು
ಸೇರ-ಬೇ-ಕೆಂಬ
ಸೇರಲು
ಸೇರಿ
ಸೇರಿ-ಕೊಂ-ಡಂ-ದಿ-ನಿಂದ
ಸೇರಿ-ಕೊಂ-ಡರು
ಸೇರಿ-ಕೊಂ-ಡಳು
ಸೇರಿ-ಕೊಂ-ಡಿದ್ದ
ಸೇರಿ-ಕೊಂಡು
ಸೇರಿ-ಕೊಂಡೆ
ಸೇರಿ-ಕೊ-ಳ್ಳಲು
ಸೇರಿತು
ಸೇರಿದ
ಸೇರಿ-ದಂತೆ
ಸೇರಿ-ದರು
ಸೇರಿ-ದ-ವನೂ
ಸೇರಿ-ದ-ವರು
ಸೇರಿ-ದಾಗ
ಸೇರಿದ್ದ
ಸೇರಿ-ದ್ದ-ರಿಂದ
ಸೇರಿ-ದ್ದರು
ಸೇರಿದ್ದು
ಸೇರಿದ್ದೆ
ಸೇರಿ-ದ್ದೆ-ನಲ್ಲ
ಸೇರಿ-ದ್ದೆವು
ಸೇರಿದ್ದೇ
ಸೇರಿ-ಯಾ-ಗಿತ್ತು
ಸೇರಿ-ರ-ಲಿಲ್ಲ
ಸೇರಿ-ಸಿ-ಕೊಂ-ಡರು
ಸೇರಿ-ಸಿ-ಕೊಂಡು
ಸೇರಿ-ಸಿ-ದರು
ಸೇರಿ-ಸಿ-ದೆವು
ಸೇರಿಸು
ಸೇರು-ತ್ತದೆ
ಸೇರುವ
ಸೇರು-ವಂ-ತಾ-ಯಿತು
ಸೇರು-ವಾ-ಗಲೇ
ಸೇರು-ವು-ದ-ರಿಂದ
ಸೇರ್ಪ-ಡೆ-ಗೊ-ಳ್ಳುವ
ಸೇರ್ಪ-ಡೆ-ಯಾದ
ಸೇವನೆ
ಸೇವಾ
ಸೇವಾ-ಕಾ-ಲ-ದ-ಲ್ಲಿದ್ದ
ಸೇವಾ-ನಿ-ವೃ-ತ್ತಿ-ಯನ್ನು
ಸೇವಾ-ರೂಪ
ಸೇವಾ-ವ-ಧಿ-ಯಲ್ಲಿ
ಸೇವಿಂಗ್
ಸೇವಿ-ಸ-ದಿ-ದ್ದರೆ
ಸೇವಿ-ಸ-ಬೇಡಿ
ಸೇವಿ-ಸ-ಲಾ-ಗ-ಲಿ-ಲ್ಲ-ವಲ್ಲ
ಸೇವಿ-ಸಿ-ದರೆ
ಸೇವಿ-ಸು-ವ-ವನು
ಸೇವೆ
ಸೇವೆ-ಗಷ್ಟೇ
ಸೇವೆಗೆ
ಸೇವೆಗೇ
ಸೇವೆ-ಗೈ-ದುದು
ಸೇವೆಯ
ಸೇವೆ-ಯನ್ನು
ಸೇವೆ-ಯಾ-ಗಲೀ
ಸೇವೆ-ಯಿಂದ
ಸೇವೆ-ಯಿಂ-ದಷ್ಟೇ
ಸೈ
ಸೈಕಲ್
ಸೊಂಟದ
ಸೊಗ-ಸಾ-ಗಿ-ರು-ತ್ತಿತ್ತು
ಸೊಗಸು
ಸೊಜಿ-ಗ-ವಾ-ದರೂ
ಸೊಧರ
ಸೊಪ್ಪಿಲ್ಲ
ಸೊರ-ಗಿತು
ಸೊಲ್ಲೆ-ತ್ತುವ
ಸೋ
ಸೋದರ
ಸೋಪಾನ
ಸೋಮ
ಸೋಮಣ್ಣ
ಸೋಮ-ಣ್ಣ-ನಿಗೆ
ಸೋಮ-ಯಾ-ಜಿ-ಗ-ಳನ್ನು
ಸೋಮ-ಯಾ-ಜಿ-ಗ-ಳಲ್ಲಿ
ಸೋಮ-ಯಾ-ಜಿ-ಗಳು
ಸೋಮ-ವಾರ
ಸೋಮ-ಸಾ-ಗ-ರದ
ಸೋರಾ-ಮ-ಸ್ವಾಮಿ
ಸೋಲು-ತ್ತಾರೆ
ಸೋಲು-ವುದು
ಸೋಲೊ-ಲ್ಲದ
ಸೌ
ಸೌಕ-ರ್ಯ-ಪ-ಡೆದೆ
ಸೌಖ್ಯ-ವನ್ನು
ಸೌಜನ್ಯ
ಸೌಜ-ನ್ಯದ
ಸೌಜ-ನ್ಯ-ದಿಂದ
ಸೌಭಾಗ್ಯ
ಸೌಭಾ-ಗ್ಯ-ವನ್ನು
ಸೌಭಾ-ಗ್ಯ-ವೆಂದು
ಸೌಭಾ-ಗ್ಯವೇ
ಸೌಮ್ಯ-ಮೂರ್ತಿ
ಸೌಲಭ್ಯ
ಸೌಹೃ-ದ-ಸ್ವ-ಭಾವ
ಸ್ಟಂಡ್
ಸ್ಟ್ಯಾಂಡಿ-ನಲ್ಲಿ
ಸ್ಣೇಹಿ-ತರ
ಸ್ತಂಭ
ಸ್ತರಕ್ಕೆ
ಸ್ತರ-ದಲ್ಲೂ
ಸ್ತರ-ವನ್ನು
ಸ್ತುತಿಃ
ಸ್ತುತಿ-ಲ-ಕ್ಷ-ಣ-ದಂತೆ
ಸ್ತುತ್ಯರ್ಹ
ಸ್ತುತ್ಯ-ರ್ಹ-ಸಂ-ಗ-ತಿ-ಯಾ-ಗಿದೆ
ಸ್ತೋತ್ರ-ಸಾ-ಹಿ-ತ್ಯದ
ಸ್ತ್ರೀಯ-ರಿ-ಬ್ಬರ
ಸ್ಥಗಿ-ತ-ಗೊಂ-ಡಿದ್ದ
ಸ್ಥರ-ಗ-ಳಲ್ಲಿ
ಸ್ಥಳ
ಸ್ಥಳ-ಗ-ಳಲ್ಲಿ
ಸ್ಥಳ-ದಲ್ಲಿ
ಸ್ಥಳ-ವೆ-ಲ್ಲಾ-ದರೂ
ಸ್ಥಳಾಂ-ತ-ರಿ-ಸ-ಲಿಲ್ಲ
ಸ್ಥಳಾ-ವ-ಕಾಶ
ಸ್ಥಳೀ-ಯ-ರಿಗೂ
ಸ್ಥಾನ
ಸ್ಥಾನ-ಗ-ಳನ್ನು
ಸ್ಥಾನ-ಗ-ಳಲ್ಲಿ
ಸ್ಥಾನದ
ಸ್ಥಾನ-ದಲ್ಲಿ
ಸ್ಥಾನ-ದ-ಲ್ಲಿ-ರುವ
ಸ್ಥಾನ-ದಿಂದ
ಸ್ಥಾನ-ವನ್ನು
ಸ್ಥಾನ-ವೆಂದು
ಸ್ಥಾನಾಂ-ತ-ರ-ವಾದ
ಸ್ಥಾನಾಂ-ತ-ರಿ-ತ-ರಾಗಿ
ಸ್ಥಾನಾಂ-ತ-ರಿ-ಸ-ಬೇ-ಕಾ-ಯಿತು
ಸ್ಥಾನಾ-ದಿ-ಗ-ಳನ್ನೂ
ಸ್ಥಾನಾ-ಪ-ನ್ನ-ರಾದ
ಸ್ಥಾನೀ-ಯ-ವಾಗಿ
ಸ್ಥಾನೇ
ಸ್ಥಾಪ-ನೆ-ಯಾ-ಗ-ಲೇ-ಬೇ-ಕೆಂದು
ಸ್ಥಾಪ-ನೆ-ಯಾಗಿ
ಸ್ಥಾಪಿ-ಸ-ಬೇ-ಕೆಂದು
ಸ್ಥಾಪಿ-ಸ-ಲೇ-ಬೇಕು
ಸ್ಥಾಪಿ-ಸ-ಲ್ಪಟ್ಟ
ಸ್ಥಾಪಿ-ಸಿದ
ಸ್ಥಾವ-ರಾ-ಣಾಂ
ಸ್ಥಿತಿ
ಸ್ಥಿತಿ-ಗ-ತಿ-ಗಳು
ಸ್ಥಿತಿಗೆ
ಸ್ಥಿತಿ-ಯನ್ನು
ಸ್ಥಿತಿ-ಯಲ್ಲಿ
ಸ್ಥಿತಿ-ಯ-ಲ್ಲಿದ್ದೆ
ಸ್ಥಿತಿ-ಯಿ-ರುವ
ಸ್ಥಿರ
ಸ್ಥಿರ-ವಾ-ಗಿದೆ
ಸ್ಥಿರ-ವಾ-ಗು-ತ್ತಿತ್ತು
ಸ್ಥಿರ-ವಾದ
ಸ್ನಾತ-ಕೋ-ತ್ತರ
ಸ್ನಾನ
ಸ್ನಾನದ
ಸ್ನಾನಾ-ನಂ-ತರ
ಸ್ನೆಹಿ-ತನ
ಸ್ನೇಯಿ-ತ-ರಂತೆ
ಸ್ನೇಹ
ಸ್ನೇಹದ
ಸ್ನೇಹ-ವಾಗಿ
ಸ್ನೇಹ-ವಿ-ಲ್ಲ-ದಿ-ದ್ದರೆ
ಸ್ನೇಹ-ಶೀ-ಲತೆ
ಸ್ನೇಹಿತ
ಸ್ನೇಹಿ-ತ-ನಂತೆ
ಸ್ನೇಹಿ-ತರ
ಸ್ನೇಹಿ-ತ-ರಾಗಿ
ಸ್ನೇಹಿ-ತ-ರಾ-ಗಿದ್ದ
ಸ್ನೇಹಿ-ತರು
ಸ್ಪಂದಿ-ಸುವ
ಸ್ಪಂದಿ-ಸು-ವುದು
ಸ್ಪಟಿಕ
ಸ್ಪರ್ದೇ-ಗ-ಳಾ-ಗಲಿ
ಸ್ಪರ್ಧಾ-ವಿ-ಷ-ಯ-ವನ್ನು
ಸ್ಪರ್ಧಿ
ಸ್ಪರ್ಧೆ
ಸ್ಪರ್ಧೆ-ಗ-ಳಲ್ಲಿ
ಸ್ಪರ್ಧೆ-ಗ-ಳಲ್ಲೂ
ಸ್ಪರ್ಧೆ-ಗ-ಳಿಂದ
ಸ್ಪರ್ಧೆ-ಗ-ಳಿಗೂ
ಸ್ಪರ್ಧೆ-ಗ-ಳಿಗೆ
ಸ್ಪರ್ಧೆ-ಗಳು
ಸ್ಪರ್ಧೆಯ
ಸ್ಪರ್ಧೆ-ಯನ್ನು
ಸ್ಪರ್ಧೆ-ಯಲ್ಲಿ
ಸ್ಪರ್ಧೆ-ಯ-ಲ್ಲಿ-ದ್ದಾ-ನೆಂ-ದರೆ
ಸ್ಪರ್ಧೇ-ಗ-ಳಾ-ಗಲಿ
ಸ್ಪರ್ಶವೇ
ಸ್ಪರ್ಷ-ಶಿ-ಬಿ-ರ-ಕ್ಕಾಗಿ
ಸ್ಪಷ್ಟ
ಸ್ಪಷ್ಟ-ಗೊ-ಳ್ಳು-ತ್ತದೆ
ಸ್ಪಷ್ಟತೆ
ಸ್ಪಷ್ಟ-ಪ-ಡಿ-ಸಿ-ದ್ದಾರೆ
ಸ್ಪಷ್ಟ-ಪ-ಡಿ-ಸುತ್ತಾ
ಸ್ಪಷ್ಟ-ವಾಗಿ
ಸ್ಪಷ್ಟ-ವಾ-ಗಿದೆ
ಸ್ಪಷ್ಟ-ವಾ-ಗಿ-ಸಿ-ದರು
ಸ್ಪಷ್ಟ-ವಾದ
ಸ್ಪಷ್ಟ-ವಾ-ಯಿತು
ಸ್ಪೀಶೀಸ್
ಸ್ಪೂರ್ತಿ
ಸ್ಪೂರ್ತಿಯ
ಸ್ಪೂರ್ತಿ-ಯಾ-ಗು-ತ್ತಾರೆ
ಸ್ಪೂರ್ತಿ-ಯಿಂದ
ಸ್ಫಟಿ-ಕದ
ಸ್ಫಟಿ-ಕೇನ
ಸ್ಫೂರ್ತಿ
ಸ್ಮರ-ಣ-ಸಂ-ಚಿಕೆ
ಸ್ಮರ-ಣಾರ್ಹ
ಸ್ಮರ-ಣೀ-ಯ-ವೆ-ನಿ-ಸು-ತ್ತದೆ
ಸ್ಮರಣೆ
ಸ್ಮರಿ-ಸದೇ
ಸ್ಮರಿಸಿ
ಸ್ಮರಿ-ಸಿ-ದ್ದಾರೆ
ಸ್ಮರಿ-ಸು-ತ್ತೇನೆ
ಸ್ಮರಿ-ಸು-ತ್ತೇ-ನೆಂ-ದರೆ
ಸ್ಮರಿ-ಸುವ
ಸ್ಮಾರ-ಕ-ವಾಗಿ
ಸ್ಮಿತ್
ಸ್ಮೃತಃ
ಸ್ಮೃತಮ್
ಸ್ಮೃತಾ
ಸ್ಮೃತಾಃ
ಸ್ಮೃತಿ
ಸ್ಮೃತಿ-ಪ-ಟ-ಲ-ದಲ್ಲಿ
ಸ್ಮೃತಿಯ
ಸ್ಮೃತಿ-ಯಲ್ಲಿ
ಸ್ಮೋ
ಸ್ವ
ಸ್ವಂತ
ಸ್ವಗ್ರಾ-ಮ-ವಾದ
ಸ್ವಚ್ಛ
ಸ್ವತ
ಸ್ವತಂ-ತ್ರ-ವಾಗಿ
ಸ್ವತಃ
ಸ್ವದೇ-ಶದ
ಸ್ವದೇಶೇ
ಸ್ವದೊ-ಷಾ-ವಿ-ಷ್ಕ-ರ-ಣವೇ
ಸ್ವಪ್ರ-ಕಾಶ
ಸ್ವಭಾವ
ಸ್ವಭಾ-ವಕ್ಕೆ
ಸ್ವಭಾ-ವತಃ
ಸ್ವಭಾ-ವದ
ಸ್ವಭಾ-ವ-ದ-ವರು
ಸ್ವಭಾ-ವನ್ನು
ಸ್ವಭಾ-ವ-ವಿಲ್ಲ
ಸ್ವಭಾ-ವ-ವು-ಳ್ಳ-ವರು
ಸ್ವಭಾ-ವವೂ
ಸ್ವಯಂ
ಸ್ವಯ-ಮಾ-ಯಾಂತಿ
ಸ್ವರ
ಸ್ವರೂಪ
ಸ್ವರೂ-ಪದ
ಸ್ವರೂ-ಪ-ದ-ಲ್ಲಿಯೂ
ಸ್ವರೂ-ಪ-ವನ್ನು
ಸ್ವರೂ-ಪ-ವಾ-ದರೂ
ಸ್ವರೂ-ಪವೇ
ಸ್ವರ್ಗ
ಸ್ವರ್ಗ-ಮೋ-ಕ್ಷ-ಗ-ಳನ್ನು
ಸ್ವರ್ಣ-ಪು-ಷ್ಪಕ್ಕೆ
ಸ್ವರ್ಣ-ವಲ್ಲಿ
ಸ್ವರ್ಣ-ವಲ್ಲೀ
ಸ್ವಲ್ಪ
ಸ್ವಲ್ಪ-ಕಾಲ
ಸ್ವಲ್ಪ-ದ-ರಲ್ಲೇ
ಸ್ವಲ್ಪ-ಮ-ಟ್ಟಿಗೆ
ಸ್ವಲ್ಪ-ಮ-ಟ್ಟಿನ
ಸ್ವಲ್ಪವೂ
ಸ್ವಲ್ವವೂ
ಸ್ವಶ್ರೇ-ಯ-ಸ್ಸಿಗೆ
ಸ್ವಾಗ-ತಕ್ಕೆ
ಸ್ವಾಗ-ತಿ-ಸಿದ
ಸ್ವಾಗ-ತಿ-ಸಿ-ದರು
ಸ್ವಾಗ-ತಿ-ಸು-ತ್ತಲೇ
ಸ್ವಾಗ-ತಿ-ಸುತ್ತಾ
ಸ್ವಾಗ-ತಿ-ಸುವ
ಸ್ವಾದು
ಸ್ವಾಭಾ-ವಿಕ
ಸ್ವಾಭಾ-ವಿ-ಕ-ವಾ-ಗಿಯೇ
ಸ್ವಾಭಿ-ಮಾನ
ಸ್ವಾಭಿ-ಮಾ-ನ-ವನ್ನು
ಸ್ವಾಮಿ
ಸ್ವಾಮಿ-ಗಳ
ಸ್ವಾಮಿ-ಗ-ಳಿಗೆ
ಸ್ವಾಮಿ-ಗಳು
ಸ್ವಾಮಿಗೆ
ಸ್ವಾಮಿ-ನಾ-ರಾ-ಯಣ
ಸ್ವಾಮಿನೌ
ಸ್ವಾಮಿ-ಯನ್ನು
ಸ್ವಾಮಿ-ಯ-ವರು
ಸ್ವಾಮೀ-ಜಿ-ಗಳ
ಸ್ವಾಮೀ-ಜಿ-ಯೊ-ಬ್ಬರು
ಸ್ವಾಯತ್ತ
ಸ್ವಾರ-ಸ್ಯ-ಕ-ರ-ವಾದ
ಸ್ವಾರ-ಸ್ಯ-ಪೂರ್ಣ
ಸ್ವಾರ-ಸ್ಯ-ವೆಂ-ದರೆ
ಸ್ವಾರ್ಥ-ಗಳೂ
ಸ್ವಾರ್ಥದ
ಸ್ವಾರ್ಥ-ವಿ-ಲ್ಲದೇ
ಸ್ವಾರ್ಥವೂ
ಸ್ವಾವ-ಲಂ-ಬನ
ಸ್ವಾಸ್ಥ್ಯ-ಗುಟ್ಟು
ಸ್ವೀಕ-ರಿ-ಸದ
ಸ್ವೀಕ-ರಿಸಿ
ಸ್ವೀಕ-ರಿ-ಸಿ-ದ್ದೇನೆ
ಸ್ವೀಕ-ರಿ-ಸುವ
ಹ
ಹಂಚ-ಬೇ-ಕೆಂದು
ಹಂಚಿ
ಹಂಚಿ-ಕೊಂ-ಡಾ-ಗ-ಲೆಲ್ಲ
ಹಂಚಿ-ಕೊಂಡು
ಹಂಚಿ-ಕೊ-ಳ್ಳಲು
ಹಂಚಿ-ಕೊ-ಳ್ಳು-ತ್ತಾರೆ
ಹಂಚಿ-ಕೊ-ಳ್ಳು-ತ್ತಿ-ದ್ದೆವು
ಹಂಚಿ-ಕೊ-ಳ್ಳು-ತ್ತೇನೆ
ಹಂಚಿ-ಕೊ-ಳ್ಳುವ
ಹಂಚಿ-ಕೊ-ಳ್ಳು-ವದು
ಹಂಚುತ್ತ
ಹಂಚು-ತ್ತಿ-ರುವ
ಹಂಡೆ
ಹಂಡೆಯ
ಹಂತ-ಗ-ಳನ್ನೂ
ಹಂತದ
ಹಂತ-ದಲ್ಲಿ
ಹಂತ-ದಲ್ಲೇ
ಹಂತ-ದ-ವ-ರೆಗೂ
ಹಂಬಲ
ಹಂಸ
ಹಕಾರಂ
ಹಕಾ-ರ-ಗಳ
ಹಗ್ಗ
ಹಗ್ಗ-ವನ್ನು
ಹಚ್ಚ
ಹಚ್ಚಿ
ಹಚ್ಚಿ-ಕೊಂ-ಡಿದ್ದ
ಹಠ
ಹಣ
ಹಣ-ಗ-ಳಿಸಿ
ಹಣ-ಗ-ಳಿ-ಸುವ
ಹಣದ
ಹಣ-ದಲ್ಲಿ
ಹಣ-ದಿಂದ
ಹಣ-ಬೇ-ಕೆಂದು
ಹಣ-ವನ್ನು
ಹಣ-ವಿ-ದ್ದಾಗ
ಹಣ-ವಿ-ಲ್ಲ-ದಿ-ದ್ದಲ್ಲಿ
ಹಣವು
ಹಣ್ಣನ್ನು
ಹಣ್ಣಿನ
ಹಣ್ಣು-ಗ-ಳನ್ನು
ಹತೈ-ಷಿ-ಗಳೂ
ಹತೋ-ಟಿ-ಯನ್ನೇ
ಹತೋ-ಟಿ-ಯ-ಲ್ಲಿ-ಲ್ಲ-ಎಂದು
ಹತೋ-ಟಿ-ಯಿ-ರು-ತ್ತದೆ
ಹತೋ-ಟಿ-ಯಿ-ರು-ವು-ದಿಲ್ಲ
ಹತ್ತನೆ
ಹತ್ತ-ನೆಯ
ಹತ್ತನೇ
ಹತ್ತ-ನೇ-ಕ್ಲಾಸ್
ಹತ್ತಲು
ಹತ್ತಾಗಿ
ಹತ್ತಾರು
ಹತ್ತಿ
ಹತ್ತಿ-ಕ್ಕು-ವುದು
ಹತ್ತಿರ
ಹತ್ತಿ-ರಕ್ಕೆ
ಹತ್ತಿ-ರದ
ಹತ್ತಿ-ರ-ದಿಂದ
ಹತ್ತಿ-ರ-ವಿ-ರುವ
ಹತ್ತಿ-ರವೇ
ಹತ್ತಿ-ಸಿ-ದರು
ಹತ್ತು
ಹತ್ತು-ವಿ-ಧ-ವಾಗಿ
ಹತ್ತು-ಸಾ-ವಿ-ರ-ಪಾಲು
ಹದ-ಗೊ-ಳಿ-ಸುವ
ಹದಾ
ಹದಿ-ನಾರು
ಹದಿ-ನೈದು
ಹನು-ಮಂತ
ಹನು-ಮಂ-ತನ
ಹನು-ಮಂ-ತ-ನಿಗೆ
ಹನು-ಮನ
ಹನು-ಮ-ನಲ್ಲಿ
ಹನು-ಮ-ನಿಗೆ
ಹನು-ಮ-ನಿದ್ದ
ಹನ್ನೊಂದು
ಹಪ-ಹ-ಪಿ-ಸು-ತ್ತಿ-ದ್ದರು
ಹಬ್ಬದ
ಹಬ್ಬ-ಹ-ರಿ-ದಿ-ನ-ಗ-ಳಲ್ಲಿ
ಹಮ್ಮಿ-ಕೊಂ-ಡಿದ್ದು
ಹಮ್ಮಿ-ಕೊಂ-ಡಿ-ರುವ
ಹಮ್ಮಿ-ಕೊಂಡು
ಹರ-ಟೆ-ಯನ್ನು
ಹರ-ಡಿ-ದೆ-ಮೊ-ದ-ಲ-ನೆ-ಯದು
ಹರ-ವನ್ನು
ಹರ-ಸಿ-ದ್ದಾರೆ
ಹರಹು
ಹರಿ-ಕಥೆ
ಹರಿ-ಕ-ಥೆ-ದಾ-ಸರ
ಹರಿದ
ಹರಿ-ದಾ-ಡ-ದಿ-ರ-ಬೇಕು
ಹರಿ-ದಾ-ಡುವ
ಹರಿ-ಯುವ
ಹರಿ-ಯು-ವಂ-ತಾ-ಗಲಿ
ಹರಿ-ಯು-ವಂತೆ
ಹರಿ-ಸ-ಬೇ-ಕೆಂದು
ಹರಿ-ಸು-ತ್ತಿ-ರ-ಲಿಲ್ಲ
ಹರ್ಯಾ-ಣದ
ಹರ್ಷಿ-ಸು-ತ್ತೇವೆ
ಹರ್ಷೋ-ತ್ಸಾ-ಹ-ದಿಂದ
ಹರ-ಸಾ-ಹ-ಸ-ದಿಂದ
ಹಲ-ಗೆಯ
ಹಲ-ವರ
ಹಲ-ವ-ರನ್ನು
ಹಲ-ವ-ರಿಗೆ
ಹಲ-ವರು
ಹಲ-ವಾರು
ಹಲವು
ಹಲ-ವು-ಕ್ಷೇ-ತ್ರ-ಗಳ
ಹಲ-ಸಿನ
ಹಲ-ಸಿ-ನ-ಹ-ಣ್ಣಿನ
ಹಲ್ಲಿ-ಲ್ಲ-ದ-ವನು
ಹಲ್ಲೆ
ಹಲ್ಲೆ-ಮಾ-ಡಿ-ದ-ವ-ನನ್ನು
ಹಳ-ತ-ಕ-ಟ್ಟ-ದ-ಲ್ಲಿನ
ಹಳ-ತಾ-ಗದ
ಹಳ-ತಾ-ಗಿಲ್ಲ
ಹಳ-ತಾ-ಗು-ವು-ದಾ-ದರೂ
ಹಳ-ದೋ-ಟದ
ಹಳಿಗೆ
ಹಳೆ
ಹಳೆಯ
ಹಳೆ-ಯ-ಕಾ-ಲದ
ಹಳೆ-ಯ-ದಾದ
ಹಳ್ಳಿ
ಹಳ್ಳಿ-ಗರ
ಹಳ್ಳಿ-ಗಳ
ಹಳ್ಳಿಯ
ಹಳ್ಳಿ-ಯಲ್ಲಿ
ಹಳ್ಳಿ-ಯಿಂದ
ಹವ-ಮಾನ
ಹವಿ-ರ್ಯಜ್ಞಾ
ಹವಿಷ್ಯಂ
ಹವಿಸ್ಸು
ಹವೀಕ
ಹವ್ಯಕ
ಹವ್ಯ-ಕ-ನಂತೆ
ಹವ್ಯ-ಕರ
ಹವ್ಯ-ಕ-ರಂತೂ
ಹವ್ಯ-ಕ-ಸಂ-ಘ-ದಲ್ಲೂ
ಹವ್ಯಾಶ
ಹವ್ಯಾಸ
ಹಸ-ನಾ-ಗಿ-ರ-ಬೇ-ಕೆಂದು
ಹಸ-ನಾ-ಗಿ-ಸಲು
ಹಸ-ನ್ಮುಖಿ
ಹಸ-ನ್ಮು-ಖಿ-ಯಾ-ಗಿ-ರು-ತ್ತಿದ್ದ
ಹಸ-ನ್ಮು-ಖಿ-ಯಾದ
ಹಸಿದು
ಹಸಿ-ಯಾಗೇ
ಹಸಿ-ರಾಗಿ
ಹಸಿ-ರಾ-ಗಿದೆ
ಹಸಿ-ರಾ-ಗಿಯೇ
ಹಸಿ-ವನ್ನ
ಹಸಿ-ವನ್ನು
ಹಸಿ-ವಿನ
ಹಸಿವು
ಹಸ್ತ
ಹಸ್ತಂ-ಗ-ತ-ವಾ-ಗುವ
ಹಸ್ತಕ್ಕೇ
ಹಸ್ತ-ಮಾ-ರೋಪ್ಯ
ಹಸ್ತರು
ಹಸ್ತೌ
ಹಾಂಗ್ಕಾಂ-ಗಿನ
ಹಾಂಗ್ಕಾಂ-ಗ್ನಲ್ಲಿ
ಹಾಕ-ಲಿ-ಲ್ಲ-ವಂತೆ
ಹಾಕಲು
ಹಾಕಿ
ಹಾಕಿ-ದರೆ
ಹಾಕಿದೆ
ಹಾಕಿ-ದ್ದರೂ
ಹಾಕಿ-ಸಿ-ಕೊಂಡು
ಹಾಕಿ-ಸಿದ
ಹಾಕುವ
ಹಾಕು-ವ-ವರು
ಹಾಕು-ವುದೇ
ಹಾಗಂತ
ಹಾಗಾ-ಗ-ದಂತೆ
ಹಾಗಾಗಿ
ಹಾಗಾ-ಗಿಯೇ
ಹಾಗಾ-ಗಿಲ್ಲ
ಹಾಗಾ-ದರೆ
ಹಾಗಿ-ದ್ದರೂ
ಹಾಗಿಲ್ಲ
ಹಾಗಿವೆ
ಹಾಗು
ಹಾಗೂ
ಹಾಗೆ
ಹಾಗೆಂದ
ಹಾಗೆಂದು
ಹಾಗೆಯೇ
ಹಾಗೇ
ಹಾಗೇ-ನಾ-ದರೂ
ಹಾಜ-ರಾ-ಗಿದ್ದೆ
ಹಾಜ-ರಾ-ಗು-ವಂತೆ
ಹಾಜ-ರಾದೆ
ಹಾಡಿ
ಹಾಡು
ಹಾಡುತ್ತಾ
ಹಾತೊ-ರೆ-ಯಲು
ಹಾದಿ
ಹಾದ್ದ-ರಿಂದ
ಹಾನ-ಗಲ್
ಹಾನಿ-ಯುಂಟು
ಹಾರ
ಹಾರಳ್ಳೆ
ಹಾರಾ-ಡು-ತ್ತಿದ್ದ
ಹಾರೈಕೆ
ಹಾರೈ-ಕೆ-ಯೊಂ-ದಿಗೆ
ಹಾರೈಸಿ
ಹಾರೈ-ಸುತ್ತಾ
ಹಾರೈ-ಸು-ತ್ತೇನೆ
ಹಾರೈ-ಸೋಣ
ಹಾರ್ಸಿ-ಕಟ್ಟಾ
ಹಾಲನ್ನು
ಹಾಲನ್ನೇ
ಹಾಲಿ
ಹಾಲು
ಹಾಲ್ಗೆ
ಹಾಳ-ದ-ಕ-ಟ್ಟಾದ
ಹಾಸ
ಹಾಸಲು
ಹಾಸ-ಲು
ಹಾಸಿಗೆ
ಹಾಸು-ಹೊ-ಕ್ಕಾ-ಗಿತ್ತು
ಹಾಸು-ಹೊ-ಕ್ಕಾ-ಗಿ-ದೆ-ಇ-ದ್ದುದು
ಹಾಸ್ಟೆ-ಲಿನ
ಹಾಸ್ಟೆಲ್
ಹಾಸ್ಟೆ-ಲ್ನಲ್ಲಿ
ಹಾಸ್ಟೇ-ಲಿನ
ಹಾಸ್ಯ-ಪ್ರಜ್ಞೆ
ಹಾಸ್ಯ-ಪ್ರಿಯೆ
ಹಿ
ಹಿಂಜ-ರಿ-ದಾಗ
ಹಿಂಜ-ರಿ-ಯು-ತ್ತಿದ್ದೆ
ಹಿಂತಿ-ರು-ಗಿ-ದರು
ಹಿಂತಿ-ರು-ಗು-ವುದೂ
ಹಿಂದಕ್ಕೆ
ಹಿಂದಿ
ಹಿಂದಿತ್ತು
ಹಿಂದಿನ
ಹಿಂದಿ-ನಿಂ-ದಲೂ
ಹಿಂದಿ-ರುಗಿ
ಹಿಂದಿ-ರು-ಗಿದ
ಹಿಂದಿ-ರು-ಗಿದೆ
ಹಿಂದಿ-ರು-ಗಿ-ದ್ದರು
ಹಿಂದಿ-ರು-ಗಿ-ದ್ದಾ-ಯಿತು
ಹಿಂದಿ-ರು-ಗು-ತ್ತಿ-ರ-ಲಿಲ್ಲ
ಹಿಂದಿ-ರು-ಗು-ವಾಗ
ಹಿಂದಿ-ರು-ಗು-ವುದು
ಹಿಂದಿ-ರುವ
ಹಿಂದೀ
ಹಿಂದು-ಮುಂದು
ಹಿಂದು-ಳಿದ
ಹಿಂದು-ಳಿ-ದ-ವರ
ಹಿಂದು-ಳಿ-ದ-ವರೇ
ಹಿಂದೂ
ಹಿಂದೆ
ಹಿಂದೆ
ಹಿಂದೆ-ಗೆ-ಯದ
ಹಿಂದೆ-ಮುಂದೆ
ಹಿಂದೆಯೇ
ಹಿಂದೊಮ್ಮೆ
ಹಿಂಪ-ಡೆ-ಯ-ಲಾ-ಯಿತು
ಹಿಂಬ-ದಿಯ
ಹಿಂಬಾ-ಗ-ದ-ಲ್ಲಿತ್ತು
ಹಿಂಬಾ-ಲಿ-ಸಲು
ಹಿಂಭಾ-ಗ-ದಲ್ಲೆ
ಹಿಂಸೆ
ಹಿಗ್ಗಿ-ಸಿ-ಕೊಂ-ಡಿತು
ಹಿಗ್ಗಿ-ಸು-ತ್ತ-ಲಿತ್ತು
ಹಿಗ್ಗುವ
ಹಿಟ್ಟಿ-ಗ-ಗ-ಲಿದ
ಹಿಡಿತ
ಹಿಡಿ-ತ-ವನ್ನು
ಹಿಡಿದ
ಹಿಡಿ-ದ-ವರು
ಹಿಡಿ-ದವು
ಹಿಡಿದು
ಹಿಡಿ-ದು-ಕೊ-ಳ್ಳು-ತ್ತೇನೆ
ಹಿಡಿದೇ
ಹಿಡಿ-ಬ-ಳ್ಳಿ-ಯನ್ನೇ
ಹಿಡಿ-ಯ-ಲಿಲ್ಲ
ಹಿಡಿ-ಯು-ತ್ತದೆ
ಹಿಡಿ-ಯು-ತ್ತಾರೆ
ಹಿಡಿ-ಸದೇ
ಹಿಡಿ-ಸಿತು
ಹಿಡಿ-ಸಿತ್ತು
ಹಿತ
ಹಿತಂ
ಹಿತ-ಕ-ರ-ವಲ್ಲ
ಹಿತ-ಕ್ಕಾಗಿ
ಹಿತ-ಕ್ಕಿಂ-ತಲೂ
ಹಿತ-ಚಿಂ-ತನ
ಹಿತದ
ಹಿತ-ದೃ-ಷ್ಟಿ-ಯಿಂದ
ಹಿತ-ಭುಕ್
ಹಿತ-ಮಿ-ತ-ವಾದ
ಹಿತ-ವನ್ನು
ಹಿತ-ವಾದ
ಹಿತ-ವಾ-ದದ್ದು
ಹಿತವೂ
ಹಿತಾಯ
ಹಿತಾ-ಸ-ಕ್ತಿಯ
ಹಿತಾ-ಹಾರ
ಹಿತೈ-ಷಿ-ಗಳ
ಹಿತೈ-ಷಿ-ಗ-ಳಾ-ಗಿ-ದ್ದರು
ಹಿತೈ-ಷಿ-ಗ-ಳಾ-ಗಿ-ರು-ತ್ತಿ-ದ್ದರು
ಹಿತೈ-ಷಿ-ಗಳು
ಹಿತೈ-ಷಿಣಃ
ಹಿತೈ-ಷಿ-ಯಾಗಿ
ಹಿತೈಷೀ
ಹಿತೈ-ಷೀ-ಮ-ನೀಷೀ
ಹಿತ್ತ-ಲ-ಗಿಡ
ಹಿತ್ತ-ಲಿ-ನಲ್ಲಿ
ಹಿನ್ನ-ಲೆ-ಯಲ್ಲಿ
ಹಿನ್ನೆಲೆ
ಹಿನ್ನೆ-ಲೆಯ
ಹಿನ್ನೆ-ಲೆ-ಯಲ್ಲಿ
ಹಿನ್ನೆ-ಲೆ-ಯ-ಲ್ಲಿ-ಯಿಂದ
ಹಿನ್ನೆ-ಲೆ-ಯ-ಲ್ಲಿಯೇ
ಹಿನ್ನೆ-ಲೆ-ಯಿಂದ
ಹಿನ್ನೆ-ಲೆ-ಯು-ಳ್ಳ-ವರು
ಹಿನ್ನೆ-ಲೆ-ಯೆಲ್ಲಿ
ಹಿಮಾ-ಲಯಃ
ಹಿರ-ಣ್ಯ-ರೇ-ತಸಂ
ಹಿರಿ-ತನ
ಹಿರಿ-ದಾದ
ಹಿರಿ-ದಾ-ದದ್ದು
ಹಿರಿ-ದೆಂ-ಬು-ದನ್ನು
ಹಿರಿಮೆ
ಹಿರಿ-ಮೆಗೆ
ಹಿರಿಯ
ಹಿರಿ-ಯಣ್ಣ
ಹಿರಿ-ಯ-ಣ್ಣನ
ಹಿರಿ-ಯರ
ಹಿರಿ-ಯ-ರಾದ
ಹಿರಿ-ಯ-ರಿಂದ
ಹಿರಿ-ಯ-ರಿಗೆ
ಹಿರಿ-ಯ-ರಿವು
ಹಿರಿ-ಯರು
ಹಿರಿ-ಯ-ರೆಲ್ಲಾ
ಹಿರಿ-ಯ-ವಿ-ದ್ಯಾ-ರ್ಥಿ-ಯೋ-ರ್ವ-ರನ್ನು
ಹಿರೆ-ಮನೆ
ಹಿಸೆ-ಯಾಗಿ
ಹೀಗಾಗಿ
ಹೀಗಿದೆ
ಹೀಗಿ-ದೆ
ಹೀಗಿ-ದ್ದರೂ
ಹೀಗಿ-ರ-ಬೇಕು
ಹೀಗಿ-ರು-ವಾಗ
ಹೀಗೆ
ಹೀಗೆಯೇ
ಹೀಗೇ
ಹೀಗೊಂದು
ಹೀರಿ-ಕೊ-ಳ್ಳುವ
ಹುಟ್ಟಿ-ಕೊಂ-ಡು-ಬಿ-ಡು-ತ್ತವೆ
ಹುಟ್ಟಿದ
ಹುಟ್ಟಿ-ದವು
ಹುಟ್ಟಿದ್ದ
ಹುಟ್ಟಿದ್ದೆ
ಹುಟ್ಟಿ-ಬೆ-ಳೆದ
ಹುಟ್ಟು-ಕು-ರುಡ
ಹುಟ್ಟು-ವಂತೆ
ಹುಟ್ಟು-ಹಾ-ಕುವ
ಹುಟ್ಟೂ-ರಿನ
ಹುಟ್ಟೂರು
ಹುಟ್ಟೂ-ರು-ಇದೇ
ಹುಡುಕಿ
ಹುಡು-ಕಿ-ಕೊಂಡು
ಹುಡು-ಕಿದೆ
ಹುಡು-ಕುತ್ತ
ಹುಡು-ಕು-ತ್ತಿದ್ದ
ಹುಡು-ಕು-ತ್ತಿ-ದ್ದಾಗ
ಹುಡು-ಕು-ತ್ತಿ-ರು-ವಾಗ
ಹುಡು-ಕುವ
ಹುಡು-ಕು-ವು-ದ-ಕ್ಕಾ-ಗಿಯೇ
ಹುಡು-ಕು-ವುದು
ಹುಡುಗ
ಹುಡು-ಗನ
ಹುಡು-ಗ-ನಾದ
ಹುಡು-ಗ-ನಿಗೆ
ಹುಡು-ಗನು
ಹುಡು-ಗ-ರಿ-ಗೆಲ್ಲಾ
ಹುಡು-ಗಿ-ಯ-ರಿಗೆ
ಹುಡು-ಗಿ-ಯರೇ
ಹುದು-ಗಿ-ರುವ
ಹುದು-ಗಿ-ಸಿ-ಟ್ಟು-ಕೊ-ಳ್ಳಲು
ಹುದ್ದೆ
ಹುದ್ದೆ-ಗ-ಳನ್ನು
ಹುದ್ದೆಗೆ
ಹುದ್ದೆಯ
ಹುದ್ದೆ-ಯನ್ನು
ಹುದ್ದೆ-ಯ-ಲ್ಲಿದ್ದು
ಹುನ್ನಾರ
ಹುರಿ-ಗೊ-ಳಿ-ಸಿ-ಕೊ-ಳ್ಳ-ಬೇ-ಕಾ-ಯಿತು
ಹುರಿ-ದುಂ-ಬಿ-ಸಿದ
ಹುರಿ-ದುಂ-ಬಿ-ಸು-ತ್ತದೆ
ಹುರುಪು
ಹೂ
ಹೂಡ-ಬೇ-ಕಾ-ಗು-ತ್ತಿತ್ತು
ಹೃದಯ
ಹೃದ-ಯ-ದಲ್ಲಿ
ಹೃದ-ಯ-ದ-ಲ್ಲಿದೆ
ಹೃದ-ಯ-ದಿಂದ
ಹೃದ-ಯ-ಪೂ-ರ್ವಕ
ಹೃದ-ಯ-ವಂ-ತರು
ಹೃದ-ಯ-ವನ್ನು
ಹೃದ-ಯವು
ಹೃದ-ಯ-ವೆಂಬ
ಹೃದಯಿ
ಹೃದಯೇ
ಹೃದ್ಗ-ತ-ಮಾ-ಡಿ-ಕೊಂ-ಡ-ವರು
ಹೆಂಡತಿ
ಹೆಂಡ-ತಿ-ಯನ್ನು
ಹೆಂಡ್ತಿ
ಹೆಗಡೆ
ಹೆಗ-ಡೆ-ಯ-ವರ
ಹೆಗ-ಡೆ-ಯ-ವರು
ಹೆಗ-ಲು-ಕೊಟ್ಟು
ಹೆಗ-ಲೆ-ಣೆ-ಯಾಗಿ
ಹೆಗೆ-ಡೆಗೆ
ಹೆಗ್ಗಡೆ
ಹೆಗ್ಗ-ಳಿಕೆ
ಹೆಗ್ಗ-ಳಿ-ಕೆಗೆ
ಹೆಗ್ಗ-ಳಿ-ಕೆಯ
ಹೆಗ್ಗ-ಳಿ-ಕೆ-ಯಿತ್ತು
ಹೆಗ್ಗು-ರು-ತಾಗಿ
ಹೆಗ್ಗು-ರುತು
ಹೆಚ್ಚಲ್ಲ
ಹೆಚ್ಚಾ-ಗ-ಬ-ಹುದು
ಹೆಚ್ಚಾಗಿ
ಹೆಚ್ಚಾ-ಗಿತ್ತು
ಹೆಚ್ಚಾ-ಗಿ-ರು-ತ್ತವೆ
ಹೆಚ್ಚಾ-ಗು-ತ್ತಿತ್ತು
ಹೆಚ್ಚಾ-ಗು-ತ್ತಿದೆ
ಹೆಚ್ಚಾ-ಯಿತು
ಹೆಚ್ಚಿತು
ಹೆಚ್ಚಿನ
ಹೆಚ್ಚಿ-ನ-ವರು
ಹೆಚ್ಚಿ-ಸಿತು
ಹೆಚ್ಚಿ-ಸೀ-ತೆಂಬ
ಹೆಚ್ಚಿ-ಸುತ್ತಾ
ಹೆಚ್ಚಿ-ಸು-ತ್ತಿತ್ತು
ಹೆಚ್ಚಿ-ಸು-ವು-ದಾ-ಗಿದೆ
ಹೆಚ್ಚು
ಹೆಚ್ಚು-ವುದು
ಹೆಚ್ಚೇಕೆ
ಹೆಜ್ಜೆ
ಹೆಜ್ಜೆ-ಯ-ನ್ನಿ-ಟ್ಟರೆ
ಹೆಣ-ಗುವ
ಹೆಣೆದು
ಹೆಣ್ಣು
ಹೆತ್ತ-ವ-ರಿ-ಗಿಂತ
ಹೆತ್ತಿದ್ದ
ಹೆದ-ರು-ತ್ತೀಯಾ
ಹೆಬ್ಬೆಟ್ಟ
ಹೆಬ್ಬೆ-ರಳು
ಹೆಮ್ಮ-ರ-ವ-ನ್ನಾಗಿ
ಹೆಮ್ಮೆ
ಹೆಮ್ಮೆಯ
ಹೆಮ್ಮೆ-ಯ-ನ್ನುಂ-ಟು-ಮಾ-ಡಿದೆ
ಹೆಮ್ಮೆ-ಯಾ-ಗ-ದಿ-ರದು
ಹೆಮ್ಮೆ-ಯಿಂದ
ಹೆಮ್ಮೆ-ಯಿತ್ತು
ಹೆಮ್ಮೆ-ಯಿದೆ
ಹೆಮ್ಮೆ-ಯೆ-ನಿ-ಸಿತು
ಹೆಮ್ಮೆ-ಯೆ-ನಿ-ಸು-ತ್ತದೆ
ಹೆಲಿ-ಕಾ-ಪ್ಟರ್
ಹೆಸ-ರನ್ನು
ಹೆಸ-ರಿ-ನಿಂ-ದಲೇ
ಹೆಸರು
ಹೆಸ-ರು-ಗಳ
ಹೆಸ-ರೇನು
ಹೆಸ-ರೇ-ನೆಂದು
ಹೇಗಿ-ರ-ಬೇಕು
ಹೇಗಿ-ರು-ತ್ತದೆ
ಹೇಗೆ
ಹೇಗೆ
ಹೇಗೇ
ಹೇಗೋ
ಹೇತು
ಹೇಮ-ಕ್ಕ-ನನ್ನು
ಹೇಮ-ಚಂದ್ರ
ಹೇಮ-ಲಂಬಿ
ಹೇಮಾ-ವತಿ
ಹೇಮಾ-ವತೀ
ಹೇಯ-ವಾ-ದೀತು
ಹೇರಂಬ
ಹೇರಂಭ
ಹೇರಂ-ಭ-ಶ್ರೀ-ಧರ
ಹೇಳದೇ
ಹೇಳ-ಬಲ್ಲೆ
ಹೇಳ-ಬ-ಹುದು
ಹೇಳ-ಬ-ಹುದೇ
ಹೇಳ-ಬೇ-ಕಾ-ಗಿಲ್ಲ
ಹೇಳ-ಬೇ-ಕಾ-ದ-ದ್ದನ್ನು
ಹೇಳ-ಬೇ-ಕಾ-ದ-ದ್ದಿಲ್ಲ
ಹೇಳ-ಬೇ-ಕಾ-ದರೆ
ಹೇಳ-ಬೇ-ಕಿಲ್ಲ
ಹೇಳ-ಬೇಕು
ಹೇಳ-ಬೇ-ಕೆಂ-ದರೆ
ಹೇಳ-ಬೇ-ಕೆಂ-ಬು-ದನ್ನು
ಹೇಳಮ್ಮಾ
ಹೇಳ-ಲಾ-ಗದು
ಹೇಳ-ಲಾ-ಗಿದೆ
ಹೇಳಲು
ಹೇಳಲೇ
ಹೇಳ-ಲೇ-ಬೇಕು
ಹೇಳ-ಹೊ-ರ-ಟಿ-ರು-ವುದು
ಹೇಳಿ
ಹೇಳಿಕೆ
ಹೇಳಿ-ಕೊಂ-ಡಾ-ಗ-ಲೆಲ್ಲಾ
ಹೇಳಿ-ಕೊಂಡು
ಹೇಳಿ-ಕೊಂಡೆ
ಹೇಳಿ-ಕೊ-ಟ್ಟಂತೆ
ಹೇಳಿ-ಕೊ-ಟ್ಟ-ವರು
ಹೇಳಿ-ಕೊ-ಟ್ಟು-ದ-ಲ್ಲದೇ
ಹೇಳಿ-ಕೊಟ್ಟೆ
ಹೇಳಿ-ಕೊ-ಡ-ಬೇಕು
ಹೇಳಿ-ಕೊ-ಡಲು
ಹೇಳಿ-ಕೊ-ಡು-ತ್ತಿ-ದ್ದರು
ಹೇಳಿ-ಕೊ-ಡು-ತ್ತೇವೆ
ಹೇಳಿ-ಕೊ-ಳ್ಳಲು
ಹೇಳಿ-ಕೊ-ಳ್ಳು-ವುದೇ
ಹೇಳಿದ
ಹೇಳಿ-ದಂತೆ
ಹೇಳಿ-ದನು
ಹೇಳಿ-ದರು
ಹೇಳಿ-ದ-ರು
ಹೇಳಿ-ದರೂ
ಹೇಳಿ-ದರೆ
ಹೇಳಿ-ದ-ವ-ರಲ್ಲ
ಹೇಳಿ-ದಷ್ಟು
ಹೇಳಿದೆ
ಹೇಳಿದ್ದ
ಹೇಳಿ-ದ್ದ-ರಂತೆ
ಹೇಳಿ-ದ್ದರು
ಹೇಳಿ-ದ್ದಾನೆ
ಹೇಳಿ-ದ್ದಾರೆ
ಹೇಳಿ-ದ್ದಿದೆ
ಹೇಳಿ-ದ್ದು
ಹೇಳಿದ್ದೂ
ಹೇಳಿದ್ದೆ
ಹೇಳಿ-ದ್ದೆಲ್ಲ
ಹೇಳಿದ್ದೇ
ಹೇಳಿ-ಮಾ-ಡಿ-ಸಿ-ದಂ-ತಿ-ದ್ದರು
ಹೇಳಿ-ರ-ಬೇಕು
ಹೇಳಿ-ರು-ವಂತೆ
ಹೇಳಿ-ರು-ವರು
ಹೇಳಿ-ರು-ವುದು
ಹೇಳಿ-ಲ್ಲವೇ
ಹೇಳಿ-ಸಿ-ಕೊಂ-ಡಿ-ದ್ದೇನೆ
ಹೇಳಿ-ಸಿ-ಕೊಂಡು
ಹೇಳಿ-ಸಿ-ಕೊಂಡೆ
ಹೇಳಿ-ಸಿ-ಕೊ-ಳ್ಳ-ತ್ತಾರೆ
ಹೇಳಿ-ಸಿ-ಕೊ-ಳ್ಳಲು
ಹೇಳಿ-ಸಿ-ಕೊ-ಳ್ಳು-ವು-ದ-ಕ್ಕಾಗಿ
ಹೇಳು
ಹೇಳು-ತ್ತದೆ
ಹೇಳುತ್ತಾ
ಹೇಳು-ತ್ತಾನೆ
ಹೇಳು-ತ್ತಾ-ನೆ
ಹೇಳು-ತ್ತಾರೆ
ಹೇಳು-ತ್ತಿದೆ
ಹೇಳು-ತ್ತಿ-ದ್ದರು
ಹೇಳು-ತ್ತಿ-ದ್ದ-ರು
ಹೇಳು-ತ್ತಿ-ದ್ದಾರೆ
ಹೇಳು-ತ್ತಿ-ದ್ದು-ದನ್ನು
ಹೇಳು-ತ್ತಿ-ರು-ತ್ತಾರೆ
ಹೇಳು-ತ್ತೇವೆ
ಹೇಳುವ
ಹೇಳು-ವಂತೆ
ಹೇಳು-ವ-ಮಾತು
ಹೇಳು-ವಲ್ಲಿ
ಹೇಳು-ವಾಗ
ಹೇಳು-ವಾ-ಗಲೂ
ಹೇಳು-ವು-ದಕ್ಕೆ
ಹೇಳು-ವು-ದನ್ನು
ಹೇಳು-ವು-ದ-ನ್ನೆಲ್ಲಾ
ಹೇಳು-ವು-ದಾ-ದರೆ
ಹೇಳು-ವುದು
ಹೇಳು-ವುದೇ
ಹೈಸ್ಕೂ-ಲಿಗೆ
ಹೈಸ್ಕೂ-ಲಿನ
ಹೈಸ್ಕೂ-ಲಿ-ನಲ್ಲಿ
ಹೈಸ್ಕೂಲು
ಹೊಂದಲಿ
ಹೊಂದಲು
ಹೊಂದಾ-ಣಿಕೆ
ಹೊಂದಿ
ಹೊಂದಿದ
ಹೊಂದಿ-ದ-ವನು
ಹೊಂದಿ-ದ-ವ-ರಲ್ಲ
ಹೊಂದಿ-ದ-ವ-ರಾಗಿ
ಹೊಂದಿ-ದ-ವ-ರಾ-ಗಿ-ದ್ದಾರೆ
ಹೊಂದಿ-ದ-ವರು
ಹೊಂದಿ-ದಾಗ
ಹೊಂದಿ-ದಾ-ಗಲೂ
ಹೊಂದಿದೆ
ಹೊಂದಿ-ದ್ದರೂ
ಹೊಂದಿ-ದ್ದಾನೆ
ಹೊಂದಿ-ದ್ದಾರೆ
ಹೊಂದಿದ್ದು
ಹೊಂದಿದ್ದೆ
ಹೊಂದಿ-ದ್ದೇನೆ
ಹೊಂದಿ-ರ-ಬೇಕು
ಹೊಂದಿ-ರುವ
ಹೊಂದಿ-ರು-ವ-ವರು
ಹೊಂದಿ-ರು-ವುದು
ಹೊಂದಿಲ್ಲ
ಹೊಂದಿ-ಲ್ಲ-ದಿ-ದ್ದರೆ
ಹೊಂದಿ-ಲ್ಲ-ದಿ-ರು-ವುದು
ಹೊಂದಿ-ಲ್ಲ-ವೆಂದು
ಹೊಂದಿವೆ
ಹೊಂದಿ-ಸಿ-ಕೊಂಡ
ಹೊಂದು-ತ್ತಾ-ನೆಂದೂ
ಹೊಂದು-ತ್ತಿ-ದ್ದಾನೆ
ಹೊಂದು-ತ್ತಿ-ದ್ದಾರೆ
ಹೊಂದು-ತ್ತಿ-ದ್ದಾ-ರೆಂದು
ಹೊಂದು-ತ್ತಿ-ರು-ವಾ-ಗಲೆ
ಹೊಂದು-ತ್ತೇವೆ
ಹೊಂದು-ವಂ-ತ-ಹುದು
ಹೊಂದು-ವಂತೆ
ಹೊಂದು-ವಂ-ಥದ್ದು
ಹೊಂದು-ವದು
ಹೊಂದು-ವುದು
ಹೊಂದು-ವುದೂ
ಹೊಗಿ
ಹೊಗು-ತ್ತಿದ್ದೆ
ಹೊಟ್ಟಿಗೆ
ಹೊಟ್ಟೆ
ಹೊಟ್ಟೆ-ಯನ್ನ
ಹೊಟ್ಟೆ-ರಾ-ಯನ
ಹೊಡೆ-ತ-ವನ್ನು
ಹೊಡೆ-ದ-ವ-ರಿಲ್ಲ
ಹೊಡೆದು
ಹೊಡೆ-ಯ-ವರು
ಹೊಡೆ-ಯು-ವುದು
ಹೊಣೆ-ಗಾ-ರಿ-ಕೆ-ಯನ್ನು
ಹೊಣೆ-ಯನ್ನು
ಹೊತ್ತ-ಗೆ-ಯನ್ನು
ಹೊತ್ತಿ-ಗಾ-ಗಲೇ
ಹೊತ್ತಿಗೆ
ಹೊತ್ತಿದ್ದ
ಹೊತ್ತಿನ
ಹೊತ್ತು
ಹೊತ್ತು-ಕೊಂಡು
ಹೊತ್ತೂ
ಹೊದಿ-ಯಲು
ಹೊನ್ನಾ-ವರ
ಹೊನ್ನಾ-ವ-ರದ
ಹೊನ್ನೆ-ಹದ್ದ
ಹೊನ್ನೇ-ಹದ್ದ
ಹೊಮ್ಮು-ವಿ-ಕೆ-ಯಿಂದ
ಹೊಯ್ಸ-ಳರ
ಹೊಯ್ಸ-ಳ-ರಾಜ
ಹೊರ
ಹೊರಕ್ಕೆ
ಹೊರ-ಗಡೆ
ಹೊರ-ಗಿದ್ದ
ಹೊರಗೆ
ಹೊರ-ಗೆ-ಡ-ಹುವ
ಹೊರ-ಜ-ಗ-ತ್ತನ್ನು
ಹೊರ-ಟಾಗ
ಹೊರಟೇ
ಹೊರ-ಟೇ-ಬಿಟ್ಟೆ
ಹೊರ-ಡ-ಬ-ಯ-ಸುವ
ಹೊರ-ಡಲು
ಹೊರ-ಡುವ
ಹೊರ-ತಂದು
ಹೊರ-ತ-ರು-ತ್ತಿ-ರುವ
ಹೊರ-ತಾಗಿ
ಹೊರ-ತಾ-ಗಿಯೂ
ಹೊರತು
ಹೊರ-ತು-ಪ-ಡಿಸಿ
ಹೊರ-ತು-ಪ-ಡಿ-ಸಿ-ದಂತೆ
ಹೊರತೂ
ಹೊರ-ದೇಶ
ಹೊರ-ನ-ಡೆ-ಯಿತು
ಹೊರ-ಬ-ರು-ತ್ತಿದೆ
ಹೊರ-ರಾ-ಜ್ಯ-ದ-ವ-ರಿಗೆ
ಹೊರ-ಹೊ-ಮ್ಮು-ತ್ತಿ-ದ್ದವು
ಹೊರ-ಹೊ-ಮ್ಮು-ವಲ್ಲಿ
ಹೊರ-ಹೋ-ಗು-ವಾಗ
ಹೊರೆ-ಯನ್ನು
ಹೊಳ-ಪಿನ
ಹೊಳೆಗೆ
ಹೊಳೆ-ದಳು
ಹೊಳೆಯ
ಹೊಳೆ-ಯ-ಲಿಲ್ಲ
ಹೊಸ
ಹೊಸ-ತನ್ನು
ಹೊಸತು
ಹೊಸ-ದ-ರಲ್ಲಿ
ಹೊಸ-ದಾಗಿ
ಹೊಸೆದು
ಹೊಸ್ತೋಟ
ಹೋಗದೆ
ಹೋಗನು
ಹೋಗ-ಬ-ಹು-ದಾ-ದಷ್ಟು
ಹೋಗ-ಬೇ-ಕಿತ್ತು
ಹೋಗ-ಬೇ-ಕೆಂದು
ಹೋಗ-ಲಿಲ್ಲ
ಹೋಗಲು
ಹೋಗಿ
ಹೋಗಿ-ದ್ದಾಗ
ಹೋಗಿ-ದ್ದಿದೆ
ಹೋಗಿದ್ದು
ಹೋಗಿದ್ದೆ
ಹೋಗಿ-ದ್ದೆವು
ಹೋಗಿ-ದ್ದೇನೆ
ಹೋಗಿಲ್ಲ
ಹೋಗಿ-ಬಂದು
ಹೋಗು
ಹೋಗು-ಗ-ಳಿಗೂ
ಹೋಗು-ತ್ತದೆ
ಹೋಗು-ತ್ತಾನೆ
ಹೋಗು-ತ್ತಿದೆ
ಹೋಗು-ತ್ತಿದ್ದೆ
ಹೋಗು-ತ್ತಿ-ರ-ಲಿಲ್ಲ
ಹೋಗು-ತ್ತಿ-ರು-ವುದು
ಹೋಗು-ತ್ತೇನೆ
ಹೋಗುವ
ಹೋಗು-ವಂತೆ
ಹೋಗು-ವಂ-ಥ-ವರು
ಹೋಗು-ವ-ವ-ರನ್ನು
ಹೋಗು-ವಾಗ
ಹೋಗು-ವಾ-ಗ-ಲೆಲ್ಲ
ಹೋಗು-ವುದು
ಹೋಗೆ
ಹೋಗ್ತೀಯಾ
ಹೋಟೆ-ಲಿ-ನಿಂದ
ಹೋಟೆಲ್
ಹೋಟೇ-ಲಲ್ಲಿ
ಹೋಟೇ-ಲು-ಗ-ಳಲ್ಲಿ
ಹೋಟೇಲ್
ಹೋದ
ಹೋದಂತೆ
ಹೋದರು
ಹೋದರೂ
ಹೋದರೆ
ಹೋದ-ವನು
ಹೋದ-ವರು
ಹೋದಾಗ
ಹೋದಾ-ಗ-ಲೆಲ್ಲ
ಹೋದುದು
ಹೋದೆ
ಹೋಮ-ದಂತೆ
ಹೋಯಿತು
ಹೋಯಿ-ತೆ-ನ್ನ-ಬೇಕು
ಹೋರಾಟ
ಹೋರಾ-ಟ-ದಲ್ಲಿ
ಹೋರಾ-ಡಿದ
ಹೋರಾ-ಡುವ
ಹೋರಾ-ಡು-ವುದು
ಹೋಲಿಸಿ
ಹೋಲಿ-ಸಿ-ದರೆ
ಹೋಲುವ
ಹೌದಾ-ದರೆ
ಹೌದು
ಹೌರಾ
ಹ್ರದೋ
}
