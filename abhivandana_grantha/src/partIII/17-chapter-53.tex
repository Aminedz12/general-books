{\fontsize{14}{16}\selectfont
\chapter{ಅಪರೂಪದ ಪ್ರಭೇದ ವಿ~॥ ಗಂಗಾಧರ ಭಟ್ಟರು}

\begin{center}
\Authorline{ಡಾ~॥ ಎಸ್. ಎನ್. ಹೆಗಡೆ,}
\smallskip

ನಿವೃತ್ತ ಪ್ರಾಧ್ಯಾಪಕರು ಹಾಗೂ ಮುಖ್ಯಸ್ಥರು\\
ಪ್ರಾಣಿಶಾಸ್ತ್ರ ಅಧ್ಯಯನ ವಿಭಾಗ\\ 
ಮಾನಸಗಂಗೋತ್ರಿ\\ 
ಮೈಸೂರು – 570006
\addrule
\end{center}

ಕಾಲಗಣನೆಗೆ ಸಂಬಂಧಿಸಿದ ಪ್ರಾಚೀನ ಮತ್ತು ಅರ್ವಾಚೀನ ಈ ಎರಡು ಪದಗಳು ವಿರುದ್ಧಾರ್ಥ ನೀಡುತ್ತವೆ.  ಸಾಮಾನ್ಯವಾಗಿ ಪ್ರಾಚೀನ ಸಿದ್ಧಾಂತಗಳನ್ನು ನಂಬುವವರು ನವೀನ ಸಿದ್ಧಾಂತಗಳನ್ನು ನಂಬಲಾರರು.  ನವೀನ ಸಿದ್ಧಾಂತಗಳನ್ನು ನಂಬುವವರು ಪ್ರಾಚೀನ ಸಿದ್ಧಾಂತಗಳನ್ನು ನಂಬಲಾರರು.  ಪ್ರತಿಯೊಂದು ಕ್ಷೇತ್ರಕ್ಕೂ ಈ ಹೇಳಿಕೆ ಅನ್ವಯಿಸುತ್ತದೆ ಎಂದು ಹೇಳಬಹುದು.  ಎರಡೂ ಸಿದ್ಧಾಂತಗಳನ್ನು ನಂಬುವವರು ಅಥವಾ ಅದನ್ನು ಪ್ರತಿಪಾದಿಸುವವರು ಕಡಿಮೆ.  ವಿದ್ಯಾನ್ ಗಂಗಾಧರ ಭಟ್ಟರು ಹಳೆಯ ಸಿದ್ಧಾಂತದ ತಳಹದಿಯ ಮೇಲೆ ಹೊಸ ಸಿದ್ಧಾಂತಗಳನ್ನು ರೂಪಿಸಬಲ್ಲವರು ಎಂದು ನನ್ನ ಅನಿಸಿಕೆ.  

ನನ್ನ ಈ ಅನಿಸಿಕೆಗೆ ಕಾರಣವಿದೆ.  ಒಮ್ಮೆ ನಾನು ಅವರಲ್ಲಿಗೆ ಹೋಗಿದ್ದಾಗ ಅವರು ನನಗೆ ವಿಕಾಸವಾದಕ್ಕೆ ಸಂಬಂಧಿಸಿದ ಕೆಲವು ಪ್ರಶ್ನೆಗಳನ್ನು ಕೇಳಿದರು.  ಬಹುಶಃ ಎಲ್ಲಿಯೋ ಯಾರೋ ಅವರೊಂದಿಗೆ ಈ ವಿಷಯವನ್ನು ಚರ್ಚಿಸಿರಬೇಕು, ಅಲ್ಲಿ ಹುಟ್ಟಿದ ಸಂದೇಹಗಳನ್ನು ಪರಿಹರಿಸಿಕೊಳ್ಳಲು ನನಗೆ ಈ ಪ್ರಶ್ನೆಗಳನ್ನು ಕೇಳುತ್ತಿದ್ದಾರೆ ಎಂದು ಸುಲಭವಾಗಿ ತಿಳಿದುಕೊಂಡೆ.  ನೂರ ಅರವತ್ತು ವರ್ಷಗಳ ಹಿಂದೆ ಪ್ರತಿಪಾದಿತವಾದ ಡಾರ್ವಿನ್ ಸಿದ್ಧಾಂತಕ್ಕೆ ಅನೇಕ ವಿಜ್ಞಾನಿಗಳು ತಮ್ಮ ವಿಶಿಷ್ಟ ಸಂಶೋಧನೆಗಳ ಮೂಲಕ ಪುರಾವೆ ಒದಗಿಸಿಕೊಟ್ಟಿದ್ದಾರೆ.  ಒಂದು ಕಾಲದಲ್ಲಿ ಡಾರ್ವಿನ್ ವಾದವನ್ನು ಸಂಪೂರ್ಣ ತಿರಸ್ಕರಿಸಿದ್ದ ಕ್ರಿಶ್ಚಿಯನ್ ಧರ್ಮ ಪ್ರಚಾರಕರು ಕೂಡ ಈಗ ಡಾರ್ವಿನ್ ವಾದವನ್ನು ಒಪ್ಪಲಾರಂಭಿಸಿದ್ದಾರೆ.  ಅಂಥದರಲ್ಲಿ ಇತ್ತೀಚೆಗೆ ನಮ್ಮ ಮೂರ್ಖ ರಾಜಕಾರಣಿಯೊಬ್ಬರು ಡಾರ್ವಿನ್‍ವಾದ ತಪ್ಪು, ಅದಕ್ಕೆ ಯಾವುದೇ ಪುರಾವೆ ಇಲ್ಲ ಎಂದು ಹೇಳಿಕೆ ನೀಡಿದ್ದಾರೆ.  ಅವರ ಆಷಾಢಭೂತಿತನಕ್ಕೆ ನಗುಬರುತ್ತಿದೆ.  ಅದೇನೇ ಇರಲಿ.  ವೈಜ್ಞಾನಿಕ ಅಧ್ಯಯನದ ಹಿನ್ನೆಲೆ ಇಲ್ಲದ ಗಂಗಾಧರ ಭಟ್ಟರಲ್ಲಿ ಜೀವ ವಿಕಾಸದ ಕುರಿತು ಹುಟ್ಟಿದ್ದ ಸಂದೇಹಗಳನ್ನು ನನಗೆ ಪರಿಹರಿಸಲು ಸಾಧ್ಯವಾದದ್ದು ಅವರ ಗ್ರಾಹಕ ಶಕ್ತಿಯಿಂದ ಅನಿಸುತ್ತಿದೆ.  

ಗಂಗಾಧರ ಭಟ್ಟರು ಅಪ್ಪಟ ಬ್ರಾಹ್ಮಣ, ಜನಿಸಿದ್ದು ಹವ್ಯಕ ವೈದಿಕ ಕುಟುಂಬ.  ಇಡೀ ಕುಟುಂಬ ಹಿಂದಿನಿಂದಲೂ ಧಾರ್ಮಿಕ ಕಾರ್ಯಗಳನ್ನು ಏರ್ಪಡಿಸುವುದರಲ್ಲಿ, ಅಥವಾ ಅವರ ಶಿಷ್ಯವರ್ಗದವರು ಅಂಥ ಕಾರ್ಯಕ್ರಮಗಳನ್ನು ಏರ್ಪಡಿಸಿದಾಗ ಸಮರ್ಪಕವಾಗಿ ಧಾರ್ಮಿಕತೆಗೆ ಚಾಚೂ ಭಂಗಬರದಂತೆ ನಡೆಸಿಕೊಡುತ್ತ ಬಂದವರು.  ಅಂಥ ಕುಟುಂಬದಲ್ಲಿ ಜನಿಸಿದ ಗಂಗಾಧರ ಭಟ್ಟರು ಕೂಡ ಧಾರ್ಮಿಕ ಮನೋಭಾವದವರೇ ಎಂದು ಯಾರಾದರೂ ಹೇಳಿದರೆ ಅದನ್ನು ಖಂಡಿತಾ ನಂಬಲೇ ಬೇಕು.  

ಡಾರ್ವಿನ್ ವಾದದ ಕುರಿತು ಚರ್ಚೆ ನಡೆಸುತ್ತ ಅವರು ಕೇಳಿದ ಪ್ರಶ್ನೆ, 'ಮಂಗನಿಂದ ಮಾನವ' ಎಂದು ಡಾರ್ವಿನ್ ಸಿದ್ಧಾಂತ ಹೇಳುತ್ತದೆ, ಅದಕ್ಕೆ ಪುರಾವೆ ಎಲ್ಲಿದೆ?  ಎಂದು ಕೇಳಿದ್ದರು.  ಅದಕ್ಕೆ ಸೂಕ್ತ ಉತ್ತರವನ್ನು ನಾನು ನೀಡಿದ್ದೆ.  ಡಾರ್ವಿನ್ ಮರಗಳ ಮೇಲೆ ಹಾರಾಡುತ್ತಿದ್ದ ಮಂಗಗಳಲ್ಲಿ ಕೆಲವು ಧುತ್ತನೆ ಭೂಮಿಗೆ ಬಂದು ಬಾಲ ಕಳಚಿಬಿದ್ದು ಮೈಮೇಲಿನ ಕೂದಲುಗಳೆಲ್ಲ ಉದುರಿ ಮನುಷ್ಯನಾಗಿ ಬದಲಾದವು ಎಂದಲ್ಲ,  ವಿಕಾಸ ಸುದೀರ್ಘಕಾಲದ ಬದಲಾವಣೆಗಳ ಒಂದು ಪ್ರಕ್ರಿಯೆ.  ಒಮ್ಮೆಲೆ ಎಂದಿಗೂ ವಿಕಾಸವಾಗದು.  ಡಾರ್ವಿನ್‍ನ ಪ್ರಕಾರ ಸುಮಾರು ಹದಿನೈದು ಮಿಲಿಯ ವರ್ಷಗಳ ಹಿಂದೆ ವಾಸವಾಗಿದ್ದ ವಾನರ ಪ್ರಭೇದದ ಕವಲೊಂದು ಪ್ರತಿಯೊಂದು ಪೀಳಿಗೆಯಲ್ಲೂ ನಿಧಾನವಾಗಿ ಅಗೋಚರ  ಮಾರ್ಪಡುಗಳನ್ನು ಪಡೆಯುತ್ತ, ಹೀಗೆ ಆದ ಮಾರ್ಪಾಡುಗಳು ಕೂಡ ನಿಧಾನವಾಗಿ ಗೋಚರಿಸಲಾರಂಭಿಸಿದಾಗ ಅಂಥ ಜೀವಿಯನ್ನು ಮಾನವ ಎಂದು ಕರೆಯಲಾಯಿತು.  ಮಾನವ ಪ್ರಭೇದದ ಉದಯವಾದದು ಇಂಥ ನಿಧಾನ ಕ್ರಿಯೆಯಿಂದ ಎಂದು ವಿವರಿಸಿದ್ದೆ.  

ಗಂಗಾಧರ ಭಟ್ಟರು ನನ್ನನ್ನು ಅಲ್ಲಿಗೇ ಬಿಡಲಿಲ್ಲ.  ಈ ಡಾರ್ವಿನ್ ಸಿದ್ಧಾಂತ ಎಲ್ಲ ಜೀವಿಗಳಿಗೂ ಅನ್ವಯಿಸುತ್ತದೆಯೇ? ಎಂದು ಕೇಳಿದರು.  ಅದಕ್ಕೂ ನಾನು ಸೂಕ್ತವಾದ ಉತ್ತರವನ್ನು ನೀಡಿದೆ.  ಡಾರ್ವಿನ್‍ನ ವಿಕಾಸ ವಾದವನ್ನು ಪ್ರತಿಪಾದಿಸಿದ್ದು, ಸಮಸ್ತ ಜೀವಿಗಳ ಉದ್ಭವ ಹಾಗೂ ಅವುಗಳ ವೈವಿಧ್ಯಕ್ಕೆ ಕಾರಣ ಹುಡುಕುವುದಕ್ಕಾಗಿಯೇ ಆಗಿತ್ತು.  ಡಾರ್ವಿನ್, ಕ್ರೈಸ್ತ ಧರ್ಮ ಅನೇಕ ಶತಮಾನಗಳ ಕಾಲ ನಂಬಿದ್ದ 'ಸಕಲವೂ ದೈವನಿರ್ಮಿತ' ಎನ್ನುವ ಸಿದ್ಧಾಂತವನ್ನು ತಿರಸ್ಕರಿಸಿ 'ಸರಳತೆಯಿಂದ ಕ್ಲಿಷ್ಟತೆ', ಅಂದರೆ ಮೊದಲು ಸರಳ ಜೀವಿಗಳು ವಿಕಾಸವಾಗಿ ಅವುಗಳಿಂದ ಪ್ರಾಕೃತಿಕ ಆಯ್ಕೆ ಪ್ರಕ್ರಿಯೆಯ ಮೂಲಕ ನಿಧಾನವಾಗಿ ಸಂಕೀರ್ಣ ಜೀವಿಗಳ ಉದ್ಭವವಾದವು ಎಂದು ತಿಳಿಸಿದ್ದಾನೆ, ಮಂಗನಿಂದ ಮಾನವ ಎನ್ನುವುದು ಈ ಪರಿಕಲ್ಪನೆಯಿಂದಲೇ ಬಂದದ್ದು, ಮಾನವನಿಗೆ ಹೋಲಿಸಿದರೆ ವಾನರ ಸರಳ ಜೀವಿ, ಇಂಥ ಸರಳ ಜೀವಿಗಳ ಕವಲೊಂದು ಮುಂದೆ ಹದಿನೈದು ದಶಲಕ್ಷ ವರ್ಷಗಳ ಪರ್ಯಂತ ಮಾರ್ಪಾಡಾಗುತ್ತ ನವಮಾನವನ ಉದಯವಾಯಿತು ಎಂದು ವಿವರಿಸಿದೆ.  ಅದಕ್ಕೆ ಪುರಾವೆ, ಇಂದು ಭೂಗರ್ಭದಲ್ಲಿ ದೊರೆಯುತ್ತಿರುವ ಪಳೆಯುಳಿಕೆಗಳು.  ಗೊರಿಲ, ಚಿಂಪಾಂಜಿಗಳಿಂದ ಹಿಡಿದು, ನವಮಾನವನನ್ನು ಹೋಲುವ ಪಳೆಯುಳಿಕೆಗಳು ದೊರೆತಿದ್ದು ಅವು ವಾನರ–ಮಾನವರ ನಡುವಿನ ಸ್ಥಿತಿಯನ್ನು ತಿಳಿಸುತ್ತವೆ ಎಂದು ಹೇಳಿದೆ.  

ಈ ವಿವರಣೆ ನೀಡುತ್ತಿದ್ದಾಗ, ಮೊದಲು ಅಕಶೇರುಕಗಳು ವಿಕಾಸವಾದವು, ನಂತರ ಕಶೇರುಕಗಳು ಹುಟ್ಟಿದವು, ಕಶೇರುಕಗಳಲ್ಲಿ ಮೊದಲು ಉದ್ಭವವಾದ ಗುಂಪು ಮತ್ಸ್ಯಗಳು, ನಂತರ ಉಭಯಜೀವಿಗಳು, ಸರೀಸೃಪಗಳು, ಪಕ್ಷಿಗಳು ಅನಂತರ ಸಸ್ತನಿಗಳು ಉದ್ಭವವಾದವು ಎಂದು ವಿಕಾಸದ ಹಂತಗಳನ್ನೂ ವಿವರಿಸಿದ್ದೆ.  ಚುರುಕುಮತಿಯಾದ ಗಂಗಾಧರ ಭಟ್ಟರು ವಿಕಾಸವಾದವನ್ನು ನಮ್ಮ ಪೌರಾಣಿಕ ಹಿನ್ನೆಲೆಯಿಂದ ಪರಿಶಿಲಿಸಿ ನನಗೊಂದು ಪ್ರಶ್ನೆಯನ್ನು ಎಸೆದರು.  ಹಾಗಾದರೆ ನೀವು ದಶಾವತಾರವನ್ನು ಒಪ್ಪುತ್ತೀರಾ? ಎಂದು.  ಗಂಗಾಧರ ಭಟ್ಟರ ತಾರ್ಕಿಕ ಮನೋಭಾವವನ್ನು ಈ ವಿಶ್ಲೇಷಣೆ ತೋರ್ಪಡಿಸುತ್ತದೆ. 

ನಿಜ, ಗಂಗಾಧರ ಭಟ್ಟರ ಊಹೆ ಖಂಡಿತಾ ನಿಜ.  ವಿಕಾಸವಾದದ ಸರಳತೆಯಿಂದ ಜಟಿಲತೆ ನಮ್ಮ ದಶಾವತಾರ ಕಲ್ಪನೆಯನ್ನೇ ಆಧರಿಸಿದೆ.  ಅಂದರೆ ನಮ್ಮ ಪೂರ್ವಿಕರು ಚಾಲ್ಸ್ ಡಾರ್ವಿನ್‍ನಿಗಿಂತ ಮೊದಲೇ ವಿಕಾಸವನ್ನು ಅರಿತಿದ್ದರೆ ಎಂಬ ಪ್ರಶ್ನೆ ಎದುರಾಗುತ್ತದೆ.  ಅದೂ ನಿಜ, ಪಾಶ್ಚಾತ್ಯರ ಅನೇಕ ಶೋಧಗಳು ಬೆಳಕಿಗೆ ಬರುವ ಮೊದಲೇ ನಮ್ಮ ಭಾರತೀಯ ಸಂಸ್ಕೃತಿ, ಇತಿಹಾಸಗಳಲ್ಲಿ ಅವು ಪ್ರತಿಫಲಿತವಾಗಿದೆ ಎಂಬುದಕ್ಕೆ ಅನೇಕ ಉದಾಹರಣೆಗಳಿವೆ.  ಪುಷ್ಪಕ ವಿಮಾನ, ಭೂಮಿಯ ಚಲನೆ, ಆಗಸದಲ್ಲಿ ಗ್ರಹ–ನಕ್ಷತ್ರಗಳ ಸ್ಥಿತಿ ಹಾಗೂ ಚಲನೆಯ ಕುರಿತು ನಮ್ಮ ಪೂರ್ವಿಕರು ಎಷ್ಟೋ ಸಹಸ್ರ ವರ್ಷಗಳ ಹಿಂದೆಯೇ ಕಲ್ಪಿಸಿದರು ಎನ್ನುವುದೂ ಸರಿ.  ಆದರೆ ದುರಾದೃಷ್ಟ, ನಿಖರವಾದ ದಾಖಲೆಗಳಿಲ್ಲದ ಕಾರಣ ನಮ್ಮವರ ಶೋಧ ಬೆಳಕಿಗೆ ಬಾರದೆ ಅವುಗಳ ಆವಿಷ್ಕಾರದ ಕೀರ್ತಿ ಪಾಶ್ಚಾತ್ಯರ ಪಾಲಾಯಿತಲ್ಲವೆ?  ಡಾರ್ವಿನ್‍ನ ವಿಕಾಸವಾದವೂ ಇದಕ್ಕೆ ಒಂದು ಉದಾಹರಣೆ.   

ಇಷ್ಟು ಸುದೀರ್ಘವಾಗಿ ಈ ಘಟನೆಯನ್ನು ನಾನು ವಿವರಿಸುವುದರ ಉದ್ದೇಶ, ನಾನು ಅರಿತಂತೆ ವಿದ್ವಾನ್ ಗಂಗಾಧರ ಭಟ್ಟರ ಸಾಮಥ್ರ್ಯ ಹಾಗೂ ವ್ಯಕ್ತಿತ್ವವನ್ನು ತಿಳಿಸಿಕೊಡುವುದು.  ಮಾನ್ಯ ಗಂಗಾಧರ ಭಟ್ಟರದು ವಿಶಿಷ್ಟ ವ್ಯಕ್ತಿತ್ವ.  ನನ್ನ ಜೀವವೈಜ್ಞಾನಿಕ ಪರಿಭಾಷೆಯಲ್ಲಿ ಇದೊಂದು ಉತ್ಕೃಷ್ಟ ಪ್ರಭೇದ.  'ಅಪರೂಪದ ಪ್ರಭೇದ', ಇಂಗ್ಲೀಷಿನಲ್ಲಿ ನಾವು ಅದನ್ನು ರೇರ್ ಸ್ಪೀಶೀಸ್ (ಖಚಿಡಿe Sಠಿeಛಿes) ಎನ್ನುತ್ತೇವೆ.  ಇಂಥ ಪ್ರಭೇದಗಳು ಈಗೀಗ ವಿನಾಶದತ್ತ ಸಾಗುತ್ತಿವೆ.  ಇಂಥ ಪ್ರಭೇದಗಳನ್ನು ನಾವು ಆಘಾತಕ್ಕೊಳಗಾದ ಅಥವಾ ವಿನಾಶದತ್ತ ಸಾಗಿರುವ ಪ್ರಭೇದ (ಎನ್‍ಡೇಂಜರ್ಡ್ ಸ್ಪೀಶೀಸ್) ಎಂದೂ ಹೇಳುತ್ತೇವೆ.  ಇಂದು ನಾವು ಕಾಣುವ ವ್ಯಕ್ತಿಗಳಲ್ಲಿ ಬಹುತೇಕರು ಒಂದೋ ಅಪ್ಪ ನೆಟ್ಟ ಆಲದ ಮರಕ್ಕೆ ಜೋತು ಬೀಳುವವರು, ಇಲ್ಲವೆ ಹೊಸತು ಹೊಸತು ಎಂದು ಉಳಿದದ್ದೆಲ್ಲವನ್ನೂ ಹೊಸೆದು ಹಾಕುವವರು.  ಹಳೆಯ ಸಿದ್ಧಾಂತದ ತಳಹದಿಯ ಮೇಲೆ ಹೊಸ ಸಿದ್ಧಾಂತಗಳನ್ನು ರೂಪಿಸಬಲ್ಲವರು ಗಂಗಾಧರ ಭಟ್ಟರಂಥ ವಿಶ್ಲೇಷಣಾ ಸಾಮಥ್ರ್ಯವುಳ್ಳವರು ಮಾತ್ರ.              

ಅಗ್ಗೆರೆ ಉತ್ತರ ಕನ್ನಡ ಜಿಲ್ಲೆಯ ಸಿದಾಪುರ ತಾಲೂಕಿನ ಕವಲಕೊಪ್ಪದ ಸಮೀಪದ ಒಂದು ಪುಟ್ಟ ಹಳ್ಳಿ.  ಅಲ್ಲಿದ್ದದ್ದೊಂದು ದೊಡ್ಡ ಅವಿಭಕ್ತ ಹವ್ಯಕ ಕುಟುಂಬ.  ಮೂಲತಃ ಕೃಷಿಕ ಮನೆತನ, ಪೌರೋಹಿತ್ಯವೂ ಅದರ ಜೊತೆಗೂಡಿ ಸಾಗಿದ ವೃತ್ತಿ.  ಈ ಕುಟುಂಬದ ನಾಲ್ವರು ಹಿರಿಯರ ಪೈಕಿ ಮೂರನೆಯವರು ವೇ ದಿ. ವಿಘ್ನೇಶ್ವರ ಭಟ್ಟರು,  ವಿಘ್ನೇಶ್ವರಭಟ್ಟರು ಹಾಗೂ ರೇವತಿಯವರ ಮೂರನೆಯ ಪುತ್ರ ಗಂಗಾಧರ.  ಈ ಪೈಕಿ ಕೊನೆಯವರು ವೇ. ರಾಮಕೃಷ್ಣ ಭಟ್ಟರು.  ಅವರ ಪತ್ನಿ ನನ್ನ ಚಿಕ್ಕಮ್ಮ.  ನನ್ನ ಮಾತೃಸ್ವರೂಪಿಣಿ ಚಿಕ್ಕಮ್ಮ ಭುವನೇಶ್ವರಿ ಈ ಕುಟುಂಬವನ್ನು ಕಿರಿಸೊಸೆಯಾಗಿ ಸೇರಿದ್ದರು.  ನನ್ನ ಮನೆಯಿಂದ ಕೇವಲ ಒಂದೂವರೆ ಕಿ.ಮೀ ದೂರದಲ್ಲಿದ್ದ ನನ್ನ ಅಜ್ಜನಮನೆ ಕೊರ್ಲಕೈಗೆ ಅಂದರೆ ಚಿಕ್ಕಮ್ಮ ತವರು ಮನೆಗೆ ಬಂದಾಗ ಪ್ರತಿಬಾರಿ ಆಕೆಯನ್ನು ಹಿಂದಕ್ಕೆ ಕಳಿಸುವ ಕಾಯಕ ನನ್ನದು.  ಈ ಅವಿಭಕ್ತ ಕುಟುಂಬದ ಓರ್ವ ಸದಸ್ಯರಾದ ವಿದ್ವಾನ್ ಗಂಗಾಧರ ಭಟ್ಟರು, ಹಿರಿಯರಿಗೆ ಗಂಗಾಧರ, ಅನೇಕ ಕಿರಿಯರಿಗೆ ಗಂಗಣ್ಣ.  ನನ್ನ ಚಿಕ್ಕಮ್ಮ ಗಂಗಣ್ಣನಿಗೂ ಚಿಕ್ಕಮ್ಮ.  ಹೀಗೆ ಗಂಗಣ್ಣ ನನ್ನ ಸಂಬಂಧಿ.  ಆದರೆ ನಮ್ಮ ನಡುವೆ ಸಂಬಂಧದ ನಡೆವಳಿಕೆಗಿಂತ ಸ್ನೇಹದ ನಂಟೇ ಹೆಚ್ಚು. 

ಕೊರ್ಲಕೈ ನನ್ನ ಅಜ್ಜನ ಮನೆ, ಅಂದರೆ ನನ್ನ ಚಿಕ್ಕಮ್ಮನ ತವರು.  ಅಲ್ಲಿಂದ ಅಗ್ಗೆರೆ 38 ಕಿಮೀ.  ಅಲ್ಲಿಗೆ ಹೋಗಲು, ಸಾಗರದಿಂದ ಸಿರ್ಸಿಗೆ ಹೋಗುವ ಮಾರ್ಗದಲ್ಲಿ ದಿನಕ್ಕೆ ಮೂರು ಬಾರಿ ಮಾತ್ರ ಬರುತ್ತಿದ್ದ ಸರ್ಕಾರಿ ಬಸ್‍ನಲ್ಲಿ ಆಡೂಕಟ್ಟೆಯಲ್ಲಿ ಹತ್ತಿ ಕುಳಿತು ಸಿದ್ದಾಪುರ ದಾಟಿ ಕೊಲಸಿರ್ಸಿ ಕತ್ತರಿಯಲ್ಲಿ ಇಳಿದು ಒಂಬತ್ತು ಕಿಮೀ ನಡೆದೇ ಹೋಗಬೇಕಿತ್ತು.  ಮುಂದೆ ಸಿದ್ದಾಪುರ ಹಾರ್ಸಿಕಟ್ಟಾ ಬಸ್ಸು ಪ್ರಾರಂಭವಾದಾಗ, ನಡೆಯುವ ದೂರ ಸ್ವಲ್ಪ ಕಡಿಮೆಯಾಗಿತ್ತಾದರೂ ಕಚ್ಚಾ ರಸ್ತೆಯ ಪ್ರಯಾಣ ತ್ರಾಸದಾಯಕವಾಗಿಯೇ ಇತ್ತು.  ಒಮ್ಮೆ ಅಲ್ಲಿ ತಲುಪಿದರೆ ಹಿಂತಿರುಗುವುದೂ ಅಷ್ಟೆ, ಪ್ರಯಾಣದ ಈ ಅನಾನುಕೂಲದಿಂದಾಗಿ ಒಮ್ಮೆ ಅಗ್ಗೆರೆಗೆ ಹೋದರೆ ಎರಡು ಮೂರು ದಿನಗಳು ಅಲ್ಲಿಯೇ ಠಿಕಾಣಿ ಹೂಡಬೇಕಾಗುತ್ತಿತ್ತು.  ಬೇಸಿಗೆ ಅಥವಾ ನವರಾತ್ರಿ ರಜೆಯ ದಿನಗಳಾದರೆ ಅದು ಇನ್ನೂ ಒಂದೆರಡು ದಿನ ಹೆಚ್ಚಾಗುತ್ತಿತ್ತು.  

ಮನೆತುಂಬ ಜನ; ಐದಾರು ಮಕ್ಕಳು, ಮನೆ ಮುಂದಿನ ವಿಶಾಲ ಅಂಗಳ, ಸುತ್ತ ಬೆಟ್ಟಗುಡ್ಡಗಳಿಂದ ಕೂಡಿದ ಸ್ಥಳ ಆಟವಾಡಲು ಬಹಳ ಪ್ರಶಸ್ತವಾಗಿತ್ತು.  ಹೀಗಾಗಿ ಅಲ್ಲಿ ಹೋದರೆ ದಿನ ಕಳೆದುದೇ ತಿಳಿಯುತ್ತಿರಲಿಲ್ಲ.  ಗಂಗಾಧರ ಭಟ್ಟರ ಹಿರಿಯಣ್ಣ ವೇ. ದಿ. ಮಂಜುನಾಥ ಭಟ್ಟರು ನನ್ನ ಸಮವಯಸ್ಕರು.  ಗಂಗಣ್ಣ ನನಗಿಂತ ಸ್ವಲ್ಪ ಕಿರಿಯ.  ಇಂಥ ದಿನಗಳನ್ನು ನೆನೆಸಿಕೊಂಡರೆ ಈಗಲೂ ಮನಸ್ಸು ಸಂತೋಷಗೊಳ್ಳುತ್ತದೆ.  ಮಂಜುನಾಥ, ಶ್ರೀಧರ ಹಾಗೂ ಗಂಗಾಧರ ಇವರೊಂದಿಗೆ ಆಡಿದ್ದ ಆಟ–ಹುಡುಗಾಟಗಳನ್ನು ನೆನೆಸಿಕೊಂಡರೆ ಈಗಲೂ ಮನಸ್ಸು ಮುದಗೊಳ್ಳುತ್ತದೆ.  ಆದರೆ ಆ ದಿನಗಳಲ್ಲಿ ಮುಂದಿನ ಜೀವನ ಕುರಿತು ನಮ್ಮಲ್ಲಿ ಯಾರಿಗೂ ಒಂದು ನಿಶ್ಚಿತ ಗುರಿಯಾಗಲಿ ಉದ್ದೇಶವಾಗಲೀ ಅಂದು ಇದ್ದಂತಿರಲಿಲ್ಲ.  ಕುಟುಂಬದ ಹಿನ್ನೆಲೆಯ ಆಧಾರದ ಮೇಲೆ ನನ್ನೊಬ್ಬನನ್ನು ಹೊರತುಪಡಿಸಿ ಉಳಿದ ಮೂವರಿಗೂ ವೇದಾಧ್ಯಯನ ಕಟ್ಟಿಟ್ಟ ಬುತ್ತಿ ಎನ್ನಬಹುದಿತ್ತೇನೋ.  ಗಂಗಣ್ಣ ಏಳೆಂಟು ವರ್ಷದವನಾಗುವವರೆಗೆ ಮಾತ್ರ ನಾನು ಅವರಲ್ಲಿಗೆ ಹೋಗುತ್ತಿದ್ದೆ.  ಮುಂದೆ ನನ್ನ ಕಾಲೇಜು ಜೀವನ ಆರಂಭವಾದಾಗ ನನ್ನ ಅಗ್ಗೆರೆ ಭೇಟಿ ತುಂಬ ಅಪರೂಪವಾಗಿ ಹೋಯಿತೆನ್ನಬೇಕು. 

ಮತ್ತೆ ನಮ್ಮಿಬ್ಬರ ಭೇಟಿಯಾದ್ದು ಮೈಸೂರು ಮಹಾರಾಜ ಸಂಸ್ಕೃತ ಕಾಲೇಜು ಹಾಸ್ಟೆಲಿನ ಒಂಬತ್ತನೆಯ ನಂಬರ್ ಕೋಣೆಯಲ್ಲಿ.  ನಾನು ಮೈಸೂರಿಗೆ ಬಂದ ಹೊಸದರಲ್ಲಿ ಅದೇ ಕೋಣೆಯಲ್ಲಿದ್ದೆ, ಗಂಗಾಧರ ಭಟ್ಟರ ಹಿರಿಯಣ್ಣ ದಿ. ವೇ. ಮಂಜುನಾಥ ಭಟ್ಟರು ಆ ರೂಮಿನಲ್ಲಿದ್ದರು.  ಅಲ್ಲಿಗೆ ನಾನು ಅವರ ಅತಿಥಿಯಾಗಿ ಬಂದು ಸೇರಿಕೊಂಡು ಕೆಲವು ದಿನಗಳವರೆಗೆ ಅವರ ಜೊತೆಯಲ್ಲಿದ್ದೆ.  ಅನಂತರದ ದಿನಗಳಲ್ಲಿ, ಶ್ರೀಧರ ಭಟ್ಟ, ಗಂಗಾಧರ ಭಟ್ಟ, ಹೇರಂಭ ಭಟ್ಟ ಹಾಗೂ ಸಹೋದರಿಯರು ಹೀಗೆ ಒಬ್ಬರಾದ ನಂತರ ಇನ್ನೊಬ್ಬರು ಬಂದು ಮಂಜುನಾಥ ಭಟ್ಟರನ್ನು ಅಥವಾ ಅಗ್ಗೆರೆ ಭಟ್ಟರ ಪಡೆಯನ್ನು ಸೇರಿಕೊಂಡರು,  ಮಂಜುನಾಥ ಹಾಗೂ ಶ್ರೀಧರ ಭಟ್ಟರು ವಿದ್ಯಾಭ್ಯಾಸದ ನಂತರ ಊರಿಗೆ ಹಿಂತಿರುಗಿದರು.  ಸಹೋದರಿಯರು ವಿವಾಹ–ಕುಟುಂಬ ಮುಂತಾದ ಕಾರಣಗಳಿಂದ ಮೈಸೂರನ್ನು ಬಿಟ್ಟು ಹಿಂತಿರುಗಿದರು.  ಗಂಗಾಧರ ಭಟ್ಟರು, ಹೇರಂಭ ಭಟ್ಟರು ಹಾಗೂ ಸೌ. ರತ್ನಾವತಿ ಮೈಸೂರಿನಲ್ಲಿಯೇ ಉಳಿದು ಇಲ್ಲಿಯೆ ನೆಲೆಯನ್ನು ಕಂಡುಕೊಂಡರು.  ಅಗ್ಗೆರೆ ಕುಟುಂಬದ ಯಾವ ಸದಸ್ಯರು ಇಲ್ಲಿದ್ದರೂ ನಾನು ಅವರೆಲ್ಲರೊಂದಿಗೆ ಅತ್ಯಂತ ನಿಕಟವಾದ ಸ್ನೇಹ ಹಾಗೂ ಬಾಂಧವ್ಯವನ್ನು ಉಳಿಸಿಕೊಂಡಿದ್ದೆ.  

ಗಂಗಣ್ಣ ವಿಶಿಷ್ಟವ್ಯಕ್ತಿ.  ಜನಿಸಿದ್ದು ಹವ್ಯಕ ವೈದೀಕ ಕುಟುಂಬ.  ಧಾರ್ಮಿಕ ಮನೋಭಾವ ಅವರಿಗೆ ಜನ್ಮತಃ ಬಂದದ್ದು, ಆದರೆ ಅವರು ಮೂಲಭೂತವಾದಕ್ಕೋ ಅಥವಾ ಕುರುಡು ಧಾರ್ಮಿಕತೆಯನ್ನೋ ನಂಬಿದವರಲ್ಲ.  ಪ್ರಾಚೀನತೆಗೆ ನವ್ಯತೆಯನ್ನು ಮೇಳೈಸಿ ಅವುಗಳನ್ನು ಅರ್ಥಮಾಡಿಕೊಂಡವರು.  ಧಾರ್ಮಿಕತೆಯ ಅನುಯಾಯಿಯಾದರೂ, ವೈಜ್ಞಾನಿಕತೆಯನ್ನು ಬಿಟ್ಟುಕೊಟ್ಟವರಲ್ಲ.  ಪ್ರಾಚೀನ ಹಾಗೂ ನವೀನ ನ್ಯಾಯಶಾಸ್ತ್ರದಲ್ಲಿ ಸಾಂಪ್ರದಾಯಿಕ ಅಧ್ಯಯನ ಮಾಡಿದ್ದ ಗಂಗಾಧರ ಭಟ್ಟರು,  ತಮ್ಮ ಜ್ಞಾನವನ್ನು ಅದೊಂದೇ ಕ್ಷೇತ್ರಕ್ಕೆ ಮೀಸಲಿಟ್ಟವರಲ್ಲ.  ಅವರ ತಿಳುವಳಿಕೆಗೆ ನಿಲುಕುವ ಯಾವುದೇ ವಿಷಯವನ್ನು ಸಮಗ್ರವಾಗಿ ಯೋಚಿಸಿ ತೀರ್ಮಾನಕ್ಕೆ ಬರುತ್ತಿದ್ದರು.  ತರ್ಕಬದ್ಧವಾದ ಚಿಂತನೆಗಳಿಂದಾಗಿ  ಗಂಗಾಧರ ಭಟ್ಟರು ವಿವಾದಾತೀತ ವ್ಯಕ್ತಿ.    

ನಾನು ವಿಜ್ಞಾನದ ಪ್ರಾಧ್ಯಾಪಕ, ಜೀವ ವಿಜ್ಞಾನ ಅದರಲ್ಲಿಯೂ ಪ್ರಾಣಿಗಳ ಅಧ್ಯಯನದಲ್ಲಿ ವಿಶೇಷ ಆಸಕ್ತಿ ವಹಿಸಿದವ.  ವಿಶ್ವವಿದ್ಯಾನಿಲಯದ ಪ್ರಾಣಿವಿಜ್ಞಾನ ವಿಭಾಗದಲ್ಲಿ ಅಧ್ಯಾಪಕನಾಗಿ ಸೇರಿದ ನಂತರ ವೈಜ್ಞಾನಿಕ ವಿಷಯಗಳನ್ನು ಮಾತೃಭಾಷೆಯಲ್ಲಿ ಬರೆಯಲು ಆರಂಭಿಸಿದೆನು.  ಮೊದಮೊದಲು ಬರೆವಣಿಗೆ ಕಷ್ಟಕರವಾಗಿ ತೋರುತ್ತಿತ್ತು.  ಪದಗಳ ಕೊರತೆ ಇದೆ ಎನ್ನಿಸುತ್ತಿತ್ತು.  ಆದರೂ ನನ್ನ ಪ್ರಯತ್ನವನ್ನು ಬಿಡುತ್ತಿರಲಿಲ್ಲ.  ಆಗೆಲ್ಲ ನನಗೆ ನೆರವಾಗುತ್ತಿದ್ದವರು ಮಾನ್ಯ ವಿದ್ವಾನ್ ಗಂಗಾಧರ ಭಟ್ಟರು. ಕನ್ನಡ ಪದಗಳು ದೊರೆಯದಿದ್ದಾಗ ಸಂಸ್ಕೃತದ ಪದಗಳನ್ನು ಆಶ್ರಯಿಸುವುದನ್ನು ರೂಢಿಸಿಕೊಂಡೆ.  ಗಂಗಾಧರ ಭಟ್ಟರನ್ನು ಸಂಪರ್ಕಿಸಿದರೆ ಸಂಸ್ಕೃತಪದಗಳು ಸುಲಭವಾಗಿ ದೊರಕುತ್ತಿದ್ದವಲ್ಲ.  ಹೀಗಾಗಿ ನನ್ನ ಬರೆವಣಿಗೆಗಳ ಮೌಲ್ಯ ಹೆಚ್ಚಿತು ಎನ್ನಬಹುದು.  

ನನಗೆ ಬಿಡುವಾದಾಗಲೆಲ್ಲ ಗಂಗಾಧರ ಭಟ್ಟರನ್ನು ಭೇಟಿಯಾಗುತ್ತಿದ್ದೆ.  ಪ್ರತಿಯೊಂದು ಭೇಟಿಯ ಸಂದರ್ಭದಲ್ಲಿಯೂ ಅನೇಕ ಹೊಸ ಹೊಸ ವಿಷಯಗಳನ್ನು ತಿಳಿದು ಕೊಳ್ಳುತ್ತಿದ್ದೆ.  ಎಷ್ಟೊ ಬಾರಿ ನಮ್ಮಿಬ್ಬರ ನಡುವೆ ಚರ್ಚೆ ಹಲವಾರು ಗಂಟೆಗಳ ಕಾಲ ನಡೆಯುತ್ತಿತ್ತು.  ವೈಜ್ಞಾನಿಕ ವಿಷಯಗಳನ್ನು ನನ್ನಲ್ಲಿ ಪ್ರಶ್ನಿಸಿ ತಿಳಿದುಕೊಳ್ಳುತ್ತಿದ್ದರು.  ನನಗೆ ವೈಜ್ಞಾನಿಕ ವಿಷಯಗಳನ್ನು ಧಾರ್ಮಿಕ, ಪೌರಾಣಿಕ ಅಥವಾ ಐತಿಹಾಸಿಕ ದೃಷ್ಠಿಕೋನದಿಂದ ಪರಿಶೀಲಿಸಲು ಸಾಧ್ಯವಾಗುತ್ತಿತ್ತು.   

ವಿಜ್ಞಾನದ ಬೆಳೆವಣಿಗೆಯ ಕುರಿತು ಇವರಿಗೆ ಬಹಳ ಆಸಕ್ತಿ. ಅವರನ್ನು ಭೇಟಿಯಾದಾಗಲೆಲ್ಲ \break ಒಂದೊಂದು ವಿಷಯಗಳ ಬಗ್ಗೆ ನಮ್ಮಲ್ಲಿ ಚರ್ಚೆ ನಡೆಯುತ್ತಿತ್ತು.  ಅಣು–ಕಣಗಳಿಂದ ಹಿಡಿದು ತಿಮಿಂಗಿಲ ಬ್ರಹ್ಮಾಂಡಗಳ ವರೆಗಿನ ವಿವಿಧ ವಿಷಯಗಳ ಕುರಿತು ನನ್ನೊಂದಿಗೆ ಚರ್ಚಿಸುತ್ತಿದ್ದರು.  ವಿವಿಧ ವಿಷಯಗಳಲ್ಲಿ ಅವರಿಗೆ ಆಸಕ್ತಿ ಇತ್ತು.  ಈ ಜ್ಞಾನದಾಹದಿಂದಾಗಿಯೇ ಅವರು ವಿಜ್ಞಾನ, ಪರಿಸರ, ಕೃಷಿ, ಆಯುರ್ವೇದ ಮುಂತಾದ ವಿಷಯಗಳಿಗೆ ಸಂಬಂಧಿಸಿದ ಅನೇಕ ಕಾರ್ಯಕ್ರಮಗಳಲ್ಲಿ ಭಾಗವಹಿಸಿದ್ದಾರೆ, ಆಯೋಜಿಸಿದ್ದಾರೆ ಹಾಗೂ ನಿರ್ವಹಣೆ ಮಾಡಿದ್ದಾರೆ.  ಯಾವುದೇ ಜವಾಬ್ದಾರಿಯನ್ನು ನೀಡಿದರೂ ಸಮರ್ಥವಾಗಿ ನಿರ್ವಹಿಸುವುದು ಗಂಗಾಧರ ಭಟ್ಟರ ವೈಶಿಷ್ಟ್ಯ. 

ಗಂಗಾಧರ ಭಟ್ಟರ ಇನ್ನೊಂದು ಸದ್ಗುಣ ಅತಿಥಿ ಸತ್ಕಾರ.  ಉತ್ತರ ಕನ್ನಡ ಜಿಲ್ಲೆಯ ಸಿರ್ಸಿ ಸಿದ್ದಾಪುರ ತಾಲೂಕಿನ ಯಾವುದೇ ಭಾಗದಿಂದ ಯಾರಾದರೂ ಮೊದಲ ಬಾರಿಗೆ ಮೈಸೂರಿಗೆ ಬಂದರೆಂದರೆ ಅವರಿಗೆ ಆಶ್ರಯ ಒದಗಿಸುತ್ತಿದ್ದವರು ಗಂಗಾಧರ ಭಟ್ಟ ದಂಪತಿಗಳು.  ಅಭ್ಯಾಗತರ ಕಾರ್ಯಭಾರಗಳಿಗೆ ಸೂಕ್ತ ಮಾರ್ಗದರ್ಶನವನ್ನು ಗಂಗಾಧರ ಭಟ್ಟರು ನೀಡುತ್ತಿದ್ದರೆ, ಶ್ರೀಮತಿ ಶೈಲಜಾ ಊಟೋಪಚಾರಗಳ ವ್ಯವಸ್ಥೆಯನ್ನು ನೋಡಿಕೊಳ್ಳುತ್ತಿದ್ದರು.  ವಾಣಿವಿಲಾಸ ಮಾರ್ಕೆಟ್ಟಿನ ಸಮೀಪದಲ್ಲಿ ಸಯ್ಯಾಜಿರಾವ್ ರಸ್ತೆಯ ಕನ್ನಡ ಪ್ರಭ ಕಛೇರಿಯ ಪಕ್ಕದ ಇವರ ಮನೆ ಸದಾ ಛತ್ರದಂತೆ ಇದ್ದದ್ದನ್ನು ನಾನು ಕಂಡಿದ್ದೇನೆ.  ಈಗಲೂ ಅಷ್ಟೆ, ತೊಣಚಿಕೊಪ್ಪಲು ಬಡಾವಣೆಯಲ್ಲಿರುವ ಇವರ ಮನೆಯಲ್ಲಿ ಸದಾ ಯಾರಾದರೊಬ್ಬರು ಅಭ್ಯಾಗತರು ಇರಲೇ ಬೇಕು.  ಇವರ ಮನೆಗೆ ಬರುವವರ ಉದ್ದೇಶವೂ ವಿಭಿನ್ನವಾಗಿರುತ್ತದೆ, ಶಾಲಾ–ಕಾಲೇಜುಗಳ ಪ್ರವೇಶಾತಿಗಾಗಿ ಬರುವವರಿಂದ ಹಿಡಿದು, ಉದ್ಯೋಗವನ್ನರಸಿ ಅಥವಾ ಉದ್ಯೋಗಕ್ಕೆ ಸೇರಿಕೊಳ್ಳಲು ಬರುವವರೋ ಇಲ್ಲವೆ ಯಾವುದೋ ಕಚೇರಿ ಕೆಲಸಕ್ಕಾಗಿ ಬರುವವರೂ ಇದ್ದರು.  ಹೀಗೆ ಬಂದ ಅಭ್ಯಾಗತರು ಬಂದ ಉದ್ದೇಶವನ್ನು ಪೂರೈಸಿಕೊಳ್ಳುವವರೆಗೆ ಗಂಗಾಧರ ಭಟ್ಟರ ಅತಿಥಿಯಾಗಿದ್ದ ಸಂದರ್ಭಗಳಿವೆ.         

ಗಂಗಾಧರ ಭಟ್ಟರು ಸಂಘಟನಾ ಚತುರರು.  ಮೈಸೂರಿನಲ್ಲಿ 1978ರಲ್ಲಿ ಪ್ರಾರಂಭವಾದ ಹವೀಕ ಸಂಘವು ಹತ್ತು ವರ್ಷಗಳ ಚಟುವಟಿಕೆಗಳ ನಂತರ ಇದ್ದಕ್ಕಿದ್ದಂತೆ ಕ್ರಿಯಾಹೀನವಾದಾಗ, ಅದರ ಕಾರ್ಯದರ್ಶಿ ಸ್ಥಾನದ ಜವಾಬ್ದಾರಿಯನ್ನು ಹೊತ್ತು ಪುನಃ ಚೇತರಿಸಿಕೊಳ್ಳುವಂತೆ ಮಾಡಿದರು.  ಗಂಗಾಧರ ಭಟ್ಟರು, ಹವೀಕ ಸಂಘ ಹಾಗೂ  ಶ್ರೀಮನ್‍ಮಹಾರಾಜ ಕಾಲೇಜಿನ ಪ್ರದೋಷ ಸಂಘದ ಆಶ್ರಯದಲ್ಲಿ ಅನೇಕ ವೈವಿಧ್ಯಮಯ ಕಾರ್ಯಕ್ರಮಗಳನ್ನು ಹಮ್ಮಿಕೊಂಡು ಅವುಗಳನ್ನು ಆಕರ್ಷಕವಾಗುವಂತೆ ನಿರ್ವಹಿಸಿ ಯಶಸ್ಸುಗಳಿಸಿದ್ದಾರೆ.  

ಗಂಗಾಧರ ಭಟ್ಟರು ಅತ್ಯುತ್ತಮವಾದ ವಾಗ್ಮಿ ಎಂಬುದರಲ್ಲಿ ಯಾವುದೆ ಶಂಕೆ ಇಲ್ಲ.  ಕನ್ನಡ ಹಾಗು ಸಂಸ್ಕೃತ ಭಾಷೆಯ ಮೇಲಿನ ಹಿಡಿತ, ಈಗಿನ ದಿನಗಳಲ್ಲಿ ಅಗತ್ಯವೆನಿಸುವಷ್ಟು ಇಂಗ್ಲೀಷಿನ ತಿಳಿವಳಿಕೆ, ವಿಷಯದ ಕುರಿತ ಅರಿವು, ಪ್ರಬುದ್ಧತೆಗಳು,  ಆಧ್ಯಯನಶೀಲತೆ ಮುಂತಾದ ಗುಣಗಳಿಂದ ಈತ ಅತ್ಯುತ್ತಮ ಭಾಷಣಕಾರ ಎಂಬ ಹೆಗ್ಗಳಿಕೆಗೆ ಪಾತ್ರರಾಗಿ, ವಿದ್ಯಾರ್ಥಿ ದೆಸೆಯಿಂದಲೇ ಅನೇಕ ಸ್ಪರ್ಧೆ, ಗೋಷ್ಠಿಗಳಲ್ಲಿ ಬಹುಮಾನ, ಪ್ರಶಸ್ತಿ ಹಾಗೂ ಪುರಸ್ಕಾರಗಳನ್ನು ದೊರಕಿಸಿಕೊಂಡಿದ್ದರಲ್ಲಿ ಆಶ್ಚರ್ಯ ಪಡುವಂತಹದೇನೂ ಇಲ್ಲ.  ಸನ್ನಡತೆ, ಸತ್–ಚಾರಿತ್ರ್ಯಗಳನ್ನು ಅಳವಡಿಸಿಕೊಂಡು ಪರಿಪೂರ್ಣತೆಯ ಕಡೆಗೆ ಸಾಗಿದ ಇಂಥ ಅಪರೂಪದ ವ್ಯಕ್ತಿಗೆ ಹಾಗೂ ಇವರ ಕುಟುಂಬಕ್ಕೆ ಅಭಿನಂದನೆಗಳು, ಶುಭಹಾರೈಕೆಗಳು.   

\articleend
}
