\chapter{ಶಿಷ್ಯರ ಮನಗೆದ್ದ ಸರಳಸ್ವಭಾವದ ವಿದ್ವಾನ್ ಗಂಗಾಧರ ಭಟ್ಟರು}

\begin{center}
\Authorline{ಡಾ । ರಾಚೋಟಿ ದೇವರು.}
\smallskip
ಅತಿಥಿ ಉಪನ್ಯಾಸಕರು\\
ಸಂಸ್ಕೃತ ಅಧ್ಯಯನ ವಿಭಾಗ,\\
ಮಾನಸ ಗಂಗೋತ್ರಿ. ಮೈಸೂರು.
\addrule
\end{center}
\begin{verse}
ಗುಕಾರಸ್ತ್ವನ್ಧಕಾರಸ್ಯಾತ್ ರುಕಾರಸ್ತನ್ನಿರೋಧಕಃ ।\\
ಅನ್ಧಕಾರನಿರೋಧಿತ್ವಾತ್ ಗುರುರಿತ್ಯಭೀಧೀಯತೇ ॥
\end{verse}
ಭಾರತೀಯ ಭವ್ಯಪರಂಪರೆಯಲ್ಲಿ ಮಾತಾಪಿತೃಗಳಂತೆ ಗುರುವಿಗೆ ಅತ್ಯಂತ ಪ್ರಾಶಸ್ತ್ಯ ಕೊಟ್ಟಿರುವುದನ್ನು ನಾವು ಕಾಣುತ್ತೇವೆ. ಅದಕ್ಕೆ ಕಾರಣವನ್ನು ಗುರು ಶಬ್ದವೇ ಹೇಳುತ್ತದೆ. “ಗು” ಶಬ್ದ ಅಂಧಕಾರವನ್ನು “ರು” ಎಂದರೆ ಪ್ರಕಾಶವನ್ನು ಸೂಚಿಸುತ್ತದೆ. ಶಿಷ್ಯರ ಅಜ್ಞಾನರೂಪ ಅಂಧಕಾರವನ್ನು ಕಳೆದು ಸುಜ್ಞಾನರೂಪ ಪ್ರಕಾಶವನ್ನು ನೀಡುವದರಿಂದ ಅಂಥವರನ್ನು ನಾವು ಗುರುವೆಂದು ಪೂಜಿಸುತ್ತೇವೆ. ಆದರೆ ಪ್ರಾಚೀನ ಕಾಲದಲ್ಲಿ ಗುರುಕುಲ ಪದ್ಧತಿಯಲ್ಲಿ ಅಧ್ಯಯನ ಮಾಡುವ ಸಂದರ್ಭದಲ್ಲಿ ಶಿಷ್ಯನ ಎಲ್ಲಾ ಅಗುಹೋಗುಗಳಿಗೆ ಗುರುವಾದವನು ಕಾರಣನಾಗುತ್ತಿದ್ದನು, ಶಿಷ್ಯನ ಏಳಿಗೆಯನ್ನು ಸದಾ ಬಯುಸುವ ಸಹೃದಯರಾಗಿರುತ್ತಿದ್ದರೆಂದು ನಾವು ತಿಳಿಯುತ್ತೇವೆ. ಆದರೆ ಇಂದು ವಿದ್ಯೆ ಎನ್ನುವುದು ಹಣದಿಂದ ಕೊಂಡುಕೊಳ್ಳುವ ವಸ್ತುವಾಗಿರುವದು ವಿಷಾದನೀಯ. ಇಂದಿನ ವಿದ್ಯೆ ಕೇವಲ ಉದ್ಯೋಗಕ್ಕಾಗಿ ನೀಡುವ ಪ್ರಮಾಣ ಪತ್ರಕ್ಕೆ ಮೀಸಲಾಗಿರುವುದು ಸತ್ಯ ಸಂಗತಿ, ಇಂದಿನ ಶಿಕ್ಷಣ ಪದ್ದತಿಯಲ್ಲಿ ವಿದ್ಯಾರ್ಥಿಗಳ ಹಿತಕ್ಕಿಂತಲೂ ಅವರಿಗೆ ಪರೀಕ್ಷೆಯನ್ನು ಉತ್ತಿರ್ಣಗೊಳಿಸುವುದು ಮುಖ್ಯವಾಗಿದೆ. ಆದ್ದರಿಂದ ಇಂದಿನ ದಿನಮಾನದಲ್ಲಿ ಗುರುಶಿಷ್ಯರ ಸಂಬಂಧ ಅವರು ಕೊಡುವ ಅಂಕಗಳಿಗೆ ನಿಗದಿತವಾಗಿದ್ದು ಅದು ಅವರ ಜೀವನಕ್ಕೆ ನಾಂದಿಯಾಗಿದೆ. ಒಂದುವೇಳೆ ಅಂಕಗಳು ಕಡಿಮೆಯಾದರೆ ಅವರಿಗೆ ಉದ್ಯೋಗಾವಕಾಶಗಳು ಸಿಗುವುದಿಲ್ಲ. ಈ ಎಲ್ಲಾ ಕಾರಣಗಳಿಂದ ಇಂದಿನ ವಿದ್ಯೆ, ಗುರು ಮತ್ತು ಶಿಷ್ಯ ವ್ಯವಹಾರ ಹಣಗಳಿಸುವ ವ್ಯಾಪಾರದ ಅನಾದರ್ಶ ಸ್ಥಿತಿಗೆ ಬಂದು ತಲುಪಿದೆಯೆಂದರೆ ಅತಿಶಯೋಕ್ತಿಯಲ್ಲ. ಆದರೂ ಎಲ್ಲೂ ಆದರ್ಶವಿಲ್ಲವೇ ಇಲ್ಲ ಎನ್ನಲಾಗದು. ಹಾಗಾಗಿ ಒಬ್ಬ ಸುಭಾಷಿತಕಾರನು ಹೀಗೆ ಹೇಳುತ್ತಾನೆ 
\begin{verse}
ಗುರವೋ ಬಹವಸ್ಸನ್ತಿ ಶಿಷ್ಯವಿತ್ತಾಪಹಾರಕಾಃ ।\\
ಗುರವೋ ದುರ್ಲಭಾಸ್ಸನ್ತಿ ಶಿಷ್ಯ ಚಿತ್ತಾಪಹಾರಕಾಃ ॥
\end{verse}
ಇಂದಿನ ಗುರುಗಳು ಅಥವಾ ಶಿಕ್ಷಕರು ವಿದ್ಯಾರ್ಥಿಗಳ, ಶಿಷ್ಯರ ಹಣವನ್ನು ಸೂರೆಗೊಳ್ಳುವವರು ಹಲವರನ್ನು ನಾವು ಕಾಣುತ್ತೇವೆ. ಆದರೆ ಶಿಷ್ಯರ ಮನಸ್ಸನ್ನು ಗೆಲ್ಲುವವರು ಎಲ್ಲೋ ಕೆಲವರು ಮಾತ್ರ.

ಪ್ರಕೃತ ನಾನು ಹೇಳಹೊರಟಿರುವುದು ಶಿಷ್ಯವಿತ್ತವನ್ನು ಕಿಂಚಿತ್ತೂ ಅಪೇಕ್ಷಿಸದ ಆದರೆ ಚಿತ್ತವನ್ನು ಸಂಪೂರ್ಣ ಆಕರ್ಷಿಸಿದ ವಿದ್ವಾನ್. ವಿ ಗಂಗಾಧರ ಭಟ್ಟರ ಬಗ್ಗೆ.

ಮೈಸೂರು ಭಾಗವು ಕಲೆ ಮತ್ತು ಸಂಸ್ಕೃತಿಯ ನೆಲೆವೀಡು. ಈ ನೆಲ ಹಲವು ವಿದ್ವಾಂಸರು ಸಾಹಿತಿಗಳು, ಕವಿಗಳಿಗೆ ಆಶ್ರಯವಿತ್ತಿದೆ. ಇದರ ಕೀರ್ತಿ ಮೈಸೂರು ಮಹಾರಾಜರಿಗೆ ಸಲ್ಲುತ್ತದೆ. ಅಷ್ಟೇ ಅಲ್ಲದೆ ಇಲ್ಲಿ ಸಂಸ್ಕೃತ ಅಧ್ಯಯನ ಅಧ್ಯಾಪನಕ್ಕೇ ಹೆಚ್ಚು ಒತ್ತನ್ನು ಕೊಟ್ಟು ಸಂಸ್ಕೃತ ವಿದ್ವಾಂಸರನ್ನು ಬೆಳೆಸಿದ ಕೀರ್ತಿ ಅವರಿಗೆ ಸಲ್ಲುತ್ತದೆ.

ನಾನು ಮೈಸೂರು ನಗರಕ್ಕೆ ಸಂಸ್ಕೃತ ಅಧ್ಯಯನಕ್ಕಾಗಿ ಬಂದು ಸುತ್ತೂರು ಶ್ರೀಮಠದ ಗುರುಕುಲದಲ್ಲಿ ಆಶ್ರಯವನ್ನು ಪಡೆದುಕೊಂಡಾಗ, ಸಂಸ್ಕೃತ ಅಧ್ಯಯನಕ್ಕೆ ಎಲ್ಲಿ ಹೋಗುವುದು ಎನ್ನುವ ಪ್ರಶ್ನೆ ಬಂದಾಗ ಅಲ್ಲಿ ಇದ್ದ ನಮ್ಮ ಎಲ್ಲಾ ಸಹಪಾಠಿಗಳು ಹೇಳಿದ್ದ ಒಂದು ಮಾತು ಇಂದಿಗೂ ನನಗೆ ನೆನಪು ಇದೆ, ಅದು ಸಂಸ್ಕೃತವನ್ನು ಕಲಿಯ ಬೇಕಾದರೆ ಇಲ್ಲಿ ಪ್ರಸಿದ್ದರಾದ ಒಬ್ಬ ವಿದ್ವಾಂಸರು ಇದ್ದಾರೆ ಅವರು ಜಪದಕಟ್ಟೆ ಮಠದಲ್ಲಿ ಅಧ್ಯಾಪಕರಾಗಿದ್ದಾರೆ ಅವರಿಂದ ಮಾತ್ರ ಒಳ್ಳೆಯ ವಿದ್ಯೆಯನ್ನು ಕಲಿಯಲು ಸಾಧ್ಯ ಎಂದು ಹೇಳಿದರು. ಅವರೇ ವಿದ್ವಾನ್ ಗಂಗಾಧರ ಭಟ್ಟರು. ಈ ಸಮಯದಲ್ಲಿ ಅವರ ಹೆಸರು ಮೊಟ್ಟ ಮೊದಲಿಗೆ ನನ್ನ ಕಿವಿಯ ಮೆಲೆ ಬಿತ್ತು, ಅನಂತರ ನಾನು ಜಪದಕಟ್ಟೆ ಮಠದಲ್ಲಿ ಅವರ ಪಾಠ ಪ್ರವಚನವನ್ನು ಕೇಳಿ ಅಂದಿನಿಂದ ಅವರ ವಿದ್ಯಾರ್ಥಿಯಾದೆ. ಇಂದಿನವರೆಗೂ ಕೂಡಾ ಏನಾದರೂ ಸಂಶಯ ಬಂದರೆ ಮೊದಲು ನಾನು ಹೋಗುವುದು ಅವರಲ್ಲಿಯೇ. ನಮ್ಮ ಜೀವನದಲ್ಲಿ ಹಲವು ಅಧ್ಯಾಪಕರನ್ನು ನಾವು ನೋಡಿದ್ದೇವೆ ಆದರೆ ಅವರಂಥವರನ್ನು ನಾವು ಕಾಣುವುದು ದುರ್ಲಭ. ಅವರು ಯಾವತ್ತೂ ತಮ್ಮ ವಿದ್ಯಾರ್ಥಿಗಳಿಗೆ ಅನುಕೂಲ ಮಾಡಿಕೊಡುವುದೇ ಉದ್ದೇಶವುಳ್ಳವರಾಗಿದ್ದಾರೆ. ಅವರ ಹೃದಯವು ಮಾತೃವಾತ್ಸಲ್ಯದಿಂದ ಕೂಡಿದ್ದು, ಅವರು ಸದಾ ಸರಳ ಸಜ್ಜನಿಕೆಯ ಸ್ವಭಾವವುಳ್ಳವರು. ಅವರ ಪ್ರೀತಿಯ ಮಾತು ಸದಾ ಮಕ್ಕಳನ್ನು ತಿದ್ದಿ ಸರಿದಾರಿಗೆ ತರುವ ತಾಯಿಯ ಮಾತಿನಂತಿರುತ್ತದೆ..

ನಾನು ಅವರ ವಿದ್ಯಾರ್ಥಿಯಾಗಿ ನ್ಯಾಯಶಾಸ್ತ್ರಕ್ಕೆ ಸರ್ಕಾರಿಮಹಾರಾಜ ಸಂಸ್ಕೃತಕಾಲೇಜಿಗೆ ಪ್ರವೇಶವನ್ನು ಪಡೆದಾಗ ತರ್ಕಶಾಸ್ತ್ರದ ಬಗ್ಗೆ ಲವಲೇಶವೂ ಗೊತ್ತಿಲ್ಲದ ನನಗೆ ಅವರು ಸಾಕಷ್ಟು ವೇಳೆ ಅವರ ಮನೆಯಲ್ಲಿಯೂ ಪಾಠ ಮಾಡುವ ಮುಖಾಂತರ ವಿದ್ಯಾದಾನ ಮಾಡಿರುತ್ತಾರೆ. ಮತ್ತು ನಾನು ಅವರ ಮನೆಗೆ ರಾತ್ರಿಯ ಯಾವ ಸಮಯದಲ್ಲಿ ಹೋದರೂ ಕೂಡ ಪಾಠವನ್ನು ಮಾಡಿದ್ದಾರೆ. ಇಂದು ಕೆಲವು ಅಧ್ಯಾಪಕರು ಸರ್ಕಾರಿ ಉದ್ಯೋಗ ಬಂದರೆ ಸಾಕು ಅವರು ತಾವೇ ದೊಡ್ಡ ವಿದ್ವಾಂಸರು ಎಂದು ಜಂಬಕೊಚ್ಚಿಕೊಂಡು ವಿದ್ಯಾರ್ಥಿಗಳಿಗೆ ಪಾಠಮಾಡದೆ ಇರುವಂಥವರನ್ನು ನಾವು ಕಾಣುತ್ತೇವೆ. ಆದರೆ ವಿದ್ವಾನ್ ಗಂಗಾಧರ ಭಟ್ಟರು ಯಾವುದೇ ವಿದ್ಯಾರ್ಥಿ ಹೋಗಿ ಕೇಳಿದರೆ ಯಾವ ಸಮಯದಲ್ಲಿಯಾದರೂ ಕೂಡ ಅವರಿಗೆ ವಿದ್ಯೆಯನ್ನು ಹೇಳಿಕೊಟ್ಟವರು. ನನಗೂ ಕೂಡಾ ಪರೀಕ್ಷೆಯ ಸಮಯದಲ್ಲಿ ರಾತ್ರಿ ಹನ್ನೊಂದು ಗಂಟೆಯವರೆಗೂ ಪಾಠಮಾಡಿರುತ್ತಾರೆ.

ವಿದ್ಯಾರ್ಥಿಗಳಿಗೆ ಪಾಠವ ಮಾಡುವುದಷ್ಟೆ ಅವರ ಕಾಯಕವಾಗದೆ ಅವರು ಹಲವು ಸಂಶೋಧನಾತ್ಮಕ ಲೇಖನಗಳನ್ನು ಮತ್ತು ಗ್ರಂಥಗಳನ್ನು ಕೂಡ ಪ್ರಕಟಿಸಿದ್ದಾರೆ. ಉದಾಹರಣೆಗೆ “ಮಂಡಿಕಲ್ ರಾಮಶಾಸ್ತ್ರೀ”ಯವರ ಪರಿಚಯಾತ್ಮಕ ಗ್ರಂಥವನ್ನು ಸಂಸ್ಕೃತದಲ್ಲಿ ರಚಿಸಿರುತ್ತಾರೆ. ಈ ಗ್ರಂಥದಲ್ಲಿ ಮಂಡಿಕಲ್ ರಾಮಶಾಸ್ತ್ರೀಗಳ ಜೀವನ ಶೈಲಿ ಮತ್ತು ಅವರ ಸಾಹಿತ್ಯಿಕ ಕೊಡಿಗೆಗಳ ವಿಷಯಕವಾಗಿ ವಿಶದೀಕರಿಸಿರುತ್ತಾರೆ.

ಶ್ರೀಯುತರು ಮಕ್ಕಳಿಗೆ ಉಪಯೋಗವಾಗುವ ರೀತಿಯಲ್ಲಿ ಸರಳ ಸಂಸ್ಕೃತದಲ್ಲಿ “ಕಥಾಸರಸ್ವತೀ” ಎನ್ನುವ ಗ್ರಂಥವನ್ನು ಸಾಹಿತ್ಯಲೋಕಕ್ಕೆ ನೀಡಿದ್ದಾರೆ. ಪ್ರಸ್ತುತ ಈ ಗ್ರಂಥದಲ್ಲಿ ಸರಸ್ವತೀ ನದಿಯ ವಿಷಯಕವಾಗಿ ಸಾಕಷ್ಟು ವಿಷಯವನ್ನು ಸಂಗ್ರಹಿಸಿರುವುದನ್ನು ನಾವು ಇಲ್ಲಿ ಕಾಣಬಹುದು. ಅಷ್ಟೆ ಅಲ್ಲದೆ ಮಂಗಳೂರು ವಿಶ್ವವಿದ್ಯಾನಿಲಯದಲ್ಲಿ ಪದವಿ ಪಠ್ಯಪುಸ್ತಕದ ಸಂಪಾದಕರಾಗಿ ಅಲ್ಲಿ ಅವರು \textbf{“ಸಂಸ್ಕೃತವಾಣಿಜ್ಯಮ್”} ಎನ್ನುವ ಗ್ರಂಥದಲ್ಲಿ ಅರ್ಥನೀತಿಯ ಬಗ್ಗೆ ಪಾಠ ಬರೆದಿದ್ದಾರೆ. ಹೀಗೆ ಶ್ರೀ ಯುತರು ವಿದ್ಯಾರ್ಥಿಗಳ ಹಾಗೂ ಸಾಮಾಜಿಕರ ಅಭಿವೃದ್ಧಿಗಾಗಿ ಶ್ರಮಿಸಿ ಸದಾ ನಗು ನಗುತ್ತ ತಮ್ಮ ಜೀವನವನ್ನು ಶಿಷ್ಯರಿಗಾಗಿ ಕಳೆಯುತ್ತಾ ಬಂದಿದ್ದಾರೆ.

ಆದರೆ ಅವರು ಇಂದು ಸಂಸ್ಕೃತ ಅಧ್ಯಾಪಕ ವೃತ್ತಿಯಿಂದ ನಿವೃತ್ತಿಹೊಂದುತ್ತಿರುವ ಶುಭ ಸಂದರ್ಭದಲ್ಲಿ ಅವರ ಶಿಷ್ಯವೃಂದವೆಲ್ಲ ಸೇರಿ ಗುರುವಂದನಾ ಕಾರ್ಯಕ್ರಮ ಮಾಡಿರುವುದು ಶ್ಲಾಘನೀಯವಾದದ್ದು ಮತ್ತು ಅವರ ವಿದ್ಯಾರ್ಥಿಗಳಾದ ನಮಗೆ ಇದು ಒಂದು ಸುದೈವವೇ ಸರಿ. ಗುರುವಿಗೆ ನಿವೃತ್ತಿಯಿರುವುದಿಲ್ಲ ಆದರೆ ಆಧುನಿಕ ಕಾಲದ ಶೈಕ್ಷಣಿಕ ಪದ್ದತಿಗೆ ಅನುಗುಣವಾಗಿ ಅವರು ನಿವೃತ್ತಿ ಹೊಂದುವುದು ಸಹಜ, ಆದರೆ ಅವರು ಸದಾ ತಮ್ಮ ಶಿಷ್ಯವೃಂದಕ್ಕೆ ಮಾರ್ಗದರ್ಶನ ಮಾಡುತ್ತಾ ಸುದೀರ್ಘ ವರ್ಷಗಳವರೆಗೆ ಅವರು ಮತ್ತು ಅವರ ಎಲ್ಲಾ ಕುಟುಂಬದ ಸದಸ್ಯರಿಗೆ ಆರೋಗ್ಯ ಭಾಗ್ಯಾದಿಗಳನ್ನು ಭಗವಂತನು ನೀಡಲೆಂದು ಪ್ರಾರ್ಥಿಸುವೆ.

\centerline{\textbf{“ಶಿವಂ ಭೂಯಾತ್”}}

\articleend
