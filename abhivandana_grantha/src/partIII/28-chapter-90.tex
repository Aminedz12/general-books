\chapter{ಅಪ್ರತಿಮ ಜ್ಞಾಪಕ ಶಕ್ತಿಯ ವಿದ್ವಾನ ಗಂಗಾಧರ ಭಟ್ಟರು}

\begin{center}
\Authorline{ಶ್ರೀಧರ್ ಆರ್ ಭಟ್ಟ}

ಪತ್ರಕರ್ತರು\\  
ನಂಜನಗೂಡು
\end{center}

ನಾನು ಜೀವನದ ದಾರಿ ಹುಡುಕುತ್ತ ನಂಜನಗೂಡಿಗೆ ಬಂದು ನೆಲಸಿದ ಪ್ರಾರಂಭದ ದಿನಗಳು ಬಹುಶಃ 1984 ರ ಆರಂಭದ ದಿನಗಳು,  ಆಗಲೇ  ಆಂದೋಲನಪತ್ರಿಕೆಯ ಬಳಗದಲ್ಲಿ ಸೇರಿಕೊಂಡಿದ್ದ ನಾನು ಒಂದು ಸಮಾರಂಭದ ನಿಮಿತ್ತ ಮೈಸೂರಿನ  ಗೋವಿಂದರಾವ್ ಮೆಮೋರಿಯಲ್ ಹಾಲ್‍ಗೆ   ಹೋಗಿದ್ದೆ ಅಲ್ಲಿ ನನ್ನ ಹಾಗೂ ಗಂಗಾಧರಭಟ್ಟರ ಪ್ರಥಮ ಬೇಟಿಯಾದ ನೆನಪು . ಅದಾಗಲೇ ನಾನು ನಮ್ಮ ತಾಲೂಕಿನವರೇ ಆಗಿದ್ದ  ಗಂಗಾಧರ ಭಟ್ಟರ ಕುರಿತಾಗಿ ಕೇಳಿದ್ದೆ .   ಉಪ್ಪಿಟ್ಟಿನ ಸುಬ್ಬಣ್ಣನೆಂದು ಅವರಿಂದ ಕರೆಸಿಕೊಳ್ಳುತ್ತಿದ್ದ ನನ್ನ ಸೊಧರ ಬಾವ ಕಸಿಗೆ ಸುಬ್ಬಣ್ಣ  ಅವರ ಪಾಂಡಿತ್ಯ  ಅಘಾದವಾದ ಜ್ಞಾನ ಭಂಡಾರದ ಬಗೆಗೆ ನನಗೆ ಆಗಲೇ ತಿಳಿಸಿದ್ದ. ಸದಾ  ಹಸನ್ಮುಖಿಯಾಗಿರುತ್ತಿದ್ದ  ಅವರನ್ನು ಸ್ನೇಹಿತರು ಶಿಷ್ಯವೃಂದ ಯಾವಾಗಲು ಮುತ್ತಿಕೊಂಡೆ ಇರುತ್ತದೆ ಅಂದೂ ಹಾಗೇ  ಇಂದೂ ಹಾಗೆ, ಅಂದು ನಾನೇ ಮುಂದಾಗಿ ಅವರ  ಪರಿಚಯ ಮಾಡಿಕೊಂಡೆ.  ಸಾದಾರಣ ಮೈಕಟ್ಟಿನ  ಗಂಗಾಧರ ಭಟ್ಟರ ವಾಮನಾವತಾರದ ಪರಿಚಯವಾದದ್ದು ನನಗೆ ಹಾಗೆ.

ನಮ್ಮ  ಸಿರ್ಸಿ ಸಿದ್ದಾಪರ  ಭಾಗದ ಜನರ ಪಾಲಿಗೆ ಮೈಸೂರಿನ ವಿದ್ಯಾಭ್ಯಾಸ  ಎಂದರೆ ಅವರೆಲ್ಲರಿಗೆ ನೆನಾಪಾಗುತ್ತಿದ್ದಿದ್ದು  ಈ ಗಂಗಾಧರ ಭಟ್ಟರು. ಬರಿಗೈನಲ್ಲಿ ಮೈಸೂರಿಗೆ ಬಂದವರ ಪಾಲಿನ ಅಪಧ್ಭಾಂದವರಾಗಿ  ಅವರ ಶಿಕ್ಷಣ ಆರೋಗ್ಯ ಹಾಗೂ ಉದ್ಯೋಗಕ್ಕಾಗಿ   ತನ್ನ ಕೈ ಮೀರಿ  ಉದಾರತೆಯ ಹಸ್ತ ಚಾಚುತ್ತಿದ್ದ  ಅವರು ಎಲ್ಲರಿಗೂ ದಾರಿ ದೀಪವೇ  ಆಗಿದ್ದಾರೆ .ಅವರೋ ಸಂಸ್ಕೃತ ವಿದ್ವಾಂಸರು. ಆದರೆ ನಾನು  ಅವರದೇ ತಾಲೂಕಿನವನು ಎಂಬುದನ್ನು ಬಿಟ್ಟರೆ  ಅವರೊಡನೆಯ  ಸ್ನೇಹ ಹಸ್ತಕ್ಕೇ ಬೇಕಾದ ಅಹೃತೆಗಳೇನೂ ನನಲ್ಲಿ ಇರಲಿಲ್ಲ   ಆದರೂ  ಅವರ ಹಾಗೂ ನನ್ನ ಭಾಂದವ್ಯ     ಅಲ್ಲಿಂದ ಪ್ರಾರಂಭವಾಗಿ ಗಟ್ಟಿಯಾಗುತ್ತ ಕೊನೆಗೆ ವಿಧ್ವಾಂಸರಾದ ಅವರು  ನನಗೆ ಗಂಗಣ್ಣನಾದರು. ಆ ಭಾಂಧವ್ಯ ಇಂದೀಗೂ ಮುಂದುವರಿದೇ ಇದೆ .  ಆಗ ಅವರ ಮನೆ  ರಮಾವಿಲಾಸ ರಸ್ತೆಯ ಕೆಳಭಾಗದಲ್ಲಿ ಅಯ್ಯಂಗಾರ ಮೆಸ್ ನ ಹಿಂಬಾಗದಲ್ಲಿತ್ತು.   ಅಂದೇ ಅವರ ಮನೆಗೆ ನನ್ನ ಪ್ರವೇಶವಾಯಿತು.  ಆಗ ಅವರ ಜೊತೆಯಲ್ಲಿ ಸಹೋಧರಿ ಹೇಮಾವತಿ ಹಾಗೂ ರತ್ನಾವತಿಯವರಿದ್ದರು  . ಮನೆ ಅಗಲೇ  ದಾಸೋಹದ ಮಠದಂತಿತ್ತು  ಸದಾ ಅವರಲ್ಲಿಗೆ ಜ್ಞಾನ ಪಿಪಾಸುಘಳಗಿ ಬರುವ ವಿದ್ಯಾರ್ಥಿಗಳು ಹಾಗೂ ಪರದೇಶಿಗಳಾಗಿ ಮೈಸೂರಿಗೆ ಆಗಮಿಸಿರುವ ಶಿಕ್ಷಾಣಾರ್ಥಿಗಳಿಂದ   ಆಗಾಗ ಭರ್ತಿಯಾಗಿರುತ್ತಿತ್ತು ಪಾಪ ಆ ಸಹೋದರಿಯರು ಶಿಕ್ಷಣಕ್ಕಾಗಿ ಮನೆಯಿಂದ ಹೊರಹೋಗುವಾಗ   ಮಾಡಿಟ್ಟಿದ್ದ ಅಡುಗೆ ಅವರು ಹಸಿದು ಬರುವ ವೇಳಗೆ ಇವರ ದಾಸೋಹದ ಪರಿಯಿಂದಾಗಿ  ಖಾಲಿಯಾಗಿ ಅವರ ಪಾಲಿಗೆ ತೊಳೆಯಬೇಕಾಗಿದ್ದ ( ನಾನು  ಸೇರಿದಂತೆ ಮನೆಗೆ ಬಂದವರೆಲ್ಲ ಊಟಮಾಡಿಟ್ಟು ಹೋದ )   ತಟ್ಟೆ ಲೋಟ ಮಾತ್ರ ಕಂಗೋಳಿಸುತ್ತಿತ್ತು . ಅವರ ನಿತ್ಯ ದಾಸೋಹದ ಕತೆ ಇದಾಗಿತ್ತು. 

ಕ್ಷಣಾರ್ಧದಲ್ಲಿ  ಸಂಸ್ಕೃತಭಾಷೆಯನ್ನು ಆಂಗ್ಲ ಭಾಷೆಗೆ ತುರ್ಜಮೆ ಮಾಡಿ  ವಿವಿರಿಸುವ ಚಾಕಚಕ್ಯತೆ ಅಥವಾ ಬುಧ್ದಿಮತ್ತೆ ಗಂಗಾಧರ ಭಟ್ಟರಿಗೆ ಕರತಲಕವಾಗಿತ್ತು.  ಆಕೃತಿಯಲ್ಲಿ ವಾಮನನಂತಿದ್ದ ಅವರು ಪಾಠ ಪ್ರವಚನಕ್ಕೆ ನಿಂತರೆಂದರೆ ಆಘಾಧವಾದ ಪಾಂಡಿತ್ಯ ಅವರ ಬಾಯಿಂದ ಸರಾಗವಾಗಿ ಮಾರ್ಧನಿಸುತ್ತಿತ್ತು ಅವರ ಈ ಪ್ರಾವಿಣ್ಯತೆಯಿಂದಾಗಿ  ಅವರಲ್ಲಿಗೆ ವಿದೇಶಿಗರೂ ಸಹ  ಅಗಮಿಸಿ ಪಾಠ ಹೇಳಿಸಿಕೊಳ್ಳತ್ತಾರೆ, ಇಂದೀಗೂ ಅದು ಮುಂದುವರಿದಿದೆ.

ಆಡು ಮುಟ್ಟದ ಸೊಪ್ಪಿಲ್ಲ ಗಂಗಾಧರ ಭಟ್ಟರ ಅರಿವೆಗೆ ಬಾರ ವಿಷಯಗಳಿಲ್ಲ, ಎಂಬ ಪಾಂಡಿತ್ಯ ಅವರದ್ದು.  ಅವರಲ್ಲಿಗೆ ಸಂಸ್ಕೃತದ ವಿದ್ಯಾರ್ಥಿಗಳು ಮಾತ್ರವಲ್ಲ  ಆಯುರ್ವೇದ ವಿಜ್ಞಾನ ಕಲೆ ಸಾಹಿತ್ಯ ಸೇರಿದಂತೆ  ಎಲ್ಲ ಪ್ರಾಕಾರದ ವರೂ ಸಹ ಜ್ಞಾನಾರ್ಜನೆಗಾಗಿ ಆಗಮಿಸುತ್ತಿರುವದನ್ನು ನಾನು ಸ್ವತಃ ಕಣ್ಣಾರೆ  ಕಂಡಿದ್ಧೇನೆ. ಅವರ ಚರ್ಚೇಯಲಿ ್ಲ ó ಕೆಲವಮ್ಮೆ ನಾನು  ಮೂಕ ಪ್ರೇಕ್ಷಕನಾಗಿರುತ್ತಿದ್ದ ದಿನಗಳೂ ಇರುತ್ತಿದ್ದವು.

ಮೈಸೂರು ನಗರದ ವಿದ್ಯಾರ್ಥಿಗಳ ಪಾಲಿಗೆ ಎಲ್ಲಿಯೇ ಭಾಷಣ ಸ್ಪರ್ಧೇಗಳಾಗಲಿ  ಪ್ರಭಂಧ ಸ್ಪರ್ದೇಗಳಾಗಲಿ  ನಡೆಯುತ್ತದೆ ಎಂದಾದಲ್ಲಿ   ಅವರೆಲ್ಲ ಧಾಂಗುಡಿ ಇಡುವದು ಜ್ಞಾನ ಭಂಡಾರವಾದ ಈ ಭಟ್ಟರ ಮನೆಗೆ.  ಗಂಗಾಧರಭಟ್ಟರು ಹೇಳಿದ  ಬರೆಸಿದ ವಿಷಯಗಳ ಭಟ್ಟಿ ಇಳಿಸಿಕೊಂಡು ಹೋಗಿ   ಅಲ್ಲಿ ಆ ವಿಷಯಗಳನ್ನು ವಿವಿರಿಸಿ( ಮಂಡಿಸಿ )   ಚರ್ಛಾ ಸ್ಪರ್ಧೆಯ ಪರ ವಿರೋಧದ ಎರಡು ಪುರಸ್ಕಾರಗಳು ಇವರ ಶಿರ್ಷಯರಿಗೆ ಲಭ್ಯವಾದ ಉದಾಹರಣೆಗಳೂ ಸಹ ಸಾಕಷ್ಟಿವೆ . ಬಂದವರಿಗೆ ನಹೀ ಎಂದು ಹೇಳಿದವರಲ್ಲ ನಮ್ಮ ಗಂಗಣ್ಣ  

ಇಂಥಹ ಅಪರೂಪದ ಪ್ರೌಢಿಮೆ ನಮ್ಮ ಗಂಗಣ್ಣನದು  ಸಂಸ್ಕೃತ ವೇದ ಉಪನಿಷತ್ ಎಲ್ಲದರ ಬಗ್ಗೆ ಅಪಾರ ಪಾಂಡಿತ್ಯ ಗಳಿಸಿದ್ದ ನಮ್ಮ ಗಂಗಣ್ಣನಲ್ಲಿ ಈ ವಿಷಯಗಳ ಕುರಿತು ಯಾವುಧೇ ಪ್ರಶ್ನೆ ಎದುರಾದರೂ ಕ್ಷಣದಲ್ಲಿ ಅದರ ವಿವಿರ ಕಂಪ್ಯೂಟರನಂತೆ ಕರಾರುವಾಕ್ಕಾಗಿ ಲಭ್ಯವಿರುತ್ತಿತ್ತು . ಹಾಗಾಗಿಯೇ ಅವರ ಸ್ಣೇಹಿತರ ಬಳಗವೂ ಸದಾವೃದ್ದಿಯಾಗುತ್ತಲೇ ಇದೆ.  ಈ ಗಂಗಣ್ಣ ಎಂದೂ ಹೆಸರು ಹಾಗೂ ಧನದ ಹಿಂದೆ ಬಿದ್ದವರಲ್ಲ, ಹಾಗೇನಾದರೂ ಆಗಿದ್ದರೆ ಅವರು ಧಾರೆ ಎರೆದ  ವಿಷಯಗಳಿಂದಲೇ  ಲಕ್ಷಾಂತರ ರೂಗಳನ್ನು ಸಂಪಾಧಿಸಬಹುದಿತ್ತು . ಪ್ರತಿ ನಿತ್ಯವೂ ಕರಾರುವಾಕ್ಕಾಗಿ ಮನೆಗೆ ಬಂದವರಿಗೆ ಪಾಠ ಮಾಡುತ್ತಿದ್ದ ಅವರು ಯಾರಿಂದಲೂ ಬಿಡಿಗಾಸು ಸಹ ಪಡೆಯದೇ ಅಲ್ಲಿಂದ ಇಲ್ಲಿಯವರಿಗೂ ಉಚಿತವಾಗಿಯೇ ಜ್ಞಾನ ದಾಸೋಹ ದ ಔತಣ  ಬಡಿಸಿ ತೃಪ್ತಿ ಕಾಣುತ್ತಿದ್ದಾರೆ ,    ಇಂಥ ಜ್ಞಾನ ಭಂಡಾರದ ಗಂಗಣ್ಣ  ನಮ್ಮೊಡನೆ ಯಕ್ಷಗಾನದ ತಾಳ ಮದಲ್ಲೆಯ ಅರ್ಥವನ್ನೂ ಹೇಳಿದ್ದರು.  ಆ ಸಮಯದಲ್ಲಿ ಅವರ ಚಾಕಚಕ್ಯತೆ ಭಾಷಾ ಪೌಢಿಮೆ, ಸಮಯ ಸ್ಪೂರ್ತಿಯ ಜ್ಞಾನ ಧಾರೆಯ ಶುಧ್ದ ತರ್ಕದಿಂದ ಕೂಡಿದ  ವಾದವಿವಾದಗಳ ಆಸ್ವಾಧಕನಾದವರಲ್ಲಿ ನಾನೂ ಒಬ್ಬನಾಗಿದ್ಧೇನೆ . ಮನಸ್ಸು ಮಾಡಿದ್ದರೆ ಅವರು ಯಕ್ಷಗಾನ ತಾಳಮದ್ದಲೆಯಲ್ಲೂ ಹೆಸರು ಗಳಿಸಬಹುದಿತ್ತು ಅಂತಹ ಪಾಡಿತ್ಯ ಅವರದಾಗಿತ್ತು.  

ಇಂಥಹ ಜ್ಞಾನ ಭಂಡಾರದ ಘಣಿ ಗಂಗಣ್ಣ ಈಗ ನಿವೃತ್ತಿಯ ಅಂಚು ತಲುಪುತ್ತಿದ್ದಾರೆ ಎಂದು ನಂಬಲೇ ಅಸಾಧ್ಯವಾಗುವದರ ಜೊತೆಗೆ ನನ್ನ ವಯಸ್ಸು  ಸಹ ಜ್ಷಾಪಕವಾಗಲಾರಂಬಿಸಿದೆ. ಸದಾ ಚಟುಚಟಿಕೆಯಿಂದ ಇರುವ ನಮ್ಮ ಪ್ರೀತಿಯ ಗಂಗಣ್ಣ    ತಮ್ಮ ವಿದ್ವತ್ತಿನ ಪರಿಮಳ ಬೀರುತ್ತ  ಆರೋಗ್ಯವಾಗಿಯೇ  ನೂರು ವಂಸಂತಗಳನ್ನು  ಕಾಣುತ್ತ  ತನ್ನ ಜ್ಞಾನ ಭಂಢಾರದ ಪರಿಮಳವನ್ನು ಪಸರಿಸುತ್ತಿರಲಿ ಎಂಬುದೇ ನಮ್ಮಲ್ಲರ ಅಪೇಕ್ಷೆಯಾಗಿದೆ




