\chapter{गङ्गाधरं वीक्ष्य विदन्तु सर्वे}

\begin{center}
\Authorline{डा~। एच् वि नागराजराव्}
\smallskip

निवृत्तसंशोधकः\\ 
प्राच्यविद्यासंशोधनालयः\\  
मैसूरु
\addrule
\end{center}

\begin{verse}
धत्ते शोभां विबुधसदसि न्यायशास्त्रप्रवीणो\\
वादं कुर्वन् सकलहितदं भट्टगङ्गाधरोयम्~।\\
श्रावं श्रावं वचननिचयं तन्मुखाम्भोरुहोत्थं\\
सर्वश्छात्रो विमलधिषणो जायते पूर्णविद्यः~॥ १ ॥
\end{verse}

\begin{verse}
विख्यातगङाधरभट्टवाचाम् आचान्तनिर्दोषसुधारसानाम्~।\\
माधुर्यमासाद्य मनोस्मदीयं न मन्यतेऽन्यन्मधुरं जगत्याम्~॥ २ ॥
\end{verse}

\begin{verse}
बौद्धे च जैने पथि वेदशीर्षे चेत्संशयस्ते व्रज संश्रय त्वम्~।\\
दयालुगङाधरभट्टपादाविति ब्रुवन्त्याशु बुधाः स्वशिष्यान्~॥ ३ ॥
\end{verse}

\begin{verse}
यस्मिन् सुविद्या विनयो न तस्मिन् वैदुष्यवान्नैव दयापरोऽस्ति~।\\
इति प्रसिद्धोक्तिरियं मृषेति गङाधरं वीक्ष्य विदन्तु सर्वे~॥ ४ ॥
\end{verse}

\begin{verse}
श्रीनागराजं कविमान्यमान्यं देशे विदेशे प्रचुरप्रसिद्धम्~।\\
गङ्गाधरं यः स्तुतिभिः तुतोष वन्दे सुशीलं सुविनीतविद्यम्~॥
\end{verse}

~\hfill\textbf{(सम्पादकः)}

\articleend
