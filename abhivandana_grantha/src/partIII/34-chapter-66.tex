{\fontsize{14}{16}\selectfont
\chapter{ಗುರುಪ್ರಸಾದ}

\begin{center}
\Authorline{ವಿ~॥  ಗುರುಪ್ರಸಾದ}
\smallskip

ಸಹಾಯಕ ಪ್ರಾಧ್ಯಾಪಕರು, ಸಂಸ್ಕೃತ ವಿಭಾಗ\\
ಎಮ್.ಐ.ಟಿ. ಫಸ್ಟ್ ಗ್ರೇಡ್ ಕಾಲೇಜ್\\
ಮೈಸೂರು
\addrule
\end{center}

 ನಮ್ಮ ಹುಟ್ಟೂರಲ್ಲಿ ಶ್ರೀಯುತರಾದ ಗಂಗಾಧರ ಭಟ್ಟರ ಮನೆಗೂ ನಮ್ಮ ಮನೆಗೂ ಐದಾರು ಕಿಲೋಮೀಟರ ದೂರವಿರಬಹುದು. ನಮ್ಮಿಬ್ಬರ ಕುಟುಂಬದ ನಂಟು ಬಹಳ ಹಳೆಯದು. ಬಹುಶಃ ಆ  ವಿಷಯವನ್ನು ಹೇಳಲು ನನಗಿರುವ ಮಾಹಿತಿಯ\break ಆಧಾರದ ಮೇಲೆ ಹೇಳುವುದಾದರೆ ೧೯೪೦ ರ ದಶಕಕ್ಕೆ ಹೋಗಬೇಕಾಗುತ್ತದೆ. ಅದು ನನ್ನ ತಂದೆಯ ಮತ್ತು ಗಂಗಾಧರ ಭಟ್ಟರ ದೊಡ್ಡಪ್ಪ \enginline{-} ಮಹಾ\-ಬಲೇಶ್ವರ ಭಟ್ಟರು, ಅವರ ಕಾಲಕ್ಕೆ ಸಂಬಂಧಿಸಿದ ವಿಷಯ. ಅವರ ದೊಡ್ಡಪ್ಪನಿಗೆ ನನ್ನ ತಂದೆಯಲ್ಲಿ ವಿಶ್ವಾಸ. ಅದೂ ಸಹ ಪೂರ್ವಜನ್ಮದ ಸಂಬಂಧವಲ್ಲದಿದ್ದರೆ ಈ ಬಂಧ ಸಾಧ್ಯವೇ ಇಲ್ಲ. ಮಹಾ\-ಬಲೇಶ್ವರ ಭಟ್ಟರು ಆಗಿನ ಕಾಲದಲ್ಲಿ ಬಹಳ ಪ್ರಕಾಂಡ ಪಂಡಿತರು. ನನ್ನ ತಂದೆಗೆ ಹೃದಯ, ಬಾಯಿ, ಕೈ ಈ ಮೂರರ ಶುದ್ಧಿಯ ಹೊರತು ಇನ್ನಾವ ಅರ್ಹತೆಯೂ ಇರಲಿಲ್ಲ !? ಒಂದು ದೃಷ್ಟಿಯಿಂದ ಅವರೆದುರು ಇವರು ನಗಣ್ಯ. ಆದರೂ ನನ್ನ ತಂದೆ ಅವರ ಮನೆಯಲ್ಲೇ ಅವರ ಔದಾರ್ಯದ ಕಾರಣದಿಂದ ಉಳಿದಿದ್ದುದು ಇತಿಹಾಸ. ಈಗ ಅದರ\break ವಿವರಣೆ ನನ್ನ ಎಣಿಕೆಗೆ ನಿಲುಕದ್ದು \enginline{-} ಪ್ರಕೃತ ಅದು ಅಪ್ರಕೃತ ಕೂಡ. ಅವರ ಮನೆ ಮತ್ತು ನಮ್ಮ ಮನೆಯ ನಂಟು ಅಷ್ಟು ಪುರಾತನವಾದುದು ಎಂಬುದಷ್ಟೇ ಆಶಯ. ಆದರೆ ಆ\break ನಂಟಿನ ಅಂಟು ಇಂದೂ ಉಂಟು. ಅದು ಮಹಾಬಲೇಶ್ವರ ಭಟ್ಟರು ಕೊಟ್ಟ ಆಶ್ರಯ\-ದಿಂದ ಆರಂಭವಾಗಿ ವಿಘ್ನೇಶ್ವರ ಭಟ್ಟರಿಂದ ಬೆಳೆದು ಗಂಗಾಧರ ಭಟ್ಟರ\break ತನಕವೂ ಮುಂದುವರೆದು ಬಂದಿದೆ. ಹಾಗಾಗಿ ಆ ಅಂಟು\enginline{-}ನಂಟಿನ ಪ್ರಯೋಜನ ನಮ್ಮ ಕುಟುಂಬಕ್ಕೆ ವಿಪುಲವಾಗಿ ಆಗಿದೆಯೆಂಬುದು ದಿಟ. ಹಾಗಾಗಿ ಅವರ ಕುಟುಂಬದ ಋಣಭಾರ ನಾವು ಹೊರಲಾರದಷ್ಟು ನಮ್ಮ ಕುಟುಂಬದ ಮೇಲಿದೆ.

ನಮ್ಮ ಬಾಲ್ಯದ ಘಟನೆಗಳ ನೆನಪು ಸಾಮಾನ್ಯವಾಗಿ ಯಾವ ವಯಸ್ಸಿನಿಂದ ಸ್ಮೃತಿಯಲ್ಲಿ ಉಳಿಯಬಹುದೋ ಅಷ್ಟು ಚಿಕ್ಕ ವಯಸ್ಸಿನಿಂದ ನಾನು ಗಂಗಾಧರ ಭಟ್ಟರನ್ನು ನೋಡುತ್ತ ಬಂದಿದ್ದೇನೆ. ಅವರ ಮನೆಯಲ್ಲೇ ಸಾಕಷ್ಟು ಸಮಯ ವಾಸ ಮಾಡಿದ್ದೇನೆ. ಅವರ ಮನೆ ನನಗೆ ಇನ್ನೊಂದು ಮನೇಯೇ ಆಗಿತ್ತು ಎಂದರೆ ತಪ್ಪಿಲ್ಲ. ಬಹುಶಃ ಆ ಸಹವಾಸ ಮೈಸೂರಿನವರೆಗೂ ನನ್ನನ್ನು ಕರೆತಂದಿತು. ಹೌದು, ನಿಜ ! ಅವರು ಮೈಸೂರಿನಲ್ಲಿ ಇದ್ದುದರಿಂದ ನಾನೂ ಮೈಸೂರಿಗೆ ಬಂದೆ. ಅದಿಲ್ಲದಿದ್ದರೆ,  ಅವರು ಇನ್ನೆಲ್ಲಿರುತ್ತಿದ್ದರೋ ಅಲ್ಲಿಗೇ ಹೋಗಿರುತ್ತಿದ್ದೆ.

ಗಂಗಾಧರ ಭಟ್ಟರಲ್ಲಿ ನನಗೆ ಅತಿಯಾದ ಪ್ರೀತಿ, ಅತಿಯಾದ ಸಲುಗೆ. ಅವರ ತಂದೆ ಅಣ್ಣಂದಿರಲ್ಲೂ ಅದೇ ಸಲುಗೆಯೇ ನನಗುಂಟು. ಆದರೆ ಅವರನ್ನೆಲ್ಲ ಬಹುವಚನದಲ್ಲಿ ವ್ಯವಹರಿಸುವುದು ರೂಢಿ, ಆದರೆ ಇವರನ್ನು ಯಾವತ್ತೂ ಬಹುವಚನದಲ್ಲಿ ಕರೆದಿದ್ದಿಲ್ಲ. ಅದಕ್ಕೆ ನಮ್ಮಿಬ್ಬರ ಸ್ವಭಾವವೂ ಕಾರಣವಿರಬಹುದು. ಏನೋ ! ಬಹುವಚನ ಪ್ರಯೋಗ ಅದೇಕೋ ನನಗೆ ಪರಸ್ಪರರ ಭಾವದಲ್ಲಿ ಅಂತರವನ್ನು ಧ್ವನಿಸುತ್ತದೆ. ಹಾಗೆಂದು ಈ ರೀತಿಯ ಚಿಂತನೆಗಳೆಲ್ಲ ಮೊಳೆಯುವ ಮೊದಲೇ \enginline{-} ಅಷ್ಟು ಚಿಕ್ಕಂದಿನಿಂದ ಆ ವ್ಯಾಪಾರ ನಡೆಯುತ್ತ ಬಂದಿದೆ, ಅಂದರೆ ಅದಕ್ಕೆ ಇನ್ನೂ ಆಳವಾದ ಕೊಂಡಿಯೇನಾದರು ಇರಲೂ ಸಾಕು. ಹಾಗೆ ನೊಡಿದರೆ ಅವರ ಮತ್ತು ನಮ್ಮ ಮನೆಯ ಸಂಬಂಧವನ್ನೇ ನೋಡಿದರೂ ಅದು ಕೇವಲ ಈ ಜನ್ಮದ ಸಂಬಂಧ ಮಾತ್ರವೆಂದು ಅನ್ನಿಸುವುದಿಲ್ಲ. ಇಂತಹ ಭಾವನೆಯನ್ನು  ಪುಷ್ಟೀಕರಿಸುವ  ಮಹಾಕವಿ ಕಾಳಿದಾಸನ ಮಾತೊಂದು ಈ ಸಂದರ್ಭಕ್ಕೆ \hbox{ಸ್ಮೃತಿಗೆ} ಬರುತ್ತಿದೆ \enginline{-} ಈಗಿನ ನಮ್ಮ ಬದುಕು\enginline{-}ಸಂಬಂಧ\enginline{-}ಭಾವವನೆಗಳಿಗೆ ಜನ್ಮ\enginline{-}ಜನ್ಮಾಂತರದ ಸಂಬಂಧ \enginline{-} ಸಂಸ್ಕಾರಗಳು ಕಾರಣವಿರಬಹುದು, ಪ್ರಕೃತ ಕಾಲದಲ್ಲಿ ಅವುಗಳಿಗೆ \break ಸಂಬಂಧಿಸಿದ ಘಟನೆ, ಸನ್ನಿವೇಶ  ಇತ್ಯಾದಿ  ಘಟಿಸಿದಾಗ ಸುಪ್ತವಾಗಿದ್ದ ಸಂಸ್ಕಾರದಿಂದ ಸ್ಮೃತಿ  ಪ್ರಬೋಧವಾಗಿಬಿಡುತ್ತದೆ  ಎಂಬುದನ್ನು ತನ್ನ ಅಭಿಜ್ಞಾನಶಾಕುಂತಲ ನಾಟಕದಲ್ಲಿ ಬಹು ರಮ್ಯವಾದ ಶ್ಲೋಕದ ಮೂಲಕ ಸಾರುತ್ತಾನೆ \enginline{-} 

\begin{verse}
	ರಮ್ಯಾಣಿ ವೀಕ್ಷ್ಯ ಮಧುರಾಂಶ್ಚ ನಿಶಮ್ಯ ಶಬ್ದಾನ್ \\
	ಪರ್ಯುತ್ಸುಖೀ ಭವತಿ ಯತ್ಸುಖಿತೋಪಿ ಜಂತುಃ ~।\\
	ತಚ್ಚೇತಸಾ ಸ್ಮರತಿ ನೂನಮಬೋಧಪೂರ್ವಂ\\
	ಭಾವಸ್ಥಿರಾಣಿ ಜನನಾಂತರಸೌಹೃದಾನಿ ~॥
\end{verse}
	\smallskip
	ಶಾಲೆಯ ರಜೆಯ ಅವಧಿಯಲ್ಲಿ ಹುಡುಗರು ಅಜ್ಜನ ಮನೆಗೆ ಹೋಗುವುದುಂಟಷ್ಟೆ. ಆದರೆ ನನಗೆ ಅದು ಅಗತ್ಯವಿರಲಿಲ್ಲ. ಕಾರಣ ನನ್ನಜ್ಜನ ಮನೆ ನಮ್ಮ ಮನೆಯಿಂದ,\break ಎಡವಿ ಬಿದ್ದರೆ ಮೂರೇ ಮಾರು. ಹಾಗಾಗಿ ರಜೆ ಬಂತೆಂದರೆ ಮಣ್ಣೀಕೊಪ್ಪಕ್ಕೆ ಹೋಗುವುದು ರೂಢಿ. ಆಗ ಅಲ್ಲಿ  ಶ್ರೀ ವಿಘ್ನೇಶ್ವರ ಭಟ್ಟರು, ಅವರ ಧರ್ಮಪತ್ನೀ ಶ್ರೀಮತಿ ರೇವತಮ್ಮ, ಅವರ ಹಿರಿಯ ಸುಪುತ್ರ  ಶ್ರೀ ಮಂಜುನಾಥ ಭಟ್ಟರು, ಸುಪುತ್ರಿಯರಾದ ಲೀಲಕ್ಕ, ಹೇಮಕ್ಕ  ಮತ್ತು ರತ್ನಕ್ಕ ಇವರು ವಾಸವಾಗಿದ್ದರು.  ವಿಘ್ನೇಶ್ವರ ಭಟ್ಟರ ದ್ವಿತೀಯ ಪುತ್ರ  ಶ್ರೀ ಶ್ರೀಧರ ಭಟ್ಟರು ಸಿದ್ದಾಪುರದಲ್ಲಿದ್ದರು, ತೃತೀಯ ಪುತ್ರ ಗಂಗಣ್ಣ ಮಾತ್ರ ಮೈಸೂರಿನಲ್ಲಿ ಇದ್ದ. ಅಂತೂ ಈ ಬಗೆಯ ತುಂಬಿದ ವೈದಿಕ ಕುಟುಂಬ. 
	
	ಮನೆಯಲ್ಲಿ ಆರ್ಥಿಕವಾಗಿ ಏನೂ ಅನುಕೂಲತೆ ಇದ್ದ ಕಾಲವಲ್ಲ ಅದು. ಆದರೆ ಅವರ ಔದಾರ್ಯದ ಕಾರಣ ಹೆಚ್ಚಿನವರಿಗೆ ಅದರ ಅರಿವೂ ಅಗುತ್ತಿರಲಿಲ್ಲ. ಅವರ ಮನೆಯಲ್ಲಿ  ಹೊರಗಿನವರು,  ದೂರದ ಊರಿನವರು ಬಂದುಳಿಯುವ ವಿಷಯ\break  ಹಾಗಿರಲಿ, ಅಲ್ಲೇ ಪಕ್ಕದ ಊರಿನವರೂ  ಸಹ ಇವರ ಮನೆಗೆ ಬಂದು  ಒಂದೆರಡು ದಿನ ಉಳಿದುಕೊಳ್ಳುತ್ತಿದ್ದುದೂ ಉಂಟು. ಹಾಗಾಗಿ ಆ ಮನೆ  ಎಷ್ಟೋ ಬಾರಿ ಸಾಕ್ಷಾತ್ ಧರ್ಮಛತ್ರದಂತೆ ಕಂಗೊಳಿಸುತ್ತಿತ್ತೆಂದರೆ,  ಅದೇನೂ ಆಲಂಕಾರಿಕ ಮಾತಾಗುವುದಿಲ್ಲ. ಇಂತಹ  ಮನೆಯ ವಾತಾವರಣದಲ್ಲಿ ನಾನೂ, ನನ್ನಂತಹ ಇನ್ನೂ ಕೆಲವು ಮಕ್ಕಳು ರಜಾ ಕಾಲದಲ್ಲಿ  ಸೇರುತ್ತಿದ್ದೆವು, ನಮ್ಮ ನಮ್ಮ ಮನೆಗಿಂತಲೂ ಹೆಚ್ಚು ಸಲುಗೆಯಿಂದ ಅಲ್ಲಿ ನಾವು ಕಾಲ ಕಳೆಯುತ್ತಿದ್ದೆವೆಂಬುದು ವಾಸ್ತವ.
	 
ಅವರ ಮನೆಯಲ್ಲಿ ಗಂಗಣ್ಣನ ಅಮ್ಮ \enginline{-} ರೇವತಮ್ಮನವರು ಉಬ್ಬು ರೊಟ್ಟಿ ಮಾಡುವುದರಲ್ಲಿ ಎತ್ತಿದ ಕೈ. ಮಕ್ಕಳೆಂದರೆ ಅವರಿಗೆ ಅಷ್ಟೇ ಅಕ್ಕರೆ. ಹಾಗಾಗಿ ರುಚಿಯಾದ ತಿಂಡಿ/ಅಡುಗೆಯಾಗುತ್ತಿತ್ತು, ಸಮೃದ್ಧವಾಗಿ ತಿಂದುಂಡು, ಮನೆಯವರಿಗೆ  ಅಷ್ಟೋ ಇಷ್ಟೋ ಕೆಲವು ಕೆಲಸಗಳಲ್ಲಿ ಸಹಾಯ ಮಾಡುತ್ತ, ನಮ್ಮ ಬಾಲ್ಯದ ಎಲ್ಲ ರೀತಿಯ ಚೇಷ್ಟೆಗಳಿಗೂ ಅವರ ಮನೆ ಆಶ್ರಯವಾಗಿತ್ತು. 
ಅವರ  ಮನೆಯ ಹಸುಗಳು ನಮಗಿಂತ ತುಂಟವಾಗಿದ್ದವು. ಸೈನಿಕರು ಭಾರತದ ಗಡಿ ಕಾಯುವಂತೆ ಅಷ್ಟದಿಕ್ಕುಗಳಲ್ಲಿಯೂ ನಿಂತು ಅವುಗಳನ್ನು  ಕಾಯಬೇಕಿತ್ತು. ನಾವೆಲ್ಲ ಹುಡುಗರು ನಮಗಿಂತ ಹಿರಿಯರಾಗಿದ್ದ ಗಂಗಣ್ಣನ ತಂಗಿಯರು \enginline{-} ಲೀಲಕ್ಕ, ಹೇಮಕ್ಕ ಮತ್ತು ರತ್ನಕ್ಕ \enginline{-} ಇವರ ಸುಪರ್ದಿಯಲ್ಲಿ ಗೋಪಾಲ ಚೇಷ್ಟೆಯನ್ನು ಮಾಡುತ್ತಿದ್ದೆವು. 

ಈ ಮಧ್ಯದಲ್ಲಿ  ಶ್ರೀ ಮಂಜುನಾಥ ಭಟ್ಟರು ಸಮಯವಿದ್ದಾಗಲೆಲ್ಲ ವೇದಪಾಠ ಮಾಡುತ್ತಿದ್ದರು. ಶ್ರೀ ವಿಘ್ನೇಶ್ವರ ಭಟ್ಟರು, ಶ್ರೀ ಮಂಜುನಾಥ ಭಟ್ಟರು ತಾವು ಪೌರೋಹಿತ್ಯಕ್ಕೆ ಹೋಗುವಾಗ, ನಮ್ಮಲ್ಲಿ ವಾಡಿಕೆಯಿರುವ ಪಡಚಾಕರಿಗೆ,  ಸಹಸ್ರನಾಮಾದಿಗಳ ಪಾರಾಯಣಕ್ಕೆ  ಕರೆದುಕೊಂಡು ಹೋಗುತ್ತಿದ್ದರು. ಈ ಸಂದರ್ಭದಲ್ಲಿ ಹೇಳಲೇ ಬೇಕಾದ ಇನ್ನೊಂದು ಅಂಶವಿದೆ \enginline{-} ಶ್ರೀ ವಿಘ್ನೇಶ್ವರ ಭಟ್ಟರು ಅನೇಕ ಶಿಷ್ಯವರ್ಗದವರ ಮನೆಗೆ ವೈದಿಕ ಕಾರ್ಯಗಳನ್ನು ಮಾಡಿಸಲು ಹೋಗುತ್ತಿದ್ದರು. ಆದರೆ ಇವರು ಹೋಗುವ ಅನೇಕ ಮನೆಗಳಲ್ಲಿ ಯಾವ ಅನುಕೂಲವೂ  ಇಲ್ಲದ ಬಡತನ. ಹಾಗಾಗಿ ಇವರ ಮನೆಲ್ಲಿಯೇ ಇದ್ದ ಪದಾರ್ಥಗಳನ್ನು  ಅವರ ಮನೆಗೆ  ತೆಗೆದುಕೊಂಡು ಹೋಗಿ ಅವರ ಮನೆಯಲ್ಲಿ ವೈದಿಕ ಕಾರ್ಯವನ್ನು ನಡೆಸಿಕೊಡುತ್ತಿದ್ದರು. ಅವರ ಈ ಔದಾರ್ಯ ಮಾತ್ರ \break ಅವಿಸ್ಮರಣೀಯವಾದುದು. (ಗಂಗಣ್ಣನಲ್ಲಿ ಅನೇಕ ಲೇಖಕರು ಹೇಳಿರುವ, ಕಂಡಿರುವ ಔದಾರ್ಯಕ್ಕೆ ಇಲ್ಲಿಯೇ ಅದರ ಬೀಜವಿರುವುದನ್ನು  ಗುರುತಿಸಬಹುದು) ಇಂತಹ\break ಸಂದರ್ಭಗಳಲ್ಲಿ  ನನ್ನನ್ನು ಅವರು ಆ ಕೆಲಸ ಕಾರ್ಯಗಳಲ್ಲಿ ತೊಡಗಿಸುತ್ತಿದ್ದರು. ಹಾಗಾಗಿ ಎಷ್ಟೋ ವೈದಿಕ ಸಂಸ್ಕಾರ ಬಾಲ್ಯದಲ್ಲೇ ಉಂಟಾಗುವಂತಾಯಿತು.  

ಅವರ ಮನೆಯಲ್ಲಿ ನಾವಿರುವ ರಜಾ ಕಾಲದಲ್ಲಿಯೇ ಮೈಸೂರಿನಿಂದ ಗಂಗಣ್ಣ ಸಹ ಮನೆಗೆ ಬರುತ್ತಿದ್ದ. ಅವನಲ್ಲಿ ನಮಗೆ ಬಹಳ ಆಕರ್ಷಣೆ. ಆಬಾಲ ವೃದ್ಧರೊಂದಿಗೆ\break  ಅವರವರಿಗೆ ಅನುಗುಣವಾಗಿ ನಡೆದುಕೊಳ್ಳುವ, ಹೊಂದಿಕೊಳ್ಳುವ ಅಸಾಧಾರಣ ಗುಣ ಅವನಲ್ಲಿದೆ. ಅವನು ಪಂಡಿತರಿಗೂ ಪಾಮರರಿಗೂ ಸಮಾನವಾಗಿ ಸಲ್ಲುವ ವ್ಯಕ್ತಿ. ಇದು ಅವನ ಜೊತೆಗಿನ ನಮ್ಮ ಸಲುಗೆಗೆ ಕಾರಣ. ಅವನಿಗೆ ಶಾಸ್ತ್ರಕೃಷಿಯಲ್ಲಿರುವ ಆಸಕ್ತಿ ಭೂ\- ಕೃಷಿಯಲ್ಲೂ ಇತ್ತು. ಮೈಸೂರಿನಲ್ಲಿ ಅಧ್ಯಯನ ಅಧ್ಯಾಪನಗಳಲ್ಲಿ ಹೇಗೆ ತೊಡಗಿಕೊಂಡಿದ್ದನೋ ಅದೇ ರೀತಿ ಮನೆಗೆ ಬಂದಾಗ ಕೃಷಿಯಲ್ಲಿ ತೊಡಗಿಕೊಳ್ಳುವುದು ಅವನ ಸ್ವಭಾವ. ಈ ಸಂದರ್ಭಗಳಲ್ಲಿ ನಾವು ಅವನ ಹಿಂದೆ ಮುಂದೆ ಓಡಾಡುತ್ತಿದ್ದೆವು. ಇನ್ನು, ಅವನು ಊರಿಗೆ ಬಂದಾಗ ನಿರ್ದಿಷ್ಟವಾದ ಮನೆಗಳಿಗೆ ಹೋಗುವ ರೂಢಿಯಿತ್ತು. ಅವನ ತಂಗಿ ವೇದಾ(ದ)ವತಿಯ ಮನೆಗೆ ಹೋಗಿ ಅಲ್ಲಿ ಒಂದು ರಾತ್ರಿ ಉಳಿದು ಮಾರನೇ ದಿನ ನಮ್ಮ ಮನೆಗೆ ಬಂದು, ಅಂದು ರಾತ್ರಿ ನಮ್ಮಲ್ಲಿ ತಂಗಿ ಅಲ್ಲಿಂದ ಮನೆಗೆ ಹೋಗುವುದು \enginline{-} ಇದು ಅವನ ವಾರ್ಷಿಕ ಹೈಸಾಲು.

ಕಾಲಕ್ರಮದಲ್ಲಿ ನಾನು ಎಸ್.ಎಸ್.ಎಲ್.ಸಿ ಮುಗಿಸಿ ಮನೆಯಲ್ಲೇ ಇದ್ದೆ. ತೋಟದ ಕೃಷಿ ಮತ್ತು ಹಸು ಮೇಯಿಸುವುದು, ಹಾಲು ಕರೆದು ಊರಿಗೆಲ್ಲ \hbox{ಕೊಡುವುದು} ನನ್ನ ನಿತ್ಯ  ಕೆಲಸ. ಅದನ್ನು  ಬಹುಶಃ ಚೆನ್ನಾಗಿಯೇ ನಿರ್ವಹಿಸುತ್ತಿದ್ದೆ. ನಾನು \hbox{ಇನ್ನೇನಕ್ಕೂ} ಯೋಗ್ಯನಲ್ಲವೆಂದು ನಮ್ಮ ಮನೆಯ ಯಜಮಾನರ  ಸ್ಪಷ್ಟ ತೀರ್ಮಾನ\-ವಾಗಿತ್ತು. ಓದು ಮತ್ತು ಕೃಷಿ ಈ ಎರಡು ಕ್ಷೇತ್ರದ ಆದ್ಯತೆಯ ವಿಷಯದಲ್ಲಿ ವ್ಯತಿರಿಕ್ತ  ನಿಲುವು ನಮ್ಮಲ್ಲಿ ವಾದದ\enginline{-}ವಿವಾದದ ವಿಷಯವಾಗಿತ್ತು. ಬಂಧುವರ್ಗದ,  ತಂದೆ \enginline{-} ತಾಯಿಯರೆಲ್ಲರ ಅಭಿಪ್ರಾಯ ಪಾಠಶಾಲೆ\-ಗಾದರೂ ನನ್ನನು \hbox{ಕಳುಹಿಸಬಹುದಿತ್ತು} ಎಂಬ \hbox{ನಿಲುವಾಗಿತ್ತು.} \textit{ನಮ್ಮಲ್ಲೆಲ್ಲ ಇರುವ ಅಭಿಪ್ರಾಯವೇ ಹಾಗೆ \enginline{-} ಏನಕ್ಕೂ \hbox{ಯೋಗ್ಯನಲ್ಲದವ} ಪಾಠಶಾಲೆಗಾದರೂ \hbox{ಹೋಗಬಹುದಲ್ಲ!} ಎಂಬುದು. ಸಂಸ್ಕೃತ ಕ್ಷೇತ್ರಕ್ಕೆ ಬರುವ ಹೆಚ್ಚಿನವರು ಅವರ ಮನೆಗಳಲ್ಲಿ ಇಂತಹ ಅಭಿಪ್ರಾಯ ಹೊಂದಿರುವವರೇ ಹೆಚ್ಚು. ಅದರೆ ಮನೆಯಿಂದ ಬಿಡಿಗಾಸನ್ನೂ ಪಡೆಯದೇ ಇಲ್ಲಿ ಬಂದು ವೇದ, ಸಂಸ್ಕೃತ, ಲೌಕಿಕ ವಿಷಯಗಳನ್ನು  ಓದುತ್ತ ವಾರಾನ್ನವನ್ನೋ, ಹೋಟೆಲ್‌ನ್ನೋ ಆಹಾರಕ್ಕಾಗಿ\break ಅವಲಂಬಿಸಿ ಪೌರೋಹಿತ್ಯ ಮಾಡಿಕೊಂಡು ಅಳಿದುಳಿದ ಹಣವನ್ನು ಊರಿನಲ್ಲಿರುವ ತಂದೆತಾಯಿಗಳಿಗೆ ಕಳುಹಿಸಿ, ಇಲ್ಲಿಂದಲೇ ಅವರನ್ನು ಸಾಕುವವರೂ ನಮ್ಮಲ್ಲಿ ಅನೇಕರಿದ್ದಾರೆ ಎಂದರೆ ಸಂಸ್ಕೃತ ಕ್ಷೇತ್ರಕ್ಕೆ ಹೊರತಾದವರು ನಂಬಲಾರರು. ನಮ್ಮ ಸಂಸ್ಕೃತ\enginline{-}ಸಂಸ್ಕೃತಿಯ ಪ್ರಕೃತ ಪರಿಸ್ಥಿತಿಗೆ ಇಂಥವುಗಳಲ್ಲೂ ಕಾರಣವನ್ನು ಅನ್ವೇಶಿಸಬೇಕಾದ\break ಜವಾಬ್ದಾರಿ ಅದಕ್ಕೆ ಸಂಬಂಧಿಸಿದವರಿಗುಂಟು}.  ಪ್ರಕೃತಕ್ಕೆ ಬರುವುದಾದರೆ, ನಮ್ಮ ಯಜಮಾನರಿಗೆ ಮಾತ್ರ ಪಾಠಶಾಲೆಗೆ ಹೋಗುವುದಕ್ಕೂ ನಾನು ಅಯೋಗ್ಯ ಎಂದಿತ್ತು. ಅಷ್ಟು ಹೊತ್ತಿಗಾಗಲೇ ನಾಲ್ಕು ವರ್ಷ ಕೃಷಿಯಲ್ಲಿ ಕಳೆಯಿತು. ನನಗೆ ನಾನೇ ಕಳೆದುಹೋಗುವ\break ಆತಂಕವಿತ್ತು. ಮೊದಲಿನಿಂದಲೂ ಸಂಸ್ಕೃತದ ವಾಸನೆಯೇನೋ ಇತ್ತು. ಗಂಗಣ್ಣ\break ಮೈಸೂರಿನಲ್ಲಿ ಇರುವುದರಿಂದ ನನಗೂ ಪ್ರತಿದಿನ  ಮೈಸೂರಿನ  ಕನಸು ಬೀಳುತ್ತಿತ್ತು. ಅನೇಕ ವೇಳೆ ಮನೆಯಿಂದ ಓಡಿಬರುವ ಪ್ರಯತ್ನವನ್ನು ಮಾಡಿಯೂ ತಕ್ಕ ಫಲ ಗಿಟ್ಟಿಸಿ\-ಕೊಳ್ಳುವಷ್ಟು ಯಜಮಾನರ ಬಲದ ಮುಂದೆ ನನ್ನ ಬಲ ಸಾಕಾಗುತ್ತಿರಲಿಲ್ಲ.\break ರಾಮಾಯಣ ಕಾಲದಲ್ಲಿದ್ದ ವಿಶೇಷ ವರವುಳ್ಳ ವಾಲಿಯಂತೆ ಯಜಮಾನರು ನನ್ನ \hbox{ಬಲವನ್ನು }ಉಡುಗಿಸಿಬಿಡುತ್ತಿದ್ದರು. ಕಾಲ ಕೂಡಿಬರದಿರುವಾಗ ಇವೆಲ್ಲ ಹೀಗೆಯೇ. ಏಕೆಂದರೆ, ದೈವಾನುಗ್ರಹದಿಂದ ಮುಂದೆ ನಡೆಯಬೇಕಾದುದು ಅದಾಗಿಯೇ ನಡೆಯಿತು, \enginline{-} ಹಿಂದಿನ ನಮ್ಮ ನಿಲುವಿನ ಪ್ರಾಮಾಣ್ಯವನ್ನು ಮುಂದೆ ಘಟಿಸುವ ಘಟನೆಗಳು ಪರೀಕ್ಷಿಸಿ ಹಾಲನ್ನು ಹಾಲು ನೀರನ್ನು ನೀರು ಎಂದುಬಿಡುತ್ತವೆ. ಅಲ್ಲಿಯ ತನಕ ಕಾಯುವಷ್ಟು ತಾಳ್ಮೆ, ತಿಳುವಳಿಕೆ ನಮಗಿರಬೇಕಷ್ಟೆ.

ಹೀಗಿರಲು ಒಮ್ಮೆ ಗಂಗಣ್ಣ ನಮ್ಮ ಮನೆಗೆ ದಯಮಾಡಿಸಿದ. ನಾನು\break ನಮ್ಮೆಜಮಾನರಿಗೆ ತಿಳಿಯದಂತೆ ಗಂಗಣ್ಣನ ಕಿವಿಯಲ್ಲಿ ನನಗೆ ಬೀಳುತ್ತಿರುವ ಕನಸನ್ನು ಉಸುರಿದೆ. ಅವನ ವ್ಯವಹಾರ ಚಾತುರ್ಯ ಯಾರಿಗೂ ವಿರೋಧವಾಗದಂತೆ ಸಮಸ್ಯೆಯನ್ನು ನಿರ್ವಾಹಮಾಡುತ್ತಿತ್ತು.\\ 
ನಮ್ಮ ಮನೆಯವರೆಲ್ಲರೂ ಇರುವಾಗ ಅವನು ನನ್ನಲ್ಲಿ ಒಂದು ಪ್ರಶ್ನೆಯನ್ನು ಕೇಳಿದ\enginline{-} 

ನಿನ್ನ ಗುರು ಯಾರು....??\\
ನಾನು ತಡಬಡಾಯಿಸಿ ಮೇಲೆ ಕೆಳಗೆ ನೋಡಿದೆ. ಪ್ರಾಥಮಿಕ ಶಿಕ್ಷಣವನ್ನು ಪಡೆದ\break ಕಾಲದಿಂದ ಹಿಡಿದು ಮಾಧ್ಯಮಿಕ ಶಿಕ್ಷಣದ ತನಕ \hbox{ಕಲಿಸಿದವರೆಲ್ಲರೂ} ಆಗಿನ ತಿಳುವಳಿಕೆಯಲ್ಲಿ ಗುರುಗಳೇ ಆಗಿದ್ದರು. ಆದರೆ ಅವರಾರನ್ನೂ ನನ್ನ ಮನಸ್ಸು ಈ "ಧ್ವನಿಯುಕ್ತ ಪ್ರಶ್ನೆಗೆ" ಉತ್ತರವಾಗಿ ಗ್ರಹಿಸಿರಲಿಲ್ಲ ಯಾ ಸ್ವೀಕರಿಸಿರಲ್ಲಿಲ್ಲ \enginline{-} ಇದೇ ನಿಜ. ನನ್ನ ಸ್ಥಿತಿಯನ್ನು  ಗ್ರಹಿಸಿದ ಗಂಗಣ್ಣ ಹೇಳಿದ \enginline{-} 

ನನ್ನನ್ನು ಕೇಳಿದರೆ, "ನನ್ನ ಗುರುಗಳು ಶ್ರೀ ಎನ್.ಎಸ್.ರಾಮಭದ್ರಾಚಾರ್ಯರು\break ಎಂದು ಹೇಳುತ್ತೇನೆ. ಅನೇಕರಿದ್ದರೂ ನಿನಗೆ ಹೀಗೆ ಹೇಳಲು ಯಾರೂ ಇಲ್ಲವಲ್ಲವೇ~!?”  

ನನಗೂ, ಹೌದಲ್ಲ !! ಎನ್ನಿಸಿತು, ನಾನು ತಲೆಯಾಡಿಸಿದೆ. 

ಆಗ ಹೇಳಿದ \enginline{-}  “ಜೀವನದಲ್ಲಿ ಸರಿಯಾದ ಗುರುವನ್ನು ಹುಡುಕಿಕೊಳ್ಳಬೇಕು”. 

ಇದು ನನನ್ನು ಗಂಭೀರ ಚಿಂತನೆಗೆ ತಳ್ಳಿತು. 
ಗಂಗಣ್ಣ ಮನೆಯ ಕಡೆಗೆ ಹೊರಟ. ನಾನು ಬಹಳ ದೂರದವರೆಗೆ ಹಿಂಬಾಲಿಸಿದೆ, ಮುಂದೆಯೂ ಹಿಂಬಾಲಿಸುವ ತೀರ್ಮಾನ ಮಾಡಿದೆ, ಮನೆಯಿಂದ ಓಡಿಹೋಗಲು ಬೇಕಾದ, ರಾಮನಿಂದ ದೊರಕಿದ ಮನೋ\- ಬಲವುಳ್ಳ ಸುಗ್ರೀವನಂತೆ ಮನೋ ಬಲವನ್ನು ಸಂಪಾದಿಸಿಕೊಂಡು  ಹಿಂದಿರುಗಿದೆ. 

ಒಬ್ಬ ಹಿತೈಷಿಗಳಿಂದ ನಾನು ಪ್ರಯಾಣಕ್ಕೆಲ್ಲ ಅಗತ್ಯವಿದ್ದ ಧನವನ್ನು ಸಾಲ ಪಡೆದುಕೊಂಡೆ. ೧೯೯೫ ರ ಜೂನ್ ತಿಂಗಳ ಅಂತಿಮ ದಿನ. ರಾತ್ರೋರಾತ್ರಿ ತಂದೆಗೆ ನಮಸ್ಕರಿಸಿ, ನನ್ನನ್ನು ಹುಡುಕದಿರುವಂತೆ ಯಜಮಾನರ ಹಾಸಿಗೆಯಲ್ಲಿ ಚಿಕ್ಕ ಚೀಟಿಯನ್ನಿಟ್ಟು ಮತ್ತಾರಿಗೂ ತಿಳಿಯದಂತೆ ಸರ್ಟಿಫಿಕೆಟ್ ನ ಒಂದು ಬ್ಯಾಗನ್ನು ಹಿಡಿದು ಉಟ್ಟ ಬಟ್ಟೆಯಲ್ಲಿ ಮನೆ ಬಿಟ್ಟು  ಹೊರಟೇ ಬಿಟ್ಟೆ. ಅಲ್ಲಿ ಇಲ್ಲಿ ಬಿದ್ದು ಎದ್ದು ಮೈಸೂರಿಗೆ ಬಂದು ತಲುಪಿದೆ. ನಮ್ಮ ಸಿದ್ದಾಪುರ ಪೇಟೆಗಿಂತ ಸ್ವಲ್ಪವೇ ಚಿಕ್ಕದಿರುವ ಮೈಸೂರಿನ ಬಸ್ ಸ್ಟ್ಯಾಂಡ್ ನಲ್ಲಿ ಇಳಿದಾಗ ಬೆಳಗಿನ ಜಾವ ೫ ಗಂಟೆ. ದಿಕ್ಕು ಕಾಣಲಿಲ್ಲ. ಯಾರ್ಯಾರನ್ನೋ ಕೇಳಿಕೊಂಡು ಸಂಸ್ಕೃತ ಪಾಠಶಾಲೆಗೆ ಬಂದು ಗಂಗಾಧರ ಭಟ್ಟರ ಮನೆಯನ್ನು ತೋರಿಸುವಂತೆ ಕೇಳಿ ಅಂತೂ ಗಂಗಣ್ಣನ ಮನೆಗೆ ಬಂದು ತಲುಪಿದೆ. 

ಆಗಲೇ ಸಂಸ್ಕೃತ ಪಾಠಶಾಲೆಯಲ್ಲಿ ಎಡ್ಮಿಶನ್ನಿನ ಸರಕಾರಿ ನಿಯತ ಸಮಯವೆಲ್ಲ\break ಮುಗಿದಿತ್ತು. ಆದರೆ ಸಂಸ್ಕೃತ ಪಾಠಶಾಲೆಯ ವ್ಯವಹಾರ ಹಾಗೆ ಮಾಡಿದರೆ ನಡೆಯು\-ವುದು ಕಷ್ಟ. ಮತ್ತೆ, ಗಂಗಣ್ಣ ಕಳುಹಿಸುವ ಹುಡುಗರು ಅಲ್ಲಿ ಯಾವಾಗಲೂ ಸೇರಲು ಅಘೋಷಿತ ಅವಕಾಶವಿದೆ !!! ಗಂಗಣ್ಣನೇ ನೇರವಾಗಿ ಬಂದು ಕೂಡಲೇ ಪಾಠಶಾಲೆಯಲ್ಲಿ ಎಡ್ಮಿಶನ್ ಮಾಡಿಸಿದ. ಸಾಹಿತ್ಯಕ್ಕೆ ಎಡ್ಮಿಶನ್ ಆಯಿತು. ಅಲ್ಲಿಗೆ ನನ್ನ ಗ್ರಾಂಥಿಕ ಕೃಷಿ ಪ್ರಾರಂಭ ಆಯಿತು. ಪಾಠಶಾಲೆಯ ಹಾಸ್ಟೇಲ್ ನಲ್ಲಿ ಒಂದೂ ರೂಮ್ ಸಹ ಖಾಲಿ ಇರಲಿಲ್ಲ. ಆದರೆ ಅಲ್ಲಿ ಮತ್ತೆ ಗಂಗಣ್ಣ  ಬುದ್ಧಿವಂತಿಕೆಯಿಂದ ಒಂದು ರೂಮಿನಲ್ಲಿ ವ್ಯವಸ್ಥೆ ಕಲ್ಪಿಸಿ\-ಕೊಟ್ಟ. ಮಧ್ಯಾಹ್ನ, ರಾತ್ರಿ ಊಟಕ್ಕೆ ವೇದಶಾಸ್ತ್ರ ಪೋಷಿಣೀ ಸಭೆಯಲ್ಲಿ\break ವ್ಯವಸ್ಥೆಯಾಯಿತು.


ಅಂತೂ ಮೂರ್ನಾಲ್ಕು ದಿನದಲ್ಲಿ ಇಷ್ಟಾಗುವ ಹೊತ್ತಿಗೆ ನಮ್ಮ ಯಜಮಾನರು ನನ್ನನ್ನು ಪುನಃ ಮನೆಗೆ ಕರೆದೊಯ್ಯಲು ಮೈಸೂರಿನಲ್ಲಿ ಸೂರ್ಯೋದಯಕ್ಕೆ ಸರಿಯಾಗಿ  ಪ್ರತ್ಯಕ್ಷವಾಗಿಬಿಟ್ಟರು. ಗಂಗಣ್ಣನೆದುರು ನನಗೆ ಅಳು ಬಂತು. ಆದರೆ ಅವನು ಬಲು ಚಾಣಾಕ್ಷ. ಅವನಿಗೂ ನಮ್ಮ ಯಜಮಾನರಿಗೂ ಸಾಕಷ್ಟೇ ವಿಶ್ವಾಸ, ಅದನ್ನು ಕೆಡಿಸಿ\-ಕೊಳ್ಳುವಂತೆಯೂ ಇಲ್ಲ. ನನ್ನ ಆಗಮನಕ್ಕೆ ತಾನೇ ಕಾರಣವೆಂದರೆ ವಿಶ್ವಾಸಕ್ಕೆ ಧಕ್ಕೆ. ಹಾಗಾಗಿ ಅವರೆದುರು, “ಈಗ ಗುರು ತಂದಿರುವ ಪದಾರ್ಥಗಳು ಇಲ್ಲೇ ನಮ್ಮ ಮನೆಯಲ್ಲಿರಲಿ. ನೀವಿಬ್ಬರೂ ಹೋಗಿ ನಿಮ್ಮ ಅನಿವಾರ್ಯತೆಯನ್ನು ಚಿಂತಿಸಿ, ಆಮೇಲೆ ಅವನು ಇಲ್ಲಿಗೆ ಬರಲಿ. ಇಲ್ಲಿ ವ್ಯವಸ್ಥೆಯಂತೂ ಆಗಿದೆ. ಮತ್ತೆ ಯಾವಾಗ ಬಂದರೂ ಆಗಬಹುದು ಎಂದ. ನನಗಾದರೋ ವಾಲಿಯ ಕಾರಣದಿಂದ ಸುಗ್ರೀವನಿಗೆ ಋಷ್ಯಮೂಕ ಪರ್ವತ\break  ಬಿಡುವುದು ಹೇಗೆ ಆತಂಕದ ವಿಷಯವಾಗಿತ್ತೋ,  ಹಾಗೆಯೇ  ಮೈಸೂರು ಬಿಟ್ಟು ಮನೆಗೆ ತೆರಳುವುದು  ನನಗೂ ಆತಂಕವಾಗಿತ್ತು. ಪುನಃ ಮೈಸೂರಿಗೆ ಬರುವ ಭರವಸೆ ಇರಲಿಲ್ಲ. ಆದರೆ ಗಂಗಣ್ಣ ಇಷ್ಟು ಹೇಳಿದ ಮೇಲೆ ಹೋಗದೇ ವಿಧಿಯಿರಲಿಲ್ಲ. ಯಜಮಾನರ ಜೊತೆಗೆ ಊರಿಗೆ ಹೋಗಿ ಮತ್ತೆ ಕುಡುಗೋಲು ಹಿಡಿದೆ, ಎಮ್ಮೆಗಳಿಗೆ ಹುಲ್ಲು ತರಲು ಗದ್ದೆಗೆ ಹೊರಟೆ. ಹಾಗೂ ಒಂದು ವಾರ ಕಳೆಯಿತು. ಇಷ್ಟರಲ್ಲೇ ನನಗೆ ರಾಮಬಲ ಒದಗಿಬಿಟ್ಟಿತ್ತು. ಸುಗ್ರೀವ ವಾಲಿಯನ್ನು ಧೈರ್ಯಮಾಡಿ ಕರೆದಂತೆ ನಾನು ಹೊರಡಲೇ ಬೇಕು, ಇದೋ ಹೊರಟೆ ಎಂದೆ. ಆಗ ಯಾರೂ ನನ್ನನ್ನು ತಡೆಯಲಾಗಲಿಲ್ಲ. ಮತ್ತೆ ಮೈಸೂರಿಗೆ ಬಂದಿಳಿದೆ. ಗಂಗಣ್ಣ ನನ್ನನ್ನು ನೋಡಿ "ಪುನರಾಯಾನ್ ಮಹಾಕಪಿಃ" ಅಂದುಕೊಂಡು ನಕ್ಕು ತನ್ನ ಜೊತೆಗೆ ಪಾಠಶಾಲೆಗೆ ಕರೆದುಕೊಂಡು ಹೊರಟ. 

ನಾನು ನಿತ್ಯವೂ ಗಂಗಣ್ಣನೊಡನೆಯೇ ಶಂಕರವಿಲಾಸ ಪಾಠಶಾಲೆಗೆ ಹೋಗುತ್ತಿದ್ದೆ. ಅಲ್ಲಿ ಅವನು ತರ್ಕ, ವ್ಯಾಕರಣಗಳನ್ನು ಪಾಠಮಾಡುತ್ತಿದ್ದ. ಅಲ್ಲದೇ ಕೆಲವು \hbox{ಕಾವ್ಯಪಾಠ} ಮಾಡುತ್ತಿದ್ದ. ಆ ಪಾಠ ಅತ್ಯಂತ ರಸಸ್ಯಂದಿಯಾಗಿರುತ್ತಿತ್ತು. ವಿಶೇಷವಾಗಿ ಕರುಣ, ವೀರ ರಸಗಳು ಅವನಲ್ಲಿ ಚೆನ್ನಾಗಿ ಅಭಿವ್ಯಕ್ತವಾಗುತ್ತಿದ್ದವು. ಶ್ಲೇಷ ಮತ್ತು ಧ್ವನಿ ಅವನ ಮಾತಿನ ದೇಹ ಮತ್ತು ಆತ್ಮಗಳೆಂದರೆ ತಪ್ಪಿಲ್ಲ. ಈ ಸಾಹಿತ್ಯದಲ್ಲೇ ಹಾಸ್ಯ  ಸಹ\break ಹಾಸು\enginline{-}ಹೊಕ್ಕಾಗಿಯೇ ಇರುತ್ತದೆ.  ಹಾಸ್ಯಕ್ಕಾಗಿಯೇ ಹಾಸ್ಯಮಾಡುವ ದಾರಿದ್ರ್ಯವೂ ಇಲ್ಲ,  ಪ್ರವೃತ್ತಿಯೂ ಇಲ್ಲ. ಹಾಗಾಗಿ ಹಾಸ್ಯ ಎಲ್ಲೂ ಅಪಹಾಸ್ಯಕ್ಕೊಳಗಾಗುವುದಿಲ್ಲ. ಅದು ಇತರರನ್ನು ಅವಾಕ್ ಆಗಿಸುವ ಅವನ ವಾಕ್ .  ಈ ಬಗೆಯ ವಾಗ್ರಂಜನೆ  ನಿತ್ಯ ಒಡನಾಟವಿದ್ದು ಆಸ್ವಾದಿಸಿದವರಿಗೆ ಮಾತ್ರ  ತಿಳಿಯುವ ವಿಷಯ. ಆ ಧ್ವನಿಯನ್ನು  ಬರೆದು ತಿಳಿಸುವುದು  ಸಾಧ್ಯವಿಲ್ಲ. ಸಾಮಾನ್ಯವಾಗಿ ಹುಡುಗರನ್ನು ತಿದ್ದುವುದು ಉಪಾಯವಾದ ಹಾಸ್ಯದಿಂದಲೇ ವಿನಾ ಗದರುವ ಸ್ವಭಾವವೇ ಇರಲಿಲ್ಲ. ಯಾವುದೇ ಭಾವವನ್ನು ಅಗತ್ಯ ಇದ್ದಲ್ಲಿ ಹದವಾಗಿ ಅಭಿವ್ಯಕ್ತಿಸುವ ಸಂಯಮ ಪ್ರಕೃತಿ. ಎಂತಹ ಪರಿಸ್ಥಿತಿಯಲ್ಲೂ ಸಮಾಧಾನವನ್ನು ಕಳೆದು\-ಕೊಳ್ಳದೇ, ಉದ್ವಿಗ್ನವಾಗದೇ ಇರುವ ಸ್ವಭಾವ ಅವನಲ್ಲಿ ಅಸಾಧಾರಣ\-ವಾಗಿದೆ. ಅದೇ ಅವನ ನಿಜವಾದ ಶಕ್ತಿ ಎಂದರೆ ತಪ್ಪಿಲ್ಲ. ಬಹುಶಃ ಇಷ್ಟು ಕಾಲದ ನನ್ನ ಒಡನಾಟದಲ್ಲಿ ಅವನಿಗೆ ಭಯಂಕರವಾಗಿ ಸಿಟ್ಟು ಬಂದಿರುವುದನ್ನು   ಒಮ್ಮೆಯೋ ಎರಡು\break ಬಾರಿಯೋ ನೋಡಿದ್ದೇನೆ. ಅನ್ಯಾಯವನ್ನು ಕಿಂಚಿತ್ತೂ ಸಹಿಸದ ಸ್ವಭಾವ, ಅಂಥದ್ದು  ಘಟಿಸಿದರೆ  ಆ ಕ್ಷಣದಲ್ಲಿಯೇ ಖಂಡನೆ ಖಂಡಿತ. ಇನ್ನು  ಅನ್ಯಾಯವೇನಾದರೂ\break ಅತಿಯಾಗಿ, ಉದ್ದೇಶಪೂರ್ವಕವಾಗಿದ್ದರಂತೂ ಎದುರಿಗಿರುವವನು ಮುಂದೆಂದೂ ಪುನಃ ಅವನೆದುರು ನಿಲ್ಲಲಾರ \enginline{-} ಅಷ್ಟಾಗುವುದು ಶತಸ್ಸಿದ್ಧ. ಆದರೆ ಅವನ ಶಕ್ತಿಯೇ\break ಸಂಯಮ. ಹಾಗಾಗಿಯೇ ಯಾವುದೇ ಸಮಸ್ಯೆಯನ್ನೂ ಹಗುರವಾಗಿ, ಎಲ್ಲೂ \break ಡ್ಯಾಮೇಜ್ ಆಗದಂತೆ ನಿರ್ವಾಹ ಮಾಡುವ ಕೌಶಲ ಅವನಿಗೆ ಸಿದ್ಧಿಸಿದೆ. ಅದೇ\break ಸಂಯಮ ಮತ್ತು ಕೌಶಲ ಪಾಠದಲ್ಲೂ ವಿಷಯ ನಿರೂಪಣೆಯಲ್ಲೂ ಕಾಣುತ್ತದೆ. ಹಾಗಾಗಿ ಆಯಾ \hbox{ಸಂದರ್ಭಕ್ಕೂ}, ಪಾಠಕ್ಕೂ ಹೊರತಾದ ಯಾವದೇ ವಿಷಯವೂ ಅಪ್ಪಿ ತಪ್ಪಿಯೂ ಅಲ್ಲಿ ನುಸುಳುತ್ತಿರ\-ಲಿಲ್ಲ. ಪಾಠದ ವಿಷಯ ಮಾತ್ರ ಗಂಭೀರ ಪ್ರವಾಹವಾಗಿ ಹರಿಯುತ್ತಿತ್ತು. ಅವನ ಪಾಠದಲ್ಲಿ  ಒಂದು ವಿಶಿಷ್ಟ ನೋಟ ಮತ್ತು ವಿಶೇಷ  ಸರಣಿ ಇತ್ತು \enginline{-} ಅದು ವಿದ್ಯಾರ್ಥಿಗಳನ್ನು  ಬಹಳ ಆಕರ್ಷಿಸುತ್ತಿತ್ತು. ಹಾಗಾಗಿ ಆಬಾಲ\enginline{-}ವೃದ್ಧರೂ ಅವನ ಪಾಠವನ್ನು ಮಂತ್ರಮುಗ್ಧರಾಗಿ ಆಸ್ವಾದಿಸುತ್ತಿದ್ದರು. ಅವನಲ್ಲಿದ್ದ  ಆ ವಿಶಿಷ್ಟ ಸರಣಿಗೆ  ಮೂಲ ಬೇರೆಯೇ ಇತ್ತು.  ಅದು ಅನಂತರದ ದಿನಗಳಲ್ಲಿ ನನಗೆ ಮನದಟ್ಟಾಯಿತು. 

ಹೀಗೆ ಶಾಲೆಯಲ್ಲಿ ಪಾಠ ನಡೆಯುತ್ತಿದ್ದರೆ, ಅವನ ಮನೆ ಇನ್ನೊಂದು \hbox{ಶಾಲೆಯಂತೆ} ನಡೆಯುತ್ತಿತ್ತು. ಅಲ್ಲಿಗೆ ಸಾಕಷ್ಟು ಜನರು ಬೇರೆ ಬೇರೆ ವಿಷಯಗಳಿಗಾಗಿ ಪಾಠಕ್ಕೆ ಬರುತ್ತಿದ್ದರು. ಆ ಪಾಠ ನನಗೂ ಆಗುತ್ತಿತ್ತು.  ಈ ರೀತಿಯಲ್ಲಿ ಸಾಹಿತ್ಯ ತರಗತಿಗೆ  ಅಗತವಿದ್ದ ತರ್ಕ ಇತ್ಯಾದಿ ಪಾಠಗಳು ಇಲ್ಲಿ ನಡೆದರೆ ಸಾಹಿತ್ಯದ ಕೆಲವು ಭಾಗ ಸಂಸ್ಕೃತ ಕಾಲೇಜಿನಲ್ಲಿ  ನಡೆದು,  ಏನೂ ಬರದ, ಹಸು, ಎಮ್ಮೆಗಳನ್ನು ಮೇಯಿಸಲು ಮಾತ್ರ ಯೋಗ್ಯನೆಂದು ನಿರ್ಣಯಿಸಲ್ಪಟ್ಟ ನಾನು ಒಂದೇ ವರ್ಷದಲ್ಲಿ ಸಾಹಿತ್ಯದ ಅಂತಿಮ ವರ್ಷದ ಪರೀಕ್ಷೆ ತೆಗೆದುಕೊಂಡು ತೇರ್ಗಡೆಯಾಗುವುದು ಸಾಧ್ಯವಾಯಿತು. ಬಹುಶಃ ಆಗ ಸಾಹಿತ್ಯದ ಕ್ಲಾಸಿನ ನಮ್ಮ ಬ್ಯಾಚಿನಲ್ಲಿ ತರ್ಕವನ್ನು ಸಂಸ್ಕೃತದಲ್ಲಿ ಬರೆದಿದ್ದು ನಾನೊಬ್ಬನೇ ಇರಬೇಕು. ಸಂಸ್ಕೃತದಲ್ಲೇ ಬರೆಯಬೇಕೆಂದು ಗಂಗಣ್ಣ ಒತ್ತಾಯಿಸಿದ್ದರಿಂದ ಅದು ಸಾಧ್ಯವಾಯಿತೇ ವಿನಾ ಇಲ್ಲದಿದ್ದರೆ ಆಗುತ್ತಿರಲಿಲ್ಲ, ಬರೆಯುತ್ತಿರಲಿಲ್ಲ.

ಅದೇ ವರ್ಷದಲ್ಲಿ ಗಂಗಣ್ಣ ನನ್ನನ್ನು ಹೊಳೇನರಸೀಪುರಕ್ಕೆ ರಾಮಾಯಣ ಪಾರಾಯಣ ಮಾಡಲು ಹೋಗುವಂತೆ ಹೇಳಿದ. ಪಾರಾಯಣವನ್ನು ನಿರ್ದಿಷ್ಟ ಅವಧಿಯಲ್ಲಿ ಮುಗಿಸಲು ಹೇಗೆ ಓದಬೇಕು ಎಂಬುದನ್ನೆಲ್ಲ ಕಲಿಸಿ ಕಳುಹಿಸಿಕೊಟ್ಟ. ನಾನು ಹತ್ತು ದಿನಗಳು ಅಲ್ಲಿದ್ದು ಅದನ್ನು ಮುಗಿಸಿ ಬಂದೆ. ನನಗೆ ಎರಡುಸಾವಿರ ಚಿಲ್ಲರೆ\break ದಕ್ಷಿಣೆ ದೊರೆಯಿತು. ಆಗ ಅವನೇ ಬಂದು ಎಸ್.ಬಿ.ಐನಲ್ಲಿ ಖಾತೆಯನ್ನು ಮಾಡಿಸಿ\-ಕೊಟ್ಟು ಆ ಹಣವನ್ನು ಅಲ್ಲಿ ಹಾಕಿಸಿದ. ಅದರಲ್ಲಿ ನಾನು ಮೈಸೂರಿಗೆ ಬರುವಾಗ ತಂದಿದ್ದ ಸಾಲವನ್ನು ತೀರಿಸುವಂತೆ ಹೇಳಿದ. ನಾನು ಹಾಗೆಯೇ ಮಾಡಿದೆ. ಮುಂದೆ ಮತ್ತೊಮ್ಮೆ\break ಬೆಳ್ತಂಗಡಿಯಲ್ಲಿ ನಡೆದ ಅಯುತಚಂಡಿ ಯಾಗಕ್ಕೂ ಅವನೇ ಒತ್ತಾಯಿಸಿ ಕಳಿಸಿದ್ದ. ಹೀಗೆ, ಕೇವಲ ನನ್ನಲ್ಲಿ ಅಂತ ಅಲ್ಲ, ಯಾರಲ್ಲೂ ಇದೇ ರೀತಿಯಲ್ಲಿ ನಿಸ್ವಾರ್ಥದಿಂದ ಪರೋತ್ಕರ್ಷವನ್ನು ಬಯಸುವ ಮಾರ್ಗದರ್ಶನದ  ಶುದ್ಧವ್ಯವಹಾರವನ್ನು ಈ ತನಕವೂ ನೋಡುತ್ತಲೇ ಬಂದಿದ್ದೇನೆ. ಎಲ್ಲೂ ಅದರ ಹದ ವ್ಯತ್ಯಾಸವಾಗಿದ್ದು ನನ್ನ ಗಮನಕ್ಕೆ ಬಂದಿಲ್ಲ.

ನನಗೆ ಸಾಹಿತ್ಯ  ಮುಗಿಯುತ್ತಿದ್ದಂತೆಯೇ ಶಂಕರವಿಲಾಸ ಪಾಠಶಾಲೆಯಲ್ಲಿ ಪ್ರಥಮಾ,\break ಕಾವ್ಯಕ್ಕೆ ಪಾಠಮಾಡುವಂತೆ ತಿಳಿಸಿದ. ನಾನು ಗಾಬರಿಯಾದೆ. ಇನ್ನೂ ನಾನು ಸಂಸ್ಕೃತ ಪುಸ್ತಕ ಹಿಡಿದು ಸರಿಯಾಗಿ ಒಂದು ವರುಷವೂ ಆಗಿಲ್ಲ. ಆಗಲೇ ಪಾಠಮಾಡುವುದು ಹೇಗೆ ಸಾಧ್ಯ. ಅದೂ ಸಹ ಅಲ್ಲಿಗೆ ಆಗಲೇ ಸಂಸ್ಕೃತ ಭಾರತಿಯಿಂದ ಸಂಸ್ಕೃತದ\break ಪರಿಚಯವಿದ್ದ, ನಮ್ಮೆದುರಿಗೇ ಅಲ್ಪಸ್ವಲ್ಪ ಸಂಸ್ಕೃತಲ್ಲಿ ಮಾತನಾಡುತ್ತಿದ್ದ ಕೆಲವು \hbox{ಸುಸಂಸ್ಕೃತ} ಹಿರಿಯ ಸ್ತ್ರೀಯರು ಪಾಠಕ್ಕೆ ಬರುತ್ತಿದ್ದರು. ಅವರಿಗೆಲ್ಲ ಪಾಠಮಾಡುವುದು ಸಾಧ್ಯವಿಲ್ಲ ಎಂದೆ. ಸಾಧ್ಯವಿದೆ ಎಂದು ಅವನ ವಾದ. ನಾನು ಸಾಹಿತ್ಯ ವಿದ್ವತ್ತನ್ನು ವಿಧಿವತ್ತಾಗಿ ಓದಿ ವ್ಯುತ್ಪತ್ತಿಯುಳ್ಳ ಹುಡುಗರನ್ನು ಅಲ್ಲಿ ಪಾಠಮಾಡುವಂತೆ ಕೇಳಿದೆ. ಯಾರೂ ಒಪ್ಪಲಿಲ್ಲ. ಗಂಗಾಧರ ಭಟ್ಟರು ನಿನಗೆ ಮಾಡಲು ಹೇಳಿದ್ದಾರೆ. ನೀನೇ ಮಾಡು, ಎಂದರು. ವಿಧಿಯಿಲ್ಲದೇ ಪಾಠವನ್ನು ಮಾಡಲೇಬೇಕಾಯಿತು. 'ಯೋಗ' ಬಲವಾಗಿದ್ದರೆ ಅದು ಫಲ ಕೊಡುವುದಕ್ಕೆ ಸುಮ್ಮನೆ ಕಾಲಕಾಯುತ್ತಿರುತ್ತದೆ. ಆ ಕಾಲ ಬಂದಾಗ ನಾವೇನೂ\break ಮಾಡದಿದ್ದರೂ ಅದೇ ಎಲ್ಲವನ್ನೂ ಮಾಡಿಸಿಬಿಡುತ್ತದೆ. ನಾವು ಅನ್ಯಥಾ ಚೇಷ್ಟೆ ಮಾಡದೆ ಪ್ರೇಕ್ಷರಂತೆ ವೀಕ್ಷಕರಾಗಿದ್ದರೆ ಸಾಕು. ಹಾಗಾಗಿ ಶುದ್ಧ ಅಯೋಗ್ಯನಾಗಿದ್ದ ನಾನೂ ಯೋಗಮಾತ್ರದಿಂದ ಯೋಗ್ಯವಾಗಿಯೇ ಪಾಠವನ್ನು  ಮಾಡುವುದು ಸಾಧ್ಯವಾಯಿತು. ಗಂಗಣ್ಣ ಕಾಲಕಾಲಕ್ಕೆ ಬೇಕಾದ ವಿಷಯಗಳನ್ನು, ಪಾಠಕ್ಕೆ ಬೇಕಾದ ಮಾರ್ಗದರ್ಶನವನ್ನು ಮಾಡುತ್ತಿದ್ದ. ಈ ಪಾಠ ಮಾಡುತ್ತಿದ್ದುದರಿಂದ ನನಗೆ ತಿಂಗಳು ರೂ/ಐನೂರು  ಸಂಬಳವನ್ನು ಆ ಪಾಠಶಾಲೆಯಿಂದ ಗಂಗಣ್ಣ ಕೊಡಿಸಿದ. ಮುಂದೆ ಇದೇ ಹಣ ನನಗೆ ಮೈಸೂರು ಮಾನಸ ಗಂಗೋತ್ರಿಯಲ್ಲಿ  ಸಂಸ್ಕೃತ ಎಂ.ಎ ಮಾಡುವುದಕ್ಕೂ ಸಹಾಯವಾಯಿತು. ನನ್ನ ಅಗತ್ಯ ಪುಸ್ತಕಗಳಿಗೆ ಬಳಕೆಯಾಯಿತು. ಇದಲ್ಲದೇ ಗಾಯನ ಮತ್ತು ತಬಲಾ ಅಭ್ಯಾಸ ಮಾಡುತ್ತಿದ್ದ ನಾನು ಅದಕ್ಕೆ ಅಗತ್ಯವಿದ್ದ ವಾದ್ಯಗಳನ್ನೂ ಇದೇ ಹಣದಿಂದ ತೆಗೆದು\-ಕೊಂಡೆ. ಹೀಗೆ ಗಂಗಣ್ಣನ ಕಾರಣದಿಂದಲೇ ಇವೆಲ್ಲ ನನಗೆ  ಕೂಡಿಬಂದವು. ಮನೆಯಿಂದ ಬಂದ ಒಂದೇ ವರ್ಷದಲ್ಲಿ ವಿದ್ಯಾರ್ಥಿಯೂ ಆಗಿ ಅಧ್ಯಾಪಕನೂ ಆಗಿದ್ದರ ಹಿಂದೆ ಇದ್ದುದು ಗಂಗಣ್ಣನ ಒತ್ತಾಸೆಯೇ ವಿನಾ ಬೇರೆಯೇನೂ ಅಲ್ಲ. 

ಮುಂದೆ ನಾನು ವಿದ್ವತ್ ತರಗತಿಯಲ್ಲಿ ತರ್ಕಶಾಸ್ತ್ರವನ್ನು ಆಯ್ಕೆ ಮಾಡಿಕೊಂಡೆ. ಗಂಗಣ್ಣನಿಂದ  ಪಾಠನಡೆಯುತ್ತಿತ್ತು. ಆದರೆ ಒಂದೇ ವರುಷದಲ್ಲಿ ಸಾಹಿತ್ಯವನ್ನು\break ಓದಿದ್ದರ ಪರಿಣಾಮ ವಿದ್ವತ್ ತರಗತಿಯ ಪಾಠವನ್ನು ಅರ್ಥಮಾಡಿಕೊಳ್ಳಲು ಅಗತ್ಯ ಇದ್ದ ಹಿಂದಿನ ಎಲ್ಲಾ ಪಾಠ ನನಗೆ ಆಗದೇ ಇದ್ದುದು ಸಾಕಷ್ಟು  ಶ್ರಮವಾಗುತ್ತಿತ್ತು. \hbox{ಪಕ್ಷತಾ} ಗ್ರಂಥ ಅಧ್ಯಯನಕ್ಕಿತ್ತು. ಆ ಗ್ರಂಥದ ಬಗ್ಗೆ, ಪಕ್ಷತಾ ಪ್ರಾಣಘಾತಿನೀ ಎಂಬ\break ನಾಣ್ಣುಡಿಯೇ ಇದೆ. ಗ್ರಂಥ ಕಬ್ಬಿಣದ ಕಡಲೆ. ಅದರ ಮೊದಲ ಪುಟ್ಟ ಪಂಕ್ತಿ “ಅಥ\break ವ್ಯಾಪ್ತ್ಯನಂತರಂ ಪಕ್ಷ\-ಧರ್ಮತಾ ನಿರೂಪ್ಯತೇ” ಎಂಬುದು. ಸರಳವಾದ ಆ \hbox{ಪಂಕ್ತಿಗೆ} ಸುಂದರವಾದ, ಆದರೆ ಅಷ್ಟೇ ಟೆಕ್ನಿಕಲ್ ಆದ ನಿರೂಪಣೆ ಅದರ ಅಡಿಯಲ್ಲಿದೆ.\break ನಿರೂಪಣಾ ಶೈಲಿಯೇ ತರ್ಕ\-ಶಾಸ್ತ್ರದ ಜೀವಾಳ. ತರ್ಕದ ಜಾಡು ಸರಿಯಾಗಿ ಸಿಕ್ಕರೆ ಅದು ವ್ಯಾಘ್ರಮುಖ ಗೋವು. ಇಲ್ಲದಿದ್ದರೆ ಅದು ವ್ಯಾಘ್ರವೇ. ಆ ಒಂದು ಪಂಕ್ತಿಯ ಅರ್ಥವನ್ನು ನಾನು ಗಂಗಣ್ಣನಿಂದ ನಾಲ್ಕು ಬಾರಿ ಪಾಠ ಮಾಡಿಸಿಕೊಳ್ಳಬೇಕಾಯಿತು. ಆದರೆ ಗಂಗಣ್ಣನ ಸ್ವಭಾವ ಹೇಗಿತ್ತೆಂದರೆ ಇನ್ನೂ ನಾಲ್ಕು ಬಾರಿ ನಾನು ಕೇಳಿದರೂ ಮೊದಲನೆಯ ಬಾರಿ ಹೇಳಿದ ಸಮಾಧಾನ, ಉತ್ಸಾಹದಲ್ಲೇ ಆಗಲೂ ಪಾಠ ನಡೆಯುತ್ತಿತ್ತು. ಅಲ್ಲದೇ ಅಷ್ಟು ವಿಭಿನ್ನ ವಿಧಾನವನ್ನು ಬಳಸಿ ಅರ್ಥೈಸುತ್ತಿದ್ದ. ಅವನ ಆ ವಾಕ್ಕೌಶಲ \hbox{ಎಂಥವರನ್ನೂ} ಆಶ್ಚರ್ಯ ಚಕಿತರನ್ನಾಗಿಸುವಂತಿದೆ.  ಆದರೆ ನಾನು ಹೀಗೆಯೇ ಆ ವರ್ಷ ದೂಡಿದೆ. ಮುಂದೆ ಎರಡನೆ ವರ್ಷದಿಂದ ತರ್ಕದ ಜಾಡು ಅರ್ಥವಾಗುತ್ತಾ ಬಂತು. ನಾನು ಮುಂದುವರೆದೆ. ಯಥಾಶಕ್ತಿ ವಿದ್ವತ್ ತರಗತಿಗಿದ್ದ ವಿಷಯಗಳನ್ನು  ಅರ್ಥೈಸಿ\-ಕೊಂಡು ಪದವಿ ಮುಗಿಸಿದೆ. ಅವನ ಆಶೀರ್ವಾದದಿಂದಲೇ ವೈಶಿಷ್ಟ್ಯಶ್ರೇಣಿಯೂ ಬಂತು. \hbox{ರಿಸಲ್ಟ್}  ಬಂದ  ಸಂದರ್ಭದಲ್ಲಿ ನಾನು ಊರಲ್ಲಿದ್ದೆ. ಎಂದೂ ಯಾವತ್ತೂ ಯಾವ ವಿಷಯಕ್ಕೂ ಫೋನ್ ಮಾಡದ ಗಂಗಣ್ಣ ಅಂದು ಮನೆಗೆ ಫೋನ್ ಮಾಡಿ ಈ ವಿಷಯವನ್ನು ತಿಳಿಸಿದ. ಅವನ ಈ ವ್ಯವಹಾರದ ಹಿಂದೆ ಇದ್ದುದು ಒಂದು ಸೂಕ್ಷ್ಮ ಅಂಶ \enginline{-} ಅದು, ಆರು ವರ್ಷದ ಹಿಂದೆ ಮನೆಯಿಂದ ಓಡಿ ಮೈಸೂರಿಗೆ ಬಂದಿದ್ದಕ್ಕೂ,  ನನ್ನ ಅಧ್ಯಯನಕ್ಕೆ ನಮ್ಮೆಜಮಾನರಿಗಿದ್ದ ವಿರೋಧಕ್ಕೂ, ನಾನು ಮೈಸೂರಿಗೆ ಬರುವುದಕ್ಕೆ ತಾನು ಕಾರಣನೆಂದು ತನ್ನ\break ಮೇಲಿರಬಹುದಾದ ಗುಮಾನಿಗೂ ಕೊಟ್ಟ ಒಂದು ಉತ್ತರವಾಗಿತ್ತು \enginline{-} ಅವನ ಒಂದು \hbox{ಫೋನ್} ಕಾಲ್~\enginline{-} ಅವನ ನಡೆಯೇ ಹಾಗೆ. ಇದು ಮನೆಯ ಉಳಿದವರಿಗೆಲ್ಲ ಖುಷಿಯಾಯಿತೇ ವಿನಾ ಗಂಗಣ್ಣನ ವ್ಯವಹಾರ ಅರ್ಥವಾಗಲಿಲ್ಲ. ಅರ್ಥವಾದ \hbox{ಯಜಮಾನರು} ಮಾತನಾಡಲಿಲ್ಲ. ನಾನೂ ಸುಮ್ಮನಿದ್ದುಬಿಟ್ಟೆ. ವಿದ್ವತ್ ತರಗತಿಯಲ್ಲಿ ಉತ್ತಮ ಶ್ರೇಣಿಯಲ್ಲಿ ಪಾಸಾದರೆ ಒಂದು ಶಾಲನ್ನು ಸರ್ಕಾರ ಕೊಡುತ್ತದೆ. ಅದು ಸರ್ಕಾರೀ \hbox{ವ್ಯವಸ್ಥೆಯಲ್ಲಿ} ಪಾಸಾದ ವರ್ಷದಲ್ಲಿಯೇ ಕೊಡುವುದೆಂದಿಲ್ಲ. ಎಷ್ಟೋ ವರ್ಷ \hbox{ಆದಮೇಲೆ} ಬಂದರೆ ಬಂತು ಅಷ್ಟೆ ! ಹಾಗೆ ನನಗೂ ಬರಬೇಕಾದ ಶಾಲು ಪಾಠಶಾಲೆಗೆ ಎಂದೋ ಬಂತು. ಆದರೆ ನಾನು ತೆಗೆದುಕೊಳ್ಳುವುದಕ್ಕೆ ಹೋಗಲೇ ಇಲ್ಲ. ಆದು ಅವಧಿ ಮುಗಿದು ವಾಪಸ್ಸು ಹೋಗುವ ಸಮಯ ಬಂದಿತ್ತು. ಆಗ ಸ್ವತಃ  ಗಂಗಣ್ಣನೇ ಆ \hbox{ಶಾಲನ್ನು} ಮನೆಗೆ ತಂದಿಟ್ಟುಕೊಂಡು ನನಗೆ ಕೊಟ್ಟ. ಗಂಗಣ್ಣನ ಈ ಬಗೆಯ ಪ್ರೀತಿ\enginline{-}ವಿಶ್ವಾಸ,  ವ್ಯವಹಾರ ಸೌಜನ್ಯ ಅವನ ನಿತ್ಯದ ಬದುಕೇ ಆಗಿದೆ.

ನನ್ನ ಅಧ್ಯಯನದ ಅವಧಿಯಲ್ಲಿ ಗಂಗಣ್ಣ ಶೈಲಜಕ್ಕ ಇವರ ವಿವಾಹ ನಿಶ್ಚಯ ಆಯಿತು. ಆ ಸಂದರ್ಭದಲ್ಲಿ ಹಣಕಾಸಿಗೆ ಅಂಥ ಅನುಕೂಲವೇನಿರಲಿಲ್ಲ. ಕಾರಣ\break ಗಂಗಣ್ಣನ ಜೀವನ "ಧನ" ಪ್ರಧಾನ ಜೀವನವಾಗಿರಲಿಲ್ಲ. "ದಾನ" ಪ್ರಧಾನ  ಜೀವನವಾಗಿತ್ತು. ಈ ಸಂದರ್ಭದಲ್ಲಿ ನನಗನ್ನಿಸಿತು \enginline{-} ನನ್ನ ಸಂಪಾದನೆಗೆ ಗಂಗಣ್ಣನೇ ಕಾರಣನಾದ್ದರಿಂದ ನಾನೂ ಒಂದು ಅಳಿಲು ಸೇವೆ ಮಾಡಬೇಕು ಎಂದು. ನನ್ನ ಬ್ಯಾಂಕಿನಲ್ಲಿರುವ ಹಣವನ್ನು ತೆಗೆದೆ. ಅದು ಒಂದು ಸಣ್ಣ ಮೊತ್ತವೂ ಆಗಲಿಲ್ಲ. ಗಂಗಣ್ಣನಿಗೆ ಹೇಳದೇ\break ಬ್ಯಾಂಕಿನ ಅಕೌಂಟ್ ನ್ನೇ ಕ್ಲೋಸ್ ಮಾಡಿದೆ. ಇರುವಷ್ಟು ಹಣವನ್ನು ಕೊಟ್ಟೆ.  ಗಂಗಣ್ಣ ಅದೆಲ್ಲ  ಬೇಡವೆಂದ. ಆದರೆ ನನ್ನ ಭಾವವನ್ನು ಅವನು ತಿರಸ್ಕರಿಸಲಿಲ್ಲ. ಅದರಲ್ಲೂ ನಾನು ಕೊಟ್ಟಿದ್ದು ನನ್ನ ಸಮಾಧಾನಕ್ಕಾಗಿತ್ತೇ ವಿನಾ ಮದುವೆಗೆ ಅಗತ್ಯ ಇರುವುದಕ್ಕೆ ಅದು ಯಾವ ಲೆಕ್ಖಕ್ಕೂ ಆಗುವಂಥದ್ದಲ್ಲ. ಅದರ ಹಿಂದಿನ ಭಾವ ಕೇವಲ, ಅಳಿಲು ಭಕ್ತಿ ಮರಳ ಸೇವೆ  ಮಾತ್ರವಾಗಿತ್ತು. ಇರಲಿ, ನನ್ನ ಮನಸ್ಸಿನಲ್ಲಿ ಅದನ್ನು ನಾನು ಅವನಿಗೇ ಕೊಟ್ಟಿದ್ದಾಗಿತ್ತು. ಏಕೆಂದರೆ ಅದು ಅವನ ಕಾರಣದಿಂದಲೇ ಪ್ರಾಪ್ತವಾಗಿತ್ತಷ್ಟೇ ! ಆದರೆ ಕೆಲವೇ ಅವಧಿಯಲ್ಲಿ ಅವನು ಅದನ್ನು ವಾಪಸ್ಸು ಕೊಟ್ಟುಬಿಟ್ಟ ಎಂಬುದು ಬೇರೆ ಮಾತು. ಆಗ ಅಕೌಂಟ್ ವಿಷಯವನ್ನು ಮಾತ್ರ ಹೇಳಲೇಬೇಕಾಯಿತು. ಲಘು ಅಸಮಾಧಾನಗೊಂಡು\break ಬ್ಯಾಂಕಿನಲ್ಲಿ  ಪುನಃ ಖಾತೆ ತೆರೆಸಿಕೊಟ್ಟ. ಈ ಮಧ್ಯದಲ್ಲಿ ಸಂಪನ್ನವಾದ ಅವನ \hbox{ವಿವಾಹದಲ್ಲಿ} ನಾನು ಎಲ್ಲ ರೀತಿಯ ಕೆಲಸ\enginline{-}ಕಾರ್ಯಗಳಲ್ಲಿ ಆದ್ಯಂತವಾಗಿ ಭಾಗವಹಿಸಿದೆ. ಅವನ ಅಣ್ಣನ ಬೈಕ್ ನಲ್ಲಿ ಅವನನ್ನು ಹಿಂದೆ ಕೂರಿಸಿಕೊಂಡು ಆಮಂತ್ರಣ ಪತ್ರಿಕೆ ಹಿಡಿದು ಅನೇಕ ಊರುಗಳಿಗೆ ಹೋಗುವ ಸುಯೋಗ ಒದಗಿತು.  ಮುಂದೆ ಸಕಲ ಸಿದ್ಧತೆಯೊಂದಿಗೆ\break ಅದ್ಭುತವಾಗಿ ವಿವಾಹ ಸಂಪನ್ನವಾಯಿತು. ಅನೇಕ ದಿಗ್ಗಜ ವಿದ್ವಾಂಸರು ಅಲ್ಲಿ ಉಪಸ್ಥಿತರಿದ್ದರು. ಸುತ್ತಮುತ್ತಲ ಪ್ರೌಢ ಸಮಾಜ ಅಲ್ಲಿ ಸೇರಿತ್ತು. ಗಂಗಣ್ಣ ಎಲ್ಲೆಡೆ ಸಾಕಷ್ಟು  ಪರಿಚಿತನಾಗಿದ್ದರೂ ವಿವಾಹ ಸಂದರ್ಭ ಮಾತ್ರ ಅವನ ಗಂಭೀರ ವ್ಯಕ್ತಿತ್ವವನ್ನು ಅಲ್ಲಿಯ ಸಮಾಜಕ್ಕೆ  ಹೆಚ್ಚು ಪ್ರಕಾಶವಾಗುವಂತೆ ಮಾಡಿತ್ತು. ಇಂತಹ ವಿವಾಹ ಮಹೋತ್ಸವ ಸಂದರ್ಭ,  ಅವಕಾಶ ನನಗೆ ಅತ್ಯಂತ ಸಂತೋಷವನ್ನು , ಧನ್ಯತೆಯನ್ನೂ ತಂದುಕೊಟ್ಟಿತು. ಮುಂದೆ ಗಂಗಣ್ಣ ದಂಪತಿಗಳು ಮೈಸೂರಿಗೆ ಬಂದರು. ಗಂಗಣ್ಣನ ವಿಶ್ವಾಸ ನನಗೆ ಹೇಗೆ\break ಒದಗಿತ್ತೋ ಅದಕ್ಕೆ ಕಿಂಚಿತ್ತೂ ನ್ಯೂನಾತಿರೇಕವಿಲ್ಲದ ವಾತ್ಸಲ್ಯ, ವಿಶ್ವಾಸ ಶ್ರೀಮತಿ \hbox{ಶೈಲಜಕ್ಕಳಿಗೂ} ನನ್ನ ಬಗ್ಗೆ ಉಂಟಾಯಿತು.  ಗಂಗಣ್ಣನ ಒಡನಾಟ ವ್ಯವಹಾರಗಳ ದೃಷ್ಟಿಯಿಂದ  ಗೃಹಕೃತ್ಯ ಸುಲಭ ಸಾಧ್ಯವಾದುದಾಗಿರಲಿಲ್ಲ. ಅದರಲ್ಲೂ ಸಂಸ್ಕೃತ ವಿದ್ವಾಂಸರ ಮನೆಯ ನಿರ್ವಹಣೆ ಇನ್ನೂ ಸ್ವಲ್ಪ ಕಷ್ಟವೇ. ಆದರೂ ಶೈಲಜಕ್ಕ ಗಂಗಣ್ಣನ ಗೃಹಿಣಿಯಾಗಿ ಗೃಹವನ್ನು ನಿರ್ವಹಿಸಿದ ಬಗೆಯನ್ನು ಲಘುವಾಗಿ  ಪರಿಗಣಿಸುವಂತಿಲ್ಲ. ಹಾಗಾಗಿ ಇಂದು  ಗಂಗಣ್ಣನ ಸಮಸ್ತ ಒಡನಾಡಿಗಳ ಗೌರವಕ್ಕೂ ಆದರಕ್ಕೂ ಸಂಪ್ರೀತಿಗೂ ಅವಳು \hbox{ಕಾರಣಳಾಗಿ} ಅವನ ಯಶಸ್ಸಿನಲ್ಲಿ ತಾನೂ ಸಮವಾಗಿ ಪಾಲುದಾರಳಾಗಿ ಗೃಹಿಣೀ ಪದಕ್ಕೆ ಅನ್ವರ್ಥಳಾದದ್ದನ್ನು ನಾವೆಲ್ಲ ನೋಡಿದ್ದೇವೆ.

ವಿವಾಹ ಮುಗಿದು ಪ್ರವೇಶದ  ದಿನವೇ  ರಾತ್ರಿ ಗಂಗಣ್ಣ ಬೆಂಗಳೂರಿಗೆ ಹೊರಡಬೇಕಾಯಿತು. ಕಾರಣ ಈ ಹೊತ್ತಿಗೆ ಮಹಾರಾಜ ಸಂಸ್ಕೃತ ಪಾಠಶಾಲೆಯಲ್ಲಿ ಹೊಸದಾಗಿ ಅನೇಕ ಶಿಕ್ಷಕರ ನೇಮಕ ನಿರ್ಣಯವಾಗಿ ಸಂದರ್ಶನ ನಡೆಯುವುದಿತ್ತು, ನಡೆಯಿತು.  ಇದರಲ್ಲಿ ಗಂಗಣ್ಣನೂ ಆಯ್ಕೆಯಾಗಿ ಶಂಕರವಿಲಾಸ ಪಾಠಶಾಲೆಯಿಂದ ಮಹಾರಾಜ ಸಂಸ್ಕೃತ ಪಾಠಶಾಲೆಗೆ ಆಗಮಿಸಿದ. ಅಲ್ಲಿ ಶಾಸ್ತ್ರ ಪಾಠ ಆರಂಭವಾಯಿತು. ಅವನ ಪಾಠಕ್ರಮ ಬಹಳ ಗಂಭೀರ. ಆದರೂ ಲಲಿತ. ಪಂಕ್ತಿಗಳನ್ನು ಅರ್ಥೈಸುವ\break ಕ್ರಮವೂ ವಿಶಿಷ್ಟ. ಯಾವ ವಿಷಯವನ್ನೇ ತೆಗೆದುಕೊಂಡರೂ ನೇರವಾಗಿ ಅದರ ಬೇರು ಅಥವಾ ಮೂಲವನ್ನು ಹಿಡಿದುಬಿಡುವ ಅಸಾಧಾರಣ ಪ್ರತಿಭೆ. ಅದನ್ನು ಅರ್ಥೈಸಲು ಅತ್ಯಂತ ಶೀಘ್ರವಾಗಿ ಸ್ಫುರಿಸುವ ದೃಷ್ಟಾಂತ. ಈ ಎಲ್ಲ ಅಂಶಗಳು ಅವನ ಪಾಠಕ್ಕೆ ಅದ್ಭುತವಾಗಿ ಪುಷ್ಟಿಕೊಡುವ ವಿಷಯಗಳಾಗಿವೆ. ಎಂಥವನಿಗೂ ಯಾವುದೇ\break ವಿಷಯವನ್ನು ಅರ್ಥಮಾಡಿಸುವ ಕಲೆ ಗಂಗಣ್ಣನಿಗೆ ಕರಗತ. ಹಾಗಾಗಿ ಎಲ್ಲ ಶಾಸ್ತ್ರದ\break ಆಸಕ್ತ ವಿದ್ಯಾರ್ಥಿಗಳು ಪಾಠ ಕೇಳುವುದಕ್ಕೆ ಕಾತರಿಸುತ್ತಿದ್ದುದುಂಟು. ಇಲ್ಲಿ ಒಂದು\break ವಿಷಯವನ್ನು ಉಲ್ಲೇಖಿಸಬೇಕು \enginline{-} ಯಾವತ್ತೂ ವಿವಿಧ  ಕಾರಣಗಳಿಂದ ಗಂಗಣ್ಣನ ಸುತ್ತ ಅನೇಕ ವಿದ್ಯಾರ್ಥಿಗಳು ಇದ್ದೇ ಇರುತ್ತಿದ್ದರು. ಯಾವುದೇ ಕಾರ್ಯಕ್ರಮ ಇರಲಿ, ಅವರೆಲ್ಲ ಭಾಗವಹಿಸುತ್ತಿದ್ದರು.  ಎಲ್ಲ ವಿದ್ಯಾರ್ಥಿಗಳ ವೈಯಕ್ತಿಕವಾದ, ಯಾ  ಪಾಠಶಾಲೆಗೆ ಸಂಬಂಧಿಸಿದ ಎಲ್ಲ ವ್ಯವಹಾರಗಳಿಗೆ, ಸಮಸ್ಯೆಗಳಿಗೆ ಅವನನ್ನೇ ಅವಲಂಬಿಸುತ್ತಿದ್ದರು. ಆದರೆ ಅವನಿಗೆ ಯಾವ ಹುಡುಗರಲ್ಲೂ , ಯಾವ ಶಾಸ್ತ್ರದ \hbox{ವಿದ್ಯಾರ್ಥಿಗಳಲ್ಲೂ} ಕಿಂಚಿತ್ತೂ ತರತಮ ಭಾವ\-ವಿರಲಿಲ್ಲ, ಎಲ್ಲರೂ ಸಮಾನರು. \hbox{ಹುಡುಗರ} ಗುಂಪಿರುವಲ್ಲಿ ಸಮಸ್ಯೆ ಸಹ ಸಾಕಷ್ಟು ಇರುವುದುಂಟಷ್ಟೆ ! ಅಲ್ಲಿ  ಸಮಸ್ಯೆಯ ಪರಿಹಾರ ಹೆಚ್ಚಾಗಿ ಸಾಮೋಪಾಯ\-ದಿಂದಲೇ ಆಗುತ್ತಿತ್ತು. ಕೆಲವೊಮ್ಮೆ ಭಾರೀ ವೀರರಸ ಉಕ್ಕಿ ಅದರಿಂದಲೂ ಸಮಸ್ಯೆಗಳು ಪರಿಹಾರ ಆಗುವುದುಂಟು.  ಕೆಲವೊಮ್ಮೆ ವಿಚಿತ್ರವಾದ ವಾದಸರಣಿಯಲ್ಲಿ  ಇಡೀ ಸಮಸ್ಯೆಯನ್ನೇ ಗೆದ್ದುಬಿಡುವುದೂ ಇದೆ.  ಈ ಪ್ರಯೋಗ ಮಾತ್ರ ಎಷ್ಟೇ ಸಂಪರ್ಕದಲ್ಲಿದ್ದ ಯಾವ\break ವ್ಯಕ್ತಿಗಳೂ ಊಹಿಸದ ರೀತಿಯಲ್ಲಿ ತತ್ಕಾಲದಲ್ಲಿ ಸಿದ್ಧವಾಗಿ ಸಮಸ್ಯೆಯನ್ನು ಜಯಿಸುವ ಕ್ರಮ, ಊಹಾತೀತ. ಗಂಗಣ್ಣನ ಪ್ರತ್ಯುತ್ಪನ್ನ ಮತಿತ್ವದ ಅನ್ವರ್ಥತೆಗೆ ಸಾಕ್ಷಿಯಾಗುವ ಸಂದರ್ಭ.   ಗಂಗಾಧರ ಭಟ್ಟರು ಈ ಸಮಸ್ಯೆಯನ್ನು ಹೇಗೆ ಜಯಿಸುತ್ತಾರೆ ಎಂಬುದು ಅನೇಕರಿಗೆ ಬಹಳ ಕುತೂಹಲದ ವಿಷಯ, ಎಷ್ಟೋ ಪ್ರಕರಣಗಳಲ್ಲಿ ಜಯಿಸಿದ\break ಮೇಲೂ ಹೆಚ್ಚಿನವರಿಗೆ ಅದರ ಒಳ ಮರ್ಮ ಅರ್ಥವಾಗುವುದಿಲ್ಲ. ಅದಕ್ಕವರು ಅಪಾರ್ಥ\-ವನ್ನೇ ಕಲ್ಪಿಸಿಕೊಳ್ಳುವಂತಾಗುತವುದೂ ಉಂಟು. ಹಾಗೆಂದು ಪ್ರತಿಯೊಂದೂ ಇಟ್ಟ ನಡೆ ನಿಲುವು, ನಿರ್ಣಯಗಳೆಲ್ಲ  ಆತ್ಯಂತಿಕ  ಪರಮ ಪ್ರಮಾಣಭೂತವೆಂದು ಇದರ ತಾತ್ಪರ್ಯವಲ್ಲ. ದೋಷವೇ ಇಲ್ಲದ ವ್ಯಕ್ತಿಗಳಿದ್ದಾರೆಯೇ ! ಅದರೆ ಅವನ ನಡೆಯ ಹಿಂದಿನ ಸಂಕಲ್ಪ, ಉದ್ದೇಶಗಳು ಮಾತ್ರ  ಶುದ್ಧ, ನಿಸ್ಸ್ವಾರ್ಥ,  ಪ್ರಾಮಾಣಿಕ ಎಂಬುದನ್ನು ಘಂಟಾಘೋಷವಾಗಿ ಹೇಳಬಹುದು. 

ಒಮ್ಮೆ ಪಾಠಶಾಲೆಯಲ್ಲಿ   ಪದೇ ಪದೇ  ಕಳ್ಳತನ ಮಾಡುತ್ತಿದ್ದ ಹುಡುಗನೊಬ್ಬನನ್ನು\break  ರೆಡ್ ಹ್ಯಾಂಡ್ ಹಿಡಿದಿದ್ದೆವು. ಗಂಗಣ್ಣನೆದುರು ಅವನನ್ನು ಹಾಜರುಪಡಿಸಿ \hbox{ಅವನೆದುರು} ಕಳ್ಳ  ಒಪ್ಪಿಕೊಂಡೂ ಆಗಿತ್ತು. ಅವನ ತಂದೆಯನ್ನು ಕರೆಸಿ ಅವನನ್ನು ಮನೆಗೆ ಕಳುಹಿಸುವುದೆಂದು ಅಂದೇ ರಾತ್ರಿ  ಗಂಗಣ್ಣನೇ ತೀರ್ಮಾನ ಕೊಟ್ಟ.  ಆದರೆ ಮಾರನೇ ದಿನ ಬೆಳಿಗ್ಗೆ  ಎಲ್ಲರೆದುರು ಇದೇ ಗಂಗಣ್ಣ ಯಾರಿಗೂ ಪ್ರತಿವಾದ \hbox{ಮಾಡಲಾಗದಂತೆ}\break ವಿಚಿತ್ರವಾಗಿ ವಾದ ಮಾಡಿ  ಅವನು ಕಳ್ಳನೇ ಅಲ್ಲ ಎಂದು ನಿರ್ಣಯಿಸಿ, ಅವನ ತಂದೆಗೆ ಅಭಯ ಕೊಟ್ಟು ಕಳುಹಿಸಿಕೊಟ್ಟುಬಿಟ್ಟ. ಯಾರಾದರೂ ನಿನ್ನನ್ನು ಕಳ್ಳ ಎಂದರೆ ನನ್ನಲ್ಲಿ ಬಂದು ಹೇಳು ಎಂದು ಕಳ್ಳನಿಗೇ ಅಭಯ ಕೊಟ್ಟು ಕಳುಹಿಸಿದ. ಕಳ್ಳನನ್ನು\break ಹಿಡಿದವರೇ ತಲೆತಗ್ಗಿಸುವ ಸಂದರ್ಭ. ಗಂಗಣ್ಣನ ವ್ಯವಹಾರದಿಂದ ಹಣ, ವಸ್ತುಗಳನ್ನು ಕಳೆದುಕೊಂಡ ನಾವು ಕಂಗಾಲಾಗಿಬಿಟ್ಟೆವು. ಬೇರೆ ವಿದ್ಯಾರ್ಥಿಗಳು ಹೆದರಿ ಸುಮ್ಮನಾಗಿಬಿಟ್ಟರು. ನಾನು ಸಲುಗೆಯಿದ್ದ ಕಾರಣ ಮಾರನೇ ದಿನ ರಾತ್ರಿ ಮನೆಗೆ ಹೋಗಿ \hbox{ಕೇಳಿದೆ.} ಗಂಗಣ್ಣ ಕೊಟ್ಟ ಕಾರಣ ನನಗೆ ಅತ್ಯಾಶ್ಚರ್ಯವನ್ನು ಉಂಟುಮಾಡಿತು.  ಆ ವಿಷಯ  ಕೇಳಿದಾಗ ಅವನು ಹಾಗೆ ತೀರ್ಮಾನಿಸಿದ್ದೇ ಸರ್ವಾತ್ಮನಾ ಯುಕ್ತ ಎನ್ನಿಸಿತು. \hbox{ಏಕೆಂದರೆ} ಆ ವ್ಯವಹಾರ  ಅಷ್ಟು ಸೂಕ್ಷ್ಮವಾಗಿತ್ತು. ಹಾಗಾಗಿ ಗಂಗಣ್ಣನ ವ್ಯವಹಾರ ಅನೇಕರಿಗೆ ನಿಲುಕುವುದೂ ಕಷ್ಟ, ಅರಗುವುದೂ ಕಷ್ಟ. ಅವನ ವಿದ್ಯಾರ್ಥಿ ದೆಸೆಯಿಂದ ಕೇಳುವ ಕೆಲವು ಘಟನೆ ಮತ್ತು ನಾನು ನೋಡಿದ ಅನೇಕ ಸಂದರ್ಭಗಳು  ಚಾಣಕ್ಯನ ಎಲ್ಲ ಬಗೆಯ  ಪ್ರಯೋಗಗಗಳಿಗೆ  ಸಾಕ್ಷಿಯಾಗಿವೆ. ಸ್ವಪ್ನೇಪಿ ನ ವೃಥಾ ಚೇಷ್ಟತೇ ಚಾಣಕ್ಯಃ ಎಂಬ ನುಡಿಗೆ ನಮ್ಮೆದುರಿಗಿರುವ ಒಂದು ಉದಾಹರಣೆ. ಆದರೂ ಸಮಸ್ಯೆಯ ಪರಿಹಾರಕ್ಕೆ\break ಭೇದೋಪಾಯ ಪ್ರಯೋಗವಾಗುತ್ತಿರಲಿಲ್ಲ. ಆ ಉಪಾಯದಿಂದ   ಅಪಾಯವೇ ಹೆಚ್ಚು.  ಕಿಂಚತ್ ವ್ಯತ್ಯಾಸವಾದರೂ  ಪರಿಹಾರಕ್ಕೆ ಬಂದ ಗುಂಪುಗಳು  ನಂಬಿಕೆ ಕಳೆದುಕೊಂಡು ತಿರುಗಿ ತಲೆಹಾಕಂದಂತಾಗಬಹುದು.  ಸಮಾಜದಲ್ಲಿ ಈ ಪ್ರಯೋಗವೇ ಹೆಚ್ಚು ಬಳಕೆಯಾಗಿ ಪರಸ್ಪರರಲ್ಲಿ ಯಾರ ಮೇಲೂ ಯಾರಿಗೂ ವಿಶ್ವಾಸವಿಲ್ಲದ ಪರಿಸ್ಥಿತಿಯನ್ನು\break ವ್ಯಾಪಕವಾಗಿ ನೋಡುತ್ತೇವೆ. ಸಂಘ ಸಂಸ್ಥೆಗಳು ಇಂತಹ ನಡೆಯಿಂದ ಅಲ್ಪಾವಧಿಯಲ್ಲಿ ಅಸ್ತಂಗತವಾಗಿಬಿಡುತ್ತವೆ. ಇಂದು ಜೊತೆಗಿದ್ದ ಹುಡುಗರು ಮುಂದೆ ಇರುತ್ತಾರೆಂಬ ಭರವಸೆಯಿಲ್ಲ. ಇರುವವರೂ ವಿಶ್ವಾಸದಿಂದ ಇರುತ್ತಾರೆಂಬ ನಂಬಿಕೆಯಿಲ್ಲ. ವೈಷಮ್ಯದ ಅಗ್ನಿ ಪ್ರಜ್ವಲಿಸುವಂತಾಗುತ್ತದೆ. ಆದರೆ ಗಂಗಣ್ಣನಲ್ಲಿ  ಮಾತ್ರ ಈ ಬಗೆಯ ವ್ಯವಹಾರವಿಲ್ಲದೇ ವಿಶ್ವಾಸವೇ ಪ್ರಧಾನವಾಗಿ ಸಾಮವೇ ಸಾಧನವಾಗಿದೆ.  ಹಾಗಾಗಿಯೇ ಅವನ \hbox{ಸಂಪರ್ಕಕ್ಕೆ } ಬಂದ ವ್ಯಕ್ತಿಗಳಾಗಲೀ, ವಿದ್ಯಾರ್ಥಿಗಳಾಗಲೀ ಮತ್ತೆ ಅವನಿಂದ ವಿಮುಖರಾದುದಿಲ್ಲ. ಸಮಾಜದೊಂದಿಗೆ  ಶುದ್ಧ ಸ್ನೇಹ, ನಿಸ್ಸ್ವಾರ್ಥ ವ್ಯವಹಾರ, ಅಲ್ಲದೇ ಎಲ್ಲ ಸಂದರ್ಭಗಳಲ್ಲಿ ಹುಡುಗರ ಪರವಾಗಿ ನಿಲ್ಲುವ ಅವನ ಛಾತ್ರಪ್ರೀತಿ ಅದಕ್ಕೆ  ಪ್ರಮುಖ ಕಾರಣವಾಗಿವೆ. 

ಕಾಲೇಜುಗಳಲ್ಲಿ  ಸ್ಪರ್ಧೆಗಳ ಕಾಲ ಬಂತೆಂದರೆ ಸ್ಪರ್ಧೆಯಲ್ಲಿ ಭಾಗ\-ವಹಿಸುವ ಹುಡುಗರಿಗಿಂತ ಗಂಗಣ್ಣ ಹೆಚ್ಚು ಬ್ಯುಸಿಯಾಗಿರುತ್ತಿದ್ದ. ಏಕೆಂದರೆ ಹೆಚ್ಚು ಕಮ್ಮಿ ಎಲ್ಲ ಶಾಸ್ತ್ರದ ವಿದ್ಯಾರ್ಥಿಗಳು ಗಂಗಣ್ಣನ ಮನೆಗೆ ಹಾಜರಾಗುತ್ತಿದ್ದರು. ಅವರಲ್ಲಿ ಒಂದೇ ಶಾಸ್ತ್ರದ ಅನೇಕ ವಿದ್ಯಾರ್ಥಿಗಳು ಇರುತ್ತಿದ್ದರು. ಅವರವರ ಶಾಸ್ತ್ರದಲ್ಲಿ ಒಂದೇ ವಿಷಯವನ್ನು ವಿಭಿನ್ನವಾಗಿ ಸಿದ್ಧಮಾಡಿಕೊಡಬೇಕಾಗುತ್ತಿತ್ತು, ಅದೇ  ಸವಾಲಿನ ವಿಷಯ. \hbox{ಶಾಸ್ತ್ರರಸಿಕರಿಗೆ} ಅದೊಂದು ಶಾಸ್ತ್ರಕ್ರೀಡೆ. ಅದನ್ನು ಗಂಗಣ್ಣ ಲೀಲೆಯಿಂದ ಆಡುತ್ತಿದ್ದುದು ನೋಡಲು ಭಲೇ ಖುಷಿಯಾಗಿರುತ್ತಿತ್ತು. ಹಾಗೆಯೇ ಹುಡುಗರಿಗೆ ಬಹುಮಾನವೂ \hbox{ಸಿಗುತ್ತಿತ್ತು.} ನಾನು ಯಾವ ಸ್ಪರ್ಧೆಗಳನ್ನೂ ಸಾಮಾನ್ಯವಾಗಿ ಬಿಡುತ್ತಿರಲಿಲ್ಲ. ಅನೇಕ ಬಹುಮಾನಗಳೂ ಬಂದಿವೆ. ಬಂದ ಬಹುಮಾನದ ಹಿಂದಿದ್ದು ಗಂಗಣ್ಣ. ಹಾಗಾಗಿ ನಾನು ಈ ಕೃತಕೃತ್ಯತೆಗೆ ಬಂದ ಬಹುಮಾನವನ್ನೆಲ್ಲ ನಾನದನ್ನು ಗಂಗಣ್ಣನ ಮನೆಯಲ್ಲಿಯೇ ಇಟ್ಟುಬಿಟ್ಟಿದ್ದೇನೆಯೇ ವಿನಾ ಯಾವುದೊಂದನ್ನೂ ನನ್ನ ರೂಮಿಗೆ ಸಾಗಿಸಿಲ್ಲ. 

ಇಷ್ಟೆಲ್ಲ ರೀತಿಯಲ್ಲಿ ವಿದ್ಯಾರ್ಥಿಗಳಿಗೆ ಉಪಕಾರಿಯಾಗಿಯೂ ವಿದ್ಯಾರ್ಥಿಗಳಿಂದ ಅಧ್ಯಯನವನ್ನು ಬಿಟ್ಟು ಇನ್ಯಾವುದನ್ನೂ ಗಂಗಣ್ಣ ಅಪೇಕ್ಷಿಸಿದವನಲ್ಲ. ಅವನು ಯಾವುದೇ ತಮ್ಮ ವೈಯಕ್ತಿಕ ಕೆಲಸಕ್ಕೆ ವಿದ್ಯಾರ್ಥಿಗಳನ್ನು ಬಳಸಲಿಲ್ಲ. ಬಳಸಿಕೊಂಡರೆ ಯಾವ ತಪ್ಪೂ ಇಲ್ಲ. ಶಿಷ್ಯ ಆಚಾರ್ಯನ ಹಕ್ಕು, ಆದರೆ ಅವನು ಮಾತ್ರ  ಅಪೇಕ್ಷಿಸಲಿಲ್ಲ. ನಾನು ಅವರ ಮನೆಯಲ್ಲಿಯೇ ಹೆಚ್ಚಾಗಿ ಇರುತ್ತಿದ್ದ ಕಾರಣ ನಮ್ಮಲ್ಲಿ ವ್ಯವಹಾರ ಕೊಂಚ \hbox{ಭಿನ್ನವಾಗಿತ್ತು.} ಬಹಳ ಕಾಲದ ನಮ್ಮ ಅವರ ಮನೆಯ ಒಡನಾಟದಿಂದ \hbox{ಪರಸ್ಪರರಲ್ಲಿ} ಮನೆಯ ಭೇದ\-ವಿರದ ಕಾರಣ ನಮ್ಮೀರ್ವರ ನಾಡಿಮಿಡಿತ \hbox{ಪರಸ್ಪರರಿಗೆ} ಚೆನ್ನಾಗಿಯೇ ಅರಿವಿದೆ. ಹಾಗಾಗಿ ನನ್ನಲ್ಲಿ ಯಾವ ವಿಷಯದಲ್ಲೂ ನಿಸ್ಸಂಕೋಚವಾಗಿ ವರ್ತಿಸು\-ವಂತಿದ್ದರೂ ಸಹ ಯಾವುದನ್ನೂ ಅಪೇಕ್ಷಿಸಿದ್ದಿಲ್ಲ. ನಾವೇ ಸ್ವತಃ ತಿಳಿದು \hbox{ಏನನ್ನಾದರೂ} ಮಾಡಬೇಕು. ಅವರ ಮನೆಯವರೆಲ್ಲರ  ಸ್ವಭಾವವೇ ಹೇಗಿದೆಯೆಂದರೆ ಅವರು ಸಾಮಾನ್ಯವಾಗಿ ಇತರರಲ್ಲಿ ಬಾಯಿಬಿಟ್ಟು ಯಾವ ಕೆಲಸವನ್ನೂ ಹೇಳುವ ರೂಢಿ ಇಲ್ಲ. ಮಾಡಬೇಕಾದವರು ತಿಳಿದು ಮಾಡಿದರೆ ಉಂಟು. ಇಲ್ಲದಿದ್ದರೆ ಇವರೇ ಅದನ್ನು ಮಾಡಿಬಿಡುತ್ತಾರೆ. ಆದರೆ ಅದು ಕಠಿಣ ಶಿಕ್ಷಣದಂತಿರುತ್ತದೆ. ಗಂಗಣ್ಣನಲ್ಲಂತೂ ಈ ಸ್ವಭಾವ ಇನ್ನೂ ಸ್ವಲ್ಪ ಜಾಸ್ತಿ. ಮೇಲೆ ಹೇಳಿದಂತೆ ನನಗೆ ಅವರ ನಾಡಿಮಿಡಿತ ಚೆನ್ನಾಗಿ\break ತಿಳಿದಿದ್ದರಿಂದಲೂ ಮತ್ತು ಸ್ವಭಾವತಃ ಅವರ ಮನೆಯಲ್ಲಿ ನನಗೆ ವಾಚ್ಯವಾಗಿ ಕೆಲಸವನ್ನು ಹೇಳುವ ಅಗತ್ಯವಿಲ್ಲದೇ ಇಂಗಿತವನ್ನು ಅರಿತು ಮಾಡುವ ಅಭ್ಯಾಸ ಇದ್ದುದರಿಂದಲೂ ನನಗೆ ಕೆಲಸ ಹುಡುಕುವ ಅಥವಾ ಕೇಳುವ ಪ್ರಮೇಯ ಬರುತ್ತಿರಲಿಲ್ಲ, ನಾನೇ ಮಾಡುತ್ತಿದ್ದುದುಂಟು. ಆದರೆ ಅವನು ಅಷ್ಟಾಗಿ ಪರಾಲಂಬಿಯಾಗುವುದನ್ನು ಇಷ್ಟಪಡದ ಕಾರಣ ನನ್ನ ಸ್ವಪ್ರೇರಿತ ಚುರುಕುತನವನ್ನು  ತಡೆಯುತ್ತಿದ್ದುದೂ ಉಂಟು. ಇಂತಹ ಸನ್ನಿವೇಶದಲ್ಲಿ ಒಮ್ಮೆ ನಾನು “ಗುರುಶುಶ್ರೂಷಯಾ ವಿದ್ಯಾ  ! ಎಂದಿದ್ದಾರೆ” ಎಂದೆ. -(ಹಾಗಾಗಿ ನಾನು ಮಾಡುವುದು  ಸರಿ ಎಂಬರ್ಥದಲ್ಲಿ) ಅದಕ್ಕವನು ತಕ್ಷಣವೇ ಹೇಳಿದ \enginline{-} \hbox{"ಶುಶ್ರೂಷಾ} ಶಬ್ದಕ್ಕೆ ನೀನಂದುಕೊಂಡ ಅರ್ಥಕ್ಕಿಂತ ಭಿನ್ನವಾದ ಅರ್ಥವಿದೆ, \hbox{ಶುಶ್ರೂಷಾ} ಎಂದರೆ \hbox{ಶ್ರೋತುಮ್} ಇಚ್ಛಾ \enginline{-}ಗುರುವಿನ ಉಪದೇಶವನ್ನು ಕೇಳಬೇಕೆಂಬ ಅಪೇಕ್ಷೆಗೆ \hbox{ಶುಶ್ರೂಷೆ} ಎಂದು ಹೆಸರು. ಗುರು ಹೇಳುವುದನ್ನು ಅಪೇಕ್ಷೆಪಟ್ಟು ಕೇಳಿದರೆ ಗುರುವಿನ ವಿದ್ಯೆ ಶಿಷ್ಯನಲ್ಲಿ ಚೆನ್ನಾಗಿ \hbox{ಬೆಳೆಯುವುದು} ಸಾಧ್ಯ. ಆದ್ದರಿಂದ ಅದಕ್ಕೆ ಈ ಅರ್ಥ ಹೆಚ್ಚು ಸಂಗತ," ಎಂದು ಹೇಳಿ, ಅವನ ಎಂದಿನ ಶೈಲಿಯಲ್ಲಿ ಮಾತನ್ನು ಮುಂದುವರೆಸಿ, ನೀನು ಹೇಳಿದ ಅರ್ಥವನ್ನು ವಿದ್ಯಾರ್ಥಿಗಳಿಂದ \hbox{ಸೇವೆಯನ್ನೇ} ಹೆಚ್ಚು ಬಯಸಿದ ಅಧ್ಯಾಪಕರು ಪ್ರಚುರಪಡಿಸಿ\-ಕೊಂಡಿದ್ದಾರೆ” ಎಂದು ನಕ್ಕ. ಹೀಗೆ ತಾನು ಎಷ್ಟೇ ಪರೋಪಕಾರಿ\-ಯಾಗಿದ್ದರೂ ತನಗೂ ಅದೇ ಸೇವೆ, ಉಪಕಾರವನ್ನು ಅಪೇಕ್ಷಿಸುವ ಸ್ವಭಾವ ಅವನದಲ್ಲ ಎಂಬುದು ಅವನ ವ್ಯಕ್ತಿತ್ವದ ಇನ್ನೊಂದು ವಿಶೇಷತೆ.
\vskip 8pt

ಗಂಗಣ್ಣ ಒಂದು ಶಾಸ್ತ್ರಕ್ಕೆ ಸೀಮಿತನಾದವನಲ್ಲ. ಆಯುರ್ವೇದ, ಅರ್ಥಶಾಸ್ತ್ರಗಳು ಅವನ ವಿಶೇಷ ಆಸಕ್ತ ವಿಷಯಗಳು. ಆಯುರ್ವೇದ ಮತ್ತು ಸಂಸ್ಕೃತವನ್ನು ಪಾಠ\-ಮಾಡಿಸಿಕೊಳ್ಳಲು ಅವನಲ್ಲಿಗೆ ವೈದ್ಯ ವಿದ್ಯಾರ್ಥಿಗಳು ಬರುತ್ತಿದ್ದರು, ಗಂಗಣ್ಣನೇ ಆಯುರ್ವೇದ ಕಾಲೇಜಿಗೆ ಹೋಗುತ್ತಿದ್ದುದೂ ಉಂಟು. ಚಾಣಕ್ಯನ ಅರ್ಥಶಾಸ್ತ್ರ ಗಂಗಣ್ಣನಿಗೆ ಪರಮ\-ಪ್ರಿಯವಾಗಿತ್ತು. ಪ್ರತಿ ವರ್ಷ ಶಿವಮೊಗ್ಗದಲ್ಲಿ ಕಾಲೇಜಿನ ಅಧ್ಯಾಪಕ ಸಮೂಹಕ್ಕೆ ಅರ್ಥಶಾಸ್ತ್ರದ ಬಗ್ಗೆ ಮೂರ್ನಾಲ್ಕು ಪ್ರವಚನಗಳನ್ನು ಅಲ್ಲಿಯವರು ಏರ್ಪಡಿಸುತ್ತಿದ್ದರು. ಆ ಸಮಯದಲ್ಲಿ ಸಿದ್ಧತಾಕಾರ್ಯ ಅತ್ಯಂತ ಭರದಿಂದ ನಡೆಯುತ್ತಿತ್ತು. ಅವನು ಸಿದ್ಧತೆ \hbox{ಮಾಡುವುದೆಂದರೆ} ಊಟ ಮತ್ತು ಮಧ್ಯೆ ಒಂದು ಚಹಕ್ಕೆ  ಮಾತ್ರ ಬಿಡುವು, ಉಳಿದಂತೆ ಇಡೀ ದಿನ ಏಳುವ ಪ್ರಶ್ನೆಯಿಲ್ಲ. ಹಾಗೆ ತಯಾರಿ ಮಾಡುವಾಗ ಅರ್ಥಶಾಸ್ತ್ರದಲ್ಲಿರುವ ನಗರ ನಿರ್ಮಾಣವೇ ಮುಂತಾದ ಅನೇಕ ವಿಷಯವನ್ನು ಮನದಟ್ಟು ಮಾಡಲು ವಿವಿಧವಾದ ಚಿತ್ರ\-ಗಳನ್ನೆಲ್ಲ ಇಲ್ಲಿ ಸಿದ್ಧಪಡಿಸಲಾಗುತ್ತಿತ್ತು. ಈ ಸಿದ್ಧತೆಯಲ್ಲಿ  ನಾವು ಸಂಪೂರ್ಣವಾಗಿ ಪಾಲ್ಗೊಳ್ಳು\-ತ್ತಿದ್ದೆವು. ಹಾಗಾಗಿ ಅರ್ಥಶಾಸ್ತ್ರದ ಪರಮಾರ್ಥದೆಡೆಗೆ ಆಗಲೇ ಒಂದು ಲಕ್ಷ್ಯ ಅಥವಾ ಸಂಸ್ಕಾರ ನನಗೂ ಒದಗುವಂತಾಗಿತ್ತು. ಆಧುನಿಕ ಅರ್ಥಶಾಸ್ತ್ರ ಭೋಗ ಮಾತ್ರ ಲಕ್ಷ್ಯವುಳ್ಳದ್ದು. ಚಾಣಕ್ಯನ ಅರ್ಥಶಾಸ್ತ್ರ ಪರಮಾರ್ಥ ಪ್ರಯೋಜಕವಾದದ್ದು  ಎಂಬುದು ಅರ್ಥಶಾಸ್ತ್ರದ ಬಗೆಗಿನ ಅವನ ನೋಟವಾಗಿದೆ.

ಶಾಸ್ತ್ರಪಾಠದ ವಿಷಗಳು ಬಂದಾಗ ಗಂಗಣ್ಣ ಆಗಾಗ ತನ್ನ ಗುರುಗಳಾದ ಶ್ರೀ ಎನ್.ಎಸ್.ರಾಮಭದ್ರಾಚಾರ್ಯರ ಬಗ್ಗೆ  ಹೇಳುತ್ತಿದ್ದ. ಅವರ ಪಾಠದ \hbox{ಶೈಲಿಯನ್ನು} ಪರಿಚಯಿಸುತ್ತಿದ್ದ. ನನಗೆ ಅವರನ್ನು ನೋಡುವ ಕುತೂಹಲ ಉಂಟಾಯಿತು. ಹಲವು \hbox{ಪ್ರಯತ್ನಗಳಲ್ಲಿ} ಅವರನ್ನು  ಭೇಟಿಯಾಗುವುದು ಸಾಧ್ಯವಾಯಿತು. ಆಮೇಲೆ ಅವರ ಪಾಠಪ್ರವಚನಗಳನ್ನು ಸಾಕಷ್ಟು ಕೇಳುವ ಅವಕಾಶವೂ ಒದಗಿತು. ಅವರ ಪಾಠದಲ್ಲಿ ಅಸಾಧಾರಣವಾದ ಚುಂಬಕ ಶಕ್ತಿಯಿತ್ತು. ಒಮ್ಮೆ ಪಾಠ ಕೇಳಿದರೆ ಮತ್ತೆ ಮತ್ತೆ ಆ ಕಡೆ ಸೆಳೆಯುತ್ತಿತ್ತು. ಅವರ ಪಾಠ ಕೇಳಿದ ನನಗೆ ಒಂದು ವಿಷಯ ಸ್ಪಷ್ಟವಾಯಿತು. \hbox{ಗಂಗಣ್ಣನಲ್ಲಿ} ಕಾಣುತ್ತಿದ್ದ ವಿಶಿಷ್ಟ ಶೈಲಿ ಅದು ರಾಮಭದ್ರಾಚಾರ್ಯರ ಪಾಠದ \hbox{ಶೈಲಿಯೇ} ಆಗಿತ್ತು. ಯಾವುದೇ ವಿಷಯವಾದರೂ ಸರಿ, ಅದರ ಜಾಡನ್ನು ಹಿಡಿದು \hbox{ಕೂಡಲೆ} ಆ ವಿಷಯದ ತಳಕ್ಕೆ ಇಳಿದು ಒಂದೊಂದೂ ಶಬ್ದವನ್ನು ಅದರ ಆಮೂಲಾಗ್ರ ಬಿಡಿಸಿ ಬಿಡಿಸಿ ನಿರೂಪಿಸುವುದು ಆಚಾರ್ಯರ ಶೈಲಿ. ತಾನು ಹೇಳುವ ಯಾವುದೇ ಒಂದು ಶಬ್ದ ಅಕ್ಷರ ಸಹ ಎದುರಿಗಿರುವವನಿಗೆ ಅರ್ಥವಾಗಿಲ್ಲ ಎಂದಾಗಬಾರದು ಎಂಬುದು ಅವರ ಪಾಠದ \hbox{ಪ್ರತಿಜ್ಞೆ.} ಗ್ರಂಥದ ಆರಂಭದ ಕೆಲವು ಭಾಗ ಪಾಠ ಮಾಡಿದರೆ ಮುಂದೆ ಅವನೇ ಗ್ರಂಥವನ್ನು ಓದಿಕೊಂಡು ಅರ್ಥಮಾಡಿಕೊಳ್ಳುವಂತಿರಬೇಕು ಎಂಬುದು ಅವರ ನಿಲುವಾಗಿತ್ತು. ಅಂತೆಯೇ ಪಾಠವಿರುತ್ತಿತ್ತು. ಈ ರೀತಿಯ ಕಲೆ ಅವರಿಂದ ಗಂಗಣ್ಣನಿಗೂ ಹರಿದಿದೆಯೆಂಬುದು ಉಭಯತ್ರ ಪಾಠ ಕೇಳಿದ ನನಗೆ ಸ್ಪಷ್ಟವಾದ ವಿಷಯ. ಅದಲ್ಲದೇ ಶ್ರೀ ಎನ್.ಎಸ್.ಆರ್ ರವರು ದೃಷ್ಟಾಂತ ಕೊಡುವ \hbox{ವಿಷಯದಲ್ಲಂತೂ} ಅವರೇ ಒಂದು ದೃಷ್ಟಾಂತ ಎನ್ನಬಹುದು. ಎಂತಹ ಕಠಿಣ ವಿಷಯ\-ವನ್ನೂ ಎದುರಿಗಿರುವ ವ್ಯಕ್ತಿಗೆ ಅರ್ಥೈಸುವ ನೈಪುಣ್ಯದ ಜೀವವೇ ಅವರ ದೃಷ್ಟಾಂತ. ಅದು ಅವರ ನಿಜ ಶಕ್ತಿ. ಅವರದೇ ರೀತಿಯಲ್ಲಿ ದೃಷ್ಟಾಂತ ಕೊಡುವ ಶೈಲಿಯೂ ಗಂಗಣ್ಣನಿಗೆ ಒಲಿದಿದೆ. ಇದಲ್ಲದೇ ಒಂದೇ ಮಾತಿನಲ್ಲಿ ಅನೇಕಾರ್ಥವನ್ನು ಹೊಮ್ಮಿಸುವ ವಿಶಿಷ್ಟ ಸಾಮರ್ಥ್ಯ ಸಹ ಅಲ್ಲೂ ಇದ್ದುದು ಇಲ್ಲೂ ಇದೆ. ಹೀಗೆ ಜನ್ಮನಾ ಪ್ರತಿಭೆ\-ಯಿದ್ದ ಗಂಗಣ್ಣ ತನ್ನ ಗುರುಗಳ ಪ್ರತಿಭಾಸಾರವನ್ನೂ  ಅದ್ಭುತವಾಗಿ ಹೀರಿಕೊಂಡಿದ್ದು ನಿಜ.\break ಗುರವಿನೊಂದಿಗೆ ಶ್ಲಿಷ್ಟನಾದಾಗ ಮಾತ್ರ ನಿಜ ಶಿಷ್ಯನಾಗುವುದು ಸಾಧ್ಯವಷ್ಟೆ ! ಅಂತಹ ಸ್ವಭಾವ ಗಂಗಣ್ಣನಿಗೆ ಇದೆ. ಅದೇ ಅಲ್ಲಿಯ ಸಾರ ಇಲ್ಲಿ ಹರಿಯುವುದಕ್ಕೂ ಕಾರಣ\-ವಾಗಿದೆ. ಪರಸ್ಪರ ಪ್ರೀತಿ \enginline{-} ಭಕ್ತಿ ಅದರ ಹರಿಯುವಿಕೆಗೆ ಕಾರಣವಾದ ಅಂಶ. ಅವರ ಮೇಲಿನ ಭಕ್ತಿಯ ಕಾರಣದಿಂದ ಅವರ ಕಷ್ಟಗಳಿಗೆ ಏನು ಬೇಕಾದರೂ ಸಹಾಯ ಇವನಿಂದ ಸಿಗುತ್ತಿತ್ತು. ಅದನ್ನು ಶ್ರೀ ಎನ್.ಎಸ್.ಆರ್ ರವರೇ ಹೇಳುತ್ತಿದ್ದರು. ಇವನ ವ್ಯವಹಾರ ನೈಪುಣ್ಯವನ್ನು, ಕಷ್ಟ ಕಾಲದಲ್ಲಿ ಒದಗಿಬರುವ, ಔದಾರ್ಯದಿಂದ ವರ್ತಿಸುವ ಗುಣ\-ವನ್ನೂ ಅವರು ಪ್ರೀತಿಯಿಂದ ಸ್ಮರಿಸುತ್ತಿದ್ದುದುಂಟು. ಆರ್ಥಿಕವಾಗಿ ಶ್ರೀ ಎನ್.ಎಸ್.ಆರ್ ರವರು ಸಾಕಷ್ಟು ಕಷ್ಟ ಅನುಭವಿಸಿದವರು. \hbox{ಸರಸ್ವತೀ ಲಕ್ಷ್ಮಿಯರಿಗೆ} ಈ ಲೋಕದಲ್ಲಿ ಅಷ್ಟಕ್ಕಷೇ ಎಂಬುದು ಲೋಕವಿದಿತವಷ್ಟೆ ! ಒಮ್ಮೆ ಅವರ \hbox{ಮನೆಯಲ್ಲಿ} ವಿವಾಹ ಸಮಾರಂಭ ಏರ್ಪಾಟಾಗಿತ್ತು. ಅದಕ್ಕಾಗಿ ತಂದ ಅಕ್ಕಿ ಅನಿರೀಕ್ಷಿತವಾಗಿ ಹಿಂದಿನ ರಾತ್ರಿಯೇ ಮುಗಿದು ಹೋಯಿತು. ಏನು ಮಾಡಬೇಕೆಂಬುದೇ \hbox{ಅವರಿಗೆ} \hbox{ತೋಚದೇ} \hbox{ಕಂಗಾಲಾಗಿಬಿಟ್ಟಿದ್ದರು.} ಆದರೆ ಗಂಗಣ್ಣನಿಗೆ ಈ ವಿಷಯ ಹೇಗೆ \hbox{ತಿಳಿಯಿತೋ} \hbox{ತಿಳಿಯದು.} ಬೆಳಗಿನ ಝಾವ ನಾಲ್ಕೂವರೆ ಗಂಟೆಯ ಸಮಯಕ್ಕೆ ಒಂದು \hbox{ಆಟೋದಲ್ಲಿ} \hbox{ಅಗತ್ಯವಿದ್ದಷ್ಟು}  ಅಕ್ಕಿಯನ್ನು ಅವರ ಮನೆಗೆ ತೆಗೆದುಕೊಂಡು ಹೋಗಿ ಗಂಗಣ್ಣ \hbox{ಕೊಟ್ಟುಬಂದ.} ಇದು ಎಲ್ಲರಿಗೂ ಆಶ್ಚರ್ಯವನ್ನುಂಟುಮಾಡಿತು. ಇದರಿಂದ ಅಂದು ಗಂಗಣ್ಣ ಅಯಾಚಿತ\-ವಾಗಿ ಅವರ ಮನೆಯ ಮರ್ಯಾದೆ \hbox{ಉಳಿಸಿದ್ದ,} ಎಂಬ ಈ ಘಟನೆಯನ್ನು ಇತ್ತೀಚೆಗೆ ಎನ್.ಎಸ್.ಆರ್ ರವರ ಸುಪುತ್ರಿ ಕಲ್ಯಾಣಿ\-ಯವರು \hbox{ಗಂಗಣ್ಣನ} ವದಾನ್ಯ  ಸ್ವಭಾವವನ್ನು ನನ್ನಲ್ಲಿ ಸ್ಮರಿಸಿಕೊಂಡಿದ್ದಾರೆ. ಇದಲ್ಲದೇ ಪಾಠಶಾಲೆ\-ಯಲ್ಲಾಗಲೀ ಬೇರೆಡೆಯಲ್ಲಾಗಲೀ ಯಾವ ವಿಷಯದಲ್ಲಿ ಅವರಿಗೆ ಕಿಂಚಿತ್ \hbox{ತೊಂದರೆಯಾದರೂ} ಗಂಗಣ್ಣ ಅವರ ಸಹಾಯಕ್ಕೆ ಒದಗುತ್ತಿದ್ದ. ಹಾಗಾಗಿ ಗುರುಗಳ ಸೇವೆಗೂ ಗಂಗಣ್ಣ \hbox{ಇತರರಿಗೆ} \hbox{ಮಾದರಿಯೇ.} ಅಂತಹ ತನ್ನ ಗುರುಗಳನ್ನು ಪಾಠಶಾಲೆ\-ಯಲ್ಲಿ ಅದ್ಭುತವಾದ \hbox{ಅಭಿವಂದನೆಯ} ಕಾರ್ಯಕ್ರಮದ ಮೂಲಕ \hbox{ಗೌರವಿಸಿದ್ದುಂಟು.} ವಾಸ್ತವವಾಗಿ ನಮ್ಮ ಪರಂಪರೆಯ \hbox{ಅಭಿವಂದನಾ} ಕಾರ್ಯಕ್ರಮ ಅಂದೇ ಆರಂಭವಾಗಿದೆ ಎಂದರೆ ತಪ್ಪಿಲ್ಲ. ಆ ಕಾರ್ಯಕ್ರಮವನ್ನು ಜೀವನದಲ್ಲಿ ಮರೆಯುವಂತಿಲ್ಲ. ಅಂದು ಆ ಕಾರ್ಯಕ್ರಮದ ಅಧ್ಯಕ್ಷತೆಯನ್ನು \hbox{ಎನ್. ಬಾಲಸುಬ್ರಹ್ಮಣ್ಯರವರು} ವಹಿಸಿದ್ದರು. ಕಾರ್ಯಕ್ರಮ ನಿರೂಪಣೆ ಗಂಗಣ್ಣನದೇ ಆಗಿತ್ತು. ಒಂದು ವಿಶೇಷವೆಂದರೆ ಅಧೈರ್ಯ ಎಂಬುದೇ ಇಲ್ಲದ ಗಂಗಣ್ಣನಿಗೆ ಗುರುಗಳೆದುರು ಮತ್ತು ತಂದೆ \hbox{ವಿಘ್ನೇಶ್ವರ} ಭಟ್ಟರೆದುರು ಮಾತನಾಡುವುದಕ್ಕೆ ಮಾತ್ರ ಭಯವಂತೆ. ಗಂಗಣ್ಣ ಅಂದು ಹೇಳಿರುವುದು ಇಂದೂ ನನಗೆ ಜ್ಞಾಪಕ\-ಇದೆ. ಹಾಗಾಗಿ ಅಂದು ಪ್ರತ್ಯೇಕವಾಗಿ ಮಾತನಾಡದೇ ನಿರ್ವಹಣೆ ಮಾತ್ರ ನಡೆದಿತ್ತು. ಅಭಿವಂದಿಸಲ್ಪಟ್ಟ ಎನ್.ಎಸ್.ಆರ್ ರವರ ಅಂದಿನ ಗಂಭೀರ ಮಾತು, ಅದಕ್ಕೆ \hbox{ತಕ್ಕನಾದ} ಎನ್.ಬಾಲಸುಬ್ರಹ್ಮಣ್ಯರವರ ಮಾತನ್ನು ಮರೆಯುವಂತಿಲ್ಲ. ನಡೆದ ಕಾರ್ಯಕ್ರಮ ಅದ್ಭುತವೂ ಅತ್ಯಂತ \hbox{ಗಂಭೀರವೂ} ಆಗಿತ್ತು. ಅದು ಎನ್.ಎಸ್.ಆರ್ \hbox{ರವರಿಗೆ} ಅತ್ಯಂತ ಸಂತೋಷವನ್ನು \hbox{ಉಂಟುಮಾಡಿತ್ತು.} ಕಾರ್ಯಕ್ರಮ ಮುಗಿದ ಮೇಲೆ ಗಂಗಣ್ಣ \hbox{ಅವರಲ್ಲಿ,} “ನಾನೇನೂ \hbox{ಮಾತನಾಡಲಾಗಲಿಲ್ಲ"} ಎಂದು ಕೊಂಚ ವಿಷಾದದಿಂದ ಹೇಳಿದ, ತಕ್ಷಣ ಅವರು, “ಕಾರ್ಯಕ್ರಮದ ನಿರೂಪಣೆಯಲ್ಲಿ ನೀನಾಡಿದ ಮಾತುಗಳನ್ನೆಲ್ಲ \hbox{ಸೇರಿಸಿದರೆ} ಒಳ್ಳೆಯ ಪ್ರವಚನವೇ \hbox{ಆಗುವಂತಿತ್ತಪ್ಪ~!} ಎಂದು ಅವರು \hbox{ಉದ್ಗರಿಸಿ} ಸಂತೋಷ ಪಟ್ಟಿದ್ದರು, ಸಂತೋಷವನ್ನು \hbox{ಉಂಟುಮಾಡಿದ್ದರು.} ಮಾತ್ರವಲ್ಲ, ಬಹಳ ಪ್ರೀತಿಯಿಂದ \hbox{ಕೆಂಪು} ಬಣ್ಣದ ಒಂದು ಉತ್ತಮವಾದ \hbox{ಶಾಲನ್ನು} ಗಂಗಣ್ಣನಿಗೆ ಆಶೀರ್ವಾದ ಮಾಡಿ \hbox{ಉಡುಗರೆಯಾಗಿ} \hbox{ಕೊಟ್ಟರು.} ಆಗ ಗಂಗಣ್ಣ, “ಹಿಂದೆ ನನಗೆ ವಿದ್ವತ್ ಶಾಲು \hbox{ಬಂದಿದ್ದರೂ} ಅದು ವಿದ್ವತ್ ಶಾಲಲ್ಲ, ನಿಜವಾದ ವಿದ್ವತ್ ಶಾಲು ಈಗ  ಬಂತು” ಎಂದು ಹೇಳಿದ್ದ \enginline{-} ಅವನ ತೃಪ್ತಿಗೆ ಪಾರವಿರಲಿಲ್ಲ. ಬಹುಶಃ ಇಂದೂ ಆ \hbox{ಶಾಲನ್ನು} \hbox{ಅತಿಯಾದ} ಗೌರವದಿಂದ {(ಒಮ್ಮೆಯೂ ಬಳಸದೆ !!?)} ಪೆಟ್ಟಿಗೆಯಲ್ಲಿ \hbox{ಜತನದಿಂದ} ಕಾಪಾಡಿಕೊಂಡಿದ್ದಾನೆ !! ಪ್ರಕೃತ, ಯಾವ ಪದಾರ್ಥ ಕೊಟ್ಟಿದ್ದು ಎಂಬುದು ಅದರ ಬೆಲೆಗೆ ಮಾನದಂಡವಲ್ಲ. ಯಾರು ಯಾವ ಭಾವದಿಂದ ಕೊಟ್ಟಿದ್ದು ಎಂಬುದೇ\break ಮಾನ್ಯತೆಗೆ ಮಾನದಂಡ. ಹಾಗಾಗಿ ಶಿಷ್ಯನಿಗೆ ಗುರು ಪ್ರಸಾದ ಭಾವದಿಂದ ಕೊಟ್ಟ\break ಪದಾರ್ಥಕ್ಕೆ ದಕ್ಕುವ ಕಿಮ್ಮತ್ತು ಉಳಿದವರು ಕೊಟ್ಟರೆ ಸಿಗುವುದು ಕಷ್ಟ. 
ಗುರುವಿನ ತೃಪ್ತಿಯ ಮಹತ್ತ್ವವನ್ನು ಶಾಸ್ತ್ರ ಹೀಗೆ ಹೇಳುತ್ತದೆ \enginline{-} 
\begin{verse}
ಗುರೌ ತುಷ್ಟೇ ತು ತುಷ್ಟಾಸ್ಸ್ಯುಃ ಸರ್ವೇ ದೇವಾಃ ಸವಾಸವಾಃ  ।\\ 
ಗುರೌ ರುಷ್ಟೇ ತು ರುಷ್ಟಾಸ್ಸ್ಯುಃ ಸರ್ವೇ ದೇವಾಃ ಸವಾಸವಾಃ ॥
\end{verse}
ಶಿಷ್ಯನ ಜೀವನವೇ ಗುರುವಿನ ತುಷ್ಟಿ, ರುಷ್ಟಿಗಳನ್ನವಲಂಬಿಸಿದೆ. “ಗುರುವು ಸಂತುಷ್ಟ\-ನಾದರೆ ದೇವತೆಗಳೆಲ್ಲ ಸಂತುಷ್ಟರಾಗುತ್ತಾರೆ, ಗುರುವು ಅಸಮಾಧಾನಗೊಂಡರೆ ದೇವತೆ\-ಗಳೂ ಹಾಗೆಯೇ” ಎನ್ನುತ್ತದೆ ಮೇಲಿನ ಶ್ಲೋಕ. ಇದು ಆತ್ಮಜ್ಞಾನವನ್ನು ಅನುಗ್ರಹಿಸುವ ಗುರುವಿಗೆ ನೇರವಾಗಿ ಅನ್ವಯಿಸುವ ವಿಷಯವಾದರೂ ಗುರು ಶಿಷ್ಯ \hbox{ಪರಂಪರೆ} ಅಂತಹ ಗುರುವಿನಿಂದಲೇ ಆರಂಭವಾಗಿ ಅದರ ಅಂಶವೇ ಪರಂಪರೆಯಲ್ಲಿ \hbox{ಹರಿದು} ಬರುವುದರಿಂದ  ಮೂಲವನ್ನು ಜ್ಞಾಪಿಸಿಕೊಳ್ಳುವಲ್ಲಿ ಅನೌಚಿತ್ಯವಿಲ್ಲ. ಅದನ್ನು\break ಜ್ಞಾಪಿಸಿಕೊಳ್ಳುತ್ತಲೇ ಸಾಗಿದಾಗ ಮಾತ್ರ ಪರಂಪರೆ ಎಂಬುದಕ್ಕೆ ಅರ್ಥ, ಇಲ್ಲದಿದ್ದರೆ ಅದು ವ್ಯರ್ಥ. ಪ್ರಕೃತ, ಶಾಲು, ಅಭಿವಂದನೆ ವಿಷಯ ಹೇಳುವುದೇ ಇಲ್ಲಿಯ ಉದ್ದೇಶವಲ್ಲ. ಅದರ ಹಿಂದಿರುವ ಭಾವ ಮತ್ತು ಸಂಬಂಧ ಎಷ್ಟು ಗಂಭೀರವಾದು, ವಾತ್ಸಲ್ಯ ಭರಿತ\-ವಾದುದು, ಭಕ್ತಿ\-ಪೂರ್ವಕವಾದುದು, ಆದರ್ಶಪ್ರಾಯವಾದುದು ಎಂಬುದು ಗಮನಿಸಬೇಕಾದ ಅಂಶ. ಎಲ್ಲರಿಗೂ ಇಂತಹ ಸನ್ನಿವೇಶಗಳು ಜೀವನದಲ್ಲಿ ಒದಗಲಾರವು. ಇಂತಹ ರಸ, ಭಾವಗಳು ಜೀವಿಗಳಿಗೆ ಒಂದು ಬಗೆಯ ಹಿತವಾದ \hbox{ಅಂತಃಸುಖವನ್ನು} ನೀಡಿ  ಜೀವನದಲ್ಲಿ ಒಂದು ಪಾಕವನ್ನು ಉಂಟುಮಾಡುತ್ತವೆ ಎಂಬುದು ಸುಳ್ಳಲ್ಲ. \hbox{ಅಂಥವು} ಅವಶ್ಯ ಪ್ರಾಪ್ತವಾಗಬೇಕಾದವುಗಳೇ ಆಗಿವೆ. ಗಂಗಣ್ಣನ ಜೀವನ ಗುರುವಿನ ಪ್ರಸಾದ \enginline{-} ಪ್ರಸನ್ನತೆ (\textit{ಪ್ರಸಾದಸ್ತು ಪ್ರಸನ್ನತಾ \enginline{-} ಅಮರಕೋಶ})ಯಿಂದಲೇ ನಿರೂಪಿತ\-ವಾಗಿರು\-ವುದು ಸ್ಪಷ್ಟ. ಹಾಗಿರುವುದರಿಂದ ಈ ಲೇಖನಕ್ಕೆ ಅದೇ ನಿಜವಾದ ಹಿನ್ನೆಲೆ. \hbox{ಇಷ್ಟಲ್ಲದೇ} ನನ್ನ ಬದುಕಿಗೂ ಸಹ ಇದೇ ಪ್ರಸಾದವೇ ಮೂಲವಾಗಿದೆ.(ಗಂಗಣ್ಣ ಮೈಸೂರಿನಲ್ಲಿಲ್ಲದಿದ್ದರೆ ನಾನು ಮೈಸೂರಿಗೆ ಬರುತ್ತಲೇ ಇರಲಿಲ್ಲ ಎಂಬುದನ್ನು ಈ ಮೊದಲೇ \hbox{ಹೇಳಿದ್ದೇನೆ}) ಒಂದುವೇಳೆ ಅದಿಲ್ಲದಿದ್ದರೆ ಇಲ್ಲಿ ಹೇಳಬೇಕಾದ ಯಾವ ವಿಷಯವೂ ಇಲ್ಲ. ಅದಕ್ಕಾಗಿಯೇ ಲೇಖನದ ತಲೆ\-ಬರಹವೂ ಗುರುಪ್ರಸಾದ ಎಂಬುದಾಗಿದೆಯೇ ವಿನಾ ನನ್ನ \hbox{ಹೆಸರಿನ} ಅಭಿಮಾನದಿಂದಲ್ಲ. ಗುರುಗಳ ಪ್ರಸಾದ ಗಂಗಣ್ಣನಿಗುಂಟು. ಅವರಿಬ್ಬರ\break ಪ್ರಸಾದವೂ ನನ್ನ ಮೇಲುಂಟು. ಆ ಕಾರಣದಿಂದಲೇ ನನಗೊಂದು ಬದುಕು \hbox{ಮೈಸೂರಲ್ಲಿ} ಸಾಧ್ಯವಾಯಿತು. ಹಾಗಾಗಿ ಗುರು\enginline{-}ಪ್ರಸಾದವೇ ಈ ಲೇಖನದ ಪ್ರಧಾನ ತಿರುಳು. 

\smallskip
ಗಂಗಣ್ಣನ ಸಹಾಯ ಸ್ವಭಾವಕ್ಕೆ ವ್ಯಕ್ತಿಗತ ಸೀಮೆಯಿರಲಿಲ್ಲ. ಪಾಠಶಾಲೆಯಲ್ಲಿ ಇನ್ನೊಬ್ಬ ಅಧ್ಯಾಪಕರಾಗಿದ್ದ ಶ್ರೀ ವೆಂಕಣ್ಣಾಚಾರ್ಯರ ಸಂಪೂರ್ಣ ವ್ಯವಹಾ\-ರಕ್ಕೆ  ಅವರು \hbox{ಗಂಗಣ್ಣನನ್ನೇ} ಅವಲಂಬಿಸಿದ್ದರು. ಅವರ ಮಕ್ಕಳಿಗಿಂತಲೂ ಗಂಗಣ್ಣನನ್ನೇ ಅವರು ನೆಚ್ಚಿ\-ಕೊಂಡಿದ್ದರು. ಶ್ರೀ ವಿಶ್ವೇಶ್ವರ ದೀಕ್ಷಿತರಿಗೂ ಅಲ್ಲದೆ ಇನ್ನೂ ಅನೇಕ ಅಧ್ಯಾಪಕರು\-ಗಳಿಗೂ ಅವನ ಸೇವಾರೂಪದ ಸಹಾಯ ಹಸ್ತ ಇದ್ದೇ ಇತ್ತು.  ಹೀಗೆ ಗಂಗಣ್ಣ ಅಧ್ಯಯನ, ಅಧ್ಯಾಪನ, ಸೇವೆ, ಸಹಾಯ, ವ್ಯವಹಾರ ಯಾವುದರಲ್ಲೂ ಮಾದರಿ\-ಯಾಗಿ ನಿಲ್ಲುವ ಗುಣಗ\-ಳುಳ್ಳವನೆಂಬುದರಲ್ಲಿ ಅನುಮಾನವಿಲ್ಲ, ಆ ಗುಣಗಳು ಸ್ವಾರ್ಥದ ಲವಲೇಶವೂ ಇಲ್ಲದೆ ಶುದ್ಧವಾಗಿ ಇರುವಂಥವುಗಳು ಎಂಬುದು  ಮುಖ್ಯವಾಗಿ ಗಮನಿಸಬೇಕಾದ ಅಂಶ. ಯಾಕೆಂದರೆ ಎಲ್ಲೆಲ್ಲೂ ಸ್ವಾರ್ಥವೇ ವಿಜೃಂಭಿಸುವುದನ್ನು ನಾವು ಕಾಣುತ್ತೇವೆ. ಇಲ್ಲಿ ಅದರ ಲವಲೇಶವೂ ಇಲ್ಲ.  ಈ ಸಂದರ್ಭಕ್ಕೆ ಕಾಳಿದಾಸನ \enginline{-} "\hbox{ಗುಣಾ} ಗುಣಾನುಬಂಧಿತ್ವಾತ್ ತಸ್ಯ ಸಪ್ರಸವಾ ಇವ" ಎಂಬ ಮಾತು ಮನಸ್ಸಿನಲ್ಲಿ ಹಾದು ಹೋಗುತ್ತದೆ. 

ಗಂಗಣ್ಣನ ಜಾತಕದಲ್ಲಿ  ಬುಧನದೇ ಸಾಮ್ರಾಜ್ಯ. ಅವನು ಪಂಚಮಸ್ಥಾನದಲ್ಲಿ  ಯಾವುದೇ ಪಾಪಗ್ರಹಗಳ ಯೋಗ, ದೃಷ್ಟಿಗಳಾವುದೂ ಇಲ್ಲದೇ ಪರಿಶುದ್ಧನಾಗಿ \hbox{ನಿಂತಿದ್ದಾನೆ.}  ಬುಧ ಸ್ವತಃ ಶುಭಗ್ರಹ, ಆದರೆ ಪಾಪಗ್ರಹಗಳ ಜೊತೆಗೂಡಿದರೆ \hbox{ಅವನೂ} ಪಾಪನಾಗಿಬಿಡುತ್ತಾನೆ. ಆದರೆ ಇಲ್ಲಿರುವ ಬುಧ ಶುದ್ಧನಾಗಿದ್ದಾನೆ. ವಾತ,ಪಿತ್ತ, ಶ್ಲೇಷ್ಮ\-ಗಳೆಂಬ ಮೂರೂ ಧಾತುಗಳ ಸಮಾನ ಹದವುಳ್ಳ ಪ್ರಕೃತಿ ಬುಧನದು, \hbox{ಯುಕ್ತಿಯುಕ್ತ} ಮತ್ತು ಶ್ಲೇಷಯುಕ್ತ ಮಾತು, ತಿಳಿ ಹಾಸ್ಯ, ಪಾಂಡಿತ್ಯ, ಕಲಾ ನೈಪುಣ್ಯ, ಸ್ನೇಹ ಇವೆಲ್ಲ ಬುಧ ಸೂಚಿ\-ಸುವ ಗುಣಗಳು. ಅಂತಹ ಬುಧನ ಸ್ಥಿತಿ ಗಂಗಣ್ಣನ ವ್ಯಕ್ತಿತ್ವವನ್ನು ಸ್ಪಷ್ಟವಾಗಿ ನಿರೂಪಿ\-ಸುತ್ತಿದೆ. ಮೇಲೆ ಹೇಳಿದ ಎಲ್ಲ ಅಂಶಗಳು ಅವನಲ್ಲಿ \hbox{ಪ್ರಭೂತವಾಗಿ} \hbox{ಗೋಚರಿಸುತ್ತವೆ}. ಅವನ ಕಲಾಭಿವ್ಯಕ್ತಿ ಇತ್ತೀಚಿನ ಜನರಿಗೆ ಪರಿಚಯವಿದೆಯೋ ಇಲ್ಲವೋ ಗೊತ್ತಿಲ್ಲ. ಹಿಂದೆ ಕೆಲವು ಕಾವ್ಯ, ನಾಟಕಗಳು \hbox{ಗಂಗಣ್ಣನ} ಪಾತ್ರ ಮತ್ತು ನಿರ್ದೇಶನದಲ್ಲಿ ರಂಗದ ಮೇಲೆ ಪ್ರಯೋಗವಾಗಿವೆ. ನಾನೂ ಸಹ \hbox{ಗಂಗಣ್ಣನ} ಪಾತ್ರ ಮತ್ತು ಸೂತ್ರಧಾರಿಕೆಯಲ್ಲಿ ನಡೆದ ತಾಳಮದ್ದೆಲೆ ರೀತಿಯ ಕಾರ್ಯಕ್ರಮದಲ್ಲಿ ಭಾಗವಹಿಸಿದ್ದೆ. ಆ ರಂಗದ ಮೇಲೆ ನನ್ನ ಪಾಲಿಗೆ ಕೃಷ್ಣ ಒಲಿದಿದ್ದ.  ನಾವೆಲ್ಲ ಅವನ \hbox{ನಿರ್ದೇಶನದಲ್ಲಿ} ಕಾರ್ಯಕ್ರಮವನ್ನು ಚೆನ್ನಾಗಿ ನಿರ್ವಹಿಸಿದ್ದೆವು. ಗಂಗಣ್ಣ ಯಕ್ಷಗಾನ ತಾಳಮದ್ದಲೆಯಲ್ಲೂ ಭಾಗವಹಿಸುತ್ತಿದ್ದ. ಅವನ ಮಾತಿನ ಕೌಶಲ, ಪ್ರತಿಭೆ, \hbox{ಉತ್ತರೋತ್ತರ} \hbox{ಯುಕ್ತಿಯ} ಸಾಮರ್ಥ್ಯ, ಆ ರಂಗದಲ್ಲಿ ಚೆನ್ನಾಗಿ ಪ್ರತಿಫಲಿಸುತ್ತಿತ್ತು. ಒಮ್ಮೆ \hbox{ದುರ್ಯೋಧನನ} ಪಾತ್ರ ಮಾಡಿದಾಗ ಕೃಷ್ಣನಿಗೆ ಪ್ರಶ್ನೆಯೊಂದನ್ನು ಕೇಳಿದ್ದ. ಅದಕ್ಕೆ ಕೃಷ್ಣನ ಪಾತ್ರ ಧಾರಿಗಳಿಗೆ ಉತ್ತರಕೊಡಲಾಗಲಿಲ್ಲ. ಅನತಿಕಾಲದ ಇನ್ನೊಂದು ಪ್ರಸಂಗ\-ದಲ್ಲಿ \hbox{ಪಾತ್ರಗಳು} ಬದಲಾಗಿ ಆಗ ಕೃಷ್ಣನ ಪಾತ್ರ ಮಾಡಿದವರು ಈಗ ದುರ್ಯೋಧನನ ಪಾತ್ರವನ್ನು ಮಾಡಿದ್ದರು. ಗಂಗಣ್ಣ ಕೃಷ್ಣನಾಗಿದ್ದ. ಹಿಂದೆ ಕೃಷ್ಣನಾಗಿ ಉತ್ತರ ಕೊಡಲಾಗದ\break ಪ್ರಶ್ನೆಯನ್ನೇ ಈಗ ದುರ್ಯೋಧನ ಪಾತ್ರಧಾರಿಗಳು ಕೃಷ್ಣ(ಗಂಗಣ್ಣ)ನೆಡೆಗೆ ಎಸೆದರು. ಆದರೆ ಅದಕ್ಕೆ ತಕ್ಕ ಉತ್ತರ ಕಷ್ಣನ ಬತ್ತಳಿಕೆಯಲ್ಲಿತ್ತು. ಪಾತ್ರದ ತನ್ಮಯತೆ ಮಾತ್ರ ಈ ರೀತಿಯಲ್ಲಿ ಪ್ರಶ್ನೋತ್ತರವನ್ನು ಸೃಷ್ಟಿಸಬಲ್ಲದು. ಅಂಥ ಅಸಾಧಾರಣ ಪ್ರತಿಭೆ \hbox{ಗಂಗಣ್ಣನಿಗಿತ್ತು.} ಈಗ ವಿದ್ಯಾರ್ಥಿಗಳ ಕಡೆಯಿಂದ ಸಂಪನ್ನವಾದ ಅಭಿವಂದನ ಸಮಾರಂಭದಲ್ಲೂ  ಅವನ ವಾಕ್ ಕೌಶಲಕ್ಕೆ ನಿದರ್ಶನವಾಗುವ ಅಂತಹದ್ದೇ ಒಂದು ಪ್ರಸಂಗ ನಡೆದಿದೆ. ಸಭೆಯಲ್ಲಿ ಜನರು ಅದನ್ನು ಗಮನಿಸಿರಲಿಕ್ಕೂ ಸಾಕು. ಆ ಸಂದರ್ಭದಲ್ಲಿ  ಶ್ರೀಯುತ ಉಮಾಕಾಂತ ಭಟ್ಟರು ಮಾತನಾಡುವಾಗ ಗಂಗಣ್ಣನ ಮತ್ತು ತಮ್ಮ ಮಧ್ಯದಲ್ಲಿ, ಬಾಲ್ಯದಿಂದ ಈ ತನಕ ಅವಿವಾದಿತ ಸ್ಪರ್ಧೆಯಿರುವುದನ್ನು ಜ್ಞಾಪಿಸಿಕೊಳ್ಳುತ್ತಾ ಹೇಳಿದರು, “ಸ್ಪರ್ಧೆಗಳಲ್ಲಿ \hbox{ಗಂಗಾಧರ} ಪ್ರಥಮ ಬಹುಮಾನವನ್ನ ನನಗೆ ಪಡೆಯಲು ಬಿಡುತ್ತಿರಲಿಲ್ಲ. ಅವನ ವಾಕ್ \hbox{ಕೌಶಲವೇ} ಹಾಗೆ, ನನಗೆ ವಿಶ್ವಾಸವಿದೆ, ಈಗಲೂ ಗಂಗಾಧರ ನನ್ನ ಮಾತು ಮುಗಿದ ಮೇಲೆ, ನನ್ನ ಮಾತು ಮರೆಯುವ ಹಾಗೆ ಮಾತನಾಡುತ್ತಾನೆ” ಎಂದು, ಜಾಣ್ಮೆಯ ಅಷ್ಟೇ\break ಚೋದ್ಯವಾದ ಮಾತನ್ನಾಡಿದರು. ಅನಂತರ ಗಂಗಣ್ಣ ತಾನು ಮಾತನಾಡುವಾಗ\enginline{-} “ಉಮಾಕಾಂತ ಹೇಳಿದ್ದಾನೆ, ತನ್ನ ಮಾತು ಮರೆಯುವಂತೆ ನಾನು ಮಾತನಾಡುತ್ತೇನೆ ಎಂದು, ಆದರೆ ಇಂದು ನಾನು ಎಲ್ಲವನ್ನೂ ಮರೆತಿದ್ದೇನೆ” ಎನ್ನುತ್ತಾನೆ. ಇದು ಆ ಸಂದರ್ಭಕ್ಕೆ ಸಭೆಯ ಮತ್ತು ಅವನ, ಅಲ್ಲದೇ ಆ ಕಾರ್ಯಕ್ರಮದ ಭಾವಕೇಂದ್ರವನ್ನು ಅಭಿವ್ಯಂಜಿಸುವ ಬಹ್ವರ್ಥಗರ್ಭಿತ ಧ್ವನಿಪೂರ್ಣ ಮಾತಾಗಿತ್ತು. ಆಗ ಸ್ಪರ್ಧೆಗಾಗಿ ಮಾತನಾಡುವ ಭಾವ ಅಲ್ಲಿರಲಿಲ್ಲ. ತನ್ನ ಬಗ್ಗೆ ವಿದ್ಯಾರ್ಥಿಗಳ ಅಭಿವಂದನೆಯ ರಸ-ಭಾವ-ಭಾರ ತುಂಬಿದ ಭಾವುಕವಾದ ಸನ್ನಿವೇಶ  ಅದಾಗಿತ್ತು. ಭಾವುಕ ಸ್ಥಿತಿ ಭಾವುಕನನ್ನು ಮೌನ\-ದೆಡೆಗೇ ಸೆಳೆಯುವುದು ಅದರ ಸ್ವಭಾವವಷ್ಟೇ ! ಅವನ ಮಾತು ಅದನ್ನೇ ಸ್ವಾರಸ್ಯವಾಗಿ ಧ್ವನಿಸುತ್ತಿತ್ತು. ಒಂದೇ ಮಾತಿನಿಂದ ಹೇಳಬೇಕಾದ್ದೆಲ್ಲವನ್ನೂ ಅವನು ಧ್ವನಿಗರ್ಭಿತವಾಗಿ \hbox{ಹೇಳಿಬಿಟ್ಟಿದ್ದ.} ಹೀಗೆ ಅವನ ಒಡನಾಟದಲ್ಲಿ ನಮಗೆ ಇಂತಹ ಧ್ವನಿಪೂರ್ಣ ಮಾತು ಮತ್ತು ವ್ಯವಹಾರಗಳ ಸನ್ನಿವೇಶ ಅಪರೂಪವಾಗಿರಲಿಲ್ಲ. 

ಸ್ವಭಾವವಾಗಿ ಗಂಗಣ್ಣನಲ್ಲಿ ಕರುಣ ರಸ ಮತ್ತು ವೀರ ರಸಗಳು ಬಹಳ ಚೆನ್ನಾಗಿ ಅಭಿವ್ಯಕ್ತವಾಗುತ್ತವೆ. ಹಾಸ್ಯ ರಸ ತಾನೇ ತಾನಾಗಿ ಏರ್ಪಡುತ್ತದೆ. ಶೃಂಗಾರ ಭಾವ ಬಹಳ ಸುಪ್ತವಾಗಿದೆ. ಗಂಭೀರ ಸನ್ನಿವೇಶವಿರುವಾಗ ಅನೇಕರಲ್ಲಿ ಹಾಸ್ಯ ಶೃಂಗಾರಗಳು\break ಮಿಶ್ರವಾಗಿ, ಎರಡರ ಹದವೂ ತಪ್ಪಿ ಕ್ಷಣಾರ್ಧದಲ್ಲಿ ವ್ಯಕ್ತಿಯನ್ನೂ  ಗಂಭೀರ \hbox{ಸನ್ನಿವೇಶ} ವನ್ನೂ ಪ್ರಪಾತಕ್ಕೆ ತಳ್ಳಿಬಿಡುವುದನ್ನು  ಸಭೆ ಸಮಾರಂಭಗಳಲ್ಲಿ ನಾವು ನೋಡುತ್ತೇವೆ. ಆದರೆ ಗಂಗಣ್ಣನಲ್ಲಿ ಯಾವ ರಸವೂ ವಿರಸವಾಗದೇ ಸರಿಯಾದ ಹದವನ್ನು ಕಾಯ್ದು\-ಕೊಂಡಿವೆ, ಅದರಲ್ಲೂ ಶೃಂಗಾರವಂತೂ ಅತ್ಯಂತ ಸುಪ್ತವಾಗಿದೆ‘. ಷಷ್ಠದ ಶುಕ್ರ ಅದನ್ನು ಹಾಗೆ ಹದಗೊಳಿಸಿದ್ದಾನೆ.  ಅದೆಲ್ಲೂ ಅಷ್ಟಾಗಿ ಪ್ರಕಟವಾಗಿದ್ದನ್ನು ನಾನು ಕಂಡಿಲ್ಲ. ಹಾಗೆಂದು ಅದು ಇರಬಾರದ್ದೆಂದು ಲೇಖಕನ ಅಭಿಪ್ರಾಯವಲ್ಲ. ರಸರಾಜ ಎಂಬ ಕೀರ್ತಿ ಶೃಂಗಾರಕ್ಕಿದ್ದೇ ಇದೆ. ಆದರೆ ಅದು ಗಂಭೀರ ರಾಜನಂತಿರಬೇಕು! ಹದ \hbox{ತಪ್ಪಿದರೆ} ರಾಜ ರಾಜ್ಯವನ್ನೇ ಕಳೆದುಕೊಳ್ಳುವ ಅಪಾಯವಿರುತ್ತದೆ. ಇನ್ನು, ಗಂಗಣ್ಣನಲ್ಲಿರುವ \hbox{ಅನಾಲಸ್ಯ,} ಉತ್ಸಾಹ, ಧೈರ್ಯ, ಮತ್ತು ಔದಾರ್ಯವನ್ನು ಸುಸ್ಥನಾದ ಕುಜ \hbox{ಕೊಟ್ಟಿದ್ದಾನೆ.} \hbox{ಧೈರ್ಯಂ} ಸರ್ವತ್ರ ಸಾಧನಂ ಎಂಬ ಮಾತು ಗಂಗಣ್ಣನಲ್ಲಿ ಸಾಕ್ಷಾತ್ತಾಗಿ ಕಾಣುವ ಗುಣ. \hbox{ಅಧೈರ್ಯ} ಎಂಬ ಪದವೇ ಅವನ ಕೋಶದಲ್ಲಿಲ್ಲ. ಬುದ್ಧಿಸೂಕ್ಷ್ಮತೆ, ಸ್ವಾಭಿಮಾನ, ಸ್ವದೇಶಭಕ್ತಿ, ವಿಭಿನ್ನ ಕ್ಷೇತ್ರದಲ್ಲಿ ಆಸಕ್ತಿ ಮತ್ತು ಸಮಾನ ಪರಿಣತಿ, ಪ್ರಾಚೀನತೆಯನ್ನು ಬಿಡದ ಆಧುನಿಕತೆ, \hbox{ಸಂಸ್ಕೃತಿಯನ್ನು} ಬಿಡದ ವಿಜ್ಞಾನಾಸಕ್ತಿ ಇವೆಲ್ಲ ಅವನಲ್ಲಿ ಮೇಳೈಸಿವೆ. ಈ ಎಲ್ಲ ಗುಣಗಳಿಂದ ಅವನು ಒಬ್ಬ ವ್ಯಕ್ತಿಯಾಗಿರದೇ ಅನೇಕ ವ್ಯಕ್ತಿಗಳ ಸಂಗಮವಾಗಿ ಕಾಣುತ್ತಾನೆ. ಪ್ರಭೂತವಾದ ಪ್ರತಿಭೆಗಳ ಆಕರವಾಗಿದ್ದಾನೆ. ಹಾಗಿದ್ದೂ ಅವನು ತನ್ನ \hbox{ಪ್ರತಿಭೆಯನ್ನು} ಅಕಾಲ, ಅದೇಶ, ಅಪಾತ್ರದಲ್ಲಿ ಚೆಲ್ಲುವ \hbox{ಚಪಲವಿರುವವನಲ್ಲ.} ಒಂದೇ \hbox{ಮಾತಿನಲ್ಲಿ} ಹೇಳುವುದಾದರೆ ನಾನು ಗಂಗಣ್ಣನನ್ನು "ಪ್ರತಿಭಾಸಂಯಮೀ" ಎನ್ನುತ್ತೇನೆ. ಯಾವ ಯೋಗ್ಯತೆಯೂ ಇಲ್ಲದೇ ಮೇರೆ ಮೀರಿ ಮೆರೆಯುವವರು ಎಲ್ಲೆಲ್ಲೂ ಕಾಣುವ ಈ \hbox{ಕಾಲದಲ್ಲಿ} ಗಂಗಣ್ಣ ಮಾತ್ರ ಅತ್ಯಂತ ಅಪರೂಪದ ವ್ಯಕ್ತಿಯಾಗಿ ನಿಲ್ಲುತ್ತಾನೆ.  

ಇಂತಹ ಅಂಶಗಳಿಂದ ರೂಪುಗೊಂಡ ವ್ಯಕ್ತಿತ್ವವುಳ್ಳ ಗಂಗಣ್ಣನ \hbox{ಅಯಾಚಿತ} ಕೀರ್ತಿ ಅತಿ ದೂರದವರೆಗೂ ವ್ಯಾಪಿಸಿದೆ. ಒಮ್ಮೆ ನಾನು ಆಂಧ್ರದ ತೆನಾಲಿಗೆ \hbox{ಪರೀಕ್ಷೆಗೆ} ಹೋದಾಗ ಅಲ್ಲಿಯ ವಿದ್ವಾಂಸರೊಬ್ಬರು, “ನೀನು ಯಾರ \hbox{ವಿದ್ಯಾರ್ಥೀ?”} ಎಂದು ಕೇಳಿದಾಗ ನಾನು ಗಂಗಾಧರ ಭಟ್ಟರ ವಿದ್ಯಾರ್ಥೀ ಎಂದೆ". “ತೇ ತು \hbox{ನಿತರಾಂ} \hbox{ತಾರ್ಕಿಕಾ} ಭವಂತಿ” ಎಂದು ಅವರು ಉದ್ಗರಿಸಿದ್ದರು. ಅಲ್ಲಿಯವರಿಗೂ ಅವನ \hbox{ಪರಿಚಯವಿತ್ತು.} ಈಗ ಅದು ವಿದೇಶದ ವರೆಗೂ ವ್ಯಾಪಿಸಿರುವುದು ಗೊತ್ತೇ ಇದೆ.  ಇನ್ನೊಂದು ತಮಾಷೆಯಾದ ಸಣ್ಣ ಘಟನೆ ಸ್ಮರಣೆಯಲ್ಲಿದೆ. ಒಮ್ಮೆ \hbox{ಮೈಸೂರಿನಲ್ಲಿರುವ} ಮನೆಗೆ ಯವುದೋ ಊರಿನಿಂದ ಪತ್ರವೊಂದು ಬಂದಿತ್ತು. ಅದರಲ್ಲಿ ಅಡ್ರೆಸ್ \enginline{-}\break “ಗಂಗಾಧರ ಭಟ್, ಮೈಸೂರು” ಎಂಬುದನ್ನು ಬಿಟ್ಟು ಮತ್ತೇನೂ ಇರಲಿಲ್ಲ.  ಅಡ್ರೆಸ್\break ವ್ಯವಸ್ಥಿತವಾಗಿದ್ದರೇ ಮನಗೆ ಪತ್ರ ಬರುವ ಭರವಸೆ ಇಲ್ಲ. ಹಾಗಿದ್ದೂ ಸಮಯಕ್ಕೆ\break ಸರಿಯಾಗಿ ಅದು ಮನೆಗೆ ಬಂದಿತ್ತು. ಗಂಗಣ್ಣನ ಬಹು ವ್ಯಾಪಕತೆಯನ್ನು ನಾವೆಲ್ಲ \hbox{ತಮಾಶೆ} ಮಾಡಿ ನಕ್ಕಿದ್ದುಂಟು. ಹೀಗೆ ಘಟನೆಗಳನ್ನು ಹೇಳುತ್ತಾ ಹೋದರೆ ಹೊತ್ತಿದ್ದರೆ ಹೊತ್ತಗೆಯನ್ನೇ ಬರೆಯ\-ಬಹುದು. ಅವನ ವ್ಯಕ್ತಿತ್ವ ಅಷ್ಟು ವ್ಯಾಪಕವೂ ಹೌದು. ಆದರೆ ದೇಶ, ಕಾಲಗಳ ಮಿತಿ ನಮ್ಮನ್ನು ನಿರ್ಬಂಧಿಸುತ್ತದೆ. 

ಇಂತಹ ವ್ಯಕ್ತಿತ್ವಕ್ಕೆ ಒಂದು ಅಭಿವಂದನ ಕಾರ್ಯಕ್ರಮ \hbox{ನಡೆದುದು ಅತ್ಯಂತ} ತೃಪ್ತಿ\-ಯನ್ನುಂಟುಮಾಡುವ ವಿಷಯ. ಈ ಕಾರ್ಯಕ್ರಮದ ಸಂದರ್ಭದಲ್ಲಿ ಉಳಿದ ವಿದ್ಯಾರ್ಥಿ\-ಗಳೆಲ್ಲ ದೂರದಲ್ಲಿದ್ದುದು ಸಾಕಷ್ಟು ಜವಾಬ್ದಾರಿಯನ್ನು ಅಯಾಚಿತ\-ವಾಗಿ ನಾನು ನಿರ್ವಹಿಸಬೇಕಾಯಿತು. ಯಾವುದೇ ಸಂದರ್ಭದಲ್ಲೂ ಕಾರ್ಯಕ್ರಮ ಗಂಗಣ್ಣನ \hbox{ಗಾಂಭೀರ್ಯಕ್ಕೆ,} ಅವನ ಮನೋಧರ್ಮಕ್ಕೆ ಚ್ಯುತಿ ಬರದಂತೆ ನಡೆಯಬೇಕೆಂಬ ಆಶಯವಿತ್ತು. ಗಂಗಣ್ಣ ಮತ್ತು ಶೈಲಜಕ್ಕನೊಡನಿದ್ದ ಒಡನಾಟ, ವಿಶ್ವಾಸಗಳಿಂದ ಅದರ \hbox{ನಿರ್ವಾಹ} ಸುಲಭವಾಯಿತು. ದೂರದಲ್ಲಿದ್ದ ವಿದ್ಯಾರ್ಥಿಗಳೊಂದಿಗೆ ಸಂವಹನ ಸಾಧ್ಯವಾಯಿತು. ದೂರದಿಂದಲೇ ಅವರು ಏನೆಲ್ಲ ಮಾಡಬಹುದೋ ಮಾಡಿದರು. ಕಾರ್ಯಕ್ರಮ ಸಂಪನ್ನವಾಯಿತು. ಕಾರ್ಯಕ್ರಮದ ಯಶಸ್ಸಿನಲ್ಲಿ ಪೂರ್ಣ ತೃಪ್ತಿಯಿಲ್ಲದಿದ್ದರೂ  ಅತೃಪ್ತಿಯಿಲ್ಲ. ತೃಪ್ತಿ\enginline{-}ಅತೃಪ್ತಿಗಳೇನಿದ್ದರೂ ಸಾಪೇಕ್ಷವಾದವು. ಅಷ್ಟಾದರೂ ನನ್ನ ಪಾಲಿಗೆ ಒದಗಿದ್ದುದು, ಕೊಂಚ ಋಣಭಾರ ಇಳಿದದ್ದು ಸಮಾಧಾನ ಕೊಟ್ಟಿದೆ. ಇದಲ್ಲದೇ ಶ್ರೀ ಉಮಾಕಾಂತ ಭಟ್ಟರು ರಚಿಸಿಕೊಟ್ಟ ಸಾಹಿತ್ಯವನ್ನು ಆಧರಿಸಿ ವಿದ್ಯಾರ್ಥಿಗಳ ಭಾವವನ್ನೂ \hbox{ಅದರೊಡನೆ}  ಬೆಸೆದು ಅಭಿವಂದನ ಪತ್ರ ಸಿದ್ಧಪಡಿಸಿ ಸಭೆಯಲ್ಲಿ ವಾಚಿಸಲು ಅವಕಾಶ ಒದಗಿದುದು ನನ್ನ ಧನ್ಯತೆಗೆ ಕಾರಣವಾಯಿತು. ಅದು ಗಂಗಣ್ಣನ ವ್ಯಕ್ತಿತ್ವಕ್ಕೆ ತಕ್ಕ ಗಂಭೀರ ಸಾಹಿತ್ಯವಾಗಿ ರೂಪುಗೊಂಡಿದೆ ಎಂದು ನನ್ನ ಭಾವನೆ. ಅದಲ್ಲದೇ ಈ ಗ್ರಂಥದ ಸಂಪಾದಕನಾಗಿ ಕೆಲಸ\-ಮಾಡಿದ್ದು ಇನ್ನೊಂದು ಭಾಗ್ಯ. ಈ ನೆಪದಲ್ಲಿ ನನಗೆ  ಸಾಹಿತ್ಯ \hbox{ಕೃಷಿ ಆದಂತಾಯಿತು.} 

ಗ್ರಂಥದ ಕೆಲಸ ನಾನಾ ಕಾರಣಗಳಿಂದ ಸಾಕಷ್ಟು ವಿಲಂಬವಾಯಿತು. ಅದಕ್ಕೆ ISBN ಲಭ್ಯವಾಗುವಲ್ಲಿ ನಿಧಾನವಾದುದು ಪ್ರಧಾನ ಕಾರಣ. ಈ ಮಧ್ಯೆ ನನ್ನ ಅತ್ತೆ ಅನಿರೀಕ್ಷಿತವಾಗಿ ಕ್ಯಾನ್ಸರ್ ರೋಗಕ್ಕೆ ತುತ್ತಾಗಿ ಅವರನ್ನು ಸುಮಾರು ಒಂದೂವರೆ ತಿಂಗಳಷ್ಟು ಕಾಲ\break ಮಂಗಳೂರಿನಲ್ಲಿಟ್ಟುಕೊಂಡು ನೋಡಿಕೊಳ್ಳಬೇಕಾಯಿತು. ಕೊನೆಗೂ ಅವರು ಬದುಕ\-ಲಿಲ್ಲ. ಮೃತಿಪೂರ್ವಾಪರ ವ್ಯವಹಾರ - ಆ ಸಂಬಂಧದ ಓಡಾಟ ನನಗೆ ISBN ಇತ್ಯಾದಿ ಗ್ರಂಥಕ್ಕೆ ಸಂಬಂಧಿಸಿದ ವ್ಯವಹಾರವನ್ನು ವೇಗವಾಗಿ ಫಾಲೋ ಅಪ್ ಮಾಡುವುದು \hbox{ಕಷ್ಟವಾಯಿತು.} ಇದಲ್ಲದೇ ನನ್ನನ್ನು ಬಿಟ್ಟು ಉಳಿದೆಲ್ಲ ವಿದ್ಯಾರ್ಥಿಗಳು ವಿಭಿನ್ನ ಊರು\-ಗಳಲ್ಲಿ ಔದ್ಯೋಗಿಕ ಒತ್ತಡದಲ್ಲಿದ್ದು ಗ್ರಂಥದ ಪ್ರೂಪ್ \hbox{ಕರೆಕ್ಷನ್} \hbox{ಇತ್ಯಾದಿಗಳನ್ನು} \hbox{ಆನ್ ಲೈನ್} ನಲ್ಲೇ ಮಡಬೇಕಾಗಿ ಬಂದುದು ಮತ್ತೊಂದು ಕಾರಣ. ಇನ್ನು \hbox{ಶ್ರೀರಂಗ} \hbox{ಡಿಜಿಟಲ್} \hbox{ಸಾಪ್ಟ್ ವೇರ್} ನವರ ವ್ಯವಹಾರ ನಮಗೆ ಬಹಳ \hbox{ಭಾರವಾಗಿ} ಗ್ರಂಥ\-ಕಾರ್ಯದ \hbox{ವಿಲಂಬಕ್ಕೆ} ಇನ್ನೊಂದು ಕಾರಣವಾಯಿತು. ಈ ನಡುವೆ, ಪೂರ್ವನಿಗದಿತ ವಿವಿಧ ಕ್ಷೇತ್ರಗಳ \hbox{ಕೆಲಸದ} ಒತ್ತಡಗಳು, ಅಲ್ಲದೇ ನನ್ನ \hbox{ದ್ರುತವಲ್ಲದ} ವಿಲಂಬಿತ ಗತಿಯೂ ಸೇರಿ \hbox{ಓಟ್ಟಾರೆ} ಗ್ರಂಥ \hbox{ಪ್ರಕಾಶನದ} ಸಮಯ ಲಂಬಿಸಿ ಅನೇಕರ \hbox{ಸಂಯಮವನ್ನು} \hbox{ಪರೀಕ್ಷಿಸುವಂತಾಯಿತು.} ಇವೆಲ್ಲವೂ ಇಲ್ಲಿ ವಿಶೇಷವಾಗಿ \hbox{ಉಲ್ಲೇಖನೀಯ} ವಿಷಯ\-ವೆಂದಲ್ಲ. ಏಕೆಂದರೆ ಗ್ರಂಥ ಸಂಪಾದನೆಯ ಸಮಸ್ಯೆ ಆ \hbox{ಕ್ಷೇತ್ರದಲ್ಲಿರುವವರಿಗೆಲ್ಲ}\break ಗೊತ್ತಿಲ್ಲದಿಲ್ಲ. ಆದರೂ ಕೆಲವೊಮ್ಮೆ ನಿರ್ದಿಷ್ಟವಾಗಿ, ಇರುವ ವಿಷಯ\-ವನ್ನು  \hbox{ವಾಚ್ಯವಾಗಿ} ವ್ಯಕ್ತಪಡಿಸದ ಹೊರತು ಕಾರಣ ಸ್ಪಷ್ಟವಾಗುವುದಿಲ್ಲ, ವಿನಾಕಾರಣ ಆಗ್ರಹಕ್ಕೆ ಆಸ್ಪದವಾಗುವ ಸಂಭವವನ್ನು ಅಲ್ಲಗಳೆಯುವಂತಿಲ್ಲ. ಅದಕ್ಕೇ ತಾನೆ, ಆಹಾರೇ ವ್ಯವಹಾರೇ ಚ ತ್ಯಕ್ತ\-ಲಜ್ಜಃ \hbox{ಸುಖೀ} ಭವೇತ್ ಎಂದ್ದಿದ್ದು. ಇನ್ನು, ಏನೆಲ್ಲ ಮಾಡಿದರೂ ಗ್ರಂಥದಲ್ಲಿ \hbox{ದೋಷಗಳಿಲ್ಲದಿಲ್ಲ.} ಅದನ್ನು ಮನ್ನಿಸಬೇಕೆಂದು ಎಲ್ಲರಲ್ಲೂ ವಿನಂತಿಸುತ್ತೇನೆ. ವಿದ್ಯಾರ್ಥಿಗಳು  ನನಗೆ ಕೊಟ್ಟ ಈ ಅವಕಾಶಕ್ಕೆ  ಅವರೆಲ್ಲರಿಗೆ  ಎಷ್ಟು  ಕೃತಜ್ಞತೆ ಹೇಳಿದರೂ ಕಡಿಮೆಯೇ. ವಿಶೇಷವಾಗಿ ಸಹಕರಿಸಿದ \hbox{ವಿದ್ಯಾರ್ಥಿಗಳಿಗೆಲ್ಲ} ಅನಂತಾನಂತ ಧನ್ಯವಾದಗಳು.

ನನ್ನ ಬಗೆಗೆ ಗಂಗಣ್ಣ, ಶೈಲಜಕ್ಕ, ಗುರುದಂಪತಿಗಳ ಪ್ರಸಾದಭಾವದ ಕಾರಣ \hbox{ಇಷ್ಟಾದರೂ} ವ್ಯಾವಹಾರಿಕ, ಸಾಹಿತ್ಯಿಕ ಕೆಲಸವನ್ನು  ನಿರ್ವಹಿಸುವುದು ನನಗೆ ಸಾಧ್ಯ\-ವಾಯಿತು. ಈ ಸೇವೆ ಆ ದಂಪತಿಗಳಿಗೆ ಅರ್ಪಿತ.

\articleend
}
