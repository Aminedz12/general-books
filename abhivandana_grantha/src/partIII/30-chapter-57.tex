\chapter{ನಾ ಕಂಡಂತೆ ಗಂಗಾಧರ}

\begin{center}
\Authorline{ನರಸಿಂಹ ಹೆಗಡೆ}
\smallskip

ಹೊನ್ನೇಹದ್ದ,\\ 
ಸಿದ್ಧಾಪುರ
\addrule
\end{center}

ನಮ್ಮ ಜೀವನದ ಗುರುಗಳು, ನಮ್ಮ ಕುಲ ಪುರೋಹಿತರೂ ಆದ ವೇ~। ಬ್ರ~। ಶ್ರೀ ವಿಘ್ನೇಶ್ವರ ಭಟ್ಟರ ಕುಟುಂಬ ಮತ್ತು ನಮ್ಮ ಕುಟುಂಬ ಒಂದು ರೀತಿಯ ಅಬಿನಾಭಾವ ಸಂಬಂಧ ಹೊಂದಿವೆ. ನಾವು ರಕ್ತಸಂಬಂಧಿಗಳೂ ಹೌದು. ಈ ಸಂಬಂಧದ ಹಿನ್ನೆಲೆಯಿಂದ ಅವರು ನನಗೆ ಚಿಕ್ಕಪ್ಪನಾಗಬೇಕು. ನನ್ನ ಬಾಲ್ಯದಿಂದ ಹಿಡಿದು ನನ್ನ ಜೀವನದ ಪ್ರತಿ ಮಜಲು ಸಹ ಚಿಕ್ಕಪ್ಪನವರಾದ ಶ್ರೀ ವಿಘ್ನೇಶ್ವರ ಭಟ್ಟರ ಮಾರ್ಗದರ್ಶನದಲ್ಲೇ ನಡೆದಿದೆ. ಅವರ ಹಿತ ನುಡಿ, ಸಾಂದರ್ಭಿಕ ಸಲಹೆ ಹಾಗು ಸಹಕಾರದಿಂದ ಇಂದು ನಾನು ಜೀವನದ ಔನ್ನತ್ಯವನ್ನು ಹೊಂದಿದ್ದೇನೆ ಎಂಬುದರಲ್ಲಿ ಯಾವುದೇ ಸಂಶಯವಿಲ್ಲ. ನಮ್ಮೀರ್ವರ ಕುಟುಂಬ ಇಂದಿಗೂ ಇದೇ ರೀತಿಯ ಸಂಬಂಧದಿಂದ ಮುಂದುವರಿದುಕೊಂಡು ಹೋಗುತ್ತಿದೆ.

ನನ್ನ ಚಿಕ್ಕಪ್ಪ ವಿಘ್ನೇಶ್ವರ ಭಟ್ಟರಿಗೆ ಮೂವರು ಗಂಡುಮಕ್ಕಳು, ನಾಲ್ವರು ಹೆಣ್ಣು ಮಕ್ಕಳು. ಗಂಡುಮಕ್ಕಳಲ್ಲಿ ಕೊನೆಯ ಕುಡಿಯೇ ಗಂಗಾಧರ. ನನಗೆ ತಿಳಿದಂತೆ ಪ್ರಾಥಮಿಕ ಪ್ರೌಢಶಿಕ್ಷಣವನ್ನು ಊರಲ್ಲೇ ಮುಗಿಸಿ ವೇದಾಧ್ಯಯನಕ್ಕೆಂದು ಶ್ರೀ ವೇ~। ರಾಮಚಂದ್ರ ಶಾಸ್ತ್ರಿಗಳ ಬಳಿಗೆ ಹೋದ. ಆದರೆ ಅಲ್ಲಿ ಅವನಿಗೆ ಅನುಕೂಲವಾಗದೇ ಮನೆಗೆ ಬಂದುಬಿಟ್ಟ. ಅವನಿಗೆ ಅತ್ಯಂತ ಅಕ್ಷರದ ದಾಹ ಇತ್ತು. ಮನೆಯಲ್ಲಿ ಕೊಂಚ ಭಯದ ವಾತಾವರಣದಿಂದ ತಂದೆಯಲ್ಲಿ ಅರುಹಲಸಾಧ್ಯವಾದ ವಿಷಯವನ್ನು ನನ್ನಲ್ಲಿ ಹಂಚಿಕೊಂಡು, ತನ್ನ ಮಾರ್ಗವನ್ನು ತಾನೇ ಹುಡುಕಿಕೊಂಡು ಮೈಸೂರಿಗೆ ಹೋಗುವ ಇಚ್ಛೆಯನ್ನು ತಿಳಿಸಿದ. ಅವನು ವಿದ್ಯೆಯಿಂದ ವಂಚಿತನಾಗುವುದು ವಿದ್ಯಾಪ್ರೇಮಿಯಾದ ನನಗೆ ಇಷ್ಟವಿರಲಿಲ್ಲ.  ಈ ಕಾರಣದಿಂದ ನಾನು ನನ್ನ ಕೈಲಾದ ಮಟ್ಟಿಗೆ ಸಹಕರಿಸಿ  ಆತ್ಮತೃಪಿ ಹೊಂದಿದೆ.

ಪ್ರತಿಭಾಸಂಪನ್ನನಾಗಿದ್ದ ಗಂಗಾಧರ ವಿದ್ಯಾರ್ಜನೆಗಾಗಿ ಮೈಸೂರಿನಲ್ಲಿ ಕಠಿಣ ಜೀವನ ನಡೆಸುತ್ತಾ ವಾರಾನ್ನದಿಂದ ಹೊಟ್ಟೆ ತುಂಬಿಸಿಕೊಂಡು  ಉನ್ನತಶಿಕ್ಷಣ ಪಡೆದು ಉತ್ತಮ ವಿದ್ಯಾವಂತನಾದ. ಇದರಲ್ಲಿ ನನ್ನ ಕೊಡುಗೆಯೂ ಕಿಂಚಿತ್ ಇದ್ದಿದ್ದರಿಂದ ನಾನು ಮಣ್ಣಿಕೊಪ್ಪ ಕುಟುಂಬದ ಒಂದು ಕೊಂಡಿಯೂ ಆಗಿ ಅವರಿಗೆ ನಾನು ನರಸಿಂಹ ಅಣ್ಣಯ್ಯನಾದೆ. 

ಗಂಗಾಧರ ಅತ್ಯಂತ ಸಹೃದಯಿ. ತನ್ನನ್ನು ಆಶ್ರಯಿಸಿ ಬಂದ ಯಾವೊಬ್ಬ ಶಿಷ್ಯನನ್ನೂ ಕಡೆಗಣಿಸದೇ ಆಶ್ರಯ ನೀಡಿ ವಿದ್ಯಾರ್ಜನೆಗೆ ಸಹಾಯಮಾಡುವ ಪ್ರವೃತ್ತಿ ಅವನಲ್ಲಿದೆ. ಬಹಳ ಜನ ವಿದ್ಯಾರ್ಥಿಗಳು ಅವನ ಗರಡಿಯಲ್ಲಿ ಬೆಳೆದಿದ್ದಾರೆ. ಇಂದು ಅವನು ಸಮಾಜದಲ್ಲಿ ಬಹಳ ಗೌರವವನ್ನು ಸಂಪಾದಿಸಿ ಸಮಾಜಕ್ಕೊಂದು ಆಸ್ತಿಯಾಗಿದ್ದಾನೆ. ಇದು ನನಗೆ ಅತ್ಯಂತ ಹೆಮ್ಮೆಯ ವಿಚಾರವಾಗಿದೆ.

ಕಾಲಕ್ರಮದಲ್ಲಿ ನನ್ನ ಕುಟುಂಬದ ಆಸ್ತಿ ವಿಚಾರದಲ್ಲಿ ದಾಯಾದಿಗಳೊಡನೆ ಭಿನ್ನಾಭಿಪ್ರಾಯದ ಸನ್ನಿವೇಶ ಉಂಟಾದಾಗ ನಾನು ಬಹಳ ತೊಂದರೆಗೊಳಗಾದೆ. ಆಗ ಚಿಕ್ಕಪ್ಪ ನನಗೆ ಉಪದೇಶ ಮಾಡಿದರು, ನರಸಿಂಹ ! ಪ್ರಪಂಚ ದೊಡ್ಡದಿದೆ. ಆಸ್ತಿ ದೊಡ್ಡದಲ್ಲ. ಮಕ್ಕಳೇ ಆಸ್ತಿ. ಅವರನ್ನು ಚೆನ್ನಾಗಿ ಓದಿಸು,” ಎಂಬುದಾಗಿ. ಇದು ಮುಂದೆ ಸಾಕಾರಗೊಂಡಿದ್ದು ಗಂಗಾಧರನ ಮೂಲಕ. ನನ್ನೆರಡು ಪುತ್ರರು – ಗಣಪತಿ ಮತ್ತು ಉಮೇಶ ಮೈಸೂರಿನಲ್ಲಿ ಗಂಗಾಧರನ ಆಶ್ರಯದಲ್ಲಿ ಅವನ ಮಾರ್ಗದರ್ಶನದಲ್ಲೇ ಮುನ್ನಡೆದರು. ಅವನೇ ಅಲ್ಲಿ ಅವರನ್ನು ಸಂಪೂರ್ಣವಾಗಿ ನೋಡಿಕೊಂಡ. ಅದರಿಂದ ನನ್ನ ಮಕ್ಕಳು ಚೆನ್ನಾಗಿ ಓದಿ ತಮ್ಮ ಕಾಲಮೇಲೆ ತಾವು ನಿಲ್ಲುವಂತಾಯಿತು.  ಅದರಿಂದ ನನ್ನ ಕುಟುಂಬಕ್ಕೂ  ಬಲ ಬರುವಂತಾಯಿತು. ಇದರಿಂದ ನನ್ನ ಇಡೀ ಕುಟುಂಬ ಗಂಗಾಧರನನ್ನೂ ಯಾವತ್ತೂ ನೆನೆದುಕೊಳ್ಳುತ್ತಲೇ ಇರುತ್ತದೆ. ಹಾಗಾಗಿ ಈ ಕಾಲದಲ್ಲಿ ನಮ್ಮೆರಡು ಕುಟುಂಬ – ಚಿಕ್ಕಪ್ಪನ ಮಕ್ಕಳು ಮತ್ತು ನನ್ನ ಮಕ್ಕಳು ಯಾವತ್ತೂ ಪರಸ್ಪರ ವಿಶ್ವಾಸದಿಂದ – ಅವಿನಾಭಾವದಿಂದ ಬದುಕುತ್ತಿರುವುದು ನನಗೆ ಪರಮ ಸುಖವನ್ನು ತಂದುಕೊಟ್ಟಿದೆ. 

ಪ್ರಕೃತ ಸಂದರ್ಭದಲ್ಲಿ ಗಂಗಾಧರ ವಿತ್ತಿಯಿಂದ ನಿವೃತ್ತನಾಗುತ್ತಿದ್ದು ಅವನ ವಿದ್ಯಾರ್ಥಿಗಳೆಲ್ಲ ಅವನಿಗೆ ಅಭಿವಂದನ ಕಾರ್ಯಕ್ರಮ ಹಮ್ಮಿಕೊಂಡಿರುವ ವಿಷಯವನ್ನು ಚಿ~। ಗುರುಪ್ರಸಾದ ತಿಳಿಸಿದ. ವಿಷಯ ತಿಳಿದು ನನಗೆ ಪರಮಾನಂದವಾಯಿತು. ಗಂಗಾಧರನನ್ನು ಅವನ ವಿದ್ಯಾರ್ಥಿಗಳು ಮತ್ತು ಸಮಾಜ ಎಷ್ಟು ಕೃತಜ್ಞತೆಯಿಂದ ಭಾವಿಸಿದರೂ ಅದು ಹೆಚ್ಚಲ್ಲ. ಏಕೆಂದರೆ ಅವನಿಂದ ಅಷ್ಟೊಂದು ಪ್ರಯೋಜನ ಸಮಾಜಕ್ಕೆ ಆಗಿದೆ. ಇದೇ ಸಂದರ್ಭದಲ್ಲಿ ಅಭಿವಂದನ ಗ್ರಂಥವನ್ನೂ ಪ್ರಕಟಿಸುತ್ತಿರುವುದು ಮತ್ತೂ ಸಂತೋಷದ ಸಂಗತಿ. ಈ ಗ್ರಂಥಕ್ಕೆ ಗಂಗಾಧರನ ಬಗ್ಗೆ ಒಂದು ಲೇಖನ ಕೊಡುವಂತೆ ಗುರುಪ್ರಸಾದ ಕೇಳಿದ್ದಾನೆ. ಬರೆಯುವುದು ಬಹಳವಿದೆ. ಆದರೆ ಕಾಲ ಮತ್ತು ವಯೋಧರ್ಮ ನನಗೆ ಸಹಕರಿಸಿದಷ್ಟು – ಏನೋ ಕಿಂಚಿತ್ ಮಾತ್ರ ನನಗೆ ಬರೆಯುವುದು ಸಾಧ್ಯವಾಗಿದೆ, ಬರೆದಿದ್ದೇನೆ.  

ನಾನು ಮತ್ತು ನನ್ನ ಮಕ್ಕಳು ಅವನಿಗೆ ಅತ್ಯಂತ ಕೃತಜ್ಞರಾಗಿದ್ದೇವೆ. ಮುಂದಿನ ಅವನ ನಿವೃತ್ತಜೀವನ ಸುಖಮಯವಾಗಿರಲೆಂಬುದೇ ನಮ್ಮೆಲ್ಲರ ಹೃದಯಪೂರ್ವಕ ಹಾರೈಕೆ.

\articleend	
