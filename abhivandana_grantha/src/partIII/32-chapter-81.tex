{\fontsize{14}{16}\selectfont
\chapter{ಗುರುರೇವ ಜಗತ್ಸರ್ವಮ್}

\begin{center}
\Authorline{ವಿದ್ವಾನ್~॥ ವಿನಾಯಕ ಭಟ್ಟ ಗಾಳಿಮನೆ}
\smallskip
ಮುಖ್ಯಸ್ಥರು, ಸಂಸ್ಕೃತ ವಿಭಾಗ \\
ಆಳ್ವಾಸ್ ಮಹಾವಿದ್ಯಾಲಯ\\
ಮೂಡಬಿದಿರೆ  \enginline{-}  574227
\addrule
\end{center}
\begin{center}
ಪರಿವರ್ತಿನಿ ಸಂಸಾರೇ ಮೃತಃ ಕೋ ವಾ ನ ಜಾಯತೇ\\
ಸ ಜಾತೋ ಯೇನ ಜಾತೇನ ಯಾತಿ ವಂಶಃ ಸಮುನ್ನತಿಮ್~॥
\end{center}
ವಂಶೋನ್ನತಿಯನ್ನು ಸಾಮಾಜಿಕವಾಗಿ ತೋರಿಕೊಟ್ಟ ಪುತ್ರರು ತಮ್ಮ ಜೀವನವನ್ನು ತನ್ಮುಖೇನ ಸಾರ್ಥಕ್ಯ ಹೊಂದುವಂತೆ ಮಾಡಿಕೊಳ್ಳುತ್ತಾರೆ. ಹೀಗೆ ಸಾಧನೆಯ ಪಥದಲ್ಲಿ ಸಾಗಿ ಯಶಃಪರ್ವತವನ್ನು ಆರೋಹಣ ಮಾಡಿ ಅಭಿವಂದ್ಯರಾದವರು ಜನವರಿ 31ಕ್ಕೆ ವೃತ್ತಿನಿವೃತ್ತರಾದ ಸಚ್ಛಿಷ್ಯಸಮೂಹವನ್ನು ನಾಡಿಗೆ ಕೊಡುಗೆಯಾಗಿ ನೀಡಿರುವ ಸಿದ್ಧಾಪುರ ತಾಲೂಕಿನ ಅಗ್ಗೆರೆ(ಮಣ್ಣಿಕೊಪ್ಪ)ಯ ಮೂಲನಿವಾಸಿ, ಮೈಸೂರಿನ ಸಂಸ್ಕೃತ ಮಹಾವಿದ್ಯಾಲಯದ ನವ್ಯನ್ಯಾಯ ವಿಭಾಗದ ಸಹಾಯಕ ಪ್ರಾಧ್ಯಾಪಕರಾಗಿದ್ದ ವಿದ್ವಾನ್॥ ಗಂಗಾಧರ ಭಟ್ಟರು.

\section*{ಜನ್ಮತ: ಪ್ರತಿಭಾಸಂಪನ್ನರು}

ವಿ~॥ ಗಂಗಾಧರ, ವಿ~॥ ಭಟ್ಟರ ತಂದೆ ಶ್ರೀ ವಿಘ್ನೇಶ್ವರ ಭಟ್ಟ ತಾಯಿ ಶ್ರೀಮತಿ ರೇವತಿ ಭಟ್ಟರು. ಸಿದ್ದಾಪುರ ತಾಲೂಕಿನ ಮಣ್ಣೀಕೊಪ್ಪ ಗ್ರಾಮದಲಿ ಜನನ. ಪ್ರಸ್ತುತ ಅಗ್ಗೆರೆಯಲ್ಲಿ ಮನೆಯವರ ವಾಸ. ಇವರು ತಮ್ಮ ಕುಟುಂಬದ ಸಂಸ್ಕಾರವಿಶೇಷದಂತೆ ಜನ್ಮಜಾತ\-ವಾಗಿಯೇ ಔದಾರ್ಯಾದಿ ಗುಣಭೂಯಿಷ್ಠರು, ವಿಶೇಷ ಪ್ರತಿಭಾಸಂಪನ್ನರು. ಗೋಕರ್ಣ\-ದಲ್ಲಿ ವೇದ ಮತ್ತು ಮೈಸೂರಿನಲ್ಲಿ ನವೀನನ್ಯಾಯಶಾಸ್ತ್ರದ ಅಧ್ಯಯನ. ಅದೇ ಸಮಯದಲ್ಲೇ ಲೌಕಿಕ ಶಿಕ್ಷಣದಲ್ಲೂ ಸೈ ಎನಿಸಿಕೊಂಡು ವಾಣಿಜ್ಯಪದವಿಯನ್ನು ಪಡೆದವರು.  ಮುಂದೆ, ಅಧ್ಯಯನ ಮಾಡಿದ ಸಂಸ್ಕೃತ ಮಹಾಪಾಠಶಾಲೆಯಲ್ಲಿ 1998ರಿಂದ ಅಧ್ಯಾಪನ ಮಾಡುವ ಯೋಗ ಮತ್ತು ಯೋಗ್ಯತಾಸಂಪನ್ನರು ಆದರು. ತಮ್ಮ ಪ್ರತಿಭಾಕೌಶಲ್ಯದಿಂದಲೇ ಸಕಲರನ್ನೂ ಆಕರ್ಷಿಸುವ ಶ್ರೀಯುತರು ಚಾಣಕ್ಯನ ಅರ್ಥಶಾಸ್ತ್ರದ ಮೇಲೆ ವಿಶೇಷಾಧ್ಯಯನ ನಡೆಸಿದವರೂ ಹೌದು. ರಾಜ್ಯದ ಏಕೈಕ ಸಂಸ್ಕೃತ ದಿನಪತ್ರಿಕೆ ಸುಧರ್ಮಾ ನಿರಾತಂಕವಾಗಿ ನಡೆಯುವಲ್ಲಿ ಭಟ್ಟರ ಪಾತ್ರ ದೊಡ್ಡದಿದೆ. ರಾಮಾಯಣ\enginline{-}ಮಹಾಭಾರತ ಪರೀಕ್ಷೆಗಳನ್ನು ಅವಿರತವಾಗಿ ನಡೆಸಿ ಮಕ್ಕಳಿಗೆ ಭಾರತೀಯ ಪುರಾಣಪ್ರಜ್ಞೆ ಬೆಳೆಸಿದ ಶ್ರೇಯಸ್ಸು ಭಟ್ಟರದ್ದು.

\section*{ಉತ್ತರಕನ್ನಡ  \enginline{-}  ಮೈಸೂರಿನ ದೃಢಬೆಸುಗೆ   \enginline{-}   ವಿ~॥ ಗಂಗಾಧರ ಭಟ್ಟರು}

ಶಿಕ್ಷಣಕ್ಕೂ ಮತ್ತು ಮೈಸೂರಿಗೆ ಅವಿನಾಭಾವ ನಂಟಿರಿವುದು ತಿಳಿದೇ ಇದೆ. ಹಾಗೆಯೇ  ಗಂಗಾಧರಭಟ್ಟರಿಗೂ ಉತ್ತರಕನ್ನಡದಿಂದ ಅಧ್ಯಯನಕ್ಕೆ ತೆರಳುವ ವಿದ್ಯಾರ್ಥಿಗಳಿಗೂ ನಂಟಿದೆ. ಅಶನ\enginline{-}ವಸನಾದಿಗಳನ್ನು ಸುವ್ಯವಸ್ಥೆಗೊಳಿಸಿ ಎಲ್ಲರೂ ತಮ್ಮವರು ಎಂದು ಭಟ್ಟರು ತೋರುವ ಔದಾರ್ಯ ಅನ್ಯಾದೃಶವಾದುದು.

ಉತ್ತರಕನ್ನಡಜಿಲ್ಲೆ ಸೇರಿದಂತೆ ರಾಜ್ಯದ ಮತ್ತು ರಾಷ್ಟ್ರದ ನಾನಾಮೂಲೆಯಿಂದ ವೇದ\enginline{-}ಶಾಸ್ತ್ರಗಳನ್ನು ಓದಲು ಬರುವ, ಹಾಗೂ ಇದರ ಹೊರತಾಗಿಯೂ ಯಾವುದೇ ವಿದ್ಯಾರ್ಥಿ ಮೈಸೂರಿಗೆ ಅಧ್ಯಯನಕ್ಕೆ ಹೋಗುತ್ತಾನೆ ಅಂತಾದರೆ ಬಹುಶ: ಮೊದಲು ಬಾಗಿಲು ತಟ್ಟುವುದು  ಭಟ್ಟರ ಮನೆಯದ್ದೇ ಎಂಬುದು ಖಂಡಿತ ಅತಿಶಯೋಕ್ತಿಯಲ್ಲ. 

\section*{ವಿದ್ಯಾರ್ಥಿಗಳ ಪಾಲಿಗೆ ಕರುಣಾಳು}

ಓದುವುದಕ್ಕೆ ಅಂತ ಹೋಗುವಾಗ ಅಶನಾದಿಗಳ ಸಮಸ್ಯೆ ಒಂದಾದರೆ, ಓದುವಾಗ ಇನ್ನಿತರ ತೊಂದರೆ ತೊಡಕು ಬರುವಂತದ್ದು ನಿರೀಕ್ಷಿತವೇ ಸರಿ. ವೈಯಕ್ತಿಕ ನೆಲೆಯಿಂದ ಆರಂಭಿಸಿ ಎಂತದ್ದೇ ಸಮಸ್ಯೆ ಇದ್ದರೂ ಭಟ್ಟರು ಬಂದರೆ ಆ ಸಮಸ್ಯೆ ಪರಿಹಾವನ್ನು ಕಾಣುತ್ತದೆ ಎಂದೆ ನಂಬಿದವರು ಅಸಂಖ್ಯರಿದ್ದಾರೆ, ಅದರ ಕೃತಜ್ಞತೆ ಇಂದಿಗೂ ಅನೇಕರ ಪಾಲಿಗೆ ದೊಡ್ಡ ಋಣಭಾರವೇ ಆಗಿರಬಹುದು. 
\newpage

ನಿರೀಕ್ಷೆ\enginline{-}ಪರೀಕ್ಷೆ  \enginline{-}  ಅಪೇಕ್ಷಾರಹಿತರಾಗಿ ಕೇವಲ ವಿದ್ಯಾರ್ಥಿಗಳ ಶ್ರೇಯಃಕಾಂಕ್ಷಿಗ\-ಳಾಗಿ ತಮ್ಮ ಕರ್ತವ್ಯಕರ್ಮನಿಷ್ಠರು, ಛಾತ್ರಮಾನಸಕಮಲವನ್ನು ಅರಳಿಸುವ ಭಗವಾನ್\break ಭಾಸ್ಕರರು. ಅವರಿಂದ ಉಪಕೃತರು ಅಸಂಖ್ಯಾತರು, ಆದರೆ ಅವರೆಂದೂ ಯಾರಿಂದಲೂ ಏನನ್ನೂ ಬಯಸದೇ ಬದುಕಿದವರು, ಇದು ಕೇವಲ ಭಟ್ಟರಿಗೆ ಮಾತ್ರ ಅನ್ವಯಿಸುವುದಿಲ್ಲ, ಅವರ ಕುಟುಂಬಕ್ಕೆ ಸಮನ್ವಯವಾಗುವ ಮಾತು. ಹೀಗೆ ಸಾಗುತ್ತದೆ  ಭಟ್ಟರ ಜೀವನಯಾನ.

\section*{ನಿರಂತರ ಅಧ್ಯಾಪನ   \enginline{-}   ಭಟ್ಟರ ವಿಶೇಷ}

ಕಾಲೇಜಿನ ಸಮಯವನ್ನು ಮತ್ತು ತಮ್ಮ ಬಿಡುವಿನ ಸಮಯವನ್ನೂ ಅಧ್ಯಯನ\enginline{-}ಅಧ್ಯಾಪನಕ್ಕೆ ಮೀಸಲಿಟ್ಟು ಬದುಕನ್ನು ಸಾಗಿಸಿದ ಪರಿ ಮೈಮನ ರೋಮಾಂಚ ಗೊಳಿಸುತ್ತದೆ. ತಮ್ಮ ವೃತ್ತಿಜೀವನವನ್ನು ಅಧ್ಯಾಪನಕ್ಕೆಂದೇ ಮೀಸಲಾಗಿರಿಸಿದವರು. ಬೆಳಿಗ್ಗೆ ಬೊಧನೆಗೆ ಕುಳಿತರೆ ನಿರಂತರ ನಾಲ್ಕೈದು ಗಂಟೆ ಶಾಸ್ತ್ರಗಳ ವಿಚಾರವನ್ನು ವಿವರಿಸಿ ಅರ್ಥ ಮಾಡಿಸುವ ಪರಿ ಭಟ್ಟರಿಗೇ ಮೀಸಲಾದದ್ದು.

ಅಧ್ಯಾಪನದಲ್ಲಿದ್ದೂ ಶ್ರೇಷ್ಠವಿದ್ವಾಂಸರಾಗಿ  ಕಂಗೊಳಿಸಿದ ಅನೇಕ ವಿದ್ವಜ್ಜನರು ನಮ್ಮ ಜೊತೆ ಇದ್ದಾರೆ. ಆದರೆ ತಾವು ವಿದ್ವಾಂಸರಾಗಿ ಅಧ್ಯಯನ ನಿರತ ವಿದ್ಯಾರ್ಥಿಗಳನ್ನು ತಮ್ಮ ಬೋಧನೆಯಿಂದ ವಿದ್ವಜ್ಜನರನ್ನಾಗಿ ಸಿದ್ಧ ಮಾಡಿದವರು ಕೇವಲ ಬೆರಳೆಣಿಕೆಯ ಅಧ್ಯಾಪಕ\-ರಿರಬಹುದು. ಅಂತವರಲ್ಲಿ ಭಟ್ಟರು ವಿರಳರಲ್ಲೂ ಅತಿವಿರಳರು.

ಇಂತಹ ಮಹನೀಯರಾದ ವಿದ್ವಾನ್॥ ಗಂಗಾಧರ ಭಟ್ಟರಿಗೆ ಫ಼ೆ\kern -4ptಬ್ರವರಿ 11ರಂದು ಅವರ ಶಿಷ್ಯವೃಂದವು  ಅಭಿವಂದನ ಕಾರ್ಯಕ್ರಮವನ್ನು ಭಟ್ಟರ ಕರ್ಮಕ್ಷೇತ್ರವಾದ ಮೈಸೂರಿನ ಮಹಾರಾಜ ಸಂಸ್ಕೃತ ಕಾಲೇಜಿನಲ್ಲಿ ಏರ್ಪಡಿಸಿತ್ತು. 

ಆರಂಭದಲ್ಲಿ ಕೇವಲ ಒಮ್ಮೆ ಮಾತ್ರ ಸಭೆ ಸೇರಿ, ಸುಮಾರು ನಾಲ್ಕು ತಿಂಗಳಿಂದ ಸಿದ್ಧತೆ\-ಯಲ್ಲಿ ತೊಡಗಿ, ಉಳಿದಂತೆ ಸಕಲವನ್ನು ಅವರವರ ಕಾರ್ಯಕ್ಷೇತ್ರದಲ್ಲಿದ್ದುಕೊಂಡೇ ಸಂವಹನಮಾಧ್ಯಮಗಳ ಮೂಲಕವೇ ನಿರ್ವಹಿಸಿ ಸಿದ್ಧವಾದ ಪುಣ್ಯತಮಕಾರ್ಯವದು. 

ಕಾರ್ಯಕ್ರಮ ಭವ್ಯವಾಗಿ ಅನಾವರಣಗೊಂಡ ಕ್ಷಣವದು. ಭಟ್ಟರ ಕುರಿತಾದ ಅಸೀಮ\-ವಾದ ಶ್ರದ್ಧಾ\enginline{-}ಭಕ್ತಿಗಳೇ ಜೀವಾಳವಾಗಿ, ಅವರಲ್ಲಿ ನ್ಯಾಯಶಾಸ್ತ್ರವನ್ನು ಅಧ್ಯಯನ ಮಾಡಿದ ಛಾತ್ರನಿವಹವೇ ಒಗ್ಗೂಡಿ ಮಾಡಿದ ಅಭಿವಂದನಕಾರ್ಯ.    ಇದು ಒಂದು ನೆಲೆಯಲ್ಲಿ ಭಟ್ಟರ ಕುರಿತು ಶಿಷ್ಯರಿಗಿರುವ ನಿಷ್ಠೆಯ ದರ್ಶನವಾದರೆ, ಇನ್ನೊಂದು ನೆಲೆಯಲ್ಲಿ  ಶಿಷ್ಯರ ಬಗ್ಗೆ ಭಟ್ಟರ ವಿಶ್ವಾಸದೃಢತೆಯನ್ನೂ ಶ್ರುತಪಡಿಸುತ್ತದೆ. ನಿವೃತ್ತಿಯ ಸಂದರ್ಭಕ್ಕೆ  ತಮ್ಮ ಶಿಷ್ಯಗಣವನ್ನು ನೋಡುವ ಬಯಕೆಯನ್ನು ಹೊಂದಿರುವ ಭಟ್ಟರ ಮನಸಲ್ಲಿರುವ ಶಿಷ್ಯವಾತ್ಸಲ್ಯಕ್ಕೆ ಪಾರವುಂಟೆ? ಎಷ್ಟೊಂದು ಉದಾತ್ತಬಗೆಯಿದು! ಎಂತ ಸಾತ್ತ್ವಿಕ ಭಾವವಿದು! ಭಟ್ಟರಲ್ಲಿ ಶಿಷ್ಯವೃತ್ತಿಪಡೆದ ವಿದ್ಯಾರ್ಥಿಗಳ ಸಾರ್ಥಕ್ಯದ ಕ್ಷಣವಿದು. ಅದರಲ್ಲೂ ನಿವೃತ್ತರಾಗುವ ಸಂದರ್ಭದಲ್ಲಿ ತಮ್ಮ ವಿದ್ಯಾರ್ಥಿವೃಂದ  \enginline{-}  ದರ್ಶನಕಾಂಕ್ಷಿಗಳೆಂಬುದು ನಿಜಕ್ಕಾದರೂ ಭಟ್ಟರ ಶಿಷ್ಯವೃಂದದ ಪುಣ್ಯಾತಿಶಯವಲ್ಲದೇ ಮತ್ತೇನು? ನಿಜಕ್ಕೂ ಶಿಷ್ಯವೃಂದ ಧನ್ಯ. ಜೀವನದ ಪರಮಸಾರ್ಥಕ್ಯದ ಕ್ಷಣವೂ ಹೌದು. 

ಇಂತ ಭಾವನಾತ್ಮಕವೂ\enginline{-}ಬೌದ್ಧಿಕವೂ ಆಗಿರುವ ಈ ಕಾರ್ಯಕ್ರಮ ನಡೆಯುತ್ತಿರುವುದು ನಮ್ಮ ನಾಡಿನ ಶ್ರೇಷ್ಠ ವಿದ್ವಜ್ಜನರು, ಪೂಜ್ಯ ಭಟ್ಟರ ಸಂಬಂಧಿಗಳು, ಸುಹೃದರು, ಸಹೋದ್ಯೋಗಿಗಳು ಸೇರಿ ನಡೆಸುತ್ತಿರುವುದು ನಮ್ಮೆಲ್ಲರ ಪಾಲಿನ ಪುಣ್ಯಾತಿಶಯ. ಇಂತಹ ನಿಃಸ್ಪೃಹ ಗುರುಸನ್ನಿಧಾನದಲ್ಲಿ ಅಧ್ಯಯನ ಮಾಡಿದ್ದೇನೆ ಎಂಬುದೇ ನನ್ನ ಪಾಲಿಗೆ ಸದಾ ಭಾವುಕತನವನ್ನು, ರೋಮಾಂಚವನ್ನೂ ಉಂಟುಮಾಡುತ್ತದೆ. ಇದಕ್ಕೆ ಕಳಶ\-ವಿಟ್ಟಂತೆ, ಪೂಜ್ಯ ಗಂಗಾಧರ ಭಟ್ಟರು ಮತ್ತು ಕುಟುಂಬ, ನಮ್ಮ ಮನೆಯ ಹಿತವನ್ನು ಸದಾ ಬಯಸುವ, ಅದನ್ನು ಧಾರ್ಮಿಕ ನೆಲೆಯಲ್ಲಿ ಸಾಧ್ಯವಾಗಿಸುವ ಸ್ಥಾನಾಪನ್ನರಾದ ನಮ್ಮ ಪುರೋಹಿತರು ಎಂಬುದು ನನಗೆ, ನಮ್ಮ ಕುಟುಂಬಕ್ಕೆ ಆತ್ಯಂತಿಕವಾದ ಧನ್ಯತೆಯನ್ನೊದಗಿಸಿದ ಸಂಗತಿ. 
\begin{center}
ಧನ್ಯಾಃ ಸ್ಮೋ ಭವತಃ ಪಾದಸ್ಪರ್ಶಜನ್ಯಸುಖಾದಿಭಿಃ~।\\
ಸದಾ ತು ತದ್ಭವೇದೇವಂ ಭಾವಸಂಸ್ಕಾರಸಾಧಕಮ್~॥
\end{center}

\articleend	
}
