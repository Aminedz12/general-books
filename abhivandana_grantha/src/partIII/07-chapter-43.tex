\chapter{ನನ್ನ  ಸಹೋದರ - ನಾನು ಕಂಡಂತೆ}

\begin{center}
\Authorline{ಶ್ರೀಧರ ವಿಘ್ನೇಶ್ವರ ಭಟ್ಟ}
\smallskip

ಮಣ್ಣಿಕೊಪ್ಪ\\
ಬಿ.ಎಸ್ಸಿ., ಐ,ಇ.ಸಿ.\\
ಆಂಗಿರಸ ಪ್ರಿಂಟರ್ಸ್,\\
ಚಂದ್ರಗುತ್ತಿರಸ್ತೆ, ಸಿದ್ದಾಪುರ  (ಉ.ಕ.)
\end{center}
`ಮಣ್ಣಿಕೊಪ್ಪ'  ಒಂದು ಕಾಲದಲ್ಲಿ ಜನರ ವಸತಿಯಿಂದ ದೂರವಿದ್ದ ಭೂಪ್ರದೇಶ. ಉತ್ತರಕನ್ನಡದ ಸಿದ್ದಾಪುರ ತಾಲೂಕಿನ  ನಾಲಿಗಾರ ಗ್ರಾಮದ; ಬೇಸಾಯದಿಂದ ಮರೆಯಾದ ತರಿಕ್ಷೇತ್ರವೆಂದು ಕಂದಾಯ ಇಲಾಖೆಯಲ್ಲಿ ನೋಂದಿತ ತಾಣ.  ಪಕ್ಕದ ಗ್ರಾಮ ಕವಲಕೊಪ್ಪ. ಅಗ್ಗೆರೆ ಎಂಬಲ್ಲಿ ವೇ| ಗಣಪ ಭಟ್ಟರ ಕೂಡುಕುಟುಂಬದ ತೃತೀಯ ಸಂತಾನವಾಗಿ ಸಂದ ವಿಘ್ನೇಶ್ವರ ಭಟ್ಟರು ಬೇಸಾಯ ಹಾಗೂ ಕುಟುಂಬಕ್ಕೆ ಒಪ್ಪಿದ ಪೌರೋಹಿತ್ಯ ನಿರ್ವಹಣೆಯಲ್ಲಿ ನೆಮ್ಮದಿ ಕಂಡ ಸಂಸಾರವದು.  ಸಮಕಾಲೀನ ಅಗತ್ಯಗನುಗುಣವಾಗಿ ಬೇಸಾಯಕ್ಕೆ ಹೆಚ್ಚು ಒತ್ತುಕೊಟ್ಟು ಪರಿಶ್ರಮಿಸಿ ಬದುಕು ಸಾಗಿಸುತ್ತಿರುವಾಗ ರೇವತಿ ಎಂಬ ವಧೂವನ್ನು ವರಿಸಿ ಸಂಸಾರಿಯಾಗಿ ಮಂಜುನಾಥ, ಶ್ರೀಧರ ಎಂಬ ಪುತ್ರರನ್ನು  ಪಡೆದು  ಮುಂದುವರೆದಿತ್ತು.  ಕುಟುಂಬದ ಆರ್ಥಿಕ ಮೂಲದ ವಿಸ್ತರಣೆಯ  ಪ್ರಯತ್ನವಾಗಿ  ಯೋಚಿಸಿ, ಯೋಜಿಸಿ  ಪ್ರಸ್ತುತ ಮಣ್ಣಿಕೊಪ್ಪ ಎಂಬಲ್ಲಿ ಶ್ರೀಮಾನ್ ನೆಲೆಮಾವು  ಮಠಕ್ಕೆ ಸೇರಿದ ತರಿಕ್ಷೇತ್ರವನ್ನು  ಪಡೆದು ಬೇಸಾಯಕ್ಕೆ,  ಅಭಿವೃಧ್ಧಿಗೆ ತೊಡಗಿಕೊಂಡು;  ಅಲ್ಲೆ ಚಿಕ್ಕದೊಂದು ಬಿಡಾರ ಮಾಡಿಕೊಂಡು ಬದುಕು ಕಟ್ಟಿಕೊಂಡು ಮುಂದುವರಿಯುತ್ತಿರುವಲ್ಲಿ  ದಂಪತಿಗಳಿಗೆ ಮೂರನೆಯ ಪ್ರಸವ ಭಾಗ್ಯ ಪ್ರಾಪ್ತಿಯಾಗಿ ಪಡೆದ ಶಿಶುವಿಗೆ  ಗಂಗಾಧರ ಎಂದು ನಾಮಕರಣ ಮಾಡಿದರು.

ಭೂ ಬೇಸಾಯದ ಕೆಲಸ ಮುಂದುವರಿದಂತೆ ಕಣ್ಣಿಗೆ ಬಿದ್ದ  ಶಿಲೆಯೊಂದು ಬೋರಲಾಗಿಸಲು ಗೋಚರಿಸಿದ್ದೇ  ಮಾರುತಿಯ  ಮೂರ್ತರೂಪ.  ಅದನ್ನೇ ಆರಾಧಿಸಿಕೊಂಡು  ನಮ್ಮ ಸಕಲ ಸಂಕಷ್ಟಗಳಿಗೆ ಪರಿಹಾರ, ರಕ್ಷಣೆ ಪಡೆದು ಬದುಕುಕಟ್ಟಿಕೊಂಡ ದಿನಗಳವು. ನಮ್ಮ ತಾಯಿಯ ಮುಗ್ಧಮನಸ್ಸಿನಲ್ಲಿ ಸುಡುಗಾಡು ಸಿದ್ದನೊಬ್ಬ ನುಡಿದ ಶಬ್ದರೂಪ ``ನಿಮ್ಮಎಲ್ಲಾಕ್ಷೇಮ, ಸೌಖ್ಯವನ್ನು ಈ ದೇವ ನೋಡಿಕೊಳ್ಳುತ್ತಾನೆ'' ಎಂದಿದ್ದನಂತೆ.  ಕಾಲಕ್ರಮಿಸಿದಂತೆ   ದೇವಾಲಯ ಹಾಗೂ  ಈ ಪ್ರದೇಶ  ಸಮೃಧ್ಧಿಯನ್ನು ಕಂಡದ್ದು ನಮ್ಮ ಕುಟುಂಬದ ಮೂಲ ಆಶ್ರಯವಾಗಿ ಸಂದಿದ್ದು, ಕಾಕತಾಳೀಯ. ಈ ಕಾಲಘಟ್ಟದಲ್ಲಿ  ಸಂಸಾರದಲ್ಲೂ ನಾಲ್ಕು ಪುತ್ರಿಯರನ್ನು  ಪಡೆದು ನೆಮ್ಮದಿಯ ಸಂಸಾರ ಸಾಗಿತ್ತು.	
	
ಕುಮಾರ ಗಂಗಾಧರ ಹತ್ತಿರದ ಕವಲಕೊಪ್ಪ ಸರಕಾರಿ ಹಿರಿಯ ಪ್ರಾಥಮಿಕ ಶಾಲೆಯಲ್ಲಿ ಪ್ರಾಥಮಿಕ ಶಿಕ್ಷಣವನ್ನು ಹೊಂದುತ್ತಿರುವಾಗಲೆ ಪ್ರಾಪ್ತ ವಯಸ್ಸಿಗೆ ಉಪನೀತನಾಗಿ  ಮಾಧ್ಯಮಿಕ ಶಿಕ್ಷಣವನ್ನು ಬಿದ್ರ್ರಕಾನ ಮಹಾತ್ಮಾಗಾಂಧಿ ಶತಾಬ್ದಿ ಪ್ರೌಢಶಾಲೆಯಲ್ಲಿ ಪೂರೈಸಿ, ವೇದಾಧ್ಯಯನ ಮಾಡಲೋಸುಗ ಹೊನ್ನಾವರದ ಹತ್ತಿರವಿರುವ ಪಾಠಶಾಲೆ ಹಿರಿಯ ವಿದ್ವಾಂಸರಲ್ಲಿ ಪಾಠ ಶಾಲೆಗೆ ಸೇರಿದ್ದ.  ಅನಿರೀಕ್ಷಿತವಾಗಿ ಅವನಿಗೆ ಆ ವ್ಯವಸ್ಥೆ ಸರಿಹೊಂದದೆ ಅಲ್ಲಿ ಅಲ್ಪಕಾಲದಲ್ಲೆ ಬಿಟ್ಟು ಬದಲಾವಣೆಯಾಗಿ ಗೋಕರ್ಣದಲ್ಲಿ ಅದೇ ವಿದ್ಯಾಭ್ಯಾಸವನ್ನು ಮುಂದುವರಿಸಿದ. ಅಲ್ಲಿ ಬಿಟ್ಟು  ಬರುವಾಗ ``ನಿನ್ನ ವಿದ್ಯಾಭ್ಯಾಸ ಮುಗಯಿತೇ?.....'' ಎಂದಿದ್ದರು. ಅದೇ ಪ್ರಾರಂಭದ  ಸೂಚನೆಯಾಯಿತು!.  ಹೇಗೇ ಇರಲಿ, ಅಧ್ಯಯನ..ಅಧ್ಯಾಪನ ಕೈಂಕರ್ಯದಲ್ಲಿ  ಮೈಸೂರು ಮಹಾರಾಜ ಸಂಸ್ಕೃತ ಪಾಠಶಾಲೆ ಆಶ್ರಯ ಒದಗಿಸಿತು. ಈ ದಿಶೆಯಲ್ಲಿ  ಸಾಕಾರ  ಬದುಕು ಸಾಕ್ಷಾತ್ಕಾರಗೊಂಡಿದೆ, ಆಧುನಿಕ ಶಿಕ್ಷಣ ವಿಭಾಗದ ಜೊತೆಜೊತೆಯಲ್ಲಿ.

ಅಧ್ಯಾಪನ ವಿಷಯದಲ್ಲಿ ತಿಳಿಹೇಳುವ ಸಾಮಥ್ರ್ಯ ಸಧೃಢವಾಗಿದೆ ಎಂದು ಮನಗಂಡಿದ್ದೇನೆ.  ಹೀಗೆ ನನ್ನ ಆರೋಗ್ಯದ ವಿಚಾರದಲ್ಲಿ ಚಿಕಿತ್ಸೆಗೆಂದು ಆಯುರ್ವೇದ ವೈದ್ಯ ಡಾ.| ಸಾಂಬಮೂರ್ತಿಯವರಲ್ಲಿ ಹೋಗಿದ್ದೆ. ತಪಾಸಣೆ, ಚಿಕಿತ್ಸೆ, ಸಲಹೆಗಳು ಮುಗಿದಮೇಲೆ, ನನ್ನ ಹಾಗೂ  ನನ್ನ ತಮ್ಮ ಗಂಗಾಧರನ ಸಾಮ್ಯದ ಮೇಲೆ ಅವರು ನಿಮ್ಮ ತಮ್ಮನಾ .. ಅಣ್ಣನಾ ಎಂದು ವಿಚಾರಿಸಿದರು. ಹಾಗೂ ಅವರ ಬೋಧನಾ ಸಾಮಥ್ರ್ಯ ಅಗಾಧವಾಗಿದೆ. ಅತಿಕ್ಲಿಷ್ಟ ವಿಷಯವನ್ನೂ ಚಿಕ್ಕಮಕ್ಕಳಿಗೆ ತಿಳಿಹೇಳುವಂತೆ ಮನನ ಮಾಡಿಸಲು ಸಮರ್ಥರು ಎಂದು ಅಂದಿದ್ದರು. ಪರೋಕ್ಷದಲ್ಲಿ ಗುಣಗಾನ.. ಗೌರವ ನೈಜ ಸಾಮಥ್ರ್ಯಕ್ಕೆ  ಪ್ರಮಾಣ ಎಂದು ಗುರುತಿಸಿದ್ದೇನೆ.

ಮಾನವ ಎಂದಮೇಲೆ ದೋಷ..ದೌರ್ಬಲ್ಯಗಳು ಇರುತ್ತವಷ್ಟೆ?  ಶೀಘ್ರಕೋಪ,  ಒಮ್ಮೆ ನಿರಾಕರಿಸಿದ ಯಾ ಒಪ್ಪದ ನಿಲುವನ್ನು ಯಾವುದೇ ಕಾರಣಕ್ಕೂ ರಾಜಿಯಾಗದೇ ಇರುವುದು,  ತನ್ನಯೋಚನಾ ನೇರಕ್ಕೆ ಮಾತ್ರ ಸ್ಪಂದಿಸುವುದು,  ಹೀಗೆ  ಸಮಗ್ರವಾಗಿ ಸ್ವೀಕರಿಸದ  ಕೆಲವು ನಿಲುವುಗಳು ಇರಬಹುದೇನೋ.

ಈ ಸಂದರ್ಭದಲ್ಲಿ ಕೆಲವು  ಘಟನೆಗಳನ್ನು  ಹಂಚಿಕೊಳ್ಳಲು ಬಯಸುತ್ತೇನೆ ....

ಬಾಲ್ಯದ ಆಟದಲ್ಲಿ ತೊಡಗಿದ ಸಮಯದಲ್ಲಿ ಸಹೋದರಿಯ ಅಚಾತುರ್ಯದಿಂದ ಸ್ನಾನದ ಮನೆಯೊಳಗೇ ಇದ್ದ ಕಟ್ಟೆಯಿಲ್ಲದ ಬಾವಿಯಲ್ಲಿ ಬಿದ್ದು ಬಿಟ್ಟ. ಅಲ್ಲೆ ಸಿಕ್ಕ ಅಂಚನ್ನೆ ಆಧರಿಸಿ ಸಾವರಿಸಿಕೊಂಡು  ಕೂಗಿಕೊಂಡ. ಆಗ ಮಕ್ಕಳ ಧ್ವನಿಯನ್ನು ಆಧರಿಸಿ ನಮ್ಮ ಮನೆಯಲ್ಲೆ ಆಶ್ರಯ ಪಡೆದ ಸಂಬಂಧಿಯೊಬ್ಬರು ಬಂದು ನೋಡಿದಾಗ ಏನು ಮಾಡುವುದು ಎಂದು ತಿಳಿಯದೇ ಚಡಪಡಿಸುತ್ತಿದ್ದರು. ಆಗ ಗಂಗಾಧರನೇ, ``ಮಾವಾ ! ಹಗ್ಗ ಬಾವಿಯಲ್ಲಿ ಬಿಡು, ನಾನು ಹಿಡಿದುಕೊಳ್ಳುತ್ತೇನೆ'. ಎಂದು ಸಲಹಿಸಿದಾಗ..... ಸಂಬಂಧಿಗಳು ಹಗ್ಗವನ್ನು ಇಳಿಬಿಟ್ಟು ಮೇಲೆ ಎತ್ತಿ ನಿಟ್ಟುಸಿರು ಬಿಟ್ಟಿದ್ದು ಆಗಿತ್ತು. ಸಿಕ್ಕ ಹಿಡಿಬಳ್ಳಿಯನ್ನೇ  ಆಶ್ರಯಿಸಿ ಮೇಲಕ್ಕೆ ಬರುವ ಧೃಢತೆ ಆಗಲೇ ಕಂಡಿತ್ತು.

ಪ್ರಾಥಮಿಕ ವಿದ್ಯಾರ್ಥಿಗಳಿಂದ ಹಿಡಿದು ಪ್ರೌಢ ಶಿಕ್ಷಣ, ಉನ್ನತ ಶಿಕ್ಷಣ ಹಾಗೂ ಆಯುರ್ವೇದ ವೈದ್ಯರಿಗೆ ಸಂಸ್ಕೃತ ಭಾಷೆಯ ಮೂಲಕ ವಿಷಯ ಬೋಧನೆ... ಮಂಡನೆಯನ್ನು ಎಲ್ಲಾ ಸ್ಥರಗಳಲ್ಲಿ ತೊಡಗಿಸಿಕೊಂಡಿದ್ದು ಈಗ ಪ್ರತ್ಯಕ್ಷದಲ್ಲೆ ಇದೆ. ವಿದ್ಯಾಭ್ಯಾಸದಲ್ಲಿ ಅನೇಕ ಅಡೆತಡೆಗಳು ವಿಫಲ..ಸಫಲ ಪ್ರಯತ್ನಗಳು ಹಾಸುಹೊಕ್ಕಾಗಿತ್ತು. ಆದರೂ ಅನೇಕ ವಿದ್ಯಾರ್ಥಿಗಳಿಗೆ  ಆಶ್ರಯ, ಸಹಾಯ,  ಮಾರ್ಗದರ್ಶನ ನೀಡಿ ಸಮಾಜದಿಂದ ಪಡೆದದ್ದನ್ನು ಮರಳಿಸುವಲ್ಲಿ ಪ್ರಾಮಾಣಿಕವಾದ ಮನಸ್ಥಿತಿಯಿಂದ ಅನುಸರಿಸಿದ್ದಾನೆ ಎಂಬುದು  ಸಮಾಧಾನಕರ.

ಕೂಡುಕುಟುಂಬದಲ್ಲಿ ಬದುಕು ಬಾಹ್ಯವಾಗಿ ಭವ್ಯವಾಗಿದ್ದರೂ ಆಂತರಿಕವಾಗಿ ಮೇಲಾಟಗಳು, ಸಬಲರು ಪ್ರಬಲರನ್ನು ಹತ್ತಿಕ್ಕುವುದು, ಅಮಾಯಕರನ್ನು ದುಡಿಸಿಕೊಳ್ಳುವುದು, ಇವೆಲ್ಲ ಲೌಕಿಕವಾಗಿ ಕ್ಷಮತೆಯುಳ್ಳದ್ದೇ ಆದರೂ ಅಧರಕ್ಕೆ ಸಗಣಿ ಒರೆಸಿ, ಉದರಕ್ಕೆ ವಿಷವುಣಿಸುವ ವಿಕೃತರೂ ಇರುವಲ್ಲಿ ಬದುಕು ಸಹ್ಯವೆನಿಸದು. ಆದರೂ ಮುಂದುವರಿಯುವುದು ಅಗೋಚರ ಸಹ್ಯತೆ. 

ಸಾಮಾಜಿಕ ಕ್ಷಮತೆಗೆ ಒಪ್ಪುವ  ಬದುಕಿನಲ್ಲಿ ಸಮರಸವನ್ನು ಕಾದುಕೊಳ್ಳುವ ವ್ಯಾವಹಾರಿಕ ಪ್ರಜ್ಞೆ,  ಸರಳ ಬದುಕು ಉದರ ಬರಿದಿದ್ದಾಗಲೂ ಅನುಸರಿಸುವ ಜಾಣ್ಮೆ, ಬಹುಶಃ ನಿತ್ಯಜೀವನವನ್ನು ಹಸನಾಗಿಸಲು ಸಹಕರಿಸಿದೆ.ನಮ್ಮಕುಟುಂಬ ಆರ್ಥಿಕವಾಗಿ ಸದೃಢವೇನಲ್ಲ.  ಉಂಡು..ಉಟ್ಟುಕೊಳ್ಳ್ಳಲು  ಸಾಕಷ್ಟು  ಮೂಲಭೂತ ಅಗತ್ಯವನ್ನು ಪೂರೈಸಲು ಹೆಣಗುವ ಸಂಸಾರವದು. ಪಡೆದ ಸಹಾಯಕ್ಕೆ ಪ್ರತಿಸ್ಪಂದಿಸುವ ಮನೋಭಾವ, ತನ್ನಾರ್ಜಿತ ಸಂಗ್ರಹದಲ್ಲಿ ಬಹುಪಾಲು ವ್ಯಯಿಸುವ ಧರ್ಮಪಾಲನೆಯಲ್ಲಿ ತೊಡಗಿಸಿಕೊಂಡಿದ್ದಾನೆ. 

ಪಿತೃಋಣದಲ್ಲಿ ಕೊರತೆ ಕಂಡರು ಎಳೆವಯಸ್ಸಿನವರನ್ನು ಹೃದಯ ಸಾಮಿಪ್ಯದಿಂದ ಕಾಣುವ ಪ್ರಯತ್ನ, ಕೊರತೆಯನ್ನು ಕಣ್ಮರೆಯಾಗಿಸುತ್ತಿದೆ.  ಆರೋಗ್ಯ  ಕ್ಷೀಣಿಸಿ, ನಿವೃತ್ತಿಯ ಕಾಲದಲ್ಲಿ ನೋವನ್ನುಕೊಟ್ಟರೂ ಜಿಜ್ಞಾಸುಗಳಿಗೆ ಗ್ರಂಥಾವಲೋಕನದಿಂದ ಸಲಹೆ ಸೂಚನೆ ನೀಡಿ ವಿಷಯ ಪ್ರತಿಪಾದನೆಯಲ್ಲಿ ಎಗ್ಗಿಲ್ಲದ ನಿಲುವು ತೋರುವ ಮೂಲಕ ನೆಮ್ಮದಿ,  ಸಮಯಕ್ಕೆ ತಕ್ಕ ವಿನಿಯೋಗವನ್ನು ಸಂಪನ್ನಗೊಳಿಸಿಕೊಂಡಿರುತ್ತಾನೆ  ಎಂಬ ಸಾಂತ್ವನ,  ಸಂತಸವಿದೆ.  ಸನ್ಮಂಗಲಾನಿ ಭವಂತು.

