z{\fontsize{14}{16}\selectfont
\chapter{ಗಂಗಾಧರ ನಮ್ಮ ಕುಟುಂಬದ ಹೆಮ್ಮೆ}

\begin{center}
\Authorline{ವಿ~। ಗಣಪತಿ ದೇವರು ಭಟ್ಟ}
\smallskip

ಅಗ್ಗೇರೆ, ಕವಲಕೊಪ್ಪ,\\ 
ಸಿದ್ದಾಪುರ
\addrule
\end{center}
ಗಂಗಾಧರ ಭಟ್ಟರ ಅಧ್ಯಾಪಕ ವೃತ್ತಿಗೆ ಜನವರಿ 30 ಕ್ಕೆ ನಿವೃತ್ತಿ. ಈ ಸಂದರ್ಭದಲ್ಲಿ ನ್ಯಾಯಶಾಸ್ತ್ರದ ವಿದ್ಯಾರ್ಥಿಗಳು ಅವರನ್ನು ಅಭಿವಂದಿಸುವ ಸಲುವಾಗಿ ಗಂಗಾಧರ ಭಟ್ಟರ ಅಭಿವಂದನ ಸಮಿತಿ ರಚಿಸಿಕೊಂಡು ತನ್ಮೂಲಕ ಹೊರತರುತ್ತಿರುವ ಅಭಿವಂದನ ಗ್ರಂಥಕ್ಕೆ ಇದೊಂದು ಕಿರು ಲೇಖನ.

ನನ್ನ ಅಜ್ಜ ಗಣಪಭಟ್ಟರು. ವೇದಾಧ್ಯಯನ ಸಂಪನ್ನರು. ಧರ್ಮಶಾಸ್ತ್ರ, ಜ್ಯೌತಿಷ ಇತ್ಯಾದಿ ಶಾಸ್ತ್ರಬಲ್ಲವರು. ಪೌರೋಹಿತ್ಯ, ಕೃಷಿ ಇವರ ವೃತ್ತಿಗಳು. ಅವರಿಗೆ ದೇವರು ಭಟ್ಟ, ಮಹಾಬಲೇಶ್ವರ ಭಟ್ಟ, ವಿಗ್ನೇಶ್ವರ ಭಟ್ಟ, ರಾಮಕೃಷ್ಣ ಭಟ್ಟ ಎಂಬ ನಾಲ್ಕು ಜನ ಮಕ್ಕಳು. ಇವರಲ್ಲಿ ನಾನು ದೇವರು ಭಟ್ಟರ ಪುತ್ರ. ವಿಘ್ನೇಶ್ವರ ಭಟ್ಟರ ಪುತ್ರ ಗಂಗಾಧರ. ಇವನ ತಾಯಿ ರೇವತೀ. ಇವನಿಗೆ ಎರಡು ಜನ ಸಹೋದರರು, ನಾಲ್ಕು ಜನ ಸಹೋದರಿಯರು.

ಗಂಗಾಧರ ಮೈಸೂರಿನಲ್ಲಿ ಓದಿ ಅಲ್ಲಿಯೇ ಅಧ್ಯಾಪಕನಾಗಿ ನಿವೃತ್ತನಾಗುತ್ತಿದ್ದಾನೆ. ಅವನು, ತಾನು ಓದಿದ್ದು ಮಾತ್ರವಲ್ಲದೇ ಅವರ ಕುಟುಂಬದವರನ್ನೆಲ್ಲ ಕರೆತಂದು ಓದಿಸಿದ್ದಾನೆ. ಈ ದೃಷ್ಟಿಯಿಂದ ಅವನು ಬಹಳ ಶ್ರಮಪಟ್ಟಿದ್ದಾನೆ. ಕುಟುಂಬಕ್ಕೆ ಅವನಿಂದ ಬಹಳವೇ ಉಪಕಾರವಾಗಿದೆ. ಅವನಿಗಿಂತ ಮೊದಲೇ ಅವನ ಅಣ್ಣಂದಿರು, ಅಲ್ಲದೇ ನಾನೂ ಸಹ ಎಲ್ಲರೂ ಮೈಸೂರಿನಲ್ಲೇ ಓದಿದವರು. ನಾನು ಅಲಂಕಾರ ವಿದ್ವತ್ ಮಾಡಿದೆ. ಅಂತೂ ನಮೆಲ್ಲರ ವಿದ್ಯಾಭ್ಯಾಸಕ್ಕೆ ಮೈಸೂರಿನ ಮಹಾರಾಜಸಂಸ್ಕೃತ ಪಾಠಶಾಲೆ ಆಶ್ರಯವಿತ್ತಿದೆ. ಆ ಬಗ್ಗೆ ನನಗೆ ಬಹಳ ಹೆಮ್ಮೆಯಿದೆ. ನಾವೆಲ್ಲ ಅ ವಿದ್ಯಾಕೇಂದ್ರಕ್ಕೆ ಕೃತಜ್ಞರಾಗಿದ್ದೇವೆ. 

ನನ್ನ ತಮ್ಮ ಗಂಗಾಧರ ಚಿಕ್ಕಂದಿನಿಂದಲೇ ಅತ್ಯಂತ ಪ್ರತಿಭಾಸಂಪನ್ನ. ಗೋಕರ್ಣ ಇತ್ಯಾದಿ ಸ್ಥಳದಲ್ಲಿ ಓದಲು ಪ್ರಯತ್ನ ಮಾಡಿದರೂ ಕೊನೆಗೆ ಮೈಸೂರಿನ ಪಾಠಶಾಲೆಯಲ್ಲಿ ಅಧ್ಯಯನ ಮಾಡಿ ಅಲ್ಲೇ ಅಧ್ಯಾಪಕನಾಗಿ ನಿವೃತ್ತನಾಗುತ್ತಿದ್ದಾನೆ. ಇದು ನನಗೆ ಹೆಮ್ಮೆಯನ್ನುಂಟುಮಾಡಿದೆ. ಅವನು ಅಧ್ಯಾಪಕನಾಗಿ ಅನೇಕ ವಿದ್ಯಾರ್ಥಿಗಳಿಗೆ ಮಾರ್ಗದರ್ಶನ ಮಾಡಿದ್ದಾನೆ. ಅವನ ಗರಡಿಯಲ್ಲಿ ಪಳಗಿದ ಆ ವಿದ್ಯಾರ್ಥಿಗಳೆಲ್ಲ ಉತ್ತಮವಾಗಿ ಜೀವನ ಸಾಗಿಸುತ್ತಿದ್ದು ಇಂದು ಅವನನ್ನು ಅತ್ಯಂತ ಗೌರವದಿಂದ ಅಭಿವಂದಿಸುತ್ತಿದ್ದಾರೆ. ಇದು ಅವನ ಜೀವನದ ಒಂದು ಸಾರ್ಥಕ್ಯ ಕ್ಷಣವೆಂದು ನಾನು ಭಾವಿಸುತ್ತೇನೆ. ಇದು ನನಗೆ, ನಮ್ಮ ಕುಟುಂಬಕ್ಕೆ ಅತ್ಯಂತ ಹೆಮ್ಮೆಯ ವಿಷಯವೇ ಸರಿ.

ಸಂಸ್ಕೃತ ಪಾಠಶಾಲೆ ಮೊದಲು ಮಹಾರಾಜರಿಂದ ಸ್ಥಾಪನೆಯಾಗಿ ಅವರ ವ್ಯವಸ್ಥೆಯಿಂದಲೇ ನಡೆಯುತ್ತಿತ್ತು. ಅಲ್ಲಿ ಓದುವವರಿಗೆ ಪೂರ್ಣಯ್ಯನ ಛತ್ರದಲ್ಲಿ ಭೋಜನ ವ್ಯವಸ್ಥೆಯಿತ್ತು. ನಾನು ಅಲ್ಲಿಯೇ ಭೋಜನ ವ್ಯವಸ್ಥೆ ಪಡೆದಿದ್ದೆ. ಅನಂತರ ಅದು ನಿಂತು ಹೋದಾಗ ವೇದಶಾಸ್ತ್ರಪೋಷಿಣೀ ಸಭಾ ಎಂಬ ಸಂಸ್ಥೆ ಜನ್ಮ ತಾಳಿ ಪಾಠಶಾಲೆಯ ವಿದ್ಯಾರ್ಥಿಗಳಿಗೆ ಭೋಜನ ವ್ಯವಸ್ಥೆಯನ್ನು ಕಲ್ಪಿಸಿತು. ಇಂದೂ ಸಹ ಇದು ಈ ಕಾರ್ಯವನ್ನು ಮಾಡುತ್ತಿದೆ. ನಮ್ಮ ಗಂಗಾಧರ ಆರಂಭದಲ್ಲಿ ಭೋಜನಕ್ಕೆ ಬಹಳ ಕಷ್ಟಪಟ್ಟು ವಾರಾನ್ನದ ವ್ಯವಸ್ಥೆಯಿಂದ ಓದಿದ್ದಾನೆ. ಅನಂತರ ಕಾಲದಲ್ಲಿ ವೇದಶಾಸ್ತ್ರಪೋಷಿಣೀ ಸಭಾದ ಪ್ರಯೋಜನವನ್ನೂ ಪಡೆದಿದ್ದಾನೆ. ಆದ್ದರಿಂದ ಈ ಸಂಸ್ಥೆಯ ಮೇಲೆ ನಮಗೆಲ್ಲರಿಗೂ ಅಭಿಮಾನವಿರುವುದಲ್ಲದೇ ಸಂಸ್ಥೆಗೆ ನಾವೆಲ್ಲ ಋಣಿಗಳಾಗಿದ್ದೇವೆ. 

ಗಂಗಾಧರ ಪಾಠಶಾಲೆಯ ಬೋಧನೆಗೆ ಮಾತ್ರ ಸೀಮಿತವಾಗದೇ ದೇಶವಿದೇಶಗಳಿಗೂ ತೆರಳಿ ವಿಷಯ ಸಮರ್ಥಿಸುವ ಸಾಮಥ್ರ್ಯ ಹೊಂದಿದ್ದಾನೆ. ಹತ್ತಾರು ಸಾವಿರ ಜನ ಸೇರಿದ ಸಭೆಯಲ್ಲಿ ಜನರನ್ನು ಮಂತ್ರಮುಗ್ಧರನ್ನಾಗಿಸುವ ವಾಕ್ಪಟುತ್ವ ಹೊಂದಿದ್ದಾನೆ. ಇದು ನಮ್ಮ ಕುಟುಂಬಕ್ಕೂ ಸಮಾಜಕ್ಕೂ ಹೆಮ್ಮೆಯ ವಿಚಾರ. ಅವನು ಉತ್ತಮ ಉಪನ್ಯಾಸಕ ಮಾತ್ರನಲ್ಲ, ಉತ್ತಮ ಲೇಖಕನೂ ಹೌದು. ಹಾಗಾಗಿ ಅನೇಕ ಜನರು ಪ್ರವಚನಕ್ಕೂ ಮತ್ತು ಬರವಣಿಗೆಗೂ ಇವನ ಮಾರ್ಗದರ್ಶನ ಪಡೆಯುತ್ತಾರೆ. 

ಇನ್ನೊಂದು ಪ್ರಮುಖ ವಿಷಯವೆಂದರೆ, ಉತ್ತರ ಕನ್ನಡ ಜಿಲ್ಲೆಯ ಶಿರಸಿಯ ಸಮೀಪವಿರುವ ಸ್ವರ್ಣವಲ್ಲೀ ಮಹಾಸಂಸ್ಥಾನದ ಪೀಠಾಧಿಪತಿಗಳಾಗಿರುವ ಶ್ರೀಶ್ರೀ ಗಂಗಾಧರೇಂದ್ರ ಸರಸ್ವತೀ ಶ್ರೀಗಳಿಗೂ ಸಹ ಪೂರ್ವಾಶ್ರಮದಲ್ಲಿ ನಮ್ಮ ಗಂಗಾಧರ ಶಾಸ್ತ್ರ ಬೋಧನೆಯನ್ನು  ಮಾಡಿದ್ದಾನೆ. ಅವರು ಇಂದು ಸಮಾಜಕ್ಕೆಲ್ಲ ಮಾನ್ಯರಾಗಿದ್ದು, ಸ್ವತಃ ಶಾಸ್ತ್ರವೇತ್ತರಾಗಿ ಹೊರಹೊಮ್ಮುವಲ್ಲಿ ಗಂಗಾಧರನ ಪಾತ್ರವೂ ಇದ್ದು, ಅದು ಅವನ ಪ್ರಾಧ್ಯಾಪಕ ವೃತ್ತಿಗೆ ಒಂದು ಸಾರ್ಥಕ್ಯದ ಭಾವವನ್ನು ಕೊಟ್ಟಿದೆಯೆಂದು ನಾನು ಭಾವಿಸುತ್ತೇನೆ. ಇದು ನಮಗೆಲ್ಲ ಹೆಗ್ಗಳಿಕೆಯ ವಿಷಯವೇ ಸರಿ. 

ಇನ್ನು, ದೇಶೀಯರು ಮಾತ್ರವಲ್ಲದೇ ವಿದೇಶೀಯರು ಸಹ ಇವನಲ್ಲಿ ಬಂದು ಅಧ್ಯಯನ ಮಾಡುತ್ತಾರೆ. ಇಂದು ವಿದೇಶಿಗರಿಗೆ ಪಾಠಮಾಡುವುದು ದೊಡ್ಡ ಬಿಜಿನೆಸ್ಸೇ ಆಗಿಬಿಟ್ಟಿದೆ. ಆದರೆ ವಿದ್ಯಾರ್ಥಿಗಳಿಂದ ಯಾವುದೇ ಪ್ರತಿಫಲಾಪೇಕ್ಷೆ ಇಲ್ಲದೇ ಉದಾರವಾಗಿ ಬೋಧನೆ ಮಾಡುತ್ತಾನೆ. ವಿದ್ಯಾರ್ಥಿಗಳನ್ನು ತನ್ನ ಸ್ವಂತ ಮಕ್ಕಳಂತೆ ಕಾಣುತ್ತಾನೆ. ಪಾಠ ಮಾಡುವುದಲ್ಲದೇ ಮಕ್ಕಳಿಗೆ ವಿವಿಧವಾಗಿ ಉಪಕರಿಸುವುದು ಇವನ ವಿಶಾಲಗುಣ, ವಿಶೇಷಗುಣ. ಇವನ ತಂದೆಯೂ ಸಹ ಸ್ವತಃ ತನಗೇ ಇದೆಯೋ ಇಲ್ಲವೋ ಎಂಬುದನ್ನೂ ಪರಿಗಣಿಸದೇ ದಾನ ಮಾಡುವ ಸ್ವಭಾವದವರು. ಅದು ಇವನಿಗೂ ಬಂದಿದೆ ಎಂಬುದರಲ್ಲಿ ಯಾವುದೇ ಸಂದೇಹವಿಲ್ಲ. ಇನ್ನು, ಯಾವುದೇ ವಿಷಯ ತನಗೆ ಹಿಡಿಸದೇ ಇರುವ ಪಕ್ಷೇ ವಜ್ರಾದಪಿ ಕಠೋರಾಣಿ ಎಂಬಂತೆಯೂ, ಸಾತ್ತ್ವಿಕರಲ್ಲಿ ಮೃದೂನಿ ಕುಸುಮಾದಪಿ ಎಂಬ ರೀತಿಯಲ್ಲಿಯೂ ವರ್ತಿಸುವ ಅವನ ಸ್ವಭಾವವೂ ನೆನಪಿಸಿಕೊಳ್ಳಬೇಕಾದ ವಿಚಾರವೇ ಹೌದು.

ಅವನ ಗೃಹಿಣಿಯಾದ ಚಿ.ಸೌ. ಶೈಲಜಾ ಸಹ ಅವನ ಗುಣಗಳಿಗೆ ಅನುಗುಣವಾಗಿದ್ದಾಳೆ. ಸುಸಂಸ್ಕೃತ ಮತ್ತು ಸಭ್ಯ ಗೃಹಿಣೀ. ಅವನ ಕೌಟುಂಬಿಕ, ಸಾಮಾಜಿಕ ಮತ್ತು ವ್ಯಾವಹಾರಿಕ ಆಗುಹೋಗುಗಳನ್ನು ಚೆನ್ನಾಗಿ ತೂಗಿಸಿಕೊಂಡು ಎಲ್ಲ ವಿಧದಲ್ಲೂ ಅವನಿಗೆ ಸಹಾಯಕಳಾಗಿದ್ದಾಳೆ. ಬಂಧುಗಳಲ್ಲೂ ಸಹ ಉತ್ತಮ ಬಾಂಧವ್ಯವನ್ನು ಇಟ್ಟುಕೊಂಡಿದ್ದಾಳೆ. 

ಗಂಗಾಧರನ ವಿಷಯದಲ್ಲಿ ಸಾಕಷ್ಟು ಹೇಳುವ ವಿಷಯವಿದ್ದರೂ ಅವನ ವ್ಯಕ್ತಿತ್ವ, ಪಾಂಡಿತ್ಯವನ್ನು ಈಗ ಇಷ್ಟೇ ಬರವಣಿಗೆಯಿಂದ ತಿಳಿಸುವುದು ಸಾಧ್ಯವಾಗುತ್ತಿದೆ. ಅವನ ಮುಂದಿನ ಜೀವನ ಸುಖಮಯವಾಗಿರಲೆಂದು ಹಾರೈಸಿ ನನ್ನ ಲೇಖನಿಗೆ ವಿರಾಮ ಕೊಡುತ್ತೇನೆ.

\centerline{ಸಮಸ್ತ ಸನ್ಮಂಗಲಾನಿ ಭವಂತು.}

\articleend
}
