\chapter{ವಿದ್ವತ್ತಿನ ಗಣಿ ಆತ್ಮೀಯ ಮಿತ್ರ}

\begin{center}
\Authorline{ವಿದ್ವಾನ್ ಗಂಗಾಧರ ವಿ. ಭಟ್ಟರು}
\smallskip

ವಿ~। ಟಿ.ವಿ. ಸತ್ಯನಾರಾಯಣ ಎಂ.ಎ.\\
ನಿವೃತ್ತ ಉಪನಿರ್ದೇಶಕ ಮತ್ತು\\ 
ಹಾಲಿ ಗೌರವಸಲಹಾಗಾರರು\\
ಪ್ರಾಚ್ಯವಿದ್ಯಾಸಂಶೋಧನಾಲಯ\\
ಮೈಸೂರು
\addrule
\end{center}

ಮೈಸೂರಿನ ಮಹಾರಾಜ ಸಂಸ್ಕೃತ ಕಾಲೇಜಿನಲ್ಲಿ ತರ್ಕಪ್ರಾಧ್ಯಾಪಕರಾಗಿರುವ ವಿದ್ವಾನ್ ಗಂಗಾಧರ ವಿ.ಭಟ್ಟ ಅವರ ಬಗ್ಗೆ ನಾಲ್ಕಾರು ಮಾತುಗಳನ್ನು ಬರೆಯುವುದೆಂದರೆ, ಅದು ಸಂತೋಷವನ್ನು ನೀಡುವ ವಿಷಯ. ವಿದ್ವತ್ತಿನ ಆಕರವಾಗಿರುವ ಅವರು ನಮ್ಮ ಸಂಸ್ಕೃತಸಾಹಿತ್ಯ ಲೋಕಕ್ಕೆ ಹಾಗೂ ಜನಸಾಮಾನ್ಯರಿಗೂ ಬೇಕಾಗಿರುವಂತಹ ಅಪರೂಪದ ಅನರ್ಘ್ಯರತ್ನವಾಗಿದ್ದಾರೆ. ನನ್ನ ಅವರ ಪರಿಚಯ ಅನೇಕ ವರ್ಷಗಳದು. 

ನಾನು ಮೈಸೂರಿನ ಮಹಾರಾಜ ಸಂಸ್ಕೃತ ಮಹಾಪಾಠಶಾಲೆಯಲ್ಲಿ ಅಲಂಕಾರಶಾಸ್ತ್ರವನ್ನು ಓದುತ್ತಿದ್ದ ಸಮಯದಲ್ಲಿ ಗಂಗಾಧರಭಟ್ಟರು, ಮೈಸೂರಿಗೆ ಬಂದವರು. ಇವರ ಹುಟ್ಟಿದ ಊರು ಉತ್ತರಕನ್ನಡಜಿಲ್ಲೆಯ ಸಿದ್ಧಾಪುರ ತಾಲ್ಲೂಕಿನ ಮಣ್ಣಿಕೊಪ್ಪ ಎಂಬ ಹಳ್ಳಿ. ಆ ಊರಿನಲ್ಲಿ ತೋಟ–ತುಡಿಕೆ ಮಾಡಿಕೊಂಡು, ಪೌರೋಹಿತ್ಯ, ಸಂಸ್ಕೃತವನ್ನು ಓದಿಕೊಂಡಿದ್ದ ತಿಳಿದವರ ಮನೆತನ ಇವರದು. ಇವರ ತಂದೆ ಶ್ರೀ ವಿಘ್ನೇಶಭಟ್ಟರು. ಇವರು ವೇದವನ್ನು, ಸಂಸ್ಕೃತವನ್ನು ಓದಿ ಆ ಹಳ್ಳಿಯಲ್ಲಿ ಬಲ್ಲವರಾಗಿದ್ದರು.

ವಿಶೇಷವಾಗಿ ಶಾಂಕರದರ್ಶನದಲ್ಲಿ ಇವರಿಗೆ ಅಪಾರವಾದ ಆಸಕ್ತಿ ಶ್ರದ್ಧೆ. ಇವರಿಗೆ ಸದಾ ಓದುವ ಹವ್ಯಾಸ. ಪೌರೋಹಿತ್ಯ, ಕೃಷಿಕಾರ್ಯಗಳಲ್ಲಿ ತಮ್ಮನ್ನು ತೊಡಗಿಸಿಕೊಂಡು ಬಾಳನ್ನು ಕಂಡವರು ಇವರು. ಗಂಗಾಧರಭಟ್ಟರ ತಾಯಿ ಶ್ರೀಮತಿ ರೇವತಿಯವರು. ಇವರೂ ಸಹ ಪತಿಗೆ ತಕ್ಕ ಹೆಂಡತಿ. ಗೃಹಕಾರ್ಯಗಳಲ್ಲಿ ನಿರತರಾಗಿ ಮಕ್ಕಳ ಪಾಲನೆ ಪೋಷಣೆಯಲ್ಲಿ ತೊಡಗಿದ್ದವರು. ಗಂಗಾಧರಭಟ್ಟರ ದೊಡ್ಡ ಅಣ್ಣ ವಿದ್ವಾನ್ ಮಂಜುನಾಥ ಭಟ್ಟರು. ಅವರು ನಮ್ಮ ಮೈಸೂರಿನ ಮಹಾರಾಜ ಸಂಸ್ಕೃತ ಮಹಾಪಾಠಶಾಲೆಗೆ ಬಂದು ಪಾಠಶಾಲೆಯ ಹಿಂದಿನ ರೂಮಿನಲ್ಲಿದ್ದುಕೊಂಡು ಅದ್ವೈತವೇದಾಂತವನ್ನು ವಿದ್ವಾನ್ ರಾಮಚಂದ್ರ ಸೋಮಯಾಜಿಗಳಲ್ಲಿ ಮತ್ತು ವಿದ್ವಾನ್ ನಾರಾಯಣಭಟ್ಟರಲ್ಲಿ ಓದುತ್ತಿದ್ದು ನಂತರ ಅದ್ವೈತವೇದಾಂತ ವಿದ್ವತ್ತನ್ನು ಪೂರೈಸಿದರು. 

ಅನಂತರ ಅವರು ತಮ್ಮ ಊರಿಗೇ ಹೋಗಿ ಸಂಸ್ಕೃತ ಪಾಠಪ್ರವಚನಗಳನ್ನು ಮಾಡುತ್ತಾ, ತಮ್ಮ ಮನೆಯ ಜವಾಬ್ದಾರಿಯನ್ನು ನೋಡುತ್ತಾ ತಂದೆ–ತಾಯಿಗಳ ಜೊತೆ ಸಂತೋಷದಿಂದ ಕಾಲ ಕಳೆಯುತ್ತಿದ್ದರು. ಗಂಗಾಧರಭಟ್ಟರ ಎರಡನೆಯ ಅಣ್ಣ ಶ್ರೀಧರಭಟ್ಟರು. ಇವರೂ ಸಹ ಸ್ವಲ್ಪಕಾಲ ಮೈಸೂರಿನ ಮಹಾರಾಜ ಸಂಸ್ಕೃತ ಕಾಲೇಜಿನ ರೂಮಿನಲ್ಲಿದ್ದುಕೊಂಡು ಸಂಸ್ಕೃತವನ್ನು ಕಲಿತರು. ಆಮೇಲೆ ಊರಿಗೆ ಹಿಂದಿರುಗಿ ಸಿದ್ಧಾಪುರದಲ್ಲಿ ಆಂಗಿರಸ ಎಂಬ ಖಾಸಗಿ ಮುದ್ರಣಾಲಯವೊಂದನ್ನು ಇಟ್ಟುಕೊಂಡು ಜೀವನ ನಡೆಸತೊಡಗಿದರು.

1973–1978ರ ಅವಧಿಯ ಕಾಲವೆಂದು ನನ್ನ ನೆನಪು. ಆವಾಗ ನಾನು ಅಲಂಕಾರಶಾಸ್ತ್ರ ವಿದ್ವತ್ತರಗತಿಯಲ್ಲಿ ಓದುತ್ತಿದ್ದೆ. ಗಂಗಾಧರಭಟ್ಟರಿಗಿಂತ ಮೊದಲೇ ನಾನು ಪಾಠಶಾಲೆಯಲ್ಲಿ ಓದುತ್ತಿದ್ದೆ. ಆ ಸಮಯದಲ್ಲಿ ಗಂಗಾಧರಭಟ್ಟರು ಮೈಸೂರಿಗೆ ಬಂದರು. ಮೊದಲಿಗೆ ಇವರು ತಮ್ಮ ಊರಿನ ಬಳಿ ಬಿದ್ರಕಾನ್ ಎಂಬ ಹಳ್ಳಿಯ ಬಳಿ ಹಳತಕಟ್ಟದಲ್ಲಿನ ಸಂಸ್ಕೃತ ಪಾಠಶಾಲೆಯಲ್ಲಿ ಪ್ರಥಮ ತರಗತಿಯನ್ನು ಸುಬ್ರಹ್ಮಣ್ಯಭಟ್ಟರಲ್ಲಿ ಓದಿದರು. ತಮ್ಮ ಊರಿನಲ್ಲೇ ಪ್ರಾಥಮಿಕ ವಿದ್ಯಾಭ್ಯಾಸ ಮಾಡಿ ಬಿದ್ರಕಾನ್‍ನಲ್ಲಿದ್ದ ಪ್ರೌಢಶಾಲೆಯಲ್ಲಿ ಎಸ್.ಎಸ್.ಎಲ್.ಸಿ. ಮುಗಿಸಿ ಮೈಸೂರಿಗೆ ಬಂದರು.

ಮೈಸೂರಿನ ಮಹಾರಾಜ ಸಂಸ್ಕೃತ ಪಾಠಶಾಲೆಯಲ್ಲಿ ಕೃಷ್ಣಯಜುರ್ವೇದ ಪ್ರಥಮ ತರಗತಿಗೆ ಸೇರಿ, ಹಿಂದಿನ ರೂಮಿನಲ್ಲಿ ವಾಸಿಸುತ್ತಿದ್ದರು. ಮೈಸೂರಿನ ಶ್ರೀರಾಮಮಿಶ್ರ ಸಂಸ್ಕೃತ ಪಾಠಶಾಲೆಯಲ್ಲಿ ಪ್ರೈವೇಟಾಗಿ ಕಾವ್ಯ ಪರೀಕ್ಷೆ ಮುಗಿಸಿ, ಅನಂತರ ಮೈಸೂರು ಮಹಾರಾಜ ಸಂಸ್ಕೃತ ಮಹಾಪಾಠಶಾಲೆಯಲ್ಲಿ ಸಾಹಿತ್ಯ ತರಗತಿಗೆ ಸೇರಿ ಸಾಹಿತ್ಯ ಪರೀಕ್ಷೆಯನ್ನು ಮುಗಿಸಿ, ಅನಂತರ ತರ್ಕಶಾಸ್ತ್ರದ ನವೀನನ್ಯಾಯ ವಿದ್ವತ್ತಿಗೆ ಸೇರಿದರು. ವಿದ್ವಾನ್ ಶ್ರೀನಾಥಾಚಾರ್ಯರು, ವಿದ್ವಾನ್ ಎ. ವೆಂಕಣ್ಣಾಚಾರ್ಯರು, ವಿದ್ವಾನ್ ಎನ್.ಎಸ್.ರಾಮಭದ್ರಾಚಾರ್ಯರು ಇವರ ತರ್ಕಶಾಸ್ತ್ರದ ವಿದ್ಯಾಗುರುಗಳು. ಓದುತ್ತಿದ್ದ ಕಾಲದಲ್ಲಿ ತುರುವೇಕೆರೆ ವಿದ್ವಾನ್ ವಿಶ್ವೇಶ್ವರದೀಕ್ಷಿತರಲ್ಲೂ ಓದಿ ತಿಳಿದುಕೊಳ್ಳುವ ಅವಕಾಶ ಇವರಿಗೆ ಒದಗಿತು. 

ವಿದ್ವಾನ್ ಎನ್.ಎಸ್.ರಾಮಭದ್ರಾಚಾರ್ಯರಿಂದ ತರ್ಕಶಾಸ್ತ್ರದಲ್ಲಿ ತುಂಬಾ ವಿಷಯಗಳನ್ನು ತಿಳಿದುಕೊಂಡು ಪ್ರಖರವಾದ ಪಾಂಡಿತ್ಯವನ್ನು ಸಂಪಾದಿಸಿದರು. ಅವರ ಪಾಠಪ್ರವಚನ, ಮಾರ್ಗದರ್ಶನಗಳು ಗಂಗಾಧರಭಟ್ಟರಿಗೆ ಬಹಳವಾಗಿ ದೊರಕಿದವು. ವಿದ್ವಾನ್ ಎನ್.ಎಸ್.ರಾಮಭದ್ರಾಚಾರ್ಯರ ಬಗ್ಗೆ ಗಂಗಾಧರಭಟ್ಟರಿಗೆ ಅಪಾರ ಗೌರವ–ಭಕ್ತಿ.

ಪಾಠಶಾಲೆಯಲ್ಲಿ ಓದುತ್ತಿದ್ದ ಕಾಲದಲ್ಲಿ ಗಂಗಾಧರಭಟ್ಟರು, ತುಂಬಾ ಪ್ರತಿಭಾಶಾಲಿ ವಿದ್ಯಾರ್ಥಿಯಾಗಿದ್ದರು. ವಿದ್ಯಾರ್ಥಿಸಂಘ, ಪ್ರದೋಷಸಂಘದ ಕಾರ್ಯಕ್ರಮಗಳಲ್ಲಿ ಸಕ್ರಿಯವಾಗಿ ಭಾಗವಹಿಸುತ್ತಿದ್ದರು. ಅಂತರವಿದ್ಯಾರ್ಥಿನಿಲಯ ಚರ್ಚಾಸ್ಪರ್ಧೆಗಳು ಹಾಗೂ ಇತರ ಸ್ಪರ್ಧೆಗಳು, ಕಾಲೇಜಿನಲ್ಲಿ ನಡೆಯುವ ಇತರ ಸ್ಪರ್ಧೆಗಳಲ್ಲೂ ಭಾಗವಹಿಸಿ, ಷೀಲ್ಡ್‍ನ್ನು, ಬಹುಮಾನಗಳನ್ನು ಪಡೆದವರು. ಅಖಿಲಭಾರತ ವಾಕ್ಪ್ರತಿಯೋಗಿತಾಸ್ಪರ್ಧೆಗಳಲ್ಲೂ ಕರ್ನಾಟಕದ ಪ್ರತಿನಿಧಿಯಾಗಿ, ಮೈಸೂರು ಪಾಠಶಾಲೆಯ ವತಿಯಿಂದ ಭಾಗವಹಿಸಿ, ಚಿನ್ನ, ಬೆಳ್ಳಿ ಪದಕಗಳನ್ನು ಪಡೆದ ಪ್ರತಿಭಾಶಾಲಿ ವಿದ್ಯಾರ್ಥಿಯಾಗಿದ್ದರು. ವಿದ್ಯಾರ್ಥಿದೆಸೆಯಲ್ಲೇ ಇವರು ತುಂಬಾ ವಾಗ್ಮಿಗಳು. ಓದುವುದರಲ್ಲೂ ಅಷ್ಟೇ ಮುಂದು. ಇವರ ಜೊತೆಯಲ್ಲಿ ಸದಾ ಇರುತ್ತಿದ್ದವರು ಇವರ ಅಚ್ಚುಮೆಚ್ಚಿನ ಸ್ನೇಹಿತರಾಗಿದ್ದ ವಿದ್ವಾನ್ ಉಮಾಕಾಂತಭಟ್ಟರು. ಇವರಿಬ್ಬರೂ ಸದಾ ಪಾಠಶಾಲೆಯ ಎಲ್ಲ ಕಾರ್ಯಚಟುವಟಿಕೆಗಳಲ್ಲಿ ಜೊತೆಜೊತೆಯಾಗಿ ಭಾಗವಹಿಸುತ್ತಿದ್ದರು.

ಗಂಗಾಧರಭಟ್ಟರು ಪಾಠಶಾಲೆಯಲ್ಲಿ ಓದುತ್ತಿದ್ದಾಗಲೇ ಮೈಸೂರಿನ ಬನುಮಯ್ಯ ಕಾಲೇಜಿಗೆ ಸೇರಿ ಬಿ.ಕಾಮ್. ಪದವಿಯನ್ನು ಪಡೆದುಕೊಂಡರು. ನಂತರ ಜೀವನ ನಿರ್ವಹಣೆ ಕೈಬೀಸಿ ಕರೆಯುತ್ತಿದ್ದುದರಿಂದ, ಮೈಸೂರಿನ ಸಂಸ್ಕೃತ ಮಹಾಪಾಠಶಾಲೆಯಲ್ಲೇ ಮೀಮಾಂಸಾಶಾಸ್ತ್ರ ಓದುತ್ತಿದ್ದ ವಿದ್ವಾನ್ ರಮಾನಂದ ಅವಭೃತ ಅವರ ಮೂಲಕ, ಮೈಸೂರಿನ ಶಂಕರಮಠದ ಬಳಿಯಿರುವ ಜಪದಕಟ್ಟೆಮಠದ ಶಂಕರವಿಲಾಸ ಸಂಸ್ಕೃತ ಪಾಠಶಾಲೆಯಲ್ಲಿ ಸಂಸ್ಕೃತ ಅಧ್ಯಾಪಕರಾಗಿ ಸೇರಿಕೊಂಡರು.

ಅಲ್ಲಿ ಪ್ರಥಮ–ಕಾವ್ಯ ತರಗತಿಗಳ ವಿದ್ಯಾರ್ಥಿಗಳಿಗೆ ಅನೇಕ ವರ್ಷಗಳು ಸಂಸ್ಕೃತವನ್ನು ಪಾಠಮಾಡಿದರು. ಕೆಲವು ದಿನಗಳ ನಂತರ ಸಾಹಿತ್ಯ ತರಗತಿಯೂ ಪ್ರಾರಂಭವಾಗಿ ಅದರಲ್ಲೂ ಪಾಠಗಳನ್ನು ಮಾಡಿದರು ಎಂದು ನೆನಪು. ಕೆಲವಾರು ವರ್ಷಗಳ ನಂತರ ಅಲ್ಲಿಯೇ ಮುಖ್ಯೋಪಾಧ್ಯಾಯರೂ ಆದರು. ಉತ್ತಮವಾದ ಸಂಸ್ಕೃತ ಪಾಂಡಿತ್ಯ ಗಳಿಸಿದ್ದ ಇವರು ಮಕ್ಕಳಿಗೆ ಅರ್ಥವಾಗುವಂತೆ ಸಂಸ್ಕೃತವನ್ನು ಪಾಠ ಮಾಡುತ್ತಿದ್ದರು. ವಿದ್ಯಾರ್ಥಿಗಳಿಗೆ ಗಂಗಾಧರಭಟ್ಟರೆಂದರೆ ಅತ್ಯಂತ ಅಚ್ಚುಮೆಚ್ಚಿನ ಅಧ್ಯಾಪಕರು ಹಾಗೂ ಗುರುಗಳು. 

ಪಾಠ ಮಾಡುವುದಷ್ಟೇ ಅಲ್ಲ. ಇತರ ಶೈಕ್ಷಣಿಕ ಚಟುವಟಿಕೆಗಳಲ್ಲೂ ವಿದ್ಯಾರ್ಥಿಗಳನ್ನು ತೊಡಗಿಸಿ ಅವರಿಗೆ ಉತ್ತಮ ಮಾರ್ಗದರ್ಶನ ನೀಡಿದ್ದಾರೆ. ಇವರಲ್ಲಿ ಸಂಸ್ಕೃತ ಓದಿದ ನೂರಾರು ವಿದ್ಯಾರ್ಥಿಗಳಲ್ಲಿ ಅನೇಕರು ಇಂದು ಉತ್ತಮ ಹುದ್ದೆಯಲ್ಲಿದ್ದು ಸಂಸ್ಕೃತ ಅಧ್ಯಾಪಕರಾಗಿ ಕೆಲಸ ಮಾಡುತ್ತಿದ್ದಾರೆ. ಇವರನ್ನು ಅವರೆಲ್ಲ ಸದಾ ನೆನಪಿಸಿಕೊಳ್ಳುತ್ತಾರೆ. ಶಂಕರವಿಲಾಸ ಸಂಸ್ಕೃತ ಪಾಠಶಾಲೆಯ ಸರ್ವತೋಮುಖ ಅಭಿವೃದ್ಧಿಗೆ ವಿದ್ವಾನ್ ಗಂಗಾಧರಭಟ್ಟರು, ತಮ್ಮನ್ನು ಗಂಧದ ಕಲ್ಲಿನ ಮೇಲೆ ತೇದುಕೊಂಡಂತೆ ತೇದುಕೊಂಡಿದ್ದಾರೆ. ಮೈಸೂರಿನಲ್ಲಿ ಆ ಪಾಠಶಾಲೆಯನ್ನು ಅಚಲವಾಗಿ ನಿಲ್ಲಿಸಿದ್ದಾರೆ. ಈ ಕೀರ್ತಿ ಇವರಿಗೇ ಸಲ್ಲುತ್ತದೆ.

ಎಷ್ಟೋ ವರ್ಷಗಳ ನಂತರ ವಿದ್ವಾನ್ ಗಂಗಾಧರಭಟ್ಟರು, ಮೈಸೂರು ಮಹಾರಾಜ ಸಂಸ್ಕೃತ ಕಾಲೇಜಿನಲ್ಲಿ ನ್ಯಾಯಶಾಸ್ತ್ರದ ಅಧ್ಯಾಪಕರಾಗಿ ನೇಮಕಗೊಂಡು, ತಾವು ಸೇರಿಕೊಂಡಂದಿನಿಂದ ಇಂದಿನವರೆಗೂ ಇವರು ನ್ಯಾಯಶಾಸ್ತ್ರ ವಿದ್ಯಾರ್ಥಿಗಳಿಗೆ ಸತತವಾಗಿ, ಪ್ರಾಮಾಣಿಕವಾಗಿ, ಉತ್ಸಾಹದಿಂದ ತರ್ಕಶಾಸ್ತ್ರ ಪಾಠಗಳನ್ನು ಮಾಡಿದ್ದಾರೆ. ಕಾರಿಕಾವಲೀ, ಮುಕ್ತಾವಲೀ, ದಿನಕರೀ, ನ್ಯಾಯಮಂಜರೀ, ಗದಾಧರೀಯ ಮೊದಲಾದ ಪ್ರಾಚೀನನ್ಯಾಯ ಮತ್ತು ನವೀನನ್ಯಾಯಶಾಸ್ತ್ರಗ್ರಂಥಗಳನ್ನು ವಿದ್ಯಾರ್ಥಿಗಳಿಗೆ ಅರ್ಥವಾಗುವ ರೀತಿಯಲ್ಲಿ ಬೋಧಿಸುತ್ತಿದ್ದರು. ತರ್ಕಶಾಸ್ತ್ರ ಕಠಿಣ ಎಂಬ ಮನೋಭಾವ ಬರದಂತೆ ಅದರಲ್ಲಿ ಆಸಕ್ತಿ ಕುತೂಹಲ ಹುಟ್ಟುವಂತೆ ತರ್ಕಶಾಸ್ತ್ರವನ್ನು ಪಾಠ ಮಾಡುವ ಕಲೆ ಇವರಿಗೆ ಕರಗತವಾಗಿದೆ. ಹಾಗಾಗಿ ವಿದ್ಯಾರ್ಥಿಗಳಿಗೆ ಮನಮುಟ್ಟುವಂತೆ ಪಾಠ ಮಾಡುವ ಶೈಲಿ ಇವರದು.

ಇವರು ಮಹಾರಾಜ ಸಂಸ್ಕೃತ ಕಾಲೇಜಿನಲ್ಲಿ ಅಧ್ಯಾಪಕ ವೃತ್ತಿಯ ಜೊತೆಜೊತೆಗೆ ಕಾಲೇಜಿನ ಅಧ್ಯಾಪಕ ಸಂಘದ ಕಾರ್ಯಗಳಲ್ಲಿ ತಮ್ಮನ್ನು ತಾವು ತೊಡಗಿಸಿಕೊಂಡು ಅಧ್ಯಾಪಕರ ಅಭಿವೃದ್ಧಿಗಾಗಿ ಶ್ರಮಿಸಿದ್ದಾರೆ. ರಾಜ್ಯದ ಬೇರೆ ಬೇರೆ ಸಂಸ್ಕೃತ ಕಾಲೇಜುಗಳಲ್ಲಿ ನಡೆಯುವ ಶೈಕ್ಷಣಿಕ ಸ್ಪರ್ಧೆಗಳಲ್ಲಿ ಭಾಗವಹಿಸಲು ತಮ್ಮ ಕಾಲೇಜಿನ ವಿದ್ಯಾರ್ಥಿಗಳನ್ನು ಕರೆದುಕೊಂಡು ಹೋಗಿ, ಅವರಿಗೆ ಮಾರ್ಗದರ್ಶನ ನೀಡಿ ಬಹುಮಾನಗಳನ್ನು ಗಳಿಸುವಂತೆ ಮಾಡಿದ್ದಾರೆ. ಅಖಿಲಭಾರತ ಸಂಸ್ಕೃತ ವಾಕ್ಪ್ರತಿಯೋಗಿತಾ ಸ್ಪರ್ಧೆಗಳಿಗೂ ವಿದ್ಯಾರ್ಥಿಗಳನ್ನು ಕರೆದುಕೊಂಡು ಹೋಗಿ, ಪದಕಗಳನ್ನು, ಬಹುಮಾನಗಳನ್ನು ಗಳಿಸುವಂತೆ ಮಾಡಿರುವುದು ಇವರ ಹೆಗ್ಗಳಿಕೆ. ಮೈಸೂರಿನ ವೇದಶಾಸ್ತ್ರ ಪೋಷಿಣೀಸಭೆಯ ಕಾರ್ಯಕಲಾಪಗಳಲ್ಲಿ ವಿದ್ವತ್ಸಮ್ಮಾನ–ಸಮಾರೋಹಗಳಲ್ಲಿ ಇವರೂ ಸಹ ಭಾಗವಹಿಸಿ ಸಹಾಯ–ಸಹಕಾರ ನೀಡಿದ್ದಾರೆ.

ಮೈಸೂರಿನ ಅನೇಕ ಸಂಘಸಂಸ್ಥೆಗಳಲ್ಲಿ, ರಾಜ್ಯದ ವಿವಿಧ ಮಠಮಾನ್ಯಗಳ ಕಾರ್ಯಕ್ರಮಗಳಲ್ಲಿ, ಶೈಕ್ಷಣಿಕಸಂಸ್ಥೆಗಳ ವಿವಿಧ ಕಾರ್ಯಕ್ರಮಗಳಲ್ಲಿ ಇವರು ಸಕ್ರಿಯವಾಗಿ ದುಡಿದಿದ್ದಾರೆ. ಅನೇಕ ಉಪನ್ಯಾಸಗಳನ್ನು, ಪ್ರವಚನಗಳನ್ನು ನೀಡಿದ್ದಾರೆ. ಮೇಲುಕೋಟೆಯ ಸಂಸ್ಕೃತ ಸಂಶೋಧನಾ ಸಂಸತ್, ರಾಜ್ಯದ ಇತರ ಸಂಸ್ಕೃತ ಕಾಲೇಜುಗಳಲ್ಲಿ ಸಂಸ್ಕೃತದ ವಿಚಾರಗೋಷ್ಠಿಗಳು, ಕಾರ್ಯಾಗಾರಗಳಲ್ಲಿ ಭಾಗವಹಿಸಿ ಕೆಲಸ ಮಾಡಿದ್ದಾರೆ. 

ಸಂಸ್ಕೃತ ಪಾಠ್ಯಪುಸ್ತಕ ರಚನಾಸಮಿತಿಯಲ್ಲಿ ಇದ್ದು ಪಾಠ್ಯಪುಸ್ತಕಗಳ ರಚನಾ ಕಾರ್ಯಗಳನ್ನು ಮಾಡಿದ್ದಾರೆ. ಗಣಪತಿ ಸಚ್ಚಿದಾನಂದಾಶ್ರಮ, ಸುತ್ತೂರು ಮಠ, ಉಡುಪಿ ಮಠ, ಸ್ವರ್ಣವಲ್ಲೀ ಮಠ ಮೊದಲಾದ ಕಡೆಗಳಲ್ಲಿ ಅನೇಕ ಸಾಂಸ್ಕೃತಿಕ, ಧಾರ್ಮಿಕ, ಶೈಕ್ಷಣಿಕ ಕಾರ್ಯಕ್ರಮಗಳಲ್ಲಿ ಭಾಗವಹಿಸಿ ಮಾರ್ಗದರ್ಶನ ನೀಡಿದ್ದಾರೆ. ಇವರು ಓದುತ್ತಿದ್ದ ಕಾಲದಿಂದಲೂ ವಿದ್ಯಾರ್ಥಿ ಪರಿಷತ್ತಿನ ಕಾರ್ಯಕರ್ತರಾಗಿ ಸಾಕಷ್ಟು ಕೆಲಸ ಮಾಡಿದ್ದಾರೆ. ಅನೇಕ ಬೈಠಕ್ಕುಗಳಲ್ಲಿ ವಿದ್ಯಾರ್ಥಿಗಳಿಗೆ ಉತ್ತಮ ವಿಷಯಗಳನ್ನು ಬೋಧಿಸಿದ್ದಾರೆ. 

ಬಹು ವರ್ಷಗಳಿಂದ ಭಾರತ ಸಂಸ್ಕೃತಿ ಯೋಜನೆಯ ನಿರ್ದೇಶಕರಾಗಿ ಪ್ರೌಢಶಾಲೆ ಮತ್ತು ಪಿಯುಸಿ ವಿದ್ಯಾರ್ಥಿಗಳಿಗೆ ರಾಮಾಯಣ–ಮಹಾಭಾರತ ಪರೀಕ್ಷೆಗಳಲ್ಲಿ ಪರೀಕ್ಷೆ ನಡಿಸುವುದರ ಮೂಲಕ ಸಾವಿರಾರು ವಿದ್ಯಾರ್ಥಿಗಳಿಗೆ ರಾಮಾಯಣ–ಮಹಾಭಾರತಗಳ ಸಾಕಷ್ಟು ವಿಷಯಗಳು ತಿಳಿಯುವಂತೆ ಮಾಡಿದ್ದಾರೆ. ಇವರು ಮಾತನಾಡುವ ಕಲೆ ಅನ್ಯಾದೃಶವಾದುದು. ಇವರ ನೇರ ನುಡಿ, ಪ್ರಾಸಬದ್ಧವಾದ ಮಾತು, ಹಿತಮಿತವಾದ ಭಾಷಣ, ವಾಸ್ತವಿಕ ವಿಷಯದ ಪ್ರಕಾಶನ, ಎಲ್ಲವೂ ಆಕರ್ಷಕವಾದವು ಅರ್ಥಗರ್ಭಿತವಾದವು ಹಾಗೂ ಉಪಯುಕ್ತವಾದವುಗಳು. ಹಾಗಾಗಿ ಎಲ್ಲರೂ ಇವರ ಭಾಷಣ, ಉಪನ್ಯಾಸಗಳನ್ನು ಆಸಕ್ತಿಯಿಂದ ಕೇಳುತ್ತಾರೆ. ಇವರ ಸಾಹಿತಕೃಷಿಯು ಬಹುಮುಖಿ ಹಾಗೂ ಸಮಾಜಮುಖಿಯೂ ಆಗಿದೆ.

ಇವರು ಸುಮಾರು 20–25 ವರ್ಷಗಳಿಂದ ವಿಶ್ವದಲ್ಲೇ ಏಕೈಕ ಸಂಸ್ಕೃತ ದಿನಪತ್ರಿಕೆಯಾದ ‘ಸುಧರ್ಮಾ’ ಸಂಸ್ಕೃತ ದಿನಪತ್ರಿಕೆಯ ಕಾರ್ಯಗಳಲ್ಲಿ ಸಕ್ರಿಯವಾಗಿ ದುಡಿದಿದ್ದಾರೆ. ಸುಧರ್ಮಾ ದಿನಪತ್ರಿಕೆಯ ಸಂಸ್ಥಾಪಕರಾದ ಪಂಡಿತ ಕೆ.ಎನ್.ವರದರಾಜ ಐಯ್ಯಂಗಾರ್ಯರ ನಿಧನಾನಂತರ ಅವರ ಮಗ ಶ್ರೀ ಕೆ.ವಿ.ಸಂಪತ್ಕುಮಾರ್ ಅವರು ಆ ಪತ್ರಿಕೆಯನ್ನು ಮುನ್ನಡೆಸಿಕೊಂಡು ಬರುತ್ತಿದ್ದಾರೆ. ಅವರಿಗೆ ಅವರ ಪತ್ನಿ ಜಯಲಕ್ಷ್ಮಿಯವರು ಸಹಾಯಕರಾಗಿ ದುಡಿಯುತ್ತಿದ್ದಾರೆ. ಮೊದಮೊದಲು ನಾನು ಮತ್ತು ವಿದ್ವಾನ್ ಎಚ್.ವಿ.ನಾಗರಾಜರಾವ್ ಅವರು, ನಾವಿಬ್ಬರು ಸುಧರ್ಮಾ ಮುಖಪುಟದ ಸುದ್ದಿಗಳನ್ನು ಬರೆಯುತ್ತಿದ್ದೆವು. 

ಹಿಂಬದಿಯ ಪುಟದಲ್ಲಿ ಆಗಾಗ್ಗೆ ಲೇಖನಗಳನ್ನು ಬರೆಯುತ್ತಿದ್ದೆವು. ನಂತರ ಗಂಗಾಧರಭಟ್ಟರೂ ಸಹ ಸುಧರ್ಮಾ ಸಂಸ್ಕೃತ ದಿನಪತ್ರಿಕೆಗಾಗಿ ಸುದ್ದಿಗಳನ್ನು, ಲೇಖನಗಳನ್ನು ಸಂಸ್ಕೃತದಲ್ಲಿ ಬರೆಯತೊಡಗಿದರು. ಪ್ರತಿವರ್ಷ ಜುಲೈ ತಿಂಗಳ 15ರಂದು ಸುಧರ್ಮಾ ಸಂಸ್ಕೃತ ದಿನಪತ್ರಿಕೆಯ ವಾರ್ಷಿಕೋತ್ಸವವು ನಡೆಯುತ್ತದೆ. ಅದರ ವಿಶೇಷ ಸಂಚಿಕೆಗಳಲ್ಲಿ ಲೇಖನಗಳನ್ನು ಬರೆಯುವುದು, ಸುದ್ದಿಗಳನ್ನು ಬರೆಯುವುದು, ಲೇಖನಗಳ ಕರಡುಪ್ರತಿ ತಿದ್ದಿಕೊಡುವುದು, ಸುಧರ್ಮಾ ಪತ್ರಿಕೆಯ ವಾರ್ಷಿಕೋತ್ಸವದಲ್ಲಿ ಪಾಠಶಾಲೆಯ ತಮ್ಮ ವಿದ್ಯಾರ್ಥಿಗಳನ್ನು ಕರೆದುಕೊಂಡು ಹಿಂದಿನ ದಿನವೇ ಬಂದು, ಮಾರನೆಯ ದಿನದ ಸಮಾರಂಭದ ಎಲ್ಲ ಸಿದ್ಧತೆಗಳನ್ನು ಮಾಡುವುದು, ಮಾಡಿಸುವುದು, ಸಮಾರಂಭದ ಎಲ್ಲ ನಿರ್ವಹಣೆಯನ್ನು ಮಾಡುವುದು ಮುಂತಾದ ಜವಾಬ್ದಾರಿಗಳನ್ನು ವಹಿಸಿಕೊಂಡು, ಸುಧರ್ಮಾ ಪತ್ರಿಕೆಗಾಗಿ ನಮ್ಮೊಡನೆ ಸೇವೆ ಮಾಡಿದ್ದಾರೆ. ಅದನ್ನು ಮರೆಯಲಾಗದು. 

ಎಲ್ಲ ರೀತಿಯ ಕೆಲಸ ಕಾರ್ಯಗಳನ್ನು ಮಾಡಿದರೂ ಯಾವ ರೀತಿಯ ಗೌರವ, ಪ್ರಶಂಸೆಗಳನ್ನು ಇವರು ಬಯಸಿದವರಲ್ಲ. ಸಂಸ್ಕೃತಭಾಷೆಯ ಪ್ರಸಾರ–ಪ್ರಚಾರವಾಗುತ್ತದಲ್ಲ ಎನ್ನುವ ಸದ್ಭಾವನೆಯಿಂದ, ಸುಧರ್ಮಾದ ಕಾರ್ಯಗಳನ್ನು ಮಾಡಿಕೊಂಡು ಬಂದಿದ್ದಾರೆ. ಈಗಲೂ ಸಾಧ್ಯವಾದಾಗ ಲೇಖನಗಳನ್ನು, ಸುದ್ದಿಗಳನ್ನು ಬರೆದುಕೊಡುವುದರ ಮೂಲಕ ಸುಧರ್ಮಾ ಪತ್ರಿಕೆಗೆ ಸಹಕಾರ ನೀಡುತ್ತಿದ್ದಾರೆ. ಈ ಎಲ್ಲ ಸಮಯಗಳಲ್ಲೂ ಗಂಗಾಧರಭಟ್ಟರು ನಾನು ಜೊತೆ ಜೊತೆಯಾಗಿ ಇದ್ದು ಕೆಲಸ ಮಾಡಿದುದು ನನಗೆ ಅಮಿತಾನಂದವನ್ನು ಉಂಟುಮಾಡಿದೆ.

ಇನ್ನು ಕರ್ನಾಟಕದಲ್ಲಿ ಸಂಸ್ಕೃತ ವಿಶ್ವವಿದ್ಯಾನಿಲಯವನ್ನು ಸ್ಥಾಪಿಸಬೇಕೆಂದು ಕರ್ನಾಟಕ ಸರ್ಕಾರವು ನಿಶ್ಚಯಿಸಿದಾಗ ಪರ–ವಿರೋಧವಾಗಿ ಅನೇಕ ಅಭಿಪ್ರಾಯಗಳು ಬಂದುವು. ಈ ಸಂಸ್ಕೃತ ವಿಶ್ವವಿದ್ಯಾನಿಲಯವು ಆರಂಭವಾಗುವುದು ಬೇಡ ಎಂದು ವಿರೋಧಿಸಿ ಚಳುವಳಿ ಮಾಡಿದವರೇ ಹೆಚ್ಚು ಜನರು. ಆಗ ಸಂಸ್ಕೃತ ವಿಶ್ವವಿದ್ಯಾನಿಲಯವನ್ನು ಸ್ಥಾಪಿಸಲೇಬೇಕು ಎಂದು ವಿದ್ವಾನ್ ವೆಂಕಟರಮಣ ಹೆಗ್ಗಡೆ, ನಾವೆಲ್ಲ ಪ್ರಯತ್ನಪೂರ್ವಕವಾಗಿ ಕ್ರಿಯಾಶೀಲರಾಗಿದ್ದಾಗ ಗಂಗಾಧರಭಟ್ಟರೂ ಸಹ ನಮ್ಮೊಡನೆ ದನಿಗೂಡಿಸಿ ಸಹಕರಿಸಿದರು.     

ಸಂಸ್ಕೃತ ವಿಶ್ವವಿದ್ಯಾನಿಲಯವು ಸ್ಥಾಪನೆಯಾಗಲೇಬೇಕೆಂದು ಮೈಸೂರಿನಲ್ಲಿ ಪ್ರಯತ್ನಪಟ್ಟು ಹೋರಾಡಿದ ವಿಷಯವನ್ನು ಬಹಳಷ್ಟು ಸಂಸ್ಕೃತದ ಜನರೇ ಈಗ ಮರೆತಿದ್ದಾರೆ. ಇದೊಂದು ವಿಪರ್ಯಾಸ. ವಿದ್ವಾನ್ ತುರುವೇಕೆರೆ ವಿಶ್ವೇಶ್ವರದೀಕ್ಷಿತರಲ್ಲಿ ಎಷ್ಟೋ ಸಂಸ್ಕೃತ ವಿಷಯಗಳನ್ನು ಕಲಿತಿದ್ದ ಇವರು ಅವರ ಅಂತ್ಯಕಾಲದಲ್ಲಿ ಅವರು ತುಂಬಾ ಕ್ಲೇಶ ಪಡುತ್ತಿದ್ದ ಸಮಯದಲ್ಲಿ ಅವರ ಮನೆಗೆ ಹೋಗಿ ಅವರ ರಕ್ಷಣೆಗಾಗಿ ನಿಂತು, ಅವರ ಸೇವೆಗೈದುದು ನಿಜಕ್ಕೂ ಇವರ ಹಿರಿತನ ಹಾಗೂ ಅಸದೃಶವಾದ ಗುರುಭಕ್ತಿ ಎಂದು ಹೇಳಲೇಬೇಕು.

ವಿದ್ವಾನ್ ಗಂಗಾಧರಭಟ್ಟರು ಸಹೃದಯರು, ಸಜ್ಜನರು, ಖಂಡಿತವಾದಿಗಳು, ಸತ್ಯನಿಷ್ಠರು. ಸಂಸ್ಕೃತ ವಿದ್ಯಾಭ್ಯಾಸದ ಎಲ್ಲ ಮಜಲುಗಳಲ್ಲೂ ವಿದ್ಯಾರ್ಥಿಗಳ ಪರವಾಗಿ ದುಡಿದಿದ್ದಾರೆ. ನಾನೂ ಅವರು ಎಷ್ಟೋ ಸಲ ಭೇಟಿಯಾದಾಗ, ಸಂಸ್ಕೃತ ಶಾಸ್ತ್ರಪರಂಪರೆ–ಸಂಸ್ಕೃತವಿದ್ಯಾಭ್ಯಾಸ, ಇಂದಿನ ಸಮಾಜದ ಸ್ಥಿತಿಗತಿಗಳು, ಸಂಸ್ಕೃತದ ಉಳಿವು–ಬೆಳವಣಿಗೆ ಹೇಗೆ ಮುಂತಾದ ಎಷ್ಟೋ ವಿಷಯಗಳನ್ನು ಚರ್ಚಿಸುತ್ತಿದ್ದೆವು. ಪರಿಹಾಸವಾಗಿ ಮಾತನಾಡಿಕೊಂಡಿದ್ದೇವೆ. ನನ್ನಲ್ಲಿ ಅವರ ಸ್ನೇಹ, ಪ್ರೀತಿ, ವಿಶ್ವಾಸ, ಸದಭಿಮಾನಗಳು ಅವ್ಯಾಜವಾದುದು. ನನಗೂ ಸಹ ಅವರ ವಿದ್ವತ್ತು, ಸದ್ಗುಣ, ಸಹೃದಯತೆ, ವಿಶ್ವಾಸಗಳ ಬಗ್ಗೆ ಅಷ್ಟೇ ಗೌರವವಿದೆ. ಪ್ರಾಚೀನಶಾಸ್ತ್ರಪರಂಪರೆಯ ಗುರುಗಳಲ್ಲಿ ಓದಿದ ನಮಗೆ ಪ್ರಾಚೀನಶಾಸ್ತ್ರಪರಂಪರೆಯು ಉಳಿಯಬೇಕೆನ್ನುವ ಅತಿಯಾದ ಹಂಬಲ. ಅದರ ಬಗ್ಗೆ ನಾನು ಅವರು ಎಷ್ಟೋ ಬಾರಿ ಮಾತನಾಡಿಕೊಂಡಿದ್ದುಂಟು.
\newpage
ಈಗ ಮೈಸೂರು ಸಂಸ್ಕೃತ ಪಾಠಶಾಲೆಯ ಅಲಂಕಾರಶಾಸ್ತ್ರ ಪ್ರಾಧ್ಯಾಪಕರಾಗಿರುವ ವಿದ್ವಾನ್ ಜಿ.ಮಂಜುನಾಥ ಅವರು ಬಹಳ ವರ್ಷಗಳ ಹಿಂದೆ ಇದೇ ಪಾಠಶಾಲೆಯಲ್ಲಿ ವಿದ್ಯಾರ್ಥಿಯಾಗಿ ಓದುತ್ತಿದ್ದಾಗ, ನನ್ನನ್ನು ಅವರ ಊರಾದ ಗೊದ್ದಲಬೀಳುವಿಗೆ ವಿಷ್ಣುಷಟ್ಪದಿಯ ಬಗ್ಗೆ ಉಪನ್ಯಾಸ ಮಾಡಿಸಲು ಕರೆದೊಯ್ದರು. ಅಲ್ಲಿನ ಶಾಲೆಯೊಂದರಲ್ಲಿ ಉಪನ್ಯಾಸ ಮಾಡಿದಾಗ ಎಲ್ಲರೂ ಮೆಚ್ಚಿದರು. ನಂತರ ಆ ದಿನ ಸಂಜೆ ರಾತ್ರಿ, ಮಾರನೆಯ ದಿನ ಆ ಊರಿನವರು ತಮ್ಮ ತಮ್ಮ ಮನೆಗಳಿಗೆ ನನ್ನನ್ನು ಕರೆದೊಯ್ದು ಉಪಚರಿಸಿ, ಆದರಾತಿಥ್ಯ ನೀಡಿ ನನ್ನಿಂದ ಮತ್ತೆ ಕೆಲವೆಡೆ ಉಪನ್ಯಾಸ ಮಾಡಿಸಿದರು. 

ಆಗ ಮಾರನೆಯ ದಿನ ವಿದ್ವಾನ್ ಜಿ.ಮಂಜುನಾಥ ಅವರು ನನ್ನನ್ನು ಅವರ ಊರಿನ ಬಳಿಯಿದ್ದ ಗಂಗಾಧರಭಟ್ಟರ ಮನೆಗೆ ಕರೆದೊಯ್ದರು. ಅಲ್ಲಿ ಗಂಗಾಧರಭಟ್ಟರ ತಂದೆಯವರಾದ ವಿದ್ವಾನ್ ವಿಘ್ನೇಶ್ವರಭಟ್ಟರನ್ನು ಪರಿಚಯ ಮಾಡಿಸಿದರು. ಆಗ ಅವರು ಹುಡುಗನಾದ ನನ್ನೊಡನೆ ಎಷ್ಟು ಪ್ರೀತಿ, ವಿಶ್ವಾಸ, ಆದರಗಳಿಂದ ಗೌರವದಿಂದ ಮಾತನಾಡಿಸಿದರು ಎನ್ನುವುದನ್ನು ನಾನು ಇನ್ನೂ ಮರೆತಿಲ್ಲ. ಅವರು ಶಂಕರಾಚಾರ್ಯರ ಅದ್ವೈತವೇದಾಂತದ ಬಗೆಗಿನ ಪುಸ್ತಕಗಳು, ಲೇಖನಗಳನ್ನು ನನಗೆ ತೋರಿಸಿ ಸಂತಸಪಟ್ಟಿದ್ದು ನನಗೆ ನೆನಪಿದೆ. ಹಾಗೆಯೇ ಶ್ರೀ ಜಿ.ಮಂಜುನಾಥರು ಗಂಗಾಧರಭಟ್ಟರ ಅಣ್ಣ ಶ್ರೀಧರಭಟ್ಟರು ಸಿದ್ದಾಪುರದಲ್ಲಿ ನಡೆಸುತ್ತಿದ್ದ ಆಂಗಿರಸ ಮುದ್ರಣಾಲಯಕ್ಕೆ ನನ್ನನ್ನು ಕರೆದೊಯ್ದು ಅವರನ್ನು ಭೇಟಿ ಮಾಡಿಸಿದ್ದರು. ಅವರೂ ಸಹ ಅಷ್ಟೇ ಸಹೃದಯತೆಯಿಂದ ನನ್ನನ್ನು ಗೌರವಿಸಿ ಮಾತನಾಡಿಸಿದುದು ನನಗಿನ್ನೂ ನೆನಪಿದೆ. ವಿದ್ವಾನ್ ಜಿ.ಮಂಜುನಾಥರು ನನಗೆ ಆತ್ಮೀಯ ಸ್ನೇಹಿತರು. ಅವರು ಆಗ ಉತ್ತರಕನ್ನಡದ ಎಷ್ಟೋ ಜಾಗಗಳಿಗೆ ಕರೆದೊಯ್ದು ಅಲ್ಲಿನ ವನಸಿರಿ, ಪ್ರಾಕೃತಿಕ ಸಂಪತ್ತುಗಳನ್ನು ತೋರಿಸುವುದರ ಜೊತೆಗೆ ತಮ್ಮ ಮನೆಗೂ ಕರೆದೊಯ್ದು ಆದರಾತಿಥ್ಯವನ್ನು ನೀಡಿ, ತಮ್ಮ ಊರಿನವರಿಗೆ ನನ್ನನ್ನು ಒಬ್ಬ ಉತ್ತಮ ಸಂಸ್ಕೃತ ವಿದ್ವಾಂಸನೆಂದು ಪರಿಚಯಿಸಿದುದನ್ನೂ ಸದಾ ನೆನಪಿಸಿಕೊಳ್ಳುತ್ತೇನೆ.
\begin{verse}
\textbf{“ಜ್ಞಾನಾಯ ವಿದ್ಯಾ ಧನಂ ದಾನಾಯ ಶಕ್ತಿಃ ಪರೇಷಾಂ ಪರಿರಕ್ಷಣಾಯ~।”}
\end{verse}
“ವಿದ್ಯೆಯು ಜ್ಞಾನಕ್ಕೆ, ಹಣವು ದಾನಕ್ಕೆ, ಶಕ್ತಿಯು ಇತರರ ರಕ್ಷಣೆಗೆ” ಎನ್ನುವ ಮಾತು ಗಂಗಾಧರಭಟ್ಟರಿಗೆ ಹೊಂದುವಂತಹುದು.

ಒಟ್ಟಾರೆ ವಿದ್ವಾನ್ ಗಂಗಾಧರಭಟ್ಟರ ಸ್ನೇಹ, ಸಹವಾಸ ಒಡನಾಟಗಳು ಮೈಸೂರಿನ ಮಹಾರಾಜ ಸಂಸ್ಕೃತ ಮಹಾಪಾಠಶಾಲೆಯಲ್ಲಿ ನನಗೆ ದೊರೆಯಿತು. ಸುಧರ್ಮಾ ದಿನಪತ್ರಿಕೆಯ ಸನಿಹದಿಂದ ಅವರ ಸ್ನೇಹ, ಆದರ, ಪ್ರೀತಿಗಳು ಮತ್ತಷ್ಟು ಹೆಚ್ಚಾಯಿತು. ಗಂಗಾಧರಭಟ್ಟರು ಸರಸಿಗಳು, ಪರಿಹಾಸಪ್ರಿಯರೂ ಹೌದು, ಎಷ್ಟೋ ವಿದ್ಯಾರ್ಥಿಗಳಿಗೆ ಮನೆಯಲ್ಲಿ ಸಂಸ್ಕೃತಪಾಠ ಮಾಡಿದ್ದಾರೆ. ಈಗಲೂ ಮಾಡುತ್ತಿದ್ದಾರೆ. ತಮ್ಮ ಬಳಿ ಕಲಿಯಬೇಕೆಂದು ಬರುವ ವಿದ್ಯಾರ್ಥಿಗಳಿಗೆ ಅವರು ಎಂದೂ ಇಲ್ಲವೆನ್ನದೆ ಸಂಸ್ಕೃತ ವಿದ್ಯೆಯನ್ನು ಕಲಿಸಿದ್ದಾರೆ, ಕಲಿಸುತ್ತಿದ್ದಾರೆ. ಎಷ್ಟೋ ಜನ ವಿದೇಶಿ ವಿದ್ಯಾರ್ಥಿಗಳಿಗೂ ಸಂಸ್ಕೃತವನ್ನು ಪಾಠ ಮಾಡಿ, ಅವರನ್ನು ತರಬೇತಿಗೊಳಿಸಿದ್ದಾರೆ. 

ನಿಜಕ್ಕೂ ವಿದ್ವಾನ್ ಗಂಗಾಧರಭಟ್ಟರು ಒಬ್ಬ ಅಪರೂಪದ ವ್ಯಕ್ತಿ. ಅವರು ಕಾಲಧರ್ಮಕ್ಕನುಗುಣವಾಗಿ ಸೇವೆಯಿಂದ ನಿವೃತ್ತರಾಗುತ್ತಿದ್ದಾರೆ. ಆದರೆ, ಅವರೆಂದೂ ಯಾವ ಕಾರ್ಯದಿಂದಲೂ ನಿವೃತ್ತರಾಗುವವರಲ್ಲ. ಸಂಸ್ಕೃತ ವಿದ್ಯಾಬೋಧನೆಯಲ್ಲಿ ಸದಾ ಪ್ರವೃತ್ತರಾಗಿರುವವರೇ. ಅವರ ನಿವೃತ್ತಿಯ ನಂತರದ ಜೀವನವು ಸುಖಮಯವಾಗಿರಲಿ, ಆನಂದವಾಗಿರಲಿ. ಅವರಿಗೆ ಭಗವಂತನು ಆಯುರಾರೋಗ್ಯಸಂಪದಾದಿಸಕಲಭಾಗ್ಯಗಳನ್ನು ನೀಡಿ ಕಾಪಾಡಲೆಂದು ತುಂಬು ಹೃದಯದಿಂದ ಹಾರೈಸುತ್ತೇನೆ.

\articleend
