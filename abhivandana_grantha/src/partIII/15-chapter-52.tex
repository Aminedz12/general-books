{\fontsize{14}{16}\selectfont
\chapter{ವಿ~। ಗಂಗಾಧರಭಟ್ಟರೊಡನಾಟದ ರಸನಿಮಿಷಗಳು} 

\begin{center}
\Authorline{ಜಿ.ಎನ್.ಅನಂತವರ್ಧನ}
\smallskip
ಚಾರ್ಟರ್ಡ್ ಅಕೌಂಟೆಂಟ್\\ 
ನಂ 65,ಗುರಿಕಾರ ದೇವಣ್ಣರಸ್ತೆ\\
ಮೈಸೂರು 570 009 
\addrule	
\end{center}

ನನಗಿಂತ ಒಂದು ವರ್ಷ ಹಿರಿಯರಾದ ವಿದ್ವಾನ್ ಗಂಗಾಧರ ಭಟ್ ಬಿ.ಕಾಂ ಪದವೀಧರರು (1979). ನಾನೂ ಬಿ.ಕಾಂ ಪದವೀಧರ(1980). ಅವರು ಸಂಸ್ಕೃತ ಸಾರಸ್ವತ ಲೋಕದಲ್ಲಿ ಈಜಾಡಿದರೆ ನಾನು ಸಿ.ಎ ಮಾಡಿ ವಾಣಿಜ್ಯ ಕ್ಷೇತ್ರದಲ್ಲೇ ಮುಂದುವರೆದೆ. ಅವರನ್ನು ಮೊದಲ ಸಲ ಭೇಟಿ ಆದದ್ದು ಮೈಸೂರಿನ ಶಿವರಾಂಪೇಟೆಯ ಮಯೂರ ಹೋಟೇಲಲ್ಲಿ ನಡೆದ ಹವೀಕ ಸಂಘದ ಬಹಿರಂಗ ಸಭೆಯಲ್ಲಿ. ಕಾರಣಾಂತರಗಳಿಂದ ಕೆಲವಷ್ಟು ವರ್ಷ ಸ್ಥಗಿತಗೊಂಡಿದ್ದ ಸಂಘವನ್ನು ಹಿರಿಯರೆಲ್ಲಾ ಸೇರಿ ಮುಂದುವರೆಸಲು ಕರೆದಿದ್ದ ಸಭೆ ಅದು. ಈ ಸಭೆಯನ್ನು ಯಶಸ್ವಿಯಾಗಿ ನಿರ್ವಹಿಸಿ ಸಂಘದ ಕಾರ್ಯದರ್ಶಿಯಾಗಿ ಹಳಿಗೆ ತಂದ ಕೀರ್ತಿ ತರುಣ ಗಂಗಣ್ಣ ಅವರಿಗೆ ಸಲ್ಲುತ್ತದೆ. ಹವೀಕ ಸಂಘದ ಚಟುವಟಿಕೆಗಳಲ್ಲಿ ಎರಡು ವರ್ಷ ಅವರೊಡನೆ ಸಕ್ರಿಯವಾಗಿ ಕೆಲಸ ಮಾಡಿದ್ದು ಭವಿಷ್ಯದ ನನ್ನ ಚಟುವಟಿಕೆಗಳಿಗೆ ಬಹಳ ಸಹಕಾರಿಯಾಗಿದೆ.

ಇದೇ ಸಂದರ್ಭದಲ್ಲಿ ಮೈಸೂರಿನಿಂದ ಭೋಗಾದಿ ರಸ್ತೆಯಲ್ಲಿ 9ನೇ ಕಿ.ಮೀಗೆ ಬರುವ ಕುಗ್ರಾಮ ಕೆ.ಹೆಮ್ಮನ\-ಹಳ್ಳಿಯಲ್ಲಿನ ಹೊಯ್ಸಳರ ಕಾಲದ ಶ್ರೀ ಮಹಾಲಿಂಗೇಶ್ವರ ದೇವಾಲಯವು ಶ್ರೀ ಧರ್ಮಸ್ಥಳ ಮಂಜುನಾಥೇಶ್ವರ ಧರ್ಮೋತ್ಥಾನ ನ್ಯಾಸದ ವತಿಯಿಂದ ಜೀರ್ಣೋದ್ಧಾರಗೊಂಡು (1994)    ಪ್ರತಿಷ್ಠೆ  \enginline{-}  ಕುಂಭಾಭಿಷೇಕಕ್ಕೆ ಸಜ್ಜುಗೊಂಡ ಸಂದರ್ಭ. ನಾನು ಈ ದೇವಾಲಯದ ಕಾರ್ಯದರ್ಶಿಯಾಗಿದ್ದೆ. ಧಾರ್ಮಿಕ ವಿಧಿವಿಧಾನಗಳನ್ನು  ನಿಶ್ಚಯಿಸಬೇಕಾಗಿದೆ. ಇದರ ಬಗ್ಗೆ ನಾನು ಪೂರ್ಣ ಅಜ್ಞಾನಿ.ಹೇಗೆ ಮುಂದುವರೆಯುವುದೆಂದು ತಿಳಿಯದೆ ಸಯ್ಯಾಜಿ ರಾವ್ ರಸ್ತೆಯಲ್ಲಿದ್ದ ಗಂಗಣ್ಣನವರನ್ನು ಅವರ ಮನೆಯಲ್ಲಿ ಭೇಟಿಯಾಗಿ ನಿರೂಪ ಕೊಟ್ಟೆ.  ‘ಧಾರ್ಮಿಕ ವಿಧಿವಿಧಾನಗಳ ಸಂಪೂರ್ಣ  ಜವಾಬ್ದಾರಿ ನನ್ನದು, ನೀವು ಬೇರೆ ಕೆಲಸ ನೋಡಿಕೊಳ್ಳಿ’  ಎಂದು ಎರಡು ಮಾತಲ್ಲಿ ನನ್ನನ್ನು ಕಳಿಸಿಕೊಟ್ಟರು. ಬೇರೆ ಮಾತೇಯಿಲ್ಲ. ಅವರು ಆಗ ಅವರ ಮನೆಯ ಸಮೀಪದ ಶಂಕರ ಮಠದ ರಸ್ತೆಯಲ್ಲಿನ ಶ್ರೀ ಶಂಕರ ವಿಲಾಸ ಸಂಸ್ಕೃತ ಪಾಠಶಾಲೆಯಲ್ಲಿ ಉಪಾಧ್ಯಾಯರಾಗಿದ್ದರು. ಅಲ್ಲದೆ ಮನೆಯಲ್ಲಿ ಸಂಸ್ಕೃತ ವ್ಯಾಕರಣ ಮತ್ತು ತರ್ಕ ಉಚಿತವಾಗಿ ಪಾಠ ಮಾಡುತ್ತಿದ್ದರು. ದೇವಾಲಯದ ಪ್ರತಿಷ್ಠೆಗೆ ತಮ್ಮ 20 ಜನ ಶಿಷ್ಯರೊಡಗೂಡಿ ಶೈವಾಗಮದ  ನಿವೃತ್ತ ಪ್ರಾಧ್ಯಾಪಕರಾದ ಶ್ರೀ ಸೀತಾರಾಮ ಸೋಮಯಾಜಿಗಳನ್ನು ಅಧ್ಯಕ್ಷರನ್ನಾಗಿ ಒಪ್ಪಿಸಿ ಕಾರ್ಯೋನ್ಮುಖರಾದರು. 

17  \enginline{-}  9  \enginline{-}  1994 ಸೋಮವಾರ ಆರಂಭಗೊಂಡ ಬೆಳಗ್ಗಿನ ಪೂಜಾ ಕಾರ್ಯಗಳು, ಸಂಜೆಯ ಸಭಾ ಕಾರ್ಯಕ್ರಮ ಗಂಗಣ್ಣನವರು ತಮ್ಮ ಶಿಷ್ಯರೊಂದಿಗೆ ನಿರ್ವಹಿಸಿದ ಪರಿ ವಿವರಿಸಲು ಅಸದಳ. ಧಾರ್ಮಿಕ ಕಾರ್ಯಗಳಲ್ಲಿ ನಿಷ್ಠೆ, ಅತ್ಯಂತ ಹಿಂದುಳಿದ ಗ್ರಾಮದಲ್ಲಿನ ಹಿಂದುಳಿದ ಜನರೊಡನೆ ಒಡನಾಟ, ಸಭಾ ಕಾರ್ಯಕ್ರಮಕ್ಕೆಆಗಮಿಸಿದಮಠಾಧಿಪತಿಗಳು, 
                                                                   
ರಾಜಕೀಯ  ಮುಖಂಡರು,  ಸರ್ಕಾರಿ ಅಧಿಕಾರಿಗಳು ಎಲ್ಲರನ್ನೂ ಸಂಭಾಳಿಸಿದ ಅವರ ಚಾಕಚಕ್ಯತೆ ಅಸಾಧಾರಣ. ಮಾತಿನಲ್ಲಿ ಸ್ಪಷ್ಟತೆ ಕೃತಿಯಲ್ಲಿ ಪಾರದರ್ಶಕತೆ, ಒಪ್ಪಿಕೊಂಡ ಜವಾಬ್ದಾರಿಯನ್ನು ತ್ರಿಕರಣಪೂರ್ಣವಾಗಿ ಶ್ರಮಿಸಿ ಮುಗಿಸಿಕೊಡುವ ಅವರ ಸ್ವಭಾವ ನನಗೆ ಬಹಳ ಹಿಡಿಸಿತು. 19  \enginline{-}  9  \enginline{-}  1994 ಬುಧವಾರ ನಮ್ಮ ದೇವಾಲಯದ ಚಾರಿತ್ರಿಕ ದಿನ. ಶ್ರೀ ಧರ್ಮಸ್ಥಳದ ಧರ್ಮಾಧಿಕಾರಿ ಡಾ~॥ ಡಿ, ವೀರೇಂದ್ರ ಹೆಗ್ಗಡೆ, ಸುತ್ತೂರು ಮಹಾ ಸಂಸ್ಥಾನದ ಪೀಠಾಧ್ಯಕ್ಷರು ಶ್ರೀ ಶಿವರಾತ್ರಿ ದೇಶೀಕೇಂದ್ರ ಮಹಾಸ್ವಾಮಿಗಳು, ಅವಧೂತ ದತ್ತಪೀಠದ ಶ್ರೀ ಗಣಪತಿ ಸಚ್ಚಿದಾನಂದ ಮಹಾಸ್ವಾಮಿಗಳು ಇನ್ನೂ ಅನೇಕ ಗಣ್ಯರೂ ಆಢ್ಯರೂ ಸೇರಿ ದೇವಾಲಯವನ್ನು ಲೋಕಾರ್ಪಣೆ ಮಾಡಿದ ದಿನ. ಬೆಳಗ್ಗಿನ ಜಾವ 1 ಗಂಟೆಗೆ ಅಷ್ಠಬಂಧ (ವಜ್ರಗಾರೆ) ಹಾಕಿ ದೇವರ ವಿಗ್ರಹಗಳನ್ನು ಪ್ರತಿಷ್ಥೆಮಾಡಬೇಕು. ಕಷ್ಟದಲ್ಲಿ ನಾಲ್ಕು ಮಂದಿ ಹೋಗಬಹುದಾದಷ್ಟು ಸಣ್ಣ ಗರ್ಭಗುಡಿ, ಹತ್ತು ಜನರಿಗೂ ಎತ್ತಲಾಗದ ಕಲ್ಲಿನ ವಿಗ್ರಹಗಳು, ಎರಡು ದಿನದಿಂದ ನಿದ್ದೆಕೆಟ್ಟು ಎಲ್ಲರಿಗೂ ಸುಸ್ತೋ ಸುಸ್ತು. ವಜ್ರಗಾರೆ ತಯಾರಾಗಿದೆ. ಸೋಮಯಾಜಿಗಳು ಧಾರ್ಮಿಕ ವಿಧಿವಿಧಾನಗಳನ್ನು ಮುಗಿಸಿ ಗಂಗಣ್ಣನವರಿಗೆ ವಿಗ್ರಹಗಳನ್ನು ಕೂರಿಸಲು ಸೂಚನೆ ಕೊಟ್ಟಿದ್ದಾರೆ. ವೈದಿಕರಿಗಾರಿಗೂ ಲಿಂಗವನ್ನು ಎತ್ತಲು ಸಾಧ್ಯವಾಗುತ್ತಿಲ್ಲ. ನಮ್ಮ ದೇವಾಲಯದ ಉಸ್ತುವಾರಿ ನೋಡಿಕೊಳ್ಳುತ್ತಿದ್ದ ಸಾಹುಕಾರ್‍ಹುಂಡಿ ನಂಜುಂಡೇಗೌಡ್ರು ಮತ್ತವರ ಹೆಂಡ್ತಿ ಚಿಕ್ಕಮ್ಮ ಹಾಗು ಮಕ್ಕಳು ಸೋಮಣ್ಣ ಮತ್ತು ಶಿವಣ್ಣ ಗಂಗಣ್ಣನವರೊಟ್ಟಿಗೇ ಇದ್ದು ಎಲ್ಲವನ್ನೂ ಗಮನಿಸುತ್ತಿದ್ದಾರೆ. ಗಂಗಣ್ಣನವರು ಸೋಮಣ್ಣನಿಗೆ ಆಜ್ಞೆ ನೀಡಿದರು. ಶುದ್ಧವಾಗಿ ಪಂಚೆ ಉಟ್ಟು. ಕಟ್ಟು ಮಸ್ತಾದ ಶರೀರದ ಸೋಮಣ್ಣ ಬರೀ ಮೈಯಲ್ಲಿ ಲಿಂಗವನ್ನು ಸಲೀಸಾಗಿ ಎತ್ತಿ ಗರ್ಭಗುಡಿ ಪ್ರವೇಶಿಸಿ ವಜ್ರಗಾರೆ ಹಾಕಿ ಪ್ರತಿಷ್ಠಾಪಿಸೇ ಬಿಟ್ಟರು. ಸೋಮಯಾಜಿಗಳು ಸಾಧು ಸಾಧು, ಶಿವ ಶುದ್ಧ ಭಕ್ತಿಗೆ ಒಲಿದಿದ್ದಾನೆಂದು ಉದ್ಗರಿಸಿಯೇ ಬಿಟ್ಟರು.

ಕುಂಭಾಭಿಷೇಕವಾಗಿ ಸಭಾಕಾರ್ಯಕ್ರಮ ಬೆಳಿಗ್ಗೆ 11ಗಂಟೆಗೆ ಆರಂಭಗೊಂಡು ಸ್ವಾಮೀಜಿಗಳ ಆಶೀರ್ವಚನವೂ ಮುಗಿದು ಗಂಗಣ್ಣನವರು ವಂದನಾರ್ಪಣೆ ಮಾಡುತ್ತಿದ್ದರು. ಆಗ ಉಪಮುಖ್ಯಮಂತ್ರಿ ಆಗಿದ್ದ ಶ್ರೀ ಸಿದ್ಧರಾಮಯ್ಯನವರು, ನಮ್ಮ ಚಾಮುಂಡೇಶ್ವರಿ ಕ್ಷೇತ್ರದಿಂದಲೇ ಗೆದ್ದಿದ್ದರಿಂದ, ನೂರಾರು ಅಭಿಮಾನಿಗಳೊಡಗೂಡಿ ವೇದಿಕೆ ಏರಿಯೇ ಬಿಟ್ಟರು. ಎಲ್ಲರೂ ಕಾರ್ಯಕ್ರಮ ಮುಗಿಸಿ ಹೊರಡುವ ತರಾತುರಿಯಲ್ಲಿದ್ದಾರೆ. ಗಂಗಣ್ಣ ಉಪಮುಖ್ಯಮಂತ್ರಿಯನ್ನು ಸ್ವಾಗತಿಸುತ್ತಾ ಶಾಲುಹೊದಿಸಿ ಹಾರ ಹಾಕಿ ಸನ್ಮಾನವನ್ನೂ ಮಾಡಿಸಿದರು. ಮಂತ್ರಿಗಳು ಆಮಂತ್ರಣ ಹಿಡಿದು ದೀರ್ಘವಾದ ಭಾಷಣಕ್ಕೆ ತಯಾರಾಗುತ್ತಿದ್ದ ಹಾಗೆ ಸಮಯ ಸ್ಪೂರ್ತಿಯಿಂದ ನಮ್ಮ ಗಂಗಣ್ಣ ಹೇಳಿದರು  \enginline{-}   “ಸ್ವಾಮಿಗಳ ಆಶೀರ್ವಚನ ಆಗಿರುವುದರಿಂದ ಮಾನ್ಯ ಮಂತ್ರಿಗಳು ತಮ್ಮ ಕೈಸನ್ನೆಯಿಂದಲೇ ನಾನು ಮಾತನಾಡುವುದಿಲ್ಲ ಎಂದು ಹೇಳುತ್ತಿದ್ದಾರೆ. ಅವರನ್ನು ಅವರ ಕ್ಷೇತ್ರದಲ್ಲಿನ ಅನ್ನದಾನ ಕಾರ್ಯಕ್ರಮವನ್ನು ಉದ್ಘಾಟಿಸಿ ಕೊಡಬೇಕೆಂದು ವಿನಂತಿಸಿಕೊಳ್ಳುತ್ತೇನೆ ’’ ಎಂದು ಸಭಾಕಾರ್ಯಕ್ರಮವನ್ನು ಮುಗಿಸಿಯೇಬಿಟ್ಟರು. ಗಣ್ಯರಿಗ್ಯಾರಿಗೂ ನೋವಾಗದಂತೆ, ಸಭಿಕರಿಗೆ ಭಾಷಣದ ಹಿಂಸೆ ಆಗದಂತೆ, ಸಚಿವರಿಗೂ ಅವರ ಆಭಿಮಾನಿ ದೇವತೆಗಳಿಗೆ ಅವಮಾನವಾಗದಂತೆ ಸುಖಾಂತ್ಯ ಮಾಡಿದ್ದಕ್ಕೆ ಪ್ರಶಂಸಿದವರು ಅನೇಕರು.

ಮುಂದೆ ದೇವಾಲಯದ ಪ್ರತಿ ವಾರ್ಷಿಕ ಮಾಹಾಶಿವರಾತ್ರಿ ಜಾತ್ರೆಯಂದು ಸಹಸ್ರಾರು ಪ್ರೇಕ್ಷಕರ ಮುಂದೆ ರಾತ್ರಿ ಇಡೀ ನಡೆಯುವ ಸುಮಾರು 60 ಕಲಾವಿದರನ್ನೊಳಗೊಂಡ ಪೌರಾಣಿಕ ನಾಟಕ ಪ್ರದರ್ಶನದ ಸಭೆ;  ಸಂತೆ, ಗರಡಿ, ಗೋಶಾಲೆ, ಕಲ್ಯಾಣ ಮಂಟಪ, ಕುಸ್ತಿ ಸ್ಪರ್ಧೆಗಳು, ಕೃಷಿ ಮತ್ತು ನಾಟಿ ಔಷಧಿಗಳಿಗೆ ಸಂಬಂಧಿಸಿದ ಸಭೆಗಳು ಇತ್ಯಾದಿಯನ್ನು ಉದ್ಘಾಟಿಸಿ ಸುಮಾರು ಹತ್ತು ವರ್ಷ ನಡೆಸಿಕೊಟ್ಟ ಕೀರ್ತಿ ಗಂಗಣ್ಣನವರಿಗೆ ಸಲ್ಲುತ್ತದೆ. ಈ ಸಭೆಗಳನ್ನು ನಡೆಸುವ ಕಲೆಯನ್ನು ಸ್ಥಳೀಯರಿಗೂ ಕಲಿಸಿಕೊಟ್ಟಿದ್ದಲ್ಲದೇ ಸಮಸ್ತ ಮುಕ್ಕೋಟಿ ಜನರನ್ನು ನಮ್ಮವರೇ ಎಂದು ಅಂತಃಕರಣ ಪೂರ್ವಕವಾಗಿ ತಿಳಿದು ವರ್ತಿಸಿದ್ದರಿಂದಲೇ ಇಂದಿಗೂ ಹಳ್ಳಿಗರ ಬಾಯಿಯಲ್ಲಿ `ನಮ್ಮ ಗಂಗಣ್ಣ ’ಬುದ್ಧಿ’ಯವರಾಗಿಯೇ’ ಉಳಿದಿದ್ದಾರೆ.

ನಮ್ಮ ಕೆ.ಹೆಮ್ಮನಹಳ್ಳಿಯ ಹುಟ್ಟುಕುರುಡ ಜವರಶೆಟ್ಟಿ ಹರಿಕಥೆ ಯಾರ್ಯಾರಿಂದಲೋ ಕೇಳಿ ಕಲಿತವರು. ಎಲ್ಲರೂ ಕೂಲಿ  \enginline{-}  ನಾಲಿ ಮಾಡಿ ಬದುಕು ನಡೆಸುವುದರಿಂದ ಪುಸ್ತಕ ಓದಿ ಅರ್ಥಮಾಡುವುದು ಇಂದಿಗೂ ಶೂನ್ಯ. ಇವರು ಕುಂಡಪ್ಪನೆಂದೇ ಪ್ರಖ್ಯಾತಿ. ಕುರುಡ ಶಬ್ದದ ಅಪಭ್ರಂಶವೇ ಕುಂಡ. ನಮ್ಮ ಗಂಗಣ್ಣನವರೇ ಮೊದಲಿಗೆ ಹರಿಕಥೆದಾಸರ ಮೂಲ ಹೆಸರು ಪತ್ತೆಮಾಡಿ ಅದೇ ಹೆಸರಿನಿಂದಲೇ ಸಭೆಗೆ ಪರಿಚಯ ಮಾಡಿಸಿದ್ದಲ್ಲದೇ ಇಂಥ ಅಡ್ಡಹೆಸರುಗಳು/ದೋಷಗಳು ಪ್ರಯೋಗ ಆಗಬಾರದೆಂದು ಎಚ್ಚರಿಕೆಯನ್ನೂ ನೀಡಿದ್ದರು.

ದೇವಾಲಯದ ಮುಂಭಾಗದಲ್ಲಿ ಕ್ರಿ.ಶ. 1188 ಇಸವಿಯಲ್ಲಿ ಹೊಯ್ಸಳರಾಜ ಬಲ್ಲಾಳ 2 ಹಾಕಿಸಿದ ಶಿಲಾಶಾಸನವಿದೆ. ಇದರಲ್ಲಿ ಈ ದೇವಾಲಯವು ಸುತ್ತಲಿನ ಹದಿನಾರು ಗ್ರಾಮಗಳಿಗೆ ನಂದಾದೀಪವಾಗಿರಲೆಂದು ಬರೆದಿದೆ. ಇದರ ಬಗ್ಗೆ ಸುತ್ತಲಿನ ಹಳ್ಳಿಗಳ ಈಗಿನ ಹೆಸರುಗಳ ಆಧಾರದ ಮೇಲೆ ನಮ್ಮ ದೇವಾಲಯಕ್ಕೆ ಸಂಬಂಧ ಕಲ್ಪಿಸಿ ಚರಿತ್ರೆಯಲ್ಲಿ ದಾಖಲಿಸಿದ ಕೀರ್ತಿಯೂ ಗಂಗಣ್ಣನವರಿಗೆ ಸಲ್ಲುತ್ತದೆ.

ಹವೀಕ ಸಂಘದ ಕಾರ್ಯದರ್ಶಿಯಾಗಿ ಎರಡು ವರ್ಷ ಸೇವೆ ಸಲ್ಲಿಸಿ ಅದರ ಜವಾಬ್ದಾರಿಯನ್ನು ನನಗೆ ವಹಿಸಿದರು. ಇಂದಿಗೂ ಈ ಸಂಘಕ್ಕೆ ನನ್ನ ಅಳಿಲು ಸೇವೆ ಮುಂದುವರೆಸಿದ್ದೇನೆ. ಗಂಗಾಧರ ಭಟ್ಟರ ಕಾಲದಲ್ಲಿಯೇ ಮೊದಲ ಬಾರಿಗೆ ವಾರ್ಷಿಕವಾಗಿ ನಡೆಯುವ ಪ್ರತಿಭಾ ಸಂವರ್ಧಿನಿ ಸ್ಪರ್ಧೆಗಳು ಆರಂಭಗೊಂಡದ್ದು. ಗಂಗಾಧರ ಭಟ್, ಉಮಾಕಾಂತ ಭಟ್, ಆದಿಚುಂಚನಗಿರಿಯ ವಿಶ್ವೇಶ್ವರ ಭಟ್, ಗ.ನಾ. ಭಟ್ ಅವರ ಚಿಕ್ಕ  \enginline{-}  ಚೊಕ್ಕ ಯಕ್ಷಗಾನ ತಾಳಮದ್ದಲೆ ತಂಡ ಅನೇಕ ಊರುಗಳಲ್ಲಿ ತನ್ನ ಪ್ರದರ್ಶನ ನೀಡಿದ್ದು, ಅದರ ಸವಿ ಇನ್ನೂ ನನ್ನ ಸ್ಮೃತಿಪಟಲದಲ್ಲಿ ದಾಖಲಾಗಿಯೇ ಇದೆ. ಗಂಗಾಧರ ಭಟ್ಟರ ಅರ್ಥವೈಭವ, ತುಂಟ ನುಡಿಗಳು, ಪದಗಳ ಬಳಕೆಯ ಸೊಗಸು ಕೇಳಿಯೇ ಅನುಭವಿಸಬೇಕು.

ನನ್ನ ತಂದೆ ದಿ. ಜಿ.ಟಿ.ನಾರಾಯಣ ರಾವ್ ಅವರು ವಿಜ್ಞಾನ ಲೇಖಕರಾಗಿ ಇದಕ್ಕೆ ಸಂಬಂಧಿಸಿದ ಅನೇಕ ಪುಸ್ತಕಗಳನ್ನು ಬರೆದಿದ್ದಾರೆ. ಸಂಸ್ಕೃತ ಪದಗಳ ಬಗ್ಗೆ, ಅದರ ಪ್ರಯೋಗ ಮತ್ತು ವ್ಯಾಕರಣದ ಬಗ್ಗೆ ಏನೇ ಸಂಶಯ ಬಂದರೂ ತಂದೆಯವರು ಮೊದಲಿಗೆ ಗಂಗಾಧರ ಭಟ್ಟರೊಡನೆ ಚರ್ಚಿಸುತ್ತಿದ್ದರು. ಮೊಗೆದಷ್ಟೂ ಅವರ ಒಡನಾಟದ ನೆನಪು ಬರುವುದು. ಆದಿ ಜಗದ್ಗುರು ಶ್ರೀ ಶಂಕರಾಚಾರ್ಯರ ಪ್ರತಿಪಾದಿಸಿದ ಪಂಚಾಯತನ ಪೂಜಾ ಪದ್ದತಿಯ ಬಗ್ಗೆ ಚರ್ಚಿಸಲು ಇತ್ತೀಚೆಗೆ ಅವರಲ್ಲಿಗೆ ಹೋಗಿದ್ದೆ. ಈ ಪದ್ಧತಿಗೂ ಯೋಗಕ್ಕೂ, ದೇಶ ಸಂಘಟನೆಗೂ, ದೇಹ ಸಂರಕ್ಷಣೆಗೂ ಇರುವ ಅಂತಸ್ಸಂಬಂಧದ ಬಗ್ಗೆ ಅನೇಕ ಮಹತ್ವದ ವಿಚಾರಗಳನ್ನು ತೆರೆದು ತೋರಿಸಿದರು. ಅವರ ಕಾಲು ನೋವಿನ ನಡುವೆಯೂ ಕೆಲವಷ್ಟು ಆಸನಗಳನ್ನು ಮಾಡಿ ತೋರಿಸಿದರು. ಯೋಗ ನಮಸ್ಕಾರ ಪಂಚಕದ ಪುಸ್ತಕ ಪ್ರಕಟನೆಗೆ ಅವರೇ ಕಾರಣರೆಂದರೂ ಅತಿಶಯೋಕ್ತಿ ಆಗಲಾರದು. 

ಸಮಾಜಕ್ಕೆ ಸದಾ ಒಳಿತನ್ನು, ಸದ್ವಿಚಾರವನ್ನು ಕೊಡಬೇಕೆಂಬುದು ಗಂಗಾಧರ ಭಟ್ಟರ ನಿಲುವು. ಅವರ ಮನೆಯಲ್ಲಿ ಸಮಾಜದ ಅದೆಷ್ಟೋ ಹಿಂದುಳಿದ ಜನಾಂಗದವರನ್ನು ಭೇಟಿಯಾಗಿದ್ದೇನೆ. ಅವರಿಗೆಲ್ಲಾ ಪಾಟ ಹೇಳಿಕೊಟ್ಟುದಲ್ಲದೇ  ಆರ್ಥಿಕವಾಗಿಯೂ ಸಹಾಯ ಮಾಡಿದ್ದಾರೆ. ವಿಶ್ವ ಹಿಂದೂ ಪರಿಷತ್ತು ನಡೆಸುವ ರಾಮಾಯಣ  \enginline{-}  ಭಾರತ ವಾರ್ಷಿಕ ಪರೀಕ್ಷೆಯ ಮೌಲ್ಯಮಾಪನಕ್ಕೆ ನನ್ನ ಹೆಂಡತಿಯನ್ನು ಕರೆಯಲು ಎಂದೂ ಮರೆಯುತ್ತಿರಲಿಲ್ಲ.   ಬಹುಮುಖ ಪ್ರತಿಭೆಯ ವಿದ್ವಾನ್ ಗಂಗಾಧರ ಭಟ್ ಅವರು ಸರ್ಕಾರದ ಸೇವೆಯಿಂದಷ್ಟೇ ನಿವೃತ್ತರಾಗುತ್ತಿರುವುದು. ಅರೆಕಾಲಿಕಾ ಸಮಾಜ ಸೇವೆಗಷ್ಟೇ ಸೀಮಿತರಾಗಿದ್ದ ಅವರು ಈಗ ಪೂರ್ಣವಧಿ ಸೇವೆಗೆ ಸೇರ್ಪಡೆಗೊಳ್ಳುವ ಈ ಶುಭ ಸಂದರ್ಭದಲ್ಲಿ ಅವರೊಡನೆಯ ರಸ ನಿಮಿಷಗಳ ನೆನಪನ್ನು ದಾಖಲಿಸಲು ಅನುವು ಮಾಡಕೊಟ್ಟ ‘ವಿದ್ವಾನ್ ಗಂಗಾಧರ ಭಟ್ ಅಭಿನಂದನಾ ಸಮಿತಿ’ಯ ಎಲ್ಲಾ ಸದಸ್ಯರಿಗೆ ಕೃತಜ್ಙತೆಗಳನ್ನು ಸಲ್ಲಿಸುತ್ತೇನೆ.                 

\articleend
}
