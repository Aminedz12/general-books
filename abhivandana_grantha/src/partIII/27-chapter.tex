{\fontsize{14}{16}\selectfont
\chapter{ಜ್ಞಾನಗಂಗಾಧರ}

\begin{center}
\Authorline{ವಿ~॥  ಉದಯ-ಭಟ್ಟ}

ಸಹಾಯಕ ಪ್ರಾಧ್ಯಾಪಕರು, ವ್ಯಾಕರಣ ಶಾಸ್ತ್ರವಿಭಾಗ\\
ಶ್ರೀಮನ್ಮಹಾರಾಜ-ಸಂಸ್ಕೃತಮಹಾಪಾಠಶಾಲೆ\\
ಮೈಸೂರು
\addrule
\end{center}

ಹುಟ್ಟು ಹೋರಾಟಗಾರರಾಗಿರುವ ಗಂಗಾಧರ ಭಟ್ಟರದು ವಿಶಿಷ್ಟ ವ್ಯಕ್ತಿತ್ವ. ಯಾವುದೇ ಅಧಿಕಾರ ಅಂತಸ್ತುಗಳಿಲ್ಲದೆಯೂ  ವ್ಯಕ್ತಿಯೊಬ್ಬ ತನ್ನ ವಿದ್ವತ್ತು, ಅಪಾರದ ಲೋಕಾನುಭವ ಕಠಿಣ ಪರಿಶ್ರಮದಿಂದ ಹೇಗೆ ಸರ್ವಮಾನ್ಯತೆಯಿಂದ ಸಂಪಾದಿಸಬಲ್ಲ ಎನ್ನುವುದಕ್ಕೆ\break ಜೀವಂತ ಉದಾಹರಣೆ ಗಂಗಾಧರ ಭಟ್ಟರು. ನಿಷ್ಠುರ ಸ್ವಭಾವದ ಭಟ್ಟರು ತಾನು ಒಪ್ಪಿದ \enginline{-}ಅಪ್ಪಿದ ವಿಷಯವನ್ನು ಪ್ರತಿಪಾದಿಸುವ ರೀತಿ, ಸ್ವಪಕ್ಷವನ್ನು ಸಮರ್ಥಿಸಿಕೊಳ್ಳುವ ಅವರ ಅದ್ಭುತ ಮಾತುಗಾರಿಕೆ ಕೆಲವೊಮ್ಮೆ ಖ್ಯಾತ ನ್ಯಾಯವಾದಿಗಳನ್ನು ಜ್ಞಾಪಿಸುತ್ತದೆ ಎಂದರೆ ಅತಿಶಯೋಕ್ತಿಯಲ್ಲ. ಅವರೊಬ್ಬ ಚತುರ ವಾಗ್ಮಿ. ಚಾಣಾಕ್ಷ ನೀತಿಜ್ಞ, ಶಾಸ್ತ್ರ\break ನದೀಷ್ಣ, ಸರಳ ಜೀವಿ. ಅವರ ಪ್ರಭಾ ವಲಯದೊಳಗೆ ಬರದಿರುವವರೇ ಅಪರೂಪ. ಕೆಲವೊಮ್ಮೆ ಅವರನ್ನು ವಿರೋಧಿಸಲೆಂದೇ ಹೊರಟವರೂ ಕೂಡ ಅನಂತರ ಅವರ\break ಸಮರ್ಥಕರಾಗಿ ಬದಲಾಗಿದ್ದುದುಂಟು. ಭಟ್ಟರ ವೈದುಷ್ಯವನ್ನು, ಪ್ರಭಾವವನ್ನು ಬಳಸಿ\-ಕೊಂಡು ದೊಡ್ಡವರೆನಿಸಿಕೊಂಡಿದ್ದಾರೆ. ಉನ್ನತ ಸ್ಥಾನವನ್ನಲಂಕರಿಸಿದ್ದಾರೆ. ಆದರೆ ಭಟ್ಟರು ಮಾತ್ರ ಯಾವುದೆ ಫಲಾಪೇಕ್ಷೆಯಿಲ್ಲದೇ ನಿರ್ವ್ಯಾಜಬುದ್ಧಿಯಿಂದ ಉಪಕರಿಸಿದರೇ ಹೊರತು ಪ್ರತ್ಯುಪಕಾರದ ಸಣ್ಣ ನಿರೀಕ್ಷೆಯನ್ನೂ ಮಾಡಿದವರಲ್ಲ. 

ಅತಿಕ್ಲಿಷ್ಟವಾದ ಗಹನವಾದ ಶಾಸ್ತ್ರ ವಿಷಯವನ್ನು ಅತ್ಯಂತ ಸರಳವಾಗಿ ಸುಂದರವಾಗಿ, ಎದುರಿಗೆ ಕುಳಿತು ಕೇಳುತ್ತಿರುವವರು “ಈ ವಿಷಯ ಇಷ್ಟು ಸರಳವೇ !” ಎಂದು ಉದ್ಗಾರ ತೆಗೆಯುವಷ್ಟರ ಮಟ್ಟಿಗೆ ಸುಲಲಿತವಾಗಿ ನಿರೂಪಿಸುವ \hbox{ಕಲೆಗಾರಿಕೆ} ಜನ್ಮಜಾತವಾಗಿಯೇ ಭಟ್ಟರಲ್ಲಿದೆ ಎನಿಸುತ್ತದೆ. ದಿನದ ಬಹು ಭಾಗವನ್ನು \hbox{ಅಧ್ಯಯನ} \hbox{ಅಧ್ಯಾಪನ}ಗಳಲ್ಲಿಯೇ ಕಳೆಯುತ್ತಿದ್ದ ಭಟ್ಟರ ಜ್ಞಾನಸಂಪತ್ತು ಅಗಾಧ ಮತ್ತು \hbox{ಅಗಣಿತ.} ಅವರು ಕೇವಲ ಹೆಸರಿನಿಂದ ಮಾತ್ರ ಗಂಗಾಧರರಲ್ಲ. ಅವರು ನಿಜವಾಗಿ ಜ್ಞಾನ\-ಗಂಗೆಯನ್ನು ಧರಿಸಿರುವ ಜ್ಞಾನಗಂಗಾಧರ. ವೇದ, ಶಾಸ್ತ್ರ, ಇತಿಹಾಸ, ಪುರಾಣ, ಆಯುರ್ವೇದ, ವಾಣಿಜ್ಯ ಶಾಸ್ತ್ರ ಅರ್ಥಶಾಸ್ತ್ರ ಕನ್ನಡ, ಸಂಸ್ಕೃತ, ಆಂಗ್ಲಭಾಷಾ ಸಾಹಿತ್ಯ... ಹೀಗೆ ಅವರ ಜ್ಞಾನದ ಹರಹು ಅತಿವಿಶಾಲ ಹಾಗೂ ಗಹನವಾಗಿದೆ. ಪ್ರಮಾಣಪತ್ರಗಳ ಸಂಪಾದನೆಯಲ್ಲಿ ಆಸಕ್ತಿ ತೋರದ ಭಟ್ಟರು ತಮ್ಮ ಸತತ \hbox{ಪರಿಶ್ರಮ} ಅಭ್ಯಾಸಗಳಿಂದ \hbox{ವಿದ್ವತ್ತಿಗೆ} ತಾವೇ ಪ್ರಮಾಣ ಭೂತರಾಗಿದ್ದುದು ವಿಶೇಷ. \hbox{ಅವರೊಂದು} ಸಣ್ಣ \hbox{ವಿಶ್ವಕೋಶವಿದ್ದಂತೆ.} ಆದ್ದರಿಂದಲೇ ಭಟ್ಟರು ಪ್ರಮಾಣಪತ್ರ ವಿದ್ವಾಂಸರಲ್ಲ. ವಿದ್ವತ್ತಿಗೆ ಅವರೇ ಒಂದು ಪ್ರಮಾಣ ಎಂದು ಅನೇಕರು ಹೇಳುವುದು. ಭಟ್ಟರ ಪರಿಶ್ರಮ ಅವರಿಗೆ ಅಪಾರ ಶಿಷ್ಯವೃಂದವನ್ನು ಅಭಿಮಾನಿಗಳನ್ನು ಸೃಷ್ಟಿಸಿ ಕೊಟ್ಟಿದೆ. ಭಟ್ಟರ ಈ ಸಾಮರ್ಥ್ಯದ ಸಹಾಯವನ್ನು ಪಡೆದ ಅನೇಕರು ಅನೇಕ ಕಾರ್ಯಗಳನ್ನು ಸಾಧಿಸಿಕೊಂಡಿದ್ದಾರೆ. ಆದರೆ ತಮ್ಮ ಪ್ರಭಾವದ ಸಾಮರ್ಥ್ಯದ ಹತ್ತನೆಯ ಒಂದಶವಷ್ಟಾದರೂ ಬಳಸಿಕೊಂಡಿದ್ದರೆ ಭಟ್ಟರಿಂದು ರಾಜ್ಯ/ರಾಷ್ಟ್ರಮಟ್ಟದ ಪ್ರಶಸ್ತಿಗಳಿಂದ ಭೂಷಿತರಾಗಿರುತ್ತಿದ್ದರು.

ಇವರು ಸರ್ಕಾರಿ ಮಹಾರಾಜ ಸಂಸ್ಕೃತ ಮಹಾವಿದ್ಯಾಲಯದಲ್ಲಿ ೧೯೯೮ ಜನೆವರಿಯಿಂದ ೨೦೧೮ ಜನವರಿಯವರೆಗೆ ಇಪ್ಪತ್ತು ವರ್ಷಗಳ ಕಾಲ ನವೀನನ್ಯಾಯ ಶಾಸ್ತ್ರವಿಭಾಗದಲ್ಲಿ ಸಹಾಯಕ ಪ್ರಾಧ್ಯಾಪಕರಾಗಿ ಸೇವೆಯನ್ನು ಸಲ್ಲಿಸಿದ್ದಾರೆ. ತಮ್ಮ ಸೇವಾವಧಿಯಲ್ಲಿ ಇವರು ಐವರು ಪ್ರಾಂಶುಪಾಲರುಗಳನ್ನು ಕಂಡಿರುವ ಇವರು ಎಲ್ಲಾ ಪ್ರಾಂಶುಪಾಲರುಗಳಿಗೂ ಆತ್ಮೀಯರಾಗಿಯೇ ಇದ್ದವರು. ಕಾಲೇಜಿನ ಯಾವುದೇ ಕಾರ್ಯಕ್ರಮವಿರಲಿ ಶೈಕ್ಷಣಿಕ ಚಟುವಟಿಕೆಗಳಿರಲಿ ರಾಜ್ಯಮಟ್ಟದ ಸ್ಪರ್ಧೆಗಳಿರಲಿ, ವಸತಿನಿಲಯದ ಸಮಸ್ಯೆಗಳಿರಲಿ.... ಏನೇ ಇದ್ದರೂ ಇವರ ಸಲಹೆ ಸೂಚನೆಗಳನ್ನು ಕೇಳುವ ಸೌಜನ್ಯವನ್ನು ಎಲ್ಲ ಪ್ರಾಂಶುಪಾಲರುಗಳು ಪಾಲಿಸುತ್ತಿದ್ದುದು ಭಟ್ಟರಿಗೆ ಕಾಲೇಜಿನಲ್ಲಿರುವ ಮಾನ್ಯತೆಯನ್ನು, ಪ್ರಾಂಶುಪಾಲರುಗಳು ಅವರ ಮೇಲೆ ಇಟ್ಟಿರುವ ಗೌರವವನ್ನು ಸೂಚಿಸುತ್ತದೆ. 

ವಿದ್ಯಾರ್ಥಿಗಳು ಕಾಲೇಜಿನ ಯಾವುದೇ ವಿಭಾಗಕ್ಕೆ (ವೇದ-ಆಗಮ-ಶಾಸ್ತ್ರ) ಪ್ರವೇಶ ಪಡೆದಿರಲಿ ಅವರೆಲ್ಲ ಒಂದಲ್ಲ ಒಂದು ಕಾರಣಕ್ಕಾಗಿ ಭಟ್ಟರನ್ನು ಆಶ್ರಯಿಸುತ್ತಿದ್ದರು. ಹೀಗಾಗಿ ಒಂದರ್ಥದಲ್ಲಿ ಕಾಲೇಜಿನ ಎಲ್ಲ ವಿದ್ಯಾರ್ಥಿಗಳೂ ಭಟ್ಟರ ಶಿಷ್ಯರೇ ಆಗಿರುತ್ತಿದ್ದರು. ವಿದ್ಯಾರ್ಥಿಗಳಲ್ಲಿ ಕೆಲವರು ನಿರ್ದಿಷ್ಟ ವಿಷಯದ ವಿಶೇಷ ಜ್ಞಾನಕ್ಕೋಸ್ಕರ ಭಟ್ಟರನ್ನು ಆಶ್ರಯಿಸಿದರೆ ಮತ್ತೆ ಹಲವರು ಸಾಮಾನ್ಯ ಜ್ಞಾನಕ್ಕೋಸ್ಕರ ವ್ಯಾಕರಣ, ತರ್ಕ, ಅಮರಕೋಶಾದಿಗಳ ಅಧ್ಯಯನಕ್ಕಾಗಿ ಭಟ್ಟರ ಬಳಿಗೆ ಬರುತ್ತಿದ್ದರು. ಅಲಂಕಾರ,\break ವ್ಯಾಕರಣ, ಮೀಮಾಂಸಾ, ಧರ್ಮಶಾಸ್ತ್ರ ಹೀಗೆ ಯಾವುದೇ ಶಾಸ್ತ್ರವಿರಲಿ, ಏನೇ\break ಸಂಶಯವಿದ್ದರೂ ವಿದ್ಯಾರ್ಥಿಗಳು ಅವುಗಳ ಪರಿಹಾರಕ್ಕಾಗಿ ಆಶ್ರಯಿಸುತ್ತಿದ್ದುದು\break ಗಂಗಾಧರ ಭಟ್ಟರನ್ನೇ. ಅವರ ಮನೆ ಸದಾ ವಿದ್ಯಾರ್ಥಿಗಳಿಂದ (ವಿವಿಧ ವಿಷಯಗಳ) ಜಿಜ್ಞಾಸುಗಳಿಂದ ತುಂಬಿರುತ್ತಿತ್ತು. ಅವರ ಮನೆ ಸಣ್ಣ ವಿದ್ಯಾಲಯವೇ ಆಗಿತ್ತೆಂದರೆ\break ತಪ್ಪಲ್ಲ. ಮೈಸೂರು ನಗರದ ವಿವಿಧ ಕಾಲೇಜುಗಳಲ್ಲಿ ನಡೆಯುತ್ತಿದ್ದ ಚರ್ಚಾಸ್ಪರ್ಧೆಗಳಲ್ಲಿ ಭಾಷಣ ಸ್ಪರ್ಧೆಗಳಲ್ಲಿ ಭಾಗವಹಿಸುವ ವಿದ್ಯಾರ್ಥಿಗಳಲ್ಲಿ ಅನೇಕರು ವಿಷಯ ಸಂಗ್ರಹಕ್ಕಾಗಿ ಬರುತ್ತಿದ್ದುದು ಭಟ್ಟರ ಬಳಿಗೆ. ಅದು ಸಂಸ್ಕೃತಭಾಷಾ ಮಾಧ್ಯಮದಲ್ಲಿ ನಡೆಯುವ ಸ್ಪರ್ಧೆಯೇ ಆಗಿರಬೇಕಿಲ್ಲ. ಕನ್ನಡ ಅಥವಾ ಆಂಗ್ಲಭಾಷೆಗಳಲ್ಲಿಯೂ ಇದ್ದಿರಬಹುದು. ಒಟ್ಟಿನಲ್ಲಿ ಭಾಷಣ ಅಥವಾ ಚರ್ಚಾಸ್ಪರ್ಧೆಗಳೆಂದರೆ ಸಾಕು ವಿದ್ಯಾರ್ಥಿಗಳಿಗೆ ನೆನಪಾಗುತ್ತಿದ್ದವರು. ಗಂಗಾಧರ ಭಟ್ಟರು. ಅವರಲ್ಲಿರುವ ಅಪಾರವಾದ ವಿಷಯ ಸಂಗ್ರಹ ಸರಳ ನಿರೂಪಣಾ ಶೈಲಿ ಸ್ಪರ್ಧಾಳುಗಳನ್ನು ಸೂಜಿಗಲ್ಲಿನಂತೆ ಅವರೆಡೆಗೆ ಸೆಳೆಯುತ್ತಿತ್ತು. ತಮ್ಮ ಬಳಿಗೆ ಬಂದ ಎಲ್ಲರಿಗೂ ಕಿಂಚಿತ್ತೂ ಬೇಸರಿಸದೇ ಪ್ರತಿಯೊಬ್ಬರಿಗೂ ವಿಭಿನ್ನ ರೀತಿಯಲ್ಲಿ ವಿಷಯವನ್ನು ಬರೆಸುತ್ತಿದ್ದ ಭಟ್ಟರ ವಿಷಯ ಸಂಗ್ರಹ, ಸಹನೆ ಸ್ತುತ್ಯರ್ಹ. ಅಖಿಲಭಾರತ ಸ್ತರದಲ್ಲಿ ನಡೆಯುವ  “ಸಂಸ್ಕೃತ ವಾಕ್ಪ್ರತಿಯೋಗಿತಾ”ದಲ್ಲಿ ಕರ್ನಾಟಕದ ವಿದ್ಯಾರ್ಥಿಗಳ ಮಾರ್ಗದರ್ಶಕರಾಗಿ ಭಾಗವಹಿಸಿ ಅನೇಕ ವರ್ಷಗಳ ಕಾಲ ಕರ್ನಾಟಕ ರಾಜ್ಯಕ್ಕೆ ವಿಜಯ ವೈಜಯಂತೀ ಪಾರಿತೋಷಕವನ್ನು ತಂದುಕೊಟ್ಟ ಹಿರಿಮೆ ಭಟ್ಟರದು.

ವಿದ್ಯಾರ್ಥಿಗಳ ಯಾವುದೇ ಸಮಸ್ಯೆಯಿರಲಿ ಅದಕ್ಕೆ ಭಟ್ಟರು ಸ್ಪಂದಿಸುವ ರೀತಿ ಅನನ್ಯ. ಸಮಸ್ಯೆ ಯಾವುದೇ ರೀತಿಯದ್ದಾಗಿರಲಿ ಭಟ್ಟರು ಅದನ್ನು ಪರಿಹರಿಸಲು ಸದಾ ಸಿದ್ಧ. ಒಬ್ಬ ವಿದ್ಯಾರ್ಥಿಗೆ ನಗರದ ಯಾವುದೋ ಕಾಲೇಜಿನಲ್ಲಿ ಪ್ರವೇಶ ದೊರಕಿಸಿಕೊಡುವುದಿರಬಹುದು, ಅಥವಾ ನಿರ್ದಿಷ್ಟ ವಸತಿನಿಲಯದಲ್ಲಿ ಅಶನ, ವಸನ ವ್ಯವಸ್ಥೆಯನ್ನು ಕಲ್ಪಿಸಿಕೊಡುವುದಿರಬಹುದು, ಅಥವಾ ನಗರದ ಯಾವುದೋ ಭಾಗದಲ್ಲಿ ಬಾಡಿಗೆ\break ರೂಮಿನ ವ್ಯವಸ್ಥೆಯನ್ನು ಮಾಡಿಕೊಡುವುದಾಗಿರಬಹುದು, ಅಥವಾ ಈಗ ವಾಸವಾಗಿರುವಲ್ಲಿ  ಯಾವುದೊ ಸಮಸ್ಯೆಯುಂಟಾಗಿರಬಹುದು..... ಈ ಎಲ್ಲ ಸಮಸ್ಯೆಗಳನ್ನು ವಿದ್ಯಾರ್ಥಿಗಳು ಹೇಳಿಕೊಳ್ಳುತ್ತಿದ್ದುದು ಭಟ್ಟರ ಬಳಿಯಲ್ಲಿಯೇ. ಭಟ್ಟರು ಪರಿಹಾರ\break ಸೂತ್ರಧಾರರಷ್ಟೇ ಆಗಿರದೇ ಸಮಸ್ಯೆ ಪರಿಹಾರವಾಗುವವರೆಗೆ, ಕೆಲವು ಸಂದರ್ಭಗಳಲ್ಲಿ ವಿದ್ಯಾರ್ಥಿಗಳನ್ನು ತಮ್ಮ ಮನೆಯಲ್ಲಿಯೇ ಇಟ್ಟುಕೊಂಡು ಪೋಷಿಸುತ್ತಿದ್ದುದುಂಟು. ಇದಕ್ಕಾಗಿ ಅದರಿಂದ ಕಿಂಚಿತ್ ಧನವನ್ನೂ ಅಪೇಕ್ಷಿಸದಿರುವುದು ಭಟ್ಟರ ಔದಾರ್ಯದ ಆಶ್ರಿತರಲ್ಲಿ ಅವರಿಗಿರುವ ವಾತ್ಸಲ್ಯದ ಚಿಕ್ಕ ಸಂಕೇತವಷ್ಟೇ. ಭಟ್ಟರದು ಇನ್ನೊಂದು ವಿಶಿಷ್ಟ ಸ್ವಭಾವವಿದೆ. ತಮ್ಮನ್ನು ನಂಬಿ ಬಂದವರನ್ನು ಅದೆಷ್ಟೇ ಕಷ್ಟಬಂದರೂ, ಬಹುಜನರ\break  ವಿರೋಧವಿದ್ದರೂ ಬಿಟ್ಟುಕೊಡುವವರಲ್ಲ. ಕೆಲವರ್ಷಗಳ ಹಿಂದಿನ ಘಟನೆಯಿದು. ನಮ್ಮ ಕಾಲೇಜಿನ ವಸತಿನಿಲಯದಲ್ಲಿ ವಾಸಿಸುತ್ತಿದ್ದ ೩೦೦ಕ್ಕೂ ಅಧಿಕ ವಿದ್ಯಾರ್ಥಿಗಳ ಯೋಗ\-ಕ್ಷೇಮ ವಿಚಾರಣೆಗೆಂದು ಒಂದು ಸಮಿತಿಯಿತ್ತು. ಆ ಸಮಿತಿಯಲ್ಲಿ ನಾನೂ ಸದಸ್ಯ\break ನಾಗಿದ್ದೆ. ಭಟ್ಟರೂ ಸದ್ಯರಾಗಿದ್ದರು. ವಿದ್ಯಾರ್ಥಿಯೊಬ್ಬ ತೀರ ಅಶಿಸ್ತಿನ ವರ್ತನೆ\break ತೋರಿದ್ದರಿಂದ ಆತನನ್ನು ವಸತಿನಿಲಯದಿಂದ ಹೊರಹಾಕಬೇಕೆಂದು ಸಮಿತಿಯ\break ಬಹುತೇಕ ಸದಸ್ಯರು ಒತ್ತಾಯಿಸಿದ್ದರು. ಕೋತಿ ತಾನು ಕೆಟ್ಟಿದ್ದಲ್ಲದೆ ವನವನ್ನೂ ಕೆಡಿಸಿತು ಎಂಬ ಗಾದೆಮಾತಿನಂತೆ ಇತರ ವಿದ್ಯಾರ್ಥಿಗಳ ಮೇಲು ಇದರ ಕೆಟ್ಟ ಪರಿಣಾಮವಾಗಬಹುದೆಂದು ಭಾವಿಸಿ ನಾವೆಲ್ಲ ಮೇಲಿನ ನಿರ್ಧಾರಕ್ಕೆ ಬಂದಿದ್ದೆವು. ಆದರೆ ಆ ವಿದ್ಯಾರ್ಥಿ ಸಹಾಯಕ್ಕಾಗಿ ಭಟ್ಟರನ್ನು ಆಶ್ರಯಿಸಿದ್ದನೆಂದು ತೋರುತ್ತದೆ. ಭಟ್ಟರು ಮಾತ್ರ ಸಮಿತಿಯ ಸಭೆಯಲ್ಲಿ ಆತನ ಪರವಾಗಿ ಬಲವಾಗಿ ವಾದಿಸಿ ಆತನಿಗೆ ವಸತಿಯ ಸೌಲಭ್ಯವನ್ನು ಕಡಿತಗೊಳಿಸದಂತೆ ಆಗ್ರಹಿಸಿ ಅದರಲ್ಲಿ ಯಶಸ್ವಿಯೂ ಆಗಿದ್ದರು. ಭಟ್ಟರ ಪ್ರಖರವಾದ\break ಪ್ರತಿಪಾದನೆಯ ಮುಂದೆ ಸಮಿತಿಯ ಸದಸ್ಯರೆಲ್ಲರೂ ಮೂಕರಾಗಿಬಿಟ್ಟಿದ್ದರು. ಆ ಸಂದರ್ಭದಲ್ಲಿ ಅನೇಕ ಅಧ್ಯಾಪಕರುಗಳಿಗೆ ಈ ಸಂಗತಿ ಬೇಸರವನ್ನುಂಟುಮಾಡಿದ್ದು \break ಸುಳ್ಳಲ್ಲ. ಅನಂತರ ಭಟ್ಟರು ಆ ವಿದ್ಯಾರ್ಥಿಯನ್ನು ತಮ್ಮ ಬಳಿ ಕರೆಸಿಕೊಂಡು ಚೆನ್ನಾಗಿ ಬಯ್ದು ಬುದ್ಧಿ ಹೇಳಿ ಆತನಿಗೆ ತಪ್ಪಿನ ಅರಿವನ್ನು ಮಾಡಿಸಿದ್ದು ಬೇರೆ ವಿಚಾರ.\break   ವಿದ್ಯಾರ್ಥಿಗಳ ಮೇಲೆ ಗಂಗಾಧರ ಭಟ್ಟರಿಗಿರುವ ಅತಿಯಾದ ವಾತ್ಸಲ್ಯದ ಪ್ರತೀಕವಾಗಿ ನಾವಿದನ್ನು ಸ್ವೀಕರಿಸಬಹುದು. ಹೀಗೆ ವಿಷಯ, ಸಂದರ್ಭ ಯಾವುದೇ ಇರಲಿ, ತನಗೆ ಸರಿ ಎನಿಸಿದ  ವಿಷಯವನ್ನು  ಸಮರ್ಥವಾಗಿ ಪ್ರತಿಪಾದಿಸುವ, ತಾನು ಹಿಡಿದ ಕಾರ್ಯವನ್ನು ಛಲದಿಂದ ಸಾಧಿಸುವ, ತನ್ನ ನಂಬಿದವರನ್ನು ಎಂದೂ ಕೈಬಿಡದ, ಸರಿತಪ್ಪುಗಳ\break  ವಿಮರ್ಶೆಯೊಂದಿಗೆ ಸನ್ಮಾರ್ಗದರ್ಶಕರಾಗಿರುವ ಗಂಗಾಧರ ಭಟ್ಟರು ನಮ್ಮ ಮಹಾ ವಿದ್ಯಾಲಯದ ಆಧಾರಸ್ತಂಭವೇ ಆಗಿದ್ದರು ಎಂದರೆ ತಪ್ಪಾಗಲಾರದು. 

ಭಟ್ಟರ ಇನ್ನೊಂದು ವಿಶೇಷಗುಣ ಅವರಲ್ಲಿರುವ ಅಪಾರವಾದ ಶಾಸ್ತ್ರಪ್ರೀತಿ. ಶಾಸ್ತ್ರವಿಷಯಕ ಚಿಂತನೆಯಲ್ಲಿ ವಿದ್ಯಾರ್ಥಿ-ಶಿಕ್ಷಕ ಎಂಬ ಭೇದಭಾವ ಅವರಲ್ಲಿರಲಿಲ್ಲ. ಅವರ\break ವಿರೋಧಿಗಳಾಗಿದ್ದರೂ ಸರಿ, ಪರಶಾಸ್ತ್ರ ವಿಷಯವಾಗಿದ್ದರೂ ಸರಿ. ಚಿಂತನೆಗೆ ಭಟ್ಟರು ತೊಡಗಿಕೊಂಡರೆಂದರೆ ಆ ಶಾಸ್ತ್ರದ ನಾನಾ ಮುಖಗಳನ್ನು ಪರಿಚಯಿಸುವುದರೊಂದಿಗೆ\break ಸರಳಪರಿಹಾರವನ್ನು ಸೂಚಿಸಿಯೇ ಆ ವಿಷಯದಿಂದ ನಿವೃತ್ತರಾಗುತ್ತಿದ್ದರು. ಇದು ಕೇಳುಗರಿಗೊಂದು ರಸದೌತಣವಿದ್ದಂತೆ. ಭಟ್ಟರೇನಾದರು ಸಮಸ್ಯೆಯನ್ನು ಮಂಡಿಸಿ ಸಹೋದ್ಯೋಗಿಗಳೊಡನೆ ಚರ್ಚೆಗೆ ಕುಳಿತರೆಂದರೆ ಅದರ ಪರಿಹಾರ ಅವರಿಂದಲ್ಲದೇ ಬೇರೆಯವರಿಂದ ಸಾಧ್ಯವಿಲ್ಲವೆಂದೇ ಎಲ್ಲರ ಭಾವನೆಯಾಗಿರುತ್ತಿತ್ತು. ಕೆಲವೊಮ್ಮೆ ಸಮಸ್ಯೆಯ ಪರಿಹಾರಕ್ಕಾಗಿ ಭಟ್ಟರು ಆಯಾಯ ಶಾಸ್ತ್ರದ ಪರಿಧಿಯನ್ನು ದಾಟುತ್ತಿದ್ದುದೂ ಉಂಟು. ಇದು ಕೆಲವರಿಗೆ ಇರುಸು\enginline{-} ಮುರುಸು ಉಂಟುಮಾಡುತ್ತಿದ್ದುದು ಸುಳ್ಳಲ್ಲ. ಆದರೆ\break ಒಂದಂತೂ ನಿಶ್ಚಿತವಾಗಿರುತ್ತಿತ್ತು. ಅದೇನೆಂದರೆ ಆ ಸಮಯದಲ್ಲಿ ಭಟ್ಟರ ಮಾತನ್ನು ವಿರೋಧಿಸುವ ಸಾಮರ್ಥ್ಯ ಮಾತ್ರ ಸಹೋದ್ಯೋಗಿಗಳಲ್ಲಿ ಯಾರಿಗೂ ಇರಲಿಲ್ಲ. 

ಯಾರನ್ನು ಯಾವ ಕೆಲಸದಲ್ಲಿ ಯಾವಾಗ ತೊಡಗಿಸಬೇಕು, ಯಾರಿಂದ ಯಾವ ಕೆಲಸವನ್ನು ತೆಗೆಸಬೇಕು  ಎಂಬುದನ್ನು ಗಂಗಾಧರ ಭಟ್ಟರು ಚೆನ್ನಾಗಿ ಅರಿತಿದ್ದರು. ಈ ಕಾರಣದಿಂದಾಗಿಯೇ ಅವರು ವಹಿಸಿಕೊಂಡ ಕೆಲಸವನ್ನು ಅಚ್ಚುಕಟ್ಟಾಗಿ ಮಾಡಿಮುಗಿಸುತ್ತಿದ್ದರು. ಯೋ ಯಸ್ಮಿನ್ ಕುಶಲಃ ತಂ ತಸ್ಮಿನ್ನೇವ ಯೋಜಯೇತ್ - (ಚಾಣಕ್ಯನೀತಿಸೂತ್ರ .ಅ\enginline{-}೨ ) ಎಂಬ ಚಾಣಕ್ಯನ ನೀತಿಸೂತ್ರವನ್ನು ಚೆನ್ನಾಗಿ ಅರಿತಿದ್ದ ಭಟ್ಟರು ಅದರಂತೆ  ನಡೆಯುತ್ತಿದ್ದರೂ ಕೂಡ. ಹಾಗಂತ ಭಟ್ಟರಲ್ಲಿ ದೊಷಗಳೇ ಇರಲಿಲ್ಲವೆಂದರ್ಥವಲ್ಲ. ಅತಿಕೋಪ, ಹಠಮಾರಿತನ ಈ ಮೊದಲಾದ ದೋಷಗಳೂ ಅವರಲ್ಲಿದ್ದವು. ವಿಪಶ್ಚಿತ್ಸು ಅಪಿ ಸುಲಭಾ ದೋಷಾಃ (ಚಾ.ನೀ.\enginline{-}ಅ\enginline{-}೩) ಎಂಬುದಾಗಿಯೂ ಆತನೇ ವಿವರಿಸಿದ್ದಾನೆ.\break ಭಟ್ಟರಲ್ಲಿರುವ ಕಾರ್ಯಕ್ಷಮತೆ, ಅಪಾರ ಲೋಕಾನುಭವ, ಅಸದೃಶ ವೈದುಷ್ಯ\break ಅವರಿಗೆ ಅಪಾರ ಅಭಿಮಾನಿಗಳನ್ನೂ, ಶಿಷ್ಯಸಂಪತ್ತನ್ನೂ ಗಳಿಸಿಕೊಟ್ಟಿದೆ. ದೈವಾನುಗ್ರಹ\-ದಿಂದಲೋ  ಪೂರ್ವಜನ್ಮದ ಸುಕೃತದಿಂದಲೋ ಅಥವಾ ಸ್ವಪರಿಶ್ರಮದಿಂಲೋ ಅವರು ಹೊಂದಿರುವ ತೀಕ್ಷ್ಣಬುದ್ಧಿಮತ್ತೆ, ವಿವೇಚನಾ ಸಾಮರ್ಥ್ಯ, ವಾಕ್ಚಾತುರ್ಯ ಮತ್ತು ಪ್ರತ್ಯುತ್ಪನ್ನಮತಿತ್ವ, ವಿಷಯದ ತಲಸ್ಪರ್ಶಿ ಅಧ್ಯಯನದಿಂದ ತಮ್ಮದಾಗಿಸಿಕೊಂಡಿರುವ ಶಾಸ್ತ್ರವೈದುಷ್ಯ ಇವುಗಳೆ ಕೆಲವೊಮ್ಮೆ ಅವರ ಉಜ್ವಲ ಭವಿಷ್ಯಕ್ಕೆ ಮುಳುವಾಗಿದ್ದು\break ಸುಳ್ಳಲ್ಲವೇನೋ. ಯಾವುದೇ ವಿಷಯವಿರಲಿ ಅದನ್ನು ಚೆನ್ನಾಗಿ ವಿಚಾರ \enginline{-} ವಿಮರ್ಶೆ ಮಾಡದೆಯೇ, ಸರಿತಪ್ಪುಗಳ ಸಂಕಲನ-ವ್ಯವಕಲನಗಳಿಲ್ಲದೆಯೇ ಯಾರೇ ಹೇಳಿದರೂ ಒಪ್ಪದ, ತಪ್ಪನ್ನು ತಪ್ಪೆಂದೇ ವಾದಿಸುವ ಅವರ ನಿರ್ದಿಷ್ಟ \enginline{-} ನಿರ್ದುಷ್ಟ ಬುದ್ಧಿಯೇ\break ಅವರಿಗೆ ಪ್ರತಿಕೂಲ\-ಪರಿಸ್ಥಿತಿಯನ್ನು ನಿರ್ಮಾಣಮಾಡಿದ್ದಿರಲೂಬಹುದು. ಹೀಗಾಗಿ\break  ಹಿತೈಷಿಗಳ ಜೊತೆಜೊತೆಯಲ್ಲಿಯೇ ಕೆಲ ಹಿತ ಶತ್ರುಗಳೂ ಅವರಿಗೆ ಹುಟ್ಟಿಕೊಂಡಿದ್ದು\break ಸುಳ್ಳಲ್ಲ. ಈ ಪಟ್ಟಭದ್ರಹಿತಶತ್ರುಗಳೇ ಭಟ್ಟರ ಉನ್ನತಿಗೆ ಬಾಧಕವಾಗಿದ್ದಿರಲೂಬಹುದು. 

“ಮನುಷ್ಯನು ಸ್ವಕ್ಷೇತ್ರದಲ್ಲಿ ಮಾತ್ರ ಗೌರವಕ್ಕೆ ಪಾತ್ರನಾಗುತ್ತಾನೆ (ಸ್ಥಾನೇ ಏವ ನರಾಃ ಪೂಜ್ಯಂತೇ.\enginline{-}೫) ಎಂಬ ಚಾಣಕ್ಯವಚನದ ತಾತ್ಪರ್ಯವನ್ನೂ ಮೀರಿ ಭಟ್ಟರು ಸ್ವಕ್ಷೇತ್ರ ಪರಕ್ಷೇತ್ರಗಳಲ್ಲಿಯೂ ಮನ್ನಣೆಯನ್ನು ಗಳಿಸಿದವರು. ದೇಶ \enginline{-} ವಿದೇಶಗಳಲ್ಲಿ ಅವರ ವೈದುಷ್ಯದ ಕಂಪು ಹರಡಿತ್ತು. ಹೀಗಾಗಿ ಭಟ್ಟರ ಪ್ರಭಾವಲಯವೂ ವಿಶಾಲ\-ವಾಗಿಯೇ ಇತ್ತು. ವೃತ್ತಿಯಲ್ಲಿ ಸಾಮಾನ್ಯ ನೈಪುಣ್ಯವುಳ್ಳ ಅನೇಕರು ರಾಜ್ಯಮಟ್ಟದ ಹಾಗು ರಾಷ್ಟ್ರಮಟ್ಟದ ಗೌರವಗಳಿಗೆ, ಪ್ರಶಸ್ತಿಗಳಿಗೆ ಭಾಜನರಾಗಿರುವ ಅನೇಕ ಉದಾಹರಣೆಗಳು ನಮ್ಮ\break ಮುಂದಿರುವಾಗ ಭಟ್ಟರಿಗೆ ಜಿಲ್ಲಾಮಟ್ಟದ ಪ್ರಶಸ್ತಿಯೂ ಬಂದಿಲ್ಲವೆಂದರೆ ಇದರ ಹಿಂದಿನ ಕಾರಣ ಸುಸ್ಪಷ್ಟ. ಭಟ್ಟರಂತೆ ವೈದುಷ್ಯವುಳ್ಳ ಪ್ರಭಾವವುಳ್ಳ ಸಮಕಾಲಿಕ ವಿದ್ವಾಂಸರು\-ಗಳನ್ನು ಸೂಕ್ಷ್ಮವಾಗಿ ಗಮನಿಸಿದಾಗ ಅವರೆಲ್ಲರೂ ವೃತ್ತಿಗೆ ಬದಲಾಗಿ ಪ್ರವೃತ್ತಿಗೆ ಅಥವಾ ಹವ್ಯಾಸಕ್ಕೆ ಹೆಚ್ಚು ನಿಷ್ಠರಾಗಿರುವುದು ಕಂಡುಬರುತ್ತದೆ. ತನ್ಮೂಲಕ ಅವರು ಅನೇಕ\break ಪ್ರಶಸ್ತಿ\enginline{-}ಪುರಸ್ಕಾರಗಳಿಗೆ ಭಾಜನಾರಾಗಿದ್ದಾರೆ. ಆದರೆ ಈ ವಿಷಯದಲ್ಲಿ ಮಾತ್ರ ಭಟ್ಟರದು ತದ್ವಿರುದ್ಧವಾದ ಸ್ವಭಾವ. ತಾನು ಯಾವ ವೃತ್ತಿಗೆ ನಿಯುಕ್ತಿಗೊಂಡಿದ್ದೇನೆಯೋ ಆ ವೃತ್ತಿಯನ್ನು ಅರ್ಥಾತ್ ಶಿಕ್ಷಕ ವೃತ್ತಿಯನ್ನು ಅಪಾರವಾಗಿ ಪ್ರೀತಿಸಿ, ಅಧ್ಯಯನ ಅಧ್ಯಾಪನ\-ಗಳಲ್ಲಿಯೇ ದಿನದ ಬಹುಭಾಗವನ್ನು ಕಳೆಯುತ್ತ ವಿದ್ಯಾರ್ಥಿಗಳ ಸರ್ವತೋಮುಖವಾದ ಅಭಿವೃದ್ಧಿಯನ್ನೇ ಸದಾ ಚಿಂತಿಸುತ್ತಾ, ಸಾಹಿತ್ಯ-ಸಂಗೀತ-ನಾಟಕ-ಯಕ್ಷಗಾನ ಹೀಗೆ ಕಲೆಯ ವಿವಿಧ ಪ್ರಕಾರಗಳಲ್ಲಿ ವಿದ್ಯಾರ್ಥಿಗಳನ್ನು ಉತ್ಸಾಹದಿಂದ ತೊಡಗಿಸಿಕೊಳ್ಳುವಂತೆ ಸದಾ ಪ್ರೋತ್ಸಾಹಿಸುತ್ತಾ ಅವರ ಬೆನ್ನೆಲುಬಿನಂತಿದ್ದ ಭಟ್ಟರು ವೃತ್ತಿಯ ಬಗ್ಗೆ ತಮಗಿರುವ ನಿಷ್ಠೆ, ಪ್ರಾಮಾಣೀಕತೆ, ಪ್ರೀತಿ, ಗೌರವಗಳನ್ನು ಈ ಮೂಲಕ ತೋರಿಸಿಕೊಟ್ಟಿದ್ದಾರೆ.

ಭಟ್ಟರ ಗುಣದೋಷಗಳ ವಿಮರ್ಶೆ ಈ ಲೇಖನದ ವಿಷಯವಲ್ಲ. ಹಾಗಿದ್ದರೂ\break ಗುಣಾಧಿಕ್ಯದಿಂದ ಬಹುಜನರ ಮನಸ್ಸನ್ನು ಗೆದ್ದಿರುವ ಭಟ್ಟರು ತಮ್ಮಲ್ಲಿರುವ ಅಲ್ಪ\break ದೋಷದಿಂದಾಗಿ ಕೆಲವೇ ಕೆಲವರ ದೂಷಣೆಗೆ, ಅಸಮಾಧಾನಕ್ಕೆ ಒಳಗಾಗದಿರುವುದೂ ಸುಳ್ಳಲ್ಲ. ನಾಸ್ತಿ ರತ್ನಮಖಂಡಿತಮ್ \enginline{-} ದೋಷವೇ ಇಲ್ಲದ ರತ್ನವಿಲ್ಲ (ಅ\enginline{-}೩) ಎಂಬ\break ಚಾಣಕ್ಯನ ಈ ನುಡಿ ಭಟ್ಟರಂತಹ ವಿದ್ವಾಂಸರನ್ನೇ ಉದ್ದೇಶಿಸಿ ಹೇಳಿದಂತಿದೆ. 
ತನ್ನ ಕಾರ್ಯವನ್ನು ಪ್ರಾಮಾಣಿಕತೆಯಿಂದ, ಪ್ರೀತಿಯಿಂದ ನಿರ್ವಹಿಸಿದ ತೃಪ್ತಿಯೊಂದಿಗೆ ವೃತ್ತಿಯಿಂದ ನಿವೃತ್ತರಾಗಿ ಪ್ರವೃತ್ತಿಯೆಡೆಗೆ ಮುಖಮಾಡಿರುವ ಭಟ್ಟರಿಗೆ ಭಗವಂತನು ಅವರ ಬಾಳನ್ನು ಸುಖಮಯವಾಗಿರಿಸಿ ಆಯುರಾರೋಗ್ಯಭಾಗ್ಯವನ್ನು \hbox{ಕರುಣಿಸಲಿ} ಎಂದು ಅವರ ಅಭಿಮಾನಿಗಳೆಲ್ಲರ ತುಂಬುಹೃದಯದ ಹಾರೈಕೆಯಾಗಿದೆ. \hbox{ಗಂಗಾಧರ} ಭಟ್ಟರ ನಿರ್ಗಮನ ನಮ್ಮ ಸಂಸ್ಕೃತ ಮಹಾವಿದ್ಯಾಲಯದಲ್ಲಿ ಒಂದು ವಿಧದ \hbox{ನಿರ್ವಾತ} ಸ್ಥಿತಿಯನ್ನು ನಿರ್ಮಾಣಮಾಡಿದೆಯೇನೋ ಅನ್ನಿಸುತ್ತಿದೆ. ಒಬ್ಬ ಘನ \hbox{ವಿದ್ವಾಂಸ,} ವಾಗ್ಮಿ, ಶ್ರೇಷ್ಠ ಸಲಹೆಗಾರ, ಹಿರಿಯ ಮಾರ್ಗದರ್ಶಕ, ಆಪದ್ಬಂಧು, ವಿದ್ಯಾಪಕ್ಷಪಾತಿ \hbox{ಎಲ್ಲಕ್ಕಿಂತ} ಹೆಚ್ಚಾಗಿ ಉತ್ತಮ ಶಿಕ್ಷಕರೊಬ್ಬರು ಈ ಮಹಾವಿದ್ಯಾಲಯದಿಂದ ನಿವೃತ್ತಿಯನ್ನು  ಹೊಂದಿದ್ದಾರೆ. ನನ್ನಂಥ ಅನೇಕರಿಗೆ ಅನೇಕ ವಿಷಯಗಳಲ್ಲಿ ಉತ್ತಮವಾದ  ಸಲಹೆಗಳನ್ನು ನೀಡುತ್ತಿದ್ದ ಶಾಸ್ತ್ರಸಂಬಂಧವಾದ ಸಮಸ್ಯೆಗಳನ್ನು ಪರಿಹರಿಸುತ್ತಾ ಜ್ಞಾನ \hbox{ದೀವಿಗೆಯನ್ನು} \hbox{ಬೆಳಗುತ್ತಿದ್ದ} ಭಟ್ಟರು ತಮ್ಮ ಯಶಸ್ಸಿನೊಂದಿಗೆ ಹುಟ್ಟೂರಿಗೆ ತೆರಳಿದ್ದಾರೆ. ಅವರಿಗೆ \hbox{ಶುಭವಾಗಲಿ.}


\articleend
}
