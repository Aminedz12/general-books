\chapter{ಸನ್ಮಿತ್ರನಾದ ಗಂಗಾಧರಣ್ಣ}

\begin{center}
\Authorline{ವಿ॥ ಹೇರಂಬ ಅರ್. ಭಟ್ಟ}
\smallskip

ನವೋದಯವಿದ್ಯಾಲಯ\\
ಸಿದ್ಧಾರ್ಥಲೇಔಟ್,\\ 
ಮೈಸೂರು
\addrule
\end{center}

\begin{verse}
ಪಾಪಾನ್ನಿವಾರಯತಿ ಯೋಜಯತೇ ಹಿತಾಯ\\
ಗುಹ್ಯಂ ನಿಗೂಹತಿ ಗುಣಾನ್ ಪ್ರಕಟೀಕರೋತಿ ।\\
ಆಪದ್ಗತಂ ಚ ನ ಜಹಾತಿ ದದಾತಿ ಕಾಲೇ\\
ಸನ್ಮಿತ್ರಲಕ್ಷಣಮಿದಂ ಪ್ರವದಂತಿ ಸಂತಃ ॥
\end{verse}
ಕೆಟ್ಟದ್ದರಿಂದ ದೂರವಿಡುತ್ತಾನೆ, ಒಳಿತಿನಲ್ಲಿ ತೊಡಗಿಸುತ್ತಾನೆ, ಗುಟ್ಟನ್ನು (secretes) ಮುಚ್ಚಿಡುತ್ತಾನೆ, ಗುಣವನ್ನು ಸಾರುತ್ತಾನೆ, ಕಷ್ಟ ಕಾಲದಲ್ಲಿ ಬಿಟ್ಟು ಹೋಗನು, ಸಂದರ್ಭದಲ್ಲಿ ಸರ್ವವನ್ನು ತ್ಯಾಗಮಾಡುತ್ತಾನೆ. ಇಂತಹ ಒಳ್ಳೆಯ ಮಿತ್ರನ ಈ ಲಕ್ಷಣವನ್ನು ಸಂತರು (ಸಜ್ಜನರು) ಸಾರುತ್ತಾರೆ. 

ಜನ್ಮಜಾತನಾಗಿ (ದೊಡ್ಡಪ್ಪನಮಗ) ಅಣ್ಣನಾದರೂ, ಗಂಗಾಧರಣ್ಣ ನನಗೆ ಒದಗಿ ಬಂದಿದ್ದು ಸನ್ಮಿತ್ರನಾಗಿ, ಸಂತರು ಸಾರುವ ಈ ಮೇಲಿನ ಸನ್ಮಿತ್ರನ ಗುಣಗಳು ನನ್ನ ಪಾಲಿಗೆ ಅವನಿಂದ ಪ್ರಕಟವಾಗಿದೆಯೆಂದರೆ, ಅದು ವಾಸ್ತವಾನುಭವ.

ನಾವಿಬ್ಬರು ಒಂದೇ ಕುಟುಂಬದ ಎರಡು ಕವಲುಗಳ ಕುಡಿಗಳು, ಅಂದರೆ ಇವನ ತಂದೆ ವೇ। ವಿಘ್ನೇಶ್ವರ ಭಟ್ಟರು ಹಾಗೂ ನನ್ನ ತಂದೆ ವೇ। ರಾಮಕೃಷ್ಣ ಭಟ್ಟರು ಅಣ್ಣತಮ್ಮಂದಿರು.

ಸನಾತನ ವೈದಿಕ ಕುಟುಂಬ ನಮ್ಮದು. ಯಜುರ್ವೇದದ  ತೈತ್ತಿರೀಯ ಶಾಖಾಧ್ಯಾಯಿಗಳಾದ ನಮ್ಮ  ಪೂರ್ವಜರು ಆಂಗಿರಸ ಗೋತ್ರೋತ್ಪನ್ನರು. ಜೀವಿಕೆಗೆ ಕೃಷಿಯನ್ನು ಅವಲಂಬಿಸಿದರೂ, ಪೌರೋಹಿತ್ಯವನ್ನು ಕರ್ತವ್ಯವಾಗಿ ಸ್ವೀಕರಿಸಿ, ಸನಾತನಧರ್ಮದ ರಕ್ಷಣೆಯಲ್ಲಿ ಅಮೂಲ್ಯಸೇವೆ ಸಲ್ಲಿಸಿದವರು. ಶಿಷ್ಯಹಿತರಕ್ಷಣೆಯಲ್ಲಿ  ಸದಾ ಉದ್ಯತರಾದ ಇವರು ನಿಜಾರ್ಥದಲ್ಲಿ ಆಚಾರ್ಯರೆನಿಸಿಕೊಂಡವರು. 

ಉತ್ತರಕನ್ನಡ ಜಿಲ್ಲೆಯ ಸಿದ್ದಾಪುರ ತಾಲ್ಲೂಕು ಕವಲಕೊಪ್ಪ ಗ್ರಾಮದ ಅಗ್ಗೆರೆ ನಮ್ಮ ಹುಟ್ಟೂರು.ಇದೇ ನಮ್ಮ ಪೂರ್ವಜರು ಬಾಳಿಬೆಳಗಿದ ಸ್ಥಳ ಕೂಡ. ಸದ್ವೈದಿಕರಾಗಿ ಹೆಸರು ಮಾಡಿದ್ದ ವೇ। ಗಣಪ ಭಟ್ಟರು ನಮ್ಮ  ಪಿತಾಮಹರು. ದೇವರು, ಮಹಾಬಲೇಶ್ವರ, ವಿಘ್ನೇಶ್ವರ ಹಾಗೂ ರಾಮಕೃಷ್ಣ ಈ ನಾಲ್ವರು ಗಣಪ ಭಟ್ಟರ ಗಂಡು ಸಂತಾನ. ಈ ನಾಲ್ವರೂ ವೇದಾಧ್ಯಯನ ಸಂಪನ್ನರಾಗಿ ಕುಲಪರಂಪರೆಯನ್ನು ಕಲಂಕರಹಿತವಾಗಿ ಮುನ್ನಡೆಸಿದವರು. ಪಿತಾಮಹ ಗಣಪ ಭಟ್ಟರನ್ನು ಹಾಗೂ ಹಿರಿಯ ದೊಡ್ಡಪ್ಪ ದೇವರು ಭಟ್ಟರನ್ನು ನಾವಿಬ್ಬರೂ ನೋಡಿಲ್ಲ. ಗಣಪ ಭಟ್ಟರ ಮಕ್ಕಳಲ್ಲಿ ಎರಡನೆಯವರಾದ ವೇ। ಮಹಾಬಲೇಶ್ವರ ಭಟ್ಟರು ಜ್ಯೌತಿಷ ಹಾಗೂ ಧರ್ಮಶಾಸ್ತ್ರಗಳಲ್ಲಿ ನಿರ್ಣಯ ನೀಡಬಲ್ಲ ಸಮಾಜಮುಖಿಯಾಗಿದ್ದ ವೈದಿಕರೆನಿಸಿಕೊಂಡಿದ್ದರು. ಮನೆಯ ಮಕ್ಕಳೆಲ್ಲ ವಿದ್ಯಾವಂತರಾಗಬೇಕೆಂಬ ಕಳಕಳಿಯುಳ್ಳ ವಿದ್ಯಾಪಕ್ಷಪಾತಿಗಳಾಗಿದ್ದರು.

ವೇ। ಗಣಪಭಟ್ಟರ ಮೂರನೆಯ ಸುಪುತ್ರರೇ ವೇ। ವಿಘ್ನೇಶ್ವರ ಭಟ್ಟರು. ತಂದೆ ಹಾಗೂ ಹಿರಿಯಣ್ಣನ ಕಾಲಾನಂತರ ಮಹಾಬಲೇಶ್ವರಣ್ಣನ ನಿರ್ದೇಶನದಲ್ಲಿ ಸಂಸಾರದ ನೊಗಕ್ಕೆ ಹೆಗಲುಕೊಟ್ಟು, ಅದನ್ನು ಮೇಲೆತ್ತಿದ ಸಾಹಸಿ ಪುರುಷ. ದೂರದೃಷ್ಟಿ ಚಾಣಾಕ್ಷ್ಷ ಬುದ್ಧಿ, ಕಷ್ಟವನ್ನು ಎದೆಗುಂದದೆ ಎದುರಿಸುವ ಧೈರ್ಯ,ಸಮಸ್ಯೆಯನ್ನು ನಯವಾಗಿ ಬಗೆಹರಿಸುವ ಕೌಶಲ ಮುಂತಾದ ಅನೇಕ ಗುಣಗಣಗಳಿಂದ ಸಮಾಜದಲ್ಲಿ ಜನಪ್ರಿಯರಾಗಿದ್ದವರು. ಪತಿವ್ರತೆ ಹಾಗೂ ಸದ್ಗೃಹಿಣಿ ರೇವತಿ ಶ್ರೀಯುತರ ಪತ್ನಿ. ಈ ದಂಪತಿಗಳಿಗೆ ಮಂಜುನಾಥ, ಶ್ರೀಧರ, ಗಂಗಾಧರ, ವೇದಾವತಿ, ಲೀಲಾವತಿ, ಹೇಮಾವತಿ, ರತ್ನಾವತಿ ಎಂಬ ಏಳು ಮಕ್ಕಳು.

ಹಿರಿಯ ಮಗನಾದ ಮಂಜುನಾಥ ವೇದ, ಜ್ಯೌತಿಷ ಹಾಗೂ ಅದ್ವೈತ ವೇದಾಂತವನ್ನು ಗುರುಮುಖೇನ ಅಧ್ಯಯನ ಮಾಡಿ, ಕುಲಪರಂಪರೆಯಾದ ಪೌರೋಹಿತ್ಯವನ್ನು ನಿರ್ವಹಿಸುತ್ತಲೇ, ಸಂಸ್ಕೃತಪಾಠಶಾಲೆಯೊಂದರ ಮುಖ್ಯೋಪಾಧ್ಯಾಯರಾಗಿ, ಕಾರ್ಯನಿರ್ವಹಿಸಿದವರು. ಶಿಷ್ಯಹಿತ ಸಾಧಿಸುತ್ತಲೆ ಇತ್ತೀಚೆಗೆ ಅಕಾಲಿಕವಾಗಿ ದಿವಂಗತರಾದವರು. ಎರಡನೆಯ ಮಗನಾದ ಶ್ರೀಧರ ವಿಜ್ಞಾನ ಪದವೀಧರನಾಗಿದ್ದು, ಮುದ್ರಣೋದ್ಯಮದಲ್ಲಿ ಜೀವಿಕೆಯನ್ನು ಕಂಡುಕೊಂಡವನು. ವೇದಾವತಿ ಹಾಗೂ ಲೀಲಾವತಿ ಕೃಷಿಕ ಸದ್ಗೃಹಸ್ಥರನ್ನು ವಿವಾಹವಾಗಿ ಆದರ್ಶ ಗೃಹಿಣಿಯರಾಗಿದ್ದಾರೆ, ಹೇಮಾವತಿ ಹಾಗೂ ರತ್ನಾವತಿ ಕ್ರಮವಾಗಿ ಬೆಂಗಳೂರಿನಲ್ಲಿ ಹಾಗೂ ಮೈಸೂರಿನಲ್ಲಿ ಉದ್ಯೋಗಿಗಳಾಗಿದ್ದು, ಸದ್ಗುಣಿಗಳನ್ನು ವಿವಾಹಿತರಾಗಿ ನೆಮ್ಮದಿಯ ಜೀವನ ಸಾಗಿಸುತ್ತಿದ್ದಾರೆ. ಮೂರನೆ ಮಗ ಗಂಗಾಧರನೆ ಮುಂದಿನ ನನ್ನ ವೃತ್ತಾಂತದ ನಾಯಕ. ಇವನನ್ನು ನಾವೆಲ್ಲ ಪ್ರೀತಿಯಿಂದ ಕರೆಯುವುದು ‘ಗಂಗಣ್ಣ’ ಎಂದು.

ಗಂಗಣ್ಣ ನನಗಿಂತ ಎರಡು ವರ್ಷ ಮೂರು ತಿಂಗಳ ಮೊದಲು ಜನಿಸಿದವನಾದರೂ, ನಮ್ಮಿಬ್ಬರ ಜೀವನದ ಅನೇಕ ಘಟನೆಗಳು ಒಟ್ಟೊಟ್ಟಿಗೆ ಸಾಗಿವೆ. ನಮ್ಮಿಬ್ಬರ ಉಪನಯನ ಸಂಸ್ಕಾರ ಒಂದೇ ಮುಹೂರ್ತದಲ್ಲಿ ಅಗ್ಗೆರೆಯ ಮನೆಯಲ್ಲಿ ನಡೆದಿರುವುದು ಇದರ ಪ್ರಾರಂಭ.

ನಮ್ಮ ಕುಟುಂಬದ  ಕೃಷಿಕ್ಷೇತ್ರ ಮೂರು ಕಡೆಯಲ್ಲಿ ಹರಡಿದೆ.ಮೊದಲನೆಯದು ಅಗ್ಗೆರೆ. ಇಲ್ಲಿ ಅಡಿಕೆತೋಟ ಹಾಗೂ ಸ್ವಲ್ಪ ಭತ್ತದಗದ್ದೆ. ನಾಲಿಗಾರು ಗ್ರಾಮದ ಮಣ್ಣಿಕೊಪ್ಪ ಎಂಬಲ್ಲಿ ಕೇವಲ ಭತ್ತದಗದ್ದೆ ಹಾಗೂ ಅಗ್ಗೆರೆಯ ಸಮೀಪದ ತಟ್ಟಿಸರ ಎಂಬಲ್ಲಿ ಸ್ವಲ್ಪ ಭತ್ತದಗದ್ದೆ ಹಾಗೂ ಸ್ವಲ್ಪ ಅಡಿಕೆತೋಟ.

ಮಣ್ಣಿಕೊಪ್ಪ, ಅಗ್ಗೆರೆಯ ಮೂಲಮನೆಯಿಂದ ಸುಮಾರು ಎರಡು ಕಿಲೋಮೀಟರ್ ದೂರದಲ್ಲಿ (ಗುಡ್ಡವೊಂದರ ಆಚೆ) ಇದ್ದು, ಅಲ್ಲಿ ಸರಿಯಾಗಿ ಕೃಷಿಕಾರ್ಯ ನಡೆಯುತ್ತಿರಲಿಲ್ಲ. ಈ ಜಮೀನನ್ನು ಊರ್ಜಿತಗೊಳಿಸುವ ಕುಟುಂಬದ ನಿರ್ಧಾರದಿಂದ, ಅಲ್ಲೊಂದು ಮನೆಯನ್ನು ನಿರ್ಮಿಸಿ, ವೇ। ವಿಘ್ನೇಶ್ವರ ಭಟ್ಟರು (1955 ರಿಂದ) ಸಪತ್ನೀಕರಾಗಿ, ಅಲ್ಲಿ ನೆಲೆಸಲು ಪ್ರಾರಂಭಿಸಿದರು. ಎಲ್ಲರ ಸಹಕಾರ ಹಾಗೂ ಶ್ರೀಯುತರ ಶ್ರಮದಿಂದ ಅಲ್ಲಿಯೂ ಒಂದು ಅಡಿಕೆತೋಟ ಸಿದ್ಧವಾಯಿತು. ಭತ್ತದಗದ್ದೆಯ ಅಂಚಿನಲ್ಲಿ ಹರಿಯುವ ಹೊಳೆಯ ಬದಿಯಲ್ಲಿ ದೊರೆತ ಹನುಮನಿಗೆ ಚಿಕ್ಕದೊಂದು  ಗುಡಿಯು ನಿರ್ಮಾಣವಾಯಿತು. 

ಹೀಗೆ ಆ ಮಣ್ಣಿಕೊಪ್ಪ ನಮ್ಮ ಕುಟುಂಬದ ಶಾಖಾ ಗೃಹವಾಗಿ ಸಕ್ರಿಯವಾಯಿತು. ಅವಿಭಕ್ತ ಕುಟುಂಬವಾದ್ದರಿಂದ ಮನೆಯ ಮಕ್ಕಳೆಲ್ಲ ತಮಗೆ ಇಷ್ಟವಾದಲ್ಲಿ ಇರಲು, ಹೋಗಿ–ಬಂದು ಮಾಡಲು ಯಾವ ನಿರ್ಬಂಧವಿರಲಿಲ್ಲ.  ನಮ್ಮ ದೊಡ್ಡಮ್ಮ (ಶ್ರೀಮತಿ ರೇವತಿ) ಮಕ್ಕಳನ್ನು ಪ್ರೀತ್ಯಾದರಗಳಿಂದ ಕಾಣುವವಳಾಗಿದ್ದು, ಆಗಾಗ ನಾವು ಮಣ್ಣಿಕೊಪ್ಪದ ಮನೆಯಲ್ಲಿ ಬಂದು ನೆಲೆಸುತ್ತಿದ್ದೆವು. ಮಣ್ಣಿಕೊಪ್ಪ ಬೇಸಿಗೆಯಲ್ಲಿ ನಮ್ಮನ್ನು ಹೆಚ್ಚಾಗಿ ಆಕರ್ಷಿಸುತ್ತಿತ್ತು. ಅಲ್ಲಿ ಬೇಸಿಗೆಯಲ್ಲಿ ಭತ್ತದ ಫಸಲನ್ನು ಬೆಳೆಯಲು ಹೊಳೆಗೆ ವಡ್ಡನ್ನು ನಿರ್ಮಿಸುತ್ತಿದ್ದರು. ಆ ಕಟ್ಟಿನ (ವಡ್ಡಿನ) ಗುಂಡಿಯ ನೀರಿನಲ್ಲಿ ಆಟವಾಡುವುದು ನಮಗೆ ಪರಮಾನಂದ ನೀಡುವುದಾಗಿತ್ತು.

ಒಮ್ಮೆ ಈ ವಡ್ಡಿನ ನೀರಿನಲ್ಲಿ ನಾವೆಲ್ಲ ಆಟವಾಡುತ್ತಿದ್ದಾಗ, ಈಜು ಬಾರದ ನಾನು, ಆಕಸ್ಮಿಕವಾಗಿ ನೀರಿನಾಳದಲ್ಲಿ ಸಿಲುಕಿ ಮುಳುಗುವ ಸ್ಥಿತಿಯಲ್ಲಿದ್ದೆ. ಆಗ, ಅಲ್ಲೆ ಸಮೀಪದಲ್ಲಿ ಈಜುತ್ತಿದ್ದ ಗಂಗಣ್ಣ ನನ್ನನ್ನು ಮೇಲೆತ್ತಿದ್ದು, ಇಂದೂ ನೆನಪಿನಾಳದಿಂದ ಮಾಸಿಲ್ಲ. ಅಂದು ಪ್ರಾರಂಭವಾದ ಅವನ ಮೇಲೆತ್ತುವ ಕ್ರಿಯೆ ಇಂದೂ ನನ್ನ ಪಾಲಿಗೆ ಮುಂದುವರೆದಿದೆ.

ಕುಮಟದ ಬೆತ್ತಗೇರಿಯ ಶ್ರೀ ಕೃಷ್ಣಾನಂದ ಭಟ್ಟರು ಕವಲಕೊಪ್ಪ ಪ್ರಾಥಮಿಕ ಶಾಲೆಯಲ್ಲಿ ನಮ್ಮ ನೆಚ್ಚಿನ ಗುರುಗಳು. ವಿವಾಹ ಪೂರ್ವದಲ್ಲಿ ನಮ್ಮ ಮನೆಯ ಸದಸ್ಯರಾಗಿ ನೆಲೆಸಿ, ನಮ್ಮ ಮೇಲೆ ಪ್ರಭಾವ ಬೀರಿದ ಸಹೃದಯರು. ಬೇಸಿಗೆಯ ರಜೆಯಲ್ಲಿ ಒಮ್ಮೆ ಅವರ ಮನೆಗೆ (ಬೆತ್ತಗೇರಿಗೆ) ನಾವಿಬ್ಬರು ತೆರಳಿದೆವು. ಅವರದೂ ಅವಿಭಕ್ತ ವೈದಿಕ ಕುಟುಂಬ. ಶ್ರೀಯುತರ ತಂದೆ ವೇ। ವಿಘ್ನೇಶ್ವರ ಭಟ್ಟರು ಹಾಗೂ ಅಣ್ಣ ವೇ। ವಿನಾಯಕ ಭಟ್ಟರು ಉತ್ತಮ ಪುರೋಹಿತರು. ಆ ಬೇಸಿಗೆಯ ರಜಾ ಅವಧಿಯಲ್ಲಿ ನಾವು ವೇ। ವಿನಾಯಕ ಭಟ್ಟರಿಂದ ಗಣಪತಿ ಉಪನಿಷತ್ತನ್ನು ಕಲಿತೆವು. ಹೀಗೆ ಅಂದು ಪ್ರಾರಂಭವಾದ ಅವರ ಕುಟುಂಬದೊಂದಿಗಿನ ನಮ್ಮ ಗುರು–ಶಿಷ್ಯ ಭಾವ ಇಂದೂ ಮುಂದುವರೆದಿದೆ. 

ಏಳನೆಯ ತರಗತಿಯನ್ನು ಮುಗಿಸಿದ ಗಂಗಣ್ಣ ಮನೆಯಿಂದ ಸುಮಾರು ಏಳು ಕಿಲೋಮೀಟರ್ ದೂರದಲ್ಲಿರುವ ಬಿದ್ರಕಾನ ಪ್ರೌಢಶಾಲೆಗೆ ಪ್ರತಿನಿತ್ಯ ಕಾಲ್ನಡಿಗೆಯಲ್ಲೆ ತೆರಳುತ್ತಿದ್ದ. ಅಲ್ಲಿಯೂ ತನ್ನ ಪ್ರತಿಭೆ ಹಾಗೂ ಸದ್ಗುಣಗಳಿಂದ ಎಲ್ಲರ ನೆಚ್ಚಿನ ವಿದ್ಯಾರ್ಥಿಯಾಗಿದ್ದ.

ಪ್ರೌಢಶಿಕ್ಷಣಾನಂತರ, ತಂದೆಯವರ ಆಗ್ರಹದಂತೆ ಕುಲಪರಂಪರೆಯ ವೈದಿಕ ಹಾಗೂ ಸಂಸ್ಕೃತ ಶಿಕ್ಷಣವನ್ನು ಕೆಲಕಾಲ ವಿದ್ವಾನ್ ಸೂರಿ ರಾಮಚಂದ್ರ ಶಾಸ್ತ್ರಿಯವರ ಬಳಿಯಲ್ಲೂ ತದನಂತರ ಗೋಕರ್ಣದ ವೇ। ದತ್ತಾತ್ರೇಯ ಭಟ್ಟರಲ್ಲಿಯೂ ಪಡೆದ.

ಅಷ್ಟೊತ್ತಿಗಾಗಲೆ ನಾನು ನನ್ನ ಪ್ರಾಥಮಿಕ ಶಿಕ್ಷಣವನ್ನು ಪೂರೈಸಿ, ಎಂಟು ಮತ್ತು ಒಂಭತ್ತನೆ ತರಗತಿಯ ಶಿಕ್ಷಣವನ್ನು ಬೆತ್ತಗೇರಿಯ ಶ್ರೀ ವಿಘ್ನೇಶ್ವರ ಭಟ್ಟರ ಪೋಷಣೆಯಲ್ಲಿ ಧಾರೇಶ್ವರ ಹೈಸ್ಕೂಲಿನಲ್ಲಿ ಮುಗಿಸಿದ್ದೆ. ಆಗತಾನೆ ಮಂಜುನಾಥ ಅಣ್ಣ ತನ್ನ ಸಂಸ್ಕೃತ– ಶಾಸ್ತ್ರಾಧ್ಯಯನವನ್ನು ಮೈಸೂರಿನಲ್ಲಿ ಮುಗಿಸಿ, ಮನೆಗೆ ಮರಳಿದ್ದರು. ಭವಿತವ್ಯದ ದೃಷ್ಟಿಯಿಂದ ಅಧ್ಯಯನಕ್ಕೆ ಮೈಸೂರು ನಮ್ಮ ಕುಟುಂಬದವರಿಗೆ ಯೋಗ್ಯವಾದ ಸ್ಥಳ.

ಪಾರಂಪರಿಕ ಹಾಗೂ ಆಧುನಿಕ ಶಿಕ್ಷಣಗಳೆರಡನ್ನೂ ಇಲ್ಲಿ ನಡೆಸಬಹುದೆಂಬುದು ಅವರ ದೂರದೃಷ್ಟಿಯ ಚಿಂತನೆಯಾಗಿತ್ತು. ಅವರ ಪ್ರೇರಣೆ ಹಾಗೂ ಮಾರ್ಗದರ್ಶನದಂತೆ ನಾವಿಬ್ಬರೂ (ಗಂಗಣ್ಣ 1974 ಜೂನ್ ತಿಂಗಳಿನಲ್ಲೂ, ನಾನು 1975 ಜೂನ್ ತಿಂಗಳಿನಲ್ಲೂ) ಮೈಸೂರಿಗೆ ಬಂದು ಮಹಾರಾಜ ಸಂಸ್ಕೃತ ಮಹಾಪಾಠಶಾಲೆಯ ವಿದ್ಯಾರ್ಥಿನಿಲಯದ ಒಂಭತ್ತನೆ ಸಂಖ್ಯೆಯ ಕೊಠಡಿಯಲ್ಲಿ ತಂಗಿದ್ದೆವು. ಪಾಠಶಾಲೆಯಲ್ಲಿ ಕೃಷ್ಣ ಯಜುರ್ವೇದ ತರಗತಿಗೆ ಪ್ರವೇಶ ಪಡೆದ ನಾವಿಬ್ಬರೂ, ವೇ। ವೆಂಕಟರಮಣ ಶಾಸ್ತ್ರಿಗಳಲ್ಲಿ ‘ಚಿತ್ತಿ’ ಮುಂತಾದ ಸೂಕ್ತಗಳನ್ನು ಅಧ್ಯಯನ ಮಾಡಿದೆವು. ದಿನದ ಮಧ್ಯಂತರದ ವೇಳೆಯಲ್ಲಿ ನಾನು ದಳವಾಯಿ ಪ್ರೌಢಶಾಲೆಯಲ್ಲಿ ಹತ್ತನೆ ತರಗತಿಯಲ್ಲಿ, ಗಂಗಣ್ಣ ಮರಿಮಲ್ಲಪ್ಪ ಕಾಲೇಜಿನ ವಾಣಿಜ್ಯ ವಿಭಾಗದಲ್ಲಿ ಪಿ.ಯು.ಸಿಗೂ ಪ್ರವೇಶ ಪಡೆದು ಅಧ್ಯಯನ ಮುಂದುವರೆಸಿದೆವು.

ಪ್ರತಿಭಾನ್ವಿತನೂ, ಮಾತುಗಾರನೂ ಆಗಿದ್ದ ಗಂಗಣ್ಣನಿಗೆ ಮರಿಮಲ್ಲಪ್ಪ ಕಾಲೇಜಿನಲ್ಲಿ ಯುಕ್ತವಾದ ವೇದಿಕೆ ಲಭ್ಯವಾಯಿತು. ಚರ್ಚೆ–ಲೇಖನ–ಭಾಷಣ ಮುಂತಾದ ಸ್ಪರ್ಧೆಗಳಲ್ಲಿ ಮರಿಮಲ್ಲಪ್ಪ ಕಾಲೇಜನ್ನು ಅಂತರ ಕಾಲೇಜು ಸ್ತರದಲ್ಲೂ, ಪ್ರತಿನಿಧಿಸುತ್ತಿದ್ದ. ಅಲ್ಲದೆ ಅಂತರ ವಿದ್ಯಾರ್ಥಿನಿಲಯ ಸ್ಪರ್ಧೆಗಳಲ್ಲಿ ಸಂಸ್ಕೃತ ಪಾಠಶಾಲೆಯನ್ನೂ ಪ್ರತಿನಿಧಿಸಿ ಯಶಸ್ವಿಯಾಗುತ್ತಿದ್ದ. 

ಈ ಮಧ್ಯೆ ಎಸ್ಸೆಸ್ಸೆಲ್ಸಿಯನ್ನು ಮುಗಿಸಿದ ನಾನು, ನನ್ನ ತಂದೆಯವರ ಆಗ್ರಹದ ಮೇರೆಗೆ ಒಂದು ವರ್ಷ ಕೋಟೆಮನೆ ಪಾಠಶಾಲೆಯಲ್ಲಿ ಸೂಕ್ತ–ರುದ್ರ–ಚಮಕಾದಿಗಳ ಹಾಗೂ ಸಂಸ್ಕೃತ (ಕಾವ್ಯ) ಅಧ್ಯಯನ ಮುಗಿಸಿ ಮರುವರ್ಷ (1977) ಪುನಃ ಮೈಸೂರಿಗೆ ಬಂದೆ. ಆಗ ಸಂಸ್ಕೃತ ಪಾಠಶಾಲೆಯ ವಿದ್ಯಾರ್ಥಿನಿಲಯದಲ್ಲಿ ನಮ್ಮ ವಾಸಸ್ಥಾನ ಇಪ್ಪತ್ತೇಳನೆ ಕೊಠಡಿಯಾಗಿತ್ತು. ಪಿ.ಯು.ಸಿಯಲ್ಲಿ ಉತ್ತಮ ಶ್ರೇಣಿಯಲ್ಲಿ ಪಾಸಾಗಿದ್ದ ಗಂಗಣ್ಣ ಮಹಾರಾಜ ಕಾಲೇಜಿನಲ್ಲಿ ಆಗತಾನೆ ಪ್ರಾರಂಭವಾಗಿದ್ದ, ಬಿ.ಬಿ.ಎಂ ತರಗತಿಗೆ ಸೇರಿಕೊಂಡಿದ್ದ. ಆ ಕಾಲೇಜಿನಲ್ಲಿ ಗಂಗಣ್ಣನ ಪ್ರತಿಭೆಗೆ ತಕ್ಕಂತಹ ಪ್ರೋತ್ಸಾಹ ದೊರೆಯಲಿಲ್ಲ. ಮತ್ಸರದ ಅನ್ಯಾಯಕ್ಕೆ ಬಲಿಯಾಗಿ ಮನನೊಂದು ಆ ಶಿಕ್ಷಣವನ್ನು ಅಲ್ಲಿಗೆ ಮೊಟಕುಗೊಳಿಸಿ, ಮುಂದಿನ ವರ್ಷ (1977) ಬನುಮಯ್ಯ ಕಾಲೇಜಿನಲ್ಲಿ ಬಿ.ಕಾಂ ಪದವಿ ತರಗತಿಯಲ್ಲಿ ಮುಂದುವರಿದ. ನಾನು ಶಾರದಾ ವಿಲಾಸ ಕಾಲೇಜಿನ ಪಿ.ಯು.ಸಿ ತರಗತಿಗೆ ಸೇರಿಕೊಂಡೆ. ಇತ್ತ ಸಂಸ್ಕೃತ ಶಿಕ್ಷಣದಲ್ಲಿ ಸಾಹಿತ್ಯ ತರಗತಿಯ ಅಧ್ಯಯನ ಜೊತೆಜೊತೆಗೆ ಸಾಗುತ್ತಿದ್ದವು. 

ನಮ್ಮಿಬ್ಬರ ಈ ವಿದ್ಯಾರ್ಥಿ ಜೀವನ ಸಂತೋಷದಾಯಕವೇನೂ ಆಗಿರಲಿಲ್ಲ. 1971ರ ಹೊತ್ತಿಗೆ ಆರ್ಥಿಕ ಹೊರೆಯನ್ನು ಸರಿದೂಗಿಸಲಾಗದೆ, ಕುಟುಂಬ	ವಿಭಕ್ತವಾಗಿತ್ತು. ಆದರೂ ಅವಿಭಕ್ತ ಕುಟುಂಬದ ಪ್ರೀತ್ಯಾದರಗಳೂ, ಪರಸ್ಪರರ ವಿಶ್ವಾಸ ಯಥಾವತ್ತಾಗಿ ಮುಂದುವರಿದುದು ಅಭಿಮಾನಾಸ್ಪದ ಸಂಗತಿ. ಆರ್ಥಿಕ ಸಂಕಷ್ಟದ ಕಾರಣದಿಂದ ಕುಟುಂಬದಿಂದ ನಮ್ಮ ಶಿಕ್ಷಣಕ್ಕೆ ಯಾವ ನೆರವನ್ನೂ ನಿರೀಕ್ಷಿಸುವಂತಿರಲಿಲ್ಲ.

ಅಧ್ಯಯನ ಮುಂದುವರೆಸುವ ಛಲದ ಹಿಂದೆ, ಹಸಿವು, ಅನ್ನಕ್ಕಾಗಿ ಯಾಚನೆ, ವಾರಾನ್ನದ ಅನ್ವೇಷಣೆ, ಅಪಮಾನ, ಪರಿಕ್ಷಾ ಶುಲ್ಕಾದಿಗಳಿಗೆ ಪರದಾಟ, ಇವೆಲ್ಲವನ್ನೂ ಎದುರಿಸಿ ಭವಿಷ್ಯವನ್ನು ಕಟ್ಟಿಕೊಳ್ಳಲು ಹೆಣಗುವ ಅವಿರತ ಹೋರಾಟ. ಇದು ನಮಗಷ್ಟೆ ಸೀಮಿತವಾಗಿರದೇ ಸಂಸ್ಕೃತ ಅಧ್ಯಯನಕ್ಕಾಗಿ ಬಂದ ಬಹುತೇಕರ ಪಾಡಾಗಿತ್ತು. ಅನೇಕ ಬಾರಿ ವಿದ್ಯಾರ್ಥಿನಿಲಯದಲ್ಲಿ ಸಹಪಾಠಿಗಳು, ಸಹವಾಸಿಗಳು ಪರಸ್ಪರ ಪರಿಹಾಸದಲ್ಲಿ ತೊಡಗಿ, ನಕ್ಕು ಸಾಂತ್ವನಗೊಂಡು, ಇಂತಹ ಸ್ಥಿತಿಯನ್ನು ನಿಭಾಯಿಸುತ್ತಿದ್ದೆವು. ಉಮಾಕಾಂತ ಭಟ್ಟ, ಕಾಗೇರಿ ಶಿವರಾಮ ಹೆಗಡೆ, ವಿಶ್ವನಾಥ ಅಗ್ನಿಹೋತ್ರಿ, ಕಿಗ್ಗ ರಾಮಚಂದ್ರ ಜೋಯ್ಸ ಹಾಗೂ ಸಹೋದರರು ಮುಂತಾದವರು ಆಗ ನಮ್ಮ ಒಡನಾಡಿಗಳಾಗಿದ್ದರು. ವೆಂಕಟರಮಣ ಹೆಗಡೆ ಕಲಗಾರ, ದೇವಿಸರ ಗಣಪತಿ ಭಟ್ಟರು, ಶ್ರೀರಾಮ ಭಟ್ಟರು, ಹಿರೆಮನೆ ಮಂಜುನಾಥ ಭಟ್ಟರು ಮುಂತಾದವರು ನಮ್ಮ ಹಿತೈಷಿಗಳಾಗಿದ್ದರು.

ಇದೇ ಸ್ಥಿತಿಯಲ್ಲಿ ನಮ್ಮ ಪಿ.ಯು.ಸಿ ಮತ್ತು ಪದವಿ ಶಿಕ್ಷಣ ಸಮಾಪ್ತವಾಗಿತ್ತು. ಬದುಕಿನ ಅಗತ್ಯಕ್ಕೆ ಏನಾದರೂ ಒಂದು  ಉದ್ಯೋಗವನ್ನು ಅವಲಂಬಿಸುವುದು ಅನಿವಾರ್ಯವಾಗಿತ್ತು. ಆ ಸಮಯದಲ್ಲಿ ಮೀಮಾಂಸಾಶಾಸ್ತ್ರದ ಅಧ್ಯಯನ ಮಾಡುತ್ತಿದ್ದ, ಶ್ರೀ ರಮಾನಂದ ಅವಭೃಥರು ಸಂಸ್ಕೃತ ಪಾಠಶಾಲೆಯ ಸಮೀಪದ ಶಂಕರಮಠ ರಸ್ತೆಯ ಎರಡನೆ ತಿರುವಿನಲ್ಲಿರುವ ಶಂಕರವಿಲಾಸ ಸಂಸ್ಕೃತ ಪಾಠಶಾಲೆಯಲ್ಲಿ ಸಂಜೆಯ ವೇಳೆಯಲ್ಲಿ ಅಧ್ಯಾಪಕರಾಗಿ ಕಾರ್ಯ ನಿರ್ವಹಿಸುತ್ತಿದ್ದರು. ಆ ಪಾಠಶಾಲೆಯಲ್ಲಿ ಮುಖ್ಯೋಪಾಧ್ಯಾಯರ ಹುದ್ದೆ ಖಾಲಿಯಾಗಿತ್ತು. 

ಗಂಗಣ್ಣನ ವ್ಯವಹಾರ ಕೌಶಲ ಮುಂತಾದ ಅರ್ಹತೆಯನ್ನು ಮನಗಂಡಿದ್ದ ಶ್ರೀಯುತ ಅವಭೃಥರು ಅವನನ್ನು ಆ ಹುದ್ದೆಗೆ ಸೂಚಿಸಿದರು. ಅವಭೃಥರ ಸೂಚನೆಯನ್ನು ಅಂಗೀಕರಿಸಿದ ಆಡಳಿತ ಮಂಡಳಿ ಗಂಗಣ್ಣನನ್ನು ಪಾಠಶಾಲೆಯ ಮುಖ್ಯೋಪಾಧ್ಯಾಯನನ್ನಾಗಿ ಆಯ್ಕೆ ಮಾಡಿಕೊಂಡಿತು. ಬರುತ್ತಿದ್ದ ಅಲ್ಪ ಆದಾಯ, ಸ್ವಲ್ಪಮಟ್ಟಿನ ನಿರಾಳತೆಯನ್ನುಂಟುಮಾಡಿತ್ತು.

ಅಷ್ಟೊತ್ತಿಗಾಗಲೆ ಗಂಗಣ್ಣನ ತಂಗಿ ಹೇಮಾವತಿ ತನ್ನ ಮೆಟ್ರಿಕ್ ಅಧ್ಯಯನವನ್ನು ಮುಗಿಸಿದ್ದಳು. ಓದನ್ನು ಮುಂದುವರಿಸಲು  ಅವಳನ್ನು ಗಂಗಣ್ಣ ಮೈಸೂರಿಗೆ ಕರೆತಂದ. ವಿದ್ಯಾರ್ಥಿನಿಲಯವನ್ನು ಬಿಟ್ಟು ಪ್ರತ್ಯೇಕ ಬಾಡಿಗೆ ಕೊಠಡಿಯಲ್ಲಿ ವಾಸಿಸತೊಡಗಿದ. ಎರಡು ವರ್ಷಗಳ ನಂತರ ಮತ್ತೊಬ್ಬ ತಂಗಿ ರತ್ನಾವತಿ ಕೂಡ ಅಧ್ಯಯನಕ್ಕಾಗಿ ಇವರನ್ನು ಸೇರಿಕೊಂಡಳು. ಮಾಡುತ್ತಿದ್ದ ಉದ್ಯೋಗದಿಂದ ನಿರ್ದಿಷ್ಟ ಆದಾಯ ಅನಿಯತವಾಗಿಯಾದರೂ ಬರುತ್ತಿದ್ದುದರಿಂದ, ಖರ್ಚುಗಳೆಲ್ಲ ಹೇಗೋ ನಿಭಾಯಿಸಲ್ಪಡುತ್ತಿದ್ದವು.

1984ರ ವೇಳೆಗಾಗಲೆ ಶ್ರೀ ರಮಾನಂದ ಅವಭೃಥರು ತಮ್ಮ ಶಾಸ್ತ್ರಾಧ್ಯಯನವನ್ನು ಮುಗಿಸಿ, ತಮ್ಮ ಊರಿಗೆ ತೆರಳಿದಾಗ ಶಂಕರವಿಲಾಸ ಸಂಸ್ಕೃತ ಪಾಠಶಾಲೆಯಲ್ಲಿ ಸಹಾಧ್ಯಾಪಕರ ಹುದ್ದೆ ರಿಕ್ತವಾಯಿತು. ಜಪದಕಟ್ಟೆ ಮಠದ ಅಂಗಸಂಸ್ಥೆಯಾಗಿದ್ದ ಈ ಸಂಸ್ಕೃತ ಪಾಠಶಾಲೆಯ ಆಗಿನ ಅಧ್ಯಕ್ಷರಾಗಿದ್ದವರು ಪೂಜ್ಯ ಶ್ರೀ ಶ್ರೀ ಮರುಳಾರಾಧ್ಯ ಶಿವಾಚಾರ್ಯ ಮಹಾಸ್ವಾಮಿಗಳು. ತನ್ನ ಕಾರ್ಯ ನಿಷ್ಠೆಯಿಂದ ಸ್ವಾಮಿಗಳಿಗೆ ವಿಶ್ವಾಸಿಕನಾಗಿದ್ದ ಗಂಗಣ್ಣ ರಿಕ್ತ ಹುದ್ದೆಗೆ ನನ್ನನ್ನೇ ಸೂಚಿಸಿದ. ಸ್ವಾಮಿಗಳು ಸ್ವತಃ ನನ್ನ ಸಂದರ್ಶನ ಮಾಡಿ ಕೆಲಸಕ್ಕೆ ಹಾಜರಾಗುವಂತೆ ಆದೇಶಿಸಿದರು. ಈ ಹುದ್ದೆಗೆ ಅನೇಕರು ಆಕಾಂಕ್ಷಿಗಳಾಗಿದ್ದರೂ, ನನ್ನ ಆರ್ಥಿಕ ಸ್ಥಿತಿ ಹಾಗೂ ಆದಾಯದ ಅನಿವಾರ್ಯತೆಯ ಅರಿವಿದ್ದ ಅವನಿಗೆ ಆ ನಿರ್ಧಾರ ತಪ್ಪಲ್ಲವೆಂದೆನಿಸಿರಬೇಕು. ಅವನ ಈ ನಿರ್ಧಾರ ನನ್ನ ಜೀವನ ನೆಲೆನಿಲ್ಲಲು ಅವಕಾಶ ಮಾಡಿತು.

ಆ ವೇಳೆಗಾಗಲೆ ನಮ್ಮ ಜೊತೆಗಿದ್ದ ನನ್ನ ತಮ್ಮ ನಾಗರಾಜ ಜ್ಯೌತಿಷ ಶಾಸ್ತ್ರದ ಪೂರ್ವಭಾಗದ ಅಧ್ಯಯನವನ್ನು ಮುಗಿಸಿ ಊರಿಗೆ ತೆರೆಳಿದ್ದ. ತಂಗಿ ಜಯಂತಿ ಊರಿನಲ್ಲಿ ಮೆಟ್ರಿಕ್ ಶಿಕ್ಷಣವನ್ನು ಮುಗಿಸಿದ್ದಳು.ಅವಳನ್ನು ಮುಂದಿನ ಶಿಕ್ಷಣಕ್ಕಾಗಿ ಮೈಸೂರಿಗೆ ಕರೆಸಿಕೊಂಡೆ. ಅಷ್ಟು ಹೊತ್ತಿಗೆ ಗಂಗಣ್ಣ ತನ್ನ ವಸತಿಯನ್ನು ಶಾಂತಲ ಟಾಕೀಸಿನ ಹತ್ತಿರದ ವೆಂಕಟರಮಣಸ್ವಾಮಿ ದೇವಾಲಯ ರಸ್ತೆಯ ಸುಂದರಮೂರ್ತಿ–ಸಾವಿತ್ರಮ್ಮ ಇವರ ಮನೆಯಿಂದ, ಅಗ್ರಹಾರದಲ್ಲಿರುವ ಲಾಯರ್ ಕೃಷ್ಣಸ್ವಾಮಿಯವರ ಮನೆಗೆ ಬದಲಾಯಿಸಿದ್ದ. ಮೊದಲನೆ ಮಹಡಿಯಲ್ಲಿ ಎರಡು ವಿಶಾಲ ಕೊಠಡಿಗಳು. ಅಲ್ಲಿ ಗಂಗಣ್ಣ ಅವನ ಇಬ್ಬರು ತಂಗಿಯರು, ನಾನು ಮತ್ತು ನನ್ನ ತಂಗಿ ಜಯಂತಿ ಹೀಗೆ ನಮ್ಮ ಅವಿಭಕ್ತ ಕುಟುಂಬ ಸಾಗಿತ್ತು. ಆ ಹೊತ್ತಿಗಾಗಲೇ ಸಂಸ್ಕೃತ ಪಾಠಶಾಲೆ ಅನುದಾನಿತ ಶಾಲೆಗಳ ಪಟ್ಟಿಯಲ್ಲಿ ಸೇರಿ ಸ್ವಲ್ಪಮಟ್ಟಿನ ನೆಮ್ಮದಿಯನ್ನು ತಂದಿತ್ತು.

1984 ರಿಂದ 1986 – ಇದು ನನ್ನ ಸಂಸ್ಕೃತ ಎಂ.ಎ ಅಧ್ಯಯನದ ಅವಧಿ. ಅದೇ ಸಮಯದಲ್ಲಿ ಹತ್ತಿರದ ಸಂಬಂಧಿಗಳಾದ ಡಾ.ಎಸ್.ಜಿ ಗಂಗೊಳ್ಳಿಯವರ ಆಗ್ರಹಕ್ಕೆ ಮಣಿದು, ಅವರು ಗಂಗೋತ್ರಿ ಲೇಔಟ್‍ನಲ್ಲಿ ನೂತನವಾಗಿ ನಿರ್ಮಿಸಿದ ಮನೆಯ ಮಹಡಿಯ ಕೊಠಡಿಯಲ್ಲಿ ತಂಗಿ ಜಯಂತಿಯೊಡನೆ (1985) ವಾಸಿಸತೊಡಗಿದೆ. ಎಂ.ಎ ಅಂತಿಮ ವರ್ಷದಲ್ಲಿ ನಮಗೆ ದಕ್ಷಿಣ  ಭಾರತದ ಶೈಕ್ಷಣಿಕ ಪ್ರವಾಸ ಮೈಸೂರು ವಿಶ್ವವಿದ್ಯಾನಿಲಯದಿಂದ ಏರ್ಪಾಡಾಗಿತ್ತು. ಪ್ರವಾಸಕ್ಕೆ ಹೊರಡುವ ಅಂತಿಮ ಸಿದ್ಧತೆಗಳೆಲ್ಲ ಪೂರ್ಣಗೊಂಡಿದ್ದವು.

ಈ ವಿಷಯವನ್ನು ಗಂಗೊಳ್ಳಿಯವರಿಗೆ ತಿಳಿಸಿದೆ.  ಹಿಂದಿರುಗುವುದು ಯಾವಾಗ ಎಂದು ಪ್ರಶ್ನಿಸಿದ ಅವರು, ಮನೆಯ ಮೊದಲ ಮಹಡಿಯ ಕಟ್ಟಡವನ್ನು ಮುಂದುವರಿಸಬೇಕಾದ್ದರಿಂದ, ಕೊಠಡಿಯಲ್ಲಿರುವ ನಮ್ಮ ಸಾಮಾನನ್ನೆಲ್ಲ ತಮ್ಮ ಮನೆಯಲ್ಲಿ ಇರಿಸಿ ಹೋಗುವಂತೆ ಸೂಚಿಸಿದರು. 

ತಂಗಿಯೂ ಜೊತೆಯಲ್ಲಿ ಇದ್ದಿದ್ದರಿಂದ ಏನು ಮಾಡಬೇಕೆಂಬುದು ತೋಚದೆ ಈ ಸಮಸ್ಯೆಯನ್ನು ಗಂಗಣ್ಣನಲ್ಲಿ ಹೇಳಿಕೊಂಡೆ. ಕ್ಷಣವೂ ಯೋಚಿಸದೇ ‘ನಿನ್ನ ವಸ್ತುಗಳನ್ನೆಲ್ಲ ಇಂದೇ ನಮ್ಮ ಮನೆಗೆ ಸಾಗಿಸು, ತಂಗಿ ಜಯಂತಿ ನಮ್ಮೊಂದಿಗೆ ಇರುತ್ತಾಳೆ’ ಎಂದ. ಸೂರ್ಯೋದಯವಾದಾಗ ಮಂಜು ಮಾಯವಾಗುವಂತೆ ಬಂದ ಸಮಸ್ಯೆ ದೂರವಾಗಿತ್ತು. 

ಮುಂದೆ ವಿವಾಹವಾಗುವವರೆಗೂ ನಮ್ಮ ಅವಿಭಕ್ತ ಕುಟುಂಬ ನಿರಾತಂಕವಾಗಿ ಒಟ್ಟಿಗೆ ಸಾಗಿತು. \textbf{ಆಪದ್ಗತಂ ಚ ನ ಜಹಾತಿ.}...... ಇದೇ ಅಲ್ಲವೇ ಸನ್ಮಿತ್ರನ ಸ್ವಭಾವ ! 

ಸಂಸ್ಕೃತ ಸಾಹಿತ್ಯ ತರಗತಿಯ ಓದನ್ನು ಮುಗಿಸಿ, ಮುಂದೆ ಅಧ್ಯಯನಕ್ಕೆ ಯಾವ ಶಾಸ್ತ್ರವನ್ನು ಆಯ್ದುಕೊಳ್ಳಬೇಕೆಂಬ ಸಂದಿಗ್ಧ ಪರಿಸ್ಥಿತಿಯಲ್ಲಿ  ಮೀಮಾಂಸಾ ಶಾಸ್ತ್ರವನ್ನು  ಅಯ್ಕೆಮಾಡಿಕೊಳ್ಳುವಂತೆ ಸೂಚಿಸಿದವನೂ ಗಂಗಣ್ಣನೇ. ಏಕೆಂದರೆ ಆ ಶಾಸ್ತ್ರ ವಿಭಾಗದಲ್ಲಿ ಪಂಡಿತರತ್ನಂ ಈ.ಶ.ವರದಾಚಾರ್ಯರು ಹಾಗೂ ವಿದ್ವಾನ್ ಗಣೇಶ ಭಟ್ಟರು ಅಧ್ಯಾಪಕರಾಗಿದ್ದು, ಪಾಠ ಪ್ರವಚನಗಳು ಸರಿಯಾಗಿ ನಡೆಯುತ್ತಿದ್ದವು.

ಆ ಹೊತ್ತಿಗಾಗಲೇ ಗಂಗಣ್ಣ ನವೀನನ್ಯಾಯ ಶಾಸ್ತ್ರವನ್ನು ಅಧ್ಯಯನಮಾಡುತ್ತಿದ್ದ. ಪಂಡಿತೋತ್ತಮರಾದ ರಾಮಭದ್ರಾಚಾರ್ಯರು, ಶಿಷ್ಯವತ್ಸಲರಾದ ವೆಂಕಣ್ಣಾಚಾರ್ಯರು ತರ್ಕಶಾಸ್ತ್ರ ವಿಭಾಗದ ಪ್ರಾಧ್ಯಾಪಕರಾಗಿದ್ದರು. ಗಂಗಣ್ಣ ಇವರೀರ್ವರಿಗೂ ನೆಚ್ಚಿನ ಶಿಷ್ಯನಾಗಿದ್ದ. ರಾಮಭದ್ರಾಚಾರ್ಯರ ಪಾಠ–ಪ್ರವಚನಗಳು ಗಂಗಣ್ಣನ ಮೇಲೆ ಅತ್ಯಂತ ಪ್ರಭಾವ ಬೀರಿವೆ. ಶಾಸ್ತ್ರದ ಗತ್ತು ಹಾಗೂ ರಹಸ್ಯಗಳನ್ನು ರಾಮಭದ್ರಾಚಾರ್ಯರೇ ಇವನಿಗೆ ಧಾರೆ ಎರೆದಿದ್ದಾರೆ.

ಅನ್ಯಾಯವನ್ನು ಪ್ರತಿಭಟಿಸುವುದು, ನ್ಯಾಯಕ್ಕಾಗಿ ಹೋರಾಡುವುದು ಗಂಗಣ್ಣನ ಜನ್ಮಜಾತ ಗುಣವಾಗಿದೆ. ಸಂಸ್ಕೃತ ಕಾಲೇಜಿನ ವಿದ್ಯಾರ್ಥಿನಿಲಯದಲ್ಲಿ ವೇದಶಾಸ್ತ್ರಗಳನ್ನು ಶ್ರದ್ಧೆಯಿಂದ ವ್ಯಾಸಂಗ ಮಾಡುವ ವಿದ್ಯಾರ್ಥಿಗಳ ಗುಂಪು ಒಂದೆಡೆಯಾದರೆ, ಸಂಸ್ಕೃತವನ್ನು ನೆಪವಾಗಿಸಿಕೊಂಡು, ವಸತಿ ಸೌಲಭ್ಯ ಪಡೆದು ಬೇರೆಡೆಗೆ ತೊಡಗಿಕೊಳ್ಳುವವರ ತಂಡ ಇನ್ನೊಂದೆಡೆ. ಇವರೀರ್ವರ ಮಧ್ಯೆ ಆಗಾಗ ವಾಗ್ವಾದ–ಜಗಳಗಳು ನಡೆಯುತ್ತಲೇ ಇರುತ್ತಿದ್ದವು. ಸಂಸ್ಕೃತ ಕಾಲೇಜಿಗೆ ಆಗ (1977) ಶತಮಾನೋತ್ಸವದ ಸಂಭ್ರಮ. ಅಲ್ಲಿ ಅಧ್ಯಯನ ಮಾಡುವ ವಿದ್ಯಾರ್ಥಿಗಳಿಗೆ ಸ್ಪರ್ಧೆ ಮುಂತಾದವುಗಳನ್ನು ಆಯೋಜಿಸುವದರ ಜೊತೆಗೆ ಸಿಹಿಯನ್ನು ವಿತರಿಸುವ ಕಾರ್ಯ ನಡೆದಿತ್ತು. ಆಗ, ಸಂಸ್ಕೃತ ಕಾಲೇಜಿಗೆ ಸಂಬಂಧಪಡದ ಯುವಕರ ಗುಂಪೊಂದು ಸರದಿಯಲ್ಲಿ ಸೇರಿಕೊಂಡು ತಮಗೂ ಸಿಹಿಯನ್ನು ನೀಡಬೇಕೆಂದು ಆಗ್ರಹಿಸುತ್ತಿತ್ತು.

ವಿದ್ಯಾರ್ಥಿಗಳಿಗೆ ಮೀಸಲಾದ ಸಿಹಿಯನ್ನು ಅನ್ಯರಿಗೆ ವಿತರಿಸಬಾರದೆಂದು ವಿದ್ಯಾರ್ಥಿಯಾದ ಶ್ರೀ ರಮೇಶ ಅಡಿಗರು ನಿಷ್ಠುರವಾಗಿ ವಿರೋಧಿಸಿದರು. ತದನಂತರ ಆ ಗುಂಪನ್ನು ಸರದಿಯಿಂದ ಬೇರ್ಪಡಿಸಲಾಯಿತು. ಇದರಿಂದ ವ್ಯಗ್ರವಾದ ಆ ಯುವಕರ ಗುಂಪು ಎಲ್ಲರನ್ನು ನಿಂದಿಸುತ್ತ ಹೊರನಡೆಯಿತು. ವಿದ್ಯಾರ್ಥಿನಿಲಯದ ಕೆಲವರ ಚಿತಾವಣೆಯೊಂದಿಗೆ ಇದೇ ಯುವಕರ ಗುಂಪು ಸಂಜೆಯ ವೇಳೆಗೆ ವಿದ್ಯಾರ್ಥಿನಿಲಯಕ್ಕೆ ನುಗ್ಗಿ ಕೋಲಾಹಲವೆಬ್ಬಿಸಿತು. 

ಇದನ್ನು ವಿರೋಧಿಸಲು ಮುಂದಾದ ವಿದ್ಯಾರ್ಥಿಗಳ ಗುಂಪಿನಲ್ಲಿದ್ದ ಶ್ರೀ ರಮೇಶ ಅಡಿಗರ ಮೇಲೆ, ಪೂರ್ವಾಗ್ರಹ ಪೀಡಿತನಾದ ವಿದ್ಯಾರ್ಥಿಯೊಬ್ಬನು ಸೈಕಲ್ ಚೈನಿನಿಂದ ಹಲ್ಲೆ ನಡೆಸಿದ. ಅಡಿಗರ ಬೆನ್ನಮೇಲೆ ಬಲವಾದ ಗಾಯವಾಯಿತು. ಪರಿಸ್ಥಿತಿ ಕೈಮೀರುತ್ತಿರುವುದನ್ನು ಅರಿತ ಅಲ್ಲೆ ಇದ್ದ ಗಂಗಣ್ಣ ಸಮಯೋಚಿತವಾಗಿ ವಿದ್ಯಾರ್ಥಿನಿಲಯದ ಗೇಟನ್ನು ಮುಚ್ಚಿ ಆ ಪುಂಡರ ಗುಂಪು ಮುಂದುವರೆದು ಬರದಂತೆ ತಡೆದ.

ಪರ ಊರಿನಿಂದ ವಿದ್ಯಾಭ್ಯಾಸಕ್ಕಾಗಿ ಬಂದ ಹಲವು ವಿದ್ಯಾರ್ಥಿಗಳು ಭಯಭೀತರಾಗಿದ್ದರು. ಯಾರೂ ಮುಂದೆ ಬರಲು ಸಿದ್ಧರಿರಲಿಲ್ಲ. ಅಷ್ಟರಲ್ಲಿ ಅಲ್ಲಿಗೆ ಪೋಲೀಸರ ಆಗಮನವಾಯಿತು. 

ಗಂಗಣ್ಣ ಅಡಿಗರ ಬೆಂಬಲಕ್ಕೆ ನಿಂತು ಪೋಲೀಸ್ ಠಾಣೆಗೆ ತೆರಳಿ ಅಡಿಗರ ಮೂಲಕ ಈ ಪುಂಡರ ಹಾಗೂ ಹಲ್ಲೆ ಮಾಡಿದವನ ವಿರುದ್ಧ ದೂರು ದಾಖಲಿಸಿದ. ಅಲ್ಲದೆ ಪೂರಕವಾದ ವೈದ್ಯರ ಪ್ರಮಾಣ ಪತ್ರ, ಶರೀರದ ಮೇಲಾದ ಗಾಯದ ಛಾಯಾಚಿತ್ರ ಮುಂತಾದ ದಾಖಲೆಗಳನ್ನು ಸಿದ್ಧಪಡಿಸಿ, ಈ ಘಟನೆಯ ಬಲವಾದ ಸಾಕ್ಷಿಯಾಗಿ ಮುಂದೆ ನಿಂತ. ಹಲ್ಲೆಮಾಡಿದವನನ್ನು ಬಂಧಿಸಿ ಜಾಮೀನಿನ ಮೇಲೆ ಬಿಡುಗಡೆ ಮಾಡಲಾಯಿತು. ಪ್ರಕರಣ ನ್ಯಾಯಾಲಯದ ಮೆಟ್ಟಿಲನ್ನೇರಿತು. ಆದ್ಯಂತವಾಗಿ ಗಂಗಣ್ಣ ಅಡಿಗರ ಬೆಂಬಲಕ್ಕೆ ಧೃಡವಾಗಿ ನಿಂತು ಹೋರಾಟ ನಡೆಸಿದ. ಎದುರಾಳಿಗಳು ಇದಕ್ಕೆ ಜಾತಿಯ ಬಣ್ಣಕಟ್ಟಲು ಮುಂದಾದರು. ಇದಾವುದಕ್ಕೂ ಜಗ್ಗದಿದ್ದಾಗ ಬಸವಳಿದ ಆರೋಪಿ ತನ್ನ ತಪ್ಪನ್ನು ಒಪ್ಪಿಕೊಂಡು, ತನ್ನನ್ನು ಈ ಪ್ರಕರಣದಿಂದ ಬಿಡುಗಡೆ ಮಾಡುವಂತೆ ಅಂಗಲಾಚಿದ. ಕ್ಷಮೆಯೇ ದೊಡ್ಡ ಗುಣವಲ್ಲವೇ ! ಅಂತಿಮವಾಗಿ ಪ್ರಕರಣವನ್ನು ಹಿಂಪಡೆಯಲಾಯಿತು. ಈ ಘಟನೆ, ಪುಂಡಾಟಿಕೆ ಮಾಡುವವರಿಗೆ ಮುಂದೆ ಎಚ್ಚರಿಕೆಯ ಘಂಟೆಯಾಯಿತು. ಈ ಮೇಲಿನ ಪ್ರಕರಣ ಗಂಗಣ್ಣನ ಧೈರ್ಯವನ್ನು, ಅನ್ಯಾಯದ ವಿರುದ್ಧ ಸೊಲ್ಲೆತ್ತುವ ಧೃತಿಯನ್ನು ಹಾಗೂ ನಿರಪರಾಧಿಗಳಿಗೆ ಬೆಂಬಲವಾಗಿ ನಿಲ್ಲುವ ವ್ಯಕ್ತಿತ್ವವನ್ನು ನಿರೂಪಿಸುತ್ತದೆ.  

ಸಂಸ್ಕೃತ ಶಿಕ್ಷಣ ಕ್ಷೇತ್ರದಲ್ಲಿದ್ದ ಗೊಂದಲವನ್ನು ಹಾಗೂ ಪರೀಕ್ಷಾ ಅಕ್ರಮಗಳನ್ನು ಬಯಲಿಗೆಳೆಯಲು ನಡೆದ ವಿದ್ಯಾರ್ಥಿ ಆಂದೋಲನದಲ್ಲಿ ಗಂಗಣ್ಣನ ಪಾತ್ರ ಪ್ರಧಾನವಾಗಿತ್ತು. ಎಂದೂ ಪಾಠ–ಪ್ರವಚನವನ್ನೇ ಮಾಡದ ಸ್ವಾಮೀಜಿಯೊಬ್ಬರು, ತಮ್ಮ  ಪ್ರಭಾವದಿಂದ ಶಾಸ್ತ್ರ ಪತ್ರಿಕೆಯ ಮೌಲ್ಯಮಾಪಕರಾಗಿ, ಉತ್ತಮ ವಿದ್ಯಾರ್ಥಿಗಳನ್ನು ಅನುತ್ತೀರ್ಣರನ್ನಾಗಿಸುತ್ತಿದ್ದರು. 

ಈ ಅನ್ಯಾಯವನ್ನು ಪ್ರತಿಭಟಿಸಲು ಆರಂಭವಾದ ಆಂದೋಲನದಲ್ಲಿ, ಪರೀಕ್ಷಾ ಫಲಿತಾಂಶ ಪ್ರಕಟಣೆಯಲ್ಲಿ ವಿಳಂಬ, ವಿದ್ವತ್ ಪದವಿಯನ್ನು ಬಿ.ಎ  ಹಾಗೂ  ಎಂ.ಎ ಪದವಿಗಳಿಗೆ ಸಮಾನವೆಂದು ಅಂಗೀಕರಿಸುವುದು, ರಿಕ್ತವಾದ ಅಧ್ಯಾಪಕರ ಸ್ಥಾನವನ್ನು  ಭರ್ತಿ ಮಾಡುವುದು, ಇವೇ ಮೊದಲಾದ ಬೇಡಿಕೆಗಳನ್ನು ಮುಂದಿಟ್ಟು, ನಾವೆಲ್ಲ ಮೈಸೂರಿನ ಪ್ರಮುಖ ಬೀದಿಗಳಲ್ಲಿ ಘೋಷಣೆ ಕೂಗುತ್ತ, ಕರಪತ್ರಗಳನ್ನು ಹಂಚುತ್ತ, ಸಂಚರಿಸಿದ್ದು, ಸಂಸ್ಕೃತ ಶಿಕ್ಷಣದ ಇತಿಹಾಸದಲ್ಲೇ ಪ್ರಥಮ. ಈ ಆಂದೋಲನದಿಂದ ಎಚ್ಚತ್ತ ಕರ್ನಾಟಕ ಪ್ರೌಢಶಿಕ್ಷಣ ಪರೀಕ್ಷಾ ಮಂಡಳಿ ಅನೇಕ ಸುಧಾರಣೆಗಳನ್ನು ತಂದಿತು. ಈ ಆಂದೋಲನದ ರೂವಾರಿ – ಸೂತ್ರಧಾರಿ ಗಂಗಣ್ಣ  ಎಂಬುದು ಮರೆಯಲಾಗದ ವಿಷಯ.

ಮಹಾರಾಜ ಸಂಸ್ಕೃತ ಮಹಾಪಾಠಶಾಲೆಯ ಎಡಪಾರ್ಶ್ವದಲ್ಲಿ ಇದ್ದ ಖಾಲಿಜಾಗವನ್ನು ಅನಧಿಕೃತ ಕಟ್ಟಡ ಕಟ್ಟಿ ಕಬಳಿಸುವ ಹುನ್ನಾರ ಕೆಲವರಿಂದ ನಡೆದಿತ್ತು. ಅದನ್ನು ನಾನು ಮೈಸೂರಿನ ಕೆಲವು  ಪತ್ರಕರ್ತರನ್ನು, ಅಧಿಕಾರಿಗಳನ್ನು ಸಂಪರ್ಕಿಸಿ ವಿರೋಧಿಸಿದ್ದೆ. ಇದಕ್ಕೆ ಕೆಲವರಿಂದ ಬೆದರಿಕೆಯೂ ನನಗೆ ಬಂದಿತ್ತು. ಪಾಠಶಾಲೆಯ ವಿದ್ವಾನ್ ಎ.ಎಂ ಚಂದ್ರಶೇಖರರವರು ಮುಂತಾದ ಅಧ್ಯಾಪಕರಹಾಗೂ ಸಹಪಾಠಿಗಳು ಸಹಕಾರ ನೀಡಿದ್ದರಿಂದ ಆ ಹುನ್ನಾರ ಸಂಪೂರ್ಣವಾಗಿ ನಿಂತಿತು. ಈ ಸಂದರ್ಭದಲ್ಲಿಯೂ ಗಂಗಣ್ಣ ನನಗೆ ನೈತಿಕ ಬಲ ನೀಡಿದ್ದನ್ನ ಮರೆಯಲಾಗದು.

ನಾನು ಮಹಾರಾಜ ಸಂಸ್ಕೃತ ಮಹಾಪಾಠಶಾಲೆ ಮೀಮಾಂಸಾ ವಿದ್ವತ್ ಪದವಿಯನ್ನು ವೈಶಿಷ್ಟ್ಯ ಶ್ರೇಣಿಯಲ್ಲೂ, ಮೈಸೂರು  ವಿಶ್ವವಿದ್ಯಾಲಯದ ಸಂಸ್ಕೃತ  ಎಂ.ಎ ಪದವಿಯನ್ನು ಎರಡು ಚಿನ್ನದ ಪದಕ ಹಾಗೂ ಪ್ರಥಮ  rankನೊಂದಿಗೆ ಪಡೆದುಕೊಂಡೆ. 1985ರಲ್ಲಿ ಮೀಮಾಂಸಾ ವಿದ್ವತ್ ಹಾಗೂ 1986 ರಲ್ಲಿ ಸಂಸ್ಕೃತ ಎಂ.ಎ ಪೂರ್ಣಗೊಳ್ಳುವುದರೊಂದಿಗೆ ಬಹುತೇಕ ನನ್ನ ವಿದ್ಯಾರ್ಥಿಜೀವನ ಸಮಾಪ್ತವಾಗಿತ್ತು. ನಾನು ವಿದ್ಯಾರ್ಥಿ ಜೀವನದಲ್ಲಿ ಮಾಡಿದ ಅನೇಕ ಸಾಧನೆಗಳಿಗೆ ಗಂಗಣ್ಣನ ಬೆಂಬಲ ಹಾಗೂ ಮಾರ್ಗದರ್ಶನ ದಾರಿದೀಪವಾಗಿತ್ತು ಎಂಬುದರಲ್ಲಿ ಎರಡು ಮಾತಿಲ್ಲ.

ಗಂಗಣ್ಣನೂ ಮೈಸೂರು ವಿಶ್ವವಿದ್ಯಾಲಯದ ಅಂಚೆ ಮತ್ತು ತೆರಪಿನ ಶಿಕ್ಷಣ ಸಂಸ್ಥೆಯ ಮೂಲಕ ಸಂಸ್ಕೃತ ಎಂ.ಎ ಪದವಿಯನ್ನೂ ಸಂಪಾದಿಸಿದ್ದ, ಶ್ರೀ ಶಂಕರವಿಲಾಸ ಪಾಠಶಾಲೆಯಲ್ಲಿ ಸಹೋದ್ಯೋಗಿಗಳಾಗಿ, ಮನೆಯಲ್ಲಿ ಸಹ ಜೀವಿಗಳಾಗಿ ನಮ್ಮ ಜೀವನ ನೆಮ್ಮದಿಯಿಂದ ಸಾಗಿತ್ತು.

ಪಾಠ–ಪ್ರವಚನ ಗಂಗಣ್ಣನಿಗೆ ನೆಚ್ಚಿನ ವಿಷಯವಾಗಿತ್ತು, ಆ ಹೊತ್ತಿಗಾಗಲೇ ಅನುಭವಿ  ಹಿರಿಯ ಪ್ರಾಧ್ಯಾಪಕರೆಲ್ಲ ಸಂಸ್ಕೃತ  ಪಾಠಶಾಲೆಯಿಂದ ವಿಶ್ರಾಂತರಾಗಿದ್ದರು. ಮನೆ ಸಂಸ್ಕೃತ ಮಹಾಪಾಠಶಾಲೆಯಿಂದ ಅನತಿದೂರದಲ್ಲಿತ್ತು. ವಿದ್ಯಾರ್ಥಿಗಳು ಶಾಸ್ತ್ರಪಾಠಕ್ಕಾಗಿ ಮನೆಗೆ ಬರುತ್ತಿದ್ದರು. ಅವರೆಲ್ಲರಿಗೆ ನಿವ್ರ್ಯಾಜದಿಂದ ಪಾಠ ಹೇಳಿದ ಶ್ರೇಯಸ್ಸು ಗಂಗಣ್ಣನಿಗೆ ಸಲ್ಲುತ್ತದೆ.

ಏತನ್ಮಧ್ಯೆ ನಾವಿಬ್ಬರೂ ಕೆಲಕಾಲ ಶಾರದಾವಿಲಾಸ ಕಾಲೇಜಿನಲ್ಲಿ (part–time) ಉಪನ್ಯಾಸಕರಾಗಿಯೂ ಕೆಲಸ ಮಾಡಿದೆವು. ಆಗ ಅಲ್ಲಿ ಸಂಸ್ಕೃತ ವಿಭಾಗಾಧ್ಯಕ್ಷರಾಗಿದ್ದವರು ಡಾ। ಸಿ. ಎಚ್. ಶ್ರೀನಿವಾಸಮೂರ್ತಿಯವರು. ಕನ್ನಡ ವಿಭಾಗದ ಉಪನ್ಯಾಸಕರಾಗಿದ್ದವರು ಶ್ರೀ ಶಿವರಾಮ ಐತಾಳರು ಮತ್ತು ಶ್ರೀ ಸೂರ್ಯನಾರಾಯಣ ಸ್ವಾಮಿಯವರು. ಆಂಗ್ಲ ವಿಭಾಗದಲ್ಲಿ ಉಪನ್ಯಾಸಕರಾಗಿದ್ದವರು ಡಾ. ಎ.ಆರ್.ನಾಗಭೂಷಣ, ಶ್ರೀ ಲಕ್ಷ್ಮೀನಾರಾಯಣ ಹಾಗೂ ಶ್ರೀ ಕೆಂಪರಾಜುರವರು. ಈ ಕಾಲೇಜಿನಲ್ಲಿ ಸಲ್ಲಿಸಿದ ಸೇವೆ ಕೆಲವೇ ತಿಂಗಳುಗಳದ್ದಾಗಿದ್ದರೂ ಇವರೆಲ್ಲರ ಸಖ್ಯ ನಮಗಾಯಿತು. 

ನಾನು ಶಾರದಾವಿಲಾಸ ಕಾಲೇಜಿನಲ್ಲಿ ನನ್ನ ಪಿ.ಯು.ಸಿ ಶಿಕ್ಷಣವನ್ನು ಪೂರೈಸಿದ್ದರಿಂದ ನಾನು ಶ್ರೀ ಶ್ರೀನಿವಾಸಮೂರ್ತಿಯವರ ವಿದ್ಯಾರ್ಥಿಯೇ ಆಗಿದ್ದೆ. ಇದರ ಹೊರತಾಗಿ ಅವರಲ್ಲಿ ‘ಸಿದ್ಧಾಂತ ಕೌಮುದಿ’ಯ ಪಾಠವನ್ನು ಅವರ ಮನೆಗೆ ತೆರಳಿ ಹೇಳಿಸಿಕೊಂಡಿದ್ದೇನೆ. ನಮ್ಮಿಬ್ಬರಲ್ಲಿರುವುದು ಗುರು–ಶಿಷ್ಯ ಭಾವ. ಇವರೆಲ್ಲರೂ ನಮ್ಮನ್ನು ಗೌರವದಿಂದ ಕಾಣುತ್ತಿದ್ದರು. ಆಂಗ್ಲ ಹಾಗೂ ಕನ್ನಡ ಅಧ್ಯಾಪಕರು ವಿಶೇಷವಾಗಿ ಗಂಗಣ್ಣನಲ್ಲಿ ತಮ್ಮ ಸಂಸ್ಕೃತ, ಸಾಮಾಜಿಕ ಸಂದಿಗ್ಧಗಳು ಹಾಗೂ ಪ್ರಸ್ತುತ ವಿದ್ಯಮಾನಗಳ ಕುರಿತು ಚರ್ಚಿಸುತ್ತಿದ್ದರು. ಇವರ ಸಂಪರ್ಕ ಬಹುಕಾಲದವರೆಗೆ ಮುಂದುವರೆದಿತ್ತು.

ದಿನದಿಂದ ದಿನಕ್ಕೆ ಗಂಗಣ್ಣನನ್ನು ಹುಡುಕಿಕೊಂಡು ಬರುವವರ ಸಂಖ್ಯೆ ಅಧಿಕವಾಯಿತು. ಮೈಸೂರಿನಲ್ಲಿ ಅಧ್ಯಯನಕ್ಕಾಗಿ ನೆಲೆಯನ್ನು ಅರಸಲು ಬರುವವರು ಒಂದೆಡೆಯಾದರೆ, ತಮಗೆ ಬಂದೊದಗಿದ ಶೈಕ್ಷಣಿಕ–ಸಾಮಾಜಿಕ ಸಂದಿಗ್ಧಗಳ ಪರಿಹಾರಕ್ಕೆ ಬರುವವರು ಮತ್ತೊಂದೆಡೆ. ಚರ್ಚಾ ಸ್ಪರ್ಧೆ, ಭಾಷಣ ಸ್ಪರ್ಧೆ, ನಿಬಂಧ ಸ್ಪರ್ಧೆ ಮುಂತಾದ ಸ್ಪರ್ಧೆಗಳಿಗೆ ಮಾರ್ಗದರ್ಶನ ಪಡೆಯಲು ಬರುವವರು ಕೆಲವರಾದರೆ, ಆಂಗ್ಲ–ಸಂಸ್ಕೃತ–ಕನ್ನಡ ಪಾಠ ಹೇಳಿಸಿಕೊಳ್ಳಲು ಬರುವವರು ಹಲವರು. ಗಂಗಣ್ಣನ ಆಂಗ್ಲ ಭಾಷೆಯ ಜ್ಞಾನವೂ ಉತ್ತಮವಾಗಿದೆ. ಅನೇಕರಿಗೆ ‘ರೆನ್ ಆಂಡ್ ಮಾರ್ಟಿನ್’ ಆಂಗ್ಲ ವ್ಯಾಕರಣವನ್ನು ಬೋಧಿಸಿದ್ದಾನೆ. ಇವೆಲ್ಲವುಗಳ ಬೋಧನೆ ಮನೆಯಲ್ಲೆ ನಡೆಯುತ್ತಿತ್ತು. ಅದೂ ನಿಃಶುಲ್ಕವಾಗಿ. ಅವನು ಎಂದೂ “ಜ್ಞಾನಪಣ್ಯ”ನಾಗಲಿಲ್ಲ.

ಗಂಗಣ್ಣನ ಉಪಕಾರ ಪ್ರಜ್ಞೆ ಸದಾ ಜಾಗೃತವಾಗಿರುತ್ತದೆ. ಅವನಿಂದ ಉಪಕೃತರಾದವರು ಹಲವರು. ಸಕಾಲದಲ್ಲಿ ಇವನ ಚಾಣಾಕ್ಷತನದಿಂದ, ಬಂದ ಅಡೆತಡೆಗಳನ್ನು ದೂರವಾಗಿಸಿಕೊಂಡು ತಮ್ಮ ವ್ಯಾಸಂಗವನ್ನು ಮುಂದುವರಿಸಿದ ಎರಡು ರೋಚಕ ಘಟನೆಗಳನ್ನು ಇಲ್ಲಿ ಉಲ್ಲೇಖಿಸುವುದು ಉಚಿತವೆನಿಸುತ್ತದೆ.

ಶ್ರೀ ಸುಂದರಮೂರ್ತಿ ಮತ್ತು ಶ್ರೀ ಸಾವಿತ್ರಮ್ಮನವರ ಮನೆಯಲ್ಲಿ ಕೆಲವು ವರ್ಷ ಗಂಗಣ್ಣ ಬಾಡಿಗೆಗಿದ್ದ. ಈ ದಂಪತಿಗಳ ಪುತ್ರ ರಾಮಸ್ವಾಮಿ. ನಮ್ಮ ವಯೋಮಾನದವನೆ. ಗಂಗಣ್ಣನನ್ನು ಇವ ಬಹಳ ಹಚ್ಚಿಕೊಂಡಿದ್ದ. ಇವನು ತಾಂತ್ರಿಕವಾಗಿ ಕುಶಲನಾಗಿದ್ದರು ಎಸ್.ಎಸ್.ಎಲ್.ಸಿಯ ಸಂಸ್ಕೃತ ವಿಷಯದಲ್ಲಿ ತೇರ್ಗಡೆಯಾಗಿರಲಿಲ್ಲ. ಗಂಗಣ್ಣನ ಪಾಠ–ಪ್ರವಚನವೂ ಫಲಕಾರಿಯಾಗಲಿಲ್ಲ. ಅಂತಿಮವಾಗಿ ಇವನನ್ನು ಮುನ್ನಡೆಸಲು ಗಂಗಣ್ಣನೆ ಯೋಜನೆಯೊಂದನ್ನು ರೂಪಿಸಿದ. ಅದನ್ನು ಕಾರ್ಯಗತಗೊಳಿಸಿದವನು ನಾನು. ಅಂತು ಹರ–ಸಾಹಸದಿಂದ ರಾಮಸ್ವಾಮಿ ಸಂಸ್ಕೃತ ಪರೀಕ್ಷೆಯಲ್ಲಿ ಉತ್ತಮಾಂಕವನ್ನು ಪಡೆದು ಉತ್ತೀರ್ಣನಾದ. ಇದರಿಂದ ಅವನ ಕುಟುಂಬದವರೆಲ್ಲ ಸಂತಸದ ನಿಟ್ಟುಸಿರು ಬಿಡುವಂತಾಯಿತು. ಇಂದು ಅದೇ ರಾಮಸ್ವಾಮಿ ಮೈಸೂರಿನ ಗಣ್ಯ ಉದ್ಯಮಿಗಳಲ್ಲ್ಲೊಬ್ಬನಾಗಿದ್ದಾನೆ.

ಎರಡನೆ ಘಟನೆ, ನವಿಲುಗೆರೆಯ ಸದಾಶಿವನಿಗೆ ಸಂಬಂಧಪಟ್ಟಿದ್ದು. ಹಳ್ಳಿಯಲ್ಲಿ ಎಸ್.ಎಸ್.ಎಲ್.ಸಿಯನ್ನು ಪೂರೈಸಿದ ಸದಾಶಿವ ತನ್ನ ಅಧ್ಯಯನವನ್ನು ಮುಂದುವರಿಸಲು ನನ್ನ ಪರಿಚಯದಿಂದ ಮೈಸೂರಿಗೆ ಬಂದ. ಮೈಸೂರಿನ ಪಾಲಿಟೆಕ್ನಿಕ್ ಕಾಲೇಜಿನ ಎಲೆಕ್ಟ್ರಿಕಲ್ ವಿಭಾಗದಲ್ಲಿ ಪ್ರವೇಶ ಪಡೆಯಲು ಅರ್ಜಿ ಸಲ್ಲಿಸಿದ್ದ. ಆರ್ಥಿಕವಾಗಿ ಹಿಂದುಳಿದವರ ಕಾಯ್ದಿರಿಸಿದ (economically backword) ಪಟ್ಟಿಯಲ್ಲಿ ಇವನ ಹೆಸರು ಪ್ರಕಟವಾಗಿತ್ತು. ಅಂತಿಮ ಕ್ಷಣದಲ್ಲಿ ಪ್ರವೇಶ ಪಡೆಯುವಂತೆ ಸೂಚನೆ ನೀಡಲಾಯಿತು.

ಅಸಹಾಯಕನಾಗಿ ನನ್ನಲ್ಲಿಗೆ ಬಂದ ಸದಾಶಿವನೊಂದಿಗೆ ಕಾಲೇಜಿಗೆ ತೆರಳಿ, ಎಲೆಕ್ಟ್ರಿಕಲ್ ವಿಭಾಗದ ಮುಖ್ಯರಲ್ಲಿ ‘ಪ್ರವೇಶಕ್ಕೆ ಇನ್ನೊಂದು ದಿನದ ಅವಕಾಶವನ್ನು ನೀಡುವಂತೆ’ ವಿನಂತಿಸಿದೆ. ಅದಕ್ಕೆ ಒಪ್ಪಿಗೆ ನೀಡಿದ ಆ ವಿಭಾಗಾಧ್ಯಕ್ಷರು ‘ನಾಳೆ ಈ ಹುಡುಗ ತನ್ನ ತಂದೆಯೊಂದಿಗೆ ಬಂದು ನನ್ನನ್ನು ಸಂಪರ್ಕಿಸಲಿ’ – ಎಂದರು. ಮರುಮಾತಿಲ್ಲದೆ ಅದನ್ನು ಒಪ್ಪಿದೆ. ಸಂಪರ್ಕ ಸಾಧನಗಳಿಲ್ಲದ ಕಾಲವದು. ಆಗಲೇ ಅರ್ಧ ದಿನ ಕಳೆದಿತ್ತು. 500 ಕಿಲೋಮೀಟರ್ ದೂರದಲ್ಲಿರುವ ಸದಾಶಿವನ ತಂದೆಯವರನ್ನು ಕರೆಸುವುದು ಸುತರಾಂ ಅಸಂಭವವಾಗಿತ್ತು. 

ದಾರಿಕಾಣದೆ ಸದಾಶಿವನೊಂದಿಗೆ ಗಂಗಣ್ಣನಿದ್ದಲ್ಲಿಗೆ ತೆರಳಿದೆ. ನಾವಿಬ್ಬರು ಈ ವಿಷಯದ ಕುರಿತು ಸಾಕಷ್ಟು ಚರ್ಚಿಸಿದೆವು. ಕೊನೆಯಲ್ಲಿ ಗಂಗಣ್ಣನೆ ಕಾರ್ಯಸಾಧುವಾಗಬಹುದಾದ ಯೋಜನೆಯೊಂದನ್ನು ಹೆಣೆದು ತನ್ನ ಗುರುಗಳಾದ ಪೂಜ್ಯ ಶ್ರೀ ವೆಂಕಣ್ಣಾಚಾರ್ಯರ ಬಳಿಗೆ ಕರೆದೊಯ್ದು ವಿಷಯ ವಿವರಿಸಿ ‘ಒಂದು ದಿನದ ಮಟ್ಟಿಗೆ’ ನೀವು ಈ ಹುಡುಗನ ತಂದೆ ಅಣ್ಣಯ್ಯ ಭಟ್ಟರಾಗಿ ವ್ಯವಹರಿಸಬೇಕೆಂದು ವಿನಂತಿಸಿಕೊಂಡ.

ಗಂಗಣ್ಣನನ್ನು ಬಹುವಾಗಿ ಪ್ರೀತಿಸುವ ವೆಂಕಣ್ಣಾಚಾರ್ಯರು ಮರುಮಾತಿಲ್ಲದೆ ಇದಕ್ಕೆ ಸಮ್ಮತಿಸಿದರು. ಕಾಲೇಜಿನ ಪ್ರಾಧ್ಯಾಪಕರು ಕೇಳುವ ಸಂಭಾವ್ಯ ಪ್ರಶ್ನೆಗಳಿಗೆ ಏನು ಉತ್ತರ ಹೇಳಬೇಕೆಂಬುದನ್ನು ಆಚಾರ್ಯರಿಗೆ ತಿಳಿಸಲಾಯಿತು. ವೆಂಕಣ್ಣಾಚಾರ್ಯರ ಮನೆಯ ಸಮೀಪವೆ ವಾಸವಾಗಿದ್ದ ಪ್ರಾಧ್ಯಾಪಕರ ಮನೆಗೆ ಮುಂಜಾವಿನಲ್ಲೆ ವೆಂಕಣ್ಣಾಚಾರ್ಯರನ್ನು ಮತ್ತು ಸದಾಶಿವನನ್ನು ಕರೆದೊಯ್ಯಲಾಯಿತು. ಮನೆಗೆ ಬಂದ ವೆಂಕಣ್ಣಾಚಾರ್ಯರನ್ನು ಸ್ವಾಗತಿಸಿದ ಆ ಪ್ರಾಧ್ಯಾಪಕರು, ನಿಮ್ಮ ಹೆಸರೇನೆಂದು ಪ್ರಶ್ನಿಸಿದರು. 

ಆಗ ವೆಂಕಣ್ಣಾಚಾರ್ಯರು (ಮೊದಲೇ ಹೇಳಿಕೊಟ್ಟಂತೆ, ಕಿವಿ ಸರಿಯಾಗಿ ಕೇಳಿಸದವರಂತೆ ನಟಿಸಿ) ‘ಆಂ..... ಏನೆಂದಿರಿ.....?’ – ಎಂದರು. ಪಕ್ಕದಲ್ಲೆ ಇದ್ದ ಗಂಗಣ್ಣ ತಕ್ಷಣ ಏರುಧ್ವನಿಯಲ್ಲಿ ಆಚಾರ್ಯರನ್ನು ಸಂಬೋಧಿಸಿ ‘ಅಣ್ಣಯ್ಯ ಭಟ್ಟರೇ ! ನಿಮ್ಮ ಹೆಸರೇನು ಎಂದು ಕೇಳುತ್ತಿದ್ದಾರೆ’ – ಎಂದ. ಇದರಿಂದ ಜಾಗ್ರತರಾದ ವೆಂಕಣ್ಣಾಚಾರ್ಯರು ‘ನಾನು ಅಣ್ಣಯ್ಯ ಭಟ್ಟ. (ಜೋಳಿಗೆಯನ್ನು ತೋರಿಸಿ) ಪೌರೋಹಿತ್ಯಕ್ಕೆ ಹೋಗುವ ಗಡಿಬಿಡಿಯಲ್ಲಿದ್ದೇನೆ ಆದ್ದರಿಂದ ಕಾಲೇಜಿಗೆ ಬರಲು ಸಾಧ್ಯವಾಗುತ್ತಿಲ್ಲ, ಈ ನಮ್ಮ ಹುಡುಗನಿಗೆ ಅನುಕೂಲ ಮಾಡಿಕೊಡಿ’ – ಎಂದು ವಿನಂತಿಸಿದರು. ಅದರಿಂದ ಸಂತೃಪ್ತರಾದ ಪ್ರಾಧ್ಯಾಪಕರು ಕೈಮುಗಿದು ‘ನೀವು ನಿಮ್ಮಕಾರ್ಯಕ್ಕೆ ತೆರಳಿ, ಚಿಂತಿಸಬೇಡಿ. ನಾನು ಎಲ್ಲವನ್ನು ನೋಡಿಕೊಳ್ಳುತ್ತೇನೆ. ಹುಡುಗ ನಿಗದಿತ ಶುಲ್ಕದೊಂದಿಗೆ ಕಾಲೇಜಿಗೆ ಬಂದು ಪ್ರವೇಶ ಪಡೆಯಲಿ’ – ಎಂದರು. ಈ ಮಾತನ್ನು ಕೇಳಿದ್ದೇ ತಡ ಬೆವರುತ್ತಿದ್ದ ಆಚಾರ್ಯರು ಅಲ್ಲಿಂದ ಕಾಲ್ಕಿತ್ತರು.

ಹೀಗೆ ವೆಂಕಣ್ಣಾಚಾರ್ಯರ ಶಿಷ್ಯವಾತ್ಸಲ್ಯದ ಪ್ರಭಾವದಿಂದ, ಸದಾಶಿವ ಆ ಕಾಲೇಜಿನಲ್ಲಿ ಪ್ರವೇಶ ಪಡೆದು ತನ್ನ ವ್ಯಾಸಂಗವನ್ನು ಮುಂದುವರಿಸುವಂತಾಯಿತು. ಗಂಗಣ್ಣನ ಸಮಯ ಪ್ರಜ್ಞೆ , ಉಪಕರಿಸಬೇಕೆಂಬ ಹಂಬಲ ವಿಷಮ ಪರಿಸ್ಥಿತಿಯನ್ನು ತೊಡಕಾಗದಂತೆ ನಿಭಾಯಿಸುವ ಕೌಶಲ್ಯಕ್ಕೆ ಇದೊಂದು ಜ್ವಲಂತ ಉದಾಹರಣೆ. 

ಗಂಗಣ್ಣನ ಸೋದರ ಭಾವ ಮುಂಡೂಸರ ಶ್ರೀ ಮಂಜುನಾಥ ಹೆಗಡೆಯವರ ಪುತ್ರಿಯಾದ ಪಾರ್ವತಿಯೊಂದಿಗೆ ನನ್ನ ವಿವಾಹ 1988 ಏಪ್ರಿಲ್‍ನಲ್ಲಿ ಸಂಪನ್ನವಾಯಿತು. ಮೈಸೂರಿನ ಶೃಂಗೇರಿ ಶಂಕರ ಮಠದ ಸಮೀಪ ಚಿಕ್ಕ ಮನೆಯೊಂದರಲ್ಲಿ ನನ್ನ ವಿವಾಹೋತ್ತರ ಸಂಸಾರ ಪ್ರಾರಂಭವಾಯಿತು. ಆದರೂ ನಮ್ಮ ಈರ್ವರ ಕುಟುಂಬಗಳ ಸಂಪರ್ಕ ಅನುಸ್ಯೂತವಾಗಿ ನಡೆದಿತ್ತು. 1990 ರಲ್ಲಿ ನನಗೆ ಪುತ್ರೋತ್ಸವವಾಯಿತು. ಕೆಲವು ಸಮಯಗಳ ನಂತರ (1993) ಗಂಗಣ್ಣ ವಾಸಿಸುತ್ತಿದ್ದ ಮನೆಯ ಕೆಳಭಾಗ ತೆರವಾಯಿತು. ಮಾಲೀಕರಾದ ಲಾಯರ್ ಕೃಷ್ಣಸ್ವಾಮಿಯವರಿಗೆ ಮನವರಿಕೆ ಮಾಡಿ ಆ ಮನೆಯನ್ನು ನಮಗೆ ಬಾಡಿಗೆಗೆ ಕೊಡುವಂತೆ ಮಾಡಿದವನು ಗಂಗಣ್ಣ. ಅನುಕೂಲವಾದ ಮನೆಯೊಂದಿಗೆ ನಾವೆಲ್ಲ ಪುನಃ ಒಂದೆ ಸೂರಿನಲ್ಲಿ (ಮೇಲೆ–ಕೆಳಗೆ) ವಾಸಿಸುವ ಸಂತಸ ನಮ್ಮದಾಯಿತು. ನನ್ನ ಪತ್ನಿಯು ತನ್ನ ಪಿ.ಯು.ಸಿ ಶಿಕ್ಷಣವನ್ನು (1979–81) ಮೈಸೂರಿನ ಅಂಚೆ ಮತ್ತು ತೆರಪಿನ ಶಿಕ್ಷಣ ಸಂಸ್ಥೆಯಲ್ಲಿ ಗಂಗಣ್ಣನ ಆಶ್ರಯದಲ್ಲಿ ಪೂರೈಸಿದಳೆಂಬುದು ಇಲ್ಲಿ ಸ್ಮರಣಾರ್ಹ ಅಂಶ.

1991 ರಲ್ಲಿ ಸಿಕಂದರಾಬಾದ್‍ನಲ್ಲಿ ನಡೆದ ಕೇಂದ್ರೀಯ ವಿದ್ಯಾಲಯದ ಸಂಸ್ಕೃತ ಶಿಕ್ಷಕ ಹುದ್ದೆಯ ನಿಯುಕ್ತಿಯ ಆದೇಶ 1993 ಡಿಸೆಂಬರ್‍ನಲ್ಲಿ ನನಗೆ ಬಂತು. ಸಂದಿಗ್ಧದಲ್ಲಿ ಮುಳುಗಿದೆ. ಮೈಸೂರು ಬಿಟ್ಟು ಬೇರೆಡೆ ತೆರಳಲು ಮನಸ್ಸಿರಲಿಲ್ಲ. ಆದರೆ, ಅದು  ಭಾರತ ಸರಕಾರದ ಮಾನವ ಸಂಪನ್ಮೂಲ ಸಂಸಾಧನ ಮಂತ್ರಾಲಯದಡಿ ಸ್ವಾಯತ್ತ ಸಂಸ್ಥೆಯಾಗಿ ಕಾರ್ಯ ನಿರ್ವಹಿಸುವ ಕೇಂದ್ರೀಯ ವಿದ್ಯಾಲಯ ಸಂಘಟನೆಯ ಉದ್ಯೋಗ. ಅದೂ ಬೆಂಗಳೂರಿನ ಯಲಹಂಕದ ವಾಯುಸೇನಾ ನೆಲೆಯ ಕೇಂದ್ರೀಯ ವಿದ್ಯಾಲಯದಲ್ಲಿ. ಉದ್ಯೋಗ ಸುರಕ್ಷೆ ಹಾಗೂ ಅಧಿಕಾದಾಯ ದೃಷ್ಟಿಯಿಂದ ಗಂಗಣ್ಣನ ಹಾಗೂ ಹಿತೈಷಿಗಳ ಸಲಹೆಯಂತೆ ಆ ನೂತನ ಉದ್ಯೋಗಕ್ಕೆ (19–01–1994 ರಂದು) ಹಾಜರಾದೆ.

1993ರಲ್ಲಿ ಮೈಸೂರಿನಲ್ಲಿಯೂ ಒಂದು ಕೇಂದ್ರೀಯ ವಿದ್ಯಾಲಯ ಪ್ರಾರಂಭವಾಗಿತ್ತು. ಮನಸ್ಸಿನಾಳದಲ್ಲಿ ಆದಷ್ಟು ಬೇಗ ಮೈಸೂರಿನ ಈ ಕೇಂದ್ರೀಯ ವಿದ್ಯಾಲಯಕ್ಕೆ ಬರಬೇಕೆಂಬ ತವಕ. ಆದ್ದರಿಂದ ಕುಟುಂಬವನ್ನು ಬೆಂಗಳೂರಿಗೆ ಸ್ಥಳಾಂತರಿಸಲಿಲ್ಲ. ಮಲ್ಲೇಶ್ವರಂನಲ್ಲಿದ್ದ ಚಿಕ್ಕಮ್ಮನ ಮನೆಯಲ್ಲಿ ವಾಸ. ಶನಿವಾರ ಸಂಜೆ ಮೈಸೂರಿಗೆ ಬಂದು, ಸೋಮವಾರ ಬೆಳಗ್ಗೆ ಪುನಃ ಪ್ರಯಾಣ. ಪುತ್ರ ಮೂರು ವರ್ಷದವ. ಆದರೂ ನಿಶ್ಚಿಂತೆ. ಏಕೆಂದರೆ ಗಂಗಣ್ಣ, ಅವನ ತಂಗಿ ರತ್ನಾವತಿ ಮೊದಲನೆ ಮಹಡಿಯಲ್ಲಿದ್ದರು.

ಹೀಗೆ ಎರಡು ವರ್ಷ ಕಳೆಯಿತು. 1996 ಜೂನ್ ತಿಂಗಳಿನಲ್ಲಿ ಬೆಂಗಳೂರು ವಿಭಾಗೀಯ ಕಛೇರಿಯಿಂದ ಮೈಸೂರಿನ ಕೇಂದ್ರೀಯ ವಿದ್ಯಾಲಯಕ್ಕೆ ಸ್ಥಾನಾಂತರವಾದ ಆದೇಶ ಬಂದರೂ ಫಲಕಾರಿಯಾಗಲಿಲ್ಲ. ಏಕೆಂದರೆ ಕೇಂದ್ರೀಯ ವಿದ್ಯಾಲಯ ಸಂಘಟನೆಯ ಕೇಂದ್ರ ಕಛೇರಿಯ ಆದೇಶ ಪಡೆದ ಪರಿಚಿತರೇ ಆದ ವಿ। ಲಿಂಗಣ್ಣ ಭಟ್ಟರು ಮೈಸೂರಿನ ಕೇಂದ್ರೀಯ ವಿದ್ಯಾಲಯಕ್ಕೆ ಸ್ಥಾನಾಂತರಿತರಾಗಿ ಸಂಸ್ಕೃತ ಅಧ್ಯಾಪಕ ಸ್ಥಾನದಲ್ಲಿ ಭರ್ತಿಯಾಗಿದ್ದರು. ಅನಿವಾರ್ಯವಾಗಿ ನನ್ನ ಕುಟುಂಬವನ್ನು ಬೆಂಗಳೂರಿನ ಯಲಹಂಕ ಕೇಂದ್ರೀಯ ವಿದ್ಯಾಲಯದ ಶಿಕ್ಷಕರ ವಸತಿಗೃಹಕ್ಕೆ ಸ್ಥಾನಾಂತರಿಸಬೇಕಾಯಿತು.

ಮೈಸೂರಿಗೆ ಬರಬೇಕೆಂಬ ಕನಸು ಮುಂದೆ ನನಸಾದದ್ದು 2008ರಲ್ಲಿ. ಉದ್ಯೋಗಕ್ಕಾಗಿ ಮೈಸೂರಿನಿಂದ ಹೊರಗಿದ್ದ (ಯಲಹಂಕ – 10 ವರ್ಷ, ಕಾಠಮಂಡು – 3 ವರ್ಷ, ಕಾರವಾರ – 2 ವರ್ಷ) 15 ವರ್ಷಗಳಲ್ಲಿ ಪ್ರಾರಂಭಿಕ ಎರಡು ವರ್ಷ ಹಾಗೂ ಕೊನೆಯ (ಕಾರವಾರದ) ಎರಡು ವರ್ಷ ನನ್ನ ಪತ್ನಿ ಹಾಗೂ ಪುತ್ರ ಮೈಸೂರಿನಲ್ಲೇ ನೆಲೆಸಿದ್ದರು. ಈ ಸಮಯದಲ್ಲಿ ಗಂಗಣ್ಣನ ಸಹಾಯ–ಸಹಕಾರ ಮರೆಯಲಾಗದ್ದು.

2006ರಲ್ಲಿ ನಾನು ನೇಪಾಳದ ಕಾಠಮಂಡು ಕೇಂದ್ರೀಯ ವಿದ್ಯಾಲಯದಿಂದ ಕಾರವಾರ ಕೇಂದ್ರೀಯ ವಿದ್ಯಾಲಯಕ್ಕೆ ವರ್ಗವಾಗಿ ಬಂದಾಗ, ನನ್ನ ಮಗ ಪರೀಕ್ಷಿತ್ ಮೈಸೂರಿನ ಕೇಂದ್ರೀಯ ವಿದ್ಯಾಲಯದಲ್ಲೆ ಪಿ.ಯು.ಸಿ ಶಿಕ್ಷಣವನ್ನು ಮುಂದುವರೆಸಿದ. ಪತ್ನಿ ಪುತ್ರರಿಬ್ಬರು ಸಿದ್ದಾರ್ಥನಗರದ ಬಾಡಿಗೆ ಮನೆಯಲ್ಲಿ ವಾಸಿಸುತ್ತಿದ್ದರು. ನಾನು ಹದಿನೈದು ದಿವಸಕ್ಕೊಮ್ಮೆ ಬಂದು–ಹೋಗುತ್ತಿದ್ದೆ. ಈ ಸಂದರ್ಭದಲ್ಲಿ, ನಾನು ಬಹು ಹಿಂದೆಯೇ ಮೈಸೂರಿನಲ್ಲಿ ಖರೀದಿಸಿದ್ದ ನಿವೇಶನವನ್ನು ಮೈಸೂರಿನಲ್ಲಿ ನನ್ನ ಅನುಪಸ್ಥಿತಿಯನ್ನು ಅರಿತ ವ್ಯಕ್ತಿಯೊಬ್ಬ ಕಬಳಿಸಲು ಮುಂದಾಗಿದ್ದ. ಆ ಸಂದರ್ಭದಲ್ಲೂ ಗಂಗಣ್ಣ ಕಾನೂನಾತ್ಮಕ ಹೋರಾಟದಲ್ಲಿ ಸಹಭಾಗಿಯಾಗಿ ಉಪಕರಿಸಿದ್ದಾನೆ. ಹೀಗೆ ನನ್ನ ಜೀವನದ ಸಂತಸ ಹಾಗೂ ಕಷ್ಟದ ದಿನಗಳೆರಡರಲ್ಲೂ ಸಮನಾಗಿ ಸಹಭಾಗಿಯಾದ ಈ ಅಣ್ಣನನ್ನು ಏನೆಂದು ಭಾವಿಸಲಿ? ಮಿತ್ರಭಾವವೇ ಉಚಿತವಲ್ಲವೆ!

ಗಂಗಣ್ಣನ ವಿವಾಹ, ವೇದಾಧ್ಯಯನ ಸಂಪನ್ನರಾದ ಮಂಜಗುಣಿ ಸಮೀಪದ ತೆಪ್ಪಾರಿನ ವೇ। ವೆಂಕಟರಮಣ ಭಟ್ಟ ದಂಪತಿಗಳ ಸುಪುತ್ರಿ ಶೈಲಜಾಳೊಂದಿಗೆ 1997ರಲ್ಲಿ ನೆರವೇರಿತು. ಮನೋನುಕೂಲೆಯಾದ ಮಡದಿಯೊಂದಿಗೆ ಅವನ ಸಂಸಾರ ಮೈಸೂರಿನಲ್ಲಿ ಆನಂದದಿಂದ ಸಾಗಿತ್ತು. ಮೈಸೂರಿಗೆ, ಪ್ರಥಮಬಾರಿಗೆ ಅಧ್ಯಯನಕ್ಕಾಗಿ ಬರುವವರ ತಾತ್ಕಾಲಿಕ ವಸತಿ, ಆಗಮಿಸುವ ವಿದ್ವನ್ಮಿತ್ರರ ವಸತಿ, ಎಸ್.ಎಸ್.ಎಲ್.ಸಿ ಹಾಗೂ ಪಿ.ಯು.ಸಿ ಪರೀಕ್ಷಾ ಮೌಲ್ಯಮಾಪನಕ್ಕೆ ಬರುವ ಊರಿನಕಡೆಯ ಶಿಕ್ಷಕರ ವಸತಿ, ಅಂಚೆ ಮತ್ತು ತೆರಪಿನ ಶಿಕ್ಷಣದ ಸ್ಪರ್ಷಶಿಬಿರಕ್ಕಾಗಿ ಬರುವವರ ವಸತಿ ಹಾಗೂ ಆದರಾತಿಥ್ಯ ಇವರಿಂದ ನಿರಂತರವಾಗಿ ಸಾಗಿಬಂದಿದೆ.

ಮನೆಯ ಸಮಸ್ತ ಆಗು–ಹೋಗುಗಳಲ್ಲಿ ಜವಾಬ್ದಾರಿಯುತ ಕಾರ್ಯ ನಿರ್ವಹಿಸುವುದು ಗಂಗಣ್ಣನ ಜಾಯಮಾನ. ತಂಗಿಯರ ವಿದ್ಯಾಭ್ಯಾಸ ಹಾಗೂ ಅವರ ವಿವಾಹ ಕಾರ್ಯಗಳಲ್ಲಿ ಗಂಗಣ್ಣನ ಪಾತ್ರ ಪ್ರಧಾನವಾದುದು. ಸಂಪೂರ್ಣ ಬೇಸಿಗೆ ರಜೆಯನ್ನು ಮಣ್ಣಿಕೊಪ್ಪದ ಮನೆಯಲ್ಲೆ ಕಳೆಯುತ್ತಿದ್ದ ಇವನು ಕೃಷಿ ಸಂಬಂಧ ಕೆಲಸ ಹಾಗೂ ಮನೆಯ ದುರಸ್ತಿಯೇ ಮುಂತಾದ ಕೆಲಸಗಳಲ್ಲಿ ಸ್ವತಃ ಆಳಾಗಿ ದುಡಿಯುತ್ತಿದ್ದ. ಅಂತೆಯೆ ಹಿರಿದಾದ ಕೆಲಸಗಳನ್ನು ತನ್ನ ಆದಾಯದ ಹಣವನ್ನು ವ್ಯಯಿಸಿ ತನ್ನ ಪರಿವೀಕ್ಷಣೆಯಲ್ಲಿ, ಕಾರ್ಮಿಕರಿಂದ ಪೂರ್ಣಗೊಳಿಸುತ್ತಿದ್ದ. ಆಯುರ್ವೇದ ಸಸ್ಯಗಳ ವಿಶೇಷ ಅಧ್ಯಯನ ಹಾಗೂ ಪ್ರಕೃತಿಯ ಮಡಿಲಿನಲ್ಲಿ ವಾಸಿಸುವುದು ಇವನಿಗೆ ಅತ್ಯಂತ ಪ್ರಿಯವಾದ ವಿಷಯವಾಗಿದೆ.

ದಾಸಶಿರೋಮಣಿ ಹನುಮಂತ ನಮ್ಮ ಕುಟುಂಬದ ಆರಾಧ್ಯ ದೈವ. ಹಿಂದೆ ಹೇಳಿದಂತೆ, ಭತ್ತದ ಗದ್ದೆಯ ಅಂಚಿನಲ್ಲಿ ಹರಿಯುವ ಸಣ್ಣ ತೊರೆಯಲ್ಲಿ ಲಭಿಸಿದ ಕಗ್ಗಲ್ಲಿನ ಮೂರ್ತಿಗೆ ದೊಡ್ಡಪ್ಪ ವೇ। ವಿಘ್ನೇಶ್ವರ ಭಟ್ಟರು ಗುಡಿಯನ್ನು ನಿರ್ಮಿಸಿ ನಿರಂತರವಾಗಿ ನಿತ್ಯ ಪೂಜೆಯನ್ನು ನೆರವೇರಿಸುತ್ತ ಬಂದಿದ್ದರು. ಈ ಗುಡಿ ಜೀರ್ಣವಾಗಿತ್ತು. ಕಾರಣ ಈ ಗುಡಿಯ ಹಿಂಭಾಗದಲ್ಲೆ ಸ್ವಲ್ಪ ಎತ್ತರದ ಜಾಗದಲ್ಲಿ ನೂತನ ಗುಡಿಯನ್ನು ನಿರ್ಮಿಸಲು ದೊಡ್ಡಪ್ಪನವರು ಕಾರ್ಯಾರಂಭಿಸಿದ್ದರು. ಮನೆಯಲ್ಲಿ ಘಟಿಸುತ್ತಿದ್ದ ಅವಘಡಗಳನ್ನು ಮನಸ್ಸಿನಲ್ಲಿಟ್ಟುಕೊಂಡು, ಇದಕ್ಕೆಲ್ಲ ಈಗಿರುವ ಹನುಮನ ಮೂರ್ತಿಯ ದೃಷ್ಟಿಯೇ ಕಾರಣವೆಂಬ ಕೊರಗು ದೊಡ್ಡಮ್ಮ (ಗಂಗಣ್ಣನ ತಾಯಿ)ನವರಲ್ಲಿತ್ತು. 

ತಾಯಿಯ ಈ ಕೊರಗನ್ನು ದೂರಮಾಡುವ ದೃಷ್ಟಿಯಿಂದಲೋ ಎಂಬಂತೆ ಗಂಗಣ್ಣ ಹನುಮಂತನ ಕಲಾತ್ಮಕವಾದ ಮೂರ್ತಿಯನ್ನು ಮೈಸೂರಿನ ನುರಿತ ಶಿಲ್ಪಿ ಶ್ರೀ ಶ್ಯಾಮಸುಂದರ ಭಟ್ ಇವರಿಂದ ಕೆತ್ತಿಸಿ, ನಿರ್ಮಾಣಗೊಂಡ ನೂತನ ಗುಡಿಯಲ್ಲಿ ಪ್ರತಿಷ್ಠಾಪಿಸಲು ಕಾರಣನಾಗಿದ್ದಾನೆ. ಮನೆಯವರೆಲ್ಲಾ ಹನುಮನಲ್ಲಿ ಅಚಲ ಭಕ್ತಿಯುಳ್ಳವರು. ದೊಡ್ಡಪ್ಪನ ನಂತರ ಮಂಜುನಾಥ ಅಣ್ಣ ತನ್ನ ದೈನಂದಿನ ಕಾರ್ಯವನ್ನು ಹನುಮನ ಪ್ರಾತಃಪೂಜೆಯಿಂದಲೇ ಪ್ರಾರಂಭಿಸುತ್ತಿದ್ದರು. ಹನುಮನಿದ್ದ ಚಿಕ್ಕ ಆಲಯವನ್ನು ವಿಸ್ತರಿಸಬೇಕೆಂಬ ಸಂಕಲ್ಪದಿಂದ ಕಾರ್ಯಾರಂಭಿಸಿದ್ದರು. ಆದರೆ, ಈ ಕಾರ್ಯ ಕೈಗೂಡುವ ಮೊದಲೇ ಅವರು ಇಹಲೋಕ ತ್ಯಜಿಸಿದ್ದು ವಿಪರ್ಯಾಸವೇ ಸರಿ. ಮುಂದೆ ಈ ಕಾರ್ಯವನ್ನು ಕೈಗೆತ್ತಿಕೊಂಡ ಗಂಗಣ್ಣ ಬಹುಧನ ವ್ಯಯಿಸಿ ಶ್ರೀಧರಣ್ಣನ ಸಹಕಾರದಿಂದ ವ್ಯವಸ್ಥಿತ ಮಂದಿರ ನಿರ್ಮಿಸಿದ್ದು ಪ್ರಶಂಸಾರ್ಹ.

1998ರಲ್ಲಿ ಗಂಗಣ್ಣ ಶ್ರೀಮನ್‍ಮಹಾರಾಜ ಸಂಸ್ಕೃತ ಮಹಾಪಾಠಶಾಲೆಯಲ್ಲಿ ನವೀನನ್ಯಾಯ ಸಹಾಯಕ ಪ್ರಾಧ್ಯಾಪಕನಾಗಿ ನಿಯುಕ್ತನಾದಾಗ, ಅವನ ಆಳವಾದ ಶಾಸ್ತ್ರಪಾಂಡಿತ್ಯ ಹಾಗೂ ಪ್ರತಿಭೆಗೆ ಹೊಸ ಆಯಾಮ ಸಿಕ್ಕಿತೆಂದೆ ಹೇಳಬೇಕು. ತನ್ನ ಶಾಸ್ತ್ರ ಜ್ಞಾನವನ್ನು ನಿರ್ವಂಚನೆಯಿಂದ ವಿದ್ಯಾರ್ಥಿಗಳಿಗೆ ಧಾರೆ ಎರೆದಿದ್ದಾನೆ. ಛಾತ್ರರ ಅಭ್ಯುದಯವನ್ನು ಸದಾ ರಕ್ಷಿಸುತ್ತಿದ್ದ  ಇವನು ನಿಜಾರ್ಥದಲ್ಲಿ ಗುರು ಎನಿಸಿಕೊಂಡನು. ಆದಿ ಗುರು ಶಂಕರಾಚಾರ್ಯರ – ‘\textbf{ಅಧಿಗತತತ್ವಃ ಶಿಷ್ಯಹಿತಾಯ ಉದ್ಯತಃ ಸತತಂ ಯಃ ಸಃ ಗುರುಃ}’ ಎಂಬ ವಾಣಿ ಇವನಿಗೆ ಅನ್ವಯಿಸುತ್ತದೆ. ಅಂತೆಯೆ ಮಹಾಪಾಠಶಾಲೆಯಲ್ಲಿ ನಡೆಯುತ್ತಿದ್ದ ರಾಷ್ಟ್ರ ಹಾಗೂ ರಾಜ್ಯಸ್ಥರೀಯ ವಿದ್ವದ್ಗೋಷ್ಠಿ, ವಾಕ್‍ಪ್ರತಿಯೋಗಿತಾ, ಅಂತ್ಯಾಕ್ಷರಿ, ನಾಟಕ ಸ್ಪರ್ಧೆ ಮುಂತಾದವುಗಳ ಆಯೋಜನೆಯಲ್ಲಿ ಪ್ರಧಾನ ಪಾತ್ರ ವಹಿಸಿದ್ದಾನೆ. ಪಂಡಿತರಾದವರಿಗೆ ವ್ಯವಹಾರ ಜ್ಞಾನದ ಕೊರತೆಯಿರುತ್ತದೆಂಬುದು ಕೆಲವರ ಅಂಬೋಣ. 

ಇದಕ್ಕೆ ಅಪವಾದವೋ ಎಂಬಂತೆ ಸಂಸ್ಕೃತ ಕಾಲೇಜಿನ ಸಮಸ್ತ ಕಾರ್ಯವನ್ನು ತನ್ನ ಕೌಶಲದಿಂದ ನಿರ್ವಹಿಸುವುದರೊಂದಿಗೆ ಅದರ ಜೀವನಾಡಿಯೇ ಆಗಿದ್ದಾನೆಂದರೆ ಅತಿಶಯೋಕ್ತಿಯಾಗಲಾರದು. ಸಾಮಾಜಿಕ ಕಾರಣಗಳಿಂದ ಸಂಸ್ಕೃತ ಶಾಸ್ತ್ರಾಧ್ಯಯನಕ್ಕೆ ಬರುವ ವಿದ್ಯಾರ್ಥಿಗಳ ಸಂಖ್ಯೆ ಕ್ಷೀಣಿಸಿದ್ದು, ಬಂದರೂ ಶಾಸ್ತ್ರಾಧ್ಯಯನವನ್ನು ಸರಿಯಾಗಿ ಮಾಡದೇ, ಅನ್ಯ ವ್ಯವಹಾರಗಳಲ್ಲಿ ವ್ಯಸ್ತರಾಗಿ ಧನಾರ್ಜನೆಯಲ್ಲಿ ತೊಡಗುತ್ತಿರುವುದು ಇವನ ಮನಸ್ಸಿಗೆ ಇನ್ನಿಲ್ಲದ ನೋವನ್ನು ತಂದಿದೆ. ‘ಗುರುವಿನ ಯಶಸ್ಸು ಶಿಷ್ಯರಲ್ಲಿದೆ’ ಎಂಬುದು ಇವನ ಅಂತರಾಳ.

ಗಂಗಣ್ಣನ ಸಾಮಾಜಿಕ ಕೊಡುಗೆ ಗುರುತರವಾದದ್ದು. ಕ್ರಿಯಾರಹಿತವಾಗಿ ಸುಪ್ತವಾಗಿದ್ದ ಮೈಸೂರಿನ ಹವೀಕ ಸಂಘಕ್ಕೆ ತನ್ನ ಮುತ್ಸದ್ಧಿತನದಿಂದ ವಿವಾದಕ್ಕೆಡೆಯಾಗದಂತೆ ಚಾಲನೆ ನೀಡಿದ್ದು ಇವನ ವ್ಯವಹಾರ ಕೌಶಲಕ್ಕೆ ಕೈಗನ್ನಡಿ. ‘ಸುರಸಾ’–ಪತ್ರಿಕೆಯ ಪ್ರಕಟಣೆಯಲ್ಲಿ ಉಂಟಾದ ವಿವಾದದಿಂದ ಸಂಸ್ಕೃತ ಕಾಲೇಜಿನ ‘ಪ್ರದೋಷ ಸಂಘ’ವನ್ನು ರಕ್ಷಿಸುವಲ್ಲಿ ಇವನದು ಪ್ರಧಾನ ಪಾತ್ರ. ಉತ್ತರ ಕನ್ನಡ ಜಿಲ್ಲಾ ಸಾಂಸ್ಕೃತಿಕ ಸಂಘ, ವೇದ ಶಾಸ್ತ್ರ  ಫೋಷಿಣೀ ಸಭಾ, ಸಂಸ್ಕೃತ ಪಾಠಶಾಲಾ ಶಿಕ್ಷಕರ ಸಂಘ ಮುಂತಾದ ಹತ್ತು ಹಲವು ಸಂಘ–ಸಂಸ್ಥೆಗಳಲ್ಲಿ ಎಲೆಮರೆಯ ಕುಸುಮದಂತೆ ಕಾರ್ಯನಿರ್ವಹಿಸಿದುದು ಇವನ ಹೆಗ್ಗಳಿಕೆ. ಅನೇಕ ಸಂಘ–ಸಂಸ್ಥೆಗಳು ಇವನನ್ನು ತಮ್ಮ ಸಂಸ್ಥೆಯ ಸಲಹೆಗಾರನನ್ನಾಗಿ ಆಯ್ದುಕೊಂಡಿರುವುದು ಗಮನಾರ್ಹ.

ಪ್ರಾಚೀನ–ನವೀನ ನ್ಯಾಯ ಶಾಸ್ತ್ರ, ಭಾರತೀಯ ತತ್ವ ಶಾಸ್ತ್ರ, ಸಾಂಖ್ಯ–ಯೋಗ ಇವುಗಳೊಂದಿಗೆ ಚಾಣಕ್ಯನ ಅರ್ಥಶಾಸ್ತ್ರ, ಚರಕ–ಶುಶ್ರುತ ಸಂಹಿತೆಗಳ ವಿಶೇಷ ಅಧ್ಯಯನವನ್ನು ಗಂಗಣ್ಣ ನಡೆಸಿದ್ದಾನೆ. ಇವುಗಳಿಗೆ ಸಂಬಂಧಪಟ್ಟಂತೆ ವಿಮರ್ಶಾತ್ಮಕ ಹಾಗೂ ಸಂಶೋಧನಾತ್ಮಕ ಲೇಖನಗಳು ಇವನಿಂದ ಪ್ರಕಟಗೊಂಡಿವೆ. ಅನೇಕ ವಿದ್ವದ್ಗೋಷ್ಠಿಗಳಲ್ಲಿ ವಿಷಯ ಮಂಡಿಸಿದ, ವಾಕ್ಯಾರ್ಥ ಸಭೆಗಳ ಶಾಸ್ತ್ರಮಥನ ಕಾರ್ಯಗಳಲ್ಲಿ ಭಾಗವಹಿಸಿದ ಹಿರಿಮೆ ಇವನದು. ಹಾಂಗ್‍ಕಾಂಗ್‍ನಲ್ಲಿ ನಡೆದ ದಕ್ಷಿಣ ಏಷ್ಯಾ ಯೋಗ ಸಮ್ಮೇಳನದಲ್ಲಿ ಸಾಂಖ್ಯ–ಯೋಗಗಳ ಕುರಿತು ಮೌಲಿಕ ವಿಷಯ ಮಂಡನೆ ಮಾಡಿದ ಶ್ರೇಯಸ್ಸು ಇವನದು. 

ಮೈಸೂರಿನ ಪ್ರಾಥಮಿಕ ಹಾಗೂ ಪ್ರೌಢಶಾಲಾ ವಿದ್ಯಾರ್ಥಿಗಳಿಗೆ ರಾಮಾಯಣ ಹಾಗೂ ಮಹಾಭಾರತ ಪರೀಕ್ಷೆಯನ್ನು ಆಯೋಜಿಸಿ ಸಾಂಸ್ಕೃತಿಕ ಜಾಗೃತಿಯನ್ನು ಮಕ್ಕಳಲ್ಲಿ ಉಂಟುಮಾಡುವ ಸ್ತುತ್ಯರ್ಹ ಕೆಲಸ ಇವನಿಂದಾಗಿದೆ. ಸಂಸ್ಕೃತ ಪಠ್ಯರಚನಾ ಸಮಿತಿಯ ಸದಸ್ಯನಾಗಿ ಛಾತ್ರೋನ್ನತಿಕಾರಕ ಹಾಗೂ ದೋಷ ರಹಿತ ಪಠ್ಯನಿರ್ಮಾಣದಲ್ಲಿ ಭಾಗಿಯಾಗಿದ್ದಾನೆ. ಪುಸ್ತಕ ರಚನೆಯಲ್ಲಿಯೂ ಹಿಂದೆ ಬಿದ್ದಿಲ್ಲ.

ಮೈಸೂರಿನಿಂದ ಪ್ರಕಟವಾಗುವ ಸಂಸ್ಕೃತ ಸುಧರ್ಮಾ ದಿನಪತ್ರಿಕೆಯ ಗೌರವ ಸಂಪಾದಕನಾಗಿ, ನಿಯತವಾಗಿ ಆ ಪತ್ರಿಕೆಯಲ್ಲಿ ಲೇಖನಗಳನ್ನು ಪ್ರಕಟಿಸುವುದರೊಂದಿಗೆ, ಸುದ್ದಿ ಮುಂತಾದವುಗಳನ್ನು ಸಂಪಾದಿಸಿ, ಭಾಷಾಂತರಿಸಿಕೊಡುವುದರೊಂದಿಗೆ ಅದರ ಜೀವನಾಡಿಗಳ್ಳಲ್ಲೊಬ್ಬನಾಗಿರುವುದು ಅಭಿನಂದನಾರ್ಹ ವಿಷಯ.

ಗಂಗಣ್ಣನ ವಿದ್ವತ್ತು, ಉಪನ್ಯಾಸ ಕಲೆ, ಸಮಾಜ ಸೇವೆ ಮುಂತಾದವುಗಳನ್ನು ಪರಿಗಣಿಸಿ, ಅವನನ್ನು ಅರಸಿಬಂದ ಪ್ರಶಸ್ತಿ, ಸಮ್ಮಾನ ಹಾಗು ಗೌರವಗಳು ಅನೇಕ.

‘ಅನ್ನದಾನ ಹಾಗೂ ವಿದ್ಯಾದಾನ’ – ಇವೆರಡು ಶ್ರೇಷ್ಠ ದಾನಗಳೆನಿಸಿಕೊಂಡಿವೆ. ಅನ್ನದಾನ ಜೀವನಕ್ಕೆ ಆಧಾರವಾದರೆ, ವಿದ್ಯಾದಾನ ಬುದ್ಧಿಯ ಪರಿಷ್ಕಾರಕ್ಕೆ ಸೋಪಾನ. ಇವೆರಡೂ ಗಂಗಣ್ಣನಿಂದ ಶಕ್ತಿ ಮೀರಿ ನಡೆದಿವೆ. 

ಒಡಹುಟ್ಟಿದವರು ಹೇಮಾವತಿ, ರತ್ನಾವತಿ, ಪ್ರವೀಣ, ರಾಮಚಂದ್ರ, ಅರವಿಂದ, ಆನಂದ, ವಸಂತ, ಭವ್ಯಾ ಮುಂತಾದವರಾದರೆ, ಸಂಬಂಧಿಕರು, ತಾಲ್ಲೂಕು ಮತ್ತು ಜಿಲ್ಲೆಯವರಾದ ಸಂಕರಗದ್ದೆ ಪರಮೇಶ್ವರ ಭಟ್ಟ, ಗೋಳಿಕೈ ಮಂಜುನಾಥ, ಹೊನ್ನೆಹದ್ದ ಗಣಪತಿ, ಶೀಬಳಿ ಶ್ರೀಪಾದ ಹೆಗಡೆ, ಕಲ್ಮನೆ ರಾಘವ ಹೆಗಡೆ, ಅಗ್ಗೆರೆ ಪರಮೇಶ್ವರ ಹೆಗಡೆ, ಅರಿಶಿನಗೋಡು ಗುರುಪ್ರಸಾದ, ಸಾಸ್ತಾನದ ವಿಜಯ ಮಂಜ ಮುಂತಾದವರು ಇವನ ಆಶ್ರಯ ಹಾಗೂ ಮಾರ್ಗದರ್ಶನದಲ್ಲಿ ತಮ್ಮ ವಿದ್ಯಾಭ್ಯಾಸವನ್ನು ಯಶಸ್ವಿಯಾಗಿ ಪೂರೈಸಿದವರು. ಇವನ ಮಾರ್ಗದರ್ಶನದಂತೆ ಮೈಸೂರಿನಲ್ಲಿ ನೆಲೆಸಿ ವಿದ್ಯಾಭ್ಯಾಸ ಪೂರೈಸಿದವರು ಅಗಣಿತರು.

ವಿದ್ಯಾರ್ಥಿಗಳಷ್ಟೇ ಅಲ್ಲದೆ, ಸಾಂಸಾರಿಕವಾಗಿ ಹಾಗೂ ಔದ್ಯೋಗಿಕವಾಗಿ ಸಮಸ್ಯೆಯನ್ನು ಎದುರಿಸುತ್ತಿದ್ದ ಹಾರಳ್ಳೆ ಮಹಾಬಲೇಶ್ವರ ಶರ್ಮಾ, ಮುದ್ರಣೋದ್ಯಮಿಯಾದ ಜ್ಞಾನಶಂಕರ ಮುಂತಾದವರು ಗಂಗಣ್ಣನಿಂದ ಬಹೂಪಕೃತರಾಗಿದ್ದಾರೆ.

ಎಲ್ಲವನ್ನು ವಿಮರ್ಶೆಯ ತಕ್ಕಡಿಯಲ್ಲಿ ತೂಗಿ, ಕಾರ್ಯರೂಪಕ್ಕೆ ತರುವ, ತನಗೆ ಸರಿಯೆನಿಸಿದ್ದನ್ನು ನಿರ್ದಾಕ್ಷಿಣ್ಯವಾಗಿ ಪ್ರತಿಪಾದಿಸುವ ಪ್ರವೃತ್ತಿ ಗಂಗಣ್ಣನದು. ಡಂಬಾಚಾರವನ್ನು ಕಟುವಾಗಿ ವಿರೋಧಿಸುವ ಇವನು ಪರರನ್ನು ಓಲೈಸುವ ಕಾರ್ಯದಿಂದ ಬಹುದೂರ. ಹೇಳಬೇಕಾದದ್ದನ್ನು ಮೊನಚಾಗಿ ಮನಸ್ಸಿಗೆ ನಾಟುವಂತೆ, ಕಡಿಮೆ ಶಬ್ದಗಳಲ್ಲಿ ಹೇಳುವ ಇವನ ಮೋಡಿಗೆ ಒಳಗಾಗದವರು ವಿರಳವೆಂದೇ ಹೇಳಬೇಕು. ಇವನ್ನೆಲ್ಲ ಅರ್ಥಮಾಡಿಕೊಳ್ಳದ ಕೆಲವರು ಇವನನ್ನು ಅಹಂಕಾರಿಯೆಂದುಕೊಂಡರೆ ಆಶ್ಚರ್ಯವಿಲ್ಲ.

ನಾನು ವಯಸ್ಸಿನಲ್ಲಿ ಚಿಕ್ಕವನಾದರು, ತನ್ನ ಅಂತರಂಗವನ್ನು ಅನೇಕ ಬಾರಿ ಗಂಗಣ್ಣ ಬಹಿರಂಗಗೊಳಿಸಿದ್ದಿದೆ. ಕುಟುಂಬದ ವಿಷಯಗಳನ್ನು ಚರ್ಚಿಸಿ ಭಿನ್ನಾಭಿಪ್ರಾಯವನ್ನು ವ್ಯಕ್ತಗೊಳಿಸಿದ್ದಿದೆ. ಅಂತೆಯೆ, ನಾನೂ ನನ್ನ ಮನದಾಳದ ಮಾತನ್ನು ಇವನಲ್ಲಿ ತೋಡಿಕೊಂಡಿದ್ದೇನೆ. ಸರಿ ಕಾಣದ್ದಕ್ಕೆ ಅಸಮ್ಮತಿಸಿದ್ದೇನೆ. ಇವಾವವೂ ನಮ್ಮಿಬ್ಬರನ್ನು ಬೇರ್ಪಡಿಸದೆ, ಇನ್ನೂ ಹತ್ತಿರ ತಂದಿವೆ.

ಸುಖ ಬಂದಾಗ ಹಿಗ್ಗುವ ದುಃಖ ಬಂದಾಗ ಕುಗ್ಗುವ ಜಾಯಮಾನ ಗಂಗಣ್ಣನದ್ದಲ್ಲ. ಎರಡನ್ನೂ ಸಮನಾಗಿ ಸ್ವೀಕರಿಸಿ, ಲಭಿಸಿದ್ದನ್ನು ತನ್ನ ಸೌಭಾಗ್ಯವೆಂದು ಸ್ವೀಕರಿಸುವ ಆದರ್ಶ ಗುಣ ಇವನದು. ನನ್ನ ಸುಖ–ದುಃಖಗಳೆರಡರಲ್ಲು ಸಮನಾಗಿ ಭಾಗಿಯಾದ ಗಂಗಣ್ಣನನ್ನು ಮಿತ್ರನೆಂದು ಪರಿಗಣಿಸುವುದೇ ಉಪಾದೇಯ. ಶ್ರೇಯಸ್ಕರವು ಹೌದು. ‘\textbf{ತನ್ಮಿತ್ರಮಾಪದಿ ಸುಖೇ ಚ ಸಮಕ್ರಿಯಂ ಯತ್}’ – ಎಂಬ ಕವಿ ಭರ್ತೃಹರಿಯ ವಾಣಿ ಇದನ್ನು ಪುಷ್ಟೀಕರಿಸುತ್ತದೆ.

ಅಣ್ಣನಾದವನು ಈ ತೆರನಾಗಿ ಸನ್ಮಿತ್ರನಾಗಿ ನನ್ನ ಜೀವನಕ್ಕೆ ಆಸರೆಯಾದುದು ಯೋಗಾಯೋಗ. ಈ ಮಿತ್ರತ್ವ ಚಿರಾಯುವಾಗಲೆಂಬುದೆ ಅಂತರಂಗದ ಪ್ರಾರ್ಥನೆ.

\articleend
