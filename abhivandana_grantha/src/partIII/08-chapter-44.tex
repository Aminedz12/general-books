{\fontsize{14}{16}\selectfont
\chapter{ಪ್ರಿಯ ಅಣ್ಣ   \eng{-}   ಗಂಗಣ್ಣ} 

\begin{center}
\Authorline{ರತ್ನಾವತೀ. ವಿ.ಭಟ್ಟ}
\smallskip

(ವೇದಾವತೀ, ಲೀಲಾವತೀ, ಹೇಮಾವತೀ)\\
ನಂ.1114/1.ಗೀತಾ ರೋಡ್\\
ಚಾಮರಾಜಪುರಂ,\\ 
ಮೈಸೂರು
\addrule
\end{center}

ನನ್ನ ಪ್ರೀತಿಯ ಅಣ್ಣ ಗಂಗಣ್ಣನೊಂದಿಗಿನ ಸವಿ ನೆನಪಿನ ಸುರುಳಿಯ ಸರಮಾಲೆ\-ಗಳನ್ನು ಬಿಚ್ಚಿಡುವ ಸುಸಂದರ್ಭ ಈಗ ಬಂದೊದಗಿದೆ.

ಉತ್ತರಕನ್ನಡ ಜಿಲ್ಲೆಯ ಸಿದ್ದಾಪುರ ತಾಲ್ಲೂಕಿನ ನಾಲಿಗಾರ ಗ್ರಾಮದ ಅಗ್ಗೇರೆ ನಿವಾಸಿ\-ಯಾದ ದಿವಂಗತ ವೇದಮೂರ್ತಿ ಶ್ರೀ ವಿಘ್ನೇಶ್ವರ ಗಣಪ ಭಟ್ ಮತ್ತು ದಿ. ಶ್ರೀಮತಿ ರೇವತಿ ಇವರ ತೃತೀಯ ಪುತ್ರನಾದ ಗಂಗಾಧರ ವಿ. ಭಟ್ಟ  ಅಪ್ಪ ಪ್ರೀತಿಯಿಂದ ಕರೆಯುತ್ತಿರುವ ಗಂಗ್ಯಾ ರವರೇ ಇವತ್ತಿನ ವಿದ್ವಾನ್ ಶ್ರೀ ಗಂಗಾಧರ ವಿ. ಭಟ್ಟರವರು.

ಇವರು ಅಗ್ಗೇರೆಯ ಒಟ್ಟು ಕುಟುಂಬದಲ್ಲಿ ಬೆಳೆದು ಪ್ರಾಥಮಿಕ ಶಿಕ್ಷಣವನ್ನು ಕವಲಕೊಪ್ಪದಲ್ಲಿ ಮುಗಿಸಿ, ಪ್ರೌಢ  ಶಿಕ್ಷಣವನ್ನು ನಮ್ಮ ದೊಡ್ಡಪ್ಪನ ಮನೆ ತಟ್ಟೀಸರದಲ್ಲಿ ಇದ್ದಕೊಂಡು ನಮ್ಮ ಅಕ್ಕನಿಗೆ ಜೊತೆಯಾಗಿ  ಬಿದ್ರಕಾನ್ ಹೈಸ್ಕೂಲಿಗೆ ಹೋಗಿ ಓದಿರುತ್ತಾರೆ.

ತದನಂತರ, ಮುಂದಿನ ವಿದ್ಯಾಭ್ಯಾಸಕ್ಕಾಗಿ ಮೈಸೂರನ್ನು ಅರಸಿಕೊಂಡು ಬಂದರು. ಹಿಂದಿನಿಂದಲೂ ನಮ್ಮ ಮನೆಗೂ ಮೈಸೂರಿಗೂ ಅವಿನಾಭಾವ ಸಂಬಂಧ ಎಂದರೂ ತಪ್ಪಿಲ್ಲ.  ಅಣ್ಣಂದಿರೆಲ್ಲ ಮೈಸೂರಿನಲ್ಲೇ ಓದಿರುವರು. ಆದ್ದರಿಂದ ಗಂಗಣ್ಣನವರಿಗೂ ಮನಸ್ಸಿನಲ್ಲಿ ಮೈಸೂರಿನಲ್ಲೇ ಓದಬೇಕೆಂಬ ಹಂಬಲ ಬಂದಿರಲೂಬಹುದು.  ಅವರೆಲ್ಲರ ಆಶೀರ್ವಾದ ಮತ್ತು ಅಣ್ಣನ ಸ್ವಯಂ ಪ್ರಯತ್ನ ಇದೆಲ್ಲದರ ಫಲಿತಾಂಶವೇ ಇವರು ಇಂದು ಮೈಸೂರಿನಲ್ಲಿ ಅತ್ಯಂತ ಕೀರ್ತಿ ಗಳಿಸಿ, ಅವರ ಹೆಸರು ರಾರಾಜಿಸುವಂತಾಗಿದೆ.

ಅಣ್ಣನವರಿಗೆ 31 ನೇ 2018 ಜನವರಿ ಮಾಹೆಯಲ್ಲಿ ಸರ್ಕಾರಿ ನಿಯಮಾವಳಿಯ ಪ್ರಕಾರ ಉಪಾಧ್ಯಾಯ ವೃತ್ತಿಯಿಂದ  ನಿವೃತ್ತಿ.  ಆದರೆ,  ಪಾಠ  \enginline{-}  ಪ್ರವಚನ, ಬರವಣಿಗೆ, ವಿದ್ಯಾದಾನ, ಅತಿಥಿ ಸತ್ಕಾರ, ಸಹಾಯ ಹಸ್ತ ನೀಡುವ ಅವರ ಕಾರ್ಯಕ್ಕೆ ಯಾವತ್ತೂ  ನಿವೃತ್ತಿ ಇಲ್ಲ. ನಿವೃತ್ತರಾಗುವುದು ಅವರಿಂದ ಸಾಧ್ಯವೂ ಇಲ್ಲ.

ನಮ್ಮ ಅಣ್ಣನ ಬಗ್ಗೆ ಹೇಳಲು ನಮಗೆ ಪದಗಳೇ ಇಲ್ಲ. ಅರ್ಹರು ಅಲ್ಲವೇನೋ ಎಂಬ ಸಣ್ಣ ಸಂಶಯವೂ ನಮ್ಮೊಳಗಿದೆ.  ಆದರೂ ಸಹ ಧೈರ್ಯ ಮಾಡಿ ಈ ಲೇಖನ ಬರೆಯುತ್ತಿದ್ದೇನೆ.  ಇದರಲ್ಲಿರುವ ತಪ್ಪುಗಳನ್ನೆಲ್ಲಾ ತಿದ್ದಿಕೊಂಡು   \enginline{-}   ಕ್ಷಮಿಸಿ ಬಿಡಿ.  ಏಕೆಂದರೆ, ಅವರು ಬರೆಯುವ ತರಹ ನಮಗೆ ಬರೆಯುವ ಅಭ್ಯಾಸವೂ ಇಲ್ಲ ಅಷ್ಟೊಂದು ಜ್ಞಾನವುಳ್ಳವರಂತು ಅಲ್ಲವೇ ಅಲ್ಲ.

ನನಗೂ ನನ್ನ ಅಣ್ಣನಿಗೂ 10 ವರ್ಷ ಅಂತರ. ನನಗೆ ಸ್ವಲ್ಪಮಟ್ಟಿನ ತಿಳುವಳಿಕೆ ಬರವುದರೊಳಗೆ ಅವರು ಮೈಸೂರು ಪ್ರಾಂತ್ಯವನ್ನು ಸೇರಿಯಾಗಿತ್ತು.  ನಮಗೆ ಅಣ್ಣನ ಸಾಮೀಪ್ಯ ದೊರಕುವುದೆಂರೆ ಅವರು ರಜಾದಲ್ಲಿ ಊರಿಗೆ ಬಂದಾಗ ಮಾತ್ರ.  ಅವರು ಮೈಸೂರಿನಿಂದ ಮನೆಗೆ ಬರುತ್ತಿದ್ದಾರೆಂದರೆ ನಮಗೆ ಎಲ್ಲಿಲ್ಲದ ಸಂಭ್ರಮ ಹಾಗು ಆನಂದ.  ಏಕೆಂದರೆ, ಬರುವಾಗ ನಮಗೋಸ್ಕರ ಪೆನ್ನು, ತಿಂಡಿ ಮುಂತಾದ ವಸ್ತುಗಳನ್ನು ತರುವರು. ಆವಾಗ ನಮಗೆ ಮೈಸೂರೆಂದರೆ ಫಾರಿನ್ ಇದ್ದ ಹಾಗೆ. 

ಎಲ್ಲರಿಗೂ ನಮ್ಮಣ್ಣ ಪೆನ್ನು ತಂದಿದ್ದಾರೆಂದು ತೋರಿಸುವುದು. ಆ ಪೆನ್ನು ಉದ್ದ\-ವಾಗಿದ್ದು ಒಳ್ಳೆ ಘಮಘಮ ವಾಸನೆ ಬರುತ್ತಿತ್ತು.  ನನಗೆ ಒಂಥರ ಜಂಬ. ಎಲ್ಲರೂ, “ನಿಂಗೆ ಚಲೊವಾ ! ಹದಾ ! ನಿನ್ನಣ್ಣ ದೂರದಿಂದ ಎಲ್ಲ ತಂದು ಕೊಡ್ತಾರೆ ! ನಮಗೆ ಆ ಭಾಗ್ಯ ಇಲ್ಲ !” ಅಂತ ವಿಷಾದಿಸುತ್ತಿದ್ದರು.  ಅವರು ಮನೆಗೆ ಬಂದಾಗ  ಮನೆಯಲ್ಲಿ ಹಬ್ಬದ ವಾತಾವರಣ.  

ಮೂರು ಜನ ಅಣ್ಣಂದಿರು ಒಟ್ಟಿಗೆ ಕೂತು ಊಟ ಮಾಡುತ್ತಿದ್ದರು. ಒಟ್ಟಿಗೆ ಸೇರಿ ಕೆಲವು ವಿಷಯಗಳ ಬಗ್ಗೆ ಚರ್ಚೆ ನಡೆಸುವರು, ಹಾಗೇ ಹರಟೆಯನ್ನು ಹೊಡೆಯವರು.  ಎಲ್ಲರೂ ಸೇರಿ ಒಟ್ಟಾಗಿ ಗದ್ದೆ ತೋಟದ ಕೆಲಸ ಮಾಡುತ್ತಿದ್ದರು. ಇಂದಿಗೂ ಆ ದೃಶ್ಯಗಳು ಕಣ್ಣಿಗೆ ಕಟ್ಟಿದ ಹಾಗಿವೆ.  ಅಣ್ಣ ತಮ್ಮ ಎಂದರೆ ಹೀಗಿರಬೇಕು ಎಂದು ನಾವು ಅಕ್ಕತಂಗಿಯರು ಮಾತನಾಡಿಕೊಳ್ಳುತ್ತಿದ್ದೆವು. ಅವರು ಊರಿಗೆ ಬಂದರೆ ಕೆಲಸದಾಳಿಂದ ಹಿಡಿದು ಎಲ್ಲರೂ ಖುಷಿಪಡುವವರೆ. ಯಾಕೆಂದರೆ, ಎಲ್ಲರನ್ನೂ ತಮಾಷೆ ಮಾಡುತ್ತಾ ಪ್ರೀತಿಯಿಂದ ಮಾತನಾಡುತ್ತಾರೆಂದು.  ನಮ್ಮನ್ನೆಲ್ಲಾ ಕೀಟಲೇ ಮಾಡುತ್ತಾ, ಅಮ್ಮನನ್ನು ರೇಗಿಸುತ್ತಾ ಹಬ್ಬಹರಿದಿನಗಳಲ್ಲಿ ಅಮ್ಮನ ಜೊತೆ ಹಾಡು ಹೇಳುತ್ತಾ ಅಡಿಗೆ ಮನೆಯಲ್ಲಿ ಸಹಾಯ ಮಾಡುತ್ತಾ ಇರುತ್ತಿದ್ದರು.  ಆ ದಿನಗಳ ನೆನಪು ಸದಾ ಹಸಿರಾಗಿದೆ.

ಹೀಗೆ ನಮ್ಮ ಬಾಲ್ಯದ ದಿನಗಳು ಕಳೆದು ನಾನು ಎಸ್.ಎಸ್.ಎಲ್.ಸಿ. ಮುಗಿದ ಬಳಿಕ ನನ್ನನ್ನು ಮೈಸೂರಿಗೆ ಕರೆದುಕೊಂಡು ಬಂದರು. ಅದಕ್ಕೂ ಮೊದಲು ಹೇಮಕ್ಕನನ್ನು ಕರೆದು\-ಕೊಂಡು ಬಂದಿದ್ದರು. ಅವಳ ಸಂಪೂರ್ಣ ಜವಾಬ್ದಾರಿ ಇವರದ್ದೆ.  ನನ್ನ ಮುಂದಿನ ವಿದ್ಯಾರ್ಜನೆ ಬಗ್ಗೆ ನನಗೆ ಏನೂ ಅರಿವೇ ಇಲ್ಲದಂತಹ ಸಂದರ್ಭದಲ್ಲಿ ಸೂಕ್ತ ವಿದ್ಯಾಭ್ಯಾಸ ಕೊಡಿಸುವ ನಿಟ್ಟಿನಲ್ಲಿ  ಸಂಪೂರ್ಣ  ಜವಾಬ್ದಾರಿಯನ್ನು ಹೊತ್ತುಕೊಂಡು ಕಾಲೇಜಿಗೆ ಸೇರಿಸಿದರು. ಇಷ್ಟಕ್ಕೂ ನನಗೆ ನನ್ನ ಬಗ್ಗೆ ವಿದ್ಯಾಭ್ಯಾಸದ ಬಗ್ಗೆ ಕಿಂಚಿತ್ ವಿಚಾರ ಮಾಡುವ ಜ್ಞಾನವೂ ಕೂಡ ಇರಲಿಲ್ಲ. ಓದಿಸ್ತೀನಿ ಅಂದ್ರು. ಓದೋದಕ್ಕೆ ಅಂತ ಬಂದೆ. 

ನಾನು ಏನಾಗಬೇಕು, ಓದಿ ಜ್ಞಾನ ಸಂಪಾದನೆ ಮಾಡಬೇಕು ಎನ್ನುವ ಗುರಿ ಇರಲೇ ಇಲ್ಲ. ನಮ್ಮ ತಂದೆ ತಾಯಿ ಕೂಡ ನಮ್ಮ ಬಗ್ಗೆ ಅಷ್ಟೊಂದು ತಲೆಕೆಡಿಸಿಕೊಳ್ಳದ ಕಾಲದಲ್ಲಿ ಅಣ್ಣ ಮಾತ್ರ ತಂಗಿಯರ ಸಂಪೂರ್ಣ ಜವಾಬ್ದಾರಿ ವಹಿಸಿಕೊಂಡು ಮೈಸೂರಿಗೆ ಬರ\-ಮಾಡಿಕೊಂಡರು.  ನಮಗೆ ಊರು ಬಿಟ್ಟರೆ ಬೇರೆ ಯಾವ ದೊಡ್ಡ ಸಿಟಿಯ ಬಗ್ಗೆ ಸ್ವಲ್ವವೂ ತಿಳುವಳಿಕೆ ಇರಲಿಲ್ಲ. ಆದರೆ, ಅವರು ಕಿಂಚಿತ್ ಬೇಸರವಿಲ್ಲದೇ ನಮ್ಮನ್ನು ಇಲ್ಲಿಗೆ ಕರೆದು\-ಕೊಂಡು ಬಂದು ಇಂದಿನವರೆಗೂ ಅವರ ಸಂಪೂರ್ಣ ಜವಾಬ್ದಾರಿಯಲ್ಲೇ ನೋಡಿ\-ಕೊಳ್ಳುತ್ತಿದ್ದಾರೆ.  ಇದು ನಮ್ಮೆಲ್ಲರ ಅತ್ಯಂತ ಸೌಭಾಗ್ಯ ಮತ್ತು ಹೆಮ್ಮೆಯ ಸಂಗತಿ. 

ನಮಗೆ ಯಾವತ್ತೂ ಕೊರತೆ ಬರುವ ಹಾಗೆ ಮಾಡಲೇ ಇಲ್ಲ. ತಂದೆ   \enginline{-}   ತಾಯಿಯಿಂದ ದೂರವಾಗಿದ್ದೀವಿ ಅನ್ನಿಸಲೇ ಇಲ್ಲ. ವಿದ್ಯಾಭ್ಯಾಸದ ಖರ್ಚು ವೆಚ್ಚ ಎಲ್ಲ ಇವರೇ ನೋಡಿಕೊಳ್ಳುತ್ತಿದ್ದರು.  ನಾವು ಸಂಕೋಚದಿಂದ ಫೀಸ್ ಕೇಳಿದರೆ ಕೇಳಿದ್ದಕ್ಕಿಂತ ಜಾಸ್ತಿನೇ ಮರು ಮಾತನಾಡದೇ ಅದರ ಲೆಕ್ಕವನ್ನು ಕೇಳದೇ ನಮಗೆ ಕೊಡುತ್ತಿದ್ದರು.  ಇವತ್ತಿಗೂ ಅವರು ಲೆಕ್ಕವಿಡುವ ಅಭ್ಯಾಸ ಇಟ್ಟುಕೊಂಡಿಲ್ಲ.

ನಾವು ಮೈಸೂರಿಗೆ ಬಂದಾಗ ಅವರು ಶಂಕರ ವಿಲಾಸ  ಪಾಠಶಾಲೆಯಲ್ಲಿ ಉಪಾ\-ಧ್ಯಾಯ\-ರಾಗಿದ್ದರು.  ಅವರ ಸಂಬಳ ತುಂಬಾ ಕಡಿಮೆ ಇತ್ತು. ಆದರೆ ನಮಗೆ ಯಾವತ್ತೂ ಯಾವುದಕ್ಕೂ ಕಡಿಮೆ ಮಾಡಲೇ ಇಲ್ಲ.  ಯಾವ ರೀತಿ ಕಷ್ಟಪಡುತ್ತಿದ್ರೋ ಆ ಭಗವಂತನಿಗೆ ಗೊತ್ತು.  ಇವತ್ತಿಗೂ ಆ ದಿನಗಳನ್ನ ನೆನಸಿಕೊಂಡರೆ ಕಣ್ಣಂಚಿನಲ್ಲಿ ನೀರು ಜಿನುಗುತ್ತದೆ. ನಮ್ಮ ಜೊತೆ ಸಾಕಷ್ಟು ಜನರು ಮನೆಗೆ ಬಂದು   \enginline{-}   ಹೋಗಿ ಮಾಡುತ್ತಿದ್ದರು. ಯಾವತ್ತೂ ನಮಗೆ ಕಷ್ಟ ಅಂತ ಅವರನ್ನೆಲ್ಲ ನಿರ್ಲಕ್ಷ್ಯ ಮಾಡಲೇ ಇಲ್ಲ.

ಆ ದಿನಗಳಲ್ಲಿ ಪ್ರತಿ ದಿನವೂ ಮನೆಯಲ್ಲಿ ನೆಂಟರು. ಊರಿಂದ ವಿದ್ಯಾಭ್ಯಾಸ\-ಕ್ಕಾಗಿ ಮೈಸೂರಿಗೆ ಬರುವವರ ಸಂಖ್ಯೆ ತುಂಬಾ ಹೆಚ್ಚಾಗಿತ್ತು.  ಪ್ರತಿಯೊಬ್ಬರು ಅಣ್ಣನ\break ಮನೆಯಲ್ಲೇ ಇರುತ್ತಿದ್ದರು.  ಕಾಲೇಜಿಗೆ ಪ್ರವೇಶ ಮಾಡಿಕೊಡುವುದರಿಂದ ಹಿಡಿದು ಬೇರೆ ಕಡೆ ಊಟ   \enginline{-}   ವಸತಿ ವ್ಯವಸ್ಥೆ ಮಾಡಿಕೊಳ್ಳುವವರೆಗೂ ಅಣ್ಣನ ಮನೆಯಲ್ಲೇ ಊಟ   \enginline{-}   ವಸತಿ, ಪ್ರತಿಯೊಂದು ನಿರಂತರವಾಗಿ ನಡೆಯುತ್ತಿತ್ತು.  ಇದಲ್ಲದೇ ಮೈಸೂರು ನೋಡಲು ಬಂದವರಿಗೆ, ಉದ್ಯೋಗಕ್ಕಾಗಿ ಬಂದವರಿಗೆ, ಮೌಲ್ಯಮಾಪನಕ್ಕಾಗಿ ಬಂದ ಗುರುಜನರಿಗೆ ಅಣ್ಣನ ಮನೆಯೇ  ಮನೆ ಆಗಿರುತ್ತಿತ್ತು.  ಸಂಕೋಚವಿಲ್ಲದೇ  ಪ್ರತಿಯೊಬ್ಬರು ಅಲ್ಲಿಯೇ ವಾಸ್ತವ್ಯ ಮಾಡುತ್ತಿದ್ದರು.  

ಇದಕ್ಕೆ ಕಾರಣ ಅಣ್ಣನವರ ಸ್ವಭಾವ. ಬರುವ ಅಲ್ಪ ಸಂಬಳದಲ್ಲಿ ಕಿಂಚಿತ್ತೂ ಬೇಸರ\-ವಿಲ್ಲದೇ ಸ್ವಾರ್ಥವಿಲ್ಲದೇ ಜಾತಿ ಮತ ಪಂಥದ ಭೇದವಿಲ್ಲದೇ ಎಲ್ಲರಿಗೂ ಸಮಾನವಾಗಿ ವ್ಯವಸ್ಥೆ ಮಾಡುತ್ತಿದ್ದರು.  ಎಷ್ಟೋ ಸಮಯ ಎಲ್ಲರಿಗೂ ವ್ಯವಸ್ಥೆ ಮಾಡುವುದು ಕಷ್ಟಕರ\-ವಾದರೂ ಸಹ ಧೃತಿಗೆಡದೇ ಯಾರಿಗೂ ಗೊತ್ತಾಗದ ಹಾಗೆ ಆ ನೋವನ್ನು ತಮ್ಮಲ್ಲೇ\break ಇಂಗಿಸಿಕೊಂಡು ಸಂಸಾರ ನೌಕೆಯನ್ನು ಶಾಂತವಾಗಿ ನಡೆಸುತ್ತಿದ್ದರು.

ವಿದ್ಯಾರ್ಜನೆಯ ಕಾಲದಲ್ಲಿ ಅವರು ವಾರಾನ್ನ ಮಾಡಿಕೊಂಡು ವಿದ್ಯಾಭ್ಯಾಸ ಮಾಡಿದಂಥವರು.  ಅದಕ್ಕೆ ಇರಬೇಕು ವಿದ್ಯೆಗೆ ಎಷ್ಟು ಮಹತ್ವ ಇದೆ, ಅದರ ಕಷ್ಟವೇನು ಎಂದು ಸ್ವತ: ಅನುಭವಿಸಿದ್ದು.  ಬೇಕಾದಷ್ಟು ನೋವು ಇದ್ದರೂ ತಮ್ಮ ಓದಿನ ಜೊತೆಗೆ ಇತರೆ ಚಟುವಟಿಕೆಗಳಲ್ಲಿ ಭಾಗವಹಿಸಿ ತಮ್ಮ ಛಾಪನ್ನು ಮೂಡಿಸಿದವರು.  ವಿದ್ಯಾದಾನ, ಅನ್ನದಾನ ಮಾಡುವುದರಲ್ಲೇ ಅವರು ತೃಪ್ತಿ ಕಂಡುಕೊಳ್ಳುತ್ತಿರುವರು.  ಇವರಿಗೆ ಮೊದಲಿನಿಂದಲೂ ತನಗಾಗಿ ಏನೂ ಮಾಡಿಕೊಳ್ಳುವ ಸ್ವಭಾವವಿಲ್ಲ. ಇದು ರಕ್ತಗತವಾಗಿ ಬಂದಂತಹ ಗುಣ. ಬೇರೆಯವರ ಕಷ್ಟಕ್ಕೆ ಬಲು ಬೇಗ ಕರಗಿ ಹೋಗುವಂಥವರು. ಎಷ್ಟೋ ಸಮಯ ಬೇರೆಯವರಿಗೆ ಊಟ ನೀಡಿ ತಾವು ಉಪವಾಸ ಮಲಗಿರುವ ಸಂದರ್ಭವೂ ಇತ್ತು.  

ಇವರು ಬಹುಮುಖ ಪ್ರತಿಭೆ.   ಪ್ರವಚನ ಮಾಡಿದರೆ ಕರ್ಣಾನಂದದ ಜೊತೆಗೆ ಮನಸ್ಸಿ\-ಗಾನಂದವೂ ದೊರಕುತ್ತದೆ.  ಅವರು ಮಾತನಾಡುವ ಪರಿ ಎಂಥವರಿಗೂ ಮುದ ನೀಡುತ್ತದೆ.  ಪಾಠ   \enginline{-}   ಪ್ರವಚನದ ಜೊತೆಗೆ ಲೇಖನ ಬರೆಯುವುದು, ಚರ್ಚಾಸ್ಪರ್ಧೆಗೆ ಪರ ವಿರೋಧ ವಿಷಯಗಳ ಬಗ್ಗೆ ತಯಾರಿ ಮಾಡಿಕೊಡುವುದಲ್ಲದೇ ಎಲ್ಲರೂ ಆ ವಿಷಯಗಳ ಬಗ್ಗೆ ಚೆನ್ನಾಗಿ ಅರಿತುಕೊಳ್ಳುವಂತೆ ಸಿದ್ಧತೆ ಮಾಡಿ ಕೊಡುತ್ತಾರೆ. ನಾಟಕ ಬರೆಯುವುದು, ನಾಟಕ ಮಾಡಿಸುವುದು ಕೂಡ ಇವರ ಒಂದು ಹವ್ಯಾಸ. ಇವರು ಸಂಸ್ಕೃತವೊಂದರಲ್ಲೇ ಪ್ರವೀಣರಲ್ಲ. ಇವರಲ್ಲಿ ಇಂದಿಗೂ ಕನ್ನಡ, ಇಂಗ್ಲೀಷ್ ಮತ್ತು ಹಿಂದಿ ಭಾಷೆಗಳಲ್ಲಿ ಪಾಠವನ್ನು ಕಲಿತು, ಅವರೆಲ್ಲ ಈ ದಿನ ದೊಡ್ಡ ದೊಡ್ಡ ಸ್ಥಾನದಲ್ಲಿ ಇದ್ದಾರೆ ಎಂದು ಹೇಳುವುದಕ್ಕೆ ತುಂಬಾ ಹೆಮ್ಮೆ ಅನಿಸುತ್ತಿದೆ. ಇಂದಿಗೂ ಬೇರೆ ಬೇರೆ ರಾಜ್ಯ, ಬೇರೆ ಬೇರೆ ದೇಶಗಳಿಂದ ಇವರಲ್ಲಿ ಪಾಠ ಪ್ರವಚನ ಮಾಡಿಸಿಕೊಳ್ಳುವುದಕ್ಕೆ  ಕಾಯುತ್ತಿರುತ್ತಾರೆ.  ಇದಕ್ಕೆಲ್ಲಾ ಅವರ ತನು  \enginline{-}  ಮನ  \enginline{-}  ಧನ ಸಹಾಯದ ಗುಣವೇ ಕಾರಣ.  ಇವರಿಗೆ ಪಾಠ ಪ್ರವಚನ ಮಾಡುವುದರಲ್ಲಿ ಅತಿಯಾದ ಶ್ರದ್ಧೆ, ಆಸಕ್ತಿ ಎಲ್ಲಾ ಇದೆ.  ಯಾವತ್ತೂ ಪಾಠ ಮಾಡುವುದಕ್ಕೆ ಬೇಸರ ಮಾಡಿಕೊಂಡವರೇ ಅಲ್ಲ. ಸ್ವದೇಶದ ಜನರಿಗಲ್ಲದೇ ವಿದೇಶದ ಜನರಿಗೂ ಕೂಡ ಇವರ ಪಾಠ ಅಚ್ಚುಮೆಚ್ಚು. ಜ್ಞಾನಾರ್ಜನೆಗೋಸ್ಕರ  ಅವರ  ಜೀವನವನ್ನೇ ಮುಡಿಪಾಗಿಟ್ಟವರು.  ಅವರಿಗೆ ಓದದೇ ಸುಮ್ಮನೇ ಕಾಲಹರಣ ಮಾಡಿದರೆ ತುಂಬಾ ಬೇಸರ ಹಾಗೂ ಕೋಪ. 

ಇನ್ನು, ಇವರು ಆಡಂಬರದ ಜೀವನವನ್ನು ಎಂದೂ ಇಷ್ಟಪಟ್ಟವರಲ್ಲ. ಯಾವತ್ತೂ ಅವರಿಗಾಗಿ ಬದುಕಿತ್ತಿಲ್ಲವೆಂಬುದು ನೂರಕ್ಕೆ ನೂರು ಸತ್ಯ. ಇವರಿಗೆ ಇರುವ ವಿದ್ಯೆ ಹಿನ್ನಲೆಯಲ್ಲಿ ಮನಸ್ಸು ಮಾಡಿದರೆ ಬೇಕಾದಷ್ಟು ಹಣ, ಕೀರ್ತಿಗಳನ್ನು ಸಂಪಾದಿಸಿಕೊಳ್ಳಬಹುದಿತ್ತು.  ಅದ್ಯಾವುದಕ್ಕೂ ಮನಸ್ಸು ಮಾಡಿದವರಲ್ಲ.  ಸ್ವಲ್ಪದರಲ್ಲೇ ತೃಪ್ತಿಕಾಣುವವರು. 

ಇನ್ನು ಇವರ ವೈವಾಹಿಕ ಜೀವನದ ಬಗ್ಗೆ ಹೇಳಬೇಕೆಂದರೆ, ನಮ್ಮ ಅತ್ತಿಗೆ ಪತಿಗೆ ತಕ್ಕ ಪತ್ನಿ. ಅತ್ತಿಗೆ   \enginline{-}   ಶೈಲಜಾ ಬಗ್ಗೆ ಒಂದೆರಡು ಮಾತು ಇಲ್ಲಿ ಹೇಳಲೇಬೇಕು. ಇವರು ಅಣ್ಣನ ಮನಸ್ಸನ್ನು ಸಂಪೂರ್ಣ ಅರ್ಥಮಾಡಿಕೊಂಡು ಸಂಸಾರ ನಡೆಸುವಂಥ ಮಡದಿ.  ನಮಗೆ ಅಣ್ಣನ ಹತ್ತಿರ ಅಷ್ಟೊಂದು ಸಲುಗೆಯಿಂದ ಮಾತನಾಡುವ ಅಭ್ಯಾಸ ತುಂಬಾ ಕಡಿಮೆ. ಅಣ್ಣನ ಮನಸ್ಸು ತುಂಬಾ ಸೂಕ್ಷ್ಮ ಹಾಗೇ ತುಂಬಾ ತೀಕ್ಷ್ಣ. ಅವರ ಕಣ್ಣಿನ ನೋಟವೇ ನಾವು ಏನು ತಪ್ಪು ಮಾಡುತ್ತಿದ್ದೇವೆ ಎಂಬುದನ್ನು ತೋರಿಸುತ್ತದೆ. ಹಾಗಂತ ನಮ್ಮ ಮಾತನ್ನು ಕೇಳುವುದಿಲ್ಲ ಎಂದಲ್ಲ. ಮುಂಚಿನಿಂದಲೂ ಅವರ ಬಳಿ ಮಾತು ಕಡಿಮೆ.

ಅದಕ್ಕೆ ಎಲ್ಲವನ್ನೂ ಅತ್ತಿಗೆಯ ಜೊತೆ ಹಂಚಿಕೊಳ್ಳುವ ಅಭ್ಯಾಸ. ನಮ್ಮ ಅನಿಸಿಕಗಳಿಗೆ ಯಾವತ್ತೂ ನಕಾರಾತ್ಮಕವಾಗಿ ನಡೆದುಕೊಂಡವರಲ್ಲ.  ಹಾಗಾಗಿ ನಮಗೆ ಅತ್ತಿಗೆಯೆಂದರೆ ತಾಯಿಗಿಂತಲೂ ಹೆಚ್ಚು.  ಅಮ್ಮನ ಬಳಿ ಹೇಳುವುದನ್ನೆಲ್ಲಾ ಅತ್ತಿಗೆ ಹತ್ತಿರವೇ ಹೇಳುವುದು. ಒಳ್ಳೆಯ ಸಹೃದಯಿಯಾದ್ದರಿಂದ ನಿಸ್ಸಂಕೋಚವಾಗಿ ಕಷ್ಟ ಕಾರ್ಪಣ್ಯಗಳನ್ನು, ನೋವು ನಲಿವುಗಳನ್ನು ಸಮಾನವಾಗಿ ಹಂಚಿಕೊಳ್ಳುತ್ತಾರೆ. ಈ ವಿಷಯದಲ್ಲಿ ನಾವುಗಳು ಪುಣ್ಯವಂತರು. ಅತ್ತಿಗೆ ! ನಾವು ನಿಮಗೆ ಸದಾ ಋಣಿಗಳು.

ಇನ್ನು, ನನಗೆ ವಿದ್ಯಾಭ್ಯಾಸ ಕೊಡಿಸಿ, ಉದ್ಯೋಗ ಕೊಡಿಸಿ, ನನ್ನ ಜೀವನ ಉಜ್ವಲ\-ವಾಗುವಂತೆ ಮತ್ತು ನನ್ನ ಸಂಸಾರ ಜೀವನ ಸಂತೋಷದಿಂದ ಇರಲು ಅನುವು ಮಾಡಿಕೊಟ್ಟ ಅಣ್ಣ, ನನ್ನ ಮಗನ ವಿದ್ಯಾಭ್ಯಾಸಕ್ಕೂ ಕೂಡ ಅವರ ಮನೆಯಲ್ಲಿರಿಸಿಕೊಂಡು ಅವರ ಮಾರ್ಗದರ್ಶನದಲ್ಲಿ ಬೆಳೆಸುತ್ತಿದ್ದಾರೆ.  ನನ್ನ ವಿದ್ಯಾಭ್ಯಾಸದ ಜವಾಬ್ದಾರಿ ಹೊತ್ತಿದ್ದ ಅಣ್ಣ  ನನ್ನ ಮಗನನ್ನೂ ತನ್ನ ಮಡಿಲ್ಲಲ್ಲಿಟ್ಟುಕೊಂಡಿರುವುದು ನನ್ನ ಸೌಭಾಗ್ಯವೇ ಸರಿ. ಇದಕ್ಕೆ ನನ್ನ ಅತ್ತಿಗೆಯ ಸಂಪೂರ್ಣ ಬೆಂಬಲವೂ ಕಾರಣವೆಂಬುದನ್ನು ಪ್ರತ್ಯೇಕವಾಗಿ ಹೇಳಬೇಕಾಗಿಲ್ಲ. 

ಈಗ ನನ್ನ ಅಣ್ಣನಿಗೆ 60 ವರ್ಷ. ಸರ್ಕಾರಿ ನಿಯಮದಂತೆ ನಿವೃತ್ತರಾಗುತ್ತಿದ್ದಾರೆ. ಅವರ ಓದು ಮತ್ತು ಪಾಠ  \enginline{-}  ಪ್ರವಚನ ಆದರ  \enginline{-}  ಆತಿಥ್ಯ ಸಹಾಯ ಹಸ್ತ ಈ ಎಲ್ಲ ಸ್ವಭಾವದ ಕೆಲಸಕ್ಕೆ ನಿವೃತ್ತಿ ವಯಸ್ಸು ಇಲ್ಲ.  ನನ್ನ ಮನದಾಳದ ಇಂಗಿತವೇನೆಂದರೆ, ಸದಾ ಅಣ್ಣ ಪಾಠ  \enginline{-}  ಪ್ರವಚನಗಳನ್ನು ಮಾಡಿಕೊಂಡು ಮೈಸೂರಿನಲ್ಲೇ ಇರಬೇಕು.  ಸಹಸ್ರಾರು ಜನರಿಗೆ ಇನ್ನೂ ವಿದ್ಯಾದಾನ ಮಾಡಬೇಕು ಎನ್ನುವುದು.  ಇಲ್ಲಿ ಸ್ವಲ್ಪ ಸ್ವಾರ್ಥವೂ ಇದೆ. ಏಕೆಂದರೆ, ನಾನೊಬ್ಬಳೆ ಇಲ್ಲಿ ಆಗಿಬಿಡುತ್ತೇನೆ ಎಂಬುದು.  ಆದರೆ, ಅವರಿಗೆ ಹುಟ್ಟೂರಿನ ಪ್ರೇಮ ಅಪಾರ.  ಅದಕ್ಕೆ ಅವರಿಗೆ ಎಲ್ಲಿ ಸುಖಶಾಂತಿ ನೆಮ್ಮದಿ ಸಿಗುತ್ತೋ ಅಲ್ಲೇ ಅವರ ವಾಸವಾಗಲಿ. ಎಲ್ಲಿದ್ದರೂ ನಮ್ಮವರೇ.  ಆ ಭಗವಂತ  ನಿವೃತ್ತಿ ಜೀವನವನ್ನು ಹರ್ಷೋತ್ಸಾಹದಿಂದ ಕಳೆಯುವಂತೆ ಆಶೀರ್ವದಿಸಲಿ ಮತ್ತು ಅಣ್ಣ ಅತ್ತಿಗೆಯರಿಗೆ ಆಯಸ್ಸು, ಆರೋಗ್ಯ, ಐಶ್ವರ್ಯ ಎಲ್ಲ ಸೌಭಾಗ್ಯವನ್ನು  ಕರುಣಿಸಿಲಿ ಎಂದು ಭಗವಂತನ ಪದತಲದಲ್ಲಿ ತಲೆಬಾಗಿ ತಂಗಿಯರೆಲ್ಲಾ ಬೇಡಿಕೊಳ್ಳುತ್ತೇವೆ.

ಪ್ರೀತಿಯ ಹಾಗೂ ಗೌರವಪೂರ್ವಕ ನಮಸ್ಕಾರಗಳೊಂದಿಗೆ ಈ ಬರವಣಿಗೆಗೆ ವಿರಾಮ\-ಕೊಡುತ್ತೇವೆ.

\articleend
}
