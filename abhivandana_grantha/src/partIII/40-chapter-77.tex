{\fontsize{15}{17}\selectfont
\chapter{तर्कातीततर्कशात्री वि~। गंङ्गाधर भट्ट}

\begin{center}
\Authorline{डा~। माणिक बेंगेरी }
\smallskip
मैसूरु
\addrule
\end{center}
सारे भारतीयों केलिए यह गर्व की बात है कि अपने भारत वर्ष में प्राचीनकाल से गुरु-महिमा तथा गुरु-शिष्य परम्परा अपने आप में सर्वश्रेष्ठ है~। अनगिनत ऋषियों ने , विद्वानों ने , सन्तों ने . भक्तों ने गुरुके महत्त्व को पेहले सम्जा है और बाद मे विशद रूप मे शब्दबद्ध भी किया है~। 

हिन्दी साहित्य के ज्ञानाश्रयी निर्गुण भक्तिशाखा सन्त परम्परा के अग्रणी सन्त कबीर दास अपने अनेक दोहो में अपनी विशिष्ट भाषा में गुरु का महत्त्व बताते हैं~। जब भि उनके दोहो में से ----- 
\begin{verse}
गुरु गोविन्द दोऊं खडे काके लागूं पाय~।\\	
बलिहारी गुरु आपणें गोबिन्द दियो दिखाया~॥
\end{verse}
यह शब्द मेरे कानो में गूञ्जते हैं तब मेरे आंखों के सामने मैसूर् के विद्वान् गङ्गाधर भट् जी और उन की शिष्यपरम्परा का दर्शन होने लगता है~। निःसंशयरूप से सच्चा गुरु महान ही होता है~। मगर गंगाधरजी के शिष्यों को तो अपने गुरु में प्रत्यक्ष भगवान के दर्शन होते हैं~। गङ्गाधरजी के असंख्य शिष्यों के मन में यह भावना फैद होने के पीछे गङाधरजीका अनेक वर्षों का मात्रवत पितृवत प्रेम ही कारण है यह सत्य है~। कबीरजी ठीक ही केहते हैं की बिना गुरु भगवान की पेहचान , उन के दर्शन होना असम्भव है, इस लिये यदि भाग्यवश दोनो सामने खडे हो जाये तो में सर्व  प्रथम गुरु के चरणों पर नतमस्तक हूंगा~। क्यूं कि उन्ही के कारण मुझे भगवान के दर्शन प्राप्त हुये हैं~। कबीरजी के इस अनमोल विचारों का प्रत्यक्ष अनुभव गङ्गाधरजी के शिष्यों ने केवल लिया नही है बल की अपनी निजी सम्पत्ति के रूप में सुरक्षित भी रखा है~। 

यह उतना ही सत्य है की गुरु चाहे गार्हस्थ्य जीवन यापन करनेवाला हो अथवा केवल ज्ञानप्रदान करनेवाला हो गुरु तो गुरु ही होता है~। मगर यदि ज्ञान दान के सात सात अनेक व्यवहारिक बुद्धियां सुलझाने में गुरु जुड जाये तो शिष्यों का जीवन तथा ज्ञानसम्पादन का कार्य सुखर बनता है~। विद्वान गङ्गाधरजी ने अपने जीवन में यही कार्य एक असिधारा व्रत के समान सम्पन्न किया~। 

गङ्गाधरजि ने यह सब इतनी सरलता तथा निष्काम भावना के साथ किया की शायद ही शिष्यों के मन में किसी प्रकार की उपकार की भावना निर्माण हो~। सही मायने में गङ्गाधरजी का जीवन स्वान्तः-सुखाय न रहकर शिष्यजन-हिताय बन गया है जिस का गङ्गाधरजी को स्म्पूर्ण तृप्ति प्राप्त होती है~।

सामान्य रूप से यह भी दिखाई देता है की कोई भी आदमी धनसम्पादन करने हेतु शिक्षक बनता है तो वह पोटार्थी शिक्षक ही रहजाता है~। अपने जीवनावश्यक वस्तुओं को जुटाने में ही समय तथा धन लग जाता है~। मगर विद्वान गङाधरजी की कथा असामान्य है इस का अनुभव उनके निकट आनेवाले हर व्यक्ती को प्राप्त होता है~। उन का सम्पूर्ण जीवन इस बात का साक्षी है~।

संस्कृत महा विद्यालय की सेवा मैं रहते हुए अथवा निवृत्ती के समय तक इस महानुभाव ने अपने सुख या आराम के लिये एक वाहन नही खरीदा अथवा रहने के लिये एक छोटा सा घर भी नही बनाया~। अपनी सारी पूञ्जी अपने रिश्तेदार, सगे-सम्बन्धी और शिष्यों के हितसंरक्षण में तथा आनेजानेवालों के अतिथि सत्कार में व्यय की~। अर्थात् यह सामान्य बात नही है~। इस कार्य में उन की धर्मपत्नी श्रीमत्ती शैलजाजी ने भी उन्हे मन से सात-संगत दी~। इसीलिये ‘गुरुमाता’ पद की अधिकारिणी बन पायी~। फ़िर से मुझे इस सन्दर्भ में कबीर जी के और एक दोहे की याद आती है जिसे इस दम्पती ने मन से अपनाया है 
\begin{verse}
साईं इतना दिजिये जामें कुटुम्ब समाये~।\\
मैं भी भूखा न रहूं, साधू न भूखा जाये~॥
\end{verse}
गृह-गृहस्थी अच्छी तरह से निभाने के लिये ऐसे ही सद्गुणों की आवश्यकता रहती है जो आज बहुत ही दुर्मिळ बने है~। अंततः असलियत तो तभी मालूम हो जाती है जब क्थनी और करनी मएं अंतर ना दिखे~। गङ्गाधरजी को कुछ दिखाने के लिये, बहुत कुछ बताने की आवश्यकता कभी भी महसूस नही हुई क्यों की उन्हे बिना बोले कार्य करने की आदत रही~। उन के लिये ज्ञानसत्र चलाना - यही उनके जीवन का ध्येय रहा, साथ ही उन के मन में -- सर्वेभ्यो यज्ञेभ्यो ज्ञानयज्ञो विशिष्यते इस तथ्य पर विश्वास था~। जिस के अनुष्ठान से उन्हें सही अर्थ में इस भव के ताप को दूर करने में सफलता प्राप्त हो सकी~। अपने ज्ञान दान सत्र में उन्हों ने अनेक तत्त्वों पर उद्देशपूर्वक ध्यान दिया - जिससे गुरु के अनेक नामों में से कंटकोद्धार गुरु यह नाम सार्थ किया~। अपने शिष्यों को केवल पाठ सिखाकर काम नही चलेगा~। जब शिष्य का पेट भूखा रहें~। उन्होने अपने हाथों से अपने अपना असामान्य ममत्व सिद्ध किया~। इसके विपरीत आज कल ऐसे गुरु भी दिखाई देते है, जिन का वर्णन नीचे दिये हुए श्लोक के प्रथम पङ्क्ति मे मिलता है~। दुसरी पङ्क्ति अर्थात विद्वान गङ्गाधरजी के लिए ही है  
\begin{verse}
गुरवो बहवः सन्ति शिष्यवित्तापहारकाः~।\\
गुरवो विरलाः सन्ति शिष्यचित्तापहारकाः~॥
\end{verse}
अर्थात शिष्य से वित्त अर्थात धन लेकर सिखाने वाले बहुत शिक्षक बहुत मात्रा में पाये जाते हैं~। मगर शिष्य का चित्त हरण  करके याने शिष्य के चित्त मे विषय के बारे में अभिरुचि निर्माण करके सिखानेवाले गुरु अतिशय दुर्लभ  होते हैं~। विद्वान गङगाधरजी की गणना दुसरे प्रकार के गुरु में ही हो सकती है~। यह निस्संशय अपने ज्ञान सम्पत्ति का वर्षाव उन्होने अपने शिष्यों पर किया, ता की शिष्य ज्ञानगुणसंपन्नता के साथ धनसंपन्न भी हो सके, मगर अपने लिये निरीच्छ भाव अपनाया~। निश्चित रूप से यह सामान्य बात नही हो सकती है~।

माता पिता ने अत्यंत प्रेम से दिया हुआ नाम गङ्गाधर को उन्हो ने अनेक रूपों से वर्धित किया, जैसे गङ्गा को धारण तो लिया, मगर ज्ञानगङगा को अपना लिया,  और प्राकृतिक रूप से जैसे गङ्गासागर को मिलती है, वैसे ही अत्यन्त सहज भाव से अपने ज्ञान को इतना वर्धित और विस्तृत किया की स्वयं ज्ञानमहार्णवा/ज्ञानसागर गङ्गाधर जी बन गये~। समुद्र मन्थन में शिवजी ने हालाहल प्राशन करके सम्पूर्ण जग का उद्धार किया यह कथा तो सर्वज्ञात है~। इस मानवरूपी गङाधर ने अपने जीवन में आये हुए कष्टदायक संकटरूपी हालाहल स्वयं पीकर अपने सगे-सम्बन्धियों को सुख प्रधान किया, अपने  शिष्यों को खाने की, रहने की व्यवस्था करने शिष्यों को संरक्षण दिया~। इतना सब कुछ करने पर भी अपने बारे में कुछ कहने के लिये उन्हों ने अपनी वाणी का उपयोग कभी भी नही किया~। ना किसी को कभी करने दिया~। सदैव प्रसिद्धिपराङ्मुख रहना ही उन्हों ने अपने लिये श्रेयस्कर समझा~। इसी मे  उन की महानता का दर्शन होता है~। गङ्गाधर जी ने स्च्चारित्र्य तथा सत्शील को ही अलंकार के रूप में धारण किया~। मुझे ऐसा लगता है कि इस का कारण उन्हों ने सही अर्थ में नीचे उद्धृत सुभाषित के तत्वों को चयन किया था, उसे अपने मन में समा लिया था~।
\begin{verse}
ऐश्वर्यस्य च विभूषणं सुजनता शौर्यस्य वाक्संयमो \\
ज्ञानस्योपशमः श्रुतस्य विनयो वित्तस्य पात्रे व्ययः~।\\
अक्रोधस्तपसः क्षमा प्रभवितुर्धर्मस्य निर्व्याजता \\
सर्वेषामपि सर्वकारणमिदं शीलं परं भूषणं~॥
\end{verse}
अर्थात सही अर्थ में सज्जनता ही ऐश्वर्य का भूषण है~। वाक्संयम, ज्ञानवन्त का भूषण चित्तशांति, विद्यावन्त का भूषण नम्रता, विनय, धनवन्त का भूषण धन-व्यय विवेकसम्पन्नता, औदार्य, सत्पात्री धनव्यय करने की क्षमता, क्रोधपर नियन्त्रण यही तपश्चर्य का, तपस्वियों का भूषण, धर्मवन्तों का भूषण सब के साथ समभाव, क्षमापूर्ण व्यवहार, धर्म का भूषण निश्चलता इन सारे सद्गुणों का अपना-अपना एक विशिष्ट स्थान जरूर है, फ़िर भी सबसे महत्त्वपूर्ण भूषण शीलता है~। इस सुभाषित का सुयोग्य अर्थ जो कोई ग्रहणकर सक्ते है~। उनें गङ्गाधरजी के सही व्यक्तित्व का परिचय हो स्क्ता है~। ऐसा कहने पर वह अत्युक्ती नही होगी~। सत्ययुग में वर्णित व्यक्ति आज कलियुग में पायी जा स्क्ती है क्या ऐसा प्रश्न अनेक लोगों के मन में पायी जा सकती है~। संस्कृत साहित्य के नीतितज्ञ और श्रेष्ठ सुभाषितकार भर्तृहरि को भी ऐसे ही प्रश्न का उत्तर शायद अपेक्षित था~। इसलिये उन्होने -
\begin{verse}
मनसि वचसि काये पुण्यपीयूषपूर्णाः \\
त्रिभुवनमुपकारश्रेणिभिः प्रीणयन्तः~।\\
परगुणपरमाणून् पर्वतीकृत्य नित्यम्\\
निजहृदिविकसन्तः सन्ति सन्तः कियन्तः~॥
\end{verse}
अर्थात् जिन्होने अपना देह मन अमृतमय पुण्यसंचय से भर दिया है, जिन्होने अपने कष्ट से कृत उपकारों का वैभव सर्वत्र विस्तारित किया, जो अन्य लोगों के परमाणु तुल्य गुण भी मेरु समान मानते हैं~। इस का प्रकटीकरण सहजभाव से करते हैं~। दुसरों के प्रगति से उन की उपलब्धियों से मन में सन्तुष्ट होते हैं ऐसे सज्जन इस पृथ्वीतल पर शायद ही दिखायी देते हैं~।  भर्तृहरि के काल में उन्हे उस प्रश्न का उत्तर मिला होगा या नही मुझे ज्ञात नही है, अगर आज यदि भर्तृहरि मुझे मिलते हैं  तो में एक क्षण मे उन्हे विद्वान गङ्गाधरजी 

का पता दे सक्ती हूं जिससे भर्तृहरि को भी मानसिक सन्तोष की प्राप्ती हो सकेगी ऐसा मुझे पूरा विश्वास है~। 

तर्कशास्त्र पंडित होने के नाते उन्हों ने तर्कशास्त्र नवीनन्यायशास्त्र में प्रावीण्य सम्पादन किया, अपने शिष्यों को भी प्रवीण बनाया मगर स्वयं तर्कातीत, अतर्क्य रह पाये यही उनके जीवन की सबसे महत्त्वपूर्ण खूबी रही~। 

ऐसे महान व्यक्तित्व का परिचय होना यह भी एक बडी उपलब्धी है और मुझे यह सुअवसर जिन विद्वानों के कारण प्राप्त हुआ उन श्री रमेश अडिगा, सहायक प्रो.फ़. संस्कृत महाविद्यालय, मैसूरु इनके प्रति कृतज्ञता व्यक्त करते हुये मुझे बहुत ही आनन्द प्राप्त होता है~। 

सन १९६९ से सन २००० तक सलग ३१ वर्षों की बेंक आफ़ इण्डिया में सेवा करने के पश्चात जब मैने स्वेच्छा निवृत्ति ले ली तब विद्वान रमेश अडिगा जी ने मुझे गंङ्गाधर जी से मिलवाया , वही मेरे जीवन का एक अविस्मरणीय क्षण रहा ! उसके बाद मेरा समय विद्वान गंङ्गाधर जी के मार्गदर्शन में बिताने का सौभाग्य मुझे मिला जिसके परिणामस्वरूप मुझे मेरे अनेक भाषाओं मे अनेक विषयों पर् किये गये लेखन में मेरे ग्रन्थों को अनेक प्रान्तीय, तथा मानवसंसाधन मन्त्रालय, केंद्र सरकार से पुरस्कार मिले जिस का पूरा का पूरा श्रेय विद्वान गङ्गाधरजी को अर्पण करते हुये मे स्वयं को धन्य मानती हूं, साथ ही जीवन की कृतार्थता का   अनुभव करती हूं~।

विद्वान गंङ्गाधर जी के बारे में लिखने के लिये बहुत कुछ है~।  मगर शब्द और समय की पावंदी होने के कारण मेरी भावनाओं को सीमित रखना अत्यावश्यक है यह जानकर विराम देना अनिवार्य है~। इस तथ्य को समझकर मुझे यहां रुकना जरूरी है~। 

विद्वान गंङ्गाधर जी जैसे महानुभावों को निवृत्ति के पश्चात भी अपना ज्ञानपत्र अक्षुण्ण रखने मे सफलता मिले यही मेरी अन्तरिक कामना है~। मेरी इस मङ्गल कामना की पूर्ती मेरी माता, जगन्माता देवी श्रीललिताम्बिका करेगी, और उन्हे दीर्घ आयुरारोग्य और शांती प्रदान करेगी यह मेरा विश्वास है~। और यही मेरी नम्र प्रार्थना भी है~।

\articleend
}
