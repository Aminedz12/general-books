{\fontsize{15}{17}\selectfont
\presetvalues
\chapter{प्रशस्यः प्राध्यापकाः}

\begin{center}
\Authorline{वि~। श्रीधर-भट्ट. के.आर्}
\smallskip
व्याकरणप्राध्यापकः\\
श्रीमन्महारजसंस्कृतमहापाठशाला, मैसूरु
\addrule
\end{center}
विदितचरमिदं समेषां यत् विद्या विनयति मानवमिति~। सति विद्यादातरि श्रेष्ठे विषयी भवितुमर्हति विद्याकाङ्क्षी।  लोके तावत् दृष्टमिदं यत् विद्यादातृषु केषुचित् वैदुष्यं पर्याप्तं न अध्यापनकौशलम्। अपरेषु केषुचित् बोधनकौशलमात्रं, न वैदुष्यम्। सन्ति विरलाः, येषु उभयमपि एकवृन्तगतफलद्वयमिव सम्मिलितं भवेत्। तादृशेष्वन्यतमाः मान्याः विद्वांसः गङ्गाधरभट्टमहोदयाः। अत एव धुरि प्रतिष्ठापयितव्याः।

शृङ्गगिरौ साहित्यपरीक्षायां समुत्तीर्य १९८३तमे वर्षे व्याकरणशास्त्रमध्येतुं श्रीमन्महाराजसंस्कृतमहापाठशामागतोऽहम्। विद्वन्मध्यमद्वितीयकक्ष्यां प्राविशम्। प्रथमकक्ष्यापाठ्य\-विषयाः अधिगन्तव्या आसन्। तेष्वन्यतमां न्यायसिद्धान्तमुक्तावलीं दिनकरीयसहिताम्\break अध्येतुं कैश्चिदत्रत्यैः उक्तप्रकारेण श्रीमतां भट्ट। पुनः वर्षेश्वतीतेषु नवसु महापाठ\-शालायां नियुक्तोऽहम् अध्यापकत्वेन तैः सह सहोद्योगित्वेन। एतत्तु मम महोदयानामन्तिकमगच्छम्। तैश्चोररीकृत्य पाठः समारब्धः। एवं मे परिचयः समजनि विदुषाम्। तेषां तेषां पाठनेन\break नितरामुपकृतोऽहम्। न केवलमेतावत्, तदात्वे कर्णाटकसर्वकारप्रायोजितराज्यस्तरीय-\break राष्ट्रस्तरीयस्पर्धार्थं यदा अहं चितः तदा तैरेव न्यायमञ्जरी न्यायभाष्यादिविषयाः प्रमुखतया अपाठ्यन्त। येन च येन व्याकरणच्छात्रोऽपि न्यायशास्त्रे प्राप्तपुरस्कारः समभवम्। अहो\break पाठप्रभावः! ततश्च व्याकरणशास्त्रे प्राप्तविद्वत्पदवीकः उद्योगकारणात्\break स्थानान्तरमगच्छम्सौभाग्यमेव, यतः कस्को न नन्देत् किरसङ्ग्रहकोशभूतैः साकं कर्तुम्।

महाभाष्यकारः पतञ्जलिराह- विद्या चतुर्भिः प्रकारैरुपय्क्ता भवति- ते च आगम\-कालेन (ग्रहणकालेन) स्वाध्यायकालेन, व्यवहारकालेन इति। एते चत्वारोऽपि प्रकाराः\break भट्टमहोदयेषु सिद्धिं गता इत्युक्ते नातिशयोऽक्तिः। अत एव एतेषाम् अध्यापनपरिपाटी\break अनन्यसदृशा। यद्यपि शास्त्राध्ययनं कष्टं, तत्रापि न्यायशास्त्रं कष्टतरमिति मनुते छात्रगणः, तथापि गङ्गाधरभट्टानां पाठपद्धतिः तादृशं परिवर्तयेत् सपदि। पाठपद्धतौ समाश्रितः कश्चन क्रमः- ज्ञातात् अज्ञातं प्रतीति, तत्सहजतया भासते एतेषामध्यापने। छात्रनिष्ठज्ञनोद्बोधनद्वारा दुरूहतर्कशास्त्रमपि सुलभायते नूनम्। अपरमपि वैशिष्ट्यमपि अन्वयश्च। एते पण्डितमण्डल्यां वाक्यार्थपटवः।छात्रसमुदाये ज्ञान्अस्यसङ्क्रामयितारः। 

भट्टमहोदयानां पुत्रनिर्विशेषं वात्सल्यं शिष्येषु सर्वदा छात्रान् हिताय योजयन्ते। दोषानपरिगणय्य गुणप्रकाशनद्वारा सत्पथमानयन्ति। रक्षन्ति चापदः। अशनं वाश्रयं वाकाले ददति। आत्मविश्वासं पूरयन्ति, वर्धयन्ति च। अहरहः प्रेरयन्त्यभ्यासितुं शास्त्रम्। भट्टानामयं विश्वासः यत् “ अयोग्यः पुरुषो नास्ति, योजकस्तत्र दुर्लभः” इति। 

छात्राणामन्याये सति झटिति प्रतिरुध्नन्ति। न्यायदापनपर्यन्तमपि न विस्मरन्ति। एतद्विषये स्मर्तुम्येकां घटनाम्। व्याकरणशास्त्रस्य एकेन छात्रेण विद्वन्मध्यमपरीक्षायाम् अष्टानां पत्राणां मध्ये सप्तसु तावत् सप्ततिप्रतिशतम् अङ्काः लब्धाः आसन्। परमेकस्यामेव चत्वारिंशत् अङ्काः लब्धाः आसन्। बुद्धिमान् स विद्यार्थी नितरां व्यथितः, यतः तस्यामपि अधिकाः अङ्काः प्राप्तव्या आसन्। एतत् विज्ञाय भट्टमहोदयैः सपद्येव सम्बद्धाधिकारिभिः समालोच्य तस्मै न्यायः दापितः। 

गङ्गाधरभट्टानां संस्कृत-कन्नड-आङ्लभाषासु अकुण्ठिता गतिः। अतश्छात्राः भाषणाय, लेखनाय, नाटकाय, कार्ययोजनाय, शास्त्रवाक्यार्थाय, समाश्रयन्ते तान्। अध्यापने कार्यकरणे वा नैव श्रान्ता भवन्ति। एतेषां मार्गमनुसरन्तश्छात्राः नैवानुत्तीर्णा दृष्टाः। एतेषां भाषा तु कर्णानन्दकरं सुविचारप्रचोदकञ्च। उक्तं हि ‘इष्टं हि विदुषां लोके समासव्यासधारणम्’ इति। तदात्मसात्कृतं सम्यगेतैः। समयानुगुणं मितेन सारेण च वचसा आनन्दपण्डितचित्तं रसयेयुः। एतेषां न केवलं न्याये, परं वेदान्ते, व्याकरणे, साङ्ख्ये, योगे, आयुर्वेदे च परिश्रमो महान्। अतः न केवलं शास्त्रच्छात्राः किन्तु पौरजनपदैः वैदेहिकैश्च ज्ञानार्थमशिश्रियन् आश्रयन्ति च। एतेषां लेखनप्रपञ्चोऽपि चित्ताकर्शकः। 

एतेषां कर्तृत्वशक्तिः अनितरसाधारणी। सङ्घटनाचातुरी तु सर्वथानुसारणीययोग्या। कार्यक्रमस्य पूर्वसिद्धता क्रमबद्धा। कार्ययशसे यं यत्र यदर्थं योजयेत् इति विषये निश्चितधियः। उपकरणायत्ता नैते। तदेव शोभनं खलु सामर्थ्यवतां, यतः क्रियासिद्धिः, सत्त्वे भवति महताम् अक्षिसात्कृतं मया। संकृताध्यापकनां प्रोत्साहं कर्तुं दानिनः समागता आसन्। छात्रप्रतिभादिदृक्षवस्ते। अङ्गीकृतकार्ययोजनाः भट्टमहोदयाः प्रतिवर्षं विभिन्नविषयेषु शास्त्रेषु, शास्त्रीयेषु  संस्कृतवाङ्मयान्तर्गतवैज्ञानिकविषयेषु वेदेषु च छात्रान् सम्प्रेर्य, दानिनः सन्तुष्टानकुर्वन्। एवं दशवर्षाणि यावत् कार्यायोजनमकार्षुः। 

अहो ! संस्कृतानुरागः। अहो ! छात्रप्रियता ! उपकर्तारं स्मरन्ति सदा। एकस्मिन् वर्षे कृतकार्योऽपि अनारोग्यकारणात् कार्यक्रमं गन्तुम् अशक्तोऽहम्~। तदा कार्यक्रम\-समाप्तिसमनन्तरमेव विद्यागणपतिप्रसादेन सह मम गृहमागत्य क्षेममैच्छन्~। एतत्प्रकाशयति तेषां सहृदयताम्। एवं गुणेन ज्ञानेन च जनानुरागिणः गुरून् गङ्गाधरभट्टान् प्रणत्य, तेषां दीर्घायुरारोग्यञ्च सम्प्रार्थ्य विरम्यते मया।  

\articleend
}
