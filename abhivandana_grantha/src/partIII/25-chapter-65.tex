{\fontsize{14}{16}\selectfont
\chapter{ಗುರುಜ್ಞಾನ ಗಂಗಾಧರರು}

~\\[-1.7cm]

\begin{center}
\Authorline{ವಿ~॥ ಸುರೇಶ ಭಟ್ಟ}

ಮುಖ್ಯಸ್ಥರು, ಕನ್ನಡಭಾಷಾ ವಿಭಾಗ\\
ಡಿ.ಬನುಮಯ್ಯ ವಾಣಿಜ್ಯ ಮತ್ತು \\
ಕಲಾ ಕಾಲೇಜು, ಮೈಸೂರು
\addrule
\end{center}

~\\[-1.5cm]

\noindent
ವಿಶಿಷ್ಟ ವಿಶೇಷ ಸಂದರ್ಭ. ಸುಮಾರು 32 ವರ್ಷಗಳ ಹಿಂದಿನ ಸವಿಸಮಯವನ್ನು ನೆನಪಿಸಿಕೊಳ್ಳುವ ಅನುಭವವನ್ನು ಮೆಲುಕು ಹಾಕುವ ಸತ್ಸಮಯ, ಸದವಕಾಶ ಪ್ರಾಪ್ತವಾಗಿದೆ. ಗುರುಗಳಾದ ವಿದ್ವಾನ್ ಗಂಗಾಧರ ಭಟ್ಟರು ವೃತ್ತಿಜೀವನದಿಂದ ನಿವೃತ್ತರಾಗು\-ತ್ತಿರುವ ಈ ಸಮಯದಲ್ಲಿ ಅವರ ಹಿರಿಯ ವಿದ್ಯಾರ್ಥಿಗಳು ಮತ್ತು ಅಭಿಮಾನಿಗಳು ಅವರಿಗೆ ಅಭಿವಂದನ ಸಮಾರಂಭ ಏರ್ಪಡಿಸಿ ಅಭಿವಂದನ ಗ್ರಂಥವನ್ನು ಹೊರತಂದು\break ಸಮರ್ಪಿಸುತ್ತಿರುವ ಕಾರಣದಿಂದ.

1980 ರಲ್ಲಿ ವಿದ್ಯಾಭ್ಯಾಸಕ್ಕಾಗಿ ನನ್ನ ಅಣ್ಣ ವಿದ್ವಾನ್ ಜಿ.\ ಮಂಜುನಾಥರ ಆಶ್ರಯದಲ್ಲಿ ಮೈಸೂರಿಗೆ ಬಂದು ಶ್ರೀಮನ್ಮಹಾರಾಜ ಸಂಸ್ಕೃತ ಕಾಲೇಜಿನಲ್ಲಿ ವೇದಾಧ್ಯಯನಕ್ಕೆ ಸೇರಿಕೊಂಡೆ. ಆಗ ಯಾವ ಭಾಷೆಯೂ ಸರಿಯಾಗಿ ಗೊತ್ತಿಲ್ಲದ ನಾನು ಶ್ರೀಶಂಕರ\-ವಿಲಾಸ ಪಾಠಶಾಲೆಯಲ್ಲಿ ಕಾವ್ಯ ತರಗತಿಗೆ ಪ್ರವೇಶ ಪಡೆದೆ. ಆ ಸಮಯದಲ್ಲಿ ವಿದ್ವಾನ್\break ಗಂಗಾಧರ ಭಟ್ಟರು ಆ ಪಾಠಶಾಲೆಯ ಮುಖ್ಯೋಪಾಧ್ಯಾಯರಾಗಿದ್ದರು. ತರಗತಿಗಳು ಸಂಜೆ ಸಮಯದಲ್ಲಿ ನಡೆಯುತ್ತಿದ್ದು ಎರಡು ವರ್ಷ ಗಂಗಾಧರ ಭಟ್ಟರ ಪಾಠ ಕೇಳುವ ಭಾಗ್ಯ ನನ್ನದಾಗಿತ್ತು. ಆ ಪಾಠ ಇಂದಿಗೂ ಹಚ್ಚ ಹಸಿರಾಗಿಯೇ ಇದೆ. ಏಕೆಂದರೆ ನಾನೂ ಅಧ್ಯಾಪಕ ವೃತ್ತಿಯನ್ನು ನಿರ್ವಹಿಸುತ್ತಿರುವುದರಿಂದ ಅವರ ಪಾಠದ ಜ್ಞಾನ ಮತ್ತೆ ಮತ್ತೆ ಬಳಕೆಯಾಗುತ್ತಲೇ ಇದೆ. ವಿಶೇಷವಾಗಿ ಅವರು ಕಲಿಸಿದ ಸಂಧಿ\enginline{-}ಸಮಾಸಗಳ ವಿಷಯ ಚಿತ್ತಭಿತ್ತಿಯ ಚಿರಮುದ್ರೆಯಾಗಿ ಉಳಿದಿದೆ. 

ವಿದ್ಯಾರ್ಥಿಗಳ ಮನಸ್ಸನ್ನು ತಮ್ಮೆಡೆಗೆ ಎಳೆದು ಸೆಳೆದು ಮನದಟ್ಟಾಗುವಂತೆ ಬೋಧಿಸುತ್ತಿದ್ದರು. ಅಂತಹ ಬೋಧನಾ ಕಲೆ ಅವರಿಗೆ ಕರತಲಾಮಲಕವಾಗಿತ್ತು. ಭಾಷೆಯ ಗಂಭೀರವಾಣಿ ಮಧುರಮಧುವಾಗಿತ್ತು. ಆದರೆ ಆಳವಾದ ಅವರ ಪಾಂಡಿತ್ಯದ ವಿಷಯ ವಿಚಾರಗಳು ಎಲ್ಲರಿಗೂ ಸುಲಭವಾಗಿ ಅರಗುತ್ತಿರಲಿಲ್ಲ. ಕಾವ್ಯತರಗತಿಯ ಅಭ್ಯಾಸದ ಸಂಗಡ ಪಿ.ಯು, ಪದವಿ ವ್ಯಾಸಂಗಮಾಡುವ ಸಮಯದಲ್ಲಿ ನಾನು ಹಲವು ಕನ್ನಡ ಚರ್ಚಾ ಸ್ಪರ್ಧೆಗಳಲ್ಲಿ ಬಹುಮಾನ ಪಡೆದಿದ್ದೇನೆ. ಇದಕ್ಕೂ ವಿದ್ವಾನ್ ಭಟ್ಟರ ವಿಶೇಷ ಮಾರ್ಗದರ್ಶನವಿತ್ತು. ಹಲವು ವಿಷಯ ವಿಚಾರಗಳ ಕುರಿತು ಹೆಚ್ಚಿನ ಮಾಹಿತಿಗಳನ್ನು ಒದಗಿಸಿ ಮಾತನಾಡುವ ರೀತಿಯನ್ನು ಕಲಿಸಿಕೊಟ್ಟಿದ್ದಾರೆ. ನನ್ನಂತೆಯೇ ಹಲವರಿಗೆ ಈ ಬಗೆಯ ಮಾರ್ಗದರ್ಶನ ಮಾಡಿದ ಸನ್ಮಾರ್ಗದರ್ಶಕರು ಶ್ರೀಗಂಗಾಧರ ಭಟ್ಟರು. 

ನನ್ನ 27 ವರ್ಷಗಳ ಅಧ್ಯಾಪನವೃತ್ತಿಯ ಅನುಭವದಿಂದ ಹೇಳುವುದಾದರೆ, ಆದರ್ಶ ಶಿಕ್ಷಕ ಆದರ್ಶ ಗುರು ಎಂಬ ಹಿರಿಮೆಗೆ ವಿ. ಗಂಗಾಧರ ಭಟ್ಟರು ಮತ್ತು ಅಂಥವರೇ ಸಾರ್ಥಕ ಅರ್ಥರೂಪ ಎಂದರೆ ಅತಿಶಯವಲ್ಲ. ಶಾಸ್ತ್ರೀಯವಾದ ಅಧ್ಯಯನ ಮತ್ತು ಅಧ್ಯಾಪನಕ್ರಮದ ಪರಿಧಿಯನ್ನು ಸಂಪನ್ನ ಗೊಳಿಸಿದ ಮೇಧಾವಿಗಳು ಇವರು. ಸದ್ಯ ಈ ಮಾನ್ಯರು ಶ್ರೀಮನ್ಮಹಾರಾಜ ಸಂಸ್ಕೃತ ಕಾಲೇಜಿನ ನವೀನ ನ್ಯಾಯ ಶಾಸ್ತ್ರ ವಿಭಾಗದ ಪ್ರಾಧ್ಯಾಪಕರಾಗಿ ಕಾರ್ಯನಿರ್ವಹಿಸಿ ನಿವೃತ್ತರಾಗಿದ್ದಾರೆ. ಅವರ ಕಲಿಯುವ ಕಲಿಸುವ ಪ್ರವೃತ್ತಿಗೆ ನಿವೃತ್ತಿಯಿಲ್ಲ. ಇದು ಸರ್ಕಾರದ ನಿಯಮದ ನಿವೃತ್ತಿಯಷ್ಟೆ. ಅಲ್ಲಿ ವಿದ್ವಾನ್ ಗಂಗಾಧರ ಭಟ್ಟರಿಂದ ಪಾಠ ಕೇಳಿದವರ ಬಾಳು ಧನ್ಯ ಎಂದು ಭಾವಿಸುತ್ತೇನೆ.

ವೃತ್ತಿಜೀವನವನ್ನು ಹೊರತುಪಡಿಸಿ ಇನ್ನೊಂದು ವಿಷಯವನ್ನು ಹೇಳಲೇ ಬೇಕು. ಏಕೆಂದರೆ ನಾನು ಈ ಗುರುಗಳ ಮನೆತನದ ಶಿಷ್ಯವರ್ಗಕ್ಕೆ ಸೇರಿದವನೂ ಆಗಿರುವುದರಿಂದ ಸಮಾಜದ ಸರ್ವರ ಹಿತದ ಭಾವವನ್ನು ಹೃದಯದಲ್ಲಿ ತುಂಬಿಕೊಂಡು ಅದಕ್ಕಾಗಿ ತನು, ಮನ, ಧನ ಅರ್ಪಿಸಿ ಕಾಯೇನ ವಾಚಾ ಮನಸಾ ಶುದ್ಧಿಯಿಂದ ಸುದೀರ್ಘಕಾಲ ಕಾರ್ಯಗೈಯ್ಯುತ್ತಾ ಶಿಷ್ಟಪರಂಪರೆಯನ್ನು ಬೆಳೆಸಿದ ಉಳಿಸಿದ ಪುರೋಹಿತ ಪದಕ್ಕೆ ಅರ್ಥಪೂರ್ಣ ನ್ಯಾಯ ಒದಗಿಸಿದ ಮನೆತನದ ಪ್ರತಿಭಾರತ್ನ ಶ್ರೀ ಗಂಗಾಧರ ಭಟ್ಟರು. 

ನನ್ನ ನೆಚ್ಚಿನ ಗುರುಗಳಲ್ಲಿ ಅಗ್ರಮಾನ್ಯರಾದ ಶ್ರೀಯುತರ ನಿವೃತ್ತಿಯ ಜೀವನ ಸುಖಮಯವಾಗಿರುವಂತೆ ಆ ಪರಮಾತ್ಮ ಕೃಪೆ ತೋರಲೆಂದು ಪ್ರಾರ್ಥಿಸುತ್ತೇನೆ. ಮುಂದೆಯೂ ಗುರುಗಳ ವಿದ್ಯಾದಾನಧಾರೆ  ಹಲವರಿಗೆ ಹರಿಯುವಂತಾಗಲಿ ಎಂದು ಅವರನ್ನು ಪ್ರಾರ್ಥಿಸುತ್ತೇನೆ. ಕೃತಜ್ಞತೆಯ ಸದ್ಭಾವದಿಂದ ಗುರುಚರಣಕ್ಕೆ ನಮಿಸುತ್ತಾ ಅವರ ಬಗ್ಗೆ ಬರೆಯಲು ಅವಕಾಶ ನೀಡಿದ ಅಭಿವಂದನ ಸಮಿತಿಗೆ ಧನ್ಯವಾದಗಳನ್ನು ಅರ್ಪಿಸುತ್ತೇನೆ.

\smallskip
\centerline{{\fontsize{10}{12}\selectfont\ding{97}\quad\ding{97}\quad\ding{97}}}
}
