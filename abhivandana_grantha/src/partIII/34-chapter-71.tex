{\fontsize{14}{16}\selectfont
\chapter{ಗುರುಗುಣ ಅನಾವರಣ}

\begin{center}
\Authorline{ವಿ~। ಶ್ರೀಕೃಷ್ಣ ಭಟ್}
\smallskip
ಸಂಸ್ಕೃತ ಅಧ್ಯಾಪಕ\\
ಕ್ಯಾತನಳ್ಳಿ ಪಾಠಶಾಲೆ,\\
ಮೈಸೂರು
\addrule
\end{center}
ಯಾರನ್ನು ನೆನೆದರೆ ಎಲ್ಲರನ್ನೂ ನೆನೆದಂತಾಗುವುದೋ, ಯಾರು ಅಳೆಯಲಾಗದ ಅಗಾಧ ಸಮುದ್ರದ ಸರಿಸಮನಾಗಿ ಗೋಚರಿಸುವರೋ, ಯಾರನ್ನು ನೋಡಿದಾಗಲೆಲ್ಲ ಮನಸ್ಸಿನ ದುಗುಡವೆಲ್ಲ ದೂರಾಗುವುದೋ, ಯಾರು ‘ಶಿಷ್ಯಚಿತ್ತಾಪಹಾರಕಾಃ’ ಎಂಬ ಮಾತಿಗೆ ಅನ್ವರ್ಥರಾಗಿರುವರೋ ಅಂಥ ಶ್ರೇಷ್ಠರಾದ ನನ್ನ ಗುರುಗಳೂ ಆದ ಶ್ರೀಯುತ ಗಂಗಾಧರ ಭಟ್ಟರ ಅಭಿನಂದಾನಾ ಕಾರ್ಯಕ್ರಮದ ಈ ಸಂದರ್ಭದಲ್ಲಿ ಅವರ ಕುರಿತು ಅತಿಶಯವಲ್ಲದ ಕೆಲವು ವಿಚಾರಗಳನ್ನು ಹಂಚಿಕೊಳ್ಳಲು ಅಣಿಯಾಗುತ್ತಿದ್ದೇನೆ. ಸದ್ಗುಣಗಳ ಸಾಗರದಂತಿರುವ  ಗುರುಗಳಾದ ಗಂಗಾಧರ ಭಟ್ಟರ ಕುರಿತು ಅವರೇ ನನಗೆ ಕಲಿಸಿದ ‘ಗುಣಿನಿಷ್ಠಗುಣಾಭಿಧಾನಂ ಸ್ತುತಿಃ’ ಎಂಬ ಸ್ತುತಿಲಕ್ಷಣದಂತೆ ನನ್ನ ಅನಿಸಿಕೆಯಂತೆ ಅವರ ಕೆಲವು ಗುಣಗಳನ್ನು ತೆರೆದಿಡುವ ಪ್ರಯತ್ನವಷ್ಟೆ.

ನನಗಿನ್ನೂ ನೆನಪಿದೆ, ನಾನು ಸಾಹಿತ್ಯ ತರಗತಿಯ ವರೆಗಿನ ವಿದ್ಯಾಭ್ಯಾಸವನ್ನು ಮುಗಿಸಿ ಉನ್ನತ ಅಧ್ಯಯನಕ್ಕಾಗಿ ಮೈಸೂರಿನತ್ತ ಹೊರಡಲು ಸಿದ್ಧನಾದೆ. ನಾನು ಅಲ್ಲಿಯವರೆಗೆ ಓದಿದ್ದ ವಿನಾಯಕ ವೈದಿಕ ಸಂಸ್ಕೃತ ಪಾಠಶಾಲೆಯ ಮುಖ್ಯೋಪಾಧ್ಯಾಯರಾಗಿದ್ದ ಶ್ರೀ ರಾಮಚಂದ್ರ ಭಟ್ಟರು ಅವರಿಗೆ ಕಾನೂನಾತ್ಮಕವಾಗಿ ಮೂಡಿದ್ದ ಸಂದೇಹವನ್ನು ನಿವಾರಿಸಿಕೊಳ್ಳಲು ಗಂಗಾಧರಭಟ್ಟರ ಸಲಹೆಯನ್ನು ಕೇಳಲು ನನಗೆ ಸೂಚಿಸಿದ್ದರು. ಅನಂತರದ ದಿನಗಳಲ್ಲಿ ಅದಕ್ಕೆ ಬೇಕಾದ ಪರಿಹಾರವನ್ನೂ  ಪಡೆದುಕೊಂಡರು. ದೇಶವಿದೇಶಗಳಿಂದ ಅನೇಕರು ಪೂಜ್ಯರಾದಗಂಗಾಧರ ಭಟ್ಟರಲ್ಲಿ ಸಂಸ್ಕೃತ ಸಾಹಿತ್ಯದ ಅನೇಕ ಜಟಿಲ ಗ್ರಂಥಗಳನ್ನು ಪಾಠಮಾಡಿಸಿಕೊಳ್ಳುತ್ತಿದ್ದ ಹಾಗೂ ಅನೇಕ ನ್ಯಾಯವಾದಿಗಳು(ವಕೀಲರು) ಪೂಜ್ಯರ ಸಲಹೆಯನ್ನು ಪಡೆಯುತ್ತಿದ್ದ ವಿಚಾರವು ಅಷ್ಟರಲ್ಲಾಗಲೇ ನನಗೆ ತಿಳಿದ್ದಿತ್ತು. 

ನಾನು ಶ್ರೀಮನ್ಮಹಾರಾಜ ಸಂಸ್ಕೃತಕಾಲೇಜಿನಲ್ಲಿ ಮೀಮಾಂಸಾಶಾಸ್ತ್ರವನ್ನು ಅಧ್ಯಯನ ಮಾಡಲು ಪ್ರವೇಶ ಪಡೆದೆ. ಉತ್ತಮರಾದ ಶಾಸ್ತ್ರಗುರುಗಳನ್ನು ಪಡೆದು ನನ್ನ ಬುದ್ಧಿಶಕ್ತಿಗನುಗುಣವಾಗಿ ಮೀಮಾಂಸಾ ಶಾಸ್ತ್ರಾಧ್ಯಯನವನ್ನು ಆರಂಭಿಸಿದೆ. ಆರಂಭದ ದಿನಗಳಲ್ಲಿ ನನ್ನ ಅನೇಕ ಮಿತ್ರರು ಸಾಹಿತ್ಯ ತರಗತಿಯ ತರ್ಕವಿಷಯದ ಪಾಠಕ್ಕಾಗಿ ಗಂಗಾಧರ ಭಟ್ಟರ ತರಗತಿಗೆ ಒಂದು ದಿನವೂ ತಪ್ಪದೇ  ತೆರಳಬೇಕೆಂದು ಹಪಹಪಿಸುತ್ತಿದ್ದರು.
\vskip 2pt

ಅವರ ಪಾಠದ ಶೈಲಿ, ವಿಶ್ಲೇಷಿಸುವ ಕ್ರಮ, ಕೊಡುವ ಉದಾಹರಣೆ ಮುಂತಾದ\break ವಿಷಯಗಳನ್ನು ಮಿತ್ರರು ನನ್ನಲ್ಲಿ ಹಂಚಿಕೊಂಡಾಗಲೆಲ್ಲ ನನ್ನ ಮನಸ್ಸು ಅವರ ಪಾಠಕ್ಕಾಗಿ ಹಾತೊರೆಯಲು ಪ್ರಾರಂಭಿಸಿತ್ತು. ಅನಂತರದ ದಿನಗಳಲ್ಲಿ ವಿದ್ವತ್ತರಗತಿಯ ಸಾಮಾನ್ಯ ವಿಷಯ(ದಿನಕರೀ)ದ ಪಾಠವನ್ನು ಕೇಳಿ, ಕಬ್ಬಿಣದ ಕಡೆಲೆಯಂತಿರುವ ತರ್ಕಶಾಸ್ತ್ರವನ್ನು ಕಾವ್ಯರಸದೌತಣದಂತೆ  ಉಣಬಡಿಸುವ ಗುರುಗಳನ್ನು ನೋಡಿ ನಾನು ಪುಲಕಿತ\-ನಾದದ್ದು ನಿಜ.  ಅವರ ಬೋಧನಾಶೈಲಿಯು ನನ್ನ ಕಿವಿಯಲ್ಲಿ ಇನ್ನೂ ಗುಂಯ್‍ಗುಡುತ್ತಿದೆ. ಅದೆಷ್ಟೋ ದಿನ ಅವರ ಮನೆಯಲ್ಲಿ ರಘುವಂಶ, ಚಂಪೂರಾಮಾಯಣ, ಸಾಂಖ್ಯ\-ದರ್ಶನ, ಇಂಗ್ಲೀಷ್ ಹೀಗೆ ಹತ್ತು ಹಲವು ವಿಷಯಗಳನ್ನು ಪಾಠಮಾಡಿಸಿಕೊಂಡು ಕೃತಾರ್ಥ\-ನಾದೆನೆಂಬ ಭಾವನೆ ನನ್ನದು.
\vskip 2pt

ಪೂಜ್ಯರ ಒಂದು ವಿಶೇಷಗುಣವನ್ನು ಹೇಳಬೇಕೆಂದರೆ ಇವರು ಪ್ರತಿಯೊಂದು ಸಂದರ್ಭ\-ದಲ್ಲೂ ವಿದ್ಯಾರ್ಥಿಗಳ ಬೆನ್ನೆಲುಬಾಗಿ ನಿಲ್ಲುತ್ತಿದ್ದುದನ್ನು ಯಾವ ವಿದ್ಯಾರ್ಥಿ\-ಯೂ ಮರೆಯಲಾರ. 
\vskip 2pt

ವಿದ್ಯಾರ್ಥಿಗಳೂ ಕೂಡ ಯಾವ ತೊಂದರೆಯಾಗಿದ್ದರೂ, ಯಾವುದಾದರೂ ತಪ್ಪಾಗಿದ್ದರೂ ಸಹ ಗಂಗಾಧರ ಭಟ್ಟರಲ್ಲಿ ತೆರಳಿ ತೊಂದರೆಯನ್ನು ನಿವಾರಿಸಿಕೊಳ್ಳುತ್ತಿದ್ದರು, ತಪ್ಪನ್ನು ತಿದ್ದಿಕೊಳ್ಳುತ್ತಿದ್ದರು ಎಂಬುವುದು ಅವರನ್ನು ಬಲ್ಲ ಎಲ್ಲರಿಗೂ ತಿಳಿದ ಸತ್ಯ. (ನನ್ನ ತಪ್ಪನ್ನು ಹಾಗೂ ನನಗಾದ ತೊಂದರೆಗಳನ್ನು ಅವರು ನಿವಾರಿಸಿದ್ದನ್ನು ನಾನೆಂದೂ ಮರೆಯ\-ಲಾರೆ.) ಯಾವುದೇ ರೀತೀಯ ರಾಜ್ಯ ಹಾಗೂ ರಾಷ್ಟ್ರಮಟ್ಟದ ಸ್ಪರ್ಧೆಗಳಲ್ಲಿ ಶಾಸ್ತ್ರಭೇದವಿಲ್ಲದೇ ಪ್ರೋತ್ಸಾಹಿಸುವ ಅವರ ಗುಣ ನಮಗೆಲ್ಲ ಆದರ್ಶ. ಇಂತಹ ಗುರುಗಳ ಅಂತೇವಾಸಿಯಾದದ್ದು ಪೂರ್ವಜನ್ಮದ ಸುಕೃತವೇ ಸರಿ.
\vskip 2pt

ಇನ್ನೊಂದು ವಿಚಾರವನ್ನು ನಾನಿಲ್ಲಿ ಹೇಳಲೇಬೇಕು. ‘ತುಂಬಿದ ಕೊಡ ತುಳುಕುವುದಿಲ್ಲ’ ಎಂಬ ಮಾತು ಇಂಥವರನ್ನು ನೋಡಿಯೇ ಹೇಳಿರಬೇಕು. ಈ ಮಾತಿಗೆ ಅನ್ವರ್ಥ\-ದಂತಿರುವ ಪ್ರಾಜ್ಞರಾದ ನನ್ನ ಗುರುಗಳಿಗೆ ಸಂಬಂಧಿಸಿದ ಆ ಸನ್ನಿವೇಶವು ಎಂಥವರ ಮನಸ್ಸನ್ನು ಗೆಲ್ಲಬಲ್ಲದು. ನಾವೆಲ್ಲರೂ ವಿದ್ವತ್ತರಗತಿಯಲ್ಲಿ ಓದುತ್ತಿರುವ ಸಂದರ್ಭವದು.  ಆ ದಿನಗಳಲ್ಲಿ  ರಾಮಾಯಣ ಮತ್ತು ಮಹಾಭಾರತ ಪರೀಕ್ಷೆಗಳ ಉಸ್ತುವಾರಿಯನ್ನು ಹೊತ್ತು ಮೌಲ್ಯಮಾಪನವನ್ನು ಮಾಡಿಸುತ್ತಿದ್ದವರು ಶ್ರೀಯುತರು. ಮೌಲ್ಯಮಾಪನಕ್ಕಾಗಿ ನಮ್ಮನ್ನೂ ಕರೆಯುತ್ತಿದ್ದರು. ನಿಸ್ವಾರ್ಥವಾಗಿ ಅವರು ಮಾಡುತ್ತಿರುವ ಕಾರ್ಯದಲ್ಲಿ ಭಾಗಿಯಾದದ್ದು ನಮ್ಮಲ್ಲಿ ಸಮಾಧಾನವನ್ನುಂಟುಮಾಡುತ್ತಿತ್ತು. ಒಮ್ಮೆ ಮೌಲ್ಯಮಾಪನ ಕೇಂದ್ರಕ್ಕೆ ನಾನು ಸ್ವಲ್ಪ ತಡವಾಗಿ ಹೋಗಿದ್ದೆ. ಅಷ್ಟರಲ್ಲಾಗಲೇ ನನ್ನ ಮಿತ್ರರೆಲ್ಲ ಅಲ್ಲೇ ಉಪಾಹಾರವನ್ನು ಮುಗಿಸಿ ತಮ್ಮ ಕೆಲಸವನ್ನಾರಂಭಿಸಿದ್ದರು. ತಡವಾಗಿ ಬಂದಿದ್ದರೂ ನನಗೆ ಉಪಾಹಾರ ಸ್ವೀಕರಿಸಿ ಮೌಲ್ಯಮಾಪನಕ್ಕೆ ತೆರಳಲು ಗುರುಗಳಾದ ಗಂಗಾಧರ ಭಟ್ಟರು ಸೂಚಿಸಿದರು. ನಾನು ತಿಂಡಿಯನ್ನು ಸ್ವೀಕರಿಸಿ ತಿಂದ ತಟ್ಟೆ(ಬಟ್ಟಲು)ಯನ್ನು ಇಡಲು ಹೋದಾಗ ನನ್ನ ಕಣ್ಣು ನಂಬಲಾಗದ ಸನ್ನಿವೇಶವೊಂದು ಎದುರಾಗಿತ್ತು. ಅನೇಕ ಶಾಸ್ತ್ರಗಳಲ್ಲಿ ನಿಷ್ಣಾತರಾದ, ದೇಶ ವಿದೇಶಗಳಲ್ಲಿ ಶಿಷ್ಯರನ್ನು ಹೊಂದಿರುವ, ರಾಷ್ಟ್ರಸ್ತರದ ವಿದ್ವನ್ಮಣಿಯಾದ, ಅನೇಕ ಮಹನೀಯರ ಮನ್ನಣೆ ಗಳಿಸಿದ್ದ ಗುರುಗಳು ಎಲ್ಲರೂ ತಿಂದು ಇಟ್ಟು ಹೋದ ತಟ್ಟೆಗಳನ್ನು ತೊಳೆಯುತ್ತಿದ್ದರು. ನನಗೆ ಕಣ್ಣಲ್ಲಿ ನೀರು ತುಂಬಿದಂತಾಯಿತು. ‘ಭಟ್ರೆ ನೀವೆಂತಕ್ ತೊಳಿತಾ ಇದ್ರಿ?’ಎಂದು ಕೇಳಿದರೆ ‘ತೊಳೆಯಲು ಒಬ್ಬರಿಗೆ ಬರಲು ಹೇಳಿದ್ದೆ. ಅವರು ಬರಲಿಲ್ಲ. ಕೆಲಸ ಮಾಡುವವನಿಗೆ ಯಾವುದಾದ\-ರೇನು?’ಎನ್ನುತ್ತಾ ತೊಳೆಯಲಾರಂಭಿಸಿದರು. ನಾನು ಮಾಡುತ್ತೇನೆಂದು ಎಷ್ಟೇ ಒತ್ತಾಯ ಮಾಡಿದರೂ ಅವರು ತಮ್ಮ ಕೆಲಸವನ್ನು ನಿಲ್ಲಿಸಲಿಲ್ಲ. ಆ ಸಂದರ್ಭದಲ್ಲಿ ಅವರ ಜೊತೆ ನಾನೂ ಕೈ ಜೋಡಿಸಿದೆ. ಇದೊಂದು ಸನ್ನಿವೇಶ ನಮ್ಮಂಥ ಅದೆಷ್ಟೋ ಯುವಕರಿಗೆ ಜೀವನದ ಪಾಠವಾಗಬಲ್ಲದು. ಇಂಥ ಅನೇಕ ಸನ್ನಿವೇಶಗಳು ಅವರ ಸರಳತೆ, ಕಾರ್ಯತತ್ಪರತೆ, ದಕ್ಷತೆ ಹಾಗೂ ಪ್ರಾಮಾಣಿಕತೆಗೆ ಹಿಡಿದ ಕನ್ನಡಿಯಂತಿವೆ. ಇವರನ್ನು ಗುರುಗಳಾಗಿ ಪಡೆದ ಎಲ್ಲರೂ ಧನ್ಯರು. ನಾನು ಅವರ ಗುಣಗಳನ್ನು ಅಳವಡಿಸಿಕೊಳ್ಳಲು ಬಹಳ ಪ್ರಯತ್ನಪಡುತ್ತಿದ್ದೇನೆ. ಅದು ಈ ಜನ್ಮದಲ್ಲಿ ಸಾಧ್ಯಗದು ಎಂಬುದು ತಿಳಿದಿದ್ದರೂ “ತಿತೀರ್ಷುರ್ದುಸ್ತರಂ ಮೋಹಾದುಡುಪೇನಾಸ್ಮಿ ಸಾಗರಮ್” ಎಂಬಂತೆ ಪ್ರಯತ್ನ ತತ್ಪರನಾಗಿದ್ದೇನೆ. 
\vskip 2pt

ತಮ್ಮ ಸರ್ವಸ್ವವನ್ನೂ ಶಿಷ್ಯರ ಹಿತಕ್ಕಾಗಿ ಮುಡುಪಾಗಿಟ್ಟ, ಎಲ್ಲರ ಉತ್ಸಾಹದ ಚಿಲುಮೆ\-ಯಾಗಿರುವ,  ಗುರುಗಳಾದ ಗಂಗಾಧರ ಭಟ್ಟರ ವಿಶ್ರಾಂತಜೀವನ ಸುಖಮಯವಾಗಿರಲೆಂದು ನನ್ನ ಆರಾಧ್ಯದೇವನಾದ ವಿದ್ಯಾಗಣಪತಿಯನ್ನು ಪ್ರಾರ್ಥಿಸಿ ಈ ವಾಕ್ಪುಷ್ಪ\-ವನ್ನು ಗುರುಚರಣಗಳಿಗೆ ಅರ್ಪಿಸುತ್ತಿದ್ದೇನೆ.

\articleend
}
