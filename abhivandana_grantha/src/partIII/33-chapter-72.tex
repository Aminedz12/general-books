{\fontsize{14}{16}\selectfont
\chapter{ಆಚಾರ್ಯ ಗಂಗಾಧರಭಟ್ಟರು}

\begin{center}
\Authorline{ಜೈರಾಮ ಭಟ್ಟ}
\smallskip
ಸಂಸ್ಕೃತ ಅಧ್ಯಾಪಕ\\
ಬೆಂಗಳೂರು
\addrule
\end{center}
\begin{center}
\textbf{ಗುರುರ್ಬ್ರಹ್ಮಾ ಗುರುರ್ವಿಷ್ಣುಃ ಗುರುರ್ದೆವೋ ಮಹೇಶ್ವರಃ~।\\
ಗುರುಃ ಸಾಕ್ಷಾತ್ ಪರಂ ಬ್ರಹ್ಮ ತಸ್ಮೈ ಶ್ರೀ ಗುರವೇ ನಮಃ~॥}
\end{center}
ಸ್ನಾನ ಮಾಡಿದ ನಂತರ ಸಂಧ್ಯಾನಮನ ಸಾಧ್ಯವಾಗದಿದ್ದರೂ ಕೊನೆಯ ಪಕ್ಷ ಈ ಮೇಲಿನ ಶ್ಲೋಕವನ್ನಂತೂ ಉಚ್ಚರಿಸುವ ಇಲ್ಲವೇ ಸ್ಮರಿಸುವ ಪರಿಪಾಠವನ್ನು ನಾನು ಹೊಂದಿದ್ದೇನೆ. ಹಾಗೆ ಕಣ್ಮುಚ್ಚಿ ಸ್ಮರಿಸುವ ಸಂದರ್ಭದಲ್ಲಿ ಪ್ರಪ್ರಥಮವಾಗಿ ಒಂದು ಮುಖಬಿಂಬ ಮನಸ್ಸಿನಲ್ಲಿ ಚಿತ್ರಿತವಾಗುತ್ತದೆ. ಅದು ಗಂಗಾಧರ ಭಟ್ಟರ ಮುಖಬಿಂಬ. ಆದ್ದರಿಂದ ಗುರು ಎಂಬ ಪದದ ಶಕ್ತಿಯನ್ನು ನಾನು ಗಂಗಾಧರ ಭಟ್ಟರಲ್ಲಿ ಗುರುತಿಸುತ್ತೇನೆ.

ವಿದ್ವತ್ ಮಾಡಲು ಮೈಸೂರಿಗೆ ಬಂದೊಡನೆ ನಮ್ಮ ತಂದೆಯವರ ಪೂರ್ವ ನಿರ್ದೇಶನದಂತೆ ಗಂಗಾಧರ ಭಟ್ಟರನ್ನು ಭೇಟಿಯಾದೆ. ಅದು ಕೇವಲ ಭೇಟಿಯಾಗಿರಲಿಲ್ಲ. ನನ್ನ ಪಾಲಿಗೆ ಜೀವನದಲ್ಲಿ ನಾನು ಪಡೆದ ಮಹಾನ್ ಉಪಾಧ್ಯಾಯರ ದರ್ಶನವಾಗಿತ್ತು ಎಂದು ಈಗ ನನಗೆ ಅನ್ನಿಸುತ್ತದೆ. ಐದು ವರ್ಷಗಳ ಕಾಲ ಅವರ ಅಂತೇವಾಸಿಯಾಗುವ ಅವಕಾಶವು ಒಂದು ಅದೃಷ್ಟವೆಂದೂ, ಛಾತ್ರ ಜೀವಿತದ ಸಾರ್ಥಕ್ಯವೆಂದೂ ನಾನು ಭಾವಿಸುತ್ತೇನೆ.

ಅಂದು ಭಟ್ಟರನ್ನು ಕಾಣಲು ಹೋದ ಸಂದರ್ಭದಲ್ಲಿ ಅವರು ಮಂಚದ ಮೇಲೆ ತಾಂಬೂಲಪೂರಿತವದನರಾಗಿ ಕಟ್ಟಾ ಹವ್ಯಕನಂತೆ ಕುಳಿತಿದ್ದರು. ಒಂದು ಹೊತ್ತಗೆಯನ್ನು ತದೇಕಚಿತ್ತದಿಂದ ಅವಗಾಹನೆ ಮಾಡುತ್ತಿದ್ದರು. ಅಣ್ಣನ ಜೊತೆ ತೆರಳಿದ್ದ ನಾನು ಅವರಿಗೆ ಕೈ ಮುಗಿದೆ. ಭಟ್ಟರು “ಎಂತಕ್ ಬೈಂದ್ಯಾ? ಮೈಸೂರಿಗೆ” ಎಂದು ಮಂದಸ್ಮಿತರಾಗಿ ಕೇಳಿದ್ದರು. ಭಟ್ಟರ ಈ ಪ್ರಶ್ನೆಗೆ ತಕ್ಷಣ ಉತ್ತರ ಹೊಳೆಯಲಿಲ್ಲ. ಇನ್ನೂ ಪರಿಚಯವಾಗದಿರುವಾಗ ಹೀಗೊಂದು ಪ್ರಶ್ನೆ ಕೇಳಬಹುದೆಂದು ನಾನು ಊಹಿಸಿಯೂ ಇರಲಿಲ್ಲ. ಭಟ್ಟರದ್ದು ನೇರ ಮಾತು. ಹೊಸದಾಗಿ ಪರಿಚಯವಾದಾಗ ಅಹಂಕಾರವೋ ಎಂದು ಭಾಸವಾಗಬಹುದಾದ ಸಹಜ ಸ್ವಾಭಿಮಾನ ಕೆಲವೇ ಮಾತುಗಳಲ್ಲಿ ಅವರ ಆತ್ಮೀಯತೆಯ ಅರಿವಾಯಿತು. ವೇದ ಶಾಸ್ತ್ರ ಪೋಷಿಣೀ ಸಭೆಯಲ್ಲಿ ಭೋಜನಕ್ಕೆ ವ್ಯವಸ್ಥೆ ಮಾಡಿಕೊಟ್ಟರು. ಹಾಸ್ಟೆಲ್ನಲ್ಲಿ ಉಳಿಯಲು ಅವಕಾಶ ಕಲ್ಪಿಸಿಕೊಟ್ಟರು. ಭಟ್ಟರ ಸಹಾಯದ ಮನೋಧರ್ಮ ಒಬ್ಬಿಬ್ಬ ಶಿಷ್ಯರಿಗೆ ಸೀಮಿತವಾದುದಲ್ಲ. ಅವರಲ್ಲಿ ಸಹಾಯಕ್ಕೆ ಕೈ ಚಾಚಿದ ಎಲ್ಲರಿಗೂ ಲಭ್ಯವಾಗುತ್ತದೆ. ಸಂಸ್ಕೃತ ವಿದ್ಯಾರ್ಥಿಗಳಲ್ಲದೇ ಬೇರೆ ಬೇರೆ ಕಾಲೇಜುಗಳಿಗೆ ಅಧ್ಯಯನಾಕಾಂಕ್ಷಿಗಳಾಗಿ ಬರುವವರೂ ಸಹ ಇವರಿಂದ ಉಪಕೃತರಾಗುತ್ತಾರೆ.

ಭಟ್ಟರದ್ದು ವೃತ್ತಿಯೇ ಪ್ರವೃತ್ತಿ. ಪ್ರವೃತ್ತಿ ಅಧ್ಯಾಪನ. ಹಾಗಾಗಿ ಪಾಠಶಾಲೆಯಲ್ಲೂ ಪಾಠ. ಅವರ ಮನೆಯಲ್ಲಿಯೂ ಪಾಠ. ಜ್ಞಾನಾರ್ಜನೆಗೆಂದು ಭಟ್ಟರನ್ನು ಹುಡುಕಿಕೊಂಡು ಬರುವವರು ಕೇವಲ ಪಾಠಶಾಲೆಯ ವಿದ್ಯಾರ್ಥಿಗಳು ಮಾತ್ರವಲ್ಲ. ಸನ್ಯಾಸಭಿಕ್ಷೆಯನ್ನು ಪಡೆದ ಸನ್ಯಾಸಿಗಳೂ ಜ್ಞಾನ ಭಿಕ್ಷುಗಳಾಗಿ ಭಟ್ಟರನ್ನು ಆಶ್ರಯಿಸುತ್ತಾರೆ. ವಿದೇಶೀಯರೂ ಭಾರತೀಯ ತತ್ತ್ವಶಾಸ್ತ್ರದ ಜಿಜ್ಞಾಸುಗಳಾಗಿ ಬರುತ್ತಾರೆ. ಯಾರಲ್ಲಿಯೂ ತಾರತಮ್ಯ ಮಾಡದೇ ಶಿಷ್ಯತ್ವವನ್ನು ಮಾತ್ರ ಗುರುತಿಸಿ ಭಟ್ಟರು ವಿದ್ಯಾವಿತರಣೆ ಮಾಡುತ್ತಾರೆ. ವಿದ್ಯಾವಿತರಣೆ ಒಂದು ಕೌಶಲ. ಭಟ್ಟರು ಜ್ಞಾನಗಂಗೆಯನ್ನೇ ಶಿರಸ್ಸಿನಲ್ಲಿ ಧರಿಸಿದವರು, ಹೃದ್ಗತಮಾಡಿಕೊಂಡವರು. ಹಾಗಿದ್ದರೂ ವಿದ್ಯಾರ್ಥಿಯ ಜ್ಞಾನದ ಸ್ತರವನ್ನು ಗುರುತಿಸಿ ಅವನ ಗ್ರಹಣಯೋಗ್ಯತೆಗೆ ತಕ್ಕಂತೆ ತಮ್ಮ ಜ್ಞಾನವನ್ನು ಪ್ರದಾನ ಮಾಡುವ ಭಟ್ಟರ ಅಧ್ಯಾಪನ ಶೈಲಿಯು ಅನುಪಮವಾದದ್ದು.

ಭಟ್ಟರಿಗೆ ಶಿಷ್ಯವೃಂದದ ಮೇಲೆ ಪ್ರತಿಫಲಾಪೇಕ್ಷೆಯಿಲ್ಲದ ಪ್ರೇಮ. ಅಲ್ಲದೇ, ಒಬ್ಬ ತಂದೆಗೆ ಮಗನ ಮೇಲಿರುವ ಕಾಳಜಿ. ನಾನು ಊರಿಗೆ ಹೋಗುವಾಗಲೆಲ್ಲ “ದುಡ್ಡು ಬೇಕನಾ?” ಎಂದು ಕೇಳುತ್ತಿದ್ದರು. ಪಾಠಕ್ಕೆಂದು ಅಥವಾ ಇನ್ಯಾವುದೋ ಕಾರಣಕ್ಕೆ ಅವರ ಮನೆಗೂ ಆಗಾಗ ಹೋಗುವ ಸಂದರ್ಭವಿರುತ್ತಿತ್ತು. ಹಾಗೆ ಹೋದಾಗ ಭಟ್ಟರ ಶ್ರೀಮತಿಯವರಾದ ಶೈಲಕ್ಕನವರೂ ಅಷ್ಟೇ ಪ್ರೀತಿಯಿಂದ ಕಾಣುತ್ತಾರೆ. ನಾನು ಭಟ್ಟರ ಮನೆಯಲ್ಲಿಯೇ ಎಷ್ಟೋ ದಿನ ಚಹಾ ಕುಡಿದಿದ್ದನ್ನೂ, ಊಟ ಮಾಡಿದ್ದನ್ನೂ ಸದಾ ಸ್ಮರಿಸುತ್ತೇನೆ.

ವೇಷಭೂಷಣಗಳ ವಿಷಯದಲ್ಲಿ ಭಟ್ಟರದ್ದು ಅಲ್ಪಾದರ. ಅವರಿಗೆ ಮಾತೇ ಭೂಷಣ. ಅವರ ಮಾತುಗಳನ್ನು ಕೇಳುತ್ತಿದ್ದರೆ ಕೇಳುಗನಿಗೆ ಸಮಯ ಕಳೆದದ್ದು ತಿಳಿಯುವುದಿಲ್ಲ. ಆದರೂ ಭಟ್ಟರು ಸಮಯ ಪ್ರಜ್ಞೆಯಿಲ್ಲದೇ ಮಾತನಾಡುವವರಲ್ಲ. ದೇಶಪ್ರಜ್ಞೆಯಿಲ್ಲದೇ ವ್ಯವಹರಿಸುವವರಲ್ಲ. ಅವರು ಸಹಜವಾಗಿ ಆಡುವ ತಿಳಿಹಾಸ್ಯಮಿಶ್ರಿತವಾದ, ವ್ಯಂಗ್ಯಪೂರಿತವಾದ ಮಾತುಗಳಲ್ಲಿ ಶಿಷ್ಯನ ಬುದ್ಧಿಯ ವಕ್ರತೆಯನ್ನು ತಿದ್ದುವ ಶಕ್ತಿ ಇರುತ್ತದೆ. ಅವು ಜೀವನದುದ್ದಕ್ಕೂ ಮಾರ್ಗದರ್ಶಕವಾಗಿರುತ್ತದೆ. ನಾನು ಬಹಳಷ್ಟು ಸನ್ನಿವೇಶಗಳಲ್ಲಿ ಭಟ್ಟರ ಮಾತುಗಳನ್ನು ನೆನಪಿಸಿಕೊಳ್ಳುತ್ತೇನೆ.

ನಾನು ಗಮನಿಸಿದಂತೆ ಭಟ್ಟರದ್ದು ನಿರಾಡಂಬರವಾದ ಆಸ್ತಿಕ್ಯಬುದ್ಧಿ. ಅದು ತೋರಿಕೆಯ ಪ್ರತಿಷ್ಠೆಯಲ್ಲ. ಭಟ್ಟರ ಮನೋಧೋರಣೆಯಲ್ಲಿ ಒಂದು ನಿಶ್ಚಲತೆಯ ನಿಲುವಿದೆ. ಆ ನಿಲುವಿಗೆ ಬದ್ಧವಾದ ಸಂತತ ಕ್ರಿಯಾಪ್ರವೃತ್ತಿಯಿದೆ. ಹಸನ್ಮುಖಿಯಾದ ಅವರ ನಡೆಯಲ್ಲಿ ನುಡಿಯಲ್ಲಿ ಬಹಳ ಗಾಂಭೀರ್ಯವಿದೆ. ಅವಸರೋಚಿತವಾದ, ಅವರ ವ್ಯಕ್ತಿತ್ವಕ್ಕೆ ತೂಗುವ, ಅನ್ಯರಿಗೆ ಕಿಂಚಿತ್ತೂ ಹಾನಿಯುಂಟು ಮಾಡದ ಆತ್ಮಾಭಿಮಾನಿಯಾದ ಗಂಗಾಧರ ಭಟ್ಟರಿಗೆ ಅವರೇ ಸಾಟಿ. ಅವರು ಎಂದೂ ಭಾವುಕತೆಗೆ ಒಳಗಾಗುವವರಲ್ಲ. ವಾಸ್ತವತೆಯ ಅರಿವಿನಲ್ಲಿ ವ್ಯವಹರಿಸುವವರು.

ನನ್ನ ದೃಷ್ಟಿಯಲ್ಲಿ ಭಟ್ಟರು ಕೇವಲ ಭಟ್ಟರಷ್ಟೇ ಅಲ್ಲ, ಭಟ್ಟಾಚಾರ್ಯರು ಎಂದು ಸಂಬೋಧಿಸಲು ಬಯಸುತ್ತೇನೆ. ಅವರ ವಿಶ್ರಾಂತ ಜೀವನವು ಸುಖಮಯವಾಗಿರಲೆಂದು ಪರಮಾತ್ಮನಲ್ಲಿ ಪ್ರಾರ್ಥಿಸುತ್ತೇನೆ. ಭಟ್ಟಾಚಾರ್ಯರಿಗೆ ಇನ್ನೊಮ್ಮೆ, ಮತ್ತೊಮ್ಮೆ ಪ್ರಣಾಮಗಳನ್ನು ಅರ್ಪಿಸುತ್ತೇನೆ.

\articleend	
}
