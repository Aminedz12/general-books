{\fontsize{14}{16}\selectfont
\chapter{ಗುರುವಿನೊಡನೆ ಸಂಸ್ಕೃತದೆಡೆಗೆ}

\begin{center}
\Authorline{ವಿ.ಪಿ ಶಂಕರ್, ಎಂ ಎ}
\smallskip

ಮೈಸೂರು
\addrule
\end{center}

\begin{center}
ಅಜ್ಞಾನ ತಿಮಿರಾಂಧಸ್ಯ ಜ್ಞಾನಾಂಜನಶಲಾಕಯಾ~।\\
ಚಕ್ಷುರುನ್ಮೀಲಿತಂ ಯೇನ ತಸ್ಮೈಶ್ರೀ ಗುರವೇ ನಮಃ~॥
\end{center}
ಅರಿವಿನ ಕಡೆಗೆ ಹೊರಡಬಯಸುವ ಪ್ರತಿಯೊಬ್ಬರಿಗೂ ಗುರುವಿನ ಅವಶ್ಯಕತೆ ಇದೆ. ಗುರುವಿನ ಉಪದೇಶವಿಲ್ಲದೆ ಎಲ್ಲವನ್ನು ನಾವೇ ತಿಳಿಯಲಾರೆವು. ಗುರುವೆಂದರೆ ಕೇವಲ ಆಕಾರವಲ್ಲ ಅದೊಂದು ಶಕ್ತಿ, ಶಿಷ್ಯ ಅಥವಾ ಅನುಯಾಯಿಗೆ ಜ್ಞಾನ ಪ್ರವಹಿಸುವ ಮಾಧ್ಯಮ. ಬದುಕಿನ ವಿಕಾಸದೆಡೆಗೆ ಒಯ್ಯುವ ಚೈತನ್ಯ. ನಮ್ಮ ನಿಜಸ್ವರೂಪವನ್ನು ನಮಗೆ ತಿಳಿಸಿಕೊಟ್ಟು ಆತ್ಮೋನ್ನತಿಯೆಡೆಗೆ ದಾರಿ ತೋರುವ ಬೆಳಕು. ನಿಜವಾದ ಗುರುಗಳು ಶಿಷ್ಯರ ಕೈಯನ್ನು ಹಿಡಿಯುತ್ತಾರೆ, ಮನಸ್ಸನ್ನು ತೆರೆಸುತ್ತಾರೆ ಹಾಗೂ ಹೃದಯವನ್ನು ಮುಟ್ಟುತ್ತಾರೆ.

ಇಂತಹ ಗುರುಗಳ ಸತ್ಸಂಗ ಎಲ್ಲರಿಗೂ ದೊರೆಯುವಂತಹದ್ದಲ್ಲ, ಅದು ನನ್ನ ಅದೃಷ್ಟವೆಂದೇ ಹೇಳಬೇಕು. ನಮ್ಮ ಕುಟುಂಬದಲ್ಲಿ ಯಾರೂ ಕೂಡ ಸಂಸ್ಕೃತ ಭಾಷೆಯ ಜ್ಞಾನ ಪಡೆದವರಿಲ್ಲ, ಆದರೂ ಬಾಲ್ಯದಿಂದಲೂ ಆ ದಿವ್ಯಸ್ವರೂಪದ ಭಾಷೆಯ ಬಗ್ಗೆ ಏನೋ ಒಂದು ಬಗೆಯ ಆಕರ್ಷಣೆ, ಅದನ್ನು ಕಲಿಯಬೇಕು, ಅದರ ಬಗ್ಗೆ ಹೆಚ್ಚು ತಿಳಿಯಬೇಕು ಎಂಬ ಹಂಬಲ. ಅದನ್ನು ಕಲಿಯದಿದ್ದರೆ ಕಲಿತ ವಿದ್ಯೆ ಅಪೂರ್ಣ ಎನ್ನುವ ಕೊರಗು. 

ನಾನು ಆಗತಾನೇ ಪಿ.ಯು.ಸಿ. ವ್ಯಾಸಂಗವನ್ನು ಪೂರ್ಣಗೊಳಿಸಿದ್ದೆ. ಸಂಸ್ಕೃತ ಕಲಿಯುವ ಉತ್ಸಾಹದಲ್ಲಿ ಸೂಕ್ತ ಶಾಲೆಯನ್ನು ಹುಡುಕುತ್ತಿರುವಾಗ ಅರಿಯದೆ ಬಂದ ಅದೃಷ್ಟ\-ದಂತೆ ಕಂಡಿದ್ದು ಅಗ್ರಹಾರದ ಶ್ರೀ ಶಂಕರ ವಿಲಾಸ ಸಂಸ್ಕೃತ ಮತ್ತು ವೇದ ಪಾಠಶಾಲೆ. ತರಗತಿಯನ್ನು ಪ್ರವೇಶಿಸಿದಾಗ ಅಲ್ಲಿ ಕಂಡಿದ್ದು ಸರಳ ಉಡುಪಿನ, ನಗು\-ಮೊಗದ, ಹೊಳಪಿನ ಕಂಗಳ ಕಾಂತಿಯುಕ್ತ ಶರೀರದ, ವಿದ್ಯಾರ್ಥಿಗಳಿಗೆ ಭೂತವನ್ನು ಅರ್ಥ\-ಮಾಡಿಸುವ, ವರ್ತಮಾನವನ್ನು ತಿಳಿಸುವ, ಭವಿಷ್ಯವನ್ನು ರೂಪಿಸುವ ತ್ರಿವೇಣಿ ಸಂಗಮದಂತೆ ಕಂಡವರು ಗುರುಗಳಾದ “ವಿದ್ವಾನ್ ಗಂಗಾಧರ ವಿ~॥ ಭಟ್ಟರು”. ಆಗ ಅವರು ಆ ಶಾಲೆಯ ಮುಖ್ಯೋಪಾಧ್ಯಾಯರಾಗಿ ಕರ್ತವ್ಯ ನಿರ್ವಹಿಸುತ್ತಿದ್ದರು. ವಿದ್ಯಾರ್ಥಿ\-ಗಳ ಸಂಖ್ಯೆಯೂ ಉತ್ತಮವಾಗಿತ್ತು. ಅಲ್ಲಿ ಪ್ರವೇಶ ಪಡೆದು ಗುರುಗಳ ಮಾರ್ಗದರ್ಶನದಲ್ಲಿ ಸಂಸ್ಕೃತ ಅಭ್ಯಾಸ ಪ್ರಾರಂಭವಾಯಿತು. ತರಗತಿಗಳು ಬೆಳಗ್ಗೆ ಮತ್ತು ಸಂಜೆ ಶಿಸ್ತು ಬದ್ಧವಾಗಿ ನಡೆಯುತ್ತಿದ್ದವು. ಅಲ್ಲಿನ ಎಲ್ಲಾ ಶಿಕ್ಷಕರು ಪ್ರತಿಯೊಂದನ್ನು ಶ್ರದ್ಧೆ, ತಾಳ್ಮೆ, ಆಸಕ್ತಿಯಿಂದ ಕಲಿಸುತ್ತಿದ್ದರು. ಗುರುಗಳ ಪರಿಚಯವಾಗಿದ್ದು ಹೀಗೆ.

ಸಂಸ್ಕೃತ ವಿಷಯ ಬೋಧನೆಯಲ್ಲಿ ಗುರುಗಳಿಗೆ ಅವರೇ ಸಾಟಿ. ಅದು ಅಮರಕೋಶ\-ವಿರಲಿ, ಗದ್ಯವಿರಲಿ, ಪದ್ಯವಿರಲಿ ವ್ಯಾಕರಣವಿರಲಿ, ಅಲಂಕಾರವಿರಲಿ, ತರ್ಕಶಾಸ್ತ್ರ\-ವಿರಲಿ, ಅವರು ಬೋಧಿಸುತ್ತಿದ್ದ ರೀತಿಯೇ ಸೊಗಸಾಗಿರುತ್ತಿತ್ತು. ಪ್ರತಿಯೊಂದನ್ನು ಕಣ್ಣಿಗೆ ಕಟ್ಟುವಂತೆ ವಿವರಿಸುತ್ತಿದ್ದರು, ಹಲವಾರು ಸರಳ ಉದಾಹರಣೆಗಳ ಮೂಲಕ ಅರ್ಥಮಾಡಿ\-ಸುತ್ತಿದ್ದರು, ಜೊತೆಗೆ ಅದು ರಂಜನೀಯವೂ ಆಗಿರುವಂತೆ ಜಾಗ್ರತೆವಹಿಸು\-ತ್ತಿದ್ದರು. ರಸಕ್ಕೆ ಅನುಗುಣವಾಗಿ ಮುಖಭಾವಾಭಿನಯ ಮಾಡಿ ಬೋಧಿಸುತ್ತಿದ್ದರು. ನಿಜವಾದ ಶಿಕ್ಷಕರು ಮಕ್ಕಳ ಮಟ್ಟಕ್ಕೆ ಇಳಿದು, ತಮ್ಮ ಆತ್ಮವನ್ನು ವಿದ್ಯಾರ್ಥಿಗಳ ಆತ್ಮಕ್ಕೆ ವರ್ಗಾಯಿಸಿ ವಿದ್ಯಾರ್ಥಿಗಳ ದೃಷ್ಟಿಯಿಂದ ನೋಡಿ, ತಮ್ಮ ಶ್ರೋತೃಗಳಿಂದ ಕೇಳಿ ತಮ್ಮ ಮನಸ್ಸಿನ ಮೂಲಕ ಅರ್ಥಮಾಡಿಕೊಳ್ಳುತ್ತಾರೆ. ಈ ಗುರುಗಳು ಕೂಡ ಹಾಗೆಯೇ ಕಬ್ಬಿಣದ ಕಡಲೆ ಎಂದು ಜನಸಾಮಾನ್ಯರು ತಿಳಿದಿರುವ ಸಂಸ್ಕೃತ ಭಾಷೆ ಇಷ್ಟು ಸರಳ ಸುಂದರ ಸುಮಧುರ ಎನಿಸಿದ್ದು ಗುರುಗಳ ಕಂಠಸಿರಿಯಿಂದ ಕೇಳಿದಾಗ. ಅವರ ಬೋಧನೆಯಲ್ಲಿ ನವರಸಗಳೂ ತುಂಬಿರುತ್ತಿತ್ತು. ಪದ್ಯಗಳನ್ನು, ಶ್ಲೋಕಗಳನ್ನು ಸುಶ್ರಾವ್ಯವಾಗಿ ಹಾಡಿ ಹೇಳಿಕೊಡುತ್ತಿದ್ದರು. ಅವರು ಕಲಿಸುವ ರೀತಿಯಲ್ಲಿ ಯಾವುದೇ ವಿದ್ಯಾರ್ಥಿಗೆ ಮನದಟ್ಟಾಗದೆ ಇರಲಾರದು ಮತ್ತು ಮರೆಯುವುದು ಸಾಧ್ಯವೇ ಇಲ್ಲ. ಅವರು ಕಲಿಸಿದ್ದು ಅಲ್ಲಿಯೇ ಕಂಠಸ್ಥವಾಗಿಬಿಡುತ್ತಿತ್ತು. ಅವರ ಬೋಧನೆಯಲ್ಲಿ ಪಾಂಡಿತ್ಯ, ಭಾಷಾ ಪ್ರಯೋಗ, ಕ್ಷಣಸ್ಪುರತೆ, ವಿವರಣೆ, ವಿಶ್ಲೇಷಣೆ ಮತ್ತು ಅದ್ಭುತ ಹಾಸ್ಯಪ್ರಜ್ಞೆ ಎಲ್ಲವೂ ಮೇಳೈಸಿರುತ್ತಿತ್ತು. ಕೇಳಲು ಕರ್ಣಾನಂದವಾಗುತ್ತಿತ್ತು, ಅವರ ಪಾಠವನ್ನು ಕೇಳುವಾಗ ಕಾಲದ ಪರಿವೇ ಇರುತ್ತಿರಲಿಲ್ಲ. ಅಯ್ಯೋ! ಉಪನ್ಯಾಸ ಇಷ್ಟು ಬೇಗ ಮುಗಿದುಹೋಯಿತೇ ಎನಿಸುತ್ತಿತ್ತು. ವಾಲ್ಮೀಕಿ ರಾಮಾಯಣದ ಸೀತಾಪರಿತ್ಯಾಗ ಮುಂತಾದ ಪ್ರಸಂಗಗಳನ್ನು ಬೋಧಿಸುವಾಗ ನಮಗರಿಯದಂತೆ ನಮ್ಮ ಕಣ್ಣಂಚಿನಲ್ಲಿ ನೀರು ತುಂಬಿಕೊಳ್ಳುತ್ತಿತ್ತು, ಅಷ್ಟರಮಟ್ಟಿಗೆ ತಮ್ಮ ವಾಗ್ಝರಿಯಿಂದ ಕಾವ್ಯಚಿತ್ರ ಮೂಡಿಸುತ್ತಿದ್ದರು ರಸದರ್ಶನ ಮಾಡಿಸುತ್ತಿದ್ದರು.

ನನ್ನ ಸಂಸ್ಕೃತ ಪ್ರಥಮಾ, ಕಾವ್ಯ ಮತ್ತು ಸಾಹಿತ್ಯದ ವ್ಯಾಸಂಗ ನಡೆದಿದ್ದು ಆ ಪಾಠಶಾಲೆಯಲ್ಲಿಯೇ. ನಂತರ ಗುರುಗಳು ಶ್ರೀ ಮನ್ಮಮಹಾರಾಜ ಸಂಸ್ಕೃತ ಪಾಠಶಾಲೆಯಲ್ಲಿ ನವೀನ ನ್ಯಾಯ ವಿಭಾಗದಲ್ಲಿ ಸಹಾಯಕ ಪ್ರಾಧ್ಯಾಪಕರಾಗಿ ನೇಮಕಗೊಂಡರು. ಅವರ ಜ್ಞಾನ ಪ್ರಭೆ, ಪಾಂಡಿತ್ಯ ಹೆಚ್ಚು ಹೆಚ್ಚು ವಿದ್ಯಾರ್ಥಿಗಳನ್ನು ಆಕರ್ಷಿಸಲಾರಂಭಿಸಿತು. ಮಹಾರಾಜ ಕಾಲೇಜಿನ ಜ್ಞಾನ ಮುಕುಟವನ್ನು ಮತ್ತೊಂದು ರತ್ನಮಣಿಯಿಂದ ಅಲಂಕರಿಸಿದಂತಾಯಿತು. ಈ ನಡುವೆ ನಾನೂ ಸಹ ವೇದ ಹಾಗೂ ವಿದ್ವತ್ ಮಧ್ಯಮಾ ತರಗತಿಗೆ ಪ್ರವೇಶ ಪಡೆದೆ. ಆದರೆ ಕಾರಣಾಂತರಗಳಿಂದ ವಿದ್ಯಾಭ್ಯಾಸ ಮುಂದುವರೆಸಲಾಗಲಿಲ್ಲ. ಅನಂತರ ಸಾಹಿತ್ಯ ಆಧಾರದ ಮೇಲೆ ಮೈಸೂರಿನ ಮಾನಸ ಗಂಗೋತ್ರಿಯಲ್ಲಿ ಸಂಸ್ಕೃತ ಎಂ.ಎ. ವ್ಯಾಸಂಗ ಮುಗಿಸಿದೆ ವಿಶ್ವವಿದ್ಯಾನಿಲಯಕ್ಕೆ ಮೊದಲಿಗನಾಗಿ ಉತ್ತೀರ್ಣನಾದೆ. ಇದರ ಶ್ರೇಯಸ್ಸು ಗುರುಗಳಿಗೇ ಸಲ್ಲಬೇಕು.

ಗುರುಗಳ ಜೊತೆ ಬಾಂಧವ್ಯ ಇನ್ನೂ ಹೆಚ್ಚಾಯಿತು ನಿಜವಾದ ಗುರು\enginline{-}ಶಿಷ್ಯರ ಬಾಂಧವ್ಯ ಆರಂಭವಾಗುವುದು ವ್ಯಾಸಂಗ ಪೂರ್ಣವಾದ ಮೇಲೆಯೇ. ಅವರ ಆಪ್ತ ಶಿಷ್ಯವರ್ಗದಲ್ಲಿ ನಾನೂ ಒಬ್ಬ ಎಂದು ಹೇಳಿಕೊಳ್ಳಲು ಹೆಮ್ಮೆಯೆನಿಸುತ್ತದೆ. ಸಮಯ ಸಿಕ್ಕಾಗಲೆಲ್ಲಾ ನ್ಯೂ ಸಯ್ಯಾಜಿರಾವ್ ರಸ್ತೆಯ ಸಮೀಪದಲ್ಲಿದ್ದ ಅವರ ಮನೆಗೆ ಹೋಗುತ್ತಿದ್ದೆ. ಅವರ ಮನೆಯ ಸದಸ್ಯರಲ್ಲಿ ನಾನೂ ಒಬ್ಬನಾಗಿಬಿಟ್ಟಿದ್ದೆ. ಅವರ ಮನೆಯೊಂದು ಗುರುಕುಲದಂತೆ ಇತ್ತು. ಅವರ ಶ್ರೀಮತಿಯವರೂ ಸಹ ಪ್ರೀತಿ, ಅತ್ಯಾದರಗಳಿಂದ ನನ್ನನ್ನು ಉಪಚರಿಸುತ್ತಿದ್ದರು. ರುಚಿ ರುಚಿಯಾದ ಅಡುಗೆಯನ್ನು ಉಣಬಡಿಸುತ್ತಿದ್ದರು. ಅವರ ಮನೆಯಿಂದ ಹೊರಡಲು ಮನಸ್ಸೇ ಇರುತ್ತಿರಲಿಲ್ಲ. ಅವರ ಮನೆಗೆ ನಿತ್ಯವೂ ಅನೇಕ ವಿದ್ಯಾರ್ಥಿಗಳು ಗಣ್ಯರು ವಿದ್ವಾಂಸರು ವಿದೇಶೀಯರು, ಶಿಕ್ಷಕರು ಸಮಾಜದ ಎಲ್ಲ ರೀತಿಯ ಜನರು ಗುರುಗಳ ಸಲಹೆಗಾಗಿ, ಮಾರ್ಗದರ್ಶನಕ್ಕಾಗಿ, ಚರ್ಚೆಗಾಗಿ ಬರುತ್ತಿದ್ದರು. ಯಾರೊಬ್ಬರೂ ನಿರಾಶರಾಗಿ ಹಿಂದಿರುಗುತ್ತಿರಲಿಲ್ಲ. ನನ್ನನ್ನು ಕಲಿಸುವಲ್ಲಿ ಒಬ್ಬ ಶಿಷ್ಯನಂತೆ ಉಪಚರಿಸುವಲ್ಲಿ ಪುತ್ರನಂತೆ ಇತರ ಸಮಯದಲ್ಲಿ ಸ್ನೇಹಿತನಂತೆ ಕಂಡಿದ್ದಾರೆ. ಎಷ್ಟು ಜನರಿಗೆ ಈ ಭಾಗ್ಯ ಲಭಿಸುವುದೋ ಗೊತ್ತಿಲ್ಲ. ಗುರುಗಳ ನಿಷ್ಕಾಮ ಪ್ರೇಮ\break ಅಂತಹುದು.

ಗುರುಗಳು ‘ಪದ್ಮಪತ್ರಮಿವಾಂಭಸಿ’ ಎನ್ನುವ ಮನೋಧರ್ಮ ಹೊಂದಿರುವವರು. ಹಿರಿಯರು ಹೇಳಿರುವಂತೆ ‘ಅಂಟಿಯೂ ಅಂಟದೆ ನೆಂಟನಾಗಿರಬೇಕು ಬಲಗೈಯ ಹೆಬ್ಬೆಟ್ಟ ಹಾಗೆ. ಕಿರಿಯನಾಗಿರೆ ನಾಲ್ವರಲಿ ಬೆರೆತಿರಬೇಕು ಕೈಯ ಕಿರುಬೆರಳ ಹಾಗೆ ಎನ್ನುವ ಸ್ವಭಾವ ಅವರದು. ಎಲ್ಲವನ್ನು ಯಥಾರ್ಥ ಜ್ಞಾನದಿಂದ ಕಾಣುವ ದೃಷ್ಟಿಕೋನ ಅವರದು. “ನಿಜವಾದ ಗುರುಗಳು ತಮ್ಮನ್ನು ಸೇತುವೆಯನ್ನಾಗಿಸಿಕೊಳ್ಳುತ್ತಾರೆ ಮತ್ತು ಆ ಸೇತುವೆಯ ಮೂಲಕ ಗುರಿತಲುಪಲು ಶಿಷ್ಯರಿಗೆ ಆಹ್ವಾನವೀಯುತ್ತಾರೆ. ಹೀಗೆ ಶಿಷ್ಯ ಗುರಿಮುಟ್ಟಿದಾಗ ಆನಂದಿಸುತ್ತಾ ಆ ಶಿಷ್ಯರುಗಳು ಮತ್ತೊಬ್ಬರಿಗೆ ಸೇತುವೆಯಾಗಲು ಸ್ಪೂರ್ತಿಯಾಗುತ್ತಾರೆ. ಅಂತಹ ಗುರುಶ್ರೇಷ್ಠರಲ್ಲಿ ನಮ್ಮ ಗುರುಗಳೂ ಒಬ್ಬರು.

ನನ್ನ ಬದುಕಿನಲ್ಲಿ ಮರೆಯಲಾಗದ ಎರಡು ಅನುಭವಗಳೆಂದರೆ, ಒಂದು; ನಾನು ಗುರುಗಳೊಡನೆ ತತ್ತ್ವಶಾಸ್ತ್ರ / ಸಂಸ್ಕೃತದ ಹೆಚ್ಚಿನ ಅಧ್ಯಯನಕ್ಕಾಗಿ ಪಾಂಡಿಚೇರಿಗೆ ಹೋಗಿದ್ದು ಮತ್ತು ಅವರ ಹುಟ್ಟೂರು ಸಿದ್ದಾಪುರದ ಮಣ್ಣಿಕೊಪ್ಪದಲ್ಲಿ ಅವರ ಕುಟುಂಬ\-ದೊಂದಿಗೆ ಹಲವು ದಿನಗಳನ್ನು ಕಳೆದಿದ್ದು.

ಪಾಂಡಿಚೇರಿಯ ಅರಬಿಂದೋ ಆಶ್ರಮ ಮತ್ತು ಅಲ್ಲಿನ ಸಂಸ್ಕೃತ ವಿಶ್ವವಿದ್ಯಾ\-ನಿಲಯಕ್ಕೆ ಹೆಚ್ಚಿನ ವ್ಯಾಸಂಗಕ್ಕಾಗಿ ಹೊರಟಾಗ ತಾವೇ ಜತೆಯಲ್ಲಿ ಬಂದು ನನಗೆ ಅಲ್ಲಿ ಇರಲು ಸೂಕ್ತ ಸ್ಥಳ ವ್ಯಾಸಂಗ ಮಾಡಲು ಸೂಕ್ತ ಪರಿಸರ, ಗ್ರಂಥಾಲಯ, ಗುರುಗಳು ಎಲ್ಲವನ್ನೂ ಪರೀಕ್ಷಿಸಿ ಸರಿಯಾದುದು ಯಾವುದೆಂದು ನಿರ್ಧರಿಸಿ ವ್ಯವಸ್ಥೆ ಮಾಡಿ ಬಂದಿದ್ದರು. ಇಡೀ ಪಾಂಡಿಚೇರಿಯನ್ನೆಲ್ಲಾ ಪರಿಚಯ ಮಾಡಿಸಿ ಅಪರಿಚಿತ ಸ್ಥಳದಲ್ಲಿ ಹೇಗಿರ\-ಬೇಕು ಎಂದು ತಿಳಿಸಿ ಅವರು ಹಿಂದಿರುಗುವಾಗ ನನ್ನ ಕಣ್ಣಂಚಿನಲ್ಲಿ ಕಂಬನಿಯಿತ್ತು. ಅಂತಹ ಮಾತೃಹೃದಯಿ ಅವರು. ಕಡೆಗೆ ಜ್ಞಾನಾರ್ಜನೆಗೆ ಮೈಸೂರೇ ತವರೂರು ಎಂದು ತಿಳಿದು ಅಲ್ಲಿಂದ ಹಿಂದಿರುಗಿದ್ದಾಯಿತು.

ಮತ್ತೊಂದು ಮರೆಯಲಾಗದ ಅನುಭವ ಅವರ ಹುಟ್ಟೂರು ಮಣ್ಣಿಕೊಪ್ಪದಲ್ಲಿ ಕೆಲದಿನಗಳನ್ನು ಕಳೆದಿದ್ದು. ಅವರ ಬಗ್ಗೆ ಒಂದು ಪ್ರತ್ಯೇಕ ಲೇಖನವನ್ನೇ ಬರೆಯಬಹುದು ಅಷ್ಟು ಸುಂದರ ಅನುಭವ ಅದು. ನಾನು ಮಣ್ಣಿಕೊಪ್ಪಕ್ಕೆ ಹೋದಾಗ ನಾನು ಅಲ್ಲಿ ಅಪರಿಚಿತ ಅಂತ ಅನ್ನಿಸಲೇ ಇಲ್ಲ. ಎಲ್ಲರೂ ತಮ್ಮ ಸಮೀಪದ ಬಂಧುವಿನಂತೆ ಸ್ವಾಗತಿಸಿದರು. ಋಷಿ ವಾಣಿ ‘ಅತಿಥಿ ದೇವೋಭವ’ ಇಲ್ಲಿ ಪ್ರತ್ಯಕ್ಷವಾಗಿ ಕಂಡೆ. ಗುರುಗಳ ಅಣ್ಣ ದಿ. ಮಂಜುನಾಥ ಭಟ್ಟರು, ಅತ್ತಿಗೆ, ಸಹೋದರ ಶ್ರೀಧರ ಭಟ್ಟರು, ರತ್ನಕ್ಕ ಎಲ್ಲರೂ ಸಜ್ಜನರು, ಹೃದಯವಂತರು ನನ್ನನ್ನು ತುಂಬು ವಾತ್ಸಲ್ಯದಿಂದ ಉಪಚರಿಸಿದರು.

ಕಾಂಕ್ರೀಟ್ ಕಟ್ಟಡಗಳ ನಡುವೆ ಸದಾ ವಾಹನಗಳ ಕರ್ಕಷ ಶಬ್ದಗಳೊಂದಿಗೆ ಕಲುಷಿತ ವಾತಾವರಣದಲ್ಲಿ ಬದುಕು ಕಂಡಿದ್ದ ನನಗೆ ಅಲ್ಲಿ ಸುಂದರ, ರಮ್ಯ ಪ್ರಕೃತಿಯ\break ದರ್ಶನವಾಯಿತು. ಸುತ್ತಲೂ ಬೆಟ್ಟ\enginline{-}ಗುಡ್ಡ ಕಾಡುಗಳ ಮಧ್ಯೆ ಗುರುಗಳ ಪೂರ್ವಿಕರ ಮನೆ ಇದೆ. ಎದುರಿಗೆ ಭತ್ತದ ಗದ್ದೆ, ಕಾಲುವೆ, ಮನೆಯ ಮುಂದೆಯೇ ಸಣ್ಣ ತೋಟ, ಪಕ್ಕದಲ್ಲಿಯೇ ಸ್ವತಃ ತಾವೇ ನಿರ್ಮಿಸಿದ ಮಾರುತಿ ದೇವಸ್ಥಾನ. ಒಂದು ಚೇತೋಹಾರಿಯಾದ ವಾತಾವರಣ ಕೇವಲ ಕತೆ ಕಾದಂಬರಿಗಳಲ್ಲಿ ಓದಿದ್ದ ವರ್ಣನೆಯನ್ನು ಪ್ರತ್ಯಕ್ಷವಾಗಿ ಕಂಡಂತಾಯಿತು. ಜೊತೆಗೆ ಸಂಪ್ರದಾಯದಲ್ಲಿ ಶ್ರದ್ಧೆ ಗೌರವ. ಮನೆಯಲ್ಲಿ ಸದಾ ಮಂತ್ರ ಘೋಷ, ಬೆಳಗಿನ ಉಪಹಾರದ ನಂತರ ವ್ಯಾಕರಣ, ಭಗವದ್ಗೀತೆ ಮತ್ತು ಹಲವಾರು ವಿಷಯಗಳ ಬಗ್ಗೆ ಚರ್ಚೆ ನಡೆಯುತ್ತಿತ್ತು. ಗುರುಗಳ ತಂದೆ ಪೂಜ್ಯ ಶ್ರೀ ವಿಘ್ನೇಶ್ವರ ಭಟ್ಟರಿಂದಲೂ ಪಾಠ ಕೇಳಿದ ಸೌಭಾಗ್ಯ ನನ್ನದು. ಹಲವಾರು ಕಥೆಗಳು, ದೈವಶಾಸ್ತ್ರ, ಧರ್ಮಗಳ ಬಗ್ಗೆ ತಿಳಿಸಿಕೊಡುತ್ತಿದ್ದರು. ಗುರುಗಳಿಗೆ ಕೃಷಿಯಲ್ಲಿ ವಿಶೇಷ ಆಸಕ್ತಿ ಇತ್ತು ತಮ್ಮ ಹೆಚ್ಚಿನ ಸಮಯವನ್ನು ಅದರಲ್ಲಿಯೇ ಕಳೆಯುತ್ತಿದ್ದರು. ನಂತರ ಗುರುಗಳೊಂದಿಗೆ ಕಾಡಿನಲ್ಲಿ ತಿರುಗಾಟ ಚರ್ಚೆ ನಡೆಯುತ್ತಿತ್ತು. ಗುರುಗಳ ಅಣ್ಣ ಮೃದು ಹೃದಯಿ. ದಿ. ಮಂಜುನಾಥ ಭಟ್ಟರ ಮಕ್ಕಳಾದ ಕಾಂತಿ ಮತ್ತು ಪ್ರವೀಣ ಆಗ ಇನ್ನೂ ಚಿಕ್ಕವರು ಅವರೊಂದಿಗೆ ಕಳೆದ ಸಮಯವಂತೂ ಮರೆಯಲಾಗುವುದಿಲ್ಲ. ಅವರಿಗೆ ಕಥೆಗಳೆಂದರೆ ತುಂಬಾ ಆಸಕ್ತಿ. ಅವರಿಗೆ ಗೊತ್ತಿಲ್ಲದ ಕಥೆಗಳನ್ನು ಹೇಳುವುದೇ ಒಂದು ಸವಾಲಿನ ಕೆಲಸ\-ವಾಗಿತ್ತು. ಮೂವರು ಕಾಡಿನ ನಡುವೆ ಹಾಡುತ್ತಾ ಕಥೆಗಳನ್ನು ಆನಂದಿಸುತ್ತಾ ತಿರುಗುತ್ತಿದ್ದೆವು. ಅತ್ತಿಗೆಯವರ ಪಾಕಶಾಲೆಯ ಔತಣದ ರುಚಿ ಇನ್ನೂ ಜಿಹ್ವೆಯಲ್ಲಿ ಹಾಗೆಯೇ ಇದೆ. ಅಲ್ಲಿ ಕೆಲಕಾಲವಾದರೂ ಗುರುಗಳ ಸೇವೆ ಮಾಡುವ ಸದವಕಾಶ ಸಿಕ್ಕಿದ್ದು ನನ್ನ ಅದೃಷ್ಟ.

ಮಹರ್ಷಿ ರಮಣರು ಹೇಳುವಂತೆ ‘ಗುರುಕೃಪೆ ಸಾಗರದಷ್ಟು, ಕೇವಲ ಒಂದು ಲೋಟದೊಂದಿಗೆ ಬಂದವರಿಗೆ ಸಿಗುವುದು ಒಂದು ಲೋಟದಷ್ಟು ಮಾತ್ರ. ಅದಕ್ಕಾಗಿ ಸಾಗರದ ಕೃಪಣತೆಯನ್ನು ದೂರಿ ಪ್ರಯೋಜನವಿಲ್ಲ. ಪಾತ್ರೆಯು ದೊಡ್ಡದಾದಷ್ಟೂ ಹೆಚ್ಚು ಹೆಚ್ಚು ತುಂಬಿಕೊಳ್ಳಬಹುದು. ಆದುದರಿಂದ ಅದು ಒಯ್ಯುವವನನ್ನು ಅವಲಂಬಿ\-ಸಿದೆ”.

ಮಾತೃ ಹೃದಯಿ ಗುರುಗಳು ನನ್ನ ಉನ್ನತಿಗಾಗಿ ಎಲ್ಲ ರೀತಿಯ ಸಹಕಾರ ನೀಡಲು ಸಿದ್ದರಿದ್ದರು. ಎಲ್ಲ ಶಿಷ್ಯರ ಬಗೆಗೂ ಅವರ ಜಾಗ್ರತೆ ಹೀಗೆಯೇ. ನಾನೇ ಅವರಿಂದ ಹೆಚ್ಚು ಹೆಚ್ಚು ಪಡೆಯಲಾಗಲಿಲ್ಲವಲ್ಲ ಎನ್ನುವ ಕೊರಗು ಇನ್ನೂ ಇದೆ. ಅವರಿಂದ ಹಲವಾರು ಜೀವನ ಮೌಲ್ಯಗಳನ್ನು ಕಲಿತಿದ್ದೇನೆ. ಅವುಗಳನ್ನು ಅನುಸರಿಸುತ್ತಿದ್ದೇನೆ. ಸದಾ ಗುರುವಿನ ಬಳಿಯೇ ಇರುವ ಶಿಷ್ಯನಿಗಿಂತ ಅದೃಷ್ಟಶಾಲಿ ಜಗತ್ತಿನಲ್ಲಿ ಬೇರೊಬ್ಬನಿಲ್ಲ.

ತನ್ನ ಶಿಷ್ಯರುಗಳಿಗೆ ಬದುಕಿನಲ್ಲಿ ಹೊಸ ನೆಲೆಗಟ್ಟುಗಳನ್ನು ಶೋಧಿಸಲು ಅಪೂರ್ಣತೆ\-ಯಿಂದ ಪರಿಪೂರ್ಣತೆಯೆಡೆಗೆ ಸಾಗಲು ಹೊಸ ಹೊಸ ಚಿಂತನಗಳ ಅನ್ವೇಷಣೆಗೆ\break ಪ್ರೇರಣೆಯಾಗಿದ್ದಾರೆ. ವಿದ್ಯಾರ್ಥಿಗಳ ಬದುಕು ಸದಾ ಹರಿಯುವ ನೀರಿನಂತೆ ಚಲನಶೀಲ\-ವಾಗಿರಬೇಕು ಮತ್ತು ಅವರ ಬದುಕು ಹಸನಾಗಿರಬೇಕೆಂದು ಅನವರತವೂ ಶ್ರಮಿಸುತ್ತಿದ್ದಾರೆ. ತಮ್ಮ ಬೋಧಕ ವೃತ್ತಿಯಲ್ಲಿ ಸಾರ್ಥಕತೆಯನ್ನು ಕಂಡಿರುವ ಗುರುಗಳು ಸರಳ ಬದುಕು, ಉನ್ನತ ಆದರ್ಶ ಮತ್ತು ವಿಶ್ವಮಾನವ ಜೀವನ ಕೃಷ್ಟಿ ಹೊಂದಿದವರು. ಅವರ ಸಹಕಾರ ಮನೋಭಾವ, ಸ್ನೇಹಶೀಲತೆ, ವಿನಯವಂತಿಕೆ, ಕಾರ್ಯತತ್ಪರತೆ ಅನುಕರಣೀಯ.

ಮತ್ತೊಬ್ಬರ ವಿದ್ವತ್ತನ್ನು ಶ್ರೇಷ್ಠತೆಯನ್ನು, ಸದ್ಗುಣಗಳನ್ನು ಗುರುತಿಸುವಲ್ಲಿ ಅವರದು ಉದಾರ ದೃಷ್ಟಿ. ಸಮಾಜದಲ್ಲಿ ಕುಟುಂಬದಲ್ಲಿ, ವೃತ್ತಿ ಬದುಕಿನಲ್ಲಿ ಹಿರಿಯ, ಕಿರಿಯರೆಲ್ಲರ ಪ್ರೀತಿ, ಗೌರವ, ವಿಶ್ವಾಸಕ್ಕೆ ಪಾತ್ರರಾಗಿದ್ದಾರೆ. ಭಗವಂತ ಅವರಿಗೆ ದೀರ್ಘ ಆಯುಷ್ಯ, ಆರೋಗ್ಯ, ಸುಖ, ಶಾಂತಿ, ನೆಮ್ಮದಿಗಳನ್ನು ಕರುಣಿಸಲಿ. ಅವರ ಜ್ಞಾನ ಭಂಡಾರದಿಂದ ಇನ್ನೂ ಸಾವಿರಾರು ಕುಸುಮಗಳು ಅರಳಲಿ. ಶಿಷ್ಯ ಕೋಟಿಯ ಮೇಲೆ ಅವರ ಅನುಗ್ರಹ ಆಶೀರ್ವಾದ ಸದಾ ಇರಲಿ. ಅವರ ಮುಂದಿನ ಬದುಕು ಸದಾ ಸುಖ\-ಕರ\-ವಾಗಿರಲಿ. ಗುರುಗಳೆಂದರೆ ಮತ್ತೆ ನೆನಪಾಗುವುದು,
\begin{center}
ಹಳೆ ಬೇರು ಹೊಸ ಚಿಗುರು ಕೂಡಿರಲು ಮರಸೊಬಗು\\
ಹೊಸ ಯುಕ್ತಿ ಹಳೆ ತತ್ತ್ವದೊಡಗೂಡೆ ಧರ್ಮ~।\\
ಋಷಿವಾಕ್ಯದೊಡನೆ ವಿಜ್ಞಾನ ಕಲೆ ಮೇಳವಿಸೆ \\
ಅಸವು ಜನಜೀವನಕೆ   \enginline{-}   ಮಂಕುತಿಮ್ಮ~॥
\end{center}

\articleend
}
