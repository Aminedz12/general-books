\chapter{"ನಾ ಕಂಡ ನನ್ನ ಸಹವರ್ತಿ ಗಂಗಾಧರ"}

\begin{center}
\Authorline{ವಿಶ್ವನಾಥ ಅಗ್ನಿಹೋತ್ರಿ}
\addrule
\end{center}

ವೇಧ–ಶಾಸ್ತ್ರಗಳ ಅಧ್ಯಯನಕ್ಕಾಗಿಯೇ ಮಹಾರಾಜರಿಂದ ಸ್ಥಾಪಿಸಲ್ಪಟ್ಟ ಒಂದು ಪ್ರಾಚೀನ ವಿದ್ಯಾಸಂಸ್ಥೆ   "ಶ್ರಿಮನ್ಮಹಾರಾಜಾಸಂಸ್ಕೃತ" ಮಹಾಪಾಠಶಾಲೆಯಾಗಿದೆ.  ಇಂತಹ ಪಾಠಶಾಲೆ ಇಂದು ನಮ್ಮ ಬದುಕಿಗೆ ಕಾರಣೀಭೂತವಾಗಿದೆ. 

1975 ಜೂನ್ ತಿಂಗಳಿನಲ್ಲಿ ಮೈಸೂರಿಗೆ ನನ್ನ ಪಾದಾರ್ಪಣೆ. ಊರನ್ನು ಬಿಟ್ಟು ಎಲ್ಲಿಗೂ ಹೋಗದೆ ಇರುವ ನನಗೆ ಮೈಸೂರಿಗೆ ಬೆಳಿಗ್ಗೆ ಬಂದಿಳಿದಾಗಾ ಯಾವುದೋ ವಿದೇಶದಲ್ಲಿ ಇದ್ದಂತೆ ಭಾಸವಾಗಿತ್ತು. ಬಡತನದ ಅವಿಭಕ್ತ ಕುಟುಂಬದಲ್ಲಿ  ಅಶಾಂತಿಯ ಬೇಗುದಿಯಲ್ಲಿ ಜೀವಿಸಿ ಮನೆಯಲ್ಲಿ ಭವಿಷ್ಯವಿಲ್ಲದ ಸಂದರ್ಭದಲ್ಲಿ 'ಕೃಷ್ಣಮಂಜಭಟ್ಟರ' ಸಹಕಾರದಿಂದ ಬಂದ ನನ್ನಂತೆ  ಬಂದ ವ್ಯಕ್ತಿ ನನ್ನ ಜೊತೆಯಲ್ಲಿ ವಾಸವಾಗಿರುವ ವ್ಯಕ್ತಿ ಗಂಗಾಧರ.

ಘಟ್ಟದ ಕೆಳಗಿನ ಹೊನ್ನಾವರ ತಾಲೂಕಿನ ಕಡತೋಕಾ ಗ್ರಾಮದವ ನಾನು. ಘಟ್ಟದ ಮೇಲಿನ ಸಿದ್ದಾಪುರ ತಾಲೂಕಿನ 'ಅಣ್ಣೆರಿ' ಗ್ರಾಮದವ ಗಂಗಾಧರ. ನಮ್ಮಿಬ್ಬರ ಹೆಸರು ಬೇರೆ– ಬೇರೆಯಾಗಿದ್ದರೂ ಅದರ ಮೂಲ ಒಂದೇ.ಎರಡೂ ಹೆಸರು   ಅನ್ವಯಿಸುವುದು ಶಿವನಿಗೆ.ಹಾಗೆ ಬೇರೆ–ಬೇರೆ  ಊರಿನವರಾದರೂ ಒಟ್ಟಿಗೆ ಸೇರಿದ್ದು ಒಂದೇ ಕೊಠಡಿಯಲ್ಲಿಅದು ಸಂಸ್ಕೃತ ಪಾಠಶಾಲೆಯ  ನಂಬರ್ 27ನೆಯ ಕೊಠಡಿಯಲ್ಲಿ. ನಮ್ಮಿಬ್ಬರ ಜೊತೆಯಲ್ಲಿ ಗಂಗಾಧರನ ಕುಟುಂಬದವರೇ ಆದ ಹೇರಂಭ ಭಟ್ಟ ಮತ್ತು ಗಂಗಾಧರ ಅಣ್ಣ ಶ್ರೀಧರ ಭಟ್ಟರು. ನಾವು ನಾಲ್ವರೂ ಬದುಕನ್ನ ಹುಡುಕಿಕೊಂಡು ಮೈಸೂರಿಗೆ ಬಂದವರು. 

1975ಕ್ಕಿಂತ  ಮೊದಲು ಗಂಗಾಧರ ಮತ್ತು ನನಗೆ ಯಾವುದೇ ರೀತಿಯ ಪರಿಚಯವಿರಲಿಲ್ಲ . ನಂತರ ಒಂದು ದೃಷ್ಟಿಯಲ್ಲಿ ಪರಿಚಿತರೆಂಬುದು ಗೊತ್ತಾಯಿತು.ಯಾಕೆಂದರೆ  ಗಂಗಧರನಿಗೆ ಮತ್ತು ಅವರ ಅಣ್ಣ ಶ್ರೀಧರ ಭಟ್ಟರಿಗೆ  ನನ್ನ ಭಾವ ಪ್ರಾಥಮಿಕ ಶಾಲೆಯಲ್ಲಿ ಅಧ್ಯಾಪಕರೆಂಬುದು  ಏನೇ ಇರಲಿ ಇಲ್ಲಿ ನಾವು ಪರಿಚಿತರಾಗಿ ಅಣ್ಣತಮ್ಮಂದಿರಂತೆ, ಸ್ನೇಯಿತರಂತೆ ಒಂದೇ ಕೊಠಡಿಯಲ್ಲಿ ಸಹವರ್ತಿಗಳಾದೆವು. 

ಹೀಗೆ ನಮ್ಮ ವಿದ್ಯಾರ್ಥಿ ಜೀವನ 1975ರಿಂದ ಸಂಸ್ಕೃತ ಪಾಠಶಾಲೆಯಲ್ಲಿ ಪ್ರಾರಂಭವಾಗಿ ಐದು ವರ್ಷಗಳ ಪರ್ಯಂತ 27ನೇ ನಂಬರಿನ ರೂಮಿನಲ್ಲಿ ಆರಂಭವಾಯಿತು.ಮತ್ತು ಅಂತ್ಯಗೊಂಡಿತು. 

ಊರಿಂದ ಬರುವಾಗ ಕಾಲಿಗೆ ಚಪ್ಪಲಿಯಿಲ್ಲದೆ ಒಂದು ಹಳೆಯದಾದ ಪಂಚೆಯನ್ನು(ಲುಂಗಿ) ಸುತ್ತಿಕೊಂಡು ಕೈಯಲ್ಲಿ ಹಳೆಯದಾದ ಒಂದು ಕಬ್ಬಿಣದ ಪೆಟ್ಟಿಗೆಯನ್ನು ಹಿಡಿದು ಮೈಸೂರು ಸೇರಿದ್ದೆವು. ಹಾಸಲು– ಹೊದಿಯಲು ಸರಿಯಾದ ಬಟ್ಟೆಯಿಲ್ಲ,ಚಾಪೆಯ ಮೇಲೆಯೇ ನಮ್ಮ ಶಯನ, ಅದೇ ನಮ್ಮ ಸಂಪತ್ತಿಗೆ ತಂದಿರುವ ಹಣದಲ್ಲಿ ಲೆಕ್ಕಾಚಾರ ಮಾಡಿ ಅನ್ನ ಮಾಡಿಕೊಂಡು ,ಹೋಟೆಲಿನಿಂದ ನಾಲ್ಕು ಆಣೆಗೆ ತಿಳಿಸಾರು  ಅಥವಾ ಸಾಂಬಾರ್ ತಂದಿಕೊಂಡು ಊಟವನ್ನು ಮಾಟುತ್ತಿದ್ದೆವು. ಒಂದೆರಡು ದಿನಗಳ ನಂತರ ಬೆಳಿಗ್ಗೆ ತಿಂಡಿಯೂ ಇಲ್ಲವಾಯಿತು. ಕಾರಣ ತಂದಿರುವ ಹಣ ಖಾಲಿ. ಆಗ ಚಿಂತೆ ಉಂಟಾಗಿ ಏನು ಮಾಡುವುದೆಂದು , ಅಂದಿನಿಂದ ಪ್ರಾರಂಭ ಊಟಕ್ಕಾಗಿ ಯಾಚನೆ– ಮೈಸೂರಿನಲ್ಲಿರುವ ಹೋಟೇಲುಗಳಲ್ಲಿ ವಾರನ್ನಕ್ಕಾಗಿ.

ಅಂದಿನ ಕಾಲದಲ್ಲಿ ವೇದಾದ್ಯಯನಕ್ಕಾಗಿ ಬರುವ ವಿದ್ಯಾರ್ಥಿಗಳಿಗೆ ವಾರಾನ್ನವನ್ನು (ಊಟ) ಕೊಡುವ ಮಹಾ ಅನ್ನಧಾನದ ಕಾರ್ಯ ಮೈಸೂರಿನ ಹೋಟೇಲ್ ಮಾಲೀಕರಿಂದ ,ಮೈಸೂರು ನಾಗರೀಕರಿಂದ ನಡೆಯುತ್ತಿತ್ತು.ಅದರಂತೆ ನಾವೂ ಕಂಡ –ಕಂಡ  ಹೋಟೇಲುಗಳಲ್ಲಿ ವಾರನ್ನಕ್ಕಾಗಿ ಹೋಗಿ ಕೇಳಲು ಪ್ರಾರಂಭಿಸಿದೆವು. ಇಂದ್ರವಿಹಾರ, ಅಶೋಕಾ ಹೋಟೇಲ್ , ಕೃಷ್ಣ ಕೆಪೆ,ಇಂದ್ರಕೆಪೆ, ಹೀಗೆ ಅನೇಕ ಹೋಟೇಲುಗಳಲ್ಲಿ ಊಟವನ್ನು ಕೇಳೀ ಪಡೆದು ಕೊಂಡೆವು. ಆದರೆ ವಾರದ ಎಲ್ಲಾ ದಿನಗಳಲ್ಲಿ ಊಟ ಸಿಕ್ಕುತ್ತಿರಲಿಲ್ಲ. ಆದರೂ ಊಟ ಕೊಟ್ಟವರನ್ನು ನಾವು ಇಂದು ನೆನಪಿಸಿಕೊಳ್ಳಲೇಬೇಕು.ಊಟವಿಲ್ಲದ ದಿನ ನಮ್ಮ ಅಧ್ಯಾಪಕರೇ ನಮಗೆ ಕರೆದು ತಮ್ಮ ಮನೆಯಲ್ಲಿ ಊಟವನ್ನು ಕೊಡುತ್ತಿದ್ದರು. 

ಗಂಗಾಧರ, ಹೇರಂಭ,ಶ್ರೀಧರ ಮತ್ತು ನಾನು ಹೀಗೆ ನಾವು ನಾಲ್ವರು ಬೆಳಿಗ್ಗೆ ತಿಂಡಿಯನ್ನು ಅಥವಾ ಹಣವಿಲ್ಲದಿದ್ದಲ್ಲಿ ಕೇವಲ   ಹಾಲನ್ನೇ ತಂದುಕೊಂಡು ಒಟ್ಟಿಗೆ ಉಪಯೋಗಿಸುತ್ತಿದ್ದೆವು. ಅಂದು ನಾಲ್ಕು ಆಣೆಯನ್ನು ಕೊಟ್ಟರೆ "ಗುರುಪ್ರಸಾದ" ಎಂಬ ಹೋಟೇಲಲ್ಲಿ  ಒಳ್ಳೆಯ ಖೀರನ್ನ ಕೊಡುತ್ತಿದ್ದರು. ಅದನ್ನ ಹಣವಿದ್ದಾಗ ಹೋಗಿ ಕುಡಿದು ಮಧ್ಯಾಹ್ನದವರೆಗೆ ಹಸಿವನ್ನು ನೀಗಿಸಿಕೊಳ್ಳುತ್ತಿದ್ದೆವು. 

ನನ್ನ ಹಾಗೂ ಗಂಗಾಧರ ಬಾಂಧವ್ಯ ಕೇವಲ ಸಹಪಾಠಿಗಳಾಗಿಯಷ್ಟೇ ಅಲ್ಲನಮ್ಮ ವಯಕ್ತಿಕ ಜೀವನದ ಆಗು ಹೋಗುಗಳಿಗೂ  ಸಂಬಂಧಪಟ್ಟಿದ್ದವು. ನಮ್ಮಿಬ್ಬರ ಸಮಸ್ಯೆಗಳನ್ನು ಪರಸ್ಪರ ಹಂಚಿಕೊಳ್ಳುತ್ತಿದ್ದೆವು. ಕೆಲವೊಮ್ಮೆ ದಿನ ಬೆಳಿಗ್ಗೆ ತಿಂಡಿಗೆ ಯಾರಲ್ಲೂ ಹಣ ಇರುತ್ತಿರಲಿಲ್ಲ. ಅಂತಹ ಸಂದರ್ಭದಲ್ಲಿ ಗಂಗಾಧರನ ಅಣ್ಣ ಶ್ರೀಧರಭಟ್ಟರು ಬಹಳ ದು:ಖದಿಂದ ಹೇಳುತ್ತಿದ್ದರು–" ವಿಶ್ವ ಈ ದಿನ 'ತಣ್ಣೀರುಬಟ್ಟೆಯೇ ಗತಿ' ಯೆಂದು, ಹಸಿವನ್ನ ಮರೆಯಲು ಅದು–ಇದು ಊರಿನ ವಿಷಯವನ್ನೊ ಅಥವಾ ಇನ್ಯಾವುದೋ ವಿಷಯವನ್ನ ತೆಗೆದುಕೊಂಡು ಮಾತಾಡುತ್ತಿದ್ದೆವು. ಅಥವಾ ಮಲಗಿ ನಿದ್ರಿಸುತ್ತಿದ್ದರು. ಊಟಕ್ಕಾಗಿ ,ತಿಂಡಿಗಾಗಿ ಪರಿತಪಿಸಿದ ನನಗೆ ಶ್ರೀ ಡಿ ವಿ ಜಿಯವರ ಕಗ್ಗದ ಒಂದು ಪದ್ಯ ನೆನಪಾಗುವುದು. 

\begin{verse}
"ಹೊಟ್ಟೆರಾಯನ ನಿತ್ಯದಟ್ಟದಸವೊ ಬಾಳು । ದುಷ್ಟ ಧಯೂಳಿಗಕೆ ನೊಟ್ಟು ಮೈ ಬಾಳು ॥ \\
ಹಿಟ್ಟಿಗಗಲಿದ ಬಾಯಿ ಬಟ್ಟೆಗೊಡ್ಡಿದ ಕೈಯಿ । ಇಷ್ಟ ನಮ್ಮೆಲ್ಲ  ಕಥೆ – ಮಂಕುತಿಮ್ಮ  ॥ 
\end{verse}

ಆರಂಭದಲ್ಲಿ ನಮ್ಮ ದಿನ ನಿತ್ಯದ ಹೊಟ್ಟೆಯನ್ನ ತುಂಬಿಸುವುದರಲ್ಲೇ ಕಳೆಯುತ್ತಿತ್ತು. ಅದರದ್ದೆ ಚಿಂತೆಯಾಗಿತ್ತು.  ಅಧ್ಯಯನಕ್ಕಾಗಿ ಮನಸ್ಸು ಒಗ್ಗುತ್ತಿರಲಿಲ್ಲ. ಯಾವಾಗ ಹೊಟ್ಟಿಗೆ  ವ್ಯವಸ್ಥೆಯಾಯಿತೊ ಆಗ ಅಧ್ಯಯದಲ್ಲಿ ಮನಸ್ಸು  ಒಗ್ಗಿತು. 

ಮೈಸೂರಿಗೆ ಬರುವಾಗ ನನಗೆ ಮತ್ತು ಗಂಗಾಧರನಿಗೆ( ಭಹುಷ:) ಹತ್ತನೇಕ್ಲಾಸ್ ಆಗಿರಲಿಲ್ಲ. ಆಗ ಮೈಸೂರಿನಲ್ಲಿ ಇರುವ "ದಳವಾಯಿ" ಹೈಸ್ಕೂಲಿನ ಮೂಲಕ ಹತ್ತನೇ ತರಗತಿಯನ್ನು ಖಾಸಗಿಯಲ್ಲಿ ಕಟ್ಟಿ ಪಾಸುಮಾಡಿಕೊಂಡೆವು. 

ಗಂಗಾಧರ ತರ್ಕಶಾಸ್ತ್ರದ ವಿದ್ಯಾರ್ಥಿ, ನಾನು ಧರ್ಮಶಾಸ್ತ್ರದ ವಿದ್ಯಾರ್ಥಿ . ಶಾಸ್ತ್ರ ಬೇರೆಯಾದರೂ ನಾವಿಬ್ಬರೂ ಅನ್ಯೋನ್ಯವಾಗಿಯೇ ಇದ್ದೆವು. ಗಂಗಾಧರ ಯಾವಾಗಲೂ ನನ್ನನ್ನು "ವಿಶ್ವ" ಎಂಬುದಾಗಿಯೇ ಕರೆಯುತಿದ್ದ. ಈಗಲೂ ಸಹ "ವಿಶ್ವ" ಎಂದೇ ಕರೆಯುತ್ತಾನೆ. ನಮಗಿಬ್ಬರಿಗೂ ಊರು ಬಹುದೂರಾ, ಹೀಗಾಗಿ ಊರಿನವರ ಸಂಪರ್ಕ ಬಹುದೂರವೇ ಕಾರಣ ಅಂದು  ಸಂಪರ್ಕಮಾಧ್ಯಮವೆಂದರೆ ಪತ್ರಮಾಧ್ಯಮವೊಂದೇ. ಇಂತಹ ಸಂದರ್ಭದಲ್ಲಿ ನಮಗೆ ನಾವೇ ಪಾಲಕರು, ಸಂಬಂಧಿಗಳು, ಹಿತೈಷಿಗಳು, ಯಾರಿಗೆ ಏನಾದರೂ ನಮ್ಮನ್ನ  ನಾವೇ ನೋಡಿಕೊಳ್ಳಬೇಕಿತ್ತು.  ಪಾಠಶಾಲೆಯ ಇನ್ನಿತರ  ರೂಮಿನಲ್ಲಿ  ಇರುವ ನಮ್ಮವರೇ ಆದ ಇನ್ನೂ ಅನೇಕ ಸ್ನೇಹಿತರು, ನಮ್ಮ ಹಿತೈಷಿಗಳಾಗಿರುತ್ತಿದ್ದರು.  ಊರಿಗೆ ಹಣಬೇಕೆಂದು ಪತ್ರ ಬರೆದರೆ ಆ ಪತ್ರ ಹೋಗಿ ತಲುಪಲು 15 ದಿನಗಳು . ಅವರು ಹಣದ ಸಿದ್ದತೆಯನ್ನು ಮಾಡಿಕೊಂಡು  ಕಳುಹಿಸಿದ ಮೇಲೆ ಅದು ಬಂದಿ ಸೇರಲು ಇನ್ನು ಹದಿನೈದು    ದಿನಗಳು. ಹೀಗೆ ಒಂದು ತಿಂಗಳು ಬೇಕಾಗಿತ್ತು.  ಅಷ್ಟರಲ್ಲಿ ನಾವು ಅವರಿವರಲ್ಲಿ ಸಾಲ ಪಡೆದು ಬದುಕುತ್ತಿದ್ದೆವು. 

ಹಣ ಬಂದ ತಕ್ಷಣ ಮಾಡಿದ ಸಾಲವನ್ನು ತೀರಿಸುವುದೇ ಕಷ್ಟ. ಮತ್ತೆ ಬರಿಗೈ. ಇದು ನಮ್ಮಿಬ್ಬರ ಬದುಕಾಗಿತ್ತು. ಏನಾದರೂ ಮಿಕ್ಕಿದ್ದರೆ ಅದರಲ್ಲಿಯೇ ಒಂದು ಸಿನೇಮಾ ನೋಡಿ ಬರುತ್ತಿದ್ದೆವು. 

ಇನ್ನೂ ಗಂಗಾಧರ ಒಬ್ಬ ನಿಷ್ಕಾಮಾ ಮನೋಭಾವದ ವ್ಯಕ್ತಿ. ತನ್ನಲ್ಲಿ ಇರುವ ಅಫಾರವಾಧ ಜ್ಙಾನವನ್ನು ಸಧಾಕಾಲ ನಿರಪೇಕ್ಷಿತನಾಗಿ  ಇನ್ನೊಬ್ಬರಿಗೆ ಧಾನ ಮಾಡುವ ವ್ಯಕ್ತಿ .

"ಶಾಸ್ತ್ರಾನ್ಯದೀತ್ಯಾಪಿ ಭವಂತಿ ಮೂರ್ಖಾಃ ಯಸ್ತು ಕ್ರಿಯಾವಾನ್ ಪುರುಷಃ ಸ ವಿದ್ವಾನ್ " ಈ ವ್ಯಕ್ತಿಗೆ ಅನ್ವಿತವಾದ ವ್ಯಕ್ತಿ ಗಂಗಾಧರ, ಯಾರು ಯಾವುದೇ ಸಮಯದಲ್ಲಾಗಲೀ ಏನನ್ನೇ ಕೇಳಿ ಬಂದರೂ  ಅದನ್ನ ಭೋಧಿಸುವ ತಿಳಿಸುವ ವ್ಯಕ್ತಿತ್ವ ಗಂಗಾಧರನದು.  ಇಂದಿಗೂ ಅವನ ಮನೆಗೆ ಹೋದರೆ ಯಾರಾದರೂ  ಅವನಲ್ಲಿ ಜ್ಙಾನಭೀಕ್ಷುಗಳಾಗಿ ಬಂದಿರುತ್ತಾರೆ. ಜೊತೆಯಲ್ಲಿ ಗಂಗಾಧರ ಸದಾ ದಾಸೋಹ ಮಾಡುವ ವ್ಯಕ್ತಿ ಊರಿಂದಾಗಲಿ ಎಲ್ಲಿಂದಾಗಲಿ  ಮನೆಗೆ ಬಂದವರಿಗೆ ಊಟವನ್ನು ಹಾಕಿ ಇರುವ ವ್ಯವಸ್ಥೆಯನ್ನು  ಮಾಡುವ ವ್ಯಕ್ತಿ. 

ನಾನು ಮತ್ತು ಗಂಗಾಧರ ಒಟ್ಟಿಗೆ ಒಂದೆರಡು ಬಾರಿ ಅಖಿಲಭಾರತ ಭಾಷಣಸ್ಪರ್ಧೆಗೆ ಹೋಗಿದ್ದೆವು. ಗಂಗಾಧರ ಅಲ್ಲಿ ಹೋದರೂ ಅಲ್ಲಿ ಪ್ರಥಮಸ್ಥಾನವನ್ನು ಪಡೆದು ಕರ್ನಾಟಕಕ್ಕೂ ಮೈಸೂರು ಸಂಸ್ಕೃತ ಕಾಲೇಜಿಗೂ ಹೆಸರನ್ನು ತಂದುಕೊಟ್ಟ ಒಬ್ಬ ವಾಗ್ಮಿ. ಅನೇಕ ಸ್ಪರ್ಧೆಗಳಲ್ಲಿ ಜಯಭೇರಿಯನ್ನು ಭಾರಿಸಿದ ಒಬ್ಬ ಸೋಲೊಲ್ಲದ ಸರದಾರನಾಗಿದ್ದ. ಗಂಗಾಧರ ಸ್ಪರ್ಧೆಯಲ್ಲಿದ್ದಾನೆಂದರೆ ಸಭಿಕರೇ ತೀರ್ಮಾನಿಸುತ್ತಿದ್ದರು ಗಂಗಾಧರನಿಗೆ ಪ್ರಥಮ ಸ್ಥಾನವೆಂದು. 

ಇನ್ನ ಗಂಗಾಧರ  ಒಬ್ಬ ಉತ್ತಮ ಕಲಾವಿಧನಾಗಿದ್ದ. ಏಕಪಾತ್ರಾಭಿನಯ, ನಾಟಕ, ಚರ್ಚಾಸ್ಪರ್ಧೆ, ಪ್ರಬಂಧ ಸ್ಪರ್ಧೆ ಹೀಗೆ    ಅನೇಕ ವಲಯಗಳಲ್ಲಿ ತನ್ನದೇ ಆದ ಛಾಪನ್ನ ಸ್ಥಾಪಿಸಿದ ಒಬ್ಬ ಕಲಾವಿಧನಾಗಿದ್ದ. ಸಂಸ್ಕೃತ ಪಾಠಶಾಲೆಯ ಶತಮಾನೋತ್ಸವ ಸಂಧರ್ಭದಲ್ಲಿ ""ಅಂಖಿ" ಎಂಬ ಐತಿಹಾಸಿಕ ನಾಟಕವನ್ನು ಮಾಡಿದ್ದರು. ಆ ನಾಟಕದಲ್ಲಿ "ಅಂಖಿ"ಯ ಪಾತ್ರವನ್ನ ಮಾಡಿದ ಗಂಗಾಧರನ ಅಂದಿನ ಅಭಿನಯ ಇಂದಿಗೂ ನನ್ನ ಕಣ್ಣ ಮುಂದೆ ನಲಿದಾಡುತ್ತಲಿದೆ. ಮನಸ್ಸಿನಲ್ಲಿ ಸ್ಥಿರವಾಗಿದೆ.

ಇನ್ನ ಗಂಗಾಧರ ಮನಸ್ಸಿಗೆ ಬಂಧ ವಿಚಾರವನ್ನು ಕಾರ್ಯರೂಪಕ್ಕೆ ತರುವ ವ್ಯಕ್ತಿಯಾಗಿದ್ದಾನೆ.

\begin{verse}
ಯಥಾ ಚಿತ್ತಂ  ತಥಾ ವಾಚಂ ಯಥಾ ವಾಚಸ್ತಥಾಕ್ರೀಯಾ  ।\\
ಚಿತ್ತೇ ವಾಚಿ ಕ್ರೀಯಾಯಾಂ  ಚ ಮಹತಾಮೇಕರೂಪತಾ ॥
\end{verse}

ಇಂತಹ ಒಬ್ಬ ಅಪರೂಪದ ವ್ಯಕ್ತಿ ಗಂಗಾಧರ.        

ಅತ್ಯಂತ ಆತ್ಮೀಯತೆಯ ಕಾರಣ ಗಂಗಾಧರನನ್ನು  ಏಕವಚನದಲ್ಲಿ ಸಂಭೋದಿಸುತ್ತಿದ್ದೆನೆ. ಇಂತಹ ಆತ್ಮೀಯ ಗೆಳೆಯ, ಇಂದು ತನ್ನ ವೃತ್ತಿ ಜೀವನದಿಂದ ನಿವೃತ್ತಿಯನ್ನು ಹೊಂದುತ್ತಿದ್ದಾನೆ. ಅವನ ನಿವೃತ್ತಿಯ ಜೀವನ ಸದಾಸುಖ, ಸಂತೋಷದಿಂದ ಕೂಡಿರಲೆಂದು, ಆಯುರಾರೋಗ್ಯವನ್ನ ದೇವರು ದಯಪಾಲಿಸಲೆಂದು ಆ ಭಗವಂತನಲ್ಲಿ ಪ್ರಾರ್ಥಿಸುತ್ತೇನೆ. 

\articleend									
