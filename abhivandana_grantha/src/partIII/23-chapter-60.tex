\chapter{ವಿದ್ವಾಂಸರಲ್ಲೊಬ್ಬ ಅಪರಂಜಿ}

\begin{center}
\Authorline{ಡಾ~॥ ಎನ್. ಆರ್. ಮುರಳೀಧರ}
\smallskip

ಸಹಾಯಕ ಪ್ರಾಧ್ಯಾಪಕರು\\
ಅಲಂಕಾರ ಶಾಸ್ತ್ರ ವಿಭಾಗ\\
ಶ್ರೀಮನ್ಮಹಾರಾಜ ಸಂಸ್ಕೃತ ಮಹಾಪಾಠಶಾಲೆ,\\ 
ಮೈಸೂರು
\addrule
\end{center}

ವಿಶಾಲ ವಸುಮತಿಯಲ್ಲಿ ಸುಮತಿಯಿಂದ ಸುವಿಖ್ಯಾತರಾದ ಸುಧೀಮಣಿಗಳು ವಿರಳಾತಿವಿರಳ.  ಅಂತಹವರಲ್ಲಿ ಅಗ್ರಗಣ್ಯರು ಅಗ್ಗೇರಿ ಗಂಗಾಧರ ವಿ. ಭಟ್ಟರು.

ನನಗಿಂತ ನಾಲ್ಕು ಐದು ವರುಷ ಹಿರಿಯರಾದ ಗಂಗಾಧರ ಭಟ್ಟರನ್ನು ಕಂಡ, ಮನಗಂಡ ಮತ್ತಾವ ವಿದ್ವಾಂಸನು ಇತರರನ್ನು ಇಷ್ಟಪಡುವ ಗೋಜಿಗೆ ಹೋಗುವುದು ಕಡಿಮೆ.  ಅಂತಹ ಆಕರ್ಷಕ ವ್ಯಕ್ತಿತ್ವ ಭಟ್ಟರದು.

ಅದ್ಭುತ ಆಕರ್ಷಕ ಕಣ್ಣುಗಳ ಕಾಂತಿ, ಬಿಳಿಯ ಮೈಬಣ್ಣದ ತೊಗಟೆ ಇವರು ಶ್ರೇಷ್ಠರೆಂದು  ಮೇಲ್ನೋಟಕ್ಕೆ ಹೊಡೆದು ಹೇಳುವುದು ಜಾಗಟೆ. ವ್ಯವಸ್ಥೆಯ ಮತ್ತೊಂದು ಹೆಸರು ಭಟ್ಟರು.

ಇವರನ್ನು ಮೊಟ್ಟಮೊದಲಿಗೆ ನಾನು ನನ್ನ ತೀರ್ಥರೂಪುಗಳಾದ ವಿದ್ವಾನ್ ಸೋ. ರಾಮಸ್ವಾಮಿ ಐಯ್ಯಂಗಾರ್ಯರ ಶ್ರೀರಾಮಮಿಶ್ರ ಸಂಸ್ಕೃತ ಪಾಠಶಾಲೆಯಲ್ಲಿ ಕಂಡದ್ದು.  ಇವರು ನನ್ನ ತಂದೆಯವರ ವ್ಯಾಕರಣದ ಕೆಲವು ಅವಧಿಗಳಿಗೆ ಬಂದು ಕುಳಿತು ಏಕಾಗ್ರತೆಯಿಂದ ಅಭ್ಯಾಸಮಾಡುವ ವಿಧಾನ, ಇವರಿಗೆ ವಿಷಯಗ್ರಹಿಕೆಯೇ ಪ್ರಧಾನ, ಅದರಲ್ಲಿ ಆದರೂ ನಿಧಾನ. ಮನದಟ್ಟಾಗುವ ತನಕ ಬಿಡದ ಸಾತ್ವಿಕ ಹಠ ಆಳವಾಗಿ ಒಂದು ವಿಷಯವನ್ನು ತಿಳಿಯುವ ನಿರಂತರ ಶ್ರಮ, ಬ್ರಹ್ಮಚರ್ಯಾಶ್ರಮ ಆಗಲೇ ಅಂಕುರವಾಗಿದ್ದವು ಈ ಎಲ್ಲಾ ಆದರ್ಶಗುಣಗಳು.

ಬೆರಳು ತೋರಿಸಿದರೆ ಹಸ್ತಂಗತವಾಗುವ ಗುಣ

ಬಾಲ್ಯದಲ್ಲಿ ಅಧ್ಯಾಪಕರು ಪಾಠ್ಯವಿಷಯಗಳೆಲ್ಲವನ್ನೂ ಮಾಡಿದರೂ ಅರ್ಥವಾಗದಿರುವುದು ವಿದ್ಯರ್ಥಿಗಳ ಸಹಜ ಸ್ಥಿತಿ.  ಆದರೆ ಭಟ್ಟರು ಬಾಲ್ಯದ ಪಾಠಗಳನ್ನು ಸರಿಯಾಗಿ ಗ್ರಹಿಸಿ ಅದನ್ನು ಅರಗಿಸಿ ಬೆಳೆಸಿ ಹೆಮ್ಮರವನ್ನಾಗಿ ಮಾಡಿ ಹಣ್ಣುಗಳನ್ನು ಪಡೆದು ತಾವೂ ಸವಿದು, ಬೇರೆಯವರಿಗೂ ಉಣ್ಣಿಸುವ ಕಾರ್ಯದಲ್ಲಿ ಸಫಲರಾಗುತ್ತಿದ್ದರು.

ದಾಖಲೆಯ ವಿದ್ಯಾರ್ಹತೆಯಿಂದ ದೂರ :– ಈ ಪ್ರಪಂಚದಲ್ಲಿ ಹಲವರು ವಿದ್ಯಾರ್ಹತೆಯ ದಾಖಲೆಗಳ ಸಂಗ್ರಹದಲ್ಲಿ ನಿರತರು.  ದಾಖಲೆಯಲ್ಲಿ ಅರ್ಹತೆ ಇರುವವರು ಬಹಳ.  ಆದರೆ ಭಟ್ಟರು ಅರ್ಹತೆಯಿಂದ ದಾಖಲೆ ಬರೆದವರು.  ಆದ್ದರಿಂದ ಅವರು ಬಿ.ಎಡ್ ಕೋರ್ಸಿಗೆ ಪ್ರವೇಶ ಪಡೆದರೂ ಜ್ಞಾನಪಿಪಾಸುಗಳಾದ ಅವರಿಗೆ ಜ್ಞಾನದ ದಾಹತೀರುವ ಸ್ವಚ್ಛ ವಿದ್ಯಾಜಲ ಸಿಗಲಿಲ್ಲವೆಂಬ ಕಾರಣದಿಂದ ಅಲ್ಲಿಂದ ನಿವೃತ್ತರಾದರು.  ತಾವೇ ತಮಗೆ ಬೇಕಾದ ಜ್ಞಾನಗಳಿಕೆಯಲ್ಲಿ ಏಕಲವ್ಯನಂತೆ ಪ್ರವೃತ್ತರಾದರು.  ಅದರಲ್ಲಿ ಸಫಲರಾದರೂ ಕೂಡ.

ಒಂದು ಜನ್ಮದಲ್ಲಿ ಒಂದೇ ಶಾಸ್ತ್ರ ಇದಕ್ಕೆ ಅಪವಾದ :– ಒಂದು ಜನ್ಮದಲ್ಲಿ ಒಂದು ಶಾಸ್ತ್ರದಲ್ಲಿ ಪ್ರಾವೀಣ್ಯವನ್ನು ಪಡೆಯುವುದೇ ಕಷ್ಟವಾಗಿರುವ ಈ ಕಾಲಘಟ್ಟದಲ್ಲಿ ಅನೇಕ ವಿದ್ಯಾಶಾಖೆಗಳಲ್ಲಿ ಪ್ರಾವೀಣ್ಯಪಡೆದ ಅಪರೂಪದ ಪಂಡಿತ ಈ ಭಟ್ಟರು ಖಂಡಿತ.  ಇವರ ಅಧಿಕೃತ ಅಧೀತ ಶಾಸ್ತ್ರ ನವೀನ ನ್ಯಾಯಶಾಸ್ತ್ರ.  ಇವರಿಗೆ ವಿಶೇಷ ಆಸಕ್ತಿಯಿರುವ ಶಾಖೆಗಳು ಆಯುರ್ವೇದ, ವೆದಾಂತಾದಿದರ್ಶನಗಳು.  ಇವರು ಆಯುರ್ವೇದಶಾಸ್ತ್ರವನ್ನು ಸ್ವತಃ ಅಧ್ಯಯನ ಮಾಡಿ ಸಂಶೋಧನೆಗೈದು ಆಯುರ್ವೇದದ ಅನೇಕ ಗ್ರಂಥಗಳನ್ನು ಆಯುರ್ವೇದ ವಿದ್ಯಾರ್ಥಿಗಳಿಗೆ ಪಾಠಮಾಡಿದ್ದಾರೆ.  ಅನರ್ಥ ವ್ಯರ್ಥವೆಂದು ಈಗಿನ ಬುದ್ಧಿಜೀವಿಗಳು ಭ್ರಮಿಸುವ ಬಾಧಿಸುವ ಭಾರತದ ಹೆಮ್ಮೆಯ ಚಾಣಕ್ಯನ ಅರ್ಥಶಾಸ್ತ್ರವನ್ನು ಅರ್ಥಪೂರ್ಣವಾಗಿ ಅಧ್ಯಯನ ಮಾಡಿ ತುಲನಾತ್ಮಕವಾಗಿ ಅನೇಕ ಕಡೆ ಪ್ರಬಂಧ ಮಂಡನೆಯನ್ನು ಮಾಡಿ ಸೈ ಎನಿಸಿಕೊಂಡಿದ್ದಾರೆ.

\section*{ವಾಮನಮೂರ್ತಿ ಆದ ತ್ರಿಭುವನ ಕೀರ್ತಿ}  

ಇವರ ವಿವಿಧ ಕ್ಷೇತ್ರಗಳಲ್ಲಿನ  ಪಾಂಡಿತ್ಯಕ್ಕೆ ಪ್ರತಿಪಾದನ ಶೈಲಿಗೆ ಮರುಳಾಗದವರಿಲ್ಲ.  ಸಡ್ಡು ಹೊಡೆದವರಿಲ್ಲ.  ಇವರ ಮಾತಿನ ಚಮತ್ಕಾರ, ವಿಷಯ ನಿರೂಪಣೆಯಲ್ಲಿನ ಸ್ಪಷ್ಟತೆ, ಗಂಟೆಗಟ್ಟಲೆ ಸಮಯ ನೀಡಿದರೆ ಗಂಟಲು ಕೆಡದೆ ಮಾತನಾಡಲು ಸಮರ್ಥವಾದ ದ್ವನಿಪೆಟ್ಟಿಗೆ, ಮಾತಿನಲ್ಲಿ ಭವನಕಟ್ಟಲು ಬಳಸುವ ಮುನ್ನ ಇಟ್ಟಿಗೆ, ಅಪಾಯವಿಲ್ಲದ ಮಾತಿನ ಪೀಠಿಕೆಯ ಅಡಿಪಾಯ, ಎಲ್ಲರಿಗೂ ಆದರ್ಶಪ್ರಾಯ.  ಆಕಾರದಲ್ಲಿ ವಾಮನಮೂರ್ತಿ ಆದರೆ ಪಾಂಡಿತ್ಯ ವ್ಯಕ್ತಿತ್ವಗಳಿಂದ ತಿಭುವನ ಕೀರ್ತಿ.

ಖಂಡಿತವಾದಿ ಲೋಕದ ಅವಿರೋಧಿ :–  ಭಟ್ಟರು ತಮ್ಮ ಒಲವಿನ ನಿಲುವನ್ನು ಸಭೆಗಳಲ್ಲಿ ಅಥವಾ ಇನ್ನೆಲ್ಲೇ ಆಗಲಿ ತಿಳಿಸುವಲ್ಲಿ ಹೊಂದಾಣಿಕೆ ಮಾಡಿಕೊಳ್ಳುವವರಲ್ಲ.  ಆದರೆ ತಿಳಿಸುವ ವಿಧಾನ ಎಲ್ಲರ ಚಿತ್ತ ತಮ್ಮತ್ತ ಸೆಳೆಯುತ್ತಾ ಶ್ರೋತೃಗಳಿಗೆ ಅರಿವಾಗದೆ ತಮ್ಮ ನಿಲುವನ್ನು ತಿಳಿಸುವ ಶಕ್ತಿ ಹಾಗೂ ಯುಕ್ತಿ ಅನಿತರ ಸಾಧಾರಣ.  ಆದ್ದರಿಂದಲೇ ಅವರು ಲೋಕದ ಅವಿರೋಧಿ.

\section*{ನಾಯಕತ್ವಗುಣ} 

ಯಾವುದೇ ಕಾರ್ಯವನ್ನು ವಹಿಸಿದರೂ ಇಲ್ಲವೆನ್ನದೇ ಎಲ್ಲವನ್ನು ಬಹಳ ಅಚ್ಚುಕಾಟ್ಟಾಗಿ ಮಾಡುವ ಹಾಗೂ ಮಾಡಿಸುವುದು ಇವರ ವಿಶೇಷಗುಣ.  ಇವರಿದ್ದೆಡೆ ಗೊಂದಲವಿಲ್ಲ.  ವ್ಯವಸ್ಥೆಯೇ ಎಲ್ಲ.  ತಕ್ಷಣಕ್ಕೆ ಬರಬಹುದಾದ ಸಮಸ್ಯೆಯನ್ನು ಬಗೆಹರಿಸುವ ಸಮಯಪ್ರಜ್ಞೆ ಅನುಪಮ.  ಎಂತಹ ಅಧಿಕ ಅರಿಗಳಾದ ಅಧಿಕಾರಿಗಳನ್ನು ತನ್ನ ವ್ಯಕ್ತಿತ್ವದಿಂದ ಸೆಳೆದು ಅಂಕುಶವಿಲ್ಲದೇ ಅಕಾರ್ಯಗಳಿಗೆ ಅಂಕುಶ ಹಾಕುವ ಪರಾಂಕುಶ ಆಚಾರ್ಯರು ನಮ್ಮ ಭಟ್ಟರು.

ಇಂತಹ ಗುಣವನ್ನು ನಮ್ಮ ಸಂಸ್ಕೃತಕಾಲೇಜಿನಲ್ಲಿ ನಡೆದ ಅನೇಕ ರಾಜ್ಯಮಟ್ಟದ ಸ್ಪರ್ಧೆ, ವಿದ್ವದ್ಗೋಷ್ಠೀ, ಮುಂತಾದವುಗಳ ಆಯೋಜನೆಗಳಲ್ಲಿ ಕಾಣಬಹುದು.  ಇದರೆಲ್ಲದರ ಸೂತ್ರಧಾರಿ ಭಟ್ಟರು, ಇತರರು  ಪಾತ್ರಧಾರಿಗಳು ಎಂಬುದು ನಿರ್ಮತ್ಸರಿಗಳಾದ ಪ್ರತಿಯೊಬ್ಬರಿಗೂ ತಿಳಿದ ವಿಷಯವೇ ಸರಿ.

ಭಟ್ಟರು ಕೃಷ್ಣನಂತೆ, ಚಾಣಕ್ಯನಂತೆ :–  ನಮ್ಮಭಟ್ಟರು ಯಾವುದೇ ವೇದಿಕೆಯ ಮುಂಭಾಗದಲ್ಲಿ ಬಾರದಿದ್ದರೂ ಕೂರದಿದ್ದರೂ ಎಲ್ಲದರ ಶ್ರೇಷ್ಠ ದೃಷ್ಟಿ  \eng{(super vision)} ಇವರದಾಗಿರುತ್ತಿತ್ತು. ಚಂದ್ರಗುಪ್ತನ ಕಾರ್ಯತಂತ್ರ, ಅದಕ್ಕೆ ಚಾಣಕ್ಯನ ಮೂಲಮಂತ್ರವಿದ್ದಹಾಗೆ.  ಅರ್ಜುನ ವಿಜಯಕ್ಕೆ ಹಿಂದಿರುವ ಕೃಷ್ಣನ ಅಭಯವಿದ್ದಹಾಗೆ. 

\section*{ಅಧ್ಯಕ್ಷರಾಗದಿದ್ದರೂ ಇವರದೇ ಅಧ್ಯಕ್ಷತೆ} 

ಪಠ್ಯಪುಸ್ತಕರಚನಾಸಮಿತಿ ಅಥವಾ ಇನ್ನಾವುದೇ ಸಲಹಾ ಸಮಿತಿಯಲ್ಲಾಗಲಿ ಅದಕ್ಷರು ಅಧ್ಯಕ್ಷರಾಗಿ ಆಯ್ಕೆಯಾಗಬಹುದು.  ಆದರೆ ಅವರ ಹಿಂದೆ ಭಟ್ಟರು ಇದ್ದರೆ ಅವರು ಅದಕ್ಷರಾಗದೇ ಅಧ್ಯಕ್ಷರೇ ಆಗಿಬಿಡುತ್ತಾರೆ.  ಸ್ಥಾನ ಅಧ್ಯಕ್ಷರದು ಅಲ್ಲಿ ಸರ್ವಕಾರ್ಯನಿರ್ವಾಹಕರು ಭಟ್ಟರಾಗಿರುತ್ತಾರೆ.

\section*{ಗುರುಭಕ್ತಿ}  

\dev{‘लब्धविद्यो गुरुं द्वेष्टि’}   ಎಂಬುದು ಈಗಿನ ವಿದ್ಯಾರ್ಥಿಗಳ ಸಹಜನಡಿಗೆ.  ಆದರೆ  \dev{‘गुरुं प्रकाशयेत् धीमान्’} ಎಂಬುದು ಭಟ್ಟರ ಸ್ವಭಾವ.  ಅಧ್ಯಾಪನ ಮಾಡಿದ ಎಲ್ಲಾ ಅಧ್ಯಾಪಕರ ಅಚ್ಚುಮೆಚ್ಚಿನ ತಮ್ಮ ತಮ್ಮ ಮಕ್ಕಳಿಗಿಂತ ಆತ್ಮೀಯರಾದ ಶಿಷ್ಯರು ನಮ್ಮ ಭಟ್ಟರು.

ತಮ್ಮ ಅತ್ಯಂತ ಆಂತರಿಕ ವ್ಯವಹಾರಗಳಿಗೂ ತಮ್ಮ ಮಕ್ಕಳನ್ನು ಬಿಟ್ಟು ಭಟ್ಟರಿಗೆ ಬಿಟ್ಟುಕೊಟ್ಟು ನೆಮ್ಮದಿಯಿಂದ ಇದ್ದವರು ಹಲವಾರು ಗುರುಗಳು. ನಂಬಿಕೆಗೆ ಮತ್ತೊಂದು ಹೆಸರು ಈ ಭಟ್ಟರು.

\section*{ಶಿಷ್ಯವಾತ್ಸಲ್ಯ} 

ತಮ್ಮ ಮಕ್ಕಳಲ್ಲೇ ತಾರತಮ್ಯಮಾಡಿ ಪಾಠಪ್ರವಚನಗಳಲ್ಲಿ ವ್ಯತ್ಯಾಸಮಾಡುತ್ತಾ ಕಾಲವ್ಯಯ ಮಾಡಿದ ಪ್ರಾಚೀನ ಪಂಡಿತರ ಬಗ್ಗೆ ನಾವು ಕೇಳಿದ್ದೇವೆ.  ಆದರೆ ಸರ್ವರಿಗೂ ಸಮರ್ಥವಾಗಿ ಅರ್ಥವಾಗುವಂತೆ ನಿರ್ವಂಚನೆಯಿಂದ ಪಾಠಮಾಡಿ ಶಾಸ್ತ್ರಪಂಡಿತರನ್ನಾಗಿ ಮಾಡಿದ ವೃತ್ತಿ, ಪ್ರವೃತ್ತಿ ಹಾಗೂ ಪ್ರಕೃತಿ ನಮ್ಮ ಭಟ್ಟರದು.  ಇದಕ್ಕೆ ಸಾಕ್ಷಿ ಈ ಕಾರ್ಯಕ್ರಮ ಹಾಗೂ ಈ ಅಭಿವಂದನಗ್ರಂಥ.

\section*{ಪರೋದ್ಧಾರದಿಂದ ಆತ್ಮೋದ್ಧಾರ} 

ಸಮಾಜದಲ್ಲಿ ತಾನು ಮೇಲೆ ಬಂದ ಮೇಲೆ ಇತರರನ್ನು ತುಳಿಯುವ, ತಡೆಯುವ ಪ್ರವೃತ್ತಿ ಜನರಲ್ಲಿ ಸಾಮಾನ್ಯವಾಗಿ ಕಂಡು ಬರುತ್ತದೆ.  ಆದರೆ ಭಟ್ಟರು ತಾವು ಮೈಸೂರಿಗೆ ಏಕಾಂಗಿಯಾಗಿ ಬಂದು ನೆಲ, ಹಾಗೂ ನೆಲೆಯನ್ನು ಕಂಡು ಬಳಿಕ ತನ್ನ ಊರಿನ  ಸಂಬಂಧಿಕರಲ್ಲಿ ಅನೇಕರನ್ನು ಇಲ್ಲಿಗೆ ಕರೆತಂದು ಪ್ರೋತ್ಸಾಹಿಸಿ ಮಾರ್ಗದರ್ಶನ, ಊಟ ವಸತಿಗಳನ್ನು ಕಲ್ಪಿಸಿ ಪಾಠಪ್ರವಚನಗಳನ್ನು ಮಾಡಿ ದೊಡ್ಡ ವಿದ್ವಾಂಸರನ್ನಾಗಿ ಮಾಡಿದ್ದಾರೆ. \dev{‘स पितरस्तासां केवलं जन्महेतवः’} ಎಂಬ ನುಡಿಗೆ ದೃಷ್ಟಾಂತವಾಗಿದ್ದಾರೆ.  ಆದರೆ ಇದನ್ನು ಅವರ ಬಾಯಿಂದ ಕೇಳಲಾಗದ ಸ್ಥಿತಿ.  ಏಕೆಂದರೆ ಇವರು \dev{‘त्यागे श्लाघाविपर्ययः’}  ಎಂಬುದಕ್ಕೆ ನಿದರ್ಶನ. ಆದ್ದರಿಂದ ಪ್ರದರ್ಶನಕ್ಕಿಂತ ದರ್ಶನ ಮುಖ್ಯ. 

ಇವರು ಜಂಭವಂತರಲ್ಲ ಜಾಂಬವಂತರು :–  ಸಾಮಾನ್ಯವಾಗಿ ಸಮಾನಕಾರ್ಯಕ್ಷೇತ್ರದಲ್ಲಿ ಅಕಾರಣವಾಗಿಯೇ ಅಸೂಯೆ ಇರುವುದು ಹಾಗೂ ಬರುವುದು.  ಆದರೆ ಗುಣೈಕ ಪಕ್ಷಪಾತಿಗಳಾದ ಇವರು ಯಾರಲ್ಲಿ ಯಾವ ಸಾಮರ್ಥ್ಯವಿರುತ್ತದೋ ಅದನ್ನು ಸೂಕ್ತ ಸಂದರ್ಭದಲ್ಲಿ ಹೊರಗೆಡಹುವ  ಸೂಕ್ತ ಸನ್ನಿವೇಶವನ್ನು ನಿರ್ಮಿಸಿ ಪ್ರೋತ್ಸಾಹಿಸುತ್ತಿದ್ದರು. 

ಇದಕ್ಕೆ ನಾನೇ ಸಾಕ್ಷಿ.  ವಿದ್ಯಾರ್ಥಿದೆಸೆಯಲ್ಲಿ ಯಾವ ವೇದಿಕೆಯಲ್ಲೂ ಕಾಣಿಸಿಕೊಳ್ಳದ, ಎದ್ದು ನಿಂತು ಏನನ್ನೂ ಮಾತನಾಡದ ನನ್ನನ್ನು ರಾಜ್ಯದ ವಿವಿಧ ಭಾಗಗಳಲ್ಲಿ ನಡೆದ ಶಿಬಿರಗಳಿಗೆ ತನ್ನೊಂದಿಗೆ ಸಂಪನ್ಮೂಲವ್ಯಕ್ತಿಯಾಗಿ ಕರೆದುಕೊಂಡು ಹೋಗಿ ವಿಷಯ ತಿಳಿಸುವ ಕಾರ್ಯದಲ್ಲಿ ತೊಡಗಿಸಿದರು.  ಅದರಿಂದ ಬಂದ ಕೀರ್ತಿಯನ್ನು ಕೇಳಿ ಸಂತಸಗೊಂಡರು.  ಆದ್ದರಿಂದ ನನ್ನ ವಿಷಯದಲ್ಲಿ ಭಟ್ಟರು ಜಂಭವಂತರಲ್ಲ,  ಹನುಮಂತನಿಗೆ  ಸಮುದ್ರತರಣದಲ್ಲಿ  ಸ್ಪೂರ್ತಿ ನೀಡಿದ  ಜಾಂಬವಂತನಂತೆ ಎಂದು.

\section*{ದೂರಶಿಕ್ಷಣ} 

ಈಗಿನ ಅಧ್ಯಾಪಕರಲ್ಲಿ ಹಲವರು ಶಿಕ್ಷಣದಿಂದ (ಅಧ್ಯಾಪನದಿಂದ) ದೂರವಾಗಿರುತ್ತಾರೆ.  ಪಾಠಮಾಡದೇ ರಾಜಕೀಯ ಮಾಡುತ್ತಿರುತ್ತಾರೆ.  ಆದರೆ ಭಟ್ಟರು ಶಾಲೆಯಲ್ಲೂ ಶಿಕ್ಷಣವನ್ನು ನೀಡಿ ವಿದೇಶದ ವಿದ್ಯಾರ್ಥಿಗಳಿಗೆ  \eng{online} ನಲ್ಲಿ ಸಂಸ್ಕೃತ, ಆಂಗ್ಲ, ಹಿಂದಿ ಮುಂತಾದ ಭಾಷೆಗಳಲ್ಲಿ ಶಿಕ್ಷಣವನ್ನು ನೀಡುತ್ತಿದ್ದಾರೆ. ಸಮೀಪಸ್ಥರಿಗೂ ಅಧ್ಯಾಪನಮಾಡಿ ದೂರದಲ್ಲಿರುವವರಿಗೂ ಜ್ಞಾನವನ್ನು ಹಂಚುತ್ತಿರುವ ಅಪರೂಪದ ಅಧ್ಯಾಪಕರು ಈ ಭಟ್ಟರು.

ಭಟ್ಟರಿಗೆ ನನ್ನೀಕವನ ಸಮರ್ಪಿತ –
\begin{verse}
ಭಟ್ಟರು ವಿದ್ವಾಂಸರಲ್ಲಿ ಅಪರಂಜಿ\\
ಅನ್ಯಥಾ ಮಾತನಾಡರು ದಾಕ್ಷಿಣ್ಯಕ್ಕೆ ಅಂಜಿ\\
ವಾದಕ್ಕೆ ಇಳಿದರೆ ಕುಡಿಸುವರು ಗಂಜಿ\\
ಇವರೆದುರಿಗೆ ಇತರರು ಕೇವಲ ಗುಲಗಂಜಿ
\end{verse}
ಒಟ್ಟಿನಲ್ಲಿ ಸಂಸ್ಕೃತವಿದ್ವಲ್ಲೋಕಕ್ಕೆ ಭಟ್ಟರು ಆಸ್ತಿ. ಭಟ್ಟರ ಒಡನಾಟ, ಪಾಂಡಿತ್ಯ ಹಾಗೂ ವ್ಯಕ್ತಿತ್ವದ ಪರಿಚಯ ಕಾರ್ಯ ಪೂರ್ವ ಜನ್ಮದ  ಪುಣ್ಯವಿಶೇಷ ಲಭ್ಯ ಅಲ್ಲದೇ ಮತ್ತಿನ್ನೇನು. 


\articleend
