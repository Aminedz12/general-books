{\fontsize{14}{16}\selectfont
%\addtocontents{toc}{\protect\newpage}
\chapter{ವಿ~॥ ಗಂಗಾಧರ ಭಟ್ಟರು ನಾ ಕಂಡಂತೆ}

\begin{center}
\Authorline{ಪ್ರೊ~। ಸುಬ್ರಾಯ ವಿ ಭಟ್ಟ}
\smallskip

ಮೀಮಾಂಸಾವಿಭಾಗಾಧ್ಯಕ್ಷರು\\   
ರಾಜೀವಗಾಂಧೀ ಪರಿಸರ\\
ಶೃಂಗೇರಿ
\addrule	
\end{center}

ಅಪ್ರತಿಮ ಪ್ರತಿಭಾವಂತರಾದ ವಿ~॥ ಗಂಗಾಧರ ಭಟ್ಟರು ನ್ಯಾಯಶಾಸ್ತ್ರ ಪ್ರಾಧ್ಯಾಪಕ\-ರಾಗಿ ಸೇವೆ ಸಲ್ಲಿಸಿ ಸರ್ಕಾರದ ನಿಯಮದಂತೆ ನಿವೃತ್ತಿ ಹೊಂದುತ್ತಿದ್ದಾರೆ. ಇದು ಅವರ ವೃತ್ತಿಯಿಂದ (ಸರ್ವಕಾರೀಯ ನಿಯಮದಂತೆ) ನಿವೃತ್ತಿಯೇ ವಿನಃ ಪ್ರವೃತ್ತಿಯಿಂದಲ್ಲ. ಏಕೆಂದರೆ   \enginline{-}   ಅವರು ನ್ಯಾಯಶಾಸ್ತ್ರವಿಭಾಗದ ಪ್ರಾಧ್ಯಾಪಕರಾಗಿ ವೃತ್ತಿಯನ್ನು ಆರಂಭಿಸುವು\-ದಕ್ಕಿಂತಲೂ ಮೊದಲೂ ಸಹ ಸಹಜವಾಗಿಯೇ ನ್ಯಾಯಶಾಸ್ತ್ರ ಓದುವ ವಿದ್ಯಾರ್ಥಿಗಳಿಗೆ ಪಾಠ  \enginline{-}  ಪ್ರವಚನವನ್ನು ನಿರ್ವಹಿಸುತ್ತ ಬಂದವರು.

ನಾನೂ ಸಹ ಅವರಲ್ಲಿ ನ್ಯಾಯಗ್ರಂಥಗಳನ್ನು ಓದಿದವರಲ್ಲಿ ಒಬ್ಬನು. ವೈಯಕ್ತಿಕವಾಗಿ ಹೇಳುವುದಾದರೆ, ನಾವು ಮೂಲತಃ ಒಂದೇ ಕುಟುಂಬದವರೇ. ಆದರೆ ಅವರ ಅಜ್ಜನ ಕಾಲದಲ್ಲಿಯೇ ಮೂಲಸ್ಥಾನವನ್ನು ಬಿಟ್ಟು ಹೊರಗಡೆ ನೆಲೆಸಿದವರು. ನಾನೂ ಸಹ ನನ್ನ ಪೂರ್ವಿಕರ ಸಮಯದಲ್ಲೇ ಮೂಲಸ್ಥಾನವನ್ನು ಬಿಟ್ಟವನು. ಕೌಟುಂಬಿಕ ಸಂಬಂಧದಲ್ಲಿ ಹೇಳುವುದಾದರೆ, ಅಶೌಚ ಸಂಬಂಧವನ್ನು ಹೊಂದಿದ ದಾಯಾದರು. ಆದರೂ ನನಗೆ ಅವರ ಪರಿಚಯವಾದದ್ದು ನಾನು ಮೈಸೂರಿಗೆ ಬಂದ ಮೇಲೆ.

ನಾನು ಮೈಸೂರಿಗೆ ಶಾಸ್ತ್ರಾಧ್ಯಯನಕ್ಕೆ ಬಂದವನು. ಮಹಾರಾಜ ಸಂಸ್ಕೃತ ಮಹಾ\-ಪಾಠಶಾಲೆಯಲ್ಲಿ ಪೂರ್ವಮೀಮಾಂಸಾಶಾಸ್ತ್ರವನ್ನು ಓದಿದವನು. ಆ ಸಮಯಕ್ಕೆ\break  ಶ್ರೀಯುತರು ಮಹಾರಾಜ ಸಂಸ್ಕೃತ ಮಹಾಪಾಠಶಾಲೆಯಲ್ಲಿ ವಿದ್ವಾನ್ ರಾಮ\-ಭದ್ರಾಚಾರ್ಯರಲ್ಲಿ ತಮ್ಮ ನ್ಯಾಯಶಾಸ್ತ್ರಾಧ್ಯಯನವನ್ನು ಮುಗಿಸಿ ಶ್ರೀ ಶಂಕರವಿಲಾಸ ಸಂಸ್ಕೃತ ಪಾಠಶಾಲೆಯಲ್ಲಿ ಮುಖ್ಯಾಧ್ಯಾಪಕರಾಗಿ ಕಾರ್ಯ ನಿರ್ವಹಿಸುತ್ತಿದ್ದರು. ಇನ್ನೂ ಮಹಾರಾಜ ಸಂಸ್ಕೃತ ಮಹಾಪಾಠಶಾಲೆಯಲ್ಲಿ ಪ್ರಾಧ್ಯಾಪಕರಾಗಿ ಸೇರಿರಲಿಲ್ಲ.

ಸಾಮಾನ್ಯ ಮಧ್ಯಮ ವರ್ಗದ ಆರ್ಥಿಕ ಸ್ಥಿತಿಯಿರುವ ಹಳ್ಳಿಯ ವೈದಿಕ ಕುಟುಂಬದ ಹಿನ್ನೆಲೆಯಿಂದ ಅಧ್ಯಯನಕ್ಕಾಗಿ ಮೈಸೂರಿಗೆ ಬಂದವರು. ಅವರ ತಂದೆಯವರೂ ಸಹ ವೇದ, ಸಂಸ್ಕೃತ ಹಾಗೂ ಜ್ಯೌತಿಷ ಮತ್ತು ಪೂರ್ವಾಪರ ಗೃಹ್ಯಪ್ರಯೋಗವನ್ನು ಚೆನ್ನಾಗಿ ಬಲ್ಲ ಅನುಷ್ಠಾನಪರರಾದ ವೈದಿಕರು. ಓದುವ ಸಮಯದಲ್ಲೂ ಸಹ ಮನೆಯಿಂದ ವಿಶೇಷ\-ವಾದ ಸಹಾಯ  \enginline{-}  ಸವಲತ್ತುಗಳನ್ನು ಹೊಂದಿದವರಲ್ಲ. ಮೈಸೂರಿನ ಶಂಕರ\-ವಿಲಾಸ ಸಂಸ್ಕೃತ ಪಾಠಶಾಲೆಯಲ್ಲಿ ಉದ್ಯೋಗಕ್ಕೆ ಪ್ರವೇಶಿಸಿದ ನಂತರ [ಸುಮಾರು ೩೫  \enginline{-}  ೪೦ ವರ್ಷಗಳ ಹಿಂದೆ] ತಮ್ಮ ತಂಗಿಯರನ್ನು ಮೈಸೂರಿಗೆ ಕರೆತಂದು ಅವರನ್ನು ಓದಿಸಿದರು. ಅವರ ಕುಟುಂಬದ ಅನೇಕ ಸದಸ್ಯರನ್ನು ಅಂದರೆ ಅಣ್ಣನ ಮಕ್ಕಳು, ಅಕ್ಕನ ಮಕ್ಕಳು, ದೊಡ್ದಪ್ಪನ ಮೊಮ್ಮಕ್ಕಳು ಹೀಗೆ ಬಂಧುವರ್ಗದ ಅನೇಕರನ್ನು ತಮ್ಮ ಜೊತೆಗೇ ಇಟ್ಟುಕೊಂಡು ಅಶನ  \enginline{-}  ವಸನಗಳೊಂದಿಗೆ ಓದಿಸಿದವರು.

ನಿರಂತರ ಅಧ್ಯಯನಶೀಲರಾದ ಗಂಗಣ್ಣನವರು ನ್ಯಾಯಶಾಸ್ತ್ರ ಗ್ರಂಥಗಳನ್ನು ಅನೇಕರಿಗೆ ಪಾಠ ಮಾಡಿದ್ದಾರೆ. ಕರ್ತವ್ಯದಲ್ಲಿ ಸೇರಿದ್ದರಿಂದ ಅಧ್ಯಾಪಕರಾಗಿ ತರಗತಿಯಲ್ಲಿ ಪಾಠ ಮಾಡುವುದು ವಿಶೇಷವೇನಲ್ಲ. ನ್ಯಾಯಶಾಸ್ತ್ರ ವಿಭಾಗದ ಪ್ರಾಧ್ಯಾಪಕರಾಗಿ ಪಾಠವನ್ನೇ ಮಾಡದೇ ಅಧ್ಯಾಪಕ ವೃತ್ತಿಯನ್ನು ಮುಗಿಸಿದವರಿರುವ ಇಂದಿನ ಕಾಲದಲ್ಲಿ ಯಾವುದೇ ಫಲಾಪೇಕ್ಷೆಯಿಲ್ಲದೇ ಅನೇಕ ವಿದ್ಯಾರ್ಥಿಗಳಿಗೆ ಪಾಠ ಮಾಡುವವರಿದ್ದಾರೆ ಎಂದಾದರೆ ಅಂಥವರಲ್ಲಿ ಶ್ರೀಯುತ ಗಂಗಾಧರ ಭಟ್ಟರೂ ಒಬ್ಬರು.

ಕೇವಲ ಗ್ರಂಥಾಧ್ಯಯನ ಮಾತ್ರವಲ್ಲದೇ ಅದರ ಮೇಲೆ ವಿಮರ್ಶೆ ಮಾಡುವ ಚಿಂತನ ಸ್ವಭಾವ ಅವರ ಅಪರೂಪದ ಅಧ್ಯಯನ ಕ್ರಮ. ಅದನ್ನೇ ಅಧ್ಯಾಪನದಲ್ಲೂ ರೂಢಿಸಿಕೊಂಡವರು. ಇನ್ನೂ ಹೇಳಬೇಕಾದರೆ ಆಂಗ್ಲ ಭಾಷೆಯಲ್ಲೂ ವಿಶೇಷವಾದ ಹಿಡಿತ\-ವನ್ನು ಹೊಂದಿರುವವರು. ಅನೇಕ ಜನ ವಿದೇಶೀಯರಿಗೂ ಶಾಸ್ತ್ರ ಪಾಠ ಮಾಡಿದವರು. ಅಷ್ಟೇ ಅಲ್ಲ ಆಯುರ್ವೇದ ಮೂಲ ಗ್ರಂಥಗಳ ಅಧ್ಯಯನವನ್ನು ಹೊಂದಿದ ಭಟ್ಟರು ಮೈಸೂರಿನ ಆಯುರ್ವೇದ ಕಾಲೇಜಿನ ಅನೇಕ ವಿದ್ಯಾರ್ಥಿಗಳಿಗೆ ಪಾಠ ಮಾಡಿದವರು. ಅಲ್ಲಿಯ ಅನೇಕ ವಿದ್ಯಾರ್ಥಿಗಳು ಮನೆಗೇ ಬಂದು ಪಾಠ ಹೇಳಿಸಿಕೊಂಡು ಹೋಗುವವರನ್ನು ನಾನು ನೋಡಿದ್ದೇನೆ.

“\textbf{ಯಸ್ತು ಕ್ರಿಯಾವಾನ್ ಪುರುಷಃ ಸ ವಿದ್ವಾನ್}” ಎಂಬ ಸುಭಾಷಿತದಂತೆ ಸರ್ವದಾ ಕ್ರಿಯಾಶೀಲರು. ಮನೆಯಲ್ಲಿ ಯಾವಾಗಲೂ ಜನಜಂಗುಳಿ. ಒಂದಲ್ಲಾ ಒಂದು ಕೆಲಸ\enginline{-}ಕಾರ್ಯಗಳಿಗಾಗಿ ಅವರನ್ನು ನೋಡಲು ಬರುವವರೇ. ಲೌಕಿಕ ಕಾರ್ಯವೇ ಇರಲಿ, ಪಾಠಕ್ಕೆ ಸಂಬಂಧಿಸಿಯೇ ಇರಲಿ ಈ ವಿಷಯದಲ್ಲಿ ಅವರ ಬಂಧುವರ್ಗದವರು ಅಥವಾ ಊರಿನವರು ಅಥವಾ ಬೇರೆಯವರು ಇತ್ಯಾದಿ ಯಾವ ಭೇದ\enginline{-}ಭಾವವೂ ಇಲ್ಲದೇ ಪ್ರಾಮಾಣಿಕವಾಗಿ ನಿಃಸ್ವಾರ್ಥದಿಂದ ತಮ್ಮ ಸಹಾಯಹಸ್ತವನ್ನು ಚಾಚುತ್ತಿದ್ದವರು. ಹಾಗಾಗಿಯೇ ಮೈಸೂರಿನ ಜನಮಾನಸದಲ್ಲಿ ಆತ್ಮೀಯ ಸ್ಥಾನವನ್ನು ಹೊಂದಿದವರಾಗಿದ್ದಾರೆ. ಈ ಮಧ್ಯೆ ಆತ್ಮೀಯರ ಒತ್ತಾಸೆಗೆ ಮಣಿದು ಸ್ಥಾನೀಯವಾಗಿ ಮೈಸೂರಿನಲ್ಲಿ ಹಾಗೂ ನಾಡಿನ ಬೇರೆ ಬೇರೆ ಸ್ಥಳಗಳಲ್ಲಿ ಧಾರ್ಮಿಕ/ಆಧ್ಯಾತ್ಮಿಕ/ಶಾಸ್ತ್ರೀಯ ಉಪನ್ಯಾಸಗಳನ್ನು ನಿರ್ವಹಿಸುವುದು ಅನಿವಾರ್ಯತೆಯಲ್ಲಿ ಒಂದಾಗಿತ್ತು. ಹೀಗೆ ಅವರ ಜೀವನವೇ ಒಂದು ರೀತಿಯ ಅವಿರತವಾದ ಯಂತ್ರವಿದ್ದಂತೆ
\vskip 2pt

ಆಸ್ತಿಕರಾಗಿ ಗುರು\enginline{-}ಹಿರಿಯರ ಬಗ್ಗೆ ಗೌರವಭಾವವನ್ನು ಹೊಂದಿದ ಪರಿಶುದ್ಧಾಂತ\-ರಂಗದವರು. 
\vskip 2pt

ಶಮ\enginline{-}ದಮ  \enginline{-}  ತಿತಿಕ್ಷಾ  \enginline{-}  ಉಪರತಿ ಇತ್ಯಾದಿ ಆತ್ಮಗುಣವನ್ನು ಜನ್ಮಶಃ ಹೊಂದಿದವ\-ರಾಗಿ ಅದನ್ನು ಶಾಸ್ತ್ರಸಂಸ್ಕಾರದಿಂದ ಬೆಳೆಸಿಕೊಂಡವರು. ಅನ್ಯಾಯವನ್ನು ಎಂದಿಗೂ ಸಹಿಸಿ ಸುಮ್ಮನಿರುವ ಸ್ವಭಾವ ಅವರದ್ದಲ್ಲ. ಹೋರಾಟ ಮಾಡಿಯಾದರೂ ನ್ಯಾಯವನ್ನು ಪಡೆಯ\-ಬೇಕೆಂಬುದು ಅವರ ಪ್ರಾಮಾಣಿಕ ಕೆಚ್ಚೆದೆಯ ಹಂಬಲ. ಅವರ ಶಿಷ್ಯರಾಗಿರು\-ವವರು ರೂಢಿಸಿಕೊಳ್ಳಲೇಬೇಕಾದ ಪ್ರಧಾನ ಗುಣಗಳಲ್ಲಿ ಇದೂ ಒಂದು.
\vskip 2pt

ಸ್ವಾಭಾವಿಕವಾಗಿಯೇ ಪರಿಶುದ್ಧ ಹಸ್ತರು. ಸರಳಸಜ್ಜನಿಕೆಯ ವ್ಯಕ್ತಿತ್ವಹೊಂದಿದ ಭಟ್ಟರನ್ನು ಅನೇಕ ಸಂಘ ಸಂಸ್ಥೆಗಳಲ್ಲಿ ಪದಾಧಿಕಾರಿಗಳನ್ನಾಗಿ ಹಾಗೂ ಮಾರ್ಗದರ್ಶಕರನ್ನಾಗಿ ನೇಮಿಸಿಕೊಂಡಿರುವ ಉದಾಹರಣೆಗಳೂ ಕಡಿಮೆ ಇಲ್ಲ.
\vskip 2pt

ಇತ್ತೀಚಿನ ದಿನಗಳಲ್ಲಿ ಸ್ವಲ್ಪಮಟ್ಟಿಗೆ ಆರೋಗ್ಯದ ಸಮಸ್ಯೆಯಿಂದ ಬಳಲಿದರೂ, ತಮ್ಮ ದೈನಂದಿನ ಅಧ್ಯಯನ ಹಾಗೂ ಕರ್ತವ್ಯ ಮತ್ತು ಪಾಠ ಪ್ರವಚನವನ್ನು ಬಿಡಲಿಲ್ಲ. ಇಂತಹ ನಿಷ್ಠೆಯೇ ಅವರ ಶಿಷ್ಯರಿಗೆಲ್ಲ ಅನುಸರಿಸಬೇಕಾದ ಆದರ್ಶ. ನಾನು ನೋಡಿದ ಹಾಗೆ ಮದುವೆಯ ಮೊದಲು ಅವರ ತಂಗಿಯರು (ಮೈಸೂರಿನಲ್ಲೇ ಕಾಲೇಜು ಶಿಕ್ಷಣ\-ವನ್ನು ಓದುತ್ತಿದ್ದವರು) ಹಾಗೂ ಮದುವೆಯ ನಂತರ ಅವರ ಧರ್ಮಪತ್ನಿ, ಅವರ ಮನಸ್ಥಿತಿಗೆ ಅನುಗುಣವಾಗಿ ಊಟ/ತಿಂಡಿ/ಚಹಾ/ಕಾಫಿ/ಕಷಾಯ ಹೀಗೆ ಆಯಾ ಸಮಯದಲ್ಲಿ ಬಂದ ಅತಿಥಿಗಳಿಗೆ ಒದಗಿಸಿ ತಮ್ಮ ಮೂಲ ಸಂಪ್ರದಾಯದಂತೆ ಅತಿಥಿಸತ್ಕಾರವನ್ನೂ ನಡೆಸಿಕೊಂಡು ಬಂದಿರುವುದು ಗಮನಿಸಲೇಬೇಕಾದ ಅಂಶ. ಮಹಾನಗರಗಳಲ್ಲಿ ತಮ್ಮ ತಮ್ಮ ದೈನಂದಿನ ಕರ್ತವ್ಯ ನಿರ್ವಹಣೆಯ ಜೊತೆಗೆ ಇದು ಕಷ್ಟಸಾಧ್ಯವೇ ಆದರೂ ಸಂತೋಷದಿಂದಲೇ ನಿರ್ವಹಿಸುತ್ತಿರುವುದನ್ನು ನಾನು ಸ್ವತಃ ನೋಡಿ ಗಮನಿಸಿದ ವಿಶೇಷ.

ಮನೆಯಲ್ಲಿ ತಂದೆ\enginline{-}ತಾಯಿ, ಅಣ್ಣ\enginline{-}ತಮ್ಮ, ಅಕ್ಕ\enginline{-}ತಂಗಿ ಜೊತೆಗೆ ಶಿಷ್ಯರು, ವಿಶ್ವಾಸಿಗಳು, ಅಭಿಮಾನಿಗಳು ಹೀಗೆ ಎಲ್ಲರಲ್ಲೂ ವಿಶ್ವಾಸದಿಂದ ನಡೆದುಕೊಂಡು ಬರುತ್ತಿರುವ ಶ್ರೀಯುತ ಗಂಗಣ್ಣನವರು ಸಾವಿರಾರು ಶಿಷ್ಯರಿಗೆ ಪ್ರಾಮಾಣಿಕವಾಗಿ ಪಾಠ ಹೇಳಿ ತಮ್ಮ ಅಧ್ಯಾಪಕವೃತ್ತಿಗೆ ನ್ಯಾಯವನ್ನು ಒದಗಿಸಿದ್ದಾರೆ ಎಂಬುದು ಅವರ ಶಿಷ್ಯರು ಹಾಗೂ ಅಭಿಮಾನಿಗಳೆಲ್ಲ ಸೇರಿ ಈ ಸಂದರ್ಭದಲ್ಲಿ ಸಲ್ಲಿಸುತ್ತಿರುವ ಗುರುವಂದನೆಯೇ ಸಾಕ್ಷಿ.

ಈಗ ವೃತ್ತಿಯಿಂದ (ಸರ್ಕಾರಿ ನಿಯಮದಂತೆ) ನಿವೃತ್ತಿಯಾಗುತ್ತಿದ್ದಾರೆಯೇ ವಿನಾ ಪ್ರವೃತ್ತಿಯಿಂದಲ್ಲ. ಏಕೆಂದರೆ ಈ ಅಧ್ಯಾಪನ ಪ್ರವೃತ್ತಿ, ಅವರು ವೃತ್ತಿಯನ್ನು ಆರಂಭಿ\-ಸುವ\-ದಕ್ಕಿಂತ ಮೊದಲೇ ಬೆಳೆಸಿಕೊಂಡಿದ್ದು. ಸಮಾಜಕ್ಕೆ ಅವರ ಮಾರ್ಗದರ್ಶನ ಇನ್ನೂ ಹೆಚ್ಚಿನ ರೀತಿಯಲ್ಲಿ ಒದಗಲಿ. ಧರ್ಮ ಮಾರ್ಗದಲ್ಲಿ ಮುಂದಿನ ಪೀಳಿಗೆ ಮುನ್ನಡೆಯುವಲ್ಲಿ ಶ್ರೀಯುತರ ಮಾರ್ಗದರ್ಶನ ಹಾಗೂ ಆಶೀರ್ವಾದ ಮುಂದುವರಿಯಲಿ. ಅದಕ್ಕಾಗಿ ಅವರಿಗೆ ಆರೋಗ್ಯ, ಆಯುಷ್ಯ ಹಾಗೂ ಭಾಗ್ಯಗಳನ್ನು ಜಗನ್ಮಾತೆಯಾದ ಶಾರದಾಂಬೆ ಅನುಗ್ರಹಿಸಲಿ ಎಂಬುದಾಗಿ ಜಗನ್ಮಾತೆಯಾದ ಶಾರದಾಂಬೆಯ ಸನ್ನಿಧಿಯಲ್ಲಿ ಪ್ರಾರ್ಥಿಸುತ್ತೇನೆ.

\articleend
}
