{\fontsize{14}{16}\selectfont
\chapter{ರತ್ನಾಕರೋಽಯಂ ಗಂಗಾಧರಃ}

\begin{center}
\Authorline{ವಿ~। ಲಕ್ಷ್ಮೀನಾರಾಯಣ ಎನ್ ಹೆಗಡೆ}
\smallskip

ಲೆಕ್ಖಪರಿಶೋಧಕರು, ಹಾಗೂ\\
ಅಧ್ಯಕ್ಷರು ಹವೀಕ ಸಂಘ (ರಿ)\\ 
ಮೈಸೂರು
\addrule
\end{center}

ಅದೊಂದು ದಿನ ವಿದ್ಯೆಯನ್ನೂ, ಅದಕ್ಕಾಗೊಂದು ನೆಲೆಯನ್ನೂ ಅರಸಿ ಎಲ್ಲೆಲ್ಲೋ ತಿರುಗಾಡುತ್ತಾ ನಿರಾಶ್ರಿತರಂತೆ ಮೈಸೂರಿಗೆ ಬಂದಂತಹ ಕಾಲ. ಜ್ಞಾನದೇಗುಲವಿದು ನಿರಾಶ್ರಿತರ ತಾಣ ಮೈಸೂರೆಂದು ಕೇಳಿ ತಿಳಿದಿದ್ದೆ. ಆದರೆ ಯಾವ ಜ್ಞಾನವನ್ನು ಸಂಪಾದಿಸಬೇಕು, ಕಲಿಯುವುದಾದರೂ ಎಲ್ಲಿ, ಆಶ್ರಯವೆಲ್ಲಿ, ಕಲಿಯುವುದಾದರೂ ಏನು ಎಂಬಂತಹ ಹತ್ತು ಹಲವು ಪ್ರಶ್ನೆಗಳು ನನ್ನನ್ನು ಕಾಡುತ್ತಿದ್ದವು.

ಸಂಭಂಧಿಕರಾದ ಶ್ರೀಯುತ ಹೇರಂಭ ಭಟ್ಟರು ಕೈ ಬೀಸಿ ಕರೆದು ನಿನಗಿದೆ ಇಲ್ಲಿ ಆಶ್ರಯ, ನಾವೆಲ್ಲಾ ನಿಮ್ಮವರು. ನಿನ್ನ ಓದನ್ನು ಸಂಸ್ಕೃತದಿಂದ ಪ್ರಾರಂಭಿಸು ಎಂದು ಹೇಳಿ ನನಗೆ ಆಶ್ರಯವನ್ನು ಕಲ್ಪಿಸಿ, ಅಣ್ಣ ಗಂಗಾಧರನನ್ನು ಪರಿಚಯಿಸಿದರು. ಮುಂದೆ ಇದೇ ಪರಿಚಯವೇ ಸಂಬಂಧವನ್ನು ದಾಟಿ ನನಗೆ ಜೀವನದ ದಾರಿಯನ್ನು ತೋರಿಸಿ\-ಕೊಟ್ಟವರು ಈ ಮಹನೀಯರು.

ಇಂದು ಮೈಸೂರಿನಲ್ಲಿ ಸರಿಯಾಗಿ ನೆಲೆ ನಿಂತಿದ್ದೇವೆ ಎಂದರೆ ಅದರ ಪೂರ್ಣಾನುಗ್ರಹದ ಫಲವನ್ನು ಇವರಿಗೆ ಸಲ್ಲಿಸಿದರೆ ತಪ್ಪಾಗಲಾರದು. ಇಂತಹ ಸಾವಿರಾರು ಅನಾಥರನ್ನು ನಾಥರನ್ನಾಗಿಸಿದವರು ಈ ಗಂಗಾಧರ ಭಟ್ಟರು. ಇವರಿಂದ ನಡೆಯುವ ಜ್ಞಾನ ದಾಸೋಹದ ಔತಣ ಸಾವಿರಾರು ಮಲೆನಾಡಿಗರನ್ನು ಸಾಗರದಾಚೆಗೂ ತಳ್ಳಿದೆ.

ಭಟ್ಟರು ಅಂದರೆ ಒಂದು ರೀತಿಯಲ್ಲಿ ಸಮುದ್ರದಂತೆ. ಇವರಲ್ಲಿ ಸಿಗದೇ ಇರುವ ವಿಷಯವೇ ಇಲ್ಲ. ಇವರೋರ್ವ ಜ್ಞಾನ ಭಂಡಾರ. ಸಾಧಾರಣವಾಗಿ ಸಂಸ್ಕೃತ ವಿದ್ವಾಂಸರಿಗೆ ಅನ್ಯಭಾಷೆ ಅಲರ್ಜಿಯಾಗಿರುತ್ತದೆ.ಕಾರಣ ಕಲಿಯಲು ಕ್ಲಿಷ್ಟ ಹಾಗು ನಿರಾಸಕ್ತಿ. ಆದರೆ ಗಂಗಾಧರ ಭಟ್ಟರು ಎಲ್ಲಾ ಭಾಷೆ, ವಿಷಯಗಳಲ್ಲಿ ಪಾಂಡಿತ್ಯವನ್ನು ಕಂಡು\-ಕೊಂಡವರು. ಹೆಚ್ಚಿನ ಸಂಸ್ಕೃತ ಪಾಠವನ್ನು ಅವರು ಹೊರದೇಶ ಹಾಗು ಹೊರರಾಜ್ಯದವರಿಗೆ ಮಾಡುತ್ತಿದ್ದದನ್ನು ನಾವು ಕಂಡಿದ್ದೇವೆ. ಸಂಸ್ಕೃತ ಭಾಷೆ ಹಾಗೂ ಶಾಸ್ತ್ರದ ತಿರುಳನ್ನು ನಿರರ್ಗಳ\-ವಾಗಿ ಭಾಷೆಯ ತೊಡಕಿಲ್ಲದೆ ಪಾಠ ಮಾಡುತ್ತಿದ್ದುದು ಅವರ ವಿಶೇಷತೆ. ಭಟ್ಟರು ಕೇವಲ ಸಂಸ್ಕೃತ ಪಾಂಡಿತ್ಯವನ್ನು ಮಾತ್ರ ಪಡೆದಿರಲಿಲ್ಲ. ಅನ್ಯ ವಿಷಯಗಳಲ್ಲೂ ಅವರ ಜ್ಞಾನ ಆಳವಾಗಿತ್ತು ಎಂಬುದೇ ನನ್ನದೊಂದು ಉದಾಹರಣೆ. ನಾನು ನನ್ನ ಕಾನೂನು ಪದವಿ ಪೂರೈಸುವಾಗ, ಅತ್ಯಂತ ಕ್ಲಿಷ್ಟಕರವಾದ, ಕಾಲೇಜಿನಲ್ಲಿಯೂ ಸಹ ಉಪನ್ಯಾಸಕರಿಗೆ ಅದನ್ನು ಅರ್ಥೈಸಲಾಗದ ಒಂದು ವಿಷಯ, ಅದನ್ನು ಇವರಲ್ಲಿ ಬಂದು ಕೇಳಿದಾಗ, ಆ ವಿಷಯವನ್ನು ಯಾವುದೇ ಗ್ರಂಥದ ಅಧಾರವಿಲ್ಲದೇ ನಿರಾಯಾಸವಾಗಿ ನನ್ನ ತಲೆಗೆ ತುಂಬಿದರು. ಅದರ ಇಂಗ್ಲೀಷನ್ನು ಕಾಠಿಣ್ಯ ಪದಪುಂಜಗಳು ಯಾವ ನಿಘಂಟುವಿನಲ್ಲೂ ಸಿಗುವಂತಿರಲಿಲ್ಲ. ನನಗೆ ತುಂಬಾ ಆಶ್ಚರ್ಯವಾಯಿತು. ಅಂತಹ ಜ್ಞಾನಬುತ್ತಿ ಗಂಗಾಧರ ಭಟ್ಟರು. ಸಾಮಾನ್ಯ ವ್ಯಾವಹಾರಿಕ ಜ್ಞಾನವಂತೂ ಅಪ್ರತಿಮ. ಕಾನೂನು ವಿಭಾಗವಿರಲಿ, ಮನೆಯ ವ್ಯಾಜ್ಯ ಪಂಚಾಯತಿಯಿರಲಿ, ಸಾಂಸ್ಕೃತಿಕ ಸಾಮಾಜಿಕ ವಿಭಾಗವಿರಲಿ ಎಲ್ಲ ವಿಭಾಗದಲ್ಲೂ ಅವರ ಅತ್ಯುತ್ತಮವಾದ ಸಲಹೆ ಅವರ್ಣನೀಯ. ಹಲವಾರು ಪಾಠಪ್ರವಚನಗಳ ಹೊರತಾಗಿಯೂ ಪ್ರಯೋಜನ, ಸಲಹೆ ಪಡೆದುಕೊಂಡವರು ಬಹಳಷ್ಟು ಜನ.

ಇನ್ನು ಗಂಗಾಧರ ಭಟ್ಟರ ಪಾಠಪ್ರವಚನಗಳ ಕುರಿತು ಹೇಳಬೇಕೆಂದರೆ, ಇರುವೆಗೆ ಸಕ್ಕರೆ, ಬೆಲ್ಲ ಎಷ್ಟು ಪ್ರೀತಿಯೋ, ಹಾಗೆ ಅವರ ಶಿಷ್ಯರಿಗೆ ಅವರ ಪಾಠ ಅಷ್ಟೇ ಪ್ರೀತಿ. ಅವರ ಪಾಠ ಎಂದರೆ ಕೊಠಡಿ ಯಾವಾಗಲೂ ತುಂಬಿರುತ್ತದೆ. ಅವರ ಪಾಠದ ಶೈಲಿಯಂತೂ ಅಪ್ರತಿಮ. \textbf{“ಬಾಲಾನಾಂ ಸುಖಬೋಧಾಯ”} ಎಂಬ ತರ್ಕಸಂಗ್ರಹದ ವಾಕ್ಯದಂತೆ ಅಬಾಲವೃದ್ಧರಿಗೂ ಸಹ ಅವರವರ ಮಟ್ಟಕ್ಕೆ ಇಳಿದು ಜ್ಞಾನವನ್ನು ತುಂಬುವುದು ಅವರ ವಿಶೇಷತೆ. ಅವರ ವಿಷಯವೇ ತರ್ಕ. ತರ್ಕ ಎಂಬ ಕಬ್ಬಿಣದ ಕಡಲೆಯನ್ನು ಹಲ್ಲಿಲ್ಲದವನು ಸಹ ಕಡಿಯುವಂತಿರುತ್ತದೆ ಅವರ ತರ್ಕ ಪಾಠ. ಯಾವ ವಿಷಯದಲ್ಲೇ ಆದರೂ ಸುಲಲಿತವಾದ ಅವರ ಪಾಠ ವಿದ್ಯಾರ್ಥಿಗಳಲ್ಲಿ ಆಸಕ್ತಿ ಹೆಚ್ಚಿಸುತ್ತಿತ್ತು. ಇನ್ನು ಮನೆ\-ಪಾಠವಂತು ಪ್ರತಿನಿತ್ಯ ಅನ್ನ ದಾಸೋಹದಂತೆ ನಡೆಯುತ್ತದೆ. ಅವರಿಗಿರುವ ಶಿಷ್ಯವೃಂದವನ್ನು ಪರಿಗಣಿಸಿದರೆ ಅವರು ಒಂದು ವಿದ್ಯಾಶಾಲೆಯನ್ನು ತೆರೆಯಬಹುದಿತ್ತು. ಅದಕ್ಕೆ ಸರಿಯಾದ ಸಹಕಾರವೂ ಇತ್ತು. ಬೇರೆಯವರಾದರೆ ಇದನ್ನೇ ಉಪಯೋಗಿಸಿ, ಹಣಗಳಿಸಿ ಇಂದು ಐಷಾರಾಮಿ ಜೀವನ ನಡೆಸುತ್ತಿದ್ದರೇನೋ? ಗಂಗಾಧರ ಭಟ್ಟರ ಜ್ಞಾನ ಸಮುದ್ರದಲ್ಲಿ ಮಿಂದೆದ್ದು ಹೋದವರು ಸಾವಿರಾರು ಮಂದಿ, ಆದರೆ ಗಂಗಾಧರರು ಮಾತ್ರ ಅದೇ ಸಮಸ್ಥಿತಿಯಲ್ಲಿಯೇ ಇದ್ದಾರೆ. \textbf{“ಪರೋಪಕಾರಾರ್ಥಮಿದಂ ಶರೀರಮ್”} ಎಂಬಂತೆ ಬೇರೆಯವರನ್ನು ಉಪಕರಿಸುವುದೆಂದರೆ ಗಂಗಾಧರ ಭಟ್ಟರಿಗೆ ಅತೀವ ಪ್ರೀತಿ. ಅದರಲ್ಲೇ ಸಾರ್ಥಕಥೆಯನ್ನು ಕಂಡುಕೊಂಡವರು ಎಂದರೆ ಅತಿಶಯೋಕ್ತಿಯಲ್ಲ.

ಮನೆಗೆ ಯಾರೇ ಏನನ್ನೇ ಕೇಳಿಕೊಂಡು ಬಂದರೂ ಗಂಗಾಧರ ಭಟ್ಟರಲ್ಲಿ ಅದಕ್ಕೆ ಪರಿಹಾರವಿರುತ್ತದೆ. ಸಂಬಂಧಪಟ್ಟ ವ್ಯಕ್ತಿಗಳನ್ನು ಸಂಪರ್ಕಿಸಿ ಬಂದವರ ಸಮಸ್ಯೆಗೆ ಪರಿಹಾರ ಮಾಡಿ ಉಪಕರಿಸುವುದು ಇವರ ಸ್ವಭಾವ. ಅದರಲ್ಲಿ ಕಿಂಚಿತ್ತೂ ಸ್ವಾರ್ಥದ ಲವಲೇಶ ಕಾಣುವುದಿಲ್ಲ. ಅದಕ್ಕೆ ಸರಿಯಾಗಿ ಅವರ ಧರ್ಮಪತ್ನಿ ಶೈಲಕ್ಕನೂ ಸಂಪೂರ್ಣ ಸಹ\-ಕಾರಿಯಾಗಿದ್ದಾರೆ.

ಇಂತಹ ಅಪೂರ್ವ ವ್ಯಕ್ತಿ ಸರ್ಕಾರಿ ವೃತ್ತಿ ಜೀವನದಿಂದ ನಿವೃತ್ತರಾಗುತ್ತಿದ್ದಾರೆ. ನಿಜವಾದ ಇಂತಹವರ ನಿಸ್ವಾರ್ಥ, ಪ್ರಾಮಾಣಿಕ ಸೇವೆ ಇನ್ನು ನಮ್ಮ ಸರಕಾರಕ್ಕೆ ಬೇಕಾಗಿದೆ. ಆದರೆ ಸರಕಾರೀಯ ನಿಯಮ ಮೀರುವಂತಿಲ್ಲ ಎಂಬುದು ಸತ್ಯ. ಇಂತಹ ಇವರ ಸೇವೆಯನ್ನು ನಮ್ಮ ಮಲೆನಾಡಿನ ಹವ್ಯಕರಂತೂ ಎಂದೂ ಮರೆಯುವಂತಿಲ್ಲ. ಹವ್ಯಕಸಂಘದಲ್ಲೂ ಸಹ ಇವರು ಅಪಾರ ಸೇವೆ ಸಲ್ಲಿಸಿದ್ದಾರೆ. ಒಂದು ಹಂತದಲ್ಲಿ ಪ್ರಪಾತಕ್ಕೆ ಬಿದ್ದ ಸಂಘವನ್ನು ಎತ್ತಿ ಹಿಡಿದವರು ಇವರು. ಇವರ ಈ ಸೇವೆಯನ್ನು ಪರಿಗಣಿಸಿ ನಾನು ನಮ್ಮ ಸಂಘಂದಿಂದ ಸನ್ಮಾನಿಸಿ ಗೌರವಿಸಿದ್ದು ನನ್ನ ಭಾಗ್ಯ ಎಂದುಕೊಂಡಿದ್ದೇನೆ.

ಇವರ ಸಾಮಾಜಿಕ, ಸಂಸ್ಕೃತ ಸೇವೆ ನಮ್ಮ ಸಮಾಜಕ್ಕೆ ಅತ್ಯಗತ್ಯವಾಗಿ ಬೇಕಾಗಿದೆ. ಇವರ ಜ್ಞಾನ ದಾಸೋಹ ಸದಾ ಸಾರ್ವಜನಿಕರಿಗೆ ಸಿಗಲಿ. ಈ ಸಂದರ್ಭದಲ್ಲಿ ನಾನು ದೇವರಲ್ಲಿ ಪ್ರಾರ್ಥಿಸುವುದೇನೆಂದರೆ ಇವರಿಗೆ ಆರೋಗ್ಯವನ್ನು, ಹೆಚ್ಚಿನ ಆಯುಷ್ಯವನ್ನು ನೀಡಿ ಇವರ ಸೇವೆ ಮುಂದುವರೆಸಲು ಅನುವು ಮಾಡಿಕೊಡಲಿ.

\articleend
}
