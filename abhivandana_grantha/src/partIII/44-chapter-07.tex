{\fontsize{16}{18}\selectfont
\presetvalues
\chapter{अभिनन्दवाचः}

\begin{center}
\Authorline{गुरुचरणवशंवदः }
\smallskip
सुधर्मापरिवारः\\
सुधर्मा\enginline{-}अद्वितीया संस्कृतदिनपत्रिका \\

\addrule
\end{center}

गुरवो हि प्रातस्स्मरणीयाः~। प्रातःकाले मनो निर्मलं दुश्चिन्ताविहीनं च भवति~। तदा देवप्राज्ञगुरवः संस्मरणीयाः~। दैनिककार्येषु निष्प्रत्यूहसिद्धये सदनुग्रहप्राप्तये च देवस्मरणम्~। जीवने सन्मार्गे गमनाय गुरूपदेशाः अनिवार्याः~। तेषां स्मरणेन तेषामनुग्रहेण सह कश्चन सन्मार्गः लभ्यते~। तस्मादेव प्राज्ञाः पूर्वजाः प्रातःकाले गुरुस्मरणम् उपदिदिशुः~। परं स्मरणीयत्वं च गुरौ भवेत्~। आदर्शगुणाः अधीतिबोधाचरणप्रचारणाख्यानि चत्वारि सोपानानि तस्मिन् गुरौ वसेयुः~। तदैव गुरून् छात्राः सास्मर्यन्ते~। 

एवम् आदर्शगुणभूयिष्ठाः प्रातस्स्मरणीयत्वगुणनिधयः सन्ति तत्र भवन्तः विद्वांसः गङ्गाधरभट्टाः~। सार्थकं वृत्तिजीवनं निरुह्य सहस्राधिकच्छात्रान् प्रबोध्य ते स्वकार्यात् निवृत्तिमेष्यन्तः सन्ति~। सर्वं हि कार्यं समयसापेक्षम्~। निवृत्तिः कार्यस्यैव~। किन्तु ज्ञानस्य च निवृत्तिः नैव भाविनी~। अत अधिकृताध्यापककार्यात् एव गुरवः निवृत्तिमेष्यन्तो वर्तन्ते~। ज्ञानं तु तेषां सर्वदा अपेक्षितमेव~।

\textbf{हितं मनोहारि च दुर्लभं वचः} इतीदं वाक्यं प्रसिद्धमेव~। वाचि पटुत्वं न सर्वेषां भवति~। ये च ज्ञानिनः ते बहिः वाग्रूपेण प्रकटयितुं न पारयन्ति~। ये च वाग्मिनः ते ज्ञाने शून्याः अवलोक्यन्ते~। यस्य ज्ञानं वाग्मित्वम् उभयं भवति स एव उत्तमः अध्यापकः भवति~। तदुक्तं महाकविना कालिदासेन --

\begin{verse}
श्लिष्टा क्रिया कस्यचिदात्मसंस्था सङ्क्रान्तिरन्यस्य विशेषयुक्ता~।\\
यस्योभयं साधु स शिक्षकाणां धुरि प्रतिष्ठापयितव्य एव~॥
\end{verse}

इति वचनानुसारेण गङ्गाधरगुरवः शिक्षकाणां धुरि नूनं प्रतिष्ठापयितव्याः एव~। समयानुसारेण श्रोत्रनुरोधेन च विषयप्रस्तुतिः वाग्मिता इत्युच्यते~। न च गङ्गाधरभट्टस्य अभिनन्दनार्थं प्रवृत्ते अस्मिन् लेखे कुतो वा वाग्मित्ववर्णनम् इति वाच्यम्~। गङ्गाधरभट्टस्य वाग्मिपदपर्यायत्वस्वीकरणात्~। श्रीमन्महाराजसंस्कृतपाठशालायां सर्वशास्त्राणां सामान्यविषयाणां च व्याख्याने परमसमर्थाः आसन् गुरुवर्याः~। 

सुधर्मायाः आरम्भात् साम्प्रतिकसमयपर्यन्तम् आधाररूपेण सन्ति प्रख्याताः विद्वांसः तत्र भवन्तो नागराजवर्याः~। परं सुधर्मा बहून् विदुषः साहायत्वेन इच्छति~। देवानुग्रहरूपेण विंशत्यधिकवत्सरेभ्यः पूर्वं सुधर्मापरिवारं प्राविशन् गङ्गाधरभट्टाः~। शास्त्रनिष्णातैः केवलैः न शक्यते सामान्यया सरलसंस्कृतभाषया वार्तादीनां लेखः~। विशेषसामर्थ्यमपेक्षितं तत्र~। गङ्गाधरास्तु सर्वत्र समर्थाः एव~। वार्तालेखादिकं विलिख्य सुधर्माम् अभ्यवर्धयन् ते~। सुधर्मायाः वार्षिकोत्सवेषु कार्यक्रमनिर्वहणमपि ते असकृदकुर्वन्~। तदा जनाः तु अतिथ्यभ्यागतभाषणानाम् अपेक्षया तेषां कार्यक्रमनिर्वहणशैल्यै एव स्निह्यन्ति स्म~। पण्डितान् पामरान् बालान् वृद्धान् च तेषां वाचः आकर्षन्ति~। नैजतया ते भाषणचुञ्चवः~। न केवलं भाषणचणाः किन्तु छात्रप्रेरकाः अप्यासन्~। अनेकधा राष्ट्रस्तरस्य भाषणादिस्पर्धार्थं ते छात्रान् अनयन्~। तदा सर्वशास्त्रेषु अपि तैरेव भाषणानि विलिख्य प्रदत्तानि~। तथासीत् तेषां वैदुष्यम्~। 

ते सरलाः विरलाः वाग्मिनः शास्त्रनदीष्णाः परोपकारनिष्ठाः च विद्वांसः~। न केवलं वाचि तेषां पटुत्वं वैदुष्यं चासीत् किन्तु प्रशासनकौशलमपि तेषामवर्तत~। ते तु अल्पतृप्ताः~। कदापि उन्नतस्थानानि न ते अवाञ्छन्~। परं तत्स्थानभाग्भ्यः प्रशासने सूचनाः प्रदाय समुपाकुर्वन्~। ते यदि अकाङ्क्षयिष्यन् तर्हि विश्वविद्यालयादीनां कुलपतिस्थानमपि व्यभूषयिष्यन्~। परं न तथा तैः काङ्क्षितम्~। विदेशीयानामपि पाठने ते निष्णाः आसन्~। तत्रापि ते नाग्रे वृत्तिम् अन्वसरन्~। एवं ते स्वजीवने सर्वक्षेत्रेषु अपि पद्मपत्रमिवाम्भसा अतिष्ठन्~। 

इतः परं तु छात्राः तेषां पाठनशैल्याः आनन्दमनुभवितुं न शक्नुवन्ति इति खेदः यद्यप्यस्ति~। तथापि तदनिवार्यम्~। ज्ञानवृद्धास्ते अधुना वयोवृद्धाः अपि~। शरीरं तु विश्रान्तिमपेक्षते~। अतः तेषां निवृत्तजीवनं सुखमयं भवतु इत्येवं महतः शिष्यवृन्दस्य आशयः~। सुधर्मापरिवारः सर्वदा तेषां सर्वविधं साह्यं काङ्क्षति~। तेभ्यः भगवान् शुभोदर्कं वेदोक्तशतसंवत्सराधिकम् आयुः उत्तमारोग्यं च प्रददातु इति प्रार्थयमानः तेषां गुरूणां पादारविन्दयोः लेखरूपमिदम्  अभिनन्दनवचः समर्पयति-

\begin{center}
\textbf{गङ्गाधरं तर्कनिष्ठं गुरुं शास्त्रमहोदधिम्~।\\
अभिनन्दति सम्प्रीत्या सुधर्मा दिनपत्रिका~॥}
\end{center}

\articleend

}
