\chapter{A Walking Encyclopedia}

\begin{center}
\Authorline{Sri Avadhany}
Engineer\\
Nanjanagood
\addrule
\end{center}

My father was a Sanskrit scholar and when he passed away in 2002, we thought of providing in his memory monetary assistance to students of Sanskrit who excelled in their studies. I had no idea about how to go about it and at last decided that I should seek help of Maharaja’s Sanskrit College, Mysore whom my father had held in high respect I think it was month of July in the year 2005, that I walked into the office of the Principal of Maharaja Sanskrit College with lots of apprehension as to what would be the reaction to my request. I was happy that my request to give an opportunity to us for extending some monetary rewards to bright students of the institute was well received. And that is how I came in contact with Sri Gangadhar Bhat.

In life one comes across many types of individuals. Some of them you respect, some of them you admire and few are such that you admire as well as respect. Sri Gangadhar Bhat is one such person whom not only do I admire a lot but have a very high respect for him. A frail figure of his camouflages the genius in him. As the time passed by my admiration and respect grew in intensity. Our interventions at anytime used to be short but to me they were very sweet. His booming voice, clear articulation, deep knowledge covering a very wide span of subjects left me spellbound and inspired. I always found him enthusiastic to my queer ideas and a person with full of energy. My concept of ‘true’ education resonated with his. Education is not just passing examinations and acquiring degrees. A true education would trigger interest in students to seek knowledge beyond the confines of university syllabi. Sri Gangadhar Bhat would therefore give assignments to his students to work on subjects beyond subject curriculum and present them in the annual functions that we used to conduct every year in the Sanskrit college. It is not easy to guide students in a subject unless you have mastery over the same. It was amazing that Sri Gangadhar Bhat would guide his students in such a wide span of subjects as mathematics and science, environment, art, literature, history and philosophy. Name any field of human endeavour and one would find Sri Gangadhar Bhat absolutely at home with the subject. He is a walking encyclopaedia and there cannot be two opinions about it. 	

Skill sets required for handling varied responsibilities are very often opposite in nature. To give examples, a person good in studies will normally be not good in sports and vice versa. A boy good in theory will be extremely uncomfortable in practical classes in laboratories but a boy good in lab work will be more often just an average in theory classes. When it comes to industries, we do find that a person with strong leadership is not technically very sound and a person very strong technically quite often is a poor leader of men. But here also Sri Gangadhar Bhat is an exception. His organizing capacity is tremendous. He leads and inspires his students to scale and achieve very ambitious targets.

Beginning of 2018 will witness Sanskrit College losing a jewel – a loss almost impossible to compensate. January 2018 will be the last month when Sri Gangadhar Bhat bids farewell to the institute whom he served with distinction, dedication, dignity and relish for nearly two decades. It is not easy to fill the vacuum created by retirement of such a versatile genius, multifaceted personality, highly knowledgeable walking encyclopaedia. For students, if there has to be an iconic person and a person whom they want to idolize, they cannot find a better person than Sri Gangadhar Bhat. I for sure have been a great admirer of him and he is also an inspiration for me in many of my activities.

I understand that he will return to his roots and is planning to settle down in his native village after his retirement. He has been keeping indifferent health for the past two three years. My prayers to the Almighty for a complete recovery and hope the native environment in his home town, free from modern day pollution will hasten up the process of speedy recovery.

I offer my humble Pranam, admiration and respects to this great ‘Saraswathiputra’. Let God continue to give him strength and vitality to share his immense knowledge with others who are interested in acquiring them. Hope he will be at ease in use of modern technology like Skype etc to spread his knowledge from the comforts of his hometown. On this point I must credit his wife for coming to his rescue and assisting him in the use of these gadgets. I cannot forget her hospitality whenever we used to visit them at their house in Mysore. We will always cherish her hospitality and remember her with great reverence.

With warm regards and deep respects to both Smt and Sri Gangadhar Bhat.

\articleend
