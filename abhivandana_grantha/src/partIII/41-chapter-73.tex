\chapter{Top Ten Qualities of My Uncle and Teacher}

\begin{center}
\Authorline{Praveen bhat}
\smallskip

\end{center}
\begin{verse}
प्रवो महे महि नमो भरद्वमाङ्गूष्यं शवसानाय साम ।\\
ये ना नः पूर्वे पितरः पदज्ञा अङ्गिरसो गा अविन्दन् ॥\\
(य.वे.३४.७)
\end{verse}
Scholars who turned the disciples as skilled and virtuous from the ignorance by giving well education, those scholars are mentioned as father by the pupils.

About 8 years I resided with my uncle in Mysore. In that time I found many qualities in him. He is not only my uncle but also my teacher. Hence I saw him as teacher and uncle.
\begin{verse}
नागुणी गुणिनं वेत्ति गुणी गुणिषु मत्सरी ।\\
गुणी च गुणरागी च विरलः सरलो जनः ॥
\end{verse}
A person without qualities does not know the person with good qualities. A person with good qualities blames and jealous about others qualities. Hence it is very rare that a person with good qualities and appreciates others qualities.

During my residence in Mysore I made out many qualities from him as student and son. I keep on trying to apply those qualities in me which made my life great. There are many things to write but I cannot explain all. So I list out top ten qualities here. We go through one by one in detail.

\section*{01 -SHARPNESS OF THE INTELLECT (तीक्ष्णमतिः):}

\begin{verse}
शुश्रूषा श्रवणं चैव ग्रहणं धारणं तथा ।\\
ऊहापोहार्थविज्ञाने तत्वज्ञानं च धीगुणाः ॥
\end{verse}
\textbf{Tendency to listen, hearing, adoption, keeping it in mind, guessing, discussing, understanding the special meaning and knowing the reality are the qualities of an intellect.}

I used to noticing this quality everyday while he was teaching. His teaching style is completely different. He can explain one single word in different ways. Always students get more than they expected. Even you cannot imagine that he can explain in that way. That much sharpness he has in his intellect. Not only he has sharpness in his intellect but also he can make student’s intellect sharp.  He is able to answer any question of any field in a single moment, at the spot. When the students are answered by him then there is no question of doubts. I wondered about his intelligence because if he read a new book once he is able to criticize or evaluate the book. The way he grasping, the way he thinking I was totally attracted by him.

\section*{02-ELOQUENCE (वाक्पटुत्वम्):}
\begin{verse}
अस्वीकृतव्याकरणौषधानामपाटवं वाचि सुगूढमास्ते ।\\
कस्मिंश्चिदुक्ते तु पदं कथंचित् स्थैर्यं वपुः स्विद्यति वेपते च ॥
\end{verse}
Grammar is supreme medicine. One who doesn’t take this medicine will suffer while talking or speaking. When he starts to talk then he is sweating out and shivering.

Speaking is a skill. Everyone does not have this quality. Somebody has this by birth somebody by experience. But my uncle has both. I liked his fluency on talk, knowledge on subject and grip on language. He is a great speaker. He can talk any topic without any preparation. He speaks limited but meaningful. He is aware about what he is talking. He always starts with introduction, continue with explanation and finish with conclusion without any diversion of the subject. He speaks according to the situation and need. The listeners never feel that he is talkative.

\section*{03-FEARLESSNESS (निर्भयत्वम्):}
\begin{verse}
सुवर्णपुष्पां पृथिवीं चिन्वन्ति पुरुषास्त्रयः ।\\
शूरश्च कृतविद्यश्च यश्च जानाति सेवितुम् ॥
\end{verse}
Brave man, well educated and efficient in social service- these three people pluck the golden flowers from the creepers in land.
He is brave himself and make others fearless. In Sanskrit college whatever problems come to him asking that to solve, once he committed for their invocation, he is ready to face any situation without fear. I heard that he fought for the sake of the students and teachers against the education department. And he got victory for his honest fight. “धैर्यं सर्वत्र साधनम्” braveness is instrument everywhere.

\section*{04-LAZILESSNESS ( अजाढ्यम् ):}
\begin{verse}
आलस्यं हि मनुष्याणां शरीरस्थो महान् रिपुः ।\\
नास्त्युद्यमसमो बन्धुः कुर्वाणो नावसीदति ॥
\end{verse}
Laziness is the supreme enemy of men possessed of limbs; there could be no other kinsman that is equivalent to effort. The active person doesn’t suffer at all.

He is always enthusiast. His enthusiasm influences on their students and people around him. I used to skip many classes while studying but there is no history that he skipped the classes. He used to teach in the class for long time and afterwards in the house continuously. When he is free he used that whole time for reading different books. I never saw him tired. He has more spirit than students.

\section*{05-LEADERSHIP (नायकत्वम्):}
\begin{verse}
त्यागी कृती कुलीनः सुश्रीको रूपयौवनोत्साही ।\\
दक्षोऽनुरक्तलोकस्तेजोवैधग्ध्यशीलोवान्नेता ॥
\end{verse}
\textbf{one who Sacrifice, active, noble, decent, young, enthusiast, efficient, deeply interested in people and bright person is called as leader.}
Intentionally he never wants to become leader. But the life how he leads, people agreed him as leader. I saw him for 5 years in Sanskrit college that he managed many state and national level functions as main role. Generally scholars don’t want to become leaders, they always one step back for leadership. But when leadership comes to my uncle’s shoulder, he agreed and proved his ability. He is model for all leaders.

\section*{06-CO-ORDINATION (व्यवस्थापकत्वम्):}
\begin{verse}
अमन्त्रमक्षरं नास्ति नास्ति मूलमनौषधम् ।\\
अयोग्यः पुरुषो नास्ति योजकस्तत्र दुर्लभः ॥
\end{verse}
There is no letter (syllable) which is meaningless (hymn), there is no plant (shrub) which is not medicine; there is none who is useless (unfit), only thing is that one has to fix it properly.

For every organization there is an organizer like that for our organization he is the organizer or coordinator. Functions which were held at Sanskrit college, almost all were done by his co-ordination itself. One who helps for organizing the functions, one who suggests the students in any condition is none other than our teacher and my uncle.
He helped lot to ‘Pradosha Sangha’ which is managed by the students themselves. Whenever students invite him to presiding to function, no history that he said no to students. So students or teachers whoever come to Sanskrit college, they all depend him one or another way for his co-ordination.

\section*{07-SERIOUSNESS (गम्भीरता):}

This quality makes the man perfect and describes character. I have been noticing this quality since my childhood. He remains same in the house as my uncle and in the college as my teacher. By his seriousness he can control us in the house and students in college. His one look changes the silly behavior of others. That doesn’t mean that he doesn’t know how to smile. He knows where to smile, where to be serious. He is aware about his behavior. All scholars may have this quality but my uncle becomes child with children, elder with elders, scholar with knowledge people and common man with common people.

\section*{08-CLARIFICATION (स्फुटत्वम्):}
\begin{verse}
प्रियवाक्यप्रदानेन सर्वे तुष्यन्ति जन्तवः।\\
तस्मात्तदेव वक्तव्यं वचने का दरिद्रता ॥
\end{verse}
With agreeable words all beings are pleased. Hence that alone should be spoken, why should one scare in the matter of words.
Whatever questions asked by the students, they are answered without any doubts. He can clarify the doubts at the spot where students ask the questions in single movement without diversion. The clarity in his voice, style he answers and subject he presents give the complete and thorough knowledge to the students more than they expect. Because many teacher answers with doubts. So usually students don’t satisfy with the answers and keep on questioning the teacher. But when we hear his teachings there are no doubts. Hence when there are no doubts no questions arise.

\section*{09-STATE OF SPIRITUAL PASSIVITY (निष्कामकर्मत्वम्):}
\begin{verse}
कर्मण्येवाधिकारस्ते मा फलेषु कदाचन ।\\
मा कर्मफलहेतुर्भूर्मा ते सङ्गोऽस्त्वकर्मणि ॥
\end{verse}
You have right to do actions, but never to the fruits (of works); Don’t be impelled by the fruits of works (at the same time), don’t be tempted to withdraw from works.

He taught many students. He teaches many students. He will teach many students. He always keeps on teaching. But he never expects the fees from them. Nowadays it is very rare. Many teachers want to earn more money so they take separate classes other than the school hours and publish many books which help students to pass the exam easily. But my teacher never taught us for the examination. He gave us thorough knowledge about the subject. He expects only student’s intellect. He helped many people instead of teaching, but he never expects fruits. But fruit comes to him. Where there is no expectation, there is fruit. If you expect or don’t expect all should eat their fruit of action.

\section*{10-TEACHING METHODOLOGY (बोधना-विधानम्):}
\begin{verse}
माधुर्यमक्षरव्यक्तिः पदच्छेदस्तु सुस्वरः ।\\
धैर्यं लयसमर्थं च षडेते पाठकागुणाः ॥
\end{verse}
Sweetness in sound, clarity in pronunciation, proper splitting of words, good voice, dare to teach, rhythmic tone -  these are the six qualities of great teacher.

My teacher was telling in the class about the method of teaching. So I tried here to explain his words. Teaching is a skill. It has two things. One is who should be taught? Second one is what should be taught? Education means the teacher, the student and the subject. Here student compared to land, subject compared to seed and student’s knowledge compared to crop. Execution of these three things decides the success or victory of the teacher. We can write a separate book, do a PhD on his teaching method. Even education department have many things to learn from him. The problem of the teachers is to teach according to different intellect level of the students. Just think if there are many students in the class. Some of them understand early, some are slowly and some are very slowly. So teachers should teach according to their intellect level.

Our teacher never gets bored to teach. Because I understand very slowly and I come under third level. So he used to teach many times the same subject until our involvement. But finally he made us understand. He never taught us for the examination. He gave us thorough knowledge of the subject. Many times he taught one line for whole month to make us understand properly.
I explained here top ten qualities. That doesn’t mean that he has only these much qualities. Like this there are many qualities in him. I can’t explain all about here. I copied many of his qualities and followed and succeed. I stayed in his house about 8 years and I learned lot such as speaking, writing, behaving, typing, cooking and etcetera. What I wrote today in English language, this is all because of him and his influence. I convey my gratitude to my uncle and aunty for bearing me for 8 years. Because I behaved silly many times. I never felt hungry during my residence with them. I gave pain to them sometime. So please forgive me. I pray to god that his transparency, honesty and efficiency will protect him throughout his life. I’m trying to apply many qualities in my life like way he treating the people, helping the people, etc.
