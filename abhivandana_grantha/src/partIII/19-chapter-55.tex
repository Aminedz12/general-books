\chapter{ನಯ ವಿನಯ ಸಂಪನ್ನ  ವಿದ್ವಾನ್ ಶ್ರೀ ಗಂಗಾಧರ ಭಟ್}

\begin{center}
\Authorline{ಡಾ|| ಎಂ.ಎ. ಜಯಶ್ರೀ}
\smallskip

ನಿವೃತ್ತ ಅಧ್ಯಾಪಕರು\\
ಡಿ.ಬನುಮಯ್ಯ ಕಾಲೇಜ್\\
ಮೈಸೂರು
\end{center}

ನಾನು ಸಂಸ್ಕೃತ ಪ್ರಾಧ್ಯಾಪಕಿಯಾಗಿ ಬನುಮಯ್ಯ ಕಾಲೇಜಿನ ಪದವಿ ತರಗತಿಯಲ್ಲಿ ಪಾಠ ಮಾಡುತ್ತಿದ್ದಾಗ ಶ್ರೀ ಗಂಗಾಧರ ಭಟ್ಟರು ಅಲ್ಲಿಯ ವಿದ್ಯಾರ್ಥಿ. ಅವರ ನಯ ವಿನಯ ಎಷ್ಟಿತ್ತೆಂದರೆ ಹಲವು ತಿಂಗಳಕಾಲ ಅವರು ಸಂಸ್ಕೃತ ವಿದ್ವಾಂಸರೆಂದೂ, ಮಹಾರಾಜ ಸಂಸ್ಕೃತ ಪಾಠಶಾಲೆಯ ಪ್ರಾಧ್ಯಾಪಕರೆಂದು ನನಗೆ ತಿಳಿದೇ ಇರಲಿಲ್ಲ. ಆಕಸ್ಮಿಕವಾಗಿ ನನಗೆ ಅವರ ವಿದ್ವತ್ತು ತಿಳಿದ ಕೂಡಲೇ ಅವರನ್ನು ನನಗೆ ಸಂಸ್ಕೃತ ವ್ಯಾಕರಣದ ಕ್ಲಿಷ್ಟ ಭಾಗವನ್ನ ಮನೆಗೆ ಬಂದು ಪಾಠ ಹೇಳುವಂತೆ ಪ್ರಾರ್ಥಿಸಿಕೊಂಡೆ ಅವರು ಅಷ್ಟೇ ಸೌಜನ್ಯದಿಂದ ನನಗೆ ಹಲವಾರು ವಿಷಯಗಳನ್ನು ಕಲಿಸಿಕೊಟ್ಟರು.

ಸಂಸ್ಕೃತ ಮಹಾರಾಜ ಪಾಠಶಾಲೆಯಲ್ಲಿ ಪ್ರಾಧ್ಯಾಪಕರಾದ ಇವರು ಅವರ ಶಿಷ್ಯರಿಗೆ ಅಚ್ಚುಮೆಚ್ಚಿನ ಗುರು. ಒಬ್ಬರಿರಲಿ, ಇಬ್ಬರಿರಲಿ ಒಂದೇ ರೀತಿಯಲ್ಲಿ ಸರಳವಾಗಿ ಅವರ ಮಟ್ಟಕ್ಕಿಳಿದು ವಿದ್ಯಾದಾನ ಮಾಡುವ ವಿಶೇಷ ಶಕ್ತಿ ಗಂಗಾಧರ ಭಟ್ಟರದು. ಇವರು ಶಾಲೆಯಲ್ಲಷ್ಟೇ ಅಲ್ಲದೆ ಅವರ ಮನೆಯನ್ನೂ ಛಾತ್ರಾವಾಸವಾಗಿ ಮಾಡಿ ಹಲವಾರು ಶಿಷ್ಯರಿಗೆ ವಿದ್ವತ್ ಪದವಿಯನ್ನು ಪಡೆಯುವಂತೆ ಮಾಡಿದ್ದಾರೆ. ಇವರ ಪತ್ನಿಯೂ ಅಷ್ಟೆ. ಹಸನ್ಮುಖಿ, ಹಾಸ್ಯಪ್ರಿಯೆ ಹಾಗೆಯೇ ಮನೆಯಲ್ಲಿಯೇ ಇದ್ದು ಅಧ್ಯಯನ ಮಾಡುತ್ತಿರುವ ಹುಡುಗರಿಗೆಲ್ಲಾ ತಾಯಿ.

ಈ ದಂಪತಿಗಳೊಂದಿಗೆ ಎರಡು ಸಾರಿಯಾದರೂ ಒಟ್ಟಿಗೆ ಹಲವಾರು ದಿನಗಳ ಕಾಲ ಹಾಂಗ್‍ಕಾಂಗಿನ ಯೋಗ ಕಾನ್‍ಫೆರೆನ್ಸ್ ಸಮಯದಲ್ಲಿ ಇದ್ದದ್ದು ನೆನಪಿನಲ್ಲಿ ಅಚ್ಚಳಿಯದೆ ಇದೆ. ಗಂಗಾಧರ ಭಟ್ಟರು ಕೇವಲ ನ್ಯಾಯ, ಮೀಮಾಂಸಾ ಶಾಸ್ತ್ರಗಳಿಗೆ ತಮ್ಮನ್ನು ತೊಡಗಿಸಿಕೊಳ್ಳದೆ ಸಂಸ್ಕೃತಸಾಹಿತ್ಯದ ವಿವಿಧ ಪ್ರಕಾರಗಳಲ್ಲಿಯೂ ಪಾಂಡಿತ್ಯ ಪಡೆದವರಾಗಿದ್ದಾರೆ. ಆಯುರ್ವೇದಿಕ್ ಕಾಲೇಜಿನಲ್ಲಿ ಅವರು ಕೊಟ್ಟ ಉಪನ್ಯಾಸಗಳನ್ನು ಆಯುರ್ವೇದಿಕ್ ಕಾಲೇಜಿನ ಉಪಾಧ್ಯಾಯವರ್ಗ ಹಾಗೂ ಅಲ್ಲಿನ ವಿದ್ಯಾರ್ಥಿವರ್ಗ ಇಂದಿಗೂ ನೆನಸಿಕೊಳ್ಳುತ್ತಿದ್ದಾರೆ.

ಗಂಗಾಧರಭಟ್ಟರು ಮೈಸೂರಿನ ಆಕಾಶವಾಣಿಯ ಮೂಲಕ ಸಂಸ್ಕೃತ ವಾಙ್ಮಯಕ್ಕೆ ಸೇರಿದ ವಿಷಯಗಳ ಮೇಲೆ ಹಲವಾರು ಕಾರ್ಯಕ್ರಮಗಳನ್ನು ಬಿತ್ತರಿಸಿದ್ದಾರೆ. ಈ ರೀತಿಯ ಕಾರ್ಯಕ್ರಮಗಳಲ್ಲಿ ನಾನು ಪಾಲ್ಗೊಂಡಿದ್ದು, ಅವುಗಳ ವೈಶಿಷ್ಟ್ಯ ಹಾಗೂ ಅವುಗಳ ರಚನೆ ಮತ್ತು ಪ್ರಸಾರದಲ್ಲಿ ಗಂಗಾಧರ ಭಟ್ಟರಿಗೆ ಇದ್ದ ವೈದುಷ್ಯಕ್ಕೆ ನಾನು ಮಾರು ಹೋಗಿದ್ದೇನೆ.

ಗಂಗಾಧರಭಟ್ಟರು ತಮ್ಮನ್ನು ನನ್ನ ಶಿಷ್ಯ ಎಂದು ಹೇಳಿದರೂ ಅದು ಕೇವಲ ಕಾಲೇಜಿನ ಒಂದು ಸಂಪ್ರದಾಯಕ್ಕೆ ಸೇರಿದ್ದೇ ಹೊರತು, ಅವರಿಗೆ ಶಿಷ್ಯೆಯಾಗಿ ನಾನು ಕಲಿತದ್ದು ಬಹಳ ಹೆಚ್ಚು. ಅವರು ಮತ್ತೆಮತ್ತೆ ನನ್ನ ಶಿಷ್ಯರೆಂದು ಎಲ್ಲರೊಂದಿಗೆ ಹೇಳಿಕೊಂಡಾಗಲೆಲ್ಲಾ ನನಗೆ ಸ್ವಲ್ಪ ನಾಚಿಕೆಯಾಗುತ್ತದೆ. ಆದರೆ ಅವರ ಸೌಜನ್ಯ, ವಿನಯದ ದ್ಯೋತಕವಾದ ಈ ಭಾವನೆಯನ್ನು ಕಂಡು ನಾನು ಅವರಿಗೆ ನನ್ನ ಧನ್ಯತಾ ಪೂರ್ವಕವಾದ ನಮನವನ್ನು ಮಾಡುತ್ತೇನೆ.

ನನ್ನ ಅಣ್ಣ ಎಂ.ಎ. ನರಸಿಂಹನ್ ಅವರು ಹೇಳುವಂತೆ, “ಶ್ರೀ ಗಂಗಾಧರ ಭಟ್ಟರು ಸೌಜನ್ಯದ ಸಾಕಾರ ಮೂರ್ತಿ. ಅವರ ಸಂಸ್ಕೃತದ ಸೇವೆಯ ವ್ಯಾಪ್ತಿ ಅರಿತವರಿಗೇ ಗೊತ್ತು. ಇವರು ಸಂಸ್ಕೃತದ ಏಕೈಕ ದಿನ ಪತ್ರಿಕೆ ಸುಧರ್ಮಾಗೆ ಒಂದು ಪ್ರಮುಖ ಆಧಾರ ಸ್ತಂಭ. ಅದರಲ್ಲಿ ಇವರು ಬರೆಯುವ ರೀತಿಯನ್ನು ನೋಡಿದರೆ ನಮ್ಮ ದೇಶದ ಪ್ರಸಿದ್ಧ ಪತ್ರಿಕಾಕರ್ತರು, ಸಂಪಾದಕರುಗಳಿಗೆ ಇವರು ಹೆಗಲೆಣೆಯಾಗಿ ನಿಲ್ಲಬಲ್ಲರು. ನಾವು ನಮ್ಮ ಅಧ್ಯಯನದಲ್ಲಿ ಸಂದೇಹಗಳು ಬಂದಾಗಲೆಲ್ಲ ಶ್ರೀ ಭಟ್ಟರೇ ನಮ್ಮ ವಿಶ್ವಕೋಶ. ಯಾವುದೇ ವಿಷಯವಾಗಿರಲಿ, ತಮಗೆ ಗೊತ್ತಿರುವುದನ್ನು ತಿಳಿಸುವುದಲ್ಲದೇ ಅಕರಗ್ರಂಥಗಳನ್ನು ನೋಡಿ ನಮ್ಮ ಸಂದೇಹವನ್ನು ಪರಿಹರಿಸುವ ಅವರ ವಿಶಾಲ ಮನಸ್ಸು ನಮ್ಮನ್ನು ವಿಸ್ಮಿತರನ್ನಾಗಿ ಮಾಡುತ್ತದೆ. ಅಧ್ಯಯನ, ಅಧ್ಯಾಪನ, ಸಂಶೋಧನ ಪ್ರಕಾರಗಳಲ್ಲಿ ಇವರು ನಮಗೆ ಮಾದರಿ. ಇವರಿಂದ ಇನ್ನೂ ಮಹತ್ತಾದ ಕಾರ್ಯಗಳಾಗಿ ಇವರ ವಿದ್ವತ್ ಪ್ರತಿಭೆಯ ಪರಿಚಯ ವಿಶ್ವಕ್ಕೆ ಆಗಲಿ ಎಂಬುದೇ ನಮ್ಮ ಆಶಯ.” 

ಭಟ್ಟರು ಇಂದು ತಮ್ಮ ಪ್ರಾಧ್ಯಾಪಕ ವೃತ್ತಿಯಿಂದ ನಿವೃತ್ತರಾಗುತ್ತಿದ್ದಾರೆ. ನೌಕರಿಯ ಶೃಂಖಲೆಯಿಂದ ಬಿಡುಗಡೆಯಾಗುವ ಅವರು ಇನ್ನು ಸ್ವತಂತ್ರವಾಗಿ ತಮ್ಮ ಪಾಂಡಿತ್ಯದ ಪ್ರಕಾಶವನ್ನು ಮುಂದಿನ ಪೀಳಿಗೆ ಹಾಗೂ ನಮ್ಮೆಲ್ಲರೊಂದಿಗೆ ಹಂಚಿಕೊಂಡು ಬರಹ, ಉಪನ್ಯಾಸಗಳ ಮೂಲಕ ಇನ್ನೂ ಹಲವಾರು ದಶಕಗಳು ಪ್ರಕಾಶಿಸುತ್ತಿರಲಿ ಎಂಬ ಹಾರೈಕೆಯೊಂದಿಗೆ ಈ ಲೇಖನವನ್ನು ಸಮಾಪ್ತಿಗೊಳಿಸುತ್ತಿದ್ದೇನೆ.
