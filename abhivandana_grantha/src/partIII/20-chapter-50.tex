{\fontsize{14}{16}\selectfont
\chapter{ನಮ್ಮ ಪುರೋಹಿತರು   \enginline{-}   ನಮ್ಮ ಹೆಮ್ಮೆ}

\begin{center}
\Authorline{ವಿ॥ಚಂದ್ರಶೇಖರ ಭಟ್ಟ }
\smallskip

ಗಾಳಿಮನೆ,\\ 
ಸಿದ್ದಾಪುರ	
\addrule
\end{center}

\textbf{ಸ್ವದೇಶೇ ಪೂಜ್ಯತೇ ರಾಜಾ ವಿದ್ವಾನ್ ಸರ್ವತ್ರ ಪೂಜ್ಯತೇ} ಎಂಬ ಮಾತು ಪ್ರಸಿದ್ಧ\-ವಾಗಿದೆ. ಈ ಮಾತಿಗನುಗುಣವಾಗಿ ವಿದ್ವಾಂಸರಾದ ಗಂಗಾಧರ ಭಟ್ಟರು ತಮ್ಮ ಅಧ್ಯಯನ\enginline{-}ಅಧ್ಯಾಪನಗಳಿಂದ ಸಾರ್ವದೇಶಿಕವಾಗಿ, ಸಾರ್ವಕಾಲಿಕವಾಗಿ ಪೂಜನೀಯರು. ಇದೀಗ ಅವರ ಶಿಷ್ಯರೆಲ್ಲರೂ ಸೇರಿ ಅವರಿಗೆ ಗುರುವಂದನೆ ಸಲ್ಲಿಸುತ್ತಿರುವುದು, ರಾಜ್ಯ\-ಸ್ತರದಲ್ಲಿ ಗೌರವಿಸುತ್ತಿರುವುದು ಅತ್ಯಂತ ಸಂತಸದ ಮತ್ತು ಸ್ತುತ್ಯರ್ಹಸಂಗತಿಯಾಗಿದೆ. 

ಗಂಗಾಧರ ಎಂಬ ಶ್ರೀಯುತರ ನಾಮಧೇಯ ಸಾರ್ಥಕವಾಗಿದೆ ಎಂಬುದು ನನ್ನ ಭಾವನೆ. ಭಗೀರಥನ ಸಂಪ್ರಾರ್ಥನೆಯನ್ನು ಅನುಸರಿಸಿ ಈಶ್ವರನು ಗಂಗೆಯನ್ನು ಧರಿಸಿದ್ದು, ನಂತರ  ದೇವಗಂಗೆ ಭೂಗಂಗೆಯೂ, ಪಾತಾಳಗಂಗೆಯೂ ಆಗಿ ಲೋಕಪಾವನೆ\-ಯಾಗಿದ್ದು ಪುರಾಣಪ್ರಸಿದ್ಧವೃತ್ತಾಂತವಾಗಿದೆ. ಇದರಂತೆ ಗಂಗಾಧರಭಟ್ಟರೂ ಕೂಡ ಸಂಸ್ಕೃತ  \enginline{-}  ಶಾಸ್ತ್ರ ಪರಂಪರೆಯನ್ನು ಸಮುದ್ಧರಣಗೊಳಿಸುವ ಕೈಂಕರ್ಯದಲ್ಲಿ ಶಾಸ್ತ್ರಗಳೆಂಬ ಗಂಗೆಯನ್ನು ಧರಿಸಿ ತಮ್ಮ ತಾಯಿ\enginline{-}ತಂದೆಯರು ಕೊಡಮಾಡಿದ್ದ ಗಂಗಾಧರ ಎಂಬ ಹೆಸರನ್ನು ಅನ್ವರ್ಥವಾಗಿಸಿಕೊಂಡ ಮಹನೀಯರು. ಹಾಗೂ ಅವರಲ್ಲಿ ಛಾತ್ರವೃತ್ತಿ ಕೈಗೊಂಡ ಶಿಷ್ಯಗಣದ ಉದ್ಧಾರವನ್ನೂ ಮಾಡಿದವರು. ಹೀಗೆ ಉಭಯಥಾಪಿ ನಾಮ\-ಸಾರ್ಥಕ್ಯ ವನ್ನು ಗ್ರಹಿಸಬಹುದು. 

\section*{ನಿಷ್ಕಾಮಕರ್ಮಯೋಗೀ} 

ನಾನು ತಿಳಿದಂತೆ ಶ್ರೀಯುತರ ಸುದೀರ್ಘವಾದ ಅಧ್ಯಾಪನ ವೃತ್ತಿಯಲ್ಲಿ ಅಸಂಖ್ಯಾತ\-ವಿದ್ಯಾರ್ಥಿಗಳಿಗೆ ಮಾರ್ಗನಿರ್ದೇಶನ ಮಾಡಿದ್ದಾರೆ. ಹೆಚ್ಚೇಕೆ, ಬಹಳಷ್ಟು ವಿದ್ಯಾರ್ಥಿ\-ಗಳಿಗೆ ವಿದ್ಯಾರ್ಜನೆಗೆ ಅಶನ\enginline{-}ವಸನಾದಿ ಸಕಲವ್ಯವಸ್ಥೆ ಮಾಡಿದ್ದಾರೆ. ಮೈಸೂರಿಗೆ ಓದಲು ಹೋಗುವ ನಮ್ಮಲ್ಲಿಯ(ಉತ್ತರಕನ್ನಡದ, ಅದರಲ್ಲೂ ಸಿದ್ದಾಪುರ,ಸಿರಸಿ,ಸಾಗರ,ಯಲ್ಲಾ\-ಪುರದ) ಸಕಲವಿದ್ಯಾರ್ಥಿಗಳೂ ಒಂದಲ್ಲಾ, ಒಂದು ತೆರನಾಗಿ ಗಂಗಾಧರ ಭಟ್ಟರಿಂದ ಸಹಾಯ\enginline{-}ಸಹಕಾರ ಪಡೆದವರೇ ಆಗಿದ್ದಾರೆ. ಇದರಲ್ಲಿ ಯಾವ ಅತಿಶಯವೂ ಇಲ್ಲ. ಪಾಠಶಾಲೆಯಲ್ಲಿ ಓದುವವರಿಗಂತೂ ಗಂಗಾಧರ ಭಟ್ಟರು \eng{godfather} ಇದ್ದಂತೆ.  

ಅವರ ಮಾರ್ಗದರ್ಶನದಲ್ಲಿ ಅನೇಕ ವಿದ್ಯಾರ್ಥಿಗಳು ಶಾಸ್ತ್ರಪರಂಪರೆಯ ಅಧ್ಯಯನ ನಡೆಸಿದ್ದಾರೆ. ಅವರು ತರ್ಕಶಾಸ್ತ್ರದ ಪ್ರಾಧ್ಯಾಪಕರಾಗಿದ್ದರೂ ಸಹ ಸಕಲ\-ಶಾಸ್ತ್ರ\-ಗಳನ್ನು ಅಸಂದಿಗ್ಧವಾಗಿ  ಪಾಠಮಾಡಬಲ್ಲವರು. ಹೀಗೆ ಪಾಠಮಾಡಿ ಬಹಳಷ್ಟು ವಿದ್ಯಾರ್ಥಿ\-ಗಳು ವಿದ್ವಾನ್ ಎಂಬ ಪದವಿ ಪಡೆಯುವಲ್ಲಿ ಶ್ರೀಯುತರ ಕೊಡುಗೆ ಅದ್ವಿತೀಯವಾಗಿದೆ. ಇಂದು ನನ್ನ ಮಗನೂ ಸೇರಿದಂತೆ ಅವರಲ್ಲಿ ವಿದ್ಯಾರ್ಜನೆ ಮಾಡಿದ ಬಹಳಷ್ಟು ವಿದ್ಯಾರ್ಥಿಗಳು ಉಪನ್ಯಾಸಕರಾಗಿ,  ರಾಜ್ಯ ಮತ್ತು ರಾಷ್ಟ್ರದ ಶ್ರೇಷ್ಠವಿದ್ಯಾಸಂಸ್ಥೆಗಳಲ್ಲಿ ಸೇವೆ ಸಲ್ಲಿಸುತ್ತಿದ್ದಾರೆ. ಇದಕ್ಕೆ ಕಾರಣ ಅಧ್ಯಾಪನ, ಅಶನ\enginline{-}ವಸನಾದಿವ್ಯವಸ್ಥೆಗಳೆಲ್ಲವನ್ನೂ ಯಾವುದೇ ಪ್ರತಿಫಲಾಪೇಕ್ಷೆ ಇಲ್ಲದೇ ಮಾಡಿದವರು, ಮಾಡುತ್ತಿರುವವರು ಎಂಬುದಾಗಿದೆ. 

ನಮ್ಮ ಶಾಸ್ತ್ರಗಳು ಸಾರಿ\enginline{-}ಸಾರಿ ಹೇಳುವ \textbf{ನಿಷ್ಕಾಮಕರ್ಮದ} ಆಚರಣೆಯನ್ನು ನಿತ್ಯ\enginline{-}ನಿರಂತರವಾಗಿ ಮಾಡಿಕೊಂಡು ಬಂದಿದ್ದಾರೆ. ಒಂದರ್ಥದಲ್ಲಿ ಎಲ್ಲವನ್ನೂ ಮಾಡಿಯೂ ತಾನೇನೂ ಮಾಡಿಲ್ಲ ಎಂಬ \textbf{ಪದ್ಮಪತ್ರಮಿವಾಂಭಸಾ} ಎಂಬ ತೆರದಲ್ಲಿ ಜೀವನಸಾಗಿಸುತ್ತಾ ಮುಂದುವರಿದವರು. ಹೀಗೆ \textbf{ನಿಷ್ಕಾಮಕರ್ಮಯೋಗಿಯಾಗಿ} ನೆಲೆಗೊಂಡಿರುವ ವಿದ್ವಾನ್॥ಗಂಗಾಧರ ಭಟ್ಟರ ಜೀವನ ಅನುಸರಣೀಯ ಎಂಬಲ್ಲಿ ಎರಡು ಮಾತಿಲ್ಲ.

\section*{ವಿದ್ವಜ್ಜನರ ಕುಟುಂಬ   \enginline{-}   ನಮ್ಮ ಪುರೋಹಿತರು}

ಸಂಬಂಧದಲ್ಲಿ ಬಹಳಷ್ಟು ವಿಧಗಳಿವೆ. ರಕ್ತಸಂಬಂಧದಷ್ಟೆ ಗುರು\enginline{-}ಶಿಷ್ಯಸಂಬಂಧಕ್ಕೂ ಲೋಕದಲ್ಲಿ ಗೌರವಾದರಗಳು, ಪ್ರಾಮುಖ್ಯವೂ ಇದೆ. ಈ ದೃಷ್ಟಿಯಿಂದ ಗಾಳಿಮನೆ  \enginline{-}  ಅಗ್ಗೆರೆಯ ಸಂಬಂಧ ಬಹಳ ಮುಖ್ಯವಾದದ್ದು.  ನಮಗೆ ಅಗ್ಗೆರೆಯ ಭಟ್ಟರ ಕುಟುಂಬ\-ದವರು ಪುರೋಹಿತರಾದರೆ, ಅವರಿಗೆ ನಾವು ಪುರೋಹಿತರು. ಈ ದೃಷ್ಟಿಯಲ್ಲಿ ನಾವಂತೂ ಧನ್ಯಾತ್ಮರು\enginline{-}ಪುಣ್ಯಾತ್ಮರೂ ಹೌದು. ಏಕೆಂದರೆ ನಮ್ಮ ಪುರೋಹಿತರು\break ಶಾಸ್ತ್ರಜ್ಞರೂ\enginline{-}ವೈದಿಕರೂ ಆಗಿರುವವರು. ಇಂತವರನ್ನು ನಮ್ಮ ಪುರೋಹಿತರೆಂದು\break ಹೇಳಲು ಬಹಳಷ್ಟು ಹೆಮ್ಮೆ ಎನಿಸುತ್ತದೆ. ಅದೇ ರೀತಿಯಲ್ಲಿ \textbf{ಸಂಸ್ಕಾರವಂತ  \enginline{-}   ಅಗ್ಗೆರೆ ಮನೆತನಕ್ಕೆ} ನಾವು ಪುರೋಹಿತರೆಂಬ ಧನ್ಯತೆಯೂ ನಮ್ಮ ಮನೆತನಕ್ಕಿದೆ. \textbf{ಪರಸ್ಪರಂ ಭಾವಯಂತ: ಶ್ರೇಯ: ಪರಮವಾಪ್ಸ್ಯಥ}  ಎಂಬಂತೆ ಎರಡೂ ಮನೆತನಗಳೂ ಗೌರವಾದರಕ್ಕೆ ಪಾತ್ರರಾದ ತೃಪ್ತಿ ಇದೆ. 

ವೈದಿಕರಾಗಿ, ಶಾಸ್ತ್ರಜ್ಞರಾಗಿ ಉಭಯಥಾಪಿ ವಿದ್ವಾಂಸರಾಗಿರುವ ವಿದ್ವಾನ್ ಗಂಗಾಧರ ಭಟ್ಟರು ನಮ್ಮ ಪುರೋಹಿತರು. ಹೌದು, ಇದು ನಮ್ಮ ಗಾಳಿಮನೆಯವರೆಲ್ಲರ ಪಾಲಿಗೊಂದು ಅತ್ಯಂತ ಹೆಮ್ಮೆಯ ಸಂಗತಿ. ವೈದಿಕರಾಗಿ ಅವರು ನಮ್ಮ ಮನೆಯ ಅನೇಕ ಧಾರ್ಮಿಕ ಕಾರ್ಯಕ್ರಮಗಳಿಗೆ ಆಗಮಿಸಿ ನಮ್ಮನ್ನೆಲ್ಲ ಆಶೀರ್ವದಿಸಿದ್ದಾರೆ. ಶಾಸ್ತ್ರಪೂತವಾದ ಮಾತುಗಳಿಂದ ನಮಗೆಲ್ಲಾ ಒಳಿತನ್ನೆ ಹರಸಿದ್ದಾರೆ. ವಕ್ತಾ ದಶಸಹಸ್ರೇಷು ಎಂಬಂತೆ ಶ್ರೀಯುತರು ಅಪರೂಪದ, ಉತ್ತಮವಾಗ್ಮಿಗಳು. ಸಂಸ್ಕೃತ, ಕನ್ನಡ, ಇಂಗ್ಲೀಶ್, ಹಿಂದಿ ಹೀಗೆ ನಾಲ್ಕಾರು ಭಾಷೆಗಳಲ್ಲಿ ನಿರರ್ಗಳವಾಗಿ ಮಾತಡಬಲ್ಲರು, ಪಾಠ \enginline{-}ಪ್ರವಚನ ಮಾಡಬಲ್ಲರು. ಶ್ರೀಯುತರಲ್ಲಿ ದೇಶ\enginline{-}ವಿದೇಶಗಳ ಎಷ್ಟೋ ವಿದ್ಯಾರ್ಥಿಗಳು ವಿದ್ಯಾರ್ಜನೆ ಮಾಡಿದ್ದಾರೆ. ಹೀಗೆ ವಿದ್ಯಾದಾನಮಾಡಿ, ಹಲವರಿಗೆ ಆರ್ಥಿಕವಾಗಿಯೂ ಸಹಾಯಮಾಡಿ ದಾನೀ ಭವತಿ ವಾ ನ ವಾ,  ಎಂಬ ಮಾತಿನ ಆಶಯದಂತೆ ವಿದ್ಯಾ  \enginline{-}  ಅರ್ಥದಾನಗಳಿಂದ ದಾನದಲ್ಲೂ ಅಗ್ರಪಂಕ್ತಿಯಲ್ಲಿ ಕಂಗೊಳಿಸುವಂತಹ ಮಹನೀಯರು. 

ವಿದ್ವಾನ್॥ ಗಂಗಾಧರ ಭಟ್ಟರ ದೊಡ್ಡಪ್ಪ ಕೀರ್ತಿಶೇಷ॥ಮಹಾಬಲೇಶ್ವರ ಭಟ್ಟರು ಬಹುಶ್ರುತವಿದ್ವಾಂಸರಾಗಿ ಅಗ್ಗೆರೆಮನೆತನದ ಹೆಸರನ್ನು ಚಿರಸ್ಥಾಯಿಗೊಳಿಸಿದಂತವರು. ಹಾಗೆಯೇ, ಸಹೋದರರಾಗಿದ್ದ ವಿದ್ವಾನ್॥ ಮಂಜುನಾಥ ಭಟ್ಟರು ಸಹ ಅದ್ವೈತ ವೇದಾಂತವನ್ನು ಅಧ್ಯಯನ ಮಾಡಿದ್ದವರು, ಪುರೋಹಿತರಾಗಿ ಗೌರವಾರ್ಹರಾಗಿ\-ದ್ದವರು. ಹಾಗೆಯೇ ಶ್ರೀಯುತರ ತಂದೆಯವರು ಕೀರ್ತಿಶೇಷ ವಿಘ್ನೇಶ್ವರ ಭಟ್ಟರು ವೈದಿಕ\-ರಾಗಿದ್ದವರು. ಹೀಗೆ ಇಡೀಮನೆತನವೇ ಶಾಸ್ತ್ರಪರಂಪರೆಯಿಂದ ಲೋಕಹಿತದಲ್ಲಿ ತೊಡಗಿಸಿಕೊಂಡಿದ್ದು ಸರ್ವರಿಗೂ ವಿದಿತವಾದ ಸಂಗತಿಯಾಗಿದೆ. 

\textbf{ಸಙ್ಘೇಶಕ್ತಿ: ಕಲೌ ಯುಗೇ} ಎಂಬಂತೆ ವಿದ್ವಾನ್ ಗಂಗಾಧರ ಭಟ್ಟರು ಓರ್ವ ಶ್ರೇಷ್ಠ ಸಂಘಟಕರು ಹೌದು. ಮಹಾರಾಜ ಸಂಸ್ಕೃತ ಕಾಲೇಜಿನಲ್ಲಿ ಅವರ ಯಶಸ್ವೀ ಸಂಘಟನೆಗಳಿಗೆ ಬಹಳಷ್ಟು ದೃಷ್ಟಾಂತಗಳಿವೆ. 

\section*{ಆದರ್ಶದಂಪತೀ}
\vskip -6pt

ಶ್ರೀಯುತರ ಅರ್ಧಾಂಗಿಯಾಗಿ ಜೀವನದ ಸುಖ\enginline{-}ದುಃಖಗಳಲ್ಲಿ ಪಾಲ್ಗೊಳ್ಳುತ್ತಿರುವ ಶ್ರೀಮತಿ ಶೈಲಾರವರು ಅಭಿನಂದನೀಯರು. ಏಕೆಂದರೆ ವ್ಯಕ್ತಿಯೋರ್ವ ಸಮಷ್ಟಿಯಲ್ಲಿ ಲೋಕಹಿತವನ್ನು ಬಯಸುವಾಗ ಮನೆಯವರ ಸಂಪೂರ್ಣ ಸಹಾಯ\enginline{-}ಸಹಕಾರ ಅತ್ಯಂತ ಅಗತ್ಯ. ಈ ದೃಷ್ಟಿಯಿಂದ, ಹಾಗೂ ಭಟ್ಟರ ವಿವಾಹೋತ್ತರ ಜೀವನದ ಎಲ್ಲಾ ವಿಧದ ಕಾರ್ಯಕೈಂಕರ್ಯಗಳಲ್ಲಿ ಸಂಪೂರ್ಣ ತೊಡಗಿಸಿಕೊಂಡು ಸದಾ ಯಶೋಮುಖಿಗಳಾಗುವಂತೆ ಮಾಡಿದ್ದಾರೆ. 

ಯಜ್ಞಪ್ರಕ್ರಿಯಗಳಲ್ಲಿ ತೊಡಗುವ ಕಾರಣಕ್ಕೆ ಪತ್ನೀ ಎಂದು ಕರೆಯುತ್ತಾರೆ ಎಂಬುದು ಸರ್ವವಿದಿತ ಸಂಗತಿ. ಶ್ರೀಮತಿ ಶೈಲಾರವರು ಭಟ್ಟರ ಅಧ್ಯಯನ\enginline{-}ಅಧ್ಯಾಪನವೆಂಬ ಯಜ್ಞದಲ್ಲಿ ಸಂಪೂರ್ಣಸಹಕಾರವನ್ನಿತ್ತು ತೊಡಗಿಸಿಕೊಂಡವರು. 									

ನನಗೆ ನೆನಪಿರುವಂತೆ ನನ್ನ ಮಗ ವಿನಾಯಕ ಮೈಸೂರು ಪಾಠಶಾಲೆಯಲ್ಲಿ ಅಧ್ಯಯ\-ನಕ್ಕೆ ತೆರಳಿದಾಗ ಅನೇಕ ದಿನಗಳವರೆಗೆ ಭಟ್ಟರ ಮನೆಯಲ್ಲೇ ವಾಸ್ತವ್ಯ ಮಾಡಿದ್ದ. ಆಗ ವಿದ್ವಾನ್॥ ಗಂಗಾಧರ ಭಟ್ಟರು ಮತ್ತು ಶ್ರೀಮತಿ ಶೈಲಾರವರು ತಂದೆ\enginline{-}ತಾಯಿಗಳಾಗಿ ಅನ್ನಪಾನಾದಿ ಸಕಲ ವ್ಯವಸ್ಥೆಗಳನ್ನೂ ಮಾಡಿದ್ದಾರೆ. 

ಅಷ್ಟೆ ಅಲ್ಲ, ವಿನಾಯಕನ ಸರ್ವಾಂಗೀಣ ಅಧ್ಯಯನಕ್ಕೆ ಸಕಲವಿಧದಲ್ಲಿ ಸಹಾಯ\enginline{-}ಸಹಕಾರ  \enginline{-}  ಮಾರ್ಗ\-ದರ್ಶನ ಇತ್ತು, ಇಂದು ಓರ್ವ ಅಧ್ಯಾಪಕನಾಗಿ ಆತ ಕಾರ್ಯ\-ನಿರ್ವಹಿಸುವಲ್ಲಿ ಸಮರ್ಥನಾಗುವಂತೆ ಮಾಡುವಲ್ಲಿಯೂ ವಿದ್ವಾನ್॥ ಗಂಗಾಧರ ಭಟ್ಟ ದಂಪತೀ ಪಾತ್ರ ಮಹತ್ತಮವಾದದ್ದು. ಈ ಹಿನ್ನೆಲೆಯಲ್ಲಿ ಇವರೀರ್ವರ ಋಣ ನನ್ನ ಮೇಲಿದೆ. ಇದು ನಾನು ಎಂದೂ ಮರೆಯಲಾರದ, ಮರೆಯಬಾರದ ವಿಷಯವೂ ಹೌದು. ಈ ಮುಖೇನ ನನ್ನ ಕೃತಜ್ಞತೆಯನ್ನು ಸಲ್ಲಿಸುವೆ. 

ಹೀಗೆಯೇ ಅನೇಕ ವಿದ್ಯಾರ್ಥಿಗಳಿಗೆ ಆಶ್ರಯವನ್ನಿತ್ತವರಾದ ಕಾರಣ ವಿದ್ವಾನ್॥ ಗಂಗಾಧರಭಟ್ಟದಂಪತೀ ಅಪುತ್ರರಾಗಿಯೂ ಪುತ್ರವಂತರು, ಏಕೆಂದರೆ ನೂರಾರು ವಿದ್ಯಾರ್ಥಿ\-ಗಳಿಗೆ ವಿದ್ಯೋಪದೇಶಮಾಡಿ ಅವರಿಗೆಲ್ಲ ದೃಢವಾದ ಬದುಕನ್ನು ನೀಡಿದ ತಂದೆತಾಯಿಗಳಾಗಿದ್ದಾರೆ. 					

ಪ್ರಸಿದ್ಧವಾದ ಮಾತೊಂದು ಹೀಗಿದೆ  \enginline{-}   
\vskip -5pt

\begin{tabular}{>{\hspace{0.8cm}}l}
ಜನಿತಾ ಚೋಪನೀತಾಚ ಯಶ್ಚ ವಿದ್ಯಾಂ ಪ್ರಯಚ್ಛತಿ~।\\
ಅನ್ನದಾತಾ ಭಯತ್ರಾತಾ ಪಂಚೈತೇ ಪಿತರ: ಸ್ಮೃತಾ:~॥ 		
\end{tabular}

\section*{ವಿದ್ಯಾದಾನ   \eng{-}} 

ಅನ್ನದಾನಗಳನ್ನು ನೀಡಿದ ಭಟ್ಟರಿಗೆ ಸಾವಿರಾರು ಮಕ್ಕಳಿದ್ದಾರೆ. ಜೀವನದಲ್ಲಿ ಕಳೆಗುಂದಿದ ಎಷ್ಟೋ ಸಂದರ್ಭದಲ್ಲಿ ವಿದ್ಯಾರ್ಥಿಗಳ ಮನದಲ್ಲಿ ಮೂಡಿದ ಭಯವನ್ನು ಕಿತ್ತೋಡಿಸಿದ ಭಯತ್ರಾತರಾಗಿಯೂ ತಂದೆಯಾಗಿರುವವರು ಹೌದು.

\textbf{ವ್ಯಷೇಮ ದೇವಹಿತಂ ಯದಾಯು:} ಎಂಬಂತೆ ಭಗವದ್ಧಿತಕ್ಕೆ ಆಯುಷ್ಯವಿನಿಯೋಗ\-ವಾಗಿ ಅದು ಸಾರ್ಥಕ್ಯ ಪಡೆದುಕೊಳ್ಳಲಿ. ವಿದ್ವಾನ್~। ಗಂಗಾಧರ ಭಟ್ಟದಂಪತೀ ಆರೋಗ್ಯವಂತರಾಗಿ ಪೂರ್ಣಾಯುಷ್ಯವನ್ನು ಹೊಂದಿ, ನಿವೃತ್ತಿಯ ಉತ್ತರಜೀವನವನ್ನು ಅರ್ಥಪೂರ್ಣಗೊಳಿಸಿಕೊಳ್ಳಲೆಂದು ಭಗವಂತನಲ್ಲಿ ಪ್ರಾರ್ಥಿಸುವೆ. ಆಯುರಾರೋಗ್ಯೈಶ್ವರ್ಯಗಳಿಂದ ಕೂಡಿದ ಪರಿಪೂರ್ಣ ಜೀವನ ವಿದ್ವಾನ್~। ಗಂಗಾಧರ ಭಟ್ಟ ದಂಪತೀಯದ್ದಾಗಲಿ. 

\centerline{ಸಮಸ್ತಸನ್ಮಂಗಳಾನಿ ಭವಂತು}

\articleend
}
