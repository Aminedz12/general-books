{\fontsize{14}{16}\selectfont
\chapter{ನಮ್ಮ ಹೆಮ್ಮೆ ನಮ್ಮ ಗಂಗಾಧರಭಟ್ಟರು}

\begin{center}
\Authorline{ವಿ~॥ ರಾಘವ ಕೆ.ಎಲ್}
\smallskip
ಯೋಜನಾಸಹಾಯಕರು\\
ಕರ್ನಾಟಕ ಸಂಸ್ಕೃತ ವಿಶ್ವವಿದ್ಯಾಲಯ\\
ಚಾಮರಾಜಪೇಟೆ, ಬೆಂಗಳೂರು
\addrule
\end{center}

ವಿದ್ಯಾರ್ಥಿಯೆಂಬ ಶಿಲೆಯನ್ನು ಜ್ಞಾನಬೋಧನೆಯ ಉಳಿಪೆಟ್ಟಿನಿಂದ ಕೆತ್ತಿ ಸುಂದರಶಿಲ್ಪವನ್ನಾಗಿಮಾಡುವಲ್ಲಿ, ಸುಸಂಸ್ಕೃತವ್ಯಕ್ತಿಯಾಗಿ ರೂಪಿಸುವಲ್ಲಿ ಬೋಧನೆ ಮಾಡಿದ ಅನೇಕ ಗುರುಗಳು ಕಾರಣ. ಆದರೆ ಅದೆಷ್ಟೋ ಶಿಲೆಗಳನ್ನು ಶಿಲ್ಪವನ್ನಾಗಿ ಮಾಡಿದ ನಿಜವಾದ ಅರ್ಥದಲ್ಲಿ ಗುರುಗಳು ಗಂಗಾಧರಭಟ್ಟರು. ಶಂಕರಾಚಾರ್ಯರು ಹೇಳಿದಂತೆ “ಕೋ ಗುರುಃ? ಅಧಿಗತತತ್ವಃ ಶಿಷ್ಯಹಿತಾಯ ಉದ್ಯತಃ ಸತತಮ್~।” ಎನ್ನುವುದು ನನ್ನ ವಿದ್ಯಾಗುರುಗಳಾದ ಗಂಗಾಧರಭಟ್ಟರಲ್ಲಿ ಸಂಪೂರ್ಣವಾಗಿ ಅನ್ವಯವಾಗುತ್ತದೆ. ನ್ಯಾಯಶಾಸ್ತ್ರದ ತತ್ವವನ್ನು ಅರಿತು ಅದನ್ನು ಬೋಧಿಸುವುದರ ಜೊತೆಗೆ ಶಿಷ್ಯರ\break ತೊಂದರೆ\-ಯನ್ನು ಪರಿಹರಿಸಿ, ಶಿಷ್ಯರ ಒಳಿತಿಗೆ ಸದಾ ಯತ್ನಿಸುವುದು ಅವರ ಗುರುತ್ವ. ನಾನು ಅವರ ವಿದ್ಯಾರ್ಥಿಯಾಗಿ ಅವರ ಬಗ್ಗೆ ಒಂದೆರಡು ಮಾತು ಬರೆಯಲು ಮನಸ್ಸಿನಲ್ಲಿ ಪುಳಕ\-ವಾದರೂ, ನನ್ನಂತಹ ಅಲ್ಪ ಅವರ ವ್ಯಕ್ತಿತ್ವವನ್ನು ಹೇಗೆ ತಾನೇ ಅರಿತು ವಿವರಿಸಿ\-ಯೇನು ಎಂಬ ಅಳುಕೂ ಆಗುತ್ತಿದೆ. ಆದರೂ ವಿದ್ಯಾಗುರುಗಳ ಗುಣಪ್ರಕಾಶನ ವಿಹಿತವೇ ಆದುದರಿಂದ ನನ್ನ ಅರಿವಿನ ಹಿನ್ನೆಲೆಯೆಲ್ಲಿ ಅವರ ಕುರಿತು ಕೆಲೆವು ವಿಷಯಗಳನ್ನು ಪ್ರಸ್ತಾಪಿಸುತ್ತೇನೆ.

ಮೊದಲನೆಯದಾಗಿ ನಾನೂ ಅವರ ಶಿಷ್ಯ ಎಂದು ಹೇಳಿಕೊಳ್ಳುವುದೇ ನಮಗೆಲ್ಲಾ ಒಂದು ಹೆಮ್ಮೆ. ಇದು ನನಗಂತೂ ವೈಯಕ್ತಿಕವಾಗಿ ಆದ ಅನುಭವ. ಕೇವಲ ಕರ್ನಾಟಕ\-ದಲ್ಲಿ ಮಾತ್ರವಲ್ಲ, ತಿರುಪತಿ, ದೆಹಲಿ ಮೊದಲಾದ ಸಂಸ್ಕೃತವಿಶ್ವವಿದ್ಯಾಲಯಗಳಲ್ಲೂ ಸಹ. ಕಾರ್ಯನಿಮಿತ್ತವಾಗಿ ಅಲ್ಲೆಲ್ಲಾ ಹೋದಾಗ ನಮ್ಮನ್ನು ಪರಿಚಯಮಾಡಿಕೊಳ್ಳುವಾಗ ಅಲ್ಲಿನ ವಿದ್ವಾಂಸರು ನನ್ನನ್ನು ಕೇಳುವುದು ‘ಎಲ್ಲಿ ಓದಿದ್ದೀರಿ?’ ಎಂದು. ನಾನು ಮೈಸೂರಿನಲ್ಲಿ ಎಂದಾಕ್ಷಣ ‘ಓ, ಗಂಗಾಧರ ಭಟ್ಟರ ಬಳಿ’ ಎಂದು ಅವರು ಹೇಳುವುದು, ಹಾಗೆ ಹೇಳುವಾಗ ಅವರಿಗೆ ಗಂಗಾಧರಭಟ್ಟರ ಮೇಲಿನ ಪ್ರೀತಿ, ಅಭಿಮಾನ ಎದ್ದು ಕಾಣುತ್ತಿತ್ತು. ಮೈಸೂರಿನಲ್ಲಂತೂ ಹಲವು ಕಾಲೇಜುಗಳಲ್ಲಿ ಕೆಲಸಮಾಡುವಾಗ ಸಹ ಆದ ಅನುಭವ. ಅಲ್ಲಿ ಯಾರಾದರೂ ನೀವು ಎಲ್ಲಿ ಓದಿದ್ದು ಎಂದು ಕೇಳಿದಾಗ ಪಾಠಶಾಲೆಯಲ್ಲಿ ಓದಿದ್ದು ಎಂದ ಕೂಡಲೇ ಅಲ್ಲಿ ಹೇಳುವುದು ಒಂದೇ ಮಾತು ‘ಅದೇ ಗಂಗಾಧರ ಭಟ್ಟರು ಇದಾರಲ್ಲ, ಅಲ್ಲೇನಾ?’ ಈ ರೀತಿ ವಿದ್ವದ್ವಲಯ ಮತ್ತು ಸಾಮಾಜಿಕ ವಲಯದಲ್ಲಿ ಅವರ ಘನತೆಯನ್ನು ಕಂಡಾಗ ಅವರ ಬಗ್ಗೆ ಯಾರಿಗಾದರೂ ಹೆಮ್ಮೆಯಾಗದಿರದು, ಹಾಗೇ ನಾನು ಅವರ ಶಿಷ್ಯ ಎನ್ನುವುದಂತೂ ಮತ್ತೂ ಸಂತೋಷವೇ.

\section*{ಅಧ್ಯಾಪನರತಿ}

ಆದರೆ ಇದೊಂದೇ ನಮಗೆ ಹೆಮ್ಮೆಯ ಸಂಗತಿಯಲ್ಲ. ಅದಕ್ಕೆ ಅನೇಕಕಾರಣಗಳಿವೆ. ಅದರಲ್ಲೊಂದು ಅಧ್ಯಯನ\enginline{-}ಅಧ್ಯಾಪನದ ಪ್ರೀತಿ, ನಿಷ್ಠೆ. ಬೆಳಿಗ್ಗೆ ೭.೩೦ ಕ್ಕೆ ಸರಿಯಾಗಿ ತರ\-ಗತಿಗೆ ಬಂದುಬಿಡುತ್ತಿದ್ದರು. ಎಷ್ಟೋ ಬಾರಿ ಪ್ರಾಂಶುಪಾಲರು ಕರೆಕಳುಹಿಸಿದರೂ, ತುಂಬಾ ಅರ್ಜೆಂಟಾ, ಇಲ್ಲಾ ಅಂದ್ರೆ ಆಮೇಲೆ ಕ್ಲಾಸು ಮುಗಿದ ಅನಂತರ ಬರ್ತೀನಿ ಅಂತ ಹೇಳಮ್ಮಾ’ ಎಂದು ಕರೆಯಲು ಬಂದ ಪ್ಯೂನ್ಗಳಿಗೆ ಹೇಳಿದ್ದೂ ಇದೆ. ಹಾಗಾಗಿ ಕೆಲವೊಮ್ಮೆ ಪ್ರಾಂಶುಪಾಲರೇ ‘ಕ್ಲಾಸ್ಮುಗಿದ ಅನಂತರ ಬರಲು ಹೇಳು’ ಎಂದು ಪ್ಯೂನ್ಗಳನ್ನು ಕಳುಹಿಸುತ್ತಿದ್ದರು. ಅವರಿಗೆ ಅಷ್ಟು ಪ್ರೀತಿ ಪಾಠಮಾಡುವುದು. ಇದು ಕೇವಲ ಕಾಲೇಜಿನಲ್ಲಿ ಮಾತ್ರವಲ್ಲ. ಮನೆಯಲ್ಲಿ ಪಾಠಮಾಡುವಾಗಲೂ ಸಹ ೧೧ ಗಂಟೆಗೆ ಪಾಠಕ್ಕೆ ಬರಲು ಹೇಳಿದರೆ ಅದಕ್ಕೂ ಮೊದಲೇ ಸಿದ್ಧವಾಗಿರುತ್ತಿದ್ದರು.
ಹಾಗೆಯೇ ಪಾಠ್ಯವಿಷಯದ ಬಗ್ಗೆ ಮೊದಲು ಆಸಕ್ತಿಬೆಳೆಸುತ್ತಿದ್ದರು. ಶಾಸ್ತ್ರವಿಷಯವೆಂದರೆ ಕಠಿಣವೆನ್ನುವ ಭಾವನೆ ಸಾಮಾನ್ಯ\-ವಾಗಿ ವಿದ್ಯಾರ್ಥಿಗಳಲ್ಲಿರುತ್ತದೆ. ಆದರೆ ವಿಷಯಕ್ಕೆ ಪೂರಕವಾದ ಲೌಕಿಕದೃಷ್ಟಾಂತ, ಪೂರ್ವಪೀಠಿಕೆಗಳಿಂದ ಅಧ್ಯೇಯವಿಷಯದಲ್ಲಿ ಮೊದಲು ಆಸಕ್ತಿಬರುವಂತೆ ಮಾಡಿ, ಶಾಸ್ತ್ರಪಾಠವನ್ನೂ ಬಹಳಸ್ವಾರಸ್ಯಪೂರ್ಣವಾಗಿಸುತ್ತಿದ್ದರು ನಮ್ಮ ಗುರುಗಳಾದ ಗಂಗಾಧರಭಟ್ಟರು. ಯಾವ ವಿಷಯದಲ್ಲಾದರೂ ಸಂದೇಹ ಬಂದರೆ ಮತ್ತೆ ವಿವರಿಸಿ ಪಾಠಮಾಡಿ ವಿವರಿಸುತ್ತಿದ್ದರು. ಹಾಗಾಗಿ ಶಂಕಯಾ ಭಕ್ಷಿತಂ ಸರ್ವಂ ತ್ರೈಲೋಕ್ಯಂ ಸಚರಾಚರಮ್~॥ ಸಾ ಶಂಕಾ ಭಕ್ಷಿತಾ ಯೇನ ಸ ಗುರುರ್ದೇವಿ ದುರ್ಲಭಃ~॥ ಎಂಬಂತೆ ಇಂತಹವರು ದುರ್ಲಭವೇ ಸರಿ.

\section*{ದೃಷ್ಟಾಂತೋಪಯೋಗ}

ಶಾಸ್ತ್ರೀಯವಿಷಯವನ್ನೂ ಸಹ ಲೌಕಿಕದೃಷ್ಟಾಂತದ ಮೂಲಕ ಸುಲಭವಾಗಿ ಅರ್ಥವಾಗುವಂತೆ ಪಾಠಮಾಡುವುದು ಇನ್ನೊಂದು ವಿಶೇಷತೆ. ಅವರು ಪಾಠಮಾಡುವಾಗ ಹೇಳುವ ಉದಾಹರಣೆಗಳು ಮರೆಯುವುದೇ ಇಲ್ಲ. ವ್ಯಾಪ್ತಿಜ್ಞಾನ ಮತ್ತು ಪಕ್ಷಧರ್ಮತಾಜ್ಞಾನದಿಂದಲೇ ಅನುಮಿತಿ ಉಂಟಾಗುವುದಿಲ್ಲ, ಪರಾಮರ್ಶ ಎಂಬ ವಿಶಿಷ್ಟಜ್ಞಾನಾಂತರಬೇಕು ಎಂಬ ನ್ಯಾಯಸಿದ್ಧಾಂತವನ್ನು ನಿರೂಪಿಸುವಾಗ ಅವರು ನೀಡಿದ ಗುಪ್ತದಾನದ ಉದಾಹರಣೆ ಇನ್ನೂ ನನ್ನ ನೆನಪಿನಂಗಳದಲ್ಲಿ ಹಸಿಯಾಗೇ ಇದೆ, ನಾನೂ ಅದೇ ಉದಾಹರಣೆಯನ್ನು ನನ್ನ ವಿದ್ಯಾರ್ಥಿಗಳಿಗೂ ಹೇಳಿದ್ದಿದೆ.

\section*{ಅಧ್ಯಯನಪ್ರೇರಣೆ}

ಕೇವಲ ಪರೀಕ್ಷೋಪಯೋಗಿಯಾಗಿ ಪಾಠ್ಯಗ್ರಂಥವನ್ನು ಮಾತ್ರ ಬೋಧಿಸದೇ, ಅದಕ್ಕೆ ಪೂರಕವಾದ ಬೇರೆಗ್ರಂಥಗಳನ್ನು ಓದುವಂತೆಯೂ ವಿದ್ಯಾರ್ಥಿಗಳನ್ನು ಪ್ರೇರೇಪಿಸು\-ತ್ತಿದ್ದರು. ಜೊತೆಗೆ ತಾವೇ ಪಾಠವನ್ನೂ ಮಾಡುತ್ತಿದ್ದರು. ಅದರ ಫಲವೇ ಪಠ್ಯದಲ್ಲಿಲ್ಲದಿದ್ದರೂ ಶಬ್ದಶಕ್ತಿಪ್ರಕಾಶಿಕಾದ ಒಂದಿಷ್ಟು ಭಾಗವನ್ನಾದರೂ ಓದುವಂತಾಯಿತು. ಜೊತೆಗೆ ಯಾವುದಾದರೂ ಸ್ಪರ್ಧೆಯ ಸಂದರ್ಭದಲ್ಲೂ ಸ್ಪರ್ಧಾವಿಷಯವನ್ನು ವಿವರಿಸುವುದರ ಜೊತೆಗೆ ತತ್ಸಂಬಂಧಿ ವಿಷಯಗಳು ಯಾವ ಗ್ರಂಥದಲ್ಲಿ ಏನೇನಿದೆ ಎಂಬುದನ್ನು ಗ್ರಂಥಸಾತ್ಪಾಠಮಾಡುತ್ತಿದ್ದರು. ಇದರಿಂದ ಹಲವು ಗ್ರಂಥಗಳ ಕೆಲವು ಭಾಗದ ಪರಿಚಯವೂ ಆಗುತ್ತಿತ್ತು. ಇದರಿಂದ ಆಯಾ ಗ್ರಂಥಗಳಲ್ಲಿ ರುಚಿಯೂ ಬೆಳೆಯುತ್ತಿತ್ತು.

\section*{ಬೇಸರವಿಲ್ಲದ ಪಾಠ}

ವಿದ್ಯಾರ್ಥಿಗಳ ಮನೋಸ್ಥಿತಿಯನ್ನು ಅರಿತು ಅವರ ಮಟ್ಟಕ್ಕೆ ಇಳಿದು ಪಾಠಮಾಡುತ್ತಿದ್ದರು. ತರಗತಿಯಲ್ಲಿ ಯಾರಿಗಾದರೂ ಯಾವುದೋ ವಿಷಯ ಅರ್ಥವಾಗಿಲ್ಲವೆಂದರೆ ಪುನಃ ಪಾಠಮಾಡಲು ಒಂಚೂರು ಬೇಸರಿಸುತ್ತಿರಲಿಲ್ಲ. ಒಂದಲ್ಲ ನಾಲ್ಕುಬಾರಿ ಹೇಳುವಾಗಲೂ ಮೊದಲಬಾರಿ ಹೇಳಿದ ತಾಳ್ಮೆ, ಪ್ರೀತಿಯಿಂದಲೇ ಹೇಳುತ್ತಿದ್ದರು. ಹಾಗಾಗಿ ಅವರಿಗಂತೂ ಪಾಠಮಾಡುವುದರಲ್ಲಿ ಬೇಸರವಿಲ್ಲ, ವಿದ್ಯಾರ್ಥಿಗಳಿಗೂ ಅವರ ಪಾಠದಲ್ಲಿ ಬೇಸರ\-ವಿಲ್ಲ. ಜೊತೆಗೆ ಅವರ ಬೋಧನಶೈಲಿ, ಉದಾಹರಣೆಗಳ ಬಳಕೆ,\break ಸಂದರ್ಭೋಚಿತ ನವಿರುಹಾಸ್ಯ ಇವೆಲ್ಲ ಅವರ ಪಾಠದ ಆಕರ್ಷಣೆ. ಹಾಗಾಗಿ ಪಾಠ್ಯೇತರ, ಲೌಕಿಕವಿಷಯದ ಸ್ಪರ್ಶವೇ ಇಲ್ಲದೇ ಇದ್ದರೂ ಅದು ಸ್ವಾರಸ್ಯಪೂರ್ಣ, ಬೇಸರ\-ವಿಲ್ಲದಪಾಠ.
\begin{verse}
ಗುರವೋ ಬಹವಃ ಸಂತಿ ಶಿಷ್ಯವಿತ್ತಾಪಹಾರಕಾಃ~।\\
ದುರ್ಲಭೋಯಂ ಗುರುರ್ದೇವಿ ಶಿಷ್ಯಚಿತ್ತಾಪಹಾರಕಃ~॥
\end{verse}
ಎಂಬಂತೆ ಪಾಠದ ಮೂಲಕ ಶಿಷ್ಯರ ಮನಸ್ಸನ್ನು ಆಕರ್ಷಿಸಿದ್ದರು.

\section*{ಶಿಷ್ಯಹಿತಾಯ ಉದ್ಯತಃ ಸತತಮ್}

ವಿದ್ಯಾರ್ಥಿಗಳ ಏಳ್ಗೆಗೆ ಸದಾ ಉದ್ಯತರಾಗಿದ್ದರು. ವಿದ್ಯಾರ್ಥಿಗಳಲ್ಲಿ ಅಧ್ಯಯನದ ಜೊತೆಗೆ ಭಾಷಣ, ವಾಕ್ಯಾರ್ಥಕಲೆ ಬೆಳೆಯಲಿ ಎಂದು ಪ್ರತಿತ್ರಯೋದಶಿಯ ದಿನ ಪಾಕ್ಷಿಕ\-ಗೋಷ್ಠಿಯಯನ್ನು ಆರಂಭಿಸಿದರು. ಪ್ರತಿಪ್ರದೋಷದ ದಿನ ಸಂಜೆ ಕೊನೆಯ ಅವಧಿ\-ಯಲ್ಲಿ ಪರ್ಯಾಯದಂತೆ ಒಂದೊಂದು ಶಾಸ್ತ್ರದ ಒಬ್ಬ ವಿದ್ಯಾರ್ಥಿ ಶಾಸ್ತ್ರ\-ಪ್ರಸ್ತಾವವನ್ನು ಮಾಡಬೇಕಿತ್ತು, ಉಳಿದ ವಿದ್ಯಾರ್ಥಿಗಳು, ಅಧ್ಯಾಪಕರು ಪ್ರಶ್ನೋತ್ತರದಲ್ಲಿ ಭಾಗವಹಿ\-ಸುತ್ತಿದ್ದರು. ಅನಂತರ ಪ್ರದೋಷಸಂಘದ ಮೂಲಕ ಇದು ಮುನ್ನಡೆ\-ಯುವಂತೆ ಮಾಡಿದ್ದರು. ಹೀಗೆ ವಿದ್ಯಾರ್ಥಿಗಳಲ್ಲಿ ಹೆಚ್ಚಿನ ಅಧ್ಯಯನಕ್ಕೆ ಪ್ರೇರಣೆ\-ಯಾಗಿದ್ದರು. ಆದರೆ ವಿದ್ಯಾರ್ಥಿಗಳಲ್ಲಿ ತಾರತಮ್ಯವಿರಲಿಲ್ಲ. ಒಮ್ಮೆ ನಾನು ಭಾಷಣ\-ಸ್ಪರ್ಧೆಗೆ ಭಾಗವಹಿಸುತ್ತೇನೆ, ಶಲಾಕಾಸ್ಪರ್ಧೆಗೆ ಭಾಗವಹಿಸುವುದಿಲ್ಲವೆಂದಾಗ ಆಗ್ರಹ\-ಪೂರ್ವಕವಾಗಿ ನನ್ನನ್ನು ಶಲಾಕಾಸ್ಪರ್ಧೆಗೆ ಸಿದ್ಧಪಡಿಸಿ, ಭಾಷಣಸ್ಪರ್ಧೆಗೆ ಹಿರಿಯ\-ವಿದ್ಯಾರ್ಥಿ\-ಯೋರ್ವರನ್ನು ಆಯ್ಕೆಮಾಡಿ ಎಲ್ಲರಿಗೂ ಸಮಾನ ಅವಕಾಶವನ್ನು ಕಲ್ಪಿಸಿದ್ದರು.

\section*{ಛಾತ್ರವಾತ್ಸಲ್ಯ}

ವಿದ್ಯಾರ್ಥಿಗಳ ಬಗ್ಗೆಯಂತೂ ತುಂಬಾ ಪ್ರೀತಿ, ವಾತ್ಸಲ್ಯವಿತ್ತು. ಇದು ಎಲ್ಲಾ ಶಾಸ್ತ್ರದ ವಿದ್ಯಾರ್ಥಿಗಳಿಗೂ ತಿಳಿದಿರುವ, ಅನುಭವವಾಗಿರುವ ವಿಷಯವೇ. ನನಗೂ ಸಹ ಅಧ್ಯಾಪನದ ಜೊತೆಗೆ ಅವರು ಮಾಡಿದ ಸಹಾಯ ಮರೆಯಲು ಸಾಧ್ಯವಿಲ್ಲ. ನಾನು ಓದುವ ಸಮಯದಲ್ಲಿ ಅವರ ಪರಿಚಯದವರು ಯಾರಿಗಾದರೂ ನೀವು ಹೇಳಿದ ವಿದ್ಯಾರ್ಥಿಗೆ ಧನಸಹಾಯಮಾಡುತ್ತೇನೆ ಎಂದು ಹೇಳಿದ್ದರಂತೆ. ಆಗ ಗಂಗಾಧರ ಭಟ್ಟರು ಅವರ ಮನೆಯಲ್ಲೇ ಅವರ ಕಂಪ್ಯೂಟರ್ನಲ್ಲಿಯೇ ವಿದ್ಯಾರ್ಥಿವೇತನಕ್ಕೆ ನನ್ನಿಂದ ಅರ್ಜಿ ಸಿದ್ಧಪಡಿಸಿ, ಆ ವ್ಯಕ್ತಿಯ ವಿಳಾಸ ನೀಡಿ ಅರ್ಜಿಕೊಡಲು ಹೇಳಿದ್ದರು, ಅದರಂತೆ ನನಗೆ ವಿದ್ಯಾರ್ಥಿವೇತನವೂ ದೊರೆಯಿತು. ಹಾಗೆಯೇ ಅವರ ಪರಿಚಯದ ಅವಧಾನಿಗಳು ಕಾಲೇಜಿನಲ್ಲಿ ನೀಡುವ ವಿದ್ಯಾರ್ಥಿವೇತನವೂ ಸಿಕ್ಕಿತ್ತು.

ಇನ್ನು ವರ್ಷದ ಆರಂಭದಲ್ಲಿ ಭೋಜನವ್ಯವಸ್ಥೆಯಾಗದಿದ್ದಾಗ ‘ರಾಘವೇಂದ್ರ\-ಭವನ’ದಲ್ಲಿ ಊಟದ ವ್ಯವಸ್ಥೆಯನ್ನೂ ದೊರಕಿಸಿಕೊಟ್ಟಿದ್ದರು. ಅದಕ್ಕಿಂತಲೂ ಹೆಚ್ಚಾಗಿ ಎಷ್ಟೋ ಬಾರಿ ಅವರ ಮನೆಯಲ್ಲೇ ಊಟವಾಗಿದ್ದೂ ಇದೆ. ಒಮ್ಮೆಯಂತೂ ಮಧ್ಯಾಹ್ನ ಶಾಲೆ ಮುಗಿದು ಮನೆಗೆ ಹೋಗುವಾಗ ’ಮಧ್ಯಾಹ್ನ ಊಟಕ್ಕೆ ಎಲ್ಲಿಗೆ ಹೋಗ್ತೀಯಾ?’ ಎಂದು ಕೇಳಿದರು. ಎಲ್ಲೂ ಇಲ್ಲ ಎಂದಾಗ ನಮ್ಮನೆಗೆ ಬಾ ಎಂದು ಹೇಳಿದರು. ಮಧ್ಯಾಹ್ನ ಅವರ ಮೆನೆಗೆ ಹೋದರೆ ಹಲಸಿನಹಣ್ಣಿನ ಕಡಬಿನ ಸಿಹಿಯೂಟ. ಹೀಗೆ ವಿದ್ಯಾರ್ಥಿಗಳ ಮೇಲೆ ಅವರಿಗೆ ತುಂಬಾ ಪ್ರೀತಿ. ಅವರಿಂದ ಉಪಕೃತರಾದ ವಿದ್ಯಾರ್ಥಿಗಳು ಅನೇಕ. ಬಹುಶಃ ವಿದ್ಯಾರ್ಥಿಗಳ ಗುರುಭಕ್ತಿಗಿಂತಿಲೂ ಅವರ ಛಾತ್ರವಾತ್ಸಲ್ಯ ದೊಡ್ಡದು.

ಅಧ್ಯಯನದ ಅನಂತರವೂ ಸಹ ಈ ವಾತ್ಸಲ್ಯ ಹಾಗೇ ಮುಂದುವರಿದಿದೆ. ನಾನು ಉದ್ಯೋಗನಿರತನಾದಾಗ ಒಮ್ಮೆ ದೂರವಾಣಿ ಕರೆಮಾಡಿ ವಿಶ್ವವಿದ್ಯಾಲಯದ ಕ್ರಾಫರ್ಡ\-ಭವನದಲ್ಲಿ ಸಂಸ್ಕೃತೋತ್ಸವದಲ್ಲಿ ವಿಷಯಮಂಡನೆ ಮಾಡಲು ಹೇಳಿದರು. ಕೇವಲ ಒಂದೆರಡು ದಿನಮಾತ್ರವಿತ್ತು ಕಾರ್ಯಕ್ರಮಕ್ಕೆ. ಹಾಗಾಗಿ ಸಿದ್ಧತೆ ಮಾಡುವುದು ಕಷ್ಟವೆಂದಾಗ ಮನೆಗೆ ಕರೆಸಿ ಅರ್ಥಶಾಸ್ತ್ರದಲ್ಲಿನ ಕರಸಂಗ್ರಹಣೆಯ ಬಗ್ಗೆ ವಿಷಯವನ್ನೂ ಸಿದ್ಧಪಡಿಸಿ ನನ್ನಿಂದ ಭಾಷಣಮಾಡಿಸಿದ್ದರು. ಜೊತೆಗೆ ಆಯೋಜಕರು ಸಂಭಾವನೆ ಕೊಡದಿದ್ದಾಗ ಬೇಸರಮಾಡಿಕೊಂಡು ತಾವೇ ಕೊಡುತ್ತೇನೆ ಎಂದಿದ್ದರು, ಕೊನೆಗೆ ಅಂತೂ ಆಯೋಜಕರಿಗೆ ಅದು ತಿಳಿದು ಅವರು ಗೌರವಸಂಭಾವನೆಯನ್ನು ನೀಡಿದರು.

ಆದರೆ ಅವರು ವಿದ್ಯಾರ್ಥಿಗಳು ತಮ್ಮ ಪ್ರತಿಭೆ, ಸಾಮರ್ಥ್ಯದಿಂದಲೇ ಬೆಳೆಯ\-ಬೇಕೆಂದು ಬಯಸುತ್ತಿದ್ದರು, ತಮ್ಮ ಪ್ರಭಾವವನ್ನು ಬಳಸುತ್ತಿರಲಿಲ್ಲ. ಇದು ಅವರ ವಿಶೇಷ\-ಗುಣ. ವಿದ್ವತ್ತು ಮುಗಿದ ಕೂಡಲೇ ನಾನು ಉದ್ಯೋಗ ಮಾಡಬೇಕೆಂದು ನಿರ್ಧರಿಸಿ, ಉದ್ಯೋಗ ಹುಡುಕುತ್ತಿದ್ದಾಗ ಅವರೇ ನನಗೆ ಸದ್ವಿದ್ಯಾ ಕಾಲೇಜಿನಲ್ಲಿ ಸಂಸ್ಕೃತ ಉಪನ್ಯಾಸಕರ ಅವಶ್ಯಕತೆಯಿದೆ ಹೋಗು ಎಂದು ದೂರವಾಣಿಯ ಮೂಲಕ\break ತಿಳಿಸಿದ್ದರು. ಆಗ ಸದ್ವಿದ್ಯಾ ಸಂಸ್ಥೆಯ ಕಾರ್ಯದರ್ಶಿಯಾಗಿದ್ದ ಪ್ರೊ.ಕೆ.ವಿ.\ ಅರ್ಕನಾಥ ಅವರು ಗಂಗಾಧರಭಟ್ಟರಿಗೆ ಈ ಬಗ್ಗೆ ಹೇಳಿದ್ದರಂತೆ. ಅದರಂತೆ ನಾನು ಸದ್ವಿದ್ಯಾ ಕಾಲೇಜಿಗೆ ಹೋದಾಗ ಅಲ್ಲಿನ ಪ್ರಾಚಾರ್ಯರು ‘ಸಂಸ್ಕೃತ ವಿಭಾಗಮುಖ್ಯರು ಮಧ್ಯಾಹ್ನ ಬರುತ್ತಾರೆ, ಅವರು ಇಂಟರ್ವ್ಯೂ ಮಾಡುತ್ತಾರೆ. ಆಗಲೇ ಬನ್ನಿ’ ಎಂದರು. ನನಗೋ ಮೊದಲ ಬಾರಿ ಇಂಟರ್ವ್ಯೂಗೆ ಹೋಗುತ್ತಿರುವುದು, ಜೊತೆಗೆ ಎಚ್.ವಿ.\ ನಾಗರಾಜರಾವ್ ಅವರಂತಹ ವಿದ್ವಾಂಸರು ಅಲ್ಲಿ ವಿಶೇಷ ಆಹ್ವಾನಿತರಾಗಿ ಕಾರ್ಯನಿರ್ವಹಿಸುತ್ತಿದ್ದರು. ಹಾಗಾಗಿ ಕೊಂಚ ಭಯವಿತ್ತು. ಪುನಃಗಂಗಾಧರಭಟ್ಟರ ಬಳಿಗೇ ಬಂದು ವಿಷಯವನ್ನು ತಿಳಿಸಿ ನೀವು ಒಂದು ಮಾತು ಹೇಳಬಹುದೇ ಎಂದಾಗ, ನೀನು ಯಾಕೆ ಹೆದರುತ್ತೀಯಾ? ಇಂಟರ್ವ್ಯೂಗೆ ಹೋಗು ಆಮೇಲೆ ನೋಡೋಣ ಎಂದು ಪ್ರೇರೇಪಿಸಿ ಕಳುಹಿಸಿದರು. ಅವರ ಪ್ರೇರೆಣೆಯಂತೆ ಹೋದೆ ಇಂಟರ್ವ್ಯೂ, ಡೆಮೋ ಎಲ್ಲಾ ಚೆನ್ನಾಗಿಯೂ ಆಗಿ, ಅವರು ಬಹಳ ಖುಶಿಯಿಂದ ನನ್ನನ್ನು ಉಪನ್ಯಾಸಕನಾಗಿ ಆಯ್ಕೆ ಮಾಡಿದರು.

ಹೀಗೆ ನನ್ನ ಅಧ್ಯಯನ, ವೃತ್ತಿಜೀವನವೆರಡರಲ್ಲೂ ನನ್ನ ವಿದ್ಯಾಗುರುಗಳಾದ ಗಂಗಾಧರಭಟ್ಟರ ಉಪಕಾರ ಅವಿಸ್ಮರಣೀಯ. ಆದ್ದರಿಂದಲೇ ಅವರು ನಿಜವಾಗಿಯೂ ‘ಅಧಿಗತತ್ವಃ, ಶಿಷ್ಯಹಿತಾಯ ಉದ್ಯತಃ ಸತತಮ್’ ಎಂಬಂತೆ ನಿಜಾರ್ಥದ ಗುರುಗಳಾಗಿದ್ದಾರೆ. ಅವರಿಗೆ ನಮ್ಮಿಂದ ಪ್ರತ್ಯುಪಕಾರವನ್ನು ಮಾಡಿ ಗುರುಋಣವನ್ನು ತೀರಿಸಲಂತೂ ಸಾಧ್ಯವಿಲ್ಲ. ಆದರೂ ಕರ್ಮಣಾ ಮನಸಾ ವಾಚಾ ನಿತ್ಯಮಾರಾಧಯೇದ್ಗುರುಮ್~। ಎಂಬಂತೆ ಈ ಮೂಲಕವಾಗಿ ವಾಚಿಕ ಅಭಿವಂದನೆಗಳನ್ನು ಸಲ್ಲಿಸುತ್ತೇನೆ.

\articleend
}
