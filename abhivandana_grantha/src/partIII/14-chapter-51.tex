{\fontsize{14}{16}\selectfont
\chapter{ಆಡದೇ ಮಾಡಿದವರು}

\begin{center}
\Authorline{ಡಾ~॥ ನಿರಂಜನ, ವಾನಳ್ಳಿ}
\smallskip

ರೆಜಿಸ್ಟ್ರಾರ್, \\ಗಂಗೂಬಾಯಿ ಹಾನಗಲ್ ಸಂಗೀತಕಲೆಗಳ ವಿಶ್ವವಿದ್ಯಾಲಯ, ಮೈಸೂರು\\
ಪ್ರಾಧ್ಯಾಪಕರು, ಪತ್ರಿಕೋದ್ಯಮ ವಿಭಾಗ\\
ಮೈಸೂರು ವಿಶ್ವವಿದ್ಯಾನಿಲಯ
\addrule
\end{center}

ಪ್ರೀತಿಯ ಗಂಗಾಧರ ಭಟ್ಟರ ಬಗ್ಗೆ ಬರೆಯಲು ಹೊರಟಾಗ ಮೊದಲು ನೆನಪಾದುದು ಸರ್ವಜ್ಞನ ಸಾಲು. ಆಡದೇ ಮಾಡುವರು ರೂಢಿಯೊಳಗುತ್ತಮರು ಎಂಬುದಾಗಿ. ನಿಜಕ್ಕೂ ಆದಕ್ಕೆ ನಮ್ಮ ನಡುವೆ ಒಬ್ಬರು ಸಾಕ್ಷಿಯಿದ್ದರೆ ಗಂಗಾಧರ ಭಟ್ಟರು ಎಂದು ದೃಢವಾಗಿ ಹೇಳಬಲ್ಲೆ.

ಗಂಗಾಧರ ಭಟ್ಟರು ಹೇಗೆ ಮತ್ತು ಎಂದು ಪರಿಚಯವಾದರು ಎಂಬುದೇ ನೆನಪಿಲ್ಲ. ನಮ್ಮ ಸ್ನೇಹ ಅಷ್ಟು ಪುರಾತನವಾದುದು ಎಂಬುದನ್ನು ಇದು ಹೇಳುತ್ತದೆ. ಈಗ ನೆನಪಿಗೆ ಬರುವುದು ಉಡುಪಿಯಲ್ಲಿ ಪಿಯುಸಿ ಮುಗಿಸಿದ ನಾನು ಪದವಿ ವ್ಯಾಸಂಗಕ್ಕಾಗಿ ಮೈಸೂರಿಗೆ ಬಂದವನು ಸೀದಾ ನ್ಯೂ ಸಯಾಜಿರಾವ್ ರಸ್ತೆಯಲ್ಲಿದ್ದ ಗಂಗಾಧರ ಭಟ್ಟರ ಮನೆಗೆ ಬ್ಯಾಗು ಟ್ರಂಕಿನ ಸಮೇತ ಬಂದಿಳಿದಿದ್ದೆ. ಅದು ಮೊದಲೇ ಅಷ್ಟು ಸ್ನೇಹವಿಲ್ಲದಿದ್ದರೆ ಸಾಧ್ಯವಾಗಲಾರದು ಅಥವಾ ಗಂಗಾಧರ ಭಟ್ಟರ ವಿಶ್ವಾಸವೇ ಅಂಥದ್ದು. ಆಗ ಗಂಗಣ್ಣನಿಗೆ ಮದುವೆಯೂ ಆಗಿರಲಿಲ್ಲ. ಶಂಕರ ಮಠದ ಸಂಸ್ಕೃತ ಪಾಠಶಾಲೆಯಲ್ಲಿ ಅಧ್ಯಾಪಕರಾಗಿದ್ದರು. ಅವರ ಮನೆಯಲ್ಲಿ ನನ್ನ ಹಾಗೆ ಇನ್ನೂ ಒಂದಿಬ್ಬರು ಇದ್ದರೆಂದು ನೆನಪು. ಆಗೆಲ್ಲ ಸಿರ್ಸಿ  \enginline{-}  ಸಿದ್ದಾಪುರದ ಹವ್ಯಕ ಮಾಣಿಗಳಿಗೆ ಮೈಸೂರಿಗೆ ಬಂದಕೂಡಲೇ ಉಳಿಯಲೊಂದು ನೆಲೆ, ಕಲಿಯಲೊಂದು ಕಾಲೇಜು ಮುಂತಾಗಿ ನೆಲೆ ಕಲ್ಪಿಸಿ ಕೊಡುತ್ತಿದ್ದುದು ಗಂಗಾಧರ ಭಟ್ಟರು. ಸಾಮಾನ್ಯವಾಗಿ ನಮ್ಮ ಕಡೆಯಿಂದ ಬರುವವರೆಲ್ಲ ಗಂಗಾಧರ ಭಟ್ಟರ ವಿಳಾಸ ಹಿಡಿದೇ ಬರುತ್ತಿದ್ದರು. ಜನರಿಗೆ ಅವರ ಬಗ್ಗೆ ಅಖಂಡ ವಿಶ್ವಾಸವಿತ್ತು. ಹಾಗೆಯೇ ಗಂಗಾಧರ ಭಟ್ಟರಿಗೂ ಯಾವ ಬೇಸರ ಮಾಡಿಕೊಳ್ಳದೇ ಬಂದವರಿಗೆ ದಾರಿ ತೋರಿಸುವ ಗುಣ ಹಾಗೂ ಸಾಮಥ್ರ್ಯವಿತ್ತು. ಅನೇಕ ಹೈಸ್ಕೂಲು ಕಾಲೇಜುಗಳಿಗೆ ಹೋಗಿ ಗಂಗಾಧರ ಭಟ್ಟರು ಕಳಿಸಿದ್ದು ಎಂದರೆ ಸೀಟು ಸಿಗಲು ಏನೂ ತೊಂದರೆಯಾಗುತ್ತಿರಲಿಲ್ಲ.

ಹಾಗೆ ಗಂಗಾಧರ ಭಟ್ಟರ ಮನೆಗೆ ಬಂದಿಳಿದ ನಾನು ರೂಮು ಹುಡುಕುವ ನೆಪದಲ್ಲಿ ಕನಿಷ್ಠ ಒಂದು ವಾರ ಅವರ ಮನೆಯಲ್ಲೇ ಕಾಲಹಾಕಿದ್ದೆ. ಆ ಒಂದು ವಾರ ಅವರ ವಿದ್ವತ್ತು, ತಾಳ್ಮೆ, ಹಲವುಕ್ಷೇತ್ರಗಳ ಪ್ರತಿಭೆ ಹತ್ತಿರದಿಂದ ಪರಿಚಯಕ್ಕೆ ಬಂದವು. ಆ ಒಂದುವಾರ ಅನ್ನದಾತರೂ ಅವರೇ ಆಗಿದ್ದರು ಎಂದು ಬೇರೆ ಹೇಳಬೇಕಿಲ್ಲ. ಆನಂತರ ನಾನು ಮಹಾರಾಜ ಕಾಲೇಜಿಗೆ ಸೇರಿದ ಮೇಲೆ ಅನೇಕ ಸಂದರ್ಭಗಳಲ್ಲಿ ಅವರ ಮನೆಗೆ ಹೋಗಿದ್ದಿದೆ. ಅವರ ಹಾಗೂ ಅತ್ತಿಗೆಯ ಆತ್ಮೀಯತೆಯನ್ನು ಸವಿದಿದ್ದಿದೆ. ನಾನಾಗ ಅಂತರ್ ಕಾಲೇಜು ಚರ್ಚಾಕೂಟಗಳಿಗೆ ಕಾಲೇಜನ್ನು ಪ್ರತಿನಿಧಿಸಿ ಹೋಗುತ್ತಿದ್ದೆ. ನಾನು ಗಳಿಸುತ್ತಿದ್ದ ಬಹುಮಾನಗಳಲ್ಲಿ ಅವರ ಪಾಲೂ ಇತ್ತೆಂದು ಧಾರಾಳ ಹೇಳಬಲ್ಲೆ. ಯಾಕೆಂದರೆ ಅನೇಕ ಬಾರಿ ಅವರ ಮನೆಗೆ ಹೋಗಿ ಚರ್ಚೆಯ ಪಾಯಿಂಟುಗಳನ್ನು ಹಾಕಿಸಿಕೊಂಡು ಬಂದಿದ್ದಿದೆ. ಆಗೆಲ್ಲ ನಾನು ಚರ್ಚಾಕೂಟಗಳಲ್ಲಿ ಗೆದ್ದಿದ್ದೇನೆ. ಅವರು ಸಂಸ್ಕೃತ ಪಾಠಶಾಲೆಯಲ್ಲಿ ಓದುವಾಗಲೂ ಅದ್ವಿತೀಯ ಚರ್ಚಾಪಟುವಾಗಿದ್ದರೆಂದು ಕೇಳಿದ್ದೇನೆ. ಚಿಂತನೆ, ಬರಹ, ಭಾಷಣ ಎಲ್ಲದರಲ್ಲೂ ಅವರದು ಪ್ರಖರವಾದ ಪ್ರತಿಭೆ. ಆದರೆ ಅದು ಹೆಚ್ಚಾಗಿ ಸಂಸ್ಕೃತ ಕಾಲೇಜಿನಿಂದ ಹೊರಗೆ ಗೊತ್ತಾಗಬೇಕಾದ ಮಟ್ಟದಲ್ಲಿ ಪಸರಿಸಲೇ ಇಲ್ಲ ಎಂಬುದು ವಿಷಾದದ ಸಂಗತಿ.

ಒಮ್ಮೆ ನಾನೊಂದು ಕತೆ ಬರೆದು ಅವರ ಬಳಿ ಓದಿದ್ದೆ. ಅದು ನಮ್ಮ ಜೊತೆ ಕನ್ನಡ ಶಾಲೆ ಓದಿ, ನಾವು ಕಾಲೇಜಿಗೆ ಸೇರುವಾಗಲೇ ಮದುವೆಯಾಗಿ ಮಕ್ಕಳನ್ನು ಹೆತ್ತಿದ್ದ ಹುಡುಗಿಯರಿಗೆ ಬರೆದ ಒಂದು ಪತ್ರದಂತಿತ್ತು. ಕೇಳಿಸಿಕೊಂಡ ಗಂಗಣ್ಣ, ಆ ಹುಡುಗಿಯರೇ  ಬರೆಯುವುದಾದರೆ ಈ ಪತ್ರ ಹೇಗಿರುತ್ತದೆ ಎಂಬುದನ್ನೂ ಸೇರಿಸು ಎಂದಿದ್ದ. ಇದು ಅವರ ಅವರ ಸೃಷ್ಟಿಶೀಲ ಮನಸ್ಸಿನ ದರ್ಶನವನ್ನು ಮಾಡಿಸಿತು. ಗಂಗಣ್ಣ ಬರೆದಿದ್ದರೆ, ತಾಳಮದ್ದಳೆಯ ಅರ್ಥಧಾರಿಯಾಗಿ ರಂಗವೇರಿದ್ದರೆ ಖ್ಯಾತನಾಮರಾಗಬಹುದಿತ್ತು. ಅವರ ಪ್ರತಿಭೆ ಖಂಡಿತವಾಗಿ  ಬಳಕೆಯಾಗಿದ್ದು ಕಡಿಮೆ. ಅದಕ್ಕೆ ಕಾರಣ ಗಂಗಣ್ಣನ ‘ಆಡದೇ ಮಾಡುವ ಸ್ವಭಾವ’ ಎನಿಸುತ್ತದೆ. ಸಂಸ್ಕೃತದಲ್ಲಿ ಅವರ ವಿದ್ವತ್ತು ಆಳವಾದುದು. ಕನ್ನಡ\-ದಷ್ಟೇ ಸರಳವಾಗಿ ಅವರು ಸಂಸ್ಕೃತದಲ್ಲಿ ಮಾತನಾಡಬಲ್ಲರು. ಶಾಸ್ತ್ರ, ತರ್ಕ ಎಲ್ಲದರಲ್ಲೂ ಅವರು ಪಾರಂಗತರು. ಅವರ ಓದಿನ ಹರಹು ವಿಸ್ತಾರವಾದುದು. ಸಂಸ್ಕೃತ,  ಕನ್ನಡ ಮತ್ತು ಇಂಗ್ಲಿಷ್ ಸಾಹಿತ್ಯಗಳಲ್ಲಿ  ಅವರಿಗೆ ಅಪಾರವಾದ ಜ್ಞಾನವಿರುವುದನ್ನು ಹತ್ತಿರದಿಂದ ಬಲ್ಲೆ. ಇಷ್ಟಾಗಿಯೂ ಯಾಕೆ ಅವರು ಸಂಸ್ಕೃತ ಕಾಲೇಜಿನ ಸರಹದ್ದು ಬಿಟ್ಟು ಹೊರಗೆ ಬರಲೇ ಇಲ್ಲ ಎಂಬುದು ಅರ್ಥವಾಗದ ವಿಚಾರ. ಅಗ್ಗದ ಪ್ರಚಾರಕ್ಕೆ ಇಳಿಯುವವರೇ ಸಾಮಾಜಿಕ\-ವಾಗಿ ಮಿಂಚುವ ಇಂದಿನ ದಿನಗಳಲ್ಲಿ ಗಂಗಾಧರ ಭಟ್ಟರ ಪ್ರತಿಭೆ ಪ್ರಕಾಶಕ್ಕೆ ಬಾರದೇ ಹೋದುದು ಅಚ್ಚರಿಯಲ್ಲ. ಆದರೆ ಅಂಥ ಪ್ರಸಿದ್ಧಿಗಳೆಲ್ಲ ಕ್ಷಣಿಕ. ಅವರು ವಿದ್ಯಾರ್ಥಿಗಳಿಗೆ ಧಾರೆಯೆರೆದ ವಿದ್ಯೆ, ಹತ್ತಿರಕ್ಕೆ ಬಂದವರ ಮೇಲೆ ಉಂಟುಮಾಡಿರುವ ಪ್ರಭಾವ ಮಾತ್ರ ಅಳಿಸಲಾಗದ್ದು. ಇಂಗ್ಲಿಷ್ ಕವಿಯೊಬ್ಬರ ಕವನದ ಸಾಲುಗಳು ನೆನಪಿಗೆ ಬರುತ್ತವೆ. ‘ದೇ ಆಲ್ಸೋ ಸರ್ವ್ ಹೂ ಸ್ಟಂಡ್ ಎಂಡ್ ವೇಟ್’. ಕಾಯಾ ವಾಚಾ ಮನಸಾ ವಿದ್ಯಾರ್ಥಿಗಳ ನೈಜಮಾಸ್ತರರಾಗಿ ಎಲ್ಲಕ್ಕಿಂತ ಮುಖ್ಯವಾಗಿ ನೈಜ ವಿದ್ಯಾವಂತರಾಗಿ, ಡೋಂಗಿತನದ ಲವಲೇಶವೂ ಹತ್ತಿರ ಸುಳಿಯಗೊಡದೇ ತಮ್ಮ ಪಾಡಿಗೆ ತಾವು ಸೇವೆ ಮಾಡಿದ ಸಂಭಾವಿತ ಗಂಗಾಧರ ಭಟ್ಟರು. 

ಅವರು ಸೇವೆಯಿಂದ ನಿವೃತ್ತರಾದರೂ ಅವರ ವಿದ್ವತ್ತಿಗೆ  ನಿವೃತ್ತಿ ಇರಲಾರದು !

\articleend
}
