{\fontsize{14}{16}\selectfont
\chapter{ಭರತ ಮಾತೆ ನಲಿಯಲಿ}

\begin{center}
\Authorline{ಡಾ॥ ಪಿ.ಎಚ್. ವಿಜಯಲಕ್ಷ್ಮಿ}
\smallskip
ಸಹಾಯಕ ಪ್ರಾಧ್ಯಾಪಕರು, ಸಾಹಿತ್ಯ ವಿಭಾಗ \\
ಸರ್ಕಾರಿ ಮಹಾರಾಜ ಸಂಸ್ಕೃತ ಕಾಲೇಜು \\
ಮೈಸೂರು
\addrule
\end{center}
\begin{verse}
\underline{ಗಂ}ಗೆ ಹರಿದ ಪುಣ್ಯ ಪೃಥಿವಿಗಿಲ್ಲ ಜ್ಞಾನವಿರಹ ಭಯ\\
ಮಮತೆ ಮಡಿಲ ನಾಡಿನಲ್ಲಿ ಮಿಂಚಿಮೆರೆದ ಧ್ಯಾನಜಯ\\
ಕಿರಿಜೀವದ ಹಿರಿಯರಿವು ಬುದ್ಧನಂತ ಬುದ್ಧಿಯೊಡಲು\\
ತಿಳಿದ ಆಳದಿಳಿವ ತಿಳಿವು ಬಿತ್ತರದಿ ನೀಲ್ಗಡಲು\\
ಮಿನುಗಿ ಮುಖದಿ ಮಂದಹಾಸ ಮೀರಿ ಉಲಿಯೆ ವಾಗ್ವಿಲಾಸ\\
ಗ್ರಹಿಕೆ ಬಗೆವ ವರ್ಣಪ್ರಾಸ ಅಂತರಾಳದಲೊಂದು ಹಾಸ
\end{verse}
\begin{verse}
\underline{ಗಾ} ಯತ್ರಿಯ ಮಂತ್ರಶಕ್ತಿ ಪ್ರವಹಿಸಿಹುದು ಪ್ರವಾಹವಾಗಿ\\
ಶಾಸ್ತ್ರದೊಲವ ಧೀಶಕ್ತಿ ಧುಮುಕಿಹುದು ಝರಿ  \enginline{-}  ತೊರೆಯಾಗಿ\\
ಚರಕ ವೈದ್ಯ ಸುಶ್ರುತನ ಶಸ್ತ್ರ ಮೇಳೈಸಿದ ಸ್ವಾಸ್ಥ್ಯಗುಟ್ಟು\\
ಪದಬ್ರಹ್ಮ ಪಾಣಿನಿಯ ನಂಟು ಸುತ್ತ ನಿರುಕ್ತ ನೆಲಗಟ್ಟು\\
ಆತ್ಮಬೆಳಕು ಬೆಳಗುವಂತ ದಿವ್ಯದೀಪ್ತಿಗೆಂತ ಉಪಮೆ\\
ಅರಿತ ಅರಿವಿನಾಚೆ ನಿಂತ ಸ್ವಪ್ರಕಾಶ ಧೀರಪ್ರತಿಮೆ
\end{verse}
\begin{verse}
\underline{ಧ} ರಿಸಿ ಜಲವ ಜಗವ ಕಾಯ್ದ ಗಂಗಾಧರನೊಲುಮೆಗೆ\\
ಧನ್ಯವಾಗಿ ಧರಣಿಯೊಡಲು ಶಯನವಾಯ್ತು ಸುಮತಿಗೆ\\
ಬರಿದೆ ಬಾಳಲಿಲ್ಲ ಇಳೆ ಕೊಳೆಯ ತೊಳೆದು ಹೊಳೆದಳು\\
ತಿರುಗಿ ಕರಗಲಿಲ್ಲ ಶಿಲೆ ಒಲವ ಬೇಡಿ ಪಡೆಯಲು\\
ಪದಜಾಲ ಪದ್ಯದೋಘ ನಳನಳಿಸುತಿಹ ಚೆಂದವೇನು\\
ಮಮತೆ ಸುರಿವ ಮಾತ ಮೋಡಿ ಮನದುಂಬುವ ಪರಿಯೇನು?
\end{verse}
\begin{verse}
\underline{ರ} ಮ್ಯನೋಟ ದೀವಟಿಗೆಗೆ ದಿಗಿಲು ಅಂತರಂಗದೊಳಗೆ\\
ಶಾಂತಿದೂತ ಕಂಡೊಡನೆ ಸಹಜಬಡಿತ ಗೂಡಿನೊಳಗೆ\\
ಧರ್ಮದೊಲವು ನಡೆಗೆಂತೋ ಕಾವ್ಯದೊಲವು ನುಡಿಗೆ ಬಂಧು\\
ಇದ್ದೂ ಇರದ ಭವದಲೇಕೋ ಇಲ್ಲವಾದ ಸಂಬಂಧ ಸಿಂಧು\\
ಇದ್ದರೇನು ಕಾಯ ಕೃಶದಿ ಉಳಿದೂ ಅಳಿವ ಕೃತಕತೆ\\
ಬುದ್ಧಿಬಲಕೆ ಸಾಟಿಯೇನು ಅಳಿದೂ ಉಳಿವ ಸಹಜತೆ
\end{verse}
\begin{verse}
\underline{ಭ} ರತಮಾತೆ ಪರಮಪುನೀತೆ ಸಂತಶೂಲ ಪ್ರಸವಿಸಿ\\
ನಿರತ ದುಡಿಮೆ ಭಾಗ್ಯ ಬಯಸಿ ಶಿಷ್ಯಕೋಟಿ ಕುಸುಮಿಸಿ\\
ಭಗೀರಥಯತ್ನಬಲದ ಶ್ರವಣ ಫಲಿಸಿ ಧೀಖನನ\\
ಅವ್ಯಾಹತ ಮನನವೀಯೆ ಅಪ್ರತಿಮಮತಿ ಪ್ರಜನನ\\
ಸ್ಪಟಿಕ ನುಡಿಯ ಉಲಿವ ಕಲೆಗೆ ಶರಣು ಸಾವಿರೊಂದು\\
ಮನದಿ ತುಡಿವ ಕಲಿವ ಸೆಲೆಗೆ ನಿಜದಿ ನೀನೆ ಬಂಧು
\end{verse}
\begin{verse}
ದ\underline{ಟ್ಟ} ಅರಿವು ದಿಟ್ಟ ನಿಲುವು ಒಳ  \enginline{-}  ಹೊರಗಿನ ಏಕತೆ\\
ಸತ್ತ ಸೆಳೆತ ಬಿದ್ದ ಬಯಕೆ ಹಿತ್ತಲಗಿಡ ಸಾಮ್ಯತೆ\\
ಮಹಾಪ್ರಮಾದ ಕೊರಗಿದ್ದೇ ಫಲ ಮರಳದೆಂದಿಗೆಂದಿಗೂ\\
ಬೋಧಿವೃಕ್ಷದ ಛಾಯಾಸಂಭ್ರಮ ಅಳಿಯದುಳಿಯಲೆಂದಿಗೂ\\
ಪ್ರಜ್ಞಾಜ್ಯೋತಿಯ ಪ್ರಭಾವಲಯದ ಸಾಕ್ಷಿಭಾಗ್ಯ ಫಲರೂಪ\\
ಎಣ್ಣೆ ಮರೆತು ಕಣ್ಣಮುಚ್ಚಿ ತಂಪತೊರೆದ ಪ್ರಪರಿತಾಪ
\end{verse}
\begin{verse}
ಗುಣಗಣನೇ ಲಂಬೋದರ ದ್ವಿಗುಣಗೊಳಿಸು ಸುಗುಣಿಯ\\
ಸನ್ಮತಿಯ ಸರಸ್ವತಿಯೆ ನಿಶಿತಗೊಳಿಸು ಸುಮತಿಯ\\
ಜಗಕೆ ಮೂಲ ಚತುರ್ಮುಖನೆ ನಾಲ್ಕಾಗಿಸು ನಲುಮೊಗವ\\
ನರಸಖ ನೀ ನಾರಾಯಣ ನೆಲೆಗೊಳಿಸು ನ್ಯಾಯಬಲವ\\
ಬೆರಗಾಗಲಿ ಛಾತ್ರನಿವಹ ಬೆಳಕಾಗಲಿ ದೇವವಾಣಿ\\
ನವಯುಗದಿ ನಿಧಿಯಾಗಲಿ ನೂರ್ಕಾಲ ಬಾಳಿ ಬೆಳಗಲಿ
\end{verse}
\articleend
}
