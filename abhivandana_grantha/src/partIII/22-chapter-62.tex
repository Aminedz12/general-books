{\fontsize{14}{16}\selectfont
\chapter{ಗುರವೋ ವಿರಲಾಸ್ಸಂತಿ ಶಿಷ್ಯ\enginline{-}ಚಿತ್ತಾಪಹಾರಕಾಃ}

\begin{center}
\Authorline{ವಿ~॥  ಶ್ರೀಪಾದ ಭಟ್ಟ}
\smallskip

ನಿವೃತ್ತ ಅಧ್ಯಾಪಕರು, ಸಂಸ್ಕೃತ ವಿಭಾಗ\\
ಶಾರದಾವಿಲಾಸ ಪ್ರೌಢಶಾಲೆ \\ 
ಮೈಸೂರು
\addrule
\end{center}

ಸಮಾಜದಲ್ಲಿ ಮಾನವನಿಗೆ ಬದುಕಿಗಾಗಿ ಅನೇಕ ಮಾರ್ಗಗಳಿವೆ, ವೃತ್ತಿಗಳಿವೆ. ಆದರೆ ಅವುಗಳಲ್ಲಿ ಅನೇಕ ವೃತ್ತಿಗಳು ಕೇವಲ ಉದರಂಭರಣಕ್ಕಾಗಿಯೇ ಇವೆ. ಕೆಲವು ವೃತ್ತಿಗಳು ಮಾತ್ರ ಸಮಾಜದ ಒಳಿತಿಗಾಗಿಯೂ ಇವೆ. ಅಂತಹವುಗಳಲ್ಲಿ ಮುಖ್ಯವಾಗಿ ವೈದ್ಯವೃತ್ತಿ ಹಾಗೂ ಉಪಾಧ್ಯಾಯ ವೃತ್ತಿಗಳನ್ನು ಉಲ್ಲೇಖಿಸಬಹುದಾಗಿದೆ. ಇವುಗಳಲ್ಲಿ ಒಂದು ಪ್ರಾಣವನ್ನು ಕಾಪಾಡುವುದಾದರೆ ಮತ್ತೊಂದು ಮನುಷ್ಯನ ಮಾನವನ್ನು(ಗೌರವವನ್ನು) ಹೆಚ್ಚಿಸುವುದಾಗಿದೆ. ಯದ್ಯಪಿ ಲೋಕದಲ್ಲಿ “ಉಪಾಧ್ಯಾಯಾಶ್ಚ ವೈದ್ಯಾಶ್ಚ ಕಾರ್ಯಾಂತೇ ನಿಷ್ಪ್ರಯೋಜಕಾಃ” ಎಂಬ ತಮಾಷೆಯ ಉಕ್ತಿಯೊಂದು ಸಂಸ್ಕೃತದಲ್ಲಿದೆ. ಆದರೂ ಅದು ವಾಸ್ತವವಾದುದಲ್ಲ. ಉಪಾಧ್ಯಾಯರನ್ನು ಗುರುವೆಂದೇ ಹೇಳುವುದು ರೂಢಿ. ಈ ಶಬ್ದಕ್ಕೆ ವಿಷೇಶ ಗೌರವವಿದೆ. ಹೆಚ್ಚೇಕೆ ಸಮಾಜದ ಯಾವುದೇ ವ್ಯಕ್ತಿಯಾದರೂ ತನ್ನ ಗುರುವನ್ನು ಒಮ್ಮೆಯಾದರೂ ಸ್ಮರಿಸದೇ ಇರಲಾರ. ಅಂದಾಗ ಅಂತಹ ಸ್ಥಾನವನ್ನು ಅಲಂಕರಿಸಿದ ವ್ಯಕ್ತಿಗಳ ಜವಾಬ್ದಾರಿ ಎಷ್ಟು ಹಿರಿದೆಂಬುದನ್ನು ಊಹಿಸಿಕೊಳ್ಳಬಹುದಾಗಿದೆ. ಸುಸಂಸ್ಕೃತ ಭಾವೀ ಸಮಾಜದ ರೂವಾರಿ ಗುರುವಾಗಿರುತ್ತಾನೆ. ಆದರೆ ಈ ವೃತ್ತಿಯನ್ನು ಅವಲಂಬಿಸಿದವರೆಲ್ಲರೂ ಯಶಸ್ವಿಯಾಗುತ್ತಾರೆಂದರೆ ಅದು ಅತಿಶಯೋಕ್ತಿಯೇ ಸರಿ.

ಇಂದಿನ ವಿದ್ಯಾಭ್ಯಾಸ ಕ್ರಮವಂತೂ ಪ್ರಾಚೀನವಾದ ಗುರು ಶಿಷ್ಯ ಪರಂಪರೆಯನ್ನು ಸಂಪೂರ್ಣ ಮರೆತಂತಿದೆ. ಆದರೆ ನಮ್ಮ ಸಂಸ್ಕೃತ ಪಾಠಶಾಲೆಗಳಲ್ಲಿ ಆ ಪರಂಪರೆ ಇದುವರೆಗೂ ಮುಂದುವರೆದು ಬಂದಿರುವುದು ಸೋಜಿಗವಾದರೂ ಸತ್ಯವಾಗಿದೆ.

ಇಷ್ಟೆಲ್ಲಾ ಪೀಠಿಕೆ ಹಾಕುವ ಉದ್ದೇಶವೇನೆಂದರೆ ನನ್ನ ಮಿತ್ರರಾದ ವಿ॥ \hbox{ಗಂಗಾಧರ} ಭಟ್ಟರು, ತರ್ಕಶಾಸ್ತ್ರದ ಪ್ರಾಧ್ಯಾಪಕರು ಜನವರಿ ೩೧ಕ್ಕೆ ವಯೋಮಾನ ನಿಮಿತ್ತ\break ವೃತ್ತಿಯಿಂದ ನಿವೃತ್ತಿ ಹೊಂದುತ್ತಿದ್ದಾರೆಂದು ತಿಳಿದು ಬಂತು. ಇದು ಸಹಜ ಪ್ರಕ್ರಿಯೆ\break  ಎನಿಸಿದರೂ ಅವರ ಮತ್ತು ನನ್ನ ಬಾಂಧವ್ಯದ ಬಗ್ಗೆ ಈ ಸಂದರ್ಭದಲ್ಲಿ ನಾಲ್ಕು ಮಾತುಗಳನ್ನು ಬರೆಯದಿದ್ದರೆ ನನ್ನ ಮನಸ್ಸಿಗೆ ಸಮಾಧಾನವಿಲ್ಲವಾಗುತ್ತದೆ. ಶ್ರೀಯುತರು ಅನೇಕ ವರ್ಷಗಳಿಂದ ಮೈಸೂರಿನ ಶ್ರೀಮನ್ಮಹಾರಾಜ ಸಂಸ್ಕೃತ ಮಹಾಪಾಠಶಾಲೆಯಲ್ಲಿ ಸೇವೆ ಸಲ್ಲಿಸುತ್ತಿದ್ದುಎಲ್ಲರಿಗೂ ಬೇಕಾದ, ವಿಷೇಶವಾಗಿ ವಿದ್ಯಾರ್ಥಿವೃಂದಕ್ಕೆ ಬೇಕಾದ ವಿಶಿಷ್ಟ ವ್ಯಕ್ತಿಯಾಗಿದ್ದಾರೆ. ಅವರ ವ್ಯಕ್ತಿತ್ವವೇ ಹಾಗೆ. ಎಲ್ಲ ವಿಷಯಗಳಲ್ಲಿಯೂ \textbf{‘ಕಡಕ್’} ಎನ್ನಬಹುದಾದ ವ್ಯಕ್ತಿತ್ವ. ಉಪಾಧ್ಯಾಯರಾದ ಅನೇಕರಲ್ಲಿ ಪಾಂಡಿತ್ಯ ಇರಬಹುದು ಆದರೆ ವ್ಯಾವಹಾರಿಕ ಚತುರತೆಯೂ ಇದ್ದೇ ಇರುತ್ತದೆಂದು ಹೇಳಲಾಗದು. ಹಿಂದಿನಿಂದಲೂ ನಾನು ಗಮನಿಸಿದಂತೆ ನಮ್ಮ ಪಾಠಶಾಲೆಯ ಉಪಾಧ್ಯಾಯ ವೃಂದದಲ್ಲಿ ಕೆಲವರು ಪ್ರಕಾಂಡ ಪಂಡಿತರಾಗಿದ್ದರೂ ಲೌಕಿಕ ವ್ಯವಹಾರದಲ್ಲಿ ಸ್ವಲ್ಪ ಹಿಂದುಳಿದವರೇ ಆಗಿದ್ದರೆಂದರೆ ತಪ್ಪಾಗಲಾರದು. ಅವರ ಪಾಂಡಿತ್ಯ, ಪಾಠ ಪ್ರವಚನ ಹಾಗೂ ಅವರದೇ ಆದ ಶಿಷ್ಯವೃಂದ ಇಷ್ಟಕ್ಕೇ ಸೀಮಿತವಾಗಿರುತ್ತಿತ್ತು. ಆದರೆ ಗಂಗಾಧರ ಭಟ್ಟರು ಶಾಸ್ತ್ರಪಾಂಡಿತ್ಯ ಹಾಗೂ ಲೌಕಿಕಜ್ಞಾನ ಎರಡರಲ್ಲಿಯೂ \textbf{‘ಸೈ’} ಎನಿಸಿಕೊಂಡವರಾಗಿ ಸವ್ಯಸಾಚಿ ಆಗಿದ್ದಾರೆ. ಅವರ ಸೇವಾವಧಿಯಲ್ಲಿ ಸಂಸ್ಕೃತ ಪಾಠಶಾಲೆಯಲ್ಲಿ ಯಾವುದೇ ವಿಶೇಷ ಸಮಾರಂಭಗಳು ಜರುಗಿದರೂ ಅವುಗಳ ಸೂತ್ರಧಾರಿಕೆ ಶ್ರೀ ಗಂಗಾಧರ ಭಟ್ಟರದ್ದೇ ಆಗಿರುತ್ತಿತ್ತು ಎಂಬುದನ್ನು ಅನೇಕ ಗೋಷ್ಠಿಗಳಲ್ಲಿ ನಾನು ಕಣ್ಣಾರೆ ಕಂಡಿದ್ದೇನೆ.

ಮಾನವ ಸಹಜವಾದ ಕೋಪತಾಪಾದಿಗಳು ಅವರಲ್ಲಿಯೂ ಇವೆ. ಅವು ಭೂಷಣವೇ ಹೊರತು ದೂಷಣವಲ್ಲ. “ಜ್ವಲಿತಂ ನ ಹಿರಣ್ಯರೇತಸಂ ಚಯಮಾಸ್ಕಂದತಿ ಭಸ್ಮನಾಂಜನಮ್” ಎಂಬ ಭಾರವಿಯ ಮಾತು ಈ ಸಂದರ್ಭದಲ್ಲಿ ಜ್ಞಾಪಕಕ್ಕೆ ಬರುತ್ತದೆ. ಅವರ ಸೇವಾಕಾಲದಲ್ಲಿದ್ದ ಎಲ್ಲ ಪ್ರಾಂಶುಪಾಲರಿಗೂ ಇವರ ಸಹಾಯ ಹಸ್ತ ಬೇಕಾಗುತ್ತಿತ್ತು. ವಿಶೇಷ ಜವಾಬ್ದಾರಿಯನ್ನು ಇವರಿಗೆ ಅವರು ಒಪ್ಪಿಸುತ್ತಿದ್ದರೆಂಬುದು ಎಲ್ಲರಿಗೂ ತಿಳಿದ ವಿಷಯವೇ ಆಗಿದೆ. ಶ್ರೀಯುತರು ಆ ಜವಾಬ್ದಾರಿಯನ್ನು ಅತ್ಯಂತ ಸಮರ್ಥವಾಗಿ ನಿರ್ವಹಿಸುತ್ತಿದ್ದರು.

ಹಿಂದೆ ಶ್ರೀಯುತರು ಶ್ರೀಶಂಕರವಿಲಾಸ ಪಾಠಶಾಲೆಯಲ್ಲಿ ಶಿಕ್ಷಕರಾಗಿ ಕಾರ್ಯವನ್ನು ನಿರ್ವಹಿಸುತ್ತಿದ್ದ ಕಾಲದಲ್ಲಿ ಪಾಠಶಾಲೆಗಳ ಉಳಿವಿಗಾಗಿ ಹಾಗೂ ಉನ್ನತಿಗಾಗಿ ಮುಂದಾಳತ್ವವಹಿಸಿ ಹೋರಾಟ ನಡೆಸಿದ್ದುದು ಅನೇಕರಿಗೆ ಗೊತ್ತು. ಇವರು ಇಂಗ್ಲಿಷ್ ನಲ್ಲಿಯೂ ಉತ್ತಮ ಜ್ಞಾನ ಪಡೆದಿದ್ದರಿಂದ ವ್ಯಾವಹಾರಿಕವಾಗಿ ಏನೆಲ್ಲ ಸಮಸ್ಯೆಗಳು ಬಂದರೂ ಅವನ್ನು ಪರಿಹರಿಸುವ ಚಾಕಚಕ್ಯತೆ ಅವರಲ್ಲಿದೆ. ಅವರಲ್ಲಿರುವ ನಾಯಕತ್ವ ಗುಣ ಬಹುಶಃ ಜನ್ಮತಃ ಬಂದಿದ್ದೇ ಇರಬೇಕು.

ಶಿಷ್ಯರ ಮನಗೆದ್ದು ಪಾಠ ಮಾಡುವುದು ಸುಲಭದ ಮಾತಲ್ಲ. ಅದರಲ್ಲಿಯೂ ಈ ಕಾಲದಲ್ಲಿ ವಿದ್ಯಾರ್ಥಿಗಳ ಮನಸ್ಸನ್ನು ಗೆಲ್ಲುವುದು ಸುಲಭವಲ್ಲ. ಏಕೆಂದರೆ ಇಂದಿನ ವಿದ್ಯಾರ್ಥಿಗಳು ಕೇವಲ ಮುಗ್ಧರಲ್ಲ. ಅವರು ಹೊರ ಜಗತ್ತನ್ನು ನೋಡುತ್ತಿದ್ದಾರೆ. ಇಂತಹ ಕಾಲದಲ್ಲಿ ಶಿಷ್ಯರ ಮನಸ್ಸನ್ನು ಸೆಳೆದು ಉತ್ತಮ ಪಾಠ ಮಾಡಿ ಅವರಿಂದ ಇಷ್ಟೊಂದು ಗೌರವ ಸಂಪಾದಿಸಿರುವುದಕ್ಕೆ ಈ ಸ್ಮರಣಸಂಚಿಕೆ ಹಾಗೂ ಬೀಳ್ಕೊಡುಗೆ\break ಸಮಾರಂಭಗಳನ್ನು ಆ ಶಿಷ್ಯಂದಿರೇ ನಡೆಸಿರುವು ಸಾಕ್ಷಿಯಾಗಿದೆ.

ಶ್ರೀಯುತರು ಕೇವಲ ಶಿಷ್ಯರ ಮನಸ್ಸನ್ನು  ಮಾತ್ರ ಗೆದ್ದಿರುವುದಲ್ಲ. ಅವರು ತಮ್ಮ ಗುರುಗಳ ಮನಸ್ಸನ್ನೂ ಗೆದ್ದವರು. ಎಲ್ಲ ಉಪಾಧ್ಯಾಯರಿಗೂ “ಗಂಗಾಧರ” ಎಂದರೆ ಇಷ್ಟ. ಇವರು ಅನೇಕ ಸಂದರ್ಭಗಳಲ್ಲಿ ಅವರ ಕಷ್ಟವನ್ನು ಪರಿಹರಿಸುತ್ತಿದ್ದರು. ಗಂಗಾಧರ ಭಟ್ಟರು ನಮ್ಮೆಲ್ಲರ ಗುರುಗಳಾದ ತುರುವೇಕೆರೆ ವಿ॥ ವಿಶ್ವೇಶ್ವರ ದೀಕ್ಷಿತರು ಹಾಗೂ ವಿ॥ ವೆಂಕಣ್ಣಾಚಾರ್ಯರಂತಹ ಗುರುವೃಂದಕ್ಕೆ ಅವರ ಕಷ್ಟಕಾಲದಲ್ಲಿ ಸಾಕಷ್ಟು ಸಹಾಯ ಹಸ್ತವ ನೀಡಿದ್ದಾರೆಂದು ನಾನು ತಿಳಿದಿದ್ದೇನೆ. ಒಟ್ಟಾರೆ ಅವರ ಉಪಾಧ್ಯಾಯ ವೃತ್ತಿ ಹಾಗೂ ಬದುಕು ಸಾರ್ಥಕತೆ ಪಡೆದು ಧನ್ಯವಾಗಿದೆ ಎಂದರೆ ಅತಿಶಯೋಕ್ತಿ ಆಗಲಾರದು.

ಅನೇಕ ಸಂದರ್ಭಗಳಲ್ಲಿ ನಾನು ವೈಯಕ್ತಿಕವಾಗಿಯೂ ಅವರಿಂದ ಸಹಾಯವನ್ನು ಪಡೆದಿದ್ದೇನೆ. ಎಂದೂ ಬೇಸರ ಪಡದೇ ಅವರು ಸಹಾಯ ಮಾಡಿದ್ದಾರೆ. ಅದನ್ನು ಎಂದಿಗೂ ಮರೆಯಲಾರೆ. ಅವರ ಈ ಬಾಂಧವ್ಯ ಎಂದೆಂದಿಗೂ ಇರಲೀ ಎಂದು ಕೇಳಿಕೊಳ್ಳುತ್ತೇನೆ. ಅಂತೆಯೇ ಅವರ ವಿಶ್ರಾಂತ ಜೀವನವು ನೆಮ್ಮದಿಯಿಂದ ತುಂಬಿರಲಿ ಎಂದು ಹಾರೈಸುತ್ತಾ ಭಗವಂತನಿಗೆ ನಮಸ್ಕರಿಸುತ್ತಾ ನನ್ನೆರಡು ಮಾತುಗಳನ್ನು ಮುಗಿಸುತ್ತೇನೆ.

\articleend
}
