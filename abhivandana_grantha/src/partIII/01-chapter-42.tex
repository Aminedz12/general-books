{\fontsize{14}{16}\selectfont
\chapter{ನನ್ನ ಸೂರು ಮೈಸೂರು}

\addtocontents{toc}{\protect\contentsline{section}{ವಿ ॥ ಗಂಗಾಧರ ವಿ.ಭಟ್}{}}

ಮೈಸೂರು ನನ್ನ ಪಾಲಿಗೆ ಕೇವಲ ಒಂದು ಊರಲ್ಲ, ಇದು ನನ್ನ ಸೂರು.  ನನ್ನ \hbox{ಬದುಕನ್ನು} ಸಹನೀಯವಾಗಿ ರೂಪುಗೊಳಿಸಿ, ಸೂರು ಹೇಗೆ ಮಳೆ, ಚಳಿ, ಬಿಸಿಲಿನಿಂದ ರಕ್ಷಣೆ \hbox{ನೀಡುತ್ತದೆಯೋ} ಹಾಗೆಯೇ ನನ್ನ ಬಾಳಿನ ಗೋಳನ್ನು ಕಳೆದು ಹುರಿದುಂಬಿಸಿದ ಊರು.

\section*{ಇನ್ಮುಂದೇನು ?} 

ಪ್ರೌಢಶಿಕ್ಷಣ ಪೂರೈಸಿದ ಅನಂತರ ಮುಂದೇನು? ಎಂಬ ಪ್ರಶ್ನೆ ಮೂಡಿತು.  ನನ್ನ ತಂದೆಯವರು ವೇದಾಧ್ಯಯನವನ್ನು ಮಾಡಿಸುವ ದೃಷ್ಟಿಯುಳ್ಳವರಾಗಿದ್ದರು.  ಒಂದು ದಿನ ಪ್ರಾತಃಕಾಲದಲ್ಲಿ ಅಮ್ಮನ ಹತ್ತಿರ ಹೇಳುತ್ತಿದ್ದುದನ್ನು ಕೇಳಿಸಿಕೊಂಡೆ.  “ಮೀಸೆ \hbox{ಮೂಡುತ್ತಿದೆ,} ಏನು ಮಾಡಬೇಕೆಂಬುದೇ ಗೊತ್ತಿಲ್ಲದೇ ಬೆಳೆಯುತ್ತಿದ್ದಾನೆ” ಎಂದು.   ನನ್ನ ಸ್ವಾಭಿಮಾನವನ್ನು ಈ ಮಾತು ಕೆರಳಿಸಿತ್ತು.  ಆದರೆ ಏನು ಮಾಡಬೇಕೆಂಬ ಬಗ್ಗೆ ಸ್ಪಷ್ಟ ಚಿತ್ರವಿರಲಿಲ್ಲ.  ಹಾಗೆಯೇ ಬೆಳೆಯುತ್ತಿದ್ದೆ.  ಒಂದು ವಾರದಲ್ಲಿ ತಂದೆಯವರು ಹೇಳಿದರು.  ಸೂರಿ ರಾಮಚಂದ್ರ ಶಾಸ್ತ್ರಿಗಳ ಹತ್ತಿರ ಮಾತನಾಡಿ ಬಂದಿದ್ದೇನೆ, ಸುಬ್ರಹ್ಮಣ್ಯ \hbox{ಪಾಠಶಾಲೆಗೆ} ಹೋಗು ಎಂದು ಆದೇಶಿಸಿದರು. ಅದರಂತೆಯೇ ಒಂದು ಪೆಟ್ಟಿಗೆ ತುಂಬಿಕೊಂಡು \hbox{ಸುಬ್ರಹ್ಮಣ್ಯ} ಪಾಠಶಾಲೆಗೆ ಹೊರಟೇಬಿಟ್ಟೆ.  

ಉದಾರಿಗಳಾದ ಶ್ರೀ ಸೂರಿ ರಾಮಚಂದ್ರಶಾಸ್ತ್ರಿಗಳು ತಮ್ಮ ಮನೆಯಲ್ಲಿಯೇ ನನ್ನನ್ನು ಉಳಿಸಿಕೊಂಡು ಉಳಿದ ವಿದ್ಯಾರ್ಥಿಗಳಿಂದ ಪಾಠಮಾಡಿಸತೊಡಗಿದರು.  ಹಿರಿಯ ವಿದ್ಯಾರ್ಥಿ ಸಣ್ಣಗದ್ದೆ ಶ್ರೀಪತಿ ಭಟ್ಟರು ಶ್ರೀಸೂಕ್ತ ಪಾಠವನ್ನು ಬೆಳಗಿನ ಜಾವ 5 ಗಂಟೆಗೆ ಪ್ರಾರಂಭಿಸಿದ್ದು ಈಗಲೂ ನನ್ನ ಕರ್ಣಕುಹರದಲ್ಲಿ  ಅನುರಣಿಸುತ್ತಿದೆ.  ಒಂದು ವಾರ ಅಲ್ಲಿಯೇ ವಾಸ.  ಉಳಿದ ಕೆಲವು ವಿದ್ಯಾರ್ಥಿಗಳೊಂದಿಗೆ ಬೇರೆಡೆ ವಾಸವ್ಯವಸ್ಥೆ ಮಾಡಿಕೊಳ್ಳಲು ಏರ್ಪಾಡಾಯಿತು.  ಒಪ್ಪಿ ಉಳಿದೆ.  ಒಂದು ಅನಧ್ಯಯನ ಬಂತು.  ನಾವೆಲ್ಲ ವಿದ್ಯಾರ್ಥಿಗಳು ಹಿತ್ತಲಿನಲ್ಲಿ ಕಳೆ ಕೀಳಲು ನಿಯುಕ್ತರಾದೆವು.  ಕಳೆಯನ್ನು ಕೀಳಲು ಇಷ್ಟು ದೂರ ಬರಬೇಕಿತ್ತೇ? ಮನೆಯ ಗದ್ದೆ, ತೋಟದಲ್ಲಿ ಕಿತ್ತು ಮುಗಿಸಲಾರದಷ್ಟು ಕಳೆಯಿದೆಯಲ್ಲಾ ಎಂದು ಅಂತಃಕರಣ ಕುದಿಯಿತು.  ಮನೆಗೆ ಹೊರಟೇ ಬಿಟ್ಟೆ.  ಅಲ್ಲಿಗೆ ಹೋಗುವಾಗ ತೆಗೆದುಕೊಂಡು ಹೋದ ಪೆಟ್ಟಿಗೆ ಅಲ್ಲಿಯೇ ಉಳಿಯಿತು. ಆಮೇಲೆ ನನ್ನ ಶ್ರೀಧರಣ್ಣ ಹೋಗಿ ಆ ಪೆಟ್ಟಿಗೆಯನ್ನು ತಂದರು. ಆಗ ಶಾಸ್ತ್ರಿಗಳು ಕೇಳಿದರಂತೆ\break  “ಗಂಗಾಧರನ ವಿದ್ಯಾಭ್ಯಾಸ ಮುಗಿಯಿತೇ” ಎಂದು. ಅದನ್ನು ಹಾಗೆಯೇ ನನಗೆ ವರದಿ ಮಾಡಿದ್ದರು. 

ಇಷ್ಟು ಶೀಘ್ರದಲ್ಲಿ ಮನೆಗೆ ಬಂದಿದ್ದನ್ನು ತಂದೆಯವರು ಸಹಿಸಲಿಲ್ಲ. ಹಾಗೆಂದು ನನಗೆ ಓದುವ ಹಂಬಲ ಅತಿಯಾಗಿತ್ತು.  ಆದ್ದರಿಂದ ಗೋಕರ್ಣದತ್ತ ನನ್ನ ಚಿತ್ತ ವಾಲಿತು.  ಅದಕ್ಕೆ ಕಾರಣವಿತ್ತು.  ನನ್ನಣ್ಣ ದಿವಂಗತ ಮಂಜುನಾಥ ಭಟ್ಟರು ಗೋಕರ್ಣದ ಶ್ರೀ ಗಣು ಭಟ್ಟರಲ್ಲಿ ಅಧ್ಯಯನ ಮಾಡಿದ್ದರು. ಈಗ ಅವರು ಇಲ್ಲ.  ಅವರ ತಮ್ಮ ಶ್ರೀ ದತ್ತ \hbox{ಭಟ್ಟರಲ್ಲಿ} ವೇದಾಧ್ಯಯನಕ್ಕೆ ಹೋಗಬೇಕೆಂದು ನಿರ್ಣಯಿಸಿದೆ.  ನಮ್ಮ ಮನೆಗೆ ಸದಾ ಬರುತ್ತಿದ್ದ ರಮಣಿ ಗಣೇಶ ಭಟ್ಟರು ತಮ್ಮ ಮನೆಯಲ್ಲಿಯೇ ವಾಸಮಾಡಿ ಓದಲು \hbox{ತಿಳಿಸಿದರು,}  ಒಪ್ಪಿದೆ.  ರಮಣಿ ಗಣೇಶಭಟ್ಟರ ಕುಟುಂಬದವರು ನನ್ನನ್ನು ಅತ್ಯಂತ ಪ್ರೀತಿ ವಿಶ್ವಾಸದಿಂದ ನೋಡಿಕೊಂಡರು.  ಪೂಜ್ಯ ದತ್ತಭಟ್ಟರು ಅತ್ಯಂತ ಪ್ರೀತಿಯಿಂದ ಪಾಠ ಪ್ರಾರಂಭಿ\-ಸಿದರು.  ಅಲ್ಲಿಯೂ ಶ್ರೀ ಯಜ್ಞಪತಿ ಭಟ್ಟರು ಪ್ರಾರಂಭಿಕ ವೇದ ಪಾಠ \hbox{ಮಾಡಿದರು.}  ಅಂತೆಯೇ ಶಿಕ್ಷಾ, ಬ್ರಹ್ಮ, ಭೃಗು, ಚಿತ್ತಿ, ನವಗ್ರಹ, ಉದಕಶಾಂತಿ, ದೇವಪೂಜಾ \hbox{ಮಂತ್ರಗಳ} ಪಾಠ\-ವಾಯಿತು.  ಪ್ರಾತಃಕಾಲದಲ್ಲೆದ್ದು ಕೋಟಿತೀರ್ಥದಲ್ಲಿ ಈಜಾಡಿ ನಿತ್ಯಕರ್ಮ ಪ್ರಾರಂಭವಾಗುತ್ತಿತ್ತು.  ಉತ್ಸಾಹದಿಂದ ಅಧ್ಯಯನ  \enginline{-}  ಅಧ್ಯಾಪನ ನಡೆಯಿತು.  ಆದರೆ ವೇದಾಧ್ಯಯನದ ಹೊರತಾಗಿ ಇನ್ನಾವ ಶಿಕ್ಷಣಕ್ಕೂ ಅಲ್ಲಿ ಮುಕ್ತಾವಕಾಶವಿರಲಿಲ್ಲ.  \hbox{ಮನಸ್ಸಿನಲ್ಲಿ} ವೇದದೊಂದಿಗೆ ಸಂಸ್ಕೃತ ಹಾಗೂ ಸಾಮಾನ್ಯ ಶಿಕ್ಷಣ ಪಡೆವ ಉತ್ಕಟೇಚ್ಛೆ ದಿನೇ ದಿನೇ ವರ್ಧಿಸುತ್ತಿತ್ತು.  ಪ್ರತಿನಿತ್ಯ ಸ್ನಾನಾನಂತರ ಸಂಧ್ಯಾವಂದನೆ ಮಾಡಿ ಮಹಾಬಲೇಶ್ವರ ದೇವಾಲಯಕ್ಕೆ ಹೋಗಿ ನಮಿಸಿ ಬರುವುದು ಪದ್ಧತಿಯಾಗಿತ್ತು.  ಅದೊಂದು ದಿನ \hbox{ಫೆಬ್ರವರಿ} ತಿಂಗಳಲ್ಲಿ ದೇವಾಲಯಕ್ಕೇ ಹೋದಾಗ ನನ್ನ ತಂಗಿ ವೇದಾವತಿ ಕಂಡಳು.  ಮಹಾಬಲದರ್ಶನಕ್ಕಾಗಿ ಬಂದಿದ್ದಳು.  ಅವಳನ್ನು ಕಂಡತಕ್ಷಣ ಮನೆಯ ಕಡೆ ನನ್ನ ಸೆಳೆತ ಅತಿಯಾಯ್ತು.  ನೆಪವು ದೊರೆಯಿತು.  ಮನೆಗೆ ಹೋಗಿ ಬರುವೆ. ನನ್ನ ತಂಗಿ ಬಂದಿದ್ದಾಳೆ ಎಂದು ಹೇಳಿ ಗೋಕರ್ಣದಿಂದ ಮನೆಗೆ ಹಿಂದಿರುಗಿದೆ.  

ವೇದಾಧ್ಯಯನ ಮಾಡಬೇಕೆಂದರೆ ಅದಕ್ಕೆ ಅನುರೂಪವಾದ ಆಕೃತಿ ಬೇಕೆಂದು ಸ್ವ ಇಚ್ಛೆಯಿಂದಲೇ ಶಿಖೆಯನ್ನು ಹೆಮ್ಮೆಯಿಂದ ಬಿಟ್ಟಿದ್ದೆ.  ಆದರೆ ನನ್ನ ಉಪನಯನ ಸಂಸ್ಕಾರ\-ದಲ್ಲಿ ಶಿಖೆಯನ್ನು ಬಿಡಿಸಬೇಕೆಂಬ ನನ್ನ ತಂದೆಯ ಒತ್ತಾಸೆಯನ್ನು ತಿರಸ್ಕರಿಸಿ \hbox{ತಂದೆಯವರ} ಹೊಡೆತವನ್ನು ಎದುರಿಸಿದ್ದೆ.  ಮನೆಗೆ ಹಿಂದಿರುಗಿದ ಸಶಿಖನನ್ನು ಕಂಡು ತಂದೆ \enginline{-}ತಾಯಿಗೆ ಏನನಿಸಿತ್ತೋ ಗೊತ್ತಿಲ್ಲ !  ಆದರೆ, ಶಿಖೆಯ ಜುಟ್ಟು ಅವಮಾನಕರ ಎಂಬ ಎಳವೆಯ ಭಾವನೆ ಅತ್ಯಂತ ಅವಿಚಾರ ರಮಣೀಯ. ಪರರಿಂದ ಅಪಹಾಸ್ಯದ ಭೀತಿ ಶಿಖೆಯ ನಕಾರಕ್ಕೆ ಕಾರಣ ಎಂಬುದು ನನ್ನ ಅಭಿಪ್ರಾಯ.  ಗೋಕರ್ಣದಲ್ಲಿ ಜುಲೈ ತಿಂಗಳಿನಿಂದ ಫೆಬ್ರವರಿವರೆಗೆ ಎಂಟು ತಿಂಗಳು ಅರ್ಥಪೂರ್ಣ ಅಧ್ಯಯನವಾಗಿತ್ತು.  ಎಲ್ಲವೂ ಕಂಠಸ್ಥ\-ವಾಗಿದ್ದವು.  ಸ್ವತಃ ನನಗೇ ನನ್ನ ಅಧ್ಯಯನದ ಬಗ್ಗೆ ಹೆಮ್ಮೆಯಿತ್ತು.

ಗೋಕರ್ಣದಲ್ಲಿ ಶ್ರೀ ದತ್ತಭಟ್ಟರು ಮತ್ತು ಶ್ರೀ ಯಜ್ಞಪತಿಭಟ್ಟರು ಹಿಸೆಯಾಗಿ ಪ್ರತ್ಯೇಕ ವಾಸಿಸಲಾರಂಭಿಸಿದರು.  ಇದರಿಂದ ನನಗೆ ಇಕ್ಕಟ್ಟಾಯಿತು. ಶ್ರೀ \hbox{ದತ್ತಭಟ್ಟರಲ್ಲಿ} ಅಧ್ಯಯನ ಮಾಡಿದ ವಿದ್ಯಾರ್ಥಿಗಳಲ್ಲಿ ಒಂದೂ ಪೆಟ್ಟು ತಿನ್ನದೇ ಓದಿದವನು ನಾನು ಎಂಬ ಹೆಗ್ಗಳಿಕೆಯಿತ್ತು.  ಅವರಿಗೂ ನನ್ನ ಅಧ್ಯಯನ ಶೈಲಿ ಹಿಡಿಸಿತ್ತು.  ಶ್ರೀ \hbox{ಯಜ್ಞಪತಿ} ಭಟ್ಟರು ಸೌಮ್ಯಮೂರ್ತಿ. ನನಗೆ ಆಯ್ಕೆ ಅಪರಿಹಾರ್ಯವಾಯಿತು.  ಆದ್ದರಿಂದ ಅಧ್ಯಯನ\-ಕ್ಕಾಗಿ ಮೈಸೂರನ್ನು ಆಯ್ಕೆ ಮಾಡಿಕೊಂಡೆನು.

\section*{ಮನಸೂರೆಗೊಂಡ ಮೈಸೂರು}

1975ನೆಯ ಇಸವಿಯ ಜೂನ್ ತಿಂಗಳಿನಲ್ಲಿ ಸಿದ್ದಾಪುರದಿಂದ ಗುರುಮೂರ್ತಿ ಟ್ರಾವಲ್ಸ್ ಮೂಲಕ 18 ರೂಪಾಯಿ ಹಾಸಲು ತೆತ್ತು ಮೈಸೂರಿಗೆ ಬಂದಿಳಿದೆನು. ಈಗಾಗಲೇ ನನ್ನಣ್ಣ ಮೈಸೂರಿನಲ್ಲಿ ವಿಧ್ಯಾಭಾಸ ಪೂರೈಸಿ ಊರಿಗೆ ಹಿಂದಿರುಗಿದ್ದರು.  ಅವರ ಆತ್ಮೀಯರಾದ ಶ್ರೀ ವೆಂಕಟರಮಣ ಕಲಗಾರರ ಸಲಹೆ ಸಹಕಾರ, ಭರವಸೆಯಿಂದ ಮೈಸೂರು ಮಹಾರಾಜ ಸಂಸ್ಕೃತ ಪಾಠಶಾಲೆಯಲ್ಲಿ ನನ್ನ ಪ್ರವೇಶವಾಯಿತು.

ನಾನು ನನ್ನ ಪ್ರೌಢಶಿಕ್ಷಣದ ಹಂತದವರೆಗೂ ಸಂಸ್ಕೃತವನ್ನು ಓದಿರಲಿಲ್ಲ.   ಆದರೆ ಹಾಳದಕಟ್ಟಾದ ಶ್ರೀ ಸಿದ್ಧಿವಿನಾಯಕ ಸಂಸ್ಕೃತ ಪಾಠಶಾಲೆಯ ಶಿಕ್ಷಕರಾದ ಶ್ರೀ ಸುಬ್ರಹ್ಮಣ್ಯ ಮತ್ತು ಶ್ರೀ ಶ್ರೀಧರ ಭಟ್ಟರು ವಾರಕ್ಕೊಂದು ದಿನ ಬಿದ್ರಕಾನ ಹೈಸ್ಕೂಲಿಗೆ ಬಂದು ಪ್ರಥಮಾ ಪಾಠ ಮಾಡುತ್ತಿದ್ದರು.  ಹಾಗಾಗಿ ಪ್ರಥಮಾ ಪರೀಕ್ಷೆಯಲ್ಲಿ ಉತ್ತೀರ್ಣನಾಗಿದ್ದೆ.  ಮೈಸೂರು ಮಹಾರಾಜ ಸಂಸ್ಕೃತ ಕಾಲೇಜಿನಲ್ಲಿ ಆ ವೇಳೆಗೆ ಪ್ರಥಮಾ, ಕಾವ್ಯ ಅಧ್ಯಯನದ ಅವಕಾಶವಿರಲಿಲ್ಲ.  ಸಾಹಿತ್ಯ ತರಗತಿಗೆ ಸೇರಲು ನನಗೆ ಅಗತ್ಯ ಅರ್ಹತೆಯಿರಲಿಲ್ಲ.  ಅದಕ್ಕಾಗಿ ಕೃಷ್ಣಯಜುರ್ವೇದ  ತರಗತಿಗೆ ಪ್ರವೇಶ ಪಡೆದೆ.  27ನೇ ಸಂಖ್ಯೆಯ ಕೊಠಡಿಯಲ್ಲಿ ವಾಸದ ವ್ಯವಸ್ಥೆಯಾಯಿತು. ಅಂತೂ ಮೈಸೂರಿನಲ್ಲಿ ನೆಲೆ ದೊರೆಯಿತು.

ಮೈಸೂರಿಗೆ ಬರುವಾಗ 50 ಕೆ.ಜಿ ಅಕ್ಕಿ ತಂದಿದ್ದೆ.  ಬೆಳಿಗ್ಗೆ ಅಕ್ಕಿ ಗಂಜಿ ಮಾಡಿಟ್ಟುಕೊಂಡು ಮೂರೂ ಹೊತ್ತೂ ಅದನ್ನು ಕುಡಿದೆ. ದಿನದ ಹಸಿವನ್ನು ನಿಭಾಯಿಸುತ್ತಿದ್ದೆ. ಹತ್ತನೆಯ ತರಗತಿಯಲ್ಲಿ ಮೊದಲ ಶ್ರೇಣಿಯಲ್ಲಿ ಉತ್ತಿರ್ಣನಾದ್ದರಿಂದ ಕಾಲೇಜಿಗೆ ಸೇರ\-ಬಹುದೆಂಬ ಹಿರಿಯರ ಸಲಹೆಯಂತೆ ಮರಿಮಲ್ಲಪ್ಪ ಕಾಲೇಜಿನಲ್ಲಿ ವಾಣಿಜ್ಯ ವಿಭಾಗದಲ್ಲಿ ಪ್ರವೇಶ ಪಡೆದೆ.  ನನ್ನ ಮಿತ್ರ ಕೆರೆಕೈ ಉಮಾಕಾಂತ ಭಟ್ಟರು ಕಲಾ ವಿಭಾಗದಲ್ಲಿ \hbox{ಅಧ್ಯಯನ} ಮಾಡುತ್ತಿದ್ದರು.  ನನ್ನ ಜತೆಯೇ ಕಾಗೇರಿ ಶಿವರಾಮ ಹೆಗಡೆಯವರು ಕಲಾ \hbox{ವಿಭಾಗದ} ವಿದ್ಯಾರ್ಥಿಯಾಗಿದ್ದರು.  ಹೀಗೆ ನನ್ನ ಸಾಮಾನ್ಯ ಶಿಕ್ಷಣವು ಸಾಯದೇ ಮರುಜೀವ ಪಡೆಯಿತು.  ಶ್ರೀರಾಮಮಿಶ್ರ ಸಂಸ್ಕೃತಪಾಠಶಾಲೆಯಲ್ಲಿ     ಶ್ರೀ ಸೋ.ರಾಮಸ್ವಾಮಿ \-ಅಯ್ಯಂಗಾರ್ಯರವರ ಹತ್ತಿರ ಕಾವ್ಯತರಗತಿಯ ವ್ಯಾಕರಣ ಅಧ್ಯಯನ ಪ್ರಾರಂಭ\-ವಾಯಿತು. 1976ರಲ್ಲಿ ಕಾವ್ಯ ತರಗತಿಯಲ್ಲಿ ಉತ್ತೀರ್ಣನಾದೆ. 1977ರಲ್ಲಿ ವೇದ ತರಗತಿ\-ಯನ್ನು ಬಿಟ್ಟು ಸಾಹಿತ್ಯ ತರಗತಿಗೆ ಪ್ರವೇಶ ಪಡೆದೆ.

ಮೈಸೂರು ನನ್ನ ಅಧ್ಯಯನದ ತೃಷೆಯನ್ನು ತೀರಿಸಿತು.  ಸಂಸ್ಕೃತಾಧ್ಯಯನಕ್ಕೆ ಬರುವ ಬಹುತೇಕ ವಿದ್ಯಾರ್ಥಿಗಳಂತೆ ನಾನೂ ಆರ್ಥಿಕ ಸಂಪನ್ನನಲ್ಲ.  ಆದರೂ \hbox{ಮೈಸೂರಿನಲ್ಲಿ} ನೆಲೆಸಿ ಅಧ್ಯಯನ ಮಾಡಿರುವುದೇ ಒಂದು ರೋಚಕ ಸಂಗತಿ. 

ಮಲೆನಾಡಿನಲ್ಲಿ ನನ್ನ ಬಾಲ್ಯ ಕಳೆಯಿತು.  ನನಗೆ ಹಸಿವು ಹೇಗಿರುತ್ತದೆ ಎಂಬ ಕಲ್ಪನೆಯೇ ಇರಲಿಲ್ಲ.  ನಾನು ಊರಿನಿಂದ  ತಂದ 50 ಕೆ.ಜಿ ಅಕ್ಕಿ ಖರ್ಚಾಗುತ್ತಾ ಬಂತು.  ಮುಂದೇನು ಎಂಬ ಚಿಂತೆ ಕಾಡಿತು.  ನನ್ನ ಆಪದ್ಬಾಂಧವರಾದ ವೆಂಕಟರಮಣ ಕಲ\-ಗಾರರು ಇದನ್ನೆಲ್ಲ ಬಲ್ಲವರಾಗಿ ನನಗೆ ಸಲಹೆ ನೀಡಿದರು.  ವಾರಾನ್ನದ ವ್ಯವಸ್ಥೆ ಮಾಡಿಕೋ ಎಂದು.  ಯಾರನ್ನು ಕೇಳಬೇಕು? ಏನನ್ನು ಕೇಳಬೇಕು? ಎಂದು ನನಗೆ ಗೊತ್ತಿಲ್ಲ.  ಒಂದು ದಿನ ಕಲಗಾರರು ನನ್ನನ್ನು ನೂರಡಿ ರಸ್ತೆಯಲ್ಲಿರುವ ವಾಸು ಅಗರಬತ್ತಿ ಕಾರ್ಖಾನೆಯ ಹತ್ತಿರ ಕರೆದುಕೊಂಡು ಹೋಗಿ ಬಿಟ್ಟರು.  ಒಳಗೆ ಹೋಗಿ ವಾರಾನ್ನಕ್ಕೆ ಕೇಳು ಎಂದರು.  ತುತ್ತಿಗಾಗಿ ಪರರನ್ನು ಅವಲಂಬಿಸುವ ನನ್ನ ಸ್ಥಿತಿಗೆ ನಾನೇ ನಾಚಿ ನೀರಾದೆ.  ಆದರೆ ಗತ್ಯಂತರವಿಲ್ಲ.  ಕಛೇರಿಯ ಎದುರಿನಲ್ಲಿ ಹತ್ತಾರು ಬಾರಿ ಹಿಂದೆಮುಂದೆ \hbox{ತಿರುಗಿದೆ.}  ಒಳಗೆ ಹೋಗುವ ಧೈರ್ಯವೇ ಬರಲಿಲ್ಲ.  ಇದನ್ನು ಗಮನಿಸಿದ ರಂಗರಾವ್‍ರವರು ಅಲ್ಲಿಯೇ ಕಛೇರಿಯಲ್ಲಿ ಸಹಾಯಕರಾಗಿದ್ದ ರಾಮರಾವ್ ಅವರಿಗೆ ಸೂಚಿಸಿದರಂತೆ,  “ಯಾವುದೋ ಹುಡುಗ ಇತ್ತಿಂದತ್ತ ಅತ್ತಿಂದಿತ್ತ  ತಿರುಗುತ್ತಿದ್ದಾನೆ, ಅವನನ್ನು ಕರೆದು\break ವಿಚಾರಿಸು” ಎಂದು.  ಕಚ್ಛೆಪಂಚೆ ಧರಿಸಿದ ಸಭ್ಯ ಗೃಹಸ್ಥರ ಅಪರಾವತಾರದಂತಿರುವ ರಾಮರಾಯರು ನನ್ನನ್ನು ಕೈಸನ್ನೆ ಮಾಡಿ ಕರೆದರು.  ಗತಿಯಿಲ್ಲದೇ ಹೋದೆ. ಏನು ಬೇಕು? ಏಕೆ ತಿರುಗುತ್ತಿರುವೆ? ಎಂದು ಕೇಳಿದರು.  ನಾನು ಸಾವರಿಸಿಕೊಂಡು ಒತ್ತರಿಸಿ ಬರುತ್ತಿರುವ ಕಣ್ಣೀರನ್ನು ಅದಮಿ ಹಿಡಿದು ಹೇಳಿದೆ.  ನಾನು ಸಂಸ್ಕೃತ ಪಾಠಶಾಲೆಯಲ್ಲಿ ಓದುತ್ತಿದ್ದೇನೆ. ನನಗೆ ವಾರಾನ್ನ ಬೇಕಿದೆ ಒಂದು ದಿನದ ಊಟದ ಸೌಲಭ್ಯ ಒದಗಿಸಬಹುದೇ? ಎಂದು.  ನನ್ನ ಭಾವ ತೀವ್ರತೆಯಿಂದ ಅವರ ಕಣ್ಣುಗಳೂ ಒದ್ದೆಯಾದವು.  ನನ್ನನ್ನು ಶ್ರೀ ರಂಗರಾವ್‍ರವರ ಬಳಿಗೆ ಕರೆದೊಯ್ದರು.  ನನ್ನನ್ನು ಅವರು ಎವಯಿಕ್ಕದೇ ನೋಡಿದರು.  ನಾನು ತಲೆತಗ್ಗಿಸಿದೆ.  ನನ್ನ ಬೇಡಿಕೆಯನ್ನು ರಾಮರಾಯರೇ ಹೇಳಿದರು.  ಈಗ 11 ಗಂಟೆ. 12 ಗಂಟೆಗೆ ಬಾ. ಕರೆದುಕೊಂಡು ಹೋಗುತ್ತೇನೆ ಎಂದರು.  ಊಟಕ್ಕಾಗಿ ಬೇರೆಯವರನ್ನು ಕೇಳಲು ನಾನು ಪಡುತ್ತಿರುವ ಮುಜುಗರವನ್ನು ಗಮನಿಸಿದ ರಾಮರಾಯರು ತನ್ನನ್ನು ಹಿಂಬಾಲಿಸಲು ತಿಳಿಸಿದರು.  ಅಲ್ಲೇ ಎದುರಿನಲ್ಲಿರುವ ಅವರ ಭಾವ ಶ್ರೀ ಅನಂದರವರ ಮನೆಗೆ ಕರೆದೊಯ್ದರು.  ಈ ಹುಡುಗನು ವಾರಾನ್ನಕ್ಕಾಗಿ ಬಂದಿದ್ದಾನೆ ಎಂದರು. ನಮ್ಮನೆಗೆ ಎಂದಾದರೂ ಬರಬಹುದು ಎಂದರು ಆನಂದ ಹಾಗೂ \hbox{ಸತ್ಯವತಮ್ಮ.} ಸ್ವತಃ ರಾಮರಾಯರು ತಮ್ಮ ಮನೆಗೆ ಪ್ರತಿ ಸೋಮವಾರ ಊಟಕ್ಕೆ ಬರಲು ತಿಳಿಸಿದರು.  ಹೀಗೆ ಸೋಮ, ಬುಧ, ಗುರುವಾರಗಳ ಭೋಜನದ ವ್ಯವಸ್ಥೆಯಾಯಿತು. 

ಅಷ್ಟರಲ್ಲಿ ಗಂಟೆ 12 ಆಯಿತು.  ರಾಮರಾಯರು ಲಗುಬಗೆಯಿಂದ ಮತ್ತೆ ವಾಸು ಅಗರಬತ್ತಿಯ ಕಛೇರಿಯತ್ತ ನನ್ನನ್ನು ಕರೆತಂದರು.  ಅಷ್ಟರಲ್ಲಿ ರಂಗರಾವ್‍ರವರು \hbox{ಕಛೇರಿಯ} ಹೊರಗಡೆ ಕಾರಿನತ್ತ ಬರುತ್ತಿದ್ದರು.  ನನ್ನನ್ನು ಕರೆದು ಕಾರಿನಲ್ಲಿ ಕುಳ್ಳಿರಿಸಿ\-ಕೊಂಡರು.  ನೇರವಾಗಿ ಅವರ ಮನೆಗೇ ಕರೆದುಕೊಂಡು ಹೋದರು.  ಅಡಿಗೆಯವನಿಗೆ ತಿಳಿಸಿ ತಮ್ಮ ಸಂಗಡವೇ ನನ್ನನ್ನು ಕುಳ್ಳಿರಿಸಿ\-ಕೊಂಡು ಊಟಮಾಡಿಸಿದರು.  ಇದು ರಂಗರಾವ್ ರವರ ಮಹದೌದಾರ್ಯದ ಒಂದು ಮಾದರಿ.  ನಾನು ಸಂಕೋಚದಿಂದ ಕುಗ್ಗಿದೆ.  ಹೀಗೆ ನನ್ನ ಮೊದಲನೆಯ ದಿನದ ವಾರಾನ್ನ ಗುರುವಾರದಿಂದ ಪ್ರಾರಂಭವಾಯಿತು. ಈ ವಿಷಯ ತಿಳಿದ ರಂಗರಾವ್‍ರವರ ಅಳಿಯ ಅಣ್ಣಾಜೀರಾವ್ ರಾಮಚಂದ್ರ ಅಗ್ರಹಾರದಲ್ಲಿರುವ ತಮ್ಮ ಮನೆಗೆ ಶುಕ್ರವಾರ ಬರಲು ತಿಳಿಸಿದರು.  ಹೀಗೆ ವಾರದ ನಾಲ್ಕು ದಿನಗಳ ಸಮಸ್ಯೆ ಒಂದೇ ಪ್ರಯತ್ನಕ್ಕೆ ಮುಗಿಯಿತು.  ಆಗ ಹೋಟೆಲ್ ಗಳಲ್ಲಿಯೂ ಸಂಸ್ಕೃತ ಕಾಲೇಜಿನ ವಿದ್ಯಾರ್ಥಿಗಳಿಗೆ ವಾರಾನ್ನ ನೀಡುತ್ತಿದ್ದರು.  ನಾನು ಸಯ್ಯಾಜಿರಾವ್ \hbox{ರಸ್ತೆಯಲ್ಲಿರುವ} ಇಂದ್ರಕೆಫೆಯಲ್ಲಿ ಶನಿವಾರ ಹಾಗೂ ಇಂದ್ರಭವನದಲ್ಲಿ ಭಾನುವಾರದ ಊಟದ ಸೌಕರ್ಯ\-ಪಡೆದೆ.  ಇಷ್ಟಾದರೂ ಇಡೀ ಎರಡು ವರ್ಷ ಮಂಗಳವಾರ ಮಾತ್ರ ನಿಟ್ಟುಪವಾಸವೇ ಗತಿಯಾಯಿತು.  ಆ ಒಂದು ದಿನಕ್ಕೆ ಭಿಕ್ಷೆ ಬೇಡಲು ಮನಸ್ಸು ಒಪ್ಪಲೇ ಇಲ್ಲ.  ಹೀಗೆ ಮೈಸೂರಿನ ಜನರ ಔದಾರ್ಯದಿಂದ ನನ್ನ ಉದರವು ಹಸಿವಿನ ತಾಪದಿಂದ\break  ಪಾರಾಯಿತು.  ಆದ್ದರಿಂದಲೇ ನನ್ನ ಸೂರು ಮೈಸೂರು ಎನ್ನಲು ನನಗೆ ಹೆಮ್ಮೆ.

\section*{ಆಚಾರ್ಯರ ಸೆರೆ}

1976ರಲ್ಲಿ  ಕೃಷ್ಣಯಜುರ್ವೇದ ಪ್ರಥಮಾ ಮೊದಲನೆಯ ವರ್ಷದಲ್ಲಿ ಉತ್ತೀರ್ಣನಾದೆ.  1977ರಲ್ಲಿ ಸಾಹಿತ್ಯ ತರಗತಿಗೆ ಪ್ರವೇಶ ಪಡೆದೆ.  ತರ್ಕಸಂಗ್ರಹ ದೀಪಿಕಾ ಗ್ರಂಥದ ಪಾಠಕ್ಕೆ ಪರಮ ಪೂಜ್ಯ ಶ್ರೀ ರಾಮಭದ್ರಾಚಾರ್ಯರ ಸನ್ನಿಧಾನಕ್ಕೆ ತೆರಳಿದೆ.  \hbox{ಸರಸ್ವತೀ} ಪ್ರಾಸಾದದ ಬಲಭಾಗದ ಮಧ್ಯದ ಕೊಠಡಿಯಿದು.  ಘನಗಂಭೀರ ಮುದ್ರೆಯಿಂದ \hbox{ಆಚಾರ್ಯರು} ಪಾಠಪ್ರಾರಂಭಿಸಿದರು.  ಅಯಸ್ಕಾಂತಕ್ಕೆ ಸೆರೆಸಿಕ್ಕ ಕಬ್ಬಿಣವಾದೆ ನಾನು.  ಪ್ರಪಂಚವನ್ನೇ ವಿಶ್ಲೇಷಿಸುವ ವಿಶೇಷ ಗ್ರಂಥ ತರ್ಕಸಂಗ್ರಹ ಎಂದು ಬಿಡಿಬಿಡಿಯಾಗಿ ವಿವರಿಸಿದರು. ಇದನ್ನು ಸವಿದು ಜೀವನವೇ ಸಾರ್ಥಕವೆಂಬ ಭಾವ ಮೂಡಿತು.  ಅಂದಿ\-ನಿಂದಲೇ ಏಕಾಗ್ರತೆಯಿಂದ ನಿರಂತರವಾಗಿ ಪೂಜ್ಯರ ಪಾಠವನ್ನು ಸತ್ಕಾರದಿಂದ ಸವಿದೆ.  ಅದು ನನ್ನ ಬೌದ್ಧಿಕ ಎಲ್ಲೆಯನ್ನು ಪ್ರತಿದಿನವೂ ಹಿಗ್ಗಿಸುತ್ತಲಿತ್ತು.  ಅದನ್ನು \hbox{ಸಂಭ್ರಮಿಸಿದೆ.}  ದೀಪಿಕಾ ವ್ಯಾಖ್ಯಾನದ ಪಾಠ ಪ್ರಾರಂಭವಾಯಿತು. ತರ್ಕಸಂಗ್ರಹವು ‘\hbox{ಬಾಲಾನಾಂ} ಸುಖ\-ಬೋಧಾಯ’ ಎಂಬುದನ್ನು ವ್ಯಾಖ್ಯಾನಿಸುತ್ತಾ ಬಾಲ ಯಾರು ಎಂಬುದನ್ನು ಬಿಡಿಸಿ ಹೇಳಿದರು.  ‘ಅಧೀತಕಾವ್ಯವ್ಯಾಕರಣಕೋಶಾದಿಮಾನ್ ಅನಧೀತತರ್ಕಶಾಸ್ತ್ರಃ ಬಾಲಃ’  \enginline{-}   ತರ್ಕಶಾಸ್ತ್ರವನ್ನು ಓದದ, ಆದರೆ ಕಾವ್ಯ, ವ್ಯಾಕರಣ, ಕೋಶಗಳನ್ನು ಓದಿದವನು ಬಾಲ.  ಆಚಾರ್ಯರು ನನ್ನನ್ನು ದಿಟ್ಟಿಸಿ ಕೇಳಿದರು.  ಕಾವ್ಯ ಓದಿರುವೆಯಾ? ಎಂದು.  ಹೌದು ಎಂದೆ.  ನಾನು ಕಾವ್ಯಪರೀಕ್ಷೆಯಲ್ಲಿ ಉತ್ತೀರ್ಣನಾಗಿದ್ದೆನಷ್ಟೆ !  ಮರುಪ್ರಶ್ನೆ ಕೇಳಿದರು.  ಯಾವ ಯಾವ ಕಾವ್ಯವನ್ನು ಓದಿರುವೆ?  ಆಗ ನನಗೆ ದಿಗಿಲಾಯಿತು.  ಏಕೆಂದರೆ ಯಾವ ಕಾವ್ಯವನ್ನೂ ನಾನು ಓದಿರಲಿಲ್ಲ.  ನಾನು ಓದಿದ ಕಾವ್ಯಪರೀಕ್ಷೆ ಇಲ್ಲಿ ಅಪೇಕ್ಷಿತ ಉತ್ತರ\-ವಲ್ಲ ಎಂಬುದು ಸ್ಪಷ್ಟವಾಯಿತು.  ಮುಖ ಬಾಡಿತು.  ವ್ಯಾಕರಣ ಕೋಶ ಪರೀಕ್ಷೆಗಾಗಿ ಮಾತ್ರ ಓದಿದಷ್ಟು ಗೊತ್ತಿತ್ತು. ಅಂತೂ ನಾನು ತರ್ಕಸಂಗ್ರಹವನ್ನು ಓದಲು ಅಧಿಕಾರಿಯಲ್ಲ ಎಂಬುದು ಸ್ಪಷ್ಟವಾಯಿತು.  ಕುಂದಿದ ನನ್ನ ವದನವನ್ನು ಕಂಡು ಹೇಳಿದರು.  ಗ್ರಹಣ ಶಕ್ತಿ ಧಾರಣಶಕ್ತಿ ಇವೆರಡಿದ್ದರೆ ಈ ಗ್ರಂಥವನ್ನು ಅಧ್ಯಯನ ಮಾಡುವ ಅರ್ಹತೆ ಇದ್ದಂತೆ ಎಂದು.  ಇಂದಿನ ದಿನಗಳಲ್ಲಿ ಶಾಸ್ತ್ರಾಧ್ಯಯನಕ್ಕೆ ಬರುವ ಅನೇಕ\break ವಿದ್ಯಾರ್ಥಿಗಳು ಕೋಶ, ಕಾವ್ಯ, ವ್ಯಾಕರಣಗಳ ಗಂಧವಿಲ್ಲದವರು.  ಅಂಥವರಿಗೆ ಎರಡು ದಶಕಗಳ ಕಾಲ ಬೋಧನೆ ಮಾಡಿದ ಸಾಹಸವನ್ನು ನೆನಪಿಸಿಕೊಳ್ಳುತ್ತಿರುತ್ತೇನೆ.    

ಶಾಸ್ತ್ರಾಧ್ಯಯನದ ನಿಜವಾದ ಸವಾಲೇ ಇದು.  ಅನಧಿಕಾರಿಗಳಿಗೆ ಬೋಧಿಸುವ ದುರ್ವಿಧಿ ಶಿಕ್ಷರದ್ದು.  ಈ ವೃತ್ತಿಗೆ ಬಂದ ಮೇಲೆ ಈ ಸವಾಲನ್ನು ಎದುರಿಸಲೇಬೇಕು.  ಇದಕ್ಕಾಗಿ ನಾನು ಪ್ರಾಮಾಣಿಕ ಪ್ರಯತ್ನ ಮಾಡಿದ್ದೇನೆ.  ಏಕೆಂದರೆ ನಾನು ಅನಧಿಕಾರಿ\-ಯಾಗಿಯೇ ಸೇರಿದ್ದೆನಲ್ಲ.  ನನ್ನ ಗುರುಗಳು ಗ್ರಹಣ ಧಾರಣ ಸಾಮರ್ಥ್ಯದ ಮೇಲೆ ಶಾಸ್ತ್ರ ಬೋಧಿಸಿದ ನಿದರ್ಶನವಿತ್ತಲ್ಲ.  ಶಾಸ್ತ್ರ ಬೋಧನೆಯಲ್ಲಿಯೂ ನವ್ಯನ್ಯಾಯದ ಪರಿಷ್ಕಾರವನ್ನು ಬೋಧಿಸುವುದು ಅತ್ಯಂತ ಕ್ಲಿಷ್ಟ.  ಪೂಜ್ಯ ರಾಮಭದ್ರಾಚಾರ್ಯರು ಇದರ ಒಂದು ವಿಧಾಯಕ ನಮೂನೆಯನ್ನು ಕರುಣಿಸಿದ್ದಾರೆ. ಅದನ್ನು ಮೆಲಕು ಹಾಕುವುದೇ ಒಂದು ಮಧುರ ಅನುಭವ.  

ಸಾಹಿತ್ಯ ತರಗತಿಯಲ್ಲಿ ತರ್ಕಸಂಗ್ರಹದ ವ್ಯಾಮೋಹಕ್ಕೊಳಗಾದ ನಾನು ತರ್ಕಶಾಸ್ತ್ರವನ್ನು ಅಧ್ಯಯನ ಮಾಡುವ ಸಂಕಲ್ಪ ಮಾಡಿದೆ.  ಅದರಂತೆ ತರ್ಕಶಾಸ್ತ್ರದಲ್ಲಿ ಪ್ರವೇಶ ಪಡೆದೆ.  ಪೂರ್ವಭಾಗದಲ್ಲಿ ವ್ಯಾಪ್ತಿಗ್ರಂಥಗಳ ಪಾಠ್ಯವಾಗಿತ್ತು.  ಪೂಜ್ಯರು ‘ವ್ಯಾಪ್ತಿ’ ಎಂಬ ಒಂದೇ ಒಂದು ಪದವನ್ನು ಸೀಳಿ ಅದು ಹೇಗೆ ಪಂಚಲಕ್ಷಣೀ, ಚತುರ್ದಶಲಕ್ಷಣೀ, ಸಿದ್ಧಾಂತಲಕ್ಷಣ ಗ್ರಂಥವಾಗಿ ಅರಳಿದೆ ಎಂದು ಪ್ರತಿಪಾದಿಸಿದ ರೀತಿ ನನ್ನ ಭೀತಿಯನ್ನು ಒರೆಸಿಹಾಕಿತ್ತು.  

ವಿ + ಆಪ್ತಿ   \enginline{-}   ವ್ಯಾಪ್ತಿಯಾಗಿದೆ.  ವಿಶೇಷ ಆಪ್ತಿಯೇ ವ್ಯಾಪ್ತಿ.  ಆಪ್ತಿ ಎಂದರೆ ಸಾಹಚರ್ಯ.  ಸಾಹಚರ್ಯದ ವಿಶೇಷ ನಿಯತತೆ.  ಆದ್ದರಿಂದ ನಿಯತ ಸಾಹಚರ್ಯವೇ ವ್ಯಾಪ್ತಿ. ಆಪ್ತಿಯನ್ನು ನಿಯತವೆಂದು ಅರ್ಥೈಸಬಹುದು.  ನಿಯತತೆಯ ವಿಶೇಷವೇ ಸಾಹಚರ್ಯ.  ಆದ್ದರಿಂದ ನಿಯತಸಾಹಚರ್ಯಮ್ ಅಥವಾ ಸಾಹಚರ್ಯದ ನಿಯಮವೇ ವ್ಯಾಪ್ತಿ.  ನೆನಪಿಸಿಕೊಳ್ಳಿ ತರ್ಕಸಂಗ್ರಹದಲ್ಲಿ ಹೇಳಿಲ್ಲವೇ ನಿಯತಸಾಹಚರ್ಯ ವ್ಯಾಪ್ತಿ   \enginline{-} ‘ಸಾಹಚರ್ಯ ನಿಯಮೋ ವಾ ವ್ಯಾಪ್ತಿಃ’ ಅಂದರೆ ವ್ಯಾಪ್ತಿ ಎಂಬ ಪದದ ವಿಕಸಿತ ಅರ್ಥವೇ  ಅಲ್ಲವೇ? ಯಾರು ಇಲ್ಲ ಎನ್ನಲು ಸಾಧ್ಯ. ಈ ಪರಿಯ ಗ್ರಂಥ ಗ್ರಂಥಿಯನ್ನು ಶಿಥಿಲಗೊಳಿಸುವ ಪ್ರಕ್ರಿಯೆಯನ್ನು ಸಿದ್ಧಾಂತ ವ್ಯಾಪ್ತಿಯ ಸ್ವರೂಪದಲ್ಲಿಯೂ ಅಚ್ಚು\-ಕಟ್ಟಾಗಿ ಅನ್ವಯಿಸಿ ನನ್ನ ಮನೋಬಲವನ್ನು ವರ್ಧಿಸಿದ ಗುರುವರ್ಯರು ಶ್ರೀ ರಾಮಭದ್ರಾಚಾರ್ಯರು.  ಈ ಮಾದರಿಯನ್ನು ನನ್ನ ಪಾಠಕ್ರಮದಲ್ಲಿ ಸಾಧ್ಯವಾದಷ್ಟು ಅಳವಡಿಸಿ\-ಕೊಂಡಿದ್ದೇನೆ.

ನಾನು ವಿದ್ಯಾರ್ಥಿಗಳಿಗೆ ಕಲಿಸಿದ್ದೇನೆಯೋ ಇಲ್ಲವೋ.  ಆದರೆ ವಿದ್ಯಾರ್ಥಿಗಳಿಂದಲೇ ನಿರಂತರವಾಗಿ ಕಲಿಯುತ್ತಲೇ ಇದ್ದೇನೆ.  ಶ್ರೀ ಶಂಕರವಿಲಾಸ ಸಂಸ್ಕೃತ ಪಾಠಶಾಲೆಯಲ್ಲಿ ಎಳೆಯ ಮಕ್ಕಳಿಗೆ ಬಾಲಪಾಠ ಮಾಡುವುದರ ಮೂಲಕ ನನ್ನ ಅಧ್ಯಾಪನ ಕಲೆಯನ್ನು ಬೆಳೆಸಿ\-ಕೊಂಡೆ.

\section*{ಉದ್ಯೋಗ ಯೋಗ}

ನಾನು 1978 ರಲ್ಲಿಯೇ ಉದ್ಯೋಗಿಯಾದೆ.  ಇದು ನನ್ನ ಯೋಗ.  ಶ್ರೀ ರಮಾ~ನಂದ ಅವಭೃಥರು ಶ್ರೀ ಶಂಕರವಿಲಾಸ ಸಂಸ್ಕೃತ ಪಾಠಶಾಲೆಯಲ್ಲಿ ಶಿಕ್ಷಕರಾಗಿದ್ದರು.\break  ಮೀಮಾಂಸಾಶಾಸ್ತ್ರದ ವಿದ್ಯಾರ್ಥಿಯವರು.  ಆಗ ಸಂಸ್ಕೃತ ಪಾಠಶಾಲೆಗಳಿಗೆ ಅನುದಾನ ಸಂಹಿತೆಯನ್ನು ಜಾರಿಗೊಳಿಸುವ ಪ್ರಕ್ರಿಯೆ ಆರಂಭವಾಗಿತ್ತು.  ಅದಕ್ಕೆ ಸಂಬಂಧಿಸಿದಂತೆ ಶಾಲೆಯಲ್ಲಿ ಅನೇಕ ದಾಖಲಾತಿಗಳನ್ನು ಕಾಗದಪತ್ರಗಳನ್ನು ತಯಾರಿಸಬೇಕಾಗಿತ್ತು.\break   ಅವಭೃಥರು ಇಂತಹ ಲೌಕಿಕ ವ್ಯವಹಾರ ಬಲ್ಲವರಲ್ಲ.  ಸಂಸ್ಕೃತದಲ್ಲಿ ಅನೇಕರು ವ್ಯವಹಾರದಿಂದ ಗಾವುದ ದೂರ.  ಆದರೆ ನಿಷ್ಠೆಯಿಂದ ಸಂಸ್ಕೃತವನ್ನು ಬೋಧಿಸಿದ್ದಾರೆ.  ಆ ನಿಷ್ಠೆಗೆ ದೊರೆತ ಅನುದಾನ ವಾರ್ಷಿಕವಾಗಿ ನಾಲ್ಕು ನೂರು ರೂಪಾಯಿ.  ಅದೂ ಪ್ರತಿ ತಿಂಗಳು ದೊರೆಯದು.  ವರ್ಷಕ್ಕೊಮ್ಮೆ ಮಾರ್ಚ್ ತಿಂಗಳಲ್ಲಿ ಸಂಬಳ.   ಪ್ರತಿಫಲದ\break ನಿರೀಕ್ಷೆಯೇ  ಇಲ್ಲದೇ ಕರ್ತವ್ಯಪ್ರಜ್ಞೆಯಿಂದ ಬೋಧಿಸಿದ ಶಿಕ್ಷಕರ ಸಂತತಿಗೆ ಸೇರಿದವರು ಅವ\-ಭೃಥರು.  ಅಂಥವರ ನಿಷ್ಠೆಯಿಂದ ಸಂಸ್ಕೃತ ಇಂದಿನವರೆಗೂ ಉಳಿದು ಬಂದಿದೆ.  ಅವಭೃಥರು ನನ್ನನ್ನು ಸಂಪರ್ಕಿಸಿ ಕೇಳಿದರು. ನಮ್ಮ ಶಾಲೆಯ ದಾಖಲಾತಿಗಳನ್ನು\break ತಯಾರಿಸಲು ಸಾಧ್ಯವೇ? ಎಂದು.  ಯಾವ ಸವಾಲನ್ನೂ ತಿರಸ್ಕರಿಸುವ ಸ್ವಭಾವ ನನ್ನದಲ್ಲ, ಒಪ್ಪಿದೆ.  ಅನುದಾನ ಸಂಹಿತೆಯು ಯಾವ ದಾಖಲೆಗಳನ್ನು ನಿಯಮಿಸಿದೆಯೋ ಅವನ್ನೆಲ್ಲ ನಾಲ್ಕೇ ದಿನಗಳಲ್ಲಿ ತಯಾರಿಸಿಕೊಟ್ಟೆ.  ಅಚ್ಚುಕಟ್ಟಾಗಿ ನಿಯಮಾನುಸಾರ\break  ಸಜ್ಜುಗೊಳಿಸಿದ ದಾಖಲೆ\-ಗಳನ್ನು ಪಡೆದ ಅವಭೃಥರು ತೃಪ್ತರಾದರು.  ಈ \hbox{ವಿಷಯವನ್ನು } ಆಗಿನ ಕಾರ್ಯದರ್ಶಿಗಳಾದ ಶ್ರೀ ಶಿವಕುಮಾರ ಸ್ವಾಮಿಗಳಿಗೆ ತಿಳಿಸಿದರು.\break ಶಿವಕುಮಾರ ಸ್ವಾಮಿಗಳು ಶಿರಸಿಯ ಗುರುಶಾಂತ ಸ್ವಾಮಿಗಳ ಮಕ್ಕಳು.  ಶಿರಸಿಯ ಶ್ರೀ ಗುರುಶಾಂತ ಸ್ವಾಮಿಗಳು ಸಂಸ್ಕೃತ ಕಾಲೇಜಿನಲ್ಲಿ  ಶಕ್ತಿವಿಶಿಷ್ಟಾದ್ವೈತ ವೇದಾಂತದ\break ಪ್ರಾಧ್ಯಾಪಕರಾಗಿದ್ದರು.  ಆದ್ದರಿಂದ ಶ್ರೀ ಶಿವಕುಮಾರ ಸ್ವಾಮಿಗಳು ಸಂಸ್ಕೃತದ\break ಅಭಿಮಾನಿಗಳು. ವಿದ್ವತ್ ಪ್ರಿಯರೂ ಆಗಿದ್ದರು.  ಅವಭೃಥರು ಸಲ್ಲಿಸಿದ \hbox{ದಾಖಲೆಗಳನ್ನು} ಪರಿಶೀಲಿಸಿ ಹೆಮ್ಮೆಯಿಂದ ಗಂಗಾಧರ ಭಟ್ಟರನ್ನು ಕರೆದುಕೊಂಡು ಬನ್ನಿ ಎಂದು \hbox{ಅವಭೃಥರಿಗೆ} ತಿಳಿಸಿದರು.  ಅದರಂತೆ ನನ್ನನ್ನು ಅವ\-ಭೃಥರು ಶಿವಕುಮಾರ ಸ್ವಾಮಿಗಳ ಹತ್ತಿರ ಕರೆದೊಯ್ದರು.  ನನ್ನನ್ನು ನೋಡುತ್ತಲೇ ಶ್ರೀ ಶಿವ\-ಕುಮಾರಸ್ವಾಮಿಗಳು ನಾಳೆಯಿಂದಲೇ ತಾವು ನಮ್ಮ ಪಾಠಶಾಲೆಗೆ ಬಂದು ಪಾಠಮಾಡಿ ಎಂದರು.  ಹಿಂದುಮುಂದು ನೋಡದೆ ಒಪ್ಪಿಕೊಂಡೆ.  ಹೀಗಾಗಿ ನಾನು ಶ್ರೀ ಶಂಕರ\-ವಿಲಾಸ ಸಂಸ್ಕೃತ ಪಾಠಶಾಲೆಯಲ್ಲಿ ಬೋಧಕನಾದೆ.  

ಶಾಲೆಯಲ್ಲಿ ವಿದ್ಯಾರ್ಥಿಗಳ ಸಂಖ್ಯೆ ಸಮಾಧಾನಕರವಾಗಿರಲಿಲ್ಲ.  ಹೆಚ್ಚು ವಿದ್ಯಾರ್ಥಿ\-ಗಳನ್ನು ಕರೆತರಲು ಉಪಾಯವೊಂದನ್ನು ಹುಡುಕಿದೆ.  ಸಂಸ್ಕೃತ ಪಾಠಶಾಲೆಯಲ್ಲಿ ಇಂಗ್ಲೀಷ್ ವ್ಯಾಕರಣವನ್ನು ಉಚಿತವಾಗಿ ಹೇಳಿಕೊಡುತ್ತೇವೆ, ಆದರೆ ಸಂಸ್ಕೃತ\break ತರಗತಿಗೆ ಕಡ್ಡಾಯವಾಗಿ ಸೇರಬೇಕೆಂಬ ನಿಬಂಧನೆ ಹಾಕಿದೆ.    ಇದಕ್ಕೆ ಆಕರ್ಷಿತರಾಗಿ ಅನೇಕ ವಿದ್ಯಾರ್ಥಿಗಳು ಪ್ರಥಮ ಹಾಗೂ ಕಾವ್ಯ ತರಗತಿಗೆ ಪ್ರವೇಶ ಪಡೆದರು.  ಪಾಠಶಾಲೆಯು ನಿಷ್ಕಂಟಕವಾಗಿ ಕಾರ್ಯನಿರ್ವಹಿಸುತ್ತಿತ್ತು.  ನನಗೆ ಜೀವಿಕೆಯು ದೊರೆತಿತ್ತು.  

06.03.1980 ರಂದು ಒಂದು ಆದೇಶ ಕೈ ಸೇರಿತು.  ಆದೇಶ ಓದಿದೆ.  ನನ್ನನ್ನೇ ನಾನು ನಂಬದಾದೆ.  01.06.1979 ರಿಂದ ಶ್ರೀ ಶಂಕರವಿಲಾಸಸಂಸ್ಕೃತ ಪಾಠಶಾಲೆಯ ಮುಖ್ಯೋಪಾಧ್ಯಾಯನಾಗಿ ನಿಯುಕ್ತನಾಗಿದ್ದೆ.  300  \enginline{-}  700 ವೇತನ ಶ್ರೇಣಿಯಲ್ಲಿ ವೇತನ ಅನುದಾನವು ದೊರೆತಿತ್ತು.  ಮಾನ್ಯ ಶ್ರೀ ಶಿವಕುಮಾರ ಸ್ವಾಮಿಗಳು ಪಾಠಶಾಲಾ ಸಮಿತಿಯ ಕಾರ್ಯದರ್ಶಿಗಳಾಗಿದ್ದು ನನ್ನನ್ನು ಮುಖ್ಯೋಪಾಧ್ಯಾಯನೆಂದು\break  ನಮೂದಿಸಿ ಪ್ರಸ್ತಾವನೆಯನ್ನು ಇಲಾಖೆಗೆ ಸಲ್ಲಿಸಿದ್ದರು.  ಇದು ನನಗೆ ಗೊತ್ತಿರಲಿಲ್ಲ. ಹೀಗಾಗಿ ನನಗೇ ತಿಳಿಯದಂತೆ ನಾನು ಉದ್ಯೋಗಿಯಾಗಿ ಒಂದು ವರ್ಷ ಸಂದು\break ಹೋಗಿತ್ತು.  ಆ ಒಂದು ವರ್ಷದ ವೇತನವೆಲ್ಲ ಒಂದೇ ಸಾರಿ ನನ್ನ ಕೈ ಸೇರಿತು.  ಇದು ಭವಿತವ್ಯಕ್ಕೆ ಬೆಳಕಾಯಿತು.  ಈ 20 ವರ್ಷಗಳಲ್ಲಿ ಅನೇಕ ಬದಲಾವಣೆಗಳಾಗಿವೆ.  ಇದೇ ಹುದ್ದೆಗೆ ಈ ಸಮಯದಲ್ಲಿ ನಾಲ್ಕರಿಂದ ಐದು ಲಕ್ಷ ರೂಪಾಯಿಗಳ ವ್ಯವಹಾರ ನಡೆಯುತ್ತದೆ ಎಂಬುದು ಒಂದು ದುರಂತ.  

ಅನಿರೀಕ್ಷಿತವಾಗಿ ಉದ್ಯೋಗವೇನೋ ದೊರಕಿತು.  ಉದ್ಯೋಗಿಯಾಗಿ ಪಾಠ\-\break ಶಾಲೆಯ ವಸತಿ ನಿಲಯದಲ್ಲಿ ವಾಸಿಸುವುದು ನಿಯಮ ಬಾಹಿರವೆಂದು ಸ್ವಯಂ ಪ್ರೇರಣೆಯಿಂದ ವಸತಿ ನಿಲಯವನ್ನು ಬಿಟ್ಟು, ಸರಸ್ವತೀ ನಿಲಯದಲ್ಲಿ ವಾಸವ್ಯವಸ್ಥೆಯನ್ನು ಮಾಡಿಕೊಂಡೆ. ಈ ಸಮಯದಲ್ಲಿ ನನ್ನ ಶ್ರೀಧರಣ್ಣ ಮೈಸೂರಿನಲ್ಲಿ ಉದ್ಯೋಗಿಯಾಗಿದ್ದ.  ಪ್ರತಿತಿಂಗಳು ನನಗೆ ಆರ್ಥಿಕ ಸಹಾಯ ನೀಡುತ್ತಿದ್ದ.  ಇಬ್ಬರೂ ಸರಸ್ವತೀ ನಿಲಯದಲ್ಲಿ ವಾಸಿಸುತ್ತಿದ್ದೆವು.  ಇಬ್ಬರಿಗೂ ಹಾಸಿಗೆ ಒಂದೇ ಆಗಿತ್ತು.  ಅದೊಂದು ದಿನ ಅಕಸ್ಮಾತ್ ಆಗಿ ಕೊಠಡಿಯ ಮೇಲ್ಚಾವಣಿಯು ಕುಸಿದು ಶ್ರೀಧರಣ್ಣನ ಮೇಲೆ ಬಿದ್ದು ಪೆಟ್ಟಾಯಿತು. ನಾನು ಪಕ್ಕದಲ್ಲೇ ಕುಳಿತು ಓದುತ್ತಲಿದ್ದೆ.  ಧೂಳಿನಿಂದ ಕೊಠಡಿ ತುಂಬಿಹೋಯಿತು. ಭಯಭೀತನಾದೆ. ಆದ್ದರಿಂದ ಸರಸ್ವತೀ ನಿಲಯವನ್ನು ಬಿಟ್ಟು ತಾತ್ಕಾಲಿಕವಾಗಿ ಪುನಃ ಪಾಠಶಾಲೆಗೇ ಬಂದೆ. ಪಾಠಶಾಲೆ ನನ್ನ ತವರಲ್ಲವೇ? ಅನಂತರ ಪುನಃ ಬೇರೆ ವಾಸವ್ಯವಸ್ಥೆ ಮಾಡಿಕೊಂಡೆ. 

ಸಂಸ್ಕೃತಾಧ್ಯಯನದಿಂದ ನನ್ನ ಬದುಕು ರೂಪುಗೊಂಡಿತು.  ಆದ್ದರಿಂದ ಸಂಸ್ಕೃತ ಅಧ್ಯಯನ ಮಾಡಲು ಇತರರಿಗೆ ಪ್ರೇರೇಪಣೆ ನೀಡುವುದು ಹಾಗೂ ಸಹಾಯಹಸ್ತ ಚಾಚುವುದು ನನ್ನ ಕರ್ತವ್ಯವೆಂದು ಬಗೆದೆ.  ಸಾಮಾನ್ಯವಾಗಿ  ಸಂಸ್ಕೃತಜ್ಞರ \hbox{ಕುಟುಂಬದ} ಸದಸ್ಯರು ಸಂಸ್ಕೃತಾಧ್ಯಯನದತ್ತ ವಾಲುವುದು ಅಪರೂಪ.  ಆದರೆ ನಾನು ನನ್ನ \hbox{ಕುಟುಂಬದ} 12 ಸದಸ್ಯರನ್ನು ಮೈಸೂರಿಗೆ ಕರೆತಂದು ನನ್ನೊಂದಿಗೇ ಇರಿಸಿಕೊಂಡು ಇತರ\break ಶಿಕ್ಷಣದ ಜೊತೆಗೆ ಸಂಸ್ಕೃತ ಪ್ರಥಮಾ, ಕಾವ್ಯ ಹಾಗೂ ಸಾಹಿತ್ಯ ಅಧ್ಯಯನ ಮಾಡಿಸಿದ್ದೇನೆ.  ಅದರಂತೆ ಇತರ ಕಾಲೇಜುಗಳಲ್ಲಿ ಅಧ್ಯಯನ ಮಾಡಲು ಬಯಸಿ ಮೈಸೂರಿಗೆ ಬಂದು ವಿದ್ಯಾಲಯದಲ್ಲಿ ಪ್ರವೇಶ, ವಾಸ ಮತ್ತು ಭೋಜನ ವ್ಯವಸ್ಥೆಯನ್ನು ಬಯಸಿದ ವಿದ್ಯಾರ್ಥಿಗಳಿಗೆ ನಿರ್ವ್ಯಾಜ ಪ್ರೀತಿಯಿಂದ ವ್ಯವಸ್ಥೆ ಕಲ್ಪಿಸಿಕೊಟ್ಟಿದ್ದೇನೆ.\break  ಪ್ರತಿಯಾಗಿ ಸಂಸ್ಕೃತಾಧ್ಯಯನ ಮಾಡಲು ಪ್ರೇರೇಪಿಸಿದ್ದೇನೆ.  ಹೀಗಿದ್ದರೂ ನನ್ನ ಸಹಾಯ ಮಾತ್ರ ಅತ್ಯಂತ ನಿರ್ಲಿಪ್ತನಾಗಿದ್ದ ಕಾರಣ ಈಗಲೂ ನನಗೆ ಆ ವಿದ್ಯಾರ್ಥಿಗಳ ಪರಿಚಯ\-ವಾಗಲೀ, ಕನಿಷ್ಠ ಅವರ ವಿಳಾಸ\-ವಾಗಲಿ ನನ್ನಲ್ಲಿ ಇಲ್ಲ.  ಆದರೆ ಅಷ್ಟು ವ್ಯಕ್ತಿಗಳಿಗೆ ಸಂಸ್ಕೃತದ ಸವಿ ಸವಿಯಲು ಅನುವು ಮಾಡಿಕೊಟ್ಟ ಸಂತೃಪ್ತಿ ನನಗಿದೆ.  

\section*{ಸಂತೋಷದಲ್ಲಿ ಪರ್ಯವಸಾನವಾದ ದೌರ್ಭಾಗ್ಯ}

ನನಗೆ ಉದ್ಯೋಗ ದೊರೆತದ್ದರಿಂದ ಶ್ರೀಮನ್ಮಹಾರಾಜ ಸಂಸ್ಕೃತ ಮಹಾಪಾಠಶಾಲೆಯ ಅಧಿಕೃತ ವಿದ್ಯಾರ್ಥಿಯಾಗುವ ಸದವಕಾಶದಿಂದ ವಂಚಿತನಾದೆ. ಆದ್ದರಿಂದ ಸಾಹಿತ್ಯದ ಅನಂತರ ವಿದ್ವನ್ಮಧ್ಯಮಾ ಹಾಗೂ ವಿದ್ವದುತ್ತಮಾ ಪರೀಕ್ಷೆಯನ್ನು ಖಾಸಗೀ ಅಭ್ಯರ್ಥಿಯಾಗಿಯೇ ಎದುರಿಸಬೇಕಾಯಿತು.  ಕಾಲೇಜಿನ ಮೂಲಕ ವಿದ್ಯಾರ್ಥಿಗಳಿಗಾಗಿ ಏರ್ಪಡಿಸುವ ಯಾವ ಚಟುವಟಿಕೆಯಲ್ಲಿಯೂ ಭಾಗವಹಿಸುವ ಭಾಗ್ಯದಿಂದ ವಂಚಿತನಾದೆ.  ರಾಜ್ಯ ಹಾಗೂ ರಾಷ್ಟ್ರಮಟ್ಟದ ವಾಕ್ ಪ್ರತಿಯೋಗಿತಾದಲ್ಲಿ ಭಾಗವಹಿಸುವ ಅವಕಾಶವೇ ಇಲ್ಲದಂತಾಯಿತು.  ಈ ಬಗ್ಗೆ ಈಗಲೂ ನನಗೆ ಖೇದವಿದೆ.  

ಸ್ವತಃ ಇಂಥ ಸ್ಪರ್ಧೆಯಲ್ಲಿ ಭಾಗವಹಿಸಲು ಸಾಧ್ಯವಾಗದಿದ್ದರೂ ಭಾಗವಹಿಸುವ ವಿದ್ಯಾರ್ಥಿಗಳಿಗೆ ಮಾರ್ಗದರ್ಶನ ಮಾಡುವ ಅವಕಾಶವು 1999 ರಿಂದಲೂ ನಿರಂತರವಾಗಿ ದೊರಕಿರುವುದರಿಂದ ಸಂತುಷ್ಟನಾಗಿದ್ದೇನೆ. 

\section*{ಹೋರಾಟ}

ಕಲಿಸುತ್ತಾ ಕಲಿಯುತ್ತಾ ಬದುಕು ಕಂಡವನು ನಾನು. ಸಾಮಾನ್ಯವಾಗಿ ಸಂಸ್ಕೃತ ಶಿಕ್ಷಕರು ಉದಾರಿಗಳು.  ಯಥೇಚ್ಛವಾಗಿ ಅಂಕ ನೀಡುತ್ತಾರೆ.  ಸಂಸ್ಕೃತ ತರಗತಿಯಲ್ಲಿ ಅನುತ್ತೀರ್ಣರಾಗುವುದು ಅತಿ ವಿರಳ. ಈ ರೀತಿ ಪ್ರಸಿದ್ಧಿಯಿದೆ. ಆದರೆ ನನ್ನ ವಿದ್ಯಾರ್ಥಿ ಬದುಕಿನಲ್ಲಿ ಇದಕ್ಕೆ ತದ್ವಿರುದ್ಧವಾದ ಪರಿಸ್ಥಿತಿಯನ್ನು ಎದುರಿಸಿದ್ದೇನೆ.  
~\\[0.4cm]
ನಾನು 1981ರಲ್ಲಿ ನವೀನನ್ಯಾಯ ವಿದ್ವನ್ಮಧ್ಯಮಾ ಪರೀಕ್ಷೆಗೆ ಹಾಜರಾಗಿದ್ದೆ.  ಫಲಿತಾಂಶ ಪ್ರಕಟವಾದಾಗ ಆಶ್ಚರ್ಯ ಕಾದಿತ್ತು.  ದಿನಕರೀಯ ಪತ್ರಿಕೆಯಲ್ಲಿ 75 ಅಂಕಗಳಿಗೆ 14 ಅಂಕಗಳು ಮಾತ್ರ ನನಗೆ ಬಂದಿತ್ತು. ಹಾಗೆಂದು, ವಾಕ್ ಪರೀಕ್ಷೆಯಲ್ಲಿ 25 ಅಂಕ\-ಗಳಿಗೆ 22 ಅಂಕ ಗಳಿಸಿದ್ದೆ.  ಆದರೂ ಅನುತ್ತೀರ್ಣನಾದೆ.  ಇದು ನನ್ನನ್ನು ಕೆರಳಿಸಿತು.\break  ಸಿಡಿದೆದ್ದೆ. ಆಗ ಮರುಮೌಲ್ಯಮಾಪನದ ಅವಕಾಶವಿರಲಿಲ್ಲ.  ಮರು ಎಣಿಕೆ ಮಾಡಿಸಿದೆ.  ಸರಿ ಇದೆ ಎಂಬ ಉತ್ತರ ಬಂತು.  ತಟಸ್ಥನಾಗಲಿಲ್ಲ.  ಯಾವುದೋ ಅವಿವೇಕಿ ಅನುತ್ತೀರ್ಣ\-ಗೊಳಿಸಿದ್ದಾನೆಂಬ  ವಿಷಯ ಸ್ಪಷ್ಟವಾಯಿತು.  ಶಾಸ್ತ್ರಗಂಧವಿಲ್ಲದ ಬೋಧನಾನುಭವ\-ವಿಲ್ಲದ ಕೊಳದ ಮಠದ ಸ್ವಾಮಿ ಎಂಬ ಒಬ್ಬ, ವಶೀಲಿಯಿಂದ ತರ್ಕಶಾಸ್ತ್ರದಲ್ಲಿ ಸರ್ವ\-ಪ್ರಾವೀಣ್ಯ ಪಡೆದಿದ್ದಲ್ಲದೇ ಮೌಲ್ಯಮಾಪಕನಾಗಿ, ವಾಕ್ ಪರೀಕ್ಷಕನಾಗಿ ಬರಬೇಕೆಂಬ ತೆವಲಿನಿಂದ, ಅಧಿಕಾರಿಗಳಿಗೆ ಒತ್ತಡಹಾಕಿ ಪರೀಕ್ಷಕನಾಗಿ ನಿಯುಕ್ತನಾಗಿದ್ದ.  ಪ್ರತಿಭಾ\-ವಂತರನ್ನೆಲ್ಲ ಸಾರಾಸಗಟಾಗಿ ಅನುತ್ತೀರ್ಣಗೊಳಿಸಿದ್ದ.  ಅವನು ಪ್ರಭಾವಲಯವುಳ್ಳವನು.  ಆದ್ದರಿಂದ ಸುಮ್ಮನಿರುವುದೇ ಲೇಸೆಂಬ ಸೂಚನೆ ಎಲ್ಲಾ ಕಡೆಯಿಂದ ಬಂತು.  ಆದರೆ ನನ್ನ ಅಂತಃಕರಣದ ಕುದಿ ತಣಿಯಲಿಲ್ಲ.  ಅನುತ್ತೀರ್ಣರಾದ ಇತರ \hbox{ವಿದ್ಯಾರ್ಥಿಗಳಿಗೆ} \hbox{ತಿಳುವಳಿಕೆ} ನೀಡಿದೆ.  ಒಕ್ಕೊರಲಿನಿಂದ ಬಹಿರಂಗವಾಗಿ ಪ್ರತಿಭಟಿಸಬೇಕು. \hbox{ಪರೀಕ್ಷಕತ್ವ} ಸ್ಥಾನದಿಂದ ದೂರಿಡಬೇಕು ಎಂದು ದೂರು ನೀಡಿ ಪ್ರತಿಭಟನೆ ನಡೆಸಿದೆವು.  ಸಂಸ್ಕೃತ ಪಾಠಶಾಲೆಯ ಇತಿಹಾಸದಲ್ಲಿ ವಿದ್ಯಾರ್ಥಿಗಳು ಹೀಗೆ ಪ್ರತಿಭಟಿಸಿದ ನಿದರ್ಶನವೇ ಇರಲಿಲ್ಲ.  ನಾನಾಗ ವಿದ್ಯಾರ್ಥಿ ಪರಿಷತ್ತಿನಲ್ಲಿ ಸಕ್ರಿಯವಾಗಿ ಭಾಗವಹಿಸುತ್ತಿದೆ.  ಪರಿಷತ್ತಿನ ಸಹಾಯ ಪಡೆದು ಪ್ರತಿಭಟನೆ ಮಾಡಿದೆವು.  ಪ್ರಯತ್ನದಲ್ಲಿ ಯಶಸ್ಸು ದೊರೆಯಿತು.  ಸ್ವತಃ ಆಯುಕ್ತರು ಪಾಠಶಾಲೆಗೆ ಬಂದು ವಿದ್ಯಾರ್ಥಿಗಳನ್ನು ಕಣ್ಣಾರೆ ಕಂಡು ಮಾತನಾಡಿಸಿದಾಗ ಸಂಭವಿ\-ಸಿದ ಅನ್ಯಾಯದ ಆಳವನ್ನು ಮನಗಂಡು ನೊಂದರು.  ಇನ್ನು ಮುಂದೆ ಇಂಥ ಅಕ್ರಮ ನಡೆಯದಂತೆ ತಡೆಯುವುದಾಗಿ ಹೇಳಿ ಹೋದರು. 
~\\[0.4cm]
ಆದರೆ ನಾನು ಇಷ್ಟರಿಂದ ತೃಪ್ತನಾಗಲಿಲ್ಲ.  ವಿದ್ಯಾರ್ಥಿ ಪರಿಷತ್ತಿನ ಕಾರ್ಯಕ್ರಮಕ್ಕೆ ಮಾನ್ಯ ಶ್ರೀ ಬಿ.ಎಸ್. ಯಡಿಯೂರಪ್ಪನವರು ಬಂದಿದ್ದರು.  ಅವರನ್ನು ಪರಿಚಯಿಸಿಕೊಂಡು ಆದ ಅನ್ಯಾಯವನ್ನು ನಿವೇದಿಸಿದೆ.  ಅವರು ನಾಳೆಯೇ ಬೆಂಗಳೂರಿಗೆ ಬಂದು ವಿಧಾನಸೌಧದಲ್ಲಿ ಭೇಟಿಯಾಗಲು ತಿಳಿಸಿದರು.  ನಾನು ತಡಮಾಡದೆ ಮರುದಿನವೇ ವಿಧಾನಸೌಧಕ್ಕೆ ಹೋಗಿ ಶ್ರೀ ಯಡಿಯೂರಪ್ಪನವರನ್ನು ಭೇಟಿ ಮಾಡಿದೆ.  ಅವರು ನನ್ನನ್ನು ಆಗಿನ ಶಿಕ್ಷಣ ಸಚಿವರಾದ ಶ್ರೀ ರಾಚಯ್ಯನವರ ಹತ್ತಿರ ಕರೆದುಕೊಂಡು ಹೋಗಿ ಪರಿಚಯ ಮಾಡಿಸಿದರು.  ಸಚಿವರಲ್ಲಿ ನನ್ನ ಅಳಲು ತೋಡಿಕೊಂಡೆ.  ಸಹೃದಯರಾದ ಸಚಿವರು ಅನ್ಯಾಯದ ಬಗ್ಗೆ ಅರಿತು ಸರಿಪಡಿಸಲು ಮಾನ್ಯ ಆಯುಕ್ತರನ್ನು ಕರೆದು ಸೂಚಿಸಿ\-ದರು.  ಅದರಂತೆ ಆ ಸ್ವಾಮಿಗೆ ಈ ವಿಷಯಗಳ ಪರೀಕ್ಷಕತ್ವವನ್ನು ಮುಂದಿನ ವರ್ಷಗಳಲ್ಲಿ ನೀಡಲಿಲ್ಲ.  ಹಾಗಾಗಿ ನಾನು 1983 ರಲ್ಲಿ ಮರುಪರೀಕ್ಷೆಯನ್ನು ಎದುರಿಸಿ ಉತ್ತೀರ್ಣನಾದೆ.  
~\\[0.4cm]
ನಾನು ಆ ಸ್ವಾಮಿಯನ್ನು ಅಭಿನಂದಿಸುತ್ತೇನೆ.  ಅವನ ಮೊಂಡುತನದಿಂದ ಆದ ಅನುತ್ತೀರ್ಣತೆ ನನ್ನ ಕೆಚ್ಚನ್ನು ಹೆಚ್ಚಿಸಿತು.  1982ರಿಂದಲೇ ಯಾವ ವಿಷಯದಲ್ಲಿ ಅನುತ್ತೀರ್ಣ\-ನಾಗಿದ್ದೇನೋ ಅದೇ ದಿನಕರೀಯವನ್ನು ಬಯಸಿ ಬಂದ ವಿದ್ಯಾರ್ಥಿಗಳಿಗೆ ಪಾಠ\-ಮಾಡಲಾರಂಭಿಸಿದೆ. ನಮ್ಮನೆಗೇ ಬಂದು ಸಂಸ್ಕೃತ ಕಾಲೇಜಿನ ವಿದ್ಯಾರ್ಥಿಗಳು ಪಾಠ\-ಮಾಡಿಸಿಕೊಳ್ಳುತ್ತಿದ್ದರು.  1998ರಲ್ಲಿ ನಾನು ಸಹಾಯಕ ಪ್ರಾಧ್ಯಾಪಕನಾಗಿ ನಿಯುಕ್ತನಾಗುವವರೆಗೂ ವ್ರತದಂತೆ ಸತತವಾಗಿ ದಿನಕರೀಯವನ್ನು ಪಾಠಮಾಡಿದೆ.  ಹೀಗೆ ಪಾಠನ ಸಾತತ್ಯ\-ದಿಂದ ನನ್ನ ಅರಿವು ತನ್ನ ಆಳ ಅಗಲವನ್ನು ಹಿಗ್ಗಿಸಿಕೊಂಡಿತು. ಆದ್ದರಿಂದ ನಾನು ವಿದ್ಯಾರ್ಥಿಗಳಿಗೆ ಕೃತಜ್ಞನಾಗಿದ್ದೇನೆ.  ನಾನು ಪಾಠಮಾಡಿದ ವಿದ್ಯಾರ್ಥಿ\-ಗಳೊಂದಿಗೆ ನಾನು ಪರೀಕ್ಷಾ ಅಭ್ಯರ್ಥಿಯಾಗಿ ಪರೀಕ್ಷೆಯನ್ನು ಎದುರಿಸಿರುವುದೂ ಒಂದು ವಿಶೇಷ.

\section*{ಛಾತ್ರಸಂಪತ್ತು}

ನಾನು ಶ್ರೀ ಶಂಕರವಿಲಾಸ ಸಂಸ್ಕೃತ ಪಾಠಶಾಲೆಯ ಮುಖ್ಯೋಪಾಧ್ಯಾಯನಾಗಿ ಕಾರ್ಯ ನಿರ್ವಹಿಸುತ್ತಿರುವಾಗ ಪ್ರತಿ ವರ್ಷ ಶಾಲೆಗೆ ಪ್ರವೇಶ ಪಡೆಯಲು ಅನುಕೂಲವಾಗುವಂತೆ ಪತ್ರಿಕೆಗಳಲ್ಲಿ ಜಾಹೀರಾತು ನೀಡುತ್ತಿದ್ದೆ.  ಈ ಜಾಹೀರಾತನ್ನು ನೋಡಿ ರಾಜ್ಯಪುರಾತತ್ತ್ವ ಇಲಾಖೆಯ ಕೆಲವರು ಸಂಸ್ಕೃತಾಧ್ಯಯನಕ್ಕೆ ಪ್ರವೇಶ ಪಡೆದರು.  ಅವರಲ್ಲಿ ಶಾಸನತಜ್ಞ ಪಾಟೀಲ್ ಎಂಬ ಪದವೀಧರರು ಒಬ್ಬರು.  ಪಾಟೀಲರು ಒಂದು ದಿನ ಶಿಲ್ಪದ ಪಡಿಯಚ್ಚನ್ನು ತಂದು ಇದೇನೆಂದು ಕೇಳಿದರು.  ಪರಾಮರ್ಶಿಸಿದಾಗ ಪಂಚತಂತ್ರದ ಕಥೆಯನ್ನು ಶಿಲ್ಪದಲ್ಲಿ ಬಿಂಬಿಸಿರುವುದು ಗೊತ್ತಾಯಿತು.  ಇಂಥ ಅನೇಕ ಶಿಲ್ಪಗಳು ಬೆಳ್ಳಾವೆ ದೇವಾಲಯದ ಆವರಣದಲ್ಲಿವೆ.  ಅವುಗಳ ಕಥೆಯನ್ನು ವಿವರಿಸಬೇಕೆಂದರು.  ಶಿಲ್ಪದ ಹಿನ್ನೆಲೆಯಲ್ಲಿ ಕಥೆಯನ್ನು ವಿವರಿಸಿದೆ.  ಅತ್ಯಂತ ಸಂತುಷ್ಟರಾದ ಪಾಟೀಲರು ನಾನು ಸಂಸ್ಕೃತ ಅಧ್ಯಯನ ಮಾಡಲು ಬಂದಿರುವುದು ಸಾರ್ಥಕವಾಯಿತು ಎಂದರು.  ಈ ಕಥೆಯನ್ನು ಆಧರಿಸಿ ಗ್ರಂಥವನ್ನು ರಚಿಸಬೇಕಿದೆ.  ಅದಕ್ಕೆ  ತಮ್ಮ ಸಹಕಾರ ಬೇಕೆಂದು ಕೇಳಿದರು. ಒಪ್ಪಿದೆ.  ಅದರಂತೆ  ‘\textbf{ಶಿಲ್ಪದಲ್ಲಿ ಪಂಚತಂತ್ರ}’ ಎಂಬ ಗ್ರಂಥ ರಚಿಸಿದರು. ವಿಭಿನ್ನ ಕ್ಷೇತ್ರದಲ್ಲಿ ಕಾರ್ಯ ನಿರ್ವಹಿಸುವ ಜನರಿಗೆ ಸಂಸ್ಕೃತದ ಕಿರುಪರಿಚಯವೂ ಮಹಾಕಾರ್ಯ ಸಾಧನೆಗೆ ಸಹಕಾರಿಯಾಗುತ್ತದೆಂಬ ಬಗ್ಗೆ ಸಾಕ್ಷ್ಯ ದೊರೆಯಿತು.  
~\\[0.3cm]
ಇದೇ ರೀತಿ ಜಾಹಿರಾತು ಗಮನಿಸಿ ಮೈಸೂರಿನ ಆಯುರ್ವೇದ ವಿದ್ಯಾಲಯದ ವಿದ್ಯಾರ್ಥಿಗಳಿಬ್ಬರು ಕಾವ್ಯ ತರಗತಿಗೆ ಪ್ರವೇಶ ಪಡೆದರು.  ಸಂಸ್ಕೃತ ಭಾಷೆಯ ಮೂಲಭೂತ ಅಂಶಗಳ ಪರಿಚಯವಾದ ತಕ್ಷಣ ಸಂತೃಪ್ತರಾದರು.  ಆಯುರ್ವೇದ ವಿದ್ಯಾರ್ಥಿ\-ಗಳಿಗೆ ಪಠ್ಯಗ್ರಂಥ ಸಂಸ್ಕೃತ ಮೂಲದ್ದು.   ಅದನ್ನು ಇತರ ಭಾಷೆಗಳ ಅನುವಾದದ ಮೂಲಕವಾಗಿಯೇ ಅಧ್ಯಯನ ಮಾಡಬೇಕಾದ ದುರ್ವಿಧಿ, ಅದು ಅನಿವಾರ್ಯ\-ವಾಗಿತ್ತು. ಸಂಸ್ಕೃತದ ಮೂಲಕವೇ ಆಯುರ್ವೇದ ಗ್ರಂಥವನ್ನು ಅಧ್ಯಯನ ಮಾಡುವ ಆಸಕ್ತಿಯಿಂದ ವಿದ್ಯಾರ್ಥಿಗಳು ಚರಕಸಂಹಿತೆಯನ್ನು ಚಕ್ರಪಾಣಿ ವ್ಯಾಖ್ಯಾನದ ಸಹಿತ ಪಾಠಮಾಡಬೇಕೆಂಬ ಬೇಡಿಕೆಯಿಟ್ಟರು. ಆಗುವುದಿಲ್ಲ ಎನ್ನಲಿಲ್ಲ,  ಒಪ್ಪಿಕೊಂಡೆ.  ಪಾಠ ಪ್ರಾರಂಭ\-ವಾಯಿತು.
~\\[0.3cm]
ಚರಕಸಂಹಿತೆ ವೈಶೇಷಿಕದರ್ಶನದ ದಾರ್ಶನಿಕ ಹಿನ್ನೆಲೆಯ ಗ್ರಂಥವಾದ್ದರಿಂದ ನನಗೆ ಅತ್ಯಂತ ಆಸಕ್ತಿ ಕೆರಳಿತು. ವೈಶೇಷಿಕದರ್ಶನದ ಷಟ್‍ಪದಾರ್ಥಚಿಂತೆಯನ್ನು ಆರೋಗ್ಯ\-ಶಾಸ್ತ್ರದಲ್ಲಿ ಪ್ರಾಯೋಗಿಕವಾಗಿ ಅನ್ವಯಿಸಿರುವುದು ಅತ್ಯಂತ ಗಂಭೀರ ವಿಷಯ.  ಸಂಸ್ಕೃತ ಶಾಸ್ತ್ರ ಚಿಂತನೆಯ ಅನ್ವಯಿಕ ಮುಖ ಕಂಡು ಹೆಮ್ಮೆಯೆನಿಸಿತು. ವಿದ್ಯಾರ್ಥಿಗಳು ಶ್ರದ್ಧೆಯಿಂದ ಚಕ್ರಪಾಣಿ ವ್ಯಾಖ್ಯಾನವನ್ನು ಓದಿದರು.  ಅವರ ಪರಿಚಿತ ಕೆಲವು ವಿದ್ಯಾರ್ಥಿಗಳೂ ಈ ಪಾಠಕ್ಕೆ ಬರಲಾರಂಭಿಸಿದರು.  ಇದು ಆಯುರ್ವೇದ ಕಾಲೇಜಿನಲ್ಲಿ ಒಂದು ಹೊಸ ವಾತಾವರಣ ಸೃಷ್ಟಿಸಿತು.  ಏಕೆಂದರೆ ಇಂಥ ಉದ್ಗ್ರಂಥವನ್ನು  ಆಂಗ್ಲ ಅಥವಾ ಕನ್ನಡಾನು\-ವಾದದ ಮೂಲಕ ಅಧ್ಯಯನ ಮಾಡುತ್ತಿರುವ ವಿದ್ಯಾರ್ಥಿಗಳಿಗೆ ಶಾಸ್ತ್ರೀಯ ವಿಶ್ಲೇಷಣೆ ಸೂಕ್ಷ್ಮ ಅಂಶಗಳು ದೊರೆಯುತ್ತಿರಲಿಲ್ಲ.  ಈ ಪಾಠದಿಂದ ಮೂಲಭೂತ ಮಾಹಿತಿ ದೊರೆಯಿತು.  ಅದನ್ನು ಎಲ್ಲರಿಗೂ ಹಂಚಬೇಕೆಂದು ಬಯಸಿದ ವಿದ್ಯಾರ್ಥಿಗಳು ಅವರ ಶಿಕ್ಷಕರಲ್ಲಿ ನಿವೇದಿಸಿದಾಗ ಒಂದು ಕಾರ್ಯಕ್ರಮವನ್ನೇ ರೂಪಿಸಿದರು.  ವಾರಕ್ಕೊಮ್ಮೆ ಆಯುರ್ವೇದ ಕಾಲೇಜಿನಲ್ಲಿ ಅಂತಿಮ ಪದವಿಯ ವಿದ್ಯಾರ್ಥಿಗಳು ಹಾಗೂ\break ಶಿಕ್ಷಕರ ಸಮ್ಮುಖದಲ್ಲಿ ಚಕ್ರಪಾಣಿ ವ್ಯಾಖ್ಯಾನದ ಚಿಂತನೆ ಮಾಡಲು ನನ್ನನ್ನು ಕೇಳಿದರು.  ಒಪ್ಪಿದೆ.  ನನಗೆ ಪ್ರಾಯೋಗಿಕ ಅರಿವಿಲ್ಲ.  ಭಾಷಾಪ್ರಭುತ್ತ್ವವಿದೆ. ಅವರಿಗೆ ಪ್ರಾಯೋಗಿಕ ಅರಿವಿದೆ. ಭಾಷೆಯ ಬಿಗುವಿಲ್ಲ.  ಚಿಂತನೆಯಿಂದ ಪರಸ್ಪರರು ತಿಳುವಳಿಕೆಯನ್ನು ನೇರ್ಪುಗೊಳಿಸಿಕೊಂಡೆವು.  ಆಗ ನನಗೂ ಆಯುರ್ವೇದ ವಿದ್ಯಾರ್ಥಗಳಿಗೂ ಒಂದು ಹಿತವಾದ ಸಂಬಂಧ ಏರ್ಪಟ್ಟಿತು.  ಇದರ ಪರಿಣಾಮವಾಗಿ ಆಯುರ್ವೇದ ವಿದ್ಯಾರ್ಥಿಗಳಿಗಾಗಿ ಅಕ್ಷರಾಭ್ಯಾಸದಿಂದ ಪ್ರಾರಂಭಿಸಿ ಪದ, ಸಂಧಿ, ಸಮಾಸ, ಕಾರಕ, ವಾಕ್ಯ, ಕೃದಂತ, ತದ್ಧಿತಾಂತ ಅನ್ವಯ ಕ್ರಮಗಳ ಬಗ್ಗೆ ಪಾಠಮಾಡಿ ಶ್ಲೋಕವನ್ನು ಅರ್ಥೈಸುವ ಕಲೆಯನ್ನು ಹೇಳಿಕೊಟ್ಟೆ.  ಇದರಿಂದ ಪ್ರೇರಿತರಾದ ಕೆಲವು ವಿದ್ಯಾರ್ಥಿಗಳು 2007 ರಿಂದಲೂ ನಮ್ಮ ಕಾಲೇಜಿನ ಯೋಗಾಸನ ವಿಭಾಗದ ಎದುರಿನಲ್ಲಿ ದಿನಾಲೂ ಬೆಳಿಗ್ಗೆ ಬಂದು ಆಯುರ್ವೇದ ಗ್ರಂಥದ\ ಮೂಲವನ್ನು ಪಾರಾಯಣ ಮಾಡುತ್ತಿದ್ದಾರೆ.  ಹಿಂದೆ ಇದೇ ಸಂಸ್ಕೃತ ಕಾಲೇಜಿನಲ್ಲಿ ಮಹಾರಾಜರು ಆಯುರ್ವೇದ ವಿಭಾಗದ ಅಧ್ಯಯನಕ್ಕೂ ಅವಕಾಶ ಕಲ್ಪಿಸಿದ್ದರು.  ಕಾಲಕ್ರಮದಲ್ಲಿ ಆಯುರ್ವೇದ ಅಧ್ಯಯನ ವಿಭಾಗ ಸಂಸ್ಕೃತ ಕಾಲೇಜಿನಿಂದ ಬೇರ್ಪಟ್ಟಿತು.  ಇದರಿಂದ ಎರಡೂ ಅಧ್ಯಯನ ಶಾಖೆ ಸೊರಗಿತು.  ಸಂಸ್ಕೃತದ ಸಂಗ ತೊರೆದು ಆಯುರ್ವೇದ ಬಾಡಿತು.  ಪ್ರಾಯೋಗಿಕ  ಮೌಲ್ಯವಿರುವ  ಆಯುರ್ವೇದವನ್ನು ಕಡೆ\-ಗಣಿಸಿ ಸಂಸ್ಕೃತ ಸೊರಗಿತು.  ಆದರೆ ನಮ್ಮ ಕಿರು ಪ್ರಯತ್ನದಿಂದ ಪುನಃ ಆಯುರ್ವೇದಕ್ಕೆ ಸಂಸ್ಕೃತ ಪಾಠಶಾಲೆಯ ಹಾಗೂ ಸಂಸ್ಕೃತದ ನಂಟು ಮೂಡಿದ್ದು ಸಮಾಜದ ಸೌಭಾಗ್ಯ.   

\section*{ಕಲಿಸುತ್ತಾ ಕಲಿತೆ}

ನಾನು ಖಾಸಗಿ ವಿದ್ಯಾರ್ಥಿಯಾಗಿ ಓದಿದ್ದರಿಂದ ಸ್ಪರ್ಧೆಗಳಿಂದ ದೂರನಾಗಿದ್ದೆ.  ಆದರೆ ಪದವಿಪೂರ್ವ ಮತ್ತು ಪದವೀ ಶಿಕ್ಷಣವನ್ನು ಸಂಸ್ಕೃತದ ಜೊತೆ ಜೊತೆಗೇ ಮುಂದುವರಿಸಿದ್ದೆ.  ಆದ್ದರಿಂದ  ಆ ಕಾಲೇಜಿನ ಮೂಲಕ ಚರ್ಚಾಸ್ಪರ್ಧೆಗೆ ಹೋಗುತ್ತಿದ್ದೆ.  ಮೈಸೂರಿನಲ್ಲಿ ಆಗ ಕಾಲೇಜು ಹಂತದಲ್ಲಿ ಸಾಂಸ್ಕೃತಿಕ ಚಟುವಟಿಕೆಗಳಿಗೆ ಬರವಿರಲಿಲ್ಲ.  ನಾನು ಮರಿಮಲ್ಲಪ್ಪನವರ ಕಾಲೇಜಿನಿಂದ ಸ್ಪರ್ಧೆಗಳಲ್ಲಿ ಭಾಗವಹಿಸುತ್ತಿದ್ದೆ.  ಸತತವಾಗಿ ಚರ್ಚೆಯಲ್ಲಿ ಭಾಗವಹಿಸಿದ್ದರಿಂದ ಭಾಷೆ ತೀವ್ರಗೊಂಡಿತು.  ವಿಷಯ ಸಂಗ್ರಹವಾಯಿತು.  ಎಲ್ಲೆಡೆ ಪ್ರಥಮ ಬಹುಮಾನ ಬರುವುದು ಸಾಮಾನ್ಯವಾಯಿತು.  ನಾನು ವಿದ್ವನ್ಮಧ್ಯಮಾ ಮೊದಲನೆಯ ತರಗತಿಯಲ್ಲಿ ಸೇರಿದಾಗ ಭೋಜನಕ್ಕೆ ವೇದಶಾಸ್ತ್ರ ಪೋಷಿಣೀ ಸಭೆಯಲ್ಲಿ ಅವಕಾಶ ದೊರಕಿತು.  ಅಲ್ಲಿ ಉಮಾಕಾಂತನೂ ಊಟಮಾಡುತ್ತಿದ್ದ.  ಹಾಗಾಗಿ ನಾನು ಉಮಾಕಾಂತ ಮೈಸೂರಿನ ವಿದ್ಯಾರ್ಥಿನಿಲಯಗಳಲ್ಲಿ ನಡೆಯುವ ಚರ್ಚಾ \hbox{ಸ್ಪರ್ಧೆಗಳಲ್ಲಿ} ಭಾಗವಹಿಸುತ್ತಿದ್ದೆವು.  ಪರ್ಯಾಯದಂತೆ ನನಗೂ ಉಮಾಕಾಂತನಿಗೂ ಪ್ರಥಮ, ದ್ವಿತೀಯ ಬಹುಮಾನಗಳು ಬರಲಾರಂಭಿಸಿದವು.  ಆದ್ದರಿಂದ ಎಲ್ಲಾ ಕಡೆ ಪರ್ಯಾಯ ಪಾರಿ\-ತೋಷಕ ನಮ್ಮ ಕಾಲೇಜಿನ ಪಾಲಾಗುತ್ತಿತ್ತು. ಇದರ ಅಭಿಜ್ಞಾಪಕವಾಗಿ ಈಗಲೂ \hbox{ಮಹಾರಾಜ} ಸಂಸ್ಕೃತ ಕಾಲೇಜಿನ ಪ್ರಾಂಶುಪಾಲರ ಕೊಠಡಿಯಲ್ಲಿ ಈ ಪಾರಿತೋಷಕಗಳು ಕಂಗೊಳಿಸುತ್ತಿವೆ.  
~\\[0.3cm]
ಮಾನವ ಸಂಪನ್ಮೂಲ ಸಚಿವಾಲಯವು ರಾಷ್ಟ್ರಮಟ್ಟದಲ್ಲಿ ಸಂಸ್ಕೃತ ವಾಕ್ ಪ್ರತಿ\-ಯೋಗಿತೆಯನ್ನು ನಡೆಸುತ್ತಿದೆ.  ನಾನು ವಿದ್ಯಾರ್ಥಿಯಾಗಿ ಈ ಸ್ಪರ್ಧೆಗಳಲ್ಲಿ ಭಾಗ\-ವಹಿಸಲು ಆಗಲೇ ಇಲ್ಲ.  ಆದರೆ ಕಾಲೇಜಿನಲ್ಲಿ ನಿಯುಕ್ತನಾದ ನಂತರ ಕನಿಷ್ಠ ಹತ್ತು ವರ್ಷಗಳ ಕಾಲ ಕರ್ನಾಟಕ ರಾಜ್ಯದ ವಿದ್ಯಾರ್ಥಿಗಳ ಮಾರ್ಗದರ್ಶಕನಾಗಿ ನಿಯುಕ್ತನಾದ್ದರಿಂದ ಭಾಗವ\-ಹಿಸಬೇಕಾಯಿತು.  ರಾಜ್ಯಮಟ್ಟದಲ್ಲಿ ನಮ್ಮ ಕಾಲೇಜಿನ ವಿದ್ಯಾರ್ಥಿಗಳು\break ಅತ್ಯಧಿಕ ಬಹುಮಾನ ಗಳಿಸುವಂತೆ ವಿದ್ಯಾರ್ಥಿಗಳನ್ನು ಸಜ್ಜುಗೊಳಿಸುತ್ತಿದ್ದೆ.  ಅದರಂತೆ ರಾಷ್ಟ್ರಮಟ್ಟದಲ್ಲಿಯೂ ಕರ್ನಾಟಕದ ವಿದ್ಯಾರ್ಥಿಗಳು ಅಧಿಕ ಬಹುಮಾನ\break ಗಳಿಸುತ್ತಿದ್ದರು.  ಇದರ ಹೆಗ್ಗುರುತಾಗಿ ವಿಜಯ ವೈಜಯಂತಿ ಪಾರಿತೋಷಕವು\break ಕರ್ನಾಟಕದ ಪಾಲಾಗುತ್ತಿತ್ತು.  ಅದರಲ್ಲಿಯೂ ಮೈಸೂರು ಮಹಾರಾಜ ಪಾಠಶಾಲೆಯ ವಿದ್ಯಾರ್ಥಿ\-ಗಳು ಅಧಿಕ ಬಹುಮಾನ ಗಳಿಸಿದ್ದರಿಂದ ವಿಜಯ \hbox{ವೈಜಯಂತಿಯನ್ನು} ನಮ್ಮ ಕಾಲೇಜಿನಲ್ಲಿಯೇ ಇಡಲಾಗಿದೆ.  ಈಗಲೂ ಪ್ರಾಂಶುಪಾಲರ ಕೊಠಡಿಯಲ್ಲಿ ವಿಜಯ\hbox{ವೈಜಯಂತಿಯ} ಪರ್ಯಾಯ ಪಾರಿತೋಷಕಗಳು ಶೋಭಿಸುತ್ತಿವೆ.  ಹೀಗೆ ವಿಭಿನ್ನ ಶಾಸ್ತ್ರದ ವಿಷಯಗಳನ್ನು ಸಿದ್ಧಪಡಿಸಬೇಕಾಗಿರುವುದರಿಂದ ಎಲ್ಲಾ ಶಾಸ್ತ್ರಗಳ\break ಅವಲೋಕನ ಮಾಡುವ ಅನಿವಾರ್ಯತೆಯನ್ನು ವಿದ್ಯಾರ್ಥಿಗಳೇ ಸೃಷ್ಟಿಸಿದರು.   ಆದ್ದರಿಂದ ನಾನು ವಿದ್ಯಾರ್ಥಿಗಳಿಂದ ಕಲಿತೆ ಎಂಬುದು ನನಗೆ ಮನದಟ್ಟಾಗಿದೆ.  

ಹೀಗೆ ಮೈಸೂರು ನನ್ನನ್ನು ದಿನೇ ದಿನೇ ಬೆಳೆಸಿತು.  ಶಾಸ್ತ್ರಚಿಂತನೆಯಲ್ಲಿ ತೊಡಗಿಸಿತು.  ಮೈಸೂರಿಗೆ ಬಂದ ನಂತರ ಸಂಸ್ಕೃತದ ಜೊತೆಗೆ ಬನುಮಯ್ಯನವರ ಕಾಲೇಜಿನಲ್ಲಿ ವಾಣಿಜ್ಯ ಪದವಿಯನ್ನು ಪಡೆದೆ. ಆಮೇಲೆ ಮುಕ್ತ ವಿಶ್ವವಿದ್ಯಾಲಯದಿಂದ ಸಂಸ್ಕೃತ ಎಮ್.ಎ ಪದವಿ ಪಡೆದೆ. ಹಾಗೆಯೇ ಬಿ.ಎಡ್ ಮಾಡಬೇಕೆಂದು ಮುಕ್ತ ವಿಶ್ವ\-ವಿದ್ಯಾಲಯದ ಮುಖೇನ ಪ್ರವೇಶ ಪಡೆದೆ. ಪದವಿ ಮುಗಿಸುವ ಮೊದಲೇ ಮಹಾರಾಜ ಕಾಲೇಜಿ\-ನಲ್ಲಿ ಉದ್ಯೋಗ ದೊರಕಿದ್ದರಿಂದ ಪರೀಕ್ಷೆಗೆ ಹೋಗಲಿಲ್ಲ.  

ಬಿ.ಎಡ್ ತರಗತಿಯನ್ನು ಓದುತ್ತಿದ್ದಾಗ ಒಂದು ಘಟನೆ ನಡೆಯಿತು.  ಬಿ.ಎಡ್ ವಿದ್ಯಾರ್ಥಿಗಳಿಗೆ ಸಂಪರ್ಕ ತರಗತಿ ನಡೆಸುತ್ತಿದ್ದರು. ಶೈಕ್ಷಣಿಕ ಮನೋವಿಜ್ಞಾನದ \hbox{ತರಗತಿ} ನಡೆಯುತ್ತಿತ್ತು.  ಶಿಕ್ಷಕರೊಬ್ಬರು ಶಿಕ್ಷಣ ಪದ್ಧತಿಗಳನ್ನು ವಿಶ್ಲೇಷಿಸುತ್ತಿದ್ದರು.  \hbox{ಭಾರತದಲ್ಲಿ} ಪ್ರಾಚೀನ ಶಿಕ್ಷಣ ಪದ್ಧತಿಯಲ್ಲಿ ವಿದ್ಯಾರ್ಥಿಗಳಿಗೆ ಹೊಡೆಯುವುದು \hbox{ಸಾಧಾರಣವಾಗಿತ್ತು.} ದಂಡನೆಯಿಲ್ಲದೆ ಶಿಕ್ಷಣವಿಲ್ಲವೆಂಬ ಸಿದ್ಧಾಂತವಿತ್ತು.  ಅದಕ್ಕೆ ‘\textbf{ದಂಡಂ ದಶಗುಣಂ ಭವೇತ್}’ ಎಂದು ಹೇಳಲಾಗಿದೆ ಎಂದರು.  ನಾನು ಶೀಘ್ರವಾಗಿ ಪ್ರತಿಭಟಿಸಿದೆ.  ಸಂಸ್ಕೃತ\-ವನ್ನು ತಿಳಿಯದೇ ಸಂಸ್ಕೃತದ ನುಡಿಗಟ್ಟನ್ನು ಬಳಸುವ ಪರಿಪಾಠವಿದೆ.  ದಂಡಂ ದಶಗುಣಂ ಭವೇತ್ ಎಂಬ ನುಡಿ ಶಿಕ್ಷಣದಲ್ಲಿ ದಂಡನೆಯ ಅಗತ್ಯವನ್ನು ಹೇಳುವ ನುಡಿಯಲ್ಲ.  ಆದರೆ ಒಂದು ದಂಡ  \enginline{-}  ಕೋಲು ಹತ್ತಾರು ಬಗೆಯಲ್ಲಿ ಉಪಯುಕ್ತ ಎಂಬುದನ್ನು ಸಾರುವ ಸ್ವಾರಸ್ಯಕರವಾದ ನಾಣ್ಣುಡಿ.  ಇದರ ಪೂರ್ಣಪಾಠ \hbox{ಹೀಗಿದೆ.}  ಇದನ್ನು ಕೇಳಿದರೂ ಅರ್ಥವಾಗದು. 
\begin{verse}
ವಿಶ್ವಾಮಿತ್ರಾಹಿಪಶುಷು ಕರ್ದಮೇಷು ಜಲೇಷು ಚ~।\\
ಅಂಧೇ ತಮಸಿ ವಾರ್ಧಕ್ಯೇ ದಂಡಂ ದಶಗುಣಂ ಭವೇತ್~॥
\end{verse}
ಇಲ್ಲಿ \textbf{ವಿ}  \enginline{-}  ಹಕ್ಕಿ, \textbf{ಶ್ವಾ}  \enginline{-}  ನಾಯಿ, \textbf{ಅಮಿತ್ರ}  \enginline{-}  ವೈರಿ, ಇವು ಸೇರಿ ವಿಶ್ವಾಮಿತ್ರ.  ಈ ವಿಶ್ವಾಮಿತ್ರ ನಮಗೆ ಪರಿಚಿತವಾಗಿರುವ ಮುನಿಯಲ್ಲ.  \textbf{ಅಹಿ}   \enginline{-}   ಸರ್ಪ, \textbf{ಪಶು}   \enginline{-}   ಪ್ರಾಣಿ, \textbf{ಕರ್ದಮ}   \enginline{-}   ಕೆಸರು, \textbf{ಜಲ}   \enginline{-}   ನೀರು.  ಹೀಗೆ ಹತ್ತು ಅಪಾಯ ಸನ್ನಿವೇಶದಲ್ಲಿ ರಕ್ಷಣೆ ಮಾಡುವುದರಿಂದ ದಂಡವು ದಶಗುಣವುಳ್ಳದ್ದು   \enginline{-}   ಹತ್ತುವಿಧವಾಗಿ ಉಪಯೋಗಿಸಲು ಅದರಲ್ಲಿ ಆನುಕೂಲ್ಯವಿದೆ ಎಂಬುದು ಈ ಶ್ಲೋಕದ ತಾತ್ಪರ್ಯ.   ಈ ವಿವರಣೆಯನ್ನು ಕೇಳಿ ಆ ಶಿಕ್ಷಕರಿಗೂ ಸಹಪಾಠಿಗಳಿಗೂ ಅತ್ಯಾನಂದವಾಯಿತು.  ಸಂಸ್ಕೃತದ ಅಪಬಳಕೆಯನ್ನು ಕಂಡಲ್ಲಿ ಅಲ್ಲಿಯೇ ವಿರೋಧಿಸುವುದು ನನ್ನ ಜಾಯಮಾನ.  

ವಿದ್ಯಾರ್ಥಿ ಪರಿಷತ್‍ನ ಒಂದು ಕಾರ್ಯಕ್ರಮ ನಡೆದಿತ್ತು.  ಈ ಕಾರ್ಯಕ್ರಮದ ಅಧ್ಯಕ್ಷತೆ\-ಯನ್ನು ಶಾರದಾ ವಿಲಾಸ ಕಾನೂನು ಕಾಲೇಜಿನ ಪ್ರಾಂಶುಪಾಲರು ವಹಿಸಿದ್ದರು.  ಮಾತನಾಡುವಾಗ ರಾಮರಾಜ್ಯದ ಬಗ್ಗೆ ಪ್ರಸ್ತಾಪಿಸುತ್ತಾ ಒಂದು ಕುಹಕವನ್ನು ನುಡಿದರು.  ರಾಮನ ಪಟ್ಟಾಭಿಷೇಕದಲ್ಲಿ ಅಯೋಧ್ಯೆಯ ಹೊರಗಡೆ ಒಂದು ಹಂಡೆ ಇಟ್ಟಿದ್ದ\-ರಂತೆ.  ಅದರಲ್ಲಿ ಹಾಲನ್ನು ಹಾಕಿ ತುಂಬಲು ಪ್ರಜೆಗಳಿಗೆ ಆದೇಶಿಸಿದ್ದರಂತೆ.  ತಾನೊಬ್ಬ ನೀರು ಹಾಕಿದರೆ ಯಾರಿಗೂ ತಿಳಿಯದೆಂದು ಪ್ರಜೆಗಳು ಆಲೋಚಿಸಿದರಂತೆ. ಪರೀಕ್ಷಿ\-ಸಲಾಗಿ ಹಂಡೆಯ ತುಂಬಾ ನೀರೇ ಇತ್ತಂತೆ.  ಯಾರೂ ಹಾಲನ್ನು ಹಾಕಲಿಲ್ಲವಂತೆ.  ಅಂದರೆ ರಾಜಶಾಸನವನ್ನು ಪರಿಪಾಲಿಸದ ಪ್ರಜೆಗಳಿರುವ ರಾಜ್ಯ ರಾಮರಾಜ್ಯ.  ಇದೆಂಥ ಆದರ್ಶ ಎಂದು ಗಹಗಹಿಸಿ ನಕ್ಕರು.  ಜನ ಎಂದಿನಂತೆ ಚಪ್ಪಾಳೆ ತಟ್ಟಿದರು.  ಅಂದು ನಾನು ವಂದನಾರ್ಪಣೆಯನ್ನು ಮಾಡಬೇಕಿತ್ತು.  ವಂದನಾರ್ಪಣೆಗೆ ಪ್ರಾರಂಭಿಸುವ ಮೊದಲು ಅಧ್ಯಕ್ಷರಿಗೆ ಸವಾಲನ್ನು ಎಸೆದೆ  “ವಾಲ್ಮೀಕಿ ರಾಮಾಯಣದ ಯಾವ ಕಾಂಡದಲ್ಲಿ ಈ ವೃತ್ತಾಂತವನ್ನು ತಾವು ಓದಿದ್ದೀರಿ?  ನಾನು ಓದಿದ ರಾಮಾಯಣದಲ್ಲಿ ಈ ವೃತ್ತಾಂತವಿಲ್ಲ !” ಎಂದು.  ಇದರಿಂದ ಕಂಗಾಲಾದ ಅಧ್ಯಕ್ಷರು, “ಇದು ರಾಮಾಯಣದ ಕಥೆಯಲ್ಲ. ಇದು ಕರ್ಣಾಕರ್ಣಿಯಾಗಿ ಪ್ರಸಿದ್ಧವಿರುವ ವೃತ್ತಾಂತ” ಎಂದರು.  ಆಗ ನಾನು ಹೇಳಿದೆ.  “ಇದು ಮೂಲಕ್ಕೆ ಅಪಚಾರ.  ಅಲ್ಲಿ ಕರ್ಣಾಕರ್ಣಿಯಾಗಿ ಕೇಳಿದ ಇನ್ನೊಂದು ಕಥೆ ನನಗೆ ಗೊತ್ತು. ಅಂದು ಮಾಡಿದ ಆದೇಶದ ಪೂರ್ಣಪಾಠ ಇದು.  ‘\textbf{ಭರತಾಂ ಪಯಃ}’ ಎಂದು.  ಪ್ರಜೆಗಳು ವಿವೇಕಿಗಳಾಗಿದ್ದರು.  ಪಯಃ ಎಂಬ ಶಬ್ದಕ್ಕೆ ನೀರು ಹಾಲು ಎಂಬ ಎರಡು ಅರ್ಥವಿರುವುದು ಅವರಿಗೆ ಗೊತ್ತಿತ್ತು.  ರಾಜಾಭಿಷೇಕ ಪ್ರಸಂಗವಾದ್ದರಿಂದ ಪ್ರಜೆಗಳು ಪಯಃ ಎಂಬ ಪದಕ್ಕೆ ನೀರು ಎಂದೆ ಅರ್ಥೈಸಿ ನೀರನ್ನು ತುಂಬಿದ್ದಾರೆ. ಆದ್ದರಿಂದ ವಿವೇಚನಾ ಶಕ್ತಿಯಿರುವ ಪ್ರಜೆಗಳನ್ನು ಹೊಂದಿದ ರಾಜ್ಯ ರಾಮರಾಜ್ಯ.  ಇದು ಆದರ್ಶ” ಎಂದೆ.  ಈ ಘಟನೆ ನಡೆದ ಸಮಯದಲ್ಲಿ ನಾನು ಸಾಹಿತ್ಯ ತರಗತಿಯ ವಿದ್ಯಾರ್ಥಿ\-ಯಾಗಿದ್ದೆ.  ಸಂಸ್ಕೃತ ಭಾಷೆ, ಸಾಹಿತ್ಯಗಳ ಆಶಯಕ್ಕೆ ಧಕ್ಕೆ ಬಂದರೆ ತಕ್ಷಣ ಪ್ರತಿಕ್ರಿಯಿಸುವ ಪ್ರವೃತ್ತಿ ಇನ್ನೂ ಹಾಗೆಯೇ ಮುಂದುವರೆದಿದೆ.   

ಮಂಗಳೂರಿನಲ್ಲಿ ಸಂಸ್ಕೃತ, ಕನ್ನಡ, ಹಿಂದಿ, ಇಂಗ್ಲೀಷ್ ಭಾಷಾ ಉಪನ್ಯಾಸಕರಿಗಾಗಿ ಕಾರ್ಯಾಗಾರವನ್ನು ಏರ್ಪಡಿಸಿದ್ದರು.  ಸಂಪನ್ಮೂಲ ವ್ಯಕ್ತಿಯಾಗಿ ನಾನು ಭಾಗವಹಿಸಿದ್ದೆ.  ಸಂಸ್ಕೃತ ಭಾಷಾ ಶಿಕ್ಷಕರು ವಿಶೇಷವಾಗಿ ಶುದ್ಧ ಉಚ್ಚಾರಣೆಯ ಬಗ್ಗೆ ಗಮನ ಹರಿಸ\-ಬೇಕೆಂದು ತಿಳಿಸುತ್ತಾ ವ್ಯಾಪಕವಾಗಿ ಕೇಳಿಬರುವ ಉಚ್ಚಾರಣಾ ದೋಷಗಳನ್ನು ಸರಿಪಡಿಸುವ ಉಪಾಯಗಳ ಬಗ್ಗೆ ತಿಳಿಸಿದೆ.  ಉಪನ್ಯಾಸಕರಲ್ಲಿ ಹಲವರು ಸಿಡಿಮಿಡಿಗೊಂಡರು.  “ಈ ದೇಶದ ಗ್ರಾಮೀಣ ಜನರಿಗೆ ನೀವು \textbf{ಹವಮಾನ} ಮಾಡುತ್ತಿದ್ದೀರಿ.  \textbf{ಹಾದ್ದರಿಂದ} ನಾವು ಪ್ರತಿಭಟಿಸುತ್ತೇವೆ” ಎಂದರು.  ಅನೇಕರ ವಿರೋಧವನ್ನು ಕಂಡು ವ್ಯವಸ್ಥಾಪಕರು ಕಂಗಾಲಾದರು. ನಾನು ಕ್ಷಮೆಯಾಚಿಸುತ್ತೇನೆ  ಎಂದೆ.   ಆದರೆ ನಿಮ್ಮ ಆಲಸ್ಯವನ್ನು ಸಹಿಸು\-ವುದಿಲ್ಲ.  ನಿಮಗೆ ಹ ಕಾರವನ್ನು, ಅ ಕಾರವನ್ನು ಉಚ್ಚರಿಸುವ ಶಕ್ತಿಯಿದೆ.  ಉಚ್ಚಾರಣಾಂಗದ ವೈಕಲ್ಯವಿಲ್ಲ.  ಆದರೆ ಅದಲು ಬದಲಾಗಿ ಉಚ್ಚರಿಸುತ್ತೀರಿ.  ಇದನ್ನು ಹೇಳಿದರೆ ನಿಮಗೆ ಅವಮಾನವಾಗುತ್ತದೆ ಎಂದರೆ ಭಾಷೆಗೂ ಮಾನ  \enginline{-}  ಮರ್ಯಾದೆ ಇದೆ.  ಅದನ್ನು ಮೀರಬಾರದು.  ಭಾಷಾಶಿಕ್ಷಕರ ಆದ್ಯ ಕರ್ತವ್ಯವೇ ಭಾಷಾ ಮರ್ಯಾದಾ ರಕ್ಷಣೆ ಎಂದು ಸಮರ್ಥಿಸಿಕೊಂಡೆ.  ವಿರೋಧದ ಧ್ವನಿ ಇಂಗಿ ಹೋಯಿತು. 

ಸ್ವರ ವರ್ಣ ಲೋಪವನ್ನೂ ಸಹಿಸದಂತೆ ಭಾಷೆಯನ್ನು ಕಾಪಾಡಿದ ಪರಂಪರೆಯನ್ನು ಉಳಿಸುವುದು ನಮ್ಮ ಆದ್ಯ ಕರ್ತವ್ಯವೆಂದು ಬಗೆಯುತ್ತೇನೆ.  ನಾನು ನನ್ನ ಕರ್ತವ್ಯವನ್ನು ಮಾಡಿದ್ದೇನೆ.  ತತ್ ಕ್ಷಣದಲ್ಲಿ ನನ್ನ ಪ್ರತಿಕ್ರಿಯೆ ಕಠೋರವೆನಿಸಿ ಇತರರ ಅಸಮಾಧಾನಕ್ಕೆ ಕಾರಣವಾದರೂ ನನ್ನ ಕರ್ತವ್ಯ ನೆರವೇರಿಸಿದ ತೃಪ್ತಿಯ ದೃಷ್ಟಿಯಲ್ಲಿ ಈ ಅಸಮಾಧಾನ ನಗಣ್ಯ.

\section*{ಇಷ್ಟಾರ್ಥಸಿದ್ಧಿ }

1998ರಲ್ಲಿ ನಾನು ಅಧ್ಯಯನ ಮಾಡಿದ ಮಹಾಪಾಠಶಾಲೆಯಲ್ಲಿ ಅಧ್ಯಾಪಕ ವೃತ್ತಿ ಲಭಿಸಿತು.  ವಿದ್ಯಾರ್ಥಿ ವಲಯದಲ್ಲಿ ಈಗಾಗಲೇ ಪಾಠಕನಾಗಿ ಪರಿಚಿತನಾಗಿದ್ದರಿಂದ ವಿಭಿನ್ನ ಶಾಸ್ತ್ರದ ವಿದ್ಯಾರ್ಥಿಗಳು ತಮ್ಮಲ್ಲಿ ಉದ್ಭವಿಸಿದ ಸಂಶಯ ನಿವಾರಣೆಗೆ ನಿರ್ಭೀತರಾಗಿ ನನ್ನತ್ತ ಬರುತ್ತಿದ್ದರು.  ಹೀಗೆ ಅನಿವಾರ್ಯವಾಗಿ ನಾನು ಶಾಸ್ತ್ರಾಂತರ ಅವಲೋಕನ ಮಾಡುವಂತಾಯಿತು.  ಇದರಿಂದ ನನ್ನ ಜ್ಞಾನ ಪರಿಧಿ ವರ್ಧಿಸಿತು.  

ಮಹಾರಾಜ ಸಂಸ್ಕೃತ ಕಾಲೇಜಿನಲ್ಲಿ ಸಹೋದ್ಯೋಗಿಗಳು ನನ್ನ ಬಗ್ಗೆ ಅಪಾರ\break ಅಭಿಮಾನ ಹೊಂದಿದ್ದಾರೆ.  ನನ್ನ ಅನೇಕ ವರ್ತನೆಗಳು ಅವರಿಗೆ ಪಥ್ಯವೆನಿಸದಿರಬಹುದು.   ಆದರೆ ಕಾಲೇಜಿನ ಯಾವುದೇ ಕೆಲಸವನ್ನು ವಹಿಸಿಕೊಟ್ಟಾಗ ವ್ಯಕ್ತಿಗತ\break ಅಸಮಾಧಾನವನ್ನು ಲೆಕ್ಕಿಸದೇ ಯಶಸ್ವಿಯಾಗಿ ನಿರ್ವಹಿಸಿ ಸಹಕರಿಸಿದ್ದಾರೆ.  ಇದರಿಂದ ಇಲ್ಲಿ ಏರ್ಪಡಿ\-ಸಿದ ಸಾಂಸ್ಕೃತಿಕ ಹಾಗೂ ಶೈಕ್ಷಣಿಕ ಕಾರ್ಯಕ್ರಮಗಳು ಯಶಸ್ವಿಯಾಗಲು ಸಾಧ್ಯವಾಗಿದೆ.  ಶಾಲೆಯ ಪ್ರತಿವರ್ಷದ ಆರಂಭದ ದಿನಗಳಲ್ಲಿ ವಿದ್ಯಾರ್ಥಿಗಳ ಪ್ರವೇಶವಾಗುವವರೆಗೆ ಉಪನ್ಯಾಸಕರು ಶೈಕ್ಷಣಿಕ ಚಟುವಟಿಕೆಯಲ್ಲಿ ತೊಡಗಿಕೊಳ್ಳಲು ‘ಶಾಸ್ತ್ರ ಸಂವಾದ’ ಎಂಬ ಕಾರ್ಯಕ್ರಮವನ್ನು ಯೋಜಿಸಿದೆ.  ಈ ಸಂವಾದ ವರ್ಷದ ಆದಿ ಅಂತ್ಯದಲ್ಲಿ ನಡೆಯುತ್ತಿತ್ತು.  ವೇದ, ಶಾಸ್ತ್ರ, ಹಾಗೂ ಆಗಮ ಸಂಬಂಧಿಯಾದ ಒಂದು ವಿಷಯವನ್ನು ಒಬ್ಬ ಉಪನ್ಯಾಸಕರು ನಿರೂಪಿಸಬೇಕು.  ಯಾರು ಬೇಕಾದರೂ ಪ್ರಶ್ನಿಸಲು ಮುಕ್ತ ಅವಕಾಶವಿತ್ತು.  ಉತ್ತರವನ್ನು ಯಾರಾದರೂ ನೀಡಬಹುದಿತ್ತು.  ಈ ಶಾಸ್ತ್ರ\break ಸಂವಾದದಿಂದ ವಿದ್ಯಾರ್ಥಿ\-ಗಳಿಗೂ ಉಪನ್ಯಾಸಕರಿಗೂ ವಿಭಿನ್ನ ವಿಷಯಗಳ ಬಗ್ಗೆ ಸಾಧಾರ ತಾರ್ಕಿಕ ವಿವರ ದೊರೆಯುತ್ತಿತ್ತು.  ಇದು ಹೆಚ್ಚು ವರ್ಷ ನಡೆಯಬೇಕಾದ ಕಾರ್ಯಕ್ರಮವಾದರೂ ನಡೆದಿಲ್ಲವೆಂಬ ಖೇದ ಅನೇಕರಲ್ಲಿದೆ.  ಯಾವುದೇ ವಿದ್ಯಾ\break ಸಂಸ್ಥೆಯ ಉನ್ನತಿಗೆ ಇಂಥ ಶಾಸ್ತ್ರ ಸಂವಾದ ಅತ್ಯಗತ್ಯ.  

ಉಪನ್ಯಾಸಕರು ಭಾಷಣಕಲೆಯನ್ನು ಬೆಳೆಸಿಕೊಳ್ಳಲು ಅಧ್ಯಾಪಕ ಸಂಘದ \hbox{ವತಿಯಿಂದ} ಪ್ರದೋಷೋಪನ್ಯಾಸಮಾಲೆಯನ್ನು ಯೋಜಿಸಿದೆ. ಒಂದು ವರ್ಷ ಈ ಕಾರ್ಯಕ್ರಮ ನಡೆಯಿತು.  ಸಂಸ್ಕೃತ ಕಾಲೇಜಿನ ಶಿಕ್ಷಕರ ವಾಕ್ ಶಕ್ತಿಯನ್ನು ತೀಕ್ಷ್ಣಗೊಳಿಸುವಲ್ಲಿ ಈ ಕಾರ್ಯಕ್ರಮ ನೆರವಾಗುತ್ತದೆ.  

ಸಂಸ್ಕೃತ ಕಾಲೇಜಿಗೂ ಸಾರ್ವಜನಿಕರಿಗೂ ಸಂಪರ್ಕ ಸಾಧಿಸುವ ದೃಷ್ಟಿಯಿಂದ ಸಾರ್ವಜನಿಕ ಹಿತಾಸಕ್ತಿಯ ಕೆಲವು ಕಾರ್ಯಕ್ರಮವನ್ನು ಯೋಜಿಸಿದ್ದೆ.  ಹತ್ತು ದಿನಗಳ ಅವಧಿಯಲ್ಲಿ ಶಿವಸಹಸ್ರನಾಮ, ವಿಷ್ಣುಸಹಸ್ರನಾಮ, ಲಲಿತಾಸಹಸ್ರನಾಮ ಮುಂತಾದ ಸಹಸ್ರನಾಮಗಳ ಹಾಗೂ ಸ್ತೋತ್ರಸಾಹಿತ್ಯದ ಪಾಠವನ್ನು ಮಾಡಲಾಗಿತ್ತು.  ಅರ್ಥಸಹಿತವಾಗಿ ಪರಿಶುದ್ಧ ಉಚ್ಚಾರಣೆಯ ಶಿಕ್ಷಣ ನೀಡಲಾಗಿತ್ತು.  ಈ ಕಾರ್ಯಕ್ರಮ ಸಾರ್ವಜನಿಕರ ಪ್ರಶಂಸೆಗೆ ಪಾತ್ರವಾಯಿತು.  ಇದನ್ನು ಮುಂದುವರಿಸುವುದು ಸಂಸ್ಕೃತ ಕಾಲೇಜು ಹಾಗೂ ಸಾರ್ವಜನಿಕರ ಹಿತದೃಷ್ಟಿಯಿಂದ ಅಗತ್ಯವೆಂದು ಭಾವಿಸುತ್ತೇನೆ.  

ಯಾವುದೇ ಶಾಲೆಯ ಘನತೆ ಗೌರವ ಆ ಶಾಲೆಯ ವಿದ್ಯಾರ್ಥಿಗಳ ಸಾಮರ್ಥ್ಯ ಹಾಗೂ ಪ್ರತಿಭೆಯನ್ನು ಅವಲಂಬಿಸುತ್ತದೆ.   ಹಳೆಯ ವಿದ್ಯಾರ್ಥಿಗಳ ಜೊತೆಗೆ ಹೊಸ ಹೊಸ ವಿದ್ಯಾರ್ಥಿಗಳು ಪ್ರತಿವರ್ಷ ಸೇರುವುದರಿಂದ ಶಾಲೆಯ ಗಾತ್ರ ಹಾಗೂ ಪಾತ್ರ ಬದಲಾಗುತ್ತದೆ.  ವಿದ್ಯಾರ್ಥಿಗಳ ಪ್ರತಿಭೆಯನ್ನು ಪತ್ತೆ ಹಚ್ಚಿ ಅದನ್ನು ಬೆಳೆಸಲು ಪ್ರೋತ್ಸಾಹ ನೀಡುವುದು ಶಿಕ್ಷಕರ ಗುರುಕೃತ್ಯ.  ಈ ದೃಷ್ಟಿಯಿಂದ ‘ಪ್ರತಿಭಾಪ್ರಕಾಶ’ವೆಂಬ ಕಾರ್ಯಕ್ರಮವನ್ನು ಅಯೋಜಿಸಿದೆ.  ವಿದ್ಯಾರ್ಥಿಗಳ ಪ್ರವೇಶಾವಧಿ ಮುಗಿದ ಅನಂತರ ಈ ಕಾರ್ಯಕ್ರಮ\-ವನ್ನು ನಡೆಸಲಾಗಿತ್ತು.  ವಿದ್ಯಾರ್ಥಿಗಳು ಮುಕ್ತವಾಗಿ ತಮ್ಮ ಪ್ರತಿಭೆಯನ್ನು ಪ್ರದರ್ಶಿಸಬಹುದಿತ್ತು.  ಇದೊಂದು ಚೇತೋಹಾರಿ ಕಾರ್ಯಕ್ರಮ.  ವಿದ್ಯಾರ್ಥಿಗಳು ಹಾಗೂ ಉಪನ್ಯಾಸಕರು ಆಸಕ್ತಿಯಿಂದ ಪಾಲ್ಗೊಳ್ಳುವ ಕಾರ್ಯಕ್ರಮ.  ಪ್ರತಿವರ್ಷ ಕಾಲೇಜಿನ ಆಂತರಿಕ ಸ್ವರೂಪವನ್ನು ನಿರ್ಣಯಿಸುವ ಕಾರ್ಯಕ್ರಮವಾದ್ದರಿಂದ ಸದಾ ಮುಂದುವರಿಸ\-ಬೇಕಾದ ಕಾರ್ಯಕ್ರಮ.

ಶಾಸ್ತ್ರ ತರಗತಿಗಳಿಗೆ ಪ್ರವೇಶ ಪಡೆದ ವಿದ್ಯಾರ್ಥಿಗಳನ್ನು ಶಾಸ್ತ್ರಗ್ರಂಥಗಳ ಅಧ್ಯಯ\-ನಕ್ಕೆ ಸಜ್ಜುಗೊಳಿಸುವ ದೃಷ್ಟಿಯಿಂದ ವರ್ಷದ ಆದಿಯಲ್ಲಿ ಸೇತುಬಂಧ ಎಂಬ ಯೋಜನೆ\-ಯನ್ನು ನಿರೂಪಿಸಿದೆ.  ಇದರಲ್ಲಿ, ಅಕ್ಷರಾಭ್ಯಾಸದಿಂದ ಪ್ರಾರಂಭಿಸಿ ಅನ್ವಯಾನು\-ಸಾರ ಶ್ಲೋಕಗಳನ್ನು ಅರ್ಥೈಸಿಕೊಳ್ಳುವ ಸಾಮರ್ಥ್ಯವನ್ನು ಪಡೆಯುವಂತೆ ಕಾರ್ಯಶಾಲೆಯನ್ನು ನಡೆಸಿದೆವು.  ಅಕ್ಷರ, ಪದ, ಸಂಧಿ, ಸಮಾಸ, ಕಾರಕ, ಕೃದಂತ, ತದ್ಧಿತ, ತಿಙಂತ, ಅಲಂಕಾರ, ವೃತ್ತ, ಅನ್ವಯಗಳ ಬಗ್ಗೆ ಕ್ರಮವಾಗಿ ಮೂಲಭೂತ ಮಾಹಿತಿ ನೀಡಿ ವಿದ್ಯಾರ್ಥಿ\-ಗಳನ್ನು ತಯಾರಿಮಾಡುತ್ತಿದ್ದೆವು. ಇದರಿಂದ ವಿದ್ಯಾರ್ಥಿಗಳಿಗೆ ಶಾಸ್ತ್ರ ಗ್ರಂಥವನ್ನು ಅರ್ಥಮಾಡಿಕೊಳ್ಳುವ ಶಕ್ತಿ ಪ್ರಾಪ್ತವಾಗುತ್ತಿತ್ತು.  ಈ ಕಾರ್ಯಕ್ರಮ ಯಾವುದೇ ಸಂಸ್ಕೃತ ಪಾಠಶಾಲೆಯು ಅನುಸರಿಸಲು ಅರ್ಹವಾಗಿದೆ.  

ನಮ್ಮ ಕಾಲೇಜಿನಲ್ಲಿ ಯೋಗಶಾಸ್ತ್ರದ ಅಧ್ಯಾಪಕರಾದ ಶ್ರೀಮಾನ್ ಮುಚುಕುಂಟೆ ಕೃಷ್ಣಮಾಚಾರ್ಯರ ಸ್ಮಾರಕವಾಗಿ ಒಂದು ವೇದಾಂತ ವಾಕ್ಯಾರ್ಥ ಸ್ಪರ್ಧೆಯನ್ನು ರಾಜ್ಯಮಟ್ಟದಲ್ಲಿ ನಡೆಸುವ ಹೊಣೆಗಾರಿಕೆಯನ್ನು ನನಗೆ ನೀಡಿದ್ದರು.  ಶ್ರೀ ಮುಚುಕುಂಟೆ\break ಕೃಷ್ಣಮಾಚಾರ್ಯರ ಪುತ್ರರಾದ ಶ್ರೀ ಟಿ.ಕೆ. ಭಾಷ್ಯಂ ರವರು ವಯಕ್ತಿಕ \hbox{ಆಸಕ್ತಿಯಿಂದ} ಈ ಸ್ಪರ್ಧೆಯನ್ನು ನಡೆಸಲು ಪ್ರಚೋದನೆ ನೀಡುತ್ತಿದ್ದರು.  ವರ್ಷದ ಆದಿಯಲ್ಲಿ \hbox{ವೇದಾಂತದ} ಯಾವುದಾದರೂ ಮೂರು ವಿಷಯಗಳನ್ನು ಸಿದ್ಧಪಡಿಸಿಕೊಳ್ಳಲು ಸೂಚಿಸಲಾಗುತ್ತಿತ್ತು.   ಸ್ಪರ್ಧೆಯ ಸಮಯಕ್ಕಿಂತ ಒಂದು ಗಂಟೆಯ ಮೊದಲು ಈ ಮೂರು\break ವಿಷಯಗಳಲ್ಲಿ ಯಾವುದಾದರೂ ಒಂದನ್ನು ಆಯ್ಕೆ ಮಾಡಿ ಭಾಷಣ ಮಾಡಲು ಸೂಚಿಸಲಾಗುತ್ತಿತ್ತು.  ವಿದ್ಯಾರ್ಥಿಗಳು ಆಕರ್ಷಕ ಬಹುಮಾನವಿರುವುದರಿಂದ \hbox{ಆಸಕ್ತಿಯಿಂದಲೇ} ಭಾಗವಹಿಸುತ್ತಿದ್ದರು.  ಇಷ್ಟುಮಾತ್ರವಲ್ಲದೇ ಈ ಯೋಜನೆಯಿಂದ ಪ್ರತಿಯೊಬ್ಬ ಸ್ಪರ್ಧಿ\break ಅನಿವಾರ್ಯವಾಗಿ ಮೂರು ವಿಷಯಗಳಲ್ಲಿ ತಜ್ಞನಾಗಿರುತ್ತಿದ್ದ.  

ಶ್ರೀ ಟಿ.ಕೆ. ಭಾಷ್ಯಂರವರ ಒತ್ತಾಸೆಯ ಮೇರೆಗೆ ಜರ್ಮನಿಯ ಶ್ರೀ ಜಾನ್ ಸ್ಮಿತ್ ಗ್ಯಾರೆರವರು ನಿರ್ಮಿಸಿದ ಲೈಫ್ ಸೇವಿಂಗ್ ಯೋಗಾ ಸೆಶನ್ ಎಂಬ ಸಾಕ್ಷ್ಯಚಿತ್ರದಲ್ಲಿ ಶ್ರೀ ಮುಚುಕುಂಟೆ ಕೃಷ್ಣಮಾಚಾರ್ಯರ ಪಾತ್ರವನ್ನು ನಿರ್ವಹಿಸಬೇಕಾಯಿತು.  ಹಾಗೆಯೇ ಅಮೆರಿಕಾದ ಎಂಡ್ರ್ಯೂ ಎಪ್ಲರ್‍ರವರು ನಿರ್ಮಿಸಿದ ಮೈಸೂರು ಯೋಗ ಟ್ರೆಡಿಶನ್  ಸಾಕ್ಷ್ಯಚಿತ್ರದಲ್ಲಿ ಪಾಲ್ಗೊಳ್ಳಬೇಕಾಯಿತು. 

ನಂಜನಗೂಡಿನ ಶ್ರೀ ತಮ್ಮಯ್ಯಾವಧಾನಿಗಳ ಸ್ಮೃತಿಯಲ್ಲಿ ವಿದ್ಯಾರ್ಥಿಗಳಿಗೆ 2005 ರಿಂದ 2015ರ ವರೆಗೆ ಪ್ರತಿಭಾ ಪುರಸ್ಕಾರ ನೀಡುವ ಕಾರ್ಯದ ಯೋಜನೆಯನ್ನು ರೂಪಿಸಲು ನನಗೆ ವಹಿಸಿದ್ದರು.  ಅವಧಾನಿಗಳ ಪುತ್ರರಾದ ಶ್ರೀ ವಿ.ಟಿ. ಅವಧಾನಿಯವರು ಈ ವಿಷಯದಲ್ಲಿ ಅತ್ಯಂತ ಆಸಕ್ತರಾಗಿದ್ದರು.  ವಿದ್ಯಾರ್ಥಿಗಳ ನೈಜ ಪ್ರತಿಭೆಯನ್ನು ಗುರುತಿಸಲು ಒಂದು ಆಶು ಪ್ರಬಂಧ ಸ್ಪರ್ಧೆಯನ್ನು ಏರ್ಪಡಿಸಿ ವಿಜೇತ ಅಭ್ಯರ್ಥಿ\-ಗಳಿಗೆ ಬಹುಮಾನ ನೀಡುತ್ತಿದ್ದೆವು.  ಇದರೊಂದಿಗೆ ಪ್ರತಿಭಾ ಪುರಸ್ಕಾರ ಸಮಾರಂಭದ ದಿನ ವೇದ ಹಾಗೂ ಶಾಸ್ತ್ರಗಳಲ್ಲಿ ವಿಶೇಷ ಕಾರ್ಯಕ್ರಮವನ್ನು ಏರ್ಪಡಿಸುತ್ತಿದ್ದೆವು.  ವಿದ್ಯಾರ್ಥಿ\-ಗಳು ಸಂಸ್ಕೃತ ಸಾಹಿತ್ಯದ ಪ್ರಕೃತೋಪಯೋಗಿಯಾದ ಒಂದು ಪಠ್ಯೇತರ ವಿಷಯವನ್ನು ಪ್ರಸ್ತುತ ಪಡಿಸುತ್ತಿದ್ದರು.  ಋಗ್, ಶುಕ್ಲ ಯಜುರ್ವೇದ, ಕೃಷ್ಣಯಜುರ್ವೇದ, ಹಾಗೂ ಸಾಮವೇದದಲ್ಲಿ ವಿಕೃತಿಗಳ ಜೊತೆಗೆ ಪ್ರಕೃತಿ ಪಾಠವನ್ನು ಮಾಡುತ್ತಿದ್ದರು.  ಸರಸ್ವತೀ ಪ್ರಾಸಾದದ ಭವ್ಯಸನ್ನಿವೇಶದಲ್ಲಿ ವೇದಘೋಷದ ಮೊಳಗುವಿಕೆಯೇ ಎಲ್ಲರನ್ನು ಆಕರ್ಷಿಸುತ್ತಿತ್ತು.  ಈ ಕಾರ್ಯಕ್ರಮವನ್ನು ಯೂಟ್ಯೂಬ್ ಮೂಲಕ ಪ್ರಸಾರಗೊಳಿಸುತ್ತಿದ್ದೆವು.  ಈ ಕಾರ್ಯಕ್ರಮದ ಪ್ರಭಾವ ಎಷ್ಟೆಂದರೆ ಇದನ್ನು ಕೇಳಿದ ಮುಂಬೈಯ ಮೌನೀಶ ಭಾರದ್ವಾಜ ಎಂಬ ವಿದ್ವಾಂಸರು ವೇದಘೋಷ ಮಾಡಿದ ವಿದ್ಯಾರ್ಥಿಗಳನ್ನು ಸಂದರ್ಶಿಸಲು ಮೈಸೂರಿಗೆ ಧಾವಿಸಿ ಬಂದಿದ್ದರು.  ಈ ಕಾರ್ಯಕ್ರಮದ ದಶಮಾನೋತ್ಸವದಲ್ಲಿ  ಮುದ್ರಾರಾಕ್ಷಸ ನಾಟಕದ ಒಂದು ಸನ್ನಿವೇಶವನ್ನು ಸಂಸ್ಕೃತದಲ್ಲಿ ಪ್ರದರ್ಶಿಸಲಾಗಿತ್ತು. 

ಈ ರೀತಿ ಕಾಲೇಜಿನ ಚಟುವಟಿಕೆಯೊಂದಿಗೆ ಶಾಸ್ತ್ರಗ್ರಂಥಗಳ ಪಾಠಮಾಡುವ ಅವಕಾಶ ಅವಿರತವಾಗಿ ದೊರಕುತ್ತಲೇ ಇದೆ.  ಗುಜರಾತಿನ ಸ್ವಾಮಿನಾರಾಯಣ ಪಂಥದ ಯುವಸಂನ್ಯಾಸಿಗಳು, ಚಾತುರ್ಮಾಸ್ಯ ವ್ರತ ನಿಮಿತ್ತ ಮೈಸೂರಿಗೆ ಆಗಮಿಸುವ ಜೈನಮುನಿಗಳು ಹಾಗೂ ಸಾಧ್ವಿಯರು, ಹರ್ಯಾಣದ ಗುರು\-ಕುಲದ ಶಿಕ್ಷಕರು, \hbox{ನೇಪಾಳದ} ಕೆಲವು ವಿದ್ಯಾರ್ಥಿಗಳು, ರಾಮಕೃಷ್ಣಾಶ್ರಮದ ಸ್ವಾಮಿಗಳು, ಚೆನ್ನೇನಹಳ್ಳಿ ಗುರು\-ಕುಲದ ಹಿರಿಯ ವಿದ್ಯಾರ್ಥಿಗಳು ತರ್ಕಾಧ್ಯಯನಕ್ಕಾಗಿ ಬರುತ್ತಿದ್ದಾರೆ.  ಇದಲ್ಲದೆ ಯೋಗ\-ನಗರಿ\-ಯಾದ ಮೈಸೂರಿನಲ್ಲಿ ಯೋಗಾಭ್ಯಾಸಕ್ಕಾಗಿ ಬರುವ ವಿದೇಶೀಯರು ವಿಶೇಷವಾಗಿ ಯೋಗಭಾಷ್ಯ, ಸಾಂಖ್ಯ, ಭಗವದ್ಗೀತೆ, ಉಪನಿಷತ್ತು ಮುಂತಾದವನ್ನು ಸಂಸ್ಕೃತ\-ಮಾಧ್ಯಮದಲ್ಲಿ ಓದುವ ಬಯಕೆಯಿಂದ ನನ್ನನ್ನು ಆಶ್ರಯಿಸುತ್ತಿದ್ದಾರೆ.  ಈಗಾಗಲೇ ಸಿ.ಡಿ ರೂಪದಲ್ಲಿ ದಿನಕರೀಯ ಗ್ರಂಥದ ಪಾಠವನ್ನು ದಾಖಲಿಸಲಾಗಿದೆ.  ಅದರಂತೆಯೇ ಶಾಬ್ದಬೋಧ ಕ್ರಮದಲ್ಲಿ ತರ್ಕಸಂಗ್ರಹದ ಪಾಠ ಸಿ.ಡಿ ರೂಪದಲ್ಲಿ ಸೆರೆಹಿಡಿಯ\-ಲಾಗಿದೆ.

\section*{ಕೃತಕೃತ್ಯತೆ}

ಎರಡು ದಶಕಗಳ ಸೇವಾವಧಿಯಲ್ಲಿ  ಸಂಸ್ಕೃತ ಕಾಲೇಜಿನ ವಿದ್ಯಾರ್ಥಿಗಳನ್ನು, \hbox{ಶಿಕ್ಷಕರನ್ನು} ಸದಾ ಶೈಕ್ಷಣಿಕ ಚಟುವಟಿಕೆಯಲ್ಲಿ ತೊಡಗಿಸಿಕೊಳ್ಳುವಂತೆ ಕಾರ್ಯಕ್ರಮ ನಿರೂಪಿಸಿ ಯಶಸ್ವಿಗೊಳಿಸುವ ಜವಾಬ್ದಾರಿಯನ್ನು ನನಗೆ ನೀಡುತ್ತಿದ್ದರು. ಇದು ನನಗೆ ಅತ್ಯಂತ ಹೆಮ್ಮೆಯ ವಿಷಯ. ಹೀಗೆ ಎರಡು ದಶಕಗಳ ಸೇವೆ ಸಲ್ಲಿಸಿ ನಾನು ತೃಪ್ತಿಯಿಂದ\break ನಿವೃತ್ತನಾಗುತ್ತಿದ್ದೇನೆ. ಇಷ್ಟಾರ್ಥ ಸಿದ್ಧಿಸಿದೆ.  

ಮೈಸೂರಿನ ಶ್ರೀಮನ್ಮಹಾರಾಜಸಂಸ್ಕೃತ ಮಹಾಪಾಠಶಾಲೆಯು ಮೈಸೂರಿನ ಮಹಾರಾಜರ ಔದಾರ್ಯದಿಂದ ಮತ್ತು ಮುನ್ನೋಟದಿಂದ ಉದಯವಾಗಿದೆ. ಈ ಪಾಠಶಾಲೆ ನನ್ನಂತಹ ಅನೇಕ ಸಂಸ್ಕೃತ ವಿದ್ವಾಂಸರಿಗೆ ಆಸರೆಯನ್ನೊದಗಿಸಿದೆ. ಅದಕ್ಕೆ ಶ್ರೀಮನ್ಮಹಾ\-ರಾಜರು ಯಾವತ್ತೂ ಸಂಸ್ಕೃತಜ್ಞರ ಕೃತಜ್ಞತೆಗೆ ಪಾತ್ರರಾಗಿದ್ದಾರೆ. ನಾನು ಅವರಿಗೆ ಅತ್ಯಂತ ಕೃತಜ್ಞತೆಯನ್ನು ಭಾವಿಸುತ್ತೇನೆ. 

ನನ್ನನ್ನು ಬೆಳೆಸಿ, ಪೋಷಿಸಿದ ಮೈಸೂರಿನ ಶ್ರೀಮನ್ಮಹಾರಾಜ ಸಂಸ್ಕೃತ ಪಾಠ\-ಶಾಲೆಯು ಇತೋಪ್ಯತಿಶಯವಾಗಿ ಬೆಳೆದು ದಿಗಂತ ಪ್ರಸಿದ್ಧಿಯನ್ನು ಪಡೆಯಲಿ ಎಂದು ಬಯಸುತ್ತೇನೆ.  

~\hfill
\begin{tabular}{c}
ಸಜ್ಜನ ವಿಧೇಯ\\
\textbf{ವಿ ॥ ಗಂಗಾಧರ ವಿ.ಭಟ್}\\ 
ಮೈಸೂರು
\end{tabular}

\articleend
}
