\chapter{गङ्गाधरवन्दनम्}

\begin{center}
\Authorline{डा.चन्द्रकान्तः}
\smallskip

प्राध्यापकः\\
राजीवगान्धिपरिसरः,\\
शृङ्गगिरिः
\addrule
\end{center}

\begin{verse}
विघ्नेश्वरात्मजं वन्द्यं  रेवत्यानन्दवर्धनम् ।\\
महीसुरपुरीवासं वन्दे गङ्गाधरं गुरुम् ।। 1 ।।
\end{verse}

\begin{verse}
न्यायशास्त्रविशालाक्षं न्यायवादविशारदम् ।\\
आयुर्वेदादिशास्त्रज्ञं वन्दे गंगाधरं गुरुम् ।। 2 ।।
\end{verse}

\begin{verse}
सर्वदा सत्यवक्तारम् आङ्ग्लभाषाविचक्षणम् ।\\
हिन्दिभाषाप्रवीणं तं वन्दे गङ्गाधरं गुरुम् ।। 3 ।।
\end{verse}

\begin{verse}
विद्यासङ्क्रान्तिसंशीलं प्रतिवादिभयङ्करम् ।\\
शिष्योत्कर्षं चिकीर्षन्तं वन्दे गंगाधरं गुरुम् ।। 4 ।।
\end{verse}

\begin{verse}
लोकनीतेर्लब्धवर्णम् अर्थशास्त्रार्थभूषणम् ।\\
व्यर्थवादविभेत्तारं वन्दे गङ्गाधरं गुरुम् ।। 5।।
\end{verse}

\begin{verse}
लालित्यपूर्णैः पदसङ्कुलैश्च स्पष्टैश्च वाक्यैः परिभूषितैश्च ।\\
सर्वार्थसिद्धैः प्रतिबोधनैश्च तुष्टो हि चन्द्रो नतमस्तकोऽस्मि ।। 6 ।।
\end{verse}

\begin{verse}
इत्थं च चन्द्रकान्तेन नमस्कारो विधीयते ।\\
कुर्यात्सन्मङ्गलं नित्यं लोके गंगाधरो गुरुः।। 7 ।।
\end{verse}

\begin{verse}
गंगाधरामृतं नाम सप्तश्लोकभूषितम्  ।\\
पद्यं समर्पितं भक्त्या गुरूणां पादपद्मयोः ।। 8 ।।
\end{verse}

\articleend
