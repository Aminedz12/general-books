{\fontsize{16}{18}\selectfont
\presetvalues
%\addtocontents{toc}{\protect\newpage}
\chapter{गङ्गाधरवन्दनम्}


\centerline{\Authorline{डा~॥ चन्द्रकान्तः}}

\centerline{सहायकप्राध्यापकः}

\centerline{राजीवगान्धिपरिसरः, शृङ्गगिरिः}

\centerline{\addrule}

\begin{verse}
विघ्नेश्वरात्मजं वन्द्यं  रेवत्यानन्दवर्धनम् ।\\
महीसुरपुरीवासं वन्दे गङ्गाधरं गुरुम्~॥ 1~॥\\
न्यायशास्त्रविशालाक्षं न्यायवादविशारदम् ।\\
आयुर्वेदादिशास्त्रज्ञं वन्दे गङ्गाधरं गुरुम्~॥ 2~॥\\
सर्वदा सत्यवक्तारम् आङ्ग्लभाषाविचक्षणम् ।\\
हिन्दिभाषाप्रवीणं तं वन्दे गङ्गाधरं गुरुम्~॥ 3~॥\\
विद्यासङ्क्रान्तिसंशीलं प्रतिवादिभयङ्करम् ।\\
शिष्योत्कर्षं चिकीर्षन्तं वन्दे गङ्गाधरं गुरुम्~॥ 4~॥\\
लोकनीतेर्लब्धवर्णम् अर्थशास्त्रार्थभूषणम् ।\\
व्यर्थवादविभेत्तारं वन्दे गङ्गाधरं गुरुम्~॥ 5~॥\\
लालित्यपूर्णैः पदसङ्कुलैश्च स्पष्टैश्च वाक्यैः परिभूषितैश्च ।\\
सर्वार्थसिद्धैः प्रतिबोधनैश्च तुष्टो हि चन्द्रो नतमस्तकोऽस्मि~॥ 6~॥\\
इत्थं च चन्द्रकान्तेन नमस्कारो विधीयते ।\\
कुर्यात्सन्मङ्गलं नित्यं लोके गङ्गागाधरो गुरुः~॥ 7~॥\\
गङ्गाधरामृतं नाम सप्तश्लोकविभूषितम्  ।\\
पद्यं समर्पितं भक्त्या गुरूणां पादपद्मयोः~॥ 8~॥
\end{verse}

\centerline{{\fontsize{10}{12}\selectfont\ding{97}\quad\ding{97}\quad\ding{97}}}
}
