{\fontsize{14}{16}\selectfont
\chapter{ನೆನಪು ಹಳತಾಗಿಲ್ಲ}

\begin{center}
\Authorline{ಶ್ರೀವೆಂಕಟರಮಣ ಹೆಗಡೆ, ಕಲಗಾರು}
\smallskip

ನಿವೃತ್ತ ಅಧ್ಯಾಪಕ\\
ನಂಜನೂಗುಡು ಸರ್ಕಾರಿ ವಿದ್ಯಾಲಯ
\addrule
\end{center}

ನಮಗೆ ನೆರವು ನೀಡಿರುವವರನ್ನು, ಸಹೃದಯರನ್ನು ನೆನಪು ಮಾಡಿಕೊಳ್ಳುವುದು ಅಥವಾ ಮಾಡಿಕೊಳ್ಳುತ್ತಲೇ ಇರುವುದು ನಮ್ಮ ಧರ್ಮ, ನಮ್ಮ ಸಂಸ್ಕೃತಿ. ಈ ಹಿನ್ನೆಲೆಯಲ್ಲಿ ಒಂದೆರಡು ಮಾತು.

ಸುಮಾರು 1972   \enginline{-}   73 ರ ಸಮಯ. ಮೈಸೂರಿನ ಸಂಸ್ಕೃತ ಪಾಠಶಾಲೆಯಲ್ಲಿ ಇದ್ದ ಸಂದರ್ಭ. ಅಲ್ಲಿಯೇ ಅಧ್ಯಯನ ಮಾಡುತ್ತಿದ್ದ ಅಗ್ಗೇರೆ ಗಂಗಾಧರ ಭಟ್ಟರ ಅಣ್ಣ ಶ್ರೀ ಮಂಜುನಾಥ ಭಟ್ಟರ ಪರಿಚಯ ಆಯಿತು. ಅದು ಪರಿಚಯ ಮಾತ್ರವಾಗಿರದೇ ಕ್ರಮವಾಗಿ ಸ್ನೇಹವಾಗಿ ಪರಿವರ್ತಿತವಾಯಿತು. ಹಾಗಂತ ನಾನು ಯಾವ ವಿಷಯದಲ್ಲೂ ಅವರಿಗೆ ಸರಿಸಮನಾಗಿರಲಿಲ್ಲ. ಆದರೂ ಆ ಅವರ ಸ್ನೇಹ ಗಾಢವಾಗಿತ್ತು. ಕಾಲಕ್ರಮದಲ್ಲಿ ಅವರು ವಿದ್ಯಾಭ್ಯಾಸ ಮುಗಿಸಿ ಊರಿಗೆ ತೆರಳಿದರು. ನಾನು ಊರಿಗೆ ಹೋದಾಗ ಅವರ ಮನೆಗೂ ಹೋಗಿ ತಂಗುತ್ತಿದ್ದುದುಂಟು. ಅಲ್ಲೆಲ್ಲ ಊರೂರು ಓಡಾಡುತ್ತಿದ್ದುದುಂಟು. ಇದು ಹಳತಾಗದ   \enginline{-}   ಮರೆಯಲಾಗದ ಅನುಭವ. ಹೀಗಾಗಿ ಶ್ರೀಮಂಜುನಾಥ ಭಟ್ಟರೊಡನೆ ಒಂದು ಆತ್ಮೀಯ ಒಡನಾಟ ಯಾವತ್ತೂ ಇತ್ತು. ಈ ಒಡನಾಟವೇ ಮುಂದೆ ಗಂಗಾಧಾರ ಭಟ್ಟರೊಂದಿಗೆ ಮುಂದುವರಿಯಿತು. ಅವರು ಮಂಜುನಾಥ ಭಟ್ಟರ ಅನಂತರ ಮೈಸೂರಿಗೆ ಬಂದವರು.

ಪಾಠಶಾಲೆಯಲ್ಲಿ ಅವರ ಅಧ್ಯಯನ ಸಾಗಿತ್ತು. ಜೊತೆಜೊತೆಗೆ ಲೌಕಿಕ ಅಧ್ಯಯನವೂ ಸಹ. ಇವರು ಅಪಾರ ಪ್ರತಿಭಾವಂತರು. ಅಧ್ಯಯನ ಮಾತ್ರವಲ್ಲದೇ ಇವರಿಗೆ ಸಿದ್ಧಿಸಿದ್ದು ಭಾಷಣ ಕಲೆ. ಅದರಿಂದ ಆ ಕಾಲದಲ್ಲಿ ಅವರು ಅಗ್ರಶ್ರೇಣಿಯ ಬಹುಮಾನವನ್ನು ಬಹುವಾಗಿ ಪಡೆಯುತ್ತಿದ್ದರು. 

ಹಿರಿಯರಿಂದ ಬಂದ ಸಂಸ್ಕಾರ ಸತತ ಅವಕಾಶದಿಂದ ಉತ್ತಮ ವಾಗ್ಮಿಗಳಾದರು. ಸಂಸ್ಕೃತ ಮಾತ್ರವಲ್ಲದೇ ಇಂಗ್ಲಿಷ್, ಹಿಂದಿ ಭಾಷೆಯಲ್ಲಿಯೂ ಹಿಡಿತ ಸಾಧಿಸಿದರು. ಯಾವುದೇ ವಿಷಯವಾದರು ತಳಸ್ಪರ್ಶಿಯಾಗಿ ನೋಡುವ, ಮಾತನಾಡುವ ಪ್ರತಿಭೆ ಅವರಲ್ಲಿದೆ. ಹಾಗಾಗಿ ಅವರ ತರ್ಕಬದ್ಧ ಭಾಷೆಯ ಪ್ರಯೋಗ, ವಿಷಯ ವಿಮರ್ಶನ ಶೈಲಿ ವಿದ್ಯಾರ್ಥಿಗಳನ್ನಲ್ಲದೇ ಸಾರ್ವಜನಿಕರನ್ನೂ ಸಾಕಷ್ಟು ಆಕರ್ಷಿಸಿದೆ. 

ಅವರ ಇನ್ನೊಂದು ಬಹಳ ದೊಡ್ಡ ಗುಣ ಪರೋಪಕಾರ. ಅವರು ನನಗುಂಟಾದ ಸಂದಿಗ್ಧ ಸಮಯದಲ್ಲಿ  ಖುದ್ದು ಸಹಾಯಮಾಡಿದ್ದರಿಂದ ನನಗೆ ಸರ್ಕಾರಿ ಉದ್ಯೋಗ ಸಿಗಲು ಸಹಕಾರವಾಯಿತು. ಈ ದೃಷ್ಟಿಯಿಂದ ಅವರು ಜೀವನಕ್ಕೊಂದು ಆಸರೆಯೇ ಆದರು ಎಂದರೆ ತಪ್ಪಿಲ್ಲ. ಹೀಗಿರುವಾಗ ಇದು ಮನೊ  \enginline{-}  ಬುದ್ಧಿಭೂಮಿಕೆಗಳಲ್ಲಿ ಹಳತಾಗುವುದಾದರೂ ಎಂತು !

ಅಂದು ರಿಕ್ತಹಸ್ತನಾಗಿದ್ದ ನನಗೆ ಸಹಾಯ ಹಸ್ತ ನೀಡಿದ್ದಕ್ಕೆ ಅಗ್ಗೇರೆ ಕುಟುಂಬದವರಿಗೆ ನಾನು ಋಣಿಯಾಗಿದ್ದೇನೆ. ಈಗ ಕಾಲ ಬಹಳ ಸಂದಿದೆ.  ಅದೆಷ್ಟೇ ಸಂದರೂ ನೆನಪು ಹಳತಾಗಿಲ್ಲ, ಹಸಿರಾಗಿಯೇ ಇದೆ.

\articleend
}
