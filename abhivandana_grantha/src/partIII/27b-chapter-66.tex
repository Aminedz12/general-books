{\fontsize{14}{16}\selectfont
\chapter{ಗುರುಪ್ರಸಾದ}

\begin{center}
\Authorline{ವಿ || ಗುರುಪ್ರಸಾದ}
\smallskip

ಅಧ್ಯಾಪಕ, ಸಂಸ್ಕೃತ ವಿಭಾಗ\\
ಎಮ್.ಐ.ಟಿ. ಫಸ್ಟ್ ಗ್ರೇಡ್ ಕಾಲೇಜ್, ಮೈಸೂರು
\addrule
\end{center}

ನಮ್ಮ ಹುಟ್ಟೂರಲ್ಲಿ ಶ್ರೀಯುತರಾದ ಗಂಗಾಧರ ಭಟ್ಟರ ಮನೆಗೂ ನಮ್ಮ ಮನೆಗೂ ಐದಾರು ಕಿಲೋಮೀಟರ ದೂರವಿರಬಹುದು. ನಮ್ಮಿಬ್ಬರ ಕುಟುಂಬದ ನಂಟು ಬಹಳ ಹಳೆಯದು. ಬಹುಶಃ ಆ  ವಿಷಯವನ್ನು ಹೇಳಲು ನನಗಿರುವ ಮಾಹಿತಿಯ\break ಆಧಾರದ ಮೇಲೆ ಹೇಳುವುದಾದರೆ ೧೯೪೦ ರ ದಶಕಕ್ಕೆ ಹೋಗಬೇಕಾಗುತ್ತದೆ. ಅದು ನನ್ನ ತಂದೆಯ ಮತ್ತು ಗಂಗಾಧರ ಭಟ್ಟರ ದೊಡ್ಡಪ್ಪ \enginline{-} ಮಹಾ\-ಬಲೇಶ್ವರ ಭಟ್ಟರು, ಅವರ ಕಾಲಕ್ಕೆ ಸಂಬಂಧಿಸಿದ ವಿಷಯ. ಅವರ ದೊಡ್ಡಪ್ಪನಿಗೆ ನನ್ನ ತಂದೆಯಲ್ಲಿ ಬಹಳ ವಿಶ್ವಾಸ. ಅದೂ ಸಹ ಪೂರ್ವಜನ್ಮದ ಸಂಬಂಧವಲ್ಲದಿದ್ದರೆ ಈ ಬಂಧ ಸಾಧ್ಯವೇ ಇಲ್ಲ. ಮಹಾ\-ಬಲೇಶ್ವರ ಭಟ್ಟರು ಆಗಿನ ಕಾಲದಲ್ಲಿ ಬಹಳ ಪ್ರಕಾಂಡ ಪಂಡಿತರು. ನನ್ನ ತಂದೆಗೆ ಹೃದಯ, ಬಾಯಿ, ಕೈ ಈ ಮೂರರ ಶುದ್ಧಿಯ ಹೊರತು ಇನ್ನಾವ ಅರ್ಹತೆಯೂ ಇರಲಿಲ್ಲ. ಅವರೆದುರು ಇವರು ನಗಣ್ಯ. ಆದರೂ ನನ್ನ ತಂದೆ ಅವರ ಮನೆಯಲ್ಲೇ ಅವರ ಔದಾರ್ಯದ ಕಾರಣದಿಂದ ಉಳಿದಿದ್ದರು. ಇವೆಲ್ಲ  ಇತಿಹಾಸ, ಅದರ ವಿವರಣೆ ನನ್ನ\break ಎಣಿಕೆಗೆ ನಿಲುಕದ್ದು \enginline{-} ಪ್ರಕೃತ ಅದು ಅಪ್ರಕೃತ ಕೂಡ. ಅವರ ಮನೆ ಮತ್ತು ನಮ್ಮ ಮನೆಯ ನಂಟು ಅಷ್ಟು ಪುರಾತನವಾದುದು ಎಂಬುದಷ್ಟೇ ಆಶಯ. ಆದರೆ ಆ\break ನಂಟಿನ ಅಂಟು ಇಂದೂ ಉಂಟು. ಅದು ಮಹಾಬಲೇಶ್ವರ ಭಟ್ಟರು ಕೊಟ್ಟ ಆಶ್ರಯ\-ದಿಂದ ಆರಂಭವಾಗಿ ವಿಘ್ನೇಶ್ವರ ಭಟ್ಟರಿಂದ ಮುಂದುವರೆದು ಗಂಗಾಧರ ಭಟ್ಟರ\break ತನಕವೂ ಬಂದಿದೆ. ಹಾಗಾಗಿ ಆ ಅಂಟು\enginline{-}ನಂಟಿನ ಪ್ರಯೋಜನ ನಮ್ಮ ಕುಟುಂಬಕ್ಕೆ\break ವಿಪುಲವಾಗಿ ಆಗಿದೆ. ಅವರ ಕುಟುಂಬದ ಋಣಭಾರ ನಾವು ಹೊರಲಾರದಷ್ಟು ನಮ್ಮ ಕುಟುಂಬದ ಮೇಲಿದೆ.

ನಮ್ಮ ಬಾಲ್ಯದ ಘಟನೆಗಳ ನೆನಪು ಸಾಮಾನ್ಯವಾಗಿ ಯಾವ ವಯಸ್ಸಿನಿಂದ ಸ್ಮೃತಿಯಲ್ಲಿ ಉಳಿಯಬಹುದೋ ಅಷ್ಟು ಚಿಕ್ಕ ವಯಸ್ಸಿನಿಂದ ನಾನು ಗಂಗಾಧರ ಭಟ್ಟರನ್ನು ನೋಡುತ್ತ ಬಂದಿದ್ದೇನೆ. ಅವರ ಮನೆಯಲ್ಲೇ ಸಾಕಷ್ಟು ಸಮಯ ವಾಸ ಮಾಡಿದ್ದೇನೆ. ಅವರ ಮನೆ ನನಗೆ ಇನ್ನೊಂದು ಮನೇಯೇ ಆಗಿತ್ತು ಎಂದರೆ ತಪ್ಪಿಲ್ಲ. ಬಹುಶಃ ಆ ಸಹವಾಸ ಮೈಸೂರಿನವರೆಗೂ ನನ್ನನ್ನು ಕರೆತಂದಿತು.

ಗಂಗಾಧರ ಭಟ್ಟರಲ್ಲಿ ನನಗೆ ಅತಿಯಾದ ಪ್ರೀತಿ, ಅತಿಯಾದ ಸಲುಗೆ. ಅವರ ತಂದೆ ಅಣ್ಣಂದಿರಲ್ಲೂ ಅದೇ ಸಲುಗೆಯೇ ನನಗುಂಟು. ಆದರೆ ಅವರನ್ನೆಲ್ಲ ಬಹುವಚನದಲ್ಲಿ ವ್ಯವಹರಿಸುವ ನಾನು ಇವರನ್ನು ಯಾವತ್ತೂ ಬಹುವಚನದಿಂದ ಕರೆದಿದ್ದಿಲ್ಲ. ಅದಕ್ಕೆ ನಮ್ಮಿಬ್ಬರ ಸ್ವಭಾವವೂ ಕಾರಣವಿರಬಹುದು. ಬಹುವಚನ ಪ್ರಯೋಗ ಅದೇಕೋ ನನಗೆ ಈರ್ವರ ಪರಸ್ಪರ ಆತ್ಮೀಯಭಾವದಲ್ಲಿ ಅಂತರವನ್ನು ಧ್ವನಿಸುತ್ತದೆ. ಹಾಗೆಂದು ಈ ರೀತಿಯ ಚಿಂತನೆಗಳೆಲ್ಲ ಮೊಳೆಯುವ ಮೊದಲೇ \enginline{-} ಅಷ್ಟು ಚಿಕ್ಕಂದಿನಿಂದ ಆ ವ್ಯಾಪಾರ ನಡೆಯುತ್ತ ಬಂದಿದೆ, ಅಂದರೆ ಅದಕ್ಕೆ ಇನ್ನೂ ಆಳವಾದ ಕೊಂಡಿಯೇನಾದರು ಇರಲೂ ಸಾಕು.

ಶಾಲೆಯ ರಜೆಯ ಅವಧಿಯಲ್ಲಿ ಹುಡುಗರು ಅಜ್ಜನ ಮನೆಗೆ ಹೋಗುವುದುಂಟು. ಆದರೆ ನನಗೆ ಅದು ಅಗತ್ಯವಿರಲಿಲ್ಲ. ಕಾರಣ ನನ್ನಜ್ಜನ ಮನೆ ನಮ್ಮ ಮನೆಯಿಂದ,\break ಎಡವಿ ಬಿದ್ದರೆ ಮೂರೇ ಮಾರು. ಅದಕ್ಕೆ ರಜೆ ಬಂತೆಂದರೆ ಮಣ್ಣೀಕೊಪ್ಪಕ್ಕೆ ಹೋಗುವುದು ರೂಢಿ. ನನ್ನಂತೆಯೇ ಇನ್ನೂ ಅನೇಕ ಹುಡುಗರು ಅಲ್ಲಿ ಸೇರುತ್ತಿದ್ದರು. ಅವರಲ್ಲಿ ಈ\break ಪುಸ್ತಕಕ್ಕೆ ಧನಸಹಾಯ ಮಾಡಿದ ಹೊನ್ನೇಹದ್ದ ಸೂರಿ\enginline{-}ಸೂರ್ಯನಾರಾಯಣನೂ ಒಬ್ಬ. ಅಲ್ಲಿ ರಜೆಯನ್ನು ಆಡುತ್ತ ಕಲಿಯುತ್ತ ಕಳೆಯುತ್ತಿದ್ದೆವು. ಅಲ್ಲಿ ಗಂಗಣ್ಣನ ಅಮ್ಮ \enginline{-} \hbox{ರೇವತಮ್ಮ} ಉಬ್ಬಿದ ರೊಟ್ಟಿ ಮಾಡುವುದರಲ್ಲಿ ಎತ್ತಿದ ಕೈ. ಮಕ್ಕಳೆಂದರೆ ಅವರಿಗೆ ಅಷ್ಟೇ ಅಕ್ಕರೆ. ಹಾಗಾಗಿ ರುಚಿಯಾದ ಅಡುಗೆಯಾಗುತ್ತಿತ್ತು, ಸಮೃದ್ಧವಾಗಿ ತಿಂದುಂಡು ಬಾಲ್ಯದ ಎಲ್ಲ ರೀತಿಯ ಚೇಷ್ಟೆಗಳಿಗೂ ಅವರ ಮನೆ ಆಶ್ರಯವಾಗಿತ್ತು. ಅವರ ಮನೆಯಲ್ಲಿ ಆರ್ಥಿಕವಾಗಿ ಅನುಕೂಲತೆಯೇನೂ ಇರಲಿಲ್ಲ. ಅವರ ಔದಾರ್ಯದ ಕಾರಣ ಮಕ್ಕಳಿಗೆ ಅದರ ಅರಿವೂ ಅಗುತ್ತಿರಲಿಲ್ಲ. ಅವರ ಮನೆಯಲ್ಲಿನ ಹಸುಗಳು ನಮಗಿಂತಲೂ ತುಂಟವಾಗಿದ್ದವು. ಅವುಗಳನ್ನು ಭಾರತ ಪಾಕಿಸ್ತಾನದ ಗಡಿಯನ್ನು ಸೈನಿಕರು ಕಾಯುವಂತೆ ಅಷ್ಟದಿಕ್ಕುಗಳಲ್ಲಿಯೂ ನಿಂತು ಕಾಯಬೇಕಿತ್ತು. ನಾವೆಲ್ಲ ಹುಡುಗರು ವಯಸ್ಸಿನಲ್ಲಿ ನಮಗಿಂತ ಹಿರಿಯರಾಗಿದ್ದ ಗಂಗಣ್ಣನ ತಂಗಿಯರು \enginline{-} ಲೀಲಕ್ಕ, ಹೇಮಕ್ಕ ಮತ್ತು ರತ್ನಕ್ಕ \enginline{-} ಇವರ ಸುಪರ್ದಿಯಲ್ಲಿ ಗೋಪಾಲ ಚೇಷ್ಟೆಯನ್ನು ಮಾಡುತ್ತಿದ್ದೆವು. ಅವರು ನಾಲ್ಕು ಜನ ಅಕ್ಕ ತಂಗಿಯರು \enginline{-} ವೇದಕ್ಕ, ಲೀಲಕ್ಕ, ಹೇಮಕ್ಕ, ರತ್ನಕ್ಕ ನನ್ನನ್ನು ಅವರ ಮನೆಯ\break ಮಗನೆಂಬಂತೆಯೇ ನೋಡುತ್ತಾರೆ. ಬಹಳ ಕಾಲದಿಂದ ವಾಸ ದೂರವಾದರೂ ಆ ವಾಸದ ನಂಟು ಇಂದಿನವರೆಗೂ ಕಿಂಚಿತ್ತೂ ವ್ಯತ್ಯಾಸವಾಗದೇ ಹಾಗೆಯೇ ಇದೆ, ಆಗ ಈಗ ಭೇಟಿಯಾದಾಗ ರೀಪ್ರೆಶ್ ಅಥವಾ ರೀಚಾರ್ಜ್ ಆಗುತ್ತಿರುತ್ತದೆ. 

ಈ ಮಧ್ಯದಲ್ಲಿ ಶ್ರೀ ಮಂಜುನಾಥ ಭಟ್ಟರು ನಮಗೆ ವೇದದ ಪಾಠಗಳನ್ನು ಮಾಡುತ್ತಿದ್ದರು. ಅಲ್ಲದೆ ಪೌರೋಹಿತ್ಯಕ್ಕೆ ಹೋಗುವಾಗ ನಮ್ಮಲ್ಲಿ ವಾಡಿಕೆಯಿರುವ ಪಡ\-ಚಾಕರಿಗೆ,  ಸಹಸ್ರನಾಮಾದಿಗಳ ಪಾರಾಯಣಕ್ಕೆ ಕರೆದುಕೊಂಡು ಹೋಗುತ್ತಿದ್ದರು. ಅದರಿಂದ ನಮಗೆ ಒಂದು ವೈದಿಕ ಸಂಸ್ಕಾರ ಉಂಟಾಗುತ್ತಿತ್ತು. 

ಇಂತಹ ಸಮಯದಲ್ಲಿ ಮೈಸೂರಿನಿಂದ ಗಂಗಣ್ಣ ಮನೆಗೆ ಬರುತ್ತಿದ್ದ. ಅವನಲ್ಲಿ ನಮಗೆ ಬಹಳ ಆಕರ್ಷಣೆ. ಅವನಿಗೆ ಶಾಸ್ತ್ರಕೃಷಿಯಲ್ಲಿರುವ ಆಸಕ್ತಿ ಕ್ಷೇತ್ರಕೃಷಿಯಲ್ಲೂ ಇತ್ತು. ನಾವು ಅವನ ಹಿಂದೆ ಮುಂದೆ ಓಡಾಡುತ್ತಿದ್ದೆವು. ಅವನು ಊರಿಗೆ ಬಂದಾಗ\break ನಿರ್ದಿಷ್ಟವಾದ ಮನೆಗಳಿಗೆ ಹೋಗುವ ರೂಢಿಯಿತ್ತು. ಅವನ ತಂಗಿ ವೇದಾ(ದ)ವತಿಯ ಮನೆಗೆ ಹೋಗಿ ಅಲ್ಲಿ ಒಂದು ರಾತ್ರಿ ಉಳಿದು ಮಾರನೇ ದಿನ ನಮ್ಮ ಮನೆಗೆ ಬಂದು, ಅಂದು ರಾತ್ರಿ ನಮ್ಮಲ್ಲಿ ತಂಗಿ ಅಲ್ಲಿಂದ ಮನೆಗೆ ಹೋಗುವುದು \enginline{-} ಇದು ಅವನ ವಾರ್ಷಿಕ ಹೈಸಾಲು.

ಮುಂದೆ ನಾನು ಎಸ್.ಎಸ್.ಎಲ್.ಸಿ ಮುಗಿಸಿ ಮನೆಯಲ್ಲೇ ಇದ್ದೆ. ನಿತ್ಯ ತೋಟದ ಕೃಷಿ, ಮತ್ತು ಹಸು ಮೇಯಿಸಿ ಹಾಲು ಕರೆದು ಊರಿಗೆಲ್ಲ ಕೊಡುವ ಕೆಲಸ. ಅದನ್ನು ನಾನು ಚೆನ್ನಾಗಿಯೇ ನಿರ್ವಹಿಸುತ್ತಿದ್ದೆ. ನಾನು ಇನ್ನೇನಕ್ಕೂ ಯೋಗ್ಯನಲ್ಲವೆಂದು ನಮ್ಮ ಮನೆಯ ಯಜಮಾನ \enginline{-} ನನ್ನಣ್ಣನ ಸ್ಪಷ್ಟ ತೀರ್ಮಾನವಾಗಿತ್ತು. ಓದು ಮತ್ತು ಕೃಷಿ ಈ ಎರಡು ಕ್ಷೇತ್ರದ ಆದ್ಯತೆಯ ವಿಷಯದಲ್ಲಿ ವ್ಯತಿರಿಕ್ತ  ನಿಲುವು ನಮ್ಮಲ್ಲಿ ವಾದದ\enginline{-}ವಿವಾದದ ವಿಷಯವಾಗಿತ್ತು. ಬಂಧುವರ್ಗದ, ತಂದೆತಾಯಿಯರೆಲ್ಲರ ಅಭಿಪ್ರಾಯ ಪಾಠಶಾಲೆ\-ಗಾದರೂ ನನ್ನನು ಕಳುಹಿಸಬಹುದಿತ್ತು ಎಂಬ ನಿಲುವಾಗಿತ್ತು. \textit{ನಮ್ಮಲ್ಲಿ ಇರುವ\break ಅಭಿಪ್ರಾಯವೇ ಹಾಗೆ \enginline{-} ಏನೂ ಆಗದವನು ಪಾಠಶಾಲೆ\-ಗಾದರೂ \hbox{ಹೋಗಬಹುದಲ್ಲ!} ಎಂಬುದು. ಸಂಸ್ಕೃತ ಕ್ಷೇತ್ರಕ್ಕೆ ಬರುವ ಹೆಚ್ಚಿನವರು ಅವರ ಮನೆಗಳಲ್ಲಿ ಇಂಥದ್ದೇ ಅಭಿಪ್ರಾಯ ಹೊಂದಿರುವವರೇ ಹೆಚ್ಚು. ಅದರೆ ಮನೆಯಿಂದ ಬಿಡಿಗಾಸನ್ನೂ ಪಡೆಯದೇ ಇಲ್ಲಿ ಬಂದು ವೇದ, ಸಂಸ್ಕೃತ ಓದುತ್ತ ವಾರಾನ್ನವನ್ನೋ, ಹೋಟೆಲ್‌ಗಳಲ್ಲೋ ಆಹಾರಕ್ಕಾಗಿ ಅವಲಂಬಿಸಿ ಪೌರೋಹಿತ್ಯವನ್ನೂ ಮಾಡಿಕೊಂಡು ಅಳಿದುಳಿದ ಹಣವನ್ನು ಊರಿನಲ್ಲಿರುವ ತಂದೆತಾಯಿಗಳಿಗೆ ಕಳುಹಿಸಿ, ಅಥವಾ ಇಲ್ಲಿಂದಲೇ ಅವರನ್ನು\break ಸಾಕುವವರೂ ನಮ್ಮಲ್ಲಿ ಅನೇಕರಿದ್ದಾರೆ ಎಂದರೆ ಹೆಚ್ಚಿನವರು ನಂಬಲಾರರು. ನಮ್ಮ ಸಂಸ್ಕೃತ\enginline{-}ಸಂಸ್ಕೃತಿಯ ಪ್ರಕೃತ ಪರಿಸ್ಥಿತಿಗೆ ಇಂಥವುಗಳಲ್ಲೂ ಕಾರಣವನ್ನು ಅನ್ವೇಶಿಸ\-ಬೇಕಾದ ಜವಾಬ್ದಾರಿ ಅದಕ್ಕೆ ಸಂಬಂಧಿಸಿದವರಿಗುಂಟು}.  ಪ್ರಕೃತಕ್ಕೆ ಬರುವುದಾದರೆ, ನಮ್ಮ ಯಜಮಾನರಿಗೆ ಮಾತ್ರ ಅದಕ್ಕೂ ನಾನು ಅಯೋಗ್ಯ ಎಂದಿತ್ತು.  ಅಷ್ಟು ಹೊತ್ತಿಗಾಗಲೇ ನಾಲ್ಕು ವರ್ಷ ಕೃಷಿಯಲ್ಲಿ ಕಳೆಯಿತು. ನನಗೆ ನಾನೇ ಕಳೆದುಹೋಗುವ\break ಆತಂಕವಿತ್ತು. ಮೊದಲಿನಿಂದಲೂ ಸಂಸ್ಕೃತದ ವಾಸನೆಯೇನೋ ಇತ್ತು. ಗಂಗಣ್ಣ\break ಮೈಸೂರಿನಲ್ಲಿ ಇರುವುದರಿಂದ ನನಗೂ ಪ್ರತಿದಿನ  ಮೈಸೂರಿನ  ಕನಸು ಬೀಳುತ್ತಿತ್ತು. ಅನೇಕ ವೇಳೆ ಮನೆಯಿಂದ ಓಡಿಬರುವ ಪ್ರಯತ್ನವನ್ನು ಮಾಡಿಯೂ ತಕ್ಕ ಫಲ ಗಿಟ್ಟಿಸಿ\-ಕೊಳ್ಳುವಷ್ಟು ಯಜಮಾನರ ಬಲದ ಮುಂದೆ ನನ್ನ ಬಲ ಸಾಕಾಗುತ್ತಿರಲಿಲ್ಲ. ರಾಮಾಯಣ ಕಾಲದಲ್ಲಿದ್ದ ವಿಶೇಷ ವರವುಳ್ಳ ವಾಲಿಯಂತೆ ಯಜಮಾನರು ನನ್ನ ಬಲವನ್ನು ಉಡುಗಿಸಿಬಿಡುತ್ತಿದ್ದರು. ಕಾಲ ಕೂಡಿಬರದಿರುವಾಗ ಇವೆಲ್ಲ ಹೀಗೆಯೇ. ಏಕೆಂದರೆ,\break ದೈವಾನುಗ್ರಹದಿಂದ ಮುಂದೆ ನಡೆಯಬೇಕಾದು ಅದಾಗಿಯೇ ನಡೆಯಿತು, \enginline{-} ಹಿಂದಿನ ನಮ್ಮ ನಿಲುವಿನ ಪ್ರಾಮಾಣ್ಯವನ್ನು ಮುಂದೆ ಘಟಿಸುವ ಘಟನೆಗಳು ಪರೀಕ್ಷಿಸಿ ಹಾಲನ್ನು ಹಾಲು ನೀರನ್ನು ನೀರು ಎಂದುಬಿಡುತ್ತವೆಯಷ್ಟೆ!!!.

ಹೀಗಿರಲು ಒಮ್ಮೆ ಗಂಗಣ್ಣ ನಮ್ಮ ಮನೆಗೆ ದಯಮಾಡಿಸಿದ. ನಾನು\break ನಮ್ಮೆಜಮಾನರಿಗೆ ತಿಳಿಯದಂತೆ ಗಂಗಣ್ಣನ ಕಿವಿಯಲ್ಲಿ ನನಗೆ ಬೀಳುತ್ತಿರುವ ಕನಸನ್ನು ಉಸುರಿದೆ. ಅವನ ವ್ಯವಹಾರ ಚಾತುರ್ಯ ಯಾರಿಗೂ ವಿರೋಧವಾಗದಂತೆ ಸಮಸ್ಯೆಯನ್ನು ನಿರ್ವಾಹಮಾಡುತ್ತಿತ್ತು. 

ನಮ್ಮ ಮನೆಯವರೆಲ್ಲರೂ ಇರುವಾಗ ಅವನು ನನ್ನಲ್ಲಿ ಒಂದು ಪ್ರಶ್ನೆಯನ್ನು ಕೇಳಿದ \enginline{-} ನಿನ್ನ ಗುರು ಯಾರು ? ನಾನು ತಡಬಡಾಯಿಸಿ ಮೇಲೆ ಕೇಳಗೆ ನೋಡಿದೆ.\break ಪ್ರಾಥಮಿಕ ಶಿಕ್ಷಣದ ಕಾಲದಿಂದ ಹಿಡಿದು ಮಾಧ್ಯಮಿಕ ಶಿಕ್ಷಣದ ತನಕ \hbox{ಕಲಿಸಿದವರೆಲ್ಲರೂ} ಆಗ ನಮಗೆ ತಿಳಿದ ಗುರುಗಳೇ ಆಗಿದ್ದರು. ಆದರೆ ಅವರಾರನ್ನೂ ನನ್ನ ಮನಸ್ಸು ಈ ಪ್ರಶ್ನೆಗೆ ಉತ್ತರವಾಗಿ ಗ್ರಹಿಸಿರಲಿಲ್ಲ ಅಥವಾ ಸ್ವೀಕರಿಸಿರಲ್ಲಿಲ್ಲ \enginline{-} ಇದೇ ನಿಜ.  ಮತ್ತೆ ಗಂಗಣ್ಣ ಹೇಳಿದ \enginline{-} ನನ್ನನ್ನು ಕೇಳಿದರೆ ನನ್ನ ಗುರುಗಳು ಎನ್.ಎಸ್.ರಾಮಭದ್ರಾಚಾರ್ಯರು ಎಂದು ಹೇಳುತ್ತೇನೆ. ಅನೇಕರಿದ್ದರೂ ನಿನಗೆ ಹೀಗೆ ಹೇಳಲು ಯಾರೂ ಇಲ್ಲವಲ್ಲವೇ !!”  ನನಗೂ ಹೌದಲ್ಲ !! ಎನ್ನಿಸಿತು, ನಾನು ತಲೆಯಾಡಿಸಿದೆ. ಆಗ ಹೇಳಿದ, “ಜೀವನದಲ್ಲಿ ಸರಿಯಾದ ಗುರುವನ್ನು ಹುಡುಕಿಕೊಳ್ಳಬೇಕು” ಎಂದು. ಇದು ನನನ್ನು ಗಂಭೀರ ಸ್ಥಿತಿಗೆ ತಳ್ಳಿತು. ಗಂಗಣ್ಣ ಮನೆಯ ಕಡೆಗೆ ಹೊರಟ. ನಾನು ಬಹಳ ದೂರ ಜೊತೆಯಲ್ಲೇ ಹೋಗಿ ಮನೆಯಿಂದ ಓಡಿಹೋಗಲು ಬೇಕಾದ, ರಾಮನಿಂದ ದೊರಕಿದ ಮನೋಬಲವುಳ್ಳ\break ಸುಗ್ರೀವನಂತೆ ಮನೋ ಬಲವನ್ನು ಸಂಪಾದಿಸಿಕೊಂಡು  ಹಿಂದಿರುಗಿದೆ. 

ಒಬ್ಬ ಹಿತೈಷಿಗಳಿಂದ ನಾನು ಪ್ರಯಾಣಕ್ಕೆಲ್ಲ ಅಗತ್ಯವಿದ್ದ ಧನವನ್ನು ಸಾಲ ಪಡೆದುಕೊಂಡೆ. ೧೯೯೫ ರ ಜೂನ್ ತಿಂಗಳ ಅಂತಿಮ ದಿನ. ರಾತ್ರೋರಾತ್ರಿ ತಂದೆಗೆ ನಮಸ್ಕರಿಸಿ, ನನ್ನನ್ನು ಹುಡುಕದಿರುವಂತೆ ಯಜಮಾನರ ಹಾಸಿಗೆಯಲ್ಲಿ ಚಿಕ್ಕ ಚೀಟಿಯನ್ನಿಟ್ಟು ಮತ್ತಾರಿಗೂ ತಿಳಿಯದಂತೆ ಸರ್ಟಿಫಿಕೆಟ್ ನ ಒಂದು ಬ್ಯಾಗನ್ನು ಹಿಡಿದು ಹೊರಟೇ ಬಿಟ್ಟೆ.

ಅಲ್ಲಿ ಇಲ್ಲಿ ಬಿದ್ದು ಎದ್ದು ಮೈಸೂರಿಗೆ ಬಂದು ತಲುಪಿದೆ. ಆದರೆ ನಮ್ಮ ಸಿದ್ದಾಪುರಕ್ಕಿಂತ ಸ್ವಲ್ಪವೇ ಚಿಕ್ಕದಿರುವ ಮೌಸೂರಿನ ಬಸ್ ಸ್ಟ್ಯಾಂಡ್ ನಲ್ಲಿ ಇಳಿದಾಗ ಬೆಳಗಿನ ಜಾವ ೫ ಗಂಟೆ. ದಿಕ್ಕು ಕಾಣಲಿಲ್ಲ. ಯಾರ್ಯಾರನ್ನೋ ಕೇಳಿಕೊಂಡು ಸಂಸ್ಕೃತ ಪಾಠಶಾಲೆಗೆ ಬಂದು ಗಂಗಾಧರ ಭಟ್ಟರ ಮನೆಯನ್ನು ತೋರಿಸುವಂತೆ ಕೇಳಿ ಅಂತೂ ಗಂಗಣ್ಣನ ಮನೆಗೆ ಬಂದು ತಲುಪಿದೆ. 

ಆಗಲೇ ಸಂಸ್ಕತ ಪಾಠಶಾಲೆಯಲ್ಲಿ ಎಡ್ಮಿಶನ್ನಿನ ಸರಕಾರಿ ನಿಯತ ಸಮಯವೆಲ್ಲ\break ಮುಗಿದಿತ್ತು. ಆದರೆ ಸಂಸ್ಕೃತ ಪಾಠಶಾಲೆಯ ವ್ಯವಹಾರ ಹಾಗೆ ಮಾಡಿದರೆ ನಡೆಯು\-ವುದು ಕಷ್ಟ. ಮತ್ತೆ, ಗಂಗಣ್ಣ ಕಳುಹಿಸುವ ಹುಡುಗರು ಅಲ್ಲಿ ಯಾವಾಗಲೂ ಸೇರ\-ಬಹುದು !!! ಗಂಗಣ್ಣನೇ ನೇರವಾಗಿ ಬಂದು ಕೂಡಲೇ ಪಾಠಶಾಲೆಯಲ್ಲಿ ಎಡ್ಮಿಶನ್ ಮಾಡಿಸಿದ. ಸಾಹಿತ್ಯಕ್ಕೆ ಎಡ್ಮಿಶನ್ ಆಯಿತು. ಅಲ್ಲಿಗೆ ನನ್ನ ಗ್ರಾಂಥಿಕ ಕೃಷಿ ಪ್ರಾರಂಭ\-ವಾಯಿತು. ಪಾಠಶಾಲೆಯ ಹಾಸ್ಟೇಲ್ ನಲ್ಲಿ ಒಂದೂ ರೂಮ್ ಸಹ ಖಾಲಿ ಇರಲಿಲ್ಲ. ಆದರೆ ಅಲ್ಲಿ ಮತ್ತೆ ಗಂಗಣ್ಣ  ಬುದ್ಧಿವಂತಿಕೆಯಿಂದ ಒಂದು ರೂಮಿನಲ್ಲಿ ವ್ಯವಸ್ಥೆ ಕಲ್ಪಿಸಿ\-ಕೊಟ್ಟ. ಮಧ್ಯಾಹ್ನ, ರಾತ್ರಿ ಊಟಕ್ಕೆ ವೇದಶಾಸ್ತ್ರ ಪೋಷಿಣೀ ಸಭೆಯಲ್ಲಿ\break ವ್ಯವಸ್ಥೆಯಾಯಿತು. 

ಗಂಗಣ್ಣನೊಡನೆಯೇ ಶಂಕರವಿಲಾಸ ಪಾಠಶಾಲೆಗೆ ಹೋಗುತ್ತಿದ್ದೆ. ಅಲ್ಲಿ ಅವನು ತರ್ಕ ವ್ಯಾಕರಣಗಳನ್ನು ಪಾಠಮಾಡುತ್ತಿದ್ದ. ಅಲ್ಲದೇ ಕೆಲವು ಕಾವ್ಯಪಾಠ ಮಾಡುತ್ತಿದ್ದ. ಆ ಪಾಠ ರಸಸ್ಯಂದಿಯಾಗಿತ್ತು. ವಿಶೇಷವಾಗಿ ಕರುಣ, ವೀರ ರಸಗಳು ಅವನಲ್ಲಿ ಚೆನ್ನಾಗಿ ಅಭಿವ್ಯಕ್ತವಾಗುತ್ತಿದ್ದವು. ಶ್ಲೇಷ ಮತ್ತು ಧ್ವನಿ ಅವನ ಮಾತಿನಲ್ಲಿ ಹಾಸುಹೊಕ್ಕಾಗಿರುತ್ತದೆ. ಅಲ್ಲೇ ಹಾಸ್ಯ ಹೊರಹೊಮ್ಮುತ್ತದೆಯೇ ವಿನಾ ಹಾಸ್ಯಕ್ಕಾಗಿ ಹಾಸ್ಯಮಾಡುವ ದಾರಿದ್ರ್ಯ ಇಲ್ಲ, ಪ್ರವೃತ್ತಿಯೂ ಇಲ್ಲ. ಹಾಗಾಗಿ ಹಾಸ್ಯ ಎಲ್ಲೂ ಅಪಹಾಸ್ಯಕ್ಕೊಳಗಾಗಿದ್ದಿಲ್ಲ.\break ಸಾಮಾನ್ಯವಾಗಿ ಹುಡುಗರನ್ನು ತಿದ್ದುವುದು ಉಪಾಯವಾದ ಹಾಸ್ಯದಿಂದಲೇ ವಿನಾ ಗದರುವ ಸ್ವಭಾವವೇ ಇರಲಿಲ್ಲ. ಯಾವುದೇ ಭಾವವನ್ನು ಅಗತ್ಯವಿದ್ದಲ್ಲಿ ಹದವಾಗಿ ಅಭಿವ್ಯಕ್ತಿಸುವ ಸಂಯಮ ಪ್ರಕೃತಿ. ಎಂತಹ ಪರಿಸ್ಥಿತಿಯಲ್ಲೂ ಸಮಾಧಾನವನ್ನು ಕಳೆದು\-ಕೊಳ್ಳದೇ, ಉದ್ವಿಗ್ನವಾಗದೇ ಇರುವ ಸಂಯಮ ಸ್ವಭಾವ ಅವನಲ್ಲಿ ಅಸಾಧಾರಣ\-ವಾಗಿದೆ. ಅದೇ ಅವನ ನಿಜವಾದ ಶಕ್ತಿ ಎಂದರೆ ತಪ್ಪಿಲ್ಲ. ಬಹುಶಃ ಇಷ್ಟು ಕಾಲದ ನನ್ನ ಒಡನಾಟದಲ್ಲಿ ಅವನಿಗೆ ಸರಿಯಾಗಿ ಸಿಟ್ಟು ಬಂದಿರುವುದನ್ನು   ಒಮ್ಮೆಯೋ ಎರಡು\break ಬಾರಿಯೋ ನೋಡಿದ್ದೇನೆ. ಅನ್ಯಾಯವನ್ನು ಕಿಂಚಿತ್ತೂ ಸಹಿಸದ ಸ್ವಭಾವ, ಅದು ಘಟಿಸಿದರೆ ಅಲ್ಲಿಯೇ ಪ್ರತಿಭಟನೆ ಇದ್ದೇ ಇರುತ್ತದೆ. ಅನ್ಯಾಯವೇನಾದರೂ  ಅತಿಯಾಗಿ ಉದ್ದೇಶಪೂರ್ವಕವಾಗಿದ್ದರಂತೂ ಎದುರಿಗಿರುವವನು ಮುಂದೆಂದೂ ಮತ್ತೆ\break ಅವನೆದುರು ನಿಲ್ಲಲಾರ \enginline{-} ಅಷ್ಟಾಗುವುದು ಶತಸ್ಸಿದ್ಧ. ಆದರೆ ಅವನ ಶಕ್ತಿಯೇ ಸಂಯಮ. ಆದ್ದರಿಂದಲೇ ಎಂತಹ ವ್ಯವಹಾರವನ್ನೂ ಹಗುರವಾಗಿ, ಎಲ್ಲೂ ಡ್ಯಾಮೇಜ್ ಆಗದಂತೆ ನಿರ್ವಾಹ ಮಾಡುವ ಕೌಶಲ ಅವನಿಗೆ ಸಿದ್ಧಿಸಿದೆ. ಅದೇ ಸಂಯಮ ಮತ್ತು ಕೌಶಲ\break ಪಾಠದಲ್ಲೂ ವಿಷಯ ನಿರೂಪಣೆಯಲ್ಲೂ ಕಾಣುತ್ತದೆ. ಹಾಗಾಗಿ ಆಯಾ \hbox{ಸಂದರ್ಭಕ್ಕೂ}, ಪಾಠಕ್ಕೂ ಹೊರತಾದ ಯಾವ ವಿಷಯವೂ ಅಪ್ಪಿತಪ್ಪಿಯೂ ಅಲ್ಲಿ ನುಸುಳುತ್ತಿರ\-ಲಿಲ್ಲ. ಅವನ ಪಾಠದಲ್ಲೂ ಸಹ ಒಂದು ವಿಶಿಷ್ಟ ನೋಟ, ಸರಣಿ ಇರುತ್ತಿತ್ತು, ಅದು ನಮ್ಮನ್ನು ಬಹಳ ಆಕರ್ಷಿಸುತ್ತಿತ್ತು. ಅದು ಯಾವ ಮೂಲದ್ದೆಂದು ಅನಂತರದ ದಿನಗಳಲ್ಲಿ ನನಗೆ ಮನದಟ್ಟಾಯಿತು. ಹೀಗೆ ಪಾಠ ಶಾಲೆಯಲ್ಲಿ ನಡೆದರೆ ಮನೆ ಇನ್ನೊಂದು \hbox{ಶಾಲೆಯಂತೆ} ನಡೆಯುತ್ತಿತ್ತು. ಅಲ್ಲಿಗೆ ಸಾಕಷ್ಟು ಜನರು ಬೇರೆ ಬೇರೆ ವಿಷಯಗಳಿಗಾಗಿ ಪಾಠಕ್ಕೆ ಬರುತ್ತಿದ್ದರು. ಆ ಪಾಠ ನನಗೂ ಆಗುತ್ತಿತ್ತು. ಸಾಹಿತ್ಯದ ಕೆಲವು ಪಾಠ ಮಹಾರಾಜ ಪಾಠ\-ಶಾಲೆಯಲ್ಲಿ ಆಗುತ್ತಿತ್ತು. ಈ ಕಾರಣದಿಂದ ನಾನು ಏನೂ ಬರದವನು ಹಸು\break ಮೇಯಿಸಲು ಮಾತ್ರ ಯೋಗ್ಯನೆಂದು ನಿರ್ಣಯಿಸಲ್ಪಟ್ಟವನು ಒಂದೇ ವರ್ಷದಲ್ಲಿ ಸಾಹಿತ್ಯದ ಅಂತಿಮ ವರ್ಷದ ಪರೀಕ್ಷೆ ತೆಗೆದುಕೊಂಡು ತೇರ್ಗಡೆಯಾಗುವುದು ಸಾಧ್ಯ\-ವಾಯಿತು. ಬಹುಶಃ ಆಗ ಸಾಹಿತ್ಯದ ಕ್ಲಾಸಿನ ನಮ್ಮ ಬ್ಯಾಚಿನಲ್ಲಿ ತರ್ಕವನ್ನು ಸಂಸ್ಕೃತದಲ್ಲಿ\break ಬರೆದಿದ್ದು ನಾನೊಬ್ಬನೇ ಇರಬೇಕು. ಸಂಸ್ಕೃತದಲ್ಲೇ ಬರೆಯಬೇಕೆಂದು ಗಂಗಣ್ಣ ಹೇಳಿದ್ದರಿಂದ ಅದು ಸಾಧ್ಯವಾಯಿತೇ ವಿನಾ ಇಲ್ಲದಿದ್ದರೆ ಆಗುತ್ತಿರಲಿಲ್ಲ.

ಅದೇ ವರ್ಷದಲ್ಲಿ ಗಂಗಣ್ಣ ನನ್ನನ್ನು ಹೊಳೇನರಸೀಪುರಕ್ಕೆ ರಾಮಾಯಣ ಪಾರಾಯಣ ಮಾಡಲು ಹೋಗುವಂತೆ ಹೇಳಿದ. ಪಾರಾಯಣವನ್ನು ನಿರ್ದಿಷ್ಟ ಅವಧಿಯಲ್ಲಿ ಮುಗಿಸಲು ಹೇಗೆ ಓದಬೇಕು ಎಂಬೆಲ್ಲ ಅಂಶವನ್ನು ಕಲಿಸಿ ಕಳುಹಿಸಿಕೊಟ್ಟ. ನಾನು ಹತ್ತು ದಿನಗಳು ಅಲ್ಲಿದ್ದು ಅದನ್ನು ಮುಗಿಸಿ ಬಂದೆ. ನನಗೆ ಎರಡುಸಾವಿರ ಚಿಲ್ಲರೆ\break ದಕ್ಷಿಣೆ ದೊರೆಯಿತು. ಆಗ ಅವನೇ ಬಂದು ಎಸ್.ಬಿ.ಐನಲ್ಲಿ ಖಾತೆಯನ್ನು ಮಾಡಿಸಿ\-ಕೊಟ್ಟು ಆ ಹಣವನ್ನು ಅಲ್ಲಿ ಹಾಕಿಸಿದ. ಅದರಲ್ಲಿ ನಾನು ಮೈಸೂರಿಗೆ ಬರುವಾಗ ತಂದಿದ್ದ ಸಾಲವನ್ನು ತೀರಿಸುವಂತೆ ಹೇಳಿದ. ನಾನು ಹಾಗೆಯೇ ಮಾಡಿದೆ. ಮುಂದೆ ಮತ್ತೊಮ್ಮೆ\break ಧರ್ಮಸ್ಥಳದ ಸಮೀಪ ನಡೆದ ಅಯುತಚಂಡಿ ಯಾಗಕ್ಕೂ ಅವನೇ ಒತ್ತಾಯಿಸಿ ಕಳಿಸಿದ್ದ. ಹೀಗೆ, ಕೇವಲ ನನ್ನಲ್ಲಿ ಅಂತ ಅಲ್ಲ, ಯಾರಲ್ಲೂ ಇದೇ ರೀತಿಯಲ್ಲಿ ಪರಹಿತ ಬಯಸುವ ಶುದ್ಧವ್ಯವಹಾರವನ್ನು ಈ ತನಕವೂ ಅವನಲ್ಲಿ ನೋಡುತ್ತಲೇ ಬಂದಿದ್ದೇನೆ. ಎಲ್ಲೂ ಅದರ ಹದ ವ್ಯತ್ಯಾಸವಾಗಿದ್ದು ನನ್ನ ಗಮನಕ್ಕೆ ಬಂದಿಲ್ಲ.

ಸಾಹಿತ್ಯ  ಮುಗಿಯುತ್ತಿದ್ದಂತೆಯೇ ಶಂಕರವಿಲಾಸ ಪಾಠಶಾಲೆಯಲ್ಲಿ ಪ್ರಥಮಾ,\break ಕಾವ್ಯಕ್ಕೆ ಪಾಠಮಾಡುವಂತೆ ತಿಳಿಸಿದ. ನಾನು ಗಾಬರಿಯಾದೆ. ಇನ್ನೂ ನಾನು ಸಂಸ್ಕೃತ ಪುಸ್ತಕ ಹಿಡಿದು ಸರಿಯಾಗಿ ಒಂದು ವರುಷವೂ ಆಗಿಲ್ಲ. ಆಗಲೇ ಪಾಠಮಾಡುವುದು ಹೇಗೆ ಸಾಧ್ಯ. ಅದೂ ಸಹ ಅಲ್ಲಿಗೆ ಆಗಲೇ ಸಂಸ್ಕೃತ ಭಾರತಿಯಿಂದ ಸಂಸ್ಕೃತದ\break ಪರಿಚಯವಿದ್ದ ನಮ್ಮೆದುರಿಗೇ ಅಲ್ಪಸ್ವಲ್ಪ ಸಂಸ್ಕೃತಲ್ಲಿ ಮಾತನಾಡುತ್ತಿದ್ದ ಕೆಲವು ಸುಸಂಸ್ಕೃತರಾದ ಹಿರಿಯ ಸ್ತ್ರೀಯರು ಪಾಠಕ್ಕೆ ಬರುತ್ತಿದ್ದರು. ಅವರಿಗೆಲ್ಲ ಪಾಠಮಾಡುವುದು ಸಾಧ್ಯವಿಲ್ಲ ಎಂದೆ. ಸಾಧ್ಯವಿದೆ ಎಂದು ಅವನ ವಾದ. ನಾನು ಸಾಹಿತ್ಯ ವಿದ್ವತ್ತನ್ನು ವಿಧಿವತ್ತಾಗಿ ಓದಿ ವ್ಯುತ್ಪತ್ತಿಯುಳ್ಳ ಹುಡುಗರನ್ನು ಅಲ್ಲಿ ಪಾಠಮಾಡುವಂತೆ ಕೇಳಿದೆ. ಯಾರೂ ಒಪ್ಪಲಿಲ್ಲ. ಗಂಗಾಧರ ಭಟ್ಟರು ನಿನಗೆ ಮಾಡಲು ಹೇಳಿದ್ದಾರೆ. ನೀನೇ ಮಾಡು, ಎಂದರು. ವಿಧಿಯಿಲ್ಲದೇ ಪಾಠವನ್ನು ಮಾಡಲೇಬೇಕಾಯಿತು. ಯೋಗ ಬಲವಾಗಿದ್ದರೆ ಅದು ಫಲ ಕೊಡುವುದಕ್ಕೆ ಸುಮ್ಮನೆ ಕಾಲಕಾಯುತ್ತಿರುತ್ತದೆ. ಆ ಕಾಲ ಬಂದಾಗ ನಾವೇನೂ ಮಾಡಬೇಕಿಲ್ಲ. ಅದೇ ಎಲ್ಲವನ್ನೂ ಮಾಡಿಬಿಡುತ್ತದೆ. ನಾವು ಅನ್ಯಥಾ ಚೇಷ್ಟೆ ಮಾಡದೆ ಪ್ರೇಕ್ಷರಂತೆ ವೀಕ್ಷಕರಾಗಿದ್ದರೆ ಸಾಕು. ಹಾಗಾಗಿ ಶುದ್ಧ ಅಯೋಗ್ಯನಾಗಿದ್ಧ ನಾನು ಪಾಠವನ್ನು\break ಯೋಗ್ಯವಾಗಿಯೇ ಮಾಡುವುದು ಸಾಧ್ಯವಾಯಿತು. ಗಂಗಣ್ಣ ಕಾಲಕಾಲಕ್ಕೆ ಬೇಕಾದ ವಿಷಯಗಳನ್ನು ಪಾಠಕ್ಕೆ ಬೇಕಾದ ಮಾರ್ಗದರ್ಶನವನ್ನು ಮಾಡುತ್ತಿದ್ದ. ಈ ಪಾಠ\break ಮಾಡುತ್ತಿದ್ದುದರಿಂದ ನನಗೆ ತಿಂಗಳು ಐನೂರು ರೂಪಾಯಿಯ ಸಂಬಳವನ್ನು ಆ ಪಾಠಶಾಲೆಯಿಂದ ಗಂಗಣ್ಣ ಕೊಡಿಸಿದ. ಮುಂದೆ ಇದೇ ಹಣ ನನಗೆ ಮಾನಸ ಗಂಗೋತ್ರಿಯಲ್ಲಿ ಸಂಸ್ಕೃತ ಎಂ.ಎ ಮಾಡುವುದಕ್ಕೂ ಸಹಾಯವಾಯಿತು. ನನ್ನ ಅಗತ್ಯ\break ಪುಸ್ತಕಗಳಿಗೆ ಬಳಕೆಯಾಯಿತು. ಇದಲ್ಲದೇ ಸಂಗೀತ \enginline{-} ಗಾಯನ ಮತ್ತು ತಬಲಾ ಕಲಿಯುತ್ತಿದ್ದ ನಾನು ಅದಕ್ಕೆ ಅಗತ್ಯವಿದ್ದ ವಾದ್ಯದ ಸೆಟ್ ಗಳನ್ನೂ ಇದೇ ಹಣದಿಂದ ತೆಗೆದು\-ಕೊಂಡೆ. ಹೀಗೆ ಗಂಗಣ್ಣನ ಕಾರಣದಿಂದಲೇ ಇವನ್ನೆಲ್ಲ ನಾನು ಮಾಡುವುದು ಸಾಧ್ಯ\-ವಾಯಿತು. ಮನೆಯಿಂದ ಬಂದ ಒಂದೇ ವರ್ಷದಲ್ಲಿ ವಿದ್ಯಾರ್ಥಿಯೂ ಆಗಿ\break ಅಧ್ಯಾಪಕನೂ ಆಗಿದ್ದರ ಹಿಂದೆ ಇದ್ದುದು ಗಂಗಣ್ಣ ಒತ್ತಾಸೆಯೇ ವಿನಾ ಬೇರೆಯೇನೂ ಅಲ್ಲ. 

ಮುಂದೆ ನಾನು ತರ್ಕಶಾಸ್ತ್ರವನ್ನು ವಿದ್ವತ್ ತರಗತಿಯಲ್ಲಿ ಆಯ್ಕೆ ಮಾಡಿಕೊಂಡೆ. ಗಂಗಣ್ಣನಲ್ಲಿ  ಪಾಠ ಮಾಡಿಸಿಕೊಳ್ಳುತ್ತಿದ್ದೆ. ಒಂದೇ ವರುಷದಲ್ಲಿ ಸಾಹಿತ್ಯವನ್ನು\break ಓದಿದ್ದರ ಪರಿಣಾಮ ವಿದ್ವತ್ ತರಗತಿಯ ಪಾಠವನ್ನು ಅರ್ಥಮಾಡಿಕೊಳ್ಳಲು ಅಗತ್ಯ\-ವಿರುವ ಹಿಂದಿನ ಪಾಠ ನನಗೆ ಆಗದೇ ಇದ್ದುದು ನನಗೆ ಶ್ರಮವಾಗುತ್ತಿತ್ತು. \hbox{ಪಕ್ಷತಾ} ಗ್ರಂಥ ಅಧ್ಯಯನಕ್ಕಿತ್ತು. ಆ ಗ್ರಂಥದ ಬಗ್ಗೆ, ಪಕ್ಷತಾ ಪ್ರಾಣಘಾತಿನೀ ಎಂಬ\break ನಾಣ್ನುಡಿಯೇ ಇದೆ. ಗ್ರಂಥ ಕಬ್ಬಿಣದ ಕಡಲೆ. ಅದರ ಮೊದಲ ಪುಟ್ಟ ಪಂಕ್ತಿ “ಅಥ\break ವ್ಯಾಪ್ತ್ಯನಂತರಂ ಪಕ್ಷ\-ಧರ್ಮತಾ ನಿರೂಪ್ಯತೇ” ಎಂಬುದು. ಸರಳವಾದ ಆ \hbox{ಪಂಕ್ತಿಗೆ} ಸುಂದರವಾದ, ಆದರೆ ಅಷ್ಟೇ ಟೆಕ್ನಿಕಲ್ ಆದ ನಿರೂಪಣೆ ಅದರ ಅಡಿಯಲ್ಲಿದೆ.\break ನಿರೂಪಣಾ ಶೈಲಿಯೇ ತರ್ಕ\-ಶಾಸ್ತ್ರದ ಜೀವಾಳ ಎಂದರೆ ತಪ್ಪಲ್ಲ. ತರ್ಕದ ಜಾಡು ಸರಿಯಾಗಿ ಸಿಕ್ಕರೆ ಅದು ವ್ಯಾಘ್ರಮುಖ ಗೋವು. ಇಲ್ಲದಿದ್ದರೆ ಅದೂ ವ್ಯಾಕರಣದಂತೆ ಗೋಮುಖ ವ್ಯಾಗ್ರವೇ ಸರಿ. ಆ ಒಂದು ಪಂಕ್ತಿಯ ಅರ್ಥವನ್ನು ನಾನು ಗಂಗಣ್ಣನಿಂದ ನಾಲ್ಕು ಬಾರಿ ಪಾಠ ಮಾಡಿಸಿಕೊಳ್ಳಬೇಕಾಯಿತು. ಆದರೆ ಗಂಗಣ್ಣನ ಸ್ವಭಾವ\break ಹೇಗಿತ್ತೆಂದರೆ ಇನ್ನೂ ನಾಲ್ಕು ಬಾರಿ ನಾನು ಕೇಳಿದರೂ ಮೊದಲನೆಯ ಬಾರಿ ಹೇಳಿದ ಸಮಾಧಾನ, ಉತ್ಸಾಹದಲ್ಲೇ ಆಗಲೂ ಪಾಠ ನಡೆಯುತ್ತಿತ್ತು. ಅಲ್ಲದೇ ಅಷ್ಟು ವಿಭಿನ್ನ\break ವಿಧಾನವನ್ನು ಬಳಸಿ ಅರ್ಥೈಸುತ್ತಿದ್ದ. ಅವನ ಆ ವಾಕ್ಕೌಶಲ ಎಂಥವರನ್ನೂ ಆಶ್ಚರ್ಯ\break ಚಕಿತರನ್ನಾಗಿಸುವಂತಿದೆ,  ಎರಡನೆ ವರ್ಷದಿಂದ ತರ್ಕದ ಜಾಡು ಅರ್ಥವಾಗುತ್ತಾ ಬಂತು. ನಾನು ಮುಂದುವರೆದೆ. 

ನಾನು ವಿದ್ವತ್ತಿನ ಎರಡನೇ ಮೂರನೇ ವರ್ಷದಲ್ಲಿದ್ದಾಗ ಶೈಲಜಕ್ಕನೊಂದಿಗೆ\break ಗಂಗಣ್ಣನ ವಿವಾಹ ನಿಶ್ಚಯವಾಯಿತು. ಹಣಕಾಸಿಗೆ ಅಂಥ ಅನುಕೂಲವೇನಿರಲಿಲ್ಲ. ನನ್ನ ಸಂಪಾದನೆಗೆ ಗಂಗಣ್ಣನೇ ಕಾರಣನಾದ್ದರಿಂದ ನಾನೂ ಒಂದು ಅಳಿಲು ಸೇವೆ ಮಾಡ\-ಬೇಕೆನಿಸಿತು, ನನ್ನ ಬ್ಯಾಂಕಿನಲ್ಲಿರುವುದನ್ನು ತೆಗೆದೆ. ಅದು ಒಂದು ಸಣ್ಣ ಮೊತ್ತವೂ\break ಆಗಲಿಲ್ಲ. ಬ್ಯಾಂಕಿನ ಎಕೌಂಟ್ ನ್ನೇ ಕ್ಲೋಸ್ ಮಾಡಿ ಹಣವನ್ನು ಕೊಟ್ಟೆ. ಅಕೌಂಟ್ ಕ್ಲೋಸ್ ಮಾಡಿದ್ದನ್ನು ಆಗ ಹೇಳಲಿಲ್ಲ. ಆದರೂ ಹಣ ಬೇಡವೆಂದ. ಅದರಲ್ಲೂ ನಾನು ಕೊಟ್ಟಿದ್ದು ನನ್ನ ಸಮಾಧಾನಕ್ಕಾಗಿತ್ತೇ ವಿನಾ ಮದುವೆಗೆ ಅಗತ್ಯ ಇರುವುದಕ್ಕೆ ಅದು ಯಾವ ಲೆಕ್ಕಕ್ಕೂ ಆಗುವಂಥದ್ದಲ್ಲ. ನನ್ನ ಮನಸ್ಸಿನಲ್ಲಿ ಅದನ್ನು ನಾನು ಅವನಿಗೇ ಕೊಟ್ಟಿದ್ದಾಗಿತ್ತು ಏಕೆಂದರೆ ಅದು ಅವನ ಕಾರಣದಿಂದಲೇ ಪ್ರಾಪ್ತವಾಗಿತ್ತು. ಆದರೆ ಕೆಲವೇ ಅವಧಿಯಲ್ಲಿ ಅವನು ಅದನ್ನು ವಾಪಸ್ಸು ಕೊಟ್ಟುಬಿಟ್ಟ. ಆಗ ಅಕೌಂಟ್ ವಿಷಯ ಹೇಳಲೇಬೇಕಾಯಿತು. ಲಘು ಅಸಮಾಧಾನೊಂಡು ಪುನಃ ಬ್ಯಾಂಕಿನಲ್ಲಿ ಖಾತೆ ತೆರೆಸಿಕೊಟ್ಟ.  ಈ ಮಧ್ಯದಲ್ಲಿ ಸಂಪನ್ನವಾದ ವಿವಾಹದಲ್ಲಿ ನಾನು ಎಲ್ಲ ರೀತಿಯ ಕೆಲಸ\enginline{-}ಕಾರ್ಯಗಳಲ್ಲಿ ಆದ್ಯಂತವಾಗಿ ಭಾಗವಹಿಸಿ ಗಂಗಣ್ಣನಿಗೆ ಸಂಪೂರ್ಣವಾಗಿ ಅಸಿಸ್ಟೆಂಟ್ ಆಗಿ ಕೆಲಸ ಮಾಡಿದೆ. ಈ ಅವಕಾಶ ನನಗೆ ಅತ್ಯಂತ ಸಂತೋಷವನ್ನು ತಂದುಕೊಟ್ಟಿತು. ಗಂಗಣ್ಣ ದಂಪತಿಗಳು ಮೈಸೂರಿಗೆ ಬಂದರು. ಗಂಗಣ್ಣನ ವಿಶ್ವಾಸ ನನಗೆ ಹೇಗೆ ಒದಗಿತ್ತೋ ಅದಕ್ಕೆ ಕಿಂಚಿತ್ತೂ ಕೊರತೆಯಿಲ್ಲದ ಪ್ರೀತಿ ವಿಶ್ವಾಸ ಶ್ರೀಮತಿ ಶೈಲಜಕ್ಕಳಿಗೂ ನನ್ನ ಬಗ್ಗೆ ಉಂಟಾಯಿತು.

ಈ ಹೊತ್ತಿಗೆ ಮಹಾರಾಜ ಸಂಸ್ಕೃತ ಪಾಠಶಾಲೆಯಲ್ಲಿ ಹೊಸದಾಗಿ ಅನೇಕ ಶಿಕ್ಷಕರ\break ನೇಮಕವಾಯಿತು. ಇದರಲ್ಲಿ ಗಂಗಣ್ಣನೂ ಅಲ್ಲಿಗೆ ಆಯ್ಕೆಯಾಗಿ ಶಂಕರವಿಲಾಸ ಪಾಠಶಾಲೆಯಿಂದ ಮಹಾರಾಜ ಸಂಸ್ಕೃತ ಪಾಠಶಾಲೆಗೆ ಆಗಮಿಸಿದ. ಅಲ್ಲಿ ಶಾಸ್ತ್ರ ಪಾಠ\break ಆರಂಭವಾಯಿತು. ಅವನ ಪಾಠಕ್ರಮ ಬಹಳ ಗಂಭೀರ. ಆದರೂ ಲಲಿತ. ಪಂಕ್ತಿಗಳನ್ನು ಅರ್ಥೈಸುವ ಕ್ರಮವೂ ವಿಶಿಷ್ಟ. ಯಾವ ವಿಷಯವನ್ನೇ ತೆಗೆದುಕೊಂಡರೂ ನೇರವಾಗಿ ಅದರ ಬೇರು ಅಥವಾ ಮೂಲವನ್ನು ಹಿಡಿದುಬಿಡುವ ಅಸಾಧಾರಣ ಪ್ರತಿಭೆ. ಅದನ್ನು ಅರ್ಥೈಸಲು ಅತ್ಯಂತ ಶೀಘ್ರವಾಗಿ ಸ್ಫುರಿಸುವ ದೃಷ್ಟಾಂತ. ಈ ಎಲ್ಲ ಅಂಶಗಳು ಅವನ ಪಾಠಕ್ಕೆ ಅದ್ಭುತವಾಗಿ ಪುಷ್ಟಿಕೊಡುವ ವಿಷಯಗಳಾಗಿವೆ. ಎಂಥವನಿಗೂ ಯಾವುದೇ\break ವಿಷಯವನ್ನು ಅರ್ಥಮಾಡಿಸುವ ಕಲೆ ಗಂಗಣ್ಣನಿಗೆ ಕರಗತ. ಹಾಗಾಗಿ ಎಲ್ಲ ಶಾಸ್ತ್ರದ\break ಆಸಕ್ತ ವಿದ್ಯಾರ್ಥಿಗಳು ಪಾಠ ಕೇಳುವುದಕ್ಕೆ ಕಾತರಿಸುತ್ತಿದ್ದುದುಂಟು. ಇಲ್ಲಿ ಒಂದು\break ವಿಷಯವನ್ನು ಉಲ್ಲೇಖಿಸಲೇ ಬೇಕಾದ್ದು \enginline{-} ಯಾವತ್ತೂ ನಾನಾ ಕಾರಣಗಳಿಂದ\break ಗಂಗಣ್ಣನ ಸುತ್ತ ಅನೇಕ ವಿದ್ಯಾರ್ಥಿಗಳು ಇದ್ದೇ ಇರುತ್ತಿದ್ದರು. ವಿದ್ಯಾರ್ಥಿಗಳ ವೈಯಕ್ತಿಕ, ಮತ್ತು ಪಾಠಶಾಲೆಗೆ ಸಂಬಂಧಿಸಿದ ಎಲ್ಲ ವ್ಯವಹಾರಗಳಿಗೆ, ಸಮಸ್ಯೆಗಳಿಗೆ\break ಗಂಗಣ್ಣನನ್ನೇ ಅವಲಂಬಿಸುತ್ತಿದ್ದರು. ಆದರೆ ಯಾವ ಹುಡುಗರಲ್ಲೂ ಕಿಂಚಿತ್ತೂ ತರತಮ ಭಾವ\-ವಿರಲಿಲ್ಲ. ಅವನಲ್ಲಿ ಎಲ್ಲರೂ ಸಮಾನರು. ಹಾಗೆಯೇ ಯಾವುದೇ\break ಹುಡುಗರ ಗುಂಪುಗಳಲ್ಲಿ ಉಂಟಾಗುವ ಸಮಸ್ಯೆಯ ಪರಿಹಾರ ಸಾಮೋಪಾಯದಿಂದ ಆಗುತ್ತಿತ್ತೇ ವಿನಾ ಭೇದೋಪಾಯ ಪ್ರಯೋಗವಾಗುತ್ತಿರಲಿಲ್ಲ. ಭೇದೋಪಾಯದ ಪ್ರಯೋಗ ಅಷ್ಟೇ ಅಪಾಯ. ಕಿಂಚತ್ ವ್ಯತ್ಯಾಸವಾದರೂ  ಪರಿಹಾರಕ್ಕೆ ಬಂದ\break ಗುಂಪುಗಳು ತಿರುಗಿ ತಲೆ\-ಹಾಕಂದಂತಾಗಬಹುದು. ಅಷ್ಟು ಜಾಗ್ರತವಾಗಿ ಬಳಸಬೇಕಾದ\break ಉಪಾಯವದು. ಗಂಗಣ್ಣನಲ್ಲಿ ಬೇದ ಅತ್ಯಂತ ಅನಿವಾರ್ಯದಲ್ಲಿ ಕ್ವಚಿತ್ತಾಗಿತ್ತೇ ವಿನಾ ಬಹುಶಃ ಎಲ್ಲಕ್ಕೂ ಸಾಮೋಪಾಯವೇ ಪರಿಹಾರದ ಮಾರ್ಗವಾಗಿತ್ತು. ಅದು ಅದು\break ಸ್ಮಸ್ಯಾ ಪರಿಹಾರದ  ಮಾರ್ಗಗಳಲ್ಲಿ ರಾಜ. ಹಾಗಾಗಿಯೇ ಅವನ ಸಂಪರ್ಕಕ್ಕೆ ಬಂದ ವಿದ್ಯಾರ್ಥಿಗಳಾರೂ ಮತ್ತೆ ಅವನಿಂದ ವಿಮುಖರಾದುದಿಲ್ಲ. ಇದು ಪ್ರಾಸಂಗಿಕವಾಗಿ ಆಡಿದ ಮಾತು.  ಕಾಲೇಜುಗಳಲ್ಲಿ  ಸ್ಪರ್ಧೆಗಳ ಕಾಲ ಬಂತೆಂದರೆ ಸ್ಪರ್ಧೆಯಲ್ಲಿ ಭಾಗ\-ವಹಿಸುವ ಹುಡುಗರಿಗಿಂತ ಗಂಗಣ್ಣ ಹೆಚ್ಚು ಬ್ಯುಸಿಯಾಗಿರುತ್ತಿದ್ದ. ಏಕೆಂದರೆ ಹೆಚ್ಚು ಕಮ್ಮಿ ಎಲ್ಲ ಶಾಸ್ತ್ರದ ವಿದ್ಯಾರ್ಥಿಗಳು, ಅದರಲ್ಲೂ ಒಂದೇ ಶಾಸ್ತ್ರದ ಅನೇಕ ವಿದ್ಯಾರ್ಥಿಗಳು\break ಹಾಜರಾಗುತ್ತಿದ್ದರು. ಬೇರೆ ಬೇರೆ ಶಾಸ್ತ್ರದ ಹುಡುಗರಿಗೆ ಆಯಾಯಾ ಶಾಸ್ತ್ರದಲ್ಲಿ ಒಂದೇ ವಿಷಯವನ್ನು ವಿಭಿನ್ನವಾಗಿ ಸಿದ್ಧಮಾಡಿಕೊಡಬೇಕಾಗುತ್ತಿತ್ತು. ಶಾಸ್ತ್ರರಸಿಕರಿಗೆ ಅದೊಂದು ಶಾಸ್ತ್ರಕ್ರೀಡೆಯಂತೆ ಆಗುತ್ತದೆ. ಅದನ್ನು ಗಂಗಣ್ಣ ಲೀಲೆಯಿಂದ  ಆಡುತ್ತಿದ್ದುದು ನಮಗೆ ನೋಡಲು ಭಲೇ ಖುಷಿಯಾಗಿರುತ್ತಿತ್ತು.

ಗಂಗಣ್ಣ ಒಂದು ಶಾಸ್ತ್ರಕ್ಕೆ ಸೀಮಿತನಾದವನಲ್ಲ. ಆಯುರ್ವೇದ ಮತ್ತು ಅರ್ಥ\-ಶಾಸ್ತ್ರಗಳು ಅವನ ವಿಶೇಷ ಆಸಕ್ತ ವಿಷಯಗಳು. ಆಯುರ್ವೇದ ಮತ್ತು ಸಂಸ್ಕೃತವನ್ನು ಪಾಠಮಾಡಿಸಿಕೊಳ್ಳಲು ಅವನಲ್ಲಿಗೆ ವೈದ್ಯ ವಿದ್ಯಾರ್ಥಿಗಳು ಬರುತ್ತಿದ್ದರು, ಗಂಗಣ್ಣನೇ ಆ ಕಾಲೇಜಿಗೆ ಹೋಗುತ್ತಿದ್ದುದೂ ಉಂಟು. ಚಾಣಕ್ಯನ ಅರ್ಥಶಾಸ್ತ್ರ ಗಂಗಣ್ಣನಿಗೆ ಪರಮ\-ಪ್ರಿಯವಾಗಿತ್ತು. ಪ್ರತಿ ವರ್ಷ ಶಿವಮೊಗ್ಗದಲ್ಲಿ ಕಾಲೇಜಿನ ಅಧ್ಯಾಪಕ ಸಮೂಹಕ್ಕೆ ಅರ್ಥಶಾಸ್ತ್ರದ ಬಗ್ಗೆ ಮೂರ್ನಾಲ್ಕು ಪ್ರವಚನಗಳನ್ನು ಅಲ್ಲಿಯವರು ಏರ್ಪಡಿಸುತ್ತಿದ್ದರು. ಆ ಸಮಯದಲ್ಲಿ ಸಿದ್ಧತಾಕಾರ್ಯ ಇಲ್ಲಿ ಅತ್ಯಂತ ರಭಸವಾಗಿ ನಡೆಯುತ್ತಿತ್ತು. ಅವನು ಸಿದ್ಧತೆ ಮಾಡುವುದೆಂದರೆ ಊಟ ಮತ್ತು ಮಧ್ಯೆ ಒಂದೆರಡು ಚಹಕ್ಕೆ  ಮಾತ್ರ ಬಿಡುವು, ಉಳಿದಂತೆ ಇಡೀ ದಿನ ಏಳುವ ಪ್ರಶ್ನೆಯಿಲ್ಲ. ಆಗ ಶಾಸ್ತ್ರದಲ್ಲಿರುವ ನಗರ ಮುಂತಾದವುಗಳ ನಿರ್ಮಾಣ ಮಾಡುವ ವಿಷಯವನ್ನು ಮನದಟ್ಟು ಮಾಡಲು ವಿವಿಧವಾದ ಚಿತ್ರ\-ಗಳನ್ನೆಲ್ಲ ಇಲ್ಲಿ ಸಿದ್ಧಪಡಿಸಲಾಗುತ್ತಿತ್ತು. ಈ ಸಿದ್ಧತೆಯಲ್ಲಿ  ನಾವು ಸಂಪೂರ್ಣವಾಗಿ ಪಾಲ್ಗೊಳ್ಳು\-ತ್ತಿದ್ದೆವು. ಹಾಗಾಗಿ ಅರ್ಥಶಾಸ್ತ್ರದ ಪರಮಾರ್ಥದೆಡೆಗೆ ಆಗಲೇ ಒಂದು ಲಕ್ಷ್ಯ ಅಥವಾ ಸಂಸ್ಕಾರ ನನಗೂ ಒದಗುವಂತಾಗಿತ್ತು.

ಗಂಗಣ್ಣ ಆಗಾಗ ತನ್ನ ಗುರುಗಳಾದ ಶ್ರೀ ಎನ್.ಎಸ್.ರಾಮಭದ್ರಾಚಾರ್ಯರ ಬಗ್ಗೆ ಹೇಳುತ್ತಿದ್ದ. ಅವರ ಪಾಠದ ಶೈಲಿಯನ್ನು ಪರಿಚಯಿಸುತ್ತಿದ್ದ. ನನಗೆ ಅವರನ್ನು ನೋಡುವ ಕುತೂಹಲ ಉಂಟಾಯಿತು. ಹಲವು ಪ್ರಯತ್ನಗಳಲ್ಲಿ ಅವರನ್ನು  ಭೇಟಿಯಾಗುವುದು ಸಾಧ್ಯವಾಯಿತು. ಆಮೇಲೆ ಅವರ ಪಾಠಪ್ರವಚನಗಳನ್ನು ಸಾಕಷ್ಟು ಕೇಳುವ ಅವಕಾಶವೂ ಒದಗಿತು. ಅವರ ಪಾಠದಲ್ಲಿ ಅಸಾಧಾರಣವಾದ ಚುಂಬಕ ಶಕ್ತಿಯಿತ್ತು. ಒಮ್ಮೆ ಪಾಠ ಕೇಳಿದರೆ ಮತ್ತೆ ಮತ್ತೆ ಆ ಕಡೆ ಸೆಳೆಯುತ್ತಿತ್ತು. ಅವರ ಪಾಠ ಕೇಳಿದ ನನಗೆ ಒಂದು ವಿಷಯ ಸ್ಪಷ್ಟವಾಯಿತು. ಗಂಗಣ್ಣನಲ್ಲಿ ಕಾಣುತ್ತಿದ್ದ ಒಂದು ವಿಶಿಷ್ಟ ಶೈಲಿ ಅದು ಶ್ರೀ ರಾಮಭದ್ರಾಚಾರ್ಯರ ಪಾಠದ ಶೈಲಿಯೇ ಆಗಿತ್ತು. ಯಾವುದೇ ವಿಷಯ\-ವಾದರೂ ಸರಿ, ಅದರ ಜಾಡನ್ನು ಹಿಡಿದು ಕೂಡಲೆ ಆ ವಿಷಯದ ತಳ ತಲುಪು\-ವುದು ಒಂದು ವಿಶಿಷ್ಟ ಪ್ರತಿಭೆ. ಒಂದೊಂದೂ ಶಬ್ದವನ್ನು ಅದರ ಆಮೂಲಾಗ್ರ ಬಿಡಿಸಿ ಬಿಡಿಸಿ ನಿರೂಪಿಸಿಬಿಡುತ್ತಿದ್ದರು. ನಾನು ಹೇಳುವ ಯಾವುದೇ ಒಂದು ಶಬ್ದ ಅಕ್ಷರ ಸಹ\break ಎದುರಿಗಿರುವವನಿಗೆ ಅರ್ಥವಾಗಿಲ್ಲ ಎಂದಾಗಬಾರದು ಎಂಬುದು ಅವರ ಪಾಠದ\break ಪ್ರತಿಜ್ಞೆ. ಪಾಠದ ಆರಂಭದ ಕೆಲವು ಭಾಗ ಮಾಡಿಬಿಟ್ಟರೆ ಮುಂದೆ ಅವನೇ ಗ್ರಂಥವನ್ನು ಓದಿಕೊಂಡು ಅರ್ಥಮಾಡಿಕೊಳ್ಳುವಂತಿರಬೇಕು ಎಂಬ ರೀತಿಯಲ್ಲಿ ಅವರ ಪಾಠವಿರುತ್ತಿತ್ತು. ಈ ರೀತಿಯ ಕಲೆ ಅವರಿಂದ ಗಂಗಣ್ಣನಿಗೂ ಹರಿದಿದೆಯೆಂಬುದು ಉಭಯತ್ರ ಪಾಠ ಕೇಳಿದ ನನಗೆ ಸ್ಪಷ್ಟವಾದ ವಿಷಯ. ಅದಲ್ಲದೇ ಶ್ರೀ ಎನ್,ಎಸ್,ಆರ್ ರವರು\break ದೃಷ್ಟಾಂತ ಕೊಡುವ ವಿಷಯದಲ್ಲಿ ಅವರೇ ಒಂದು ದೃಷ್ಟಾಂತ. ಎಂತಹ ಕಠಿಣ ವಿಷಯ\-ವನ್ನೂ ಎದುರಿಗಿರುವ ವ್ಯಕ್ತಿಗೆ ಅರ್ಥೈಸುವ ನೈಪುಣ್ಯದ ಜೀವವೇ ಅವರ ದೃಷ್ಟಾಂತ\-ವಾಗಿತ್ತು. ಅದು ಅವರ ನಿಜ ಶಕ್ತಿ. ಅವರದೇ ರೀತಿಯಲ್ಲಿ ದೃಷ್ಟಾಂತ ಕೊಡುವ ಶೈಲಿಯೂ ಗಂಗಣ್ಣನಿಗೆ ಒಲಿದಿದೆ. ಇದಲ್ಲದೇ ಒಂದೇ ಮಾತಿನಲ್ಲಿ ಅನೇಕಾರ್ಥವನ್ನು ಹೊಮ್ಮಿಸುವ ವಿಶಿಷ್ಟ ಸಾಮರ್ಥ್ಯ ಸಹ ಅಲ್ಲೂ ಇದ್ದುದು ಇಲ್ಲೂ ಇದೆ. ಹೀಗೆ ಜನ್ಮನಾ ಪ್ರತಿಭೆ\-ಯಿದ್ದ ಗಂಗಣ್ಣ ತನ್ನ ಗುರುಗಳ ಪ್ರತಿಭಾಸಾರವನ್ನೂ  ಅದ್ಭುತವಾಗಿ ಹೀರಿಕೊಂಡಿದ್ದು ನಿಜ.\break ಗುರವಿನೊಂದಿಗೆ ಶ್ಲಿಷ್ಟನಾದಾಗ ಮಾತ್ರ ನಿಜ ಶಿಷ್ಯನಾಗುವುದು ಸಾಧ್ಯವಷ್ಟೆ. ಅಂತಹ ಸ್ವಭಾವ ಗಂಗಣ್ಣನಿಗೆ ಇದೆ. ಅದೇ ಅಲ್ಲಿಯ ಸಾರ ಇಲ್ಲಿ ಹರಿಯುವುದಕ್ಕೂ ಕಾರಣ\-ವಾಗಿದೆ. ಪರಸ್ಪರ ಪ್ರೀತಿ ಅದರ ಹರಿಯುವಿಕೆಗೆ ಕಾರಣವಾದ ಅಂಶ. ಅವರ ಮೇಲಿನ ಭಕ್ತಿಯಕಾರಣದಿಂದ ಅವರ ಕಷ್ಟಗಳಿಗೆ ಏನು ಬೇಕಾದರೂ ಸಹಾಯ ಇವನಿಂದ\break ಸಿಗುತ್ತಿತ್ತು. ಅದನ್ನು ಶ್ರೀ ಎನ್.ಎಸ್.ಆರ್ ರವರೇ ಹೇಳುತ್ತಿದ್ದರು. ಇವನ ವ್ಯವಹಾರ\break ನೈಪುಣ್ಯವನ್ನು, ಕಷ್ಟ ಕಾಲದಲ್ಲಿ ಒದಗಿಬರುವ, ಔದಾರ್ಯದಿಂದ ವರ್ತಿಸುವ ಗುಣ\-ವನ್ನೂ ಅವರು ಪ್ರೀತಿಯಿಂದ ಸ್ಮರಿಸುತ್ತಿದ್ದುದುಂಟು. ಆರ್ಥಿಕವಾಗಿ ಶ್ರೀ ಎನ್.ಎನ್.ಆರ್ ರವರು ಸಾಕಷ್ಟು ಕಷ್ಟ ಅನುಭವಿಸಿದವರು. ಸರ್ಸ್ವತಿ ಲಕ್ಷ್ಮಿಯರು ಕ್ವಚಿತ್ ಮಾತ್ರ\break ಒಟ್ಟಾಗಿರುವುದು ಲೋಕವಿದಿತವಷ್ಟೆ !! ಒಮ್ಮೆ ಅವರ ಮನೆಯಲ್ಲಿ ವಿವಾಹ ಸಮಾರಾಂಭ ಏರ್ಪಾಟಾಗಿತ್ತು. ಅದಕ್ಕಾಗಿ ತಂದ ಅಕ್ಕಿ ಅನಿರೀಕ್ಷಿತವಾಗಿ ಹಿಂದಿನ ರಾತ್ರಿಯೇ ಮುಗಿದು ಹೋಯಿತು. ಏನು ಮಾಡಬೇಕೆಂಬುದೇ ಅವರಿಗೆ ತೋಚದೇ ಕಂಗಾಲಾಗಿಬಿಟ್ಟಿದ್ದರು. ಆದರೆ ಗಂಗಣ್ಣನಿಗೆ ಈ ವಿಷಯ ಹೇಗೆ ತಿಳೀಯಿತೋ ತಿಳಿಯದು. ಬೆಳಗಿನ ಝಾವ ನಾಲ್ಕೂವರೆ ಗಂಟೆಯ ಸಮಯಕ್ಕೆ ಒಂದು ಆಟೋದಲ್ಲಿ ಅಗತ್ಯವಿರುವಷ್ಟು\break ಅಕ್ಕಿಯನ್ನು ಅವರ ಮನೆಗೆ ತೆಗೆದುಕೊಂಡು ಹೋಗಿ ಗಂಗಣ್ಣ ಕೊಟ್ಟುಬಂದ. ಇದು ಎಲ್ಲರಿಗೂ ಆಶ್ಚರ್ಯವನ್ನುಂಟುಮಾಡಿತು. ಇದರಿಂದ ಅಂದು ಗಂಗಣ್ಣ ಅಯಾಚಿತ\-ವಾಗಿ ಅವರ ಮನೆಯ ಮರ್ಯಾದೆ ಉಳಿಸಿದ ಎಂಬ ಈ ಘಟನೆಯನ್ನು ಇತ್ತೀಚೆಗೆ ಎನ್.ಎಸ್.ಆರ್ ರವರ ಸುಪುತ್ರಿ ಕಲ್ಯಾಣಿಯವರು ಗಂಗಣ್ಣನ ವದಾನ್ಯ ಸ್ವಭಾವ\-ವನ್ನು ನನ್ನಲ್ಲಿ ಸ್ಮರಿಸಿಕೊಂಡ್ಡಿದ್ದಾರೆ. ಹಾಗಾಗಿ ಗುರುಗಳ ಸೇವೆಗೂ ಗಂಗಣ್ಣ ಮಾದರಿಯೇ. ಅಂತಹ ತನ್ನ ಗುರುಗಳನ್ನು ಪಾಠಶಾಲೆಯಲ್ಲಿ ಅಧ್ಭುತವಾದ ಅಭಿವಂದನೆಯ ಕಾರ್ಯಕ್ರಮದ ಮೂಲಕ ಗೌರವಿಸಿದ್ದುಂಟು. ವಾಸ್ತವವಾಗಿ ನಮ್ಮ ಪರಂಪರೆಯ ಅಭಿವಂದನಾ ಕಾರ್ಯಕ್ರಮ ಅಂದೇ ಆರಂಭವಾಗಿದೆ ಎಂದರೆ ತಪ್ಪಿಲ್ಲ. ಅಂದು ಆ ಕಾರ್ಯಕ್ರಮದ ಅಧ್ಯಕ್ಷತೆಯನ್ನು ಎನ್. ಬಾಲಸುಬ್ರಹ್ಮಣ್ಯರವರು ವಹಿಸಿದ್ದರು. ಕಾರ್ಯಕ್ರಮ ನಿರೂಪಣೆ ಗಂಗಣ್ಣನದೇ ಆಗಿತ್ತು. ಒಂದು ವಿಶೇಷವೆಂದರೆ ಅಧೈರ್ಯ ಎಂಬುದೇ ಇಲ್ಲದ ಗಂಗಣ್ಣನಿಗೆ ಗುರುಗಳೆದುರು ಮತ್ತು ತಂದೆ ವಿಘ್ನೇಶ್ವರ ಭಟ್ಟರೆದುರು ಮಾತನಾಡುವುದಕ್ಕೆ ಮಾತ್ರ ಭಯವಂತೆ. ಗಂಗಣ್ಣ ಅಂದು ಹೇಳಿರುವುದು ಇಂದೂ ನನಗೆ ಜ್ಞಾಪಕವಿದೆ. ಹಾಗಾಗಿ ಅಂದು ಪ್ರತ್ಯೇಕವಾಗಿ ಮಾತನಾಡದೇ ನಿರ್ವಹಣೆ ಮಾತ್ರ ನಡೆದಿತ್ತು. ಅಭಿವಂದಿಸಲ್ಪಟ್ಟ ಎನ್.ಎಸ್.ಆರ್ ರವರ ಅಂದಿನ ಗಂಭೀರ ಮಾತು. ಅದಕ್ಕೆ ತಕ್ಕನಾದ ಎನ್.ಬಾಲಸುಬ್ರಹ್ಮಣ್ಯರವರ ಮಾತನ್ನು ಮರೆಯುವಂತಿಲ್ಲ. ನಡೆದ ಕಾರ್ಯಕ್ರಮ ಅದ್ಭುತವೂ ಅತ್ಯಂತ ಗಂಭೀರವೂ ಆಗಿತ್ತು. ಅದು ಎನ್.ಎಸ್.ಆರ್ ರವರಿಗೆ ಅತ್ಯಂತ ಸಂತೋಷವನ್ನು ಉಂಟುಮಾಡಿತ್ತು. ಕಾರ್ಯಕ್ರಮ ಮುಗಿದ ಮೇಲೆ ಗಂಗಣ್ಣ ಅವರಲ್ಲಿ, “ನಾನೇನೂ ಮಾತನಾಡಲಾಗಲಿಲ್ಲ” ಎಂದು ಹೇಳಿದ, ತಕ್ಷಣ ಅವರು, “ನಿನ್ನ ನಿರೂಪಣೆಯ ಮಾತುಗಳನ್ನೆಲ್ಲ ಸೇರಿಸಿದರೆ ಒಳ್ಳೆಯ ಪ್ರವಚನವೇ ಆಗುವಂತಿತ್ತಪ್ಪ ! ಎಂದು ಹೇಳಿ, ಬಹಳ ಪ್ರೀತಿಯಿಂದ ಕೆಂಪು ಬಣ್ಣದ ಒಂದು ಉತ್ತಮವಾದ ಶಾಲನ್ನು ಗಂಗಣ್ಣನಿಗೆ ಆಶೀರ್ವಾದ ಮಾಡಿ ಉಡುಗೋರೆಯಾಗಿ ಕೊಟ್ಟರು. ಆಗ ಗಂಗಣ್ಣ, “ಹಿಂದೆ ನನಗೆ ವಿದ್ವತ್ ಶಾಲು ಬಂದಿದ್ದರೂ, ನಿಜವಾದ ವಿದ್ವತ್ ಶಾಲು ಈಗ ನನಗೆ ಬಂತು” ಎಂದು ಹೇಳಿದ್ದು ನನಗಿನ್ನೂ ಚೆನ್ನಾಗಿ ನೆನಪಿದೆ. ಬಹುಶಃ ಆ ಶಾಲು ಇನ್ನೂ ಸುವ್ಯವಸ್ಥಿತವಾಗಿ ಅವನಲ್ಲಿ ಇದ್ದಿರಬೇಕು. ಅಂದರೆ ಯಾವ ಪದಾರ್ಥ ಕೊಟ್ಟಿದ್ದು ಎಂಬುದು ಅದರ ಬೆಲೆಗೆ ಮಾನದಂಡವಲ್ಲ. ಯಾರು ಯಾವ ಭಾವದಿಂದ ಕೊಟ್ಟಿದ್ದು ಎಂಬುದೇ ಮಾನ್ಯತೆಗೆ ಮಾನದಂಡ. ಹಾಗಾಗಿ ಶಿಷ್ಯನಿಗೆ ಗುರು ಕೊಟ್ಟ ಪದಾರ್ಥಕ್ಕೆ ದಕ್ಕುವ ಕಿಮ್ಮತ್ತು ಉಳಿದವರು ಕೊಟ್ಟರೆ ಸಿಗುವುದು ಕಷ್ಟ. 

ಗುರುವಿನ ತೃಪ್ತಿಯ ಮಹತ್ತ್ವವನ್ನು ಶಾಸ್ತ್ರ ಹೀಗೆ ಹೇಳುತ್ತದೆ \enginline{-} 
\begin{verse}
ಗುರೌ ತುಷ್ಟೇ ತು ತುಷ್ಟಾಸ್ಸ್ಯುಃ ಸರ್ವೇ ದೇವಾಃ ಸವಾಸವಾಃ | 
ಗುರೌ ರುಷ್ಟೇ ತು ರುಷ್ಟಾಸ್ಸ್ಯುಃ ಸರ್ವೇ ದೇವಾಃ ಸವಾಸವಾಃ ||
\end{verse}
ಶಿಷ್ಯನ ಜೀವನವೇ ಗುರುವಿನ ತುಷ್ಟಿ, ರುಷ್ಟಿಗಳನ್ನವಲಂಬಿಸಿದೆ. “ಗುರುವು ಸಂತುಷ್ಟನಾದರೆ ದೇವತೆಗಳೆಲ್ಲ ಸಂತುಷ್ಟರಾಗುತ್ತಾರೆ, ಗುರುವು ಅಸಮಾಧಾನ ಹೊಂದುವಂತಾದರೆ ದೇವತೆಗಳೂ ಹಾಗೆಯೇ” ಎನ್ನುತ್ತದೆ ಮೇಲಿನ ಶ್ಲೋಕ. ಇದು ಆತ್ಮಜ್ಞಾನವನ್ನು ಅನುಗ್ರಹಿಸುವ ಗುರುವಿಗೆ ನೇರವಾಗಿ ಅನ್ವಯಿಸುವ ವಿಷಯವಾದರೂ ಗುರು ಶಿಷ್ಯ ಪರಂಪರೆ ಅಂತಹ ಗುರುವಿನಿನಿಂದಲೇ ಆರಂಭವಾಗಿ ಅದರ ಅಂಶವೇ ಪರಂಪರೆಯಲ್ಲಿ ಹರಿದುಬರುವುದರಿದ  ಮೂಲವನ್ನು ಜ್ಞಾಪಿಸಿಕೊಳ್ಳುವಲ್ಲಿ ಅನೌಚಿತ್ಯವಿಲ್ಲ. ಅದನ್ನು ಜ್ಞಾಪಿಸಿಕೊಳ್ಳುತ್ತಲೇ ಸಾಗಿದಾಗ ಮಾತ್ರ ಪರಂಪರೆ ಎಂಬುದಕ್ಕೆ ಅರ್ಥ, ಇಲ್ಲದಿದ್ದರೆ ಅದು ವ್ಯರ್ಥ,  ಪ್ರಕೃತ ಶಾಲು ಕೊಟ್ಟ ವಿಷಯ ಹೇಳುವುದೇ ಉದ್ದೇಶವಲ್ಲ. ಅದರ ಹಿಂದಿರುವ ಭಾವ ಮತ್ತು ಸಂಬಂಧ ಎಷ್ಟು ಗಂಭೀರವಾದದು, ವಾತ್ಸಲ್ಯ ಭರಿತವಾದುದು, ಭಕ್ತಿಪೂರ್ವಕವಾದುದು, ಆದರ್ಶಪ್ರಾಯವಾದುದು ಎಂಬುದು ಗಮನಿಸಬೇಕಾದ ಅಂಶ. ಎಲ್ಲರಿಗೂ ಇಂತಹ ಸನ್ನಿವೇಶಗಳು ಜೀವನದಲ್ಲಿ ಒದಗಲಾರವು. ಇಂತಹ ರಸ, ಭಾವಗಳು ಜೀವಿಗಳಿಗೆ ಒಂದು ಬಗೆಯ ಹಿತವಾದ ಅಂತಃಸುಖವನ್ನು ನೀಡಿ  ಜೀವನದಲ್ಲಿ ಒಂದು ಪಾಕವನ್ನು ಉಂಟುಮಾಡುತ್ತವೆ ಎಂಬುದು ಸುಳ್ಳಲ್ಲ. ಅಂಥವು ಅವಶ್ಯ ಪ್ರಾಪ್ತವಾಗಬೇಕಾದವುಗಳೇ ಆಗಿವೆ. ಗಂಗಣ್ಣನ ಅಥವಾ ನನ್ನ ಜೀವನ ಗುರುವಿನ ಪ್ರಸಾದ \enginline{-} ಪ್ರಸನ್ನತೆ (\textit{ಪ್ರಸಾದಸ್ತು ಪ್ರಸನ್ನತಾ \enginline{-} ಅಮರಕೋಶ}) ಯಿಂದಲೇ ನಿರೂಪಿತವಾಗಿರುವುದು ಸ್ಪಷ್ಟ. ಹಾಗಿರುವುದರಿಂದ ಈ ಲೇಖನಕ್ಕೆ ಅದೇ ಹಿನ್ನೆಲೆ. ಅದಿಲ್ಲದಿದ್ದರೆ ಇಲ್ಲಿ ಹೇಳಬೇಕಾದ ಯಾವ ವಿಷಯವೂ ಇಲ್ಲ. ಅದಕ್ಕಾಗಿಯೇ ಲೇಖನದ ತಲೆಬರಹವೂ ಗುರುಪ್ರಸಾದ ಎಂಬುದಾಗಿದೆಯೇ ವಿನಾ ನನ್ನ ಹೆಸರಿನ ಅಭಿಮಾನದಿಂದಲ್ಲ. ಗುರುಗಳ ಪ್ರಸಾದ ಗಂಗಣ್ಣನಿಗುಂಟು. ಅವರಿಬ್ಬರ ಪ್ರಸಾದವು ನನ್ನ ಮೇಲುಂಟು. ಹಾಗಾಗಿ ಗುರು\enginline{-}ಪ್ರಸಾದವೇ ಈ ಲೇಖನದ ತಿರುಳು. 

ಗಂಗಣ್ಣನ ಸಹಾಯ ಸ್ವಭಾವಕ್ಕೆ ವ್ಯಕ್ತಿಗತ ಸೀಮೆಯಿರಲಿಲ್ಲ. ಪಾಠಶಾಲೆಯಲ್ಲಿ ಇನ್ನೊಬ್ಬ ಅಧ್ಯಾಪಕರಾಗಿದ್ದ ಶ್ರೀ ವೆಂಕಣ್ಣಾಚಾರ್ಯರು ಅವರ ಸಂಪೂರ್ಣ ವ್ಯವಹಾರಕ್ಕೆ ಗಂಗಣ್ಣನನ್ನೇ ಅವಲಂಬಿಸಿದ್ದರು. ಅವರ ಮಕ್ಕಳಿಗಿಂತ ಗಂಗಣ್ಣನಲ್ಲೇ ಅವರಿಗೆ ಹೆಚ್ಚು ವಿಶ್ವಾಸವಿತ್ತು. ಶ್ರೀವಿಶ್ವೇಶ್ವರ ದೀಕ್ಷಿತರಿಗೂ ಅಲ್ಲದೆ ಇನ್ನೂ ಅನೇಕ ಅಧ್ಯಾಪಕರುಗಳಿಗೂ ಅವನ ಸೇವಾರೂಪದ ಸಹಾಯ ಹಸ್ತ ಇದ್ದೇ ಇತ್ತು.  ಹೀಗೆ ಗಂಗಣ್ಣ ಅಧ್ಯಯನ ಅಧ್ಯಾಪನ, ಸೇವೆ, ಸಹಾಯ, ವ್ಯವಹಾರ ಯಾವುದರಲ್ಲೂ ಮಾದರಿಯಾಗಿ ನಿಲ್ಲುವ ಗುಣಗಳುಳ್ಳವನೆಂಬುದರಲ್ಲಿ ಅನುಮಾನವಿಲ್ಲ, ಆ ಗುಣಗಳು ಸ್ವಾರ್ಥದ ಲವಲೇಶವೂ ಇಲ್ಲದೆ ಪರಿಶುದ್ಧವಾಗಿ ವ್ಯಾಪಾರ ಮಾಡುವಂಥವುಗಳು ಎಂಬುದು ಬಹಳ ಮುಖ್ಯವಾಗಿ ಗಮನಿಸಬೇಕಾದು. ಯಾಕೆಂದರೆ ಸರ್ವಃ ಸ್ವಾರ್ಥಂ ಸಮೀಹತೇ. ಎಲ್ಲೆಲ್ಲೂ ಇದೇ ಕಾಣುತ್ತದೆ. ಇಲ್ಲಿ ಅದರ ಲವಲೇಶವೂ ಇಲ್ಲ.  ಇಂಥ ಸ್ಥಾನದಲ್ಲಿ ಕಾಳಿದಾಸನ ಮಾತು \enginline{-} ಗುಣಾ ಗುಣಾನುಬಂಧಿತ್ವಾತ್ ತಸ್ಯ ಸಪ್ರಸವಾ ಇವ ಎಂಬುದು ತಲೆಯಲ್ಲಿ ಹಾದು ಹೋಗದೇ ಇರದು. ಈ ಯೋಗ್ಯತೆಗಳು ಗಂಗಣ್ಣನ ಜಾತಕದಲ್ಲೂ ಸ್ಪಷ್ಟವಾಗಿ ಗುರುತಿಸುವಂತಿವೆ. ಪಂಚಮಸ್ಥಾನದಲ್ಲಿ ಬುದ್ಧಿಕಾರಕನಾದ ಬುಧ ಯಾವ ಪಾಪಗ್ರಹಗಳ ಯೋಗ ದೃಷ್ಟಿಗಳಾವುದೂ ಇಲ್ಲದೇ ಪರಿಶುದ್ಧನಾಗಿ ನಿಂತಿದ್ದಾನೆ. ಇಡೀ ಜಾತಕದಲ್ಲಿ ಬುಧನದೇ ಸಾಮ್ರಾಜ್ಯ. ಬುಧ ಸ್ವತಃ ಶುಭಗ್ರಹ, ಆದರೆ ಪಾಪಗ್ರಹಗಳ ಜೊತೆಯಲ್ಲಿದ್ದರೆ ಅವನೂ ಪಾಪನಾಗಿಬಿಡುತ್ತಾನೆ. ಆದರೆ ಇಲ್ಲಿರುವ ಬುಧ ಶುದ್ಧನಾಗಿದ್ದಾನೆ. ವಾತಪಿತ್ತಶ್ಲೇಷ್ಮಗಳೆಂಬ ಮೂರೂ ಧಾತುಗಳ ಸಮಾನ ಹದವುಳ್ಳ ಪ್ರಕೃತಿ, ಯುಕ್ತಿಯುಕ್ತ ಮತ್ತು ಶ್ಲೇಷಯುಕ್ತ ಮಾತು. ತಿಳಿ ಹಾಸ್ಯ, ಪಾಂಡಿತ್ಯ, ಕಲಾ ನೈಪುಣ್ಯ, ಸ್ನೇಹ ಇವೆಲ್ಲ ಬುಧ ಸೂಚಿಸುವ ಗುಣಗಳು. ಅಂತಹ ಬುಧನ ಸ್ಥಿತಿ ಗಂಗಣ್ಣನ ವ್ಯಕ್ತಿತ್ವವನ್ನು ಸ್ಪಷ್ಟವಾಗಿ ನಿರೂಪಿಸುತ್ತಿದೆ. ಮೇಲೆ ಹೇಳಿದ ಎಲ್ಲ ಅಂಶಗಳೂ ಅವನಲ್ಲಿ ಪ್ರಭೂತವಾಗಿ ಗೋಚರಿಸುತ್ತವೆ. ಅವನ ಕಲಾಭಿವ್ಯಕ್ತಿ ಇತ್ತೀಚಿನ ಜನರಿಗೆ ಪರಿಚಯವಿದೆಯೋ ಇಲ್ಲವೋ ಗೊತ್ತಿಲ್ಲ. ಹಿಂದೆ ಕೆಲವು ಕಾವ್ಯ, ನಾಟಕಗಳು ಗಂಗಣ್ಣನ ಪಾತ್ರ ಮತ್ತು ನಿರ್ದೇಶನದಲ್ಲಿ ರಂಗದ ಮೇಲೆ ಪ್ರಯೋಗವಾಗಿವೆ. ನಾನೂ ಸಹ ಒಂದು ನಾಟಕ ಮತ್ತು ತಾಳಮದ್ದೆಲೆಯಂತಹ ಒಂದು ಕಾರ್ಯಕ್ರಮದಲ್ಲಿ ಗಂಗಣ್ಣನ ಪಾತ್ರ ಮತ್ತು ಸೂತ್ರಧಾರಿಕೆಯಲ್ಲಿ ಭಾಗವಹಿಸಿದ್ದೆ. ಆ ರಂಗದ ಮೇಲೆ ನನ್ನ ಪಾಲಿಗೆ ಕೃಷ್ಣ ಒಲಿದಿದ್ದ. ಅದನ್ನು ನಾವೆಲ್ಲ ಅವನ ನಿರ್ದೇಶನದಲ್ಲಿ ಚೆನ್ನಾಗಿ ನಿರ್ವಹಿಸಿದ್ದೆವು. ಗಂಗಣ್ಣ ಯಕ್ಷಗಾನದ ತಾಳಮದ್ದಲೆಯಲ್ಲೂ ಭಾಗವಹಿಸುತ್ತಿದ್ದ. ಅವನ ಪ್ರತಿಭೆ ಅಲ್ಲಿ ಚೆನ್ನಾಗಿ ಪ್ರತಿಫಲಿಸುತ್ತಿತ್ತು. ಒಮ್ಮೆ ದುರ್ಯೋಧನನ ಪಾತ್ರ ಮಾಡಿದಾಗ ಕೃಷ್ಣನಲ್ಲಿ ಪ್ರಶ್ನೆಯೊಂದನ್ನು ಕೇಳಿದ್ದ. ಅದಕ್ಕೆ ಕೃಷ್ಣನ ಪಾತ್ರಧಾರಿಗಳಿಗೆ ಉತ್ತರಕೊಡಲಾಗಲಿಲ್ಲ. ಅನತಿಕಾಲದ ಇನ್ನೊಂದು ಪ್ರಸಂಗದಲ್ಲಿ ಪಾತ್ರಗಳು ಬದಲಾಗಿ ಕೃಷ್ಣ ಪಾತ್ರ ಮಾಡಿದವರು ದುರ್ಯೋಧನನ ಪಾತ್ರವನ್ನು ಮಾಡಿದ್ದರು. ಆಗ ಗಂಗಣ್ಣ ಕೃಷ್ಣನಾಗಿದ್ದ. ಹಿಂದೆ ಕೃಷ್ಣನಾಗಿ ಉತ್ತರ ಕೊಡಲಾಗದ ಪ್ರಶ್ನೆಯನ್ನೇ ಈಗ ದುರ್ಯೋಧನ ಪಾತ್ರಧಾರಿಗಳು ಕೃಷ್ಣನೆಡಗೆ ಎಸೆದರು. ಆದರೆ ಅದಕ್ಕೆ ತಕ್ಕ ಉತ್ತರ ಕಷ್ಣನ ಬತ್ತಳಿಕೆಯಲ್ಲಿತ್ತು. ಹೀಗೆ ಉತ್ತರೋತ್ತರ ಯುಕ್ತಿಯಲ್ಲಿ ಅಸಾಧಾರಣ ಸಾಮರ್ಥ್ಯ ಗಂಗಣ್ನನಿಗಿತ್ತು. ಬಹುಶಃ ಅವನು ಭಾಗವಹಿಸಿದ ಕೊನೆಯ ತಾಳಮದ್ದಲೇ ಅದೇ ಇರಬೇಕು. ಈಗ ವಿದ್ಯಾರ್ಥಿಗಳ ಕಡೆಯಿಂದ ಸಂಪನ್ನವಾದ ಅಭಿವಂದನ ಸಮಾರಂಭದಲ್ಲಿ ಅವನ ವಾಕ್ಕಿಗೆ ನಿದರ್ಶನವಾಗುವ ಅಂತಹದ್ದೇ ಒಂದು ಪ್ರಸಂಗ ನಡೆದಿದೆ. ಎಷ್ಟು ಜನರು ಅದನ್ನು ಗಮನಿಸಿದರೋ ಗೊತ್ತಿಲ್ಲ. ಆ ಸಂದರ್ಭದಲ್ಲಿ  ಶ್ರೀಯುತ ಉಮಾಕಾಂತ ಭಟ್ಟರು ಮಾತನಾಡುವಾಗ ಗಂಗಣ್ಣ ಮತ್ತು ತಮ್ಮ ಬಾಲ್ಯದಿಂದ ಈಗಿನ ತನಕ ಅವಿವಾದಿತ ಸ್ಪರ್ಧೆಯಿರುವುದನ್ನು ಜ್ಞಾಪಿಸಿಕೊಳ್ಳುತ್ತಾ , ಅವರು “ಸ್ಪರ್ಧೆಗಳಲ್ಲಿ ಗಂಗಾದರ ಪ್ರಥಮ ಬಹುಮಾನವನ್ನ ನನಗೆ ಪಡೆಯಲು ಬಿಡುತ್ತಿರಲಿಲ್ಲ. \enginline{-} ಈಗಲೂ ನನಗೆ ವಿಶ್ವಾಸವಿದೆ, ಈಗಲೂ ಗಂಗಾಧರ ನನ್ನ ಮಾತು ಮುಗಿದ ಮೇಲೆ, ನನ್ನ ಮಾತು ಮರೆಯುವ ಹಾಗೆ ಮಾತನಾಡುತ್ತಾನೆ” ಎಂದು ಜಾಣ್ಮೆಯ ಅಷ್ಟೇ ಚೋದ್ಯವಾದ ಮಾತನ್ನಾಡಿದರು. ಆದರೆ ಗಂಗಣ್ಣ ತಾನು ಮಾತನಾಡುವಾಗ, “ಉಮಾಕಾಂತ ಹೇಳಿದ್ದಾನೆ, ತನ್ನ ಮಾತು ಮರೆಯುವಂತೆ ನಾನು ಮಾತನಾಡುತ್ತೇನೆ ಎಂದು, ಆದರೆ ಇಂದು ನಾನು ಎಲ್ಲವನ್ನೂ ಮರೆತಿದ್ದೇನೆ” ಎನ್ನುತ್ತಾನೆ. ಇದು ಆ ಸಂದರ್ಭಕ್ಕೆ ಸಭೆಯ ಮತ್ತು ಅವನ, ಅಲ್ಲದೇ ಆ ಕಾರ್ಯಕ್ರಮದ ಭಾವಕೇಂದ್ರವನ್ನು ಅಭಿವ್ಯಂಜಿಸುವ ಬಹ್ವರ್ಥಗರ್ಭಿತ ಧ್ವನಿಪೂರ್ಣ ಮಾತಾಗಿತ್ತು. ಆಗ ಸ್ಪರ್ಧೆಗಾಗಿ ಮಾತನಾಡುವ ಭಾವ ಅಲ್ಲಿರಲಿಲ್ಲ. ಭಾವುಕವಾದ ಸ್ಥಿತಿಯಿತ್ತು. ಭಾವುಕ ಸ್ಥಿತಿ ಭಾವುಕನನ್ನು ಮೌನದೆಡೆಗೇ ಸೆಳೆಯುತ್ತದೆ. ಅದು ಅದರ ಸ್ವಭಾವ. ಅವನ ಮಾತು ಅದನ್ನೇ ಸ್ವಾರಸ್ಯವಾಗಿ ಹೇಳುತ್ತಿತ್ತು. ಹೀಗೆ ಅವನ ಒಡನಾಟದಲ್ಲಿ ನಮಗೆ ಇಂತಹ ಸನ್ನಿವೇಶ ಅಪರೂಪವಾಗಿರಲಿಲ್ಲ. 

ಸ್ವಭಾವವಾಗಿ ಗಂಗಣ್ಣನಲ್ಲಿ ಕರುಣ ರಸ ಮತ್ತು ವೀರ ರಸಗಳು ಬಹಳ ಚೆನ್ನಾಗಿ ಅಭಿವ್ಯಕ್ತವಾಗುತ್ತವೆ. ಹಾಸ್ಯ ರಸ ತಾನೇ ತಾನಾಗಿ ಏರ್ಪಡುತ್ತದೆ. ಶೃಂಗಾರ ಭಾವ ಬಹಳ ಸುಪ್ತವಾಗಿದೆ. ಗಂಭೀರ ಸನ್ನಿವೇಶವಿರುವಾಗ ಅನೇಕರಲ್ಲಿ ಹಾಸ್ಯ ಶೃಂಗಾರಗಳು ಮಿಶ್ರಣವಾಗಿ ಎರಡರ ಹದವೂ ತಪ್ಪಿ ಕ್ಷಣಾರ್ಧದಲ್ಲಿ ವ್ಯಕ್ತಿಯನ್ನೂ ಸನ್ನಿವೇಶವನ್ನೂ ಪ್ರಪಾತಕ್ಕೆ ತಳ್ಳಿಬಿಡುವುದನ್ನು  ಸಭೆ ಸಮಾರಂಭಗಳಲ್ಲಿ ನಾವು ನೋಡುತ್ತೇವೆ. ಆದರೆ ಗಂಗಣ್ಣನಲ್ಲಿ ಯಾವ ರಸವೂ ವಿರಸವಾಗದೇ ಸರಿಯಾದ ಹದವನ್ನು ಕಾಯ್ದುಕೊಂಡಿವೆ, ಅದರಲ್ಲೂ ಶೃಂಗಾರವಂತೂ ಅತ್ಯಂತ ಸುಪ್ತವಾಗಿದೆ‘. ಷಷ್ಠದ ಶುಕ್ರ ಅದನ್ನು ಹಾಗೆ ಹದಗೊಳಿಸಿದ್ದಾನೆ.  ಅದೆಲ್ಲೂ ಅಷ್ಟಾಗಿ ಪ್ರಕಟವಾಗಿದ್ದನ್ನು ನಾನು ಕಂಡಿಲ್ಲ. ಹಾಗೆಂದು ಅದು ಇರಬಾರದ್ದೆಂದು ಲೇಖಕನ ಅಭಿಪ್ರಾಯವಲ್ಲ. ರಸರಾಜ ಎಂಬ ಕೀರ್ತಿ ಅದಕ್ಕಿದೆ. ಆದರೆ ಅದು ರಾಜನಂತೆ ಗಂಭೀರವಾಗಿರಬೇಕಷ್ಟೆ ! ಇಲ್ಲದಿದ್ದರೆ ಅಂತಹ ರಾಜ ರಾಜ್ಯ ಕಳೆದುಕೊಳ್ಳುವ ಅಪಾಯವಿರುತ್ತದೆ. ಇನ್ನು, ಗಂಗಣ್ಣನಲ್ಲಿರುವ ಅನಾಲಸ್ಯ, ಉತ್ಸಾಹ, ಧೈರ್ಯ, ಮತ್ತು ಔದಾರ್ಯವನ್ನು ಸುಸ್ಥನಾದ ಕುಜ ಕೊಟ್ಟಿದ್ದಾನೆ. ಧೈರ್ಯಂ ಸರ್ವತ್ರ ಸಾಧನಂ ಎಂಬ ಮಾತು ಗಂಗಣ್ಣನಲ್ಲಿ ಸಾಕ್ಷಾತ್ತಾಗಿ ಕಾಣುವ ಗುಣ. ಅವನ ಕೋಶದಲ್ಲಿ ಅಧೈರ್ಯ ಎಂಬ ಪದಕ್ಕೆ ಸ್ಥಾನವೇ ಇಲ್ಲ. 

ಈ ಮೇಲಿನ ಎಲ್ಲ ಅಂಶಗಳಿಂದ ರೂಪುಗೊಂಡ ವ್ಯಕ್ತಿತ್ವವುಳ್ಳ ಗಂಗಣ್ಣನ ಅಯಾಚಿತ ಕೀರ್ತಿ ಇತರ ಅನತಿ ದೂರದವರೆಗೂ ವ್ಯಾಪಿಸಿದೆ. ಒಮ್ಮೆ ನಾನು ಆಂಧ್ರದ ತೆನಾಲಿಗೆ ಪರೀಕ್ಷೆಗೆ ಹೋದಾಗ ಅಲ್ಲಿಯ ವಿದ್ವಾಂಸರೊಬ್ಬರು, “ನೀನು ಯಾರ ವಿದ್ಯಾರ್ಥಿ ?” ಎಂದು ಕೇಳಿದಾಗ ನಾನು ಗಂಗಾಧರ ಭಟ್ಟರು ಎಂದೆ. “ತೇ ತು ನಿತರಾಂ ತಾರ್ಕಿಕಾ ಭವಂತಿ” ಎಂದು ಅವರು ಉದ್ಗರಿಸಿದ್ದರು. ಅಲ್ಲಿಯವರಿಗೂ ಅವನ ಪರಿಚಯವಿತ್ತು. ಈಗ ಅದು ವಿದೇಶದ ವರೆಗೂ ವ್ಯಾಪಿಸಿರುವುದು ಗೊತ್ತೇ ಇದೆ.  ಇನ್ನೊಂದು ತಮಾಷೆಯಾದ ಸಣ್ಣ ಘಟನೆ ಸ್ಮರಣೆಯಲ್ಲದೆ. ಒಮ್ಮೆ ಮೈಸೂರಿನಲ್ಲಿರುವ ಮನೆಗ ಯವುದೋ ಊರಿನಿಂದ ಪತ್ರವೊಂದು ಬಂದಿತ್ತು. ಅದರಲ್ಲಿ ಅಡ್ರೆಸ್ \enginline{-} “ಗಂಗಾಧರ ಭಟ್, ಮೈಸೂರು” ಎಂಬುದನ್ನು ಬಿಟ್ಟು ಮತ್ತೇನೂ ಇರಲಿಲ್ಲ.  ಅಡ್ರೆಸ್ ವ್ಯವಸ್ಥಿತವಾಗಿದ್ದರೇ ಮನಗೆ ಪತ್ರ ಬರುವ ಭರವಸೆ ಇಲ್ಲ. ಹಾಗಿದ್ದೂ ಸಮಯಕ್ಕೆ ಸರಿಯಾಗಿ ಅದು ಮನೆಗೆ ಬಂದಿತ್ತು. ಗಂಗಣ್ಣ ಎಷ್ಟು ಚಿರಪರಿಚಿತ ಎಂದು ನಾವೆಲ್ಲ ತಮಾಶೆ ಮಾಡಿ ನಕ್ಕಿದ್ದುಂಟು. 

ಹೀಗೆ ಘಟನೆಗಳನ್ನು ಹೇಳುತ್ತಾ ಹೋದರೆ ಹೊತ್ತಿದ್ದರೆ ಹೊತ್ತಗೆಯನ್ನೇ ಬರೆಯಬಹುದು. ಅಷ್ಟೊಂದು ವಿಮರ್ಶನೀಯ ವಿಷಯಗಳಿವೆ. ಅವನು ಒಂದು ಕಾದಂಬರಿಗೆ ವಸ್ತು ಎಂದು ಶ್ರೀ ಉಮಾಕಾಂತ ಭಟ್ಟರು ಹೇಳುತ್ತಿರುತ್ತಾರೆ. ಗಂಗಣ್ಣನ ಸಾಮಾಜಿಕ, ಮತ್ತು ಪಾಠಶಾಲೆಗೆ ಸಂಬಂಧಿಸಿದ, ಎಲ್ಲ ಕಡೆಯಿಂದ ಬರುವ ಸಾವಿರಾರು ವಿದ್ಯಾರ್ಥಿಗಳಿಗೆ ವಾಸ\enginline{-}ಅಶನ\enginline{-}ವಸನ ಇತ್ಯಾದಿ ಸಹಾಯಮಾಡಿದ ವಿಷಯಗಳೆಲ್ಲ ವಿವಿಧ ಲೇಖನದಲ್ಲಿ ಅಭಿವ್ಯಕ್ತವಾಗಿವೆ. ಅವನ ವ್ಯಕ್ತಿತ್ವದ ವ್ಯಾಪ್ತಿ ಅಷ್ಟೇ ವ್ಯಾಪ್ತವಾದುದು. ಆದರೆ ದೇಶ, ಕಾಲಗಳ ಮಿತಿ ನಮ್ಮನ್ನು ನಿರ್ಬಂಧಿಸುತ್ತದೆ. 

ಇಂತಹ ಒಂದು ವ್ಯಕ್ತಿತ್ವಕ್ಕೆ ಒಂದು ಅಭಿವಂದನ ಕಾರ್ಯಕ್ರಮ ನಡೆದುದು ಅತ್ಯಂತ ತೃಪ್ತಿಯನ್ನು ಉಂಟುಮಾಡುವ ವಿಷಯ. ಈ ಕಾರ್ಯಕ್ರಮದ ನೆಪದಲ್ಲಿ ಉಳಿದ ವಿದ್ಯಾರ್ಥಿಗಳೆಲ್ಲ ದೂರದಲ್ಲಿದ್ದುದರಿಂದ ಸಾಕಷ್ಟು ಜವಾಬ್ದಾರಿಯನ್ನು ಅಯಾಚಿತವಾಗಿ ನಾನು ನಿರ್ವಹಿಸಬೇಕಾಯಿತು. ಅದೆಲ್ಲಕ್ಕಿಂತ ಕಾರ್ಯಕ್ರಮ ಹೀಗೆಯೇ ಆಗಬೇಕೆಂಬುದು ನನಗಿತ್ತು. ಗಂಗಣ್ಣ ಮತ್ತು ಶೈಲಜಕ್ಕನೊಡನಿದ್ದ ಸಲುಗೆ ವಿಶ್ವಾಸಗಳಿಂದ ಅದರ ನಿರ್ವಾಹ ಸುಲಭವಾಯಿತು. ಕಾರ್ಯಕ್ರಮ ಸಾಧಿಸಿದ ಯಶಸ್ಸಿನಲ್ಲಿ ಪೂರ್ಣ ತೃಪ್ತಿಯಿಲ್ಲದಿದ್ದರೂ  ಅತೃಪ್ತಿಯಿಲ್ಲ. ತೃಪ್ತಿ\enginline{-}ಅತೃಪ್ತಿಗಳೇನಿದ್ದರೂ ಸಾಪೇಕ್ಷವಷ್ಟೆ. ಅಷ್ಟಾದರೂ ನನ್ನ ಪಾಲಿಗೆ ಒದಗಿದ್ದು ಅದರಿಂದ ಕೊಂಚ ಭಾರ ಇಳಿದದ್ದು ಸಮಾಧಾನದ ವಿಷಯ. ಅದಲ್ಲದೇ ಈ ಗ್ರಂಥದ ಸಂಪಾದಕನಾಗಿ ಕೆಲಸಮಾಡಿದ್ದು ನನ್ನ ಭಾಗ್ಯ. ಈ ನೆಪದಲ್ಲಿ ನನಗೆ ಉತ್ತಮವಾದ ಸಾಹಿತ್ಯ ಕೃಷಿಯಾಯಿತು. ಇದಲ್ಲದೇ ಶ್ರೀ ಉಮಾಕಾಂತ ಭಟ್ಟರು ರಚಿಸಿಕೊಟ್ಟ  ಸಾಹಿತ್ಯದ ಆಧಾರದ ಮೇಲೆ ನನ್ನದನ್ನೂ ಸೇರಿಸಿ ಸಿದ್ಧಪಡಿಸಿದ ಅಭಿವಂದನ ಪತ್ರ ನನ್ನ ಹೃದಯವನ್ನು ಸಾಹಿತ್ಯದ ಮೂಲಕ ಅಭಿವ್ಯಕ್ತಪಡಿಸುವುದಕ್ಕೆ ಆಧಾರವಾಯಿತು. ಕಾರ್ಯಕ್ರಮದಲ್ಲೂ ಅದರ ವಾಚನದ ಅವಕಾಶವೊದಗಿತು. ಅದು ಗಂಗಣ್ಣನ ವ್ಯಕ್ತಿತ್ವಕ್ಕೆ ತಕ್ಕ ಗಂಭೀರ್ಯವುಳ್ಳ ಸಾಹಿತ್ಯವಾಗಿ ರೂಪುಗೊಂಡಿದೆ ಎಂದು ನನಗನ್ನಿಸಿದೆ. ಸಮಿತಿ  ನನಗೆ ಕೊಟ್ಟ ಅವಕಾಶಕ್ಕೆ ಅದರ ಸದಸ್ಯರುಗಳಿಗೆಲ್ಲ ನನ್ನ ಅನಂತ ಕೃತಜ್ಞತೆಗಳು. 

ಗ್ರಂಥದ ಕೆಲಸ ನಾನಾ ಕಾರಣಗಳಿಂದ ಸಾಕಷ್ಟು ವಿಲಂಬವಾಯಿತು. ಅದಕ್ಕೆ ISBN ಲಭ್ಯವಾಗುವಲ್ಲಿ ನಿಧಾನವಾದು ಪ್ರಧಾನಕಾರಣ. ಈ ಮಧ್ಯೆ ನನ್ನ ಅತ್ತೆ ಅನಿರೀಕ್ಷಿತವಾಗಿ ಕ್ಯಾನ್ಸರ್ ರೋಗಕ್ಕೆ ತುತ್ತಾಗಿ ಅವರನ್ನು ಸುಮಾರು ಒಂದೂವರೆ ತಿಂಗಳಷ್ಟು ಕಾಲ ಮಂಗಳೂರಿನಲ್ಲಿಟ್ಟುಕೊಂಡು ನೋಡಿಕೊಳ್ಳಬೇಕಾಯಿತು. ಕೊನೆಗೆ ಅವರು ಬದುಕಲಿಲ್ಲ. ಈ ಸಂಬಂಧದ ಓಡಾಟ ನನಗೆ ISBN ಇತ್ಯಾದಿ ಗ್ರಂಥಕ್ಕೆ ಸಂಬಂಧಿಸಿದ ವ್ಯವಹಾರವನ್ನು ವೇಗವಾಗಿ ಫಾಲೋ ಮಾಡುವುದು ಕಷ್ಟವಾಯಿತು. ಇದಲ್ಲದೇ ನನ್ನನ್ನು ಬಿಟ್ಟು ಉಳಿದೆಲ್ಲ ವಿದ್ಯಾರ್ಥಿಗಳು ಮೈಸೂರಿನಲ್ಲಿಲ್ಲದೇ ವಿಭಿನ್ನ ಊರುಗಳಲ್ಲಿ ಔದ್ಯೋಗಿಕ ಒತ್ತಡದಲ್ಲಿದ್ದು ಗ್ರಂಥದ ಪ್ರೂಪ್ ಕರೆಕ್ಷನ್ ಇತ್ಯಾದಿಗಳನ್ನು ಆನ್ ಲೈನ್ ನಲ್ಲೇ ಮಡಬೇಕಾಗಿ ಬಂದುದು ಮತ್ತೊಂದು ಕಾರಣ. ಅಷ್ಟಕ್ಕೂ ನನ್ನ ಗತಿಯೂ ವಿಲಂಬಿತವೇ ವಿನಾ ದೃತವಲ್ಲ, ಇವೆಲ್ಲವೂ ಇಲ್ಲಿ ಉಲ್ಲೇಖನೀಯ ವಿಷಯವಲ್ಲ. ಯಾಕೆಂದರೆ ಲೇಖಕರಿಂದ ಲೇಖನಗಳನ್ನು ಪಡೆದು ನಿರ್ವಹಿಸುವ ಗ್ರಂಥ ಸಂಪಾದನೆಯ ಕಾರ್ಯದ ಸಮಸ್ಯೆ ಆ ಕ್ಷೇತ್ರದಲ್ಲಿರುವವರಿಗ ಗೊತ್ತಿಲ್ಲದಿಲ್ಲ. ಆದರೂ ಕೆಲವೊಮ್ಮೆ ನಿರ್ದಿಷ್ಟವಾಗಿ ತಿಳಿಸಿದ ಹೊರತು ಕಾರಣ ಸ್ಪಷ್ಟವಾಗುವುದಿಲ್ಲ. ವಿನಾಕಾರಣ ಆಗ್ರಹಕ್ಕೆ ಆಸ್ಪದವಾಗುವ ಸಂಭವ. ಅದಕ್ಕೇ ತಾನೆ, ಆಹಾರೇ ವ್ಯವಹಾರೇ ಚ ತ್ಯಕ್ತಲಜ್ಜಃ ಸುಖೀ ಭವೇತ್ ಎಂದ್ದಿದ್ದು. ಇನ್ನು, ಏನೆಲ್ಲ ಮಾಡಿದರೂ ಗ್ರಂಥದಲ್ಲಿ ದೋಷಗಳಿಲ್ಲದಿಲ್ಲ. ಸಾಕಷ್ಟೇ ಇರಬಹುದು. ಅದನ್ನು ಮನ್ನಿಸಬೇಕೆಂದು ವಿನಂತಿಸುತ್ತೇನೆ. ನನಗೆ ಈ ಕಾರ್ಯಕ್ಕೆ ಸಹಕರಿಸಿದ ಎಲ್ಲ ಸಮಿತಿ ಸದಸ್ಯರಿಗೂ ಅನಂತ ಧನ್ಯವಾದಗಳು.

ನನ್ನ ಬಗೆಗೆ ಗಂಗಣ್ಣ, ಶೈಲಜಕ್ಕನವರ ಪ್ರಸಾದಭಾವದ ಕಾರಣ ಇಷ್ಟನ್ನಾದರೂ ನಾನು ನಿರ್ವಹಿಸುವುದು. ಸಾಧ್ಯವಾಯಿತು,

ಈ ಸೇವೆ ಆ ದಂಪತಿಗಳಿಗೆ ಅರ್ಪಿತ

\articleend
}
