\chapter{ಸಜ್ಜನಪ್ರಿಯ ಗಂಗಾಧರ ಭಟ್ಟರು}

\begin{center}
\Authorline{ಮಹಾಬಲೇಶ್ವರಶರ್ಮಾ}
\smallskip

ನಿವೃತ್ತ ಅಧ್ಯಾಪಕ,\\ 
ಸಾಗರ
\end{center}

ಸಜ್ಜನಪ್ರಿಯ ಗಂಗಾಧರ ಭಟ್ಟ ಅಗ್ಗೇರೆ ಇವರಿಗೆ ಅಭಿನಂದನೆಗಳು

ಶ್ರೀ ಗಂಗಾಧರ ಭಟ್ಟರು ಪುರೋಹಿತ ಮನೆತನದವರು. ಪ್ರಕೃತ  ಉತ್ತರಕನ್ನಡ ಜಿಲ್ಲೆಯ ಸಿದ್ಧಾಪುರ ತಾಲೂಕಿನ ಮಣ್ಣೀಕೊಪ್ಪ ನಿವಾಸಿಗಳು. ವಾಸ್ತವವಾಗಿ ಇವರು ಮೂಲತಃ ಇದೇ ತಾಲೂಕಿನ ಬೀಳಗಿ ಸಮೀಪದ ಇಟಗಿ ಗ್ರಾಮದವರು. ಅಷ್ಟೇ ಅಲ್ಲ, ಅಲ್ಲಿಯ ದೇವಸ್ಥಾನದ ಉಪಾಧಿವಂತರೂ ಹೌದು. ಹಾಗಾಗಿ ಇವರು ಪೂರ್ವಕಾಲದಿಂದಲೂ ಪುರೋಹಿತ ಮನೆತನದವರು. ಅಂತೆಯೇ ವಿದ್ವಾಂಸರ ಕುಟುಂಬ ಕೂಡ. ಗಂಗಾಧರ ಭಟ್ಟರ ದೊಡ್ಡಪ್ಪ ಶ್ರೀ ಮಹಾಬಲೇಶ್ವರ ಭಟ್ಟರು ಗೋಕರ್ಣಮಂಡಲದಲ್ಲಿಯೇ ಗೌರವಾನ್ವಿತರಾಗಿದ್ದು, ಮಂಡಲದ ವಿದ್ವಾಂಸರೂ ಸಹ ಧರ್ಮಶಾಸ್ತ್ರಾದಿಗಳಲ್ಲಿ, ಯಜ್ಞ-ಯಾಗಾದಿಗಳಲ್ಲಿ ಸಂದೇಹ ಬಂದರೆ ಪರಿಹಾರಕ್ಕಾಗಿ ಇವರಲ್ಲಿ ಬರುತ್ತಿದ್ದುದು ಸಾಮಾನ್ಯವಾಗಿತ್ತು. ಇಂತಹ ಕುಟುಂಬದ ಉತ್ತಮ ಸಂಸ್ಕಾರದ ಹಿನ್ನೆಲೆಯುಳ್ಳವರು ನಮ್ಮ ಗಂಗಾಧರ ಭಟ್ಟರು.

ಈ ಕುಟುಂಬದ ಪ್ರಧಾನ ಗುಣ ಪರೋಪಕಾರ. ಈ ವಿಷಯದಲ್ಲಿ ಕುಟುಂಬದ ಎಲ್ಲ ಪುರುಷ, ಸ್ತ್ರೀಯರಿಬ್ಬರ ಸ್ವಭಾವವೂ ಸಮಾನವಾಗಿದೆ. ಪರೋಪಕಾರಾರ್ಥಮಿದಂ ಶರೀರಮ್ ಎಂಬುದನ್ನು ಈ ಕುಟುಂಬ ಸಾಕ್ಷಾತ್ಕರಿಸಿದೆಯೆಂದರೆ ಏನೇನೂ ಅತಿಶಯವಲ್ಲ. ಇದನ್ನು ನಾವು ಗಂಗಾಧರ ಭಟ್ಟರು ಮತ್ತು ಅವರ ಶ್ರೀಮತಿಯವರನ್ನು ನೋಡಿಯೇ ಸುಲಭವಾಗಿ ಊಹಿಸಬಹುದು. ಗಂಗಾಧರ ಭಟ್ಟರ ಧರ್ಮಪತ್ನಿಯೂ ಭಟ್ಟರಿಗೆ ತಕ್ಕವಳೇ. ಮೈಸೂರಿಗೆ ಬಂದ ಬಹುತೇಕ ವಿದ್ಯಾರ್ಥಿಗಳಿಗೆ ಗಂಗಾಧರ ಭಟ್ಟರೇ ಅಪ್ಪ, ಅಮ್ಮ ಎಲ್ಲವೂ. ಅದೆಷ್ಟು ವಿದ್ಯಾರ್ಥಿಗಳು ಇವರಿಂದ ಅಧ್ಯಯನ, ಅಧ್ಯಾಪನ, ಭಾಷಣ, ಸ್ಪರ್ಧೆಗಳಲ್ಲಿ ಇವರಿಂದ ಉಪಕೃತರಾಗಿದ್ದಾರೋ ಗಣಿಸುವುದು ಕಷ್ಟ. ಆದರೆ ಈ ವಿಷಯ ಮಾತ್ರ ವಿಶೇಷವಾಗಿ ಪರಿಗಣಸಬೇಕಾದದ್ದು. ಅದರ ಪರಿಣಾಮ ಇಂದು ಅವರಿಗೆ ವಿದ್ಯಾರ್ಥಿಗಳು ಸಲ್ಲಿಸುತ್ತಿರುವ ಕಿರು ಕೃತಜ್ಞತೆ - ಅವರ ನೆನಪಿನಲ್ಲೊಂದು ಗ್ರಂಥ ಸಮರ್ಪಣೆ ಎಂದು ನಾನು ಭಾವಿಸುತ್ತೇನೆ.

ಶ್ರೀಯುತರು ಬಹುಮುಖ ಪ್ರತಿಭೆಯುಳ್ಳವರು. ಶಾಸ್ತ್ರಜ್ಞರು. ಸಜ್ಜನರು. ಕಲಾ ರಸಿಕರು. ಆಯುರ್ವೇದದಲ್ಲೂ ಆಸಕ್ತರು. ಆಯುರ್ವೇದ ವಿದ್ಯಾರ್ಥಿಗಳಿಗೆ ಸಂಸ್ಕೃತ ಮತ್ತು ಆಯುರ್ವೇದ ಗ್ರಂಥವನ್ನು ಪಾಠಮಾಡಿದವರು. ನಗರವಾಸಿಗಳಾದರೂ ಕಷಿತಜ್ಞರು. ಹೀಗೆ ಶ್ರೀಯುತರು ಬಹು ಕ್ಷೇತ್ರದಲ್ಲಿ ಜ್ಞಾನವನ್ನು ಸಂಪಾದಿಸಿಕೊಂಡಿದ್ದಾರೆ. ಹಾಗಾಗಿ ಅವರಿಗೆ ಸಮಾಜದ ನಾನಾ ಕ್ಷೇತ್ರದ ವಿದ್ವಾಂಸರುಗಳ, ಸಜ್ಜನರ ಕಲಾವಿದರ, ವೈದ್ಯರ, ವಕೀಲರು- ನ್ಯಾಯಮೂರ್ತಿಗಳ ಸಮೀಪದ ಆತ್ಮೀಯ ಪರಿಚಯ - ಒಡನಾಟವಿದೆ. ಹಾಗಾಗಿ ಇವರು ಪರರಿಗೆ ಯಾವ ಕ್ಷೇತ್ರದ ಸಮಸ್ಯಾ - ಸಂದರ್ಭಗಳಿಗೂ ಸ್ಪಂದಿಸುವ, ವ್ಯವಹರಿಸುವ, ಉಪಕರಿಸುವ ಯೋಗ್ಯತೆಯನ್ನು, ಚಾಕಚಕ್ಯತೆಯನ್ನು ಹೊಂದಿದ್ದಾರೆ, ಎಂದರೆ ಇವರ ಹರವನ್ನು ನಾವು ಊಹಿಸಬೇಕು. 

ನನ್ನ ಜೀವನದ ಒಂದು ಸಂದರ್ಭದಲ್ಲಿ  ಬಂದ ಸಮಸ್ಯೆಗೆ ನಾನು ನ್ಯಾಯಾಲಯವನ್ನು ಪ್ರವೇಶಿಸಬೇಕಾದಾಗ ಶ್ರೀಯುತ ಗಂಗಾಧರ ಭಟ್ಟರು ಮತ್ತು ಅವರ ಅಣ್ಣ ಮಂಜುನಾಥ ಭಟ್ಟರು ಮಾಡಿದ ಸಹಾಯ ಎಂದೂ ಮರೆಯಲಾಗದ ವಿಷಯ.

ತುರುವೇಕೆರೆ ವಿಶ್ವೇಶ್ವರ ದೀಕ್ಷಿತರು ಎಂಬವರು ಮಹಾಪಂಡಿತರು, ಪಾಠಶಾಲೆಯಲ್ಲಿ ಅಧ್ಯಾಪಕರಾಗಿದ್ದರು. ಬ್ರಹ್ಮಚಾರಿಗಳಾಗಿ ಜೀವನ ಸಾಗಿಸುತ್ತಿದ್ದರು. ವೃದ್ಧಾಪ್ಯದಲ್ಲಿ ಅವರು ಅಸಹಾಯರಾಗಿಬಿಟ್ಟರು. ಅಂತಹ ಸಂದರ್ಭದಲ್ಲಿ ಗಂಗಾಧರ ಭಟ್ಟರು ಅವರಿಗೆ ಸಾಕಷ್ಟು ಸಹಾಯ ಮಾಡಿದ್ದಾರೆ. 

ಮಹಾಮಹೋಪಾಧ್ಯಾಯ ಎನ್.ಎಸ್. ರಾಮಭದ್ರಾಚಾರ್ಯರು, ಪಾಠಶಾಲೆಯಲ್ಲಿ ನ್ಯಾಯಶಾಸ್ತ್ರದ ಅಧ್ಯಾಪಕರಾಗಿದ್ದರು. ಇವರಲ್ಲಿ ಗಂಗಾಧರ ಭಟ್ಟರಿಗೆ ಅತಿಶಯವಾದ ಗೌರವ. ಅವರು ಜೀವನದಲ್ಲಿ ಬಹಳ ಕಷ್ಟ - ನಷ್ಟ ಅನುಭವಿಸಿದವರು. ಸ್ವಭಾವತಃ ಸೂಕ್ಷ್ಮ ಮತ್ತು ಮೃದು ಸ್ವಭಾವದ ಅವರು ಮನೆ-ಸಂಸಾರದಲ್ಲೂ, ಪಾಠಶಾಲೆಯಲ್ಲಿಯೂ ಅನೇಕ ತೊಂದರೆಯನ್ನು ಎದುರಿಸಿದವರು. ಆಗೆಲ್ಲ ತಾವು ಮಾತ್ರವಲ್ಲದೆ ಸಹ ವಿದ್ಯಾರ್ಥಿಗಳನ್ನೂ ಸೇರಿಸಿಕೊಂಡು ಅವರ ಸಹಾಯಾಕ್ಕೆ ನಿಂತವರು ಅವರ ವಿದ್ಯಾರ್ಥಿಯೂ ಆದ ಗಂಗಾಧರ ಭಟ್ಟರು. ಅವರಿಗೆ ಮಾತ್ರವಲ್ಲ, ಅವರ  ಕುಟುಂಬವರ್ಗಕ್ಕೂ ಸಮಯದಲ್ಲಿ ಸಾಕಷ್ಟು ಸಹಾಯ ಮಾಡಿ ಉಪಕರಿಸಿದ್ದಾರೆ. ಇನ್ನೊಂದು ಸ್ವಾರಸ್ಯವೆಂದರೆ, ಶ್ರೀಯುತ ರಾಮಭದ್ರಾಚಾರ್ಯರು ಗಂಗಾಧರ ಭಟ್ಟರಿಗೆ ಪಾಠಶಾಲೆಯಲ್ಲಿ ಪಾಠಮಾಡುತ್ತಿದ್ದರು. ಆದರೆ ಗಂಗಾಧರ ಭಟ್ಟರು ರಾಮಭದ್ರಾಚಾರ್ಯರ  ಮನೆಗೆ ಹೋಗಿ ಅವರ ಮಕ್ಕಳಿಗೆ ಸಂಸ್ಕೃತ ಪಾಠಮಾಡುತ್ತಿದ್ದರು. ಹೀಗೊಂದು ವಿಶಿಷ್ಟ ವ್ಯವವಹಾರ ಅವರಲ್ಲಿತ್ತು. ತನ್ನ ವಿದ್ಯಾರ್ಥಿಯನ್ನು ತನ್ನ ಮಕ್ಕಳಿಗೆ ಪಾಠಮಾಡಲು ಅಪೇಕ್ಷಿಸಿ ತನ್ನ ಪಾಠದ ಪರಿಣಾಮ ಮತ್ತು ಪ್ರಯೋಗವನ್ನು ತನ್ನ ಮಕ್ಕಳಿಗೇ ಪಾಠಮಾಡಿಸಿ ನೋಡುವ ಒಂದು ಪ್ರಯೋಗ ಅಲ್ಲಿ ನಡೆಯಿತು.  ಹಾಗಾಗಿ ಆಚಾರ್ಯರಿಗೆ ವಿದ್ಯಾರ್ಥಿಯಾಗಿದ್ದ ಗಂಗಾಧರ ಭಟ್ಟರು  ಆಚಾರ್ಯರ ಮಕ್ಕಳಿಗೆ ಆಚಾರ್ಯರಾಗಿದ್ದರು.

ಕೆಲವು ಸಂಘ, ಸಂಸ್ಥೆಗಳಲ್ಲಿ ಅಧ್ಯಕ್ಷ, ಕಾರ್ಯದರ್ಶೀ ಮುಂತಾದ ಹುದ್ದೆಗಳನ್ನು ನಿರ್ವಹಿಸಿದ್ದಾರೆ. ಮೈಸೂರಿನ ಹವೀಕ ಸಂಘದಲ್ಲಿ ಇವರ ಕಾರ್ಯ ನಿರ್ವಹಣೆಯನ್ನು ಸಂಘ ಇಂದೂ ಜ್ಞಾಪಸಿಕೊಳ್ಳುತ್ತದೆ. 

ಸ್ವಭಾವತಃ ಇವರು ಗಂಭೀರ. ಸದಾ ಹಸನ್ಮುಖಿ. ನ್ಯಾಯ್ಯಾತ್ ಪಥಃ ಪ್ರವಿಚಲಂತಿ ಪದಂ ನ ಧೀರಾಃ ಎಂದು ಸುಭಾಷಿತ ಹೇಳುವಂತೆ ಅನ್ಯಾಯವನ್ನು ಎಂದೂ ಸಹಿಸದೇ ತಮ್ಮ ವ್ಯಪ್ತಿಯಲ್ಲಿ ತತ್ಕ್ಷಣ ಪ್ರತಿಭಟಿಸುವವರು. ವ್ಯವಹಾರದಲ್ಲಿ ಒಂದು  ನಿರ್ಣಯದಿಂದ ಹೆಜ್ಜೆಯನ್ನಿಟ್ಟರೆ ಯಾವ ಕಾರಣಕ್ಕೂ ಹಿಂದೆಗೆಯದ ಧೀರ ಸ್ವಭಾವದವರು. ಇವರಲ್ಲಿ ಸಹಜವಾಗಿ ಜನರಿಗೇನೋ ಒಂದು ಆಕರ್ಷಣೆ ಉಂಟಾಗುವುದನ್ನು ನಾನು ಕಂಡಿದ್ದೇನೆ. ಆದರೆ ಆ ಆಕರ್ಷಣೆಯನ್ನು ಎಂದೂ ತಮಗಾಗಿ ಬಳಸಿಕೊಳ್ಳುವ ಪ್ರವೃತ್ತಿ ಇವರಲ್ಲಿಲ್ಲ.  ಯಾರಲ್ಲೂ ಮೇಲು-ಕೀಳು, ಬಡವ-ಬಲ್ಲಿದ, ಪಂಡಿತ-ಪಾಮರ ಮತ್ತು ಜಾತಿ-ಮತಗಳಲ್ಲಿ ಭೇದಭಾವ ಇಟ್ಟುಕೊಳ್ಳದೇ ಆತ್ಮೀಯವಾಗಿ ಪ್ರೀತಿಯಿಂದ ವಿಷಯಾನುಗುಣವಾಗಿ ಅವರೆಲ್ಲರನ್ನೂ ಆದರಪೂರ್ವಕವಾಗಿ ಭಾವಿಸುವ ಸ್ವಭಾವದವರು. ಆದ್ದರಿಂದ ಇವರ ಸಾಂಘಿಕ ಸ್ವಭಾವಕ್ಕೆ ಅವರವರ ವಿವಿಧ ಉಪಾಧಿಗಳಾವುವೂ  ತೊಡಕಾಗಿದ್ದಿಲ್ಲ.

ಹೀಗೆ ಒಬ್ಬ ವಿಶಿಷ್ಟ ವ್ಯಕ್ತಿ ಹಳ್ಳಿಯಿಂದ ಮೈಸೂರಿಗೆ ಬಂದು, ಅಲ್ಲಿಯೇ ಓದಿ, ವೃತ್ತಿ ಮಾಡಿ ಸಮಾಜದಲ್ಲಿ ಬಹಳ ಗೌರವಗಳಿಸಿ ವೃತ್ತಿಯಿಂದ ನಿವೃತ್ತರಾಗಿದ್ದಾರೆ. ಈ ವರೆಗೆ ಸಮಾಜಕ್ಕೆ ಬಹಳವಾಗಿಯೇ ಉಪಕರಿಸಿ ಅವರಿಂದ ಕೃತಜ್ಞತೆಗೆ ಅರ್ಹರಾಗಿದ್ದಾರೆ. ಇಂಥವರ ಬಗ್ಗೆ ಎಷ್ಟು ಹೇಳಬಹುದು !? ಎಷ್ಟೂ ಹೇಳಬಹುದು. ನನಗೆ ತಿಳಿದಿರುವುದು ಕಡಿಮೆಯೇ. ಅವುಗಳಲ್ಲಿ ಕೆಲವು ಅಂಶವನ್ನು ಮಾತ್ರ ಇಲ್ಲಿ ಈಗ ವಿಜ್ಞಾಪಿಸಿದ್ದೇನೆ. ಇವರು ದೀರ್ಘಕಾಲ ಬದುಕಿ ಅವರ ಜೀವನ ಭಗವಂತನ ಕೃಪೆಗೆ ಪಾತ್ರವಾಗಿ ಸಾರ್ಥಕತೆಯನ್ನು ಹೊಂದಲಿ ಎಂದು ಪರಮಾತ್ಮನಲ್ಲೂ ಸದ್ಗುರುವಿನಲ್ಲೂ ಪ್ರಾರ್ಥಿಸುತ್ತೇನೆ.

