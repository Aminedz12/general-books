{\fontsize{15}{17}\selectfont
\presetvalues
\chapter{गङ्गाधरो भट्टवरो विराजताम्}

\begin{center}
\Authorline{गोद्लबीलाभिजनो वि~। मञ्जुनाथभट्टः}
\smallskip

निवॄत्तः\\ 
सहायकप्राध्यापकः\\ 
महाराजसंस्कृतमहापाठशाला,\\ 
मैसूरु
\addrule
\end{center}

\begin{verse}
षष्ट्यब्देः प्राघ्हेमलम्बस्य वर्षे पौषे कृष्णे ज्येष्ठभे जीववारे~।\\
द्वादश्यां वै जीवतुल्यो हि जीवी श्रीविघ्नेशाद्रेवतीगर्भजातः~॥ १~॥
\end{verse}

\begin{verse}
ज्येष्ठा तारा जन्मकाले शिशोर्हि तस्मादेषो देवगङाधरस्य~।\\
नाम्ना भूयाज्ज्येष्ठ इत्येव तस्य चक्राते तौ दम्पती पुण्यनाम~॥ २~॥
\end{verse}

\begin{verse}
काले काले सर्वसंस्कारपूतो बालोऽकार्षीद्बाललीलाकलापम्~।\\
तस्मात्तुष्टा वंशजा बालमेनं विद्यापूतं कर्तुमैच्छन्वरिष्ठाः~॥ ३~॥
\end{verse}

\begin{verse}
स्वग्रामे स्वप्राक्तनैः स्थापितायां शालायां यो बाल्यविद्यामधीत्य~।\\
पश्चादाराद्वेणुकाकाननस्थां प्रौढिं शालामेत्य तत्राध्यशीष्ट~॥ ४~॥
\end{verse}

\begin{verse}
प्राचीनविद्यामधिगन्तुकामो गोकर्णमासाद्य हरिप्रसादात्~।\\
गुरोर्दयालोरथ दत्तभट्टात् स्वाध्यायमध्यैष्ट कुशाग्रबुद्धिः~॥ ५~॥
\end{verse}

\begin{verse}
गीर्वाणवाणीप्रणिबद्धशास्त्रजालानि विज्ञातुमना मनीषी~।\\
तस्मान्महीशूरपुरीं पुराणीं समाविशत् संस्कृतपाठशालाम्~॥ ६~॥
\end{verse}

\begin{verse}
श्रीमन्महाराजकृपाकटाक्षात् संवर्धिता संस्कृतपाठशाला~।\\
अस्यां तदानीं निगमागमज्ञाः प्रबोधयन्ति स्म बुधा मनोज्ञम्~॥ ७~॥
\end{verse}

\begin{verse}
तेषां बुधानां सविधे सहर्षमधीत्य साहित्यमथ प्रकामम्~।\\
व्युत्पत्तिमासाद्य ततोधिगन्तुं यः प्राविशद्गौतमतर्कशास्त्रम्~॥ ८~॥
\end{verse}

\begin{verse}
श्रीरामभद्रार्यमुखारविन्दादामूलचूलं समधीत्यशास्त्रम्~।\\
शास्त्रस्थसूक्ष्मानधिगम्य विद्वान् गङाधरोभूद्बुधवर्गमान्यः~॥ ९~॥
\end{verse}

\begin{verse}
अध्यापको न्यायनयस्य भूत्वा सम्बोध्यशास्त्रं शतशो हि छात्रान्~।\\
अग्गेरिविघ्नेश्वरभट्टसूनुर्चकास्ति गङाधरभट्टवर्यः~॥ १०~॥
\end{verse}

\begin{verse}
शास्त्रार्थधर्ता बहुरम्यवक्ता कर्ता यथा शास्त्रसमन्वयस्य~।\\
भर्तास्ति शास्त्राध्ययनार्थिनां हि हर्ता तथा हृद्गतदुर्गुणानाम्~॥ ११~॥
\end{verse}

\begin{verse}
शास्त्राब्धिपारङ्गतधीर एषः शास्त्रार्थसङ्क्रान्तिकलाप्रवीणः~।\\
आङ्लादिविद्यास्वपि मान्यमान्यो गर्वादिशून्यः सरलो दयालुः~॥ १२~॥
\end{verse}

\begin{verse}
सद्यो व्यतीतानि बुधस्य षष्टिर्वर्षाणि गङाधरसंज्ञितस्य~।\\
तस्मादिदं शिष्यगणेन क्ल्प्तं प्रीत्यादराभ्यामभिवन्दनं हि~॥ १३~॥
\end{verse}

\begin{verse}
कार्यान्निवृत्तिर्नियमानुरोधाद्भवेत्समेषां किमु तत्र चित्रम्~।\\
केचिन्निवृत्ता नितरां परन्तु केचित्सदा बोधनतत्परा स्युः~॥ १४~॥
\end{verse}

\begin{verse}
भट्टप्रवृत्तिः परबोधनात्मिका तस्या निवृत्तिः किमु वा भविष्यति~।\\
धीमान् सदासौ परबोधतत्परो गङ्गाधरो भट्टवरो विराजताम्~॥ १५~॥
\end{verse}

\begin{verse}
निष्कल्मषः सर्वजनानुरञ्जको विद्यानिधिर्लौकिककार्यकोविदः~।\\
लोकोपकारी सह कान्तया महान् गङ्गाधरः शैलजया विराजताम्~॥ १६~॥
\end{verse}

\begin{verse}
गग्ङाधरः शास्त्रसुधानिधिर्गुणी सर्वत्र सम्मानविशेषभाजनः~।\\
नित्यं स्वशिष्योद्धरणोद्यतः स्वयमाचार्यवर्यो नितरां विराजताम्~॥ १७~॥
\end{verse}

\begin{verse}
श्रीहेमलम्बस्य च माघकृष्णके चैकादशी भानुदिने समर्पिता~।\\
शिष्यैः कृतेऽस्मिन्नभिनन्दनोत्सवे मोदायभूयान्मम पद्यमालिका~॥ १८~॥
\end{verse}

\centerline{$*****$}

\begin{verse}
श्रीमञ्जुनाथप्रणिबद्धपद्यं गग्ङाधरोज्जीवनजन्मशीलम्~।\\
हृद्यं त्विदं पङ्कजलोलमालं सुसङ्गतं शब्दसुबोधबद्धम्~॥
\end{verse}


\hspace{6cm}\textbf{(सम्पादकः)}

\articleend
}
