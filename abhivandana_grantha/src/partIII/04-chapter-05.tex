{\fontsize{15}{17}\selectfont
\presetvalues
\chapter{गङ्गाधरसुधीरत्र हव्यकव्योमभास्करः}

\begin{center}
\Authorline{वि~। अरैयर् रामशर्मा}
\smallskip

निवॄत्ताध्यापकः\\
संस्कृतमहापाठशाला\\
मेलुकोटे
\addrule
\end{center}

\begin{verse}
शैलजा शैलजैवासौ तस्या गङ्गाधरोऽप्यसौ~।\\
गङाधर इवास्त्येव लोकस्योपकृतावुभौ~॥ १~॥
\end{verse}

\begin{verse}
न्यायशीलनयैरेकः त्रैविध्येप्येक एव सन्~।\\
विधूमः पावक इव ज्ञानाज्याहुतिभिर्ज्वलन्~॥ २~॥
\end{verse}

\begin{verse}
गङ्गाधरीकृता यस्य कीर्त्या मूर्तिरपि स्वयम्~।\\
गुरुरागरसैश्छात्रैः पुष्करैररुणायिता~॥ ३~॥
\end{verse}

\begin{verse}
गङ्गाधरसुधीरत्र हव्यकव्योमभास्करः~।\\
प्रातर्मध्यन्दिने सायं सुहृद्भिरभिवन्दितः~॥ ४~॥
\end{verse}

\begin{verse} 
इतिश्रीरामशर्मा तु यद्वद्रिश्रीशगायकः~।\\
महीसुरपुरे स्मेरं गङ्गाधरमुदञ्चति~॥ ५~॥
\end{verse}

\begin{center}
श्रीरामनामा कविकाव्यधर्मा यद्वद्रिवासी हरिपादसक्तः~।\\
\textbf{गङाधरस्नेहपरागस्पर्शात् ससृष्ट} पद्यं सुपदैस्सभायाम्~॥ 
\end{center}

~\hfill\textbf{(सम्पादकः)}

\articleend
}
