{\fontsize{14}{16}\selectfont
\chapter{ಬಹುಕಾಲದ ಒಡನಾಡಿ}

\begin{center}
\Authorline{ವಿ~।। ಉಮಾಕಾಂತ ಭಟ್ಟ}
\smallskip
ಪ್ರಾಂಶುಪಾಲರು\\
ಸರ್ಕಾರಿ ಸಂಸ್ಕೃತ ಪಾಠಶಾಲೆ, ಮೇಲುಕೋಟೆ
\addrule
\end{center}

ನಾನು ಮತ್ತು ಗಂಗಾಧರ ಬಹುಕಾಲದ ಒಡನಾಡಿಗಳು. 1975ರ ಜೂನ್ ತಿಂಗಳ ಕೊನೆಯ ವಾರ ನಮ್ಮ ಮೊದಲ ಪರಿಚಯ. ರಾತ್ರಿ 10 ರ ಸಮಯ. ಸಂಸ್ಕೃತ ಕಾಲೇಜಿನ ಹಾಸ್ಟೇಲ್ ರೂಂ ನಂ \enginline{-} 9 ರ ಒಳಗೆ ಬರುತ್ತಿದ್ದಂತೆ ನನ್ನ ಮತ್ತು ಗಂಗಾಧರನ ಮುಖ ದರ್ಶನವಾಯಿತು. ನಗು ಮಾತಿಗೆ ತಿರುಗಿತು. ಮಾತು ಆಸಕ್ತಿಯನ್ನು ಅರುಹಿತು. ಆಸಕ್ತಿ ಮತ್ತು ಆಯ್ಕೆಗಳು ಮೈತ್ರಿಯ ಕಟ್ಟನ್ನು ಬಲಪಡಿಸಿದವು. ಅಂದಿನಿಂದ ಇಂದಿನ\-ವರೆಗೂ ನಮ್ಮ ಸ್ನೇಹ ವಿಶ್ವಾಸಗಳು ಬೆಳೆಯುತ್ತಲೇ ಬಂದಿವೆ. ಈ ನಡುವೆ ಆಗಾಗ ಏರುಪೇರುಗಳಾಗಿವೆ, ಸ್ಪರ್ಧೆ ಸಂಘರ್ಷಗಳು ಒದಗಿ ಬಂದಿವೆ. ಆದರೂ ಅವುಗಳಿಂದ ನಮ್ಮ ಪ್ರೀತಿ ವಿಶ್ವಾಸಗಳಿಗೆ ಧಕ್ಕೆಯಾಗಲಿಲ್ಲ. ಮೈತ್ರಿಯ ಈ ಸಂಬಂಧವನ್ನು ನಾವಿಬ್ಬರೂ ಹೊಣೆಗಾರಿಕೆಯಿಂದ ಬೆಳೆಸಿಕೊಂಡು ಬಂದಿದ್ದೇವೆ. ಆದುದರಿಂದ ಅದು ಈಗ ಯಾಗಶಾಲೆಯ ಅಗ್ನಿಯಂತೆ ಇದೆ. ಆಹುತಿಯನ್ನು ಅರ್ಪಿಸಿದಾಗ ಅದು ಸಂಧುಕ್ಷಿತವಾಗಿ ಬೆಳಗುತ್ತದೆ. ಉಳಿದ\break ಹೊತ್ತಿನಲ್ಲಿ ಬೆಚ್ಚನೆಯ ಕಾವಾಗಿ ಕುಂಡದೊಳಗೆ ಅಡಗಿ\-ಕೊಂಡಿರುತ್ತದೆ. ನಮ್ಮಿಬ್ಬರ\break ಇಂಗಿತ, ನಿಮಿಷಿತ, ಚೇಷ್ಟಿತ, ಆಶಯ, ರುಚಿ, ಅಭಿರುಚಿ, ನಿರ್ಣಯ, \hbox{ಮೊದಲಾದವು} ತುಂಬಾ ನಿಕಟವಾಗಿವೆ ಎಂದು ನಮ್ಮೀರ್ವರ   \hbox{ಅನುಭವಕ್ಕೂ} ಬಂದಿದೆ. ನಮ್ಮಿಬ್ಬರಲ್ಲಿ ಅನೇಕ ಮೂಲಭೂತವಾದ ನಿಲು\enginline{-}ಭೇದಗಳಿವೆ. ಒಲವುಗಳ ವೈವಿಧ್ಯವಿದೆ. ಅವುಗಳ ಸ್ಪಷ್ಟವಾದ ಅರಿವೂ ನಮಗೆ ಇದೆ. ಇದರಿಂದ ನಮ್ಮ ಒಡನಾಟದಲ್ಲಿ  ಸಾಮಿಪ್ಯದ ಸಂತಸವಿದೆ, ಗೌರವದ ದೂರವಿದೆ, ಹೊಣೆಗಾರಿಕೆಯ ಸಮತೋಲ\-ವಿದೆ. 

ಭೌತಿಕವಾಗಿ ಹತ್ತಿರವಿರಲಿ, ದೂರವಿರಲಿ, ಎದುರುಗಡೆ ಇರಲಿ, ಹಿಂದುಗಡೆ ಇರಲಿ, ಇಬ್ಬರ ಆಲೋಚನೆಯ ದಿಕ್ಕು\enginline{-}ದೆಸೆಗಳೂ ಒಂದೇ. ಒಂದೇ ರೀತಿಯ ಚುರುಕುತನ, ತೀವ್ರತೆ, ವಿಷಯದ ಒಳಹೊಕ್ಕು ನಿರ್ಣಯಿಸುವ ಕೌಶಲಗಳು ಇಬ್ಬರಿಗೂ ಕೂಡಿ\-ಬಂದಿರುವುದು  ಅಚ್ಚರಿಯ ಸಂಗತಿ. ಸಂದರ್ಭಗಳ ನಿರ್ಣಯ, ವ್ಯಕ್ತಿಗಳ ವ್ಯಕ್ತಿತ್ವ \hbox{ಚಿತ್ರ,} ವ್ಯವಹಾರಗಳ ಮುಂಧೋರಣೆ ಮೊದಲಾದವು  ಹೆಚ್ಚು ಕಡಿಮೆ  ಒಂದೇ ತೆರನಾಗಿ ಬಂದು ಒದಗುತ್ತವೆ.  ಆದರೆ ಅವುಗಳ ನಿರ್ವಹಣ ಸರಣಿಯಿಂದ  ನಾವು ಭಿನ್ನರಾಗುತ್ತೇವೆ.\hbox{ಕ್ರಿಯೆಗಳ}ಆಯ್ಕೆಯಲ್ಲಿ ನಾವು ಬೇರೆಯಾಗುತ್ತೇವೆ. ನಮ್ಮಿಬ್ಬರ ಜೊತೆ ಎಂದರೆ ಕೆಲವರಿಗೆ ಹಿಗ್ಗು; ಕೆಲವರಿಗೆ ಕಣ್ಣು; ಇನ್ನೂ ಕೆಲವರಿಗೆ  ಭಯ. ನಮ್ಮ ಜೊತೆ ಸಹಪಾಠಿಗಳು ಯಾರೂ ಸುಲಭವಾಗಿ ಚರ್ಚೆಗೆ ಇಳಿಯುತ್ತಿರಲಿಲ್ಲ.  ನಮ್ಮ ಪ್ರಶ್ನೆಯ \hbox{ಬಿರುಸಿಗೆ} ಅವರು ತತ್ತರಿ\-ಸುತ್ತಿದ್ದರು. ನಮ್ಮ ಪ್ರತಿಭೆಗೆ ಮಂಕಾಗಿ ಬಿಡುತ್ತಿದ್ದರು. ಪ್ರತ್ಯೇಕವಾಗಿ ನಮ್ಮ ನೈಕಟ್ಯ\-ವನ್ನು ಬಳಸಿಕೊಳ್ಳುತ್ತಿದ್ದವರೂ ನಾವಿಬ್ಬರೂ ಒಟ್ಟಿಗೆ ಇರುವಾಗ \hbox{ಮೆಲ್ಲಗೆ} \hbox{ಜಾರಿಕೊಂಡು} ಬಿಡುತ್ತಿದ್ದರು. 

ಶ್ರೀಮನ್ಮಹಾರಾಜ ಸಂಸ್ಕೃತ ಮಹಾಪಾಠಶಾಲೆ ನಮ್ಮ ಮೈ\enginline{-}ಮನಗಳನ್ನು ಬೆಳೆಸಿ\break ಬೆಳಗಿಸಿದ 'ತೀರ್ಥಕ್ಷೇತ್ರ'. ನಾವಿಬ್ಬರೂ ನ್ಯಾಯಶಾಸ್ತ್ರದ ವಿದ್ಯಾರ್ಥಿಗಳು. ನನ್ನದು ಒಂದು ತರಗತಿ ಮುಂದಿನದು. ಗಂಗಾಧರ ಮತ್ತು ಶಂಕರ ಒಂದೇ ತರಗತಿಯ\break ವಿದ್ಯಾರ್ಥಿಗಳು. ಪಾಠ ಒಟ್ಟಿಗೆ  ನಡೆಯುತ್ತಿತ್ತು. ಮಹಾಮಹೋಪಾಧ್ಯಾಯ  ಶ್ರೀ ಎನ್,ಎಸ್. ರಾಮಭದ್ರಾಚಾರ್ಯರು ನಮ್ಮನ್ನು ಅತ್ಯಂತ ಆಕರ್ಷಿಸಿದ ಅಧ್ಯಾಪಕರು. ಅವರ ನಡೆ, ನುಡಿ, ನಗು, ನಿಲುವು, ಪಾಠ, ಪ್ರವಚನ ಎಲ್ಲವೂ ನಮಗೆ ಆದರ್ಶ\enginline{-} ಆಪ್ಯಾಯನ. ನಮ್ಮೊಳಗೆಲ್ಲ ಅವರೇ ತುಂಬಿಕೊಂಡು ಬಿಟ್ಟಿದ್ದರು. ಅವರು ನಮ್ಮ ಭಾವವನ್ನು ಸಂಸ್ಕರಿಸಿದರು., ಭಾಷೆಯನ್ನು ಪರಿಷ್ಕರಿಸಿದರು. ಪ್ರವೃತ್ತಿ ನಿವೃತ್ತಿಗಳನ್ನು\break ತಿದ್ದಿದ್ದರು. ಮನಸ್ಸನ್ನು ಬೆಳೆಸಿದರು. ಉಳಿದ ಅಧ್ಯಾಪಕರಲ್ಲಿ  ಗಂಗಾಧರನಿಗೆ ವಿದ್ವಾನ್ ಎ.ವೆಂಕಣ್ಣಾಚಾರ್ಯರ ವಿಶ್ವಾಸ\enginline{-}ಬಳಕೆಗಳು ಹೆಚ್ಚಾಗಿ ಸಿಗುತ್ತಿದ್ದವು. ಆ ದಿನಗಳಲ್ಲಿ ಸಂಸ್ಕೃತ  ಮಹಾಪಾಠಶಾಲೆಯ  ಬಹುತೇಕವಾಗಿ ದೊಡ್ಡ ದೊಡ್ಡ ವಿದ್ವಾಂಸರೆಲ್ಲ  ನಮ್ಮನ್ನು ಪ್ರೀತಿಯಿಂದ, ಸಲುಗೆಯಿಂದ, ಅಭಿಮಾನದಿಂದ ಮಾತನಾಡಿಸಿ, ಸಂದೇಹಗಳಿಗೆ ಉತ್ತರವನ್ನು ಕೊಟ್ಟು ತಮ್ಮ ಶಾಸ್ತ್ರದ ವಿದ್ಯಾರ್ಥಿಗಳಂತೆ ನೋಡಿಕೊಳ್ಳುತ್ತಿದ್ದರು. ಇದರ ಜೊತೆಗೆ\break ಗಂಗಾಧರನಿಗೆ ವೇದತೊಟ್ಟಿಯ  ಪಂಡಿತರ ಬಳಕೆಯೂ ಗಣನೀಯವಾಗಿ ಇತ್ತು. 

ಎರಡರಿಂದ ಮೂರು ವರ್ಷಗಳ ಅವಧಿಯಲ್ಲಿ ನಮ್ಮಿಬ್ಬರ ಹೊರಗಿನ  ಅಂದರೆ\break ಕಾಲೇಜಿನ ಓದು ಮತ್ತು ಸಂಸ್ಕೃತ ವಿದ್ಯಾಭ್ಯಾಸ ಎಷ್ಟು ಬಲವಾಗಿ ಸಾಗಿತ್ತು ಎಂಬುದನ್ನು  ಜ್ಞಾಪಿಸಿಕೊಂಡರೆ ಈಗಲೂ ರೋಮಾಂಚನವಾಗುತ್ತದೆ. ಬೆಳಿಗ್ಗೆ ಸಂಸ್ಕೃತ ಶಾಸ್ತ್ರಪಾಠ, ಮದ್ಯಾಹ್ನ ಹೊರಗಡೆ ಕಾಲೇಜು ಪಾಠ, ಸಾಯಂಕಾಲ ಶಾಸ್ತ್ರಾರ್ಥ ಚಿಂತನ. ಯಜ್ಞಪ್ಪನ ಕಟ್ಟೆಯ  ಅರಳಿಮರ ನಮ್ಮ ಶಾಸ್ತ್ರ ಚಿಂತನೆಯ ತಾಣ. ಮುಕ್ತವಾಗಿ ಆಲೋಚನೆ ಚರ್ಚೆ\-ಗಳನ್ನು ಮಾಡುತ್ತಿದ್ದ ಸಮಯವದು. ನಾವು ಮೂವರೂ ಸೇರಿ ಪಂಚಲಕ್ಷಣೀ, ದಿನಕರೀಯದ ಕೆಲವು ಭಾಗಗಳು ಮತ್ತು ಸಿದ್ಧಾಂತಲಕ್ಷಣದ ಕೆಲವು ಅಂಶಗಳನ್ನು ಚಿಂತಿಸಿದೆವು. ಈ  ವಿಧಾನದಿಂದ ನಮ್ಮ  ನಮ್ಮ ಮನಸ್ಸು  ಶಾಸ್ತ್ರದ ಒಳಗೆ ಸುಲಭವಾಗಿ ಪ್ರವೇಶ ಮಾಡಿ, ಸ್ವತಂತ್ರವಾಗಿ ಯುಕ್ತಿಗಳನ್ನು  ಕಲ್ಪಿಸುವಷ್ಟು ಸಶಕ್ತವಾಗಿತ್ತು. ಆದರೆ ಕಾರಣಾಂತರಗಳಿಂದ ನಮ್ಮ ಚಿಂತನಕಾರ್ಯ ನಿಂತುಕೊಂಡಿತು. ಅದನ್ನು ನಾವು ಮುಂದುವರಿಸುತ್ತಿದ್ದರೆ\break ಶಾಸ್ತ್ರಾರ್ಥಗಳು ಇನ್ನಷ್ಟು ಜಿತವಾಗಿರುತ್ತಿದ್ದವು. ನಾನು ಗಂಗಾಧರ ಆಗಾಗ ಚಿಂತನ ಮತ್ತು ಚರ್ಚೆಗಳನ್ನು ಮಾಡುತ್ತಿರುತ್ತೇವೆ. ಆದರೆ ಅವು ನಿಯಮಿತವಾಗಿಲ್ಲ. ಅವು ಆ ಆ ಕಾಲದಲ್ಲಿ ನಮ್ಮ ಬೌದ್ಧಿಕ ವ್ಯಾಯಾಮ ಮತ್ತು ಬಲಗಳು ಹೇಗಿವೆ ಎಂಬ ಪರೀಕ್ಷೆಗಾಗಿ ಮಾತ್ರ.

ಈ ಅವಧಿಯಲ್ಲಿ ನಾವಿಬ್ಬರೂ ವ್ಯಕ್ತಿತ್ವವನ್ನು ಚೆನ್ನಾಗಿ ಬೆಳೆಸಿಕೊಳ್ಳಲು ಯತ್ನಿಸ\-ತೊಡಗಿದೆವು. ಆಗ ಮೈಸೂರು ನಗರದಲ್ಲಿ ಅಂತರಕಾಲೇಜು ಚರ್ಚಾಸ್ಪರ್ಧೆಗಳೂ, ಅಂತರ ವಿದ್ಯಾರ್ಥಿನಿಲಯಗಳ ಚರ್ಚಾಗೋಷ್ಠಿಗಳೂ ಬಹಳ ಸಂಖ್ಯೆಯಲ್ಲಿ ನಡೆಯು\-ತ್ತಿದ್ದವು. ನಾವು ಯಾವ ಸ್ಪರ್ಧೆಯನ್ನೂ ಬಿಡುತ್ತಿರಲಿಲ್ಲ. ಭಾಷಣದ ಅವಕಾಶವನ್ನು\break ಔದಾಸೀನ್ಯದಿಂದ ಬಿಟ್ಟುಕೊಟ್ಟ ನೆನಪೇ ಇಲ್ಲ. ಬಹುಮಾನಗಳು ನಮ್ಮನ್ನು ಹುರಿದುಂಬಿ\-ಸಿದವು. ಶೀಲ್ಡಗಳು ನಮ್ಮ ಹೆಗಲು ತಟ್ಟಿದವು. ಸುಮಾರು ವರ್ಷಗಳ ವರೆಗೆ ಮೊದಲನೆಯ ಮತ್ತು ಎರಡನೆಯ ಸ್ಥಾನಗಳು ನಮ್ಮಿಬ್ಬರನ್ನು ಬಿಟ್ಟು ಕದಲುತ್ತಿರಲಿಲ್ಲ. ಒಂದು ಸಾರಿ ನಾನು ಪ್ರಥಮ, ಇನ್ನೊಂದು ಬಾರಿ ಗಂಗಾಧರ  ಪ್ರಥಮ. ಇಬ್ಬರ ಭಾಷೆಯಲ್ಲೂ ಡಿ.ವಿ.ಜಿ ಯವರ ಗಾಢವಾದ ಪ್ರಭಾವವಿತ್ತು. 'ಬಾಳಿಗೊಂದು ನಂಬಿಕೆ', ನನ್ನ ಕೈಪಿಡಿಯಾದರೆ 'ಜೀವನಧರ್ಮಯೋಗ' ಗಂಗಾಧರನ ಪರಿವಿಡಿಯಾಗಿತ್ತು. ನಾನು ಆಗಾಗ ಅಡಿಗರ ಗದ್ಯಕ್ಕೆ   ಮನಸೋತರೆ ಗಂಗಾಧರ ಗೌರೀಶ ಕಾಯ್ಕಿಣಿಯವರ ಓಜಸ್ಸಿಗೆ ಮೊರೆಹೋಗುತ್ತಿದ್ದ. ನಾವಿಬ್ಬರೂ ಪಂಚೆಯುಟ್ಟೇ ಸಭೆಗೆ ಹೋಗುತ್ತಿದ್ದೆವು. ಅದರಿಂದ 'ಪಂಚೆ ಕಂಪನಿ' ಎಂದು ನಮ್ಮನ್ನು ಹೆಸರಿಸಿ ಸಹಪಾಠಿಗಳು  ಕರುಬುತ್ತಿದ್ದರು ಎಂದು ಅನಿ\-ಸಿತ್ತು. ಎಷ್ಟೋ ಸಾರಿ ಹೊರಗಡೆ ಭಾಷಣದ ಸ್ಪರ್ಧೆಗಳಲ್ಲಿ ಗೆದ್ದು ಬಹುಮಾನಗಳನ್ನು ಪಡೆದು ಹಾಸ್ಟೆಲ್ ಗೆ  ಹಿಂದಿರುಗುವ  ಹೊತ್ತಿಗೆ ಊಟವೇ ಇರುತ್ತಿರಲಿಲ್ಲ. ಉಳಿದವರು\break ಸರಸರನೆ ಊಟ ಮಾಡಿ ಉಳಿದ ಅನ್ನ, ಸಾರು, ಮಜ್ಜಿಗೆಗಳನ್ನು ಕೆಲಸಗಾರರಿಗೆ ಹಂಚಿ ' ನಿಮಗೆ ಈ ಹೊತ್ತು ಊಟ ಇಲ್ಲ' ಎಂದು ಹೇಳಿ  ಸಂತೋಷಪಟ್ಟ ದಿನಗಳೂ ಇವೆ. ಆಗ\break ನಮ್ಮಿಬ್ಬರಿಗೂ ಕೈಯಲ್ಲಿ ಇರುವ ಬಹುಮಾನದ ಸಂತಸಕ್ಕಿಂತಲೂ ಹೊಟ್ಟೆಯೊಳಗಿನ\break ಹಸಿವಿನ ಸಂಕಟವೇ ಹೆಚ್ಚಾಗುತ್ತಿತ್ತು.

ಈ ನಡುವೆ ಪಾಠಶಾಲೆಯ ವಿದ್ಯಾರ್ಥಿಗಳ ಗುಂಪುಗಾರಿಕೆಯ ಕೆಲವು ವ್ಯವಹಾರ\-ಗಳಲ್ಲಿ ನಾವಿಬ್ಬರೂ ಒಂದೊಂದು ಪಕ್ಷದಲ್ಲಿ ನಿಂತು ಮುಖಾಮುಖಿಯಾಗಿ \-ಸಂಘರ್ಷಕ್ಕೆ ನಿಲ್ಲುವ  ಅನಿವಾರ್ಯತೆ ಬಂದಿತ್ತು. ಆಗ ನಮ್ಮಿಬ್ಬರಿಗೂ ನಮ್ಮ ಸ್ನೇಹವು ಮುರಿದು\break ಬೀಳುವ ಆತಂಕ ಹುಟ್ಟಿತ್ತು. ಏಕಾಂತದಲ್ಲಿ ನಾವು ಮಾತಾಡಿಕೊಂಡೆವು. ಯಾವ\break ಕಾರಣಕ್ಕೂ ನಾವಿಬ್ಬರೂ ವ್ಯವಹಾರದಲ್ಲಿ ಎದುರು ಬದುರಾಗಿ ನಿಂತು ಹೋರಾಡ\-ಬಾರದು ಎಂದು ನಿಶ್ಚಯ ಮಾಡಿಕೊಂಡೆವು. ಇಕ್ಕಟ್ಟು ಬಂದಾಗ ಎಲ್ಲಿ ನಾವಿಬ್ಬರೂ\break ಮಾತಾಡಿಕೊಳ್ಳದೆ ನಿರ್ಣಯವನ್ನು ಪ್ರಕಟಿಸುತ್ತಿರಲಿಲ್ಲ. 

ಗಂಗಾಧರ ವ್ಯವಹಾರದಲ್ಲಿ ಚತುರ, ಅವನ ಧೈರ್ಯ ಮತ್ತು ಸಮಯ ಸ್ಫೂರ್ತಿ\-ಗಳು ಅನ್ಯಾದೃಶವಾದವು. ಸಮಸ್ಯೆಯನ್ನು ಪರಿಶೀಲಿಸಿ ಅದರ ಪರಿಹಾರವನ್ನು ಸುಸೂತ್ರ\-ವಾಗಿ ನಿರ್ವಹಿಸುವ ನೈಪುಣ್ಯ ಅವನಲ್ಲಿದೆ. ಪಾಠಶಾಲೆಯ ವಾರ್ಷಿಕೋತ್ಸವ ಮುಗಿಯು\-ತ್ತಿದ್ದಂತೆ ಇಬ್ಬರು ವಿದ್ಯಾರ್ಥಿಗಳಲ್ಲಿ ಹೊಡೆದಾಟ ಬಿತ್ತು. ಅವರಲ್ಲಿ ಒಬ್ಬ ಹೊರಗಡೆಯ ಕೆಲವು ಗೂಂಡಾಗಳನ್ನು ಕರೆಸಿಕೊಂಡಿದ್ದ. ಅವರು ಚಾಕು ಚೈನಗಳಿಂದ ಸಜ್ಜಿತರಾಗಿದ್ದರು. ಆ ಕ್ಷಣದಲ್ಲಿ ಗಂಗಾಧರ ಜಾಗ್ರತನಾದ. ಕಾಲೇಜಿನ ಗೇಟು ಹಾಕಿ ಗೂಂಡಾಗಳ ಪೆಟ್ಟಿನಿಂದ ಮುಗ್ಧ ಹುಡುಗರನ್ನ ರಕ್ಷಿಸಿದ. ಆಮೇಲೆ ಪೋಲಿಸ್ ಕಂಪ್ಲೆಂಟ್ ಮೊದಲಾದವು\break ನಡೆದವು. ಹತ್ತು ರೂಪಾಯಿ ಕಿಸೆಯಲ್ಲಿ ಇಲ್ಲದೆ ಕಾನೂನಿನ ಎಲ್ಲ ವ್ಯವಹಾರಗಳನ್ನು ಗಂಗಾಧರ   ನಿರ್ವಹಿಸುವ ವಿಧಾನ ಈಗಲೂ ಅಚ್ಚರಿಯನ್ನು ಹುಟ್ಟಿಸುತ್ತದೆ. ಅಷ್ಟೇ ಅಲ್ಲ ; ಹೊಡೆದಾಟ ಮಾಡಿದ ಹುಡುಗ ಸಭ್ಯನಾಗಿ ಗಂಗಾಧರನ ಸ್ನೇಹ, ವಿಶ್ವಾಸಗಳಿಗೆ ಕೈ ಚಾಚಿದ.  ತನ್ನ ಗೂಂಡಾಗಿರಿಯನ್ನು ಬಿಟ್ಟು ಬಿಟ್ಟ. ಅವನ ಸಮಯ ಸ್ಫೂರ್ತಿಯ ಬಗ್ಗೆ ಆಲೋಚಿಸುವಾಗ ಇನ್ನೊಂದು ಘಟನೆ ನೆನಪಿಗೆ ಬರದೆ ಇರಲಾರದು. ವೇದಶಾಸ್ತ್ರ\break ಪೋಷಣೀ ಸಭೆಯಲ್ಲಿ ನಮಗೆ ಎರಡು ಹೊತ್ತಿನ ಊಟ. ಸಭೆಯ ಪಕ್ಕದ ಮನೆಯಲ್ಲಿ ಎರಡು ಮೂರು ಹುಡುಗರು ನಮಗಿಂತ ಸ್ವಲ್ಪ ದೊಡ್ಡವರೆ ಇದ್ದರು. ನಮ್ಮವರಲ್ಲಿ ಒಬ್ಬ ಅವರನ್ನು ನೋಡಿ ನೆಲಕ್ಕೆ ತಿರಸ್ಕಾರದಿಂದ ಉಗುಳಿದ ಎಂಬ ಕ್ಷಲ್ಲಕ ಕಾರಣಕ್ಕೆ ಅವರು ಜಗಳ ಎತ್ತಿದರು. ನಾವೂ ನಾಲ್ಕು ಐದು ಜನ ಇದ್ದೆವು. ಈ ಧೈರ್ಯದಿಂದ ನಾವೂ\break ಕೂಗಾಡಿದೆವು. ಅವರು ಕೆಟ್ಟ ಕೆಟ್ಟ ಬೈಗುಳಗಳನ್ನು ಉಗುಳುತ್ತ ನುಗ್ಗಿ ಹೊಡೆದಾಡಲು\break ಬಂದೇ ಬಿಟ್ಟರು. ಅವರು ಸ್ಥಳೀಯರು. ಬೇಕಾದರೆ ಆ ಬೀದಿಯ ಸಹಾಯವನ್ನು ಪಡೆಯ\-ಬಹುದು. ಇದನ್ನು ಪರಾಂಬರಿಸಿದ ಗಂಗಾಧರ ಒಂದು ಉಪಾಯವನ್ನು ಮಾಡಿದ ಅವರ ಒಂದು ಬೈಗುಳವೂ ನನಗೆ ತಿಳಿಯಲೇ ಇಲ್ಲ ಎಂದು ನಟಿಸಿ, ಅವರ ಮನೆಯ ತಾಯಂದಿರಲ್ಲಿ ವಿಷಯ ತಿಳಿಸಿ ಸಹಾನುಭೂತಿಗಳಿಸಿದ. ಅವರ ಮದ್ಯಸ್ಥಿಕೆ\-ಯಲ್ಲಿ ಹೊಡೆದಾಟ ತಪ್ಪಿತು. ಕಾಲಕ್ರಮೇಣ ಸ್ನೇಹವೂ ಬೆಳೆಯಿತು. 

ತನ್ನ  ಮನಸ್ಸಿಗೆ ಬಂದ ವಿಷಯವನ್ನು ಬೇರೆಯವರಿಗೆ ಮನದಟ್ಟು ಮಾಡುವಲ್ಲಿ ಅವನು ನಿಸ್ಸೀಮ. ಪ್ರದೋಷ ಪೂಜೆಯ ಕಾಲದಲ್ಲಿ ಎಲ್ಲ ವಿದ್ಯಾರ್ಥಿಗಳು ಉತ್ತರೀಯವನ್ನು ಹೊದೆದು ಬ್ರಹ್ಮವಸ್ತ್ರದಲ್ಲಿ ಬರಬೇಕು ಎಂದು ನಮ್ಮ ಇಂಗಿತವಾಗಿತ್ತು. ನಾಲ್ಕಾರು ವಿದ್ಯಾರ್ಥಿಗಳು ಇದಕ್ಕೆ ಅಪವಾದ ಆಗಿದ್ದರು. ಅವರನ್ನು ವಿಚಾರಿಸಲಾಯಿತು. ತಮ್ಮಲ್ಲಿ ಉತ್ತರೀಯವೇ ಇಲ್ಲ; ಅದನ್ನು ಸದ್ಯಕ್ಕೆ ಖರೀದಿಸಲು ಸಾದ್ಯವಿಲ್ಲ ಎಂದು ಅವರು\break ಉತ್ತರಿಸಿದರು. ಗಂಗಾಧರ ಈ ಸಂಗತಿಯ ಸತ್ಯಾಸತ್ಯತೆಯನ್ನು ಪರೀಕ್ಷಿಸಲಿಲ್ಲ.\break ಎಲ್ಲಿಂದಲೋ ನಾಲ್ಕಾರು ಉತ್ತರೀಯಗಳನ್ನು ಸಂಪಾದಿಸಿ ಅವರ   ಉಪಯೋಗಕ್ಕೆ ಅವು\-ಗಳನ್ನು ಕಾಯ್ದು ಇರಿಸಿದ. ಮುಂದಿನ ಪ್ರದೋಷ ಪೂಜೆಗಳಲ್ಲಿ  ಯಾರೋಬ್ಬರೂ ಶರ್ಟ ಹಾಕಿಕೊಳ್ಳದೆ ಬ್ರಹ್ಮವಸ್ತ್ರದಲ್ಲಿಯೇ ಬರುವಂತೆ ಆಯಿತು. ವ್ಯವಹಾರವನ್ನು ಗೆಲ್ಲುವ ಸಲುವಾಗಿ ಸರಿಯಾದ ಸಮಯಕ್ಕಾಗಿ ಕಾಯಬೇಕು ಎಂದು ಅವನ ಆಶಯ. ಪ್ರದೋಷ ಸಂಘದಲ್ಲಿ ನನಗಾದ  ಅನಾನುಕೂಲವನ್ನು ಸರಿಮಾಡಲು ಅವನ ಸಲಹೆಯಂತೆ ಒಂದು ವರ್ಷದವರೆಗೆ ಕಾಯಬೇಕಾಯಿತು. ವಾರ್ಷಿಕ ಸರ್ವಸಾಧಾರಣ ಸಭೆಯ ದಿನ ಅದನ್ನು ಸರಿಪಡಿಸಿ ಎಂದಿನಂತೆ ನನ್ನ ಪ್ರಾತಿನಿಧ್ಯವನ್ನು ಊರ್ಜಿತಗೊಳಿಸಿಕೊಳ್ಳುವಂತಾಯಿತು.  ಆ ವ್ಯವಹಾರದ ಗೆಲುವಿನಲ್ಲಿ ಅವನದೇ ಪೂರ್ಣಪಾತ್ರ. 

ಇಷ್ಟರಲ್ಲಿ ಗಂಗಾಧರನ ಹೋರಾಟಮಯ ಜೀವನ ಪ್ರಾರಂಭವಾಗಿತ್ತು. ಅವನು  ಬಿ.ಬಿ.ಎಮ್ ಕೋರ್ಸನ್ನು ಬಿಟ್ಟು  ಬಿ.ಕಾಮ್ ಗೆ ಬರುವಂತಾಯಿತು. 'ವಿದ್ವನ್ಮಧ್ಯಮಾ' ಪರೀಕ್ಷೆಯ  ದಿನಕರೀಯ ಪತ್ರಿಕೆಯಲ್ಲಿ ದುರುದ್ದೇಶ ಪೂರ್ವಕವಾಗಿ ಅವನನ್ನು ಅನು\-ತ್ತೀರ್ಣನನ್ನಾಗಿ ಮಾಡಿದರು. ಇದರಿಂದ ನವೀನನ್ಯಾಯ ರೆಗ್ಯೂಲರ್ ಕೋರ್ಸನ್ನು ಬಿಟ್ಟು ಹೊರಬರಬೇಕಾಯಿತು. ಈ ಸಂಗತಿಯನ್ನು ಪಂಥಾಹ್ವಾನವಾಗಿ ಗಂಗಾಧರ ತೆಗೆದುಕೊಂಡ. ವಿದ್ಯಾರ್ಥಿನಿಲಯದಿಂದ ಹೊರ ಬಂದು ಪ್ರತ್ಯೇಕ ಮನೆಮಾಡಿದ.

ಊರಿಂದ ತಂಗಿ ತಮ್ಮಂದಿರನ್ನು ಕರೆಸಿಕೊಂಡು ಅವರ ವಿದ್ಯಾಭ್ಯಾಸಕ್ಕೆ ಆಶ್ರಯನಾಗಿ ನಿಂತ. ಶಂಕರ ವಿಲಾಸ ಸಂಸ್ಕೃತ ಪಾಠಶಾಲೆಯ ಅಭಿವೃದ್ಧಿಯ ಕಾರ್ಯಗಳನ್ನು ಕೈಗೆತ್ತಿಕೊಂಡ. ತಾನು ಸಹಾಯಕ ಉಪಾಧ್ಯಾಯನಾಗಿ  ಸೇರಿಕೊಂಡರೂ ಪಾಠಶಾಲೆಯ ಪಾಠ\-ಪ್ರವಚನಗಳಿಂದ  ತೊಡಗಿ ಅನುದಾನ, ಸಂಬಳ-ಸಾರಿಗೆಗಳವರೆಗೆ ಹೊಣೆಗಾರಿಕೆಯನ್ನು ಹೊತ್ತುಕೊಂಡ. ಆಮೇಲೆ ಅಲ್ಲಿಯೇ ಮುಖ್ಯೋಪಾಧ್ಯಾಯನೂ ಆಗಿ ಸೇವೆಯನ್ನು  ಮುಂದುವರೆಸಿದ. ಈಗ ಈ ಸಂಸ್ಥೆ ಒಂದು ಸ್ವತಂತ್ರ ಸಂಸ್ಕೃತ ಕಾಲೇಜು ಆಗಿ ಬೆಳೆದಿದೆ. ಹತ್ತಾರು ಸಂಸ್ಕೃತ ತಜ್ಞರ ಜೀವನ ಯಾತ್ರೆಗೆ ಆಶ್ರಯವಾಗಿ ನಿಂತಿದೆ. ಈ ಸಂಸ್ಥೆಯ\break ಪ್ರಗತಿಯ ಪ್ರತಿಯೊಂದು ಹೆಜ್ಜೆಯಲ್ಲೂ ಗಂಗಾಧರನ ದೂರದೃಷ್ಟಿಯ ಪರಿಶ್ರಮವಿದೆ. 

ವಿದ್ಯಾರ್ಥಿಗಳ ಬದುಕಿನ ಜೊತೆಗೆ ವಿಕಟವಾಗಿ ಆಟವಾಡುವ ಕೀಳುಮನಸ್ಸಿನ\break ಪರೀಕ್ಷಕರನ್ನು  ಪರೀಕ್ಷಾಮಂಡಳಿಯಿಂದ ಹೊರಗುಳಿಸುವ ಹೋರಾಟ ಗಂಗಾಧರನ\-ದಾಗಿತ್ತು.  ಅದರಲ್ಲಿ ಅವನು ಪೂರ್ತಿಯಾಗಿ ಯಶಸ್ವಿಯಾದ. ಮತ್ತೊಮ್ಮೆ ಪರೀಕ್ಷೆಯನ್ನು  ಎದುರಿಸಿ ಉತ್ತಮ ಶ್ರೇಣಿಯಲ್ಲಿ ಉತ್ತೀರ್ಣನಾದ. ಅಷ್ಟರಲ್ಲಿ ಪ್ರಾಧ್ಯಾಪಕ\-ರಾದ ಪಿ. ಶ್ರೀನಾಥಾಚಾರ್ಯರು ನಿವೃತ್ತರಾಗಿದ್ದರು. \-ಶ್ರೀರಾಮಭದ್ರಾಚಾರ್ಯರೂ ನಿವೃತ್ತ\-ರಾಗಿ  ಕಾರಣಾಂತರದಿಂದ ಕೇರಳ ಕಡೆಗೆ ಪ್ರಯಾಣಿಸಿದರು. ಇದರಿಂದ \-ವಿದ್ವದುತ್ತಮಾ ತರಗತಿಯಿಂದ ಪಾಠಗಳಿಗಾಗಿ ನಾನು ಮತ್ತು ಶಂಕರ ಪ್ರಾಂಶುಪಾಲರಾದ ಶ್ರೀ ಈ.ಶ.ವರದಾಚಾರ್ಯರನ್ನೂ, ವೇದಾಂತ ಪ್ರಾಧ್ಯಾಪಕರಾದ ಶ್ರೀ ಕೆ.ನಾರಾಯಣ ಭಟ್ಟರನ್ನೂ ಆಶ್ರಯಿಸಿದೆವು. ಗಂಗಾಧರ ವಿಶೇಷ ಪಾಠಗಳಿಗಾಗಿ ಪಂಡಿತರತ್ನಂ ಕೆ.ಎಸ್. ವರದಾ\-ಚಾರ್ಯರಲ್ಲಿ ಶಿಷ್ಯವೃತ್ತಿಯನ್ನು ಮುಂದುವರೆಸಿದ. 

ಇಂಥ ಹೋರಾಟಗಳಿಂದ ಗಂಗಾಧರನಿಗೆ ಕೆಲವು ಲಾಭಗಳೂ ಆದವು. ಹಿರಿಯ ಪಂಡಿತರ ಮಾರ್ಗದರ್ಶನದಲ್ಲಿ ತನ್ನ ವ್ಯಾಸಂಗವನ್ನು ತಾನೇ ಮುಂದುವರೆಸಿ ಸ್ವಾವಲಂಬಿ\-ಯಾದ. ಸಮಕಾಲದಲ್ಲಿ ನೂರಾರು ಮಕ್ಕಳಿಗೆ ಪಾಠಮಾಡಿ ಒಳ್ಳೆಯ ಅಧ್ಯಾಪಕ ಎಂಬ ಕೀರ್ತಿಗೆ ಭಾಜನನಾದ. ಬಿ.ಬಿ.ಎಮ್ ಮತ್ತು ಎಮ್. ಎ. ಪರೀಕ್ಷೆಗಳು ಒಂದು ಕಡೆಯಾದರೆ ಉತ್ತಮ ಅಂಕಗಳ ಉತ್ತಮ ಶ್ರೇಣಿಯ ನವೀನನ್ಯಾಯ ವಿದ್ವತ್ ಇನ್ನೊಂದು ಕಡೆ. ಇವುಗಳ ಜೊತೆಗೆ ಲೆಕ್ಕಾಚಾರ, ಮುಂದೋರಣೆ, ಪ್ರಯತ್ನಪೂರ್ವಕವಾದ ಪ್ರಾಯೋಗಿಕ ಸಾಮಾಜಿಕ   ಜೀವನಾನುಭವಗಳ ಸರಮಾಲೆ ಅವನಿಗೆ ಒಲಿದು ಬಂದಿದ್ದವು. 

ಸ್ವಾಭಿಮಾನ ಗಂಗಾಧರನ ಸ್ವಭಾವ. ಸ್ವಾವಲಂಬನ ಅವನ ಶೀಲ. ಕಷ್ಟದ ಕಾಲ\-ದಲ್ಲಿಯೇ ಈ ಗುಣಗಳು ಬೆಳಗುವುದು. ವ್ಯಕ್ತಿತ್ವದ ಬೆಳಕಾಗಿ ಬೆಳೆಯುವುದು. ಪಾಠಶಾಲೆಯ    ಉಪಾಧ್ಯಾಯ ವೃತ್ತಿಯಿಂದ ಸಂಪಾದನೆ ಸಾಲದು ಎಂದು ಯೋಚಿಸಲಿಲ್ಲ. ಅದಕ್ಕಾಗಿ ಸಾಲ ಮಾಡುವ ದಾರಿಯನ್ನೂ ಹುಡುಕಲಿಲ್ಲ. ಆಸೆಯನ್ನು ಕಡಿಮೆ ಮಾಡಿಕೊಂಡ. ಖರ್ಚಿಗೆ ಕಡಿವಾಣ ಹಾಕಿದ. ಸಂಪಾದನೆಗೆ ಅಕ್ರಮದ ಮಾರ್ಗವನ್ನು ಹುಡುಕುವ ಜಾಯಮಾನವೇ ಅವನದಲ್ಲ. ಹಾಗೆ ಹುಡುಕಿದವರನ್ನು ಹತ್ತಿರ ಸುಳಿಯಲೂ\break ಬಿಡುವವನಲ್ಲ. ಪ್ರತಿ ದಿನ ನಾಲ್ಕಾರು ಮಂದಿ ಮನೆಯವರೆ, ಊಟ ತಿಂಡಿ ಎಲ್ಲ\break ಸಾಗಬೇಕು. ಅಗ್ಗೆರೆ ಮನೆತನ ಆದರೋಪಚಾರಕ್ಕೆ ಹೆಸರುವಾಸಿ. ಮೈಸೂರಿನ ಬಾಡಿಗೆ ಮನೆಯೂ ಗಂಗಾಧರನಿಗೆ ಅಗ್ಗೆರೆಯೇ ಆಗಿ ಹೋಯಿತು. ಬರುವವರೆಷ್ಟು! ಉಳಿದುಕೊಳ್ಳುವವರೆಷ್ಟು ! ಎಲ್ಲರಿಗೂ ಕಾಪಿ, ತಿಂಡಿ, ಊಟ, ಉಪಚಾರಗಳು, ಆಯಾ ಹೊತ್ತಿಗೆ ಯಾರಿದ್ದರೂ ಸರಿ 'ತಿಂಡಿಗೆ ಏಳಿ', 'ಊಟ ಮಾಡೋಣ' ಆತ್ಮೀಯ ಆಮಂತ್ರಣ\break ತಪ್ಪದು.  ಮೈಸೂರಿನಲ್ಲಿ ಮನೆ ಇದ್ದು, ವಾಸ ಮಾಡಿ, ಅನುಕೂಲವಿದ್ದವರೂ ಆ ಹೊತ್ತಿ\-ನಲ್ಲಿ ಗಂಗಾಧರನಿಗೆ ಅತಿಥಿಗಳೆ. ನಾನಂತೂ ಮೈಸೂರಿಗೆ ಹೋದಾಗಲೆಲ್ಲಾ\break ಗಂಗಾಧರನ ಮನೆಗೆ ಹೋಗಿ ಆತಿಥ್ಯವನ್ನು ಪಡೆಯುತ್ತಿದ್ದವನೆ. ನನಗದು ಮತ್ತೊಂದು ಮನೆ. ಪ್ರತಿಸಾರಿಯೂ ನನಗೆ ಆಶ್ಚರ್ಯ ಕಾದಿರುತ್ತಿತ್ತು. ಊರಿನ ಕಡೆಯವರು ಯಾರಾದರೂ ಅಥಿತಿಗಳಾಗಿ ಇರುತ್ತಿದ್ದರು. ಸಿದ್ದಾಪುರ, ಶಿರಸಿ, ಯಲ್ಲಾಪುರ, ಕುಮಟಾ, ಹೊನ್ನಾವರ, ಸಾಗರ ತಾಲೂಕಿನ ಹವ್ಯಕ ಬಾಂಧವರಿಗೆ ಗಂಗಾಧರನ ಮನೆ 'ಮೈಸೂರಿನ ಕಾಶಿ' ಯಾವ ಕಾರಣಕ್ಕಾಗಿ  ಅವರು ಮೈಸೂರಿಗೆ ಬಂದಿದ್ದರೂ ಗಂಗಾಧರನ ಮನೆಗೆ ಬರುವದಕ್ಕೆ\break ಕಾರಣವೇ ಬೇಡ. ಈ ಸಂಸ್ಕಾರ ಅವನ ತಂಗಿಯರಲ್ಲೂ  ಮನೆ ಮಾಡಿತ್ತು. ಅವರು ಮನೆಗೆ ಬಂದವರನ್ನು ಆದರದಿಂದ ಉಪಚರಿಸಿದರು. ಕಷ್ಟಪಟ್ಟು ಓದಿ ಮುಂದೆ ಬಂದರು. ಈಗ ಮದುವೆಯಾಗಿ ತಮ್ಮ ತಮ್ಮ ಮನೆಗಳಲ್ಲಿ ಆದರ್ಶ ಗೃಹಿಣಿಯರಾಗಿ ಜೀವನ ಮಾಡುತ್ತಿದ್ದಾರೆ. ಇರಲಿ, ಕಷ್ಟದ ಕಾಲದಲ್ಲಿ, ಇಷ್ಟೊಂದು ಖರ್ಚುವೆಚ್ಚಗಳು ಇರುವಾಗ ಬ್ರಹ್ಮಚಾರಿಯಾಗಿದ್ದು, ತಾನೂ ತನ್ನ ಓದನ್ನು ಮುಂದುವರೆಸುತ್ತ,  ತನ್ನ ಆಶ್ರಯದಲ್ಲಿ  ಇದ್ದ ತಂಗಿಯರನ್ನೂ, ತಮ್ಮಂದಿರನ್ನೂ, ಅಣ್ಣನ ಮಕ್ಕಳನ್ನೂ ಓದಿಸುತ್ತ, ಬಂದವರನ್ನೂ ಪ್ರೀತಿಯಿಂದ ಉಪಚರಿಸುತ್ತ  ಹೇಗೆ ಗಂಗಾಧರ ಕುಟುಂಬವನ್ನು 'ನಿರ್ವಹಿಸಿದ' ? ಎಂಬ ಸಂಗತಿ ನಿಬ್ಬೆರ\-ಗನ್ನು ಹುಟ್ಟಿಸುವ ಪ್ರಶ್ನೆಯಾಗಿ ಉಳಿದಿದೆ. 

ಗಂಗಾಧರನಿಗೆ ಶ್ರೀಮನ್ಮಹಾರಾಜ ಸಂಸ್ಕೃತ  ಕಾಲೇಜಿನ ನವೀನನ್ಯಾಯ  ಸಹಾಯಕ ಪ್ರಾಧ್ಯಾಪಕ ಹುದ್ದೆ ದೊರಕಿತು. ಅದು ಆ ಸಂಸ್ಥೆಯ ಭಾಗ್ಯ ಎಂದು ತಿಳಿದ\-ವರು ಸಂತಸಪಟ್ಟರು. ಅವನು ಮದುವೆಯಾದ. ಶ್ರೀಮತಿ ಶೈಲಜಾ ಅವನಿಗೆ ಅನುರೂಪಳಾದ ಮಡದಿಯಾದಳು. ಈಗ ಅವನ ಸ್ವಾವಲಂಬನಕ್ಕೆ ಸಂಬಳದ ಬಲ ಕೂಡಿ ಬಂತು. ಅವನ ಸ್ವಾಭಿಮಾನಕ್ಕೆ ಸಂಪನ್ನತೆಯ ತೇಜಸ್ಸು ಆವರಿಸಿತು. ಅತಿಥಿ ಸತ್ಕಾರಕ್ಕೆ\break ನಿರ್ದಾಕ್ಷಿಣ್ಯದ ಮೆರಗು ಬಂತು. ಅವನ ಜೀವನ ಮೊದಲಿನಂತೆ ಮುಂದುವರೆದಿದೆ. ಪ್ರೀತಿಯಿಂದ ಓದಿದ   ಶಾಸ್ತ್ರವನ್ನು ಪಾಠ ಮಾಡುವ ಯೋಗ ಅವನ ಪ್ರಯೋಗ ಕುತೂಹಲ\-ವನ್ನು ಹೊಮ್ಮಿಸಿದೆ. ಸಂಸ್ಕೃತ ಮಹಾಪಾಠಶಾಲೆಯಲ್ಲಿ  ಉತ್ಸಾಹದ ಹೊಸಯುಗ ಪ್ರಾರಂಭವಾಯಿತು. ಹವ್ಯಕ ಮಕ್ಕಳು ಎಲ್ಲಾ ಶಾಸ್ತ್ರಗಳನ್ನು ತುಂಬಿಕೊಂಡರು. ಅವರಿಗೆಲ್ಲಾ ಗಂಗಾಧರನೆ ಮಾರ್ಗದರ್ಶಕ. ಕೆಲವರಿಗೆ ಪ್ರತ್ಯಕ್ಷ ಗುರು. ಕೆಲವರಿಗೆ ಸ್ಪರ್ಧೆ ಮೊದಲಾದವುಗಳಿಗೆ ಪ್ರೇರಕ. ಮತ್ತೆ ಕೆಲವರಿಗೆ ಊಟ ವಸತಿಗಳಿಗೆ ವ್ಯವಸ್ಥಾಪಕ. ಈ\break ಅವಧಿಯಲ್ಲಿ ಅವನ ಆಚಾರ್ಯತ್ವ ಬೆಳಗಿತು.  ಬೆಳಗಿನಿಂದ ರಾತ್ರಿಯವರೆಗೆ ಶಾಲೆಯಲ್ಲೂ, ಮನೆಯಲ್ಲೂ ಪಾಠಮಾಡಿದ. ತರ್ಕಶಾಸ್ತ್ರ, ವ್ಯಾಕರಣ, ಕಾವ್ಯ, \hbox{ನಾಟಕಗಳು,} ಆಯುರ್ವೇದ, ಅರ್ಥಶಾಸ್ತ್ರ, ಇತಿಹಾಸಪುರಾಣಗಳು .....ಒಂದೇ....ಎರಡೇ ಹಲವು ವಿಷಯಗಳನ್ನು ಬೋಧಿಸಿದ. ಸ್ವಾಧ್ಯಾಯ ಪ್ರವಚನಗಳು ಎಡಬಿಡದೆ ನಡೆದವು. ನೂರಾರು ವಿದ್ಯಾರ್ಥಿಗಳು ಉಪಕೃತರಾದರು. ಬೆಳಕಿನ ಬಾಳನ್ನು ಕಟ್ಟಿಕೊಂಡರು. 

ಕಾಲ ಸರಿಯುತ್ತಿರುವುದು  ಗೊತ್ತಾಗುವುದಿಲ್ಲ. 2018 ಜನವರಿಯಲ್ಲಿ \hbox{ಗಂಗಾಧರನಿಗೆ} ಸರಕಾರಿ ಸೇವೆಯಿಂದ ನಿವೃತ್ತಿ. ನನಗೂ ಇದೇ ವರ್ಷ ಜುಲೈ 31ರಂದು ಸೇವೆಯಿಂದ ನಿವೃತ್ತಿ. ನಾವಿಬ್ಬರೂ ಮೈಸೂರಿಗೆ ಬಂದು ನಾಲ್ವತ್ತಮೂರು ವರ್ಷಗಳು ಕಳೆದವು. ಮೈಸೂರು ನಮ್ಮ ಮನಸ್ಸನ್ನು ಬೆಳೆಸಿದೆ. ನಮಗೆ ಅನ್ನವನ್ನು ಕೊಟ್ಟಿದೆ. ನಮ್ಮ ಬದುಕನ್ನು ಕಟ್ಟಿಕೊಟ್ಟಿದೆ. ಇದಕ್ಕಿಂತ ಮಿಗಿಲಾಗಿ ಸಂಸ್ಕೃತ ಸಂಬಂಧವನ್ನೂ, ಜೀವನದ ಇಹ\enginline{-}ಪರಗಳ ಮರ್ಮವನ್ನು ಬೋಧಿಸಿದ ಪ್ರಾತಸ್ಮರಣೀಯ ಶ್ರೀರಾಮಭದ್ರಾಚಾರ್ಯರಂಥ\break ಆಚಾರ್ಯರನ್ನೂ, ಶಾಸ್ತ್ರಸಂಬಂಧವನ್ನೂ ದಯಪಾಲಿಸಿದೆ. ಸದಾ ಮನಸ್ಸನ್ನು ಜಾಗೃತವಾಗಿ ಇರಿಸುವ ಸ್ನೇಹ\enginline{-}ಸಾಂಗತ್ಯವನ್ನೂ ನೀಡಿದೆ. 

ಗಂಗಾಧರನಲ್ಲಿ ಅವನ ತಂದೆ ವಿಘ್ನೇಶ್ವರ ಭಟ್ಟರ ವ್ಯವಹಾರ ಚಾತುರ್ಯ,\break ಸ್ವಾವಲಂಬನ ಮತ್ತು ಸ್ವಾಭಿಮಾನಗಳು ಬದುಕಿ ಉಳಿದಿವೆ. ಅವನ ದೊಡ್ಡಪ್ಪ  ಮಹಾ\-ಬಲೇಶ್ವರ ಭಟ್ಟರ ಪಾಂಡಿತ್ಯ ಮುಂದುವರೆದಿದೆ.  ಅವನ ಅಣ್ಣ ಮಂಜುನಾಥ ಭಟ್ಟರ ಪ್ರತಿಭೆ ಮತ್ತು ಸೌಮನಸ್ಸುಗಳು ಬೆಳಗುತ್ತಿವೆ. ಆದುದರಿಂದ ಗಂಗಾಧರ ದೀರ್ಘಕಾಲ ಬಾಳಲಿ. ಅವನಿಗೆ ಆರೋಗ್ಯ ಭಾಗ್ಯ ಕೂಡಲಿ. ಆನಂದ ಅವನಿಗೆ ಸ್ವರೂಪವಾಗಲಿ. ಈ ಸದಾಶಯದೊಂದಿಗೆ ನಮ್ಮ ಬಹುಕಾಲದ ಒಡನಾಟದ ನೆನಪುಗಳ ಗಂಟನ್ನು ಜತನದಿಂದ ಕಟ್ಟಿ ಕಾಪಿಡುತ್ತಿದ್ದೇನೆ.

\articleend

}
