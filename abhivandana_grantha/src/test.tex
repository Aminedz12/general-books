
\begin{spacing}{1.1}
{\fontsize{15}{17}\selectfont\dev{जगतः सृष्टेः प्रारम्भे अग्नि-वायु-आदित्य-अङ्गिरा इति चतुर्ण्णाम् ऋषीणां {\fontsize{14}{15}\selectfont\kan{ಮೈಸೂರಿನ ಮಹಾರಾಜ ಸಂಸ್ಕೃತ ಪಾಠಶಾಲೆಯಲ್ಲಿ }} हृदये समुद्भूताः ऋग्यजुस्सामाथर्वेति नामधेयैः ख्यातिं गताः चत्वारो वेदाः अधुना लोके समुल्लसन्ति~। चतुर्षु वेदेष्वपि अयमथर्ववेदः क्रमानुरोधेन चतुर्थः~। चत्वारोऽपि वेदाः समानपदभाज एव~। न हि तेषु महत्वविषयकं तारतम्यं पदं धत्ते~। चतुर्षु वेदेषु अथर्ववेदोऽसौ यद्यपि चतुर्थः अथापि चतुर्ण्णां वेदानां मध्ये कर्मकाण्डदृष्ट्या अतिशयितं पदमालम्बते~। कर्मकाण्डे तु यज्ञयागादिश्रौतस्मार्तकार्यनिर्वहणाय चत्वारः ऋत्विजः होता अध्वर्युः उद्गाता ब्रह्मा इति क्रमशः विद्यन्ते~। तेषु प्रमुखस्थाने अथर्ववेदज्ञः ब्रह्मा एव प्रमुखं विशिष्टतमं च स्थानं भवितुमर्हति नान्यः~। \textbf{यज्ञैरथर्वा प्रथमः पथस्तते} (ऋ.सं.१-८३-५) इति ऋग्वेदे एव उक्तत्वात्~। } {\fontsize{12}{11}\selectfont\linespread{1}\eng{There is so much one can learn just listening to Gangadhar Bhat}}} {\fontsize{15}{17}\selectfont\dev{जगतः सृष्टेः प्रारम्भे अग्नि-वायु-आदित्य-अङ्गिरा इति चतुर्ण्णाम् ऋषीणां {\fontsize{14}{15}\selectfont\kan{ಮೈಸೂರಿನ ಮಹಾರಾಜ ಸಂಸ್ಕೃತ ಪಾಠಶಾಲೆಯಲ್ಲಿ }} हृदये समुद्भूताः ऋग्यजुस्सामाथर्वेति नामधेयैः ख्यातिं गताः चत्वारो वेदाः अधुना लोके समुल्लसन्ति~। चतुर्षु वेदेष्वपि अयमथर्ववेदः क्रमानुरोधेन चतुर्थः~। चत्वारोऽपि वेदाः समानपदभाज एव~। न हि तेषु महत्वविषयकं तारतम्यं पदं धत्ते~। चतुर्षु वेदेषु अथर्ववेदोऽसौ यद्यपि चतुर्थः अथापि चतुर्ण्णां वेदानां मध्ये कर्मकाण्डदृष्ट्या अतिशयितं पदमालम्बते~। कर्मकाण्डे तु यज्ञयागादिश्रौतस्मार्तकार्यनिर्वहणाय चत्वारः ऋत्विजः होता अध्वर्युः उद्गाता ब्रह्मा इति क्रमशः विद्यन्ते~। तेषु प्रमुखस्थाने अथर्ववेदज्ञः ब्रह्मा एव प्रमुखं विशिष्टतमं च स्थानं भवितुमर्हति नान्यः~। \textbf{यज्ञैरथर्वा प्रथमः पथस्तते} (ऋ.सं.१-८३-५) इति ऋग्वेदे एव उक्तत्वात्~। } {\fontsize{12}{11}\selectfont\linespread{1}\eng{There is so much one can learn just listening to Gangadhar Bhat}}} 
\end{spacing}
\vskip 5pt
%~ \dev{अथर्ववेदे २० काण्डास्सन्ति~। ७३१ सूक्तानि ५९८७ मन्त्राश्च अत्र विद्यन्ते~। एषु मन्त्रेषु द्वादशशतमिता मन्त्रास्तु ऋग्वेदेऽपि उपलभ्यन्ते~। अथर्ववेदस्य प्रायशः अशीतिमितानि सूक्तानि गद्यरूपाणि~। अस्य वेदस्य षष्ठांशमितो भागः गद्यात्मक एव~। }


\begin{spacing}{1.1}
{\fontsize{12}{14}\selectfont\linespread{1}\eng{There is so much one can learn just listening to Gangadhar Bhat Sir. I have personally imbibed a whole lot more than my work at Cycle Pure Agarbathies required. His thoughts are so evolved that he never imposes his beliefs. On the contrary,this gentleman teacher is also happy to learn if he must.}}
\end{spacing}
\vskip 5pt

%~ {\fontsize{16}{19}\selectfont\dev{इति चतुर्ण्णाम् ऋषीणां हृदये समुद्भूताः}} {\fontsize{13}{15}\selectfont\eng{There is so much one can learn just listening}} {\fontsize{14}{16}\selectfont\kan{ಪಾಠಶಾಲೆಯಲ್ಲಿ ಕೃಷ್ಣಯಜುರ್ವೇದ}} 


\begin{spacing}{1.1}
{\fontsize{14}{16}\selectfont\kan{ಮೈಸೂರಿನ ಮಹಾರಾಜ ಸಂಸ್ಕೃತ ಪಾಠಶಾಲೆಯಲ್ಲಿ ಕೃಷ್ಣಯಜುರ್ವೇದ ಪ್ರಥಮ ತರಗತಿಗೆ ಸೇರಿ, ಹಿಂದಿನ ರೂಮಿನಲ್ಲಿ ವಾಸಿಸುತ್ತಿದ್ದರು. ಮೈಸೂರಿನ ಶ್ರೀರಾಮಮಿಶ್ರ ಸಂಸ್ಕೃತ ಪಾಠಶಾಲೆಯಲ್ಲಿ ಪ್ರೈವೇಟಾಗಿ ಕಾವ್ಯ ಪರೀಕ್ಷೆ ಮುಗಿಸಿ, ಅನಂತರ ಮೈಸೂರು ಮಹಾರಾಜ ಸಂಸ್ಕೃತ ಮಹಾಪಾಠಶಾಲೆಯಲ್ಲಿ ಸಾಹಿತ್ಯ ತರಗತಿಗೆ ಸೇರಿ ಸಾಹಿತ್ಯ ಪರೀಕ್ಷೆಯನ್ನು ಮುಗಿಸಿ, ಅನಂತರ ತರ್ಕಶಾಸ್ತ್ರದ ನವೀನನ್ಯಾಯ ವಿದ್ವತ್ತಿಗೆ ಸೇರಿದರು. ವಿದ್ವಾನ್ ಶ್ರೀನಾಥಾಚಾರ್ಯರು, ವಿದ್ವಾನ್ ಎ. ವೆಂಕಣ್ಣಾಚಾರ್ಯರು, ವಿದ್ವಾನ್ ಎನ್.ಎಸ್.ರಾಮಭದ್ರಾಚಾರ್ಯರು ಇವರ ತರ್ಕಶಾಸ್ತ್ರದ ವಿದ್ಯಾಗುರುಗಳು. ಓದುತ್ತಿದ್ದ ಕಾಲದಲ್ಲಿ ತುರುವೇಕೆರೆ ವಿದ್ವಾನ್ ವಿಶ್ವೇಶ್ವರದೀಕ್ಷಿತರಲ್ಲೂ ಓದಿ ತಿಳಿದುಕೊಳ್ಳುವ ಅವಕಾಶ ಇವರಿಗೆ ಒದಗಿತು. }}
\end{spacing}
\vskip 5pt
%~ \kan{ವಿದ್ವಾನ್ ಎನ್.ಎಸ್.ರಾಮಭದ್ರಾಚಾರ್ಯರಿಂದ ತರ್ಕಶಾಸ್ತ್ರದಲ್ಲಿ ತುಂಬಾ ವಿಷಯಗಳನ್ನು ತಿಳಿದುಕೊಂಡು ಪ್ರಖರವಾದ ಪಾಂಡಿತ್ಯವನ್ನು ಸಂಪಾದಿಸಿದರು. ಅವರ ಪಾಠಪ್ರವಚನ, ಮಾರ್ಗದರ್ಶನಗಳು ಗಂಗಾಧರಭಟ್ಟರಿಗೆ ಬಹಳವಾಗಿ ದೊರಕಿದವು. ವಿದ್ವಾನ್ ಎನ್.ಎಸ್.ರಾಮಭದ್ರಾಚಾರ್ಯರ ಬಗ್ಗೆ ಗಂಗಾಧರಭಟ್ಟರಿಗೆ ಅಪಾರ ಗೌರವ–ಭಕ್ತಿ.}

\eject
%~ \eng{Despite being a traditionalist he taught and guided me well enough to collate some Hindu prayer instructions. His contributions significant in preserving our prayer rituals and tradition, and+ appropriately guiding the emerging generations world over.}

