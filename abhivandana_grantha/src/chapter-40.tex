\chapter{ಜಪಯಜ್ಞ}

\begin{center}
\Authorline{ವಿ. ಪರಮೇಶ್ವರ}
\smallskip

ಪುಟ್ಟನಮನೆ\\ 
ಶಿರಸಿ
\end{center}

\begin{verse}
ಆರಂಭಯಜ್ಞಾಃ ಕ್ಷತ್ರಸ್ಯ ಹವಿರ್ಯಜ್ಞಾ ವಿಶಃ ಸ್ಮೃತಾಃ |\\
ಪರಿಚಾರಯಜ್ಞಾಃ ಶೂದ್ರಾಶ್ಚ ಜಪಯಜ್ಞಾ ದ್ವಿಜಾತಯಃ ||
\end{verse}
ಇತ್ತೀಚಿನ ದಿನಗಳಲ್ಲಿ ರೋಗರುಜಿನಗಳ ಸಂಖ್ಯೆ ಹೆಚ್ಚಾಗುತ್ತಿದೆ. ದಿನಕ್ಕೊಂದರಂತೆ ಹೊಸ ಹೊಸ ರೋಗಲಕ್ಷಣಗಳು ಕಾಣಿಸಿಕೊಳ್ಳುತ್ತಿವೆ. ಹಾಗೆಂದು ಹಿಂದಿನ ಕಾಲದಲ್ಲಿ ಖಾಯಿಲೆಗಳು ಇರಲೇ ಇಲ್ಲ ಎಂದರ್ಥವಲ್ಲ! ಆಗ ಪೌಷ್ಟಿಕತೆಯ ಕೊರತೆಯಿಂದ ಬಹಳಷ್ಟು ರೋಗಗಳು ಬರುತ್ತಿದ್ದವು. ಇನ್ನು ಬಂದ ರೋಗವೂ ಸರಿಯಾದ ಉಪಚಾರವಿಲ್ಲದೇ ಮಾರಣಾಂತಿಕವಾಗಿ ಬಿಡುತ್ತಿತ್ತು. ಆದರೆ ಈಗ ಹಾಗಿಲ್ಲ, ಒಂದೆರಡು ರೋಗಗಳನ್ನು ಬಿಟ್ಟರೆ ಬಹುತೇಕವಾಗಿ ರೋಗಗಳನ್ನೆಲ್ಲಾ ಗುಣಪಡಿಸಬಹುದಾಗಿದೆ. ಆದರೂ ಶೇಕಡಾ ೪೦ರಷ್ಟು ರೋಗಿಗಳು ಪೂರ್ಣವಾಗಿ ಗುಣಮುಖರಾಗುತ್ತಿಲ್ಲ! ಎಂದರೆ, ’ಎಲ್ಲ ಇದ್ದೂ ಏನೂ ಇಲ್ಲ! ’ ಎಂಬಂತಾಗಿದೆ.

ಆಧುನಿಕ ಜೀವನ ಕ್ರಮದಲ್ಲಿ ಒತ್ತಡವೇ ಹಲವು ರೋಗಗಳಿಗೆ ಮೂಲವಾಗಿದೆ. ಜೊತೆಗೆ ಅವೈಜ್ಞಾನಿಕ ಆಹಾರ ಪದ್ಧತಿ, ಚಿಂತೆ ಮತ್ತು ಸ್ವಾರ್ಥಗಳೂ ನೀರೆರೆದು ರೋಗಗಳನ್ನು ಬೆಳೆಸುತ್ತಿವೆ. ಸ್ವಾರ್ಥದ ಬಲದಿಂದ “ ತನ್ನ ಮನೆತನ ಮತ್ತು ಸಂಪ್ರದಾಯಗಳನ್ನು ತಿರಸ್ಕರಿಸಿ ನಡೆದವರಲ್ಲಿ ಮತ್ತು ಅವರ ಮುಂದಿನ ಪೀಳಿಗೆಯಲ್ಲಿ ಹೃದಯ ಸಂಬಂಧಿಯಾದ ಖಾಯಿಲೆಗಳು ಹೆಚ್ಚಾಗಿರುತ್ತವೆ.” ಎಂದು ಒಂದು ಅಧ್ಯಯನದ ವರದಿ ತಿಳಿಸುತ್ತದೆ. ಅದೇ ರೀತಿ ’ತಾನು’ ಎಂಬ ಪದವನ್ನು ಹೆಚ್ಚು ಹೆಚ್ಚು ಬಳಸುವವರಲ್ಲೂ ಇದೇ ಅಂಶ ಕಂಡು ಬಂದಿದ್ದು ಸ್ಪಷ್ಟವಾಗಿದೆ. ತನ್ನನ್ನು ಬಿಟ್ಟು ಉಳಿದವರ ಅನುಕೂಲ, ಆಸ್ತಿ, ಅಂತಸ್ತುಗಳೊಂದಿಗೆ ತನ್ನ ಹೋಲಿಸಿ ಕೊಂಡು ಚಿಂತೆಯಿಂದ ಮನಶ್ಶಾಂತಿ ಕೆಡಿಸಿ ಕೊಂಡವರಿಗೇನು ಕೊರತೆಯೇ? ಅಂಥವರು ನಮ್ಮ ಸುತ್ತಲೂ ಬಹುವಾಗಿ ಕಾಣಸಿಗುತ್ತಾರೆ. ಒಟ್ಟಿನಲ್ಲಿ ಎಲ್ಲದಕ್ಕೂ ಕಾರಣ ಈ ಮನಸ್ಸು! ಎಂಬಲ್ಲಿಗೆ ನಾವು ನಿಲ್ಲಬೇಕಾಗುತ್ತದೆ. ಮನಸ್ಸು
\begin{verse}  
ಮನ ಏವ ಮನುಷ್ಯಾಣಾಂ ಕಾರಣಂ ಬಂಧ ಮೋಕ್ಷಯೋಃ |
\end{verse}
ಮನಸ್ಸು ಜಗತ್ತಿನ ಅತ್ಯದ್ಭುತಗಳಲ್ಲಿ ಒಂದು. ಕಣ್ಣಿಗೆ ಕಾಣದೆ, ಕೈಗೆ ಸಿಗದೆ ಅದು ಅನುಭವಕ್ಕೆ ಮಾತ್ರ ಸಿಗುತ್ತದೆ. ಅದರ ಆಳ ವಿಸ್ತಾರವನ್ನು ಅರಿತವರಿಲ್ಲ. ಮನಸ್ಸನ್ನು ಅರಿತುಕೊಳ್ಳಲು ಅದರ ಕಾರ್ಯವೈಖರಿಯನ್ನು ತಿಳಿದುಕೊಳ್ಳಲು ಅನಾದಿ ಕಾಲದಿಂದ ಮನುಷ್ಯ ಪ್ರಯತ್ನ ಪಡುತ್ತಲೇ ಇದ್ದಾನೆ. ಮನಸ್ಸಿನ ರಚನೆ, ಸಾಮರ್ಥ್ಯವನ್ನು ತಿಳಿಯಲು ಜ್ಞಾನಿಗಳು, ಮನೋ ವಿಜ್ಞಾನಿಗಳು ಸತತವಾಗಿ ಅಧ್ಯಯನ ನಡೆಸುತ್ತಿದ್ದಾರೆ. ಮನಸ್ಸು ಹೃದಯದಲ್ಲಿದೆ. ಮನಸ್ಸು ಮಿದುಳಿನಲ್ಲಿದೆ. ಮನಸ್ಸು ದೇಹದ ಹೊರಗಡೆ ಇದೆ. ಅದು ಭೌತಿಕದೇಹದ ಹತೋಟಿಯಲ್ಲಿಲ್ಲ! ಎಂದು ಒಬ್ಬೊಬ್ಬರು ಒಂದೊಂದು ರೀತಿಯಲ್ಲಿ ಹೇಳುವುದನ್ನು ಕೇಳಿದ್ದೇವೆ. ಆದರೆ ವೈದ್ಯಕೀಯ ವಿಜ್ಞಾನದ ಪ್ರಕಾರ ಮನಸ್ಸಿನ ಸ್ಥಾನ ಮಿದುಳು. ನರಮಂಡಲದ ಕೇಂದ್ರವೇ ಮಿದುಳು. ಕೋಟಿ ಕೋಟಿ ನರಕೋಶಗಳ ಸಮೂಹದಿಂದ, ಅಷ್ಟೇ ಅಸಂಖ್ಯಾತ ನರತಂತುಗಳ ಜಾಲದಿಂದ ಮಿದುಳು ಮನಸ್ಸಿನ ವಿವಿಧ ಕಾರ್ಯಗಳನ್ನು ನಿಯಂತ್ರಿಸಿ ನಿರ್ದೇಶಿಸುತ್ತದೆ.

\textbf{ಮನಸ್ಸು ಒಂದು ಜಾಲ! ಇದು ಒಂದು ಮಂಡಲ!} ಈ ಮನೋ ಮಂಡಲವಿಲ್ಲದ ದೇಹ, ಕೇವಲ ತೊಗಲು, ಮಾಂಸ, ಮೂಳೆಗಳ ಗೊಂಬೆ. ಅದು ಅರ್ಥವಿಲ್ಲದೇ ಹೋಗುತ್ತದೆ. ಹಾಗಾಗಿ ಈ ದೇಹದ ಮೌಲ್ಯ ಹೆಚ್ಚುವುದು ಮನಸ್ಸು, ಬುದ್ಧಿಗಳಿಂದ. ಎಂಬಲ್ಲಿಗೆ ಮನಸ್ಸಿನ ಮಹತ್ವ ಸ್ಪಷ್ಟಗೊಳ್ಳುತ್ತದೆ.

ಮನಸ್ಸಿಗೆ ಇರುವ ಆಯಾಮಗಳು ಅಸಂಖ್ಯ. ಪ್ರಜ್ಞೆ, ಬುದ್ಧಿ, ಯೋಚನೆ, ಚಿಂತನೆ, ಅಂತಃಕರಣ, ಕನಸು ಹೀಗೆ ಬಹು ವಿಧದಲ್ಲಿ ಮನಸ್ಸು ಕೆಲಸ ಮಾಡುತ್ತಿರುತ್ತದೆ. ಆದರೆ ಇವಾವುದನ್ನೂ ‘ಹೀಗೇ’ ಎಂದು ಖಚಿತವಾಗಿ ವ್ಯಾಖ್ಯಾನಿಸಲು ಸಾಧ್ಯವಿಲ್ಲ. ಆದರೂ ಮನಸ್ಸನ್ನು ಎರಡು ಭಾಗಗಳಾಗಿ ವಿಂಗಡಿಸಬಹುದು. ಹೊರ ಅಥವಾ ಜಾಗ್ರತ ಮನಸ್ಸು. ಒಳ ಅಥವಾ ಸುಪ್ತ ಮನಸ್ಸು. ಜಾಗೃತ ಮನಸ್ಸಿನಲ್ಲಿ ನಡೆಯುವ ಎಲ್ಲಾ ಚಟುವಟಿಕೆಗಳು ನಮಗೆ ಅರಿವಿದ್ದು ಅವುಗಳ ಮೇಲೆ ನಮಗೆ ಹತೋಟಿಯಿರುತ್ತದೆ. ಅದೇ ರೀತಿ ಸುಪ್ತ ಮನಸ್ಸಿನಲ್ಲಿ ನಡೆಯುವ ವಿದ್ಯಮಾನಗಳು ಮತ್ತು ಅದು ಹೀರಿಕೊಳ್ಳುವ ವಿಷಯಗಳು ನಮ್ಮ ಅರಿವಿಗೆ ಬರುವುದಿಲ್ಲ! ಅವುಗಳ ಮೇಲೆ ನಮಗೆ ಸ್ವಲ್ಪವೂ ಹತೋಟಿಯಿರುವುದಿಲ್ಲ. ನಮ್ಮ ಜಾಗ್ರತ ಮನಸ್ಸಿಗೆ ಆತಂಕ, ಬೇಸರ, ನೋವುಂಟುಮಾಡುವ ವಿಷಯ, ಆಸೆ, ಅನುಭವಗಳನ್ನು ನಾವು ಸುಪ್ತಮನಸ್ಸಿನೊಳಗೆ ತಳ್ಳಿಬಿಡುತ್ತೇವೆ. ಅವುಗಳೇ ಮುಂದೆ, ಸಂದರ್ಭ ಒದಗಿದಾಗ, ಒತ್ತಡ ನಿರ್ಮಾಣವಾದಾಗ, ಮನೋಬೇನೆಗಳಾಗಿ ಪರಿವರ್ತನೆ ಹೊಂದುವದು. ಎಂಬುದಾಗಿ ಸಿಗ್ಮಂಡ್‌ ಫ್ರಾಯ್ಡ್ ಅಭಿಪ್ರಾಯ ಪಡುತ್ತಾನೆ.

\textbf{ಉದಾಹರಣೆಗೆ} – ಕೆರೆಯ ತಳದಲ್ಲಿ ಕಲ್ಲು, ಕೆಸರು, ಕೊಳೆಯುತ್ತಿರುವ ಪಾಚಿ, ಜೈವಿಕವಸ್ತುಗಳಿದ್ದರೂ ಮೇಲೆ ಮಾತ್ರ ನೀರು ತಿಳಿಯಾಗಿರುತ್ತದೆ. ಒಂದು ಕಲ್ಲು ಅಥವಾ ಉದ್ದನೆಯ ಕೋಲು ಹಾಕಿ ಕಲಕಿದರೆ ಅಥವಾ ಮೀಟಿದರೆ ತಳದಲ್ಲಿರುವ ವಸ್ತುಗಳು, ಮೇಲಕ್ಕೆ ಬಂದು ಕಣ್ಣಿಗೆ ಗೋಚರವಾಗುವುದಲ್ಲದೆ, ತಿಳಿ ನೀರನ್ನು ರಾಡಿ ಮಾಡುತ್ತದೆ. ಹೀಗೆಯೇ ಯಾವುದೋ ಒಂದು ಘಟನೆ ಅಥವಾ ಸಮಸ್ಯೆ ಜಾಗೃತ ಮನಸ್ಸನ್ನು ಕೆಣಕಿದಾಗ ಅದು ಮೀಟುಗೋಲಿನಂತೆ ಕೆಲಸ ಮಾಡಿ, ಸುಪ್ತ ಮನಸ್ಸಿನಲ್ಲಿ ಹುದುಗಿರುವ ಯಾವುದೋ ನೋವನ್ನು ಅಥವಾ ಸಮಸ್ಯೆಯನ್ನು ಕೆದಕಿ ಮೇಲೆ ತರುತ್ತದೆ. ಹೀಗೆ ಮೇಲೆ ಬಂದ ವಸ್ತು ಮೊದಲಿನಂತಿರದೆ, ವಿಕೃತ ಅಥವಾ ಬೇರೆ ರೂಪದಲ್ಲಿದ್ದು, ಅದೇನೆಂಬುದು ವ್ಯಕ್ತಿಗೆ ಅರ್ಥವಾಗದೇ ಮನೋಮಾಲಿನ್ಯವಾಗಬಹುದು. ಇಂಥ ಮನೋಮಾಲಿನ್ಯವನ್ನು ನಿವಾರಿಸುವಲ್ಲಿ ನಮ್ಮ ಪೂರ್ವಜರು ತೋರಿದ ಮಾರ್ಗವೇ ಕರ್ಮಮಾರ್ಗ. ಇದರ ಉದ್ದೇಶವನ್ನು ಆದಿಶಂಕರರು “ಚಿತ್ತಸ್ಯ ಶುದ್ಧಯೇ ಕರ್ಮ ನ ತು ವಸ್ತೂಪಲಬ್ಧಯೇ” | ಎಂದು ಸ್ಪಷ್ಟಪಡಿಸಿದ್ದಾರೆ.

ಪ್ರಕೃತಿಯಲ್ಲಿ ಕಂಡ ಪ್ರಕ್ರಿಯೆಗಳನ್ನೆಲ್ಲಾ ನಮ್ಮ ಪೂರ್ವಜರು ಯಜ್ಞದೃಷ್ಟಿಯಿಂದಲೂ ನಿರೂಪಿಸಿದ್ದಾರೆ. ಇಂಥ ಯಜ್ಞದೃಷ್ಟಿಗೆ ಗೋಚರಿಸುವ ತತ್ವಗಳು ಎರಡೇ ಎರಡು; ಒಂದು ಹವ್ಯಾಶ, ಇನ್ನೊಂದು ಹವಿಸ್ಸು. ಹೀಗೆ ಗೋಚರವಾದ ಯಜ್ಞತತ್ವವನ್ನು ವರ್ಣಕ್ರಮದಲ್ಲಿ ನಾಲ್ಕು ಭಾಗಗಳನ್ನು ಮಾಡಲಾಗಿ - ಮೊದಲನೆಯದು ಆರಂಭಯಜ್ಞ - ಅಂದರೆ ರಾಜಕಾರ್ಯ ಮತ್ತು ರಾಜತಂತ್ರ (ಯುದ್ಧ) ಇವುಗಳಲ್ಲಿ ನಿರತವಾಗುವುದು. ಎರಡನೆಯದು ದ್ರವ್ಯಯಜ್ಞ - ಕಾಮನೆಗಳ ಪೂರ್ತಿಗೆ ಮಾಡುವ ಕಾರ್ಯ ಕಲಾಪ. ಮೂರನೆಯದು ಪರಿಚಾರಯಜ್ಞ - ಸೇವಾರೂಪ ಜೀವನ ನಿರ್ವಹಣೆ ಅಥವಾ ಸಹಾಯಕ ವೃತ್ತಿ. ನಾಲ್ಕನೆಯ ಮತ್ತು ಕೊನೆಯದಾಗಿರುವ ಜಪಯಜ್ಞ - ಇದು ಅತ್ಯಂತ ಪ್ರಮುಖವೂ, ಪ್ರಶಸ್ತವೂ ಆಗಿದ್ದು, 
\begin{verse}
“ಯಜ್ಞಾನಾಂ ಜಪಯಜ್ಞೋಽಸ್ಮಿ ಸ್ಥಾವರಾಣಾಂ ಹಿಮಾಲಯಃ | 
\end{verse}
ಎನ್ನುವ ಮೂಲಕ ಶ್ರೀಕೃಷ್ಣನಿಗೆ ಅಚ್ಚು ಮೆಚ್ಚಿನದಾಗಿತ್ತು. “ 
\begin{verse}
ಸರ್ವೇ ತೇ ಜಪಯಜ್ಞಸ್ಯ ಕಲಾಂ ನಾರ್ಹಂತಿ ಷೋಡಶೀಮ್  || 
\end{verse}
ಎಂಬ ವಚನದಂತೆ ’ಜಪಯಜ್ಞ ಮನುವಿಗೂ ಮಾನ್ಯವಾಗಿತ್ತು’ ಎಂದು ಮನದಟ್ಟಾಗುತ್ತದೆ. \textbf{ಜಪತಾಂ ವರಮ್ | ಜಪತಾಂ ಶ್ರೇಷ್ಠಃ |} ಎಂಬುದಾಗಿ ರಾಮಾಯಣ ಮತ್ತು ಮಹಾಭಾರತಗಳಲ್ಲಿ ಬಂದದ್ದನ್ನು ಗಮನಿಸಿದರೆ, ವ್ಯಾಸವಾಲ್ಮೀಕಿಯರು ವ್ಯಕ್ತಿಯ ಔನ್ನತ್ಯವನ್ನು ವ್ಯಕ್ತಗೊಳಿಸಲು, ಅಳತೆಗೋಲಾಗಿ ಜಪದಲ್ಲಿ ಭಾಗಿಯಾಗುವದನ್ನು ಎತ್ತಿ ತೋರಿಸಿರುತ್ತಾರೆ. ಹೀಗೆ ಜಪಯಜ್ಞದ ಪ್ರಾಶಸ್ತ್ಯ ಪ್ರಕಟವಾಗುತ್ತದೆ.

\textbf{“ಜಪ”} ಇದು ಸಂಸ್ಕೃತದ \textbf{ಜಪ ವಾಚಿ ವ್ಯಕ್ತಾಯಾಂ} ಎಂಬ ಧಾತುವಿನಿಂದ ಮತ್ತು \textbf{ಜಪ ಮಾನಸೇ} ಎಂಬ ಧಾತುವಿನಿಂದಲೂ ಶಬ್ದ ಜಗತ್ತನ್ನು ಪ್ರವೇಶಿಸಿದೆ. \textbf{ಜಲ್ಪ ವಾಚಿ ವ್ಯಕ್ತಾಯಾಂ} ಎಂಬ ಧಾತುವನ್ನೂ ಗಮನಿಸಿದಾಗ ಬಾಯಿಂದ ಹೊರ ಬಂದ ಮೇಲೆ ಜಪ ಮತ್ತು ಜಲ್ಪ ಎರಡೂ ಒಂದೇ! ಎಂಬುದು ಜಿಜ್ಞಾಸೆಯ ಸಂಗತಿ. ಆದರೆ ಮನೋಭೂಮಿಕೆಯೆಲ್ಲಿ ಜಲ್ಪಕ್ಕೆ ಅವಕಾಶವಿಲ್ಲ; ಅಲ್ಲಿ ಜಪವೇ ಜಯಭೇರಿ ಬಾರಿಸುತ್ತದೆ ಎಂಬುದು ಗಮನಾರ್ಹ ಅಂಶ.
\begin{verse}
"ಜ" ಕಾರೋ ಜನ್ಮ ವಿಚ್ಚೇದಃ "ಪ" ಕಾರಃ ಪಾಪನಾಶಕಃ |\\
ತಸ್ಮಾಜ್ಜಪ ಇತಿ ಪ್ರೊಕ್ತೋ ಜನ್ಮಪಾಪ ವಿನಾಶಕಃ ||
\end{verse}
ಎಂದೆಲ್ಲಾ ಹೇಳಿ ಜಪವನ್ನು ಇವತ್ತಿಗೂ ವೈಭವೀಕರಿಸುತ್ತಿರುವುದನ್ನು ನೋಡಿದರೆ; ’ಜಪ’ ಎಂಬ ಎರಡು ಅಕ್ಷರಗಳಿಗೆ ಜಗದ ಎಲ್ಲ ಧರ್ಮಗಳೂ, ಮತಗಳೂ, ಪಂಥಗಳೂ ನತ ಮಸ್ತಕವಾಗಿರುವುದನ್ನು ಕಂಡರೆ; ’ಜಪದ ಅರ್ಹತೆ’ ಏನೆಂಬುದು ಸ್ಪಷ್ಟವಾಗಿ ಗೋಚರಿಸುತ್ತದೆ.

|“\textbf{ಜಪಸ್ಸ್ಯಾದಕ್ಷರಾವೃತ್ತಿಃ} “ ಕೆಲವೊಂದು ವಿಶಿಷ್ಟ ರೀತಿಯ ಶಬ್ದ ಸಂಯೋಜನೆಯುಳ್ಳ ಸಾಲುಗಳನ್ನು ಹಲವು ಬಾರಿ ಪುನರಾವರ್ತಿಸುವುದರಿಂದ ಅಥವಾ ಅಕ್ಷರಾವೃತ್ತಿಯಿಂದ ಶಕ್ತಿ ಸಂಚಯವಾಗುತ್ತದೆಂದು ವಿಜ್ಞಾನವೂ ಈಗ ಒಪ್ಪಿಕೊಂಡಿದೆ. ನಾವಾಡುವ ಮಾತೂ ಕೂಡ ಶಕ್ತಿ ರೂಪವೇ ಆಗಿದೆ. ಪ್ರತಿಯೊಂದು ವಸ್ತುವಿನಿಂದ ಶಕ್ತಿಯನ್ನು ಪಡೆಯುವಂತೆ, ಪ್ರತಿಯೊಂದು ಶಕ್ತಿಯಿಂದಲೂ ಮತ್ತೊಂದು ಶಕ್ತಿ ಅಥವಾ ವಸ್ತುವನ್ನು ಪಡೆಯಲು ಸಾಧ್ಯವಿದೆ. ಸೂರ್ಯನ ಬಿಸಿಲು ನಮಗೆ ಬೆಳಕು ಮತ್ತು ಬೇಗೆಗಳನ್ನು ನೀಡುವುದು. ಅದೇ ಬಿಸಿಲು ಗಿಡಗಳ ಪಾಲಿನ ಆಹಾರವೂ ಹೌದು ತಾನೆ? ಹೇಗೆ ಇಂದಿನ ಮಾನವ ಅದೇ ಬಿಸಿಲಿನಿಂದ ವಿದ್ಯುತ್ ಮತ್ತು ತಾಪ ಶಕ್ತಿಯನ್ನು ಪಡೆಯುವಲ್ಲಿ ಸಫಲನಾಗಿದ್ದಾನೋ, ಹಾಗೆಯೇ ಶಬ್ದ ತರಂಗಗಳನ್ನು ಬಳಸಿ, ಇನ್ನೂ ಬಹಳಷ್ಟನ್ನು ಸಾಧಿಸ ಬಹುದಾಗಿದೆ. ಈಗಾಗಲೇ ಶಬ್ದತರಂಗಗಳು ಶಸ್ತ್ರಕ್ರಿಯೆಗಾಗಿ ಬಳಕೆಯಾಗುತ್ತಿರುವುದನ್ನು ಇಲ್ಲಿ ಮುಖ್ಯವಾಗಿ ಗಮನಿಸಬಹುದಾಗಿದೆ.

ಪರಮಾತ್ಮನನ್ನು ಮಂತ್ರದ ರೂಪದಲ್ಲಾಗಲೀ, ನಾಮದ ರೂಪದಲ್ಲಾಗಲೀ ಆವರ್ತಿಸುವುದೇ "ಜಪ". ಕೆಲವರು ಜೋರಾಗಿ ನಾಮ ಸ್ಮರಣೆ ಮಾಡುತ್ತಾ, ಬೇರೆಯವರ ಕಿವಿಗೂ ಬೀಳುವಂತೆ ಮಾಡುತ್ತಾರೆ. ಈ ರೀತಿ ಮಾಡುವುದರಿಂದ ಪಠಿಸುವವರಿಗೂ, ಕೇಳುವವರಿಗೂ ಏಕ ಕಾಲದಲ್ಲಿ ಫಲ ದೊರಕುತ್ತದೆ. ಅಮೇರಿಕಾದ ನರತಜ್ಞ ಎಡ್ವರ್ಡ್ ಮೆಕ್ಲಿನೋ, ಕ್ರೈಸ್ತ ಮಿಷನರಿಗಳ ಜೊತೆ ಅಮೇಜಾನ್-ನ ಕೇಂದ್ರಗಳಲ್ಲಿ ಕಾರ್ಯನಿರ್ವಹಿಸುತ್ತಿದ್ದಾಗ ಆಳವೊಂದಕ್ಕೆ ಬಿದ್ದು ತೀವ್ರವಾಗಿ ಗಾಯಗೊಂಡು ಸೊಂಟದ ಕೆಳಗೆ ಹತೋಟಿಯನ್ನೇ ಕಳಕೊಂಡ. ಅವರನ್ನು ಅಲ್ಲಿ ಬಿಟ್ಟು ಹೋದ ಹೆಲಿಕಾಪ್ಟರ್ ಮುಂದೆ ಮೂರು ದಿನಗಳ ನಂತರ ಮತ್ತೆ ಬರುವುದಿದ್ದರೂ, ಆ ತನಕ ಆತನ ಪ್ರಾಣವುಳಿಯುವ ಸಾಧ್ಯತೆಯಿರಲಿಲ್ಲ. ಆದರೂ ಸ್ವತಃ ನರ ತಜ್ಞನಾದ ಆತ ಮೊದಲು ಆದಿವಾಸಿಗಳಿಂದ ಚಿಕಿತ್ಸೆಗೆ ಒಪ್ಪಲಿಲ್ಲ! ಮುಂದೆ ಪರಿಸ್ಥಿತಿ ಕೈ ಮೀರಿದ ಕಾರಣ, ಸಾಯುವ ಬದಲು ಚಿಕಿತ್ಸೆಗೆ ಒಪ್ಪಿಕೊಂಡ. ಆದಿವಾಸಿಗಳು ಒಂದಿಷ್ಟು ಎಲೆಗಳನ್ನು ಉರಿಸಿ, ಅದೇನೋ ಶಬ್ದಗಳನ್ನು ಪದೇ ಪದೇ ಉಚ್ಚರಿಸುತ್ತಿರುವುದಷ್ಟೇ ಆತನಿಗೆ ನೆನಪಿದೆ. 

ಹಲವು ಗಂಟೆಗಳ ನಂತರ ಪ್ರಜ್ಞೆ ಬಂದಾಗ ಆತ ಮೊದಲಿನಂತಾಗಿದ್ದ. ನೋವು ಪೂರ್ತಿ ವಾಸಿಯಾಗಿತ್ತು. ಹಾಗಾಗಿ ಜಪ ಮಾಡುವವನನ್ನು ಅನುಸರಿಸದಿದ್ದರೂ ಕೇಳುಗನಿಗೆ ಪರಿಣಾಮವಾಯಿತೆಂದರೆ; ಅದನ್ನನುಸರಿಸಿದರೆ ಫಲನಿಶ್ಚಯದಲ್ಲಿ ಸಂಶಯವಿಲ್ಲವಷ್ಟೇ?

ಜಪಿಸುವ ಯಾವುದೇ ಮಂತ್ರವಾಗಲೀ, ಅದನ್ನು ಗುರುಗಳಿಂದ, ಜ್ಞಾನಿಗಳಿಂದ ಅಥವಾ ಕರ್ಮನಿಷ್ಠರಿಂದ ಉಪದೇಶ ಪಡೆಯ ಬೇಕೆಂಬ ನಿಯಮವಿದೆ. ಇದು ನಿಯಮ ಮಾತ್ರವಾಗಿರದೇ, ಮಂತ್ರೋಚ್ಚಾರಣೆ ಸ್ಪಷ್ಟವಾಗಿ ನಡೆಯಬೇಕೆಂಬ ಕಾಳಜಿಯೂ ಇಲ್ಲಿದ್ದು, ಇದು ವಿಹಿತವೇ ಆಗಿದೆ. ಅಲ್ಲದೇ ಮಂತ್ರದ ಮಂತ್ರತ್ವವೂ ಅವರಿಂದಲೇ ಸಿದ್ಧಗೊಳ್ಳುವದರಿಂದ ಇದು ಅನಿವಾರ್ಯವೂ ಹೌದು. “ಮಂತ್ರ” ಯಾವುದು? ಎಂಬ ಸಂದರ್ಭ ಬಂದಾಗ, ಸಾಯಣಾಚಾರ್ಯರು ವೇದಭಾಷ್ಯಭೂಮಿಕೆಯಲ್ಲಿ ಹೀಗೆ ಹೇಳುತ್ತಾರೆ. “ಅಭಿಯುಕ್ತರು (ಶಿಷ್ಟರು, ಪ್ರಾಜ್ಞರು, ಕರ್ಮಾನುಷ್ಠಾತೃಗಳು) ಯಾವ ಮಾತನ್ನು ಅಥವಾ ಶಬ್ದ ಸಮೂಹವನ್ನು ಮಂತ್ರ ಎಂದು ಕರೆಯುತ್ತಾರೋ, ಆ ಸಮಾಖ್ಯಾನವೇ ಮಂತ್ರ ಲಕ್ಷಣವು.” 
\begin{verse}
ತಸ್ಮಾದಭಿಯುಕ್ತಾನಾಂ ಮಂತ್ರೋಽಯಮಿತಿ ಸಮಾಖ್ಯಾನಂ ಲಕ್ಷಣಮ್ ||
\end{verse}
’ಮಂತ್ರಗಳ ಆವೃತ್ತಿಯೇ ಜಪ’ ಎಂಬುದನ್ನೇ ಶಾಂಡಿಲ್ಯೋಪನಿಷತ್ತಿನಲ್ಲಿ ಹೀಗೆ ವ್ಯವಸ್ಥೆಗೊಳಿಸಿದ್ದಾರೆ. “\textbf{ವಿಧಿವದ್ಗುರೂಪದಿಷ್ಟವೇದಾವಿರುದ್ಧಮಂತ್ರಾಭ್ಯಾಸಃ ಜಪಃ}” || ಅಭ್ಯಾಸವೇ1 ಜಪ ಶಬ್ದಾರ್ಥವಾಗಿದ್ದರೂ, ಆ ಅಭ್ಯಾಸ “ಮಂತ್ರಾಭ್ಯಾಸವಾಗಿರಬೇಕು.” ಎಂಬಲ್ಲಿಗೆ ಮನ್ನಣೆಗೆ 2 ಒತ್ತು ನೀಡಲಾಯಿತು. ಇಲ್ಲದಿದ್ದರೆ ಪದೇ ಪದೇ ಹೇಳಿದ್ದೆಲ್ಲ ಜಪವಾಗಿ ಬಿಡುತ್ತಿತ್ತು. “ಜಗತ್ತಿನ ಒಳಿತನ್ನು ಲಕ್ಷ್ಯದಲ್ಲಿಟ್ಟು ಸಮಾಜಹಿತವನ್ನು ಸಾರುವ ವೇದ ಮಂತ್ರಗಳಿಗೆ ವಿರೋಧಿಯಾದ ಶಬ್ದಭಾಗ ವ್ಯಕ್ತಿಯಲ್ಲಿ ಪ್ರವಾಹ ವಿರುದ್ಧತೆಯನ್ನೂ ನಕಾರಭಾವವನ್ನೂ ಹೆಚ್ಚಿಸೀತೆಂಬ ಕಾರಣದಿಂದ ಪೂಜ್ಯವಾದ ಮಂತ್ರಾಭ್ಯಾಸವು ವೇದಕ್ಕೆ ವಿರೋಧಿಯಾಗದೇ ಸಂವಾದಿಯಾಗಿರಬೇಕು” ಎಂದರು. ಹೀಗೆ ಮಹತ್ತರವಾದ ಮಂತ್ರಗಳ ಉಚ್ಚಾರ ಮತ್ತು ಆಚಾರ ಸಂಹಿತೆಗಳ ಬಗ್ಗೆಯೂ ಗಮನಕೊಟ್ಟು, ಅದು ಅಪದ್ಧವಾಗಬಾರದೆಂದು ಗುರುವಿನಿಂದ ಉಪದಿಷ್ಟವಾಗಿರಬೇಕೆಂದರು. ’ಇಂಥ ಉಪದೇಶ ಕಲ್ಪಗಳಲ್ಲೋ, ಸ್ಮೃತಿ, ಪುರಾಣಗಳಲ್ಲೋ, ತಂತ್ರಾಗಮಗಳಲ್ಲೋ ವಿಹಿತವಾಗಿರಬೇಕು.’ ಎಂದು ಸ್ಪಷ್ಟವಾಗಿಸಿದರು. ಇದರಿಂದ ’ನಾನೇ ಗುರು’ ಎಂದು ತಿರುಗುತ್ತಿರುವವರು, ಕಿವಿಯಲ್ಲಿ ಉಸುರಿದ್ದೆಲ್ಲ ಮಂತ್ರಜಪವಾಗಿಬಿಡುವುದನ್ನು ತಪ್ಪಿಸಿದ್ದಾರೆ. ಒಟ್ಟಿನಲ್ಲಿ “ವಿಹಿತವೂ, ಗುರೂಪದಿಷ್ಟವೂ, ಆಗಿದ್ದು; ವೇದಸಂವಾದಿಯಾದ ಮಂತ್ರಗಳ ಅಭ್ಯಾಸವೇ ಜಪ” ಎಂದು ನಿರ್ಣಯವಾಯಿತು.
\begin{verse}
"ಜಪ" ಎರಡು ತೆರನಾಗಿರುತ್ತದೆ. ವಾಚಿಕ ಜಪ ಮತ್ತು ಮಾನಸ ಜಪ.
\end{verse}
\textbf{(೧) ವಾಚಿಕ ಜಪ :} ಇದರಲ್ಲಿ ಮತ್ತೆ ಎರಡು ಪ್ರಕಾರಗಳು. ಒಂದು ಉಚ್ಚೈರ್ವಾಚಿಕ, ಇನ್ನೊಂದು ಉಪಾಂಶು ವಾಚಿಕ. ಮತ್ತೊಬ್ಬರ ಕಿವಿಗೆ ಬೀಳುವಂತೆ ದೊಡ್ಡದಾಗಿ ಪಠಿಸುವುದು. ಉಚ್ಚೈಃ ಪಾಠವಾದರೆ, ತುಟಿಗಳಲುಗುತ್ತಿದ್ದರೂ ಶಬ್ದವು ಹೊರಗೆ ಬಾರದಂತೆ ಜಪಿಸುವುದು ಉಪಾಂಶುವೆನಿಸುವುದು.

\textbf{(೨) ಮಾನಸ ಜಪ :} ಮನದಲ್ಲಿ ಧ್ಯಾನಿಸುವುದು. ಇದನ್ನು "ಮೌನ ಜಪ"ವೆಂದೂ ಕರೆಯುತ್ತಾರೆ. ವಾಚಿಕ ಜಪಕ್ಕಿಂತಲೂ ಸಾವಿರ ಪಾಲು ಶ್ರೇಷ್ಟವಾದದ್ದು ಉಪಾಂಶು ಜಪ. ಈ ಉಪಾಂಶುವಿಗಿಂತ ಹತ್ತುಸಾವಿರಪಾಲು ಹಿರಿದಾದದ್ದು, ಮಾನಸಿಕ ಜಪ. ಆದ್ದರಿಂದ ಮನೋಭೂಮಿಕೆಯಲ್ಲಿ ಜಪದ ಪಾತ್ರ ಅತ್ಯಂತ ಮಹತ್ವದ್ದಾಗಿದೆ. ಅದಕ್ಕೆಂದೇ ಯೋಗಧ್ಯಾನಗಳಂತೆ ಜಪದ ಪ್ರಕ್ರಿಯೆಯಲ್ಲೂ ಮಧ್ಯಮಾರ್ಗಕ್ಕೆ (ಎರಡು ಅತಿಗಳ ನಡುವಿನ ಬಿಂದುವಿಗೆ) ಒತ್ತುನೀಡಲಾಗಿದೆ. ಉದ್ರಿಕ್ತನಾಗದ; ನಿದ್ರೆಗೂ ಜಾರದ ಮಧ್ಯ ಮನಸ್ಸಿನಿಂದ ಮತ್ತು ಅತೀ ವೇಗವೂ ಅಲ್ಲದ; ಮಂದವೂ ಆಗದ ಮಧ್ಯಗತಿಯಲ್ಲಿ ಪಠಿಸುವುದು ಜಪದ ಮುಖ್ಯ ಇತಿ ಕರ್ತವ್ಯವಾಗಿದೆ. ಇದಕ್ಕೆ ಸಂವಾದಿಯಾಗಿ ಮಂತ್ರಾರ್ಥದಲ್ಲಿ ನೆಟ್ಟ ಮನಸ್ಸುಳ್ಳವನಾಗಿರಬೇಕೆನ್ನುತ್ತದೆ ತಂತ್ರಸಾರ ಗ್ರಂಥ.
\begin{verse}
ಮನಃ ಸಂಹೃತ್ಯ ವಿಷಯಾನ್ ಮಂತ್ರಾರ್ಥಗತಮಾನಸಃ |\\
ನ ದ್ರುತಂ ನ ವಿಲಂಬಂ ಚ ಜಪೇನ್ಮೌಕ್ತಿಕಪಂಕ್ತಿವತ್ ||
\end{verse}
ಮನಸ್ಸು ಅತ್ತಿತ್ತ ಹರಿದಾಡದಿರಬೇಕು ಎಂದರೆ ಜಪಸ್ಥಾನವೂ ಇದಕ್ಕೆ ಪೂರಕವಾಗಿರಬೇಕಾಗುತ್ತದೆ. ಈ ಕುರಿತು ಬೇರೆ ಬೇರೆ ಗ್ರಂಥಗಳು ಬೇರೆ ಬೇರೆ ಅಭಿಪ್ರಾಯವನ್ನೇ ನೀಡುತ್ತವೆ. ಒಟ್ಟಾರೆಯಾಗಿ ಗಮನಿಸಿದರೆ “ಎಲ್ಲಿ ಜಪಮಾಡುವವನ ಮನಸ್ಸು ನಿಲ್ಲುತ್ತದೆಯೋ ಅಲ್ಲಿ ಜಪ ಸಲ್ಲುತ್ತದೆ” ಎಂದು ಅರ್ಥವಾಗುತ್ತದೆ. ಅಂಥ ಸ್ಥಾನ ಅತೀ ಪ್ರಶಸ್ತ ಎನ್ನಬಹುದು. ಆದರೂ ಆ ಕುರಿತು ಮಂತ್ರಕೌಮುದಿಯಲ್ಲಿ ಒಂದು ವ್ಯವಸ್ಥೆಯನ್ನೇ ತೋರಿಸಿದ್ದಾರೆ.
\begin{verse}
ಗೃಹೇ ಜಪಃ ಸಮಃ ಪ್ರೋಕ್ತೋ ಗೋಷ್ಠೇ ದಶಗುಣಃ ಸ್ಮೃತಃ |\\
ಪುಣ್ಯಾರಣ್ಯೇ ತಥಾರಾಮೇ ಸಹಸ್ರಗುಣ ಈರಿತಃ ||\\
ಅಯುತಂ ಪರ್ವತೇ ಪುಣ್ಯೇ ನದ್ಯಾಂ ಲಕ್ಷಮುದೀರಿತಮ್ |\\
ಕೋಟಿಂ ದೇವಾಲಯೇ ಪ್ರಾಹುರನಂತಂ ಶಿವಸನ್ನಿಧೌ ||
\end{verse}
ಒಟ್ಟಿನಲ್ಲಿ “ಶಿವಸ್ಥಾನದಲ್ಲಿ ಪ್ರಶಾಂತಭಾವದಿಂದ ಮಂತ್ರಾರ್ಥದಲ್ಲಿ ಬುದ್ಧಿಯನ್ನಿಟ್ಟು ಮಧ್ಯಗತಿಯಿಂದ ಜಪಿಸಬೇಕು” ಎಂದಾಯಿತು. ಬರೀ ನೆಲದ ಮೇಲೆ ಕುಳಿತು ಜಪ ಮಾಡಬಾರದು. ಮರದ ಮಣೆ ಅಥವಾ ಚಾಪೆಯ ಮೇಲೆ ಪೂರ್ವಾಭಿಮುಖವಾಗಿ ಅಥವಾ ಉತ್ತರಾಭಿಮುಖವಾಗಿ ಕುಳಿತು ಏಕ ಮನಸ್ಸಿನಿಂದ ಜಪಿಸಬೇಕು. ಸಂಧ್ಯಾಕಾಲದಲ್ಲಿ ಅಥವಾ ಕಲ್ಪೋಕ್ತ ಕಾಲಮಿತಿಯಲ್ಲಿ ಜಪವನ್ನು ಕೈಗೊಳ್ಳಬೇಕು. ಹೀಗೆ ಜಪಾಚರಣೆಯ ದಿನಗಳಲ್ಲಿ ನಿಯಮಿತವೂ, ಹಿತವೂ, ಪರಿಶುದ್ಧವೂ ಆದ ಆಯಾ ದೇಶ, ಕಾಲ, ಪರಿಸ್ಥಿತಿಗನುಗುಣವಾಗಿ ಆಹಾರಕ್ರಮವನ್ನೂ ಹೊಂದಿರಬೇಕು. 
\begin{verse}
ಭಕ್ಷ್ಯಂ ಹವಿಷ್ಯಂ ದುಗ್ಧಾನ್ನಂ ಶಾಕಂ ಚ ದಧಿಯಾವಕಮ್ | \\
ಪಯೋ ಮೂಲಂ ಫಲಂ ವಾಪಿ ಯದ್ಯದ್ಯತ್ರೋಪಪದ್ಯತೇ || 
\end{verse}
ಹೀಗಿದ್ದರೂ ಎಷ್ಟೋ ಬಾರಿ ಜಪಕಾಲದಲ್ಲೇ ಆಹಾರ ಕ್ರಮ ವ್ಯತ್ಯಾಸವಾಗಿಬಿಡುವುದನ್ನು ನೋಡುತ್ತೇವೆ. ಹಾಗಾಗದಂತೆ ಎಚ್ಚರ ವಹಿಸಬೇಕಾದ್ದು ಅನಿವಾರ್ಯ. ಅದಕ್ಕೆಂದೇ ಗೌತಮನಂತೂ 
\begin{verse}
ತಸ್ಮಾನ್ನಿತ್ಯಂ ಪ್ರಯತ್ನೇನ ಶಸ್ತಾನ್ನಾಶೀ ಭವೇನ್ನರಃ || 
\end{verse}
ಎಂದು ವಿಶೇಷವಾಗಿ ಹೇಳಿದ್ದಾನೆ. ಜಪಮಾಡುವಾಗಿನ ಸ್ಥಾನ, ಕಾಲ ಮತ್ತು ಅನ್ನಗಳ ಬಗ್ಗೆ ನೋಡಿಯಾಯಿತು. ಈಗ ಜಾಪಕ ಧರ್ಮಗಳ ಬಗ್ಗೆ ಗಮನಿಸೋಣ. 
\begin{verse}
ಹೃದಯೇ ಹಸ್ತಮಾರೋಪ್ಯ ತಿರ್ಯಕ್ಕೃತ್ವಾ ಕರಾಂಗುಲಿಃ | \\
ಆಚ್ಛಾದ್ಯ ವಾಸಸಾ ಹಸ್ತೌ ದಕ್ಷಿಣೇನ ಸದಾ ಜಪೇತ್ || 
\end{verse}
ಬಲಗೈಯಲ್ಲಿ ಜಪಮಾಲೆ ಹಿಡಿದು ಎದೆ ಪರ್ಯಂತ ಎತ್ತಿ ಕೊಂಡು ಜಪಿಸಬೇಕು. ಜಪಮಾಲೆಯಲ್ಲಿ ನೂರಾ ಎಂಟು ಮಣಿಗಳಿರುತ್ತವೆ. ಈ ಮಣಿಗಳನ್ನು ತುಳಸಿ ಮೊದಲಾದ ಗಿಡ-ಮರಗಳ ಕಾಂಡದಿಂದ, ಇಲ್ಲವೆ ಚಿಕ್ಕ ರುದ್ರಾಕ್ಷಿಗಳನ್ನು ಪೋಣಿಸಿ, ಶಂಖಗಳನ್ನು ಜೋಡಿಸಿ ಅಥವಾ ಸ್ಫಟಿಕದ ಮಣಿಗಳನ್ನು ಪೊಣಿಸಿ ತಯಾರಿಸಿತ್ತಾರೆ. 4 ಜಪಸರದಲ್ಲಿ ಇರುವ ನೂರಾ ಎಂಟು ಮಣಿಗಳು ಉಪನಿಷತ್ತನ್ನು ಪ್ರತಿನಿಧಿಸುತ್ತವೆ. ಅಲ್ಲದೆ ಅಷ್ಟೋತ್ತರಶತ ಎಂಬ ಸಂಖ್ಯೆಗೆ ಸಂಕೇತವನ್ನು ಬಹುವಾಗಿ ಸಂಖ್ಯಾಶಾಸ್ತ್ರದಲ್ಲಿ ವಿವರಿಸಿದ್ದಾರೆ. ಪ್ರಸ್ತುತ ಭಗವಂತನಿಗೆ ಪ್ರಿಯವಾದ ಸಂಖ್ಯೆಯೆಂದೂ, ಇದನ್ನು ತಂತ್ರಗ್ರಂಥಗಳು ವಿಧಿಸಿರುವವೆಂದೂ, ಪೂಜೆಯಲ್ಲಿ ಅಷ್ಟೋತ್ತರಶತ ಸಂಖ್ಯೆಯಿಂದ ಭಗವಂತನನ್ನು ಅರ್ಚಿಸಲಾಗುತ್ತದೆಂದೂ, ಹೇಗೆ ನೋಡಿದರೆ ಹಾಗೆ ವಿದಿತವಾಗುತ್ತದೆ. ಅದೇನೇ ಇದ್ದರೂ ಜಪವನ್ನು ಎಣಿಸಲೇ ಬೇಕೆನ್ನುತ್ತದೆ ತಂತ್ರಶಾಸ್ತ್ರ.
\begin{verse}
ಮುಕ್ತಾಫಲೈರ್ವಿದ್ರುಮೇಣ ರುದ್ರಾಕ್ಷೈಃ ಸ್ಫಟಿಕೇನ ವಾ |\\
ಗಣನಾ ಸರ್ವಥಾ ಕಾರ್ಯಾ ಸಮ್ಯಗಂಗುಲಿಪರ್ವಭಿಃ ||
\end{verse}
ಗಣನೆಯಿಲ್ಲದ ಜಪ ಆಸುರವಾದೀತು! ಎಂದು ಹೇಳಿ, ಜಪಗಣನೆಗೆ ತುಂಬ ಒತ್ತು ನೀಡಿರುವರು. ಜಪವನ್ನು ಎಣಿಸದಿದ್ದರೆ ಆ ಜಪ ’ನೀರಿನಲ್ಲಿ ಮಾಡಿದ ಹೋಮದಂತೆ’ ಎಂದು ಜಪಗಣನೆಯನ್ನು ಕಡ್ಡಾಯಗೊಳಿಸಿದ್ದಾರೆ.
\begin{verse}
ವಿನಾ ದರ್ಭೈಸ್ತು ಯತ್ಸ್ನಾನಂ ಯಚ್ಚ ದಾನಂ ವಿನೋದಕಮ್ |\\
ಅಸಂಖ್ಯಾತಂ ತು ಯಜ್ಜಪ್ತಂ ಸರ್ವಂ ತದಫಲಂ ಸ್ಮೃತಮ್ ||
\end{verse}
ಜಪ ಮಣಿ ಎಣಿಸುವಾಗ ಹೆಬ್ಬೆರಳು ಮತ್ತು ಕನಿಷ್ಠಿಕಾದಿ ಮೂರು ಬೆರಳುಗಳನ್ನು ಮಾತ್ರ ಉಪಯೋಗಿಸ ಬೇಕು. ತೋರು ಬೆರಳು ಅಹಂಕಾರ ಸೂಚಕವೆಂದು ಪರಿಗಣಿಸಿ, ಅದನ್ನು ಕೈಬಿಡಲಾಗಿದೆ. ಹಾಗಂತ ಬೆರಳುಗಳಿಂದಲೇ ಜಪವನ್ನು ಎಣಿಸುವಾಗ ಅನಾಮಿಕದ ಮಧ್ಯ ಪರ್ವದಿಂದಾರಂಭಿಸಿ, ತರ್ಜನಿಯ ಮೂಲಪರ್ವದವರೆಗೆ ಪ್ರದಕ್ಷಿಣಕ್ರಮದಲ್ಲಿ ಹತ್ತು, ಹತ್ತಾಗಿ ಲೆಕ್ಕಮಾಡಿ ಕೊಳ್ಳ ಬೇಕೆನ್ನುತ್ತದೆ ಶಾಸ್ತ್ರ.
\begin{verse}
ಆರಭ್ಯಾನಾಮಿಕಾಮಧ್ಯಾತ್ಪ್ರದಕ್ಷಿಣಮನುಕ್ರಮಮ್ |\\
ತರ್ಜನೀ ಮೂಲಪರ್ಯಂತಂ ಜಪೇದ್ದಶಸು ಪರ್ವಸು || 
\end{verse}
ಹೀಗೆ ಲೆಕ್ಕವಿಡುವಾಗ ನೆನಪಿಗಾಗಿ ಅಗರು, ಕೃಷಿವಸ್ತು, ಸಿಂಧೂರ, ಗೋಮಯ, ಜೇಡಿಮಣ್ಣು ಇವುಗಳಿಂದ ಗುಳಿಗೆಗಳನ್ನು ಮಾಡಿ ಪಯೋಗಿಸಬಹುದ 5ಎಂದಿದ್ದಾರೆ. ಆದರೆ ಈ ಕೆಳಗಿನ ವಸ್ತುಗಳನ್ನು ಲೆಕ್ಕದ ಅನುಕೂಲಕ್ಕೆ ತೆಗೆದು ಕೊಳ್ಳಬೇಡಿ! ಎಂದೂ ಹೇಳಿರುವರು.
\begin{verse}
ನಾಕ್ಷತೈರ್ಹಸ್ತಪರ್ವೈರ್ವಾ ನ ಧಾನ್ಯೈರ್ನ ಚ ಪುಷ್ಪಕೈಃ |\\
ನ ಚಂದನೈರ್ಮೃತ್ತಿಕಯಾ ಜಪಸಂಖ್ಯಾಂ ಚ ಕಾರಯೇತ್ ||
\end{verse}
ಹೀಗೆಯೇ ಮುಂದುವರಿದು ಜಪಕಾಲದಲ್ಲಿ ಏನೇನನ್ನು ಮಾಡ ಬಾರದು? ಎಂದರೆ ಅದನ್ನೂ ತಂತ್ರಗ್ರಂಥಗಳಲ್ಲಿ ಶಿವ-ಪಾರ್ವತೀ ಸಂವಾದ ಮೂಲಕವಾಗಿ ನಮೂದಿಸಿದ್ದಾರೆ.
\begin{verse}
ವಿಣ್ಮೂತ್ರೋತ್ಸರ್ಗಶಂಕಾದಿಯುಕ್ತಃ ಕರ್ಮ ಕರೋತಿ ಚ |\\
ಜಪಾರ್ಚನಾದಿಕಂ ಸರ್ವಮಪವಿತ್ರಂ ಭವೇತ್ಪ್ರಿಯೇ ||\\
ಮಲಿನಾಂಬರಕೇಶಾದಿಮುಖದೌರ್ಗಂಧ್ಯಸಂಯುತಃ |\\
ಯೋ ಜಪೇತ್ತಂ ದಹತ್ಯಾಶು ದೇವತಾ ಶುಪ್ತಿಸಂಸ್ಥಿತಾ ||\\
ಆಲಸ್ಯಂ ಜೃಂಭಣಂ ನಿದ್ರಾಂ ಕ್ಷುತ್ತ್ರಂ ನಿಷ್ಠೀವನಂ ಮಯಮ್ |\\
ನೀಚಾಂಗಸ್ಪರ್ಶನಂ ಕೋಪಂ ಜಪಕಾಲೇ ವಿವರ್ಜಯೇತ್ ||
\end{verse}
ಜಪವನ್ನು ಆಯಾ ಜಪದ ವಸ್ತು ಮತ್ತು ಉದ್ದೇಶ ಹೀಗೆ ಎರಡು ದೃಷ್ಟಿಕೋಣದಿಂದ ವಿಭಾಗಿಸಬಹುದಾಗಿದೆ. ಮೊದಲನೆಯದರಲ್ಲಿ ವೇದಮಂತ್ರ ಜಪಗಳು ಮತ್ತು ತಂತ್ರೋಕ್ತ ಮಂತ್ರ ಜಪಗಳು ಎಂದು ಪುನಃ ವಿಭಾಗಿಸುತ್ತಾರೆ. ವೇದಮಂತ್ರ ಜಪಗಳಲ್ಲಿ ಪ್ರತ್ಯೇಕ ಮಂತ್ರ ಜಪ ಮತ್ತು ಸೂಕ್ತ ಅಥವಾ ಅನುವಾಕ ಜಪ ಎಂದು ಮತ್ತೆರಡು ಭಾಗಗಳನ್ನು ಕಾಣುತ್ತೇವೆ. ಇನ್ನು ತಂತ್ರೋಕ್ತ ಮಂತ್ರ ಜಪಗಳಲ್ಲಿ ಬೀಜ ಮಂತ್ರ ಜಪ, ಮೂಲ ಮಂತ್ರ ಜಪ ಮತ್ತು ಮಾಲಾ ಮಂತ್ರ ಜಪ ಎಂದು ಮೂರು ವಿಭಾಗಗಳು ಕಾಣಸಿಗುತ್ತವೆ. ಇನ್ನು ಜಪದ ಉದ್ದೇಶವನ್ನು ಲಕ್ಷಿಸಿ ಮುಖ್ಯವಾಗಿ ಮೂರು ವಿಭಾಗಗಳನ್ನು ನಿತ್ಯ ಜಪ, ನೈಮಿತ್ತಿಕ ಜಪ, ಮತ್ತು ಕಾಮ್ಯ ಜಪ ಎಂದು ವಿವೇಚಿಸುತ್ತಾರೆ. ಇಷ್ಟಾದರೂ ಈ ಮೇಲಿನ ವಿಭಾಗಗಳ ವ್ಯಾಪ್ತಿಗೆ ನಿಲುಕದಂತಿರುವ ಜಪದ ಒಂದು ಪ್ರಕಾರದ ಬಗ್ಗೆ ನೋಡಲೇ ಬೇಕು. ಅದೇ ಅಜಪಾ ಜಪ.

’ಅಜಪಾ’ದ ಅರ್ಥವೇನೆಂದರೆ ನಾವು ಜಪವನ್ನು ಮಾಡದಿದ್ದರೂ, ಜಪವು ನಮ್ಮೊಂದಿಗೇ ಇರುವುದು! ಇಂಥ ಸ್ವಾಭಾವಿಕ ಜಪದ “\textbf{ಅಜಪಾ.}”ಹೆಸರೇ 6 ಪ್ರತೀ ಜೀವವೂ ಸಕಾರ, ಹಕಾರಗಳ ಜಪವನ್ನು ನಡೆಸುತ್ತಿದೆ. ಇದು ನಮ್ಮೆಲ್ಲರ ಪಾಲಿಗೆ ಅನೈಚ್ಛಿಕವಾಗಿ ನೆರವೇರುತ್ತಿದೆ. ಒಂದು ವೇಳೆ ಈ ಕ್ರಿಯೆ ನಮ್ಮ ಅಧಿಕಾರ ವ್ಯಾಪ್ತಿಯಲ್ಲಿ ಬಂದರೆ (ನಾವು ಐಚ್ಛಿಕವಾಗಿ ಈ ಜಪವನ್ನು ಮಾಡುವಂತಾದರೆ) ಅದು ಮಾನವ ಪ್ರಪಂಚದ ಅಪರೂಪವೂ, ಅದ್ಭುತವೂ ಆದ ಸಾಧನೆಯಾಗುವಲ್ಲಿ ಸಂಶಯವಿಲ್ಲ! ಈ ಮೈಲಿಗಲ್ಲನ್ನು ದಾಟಿದ ಋಷಿಗಳು ಇದನ್ನು “\textbf{ಅಜಪಾ ಜಪ}” ಎಂದು ಗುರುತಿಸಿದ್ದಾರೆ.
\begin{verse}
ಸಕಾರಂ ಚ ಹಕಾರಂ ಚ ಜೀವೋ ಜಪತಿ ಸರ್ವದಾ |  \\
ಹಂಸ ಹಂಸ ವದೇದ್ವಾಕ್ಯಂ ಪ್ರಾಣಿನಾಂ ದೇಹಮಾಶ್ರಿತಃ |\\
ಸ ಪ್ರಾಣಾಪಾನಯೋರ್ಗ್ರಂಥಿರಜಪೇತ್ಯಭಿಧೀಯತೇ || ಬ್ರಹ್ಮವಿದ್ಯೋಪನಿಷತ್.
\end{verse}
ಪ್ರತಿ ನಿತ್ಯವೂ ಜಪ ಮಾಡುವುದರಿಂದ ಅನೇಕ ಅನುಕೂಲಗಳಿವೆ. ವಿಷಯ ವಸ್ತುಗಳತ್ತ ಹರಿದಾಡುವ ಮಾನಸಿಕ ಚಿಂತೆಗಳನ್ನು ನಿಯಂತ್ರಿಸುತ್ತದೆ. ಮನಸ್ಸನ್ನು ಭಗವಂತನ ಕಡೆಗೆ ಹರಿಯುವಂತೆ ಮಾಡುವುದೇ ಅಲ್ಲದೆ ದುಷ್ಟ ಕಾರ್ಯಗಳನ್ನು ಮಾಡದಂತೆ ತಡೆ ಹಿಡಿಯುತ್ತದೆ. ಮನಸ್ಸಿಗೆ ಶಾಂತಿ ತರುತ್ತದೆ. ಜಪಿಸುವ ಪ್ರತಿ ಮಂತ್ರದಲ್ಲಿಯೂ ಚೈತನ್ಯ ಶಕ್ತಿ ಅಡಗಿದ್ದು, ಸಾಧಕನ ಶಕ್ತಿ ಕುಂದಿದಾಗ ಮಂತ್ರಶಕ್ತಿ ಸಾಧನಾ ಶಕ್ತಿಯಾಗಿ ನಿಂತು ಸಾಧಕನಿಗೆ ಹುರಿದುಂಬಿಸುತ್ತದೆ. ರಜೋ ಗುಣವನ್ನು ಸತ್ವ ಗುಣವನ್ನಾಗಿ ಪರಿವರ್ತಿಸುತ್ತದೆ. ವ್ಯಾಯಾಮದಿಂದ ಶರೀರವು ದೃಢವಾಗಿ, ಆರೋಗ್ಯವನ್ನು ಹೊಂದುವಂತೆ, ಜಪದಿಂದ ಮನಸ್ಸಿಗೆ ವ್ಯಾಯಾಮ ದೊರೆತು ಸ್ಥಿರವಾದ ಮನಸ್ಸನ್ನು ಪಡೆಯಲು ಸಹಾಯಕವಾಗುತ್ತದೆ. ಸ್ಥಿರ  ಚಿತ್ತದಿಂದ ಮನಸ್ಸಿಗೆ ಶಾಂತಿ ದೊರೆಯುತ್ತದೆ.

’\textbf{ಅಶಾಂತಿ}’ ಎನ್ನುವುದು “\textbf{ಈ ವಿಶ್ವದ ಅಪಾರ ಪೂರ್ಣತೆಯನ್ನು ಗ್ರಹಿಸಲಾರದವನ ಒಂದು ಭ್ರಮೆ.}” ಎಂದರೆ ತಪ್ಪಾಗಲಾರದು. ಈ ಭ್ರಮಾ ಕವಾಟವನ್ನು ತೆರೆದು ಹೊರ ಬರಬೇಕಾದ ಅನಿವಾರ್ಯತೆ ಇದೆ. ಅದಕ್ಕೆ ಜಪಯಜ್ಞ ಅತ್ಯಂತ ಸಹಕಾರಿಯಾಗಿದೆ. ಸ್ವಾಭಿಮಾನ, ದುರಭಿಮಾನಗಳಿಂದ ನಾವೇ ತೋಡಿಕೊಂಡ ಗುಂಡಿಗೆ ನಮಗೆ ಗೊತ್ತಾಗದ ಹಾಗೆ ಬಿದ್ದು ಬಿಟ್ಟಿರುತ್ತೇವೆ. ಮೇಲೆ ಹತ್ತಲು ಮಾಡುವ ಪ್ರಯತ್ನಗಳಿಗೆ ರಾಗ, ದ್ವೇಷಗಳು ಅಡ್ಡಿಯಾಗಿ ಪುನಃ ಪುನಃ ಕೆಳಗೆ ಜಾರಿಸುತ್ತಿರುತ್ತವೆ. ಯಾವ ದುರಭಿಮಾನದಿಂದ ಬಿದ್ದಿರುತ್ತೇವೆಯೋ, ಅದೇ ದುರಭಿಮಾನ ಇತರರ ಸಹಾಯ ಪಡೆಯದಂತೆ ಪ್ರಚೋದಿಸುತ್ತದೆ. ಕೆಲವೊಮ್ಮೆ ಬಿದ್ದಿರುವ ಗುಂಡಿಯೇ ಎಷ್ಟು ಸುಂದರವಾಗಿದೆ! ಎಂಬ ಹಿತವಾದ ಸುಳ್ಳು ಹೇಳಿಕೊಂಡು ಸಮಾಧಾನಪಟ್ಟುಕೊಳ್ಳುತ್ತೇವೆ. “ನಾವು ಬಿದ್ದಿರುವ ಗುಂಡಿ ಅಷ್ಟೇನೂ ದೊಡ್ಡದಲ್ಲ, ಅವರು ಬಿದ್ದಿರುವ ಗುಂಡಿ ನಮ್ಮ ಗುಂಡಿಗಿಂತ ಆಳವಾಗಿದೆ” ಎಂದು ವಿಕೃತ ಸಮಾಧಾನವನ್ನೂ ಹೊಂದುತ್ತೇವೆ. ಗುಂಡಿಯಲ್ಲಿ ಬೀಳಲು ಅವರು ಕಾರಣ, ಇವರು ಕಾರಣ ಎಂದು ಆರೋಪಿಸುತ್ತೇವೆ. ಅವರಿವರಿಗಿಂತ ನಾನೇ ಮುಖ್ಯ ಕಾರಣ! ಎಂಬ ಕಟುಸತ್ಯವನ್ನು ಮಾತ್ರ ನಾವು ಒಪ್ಪಿಕೊಳ್ಳುವುದಿಲ್ಲ.

\begin{verse}
ವಾಸನೆಯು ಬಿರುಗಾಳಿ, ವಿಚಾರ ತರಗೆಲೆಯು, ಆಸೆ ನಗುವುದು ವಿಚಾರ ಸೋಲುವುದು |\\
ಜಾರುವುದನರಿತರೂ ನಾಶವಾಗದು ಚಪಲ, ಸಕಲ ಸಂಕಟಕೆ ಮೂಲವಿದು ಮೂಢ | 7
\end{verse}
ಎಲ್ಲ ಸಂಕಷ್ಟಗಳಿಗೂ ಮೂಲಸ್ಥಾನವಾದ ಮನಸ್ಸನ್ನು ಹದಗೊಳಿಸುವ ಈ ’ಜಪಯಜ್ಞ’ ನಮಗಾಗಿ ಪೂರ್ವಿಕರು ತೋರಿದ ಮೃತ್ಯುಂಜಯ ಮಾರ್ಗ. ಇದನ್ನನುಸರಿಸಿ ನೀರೋಗವನ್ನೂ, ನಿರುಪದ್ರವವನ್ನೂ ಸಂಪಾದಿಸಿ, ಶಾಶ್ವತದತ್ತ ಮುಖಮಾಡೋಣ.
\begin{verse}	
ವೇದಾಃ ಶ್ರೇಷ್ಠಾಃ ಸರ್ವಯಜ್ಞಕ್ರಿಯಾಶ್ಚ ಯಜ್ಞಾಜ್ಜಪ್ಯಂ ಜ್ಞಾನಮಾರ್ಗಶ್ಚ ಜಪ್ಯಾತ್ |\\
ಜ್ಞಾನಾದ್ಧ್ಯಾನಂ ಸಂಗರಾಗವ್ಯಪೇತಂ ತಸ್ಮಿನ್ಪ್ರಾಪ್ತೇ ಶಾಶ್ವತಸ್ಯೋಪಲಬ್ಧಿಃ || 8
\end{verse}	

