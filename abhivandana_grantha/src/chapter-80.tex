\chapter{अलंकारशस्त्रे ध्वनि:}

\begin{center}
\Authorline{डा । एन्.आर्.मुरलीधरः बि.काम्}
\smallskip
सहायकाध्यापकः\\
महाराजसंस्कृतमहापाठशाला\\
मैसूरु

\end{center}
विशालेस्मिन् प्रपञ्चे नानाविधा: स्थावरजङ्गमात्मका: प्राणिनो जीवन्ति म्रियन्ते चेति नाविदितं समेषां विदुषामाविदुषाम् च। तत्र प्राणिषु के श्रेष्ठा इति माहामेधावी मनुनोक्तम्-
\begin{verse}
भूतानां प्रणिन: श्रेष्ठा: प्राणीनां बुद्धिजीविन:।\\
बुद्धिमत्सु नरा: श्रेष्ठा: नरेषु ब्राह्मणा: स्मृता:॥
\end{verse}
ब्रह्मज्ञानं चाध्यात्माध्ययनं विना न सम्भवति । तदर्थं वेदा: शास्त्राणि पुराणानि इतिहासादयो ग्रन्थराशय: आविरभवन्। एतच्च विद्यास्थानम् इति विश्रुतम् । तानि च इत्थमुक्तमभियुक्तै: 
\begin{verse}
अङ्गानि वेदाश्चत्वारो मीमांसान्यायविस्तर:।\\
पुराणं धर्मशास्त्रं च चिद्या ह्येताश्चतुर्दश॥\\
आयुर्वेदो धनुर्वेदो गन्धर्वश्चेति ते त्रय:।\\
अर्थशास्त्रं चतुर्थं च विद्या ह्यष्टादश स्मृता:॥
\end{verse}
एतेषां शास्त्राणामयमुद्घोष:-
\begin{verse}
शमार्थं सर्वशास्त्राणि विहितानि मनीषिभि:।\\
स एव सर्वशास्त्रज्ञ: यस्य शान्तं मन: सदा॥
\end{verse}
तदर्थं शास्त्राणि मानवं प्रवर्तयन्ति निवर्तयन्ति वा। शास्त्रलक्षणमित्थमुक्तमभियुक्तै:-
\begin{verse}
प्रवृर्तिर्वा निवृत्तिर्वा नित्येन कृतकेन वा।\\
पुंसां येनोपदिश्येत तच्छास्त्रमभिधीयते॥ इति
\end{verse}
लोककल्याणकांक्षिभि: क्रान्तदर्शिभि: महात्मभि: मुनिभि: मुनिसदृशैश्च नाना ग्रन्थराशयो विरचिता: विरच्यन्तेद्यापि। ते च स्वभाव-कृत्यनुगुणं त्रिधा विभक्ता:। १. प्रभुसम्मिताः   २. सुहृत्सम्मिताः  ३.कान्तासंहिताश्चेति।

तत्र प्रभुसम्मितं काव्यजातं वेदा:। ते च अपौरुषेया: इत्यस्माकीना श्रद्धा। तेषामर्थावगतौ शास्त्राण्युपकुर्वन्ति। तादृशानां वेदानां करणं सर्वज्ञतुल्यानामपि ऋषीणां न सम्भवम्। मुनय: तपसा ध्यानेन योगाभ्यासेन च समन्त्रकान् वेदार्थानवजग्मु:। अत एव ते वेदद्रष्टार: न तु कर्तार:। अत वेदरचना कर्तुमशक्येति तत्र नियमा: नावश्यका:।

सुहृत्सम्मितं काव्यजातं रामायण्महाभारतादिसर्वज्ञसदृशैर्वाल्मीकिव्यासमुख्यै: मुनिभि: विरचितम्। तेषामपि तादृशासाहित्यकरणे नियमा: अनावश्यका:।

किन्तु तृतीयप्रकारस्य कान्तासम्मिताख्यकाव्यजातस्य निर्माणे तावत् नियमा: आवश्यका:। तत्रापि सहजप्रज्ञावतां कवीनामपि ते नियमा: अनावश्यका: एव । ते च तादृशनियमै: बद्धा न भवन्ति ।  अत एव उक्तं - “निरङ्कुशा: कवय:” इति।
\begin{verse}
अपारे काव्यसंसारे कविरेक: प्रजापति:।\\
यथास्मै रोचते विश्वं तथेदं परिवर्तते॥ इति च।
\end{verse}
जगत्सृष्टिकर्तु: काव्यजगत्स्रष्टा व्यतिरिक्त इति अनेन सुष्ठु प्रतिपादितम्। तादृशा: कवय: काव्यसंसारे विरलातिविरला:।

किन्तु येषां काव्यकरणे काव्यावलोकने च अभिलाष: तेषां कृते नियमा: खल्वावश्यका:। अत एवाग्निपुराणे उक्तम्-
\begin{verse}
नरत्वं दुर्लभे लोके विद्या तत्र सुदुर्लभा।\\
कवित्वं दुर्लभं लोके शक्तिस्तत्र सुदुर्लभा॥ इति
\end{verse}
अत: कवित्वाऽऽपादकं शास्त्रमेव अलङ्कारशाशास्त्रपदेन व्यपदिश्यते । अस्यैव साहित्यशास्त्रमिति व्यपदेशोऽप्यस्ति। अत्र साहित्यापरनामधेये अलङ्कारशास्त्रे निरूपिता: विषया: के के इति भोजराजेन स्वीये शृङ्गारप्रकाशाख्ये ग्रन्थे इत्थमुक्तम्-

“किं साहित्यम्? य: शब्दार्थयोः सम्बन्ध: स च द्वादशधा -

अभिधा विवक्षा तात्पर्यं प्रविभाग: व्यपेक्षासामर्थ्यम् अन्वय: एकार्थो भाव: दोषहानं गुणोपादनम् अलङ्कारयोग: रसाऽवियोगश्च” इति। अत एव साहित्यपुरुषस्य (काव्यपुरुषस्य) परिचये प्रवृत्तानां शब्दार्थालङ्कार-रीति-रसादीनां प्राधान्याऽप्रधान्ये मनसि निधाय विविधा: सम्प्रदाया: प्रवर्तिता: प्रवृद्धाश्च । तेषां परस्परं विरोध: आपातत: प्रतीयमानोऽपि साहित्यरसिकानां तु पृथग्दृष्टावयवानां  परिचय: अन्धराजन्यायेनेव उपकरोति।

अत: लोके सर्वथा प्रमाणं, सर्वथाऽप्रमाणं किञ्चिद्वस्तु विषय: शास्त्रं वा नास्ति वेदान् विहायेत्यस्माकं दृढविश्वासः । अत एव “वेदात् शास्त्रं परं नास्ति” इति प्रथितम् । तदर्थावगतौ अलंकारशास्त्रमपि शास्त्रेषु अन्यतमम् । अत एव अलंकारिकशेखरेण राजशेखरेण उद्घुष्टम् _
\begin{verse}
अलंकारः सप्तमम….मिति यायावरीयः इति।\\
गौर्गौ कामदुघा सम्यक् प्रयुक्ता स्मर्यते बुधैः।\\
दुष्प्रयुक्ता पुनर्गोत्वं प्रयोक्तुः सैव शंसति ॥\\
तदल्पमपि नोपेक्ष्यं काव्यं दुष्टं कथञ्चन।\\
स्याद्वपु: सुन्दरमपि श्वित्रेणैकेन दुर्भगम्॥\\
सर्वथा पदमप्येकं न निगद्यमवद्यवत्।\\
विलक्ष्मणा हि काव्येन दु:सुतेनेव निन्द्यते॥
\end{verse}
दुष्टं काव्यं मह्यां भारायते। साधु काव्यं महाभारतायते । तदर्थमुपकारकाणि सर्वव्यापि शास्त्राणि।

पदवाक्यप्रमाणादिशास्त्राणां सार्थक्यं साहित्यरचनयैव भवतीति रुद्रट: स्वीये काव्यालंकारग्रन्थे सश्रद्धं सगौरवं स्पष्टं कथयति-
\begin{verse}
फलमिदमेव हि विदुषां शुचिपदवाक्यप्रमाणशास्त्रेभ्य:।\\
यत्संकारो वाचां वाचश्च सुचारुकाव्यफला:॥
\end{verse}
निरूपणेनानेन अलंकारशास्त्रस्य प्रामुख्यमवगतं भवति।

\section*{ध्वनि:}

सत्स्वपि अनेकेषु सम्प्रदायेषु ध्वनिसम्प्रदाय एव प्राधान्यमावहति। अत: ध्वनिशब्दार्थ: क: ? ध्वनिशब्द: इत: पूर्वं कुत्र प्रयुक्त:? किं लक्षणं ध्वने: तद्विभाग: क:? तस्य वाच्यलक्ष्यापेक्षया भिन्नत्वं कथम् ? इति संक्षेपेण निरूप्यतेऽत्र लेखने।

ध्वन् इति शब्दार्थकात् धातो: ‘खनि-कष्यज्यसिवसिवनिसनिध्वनिग्रन्थिचूलिभ्यश्च’ इति उणादि सूत्रेण कर्त्रर्थे इप्रत्यये कृते ध्वनि: इति शब्दो निष्पन्नो भवति। ध्वनति अर्थमिति कर्तृव्युत्पत्या ध्वनि शब्द: शब्दात्मके काव्ये प्रयुक्तो वर्तते।

वैयाकरणै: पदार्थोपस्थित्यनुकूले शब्दे वाक्यार्थानुभावानुकूले वाक्ये च स्फोटो वर्तते इति स्फोटाभिव्यंञ्जक: शब्द: ध्वनिरिति उक्त:। भाषापरिच्छेदे शब्दस्य द्वैविध्यं प्रदर्शितम्।

यथा-
\begin{verse}
शब्दो ध्वनिश्च वर्णश्च मृदङ्गादिभवध्वनि:।\\
कण्ठसंयोगजन्मनो वर्णास्ते कादयो मता:॥
\end{verse}
अस्यैव ध्वनिशब्दस्य व्युत्पत्त्यन्तराणि अर्थानुगुणं दृश्यन्ते। तद्यथा-
\begin{enumerate}
\item ध्वनिव्यञ्जनया रसादीन् प्रत्याययति - शब्दार्थसमुदाय:
\item ध्वन्यतेऽनया वाच्यार्थलक्ष्यार्थभिन्नार्थ: इति - व्यञ्जनावृत्तिर्नाम व्यापर: 
\item ध्वन्यतेऽसौ - रसादि:
\item ध्वन्यते व्यज्यतेऽस्मात् इति काव्यम् व्यञ्जकत्वसाम्यादालंकारिकैरपि ध्वनिरिति व्यपदिश्यते ।
\end{enumerate}
महाभाष्ये-“अथ गौरीत्यत्र क: शब्द:? इत्युपक्रम्योक्तं _

येनोच्चारितेनसास्नालाङ्गूलककुदखुरविषाणिनां संप्रत्ययो भवति स शब्द:

अथवा

प्रतीतिपदार्थको लोके ध्वनि: शब्द: इत्युच्यते । तद्यथा _ शब्दं कुरु मा शब्दं कार्षी: ।

शब्दकार्ययं माणवक: इति ध्वनिं कुर्वन्नेवम् उच्यते तस्मात् ध्वनि: शब्द: । अत्र प्रतीतपदार्थक: इत्यस्य प्रसिद्धपदार्थहेतु: इत्यर्थ:। एवं च ध्वनिशब्द: ध्वनिरिति व्यवहार: स्वकपोलकल्पित: नेति ध्वनितं भवति । अत एव अलंकारिकमूर्धन्येन रसिकानामानन्दवर्धनेन प्रथमो हि विद्वांसो वैयाकरणा इति सश्रद्धं सगौरवमुद्घोषितम्।

तत्र ध्वन्याल्लोकनाम्नि ग्रन्थे आनन्दवर्धन: ध्वनिमित्थं लक्षितवान् । तत्कारिका चेत्थम् अस्ति।
\begin{verse}
यत्रार्थ: शब्दो वा तमर्थमुपार्जनीकृतस्वार्थौ ।\\
व्यङ्क्त: काव्यविशेष: स ध्वनिरिति सूरिभि: कथित:॥
\end{verse}
तत्रैव वृत्तौ विवृतम् - यत्र अर्थ: = वाच्यविशेष: शब्द: वाचकविशेषो वा तमर्थ: व्यङ्कः स: काव्यविशेष: ध्वनिरिति । अत्र ध्वन्यतेऽस्मादिति व्युत्पत्तिं आश्रित्य काव्यमित्युक्तं लक्षितं च। तत्रैव लोचने इत्थमुक्तम् - स इति । अर्थो वा शब्दो वा व्यापारो वा अर्थोऽपि वाच्य: ध्वनतीति शब्दोप्येवम् व्यङ्ग्यो वा ध्वन्येते इति व्यापारो वा शब्दार्थयो: ध्वननमिति ।

कारिकया तु प्राधान्येन समुदाय एव काव्यरूपो मुख्यतया ध्वनिरिति प्रतिपादितम्। पूर्वश्लोके “तमर्थम्” इत्येतदंशं विवृतवान् आनन्दवर्धन आभां श्लोकाभ्याम् _
\begin{verse}
प्रतीयमानं पुनरन्यदेव वस्त्वस्ति वाणीषु महाकवीनाम्।\\
यत् तत्प्रसिद्धावयवातिरिक्तं विभाति लावण्यमिवाङ्गनासु॥
\end{verse}
\begin{verse}
सरस्वती स्वादुतदर्थवस्तु निष्यन्दमाना महतां कवीनाम्।\\
अलोकसामान्यमभिव्यनक्ति परिस्फुरन्तं प्रतिभाविशेषम्॥
\end{verse}
ध्वनिरयं प्रथमं आलंकारिकै: त्रिधा विभक्त:
\begin{enumerate}
\item वस्तुध्वनि:
\item अलंकारध्वनि:
\item रसध्वनि: इति।
\end{enumerate}

तत्र रस्यते - अस्वाद्यते इति योगशक्त्या रसपदेन भावाभासभावोदय भावसन्धिभावशान्त्यादयोऽपि गृह्यन्ते। अयमपि ध्वनि: अन्ययापि विभक्त: ।
\begin{enumerate}
\item विवक्षितान्यपरवाच्य:
\item अविवक्षितवाच्यश्चेति।
\end{enumerate}
विवक्षितान्यपरवाच्य: अभिधामूल:।

अविवक्षितवाच्य: लक्षाणामूल:।

अभिधामूलके ध्वनौ वाच्यं विवक्षितमपि व्यङ्ग्ये पर्यवसितं भवति । अतश्चैनं विवक्षितान्यपरवाच्यमाहु:।

लक्षणामूलके ध्वनौ तु वाच्यमविवक्षितं भवति । अतस्तं अविवक्षितवाच्यमित्याचक्षते ।

लक्षणामूलध्वनि: - अर्थान्तरसंक्रमितवाच्य: अत्यन्ततिरस्कृतवाच्यत्वेन पुनर्द्विविध: ।

अयं लक्षणामूलध्वनिर्लक्षणामाश्रित्य प्रवर्तते । अभिधामूलध्वनिरपि असंलक्ष्य,  संलक्ष्यक्रमभेदेन प्रथमं द्विविध:।  संलक्ष्यक्रमभेदो ध्वनि: शब्दार्थशक्त्युभयशक्त्युद्भवेन त्रिविध:। वस्त्वंलकारगतत्वेन शब्दशक्त्युद्भवो ध्वनि: द्विविध:। अर्थशक्त्युद्भवो ध्वनिश्च द्वादशविध: । उक्तश्च द्वादशविधत्वं दर्पणे विश्वनाथेन _
\begin{verse}
वस्तु वालंकृतिर्वापि द्विधार्थसम्भवी स्वत:।\\
कवे: प्रौढोक्तिसिद्धो वा तन्निबद्धस्य चेति षट्॥\\
षड्भिस्तैर्व्यज्यमानस्तु वस्त्वलंकाररूपक:।\\
अर्थशक्त्युद्भवो व्यङ्ग्यो याति द्वादशभेदताम्॥
\end{verse}
उभयशक्त्युद्भवो ध्वनिरेक एव। संकलक्ष्य सर्वान् इमान् भेदान् ध्वनिरष्टादशधा। ध्वनेर्मुख्य- भेदास्तु एत एव अवान्तरभेदेन सहस्रशो भवन्ति। तेषां गणना ध्वनेर्व्यापकतां बोधयति। अत: तेषां संख्याभि: गणना असंख्यावतामायासायैव कल्पते इति विरम्यते ।

ध्वने: वाच्याद्विभेदं साधयितुमानन्दवर्धनेन बहून्युदाहरणानि दत्तानि । तत्र वाच्ये विधिरूपे प्रतिषेधरूप: यथा _
\begin{verse}
भम धम्मिअ विसत्थो सो सुणओ अज्ज मारिओ देण।\\
गोलणैकच्छकुडङ्गवासिणा दरिअसीहोण॥
\end{verse}
\begin{verse}
(भ्रम धार्मिक विस्रब्ध: स शुनकोऽद्य मारितस्तेन।\\
गोदावरी नदीकूललतगहनवासिना हप्तसिंहेन॥)
\end{verse}
हे धार्मिक !  विस्रब्ध: त्वं भ्रम। अद्य स : शुनक: गोदावरी नदीकूललतगहनवासिना दृप्तसिंहेन मारित:। इत्यन्वय:

तत्रैव लोचने प्रकरणमित्थं विवृतम्-

कस्याञ्चित् संकेतस्थानं जीवितसर्वस्वायमानं धार्मिकसञ्चरणान्तरायदोषात् तदवलुप्यमानपल्लवकुसुमादि विच्छायीकरणाच्च परित्रातुम् इयमुक्ति: इति।

अत्र भ्रम इति लोट् प्रयोगात् भ्रमणं विधिरूपं विहितम्। किन्तु सिंहसद्भावात् तदनुचितमिति मा भ्रम इति निषेधरूपार्थ: वाच्यात् भिन्नोऽर्थ: प्रतीयते स: वाच्यार्थ: न, लक्ष्यार्थोऽपि न मुख्यार्थबाधाविरहात् किन्तु तद्भिन्न: व्यङ्ग्य: ध्वनिरूप: इति । क्वचित् वाच्ये प्रतिबोधरूपे विधिरूप: ।
\begin{verse}
अत्ता एत्थ णिमज्जइ एत्थ अहं दिवसां पलोएहि।\\
मा पहिअ रत्तिअन्धअ सेज्जाए मह णिमज्जइसि\\
(श्वश्रूरत्र निमज्जति अत्राहं दिवसकं विलोकय ।\\
मा पथिक रात्र्यन्धक शय्यायामावयो: शयिष्ठा:)
\end{verse}
प्रकरणं - वसतीं प्रार्थयमानं सञ्जातकामं पथिकं प्रति प्रोषितभर्तृकाया: व्यभिचारिण्या: स्वयं दूत्या: उक्तिरियम्।

अत्र वक्त्री पुंश्चली आवयो: शय्यायां मा शयिष्ठा: इति निषेधं करोतीति वाच्यार्थ: । किन्तु तत्र प्रयुक्त पदस्वारस्येन - अत्र गृहे श्वश्रूरहं च, श्वश्रूश्च जरत्तरत्वेन बधिरा निस्पन्दा च। जनसञ्चारस्तु नास्त्येव अतो यथेष्टं मम शय्यायामेव स्वपिहि इति व्यङ्ग्य: व्यभिचारिणो: वक्तृबोद्धव्ययो: वैशिष्ट्यात् सहृदयानां प्रतीयते। अत्र वाच्य: प्रतिषेधरूप: व्यङ्ग्य: विधिरूपं इति स्पष्टं ज्ञायते ।

****
क्वचिद्वाच्ये विधिरूपेऽनुभूयरूपो यथा-
\begin{verse}
वच्च महविवा एक्केइ होन्तुणीसासरोइ अव्वाई।\\
मा तुज्ज वि तिअ विणा दख्षिण्णहअहस्स जायन्तु॥
\end{verse}
\begin{verse}
(व्रज ममैकस्या भ्वन्तु नि:श्वासरोदितव्यानि ।\\
मा तवापि तया विना दाक्षिण्यहतस्य जनिषत ॥)
\end{verse}
तत्रैव लोचने-“अत्र व्रजेति विधि:। न प्रमादादेव नायिकान्तरसंगमनं तव अपि तु गाढानुरागात्, येनान्यादृङ्मुखराग: गोत्रस्खलनादि च, केवलं पूर्वकृतानुपालात्मना दाक्षिण्येनैकरूपत्वाभिमानेनैव त्वमत्र स्थित:; तत्सर्वथा षठोऽसीति गाढमन्युरूपोऽयं ; खण्डितनायिकाभिप्रायोऽत्र प्रतीयते। न चासौ व्रज्याभावरूपो निषेध: नापि विध्यन्तरमेवान्यनिषेधाभाव: इति ।

एवमालंकारिककुलतिलकेन आनन्दवर्धनेन ध्वनिस्थापनार्थं कृत्स्ने ग्रन्थे अवकाशो दत्त:। ध्वनिस्तावत् वस्तु-अलंकार-ध्वनित्वेन पूर्वमुक्त:। तस्य समन्वय: मङ्गलपद्ये एव प्रदर्शित: लोचनकारेणाभिनवगुप्ताचार्येण स्वगुरुनामग्राहं सम्यक् विवृत: । तद्यथा-
\begin{verse}
स्वेच्छाकेसरिण: स्वच्छस्विच्छायायासितेन्दव:।\\
त्रायन्तां वो मधुरिपो: प्रपन्नार्तिच्छिदो नखा:॥
\end{verse}
तत्र लोचनग्रन्थ: इत्थमस्ति-

मधुरिपो: नखा: वो युष्मान् व्याख्यातॄन् श्रोतॄंश्च त्रायन्ताम् । तेषामेव सम्बोधनयोग्यत्वात् सम्बोधनसारो हि युष्ममदर्थ:, त्राणं च अभिष्टलाभं प्रति साहायकाचरणं भवतीति, इयदत्र त्राणं विवक्षितम्, नित्योद्योगिनश्च भगवतोऽसम्मोहाताध्यवसाययोगित्वेनोत्साहप्रतीते: वीररसो ध्वन्यते ।

नखानां प्रहणत्वेन प्रहरणेन च रक्षणे कर्तव्ये नखानामव्यतिरिक्तत्वेन करणत्वात् सातिशयशक्तिता कर्तृत्वेन सूचिता, ध्वनितश्च परमेश्वरस्य व्यतिरिक्तकारणापेक्षाविरह:, मधुरिपोरित्यनेन तस्य सदैव जगत्रासापसारणोद्यम उक्त:, कीदृशस्य मधुरिपो:? स्वेच्छया केसरिण: न तु कर्मपारतन्त्र्येण, नाप्यन्दीयेच्छया, अपि तु विशिष्टदानवहननोचिततथाविधेच्छापरिग्रहोचित्यादेव स्वीकृतसिंहरूपस्येत्यर्थ;, कीदृशा: नखा:? प्रपन्नामार्तिं ये छिन्दन्ति, नखानां हि छेदकत्वमुचितम्, आर्ते: पुनच्छेद्यत्वं नखानात्यसम्भवनीयमपि तदीयानां नखानां स्वेच्छानिर्माणौचित्यात् सम्भाव्यत एवेति भाव: ।

अथवा त्रिजगत्कण्टको हिरण्यकशिपु: विश्वस्योत्क्लेशकर इति स एव वस्तुत: प्रपन्नानां भगवदेकशरणानां जनानाम् आर्तिकरणत्वात् मूर्तेवास्तित्वं विनाशयद्भि: आर्तिरेवोच्छिन्ना भवतिति परमेश्वरस्य तस्यामप्यवस्थायां परमकारुणिकत्वमुक्त्म् । किं च ते नखा: स्वच्छेन स्वच्छतागुणेन, स्वच्छमृदुप्रभृतयो हो मुख्यतया भाववृत्तय एव, स्वच्छायया च वक्रहृद्यरूपया आकृत्याऽऽयासित: खेदित: इन्दुर्यै:, अत्रार्थशक्तिमूलेन ध्वनिना बालचन्द्रत्वं ध्वन्यते, आयासनेन तत्सन्निधौ चन्द्रस्य विच्छायत्वप्रतीतिरहृद्यत्वप्रतीतिश्च ध्वन्यते, आयासकारित्वं च नखानां सुप्रसिद्धम् । नरहरिनखानां तच्च लोकोत्तररूपेण प्रतिपादितम्। किं च तदीयां स्वच्छतां कुटिलिमानं चावलोक्य बालचन्द्र: स्वात्मनि खेदमनुभवति, तुल्योऽपि स्वच्छकुटिलाकारयोगेऽपि प्रपन्नार्तिनिवारणकुशला: न तु अहमिति व्यतिरेकालंकारोऽपि ध्वनित: । किंचाहं पूर्वमेकमेवासाधरणवैषिष्ट्य-हृद्याकारयोगात् समस्त-जनाभिलषणीयता-भाजनम् अभवम्, अद्य पुनरेवंविधा नखा दश बालचन्द्राकारा: सन्तापार्तिच्छेद-कुशलश्चेति तानेव लोको बालेन्दुबहुमानेन पश्यति । न तु मामित्याकलयन् बालेन्दुरविरतमायासम् अनुभवतीवेत्युत्प्रेक्षापहति । ध्वनिरपि एवं वस्त्वलंकाररसभेदेन त्रिधा ध्वनिरत्र श्लोकेऽस्मद्गुरुभि: व्याख्यात: इति।

अन्येषां ध्वनिप्रभेदानां सोदाहरण्समन्वय: तत्तसंबद्धग्रन्थेष्वेव अवलोकनीय: । अत्र संक्षेपेण ध्वनि: निरूपित: । 
\begin{verse}
तरू: सृजति पुष्पाणि मरुद्वहति सौरभम् ।\\
कवि: सृजति काव्यानि रसं जानन्ति पण्डिता: ॥
\end{verse}
इति श्लोकानुसारेण वयं काव्यसौरभम् अनुभवेम।
