{\fontsize{14}{16}\selectfont
%~ \hyphenchar\font=-1
\righthyphenmin=800
\lefthyphenmin=800
\chapter{ಸಂಪಾದಕರ ಹೃದಯ}

ಶಿಷ್ಯವತ್ಸಲರಾದ ಶ್ರೀಮಾನ್ ಗಂಗಾಧರ ಭಟ್ಟರಿಗೆ ಅವರ ವಿದ್ಯಾರ್ಥಿಗಳಾದ ನಾವು ಯಥಾಶಕ್ತಿ ಅಭಿವಂದನೆಯನ್ನು ಸಾಂಗೋಪಾಂಗವಾಗಿ ಸಮರ್ಪಿಸಿದ್ದೇವೆಂದು\break  ಭಾವಿಸಿದ್ದೇವೆ. ಕಾರ್ಯಕ್ರಮದ ಅಂಗವಾಗಿ ಅಂದೇ ಅಭಿವಂದನ ಗ್ರಂಥಮುಕುಲ\-ವೊಂದನ್ನೂ ಸಮರ್ಪಿಸಿದ್ದೆವು. ಅಂದರೆ, ಅಂದು ಅದು ಪೂರ್ಣವಾಗಿ ಅರಳಿರಲಿಲ್ಲ. ಈಗ ಅದು ಸುಗಂಧಬೀರುವ ಮಕರಂದಸ್ರವಿಸುವ ಪರಾಗಗಳಿಂದ ಪೂರ್ಣವಾದ\break  ಗ್ರಂಥಸುಮವಾಗಿ ಅರಳಿ ನಿಂತಿದೆ. ಈ ಸುಮ, ಭ್ರಮರಗಳ ಮನಸ್ಸಿಗೆ ಸುಮನೋಹರಾ\-ವಾಗಬಹುದೆಂಬ ನಂಬಿಕೆ ನಮ್ಮದು.

ಅಭಿನಂದನ ಶಬ್ದ ಮತ್ತು ಅದರ ಪ್ರಯೋಗ ನಮಗೆಲ್ಲ ಚಿರಪರಿಚಿತವಾಗಿದೆ. ಆದರೆ ವಿದ್ಯಾರ್ಥಿಗಳಾದ ನಾವು ನಮ್ಮ ಅಧ್ಯಾಪಕರಿಗೆ ವಂದನೆಯನ್ನು ಭಾವಿಸಿ ಅದರ\break ಸಮರ್ಪಣೆಗೆ ‘ಅಭಿವಂದನ’ ಕಾರ್ಯಕ್ರಮ ಎಂದು ಹೆಸರಿಸುವುದು ಉಚಿತವೆಂದು ತಿಳಿದು ಆ ಕಾರ್ಯವನ್ನು ಸಂಪನ್ನಗೊಳಿಸಿದ್ದೇವೆ. ಗ್ರಂಥವೂ ಅಭಿವಂದನ ಗ್ರಂಥವೇ ಆಗಿದೆ. ಇದಕ್ಕೂ ನಮಗೆ ಗಂಗಾಧರ ಭಟ್ಟರೇ ಸ್ಫೂರ್ತಿ. ಅವರು ಈ ಹಿಂದೆಯೇ ತಮ್ಮ ಗುರುಗಳಾದ ಮಹಾಮಹೋಪಾಧ್ಯಾಯ ಎನ್.ಎಸ್. \hbox{ರಾಮಭದ್ರಾಚಾರ್ಯರನ್ನು}  ಗೌರವಿಸಿ, ಕೃತಜ್ಞತೆಯನ್ನು ಸಮರ್ಪಿಸಲು ಹಮ್ಮಿಕೊಂಡ ಕಾರ್ಯಕ್ರಮಕ್ಕೆ,\break  ಸಾಮಾನ್ಯವಾಗಿ ಬಳಕೆಯಲ್ಲಿರುವ ಅಭಿನಂದನ ಶಬ್ದಕ್ಕಿಂತ ಅಭಿವಂದನ \hbox{ಎಂಬುದರಲ್ಲಿ} ಹೆಚ್ಚು ಔಚಿತ್ಯವನ್ನು ಭಾವಿಸಿ ಮೊದಲ ಬಾರಿಗೆ ಅಭಿವಂದನ ಪದವನ್ನು ಬಳಸಿ\break ಕಾರ್ಯಕ್ರಮ ಮಾಡಿದರು. ವಾಸ್ತವವಾಗಿ ಅಭಿವಂದನೆಯಿಂದ ಉಭಯತ್ರ\break  ಉಂಟಾಗುವುದು ಅಭಿನಂದನೆಯೆ. ಅಭಿನಂದನೆಯೆಂದರೆ ಅದು ಆನಂದಕ್ಕಿಂತ\break  ಬೇರೆಯೇನಲ್ಲ. ಆದರೆ ರೂಢಿಯಲ್ಲಿ ಆ ಪದ, ವಿದ್ಯಾರ್ಥಿಗಳು ಅಧ್ಯಾಪಕರಿಗೆ\break ಸಮರ್ಪಿಸುವ ವಂದನೆಗಿಂತ ಸ್ವಲ್ಪ ಭಿನ್ನವಾದ ಅರ್ಥದಲ್ಲಿಯೇ ಬಳಸಲ್ಪಡುತ್ತಿದೆ.\break ಸಮಾಜದಲ್ಲಿ ಆ ಹೆಸರಿನಲ್ಲಿ ನಡೆಯುವ ಕಾರ್ಯಕ್ರಮಗಳನ್ನು ನಾವು ದಿನಂಪ್ರತಿ\break ನೋಡುತ್ತಿದ್ದೇವೆ. ಎಷ್ಟೇ ಉತ್ಕೃಷ್ಟವಾದ ಅರ್ಥವುಳ್ಳ ಪದಗಳಾದರೂ ಅದು \hbox{ಅತಿಯಾಗಿ} ಪ್ರಯುಕ್ತವಾಗಲು ಪ್ರಾರಂಭವಾದರೆ ಆ ’ಪದ’ ತನ್ನ ಸ್ಥಾನದಿಂದ ಚ್ಯುತವಾಗಿ, ಪದ\enginline{-}ಪದಾರ್ಥಗಳ ಭಾವ ಲಘುವಾಗಿಬಿಡುವ ಸಂಭವವೇ ಹೆಚ್ಚು. ಸದ್ಯ ಅಭಿವಂದನ ಪದಕ್ಕೆ ಇನ್ನೂ ಆ ಪರಿಸ್ಥಿತಿ ಬಂದಿಲ್ಲ. ಹಾಗಾಗಿ ಇದೇ ಪದ ನಮ್ಮ ನಡೆನುಡಿಗೆ\break ಅನುಗುಣವಾಗಿದೆಯೆನ್ನಿಸಿತು. ಅದನ್ನೇ ಬಳಸಿ ನಡೆಸಿದ ಕಾರ್ಯಕ್ರಮ\break  ಸಾರ್ಥಕವಾಯಿತು. ಗ್ರಂಥವೂ ಅರ್ಥಪೂರ್ಣವಾಯಿತು.

ನಾವು ಕಾರ್ಯಕ್ರಮ ಮಾಡುವ ಸಂದರ್ಭ ವಿಲಂಬ ಸಂವತ್ಸರದೆಡೆಗೆ \hbox{ಸಾಗುತ್ತಿದ್ದ} ಕಾಲ.  ಆ ಕಾಲಧರ್ಮವೋ ಎಂಬಂತೆ ಕಾರ್ಯಕ್ರಮದ ಅನಂತರ ನಡೆಯಬೇಕಿದ್ದ\break ಗ್ರಂಥಪ್ರಸವ ಗಜಗರ್ಭದಂತೆಯೇ ಸಾಕಷ್ಟು ವಿಲಂಬವಾಗಿದೆ. ಅದಿರಲಿ, ಆದರೆ ಒಂದು ಗ್ರಂಥದ ಪ್ರಕಾಶನ ಒಬ್ಬ ಗರ್ಭಿಣಿಯ ಜೀವನಕ್ಕಿಂತ ಭಿನ್ನವೇನಲ್ಲ, ಅಷ್ಟೇ \hbox{ಶ್ರಮದಾಯಕ}. ಶ್ರಮ ಆ ಆಶ್ರಮಕ್ಕೆ ಅನಿವಾರ್ಯ. ಗರ್ಭವತಿಗೆ ತನ್ನೊಳಗಿರುವ ಗರ್ಭದ ಬಗ್ಗೆ ಯಾವೆಲ್ಲ ಕುತೂಹಲ ಇರಬಹುದೋ ಅದು ಸಂಪಾದಕನಿಗೂ \hbox{ಇರುತ್ತದೆ}. “ಕ್ಲೇಶಃ ಫಲೆನ ಹಿ ಪುನರ್ನವತಾಂ ವಿಧತ್ತೇ” ಎಂಬಂತೆ ಪ್ರಸವವಾದೊಡನೆ\break ಕ್ಲೇಶವೆಲ್ಲ ಮರೆತು ಯಾವ ಆನಂದ ತಾಯಿಗಾಗುವುದೋ, ಗ್ರಂಥ ಪ್ರಕಟವಾದಾಗ\break ಸಂಪಾದಕನಿಗಾಗುವ ಸಂತೋಷ ಅದಕ್ಕಿಂತ ಕಡಿಮೆಯದೇನೂ ಅಲ್ಲ. ಮಗುವನ್ನು\break ನೋಡುವತನಕ ತಾಯಿಗೆ ನೆಮ್ಮದಿಯಿಲ್ಲ, ಮಗುವನ್ನು ಪತಿಯ ಕೈಯ್ಯಲ್ಲಿ \hbox{ಇಡುವತನಕ} ಸಮಾಧನವಿಲ್ಲ.  ಅಂತೆಯೇ ಸಂಪಾದಕನಿಗೂ ಮುದ್ರಣಗೊಂಡ ಪುಸ್ತಕ ನೋಡುವ ತನಕ ನೆಮ್ಮದಿಯಿಲ್ಲ. ಲೇಖಕರ ಕೈಯ್ಯಲ್ಲಿ ಇಡುವ ತನಕ ಸಮಾಧಾನವಿಲ್ಲ. ಆದ್ದರಿಂದ ಒಂದು ಕೃತಿ ಅದು ಸಂಪಾದಕನ ಶಿಶುವೇ ಎಂಬುದರಲ್ಲಿ ಅನುಮಾನವಿಲ್ಲ. \hbox{ಲೇಖನಗಳ} ಆಕಲನ ರೂಪವಾದ ಇಂತಹ ಗ್ರಂಥದ ಮಟ್ಟಿಗೆ ಹೇಳುವುದಾದರೆ ಸಂಪಾದಕನಷ್ಟೇ ಸಂತೋಷ ಲೇಖಕರಿಗೂ ಉಂಟು. ಇದರಲ್ಲಿ ಅವರ ಪಾತ್ರವೇ ಪ್ರಧಾನವಾಗಿದೆ. ಈ\break ಸಂತೋಷಾನುಭವಕ್ಕೆ ನಮ್ಮೆಲ್ಲರ ನಿರೀಕ್ಷೆ ಮೀರಿದ ಸಮಯ ಸಂದಿದೆ. ನೈಯ್ಯಾಯಿಕರು ಹೇಳುವಂತೆ ಇಲ್ಲಿಯೂ ವಿಘ್ನಪ್ರಾಚುರ್ಯವನ್ನು ಭಾವಿಸಬೇಕಷ್ಟೆ !!

ಪ್ರಕಾಶನದ ವಿಷಯದಲ್ಲಿ ‘ಆಕ್ಸ್ ಫರ್ಡ್’ ನಂತಹ ಅತ್ಯಂತ ಶಿಸ್ತಿನ ಸಂಸ್ಥೆ ಒಂದು\break ಪುಸ್ತಕವನ್ನು ಹದಿಮೂರು ವರ್ಷಗಳಲ್ಲಿ ಪ್ರಕಟಿಸಬೇಕೆಂದು ಯೋಜನೆಯನ್ನು\break ಹಾಕಿಕೊಂಡಿತು. ಆದರೆ ಆ ಯೋಜನೆಯನ್ನು ಮುಗಿಸಲು ಅದು ಎಪ್ಪತ್ತು \hbox{ವರ್ಷಗಳನ್ನು} ತೆಗೆದುಕೊಂಡಿತು ಎಂದು ವಿದ್ವಾಂಸರಾದ ಎನ್.\ ಬಾಲಸುಬ್ರಹ್ಮಣ್ಯನವರು\break  ಉಲ್ಲೇಖಿಸಿದ್ದಾರೆ. ಹಾಗಾಗಿ ಪುಸ್ತಕ ಪ್ರಕಾಶನದಲ್ಲಿ ಅದರಲ್ಲೂ ಅನೇಕ ಲೇಖಕರನ್ನು\break ಅವಲಂಬಿಸಿ ಸಿದ್ಧವಾಗಬೇಕಾದ ಪುಸ್ತಕವನ್ನು ಅಲ್ಪಾವಧಿಯಲ್ಲಿ ಮುದ್ರಿಸುವ ಪ್ರತಿಜ್ಞೆ\break ಪ್ರತಿಜ್ಞೆಯೇ ಆಗಿ ಉಳಿಯುವುದೇ ಹೆಚ್ಚು ಎಂದರೆ ಅತಿಶಯವಲ್ಲ. 

ಆರಂಭದಲ್ಲಿ ಲೇಖನ ಸಂಗ್ರಹಕ್ಕಾಗಿ ನಾವು ತೊಡಗಿದಾಗ ಗ್ರಂಥ ವಿಷಯ\break ಗ್ರಂಥಿಯಾಗಿಯೇ ಉಳಿದುಬಿಡುವ ಆತಂಕವಿತ್ತು. ಯದ್ಯಪಿ ಗ್ರಂಥವನ್ನು ಅಭಿವಂದನ ಕಾರ್ಯಕ್ರಮದಲ್ಲೇ ಸಮರ್ಪಿಸಬೇಕೆಂದಿದ್ದರೂ ಅದನ್ನು ಮುಂದೂಡುವುದಕ್ಕೆ ಇದೂ ಒಂದು ಕಾರಣ. ಆದರೆ ನಿಧಾನವಾಗಿ ಲೇಖನಗಳು ಬರಲು ಪ್ರಾರಂಭವಾದವು. ಬಹು ಲೇಖನಗಳು ನಮ್ಮ ಕೈಸೇರಲು ವಿಲಂಬವಾದರೂ ಕೊನೆಯಲ್ಲಿ ಬಂದವುಗಳ\break ಸಂಖ್ಯೆಯನ್ನು ಗಮನಿಸಿದಾಗ ಅಭಿಮಾನಿ ಲೇಖಕರ ಬಗೆಗೆ ನಮ್ಮ ಅಭಿಮಾನ\break  ಇಮ್ಮಡಿಸಿತು. ಆದರೆ ಲೇಖನಗಳನ್ನೆಲ್ಲ ಗ್ರಂಥದ ಸ್ವರೂಪಕ್ಕೆ ತಂದು ಅಳವಡಿಸಿದಾಗ, ಆದ ಗ್ರಂಥದ ಗಾತ್ರ ನೋಡಿ ಮತ್ತೆ ಗಾಬರಿಯಾಯಿತು. ಕಾರಣ ಅದು ಐನೂರು ಪುಟಗಳನ್ನು ದಾಟಿತ್ತು. ಆದರೆ ಶ್ರೀಮಾನ್ ನರಸಿಂಹ ಹೆಗಡೆ, ಹೊನ್ನೇಹದ್ದ,  ಅವರ ಸುಪುತ್ರರಾದ ಶ್ರೀ ಗಣಪತಿ ಹೆಗಡೆ ಮತ್ತು ಶ್ರೀ ಸೂರ್ಯನಾರಾಯಣ ಹೆಗಡೆಯವರ ಔದಾರ್ಯ ನಮ್ಮ ಕಾರ್ಯವನ್ನು ಅನಾಯಾಸವಾಗಿ ಸಂಪನ್ನಗೊಳಿಸಿತು. ಇಂತಹ ಮಹನೀಯರಿಗೆ ನಮ್ಮ ಅನಂತಾನಂತ ಕೃತಜ್ಞತೆಗಳು ಸಲ್ಲುತ್ತವೆ. 

ಏನೆಲ್ಲ ಪರಿಕರಗಳಿದ್ದರೂ ಇಂತಹ ಗ್ರಂಥಕ್ಕೆ ಜೀವಾಳ ಲೇಖಕರೇ ವಿನಾ ಮತ್ತಾರೂ ಅಲ್ಲ. ಪ್ರಕೃತ ಈ ಎಲ್ಲ ಲೇಖಕರು ಗಂಗಾಧರ ಭಟ್ಟರ ಬಗೆಗೆ ತಮ್ಮ ಆತ್ಮೀಯತೆಯನ್ನು, ಗೌರವವನ್ನು ಲೇಖನಗಳ ಮೂಲಕ ವ್ಯಕ್ತಪಡಿಸಿದ್ದಾರೆ. ಇಲ್ಲಿರುವ ಲೇಖಕರಲ್ಲಿ\break  ಬಹುತೇಕರು ನಮ್ಮ ಅಧ್ಯಾಪಕ ವೃಂದದವರೇ ಆಗಿದ್ದಾರೆ. ಉಳಿದವರೂ ಸಹ ತತ್ಸಮಾನರೇ ಹೊರತು ಅನ್ಯರಲ್ಲ. ಗಂಗಾಧರ ಭಟ್ಟರ ವಿದ್ಯಾರ್ಥಿಗಳಲ್ಲೂ ಅನೇಕರು ಲೇಖನಸೇವೆ ಸಲ್ಲಿಸಿದ್ದಾರೆ. ವಾಸ್ತವವಾಗಿ ಮೊದಲು ಲೇಖಕರನ್ನು ವಿನಂತಿಸುವ ಪತ್ರದಲ್ಲಿ ನಾವು ಗೌರವ ಸಂಭಾವನೆಯನ್ನು ನೀಡುವ ಬಗ್ಗೆ ಪ್ರಸ್ತಾಪಿಸಿದ್ದೆವು, ಆದರೆ ಕೆಲವರು ಇದನ್ನು ವಿರೋಧಿಸಿದರು \enginline{-} “ಗಂಗಾಧರ ಭಟ್ಟರ ಬಗ್ಗೆ ಮತ್ತು ಅವರ ವಿದ್ಯಾರ್ಥಿಗಳಾದ ನಿಮ್ಮಗಳ ಬಗ್ಗೆ ನಮಗಿರುವ ಅಭಿಮಾನದಿಂದ ನಾವಿದನ್ನು ಕೊಡುತ್ತಿರುವುದೇ ವಿನಾ ಸಂಭಾವನೆಯನ್ನು ಭಾವಿಸಿ ಕೊಡುತ್ತಿರುವುದಲ್ಲ. ಅಭಿವಂ(ನಂ)ದನ ಗ್ರಂಥಕ್ಕೆ\break ಅಭಿಮಾನಿಗಳು ಬರೆಯುತ್ತಾರೆಯೇ ಹೊರತು ಸಂಭಾವನೆಯ ಅಪೇಕ್ಷೆಯುಳ್ಳವರಲ್ಲ, ಅದು ರೂಢಿಯಲ್ಲೂ ಇಲ್ಲ. ಹಾಗಾಗಿ ನಿಮ್ಮ ನಿರ್ಧಾರ ಉಚಿತವಾಗಿ ಕಾಣುತ್ತಿಲ್ಲ”, ಎಂಬ ಖಡಕ್ಕಾದ ಅಭಿಪ್ರಾಯವನ್ನು ವ್ಯಕ್ತಪಡಿಸಿದರು. ನಾವು ಸಂಭಾವನೆಯ ಯೋಚನೆಯನ್ನು ಅಲ್ಲಿಗೆ ಕೈಬಿಡುವ ತೀರ್ಮಾನ ತೆಗೆದುಕೊಂಡೆವು. ಈ ಮೂಲಕ ಲೇಖಕರು\break  ಉತ್ತಮರ್ಣರಾದರು. ಹಾಗಾಗಿ, ಗಂಗಾಧರ ಭಟ್ಟರ ಆತ್ಮೀಯರೂ ಅವರ\break  ವಿದ್ಯಾರ್ಥಿಗಳಾಗಿರುವ ನಮ್ಮಲ್ಲಿ ಪ್ರಸಾದಭಾವಪೂರಿತರೂ ಆದ ಸಮಸ್ತ ಲೇಖಕರಿಗೆ ವಯಕ್ತಿಕವಾಗಿ ಸಂಪಾದಕನ, ಮತ್ತು ವಿ| ಗಂಗಾಧರ ಭಟ್ಟರ ಅಭಿವಂದನ ಸಮಿತಿಯ ಗೌರವಪೂರ್ವಕವಾದ ಭೂರಿ ಭೂರಿ ಕೃತಜ್ಞತೆಗಳು ಸಲ್ಲುತ್ತವೆ.

ಈ ಗ್ರಂಥವನ್ನು ಮೂರು ವಿಭಾಗವಾಗಿ ವಿಂಗಡಿಸಿದೆ. ಮೊದಲನೆಯದು\break ಅಭಿವಂದನ ಕಾರ್ಯಕ್ರಮಕ್ಕೆ ಸಂಬಂಧಿಸಿದ ಸಚಿತ್ರ ವರದಿ. ಹಾಗಾಗಿ ಇದು ಚಿತ್ರ ಸಂಪುಟ. ಎರಡನೆಯದು ಶಾಸ್ತ್ರಸಂಪುಟ. ಇದು ವಿವಿಧ ಶಾಸ್ತ್ರಸಂಬಂಧಿ ಲೇಖನಗಳ ವಿಭಾಗ. ಇದು ಸಂಸ್ಕೃತಭಾಷೆಯಲ್ಲಿದೆ. ಮೂರನೆಯದು  ಒಳನುಡಿ ಸಂಪುಟ.\break ಗಂಗಾಧರ ಭಟ್ಟರ ಒಡನಾಡಿಗಳು ಅವರೊಡನೆ ಇರುವ ಒಡನಾಟದ ಅನುಭವಗಳನ್ನು ಇಲ್ಲಿ ಸ್ವರಸವಾಗಿ ಸ್ಮರಿಸಿಕೊಂಡಿದ್ದಾರೆ. ಈ ಲೇಖನಗಳು ಕನ್ನಡ ಸಂಸ್ಕೃತ, ಹಿಂದಿ ಇಂಗ್ಲಿಷ್ ಭಾಷೆಗಳಲ್ಲಿವೆ. ಲೇಖಕರಲ್ಲಿ ಕೆಲವರು ಶಾಸ್ತ್ರದ ಲೇಖನವನ್ನೂ, ಕೆಲವರು ಗಂಗಾಧರ ಭಟ್ಟರ ಬಗೆಗಿನ ಲೇಖನವನ್ನು ಮತ್ತೆ ಕೆಲವರು ಎರಡೂ ವಿಭಾಗಕ್ಕೆ ಸಲ್ಲುವ ಲೇಖನವನ್ನೂ ಕೊಟ್ಟು ಉಪಕರಿಸಿದ್ದಾರೆ. ಈ ಎಲ್ಲ ವಿಷಗಳನ್ನು ಒಳಗೊಂಡು ಗ್ರಂಥ ೫೦೦ಪುಟಗಳನ್ನು ದಾಟಿದೆ. ನಮಗೆ ಗ್ರಂಥ ಬಹಳ ಬೃಹತ್ತಾಗಿದೆ ಎಂಬಂಶವೇ ಬೀಗುವ ವಿಷಯವಾಗಿಲ್ಲ. ಇಲ್ಲಿರುವ ಶಾಸ್ತ್ರದ ವಿಷಯಗಳು ಬಹಳ ಗಂಭೀರವಾಗಿ ಆಯಾ ಶಾಸ್ತ್ರದ ಗಂಭೀರ ಚಿಂತನೆಗಳಿಂದ ಕೂಡಿವೆ. ಕೆಲವು ಲೇಖನಗಳು ಸಂಶೋಧನ ಪ್ರಬಂಧಗಳಾಗಿವೆ. ಒಂದೇ ಗ್ರಂಥ ಅನೇಕರಿಗೆ ಅನೇಕ ಶಾಸ್ತ್ರವಿಷಗಳನ್ನು ಪರಿಚಯಿಸುವ ವಿಷಯಗಳ ಆಕರವಾಗಿದೆ. ಇದು ನಮ್ಮ ನೆಮ್ಮದಿಗೆ ಕಾರಣವಾದ ಸಂಗತಿ. ಇಷ್ಟಾಗಿಯೂ ಎಲ್ಲ ವೇದಗಳ ಬಗ್ಗೆಯೂ ಒಂದೊಂದು ಲೇಖನಗಳನ್ನು ಪಡೆದುಕೊಳ್ಳುವ ನಮ್ಮ ಅಪೇಕ್ಷೆ ಫಲಿಸಿಲ್ಲ. ಡಾ | ಟಿ.ವಿ.ಸತ್ಯನಾರಾಯಣರವರು ಕೊಡಮಾಡಿದ ಅಥರ್ವವೇದದ ಲೇಖನವೇ ವೇದಸಂಬಂಧಿ ಲೇಖನಗಳ ಪ್ರತಿನಿಧಿಯಾಗಿದೆ. ಉಳಿದ ವೇದಗಳ ಲೇಖನಗಳೂ ಇಲ್ಲಿ ಒದಗಿಬಂದಿದ್ದರೆ ಗ್ರಂಥಗೌರವ ಹಿಗ್ಗುತ್ತಿತ್ತೇನೋ ಎಂದು ಆ ನ್ಯೂನತಾ ಭಾವ ನಮ್ಮನ್ನು ಕಿಂಚಿತ್ ಕಾಡಿದ್ದು ಸುಳ್ಳಲ್ಲ !!  

ಇನ್ನು ಮೂರನೇ ವಿಭಾಗದ ಲೇಖನಗಳು ಶ್ರೀಯುತರ ಬಗೆಗೆ ಅನೇಕ\break ಸ್ವಾರಸ್ಯಕರವಾದ ವಿಷಯಗಳನ್ನು ಅನಾವರಣ ಮಾಡುವಂಥವುಗಳಾಗಿವೆ. ಈ\break ವಿಭಾಗದಲ್ಲಿ ಗಂಗಾಧರ ಭಟ್ಟರ ಅಣ್ಣತಂಗಿಯರೂ ಲೇಖನಗಳನ್ನು ಬರೆದಿದ್ದಾರೆ. ಇವು ಅವರ ಬಾಲ್ಯಕಾಲದ ಕೆಲವು ಅಪರೂಪದ ಘಟನೆಗಳನ್ನು ತೆರೆದಿಡುತ್ತವೆ. ಇವುಗಳಿಂದ ವಿಶೇಷವಾಗಿ ಒಂದು ಅಂಶವನ್ನು ನಾವು ಗುರುತಿಸಬೇಕು \enginline{-} ಅದೇನೆಂದರೆ, ಗಂಗಾಧರ ಭಟ್ಟರು ಮನೆಗೆ ಮಾರಿಯಾಗಿ ಊರಿಗೆ ಉಪಕಾರಿಯಾದವರೂ ಅಲ್ಲ. ಕುಟುಂಬ ಸ್ವಾರ್ಥಿಯಾಗಿ ಊರಿಗೆ\enginline{-}ಸಮಾಜಕ್ಕೆ ದಕ್ಕದವರೂ ಅಲ್ಲ. ಅವೆರಡನ್ನೂ ಅವರು ಸಮಾನವಾಗಿ ಭಾವಿಸಿದರು, ಅಂತೆಯೇ ಅವೆರಡನ್ನೂ ಸಮಾನವಾಗಿ ತೂಗಿಸಿದರು. ಇದು ಗಂಗಾಧರ ಭಟ್ಟರ ವ್ಯಕ್ತಿತ್ವದ ವೈಶಿಷ್ಟ್ಯ.

ಇನ್ನು ಈ ವಿಭಾಗದಲ್ಲಿ ನಮ್ಮ ನಾಡಿನ ಪ್ರಸಿದ್ಧ ವಿದ್ವಾಂಸರೂ ಆಶುಕವಿಗಳೂ ಆಗಿರುವ ಡಾ| ಹೆಚ್.ವಿ.ನಾಗರಾಜರಾವ್ ರವರು ಗಂಗಾಧರ ಭಟ್ಟರ ವ್ಯಕ್ತಿತ್ವದ ಬಗೆಗೆ ಸುಂದರವಾದ ಶ್ಲೋಕಗಗಳನ್ನು ರಚಿಸಿ ಕೊಟ್ಟಿದ್ದಾರೆ. ವಿದ್ಯೆಯಿರುವವನಲ್ಲಿ ವಿನಯವಿಲ್ಲ, ವೈದುಷ್ಯವಿರುವವನು ದಯಾವಂತನಲ್ಲ ಎಂಬ ಈ ಲೋಕೋಕ್ತಿ ಸುಳ್ಳು ಎಂಬುದು ಗಂಗಾಧರ ಭಟ್ಟರನ್ನು ನೋಡಿದರೆ ಸ್ಪಷ್ಟವಾಗುತ್ತದೆ ಎಂಬಿತ್ಯಾದಿ ಅಂಶಗಳನ್ನು ಬಹಳ ಸ್ವಾರಸ್ಯವಾಗಿ ಬರೆದಿದ್ದಾರೆ. ಮೇಲಿಕೋಟೆಯ ವಿದ್ವಾಂಸರಾದ ಅರೈಯ್ಯರ್ ರಾಮಶರ್ಮರು ಕಾವ್ಯರಚನಾ ಚತುರರು. ಅವರು ವಯೋಧರ್ಮದ ಅಸಹಕಾರದ ನಡುವೆಯೂ ಬಹಳ ಪ್ರೀತಿಯಿಂದ ಅಭಿವಂದನ ಕಾರ್ಯಕ್ರಮದಲ್ಲಿ ಬೆಳಿಗ್ಗೆಯಿಂದ ಸಾಯಂಕಾಲದವರೆಗೂ ಉಪಸ್ಥಿತರಿದ್ದರು. ಅದರ ಪ್ರೇರಣೆಯಿಂದ ಸಭೆಯಲ್ಲಿಯೇ ಕೆಲವು ಶ್ಲೋಕಗಳನ್ನು ರಚಿಸಿದರು. ಅವುಗಳಲ್ಲಿ ಅವರು ಗಂಗಾಧರ ಭಟ್ಟರನ್ನು “ಹವ್ಯಕ\enginline{-}ವ್ಯೋಮಭಾಸ್ಕರ” ಎಂದಿದ್ದಾರೆ. ಭಟ್ಟರು ನಿತ್ಯ ವಹ್ನಿಧೂಮಗಳ ದೃಷ್ಟಾಂತದೊಡನೆ ಪಾಠಮಾಡುತ್ತಿದ್ದವರಾದ್ದರಿಂದ (ತರ್ಕ ಎಂದರೆ ಮೊದಲು ನೆನಪಾಗುವುದೇ\break ವಹ್ನಿಧೂಮಗಳಷ್ಟೇ, ಹಾಗಾಗಿ ತಮಾಶೆಯಾಗಿ ಹೇಳಬೇಕೆಂದರೆ ನೈಯಾಯಿಕರಿಗೂ ವಹ್ನಿಧೂಮಗಳಿಗೂ ನಿಯತ ಸಾಹಚರ್ಯ) ಅವುಗಳನ್ನೇ ಬಳಸಿದ್ದಾರಾದರೂ ಅದು ವಿಭಿನ್ನವಾಗಿದೆ \enginline{-} ಜ್ಞಾನವೆಂಬ ಆಜ್ಯಾಹುತಿಯಿಂದ ಹೊಗೆ(ದೋಷ)ಯಿಲ್ಲದ\break ಶುದ್ಧಾಗ್ನಿಯಂತೆ ಗಂಗಾಧರ ಭಟ್ಟರು ಬೆಳಗುತ್ತಾರೆ \enginline{-} ವಿಧೂಮಃ ಪಾವಕ ಇವ ಜ್ಞಾನಾಜ್ಯಾಹುತಿಭಿರ್ಜ್ವಲನ್ ಎನ್ನುವ ಈ ಪದ್ಯಪಂಕ್ತಿ ಒಂದು ಗಂಭೀರಭಾಕ್ಕೆ\break ಸೆಳೆದುಬಿಡುತ್ತೆ. ಅಷ್ಟು ಗಂಭೀರವಾಗಿದೆ ಅವರ ರಚಿಸಿದ ಪದ್ಯ ಸರಣೀ. ಅಂತೆಯೇ ಇನ್ನೊಂದು ಪದ್ಯಮಾಲಿಕೆಯನ್ನು ವಿ| ಮಂಜುನಾಥ.ಜಿ.ಭಟ್ಟರು ರಚಿದ್ದಾರೆ. ಇವರು ಗಂಗಾಧರ ಭಟ್ಟರನ್ನು ಬಾಲ್ಯದಿಂದಲೂ ಬಲ್ಲವರು. ಅವರು ಶ್ರೀಯುತರ ಜೀವನದ ಬಗ್ಗೆ “ಗಂಗಾಧರೋ  ಭಟ್ಟವರೋ ವಿರಾಜತಾಮ್  ಎಂಬ ಲಲಿತವಾದ ಪದ್ಯ\break ಮಾಲಿಕೆಯನ್ನು ರಚಿಸಿ ಸಮರ್ಪಿಸಿದ್ದಾರೆ. ಆ ಮಾಲಿಕೆ ಸಂಸ್ಕೃತದಲ್ಲಿದ್ದರೂ ಭಿನ್ನ\break ಭಾಷಭಿಜ್ಞರನ್ನೂ ಆಕರ್ಷಿಸುವುದರಲ್ಲಿ ಸಂದೇಹವಿಲ್ಲ. ಮಾಲಿಕಾರಚನೆಯೇ\break ದೊಡ್ಡವಿಷಯವಲ್ಲ. ಅದು ಮೂರ್ತಿಗೆ ತಕ್ಕ ಅಲಂಕಾರವಾಗಿ ಮೂರ್ತಿಯ ಸಹಜ ಸೊಬಗು ಇನ್ನೂ ಎದ್ದು ತೋರುವಂತಾದರೆ ಆಗ ಮಾಲಿಕಾಲಂಕಾರ ಸಾರ್ಥಕ, ಒಂದುವೇಳೆ ಮಾಲಿಕೆ ಮೂರ್ತಿಯನ್ನೇ ಮುಚ್ಚಿ ತಾನೇ ಎದ್ದು ತೋರುವಂತಾದರೆ, ಅಂತಹ ಅಲಂಕಾರವನ್ನು ಅಲಂ \enginline{-} ಮತ್ತೆ ಬೇಡ ಎನ್ನಬೇಕಾಗುತ್ತದೆ. ಆದರೆ ಈ ಮಾಲಿಕೆ ಮೂರ್ತಿಯ ಸೊಬಗನ್ನು ಹೆಚ್ಚಿಸಿದೆ ಎಂಬುದರಲ್ಲಿ ಸಂದೇಹವಿಲ್ಲ. ಈ ಎಲ್ಲ ಕಾವ್ಯಕೋವಿದರಿಗೆ ನಮ್ಮ ಅನಂತ ವಂದನೆಗಳು.

ಈ ಎರಡು ವಿಭಾಗದ ಮಧ್ಯದಲ್ಲಿ ಮಧ್ಯಮಣಿಯಾಗಿ ಲೇಖನವಿಶೇಷವೊಂದು ಪೊಣಿಸಲ್ಪಟ್ಟಿದೆ. ಅದು ಅಭಿವಂದ್ಯರಾದ ಗಂಗಾಧರ ಭಟ್ಟರದೇ. ಅವರನ್ನು ನಾವು, “ಗ್ರಂಥಕ್ಕೆ ತಾವೂ ತಮ್ಮ ಜೀವನದ ಬಗ್ಗೆ ಒಂದು ಲೇಖನವನ್ನು ನೀಡಬೇಕೆಂದು ವಿನಂತಿಸಿದೆವು. ಅವರೆಂದೂ ಯಾರ ಅಪೇಕ್ಷೆಯನ್ನೂ ಉಪೇಕ್ಷಿಸಿದವರಲ್ಲವಲ್ಲ.\break ಅನಾರೋಗ್ಯದ ಮಧ್ಯೆಯೂ ಲೇಖನವನ್ನು ದಯಪಾಲಿಸಿದ್ದಾರೆ. ಅವರ ಲೇಖನವನ್ನು ಅವರ ಕುಟುಂಬಿನಿಯಾದ \enginline{-} ಶ್ರೀಮತಿ ಶೈಲಜಾರವರೇ ಟಂಕಿಸಿಕೊಟ್ಟಿದ್ದಾರೆ. ಅದಕ್ಕಾಗಿ ನಾವು ಅವರಿಗೆ ಪ್ರತ್ಯೇಕವಾಗಿ ಕೃತಜ್ಞತೆಯನ್ನು ಸಮರ್ಪಿಸುತ್ತೇವೆ.

ಈ ಗ್ರಂಥದ ಮುಖಪುಟ ಚಿತ್ರದ ಬಗೆಗೆ ಮತ್ತು ಗ್ರಂಥಕ್ಕೆ ಬಳಸಿದ ಹೆಸರಿಗೆ ಹಿನ್ನೆಲೆಯಲ್ಲಿರುವ ಕೆಲವು ಚಿಂತನೆಗಳನ್ನು ಇಲ್ಲಿ ಉಲ್ಲೇಖಿಸಬೇಕಿದೆ

ನಮ್ಮ ಅಧ್ಯಾಪಕರಾದ ಗಂಗಾಧರ ಭಟ್ಟರು ಬಹುಮುಖ ಪ್ರತಿಭೆ. ವ್ಯವಹಾರ ಚತುರರು. ಔದಾರ್ಯ ಅವರ ಸ್ವಭಾವ. ಸಮಾಜಕ್ಕೆ ಅವರು ಬಹು ಆಕರ್ಷಕ ವ್ಯಕ್ತಿ. ಕ್ಷೇತ್ರ ಸಂಸ್ಕೃತದ್ದಾದರೂ ಅದಕ್ಕೇ ಸೀಮಿತರಾದವರಲ್ಲ. ವಿದ್ಯಾರ್ಥಿದೆಸೆಯಿಂದಲೇ ಆರಂಭವಾದ ಅವರ ಪಾಠ ನಿವೃತ್ತಿಯಾದರೂ ನಿಂತಿಲ್ಲ. ಅವರ ಪಾಠದ ಪಾಟವಕ್ಕೆ ಆಕರ್ಷಿತರಾಗದ ವಿದ್ಯಾರ್ಥಿಗಳಿಲ್ಲ. ತರ್ಕಶಾಸ್ತ್ರ ಪ್ರಧಾನ ವಿಷಯವಾದರೂ ಇತರ ಶಾಸ್ತ್ರಗಳಲ್ಲೂ ಅವರ ಕೃಷಿ ಕಡಿಮೆಯದೇನಲ್ಲ. ತರ್ಕ ಕರ್ಕಶವಾದರೂ, ಕಾವ್ಯ, ನಾಟಕಗಳ ಪಾಠದಲ್ಲೂ ಅವರು ರಸವನ್ನು ಜಿನುಗಿಸಿದವರು. ಹಾಗೆಂದು ಕೇವಲ ಪಾಠಶೂರರಲ್ಲ. ವಿದ್ಯೆಯನ್ನು  ಕಲೆಯಾಗಿಸಿ ಅಭಿನಯಿಸುವ ನಯ, ನೈಪುಣ್ಯವೂ ಅವರಿಗುಂಟು. ಕಾವ್ಯ, ಶಾಸ್ತ್ರ ಕೃಷಿಯಲ್ಲೆನೋ ಪರಿಣತರೆಂದು ಭೌತಿಕ ಕೃಷಿಯಲ್ಲೇನೂ ಹಿಂದೆಬಿದ್ದವರಲ್ಲ, ಇಷ್ಟೆಲ್ಲ ಇದ್ದರೂ ಅವರಾಗಿ ಅವರು ಯಾವುದರ ಹಿಂದೂ\break ಬಿದ್ದವರಲ್ಲ!!! ಎಲ್ಲವನ್ನೂ ದೆಶ, ಕಾಲ, ಅಗತ್ಯ ಮತ್ತು ಔಚಿತ್ಯಕ್ಕನುಗುಣವಾಗಿ\break ನಿರ್ವಾಹಮಾಡಿದವರು. ಇವೆಲ್ಲ ಇದ್ದೂ ಖ್ಯಾತಿ, ಲಾಭ, ಪೂಜೆಗಳಿಂದ ಮಾತ್ರ ಗಾವುದ ದೂರವುಳಿದವರು. ಅವರ ಜೀವನವನ್ನು ಆಧರಿಸಿ ಒಂದು ಸಿನೆ\enginline{(cine)}ಯನ್ನೇ ತಯಾರಿಸಬಹುದು. ಹೀಗಿರುವಾಗ ಇಂತಹ ಬಹುಮುಖ ವ್ಯಕ್ತಿತ್ವಕ್ಕೆ ಹೊಂದುವ ಒಂದು ಮುಖಚಿತ್ರ ನಿರ್ಮಾಣ ಸುಲಭದ ವಿಷಯವಲ್ಲ, ಅದೊಂದು ಸವಾಲು.

ಇಷ್ಟೆಲ್ಲ ಏನೇ ಇದ್ದರೂ ಗಂಗಾಧರ ಭಟ್ಟರೆಂದರೆ ಎಲ್ಲಕ್ಕಿಂತ ಮೊದಲು ನಮ್ಮ ಕಣ್ಣಿಗೆ ಕಟ್ಟುವುದು ಅವರ ಪಾಠವೇ. ಪಾಠ ಅವರ ಜೀವನದಲ್ಲಿ ಹಾಸು\enginline{-}ಹೊಕ್ಕಾಗಿ ಸೇರಿಕೊಂಡಿದೆ. ಹಾಗಾಗಿ ಅವರ ಜೀವನವೇ ಪಾಠ, ಪಾಠವೇ ಅವರ ಜೀವಾಳ. ಪಾಠವೆಂದಾಗ ಅದು ಜ್ಞಾನ\enginline{-}ವಿಜ್ಞಾನಮೂಲವಾದುದಷ್ಟೆ ! ಆ ಜ್ಞಾನ\enginline{-}ವಿಜ್ಞಾನಗಳಾದರೋ ಪರಂಪರಾಪ್ರಾಪ್ತವಾದುವು. ಪರಂಪರೆ ತಪೋಮೂಲವಾದುದು. ತಪಸ್ಸು\break ಭಗವನ್ಮೂಲವಾದುದು, ಇದು ನಮ್ಮ ಭಾರತೀಯ ವಿದ್ಯಾಸಂಪ್ರದಾಯದ ಸರಣಿ. ಆದ್ದರಿಂದ ಯಾರಲ್ಲೇ ಆಗಲಿ ಅವರಲ್ಲಿರುವ ವಿದ್ಯೆ, ಕಲೆ, ಪ್ರತಿಭೆಗಳಿಗೆಲ್ಲ ಈ ಸರಣಿಯೇ ಮೂಲ ಸ್ರೋತಸ್ಸು. ಈ ದೃಷ್ಟಿಯಿಂದ ಈ ಅಭಿವಂದನ ಗ್ರಂಥದ ಮುಖಚಿತ್ರ ಈ ಚಿಂತನೆಯನ್ನೇ ಅವಲಂಬಿಸುವುದೇ ಉಚಿತ, ಅದೇ ನಮ್ಮ ಸತ್ಸಂಪ್ರದಾಯ, ಅದರಲ್ಲೇ ಎಲ್ಲವೂ ಗತಾರ್ಥವಾಗುವುದೆಂದು ಭಾವಿಸಿ ಆ ಭಾವವನ್ನೇ ಚಿತ್ರಿಸುವ ಕಿರು ಪ್ರಯತ್ನ ಇಲ್ಲಿ ನಡೆದಿದೆ.

ಈ ಹಿನ್ನೆಲೆಯಲ್ಲಿ ಯೋಚಿಸಿದಾಗ ಒಂಟಾದ ಸ್ಫುರಣೆಯಿಂದ ಒಂದು ಚಿತ್ರ ಕಣ್ಣಮುಂದೆ ಬಂದಿತು. ಆ ಬೌದ್ಧಿಕ ಚಿತ್ರವನ್ನು ಭೌತಿಕರೂಪದಲ್ಲಿ ಚಿತ್ರಿಸಲು\break ಬಹುಮಟ್ಟಿಗೆ ನಾವು ಸಫಲರಾಗಿದ್ದೇವೆ. ಅದರ ಸ್ವರೂಪ, ಆಶಯ ಹೀಗಿದೆ  \enginline{-} 

ಚಿತ್ರದ ಉತ್ತುಂಗದಲ್ಲಿ ಬೆಳಕಿನರೂಪದಲ್ಲಿ ಮಾತ್ರ ಇರುವ ಎಲ್ಲಕ್ಕೂ ಮೂಲವಾದ ಪರಬ್ರಹ್ಮದ ಕಲ್ಪನೆಯಿದೆ. ಅದರ ಕೆಳಗೆ ಸೂಕ್ಷ್ಮದೃಷ್ಟಿಗೆ ಮಾತ್ರ ಗೋಚರವಾಗುವ ಯೋಗಶಾಯಿಯಾದ ವಿಷ್ಣುವಿದ್ದಾನೆ. ಆ ವಿಷ್ಣುವಿನ ಪಾದದ ಬಳಿ ಚತುರ್ಮುಖ ಬ್ರಹ್ಮ(ಬ್ರಹ್ಮಾ)ನಿದ್ದಾನೆ. ಅವನು ತನ್ನ ಕಮಂಡಲು ತೀರ್ಥದಿಂದ ವಿಷ್ಣುಪಾದವನ್ನು ಅಭಿಷೇಚಿಸುತ್ತಿದ್ದಾನೆ. ಆ ಅಭಿಷೇಕ ತೀರ್ಥ ಗೌರೀಶಂಕರ ಶಿಖರದಲ್ಲಿರುವ ಶಿವನ ಶಿರಸ್ಸನ್ನು ಅಲಂಕರಿಸಿದೆ. ಅಲ್ಲಿಂದ ಅದು ಸಪ್ತ ಸ್ರೋತಸ್ಸಾಗಿ ಇಳೆಗೆ ಇಳಿದು ನದಿಯಾಗಿ ಹರಿದು ಭೂಮಿಯನ್ನೆಲ್ಲ ಪಾವನಗೊಳಿಸುತ್ತಾ ಸರಿತ್ಪತಿಯಾದ ಸಾಗರವನ್ನು ಸೇರುತ್ತದೆ. ಆ ನದಿಯನ್ನು ನಾವು ಗಂಗಾನದೀ ಎನ್ನುತ್ತೇವೆ. ಇದನ್ನು ಜ್ಞಾನಿಗಳು ಜ್ಞಾನದ ಪ್ರವಾಹವೆಂದು ತಿಳಿದು ಜ್ಞಾನಗಂಗಾ ಎನ್ನುತ್ತಾರೆ. ಈ ಭೌತಿಕ ಪ್ರವಾಹ ಜ್ಞಾನಾನುಭವಕ್ಕೆ ತೊಡಕಾಗಿರುವ ಪಾಪಗಳನ್ನು ಪರಿಹರಿಸಿಕೊಂಡು ಜ್ಞಾನಾನುಭವವನ್ನು ಹೊಂದಲು ಉತ್ತಮ ಸಾಧನವಾಗಿದೆ \enginline{-}ಎಂಬುದು ಆಶಯ.  ಜ್ಞಾನಪ್ರವಾಹ ರೂಪದ ಗಂಗೆ ಮಾನವ ಶರೀರದಲ್ಲಿ ಶಕ್ತಿಯ ರೂಪದಲ್ಲಿ ಪ್ರವಹಿಸುವುದನ್ನು ಯೋಗಿಗಳು ಗುರುತಿಸುತ್ತಾರೆ. ಯೋಗಿಗಳು ಅದರಲ್ಲಿ ಅವಗಾಹನ ಮಾಡುತ್ತಾ ಜ್ಞಾನಾನುಭವದೆಡೆಗೆ ಸಾಗುತ್ತಾರೆ. ಆ ಅವಗಾಹನದ ಪ್ರತೀಕವಾಗಿ ಬೌತಿಕವಾದ ಗಂಗೆಯ ಅವಗಾಹನ ಸಂಪ್ರದಾಯದಲ್ಲಿ ಬಂದಿದೆ. ಇಂತಹ ಜ್ಞಾನನದಿಯ ಬದಿಯಲ್ಲಿ ಜ್ಞಾನಾನುಭವಕ್ಕಾಗಿ ಅದಕ್ಕೆ ಪೋಷಕವಾದ ಮಾನವ ಶರೀರಕ್ಕೂ ಸಂಕೇತವಾದ ಅಶ್ವತ್ಥ ವೃಕ್ಷದ ಮೂಲದಲ್ಲಿ ತಪಸ್ಸುಮಾಡುತ್ತಿರುವ ಒಬ್ಬ ಯೋಗಿಯಿದ್ದಾನೆ. ಇನ್ನೊಂದು ಬದಿಯಲ್ಲಿ ಅದೇ ಯೋಗಿ ತಾನು ತಪಸ್ಸಿನಿಂದ ಪಡೆದ ಜ್ಞಾನ, ವಿಜ್ಞಾನಗಳನ್ನು ಶಿಷ್ಯರಿಗೆ ಅದೇ ವೃಕ್ಷದ ಪದದಲ್ಲಿ ಕುಳಿತು\break ಬೋಧಿಸುತ್ತಿದ್ದಾನೆ. ಸುತ್ತಮುತ್ತಲ ಪ್ರದೇಶ ಪ್ರಶಸ್ತವಾದ ಧರ್ಮಭೂಮಿಯೆಂದು ನಿರ್ಭಯವಾಗಿ ನಿಂತಿರುವ ಕೃಷ್ಣಮೃಗಗಳು ಸಾರುತ್ತವೆ.  ಇವೆಲ್ಲ ಪರೋಕ್ಷವಾದ ವಿಷಯಗಳಾಗಿವೆ. ಅದರಿಂದ ಈ ವಿಷಯಗಳ ಸಾರವನ್ನು ಸಾರುವ ಚಿತ್ರಗಳನ್ನು ನೆರಳು ಚಿತ್ರವಾಗಿ ಚಿತ್ರಿಸಿದೆ. (ಆದರೆ, ಬ್ರಹ್ಮವಿಷ್ಣುಮಹೇಶ್ವರರ ಚಿತ್ರ ಮಾತ್ರ ಗುಪ್ತವಾಗಿದೆ. ಕಾರಣ ದೇವತೆಗಳು ಪರೋಕ್ಷಪ್ರಿಯರು. ಸ್ಥೂಲದೃಷ್ಟಿಗೆ ಗೋಚರಿಸುವವರಲ್ಲ. ಹಾಗಾಗಿ ಅದನ್ನು ಸಂಕೇತಿಸಲು ಸೂಕ್ಷ್ಮವಾಗಿಯೇ ಚಿತ್ರಿಸಲಾಗಿದೆ. ಆದರೆ ಎಲ್ಲದಕ್ಕೂ ಮೇಲಿರುವ ಪರಬ್ರಹ್ಮ ಪರಾದೃಷ್ಟಿಗೆ ಮಾತ್ರ ನಿಲುಕುವಂತಹುದು. ಅದು ದೇವತೆಗಳಿಗಿಂತ ಇನ್ನೂ ಸೂಕ್ಷ್ಮವಾದುದು, ಅದನ್ನು ಮತ್ತೂ ಸೂಕ್ಷ್ಮವಾಗಿ ಚಿತ್ರಿಸಬೇಕು. ಆದರೆ ಅದರ ಸ್ವರೂಪವೇ ಬೆಳಕಾಗಿರುವ ಕಾರಣದಿಂದಲೂ ಅದು ನಿರಾಕಾರವಾಗಿರುವುದರಿಂದಲೂ ಬೆಳಕನ್ನು ಬೆಳಕಾಗಿಯೇ ಚಿತ್ರಿಸಲಾಯಿತು.)  ಗುರು\enginline{-}ಶಿಷ್ಯ ಸಂಬಂಧ, ಅಧ್ಯಯನ\enginline{-}ಅಧ್ಯಾಪನಗಳು ಕಾಲ\enginline{-}ದೇಶಗಳಲ್ಲಿ ಯಾರಿಂದಲೇ ಸಂಪನ್ನವಾಗುತ್ತಿವೆಯೆಂದರೂ\break ಅವೆಲ್ಲವೂ ಭಾರತೀಯ ವಿದ್ಯಾಸಂಪ್ರದಾಯದ ನೆರಳಿನಲ್ಲಿಯೇ ಸಾಗುತ್ತವೆ ಎಂಬುದನ್ನು ಪ್ರತ್ಯೇಕವಾಗಿ ಹೇಳಬೇಕಿಲ್ಲ. (ಹೊರದೃಷ್ಟಿಯಿಂದ ಗುರುತಿಸುವುದು\break ಸಾಧ್ಯವಾಗದಿದ್ದರೂ ವಿದ್ಯೆಯ ಅಂತಃಸ್ರೋತಸ್ಸು ಹರಿಯದೆ ವಿದ್ಯೋಪಾಸನೆ ಸಾಧ್ಯವಿಲ್ಲ, ಕೆಲವೊಮ್ಮೆ ಆ ಸ್ರೋತಸ್ಸು ಪ್ರಭೂತವಾಗಿ ವ್ಯಕ್ತವಾಗಬಹುದು, ಇನ್ನೊಮ್ಮೆ\break ಸುಪ್ತವಾಗಿರಬಹುದು, ಇರುವ ವಿಷಯ ಮಾತ್ರ ಅದೇ !) ಪ್ರಕೃತ ಗಂಗಾಧರ ಭಟ್ಟರು ಅಧ್ಯಯನ\enginline{-}ಅಧ್ಯಾಪನದಲ್ಲೇ ತಮ್ಮ ಜೀವನದ ಸಾರವನ್ನು ಭಾವಿಸಿ ಬಾಳಿದವರಾದ್ದರಿಂದ ಒಮ್ಮೆ ವಿದ್ಯಾಸರಣಿ ಒಂದು ಮಟ್ಟದಲ್ಲಿ ಜಾಗೃತವಾಗಿದೆಯೆಂದರೆ ಅತಿಶಯವಲ್ಲ. ಇದು ನಮಗೆ ಪ್ರತ್ಯಕ್ಷ ವಿಷಯವೇ ಆಗಿದೆ. ಹಾಗಾಗಿ ಆ ಜ್ಞಾನಗಂಗೆಯ ಪ್ರವಾಹವೆಂಬ ಪರಂಪರೆಯಲ್ಲೇ ಶ್ರೀಯುತರ ಬದುಕಿಗೆ ಸ್ಥಾನವಿದೆ ಎಂಬಂಶವನ್ನು ಲಕ್ಷಿಸಿ ಅದಕ್ಕೆ ಸಾಂಕೇತಿಕವಾಗಿ ಗಂಗಾನದಿಯ ತಟದಲ್ಲಿ ಗಂಗಾಧರ ಭಟ್ಟರ ಅಧ್ಯಾಪನದ ಚಿತ್ರವನ್ನು ಅಳವಡಿಸಿದೆ. ಒಟ್ಟಾರೆ ಚಿತ್ರ ಅವರ ನಾಮಧೇಯದೊಂದಿಗೂ ಸಹ ವಿಶಿಷ್ಟವಾದ ಸಂಬಂಧವನ್ನು ಸಾಧಿಸುತ್ತದೆ. ‘ಗಂಗಾಧರ’ ಎಂಬುದು ವಿದ್ಯಾಧಿಪತಿಯಾದ ಶಿವನ ಹೆಸರೂ ಹೌದು. ಗಂಗೆಯನ್ನು ಧರಿಸಿ ಜ್ಞಾನಭೂಮಿಯೆನಿಸಿದ ಭಾರತದೇಶಕ್ಕೆ ಸಲ್ಲುವ ಹೆಸರೂ ಹೌದು. ಎ ಎಲ್ಲ ಹಿನ್ನೆಲೆಯಲ್ಲಿ ಗ್ರಂಥಕ್ಕೂ ಸಹ ಜ್ಞಾನಗಂಗಾಧರ ಎಂದು ನಾಮಕರಣ ಮಾಡಲಾಗಿದೆ. 

ಇವೆಲ್ಲದರಿಂದ ಗಂಗಾಧರ ಭಟ್ಟರನ್ನು ಅತಿಶಯಿಸಿ ಹೊಗಳಿ ಅಟ್ಟಕ್ಕೇರಿಸುವುದು ಇದರ ಉದ್ದೇಶವಲ್ಲ. ಅವರಂತೂ ಅದನ್ನು ಕಿಂಚಿತ್ತೂ ಇಷ್ಟಪಡುವವರಲ್ಲ. ನಮಗೂ ಅತ್ಯುಕ್ತಿ ಅಭ್ಯಾಸವಿಲ್ಲ. ಹಾಗಾಗಿ ಅವರಲ್ಲಿಯೂ ಮನುಷ್ಯ ಸಹಜವಾದ ದೊಷಗಳೇನೂ ಇಲ್ಲ ಎಂಬುದು ನಮ್ಮ ಅಭಿಪ್ರಾಯವಲ್ಲ. ಆದರೆ, ಏಕೋ ಹಿ ದೋಷೋ ಗುಣಸನ್ನಿಪಾತೇ \enginline{-} ಗುಣಗಳ ಗಣಗಳೇ ಇರುವಾಗ, ಅವುಗಳನ್ನೆಲ್ಲ ಮರೆಮಾಡುವ ದೋಷವಿಲ್ಲದೇ ಗೌಣವಾದ ಮಾನವ ಸಹಜವಾದ ದೋಷವಿದ್ದರೆ ಅದು ಎತ್ತಿ ಆಡುವ ವಿಷಯವೂ ಅಲ್ಲ, ಪ್ರಕೃತವೂ ಅಲ್ಲ. ಸಕಲಕಲಾಪರಿಪೂರ್ಣತೆ ಭಗವಂತನಿಗೆ ಮಾತ್ರ  ಎಂಬುದು ಸರ್ವವಿದಿತ. ಈ ದೃಷ್ಟಿಯಿಂದ ಅವರಿಂದ ಉಪಕೃತರಾದ ನಾವು \enginline{-} ವಿದ್ಯಾರ್ಥಿಗಳು, ಅಲ್ಲದೇ ಉಪಕೃತರಾದ ಇನ್ನೂ ಅನೇಕರ ಪ್ರತಿನಿಧಿಗಳಾಗಿ ಅವರನ್ನು ಗೌರವಿಸುವ ಕಾರ್ಯ ಕೈಗೊಂಡೆವು. ಗ್ರಂಥಪ್ರಕಟಿಸುವ ಸಂಕಲ್ಪ ಮಾಡಿದೆವು. ಈ ನಮ್ಮ ಕಾರ್ಯದಿಂದ ಸಮಾಜಕ್ಕೂ ಒಂದು  ಉತ್ತಮ ಸಂದೇಶ ಸಿಕ್ಕಿರಲೂ ಸಾಕು. ಅಂದರೆ, ಸಹ ನಾವವತು | ಸಹ ನೌ ಭುನಕ್ತು | ಸಹ ವೀರ್ಯಂ ಕರವಾವಹೈ | ತೇಜಸ್ವಿ  ನಾವಧೀತಮಸ್ತು | ಮಾ ವಿದ್ವಿಷಾವಹೈ | ಎಂಬ ಋಷಿನುಡಿಯಂತೆ ವಿದ್ಯಾರ್ಥಿ ಮತ್ತು ಅಧ್ಯಾಪಕರ ಉತ್ತಮ ಬಂಧಕ್ಕೂ, ಅಧ್ಯಾಪಕರಲ್ಲಿ ವಿದ್ಯಾರ್ಥಿಗಳಿಗಿರಬೇಕಾದ ಗೌರವಾದರಗಳಿಗೂ ಇದೊಂದು ಮಾದರಿಯೆಂದು ಯಾರಿಗಾದರೂ ಅನ್ನಿಸಿದರೆ ನಮ್ಮ ಕಾರ್ಯಕ್ರಮದ ಅವಾಂತರ ಫಲವೂ ನಮಗೆ ದೊರಕಿತೆಂದು ಸಂತೋಷಪಡುತ್ತೇವೆ.

ಇಂತಹ ನಮ್ಮ ಕಾರ್ಯಕ್ರಮಕ್ಕೆ ರಾಜಮಾತೆ ಡಾ | ಪ್ರಮೋದಾದೇವಿಯವರು ಆಗಮಿಸಿರುವುದು ಕಾರ್ಯಕ್ರಮದ ಗೌರವವನ್ನು ಹೆಚ್ಚಿಸಿದ್ದು ನಿಜ, ಅಂತೆಯೇ ಪ್ರಕೃತ ಗ್ರಂಥಕ್ಕೆ ಶುಭಸಂದೇಶವನ್ನು ದಯಪಾಲಿಸಿ ಗ್ರಂಥಗೌರವದ ವೃದ್ಧಿಗೂ ಅವರು ಕಾರಣರಾಗಿದ್ದಾರೆ. ಅವರಿಗೆ ಗಂಗಾಧರ ಭಟ್ಟರು ನಡೆದುಕೊಂಡ ರೀತಿಯಿಂದ ಅವರ ಬಗೆಗೂ ಪಾಠಶಾಲೆಯ ಬಗೆಗೂ  ಅಭಿಮಾನ ಮೂಡಿದೆ. ಅದರ ಫಲವಾಗಿ ಅವರು ಕಾರ್ಯಕ್ರಮಕ್ಕೂ ಆಗಮಿಸಿ ಸಂದೇಶವನ್ನೂ ನೀಡಿದ್ದಾರೆ. ಅವರಿಗೆ ನಾವು ಎಷ್ಟು ಕೃತಜ್ಞತೆಗಳನ್ನು ಸಲ್ಲಿಸಿದರೂ ಕಡಿಮೆಯೇ. ಅಂತೆಯೇ ಅರಮನೆಯ ಸಿಬ್ಬಂದಿ ವರ್ಗ ಸಹ ರಾಜಮಾತೆಯವರ ಸಂಪರ್ಕಕ್ಕೆ ಮತ್ತು ಅವರು ನಮ್ಮಲ್ಲಿಗೆ ದಯಮಾಡಿಸುವುದಕ್ಕೆ ಬಹಳ ಸಹಕಾರವನ್ನು ಕೊಟ್ಟಿದ್ದಾರೆ. ಅವರಿಗೆ ಅನಂತ ಕೃತಜ್ಞತೆಗಳು.

ಅಭಿವಂದನ ಸಮಿತಿಯ ಗೌರವಾಧ್ಯಕರಾದ ಡಾ | ನಿರಂಜನ ವಾನಳ್ಳಿಯವರು ನಮಗೆ ಆರಂಭದಿಂದಲೂ ಅತ್ಯಂತ ಉತ್ಸಾಹವನ್ನೂ, ಬೆಂಬಲವನ್ನೂ ನೀಡುತ್ತಾ ಒಡನಾಡಿ ವಿಭಾಗಕ್ಕೆ ಲೇಖನವನ್ನೂ ಕೊಟ್ಟು, ಗೌರವಾಧ್ಯಕ್ಷರ ನುಡಿಯನ್ನೂ ನೀಡಿದ್ದಾರೆ. ಅವರಿಗೆ ನಮ್ಮ ಅನಂತ ವಂದನೆಗಳು.

ಇದೇ ರೀತಿಯಲ್ಲಿ ಸಮಿತಿಯ ಅಧ್ಯಕ್ಷರಾದ ವಿ | ಕೆ.ಎಲ್.ರಾಘವರವರು ಎಲ್ಲ ಕಾರ್ಯವನ್ನೂ ಶ್ರಮವಹಿಸಿ ನಿರ್ವಹಿಸಿ ಶಾಸ್ತ್ರವಿಭಾಗಕ್ಕೆ ಸಂಶೋಧನ ಲೇಖನವನ್ನೂ, ಒಳನುಡಿ ವಿಭಾಗಕ್ಕೆ ಇನ್ನೊಂದು ಲೇಖನವನ್ನೂ ದಕ್ಷವಾದ ಅಧ್ಯಕ್ಷೀಯವನ್ನೂ ನೀಡುವ ಮೂಲಕ ಮುಖ್ಯಪಾತ್ರ ನಿರ್ವಹಿಸಿದ್ದಾರೆ. ಅವರಿಗೆ ಸಂಪಾದಕನ ಕೃತಜ್ಞತೆಗಳು ಸಲ್ಲುತ್ತವೆ

ಇನ್ನು, ಈ ಲೇಖನಗಳ ಡಿಟಿಪಿ ಕಾರ್ಯ ಸಾಮಾನ್ಯ ವಿಷಯವಾಗಿರಲಿಲ್ಲ, ಅವೆಲ್ಲವನ್ನೂ ಟಂಕಿಸಿರುವುದಲ್ಲದೇ, ತದನಂತರದ ಸ್ಖಾಲಿತ್ಯ ಪರಿಶೀಲನೆಯನ್ನೂ ಬಹಳ ಶ್ರಮ ವಹಿಸಿ ವಿದ್ಯಾರ್ಥಿವೃಂದ ನಿರ್ವಹಿಸಿದೆ. ವಿದ್ಯಾರ್ಥಿಗಳೆಲ್ಲರೂ ಸಹ ಪರಸ್ಥಳದಲ್ಲಿ ವಾಸಿಸುವವರು. ಅವರಲ್ಲೊಬ್ಬರಂತೂ ದೂರದ ಕಲ್ಕತ್ತಾ ವಾಸಿ \enginline{-} ವಿ | ನಾಗರಾಜ ಭಟ್ಟರು, ಅವರು ಉತ್ತಮವಾದ ಲೇಖನವನ್ನೂ ಕೊಟ್ಟು ಉಳಿದ ಲೇಖನಗಳ ಕರಡು ಪರಿಶೀಲನೆಯಲ್ಲೂ ಸಹಕರಿಸಿದ್ದಾರೆ. ಈ ಎಲ್ಲ ಕಾರ್ಯಗಳೂ ಅಂತರ್ಜಾಲದ\break ವ್ಯವಸ್ಥೆಯಲ್ಲಿಯೇ ನಡೆದಿವೆ. ಅವನ್ನೆಲ್ಲ ನಿರ್ವಹಿಸಿಕೊಟ್ಟ ನಮ್ಮೆಲ್ಲ ವಿದ್ಯಾರ್ಥಿ\break ಬಂಧುಗಳಿಗೆ ಸಂಪಾದಕನ ಮತ್ತು ಸಮಿತಿಯ ಅನಂತ ಕೃತಜ್ಞತೆಗಳು. 

ವಿದ್ಯಾರ್ಥಿಗಳು ಅವರವರಿಗೆ ಸೌಲಭ್ಯವಿರುವ ತಾಂತ್ರಿಕ ವ್ಯವಸ್ಥೆಯಲ್ಲಿ\break ಲೇಖನಗಳನ್ನು ಡಿಟಿಪಿ ಮಾಡಿದ್ದಾರೆ. ಕೆಲವು ಲೇಖಕರು ತಮ್ಮ ಲೇಖನವನ್ನು ತಾವೇ ಡಿಟಿಪಿ ಮಾಡಿಕೊಟ್ಟಿರುವುದೂ ಇದೆ. ಆದರೆ ಡಿಟಿಪಿಯಾದ ಲೇಖನಗಳು, ವಿವಿಧ ಮಾದರಿಯ ತಂತ್ರಜ್ಞಾನದ ಬಳಕೆಯ ಕಾರಣದಿಂದಲೂ, ಭಿನ್ನ ಭಿನ್ನ ಅಕ್ಷರಗಳ ಅಳತೆ ಮತ್ತು ವಿನ್ಯಾಸಗಳ ಕಾರಣದಿಂದಲೂ ಗ್ರಂಥಕ್ಕೆ ಬೇಕಾದಂತೆ ಒಂದೇ ಸ್ವರೂಪಕ್ಕೆ ಅಳವಡಿಸುವ ಕಾರ್ಯ ಮಾತ್ರ ನಮ್ಮೆಲ್ಲ ನಿರೀಕ್ಷೆಯನ್ನು ಮೀರಿ ಶ್ರಮ ಮತ್ತು ಕಾಲವನ್ನು ಅಪೇಕ್ಷಿಸಿತು. ಲೇಖನದ ಎಷ್ಟೋ ಅಂಶಗಳನ್ನು ಪುನಃ ಟಂಕಿಸುವಂತಾದುದೂ ಉಂಟು. ಅಷ್ಟಾದರೂ ಒಂದೆರಡು ಲೇಖನಗಳನ್ನು ಮಾತ್ರ ಅವುಗಳ ಲೇಖಕರೂ ಸಹ ಸಂಪರ್ಕಕ್ಕೆ ಸಿಗದೆ ಅವನ್ನು ಕೈಬಿಡುವ ಅನಿವಾರ್ಯತೆಯೂ ಉಂಟಾಯಿತು. ಈ ಕಾರ್ಯದಲ್ಲಿ ಬಹಳ ಶ್ರಮ ವಹಿಸಿದವರು ಇಂತಹ ಕಾರ್ಯಕ್ಕಾಗಿ ತಮ್ಮದೇ ಆದ ವಿಶಿಷ್ಟ ವ್ಯವಸ್ಥೆಯನ್ನು ಅಳವಡಿಸಿಕೊಂಡಿರುವ ಶ್ರೀರಂಗ ಡಿಜಿಟಲ್ ಸಾಫ್ಟ್ ವೇರ್ ಟೆಕ್ನಾಲಾಜಿಕಲ್ ಪ್ರೈವೇಟ್ ಲಿಮಿಟೆಡ್ ನ ತಂತ್ರಜ್ಞರು, ತತ್ರಾಪಿ ಸಂಸ್ಥೆಯ ನಿರ್ವಾಹಕರಾದ ಶ್ರೀ ಅರ್ಜುನ್, ಸಹ ಸಿಬ್ಬಂದಿಗಳಾದ ಶ್ರೀ ಡಿ.ಶಿವಶಂಕರ್ ಮತ್ತು ಶ್ರೀ ಕಿಶೋರ್ ರವರು ಅತ್ಯಂತ ವಿಶ್ವಾಸಪೂರ್ವಕವಾಗಿಯೂ ಗೌರವಪೂರ್ವಕವಾಗಿಯೂ ಕಾಲದ ಮಿತಿಯಲ್ಲಿ ಗ್ರಂಥಕ್ಕೆ ವ್ಯವಸ್ಥಿತಸ್ವರೂಪ ಕೊಡುವಲ್ಲಿ ಬಹಳ ಶ್ರಮಿಸಿದ್ದಾರೆ. ಇವರಿಗೆ ಸಂಪಾದಕನ ಮತ್ತು ಸಮಿತಿಯ ಅನಂತ ಕೃತಜ್ಞತೆಗಳು.

ಗ್ರಂಥದ ಮುಖಪುಟಕ್ಕೆ ಚಿತ್ರವನ್ನು ಸಾಕ್ಷಾತ್ಕರಿಸಿಕೊಟ್ಟವರು ಶ್ರೀಮಾನ್ ಲೋಕೇಶ್ ರವರು. ನಮ್ಮ ಕಲ್ಪನೆಯನ್ನು ಚಿತ್ರವಾಗಿಸಲು ಮತ್ತು ಇರುವ ಜಾಗದಲ್ಲಿ ಅದನ್ನೆಲ್ಲ ಅಳವಡಿಸಲು ಅವರು ಸಾಕಷ್ಟು ಶ್ರಮಿಸಬೇಕಾಯಿತು. ಆದರೂ ಬಹುಮಟ್ಟಿಗೆ ಅದು ಕೂಡಿಬಂದಿದೆಯೆಂದು ನಮ್ಮ ಭಾವನೆ. ಚಿತ್ರಕಾರರಾದ ಲೋಕೇಶರಿಗೆ ಕೃತಜ್ಞತೆಗಳನ್ನು ಸಲ್ಲಿಸುತ್ತೇವೆ.

ಪುಸ್ತಕವನ್ನು ಸರಸ್ವತೀಪುರಂನ ಮಹಾಲಕ್ಷ್ಮಿ ಮುದ್ರಣಾಲಯದವರು\break ಮುದವಾಗುವಂತೆ ಮುದ್ರಿಸಿಕೊಟ್ಟಿದ್ದಾರೆ. ಅದರ  ಮಾಲೀಕರಿಗೆ ನಾವು ಕೃತಜ್ಞರಾಗಿದ್ದೇವೆ.

ಗ್ರಂಥದ ಗುಣಮಟ್ಟಕ್ಕೆ ಪ್ರಮಾಣ ಸಹೃದಯರಾದ ಓದುಗರೇ ಎಂಬುದು\break ನಿರ್ವಿವಾದ. ಆದರೆ ವ್ಯಾವಹಾರಿಕ ದೃಷ್ಟಿಯಿಂದ ಕಾಲಾನುಗುಣವಾಗಿ ಗುಣಮಟ್ಟಕ್ಕೆ ಮಾನದಂಡದ ಪ್ರಕಾರಗಳು ವಿಭಿನ್ನವಾಗಿವೆ. ಹಾಗೆ ಗುಣಮಟ್ಟವನ್ನು\break ನಿರ್ಧರಿಸುವುದಕ್ಕಾಗಿಯೇ ದೆಹಲಿಯಲ್ಲಿ \enginline{Rajaram Mohan Roy National Agency for ISBN} ಎಂಬ ನಿರ್ದಿಷ್ಟವಾದ ಸಂಸ್ಥೆ ಸ್ಥಾಪಿತವಾಗಿದೆ. ಅದು ಕೆಲವು ಮಾನದಂಡಗಳನ್ನು ಅನುಸರಿಸಿ ಗುಣಮಟ್ಟ ನಿರ್ಧರಿಸುತ್ತದೆ. ಅನಂತರ ಗ್ರಂಥಕ್ಕೆ ಒಂದು ಸಂಖ್ಯೆಯನ್ನು ನೀಡುತ್ತದೆ. ಈ ಸಂಖ್ಯೆಯ ಉಲ್ಲೇಖ ಗ್ರಂಥದಲ್ಲಿದ್ದಾಗ ಅದಕ್ಕೊಂದು ವಿಶೇಷ ಮಾನ್ಯತೆ ಇದೆ. ಅಂತಹ ಗ್ರಂಥದಲ್ಲಿ ಲೇಖನ ಪ್ರಕಟವಾದರೆ ಅದರಿಂದ ಸಂಶೋಧಕ ಲೇಖಕರಿಗೆ ಕೆಲವು ಅನುಕೂಲತೆಗಳಿವೆ. ಈಗಿನ ವ್ಯಾವಹಾರಿಕ ಬದುಕಿಗೆ ಆ ಅನುಕೂಲತೆಯನ್ನು ನಾವು ನಿರಾಕರಿಸುವಂತಿಲ್ಲ. ಲೇಖನದ ಹಿಂದೆ ಸಾಕಷ್ಟು ಶ್ರಮವಿರುವುದರಿಂದ ನಮ್ಮ ಗ್ರಂಥದಲ್ಲಿ ಲೇಖನ ಪ್ರಕಟವಾಗಿ ಕನಿಷ್ಠ ಅವರಿಗೆ ಈ ಅನುಕೂಲವಾದರೂ ಆದರೆ ಅಷ್ಟರಮಟ್ಟಿಗೆ ನಮಗೊಂದು ಸಾರ್ಥಕ ಭಾವ. ಈ ದೃಷ್ಟಿಯಿಂದ ನಾವು \enginline{ISBN} ಸಂಖ್ಯೆಗೆ ಎಲ್ಲ ರೀತಿಯಿಂದ ಪ್ರಯತ್ನಿಸಿದೆವು. ಈ ವ್ಯವಹಾರ ನಮಗೆ ಮೊದಲ ಅನುಭವವಾದ್ದರಿಂದಲೂ ಅದನ್ನು ಪಡೆದುಕೊಳ್ಳುವಲ್ಲಿ ಸಾಕಷ್ಟು ವಿಲಂಬವಾಗಿದೆ. ಗ್ರಂಥ ಮುದ್ರಣದಲ್ಲಾದ ವಿಲಂಬಕ್ಕೆ  ಇದೇ ಒಂದು ಪ್ರಧಾನ ಕಾರಣ. ಕಡೆಗೂ ನಮಗೆ ಅದು ದೊರಕಿದೆ. ಇದರಿಂದ ಲೇಖಕರಿಗಾಗುವ ಅಲ್ಪ ಪ್ರಯೋಜನ ನಮಗೆ ಸಮಾಧಾನವನ್ನು ಉಂಟುಮಾಡಿದೆ. \enginline{ISBN} ಗೆ ನಮ್ಮ ಗ್ರಂಥವನ್ನು ಪರಿಗಣಿಸಿರುವುದಕ್ಕೆ ಸಂಸ್ಥೆಗೆ ನಾವು ಕೃತಜ್ಞತೆಯನ್ನು ಸಲ್ಲಿಸುತ್ತೇವೆ.
\vskip 4pt

ಇಷ್ಟು ಗುರುಗಾತ್ರದ ಗ್ರಂಥದಲ್ಲಿ ಸ್ಖಾಲಿತ್ಯಗಳಿಲ್ಲವೆಂದರೆ ಅದು ಸಾಹಸದ\break ಮಾತಾದೀತು. ಪುಸ್ತಕ ಪರಿಶ್ರಮಿಗಳಿಗೆಲ್ಲ ಇದು ತಿರೋಹಿತವಾದ ವಿಷಯವೇನೂ ಅಲ್ಲ. ಸದಸದ್ವಿವೇಕವುಳ್ಳ ಸಹೃದಯಿಗಳು ಗುಣವನ್ನು ಗ್ರಹಿಸುವರೆಂಬ ವಿಶ್ವಾಸ ನಮಗಿದೆ. ಇದು ವಿದ್ಯಾರ್ಥಿಗಳ ಪರಿಶ್ರಮ. ಇಲ್ಲಿ ದೋಷಗಳೇನಾದರೂ ಕಂಡುಬಂದಲ್ಲಿ\break ಅವುಗಳನ್ನು ಗುರುಜನರು ನಮ್ಮ ಗಮನಕ್ಕೆ ತಂದರೆ ಪುನಃ ಪರಿಷ್ಕಾರಕ್ಕೆ\break ಅನುಕೂಲವಾಗುವ ದೃಷ್ಟಿಯಿಂದ ಅಂತಹ ಆಪ್ತ ಸಲಹೆಗಳನ್ನೆಲ್ಲ ಸ್ವಾಗತಿಸುತ್ತೇವೆ, ಉಪಕೃತಿಯನ್ನು ಭಾವಿಸುತ್ತೇವೆ.
\vskip 4pt

ಅಂತಿಮವಾಗಿ \enginline{-} ನಮ್ಮ ಕಾರ್ಯಕ್ಕೆ ಪ್ರತ್ಯಕ್ಷ\enginline{-}ಪರೋಕ್ಷವಾಗಿ ಸಹಕರಿಸಿದ ಎಲ್ಲರಿಗೂ ಅನಂತ ಕೃತಜ್ಞತೆಗಳು. ಈ ಎಲ್ಲ ನಮ್ಮ ಕಾರ್ಯಕಲಾಪ ಸಂಪನ್ನವಾಗಿದ್ದರೂ, ಶ್ರೀಮಾನ್ ಗಂಗಾಧರ ಭಟ್ಟರು ನಮ್ಮಲ್ಲಿ ಹೊಂದಿರುವ ವಾತ್ಸಲ್ಯ\enginline{-}ವಿಶ್ವಾಸಗಳೇ ಅವೆಲ್ಲಕ್ಕೂ ಕಾರಣ. ಅಷ್ಟೆ ಸಮ \enginline{-} ಸಮ ಪಾಲು ಶ್ರೀಯುತರ ಸಹಧರ್ಮಿಣೀ \enginline{-} ಶೈಲಜಾರವರಿಗೂ ಸಲ್ಲಲೇ ಬೇಕು.  ನಮ್ಮೆಲ್ಲ ಕಾರ್ಯದಿಂದ ಅವರಿಬ್ಬರ ಋಣ ತೀರಿಸಿದೆಂಬ ಭ್ರಮೆ ನಮಗಿಲ್ಲ. ಅವರ್ರು ಕೊಟ್ಟಿರುವುದನ್ನು ವ್ಯ್ಕ್ತಪಡಿಸುವ ಪ್ರಯತ್ನವಿದಾಗಿದೆಯೇ ವಿನಾ ನಾವವರಿಗೆ ಏನನ್ನೂ ಕೊಟ್ಟಿಲ್ಲ. ಅವರ ಪರಿಶ್ರಮ ನಮ್ಮ ಪರಿಶ್ರಮವೂ ಆದಾಗ ಅಲ್ಲೊಂದು ತೃಪ್ತಿಗೆ ವಿಷಯವಿರಬಹುದಷ್ಟೆ. ನಾವು  ಈ ಕಾರ್ಯಕಲಾಪಗಳ ನೆಪದಲ್ಲಿ ನಡೆಸಿದ ಚಟುವಟಿಕೆಗಳಲ್ಲಿ ಅವರ ವ್ಯವಹಾರ ಗಾಂಭೀರ್ಯಕ್ಕಾಗಲೀ ವಿಷಯ\break ಗಾಂಭೀರ್ಯಕ್ಕಾಗಲೀ ಹೊಂದದ ಅಂಶಗಳಿದ್ದರೆ ಅಂಥವುಗಳನ್ನು ಅವರು\break ಮನ್ನಿಸಬೇಕೆಂದು ವಿನಂತಿಸುತ್ತೇವೆ. ಅವರಿಗೆ ಆಯುರಾರೋಗ್ಯಭಾಗ್ಯ ಕೂಡಿಬರಲೆಂದು ನಮ್ಮೆಲ್ಲರ ಪ್ರಾರ್ಥನೆ. ಒಬ್ಬ ಅಧ್ಯಾಪಕರಾಗಿ ಪಾಠವನ್ನೇ ಜೀವನವಾಗಿಸಿಕೊಂಡ ಜೀವನವನ್ನೆಲ್ಲ ಪಾಠವಾಗಿಸಿದ ಚೇತನಕ್ಕೆ ಈ ಗ್ರಂಥಸುಮ ಸಮರ್ಪಿತ.
\bigskip

\centerline{|| ಭದ್ರಂ ಶುಭಂ ಮಂಗಳಮ್ ||}
\bigskip

\noindent
ಸ್ಥಳ : ಮೈಸೂರು\hfill \textbf{ವಿ | ಗುರುಪ್ರಸಾದ}

\noindent	
ವಿಲಂಬ ನಾಮ ಸಂವತ್ಸರ\hfill ಪ್ರಧಾನ ಸಂಪಾದಕ

\noindent	
ಅಧಿಕಜ್ಯೆಷ್ಠ ಕೃಷ್ಣ ತೃತೀಯಾ, ಗುರುವಾರ\hfill ವಿ | ಗಂಗಾಧರ ಭಟ್ಟರ

\noindent
ದಿನಾಂಕ : ೩.೦೫.೨೦೧೮\hfill ಅಭಿವಂದನಗ್ರಂಥ ಸಂಪಾದನ ಸಮಿತಿ
}				
