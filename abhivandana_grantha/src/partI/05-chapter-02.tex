{\fontsize{14}{16}\selectfont
\chapter{ಗೌರವಾಧ್ಯಕ್ಷರ ನುಡಿ}

\begin{wrapfigure}{l}{0.25\textwidth}
\centerline{\includegraphics[scale=0.5]{figures/niranjana.png}}
\end{wrapfigure}
ಶ್ರೀಮಾನ್ ಗಂಗಾಧರ ಭಟ್ಟರು ವಿದ್ವನ್ಮಾನ್ಯರು, ಅಂತೆಯೇ ವಿದ್ಯಾರ್ಥಿಗಳಿಗೆ ಅತ್ಯಂತ ಪ್ರೀತಿಪಾತ್ರರು. ಪ್ರಕೃತ\break ಶ್ರೀಮನ್ಮಹಾರಾಜ ಸಂಸ್ಕೃತ ಕಾಲೇಜಿನಲ್ಲಿ ಅಧ್ಯಾಪನ ಮಾಡಿ ಸೇವಾನಿವೃತ್ತಿಯನ್ನು ಪಡೆದಿದ್ದಾರೆ. ಇವರು 1975ರಲ್ಲಿಯೇ ಮೈಸೂರಿಗೆ ಬಂದು ಬಹಳ ಕಷ್ಟದಿಂದ ಜೀವನ ಸಾಗಿಸಿದವರು. ಕಷ್ಟವೆಷ್ಟಿದ್ದರೂ ಅದನ್ನೆಲ್ಲ ತಾವೇ ಜೀರ್ಣಿಸಿಕೊಂಡು ತಮ್ಮ ಕುಟುಂಬದ ಸದಸ್ಯರನ್ನೆಲ್ಲ ಇಲ್ಲಿಗೆ ಕರೆತಂದು ಅಧ್ಯಯನ ಮಾಡಿಸಿದರು. ಅದಕ್ಕಿಂತ ವಿಶೇಷವಾಗಿ ಉತ್ತರಕನ್ನಡ ಜಿಲ್ಲೆಯಿಂದ ಮತ್ತು ಸಾಗರ ತಾಲೂಕಿನಿಂದ ಸಂಸ್ಕೃತ ಓದಬರುವ ವಿದ್ಯಾರ್ಥಿಗಳಿಗೆ ಆಶ್ರಯ ಕೊಟ್ಟು, ವಿದ್ಯಾರ್ಜನೆಗೆ ಸಹಾಯ ಮಾಡಿ, ವಸತಿ, ಊಟ ವ್ಯವಸ್ಥೆ ಮಾಡಿ, ವ್ಯಾವಹಾರಿಕವಾಗಿಯೂ ಸಲಹೆ\-ಗಳಿಂದ ಸಲುಹಿ ಅವರೆಲ್ಲ ಇಂದು ಉತ್ತಮವಾಗಿ ಜೀವನ ಸಾಗಿಸುವಲ್ಲಿ ಹೆತ್ತವರಿಗಿಂತ ಮಿಗಿಲಾದ ಪಾತ್ರ\-ವಹಿಸಿದ್ದಾರೆ. ಹಿಂದಿನ ಕಾಲದ ಗುರುಕುಲ ಈಗಿಲ್ಲ. ಆದರೆ ಮೈಸೂರಿನಲ್ಲಿ ಒಂದು ಗುರುಕುಲದ ಕಾರ್ಯವನ್ನು ಶ್ರೀಮಾನ್ ಗಂಗಾಧರ ಭಟ್ಟರು ನಿರ್ವಹಿಸಿದ್ದಾರೆ ಎಂಬುದಕ್ಕೆ ಅವರ ವಿದ್ಯಾರ್ಥಿಸಮೂಹ ಸಾಕ್ಷಿಯಾಗಿದೆ. ನಾನೂ ಸಹ ಅವರ ಮನೆಯಲ್ಲಿ ಊಟಮಾಡಿದವನು. ಹಾಗಾಗಿ ಅವರ ಈ ಗುಣಗಳಿಗೆ ಸಾಕ್ಷಿಯಾಗಿ ಕೃತಜ್ಞನಾಗಿದ್ದೇನೆ.
	
ಅವರು ಪಾಠ-ಪ್ರವಚನ, ವಾಕ್ಪಟುತ್ವಗಳಿಂದ ಮಾತ್ರ ಎಲ್ಲರನ್ನು ಗೆದ್ದವರಲ್ಲ. ವಿದ್ಯೆಗೆ ತಕ್ಕನಾದ ಅಂತರ್ಬಾಹ್ಯ ವ್ಯವಹಾರ ಸಾಮ್ಯ, ವದಾನ್ಯ ಸ್ವಭಾವ, ಸಕ್ಕರೆಯಂಥ ಸೌಹೃದಸ್ವಭಾವ ಇವುಗಳು ಸೂಜಿಗಲ್ಲಿನಂತೆ ಲೋಕವನ್ನು ಅವರತ್ತ ಸೆಳೆದಿವೆ. ಮಾತ್ರವಲ್ಲ ಅವರ ಆ ಭಾವಗಳು ಕಿಂಚಿತ್ತೂ ಕೃತಕವಾಗಿರದ ಕಾರಣ ಆಕರ್ಷಿತ ಲೋಕ ಅವರಲ್ಲಿ ಶಿಥಿಲವಾಗದ ಮೈತ್ರಿಯನ್ನು ಸರಸವಾಗಿ ಉಳಿಸಿಕೊಂಡಿರಿವುದೂ ಅಷ್ಟೆ ವಿಶೇಷ. ಲೋಕದಲ್ಲಾದರೋ ಮೈತ್ರಿ ಕ್ಷಣಿಕವಾಗುತ್ತಿರುವುದನ್ನು ನೋಡುತ್ತಲೇ ಇದ್ದೇವೆ ತಾನೆ. ಇಲ್ಲಿ ಅದು ಹಾಗಾಗಿಲ್ಲ.
\eject
 
ಅವರು 2018 ಜನವರಿ 30ರಂದು ನಿವೃತ್ತರಾಗಿದ್ದಾರೆ. ಅವರ ಆಪ್ತ ವಿದ್ಯಾರ್ಥಿ ಸಮೂಹ ಅವರಿಗೆ ಅಭಿವಂದನ ಕಾರ್ಯಕ್ರಮ ಹಮ್ಮಿಕೊಂಡಿದ್ದು ಕೃತಜ್ಞತಾ ಭಾರವನ್ನು ಸಾಧ್ಯವಾದಷ್ಟು ಇಳಿಸಿಕೊಳ್ಳುವ ಪ್ರಯತ್ನ ಮಾಡಹೊರಟಿದೆ. 
\vskip 4pt

ಅವರ ಕಾರ್ಯಕ್ರಮ ನಿರ್ವಹಣೆಗಾಗಿ ಮಾಡಿಕೊಂಡ ಸಮಿತಿಗೆ ನನ್ನನ್ನು\break ಗೌರವಾಧ್ಯಕ್ಷರನ್ನಾಗಿ ಗುರುತಿಸಿ ಸಮಿತಿಯ ವತಿಯಿಂದ ಶ್ರೀಗುರುಪ್ರಸಾದರವರು ನನ್ನ ಬಳಿಗೆ ಬಂದು ತಮ್ಮ ಅಭಿಲಾಷೆಯನ್ನು ವ್ಯಕ್ತಪಡಿಸಿದಾಗ ಆ ಅವಸರ ನನ್ನನ್ನು ಭಾವತೀವ್ರತೆಗೆ ತಳ್ಳಿಬಿಟ್ಟಿತು. ಒಮ್ಮೆ ಹಳೆಯ ಸುರುಳಿಯೆಲ್ಲ ಬಿಚ್ಚಿಕೊಂಡು\break ಸುಧಾರಿಸಿಕೊಳ್ಳಲು ಕೆಲವು ಕ್ಷಣಗಳೇ ಹಿಡಿದವು. ಇಷ್ಟೆಲ್ಲ ವಿದ್ವಾಂಸರ ಮಧ್ಯದಲ್ಲಿ ಚಿಕ್ಕವನಾದ ನನಗೆ ಈ ಹುದ್ದೆ ಬಹಳ ಭಾರವೆನಿಸಿತು. ನಾನು ಈ ಹುದ್ದೆಗೆ ಬಹಳ ಚಿಕ್ಕವನು ಎಂದೆ. ಆದರೆ ಅವರ ಒತ್ತಾಸೆ ಅಚಲವಾಗಿತ್ತು. ನಾನು ಇದೊಂದು ಭಾಗ್ಯವೆಂದು ಭಾವಿಸಿದೆ.
\vskip 4pt

ಇಲ್ಲಿ ಗಂಗಾಧರ ಭಟ್ಟರ ಶಿಕ್ಷಣದಿಂದ ಶಿಕ್ಷಿತರಾದ ವಿದ್ಯಾರ್ಥಿಗಳೇ ಎಲ್ಲ\break ಜವಾಬ್ದಾರಿಯನ್ನು ಬಹಳ ಪ್ರೌಢವಾಗಿ ನಿರ್ವಹಿಸಿದ್ದಾರೆ. ಕಾರ್ಯಕ್ರಮ ಬಹಳ ಚೆನ್ನಾಗಿ ಆಯೋಜನೆಯಾಗಿದೆ. ಐತಿಹಾಸಿಕ ಗಾಂಭೀರ್ಯವುಳ್ಳ ಪಾಠಶಾಲೆಯ ಸರಸ್ವತೀ ಪ್ರಾಸಾದದಲ್ಲಿ ಅದು ಸಂಪನ್ನವಾಗಲಿದೆ. ಬೆಳಿಗ್ಗೆ ಹಿರಿಯ ವಿದ್ವಾಂಸರಿಂದ ವಿದ್ವದ್ಗೋಷ್ಠೀ.  ಮಧ್ಯಾಹ್ನ ಅಭಿವಂದನ ಕಾರ್ಯಕ್ರಮದಲ್ಲಿ ರಾಜಮಾತೆ ಡಾ। ಪ್ರಮೋದಾ\break ದೇವಿಯವರ ವಿಶೇಷ ಉಪಸ್ಥಿತಿ ಅವಸರದ ಸ್ತರವನ್ನು ಬಹಳ ಎತ್ತರಕ್ಕೆ ತಂದು ನಿಲ್ಲಿಸಿದೆ. ಇದು ಇನ್ನೊಂದು ಐತಿಹಾಸಿಕ ಸಂದರ್ಭಕ್ಕೆ ಸಾಕ್ಷಿಯಾಗಿದೆ. ಇದೇ ಸಂದರ್ಭದಲ್ಲಿ ಗಂಗಾಧರ ಭಟ್ಟರ ಅಭಿಮಾನಿ ವಿದ್ವಾಂಸರುಗಳು ಮತ್ತು ವಿದ್ಯಾರ್ಥಿಗಳು ಬರೆದ ಲೇಖನ ಸುಮಗಳಿಂದ ಅಲಂಕೃತವಾದ ಬೃಹದ್ ಗ್ರಂಥಹಾರವು ಜ್ಞಾನಗಂಗಾಧರ ಎಂಬ ಶುಭನಾಮಾಂಕಿತವಾಗಿ ಅಭಿವಂದನೆಯ ಮತ್ತೊಂದು ರೂಪವಾಗಿ ಲೋಕಕ್ಕೆ ಸಮರ್ಪಿತವಾಗುತ್ತಿದೆ. ಅನೇಕ ಲೇಖಕರ ವಿದ್ವತ್ಪೂರ್ಣ ಲೇಖನಗಳನ್ನೊಳಗೊಂಡ ಈ ಅಭಿವಂದನ ಗ್ರಂಥ ವಿದ್ವಾಂಸರಿಗೂ ವಿದ್ಯಾರ್ಥಿಗಳಿಗೂ ಒಂದು ಉತ್ತಮ\break ಕೈಪಿಡಿಯಾಗುವುದರಲ್ಲಿ ಸಂಶವಿಲ್ಲ. ಇದರ ಜೊತೆಗೆ ಗಂಗಾಧರ ಭಟ್ಟರ ಒಡನಾಡಿಗಳು ಬರೆದಿರುವ ಲೇಖನಗಳು ಜೀವನದ ಅನೇಕ ರಸಮಯ \enginline{ -} ಸಮಯವನ್ನು ಅನಾವರಣ ಮಾಡುವ ಮೂಲಕ ರಸಿಕರಿಗೆ ಆಸ್ವಾದನೀಯ\-ವಾಗುವುದೆಂಬ ವಿಶ್ವಾಸವಿದೆ. ಇಷ್ಟಲ್ಲದೇ ಭಟ್ಟರ ಕುರಿತಾದ ಸಾಕ್ಷ್ಯಚಿತ್ರವೊಂದನ್ನು ಸಾಕ್ಷವಾಗಿಸುವ ಸಾಹಸವನ್ನು ಇದೇ\break ವಿದ್ಯಾರ್ಥಿಗಳು ಮಾಡಿರುವುದು ಈ ಸಂದರ್ಭದ ಇನ್ನೊಂದು ವಿಶೇಷವೇ ಸರಿ. 

ಅಂತೂ ಒಂದು ಭಾವಪೂರ್ಣವೂ ವಿದ್ವಜ್ಜನಮಾನ್ಯವೂ ಆದ ಅಭಿವಂದನ\break ಕಾರ್ಯಕ್ರಮ ಏರ್ಪಟ್ಟಿರುವುದು ಈ ನಾಡಿಗೆ ಬಹಳ ಹೆಮ್ಮೆಯ ವಿಚಾರ. ಕೃತಜ್ಞ ಸಮಾಜ ಅಪೇಕ್ಷಿಸಿದ ಶುಭಾವಸರ. ನಾನು ಇಲ್ಲಿ ಭಾಗವಹಿಸುವಂತಾದುದೊಂದು ನನ್ನ ಜೀವನದ ಒಂದು ಭಾಗ್ಯ ಎಂದು ಭಾವಿಸುತ್ತೇನೆ.

ಇಷ್ಟೆಲ್ಲ ವಿದ್ಯಾರ್ಥಿಗಳ ಭಾವನೆ, ಉತ್ಸಾಹಗಳಿಗೆ ಮೂಲ ಸೆಲೆ ಶ್ರೀಮಾನ್\break ಗಂಗಾಧರ ಭಟ್ಟರು. ಅವರು ನಿವೃತ್ತರಾದರೂ ನಿವೃತ್ತರಲ್ಲ ಎಂಬುದು ಸರಳಸತ್ಯ. ಅವರ ಮುಂದಿನ ಜೀವನ ಮಧುರವಾಗಿಯೂ ಸರಸವಾಗಿಯೂ ಆನಂದವಾಗಿಯೂ  ಇರಲೆಂದು ನಾವೆಲ್ಲ ಒಕ್ಕೊರಲಿಂದ ಹಾರೈಸೋಣ.

\bigskip

\bigskip

\bigskip

\noindent
ಹೇಮಲಂಬಿ ಸಂವತ್ಸರ, ಮಾಘಮಾಸ\hfill \textbf{ಡಾ. ನಿರಂಜನವಾನಳ್ಳಿ}\\
ಭಾನುವಾರ, 4\eng{-}2\eng{-}2018\hfill ಕುಲಸಚಿವರು\\
ಮೈಸೂರು\hfill ಡಾ. ಗಂಗೂಬಾಯಿ ಹಾನಗಲ್ ಸಂಗೀತ \\
\phantom{i}\hfill ಮತ್ತು ಪ್ರದರ್ಶಕ ಕಲೆಗಳ ವಿಶ್ವವಿದ್ಯಾಲಯ,\\
\phantom{i}\hfill ಮೈಸೂರು

\articleend}
