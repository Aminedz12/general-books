{\fontsize{14}{16}\selectfont

\chapter{ಶಿಷ್ಯಪ್ರಿಯ ಆಚಾರ್ಯರಿಗೊಂದು ಆತ್ಮೀಯ ಅಭಿವಂದನೆ \eng{--} ಒಂದು ಪಕ್ಷಿನೋಟ}

ಸಿದ್ಧಾಪುರ ತಾಲೂಕಿನ ಮಣ್ಣಿಕೊಪ್ಪ ಗ್ರಾಮದವರಾದ ಶ್ರೀಮಾನ್ ಗಂಗಾಧರ ಭಟ್ಟರು ಮೈಸೂರಿನಲ್ಲಿ ತಮ್ಮ ಅಧ್ಯಯನವನ್ನೆಲ್ಲ ಮುಗಿಸಿ ೧೯೯೮\enginline{-}೯೯ರಲ್ಲಿ ಶ್ರೀಮನ್ಮಹಾರಾಜ ಸಂಸ್ಕೃತ ಮಹಾಪಾಠಶಾಲೆಯಲ್ಲಿ ನ್ಯಾಯಶಾಸ್ತ್ರದ ಅಧ್ಯಾಪಕರಾಗಿ \hbox{ನಿಯುಕ್ತರಾದರು}. ಅವರ ನಿಯುಕ್ತಿಯಿಂದ ನಿವೃತ್ತಿಪರ್ಯಂತ ಶ್ರೀಯುತರ ಛಾತ್ರನಿಕಾಯದಲ್ಲಿ ನಾವು ಸುಮಾರು ೨೫ ವಿದ್ಯಾರ್ಥಿಗಳು \enginline{-} ಶಾಸ್ತ್ರಗಳಲ್ಲಿ ಕಬ್ಬಿಣದ ಕಡಲೆಯೆಂದು ಪ್ರಸಿದ್ಧಿ\-ಯಿದ್ದರೂ ವಿವಿಧ ಕಾರಣಗಳಿಂದ ನ್ಯಾಯಶಾಸ್ತ್ರದ ಕಕ್ಷೆಗೆ ಬಂದವರು. ಆ\break ಕಾರಣಗಳಲ್ಲಿ ವಿದ್ಯಾರ್ಥಿಗಳಿಗೆ ಗಂಗಾಧರ ಭಟ್ಟರಲ್ಲಿರುವ ಅವರ ಪಾಠದ ಶೈಲಿಯಲ್ಲಿರುವ ಆಕರ್ಷಣೆಯೂ ಒಂದು ಪ್ರಧಾನ ಅಂಶ. ಆ ಕರ್ಷಣೆಯ ಪ್ರಭಾವ ಎಲ್ಲ\break ವಿದ್ಯಾರ್ಥಿಗಳಲ್ಲಿಯೂ ಆಳವಾಗಿ ಇರುವುದರಿಂದ ಅವರ ನಿವೃತ್ತಿಯ ಅವಸರದಲ್ಲಿ\break ಅಭಿವಂದನಕಾರ್ಯಕ್ರಮದ ವ್ಯಾಜದಿಂದ ಅವರ ಪಾಠಮಾತ್ರವಲ್ಲದ ಸರ್ವತೋಪ\-ಕಾರಕ ಸ್ವಭಾವಕ್ಕೆ ಕೃತಜ್ಞತಾಭರಿತವಾದ ಗುರುವಂದನೆಯನ್ನು ಸಮರ್ಪಿಸುವ ಆಶಯ\-ವುಳ್ಳವರಾದೆವು. ಆದರೆ ವಿದ್ಯಾರ್ಥಿಗಳಲ್ಲಿ ವಿ । ಗುರುಪ್ರಸಾದರನ್ನು ಬಿಟ್ಟರೆ ಉಳಿದವ\-ರಾರೂ ಮೈಸೂರಿನಲ್ಲಿಲ್ಲ, ಉದ್ಯೋಗಕ್ಕಾಗಿ ದೂರದ ಊರಿನಲ್ಲಿ ನೆಲೆಸಿದ್ದೇವೆ.\break ಉದ್ದಿಷ್ಟ ಕಾರ್ಯಕ್ಕಾಗಿ ಎಲ್ಲರೂ ಕಲೆಯುವುದು ಅವಶ್ಯವಾದರೂ ಸುಲಭಸಾಧ್ಯ\-ವಾದುದಲ್ಲ. ಹಾಗಾಗಿ ೧೩.೧೦.೨೦೧೭ರಂದು ಒಂದು ವಾಟ್ಸಾಪ್ ಗ್ರೂಪ್ ರಚಿಸಿ\-ಕೊಂಡೆವು. ಒಂದು ದಿನ ಕಾರ್ಯಕ್ರಮದ  ರೂಪುರೇಷೆಯನ್ನು ರೂಪಿಸಲು ಒಂದು ಜಾಗದಲ್ಲಿ ಒಂದಾಗಬೇಕೆಂದು ತೀರ್ಮಾನಿಸಿ ೧೨.೧೧.೨೦೧೭ರಂದು ಮೈಸೂರಿನಲ್ಲಿ ಕಲೆತೆವು. ಕಲೆತು ಒಂದು ಜವಾಬ್ದಾರಿಯನ್ನು ನಿರ್ವಹಿಸುವ ದೃಷ್ಟಿಯಿಂದ ನಾಮ್ಕೇವಾಸ್ತೇ ಕಮಿಟಿಯೊಂದನ್ನು ಕಟ್ಟಿಕೊಂಡೆವು. (“ನಾಮ್ಕೇವಾಸ್ತೆ ಎಂದರೆ \enginline{-} ಯಾವತ್ತೂ ನಮ್ಮ ಗೌರವಕ್ಕೆ ಪಾತ್ರರಾದ ಪಾಠಶಾಲೆಯ ವಿದ್ಯಾಗುರುಗಳ ಎದುರಿನಲ್ಲಿ ಅದರಲ್ಲೂ ವಿಶೇಷವಾಗಿ ನಾವೇ ಅಭಿವಂದಿಸಿ ಗೌರವಿಸಲಿರುವ ಶ್ರೀಯುತರ ಸಮಕ್ಷದಲ್ಲಿ, “ನಾನು \enginline{-} ಅಧ್ಯಕ್ಷ, ಉಪಾಧ್ಯಕ್ಷ, ಕಾರ್ಯದರ್ಶೀ ಎಂದೆಲ್ಲ ದೊಡ್ದ ದೊಡ್ಡ ಹುದ್ದೆಯ\break ಹೆಸರಿನಿಂದ ಕರೆದುಕೊಂಡು, ಕರೆಸಿಕೊಂಡು ಓಡಾಡುವುದು ಉಚಿತವೂ ಅಲ್ಲದ,\break ಇಷ್ಟವೂ ಇಲ್ಲದ ಅತ್ಯಂತ ಮುಜುಗರದ ವ್ಯವಹಾರವಾಗಿ ಕಂಡಿತ್ತು. ಆದರೆ ಜವಾಬ್ದಾರಿ\-ಯನ್ನು ಸೂಕ್ತ\-ವಾಗಿ ನಿಭಾಯಿಸಲು ಒಂದು ದೃಷ್ಟಿಯಿಂದ ‘ಸಮಿತಿ’ ಅವಶ್ಯಕವೂ ಆಗಿತ್ತು ಅನ್ನುವುದು ಬೇರೆಮಾತು. ಆದರೆ ಇದೆಲ್ಲಕ್ಕಿಂತ ಮಿಗಿಲಾಗಿ, “ವಿದ್ಯಾರ್ಥಿ ಎಷ್ಟೇ ದೊಡ್ಡವನಾದರೂ ವಿದ್ಯೆ ಕೊಟ್ಟ ಗುರುವು ಸಹಜವಾಗಿ ಯಾವತ್ತೂ ತನ್ನ\break ವಿದ್ಯಾರ್ಥಿಗಿಂತ ಉನ್ನತ ಸ್ಥಾನದಲ್ಲಿಯೇ ಇರುವನೆಂಬ” ಗುರುಗಳೇ ಕೊಟ್ಟ \hbox{ವಿವೇಕದ} ಹಿನ್ನೆಲೆಯಲ್ಲಿ ಮುಜುಗರಪಡದೇ ಕಾರ್ಯ ನಿರ್ವಹಿಸಲು ಸಮಿತಿ ರಚಿಸಿ ತಾತ್ಕಾಲಿಕ\-ವಾದ ಅಧ್ಯಕ್ಷ ಇತ್ಯಾದಿ ಸ್ಥಾನ  ಸ್ವೀಕರಿಸಲಾಯಿತು. ಈ ದೃಷ್ಟಿಯಿಂದ ರಚಿಸಿಕೊಂಡ ಸಮಿತಿ ಅದು ನಾಮ್ಕೇವಾಸ್ತೆ ಸಮಿತಿ ಮಾತ್ರ.) ಈ ಸಮಿತಿಗೆ ವಿ । ಕೆ.ಎಲ್ ರಾಘವ\-ರನ್ನು ಅಧ್ಯಕ್ಷರನ್ನಾಗಿ ಮಾಡಲಾಯಿತು. ಡಾ । ನಿರಂಜನ್ ವಾನಳ್ಳಿಯವರನ್ನು ಗೌರವಾಧ್ಯಕ್ಷರನ್ನಾಗಿ ಸ್ವೀಕರಿಸಿ, ನಾಡಿನ ಹಿರಿಯ ವಿದ್ವಾಂಸರಾದ  ಹೆಚ್.ವಿ.ನಾಗರಾಜ ರಾವ್, ಡಾ । ಟಿ.ವಿ.ಸತ್ಯನಾರಾಯಣ, ವಿ । ಉಮಾಕಾಂತ ಭಟ್ಟರೇ ಮುಂತಾದವ\-ರನ್ನು ಗೌರವ ಸಲೆಹೆಗಾರರನ್ನಾಗಿ ಸ್ವೀಕರಿಸಿದೆವು. ಅದೇ ಸಮಯದಲ್ಲಿ ಕಾರ್ಯಕ್ರಮದ ಒಂದು ಸ್ಥೂಲ ಸ್ವರೂಪನ್ನು ನಿರ್ಣಯಿಸಲಾಯಿತು. ಸ್ಪಷ್ಟ ರೂಪ ಕಾಲಕ್ರಮದಲ್ಲಿ\break ಕೂಡಿಬಂತು. ಅದರ ಪ್ರಕಾರ ಒಂದು ದಿನ ಪೂರ್ತಿ ಕಾರ್ಯಕ್ರಮವನ್ನು ಹಮ್ಮಿಕೊಂಡು ಬೆಳಿಗ್ಗೆ ವಿದ್ವದ್ಗೋಷ್ಠೀ, ಮಧ್ಯಾಹ್ನ ಅಭಿವಂದನ ಕಾರ್ಯಕ್ರಮ ನೆರವೇರಿಸುವು\-ದೆಂದು ತೀರ್ಮಾನಿಸಿದೆವು. ಇದಕ್ಕೆ ಸಂಬಂಧಿಸಿದ ಕೆಲಸಗಳನ್ನು ಎಲ್ಲರೂ ಹಂಚಿಕೊಂಡೆವು. ಇದೇ ಸಂದರ್ಭದಲ್ಲಿ ಅಭಿವಂದನ ಗ್ರಂಥವೊಂದನ್ನು ಸಿದ್ಧಪಡಿಸಿ ಪ್ರಕಾಶಿಸ\-ಬೇಕೆಂದು ಮತ್ತೊಂದು ಪ್ರಧಾನವಾದ ನಿರ್ಣಯವನ್ನೂ ಅಂಗೀಕರಿಸಿದೆವು. ಆ ಗ್ರಂಥದ\break ಜವಾಬ್ದಾರಿಯನ್ನು ವಿ । ಗುರುಪ್ರಸಾದರಿಗೆ ವಹಿಸಲಾಯಿತು. ಹೀಗೆ ಅಂದಿನ ಸಭೆ ಮುಗಿಸಿ ಗಂಗಾಧರ ಭಟ್ಟರಲ್ಲಿ ಇಷ್ಟೂ ವಿಷಯವನ್ನು ತಿಳಿಸಲು ಅನುವಾದೆವು.

ಆದರೆ ಗಂಗಾಧರ ಭಟ್ಟರನ್ನು ಹತ್ತಿರದಿಂದ ಬಲ್ಲ ನಮಗೆ ಒಂದು ವಿಷಯ ಸ್ಪಷ್ಟವಿತ್ತು. ಇಂದಿನ ಕಾಲದಲ್ಲಿ ಸರ್ವೇ ಸಾಮಾನ್ಯವಾಗಿರುವ ಅಭಿನಂದನ ಕಾರ್ಯಕ್ರಮವೆಂಬ ಶೋ \enginline{-} ವೃಥಾ ಪ್ರದರ್ಶನವನ್ನೆಲ್ಲ ಅವರು ಬಿಲ್ ಕುಲ್ ಒಪ್ಪುವವರಲ್ಲ. ಅಲ್ಲದೇ ಅಭಿವಂದನ ಕಾರ್ಯಕ್ರಮವೆಂದು ತನ್ನ ಬಡ ವಿದ್ಯಾರ್ಥಿಗಳು ಅಥವಾ ಮತ್ತಾರೂ ಧನವನ್ನು ವ್ಯಯಿಸುವುದು ಅವರಿಗೆ ಸರ್ವಥಾ ಸಮ್ಮತವಾಗದ ವಿಚಾರ. ಆದ್ದರಿಂದ ನಮ್ಮ ಯೋಜನೆಯನ್ನು ಬಡಪೆಟ್ಟಿಗೆ ಒಪ್ಪದ  ಅವರನ್ನು ನಮ್ಮ ಕಾರ್ಯಕ್ರಮಕ್ಕೆ ಒಪ್ಪಿಸುವುದು ಒಂದು ಸವಾಲಾಗಿ ಕಂಡಿತು. ಆದರೆ ಅವರಿಂದ ವಿದ್ಯೆಪಡೆದ ವಿದ್ಯಾರ್ಥಿಸಮೂಹ, ಮಾತ್ರವಲ್ಲ  ಅವರಿಂದ ಯಾವೆಲ್ಲ ಪ್ರಯೋಜನವನ್ನು ಸಮಾಜ ಪಡೆದಿದೆಯೋ ಅಂತಹ ಸಮಾಜ ಇಂತಹ ಮಹನೀಯರನ್ನು ಒದಗಿಬಂದ ಅವಸರದಲ್ಲೂ  ಗೌರವಿಸದಿರುವುದು ಕೃತಘ್ನತೆಯಲ್ಲದೇ ಮತ್ತೇನೂ ಅಲ್ಲ. ಕೃತಘ್ನೇ ನಾಸ್ತಿ ನಿಷ್ಕೃತಿಃ \enginline{-} ಕೃತಘ್ನತೆಗೆ ಮಾತ್ರ\break ಪ್ರಾಯಶ್ಚಿತ್ತವಿಲ್ಲ ಎಂಬುದು ಮರ್ಯಾದಾ ಪುರುಷೋತ್ತಮನಾದ ಶ್ರೀರಾಮನ ಮಾತು, ಅದು ನಮಗೆ ಆದರ್ಶ ! 

ನಾವೆಲ್ಲ ಅವರ ಮನೆಯನ್ನು ತಲುಪಿದೆವು. ಅಂದುಕೊಂಡಂತೆ ಮೊದಲಮಾತಿನಲ್ಲೇ ಅವರು ಈ ಎಲ್ಲವನ್ನೂ ನಿರಾಕರಿಸಿಬಿಟ್ಟರು. ಆದರೆ ಅವರು ಯಾವುದಕ್ಕಾಗಿ ಜೀವನವನ್ನು ಸಾಗಿಸಿದರೋ ಅಂತಹ ಶಾಸ್ತ್ರದ ಚರ್ಚಾರೂಪದ \enginline{-} \hbox{ವಿದ್ವದ್ಗೋಷ್ಠಿಯೆಂದು} ವ್ಯವಹರಿಸುವ ಗೋಷ್ಠಿಯನ್ನು ಮುಂದೆಮಾಡಿ ಅಭಿವಂದನವನ್ನು ಗೌಣಮಾಡಿ ಅವರಲ್ಲಿ ವ್ಯವಹರಿಸಿದೆವು. ಅವರಿಗೆ ಅದು ಒಪ್ಪಿತವಾಯಿತು. ಆದರೆ ಅವರ ಮಾತಿನಲ್ಲಿ ‘ಶರತ್ತು\-ಗಳು ಅನ್ವಯ’ ಎಂಬ ಧ್ವನಿಯಿತ್ತು. ಅದೇನು ? \enginline{-} “ಕಾರ್ಯಕ್ರಮ ಯಾವುದೇ ವೃಥಾ ವೈಭವ\-ವಿಲ್ಲದೇ, ಬಾಹ್ಯಪ್ರಸಿದ್ಧರನ್ನಾಗಲೀ, ರಾಜಕಾರಣಿಗಳನ್ನಾಗಲೀ ಆಹ್ವಾನಿಸದೇ ನೀವಿಷ್ಟೇ ವಿದ್ಯಾರ್ಥಿಗಳು ಸೇರಿ ಮಾಡುವುದಾದರೆ ಆಗಬಹುದು” ಎಂದರು. ಇದರೊಂದಿಗೆ ಇನ್ನೊಂದು ಬಹು ಮುಖ್ಯವಾದ ಆಶಯವನ್ನು ವ್ಯಕ್ತಪಡಿಸಿದರು, \enginline{-} “ಮೈಸೂರು ಮಹಾರಜರು ಪಾಠಶಾಲೆಯನ್ನು ಸ್ಥಾಪಿಸಿದ್ದರಿಂದ ನಾವೆಲ್ಲ ಇಲ್ಲಿ \hbox{ಕಲೆತು}, ಕಲಿತು ಜೀವನವನ್ನು ಸಾಗಿಸಲು ಆಶ್ರಯ ಒದಗಿದೆ, ಅವಕಾಶವಾಗಿದೆ. ಆದರೆ ಅಂಥವರನ್ನು ಒಮ್ಮೆಯೂ ನಮ್ಮಲ್ಲಿಗೆ ಆಮಂತ್ರಿಸಿ ನಾವಾರೂ ಕೃತಜ್ಞತೆಯನ್ನು ಸಲ್ಲಿಸಿಲ್ಲ, ಇದು ಕೃತಘ್ನತೆಯಲ್ಲದೇ ಮತ್ತೇನು ! ಒಂದುವೇಳೆ ಕಾರ್ಯಕ್ರಮಕ್ಕೆ ಅರಮನೆಯವರು ಆಗಮಿಸಲು ಸಾಧ್ಯವಾಗಿ ಅವರಿಗೆ ಒಂದು ಕೃತಜ್ಞತೆಯನ್ನು ಸಲ್ಲಿಸಲು ಅವಕಾಶವಾಗುವು\-ದಾದರೆ ಕಾರ್ಯಕ್ರಮ ನಡೆಯಲಿ, ನನ್ನ ಅಭ್ಯಂತರವಿಲ್ಲ” ಎಂದರು. 

ಫೆಬ್ರವರಿ ೧೧, ೨೦೧೮ \enginline{-} ಭಾನುವಾರ ಕಾರ್ಯಕ್ರಮ ನಿಗದಿಗೊಳಿಸಿದೆವು.

ಕಾರ್ಯಕ್ರಮಕ್ಕೆ ಸಿದ್ಧತೆಗಳೆಲ್ಲ ನಮ್ಮ ವಾಟ್ಸಾಪ್ ಗ್ರೂಪ್ ನ ಮೂಲಕವೇ ಸಾಗುತ್ತಿತ್ತು. ಈ ಮಧ್ಯೆ ನಮಗೆ ಅರಮನೆಯವರನ್ನು ಸಂಪರ್ಕಿಸಿ ರಾಜಮಾತೆಯವರನ್ನು ಆಹ್ವಾನಿಸಿ ಅವರು ದಯಮಾಡಿಸುವಂತಾಗಲು ಶ್ರೀಮಾನ್ ರಮೇಶ ಅಡಿಗರು ಮತ್ತು ಶ್ರೀಮಾನ್ ಆಳ್ವಾರ್ ರವರು ನಮ್ಮ ಸಹಕಾರಕ್ಕೆ ಬಂದರು. ತನ್ಮೂಲಕ ಅರಮನೆಯವರನ್ನು ಸಂಪರ್ಕಿಸಲಾಯಿತು. ರಾಜಮಾತೆ ಪ್ರಮೋದಾದೇವಿಯವರು ಕಾರ್ಯಕ್ರಮಕ್ಕೆ ದಯಮಾಡಿಸಲು ಸಂತೋಷದಿಂದ ಅನುಮತಿಸಿದರು, ಕಾರ್ಯಕ್ರಮ ಈ ಹಿಂದೆ\break ಸಂಕಲ್ಪಿಸಿದ ದಿನವೇ ದೃಢವಾಯಿತು. ೧೯೨೭ ರ ಅನಂತರ ಅರಮನೆಯಿಂದ ಮಹಾರಾಜರಾರೂ ಪಾಠಶಾಲೆಗೆ ಆಗಮಿಸಿರಲಿಲ್ಲ. ಈಗ ಈ ನಮ್ಮ ಕಾರ್ಯಕ್ರಮಕ್ಕೆ ರಾಜಮಾತೆಯವರು ಆಗಮಿಸುತ್ತಿದ್ದುದು ಒಂದು ಐತಿಹಾಸಿಕವಾಗಿ ಮಹತ್ತ್ವದ ಘಟನೆ\-ಯಾಗುತ್ತಿರುವ ಕಾರಣದಿಂದ ಕಾರ್ಯಕ್ರಮವನ್ನು ಅದರ ಘನತೆಗೆ ತಕ್ಕಂತೆ ನಿರ್ವಹಿಸಬೇಕೆಂದು ನಿಶ್ಚಯಿಸಿದೆವು. ಸಿದ್ಧತೆ ಸಾಗುತ್ತಿತ್ತು. ವಾಸ್ತವವಾಗಿ ಅದೇ ದಿನ ಅಭಿವಂದನ ಗ್ರಂಥವನ್ನೂ ಪ್ರಕಾಶಪಡಿಸಬೇಕೆಂದುಕೊಂಡಿದ್ದೆವು. ಲೇಖಕರಿಗೂ ಇದನ್ನೇ ತಿಳಿಸಿ\break ಗ್ರಂಥಕ್ಕಾಗಿ ಲೇಖನಗಳನ್ನು ವಿನಂತಿಸಿದ್ದೆವು. ಆದರೆ ವಿದ್ವಾಂಸರು ಸುಸಂಸ್ಕೃತ ಸಮಾಜ\-ಕ್ಕಾಗಿ ತಮ್ಮನ್ನು ಸಂಪೂರ್ಣ ಸಮರ್ಪಿಸಿಕೊಂಡಿರುವ ವಿಷಯ ಈ ಸಂದರ್ಭದಲ್ಲಿ ನಮಗೆ ಚೆನ್ನಾಗಿ ಅರಿವಿಗೆ ಬಂತು. ಅಂದರೆ ಯಾರೂ ಬಿಡುವಾಗಿರಲಿಲ್ಲ. ಹಾಗೆಂದು ಕಾರ್ಯಕ್ರಮಕ್ಕೆ ಮೊದಲು ಕೆಲವು ಲೇಖನಗಳೂ ಬಂದಿದ್ದವು. ಆದರೆ ಪ್ರಕಟವಾಗುವ ಗ್ರಂಥವನ್ನು ಒಂದು ಐತಿಹಾಸಿಕ ಘಟನೆಗೆ ಸಾಕ್ಷೀಕರಿಸುವಂತೆ ಮಾಡುವುದರಲ್ಲಿ  ಔಚಿತ್ಯವನ್ನು ಮನಗಂಡು ಕಾರ್ಯಕ್ರಮದಲ್ಲಿ ಸಾಂಕೇತಿಕವಾಗಿ ಪುಸ್ತಕವನ್ನು ಬಿಡುಗಡೆಗೊಳಿಸಿ ಮುಂದೆ ಈ ಎಲ್ಲ ಕಾರ್ಯಕ್ರಮದ ಸಂಪೂರ್ಣ ವರದಿಯನ್ನು ಚಿತ್ರಸಹಿತವಾಗಿ ದಾಖಲಿಸಿ ಪುಸ್ತಕ ಮುದ್ರಣಮಾಡುವುದೆಂದು ತೀರ್ಮಾನಿಸಿದೆವು. ಹಾಗಾಗಿ ಕಾರ್ಯಕ್ರಮದ ದಿನ ಬೆಳಿಗ್ಗೆ ವಿದ್ವದ್ಗೋಷ್ಠಿ, ಮಧ್ಯಾಹ್ನ ಅಭಿವಂದನ ಸಮಾರಂಭ ನಿಗದಿಯಾಯಿತು. ವಿದ್ವದ್ಗೋಷ್ಠಿಗಾಗಿ ವಿದ್ವಾನ್ ಉಮಾಕಾಂತ ಭಟ್ಟರು ಮತ್ತು ವಿದ್ವಾನ್ ಆಳ್ವಾರ್ ರವರು ಮಾರ್ಗದರ್ಶನ ಮಾಡಿದರು. “ತರ್ಕಸ್ವರೂಪ” ಎಂಬ ವಿಷಯದಲ್ಲಿ ಗೋಷ್ಠಿಯೆಂದು ನಿರ್ಣಯವಾಯಿತು. ಈ ಗೋಷ್ಠಿಗೆ ಧಾರವಾಡದ ಮಹಾನ್ ವಿದ್ವಾಂಸರಾದ ವೇ|ಬ್ರ|ಶ್ರೀ ರಾಜೇಶ್ವರ ಶಾಸ್ತ್ರಿಗಳನ್ನು ಅಧ್ಯಕ್ಷರನ್ನಾಗಿ ಆಮಂತ್ರಿಸಿದೆವು. ಅವರು ನಮ್ಮ ಆಹ್ವಾನವನ್ನು ಸ್ವೀಕರಿಸಿದರು. ಅಂತೆಯೇ ತುಮಕೂರಿನ ಸಂಸ್ಕೃತ ವಿದ್ವಾಂಸರಾದ ಶ್ರೀಯುತ ಶ್ರೀಧರ ಶಾಸ್ತ್ರಿಗಳನ್ನು,  ಶ್ರೀಯುತ ಉಮಾಕಾಂತ ಭಟ್ಟರನ್ನು, ಶ್ರೀಯುತ ಆಳ್ವಾರ್ ರವವರನ್ನೂ ಗೋಷ್ಠಿಯನ್ನು ನಡೆಸಿಕೊಡುವಂತೆ ಪ್ರಾರ್ಥಿಸಿದೆವು. 
\vskip 8pt

ಇನ್ನೊಂದು ಮಹತ್ತ್ವದ ಅಂಶವೆಂದರೆ ಮಧ್ಯಾಹ್ನದ ಕಾರ್ಯಕ್ರಮಕ್ಕೆ ಗಂಗಾಧರ ಭಟ್ಟರಿಗೆ ಆತ್ಮೀಯರಾದ ಶ್ರೇಷ್ಠ ವಿದ್ಮಾಂಸರನ್ನು ಆಮಂತ್ರಿಸಬೇಕೆಂದು ಅದಕ್ಕೆ ರಾಷ್ಟ್ರಿಯ ಸಂಸ್ಕೃತ ಸಂಸ್ಥಾನದ ಕುಲಪತಿಗಳಾದ ಡಾ । ಪಿ.ಎನ್.ಶಾಸ್ತ್ರಿಗಳೇ ಒಪ್ಪುವ ವ್ಯಕ್ತಿಯೆಂದು ನಿರ್ಣಯಿಸಿ ಅವರನ್ನು ವಿನಂತಿಸಿದೆವು. ಗಂಗಾಧರ ಭಟ್ಟರ ಮೇಲಿನ ಅಭಿಮಾನದಿಂದ ಅವರೂ ಸಂತೋಷದಿಂದ ಸಮ್ಮತಿಸಿದರು. 

ಆಮಂತ್ರಣ ಪತ್ರಿಕೆ ಸಿದ್ಧವಾಯಿತು. ದಿನಪತ್ರಿಕೆಗಳಿಗೆ ಲೇಖನಸಹಿತವಾಗಿ ಮಾಹಿತಿ ತಲುಪಿಸಿದೆವು. ಸುಮಾರು ಎಲ್ಲ ಪತ್ರಿಕೆಗಳಲ್ಲಿಯೂ ವಿಷಯ ಪ್ರಕಟವಾಯಿತು. ಆ ಮೂಲಕವೂ ಜನರನ್ನು ಆಹ್ವಾನಿಸಿದೆವು. ಜನರು ಸ್ವಾಗತಿಸಿದರು, ಸಮಾಜ \hbox{ಸ್ವಾಗತಿಸಿತು}. ಭಟ್ಟರ ವಿದ್ಯಾರ್ಥಿಸಮೂಹವೆಲ್ಲ ಸಂಭ್ರಮಿಸಿತು. ಎಲ್ಲ ಕಡೆಗೂ ಆಮಂತ್ರಣ ಪತ್ರಿಕೆ ಸುಲಭವಾಗಿ ತಲುಪಿತು.  

ಇಷ್ಟಾಗುವ ಹೊತ್ತಿಗೆ ಕಾರ್ಯಕ್ರಮದ ಆ ದಿನ ಬಂದೇಬಿಟ್ಟಿತು. ಹಿಂದಿನ ದಿನ ಬಹುತೇಕ ಎಲ್ಲ ವಿದ್ಯಾರ್ಥಿಗಳು ಮೈಸೂರಿನಲ್ಲಿ ಸೇರಿದರು. ಮಾರನೆಯ ದಿನದ ಕಾರ್ಯಕ್ರಮಕ್ಕೆ ಭರದ ಸಿದ್ಧತೆಯಾಯಿತು. ರಾಜ್ಯದ ವಿವಿಧ ಭಾಗಗಳಿಂದ ಭಟ್ಟರ ಅಭಿಮಾನಿಗಳು ಆಗಮಿಸುವ ನಿರೀಕ್ಷೆಯಿತ್ತು. ಅವರಿಗೆಲ್ಲ ವಸತಿ ಭೋಜನಾದಿ ವ್ಯವಸ್ಥೆ\-ಗಳೆನ್ನೆಲ್ಲ ಸಿದ್ಧಗೊಳಿಸಲಾಗಿತ್ತು. ಆಗಮಿಸುವ ಮಹಾರಾಜ ಸಂಸ್ಕೃತ ಮಾಹಾಪಾಠ\-ಶಾಲೆಯ ಮಧ್ಯದ ಸಭಾಗೃಹ ನವವಧುವಿನಂತೆ ಕಂಗೊಳಿಸಿತು. ನಿಗದಿಯಂತೆ ಬೆಳಿಗ್ಗೆ ೯.೩೦ ಕ್ಕೆ ದೀಪ ಬೆಳಗಬೇಕು, ವಿದ್ಯಾಗಣಪತಿಗೆ ಕಲ್ಪೋಕ್ತಪೂಜೆ ಸಂಪನ್ನವಾಗಬೇಕು. ಗೋಷ್ಠಿಯ ಉದ್ಘಾಟನೆಯಾಗಬೇಕು. ಗೋಷ್ಠಿ ಸಂಪನ್ನವಾಗಬೇಕು.

ದಿನಾಂಕ ೧೧.೨.೨೦೧೮, ಸೂರ್ಯೋದಯವಾಯಿತು. ಮಹಾರಾಜ ಸಂಸ್ಕೃತ ಮಹಾಪಾಠಶಾಲೆ ರಂಗವಲ್ಲೀ ತಳಿರುತೋರಣಗಳಿಂದ ಅಲಂಕೃತವಾಗಿತ್ತು. ನಿಗದಿತ ಸಮಯಕ್ಕೆ ಅಭ್ಯಾಗತರು, ಗೋಷ್ಠಿಯನ್ನು ನಡೆಸಿಕೊಡುವ ವಿದ್ವಾಂಸರೆಲ್ಲ ಧಯಮಾಡಿಸಿದ್ದರು. ಮೈಸೂರಿನ ನಾಗರಿಕರು ನೆರೆದಿದ್ದರು. ದೇಶ, ವಿದೇಶದ ವಿದ್ಯಾರ್ಥಿಗಳು ಹಾಜರಾಗಿದ್ದರು. ಗಂಗಾಧರ ಭಟ್ಟರ ಊರಿನಿಂದ ಕುಟುಂಬವರ್ಗದವರೆಲ್ಲ ಆಗಮಿಸಿದ್ದರು. 

ಇಂತಹ ಸನ್ನಿವೇಶದಲ್ಲಿ ಶ್ರೀಯುತ ಗಂಗಾಧರ ಭಟ್ಟರನ್ನು ಸ್ವಾಗತಿಸಲು ಪಾಠ\-ಶಾಲೆಯ ಮಹಾದ್ವಾರದಲ್ಲಿ  ಎಲ್ಲರೂ ಸೇರಿದೆವು. ನಿಗದಿತ ಸಮಯಕ್ಕೆ ಸರಿಯಾಗಿ ಧರ್ಮಪತ್ನಿ ಶ್ರೀಮತಿ ಶೈಲಜಾ ಮತ್ತು ಕುಟುಂಬಪರಿವಾರದೊಡನೆ ಪರಿವೃತರಾದ ಶ್ರೀಮಾನ್ ಗಂಗಾಧರ ಬಟ್ಟರು ಪಾಠಶಾಲೆಯ ಮಹಾದ್ವಾರದ ಬಳಿ ದಯಮಾಡಿಸಿದರು. ವಿದ್ಯಾರ್ಥಿಗಳು ಅಭಿಮಾನಿಗಳು ಪಾಠಶಾಲೆಯ ಅಧ್ಯಾಪಕರು ನಾಗರಿಕರು ಎಲ್ಲರೂ ಸೇರಿ ಅವರೆಲ್ಲರನ್ನು ಪೂರ್ಣಕುಂಭದೊಡನೆ ಬರಮಾಡಿಕೊಂಡೆವು. ಮಹಾದ್ವಾರದಿಂದ ಪಾಠಶಾಲೆಯ ಸಭಾಂಗಣದ ವರೆಗೂ ವೇದಘೋಷದೊಡನೆ ಅವರನ್ನು ಕರೆತರಲಾಯಿತು. ಸಭಾಂಗಣದಲ್ಲಿ ವಿದ್ಯಾಗಣಪತಿಯ ಸಾನ್ನಿಧ್ಯಕ್ಕೆ ಬರುವಷ್ಟರಲ್ಲಿ ವೇದಘೋಷ ಮುಗಿಲಮುಟ್ಟಿತು. ಅರ್ಚಕರು  ಮಹಾಗಣಪತಿಗೆ ಆರಾರ್ತಿಯನ್ನು\break ಬೆಳಗಿದರು. ಗಂಗಾಧರ ಭಟ್ಟ ದಂಪತಿಗಳು ಆರತಿ, ಪ್ರಸಾದ, ಅರ್ಚಕರ ಆಶೀರ್ವಾದ ಪಡೆದರು.

ಇಲ್ಲಿಂದ ಮುಂದೆ ಸಭಾಗಮನ. ಸಭೆ ಗೋಷ್ಠಿಗೆ ಅನುಕೂಲವಾಗಿ ಸಿದ್ಧವಾಗಿತ್ತು. ವಿಶೇಷವಾಗಿ ಸಭೆಯಲ್ಲಿ ಗಂಗಾಧರ ಭಟ್ಟರ ವಿದ್ಯಾಗುರುಗಳಾದ ಮಹಾಮಹೋಪಾ\-ಧ್ಯಾಯ ಎನ್.ಎಸ್.ರಾಮಭದ್ರಾಚಾರ್ಯ ದಂಪತಿಗಳ ಭಾವಚಿತ್ರವನ್ನು ಅಲಂಕರಿಸಲಾಗಿತ್ತು.  ಕಾರ್ಯಕ್ರಮದ ಉದ್ಘಾಟನೆಗಾಗಿ ದೀಪಬೆಳಗುವ ಸಂದರ್ಭ. ಗೋಷ್ಠಿಯ ಅಧ್ಯಕ್ಷರಾದ ವಿದ್ವಾನ್ ರಾಜೇಶ್ವರ ಶಾಸ್ತ್ರಿಗಳು, ವಿದ್ವಾನ್ ಶ್ರೀಧರಶಾಸ್ತ್ರಿಗಳು, ವಿದ್ವಾನ್ ಉಮಾಕಾಂತ ಭಟ್ಟರು ಇವರೊಡಗೂಡಿ ಶ್ರೀಯುತ ಗಂಗಾಧರ ಭಟ್ಟರು ದೀಪಬೆಳಗಿ\-ಸಿದರು. ಮುಂದೆ ಗೋಷ್ಠಿ. “ತರ್ಕಸ್ವರೂಪ” ಎಂಬದು ಗೋಷ್ಠಿಯ ವಿಷಯ.

ಇದೇನು ಎಂಬ ಪ್ರಶ್ನೆ ಅನೇಕರಿಗೆ ಸಹಜ. ಅದಕ್ಕಾಗಿ  ಎರಡು ಮಾತುಗಳು \enginline{-}

ನ್ಯಾಯಶಾಸ್ತ್ರ ಎಂಬುದು ಪ್ರಸಿದ್ಧವಾಗಿದೆ. ಗೌತಮಮಹರ್ಷಿಗಳು ಈ ಶಾಸ್ತ್ರವನ್ನು ಅನ್ವೇಶಿಸಿ ಲೋಕದಲ್ಲಿ ಪ್ರಕಾಶಪಡಿಸಿದರು. ಈ ಶಾಸ್ತ್ರ “ಪ್ರಮೇಯ” ಪ್ರಧಾನವಾಗಿದೆ \enginline{-} (ಪ್ರಮೇಯ ಎಂದರೆ ಪ್ರಮಾಣಗಳಿಂದ ತಿಳಿಯಬಹುದಾದ ಸಮಸ್ತ ವಿಷಯಗಳು.) ಈ  ಶಾಸ್ತ್ರಕ್ಕೆ ಸಂವಾದಿಯಾದ ಶಾಸ್ತ್ರ ವೈಶೇಷಿಕಶಾಸ್ತ್ರ. ಇದನ್ನು ಕಣಾದ ಮಹರ್ಷಿಗಳು ಪ್ರಚುರಪಡಿಸಿದರು. ಮುಂದೆ ಮುಂದೆ ಈ ಎರಡೂ ಶಾಸ್ತ್ರಗಳು ಪರಸ್ಪರ ಸಾಮ್ಯಾಂಶದ ಕಾರಣದಿಂದ, ಗಂಗೆ ಮತ್ತು ಯಮುನೆ ಪ್ರಯಾಗದಲ್ಲಿ ಸೇರಿ ಒಂದಾಗಿ ಹರಿಯುವಂತೆ  ಈ ಶಾಸ್ತ್ರಗಳು ತರ್ಕಶಾಸ್ತ್ರ ಎಂಬ ಒಂದೇ ಹೆಸರಿನಿಂದ ಒಂದಾಗಿ ಹರಿದಿರುವುದೂ ಉಂಟು. ಇದರದೇ ಇನ್ನೊಂದು ಕವಲಾಗಿ ಮುಂದೆ ನವೀನ ನ್ಯಾಯಶಾಸ್ತ್ರ ಎಂಬುದು ಗಂಗೇಶೋಪಾಧ್ಯಾಯ ಎಂಬುವರಿಂದ ಹರಿದು ಗೌತಮರು ಹೇಳಿದ “ಪ್ರಮಾಣ” ಎಂಬ ವಿಷಯದ ಸ್ವರೂಪವನ್ನೇ ಪ್ರಧಾನವಾಗಿ ಚಿಂತಿಸುತ್ತಾ ಅದನ್ನೇ ಪರಿಷ್ಕರಿಸಿ\break ಪರಿಷ್ಕರಿಸಿ ನಿಷ್ಕರ್ಷಿಸಲು ಪ್ರವೃತ್ತವಾಗಿದೆ. ನಾವು ಒಂದು ವಿಷಯವನ್ನು ಅಳೆಯ\-ಬೇಕಾದರೆ ಅಳತೆಗೋಲು ಸರಿಯಿದ್ದರೆ ತಾನೆ ಅಳೆದಿದ್ದು ಸರಿಯಾಗುವುದು. ಕಣ್ಣು ಇತ್ಯಾದಿ ಇಂದ್ರಿಯಗಳು ಮೊದಲು ಸರಿಯಿದ್ದರೆ ತಾನೆ ನೋಡಬೇಕಿರುವ ವಿಷಯ ಸರಿಯಾಗಿ ತೋರುವುದು. ಆದ್ದರಿಂದ ಒಂದು ತಿಳುವಳಿಕೆಗೆ ಅಗತ್ಯವಿರುವ ಸಾಧನ = ಪ್ರಮಾಣ ನಿರ್ದೋಷವಾಗಿದ್ದರೆ ಮಾತ್ರ ಅದನ್ನು ಅವಲಂಬಿಸಿ ಲಭಿಸುವ ತಿಳುವಳಿಕೆ ಶುದ್ಧವಾಗಿವುದು ಸಾಧ್ಯ. ಆದ್ದರಿಂದ ಪ್ರಮಾಣ = ಅಳತೆಗೋಲು ಅದು ಅತ್ಯಂತ\break ನಿರ್ದೋಷವಾಗಿರಬೇಕೆಂಬ ಸಂದೇಶವನ್ನು ಆ ಶಾಸ್ತ್ರ ಸಾರುತ್ತದೆ. ನಿರ್ದೋಷ ಎಂದಾಗ ದೋಷದ ಬಗ್ಗೆ ತಿಳಿಯಬೇಕಲ್ಲ. ಆ ಚಿಂತನೆಯಂತೂ ಅಲ್ಲಿ ವಿಪುಲವಾಗಿದೆ. ಇವೆಲ್ಲ ಏನೇ ಇದ್ದರೂ ಪ್ರತಿಯೊಂದು ಶಾಸ್ತ್ರಗಳಿಗೆ ಧರ್ಮ, ಅರ್ಥ, ಕಾಮ, ಮೋಕ್ಷಗಳೇ ಲಕ್ಷ್ಯ ವಿಷಯಗಳು ಎಂಬುದನ್ನು ಮರೆಯುವಂತಿಲ್ಲ. ಆದರೆ ಆಯಾ ಶಾಸ್ತ್ರಗಳು ಅವುಗಳ ಬಗೆಗಿನ ಚಿಂತನೆಗಳನ್ನು ಮಾತ್ರ ವಿವಿಧವಾದ ಮಜಲುಗಳಿಂದ ತಮ್ಮದೇ ಆದ ಚೌಕಟ್ಟನ್ನು ಹಾಕಿಕೊಂಡು ಆ ಪರಿಧಿಯಲ್ಲೇ ವಿಷಯವನ್ನು ನಿಷ್ಕರ್ಷಿಸುತ್ತ ಹೋಗುತ್ತವೆ. ಆದರೂ ಪ್ರತಿಯೊಂದು ಶಾಸ್ತ್ರವೂ ತಮ್ಮದೇ ಆದ ವಿಶಿಷ್ಟವಾದ ವೈಶಿಷ್ಟ್ಯವನ್ನು ಹೊಂದಿರುವುದೊಂದು ಅಧ್ಭುತ. ನವೀನತರ್ಕಶಾಸ್ತ್ರ ಅಥವಾ ನವೀನನ್ಯಾಯಶಾಸ್ತ್ರವೂ ಹಾಗೆಯೇ. ಈ ಶಾಸ್ತ್ರದ ವೈಶಿಷ್ಟ್ಯಗಳಲ್ಲಿ ವಿಶೇಷವಾಗಿ ಉಲ್ಲೇಖಿಸುವಂಥದೇನೆಂದರೆ ಅದು ಎಲ್ಲ ಶಾಸ್ತ್ರಗಳಿಗೆ ವಿಷಯಗಳನ್ನು ಹಿಂಜಿ ಹಿಂಜಿ ನಿಷ್ಕರ್ಷೆ ಮಾಡಲು ಬೇಕಾದ ಒಂದು ಬಗೆಯ ಚಿಂತನಾಶೈಲಿ\enginline{-}ವಾದ ಸರಣಿಯ ಹಾದಿಯನ್ನು ಆವಿಷ್ಕರಿಸಿಕೊಟ್ಟಿದೆ. ಯಾವ ಶಾಸ್ತ್ರಗಳೂ ಅದನ್ನು ಅವಲಂಬಿಸಿಯೇ ಬೆಳೆದಿವೆ. 

ತರ್ಕಸ್ವರೂಪ ಎಂಬ ವಿಷಯದಲ್ಲಿ ಗೋಷ್ಠಿ ಎಂದಾಗ ಸಾಮಾನ್ಯರಿಗೆ ಇದರಲ್ಲೇನು ಗೋಷ್ಠಿ ಮಾಡುವುದಿದೆ ಎನ್ನಿಸದಿರದು. ಅಥವಾ ವಿಷಯವನ್ನು ಶಾಸ್ತ್ರದಲ್ಲಿ\break ನಿರ್ಣಯಿಸಿಟ್ಟಮೇಲೆ ಮತ್ತೆ ಆ ವಿಷಯದಲ್ಲಿ ವಾದಮಾಡುವುದೇನಿದೆ ಎಂಬ ಪ್ರಶ್ನೆಯೇಳಬಹುದು. ಏಕೆಂದರೆ  “ತರ್ಕ” ಎಂಬ ಪದ ಲೋಕದಲ್ಲಿ, “ಕರ್ಕಷವಾದ ವಾದ \enginline{-} ಎದುರಾಳಿಯನ್ನು ಬುದ್ಧಿವಂತಿಕೆಯಿಂದಲೋ ಮೊಂಡು ಮಾತುಗಳಿಂದಲೋ ಕಟ್ಟಿಹಾಕಿ  ಭಯವುಂಟುಮಾಡುವ ಮಾತಿನ ಸರಣಿ” ಎಂಬ ಅರ್ಥದಲ್ಲಿ ಬಳಕೆಯಲ್ಲಿದೆ. ತರ್ಕಶಾಸ್ತ್ರ ಓದಿದವನು ಎಂದರೆ ಅವನಲ್ಲಿ ಮಾತಾಡಲು ಹಿಂಜರಿಯುವ ಜನರೂ ಇದ್ದಾರೆ. ಆದರೆ ತರ್ಕಶಾಸ್ತ್ರ “ತರ್ಕ” ಎಂಬ ಶಬ್ದವನ್ನು ಈ ಅರ್ಥದಲ್ಲಿ ಬಳಸಲಿಲ್ಲ. ಅದರ ಸ್ವರೂಪ ಒಂದು ದೃಷ್ಟಿಯಿಂದ ಆ ಶಾಸ್ತ್ರದಲ್ಲಿ ನಿಷ್ಕರ್ಷೆಯಾಗಿರುವುದೂ ನಿಜ. ಆದರೂ ಆ ವಿಷಯವನ್ನು ಸೂಕ್ಷ್ಮವಾಗಿ ಚಿಂತಿಸಹೊರಟಾಗ ವಿದ್ವಾಂಸರ ಮಟ್ಟದಲ್ಲಿ ಚಿಂತನೀಯ ವಿಷಯಗಳು ಬೇಕಾದಷ್ಟು ಇರುತ್ತವೆ. ಅಲ್ಲದೇ ಅನೇಕರಿಗೆ ಅದರ ಸ್ವರೂಪ ಸ್ಪಷ್ಟವಾಗ\-ದಿರುವುದೂ, ಭ್ರಾಮಕವಾದ ವಿಷಯಗಳೂ ಇರಬಹುದು. ಹಾಗಾಗಿ ಅದರ ಬಗ್ಗೆ ಒಂದು ಚಿಂತನೆಯನ್ನು, ಬಲ್ಲ ವಿದ್ವಾಂಸರು ನಡೆಸಿಕೊಟ್ಟರೆ ಜಿಜ್ಞಾಸುಗಳಿಗೆ ಅದೊಂದು ಸುಗ್ರಾಸವಾಗಬಹುದು. ಇಷ್ಟಲ್ಲದೇ ಕಾವ್ಯಶಾಸ್ತ್ರವಿನೋದೇನ ಕಾಲೋ ಗಚ್ಛತಿ ಧೀಮತಾಮ್ ಎಂಬಂತೆ ಕಾವ್ಯರಚನೆ, ಶಾಸ್ತ್ರಚರ್ಚೆಗಳೆಲ್ಲ ಆ ಕ್ಷೇತ್ರದಲ್ಲಿರುವವರಿಗೆ ರಸಗವಳವಿದ್ದಂತೆ. ಎಷ್ಟೇ ಊಟ, ತಿಂಡಿ, ಪಾನೀಯಗಳನ್ನು ಸ್ವೀಕರಿಸಿದರು ಅದೆಲ್ಲ ಆದಮೇಲೆ ಹೇಗೆ ಎಲೆ\enginline{-}ಅಡಿಕೆ ಕವಳ ಬೇಕೊ ಹಾಗೆ. ಈ ಎಲ್ಲ ದೃಷ್ಟಿಯಿಂದ ಶಾಸ್ತ್ರಗೋಷ್ಠಿ ಅರ್ಥಪೂರ್ಣ.  

ಇನ್ನು ತರ್ಕ\enginline{-}ಸ್ವರೂಪ ಎಂದಾಗ ತರ್ಕ ಎಂದರೆ ಏನು ? ಇಂತಹ ಶಾಸ್ತ್ರ ವಿಷಯ\break ಸಾಮಾನ್ಯರಿಗೆ ಅರ್ಥವಾಗದ್ದೆಂದು ತಿಳಿಯಬೇಕಿಲ್ಲ. ಅದು ಸಹಜವಾಗಿ ಸೃಷ್ಟಿಯಲ್ಲಿರುವ ಜೀವಿಗಳಲ್ಲೆಲ್ಲ ಇರುವಂಥದ್ದೇ ಆಗಿದೆ. ಮನುಷ್ಯರಿಗೆ ಮಾತ್ರವೂ ಅಲ್ಲ, ಮನುಷ್ಯರಿ\-ಗಿಂತ ನಿರ್ದುಷ್ಟವಾಗಿ ಪ್ರಾಣಿ, ಪಕ್ಷಿ ಮರಗಿಡಗಳಲ್ಲೂ ಇದು ಸಹಜವಾಗಿಯೇ ಇರುತ್ತದೆ.  ಆಗ ತಾನೆ ಹುಟ್ಟುವ ಮಗುವಿನ ಸಂಸ್ಕಾರದಲ್ಲೇ ಈ ತಿಳುವಳಿಕೆ ಇರುತ್ತದೆ. ಇನ್ನೂ ವ್ಯಕ್ತಿಗಳ ಗುರುತು ಹಿಡಯದ ಶಿಷು ತಾಯಿ ಎತ್ತಿಕೊಂಡರೆ ಅಳು ನಿಲ್ಲಿಸಿ\break ಆನಂದಿಸುತ್ತದೆ. ಅದೇ ಬೇರೆಯವರು ಎತ್ತಿಕೊಂಡರೆ ಅಳುತ್ತದೆ. ಇನ್ನೂ ಗುರುತು ಹಿಡಿಯ\-ಲಾಗದ ಮಗು\-ವಿಗೆ ತನ್ನನ್ನು ಎತ್ತಿಕೊಂಡಿದ್ದು ತಾಯಿಯೇ ಎಂದು ತಿಳಿಯುವುದಾದರು ಹೇಗೆ? ಅದಕ್ಕೆ ಗೊತ್ತು ತನ್ನ ಸುಖಕ್ಕೆ ತಾಯಿಯೇ ಕಾರಣ ಎಂದು. ಹೇಗೆ ಅದನ್ನು ನಿರ್ಣಯಿಸಿ\-ಕೊಂಡಿತು? ಒಂಭತ್ತು ತಿಂಗಳು ಇದ್ದು ಬೆಳೆದದ್ದು ತಾಯಿಯ\break ಹೊಟ್ಟೆಯಲ್ಲಿಯೇ ಅಲ್ಲವೇ! ಹಾಗಾಗಿ ಅದು ತಾಯಿಯ ಸ್ಪರ್ಷ ಮತ್ತು ಸುಖ ಎರಡನ್ನೂ ಬಲ್ಲದು, ಹಾಗಾಗಿಯೇ ತಾಯಿಯ ಸ್ಪರ್ಷವನ್ನು ಪತ್ತೆಹಚ್ಚಲೂ ಬಲ್ಲದು. ಅದರಿಂದ ತನ್ನನ್ನು ಎತ್ತಿಕೊಂಡವಳು ತಾಯಿ ಎಂದು ನಿರ್ಣಯಿಸಿಕೊಳ್ಳುವುದು ಸಾಧ್ಯವಾಗುತ್ತದೆ. ಒಂದುವೇಳೆ ತಾಯಿ ಎತ್ತಿಕೊಳ್ಳದಿದ್ದರೆ ನನಗೆ ಸುಖವೂ ಆಗುತ್ತಿರಲಿಲ್ಲ. ಆದರೆ ನನಗೆ\break ಸುಖವಾಗುತ್ತಿದೆ. ಹಾಗಾಗಿ ತಾಯಿಯೇ ಎತ್ತಿಕೊಂಡಿದ್ದಾಳೆ ಎಂದುಕೊಳ್ಳುತ್ತದೆ.\break ಇಲ್ಲಿಯೇ ಆ ತರ್ಕವಿರುವುದು. ಆ ಅಮಗುವಿನ ಸುಖಕ್ಕೆ ತಾಯಿಯೇ ಕಾರಣ. ಶಿಶುವಿಗೆ ತನ್ನನ್ನು ತಾಯಿ ಎತ್ತಿಕೊಳ್ಳದಿದ್ದರೆ ಇಂತಹ ಸುಖ ಸಿಗುತ್ತಿರಲಿಲ್ಲ. ನನಗೆ ಸುಖ\break ಉಂಟಾಗುತ್ತಿದೆ. ಹಾಗಾಗಿ ತಾಯಿಯೇ ಎತ್ತಿಕೊಂಡಿದ್ದಾಳೆ ಎಂದು ನಿರ್ಣಯಿಸಿಕೊಳ್ಳುತ್ತದೆ. ಅಳು ನಿಲ್ಲಿಸುತ್ತದೆ. ಈ ತರ್ಕ ಈಗ ಹುಟ್ಟಿದ ಮಗುವಿನಲ್ಲಿಯೇ ಇರುವಂಥದ್ದು. ಅದಕ್ಕೆ ವಿದ್ವಾಂಸರ ತರ್ಕ ಗೊತ್ತಿಲ. ಹಾಗಾದರೆ ? ಅದು ಸೃಷ್ಟಿಸಹಜ. 

ಎರಡು ವಿಷಯಗಳ ಅಥವಾ ವ್ಯಕ್ತಿಗಳ ಸಂಬಂಧದ ವಿಷಯದಲ್ಲಿ ಸಂಶಯವೇ ಇಲ್ಲದಂತೆ ತಿಳುವಳಿಕೆ \enginline{-} ಜ್ಞಾನ ಬರುವುದಾದರೆ ಅಲ್ಲೆಲ್ಲ ಈ “ತರ್ಕ”ದ ಪಾತ್ರವಿದ್ದೇ ಇದೆ. ‘ಆದರೆ ಇದೇ ವಿಷಯ ಶಾಸ್ತ್ರದ ಚೌಕಟ್ಟಿಗೊಳಪಟ್ಟಾಗ ಒಂದು ತಾತ್ತ್ವಿಕ ವಿಷಯವನ್ನು ನಿರ್ಣಯಿಸಲು ಬಳಕೆಯಾಗುವಾಗ ಅದು ಸಾಕಷ್ಟು ಸೂಕ್ಷ್ಮತೆಯನ್ನು ಪಡೆದುಕೊಂಡು  ಚಿಂತಿಸಬೇಕಾದ ಅನೇಕ ವಿಷಯಗಳನ್ನು ಒಳಗೊಳ್ಳುತ್ತದೆ. ಆ ದೃಷ್ಟಿಯಿಂದ ಗೊಷ್ಠಿ ಅಗತ್ಯ. 

ವ್ಯಭಿಚಾರಶಂಕಾನಿವರ್ತಕಃ ತರ್ಕಃ ಎಂಬುದು ಈ ಸಂಬಂಧವಾದ ಶಾಸ್ತ್ರ ವಾಕ್ಯ, ಅದರ ಸ್ಥೂಲ ಅಭಿಪ್ರಾಯ ಹೀಗೆ \enginline{-} ಯಾವದೇ ದೃಷ್ಟಿಯಿಂದ ನೋಡಿದರೂ ನಿರ್ಣಯಿ\-ಸಿದ ವಿಷಯದಲ್ಲಿ ದೊಷವೇ ಇಲ್ಲದಂತೆ ನಿರ್ಣಯಿಸಲು ಅನುಕೂಲವಾಗುವಂತಹ ಒಂದು ಬಗೆಯ ವಾದ ಸರಣಿ \enginline{-} ತರ್ಕ. ನಿರ್ಣಯವಾದ ಮೇಲೆ ಆ ನಿರ್ಣಯದಲ್ಲಿ ವಿರೋಧವಿರುವವರು ಯಾವುದೇ ರೀತಿಯ ದೋಷವನ್ನು ಎತ್ತಲಾಗದಂತೆ, ನಿರ್ಣಯಿಸಿದ ವಿಷಯದಲ್ಲಿ ದಾರ್ಢ್ಯವನ್ನುಂಟುಮಾಡಿಕೊಳ್ಳಲು ಅನುಕೂಲಿಸುವುದು ತರ್ಕ.

ಹೀಗೆ ಈ ವಿಷಯದಲ್ಲಿ ಗೋಷ್ಠಿ ಅಧ್ಭುತವಾಗಿ ಸಂಪನ್ನವಾಯಿತು. ಇದರಲ್ಲಿ\break ಶ್ರೀಯುತರಾಜೇಶ್ವರ ಶಾಸ್ತ್ರಿಗಳು ಸಭಾಧ್ಯಕ್ಷರಾಗಿದ್ದು ಶ್ರೀಮಾನ್ ಗಂಗಾಧರ ಭಟ್ಟರು ವಿರಾಜಿಸಿದ್ದರು. ಗೋಷ್ಠಿಯನ್ನು ನಡೆಸಿಕೊಡಲು ವಿ~। ಶ್ರೀಧರ ಶಾಸ್ತ್ರಿಗಳು,\break ವಿ । ಉಮಾಕಾಂತ ಭಟ್ಟರು, ವಿದ್ವಾನ್ ಎಮ್.ಎ.ಆಳ್ವಾರವರು, ವಿದ್ವಾನ್ ವಾಚಸ್ಪತಿ\break ಶಾಸ್ತ್ರಿಯವರು ಉಪಸ್ಥಿತರಿದ್ದರು. 

ಗೊಷ್ಠಿಯ ಆರಂಭದಲ್ಲಿ ಗಂಗಾಧರ ಭಟ್ಟರ ವಿದ್ಯಾರ್ಥಿಗಳಾದ ವಿ । ರಾಘವ ಕೆ.ಎಲ್, ಮತ್ತು ವಿ । ಶಿವರಾಮ ಭಟ್ಟ ಎರಡು ವಿಷಯವನ್ನು ಪ್ರಸ್ತುತಪಡಿಸಿದರು. ಅನಂತರ ತರ್ಕಸ್ವರೂಪನಿರ್ಣಯಕ್ಕೆ  ಮೂಲವಾದ (“ಅವಿಜ್ಞಾತತತ್ತ್ವೇ ಅರ್ಥೇ\break ಕಾರಣೋಪಪತ್ತಿತಃ ತತ್ತ್ವಜ್ಞಾನಾರ್ಥಮೂಹಃ ತರ್ಕಃ” ಎಂಬ) ಗೌತಮಸೂತ್ರದ\break ವ್ಯಾಖ್ಯಾನವನ್ನು ಶ್ರೀಯುತ ಉಮಾಕಾಂತ ಭಟ್ಟರು ವಿವರಿಸಿದರು. ತರ್ಕಸ್ಯ\break ಪ್ರಮಾಣಾನಾಮನುಗ್ರಾಹಕತ್ವಮ್ \enginline{-} ಪ್ರಮಾಣದಿಂದ ವಿಷಯವನ್ನು ನಿರ್ಣಯಿಸುವ ಸಂದರ್ಭದಲ್ಲಿ ವಿಷಯದ  ನಿರ್ಣಯಕ್ಕೆ ಅವಲಂಬಿಸುವ ಪ್ರಮಾಣ ಇನ್ನೂ ದೃಢ\-ವಾಗಲು ಅನುಕೂಲಿಸುವುದು ತರ್ಕ. ಎಂಬ ವಿಷಯವನ್ನು ಶ್ರೀಮಾನ್ ಆಳ್ವಾರ್ ರವರು ಮಂಡಿಸಿದರು. “ತರ್ಕಲಕ್ಷಣಸ್ವರೂಪಮ್” \enginline{-} ತರ್ಕ ಎಂಬುದರ ಲಕ್ಷಣದ ಸ್ವರೂಪವೇನು ಎಂಬ ವಿಷಯವನ್ನು ವಿದ್ವಾನ್ ವಾಚಸ್ಪತಿಮಿಶ್ರರು ಪ್ರಸ್ತಾಪಿಸಿದರು. ಅನಂತರ ಶ್ರೀಯುತ ಶ್ರೀಧರ ಶಾಸ್ತ್ರಿಗಳು ತರ್ಕಲಕ್ಷಣಪರಿಷ್ಕಾರಃ ಆ ಲಕ್ಷಣವನ್ನು ಇನ್ನೂ ಸೂಕ್ಷ್ಮವಾಗಿ ಚಿಂತಿಸುವಿಕೆ ಎಂಬ ವಿಷಯವನ್ನು ಪ್ರತಿಪಾದಿಸಿದರು. ಇವಿಷ್ಟು ಗೋಷ್ಠಿಯ ಸಾರ.
\eject

ಗೋಷ್ಠಿ ಮುಗಿದಮೇಲೆ ಪಾಠಶಾಲೆಯಲ್ಲೇ ಮಧ್ಯಾಹ್ನ ಸುಗ್ರಾಸ ಭೋಜನಕ್ಕೆ ವ್ಯವಸ್ಥೆ ಮಾಡಲಾಗಿತ್ತು. ಸುಮಾರು ೨೦೦\enginline{-}೨೫೦ ಜನರು  ಭೋಜನಮಾಡಿದರು. ಎರಡನೇ ಪಂಕ್ತಿ ಭೋಜನ ಮುಗಿಯುತ್ತಿದ್ದಂತೆ ನಾವೆಲ್ಲರೂ ಉದ್ದಿಷ್ಟವಾದ ಕಾರ್ಯ\-ಕ್ರಮಕ್ಕೆ ಸಿದ್ಧವಾದೆವು. ಕಾರ್ಯಕ್ರಮ ೩ ಗಂಟೆಗೆ ಪ್ರಾರಂಭವಾಗಬೇಕು. ಈ\break ಸಂದರ್ಭದಲ್ಲಿ ರಾಜಮಾತೆ ಪ್ರಮೋದಾದೇವಿಯವರು ದಯಮಾಡಿಸುತ್ತಿರುವುದು ಅತ್ಯಂತ ವಿಶೇಷ. ಇಬ್ಬರು ಅರಮನೆಗೆ ತೆರಳಿ ಅಲ್ಲಿಂದ  ಅವರನ್ನು ಪಾಠಶಾಲೆಗೆ ಬರ\-ಮಾಡಿ\-ಕೊಂಡೆವು. ಪಾಠಶಾಲೆಯ ಮಹಾದ್ವಾರದಲ್ಲಿ ಶ್ರೀಯುತ ಗಂಗಾಧರ ಭಟ್ಟರು ಮತ್ತು ಶ್ರೀಯುತ ಪಿ.ಎನ್.ಶಾಸ್ತ್ರಿಗಳು ಮತ್ತಿತರರೆಲ್ಲ ಉಪಸ್ಥಿತರಿದ್ದರು. ರಾಜಮಾತೆ\break ಡಾ । ಪ್ರಮೋದಾದೇವಿ ಒಡೆಯರ್ ರವರನ್ನು ಪೂರ್ಣಕುಂಭದೊಂದಿಗೆ ವೇದಘೋಷಗಳೊಡನೆ ಸ್ವಾಗತಿಸಲಾಯಿತು. ಅತ್ಯಂತ ವಿಜೃಂಭಣೆಯಿಂದ ಅವರನ್ನು ಪಾಠಶಾಲೆಯ ಮಧ್ಯಮಾವರಣದಲ್ಲಿರುವ ವಿದ್ಯಾಗಣಪತಿ ಸನ್ನಿಧಿಗೆ ಕರೆದುಕೊಂಡುಬಂದೆವು. ಅಲ್ಲಿ\break ವಿದ್ಯಾಗಣಪತಿಗೆ ಆರಾರ್ತಿಸೇವೆ ನಡೆಯಿತು. ರಾಜಮಾತೆಯವರು ತೀರ್ಥ, ಪ್ರಸಾದಗಳನ್ನು ಸ್ವೀಕರಿಸಿದರು. ಅನಂತರ ಅವರನ್ನು ಸಭೆಗೆ ಬಿಜಯಮಾಡಿಸಿದೆವು. 

ರಾಜಮಾತೆ ಡಾ । ಪ್ರಮೋದಾದೇವಿಯವರು, ಸನ್ಮಾನ್ಯ ಗಂಗಾಧರ ಭಟ್ಟರು,\break ಗೌರವಾಧ್ಯಕ್ಷರಾದ ಡಾ । ನಿರಂಜನ ವಾನಳ್ಳಿಯವರು, ವಿದ್ವಾಂಸರಾದ ರಾಜೇಶ್ವರ ಶಾಸ್ತ್ರಿಗಳು, ವಿ । ಉಮಾಕಾಂತ ಭಟ್ಟರು ವೇದಿಕೆಯಲ್ಲಿ ವಿರಾಜಮಾನರಾದರು.

ವಿದ್ವಾನ್ ನರಸಿಂಹ ಭಟ್ಟ ಮತ್ತು ವಿದ್ವಾನ್ ನಾರಾಯಣ ವೈದ್ಯ ಇವರು ಕಾರ್ಯ\-ಕ್ರಮದ ನಿರೂಪಣೆ ಪ್ರಾರಂಭಿಸಿದರು. ವೇ । ಬ್ರ । ಶ್ರೀ ರಮೇಶ ಅಡಿಗರು ತಮ್ಮ ವಿದ್ಯಾರ್ಥಿಗಳೊಂದಿಗೆ ನಡೆಸಿಕೊಟ್ಟ ಸುಶ್ರಾವ್ಯ ವೇದಘೋಷದೊಂದಿಗೆ ಅಭಿವಂದನ ಕಾರ್ಯಕ್ರಮದ ಶುಭಾರಂಭಾವಾಯಿತು. ಸಮಿತಿಯ ಅಧ್ಯಕ್ಷರಾದ ವಿದ್ವಾನ್ ಕೆ.ಎಲ್. ರಾಘವ ಪ್ರಾಸ್ತಾವಿಕ ಮಾತುಗಳನ್ನಾಡಿದರು \enginline{-} “ಈ ಸಂದರ್ಭದಲ್ಲಿ ಪೂಜ್ಯರಾದ ಗಂಗಾಧರ ಭಟ್ಟರು ಯಾವುದೇ ಸನ್ಮಾನ ಪ್ರದರ್ಶನದ ಕಾರ್ಯಕ್ರಮವನ್ನು ಇಷ್ಟ\-ಪಡುವವರಲ್ಲ. ಆದ್ದರಿಂದ ಮೊದಲು ಇದನ್ನು ತಿರಸ್ಕರಿಸಿದರೂ ಆಮೇಲೆ ವಿದ್ಯಾರ್ಥಿಗಳ ಆತ್ಮೀಯ ಒತ್ತಾಸೆಯನ್ನು ತಿರಸ್ಕರಿಸಲಾರದೇ ಅಭಿವಂದನ ಕಾರ್ಯಕ್ರಮವನ್ನು ಪುರಸ್ಕರಿಸಿದರು. ಅದು ನಮ್ಮ ಭಾಗ್ಯ, ಪ್ರತ್ಯಕ್ಷೇ ಗುರವಸ್ತುತ್ಯಾಃ ಎಂಬಂತೆ ನಮ್ಮ ಭಕ್ತಿಯ ಅಭಿವ್ಯಕ್ತಿಗೊಂದು ಅವಕಾಶ  ಸಿಕ್ಕಿದೆ ಅಷ್ಟೆ , ಗಂಗಾಧರ ಭಟ್ಟರು ಎಲ್ಲ ವಿದ್ಯಾಗುರು\-ಗಳಂತೆ ಕೇವಲ ವಿದ್ಯಾಗುರುಗಳಾಗಿಲ್ಲ, ಅವರನ್ನು ತಂದೆ ಎಂದೂ ಹೇಳಬಹುದು, ತಾಯಿ ಎಂದೂ ಹೇಳಬಹುದು. ಏಕೆಂದರೆ ಆ ವಾತ್ಸಲ್ಯವನ್ನೆಲ್ಲ ನಾವು ಅವರಲ್ಲಿ ಅನುಭವಿ\-ಸಿದ್ದೆವೆ” ಎಂದರು. 

ರಾಜಮಾತೆ ಪ್ರಮೋದಾ ದೇವಿಯವರನ್ನು ಸ್ವಾಗತಿಸುತ್ತಾ, “ಮೈಸೂರಿನ ಹೃದಯ\-ಭಾಗದಲ್ಲಿರುವ ಸಂಸ್ಕೃತ ಪಾಠಶಾಲೆಯ ಸ್ಥಾಪನೆಗೆ ಕಾರಣೀಕರ್ತರೇ ಮೈಸೂರಿನ ಮಹಾರಾಜರು, ಮಾತ್ರವಲ್ಲ ಕಲೆ, ಸಂಸ್ಖೃತಿ, ಸಂಗೀತ, ಶಿಕ್ಷಣ, ಬ್ಯಾಂಕಿಂಗ್ ಉದ್ಯಮ, ನೀರಾವರಿ, ಕೃಷಿ ಹೀಗೆ ಇವತ್ತಿನ ಸಮೃದ್ಧಕರ್ನಾಟಕಕ್ಕೇ ಅವರ ಆಡಳಿತದ ಕೊಡುಗೆ\-ಯಿದೆ. ಸಂಸ್ಕೃತ ಪಾಠಶಾಲೆಯ ಸಂಸ್ಥಾಪನೆಗೆ ಕಾರಣೀಭೂತರಾದ ಅವರು ನಮ್ಮ ಕೃತಜ್ಞತೆಗೆ ಆದ್ಯ ಪಾತ್ರರು.  ಹಾಗಾಗಿ ಅವರನ್ನು ವಿಶೇಷವಾಗಿ ಭಾವಿಸಿ ಪಾಠಶಾಲೆಗೆ ಆಹ್ವಾನಿಸಿ ಕೃತಜ್ಞತೆಯನ್ನು ಸಲ್ಲಿಸಲೇಬೇಕಾದ ಅಗತ್ಯವಿದೆ ಎಂಬ ಶ್ರೀಯುತ ಗಂಗಾಧರ ಭಟ್ಟರ ಅಪೇಕ್ಷೆಯ ಮೇರೆಗೆ ನಾವು ರಾಜಮಾತೆಯವರನ್ನು ಇಲ್ಲಿಗೆ ಆಹ್ವಾನಿಸಿದ್ದೇವೆ, ನಮ್ಮ ಮನವಿಯನ್ನು ಮನ್ನಿಸಿ ರಾಜಮಾತೆಯವರು ದಯಮಾಡಿಸಿರುವುದು ಒಂದು ಐತಿಹಾಸಿಕ ಕ್ಷಣವಾಗಿದೆ, ಅವರನ್ನು ನಾವೆಲ್ಲರೂ ಹೃತ್ಪೂರ್ವಕವಾಗಿ ಸ್ವಾಗತಿಸುತ್ತೇವೆ” ಎಂದರು.

ಗಂಗಾಧರ ಭಟ್ಟರು ಓದುವಾಗ ಇವರ ಜೊತೆಯಲ್ಲೇ ಸಂಸ್ಕೃತ ಮಹಾರಾಜ ಪಾಠಶಾಲೆಯಲ್ಲಿ ಅಧ್ಯಯನ ಮಾಡುತ್ತಿದ್ದವರು ಡಾ । ಪಿ.ಎನ್.ಶಾಸ್ತ್ರಿಗಳು. ಅವರು ಇಂದು ರಾಷ್ಟ್ರಿಯ ಸಂಸ್ಕೃತ ಸಂಸ್ಥಾನದ ಕುಲಪತಿಗಳು. ಅವರು ನಮ್ಮ ಆಹ್ವಾನವನ್ನು ಪ್ರೀತಿಯಿಂದ ಸ್ವೀಕರಿಸಿ ಗಂಗಾಧರ ಭಟ್ಟರ ಬಗೆಗಿನ ವಿಶೇಷ ಅಭಿಮಾನದಿಂದ ಈ\break ಕಾರ್ಯಕ್ರಮಕಾಗಿಯೇ ದೂರದ ದೆಹಲಿಯಿಂದ ದಯಮಾಡಿಸಿದ್ದಾರೆ. ಅವರಿಗೆ ಆತ್ಮೀಯ ಸ್ವಾಗತ ಬಯಸುತ್ತೇವೆ.

ಹೀಗೆ ಮುಂದುವರಿದು ನಿರಂಜನವಾನಳ್ಳಿಯವರನ್ನೂ, ವಿ । ರಾಜೇಶ್ವರ ಶಾಸ್ತ್ರಿ\-ಗಳನ್ನೂ, ವಿ । ಉಮಾಕಾಂತ ಭಟ್ಟರನ್ನೂ ಸಭೆಗೆ ಪರಿಚಯಿಸಿ ಸ್ವಾಗತಿಸಿದರು.

ಸ್ವಾಗತದ  ಅನಂತರ ಸ್ವರ್ಣವಲ್ಲೀ ಸಂಸ್ಥಾನದ ಪೀಠಾಧಿಪತಿಗಳಾದ ಶ್ರೀಶ್ರೀ ಗಂಗಾಧರೇಂದ್ರ ಸರಸ್ವತೀ ಸ್ವಾಮಿಗಳು  ಗಂಗಾಧರ ಭಟ್ಟರಲ್ಲಿ ಅನೇಕ ಕಾಲ ಪಾಠ\-ಪ್ರವಚನವಗಳನ್ನು ಕೇಳಿದ ಕೃತಜ್ಞತೆಯಿಂದ ತಾವೇ ಸ್ವಪ್ರೇರಣೆಯಿಂದ ಪ್ರಸಾದಿಸಿರುವ ಆಶೀರ್ವಾದಪೂರ್ವಕ ಫಲ, ಮಂತ್ರಾಕ್ಷತೆಗಳನ್ನು ವಿ । ಶಂಕರ ಭಟ್ಟರು ಶ್ರೀಮಾನ್ ಗಂಗಾಧರ ಭಟ್ಟರಿಗೆ ನೀಡಿದರು. ಅಂತೆಯೇ ಶ್ರೀಶ್ರೀಗಳ ಸಂದೇಶವನ್ನು ಸಭೆಯಲ್ಲಿ ಶಂಕರ ಭಟ್ಟರು ನಿವೇದಿಸಿದರು. ಇದೇ ಸಂದರ್ಭದಲ್ಲಿ ಶಂಕರ ಭಟ್ಟರು ತಮ್ಮ ವಿದ್ಯಾಭ್ಯಾಸದ ಸಂದರ್ಭವನ್ನೂ ಜ್ಞಾಪಿಸಿಕೊಂಡು ಆ ಕಾಲದಲ್ಲಿ ಗಂಗಾದರ ಭಟ್ಟರು ವಿದ್ಯಾರ್ಥಿಗಳಾದ ನಮಗೆಲ್ಲ ಆತ್ಮೀಯ ಧ್ವನಿಯಾಗಿದ್ದರು ಎನ್ನುತ್ತ ತಮ್ಮ ಕೃತಜ್ಞತೆಯನ್ನು ಸಲ್ಲಿಸಿದರು.

ಏತದನಂತರ ಪಾಠಶಾಲೆಯ ಅಧ್ಯಾಪಕ ಸಂಘದ ಪರವಾಗಿ ವಿ । ಎಮ್ ಎ ಆಳ್ವಾರ್ ರವರು ಅಭಿಪ್ರಾಯವನ್ನು ಹಂಚಿಕೊಂಡರು. ಅವರು, “ವಿದ್ವಾಂಸರಾದ ಗಂಗಾಧರ ಭಟ್ಟರು ಶಿಕ್ಷಕರಿಗೆಲ್ಲ ಆದರ್ಶ ಶಿಕ್ಷಕರು, ಯಾವತ್ತೂ ವಿದ್ಯಾರ್ಥಿಗಳ ಹಿತೈಷಿಗಳು, ಕಾಳಿದಾಸೋಕ್ತಿಯಂತೆ ಪಾಠಮಾಡುವ ಕಲೆಯಲ್ಲಿ ವಿಶೇಷವಾಗಿ ಪ್ರಾವೀಣ್ಯವನ್ನು ಹೊಂದಿದವರು ಎಂದು ಅಭಿಪ್ರಾಯಪಟ್ಟರು.

ಆಮೇಲೆ ಶ್ರೀಯುತ ಗಂಗಾಧರ ಭಟ್ಟರ ವಿದ್ಯಾರ್ಥಿಗಳ ಪರವಾಗಿ ವಿ । ವಿನಾಯಕ ಭಟ್ಟ ಗಾಳೀಮನೆಯವರ ಮಾತು \enginline{-} “ಶಾಸ್ತ್ರಗಳನ್ನು ಶ್ರೀಯುತರು ನನಗೆ ಬೋಧಿ\-ಸಿದರು, ಅದು ನನಗೆ ಯಾವುದೂ ನೆನಪಿಲ್ಲ ಎಂಬುದು ನನಗೆ ಗೊತ್ತು, ಅವರಿಗೂ ಗೊತ್ತು, ಆದರೆ ಅವರು ನನಗೆ ಕಲಿಸಿದ ಜೀವನ ಪರಿಷ್ಕಾರ ಮಾತ್ರ ಇನ್ನೂ ಸ್ಮರಣೆಯಲ್ಲಿದೆ” ಎಂದರು. 

ಇನ್ನೊಬ್ಬ ವಿದ್ಯಾರ್ಥಿ ಡಾ । ಶಿವರಾಮ ಮಾತನಾಡುತ್ತಾ, “ನಾನು ಮೈಸೂರಿಗೆ ಬಂದು ಗಂಗಾಧರ ಭಟ್ಟರನ್ನು ಮೊದಲು ಪರಿಚಯ ಮಾಡಿಕೊಂಡೆ. ಆದರೆ ಇನ್ನೂ ಅವರನ್ನು ಪರಿಚಯ ಮಾಡಿಕೊಳ್ಳಬೇಕಾದದ್ದು ಇದೆ ಎಂದು ಅನ್ನಿಸುತ್ತದೆ. ಸಾಮಾನ್ಯ\-ವಾಗಿ ಬಹುತೇಕ ಶಿಕ್ಷಕರು ’ಇವನು ನಮ್ಮ ಶಾಸ್ತ್ರದ ವಿದ್ಯಾರ್ಥಿ, ಇವನು ನಮ್ಮ\-ಶಾಸ್ತ್ರದವನಲ್ಲ ಎಂಬ ತರತಮ ಭಾವದಿಂದ ವಿದ್ಯಾರ್ಥಿಗಳನ್ನು ನೋಡುತ್ತಾರೆ. ಆದರೆ ಶ್ರೀಯುತರು ವಿದ್ಯಾರ್ಥಿಗಳಲ್ಲಿ ಭೇದಭಾವವನ್ನೇ ಎಣಿಸಿದವರಲ್ಲ. ಹೊತ್ತುಗೊತ್ತು ಇಲ್ಲದೇ ಅವರ ಮನೆಗೆ ಪಾಠಕ್ಕೆ ಹೋಗುತ್ತಿದ್ದೆವು. ಆದರೂ ಪಾಠ ನಡೆಯುತ್ತಿತ್ತು. ನಮ್ಮ ಈ ನಡೆಗೆ ಗುರುಪತ್ನಿಯವರೂ ಸಹ ಸ್ವಲ್ಪವು ಬೇಸರಿಸಿಕೊಳ್ಳದೇ ಸಂತೋಷದಿಂದ ನಮ್ಮನ್ನು ಸ್ವಾಗತಿಸುತ್ತಿದ್ದರು. ಅಂತಹ ಗುರುದಂಪತಿಗಳಿಗೆ ನನ್ನ ನಮಸ್ಕಾರ” ಎಂದು\break ಭಾವುಕರಾಗಿ ನುಡಿದರು.

ವಿ । ಅನಂತ ಎಮ್.ಎ. ಇವರು ಇನ್ನೊಬ್ಬ ವಿದ್ಯಾರ್ಥಿ, ಅವರು, “ಗಂಗಾಧರ ಭಟ್ಟರು ಮತ್ತು ನನ್ನ ಅಣ್ಣ (ವಿ । ಎಮ್ ಎ ಆಳ್ವಾರ್)ನವರ ಅಧೀನದಲ್ಲಿ ಪಾಠವನ್ನು ಕೇಳುವ ಭಾಗ್ಯ ನನಗೊದಗಿತು. ಗಂಗಾಧರ ಭಟ್ಟರಿಂದ ಜ್ಞಾನಧಾರೆ ಪುಂಖಾನುಪುಂಖ\-ವಾಗಿ ಹರಿಯುತ್ತಿತ್ತು. ನ್ಯಾಯಶಾಸ್ತ್ರದ ವಿಷಯವನ್ನು ಆಂಗ್ಲಭಾಷೆಯಲ್ಲಿ ಹೇಳುವವರು ತುಂಬಾ ವಿರಳ, ಗಂಗಾಧರ ಭಟ್ಟರು ಶಾಸ್ತ್ರ ಕಠಿಣವಾದರೂ ಸರಳವಾಗಿ ಬೋಧಿಸುತ್ತಾರೆ” ಎಂದು ಅಭಿಪ್ರಾಯಪಟ್ಟರು.

ಗಂಗಾಧರ ಭಟ್ಟರು ಸಮಾಜಕ್ಕೆ ತೆರೆದುಕೊಂಡವರಷ್ಟೆ ! ಬಹಳ ಜನ ವಿವಿಧ ಕ್ಷೇತ್ರ\-ದಲ್ಲಿರುವವರೂ ಸಹ ಅವರ ಅಭಿಮಾನಿಗಳಾಗಿದ್ದಾರೆ. ಆದ್ದರಿಂದ ಅಂಥವರ ಅಪೇಕ್ಷೆಯ ಮೇರೆಗೆ ಹಿರಿಯ ವಕೀಲರಾದ ಶ್ರೀಯುತ ಕೆ.ಆರ್ ಶಿವಶಂಕರ್ ರವರ ಅಭಿಪ್ರಾಯಕ್ಕೆ ಅವಕಾಶ ಕೊಡಲಾಗಿತ್ತು, ಅವರು ಸಭೆಯಲ್ಲಿ ಮಾತನಾಡುತ್ತಾ, “ಹುಲ್ಲಾಗು  ಬೆಟ್ಟದಲಿ ಮನೆಗೆ ಮಲ್ಲಿಗೆಯಾಗು” ಎಂಬ ಡಿವಿಜಿಯವರ ಮಾತನ್ನು ಗಂಗಾಧರ ಭಟ್ಟರಿಗೆ ಅನ್ವಯಿಸಿದರು. “ಸ್ನೇಹ, ಸೇವೆ, ಸಹಾಯ, ಸಂಘಟನೆ” ಇದು ಗಂಗಾಧರ ಭಟ್ಟರ ವ್ಯಕ್ತಿತ್ವ, ಒಮ್ಮೆ ಅವರೊಡನೆ ಸ್ನೇಹವುಂಟಾದರೆ ಅದು ಜೀವನ ಪರ್ಯಂತ. ಅವರು ಗುರುಜನರಿಗೆ ಮಾಡುವ ಸೇವೆ ಮಾದರಿಯಾದದ್ದು. ನಿಸ್ವಾರ್ಥ ಸಹಾಯ ಮಾಡುವ ಸ್ವಭಾವವಂತೂ ಅನಿತರ ಸಾಧಾರಣವಾದುದು. ಆ ಕಾಲದಲ್ಲಿ ವಿದ್ಯಾರ್ಥಿಗಳಿಗೆ ಊಟ, ವಸತಿ, ಇಲ್ಲದೇ ವಿದ್ಯಾರ್ಥಿಗಳಿಗೆ ಕಾಲೇಜಿನಲ್ಲಿ ಅಧ್ಯಯನಕ್ಕೆ ಮತ್ತು ಪರೀಕ್ಷೆ ಬರೆಯುವಲ್ಲಿ  ಉಂಟಾಗುವ ಕಾನೂನು ತೊಡಕುಗಳನ್ನು ಬಗೆಹರಿಸಿಕೊಡುತ್ತಿದ್ದರು. ಹೀಗಾಗಿ ಒಮ್ಮೆ ಸುಮಾರು ನಲವತ್ತು ವಿದ್ಯಾರ್ಥಿಗಳು ಪರೀಕ್ಷೆ ಬರೆಯಲು ವಂಚಿರಾಗುವ ಸನ್ನಿವೇಶ ಉಂಟಾದಾಗ ಅವರ ಸಹಕಾರದಿಂದಲೇ ಪರೀಕ್ಷೆಬರೆಯುವಂತಾಯಿತು. ನಾನು ಇತ್ತೀಚೆಗೆ ಕಟ್ಮಂಡುವಿಗೆ ಹೋದಾಗ ಅಲ್ಲಿಯವರೊಬ್ಬರು ನನ್ನನ್ನು ಇವರ ಸ್ನೇಹಿತನೆಂದು ಗುರುತಿಸಿ ನನಗೆ ಸಕಲ ವ್ಯವಸ್ಥೆಗಳನ್ನೂ ಮಾಡಿದರು. ಆಶ್ಚರ್ಯವಾಯಿತು, ವಿದ್ವಾನ್ ಸರ್ವತ್ರ ಪೂಜ್ಯತೇ ಎಂಬ ವಿಷಯ ಮನವರಿಕೆಯಾಯಿತು, ಹಾಗೆಯೇ ನನ್ನಿಂದ ಭಾರತೀಯ ಕಾನೂನು\break ವಿಚಾರವಾಗಿಯೂ ಮತ್ತು ಕಾನೂನಿನಲ್ಲಿ ನಮ್ಮ ಧರ್ಮಶಾಸ್ತ್ರವನ್ನು ಎಷ್ಟು ಅಳವಡಿಸಿಕೊಂಡಿದ್ದಾರೆಂಬ ವಿಷಯವನ್ನೂ ವಿದ್ಯಾರ್ಥಿಗಳಿಗೆ ತಿಳಿಸಬೇಕೆಂದು ಆದೇಶಿಸಿ ನನ್ನಿಂದ ಪಾಠ \enginline{-} ಪ್ರವಚನಗಳನ್ನು ಮಾಡಿಸಿದರು. ಇಂತಹ ವಿಶಿಷ್ಟ ವ್ಯಕ್ತಿಯಾದ ಶ್ರೀಯುತರಿಗೆ ಮತ್ತು ಅವರ ಕುಟುಂಬಕ್ಕೆ ನನ್ನ ಅನಂತ ವಂದನೆಗಳು” ಎಂದು ಗಂಗಾಧರ ಭಟ್ಟರ ವಿವಿಧ ಮುಖಗಳ ಪರಿಚಯವನ್ನು ಸಭೆಗೆ ಮಾಡಿಕೊಟ್ಟರು.

ಈ ಮೊದಲೇ ತಿಳಿಸಿದಂತೆ ಈ ಕಾರ್ಯಕ್ರಮದಲ್ಲಿ ಸಾಕೇತಿಕವಾಗಿ ಅಭಿವಂದನ\-ಗ್ರಂಥದ ಬಿಡುಗಡೆಯಾಗಬೇಕಿರುವುದನ್ನು ಈ ಸಂದರ್ಭದಲ್ಲಿ ನೆರವೇರಿಸಲಾಯಿತು. ಬಿಡುಗಡೆಗಾಗಿ ಡಾ । ಪಿ ಎನ್ ಶಾಸ್ತ್ರಿಗಳನ್ನು ವಿನಂತಿಸಲಾಗಿ ಅವರು “ಜ್ಞಾನಗಂಗಾಧರ” ಎಂಬ ಶುಭ ಶಿರೋನಾಮೆಯಿಂದ ಶೋಭಿಸುವ ಗ್ರಂಥವನ್ನು ಬಿಡಿಗಡೆಗೊಳಿಸಿದರು. 

ಈ ಕಾರ್ಯಕ್ರಮದ ಅನಂತರ, ಸುಧರ್ಮಾ ಸಂಸ್ಕೃತ ಪತ್ರಿಕೆಯ ಸಂಪಾದಕರಲ್ಲಿ ಗಂಗಾಧರ ಭಟ್ಟರೂ ಒಬ್ಬರಾದ್ದರಿಂದ ಆ ಪತ್ರಿಕೆಯ ವಾರ್ಷಿಕ ಸಂಚಿಕೆಯನ್ನು ಗಂಗಾಧರ ಭಟ್ಟರಿಂದಲೇ ಬಿಡುಗಡೆಗೊಳಿಸಬೇಕೆಂಬ ಸಂಕಲ್ಪ ಸುಧರ್ಮಾ ಮಾಲೀಕರದು. ಅಲ್ಲದೇ ಇದೇ ಸುಧರ್ಮಾಪ್ರಕಾಶನದಿಂದ ಪ್ರಕಾಶಿಸಲು ಉದ್ದೇಶಿಸಿರುವ ವಿ । ಶಿಂಗಪ್ಪನವರ ಮಹಾತ್ಮನಾಂ ಚರಿತಮ್ ಎಂಬ ಪುಸ್ತಕವನ್ನು ಸಹ ಬಿಡುಗಡೆಗೊಳಿಸಬೇಕೆಂಬ ವರಾತದಿಂದ ಗಂಗಾಧರ ಭಟ್ಟರು ಅವನ್ನು ಬಿಡುಗಡೆಗೊಳಿಸಿದರು.
\vskip 5pt

ಇದಾದ ಅನಂತರ ಡಾ । ಪಿ.ಎನ್.ಶಾಸ್ತ್ರಿಯವರ ಮಾತಿಗೆ ವೇದಿಕೆಯನ್ನು ಕಲ್ಪಿಸ\-ಲಾಯಿತು. ಅವರು, “ಋಷೀಣಾಂ ನಿಧಿಗೋಪಃ ಅನೂಚಾನಃ ಎಂದು \hbox{ಮಾತನ್ನಾರಂಭಿಸಿ} ಅದನ್ನು ವಿವರಿಸಿದರು. ಅನೂಚಾನ ಎಂದರೆ ಪರಂಪರಾಗತವಾಗಿ ನಡೆದುಬಂದಿದ್ದು ಎಂಬ ಅರ್ಥವಿದೆ. ಆದರೆ ನಿಜವಾಗಿ ಋಷಿಗಳ ನಿಧಿಯನ್ನು ರಕ್ಷಿಸುವವನಿಗೆ ಅನೂಚಾನ ಎನ್ನುತ್ತಾರೆ. ಋಷಿಗಳ ನಿಧಿ ಅಂದರೆ \enginline{-} ವೇದ, ಶಾಸ್ತ್ರಗಳು. ಇವುಗಳನ್ನು ರಕ್ಷಿಸುವವನಿಗೆ ಅನೂಚಾನ ಎಂದು ಹೆಸರು. ಇಂತಹ ಅನೂಚಾನರಿಗೆ ಆಶ್ರಯವಾದದ್ದು ಮಹಾರಾಜ ಸಂಸ್ಕೃತ ಮಹಾಪಾಠಶಾಲೆ. ಮೈಸೂರಿನ ಮಹಾರಾಜರು ಇದನ್ನು ಸ್ಥಾಪನೆ ಮಾಡಿದರು. ಅವರ ಪ್ರತಿನಿಧಿಯಾಗಿ ರಾಜಮಾತೆಯವರು ಇಲ್ಲಿ ಉಪಸ್ಥಿತರಿದ್ದಾರೆ. ಅದು ನಮಗೆಲ್ಲ ಬಹಳ ಸಂತೋಷದ ಸಂಗತಿಯಾಗಿದೆ. 
\vskip 5pt

ಶಾಸ್ತ್ರಿಗಳು ಮುಂದುವರೆದು ಗಂಗಾಧರ ಭಟ್ಟರ ವಿಷಯದಲ್ಲಿ ಮಾತನಾಡುತ್ತಾ, \enginline{-} ಮಿತ್ರರಾದ ಗಂಗಾಧರ ಭಟ್ಟರನ್ನು ನೋಡಿದರೆ ಕೆಲವೊಮ್ಮೆ ಅಸೂಯೆಯಾ\-ಗುತ್ತದೆ. ನಾನೂ ನ್ಯಾಯವನ್ನು ಓದಿದ್ದೇನೆ, ಇಲ್ಲೂ ಅನೇಕರು ನ್ಯಾಯವನ್ನು ಓದಿದವರು ಇರಬಹುದು, ಆದರೆ ಗಂಗಾಧರ ಭಟ್ಟರು ಶಾಸ್ತ್ರಪಾಠಮಾಡಿ ಇಷ್ಟೊಂದು ಜನ ವಿದ್ಯಾರ್ಥಿಗಳನ್ನು ತಯಾರು ಮಾಡಿದಂತೆ ನಾವ್ಯಾರೂ ತಯಾರುಮಾಡಿಲ್ಲ. ಈ ದೃಷ್ಟಿಯಿಂದ ಗಂಗಾಧರ ಭಟ್ಟರ ವಿದ್ಯಾರ್ಥಿಬಳಗವನ್ನು ನೋಡಿದಾಗ ಅವರ ಬಗ್ಗೆ ಅಪಾರವಾದ ಗೌರವ ಮೂಡುತ್ತದೆ. ಹಾಗಾಗಿ ಪತ್ನೀ\enginline{-} ಶೈಲಜಾಸಹಿತರಾದ ಗಂಗಾಧರ ಭಟ್ಟರನ್ನು ವಂದಿಸುತ್ತೇನೆ. ಇವರ ಶ್ರೇಯಸ್ಸಿಗೆ ಪತ್ನಿಯ ಪಾತ್ರ ಬಹಳ ಮುಖ್ಯ\-ವಾದುದು, ಅದನ್ನು ಶಂಕರ ಭಗವತ್ಪಾದರೂ ಸಹ ವ್ಯಕ್ತಪಡಿಸಿದ್ದಾರೆ ಚಿತಾಭಸ್ಮಾಲೇಪೋ ಎಂಬ ಶ್ಲೋಕದಲ್ಲಿ \enginline{-} “ದೇವಿ ನೀನು ಶಿವನ ಕೈಹಿಡಿದಿರುವುದರಿಂದ ತಾನೆ ಶಿವನು ಜಗದೀಶ ಎಂಬ ಪದವಿಯನ್ನು ಪಡೆಯಲು ಸಾಧ್ಯವಾಗಿರುವುದು” ಎಂದಿದ್ದಾರೆ. ಹಾಗಾಗಿ ಪತಿಯ ಶ್ರೇಯಸ್ಸಿನಲ್ಲಿ ಪತ್ನಿಯ ಪಾತ್ರ ಅತ್ಯಂತ ಪ್ರಧಾನವಾದುದು. ಹಾಗಾಗಿ ಪತ್ನೀ\-ಸಹಿತರಾದ ಭಟ್ಟರಿಗೆ ವಂದೆನೆ ಸಲ್ಲಿಸುತ್ತೇನೆ ಎಂದರು.

ಈ ಮಧ್ಯದಲ್ಲಿ ಅವರು ಇನ್ನೂ ಒಂದು ಅಂಶವನ್ನು ಸಭೆಯ ಗಮನಕ್ಕೆ ತಂದರು. ಇಂದಿನ ಕಾಲದಲ್ಲಿ ಸಂಸ್ಕೃತವನ್ನು ಓದಲು ನಮ್ಮವರು ಬಯಸುತ್ತಿಲ್ಲ. ಅದಕ್ಕೆ ಪ್ರಮುಖವಾಗಿ ತಂದೆತಾಯಿಗಳೇ ಕಾರಣ. ಆದರೆ ಇನ್ನು ಮುಂದಿನ ಅವಧಿಗಳಲ್ಲಿ ಎಲ್ಲ ಸ್ಪರ್ಧಾತ್ಮಕ ಪರೀಕ್ಷೆಗಳ ಪ್ರಶ್ನೆಪತ್ರಿಕೆಗಳಲ್ಲಿ ಸಂಸ್ಕೃತಕ್ಕೆ ಸಂಬಂಧಿಸಿದ ಪ್ರಶ್ನೆಗಳು ಕಡ್ಡಾಯವಾಗಿ ಇರುತ್ತವೆ. ಅದಕ್ಕಾಗಿ ಎಲ್ಲರೂ ಸಂಸ್ಕೃತವನ್ನು ಓದುವುದು ಅತ್ಯವಶ್ಯ.  

ಹೀಗೆ ತಮ್ಮ ಭಾವನೆಯನ್ನು ವ್ಯಕ್ತಪಡಿಸಿದ ಅವರು ತಾವೇ ತಂದಿದ್ದ ಫಲವಸ್ತ್ರಾದಿ\-ಗಳನ್ನು ಗಂಗಾಧರ ಭಟ್ಟರಿಗೆ ಸಮರ್ಪಿಸಿ ಗಂಗಾಧರ ಭಟ್ಟರನ್ನು ಅಭಿನಂದಿಸಿದರು.

ಇದಾದ ಅನಂತರ ಕಾರ್ಯಕ್ರಮದ ಪ್ರಧಾನ ಸಂದರ್ಭವಾದ ಅಭಿವಂದನ ಕಾರ್ಯಕ್ರಮ. ಅಭಿವಂದನೆಯ ಸ್ವೀಕಾರಕ್ಕಾಗಿ ದಂಪತಿಗಳನ್ನು ಸಭೆಯ ಮಧ್ಯದಲ್ಲಿ ಆಸೀನರಾಗುವಂತೆ ಪ್ರಾರ್ಥಿಸಲಾಯಿತು. ಈ ಸಂದರ್ಭದಲ್ಲಿ ವಿದ್ಯಾರ್ಥಿಗಳು\break ಶ್ರೀಯುತ\-ರೊಡನೆ ಒಡನಾಡಿದ ಕಾಲದಲ್ಲಿ ಗ್ರಹಿಸಿದ ಅವರ ಜೀವನದ ಸಾರರೂಪವಾದ ಅಂಶಗಳನ್ನು ಸಂಗ್ರಹಿಸಿ ಸಾಹಿತ್ಯದ ರೂಪವನ್ನು ಕೊಟ್ಟು ಈ ಸಂದರ್ಭದಲ್ಲಿ ಅವರಿಗೆ ಸಮರ್ಪಿಸಲು “ಅಭಿವಂದನಪತ್ರ”ವೊಂದನ್ನು ಸಿದ್ಧಪಡಿಸಲಾಗಿತ್ತು. ಅದನ್ನು ಪ್ರಕೃತ\break ನೆರವೇರಲಿರುವ ಅಭಿವಂದನೆಯ ಪೂರ್ವಾಂಗವಾಗಿ ಸಭೆಯಲ್ಲಿ ವಾಚಿಸಲಾಯಿತು. ಈ ವಾಚನವನ್ನು ವಿ । ಗುರುಪ್ರಸಾದರು ನೆರವೇರಿಸಿದರು. 

ಈಗ ಅಭಿವಂದನೆ \enginline{-} ವಿದ್ಯಾರ್ಥಿಗಳು ಮತ್ತು ಸಮಾಜ ವಿಶೇಷವಾಗಿ ಅಪೇಕ್ಷಿಸಿ\break ನಿರೀಕ್ಷಿಸುತ್ತಿದ್ದ ಸುಸಂದರ್ಭ. ಯಾರ ಜೀವನ ವಿದ್ಯಾಪ್ರವಾಹವಲ್ಲದೇ ಮತ್ತೇನೂ ಆಗಿರ\-ಲಿಲ್ಲವೋ, ಅಂತಹ ವಿದ್ಯಾ ಪ್ರವಾಹದಲ್ಲಿ ಅವಗಾಹನ ಮಾಡಿ ವಿದ್ಯಾಫಲವನ್ನು ಪಡೆದ, ವಿದ್ಯಾತೀರ್ಥ ಪ್ರಾಶನಮಾಡಿದ, ಅದನ್ನು ಪ್ರೋಕ್ಷಿಸಿಕೊಂಡ ವಿದ್ಯಾರ್ಥಿಗಳು ಮತ್ತು ಸಮಾಜ ಆ ನದಿಗೆ ಕೃತಜ್ಞತಾ ಭಾವದ ಬಾಗಿನ ಅರ್ಪಿಸುವ ಸಂದರ್ಭ. ವಿದ್ಯಾರ್ಥಿಗಳ ಮೇಲಿನ ವಾತ್ಸಲ್ಯ, ಅಭಿಮಾನಗಳಿಂದ ದಂಪತಿಗಳು ಸಂತುಷ್ಟಾಂತರಂಗ\-ರಾಗಿ ವಿದ್ಯಾರ್ಥಿಗಳ ಮನೋ ಇಂಗಿತವನ್ನು ಸ್ವೀಕರಿಸಲು ಅನುವಾಗಿದ್ದರು. 

ಈ ಸಮಾರಂಭವನ್ನು ಸಂಪನ್ನಗೊಳಿಸಲು ರಾಜಮಾತೆ ಪ್ರಮೋದಾದೇವಿಯವ\-ರನ್ನು, ಡಾ । ಪಿ.ಎನ್.ಶಾಸ್ತ್ರಿಗಳನ್ನು ಮತ್ತೆಲ್ಲ ಗಣ್ಯರನ್ನು ಬರಮಾಡಲಾಯಿತು.ಎಲ್ಲ ವಿದ್ಯಾರ್ಥಿಗಳು ಶ್ರೀಯುತರ ಸಮೀಪದಲ್ಲಿ ಸಮಾವಿಷ್ಟರಾದರು. ಶ್ರೀಮಾನ್ ರಮೇಶ ಅಡಿಗರು ವಿದ್ಯಾರ್ಥಿಗಳೊಡನೆ ವೇದಘೋಷವನ್ನು ಆರಂಭಿಸಿದರು. ಭಾರತೀಯ ಸಂಪ್ರದಾಯದಲ್ಲಿ ಸ್ತ್ರೀಯರಿಗೆ ಪ್ರಥಮ ಪೂಜೆಯಷ್ಟೆ ! ಹಾಗಾಗಿ ಮೊದಲು ಶ್ರೀಯುತರ ಧರ್ಮಪತ್ನಿ ಶ್ರೀಮತಿ ಶೈಲಜಾರವರಿಗೆ  ಹರಿದ್ರಾಕುಂಕುಮ, ಫಲವಸನಗಳೊಂದಿಗೆ ಉಡಿತುಂಬುವ ಮೂಲಕ ಉಪಾಯನ(ಬಾಗಿನ)ವನ್ನು ಸಮರ್ಪಿಸಲು ಅವರ ಅತ್ತಿಗೆಯವರೇ ಆದ  ಶ್ರೀಮತಿ ವೇದಾವತಿಯವರನ್ನು ವಿನಂತಿಸಲಾಯಿತು. ವಿದ್ಯಾರ್ಥಿಗಳ ಪರವಾಗಿ ಅವರು ಪ್ರೀತಿಪೂರ್ವಕವಾಗಿ ಉಪಾಯನವನ್ನು ಶ್ರೀಮತಿಯವರಿಗೆ\break ಸಮರ್ಪಿಸಿದರು. ರಾಜಮಾತೆಯವರು ಮತ್ತು ಶಾಸ್ತ್ರಿಗಳು ಗಂಗಾಧರ ಭಟ್ಟರಿಗೆ\break ಮೈಸೂರಿನ ಪೇಟವನ್ನು ತೊಡಿಸಿ, ಶಾಲು ಹೊದಿಸಿ, ಹಾರವನ್ನು \hbox{ಸಮರ್ಪಿಸಿದರು}. ವಿದ್ಯಾರ್ಥಿಗಳು ಪರಮ ಆದರದಿಂದ ಸಮರ್ಪಿಸಲು ಉದ್ದೇಶಿಸಿದ್ದ ಬೆಳ್ಳಿಹರಿವಾಣ, ಪಂಚಪಾತ್ರೆ ಮತ್ತು ಉದ್ಧರಣೆಗಳನ್ನು ಸಮರ್ಪಿಸಲಾಯಿತು. ಗಂಗಾಧರ ಭಟ್ಟರು ಯಾರನ್ನು ಗುರುಗಳೆಂದು ಅತಿಶಯವಾದ ಗೌರವದಿಂದ ಭಾವಿಸಿದ್ದರೋ ಅಂತಹ ಮಹಾಮಹೋಪಾಧ್ಯಾಯ ಎನ್.ಎಸ್.ರಾಮಭದ್ರಾಚಾರ್ಯರ ಭಾವಚಿತ್ರವನ್ನು\break ಅಂತೆಯೇ ಅಭಿವಂದನ ಪತ್ರವನ್ನು ಸಮರ್ಪಿಸಲಾಯಿತು. ಹೀಗೆ ಅಭಿವಂದನ ಕಾರ್ಯಕ್ರಮ ಸಂಪನ್ನವಾಗಿ ಬಹುಜನರ ಬಹುದಿನಗಳ  ಕನಸು ನನಸಾಯಿತು. 
\vskip 8pt

ಇದಾದ ಅನಂತರ ವಿದ್ಯಾರ್ಥಿವೃಂದದಲ್ಲೊಬ್ಬರಾದ ಶ್ರೀ ವಿಜಯಕುಮಾರರವರು ಬಹುಪರಿಶ್ರಮದಿಂದ ಸಿದ್ಧಪಡಿಸಿದ, ಗಂಗಾಧರ ಭಟ್ಟರ ಬಗ್ಗೆ ಅನೆಕ ಮಹನೀಯರು ತಮ್ಮ ಅಭಿಪ್ರಾಯವನ್ನು ಹಂಚಿಕೊಂಡ ವೀಡಿಯೋ ಸಂಗ್ರಹದ \enginline{DVD} ಯನ್ನು ಬಿಡುಗಡೆ\-ಗೊಳಿಸಬೇಕು ಎಂದು ರಾಜಮಾತೆಯವರನ್ನು ಪ್ರಾರ್ಥಿಸಲಾಯಿತು. ಅವರು \enginline{DVD} ಯನ್ನು ಬಿಡುಗಡೆಗೊಳಿಸಿದರು. ಅದರಲ್ಲಿ ಸುಮಾರು ಒಂದು ಗಂಟೆಯ\break ವೀಡಿಯೋ ಇದ್ದು ಬಿಡುಗಡೆಯಾದ ತಕ್ಷಣ ಅದರ ಸಂಗ್ರಹ ರೂಪವನ್ನು ಸಭೆಯಲ್ಲಿ ತೋರಿಸಲಾಯಿತು. ಸಭಾ ಕಾರ್ಯಕ್ರಮ ಮುಗಿದ ಮೇಲೆ ಪೂರ್ಣವಾಗಿ ಪ್ರದರ್ಶನ ಮಾಡಲಾಯಿತು. ಅದು ಗಂಗಾಧರ ಭಟ್ಟರ ಅಣ್ಣ, ತಂಗಿಯರೊಡನೆ ಆಟವಾಡಿದ ಕಾಲದಿಂದ ಹಿಡಿದು ಪತ್ನಿ \enginline{-} ಶೈಲಜಾರವರ ವಿವಾಹದ ಸಂದರ್ಭದ ಅಭಿಪ್ರಾಯ,\break ಕುಟುಂಬದವರ ಸ್ವರಸವಾದ ವಿಷಯಗಳು, ಬಹಳ ಕಾಲದಿಂದ ಗಂಗಾಧರ ಭಟ್ಟರನ್ನು ಹತ್ತಿರದಿಂದ ಬಲ್ಲ ವಿವಿಧ ಕ್ಷೇತ್ರದ ವಿದ್ವಾಂಸರ, ಪ್ರೌಢ ಚಿಂತಕರ ಅಭಿಪ್ರಾಯಕಥನ\-ಗಳನ್ನು ಅದು ಒಳಗೊಂಡಿದ್ದು ಗಂಗಾಧರ ಭಟ್ಟರ ನಿಜವಾದ ಅಂತಸ್ಸಾಮರ್ಥ್ಯದ ವಿವಿಧ ಮುಖಗಳನ್ನು ಪರಿಚಯಿಸುವ ಕಾರ್ಯಕ್ರಮವಾಗಿ ಸಭೆಯನ್ನು ಬಹಳವಾಗಿಯೇ\break  ರಂಜಿಸಿತು.

ಇದಾದ ಅನಂತರ, ಒಬ್ಬ ಮಹನೀಯರನ್ನೇ ತಮ್ಮಿಬ್ಬರ ಗುರುಗಳೆಂದು\-ಕೊಂಡು ಗಂಗಾಧರ ಭಟ್ಟರೊಂದಿಗೆ ಒಟ್ಟಿಗೆ ಓದಿ\enginline{-}ಬೆಳೆದ, ಭಟ್ಟರ ಒಳಹೊರ ಬದುಕನ್ನು ಬಲ್ಲ ಬಹುಕಾಲದ ಒಡನಾಡಿಯಾದ ಶ್ರೀಉಮಾಕಂತ ಭಟ್ಟರನ್ನು ಮಾತಿಗೆ ವಿನಂತಿಸ\-ಲಾಯಿತು. ಸಭಾರಂಜಕವಾದ ಮಾತಿನಿಂದಲೇ ಪ್ರಸಿದ್ಧಿಹೊಂದಿದ ಉಮಾಕಾಂತ ಭಟ್ಟರು “ಮಾತನಾಡಲು ಸಾವಿರ ಮಾತುಗಳಿವೆ, ಆದರೆ ನಾನು ಸಾವಿರದ ಕೆಲವು\break ಮಾತುಗಳನ್ನಾಡಬಯಸುತ್ತೇನೆ” ಎಂದೇ ಆರಂಭಿಸಿದರು. ಈ ಪಾಠಶಾಲೆ ದಿವ್ಯತೆಯ ಸ್ಥಳ. ವಿದ್ಯಾಗಣಪತಿಯ ಸನ್ನಿಧಾನವಿಲ್ಲಿದೆ. ಒಂದು ಬಗೆಯ ತೀರ್ಥಕ್ಷೇತ್ರವಿದು.\break ತೀರ್ಥವೆಂದರೆ ಶಾಸ್ತ್ರ. ನಾವು ಪಡೆದುಕೊಂಡು ನಮ್ಮೊಳಗೆ ತುಂಬಿಕೊಳ್ಳುವ ಶಾಸ್ತ್ರ \enginline{-} ಅದು ತೀರ್ಥ. ಶಾಸ್ತ್ರವೆಂಬ ತೀರ್ಥವನ್ನು ಶಿಷ್ಯರ ಅಂತರಂಗಕ್ಕೆ ತಲುಪಿಸುವ ಗುರು \enginline{-} ಅವನೂ ತೀರ್ಥ. ಇದೆಲ್ಲ ಇಲ್ಲಿರುವುದರಿಂದ ಇದು ತೀರ್ಥಕ್ಷೇತ್ರ. ಇಂತಹ\break ತೀರ್ಥ ಕ್ಷೇತ್ರದಲ್ಲಿ ನಾವು ಮಿಂದವರು, ಮುಳುಗಿದವರು, ಎದ್ದವರು, ಗೆದ್ದವರು. ನಮ್ಮ ಬದುಕಿನ ತೀರ್ಥ\-ಯಾತ್ರೆಯನ್ನು ಇದರ ಅನುಭವದ ನೆರವಿನಿಂದ ನೆರವೇರಿಸಿಕೊಂಡು ಹೋಗುತ್ತಿರು\-ವವರು. ಇಂತಹ ಕ್ಷೇತ್ರ \enginline{-}  ಇದು ಸರಸ್ವತೀ ಪ್ರಾಸಾದ. ಇಂತಹ ಪ್ರಾಸಾದವನ್ನು ನಿರ್ಮಿಸಿ ನಮಗೆಲ್ಲ ಆಶ್ರಯಕೊಟ್ಟವರು ಮಹಾರಾಜರು. ಅವರಿಗೆ ನಾವು ಎಷ್ಟು ಕೃತಜ್ಞ\-ರಾಗಿದ್ದರೂ ಸಾಲದು.

ಗಂಗಾಧರ ಭಟ್ಟರು ಶುದ್ಧ ಮಲೆನಾಡಿನಿಂದ ಬಂದವರು. ಈ ಸರಸ್ವತಿ ಪ್ರಾಸಾದಕ್ಕೆ ಬಂದು ಸರಸ್ವತಿಯ ಪ್ರಸಾದವನ್ನು ತಾವು ಪಡೆದು ಅದನ್ನು ವಿದ್ಯಾರ್ಥಿಗಳಿಗೆ ವಿತರಿಸುವ ಕಾರ್ಯವನ್ನು ಇಲ್ಲೇ ಮಾಡಿ ಒಬ್ಬ ಉದಾತ್ತ ಶಿಕ್ಷಕರು, ಆದರ್ಶ ಆಚಾರ್ಯರು ಎನ್ನಿಸಿಕೊಂಡರು. ಮಹಾಮಹೋಪಾಧ್ಯಾಯ ಎನ್.ಎಸ್.ರಾಮಭದ್ರಾಚಾರ್ಯರು ನಮಗಿಬ್ಬರಿಗೂ ಗುರುಗಳು. ಅವರ ವಿದ್ಯಾಪರಂಪರೆಯಲ್ಲಿ  ಬಂದವರು ನಾವು. ಅವರ ಅಂತರಂಗದಲ್ಲಿ ತುಂಬಿದ ಪ್ರೀತಿ ನಮ್ಮಂತಹ ಅನೇಕ ಮೂರ್ತಿಗಳಲ್ಲಿ ತುಂಬಿ ಅಮೂರ್ತ\-ವಾದ ವಿದ್ಯೆಯನ್ನು ಬೆಳಗಿಸುತ್ತಿದೆ. ಈ ಪ್ರಾಸಾದವನ್ನು ತುಂಬಿಸಿದ್ದು ಆ ಆಚಾರ್ಯ ಪ್ರೀತಿ,  ಗಂಗಾಧರ ಭಟ್ಟರ ಮನೆಯನ್ನು ತುಂಬಿಸಿದ್ದು ಆ ಪ್ರೀತಿ. ನಮ್ಮ ನಮ್ಮ ಮನೆ, ಮನಗಳನ್ನು ತುಂಬಿದ್ದು ಆ ಪ್ರೀತಿ. ಅದು ವಿದ್ಯಾಪ್ರೀತಿ. ಆ ಪ್ರೀತಿ ಇಂದು ಈ ಕಾರ್ಯಕ್ರಮದ ವರೆಗೆ ನಮ್ಮನ್ನು ತಂದು ನಿಲ್ಲಿಸಿದೆ. ಈ ಹಿಂದೆಯೂ ಈಗಲೂ ಇದಕ್ಕೆ ಸಾಕ್ಷಿಯಾದವರು ಅನೇಕರಿದ್ದಾರೆ. ಆದರೆ ಇಂದು ರಾಜಮಾತಾ\enginline{-}ರಾಜಸತ್ತೆಯೂ ಅದಕ್ಕೆ ಸಾಕ್ಷಿಯಾದುದು ನಮಗೆಲ್ಲ ಆ ಪ್ರೀತಿ ಇಮ್ಮಡಿಯಾಗುವಂತೆ ಮಾಡಿದೆ.

ಗಂಗಾಧರ ಭಟ್ಟರ ಪಾಠಪ್ರವಚನಗಳು ಅತ್ಯಂತ ಬೆಲೆಯುಳ್ಳವು, ಅಂದರೆ ಅವರು ಪಂಕ್ತಿಯ ಅರ್ಥ ಹೆಳಿದ್ದಾರೆ ಎಂದಲ್ಲ.  ಅದೊಂದು ಆದರ್ಶಜೀವನದ ಪದ\enginline{-}ಪದದ ಶಿಕ್ಷಣ. ಅವರು ಆರಂಭದಿಂದಲೇ ಸ್ವಾಲಂಬನೆಯೇ ಸ್ವಭಾವವಾಗಿ ಬೆಳೆದವರು.\break ಸ್ವಾವಲಂಬನೆ ಮತ್ತು ಸ್ವಾಭಿಮಾನ ಅವರ ಬದುಕಿನ ಎರಡು ಪ್ರಕಾರಗಳು. ಅದನ್ನು ಬೇರೆಯವರು ಸುಲಭವಾಗಿ ಅನುಕರಿಸಲಾಗದು. ೧೯೭೫ರಿಂದ ನಾನು ಗಂಗಾಧರ ಒಡನಾಡಿಗಳು. ಕೆಲವರು ನಮ್ಮನ್ನು ಒಳನಾಡಿಗಳು ಎನ್ನುತ್ತಾರೆ. ಅದು ಸರಿ. ನಮ್ಮಿಬ್ಬರ ನಾಡಿ ನಮ್ಮಿಬ್ಬರಿಗೂ ಗೊತ್ತು. ನಾವು ಅಕ್ಕ ಪಕ್ಕದ ಕೋಣೆಯಲ್ಲಿದ್ದರೆ ಒಂದೇ ಮಾತು, ಒಂದೇ ಶಬ್ದ, ಒಂದೇ ಪ್ರಯೋಗವನ್ನು ಮಾಡಬಲ್ಲೆವು. ನಬ್ಬಿಬ್ಬರಲ್ಲಿ ಯಾವತ್ತೂ\break ಸ್ಫರ್ಧೆಯಿದೆ, ಆದರೆ ಜಗಳವಿಲ್ಲ. ೧೯೭೫ \enginline{-}೭೬ರ ಸಾಲಿನಿಂದ ಕನಿಷ್ಠ ಆರೇಳು ವರ್ಷ ಮೈಸೂರಿನ ಎಲ್ಲ ಚರ್ಚಾಸ್ಪರ್ಧೆಗಳಲ್ಲಿ ಪ್ರಥಮ ದ್ವಿತೀಯ ಬಹುಮಾನವನ್ನು ನಾವು ಪಡೆದು\-ಕೊಳ್ಳುತ್ತಿದ್ದೆವು.

ಗಂಗಾಧಾರ ಭಟ್ಟರು ಇಷ್ಟು ಕಾಲ ಮೈಸೂರಿನಲ್ಲಿದ್ದೂ ಒಂದು ಸೈಟು\enginline{-}ಮನೆ ಮಾಡಿದವರಲ್ಲ. ಮೈಸೂರಿನಲ್ಲಿ ಸೈಟು ಮನೆ ಮಾಡಿದವರಿಗೆ ಇಷ್ಟು ಜನರ ಹೃದಯದಲ್ಲಿ ತಮ್ಮ ಪ್ರೀತಿಯನ್ನು ಹಂಚುವ ಅವಕಾಶವಿಲ್ಲ. ಆದ್ದರಿಂದ ಅವರ ಸೈಟು\enginline{-}ಮನೆ ಎಲ್ಲಿದೆ ಅಂದರೆ ವಿದ್ಯಾರ್ಥಿಗಳ ಮನದಲ್ಲಿದೆ. ಅವರ ಅಭಿಮಾನಿಗಳ ಧಮನಿಯಲ್ಲಿದೆ. ಗಂಗಾಧರ ಭಟ್ಟರ ಪ್ರೀತಿ, ಆದರೋಪಚಾರಗಳನ್ನು ವಿಶೇಷವಾಗಿ ಜ್ಞಾಪಿಸಿಕೊಳ್ಳಬೇಕು. ಲೆಕ್ಖ ಇಟ್ಟಿದ್ದರೆ ೧೦,೦೦೦ಕ್ಕಿಂತ ಹೆಚ್ಚು ಜನ ಅವರ ಆತಿಥ್ಯದ ಪ್ರಯೋಜನ ಪಡೆದುಕೊಂಡವರಿರಬಹುದು. ಗಂಗಾಧರ ಭಟ್ಟರಂತೆಯೇ ಅವರ ಹೆಂಡತಿ ಅವರ ತಂಗಿಯರು ಅಣ್ಣ\enginline{-}ತಮ್ಮಂದಿರು ಎಲ್ಲರೂ ಆತಿಥ್ಯದ ಸ್ವಭಾವದವರು. ಮನೆಯಲ್ಲಿ ಹಾಲಿಲ್ಲದಿದ್ದರೆ\break ಪಕ್ಕದಬೀದಿಯಲ್ಲಿ ಹಸುವನ್ನು ಕರೆಸಿಕೊಂಡು ಬಂದು ಪಾನೀಯವನ್ನು ಕೊಟ್ಟಿರು\-ವುದೂ ಉಂಟು. ನಮ್ಮಲ್ಲಿ ಈ ಮಟ್ಟದ ಉಪಚಾರ ಮಾಡುವವರು ಯಾರಿದ್ದಾರೆ. ಅವರ ಮನೆಗೆ ಬಂದು ಆತಿಥ್ಯ ಸ್ವೀಕರಿಸದೇ ಹೋದವರು ಯಾರೂ ಇಲ್ಲ. ಆರಂಭದ ಕಾಲದಲ್ಲಂತೂ ಅವರಿಗೆ ಸಂಬಳ ಇನ್ನೂರು ಮುನ್ನೂರಕ್ಕಿಂತ ಹೆಚ್ಚಿರಲಿಕ್ಕಿಲ್ಲ. ಅಂಥಾದ್ದರಲ್ಲಿ ಈ ಮಟ್ಟದ ಉಪಚಾರ ಅವರ ಹೇಗೆ ಸಾಧ್ಯವಾಗುತ್ತಿತ್ತು ಎಂಬುದು ಆಶ್ಚರ್ಯವಲ್ಲದೇ ಮತ್ತೇನು !! ಆದ್ದರಿಂದ ಆಗಾಗ ನಾನು ಹೇಳುತ್ತೇನೆ ಗಂಗಾಧರ ಭಟ್ಟರ ಬದುಕು ಅದು \hbox{ಕಾದಂಬರಿಗೆ} ವಸ್ತುವಾಗಬಹುದು. ಅವರ ಬಾಳ್ವೆ ಕಾದಂಬರಿಯ ವಾಸ್ತುವನ್ನು ಪದೆದುಕೊಂಡಿದೆ. ಅದರ ಅರ್ಥ ಕಲ್ಪನೆಯೆಂದಲ್ಲ. ಆದರೆ ಕವಿಗಳು ಏನನ್ನು ಕಲ್ಪಿಸಬಹುದೋ ಅದನ್ನು ಬಾಳಿ ತೋರಿಸಿದ್ದಾರೆ.

ಭಾರತಕ್ಕೆ ತುರ್ತುಪರಿಸ್ಥಿತಿಯ ಕಾಲ \enginline{-} ಆಗ ಮಾತಿನ, ಅಭಿವ್ಯಕ್ತಿಯ ಹಕ್ಕನ್ನು ಸರ್ಕಾರ ಕಸಿದುಕೊಂಡಿತ್ತು.  ಅದನ್ನು ಹಿಂಪಡೆದ ಕೂಡಲೇ ಅಲ್ಲಿಯವರೆಗೆ ಮಡುಗಟ್ಟಿದ್ದ ಮೌನ ಎಲ್ಲರ ಬಾಯಲ್ಲೂ ಆ ಶಾಸನದ ವಿರುದ್ಧವಾಗಿ ಮಾತನ್ನಾಡಿಸುತ್ತಿತ್ತು. ಆಗ ಮೈಸೂರಿನ ಟೌನ್ ಹಾಲ್ ನಲ್ಲಿ ಲಕ್ಷಾಂತರ ಜನ ಸೇರಿದ್ದರು. ಅಂದು ಪ್ರಸಿದ್ಧರಾಗಿದ್ದ ಶಾಂತಿ\break ಭೂಷಣರವರು ಅಲ್ಲಿಗೆ ಬಂದು ಮಾತನಾಡಬೇಕಿತ್ತು. ಅವರು ಬರುವುದಕ್ಕೆ ಸ್ವಲ್ಪ ತಡವಾಯಿತು. ಜನರ ಮಧ್ಯೇ ಕೋಲಾಹಲವಾಗುತ್ತಿತ್ತು. ಆ ಕ್ಷಣದಲ್ಲಿ ಲಕ್ಷಾಂತರ ಜನರನ್ನು ನಿಭಾಯಿಸಲು ಒಬ್ಬ ಮಾತುಗಾರ ಬೇಕಿತ್ತು. ಆ ಮಾತುಗಾರನಾಗಿದ್ದು ಮತ್ತಾರೂ ಅಲ್ಲ ಇಪ್ಪತ್ತು ವರ್ಷದ ನಮ್ಮ ಗಂಗಾಧರ. ಸಂಸ್ಕೃತದವರು ಹೀಗೆಲ್ಲ ಸಾರ್ವಜನಿಕ\-ವಾಗಿ ಮಾತನಾಡುವು\-ದೆಂಬುದಿಲ್ಲ. ಮಾರನೇ ದಿನ ಪಾಠಶಾಲೆಯಲ್ಲಿ ಎಲ್ಲರ ಬಾಯಲ್ಲೂ ಗಂಗಾಧರನ ಬಗ್ಗೆ ಮಾತೇ ಮಾತು. ಇತರ ಅಧ್ಯಾಪಕರು ಗಂಗಾಧರನ ಅಧ್ಯಾಪಕರಲ್ಲಿ ಬಂದು, ಏನ್ರೀ !! ನಿಮ್ಮ ಗಂಗಾಧರ ಭಟ್ಟ \enginline{-} ಸಿಂಹದ ಶಿಶು \enginline{-} ಹೇಗೆ ಅಲ್ಲಿ ಮಾತನಾಡಿ\-ಬಿಟ್ಟ !!” ಎಂದೆಲ್ಲ ಮೂಗಿನ ಮೇಲೆ ಬೆರಳಿಟ್ಟು ಆಶ್ಚರ್ಯದಿಂದ ಬೆರಗಾಗಿ ಮಾತನಾಡಿದ್ದರು. ಅಂದು ಮೈಸೂರು ಮಾತ್ರವಲ್ಲದೇ ಕರ್ನಾಟಕ್ಕೇ ಇವನು ಪರಿಚಿತನಾದ ಅಲ್ಲದೇ ಶಾಂತಿಭೂಷಣರಿಗೆ  ಮಾತನಾಡಲು ಸರಿಯಾದ ಭೂಮಿಕೆ ಹಾಕಿ ವೇದಿಕೆ ನಿರ್ಮಾಣ ಮಾಡಿಕೊಟ್ಟ. ಗಂಗಾಧರ, ವೇದಿಕೆಯನ್ನು ಬಳಸಿಕೊಂಡ ಮಾತುಗಾರ ಮಾತ್ರನಲ್ಲ ಗಂಗಾಧರ. ವೇದಿಕೆಯನ್ನು ನಿರ್ಮಾಣವನ್ನೂ ಮಾಡಿದವ. 

ಉಮಾಕಾಂತ ಭಟ್ಟರ ವಾಗ್ ಝರಿ  ಹೀಗೆ ಮುಂದುವರೆದು ಗಂಗಾಧರ ಭಟ್ಟರು ಹೇಗೆ ಕುಟುಂಬದ ಕಷ್ಟವನ್ನೆಲ್ಲ ಧಾರಣೆಮಾಡಿದವರು. ಸಮಾಜಕ್ಕೆ ಯಾವೆಲ್ಲ ರೀತಿಯಲ್ಲಿ ಉಪಕಾರವನ್ನು ಮಾಡಿದವರು ಎಂಬುದನ್ನು ಸ್ವಾರಸ್ಯವಾಗಿ ವಿವರಿಸಿ ಅಂತಿಮವಾಗಿ, \enginline{-} ಪಾಠವನ್ನು, ಪಂಕ್ತಿಯ ಅರ್ಥವನ್ನು ಒಬ್ಬರಿಗಿಂತ ಇನ್ನೊಬ್ಬರು  ಚೆನ್ನಾಗಿ ಹೇಳಬಹುದು. ಅದರೆ ಬದುಕಿನ ಪಂಕ್ತಿಯಲ್ಲಿ ನಿಜವಾದ ಅರ್ಥ ಪದಾರ್ಥವಲ್ಲ ಅದು ಪುರುಷಾರ್ಥ. ಅಂತಹ ಪುರುಷಾರ್ಥವನ್ನು ತೋರಿಸುವ ಆದರ್ಶಜೀವನವನ್ನು ಮೈಸೂರಿನಲ್ಲಿ\break ಮಾಡುತ್ತಾ ಮೈಸೂರಿನ ಶ್ರೀಗಂಧಕ್ಕೆ ಹೊಸದಾದ ಶೈತ್ಯವನ್ನು ಪರಿಮಳವನ್ನು ತಂದು\-ಕೊಟ್ಟ ಗಂಗಾಧರ ನಮಗೆ ಯಾವಾಗಲೂ ಅಭಿನಂದನೀಯ ಅಭಿವಂದನೀಯ ಎಂದು ತಮ್ಮ ಮಾತನ್ನು ಮುಗಿಸಿದರು. ಗಂಗಾಧರ ಭಟ್ಟರ ಬಗೆಗಿನ ಇವರ ಮಾತು ಸಭೆಯನ್ನು ಅತಿಶಯವಾಗಿ ಪ್ರಭಾವಗೊಳಿಸಿತು. ಇವರ ಮಾತಿಗೆ ಸಭೆ ಮಧ್ಯೇ ಮಧ್ಯೇ ಕರತಾಡನದಿಂದ ಪ್ರಚೋದಿಸಿತು.

ಅದಾದ ಅನಂತರ ಉದ್ಘೋಷಕರು ಕಾರ್ಯಕ್ರಮದ ಕೇಂದ್ರ ಶ್ರೀಮಾನ್ ಗಂಗಾಧರ ಭಟ್ಟರನ್ನು ಮಾತಿಗಾಗಿ ವಿನಂತಿಸಿದರು. ತಕ್ಷಣ ಶ್ರೀಯುತರು ಧೀರ\enginline{-}ಗಂಭೀರ ಧ್ವನಿಯಲ್ಲಿ ಸಭಾಕರ್ಷಕವಾಗಿ “ರಾಮಾಯ ರಾಮಭದ್ರಾಯ ರಾಮಂಚಂದ್ರಾಯ” ಎಂಬ ಶ್ಲೋಕದೊಡನೆ ಪ್ರವಚನವನ್ನು ಆರಂಭಿಸಿದರು. ಸಾಮಾನ್ಯವಾಗಿ ಅವರು ಶ್ರೀರಾಮ\-ನನ್ನೂ ತಮ್ಮ ಗುರುಗಳಾದ ರಾಮಭದ್ರರನ್ನೂ ಮನಸ್ಸಿನಲ್ಲಿಟ್ಟುಕೊಂಡು ಈ ಶ್ಲೋಕವನ್ನು ಹೇಳುತ್ತಾರೆ. ಒಂದು ಕ್ರಿಯೆಯಿಂದ ಅನೇಕ ಫಲ ಸಾಧಿಸುವುದು ಅವರ ಮಾತು ಕೃತಿ ಎರಡರಲ್ಲೂ ನಮಗೆ ಆಗಾಗ ಕಂಡುಬರುವ ಅಂಶ. ಅವರು ಮಾತನ್ನು ಆರಂಭಿಸಿದರು \enginline{-} “ನನಗೆ ಹಿನ್ನೆಲೆ ರಾಮಭದ್ರಾಚಾರ್ಯರು. ಆ ದಂಪತಿಗಳಿಗೆ ಇಲ್ಲಿಂದಲೇ ನಮಸ್ಕರಿಸುತ್ತೇನೆ. ನನ್ನ ಬೌದ್ಧಿಕವಾದ ಯಾವುದೇ ವ್ಯವಹಾರ ಇದ್ದರೆ ಅದನ್ನು ರೂಪಿಸಿದವರು ನನ್ನ ಗುರು ರಾಮಭದ್ರಾಚಾರ್ಯರು. 

ಈಗಷ್ಟೆ ಉಮಾಕಾಂತ ಮಾತನಾಡುತ್ತಾ ಹೇಳಿದ, “ಇಂದು ಗಂಗಾಧರ ನನ್ನ ಮಾತು ಮರೆಸುತ್ತಾನೆ” ಎಂದು. ಆದರೆ ಇಂದು ನಾನು ಎಲ್ಲವನ್ನೂ ಮರೆತಿದ್ದೇನೆ. ಕಾರಣ ಇಷ್ಟೆ, ನಾನು ನನ್ನ ಕರ್ತವ್ಯ ಮಾಡಿದ್ದೇನೆ. ಆದರೆ ಇಪ್ಪತ್ತು ವರ್ಷದ ನನ್ನ ಸರ್ವಿಸ್ ನಲ್ಲಿ ಅಂದು ಪಾಠ ಕೇಳಿದ ನಮ್ಮ ಮಕ್ಕಳು ಆನಂದದಿಂದ ಇನ್ನೂ ನನ್ನನ್ನ ನೆನಪಿಟ್ಟುಕೊಂಡಿದ್ದಾರಲ್ಲ !! ನನಗೆ ಅದೇ ಸಮಾಧಾನಕೊಡುವ ಸಂಗತಿ. ಇಂದು ಆದ ಆ ಆನಂದ ನನ್ನನ್ನು ಮೂಕನನ್ನಾಗಿಸಿದೆ. ಏಕೆಂದರೆ ವಿದ್ಯಾರ್ಥಿಗಳಿಗೆ ನಿಮ್ಮ ಗುರುಗಳು ಯಾರು ಎಂದು ಕೇಳಿದರೆ ಅನೇಕರು ಗಾಬರಿಯಾಗಿಬಿಡುತ್ತಾರೆ. ಅಂತೆಯೇ ಶಿಕ್ಷಕರಿಗೆ ನಿಮ್ಮ ಶಿಷ್ಯರು ಯಾರು ಎಂದರೆ ಗಾಬರಿಯಾಗತ್ತೆ. ಯಾರೂ ಜ್ಞಾಪಕಕ್ಕೆ ಬರುವುದಿಲ್ಲ. ಹೀಗೆ ಆಧುನಿಕ ಭಾರತ\-ದಲ್ಲಿ ಗುರುಶಿಷ್ಯರಿಬ್ಬರೂ ಪರಿಚಯವೇ ಇಲ್ಲದೇ ವ್ಯವಹಾರ ನಡೆಯುತ್ತಿರುವುದು\break ಶೋಚನೀಯ. ಇದು ನಿಜವಾದ ಶೈಕ್ಷಣಿಕ ಪರಂಪರೆ ಅಲ್ಲ. ಗುರುಶಿಷ್ಯ ಪರಂಪರೆ ದಿವ್ಯ\-ವಾಗಿ ನಡೆಯಬೇಕು. ಅದನ್ನು ಬಲ್ಲದ್ದು ರಾಜಸತ್ತೆ. ಆ ರಾಜಸತ್ತೆಗೆ ಆದರ್ಶ\-ವಾದದ್ದು ಮೈಸೂರು ಸಂಸ್ಥಾನ. ಎಲ್ಲರೂ ಸಂಸ್ಥೆಯನ್ನು ಸ್ಥಾಪಿಸಬಹುದು. ಆದರೆ ಉದ್ದೇಶ ಏನಿರ\-ಬೇಕು? ಅಖಂಡ ವಿಶ್ವದಲ್ಲಿ ಅಭೂತಪೂರ್ವವಾದುದನ್ನು ಕಟ್ಟುವ ಕೆಚ್ಚು ನಮ್ಮ ಅರಮನೆ\-ಯವರಿಗಿತ್ತು. ಅದಕ್ಕೆ  ಅರಮನೆಯ ನೆರೆಮನೆಯೇ ನಮ್ಮ ಪಾಠಶಾಲೆ\-ಯಾಯಿತು. ಅದಕ್ಕಾಗಿಯೇ ವಿದ್ಯಾರ್ಥಿಗಳು ಈ ರೀತಿಯಾಗಿ ಕಾರ್ಯಕ್ರಮ ಮಾಡಬೇಕು ಎಂದಾಗ ನಾನು ಹೇಳಿದೆ \enginline{-} ನಮಗೂ ನಿಮಗೂ ಆಶ್ರಯ ಕೊಟ್ಟಿದ್ದು ಮಹಾರಾಜರ ಔದಾರ್ಯ(ಪಾಠಶಾಲೆ). ನಾವು ಎಂದೋ ಅವರನ್ನ ಕರೆದು\break ಕೃತಜ್ಞತೆಯನ್ನು ನಾವು (ಪಾಠಶಾಲೆಯ ಕಡೆಯಿಂದ)  ಹೇಳಬೇಕಾಗಿತ್ತು. ಹೇಳಿಲ್ಲ. ಅದು ಕೃತಘ್ನತೆಯ ಪರಮಾವಧಿ. ಆದ್ದರಿಂದ ಅರಮನೆಯಿಂದ ಯಾರಾದರೂ ಈ ಕಾರ್ಯಕ್ರಮಕ್ಕೆ ಬರುತ್ತೇವೆ ಎಂದು ಒಪ್ಪುವುದಾದರೆ ನಾನೂ ಬರುತ್ತೇನೆ. ಕಾರಣ ನಾನು ಕೃತಜ್ಞತೆಯನ್ನು ಹೇಳಿ ಋಣವನ್ನು ವ್ಯಕ್ತಪಡಿಸಬೇಕಿದೆ. ಇದನ್ನು \hbox{ಅರಮನೆಯ} ಯಾವುದಾದರೂ ಲಾಭದ ನಿರೀಕ್ಷೆಯಿಂದ ಹೇಳುತ್ತಿಲ್ಲ. ಖಂಡಿತವಾಗಿಯೂ ಈ ತನು, ಮನ, ಬುದ್ಧಿ ಎಲ್ಲವೂ ರೂಪಿತವಾಗುವುದಕ್ಕೆ ಸನ್ನಿವೇಶವನ್ನು ಕೊಟ್ಟ ಅವರಿಂದ ನಿರ್ಮಾಣಗೊಂಡ ನಿವೇಶನ ಇಲ್ಲಿದೆ. ಆದ್ದರಿಂದ ನಾವು ಮೂಲವನ್ನು ಬರೆಯ\-ಬಾರದು. ನಾವು ಎಲ್ಲ ಕಷ್ಟಗಳಿಗೆ ಅರಮನೆಗೆ ಮುಖಮಾಡುತ್ತೇವೆ. ಆದರೆ ಅವರ ಕಷ್ಟ ಏನೆನ್ನುವುದು ಅವರಿಗೆ ಮಾತ್ರ ಗೊತ್ತು. ನಾನು ಅವರ ಸಂಪರ್ಕದಲ್ಲಿಲ್ಲ. ಆದರೆ ಧೀರ ಮಾತೆ ದುಷ್ಟರ ಸಂಚಿನಿಂದ ಉಂಟಾದ ಕಷ್ಟಗಳನ್ನೆಲ್ಲ ಎದುರಿಸಿದ್ದು  ರಕ್ಷಣೆ ಮಾಡಬೇಕಾದ್ದನ್ನು ಹೇಗೆ ರಕ್ಷಿಸುತ್ತಿದ್ದಾರೆ ಎಂಬುದನ್ನು ಬಲ್ಲೆ. ಅದನ್ನು ನೋಡಿದಾಗ ಎದೆ ತುಂಬಿ ಬರುತ್ತದೆ. ಈಗ ಪ್ರಜಾಪ್ರಭುತ್ವವಿದ್ದರೂ ರಾಜರಿಂದ ಉಪಕೃತರಾದ  ಮೈಸೂರಿನ ಜನರು ಕೃತಜ್ಞರಾಗಿರಬೇಕಾದ್ದು ಕರ್ತವ್ಯವೆಂಬುದು ನನ್ನ ವಾದ. ನಾನು ಪ್ರಜಾಪ್ರಭುತ್ವದ\break ವಿರೋಧಿಯಾಗಿದ್ದೇನೆಂದು  ಇದರ ಅರ್ಥವಲ್ಲ. ಅರಮನೆ ಕೊಟ್ಟ ಕೊಡುಗೆಯನ್ನು\break ಕಳೆದರೆ ಉಳಿಯುವುದೇನಿದೆ ಎನ್ನುವುದನ್ನು  ಅರ್ಥಮಾಡಿಕೊಂಡರೆ ಸಾಕು. ಮೈಸೂರಿನ ಚಿತ್ರ ಬಿಡಿಸಿ ಅರಸರು ಕೊಟ್ಟಿರುವುದನ್ನು ಕಳೆಯುತ್ತ ಬಂದರೆ ಆಗ ಈ ವಿಷಯ\break ಸ್ಪಷ್ಟ\-ವಾಗುತ್ತದೆ. 

ನಾನು ಮೂಲತಃ ಒಬ್ಬ ಚಿಕ್ಷಕ. ಅದರ ಫಲವಾಗಿ ನಮ್ಮ ಮಕ್ಕಳು ವಿದ್ಯಾರ್ಥಿಗಳು ಕಾರ್ಯಕ್ರಮ ಮಾಡಿದ್ದಾರೆ. ಅವರು ಚೆನ್ನಾಗಿ ತರ್ಕ ಓದಿದ್ದಾರೆ. ಈಗಿನ ಕಾಲದಲ್ಲಿ ಪಾಠಮಾಡುತ್ತೇನೆಂದರೂ ಕೇಳುವವರಿಲ್ಲ. ಅದರಲ್ಲೂ ತರ್ಕ ಎಂದರೆ ಓಡಿಹೋಗು\-ತ್ತಾರೆ. ಆದರೆ ನಮ್ಮ ಮಕ್ಕಳು ಇಷ್ಟು ಪರಿಶ್ರಮ ಮಾಡಿರುವುದು ಸಂತದ ಸಂಗತಿ. ಏಕೆ ಇದನ್ನೆಲ್ಲ ಮಾಡಿದ್ದು ? ಸಂಸ್ಕೃತಿಯನ್ನು ರಕ್ಷಿಸುವ ಮನಸ್ಸಿನಿಂದ ಅವರು ಇದನ್ನು ಮಾಡುತ್ತಿದ್ದಾರೆ. ನಾವು ಏನೋ ಸ್ವಲ್ಪ ಹೇಳಿಕೊಟ್ಟಿರುವುದನ್ನು ಹುಡುಗರು ಚೆನ್ನಾಗಿ\break ಬೆಳೆಸಿಕೊಂಡಿದ್ದಾರೆ. ಉಮಾಕಾಂತ ಹೇಳಿದಂತೆ ಅವರೆಲ್ಲ ಪ್ರೀತಿಯಿಂದ ಅದನ್ನು ತುಂಬಿಕೊಂಡಿದ್ದಾರೆ. ಅದನ್ನು ಇಂದು ತುಂಬುತ್ತಿದ್ದಾರೆ. ನನ್ನ ಮನೆಯವರು ನನ್ನ ಬಗ್ಗೆ ಆಗಾಗ ಆಕ್ಷೇಪಿಸಿದ್ದುಂಟು. ನಿಮಗೆ ವಿದ್ಯಾರ್ಥಿಗಳಿದ್ದರೆ ಮತ್ತೆ ಏನೂ ಬೇಡ ಎಂಬುದಾಗಿ. ಅದು ಹೌದು, ವಿದ್ಯಾರ್ಥಿಗಳು ಬಂದರೆ ನಾನು ಎಲ್ಲವನ್ನೂ ಮರೆಯುತ್ತೇನೆ. ನಾನು ಶಿಕ್ಷಕ. ವಿದ್ಯಾರ್ಥಿಗಳಿಲ್ಲದೇ ನಾನು ಶಿಕ್ಷಕನಾಗುವುದಾದರೂ ಹೇಗೆ ? ಆದ್ದರಿಂದ ನಾನು ವಿದ್ಯಾರ್ಥಿಗಳನ್ನು ಪ್ರೀತಿಸಿದ್ದು ಎನ್ನುವುದಕ್ಕಿಂತ ನಾನು ಪ್ರೀತಿಸಿದ್ದು ನನ್ನನ್ನೇ. (ಇಲ್ಲಿ ಶ್ರೀಯುತರು ಸಂಪೂರ್ಣ ಭಾವುಕರಾಗಿದ್ದಾರೆ) ನಮ್ಮ ಮಕ್ಕಳು ನಾನು ಕೊಟ್ಟಿದ್ದನ್ನು ಸಾವಿರ ಪಾಲು (ಮಾಡಿ) ಕೊಟ್ಟಿದ್ದಾರೆ. ಅಂದರೆ ಸಾವಿರ ಪಾಲು ಮಾಡಿ \hbox{ಕೊಟ್ಟಿದ್ದಲ್ಲ}. ಸಾವಿರ ಪಾಲು ಹೆಚ್ಚಿಸಿ ಕೊಟ್ಟಿದ್ದಾರೆ.(ಓದುಗರು ಇಲ್ಲೆಲ್ಲ ಶ್ಲೇಷವನ್ನು ಗ್ರಹಿಸಿಕೊಳ್ಳ\-ಬೇಕು) ಕೊಟ್ಟಿದ್ದನ್ನು ಸಾವಿರ ಪಾಲು ಮಾಡುವವರೇ ಹೆಚ್ಚು. ಆದರೆ ಇವರು ಮಾಡಿದ್ದು ಹೆಚ್ಚು. ಎಲ್ಲರ ಪ್ರೀತಿ ವಿಶ್ವಾಸ ಹೀಗೇ ಇರಲಿ. ನನ್ನ ಕೊನೆಯ ಉಸಿರಿರುವ\-ವರೆಗೂ ಯಾವುದೇ ವಿಷಯವನ್ನು ಯಾರೇ ಕೇಳಿದರೂ ಗೊತ್ತಿದ್ದರೆ ಆಗಲೇ ಹೇಳುತ್ತೇನೆ. ಇಲ್ಲದಿದ್ದರೆ ನೀವು ನನ್ನನ್ನ ತಿಳಿದುಕೊಳ್ಳುವುದಕ್ಕೆ ಪ್ರೇರೇಪಿಸುತ್ತಿದ್ದೀರಿ ಎಂದುಕೊಳ್ಳುತ್ತೇನೆ. ಜಗತ್ತಿನಲ್ಲಿ ಜಿಜ್ಞಾಸುಗಳು ಜಾಸ್ತಿ ಆಗಲಿ. 
\vskip 4pt

ಈಗ ನಾನು ನಿವೃತ್ತನಾಗುತ್ತಿದ್ದೇನೆ. ಈ ಕಾರ್ಯಕ್ರಮಕ್ಕೆ ನನ್ನ ಮೇಲಿನ ಅಭಿಮಾನದಿಂದ ತಾವೆಲ್ಲ  ಬಂದಿದ್ದೀರಿ ಇಲ್ಲಿರುವ ಅಧ್ಯಕ್ಷರು ನಿರಂಜನ ವಾನಳ್ಳಿಯವರು ನನಗೆ ಆರಂಭದಿಂದಲೂ ನನ್ನ ಜೊತೆಯಲ್ಲಿರುವವರು. ಹಾಗೆಯೇ ರಾಜೇಶ್ವರ ಶಾಸ್ತ್ರಿಗಳು ದೂರದಿಂದ ದಯಮಾಡಿಸಿದ್ದಾರೆ. ನನ್ನ ಮಿತ್ರರಾದ ಪಿ.ಎನ್.ಶಾಸ್ತ್ರಿಗಳು ನನ್ನ ಮೇಲಿನ ಅಭಿಮಾನದಿಂದ ದಿಲ್ಲಿಯಿಂದ ಇಲ್ಲಿಗೆ ಆಗಮಿಸಿದ್ದಾರೆ. ಮಿತ್ರ ಉಮಾಕಾಂತನಿದ್ದಾನೆ. ವಿಶೇಷವಾಗಿ ನಮಗೆ ಈ ಎಲ್ಲ ಸೌಕರ್ಯ ಒದಗಿಸಿದಂತಹ ರಾಜಮಾತೆಯವರು ದಯಮಾಡಿಸಿ ಇಷ್ಟು ಹೊತ್ತು ನಮ್ಮೊಡನೆ ಇದ್ದಾರೆ. ಅವರ ಇಡೀ ಪರಂಪರೆಗೆ ಕೃತಜ್ಞತೆಯನ್ನು ಸಲ್ಲಿಸಿ ನನ್ನ ಮಾತನ್ನು ಮುಗಿಸುತ್ತೇನೆ. \enginline{-} ಹೀಗೆ ಶ್ರೀಯುತರ ಭಾವ ಪ್ರವಾಹವಾಗಿ ಸಭೆಯಲ್ಲಿ ಪ್ರವಹಿಸಲಾಗಿ ಸಭೆ ಮಂತ್ರಮುಗ್ಧವಾಗಿತ್ತು.
\vskip 4pt

(ಇಲ್ಲಿ ಶ್ರೀಯುತ ಗಂಗಾಧರ ಭಟ್ಟರ ಮಾತನ್ನು ಇಷ್ಟು ಬರೆದಿದ್ದರೂ\break ಬರೆದಂತಾಗಿಲ್ಲ. ಕಾರಣ ಅವರ ಮಾತಿನಲ್ಲಿ ಧ್ವನ್ಯರ್ಥವೇ ಪ್ರಧಾನ. ಧ್ವನ್ಯರ್ಥಗಳನ್ನೆಲ್ಲ\break ಪದದಲ್ಲಿ ಬರೆಯುವುದು ಸಾಧ್ಯವೇ ಇಲ್ಲ. ಒಂದು ವೇಳೆ ಬರೆದರೂ ಅದಕ್ಕೆ ಆ ಸ್ವಾರಸ್ಯ\-ವಿರುವುದಿಲ್ಲ. ಇಲ್ಲಿ ಬರೆದಿರುವುದನ್ನು ಓದುವುದಕ್ಕೂ ಅವರ ಮಾತನ್ನೂ\break ಕೇಳು\-ವುದಕ್ಕೂ ಉಂಟಾಗುವ ಅರ್ಥ ಭಾವಗಳಿಗೆ ಬಹಳ ಅಂತರವಿದೆ. ಆದ್ದರಿಂದ ಯಥಾಶಕ್ತಿ ನಿರೂಪಿಸಲಾಗಿದೆ. ಅನೇಕ ಮಾತುಗಳನ್ನು ಬರೆಯದೇ ಬಿಡಲಾಗಿದೆ)

ಈ ಮುಂದೆ ವಿದ್ಯಾರ್ಥಿ ಶ್ರೀ ಸೀತಾಮರ ವಂದನಾರ್ಪಣೆಯಿಂದ ಕಾರ್ಯಕ್ರಮ ಸುಸಂಪನ್ನಗೊಂಡು ಮಾತೃಗೀತೆಯೊಂದಿಗೆ ಅಂತ್ಯಮಂಗಳ ಕಂಡಿತು. 

ಇದಾದ ಅನಂತರ ವಯಕ್ತಿಕವಾಗಿ ಸಾಂಘಿಕವಾಗಿ ಶ್ರೀಯುತರನ್ನು ಅಭಿವಂದಿಸಲು ಬಂದ ಅನೇಕರಿಗೆ ಅವಕಾಶ ಮಾಡಿಕೊಡಲಾಯಿತು. ಹವೀಕಸಂಘದಿಂದ ಶ್ರೀಲಕ್ಷ್ಮೀ\-ನಾರಾಯಣ, ಸುಧರ್ಮಾದಿಂದ ಶ್ರೀಸಂಪತ್ಕುಮಾರ್ ಅವರು ಹೀಗೆ ಇನ್ನೂ ಅನೇಕರು ತಮ್ಮ ಭಾವಸಮರ್ಪಣೆ ಮಾಡಿದರು. ಮುಂದೆ ಸಾಕ್ಷ್ಯಚಿತ್ರ ಪ್ರದರ್ಶಿಸಲಾಯಿತು. ಅಂತೂ ನಾವಂದುಕೊಂಡ ಕಾರ್ಯಕ್ರಮ ಸುಖಾಂತ್ಯವನ್ನು ಕಂಡು ನಾವೆಲ್ಲ ತೃಪ್ತಿಯ ನಿಟ್ಟುಸಿರನ್ನು ಬಿಟ್ಟೆವು.

ಕೃತಜ್ಞತಾ ಸಮರ್ಪಣೆ \enginline{-}

ಈ ಕಾರ್ಯಕ್ರಮ ಹೀಗೆ ಯಶಸ್ವಿಯಾಗಲು  ಅನೇಕರು ಕಾರಣಕರ್ತರಿದ್ದಾರೆ. ಮುಖ್ಯವಾಗಿ ರಾಜಮಾತೆಯವರು ದಯಮಾಡಿಸಿದ್ದು ನಮ್ಮ ಭಾಗ್ಯ. ಅವರಿಗೆ ನಾವು ಎಷ್ಟು ಕೃತಜ್ಞರಾಗಿದ್ದರೂ ಸಾಲದು. ಗಂಗಾಧರ ಭಟ್ಟರನ್ನು ಗಣಪತಿಸಚ್ಚಿದಾನಂದ ಆಶ್ರಮ\-ದಲ್ಲಿ ಆಸ್ಥಾನ ವಿದ್ವಾಂಸರೆಂದು ಸ್ವೀಕರಿಸಿದ್ದಾರೆ. ಪೂಜ್ಯ ಸ್ವಾಮಿಗಳು ಸ್ವತಃ ಫಲ ಮಂತ್ರಾಕ್ಷತೆಯನ್ನು ವಿ । ವಂಶಿಯವರ ಮೂಲಕ ಕಳುಸಿಕೊಟ್ಟಿದ್ದಾರೆ. ಅವರಿಗೆ ನಮ್ಮ ಅನಂತ ನಮನಗಳು. ಅಲ್ಲದೇ ವಂಶಿಯವರು  ಶ್ರೀ ರಾಜೇಶ್ವರ ಶಾಸ್ತ್ರಿಯವರ ವಾಸ ಮತ್ತು ಊಟೋಪಚಾರದ ವ್ಯವಸ್ಥೆಯನ್ನು ಕಲ್ಪಿಸಿದ್ದು ನಮಗೆ ಅತ್ಯಂತ ಅನುಕೂಲ\-ವಾಯಿತು, ಅವರಿಗೆ ನಮ್ಮ ಅನಂತ ಕೃತಜ್ಞತೆಗಳು. ನಮ್ಮ ಸಮಿತಿಗೆ ಅಧ್ಯಕ್ಷರಾಗಲು ಶ್ರೀ ನಿರಂಜನ ವಾನಳ್ಳಿಯವರು ನಮ್ಮ ಒತ್ತಾಯಕ್ಕೆ ಒಪ್ಪಿಕೊಂಡರು.  ನಮಗೆ ಎಲ್ಲ ರೀತಿಯಿಂದ ಪ್ರೋತ್ಸಾಹಿಸಿದ್ದು ಅತ್ಯಂತ ಸಹಕಾರಿಯಾಯಿತು. ಅವರಿಗೆ ನಮ್ಮೆಲ್ಲರ ವಂದನೆಗಳು. ಶ್ರೀ ಟಿ.ವಿ. ಸತ್ಯನಾರಾಯಣರಂತೂ ಬೇರೆ ದೇಶಕ್ಕೆ ತೆರಳಿದ್ದರು.\break ಅಲ್ಲಿಂದಲೂ ಅವರು ನಡೆಯುತ್ತಿರುವ ತಯಾರಿಯನ್ನು ವಿಚಾರಿಸುತ್ತಿದ್ದರು.\break ವಾಟ್ಸಾಪ್ ಮಾಡುತ್ತಿದ್ದರು.  ಅವರ ಪ್ರೀತಿ ವಿಶ್ವಾಸಗಳಿಗೆ ಆನಂತ ಕೃತಜ್ಞತೆಗಳು. ಶ್ರೀ ಹೆಚ್.ವಿ.ನಾಗರಾಜರಾವ್ ರವರು ಗಂಗಾಧರ ಭಟ್ಟರಲ್ಲಿ ವಿಶೇಶ ವಿಶ್ವಾಸವಿರುವವರು, ಅವರಲ್ಲಿ ಗೌರವ ಸಲಹೆಗಾರರೆಂದು ತಮ್ಮ ಹೆಸರನ್ನು ಬಳಸಿಕೊಂಡಿದ್ದೇವೆಂದು ಅನಂತರ ತಿಳಿಸಿದರೂ ವಿಶ್ವಾಸಪೂರ್ವಕವಾಗಿ ಅವರದೇ ಶೈಲಿಯಲ್ಲಿ ನಕ್ಕು ತಮ್ಮ ಸೌಹಾರ್ದ ಒಪ್ಪಿಗೆಯನ್ನು ವ್ಯಕ್ತಪಡಿಸಿದರು. ಇವರಿಗೆ ನಮ್ಮ ನಮಸ್ಕಾರಗಳು. ವಿದ್ವಾನ್ ಶ್ರೀಧರ ಶಾಸ್ತ್ರಿಗಳು ನಾವು ವಿಷಯ ತಿಳಿಸಿದೊಡನೆ ಯಾವಾಗ ಏನು ಬೇಕಾದರೂ ಸಂಕೋಚವಿಲ್ಲದೇ ಕೇಳಿ. ಕಾರ್ಯಕ್ರಮ ಮಾತ್ರ ಚೆನ್ನಾಗಿ ಮಾಡಿ ಎಂದರು. ಅವರಿಗೆ ಅನಂತ ನಮನ\-ಗಳು. ಶ್ರೀ ಉಮಾಕಾಂತ ಭಟ್ಟರಂತೂ ನಮಗೆ ಅತಿಯಾದ ಸಲುಗೆಯವರು. ಅವರೇ ಎಲ್ಲವನ್ನೂ ವಿಚಾರಿಸಿ ಸಲಹೆಗಳನ್ನು ಸಲುಗೆಯಿಂದ ಕೊಡುತ್ತಿದ್ದುದು ಅತ್ಯಂತ ಅನುಕೂಲ\-ವಾಯಿತು. ಶಾಸ್ತ್ರಗೋಷ್ಠಿಯು ಯಶಸ್ವಿಯಾಗುವುದಕ್ಕೆ ಅವರೇ ಪ್ರಧಾನ ಕಾರಣ. ವಿ । ರಾಜೇಶ್ವರ ಶಾಸ್ತ್ರಿಯವರೂ ಅವರ ಪ್ರಯತ್ನದಿಂದಲೇ ಬರುವಂತಾಯಿತು. ಶ್ರೀಯುತರಿಗೆ ಅನಂತ ವಂದನೆಗಳು. ಪಾಠಶಾಲೆಯ ಹಾಲಿ ಪ್ರಾಂಶುಪಾಲರು\break ಡಾ । ಕೆ.ಎಮ್.ಮಹದೇವಯ್ಯನವರು ಪಾಠಶಾಲೆಯಲ್ಲಿ ಅವಕಾಶವನ್ನು ಕಲ್ಪಿಸಿ\-ಕೊಟ್ಟುದು ಕಾರ್ಯಕ್ರಮದ ಯಶಸ್ಸಿಗೆ ಇನ್ನೊಂದು ಕಾರಣ. ಅವರಿಗೆ ಧನ್ಯವಾದಗಳು. ಶ್ರೀಯುತ ಆಳ್ವಾರ್ ರವರು ಅಗತ್ಯ ಸಲಹೆಗಳೊಂದಿಗೆ ಗೋಷ್ಠಿಯ ವಿಷಯಗಳನ್ನು\break ವ್ಯವಸ್ಥೆಗೊಳಿಸಿಕೊಟ್ಟರು. ಅಲ್ಲದೇ ರಾಜಮಾತೆಯವರು ಬರುವಲ್ಲಿ ನಮಗೆ ಅನುಕೂಲವಾದರು. ಅವರಿಗೆ ಅನಂತ ಕೃತಜ್ಞತೆಗಳು. ವೇ ಶ್ರೀ ರಮೇಶ ಅಡಿಗರು ಕೊಟ್ಟ ಸಹಕಾರ, ಸ್ವತಃ ಅವರ ಪ್ರಯತ್ನಶೀಲತೆ ಅಸಾಧಾರಣ, ಅವರಿಗೆ ನಮ್ಮ ಅನಂತ ನಮನಗಳು.  ಶ್ರೀವಾಚಸ್ಪತಿ ಶಾಸ್ತ್ರಿಯವರು  ಆಗಮಿಸಿದ್ದುದು ಗೋಷ್ಠಿಗೊಂದು ವಿಶೇಷ ಹುರುಪು ಬಂದು ಅದರ  ಶೋಭನಕ್ಕೆ ಕಾರಣವಾಯಿತು, ಅವರಿಗೆ ನಮ್ಮ ಕೃತಜ್ಞತೆಗಳು. ಹವೀಕ ಸಂಘದ ಶ್ರೀಲಕ್ಷ್ಮೀನಾರಾಯಣ ನಮ್ಮ ಆತ್ಮೀಯ ಲಚ್ಚ ನಮಗೆ ಎಲ್ಲ ರೀತಿಯಲ್ಲಿ ಸಹಕರಿಸಿದರು. ಅವರಿಗೆ ನಮ್ಮ ವಂದನೆಗಳು. ವಿದ್ವಾನ್ ಹೇರಂಭ ಭಟ್ಟರು ಆದ್ಯಂತವಾಗಿ ನಮ್ಮ ಬೆನ್ನಿಗೆ ನಿಂತರು. ಅವರ ಸಲಹೆ ಸಹಕಾರ ಅನನ್ಯ. ಅವರಿಗೆ ನಮ್ಮೆಲ್ಲರ ವಂದನೆಗಳು. 

ಇವೆಲ್ಲ ಸಂಪನ್ನವಾಗಲು ಸಂಪತ್ತು ಅನಿವಾರ್ಯ. ನಾವಾಗಿ ಯಾರನ್ನೂ ಆ ಬಗ್ಗೆ ಕೇಳುವ ಮಾತೇ ಇರಲಿಲ್ಲ. ಆದರೆ ಅನೇಕರು ಸ್ವತಃ ಪ್ರೇರಿತರಾಗಿ ಕೃತಜ್ಞರಾಗಿ ನಮಗೆ ಸಂಪತ್ತಿನ ಕೊರತೆಯಾಗದಂತೆ ಸಹಕರಿಸಿದ್ದಾರೆ. ಅವರಿಗೆ ಇನ್ನೂ ಅವರ ಸಂಪತ್ತು ವೃದ್ಧಿಸಲೆಂದು ಹಾರೈಸಿ ನಮ್ಮ ಕೃತಜ್ಞತೆಗಳನ್ನು ತಿಳಿಸಬಯಸುತ್ತೇವೆ.

ಕಾರ್ಯಕ್ರಮಕ್ಕೆ ಅನೇಕ ಹಿರಿಯರು ದಯಮಾಡಿಸಿದ್ದರು. ನಾವು ಪ್ರತ್ಯೇಕವಾಗಿ ಅವರನ್ನೆಲ್ಲ ಯಥೋಚಿತವಾಗಿ ಸತ್ಕರಿಸಲಾಗಲಿಲ್ಲ. ನಮ್ಮ ಮೇಲಿನ ಅವರ ಪ್ರೀತಿಗೆ ಅನಂತ ಕೃತಜ್ಞತೆಗಳು ಸಲ್ಲುತ್ತವೆ. 
\eject

ಕಾರ್ಯಕ್ರಮಕ್ಕೆ ಗಂಗಾಧರ ಭಟ್ಟರ ಕುಟುಂಬ ವರ್ಗದವರು ದೂರದಿಂದ\break ದಯಮಾಡಿಸಿದ್ದುದು ನಮಗೆಲ್ಲ ಅತ್ಯಂತ ಹರ್ಷವನ್ನು ಉಂಟುಮಾಡಿತು. ಅವರಿಗೆ ನಮ್ಮೆಲ್ಲರ ಅನಂತ ನಮನಗಳು. ಅಂತೆಯೇ ಸಮಸ್ತ ಅಭಿಮಾನಿಗಳು ಆಗಮಿಸಿ ಕಾರ್ಯಕ್ರಮವನ್ನು ಯಶ್ವಿಗೊಳಿಸಿದ್ದಾರೆ. ಅವರನ್ನೆಲ್ಲ ನಾವು ಕೃತಜ್ಞತಾ ಭಾವದಿಂದ\break ಸ್ಮರಿಸುತ್ತೇವೆ 

ಅಂತಿಮವಾಗಿ ಪ್ರತ್ಯಕ್ಷ ಪರೋಕ್ಷ ಸಹಕರಿಸಿದ ಸಮಸ್ತರಿಗೂ ವಿ । ಗಂಗಾಧರ ಭಟ್ಟರ ಅಭಿವಂದನ ಸಮಿತಿಯು ಅನಂತ ನಮಸ್ಕಾರಗಳನ್ನು ತಿಳಿಸುತ್ತದೆ.

\begin{center}
ಈ ಎಲ್ಲ ನಮ್ಮ ಸಮಸ್ತ ಆಚಾರ  ಆಚಾರ್ಯ ಗಂಗಾಧರ ಭಟ್ಟರಿಗೂ\\ ಅವರ ಧರ್ಮಪತ್ನಿ ಶ್ರೀಮತಿ ಶೈಲಜಾರವರಿಗೂ\\ ಸಮರ್ಪಿತವಾಗಲೆಂದು ಬೇಡಿಕೊಳ್ಳುತ್ತೇವೆ.
\medskip

ಬದ್ರಂ ಶುಭಂ ಮಂಗಳಮ್
\end{center}

\articleend
}
