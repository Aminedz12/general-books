{\fontsize{14}{16}\selectfont
\chapter{ಅಧ್ಯಕ್ಷೀಯ}

ಗುರುವಿನಿಂದಲೇ ಗುರಿ ತಲುಪಲು ಸಾಧ್ಯ. ಜ್ಞಾನದೀವಿಗೆಯಿಂದ ಶಿಷ್ಯನ ಬಾಳಿನಲ್ಲಿ ಬೆಳಕನ್ನು ಪಸರಿಸಿ ಆತನ ಭವ್ಯಜೀವನದ ಅಡಿಪಾಯವನ್ನು ನಿರ್ಮಿಸುತ್ತಾನೆ ಗುರು. ಅದರ ಮೇಲೆ ಶಿಷ್ಯನಾದವನು ಎಷ್ಟು ಎತ್ತರದ ಭವನವನ್ನಾದರೂ ನಿರ್ಮಿಸಬಹುದು. ಆಚಾರ್ಯ ಶಂಕರರು ’ಅಧಿಗತತ್ವಃ, ಶಿಷ್ಯಹಿತಾಯ ಉದ್ಯತಃ ಸತತಮ್’ \eng{-} ಗುರುವು ಶಿಷ್ಯಹಿತಕ್ಕಾಗಿ ಸದಾ ಉದ್ಯನತನಾಗಿರುತ್ತಾನೆ ಎಂದಂತೆ ಅಂತಹ ನಿಜಾರ್ಥದ ಗುರುಗಳು ಗಂಗಾಧರ ಭಟ್ಟರು. ನೂರಾರು ವಿದ್ಯಾರ್ಥಿಗಳಿಗೆ ಶಾಸ್ತ್ರಬೋಧನೆಯ ಬೆಳಕನ್ನು ಬೆಳಗಿಸಿ ಸದಾ ಶಿಷ್ಯರ ಏಳ್ಗೆಯನ್ನು ಬಯಸಿದವರು, ಅದಕ್ಕೆ ಉದ್ಯತರಾದವರು. ಪಾಠ\eng{-}ಪ್ರವಚನ, ಕಾಲೇಜು, ಹಾಸ್ಟೆಲ್, ಉದ್ಯೋಗ ಹೀಗೆ ಒಂದಿಲ್ಲೊಂದು ವಿಷಯದಲ್ಲಿ ಅದೆಷ್ಟೋ ಜನ ಅವರಿಂದ ಉಪಕೃತರಾಗಿದ್ದಾರೆ. ಮಹಾರಾಜ ಪಾಠಶಾಲೆಯಲ್ಲಿ ಅವರಲ್ಲಿ ನ್ಯಾಯಶಾಸ್ತ್ರವನ್ನೇ ಪ್ರಧಾನವಾಗಿ ಅಧ್ಯಯನ ಮಾಡಿದವರೇ ೨೫ ಕ್ಕೂ ಹೆಚ್ಚು ವಿದ್ಯಾರ್ಥಿಗಳು. ಇನ್ನು ಬೇರೆ ಶಾಸ್ತ್ರದ ವಿದ್ಯಾರ್ಥಿಗಳಂತೂ ಬಹುಶಃ ಸಾವಿರಕ್ಕೂ ಹೆಚ್ಚಾಗಬಹುದು. ಇಷ್ಟೊಂದು ವಿದ್ಯಾರ್ಥಿಗಳಿಗೆ ಪಾಠ ಮಾಡಿದ್ದಕ್ಕಿಂತ ಅಷ್ಟೂ\break ವಿದ್ಯಾರ್ಥಿಗಳಿಗೆ ಅವರು ತೋರಿದ ಪ್ರೀತಿ ಮತ್ತು ಕಾಳಜಿಗೆ ಬೆಲೆ ಕಟ್ಟಲು ಸಾಧ್ಯವಿಲ್ಲ. ವಿದ್ಯಾರ್ಥಿಗಳ ಮೇಲೆ ಎಷ್ಟು ಪ್ರೀತಿಯೆಂದರೆ ಬೇರೆ ಊರಿನಿಂದ ಬಂದ ಅದೆಷ್ಟೋ ವಿದ್ಯಾರ್ಥಿಗಳಿಗೆ ವಸತಿ, ಊಟೋಪಚಾರದ ವ್ಯವಸ್ಥೆಯನ್ನು ಮಾಡಿದ್ದಾರೆ. ಹಾಸ್ಟೆಲ್, ಊಟದ ವ್ಯವಸ್ಥೆಯಾಗದಿದ್ದಾಗ ಅವರ ಮನೆಯಲ್ಲಿಯೇ ವಸತಿ, ಊಟ. ಹೀಗೆ ವಿದ್ಯೆ ಮತ್ತು ವಿದ್ಯಾರ್ಥಿಗಳ ಮೇಲಿನ ಪ್ರೀತಿಯಿಂದ ಅನೇಕ ವಿದ್ಯಾರ್ಥಿಗಳಿಗೆ ಜ್ಞಾನದಾತರೂ ಅನ್ನದಾತರೂ ಹೌದು. ಇವರಿಂದ ಉಪಕೃತರಾದ ಇಂತಹ ಅದೆಷ್ಟೋ ವಿದ್ಯಾರ್ಥಿಗಳು ಇಂದು ಉತ್ತಮಸ್ಥಾನದಲ್ಲಿದ್ದಾರೆ. ಆದ್ದರಿಂದಲೇ ಅವರು ನಿಜಾರ್ಥದ ಗುರುಗಳು.

ಅಂತಹ ವಿದ್ಯಾಗುರುಗಳಿಗೆ ಭಕ್ತಿಯ ಅಭಿವಂದನೆಯನ್ನು ಸಲ್ಲಿಸಬೇಕಾದದ್ದು\break ಶಿಷ್ಯರಾದ ನಮ್ಮೆಲ್ಲರ ಆದ್ಯ ಕರ್ತವ್ಯ.
\begin{verse}
ಶಿಷ್ಯೇಣಾಪಿ ತಥಾ ಕಾರ್ಯಂ ಯಥಾ ಸಂತೋಷಿತೋ ಗುರುಃ~।\\
ಪ್ರಿಯಂ ಕುರ್ಯಾಚ್ಚ ದೇವೇಶಿ ಮನೋವಾಕ್ಕಾಯಕರ್ಮಭಿಃ~॥\\
ಗುರೋರ್ಹಿತಂ ಹಿ ಕರ್ತವ್ಯಂ ಮನೋವಾಕ್ಕಾಯಕರ್ಮಭಿಃ~।\\ 
\hspace{6cm}(ಕುಲಾರ್ಣವತಂತ್ರ)
\end{verse}
ಎಂಬ ತಂತ್ರಶಾಸ್ತ್ರದ ಮಾತಿನಿಂತೆ ಶಿಷ್ಯನಾದವನು ಕಾಯಿಕ, ವಾಚಿಕ ಹಾಗೂ\break ಮಾನಸಿಕವಾಗಿಯೂ ಸಹ ಸತತವೂ ಗುರುವಿನ ಹಿತವನ್ನು, ಹಾಗೂ ಗುರುವಿಗೆ ಸಂತೋಷವುಂಟು ಮಾಡುವುದು ಶಿಷ್ಯನ ಧರ್ಮ. ಗಂಗಾಧರ ಭಟ್ಟರ ಶಿಷ್ಯರಾದ ನಮಗೆ ಈ ಸದವಕಾಶ ಸಿಕ್ಕಿದ್ದು ಅವರ ನಿವೃತ್ತಿಯ ಸಂದರ್ಭದಲ್ಲಿ. ಅವರ ವಿದ್ಯಾರ್ಥಿಗಳೆಲ್ಲಾ ಸೇರಿ ನಮ್ಮ ನೆಚ್ಚಿನ ವಿದ್ಯಾಗುರುಗಳಿಗೆ ಅಭಿವಂದನಾ ಕಾರ್ಯಕ್ರಮವನ್ನು\break ನೆರವೇರಿಸಬೇಕೆಂದು ನಿಶ್ಚಯಿಸಿ ಮೊದಲಿಗೆ ಒಂದು ವಾಟ್ಸಪ್ ಸಮೂಹವನ್ನು ತೆರೆದು ನಮ್ಮ ಸಹಾಧ್ಯಾಯಿಗಳನ್ನು ಸೇರಿಸಿದೆವು. ಗಂಗಾಧರ ಭಟ್ಟರಲ್ಲಿ ನ್ಯಾಯಶಾಸ್ತ್ರವನ್ನೇ ಪ್ರಧಾನವಾಗಿ ಓದಿದ ಸುಮಾರು ೨೫ ಶಿಷ್ಯರು ಈ ಗಣಕ್ಕೆ ಸೇರಿದರು. ಹೀಗೆ ಚಾಲನೆ ಸಿಕ್ಕಿದ್ದು ಮಿತ್ರ ವಿನಾಯಕನಿಂದ.

ಆದರೆ ಭಟ್ಟರ ಸ್ವಭಾವನ್ನು ಚೆನ್ನಾಗಿ ಅರಿತ ನಮಗೆ ಈ ಕಾರ್ಯಕ್ರಮಕ್ಕೆ ಒಪ್ಪಿಸುವುದು ಹೇಗೆ ಎಂಬ ಚಿಂತೆಯೊಂದಿತ್ತು. ಏಕೆಂದರೆ ಅವರು ಪ್ರಚಾರಪ್ರಿಯರಲ್ಲ. ವಿಚಾರ ಮತ್ತು ಆಚಾರಪರರು. ಹೀಗಾಗಿ ಅವರನ್ನು ಒಪ್ಪಿಸುವ ಜವಾಬ್ದಾರಿಯನ್ನು ಮೈಸೂರಿನಲ್ಲಿರುವ ಸ್ನೇಹಿತ ಗುರುಪ್ರಸಾದ ಮತ್ತು ಸುರೇಶ್ ಹೆಗೆಡೆಗೆ ವಹಿಸಲಾಯಿತು ಅವರು ಅದನ್ನು ನಿರ್ವಹಿಸಿದರು. ಅಂತೂ ಭಟ್ಟರು ವಿದ್ಯಾರ್ಥಿಗಳೇ ಮಾಡುವ ಕಾರ್ಯಕ್ರಮ ಎಂಬ ಕಾರಣಕ್ಕೆ ವಿದ್ಯಾರ್ಥಿಗಳ ಪ್ರೀತಿಯನ್ನು ತಿರಸ್ಕರಿಲಾಗದೇ ಒಪ್ಪಿದರು. ನಂತರ ೨೦೧೭ ರ ನವೆಂಬರ್ ತಿಂಗಳಲ್ಲಿ ಮೈಸೂರಿನಲ್ಲಿ ಕೆಲವು ವಿದ್ಯಾರ್ಥಿಗಳು ಸಭೆ ಸೇರಿ ಒಂದು ದಿನದ ಕಾರ್ಯಕ್ರಮದ ರೂಪುರೇಷೆ ಸಿದ್ಧಪಡಿಸಿದೆವು. ಗಂಗಾಧರ ಭಟ್ಟರಿಗೆ ಸದಾ ಅಧ್ಯಯನ ಅಧ್ಯಾಪನವೇ ಪ್ರಿಯವಾದದ್ದು, ಹಾಗಾಗಿ ಶಾಸ್ತ್ರಾರ್ಥ ಗೋಷ್ಠಿ ನಡೆದರೆ ಅವರಿಗೆ ಸಂತೋಷವಾಗುತ್ತದೆ ಎಂದು ಬೆಳಿಗ್ಗೆ ಒಂದು ಅವಧಿಗೆ ಗೋಷ್ಠಿ, ಮಧ್ಯಾಹ್ನ ಅಭಿವಂದನಾ ಸಮಾರಂಭ ಮತ್ತು ಅಭಿವಂದನ ಗ್ರಂಥ ಇದು ನಮ್ಮ ಯೋಜನೆ. ಅದಕ್ಕೆ ತಕ್ಕಂತೆ ಜವಾಬ್ದಾರಿಯನ್ನು ಹಂಚಿ, ಕಾರ್ಯಭಾರದ ನಿರ್ವಹಣೆಯ ದೃಷ್ಟಿಯಿಂದ ಒಂದು ಸಮಿತಿಯನ್ನು ಅಂದೇ ರೂಪಿಸಿಕೊಂಡೆವು. ಆದರೆ ನಮ್ಮ ಸಮಿತಿಯಲ್ಲಿ ಎಲ್ಲರೂ ಕಾರ್ಯಕರ್ತರೇ, ಅಧ್ಯಕ್ಷ, ಕಾರ್ಯದrರ್ಶೀ ಎನ್ನುವುದು ಕೆಲಸದ ನಿರ್ವಹಣೆಗಷ್ಟೆ ಎಂಬುದು ನಮ್ಮ ತೀರ್ಮಾನವಾಗಿತ್ತು. ಅದರಂತೆ ಎಲ್ಲರೂ ಅವರವರ ಕಾರ್ಯದಲ್ಲಿ ತೊಡಗಿದರು. ನಂತರದ ಹೆಚ್ಚಿನ ಸಂವಹನಗಳೆಲ್ಲಾ ವಾಟ್ಸಪ್ ಮೂಲಕವೇ ನಡೆಯಿತು. ಅನೇಕ ವಿಷಯಗಳನ್ನು ಅಲ್ಲೇ ಚರ್ಚಿಸಿ ತೀರ್ಮಾನಿಸುತ್ತಿದ್ದೆವು. ಗುಂಪಿನ ಎಲ್ಲರೂ ಸಕ್ರಿಯ\-ವಾಗಿ ಸಲಹೆ ಸೂಚನೆಗಳನ್ನು ನೀಡುತ್ತಿದ್ದರು.

ಅಂತೆಯೇ ಫೆಬ್ರುವರಿ ೧೧ ಭಾನುವಾರದಂದು ಕಾರ್ಯಕ್ರಮ ನಿಶ್ಚಿತವಾಯಿತು. ಅದಕ್ಕೆ ತಕ್ಕಂತೆ ಅತಿಥಿ ಅಭ್ಯಾಗತರನ್ನು ಸಂಪರ್ಕಿಸುವಲ್ಲಿ ಅನೇಕ ಮಿತ್ರರು, ಗುರುಗಳು ಸಹಕಾರ ನೀಡಿದ್ದಾರೆ. ಅದರಲ್ಲಿ ಮಂಜುನಾಥ ಹೆಗಡೆ ಮತ್ತು ನಾರಾಯಣ ವೈದ್ಯನನ್ನು ಅಭಿನಂದಿಸುತ್ತೇನೆ. ಜೊತೆಗೆ ವಿದ್ವಾಂಸರನ್ನು ಸಂಪರ್ಕಿಸಿ ಲೇಖನಗಳನ್ನು ಸಂಗ್ರಹಿಸುವ ಕೆಲಸ ಮೊದಲು ಆಗಬೇಕಾಗಿತ್ತು. ಅದಕ್ಕೆ ಸಮಿತಿಯ ಎಲ್ಲರೂ ಅವರ ಒಡನಾಟದಲ್ಲಿ ಪ್ರಯತ್ನಿಸಿದ್ದರಿಂದ ಇಂದು ಅಭಿವಂದನ ಗ್ರಂಥ ೩೦೦ ಕ್ಕೂ ಹೆಚ್ಚು ಪುಟಗಳ ಸಂಗ್ರಾಹ್ಯ ಪುಸ್ತಕವಾಗಿ ಹೊರಬರುತ್ತಿದೆ.

ಕಾರ್ಯಕ್ರಮದ ದಿನ ಹತ್ತಿರ ಬರುತ್ತಿದ್ದಂತೆಯಂತೂ ಎಲ್ಲರಲ್ಲೂ ಹುರುಪು. ಬಹಳ ಚುರುಕಾಗಿ ನಿರಂತರ ಸಂಪರ್ಕದಲ್ಲಿದ್ದರು. ಹೆಚ್ಚಿನವರು ಮೈಸೂರಿನಲ್ಲಿಲ್ಲದಿದ್ದರೂ ಒಂದು ದಿನ ಮುಂಚಿತವಾಗಿ ಅಂದರೆ ೧೦ ನೇ ದಿನಾಂಕದಂದೇ ಮೈಸೂರಿಗೆ ಬಂದು ಸಭಾಸಿದ್ಧತೆ, ಸರಂಜಾಮುಗಳ ಖರೀದಿ, ತಳಿರುತೋರಣ ಹೋಗೆ ಸಕಲ ವ್ಯವಸ್ಥೆಯನ್ನೂ ಅಣಿಗೊಳಿಸಿದರು. ಎಲ್ಲರ ಸಹಕಾರದಿಂದ ಕಾರ್ಯಕ್ರಮ ಸಾಂಗವಾಗಿ ಸಂಪನ್ನ\-ವಾಯಿತು ಎನ್ನುವುದು ಸಂತೋಷದ ಸಂಗತಿ. ಕಾರ್ಯಕ್ರಮಕ್ಕೆ ಸ್ಥಳಾವಕಾಶ ನೀಡಿ, ಹಳೆಯ ನೆನಪುಗಳನ್ನು ಮೆಲುಕು ಹಾಕಲು ಅವಕಾಶ ನೀಡಿದ ನಮ್ಮ ಪಾಠಶಾಲೆ ಮತ್ತು ಪ್ರದೋಷಸಂಘಕ್ಕೆ ಸಮಿತಿಯ ಪರವಾಗಿ ವಂದನೆಗಳನ್ನು ಸಲ್ಲಿಸುತ್ತೇನೆ.

ಮೈಸೂರಿನ ಮಹಾರಾಣಿಯವರು ಕಾರ್ಯಕ್ರಮಕ್ಕೆ ಆಗಮಿಸಿದ್ದು ವಿಶೆಷ.\break ಹಾಗಾಗಿಯೇ ಕಾರ್ಯಕ್ರಮದಲ್ಲಿ ಪುಸ್ತಕವನ್ನು ಸಾಂಕೇತಿಕವಾಗಿ ಬಿಡುಗಡೆ ಮಾಡಿ, ಕಾರ್ಯಕ್ರಮದ ಫೋಟೋ, ವರದಿಯೊಂದಿಗೆ ನಂತರ ಮುದ್ರಣ ಮಾಡುವ\break ತೀರ್ಮಾನ ಮಾಡಿದ್ದೆವು. ಅದರಂತೆ ಈಗ ಅಭಿವಂದನ ಗ್ರಂಥ ಸುಂದರವಾಗಿ\break ಹೊರಬರುತ್ತಿದೆ. ಇದರ ಸಂಪಾದಕರಾಗಿ ಗುರುಪ್ರಸಾದರ ಶ್ರಮ ಶ್ಲಾಘನೀಯ.

ನಮ್ಮ ವಿದ್ಯಾರ್ಥಿ ಸಮಿತಿಗೆ ಗೌರವ ಸಲಹೆಗಾರರಾಗಿ ಡಾ. ಎಚ್. ವಿ. ನಾಗರಾಜ ರಾಯರು, ಡಾ. ಟಿ.ವಿ. ಸತ್ಯನಾರಾಯಣರು, ವಿ. ಉಮಾಕಾಂತ ಭಟ್ಟರು, ಡಾ. ಎಮ್.ಎ. ಆಳ್ವಾರ್ ಅವರು ಸದಾ ಸೂಕ್ತ ಸಲಹೆ ನೀಡಿ ನಮ್ಮನ್ನು ಮುನ್ನಡೆಸಿದರು. ಅವರಿಗೆ ಗೌರವಪೂರ್ವಕ ವಂದನೆಗಳು.
\newpage

ವಿದ್ಯಾರ್ಥಿಗಳ ಗುರುಭಕ್ತಿಯ ಅಭಿವ್ಯಕ್ತಿಯ ಈ ಕಾರ್ಯಕ್ರಮದ ಯಶಸ್ಸಿಗೆ ಸಹಕರಿಸಿ\-ದವರು ಬಹಳ. ವೈಯಕ್ತಿಕವಾಗಿ ಹೆಸರು ಹೇಳುವುದಾದರೆ ಅದೇ ಒಂದು ಪುಸ್ತಕವಾದೀತು. ಆದ್ದರಿಂದ ಗ್ರಂಥಕ್ಕೆ ಲೇಖನ ನೀಡಿದ ವಿದ್ವಾಂಸರಿಗೂ, ಪ್ರತ್ಯಕ್ಷವಾಗಿ ಪರೋಕ್ಷ\-ವಾಗಿ ಸಹಕರಿಸಿದ ಎಲ್ಲ ಮಹನೀಯರಿಗೂ ಸಮಿತಿಯ ಪರವಾಗಿ\break ಧನ್ಯವಾದಗಳನ್ನು ಸಲ್ಲಿಸುತ್ತೇನೆ. ಪೂಜ್ಯಪೂಜೆಯಿಂದ ಸ್ವಶ್ರೇಯಸ್ಸಿಗೆ ಭಾಜನರಾದ ಎಲ್ಲಾ ಮಿತ್ರರಿಗೂ ನನ್ನ ಕೃತಜ್ಞತೆಗಳು.
\bigskip

\noindent ಸ್ಥಳ: ಮೈಸೂರು\hfill				           \textbf{ವಿ~। ರಾಘವ ಕೆ.ಎಲ್.}\\
ದಿನಾಂಕ: ೧೭.೦೪.೨೦೧೮\hfill  						    ಅಧ್ಯಕ್ಷರು\\
\phantom{i}\hfill ಗಂಗಾಧರ ಭಟ್ಟರ ಅಭಿವಂದನ ಸಮಿತಿ


\articleend
}
