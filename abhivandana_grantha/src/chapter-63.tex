\chapter{ಗುರುಭ್ಯೋ ನಮಃ}

\begin{center}
\Authorline{ಡಾ| ವೆಂಕಟರಮಣ ಹೆಗಡೆ}
\smallskip

ಪ್ರಾಧ್ಯಾಪಕರು,\\ 
ಧರ್ಮಶಾಸ್ತ್ರವಿಭಾಗ
ಮಹಾರಾಜಸಂಸ್ಕೃತಪಾಠಶಾಲೆ,\\ 
ಮೈಸೂರು
\end{center}

೧೯೮೬ನೇ ಇಸವಿ. ಆಗತಾನೇ ನಾನು ನನ್ನ ಸಾಹಿತ್ಯ ತರಗತಿಯ ವ್ಯಾಸಂಗವನ್ನು ಮುಗಿಸಿ ವಿದ್ವತ್ತರಗತಿಯಲ್ಲಿ ಅಧ್ಯಯನ ನಡೆಸಲು ಮೈಸೂರಿಗೆ ಬರುವ ಸಂಕಲ್ಪ ಮಾಡಿದ್ದೆ. ಮನೆಯಲ್ಲಿ ಹಿರಿಯ ಮಗನಾದ್ದರಿಂದ ದೇವರ ಪೂಜಾವಿಧಿಗಳನ್ನು ಕಲಿತುಕೊಂಡು ಮನೆಗೆ ಬಂದು ಕೃಷಿಯನ್ನು ಮಾಡಿಕೊಂಡು ತಮ್ಮೊಂದಿಗೇ ಇರಲೆಂಬ ಉದ್ದೇಶ ಮತ್ತು ಆಸೆಯನ್ನಿಟ್ಟುಕೊಂಡು ಒತ್ತಾಯ ಪೂರ್ವಕವಾಗಿ ನನ್ನನ್ನು ಸೋಮಸಾಗರದ ಪಾಠಶಾಲೆಗೆ ಅಟ್ಟಿದ್ದ ನನ್ನ ತಂದೆಯ ಬಲವಾದ ವಿರೋಧದ ನಡುವೆಯೂ ನಾನು ಮೈಸೂರಿಗೆ ಹೊರಡಲು ಸಂಕಲ್ಪಿಸಿದ್ದೆ. ಪರಮಾತ್ಮನ ಅನುಗ್ರಹವೇ ಮೂರ್ತಿವೆತ್ತಂತಿರುವ ನನ್ನ ತಾಯಿಯ ದೆಸೆಯಿಂದ ನನ್ನ ಸುಬ್ಬು ಮಾವ ನನ್ನನ್ನು ಮೈಸೂರಿಗೆ ತಲುಪಿಸುವ ವ್ಯವಸ್ಥೆಯ ಪ್ರಯತ್ನ ಮಾಡತೊಡಗಿದ. ಆಗ ನನ್ನ ಪಿತಾಮಹಿಯ ತಮ್ಮ ನಮಗನಾದ ಹಳದೋಟದ ಮಾವನಾದ ಎಮ್.ಅರ್.ಹೆಗಡೆ ನನ್ನನ್ನು ಗಂಗಾಧರ ಭಟ್ಟರ ಮನೆಯಾದ ಅಗ್ಗೇರೆಗೆ ಕರೆದುಕೊಂಡು ಹೊಗಿ ನನಗೆ ಗಂಗಾಧರ ಭಟ್ಟರನ್ನು ಪರಿಚಯಿಸಿ, ಅವರ ಆಶ್ರಯವನ್ನು ಕಲ್ಪಿಸಿ ನಾನು ಮೈಸೂರನ್ನು ತಲುಪುವಂತಾಯಿತು. ಇದು ನಾನು ಗಂಗಾಧರ ಭಟ್ಟರನ್ನು ಭೇಟಿಮಾಡಿದ ಮೊದಲ ಪ್ರಸಂಗ. ಇದು ಕೇವಲ ನನ್ನೊಬ್ಬನ ಕಥೆಯಲ್ಲ. ಹೀಗೇ ನನ್ನಂತಹ ಬಹುಶಃ ಶಿರಸಿ, ಸಿದ್ದಾಪುರ, ಯಲ್ಲಾಪುರ, ಕುಮಟಾ, ಹೊನ್ನಾವರ ಮುಂತಾದ ಉತ್ತರಕನ್ನಡ ಜಿಲ್ಲೆಯ ತಾಲ್ಲೂಕುಗಳ ಮತ್ತು ಶಿವಮೊಗ್ಗ, ಸಾಗರ, ದಕ್ಷಿಣಕನ್ನಡ ಪ್ರದೇಶಗಳ ವಿದ್ಯಾರ್ಥಿಗಳಿಗೆ ಮೈಸೂರಿನಲ್ಲಿ ಸಂಸ್ಕೃತ ಮತ್ತು ಇತರ ವಿದ್ಯೆಗಳನ್ನು ಕಲಿಯಲು ಆಶ್ರಯವನ್ನು ಕಲ್ಪಿಸಿದವರು ಗಂಗಾಧರ ಭಟ್ಟರು. ಇದೇ ೨೦೧೮ ರ ಜನವರಿ ೩೧ರಂದು ನಿಯಮಾನುಸಾರ ಸರ್ಕಾರಿಸೇವೆಯಿಂದ ಗಂಗಾಧರಭಟ್ಟರು ನಿವೃತ್ತರಾಗುತ್ತಿರುವುದರಿಂದ ಅವರನ್ನು ನಾನು ನೋಡಿದಂತೆ ಮತ್ತು ನನ್ನ ಅನುಭವದಂತೆ ಅದೀರ್ಘವಾಗಿ ಅನಲ್ಪಾಕ್ಷರಗಳಿಂದ ಚಿತ್ರಿಸುವ ಮೂಲಕ ನನ್ನ ಗೌರವ, ವಂದನೆಗಳನ್ನು ಅವರಿಗೆ ಸಮರ್ಪಿಸುತ್ತಿದ್ದೇನೆ.

\section*{ಸಂಸ್ಕೃತಾಧ್ಯೇತೃಗಳ ಕಲ್ಪವೃಕ್ಷ}

ಶ್ರೀಯುತ ವಿದ್ವಾಂಸರಾದ ಗಂಗಾಧರ ಭಟ್ಟರು ಸಂಸ್ಕೃತಾಧ್ಯಯನವನ್ನು ಬಯಸಿ ಮೈಸೂರಿಗೆ ಬರುವವರ ಪಾಲಿಗೆ ಕಲ್ಪವೃಕ್ಷವೇ ಆಗಿದ್ದರೆಂದರೆ ಅದು ಅತಿಶಯೋಕ್ತಿಯಾಗಲಾರದು. ಬಹುಶಃ ಕಳೆದ ನಲವತ್ತು ವರ್ಷಗಳಿಂದ ಮೈಸೂರಿಗೆ ಬಂದ ಸಂಸ್ಕೃತಾಧ್ಯಯನದ ಆಕಾಂಕ್ಷಿಗಳಲ್ಲಿ ಬಹುಪಾಲು ವಿದ್ಯಾರ್ಥಿಗಳು ಗಂಗಾಧರಭಟ್ಟರ ಆಶ್ರಯದ ಅನುಭವವನ್ನು ಹೊಂದಿದ್ದಾರೆ. ಇಲ್ಲಿನ ಪ್ರಸಿದ್ಧವಾದ ಸರ್ಕಾರಿ ಮಹಾರಾಜ ಸಂಸ್ಕೃತಕಾಲೇಜಿನಲ್ಲಿ ಪ್ರವೇಶಪಡೆಯಲು, ಯಾವ ಶಾಸ್ತ್ರವನ್ನು ಅಧ್ಯನಮಾಡಬೇಕೆಂಬ ಗೊಂದಲವನ್ನು ಪರಿಹರಿಸಿಕೊಳ್ಳಲು, ವಿದ್ಯಾರ್ಥಿನಿಲಯದಲ್ಲಿ ಪ್ರವೇಶವನ್ನು ಪಡೆಯಲು, ಮತ್ತು ಪ್ರತಿದಿನದ ಅಶನವ್ಯವಸ್ಥೆಯನ್ನು ಹೊಂದಲು, ಹೀಗೆ ಪ್ರತಿಯೊಂದು ಅಗತ್ಯತೆಗಳನ್ನು ಪೂರೈಸಿಕೊಳ್ಳಲು ಗಂಗಾಧರಭಟ್ಟರು ವಿದ್ಯಾರ್ಥಿಗಳಿಗೆ ನೆರವಾಗುತ್ತಿದ್ದರು. ಇಂತಹ ಸಹಸ್ರರಲ್ಲಿ ನಾನೂ ಒಬ್ಬನೆಂದು ಬಿಡಿಸಿ ಹೇಳಬೇಕಾದದ್ದಿಲ್ಲ. ೧೯೮೬ರ ಜೂನ್ ತಿಂಗಳಿನಲ್ಲಿ ನನ್ನ ಮಾವಂದಿರೊಂದಿಗೆ ಶಿರಸಿಯಿಂದ ಬಸ್ನಲ್ಲಿ ಪ್ರಯಾಣಿಸಿ ನೇರವಾಗಿ ಬಂದದ್ದೇ ಗಂಗಾಧರ ಭಟ್ಟರ ಮನೆಗೆ. ಅಂದು ನನ್ನೊಂದಿಗೆ ನನಗೆ ತಿಳಿಯದ ಮತ್ತು ಮುಂದೆ ನನ್ನ ಮಿತ್ರರಾಗಿ ನನ್ನ ಕಾಲೇಜಿನ ಮತ್ತು ಹಾಸ್ಟೇಲಿನ ಸಹವಾಸಿಗಳಾದ ವಿಘ್ನೇಶ್ವರಭಟ್ಟ ಕೋಟೆಮನೆ, ಸತೀಶ ದೇವರು ಹೆಗಡೆ ಗೋಳಿಕೈ, ಗಿರೀಶ ಶಂಕರ ಹೆಗಡೆ ಇವರುಗಳೂ ಬಂದಿದ್ದರು. ಬಂದವರಿಗೆಲ್ಲ ಗಂಗಾಧರ ಭಟ್ಟರೇ ಆಸರೆ-ಆಶ್ರಯ. ಹಾಗೆ ನಾನು ನಾಲ್ಕು ದಿನ ಭಟ್ಟರ ಮನೆಯ ಆಶ್ರಯ ಪಡೆದೆ. ಈ ನಾಲ್ಕು ದಿನಗಳಲ್ಲಿ ತಮ್ಮ ಹಿರಿಯ ಶಿಷ್ಯರುಗಳ ಮೂಲಕ ನನ್ನನ್ನು ನನ್ನಿಷ್ಟದಂತೆ ಜ್ಯೋತಿಷಶಾಸ್ತ್ರಕ್ಕೆ ಅಡ್ಮಿಷನ್ ಮಾಡಿಸಿ ರೂಂ ನಂ. ೧೩ ಕ್ಕೆಸೇರಿಸಿದರು. ಆಗ ವ್ಯಾಕರಣ ಶಾಸ್ತ್ರಾಧ್ಯಯನ ಮಾಡುತ್ತಿದ್ದ ವೆಂಕಟರಮಣ ಭಟ್ಟ ಕಲ್ಮನೆ (ಕಾನಸೂರು) ಇವರು ಗಂಗಾಧರ ಭಟ್ಟರ ಮಾತಿನಂತೆ ನನ್ನನ್ನು ಬಹಳ ಪ್ರೀತಿಯಿಂದ ತಮ್ಮ ರೂಮಿಗೆ ಸೇರಿಸಿಕೊಂಡರು. 

ಹೀಗೆ ನನ್ನ ಮೈಸೂರಿನ ಬದುಕು ಆರಂಭವಾದದ್ದೇ ಗಂಗಾಧರ ಭಟ್ಟರ ಮನೆಯಿಂದ. ಅನಂತರದ ದಿನಗಳಲ್ಲಿ ಗಂಗಾಧರ ಭಟ್ಟರೊಂದಿಗೆ ನಿರಂತರ ಸಂಭಂಧವನ್ನು ಇರಿಸಿಕೊಂಡು ಬಂದವನು ನಾನು. ಅನೇಕ ಮತಭೇದಗಳ ನಡುವೆಯೂ ಗಂಗಾಧರಭಟ್ಟರೊಂದಿಗಿನ ಸಂಬಂಧ ಎಂದೂ ಕಡಿದು ಹೋಗಿಲ್ಲ. ಹೀಗೆ ಮೈಸೂರಿಗೆ ಬಂದ ಸಂಸ್ಕೃತಾಧ್ಯೇತೃಗಳಿಗೆ ಗಂಗಾಧರ ಭಟ್ಟರು ಕಲ್ಪವೃಕ್ಷದಂತಿದ್ದಾರೆಂದರೆ ಅದು ಸತ್ಯಕ್ಕೆ ದೂರವಾದದ್ದಾಗಲಾರದು.

\section*{ಅಧ್ಯಾಪನ ಮತ್ತು ಭಟ್ಟರು}

ನಿರಂತರ ಅಧ್ಯಾಪನ ಗಂಗಾಧರ ಭಟ್ಟರ ಸ್ವರೂಪವೇ ಆಗಿದೆ ಎನ್ನಬಹುದು. ಗಂಗಾಧರ ಭಟ್ಟರು ಮೈಸೂರಿನ ಶಂಕರವಿಲಾಸ ಸಂಸ್ಕೃತ ಪಾಠಶಾಲೆಯಲ್ಲಿ ಅಧ್ಯಾಪನವೃತ್ತಿಯನ್ನು ಆರಂಭಿಸಿದರು. ಕಾವ್ಯಾಂತ ತರಗತಿಯ ಪಾಠವನ್ನು ಮಾಡುವ ಹೊಣೆಯನ್ನು ಹೊತ್ತಿದ್ದ ಅವರು ಮಕ್ಕಳ ಮನಸ್ಸನ್ನು ಮುಟ್ಟುವಂತೆ ಪಾಠಮಾಡುತ್ತಿದ್ದರೂ ಈ ಬಾಲಪಾಠಗಳು ಅವರ ಮನಸ್ಸನ್ನು ಸಂತಸಗೊಳಿಸಲಿಲ್ಲವೇನೋ? ಅದಕ್ಕಾಗಿಯೇ ಅವರು ಪ್ರೌಢಪಾಠವಾದ ವಿದ್ವತ್ಪಠ್ಯಗಳ ಪಾಠವನ್ನು ತಮ್ಮ ಮನೆಯಲ್ಲಿಯೇ ಆರಂಭಿಸಿದರು. ಮಹಾರಾಜ ಸಂಸ್ಕೃತ ಕಾಲೇಜಿನ ವಿದ್ಯಾರ್ಥಿಗಳಿಗೆ ಇವರು ತಮ್ಮ ವಿದ್ವತ್ಪ್ರೌಢಿಮೆಯನ್ನು ಧಾರೆಯೆರೆದರು. ಬಹುಶಃ ಭಟ್ಟರ ವಿರಾಮದ ಸಮಯವೆಲ್ಲಾ ಈ ರೀತಿಯ ಪಾಠಪ್ರವಚನ ಕಾರ್ಯದಲ್ಲೇ ಕಳೆಯುತ್ತಿತ್ತು. ಬಹುಶಃ ಇದು ಭಟ್ಟರಿಗೆ ಆತ್ಮತೃಪ್ತಿಯನ್ನು ನೀಡುವ ವಿಷಯವಾಗಿತ್ತು. ಅಭಿಜಾತ ಪ್ರತಿಭೆಯಾದ ಭಟ್ಟರಿಗೆ ಬಾಲಪಾಠ ಮಾನಸಿಕ ತೃಪ್ತಿಯನ್ನು ನೀಡಲಿಲ್ಲವೇನೋ? ಇದೇ ಅವರ ಪ್ರೌಢ ಪಾಠಕ್ಕೆ ಪ್ರೇರಣೆಯಾದದ್ದು. ದಿನಕರೀಯ ಪಾಠವನ್ನು ಗಂಗಾಧರಭಟ್ಟರಲ್ಲಿ ಹೇಳಿಸಿಕೊಳ್ಳಲು ವಿದ್ಯಾರ್ಥಿಗಳಿಗೆ ಬಲು ಅಚ್ಚುಮೆಚ್ಚು. ಇದಲ್ಲದೇ ನವೀನನ್ಯಾಯಶಾಸ್ತ್ರದ ಮತ್ತು ಪ್ರಾಚೀನನ್ಯಾಯಶಾಸ್ತ್ರದ ಪಠ್ಯಗಳು, ದರ್ಶನಗಳು ಅವರ ಅಚ್ಚುಮೆಚ್ಚಿನ ಪಾಠವಿಷಯಗಳು. ಅವರ ಅಧ್ಯಾಪನ ಪ್ರೀತಿ ಎಂಥದ್ದಾಗಿತ್ತೆಂದರೆ ಯಾರು ಬಂದು ಕೇಳಿದರೂ ಅವರು ಇಲ್ಲವೆಂದ ಉದಾಹರಣೆಯೇ ಇಲ್ಲವೇನೋ. ಅದು ಕೇವಲ ಶಾಸ್ತ್ರಪಾಠಕ್ಕೆ ಮಾತ್ರ ಸೀಮಿತವಾದದ್ದಲ್ಲ. ಏನನ್ನು ಯಾರು ಕೇಳಿಬರುತ್ತಾರೋ ಅದನ್ನವರಿಗೆ ಪಾಠಮಾಡುತ್ತಿದ್ದರು. ನಾನು ಅವರಿಂದ ಪಾಠಮಾಡಿಸಿಕೊಂಡವನೇ. ನಾನಾಗ ಪಿಯುಸಿ ರಾತ್ರಿಕಾಲೇಜಿಗೆ ಹೊಗುತ್ತಿದ್ದೆ ಮತ್ತು ಜ್ಯೋತಿಷವಿದ್ವನ್ಮಧ್ಯಮಾವನ್ನು ಓದುತ್ತಿದ್ದೆ. ಆಗ ನಾನು ಪಿಯುಸಿ ಆಂಗ್ಲಪಾಠವನ್ನು ಮತ್ತು ದಿನಕರೀಯ ಪಾಠವನ್ನು ಭಟ್ಟರಿಂದ ಹೇಳಿಸಿಕೊಂಡೆ. ಅವರ ಪಾಠದ ಶೈಲಿ ಸರಳ ಮತ್ತು ಅನನ್ಯ. ನನ್ನಂತಹ ಮಂದಾಧಿಕಾರಿಗಳಿಗೆ ಮತ್ತು ಉತ್ತಮಾಧಿಕಾರಿಗಳಿಗೆ ಅವರು ತಕ್ಕಪಾಠವನ್ನು ಹೇಳುತ್ತಿದ್ದರು.

ಹೀಗೆ ಸಂಸ್ಕೃತಶಾಸ್ತ್ರ ಪರಂಪರೆಯ ಶಿಷ್ಯರು ಗಂಗಾಧರಭಟ್ಟರಿಗೆ ಸಹಸ್ರಾರು. ಅವರಿಗೆ ನನ್ನಂತಹವರು ಸಲ್ಲಿಸಬಹುದಾದದ್ದು ಕೇವಲ ಗೌರವವೊಂದೇ, ಕೃತಜ್ಞತೆಯ ನುಡಿಯೊಂದೇ. ಹೀಗೆ ನನ್ನಂತಹ ಸಾವಿರಾರು ವಿದ್ಯಾರ್ಥಿಗಳು ಅವರಿಂದ ಕೃತಾರ್ಥರಾಗಿದ್ದಾರೆ. ಅವರ ಪಾಠಪ್ರವಚನದ ಪ್ರೀತಿ ಮತ್ತು ವಿದ್ವತ್ತು ಕೇವಲ ಸಂಸ್ಕೃತ ಶಾಸ್ತ್ರಕ್ಕೆ ಸೀಮಿತವಾಗಿರಲಿಲ್ಲ. ಹಿಂದೆ ತಿಳಿಸಿದಂತೆ ಆಂಗ್ಲಭಾಷೆಯ ಪಾಠಕ್ಕೆ ಅವರು ಹೇಳಿಮಾಡಿಸಿದಂತಿದ್ದರು. ಸಂಸ್ಕೃತದ ವ್ಯಾಕರಣವನ್ನು ವಿವೇಚಿಸಿದವರಿಗೆ ಆಂಗ್ಲ, ವ್ಯಾಕರಣ ಸುಲಲಿತ, ಸುಲಭ. ಸ್ವಯಂ ಪ್ರತಿಭಾಸಂಪನ್ನರಾದ ಭಟ್ಟರಿಗೆ ಆಂಗ್ಲಭಾಷೆಯಲ್ಲಿ ಪಾಂಡಿತ್ಯ ಸಿದ್ಧಿಸಿದ್ದು ಆಶ್ಚರ್ಯಕರವೇನೂ ಅಲ್ಲ. ಹೀಗೆ ಕನ್ನಡ, ಸಂಸ್ಕೃತ, ಹಿಂದೀ ಭಾಷೆಗಳಲ್ಲಿ ಸಾಕಷ್ಟು ಪ್ರಾವೀಣ್ಯವನ್ನು ಪಡೆದ ಬಹುಭಾಷಾ ವಿದ್ವಾಂಸರಾದ ಭಟ್ಟರು ತಮ್ಮ ಪಾಂಡಿತ್ಯವನ್ನೆಂದೂ ತಮ್ಮಲ್ಲಿಯೇ ಹುದುಗಿಸಿಟ್ಟುಕೊಳ್ಳಲು ಬಯಸಲಿಲ್ಲ. ಬದಲಾಗಿ ಅಧ್ಯಾಪನ ಪರಂಪರೆಯಲ್ಲಿ ತಮ್ಮನ್ನು ತೊಡಗಿಸಿಕೊಂಡು ಬಂದರು. ಖ್ಯಾತಿಯನ್ನೋ, ಅರ್ಥಾದಿ ಇತರ ಸಂಪತ್ತನ್ನೋ ಅಪೇಕ್ಷಿಸದೇ ನಿಸ್ವಾರ್ಥದಿಂದ ವಿದ್ಯಾದಾನಮಾಡಿದವರು ಭಟ್ಟರು. ಅವರ ಈ ಸೇವೆ ಆಯುರ್ವೇದ ವಿಷಯದಲ್ಲೂ ಅಡೆತಡೆಯಿಲ್ಲದೇ ಸಾಗಿತ್ತು. ಹಿಂದೊಮ್ಮೆ ಆಯುರ್ವೇದದ ಅಧ್ಯಯನ ಇದೇ ಕಾಲೇಜಿನಲ್ಲಿ ನಡೆಯುತ್ತಿತ್ತಂತೆ. ಅದೇ ಜಾಡನ್ನು ಹಿಡಿದು ಬರುತ್ತಿದ್ದ ಆಯುರ್ವೇದ ವಿದ್ಯಾರ್ಥಿಗಳಿಗೆ ಚರಕಸಂಹಿತಾ, ಸುಶ್ರುತಸಂಹಿತಾ, ಅಷ್ಟಾಂಗಹೃದಯದಂತಹ  ಮೂಲಗ್ರಂಥಗಳನ್ನು ತಲಸ್ಪರ್ಶಿಯಾಗಿ ಬೋಧಿಸುತ್ತಿದ್ದರು. 

ಗಂಗಾಧರಭಟ್ಟರ ಮತ್ತು ಆಯುರ್ವೇದ ಕಾಲೇಜಿನ ಸಂಭಂಧವೇ ಒಂದು ವಿಶಿಷ್ಟಾಧ್ಯಾಯ. ಬಹುಶಃ ನೂರಾರು ಆಯುರ್ವೇದ ವೈದ್ಯರಿಗೆ ಪಾಠಮಾಡಿದ ಕೀರ್ತಿ ಭಟ್ಟರಿಗೆ ಇದೆ. ಅಲ್ಲದೇ ಆಯುರ್ವೇದ ಕಾಲೇಜಿನಲ್ಲಿ ಅನೇಕ ಪ್ರವಚನಗಳನ್ನೂ ಭಟ್ಟರು ನೀಡಿರುವುದು ನನಗೆ ಮರೆಯದಾಗಿದೆ. ಹೀಗೆ ಗಂಗಾಧರ ಭಟ್ಟರ ಅಧ್ಯಾಪನ ಪ್ರೀತಿ ಸಾವಿರಾರು ವಿದ್ಯಾರ್ಥಿಗಳ ಜ್ಞಾನದಾಹವನ್ನು ಪರಿಹರಿಸಿದೆಯೆಂಬುದು, ಅವರು ನಮ್ಮೊಂದಿಗಿದ್ದಾರೆ ಎಂಬುದು ನಮಗೆ ಹೆಮ್ಮೆಯ ವಿಷಯ.

\section*{ನಿರಂತರ ಅಧ್ಯಯನ}

ಗಂಗಾಧರ ಭಟ್ಟರ ವಿದ್ಯಾರ್ಥಿಜೀವನವನ್ನು ನಾನು ಬಲ್ಲವನಲ್ಲವಾದರೂ ಅನುಭವದ ಮಾತುಗಳನ್ನು ಕೇಳಿ ತಿಳಿದಿರುವಂತೆ ಅವರು ಸದಾ ಅಧ್ಯಯನಶೀಲರು. ೧೯೮೬ ರ ಅನಂತರವಂತೂ ನಿರಂತರವಾಗಿ ನಾನು ಅವರ ಸಂಪರ್ಕವನ್ನು ಇಟ್ಟುಕೊಂಡವನಾಗಿದ್ದು ಈ ಸಮಯದಲ್ಲಿ ನನಗೆ ಅನುಭವಕ್ಕೆ ಬಂದಂತೆ ಅವರು ನಿರಂತರ ಅಧ್ಯಯನಶೀಲರು. ಮನೆಯಲ್ಲಿ ಒಂಟಿಯಾಗಿರುವಷ್ಟೂ ಹೊತ್ತು ಅವರು ಸದಾ ಶಾಸ್ತ್ರಗಳ ಚಿಂತನೆಯಲ್ಲಿ ತೊಡಗಿಸಿಕೊಂಡವರು. ಅಧ್ಯಯನ ಮಾಡಿದ ನ್ಯಾಯಶಾಸ್ತ್ರವು ವ್ಯಾಕರಣ, ಮೀಮಾಂಸಾ, ದರ್ಶನಶಾಸ್ತ್ರಗಳ ವಿಸ್ತೃತ ಪರಿಚಯವನ್ನೂ, ಆಳವಾದ ಜ್ಞಾನವನ್ನೂ ಅಪೇಕ್ಷಿಸುವಂಥದ್ದಾದ್ದರಿಂದ ಈ ಎಲ್ಲ ಶಾಸ್ತ್ರಗಳ ಸಾರವತ್ತಾದ ಅಧ್ಯಯನವನ್ನು ಗಂಗಾಧರ ಭಟ್ಟರು ಗಳಿಸಿಕೊಂಡಿದ್ದರು. ಇದಕ್ಕಾಗಿಯೇ ಅವರು ಇಂದಿಗೂ ನಿರಂತರ ಅಧ್ಯಯನಾರ್ಥಿಯಾಗಿದ್ದಾರೆ. ಬಹುಶಃ ಶಾಸ್ತ್ರಚಿಂತನೆಯಿಲ್ಲದಿದ್ದರೆ ಗಂಗಾಧರ ಭಟ್ಟರ ನಾಡಿ ಬಡಿತ ಲಯ ತಪ್ಪಬಹುದು. ಹೀಗೆ ಗಂಗಾಧರ ಭಟ್ಟರು ಸರ್ವಶಾಸ್ತ್ರ, ದರ್ಶನಗಳನ್ನು ಅಧ್ಯಯನ ಮಾಡಿದವರಾಗಿದ್ದಾರೆ. ಷಡಂಗಗಳಲ್ಲೊಂದಾದ ಜ್ಯೋತಿಷ ಮತ್ತು ಧರ್ಮಶಾಸ್ತ್ರಗಳು ತಾರ್ಕಿಕವಾದ ಸರಣಿಯನ್ನು ಹೊಂದಿಲ್ಲವೆಂದು ಅವರು ಈ ಶಾಸ್ತ್ರಗಳ ಕುರಿತು ಅಷ್ಟಾಗಿ ಆಸಕ್ತಿಯನ್ನು ಹೊಂದಿಲ್ಲದಿರುವುದು ನನಗೆ ವೇದ್ಯವಾದ ಸಂಗತಿ. ಆದರೆ ಜ್ಯೋತಿಷ ಶಾಸ್ತ್ರದ ಗಣಿತಸ್ಕಂಧ ಅವರ ಅಧ್ಯಯನಶೀಲ ಮನಸ್ಸನ್ನು ಸೆಳೆದಿತ್ತು. ಹಾಗೆ ಜೀವನಾನುಭವವಾದಂತೆ ತಮ್ಮ ಉತ್ತರ ವಯಸ್ಸಿನಲ್ಲಿ ಭಟ್ಟರು ಧರ್ಮಶಾಸ್ತ್ರದ ಕಡೆಗೆ ವಾಲಿರುವುದು ಅತ್ಯಂತ ಸಹಜ. ಪ್ರತಿಯೊಬ್ಬ ಸಂಸ್ಕಾರವಂತ ವಿದ್ವಾಂಸನೂ ಒಂದಿಲ್ಲೊಂದು ದಿನ ಧರ್ಮಶಾಸ್ತ್ರಕ್ಕೆ ಶರಣಾಗಲೇಬೇಕು.

ಏಕೆಂದರೆ ಧರ್ಮಶಾಸ್ತ್ರವೆಂದರೆ ಅದು ಜೀವನಶಾಸ್ತ್ರ. ಅದನ್ನರಿಯದೇ ಅದನ್ನು ಸ್ಮರಿಸದೇ ಜೀವನ ಹೇಯವಾದೀತು. ಆದ್ದರಿಂದಲೇ ಬೊಧಾಯನರು ಮನುಷ್ಯನ ಮನಸ್ಸನ್ನು ಅರಿತವರಾಗಿ ಹೀಗೆ ಹೇಳಿದ್ದಾರೆ.
\begin{verse}
ಯದುತ್ತರೇ ಚೇದ್ವಯಸಿ ಸಾಧುವೃತ್ತಃ ತದೇವ ಭವತಿ ನೇತರಾಣಿ ಎಂದು.
\end{verse}
ಅಂದರೆ ಮಾನವನು ಕಾಮಾದಿ ಅರಿಷಡ್ವೈರಿಗಳ ಬಲೆಯಲ್ಲಿ ಸಿಕ್ಕಿ ಪೂರ್ವವಯಸ್ಸಿನಲ್ಲಿ ಧರ್ಮಾನುಷ್ಠಾನ, ಚಿಂತನೆಗಳನ್ನು ಮಾಡದಿದ್ದರೂ ಆತನು ತನ್ನ ಉತ್ತರವಯಸ್ಸಿನಲ್ಲಿ ಸಾಧುವೃತ್ತನಾದಲ್ಲಿ ಅವನಿಗೆ ಸಾಧುವೃತ್ತದ ಫಲವೇ ದೊರೆಯುತ್ತದೆ ಎಂದಿದ್ದಾರೆ. ಇದರ ಅಭಿಪ್ರಾಯ ಇಷ್ಟೇ. ಪ್ರತಿಯೊಬ್ಬನೂ ಧರ್ಮಜ್ಞಾನವನ್ನು ಹೊಂದಿ ಧರ್ಮಾನುಷ್ಠಾನದಿಂದ ತನ್ನನ್ನು ಪಾವನಗೊಳಿಸಿಕೊಳ್ಳಲಿ ಎಂದು. ಇದೇ ಮಾನವನ ಸಾರ್ಥಕತೆ, ಪುರುಷಾರ್ಥ. ತರ್ಕವ್ಯಾಕರಣಾದಿ ಶಾಸ್ತ್ರಗಳು ವೇದೋಕ್ತ ಧರ್ಮಾನುಷ್ಠಾನಗಳಿಗೆ ಉಪಕಾರಕಗಳಾದುದರಿಂದ ಧರ್ಮಾನುಷ್ಠಾನವೇ ಪ್ರಧಾನವಾದುದರಿಂದ ಧರ್ಮಶಾಸ್ತ್ರಕ್ಕೆ ಪ್ರಾಧಾನ್ಯ ಬಂದಿದೆ. ಅದಕ್ಕಾಗಿಯೇ ಬೃಹಸ್ಪತಿ ಹೀಗೆ ಹೇಳುತ್ತಾನೆ-
\begin{verse}
ತಾವಚ್ಛಾಸ್ತ್ರಾಣಿ ಶೋಭಂತೇ ತರ್ಕವ್ಯಾಕರಣಾನಿ ತು |\\
ಧರ್ಮಾರ್ಥಮೋಕ್ಷೋಪದೇಷ್ಟಾ ಮನುರ್ಯಾವನ್ನ ದೃಶ್ಯತೇ || ಎಂದು.
\end{verse}
ಇದಕ್ಕನುಗುಣವಾಗಿ ಗಂಗಾಧರ ಭಟ್ಟರೂ ಧರ್ಮಶಾಸ್ತ್ರದ ಸೆಳೆತಕ್ಕೆ ಸಿಕ್ಕಿ ’ಶಾಸ್ತ್ರವೆಂದರೆ ಧರ್ಮಶಾಸ್ತ್ರ’ವೆಂದು ನಂಬಿ ಈ ಶಾಸ್ತ್ರಜ್ಞಾನಕ್ಕಾಗಿ ಅದರ ಅಧ್ಯಯನದಲ್ಲಿ ತೊಡಗಿ ತಮ್ಮ ಜಿಜ್ಞಾಸೆಯನ್ನು ನೀಗಿಸಿಕೊಂಡಿದ್ದಾರೆ. ಹೀಗೆ ಸದಾ ಅಧ್ಯಯನಶೀಲತೆಯಿಂದ ಸರ್ವಶಾಸ್ತ್ರಪರಿಚಯವನ್ನು ಮಾಡಿಕೊಂಡವರು ಗಂಗಾಧರ ಭಟ್ಟರು. ಅವರು ನಮಗೆಲ್ಲಾ ಆದರ್ಶ. “ಕಾವ್ಯಶಾಸ್ತ್ರವಿನೋದೇನ ಕಾಲೋ ಗಚ್ಛತಿ ಧೀಮತಾಮ್” ಎನ್ನುವ ಸೂಕ್ತಿ, ಗಂಗಾಧರ ಭಟ್ಟರಿಗೆ ಪೂರ್ಣವಾಗಿ ಅನ್ವಯ ಹೊಂದುವಂಥದ್ದು. ಬಹುಶಃ “\textbf{ಅನಭ್ಯಾಸೇ ವಿಷಂ ಶಾಸ್ತ್ರಮ್}” ಎನ್ನುವ ಸುಭಾಷಿತ ಅವರ ರಕ್ತಗತವಾಗಿಯೇ ಬಂದಿರಬೇಕು. ಅದಕ್ಕಾಗಿಯೇ ಅವರೆಂದೂ ಶಾಸ್ತ್ರಾಧ್ಯಯನವನ್ನು ಬಿಟ್ಟವರಲ್ಲ. ಸಣ್ಣ ವಯಸ್ಸಿನಿಂದಲೇ ತಮ್ಮ ಸಹೊದರಿಯನ್ನು ವಿದ್ಯಾಭ್ಯಾಸಕ್ಕಾಗಿ ಕರೆಸಿಕೊಂಡು ಅವರ ಸಂಪೂರ್ಣ ಜವಾಬ್ದಾರಿಯನ್ನು ಹೊತ್ತುಕೊಂಡು, ತಮ್ಮ ಕುಟುಂಬದ ಬಂಧು, ಬಾಂಧವರ ಅನೇಕರ ಮಕ್ಕಳನ್ನು ತಮ್ಮ ಮನೆಯಲ್ಲಿರಿಸಿಕೊಂಡು ಅವರ ಜವಾಬ್ದಾರಿಯನ್ನೆಲ್ಲ ನಿರ್ವಹಿಸುವಾಗಲೂ ಭಟ್ಟರು ತಮ್ಮ ಅಧ್ಯಯನಾಧ್ಯಾಪನಗಳನ್ನು ಮರೆತವರಲ್ಲ. ಇಂದಿನ ಈ ನಿವೃತ್ತಿ ವಯಸ್ಸಿನಲ್ಲೂ ಅವರ ಈ ಅಧ್ಯಯನ ಅಧ್ಯಾಪನದ ಪ್ರೆಮ ನಮಗೆಲ್ಲ ಆದರ್ಶ. ಇದಕ್ಕಾಗಿ ಅವರು ಸದಾ ನಮಗೆ ಅನುಗಮ್ಯರು.

\section*{ಅಭಿಜಾತ ಪ್ರತಿಭೆ}  

“ಪ್ರಜ್ಞಾ ನವನವೋನ್ಮೇಷಶಾಲಿನೀ ಪ್ರತಿಭಾ ಮತಾ” ಇದು ಪ್ರತಿಭೆಯ ಸರ್ವಗ್ರಾಹಿಯಾದಿ ನಿರ್ವಚನ. ಇಂತಹ ಪ್ರತಿಭೆ ಅಭ್ಯಾಸದಿಂದ ಉಂಟಾಗುತ್ತದೆಂದು ಕೆಲವರು ಹೇಳಿದರೆ ಇನ್ನು ಕೆಲವರು ಇದು ಜನ್ಮದಿಂದಲೇ ಸಿದ್ಧವಾಗಿರುತ್ತದೆ ಎಂದು ಅಭಿಪ್ರಾಯ ಪಡುತ್ತಾರೆ. ಅವರಲ್ಲಿ ಹೇಮಚಂದ್ರ ತನ್ನ ಕಾವ್ಯಾನುಶಾಸನದಲ್ಲಿ ಈ ವಿಷಯವನ್ನು ಸ್ಪಷ್ಟಪಡಿಸುತ್ತಾ ಪ್ರತಿಭೆ ಜನ್ಮತಃ ಸಿದ್ಧವಾಗಬೆಕಾದುದು. ಅಭ್ಯಾಸದಿಂದ ಅದು ಪ್ರಖರವಾಗುತ್ತದೆ ಎಂದು ಹೇಳುತ್ತಾನೆ. ಇದೇ ನಿಜ. ಪ್ರತಿಭೆಯೆಂದರೆ ಅದೇನೂ ಅಭ್ಯಾಸದಿಂದ ಉಂಟಾಗುವ ಪಟುತ್ವವಲ್ಲ. ಅದಕ್ಕಿಂತ ಭಿನ್ನವಾದ ಪ್ರಜ್ಞೆಯದು, ಹೊಸತನ್ನು ಹುಟ್ಟುಹಾಕುವ ಶಕ್ತಿ. ಇಂತಹ ಪ್ರಜ್ಞೆ ಶಕ್ತಿವಿಶೇಷವನ್ನು ಹೊಂದಿದವರು ಗಂಗಾಧರ ಭಟ್ಟರು. ಕೆಲ ವಿದ್ವಾಂಸರು ಮೂಲ ಗ್ರಂಥಗಳ ಧಾರಣಾಶಕ್ತಿಯಿಂದ ಬೆಳಗಿದರೆ ಇನ್ನು ಕೆಲವರು ಹೊಸ ಹೊಸ ಕಾವ್ಯಾದಿಗಳನ್ನು ಸೃಷ್ಟಿಸಿ ಕೀರ್ತಿಗಳಿಸುತ್ತಾರೆ. 

ಆದರೆ ಇವೆರಡನ್ನು ಹೊರತುಪಡಿಸಿದಂತೆ ಗಂಗಾಧರ ಭಟ್ಟರದು ವಿಶಿಷ್ಟವಾದ ಚಿಂತನಾ ಶಕ್ತಿ. ಅವರ ಬಗ್ಗೆ ಜಿ. ಮಂಜುನಾಥರು, “ಗಂಗಾಧರ ಭಟ್ಟರು ಪ್ರತ್ಯುತ್ಪನ್ನ ಮತಿಗಳು” ಎಂದು ಹೇಳುತ್ತಿರುತ್ತಾರೆ, ಬಹುಶಃ ಗಂಗಾಧರ ಭಟ್ಟರು ಕೀರ್ತಿಯನ್ನು ಹೊಂದಲು ಅವರ ಈ ಪ್ರತ್ಯುತ್ಪನ್ನಮತಿತ್ವವೂ ಒಂದು ಕಾರಣ. ಯಾವುದೇ ವಿಷಯ ಇರಲಿ ಅದು ಅವರ ಗಮನಕ್ಕೆ ಬಂದಕೂಡಲೇ ಅವರಲ್ಲಿ ಆ ವಿಷಯದ ಬಗೆಗೆ ಅನೇಕ ಸಂಶಯ ಕೋಟಿಗಳು ಅನೇಕ ಯೋಚನೆಗಳು ಹುಟ್ಟಿಕೊಂಡುಬಿಡುತ್ತವೆ. ಏಕಸಂಬಂಧಿಜ್ಞಾನಮ್ ಅಪರಸಂಬಧಿಸ್ಮಾರಮ್ ಎಂಬಂತೆ ವಿಷಯದ ಸಂಬಂಧ ಪ್ರಶ್ನಾನುಪ್ರಶ್ನೆಗಳನ್ನು ಮಾಡುವ ತನ್ಮೂಲಕ ವಿಷಯವನ್ನು ಕಡೆಯುವ ಅವರ ಪ್ರತಿಭೆ ಅನನ್ಯ. ಇದು ಅವರ ಜನ್ಮತಃ ಪ್ರಾಪ್ತವಾದದ್ದು. ಇದೇ ಶಕ್ತಿ ಅವರಿಗೆ ಯಶಸ್ಸನ್ನೂ ಕೀರ್ತಿಯನ್ನೂ ಗೌರವಾದರಗಳನ್ನು ತಂದುಕೊಟ್ಟಿದ್ದು. ಅಂಥಹ ಅಭಿಜಾತ ಪ್ರತಿಭಾ ಸಂಪನ್ನರು ನಮ್ಮ ಶೈಕ್ಷಣಿಕ ಪರಿಸರದಲ್ಲಿ ನಮಗೆ ಸಿಕ್ಕಿರುವುದು ನಮ್ಮ ಅದೃಷ್ಟ ಎಂದು ನನಗನಿಸುತ್ತಿದೆ.

\section*{ಗಂಗಾಧರ ಭಟ್ಟರು ಮತ್ತು ಆರ್ಥಿಕತೆ}

ಮೂಲತಃ ದೊಡ್ಡ ಪುರೊಹಿತ ಕುಟುಂಬದಲ್ಲಿ ಹುಟ್ಟಿಬೆಳೆದ ಭಟ್ಟರಿಗೆ ದಾರಿದ್ರ್ಯ ಯಾವಾಗಲೂ ಪ್ರಾಪ್ತವಾಗಿಲ್ಲದಿದ್ದರೂ ಸಾಕಷ್ಟು ಆರ್ಥಿಕ ಕಷ್ಟಗಳನ್ನು ಕಂಡವರು ಭಟ್ಟರು. ಶೌಚಗಳಲ್ಲೆಲ್ಲ ಅರ್ಥಶೌಚವು ಪ್ರಧಾನವಾದದ್ದೆಂದು ಮನು ಹೇಳುತ್ತಾನೆ. “ಸರ್ವೆಷಾಮೇವ ಶೌಚಾನಾಮರ್ಥಶೌಚಂ ಪರಂ ಸ್ಮೃತಮ್”|| ಯೊ ಅರ್ಥೆ ಶುಚಿಃ ಸ ಶುಚಿಃ ನ ಮೃದ್ವಾರಿಶುಚಿಃ ಶುಚಿಃ | ಇದು ಮನುವಿನ ಸ್ಪಷ್ಟ ನುಡಿ. 

ಮಣ್ಣು ನೀರು ಮೊದಲಾದವುಗಳಿಂದ ಸಂಪಾದಿಸುವ ಶುಚಿ ಶುಚಿಯಲ್ಲ, ಆದರೆ ಅರ್ಥ ವಿಷಯದಲ್ಲಿ ಯಾರು ಶುಚಿಯೋ ಅವರು ಶುಚಿ. ಎಲ್ಲಾ ಶೌಚಕ್ಕಿಂತಲೂ ಅರ್ಥಶೌಚ ಉತ್ಕೃಷ್ಟವಾದುದು ಎಂದು ಹೇಳುತ್ತಾನೆ. ಬೃಹಸ್ಪತಿಯೇ ಮೊದಲಾದವರೂ ಇದನ್ನು ಯಥಾವತ್ತಾಗಿ ಅಂಗೀಕರಿಸಿ ಹೇಳಿದ್ದಾರೆ. ವ್ಯಾಸರು ಇದನ್ನೆ ಚತುರ್ವಿಧ ಶೌಚಗಳನ್ನು ಹೇಳುತ್ತಾ ಕಾಯಿಕ ವಾಚಿಕ ಮಾನಸಿಕ ಶುದ್ಧಿಯೊಂದಿಗೆ ದ್ರವ್ಯ ಶುದ್ಧಿಯನ್ನು ಹೇಳಿದ್ದಾರೆ. ಇಂತಹ ದ್ರವ್ಯಶುದ್ಧಿ ಅಥವಾ ಅರ್ಥಶೌಚವೆಂದರೆ ಧರ್ಮಾನುಸಾರವಾಗಿ ಅರ್ಥಸಂಪಾದನೆ ಮಾಡುವುದು,

ಲಭ್ಯವಾದ ಅಲ್ಪಾರ್ಥದಿಂದಲೂ ತೃಪ್ತಿಹೊಂದುವುದು. ಅಸತ್ಪ್ರತಿಗ್ರಹ ಮಾಡದಿರುವುದು, ಅಹರಹರ್ದಾನ ಮಾಡುವುದು ಇವೆಲ್ಲವೂ ಸೇರುತ್ತದೆ. ಇಂತಹ ಅರ್ಥ ಶೌಚವನ್ನು ಹೊಂದಿದವರು ಗಂಗಾಧರ ಭಟ್ಟರು. ಅವರು ಇಷ್ಟ ಪಟ್ಟಿದ್ದರೆ ಅವರ ಪ್ರತಿಭೆ, ವಿದ್ವತ್ತು ಪೌರೋಹಿತ್ಯಗಳಿಂದ ಹಣ ಸಂಪಾದನೆ ಮಾಡಬಹುದಾಗಿತ್ತು. ಆದರೆ ಅವರೆಂದೂ ಹಣ ಸಂಪಾದನೆಯ ಹಾದಿ ಹಿಡಿಯಲಿಲ್ಲ. ಪರಂತು ತಮಗೆ ಸಿಕ್ಕಿದ ಚಿಕ್ಕ ಅರ್ಥ ಮೂಲವಾದ ಪ್ರಾಥಮಿಕ ಹಂತದ ಪಾಠಶಾಲಾ ಅಧ್ಯಾಪಕ ವೃತ್ತಿಯಲ್ಲೇ ತೃಪ್ತಿಯನ್ನು ಕಂಡವರು. ಮುಂದೆ ೧೯೯೭-೯೮ರಲ್ಲಿ ಸರ್ಕಾರಿ ಮಹಾರಾಜ ಸಂಸ್ಕೃತ ಕಾಲೇಜಿಗೆ ನೇಮಕಾತಿ ಹೊಂದಿದಾಗಲೂ ಅದೇ ತೃಪ್ತಿ. ಆದರೆ ಕಾಲೇಜಿಗೆ ನೇಮಕಾತಿ ಹೊಂದಿದಾಗ ಅವರು ತೃಪ್ತಿಪಟ್ಟಿದ್ದು ಅರ್ಥಸಂಪಾದನೆ ಹೆಚ್ಚಾಯಿತು ಎಂದಲ್ಲ. ಅದರ ಬದಲಾಗಿ ವಿದ್ವತ್ತರಗತಿಯ ವಿದ್ಯಾರ್ಥಿಗಳಿಗೆ ಶಾಸ್ತ್ರಪಾಠವನ್ನು ಮಾಡಬಹುದಲ್ಲಾ ಎನ್ನುವ ಕಾರಣಕ್ಕಾಗಿ. ಅವರನ್ನು ಅನೇಕರೀತಿಯಲ್ಲಿ ಅನೇಕ ಪರಿಸರಗಳಲ್ಲಿ ಅನೇಕ ವ್ಯವಹಾರಗಳಲ್ಲಿ ಈಕ್ಷಿಸಿ ನನಗೆ ಅನುಭವಕ್ಕೆ ಬಂದದ್ದು ಈ ಅಂಶ - ಅದು ಅವರ ಅರ್ಥ ಶೌಚ. ಇದು ನಮಗೆಲ್ಲ ಆದರ್ಶ 

\section*{ಗಂಗಾಧರ ಭಟ್ಟರು ಮತ್ತು ಅದೃಷ್ಟ - ದೈವ}

ಯಾಜ್ಞವಲ್ಕ್ಯರು ತಮ್ಮ ಸ್ಮೃತಿಯಲ್ಲಿ ಈ ದೈವ/ಅದೃಷ್ಟ ಮತ್ತು ಪುರುಷ ಪ್ರಯತ್ನ ಇವೆರಡೂ ಕರ್ಮ ಸಿದ್ಧಿಗೆ ಅಂದರೆ ಕರ್ಮಫಲಸಿದ್ಧಿಗೆ ಕಾರಣವೆಂದು ಹೇಳಿದ್ದಾರೆ.
\begin{verse}
ದೈವೇ ಪುರುಷಕಾರೇ ಚ ಕರ್ಮಸಿದ್ಧಿರ್ವ್ಯವಸ್ಥಿತಾ |\\
ತತ್ರ ದೈವಮಭಿವ್ಯಕ್ತಂ ಪೌರುಷಂ ಪೌರ್ವದೇಹಿಕಮ್ ||
\end{verse}
ಇದು ಅವರ ವಚನ. ಇಲ್ಲಿ ದೈವವೆಂದರೆ, ಪುರುಷ ತನ್ನ ಪೂರ್ವಜನ್ಮದ ದೇಹದಿಂದ (ಕರ್ಮದಿಂದ) ಪುರುಷ ಪ್ರಯತ್ನದಿಂದ ಪಡೆದ ಕರ್ಮಫಲದ ಅಭಿವ್ಯಕ್ತರೂಪ. ಹಿಂದಿನ ಜನ್ಮಗಳಲ್ಲಿ ನಾವು ಮಾಡಿದ ಕರ್ಮಗಳ ಫಲದ ವ್ಯಕ್ತರೂಪವೇ ದೈವ ಅಥವಾ ನಾವು ಕರೆಯುವ ಅದೃಷ್ಟ ಅಥವಾ ಭಾಗ್ಯ. ಸಂಚಿತ, ಆಗಾಮಿ ಪ್ರಾರಭ್ಧವೆಂಬ ಮೂರು ವಿಧದ ಕರ್ಮಫಲಗಳಲ್ಲಿ ಪ್ರಾರಭ್ಧ ಕರ್ಮಫಲವೇ ದೈವ ಶಬ್ದ ವಾಚ್ಯವಾಗಿದೆ. ಇದು ಈ ಜನ್ಮದ ಪುರುಷ ಪ್ರಯತ್ನದೊಂದಿಗೆ ಫಲವನ್ನು ನೀಡುತ್ತದೆ. ಅಂದರೆ ಆಯಾ ಫಲಭೋಗಕ್ಕೆ ಅಗತ್ಯವಾದ ಪ್ರಯತ್ನವನ್ನು ನಾವು ಮಾಡಿದಾಗ ಮಾತ್ರ ಅದು ನಮಗೆ ಫಲರೂಪದಲ್ಲಿ ಸಿದ್ಧಿಸುತ್ತದೆ. ಇಲ್ಲಿ ಧರ್ಮಜ್ಞರು ಹೇಳುವಂತೆ ಕೆಲವರು ದೈವಮಾತ್ರದಿಂದ, ಸ್ವಭಾವ ಮಾತ್ರದಿಂದ, ಕಾಲಮಾತ್ರದಿಂದ ಪುರುಷ ಪ್ರಯತ್ನ ಮಾತ್ರದಿಂದ ಫಲಸಿದ್ಧಿಯಾಗುತ್ತದೆಯೆಂದು ಹೇಳುತ್ತಾರೆ. ಇನ್ನು ಕೆಲವರು ಇವುಗಳೆಲ್ಲದರ ಸಂಯೋಗದಿಂದ ಫಲಸಿದ್ಧಿಯನ್ನು ಹೇಳುತ್ತಾರೆ. ಇಲ್ಲಿ ಯಾಜ್ಞವಲ್ಕ್ಯರು ತಮ್ಮ ಅಭಿಮತವನ್ನು ಹೇಳುತ್ತಾ” ಹೇಗೆ ರಥವು ಒಂದೇ ಚಕ್ರದಿಂದ ಚಲಿಸಲಾರದೋ ಹಾಗೇ ದೈವವು ಪುರುಷಕಾರದ ಸಂಯೋಗದ ಹೊರತು ಸಿದ್ಧಿಸಲಾರದು ಎಂದು ಹೇಳುತ್ತಾರೆ. ಈ ಬಗ್ಗೆ ಯೋಗವಾಸಿಷ್ಠಕಾರರು ಬಹುವಾಗಿ ಚರ್ಚಿಸಿದ್ದಾರೆ. ಅದರೂ ನಮ್ಮ ಅನುಭವ ದೈವವೇ ಪ್ರಬಲವೆಂದು ಹೇಳುತ್ತದೆ. ಕೆಲವರು ಯಾವುದೋ ಫಲಕ್ಕಾಗಿ ಎಷ್ಟೆಷ್ಟೋ ಪ್ರಯತ್ನಿಸಿದರೂ ಫಲಸಿದ್ದಿಯನ್ನು ಕಾಣಲಾರರು. ಆದರೆ ಇನ್ನು ಕೆಲವರಿಗೆ ಅಳಿಲು ಪ್ರಯತ್ನಮಾತ್ರದಿಂದಲೇ ಫಲಸಿದ್ಧಿ ಉಂಟಾಗುತ್ತದೆ. ಇಂತಹ ನಮ್ಮ ಅನುಭವವೇ ದೈವವು ಪ್ರಧಾನ, ಅದೃಷ್ಟ ಪ್ರಧಾನವಾದದ್ದು ಎಂದು ದೃಢಪಡಿಸುತ್ತದೆ. ದೈವವಿದ್ದಲ್ಲಿ ಅಳಿಲುಪ್ರಯತ್ನಮಾತ್ರದಿಂದ ಅದು ಸಿದ್ಧಿಯಾಗುತ್ತದೆ. ಆದರೆ ಅಷ್ಟಾದರೂ ಪುರುಷ ಪ್ರಯತ್ನವಿಲ್ಲದಿದ್ದರೆ ಯವುದೇ ದೈವ ಸಿದ್ಧಿಸಲಾರದು ಎನ್ನುವುದೇ ಆಚಾರ್ಯರ ಅಭಿಪ್ರಾಯ. 

ಆದರೆ ಇಂತಹ ಅದೃಷ್ಟವೂ ತನ್ನದೇ ಆದ ಕಾಲವನ್ನು ಅಪೇಕ್ಷಿಸುತ್ತದೆ. ಎನ್ನುವುದೂ ಅಷ್ಟೇ ಸತ್ಯ. ಆದ್ದರಿಂದಲೇ ಸುಭಾಷಿತಕಾರರು ಸುಂದರವಾದ ಮಾತೊಂದನ್ನು ಹೀಗೆ ಹೇಳುತ್ತಾರೆ. ಅವರು ಹೇಳುವಂತೆ --
\begin{verse}
ನೈವಾಕೃತಿಃ ಫಲತಿ ನೈವ ಕುಲಂ ನ ಶೀಲಂ ವಿದ್ಯಾಪಿ ನೈವ ನ ಚ ಯತ್ನ ಕೃತಾಪಿ ಸೇವಾ |\\
ಭಾಗ್ಯಾನಿ ಪೂರ್ವತಪಸಾ ಖಲು ಸಂಚಿತಾನಿ ಕಾಲೇ ಫಲಂತಿ ಪುರುಷಸ್ಯ ಯಥೈವ ವೃಕ್ಷಾಃ ||
\end{verse}
ಅಂದರೆ ಪುರುಷನ ಶರೀರಾಕೃತಿಯಾಗಲಿ, ಕುಲವಾಗಲೀ, ಶೀಲವಾಗಲೀ, ವಿದ್ಯೆಯಾಗಲೀ, ಪ್ರಯತ್ನದಿಂದ ಮಾಡಿದ ಸೇವೆಯಾಗಲೀ ಫಲವನ್ನು ನೀಡಲಾರದು, ಆದರೆ ಪುರುಷನ ಪೂರ್ವ ತಪಸ್ಸಿನ (ಕರ್ಮದ) ಫಲದಿಂದ ಸಂಗ್ರಹಿಸಲ್ಪಟ್ಟ, ಭಾಗ್ಯಗಳು, ಅದೃಷ್ಟ ದೈವವು ಸಕಾಲದಲ್ಲಿ ಮಾತ್ರ ಫಲಿಸುತ್ತವೆ, ವೃಕ್ಷಗಳಂತೆ. 

ಈ ಮಾತನ್ನು ಗಂಗಾಧರ ಭಟ್ಟರ ಸಂಬಂಧವಾಗಿ ನಾನು ಏಕೆ ಸ್ಮರಿಸುತ್ತೇನೆಂದರೆ, ಸುಂದರಾಕೃತಿ ಹೊಂದಿದವರು ಕುಲೀನರು, ಶೀಲವಂತರು ವಿದ್ಯಾವಂತರು, ಪ್ರಯತ್ನಪೂರ್ವಕವಾಗಿ ಕೆಲಸ ಮಾಡುವವರು. ಆದರೂ ಅವರಾದೃಷ್ಟ ಸಾಲದೇನೋ. ಅವರ ಯಾವ ಕುಲಶೀಲಾದಿ ಯೊಗ್ಯತೆಗಳೂ ಅವರಿಗೆ ತಕ್ಕುದಾದ ಸ್ಥಾನಾದಿಗಳನ್ನೂ, ಆರ್ಥಿಕ ಸ್ಥಾನ ಮಾನಾದಿಗಳನ್ನೋ ಯಾವುದನ್ನೂ ಕೊಡಲಿಲ್ಲ. ಕೇವಲ ಅವರ ಪ್ರ‍ತಿಭೆಯ ಮೂಲಕ ಒಂದಿಷ್ಟು ಶುಷ್ಕ ಗೌರವ ಮಾತ್ರ ಸಿಕ್ಕಿದೆಯೆನ್ನಬಹುದು. ಆದರೆ ಅವರೆಂದೂ ಸ್ಥಾನ ಮಾನ ಅರ್ಥಗಳಿಗೋಸ್ಕರ ಆಸೆ ಪಟ್ಟವರಲ್ಲ. ಬಹುಶಃ ಅಸೆ ಪಡದಿದ್ದರೆ ಏನೂ ಸಿಗಲಾರದು ಎನ್ನುವುದಕ್ಕೆ ಭಟ್ಟರೂ ಒಂದು ಉದಾಹರಣೆಯಿರಬಹುದು. ಶಿಶುಪಾಲವಧ ಮಹಾಕಾವ್ಯದಲ್ಲಿ ಕೃಷ್ಣನ ಬಾಯಿಂದ ಒಂದು ಮಾತು ಬರುತ್ತದೆ ಅದೆಂದರೆ ಬಯಸದೇ ಏನೂ ದೊರೆಯದು. ಸಮುದ್ರ ತನ್ನಲ್ಲಿ ಸಾಕಷ್ಟು ನೀರನ್ನು ಹೊಂದಿದ್ದರೂ ಬರುವ ನದಿಗಳನ್ನು ನಿರಾಕರಿಸುವುದಿಲ್ಲ. ಅಂತೆಯೇ ನಾವು ಏನನ್ನಾದರು ಬಯಸಿದರೆ ಮಾತ್ರ ಅದು ಪ್ರಾಪ್ತಿಯಾಗಬಹುದು ಎಂದು. ಅದೇ ರೀತಿ ಮನುವು ಕೂಡ “ಕಾಮಾತ್ಮತಾ ನ ಪ್ರಶಸ್ತಾ” ಎಂದು ಹೇಳಿದರೂ “ ನ ಚೈವೇಹಾಸ್ತ್ಯಕಾಮತಾ. ಕಾಮ್ಯೋ ಹಿ ವೇದಾಧಿಗಮಃ ಕರ್ಮ ಯೋಗಶ್ಚ ಶಾಶ್ವತಃ ಎಂದು ಅಗತ್ಯವಾದದ್ದನ್ನು ಅಲಭ್ಧವಾದದ್ದನ್ನು ಬಯಸಬೇಕೆಂದು ಅಭಿಪ್ರಾಯಪಡುತ್ತಾನೆ. ರಾಜ ಧರ್ಮವನ್ನು ಉಪದೇಶಿಸುತ್ತಾ ಅಲಭ್ಧಮೀಹೇದ್ಧರ್ಮೆಣ ಎಂದೂ ಉಪದೇಶಿಸುತ್ತಾನೆ. ಒಟ್ಟಾರೆ ಭಟ್ಟರ ಬಯಕೆಯ ಅಭಾವದಿಂದಲೋ ದೈವಕಾರಣದಿಂದಲೋ ಅವರ ಯೊಗ್ಯತೆಗೆ ತಕ್ಕುದಾದದ್ದು ಅವರಿಗೆ ತಲುಪಿಲ್ಲ ಎಂಬುದು ನನಗೆ ಬೇಸರ ತರಿಸುತ್ತದೆ.

\section*{ಗಂಗಾಧರ ಭಟ್ಟರ ಆತಿಥ್ಯ}

ಗಂಗಾಧರ ಭಟ್ಟರ ಮನೆಯೆಂದರೆ ಸದಾ ನೆಂಟರು ಇಷ್ಟರು ಆಪ್ತರು ಶಿಷ್ಯರು ಅತಿಥಿಗಳಿಂದ ತುಂಬಿರುವ ಮನೆ. ಧರ್ಮಶಾಸ್ತ್ರದಲ್ಲಿ ವಿಧಿಸುವಂತೆ ಅತಿಥಿಯಜ್ಞ ಒಂದು ಮಹಾಯಜ್ಞ. ಹಿಂದಿನ ಕಾಲದಂತೆ ಇಂದು ಅತಿಥಿ ಯಜ್ಞ ನಡೆಯದಿದ್ದರೂ ಅದು ಬೇರೆಯದೇ ಸ್ವರೂಪವನ್ನು ಹೊಂದಿದೆ. ಸುಸಂಸ್ಕೃತ ಪುರೋಹಿತ ಕುಟುಂಬದ ಭಟ್ಟರಿಗೆ ಇಂತಹ ಸದಾಚಾರ ರಕ್ತಗತವಾಗಿಯೇ ಬಂದಿರಬಹುದು. ಆಪಸ್ತಂಭರು ಅತಿಥಿಧರ್ಮವನ್ನು ಉಪದೇಶಿಸುತ್ತಾ “ಕಾಲೇ ಸ್ವಾಮಿನೌ ಅನ್ನಾರ್ಥಿನಂ ನ ಪ್ರತ್ಯಾಚಕ್ಷೀತ” ಎಂದು ಹೇಳಿ “ಅಭಾವೇ ಭೂಮಿರುದಕಂ ತೃಣಾನಿ ಕಲ್ಯಾಣೀವಾಗಿತ್ಯೇತಾನಿ ವೈ ಸತೋಽಽಗಾರೇ ನ ಕ್ಷೀಯಂತೇ ಕದಾಚನೇತಿ” ಎಂದು ಹೇಳುತ್ತಾರೆ. ಗಂಗಾಧರ ಭಟ್ಟರಿಗೆ ಬ್ರಹ್ಮಚರ್ಯಕ್ರಮದಿಂದಲೂ ಈ ಧರ್ಮದ ಬಗ್ಗೆ ಪ್ರೀತಿ. ಆ ಕಾಲದಿಂದಲೂ ಅವರು ಮನೆಗೆ ಬಂದವರಿಗೆ ಕಾಲದಲ್ಲಿ ಭೊಜನ, ಉಪಾಹಾರ ಚಾಪೆ , ಹಾಸಿಗೆ ವಾಸವ್ಯವಸ್ಥೆ ಮಂಗಲಮಾತುಗಳನ್ನು ನೀಡಿದವರು. ಈ ಧರ್ಮಪಾಲನೆ ಅವರ ಗಾರ್ಹಸ್ಥ್ಯ ಜೀವನದಲ್ಲಿ ಅವರ ಧರ್ಮ ಪತ್ನಿಯ ಸಹಾಯದಿಂದ ಇನ್ನಷ್ಟು ಹೆಚ್ಚಾಯಿತು. ಇವೆಲ್ಲ ನನ್ನ ಸ್ವಯಂ ಅನುಭವ. ಭಟ್ಟರಿಗೆ ಮತ್ತು ಶೈಲಕ್ಕನಿಗೆ ನನ್ನ ಮೇಲಿನ ಪ್ರೀತಿ ಅಪಾರ. ಬಹುಶಃ ಅವರ ಮನೆಯಲ್ಲಿ ಸಿಹಿತಿಂಡಿಮಾಡಿದಾಗಲೆಲ್ಲಾ ಅವರು ನನ್ನನ್ನು ಸ್ಮರಿಸಿದ್ದಾರೆ. ನನ್ನನ್ನು ಕರೆದು ಹಲಸಿನ ಹಣ್ಣಿನ ಕಡುಬು, ಮಾವಿನ ಹಣ್ಣಿನ ಪಾಯಸ ಹೀಗೆ ಉಣಬಡಿಸಿದ್ದಾರೆ. ಇದನ್ನೆಲ್ಲಾ ಮರೆಯುವುದು ಹೇಗೆ? ಸಾಧ್ಯವೇ ಇಲ್ಲ. ಅವರ ಹೃದಯ ವೈಶಾಲ್ಯ ಅಸೀಮ. ತಮ್ಮನ್ನು ಓಲೈಸುವ ವಿರೋಧಿಸುವ ಎಲ್ಲರಿಗೂ ತಮ್ಮಿಂದಾಗುವ ಸಹಾಯ ಸಹಕಾರ ಆತಿಥ್ಯಗಳನ್ನು ಸಮಾನವಾಗಿ ವಿತರಿಸುವ ಹೃದಯ ವೈಶಾಲ್ಯ ಅವರದ್ದು. ಹೀಗಾಗಿ ಅವರು ಜನಾನುರಾಗಿಗಳು.

ಬಹುಶಃ ಗಂಗಾಧರ ಭಟ್ಟರ ಗುಣಗಾನಗಳ ಬಗ್ಗೆ ಮತ್ತೆ ಮತ್ತೆ ಸ್ಮರಿಸಿ ಬರೆಯುತ್ತಿದ್ದರೆ ಅದೊಂದು ಪುಸ್ತಕವಾದೀತು. ಅದಕ್ಕಿದು ಅವಸರವಲ್ಲ. ಇದೊಂದು ಲೇಖನ ಮಾತ್ರ, ಆದರೆ ಇಲ್ಲಿ ನಿಮಗೊಂದು ಸಂಶಯ ಮೂಡಬಹುದು ಅದೆಂದರೆ ನಾನು ಹೇಳುವಂತೆ ಗಂಗಾಧರ ಭಟ್ಟರು ಸರ್ವಗುಣಸಂಪನ್ನರೇ, ಅವರಲ್ಲಿ ಗುಣದೋಷಗಳೇ ಇಲ್ಲವೇ? ಅವರೇನು ಸರ್ವೇಶ್ವರರೇ ಸರ್ವಜ್ಞರೇ, ಇತ್ಯಾದಿ. ಆದರೆ ನಾನೇನು ಅಂಥಹ ಭ್ರಮೆಯನ್ನು ಹೊಂದಿಲ್ಲ. ಮನುಷ್ಯಮಾತ್ರನಾದ ಮೇಲೆ ಅಥವಾ ಪ್ರಾಣಿ ಮಾತ್ರನಾದಮೇಲೆ ದೋಷಗಳಿರಲೇಬೇಕು. ಆದರೆ ದೋಷಗಳು ಕೆಲವರಿಗೆ ದೋಷವಾಗಿ ಕಾಣಬಹುದು, ಕೆಲವರಿಗೆ ಕಾಣದಿರಬಹುದು. ಆದ್ದರಿಂದಲೇ ಪರದೋಷಗಳನ್ನು ಹೇಳುವುದು ಧರ್ಮಶಾಸ್ತ್ರಕ್ಕೆ ವಿರುದ್ಧವಾದುದು. ಹಾಗಾಗಿ ಗಂಗಾಧರ ಭಟ್ಟರ ಅನೇಕ ಸಿದ್ಧಾಂತಗಳು ವಿಷಯ ವಿಚಾರಗಳು, ನಿರ್ಣಯಗಳು ನಡವಳಿಕೆಗಳ ಬಗ್ಗೆ ಅವರೊಂದಿಗೆ ಮತಭೇದ, ಮತಿಭೇದಗಳ ಕಾರಣದಿಂದ ವಿರೋಧಾಭಿಪ್ರಾಯಗಳನ್ನು ಹೊಂದಿರುವುದು ಪ್ರದರ್ಶಿಸುವುದು ಸತ್ಯವೇ. ಆದರೆ ಅವೆಲ್ಲವೂ ವಿಷಯಾಧಾರಿತ ಸಿದ್ಧಾಂತಾಧಾರಿತ ಮಾತ್ರ. ಅಥವಾ ಮತ ಮತಿ ಆಧಾರಿತ ಮಾತ್ರ, ಇವುಗಳಾವುವೂ ನನ್ನಲ್ಲಿನ ಗಂಗಾಧರಭಟ್ಟರ ಗೌರವಕ್ಕೆ ಚ್ಯುತಿಯನ್ನುಂಟುಮಾಡಿಲ್ಲ. ಮತಭೇದ ಮತ್ತು ಮತಿಭೇದಗಳನ್ನು ನನ್ನ ತಂದೆಯವರೊಂದಿಗೂ ನಾನು ಹೊಂದಿದ್ದೆ. ಹಾಗೆಂದ ಮಾತ್ರಕ್ಕೆ ನನ್ನ ತಂದೆಯ ಬಗ್ಗೆ ಗೌರವ ಪ್ರೀತಿ ಇರಲಿಲ್ಲವೆಂದಲ್ಲ. ಆದರೂ ಧರ್ಮಶಾಸ್ತ್ರ ಗುರುಜನರೊಂದಿಗೆ ವೃಥಾ ವಾದ ವಿರುದ್ಧವಾದಗಳನ್ನು ನಿಷೇಧಿಸುತ್ತದೆ. ಆದರೆ ನನ್ನ ಮನೋ ನಿಗ್ರಹದ ಕೊರತೆಯಿಂದ ವಿರುದ್ಧವಾದಗಳು ಸಂಭವಿಸಿವೆಯೆಂದೇ ನಾನು ನಂಬುತ್ತೇನೆ. ಹೀಗೆ ಪರದೋಷಗ್ರಹಣವಾದರೆ ಅದು ಸ್ವದೊಷಾವಿಷ್ಕರಣವೇ. ಆದ್ದರಿಂದಲೇ ಪರದೋಷ ಕೀರ್ತನೆಯನ್ನು ಧರ್ಮಶಾಸ್ತ್ರಗಳು ನಿಷೇಧಿಸಿದ್ದು. ದೋಷಗಳು ಸಹಜ, ಗುಣಗಳು, ಆಧೇಯ. ದೋಷಗಳನ್ನು ಸಂಸ್ಕಾರಗಳಿಂದ ಯಾವಾಗಬೇಕಾದರೂ ಪರಿಹರಿಸಿಕೊಂಡು ಗುಣಗಳನ್ನು ಗಳಿಸಿಕೊಳ್ಳಬಹುದು. ಹಾಗೆ ಗುಣಗಳನ್ನು ಗಳಿಸಿಕೊಳ್ಳಬೇಕೆಂಬುದೇ ಧರ್ಮಶಾಸ್ತ್ರದ ಉಪದೇಶ. ನಲವತ್ತು ಸಂಸ್ಕಾರಗಳಲ್ಲಿ ಒಂದು ಸಂಸ್ಕಾರ ಮಾತ್ರ ಇದ್ದರೂ ಎಂಟು ಆತ್ಮಗುಣಗಳನ್ನು ಹೊಂದಿದವನು ಸ್ವರ್ಗಮೋಕ್ಷಗಳನ್ನು ಹೊಂದುತ್ತಾನೆಂದೂ ಅದಕ್ಕೆ ಬದಲಾಗಿ ನಲವತ್ತು ಸಂಸ್ಕಾರಗಳನ್ನು ಹೊಂದಿದ್ದು ಎಂಟು ಆತ್ಮಗುಣಗಳನ್ನು ಹೊಂದಿಲ್ಲದಿದ್ದರೆ ಸ್ವರ್ಗ ಸಿದ್ಧಿಯಾಗದು ಎಂದು ಗೌತಮರು ಕ್ಷಮಾದಿ ಅಷ್ಟಗುಣಗಳನ್ನು ವಿಧಿಸುತ್ತಾರೆ.

ಗಂಗಾಧರ ಭಟ್ಟರಂಥಹ ಗುರುಜನರನ್ನು ಆದರ್ಶವಾಗಿಟ್ಟುಕೊಂಡು ನಮ್ಮಲ್ಲಿ ಗುಣಾವಿರ್ಭಾವವನ್ನು ಮಾಡಿಕೊಳ್ಳಬೇಕೆಂಬುದು ಈ ಲೇಖನದ ಒಂದು ಆಶಯ. ಗಂಗಾಧರ ಭಟ್ಟರು ನಮಗೊಂದು ಆದರ್ಶ. ಅಂತಹವರು ನಿಜವಾದ ಗುರುಜನರಾಗಲು ಯೋಗ್ಯ. ತೇಭ್ಯೋ ಗುರುಭ್ಯೋ ನಮೋ ನಮಃ .
