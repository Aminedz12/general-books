\chapter{ಉತ್ಸಾಹದ ಚಿಲುಮೆ ಗಂಗಾಧರ ಭಟ್ಟರು}

\begin{center}
\Authorline{ಡಾ|| ಶಿವರಾಮ ಭಟ್ಟ,}
\smallskip
ಯಲ್ಲಾಪುರ.\\
ಪೂರ್ವವಿದ್ಯಾರ್ಥೀ 

\end{center}
\begin{verse}
ಗುರುಃ ಪಿತಾ ಗುರುರ್ಮಾತಾ ಗುರುದೆವ ಹಿ ಬಾಂಧವಾಃ |\\
ಗುರುದೇವಾತ್ ಪರಂ ನಾಸ್ತಿ ತಸ್ಮೈ ಶ್ರೀ ಗುರವೇ ನಮಃ ||
\end{verse}
ಅಂದು ಉಮ್ಮಚಗಿ ಸಂಸ್ಕೃತ ವಿದ್ಯಾಲಯದಲ್ಲಿ ಸಾಹಿತ್ಯ ಪರೀಕ್ಷೆಯನ್ನು ಮುಗಿಸಿದ ಸಂಭ್ರಮ. ಆ ದಿನ ಉಮ್ಮಚಗಿಯ ಬಸ್ ಸ್ಟ್ಯಾಂಡಿನಲ್ಲಿ ಉಮಾಕಾಂತ ಭಟ್ಟರ ದರ್ಶನವಾಯಿತು. ನ್ಯಾಯಶಾಸ್ತ್ರವನ್ನು ಓದಲು ಪ್ರೇರಣೆಯೂ ದೊರೆಯಿತು. ಅಧ್ಯಯನಕ್ಕಾಗಿ ಮೈಸೂರಿಗೆ ತೆರಳಿದೆ. ಶ್ರೀಯುತರನ್ನ ನಾನು ಮೊದಲು ನೋಡಿರಲಿಲ್ಲ. ಅಲ್ಲಿರುವ ವಿದ್ಯಾರ್ಥಿಗಳು ನನಗೆ ಹೇಳಿದ್ದು- ಯಾರು ಅಂಗಿಯ ಮೇಲೆ ಶಲ್ಯವನ್ನು ಧರಿಸಿ ಗಂಭೀರವಾಗಿ ಪಾಠಶಾಲೆಯಲ್ಲಿ ನಡೆದಾಡುತ್ತಾರೋ ಅವರೇ ಗಂಗಾಧರ ಭಟ್ಟರು.

ಅವರನ್ನು ಕಂಡೆ, ಮಾತನಾಡಿದೆ. ಅವರ ಪಾಂಡಿತ್ಯಪೂರ್ಣವಾದ ಮಾತು, ತಿಳಿಹಾಸ್ಯ ಇವುಗಳು ನನ್ನ ಮನಸ್ಸನ್ನು ಸೂರೆಗೊಂಡವು.

ಭಟ್ಟರನ್ನು ಇಷ್ಟಪಡದ ವಿದ್ಯಾರ್ಥಿ ಯಾವೊಬ್ಬನೂ ಅಲ್ಲಿ ಇರಲಿಲ್ಲ. ಎಲ್ಲಾ ಶಾಸ್ತ್ರಗಳ ವಿದ್ಯಾರ್ಥಿಗಳೂ ಅವರ ಮನೆಗೆ ಹೋಗಿ ಸಂದೇಹವನ್ನು ಪರಿಹರಿಸಿಕೊಳ್ಳುತ್ತಿದ್ದರು. ಭಟ್ಟರ ಧರ್ಮಪತ್ನಿ(ಅಕ್ಕಯ್ಯ) ಎಲ್ಲರನ್ನೂ ವಾತ್ಸಲ್ಯದಿಂದ ಬರಮಾಡಿಕೊಳ್ಳುತ್ತಿದ್ದರು. ಭಟ್ಟರ ಸಾಧನೆಗೆ ಧರ್ಮಪತ್ನಿಯ ಸಹಕಾರವೂ ಕಾರಣವಾಗಿದೆ.

ಸಮಯೋಚಿತವಾದ ಮಾತುಗಳು, ಸಂದರ್ಭಕ್ಕೆ ಉಚಿತವಾದ ನಿರ್ಧಾರ, ಮೈ ರೋಮಾಂಚನಗೊಳಿಸುವ ಉಪನ್ಯಾಸಗಳು, ಸಮಯಪಾಲನೆ, ಎಲ್ಲರನ್ನೂ ಪ್ರೀತಿಸುವ ಹೃದಯ ಇವು ಯಾರಿಗೆ ತಾನೇ ಇಷ್ಟವಾಗಲಾರವು ? ಪಂಡಿತಾಃ ಸಮದರ್ಶಿನಃ ಎಂಬಂತೆ ಚಿಕ್ಕವರು, ದೊಡ್ಡವರು, ಪಂಡಿತರು, ಸಾಮಾನ್ಯಜನರು ಎಲ್ಲರೊಂದಿಗೂ ಸಮಾನವಾಗಿ ಬೆರೆಯುವುದು ಭಟ್ಟರ ವಿಶೇಷತೆ. ಕನ್ನಡ, ಸಂಸ್ಕೃತ, ಆಂಗ್ಲಭಾಷೆಗಳಲ್ಲಿ ನಿರರ್ಗಳವಾಗಿ ಮಾತನಾಡುವ ಸಾಮರ್ಥ್ಯ ಭಟ್ಟರಿಗಿದೆ. ಅಶ್ಲೀಲವನ್ನು ದೂರವಿಟ್ಟು, ರಸಭರಿತವಾಗಿ ಶಾಸ್ತ್ರೀಯವಾಗಿ ನಿರೂಪಿಸುವ ವಾಗ್ವೈಖರಿ ಇವರಿಗೆ ಸಿದ್ಧಿಸಿದೆ.

ಸ್ಪರ್ಧೆಗಳು ಬಂದರೆ ಸಾಕು; ಭಟ್ಟರ ಮನೆಗೆ ಎಲ್ಲಾ ಶಾಸ್ತ್ರಗಳ ವಿದ್ಯಾರ್ಥಿಗಳೂ ಲಗ್ಗೆಯಿಡುತ್ತಿದ್ದರು. ವಿಷಯವನ್ನು ಕೇಳಿದ ತಕ್ಷಣ ಅವರ ಮುಖದಿಂದ ಪುಂಖಾನುಪುಂಖವಾಗಿ ಶಾಸ್ತ್ರವಾಕ್ಯಗಳು ಹೊರಹೊಮ್ಮುತ್ತಿದ್ದವು. ಸನ್ಮಿತ್ರನಾದ ಅನಂತನ ಜೊತೆಗೂಡಿ ನ್ಯಾಯಶಾಸ್ತ್ರವನ್ನು ಪಠಿಸಿದೆ. ನನಗೆ ಜೀವನವೇ ಒಂದು ಪಾಠವಾದರೆ ಭಟ್ಟರಿಗೆ ಇಂದಿಗೂ ಪಾಠವೇ ಜೀವನವಾಗಿದೆ. ಗುರುಗಳಿಗೆ ಶ್ರಮವನ್ನು ಕೊಟ್ಟು ನಾನು ಓದಿ, ಅವರ ಪ್ರೋತ್ಸಾಹದಿಂದಲೇ ನ್ಯಾಯೇತರಶಾಸ್ತ್ರಗಳ ಅಧ್ಯಯನವನ್ನೂ ಮಾಡಿದೆ. ಹೀಗೆ ತೀರಿಸಲಾರದ ಗುರುವಿನ ಋಣದಲ್ಲಿ ನಾನಿದ್ದೇನೆ. ಭಗವಂತ ನನ್ನ ಗುರುಗಳಿಗೆ ಆಯುಷ್ಯಆನಂದ ಆರೋಗ್ಯಗಳನ್ನು ಕೊಟ್ಟು ಕಾಪಾಡಲಿ ಎಂದು ಪ್ರಾರ್ಥಿಸುತ್ತೇನೆ. ಗುರುಚರಣಾರವಿಂದಗಳಿಗೆ ಅನಂತ ಪ್ರಣಾಮಗಳು.
