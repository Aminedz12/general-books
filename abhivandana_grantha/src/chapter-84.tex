\chapter{ವೇ||ಬ್ರ||ಶ್ರೀ ವಿದ್ವಾನ್ ಗಂಗಾಧರ ಭಟ್ಟರು}

\begin{center}
\Authorline{ವಿ | ಆರ್. ಶ್ರೀಧರ ಶಾಸ್ತ್ರೀ, ಸೂರಿ}
\smallskip
ನಿವೃತ್ತ-ನ್ಯಾಯಪ್ರಾಧ್ಯಾಪಕ\\
ಸಿದ್ಧಲಿಂಗೆಶ್ವರ ಸಂಸ್ಕೃತಕಾಲೇಜ್,\\
ತುಮಕೂರು
\end{center}
\begin{verse}
ನಮಿಪೆ ಶಾರದೇ! ನಿನ್ನ ಚರಣಕೆ ಶಿರವ ಬಾಗುತ ಮುದದಲಿ ||\\
ಕರವ ಜೋಡಿಸಿ ಮನದಿ ಧ್ಯಾನಿಸಿ ವಿಶ್ವರೂಪದಿ ನಿನ್ನನು\\
ಶ್ರೀ ಗುರು ಚರಣಾರವಿಂದಾಭ್ಯಾಂ ನಮಃ||
\end{verse}

ಪ್ರತಿಯೊಂದು ಮಹಾತ್ಮರು ವ್ಯಕ್ತಿಯ ಚರಿತ್ರೆಯನ್ನು ಬರೆಯುವಾಗ ಅದರ ಪರಮೋದ್ದೇಶ ಜೀವನದ ಮಾರ್ಗದರ್ಶನ. ಶ್ರೀ ವೇದವ್ಯಾಸ ಮುನಿಗಳೂ ಸಹ ಶ್ರೀಮದ್ ಭಾಗವತವನ್ನು ರಚಿಸುವಾಗ, ವಿದ್ಯಾರಣ್ಯ ಮುನಿಗಳು ಶ್ರೀ ಶಂಕರರ ದಿವ್ಯ ಚರಿತಾಮೃತವನ್ನು ಕಾವ್ಯ ರೂಪದಲ್ಲಿ ಇಳಿಸುವಾಗ ಇದೇ ಉದ್ದೇಶವನ್ನು ಉಲ್ಲೇಖಿಸಿದ್ದಾರೆ. “ ಕಲಿಮಲಪ್ರಧ್ವಂಸಿ ನಃ ಶ್ರೇಯಸೇ” ಎಂದು ಹೇಳಿದ್ದಾರೆ. ಹಾಗಾಗಿ ನನ್ನ ಸ್ನೆಹಿತನ ಆ ಜನ್ಮ ಕಥೆಯನ್ನು ಬರೆಯುವಾಗ ನಾನೂ ಅದನ್ನು ಅನುಸರಿಸಿದರೆ ಸರಿ ತಾನೆ.

ಕರ್ನಾಟಕದ ಉತ್ತರಕನ್ನಡ ಜಿಲ್ಲೆಯ ಸಿದ್ದಾಪುರ ತಾಲ್ಲೂಕಿನ ಅಗ್ಗೆರೆ ಎಂಬ ಗ್ರಾಮದಲ್ಲಿ 1957ರಲ್ಲಿ ಜನಿಸಿದ ಗಂಗಾಧರ ಭಟ್ಟರು ತಂದೆಯವರ ಮೂರನೆಯ ಮಗ. ಅವರ ಪೂರ್ವಾಪರ ಇತಿಹಾಸವನ್ನು ಈ ಅಭಿನಂದನಾ ಮಹೋತ್ಸವದ ಸಂದರ್ಭದಲ್ಲಿ  ಹಲವಾರು ಜನರು ಉಲ್ಲೇಖಿಸಿರುವದರಿಂದ ಅಜಾಮಿತ್ವಾಯ ಬೇರೊಂದು ಮಾರ್ಗವನ್ನು ಅನುಸರಿಸಿದ್ದೇನೆ. ಬಾಲ್ಯದ ಶಿಕ್ಷಣವು ಮುಂದಿನ ಪ್ರತಿಭಾಪ್ರಕಾಶಕ್ಕೆ ಭೂಮಿಕೆಯಾಗುತ್ತದೆ ಎಂದು ಬಲ್ಲವರು ಹೇಳುತ್ತಾರೆ. ಭಟ್ಟರ ತೀರ್ಥರೂಪರು ಅತ್ಯಂತ ಸುಸಂಸ್ಕೃತರಾದುದರಿಂದ ಉಚಿತಕಾಲದಲ್ಲಿ ಉಪನಯನಸಂಸ್ಕಾರವನ್ನು ಕೊಟ್ಟಿದ್ದು ಅವರ ವ್ಯಕ್ತಿತ್ವಕ್ಕೆ ಮಾನದಂಡ. ವಿಶಾಲವಾದ ಪ್ರಪಂಚದ ಅರಿವಿಗೆ ವಿರಳಾವಕಾಶವಿದ್ದ ಕಾಲದಲ್ಲಿ ತನ್ನ 18ನೇ ವರ್ಷದ ಪ್ರಾಯದಲ್ಲಿ ವಿದ್ಯಾಕಡಲೆಂದು ಪ್ರಸಿದ್ಧವಾದ ಮೈಸೂರು ನಗರವನ್ನು ಆಯ್ದುಕೊಂಡು ಮನೆಬಿಟ್ಟು ಸಂಸ್ಕೃತಾಧ್ಯಯನಕ್ಕೆ ಟೊಂಕ ಕಟ್ಟಿ ನಿಂತರು. ಇದು ನನ್ನ ಸ್ವಂತ ಅನುಭವವೂ ಹೌದು. ವಿದ್ಯಾಸಾರ್ವಭೌಮರಾದ ತೀರ್ಥರೂಪರೇ ನನ್ನ ವಿದ್ಯಾಭ್ಯಾಸದ ಮಾರ್ಗದರ್ಶನಕ್ಕೆ ಉತ್ಸುಕರಾಗಿದ್ದರೂ ಶಾರದೆಯ ಮಂದಿರವಾದ ಶೃಂಗೇರಿಯನ್ನು ಆಯ್ದುಕೊಂಡು ಅದರಲ್ಲು ಜಗದ್ಗುರುಗಳಾದ ಶ್ರೀ ಶ್ರೀ ಶ್ರೀಮದಭಿನವ ವಿದ್ಯಾತೀರ್ಥರಲ್ಲಿ ವಿದ್ಯಾತೀರ್ಥವನ್ನು ಬೇಡಿ ಹೋದವನು ನಾನು. ಆ ಕಾರಣದಿಂದ ಇಂದು ಗಂಗಾಧರ ಭಟ್ಟರ ಜೀವನದ ಘಟನೆಯನ್ನು ಉಲ್ಲೇಖಿಸಲು ಸಮರ್ಥನಾಗಿದ್ದೇನೆ.

ಅಜಮಾಸು 1974ನೇ ಇಸವಿ ಇರಬಹುದು. ಶ್ರೀ ಶ್ರೀ ಶ್ರಿಂಗೇರಿ ಜಗದ್ಗುರುಗಳ ಶಿಷ್ಯನಾಗಿ ಮೈಸೂರಿನ ಮೊಕ್ಕಾಮಿನಲ್ಲಿ ಇರುವಾಗ ಗಂಗಾಧರ ಭಟ್ಟರ ಮೊದಲ ಭೇಟಿ ನನಗಿಂತಲೂ ಚಿಕ್ಕವರಾದರೂ ಚೊಕ್ಕಮನಸ್ಸಿನವರು.” समान-सख्य-व्यसनेषु शीलता” ಎಂಬ ಮಾತು ಸತ್ಯವಾದುದು. 1977ರಲ್ಲಿ ತುಮಕೂರಿಗೆ ಉದ್ಯೊಗಕ್ಕಾಗಿ

ಸಂಸ್ಕೃತ ನ್ಯಾಯ ಶಾಸ್ತ್ರ ಪ್ರಾಧ್ಯಾಪಕನಾಗಿ ಬಂದವನು ನಾನು. ಅದಾಗಲೇ ಸಂಸ್ಕೃತ ಸಾಹಿತ್ಯ ಪರೀಕ್ಷೆಯನ್ನು ಮುಗಿಸಿದ ಭಟ್ಟರು ಮೈಸೂರಿನಲ್ಲೆ ಒಂದು ಸಂಸ್ಕೃತ ಪಾಠ ಶಾಲೆಯ ಅಧ್ಯಾಪಕರು. अधीति बोधाचरणप्रचारणैः ಇತ್ಯಾದಿ ಮಹಾಕವಿಯ ಮಾತು ಅಕ್ಷರಶಃ ಹೊಂದಿಸಿಕೊಂಡ ಭಟ್ಟರು ನವೀನನ್ಯಾಯಶಾಸ್ತ್ರ ಅಧ್ಯಯನಕ್ಕಾಗಿ ಶ್ರೀಮನ್ಮಹಾರಾಜ ಸಂಸ್ಕೃತ ಕಾಲೇಜಿನ ವಿದ್ಯಾರ್ಥಿಯಾಗಿ ತೊಡಗಿಸಿಕೊಂಡಿದ್ದರು. ಅಧ್ಯಾಪಕರು ಯಾರು? ಸಾಕ್ಷಾತ್ ಶ್ರೀಮದುದಯನಾಚಾರ್ಯರ ಅವತಾರವೋ ಎಂಬಂತಿರುವ ಶ್ರೀ ಶ್ರೀ ರಾಮಭದ್ರಾಚಾರ್ಯರು. ಭಟ್ಟರ ಅದೆಷ್ಟೋ ಜನುಮದ ಪುಣ್ಯ ಫಲ. ನಮ್ಮಿಬ್ಬರ ಮೊದಲನೇ ಭೇಟಿಯಲ್ಲೇ “ಸತ್‍ಪ್ರತಿಪಕ್ಷದ” ಸ್ವರೂಪದ ಬಗ್ಗೆ ಚಿಂತನೆ. ಸತ್‍ಪ್ರತಿಪಕ್ಷಸ್ಥಲದಲ್ಲಿ ಸಾಧ್ಯಸಾಧಕ ಹೇತು, ಮತ್ತು ಸಾಧ್ಯಾಭಾವಸಾಧಕ ಹೇತು ಎಂದು ಎರಡು ಇರುತ್ತದಲ್ಲ ಆದರೆ ಇಲ್ಲಿ ಸಂದೇಹವೇನೆಂದರೆ, ವಾದಸ್ಥಲದಲ್ಲಿ ಪ್ರತಿವಾದಿಯು ಹೇಳುವಮಾತು “ಸಾಧ್ಯಾಭಾವಸಾಧಕಹೇತುಂ ಸಾಧ್ಯಸಾಧಕತ್ವೇನ ಉಪನ್ಯಾಸಾತ್ ಅಶಕ್ತಿವಿಶೇಷ ಉಪಸ್ಥಾಪನಂ.” ಎಂದು. ಅಂದರೆ ವಾದಿಯು “ಹ್ರದೋ ವಹ್ನಿಮಾನ್ ಜಲಾತ್” ಎಂದು ಪ್ರಯೋಗಿಸಿದ ಎಂತಲ್ಲವೇ. ಇದರ ಮರ್ಮವೇನು? ಅಲ್ಲಿಂದ ಈಚೆ ಗಂಗಾಧರ ಭಟ್ಟರ ಸಖ್ಯ ದೃಢವಾಗುತ್ತಾ ಬಂತು. ಕಾರಣ ವಿದ್ಯೆಯ ಕಂಡೂತಿ. ಎರಡು ಗವಯಮೃಗಗಳು ಶೃಂಗದಿಂz  ಉಜ್ಜಾಡಿದರೆ ಉಂಟಾಗುವ ಆನಂದವು ಇಬ್ಬರಿಗೂ ಉಂಟು.

ಹಲವರು ಪಾಠ ಮಾಡುತ್ತಾರೆ ಆದರೆ ಅದು ವ್ಯುತ್ಪಾದಕವಾಗಿರುವುದಿಲ್ಲ “ವ್ಯುತ್ಪತ್ತಿ” ಎಂಬುದೊಂದು ದೊಡ್ಡಸಂಸ್ಕಾರ. ಸಂಸ್ಕಾರ ಪಡೆದ ವ್ಯಕ್ತಿ ಗುರುಸಾನ್ನಿಧ್ಯವಿಲ್ಲದೆ ಗುರ್ವನುಗ್ರಹದಿಂದಲೇ ಅನೇಕ ಗ್ರಂಥಗಳ ಸಮನ್ವಯವನ್ನು ಕಂಡುಕೊಂಡು ವಿದ್ವಜ್ಜನರು ತಲೆದೂಗುವಂತೆ ತಿಳಿಸುತ್ತಾರೆ. ಭಟ್ಟರ ಗುರಗಳ ಪಾಠವು ಅಂತಹದ್ದು.

ಗಂಗಾಧರ ಭಟ್ಟರ  ಔದಾರ್ಯ ಎಂತಹದ್ದು ಎಂದರೆ, ಓದಲು ಬರುವ ಮಕ್ಕಳಿಗೆ ತೋರಿಸುವದಾರಿ, ಸಹಾಯ ಇವುಗಳನ್ನೆಲ್ಲ ಈ ಅಭಿನಂದನಗ್ರಂಥದಲ್ಲಿ ಹಲವರು ಉಲ್ಲೇಖಿಸಿದ್ದಾರೆ. ಭಟ್ಟರ ಜೀವನದಲ್ಲಿ ಅವರ ಆಂಗ್ಲವಿದ್ಯಾರ್ಜನಾವಿಧಾನ, ವಾಕ್ಪಟುತ್ವ, ವಿಭಿನ್ನರೀತಿಯ ಚಿಂತನೆ ಇವುಗಳೆಲ್ಲ ತುಂಬಿದ್ದರೂ ನಾನು ಬೇರೆ ದೃಷ್ಟಿಯಿಂದಲೇ ಜೀವನದ ಘಟನೆಗಳನ್ನು ಆಯ್ಕೆ ಮಾಡಿಕೊಂಡೆ. ಶ್ರೀಮನ್ಮಹಾರಾಜಾ ಸಂಸ್ಕೃತ ಕಾಲೇಜಿನ ನ್ಯಾಯಶಾಸ್ತ್ರ ಅಧ್ಯಾಪಕರಾದ ಅನಂತರ ಅವರ ಆತಿಥ್ಯವನ್ನು ಸ್ವೀಕರಿಸಿದ್ದೇನೆ. ಮನೆಯ ಬಾಗಿಲು ಹಾಕಿದ್ದರೂ ಅದು ಸ್ನೇಹಿತರ ಸ್ವಾಗತಕ್ಕೆ ಮುಕ್ತವಾಗಿರುತ್ತಿತ್ತು. ಮಾರ್ಗದಲ್ಲಿ ಬರುವಾಗಲೇ ಗಂಧವೇದಿಯೊ ಶಬ್ದವೇದಿಯೊ ಆದ ಭಟ್ಟರು ನನ್ನನ್ನು ಸ್ವಾಗತಿಸುವ ಲಕ್ಷಣವೇ ಬೇರೆ .

ಒಂದು ದಿನ ಕಾರ್ಯಾಂತರದಿಂದ ಭಟ್ಟರ ಮನೆಗೆ ಹೋದ ನನಗೆ ಗಾದಾಧರೀ ಪಂಚಲಕ್ಷಣಿಯಲ್ಲೊಂದು ಚರ್ಚೆ ಕಾದಿತ್ತು. ಅದು ಗ್ರಂಥಾರೂಢವೇ ಆಗಿದ್ದರೂ ಪ್ರಾಚೀನ ಸಂಪ್ರದಾಯದಲ್ಲಿ ಅಧ್ಯಯನ ನಡೆಸಿದ ನನ್ನ ಜೊತೆ ಹಂಚಿಕೊಳ್ಳುವದು. ನಾಲ್ಕನೆ ಲಕ್ಷಣದಲ್ಲಿ ಸಾಕಲ್ಯದ ಸ್ವರೂಪ, ಸಾಕಲ್ಯದ ಹೊಂದಾಣಿಕೆ, ಸಾಕಲ್ಯವು ಎಲ್ಲಿಗೆ ಯಾವಾಗ ವಿಶೇಷಣ. ವಿಶೇಷಣ ವಿಶೇಷ್ಯಭಾವದಲ್ಲಿ ಪೌರ್ವಾಪರ್ಯದ ತಂತು ಮತ್ತು ಸಾಧ್ಯಾಭಾವದಲ್ಲಿ साध्यतावच्छेदकावच्छिन्न-प्रतियोगिताक्त्व-विशेषण ದ ಜೊತೆಗೆ ವಿಕಲ್ಪ ವಿಚಾರ. ಇದರಲ್ಲಿ ಭಟ್ಟಾಚಾರ್ಯರ ಮತ್ತು ಜಗದೀಶತರ್ಕಾಲಂಕಾರರ ಭಿನ್ನಾಭಿಪ್ರಾಯದ ಆಕೂತ. ಈ ರೀತಿಯಾದ ಶಾಸ್ತ್ರಪಂಕ್ತಿಗಳ ಚರ್ಚೆ ನಮ್ಮಿಬ್ಬರಲ್ಲಿ ಅನೇಕಬಾರಿ ನಡೆದದ್ದು ನನ್ನ ಜೀವನದ ಆನಂದ ಕ್ಷಣಗಳು.

ಭಟ್ಟರ ಜೀವನಗಮನದ ಸಂಗಾತಿಯಾದ ಶ್ರೀಮತೀ ಶೈಲಜಾರವರ ದಕ್ಷತೆಯನ್ನು ಹೇಳದೇ ಹೋದರೆ ಮಹಾ ಅಪರಾಧವಾದೀತು. ಊಟದ ಸಮಯ ಮೀರದ ಹಾಗೆ ಊಟದಲ್ಲಿಯೂ ಅಡಿಗೆಯ ರುಚಿಯನ್ನು ಮರೆಯದ ಹಾಗೆ ಮಾಡುವ ಎಲ್ಲ ಉಪಾಯಗಳನ್ನೂ ಕಂಡಿದ್ದೇನೆ. ಎಲ್ಲಿಯೂ ಆತಿಥ್ಯದಲ್ಲಿ ಕಿಂಚಿತ್ತೂ ಲೋಪವಾಗದ ರೀತಿ ಅವರದು. ಬೇಕಾದುದನ್ನೇ ಮಾಡಿ ಬಡಿಸುವ ಮಹೋನ್ನತ ಗುಣ.

ಇನ್ನೊಂದು ಬಹುಮುಖ್ಯವಾದದ್ದು. ಶ್ರೀಯುತ ಭಟ್ಟರ ಸ್ವಗ್ರಾಮವಾದ ಅಗ್ಗೆರೆಯಲ್ಲಿ ಭಟ್ಟರ ವಿವಾಹಮಹೋತ್ಸವ. ಅಂದು ನನ್ನ ಜೊತೆ ಇನ್ನೊಬ್ಬ ಪ್ರಾಚೀನ ನೈಯಾಯಿಕ ಕೆರೇಕೈ ಶ್ರೀ ಉಮಾಕಾಂತಭಟ್ಟರು. ಹವ್ಯಕರ ಸಂಪ್ರದಾಯದಂತೆ ಊಟದಲ್ಲಿ ಸಿಹಿಯನ್ನು ಬೇರೆಯವರು ಎಷ್ಟೇ ಉಪಚರಿಸಿದರೂ ನವದಂಪತಿಗಳು ಜೋಡಿಯಾಗಿ ಪಂಕ್ತಿಯಲ್ಲಿ ಉಪಚರಿಸಲು ಕಾಣಿಸಿಕೊಳ್ಳಲೇಬೇಕು. ಮಧುರಸೇವನೆಯಲ್ಲಿ ನಾವಿಬ್ಬರೂ ಅಗ್ರೇಸರರು. ಯಾರು ಮೊದಲಿಗರೆಂದು ಹೇಳುವುದು ಬಹಳ ಕಷ್ಟ. ಅದು ಮಾವಿನಹಣ್ಣಿನ ಕಾಲ. ಯಥೇಚ್ಛವಾಗಿ ರಸಾಯನ. ನವದಂಪತಿಗಳು ನಮ್ಮ ಪಂಕ್ತಿಯನ್ನು ಪ್ರವೇಶಿಸುತ್ತಿದ್ದಂತೇ ನಮ್ಮಿಬ್ಬರೊಳಗೆ ದೃಢಸಂಕಲ್ಪ. ಮುಂದೆ ದಾಟಲು ಬಿಡಲೇಕೂಡದು. ಹಾಗೆ ಮಾಡಬೇಕೆಂಬ ನಿರ್ಣಯ. ರಸಾಯನದ ದಳ್ಳೆ ಖಾಲಿಯಾಗಿ ಹೊಟ್ಟೆ ತುಂಬಿ ಹರಿದ ಆನಂದದ ಅಲೆಗಳನ್ನು ನೋಡಿದ ಭಟ್ಟರು ಮಹಾಭಾರತದ ವಿದುರನ ಮನೆಗೆ ಬಂದ ಕೃಷ್ಣನ ಆತಿಥ್ಯವನ್ನು ನೆನಪಿಸಿದ ಸಂದರ್ಭ ಎಂದೂ ಮರೆಯಲು ಸಾಧ್ಯವಿಲ್ಲ . ಮಹಾಕವಿಗಳೂ ಮುಗ್ಧರಾಗುವ, ಅವರಿಗೆ ಸಿಗದ ಈ ಕಾವ್ಯದ ಸಂಭಾಷಣೆಯು ಅಂದವೋ ಅಂದ .

ನವರಸಭರಿತವಾದ ಭಟ್ಟರ ಜೀವನದಲ್ಲಿ ವೀರರಸವೂ ಹಾಸುಹೊಕ್ಕಾಗಿದೆ.ಇದ್ದುದು ರಜೋಗುಣದ ಪ್ರಭಾವ. ಭಗವಂತನು ಸಾಕ್ಷಾತ್ ಅವತರಿಸುವುದು ಮಾತ್ರವಲ್ಲದೆ ಜೀವರಲ್ಲಿಯೂ ತ್ರಿಗುಣಗಳ ವೈಶಮ್ಯದಿಂದ ಸೃಷ್ಟಿ-ಸ್ಥಿತಿ-ಲಯಗಳ ಶಕ್ತಿಯನ್ನು ತುಂಬುತ್ತಾನೆ ಅದರಿಂದ ದುಷ್ಟಶಕ್ತಿ ಶಮನವಾಗಿ ಲೋಕ ಕಲ್ಯಾಣವಾಗುತ್ತದೆ. ಭಟ್ಟರಿಂದ ಈ ಕಾರ್ಯ ನೆಡೆಯಿತು. ಅದೇನೆಂದರೆ ಸಂಸ್ಕೃತ ವಿದ್ವತ್ ಪರೀಕ್ಷೆಯಲ್ಲಿ ಓದಲು ಬರುವ ಮಕ್ಕಳಿಗೆ ಓದಲು ಬರದವರು ಕಾವಿಗೆ ಅಗೌರವ ತರುವ ಕಾವಿಧಾರಿಯು ಖರಪ್ರಮುಖರಿಗಿಂತ ಹೆಚ್ಚಿನ ಉಪದ್ರವವನ್ನು ಉಂಟುಮಾಡುತ್ತಿದ್ದ ಕಾಲ. ಶ್ರೀಯುತ ಭಟ್ಟರು ಅಂದು ಪರಶುರಾಮರಂತೆ ವರ್ತಿಸಿದರು. ಇಲ್ಲದಿದ್ದರೆ ಅನೇಕ ಪ್ರಾಮಾಣಿಕ ಮುಗ್ಧ ವಿದ್ಯಾರ್ಥಿಗಳಿಗೆ ಆಗುವ ಅನ್ಯಾಯವನ್ನು ಇಂದಿಗೂ ಅನುಭವಿಸಬೇಕಿತ್ತು. ಭಟ್ಟರ ನಿಸ್ವಾರ್ಥ ಸಂಸ್ಕೃತ ಸೇವೆಗೆ ಸಂಸ್ಕೃತ ಕಲಿಕೋತ್ಸಾಹಿಗಳ ಧನ್ಯತೆಗಳನ್ನು ಸಮರ್ಪಿಸೋಣ.

2017ನೆಯ ಸಾಲಿನ ಕೊನೆಯ ದಿನದಲ್ಲಿ ಪ್ರಾಧ್ಯಾಪಕ ಪದವಿಯಿಂದ ನಿವೃತ್ತರಾದ ಭಟ್ಟರಿಗೆ ಅವರ ಅಂತೆವಾಸಿಗಳು ಆಯೋಜಿಸಿದ ಅಭಿವಂದನಾಸಮಾರಂಭದ ಸಂದರ್ಭದಲ್ಲಿ ನನ್ನ ಸ್ಮೃತಿಯಲ್ಲಿ ಅರಳಿದ ಈ ಪುಷ್ಪವು ಭಟ್ಟರ ಶಿರೋಲಂಕಾರವಾದರೆ  ಆಯೋಜಕರಾದ ಅಂತೆವಾಸಿಗಳ ಪ್ರಯತ್ನವು ಸಫಲವೆಂದು ಭಾವಿಸುತ್ತೇನೆ.
