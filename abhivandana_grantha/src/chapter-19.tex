\chapter{स्पर्शाशौचविचारः}

\begin{center}
\Authorline{डा ॥ वेङ्कटरमण हेगडे}
\smallskip

प्राध्यापकः धर्मशास्त्रविभागः\\
श्रीमन्महाराजसंस्कृत महापाठशाला ,\\ 
मैसूरु
\end{center}

\section*{पीठिका-}

सनातनवैदिकसंस्कृत्या धर्मपरम्परानुचरणेन च प्रपञ्चेऽस्मिन् भारतदेशः स्वकीयां प्रसिद्धिं लेभे इत्यत्र नास्ति सन्देहः विपश्चित्सु । षडङ्गयुक्ताः चत्वारो वेदाः, पुराण-न्याय-मीमांसा-धर्मशास्त्राणीति चतुर्दश विद्यास्थानानि धर्ममूलानि च तन्मूलानि सन्ति इति तु सर्वेषां विदितमेव । एवञ्च चतुर्षु पुरुषार्थेषु प्रथमस्थानीयो धर्मः भरतभूमेः प्रतिष्ठामूलं वर्तते । स च धर्मः तदितरपुरुषार्थत्रयाणामपि साधक इति तस्य प्राधान्यं प्राथम्यं च प्रदत्तं सद्भिः। तादृशो धर्मः श्रुति-स्मृति-पुराणेतिहासैः विधीयते ज्ञायते च । अत एव “वेदोऽखिलो धर्ममूलं स्मृतिशीले च अतद्विदाम्” इत्युक्तम् । तैर्विहितो धर्मः क्रियारूपः अनुष्ठेययागादिरूपः बहुप्रकारको भवति । प्रकारतो धर्मबहुत्वेऽपि स्वरूपत एक एव । अत एव “यागादिरेव धर्मः । विद्वद्भिः सद्भिः अद्वेषरागिभिः सेवितो धर्मः, यं त्वार्याः क्रियमाणं प्रशंसन्ति स धर्मः” इति तत्स्वरूपं प्रकाशयामासुः धर्मज्ञाः । एवञ्च विहितक्रियारूपोऽर्थः धर्मः । स च शौचम् अपेक्षते । अत एव भृगुणा उक्तम्, आचारेन्दौ-

श्रुतिस्मृत्युदितं कर्म  न कुर्यादशुचिः क्वचित् । इति । 

एवञ्च दक्षः शौचं विना कर्मासिद्धिंं प्रकाशयन् शौचस्य कर्मफलहेतुत्वं स्पष्टयति- 
\begin{verse}
शौचे यत्नः सदा कार्यः शौचमूलो द्विजः स्मृतः । \\
शौचाचारविहीनस्य समस्ता निष्फलाः क्रियाः ॥ इति । 
\end{verse}
सन्ध्यावन्दनाभावे अशुचित्वं तत्सत्वे च सर्वकर्मसु अनर्हत्वं च आह दक्षः- 
\begin{verse}
सन्ध्याहीनोऽशुचिः नित्यमनर्हः सर्वकर्मसु । इति ।
\end{verse}
तथा च “शुचिना कर्म कर्तव्यम्, आचान्तेन कर्म कर्तव्यम्” इत्यादयो विधयोऽपि शुचेरावश्यकतां द्योतयन्ति । अतः श्रौतस्मार्तकर्मसु शौचस्य प्राधान्येन अपेक्षा वर्तते । ततश्च अयमवसरः प्राप्यते शौचज्ञानस्य । शौचाशौचपदार्थः कः? तद्भेदाः के? अशौचनिवारणोपायाः के? इत्यादीनां ज्ञानस्य अवसरः इदानीं प्राप्तः । अतस्तद्विचारं प्रकृत्य चिन्तयामः ।

\section*{शौचाशौचपदार्थविमर्शः}

तत्र किं नाम शौचम्, किं नाम अशौचम् इति पदार्थचिन्तनं प्रचक्रिरे धर्मशास्त्रनिबन्धकाराः । तत्र मिताक्षरायां विज्ञानेश्वरस्तु- “आशौचशब्देन च कालस्नानाद्यपनोद्यः पिण्डोदकदानादिविधेः अध्ययनादिपर्युदासस्य च निमित्तभूतः पुरुषगतः कश्चनातिशयः कथ्यते; न पुनः कर्मानधिकारमात्रम्” इत्याह । तत्र पितृमरणाद्युद्देशेन अशौचविधाने न केवलं सन्ध्यादीनां कर्मणां पर्युदासो बोध्यते; न वा काम्यकर्मादिषु अधिकारनिराकरणमात्रं बोध्यते; किन्तु पिण्डोदकदानादिकमपि कर्तव्यत्वेन विधीयते । अतः तेषां विधिनिषेधपर्युदासादीनां निमित्तभूतः पुरुषगतः अतिशयः अशौचम् इत्युक्तम् । तत्र च शुचेर्भावः शौचम्, न शौचम् अशौचम्, इति शुचेरभावः एवाशौचम् इति च पदार्थविमर्शः कृतः। शुचिः, शुद्धिः, शौचम् इत्यनर्थान्तरम् । अत एव “अशुद्धा बान्धवाः सर्वे” “शेषाहोभिर्विशुद्ध्यति” “तदहः शुद्धिकारणम्” इत्यादिवाक्येषु शुद्धिपदप्रयोगः कृतः । मिताक्षरस्तु शुद्धिप्रकाशे - शुद्धिर्नाशौचसंसर्गाभावः, आशौचं च सन्ध्यापञ्चमहायज्ञादि-कर्मानधिकार-सम्पादकोऽतिशयविशेषः, स च चेतने जनन- मरणास्पृश्य-स्पर्शाद्याहितोऽदृष्टविशेष एव, ताम्रकांस्याद्यचेतने तु चण्डालाद्यस्पृश्यस्पर्शादिजन्यो आम्लादिसंसर्गनाश्यश्च आधेयशक्तिविशेषो न तु अदृष्टम्, तस्य चेतनगुणत्वात् । उभयत्रापि वा चेतनाचेतनयोः धर्माधर्मविलक्षणाधेयशक्तिविशेष एवातिशयः, तस्य सुखदुःखान्यतरजनकत्वं प्रमाणाभावेन धर्माधर्मरूपत्वाभावात् । अस्तु वा तत्तच्चण्डालाद्यस्पृश्य-संसर्गोत्पत्तिकालीन- यावदम्लादि-संसर्गाभावविशिष्टः तत्तच्चण्डालाद्यस्पृश्य-संसर्गाद्यसमयध्वंस एव । अतश्च युक्तं सर्वेषाममीषां तत्तत्ताम्र-कांस्यादि-द्रव्याङ्गक-सन्ध्यादि-सुकृतविरोधित्वात् अशौचपदवाच्यम् । अतः तत्संसर्गाद्यभाव एव शौचमिति सिद्धम्” इति व्याख्यातवान् । हारलतादीनामयमेवाभिप्रायः। रुद्रधरादयः तावत् शौचाशौचयोरुभयोरपि भावरूपत्वमेव स्वीकुर्वन्ति; अभावे विनिगमकाभावहेतोः ।  इत्थञ्च मित्रमिश्रवचनेन अशौचसंसर्गाद्यभाव एव शौचम् इति ।

शुद्धिभेदाः - प्राधान्येन शौचं द्विविधम् - बाह्यम् आभ्यन्तरञ्चेति । तदुक्तं स्मृतिचन्द्रिकायां व्याघातपादेन _ 
\begin{verse}
शौचन्तु द्विविधं प्रोक्तं बाह्यमाभ्यन्तरं तथा ।\\
मृज्जलाभ्यां स्मृतं बाह्यं भावशुद्धिस्तथान्तरम् ॥ इति । 
\end{verse}
इत्थं मृज्जलाभ्यां गात्रशोधनं बाह्यशुद्धिः । भावशुद्धिः आन्तरशुद्धिः इति । व्यासेन तु वाक् शुद्धिः, कायशुद्धिः, मनःशुद्धिः, अर्थशुद्धिः इति भेदाः प्रकाशिताः । 
\begin{verse}
कालोऽग्निकर्ममृद्वायुः मनो ज्ञानं तपो जलम् । \\
पश्चात्तापो निराहारः सर्वेऽमी शुद्धिहेतवः ॥ इति ।
\end{verse}
सामान्येन शुद्धिहेतूनभिधाय दानादीनां विशेषेण शुद्धिहेतुत्वं प्राह याज्ञवल्क्यः । अनेन च अकार्यशुद्धिः, नदीशुद्धिः, द्रव्यशुद्धिः, द्विजशुद्धिः, वेदविच्छुद्धिः, विदुषां शुद्धिः, वर्ष्मशुद्धिः, रहस्यपापशुद्धिः, मनःशुद्धिः, भूतात्मशुद्धिः, क्षेत्रज्ञशुद्धिः इति भेदाः लक्ष्यन्ते।  इत्थञ्च बहुभेदभिन्ना शुद्धिः इति ज्ञायते । 

\section{ज्ञानस्याशौचप्रयोजकत्वम्}  

जनन-मरण-अस्पृश्यस्पर्शादिनिमित्तं यद्यपि अशौचकारणं तथापि ज्ञानमेव निमित्तम् अशौचप्रयोजकं भवति । अत एव मनुः - निर्दशं ज्ञातिमरणं श्रुत्वा पुत्रस्य जन्म च” इति निमित्तज्ञानस्यावश्यकतां कथयामास । तच्च सर्वसम्मतम् । अत एव आह्निकप्रकाशे मितमिश्र आह- अत्र चाशौचे निमित्तनिश्चय एव प्रयोजको न तु निमित्तोत्पत्तिमात्रम् । ‘भिन्ने जुहोति’ इत्यादौ निमित्तनिश्चयस्यैव प्रयोजकत्वात् । किञ्च जननादेः अशौचनिमित्तत्वे देशान्तरीयनिमित्तश्ङ्कया सर्वदा विहितकर्मानुष्ठानं न स्यात् , इति । इत्थम् अस्पृश्यस्पर्शादि-निमित्तज्ञाने सति अशौचमुत्पद्यते । तं यथाकालं स्नानाचमनादिना परिहरेत् । तदेतस्याशौचविषयस्य बहुविस्तरत्वात् यथावसरं स्पर्शाशौचविषयमात्रं चिन्तयामः ।

\section*{स्पर्शाशौचनिमित्तानि}

तत्र स्मृतिषु नानाविधानि स्पर्शाशौचानि विहितानि । प्रक्षालनापनेयानि आचमनापनेयानि स्नानापनेयानि प्रायाश्चित्तापनेयानि इति दोषाल्पत्वबहुत्वापेक्षया तानि भिन्नानि उपदिष्टानि । प्रक्षालनेन यस्य स्पर्शाशौचदोषस्य निवृत्तिः तत् प्रक्षालनापनेयाशौचम् । तत्राङ्गिरा आह -
\begin{verse}
ऊर्ध्वं नाभौ करौ मुक्त्वा यदङ्गमुपहन्यते ।\\
तत्र स्नानमधस्तात्तु क्षालनेनैव शुद्ध्यति ॥ इति । 
\end{verse}
तत्र करौ मुक्त्वा नाभेरूर्ध्वं विण्मूत्रोच्छिष्टस्पर्शे स्नानम्, नाभेरधः स्पर्शे क्षालनेन शुद्धिः इति । शुद्धिप्रकाशे हारीतोऽपि - 
\begin{quote}
दुष्टाभिशस्त-तिर्यगधोवर्णोपहतानां संस्पर्शे श्वासस्वेद-पूय-शोणित-घर्दित- लालानिष्ठीवित-रेणुकर्दमोच्छिष्टजल-विण्मूत्रपुरीषादिभिर्बाह्य-शरीरोपघात-निरूपहताभिरद्भिः मृद्भिः भस्मगोमयौषधिमन्त्रमङ्गलाचार भवति” -
\end{quote}
इत्युवाच । अत्र यद्यपि जलादीनामुपदेशः अथापि तेषां समुदायेन शुद्धिः अभिप्रेता; किन्तु दोषाल्पत्वगुरुत्वाभ्याम् एकद्वित्रयाणां च प्रत्येकस्यापि शुद्धिहेतुत्वमभिप्रेतम्। अत एव मनुः विण्मूत्रोत्सर्गशुद्ध्यर्थं दैहिकानां द्वादशमलानां च शुद्ध्यर्थं मृद् जलं चादेयम् इत्याह। तत्र 
\begin{verse}
वसाशुक्रमसृक्मृज्जा मूत्रविट्कर्णविण्नखाः। \\
श्लेष्माश्रुदूषिकास्वेदो द्वादशैते नृणां मलाः॥
\end{verse}
इत्युक्तानां दैहिकमलानां शुद्ध्यर्थं मृज्जलयोरादानं कुर्यात्। एवं यथार्हं पूर्वोक्तानामादानं कार्यम्। बोधायनस्तु पूर्वोक्तद्वादशमलेषु आद्यानां षण्णां शुद्ध्यर्थं मृत् जलं च तथा ष्ण्णामन्त्यानां शुद्ध्यर्थं केवलजलं विहितवान्।

स्नानापनेयानि \_

मन्वादिभिर्महर्षयः स्नानापनेयाशौचमुपदिदिशुः। दिवाकीर्त्यादीनां स्पर्शे जाते स्नानेन शुद्धो भवति इत्यभिप्रायः। तत्र मनुः _
\begin{verse}
दिवाकीर्तिमुदक्यां च सूतिकां पतितं तथा ।\\
शवं तत्स्पृष्टिनं चैव स्पृष्ट्वा स्नानेन शुद्ध्यति ॥ इति । 
\end{verse}
अत्र विष्णुरपि _
\begin{verse}
शवस्पृष्टं दिवाकीर्तिं चितिं पूयं रजस्वलाम्।\\
स्पृष्ट्वा त्वकामतो विप्रः स्नानं कृत्वा विशुद्ध्यति ॥ 
\end{verse}
इति जगाद । तत्र विप्रः अमतिपूर्वं एतेषां स्पर्शं प्राप्य स्नानानन्तरं शुद्धो भवति । याज्ञवल्क्योऽपि 
\begin{verse}
उदक्याशुचिभिः संस्पृष्टः स्नायात् -
\end{verse}
इति विहितवान् । गौतमस्तु- 
\begin{verse}
पतितचण्डालसूतिकोदकया शवस्पृष्ट्युपस्पर्शने सचैलमुदक्पस्पर्शनात्  शुद्ध्येत् -
\end{verse}
इति प्रोवाच। बृहस्पतिस्तु तत्र दोषतरतमत्वं प्रकाशयामास । तद्यथा _
\begin{verse}
युगञ्च द्वियुगं चैव त्रियुगं च चतुर्युगम् । \\
चण्डालसूतिकोदकयापतितानामधः क्रमात्। इति । 
\end{verse}
स्मृत्यन्तरे शुद्धिप्रकाशे - 
\begin{verse}
स्पृष्ट्वा देवलकं चैव सवासा जलमाविशेत् ॥
\end{verse}
इति विहितं पश्यामः। तत्र देवलको नाम वित्तापेक्षया वत्सरत्रयं देवार्चनपरो विप्रः, स च हव्यकव्येषु गर्हितः। ब्रह्माण्डेऽपि 
\begin{verse}
शैवान् पाशुपतान् स्पृष्ट्वा लोकायतिकनास्तिकान् ।\\
विकर्मस्थान् द्विजान् शूद्रान् सवासा जलमाविशेत्  ॥ 
\end{verse}
इति विधानं पश्यामः।  देवलस्तु _ 
\begin{verse}
श्वपाकं पतितं व्यङ्गम् उन्मत्तं शवहारकम् । \\
सूतिकां सूयिकां चैव रजसा च परिप्लुताम् ॥ \\
श्वकुक्कुटवराहांश्च ग्राम्यान् संस्पृश्य मानवः । \\
सचैलं सशिरः स्नात्वा तदानीमेव शुद्ध्यति ॥
\end{verse}
इत्याह। तदत्रावधेयं यत् यत्र यत्र स्नानमात्रस्यापि विधिः अशौचनिरसनार्थमुपदिशन्ति आचार्याः तत्र तत्र “सचेलं सशिरः स्नानम्” इति उपदिष्टवान् । यदि नित्यस्नानमेव सशिरः कुर्यात् । कुर्यादिति चेत् किमु वक्तव्यं शुद्धिस्नानविषये इति कैमुतिकन्यायेनापि तत्प्राप्यते । एवमुक्तप्रकारेण अशुद्धान् स्वयमशुद्धः यदि स्पृशति मतिपूर्वं तदा उपवासेन कृच्छ्रेण वा शुद्ध्यति । तथा _
\begin{verse}
मानुषास्थिवसांविष्ठामार्तवं मूत्ररेतसी। \\
मज्जनं शोणितं वापि यदि संस्पृशेत् ॥ \\
स्नात्वापमृज्यलेपादीन् आचम्य स शुचिर्भवेत् ।\\ 
तान्येव स्वानि संस्पृश्य पूतः स्यात् परिमार्जनात् ॥
\end{verse}
इति च विधीयते । अत्र स्नात्वापमृज्य इत्यत्र आर्थिकः क्रमः न तु पाठक्रमः। अतः विष्ठादीनां लेपमपमृज्य स्नात्वा आचम्य शुद्धः स्यात् । पराशरेण चितिवृक्ष-चिति-यूप-चण्डालान् सोमविक्रयिणं च विप्रः स्पृष्ट्वा सचेलः स्नायात् इति विहितम् । पुनर्हारीतेन- श्वपच-मुष्टिक-प्रेत-हारक-वसादि संस्पृश्य ‘देवीराप’ इत्येताभिः अन्तर्जले स्नातः पूतो भवति इत्युक्तम् । व्यसेनापि - 
\begin{verse}
भासवानरमार्जारखरोष्ट्राणां शुनां तथा ।\\
शूकराणाममेध्यं च स्पृष्ट्वा स्नायात् सचैलकम् ॥  
\end{verse}
इति एतेषां मलस्पर्शने स्नानमुक्तम्। च्यवनस्तु शुद्धिप्रकाशे _ 

श्वपाकं प्रेतधूमं देवद्रव्योपजीवनम् । ग्रामयाजकं यूपं चितिकाष्ठमद्यं मद्यभाण्डं सस्नेहं मानुषास्थिशवस्पृष्टं महापातकिनं शवं स्पृष्ट्वा सचैलमम्भो अवगत्य उत्तीर्य अग्निमुपस्पृशेत् ; गायत्र्यष्टशतं जपेत् । घृतं प्राश्य पुनः स्नात्वा त्रिराचमेत् इति विहितं दीर्घशौचम् । किन्तु अन्यवचनबलात् अग्निस्पर्शादिकं कामकृते ; कामतोऽभ्यासेन स्पर्शे च वेदितव्यम् । अन्यथा स्नानादिमात्रमेव। शङ्खेनापि _  रथ्याकर्दमतोयेन ष्ठीवनाद्येन वा पुनः। नाभेरूर्ध्वं नरः स्पृष्टः सद्यः स्नानेन, निस्नेहं स्पृष्ट्वा आचमनेन गवालम्बनेन भास्करेक्षणेन वा भवतीत्युक्तम्।

\section*{अथ रजस्वलाशौचविचारः}

पूर्वोक्तेषु वचनेषु रजस्वलास्पर्षे सचेलस्नानं विहितम् । स्त्रीणां रजो दर्शनं स्वभावसिद्धम् । “कजेंदुहेतुप्रतिमासर्तवम्” इति बृहज्जातकादिष्वपि ज्यौतिषग्रन्थेषु तत्कारणं प्रकाशितम् । तस्मिन् काले स्त्री अशुचिर्भवति। तादृशं च रजः त्रिदिनं विद्यत इति त्रिदिनाशौचं प्रतिपादितम् । क्वचित् चतुर्थेऽह्नि अपि तद्दर्शनात् यावत् रजोदर्शनं तावत् दैवपितृकर्मसु अनधिकारः उक्तः। प्रथमत्रिदिनेषु असृग्दर्शनात् अस्पृष्यरूपमशौचं विहितम् । जननमरणनिमित्तादृष्टाशौचवतः स्पर्शन एव स्नानस्य विधानात् स्वयं दृष्टाशौचवत्याः स्पर्शने स्नानविधानं नानुचितम्।

अत्र काश्चन विवदन्ते स्वच्छन्देन, यत् रजोदर्शनं स्वाभाविकम् । अपि चाशौचविधानं शोणात्मकम्। स्वातन्त्र्यापहारकम्, आधुनुककाले कार्यभारे च अनुष्टातुमयोग्यम् इति च। तदत्रावधेयं यत् मानवमलानां मूत्रपुरीषादीनां विसर्गे यथा अस्माभिः शौचं क्रियते तथा रजोदर्शने शौचं न साध्यम् । मूत्रपुरीषादीनां विसर्गो यथा अस्मदधीनः तथा रजोदर्शनं न स्त्र्यधीनम् । अपि च तदसृग्दर्शनं न रोधनीयञ्च। तथा सति तन्निमित्तमशौचविधानं नास्वाभाविकम् । यथा मूत्रपुरीषादीनामवरोधं कृत्वापि दैवपित्रादिकर्म कर्तुं शक्यते तथावरोधं कर्तुं आर्तवविषये न साध्यमिति आर्तवे स्त्रीणां त्रिदिनमशौचं नास्वाभाविकम् । निमित्ते सति नैमित्तिकं सामान्यमेव। अत एव च यावद्रजोदर्शनं तावत् कर्मानधिकाररूपमशौचमपि विशेषेण विहितम्। त्रिदिनं तु अस्पृश्यत्वरूपं सामान्यमशौचम् । नात्र स्त्रीशोषणमुद्दिष्टम् । यथा पैठीनसिना; स्कन्दने छर्दने काककेशमलने अनुदकमूत्रमलकरणे खरोष्ट्रचाण्डालस्पर्शने च स्नानं विहितं यथा वा च्यवनेन श्वपाक-प्रेतधूम-चितिकाष्ठ-महापातकि-मद्यभाण्ड-शवानां च स्पर्शे स्नानाग्निस्पर्शन-गायत्रीजप-घृतप्राशनमिति विहितं , तथा अत्रापि यावद्रजोदर्शनं तावदाशौचविधानं कृतम् । यदि स्त्रीणां रजोदर्शनं स्वाभाविकम् अतो नाशौचमनुष्ठेयमिति उच्यते तदा पूर्वोक्तानामपि स्वाभाविकत्वात् तथा जननमरणादीनामपि स्वाभाविकत्वात् अशौचमवधेयं स्यात् । यद्येवं तदा धर्मव्यवस्था एव उच्छिद्येत । अतोऽत्र निष्कारणं धर्मोऽनुष्ठेयः। तमिममेवांशमभिप्रेत्य आपस्तम्बेनोक्तम्  _
\begin{verse}
नेमं लौकिकमर्थं पुरस्कृत्य धर्ममाचरेत् । इति । 
\end{verse}
अशौचं च अदृष्टरूपत्वात् वचनादेव तदनुष्ठानं कार्यम् । तथात्र पुराणे कथापि अर्थवादरूपेण श्रूयते । कदाचित् इन्द्रो वृत्रासुरं वज्रायुधेन जघान । तेन ब्रह्महत्यादोषेण लिप्तः स देवादीन् पप्रच्छ तद्दोषं स्वीकर्तुम् । ते सर्वेऽपि नाङ्गीचक्रुः । तदा तेन प्रार्थिताः स्त्रियः तद्दोषं निवारणवरं सन्तानलाभवरञ्च लेभिरे इति । अतः प्रतिमासावर्ते सा अस्पृश्या भवति इति च । 

एवं दृष्टादृष्टोभयरूपदोषसद्भावात् तदिदमशौचमवश्यमनुष्टेयमिति भाति । किन्त्विदानीं नगरग्रामवासिभिः बहुभिः सोऽयं विधिः नोपचर्यत इति तु चिन्तनीयो विषयः ।

\section*{अथ वस्त्रविचारः}

शौचे वस्त्रविचारोऽपि प्राधान्यं भजते । यदि वस्त्रं शुद्धं भवति तदा कर्मसु शुद्धो भवति अधिकारी । अस्मत् कर्णाटकप्रदेशे कन्नडभाषया ‘मडि’ ‘मैलिगे’ इति पदप्रयोगः क्रमेण शुद्ध्यशुद्ध्यर्थे क्रियते । तत्रापि वाससः विषये एव अयं प्रयोगः बाहुल्येन क्रियते । प्रायः धौतवस्त्रे ‘मडि’ पदप्रयोगः रूढः । ताद्शं ‘मडि’वस्त्रं धौतम् अन्यैरस्पृष्टं च भवति । तच्च वस्त्रं अधौतमपि ‘मडि’ पदेन बोध्यते । स्त्रीवस्त्रविषयेऽपि इयमेव रूढिः दृश्यते । धौतं कौशेयं च ताभिः ध्रियते ‘मडि’वस्त्रार्थम् । तदिदं वस्त्रं कर्माङ्गम् । अत एव “शुचिना कर्म कर्तव्यम्” इत्यादिषु वस्त्राशौचमपि अन्तर्हितं वर्तते । यथा आचमनस्नानादिना शुद्धिः , अशुचिवस्त्रेणाशुद्धिश्च प्रभवति । अतश्च नित्यनैमित्तिककाम्यकर्मसु वस्त्रशौचं सुतरामपेक्षितम् । तदर्थं वस्त्रविधयः स्मृताः । तत्र मितमिश्रकृताह्निकप्रकरणे महाभारतवचनं स्मर्यते, यथा _
\begin{verse}
अन्यदेव भवेद्वासः शयनीये सदैव हि ।\\
अन्यद्रथ्यासु देवानामर्चायामन्यदेव हि ॥ इति ।
\end{verse}
अत्र शयनरथ्यादेवार्चनादिषु  पृथग् वासो विधीयते। स्मृत्यनुगुणं प्रातः, रात्रेः पश्चिमयामरूपे काले, रात्रेः चतुर्दशे मुहूर्ते ब्रह्माख्ये वा उत्थाय हरिं स्मृत्वा मुखप्रक्षालनादि शौचं कृत्वा शयनीयवस्त्रं विसृज्य वस्त्रान्तरं परिधाय आचम्य उपदिष्टं कर्म वेदाध्ययनादिकं कुर्यात् । अत्र व्यवस्थाविशेषस्तु आह्निकप्रकाशस्मृतिचन्द्रिकादिषु महानिबन्धेषु प्रदर्शितः। रात्रेरुपान्त्यब्राह्ममुहूर्ते उत्थाने स्नानानन्तरमेव वस्त्रान्तरधारणं कृत्वा सन्ध्यावन्दनाङ्गवस्त्रान्तरधारणाङ्गं च द्विराचमनं तन्त्रेण कुर्यात् । एवमार्यसमयदर्शनात् शयनोपयुक्तं वासोऽशुद्धम् । तेन दैवादि कर्म न कार्यम्। दक्षेण तु स्पष्ट्तया उक्तम् । तत्र अशुचिर्भवेत् इत्यनुवदन् दक्षः- 
\begin{verse}
स्वप्नात् वस्त्रविपर्यासात् क्षुतादध्वपरिश्रमात् । 
\end{verse}
इत्युवाच । परस्य शयनस्पर्शनादपि अशुचिर्भवति। एवञ्च वस्त्रस्य विपर्यासात् अपि अशुचित्वं भवति। विपर्यासो नाम अधोवस्त्रस्य उपरि धारणम्, उपरि वस्त्रस्य अदोधारणम् । तथा च क्षुतेन अध्वपरिश्रमेण जनितस्वेदादिना च वस्त्रं व्यक्तिश्च अशुचिर्भवति। व्यक्तेरशुचित्वञ्च आह्निके 
\begin{verse}
मुखे पर्युषिते नित्यं भवत्यप्रत्ययो नरः । \\
लालादिस्वेदसमाकीर्णः शयनाद्युत्थितः पुमान् ॥\\
अस्नात्वा नाचरेत् कर्म जपहोमादि किञ्चन ।
\end{verse}
इति च उक्तम्। यत्र स्नानमुच्यते तत्र “स्नातः पूर्ववस्त्रं नाप्रक्षालितं बिभृयात्” इति वचनेन अशौचज्ञानकाले धृतवस्त्रस्याप्रायत्यमभिप्रेतं भवति। ततः स्नात्वा वस्त्रान्तरं शुद्धम् वा पूर्वधृतं प्रक्षालितं वा बिभृयात् । तत एव स शुद्धो भवति । इत्थं चाशुचिस्पर्शे वस्त्रस्य व्यक्तेश्चा शुचित्वम् । अङ्गिरास्तु _
\begin{verse}
उत्थाय पश्चिमे यामे रात्रिवासः परित्यजेत्  ।
\end{verse}
इत्युपदिदेश । यदि रात्रेः पश्चिमे यामे उत्थाय स्नानात् पूर्वं वेदमभ्यसति चेत् तदा प्रबोधनिमित्तमाचमनं नेत्रप्रक्षालनोत्तरं कार्यम् । किन्तु भृगुणा अशुचिवस्त्रेणाचमनं निन्दितम्, यथा् \_
\begin{verse}
विना यज्ञोपवीतेन तथा धौतेन वाससा। \\
मुक्त्वा शिखां वाप्याचामयेत् ( वा नाचामेत्?) कृतस्यैव पुनः क्रिया इति ॥ 
\end{verse}
यदि अधौतवाससा आचामेत् तदा कृतस्य पुनः क्रिया कार्या इति। अतः प्रबोधननिमित्तकं वासो धारणनिमित्तं द्विराचमनं तन्त्रेण कार्यम् । तदा वेदाध्ययनादौ शुद्धो भवति। तत्र वाससो विपरिधानं व्यासेन निषिद्धम् । यथा \_
\begin{verse}
नोत्तरीयमधः कुर्यात् नोपर्यधस्तमम्बरम्। इति । 
\end{verse}
अत्र वस्त्रद्वयपरीधानं त्वावश्यकम् । उत्तरीयमधोवस्त्रं चेति वस्त्रद्वयमुच्यते। उत्तरीयं विना श्रौतं स्मार्तं च कर्म न कार्यम् । तदाह भृगुः \_ 
\begin{verse}
विकच्छोऽनुत्तरीयश्च नग्नश्चावस्त्र एव च । \\
श्रौतं स्मार्तं तथा कर्म न नग्नः चिन्तयेदपि ॥ इति। \\
उत्तरीयं यज्ञोपवीतं तु  धार्यते च द्विजोत्तमैः । \\
तथा सन्धार्यते यत्नादुत्तराच्छादनं शुभम् ॥ इति। 
\end{verse}
यदा धौतवस्त्राभावः तदा योगियाज्ञवल्क्यवचनात् शाण्मादिकं धार्यम् । 
\begin{verse}
अभावे धौतवस्त्रस्य शाणक्षौमाविकानि च ।\\
कुतपं योगपट्टं वा विवासा येन वै भवेत् ॥ इति । 
\end{verse}
उत्तरीयाभावे पारस्करवचनेन\_ एकं चेद्वासो भवति चेत् -
\begin{verse}
तस्यैवोत्तरवर्गेणैवोत्तरीयं कार्यम् । \\
नार्द्रमेकं च वसनं परिदध्यात् कदाचन ॥ 
\end{verse}
इति । जाबालिवचनेनापि शुष्कवस्त्रं एव धार्यम् । वस्तुतः जले आर्द्रवाससा स्थले शुष्कवस्त्रेण कर्मकुर्यात् इत्यस्य नियमस्यायमपवादरूपः। तत्र शुष्कवस्त्रस्य सर्वथाभावे एव अपवादनियमोऽनुसरणीयः। किन्तु केषुचिज्जनपदेषु आर्द्रवस्त्रधारणम् आचाररूपेण दृश्यते। कथञ्चित् तस्य साधुत्वं तैरङ्गीक्रियते। शुद्धेरेव प्राधान्यं स्वीकृत्य ईदृशाचारः प्रवृत्तः। सप्तवाताहतत्वात् “आर्द्रमपि शुष्कवत् भवतीति” वचनबलात् शुष्कवस्त्रप्रतिनिधित्वमस्य सिद्ध्यति । ततश्च शुष्कवस्त्रेण यत्कर्तव्यं तत् तत्प्रतिनिधिना सप्तवाताहतार्द्रवस्त्रेण कर्तुं शक्यते । अयमेवार्द्रवस्त्रधारणाचारस्य हेतुः। वचनमेवात्र प्रमाणमपि। अथापि शुष्कवस्त्रस्य प्रथमकल्पत्वात् धौतशुष्ककार्पासस्याभावे शाणक्षौमाविककुतपयोगपट्टं वा धार्यम् इत्युक्तत्वात् शुष्कवस्त्रधारणमेव प्रशस्तम् । तत्रापि आविकमपि प्रशस्तम् । तदुक्तं भारते- 
\begin{verse}
आविकं तु सदा वस्त्रं पवित्रं राजसत्तम । \\
पितृदेवमनुष्याणां क्रियायां च प्रशस्यते॥ \\
धौताधौतं तथा दग्धं सन्धितं रजकाहृतम् । \\
रक्तमूत्रशकृल्लिप्तं तथापि परमं शुचि ॥\\
रेतस्पृष्टं रजस्पृष्टं स्पृष्टं मूत्रपुरीषयोः ।\\
रजस्वलाभिसंस्पृष्टम् आविकं सर्वदा शुचि ॥ इति ।
\end{verse}
तत्राविकेतराणि दग्धादीनि वस्त्राणि त्याज्यानि। तदर्थमेव आह्निके स्मृतिवचनम् _
\begin{verse}
दग्धं जीर्णं च मलिनं मूषकोपहतं तथा । \\
खादितं गोमहिष्याद्यैः तत्त्याज्यं सर्वदा द्विजैः॥ इति । 
\end{verse}
एवञ्च नेजकधौतमशुचि । पुरुषान्तरधृतमशुचि । ईषद्भागधौतं स्त्रीधौतं पूर्वेद्युर्धौतम् अधौतवस्त्रसम्स्पृष्टं चाशुचि। तत्र धौतवस्त्रासम्भवे प्रणवेन सम्प्रोक्ष्य परिदध्यात् आधौतमेव । 

एवं वस्त्रनियमे दृष्टेऽपि अष्टहस्तं नवं श्वेतं, ईषद्धौतं नवं श्वेतं , उत्थायवाससी शुक्ले इति च शुक्लवस्त्रस्यैव प्राधान्येन धारणं विहितम् । तथा नीलातिरक्तवस्त्रयोः निषेधोपि दृश्यते । कन्नडभाषयां मडिपदं श्वेतार्थे वर्तते । “मडिशश्कुली,” “मडिमाणि” इत्यादि प्रयोगे तस्य स्पष्टता भवति । पितृकर्मणि च श्वेतवस्त्रस्यैव विधानं दृश्यते । तदित्थं मडि इत्युक्ते श्वेतवस्त्रमेव भवेत् । मडिवस्त्रधारणं नाम शुद्धश्वेतवस्त्रधारणमेव । तादृशवस्त्रसम्पर्केण पुरुषेपि मडि प्रयोगः क्रियते । “स मडिपुरुषः मा स्पृश” ! इत्यादिषु तत्स्पष्टं भवति । एवं मडिवस्त्रं नाम श्वेतं शुद्धं च वस्त्रम् । श्वेताभावे शुद्धं वा भवेदेव । शुद्धिश्च प्रोक्षणप्रक्षालनादिना ।

\section*{उपसंहारः-} 

इत्थं बहवो विधिनिषेधात्मका नियमाः शास्त्रे उपदिष्टाः। यत्नः सदा कार्यः, इत्यादीनि वचनानि शौचस्यावश्यकतां बोधयन्ति। अतः पूर्वोक्तेषु स्पर्शाशौचेषु आचमनादीनि उक्तशौचविधानानि चरणीयानि। आचमनमङ्गलदर्शनादिषु शौचापादकत्वं पावकत्वं च वर्तते। अत एव आचारमयूखे “शुद्धिर्नाम अनर्हतापादकदोषानिवर्तकः संस्कारविशेषः” इत्युक्त्वा मङ्गलदर्शनादिकमुक्तम्। तत्र “लोकेऽस्मिन् मङ्गलान्यष्टौ ब्राह्मणा गौः हुताशनः। हिरण्यं सर्पिषादित्या आपो राजा तथाष्टमः”॥ इति नारदवचनम् तथा “अग्निचित् कपिला सत्री राजा भिक्षुः महोदधिः। दृष्टमात्राः पुनन्त्येते तस्मात् पश्येत् सदा बुधः” इति मनुष्यवचनं च उदाहृतम् । एतेषां दर्शनं स्पर्शनभक्षणादिना शुद्धिरापाद्यते। तां शुद्धिं देशकालाद्यनुगुणं प्रकल्पयेत् । तदुक्तमाचारेन्दौ बोधायनेन _
\begin{verse}
देशं कालं तथात्मानं द्रव्यं द्रव्यप्रयोजनम् । 
\end{verse}
उपपत्तिमवस्थां च ज्ञात्वा तथापि आभ्यन्तरशौचविहीनस्य बाह्यशौचस्य परमार्थता नास्ति। अत एव व्याघ्रपादेनोक्तम् _
\begin{verse}
गङ्गातोयेन कृत्स्नेन मृद्भारैश्च नगोपमैः। \\
आमृत्योश्चाचरन् शौचं भावदुष्टो न शुद्ध्यति ॥ इति ।
\end{verse}
बाह्यशौचेन युक्तेन कृतकर्मणा स्वर्गादिफलविपाकशुद्ध्यभावात् ‘न शुद्ध्यति’ इत्यभिप्रायः। अत एव यमेनोक्तं शुद्धिप्रकाशे _
\begin{verse}
सत्यं शौचं तपः शौचं शौचमिन्द्रियनिग्रहः। \\
सर्वभूतदयाशौचमद्भिः शौचं च पञ्चमम् ॥ इति । 
\end{verse}
इत्येवं गोऽश्वपशुन्यायेन आभ्यन्तरशौचमेव शौचम् इति । 

एवञ्च तदुभयं शौचं स्पृष्टाशौचादिदोषनिवर्तकं सर्वैरनुष्ठेयमिति लेखनस्यास्योद्देशः।
