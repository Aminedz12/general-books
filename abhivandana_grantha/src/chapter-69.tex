\chapter{ಗಂಗಾಧರ ಭಟ್ಟರು ನಾ ಕಂಡಂತೆ...}

\begin{center}
\Authorline{ಡಾ. ಪ್ರಶಾಂತ್}
\smallskip
ಆಯುರ್ವೇದ ವೈದ್ಯರು\\
ಪ್ರಶಾಂತಿ ಆಯುರ್ವೇದ ಕ್ಲಿನಿಕ್\\
ಮೈಸೂರು

\end{center}
2005 ನೇ ಇಸವಿ, ಮೈಸೂರಿಗೆ ಬಂದು ಆರು ತಿಂಗಳುಗಳು ಕಳೆದಿತ್ತಷ್ಟೇ, ಆಯುರ್ವೇದದಲ್ಲಿ ಸ್ನಾತಕೋತ್ತರ ಪದವಿ ಮಾಡಲೆಂದು ಮೈಸೂರಿನ ಪ್ರಸಿದ್ಧ ಸರ್ಕಾರಿ ಆಯುರ್ವೇದ ವೈದ್ಯಕೀಯ ಕಾಲೇಜಿಗೆ ಸೇರಿದ್ದೆ. ಆಯುರ್ವೇದಕ್ಕೂ ಸಂಸ್ಕೃತಕ್ಕೂ ಅತ್ಯಂತ ಹತ್ತಿರದ ನಂಟು. ಯಾಕೆಂದರೆ ಆಯುರ್ವೇದದ ಮೂಲ ಗ್ರಂಥಗಳೆಲ್ಲಾ ಇರುವುದು ಸಂಸ್ಕೃತದಲ್ಲೇ. ನಮಗಾದರೋ ಸಂಸ್ಕೃತವೆಂದರೆ ಒಂದು ರೀತಿ ಕಬ್ಬಿಣದ ಕಡಲೆಯಂತೆ. ನಮ್ಮಲ್ಲಿನ 4-5 ಸ್ನಾತಕೋತ್ತರ ವಿದ್ಯಾರ್ಥಿಗಳು ಒಂದು ದಿನ ಚರ್ಚೆ ಮಾಡುತ್ತಾ ಆಯುರ್ವೇದದ ಮೂಲಗ್ರಂಥವನ್ನು ಎಲ್ಲರೂ ಒಟ್ಟಿಗೆ ಎಲ್ಲಾದರೂ ಓದಬೇಕು ಎಂದು ಆಲೋಚಿಸಿದೆವು. ಆದರೆ ಸಂಸ್ಕೃತದ ಅನೇಕ ವಾಕ್ಯಗಳು ಅರ್ಥವಾಗದೆ ಹೋದಾಗ ಯಾರನ್ನು ಕೇಳಬೇಕು? ಎಂಬ ಪ್ರಶ್ನೆ ತಲೆದೋರಿತು. ಒಂದು ದಿನ ಸಾಯಂಕಾಲ ಕಾಲೇಜು ಮುಗಿಸಿ ಕಾಲೇಜಿಗೂ ಬಸ್ ನಿಲ್ದಾಣಕ್ಕೂ ಸಮೀಪವಿರುವ ಓದಲು ಅವಕಾಶವಿರುವ ಸ್ಥಳವೆಲ್ಲಾದರೂ ದೊರೆಯುತ್ತದೆಯೇ ನೋಡೋಣವೆಂದು ಹೊರಟೇಬಿಟ್ಟೆ. ಆಗ ನನ್ನ ಗಮನಕ್ಕೆ ಬಂದಿದ್ದು ಮೈಸೂರಿನ ಪ್ರಖ್ಯಾತ ಮಹಾರಾಜ ಸಂಸ್ಕೃತ ಪಾಠಶಾಲೆ. ಮೊದಲ ಸಲ ಒಳಗೆ ಹೋಗಿ ಅಲ್ಲಿ ಒಂದಿಬ್ಬರು ಸಿಬ್ಬಂದಿವರ್ಗದವರು ಸಿಕ್ಕರು. “ನಾನೊಬ್ಬ ಆಯುರ್ವೇದ ವಿದ್ಯಾರ್ಥಿ. ಆಯುರ್ವೇದ ಮೂಲ ಗ್ರಂಥವನ್ನು ಓದಲು ಸಂಸ್ಕೃತ ಕಲಿಯಬೇಕಿತ್ತು” ಎಂದು ಹೇಳಿದೆ. ಹೇಳಿದ್ದೇ ತಡ,  “ನೋಡಿ ಆ ಕೊನೆಯ ಕೋಠಡಿಗೆ ಹೋಗಿ. ಅಲ್ಲಿ ಗಂಗಾಧರ ಭಟ್ಟ ಎಂಬುವವರಿದ್ದಾರೆ. ನಿಮಗೇನು ಬೇಕೋ ಅದನ್ನು ಅವರು ಕೊಡಿಸುತ್ತಾರೆ” ಎಂದು ಆ ತಕ್ಷಣ ಹೇಳಿದರು.

ಕುತೂಹಲದಿಂದ ಕೋಠಡಿಯೊಳಕ್ಕೆ ಹೋದೆ. ಇನ್ನೂ ವಿದ್ಯಾರ್ಥಿಗಳಿಗೆ ಪಾಠ ನಡೆಯುತ್ತಿತ್ತು. ಸ್ವಲ್ಪ ಹೊತ್ತಿನ  ಅನಂತರ ತರಗತಿ ಮುಗಿದ ಮೇಲೆ ನನ್ನನ್ನು ಕರೆದರು. ಇದು ಗಂಗಾಧರ ಭಟ್ಟರ ಜೊತೆಗಿನ ನನ್ನ ಮೊಟ್ಟ ಮೊದಲ ಭೇಟಿ. ನಮಗೆ ಓದಲು  ಸ್ಥಳಾವಕಾಶ ಕೊಡಬೇಕೆಂದು ಹೇಳಿದ ತಕ್ಷಣ ಅತ್ಯಂತ ಸಂತಸದಿಂದ, “ಇಲ್ಲಿ ಎಲ್ಲಾ ವೇದಗಳ ಅಧ್ಯಯನ ನಡೆಯುತ್ತದೆ. ಆಯುರ್ವೇದಾಧ್ಯಯನವೂ ನಡೆಯುತ್ತದೆ ಎಂದರೆ ನನಗೆ ಇನ್ನೂ ಸಂತೋಷ” ಎಂದು ಸ್ವತಃ ತಾನೇ ನನ್ನನ್ನು ಅಂದಿನ ಪ್ರಾಂಶುಪಾಲರ ಬಳಿ ಕರೆದುಕೊಂಡು ಹೋಗಿ ನಮ್ಮ ಪರವಾಗಿ ಅವರೇ ಪ್ರಾಂಶುಪಾಲರ ಬಳಿ ಮಾತನಾಡಿ ಆಯುರ್ವೇದಾಧ್ಯಯನಕ್ಕೇಂದೇ ಒಂದು ಕೊಠಡಿಯನ್ನು ನೀಡಿದರು. 
ಮೊಟ್ಟಮೊದಲನೆ ದಿನ 4-5 ಜನ ಮಾತ್ರ ಇದ್ದ ನಮ್ಮನ್ನು ಉದ್ದೇಶಿಸಿ ಹೇಳಿದ ಮಾತುಗಳು ಈಗಲೂ ನನಪಿವೆ. ಆಯುರ್ವೇದ ಕಾಲೇಜು ಈಗ ಬೇರೆ ಕಟ್ಟಡದಲ್ಲಿರಬಹುದು, ಅದರೆ ಕಾಲೇಜು ಹುಟ್ಟಿದ್ದೆ ಇಲ್ಲಿ. ಮೊದಲು ಆಯುರ್ವೇದಾಧ್ಯಯನ ಅಧ್ಯಾಪನ ಇಲ್ಲಿ ನಡೆಯುತ್ತಿದ್ದಿದ್ದನ್ನು ನೀವು ಈಗ ಮತ್ತೆ ಪ್ರಾರಂಭ ಮಾಡಿದ್ದೀರಿ, ಅದು ನಿರಂತರವಾಗಿ ನಡೆಯಲಿ ಎಂದು ಆಶೀರ್ವಾದ ಮಾಡಿದರು. 


ದೈವೇಚ್ಛೆಯಿಂದ ಅಂತಹ ಗುರುಗಳ ಆಶಿರ್ವಾದದಿಂದ ಇವತ್ತಿಗೂ ಬೆಳಿಗ್ಗೆ 7.15 ರಿಂದ 8.30ರ ತನಕ ಆಯುರ್ವೇದಾಧ್ಯಯನ ನಡೆಯುತ್ತಲೇ ಬಂದಿದೆ. ಅಷ್ಟು ಮಾತ್ರವಲ್ಲ 4-5 ರಷ್ಟಿದ್ದ ಸಂಖ್ಯೆ ಇವತ್ತು 20-25ರ ವರೆಗೂ ಬಂದಿದೆ. ಸಂಸ್ಕೃತ ಪಾಠಶಾಲೆಯ ಅಧ್ಯಾಪಕರು, ಸಿಬ್ಬಂದಿ ಎಲ್ಲರ ಜೊತೆಗೂ ಒಂದು ರೀತಿಯ ಭಾವನಾತ್ಮಕ ಸಂಬಂಧವೂ ಬೆಳೆದಿದೆ. 2005 ರಿಂದ ಇವತ್ತಿನವರೆಗೂ ಗೂಗಂಗಾಧರ ಭಟ್ಟರ ಜೊತೆ ಆತ್ಮೀಯತೆಯ ಒಡನಾಟ ನಮ್ಮೆಲ್ಲರಿಗೂ ಇದೆ.

\section*{ಗಂಗಾಧರ ಭಟ್ಟರೊಳಗಿನ ಅದ್ಭುತ ಶಿಕ್ಷಕ }

“ತ್ರಿವಿಧಶಿಷ್ಯಬುದ್ಧಿಹಿತಮ್” ಅಂದರೆ ದಡ್ಡ, ಮಧ್ಯಮ ಹಾಗು ಬುದ್ಧಿವಂತ ಎಂಬ ಮೂರೂ ಪ್ರಕಾರದ ಶಿಷ್ಯರಿಗೂ ಹಿತವಾದದ್ದು ಎಂದು. ಒಬ್ಬ ಶಿಕ್ಷಕನಿಗೂ, ಎಲ್ಲಾ ವರ್ಗದ ವಿದ್ಯಾರ್ಥಿಗಳಿಗೂ ಅರ್ಥವಾಗುವಂತೆ ಪಾಠಮಾಡುವ ಸಾಮರ್ಥ್ಯವಿರಬೆಕಂತೆ. ಅಂತಹ ಸಾಮರ್ಥ್ಯವಿರುವ ಶಿಕ್ಷಕರಲ್ಲಿ ಒಬ್ಬರು ಗಂಗಾಧರ ಭಟ್ಟರು. ತರ್ಕಶಾಸ್ತ್ರದಂತಹ ಗಂಭೀರ ವಿಷಯದ ಅಳವಾದ ಜ್ಞಾನ ಹಾಗು ಅನೇಕ ವರ್ಷಗಳ ಅನುಭವ ಹೊಂದಿರುವ ಅವರು ಇದೀಗತಾನೆ ಸಂಸ್ಕೃತಕಲಿಯುವ ಮೊದಲ ಹೆಜ್ಜೆ ಇಟ್ಟಿರುವವರಿಗೂ ಅವರ ಸ್ತರಕ್ಕೆ ಇಳಿದು ಅವರಿಗೆ ತಿಳಿಯುವ ಹಾಗೆ ಪಾಠ ಮಾಡುವ ಕಲೆಯನ್ನು ಕಣ್ಣಾರೆ ಕಂಡಿದ್ದೇವೆ.

ಆಯುರ್ವೇದಕ್ಕೆ ಇದೀಗ ತಾನೇ ಕಾಲಿಟ್ಟಂತಹ ಹೊಸ ವಿದ್ಯಾರ್ಥಿಗಳಿಗೆ ಅರ್ಥವಾಗುವಂತೆ ಶ್ಲೋಕಗಳನ್ನು ಅನ್ವಯಮಾಡುವ “ಆಕಾಂಕ್ಷಾ ಕ್ರಮ”ದ ಪಾಠ, ಅದೇ ರೀತಿ ಸ್ನಾತಕೋತ್ತರ ವಿದ್ಯಾರ್ಥಿಗಳಿಗೆ ಚರಕ ಸಂಹಿತೆಯಲ್ಲಿನ ತರ್ಕಶಾಸ್ತ್ರಕ್ಕೆ ಸಂಬಂಧಿಸಿದ ಗಂಭೀರ ವಿಷಯಗಳ ಬಗ್ಗೆ ಉಪಾನ್ಯಾಸ, ಇವೆರಡನ್ನು ಕೇಳಿದ ನಮಗೆಲ್ಲರಿಗೂ ಸಮಯ ಕಳೆದದ್ದೇ ತಿಳಿಯುತ್ತಿರಲಿಲ್ಲ.
ಒಬ್ಬ ಶಿಕ್ಷಕ ತನ್ನ ಶಿಷ್ಯನ ವಿಧ್ಯಾಭ್ಯಾಸದ ಬಗ್ಗೆ ಮಾತ್ರ ಯೋಚನೆ ಮಾಡುವುದಿಲ್ಲ, ಅವನ ಸರ್ವಾಂಗೀಣ ಅಭಿವೃದ್ಧಿಯ ಕಡೆಗೆ ಗಮನವನ್ನು ಸದಾ ಇಟ್ಟಿರುತ್ತಾನೆ. ಇದೇ ರೀತಿ ವಿದ್ಯಾರ್ಥಿಗಳ ಬಗ್ಗೆ ಅತೀವ ಕಳವಳ ಇರುವ ಶಿಕ್ಷಕರಲ್ಲಿ ಗಂಗಾಧರ ಭಟ್ಟರು ಒಬ್ಬರು. ಎರಡು ಸಂದರ್ಭಗಳಲ್ಲಿ ಅವರ ಹೃದಯ ವೈಶಾಲ್ಯದ ನೇರ ಅನುಭವ ನನಗಾಯಿತು. ಒಮ್ಮೆ ಆಯುರ್ವೇದ ವಿದ್ಯಾರ್ಥಿನಿಯೊಬ್ಬಳು “ತನಗೆ ಸಂಸ್ಕೃತ ಏನೇನೂ ಬರುವುದಿಲ್ಲ, ನನಗೆ ಸಹಾಯ ಮಾಡಿ” ಎಂದು ಕೇಳಲು ಬಂದಿದ್ದಳು. ಆ ಸಂದರ್ಭದಲ್ಲಿ ಭಟ್ಟರು ಅದಾಗಲೇ ತುಂಬಾ ವ್ಯಸ್ತರಾಗಿದ್ದರು. ನೇರವಾಗಿ ಅವಳಿಗೆ ಯಾವುದೇ ರೀತಿಯಲ್ಲಿ ಸಹಾಯ ಮಾಡಲಾಗಲಿಲ್ಲ. ಆದರೂ ಅವರು ಇನ್ನೆರಡು ದಿನದಲ್ಲಿ ನೀನು ಬಾ ವ್ಯವಸ್ಥೆ ಮಾಡುತ್ತೇನೆ ಎಂದರು. ಪುನಃ ಆ ವಿದ್ಯಾರ್ಥಿನಿ ಹೋದ ತಕ್ಷಣವೇ ತನ್ನ ಯಾರೋ ಒಬ್ಬ ಶಿಷ್ಯನನ್ನು ಕರೆದು, “ಆ ಆಯುರ್ವೇದ ವಿದ್ಯಾರ್ಥಿನಿಗೆ ಸಂಜೆ ಸಂಸ್ಕೃತ ವ್ಯಾಕರಣ ಹೇಳಿಕೊಡಬೇಕು” ಎಂದರು. ಆ ಹುಡುಗ ಸಲ್ಪ ಹಿಂಜರಿದಾಗ “ಹೇಳಿದಷ್ಟು ಮಾಡು ಅಷ್ಟೇ” ಎಂದರು. ಅನಂತರ ಅವರು ಹೇಳಿದ ಮಾತು ಇವತ್ತಿಗೂ ನೆನಪಿದೆ. “ಮೊದಲೇ ಸಂಸ್ಕೃತವೆಂದರೆ ಮೂಗು ಮುರಿಯುವ ಕಾಲವಿದು. ಅದರಲ್ಲೂ ಯಾರೋ ಒಬ್ಬರು ಸಂಸ್ಕೃತ ಕಲಿಸಿ ಎಂದು ಬಂದಾಗ ನಾವಲ್ಲದೇ ಮತ್ತಾರು ಈ ಜವಾಬ್ದಾರಿ ತೆಗೆದುಕೊಳ್ಳಬೇಕು?” ಎಂದರು.

“ಇಲ್ಲಿನ ವಿದ್ಯಾರ್ಥಿಗೆ ಒಂದು ಅವಕಾಶ ಸಿಕ್ಕರೆ ಅವನಿಗೂ ಕಲಿಸಲು ಒಂದು ಅನುಭವ ದೊರೆತಂತಾಗುತ್ತದೆ. ಅಷ್ಟು ಮಾತ್ರವಲ್ಲ ಪಾಠಶಾಲೆಗೆ ಬರುವ ಎಷ್ಟೊ ವಿದ್ಯಾರ್ಥಿಗಳು ಆರ್ಥಿಕ ಸಂಕಷ್ಟದಲ್ಲಿರುತ್ತಾರೆ. ಅವರಿಗೆ ಅಷ್ಟಿಷ್ಟಾದರೂ ಸಹಾಯವಾದಂತಾಗುತ್ತದೆ” ಎಂದರು. 
ಎಷ್ಟೊಂದು ಆಳವಾದ ಚಿಂತನೆ ತನ್ನ ನೈತಿಕಜವಾಬ್ದಾರಿ, ಆದರೂ ತನಗೆ ಲಾಭ ಬೇಕೆಂದು ದುರುದ್ದೇಶವಿಲ್ಲದೆ ಮುಂದಿನ ಪೀಳಿಗೆಯನ್ನು ತಯಾರು ಮಾಡಬೇಕೆಂಬ ಅವರ ಕಳಕಳಿಯೇ ಅವರಲ್ಲಿನ “ಆದರ್ಶ ಶಿಕ್ಷಕ”ನ ಹೆಗ್ಗುರುತು.

ಅದೊಂದು ದಿನ ಭಟ್ಟರ ಜೊತೆ ಪಾಠಶಾಲೆಯ ವಿದ್ಯಾರ್ಥಿನಿಲಯಕ್ಕೆ ಭೇಟಿ ನೀಡುವ ಅವಕಾಶ ಸಿಕ್ಕಿತ್ತು. ಒಂದು ಕೊಠಡಿಯಲ್ಲಿ ವಿದ್ಯಾರ್ಥಿಯೊಬ್ಬ ಮಲಗಿದ್ದ. ಅವನಿಗೆ ಜ್ವರವಿದೆಯೆಂದು ತಿಳಿಯಿತು ತಕ್ಷಣ ಅವನ ಸಹಪಾಠಿಯೊಬ್ಬನನ್ನು ಕರೆಸಿ ತಕ್ಷಣ ವೈದ್ಯರಲ್ಲಿ ಕರೆದುಕೊಡುಹೋಗಿ ಔಷಧಿ ಕೊಡಿಸಿ ಬಂದ ಮೇಲೆ ಏನಾಯಿತೆಂದು ನನಗೆ ಹೇಳಬೇಕು ಎಂದು ತಾವೇ ದುಡ್ಡನ್ನು ಕೊಟ್ಟು ಕಳುಹಿಸಿದರು. ಆ ಸರಳತೆ ಸಹೃದಯತೆ ನಾವೆಲ್ಲರೂ ಕಲಿಯಬೇಕಾಗಿರುವ ಅಮೂಲ್ಯವಾದ ಪಾಠವಾಗಿದೆ.

ನಮ್ಮ ಚಿಕಿತ್ಸಾಲಯದಲ್ಲೇ ಆಗಲಿ ಅಥವಾ ಬೇರೆ ಎಲ್ಲೇ ಆಗಲಿ  ಯಾರಾದರೂ ಸಂಸ್ಕೃತ ಬಲ್ಲವರೆಂದು ಗೊತ್ತಾದರೆ ಸಾಕು ನಿಮಗೆ ಪಾಠಶಾಲೆ ಗೊತ್ತೇ?  ಅಲ್ಲಿ ಗಂಗಾಧರ ಭಟ್ಟರು ಗೊತ್ತೇ? ಎಂದು ಕುತೂಹಲದಿಂದ ಕೇಳುವುದು ಸಾಮಾನ್ಯವೇ ಆಗಿತ್ತು. ಆಗಲೇ ಗೊತ್ತಾಗಿದ್ದು ಗಂಗಾಧರ ಭಟ್ಟರ ಅದ್ಭುತ ವ್ಯಕ್ತಿತ್ವ. ಗಂಗಾಧರ ಭಟ್ಟರ ಪರಿಚಯವಿಲ್ಲದವರೇ ಇರಲಿಲ್ಲ. ಆಷ್ಟು ಮಾತ್ರವಲ್ಲದೇ ಎಲ್ಲರೂ ಅವರನ್ನು “ಗುರು” ಎಂಬ ಗೌರವ ಭಾವನೆಯಿಂದ ವ್ಯಕ್ತಪಡಿಸುತ್ತಿದ್ದುದನ್ನು ನೋಡಿದಾಗ ನನಗೂ ಒಂದು ರೀತಿಯ ಹೆಮ್ಮೆ ಎನಿಸುತ್ತಿತ್ತು.

ಅಂತಹ ಗಂಗಾಧರ ಭಟ್ಟರು ನಿವೃತ್ತಿಯಾಗುತ್ತಿದ್ದಾರೆ. ಆದರೆ ನಿವೃತ್ತಿ ಎನ್ನುವುದು ಅವರ ಸರಕಾರೀ ಸೇವೆಗೇ ಹೊರತು ಅವರೊಳಗಿನ ಶಿಕ್ಷಕನಿಗಲ್ಲ. ಅವರೊಳಗಿನ ಶಿಕ್ಷಕ ನಿವೃತ್ತಿ ಹೊಂದುವುದೂ ಇಲ್ಲ. ಈ ಸಂದರ್ಭದಲ್ಲಿ ದೇವರು ಅವರಿಗೆ ಆಯುರಾರೋಗ್ಯವನ್ನು ನೀಡಲಿ ಎಂದು ಎಲ್ಲರೂ ಪ್ರಾರ್ಥಿಸೋಣ.
