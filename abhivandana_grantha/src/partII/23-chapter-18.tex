{\fontsize{15}{17}\selectfont
\presetvalues
\chapter{मीमांसादर्शनसम्मतः मोक्षोपायः}

\begin{center}
\Authorline{प्रो. सुब्राय वि. भट्टः}
\smallskip

मीमांसाविभागाध्यक्षः\\
राष्ट्रियसंस्कृतसंस्थानम् \\
(मानितविश्वविद्यालय:)\\
राजीवगान्धीपरिसरः शृङ्गेरी
\addrule
\end{center}
धर्मार्थकाममोक्षाख्येषु चतुर्विधपुरुषार्थेषु मोक्ष एव परमपुरुषार्थः~। तत्र आद्यास्त्रयः साधनरूपाः अन्तिमस्तु मोक्षः साध्यरूपः~। मानवजन्मनः परममुद्देश्यं तु निरतिशयसुखप्राप्तिरूपः मोक्षः~। तदतिरिच्य सर्वेऽपि सुखविशेषाः सातिशयाः इति वैदिकसिद्धान्तः~। 

सर्वेष्वपि दर्शनेषु मोक्षः तत्साधनानि तत्प्राप्तौ साधनसम्पत्तिः इत्येवमादयः विचाराः विशिष्य विमृष्टाः~। आस्तिकदर्शनेष्वपि विशेषतः उपनिषदर्थप्रतिपादिते वेदान्तदर्शने “नान्यः पन्था विद्यतेऽयनाय” इत्यादिवेदवाक्यानुसारेण ज्ञानादेव हि कैवल्यमिति ब्रह्मज्ञानमेव मोक्षसाधनमिति वैदिकसिद्धान्तः प्रतिपादितः~। 

कर्मकाण्डात्मकस्य वेदभागस्य अर्थनिर्णयार्थं प्रवृत्ते पूर्वमीमांसाशास्त्रेऽपि मोक्षतत्साधनविचारः कृतः~। धर्मविचारार्थं प्रवृत्तस्य पूर्वमीमांसाशास्त्रस्य मोक्षविचारविमर्शने का प्रसक्तिः इति चिन्तायाम् इयमस्ति सङ्गतिः~। 

तथा हि - ‘ज्योतिष्टोमेन स्वर्गकामो यजेत’ इत्यादिभिः वेदवाक्यैः स्वर्गादिफलोद्देशेन ज्योतिष्टोमादयः यागाः विहिताः~। तत्र च स्वर्गपदार्थः कः ? मोक्षापेक्षया भिन्नः उत स्वर्ग एव मोक्षः इति विचारः प्राप्तः~। 

तदा स्वर्गः कः ? मोक्षः कः ? तयोः साधनानि कानि इत्यादिविचाराः प्राप्ताः~। एवमेव कालान्तरदेहान्तरदेशान्तरभाविस्वर्गादिफलभोक्ता कः ? इति प्रश्ने स तु नित्यः शरीराद्यपेक्षया अन्यः जीवः इति सिद्धान्तितं भाष्ये~। तदा जीवः परमात्मापेक्षया भिन्नः उताभिन्नः, तस्य कर्तृत्वादिकं कथम् ? तस्य मोक्षः कथं सिध्यति ? इत्यादयः मोक्षतदुपायविचाराः प्राप्ताः~। 

कर्मसमुच्चितज्ञानसाध्यो मोक्षः इति मीमांसासम्मतः पक्षः~। तथा हि –
\begin{verse}
सर्वत्रैव हि विज्ञानं संस्कारत्वेन गम्यते~। \\
पराङ्गं चात्मविज्ञानादन्यत्रेत्यवधारणात्~॥ (श्लो.वा.व्या.अ.)
\end{verse}
इत्यादिना कुमारिलभट्टपादैः वार्तिके त्रिविधमात्मज्ञानम् अभिहितम्~। क्रत्वर्थं पुरुषार्थं मोक्षार्थञ्चेति~। 

“अविनाशी वा अरेऽयमात्मानुच्छित्तिधर्मा मात्राऽसंसर्गस्तु अस्य भवति” इति मात्राशब्देन भूतेन्द्रियधर्माधर्मावभिधीयेते~। “आत्मा वाऽरे द्रष्टव्यः” इति शरीरादिव्यतिरिक्तसंसारिकर्तृभोक्तृनित्यात्मप्रतिपादकस्य आत्मज्ञानविधेः साध्याकाङ्क्षायां फलश्रवणाभावेन दृष्टसम्भवे अदृष्टकल्पनानुपपत्तेः योग्यतया कर्मप्रवृत्तिसिध्यर्थत्वेन क्रत्वर्थत्वमङ्गीकृतम्~। 

अनारभ्याधीतस्य च “यदेव विद्यया करोति श्रद्धयोपनिषदा तदेव वीर्यवत्तरं भवति” इति श्रुत्या उपनिषच्छब्दवाच्यस्य आत्मज्ञानस्य क्रत्वर्थत्वम् अङ्गीकृतम्~। तदुक्तं सम्बन्धवार्तिके –
\begin{verse}
आत्मा ज्ञातव्य इत्येतन्मोक्षार्थं न च चोदितम्~। \\
कर्मप्रवृत्तिहेतुत्वमात्मज्ञानस्य लक्ष्यते~॥ इति~। 
\end{verse}
अतश्च संसारिकर्तृभोक्तृनित्यात्मज्ञानस्य मोक्षार्थत्वं निरस्य पारलौकिकफलककर्मप्रवृत्ति निवृत्तिहेतुत्वं स्वीकृतम्~। 

\textbf{आत्मज्ञानस्य पुरुषार्थत्वम्}, यथा – “अपहतपाप्मा विजरो विमृत्युर्विशोको विजिघत्सोऽपिपासः सत्यकामस्सत्यसङ्कल्पः सोऽन्वेष्टव्यः स विजिज्ञासितव्यः” इत्यादिश्रुत्या पाप-जरामृत्यु-शोक-क्षुत्पिपासाराहित्य-काम्यमानफलप्राप्ति-प्रयत्नानपेक्षसङ्कल्पमात्राधीनसिद्धि\-रूप- गुणविशिष्टात्मज्ञानपूर्वकं यद्वेदान्तवाक्यार्थावधारणात्मकजिज्ञासासहितस्य तदवधारितात्मरूपार्थानुचिन्तनात्मकान्वेषणात्मकमपहतपाप्मत्वादिगुणविशिष्टात्मज्ञानं विहितम्~। 

तथा “बोद्धव्यो मन्तव्यः” इति श्रुत्या न्यायविमर्शनाख्यमननसहितस्य तन्निर्धारितात्मरूपसाक्षात्कारबोधात्मकस्य सगुणात्मज्ञानस्य विधानम्, तदुभयविध्योः पुरुषे कर्मणि वा दृष्टफलासम्भवेऽदृष्टापेक्षायां श्रुत्याद्यसम्भवेन क्रत्वर्थस्याप्यभावात् च “सर्वांश्च लोकानाप्नोति सर्वांश्च कामानाप्नोति” “तरति शोकमात्मवित्” इत्यादिवाक्यशेषसमर्पित अभ्युदयफलार्थत्वात् पुरुषार्थत्वमेव~। ततश्च तयोः पुरुषार्थत्वम्~। 
\begin{verse}
तृतीयप्रकारः – \textbf{आत्मज्ञानस्य मोक्षार्थत्वम्~। }
\end{verse}
तथा हि – “आत्मानमुपासीत” इति श्रुत्या केवलनिर्गुणात्मविषयावबोधपर्यन्तं स्पष्टम् अपरोक्षरूपात्मतत्त्वज्ञानं विहितम्~। तस्य “स खल्वेवं वर्तयन् यावदायुषं ब्रह्मलोकमभिसम्पद्यते न स पुनरावर्तते” इत्यपुनरावृत्तिरूपं परमात्मप्राप्त्यवस्थारूपं वाक्यशेषोपनीतं फलमेव मोक्षः~। 

परमात्मप्राप्तेः शरीरसम्बन्धोपाधिककर्तृत्वभोक्तृत्वात्मकसंसाररूपावस्थापरित्यागेन अकर्तृभोक्त्रात्मकासंसारिरूपनिजावस्थात्मकत्वेन अजन्यत्वात् युक्तमक्षय्यत्वमिति परमात्मप्राप्त्यवस्थारूपत्वमुक्तं वार्तिके~। 

अनौपाधिकत्वरूपाभिप्रायः परमशब्दः~। पूर्वकृताशेषबन्धहेतुकर्मक्षयाच्च असंसारित्वप्राप्तौ कर्तृत्वाभावेन कर्मान्तरत्वं नोपपद्यते~। तदा कारणाभावात् पुनर्बन्धासम्भवेन प्राप्तापुनरावृत्तेः अनुपादानात् फलतदक्षय्यरूपार्थद्रव्याभिधाननिमित्तो वाक्यभेदो नास्तीति अपुनरावृत्तेः ब्रह्मप्राप्तितो भेदाभावेन अपुनरावृत्त्यात्मकत्वमुक्तं वार्तिके~। 

ननु असंसारिरूपात्मज्ञानस्य मोक्षहेतुत्वे तेनैव मोक्षसिद्धौ कर्मापेक्षा न स्यात् इति चेत् – तत्रोक्तं बृहद्दीपिकायाम् –
\begin{verse}
ननु निःश्रेयसं ज्ञानात् बन्धहेतोर्न कर्मणः~। \\
नैकस्मादपि तत् किन्तु ज्ञानकर्मसमुच्चयात्~॥ इति~। 
\end{verse}
ज्ञाने सत्यपि नित्यनैमित्तिकानुष्ठानं विना पूर्वकृतदुरितक्षयाभावात् अकरणनिमित्तानागतप्रत्यवायपरिहारायोगाच्च तत्फलोपभोगार्थं बन्धप्रसक्तेः मोक्षप्राप्तिः न स्यात्~। कर्मनाशार्थं ज्ञानविधिः नोपलभ्यते~। 

अस्ति खलु “ज्ञानाग्निः सर्वकर्माणि भस्मसात् कुरुतेऽर्जुन” इत्यादिस्मृतयः इति चेत् – तासां स्मृतीनाम् औपचारिकत्वेन व्याख्यानं कृतम्~। 
\begin{verse}
यद्यप्यज्ञानजन्यत्वं कर्मणामवगम्यते~। \\
रागादिवत्तथाप्येषां न ज्ञानेन निराक्रिया~॥\\
कर्मक्षयो हि विज्ञानादित्येतच्चाप्रमाणवत्~। \\
फलस्याल्पस्य वा दानं राजपुत्रापराधवत्~॥
\end{verse}
इति सम्बन्धवार्तिके ज्ञानस्य कर्मनाशकत्वं निराकृतम्~। ज्ञानस्य मोक्षसाधनतया कर्मणो मोक्षसम्बन्धनिवारणानुपपत्तेः~। 

असंसार्यात्मज्ञानात्पूर्वं कर्तृत्वभोक्तृत्वाभिमानानिवृत्तौ कर्मप्रवृत्तिरस्त्येव~। परिपक्वज्ञानेनैव आत्मनः कर्मसम्बन्धः वार्यते~। अन्यथा विधानावस्थायां कर्तृत्वाभिमाननिवृत्तौ ज्ञाने विनियोज्यत्वं न स्यात्~। यदा मोक्षसाधनतया ज्ञाने प्रवृत्त्या ज्ञानमुत्पद्यते तदा कर्तृत्वाभिमाननिवृत्त्यभावेन मोक्षप्रतिबन्धकीभूतबन्धकारणकर्मक्षयाय नित्यादिकर्मकर्तव्यमिति न दोषः~। नन्वेवमपि –
\begin{verse}
मोक्षार्थी न प्रवर्तेत तत्र काम्यनिषिद्धयोः~। \\
नित्यनैमित्तिके कुर्यात् प्रत्यवायजिहासया~॥
\end{verse}
इति सम्बन्धवार्तिके काम्यनिषिद्धवर्जनात् तत्फलभोगार्थशरीरानुत्पत्तेः नित्यनैमित्तिकानुष्ठानाच्च तदकरणनिमित्तप्रत्यवायफलोपभोगार्थशरीरोत्पत्त्यभावेन ज्ञानवाक्यानां च औपचारिकतया व्याख्यानमुपपद्यते~। 

ननु कर्मणा ज्ञानं बाध्यतां तुल्यबलत्वाद्विकल्पो वा अस्तु~। विरोधपरिहाराय वा अङ्गाङ्गिभावः कल्प्यताम्, न तु निरपेक्षविहितयोः समुच्चयो युक्तः इति चेत् न~। अवान्तरकार्यैक्ये सति बाधो विकल्पो वा स्यात्~। अत्र कर्मणां पूर्वकृतपापनाशावश्यविहिताकरणनिमित्तपापपरिहारफलत्वात्, निर्गुणात्मज्ञानस्य च तत्त्वप्रकाशफलकत्वात् न अवान्तरकार्यैक्यमस्ति~। 

ननु बन्धहेतुकर्मक्षयादेव मोक्षसिद्धेः किं तत्त्वज्ञानेन इति चेन्न~। पूर्वकृतकर्मक्षये सत्यपि कर्तृत्वभोक्तृत्वाभिमाननिवृत्त्यभावात्~। निर्व्यापारत्वं नोपपन्नम्~। किञ्चित् बन्धहेतुकर्म भवेदेव~। तदर्थम् अपेक्षितकर्तृत्वभोक्तृत्वाभिमाननिवृत्तिः तत्त्वज्ञानं विना न सम्भवति~। अनेनैवाभिप्रायेणोक्तं बृहद्दीपिकायाम् –
\begin{verse}
नित्यनैमित्तिकैरेव कुर्वाणो दुरितक्षयम्~। \\
ज्ञानं च विमलीकुर्वन् अभ्यासेन च पाचयेत्~। \\
वैराग्यात् पक्वविज्ञानात् कैवल्यं भजते नरः~॥ इति~। 
\end{verse}
अतः अवान्तरकार्यं न तावत् बन्धविकल्पौ~। जीवनादिनिमित्तवतः कर्मस्वधिकारः~। मुमुक्षोश्च ज्ञानेऽधिकारः~। अधिकारिभेदात् भिन्नमार्गत्वात् न अङ्गाङ्गिभावः सम्भवति इति परिशेषात् समुच्चयसिद्धिरिति~। 

ज्ञानस्य “आत्मा वाऽरे द्रष्टव्य” इत्यादिविधिविधेयत्वाङ्गीकारेण मोक्षार्थज्ञानपरिपाकात्पूर्वम् उत्पन्नात्मतत्त्वज्ञानस्य असम्भावनाविपरीतभावनानिवृत्त्यर्थं श्रवणाद्यभ्यासकाले कर्मानुष्ठानमावश्यकम्~। तेन विना दुरित(कर्म)नाशासम्भवात्~। ज्ञानस्य कर्मनाशकत्वे प्रामाणाभावात् इत्येवं वार्तिकसम्मतो ज्ञानकर्मसमुच्चयपक्षः~। 

तदुपरि चिन्त्यते – परिपक्वज्ञानस्यैव मोक्षसाधनत्वात् तस्मिन् कर्मसाहित्यासम्भवात् कथं समुच्चयः इति चेत् – अपरिपक्वज्ञानस्यैव परिपक्वतावस्थायां मोक्षसाधनत्वमङ्गीकृतम् इति न कर्मसाहित्यं विरुध्यते~। परिपक्वं ज्ञानं न ज्ञानान्तरम्~। तस्यैव दार्ढ्याय श्रवणाद्यभ्यासविधानात्~। 

यद्यपि परिपक्वज्ञानोत्तरमपि कर्मानुष्ठानं दृश्यते~। यथा नित्यसिद्धात्मज्ञानवतोऽपि स्वस्य कर्म~। “न मे पार्थास्ति कर्तव्यम्” इत्यादिना भगवता प्रतिपादितम्~। तथा “विद्वान्युक्तः समाचरन्” इत्यादिना~। एवञ्च परिपक्वज्ञानेनापि समुच्चयो वक्तुं शक्यते~। 

तथापि मोक्षप्रतिबन्धकीभूतदुरितक्षयार्थं कर्मानुष्ठातुः कर्ताऽहम् अस्य फलभोक्ताहमित्येवं देहादिव्यतिरिक्तसंसार्यात्मज्ञानरूपाङ्गवैकल्ये सति अधिकारिविशेषणरूपविद्वत्तायाः अभावेन तादृशकर्मणः प्रत्यवायक्षयफलानुत्पादात् वैयर्थ्यमेव~। अत एव भगवदादीनां लोकसङ्ग्रहार्थमेव तदाचरणं न तु दुरितक्षयाय इति न दोषः~। 

एवं तर्हि अपरिपक्वज्ञानसमकाले नित्यकर्माद्यनुष्ठानवत् बन्धकस्य काम्यस्यापि अनुष्ठानप्रसङ्गः इति चेत् – ऐहिकामुष्मिकफलभोगविरक्तस्यैव ज्ञाने अधिकारात् तत्काले तत्तत्फलकामनाभावे तज्जनककर्मानुष्ठानं नोपपद्यते~। अत एव तेषामपि –
\begin{verse}
यज्ञो दानं तपश्चैव पावनानि मनीषिणाम्~। \\
एतान्यपि तु कर्माणि सङ्गं त्यक्त्वा फलानि च~। \\
कर्तव्यानीति मे पार्थ निश्चितं मतमुत्तमम्~॥
\end{verse}
इत्यादिवचनैः कामनां विहाय अनुष्ठाने न कश्चन दोषः~। एवञ्च “नित्यनैमित्तिकैरेव कुर्वाणो दुरितक्षयम्” इति वार्तिके नित्यनैमित्तिकपदं तादृशकर्मणामपि उपलक्षणमिति एवकारः तत्र कामनापरित्यागपरः~। 

ननु विविदिषावाक्येन कर्मणां ज्ञानार्थत्वेन विनियोगे सति, ज्ञानव्यापारेणैव कर्मणां मोक्षजनकत्वं कल्प्यतामिति चेत् – ज्ञानोत्तरमनुष्ठेयानां कर्मणां पूर्वोत्पन्नज्ञानजनकत्वं न सम्भवति~। उत्पत्त्यपूर्वस्थानीया चित्तशुद्धिः, परमापूर्वस्थानीयं तत्त्वज्ञानमिति कर्मणां मोक्षं प्रत्यपि साधनत्वं विहितम्~। 

तर्हि कर्मणां चित्तशुद्धावेव उपयोगात् चित्तशुद्ध्या उत्पन्नज्ञानादेव अस्तु मोक्षः इति चेत्, चेतसः रजस्तमोगुणसंसर्गराहित्येन सत्वाविर्भावे सति आत्मज्ञानोत्पत्तावपि आत्मधर्माधर्मयोरुच्छेदासम्भवेन तदर्थं कर्मानुष्ठानमावश्यकम्~। ज्ञानस्य च न कर्मनाशकत्वमिति वार्तिके एव उक्तम्~। 

ज्ञानसत्वे कर्मणामननुष्ठानसम्भवेऽपि कृतानां कर्मणां शक्त्यात्मनावस्थितानां न ज्ञानेन नाशः सम्भवति~। ज्ञानमपि न वस्तुस्वभावान्यथाकरणक्षमम्~। अयं च वस्तूनां नाशः यत् शक्त्यात्मना अवस्थानम्~। 

येषां मते ज्ञाने सति कर्मनाशः तेषां मतेऽपि प्रारब्धव्यतिरिक्त- कर्मनाशकत्वमङ्गीक्रियते~। तद्दृष्टान्तेन इतरकर्मनाशाभावः अनुमीयते~। यथा लोके रज्जुज्ञानेन सर्पनाशेऽपि तत्कृतभयकम्पादिकृतः शिरःस्फोटादिः दृश्यते~। अतः फलोपभोगवारणाय अवस्थापेक्षितधर्माधर्मोच्छेदार्थं कर्मानुष्ठानमाश्यकमेव~। 

सन्यासविधिस्तु – “काम्यानां कर्मणां न्यासं सन्यासं कवयो विदुः” इत्यभिप्रायेण~। अथवा वैराग्यात् द्रव्यार्जनाशक्तस्य द्रव्याद्यभावे कर्मानुष्ठानाशक्ताधिकारिकतया व्याख्येयः~। 	अत एव –
\begin{verse}
नियतस्य तु सन्यासः कर्मणो नोपपद्यते~। \\
मोहात्तस्य परित्यागः तामसः परिकीर्तितः~॥
\end{verse}
इत्यादिना शक्तस्य नित्यादिकर्मणां सन्यासे तामसत्वेन निन्दा दर्शिता~। 		
\begin{verse}
एषा तेऽभिहिता साङ्ख्ये बुद्धिर्योगे त्विमां श्रुणु~। \\
बुद्ध्या युक्तो यथा पार्थ कर्मबन्धं प्रहास्यति~॥
\end{verse}
इत्यनेन केवलसाङ्ख्यबुद्ध्या कर्मनाशाभावमभिप्रेत्य योगबुद्धेरेव कर्मबन्धनाशकत्वमुक्तम्~। एतद्विरुद्धवचनानि च औपचारिकतया व्याख्येयानि~। अत एव –
\begin{verse}
कर्मणैव हि संसिद्धिमास्थिता जनकादयः~। \\
सर्वकर्माण्यपि सदा कुर्वाणो मद्व्यपाश्रयः~। \\
मत्प्रसादादवाप्नोति शाश्वतं पदमव्ययम्~॥
\end{verse}
इत्यादिना कर्मणा मोक्षाख्यशाश्वतपदप्रतिपादनस्य औपचारिकतया व्याख्यातम्~। 

एवञ्च “क्षीयन्ते चास्य कर्माणि” इत्यादिश्रुतीनां “ज्ञानाग्निः सर्वकर्माणि” इत्यादिस्मृतीनां च स्तावकत्वम्~। अथवा कर्मसहितज्ञानपरतया वा औपचारिकत्वमेव~। अतः शक्तस्य ज्ञानसमकाले कर्मानुष्ठानमावश्यकमेव~। अशक्तानां तु सन्यासेनैव दुरितक्षयोत्पत्तिसम्भवात् न कर्मानुष्ठानस्यावश्यकता~। अपि तु तेषां सन्यास एव युक्तः~। 
\begin{verse}
ये च सन्यासजा दोषाः ये च स्युः कर्मसम्भवाः~। \\
सन्यासस्तान् दहेत् सर्वान् तुषाग्निरिव काञ्चनम्~॥
\end{verse}
इत्यादिना प्रतिपादितम्~। 

अतः तेषां सन्याससमुच्चितज्ञानात् अथवा स्वाश्रमोचिततत्तत्कर्मसमुच्चितज्ञानादेव वा मोक्षः इति सिद्धम्~। 

एवञ्च विविदिषावाक्येन ज्ञानप्रतिबन्धकीभूतरजस्तमःकलुषितान्तःकरणशुद्धिद्वारा कर्मणः ज्ञानार्थत्वं तथैव संयोगपृथक्त्वन्यायेन मोक्षप्रतिबन्धकीभूतधर्माधर्मोच्छेदद्वारा मोक्षार्थत्वमप्युपपद्यत इति सिद्धो ज्ञानकर्मसमुच्चयः~। 

नित्यनैमित्तिककर्मणां दुरितक्षयार्थत्वश्रवणेन तदर्थम् अनुष्ठानसम्भवेऽपि काम्यानां तत्तत्कामनापरित्यागेन मोक्षार्थमनुष्ठाने कथं मोक्षप्राप्तिः इति न शङ्क्यम्~। तदुक्तम् –
\begin{verse}
वेदोदितानि कर्माणि कुर्यादीश्वरतुष्टये~। \\
यथाश्रमं त्यक्तकामः प्राप्नोति परमं पदम्~॥\\
काम्यं विषयभोगार्थमिहामुत्र प्रयुज्यते~। \\
मोक्षाय सुकृतं विद्धि ब्रह्मार्पणधिया कृतम्~॥
\end{verse}
इत्यादिवचनैः ईश्वरप्रसादार्थत्वस्य पुण्यपापक्षयार्थत्वस्य च अवगमनद्वारा मोक्षार्थमनुष्ठानमुपपद्यते~। एतेन नित्यादिभिः अधर्मनाशवत् तस्यामवस्थायां सुकृतनाशोऽपि सम्भवतीति सूचितम्~। सत्वशुद्धि, आत्मधर्मपुण्यपापक्षयद्वारभेदेन एकार्थत्वं नास्ति~। “ज्ञानादेव तु कैवल्यम्” इति श्रुतिविरोधोऽपि नास्ति~। ज्ञानस्यापि मोक्षसाधनत्वाङ्गीकारेणाविरोधात्~। “यदि रथन्तरसामा सोमः स्यात् ऐन्द्रवायवाग्रान् ग्रहान् गृह्णीयात्” इत्यादौ स्वप्रतिस्पर्धिनो बृहत्साम्न एव रथन्तरपदेन व्यावृत्तिवत् ज्ञानविरोधि अज्ञानस्यैव व्यावृत्त्या एवकारः उपपद्यते~। 

एवं तमेव विदित्वेति श्रुतिविरोधोऽपि नास्ति~। तत्र ‘तमेव’ इत्येवकारेण आत्मभिन्नपदार्थज्ञानस्यैव मोक्षसाधनतानिषेधकरणेन कर्मणां मोक्षसाधनत्वे विरोधो नास्ति~। नापि ‘न कर्मणे’ति श्रुतिविरोधः, तस्याः सामान्यविषयत्वेन भगवदर्पणशून्यकर्मविषयतयोपपद्यते~। 

कर्मणां पुण्यपापक्षयफलकत्वमेव अस्तु~। न मोक्षफलकत्वमिति चेत् – ज्ञानस्यापि दृष्टद्वारा अज्ञाननिवृत्तिफलकत्वेनैवोपपत्तौ मोक्षफलकत्वं न स्यात्~। व्यापारेण व्यापारिणोऽन्यथासिद्ध्यभावस्य उभयत्र तुल्यत्वाच्च~। 

तर्हि निषिद्धकर्मणामपि भगवदर्पणबुद्ध्या अनुष्ठानापत्तिः इति चेत् –
\begin{verse}
देवतानां गुरूणाञ्च मातापित्रोस्तथैव च~। \\
पुण्यं देयं प्रयत्नेन नापुण्यं चोदितं क्वचित्~॥
\end{verse}
इति माधवोदाहृताङ्गिरो वचसा निषेधेन पापकर्मणोऽर्पणानुपपत्तेः~। कर्मजन्यत्वे मोक्षस्य अनित्यत्वप्रसङ्गः इति तन्न~। नाशकसामान्याभावेन ध्वंसवत् नित्यत्वमुपपद्यते~। मोक्षविरोधिपुनर्बन्धोत्पत्त्या मोक्षो नश्येत्~। तदा पुनर्बन्धोत्पादे न किञ्चित्कारणमस्ति~। अतः तदा काम्यानां फलकामनापरित्यागे तथा नित्यानामपि भगवत्प्रीत्यर्थमनुष्ठितानां भगवत्प्रसादद्वारा पुण्यपापक्षयद्वारा वा मोक्षार्थत्वे न कोऽपि विरोधः~। 

मीमांसकानां मोक्षसाधनविषये मोक्षस्वरूपविषये वा वेदान्तिनामपेक्षया नातीव भेदः गम्यते~। भट्टपादैः आत्मवादे –
\begin{verse}
इत्याह नास्तिक्यनिराकरिष्णुः आत्मास्तितां भाष्यकृदत्र युक्त्या~। \\
दृढत्वमेतद्विषयप्रबोधः प्रयाति वेदान्तनिषेवणेन~॥ (श्लो.वा.)
\end{verse}
इत्यादिना वेदान्तसम्मतमोक्षमार्ग एव सूचितः~। 

एवं च अनौपाधिकस्यात्मनः परमात्मप्राप्त्यवस्थारूपः मोक्षः~। स च नित्यः निरतिशयः सत्-चित्-आनन्दस्वरूपः ज्ञानकर्मसमुच्चयसाध्य इति मीमांसकसम्मतः पक्षः~। 

\centerline{॥ इति शम्~॥}

\articleend
}
