\chapter{अनुपलब्धेः प्रमाणान्तरत्वस्य निराकरणम्}

\begin{center}
\Authorline{वि.मधुसूदन अडिगः}
\smallskip

उपप्राध्यापकः\\
एस्.के.एस्.वि.ए.विद्यालयः\\
आदिचुञ्चनगिरिः, मण्ड्या
\addrule
\end{center}

\begin{verse}
कणभक्षाक्षपादौ तौ तत्त्वालोकविचक्षणौ ।\\
यथादृष्टोपदेष्टारौ परमाप्तौ मुनी स्तुमः ॥
\end{verse}
लोके  'घटोऽस्ति, पटोऽस्ति' इत्यादिव्यवहारवत् 'घटो नास्ति, पटो नास्ति' इत्यादयोऽपि व्यवहारा दृश्यन्ते । अत्रास्तीति व्यवहारकारणं "भावप्रतीतिः", तत्र विषयः भावः ; नास्तीति व्यवहारकारणम् "अभावप्रतीतिः" तत्र विषयः अभावः । इयमभावप्रतीतिः केन प्रमाणेन गृह्यते? इत्यत्र विवदन्ते शास्त्रकाराः । नैयायिका वैशेषिकाश्च "अभावप्रतीतिः क्वचित्प्रत्यक्षेण, क्वचिदनुमानेन च गृह्यते" इत्यङ्गीकुर्वन्ति । भाट्टाः तदनुयायिनश्च "अभावोऽनुपलब्धिप्रमाणेन गृह्यते" इति प्रतिपादयन्ति । प्राभाकारास्तु "अभावाख्यप्रमेयस्याधिकरणस्वरूपत्वादतिरिक्ततया न गृह्यते इति नास्ति तस्य प्रमाणचिन्ता" इति परिभावयन्ति । तस्मादभावग्राहकं प्रमाणं विचारपदवीमधिरोहति ।
 
\section*{नैयायिकानां वैशेषिकानाञ्चायमाशयः} 

समीपदेशे प्रतीयमानोऽभावः प्रत्यक्षेण, परोक्षे विद्यमानोऽभावः अनुमानादिना च गृह्यते इति । कश्चिज्जलार्थी जलाहरणाय घटमन्विष्यन् यदि भूतलमात्रं पश्यति, तदा तत्र घटाभावं प्रतीत्य ततो निवर्तते, "घटो नास्ति" इति व्यवहरति वा । "विस्फारिते चक्षुषि भूतलज्ञानं तदभावे तदभाव" इत्यन्वयव्यतिरेकाभ्यां भूतलप्रतिपत्तौ यथा चक्षुरिन्द्रियं कारणमिति ज्ञायते तथैव "विस्फारिते चक्षुषि घटाभावज्ञानं तदभावे तदभाव" इत्यन्वयव्यतिरेकाभ्यां तद्वृत्तिघटाभावस्य प्रतिपत्तावपि चक्षुरिन्द्रियमेव कारणमिति ज्ञायते । तत्र भूतलग्रहणे यावन्त्यः सामग्र्यः अपेक्षिताः ताभिः सह प्रतियोगिस्मरणं, तज्ज्ञानानुपलब्धिश्चाभावप्रत्यये अपेक्षितः । तत्र सन्निकर्षो विशेषणविशेष्यभावाख्यः स्वीक्रियत इति । परोक्षे विद्यमानस्त्वभावः स्मृत्यभावादिलिङ्गेन गृह्यते । तदत्रोपरिष्टान्निरूपयिष्यामः । अपि च अनुपलब्धि: प्रमाणान्तरमिति अन्यैर्यदुच्यते तत्तु अनुमानमेवेति प्रशस्तपादाचार्या आमनन्ति । तदुक्तं भाष्ये "\textbf{अभावोप्यनुमानमेव । यथा उत्पन्नं कार्यं कारणसद्भावे लिङ्गं तथानुत्पन्नं कार्यं कारणासद्भावे लिङ्गम्}" इति । अनुपलब्धिर्नाम ज्ञानानुपलम्भः । यस्तावदनुपलब्धिं प्रमाणमिच्छति सोऽपि केवलज्ञानानुपलम्भं प्रमाणमिति नाभ्युपगच्छति; स्वरूपतः विप्रकृष्टस्याप्यभावत्वप्रसङ्गात् । परमाणुः स्वरूपतो विप्रकृष्टः अतीन्द्रियत्वात्, अतः परमाणोरनुपलम्भात्तदभावप्रतीतिप्रसङ्गः । एवमेव भूतलान्तर्वर्तिजलमपि स्वरूपतो विप्रकृष्टम्, तस्याप्यभावप्रतीत्या न कस्यापि कूपखनने प्रवृत्तिः स्यात् । अतः "\textbf{ज्ञानकारणेषु सत्सु ज्ञानयोग्यस्य यो ज्ञानानुपलम्भः सः अभावप्रतीतौ कारण}"मिति वक्तव्यम् । अयञ्चानुपलम्भः योग्यानुपलम्भः ।

वस्तुतः अयोग्यानुपलम्भस्य योग्यानुपलम्भस्य च नास्ति स्वरूपतो भेदः, अभावस्य निरतिशयत्वात् । लोके घटाभावपटाभावाद्यसङ्ख्यसङ्ख्यानामभावानां व्यवहारसत्त्वेऽपि अभावेषु स्वरूपतो भेदो नास्ति, सर्वेषामपि भावभिन्नरूपत्वात्; किन्तु प्रतियोगिभेदाद्वा प्रतियोगितावच्छेदकभेदाद्वा प्रतियोगितावच्छेदकसंसर्गभेदाद्वा अभावभेदः कल्प्यते, सम्बन्धविशेषा अपि प्रतियोगिना साकमन्वयं प्राप्यैव अभावेऽतिशयं जनयन्ति न तु साक्षादभावेनान्वयं प्राप्नुवन्ति । तस्मान्निरतिशयोऽनुपलम्भः इन्द्रियवत्स्वशक्त्यैव बोधजनको न भवति ; किन्तु अविनाभावसापेक्ष एव । योग्यानुपलम्भसत्वे अभावोपपत्तिः तदभावे तदभाव इत्यन्वयव्यतिरेकाभ्यां योग्यानुपलम्भः अभावप्रतिपत्तिं न व्यभिचरतीति ज्ञायते । अयोग्यानुपलम्भसत्वे अभावोपपत्ति इति नास्ति, सत्यपि ज्ञेये तस्य सम्भवः(अयोग्यानुपलम्भसम्भवः) परमाण्वादौ प्रसिद्ध इत्ययोग्यानुपलम्भः अभावप्रतिपत्तिं व्यभिचरति । तस्माद्योग्यानुपलम्भ एव अभावपरिच्छेदक इति वक्तव्यः । एतादृशः योग्यानुपलम्भः अनुमानमेव, अविनाभावसापेक्षत्वात् । यद्यविनाभावो नापेक्षितस्तर्हि तस्यायोग्यानुपलम्भवदबोधकत्वप्रसङ्गः । केवलं भूतलं पश्यता घटार्थिना "भूतलं घटाभाववत्, विशेष्यता-सम्बन्धेनानुपलम्भात्, विशेष्यतासम्बन्धेन यत् यदनुपलम्भवत् तत्तदभाववत् यथा पटानुपलम्भवद्भूतलं पटाभाववत्, तथा चायम्, (यद्यत्रास्ति तत्तत्रोपलभ्यते यथा घटः, न चेदं तथा ) तस्मात्तथा" इत्यनुमानेन घटाभावोऽनुमीयते इति ।

\section*{अनुपलब्धिः प्रमाणान्तरमिति वदतामाशयः} 

भाट्टाः तदनुयायिश्च अभावप्रतीतिः अनुपलब्ध्या गृह्यते इत्यङ्गीकुर्वन्ति । यदा यत्र चक्षुरुन्मीलनादि सहकारिकारणसत्त्वेऽपि प्रत्यक्षादिप्रमाणपञ्चकाप्रवृत्या यज्ज्ञानानुपलम्भः सः प्रमाणभूतः अभावः, \textbf{अनुपलब्ध्यपरनामा} । तच्च सत्तामात्रेणाभावं बोधयतीति न तत्रानवस्थाशङ्का । कश्चित् कदाचिज्जलाहरणार्थं घटमन्विष्यन् केवलं भूतलं पश्यति, तदा स्मृत्यारूढे प्रतियोगिनि प्रत्यक्षादिपञ्चभिरपि प्रमाणैर्घटमलभमानः, प्रमाणाप्रवृत्तिप्रयुक्तघटज्ञानानुपलम्भात् घटाभावमधिगच्छति । अत्र प्रमाणाप्रवृत्तिप्रयुक्तो ज्ञानानुपलम्भः प्रमाणम्, 'घटो नास्ति' इति विषयाभावबुद्धिः प्रमा । ननु "घटः यदि स्यात्तर्हि भूतलमिवोपलभ्येत, नोपलभ्यते," इति तर्कानुग्रहेण चक्षुरादिनैव भूतलादौ घटाभावादिप्रतीतिः गृह्यते इति \textbf{चेन्न} । विषयेन्द्रियसन्निकर्षाभावादभावो न प्रत्यक्षगम्यः । भूतले घटाभावग्रहणकाले इन्द्रियसन्निकर्षः भूतलेन साकं भवति योग्यत्वात्, न घटाभावेन साकमयोग्यत्वात् । ननु कार्यगम्या हि योग्यता; उपरतचक्षुरिन्द्रियस्य भूतलग्रहणमपि नास्ति, घटाभावग्रहणमपि नास्ति । अतः यथा इन्द्रियव्यापारसत्त्वे भूतलग्रहणं तदभावे तदभाव इत्यन्वयव्यतिरेकाभ्यां भूतलमिन्द्रियग्रहणयोग्यमिति ज्ञायते तथा इन्द्रियव्यापारसत्त्वे घटाभावग्रहणं तदभावे तदभाव इत्यन्वयव्यतिरेकाभ्यां घटाभावोऽपि इन्द्रियग्रहणयोग्य इति ज्ञायते । सन्निकर्षस्तूपरिष्टात्कल्पयिष्यते इति चेन्न । अभावेन साकमिन्द्रियस्य सन्निकर्षासम्भवात् । इन्द्रियार्थ-सन्निकर्षास्तु संयोगसमवायादयो भवन्ति । अभावेन साकं न संयोगः सन्निकर्षः अद्रव्यत्वात् । न वा समवायः सन्निकर्षः गुणादिवैलक्षण्यात् । अत एव संयुक्तसमवायादयोऽपि सन्निकर्षा न भवन्ति । न च "घटाभाववद्भूतलमि"त्यत्र संयुक्तविशेषणतासन्निकर्षेण घटाभावग्रहणं ; चक्षुस्संयुक्ते भूतले घटाभावस्य विशेषणत्वादिति वाच्यम् । संयुक्तं वा समवेतं वा विशेषणं भवति यथा \textbf{दण्डी}त्यत्र संयुक्तो दण्डः, \textbf{शुक्लः पटः} इत्यत्र समवेतं शुक्लं रूपम् ; असंयुक्तमसमवेतं विशेषणं न भवतीति घटाद्यभावः न भूतलदिविशेषणम् । तस्मान्न संयुक्तविशेषणतासन्नकर्षेणाभावग्रहणं युक्ततरम् । यदि च घटाभावः भूतलविशेषणमेवेत्यङ्गीकृत्य संयुक्तविशेषणतासन्निकर्षोपकृतं चक्षुः तद्गृह्णाती-त्यङ्गीक्रियते तर्हि चक्षुस्संयुक्तगुडखण्डादिषु विद्यमानान् रसादीनपि चक्षुरेव गृह्णीयात् तत्र तेषामपि विशेषणत्वात् । यद्ययोग्यत्वान्नगृह्णाति तर्हि अयोग्यत्वादभावमपि न गृह्णीयात् । अतः सन्निकर्षाभावादेव पूर्वमुपवर्णितः सहचारः इन्द्रियाभावयोर्नैव घटते । उपवर्णितो भूतलेन्द्रिययोः सहचारः, यथा विदूरदेशस्थेन ज्वालावलीघटितवह्निना उष्णस्पर्शानुमितौ इन्द्रियार्थसन्निकर्षः अन्यथासिद्धः, तथा अन्यथासिद्धः । अतः प्रकृते घटाभावः न प्रत्यक्षेण गृह्यते । तदिदं प्रतिपादितं श्लोकवार्तिके –
\begin{verse}
\textbf{गृहीत्वा वस्तुसद्भावं स्मृत्वा च प्रतियोगिनम् ।}\\
\textbf{मानसं नास्तिताज्ञानं जायतेऽक्षानपेक्षया ॥  इति ।}
\end{verse}  
न वा अभावः अनुमानगम्यः । घटाभाववद्भूतलमित्यत्र भूतलं वा घटादर्शनं वा लिङ्गं न भवतः । यदि \textbf{भूतलं लिङ्गं} स्यात्तर्हि घटाभावाभाववति नाम घटवति प्रदेशेऽपि भूतलज्ञानसम्भवात् व्यभिचारः, भूतलं भूतलधर्मो न भवतीत्यपक्षधर्मत्वात् स्वरूपासिद्धो हेतुः, भूतलस्याभावस्य चानन्त्येन सम्बन्धाग्रहणाद् व्याप्त्यभावश्च । अतः भूतलं न लिङ्गम् ।  यदि \textbf{घटादर्शनं लिङ्गं} स्यात् तर्ह्यपक्षधर्मत्वात् स्वरूपासिद्धो हेतुः, व्याप्तेः व्याप्यव्यापकयोर्ग्रहणपूर्वकत्वात् घटादर्शनस्य घटाभावस्य चाग्रहणे व्याप्तिज्ञानस्याप्यसिद्धिः । अदर्शनं च दर्शनाभावः ; तस्य परिच्छेदेऽपि इयमेव गतिः । सन्दिग्धं लिङ्गं कथं वा साध्यं साधयति? अतः घटादर्शनमपि न लिङ्गम् । उपमानं, शब्दः, अर्थापत्तिश्चात्र न प्रमाणं, प्रत्यक्षनिराकरणेनैव तेषामप्यभावग्राहकत्वनिराकरणसम्भवात् । अतः यथा भावात्मकप्रमेयपरिग्रहे भावात्मक-प्रमाणमेवाङ्गीक्रियते तथा विचारचर्चायां बहुदूरं गत्वाप्यभावात्मकप्रमेयपरिग्रहे \textbf{अभावात्मकप्रमाण}मेवाङ्गीकरणीयः । स च प्रमाणभूतः अभावः अनुपलब्धिरेव । तदुक्तम्-
\begin{verse}
\textbf{भावात्मके प्रमेये हि नाभावस्य प्रमाणता ।\\
अभावेपि प्रमेये स्यात् न भावस्य प्रमाणता ॥\\
न प्रमेयमभावाख्यं निह्नुतं बौद्धवत्त्वया ।\\
प्रमाणमपितेनेदमभावात्मकमिष्यताम् ॥} इति । तन्न -
\end{verse}

\section*{अनुपलब्धेः प्रमाणान्तरत्वनिरासः} 

भावात्मके प्रमेये भावात्मकं प्रमाणमभावात्मके प्रमेयेऽभावात्मकं प्रमाणमिति तु न नियमः । क्लृप्तेन प्रत्यक्षेण वा अनुमानेन वा अभावप्रतीतिसम्भवे अभावाख्यप्रमाणान्तरपरिकल्पनं न युक्तमिति प्रतिभाति । अत एव इदमुक्तम् –
\begin{verse}
\textbf{अदूरमेदिनीदेशवर्तिनस्तस्य चक्षुषा ।\\
परिच्छेदः परोक्षस्य क्वचिन्मानान्तरैरपि ॥ इति ।}
\end{verse}
यथा "\textbf{घटवद्भूतलम्}" इति घटविशेषणक-भूतलविशेष्यक-विशिष्टबुद्धिः इन्द्रियान्वयव्यतिरेकानुविधानात् चाक्षुषि एकैव, तथा "\textbf{घटाभाववद्भूतलम्}" इति घटाभावविशेषणक-भूतलविशेष्यक-विशिष्टबुद्धिरपि इन्द्रियान्वयव्यतिरेकानुविधानात् चाक्षुषि एकैव स्यात्; कथं वा तत्र भूतलज्ञानं प्रत्यक्षं, घटाभावज्ञानं च प्रत्यक्षविलक्षणप्रमाणजन्यमिति बुद्ध्यामहे? न च यथा 'पर्वतो वह्निमान्' इत्यत्र पर्वतांशे प्रत्यक्षत्वं, वह्न्यंशे परोक्षत्वं तथा "\textbf{घटाभाववद्भूतलम्}" इत्यत्र भूतलांशे प्रत्यक्षत्वं, घटाभावांशे परोक्षत्वं युज्यते इति वाच्यम् । 'पर्वतो वह्निमान्' इति विशिष्टज्ञानस्य परोक्षत्वात् । अपि च पर्वते धूमदर्शनानन्तरं व्याप्तिस्मरणादिना व्यवहितत्वात् युक्तमेव 'पर्वतो वह्निमान्' इति ज्ञानस्य परोक्षत्वम् । प्रकृते न तथा ; "\textbf{घटाभाववद्भूतलम्}" इत्यत्र इन्द्रियव्यापारसमनन्तरं भूतलज्ञानवत् घटाभावज्ञानमपि अव्यवधानेनैव भवति । अतः न तत्र परोक्षशङ्कापि । अपि च 'पर्वतो वह्निमान्' इत्यत्र व्यापृताक्षोsपि पर्वते वह्निं नावलोकयति, अतः तज्ज्ञानं परोक्षमेव । प्रकृते तु अनुपरतचक्षुरिन्द्रियव्यापारश्चेत् घटाभाव-मवलोकयति (घटाभावमनुभवति) नोचेन्न, तस्मात् घटाभावज्ञानं प्रत्यक्षमेव । दूरदेशस्थवह्नि-'रूप' दर्शनादुष्णस्पर्शानुमाने इन्द्रियार्थसन्नकर्षः यथा अन्यथासिद्धः तथा प्रकृते नास्ति । तत्र चक्षुरिन्द्रियस्य स्पर्शदर्शनाकौशलं, स्पर्शज्ञाने त्वगिन्द्रियस्य कारणत्वं, रूपस्पर्शयोर्ज्ञाने तत्तदिन्द्रियाणामविनाभाविता च सम्यगवधारितमिति वह्नि-'रूप'दर्शनात् स्पर्शस्त्वनुमीयते इति इन्द्रियार्थसन्नकर्षः अन्यथासिद्धः । "\textbf{घटाभाववद्भूतलम्}" इत्यत्र नैवं क्रमः । तस्माद्घटाभावः प्रत्यक्षविषय एव ।

ननु केवलमभाव इति पदार्थो नास्ति, तस्य प्रतियोग्यनुयोगिसापेक्षत्वात् । प्रायः सर्वदा 'इह इदं नास्ति' इत्येव प्रतीयते । तत्रेन्द्रियसन्निकर्षस्त्वधिकरणेनैव नत्वभावेन इत्यभावः न प्रत्यक्षविषयः। अपि च नीरूपस्येन्द्रियेणासम्बद्धस्य चाभावस्य प्रतीतिश्चक्षुषीत्यपि न भवतीति चेन्न । रूपवत्वात् चाक्षुषत्वं प्रतीतेः न सिद्ध्यति, अपि तु चक्षुरिन्द्रियजन्यत्वादेव । यदि रूपवत्वात्प्रतीतिश्चाक्षुषी स्यात् तर्ह्यतीन्द्रियाणां परमाणूनामपि प्रतीतिश्चाक्षुषी स्यात् रूपवत्वात् । एवं सम्बद्धं सर्वमेव चाक्षुषप्रतीतिं जनयतीति न नियमः । तथा चेत् चक्षुषा सम्बद्धस्याकाशस्यापि प्रतीतिश्चाक्षुषी स्यात् (विभुत्वादाकाशस्य चक्षुरिन्द्रियसम्बन्धः युक्तः) तथा तु नास्ति । 'चाक्षुषं चक्षुषा सम्बद्धमेव भवति' इत्ययं नियमः भावेष्वेव, संयोगादिसन्निकर्षाः भावानामेव, इन्द्रियाणां प्राप्यकारित्वनियमोऽपि भावाभिप्रायेणैव । \textbf{अवस्तुभूतस्याभावस्य न इमे नियमाः} । तस्माद्रूपाभावेऽपि सन्निकर्षाभावेऽपि इन्द्रियमभावप्रतीतिं जनयति; सा चाक्षुषी एव । न चासम्बद्धत्वाविशेषादेकाभावग्रहणे देशादिव्यवहितसर्वाभावग्रहणप्रसङ्‌ग इति वाच्यम् । अभावप्रतीतावाश्रयग्रहणस्यापेक्षत्वान्न देशादिव्यवहितसर्वाभावग्रहणप्रसङ्गः । अथ चासम्बद्धमिन्द्रियं न गृह्णाति प्राप्यकारित्वनियमात्तस्य इत्यङ्गीकारे संयुक्तविशेषणता-सन्निकर्षोपकृतं चक्षुरिन्द्रियमेवाभावं गृह्णातीति तत्प्रतीतिश्चाक्षुषी इत्यपि वक्तुं शक्यते । न चासंयुक्तमसमवेतं वा विशेषणं न भवतीत्यभावः न कस्यचिद्विशेषणम्, अपि च सम्बन्धान्तरमूलकत्वात् संयुक्तविशेषणतादिसन्निकर्षः न वास्तवः सम्बन्धः किन्तु कल्पित एव, तस्मादभावश्चक्षुषा गृह्णातीति वादः नैव शोभते इति वाच्यम् । 'संयुक्तं वा समवेतं वा विशेषणं भवति', 'सम्बन्धान्तरमूलकत्वात् संयुक्तविशेषणतादिसन्निकर्षः न वास्तवः सम्बन्धः किन्तु कल्पित एव' इत्यादयो नियमाः केवलं भावेष्वेव नत्वभावेषु ; अभावस्य भावविरुद्धस्वभावत्वात् । अथवा यत्सम्बद्धं तद्विशेषणमेवेति रीतिर्नास्ति, तद्यथा पादेन पीडिते शिरसा धारिते वा दण्डे '\textbf{दण्डी}'ति व्यवहारो लोके न दृश्यते , एवमेव यद्विशेषणं तत्सम्बद्धमेवेत्यपि रीतिर्नास्ति, समवायस्य विशेषणत्वेऽपि सम्बन्धान्तराभावात् । तस्मादसम्बद्धोप्यभावः विशेषणं भवितुमर्हतीति विज्ञायते । अतः भूतलादिविशेषणमभावः संयुक्तविशेषणतासन्निकर्षोपकृतचक्षुषैव गृह्यत इति । अपि च "संयुक्तविशेषणतासन्निकर्षेणाभावस्य ग्रहणे तेनैव सन्निकर्षेण रसादीनामपि ग्रहणं यच्चोदितं तदपि न युज्यते । संयुक्तसमवायादि सन्निकर्षाणामङ्गीकारेऽपि दोषस्यास्य समानत्वात् । संयुक्तसमवायसन्निकर्षेण चक्षुषा घटगतरूपग्रहणे घटसमवायस्यैकत्वात्तद्गतरसस्यापि ग्रहणप्रसङ्गः इति संयुक्तसमवायादयोऽपि सन्निकर्षा न स्युः । न चेष्टापत्तिः, असन्निकृष्टान्येवेन्द्रियाणि रूपादिकं गृह्णीयुरित्यापत्तिः स्यात् । नह्यसन्निकृष्टानीन्द्रियाणि रूपादिकं गृह्णन्ति, न वा संयुक्तसमवायादिभ्योऽन्यः रूपादिग्रहणे सम्बन्धो विद्यते । न चार्थग्रहणात्मकश्चक्षुषो व्यापार एव सन्निकर्षपदवाच्यः, तत्सत्त्वाद्घटगतरूपं गृह्यते तदभावाद्रसो न गृह्यते, न पुनः रूपादिग्रहणार्थमतिरिक्ततया सन्निकर्षाणामपेक्षा इति वाच्यम् । तदिन्द्रियव्यापारसत्त्वादभावोऽपि गृह्यताम् । अपि च वस्तुग्रहणाय प्राप्यकारीणीन्द्रियाणि स्वीकृत्यापि सन्निकर्षो निराक्रियते इति विरुद्धमेतत् । तस्माद्भावग्राहकाः सन्निकर्षाः पञ्च, अभावग्राहकः एक इत्याहत्य षड्विधसन्निकर्षा अङ्गीकरणीयाः । चक्षुषा रसादिग्रहणवारणाय षड्विधसन्निकर्षानुगामिनी योग्यता वक्तव्या । योग्यतायाः सन्निकर्षव्याप्यत्वात् \textbf{यत्र योग्यता तत्र सन्निकर्षः सम्भवतीति रूपं चक्षुषा गृह्यते, यत्र सन्निकर्षस्तत्र योग्यता इत्यभावात्} सत्यपि सन्निकर्षे चक्षुषा रसो न गृह्यते । अतः चक्षुषा अभावस्य ग्रहणात् तत्प्रतीतिः प्रत्यक्षमेव । 

ननु क्वचिदिन्द्रियव्यापाराभावेऽपि अभावः प्रतीयते । तथा हि- कश्चित् स्वरूपतो देवकुलं प्रतीत्य देशान्तरं गतः केनचित् ‘देवकुले देवदत्तः अस्ति वा नास्ति वा’ इति पृष्टः केवलं देवकुलं स्मरन् तदानीमेव जातजिज्ञासः देवदत्ताभावं प्रतीत्य देवदत्तो नास्तीति व्यवहरति । अत्र देशान्तरे स्थितस्य देवकुलगतदेवदत्ताभावेन साकमिन्द्रियार्थसन्निकर्षः न भवतीति न तत्र प्रत्यक्षेणाभावप्रतीतिः । पूर्वं देवदत्ताभावस्याग्रहणान्न तदानीं स्मृतिः । अपक्षधर्मत्वान्न देवकुलं लिङ्गं भवति , अग्रहणान्न वा देवदत्तादर्शनम् इति तत्र नानुमानं प्रमाणम् । अतः प्रतियोगिनः स्मरणे सति ज्ञानानुपलम्भात् देवदत्ताभावप्रतीतिः गृह्यते । एवमेकत्राभावः अभावपरिच्छेदकः इत्यङ्गीकारे सर्वत्रापि अयमेव प्रकारः शोभते । तस्मादनुपलब्धिः प्रमाणान्तरमिति \textbf{चेन्न} ।

तदत्र विकल्पः - स्वरूपतो देवकुलं प्रतीत्य देशान्तरं गतः केनचित् 'देवकुले देवदत्तः अस्ति वा नास्ति वा' इति पृष्टः, स्मृत्यारूढे प्रतियोगिनि किमिदानीन्तनानुपलम्भेनेदानीन्तनाभावं प्रत्येति? उत प्राक्तनानुपलम्भेन प्राक्तनाभावम्? इति \textbf{चेत् न प्रथमः पक्षः} - इदानीन्तनानुपलम्भस्य देशान्तरव्यवधानेन अयोग्यानुपलम्भत्वात् । तथा च इदानीं देवदत्तागमनस्यापि सम्भवात् तदभावः सन्दिग्धोऽपि ।

\subsection*{न द्वितीयः पक्षः} 

प्राक्तनानुपलम्भस्येदानिमनुवृत्यभावादविद्यमानोसौ न तत्प्रतीति कारणम् । नापि तदनुपलम्भस्य स्मृतिः, पूर्वमग्रहणात् ; न वा प्रमाणान्तरं तत्रास्ति, अभावरूपत्वात् ; अनुपलम्भान्तरस्वीकारे अनवस्था । अतः कथमपि अनुपलम्भस्याग्रहणात् तत्र अभावः न प्रतीयेत । न च वस्त्वनुपलम्भः वस्त्वुपलम्भेन निवर्तनीयः, अतः देशानतरं गतस्य देवदत्तोपलम्भाभावादनुवर्तते तदनुपलम्भः, तेन च भवत्येव देवदत्ताभावप्रतीतिः इति वाच्यम् । यदि स्वरूपतः अधिकरणमात्रं गृहीत्वा पश्चात् कालान्तरे वस्तुग्रहणात् "\textbf{इह इदं नासीत्}" इति प्राक्कालीनाभावज्ञानं प्राप्नुयात् तर्हि तत्र कथं प्रतीतिनिर्वाहः? वस्तुनो ग्रहणात् अनुपलम्भस्तु निवृत्तः, यच्चाभावप्रतीतिकारणम् । यदनुपलम्भः पूर्वमासीत् स इदानीमविद्यमानोऽपि सम्प्रत्यभावहेतु इत्युच्यते तर्हि प्रनष्टेन्द्रियेणापि विषयग्रहणप्रसङ्गः । न च अद्यतनोपलम्भेन अद्यतनानुपलम्भः नष्टः न प्राक्कालीनानुपलम्भ इति प्राक्कालीनानुपलम्भेन प्राक्कालीनाभावः गृह्यते "\textbf{इह इदं नासीत्}" इति वाच्यम् । अनुपलम्भो नाम उपलम्भप्रागभावः, स च वस्तुप्राप्तिपर्यन्तमेक एव, न तत्र प्राक्तनाद्यतनभेदः ; तर्हि अद्यतनो निवृत्तः, प्राक्तनो न निवृत्तः इति कथं सङ्गच्छते । तस्मादभावः अभावेन परिच्छिद्यते इति न युक्तियुक्तः प्रतिभाति । कथं तर्ह्यसतीन्द्रियव्यापारे अभावग्रहणम् इति चेत् \textbf{अनुमाना}दिति । स्वरूपतः देवकुलं प्रतीत्य स्थानान्तरं प्राप्तः 'देवदत्तोsस्ति वा नास्ति वा' इति पृष्टः अनुमानाद्देवदत्ताभावं गृह्णाति । तदित्थं - देवकुलं देवदत्ताभाववत् तत्स्मृत्यभावात्, यस्मिन् स्मर्यमाणे सत्यामपि सुस्मूर्षायां यन्न स्मरति तत्तस्य ग्रहणकाले नासीत् यथा केवलभूतले स्मर्यमाणे अस्मर्यमाणो घटः । देवकुले स्मर्यमाणे सत्यामपि सुस्मूर्षायां ग्रहणयोग्यो देवदत्तः न स्मर्यते । तस्मात् देवकुलग्रहणकाले स नासीदिति ज्ञायते ।

ननु एकज्ञानसंसर्गिणोः वस्तुनोः संस्कारपाटवादिविरहवशादेकस्यास्मरणे, संस्कारपाटवादपरस्य स्मरणे च स्मृत्यभावादभावानुमानं न भवति इति तद्व्यभिचरितम् । तद्यथा अधीतस्य श्लोकस्य पदान्तर स्मरणे पदान्तरस्यास्मरणं लोके दृष्टम् इति चेन्न । अनुपलब्धिं षष्ठं प्रमाणमिति स्वीकृत्यापि सहोपस्थितयोः घटभूतलयोः कदाचित् कारणान्तरवशात् भूतलमुपलभ्यते, घटश्च नोपलभ्यते तदा घटाभावप्रतीतिः कुतः? यदि एकज्ञानसंसर्गिणोः वस्तुनोः एकस्योपलम्भेऽपि यस्यानुपलम्भः स एव तदभावसाधनं भवति न सर्वानुपलम्भः । अपि च एकज्ञानसंसर्गिणोः घटभूतलयोः भूतलग्रहणे या सामग्री सैव घटग्रहणेऽपि, "यदि भूतले घटोऽभविष्यत् तर्हि भूतलमिव व्यज्ञास्यत तत्तुल्यसामग्रीकत्वात्, न गृह्यते तस्मान्नासति" इत्येवं तर्कानुग्रहेण तदनुपलम्भं प्राप्य घटाभावः निश्चीयते तर्हि अनुमानपक्षेऽपि समानोऽयं परिहारः "यश्चोभयोः समो दोषः परिहारोपि तादृशः" इति न्यायात् । एकज्ञानसंसर्गिणोः वस्तुनोः एकस्य स्मरणेऽपि यस्यास्मरणं तत्तदभावं साधयति । प्रकृतेऽपि या देवकुलग्रहणसामग्री सैव देवदत्तस्यापि ग्रहणसामग्री, एवमेव या देवकुलस्य स्मरणसाग्री सैव देवदत्तस्यापि स्मरणसामग्री तदेकज्ञानसंसर्गित्वात् । अतः "यदि देवकुलग्रहणकाले देवदत्तोsभविष्यत् तर्हि सोऽपि देवकुलस्मरणकाले देवकुलमिवास्मरिष्यत् तत्तुल्यसामग्रीकत्वात्, न च स्मर्यते, तस्मान्नासीत्" इति तर्कबलात् मानसप्रत्यक्षेणैव स्मृत्यभावः निश्चीयते । तेन च देवदत्ताभावः अनुमीयते । श्लोकादिषु पदानामुच्चारणं क्रमेण भवति इति न तान्येकज्ञानसंसर्गीणि । अतः यत्र पटुतरः संस्कारः तत् स्मर्यते, यत्र न पटुतरः संस्कारः तन्नस्मर्यते इति न काप्यनुपपत्तिः।

एवम् अदूरदेशस्थः प्रत्यक्षेण, दूरस्थः अनुमानेन च अभावः गृह्यते इति अनुपलब्धिः न प्रमाणान्तरम् ।

\articleend
