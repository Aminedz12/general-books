\chapter{पुराणलक्षणसमीक्षा}

\begin{center}
\Authorline{वि~। दिद्दिगि वंशीकृष्णः  एम्  ए }
\smallskip

सहायकसंशोधकः\\
प्राच्यविद्यासंशोधनालयः\\
मैसूरु विश्वविद्यानिलयः, मैसूरु
\rule{\textwidth}{0.1pt}
\end{center}

र्महर्षिभिश्च प्रणीतानां संस्कृतसाहित्यस्य ग्रन्थानां स्वाध्यायेन प्रतीयते यत् पुराणं वेदश्च भारतीयसंस्कृतेः संरक्षकौ सहोदरौ प्रहरिणौ स्तः~। वेदकालत एव भारतीयैर्विशिष्टैर्विद्वद्भिः स्वस्वग्रन्थेषु महता सम्मानेन सह पुराणानामुल्लेखः कृतोऽस्ति~। यतो हि वेदपुराणानामैहलौकिक-पारलौकिकाभ्युदयसाधकानामुपदेशानां बलेनैव अयं समाजः सुस्थिरोऽस्ति~। तथा विश्वस्मिन् सभ्यजातिषु अग्रणीर्वर्तते~।

लक्षणप्रमाणाभ्यां वस्तुसिद्धिरिति नियमाल्लक्षणज्ञानस्यावश्यकत्वेन किं लक्षणकं पुराणमिति जिज्ञासायां तत्तत्पुराणेषु पुराणलक्षणवर्णनप्रसङ्गे प्रोक्तमस्ति यत् सृष्टिः प्रलयः प्रसिद्धराजवंशपरम्परा मन्वन्तराणि देवर्षिराजवंशेषु समुत्पन्नानां विशिष्टव्यक्तीनां पावनचरितवर्णनञ्चेति पञ्चविषयाः प्रधानतया यस्मिन् ग्रन्थे विवेचिताः सन्ति तत् पुराणमुच्यते 
\begin{verse}
सर्गश्च प्रतिसर्गश्च वंशो मन्वन्तराणि च~।\\
वंशानुचरितञ्चैव पुराणं पञ्चलक्षणम्~॥ इति~।
\end{verse}
अत्र सर्गो नाम सृष्टिः - जगत उत्पत्तिः~। प्रतिसर्गो नाम दृश्यमानस्य समस्तस्य प्रपञ्चस्य प्रलयः~। वंशो विश्वस्योपादाभूतानां तत्त्वानां देवर्षिमनुष्याणां चोत्पत्तिपरम्परा~। मन्वन्तरं नाम सृष्ट्यादीनां कालव्यवस्थापनम्~। वंशानुचरितं च तत्तद्वंशभवानां मनुष्याणां राजर्षीणां महर्षीणां च विषये यद्वक्तव्यं तद्विवरणोपस्थानम्~।

\section*{पुराणस्य पञ्चलक्षणसमन्वयः}

विष्णुपुराणे प्रथमांशस्य प्रथमेऽध्याये पराशरं प्रति मैत्रेयस्य प्रार्थनावसरे लक्षणानां समन्वयः दृश्यते -
\begin{verse}
यन्मयं च जगद्ब्रह्मन् यतश्चैतच्चराचरम्~।\\
लीनमासीद्यथा यत्र लयमेष्यति यत्र च~॥\\
यत्प्रमाणान् भूतानि देवादीनाञ्च सम्भवम्~।\\
समुद्रपर्वतानां च संस्थानं च यथा भुवः~॥\\
सूर्यादीनां च संस्थानं प्रमाणं मुनिसत्तम्~।\\
देवादीनां तथा वंशान् मनून् मन्वन्तराणि च~॥\\
कल्पान् कल्पविभागांश्च चातुर्युगविकल्पितान्~।\\
कल्पान्तस्य स्वरूपं च युगधर्माश्च कृत्स्नशः~॥\\
देवर्षिपार्थिवानां च चरितं यन्महामुने~।\\
वेदशास्त्राप्रणयनं यथावद्व्यासकर्तृकम्~॥\\
धर्माश्च ब्राह्मणादीनां तथा चाश्रमवासिनाम्~।\\
श्रोतुमिच्छाम्यहं सर्वं त्वत्तो वासिष्ठनन्दनम्~॥ इति~। (वि० पु० १।१।१-१४)
\end{verse}
अत्र भूतानां प्रमाणं देवादीनां सम्भवः~। समुद्रपर्वतादीनां भुवः सूर्यादीनां च संस्थानमिति सर्वं सर्गेऽन्तर्भवति~। चराचरस्य प्रलयः प्रतिसर्गे, देवादीनां वंशः वंशे, कल्पकल्पविभागयुगधर्माश्च मन्वन्तरे, देवर्षिपार्थिवादीनां चरितं वेदशाखाविभागश्चेति सर्वं वंशानुचरितेऽन्तर्भवति~।

प्रशस्तं पुराणलक्षणमिदं किञ्चित् पाठभेदेन वा एकरूपेण विष्णुपुराणे (३।६।२४)~। मत्स्यपुराणे (४३।६४)~। मार्कण्डेयपुराणे (१३७।१३)~। अग्निपुराणे (१।१४)~। भविष्यपुराणे (२।२५)~। वरहपुराणे (२।४)~। गरुडपुराणस्याचारकाण्डे (२।२८)~। स्कन्दपुराणस्य प्रभासखण्डे (२।८४)~। ब्रह्माण्डपुराणस्य पूर्वभागे (१।३८) समुपलभ्यते~। अमरसिंहादिभिः कोशकारैरपि अदःपुराणलक्षणं यथैवाङ्गीकृतम्~। किन्तु श्रीमद्भागवतस्य द्वितीये स्कन्धे दशमाध्याये पुराणानि दशलक्षणलक्षितानि प्रोक्तानि -
\begin{verse}
अत्र सर्गो विसर्गश्च स्थानं पोषणमूतयः~।\\
मन्वन्तरेशानुकथा निरोधो मुक्तिराश्रयः~॥\\
दशमस्य विशुद्ध्यर्थं नवानामिह लक्षणम्~।\\
वर्णयन्ति महात्मानः श्रुतेनार्थेन चाञ्जसा~॥ इति~। (भा० पु० २।१०।१-२)
\end{verse}

\section*{दशलक्षणसमन्वयः}

श्रीमद्भागवते पुराणे एतल्लक्षणसमन्वयकारिणां विपश्चितां मतमस्ति यत् भागवतस्य प्रथमद्वितीयौ स्कन्धौ उपोद्घातस्वरूपौ स्तः~। अत एतयोराद्यद्वयोः स्कन्धयोर्न हि लक्षणसमन्वयो भवितुमर्हति~। किन्तु तृतीयस्कन्धतो द्वादशस्कन्धपर्यन्तं दशसु स्कन्धेषु क्रमशः दशानां लक्षणानां समन्वयः भवति~। तथा हि तृतीये स्कन्धे ३३ अध्यायाः सन्ति~। तेषु सर्गस्य कारणसृष्टेः तत्त्वानामुत्पत्तिप्रक्रियाया वर्णनं विद्यते~। चतुर्थे स्कन्धे ३१ अध्यायाः सन्ति येषु विसर्गस्य कार्यसृष्टेश्चराचरस्य जगतः वर्णनमस्ति। पञ्चमे स्कन्धे २६ अध्यायाः सन्ति यत्र तृतीयलक्षणस्य स्थानस्य जम्बूद्वीपादीनां स्थितेर्वर्णनं जातमस्ति~। षष्ठे स्कन्धे १९ अध्यायाः सन्ति~। तेषु चतुर्थलक्षणस्यानुग्रहणरूपस्य पोषणस्याजामिलादिरक्षायाः पष्टिः कृतास्ति~। सप्तमे १९ अध्यायाः सन्ति~। ऊतेः प्रह्लादहिरण्यकशिपुप्रभृतीनां शुभाशुभकर्मवासनाया उत्कर्षाकर्षौ वर्णितौ स्तः~। अष्टमे २४ अध्यायाः सन्ति~। तेषु षष्ठलक्षणस्य मन्वन्तरस्य निरूपणमस्ति~। नवमे २४ अध्यायाः सन्ति यत्र १-१३ अध्यायेषु भगवतो विष्णोरनुवर्तिनां सूर्यवंशसमुद्भूतानां राज्ञाम् १४-२४ अध्यायेषु चन्द्रवंशीयानां च राज्ञां नानाख्यानोपबृंहितकथापुरस्सरं सप्तमलक्षणस्य ईशानुकथायाः प्रामाणिकं विवरणमुपस्थापितम्~। दशमे ९० अध्यायाः सन्ति येषु अष्टमलक्षणस्य निरोधस्य दुष्टानां राज्ञां दैत्यानां च वधस्य विवरणं दत्तमस्ति~। एकादशे ३१ अध्यायाः सन्ति यत्र नवमस्य मुक्तेः स्वस्वरूपेणावस्थितेर्विवेचनं वर्तते~। द्वादशे च १३ अध्यायाः सन्ति येषु दशमस्य आश्रयतत्त्वस्य परमेश्वरस्योपदेशो दत्तोऽस्ति~। इदं दशममेव आश्रयतत्त्वं भागवतस्यान्तिमं लक्ष्यम् भवति~। अर्थात् मानवजन्मनो वास्तविकलक्ष्यज्ञानाय नवसंख्याकानां लक्षणानां विवरणमुपस्थापितम्~। इत्थं समान्यतया दशानां पुराणलक्षणानां भागवतपुराणस्य क्रमशः दशसु (३-१२) स्कन्धेषु समन्वयः भवति~। अपि च श्रीमद्भागवतस्य द्वादशस्कन्धे सप्तमाध्याये कतिपयशब्दभेदेन भूयो दशलक्षणानीत्थं परिगणितानि सन्ति -
\begin{verse}
सर्गोऽस्याथ विसर्गश्च वृत्ती रक्षान्तराणि च~।\\
वंशो वंशानुचरितं संस्था हेतुरपाश्रयः~॥\\
दशभिर्लक्षणैर्युक्तं पुराणं तद्विदो विदुः~।\\
केचित् पञ्चविधं ब्रह्मन् महदल्पव्यवस्थया~॥ इति~। (भा० पु० १२।७।९-१०)
\end{verse}
(१) सर्गः (२) विसर्गः (३) वृत्तिः (४) रक्षा (५) अन्तराणि (६) वंशः (७) वंशानुचरितम् (८) संस्था (९) हेतुः (१०) अपाश्रयः~। एभिर्दशभिर्लक्षणैर्लक्षितं साहित्यं पुराणविद्भिर्विद्वद्भिः पुराणं प्रोच्यते~।

येषु पुराणेष्वेषां दशानां लक्षणानां पृथक् पृथग्वर्णनं विद्यते तानि महापुराणानि कथ्यन्ते~। येषु च पञ्चलक्षणानां प्राधान्यं विद्यते तानि लघुपुराणान्युपपुराणानि वा प्रोच्यन्ते महदल्पव्यवस्थया = पुराणमहापुराणव्यवस्थया पुराणलक्षणं ज्ञेयम्~।

श्रीमद्भागवते दशानां सर्गादीनां लक्षणान्येवमुक्तानि सन्ति -

(१) ईश्वरप्रेरणया गुणत्रयसाम्यावस्थापन्नायां प्रकृतौ जायमानात् क्षोभात् महत्तत्त्वस्याविर्भावो भवति~। ततोऽहङ्कारः तत एकादशेन्द्रियाणि पञ्चतन्मात्राश्च पञ्चतन्मात्राभ्यः पञ्चमहाभूतान्युत्पद्यन्त इत्येवं तत्त्वानामुत्पत्तिक्रमः सर्गः प्रोच्यते~। तथा हि 
\begin{verse}
अव्याकृतगुणक्षोभान् महतस्त्रिवृतोऽहमः~।\\
भूतमात्रेन्द्रियार्थानां सम्भवः सर्ग उच्यते~॥ इति~। (भा० पु० १२।७।११)
\end{verse}
(२) परमात्मनोऽनुग्रहात् सृष्ट्यर्थं सामर्थ्यं प्राप्य ब्रह्मद्वारा पूर्वकर्मानुसारं सदसद्वासनाप्राधान्येन बीजाद्बीजमिव प्रवाहापन्नं यत् चराचरात्मकं जगज्जायते तद्विसर्गः कथ्यते -
\begin{verse}
पुरुषानुगृहीतानामेतेषां वासनामयः~।\\
विसर्गोऽयं समाहारो बीजाद्बीजं चराचरम्~॥ इति~। (भा० पु० १२।७।१२)
\end{verse}
(३) वृत्तिः प्राणिनां जीवननिर्वाहसामग्री~। जीवननिर्वाहार्थं चराणां प्राणिनामचराणि भूतानि वृत्तिः~। मानवो जीवननिर्वाहार्थं येषां वस्तूनामुपयोगं विधत्ते तदेव तद्वृत्तिः~। मनुष्यैः स्वस्वभावतः स्वेच्छानुसारं काश्चन वृत्तयो निश्चिताः काश्चन च शास्त्रादेशतो गृहीताः~। उभयविधाया वृत्तेरुद्देश्यं मानवजीवनस्थितिसंरक्षणमेव भवति -
\begin{verse}
वृत्तिर्भूतानि भूतानां चराणामचराणि च~।\\
कृता स्वेन नृणां तत्र कामाच्चोदनयापि वा~॥ इति~। (भा० पु० १२।७।१३)
\end{verse}
(४) प्रतियुगमीश्वरः पशुपक्षिमनुष्यर्षिदेवतादिरूपेणावतीर्य विविधां लीलां विधत्ते~। वेदद्विषो निहत्य च धर्मं स्थापयति~। तस्यावतारलीला विश्वस्य रक्षार्थमेव भवति~। अतो भगवल्लीलेयं रक्षा व्यपदिश्यते -
\begin{verse}
रक्षाच्युतावतारेह विश्वस्यानु युगे युगे~।\\
तिर्यङ्मर्त्यर्षिदेवेषु हन्यन्ते यैः त्रयी द्विषः~॥ (भा० पु० १२।७।१४)
\end{verse}
(५) मनुः मनुपुत्रा ऋषयो देवता इन्द्रो विष्णोरंशावतारश्चेत्येवं विधाभिः षड्भिर्विशेषताभिर्युक्तः कालो मन्वन्तरमुच्यते~। मन्वन्तरेणैव सर्वो व्यवहारो निर्वहति~। प्रत्येकमन्वन्तरस्याधिपतिरेकः कश्चन विशिष्टो मनुर्भवति यस्य सहयोगिनः पञ्चान्येऽपि भवन्ति -
\begin{verse}
मन्वन्तरं मनुर्देवा मनुपुत्राः सुरेश्वरः~।\\
ऋषयोंऽशावतारश्च हरेः षड्विधमुच्यते~॥ इति~।(भा० पु० १२।७।१५)
\end{verse}
(६-७) ब्रह्मणः सकाशादुत्पन्नानां भूतभविश्यद्वर्तमानानां राज्ञाम् ऋषीणाञ्च सन्तानपरम्परया वंशः~। तेषां वंशधराणां चरित्रचर्चा च वंशानुचरितं कथ्यते -
\begin{verse}
राज्ञां ब्रह्मप्रसूतां वंशस्त्रैकालिकोऽन्वयः~।\\
वंशानुचरितं तेषां वृत्तं वंशधराश्च ये~॥ इति~।(भा० पु० १२।७।१६)
\end{verse}
(८) संस्था प्रलयः~। स्वभावादेव ब्रह्माण्डस्य प्रलयो जायते~। स च चतुर्विधः - नैमित्तिक-प्राकृतिक-नित्य-आत्यन्तिकप्रलयभेदात्~। तत्त्वज्ञानिभिर्विद्वद्भिः प्रलय एव संस्थेति नाम्ना अभिधीयते -
\begin{verse}
नैमित्तिकः प्राकृतिको नित्य आत्यन्तिको लयः~।\\
संस्थेति कविभिः प्रोक्ता चतुर्धास्य स्वभावतः~॥ इति~।(भा० पु० १२।७।१७)
\end{verse}
(९) वस्तुतो जीव एवात्र हेतुशब्देन व्यवह्रियते यतः स एवादृष्टद्वारा सर्वविसर्गादीनां कर्माणां हेतुः कर्ता भवति~। अविद्याविवशो जीवो विविधकर्मकलापेष्वासक्तो भवति~। य एनं च प्रधानतत्त्वदृष्ट्या पश्यन्ति त एनमनुशयिनं प्रकृतौ शयनकर्तारं कथयन्ति~। ये चोपाधिदृष्ट्या अवलोकयन्ति ते तमव्याकृतं प्रकृतिरूपं वदन्ति -
\begin{verse}
हेतुर्जीवोऽस्य सर्गादेरविद्याकर्मकारकः~।\\
यं चानुशयिनं प्राहुरव्याकृतमुतापरे~॥ इति~। (भा० पु० १२।७।१८)
\end{verse}
(१०) यत्तिसृषु जागृत्स्वप्नसुषुप्तिष्ववस्थासु ततः परं तुरीयतत्त्वरूपेण (साक्षिरूपतया) लक्ष्यते~। तदेव ब्रह्म अत्र अपाश्रयशब्देन प्रोच्यते -
\begin{verse}
व्यतिरेकान्वयो यस्य जागृत्स्वप्नसुषुप्तिषु~।\\
मायामयेषु तद्ब्रह्म जीववृत्तिष्वपाश्रयः~॥ इति~। (भा० पु० १२।७।१९)
\end{verse}

\section*{उभयोः लक्षणयोः सामञ्जस्यम्}

तत्र सर्गविसर्गावुभयत्र समानौ स्तः~। द्वितीयस्कन्धे निर्दिष्टस्य स्थानशब्दस्य स्थाने द्वादशस्कन्धे वृत्तिपदस्य प्रयोगः विद्यते~। द्वितीये प्रोक्तमनुग्रहरूपं पोषणं द्वादशे रक्षापदेन प्रोच्यते~। द्वितीये प्रयुक्तस्य जीवस्य संसारप्राप्तिहेतुकाविद्याकर्मवासनादिबोधकस्य ऊतिपदस्य स्थाने द्वादशे हेतुपदं प्रयुक्तम्~। द्वितीये उल्लिखितस्य मन्वन्तरमित्यस्य स्थाने द्वादशेऽन्तरपदमुपात्तम्~। द्वितीये दर्शितस्य ईशानुकथापदस्य स्थाने द्वादशे वंशवंशानुचरितपदे प्रयुक्ते~। द्वितीये उक्तस्य निरोधस्य स्थाने द्वादशे संस्थापदस्य प्रयोगो जातः यत्र नित्यनैमित्तिकप्राकृतिकप्रलया अन्तर्भवन्ति~। द्वितीये प्रोक्ता मुक्तिर्द्वादशे संस्था इति आत्यन्तिकप्रलयाभेदेन पदेन गृह्यते~। तथा द्वितीयस्कन्ध उपात्तस्याश्रयस्य स्थाने द्वादशेऽपाश्रयशब्दो व्यपदिष्टः~।

एवं श्रीमद्भागवतवद्ब्रह्मवैवर्तपुराणस्य श्रीकृष्णजन्मखण्डेऽपि महापुराणानि दशलक्षणलक्षितानि प्रोक्तानि -
\begin{verse}
सृष्टिश्चापि विसृष्टिश्च स्थितिस्तेषां च पालनम्~।\\
कर्मणां वासना वार्ता मनूनां च क्रमेण च~॥\\
वर्णनं प्रलयानां च मोक्षस्य च निरूपणम्~।\\
उत्कीर्तनं हरेरेव देवानां च पृथक् पृथक्~॥\\
दशाधिकं लक्षणञ्च महतां परिकीर्तितम्~। इति~। (ब्र० वै० १३१।८७-८९$*$)
\end{verse}
एवं सृष्टिः, विसृष्टिः, स्थितिः, पालनम्, कर्मणां वासना, मनूनां वार्ता, प्रलयानां वर्णनम्, मोक्षस्य च निरूपणमित्यष्टावुभयत्र समानान्येव सन्ति~। हरेरुत्कीर्तनमित्यस्याश्रयेऽन्तर्भावो जायते~। देवानाञ्च पृथक् पृथगित्यस्य ईशानुकथायां वंशानुचरिते वा अन्तर्भावो भवति~। इत्येवं शब्दान्तरैः श्रीमद्भागवतस्य लक्षणानि ब्रह्मवैवर्तेऽपि प्रोक्तानि~।

\section*{पञ्चलक्षणे दशलक्षणस्यान्तर्भावः}

ननु दशलक्षणस्य पञ्चलक्षणेन विरोधात् कथं तयोः प्रामाण्यमित्येवं स्थितौ कियन्तो विद्वांसः दशलक्षणस्य पञ्चलक्षण एवान्तर्भावं कुर्वन्ति~। केचन च महापुराणानि दशलक्षणानि उपपुराणानि पञ्चलक्षणानीति च मन्यन्ते~। तत्र पुराणं पञ्चलक्षणमिति वादिनां विपश्चितां मते श्रीमद्भागवते प्रोक्तानि दशलक्षणानि पुराणान्तरप्रोक्तपञ्चलक्षणस्य आवश्यकतानुसारं विस्तारमात्रं भवति~। यतो हि श्रीमद्भागवते सर्गविसर्गवंशवंशानुचरितमन्वन्तराणि तु शब्दत एव प्रोक्तानि सन्ति~। अवशिष्टानामन्येषामप्यन्तर्भाव एतेष्वेव सम्भवति~। तथा हि सर्गो नाम प्रकृतेर्गुणवैषम्याज्जायमाना विराट्सृष्टिः (ब्रह्माण्डस्य समष्टिसृष्टिः)~। विसर्गश्च ब्रह्माण्डान्तर्वर्तिनां ब्रह्मणो जायमानानां जीवानां व्यष्टिसृष्टिः~। परिणामतः सर्गे विसर्गस्यान्तर्भावो नानुचितः~। तथा च विसर्गः किल सर्गस्यावान्तरभेदः~। एवमाश्रयापाश्रयशब्दाभ्यामुपात्त ईश्वरः सर्गकर्तृत्वेन हेतुः ऊतिर्वा सर्ग एवान्तर्भवितुमर्हति~। प्रतिसर्गे संस्थानिरोधयोरन्तर्भावः~। मन्वन्तरेऽन्तरस्यान्तर्भावः~। वृत्तिः स्थानम् ईशानुकथा पोषणम् रक्षा वा वंशवंशानुचरितयोः स्फुटमन्तर्भवन्ति~।

तथा हि \_
\begin{verse}
सर्गे विसर्ग-हेतु-ऊति-आश्रयाणामन्तर्भावः~।\\
प्रतिसर्गे संस्था-निरोधयोरन्तर्भावः~।\\
मन्वन्तरे अन्तरस्यान्तर्भावः~।
\end{verse}
वंशवंशानुचरितयोः वृत्ति-स्थान-रक्षा-पोषण-ईशानुकथानामन्तर्भावः~।

तस्मात् श्रीमद्भागवतीयानि दशलक्षणानि पुराणान्तरोक्तस्य पञ्चलक्षणस्य प्रपञ्चमात्रमुपबृंहितरूपं वेति स्वीकारे न काऽप्यनुपपत्तिः~। दशलक्षणं पुराणमिति मन्यमानानां मनीषिणां मते तु येषु दशलक्षणानां पृथक् पृथग्वर्णनं वर्तते तानि महापुराणानि कथ्यन्ते~। यत्र च पञ्चानामेव लक्षणानां प्राधान्यं वर्तते तान्युपपुराणानि प्रोच्यन्ते~। अत एव ब्रह्मवैवर्तपुराणस्य श्रीकृष्णजन्मखण्डे प्रथमं पूर्वोक्तं सर्गश्च प्रतिसर्गश्चेत्यादि पञ्चलक्षणमुक्त्वा प्रोक्तं यत् -
\begin{verse}
एतदुपपुराणानां लक्षणं च विदुर्बुधाः~। -इति~।
\end{verse}
ततो दशलक्षणविवक्षया सङ्केतितम् -
\begin{verse}
महतां च पुराणानां लक्षणं कथयामि ते~॥ -इति~। (ब्र० वै० १३१।७)
\end{verse}
भागवतेऽपि महापुराणोपपुराणलक्षणनिर्णयप्रसङ्गे प्रोक्तं यत् पुराणविदो विद्वांसो दशभिर्लक्षणैर्लक्षितं पुराणमिति वदन्ति~। अपरे च पञ्चलक्षणलक्षितं पुराणमित्युच्यन्ते~। तदित्थं दशलक्षणलक्षितानि महापुराणानि पञ्चलक्षणलक्षितानि चोपपुराणानीति तेषामभिप्रायः -
\begin{verse}
दशभिर्लक्षणैर्युक्तं पुराणं तद्विवो विदुः~।\\
केचित् पञ्चविधं प्राहुर्महदल्पव्यवस्थया~॥ इति~। (भा० पु० १२।७।१०)
\end{verse}
वस्तुतो पुराणसामान्यस्य दशलक्षणानि न सन्ति अपि त्विमानि ब्रह्मवैवर्तस्य भागवतस्य च नैजलक्षणानि सन्ति~। यतो हि भगवत्स्वरूपविवेचनमेव श्रीमद्भागवतस्य प्रधानं कारणम्~। अतो दशमस्याश्रयतत्त्वस्य परमेश्वरस्य स्वरूपपरिज्ञानायैवात्र नवानां लक्षणानां विवेचनं कृतमस्ति 
\begin{verse}
दशमस्य विशुद्ध्यर्थं नवानामिह लक्षणम्~।\\
वर्णयन्ति महात्मानः श्रुतेनार्थेन चाञ्जसा~॥ इति~। (भा० पु० १२।७।२)
\end{verse}
एवं दशमं लक्षणं प्रापयितुं पूर्वोक्तानां नवानां लक्षणानां प्रयोजनमुक्तम् श्रीमद्भागवते साक्षाद्व्यासेनैव~।

\articleend
