\chapter{शक्तिविशिष्टाद्वैत वेदान्ते प्रमाणानि}

\begin{center}
\Authorline{डा. एम्.महादेवय्यः}
\smallskip
प्रांशुपालः, श.वि.वेदान्तप्राध्यापकःच,\\
महाराज-संस्कृत-महापाठशाला,\\
मैसूरु
\addrule
\end{center}
न्यायः, वैशेषिकं, सांख्यं, योगः, पूर्वमीमांसा, उत्तरमीमांसाचेति आस्तिकदर्शनानि षट्। तत्र उत्तरमीमांसादर्शनमेव वेदान्तदर्शनमितिप्रथते। वेदान्तदर्शनेषु च चत्वारिमतानि अद्वैत-विशिष्टाद्वैत-द्वैत-शक्तिविशिष्टाद्वैत-नामानि विविधाचार्योपज्ञानि आगच्छता कालेन अभिवृद्धानि। प्रकृतप्रबन्धः शक्तिविशिष्टाद्वैतवेदान्तानुरोधेन प्रमाणानां विचारं कर्तुं प्रवृत्तोऽस्ति।

सर्वप्रथमं कोऽयं शक्तिविशिष्टाद्वैतवेदान्तो नाम इति किञ्चितप्रस्तूयते। प्रायःविदुषां अस्य सिद्धान्तस्य विषये वास्तविकांशानां परिचयः स्वल्प इति विज्ञायते।

तत्र  शक्तिविशिष्टाद्वैतम्’ इति पदस्य व्युत्पत्तिः एवं भवति -

शक्तिश्च शक्तिश्च शक्ती, ताभ्यां विशिष्टौ -  शक्तिविशिष्टौ शिवजीवौ, तयोः अद्वैतं शक्तिविशिष्टाद्वैतं इति ।(शिवाद्वैतपरिभाषा)

अयञ्च शक्तिविशिष्टाद्वैतवेदान्तः वीरशैवाभिधैः अनुस्रियते इति अभिप्रायः लोके वर्तते । अस्यैव सिद्धान्तस्य् अन्यान्यपि नामान्तराणि भवन्ति, अवमस्यसिद्धान्तस्य प्रधानांशश्च तत्रैव शिवाद्वैतपरिभाषां एवं प्रोक्तम् -

सकलनिगमागमप्रसिद्धस्यास्य सिद्धांतस्य शिवाद्वैत-शक्तिविशिष्टाद्वैत – विशेषाद्वैत – षट्स्थल – लिंगांगसामरस्य – शिवयोगाद्यभिधेयत्वं महापाशुपतपर्यायत्वमप्यस्ति । तथा शिवागमेषु एतत्सर्वनामसमन्वितोऽयं वीरशैवशब्देन व्यवह्रियते । सत्यपि अभिधानबेदे सर्वे तदाश्रिताशयाः खलेकपोतन्यायेनशिवजीवैक्यबोधे एव समासमापतंति । इति ।

एवं सति शिवजीवैक्यबोधकशिवाद्वैतशब्दस्यार्थः क  इति जिज्ञासायां, अत्रोच्यते -

शिवाद्वैतमित्यत्र शिवे अद्वैतमिति व्युत्पत्तिः । मलत्रयसंबंधेन शरीरत्रयावच्छिन्नं सत्काष्ठसंयोगावस्थायां वह्नेः अनंतविस्फुलिंगाविर्भाववत्, तदीयेच्छशक्तिवशात्, विभक्तमिदं स्वस्वरूपगोपनावशेन अनंतजीवभावमापन्नं यत्परशिवचैतन्यं, तत्, स्वशक्तिविकाशवशात्, मलत्रयापगमेन शरीरत्रयानवच्छ्न्नं सत्,  स्वस्वरूपाविर्भावप्रभावेन

‘यथा नद्यः स्यंदमानाः समुद्रेऽस्तं गच्छंति नामरूपे विहाय’ (मुंडकोपनिषत् ३-२-८)

इत्यादि श्रुतिशत-प्रमाणसमासादि तं स्वस्वरूपानंदरसनिर्भरं स्वयमेवभवतीति ।

तथाचोक्तम् –
\begin{verse}
चैतन्यमात्मनोरूपंसच्चिदानंदलक्षणम् ।\\
तस्यानावृतरूपत्वात्शिवत्वंकेनवार्यते ? ॥इति
\end{verse}
तत्र वीरशैवशब्दस्य निवचनम् एवमस्ति इति तदीयग्रन्थपठनेनज्ञायते -
\begin{verse}
वीशब्देनोच्यतेविद्याशिवजिवैक्यबोधिनी ।\\
तस्यांरमंतेयेशैवाःवीरशैवाःप्रकीर्तिताः ॥’इति ।
\end{verse}
तत्रापि -

विशेषाद्वैतमित्यत्र – विश्वशेषश्च विशेषौ, शिवजीवौ, तयोरद्वैतम्विशेषाद्वैतमिति । अत्र ‘वि’ – शब्देन सूक्ष्मचिदचिद्रूपशक्तिविशिष्टः परशिवः उच्यते । शेषशब्देन शरीरम्, अर्थात्स्थूलचिदचिद्रूपशक्तिविशिष्टो जीवः उच्यते । तयोरद्वैतमेकत्वम् अस्तीति वाक्यार्थबोधः । अमुमेवार्थमभिप्रेत्य ब्रह्मसूत्रभाष्ये श्रीश्रीपतिपंडितभगवत्पादैः बहुधाप्रपंचितम् । तत्र षट्स्थलाष्टावरणादिक्रमोऽपि शक्तिविशिष्टाद्वैतप्रक्रियामनुसृत्य सृष्टतयाप्रतिपादितः । स च तत्कृतश्रीकरभाष्येद्रष्टव्यः ।इति ।

\section*{शक्तिविशिष्टाद्वैतवेदान्ते प्रमाणानि}

तत्र शक्तिविशिष्टाद्वैतवेदान्ते त्रीणिप्रमाणानि अङ्गीक्रियन्ते। तानि यथा - प्रत्यक्षं, अनुमानं, शब्दः चेति। एतैः प्रमाणैः उत्पद्यमानाः प्रमितयः यथाक्रमं प्रक्षं, अनुमितिः, शाब्दबोधः इति च उच्यते।

प्रत्यक्षम्:-

इदं च सर्वप्रमाणमूर्धन्यतमम् इति सर्वसम्मतम्। अक्षमक्षं प्रतीत्य - प्रत्यक्षम् इति प्रत्यक्षपदस्य व्युत्पत्तिः। प्रत्यक्षप्रमाणम् इत्यत्र विद्यमानं प्रत्यक्षपदम् इन्द्रियार्थसन्निकर्षः एव अत्रप्रत्यक्षशब्देन बोध्यते। इन्द्रियाणि च घ्राणरसनचक्षुस्वक्श्रोत्रमनांसि। एतेषाम् इन्द्रियाणां तत्तदिन्द्रियग्राह्यवस्तुभिः साकं सम्मन्धः यदा भवति तदानीं तत्तद्विषयभूतवस्तुविषयकज्ञानं जायते। एवं जायमानस्य ज्ञानस्य प्रत्यक्षप्रमितिः इतिनाम। तत्कारणभूतस्य इन्द्रियार्थसन्निकर्षस्य प्रत्यक्षप्रमाणम इति च नाम। प्रत्यक्षप्रमितिहि बाह्यम् आन्तरं चेति द्विविधम्। मन आख्येन अन्तरिन्द्रियेण जायमानं सुखदुःखादिविषयकप्रत्यक्षम् आन्तरप्रत्यक्षम् इत्युच्यते। घ्राणादिबाह्येन्द्रियैः जायमानं प्रत्यक्षं तु बाह्यप्रत्यक्षम् इत्युच्यते।

एवमेव प्रत्यक्षप्रमितिकारणभूतः इन्द्रियार्थसन्निकर्षोऽपि द्विविधः -
\begin{enumerate}
\item लैकिकसन्निकर्षः,
\item अलौकिकसन्निकर्षश्चेति।
\end{enumerate}
तत्रलौकिकसन्निकर्षःषड्विधः -
\begin{enumerate}
\item संयोगः (चक्षुषाघटप्रत्यक्षजनने उपयुज्यते)
\item	संयुक्तसमवायः (घटरूपप्रत्यक्षे अयं सन्निकर्षः)
\item	संयुक्तसमवेतसमवायः (रूपत्वादिसाक्षात्कारे अयं सन्निकर्षः)
\item 	समवायः (श्रोत्रेणशब्दसाक्षात्कारे अयं सन्निकर्षः)
\item 	समवेतसमवायः (शब्दत्वसाक्षात्कारे अयं सन्निकर्षः उपयुज्यते)
\item 	विशेषणविशेष्यभावः (अभावप्रत्यक्षे अयं सन्निकर्षः उपयुज्यते)
\end{enumerate}
अलौकिकसन्निकर्षस्तु त्रिविधः -
\begin{enumerate}
\item सामान्यलक्षणः
\item ज्ञानलक्षणः
\item योगलक्षणः
\end{enumerate}
(एतेषांविवरणानिविस्तरभयात्अत्रनप्रतिपादितानि)

\section*{अनुमानप्रमाणम् -}

इदं प्रमाणेषु द्वितीयम्। दृष्टं किञ्चित्वस्तु आदाय तदानीम् अदृश्यमानस्य अन्यस्य कस्यचित्वस्तुनः ज्ञानावाप्तिः अनुमितिरित्युच्यते। तादृश अनुमितेः कारणभूतं यत्तदेव अनुमानमित्युच्यते। सामान्यतः अनुमानं नाम व्याप्तिज्ञानम्। व्याप्तिज्ञानमन्तरा अनुमितेः अनुदयात्। अनुमितिप्रक्रियाच एवं भवति - यथा यदि कश्चित्पर्वतादौ धूमादिदर्शनानन्तरं तेन वह्न्यादेः ज्ञानम् अवाप्नोति चेत्तत्र प्रथमं सःपुरुषः पक्षत्वेन व्यवह्रीयमाणे पर्वते हेतुरितिव्यवह्रीयमाणस्य धूमस्य दर्शनम् अवाप्नोति- धूमवान्पर्वतः इति। इदमेव पक्षधर्मताज्ञानम् इत्युच्यते। ततःपरं यत्र यत्र धूमः तत्र तत्र वह्निः यथा महानसः | इति ज्ञानम् अवाप्नोति। इदमेवव्याप्तिज्ञानम् इत्युच्यते। ततः परं वह्निव्याप्यधूमवान् अयं पर्वतः इतिज्ञानमाप्नोति। यच्चपरामर्शः इत्युच्यते। तेनपर्वतोवह्निमान् इत्यनुमितिरुत्पद्यते। एवंरीत्या अनुमितिप्रक्रियानिरूपितावर्तते। अत्र केचित्व्याप्तिज्ञानमेव अनुमानमिति, अन्ये तु परामर्श एव अनुमानमिति च वदन्ति।

शाब्दबोधः -

अयं च शाब्दबोधः शब्दादुत्पद्यते। शब्दो हि नाम आप्तवाक्यम्। आप्तस्तु यथार्थवक्ता। यथा ’नद्यास्तीरे फलानि सन्ति’ इति आप्तोक्तं वाक्यं श्रुत्वा कश्चित्पुरुषः तादृशवाक्यार्थज्ञानमवाप्य तदनुगुणं नदीतीरं प्राप्य फलानि लभते तत्र तादृशवाक्यश्रवणेन जातः बोधः एव शाब्दबोधः इत्युच्यते। एतादृशशाब्दबोधे तावत्पदज्ञानं करणं, पदजन्यपदार्थस्मरणंव्यापारः, शक्तिज्ञानं च सहकारि। किञ्च आकांक्षा, योग्यता, सन्निधिः, तात्पर्यज्ञानम् इत्येतान्यपि शाब्दबोधकारणवर्गे अन्तर्भवन्ति।

एवं रीत्या त्रीणि प्रमाणानि शक्तिविशिष्टाद्वैतवेदान्ते निरूपितानि। एतदतिरिच्यइतोऽपिपञ्चप्रमाणानि अन्यैः दार्शनिकैः प्रतिपादितानि। तानि यथा -
\begin{enumerate}
\item उपमानम् - उपमितिकरणम् उपमानम्। उपमितिः हि नाम संज्ञासंज्ञिसंबन्धज्ञानम्। एतदवाप्तिप्रकारस्तु - कश्चित्पुरुषः गवय इति पदस्य अर्थं अजानन्तादृशपदार्थज्ञानवतः पुरुषस्य सकाशात् ’  गोसदृशः गवयः’ इति श्रुत्वाततः परम् अरण्यं गतः तादृशं प्राणिनं दृष्ट्वा पूर्वोक्तं तदभिज्ञवाक्यं स्मरन् “असौ गवयपदवाच्यः” इति ज्ञानमवाप्नोति। इदमेव ज्ञानम् उपमितिरित्युच्यते। अस्य कारणमेव उपमानम् इत्युच्यते। सादृश्यज्ञानमेव उपमानम्।
\item अर्थापत्तिः -अत्र अर्थादापद्यते इत्यर्थापत्तिः इति व्युत्पत्या, ’ पीनोदेवदत्तः दिवा न भुङ्क्ते ’ इत्यनेन देवदत्तस्य रात्रिभोजित्वं कल्प्यते।
\item अनुपलब्धिः - भूतलेघटोनास्ति इति ज्ञानस्य भूतलवृत्तिघटानुपलब्धिद्वारा ग्रहणात् अभावग्रहणकारणभूतं प्रमाणम् अनुपलब्धिः इत्युच्यते।
\item संभवः - शते पञ्चाशन्न्यायेन किञ्चित् एवं संवृत्तं स्यात् इति पूर्वतनानां वचांसि संभवः इत्युच्यते।
\item ऐतिह्यम् - अज्ञातवक्तृप्रोक्तं बहुकालिकपरंपरातः प्राप्तं इतिहासादिसम्बद्धः वचः एवऐतिह्यमित्युच्यते । एतेषु प्रमाणेषु उपमानम्, संभवः, ऐतिह्यं इतीमानि पृथक्प्रमाण्त्वेन प्रथमानानि शब्दप्रमाणे अन्तर्भवन्ति। अर्थापत्तिः अनुमानप्रमणे अन्तर्भवति। अनुपलब्धिः अभावे अन्तर्भवति। अतः आहत्यत्रीण्येवप्रमाणानि अन्ततः सिध्यन्तीति शक्तिविशिष्टाद्वैतवेदान्तसिद्धान्तः।
एवम्शक्तिविशिष्टाद्वैतवेदान्ते प्रमाणानि निरूपितानि।
\end{enumerate}
\articleend
