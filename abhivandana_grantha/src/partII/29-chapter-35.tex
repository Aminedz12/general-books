{\fontsize{15}{17}\selectfont
\presetvalues
\chapter{सांख्यदर्शनस्य संक्षिप्तपरिचयः}

\begin{center}
\Authorline{वि~॥ श्रीपाद-भट्टः}
\smallskip

निवृत्ताध्यापकः\\
शारदाविलासशिक्षणसंस्था, मैसूरु
\addrule
\end{center}

भारतं दर्शनानां जन्मभूमिः इति तु विदितमेव~। अत एव इयं पुण्यभूमिः इति कथ्यते~। पञ्चसहस्रवर्षेभ्यः प्रागेव वेदाः प्रादुरभूवन्~। तेषां सारं तु दर्शनान्येव इति तु ज्ञातमेवास्ति~। अवैदिकान्यपि दर्शनानि सन्त्येव~। अत एव दर्शनानि आस्तिकानि नास्तिकानि इति द्वेधा विभज्यन्ते~। वेदानुसारीणि आस्तिकानि चेत् तदितराणि नास्तिकानि भवन्ति~। “अस्ति नास्ति दिष्टं मतिः” इति पाणिनि-सूत्रानुसारेण देवस्य अस्तित्वनास्तित्वाङ्गीकारेणैव आस्तिकता नास्तिकता च स्वीकृता भवेत्~। परन्तु उत्तरत्र वेदस्य प्रामण्याप्रामाण्याङ्गीकारेणैव तत्स्वीकृतं दृश्यते~। तानि च आस्तिकानि षट् नास्तिकानि च षडिति प्राधान्येन कथ्यन्ते~। 

तादृशदर्शनपर्वतसमूहे अन्यतमं भवति साङ्ख्यम्~। अस्य प्रवक्ता कपिलो मुनिरिति कथ्यते~। इदं च दर्शनं प्राचीनं इत्यपि कथयन्ति~। यतः श्वेताश्वतरोपनिषदि एव प्रकृति पुरुषयोः उल्लेखः बीजरूपेण दृश्यते~। 
\begin{verse}
अजामेकां लोहितशुक्लकृष्णां बह्वीः प्रजा जनयन्ती सरूपाः~।\\
अजो ह्येको जुषमणोऽनुषेते जहात्येनाम्भुक्तभोगामजोन्यः~॥
\end{verse}
एवमेव कठोपनिषद्यपि अव्यक्तमहङ्कारश्चेति सांख्यिकपारिभाषिके पदे दरीदृश्यते~। तथा प्रश्नोपनिषद्यपि अस्य दर्शनस्यांशाः सूचिता दृश्यन्ते~। अर्वाचीनेति विक्यातायां मैत्रायणीयोपनिषदि च साङ्ख्यतत्वानां स्वीकरणं सुस्पष्टतया वयं पश्यामः~। अस्य च दर्शनस्य साङ्ख्य़मिति प्राप्तौ कारणद्वयं वर्तते~। तत्र प्रथमं तावत् संख्यातः पञ्चविंशति तत्वानि अत्र उल्लिखितानि~। एवं संख्यानं नाम सयुक्तिविश्लेषणम् अवलोकनञ्च~। तथा च विवेचनप्रधानमिदं दर्शनमिति हेतोः सांख्यमिति नाम प्राप्तमिति पण्डिताः कथयन्ति~। महाभारतग्रन्थेऽपि षड्विंशति तत्वयुक्तं सेश्वरसांख्यं पुरस्कृतं दृश्यते~। 

अस्य च दर्शनस्य ईश्वरकृष्णीया सांख्यकारिका एव इदानीं प्रमाणभूतो ग्रन्थः दृश्यते~। असौ- ईश्वरकृष्णः च आसुरी पञ्चशिखा नाम्ना कपिलशिष्ययोः परम्परायामेव जातः इति ज्ञायते~। इतः पूर्वम् चरकः चतुर्विंशति तत्वयुक्तं सांख्यदर्शनमेव उररीकृतवान्~। अयं च पञ्चशिखा सिद्धान्तः इति कथ्यते~। षड्विंशति तत्वयुक्तं सेश्वरसांख्यम् च अहिर्बुध्न्यसंहितायां दृश्यते इति पण्डिताः कथयन्ति~। प्रायः इदमेव कापिलं दर्शनं भवेदिति बहवः विद्वांसः अभिप्रयन्ति~। कपिलः सांख्यप्रवचनसूत्राणि व्यरचयदिति केचन पण्डिताः वदन्ति~। सांख्यकारिकायाः पुनः गौडपादीयं भाष्यं वाचस्पतेः तत्वकौमुदी पण्डितनारायणीया चन्द्रिका च विख्याताः विवरणग्रन्थास्सन्ति~। सांख्यप्रवचनसूत्रस्य भाष्यकारः विज्ञानभिक्षुस्सांख्यस्य तथा वेदान्ते प्रतिपादितस्य ईश्वरतत्वस्य तावान्भेदो नास्तीति प्रत्यपादयत् 

\section*{उद्देशः}

आधिदैविकम् , आधिभौतिकम्, आध्यात्मिकं चेति तापत्रयनिवारणाय कपिलः आसुरी\-नामकं शिष्यं प्रति दर्शनमिदमुपदिष्टवानिति श्रूयते~। अत्रोक्तानि तत्वानि जीवनानुभवानु\-सारेणैव निरूपितानि~। प्रकृतिः बाह्यप्रपञ्चात्मिका भवति तज्ज्ञातारो जीवास्तु तस्याः भिन्नाः एव भवन्ति~। ते एव चेतनात्मकाः पुरुषाः भवन्ति~। प्रकृतिस्तु जडस्वरूपाऽस्ति~। पङ्ग्वन्धन्यायेन तयोः परस्परसाहाय्येन लौकिको व्यवहारः प्रवर्तते~। कथं पुरुषः प्रकृतेर्भिन्नो भवतीति निरूपणमेव दर्शनस्य उद्देशो भवति~। इन्द्रियप्रत्यक्षज्ञानं पुनस्तर्केणपरिशील्यते~। एवं च इदं विवेचनं प्रकृतिपुरुषाभ्याम् भिन्नंं भवति~। 

\section*{प्रमाणानि}
 सांख्यदर्शनकारैः प्रत्यक्षानुमानं शब्दश्चेति त्रीण्येव प्रमणानि स्वीकृतानि~। उपमानादिकञ्चानुमाने एवान्तर्भवतीति तेषामभिप्रायोस्ति~। तदुक्तं कारिकायाम् - 
\begin{verse}
दृष्टमनुमानमाप्तवचनञ्च सर्वप्रमाणसिद्धत्वात्~। \\
त्रिविधं प्रमाणमिष्टं प्रमेयसिद्धिः प्रमाणाद्धि~। 
\end{verse}
इन्द्रियगोचरविषयकं ज्ञानं प्रत्यक्षेण भवति~। इन्द्रियागोचरं ज्ञानमनुमानप्रमाणेन भवति~। अनुमानेनालभ्यज्ञानमाप्तवचनरूपेण शब्देन भवति~। 

\section*{प्रकृतिः गुणाश्च}

सांख्यमते अव्यक्तं प्रधानं प्रकृतिरिति अनर्थान्तरम्~। सा च सत्वरजस्तमोगुणात्मिका भवति~। प्रकरोतीति प्रकृतिः~। सैव महदादिरूपेण परिणमिता सती प्रत्यक्षा भवति~। कार्यकारणयोर्मध्ये भेदः अभेदश्च स्वीक्रियते सांख्यैः~। अतः कर्यं कारणात् भिन्नाभिन्नरूपं भवति~। यथा प्रकृतिः गुणत्रयात्मिका तथैव तत्कार्यमपि त्रिगुणात्मकं भवति~। यथा तन्तवः त्रयः मिलित्वा एको गुणो भूत्वा कार्यकारिणः भवन्ति तथैव त्रयो गुणाः मिलित्वा सृष्ट्यादिकार्यं प्रति कारणं भवति~। एते गुणाः भोगापवर्गौ प्रत्यपि कारणानि भवन्ति इति वाचस्पतिमिश्राः कथयन्ति~। सत्वेनप्रकाशः [ ज्ञानम् ] सुखं च , एवं रजसा दुःखं प्रचोदनं, तमसा च बन्धः औदासीन्यं चानुभूयते~। यथा एकस्मिन् दीपे द्रवरूपं तैलं जडा वर्ती ,एवम्प्रकाशरूपं ज्वलनं च परस्परविरोधे सत्यपि मिलित्वा उपकारकाणि भवन्ति तद्वद्गुणाः अपि परस्परविरोधे सत्यपि मिलित्वा उपकारकाः भवन्ति~।   

\section*{पुरुषः}

पुरुषस्तु कूटस्थनित्योऽपरिणामेऽनुभयात्मकः चेतनः~। प्रकृतिश्च जडा~। पङ्ग्वन्धाविव तौ मिलित्वा कार्यं प्रवर्तेते~। फलापेक्षां विनापि प्रकृतिः परोपकारार्थं कार्ये प्रवर्तते~। इयं प्रकृतिः पुरुषार्था भवति~। यदा पुरुषे नित्यानित्यविवेकः समुत्पद्यते तदा सः दुःखाद्विमुक्तो भवति~। जननमरणादिकं प्रत्येकं भिन्नमिति दर्शनात्पुरुषास्तु अनेका एव भवन्ति~। अविवेक एव पुरुषस्य बन्धकारणम्~। तन्निरासे स इह मुक्तो भवति~।तदुक्तम्\enginline{-} वत्सविवृद्धि-निमित्तं क्षीरस्य यथा प्रवृत्तिरज्ञस्य~। पुरुषविमोक्षनिमित्तं तथा प्रवृत्तिः प्रधानस्येति~। 

\section*{सृष्टिस्वरूपम्-} 

प्रकृतेः परिणामेषु कार्येषु प्रथमं महत्वं (बुद्धिः)  ततोऽहङ्कारः  तस्मात्पञ्च ज्ञानेन्द्रियाणि\break पञ्चकर्मेन्द्रियाणि च उत्पद्यन्ते~। ततः ज्ञानेन्द्रियविषयाः शब्दादयः पञ्च~। एवं पञ्चतन्मात्राः, पञ्चभूताः पृथिव्यादयः समुत्पद्यन्ते~। त्वक्चक्षुश्रोत्रजिह्वाघ्राणेति पञ्चकं ज्ञानेन्द्रियं भवति~।\break वक्पाणिपादपायूपस्थाश्चेति पञ्चकं कर्मेन्द्रियम्भवति~। सङ्कल्पविकल्पात्मकं मनश्च उभयात्मकं भवति~। बुद्धेः अहङ्कारस्य मनसश्च अन्तःकरणानीत्यपि नामधेयं भवति~।\break अन्यानीन्द्रियाणि द्वारमात्राणि भवन्ति~। तानि स्वानुभवं मनः प्राप्नुवन्ति~। तानि सङ्गृहीत्वा अहङ्कारं प्रति प्रेषयति~। अहङ्कारश्च बुद्धिं प्रति प्रेषयति~। एतानि अन्तःकरणानि समेत्य\break पुरुषापेक्षितमर्थं पूरयन्ति~। तदुक्तम् 
\begin{verse}
प्रकृतेर्महान् ततोऽहङ्कारस्तस्माद्गणश्च षोडशकः~। \\
तस्मादपि षोडशकात्पञ्चश्यः पञ्चभूतानि~॥ इति~॥
\end{verse}
\vskip -48pt

\section*{पुरुषाय अष्टसिद्धयः सन्ति}
\vskip -5pt

१.साङ्ख्यागमानामध्ययनम् 	२.आप्तोपदेशः ३.  उपदिष्टस्य तर्केण मननम्
४. सहाध्यायिभिस्सः विमर्शः 	५. स्वत्वभावनात्यागेन बुद्धेः परिशुद्धता	६. प्रमोदः
७. मुदितः	८. मोदः च~। अन्ते विद्यमानास्त्रयः दुःखनिरोधकाः~। एतानि सर्वाणि मुक्तिसाधनानि भवन्ति~। 

परमप्राप्तिः मुक्तिरेव~। अविवेकेन प्रकृतिपुरुषयोः क्रीडा एवपुरुषस्य संसारिकजीवनं प्रति कारणमिति साङ्ख्यदर्शनं कथयति~। अन्ते उत्प्द्यमानविवेकेन मोक्षः लभ्यते इत्यपि सूचयति~। अतः प्रकृतिपुरुषाभ्यां भिन्नः ईश्वरः नोपेक्ष्यते इति निरीश्वरसांख्याः कथयन्ति~। यदि कश्चित्सार्वभौमः ईश्वरः स्वीक्रियते तर्हि यः कोपि जन्तुः दुःखितो न भवेदेव इति वदन्ति~। यद्यपि श्वेताश्वतरोपनिषत्शैवागमाश्च ईश्वरमङ्गीकुर्वन्ति~। तथापि भौतविषयभूतं शरीरमेव चैतन्यम्~॥ प्रकृत्या क्रियमाणं कार्यजातं स्वकीयमिति बुद्धिः अविवेकः~। अहं तावन्न तादृशः इति ज्ञानमेव विवेकः~। ईदृशो विवेक एव मुक्तिं प्रति कारणं भवति इत्येव सांख्यानामाघोषो भवति~। 

\centerline{{\fontsize{10}{12}\selectfont\ding{97}\quad\ding{97}\quad\ding{97}}}
}
