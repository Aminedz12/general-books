{\fontsize{15}{17}\selectfont
\presetvalues
\chapter{योगवेदान्तदर्शनयोः  उपायोपेयभावविमर्शः}

\begin{center}
\Authorline{डा~॥ नागराजभट्टः}
\smallskip

सहायक-प्राध्यापकः, संस्कृताध्ययनविभागः\\
रामकृष्णमिशन्-विवेकानन्दविश्वविद्यालयः\\
बेलूरुमठः, कोलकात्ता, पश्चिमवङ्गः
\addrule
\end{center}

\section*{विषयसङ्गतिः}

योगवेदान्तदर्शनयोः विशिष्टः कश्चित्सम्बन्धो विलसति~। परमतात्पर्यार्थे तात्त्विकांशे वा नूनमनयोः दर्शनयोः सत्यपि विरोधे बहवो विषयाः अविरोधेनैव प्रतिपादिताः~~। योगदर्शनं वेदान्तम् अनुष्ठानस्तरे व्यवहारकाले वा महदुपकरोत्येव~। एवं योगवेदान्तदर्शनयोः विरोधाविरोधरूपोऽयं सम्बन्धः ब्रह्मसूत्रे तद्भाष्ये च प्रतिभाति~। भगवता बादरायणेन ब्रह्मसूत्रस्य द्वितीयाध्याये "योगप्रत्युक्त्यधिकरणम्” (ब्रह्मसूत्रम्-२/१/२/१) इत्यनेन योगस्मृतिः निराक्रियते~। तत्र भाष्यकारः शङ्कराचार्योऽपि “द्वैतिनो हि ते सांख्याः योगाश्च” (२/१/२/३) इति योगदर्शनस्य निराकरणपर इव आभाति~। किन्त्वन्यत्र “उपनिषदुपायः सम्यग्दर्शनोपायो वा योग” इति वेदान्तिभिरेव महता आदरेण योगः समाद्रियते~। 

एवंप्रकारेण विशिष्टेऽस्मिन्नधिकरणे योगवेदान्तयोः केचन सदृशासदृशविचाराः संक्षेपेणसमालोचिताः सन्ति~। व्याख्याकारैरपि विरोधाभासत्वं सप्रमाणं निरूप्य व्यवहारकाले तयोः दर्शनयोरविरोधः प्रदर्श्यते~। शङ्कराचार्येण, व्याख्यात्रा वाचस्पतिमिश्रेण च पातञ्जलयोगदर्शनं स्तूयते यथा "योगशास्त्रेऽपि अथ तत्त्वदर्शनोपायो योगः इति सम्यग्दर्शनाभ्युपायत्वेनैव योगोऽङ्गीक्रियते","उपनिषदुपायस्य च तत्वज्ञानस्य योगापेक्षास्ति" इति (ब्रह्मसूत्रशाङ्करभाष्यम् २/१/२/३,भामती २/१/२/३)~। पुनश्च "सांख्ययोगौ हि परमपुरुषार्थसाधनत्वेन लोके प्रख्यातौ, शिष्टैश्च परिगृहीतौ, लिङ्गेन च श्रौतेनोपबृंहितौ" (ब्रह्मसूत्रशाङ्करभाष्यम् २/१/२/३) इति भाष्यवचनात् योगविधेः लोकप्रसिद्धिः, शिष्टपरिगृहीतत्वं चावगम्यते~। "आचाराच्च स्मृतिं ज्ञात्वा स्मृतेश्च श्रुतिकल्पनात्" इति न्यायेन शिष्टपरिगृहीतस्य योगस्य श्रुतिमूलकत्वमपि सम्भवति~। 

सूत्रकार-भाष्यकार-व्याख्याकारैरन्यत्र परिगृहीता योगस्मृतिः योगप्रत्युक्त्यधिकरणेन कुतो निराकृता, निराकृतापि सर्वांशेन उत एकांशेनेति संक्षेपेण विचार्यते~। 

ब्रह्मविद्यायाम् अन्तरङ्ग-बहिरङ्ग-साधनोपदेशो विशिष्टं स्थानं भजते~। अन्तरङ्ग-बहिरङ्ग-साधनतथ्यं कानिचन अधिकरणानि संक्षेपेणैव विवृण्वन्ति~। अतो योगप्रत्युक्त्यधिकरण-परीक्षणव्याजेन योगवेदान्तयोः विरोधाविरोध-विचारो विमर्शपदवीमानीयते~। तदङ्गतया\break वेदान्तविद्याया अन्तरङ्ग-बहिरङ्ग-साधनानुष्ठानकाले योगदर्शनस्य परमुपादेयत्वं, योगदर्शनस्य उपनिषदुपायत्वं चेति विषयाः यथामति उपस्थाप्यन्ते~। 

\section*{योगप्रत्युक्त्यधिकरणस्य संक्षेपार्थः}

ब्रह्मसूत्रस्य प्रथमाध्यायेन वेदान्तवाक्यानाम् अद्वितीयब्रह्मणि समन्वयः प्रदर्शितः~। वेदान्तानाम् अद्वितीयब्रह्मसमन्वये सति स्मृतिन्यायपूर्वकाक्षेपं समाधातुमविरोधाध्यायो नाम द्वितीयोऽध्यायः प्रवर्तते~। द्वितीयाध्यायगत-प्रथमपादेन स्मृतिपूर्वकाक्षेप-परिहारो विधीयते~। तत्र द्वितीयाध्यायस्य प्रथमपादे द्वितीयमधिकरणमस्ति योगप्रत्युक्त्यधिकरणम्~। 

\begin{description}
\item[विषयः] - योगस्मृतिः हिरण्यगर्भस्मृतिर्वा अस्याधिकरणस्य विषयः~। 
\item[संशयः] - वेदान्तवाक्यानाम् अद्वितीयब्रह्म-समन्वये किं योगशास्त्रस्य विरोधो वर्तते वा न वेति संशीतिः~। 
\item[पूर्वपक्षः] - अष्टाङ्गयोग-निरूपणपरो वैदिको तत्त्वदर्शनोपायो योगशास्त्रेण विधीयते~।\break प्रधानकारणवादपरत्वात् प्रमाणत्वेन लोके शिष्टैः परिगृहीतत्वाच्च योगशास्त्रं वेदान्तवाक्यानाम् अद्वितीय-ब्रह्मसमन्वयं न सहत इति पूर्वपक्षसंक्षेपः~। 
\item[सिद्धान्तः] - योगशास्त्रं वेदान्तानां ब्रह्मसमन्वयं न तात्पर्येण विरुध्यति~। वेदान्तेषु वैदिकमेव ज्ञानं ध्यानं च सांख्य-योगशब्दाभ्यां निरूप्यते~। जडप्रधानकारणवादे सत्यपि विरोधे योगशास्त्रस्य वेदान्तेन सह संवादो वरीवर्ति~। अतः पातञ्जलयोगस्मृतिः ब्रह्मवादं न विरुध्यति~। 
\item[सङ्गतिः] - ब्रह्मसूत्र-प्रथमाध्याये जडप्रधान-जगत्कारणत्ववादे श्रुतिविरोधः प्रदर्शितः~। सत्यपि श्रुतिविरोधे स्मृतयः तत्र प्रमाणमित्याक्षिप्य परिह्रियत इति आक्षेपसङ्गतिः~। 
\item[फलम्] - योगस्मृतिविरोधे सति अद्वितीय-ब्रह्म-समन्वयासिद्धिरेव पूर्वपक्षे फलम्~। सिद्धान्ते तु योग-स्मृतेरविरोधेन वेदान्तानां ब्रह्मसमन्वय एव फलम्~। 
\item[सूत्रार्थः] - एतेन योगः प्रत्युक्तः~। 
\item[एतेन] - सांख्यस्मृति-निराकरणेन, योगः- योगस्मृतिः हिरण्यगर्भस्मृतिर्वा, प्रत्युक्तः- निराकृतेति बोद्धव्या~। 
\end{description}

\section*{ब्रह्मसूत्रदिशा योगवेदान्तयोः विरोधाविरोधविचारः}

योगप्रत्युक्त्यधिकरणस्य "एतेन योगः प्रत्युक्त" इति सूत्रेण कृत्स्नस्य योगशास्त्रस्य प्रामाण्यं न बाध्यते~। किन्तु येनांशेन तत्त्वतः प्रमेयतो वा वेदान्तविरोधो योगशास्त्रस्य तन्मात्रेणैव तत्प्रत्युक्तिः~। तथा च स्वतन्त्र-प्रकृतिवादिनो हि योगशास्त्रविदः, किन्तु "मायां तु प्रकृतिं विद्यात् मायिनं तु महेश्वरम्" (श्वेताश्वतरोपनिषत् ४/१०) इति श्रुतेः "अविद्यात्मिका हि बीज-शक्तिरव्यक्त-शब्दनिर्देश्या परमेश्वराश्रया मायामयी महासुप्तिः यस्यां स्वरूप-प्रतिबोध-रहिता शेरते संसारिणो जीवाः" (ब्रह्मसूत्र-शाङ्करभाष्यम् १/४/३) इति भाष्यवाक्य-प्रमाणाच्च नाम\-रूपात्मक-प्रपञ्च-बीजभूत-प्रकृतेर्न सर्वथा स्वातन्त्र्यम्~। किन्तु ब्रह्मण एव निरवधिक-\break स्वातन्त्र्यमिति वेदान्तविदो वदन्ति~। जडस्वरूपाया अपि प्रकृतेर्जगन्निमित्तकारणत्वं मन्वते सांख्याः योगशास्त्रविदश्च~। "तस्य भासा सर्वमिदं विभाति" इति श्रुतेः चिन्मात्र-स्वरूपस्य स्वप्रकाशात्मन एव जगन्निमित्तत्वं तन्नियामकत्वमिति वेदान्तराद्धान्तः~। ईक्षत्यधिकरणस्मृत्यधिकरणादिभिः बहुभिः ब्रह्मसूत्राधिकरणैरपि (ब्रह्मसूत्रम्-१/१/५, २/१/१) सांख्यनिराकरणव्याजेन योगशास्त्रानुमतमपि जडप्रकृतेः जगन्निमित्तत्वं सयुक्ति निराक्रियते~। स्वरूपतः आत्मभेदाङ्गीकारेण पातञ्जलयोगदर्शनं द्वैतसिद्धान्तं पुरस्कुर्वच्छाङ्करवेदान्तं विवदते~। यद्यपि "नेतरोऽनुपपत्तेः, भेदव्यपदेशाच्चान्यः"(ब्रह्मसूत्रम् १/१/६/५, १/१/७/२) इत्यादिसूत्रैः जीवस्य सर्वज्ञत्व-सर्वशक्तत्वानुपपत्तेः, व्यवहारे जीवेशभेदमङ्गीकुर्वन्ति वेदान्तिनः~। सोऽयं भेदाङ्गीकारः न तात्त्विकः~। परं पातञ्जलदर्शने जीवेशभेदः पारमार्थिकाकारेणैव स्वीकृतः~। यथा "क्लेशकर्मविपाकाशयैरपरामृष्टः पुरुषविशेष ईश्वरः" (योगसूत्रम् १/२४) इति~। पुरुषप्रकृतिविवेकख्यात्या मोक्षमिच्छन्तः योगशास्त्रविदो नाम एकत्वदर्शिनः~। वेदोक्तादात्मैकत्व-विज्ञानादेव खलु मुक्तिरित्यद्वैतवेदान्तिनो वदन्ति~। किन्तु जीवेशभेदस्वरूपाङ्गीकारेण परमार्थतो द्वैतमेव साधयन्तो वेदान्तविरुद्धं मन्वते योगशास्त्रविदः, तदुक्तं शाङ्करब्रह्मसूत्रभाष्ये यथा "द्वैतिनो हि ते सांख्या योगाश्च न आत्मैकत्वदर्शिनः" (२/१/२/३) इति~। एवं प्रकारेण योगशास्त्रविदः व्यवहारे सांख्यदर्शनमेव प्राधान्येन अनुसरन्तः प्रकृतेः स्वातन्त्र्यम्, अचेतन\-प्रकृतेः जगन्निमित्तत्वं, चित्स्वरूपस्यात्मनः तात्त्विकाकारेण भोक्तृत्वं, जीवेशभेदं\break चाङ्गीचक्रुः~। अत एव भगवता बादरायणेन योगप्रत्युक्त्यधिकरणद्वारा योगशास्त्रस्य तत्त्वमीमांसायामेव विरोधः प्रदर्श्यते~। न तु तदुक्ताचारप्रक्रियायामिति बोद्धव्यम्~। 

योगशास्त्रप्रतिपादितानां स्वतन्त्राचेतनप्रकृतिवादादीनां निराकरणेन तत्प्रतिपादकस्य\break पातञ्जलयोगदर्शनस्यापि निराकरणं स्यादिति आशङ्क्येत~। तदाशङ्कापनयनपूर्वकं वाचस्पति\-मिश्रः योगवेदान्तयोरुभयोः समन्वयं प्रतिपादयति~। न ह्यचेतनप्रकृतेः जगत्कारणत्व\-साधनायां, परमार्थतः पुरुषस्य भोक्तृत्वप्रतिपादने वा योगदर्शनस्यापि तात्पर्यमस्ति~। यदि प्रधानस्य,\break तद्विकाराणां महदहङ्कारादीनां, पञ्चतन्मात्राणाञ्च निरूपणे एव योगशास्त्रस्य तात्पर्यं स्यात्, तर्हि तद्बाधने तत्प्रतिपादकस्य पातञ्जलयोगदर्शनस्यापि नूनमप्रामाण्यं स्यात्~। परन्तु योगदर्शनं न प्रधानादिसद्भावपरम्~। अतः प्रधानादिनिराकरणेन योगशास्त्रगतं प्रामाण्यं न हीयते~। एवं तर्हि योगदर्शनेन सांख्यानुमतो जडप्रकृतिकारणवादः किमुद्देश्येन प्रतिपादित इति विज्ञेयम्~। तथा च योगशास्त्रं तात्पर्येण योगस्वरूपनिरूपणोद्देश्येन प्रवृत्तं सत्तत्साधन-तत्फलव्युत्पादनपरम्~। \hbox{तद्व्युत्पादने} निमित्तापेक्षा वर्तते~। अतो योगस्वरूपस्य, तत्साधनस्य, तत्फलस्य च\break व्युत्पिपादयिषया पतञ्जलिना निमित्तमात्रेण प्रधानादयः प्रतिपादिताः न तु तात्पर्येणेति वाचस्पतिमिश्रस्याकूतम्~। तदुक्तं वाचस्पतिना "नानेन योगशास्त्रस्य हैरण्यगर्भपातञ्जलादेः सर्वथा\break प्रामाण्यं निराक्रियते,किन्तु जगदुपादानस्वतन्त्रप्रधान-तद्विकारमहदङ्कारपञ्चतन्मात्रगोचरं\break प्रामाण्यं नास्तीत्युच्यते (भामती २/१/२/३)~। यथा ब्रह्मनिरूपणाय वेदान्तेषु या सृष्टि\-प्रक्रिया अभिधीयते न तु तत्तात्पर्येण~। तथैव योगशास्त्रेऽपि योगस्वरूपं व्युत्पादयितुं प्रधाना\-दयो निरूप्यन्ते~। एवं "योगप्रत्युक्त्यधिकरणेन"सूत्रकारबादरायणकृतस्य तात्विकविरोधस्याप्यविरोधः प्रदर्शितः वाचस्पतिमिश्रेण~। 

एवमेव शङ्कराचार्येण प्रतिपादिततात्त्विकविरोधमपि विरुध्य वाचस्पतिमिश्रः क्वचित्तत्त्वाविरोधं संपादयति~। भगवता भाष्यकारेण शङ्कराचार्येण साक्षादेव योगशास्त्रस्य तत्त्वमीमांसा निराक्रियते~। तदुक्तं-"तत्रापि श्रुतिविरोधेन प्रधानं स्वतन्त्रमेव कारणं, महदादीनि च\break कार्याण्यलोकवेदप्रसिद्धानि"(ब्रह्मसूत्रशाङ्करभाष्यम् २/१/२/३),“द्वैतिनो हि ते सांख्या\break योगाश्च नात्मैकत्वदर्शिनः”(ब्रह्मसूत्रशाङ्करभाष्यम् २/१/३) इति~। 

विषयेऽस्मिन् प्रसङ्गान्तरे वाचस्पतिमिश्रः भाष्यकारसम्मतं योगवेदान्ततात्त्विकविरोधमप्यविरोधरूपेण व्याख्याति~। यथा “चेतनाधिष्ठितमचेतनं प्रवर्तते यथा योगिनामीश्वरवादिनाम्”(भामती २/२/२)इति~। एवमेव योगवेदान्तयोः समन्वय\-मिच्छता वाचस्पतिमिश्रेण योगसूत्रव्याख्याने चेतनाधिष्ठितेनैव प्रधानेन सृष्टिरथवा ईश्वरस्यैव सृष्टिलयकर्तृत्वमिति\break निरूपितं यथा- “अनादौ तु सर्गसंहारप्रबन्धे सर्गान्तरसमुत्पन्न\-सञ्जिहीर्षाऽवधिसमये पूर्णे मया सत्वप्रकर्ष उपादेय इति प्रणिधानं कृत्वा भगवान् जगत्सञ्जहार” (तत्त्ववैशारदी-१/२४)~।\break  ए~वञ्च योगप्रत्युक्त्यधिकरणेन योगशास्त्रानुमतस्य प्रधानकारणवादस्यैव निराकरणं क्रियते इति निष्कर्षः~। योगदर्शनम् अष्टाङ्गयोगमुखेन वेदान्त\-विद्यासाधनमित्यत्र नास्ति कस्यापि\break  संशयलेशोऽपि, तदुक्तं भाष्यकारेण यथा "योगशास्त्रेऽपि सम्यग्दर्शनोपायत्वेनयोगोऽङ्गीक्रियते" (ब्रह्मसूत्रशाङ्करभाष्यम् २/१/२/३) इति~। 

अथेदानीं पातञ्जलयोगदर्शनप्रसिद्धस्य अष्टाङ्गयोगस्य वेदान्तदर्शनप्रसिद्धस्य अन्तरङ्ग\-बहिरङ्गसाधनस्य च कीदृशः सम्बन्ध इति तुलनात्मकदृष्ट्या चिन्त्यते~। "शान्तो दान्त\break  उपरतः तितिक्षुः" (बृहदारण्यकम् ४/४ ) इत्यादिना निवृत्तिप्रधानतया विहितं शमादिषट्कमेव पतञ्जलिः प्रकारान्तरेण पञ्चधा विभज्य दर्शयति यथा-"अहिंसा-सत्य-अस्तेय-ब्रह्मचर्य-अपरिग्रहा यमाः" (योगसूत्रम् २/३०) इति~। यद्यपि ब्रह्मसूत्रेण साक्षादेव \hbox{अहिंसादयो} न विचार्यन्ते, तथापि शमादयो विमृश्यन्त एव~। सर्वापेक्षाधिकरणेन(ब्रह्म.सू ३/४/६/२७) "शमादयः साक्षात्कारपर्यन्तं पालनीयाः, साधकस्य विक्षेपनिवृत्तिद्वारा ते ज्ञानोत्पत्तये \hbox{हेतवः"} इति अभिहितम्~। आधिकारिकाधिकरणेन (ब्रह्म.सू ३/४/११) च ब्रह्मचर्यपालनं स्तूयते~। 

यज्ञादीनि नित्यकर्माणि,जपोपवासादयश्च ब्रह्मविद्यायां बहिरङ्गसाधनानि भवन्ति~। तानि प्रवृत्तिरूपाणीत्यतो नियमत्वेन पतञ्जलिना विहितानि यथा "शौचसन्तोषतपःस्वाध्यायेश्वरप्रणिधानानि नियमाः"(योगसूत्रम् २/३२ ) इति~। नियम्यते कर्तव्यतया बुद्धौ निश्चीयते इति नियमः~। यद्यपि योगशास्त्रोक्तक्रमेण ब्रह्मसूत्रेषु नियमाः नैव विहिताः~। तथापि बहुभिरधिकरणैः विहितानि साधनानि नियमरूपाणि भवन्ति~। यथा सर्वापेक्षाधिकरणेन (ब्रह्मसूत्रम् ३/४/६) यज्ञादीनां नित्यकर्मणां, विधुराधिकरणेन (ब्रह्मसूत्रम् ३/४/९) जपोपवासदेवताराधनादीनां विहितकर्मणां च विद्योत्पत्तौ सहकारिकारणत्वं प्रसाध्यते~। 

“त्रिरुन्नतं स्थाप्य समं शरीरम् (श्वेताश्वतरोपनिषत् २/८), शुचौ देशे प्रतिष्ठाप्य\break स्थिरमासनमात्मनः, उपविश्यासने युञ्ज्यात्, समं कायशिरोग्रीवम्”(भगवद्गीता ६/११-१३) इति श्रुतिस्मृतिसिद्धमेव आसनं विहितं योगसूत्रकारेण "स्थिरसुखमासनम्”(योगसूत्रम्२/४६) इति~। आसनं नाम निश्चलं सुखकरं च भवेत्~। उपासनाङ्गतया "आसीनः सम्भवात्" (ब्रह्म\-सूत्रम् ४/१/६ )इत्यासनम् उपदिशति भगवान्बादरायणः~। निदिध्यासनं प्रति निश्चलं\break समाहितं च मनः कारणमित्युच्यते~। आसनप्राणायामं विना ध्येयवस्तुनि मनःसमाधानासम्भवात् आसनप्राणायामप्रत्याहारतत्त्वं शास्त्रतः गुरूपदेशतश्च विज्ञेयं ब्रह्मविद्योपासकैः~। ब्रह्मसूत्रभाष्यकारोऽपि आसीनाधिकरणस्य "स्मरन्ति च"(ब्रह्मसूत्रम् ४/१/६/४) इत्यस्मिन्सूत्रे, "शुचौ देशे प्रतिष्ठाप्य स्थिरमासनमात्मनः"(६/११) इति गीतावचनमात्रमुदाहृत्य ततोऽधिकं योगशास्त्रे प्रसिद्धमित्युक्त्वा विरमति~। अतः कानि तान्यासनानि पद्म- भद्र -वीर -स्वस्तिकादीनि , कथं वा अनुष्ठेयानीत्यादयः संशयाः योगशास्त्रेण एव निवर्त्यन्ते~। 

अध्यात्मविद्यायां साधनचतुष्टयसहितश्रवणादिकमेव अन्तरङ्गसाधनम्~। एवं योगशास्त्रेऽपि अन्तरङ्गसाधनानि धारणाध्यानसमाधयो भवन्ति~। वेदान्तशास्त्रे उपासनं निदिध्यासनं ध्यानं वा धारणाध्यानसमाधिसाधारणम्~। बाह्याभ्यन्तरविषयेषु चित्तस्य बन्धरूपा धारणा एव\break उपासनस्यादिमं सोपानम्~। यथोक्तं"देशबन्धश्चित्तस्य धारणा" (योगसूत्रम् ३/१) इति~। धारणायां सिद्धायां समानप्रत्ययानाम् एकतानतासम्पादनमेव ध्यानम्~। तदुक्तं “तत्र प्रत्ययैकतानता ध्यानम्” (योगसूत्रम् ३/२)इति~। अस्यैव ध्यानस्योपासनस्य वा चरमोत्कर्षः\break समाधिः~। समाध्यवस्थायां चित्तं ध्येयस्वरूपं भवति~। प्रत्ययाकारकं वृत्यन्तरं नोदेति\break इत्यर्थः~। एतदेव सूत्रितं यथा “तदेवार्थमात्रनिर्भासं स्वरूपशून्यमिव समाधिः”(योगसूत्रम् ३/३) इति~। ध्यातृध्येयध्यानत्रिपुटीलयः एव समाधिशब्दवाच्यः~। एवञ्च ध्यानं निदिध्यासनं वा उत्तरोत्तरोत्कर्षं मनसि निधाय धारणाध्यानसमाधिभेदः प्रदर्श्यते योगसूत्रकारेण~~। अतो\break वैदिकोऽयमष्टाङ्गयोगो वेदान्तसाधकानां प्रकृष्टोपकारकत्वेन उपेयब्रह्मविद्यायां पुष्कलोपायभावं भजते~। 

\section*{वेदान्तदिशा अन्तरङ्गबहिरङ्गसाधनविचारः}

सनातनवेदान्तदर्शने अन्तरङ्गबहिरङ्गसाधनोपदेशः सोपानारोहणक्रमेण समुपदिश्यते~। वस्तुतस्तु ऋषीणां साक्षात्कृतात्मतत्त्वानां च वेदान्तोपदेशो गहनो गभीरश्च~। मन्दाधिकारिभ्यः साधकेभ्यः तेषां वेदान्तोपदेशस्यावबोधः क्वचिद्दुष्करः स्यात्~। वेदान्तेषु अधिकारिभेदेन \break अन्तरङ्ग-बहिरङ्गसाधनानां नियतक्रमोपदेशो न विहितः~। एवं सति वेदान्तदर्शने यान्यन्तरङ्गबहिरङ्गसाधनानि विहितानि तान्येव पतञ्जलिमहर्षिणा विलक्षणतया विधीयन्ते यथा- "यम-नियमासन-प्राणायाम-प्रत्याहार-धारणाध्यान- समाधयोऽष्टावङ्गानि" (योगसूत्रम् २/२९) इति~। वेदान्तोपदर्शितानामेव अन्तरङ्ग-बहिरङ्ग- साक्षात्कार-साधनानां नियतक्रमोपदेश एव अयमष्टाङ्गयोगः~। साधकोपयोगितया नियतक्रमेण वेदान्तोपायानि योगशास्त्रे भगवान्पतञ्जलिरनुशास्ति~~। अनुष्ठानक्रममाश्रित्य व्यक्तिस्तरे योगसाधनानां योऽयं नियतक्रमोपदेश नान्येषु भारतीयदर्शनसूत्रेषु संदृश्यते~। 

अथेदानीं वेदान्तोपदिष्टं योगशास्त्रसम्मतम् अन्तरङ्गबहिरङ्गसाक्षात्कारसाधनजातं\break संक्षेपेण विचार्यते~। यद्यपि वेदान्तवाक्यमेव साक्षादिह साक्षात्कारसाधनमिति परमार्थः\break ज्ञानस्य प्रमाणफलत्वात्~। “तं त्वौपनिषदं पुरुषं”(बृहदारण्यकम् ३.९.२६), “वेदान्तविज्ञान\-सुनिश्चितार्थाः” (मुण्डकोपनिषत् ३.२.६) इत्यादिश्रुतेश्च~। तथापि संप्रदायप्रसिद्धमन्यदपि\break प्रसङ्गसङ्गत्या चिन्त्यते~। 

यत् विप्रकृष्टं सदुपकारकं तत् बहिरङ्गसाधनमितिव्यवहारः~। यथा धनुः लक्ष्यवेधने\break बहिष्ठमेव सदुपकारकमिति बहिरङ्गसाधनम्~। तथा नित्यनैमित्तिककर्माणि यज्ञादीनि श्रौतस्मार्तकर्माणि विक्षेपादिदोषनिराकरणेन चित्तशुद्धिं चित्तैकाग्र्यं च विधाय जिज्ञासोत्पत्तिद्वारा साक्षात्कारे बहिरङ्गसाधनानीत्युच्यन्ते आरादुपकारकत्वात्~। तदुक्तं-“तमेतमात्मानं वेदानु\-वचनेन ब्राह्मणा विविदिषन्ति यज्ञेन दानेन तपसा अनाशकेन”(बृहदारण्यकम् ४.४.२२), आरुरुक्षोर्मुनेर्योगं कर्म कारणमुच्यते~। योगारूढस्य तस्यैव शमः कारणमुच्यते॥(गीता.६.३)~। निष्कामबुध्या अनुष्ठीयमानानि यज्ञदानतपांसि साक्षात्कारं प्रति बहिरङ्गसाधनानि भवन्ति~। तानि उत्कटजिज्ञासोदयपर्यन्तमनुवर्तनीयानि एव~। उत्कटजिज्ञासोदये सति त्यक्तबहिरङ्गसाधनोऽन्तरङ्गसाधनानि अनुतिष्ठेदिति वेदान्तसंप्रदायः~। 

अथ च सन्निकृष्टं सदुपकारकम् अन्तरङ्गसाधनमित्युच्यते~। यथा धनुषो मुक्तो बाणो\break लक्ष्यवेधनेऽन्तरङ्गं भवति तथा विवेकादिचतुष्टयं, श्रवणमनननिदिध्यासनं च ज्ञानस्यान्त\-रङ्गसाधनमिति स्थितिः~। तत्रापि विवेकादि–साधनचतुष्टयसंपन्नस्यैव वेदान्तवाक्यश्रवणादावधिकार इत्यतो विवेकादिचतुष्टयं श्रवणादौ प्रयोजकम्~। साधको वेदान्तवाक्यार्थस्य\break श्रवणमनननिदिध्यासनकालेऽपि नियमेन विवेकादिसाधन- चतुष्टयमभ्यसेत्~। यदि\break श्रवणादौ प्रवृत्तस्य पूर्ववासनाबलात् विषयेषु प्रवृत्तिः स्यात्, तर्हि तस्य श्रवणादिसिद्धिः नैव भवेत्~। अत एव श्रवणमनननिदिध्यासनेषु सिद्धिकामो विवेकादिसाधनचतुष्टयं श्रवणादिभिः सह साक्षात्कारपर्यन्तमभ्यसेत्~। विवेकादि– साधनचतुष्टयं मुमुक्षुणा नैव यज्ञदानवद्धातव्यमिति हेतोः नित्यानित्यवस्तु- विवेकादि-साधनचतुष्टयस्य ज्ञानं प्रत्यन्तरङ्गसाधनत्वमङ्गीक्रियते~। 

\section*{विवेकादि-साधनचतुष्टयसंपत्तिः}

“शान्तो दान्त उपरतः तितिक्षुः” (बृहदारण्यकम् ४.४.२३) इत्यादिना वेदान्तेषु\break साक्षात्कारं प्रत्यन्तरङ्गसाधनत्वेन विवेकादिसाधनचतुष्टयं विधीयते~। यज्ञदानादीनि निष्काम\-कर्माणि बहिरङ्गसाधनानि चित्तशुद्धिद्वारा विवेक-वैराग्यादिसाधनचतुष्टयं भावयन्ति~। तत्र\break बृहदारण्यकश्रुतेः पाठक्रममाश्रित्य भाष्यकारशङ्कराचार्याः साधनचतुष्टयस्य नियतक्रमं  दर्शयन्ति यथा- "नित्यानित्यवस्तुविवेकः, इहामुत्रार्थफलभोगविरागः, शमदमादिसाधनसम्पत्, मुमुक्षुत्वं" (ब्रह्मसूत्रशाङ्करभाष्यम् १/१/१) चेति~। तत्र साधनचतुष्टये उत्तरोत्तरं प्रति\break पूर्वस्य कारणत्वमिति बोद्धव्यम्~। नित्यानित्यवस्तुविवेकः वैराग्यहेतुः, वैराग्यं शमादिसिद्धिहेतुः, शमादिना मुमुक्षुत्वमिति च कारणादेव साधन-चतुष्टय-संपत्तौ कश्चिद्विशिष्टक्रमः समाश्रीयते~। 

\subsection*{1. नित्यानित्यवस्तुविवेकः}

देशकालनिमित्तैरपरिच्छिन्नं चिन्मात्रं नित्यं, नामरूपात्मकं जगदिदं परिच्छिन्नं सत्\break अनित्यमिति यः परोक्षानुभवः स एव नित्यानित्यानित्यवस्तुविवेकः~। तत्रायं विशेषो यत् अशुद्धान्तःकरणानामीदृशे विवेके सत्यपि वैराग्यादिकं नोदेति~। शुद्धान्तःकरणानामेव नित्यानित्यात्मकपरोक्षानुभवो वैराग्यहेतुः~। अतो निष्कामकर्मादिना बहिरङ्गसाधनेन चित्तशुद्धिः \-नियमेन कार्या~। अनुष्ठितबहिरङ्गसाधनस्यैव नित्यानित्यवस्तुविवेकाधिकार इति गूढार्थः~। 

\subsection*{2. इहामुत्रार्थफलभोगविरागः}

दोषदृष्ट्या विषयभोगेषु अनादर इत्यर्थः~। आब्रह्मलोकात् विपरिवर्तमानेषु भोगेषु अनादररूपा उपेक्षैव वैराग्यम्~। "तद्यथेह कर्मजितो लोकः क्षीयते एवमेव अमुत्र पुण्यजितो लोकः क्षीयते"(छान्दोग्यम् ८.१.९), "परीक्ष्य लोकान्कर्मचितान् ब्राह्मणो निर्वेदमायात् नास्त्यकृतः कृतेन" (मुण्डकोपनिषत् - १.२.१२) इत्यादिश्रुतिप्रतिपादितो विरागभाव एव शमदमादि उत्पत्यनुकूलतया साक्षात्कारसोपानमिति ज्ञायते~। 

\subsection*{3. शमदमादिसाधनसम्पत्}

शमादयस्तावत् "शम-दमोपरति-तितिक्षा-समाधान-श्रद्धाख्याः "भवन्ति~। शम-दमादीनां क्रमः बृहदारण्यकस्य काण्वपाठे निर्दिष्टः~~। यथा “शान्तो दान्त उपरतः तितिक्षुः समाहितो भूत्वा आत्मन्येवात्मानं पश्यति”( ४.४.२३) इति~। बृहदारण्यकस्य काण्व-माध्यन्दिनोभयपाठानुगुणतया शमदमादौ श्रद्धासमाधानयोरङ्गीकार इति ज्ञेयम्~। 

\begin{enumerate}	
\itemsep=2pt
\item शमः- विषयदोषदृष्टा 	विषयेभ्यो 	मनसः 	प्रतिनिवर्तनम्~। 
\item दमः-विषयदोषदृष्टा 	विषयेभ्यः 	बाह्येन्द्रियाणां 	प्रतिनिवर्तनम्~। 
\item उपरतिः-ऐहिकामुष्मिकभोगहेतूनां 	लौकिकवैदिककर्मणां, 	विषयाणां 	च   परित्यागः~। शमदमयोः 	परिपाकावस्था 	एव 	उपरतिरित्यपि 	वदन्ति केचित्~। 
\item तितिक्षा-शीतोष्णादिद्वन्द्वसहनम्~। 
\item समाधानम्- लक्ष्ये ब्रह्मणि चित्तैकाग्र्यम्~। सगुणोपासनभाव्यं चित्तैकाग्र्यं 	 सगुणब्रह्मणि मनोनिवेशनहेतुः किन्तु 	इह निर्गुणब्रह्मण्यनवरतस्थितिमुद्दिश्य 	चित्तैकाग्र्यमुच्यते~। 
\item श्रद्धा-आस्तिक्यं गुरुवेदान्तवाक्येषु विश्वासो वा~। 
\end{enumerate}

\textbf{अत्रेदं चिन्त्यं}-शमादीनां षण्णामपि साधनानां परस्परसापेक्षत्वात्,मिलितानि तानि एकं साधनम्~। उत्तरोत्तराणां दमादीनां निदानं शम इत्यतः तेषु शमस्यैव मुख्यत्वं बेध्यम्~। यः अनिगृहीतमना बहिर्विषयेषु आसक्तः स इन्द्रियाणि जेतुं न शक्नोति~। न काम्यकर्माणि सन्यसति~। क्षुत्पिपासादिद्वन्द्वं सोढुं न क्षमते~। ब्रह्मण्येव मनःसमाधानं न ईष्टे~। न गुरुवेदान्तवाक्येषु श्रद्धा भवति तस्य~। अतः “शान्तो दान्त उपरतः" इति श्रुतिः शमस्यैव मुख्यत्वं दमादिहेतुत्वं च दर्शयति~। 

\subsection*{4. मुमुक्षुत्वम् } 

ब्रह्मावाप्तिः अनर्थनिवृत्तिश्च 	मोक्षः, तस्मिन् तीव्रतरेच्छा 	एव श्रवणादिप्रवृत्तौ प्रयोजकम्~।  सामान्याकारकेच्छा न श्रवणप्रयोजकम्~। 

\section*{श्रवणादेरन्तरङ्गसाधनत्वविचरः}

\subsection*{1. श्रवणम्} 

“आत्मा वा अरे द्रष्टव्यः श्रोतव्य” (बृहदारण्यकम् - २/४/५) इति श्रुतेः साक्षात्कारं प्रति श्रवणादेरन्तरङ्ग-साधनत्वमवगम्यते~~। तत्र प्रत्यगभिन्नब्रह्मणि तात्पर्यनिर्णयानुकूल- चेतोवृत्तिविशेष एव श्रवणम्~। तात्पर्यनिर्णयश्च उपक्रमोपसंहारादिभिः षड्विधतात्पर्यलिङ्गैः सम्भवतीति वेदान्तमर्यादा~। 

\subsection*{2. मननम् }

ऐक्यसाधकयुक्तिभिः, भेदबाधकयुक्तिभिश्च श्रुतार्थे प्रत्यगभिन्नब्रह्मणि एव अनुसन्धानम्~। 

\subsection*{3.निदिध्यासनम् }

अनात्म-प्रत्ययानन्तरित-ब्रह्माकारकप्रत्ययप्रवाह एव निदिध्यासनम्~। साक्षात्कारपर्यन्तमिदं निदिध्यासनम् अभ्यसनीयम्~। “अहं ब्रह्मास्मि” इति स्वप्रयत्नसाध्या ब्रह्माकारा वृत्तिः निदिध्यासनम्~। स्वप्रयत्नं विनैव “अहं ब्रह्मास्मि” इति वेदान्तप्रमाणजन्य-ब्रह्माकारा वृत्तिरेव साक्षात्कार इत्येवं निदिध्यासन-साक्षात्कारयोः भेदः~। 

निदिध्यासनस्यैव परिपक्वावस्था योगशास्त्रप्रसिद्धसमाधिः, “ते ध्यानयोगानुगता अपश्यन् देवात्मशक्तिम्” (श्वेताश्वतरोपनिषत् १/३) इति श्रुतेः~। तथा हि निदिध्यासनस्यादिमावास्था सविकल्पकसमाधिः,यत्र ध्यातृध्यानध्येयाकारा त्रिपुटी भासते~। तस्यैव ध्यातृ-ध्यान-ध्येयरूपत्रिपुटीभानरहिता परिपाकावस्था निर्विकल्पकसमाधिरिति विवेकः~। तदुक्तं - 

\begin{verse}
ताभ्यां निर्विचिकित्सेऽर्थे चेतसः स्थापितस्य यत्~। \\
एकतानत्वमेतद्धि निदिध्यासनमुच्यते~॥ \\
ध्यातृध्याने परित्यज्य क्रमात् ध्येयैकगोचरम्~। \\ 
निवातदीपवच्चित्तं समाधिरभिधीयते~॥ (पञ्चदशी १/५४-५५)
\end{verse}

\section*{श्रवणादीनां प्रयोजनम्}

ज्ञानप्रतिबन्धकीभूतयोः संशयविपर्यययोः निरसनमेव श्रवणादीनां प्रयोजनम्~। प्रमाणगतसंशयनिवृत्तिः श्रवणेन, प्रमेयगतसंशयनिवृत्तिः मननेन, विपर्ययनिवृत्तिश्च निदिध्यासनेन भवतीति परिनिष्ठितार्थः~। 

\section*{उपसंहारः}

सनातनदर्शनपरम्परायां योगवेदान्तदर्शने महनीयं स्थानं प्राप्नुतः~। "अथातो ब्रह्मजिज्ञासा" इत्यादिना बादरायणीयवेदान्तदर्शनं पुरुषं प्रति "प्रत्यगभिन्नब्रह्मविचारः कर्तव्य" इत्युपदिशति~।"अथ योगानुशासनम्" इत्यादिना पातञ्जलयोगदर्शनं मुमुक्षुं योग्याधिकारिणं प्रति \hbox{साक्षात्कारोपायानि} अनुशास्ति~। एवञ्च अनुशासनेन उपदेशेन च योगवेदान्तदर्शने पुरुषं स्वरूपावस्थानात्मकं मोक्षं गमयतः~। तत्र वेदान्तदर्शनेन उपेयार्थसिद्धिः, योगेन चोपाय\-सिद्धिरिति योगवेदान्तयोरुपायोपेयभावो वरीवर्तीति शिवम्~। 

\articleend
}
