{\fontsize{15}{17}\selectfont
\presetvalues
\chapter{यज्ञपत्युपाध्यायमतेन सामान्याभावस्य अतिरिक्तत्वसाधनम्}

\begin{center}
\Authorline{वि~॥ राघवः के.एल्.}
\smallskip

योजनसहायकः\\
कर्नाटकसंस्कृतविश्वविद्यालयः\\
चामराजपेटे, बेङ्गलूरु
\addrule
\end{center}

व्याप्तिसिद्धान्तलक्षणनिरूपणानन्तरं सामान्याभावप्रकरणे मणिकारः  सामान्याभावस्य\break विशेषाभावकूटातिरिक्तत्वं साधयति~। अत्रेदं मणिवाक्यम् ‘ अन्यनिष्ठवह्नेः धूमवत्पर्वतवृत्यत्यन्ताभावप्रतियोगित्वेऽपि तत्प्रतियोगिता न वह्नित्वेनावच्छिद्यते, धूमवति वह्निर्नास्तीति\break प्रतीतेः’ इति~। ‘पर्वते महानसीय वह्निर्नास्ति इति प्रतीत्या धूमवति पर्वते विद्यमानाभावप्रतियोगित्वं महानसीय वह्नेः~। तत्प्रतियोगितावच्छेदकमेव भवति वह्नित्वं चेत् पुनः उक्तव्याप्तिलक्षणस्य वह्निमान् धूमात् इत्यत्रैवाव्याप्तिः~। किन्तु उक्ताभावप्रतियोगितायाः महानसीयवह्नौ सत्वेऽपि प्रतियोगितावच्छेदकं न वह्नित्वम्~। विपक्षे बाधकमुक्तम् - यदि तादृशाभावप्रतियोगितावच्छेदकं वह्नित्वं स्यात् तर्हि धूमवत्यादि वह्निविशेषाभावस्य सत्वेन तत्रापि वह्निर्नास्ति इति वह्निसामान्याभावप्रतीतिः स्यात्~। न भवति च तथेति वह्नित्वस्य तादृशाभावप्रतियोगितावच्छेदकत्वम् अपि नास्तीति नाव्याप्त्यवकाशः~।  

कुतः वह्नित्वस्य न प्रतियोगितावच्छेदकत्वम् इति चेत् सामान्यधर्मावच्छिन्नाभावः अन्यः, विशेषधर्मावच्छिन्नश्च अन्यः विशेषाभावकूटः~। अत्र संशयन्यथानुपपत्तिः प्रमाणितः~। तथा हि यदि विशेषाभावकूटः एव सामान्याभावः इत्युच्यते तर्हि वायौ यावद्रूप विशेषाभावनिश्चये सति वायुः रूपवान् न वा इति संशयो न स्यात्~। रूपयावद्विशेषाभावकूटवत्वस्य तत्र निश्चितत्वात्, तस्य च रूपसामान्याभावरूपत्वेन सत्संशयप्रतिबन्धकत्वात्~। किन्तु भवति च तादृशसंशयः इति तदुपपत्तये सामान्याभावस्य पार्थक्याभ्युपेयम्~। तथा सति विशेषाभावकूटनिश्चयस्य सामान्यसंशयाप्रतिबन्धकत्वात् तत्संशयोपपत्तिः~। 

अत्र प्रभाकारः यज्ञपत्युपाध्यायः सामान्याभावस्य पार्थक्ये बीजं प्रदर्शयन् ग्रन्थमवतारयति नन्वेवम् इत्यादिना~। नन्विति शङ्काया अयमाशयः- यदि सामान्यधर्मः विशेषाभाव\-प्रतियोगितावच्छेदकः न भवति चेत् सः प्रतियोगितावच्छेदकः एव न भवति~।  यतः\break सामान्याभावः विशेषाभावकूटातिरिक्तः न इत्यङ्गीकारे तस्य तदनतिरेकात् यदि सामान्यधर्मः विशेषाभावप्रतियोगितावच्छेदकः न, तदा सामान्याभावप्रतियोगितावच्छेदकोऽपि न भवतीति सामान्यधर्मे अभावप्रतियोगितावच्छेदकत्वमेव न स्यात्~। तथा सति\break उक्तलक्षणस्य धूमवान् वह्नेरित्यादौ अतिव्याप्तिः~। तथा हि- हेतुसमानाधिकरणात्यन्ताभावः धूमविशेषाभावः~। तादृशाभावकूटः न धूमत्वसामान्यधर्मावच्छिन्नः, सामान्यधर्मस्य विशेषा\-भावप्रतियोगिता-ऽनवच्छेदकत्वात्~। एवञ्च हेतुसमानाधिकरणात्यन्ताभावप्रतियोगिता\-\break नवच्छेदकं च साध्यतावच्छेदकं धूमत्वम्, तदवच्छिन्नसामानाधिकरण्यं वह्नेः महानसादौ\break वर्तते इति लक्षणगमनात् अतिव्याप्तिः~। अतः सामान्यधर्मावच्छिन्नाभावस्य पार्थक्यमावश्य\-कम्~। तथा च वह्निसमानाधिकरणः धूमविशेषाभावकूटः अन्यः, धूमत्वावच्छिन्नाभावः\break अन्यः~। वह्न्यधिकरणे अयोगोलके ’धूमो नास्त” इति प्रतीत्या धूमसामान्याभावप्रतियोगितावच्छेदकमेव भवति साध्यतावच्छेदकम् धूमत्वं इति साध्यतावच्छेदके प्रतियोगितावच्छेदकभेदाभावात् नातिव्याप्तिः~। एवं च सामान्यधर्मस्य अभावप्रतियोगितावच्छेदकत्वनिर्वाहार्थमपि सामान्याभावस्या अतिरिक्तत्वमावश्यकम् इत्येवं साकूतः ’नन्वेवम् इत्यादिः\break सामान्यावच्छिन्नेति”(१) इत्यन्तः ग्रन्थः  ~। माथुर्यामपि ’ननु विशेषाभावानां ……\break सामान्यरूपमन्यथा अभावप्रतियोगितावच्छेदकमेव तन्न स्यात्’ इत्येवम् इयमेवाशङ्कोद्भाविता~। 

ततश्च संशयान्यथानुपपत्तिः प्रदर्शिता - रूपसामान्याभावरूपविशेषाभावकूटयोः\break अभिन्नत्वे वायौ रूपविशेषयावदभावकूटनिश्चये सति तत्सामान्यसंशयो न स्यात्, रूपसामान्या\-भावरूपतद्विशेषयावदभावकूटस्य निश्चितत्वेन संशयविषयासम्भवात्, तदभावनिश्चयस्य\break प्रतिबन्धकत्वाच्च~। भवति च तथेति तयोः पार्थक्यमपि सम्भवत्येव~। 

\section*{संशयोपपादनाशङ्का}

ननु वायौ रूपसामान्याभावरूपस्य प्रसिद्धरूपविशेषाभावकूटस्य निश्चितत्वेऽपि, तादृशविशेषाभाव एव रूपसामान्याभावत्वेन रूपेण भासते~। तत्र विशेषाभावकूटस्य निश्चितत्वेऽपि सामान्याभावस्यानिश्चितत्वात् वायौ रूपं न वा इति प्रतीतौ रूपसामान्याभावत्वेन अभावस्य भानमिति चेन्न~। किं रूपाभावत्वं तत्र अनिश्चितम् ? तच्च रूपत्वविशिष्टधर्मिप्रतियोगिकाभावत्वम् वा ? रूपसामान्याभावरूपत्वं वा ? नाद्यः, रूपविशेषाभावस्य तत्र निश्चितत्वेन, तत्प्रतियोगिनि रूपत्वस्य निश्चितत्वात्~। अभावकूटकथनात् प्रतियोगिवर्गे इत्युक्तं, प्रतियोगिगणे इति यावत्~। एवं च प्रतियोगिनि रूपत्वस्य निश्चितत्वेन रूपत्वविशिष्टप्रतियोगिकाभावत्वमपि निश्चितमिति रूपाभावत्वेन नाभावस्य संशयविषयत्वम्~। नापि द्वितीयः – रूपविशेषाभावकूटस्यैव रूपसामान्याभावरूपत्वेन तस्य च तत्र निश्चितत्वात् रूपसामान्याभावत्वमपि निश्चितमेव इति तेन रूपेणापि न संशयविषयता इत्युभयथापि शङ्कानवकाशः~। 

पुनः सामान्याभावस्य अतिरिक्तत्वेऽपि संशयानुपपत्तिं प्रदर्श्य अनतिरिक्तत्वं शङ्कते पूर्वपक्षी - अथ सामान्याभावः अतिरिक्तः एव भवतु~। तथापि अन्यत्र कुत्रचित् सन्दिग्धरूपादेः वायौ वृत्तित्वसंशयः भवेदेव~। अतः तत्र वायौ तादृशरूपाभावस्य अनिश्चितत्वात् स एव ‘ रूपं न वा’ इति संशयविषयः भवति~। सामान्याभावस्यातिरिक्तत्वपक्षेऽपि रूपवत्वसंशयं प्रति रूपसामान्याभावत्वप्रकारकनिश्चयः प्रतिबन्धकः वाच्यः~। यदि रूपसामान्याभावत्वप्रकारकनिश्चयः स्यात् तर्हि अन्यत्रप्रसिद्धस्य रूपादेः वायौ संशयो न भवेत् प्रतिबन्धकस्य सत्वात्~। किन्तु भवति च तत्संशयः इति तत्र विशेषाभावकूटनिश्चयेऽपि रूपाभावत्वेन  न तन्निश्चितम्, स एव संशयविषयः इति चेन्न~। उक्तसंशयविरहदशायामपि नाम अन्यत्रप्रसिद्धस्य रूपस्य सन्दिग्धत्वविरहदशायामपि वायौ रूपं न वेति संशयः सम्भवत्येवेति, रूपसामान्याभावः एव संशयविषयः, न तु अन्यत्रप्रसिद्धरूपादिः~। 

\section*{रूपस्य संशयाविषयत्वशङ्का}   

न च यदि ‘वायौ रूपं न वा’ इति संशयविषयता यदि प्रसिद्धरूपविशेषाभावे नास्ति, तर्हि रूपविशेषेऽपि तद्विषयता न स्यात्~। रूपविशेषाभावस्य तत्र निश्चितत्वात् स यदि संशयविषयः न भवति तर्हि तत्प्रतियोगिरूपविशेषोऽपि तद्विषयः न भवेत्~। तथा च किं तस्य विषयः इति चेत् प्रसिद्धरूपविशेषात् भिन्नं रूपम् इति वक्तुम् न शक्यते, तस्यात्यन्तमसत्वात्, एवञ्च तस्य संशयस्य रूपविषयकत्वमेव न निर्वहेत् इति चेन्न~। ‘यद्दर्मावच्छिन्नप्रतियोगिताकाभावः यस्य यत्र निश्चितः, तत्र तेनरूपेण सः  संशयविषयो न भवति’ इति नियमः~। एवञ्च वायौ रूपविशेषाभावकूटस्य निश्चितत्वेऽपि स अभावः न रूपत्वावच्छिन्नप्रतियोगिताकाभावः~। रूपसामान्याभावस्यैव तादृशप्रतियोगितात्वात्~। तस्य च तत्र अनिश्चितत्वेन रूपत्वेन रूपस्य संशयविषयत्वं भवेदेव~। 

\section*{व्याख्यातुरभिमतम्} 

उपरिष्टात्सर्वं यथाग्रन्थं तदनुरोधिशङ्कानिरासादिन व्याख्यातम्~। ‘वयं तु ब्रूमः,’ इत्यादिना स्वाभिमतमाविष्करोति प्रभाकारः~। यदि ‘वायौ रूपं नास्ति’ इत्यपि रूपसामान्याभावत्वप्रकारिका प्रतीतिः रूपविशेषाभावकूटविषयिणी एव, तयोरभावयोरनतिरेकादित्युच्यते, तदा ‘इह घटद्वये रूपं नास्ति’ इति प्रतीत्यापत्तिः, प्रत्येकं घटे कस्यचिद्रूपस्य सत्वेऽपि यावद्रूपविशेषस्य घटद्वयेऽसत्वात् यावद्रूपाभावकूटस्य तत्र अबाधितत्वात्~। रूपविशेषस्य घटव्यक्तौ सत्वेऽपि स्वानधिकरणघटव्यक्त्यवच्छेदेन तदभावस्य सत्वात्~। यदि एवं नाङ्गीक्रियते तर्हि यावद्रूपविशेषाभावोऽपि न सिध्येत्~। तथा च ‘इह समुदितघटद्वये यावद्रूपविशेषाभावः अस्ति’ इत्यपि प्रामाणिकप्रतीतिः न स्यात्~। तथा च घटद्वस्य यावद्रूपविशेषवत्वापत्तिः~। 

\section*{ऐकाधिकरण्यविशेषणेन उक्तापत्तिवारणशङ्का} 

अथ एकाधिकरणः विशेषाभावकूटः सामान्याभावरूपः इत्युच्यते~। प्रकृते घटद्वयवृत्तियावद्विशेषाभावकूटः न एकाधिकरणवृत्तिः~। प्रत्येकस्मिन् घटे तादृशाभावकूटस्याभावात्, अतः तस्य न सामान्याभावरूपत्वमिति न तादृशप्रतीत्यापत्तिः इतिचेत्, समानाधिकरणः विशेषा\-भावकूटः सामान्याभावः इत्यत्र सामानाधिकरण्यं कुत्र विशेषणम् ? सामान्याभावे वा ?\break अथवा यः वास्तविकरूपेण विशेषाभावसमानाधिकरणः, स सामान्याभावे विशेषितो वा ? आद्ये – रूपाभावस्य च रूपं प्रतियोगि, प्रतियोगित्वं च अभावविरहात्मत्वम् इति रूपस्य स्वाभावाभावरूपत्वमावश्यकम्~। किन्तु एकाधिकरणः विशेषाभावकूटः सामान्याभाव इत्यङ्गी\-कारे, रूपस्य तद्विशेषाभावविरहात्मत्वं विशेषाभावनिष्ठसामानाधिकरण्याभावरूपत्वं इत्युभय\-वस्तुविरहात्मता अङ्गीकरणीया इति रूपस्य  भावाभावोभयात्मकत्वापत्तिः~। विशेषाभावा\-भावस्य भावरूपत्वात् सामानाधिकरण्याभावस्य च अभावरूपत्वात् तदुभयात्मकत्वात्\break रूपस्य~। यदि विशेषाभावतन्निष्ठैकाधिकरणत्वाभावात्मकत्वं नाङ्गीक्रियते, एकाधिकरणवृत्तित्वविशिष्टविशेषाभावकूटाभावरूपत्वं रूपस्याङ्गीक्रियते, तदा रूपविशेषाभावो रूपानात्मकः स्यात्~। यथा घटाभावाभावस्य घटात्मकत्वेऽपि अधिकरणविशिष्टः घटाभावाभावः घटरूपः न भवति तद्वत् एकाधिकरणत्वविशिष्टरूपविशेषाभावाभावोऽपि रूपात्मकः न भवति इति\break महदसामञ्जस्यम्~। 

अन्त्ये घटद्वये इहरूपं नास्ति इति प्रतीत्यापत्तिः इति पूर्वोक्तः एव दोषः~। यतः कुत्रचित् एकाधिकरणवृत्तिविशेषाभावकूटस्यैव घटद्वयसमुदायवृत्तित्वात्, तस्यैव च रूपसामान्याभावात्मकत्वात्~। 

\section*{एकाधिकरणस्य विशेषाभावकूटस्य सामान्याभावत्वेन प्रतीतिशङ्का~। }

अथ विशेषाभावकूट एव सामान्याभावः सः नानाधिकरणश्चेत् सामान्याभावत्वेन न प्रतीयते~। एकाधिकरण एव तथा प्रतीयते, तथा च घटद्वयवृत्तिविशेषाभावकूटस्य एकाधिकरणत्वाभावात् रूपं नास्ति इति सामान्याभावत्वेन प्रतीतिर्न भवति इति चेन्न~। प्रतीतौ तद्विषयस्य कारणत्वम्, न तु तदधिकरणम् इति प्रतीतिविषयः एकाधिकरणो वा नानाधिकरणो वा, प्रतीत्युत्पत्तौ\break तस्याकिञ्चित्करत्वात्~। तथा च प्रतीतिविषयस्य विशेषाभावकूटस्य सामान्याभावकूटात्म\-कस्य नानाधिकरणत्वेऽपि तत्प्रतीतिनिवारणं गीर्वाणगुरूणामप्यशक्यम्~। तदेतदुक्तं दूषणान्त\-रम् – प्रत्येतव्यशरीरस्य नानाधिअकरणेऽपि सत्वे तत्र तत्प्रतीतिनिवारणस्य ब्रह्मणोऽप्यशक्यत्वात्  इति~। 

अपि च तथाङ्गीकारेऽपि यत्किञ्चित्संयोगवति घटे इह संयोगो नास्ति इति संयोग\-सामान्याभवप्रतीत्यापत्तिः~। तत्र यावत्संयोगविशेषाभावकूटस्य सत्वात् इत्यन्या आपत्तिः प्रदर्शिता~। इदं तु चिन्त्यं यत्किञ्चित्संयोगसत्वे तद्यावाद्विशेषाभावः कथं सम्भवतीति~। तथापि एकाधिकरणकयावत्संयोग-विशेषाणां तत्रासत्वात् तथाप्रतीतिराशङ्किता इति भाति~। 

अपि च विशेषाभावकूटः एव सामान्याभावः चेत् कूटपदार्थः समूहः किं विशेषाभाव एव वा ? उत तदतिरिक्तः ? समूहः भावरूपो वा ? यदि समूहः विशेषाभावः एव, तदा इह रूपं इति, इह रूपाभावः इति एकवचनोल्लेखप्रतीतिर्न स्यात्~। समूहस्य च विशेषाभावरूपत्वात्, तस्य च नैकत्वात्~। यतः अनेके विशेषाभावाः वर्तन्ते, तेषामेव च समूहरूपत्वात् बहुत्वस्य सत्वात्, सर्वदा बहुत्वोल्लेखप्रतीतिरेव ’इह रूपाभावाः’ इति स्यात् इत्यर्थः~। 

द्वितीये यदि विशेषाभावातिरिक्तः समूहः इत्युच्यते, तर्हि सः किं भावरूपो वा अभावरूपो वा? यदि भावरूपः तदा तस्य सामान्याभावरूपत्वं न स्यात्~। भावाभावयोर्विरोधात्~। यथा रूपविशेषाभावसमूहः रूपविशेषाभावातिरिक्तः भावरूपश्चेत् तदानीं सः रूपप्रतियोगिकाभावः एव न भवति, तस्य भावरूपत्वात्~। यदि अतिरिक्तः सः अभावरूपः इत्युच्यते तर्हि सिद्धमेव सामान्याभावस्य अतिरिक्तत्वम् इति~। 

\textbf{गम्यगमकभावोपपत्तिः-} अतः सामान्याभावः विशेषाभावकूटातिरिक्तः इति केनापि \-स्वीकर्तव्यमेव~। तयोश्च भेदे सिद्धे ’यदि गृहे न घटसामान्यं, तदा न घटविशेषः’ इत्यादि गम्यगमकभावप्रतीतिनिर्वाहः~। घटसामान्यस्यैवाभावात् तद्विशेषोऽपि नास्ति इति गम्यस्य घटविशेषाभावस्य गमकः घटसामान्याभावः~। तथा च यो यत्सामान्याभावात् स तद्विशेषाभावात् इति व्याप्त्या च सामान्याभावविशेषाभावयोः साध्यसाधनभावः~। एवं ’यदि गृहे न घटविशेषः तदा तत्र न घटसामान्यं ’ इति च, यो यदीययावद्विशेषाभाववान् स तत्सामान्याभाववन् इति व्याप्तेः~। अयं च गम्यगमकभावः सामान्याभावविशेषाभावकूटयोर्भेदनिबन्धनः तयोः भेदमन्तरा न सम्भवतीति एतदप्युपष्टम्भकम्~। 
 	
\section*{यत्तुकारमतम्} 

सामान्यधर्मः विशेषाभावप्रतियोगितावच्छेदको न भवति इति तु मूले एव प्रतिपादितम्~। किन्तु केचित् वायौ रूपं न वा इति संशयो रूपत्वावच्छिन्न- प्रतियोगिताकत्वेन रूपाभावं विषयीकरोति~। रूपविशेषाभावकूटश्च न रूपत्वावच्छिन्नप्रतियोगिताकः~। अतः विशेषाभावकूटनिश्चये संशयानुपपत्या विशेषाभावकूटातिरिक्तत्वं सामान्याभावस्य, तस्यैव च संशयविषयत्वमङ्गीकार्यम् इति ग्रन्थतात्पर्यं वर्णयन्ति~। तन्न समीचीनम्~। वायौ रूपविशेषाभावकूटनिश्चये ’रूपं न वा’ इति संशये रूपत्वावच्छिन्नप्रतियोगिताकत्वेन अभावस्य विषयत्वाभावात्,\break किन्तु रूपत्वविशिष्टप्रतियोगिताकत्वेन विशेषाभावकूटसाधारण्येन अभावमवगाहते सः\break संशयः, ’वायौ न रूपं’ इति निश्चयो वा~। यतः प्रतियोगित्वान्यूनानतिरिक्तवृत्तित्वस्य\break अवच्छेदकत्वस्य रूपत्वेऽस्फुरणात्~। यदि रूपत्वस्यैव प्रतियोगितावच्छेदकत्वावगाहनं स्यात् तदा ’वायौ न रूपम्’ इति निश्चयः एव अतिरिक्तसामान्याभावे प्रमाणां स्यात्~। तथा च संशयानुपपत्तिप्रदर्शनप्रयासोऽनुचितः स्यात्~। विशेषाभावकूटस्यैव संशयविषयताप्रयोजकत्वेन वायौ तन्निश्चितत्वाभिधानमपि व्यर्थं स्यात्, वायौ रूपविशेषाभावकूटानिश्चयेऽपि तस्य\break रूपत्वावच्छिन्नप्रतियोगिताकत्वाभावात् रूपत्वावच्छिन्नप्रतियोगिताकत्वेन संशयविषयत्वसम्भावनासम्भवात्~। 

यद्यप्यत्र यत्तुकारमते संशयानुपपत्तिः कथं इति तु नोक्तं, तच्चिन्त्यम्~। यतः तैः \-संशयस्य रूपत्वावच्छिन्नप्रतियोगिताकत्वं अङ्गीक्रियते  रूपविशेषाभावकूटस्य च न तथात्वं, अतः\break विशेषाभावनिश्चयस्य च रूपत्वावच्छिन्नप्रतियोगिताकत्वानवगाहनात् तत्सत्वेऽपि रूपत्वावच्छिन्नप्रतियोगिताकसंशयस्य बाधकाभावात् कथं तदनुपपत्तिः ? अत एव प्रभाकारेण\break अन्तिमदोषः--’तत्र तदनिश्चयेऽपि तस्य रूपत्वावच्छिन्नप्रतियोगिताकत्वविरहादेव तादृशसंशयविषयत्वसम्भावनानवकाशात्’ इति विशेषाभावस्य संशयविषयत्वकल्पनं निराकृतम्~। एवं च रूपविशेषाभावस्य संशयविषयत्वायोगात् न अतिरिक्तसामान्याभावसिद्धिः, किन्तु\break संशयानुपपत्या एव तत्सिद्धिरिति यज्ञपतिः यथाग्रन्थं समर्थयति~। इदं च यत्तु इत्यादिनोक्तं अभिनववाचस्पतिमिश्राणां मतमिति धर्मराजाध्वरीन्द्रकृततर्कचूडामणिनाम्न्या रुचिदत्तकृतप्रकाशस्य व्याख्ययावगम्यते~। 

एवं प्रकाशकारेण तु यज्ञपतिकृता ’समुदितघटद्वये इह रूपं नास्ति’ इति प्रतीत्यापत्तिः\break निराकृता~। यथोक्तं-’यत्राधिकरणे यदवच्छेदेन विशेषाभावकूटो वर्तते तत्र तदवच्छेदेन\break सामान्याभवप्रतीतिं जनयति नान्यत्रापि~। घटद्वये च प्रत्येकं यावद्विशेषाभावाभावे समुदितेऽपि तथात्वमेव~। प्रत्येकोभयातिरिक्तस्य समुदायस्यानभ्युपगमात्~। अभ्युपगमे वा तत्र रूपसामान्याभावप्रतीतेरिष्टत्वात्’ इति~। एवं च समुदितघटद्वये घटत्वावच्छेदेन विशेषाभावकूटो नास्ति धटे कस्यचिद्रूपस्य सत्वादिति घटत्वावच्छेदेन न रूपसामान्याभावप्रतीतिः, एवं च प्रत्येकं घटद्वये रूपाविशेषाभावकूटाभावे तत्समुदायेऽपि तथैव समुदायस्य प्रत्येकानतिरिक्तत्वादिति समुदायेऽपि न रूपसामान्याभावप्रतीतिः~। यदि तु  समुदायस्य प्रत्येकातिरिक्तत्वमभ्युपेयते तर्हि तत्र तत्प्रतीतिरिष्टैवेति न तया प्रतीत्या अतिरिक्तसामान्याभावसाधनम्~। एवं एकत्वो\-ल्लेखिनी इह रूपाभावः इति प्रतीतिरपि न स्यात् इति यज्ञप्त्युक्तमपि दूषितं–आद्यक्षणावच्छिन्ने घटे इह रूपाभावः इति प्रतीतिरपि न स्यात्, तत्र रूपसान्याभावस्य वक्तुमशक्यत्वात् ध्वंसप्रागभावयोरत्यन्ताभावेन सह विरोधात्~। किन्तु तत्र रूपविशेषाणां अनेके  अभावाः एव तादृशप्रतीतिविषया इति तत्रापि इह रूपाभावा इत्येव प्रतीतिस्स्यात्~। अतः बहुत्वोल्लेखप्रतीत्या अतिरिक्तसामन्याभावसाधनमशक्यमिति~। 

एवं दीधितिगादाधर्यामपि प्रकारान्तरेणैव सामान्याभवस्यातिरिक्तत्वं साधितम्~। यथा विशेषाभावकूटस्यैव सामान्याभावरूपत्वे परस्परं अभावभेद एव न स्यात्~। तदुक्तं-एवं हि तत्तद्देशकालाद्यवच्छिन्नानवच्छिन्नतत्तदधिकरणवृत्तित्वेन विशिष्टस्य तत्तदवच्छेदेन वा एकस्यैवाभावस्य घटपटगोत्वाश्वत्वादिप्रतियोगिकत्वसम्भवेन तत्तद्भेदोऽपि विलुप्येत~। इति~। अपि च व्यासज्यवृत्तिधर्मज्ञानार्थं न यावदाश्रयसन्निकर्षापेक्षा, यावदवयवसन्निकर्षाभावेऽपि तत्समवेतावयविप्रत्यक्षात्~। एवं अत्रपि यत्किञ्चिदेकाभावसन्निकएषेण सामान्याभवप्रतीतिः स्यात्~। न भवति च तथेति तस्य पार्थक्यम् इति च दीधितिकारेण युक्तियुक्तं तत्साधनम्~। 

शास्त्रे खण्डनमण्डनं तु वर्तत एव~। किन्तु यज्ञपतिना उपाध्यायेन नूतनाः विषयाः प्रतिपादिताः, युक्तयश्च  प्रदर्शिताः~। तेन तद्व्याख्यानं मण्यर्थावगमाय, शात्रसिद्धान्तबोधाय, शास्त्रार्थचिन्तनाय च उपकरोति इति तु निश्चप्रचम्~। 

\section*{परामृष्टग्रन्थाः -}

\begin{thebibliography}{99}
\bibitem{chap29key1} तत्वचिन्तामणिप्रभा 
\bibitem{chap29key2} माथुरी 
\bibitem{chap29key3} तत्वचिन्तामणिप्रकाशः तर्कचूडामणिसहितः
\bibitem{chap29key4} अनुमानगादाधरी
\end{thebibliography}

\articleend
}
