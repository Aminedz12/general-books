{\fontsize{15}{17}\selectfont
\chapter{परमाणुः}

\begin{center}
\Authorline{वि । भास्कर-हेगडे}
\smallskip
पूर्वविद्यार्थी
\addrule
\end{center}
जगदिदंकार्यरूपम् । कार्यञ्चजन्यम्। जन्यं तावत् अनेकावयवसंयोगरूपम् । यथा घटापटादिः । तादृशस्य घटादिकार्यस्य निरवयवत्त्वं वा ? अवयवानन्त्यं वा ? परमाण्वन्तता वा ? इति जिज्ञासायां सत्याम् अत्र प्रतिपाद्यते । यदि निरवयवत्त्वं स्वीक्रियते तस्य तर्हि घटादेर्नित्यत्वप्रसङ्गः । यन्निरवयवी तन्नित्यम् इति न्यायात् । कपालद्वयसंयोगात् जायमाने घटे अनेकावयवदर्शनात् । ध्वस्तो घटः इति प्रतीतेः च निरवयवत्त्वं नोपपद्यते ।

अवयवानन्त्यमप्यप्रामाणिकम्।मेरु-सर्षपयोः अनन्तावयवयोगित्व-विशेषगुणपरिमाणत्व-प्रसङ्गात् । तस्मात् परमाण्वन्तता एव कार्यस्य युक्तिमती ।

तथा च पार्थिवम् आप्यं तैजसं वायवीयञ्चेति चतुर्विधं कार्यं स्वावयवमाश्रितमुपलभ्यते । यथा घटः स्वावयवं कपालमाश्रितः । तादृशावयवानामवयवान्तरेषु परमाणुरेव निरवयवी । परमाणुषु सावयवत्त्वस्य च हेतोरसिद्धत्वात् न अवयवानन्त्यकल्पना अनवस्थापातात् । तेषामप्यवयवत्त्व ते एव परमाणवः भवेयुः । तेषां परिमाणमाह-
\begin{verse}
जालसूर्यमरीचिस्थं यत्सूक्ष्मं दृश्यते रजः ।\\
तस्य षष्ठस्तु यो भागः परमाणुः स उच्यते ॥ इति ।
\end{verse}
अस्यार्थस्तु यत्सूक्ष्मं रजः गवाक्षे आलोकसहायतया दृश्यते तस्य षष्ठः भागः परमाणुरिति गीयते । अतः अवयवधारायाः यत्र विरामः स एव परमाणुः । परमाणुस्तु अणुपरिमाणकः । तस्मात् अपि सूक्ष्मपरिमाणस्य अभावात् अणुरेव सूक्ष्मतमं परिमाणम् । त्रपरमाणौ किं मानमिति चेत् उच्यते । यावत् कार्यजातस्य स्वावयवाश्रितस्य प्रत्यक्षेण ग्रहणं तत्र तदेवप्रमाणम् । अत्र अनुमानमपि । यतः प्रत्यक्षपरिकलितमप्यर्थम् अनुमानेन बुभुत्सन्ते तर्करसिकाः । तद्यथा- कार्यं स्वावयवाश्रितं सावयवत्वात् परिदृश्यमानकार्यवत् ।

इमे परमाणवस्तावत् अचेतनाः । अतः कार्यार्थं काञ्चनचेतनश्क्तिमपोन्ते । तादृशचेतनशक्तिरेव ईश्वरः । अतः सः कार्यसामान्यस्य निमित्तकारणम् ।

तत्र परमाणौ योगिप्रत्यक्षं, सम्पतत्रैः द्यावाभूमिं जनयन् देव एकः इति च श्रुतिरपि प्रमाणम् । यतो हि गत्यर्थक “पत्ल्” धातोः अत्र न प्रत्यये प्राप्ते पतत्रम् इति शब्दः निष्पद्यते । गतिशीलत्वम् अस्यार्थः । अनेन सततगतिशीलाः परमाणवः पतत्रशब्देन लक्ष्यन्ते इति सिद्धम् ।

अपि च सकृदेव सर्वे कार्यनिर्वर्त्यमानाः परिमाणवः परिमाण्वनुगुणसङ्ख्यया एकत्र संयोज्य कार्यमारभन्ते वा इति ते सकृदारम्भे कुम्भे विभज्यमाने कपाल-शर्करा-कण-चूर्णादिकमपहाय साक्षात् परमाण्वन्तताभावात् तथा न सङ्गच्छते । तस्मात् द्व्यणुकादि क्रमेणपरमाणवः कार्यमारभन्ते । 

अपि च अनित्यं द्व्यणुकादौ तु सङ्ख्याजन्यमुदाहृतम् इत्युक्त्या त्र्यनुकं प्रति द्व्यणुकपरिमाणं न कारणम् । द्व्यणुकं प्रति परमाणुपरिमाणं न कारणम् इति ज्ञायते । यतः परिमाणस्य स्वसमानजातीयोत्कृष्टपरिमाणजनकत्वनियमात् । 

अतः द्वावेव परमाणूप्रथमं सङ्घटेते । बहुत्वसङ्ख्यायाः महत्परिमाणकारणत्वदर्शनात् । त्रिषु परमाणुषु मिलत्सु तत्कार्ये बहुत्वसङ्ख्यायाः महत्वारम्भकत्वात् तत्प्रत्यक्षत्वं प्रसृजेत । अतिसूक्ष्मात् तन्न प्रत्यक्षम् । अतः द्व्यणुकादिक्रमेणकार्योत्पत्तिः । परमाणुद्वयसंयोगात् द्व्यणुकं, द्व्यणुकत्रयसंयोगात् त्र्यणुकं, त्र्यणुकचतुष्टयसंयोगात् चतुरुणुकम् इति कार्योत्पत्तिक्रमः । 

यस्य निरवयवत्त्वं तन्नित्यम् इति नियमात् परमाणुः नित्यद्रव्यम् । एते नपरमाणौ अणुपरिमाणवत्त्वं नित्यत्वं तत्परिमाणस्य कारणाभाववत्त्वञ्च अबाधितम् । 

\articleend
}
