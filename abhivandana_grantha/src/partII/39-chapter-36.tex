{\fontsize{15}{17}\selectfont
\presetvalues
\chapter{व्यक्तित्वविकसने भगवद्गीताया योगदानम्}

\begin{center}
\Authorline{स्वामी करुणाकरानन्दः}
\smallskip
रामकृष्णाश्रमः\\
बेङ्गळूरु
\addrule
\end{center}

\begin{center}
गीता सुगीता कर्तव्या किमन्यैः शास्त्रविस्तरैः~। \\
यः स्वयं पद्मनाभस्य मुखपद्माद् विनिस्सृतः~॥
\end{center}
इति गीतामाहात्म्ये निक्षिप्तोऽयं श्लोकः गीताया विषये न अतिशयोक्तिः~। न केवलं भारतीयानाम् अपि तु विश्ववासिनां सर्वेषां कर्मक्षेत्रे आध्यात्मिकविकसने च मुख्यतः इदं शास्त्रं शासनं कृतवत् जीवनपथप्रदर्शकेन ज्योतिषा विराजते~। 

भगवद्गीताया अध्ययनेन ज्ञायते यत् तत्र द्विप्रकारिके व्यक्ती निरूपिते स्तः~। एका अविनाशिनी नित्या सर्वव्यापिनी द्वन्द्वातीता परिणामरहिता महती~। 
\begin{verse}
नित्यः सर्वगतः स्थाणुरचलोऽयं सनातनः~॥ \\
अव्यक्तोऽयमचिन्त्योऽयमविकार्योऽयमुच्यते~। ' (२.२४-२५), ]\\
सर्वतः पाणिपादं तत्सर्वतोऽक्षिशिरोमुखम्~। \\
सर्वतः श्रुतिमल्लोके सर्वमावृत्य तिष्ठति~॥ (१३.१३), 
\end{verse}
इत्यादिभिः श्लोकैः तस्याः स्वरूपं वर्णितमत्र~। श्रीकृष्णस्य अहङ्काररूपेण 
\begin{verse}
‘यो मां पश्यति सर्वत्र सर्वं च मयि पश्यति~। \\
तस्याहं न प्रणश्यामि स च मे न प्रणश्यति॥' (६.३०)\\
न मे पार्थास्ति कर्तव्यं त्रिषु लोकेषु किञ्चन~। \\
नानवाप्तमवाप्तव्यं वर्त एव च कर्मणि~॥' (३.२२) 
\end{verse}
इत्यादिभिः तस्या अभिव्यक्तिः लोके वर्णिता~॥ अपरा तु असङ्ख्यरूपैः चिज्जडैः नित्यं\break परिणामवती परिच्छिन्ना त्रिगुणैर्बद्धा रागद्वेषमुखैः द्वन्द्वैरभिभूता क्षुद्रा च~। तस्याः स्वरूपं तु\break त्रयोदशाध्याये क्षेत्रनिरूपणेन, सप्तमाध्याये परापराप्रकृतिनिरूपणेन, षोडशाध्याये\break दैवासुरसम्पद्विभागेन, सप्तदशाध्याये गुणत्रयनिरूपणेन उपपादितम्~। धृतराष्ट्रस्य अहङ्काररूपेण ‘मामकाः पाण्डवाश्चैव' इति, दुर्योधनस्य अहङ्काररूपेण ‘मदर्थे त्यक्तजीविताः' इति, अर्जुनस्य अहङ्काररूपेण ‘यदि मामप्रतीकारमशस्त्रं शस्त्रपाणयः' इति च तस्या अभिव्यक्तयः दर्शिताः~। एषा द्वितीयाभिव्यक्तिः जीवस्य व्यक्तित्वं सङ्कोचयति लोकेऽसामरस्यस्य बीजं च वपति~। तस्मात् निजजीवने सुखं शान्तिं स्थापयितुं सामाजिकसामरस्यं च निर्वोढुं प्रथमं\break व्यक्तित्वमाश्रयणीयम्~। भारतीयसांस्कृतिकजीवनस्य परमादर्शः इदमेव~। भारतीयस्य प्रत्येकं जीवने इममेवादर्शम् आनेतुम् सर्वाणि शास्त्राणि विधिनिषेधरूपाणि नियोजितानि~। किन्तु\break प्रकृते द्वितीयाद् व्यक्तित्वात् प्रथमव्यक्तित्वे आत्मानं स्थापयितुं केनचित् सेतुना विना कथं शक्यते? इति चेत् वक्ष्यामः तदर्थमेव भगवता धर्मसेतुरयं निर्मितः इति~। ‘स्वल्पमप्यस्य\break धर्मस्य'(२.४०),‘ये तु धर्म्यामृतमिदं यथोक्तं पर्युपासते' (१२.२०), ‘अध्येष्यते च य इमं धर्म्यम्' (१८.७०) इति तेनैव उक्तत्वात्~। सम्पूर्णा भगवद्गीतापि ‘धर्' (धर्मक्षेत्रे) ‘म' (मम) इति द्वयोर्मध्य एव निहिता~। 
\begin{verse}
व्यक्तित्वपरिवर्तने को धर्मः आश्रयणीयः ?
\end{verse}
मुहुर्मुहुः आदर्शव्यक्तित्वविषये श्रवणं कर्तव्यम्~। तस्मात् गीतायां तत्र तत्र भगवता श्रीकृष्णेन तस्य सविस्तरं प्रवचनं कृतम्~। 
\begin{verse}
न जायते म्रियते वा कदाचिन्नायं भूत्वा भविता वा न भूयः~। \\
अजो नित्यः शाश्वतोऽयं पुराणो न हन्यते हन्यमाने शरीरे~॥ (२.२०) \\
यस्त्वात्मरतिरेव स्यादात्मतृप्तश्च मानवः~। \\
आत्मन्येव च सन्तुष्टस्तस्य कार्यं न विद्यते~॥ (३.१७)\\
न च मां तानि कर्माणि निबध्नन्ति धनञ्जय~। \\
उदासीनवदासीनमसक्तं तेषु कर्मसु~॥(९.९)\\
समोऽहं सर्वभूतेषु न मे द्वेष्योऽस्ति न प्रियः~। (९.२९)
\end{verse}
इत्यादिभिः श्लोकैः यद्व्यक्तित्वं प्रतिपादितं तस्य चिन्तनं सदा कुर्वन् उद्देशात् कदाऽपि न प्रमाद्यति~। ततः तदर्थं साधनानि सम्पादयितव्यानि~। तस्माद् भगवता आदौ निष्कामकर्म उक्तम्~। सर्वाणि कर्माणि ज्ञाने परिसमाप्यन्ते~। तस्मात् चित्तशुद्धिद्वारा ज्ञानार्थमेव कर्तव्यं कर्म करणीयम्~। ज्ञानादृते अन्यत् फलं कर्मद्वारा नापेक्षणीयम्~। तस्मात् 
\begin{verse}
कर्मण्येवाधिकारस्ते मा फलेषु कदाचन~। \\
मा कर्मफलहेतुर्भूर्मा ते सङ्गोऽस्त्वकर्मणि~॥ (२.४७)
\end{verse}
इत्यादिशति भगवान्~। कर्मणा यदि फलमपेक्षेत तर्हि ‘बहुजनहिताय बहुजनसुखाय' एव~। ततः स्वार्थः विनश्यति~। अपि च निरपेक्षकर्मणः सर्जनशक्तिर्विलक्षानन्दश्च जायेते इति वैज्ञानिकैः संशोधितम्~। कर्मफलापेक्षया तयोर्निरोधो जायते~। १९४९तमे वर्षे मनश्शास्त्रीयप्राध्यापकेन केनचित् ह्यारिहारलो महाशयेन कपिषु कृतेन संशोधनेन अयं विषयः आविष्कृतः~। बहुजनहिताय यत् कर्म कुर्मः सः यज्ञनामकः भवति~। गीतायां यज्ञस्य बहु प्रशंसा कृता~। तस्य अवश्यकर्तव्यत्वं प्रतिपादितम्~। तपोदानेऽपि साधनरूपेण विहिते~। प्रतिकूलप्राप्तौ मनसः समस्थितेः रक्षणं, इन्द्रियनिग्रहं च तपो भवति~। ततः अन्तःशक्तिः वर्धते~। यथाशक्ति स्वस्वामित्वे वर्तमानं यत्किञ्चिदपि सत्पात्राय अनुपकारिणे देयमिति बुद्ध्या परस्वामित्वकरणं दानमिति कथ्यते~। ततः उदारमना भवति, स्वार्थत्यागः सिद्ध्यति~। अभयमित्यादीन् दैवीगुणान् सम्पादयितुं , दम्भादीन् आसुरीगुणांश्च हातुं तेषामुपदेशः कृतः~। ताभ्यां हेयोपादानाभ्यां सीमिताहङ्कारः नश्यते ‘अहं सर्वस्य प्रभवो मत्तः सर्वं प्रवर्तते' (१०.८) इत्याकारकः पक्वाहङ्कारो जायते ततः लोकसङ्ग्रहो भवति, बहुजनहितं च भवति~। गुणत्रयं विभज्य अनुष्ठानयोग्यसत्त्वगुणलक्षणं दर्शयति भगवान्~। स्वस्मिन् जीवने सत्त्वगुणादानेन विकासो जायते  जीवनश्रेणीमुपर्युपरि गच्छति, रजस्तमोभ्यां सङ्कोचो जायते अधःपतनं तस्य सिद्धं भवति~। 
\newpage

\section*{विकसितव्यक्तित्वस्य लक्षणम्} 

सम्पूर्णतया यस्य व्यक्तित्वं विकसितं सः कीदृशः भवति ? इति चेत् ब्रूमः सः सम्पूर्णस्वातन्त्र्येण सर्वं कार्यं लोकहिताय एव करोति~। तैः कर्मभिः सः न बध्यते~। यत् किञ्चिदपि तस्य मन उद्विग्नं कर्तुं न शक्नोति~। यः कोऽपि तमुपसर्पति सः मनश्शान्तिमाप्नोति~। निर्दुष्टतया सः सम्यक्चिन्तनं कर्तुं शक्नोति यतः सर्वेषां कल्याणं भूयात्~। पूर्णकामः सः स्वार्थचिन्तनरहितः सदा इतरेषां कामनापूर्तये यतते~। सर्वेषां हृदि सः वर्तते सर्वे तस्य हृदि वर्तन्ते च~। सः विश्वमानवो भवति~। जीवने आगमापायिनी सुखदुःखे सः न इच्छति न द्वेष्टि च~। लाभालाभौ समेन स्वीकरोति सः~। तस्य मन अत्यन्तमानन्दमनुभवत् गुरुणापि दुःखेन न विचाल्यते~। यस्मिन् समाजे एतादृशा परिपूर्णा व्यक्तयः सन्ति तस्मिन् समाजे स्थिरशान्तिः वर्तते, सर्वतः नरा अभ्युदयम् आप्नुवन्ति~। आध्यात्मिकवैज्ञानिकक्षेत्रयोर्मध्ये सामरस्यं जायते~। प्रकृतिपशुमनुष्येषु सामरस्यं च स्थाप्यते~। एवं भगवद्गीता जीविनं पशुत्वात् नरत्वं, नरत्वात् च दैवत्वं गमयती जगति देवलोकमेव आनयति~। तस्मात् सर्वभिर्नरैः तस्या अध्ययनं कृत्वा तासु उक्तानि मौल्यानि अवश्यं स्वजीवने अनुष्ठातुं प्रयत्नः कर्तव्यः~। 

\articleend
}
