{\fontsize{15}{17}\selectfont
\presetvalues
\chapter{नैषा तर्केण मतिरापणेया}

\begin{center}
\Authorline{आचार्य: महाबलेश्वर पि भट्ट:}
\smallskip

विभागध्यक्ष:, अद्वैतवेदान्तविभाग:\\
राष्ट्रियसंस्कृतसंस्थानम् (मानितविश्वविद्यालय:)\\
राजीवगान्धीपरिसर:, मेणसे,\\ 
शृङ्गेरी -577139
\addrule
\end{center}

\begin{center}
शंकारूपेण मच्चित्तं पङ्कीकृतमभूद्यया~। \\
किङ्करी यस्य सा माया शङ्कराचार्यमाश्रये~॥
\end{center}

भारतीयार्षपरम्परायां लब्धजन्मानामस्माकं परमं प्रमाणम् अपौरुषेयं वाक्यं वेदेति विश्वजनीनम्~। सर्वदा सर्वथा सुखं मे भूयात् दु:खं मनागपि मा  भूदिति मनसिकृत्य दिवानक्तं प्रवर्तमानानां नरजन्मनां धर्मब्रह्मस्वरूपं करतलन्यस्तामलकवत् वेदो निवेदयतीति निश्चप्रचं सर्वेषां प्रेक्षावताम्~। अत: ऋग्यजुसामाद्यात्मना चतुर्धा विभक्त: आम्नायसमुदाय एव शास्त्र\-शब्दस्य मुख्यो अर्थ:~। तथा च निरटङ्कि शङ्करभगवत्पादै: शारीरकमीमांसायां शास्त्रयोनित्वादिति तृतीयसूत्रव्याख्यानावसरे “महत: ऋग्वेदादिशास्त्रस्य अनेकविद्यास्थानोपबृंहितस्य प्रदीपवत् सर्वार्थावद्योतिन:” इति~। प्रत्यक्षादिप्रमाणै: कथञ्चिदपि अनवगतम् अलौकिक\-सुखसाधनम् आत्यन्तिकसंसारदु:खनिवृत्तिसाधनं च उपदिशतीति चिरन्तना परमर्षय:\break श्रुतिराशे: संरक्षणे कटिबद्धाबभूवु:~। वर्णस्वरपदवाक्यवाक्यार्थतात्पर्यादिसम्यक् ज्ञानाय\break प्रचाराय रक्षणाय च चतुर्दशविद्यास्थानानि प्रणिन्यु:~। अत एव वेदस्य अप्रामाण्यशङ्कास्पर्श: स्वप्नेऽपि न स्पृशति~। 

सोऽयं वेद: ऐहिकामुष्मिकसुखलिप्सूनां प्रवृत्तिमार्गं निवेदयति दृष्टानुश्रविकविषय\-वितृष्णानां संसारव्याविवृत्सूनां निवृत्तिमार्गं दर्शयति~। तदनुसारिण्य: स्मृतय:, पुराणानि, इतिहासा:, काव्यनाटकादिकृतय: श्रौतीं सरणिम् अनुसरन्त्य: प्रवृत्तिनिवृत्तिमार्गद्वयमेव अनुवदन्ति~। अत एव स्मृत्यादीनां श्रुतिमूलकत्वेन प्रामाण्यं
\begin{verse}
गौणं शास्त्रत्वञ्च घटते~। तथा चोक्तम् –\\
प्रवृत्तिर्वा निवृत्तिर्वा नित्येन कृतकेन वा~। \\
पुंसां येनोपदिश्येत तच्छास्त्रमभिधीयते~॥
\end{verse}

शास्त्रेषु मनुष्याणामेवाधिकार: नान्येषां प्राणिनाम्~। यत: तिर्यक्प्राणिन: वागिन्द्रियविकला: दुर्बलमनोबला: केवलम् अहारनिद्रासन्तानवर्धनादिषु दत्तचित्ता:~। मानवशरीरं\break एकादशेन्द्रियसंवलितं अलौकिकं वस्तु चिन्तयितुम् अणिमाद्यष्टैश्वर्यं सम्पादयितुं सर्वानपि प्राणिन: स्वायत्तीकर्तुं समर्थमस्ति~। 

अत एतादृशं सर्वकार्यसमर्थं शरीरं सुदुर्लभं सम्प्राप्य किं कर्तव्यम् ? किं न कर्तव्यम् ? किं साधनीयम् ? किं भोक्तव्यम् ? किं द्रष्टव्यम् ? किं श्रोतव्यम् ? किं मन्तव्यम् ? किं विज्ञेयम् इत्येवं प्रेक्षावन्त: विचारयन्ति~। तान् प्रति गीता भगवती गायति –
\begin{verse}
\textit{तस्माच्छास्त्रं प्रमाणं ते} कार्याकार्यव्यवस्थितौ~। \\
ज्ञात्वा शास्त्रविधानोक्तं कर्म कर्तुमिहार्हसि~॥ - भगवद्गीता - 16.24
\end{verse}
तस्मात् यदा कार्याकार्ययोर्विषये सन्देहो भवति तदा शास्त्रं प्रमाणम्~। तत्र स्थितेन वाक्येन यत् प्रतिपाद्यते तदेव कार्यं कर्तुमर्हसि इति गीतोपदेश:~। 

“हृद्यपेक्षयातु मनुष्याधिकारत्वात्” (ब्र. सू – 1.3.25) इति भगवान् बादरायण: मनुष्याणां शास्त्राधिकारित्वं स्पष्टीचकार~। अङ्गुष्ठपरिमाणं हृदयं मनुष्याणामेवेति तस्यैव अधिकार:~। हृत् नाम मन:~। स्वान्तं हृन्मानसं मन: इति चामर:~। विशिष्टविचारशक्तिसम्पन्नं मन: मनुष्याणामेवेति तस्यैव शास्त्रे अधिकार:~। 

एवं विवेकशक्तिसम्पन्नं ऊहापोहकुशलं मानवं आश्रित्य शास्त्राणि प्रवर्तन्ते~। दुर्लभं मानवजन्म सफलीकर्तुं शास्त्राणि साक्षात् प्रणाड्या वा तमेव बहुधा बोधयन्ति~। विशेषत: न्यायशास्त्रं मानवमनस: विवेचनशक्ते: संवर्धनाय प्रवृत्तमिति प्रसिद्धम्~। मुनिना अक्षपादेन प्रणीतं न्यायदर्शनं, महर्षिणा कणादेन प्रणितं वैशेषिकदर्शनं, तयो: व्याख्यानं भाष्यग्रन्था: टीकोपटीकाश्च कालक्रमेण तर्कशास्त्रमिति प्रथितमभूत्~। कालक्रमेण युक्तिप्रधानानि सांख्यबौद्धजैनादीन्यपि दर्शनानि तर्कशास्त्रोपलक्षितानि सञ्जातानि~। 

औपनिषदं अद्वितीयमात्मतत्त्वं केवलं शुष्कतर्कविषयो न भवतीति विवक्षति –
\begin{verse}
नैषा तर्केण मतिरापनेया प्रोक्तान्येनैव सुज्ञानाय प्रेष्ठ~। \\
यां त्वमापः सत्यधृतिर्बतासि त्वादृङ्नो भूयान्नचिकेतः प्रष्टा~॥ - कठोपनिषदत् - 1.2.9
\end{verse}
एषा मति: श्रुतिशिरोवचननिचयनिर्णीता अद्वितीयब्रह्मविषया बुद्धि: तर्केण केवलेन\break आगममन्तरेण युक्त्या प्राप्तुं न शक्या~। श्रोततेनैव वचसा आत्मयाथात्म्यविज्ञानप्राप्ति:~। अन्योऽप्यर्थ: - गुरुवेदान्तवचन- श्रवण-मनन-निधिध्यासनेन समासाधिता अनुभवपद्धति\-मध्यस्था अहं ब्रह्मास्मीति अवगतिरूपा एषा मति: तर्केण श्रुतिबलशून्येन अपनेतुं बाधितुं निराकर्तुं वा न शक्या~। प्रमाणसिद्धत्त्वात् अनुभवगोचरत्त्वाच्च~। 

अद्वैतानुभूति: मनुष्यजन्मन: परमं चरमं च लक्ष्यमिति औपनिषत् सिद्धान्त:~। सैव मुक्तिरित्यभिधीयते~। सा च सुदुर्लभा~। देहेन्द्रियमनोबुद्ध्यादिषु अनात्मसु आत्मबुद्धि: अनाद्यनिर्वचनीया अविद्याकल्पिता अनादिकालात् प्रवृता दृढिष्ठा च विद्यते~। चक्षुरादीनि इन्द्रियाण्यपि निसर्गत: बहिर्मुखानि रूपरसादिबाह्यविषयानेव अनायासेनावगमयन्ति~। महता प्रयत्नेनापि प्रत्यगात्मविषयाणि न भवन्ति~। प्रत्यक्षादिसिद्धलौकिकविषयेषु मन: लगति~। बाह्येन्द्रियै: प्रचलितं मन: आत्मनि न समादधाति~। अत: सर्वेषां दर्शनानां समन्वयेन सारभूतस्य तत्त्वस्य विषयस्य च स्वीकारेण विचार: प्रवर्तनीय:~। ऋषिप्रणीतं सर्वमपि शास्त्रम् उपादेयमेव~। तथापि हंसक्षीरन्यायेन सार एव ग्राह्य:~। मिथ्याज्ञानं रागद्वेषजननद्वारा कर्म कारणं, पश्चात् जन्म हेतु:, तदनु सुखदु:खकारणमिति विवक्षया सूत्रं प्रणीतं गौतमेन~। इदं च सूत्रं सगौरवं स्मरन्ति भाष्यकृत: शारीरकमीमांसायां चतुर्थे समन्वयाधिकरणे~। तथाहि आचार्यप्रणीतं न्यायोपबृंहितं सूत्रं “दु:खजन्मप्रवृत्तिदोषमिथ्याज्ञानानाम् उत्तरोत्तरापाये तदनन्तरापायादपवर्ग:” (न्या. द – 1.1.2) इति~। कारणनाशात् कार्यनाश: इति सर्वसम्मता युक्ति:~। जन्मनाशे सुखदु:खयो: नाश:~। कर्मनाशे जन्मनाश:~। रागद्वेशयो: नाशे कर्म नाश:~। अज्ञाननाशे रगद्वेषयो: नाश:~। अज्ञानं तु अद्वैतमात्मस्वरूपम् आच्छादयति~। अत एव द्वैतं भासते~। सति द्वैते रागद्वेषौ प्रभवत:~। इष्टे राग: अनिष्टे द्वेष:~। तत: कर्म पश्चात् जन्म तदनु सुखदु:खानुभव: संसार:~। अनेन अक्षपादीयं दर्शनं युक्तियुक्तं सुग्राह्यमिति समसूचि~। “यतोऽभ्युदयनि:श्रेयससिद्धि: स धर्म:” इति महर्षिणा कणादेन प्रणीतम्~। इदं च धर्मलक्षणम् आस्तिका: सर्वे सशिर:कम्पं उररीकुर्वन्ति~। “काणादं पाणिनीयं च सर्वशास्त्रोपकारकम्” इति प्रथितमस्ति~। पदवाक्यप्रमाणज्ञा: वेदान्तसूत्रभाष्यप्रणेतार: वेदान्तान् व्याचचक्षिरे~। मधुसूदनसरस्वत्याचार्या: समेषां वेदशास्त्रपुराणादीनाम् अद्वैते पर्यवसानमिति प्रस्थानभेदाख्ये ग्रन्थे निरूपयामासु:~। आगमोत्पत्तिभ्यां तत्त्वनिर्णय: कार्य: इति मनुनाप्युक्तम् –
\begin{verse}
आर्षं धर्मोपदेशं च वेदशास्त्राविरोधिना~। \\
यस्तर्केणानुसन्धत्ते स धर्मं वेद नेतरः~॥ - मनु० 12. 106
\end{verse}
उक्तं भगवत्पादाचार्यै: साधनापञ्चके –
\begin{verse}
वाक्यार्थश्च विचार्यतां श्रुतिशिरःपक्षः समाश्रीयतां\\
दुस्तर्कात् सुविरम्यतां श्रुतिमतस्तर्कोऽनुसंधीयताम्~। \\  
ब्रम्हास्मीति विभाव्यतामहरहर्गर्वः परित्यज्यताम्\\  
देहेऽहम्मतिरुज्झ्यतां बुधजनैर्वादः परित्यज्यताम्~॥\\
\hspace{5cm} - साधनापञ्चकम् - श्लो. 3
\end{verse}
श्रुतिश्च भवति 'आत्मा वा अरे दृष्टव्यः श्रोतव्य: मन्तव्य: निदिध्यासितव्य:' (बृ. उ – 2.4.5) इति आत्मदर्शनमुद्दिश्य श्रवणमनननिधिध्यासनानि विहितानि~। तत्र प्रमाणासम्भावना निवृत्तये असकृत् श्रवणं विहितम्~। श्रुतस्य अर्थस्य दार्ढ्याय प्रमेयासम्भावनाव्यावर्तनाय मननं विहितम्~। मननं नाम युक्तिभि: अनुचिन्तनम्~। अनुमानम् अर्थापत्तिर्वा युक्तिबलेन विवक्षिता~। पण्डितो मेधावी (छा. उ – 14.2) इत्यपि श्रुति: पण्डितशब्देन शाब्दज्ञानम्, मेधावी इत्यनेन युक्तिसिद्धं निश्चयात्मकं ज्ञानम् विवक्षितम्~। अन्वयव्यतिरेकाभ्यां शरीरत्रयातिरिक्त: अवस्थात्रयातिरिक्त: असङ्ग: स्वप्रकाश: आत्मा इति निर्णय: युक्त्या निश्चेतुं शक्यते~। अत: आत्मदर्शने अस्ति तर्कस्यापि महद्योगदानम्~। तर्काप्रतिष्ठानादिति बादरायणप्रणीतसूत्रेण श्रुतिविरुद्ध: तर्क: अप्रतिष्ठित:~। यतो हि शब्दस्पर्शरूपरसादिरहितम् आत्मतत्त्वं बाह्येन्द्रियागोचरं मनसापि अचिन्त्यम्~। अत: अचिन्त्यविषये केवल: तर्क:
\begin{verse}
निश्चयज्ञानं न जनयति~। तथाचोक्तम् अभियुक्तै: -\\
\textit{अचिन्त्या: खलु ये भावा: न तांस्तर्केण योजयेत्~। }\\
प्रकृतिभ्यः परं यत्तु तदचिन्त्यस्य लक्षणम्~॥ - ब्रह्मवैवर्तपुराणम्
\end{verse}
तस्मात् अभौतिकं ब्रह्मस्वरूपं धर्मस्वरूपं वा वेदैकवेद्यं तदनुसारिणा तर्केण निश्चयबुद्धि: सम्पादनीया~। अत: आगमाविरोधितर्केण ब्रह्मात्मानुभूति: 

\centerline{$*****$}

नितान्तं सुहृद्वरेण्या: विद्वांस: गङ्गाधरभट्टवर्या: तर्कशास्त्रे अन्यान्यदर्शनेषु च कृतभूरिपरिश्रमा:, आधुनिकविषयेष्वपि तलस्पर्शिज्ञानसम्पन्ना:~। अत: सन्मार्गे छात्रान् सुहृदश्च प्रेरयन्ति~। दुर्मार्गात् निवर्तयन्ति~। तस्मात् तेषाम् अभिनन्दनावसरे इदम् अभिनन्दनकुसुमं\break लेखनं समर्प्यते~। 

\articleend
}
