{\fontsize{14}{16}\selectfont
\pretolerance=9000
\tolerance=200
\chapter{ಕಲೌ ವಾಗ್ಭಟನಾಮಾ} 

\begin{center}
\Authorline{ಡಾ~। ಪ್ರಶಾಂತ. ಎಮ್. ಡಿ}
\smallskip

ಆಯುರ್ವೇದ ವೈದ್ಯರು\\
ಮೈಸೂರು
\addrule
\end{center}

ವಾಗ್ಭಟ ಎಂಬ ಶೇಷ್ಠ ವೈದ್ಯನ ಹೆಸರು ಆಯುರ್ವೇದದ ಕ್ಷೇತ್ರದಲ್ಲಿ ಸುಪ್ರಸಿದ್ಧ. ಪ್ರಾಚೀನ ವೈದ್ಯ ವಿಶಾರದರು ಈ ವಾಗ್ಭಟನ ಬಗ್ಗೆ ಅಂದಿನ ಕಾಲದಿಂದಲೂ ಪ್ರಚಲಿತವಾಗಿದ್ದ ಕಥೆಯೊಂದನ್ನು ಹೇಳುತ್ತಾರೆ. ಆದರಂತೆ ಒಂದೊಮ್ಮೆ ಆದಿದೇವನಾದ  ಸಾಕ್ಷಾತ್ ಶ್ರೀ ಮನ್ನಾರಾಯಣನ ಅಪರಾವತಾರವಾದ ಧನ್ವಂತರಿಯು ಕಲಿಕಾಲವಶದಿಂದಾಗಿ ಈ ಭಾರತವರ್ಷದಲ್ಲಿ ಎಷ್ಟು ಜನ ಆರೋಗ್ಯವಂತರಾಗಿಯೇ ಉಳಿದಿದ್ದಾರೆ? ಯಾವ ರೋಗನಿವಾರಣಾ ಉಪಾಯವನ್ನು ಜನಸಾಮಾನ್ಯರು ಅನುಸರಿಸುತ್ತಿದ್ದಾರೆ? ಅಂತಹ ಶ್ರೇಷ್ಠ ವೈದ್ಯರಾದರೂ ಯಾರಿದ್ದಾರೆ? ಎಂದು ಪರೀಕ್ಷಿಸಬೇಕು ಎಂದು ಒಂದು ಪಕ್ಷಿಯರೂಪವನ್ನು ತಾಳಿ ಭಾರತವರ್ಷದಲ್ಲೇ ಶ್ರೇಷ್ಠರೆಂದು ಖ್ಯಾತರಾಗಿರುವ ವೈದ್ಯರ ಮನೆ ಮುಂದಿರುವ ಮರದ ಮೇಲೆ ಕುಳಿತು,“ಕೋsರುಕ್ ಕೋsರುಕ್” (ಕಃ ಅರುಕ್ = ಯಾರು ರೋಗಿ) ಎಂದು ಮೂರುಸಲ ಕೂಗಿತಂತೆ. ಅದಕ್ಕೆ ಯಾವ ಉತ್ತರವೂ ಸಿಗುವುದಿಲ್ಲ. ಎಲ್ಲಾ ಕಡೆ ಸುತ್ತಾಡಿ ನಿರಾಸೆಯಿಂದ ಕೊನೆಗೆ ಸಿಂಧೂ ದೇಶದ ಒಂದು ಮನೆ ಮುಂದಿನ ಫಲ ಪುಷ್ಪಗಳಿಂದ ಸಂಪದ್ಭರಿತವಾದ ಮರದ ಮೇಲೆ ಕುಳಿತು ಮತ್ತೆ ಕೋsರುಕ್ ಕೋsರುಕ್ ಎಂದು ಕೂಗಿತು. ರೋಗಿಯ ಶುಶ್ರೂಷೆಯಲ್ಲಿದ್ದ ವಾಗ್ಭಟ ಇಂತಹ ಸ್ಪಷ್ಟವಾದ ವಾಣಿಯನ್ನು ಕೇಳಿ ಯಾರಿರಬಹುದು ಎಂದು ನೋಡುತ್ತಾ ಹೊರಕ್ಕೆ ಬಂದನು. ಮರದ ಮೇಲಿನ ಪಕ್ಷಿಯು ಈ ಮಾತನ್ನು ಹೇಳುತ್ತಿದೆ ಎಂದು ತಿಳಿದು ಇದು ಸಾಮಾನ್ಯ ಪಕ್ಷಿಯಲ್ಲ, ಪಕ್ಷಿರೂಪದಲ್ಲಿರುವ ದೇವತೆಯೇ ಆಗಿರಬೇಕೆಂದು ಒಂದು ಬಟ್ಟಲಿನಲ್ಲಿ ನೀರು ಹಾಗೂ ಹಾಲನ್ನು ತಂದಿಟ್ಟನು. ಆ ಪಕ್ಷಿಯು ನೋಡಿಯೂ ನೋಡದಂತೆ ಇದ್ದು ಮತ್ತೆ ಕೋsರುಕ್ ಕೋsರುಕ್ ಎಂದು ಕೂಗಿತು. ತನ್ನ ಪ್ರಶ್ನೆಗೆ ಉತ್ತರ ಸಿಗುವವರೆಗೆ ಇದು ಹಣ್ಣನ್ನು ತಿನ್ನುವುದಿಲ್ಲ ಎಂದು ತಿಳಿದು “ ಹಿತಭುಕ್, ಮಿತಭುಕ್, ಅಶಾಕಭುಕ್” ಅಂದರೆ ಹಿತಾಹಾರ, ಮಿತಾಹಾರ ಹಾಗೂ ಅಲ್ಪ ತರಕಾರಿ ಸೇವಿಸುವವನು ಎಂದು ಹೇಳಿದನು. ತಕ್ಷಣವೇ ಈ ಉತ್ತರದಿಂದ ಸಂತುಷ್ಟಗೊಂಡ ಪಕ್ಷಿಯು ಖುಷಿಯಿಂದ ಬಂದು ಹಣ್ಣನ್ನು ತಿಂದು ನಾನು ಸದ್ವೈದ್ಯರ ಪರೀಕ್ಷೆಗೆಂದು ಬಂದ ಸಾಕ್ಷಾತ್ ಧನ್ವಂತರಿ ಎಂದು ತನ್ನ ನೈಜ ಸ್ವರೂಪವನ್ನು ತೋರಿಸಿ ನಿನ್ನ ಆಯುರ್ವೇದದ ಜ್ಞಾನವು ಅಪಾರವಾದದ್ದು, ನಿನ್ನಿಂದ ಆಯುರ್ವೇದದ ಅಷ್ಟಾಂಗಗಳ ಕುರಿತ ಗ್ರಂಥವು ಬರೆಯಲ್ಪಡಲಿ ಎಂದು ಆಶೀರ್ವದಿಸಿ ಅದೃಶ್ಯನಾದನಂತೆ. ಇದೇ ವಾಗ್ಭಟನು “ಅಷ್ಟಾಂಗ ಹೃದಯವೆಂಬ” ಗ್ರಂಥವನ್ನು ಬರೆದು ಕಲಿಯುಗದಲ್ಲೆ ಶ್ರೇಷ್ಠ ವೈದ್ಯನೆಂಬ ಹೆಸರು ಪಡೆದನಂತೆ. ಆದ್ದರಿಂದಲೇ ಒಂದು ಉಕ್ತಿ ಪ್ರಸಿದ್ಧವಾಗಿದೆ 
\begin{verse}
ಅತ್ರಿ: ಕೃತಯುಗೇ ವೈದ್ಯೋ ದ್ವಾಪರೇ ಸುಶ್ರುತೋ ಮತಃ~। \\
ಕಲೌ ವಾಗ್ಭಟನಾಮಾ ಚ ತಿಸ್ರ ಏವ ಯುಗೇ ಯುಗೇ~॥ ಎಂದು.
\end{verse}
ಕಲಿಯುಗದಲ್ಲಿನ ಜನರ ಬುದ್ಧಿ ಮತ್ತೆಗೆ ಅನುಗುಣವಾಗಿ ಆಯುರ್ವೇದದ ಎಂಟೂ ಅಂಗಗಳಲ್ಲೂ ಸಾರಭೂತವಾದ ವಿಷಯಗಳನ್ನು ಕ್ರೋಢೀಕರಿಸಿ ಅಷ್ಟಾಂಗಹೃದಯ ಎಂಬ ಗ್ರಂಥವನ್ನು ವಾಗ್ಭಟಾಚಾರ್ಯರು ಬರೆದರು. ಸಂಪೂರ್ಣ ಅಷ್ಟಾಂಗ\-ಹೃದಯ ಗ್ರಂಥವು 120 ಅಧ್ಯಾಯ ಹಾಗೂ 7471 ಸಂಖ್ಯೆಯ ಶ್ಲೋಕಗಳನ್ನು ಒಳ\-ಗೊಂಡಿದೆ. ಅಷ್ಟಾಂಗಹೃದಯವು 6 ಸ್ಥಾನಗಳನ್ನು ಹೊಂದಿದ್ದು ಅದರಲ್ಲಿ ಮೊದಲನೆಯದು ಸೂತ್ರಸ್ಥಾನ. ಮೂವತ್ತು ಅಧ್ಯಾಯಗಳುಳ್ಳ ಸೂತ್ರಸ್ಥಾನವು ಅದ್ಭುತವಾದ ವಿಷಯ\-ಗಳ ಜೋಡಣೆ\-ಯನ್ನು ಹೊಂದಿದ್ದು ದಾರದಲ್ಲಿ ಮಣಿಗಳನ್ನು ಜೋಡಿಸಿದಂತೆ ಕ್ರಮಬದ್ಧ\-ವಾಗಿದೆ.

ಇದೇ ಕಾರಣದಿಂದಾಗಿ ಪ್ರಾಚೀನ ವೈದ್ಯರು “ಸೂತ್ರ ಸ್ಥಾನೇ ತು ವಾಗ್ಭಟಃ” \enginline{-} (ಶ್ರೇಷ್ಠಃ)ಎಂಬ ಬಿರುದನ್ನೇ ಕೊಟ್ಟಿದ್ದಾರೆ. ಅಷ್ಟಾಂಗಹೃದಯದ ಸಂಪೂರ್ಣಗ್ರಂಥದ ವಿಷಯಗಳನ್ನು ಸೂತ್ರರೂಪದಲ್ಲಿ ಈ ಸ್ಥಾನದಲ್ಲಿ ಬರೆಯಲಾಗಿದೆ. ಇದೇ ಸೂತ್ರಗಳ ವಿಸ್ತಾರವನ್ನೇ ನಾವು ಇತರ ಸ್ಥಾನಗಳಲ್ಲಿ ಕಾಣಬಹುದಾಗಿದೆ. ಇದನ್ನೇ ಸೂತ್ರಸ್ಥಾನದ ಕೊನೆಯ ಅಧ್ಯಾಯದ ಕೊನೆಯ ಶ್ಲೋಕದಲ್ಲಿ 
\begin{verse}
“ಅತ್ರಾರ್ಥಾ: ಸೂತ್ರಿತಾಃ ಸೂಕ್ಷ್ಮಾ: ಪ್ರತನ್ಯಂತೇ ಹಿ ಸರ್ವತಃ” 
\end{verse}
ಎಂದು ಹೇಳಿರುವುದು. ಎಲ್ಲಾ ಗ್ರಂಥಕಾರರಂತೆ ವಾಗ್ಭಟಾಚಾರ್ಯರೂ ಕೂಡಾ ಪ್ರಾರ್ಥನಾ ಶ್ಲೋಕದಿಂದ ಅಷ್ಟಾಂಗ ಹೃದಯವನ್ನು ಪ್ರಾರಂಭಿಸಿದ್ದಾರೆ. ಇದಂತೂ ಅತಿ ಸುಂದರವಾದ ಶ್ಲೋಕವಾಗಿದೆ. ಆ ಶ್ಲೋಕವು ಹೀಗಿದೆ \
\begin{verse}
ರಾಗಾದಿರೋಗಾನ್ ಸತನಾನುಷಕ್ತಾನ್\\
ಅಶೇಷಕಾಯಪ್ರಸೃತಾನಶೇಷಾನ್~। \\
ಔತ್ಸುಕ್ಯಮೋಹಾರತಿದಾನ್‍ ಜಘಾನ\\
ಯೋಪೂರ್ವವೈದ್ಯಾಯ ನವೋಸ್ತು ತಸ್ಮೈ~॥
\end{verse}
ಈ ಶ್ಲೋಕದಲ್ಲಿ ಅಪೂರ್ವ ಅಥವಾ ಶ್ರೇಷ್ಠ ವೈದ್ಯನಿಗೆ ನಮಸ್ಕಾರ ಎಂದಿದೆ. ಆದರೆ ಯಾರು ಆ ಶ್ರೇಷ್ಠ ವೈದ್ಯ ಎಂದರೆ ಸಕಲ ಜೀವಿಗಳಲ್ಲಿ ಎಲ್ಲಾ ಕಾಲದಲ್ಲೂ ಯಾರನ್ನೂ ಯಾವತ್ತೂ ಬಿಟ್ಟಿರದ ಔತ್ಸುಕ್ಯ, ಮೋಹ, ಅರತಿ ಮುಂತಾದ ತೊಂದರೆಗಳಿಗೆ ಕಾರಣವಾದ ರಾಗವೇ ಮೊದಲಾದವುಗಳನ್ನು (ರಾಗ, ದ್ವೇಷ, ಲೋಭ ಇತ್ಯಾದಿ) ನಾಶಪಡಿಸುವ ಸಾಮರ್ಥ್ಯವಿದೆಯೋ ಅಂತಹ ವೈದ್ಯನೇ ಅಪೂರ್ವ ವೈದ್ಯ. ಅವನಿಗೆ ನಮಸ್ಕಾರ.

ಈ ಪ್ರಾರ್ಥನಾ ಶ್ಲೋಕ ಅನೇಕ ಆಳವಾದ ಅರ್ಥವನ್ನು ಹೊಂದಿದೆ. ಒಬ್ಬ ಆಯುರ್ವೇದ ವಿದ್ಯಾರ್ಥಿಯಾಗಿ  ಈ ಶ್ಲೋಕವನ್ನು ಓದಿದರೆ ತನ್ನ ವೈದ್ಯ ವೃತ್ತಿಯ ಗುರಿಯು ರೋಗಗಳ ಮೂಲ ಕಾರಣವನ್ನು ಹುಡುಕಿ ಆ ಕಾರಣವನ್ನೇ ತೆಗೆಯುವುದು. ಅದೇ ನಿಜವಾದ ಚಿಕಿತ್ಸೆ, ಅದನ್ನೇ ಮಾಡಬೇಕಾದ್ದು ಎಂದು ತಿಳಿಯುತ್ತದೆ. ಈ ಒಂದು ಶ್ಲೋಕವು ಇಡಿ ಗ್ರಂಥದ ಉದ್ದೇಶವನ್ನೂ ಹೇಳುತ್ತದೆ. ಔತ್ಸುಕ್ಯ ಮೋಹಾದಿಗಳನ್ನೂ ನಿವಾರಿಸುವ ಉಪಾಯವನ್ನೊಳಗೊಂಡಂತಹ ಆಯುರ್ವೇದದ ಗ್ರಂಥವಿದು ಎನ್ನುವುದು ತಿಳಿಯುತ್ತದೆ. ಈ ಶ್ಲೋಕವು ಪ್ರಾರ್ಥನಾ ಶ್ಲೋಕವಾಗಿದ್ದರೂ ಯಾವುದೇ ಜಾತಿ, ಮತ, ಪಂಥ ಧರ್ಮಗಳ ಉಲ್ಲೇಖವಿಲ್ಲದೇ ಅಂತಹ ಶ್ರೇಷ್ಠ ಗುಣಗಳುಳ್ಳ ವೈದ್ಯನಿಗೆ ನಮಸ್ಕಾರ ಎಂದಿರುವುದರಿಂದ ಈ ಆಯುರ್ವೇದವೂ ಯಾವುದೇ ಜಾತಿ, ಮತ, ಧರ್ಮಾತೀತವಾದದ್ದು ಎಂದೂ ತಿಳಿಯುತ್ತದೆ.

ಈ ಶ್ಲೋಕವು ರೋಗಗಳ ಮೂಲಕಾರಣವನ್ನು ಹೇಳುತ್ತದೆ. “ರಾಗ” ಇದು ಎಲ್ಲಾ ತೊಂದರೆಗಳಿಗೂ ಅಯುರ್ವೇದದ ಪ್ರಕಾರ ಮೂಲವಾದ ಕಾರಣ. ಹಾಗಾದರೆ ಏನಿದು “ರಾಗ” ? ಇದು ನನಗೆ ಬೇಕು, ಇದು ತನಗೆ ಬೇಕು ಎಂಬ ಬಯಕೆಯನ್ನು ಇಚ್ಛೆ ಎನ್ನಬಹುದು. ಇದರ ಮುಂದುವರೆದ ಅವಸ್ಥೆ. ಅಂದರೆ ಇದು ನನಗೆ ಬೇಕೇ ಬೇಕು ಎಂಬ ತೀವ್ರತರವಾದ ಬಯಕೆ, ಹಾಗೂ ಸಿಗದಿದ್ದಾಗ ಆಗುವ ತೊಳಲಾಟ ಇವೆಲ್ಲಾ “ರಾಗ” ಎಂದು ಕರೆಯಲ್ಪಡುತ್ತದೆ.

ಎಲ್ಲಾ ವ್ಯಾಧಿಗಳ ಮೂಲ ಇದರಲ್ಲೇ ಇದೆ ಎನ್ನುವುದು ಆಯುರ್ವೇದದ ಮತ. ಶಾರೀರಿಕವಾಗಲಿ ಮಾನಸಿಕವಾಗಲೀ ಎರಡೂ ಪ್ರಕಾರದ ಕಾಯಿಲೆಗಳಿಗೆ ಈ “ರಾಗ”ವು ಕಾರಣವಾಗುತ್ತದೆ. ಉದಾ: ಕರಿದ ತಿಂಡಿಯನ್ನು ತಿಂದರೆ ಎದೆ ಉರಿ ಬರುತ್ತದೆ ಎಂದು ತಿಳಿದಿರುತ್ತದೆ. ವೈದ್ಯರೂ ಸೇವಿಸಬೇಡಿ ಎಂದಿರುತ್ತಾರೆ. ಆದರೆ ಅದು ತನಗೆ ಬೇಕೇ ಬೇಕು ಅನ್ನುವ ಬಯಕೆ ಉತ್ಕಟವಾಗಿ, ಸೇವಿಸಿದರೆ ಕಾಯಿಲೆ ಉಲ್ಬಣವಾಗುತ್ತದೆ. ಹೇಗೋ ಆ ಆಸೆಯನ್ನು ಅದುಮಿ ಸೇವಿಸದಿದ್ದರೆ ಆ ತೊಂದರೆ ಬರುವುದಿಲ್ಲ. ಆದರೆ ತಾನು ಬಯಸಿದ್ದು ಸೇವಿಸಲಾಗಲಿಲ್ಲವಲ್ಲ ಎಂಬ ತೊಳಲಾಟ ಇನ್ನಾವುದೋ ತೊಂದರೆಗೆ  ಕಾರಣ\-ವಾಗುತ್ತದೆ. ಇಚ್ಛೆ ಒಳ್ಳೆಯದು ಆದರೆ ಆ ಇಚ್ಛೆ ರಾಗವಾಗಿ ಬದಲಾದರೆ ಕಾಯಿಲೆಗೆ ಕಾರಣ\-ವಾಗುತ್ತದೆ. ಇಚ್ಛೆ ಹಾಗು ರಾಗ ಇವುಗಳ ನಡುವಿನ ಅಂತರ ಅತೀ ಸೂಕ್ಷ್ಮ. ಎಷ್ಟೋ ಸಲ ನಮಗರಿವಿಲ್ಲದೇನೇ ಇಚ್ಛೆಯ ಅವಸ್ಥೆ ದಾಟಿ “ರಾಗ” ಆವರಿಸಿಬಿಟ್ಟಿರುತ್ತದೆ. ಈ ರಾಗವೆಂಬ ಬೀಜವೇ ರೋಗವೆಂಬ ಮರವಾಗುವುದರಿಂದ ರಾಗವೆಂಬ ಬೀಜ ಮೊಳಕೆಯೊಡೆಯುವ ಹಂತದಲ್ಲೇ ಅದನ್ನು ತಡೆಯುವ ಸಾಮರ್ಥ್ಯವುಳ್ಳ ವೈದ್ಯನೇ ಅಪೂರ್ವ ವೈದ್ಯ. ಈ ರಾಗವನ್ನುವುದು ಅಷ್ಟೊಂದು ತೊಂದರೆ ಕಾರಿಯೇ? ಹೌದಾದರೆ ಯಾಕೆ ಎಂಬುದನ್ನು ನೋಡಬೇಕಲ್ಲವೇ?

ಹೌದು ಇದು ಅಷ್ಟೊಂದು ತೊಂದರೆಕಾರಿ. ಏಕೆಂದರೆ ಇದು ನಿಯಂತ್ರಣವಿಲ್ಲದಿದ್ದರೆ  ಮುಂದಿನ ಅವಸ್ಥೆಯಾದ ಕ್ರೋಧ, ಮೋಹ ಮುಂತಾದ ಇನ್ನೂ ಗಂಭೀರ ಅವಸ್ಥೆಗಳಿಗೆ ತಳ್ಳುತ್ತದೆ. ಇಷ್ಟು ಮಾತ್ರ ಅಲ್ಲ ಈ “ರಾಗ”ದ ಸ್ವರೂಪವಾದರೂ ಯಾವ ರೀತಿ ಎಂದರೆ ಇದು ಯಾವಾಗಲೂ ನಮ್ಮ ಜೊತೆಗೆ ಇರುತ್ತದೆ. ಒಂದುಕ್ಷಣವೂ ಬಿಟ್ಟಿರುವುದಿಲ್ಲಾ \break ನಮ್ಮನ್ನು ಜೀವನದ ಒಂದೊಂದು ಕೂಡಾ ಯಾವುದಾದರೂ ಒಂದು ವಸ್ತು/ವ್ಯಕ್ತಿ/ವಿಷ\-ಯದ ಬಗೆಗಿನ ಅತಿಯಾದ ಅವಲಂಬನೆ ಇದ್ದೇ ಇರುತ್ತದೆ. ಆದರಿಂದ ತಪ್ಪಿಸಿಕೊಳ್ಳುವುದು ಸುಲಭದ ಮಾತಲ್ಲ. ಇದು ಇಷ್ಟಕ್ಕೆ ನಿಲ್ಲುವುದೂ ಇಲ್ಲ.

ಇದರ ಮುಂದುವರೆದ ರೂಪವಾಗಿ ಮನುಷ್ಯನು ತನ್ನ ವಿವೇಚನಾ ಶಕ್ತಿಯನ್ನೇ ಕಳೆದು\-ಕೊಳ್ಳುತ್ತಾನೆ. ಈ ರಾಗಾದಿಗಳು ಕೇವಲ ಮನುಷ್ಯರಲ್ಲಿ ಮಾತ್ರವಲ್ಲ. ಜೀವವಿರುವ ಪ್ರತಿಯೊಂದರಲ್ಲೂ ಕಂಡುಬರುತ್ತವೆ.  ಸಕಲಭೂತಗಳಿಗೂ ಕಾರಣೀಭೂತವಾದ ರಾಗಾದಿಗಳನ್ನು ಜಯಿಸುವುದೇ ಆಯುರ್ವೇದದ ಉದ್ದೇಶ ಎಂಬುದನ್ನು ವಾಗ್ಭಟನು ಪ್ರಾರಂಭದ ಪ್ರಾರ್ಥನಾ ಶ್ಲೋಕದಲ್ಲೇ ವಿವರಿಸಿರುವುದು ಈ ಗ್ರಂಥದ ವೈಶಿಷ್ಠ್ಯ.

ರಾಗ ಎನ್ನುವುದು ರೋಗಗಳಿಗೆ ಕಾರಣ ಆನ್ನುವುದೇನೋ ತಿಳಿಯಿತು. ಆದರೆ ಈ ರಾಗವು ಉಂಟಾಗದೇ ಇರುವ ಹಾಗೆ ಏನು ಮಾಡಬೇಕು? ಈ ಪ್ರಶ್ನೆಯ ಉತ್ತರವು ಇಡೀ ಗ್ರಂಥದಲ್ಲಿದೆ. ರೋಗಕಾರಣಗಳನ್ನು ಹೇಳಬೇಕಾದರೆ “ಪ್ರಜ್ಞಾಪರಾಧ” ಎಂಬುವುದು ಒಂದು ಕಾರಣವಾಗಿ ಹೇಳಿದ್ದಾರೆ. ಪ್ರಜ್ಞೆ ಅಂದರೆ ವಿವೇಚನೆ. ಈ ವಿವೇಚನೆಯಲ್ಲಿ ಆಗುವ ತಪ್ಪುಗಳಿಗೆ ಪ್ರಜ್ಞಾಪರಾಧ ಎಂದು ಹೆಸರು. ವಿವೇಚನೆಯಲ್ಲಿನ ತಪ್ಪು 3 ಕಾರಣಗಳಿಂದಾಗುತ್ತದೆ. ಮೊದಲನೆಯದು ಯಾವುದು ಸರಿ ಯಾವುದು ತಪ್ಪು ಎಂಬುವುದು ಗೊತ್ತಿಲ್ಲದಿರುವಿಕೆ, ಇದಕ್ಕೆ “ಧೀ ವಿಭ್ರಂಶ” ಎನ್ನುತ್ತಾರೆ. ಸದ್ಗ್ರಂಥಾಧ್ಯಯನ, ಸಜ್ಜನರ ಸಹವಾಸ ಇತ್ಯಾದಿಗಳು ಸರಿ ಮತ್ತು ತಪ್ಪುಗಳನ್ನು ಸರಿಯಾಗಿ ಗುರುತಿಸುವಲ್ಲಿ ಸಹಕಾರಿಯಾಗುತ್ತವೆ. ಕೆಲವೊಮ್ಮೆ ಯಾವುದು ಸರಿ ಯಾವುದು ತಪ್ಪು ಎಂದು ತಿಳಿದಿದ್ದರೂ ಕೂಡಾ ಅದನ್ನು ಅನುಸರಿಸುವಲ್ಲಿ ತಮ್ಮೊಳಗಿನ ದೌರ್ಬಲ್ಯದಿಂದ ಎಡವುತ್ತೇವೆ. ಉದಾ: ಛಳಿಗಾಲದಲ್ಲಿ  ಐಸ್‍ಕ್ರೀಮ್ ಸೇವನೆ ಹಿತಕರವಲ್ಲ ಎಂದು ಗೊತ್ತಿರುತ್ತದೆ. ಆದರೂ ಎದುರು ಐಸ್‍ಕ್ರೀಮ್ ಇರಬೇಕಾದರೆ ತಡೆಯಲಾಗದೇ ತಿಂದು ಬಿಡುತ್ತೇವೆ. ಇದಕ್ಕೆ “ಧೃತಿ\-ವಿಭ್ರಂಶ” ಎನ್ನುತ್ತೇವೆ. ಪದೇ ಪದೇ ಸರಿ ತಪ್ಪುಗಳ ವಿವೇಚನೆಯನ್ನು ಮನಸ್ಸಿಗೆ ತಂದುಕೊಳ್ಳುವುದು. ಇಂದ್ರಿಯಗಳ ನಿಯಂತ್ರಣ ಮಾಡುವುದರ ಮುಖಾಂತರ ಈ ತಪ್ಪನ್ನು ಸರಿಪಡಿಸ\-\break ಬಹುದು. ತಿಳಿದೋ ತಿಳಿಯದೆಯೋ ಅನೇಕ ತಪ್ಪುಗಳು ನಮ್ಮಿಂದ ಆಗುತ್ತವೆ. ಒಮ್ಮೆ ಆದ ತಪ್ಪಿನಿಂದ ಪಾಠ ಕಲಿಯಬೇಕಾದದ್ದು ನಮ್ಮ ಜವಾಬ್ದಾರಿ ಕೆಲವೊಮ್ಮೆ ಅದನ್ನು ಮರೆತು ಮತ್ತೆ ಅದೇ ತಪ್ಪನ್ನು ಮಾಡುತ್ತೇವೆ. ಇದಕ್ಕೆ ಸ್ಮೃತಿ ಭ್ರಂಶ ಎನ್ನುತ್ತೇವೆ. ಹಿಂದಿನ ಅನುಭವಗಳನ್ನು ಪದೇ ಪದೇ ಮನನ ಮಾಡುವುದರಿಂದ ಈ ತಪ್ಪನ್ನು ಸರಿಪಡಿಸಬಹುದು. ಈ ರೀತಿ ಧೀ, ಧೃತಿ, ಸ್ಮೃತಿ ಇವುಗಳನ್ನು ಹೆಚ್ಚಿಸುತ್ತಾ ಹೋದಂತೆ ವಿಷಯಗಳ ಬಗೆಗಿನ ವಸ್ತುಸ್ಥಿತಿ ಅಥವಾ ಯಥಾರ್ಥಜ್ಞಾನ ನಮಗೆ ಉಂಟಾಗಿ “ರಾಗ” ಬರದಂತೆ ತಡೆಗಟ್ಟಿ ಆರೋಗ್ಯ ಪೂರ್ಣಜೀವನವನ್ನು ನಡೆಸಬಹುದಾಗಿದೆ.

\articleend
}
