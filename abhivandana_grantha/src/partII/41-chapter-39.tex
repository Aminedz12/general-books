{\fontsize{15}{17}\selectfont
\presetvalues
\chapter{शास्त्रे वैरुद्ध्यं साम्यञ्च} 

\begin{center}
\Authorline{वि~। दीपकभट्टः~। }
\smallskip

मैसूरुनगरी
\addrule
\end{center}

बहुत्र शास्त्रेषु वैरुध्यं वर्तत इति प्रतिभाति~। वस्तुतः मुनिवचनेषु बहुत्र वैरुध्यं नास्ति~। एवं तर्हि का वा वस्तुस्थितिरिति विचारयितुकामेन मया  

न्यायवेदान्तमीमांसासाहित्यव्याकरणज्यौतिषशास्त्रेषु किञ्चित् प्रस्तूयते~। 

अर्थवादशब्दश्रवणमात्रेण बहव इत्थं व्याजिहीर्षन्ति - "अर्थवादः पूर्वमीमांसाशास्त्रे, वेदे, वेदभाष्ये वा विद्यते" इति~। वस्तुतः अर्थवादः शास्त्रान्तरेऽपि दृश्यते~। एवं तर्हि अर्थवादो नाम कः इति सोदाहरणमत्र प्रस्तूयते~। कथं भावयेदिति शाब्दीभावनायाः इतिकर्तव्यताकाङ्क्षायाम् अर्थवादज्ञाप्यं प्राशस्त्यम् इतिकर्तव्यतात्वेन अन्वेति इत्युच्यते पूर्वमीमांसाशास्त्रे~। विधिसन्निधौ श्रूयमाणः प्राशस्त्यबोधको वाक्यसमूह एव अर्थवादो भवति~। यथा - "वायव्यं श्वेतमालभेत भूतिकामः" इत्येकं वाक्यं श्रूयते~। इदञ्च वाक्यं तैत्तिरीयसंहितायाः द्वितीयकाण्डे प्रथमप्रश्ने प्रथमानुवाके वर्तते~। वायुः देवता यस्य पशोः सः वायव्यः इत्युच्यते~। स च श्वेतवर्णविशिष्टः भवेत्~। तमालभेत इत्यस्य वाक्यस्य तं संस्पृशेत् इत्यर्थः~। वस्तुतः अत्र यजेत इति पदं न श्रूयते~। एवं सत्यपि वायुदेवताको यागः इति कथं निश्चीयते इति चेत् - द्रव्यदेवतात्मको यागः सर्वत्र श्रूयते~। अत्रापि वायुः देवता, वायव्यः इति पशुरूपं द्रव्यं चोदलेखि~। अनेन कारणेन अस्यापि यागरूपत्वं सिद्ध्यति~। एवञ्च वायव्येन यजेत इत्येवं यागः परिकल्पनीयो भवति~। वायव्यं श्वेतमालभेत इति विधिवाक्यस्य अयमर्थः - धनमिच्छता पुरुषेण श्वेतगुणविशिष्टेन द्रव्येण वायुदेवताको यागः कर्तव्यः इति~। सहस्रसङ्ख्याकेषु देवेषु विद्यमानेषु वायुदेवताक एव यागः कुतोऽनुष्ठेयः ? इति प्रश्नः अत्र जागर्ति~। अयमेव प्रश्नः विधिवाक्यसमनन्तरमित्थं वेदे समाधीयते-

"वायुर्वै क्षेपिष्ठा देवता" इति~। इदं न विधिवाक्यम् अपि तु अर्थवादः एव~। वायुः क्षिप्रगामिनी देवता इति कारणेन वायुदेवताको यागो व्यधायि~। क्षिप्रगामिन्याः देवतायाः यागेन सा देवता क्षिप्रं परितुष्टा भूत्वा, क्षिप्रं फलं प्रदास्यति इत्यर्थवादः~। एवञ्च वायुदेवताकं यागं किमर्थं कुर्यात् इत्याकाङ्क्षायाम्, क्षिप्रफलप्राप्तिरूपं प्रयोजनं प्रदर्श्य, यागे प्रवृत्तिमुत्पादयत्ययमर्थवादः~। अतः "वायुर्वै क्षेपिष्ठा देवता" इति वाक्यमर्थवादो भवति~। अनेनैव कारणेन  विधिवाक्यस्य शेषः अर्थवादः इति व्यवह्रियते~। अत्र क्षेपिष्ठा इति पदेन वायोः क्षिप्रगामित्वरूपं प्राशस्त्यमुपवर्णितं दृश्यते~। विधेः कर्तव्यं यत् प्रवर्तनारूपं कार्यं तत्र प्रकारबोधनमेव इतिकर्तव्यता~। इष्टसाधनताज्ञानवन्तमपि पुरुषं यागे अप्रवृत्तं दृष्ट्वा, कर्मणि प्राशस्त्यज्ञानजननद्वारा कर्तव्यप्रकारं बोधयति अयमर्थवादः~। अनेन कारणेन अर्थवादः इतिकर्तव्यतात्वेन शाब्दीभावनायामन्वेतीति मीमांसकाः प्रतिपादयन्ति~। 

अयमेवार्थवादः शास्त्रान्तरेऽपि तत्र तत्र दृश्यते~। शब्दशास्त्रे यथा  -"तस्मादध्येयं व्याकरणम्" इति पातञ्जलमहाभाष्यस्य वाक्यमुदाहर्तुं शक्यते ~। अत्र अध्येयमिति कृत्यप्रत्ययान्तस्य प्रैषार्थत्वं विध्यर्थत्वं वा स्वीकर्तुं शक्यते~। 

स च विचारः अग्रे उपवर्ण्यते~। "तस्मादध्येयं व्याकरणम्" इति वचनात् प्राक् व्याकरणाध्ययनस्य प्रयोजनानि प्रदर्शितानि~। "रक्षोहागमलघ्वसन्देहाः प्रयोजनम्" इति कानिचन प्रयोजनानि प्रतिपाद्य, पुनरपि महाभाष्ये कथ्यते -" इमानि च भूयः शब्दानुशासनस्य प्रयोजनानि" इति~। इत्थञ्च व्याकरणाध्ययने प्रवृत्तिमुत्पादयितुं प्राशस्त्यज्ञानजनकानि प्रयोजनानि समुपवर्णितानीति कारणेन पस्पशाह्निकेऽपि अर्थवादः कथयितुं शक्यते~। 

एवमेव काव्यशास्त्रेऽपि अर्थवादो विद्यते इति विज्ञानाय श्लोकोऽयमुदाह्रियते~। सुभाषितान्यपि काव्यशास्त्रस्यैव एकदेशः~। 
\begin{verse}
उद्योगिनं पुरुषसिंहमुपैति लक्ष्मीः~। \\
दैवेन देयमिति कापुरुषा वदन्ति~॥\\
दैवं निहत्य कुरु पौरुषमात्मशक्त्या~। \\
यत्ने कृते यदि न सिद्ध्यति कोऽत्र दोषः~॥
\end{verse}
जनयिता देवः कथञ्चित् भोजयत्येव इति विश्वासेन जीवनं याप्येत ? आहोस्वित्  सिंहवत् उत्साहेन अन्वहमुद्योगः क्रियेत ? इति विकल्पे प्राप्ते, "उद्योग एव क्रियेत" इति नियमविधिं परिसङ्ख्याविधिं वा वक्तुं शक्नुमः~। बुभुक्षारूपः हेतुरेव मानवमुद्योगे प्रवर्तयति~। अनया रीत्या बुभुक्षारूपेण प्रमाणान्तरेण उद्योगरूपे अर्थे प्रमिते सति, "दैवं निहत्य आत्मशक्त्या पौरुषं कुरु" इति पुनर्विधानं प्रैषः भवितुमर्हति~। एवञ्च श्लोकेऽस्मिन् विधिर्वा भवतु~। प्रैषो वा भवतु~। नास्ति विप्रतिपत्तिः~। किन्तु "पुरुषसिंहमुद्योगिनं लक्ष्मीरुपैति" इति पुरुषसिंहस्य उद्योगिनः प्रयोजनमपि स्पष्टमभ्यधायि~। अत्र - "लक्ष्मीरुपैति" इति प्राशस्त्यज्ञानजननद्वारा, "दैवं निहत्य आत्मशक्त्या पौरुषं कुरु" विधानं दृश्यते~। इत्थञ्च कुरु इति लोट्-लकारेण विधिः, "लक्ष्मीरुपैति" इति वाक्येन अर्थवादश्च उभावपि सिद्ध्यतः~। अनया रीत्या काव्यशास्त्रेऽपि अर्थवादः सिद्धो भवति~। 

प्रैषार्थस्य विध्यर्थस्य च विषये नूतनेषु वैयाकरणेषु मीमांसकेषु च विप्रतिपत्तिः दृश्यते~। प्रैषातिसर्गसूत्रे भोजिदीक्षितेन इत्थमभ्यधीयत - "प्रैषः विधिः" इति~। मीमांसकास्तु भेदमेव प्रादीदृशन्~। अनेनैव कारणेन विषयेऽस्मिन् परस्परं वैरुद्ध्यं दृश्यते~। वस्तुतस्तु विरोधः नास्त्येव~। अयञ्च भेदः भोजिदीक्षितकृतः एव न त्वन्यः~। 

"प्रमाणान्तरेण प्रमिते अर्थे पुनर्विधानं प्रैषः" इति भा-रहस्ये मीमांसकः खण्डदेवः प्रैषार्थं स्पष्टं प्रतिपादयामास~। विधिस्तु अपूर्वो भवति अथवा अज्ञातज्ञापको भवतीति कारणेन अनयोः प्रैषविध्योर्भेदः वर्तत एव~। अस्य उदाहरणानि एवं वक्तुं शक्यन्ते~। परीक्षार्थं पठनीयमिति छात्रः जानात्येव~। अथापि गुरुः उपदिशति - परीक्षार्थं पठनीयमिति~। पठनेन विना परीक्षा नोत्तीर्यते इत्याकारकं ज्ञानमेव प्रमाणान्तरम्~। अनेनैव प्रमाणान्तरेण "परीक्षार्थं पठनीयम्" इत्यर्थे सिद्धे सति पुनः आचार्यः उपदिशति - परीक्षार्थं पठनीयम् इति~। अत्र आचार्योक्तिः प्रैषः भवति~। एवमेव बुभुक्षानिवारणाय भोक्तव्यम् इति विज्ञातेऽपि माता इत्थमुपदिशति - भोक्तुं समयः~। त्वया भोक्तव्यम् इति~। अत्र मातुरुपदेशः प्रैषः भवति~। अनया रीत्या अन्येन केनचित् प्रमाणेन कारणेन वा  प्रेरणायां लब्धायामपि, पुनः या प्रेरणा क्रियते सा प्रेरणा एव प्रैषः इत्यभिधीयते ~। किन्तु "यजेत स्वर्गकामः" इति वाक्येन विना, न केनाप्यन्येन प्रमाणेन "स्वर्गप्राप्त्यर्थं यागो विधेयः" इत्याकारकं ज्ञानं सम्पद्यते~। अतः अज्ञातज्ञापकत्वात् "यजेत" इति विधिरेव भवति न तु प्रैषः इति सिद्धं भवति~। प्राचीनाचार्यः वैयाकरणः काशिकाकारोऽपि "विधिप्रैषयोः को विशेषः ?" इति विचार्य, "केचिदाहुः -अज्ञातज्ञापनं विधिः, प्रेषणं प्रैषः" इति प्रैषार्थस्य विध्यर्थस्य च भेदमचकथत्~। महाभाष्येऽपि प्रैषातिसर्गसूत्रे इत्थं विचारितम् - "वाऽसरूपोऽस्त्रियाम्" इति सूत्रबलेन लिङ्-लकारः कृत्यप्रत्ययाश्च परस्परं विकल्पेन बाधकाः भवन्तु इति~। किन्तु वासरूपन्यायः "स्त्रियां क्तिन्" इति सूत्रपर्यन्तमेवानुवर्तते~। प्रैषातिसर्गसूत्रं तु स्त्रियां क्तिन् इति सूत्रात् अग्रे विद्यते~। अतः प्रैषादिषु अर्थेषु वासरूपन्यायः न प्रवर्तते~। एवञ्च प्रैषादिषु अर्थेषु वासरूपन्यायेन कृत्यप्रत्ययानां विधानं न सिद्ध्यति~। परन्तु प्रैषादिषु अर्थेषु कृत्यप्रत्ययाः इष्टा एव~। अनेनैव कारणेन कृत्यप्रत्ययानां पुनर्विधानं दृश्यते~। अत एव वार्तिकम् - "विध्यर्थं तु स्त्रियाः प्रागिति वचनात्" इति~। विधिसूत्रप्रैषसूत्रयोर्मध्ये एकस्यैव सूत्रस्य व्यवधानं विद्यते~। अतः विधिसूत्रात् विधिशब्दं अनुवर्त्य, "अतिसर्गप्राप्तकालयोः कृत्याश्च" इति सूत्रं च कृत्वा, विध्यर्थे कृत्यप्रत्ययानां विधानं सुलभं खलु ?~। किन्तु पाणिनिना पुनः प्रैषशब्दः प्रयुक्तः~। अनेनापि अभिज्ञायते प्रैषार्थादन्यो विध्यर्थः इति~। एवमेव "तत्प्रयोजको हेतुश्च" इति सूत्रस्थमहाभाष्येऽपि प्रैषार्थस्य विचारं महामुनिः पतञ्जलिः पाचीकटत्~।  देवदत्तः पचति इत्यस्मिन् वाक्ये "स्वतन्त्रः कर्ता" इति सूत्रेण देवदत्तशब्दस्य कर्तृसंज्ञा सिद्ध्यति~। यज्ञदत्तः देवदत्तेन पाचयति इत्यस्मिन् वाक्येऽपि, देवदत्तः एव पाकक्रियां निर्वर्तयति~। एवं तर्हि अनयोर्भेदः कः ? इति जिज्ञासायम्, कैयटः एवं स्वाभिप्रायं प्रादीदृशत्  - "प्रैषादूर्ध्वं प्रयोज्यस्य स्वव्यापारे प्रवर्तनात् स्वातन्त्र्यम्~। प्रैषकाले तु स्वव्यापाराप्रवर्तनात्, प्रवृत्तौ प्रैषवैयर्थ्यात् स्वातन्त्र्यं नास्ति~। तस्य च  प्रयोजको न हेतुः स्यात्~। ततश्च अत्र न णिज् भवेत्~। " इत्युक्त्वा, अन्ते -"तस्मादत्र प्रैषग्रहणेन नियोगमात्रमुच्यते न तु निकृष्टविषय एव नियोगः~। " इति~। अनेनापि कारणेन विध्यर्थादन्यः प्रैषार्थः इति निर्णेतुं शक्यते~। भोजिदीक्षितस्तु विधिसूत्रे - "विधिः प्रेरणम्, भृत्यादेः निकृष्टस्य प्रवर्तनम्" इत्यभिधाय, प्रैषसूत्रे पुनः "प्रैषः विधिः" इत्यवोचत्~। इदञ्च प्राचीनाभिप्रायविरुद्धमस्तीति प्रतिभाति~। केचित् इति प्रयुज्य काशिकाकारोऽपि भेदं प्रादर्शयत्~। कैयटोऽपि प्रैषार्थं स्फुटं बभाषे~। भाष्येऽपि भेदः तत्र तत्र समुदलेखि~। मीमांसकास्तु भेदमेव समुद्गीरयन्ति~। एवञ्च विधिप्रैषयोर्विषये  वैयाकरणानां मीमांसकानां च परस्परं वैरुध्यं नास्तीति सम्यगवगम्यते~। 

एवमेव नैयायिकाः प्रथमान्तमुख्यविशेष्यकं शाब्दबोधं प्रतिपिपादयिषाञ्चक्रुः~।\break वैयाकरणाः धात्वर्थमुख्यविशेष्यकं शाब्दबोधं व्याचिकीर्षाञ्चक्रुः~। मीमांसकास्तु भावनामुख्य\-विशेष्यकं शाब्दबोधं व्याजिहीर्षाञ्चक्रुः~। अस्मिन्नपि विषये वैयाकरणमीमांसकयोः\break किञ्चित् साम्यं दृश्यते~। मीमांसाग्रेसरः खण्डदेवः नैयायिकमतं निष्कारुण्यं खण्डयामास~। स च खण्डदेवः स्वकीयमभिप्रायमित्थं प्रकटीचकार - प्रथमान्तस्य मुख्यविशेष्यकस्य\break शाब्दबोधस्य उल्लेखः गौतमेन मुनिना न क्वापि कृतः~। मुनिप्रामाण्याभावात् नैयायिकमतं\break नानुसरणीयम् इति~। एवं तर्हि अयं वादः कथं प्रावर्तिष्ट ? इति चेत् लोकव्यवहारेण युक्त्या च प्रावर्ततेति वक्तव्यं भवति~। अतः विषयेऽस्मिन् मुनिवचनेषु बहुवैरुध्यं नास्त्येव~। 

एवमेव वेदान्तेऽपि - "तासु तदा भवति यदा सुप्तः स्वप्नं न कञ्चन पश्यत्यथास्मिन् प्राण एवैकधा भवति" इति वाक्यं पठन्तः बहवः सुषुप्तिर्नाम सुष्ठु सुप्तिः इत्येव अभिधित्सन्ति~। यत्र निद्रायां स्वप्नं न दृश्यते तत्र सुषुप्तिरिति व्यवहारः इति च तेऽभ्यदधत~। अत एव जाग्रत्स्वप्नसुषुप्तयः इति तिस्रः प्रसिद्धा अवस्था भवन्ति~। समाधिस्तु तुरीयावस्था इति वादं मण्डयन्ति~। किन्तु ब्रह्मसूत्रभाष्ये तृतीयाध्यायस्य द्वितीयपादे सप्तमे सूत्रे शङ्कराचार्याः सुषुप्तिविचारं व्याचचक्षिरे~। अत्र अन्ते -"ब्रह्म तु अनपायि सुप्तिस्थानम् - इत्येतत्प्रतिपादयामः~। तेन तु विज्ञानेन प्रयोजनमस्ति जीवस्य ब्रह्मात्मत्वावधारणं स्वप्नजागरितव्यवहारविमुक्तत्त्वावधारणम् च~। तस्मात् आत्मैव सुप्तिस्थानम्~। " इति च प्रोचुः~। सुष्टु सुप्तिः सुषुप्तिः इत्युक्त्वा, सुषुप्तिस्थानं शय्या इति तु नाभाषि~। यदि सुप्तिः लौकिकी गृह्यते तर्हि सुप्तेः स्थानं शय्या एव भवितुमर्हति~। अतः अत्र सुषुप्तिर्नाम निरन्तरं ब्रह्मध्यानमूलकम् आत्मन्येव अवस्थानमिति तात्पर्यं पर्यवस्यति~। इदं तु समाधौ अपि समानमेवेति कारणेन सुषुप्तिसमाध्योः महदन्तरं न विद्यत एववेति सम्यगभ्युपगम्यते~। एवं तर्हि पूर्वोक्तं वाक्यं कथङ्कारं समाधीयेत इति जिज्ञासायाम् - यदा सुप्तः न भवति, न कञ्चन स्वप्नं पश्यति इति नञः उभयत्रान्वये सर्वं सुलभं खलु ? इत्थं च सुषुप्तिसमाध्योः महदन्तरं न विद्यते इति प्रतिपादयितुं शक्यते~। 

एकस्यैव नञः उभयत्र अन्वये प्रमाणं किम् ? एकस्यैव नञः बहुत्र अन्वयः साधु भवति किम् ? इत्यादयः प्रश्नाः यदि समुदयन्ति तर्हि उदाहरणान्यपि वेदेषूपनिषत्सु वा समुपलभ्यन्ते~। यथा महानारायणोपनिषदि वाक्यमिदं श्रूयते - 

"न कर्मणा न प्रजया धनेन त्यागेनैके अमृतत्त्वमानशुः~। " इति

अत्र "कर्मणा अमृतत्त्वं न आनशुः", "प्रजया अमृतत्त्वं न आनशुः" इति वाक्यद्वयं सम्पद्यते~। धनेन इत्यस्य अन्वयः कुत्र इति अत्रत्यः प्रश्नः~। धनेन अमृतत्त्वम् आनशुः इति तु सर्वशास्त्रविरुद्धम्~। अतः धनेन इति तृतीयान्तस्य क्रियान्वये नञः आवृत्तिः कर्तव्या एव भवति~। अतः वेदे उपनिषदि वा कुत्रचित् प्रयुक्तः एकः एव नञ् अनेकासु क्रियासु अन्वयं प्राप्नोतीत्यत्र न शास्त्रविरोधः~। अतः - यदा सुप्तः न भवति, न कञ्चन स्वप्नं पश्यति इति नञः उभयत्रान्वये न शास्त्रविरोधः इति सप्रमाणं निर्धारयितुं शक्यते~। ज्यौतिषेऽपि एवमेव मतभेदो दृश्यते~। मुनिवचनस्वीकारे तु ज्यौतिषस्य अनुभवः अद्यापि भवत्येव~। तद्यथा बृहत्पराशरहोरायाम् - कुजगुरुशनीनां सप्तमे पूर्णा दृष्टिः न प्रतिपादिता दृश्यते~। तद्यथा -
\begin{verse}
शनिः पादं त्रिकोणेषु चतुरश्रे द्विपादकम्~। \\
त्रिपादं सप्तमे विप्र त्रिदशे पूर्णमेव हि~॥\\
चतुरश्रे गुरुः पादं सप्तमे च द्विपादकम्~। \\
त्रिपादं त्रिदशे विप्रः पूर्णं पश्यति कोणभे~॥\\
सप्तमे पादमेकं च द्विपादं त्रिदशे कुजः~। \\
त्रिपादं च त्रिकोणेषु चतुरश्रे तु पूर्णता~॥\\
\hspace{3cm} (बृहत्पराशरहोरा-राशिदृष्टिभेदाध्यायः-१८-२१)
\end{verse}
छन्दोभङ्गनिवारणाय अत्रत्यः अन्तिमः श्लोकः मया परिष्कृतः अस्ति~। परिष्काराभावेऽपि एषु त्रिषु श्लोकेषु सप्तमे पूर्णा दृष्टिः न प्रतिपादिता~।  

अन्ये ग्रन्थकारास्तु सप्तमे स्थाने कुजगुरुशनीनां पूर्णां दृष्टिमभ्यधिषत~। प्रायः सर्वेषु पञ्चाङ्गेषु सर्वेषां ग्रहाणां सप्तमे स्थाने पूर्णा दृष्टिः अभ्यधीयत~। किन्तु कुजगुरुशनीनां सप्तमे शत्रुदृष्टिफलं न कदापि अनुभूयते~। अतः अत्र मुनिवचनमेव वरमिति तत्रभवन्तो भवन्तो विदाङ्कुर्वन्तु~। ज्यौतिषे तथ्यमेव नास्तीति प्रतिपादयितुं प्रवृत्तानां चार्वाकाणाम् एतादृगेकमेव कारणं पर्याप्तं भवति~। एवमेव वराहमिहिरविरचिते बृहज्जातके द्वितीयाध्याये तात्कालमैत्री व्याख्याता दृश्यते~। अत्र - "तत्काले सुहृदः स्वतुङ्गभवनेऽप्येकेऽरयस्त्वन्यथा" इति वराहमिहिरः अभ्यधत्त~। अस्य तात्पर्यमेवमस्ति- स्वोच्चस्थाने ग्रहाः तत्काले  सुहृदः भवन्ति इति एके आचार्याः आमनन्ति~। अन्यथा शत्रवः इति~। केषाञ्चित् आचार्याणां मते उच्चस्थानं तत्कालमित्रं भवति~। अन्येषां मते तु उच्चस्थानं तत्कालमित्रं न भवत्येव~। उदाहरणं यथा - चन्द्रस्य उच्चस्थानं वृषभः राशिः~। स चोच्चराशिः तत्काले मित्रमिति परिगण्यते चेत् शुक्रः चन्द्रेण समः भवति~। उच्चस्थानं न मित्रमिति स्वीकारे तु शुक्रः चन्द्रस्य शत्रुरेव भवति इति परस्परं विरोधः~। प्रायः बहुत्र ग्रन्थेषु पञ्चाङ्गेषु च चन्द्रेण समः शुक्रः इति निरदेशि~। वस्तुतः जातकेषु कुजगुरुशनिभ्योऽपि शुक्रः चन्द्रम् अधिकं बाधते इति अनुभवेनैव समवधार्यते~। यश्च ग्रन्थमात्रं समधीत्य दैवज्ञकीर्तिं चिकीर्षति तस्य वचनं अनुभवे मिथ्यैव भवति~। ज्यौतिषं मिथ्या इति प्रतिपादयितुमेतादृशम् एकैकं कारणमपि चार्वाकाणामलं भवति~। 

इत्थं मीमांसावेदान्तव्याकरणसाहित्यन्यायज्यौतिषशास्त्रेषु किञ्चिदभिहितं मया~। अत्र यदुक्तं तत्सत्यमुत मिथ्येति निर्णये शास्त्रेषु पारदृश्वनः तत्त्वनिष्णाताः विद्वद्वरेण्याः एव प्रमाणं भवितुमर्हन्ति~। एकैकस्यापि शास्त्रवाक्यस्यापि तत्त्वज्ञानाय आजीवनं यत्नो विधेयः~। विहितस्य यत्नस्य सार्थक्यमन्वहमनुभवन्नहम् - "क्षणादूर्ध्वमतार्किकः" इति वाक्यमन्वहमात्मसात् कृतवद्भ्यः, जीवनस्य एकैकं क्षणमपि शास्त्रवाक्यविचारार्थमेव प्रदत्तवद्भ्यः, न्यायनिष्णातेभ्यः अस्मद्गुरुभ्यः विद्वद्भ्यः श्रीमद्भ्यः गङ्गाधरभट्टपादेभ्यः पादपातं प्रणिपत्य अनन्तान् धन्यवादान् समर्पये~॥
					
\articleend
}
