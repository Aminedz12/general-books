\chapter{काव्यतत्त्वविमर्शः}

\begin{center}
\Authorline{डा । एन्. सुब्रह्मण्यभट्टः}
\smallskip

सहायकप्राध्यापकः\\
अलङ्कारशास्त्रविभागः\\
महाराजसंस्कृतमहापाठशाला\\
मैसूरु
\addrule
\end{center}
विश्वस्मिन्नस्माकं भरतखण्डस्येतिहासः वेदवा$‰$यादेवारभ्यते । तन्मूलान्येव व्याकरण-न्याय-ज्यौतिषादीनि शास्त्राणि । तेषां सर्वेषामपि वेद-शास्त्र-दर्शनानां परमं लक्ष्यं तत्त्वसाक्षात्कारः । पूर्वजाः तत्त्वान्वेषणचक्षवः परमर्षयः स्वस्वशास्त्रसिद्धान्तपद्धतिमनुसृत्य तत्स्थापनाय शास्त्रान्तरसिद्धान्तान् खण्डयन्तोऽपि तत्त्वसमन्वये समदृष्टिस्तेषामवलोक्यते ।  तदेवमलङ्कारशास्त्रेऽपि सकलसहृदयहृदयावर्जकं किमप्यलौकिकामोददायिनं काव्यतत्त्वं जिज्ञास्यन्तः शास्त्रकाराः विविधविचारमार्गमनुसरन्तः विविधसम्प्रदायप्रवर्तका अभूवन् । तेषु लब्धानुयायिवर्गाः केचन सम्प्रदायवादाः काव्यशास्त्रसमाजे सबहुमानमभ्युपगता विस्तारिताश्च ।

आभरतं जगन्नाथपर्यन्तमप्यविरतं प्रवहिताः प्रवर्धिताश्च विविधविचारधाराः अद्यापि नोपक्षीणप्रवाहाः प्रतिभासन्ते । तेषां च विचारसरणिः धर्ममुखेन, व्यापारमुखेन, व्यङ्ग्यमुखेनेति त्रिभेदतया प्रवहितमिति समुद्रबन्धेन सूचितम् । प्राक्तनाः भामहादयः आलङ्कारिकाः काव्यानन्दमनुभवन्तोऽपि काव्यस्वरूप-प्रकारादिनिरूपणे मग्नाः काव्यसारत्वेन परिगण्यमानं तत्सौन्दर्यं गुणालङ्काररीतिष्वेव साक्षात्कृतवन्तः । किन्त्वचैतन्येऽपि काव्ये तज्जीवनाधायकमंशं मनसि निधाय सर्वप्रथममाचार्यवामनेन आत्मशब्दप्रयोगः कृतः- “रीतिरात्मा काव्यस्ये” ति ।  तदनु आनन्दवर्धनेन ध्वन्यात्मवादः, कुन्तकेन वक्रोक्तिजीवितवादः, विश्वनाथादिभी रसात्मवादः- इत्यादिस्वस्वबुद्ध्यनुसारं काव्यतत्त्वं प्रतिपादितम् । ते च \_  \eng{i)}  अलङ्कारसम्प्रदायः ।  \eng{ii)}   रीतिसम्प्रदायः ।	   \eng{iii)}  रससम्प्रदायः । \eng{iv)}  ध्वनिसम्प्रदायः ।	 \eng{v)}  वक्रोक्तिसम्प्रदायः । \eng{vi)}  औचित्यसम्प्रदायः ।

\section*{\eng{i)} अलङ्कारसम्प्रदायः \_}

यद्यपि मण्डनार्थकोऽयमलङ्कारशब्दः काव्ये वाचां चमत्कृतिं लक्ष्यीकृत्य प्रथमं भरतेन नाट्यशास्त्रे उपमा-रूपक-दीपक-यमकाश्चत्वारोऽलङ्काराः इति निर्दिष्टः । तदनन्तरवर्तिनः दण्डि-भामहादयः रसगुणादिभिर्दीप्तमपि काव्यं भूषाविरहितं वनितामुखमिव न शोभत इति मन्वानाः नानाप्रकारकोक्तिवैचित्यं विविधालङ्कारत्वेन सलक्ष्यं लक्षणं प्रतिपाद्य काव्यालङ्कारनाम्नैव स्वस्वग्रन्थान् रचयामासुः । भामहेन प्रायः सर्वोऽपि प्रतीयमानोऽर्थः वक्रोक्तिमुखेन दृष्टः । यतो हि वक्रोक्तिमेव काव्यसर्वस्वं ब्रुवताऽनेन कविप्रतिभादिवशात् सन्निविष्टं रस-भावादिकमप्यलङ्कारवर्ग एवान्तर्भावितम् । दण्डी तु अलङ्काराणां काव्यधर्मत्वं कथयन् तच्छोभाकरत्वेन हेतुना शब्दालङ्कारान्, गुणान्, शृङ्गारादींश्च रसानपि अलङ्कारपरिधौ निचिक्षेप । वामनस्तु ‘रीतिरात्मा काव्यस्य इ्रत्युद्घुष्य \textbf{काव्यं ग्राह्यमलङ्कारात्} इत्यलङ्कारादेव काव्यस्य ग्राह्यत्वं निर्दिशति ।   एवम् उद्भट-रुद्रटावपि भामहानुवर्तिनौ अलङ्काराणामेव प्राधान्यं ददतुः । वस्तुतस्तु अलङ्काराणां शब्दार्थरूपकाव्यशरीरस्य  शोभाकरत्वेनैव मुख्यत्वं परिहीयत इति नव्याः ।

\section*{\eng{ii)} रीतिसम्प्रदायः \_}

मतस्यास्य प्रवर्तकः आचार्यवामनः । यदि शब्दार्थयोः काव्यशरीरत्वम्, अलङ्काराणां तदाभूषणत्वं, तर्ह्यात्मस्थानीयं किमपि तत्त्वं स्यादिति वामनेन ऐदम्प्राथम्येन यत्कृतं तात्त्विकचिन्तनं तत् साहित्यशास्त्रकाराणामनन्तरवर्तिनां चिन्तनसरणिमेव पर्यवर्तयत् । अस्य मते \textbf{रीतिरेव काव्यस्यात्मा}, विशिष्टा पदरचना रीतिः । पदरचनायाः वैशिष्ट्यं गुणानां स्थितिमवलम्ब्य तिष्ठति । अत एव गुणाश्रितमिदं रीतिमतं गुणसम्प्रदायनाम्नापि व्यवह्रियते । भरतोदितान् दशगुणान् वामनः शब्दार्थगतत्वेन द्विगुणीकृतवान् । एवं च विशिष्टपदरचनात्वरूपलक्षणाया रीतेः शैलीत्यपि वक्तुं शक्यते । पदलालित्य-माधुर्य-ग्राम्यत्वादिस्वरूपायाः वैदर्भीरीतेः स्पष्टप्रतिपत्तये दशगुणानां, समासबहुललक्षणायाः गौडीरीतेः प्रतीतये ओजःकान्त्योः, उक्तिवैचित्र्य-असमास-सौम्यार्थविशिष्टायाः पाञ्चालीरीतेः ज्ञानाय माधुर्यप्रसादयोरावश्यकतानेन प्रदर्शिता । रसं कान्तिगुणान्तर्गतं विज्ञाय काव्ये तस्य बहुतरं प्राधान्यं प्रतिपादितम् । एवमेव काव्यपरिभाषायां गुणात्मनो रीतेरेव मुख्यत्वम्, अलङ्काराणां गौणत्वं प्रदर्शितमाचार्यवामनेन । दण्डिनापि पूर्वोक्तदशगुणा वैदर्भीमार्गस्य प्राणभूता इति स्पष्टीकृतम् ट्ट \textbf{इति वैदर्भमार्गस्य प्राणा दशगुणा स्मृताः} इ्रति । परं वैदर्भ्यादिरीतीनां गुणाश्रयत्वं, गुणानां रसधर्मत्वं चोरीकृत्य शरीरेऽवयवसंस्थानविशेषवदवस्थितायाः रीतेः काव्यात्मना सह सम्बन्धः स्थापितः मम्मटादिभिराचार्यैः ।

\section*{\eng{iii)} रससम्प्रदायः --}

‘रस’- आस्वादनस्नेहनयोरित्यनेन धातुना रस्यत आस्वाद्यत इत्यर्थे निष्पन्नोऽयं रसशब्दः सहृदयानां हृदयावर्जकत्वेन भावसंवेदनशीलः आनन्दापरपर्यायः । रससिद्धान्तस्यास्य मूलप्रवर्तको भरतमुनिः । नाट्यांशविमर्शनाय प्रवृत्तेऽत्र नाट्यशास्त्रे काव्यशास्त्रसम्बद्धा अलङ्कारादयोंऽशाः अपि प्रतिपादिताः । काव्य-नाट्ययोः श्रवणस्य पठनस्य चानन्दैकशरीरा अनुभवैकवेद्या रसानुभूतिरेव फलम् । वैदिकसाहित्येषु रसशब्दश्रवणेऽपि काव्यशास्त्रे सम्प्रदायरूपेण एतच्छब्दप्रयोक्तुर्भरतमुनेः नाट्यशास्त्रमेव प्रथमम् । तत्र च 

\begin{verse}
\textbf{विभावानुभावव्यभिचारिसंयोगाद्रसनिष्पत्ति}-रिति 
\end{verse}

रसनिष्पत्तिप्रकारः सूत्रितः । नायिका-नायकयोरन्यान्यं प्रत्यन्यान्यगता रतिः कारणं, चन्द्र-चन्द्रिका-उद्यानादीनि रत्याद्युपद्दीपकानि कार्याणि, स्तम्भस्वेदादीनि सहकारीणीति लोके व्यवहारः तान्येव काव्ये विभावानुभावव्यभिचारिभावपदसंज्ञितानि भवन्ति । एतैरभिव्यक्तोरत्यादिस्थायिभावः ‘दध्यादिन्यायेन’ रूपान्तरपरिणतः प्रपाणकरसवत् परिमिश्रित एवास्वाद्यतां नीयते । स च स्वाकारवदभिन्नत्वेनास्वाद्यमानो वेद्यान्तरस्पर्शशून्यः, सत्वोद्रेकादखण्डस्वप्रकाशानन्दचिन्मयः, ब्रह्मास्वादसदृश इत्यत्र सहृदयानां हृदयमेव प्रमाणम् । आनन्दैकरूपत्वादेक एव रसः सहृदयैरनुभूयमानोऽप्युपाधिभेदात् शृङ्गारादिरूपेण प्रतिभासते ।

भरतसूत्रे-‘संयोग’, ‘निष्पत्ति’-शब्दौ विवेचयन्तः शास्त्रकाराः नानामतप्रतिपादकाः सञ्जाताः । तेषु भट्टलोल्लटस्योत्पत्तिवादः-श्रीशङ्कुकस्यानुमितिवादः- भट्टनायकस्य भुक्तिवादः- अभिनवगुप्तस्य व्यक्तिवादः-इत्येते प्रमुखाः । साम्प्रतमेतेषामेव मतमादाय रससम्प्रदायः प्रचलति ।

यद्यपि ध्वनि-वक्रोक्ति-औचित्यप्राधान्यवादिभिरपि इदमभ्युपेयते यत् काव्ये रस एव सारभूतश्चमत्कारः सहृदयैरास्वाद्यमानः, तस्य चरमं लक्ष्यमिति ।

\section*{\eng{iv)} ध्वनिसम्प्रदायः\_}

ध्वनिसिद्धान्तस्य प्रवर्तकः आचार्यानन्दवर्धनः । ध्वनेः कल्पनं साहित्यशास्त्रप्रपञ्चे नवीनामेव चिन्तनधारामसृजत् । रसो न वाच्यः, अपि तु व्यङ्ग्य एवेति काव्ये व्यङ्ग्यस्यैव प्राधान्यं प्रदर्शितम् । \textbf{काव्यस्यात्मा ध्वनि}-रिति तस्योद्घोषः । गौणीकृतस्वार्थाभ्यां शब्दार्थाभ्यां यस्तदतिशयचार्वर्थो बोध्यते स एव वस्त्वलङ्काररसस्त्रिरूपो ध्वनिः काव्यात्मा इति ध्वनिस्वरूप कथितः । ध्वनिशब्दस्तु वैयाकरणोक्तस्फोटसिद्धान्तात् स्वीकृतः । तैर्हि स्फोटरूपमुख्यार्थाभिव्यक्तिविधायकस्य ध्वनिशब्दः पयुक्तः । तथैवानेनास्मिन्नेव साम्ये ध्वनिशब्दमादायास्य विस्तृतार्थो विहितः । प्रतीयमानं विधि-निषेधादिबहुरूपं वस्तु, प्रतीयमाना अलङ्काराः, अभिव्यङ्ग्याः शृङ्गारादयो रसाश्च ध्वनिस्थानमधिरोहन्ति । ध्वनन-गमन-प्रत्यायनादयोऽस्य पर्यायत्वेन प्रयुज्यन्ते । व्यङ्ग्यार्थस्य प्राधान्याप्राधान्यमवलम्ब्यैव काव्यमुत्तम-मध्यमाधमत्वेन त्रेधा विभज्यते ।

काव्यार्थप्रतिपत्तये अभिधा-लक्षणा-तात्पर्यभिन्नः कश्चन व्यञ्जनाव्यापारोऽभ्युपेयते । अभिधा-लक्षणा-तात्पर्याख्यासु त्रिसृषु वृत्तिषु स्वं स्वमर्थं बोधयित्वा उपक्षीणासु सतीषु योऽन्योऽप्यर्थः यया प्रतीयते सा व्यञ्जनेत्युच्यते । अभिधामूलत्वेन लक्षणामूलत्वेन च ध्वनेः अनेके प्रभेदा सम्भवन्ति । यद्यप्यनेन त्रिविधेऽपि ध्वनौ काव्यत्मत्वेनोक्तेऽपि रसध्वनेरेव प्राधान्यमनुमतम् । महाकवीनां वाणीष्वङ्गनायाः प्रसिद्धावयवातिरिक्तं लावण्यमिवान्य एव रसरूपोऽर्थः भासते, स एवार्थः काव्यस्यात्मेति सदृष्टान्तं दर्शयति 
\begin{verse}
\textbf{काव्यस्यात्मा स एवार्थः तथा चादिकवेः पुरा ।\\
क्रौञ्चद्वन्द्ववियोगोत्थः शोकः श्लोकत्वमागतः ॥} इति ।
\end{verse}
व्यङ्ग्य एव काव्यजीवितं, स क्वचिदप्यलङ्कारतामाप्तुं नार्हति । यदि व्यङ्ग्यरूपा रसादयोऽन्यस्य वाक्यार्थस्य परिपोषका भवन्ति तदा रसभावादयोऽप्यलङ्कारतां प्रतिपद्यन्ते । गुणालङ्कारास्तदुपस्कारका भवन्तीति काव्ये रसरीतिगुणालङ्काराणां काव्यतत्त्वानां यथायोगं स्थानं व्यवस्थितम् । ध्वनिकालादारभ्य तेषां विवेचनं, स्थानं, स्वरूपनिर्धारणमपि व्यङ्ग्यमर्यादामवलम्ब्यैव जातम् । अत एव ध्वनिवादिनः साहित्यसिद्धान्तव्यवस्थापका इत्युच्यन्ते ।

इत्थं च ध्वनिरेव काव्यस्यात्मा, माधुर्यादयो गुणा आत्मनो धर्माः, पुनरलङ्काराः शब्दार्थभूतं काव्यशरीरमलङ्कुर्वन्ति, रीतयोऽवयवसंस्थानविशेषवदिति तेषां मतम् । अनेन प्रतिपादितो ध्वनिरूपश्चमत्कारः पाश्चात्यसाहित्यतत्त्वविमर्शकैरपि समादृतः ।

\section*{\eng{v)} वक्रोक्तिसम्प्रदायः \_} 

वक्रोक्तिसम्प्रदायस्य समर्थकः कुन्तकः । यद्यपि सम्प्रदायोऽयमलङ्कारसम्प्रदायस्यैव रूपान्तरमिति भाति । अथापि केवलमलङ्काराणामेव काव्यसर्वस्वममन्यमानः कुन्तकः तदतिचारमपि निराकरोति । उक्तं च - 

\textbf{यदेवंविधे भावस्वभावसौकुमार्यवर्णनप्रस्तावे भूयसां न वाच्यालङ्काराणामुपमादीनामुपयोगयोग्यता सम्भवति, स्वभावसौ\-कुमार्यातिशय- म्लानताप्रसङ्गात्}- इति । 

वस्तुतस्तु भामहेन - 
\begin{verse}
\textbf{सैषा सर्वत्र वक्रोक्तिरनयार्थो विभाव्यते । }
\end{verse}
इति सर्वेष्वलङ्कारेषु समाहितः उक्तिवैचित्र्यरूपश्चमत्कारो वक्रोक्तिरित्युप्तं वक्रोक्तिबीजं कुन्तकेन पल्लवितं पुष्पितञ्च -‘वक्रोक्तिजीवित’- मित्येतत् नूतनसिद्धान्तप्रतिपादकग्रन्थरूपेण फलितम् । दण्डिनापि वक्रोक्ति-स्वभावोक्तिरूपाभ्यां वा‰यस्य द्वैविध्यं प्रदर्शितम् । वक्रा = सकलजनसाधारणवचनाद्भिन्ना अलौकिकचमत्कारयुक्ता, उक्तिः=कथनं वचनविन्यास इति वक्रोक्तिशब्दस्यार्थो विवृतः । कुन्तकोऽपि
\begin{verse}
\textbf{वक्रोक्तिरेव वैदग्ध्यभङ्गीभणितिरुच्यते इति }
\end{verse}
वक्रोक्तिस्वरूपमुक्त्वा गुणालङ्कारान् सर्वानपि वक्राक्तिविधावेवान्तर्भावयति । कविप्रस्थानमार्गत्वेन सुकुमार-विचित्र-मध्यमारूपास्त्रयो मार्गाः प्रदर्शिताः । सुकुमारमार्गे रसभावयोः प्राधान्यं, स्वकीयनायिकावदलङ्करणं च सहजं भवति । विचित्रमार्गे उक्त्यलङ्कारयोः प्राधान्यं, गणिकानायिकावत् कृत्रिमसज्जितशृङ्गाराणां प्रदर्शनं भवति । उभयोर्मध्ये द्वयोरपि सादृश्यमनुभवन् मार्गः मध्यममार्ग इति व्यपदिष्टः ।

यद्यपि सुकुमारमार्गः वैदर्भ्याः समानतां, विचित्रमार्गः गौड्या आनुकूल्यं धारयतः, तथापि नैतानि सादृश्यानि परिपूर्णानि । प्रसिद्धध्वनिमार्गेण परिचितोऽपि कुन्तकः ध्वनिसिद्धान्तोच्छेदायैव प्रवृत्त इति भाति । स तु ध्वन्यालोकोक्तलक्ष्याणि वक्रोक्तिविधेष्वेवान्तर्भावितवान् । तदभिप्रायः सर्वस्वकारेणेत्थं सूचितः- \textbf{उपचारवक्रतादिभिः समस्तो ध्वनिप्रपञ्चः स्वीकृतः । केवलमुक्तिवैचित्र्यजीवितं काव्यं, न व्यङ्ग्यजीवितमिति तदीयं दर्शनं व्यवस्थित}-मिति । वक्रोक्तेर्व्यापकत्वाय वर्णविन्यास-पदपूर्वार्ध-परार्ध-वाक्य-प्रकरण-प्रबन्धवक्रताः परिकल्प्य, आसु वाक्यवक्रतायां सर्वेषामलङ्काराणामन्तर्भावं दर्शयामास । किन्तु परवर्तिन आचार्याः वक्रोक्तिं रुद्रटोक्तप्रकारेण शब्दालङ्कारसामान्ये केचनार्थालङ्कारसामान्ये च पातयामासुः ।

\section*{\eng{vi)} औचित्यसम्प्रदायः \_} 

काव्यनिबन्धनावसरे, रसभावादीनां गुणरीत्यलङ्काराणां सन्निवेशने, कथासङ्घटनादिषु साहित्यशास्त्रकारोक्तलक्षणानुसरण एव कवीनां यदि लक्ष्यं स्यात्तदा काव्यत्वहानिर्भवेदिति धिया ध्वनिकारेणैव तृतीयोद्योते परस्परविरोधिरसानां वाच्यवाचकानां च सन्निवेशनौचित्यस्य भावना समुद्भाविता - 
\begin{verse}
\textbf{वाच्यानां वाचकानां च यदौचित्येन योजनम् ।\\
रसादिविषयेणैतत्कर्म मुख्यं महाकवेः ॥} इति । 
\end{verse}
सैवाचार्यक्षेमेन्द्रेण पूर्वग्रन्थानालोच्य औचित्यरूपकाव्यतत्त्वस्य व्यापकरूपमादातुम्- “औचित्यविचारचर्चा”- नाम्नि स्वग्रन्थे समालोचिता । औचित्यं नाम यत्सदृशः येन परस्परं सम्मिलितं भवेत् तदेवोचितं, तस्य भाव औचित्यमित्यभिहितम् । तदुक्तम्
\begin{verse}
\textbf{उचितं प्राहुराचार्याः सदृशं किल यस्य यत् । \\
उचितस्य च यो भावस्तदौचित्यं प्रचक्षते ॥} इति ।
\end{verse} 
तच्च औचित्यमेव रसजीवितभूतमिति तेन स्पष्टमुच्यते-
\begin{verse}
\textbf{औचित्यस्य चमत्कारकारिणश्चारुचर्वणे ।\\
रसजीवितभूतस्य विचारं कुरुतेऽधुना ॥} इति । 
\end{verse}
अस्यौचित्यस्य पद-वाक्यार्थ-रस-कारकाद्यौचित्यविधमभिधाय तदभावे काव्यत्वहानिरपि प्रदर्शितम् । ध्वनिकृतापि
\begin{verse}
\textbf{अनौचित्यादृतेनान्यद्रसभङ्गस्य कारणम् ।\\
प्रसिद्धौचित्यन्धस्तु रसस्योपनिषत्परा ॥} इति 
\end{verse}
औचित्यहानौ रसहानिः पूर्वमेवाभ्युपगता । पूर्ववर्तिभिः भरतभामहादिभिरप्यालङ्कारिकैः स्वस्वग्रन्थेषु गुणालङ्कारादि तत्तद्विषयकप्रतिपादनावसरे औचित्यरक्षायै सङ्केतो दत्तः । तदेव चाचार्यक्षेमेन्द्रेण वितत्य काव्यजीवितत्वेन प्रतिपादितम् । 

वस्तुतस्तु काव्यात्मनो रसस्य ध्वनेर्वा परिपुष्ट्यर्थं गुणालङ्काररीत्यङ्गरसादीनां यथोचितनिवेशने कविनावहितेन भाव्यमिति तु सत्यम्, अथापि तदवहितत्त्वरूपं यदौचित्यं तदेव कविकर्मणः सारो भवितुं नार्हति । यतः गुणालङ्कारादिषु पद-तदंशेष्वपि उचितत्वं पृथगेव भवति । अतः न कदाप्यौचित्यं काव्यात्मतापदमधिरोढुं क्षमम् ।

इत्थं च काव्य-नाटकादीनामनुसन्धानेन सहृदयानां हृदये आनन्दात्मकः यश्चित्तद्रवीभवनरूपः कश्चिदंशः, स एव रसशब्दवाच्यः इत्यत्र नास्ति विवादः । अथापि प्राचीनैरेते  रसभावादयः गुणालङ्कारेष्वन्तर्भाविताः सन्तः मेघान्तरितश्चन्द्रमा  इव काव्ये प्रधानपदे प्राकाश्यं न गताः । अत एव नव्यैरालङ्कारिकैः  - शब्दार्थौ काव्यस्य शरीरं, प्राधान्येनाभिव्यङ्ग्यो वस्त्वलङ्कारस्त्रिरूपो ध्वनिरात्मा, माधुर्यादयः काव्यात्मनो गुणाः, रीतयोऽवयवसंस्थानविशेषवत्, अलङ्काराः शब्दार्थशरीरभूतकाव्यस्य शोभादायकाः- इति यथायोगं व्यवस्थापिताः । एतेनैव गुण-रीत्यलङ्कारप्राधान्यवादिनां मतं, तथा वक्रोक्तेरर्थालङ्कारविशेषत्वेनाभ्युगमाच्च वक्रोक्तिजीवितमतमपि निराकृतं भवति । काव्ये गुणालङ्कारादिषु, पद-तदंशेष्वपि उचितत्वस्य पृथक्त्वादौचित्यं न काव्यात्मतापदमधिरोहति ।

\articleend
