\chapter{मीमांसाशास्त्रानुगुणं विधिविमर्शः}

\begin{center}
\Authorline{वि । श्रीकृष्ण-भट्टः}
\smallskip

संस्कृतशिक्षकः,जे.एस्.एस्.\\ 
संस्कृतपाठशाला, मैसूरु.
\end{center}

मीमांसकैः विधिविचारः बहुधाकृतः । कोऽयं विधिः? किं तल्लक्षणम् ? लक्ष्यं च किं तस्य ? इति । मीमांसकैः वेदः चतुर्धा विभक्तः विध्यर्थवादमन्त्रनामधेयभेदात् ।  शास्त्रेऽस्मिन् प्रथमाध्यायस्य प्रथमपादे एव विधिविमर्शः कृतः । “\textbf{चोदनालक्षणोऽर्थो धर्मः-३}” इत्यत्र श्रुतचोदनापदस्य तत्प्रतिपादकत्वं मीमांसकैः आदृतम् । तथा च “\textbf{चोदना हि क्रियायाः प्रवर्तकं वचनम् आहुः-४}”  इति भाष्ये, “\textbf{चोदनाचोपदेशश्च विधिश्चैकार्थवाचिनः-५}” इति वार्त्तिके च श्रूयते । अनेन जाज्ञापितो विषयः विध्यर्थः प्रवर्तना । युक्तमुक्तञ्च तल्लक्षणम् “\textbf{अज्ञातार्थज्ञापको वेदभागो विधिः-६}” इति । स च विधिः लिङ्लोट्लेट्तव्यप्रत्ययेभ्यः उच्यते । तत्र तावत् “\textbf{विधिनिमन्त्रणामन्त्रणाधीष्टसम्प्रश्नप्रार्थनेषु लिङ् ७}” इति अनुशासनबलात् विध्यादिकं लिङध्यर्थः इत्यवगम्यते । विधिर्नामप्रेरणा । सा च प्रवृत्यनुकूलव्यापारः । तद्यथा ‘गामानय’ “\textbf{स्वर्गकामो यजेत८}” इत्यादिलिङ्श्रवणानन्तरं पुरुषस्य प्रवृत्तिः दृश्यते । सा च प्रेरणा आख्यातार्थप्रवृत्तौ प्रयोजकसम्बन्धेन अन्वेति । किन्तु अस्मिन्नेव विधिविषये विवदन्ति दार्शनिकाः । एवमुद्घोषयन्ति तार्किकाः लिङादिश्रवणानन्तरं प्रवृत्तिदर्शनात् “\textbf{प्रवृत्तिसामग्रीजननद्वारा  लिङादिज्ञानस्यप्रवृत्तौ उपयोगः९}” इति । अत्र सामग्रीपदवाच्यं ‘बलवदनिष्टाननुबन्धित्वज्ञानम्, कृतिसाध्यत्वज्ञानम्, इष्टसाधनत्वज्ञानम् इति । एतेषु अन्यतमाभावेऽपि मधुविषसम्पृक्तान्नभोजने, चन्द्रस्पर्शने, मण्डलीकरणादौ च प्रवृत्यदर्शनात् त्रित्वेऽपि शक्तिः अभ्युपगन्तव्या । स एव विध्यर्थः इति नैयायिकाः आमनन्ति । अत्र स्वमतसंस्थापनपुरस्सरं न्यायमतम् एवं खण्डयन्ति मीमांसकाः शक्तित्र्यकल्पने गौरवात् इष्टसाधनताज्ञानादिनैव प्रवृत्तेः अनुभवसिद्धत्वाच्च बलवदनिष्टाननुबन्धित्वज्ञानं प्रवृत्तौ न कारणम् इति । एवं निषेधेषु प्राचीनकर्मवशेनैव प्रवृत्त्यप्रवृती कादाचित्के उपपादनीये । ननु एवं सति ‘तृप्तिकामः कलञ्जम्भक्षयेत् ’इत्यादिलौकिकवाक्यानामपि प्रामाण्यापत्तिः । इष्टसाधनत्वादिरूप-विध्यर्थस्य तत्र बाधकाभावात् इति चेत् उच्यते- प्रवर्तना एव विध्यर्थ इति प्रतिपादयतां मीमांसकानां मते तदनापत्तिरिति ।

नापि कृतिसाध्यत्वं विध्यर्थः, तार्किकमते पाकानुकूलस्य कृतौ भाने समानसंवित्संवेद्यतयापाकेऽपि कृतिसाध्यस्य भानापत्तेः,अन्यलभ्यतया लिङादिवाच्यत्वानुपपत्तेश्च । मीमांसकमतेऽपि प्रवर्तना आक्षेपादेव इष्टसाधनवत् भानोपपत्तेः कृतिसाध्यस्य अन्यलभ्यत्वम् । अतः परिशेषात्  इष्टसाधनस्यैव लिङाद्यर्थत्वम् इति वक्तव्यम् । किन्तु निषेधस्थले लिङादीनाम् अनिष्टसाधनतायां लक्षणा । अनेन अवगम्यते यत् लिङादेः न इष्टसाधनत्वमेवार्थः इति । तथात्वे ‘आचार्यप्रेरितोऽहं गामानयामि’ इति प्रवर्तकव्यवहारस्य अनुपपत्तिः स्यात्। अतः विधिः लिङाद्यर्थः इति मीमांसकानां सम्मतिः । तत्र प्रवर्तनाप्रवृत्तिविषयकयागादौ इष्टसाधनत्वानुमानात् अनुमानेनैव इष्टसाधनत्वलाभः । नतु तत्कल्पनमिति । निषेधस्थले तु निवर्तना एव वाक्यार्थः नञ्पदसमभिव्याहारात् । इत्थञ्च सिद्धं विध्यर्थः प्रवर्तना इति ।  विधिविषये मीमांसकैः अपूर्वनियमपरिसंख्याभेदेन त्रैविध्यं अङ्गीकृतम् । तथा च उक्तम् “\textbf{विधिरत्यन्तमप्राप्तौ नियमः पाक्षिके सति । तत्र चान्यत्रच प्राप्तौ परिसंखेतिगीयते १०}” ॥ इति । उत्पत्तिविनियोगप्रयोगाधिकारविधीनामपि अत्रैव अन्तर्भावः । 

\begin{thebibliography}{99}
\bibitem{chap31key1} शाबरभाष्यसहितंमीमांसादर्शनम्, श्रीमन्महादेवचिमणाजीआपटे, आनन्दाश्रमप्रेस्, १९३१
\bibitem{chap31key2} मीमांसाश्लोकवार्तिकम्, कुमारिलभट्टः रत्न पब्लिकेशन्स, वाराणसी २०१३
\bibitem{chap31key3} मीमांसाकौस्तुभः, खण्डदेवः, चौखम्बासंस्कृतसीरीसआफिस, बनारस १९२९
\bibitem{chap31key4} अर्थसङ्ग्रहः, लौगाक्षिभास्करः, राष्ट्रियसंस्कृतसंस्थान २००२.
\bibitem{chap31key5} भाट्टनयोद्योतःनारायणसुधी, राष्ट्रियसंस्कृतसंस्थानम्, २००६
\bibitem{chap31key6} विधिरसायनम्, श्री अप्पय्यदीक्षितः, बाबूहरिदासगुप्त, विद्याविलासयन्त्राययः, १९०१
\bibitem{chap31key7} वैयाकरणसिद्धान्तकौमुदी, भट्टोजिदीक्षितः, मोतीलालबनारसीदासदिल्ली, २००४
\bibitem{chap31key8} कृदन्तरूपनन्दिनी, जनार्दनहेगदे, संस्कृतभारती, बेङ्गलूरु, २०१४
\bibitem{chap31key9} धातुरूपनन्दिनी, जनार्दनहेगदे, संस्कृतभारती, नवदेहली, २०११
\bibitem{chap31key10} वाचस्पत्यम्, श्रीतारानाथतर्कवाचस्पतिमिश्रभाट्टाचार्यः, काव्यप्रकाश-प्रेस्कोलकत्ता, १८७३
\bibitem{chap31key11} अमरकोशः, अमरसिंहः, निर्णयसागरप्रकाशनम्, मुम्बई, १९६९
\bibitem{chap31key12} सर्वलक्षणसङ्ग्रहः, स्वामीगौरीशङ्करभिक्षुः, चौखम्बासंस्कृतसीरीजआफिस, वाराणसी, २०१०
\bibitem{chap31key13} \kan{ಶಬ್ದಾರ್ಥಕೌಸ್ತುಭ, ಚಕ್ರವರ್ತಿ ಶ್ರೀನಿವಾಸ ಗೋಪಾಲಾಚಾರ್, ಬಾಸ್ಕೊ ಪ್ರಕಾಶನ ಬೆಂಗಳೂರು,} \eng{2003}
\bibitem{chap31key14} \eng{ಭಾರತೀಯದರ್ಶನ, ಕರ್ನಾಟಕ ಸರ್ಕಾರ,} \eng{1997}
\end{thebibliography}

