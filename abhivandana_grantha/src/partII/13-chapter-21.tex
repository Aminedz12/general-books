\chapter{तर्कस्वरूपविचारः}

\begin{center}
\Authorline{डा । एम् . ए. आळ्वार्}
\smallskip

सहायकप्राध्यापकः\\
नवीनन्यायविभागः\\
महाराजसंस्कृत\\
महापाठशाला, मैसूरु
\addrule
\end{center}
न्यायवैशेषिकदर्शनयोः मिलितयोः तर्कशास्त्रं इति प्रसिद्धं नामेति सर्वसम्प्रतिपन्नम् । तत्र ‘तर्कः' इति शब्दस्यार्थस्त्वनेकविधः । ‘‘तर्क्यन्ते = प्रतिपाद्यन्ते इति तर्काः = द्रव्यादिपदार्थाः'' इति द्रव्यादिपदार्थपरतया तर्कशब्दं व्यवृणोत् तार्किकाग्रगण्यस्तर्कसंग्रहकर्ताऽन्नंभट्टः । परन्तु एतादृशार्थतया तर्कशब्दव्याख्यानन्तु तर्कसंग्रहमङ्गलश्लोकव्याख्यासंदर्भमात्रसीमितमिति विमर्शकाः । वास्तवतया तर्कशास्त्रे बहुलतया उपयुज्यमानः तर्कशब्दस्तु ‘‘प्रमाणानामनुग्राहकस्तर्कः'' इत्यस्मिन् अर्थे एव प्रयुज्यमानो दरीदृश्यते । स च कीदृश इत्याकांक्षायां ‘‘व्याप्यारोपेण व्यापकारोपः'' इत्येवंरूपो भवति । अत्र 'प्रमाणानामनुग्राहकः' इत्यत्र 'प्रमाणानाम्' इति बहुवचननिर्देशात् तर्कः प्रत्यक्षादिसर्वप्रमाणोपोद्बलकः इति विज्ञायते । 

यथा प्रत्यक्षेण आरोहपरिणाहविशिष्टे वस्तुनि नातिदूरसमीपवर्तिनि मन्दालोके गृह्यमाणे सति 'स्थाणुर्वा पुरुषो वा' इति सन्देहः सहजः । तथा सति 'स्थाणुत्वव्याप्यवक्रकोटरादिमानयम्' इति वा 'पुरुषत्वव्याप्यकरादिमानयम्' इति वा संशयोत्तरपरामर्शे जाते तदनुगुणतया 'स्थाणुरेवायम्' अथवा 'पुरुष एवायम्' इति निर्णयः जायते । तदनु पुनः संशये जाते (यदि पुरोवर्ति वस्तु पुरुषः, तदानीं) 'यदि पुरुषत्वं न स्यात् तर्हि करादिमत्त्वं न स्यात्' इति एवंरूपः तर्कः प्रत्यक्षप्रमाणानुग्राहकस्सन् प्रत्यक्षप्रमितिजननसहकारी भवति । 

एवं तर्कस्य अनुमानप्रमाणसहकारित्वं तु प्रसिद्धमेव । यथा ‘पर्वतो वह्निमान्, धूमात्' इत्यनुमानप्रयोगे कृते, ‘हेतुरस्तु, साध्यं मास्तु' (धूमोऽस्तु, वह्निर्मास्तु) इति हेतोरप्रयोजकत्वशङ्का प्रतिपक्षिणा यदि क्रियेत, तदा वादिना ‘यदि साध्यं न स्यात्, तर्हि हेतुरपि न स्यात्' (यदि वह्निर्न स्यात् तर्हि धूमोऽपि न स्यात्, धूमस्य वह्निजन्यत्वात्) इति तस्योत्तरतया तर्क उच्यते । उक्तप्रयोगे धूमो व्याप्यः, वह्निः व्यापकः । अप्रयोजकत्वशङ्काकर्तुः व्याप्यो धूमः पर्वते सम्मतः । व्यापको वह्निस्तु नेष्टः । यदि साध्यः व्यापकश्च यो वह्निः न स्यात्, तर्हि व्याप्यो हेतुश्च यो धूमः, सोऽपि न स्यात् इति वक्तव्यम् । अयमेव तर्कः । अत्र हि व्याप्यस्याभ्युपगमे, तदनभिमतस्य व्यापकस्य आपादनं क्रियत इति तर्कस्येदं स्वरूपम् । एवं रीत्या अयं तर्कः अनुमानप्रमाणस्य अनुग्राहको भवति । 

किञ्च तर्कः शब्दप्रमाणस्यापि अनुग्राहको भवति । यथा कस्यचिच्छब्दस्यार्थे निर्णेये सति इमे अंशाः सहकुर्वन्तीति शाब्दिकाग्रेसरो भर्तृहरिरभिप्रैति 
\begin{verse}
संयोगो विप्रयोगश्च साहचर्यं विरोधिता ।\\
अर्थः प्रकरणं लिङ्गं शब्दस्यान्यस्य सन्निधिः ।\\
सामर्थ्यमौचिती शक्तिः कालोव्यक्तिस्वरादयः ॥ (वाक्यपदीयम्)
\end{verse}
शब्दस्यार्थनिर्णयावसरे उपर्युक्ता अंशाः तर्कसहकारेणैवान्वयं प्राप्नुवन्ति । यथा ‘‘पार्वती नीलकण्ठमालिलिङ्ग । स च सन्तुष्टस्तामेवमुवाच'' इति वाक्यद्विकान्तर्गते प्रथमवाक्यस्थ'नीलकण्ठ'पदं परमेश्वरमेव बोधयति । नो चेदनुपपत्तिस्स्पष्टा  । तथा तत्रत्यतर्कस्वरूपं - 'यदि नीलकण्ठपदं परमेशवरं न बोधयेत् (बर्हिणं अन्यद्वा बोधयेत्), तर्हि उत्तरत्र विद्यमानं ‘‘स च सन्तुष्टस्तामेवमुवाच'' इति वाक्यार्थस्यानन्वयापत्तिः, तात्पर्यानुपपत्तिश्च' - इत्येवं भवति ।

उपमानस्य तु केवलशक्तिग्राहकप्रमाणत्वात्, वस्तुग्राहकप्रमाणत्वाभावाच्च न तत्र तर्कस्य तादृशं अनुग्राहकत्वम् इति केचित् । परन्तु प्रकृष्टन्यायग्रन्थेषु तत्रापि तर्कसहकारोऽस्त्येवेति प्रतिपादितमस्तीति विज्ञायते ।

एवञ्च अयं तर्कः प्रमाणस्य विरोधे आपादिते, तन्निरुणद्धि, प्रमाणस्य प्रमाणत्वं च रक्षतीति तर्कस्य प्रधानं प्रयोजनम् ।

अयं च तर्कः प्रधानतया द्विविधो भवति - प्रामाणिकपरित्यागाप्रामाणिक स्वीकारापादानरूपविषयद्वयगर्भः इत्यभियुक्ताभिप्रायः । तथा हि - तत्र 'एवं चेत् प्रामाणिकः पक्षः परित्याज्यो भवेत्' इत्येवं रूपेण प्रतिपाद्यमानः प्रामाणिकपरित्यागापादनरूपः प्रथमविधतर्कः । 'एवं चेत् अप्रामाणिकाङ्गीकारापत्तिः' इत्येवंरीत्या प्रतिपाद्यमानः अप्रामाणिक- स्वीकारापादनरूपः द्वितीयविधस्तर्कः । 

तत्र क्रमश उदाहरणम् - 
\begin{enumerate}
\item	'आकाशः प्रत्यक्षतया गृह्यते' (वेदान्त्यैकदिशिमतमिदम्) इत्युक्ते तस्य विभुत्वं न  स्यात्' इति तर्कप्रयोगात्, प्रामाणिकस्य नभोविभुत्वस्य परित्यागापत्तिः ।
\item	'गवि नष्टे तत्रत्यं गोत्वं अन्यत्र गच्छति' इत्यङ्गीक्रियते चेत् जातेः क्रियाश्रयत्वापत्त्या  द्रव्यत्वमेव स्यात्' इति तर्कप्रयोगात्, अप्रामाणिकस्य जातेर्द्रव्यत्वस्याङ्गीकारापत्तिः । तदनु उपर्युक्तरीत्या उपपाद्यमानस्तर्कः प्राकारान्तरेण विभज्यमानः पञ्चविधो भवति । तद्यथा आत्माश्रयः, अन्योन्याश्रयः, चक्रकम्, अनवस्था, अनिष्टप्रसङ्गश्चेति पञ्चविधः । तत्र क्रमशः उपर्युक्तानां तर्कविधानां उदाहरणानि अधो निर्दिष्टरीत्या ज्ञेयानि - 
\end{enumerate}
\begin{enumerate}
\item आत्माश्रयः - आधाराधेयभावानङ्गीकर्तृभिः बौद्धैः एवम् आक्षिप्यते - 'घटविशिष्टे भूतले घटः अस्ति, उत घटशून्ये भूतले घटः अस्ति ?' (इति प्रश्नः) । घटशून्ये भूतले घटोऽस्ति इति इत्युच्यते चेत् व्याघातदोषः स्पष्टः । 'घटविशिष्टे भूतले घटः अस्ति' इत्युच्यते चेत् घटस्य घटविशिष्टः एव आधारः इत्येवमुक्तं भवति । तथा च घटः आधेयः, घटविशिष्टः आधारः इत्येवंरीत्या कथनात् घटस्य आधारकोटौ घटोऽपि अन्तर्भवति । एतेन घटस्य घटोऽपि आधारः इत्युक्तं भवति । परन्तु स्वस्य स्वाधारत्वं लोके कुत्रापि न दृष्टम् । अतः तत्र प्रसज्यमानः दोषः आत्माश्रयः इति शब्देन व्यपदिश्यते । इदम् आत्माश्रायाख्यतर्कस्योदाहरणम् । 
\begin{verse}
न्यायशास्त्रीयपरिभाषया तु अस्यात्माश्रयस्य स्वरूपमेवं भवति - \\
स्वापेक्षापादको निष्टप्रसङ्ग आत्माश्रयः । स च पुनः त्रिविधः  -
\end{verse}
	\begin{enumerate}
		\item	ज्ञप्तौ स्वापेक्षापादकोऽनिष्टप्रसङ्गरूपः । (उदाहरणम् - तत्राद्या यथा एतद्धटज्ञानं  यद्येतद्धटज्ञानजन्यं स्यात् एतद्धटज्ञानभिन्नं स्यात् इति । अत्र ज्ञप्तौ = प्रकृतज्ञानजननार्थं पुनः तादृशवाक्यस्थपदे एव समाश्रयणीये आत्माश्रयः दोषः स्पष्टः)
		\item	उत्पत्तौ स्वापेक्षापादकोनिष्टप्रसङ्गरूपः । (उदाहरणम - यथा घटोयं यद्येतद्घटजन्यः  स्यात् एतद्घटभिन्नः स्यात् इति । अत्रोदाहरणे घटस्य स्वजन्यत्वकथनात् उत्पत्तौ आत्माश्रयः दोषः स्पष्टः ।)
		\item	स्थितौ स्वापेक्षापादकोनिष्टप्रसङ्गरूपः । (उदाहरणम् - अयं घटो यद्येतद्धटवृत्तिः  स्यात् तथात्वेनोपलभ्येत इति । अत्र घटः स्ववृत्तिः इत्यर्थकवाक्यप्रयोगात् स्थितौ आत्माश्रयः दोषः स्पष्टः ।)
	\end{enumerate}
\item	अन्योन्याश्रयः - धर्मधर्मिभावानङ्गीकर्तृभिः बौद्धैः एवम् आक्षिप्यते - 'गौः' इति  शब्दप्रयोगे कृते तत्र 'गोत्वं' विशेषणमिति धर्मधर्मिभावाङ्गीकर्तृभिः तार्किकादिभिः उच्यते । एवं 'गोत्वम्' इति शब्दप्रयोगे कृते 'गोः भावः = गोत्वम्' इति व्युत्पत्यनुरोधेन गोत्वे गौः (गोव्यक्तिः) विशेषणत्वेन अङ्गीक्रियते । एवं च गवि गोत्वं विशेषणं, गोत्वे च गौर्विशेषणम् इत्युक्त्वा एतद्वयमतिरिच्य तृतीयवस्त्वनङ्गीकारे अन्योन्याश्रयदोषः स्पष्टः इति बौद्धाशयः ।  

न्यायशास्त्रीयपरिभाषया तु अन्योन्याश्रयस्य स्वरूपमेवं भवति - 

स्वापेक्षापेक्षितत्वनिबन्धनोऽनिष्टप्रसङ्गोऽन्योन्याश्रयः । अस्योदाहरणम् - 'यथा अयं घटो यद्येतद्धटज्ञानजन्यज्ञानविषयः स्यात् एतद्धटभिन्नः स्यात् इति । अत्रापि उत्पत्तौ स्थितौ च उपर्युक्तरीत्या द्वैविध्यं स्वयमुदाहार्यम् ।
\item	चक्रकम् - परस्पराश्रितानां त्रयाणां सद्भावे चक्रकमिति व्यपदिश्यते । यथा  ज्ञानार्थयोरभेदवादे - रूपरसादिभेदः इन्द्रियभेदाधीन:, इन्द्रियभेदश्च विषयभेदाधीन इति त्रयाणां परस्पराश्रितत्वात् चक्रकमिति । 

न्यायशास्त्रीयपरिभाषया तु चक्रकस्य स्वरूपमेवं भवति -  

स्वापेक्षणीयापेक्षितसापेक्षत्वनिबन्धनोनिष्टप्रसङ्गश्चक्रकम् । पूर्वोक्त एवापादके जन्यपदान्तरम् अन्तर्भाव्योदाहार्यम् । अपेक्षा त्वत्र साक्षात्परम्परासाधारणी ग्राह्या ।
\item	अनवस्था - अप्रामाणिकानन्तपदार्थपम्पराकल्पनाधाराविश्रान्त्यभाव: । यथा - बौद्धैः  कृते धर्मधर्मिभावविषयकाक्षेपे - गवि गोत्वं धर्मः । गोत्वे पुनः गोत्वत्वम् अतिरिक्तो धर्मः न वा ? अतिरिक्तश्चेत्, तत्रापि गोत्वत्वत्वं कुतो न स्यात्, पुनस्तत्रापि गोत्वत्वत्वत्वादिकमिति अनवस्थाया उदाहरणम् ।

न्यायशास्त्रीयग्रन्थनिरूपणदृष्ट्या तु एवं प्रपञयितुं शक्यम् - यथा घटत्वं यदि यावद्धटहेतुवृत्ति स्यात् घटजन्यवृत्ति न स्यात् इति । वृत्तिकारास्तु अनवस्था च अन्यवस्थितपरम्परारोपाधीनानिष्टप्रसङ्गः । यथा यदि घटत्वम् घटजन्यत्वव्याप्यं स्यात् कपालसमवेतत्वव्याप्यं न स्यात् इत्याहुः (गौ० वृ० १।१।४०) ।
\item	अनिष्टप्रसङ्गः - प्रतिबन्दिलाघवगौरवादयोऽतिरिक्ततर्करूपभेदा अत्रान्तर्भवन्तीति  अभियुक्तानामाशयः । अस्योदाहरणानि तत्र तत्र ग्रन्तेष्वनुसन्धेयानि । तार्किकैकदेशिनां दृष्ट्या अन्तिमविधस्तर्कोऽपि द्विविधः -  व्याप्तिग्राहकः विषयपरिशोधकश्च । तत्राद्यो यथा - 'धूमो यदि वह्निव्यभिचारी स्यात् तदा वह्निजन्यो न स्यात्' इति । द्वितीयस्तु - 'पर्वतो यदि निर्वह्निः स्यात् निर्धूमः स्यात् इत्यादिः ( जग० तर्कग्रन्थ०) (गौ० वृ० १।१।४०) । 

अत्र धूमादेर्व्यभिचारशङ्कानिवृत्तिद्वारा विषयस्य वह्नयादेः निश्चायकत्वेन एतत्तर्कस्य परिशोधकत्वमित्यवधेयम् । 
\end{enumerate}

अत्र प्राचीननैयायिकास्तु 

स च तर्क एकादशविधः - 

1.व्याघातः 2.आत्माश्रयः 3.इतरेतराश्रयः 4.चक्रकाश्रयः 5.अनवस्था 6.प्रतिबन्धिकल्पना  7.लाघवकल्पना 8.गौरवम् 9.उत्सर्गः 10.अपवादः 11.वैजात्यम् इत्यङ्गीचक्रुः(सर्व०पृ० २३९ अक्ष०)।

एतान्येकादशान्यपि पञ्चस्वेवान्तर्भवन्तीति नव्याः ।

अयञ्च तर्कः अङ्गपञ्चकविशिष्टः । यथा - 
\begin{enumerate}
\item	वैपरीत्ये विश्रान्तिः = आपाद्यस्य विपर्यये पर्यवसानम् । 
\item	प्रतिहतिविरहः = प्रतितर्कापराहतिः । 
\item	अनिष्टता  = आपाद्यस्य परानिष्टत्वम् । 
\item	अनानुकूल्यम् = परपक्षासाधकत्वम् । 
\item	व्याप्तिश्च = आपाद्यापादकयोर्व्याप्तिश्च (इति पञ्च) ।
\end{enumerate}

\section*{यथाक्रमममीषामुदाहरणानि -} 

यथा उपर्युक्ते ‘यदि वह्निर्न स्यात् तर्हि धूमोऽपि न स्यात्' इत्येतादृशतर्के -  आपादकः वह्न्यभावः, धूमाभावः आपाद्यः ।
\begin{enumerate}
\item	अनयोर्व्याप्तिर्वर्तते - यत्र वह्न्यभावः, तत्र धूमभाव इति । 
\item	आपाद्यस्य विपर्यये पर्यवसानम् । पर्वते खलु धूमः प्रत्यक्षसिद्धः ।
\item	प्रतिवादिनोऽनिष्टः, ‘धूमोऽस्तु' इति खलु वदति सः । 
\item	परपक्षाननुकूलत्वं तु स्पष्टमेव । 
\item	एवं प्रतितर्केणापराहतिश्चावश्यकी । ‘वह्निर्यदि न स्यात्, तर्हि धूमोऽपि न स्यात्' इति तर्के प्रदर्शिते, ‘कुतो न स्यात्?' इति पुनस्तोनोक्ते, ‘यदि वह्न्यभावेऽपि धूमः स्यात्, तर्हि कारणं विनापि कार्यं स्यात्, धूमः खलु वह्निजन्यः' इति कार्यकारणभावरूपस्यानुकूलतर्कस्य सत्त्वात् प्रतितर्कापराहतिः वर्तते । एवं पञ्चाङ्गानि ज्ञेयानि ॥ 
\end{enumerate}
बोद्धैकदेशिनः नवीनाः  - तर्कः अनुमितेरेव प्रकारभेद इत्यभिप्रयन्ति । ननु, अनुमितिस्तु उद्देश्यविधेयवती, तर्कस्तु आपाद्यापादकभाववान् । एवं सति तर्कः अनुमितिविशेषरूपः कथं स्यात्? इति चेत् - आपाद्यापादकयोः एवं हेतुसाध्यभावापन्नत्वेन तर्कोऽप्यनुमितिविशेष एव । प्रतितर्कापराहतिः स्याच्चेत्, अलम्  । किमधिकापेक्षया ? अतस्तर्कोऽपि अनुमितिविशेषरूपः, अत एव स्वतन्त्रं प्रमाणमिति ते मन्यन्ते । 

परन्तु नैतदयुक्तमिति तार्किकाः - प्रमाणस्य वस्तुप्रदर्शनफलत्वात्तर्कस्यातथात्वादेतन्न युक्तम्  । किञ्च - तर्को नानुमितिः, किन्त्वनुमानाङ्गम् । अनुमितिर्हि फलरूपा, तर्कस्तु न फलरूपः, किन्तु साधनाङ्गमेव । प्रमाणानि तु चत्वार्येव, नातिरिक्तानि । तर्कस्तु प्रमाणानुग्राहक एवेति ।

तदेतत्सर्वमिदमित्थं समग्राहि वेदान्तदेशिकापरनामधेयैर्वेङ्कटनाथाचार्यैस्स्वीय तत्त्वमुक्ताकलापग्रन्थे \
\begin{verse}
तर्को व्याप्याभ्युपेतावभिमतिपदव्यापकस्य प्रसक्तिः\\
मानप्रत्यूहघाती द्विविषय उदितः पञ्चधाऽऽत्माश्रयादिः ।\\
विश्रान्तिर्वैपरित्ये प्रतिहतिविरहोऽनिष्टताऽनानुकूल्यं\\
व्याप्तिश्चास्याङ्गमेनं कतिचिदनुमितेस्तादृशं भेदमाहुः॥(तत्त्वमुक्ताकलापः  4/60) इति।
\end{verse}
एवं बहुधा व्यापृतोऽयं तर्कविचारः नव्यतर्कग्रन्थेषु विशेषतो न विवृत इति तद्ग्रन्थाध्ययनेन स्पष्टं ज्ञायते । अतः कारणादत्र विदुषां तोषाय यथामति न्यरूपि गुरूणामनुग्रहपरीवाहात्तदीयवचनान्यादायैवेति शम् ॥

\begin{thebibliography}{99}
\bibitem{chap21-key1}	श्रीवेदान्तदेशिकविरचितः तत्त्वमुक्ताकलापः ।
\bibitem{chap21-key2}	महामहोपाध्याय-नव्यमङ्गलाभिजन-श्रीवरदाचार्यविरचित-तत्त्वमुक्ताकलापव्याख्या सर्वङ्कषा ।
\bibitem{chap21-key3}	अनेके न्यायादिशास्त्रग्रन्थाः ।
\end{thebibliography}

\articleend
