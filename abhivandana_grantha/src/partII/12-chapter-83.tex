{\fontsize{15}{17}\selectfont
\chapter{न्यायदर्शनम्}

\begin{center}
\Authorline{वि~। श्रीधर-शास्त्री}
\smallskip
निवृत्त-न्याय-प्राध्यापकः\\
संस्कृत-पठशाला\\
तुमकूरु
\addrule
\end{center}
\begin{verse}
\textbf{ॐ नमः सर्वभूतानि विष्टभ्य परितिष्ठते~। \\
अखण्डानन्दबोधाय पूर्णाय परमात्मने~॥}
\end{verse}

इह दुःखबहुले संसारे सर्वेऽपि सुखं प्रेप्सवः दुःखं परिजिहीर्षवः तत्तदुपायान् चिन्तयन्तः एव वर्तन्ते~। शास्त्रप्रवर्तकाः ऋषयोऽपि लोकानुग्रहार्थं स्वतपसा साधितं शास्त्ररूपेण वितन्वन्त~। तेषु अन्यतममिदं प्रथितन्यायदर्शनापरपर्यायं महत् न्यायदर्शनं गौतममुनिना प्रणीतं सर्वविद्यासु उत्कृष्टं विराजते~। चरमपुरुषार्थसाधनसामग्रीषु ज्ञानमेव वरम् साधनमिति श्रुतिस्मृतियुक्तिभिः बहुधा साधनभावं प्रतिपादयन्ति~। इदमपि न्यायदर्शनं ज्ञानादेव कैवल्यं मनुते~। पर तच्च ज्ञानं आत्मज्ञानमेव~। षोडशपदार्थ तत्त्वज्ञानम् आत्मज्ञानोपाये एव पर्यवस्यति इति भाष्पावलोकनेन स्पष्टं भवति~। अत्रच सकलप्रमाणमौलिभूताः श्रुतयः “आत्मावाऽरे द्रष्टव्यः श्रोतव्यः मन्तव्यः निदिध्यासितव्यः” इत्यादयः~। तत्र विहितेषु साधनेषु मननमेव अभ्यर्हिततमम्~। मननञ्च सपरिकरहेतुभिः अनुमानमेव~। अयमेव न्यायशास्त्रस्य शास्त्रान्तरेभ्यो विशेषः~। प्रमाणविषयमधिकृत्य प्राञ्चैः कणादगौतमादिमहर्षिभिः नव्यैश्च उदयनाचार्य-गङ्गेशोपाध्याय प्रभृतिभिः कुशाग्रमतिभिः उपपादितम् ,नैवमन्यैः कैश्चिदपि तन्त्रकारैः इत्यत्र अन्तरङ्गमेव विदुषां प्रमाणम् इति किमन्यद्वक्तव्यम् ?  अत एव याज्ञवल्क्यमुनिरपि धर्मज्ञानाङ्गत्वेन न्यायं पर्यगणयत्~। यथा- 
\begin{verse}
पुराणन्यायमीमांसा धर्मशास्त्राङ्गमिश्रिताः~। \\
वेदाः स्थानानि विद्यानां धर्मस्य च चतुर्दश~॥ इत्याह~। 
\end{verse}    

मनुरपि-

\begin{verse}
आर्षं धर्मोपदेशां च वेदशास्त्राविरोधिना~। \\
यस्तर्केणानुसन्धत्ते स धर्मं वेद नेतरः~॥ इति~। 
\end{verse}

धर्मज्ञानोपायेषु तर्कस्य महत् स्थानमस्ति~। श्रीमद्भागवतपुराणे इयं तर्कविद्या अध्यात्मविद्या इति उल्लिखितं दृश्यते~। तत्र दत्तात्रेयमहामुनिः प्रह्लादाय उपादिशदिति एकम् उपाख्यानं अस्ति~। किञ्च महर्षिगौतमप्रणीते आन्वीक्षिकीत्यपरसंज्ञे न्यायशास्त्रे पदार्थनिरूपणपराणि यानि सूत्राणि तानि उपनिषत्स्वपि उपदेशरूपेण तत्तदर्थपराण्येव उपनिषद्वाक्यरूपेण उपलभ्यन्ते~। तस्मात् अस्य शास्त्रस्य श्रुतिमतत्त्वं युक्तिप्रधानत्वस्यापि सत्त्वात् महत्त्वं सुतरां सिध्यति~।                      न्यायदर्शनसारपिपासुभिः वात्स्यायनमुनिकृतभाष्यावेक्षणम् अनिवार्यमेव भवति~। प्रथमसूत्रभूमिकायां प्रमात्रादिचतुर्वर्गेषु प्रमाणस्यैव प्राधान्यं भाष्यकारः उपपादयति~। हेयोपादेयादिषु हानस्य प्रमाणाधीनत्वात् षोडश पदार्थेषु प्रमाणस्य प्राथम्यमुपपादयत्~। भाष्यकाररीत्या प्रमा-असाधारणकारणं प्रमाणम् इति प्रमाणलक्षणमवगम्यते~। 

सकलशास्त्रोपकारकत्वेन “तत्त्व” पदार्थः बहुसम्यक् निरूपितः~। “सतः सद्भावः असतः असद्भावः” इति भाष्यवचनम्~। इदमेव उत्तरत्र न्यायशास्त्राभिवृध्दौ सप्तपदार्थवादे अभावनिरूपणस्य भूमिका भवति~। अन्यथा भावभिन्नत्वं अभावत्वम् इत्यादि लक्षणे अन्योन्याश्रयः अपरिहार्यो भवति~। 

न्यायदर्शनस्य प्रथमसूत्रभाष्ये निःश्रेयससाधनमुपपादितम्~। तत्र शास्त्रनिरूपणक्रमः उद्देश-लक्षण-परीक्षा-रूपेण त्रिविधः इत्युक्तम्~। अत एव सूत्रकारैरेव आदौ ग्रन्थोद्देशं उपपाद्य क्रमेणोपपादितप्रमाणपदार्थनिरूपणं प्रादर्शिषतम्~। अत्रैव ईश्वरसिद्धिविषये उदयनाचार्यैः दृष्टिः प्राक्षिपत~। तदनुसृत्य आचार्यैः उक्तम् इत्थम् अस्ति-

\begin{verse}
साक्षात्कारिणि नित्ययोगिनि परद्वारानपेक्षस्थितौ~। \\
भूतार्थानुभवे निविष्टनिखिलप्रस्ताविवस्तुक्रमः~॥\\
लेषादृष्टिनिमित्तदुष्टिविगमप्रभ्रष्टशङ्कातुषः~। \\
शङ्कोन्मेषकलङ्किभिः किमपरैस्तन्मे प्रमाणम् शिवः~॥ न्या.कु.४.६~॥ इति~। 
\end{verse}

एवं प्रमाणनिरूपणानन्तरं द्वादशविधं प्रमेयं निरूपितम्~। तेषु आत्मा बुध्यनुमेयः~। शरीरं चेष्टाश्रयम्~। पृथिव्यादि भूतप्रकृतीनि घ्राणादि इन्द्रियाणि~। तेषां विषयीभूताः गन्धाद्यर्थाः~। आत्मविषेशगुणः बुद्धिः ज्ञानापरपर्यायः~। नैय्यायिकानां ज्ञानं वेदान्तिभ्यः विलक्षणं भवति~। वृत्तिज्ञानं स्वरूपज्ञानं इत्यादि रूपेण नोच्यते~। आत्मगुणसाक्षात्कारसाधनं सकलज्ञानकारणीभूतमिन्द्रियं प्रत्यात्मनियतं मनः~। प्रवृत्तिरपि आत्मगुण एव~। परं धर्माधर्मरूपः इति न्यायसमयः~। रागादयः दोषाः~। अतः तदुच्छेदः अत्यन्तमपेक्षितम्~। तेभ्यः एव जीवः संसरति~। प्रेत्यभावः पुनरुत्पत्तिः~। अत्र भाष्यवचनम् एवं विवृणोति - “ यत्र क्वचित् प्राणभृन्निकाये वर्तमानः पूर्वोपात्तान् देहादीन् जहाति-तत्-प्रैति~। यत् तत्र अन्यत्र वा देहादीन् अन्यान् उपादत्ते-तद्भवति~। प्रेत्यभावः मृत्वा पुनर्जन्म~। सुखदुःखसंवेदनं फलम्~। पीडालक्षणं दुःखम्~। दुःखापायः अपवर्गः~। दुःखे आत्यन्तिकविशेषणमस्ति~। अत्र अपवर्गपदार्थप्रतिपादनावसरे ब्रह्मक्षेमप्राप्तिः इति ब्रह्मभावः उपपादितः~। 

एवं संशयपदार्थः निरूपितः~। स च त्रिविधः इत्युक्तम्~। साधारणधर्म-असाधारणधर्म-विप्रतिपत्ति भेदेन~। प्रयोजनं नाम पुरुषप्रवृत्युद्देश्यं इति~। तच्च दृष्टादृष्टभेदेन द्विविधम्~। एवमेव षोडशपदार्थानां स्पष्टं निरूपणं भाष्ये अस्ति~। ततः द्वितीयाध्यायस्य प्रथमान्हिके संशयपरीक्षणमुपपाद्य अर्थापत्तेः अन्तर्भावकथनं शब्दानित्यत्वपरीक्षणं च सूपपादितम्~। अन्ते पञ्चमाध्याये सर्वं वादोपयोगितया जातिपदार्थः निग्रहपदार्थश्च उपवर्णितः~। 

इत्थं न्यायदर्शनं वात्स्यायनाचार्यभाष्योपेतं मुनिना गौतमेन ऐहिकपदार्थजातज्ञानकारणं आमुष्मिकोत्कृष्ट-अपवर्गाभिध\-फलजनक-तत्त्वज्ञानस्य इत्युपपाद्य अन्ते अपवर्गकारणं तु तत्त्वज्ञानमेव नान्यदिति सप्रमाणनिरूपणम्~। श्रुतिरपि तमेव विदित्वा अतिमृत्युमेति नान्यःपन्था विद्यतेऽयनाय इत्यादिना समर्थितः~।  अस्य शास्त्रस्य न्यायशास्त्रमिति व्यपदेशस्तु “असाधारण्येन व्यपदेशा भवन्ति” इति न्यायेन परार्थानुमानस्य सर्वविद्यानुग्राहकत्वेन सर्वकर्मानुष्ठानहेतुत्वेन प्राधान्येनैव तथा व्यवह्रियते~। “सोऽयं परमो न्यायः विप्रतिपन्नपुरुषप्रतिपादकत्वात् प्रवृत्तिहेतुत्वाच्च “ इति सर्वज्ञेनावादिषि~। महर्षिः वात्सायनोऽपि “ प्रमाणादिभिः पदार्थैः प्रविभज्यमाना इयम् आन्वीक्षिकी विद्या-

\begin{verse}
प्रदीपःसर्वविद्यानाम् उपायः सर्वकर्मणाम्~। \\
आश्रयः सर्वधर्माणां विद्योद्देशे प्रकीर्तिता~॥
\end{verse}

इत्यनेन सर्वविद्याश्रेष्ठत्वम् अस्याः न्यायविद्यायाः प्रत्यपादयत्~। समानतन्त्रे वैशेषिके नयेपि महर्षिणा कणादेन द्रव्यादिरूपसप्तपदार्थान्तःपाति आत्मतत्वज्ञानेनैव न्यायदर्शनोक्तविध-निःश्रेयसप्राप्तिरिति विदितम्~। आहत्य शास्त्रद्वयेऽपि एकप्रयोजनमिति द्वे अपि शास्त्रे “न्यायशास्त्रम्” इति सङ्गिरामः~। 

इदं न्यायदर्शनं आर्षमेव~। अलौकिकशक्त्या पदार्थतत्त्वज्ञानं ऋषेः गोचरी बभूव~। तस्यैव न्यायशास्त्रस्य वात्स्यायनः भाष्यजातं व्यरचयत्~। स्वयं भाष्यान्ते उक्तवचनेन स्पष्टं भवति~। तेन तच्छास्त्रप्रतिपादित-अणुसिद्धान्तः दृढः भवति~।  परमणुरेव आत्मा इति वदतां “तस्मादात्मन आकाशः सम्भूतः” इति श्रुतिः सङ्गच्छते~।  अयमेव सकलशास्त्रसमन्वयक्रमः इति गुरुवर्याः श्रीमद्रामभद्राचार्याः इत्थमेव गिरं सङ्गिरन्ते स्म~। न्यायदर्शनोक्तपदार्थचिन्तनं आधुनिकविज्ञानलोकस्य परमोपकारकं भवति इति संक्षेपेण न्यायभाष्यम् उपपाद्य विरमामि~। 

\articleend
}
