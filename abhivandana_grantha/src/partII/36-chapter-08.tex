{\fontsize{15}{17}\selectfont
\presetvalues
\chapter{निर्विशेषं परं ब्रह्म}

\begin{center}
\Authorline{डा॥ श्रीधरभट्टः ऐनकै}
\smallskip

निर्देशकः, वेदन्तभारती\\
योगानन्देश्वरसरस्वतीमठः\\
के.आर्.नगरम्
\addrule
\end{center}

आस्तिकनास्तिकभेदेन द्वैविध्यमापन्नेषु भारतीयदर्शनेषु आस्तिकान्येव दर्शनानि\break समुपादेयानीति निश्चप्रचं लोकानाम्~। तत्रापि न्यायवैशेषिकादिदर्शनापेक्षया साक्षान्निः\break श्रेयसहेतुप्रतिपादकत्वेन वेदान्तदर्शनमेव दर्शनीयतमं दर्शनमिति मधुसूदनसरस्वतीप्रमुखानां महामेधाविनामाशयः~। 

वस्तुतः वेदान्तशब्दस्य मुख्यगौणभेदेन अर्थभेदोऽपि दृश्यते शारीरकमीमांसाशास्त्रे~।\break तदुक्तं सदानन्दयोगीन्द्रैः ‘वेदान्तो नाम उपनिषत्प्रमाणं शारीरकसूत्रादीनि च' इति~। एवमेव\break रामतीर्थैः "उपनिषच्छब्दः ब्रह्मात्मैक्यसाक्षात्कारविषयः' इति~। वार्तिककारैः श्रीसुरेश्वरैः\break ‘तत्र उपनिषच्छब्दो ब्रह्मविद्यागोचरः' इति च व्याख्यातम्~। प्रायः वेदान्ताः नाम\break प्राधान्येन उपनिषदः एव परिदृश्यन्ते~। तत्रोक्तस्य विषयजातस्य प्रपञ्चनमेव ग्रन्थान्तरेषु\break परिदृश्यते~। भगवान् बादरायणाचार्यः आपाततः परस्परविरुद्धतया प्रतीयमानवेदान्तवाक्यानां तात्पर्यनिर्णयद्वारा परमपुरुषार्थसाधनीभूतपरतत्त्वावधारणाय "अथातो ब्रह्मजिज्ञासा"\break इत्यादिसूत्रमालिकामेकां  रचयामास~। ततः श्रीमच्छङ्करभगवत्पादानां परमगुरवः गौडपादाचार्याः वेदान्तार्थसारसंग्रहभूतप्रकरणचतुष्टयोपेतकारिकाख्य माण्डूक्योपनिषद् व्याख्यानरूप\-ग्रन्थतल्लजं व्यरचयन्~। अत एव वक्ष्यते भाष्ये- प्रकरणचतुष्टयम् ‘ओमित्येतदक्षरमित्यादीति'~। उक्तं च शास्त्रसिद्धान्तलेशसंग्रहे श्रीमदप्पयदीक्षितेन्द्रैः ‘माण्डूक्यश्रुतिव्याख्यानरूपं भाष्यं गौडपादीयविवरणमिति'~। अतिसंक्षिप्तत्वेन दुर्ग्रहार्थानि ब्रह्मसूत्राणि मन्दमतीनां\break अनायासेन अर्थावबोधनाय व्याचख्युर्भगवत्पादसंज्ञकाः श्रीशङ्कराचार्याः~। प्रस्थानत्रयात्मना विभक्तम् प्रसन्नगम्भीरं शाङ्करं भाष्यम् जीवब्रह्मणोरभेदरूपं स्वाभाविकं पारमार्थिकात्मतत्त्वम् स्फुटतया विवृणोति~। एवं प्रवर्तिते कालचक्रे भगवत्पादीयभाष्यस्य सर्वमनीषिमनोहरं\break व्याख्यानं वार्तिकं च तदन्तेवासिभिः पद्मपादसुरेश्वरादिभिराचार्यप्रवरैः व्यरचि~। 

तदनु भगवत्पादीयप्रस्थानत्रयभाष्याणां परश्शतानि व्याख्यानानि आविरभूवन्~। तेषु\break सर्वज्ञात्ममुनि-वाचस्पतिमिश्र-अमलानन्द-चित्सुख-प्रकटार्थकार-आनन्दगिरि-रामानन्द- शङ्करानन्दप्रभृतयः सर्वतन्त्रस्वतन्त्राः आचार्यप्रवराः भाष्यकृतामेव शास्त्रसरणिमनुसृत्य\break नैकग्रन्थतल्लजानरीरचन्~। सर्वेऽपि इमे आचार्यप्रवराः जीवब्रह्मणोः पारमार्थिकभेदं\break परमसिद्धान्ततया स्वीकुर्वन्तो विराजन्ते~। 

उपक्रमोपसंहारादिभिः षड्विधतात्पर्यलिङ्गैः प्रमेयभूते प्रत्यक्स्वरूपे परब्रह्मणि एव\break सर्वेषाम् वेदान्तवक्यानां पर्यवसानात् ब्रह्मणः एव विषयत्वम् वेदान्तसिद्धान्तानाम्~। प्रयोजनं तु तदैक्यप्रमेयगताज्ञाननिवृत्तिः स्वस्वरूपानन्दावाप्तिश्च~। तत् फलं च ब्रह्मात्मैकत्वलक्षण\-चिन्मात्रगत-अज्ञानस्य तत् कार्यभूतसकलप्रपञ्चस्य निवृत्तिः पुनरुत्पत्यभावरूपा स्वस्वरूपा\-खण्डानन्दप्राप्तिश्च~। "तरति शोकमात्मवित्" "ब्रह्मवेद ब्रह्मैव भवति", इत्यादीनि श्रुति\-शतवाक्यानि प्रमाणानि भवन्ति~। तत्र निर्गुणं निर्विशेषं ब्रह्म एव पारमार्थिकं परं\break सत्यम्~। सगुणं सविशेषकं ब्रह्म तु औपाधिकम्~। तत्र अविद्यायाः उपाधित्वात्~। इदमेव\break सगुणं सविशेषं ब्रह्म सृष्टिस्थितिप्रलयकर्ता ईश्वरः इति कथ्यते~। अयं ईश्वरः अविद्योपहितत्वात् अपारमार्थिकः मिथ्याभूतः~। वस्तुतः एकमेव तत्त्वं निरुपाधिकं सत् ब्रह्मरूपेण सोपाधिकं सत् ईश्वररूपेण अवभासते~। "उपासकानां कार्यार्थं ब्रह्मणो रूपकल्पना" इति श्रुत्युक्तरीत्या ईश्वरब्रह्मणोः अनन्यत्वम् उपनिषत्सु परिदृश्यते~। एवं जीवब्रह्मणोः अपि वस्तुतः न कश्चन भेदः, तत्त्वमस्यादिश्रुतेः~। तथापि अज्ञानवशात् एव जीवः बद्धः एव सन्, स्वकीयं ब्रह्मस्वरूपं (ब्रह्मभावम्) अनुभवति~। इदं जगदपि अविद्यावशात् ब्रह्मणि आरोपितम् तदपि मिथ्याभूतम्~। 

वस्तुतस्तु एकमेव परं ब्रह्म सर्वत्र अवतिष्ठते इति ब्रह्मव्यतिरिक्तं सर्वं मिथ्या केवलं ब्रह्मैव सत्यम् इति अद्वैतवेदान्तिनामाशयः~। 

निर्गुणं निर्विशेषं ब्रह्म वाङ्मनसयोः अगोचरम्~। न हि तत् केनापि धर्मेण गुणेन वा प्रकाशयितुं शक्यते~। प्रपञ्चस्य सर्वेभ्यः वस्तुभ्यः भिन्नं ब्रह्म इति निषेधमुखेन नेति नेतिमुखेन वा ब्रह्मणः लक्षणं क्रियते~। सर्वासु उपनिषत्सु सच्चिदानन्दादिरूपेण विधिमुखेन यद् ब्रह्मणः लक्षणं दृश्यते तत् सर्वम् निषेधमुखेन एवावगन्तव्यम् इति ईशाद्युपनिषत्सु परिदृश्यमानो विशेषो विचारः~। तदुक्तं भगवत्पादैः सर्ववेदान्तसिद्धान्तसारसंग्रहे- "नेति नेतीत्यरूपत्वात् अशरीरो भवत्ययम्" इति~। प्रातःस्मरणस्तोत्रे "यन्नेति नेति वचनैः निगमा अवोचुः" इति~। बृहदारण्यकवार्तिकेऽपि नेति नेति पदाभ्यां ब्रह्मबोधनप्रकारः विस्तरेणोच्चते~। "नेति नेतीति शब्दाभ्यां सत्यस्य ब्रह्मणः" इत्यारभ्य- "शब्दप्रवृत्तिहेतूनां साक्षाद् ब्रह्मण्यसम्भवात्" इत्यन्तम् जीवब्रह्मैक्यबोधनरीतिः प्रदर्शिता (बृ.वा. २.३.१७६तः २५४)

अपरोंऽशः - उपनिषत्सु प्रतिपादानप्रक्रिया द्विविधा दृश्यते~। क्वचित्तु- तदर्थेनोपक्रम्य त्वमर्थेन ऐक्यमुपदिश्यते यथा छान्दोग्ये "सदेव सोम्येदमग्र आसीदेकमेवाद्वितीयम्" (छा.उ.६..१) इति तत्पदार्थं प्रक्रम्य स आत्मा इति तस्यैव अन्ते प्रत्यगात्मत्वं दर्शयति~। क्वचित्तु- त्वमर्थेनोपक्रम्य तस्य असंसारिब्रह्मात्मत्वमुक्त्वा अन्ते ‘स वा एष महानज आत्मा"(बृ.उ.४.४.२२) इति ब्रह्मत्वं निरूपयति~। 

इयं प्रक्रिया द्वैविध्यं माण्डूक्यभाष्यारम्भे भगवत्पादैः श्लोकाभ्यां प्रादर्शि~। तत्र\break आद्येन, यत्परं ब्रह्म तन्नतोऽस्मीति अस्मदर्थस्य तत्पदलक्ष्यार्थब्रह्मैक्यानुसन्धानमेव\break नमनं प्रदर्श्य, ब्रह्मणः प्रत्यक्त्वप्रतिपादनपरत्वं सूत्रितम्~। द्वितीयेन श्लोकेन च विश्वादिशब्द\-वाच्यस्य भोक्तुः तुरीयत्वेन असंसारिब्रह्मत्वप्रतिपादनपरत्वं सूत्रितम्~। अर्थात् अत्र प्रदर्शितं प्रक्रियाद्वैविध्यं विधिमुखेन निषेधमुखेन च ब्रह्मप्रतिपादनमिति गौडपादीयभाष्यव्याख्याने\break भाष्यटिप्पण्यां श्रीमदनुभूतिस्वरूपाचार्याः अभिप्रयन्ति~। एवं सर्वविशेषं नेति नेतीति विहाय यदवशिष्यते तदव्ययम् ब्रह्म इति निर्णयः~। अर्थात् जीवभावजगद्भावबाधे प्रत्यगभिन्नब्रह्मैवावशिष्यते इत्यर्थः~। अत्र तु प्रमाणम् ‘पूर्णमेवावशिष्यते' इति~। भगवत्पादाः अपि न भूमिर्न तोयं न तेजो न वायुर्न खम्' इत्यादि दशश्लोकिप्रकरणाख्ये ग्रन्थे मोक्षसाधनीभूतज्ञानविषयपरमानन्दबोधरूपाहंप्रत्ययालम्बनाद् द्वितीयसर्वप्रमाणाबाध्यौपनिषत्सिद्धान्तनिर्णयं सर्वतः स्थितं सर्वत्रानुस्यूतबिम्बीभूत- मनोवाचामगोचरं निरतिशयानन्दरूपस्वतस्सिद्धकेवलात्मशिवस्वरूपं प्रदर्शयन्ति~। तदर्थमेवमुपनिषत्सु संस्तूयते~। 

\begin{verse}
निःशब्दं परमं ब्रह्म परमात्मानमीयते~। \\
सकले निष्कले भावे सर्वत्रात्मा व्यवस्थितः~। \\
सर्वदा सर्वकृत् सर्वः परमात्मेत्युदाहृतः~। \\
अनाद्यन्तोऽवभासात्मा परमात्मेह विद्यते~॥\\
निर्विकल्पः स्वरूपात्मा सविकल्पविवर्जितः~। \\
सदा समाधिशून्यात्मा आदिमध्यान्तवर्जितः~। 
\end{verse}

एवं सर्वोपनिषत्तात्पर्यं प्रत्यगभिन्ने अद्वये ब्रह्मण्येवेति वेदान्तराद्धान्तः  इति शम्~। 

\articleend
}
