{\fontsize{15}{17}\selectfont
\presetvalues
\chapter{पूर्वमीमांसाशास्त्रे मोक्षविमर्शः}

\begin{center}
\Authorline{वि~॥ प्रशान्तः}
\smallskip

सहायकप्राध्यापकः - संस्कृतविभागः\\
दयालसिंहमहाविद्यालयः\\
देहली
\addrule
\end{center}
 
\section*{उपक्रमः}

अखिलमपि भारतीयं दर्शनशास्त्रं प्रायेण मोक्षमेव चरमपुरुषार्थत्वेन स्वीकृत्य प्रवर्त्तते~।\break तत्र सन्ति नैका विप्रतिपत्तयो दर्शनेषु मोक्षस्वरूपनिरूपणे~। तथा हि लौकिकेषु सुख\-दुःखेष्वेव स्वर्गनरकादिव्यपदेशं भजन्तो देहात्मवादिनश्चार्वाकाः प्रत्यक्षैकशरणा देहोच्छेद एव मोक्ष\break इत्याचक्षते~। विचित्रवासनावशेन नीलपीतादिरूपेण प्रवहतो ज्ञानसन्तानस्य अशेषवासनोच्छे\-दान्नीलपीतादि वैचित्र्यं हित्वा केवलं विज्ञानसन्ततेः स्वरूपेणावस्थानं मोक्ष इति विज्ञानवादिनो बौद्धाः सड्.गिरन्ते~। ‘प्रदीपस्येव निर्वाणं विमोक्षस्तस्य तायिनः’ इतिस्वीकुर्वन्तो\break माध्यमिकाः शून्यवादिनो बौद्धाः दीपसन्तानस्येव ज्ञानसन्तानस्याप्युपरमं मोक्षं वदन्ति~।\break जैनास्तु  बन्धहेत्भावनिर्जराभ्यां कृत्स्नकर्मविप्रमोक्षणं मोक्षः~। तदनन्तरमूर्ध्वं गच्छत्या\-\break लोकान्तात्~। (तत्त्वार्थसूत्रम्-१०/२,१०/५) इति व्याहरन्तः सततमूर्ध्व गमनमालोकाकाश\-गमनं वा मोक्षमुरीकुर्वन्ति~। एकविंशतिप्रभेददुःखस्यात्यन्तिकी निवृत्तिर्मोक्ष इति नैयायिकाः~। समस्तात्मविशेषगुणोच्छेदोपलक्षिता स्वरूपस्थितिर्मुक्तिरिति वैशेषिकाः~। त्रिविधदुःखस्यैकान्तात्यन्ताभावो मोक्ष इति सांख्याः~। प्रकृतिपुरुषविवेकेन वैराग्यपरिपाकादष्टाड्गयोगानुष्ठा\-नादीश्वरप्रसादात् चित्तवृत्तेर्निरोधाद् द्रष्टुरात्मनो स्वरूपस्थितिर्मुक्तिरिति पातञ्जलाः~। प्रपञ्च\-विलयो मोक्ष इति शाङ्कराः~। इतोऽप्यधिकं विद्यन्ते वादाः मोक्षविषये, ते सर्वेऽपि सर्वदर्शनसड्.ग्रहादिग्रन्थेषु तत्तत्सम्प्रदायानुरोधेनावगन्तव्याः~। इमे सर्वेऽपि वादाः पार्थसारथिकमलाकरभट्ट प्रभृतिभिर्विद्वद्भिरेकैकशोऽनूद्य निराकृताः शास्त्रदीपिकादिग्रन्थेषु~। प्रकृते तु पूर्वमीमांसा\-शास्त्रे ‘मोक्षः’ मीमांसाचार्याणां विचारपदवीं कथड्.कारमारूढ इति निरूपयामः~। 

कर्मनिरूपणपरं मीमांसादर्शनमनन्यगत्या शरीराद् भिन्नमात्मतत्त्वं स्वीकृत्य प्रवर्त्तते~।\break अन्यथा तत्र कृतहानाकृताभ्यागमप्रसड्.ग एवापतेत्~। यद्यपि मीमांसासूत्रे न्यायादिसूत्रवदात्मास्तित्वप्रतिपादकं सूत्रं नोपलभ्यते~। तथापि 
\begin{verse}
सत्संप्रयोगे पुरुषस्येन्द्रियाणाम् 1..~। (जैमिनिसूत्रम् १/­१/४), \\
संस्कारास्तु पुरुषसामर्थ्ये.. - (जैमिनिसूत्रम् ३/­८/३) 
\end{verse}
इत्यादिसूत्रेषु पुरुषपदेन आत्मा एव गृह्यते~। शाबरभाष्ये तर्कपादे पञ्चमाधिकरणे वृत्तिकारग्रन्थे 

“यस्य तच्छरीरं सोऽपि तैर्यज्ञायुधैर्यज्ञायुधीत्युच्यते ........ प्राणादिभिरेनमुपलभामहे” इत्यादिकथनेन तर्केण श्रुत्या चात्मास्तित्वं साधितमस्ति~। श्लोकवार्त्तिकस्यात्मवादे भट्टाचार्येणाभाणि - 
\begin{verse}
शरीरेन्द्रियबुद्धिभ्यो व्यतिरिक्तत्वमात्मनः~। \\
नित्यत्वं चेष्यते शेषं शरीरादि विनश्यति~॥ (श्लो०वा०-७)\\
तस्मादुभयहानेन व्यावृत्त्यनुगमात्मकः~। \\
पुरुषोऽभ्युपगन्तव्यः कुण्डलादिषु स्वर्णवत्~॥ (श्लो०वा०-२८)\\
इत्याह नास्तिक्यनिराकरिष्णुरात्मास्तितां भाष्यकृदत्र युक्त्या~। \\ 
दृढत्वमेतद्विषयश्च बोधः प्रयाति वेदान्तनिषेवणेन~॥ (श्लो०वा०-१४८) 
\end{verse}
इत्थं यागादिकर्मणां कर्ता, तज्जन्यफलस्य च भोक्ता अनुच्छित्तिधर्मा, शरीरभिन्नः कश्चना\-वश्यकः~। अन्यथा कर्मफलव्यवस्थाया एवोच्छेदः स्यात्~। तस्मादेभिर्वचोभिरात्मास्तित्वं दृढं प्रत्यपीपदन् मीमांसकाः~। एवं सिद्धे आत्मास्तित्वे तस्यात्मनो मोक्षविचारः कर्तुम् शक्यते~। मीमांसाशास्त्रप्रतिपादितो धर्मः कर्मात्मक एवास्तीति कृत्वा सूत्रभाष्ययोः कर्मनिवृत्तिलक्षणस्य मोक्षस्य विचारो न दृश्यते~। तथापि भट्टाचार्यस्य वार्त्तिकग्रन्थे मोक्षचर्चा वर्त्तते~। इममेव\-वार्त्तिकग्रन्थमाधृत्योत्तरवर्तिभिर्मीमांसाचार्यैः स्वस्वग्रन्थेषु मोक्षवादो व्यमर्शि~। तथा चैतावता मीमांसादर्शने मोक्षविचारो नास्तीति केेषाञ्चित् कथनमपि न समीचीनं प्रतीमः~। भट्टकुमारिलेन श्लोकवार्त्तिकाख्ये स्वग्रन्थे सम्बन्धाक्षेपपरिहारवादे, आत्मवादे च प्रसक्तानुप्रसक्ततया मोक्षस्य चर्चा कृता वर्त्तते~। 	तदनन्तरमनेनैवाचार्येण तन्त्रवार्त्तिके व्याकरणाधिकरणे निःश्रेयसरूप\-फलस्य समालोचना विहितास्ति प्रसङ्गादेव~। अस्यैव तन्त्रवार्त्तिकस्य व्याख्यानं भट्टसोमेश्वर\-प्रणीतं न्यायसुधाभिधमतिविशदं तत्रायं सर्वोऽपि विषयः सम्यग्विवृतः, एतदनु षोडशतमे \-ईस्वीयशतके खण्डदेवेन स्वकीये भाट्टदीपिकाग्रन्थे चतुर्थाध्यायस्य तृतीयपादे पञ्चमेऽधिकरणे “मोक्षोऽपि यदि दुःखध्वंसरूपः...” इत्यादिना ग्रन्थेन विषयोऽयमतिसंक्षिप्ततया चर्चितः~। अस्यापि भाट्टदीपिकाग्रन्थस्य प्रभावलीनामकं व्याख्यानं खण्डदेवशिष्यशम्भुभट्टप्रणीतं, तत्र ”अधुना मीमांसकमतरीत्यापि न मोक्षस्य तज्जन्यत्वसंभवः.....”इत्यादि ग्रन्थादारभ्य आ अधिकरणान्तादमुं विषयं विस्तरेण न्यभान्त्सीद् ग्रन्थकारः शम्भुभट्टः~। एतदतिरिच्य पार्थसारथिः शास्त्रदीपिकायास्तर्कपादे कमलाकरभट्टो  मीमांसाकुतूहले च मोक्षवादनाम्नैव ग्रन्थभागमरीरचताम्~। नारायणपण्डितोऽपि मानमेयोदयग्रन्थे प्रमेयभागे द्रव्यनिरूपणप्रकरणे आत्मस्वरूपनिरूपणानन्तरं “ननु कौ पुनः स्वर्गापवर्गौ नाम ?........” इत्यादिना प्रारभ्य मोक्षमेवव्यमृक्षत्~। इतोप्यत्र मीमांसाग्रन्थेषु मोक्षविचारः उपलभ्येत~। ततो मीमांसकाः स्वर्गकामाः, न मोक्षाकाङ्क्षिण इति कथनम् अविचारितरमणीयमेव~। 
~\\[0.2cm]
सम्प्रतीदमेव सर्वं समाहृत्य मीमांसासम्मतं मोक्षवादं सन्दर्भिष्यामः~। मीमांसासूत्रेषु प्रथमाध्यायस्य तृतीयपादे प्रयोगोत्पत्यशास्त्रत्वाच्छब्देषु न व्यवस्था स्यात्~। (१/३/२४) इत्यतः सूत्रादारभ्य एकदेशत्वाच्च विभक्तिव्यत्यये स्यात्~। (१/३/२९) इति सूत्रपर्यन्तं षट्सूत्रात्मकमधिकरणं व्याकरणाधिकरणमित्युच्यते~।  याज्ञे कर्मण्यपभाषणं न कर्तव्यमिति स्वीकारादस्मिन्नधिकरणे शब्दसाधुत्वविधायकस्य व्याकरणशास्त्रस्य प्रामाण्यं वर्त्तते न वेति विचारो विहितः~। तत्र व्याकरणप्रामाण्यविचारो विषयः, तस्य प्रामाण्यं न वेति संशयः, व्याकरणं न प्रमाणं प्रामाण्यप्रतिपादिकायाः श्रुतेरभावात् इति पूर्वपक्षः, व्याकरणं प्रमाणं, न म्लैच्छित वै...... इत्यादीनां श्रुतीनां विद्यमानत्वात् इति सिद्धान्तेन व्याकरणप्रामाण्यं निरणायि~। एतावदेवाधिकरणस्यस्वरूपम्~। अस्याधिकरणस्य षट्सु सूत्रेषु प्रथमं सूत्रं (१/३/२४) पूर्वपक्षपरं, ततः द्वितीयं सूत्रं सिद्धान्तरूपम्, अवशिष्टं सूत्रचतुष्टयं सिद्धान्तमेव दृढीकरोतीति स्थितिः~। अत्र पूर्वपक्षं प्रतिपादयन् पीतमीमांसापयोनिधिरयं वार्त्तिककारो महाभाष्यवाक्यपदीयादिग्रन्थेषु वर्णितं व्याकरणशास्त्रस्य भूषणत्वं दूषणतारजोभिर्दूषयन् सिद्धान्तप्रतिपादनावसरे च तानि दूषणतारजांसि मीमांसातर्कवारिणा स्वयमेवाचिक्षलत्~। स्पष्टञ्चैतत् वार्त्तिकटीकायां न्यायसुधायां मूले च~। 

इदानीमस्मिन् व्याकरणप्रामाण्यप्रतिपादनात्मकेऽधिकरणे मोक्षचर्चा कुतोन्वागता इति प्रश्ने किमुत्तरमिति चेत्? प्रसड्गात्~। तथाहि तत्र सिद्धान्तपक्षोपस्थापनकाले साधुशब्दानां प्रयोगाश्रितानामेव फलसाधनत्वं भवति, न साधुशब्दज्ञानमात्रादेव फलं संभवतीति निरणैषीद्भट्टाचार्यः~। तथा चोक्तं तन्त्रवार्त्तिके “यथा ‘योश्वमेधेन यजते य उ चैनं वेद’ इति ज्ञानमात्रादेव ब्रह्महत्यातरणं यदि सिध्येत्, को जातुचिद् बहुद्रव्यव्ययायाससाध्यमश्वमेधं कुर्यात्~। तद्विधानं चानर्थकमेव स्यात्~। एवं शब्दज्ञानाच्चेद्धर्मः सिध्येत् को नामानेकताल्वादिव्यापारायासखेदमनुभवेत्”~। अत्रैवोक्तं “द्रव्यसंस्कारकर्मसु परार्थत्वात् फलश्रुतिरर्थवादः स्यात्~। (मीमांसासूत्रम्-४/३/१) इत्यनेन न्यायेन ज्ञानस्य पुरुषशब्दसंस्कारत्वेन निराकाङ्क्षस्य फल\-सम्बन्धाभावात्”~। 

इदमत्र तात्पर्यं  “यस्य पर्णमयी जुहूर्भवति न स पापं श्लोकं शृणोति” (तै० सं० - ३/५/७) इति द्रव्ये, “यदाड्.क्ते चक्षुरेव भातृव्यस्य वृड्क्ते” (तै०सं० ६/­१/१) इति संस्कारे “यत्प्रयाजानुयाजा इज्यन्ते” “वर्म वै तद्यज्ञाय क्रियते वर्म यजमानाय भातृव्याभिभूत्यै”\break ( तै० सं० २/६/१) इत्यसंस्कारकर्मणि वाक्यानि पठितानि सन्ति, तत्र पर्णमयी जुहूर्द्रव्यम्, अञ्जनं संस्कारः, प्रयाजानुयाजाः कर्माणि, एतेषां समीपे पठितानि फलश्रुतिवाक्यान्यर्थवादाः भवन्ति~। कुतः? परार्थत्वाद्~। जुह्वादिकं परार्थं= क्रत्वर्थं भवति~। ततः पर्णमयीत्वादेः पृथक् फलं भवितुं नार्हतीति सूत्रविवरणम्~। अनेन न्यायेन शब्दज्ञानमपि परार्थमेव भवति, क्रत्वर्थं भवतीति यावत्~। सूत्रे द्रव्यादित्रयाणामेवपारार्थ्यमुक्तं न तु शब्दज्ञानस्यापि? तर्हिशब्दज्ञानस्य पारार्थ्यं कुतः? एवं स्थिते शब्दज्ञानस्यान्तर्भाव एषु त्रिष्वेव कार्यः~। अत एव शब्दज्ञानस्यान्तर्गतिः संस्कारे क्रियते~। इदं ज्ञानं ज्ञेयं=शब्दं, ज्ञातारं=पुरुषं वा संस्करोतीति कृत्वा ‘इत्यनेन न्यायेन’ इत्यादिवार्त्तिकग्रन्थः संगच्छते~। एतदुपर्यपि प्रश्न उदेति यत् ज्ञानमात्रं संस्कारत्वेन पराड्गं भवत्यथवा किञ्चिद्ज्ञानं पराड्.गं न भवत्यपि? एतद्विषयकं वार्त्तिकमवतारयति महामेधावी भट्टः 

\begin{verse}
सर्वत्रैव हि विज्ञानं संस्कारत्वेन गम्यते~। \\
पराड्.गं चात्मविज्ञानादन्यत्रेत्यवधारणात्~॥ (तन्त्रवार्त्तिकम् ९२१)
\end{verse}

अयमत्राशयः  आत्मज्ञानं विहाय ज्ञानमात्रं संस्कार्यं संस्कुर्वन् पराड्.गं भवति अव\-धारणात् इति एवार्थे~। आत्मज्ञानादन्यत्रैव इत्यर्थः~। 7आत्मज्ञानं हि संयोगपृथक्त्वात् क्रत्वर्थ\-पुरुषार्थत्वेन ज्ञायते तेन विना परलोकफलेषु कर्मसु प्रवृत्तिनिवृत्यसम्भवात्”~। (तन्त्रवार्त्तिकम्-९२१)\break अत्रोक्तमस्ति यत् आत्मज्ञानं क्रत्वर्थमपि भवति पुरुषार्थमपि~। पुरुषार्थो हि द्विविधोत्र\break गृहीतः~। अभ्युदयरूपो निःश्रेयसरूपश्च~। सत्यामीदृश्यां स्थितौ निःश्रेयसापरपर्यायस्य\break मोक्षस्य निरूपणमवश्यमापतितम्~। अतोऽत्र “आत्मानमुपासीत” इत्यादिविधीनां साध्यस्था\-नापन्नस्य मोक्षाख्यस्य पुरुषार्थस्य विचारो विहितः~। संयोगपृथक्त्वन्यायस्तु प्रसिद्धो मीमांसा\-याम्~। तथा च सूत्रम्  

\begin{verse}
एकस्य तूभयत्वे संयोगपृथक्त्वम्~। (४/३/५)~। 
\end{verse}
अत्र तावत् संयुज्यते पदार्थः पदार्थान्तरेण विशिष्टो बोध्यते यत्र सः संयोगो वाक्यम्, तस्य पृथक्त्वात्~। विनियोजकवाक्यभेदात् इति यावत्~। मीमांसाशास्त्रं हि वाक्यविचारशास्त्रम्~। तत्र उपनिषत्सु कानिचन वाक्यान्यात्मज्ञानस्य क्रत्वर्थत्वं प्रतिपादयन्त्यपराणि तु पुरुषार्थत्वम्~। अतो विधिवाक्यभेदादेकमेवात्मज्ञानमुभयार्थं सम्पद्यते~। अयमेव संयोगपृथक्त्वन्यायः~। 

आहत्य तु मीमांसकमतरीत्या त्रिविधं सम्पद्यत आत्मज्ञानमिदं क्रत्वर्थपुरुषार्थभेदेन~। तदुपपाद्यते-

१. “अविनाशी वा अरेऽयमात्मानुच्छित्तिधर्मा मात्रासंसर्गस्त्वस्य भवति” (बृ० उप ०४/३/१५) “आत्मा वाऽरे द्रष्टव्यः” (बृ०उप०२/४/५) इत्यात्मज्ञानविधिरस्ति~। विधिरयं शरीरादात्मभेदं साधयन् कर्मप्रवृत्तिनिवृत्तिसिद्ध्यर्थत्वात् क्रत्वर्थो भवति~। क्रत्वर्थत्वाच्चास्य पारार्थ्यमस्ति, न पुरुषार्थत्वम्~। एवं चास्य मोक्षार्थत्वं निरस्तम्~। अत एवास्य फलश्रुतिवाक्यानि अर्थवादत्वेन स्वीकर्तव्यानि भवन्ति~। एतदर्थमुक्तं सम्बन्धाक्षेपपरिहारवादे
\newpage
\begin{verse}
आत्मा ज्ञातव्य इत्येतन्मोक्षार्थं न च चोदितम्~। \\
कर्मप्रवृत्तिहेतुत्वमात्मज्ञानस्य लक्ष्यते~॥ \\
विज्ञाते चास्य पारार्थ्ये यापि नाम फलश्रुतिः~। \\
सार्थवादो भवेदेव न स्वर्गादिः फलान्तरम्~॥ (श्लो०वा०-१॰३,४)\\
\end{verse}
~\\[-1cm]
तन्त्रवार्त्तिके च
~\\[-1cm]
\begin{verse}
क्रत्वर्थांशेपरार्थत्वादर्थवादः फलश्रुतिः~। (तन्त्रवा०-८८९ पूर्वार्धम्)
\end{verse}

२. “य आत्मा अपहतपाप्मा विजरो विमृत्युर्विशोको विजिघत्सोऽपिपासः सत्यकामस्सत्यसंकल्पः सोऽन्वेष्टव्यः स विजिज्ञासितव्यः” (छां.उप०-८/७/१) इति श्रुत्या तथा\break “बोद्धव्यो मन्तव्य” इत्यादिना यदात्मज्ञानविधानं वर्तते तत् “सर्वांश्च लोकानाप्नोति\break सर्वांश्च कामानाप्नोति (छां०उप०८/१२/६) इत्यादिकामवादलोकवादरूपवाक्यशेषबलाद् अभ्युदयफलरूपं पुरुषार्थं बोधयति~। अतोऽस्य विधानस्य पुरुषार्थत्वं सिध्यति~। 

३. “आत्मानमुपासीत” (बृ०उप०-१/४/६) इति श्रुत्यामात्मज्ञानस्य तृतीयस्य विधानं, तस्य “स खल्वेवं वर्तयन्यावदायुषं ब्रह्मलोकमभिसंपद्यते न च पुनरावर्त्तते” (छा०उप०-८/१५/१) इत्यपुनरावृत्यात्मकपरमात्मप्राप्त्यवस्थारूपं फलं वाक्यशेषोपनीतं मोक्षाख्यम्~। 
\hfill (भाट्टदीपिकाप्रभावली-४/३/५) 

अत्र च स खल्वेवमित्यादिफलश्रुतिवाक्यस्यार्थवादत्वं कल्पयितुं न शक्यते~। कुतः? अप्रकरणगतत्वादस्य वाक्यस्य~। विधिप्रकरणे पठितस्य फलश्रुतिवाक्यस्य अर्थवादत्वं संभवति~। इदं तु न तथा~। ततो नास्याञ्जनादिवदर्थवादता~। एवमस्यापि विधेः पुरुषार्थत्वं निःश्रेयसफलरूपेण सिद्धम्~। स खल्वेवमित्यादिवाक्यशेषस्य फलसमर्पकत्वात्~। ये च वार्त्तिककृता संयोगपृथक्त्वन्यायेनात्मज्ञानस्य क्रत्वर्थपुरुषार्थत्वे उक्ते तेऽपि समन्वगाताम्~। इत्थञ्च कर्मानुष्ठानधुरीणवाक्यार्थविचारविचक्षणा मीमांसकाः श्रुतिवाक्येभ्य एव मोक्षोऽपि सिध्यतीत्यसीषिधन्~। 

सम्प्रति मीमांसकमते मोक्षस्वरूपं कीदृशमित्यालोचयामः~। तत्र मोाक्षस्वरूपविषये विप्रतिपद्यन्ते मीमांसकाः~। भाट्टास्त्वानन्दमोक्षवादिनः सन्तीति मानमेयोदये स्पष्टमुक्तम्-
\newpage

\begin{verse}
दुःखात्यन्तसमुच्छेदे सति प्रागात्मवर्तिनः~। \\
सुखस्य मनसा भुक्तिर्मुक्तिरुक्ता कुमारिलैः~॥ (प्रमेयप्रकरणम् श्लोकः-२५)
\end{verse}
कथनस्यास्य मूलं श्लोकवार्त्तिकमस्ति~। किन्तु भाट्टपन्थानमनुरुन्धानोपि पार्थसारथिमिश्रः आनन्दमोक्षवादं न सहते~। श्लोकवार्त्तिकस्थमानन्दमोक्षवादं परमतत्वेन व्यवस्थाप्य दुःखाभाव एव मोक्ष इति मेने सः~। तथा हि
\begin{verse}
सुखोपभोगरूपश्च यदि मोक्षः प्रकल्प्यते~। \\
स्वर्ग एव भवेदेष पर्यायेण क्षयी च सः~॥ 
\end{verse}
इत्यादीनि वार्त्तिकानि दुःखाभावरूपं मोक्षं प्रतिपादयितुं युक्तित्वेन स्मरति~। एतावता भाट्टेषु मतद्वयमागतम्~। एकः आनन्दमोक्षवादः, द्वितीयस्तु दुःखाभावरूपः~। 

एवं मीमांसावेदान्तयोरपि मोक्षस्वरूपविषये मतभेदो वर्तते~। वेदान्ते प्रपञ्चविलयो मोक्षः इति स्वीक्रियते प्रपञ्चस्याविद्यानिर्मितत्वात्~। पार्थसारथिमते तु प्रपञ्चसम्बन्धविलयो मोक्षः इति वर्तते~। यतो हि मीमांसामते प्रपञ्चो नाविद्यानिर्मितः किन्तु सत्य एव~। तथा हि  “त्रेधा हि प्रपञ्चः पुरुषं बध्नाति भोगायतनं  शरीरं, भोगसाधनानीन्द्रियाणि, भोग्याः विषयाः~। भोग इति च सुखदुःखविषयोऽपरोक्षानुभव उच्यते~। तदस्य त्रिविधस्यापि बन्धस्यात्यन्तिको विलयो मोक्षः” (शा०दीपिका मोक्षवादः)

अपरंमतमस्ति प्राभाकराणां, ते तु नियोगसिद्धिरेव मोक्षः (प्रकरणपञ्चिका तत्त्वालोक\-प्रकरणम्) इति मेनिरे~। नियोगो नाम - आज्ञा~। फलेच्छां परित्यज्य कर्तव्यबुद्ध्या नित्यकर्मणामनुष्ठानमिति~। अतः प्रभाकरमतरीत्या आत्मज्ञानपूर्वकं वैदिककर्मणामनुष्ठाने धर्माधर्मौ नश्येते ततो देहेन्द्रियादिसम्बन्धो स्वभावेनैव नश्यति, स एव मोक्षः~। कमलाकरभट्टस्तु आनन्दमोक्षवादमेव स्वीकरोति~। वदति च मीमांसाकुतूहले “तत्त्वतस्त्वस्माकं मोक्षो वेदान्तिमतादभिन्न एव” ..... पार्थसारथिस्तु वेदान्ताध्ययनशून्यो भ्रान्त एव” इति~। विषयोऽयं शास्त्रदीपिकादिषु विस्तरेण विमृष्टः~। अधिकं तत एवावधेयः~। 

इदानीं मीमांसापन्थानमनुसृत्य मोक्षसाधनानि विमृश्यन्ते~। तन्त्रवार्त्तिकानुसारं आत्मानमुपासीत इति श्रुत्या विहितात्मज्ञानस्य फलं मोक्ष इति न्यरूपि~। निरूपणेनानेनायाति यत् ज्ञानमेव मोक्षसाधनम्~। तर्हि कर्म एव सर्वस्वमिति मन्यमानानां मीमांसकानां का गतिरित चेत्  सद्गतिरेव~। कुतः?
\begin{verse}
न च ज्ञानविधानेन कर्मसम्बन्धवारणम्~। 
\end{verse}

प्रत्याश्रमवर्णनियतानि नित्यनैमित्तिककर्माण्यपि पूर्वकृतदुरितक्षयार्थमकरण निमित्तानागतप्रत्यवायपरिहारार्थं च कर्तव्यानि~। (तन्त्रवा० ९२२ व्याकरणाधिकरणम्) सत्यपि ज्ञाने नित्यनैमित्तिककर्मानुष्ठानं विना पूर्वकृतदुरितक्षयो न स्यात्~। अकरणे च प्रत्यवायो भवेत्~।\break प्रत्यवायस्य फलोपभोगार्थं च शरीरधारणमनिवार्यं भवेत्~। अतः काम्यनिषिद्धे कर्मणी\break परित्यज्य नित्यनैमित्तिकानुष्ठानं कर्त्तव्यमेव~। इदमेवोक्तं श्लोकवार्त्तिकेऽपि सम्बन्धाक्षेपपरि\-हारवादे

\begin{verse}
मोक्षार्थी न प्रवर्तेत तत्र काम्यनिषिद्धयोः ~। \\
नित्यनैमित्तिके कुर्यात् प्रत्यवायजिहासया~॥ (श्लो०वा०-११०)
\end{verse}

अस्मिन् विषये “तमेतं वेदानुवचनेन ब्राह्मणा विविदिषन्ति ब्रह्मचर्येण तपसा श्रद्धया यज्ञेनानाशकेन ......” इति विविदिषा श्रुतिः प्रमाणमस्ति~। स्मृतिरपि तथैव ”न मे पार्थास्ति कर्त्तव्यं ..... वर्त एव च कर्मणि~॥ ” न केवलमेतावदपि तु “न बुद्धिभेदं जनयेत् ....... जोषयेत् सर्वकर्माणि विद्वान् युक्तः समाचरन्~॥ ” इत्यत्र ज्ञातात्मतत्त्वः अन्यानपि कर्मसु योजयेदिति प्रत्यपादि भगवता गीतायाम्~। एवं स्थिते ”क्षीयन्ते चास्यकर्माणि” इत्यादिश्रुतीनां ”ज्ञानाग्निः सर्वकर्माणि भस्मसात् कुरुतेऽर्जुन” इत्यादिस्मृतीनां ज्ञानात् मोक्षः इति प्रतिपादिकानां स्तावकत्वमथवा कर्मसहितज्ञानात् मोक्षः इति व्याख्याने न क्वापि विप्रतिपत्तिः~। ”ज्ञानादेव तु कैवल्यम्” इत्यत्र तु ज्ञानप्रतिस्पर्धिनोऽज्ञानस्य व्यावृत्तिं कृत्वा एवकारः समर्थनीयः~। ज्ञानादेव न तु अज्ञानादिति~। अत्र च आत्मज्ञानानन्तरमपि कर्मानुष्ठानं कर्त्तव्यमिति प्रतिपादनाय कथमेतावदायासः इति चेत्? श्रुतेरेव प्रामाण्यं सम्पादयितुम्~। तथा च 

\begin{verse}
अन्धन्तमः प्रविशन्ति येऽविद्यामुपासते~। \\
ततो भूय इव ते तमो य उ विद्यायां रताः~॥\\
\hspace{3cm}(ईशावास्यम् - ९+, बृहदारण्यकम् - ४.४.१०)
\end{verse}
इति ईशावास्ये बृहदार्ण्यके च प्रसिद्धञ्च~। अत्र तत इति ल्यब्लोपे पञ्चमी~। तथा च- ये अविद्याम् अग्निहोत्रादिकाम्यकर्मलक्षणामुपासते तेऽन्धं तमोऽज्ञानरूपं प्रविशन्ति~। त एव तानि कर्माणि विहाय केवलं विद्यायां रताः ते भूय इव बहुतमः प्रविशन्ति~। कुतः? कर्मणामपि त्यागादिति~। अत्र तच्छब्दोपात्तः एक एव कर्त्ता~। तस्यैव विद्या-अविद्यानुष्ठानमिति व्याख्यानात् ज्ञानकर्मणः समुच्चयः स्वीकार्यः~। तस्मात् ज्ञानकर्मसमुच्चयवन्त एव तमस्तरन्तीति~। (द्रष्टव्या प्रभावली ४/३/५) इत्थं ज्ञानकर्मसमुच्चयः मीमांसकमते मोक्षसाधनम्~। अत्र मीमांसकानां विवादो नास्ति~। 

\section*{उपसंहारः}

एवमसत्यपि मोक्षवर्णने सूत्रभाष्ययोरुपलब्धमूलो मोक्षः तन्त्रवार्त्तिके व्याकरणाधिकरणे~। तन्मूलमादायैव न्यायप्रकाशार्थसड्ग्रहादिषु “सोऽयं धर्मो यदुद्देशेन विहितः तदु्द्देशेन क्रियमाणस्तद्धेतुः~। ईश्वरार्पणबुद्ध्या क्रियमाणस्तु निःश्रेयसहेतुः” इत्यादि न्यगादि~। तथा च तत्त्वज्ञानिनामपि नैष्कर्म्यं निवारयन्तो मीमांसका वेदप्रामाण्यैकशरणाः श्रुतिप्रतिपादितं मोक्षमपि नोज्झन्ति~। तत्र मीमांसकेषु मोक्षस्वरूपविषये मतत्रयमस्ति~। भाट्टानां मतद्वयम् - आनन्दमोक्षवादः दुःखाभावरूपश्च~। प्राभाकराणांत्वेकं मतम्-नियोगसिद्धिर्मोक्षः~। मोक्षसाधनन्तु ज्ञानकर्मसमुच्चयः~। मोक्षसाधनविषये मीमांसकानां परस्परं न विवाद इति संक्षेपः~। विस्तरस्तु श्लोकवार्त्तिकतन्त्रवार्त्तिकादिग्रन्थेषु द्रष्टव्यः~। 

~\\[-1cm]
 
\section*{सन्दर्भग्रन्थाः-} 
\vskip -1cm
 
\begin{thebibliography}{99}
\itemsep=0pt
\bibitem{chap12key1} अध्वरमीमांसा कुतूहलवृत्तिः --- वासुदेवदीक्षितः~। 
\bibitem{chap12key2} अर्थसड्.ग्रहः --- लौगाक्षिभास्करः~। 
\bibitem{chap12key3} तन्त्रवार्त्तिकम् --- कुमारिलभट्टः~। 
\bibitem{chap12key4} न्यायसुधा --- भट्टसोमेश्वरः (तन्त्रवार्त्तिकटीका)~। 
\bibitem{chap12key5} प्रभावली --- शम्भुभट्टः (भाट्टदीपिकाटीका)~। 
\bibitem{chap12key6} भाट्टदीपिका --- खण्डदेवः~। 
\bibitem{chap12key7} महाभाष्यम् --- पतञ्जलिः~। 
\bibitem{chap12key8} मानमेयोदयः --- नारायणद्वयी~। 
\bibitem{chap12key9} मीमांसाउद्धरणकोशः --- डी.जे. अग्रवालः~। 
\bibitem{chap12key10} मीमांसाशाबरभाष्यम् --- युधिष्ठिरमीमांसकः~। 
\bibitem{chap12key11} मीमांसान्यायप्रकाशः --- आपदेवः~। 
\bibitem{chap12key12} मीमांसाकुतूहलम् --- कमलाकरभट्टः~। 
\bibitem{chap12key13} मीमांसानयमञ्जरी --- पट्टाभिरामशास्त्री~। 
\bibitem{chap12key14} श्लोकवार्त्तिकम् --- कुमारिलभट्टः~। 
\bibitem{chap12key15} सर्वदर्शनसड्.ग्रहः --- माधवाचार्यः~। 
\end{thebibliography}

\articleend
}
