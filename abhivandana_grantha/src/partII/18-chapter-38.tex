{\fontsize{15}{17}\selectfont
\presetvalues
\chapter{व्याप्तिः साध्यवदन्यस्मिन्  असम्बन्ध उदाहृतः}

\begin{center}
\Authorline{डा~॥ शिवरामः}
\smallskip

संस्कृताध्यापकः\\
श्रीवाणीविद्याकेन्द्रम्
\addrule
\end{center}

का नाम व्याप्तिः इति अत्र चिन्त्यते~। साध्यवान् – साध्यनिष्ठ-आधेयतानिरूपित अधिकरणतावान्, साध्यवदन्यः- साध्यवत्प्रतियोगिकभेदनिष्ठ-आधेयतानिरूपित अधिकरणतावान्~। असम्बन्धः--वृत्तित्वप्रतियोगिकाभावः~। एवञ्च– “साध्यनिष्ठ- आधेयतानिरूपित-अधिकरणतावत्प्रतियोगिकभेदनिरूपितवृत्तित्वाभावः’’ इति लक्षणं फलितम्~। 

“पर्वतो वह्निमान् धूमात्’’ इत्यत्र साध्यः वह्निः, वह्निनिष्ठ आधेयतानिरूपित अधिकरणतावान् पर्वतः~। तत्प्रतियोगिक भेदः वह्निमान् न इति प्रतिसाक्षिकः~। तादृश भेदवान् ह्रदः~। तन्नितूपितवृत्तित्वं जलशैवलादौ~। वृत्तित्वाभावः धूमे वर्तते इति समन्वयः~। 

लक्षणे केन सम्बन्धेन साध्यवान् बोध्यः इति नोक्तः~। अतः अत्र समवायसम्बन्धेन साध्यवान् स्वीक्रियते~। समवायसम्बन्धेन वह्निनिष्ठ -आधेयतानिरूपित-अधिकरणतावान् वह्नेरवयवः~। प्रकृते साध्यनिष्ठ-आधेयतानिरूपित-अधिकरणतावत्प्रतियोगिकः भेदः इत्युक्ते समवायेन वह्निमान् न इति प्रतीतिसाक्षिकः भेदः~। तादृशभेदाधिकरणं पर्वतः~। तन्निरूपितवृत्तित्वमेव धूमे वर्तते~। वृत्तित्वाभावः नास्ति इति अव्याप्तिः~। अव्याप्तिवारणाय साध्यनिष्ठ-आधेयतायां साध्यतावच्छेदकसम्बन्धावच्छिन्नत्वं निवेश्यते~। एवञ्च- लक्षणं इत्थं फलितम् – साध्यतावच्छेदकसम्बन्धावच्छिन्न- साध्यनिष्ठ-आधेयतानिरूपित-अधिकरणतावन्निष्ठ-प्रतियोगिताक-भेदवन्निरूपित- वृत्तित्वाभावः~। 

वह्निसाध्यकस्थले साध्यतावच्छेदकसम्बन्धः संयोगसम्बन्धः~। संयोगसम्बन्धावच्छिन्न आधेयतानिरूपित अधिकरणतावान् पर्वतः भवति~। न तु वह्नेरवयवः~। तादृश अधिकरणतावन्निष्ठ प्रतियोगिताकभेदः संयोगेन वह्निमान् न इति प्रतीतिसाक्षिकः~। तादृशभेदाधिकरणं ह्रदः~। तन्निरूपितवृत्तित्वाभावः धूमे वर्तते इति समन्वयः~। 

एवं सति पर्वतो वह्निमान् धूमात् इत्यत्र अव्याप्तिः~। अत्र साध्यवान् वह्निमान् पर्वतः~।\break तादृश वह्निमन्निष्ठप्रतियोगिताकः भेदः पर्वतो न इति प्रतीतिसाक्षिकः भेदः~। तादृशभेदा\-धिकरणं महानसः~। तन्निरूपितवृत्तित्वाभावः धूमे नास्ति इति अव्याप्तिः~। तन्निवारणाय\break साध्यनिष्ठप्रतियोगितायां साध्यवत्त्वावच्छिन्नत्वं निवेशनीयम्~। एवञ्च लक्षणं साध्यतावच्छेद\-कसम्बन्धावच्छिन्न-साध्यनिष्ठ-आधेयतानिरूपित- अधिकरणतानिष्ठप्रतियोगिताकभेदवन्निरूपितवृत्तित्वाभावः इति फलितम्~। इदानीम् साध्यवान् वह्निमान्, साध्यवत्वावच्छिन्न-प्रति\-योगिताकभेदः वह्निवत्वावच्छिन्न प्रतियोगिकभेदः वह्निमान् न इति प्रतीतिसाक्षिकभेदः~।\break तादृशभेदाधिकरणं महानसः न भवति~। किन्तु ह्रदादिकम्~। तन्निरूपितवृत्तित्वाभावः धूमे अस्ति इति लक्षणसमन्वयः~। 

\articleend
}
