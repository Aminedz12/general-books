\chapter{व्याकरणमहाभाष्यस्य महत्त्वं वैशिष्ट्यञ्च}

\begin{center}
\Authorline{डा. उदय भट्टः}
\smallskip

व्याकरणविभाग\\
महाराजसंस्कृतमहापाठशाला\\
मैसूरु
\end{center}
 	
इह खलु सकलव्यवहारनिर्वाहणाय भाषैव जुषते प्रधानभावम् । अन्तरेण भाषां जना न समर्था भवन्ति स्वीयमाशयम् अवबोधयितुम् । अपरञ्च भाषयापि सुसंस्कृतया भवितव्यम् । सुसंस्कृता भाषा हि शब्दसाधुत्वमपेक्षते । साधुशब्दप्रयोगे च व्याकरणाध्ययनं प्रधानं कारणम् । भगवान् भर्तृहरिरपि स्वीये वाक्यपदीयग्रन्थे एवं समुल्लिलेख-
\begin{verse}
साधुत्वज्ञानविषया सैषा व्याकरणस्मृतिः ॥ इति ।
\end{verse}
शब्दसंस्कारार्थं प्रवृत्तमिदं शब्दशास्त्रम् । शब्दानां साधुत्वविवेचनमेव शास्त्रस्यास्य मुख्यम् उद्देश्यं वर्तते । नैके शाब्दिकग्रन्था नैकेषां शब्दानां संस्कारम् अकुर्वन् । तेषु ग्रन्थेषु त्रिमुनिव्याकरणापरनामधेयः पाणिनिविरचितोऽष्टाध्यायीनामकोऽयं ग्रन्थो वैशिष्ट्यं महत्त्वञ्च भजते । एतच्छास्त्रं पाणिनिप्रणीतं कात्यायनोपबृंहितं पतञ्जलिप्रतिष्ठापितं सत् आसेतुहिमाचलम् अध्ययनाध्यापनपरम्परायाम् अद्यापि विराजते । पाणिनीयव्याकरणस्य परममहत्त्वपूर्णमङ्गं वर्तते वररुचेर्वार्तिकपाठः । वार्त्तिकम् अन्तरेण पाणिनीयं व्याकरणम् अपूर्णमेव प्रतिभाति । पतञ्जलिः कात्यायनीयं वार्तिकपाठमवलम्ब्यैव स्वीयं महाभाष्यं व्यरचयत् । सकलेऽपि संस्कृतगद्यसाहित्ये इदं महाभाष्यम् अद्वितीयोऽप्रतिमश्च ग्रन्थः ।

\section*{महाभाष्याभिदानस्य सार्थकता}

शेषावतारित्वेन लोके प्रथितः पाणिन्यनुयायिभिः सश्रद्धं सगौरवं सभक्ति समाश्रितश्च पतञ्जलिः कात्यायनस्य वार्त्तिकपाठव्याख्यानव्याजेन कृत्स्नम् अष्टाध्यायीम् अनितरसाधारण्येन वैदुष्येन स्वबुद्धिप्रतिभाकलापेन च करतलामलकमिव व्याचख्यौ । ग्रन्थस्यास्य महत्त्वं परिगणय्य वाक्यपदीयव्याख्याकारेण पुण्यराजेन प्रोक्तं ‘कृतोऽथ पतञ्जलिना....’ इति कारिकाव्याख्यावसरे - “तच्च भाष्यं न केवलं व्याकरणस्य निबन्धनं, यावत्सर्वेषां न्यायबीजानां बोद्धव्यमित्यथ एव सर्वन्यायबीजहेतुत्वादेव महच्छब्देन विशेष्य महाभाष्यम् इत्युच्यते लोके इति । अस्यायमाशयः - शास्त्रान्तरेष्वपि भाष्यग्रन्थाः सन्ति बहवः । द्वैताद्वैतविशिष्टाद्वैतशक्तिविशिष्टाद्वैतिनां तत्तदाचार्यप्रणीता भाष्यग्रन्थाः पृथक् पृथक् वर्तन्ते । उदा - मध्वभाष्यं, शाङ्करभाष्यं, श्रीभाष्यं, श्रीकरभाष्यम् इति । मीमांसान्यायशास्त्रादीनां टीकाग्रन्थाः केचन भाष्यपदेन व्यवह्रियन्ते । उदा - शाबरभाष्यं, न्यायभाष्यम् इत्यादयः । एषां समेषां ग्रन्थानां केवलं भाष्यपदेनैव व्यवहारः । किन्तु श्रीमद्भगवत्पतञ्जलिप्रणीतस्य शब्दशास्त्रटीकाग्रन्थस्य महाभाष्यपदेन व्यवहारस्त्वस्य महत्त्वबोधनाय लोके आदरेण क्रियते । इतरशास्त्रभाष्यग्रन्थानां विषये तावत् एवं वक्तुं शक्नुमः - तत्र सूत्रभाष्ययोर्मध्ये यस्मिन् कस्मिन् अपि विषये वैरुध्यं सञ्जातश्चेत् कस्याभिप्रायः स्वीकार्यः इति संशये सूत्रमतमेव परमप्रमाणत्वेनाङ्गीक्रियते । सूत्रकारा एव तत्र प्रबलाः । तेषामेवाभिमतम् अवश्यं समाश्रयणीयम् । तत्र सूत्रकारापेक्षया भाष्यकारस्य दौर्बल्यमेव । परन्तु शब्दशास्त्रे न तथा । अत्रस्था स्थितिरेवान्या । पाणिनीयव्याकरणपरम्परायां तावत् त्रिभिर्मुनिसदृशैः शाब्दिकमचर्चिकाभिः शब्दसाधुत्वविषये विचारमन्थनं कृतम् । त एव सूत्रवार्त्तिकभाष्यकाराः पाणिनिकात्यायनपतञ्जलयः । शब्दानां साधुत्वनिर्णयो न सरलः । अतीव दुष्करः पन्थाः सः । अस्मिन् सुदुष्करे पथि प्रयातः सूत्रकारः पाणिनिरुपचतुः सहस्रसूत्रैर्वैदिकलौकिकोभयविधशब्दानामनुशासनं चकार । गच्छति काले अस्मिन् शब्दसाधुत्वनिश्चयकर्मणि इतोऽप्यतिशयेन सोत्साहं प्रवृत्तेन वार्तिककृता भगवता वररुचिना स्वबुद्धिप्रतिभाकलापेन पाणिनीयं किञ्चिदिव तत्र तत्र परिष्कृतम् । तत्र उत्तरकाले कात्यायनवार्त्तिकव्याख्यानव्याजेन शेषावतारिणा पतञ्जलिमहर्षिणा अनितरसाधारण्या शेमुष्या कृत्स्ना अष्टाध्यायी निकषायिता । एवम् अनारतं प्रवृत्ताभ्यां शब्दसाधुत्वस्य चिन्तनमन्थनाभ्यां समुत्पन्नोऽयं नवनीतः सम्प्रत्यस्माकं पुरतः पाणिनीयव्याकरण परम्परारूपेण विराजते ।

अत्रावधेयोऽयमंशः - शब्दानां साधुत्वविषये सम्प्रति काले परमप्रमाणभूता मूलग्रन्थास्त्रयो वर्तन्ते - पाणिन्याचार्यकृताष्टाध्यायी, कात्यायनविरचितो वार्त्तिकपाठः, भगवत्पतञ्जलिप्रणीतं महाभाष्यञ्चेति । तत्रापि शब्दसाधुत्वविषये सूत्र- वार्त्तिकयोर्मध्ये ऐकमत्याभावे वार्त्तिककृदभिप्राय एव ग्राह्यः । सूत्रभाष्ययोर्मध्ये वार्त्तिकभाष्ययोर्मध्ये च विरोधे सञ्जाते भाष्यमतमेव परमप्रमाणत्वेनाङ्गीकार्यम् । यतो हि शब्दसाधुत्वनिर्णयावसरे सूत्राणां वार्त्तिकानाञ्च तलस्पर्श्यध्ययनं क्रुत्वा, शास्त्रान्तरप्रयोगान्यपि परिशील्य, सम्यग्विचार्य विमृश्य च भगवता पतञ्जलिना निर्णयोऽकारि । इममेवाभिप्रायं पुष्णातीदं भट्टोजिदीक्षितवचनम् - ‘यथोत्तरं मुनीनां प्रामाण्यम्’ इति - ( सि. कौ.‘न बहुव्रीहौ’ सूत्रव्याख्या ) । महर्षिः पतञ्जलिः स्वीयग्रन्थे सूत्रकारं वार्त्तिककारञ्च ‘भगवान्’ ‘आचार्य’ इति सगौरवं व्याहरत् । तयोरुभयोर्विषये विद्यमानम् अभिमानं प्रदर्शयत्ययं व्यवहारः । एवं सत्यपि कुत्रचित् सूत्रमतं पुरस्कृत्य वार्त्तिकानि प्रत्याख्याति भाष्यकारः । अपरत्र वार्त्तिकमतमवलम्ब्य सूत्रकाराभिप्रायं तिरस्करोति । क्वचिदवसरेषु सूत्रकृद्वार्त्तिककृतोर्द्वयोरप्यभिमतं परित्यज्य ‘इष्यते इष्टिः’ इत्यादिपदप्रयोगेण स्वतन्त्ररूपेण स्वाभिप्रायं प्रकटयति । न बहुव्रीहौ ( २-१-२९) इत्यादिसूत्राणां प्रत्याख्यानं विचारममुं स्पष्टयति । तस्मिन् सूत्रभाष्ये
\begin{verse}
अकच्स्वरौ तु कर्तव्यौ प्रत्यङ्गं मुक्तसंशयम् ।\\
त्वक्त्पितृकमत्पितृक इत्येव भवितव्यम्” ॥
\end{verse}
इति निर्दिश्य त्वकं पिता यस्य सः = त्वकत्पितृक, अहकं पिता यस्य सः = मकत्पितृक इत्यत्र अकच्प्रत्यय एव भवति इति स्पष्टं न्यगादि पतञ्जलिमहर्षिणा । सूत्रकारस्तु अत्र कप्रत्ययं पुरस्क्रुत्य त्वत्कपितृक-मत्कपितृक इत्यनयोः साधुत्वमङ्गीचकार । एवमत्र सूत्रकाराभिप्रायोऽस्वीकृतो भाष्यकारेण । सर्वैः पाणिन्यनुयायिभिः शाब्दिकैरयमेव निर्णयः शिरसि सन्धार्य अनुस्रियते पाल्यते च । एवञ्च स्वेष्टिकथनेन स्वाभिप्रायकथनेन स्वाभिप्रायसमर्थनेन च अतिसूत्रकारः सञ्जातः पतञ्जलिः । एवंरीत्या पतञ्जलिभाष्यस्य महत्त्वम् अधिकं वर्तते । अत एव शास्त्रान्तरेषु सूत्रकारमतमेव स्वीकुर्याणानाम् आचार्याणां ग्रन्थाः केवलं भाष्यपदभाजः सञ्जाताः । शब्दशास्त्रे तु स्वाभिप्रायप्रकटनेन समर्थनेन च सूत्रवार्त्तिकमतम् अतिक्रान्तेन पातञ्जलभाष्यग्रन्थेन महत्त्वपदवी अधिगता । अत एव महाभाष्यपदभागभवत् अयं पातञ्जलो ग्रन्थः । एवं शास्त्रान्तरप्रणीतभाष्यग्रन्थापेक्षया एतद्भाष्यम् इष्टिघटितं सत् सूत्रापेक्षया प्रामाण्यमधिकम् आवहति । अस्मादेव कारणात् अस्य ग्रन्थस्य ‘महाभाष्यम्’ इति नामधेयम् इत्यभिप्रैति वैयाकरणमूर्धन्यो नागेशभट्तः । “व्याख्यातृत्वेऽप्यस्य इष्ट्यादिकथनेन अन्वाख्यातृत्वात् इतरभाष्यवैलक्षण्येन महत्त्वम्” इति । (व्या. म. भा - उद्योतव्याख्यानम् २-२-२)

\section*{महाभाष्ये निरूपिताः प्रधानविषयाः}

महाभाष्यस्य भाषासौष्ठवं, कठिनतरस्यापि शास्त्रविषयस्य सुलभतया विवेचनं, सर्वत्र शब्दस्य शब्दार्थस्य वा साफल्यम् इत्यादिगुणगणाः सर्वान् आकर्षयति इत्यभिप्रैति पण्डितो भार्गवशास्त्रीमहाभागः । महाभाष्ये तावत् शब्दशास्त्रस्य प्रस्तावरूपेण ‘अथ शब्दानुशासनम्’ इत्यस्य ‘गौरीत्यत्रकः शब्दः’ इति प्रश्नपूर्वकं स्वयमुत्थाप्य शब्दतत्त्वस्यानावरणं कृतवान् भाष्यकारः । ‘प्रयोजनमनुद्दिश्य न मन्दोऽपि प्रवर्तते’ इति विचारं समर्थयन् पतञ्जलिः ‘कानि पुनः शब्दानुशासनस्य प्रयोजनानि ’ इति पृष्ट्वा शब्दानुशासनस्य मुख्यप्रयोजनानि, एवम् आनुषङ्गिकप्रयोजनानि सविस्तरं मनोज्ञतया निरूपयामास । शब्दस्य नित्यानित्यत्वपक्षौ साधारं प्रतिपाद्य, वर्णानां सार्थकनिरर्थकविषयनिरूपणं कृत्वा, व्याकरणपदार्थं विचार्य, शब्दानां ज्ञानेन धर्मावाप्तिः आहोस्वित् प्रयोगेण इति विचारं सयुक्ति विवृत्य शब्दार्थसम्बन्धानां नित्यत्वं लोकादेव सिध्यतीति प्रतिपादयामास भगवान् भाष्यकारः ।

\section*{महाभाष्यस्य भाषासौष्ठवम्}

पूर्वोक्तानां समेषां विषयाणां मिरूपणम् अतीव सारल्येन कृतं वर्तते । संस्कृतगद्यसाहित्ये शास्त्रसाहित्ये च एतादृशीं भाषाशैलीम्, ईदृशं वाक्यसौष्ठवम् महाभाष्यमन्तरा अन्यत्र कुत्रापि द्रष्टुमशक्ता वयम् । परन्तु तत्र भवन्तो भाष्यकारस्य सरलनिरूपणपरिपाटिम् 

कथं पुनर्ज्ञायते सिद्धः ( नित्यः ) शब्दोऽर्थः सम्बन्धश्चेति ? लोकतः ॥ यल्लोके अर्थमर्थमुपादाय शब्दान् प्रयुञ्जते, नैषां निर्वृत्तौ यत्नं कुर्वन्ति । ये पुनः कार्या (अनित्याः) भावा निर्वृत्तौ तावत् तेषां यत्नः क्रियते । तद्यथा - घटेन कार्यं करिष्यन् कुम्भकारकुलं गत्वाह - कुरु घटं, कार्यमानेन करिष्यामीति । न तद्वत् शब्दान् प्रयुयुक्षमाणो वैयाकरणकुलं गत्वाह - कुरु शब्दान्, प्रयोक्ष्य इति । तावत्येव (वैयाकरणकुलमगत्वैव, बुध्या वस्तुनिरूप्य ) अर्थमुपादाय शब्दान् प्रयुञ्जते । (व्या. म. भा. २-२-२)

एवम् अतीव सारल्येन यथा गुरुः शिष्यान् बोधयति तथा शास्त्रस्थान् गहनविचारान् प्रबोध्य शब्दतत्त्वजिज्ञासूनां सन्देहसन्दोहान् दूरीकरोति पतञ्जलिः । उदाहरणप्रत्युदाहरणैः सह लौकिकन्यायान् संयोज्य सरलया सरसया च भाषया यद्विवरणं क्रियते तत् पातञ्जले महाभाष्ये एव द्रष्टुं शक्यम् ।

\section*{महाभाष्यस्य गाम्भीर्यम्}

सरससरलनिरूपणमात्रमवलोक्य भाष्येऽस्मिन् विषयगाम्भीर्यमेव नास्तीति न भ्रमितव्यम् । अस्य विचारगाम्भीर्यमवलोक्य कैयटोपाध्यायादयः शब्दमर्मज्ञा अपि “भाष्याब्धिः क्वातिगम्भीरः क्वाहं मन्दमतिस्ततः” इति सगौरवं निजगदुः । (व्या. म. भाष्यम् - ६.१-१-१) सरलभाषया सुनिरूपितमपि शब्दतत्त्वात्मकं भाष्यसिद्धान्तं समधिगन्तुं स्वल्पप्रज्ञा अर्थान् अकृतबुद्धयोऽसमर्था एव भवन्तीत्यभिप्रैति वाक्यपदीयकारः ।
\begin{verse}
अलब्धगाधे गाम्भीर्यात् उत्तान इव सौष्ठवात् ।\\
तस्मिन्नकृतबुद्धीनां नैवावस्थितनिश्चयः ॥ इति ॥ (व्या. य २-४८६)
\end{verse}
प्रमेयबाहुल्येन दुरवगाहत्त्वं भाजमाना, शास्त्रविषयगाम्भीर्येण इयत्तापरिच्छेदकत्वाभाववतीयं कृतिः स्वभाषासौष्ठवादेव स्पष्टप्रायमिव प्रतिभाति । “सज्जनमानसमिव निसर्गसुकुमारयति अतिगम्भीरमिदं महाभाष्यम्” इति प्रोवाच वाक्यपदीयभाष्यकारः पुण्यराजः । अत एव अस्मद्गुरुचरणा महाभाष्याध्यापनावसरे वदन्ति स्म - “अयं महाभाष्यग्रन्थो गोमुखव्याघ्र इव” इति । बाह्यदृष्ट्या यदा समालोक्यते तदा गौरिव (गोर्नियन्त्रणमिव) सुलभसाध्यमध्ययनम् इति प्रतिभाति । वस्तुत अन्तर्दृष्ट्या विचार्यमाणे तत्त्वबोधस्तु जलधिरिव गाम्भीर्यमावहति । अतो व्याघ्रनियन्त्रणमिव कष्टसाध्यम् अस्याध्ययनमिति ।

\section*{महाभाष्यस्य वैशिष्ट्यम्} 

पातञ्जलमहाभाष्यस्याध्ययनेन तदानीन्तनसमाजस्य दर्शनं, विविधक्षेत्राणां सन्दर्शनं, जनानां बौद्धिकसामर्थ्यप्रदर्शनम् इत्यादीनां नैकेषां विषयाणां परिचयः सम्भवत्यस्माकम् । द्वित्राण्युदाहारणान्यवलोकयामः -

तद्धितार्थोत्तरपदसमाहारे च (२-१-५२) सूत्रभाष्ये “कश्चित् तन्तुवायमाह- अस्य सूत्रस्य शाटकं वयेति । स पश्यति - यदि शाटको न वातव्यः, वातव्यो न शाटक इति । स मन्ये - यस्मिन्नुते यस्मिन्नुते ‘शाटक इत्येतद्भवति”। एतदुदाहरणप्रदर्शनेन ज्ञायते - तन्तुवायोऽपि देशेऽस्मिन् एतादृश्ये विचक्षण (यदि शाटको न वातव्य इत्यादि विचारयति ) आसीत् इति ।

समर्थः पदविधिः (१-२-२) सूत्रभाष्यकृता लोकस्वभावनिदर्शकम् उदाहरणमेकं प्रदत्तम् । तद्यथा - “एवं हि दृश्यते लोके भिक्षुकोऽयं द्वितीयां भिक्षां समासाद्य पूर्वं न जहाति, सञ्चयायैव प्रवर्तत इति । भिक्षुको हि धनसङ्ग्रहे लुब्धः, कथं वा पूर्वां भिक्षां जह्यात् ? अनेनोदाहरणेन समाजस्य स्वाभाविकं रूपं प्रदर्शितम् ।

न मु ने (८-२-३) सूत्रभाष्ये तदानीन्तनानां व्यवहारकौशल्यं बुद्धिनैपुण्यञ्च कथमासीदिति सम्यग् निरूपितम् । “अथवा द्विगता अपि हेतवो भवन्ति”। तद्यथा - आम्राश्च सिक्ताः पितरश्च प्रीणिता भवन्ति”। ... अथवा वृद्धकुमारीवाक्यवदिदं - द्रष्टव्यम् । ‘वृद्धकुमारी इन्द्रेणोक्ता वरं वृणीष्वेति । सा वरमवृणीत् - पौत्रा मे बहुक्षीरघृतमोदनं कांस्यपात्र्यां भुञ्जीरन्निति । न च तावदस्याः पतिर्भवति कुतः पुत्राः ? कुतः पौत्राः ? कुतो गावः ? कुतो धान्यम् ?  तत्रानया वाक्येनैकेन पतिः, पुत्रा, गावो धान्यम् इति सर्वं सङ्ग्रहीतं भवति । आभ्यामुदाहरणाभ्यामनुमातुं शक्यते - भाष्यप्रणयनकाले व्युत्पन्नाः, सुचतुरा धार्मिकाश्च जनाः भारतवर्षेऽस्मिन् निवसन्ति स्मेति । भाष्यकारसमये वैद्यकप्रबन्धः कीदृश आसीत् इति विचार्यमाणे मुक्तकण्ठेन एवं वक्तुं पारयामः - इदानीं परिदृश्यमानो वैद्यकव्यवहारः तदानीन्तनकाले न दृष्टिगोचरो भवति । ‘नड्वलोदकं पादरोगाः, पादरोगनिमित्तमिति गम्यते । दधित्रपुसं प्रत्यक्षो ज्वरः’ इत्येवंविधा व्यवहारा भवन्ति दृष्टिगोचराः । रोगोत्पत्त्यनन्तरं तदुपशमोपायश्च इदानीन्तनानाम्, रोगोत्पत्तिरेव प्रतिपाद्यन्ते स्म प्राचीनैः ।

अपि च महाभाष्यस्य तलस्पर्शाध्ययने तदानीन्तनकालीनाः समग्रा विषया ज्ञातुं शक्याः । अनेनाध्येतॄणां व्यावहारिकज्ञानमप्यभिवर्धत इत्यत्र नास्ति संशयलेशः । भाष्यकारस्य समये समाजेऽस्मिन् कृषिकार्यं कथम् आसीत् ? धान्यसञ्चयः कथं भवति स्म? कियद्धान्यमूल्यम् ? शालिप्रकाराः कीदृशा आसन् ? कृषीवलानां कीदृशी स्थितिरासीत् ? ते ऋणग्रस्ता उत न? वस्त्रनिर्माणप्रकाराः के ? ग्रामनगरादिविभागः कथमासीत् ? मार्गप्रबन्धः कथं भवति स्म ? रोगप्रतिबन्धार्थं दृष्टोपायाः के ? अदृष्टोपायाः के ? जातिव्यवस्था- जातिभेदाः - सामाजिकव्यवहारः - छात्रवृत्तिः -समाजसेवा-स्त्रीसम्मान इत्यादयः सर्वे विषया अवगता भवन्ति । अत एव “महाभाष्यं वा पठनीयं, महाराज्यं वा पालनीयम्” इत्यभियुक्तोक्तिरद्यापि सर्वत्र प्रचलिता वर्तते ।

\section*{महाभाष्यस्य महत्त्वम्}

पाणिनीयव्याकरणस्य एवं पाणिनीतरशास्त्राणाम्, अपि च इतरशास्त्रग्रन्थानां लौकिकव्यवहाराणाञ्च सारभूतमंशं पाणिनीयव्याकरणव्याख्याव्याजेनैकत्र सञ्जग्राह पतञ्जलिमहर्षिः । ‘न केवलं शब्दशास्त्रस्य, अपि तु सकलविद्यानामयमाकर’ इति प्रशशंस भगवान् भर्तृहरिः ।
\begin{verse}
कृतोऽथ पतञ्जलिना गुरुणा तीर्थदर्शिना ।\\
सर्वेषां न्यायबीजानां महाभाष्यनिबन्धने ॥ इति ॥ (वा. य २-४८५)
\end{verse}
ग्रन्थस्यास्य महत्त्वमभिसन्धार्यैव दयानन्दसरस्वतीनां गुरुचरणैः विरजानन्दस्वामिभिः समुद्घोषणमिदं कृतमासीत् 
\begin{verse}
अष्टाध्यायी महाभाष्ये द्वे व्याकरणपुस्तके ।\\
ततोऽन्यत्पुस्तकं यत्तु तत्सर्वं धूर्तचेष्टितम् ॥ इति ॥
\end{verse}

किञ्चास्य प्रामुख्यमभिलक्ष्य अर्वाचीनैर्विद्वद्भिरपि समनुकीर्तितेऽयं कृतिः । विद्वद्वरो माधवकृष्णशर्मा एवमभिप्रैति  ‘अष्टाध्यायीमधिकृत्य नैके व्याख्यानग्रन्था विरचिताः सन्ति । किन्तु पतञ्जलेर्व्याख्यासौन्दर्यं, सारल्यं, परिश्रमं प्रामाण्यञ्चातिरिच्य गन्तुं नेतरो टीकाग्रन्थाः नितराम् असमर्था एवाभूवन्’ इति ।

\selectlanguage{english}
\begin{english}
Among the large number of commentaries on the \textit{Astadhyayi} none exels \textit{Patanjali’s} effort in the scereness of the critical touch and in the unimpeachability of the authority. \textit{(Panini- KatyAyana- Patanjali)}
\end{english}

\selectlanguage{sanskrit}
एतादृशवैशिष्ट्येन समायुक्तस्य महत्त्वमधिगतस्य ग्रन्थस्य अध्ययनाध्यापनप्रणालिरिदानीम् उच्छिन्नेव भवति । अस्याः प्रणाल्याः पुनरुज्जीवनार्थं संस्कृताभिमानिभिः सर्वैः प्रयतनीयम् । एतदर्थं पाणिन्यनुयाभिः शाब्दिकैः सश्रद्धं महाभाष्यम् अध्येयम् । भाष्याब्धेः रत्नोपमं ज्ञानमधिगन्तुं प्रामाणिकः प्रयासो विधेयः । अस्मिन् ज्ञानयज्ञे प्राप्यमाणम् अल्पमपि साफल्यं महर्षये पतञ्जलये अस्माभिः समर्पयिष्यमाणा कृतज्ञताकुसुमोपमं भविता ।
 
\begin{thebibliography}{99}
\bibitem{ch11key1} पतञ्जलिविरचितं व्याकरणमहाभाष्यम्
\bibitem{ch11key2} पाणिनिविरचिता अष्टाध्यायी
\bibitem{ch11key3} भर्तृहरिप्रणीतं वाक्यपदीयम्
\bibitem{ch11key4} भट्टोजिदीक्षितकृता वैयाकरणसिद्धान्तकौमुदी
\bibitem{ch11key5} डा. रामचन्द्रशुक्लस्य प्रबन्धरत्नाकरः
\bibitem{ch11key6} गोपालदत्तपाण्डेयलिखितं अष्टाध्यायीप्रास्ताविकम्
\bibitem{ch11key7} भार्गवशास्त्रीकृतो महाभाष्यप्रस्तावः
\bibitem{ch11key8} माधवकृष्णशर्मविरचिता \eng{‘Panini-kAtyAyana-Patanjali’} कृतिः ।
\end{thebibliography}
