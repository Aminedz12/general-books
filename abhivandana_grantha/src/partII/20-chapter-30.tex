\chapter{सन्निकर्षः}

\begin{center}
\Authorline{प्रवीण मण्णिकोप्पा}
\smallskip

योगाध्यापकः मणिपाल्,\\
पूर्वविद्यार्थी (२०१२-२०१६)
\addrule
\end{center} 
न्यायशास्त्रे प्रत्यक्षानुमानोपमानशब्दभेदेन चत्वारिप्रमाणानि अङ्गीकृतानि । तेषु मध्ये प्रत्यक्षप्रमाणं मूर्धन्यतमम् । यतः प्रत्यक्षप्रमाणम् उपजीव्यैव इतराणि प्रमाणानि प्रमां जनयन्ति । इदं च प्रत्यक्षं कीदृशम् इति जिज्ञासायां इन्द्रियार्थसन्निकर्ष एव प्रत्यक्षप्रमाणव्यवहार्यः भवति । अत्र इदमवधेयम्- आत्मा मनसा संयुज्यते । मनः इन्द्रियेण, इन्द्रियम् अर्थेन, ततः प्रत्यक्षम् इति प्रत्यक्षात्मकप्रमायाः उदयस्य क्रमः । एकः घटः अस्माभिः दृश्यते चेत् ‘अयं घटः’ इति प्रत्यक्षात्मिका प्रमा जायते । एतादृशप्रमापूर्वभाविनी इयं प्रक्रिया । प्रथमतः दृष्टु आत्मनः तदीय मनसा सह संयोगो भवति । ततः तादृश मनसः तदीय चक्षुरिन्द्रियस्य च संयोगो जायते । ततः परं तादृश चक्षुरिन्द्रियस्य घटेन सह सम्पर्को भवति । एतादृश सम्पर्क एव सन्निकर्ष इत्युच्यते । अतः इन्द्रियार्थसन्निकर्ष एव प्रत्यक्षप्रमां प्रति कारणम् । एतादृश इन्द्रियार्थसन्निकर्ष एव प्रत्यक्षप्रमाणपदवाच्यः । 

सन्निकर्षः द्विविधः । लौकिकः अलौकिकश्चेति । लौकिकः सन्निकर्षः षड्विधः । यथा-संयोगः, संयुक्तसमवायः, संयुक्तसमवेतसमवायः, समवायः, समवेतसमवायः, विशेषणविशेष्यभावश्चेति । एतेषां षड्विधसन्निकर्षाणां क्रमशः उदाहरणानि एवं ज्ञेयानि । 

चक्षुषा घटप्रत्यक्षजनने संयोगः सन्निकर्षः । चक्षुरिन्द्रियं हि तैजसम् । तस्मात् तेजोरूपात् चक्षुरिन्द्रियात् आलोकादिसहकारिकारणेषु विद्यमानेषु निःसृताः चाक्षुषरश्मयः चक्षुरिन्द्रियसकाशात् घटादिवस्तुना सह संयुज्यन्ते इति । इत्यतः चक्षुरिन्द्रियवस्तुनोः संयोगः सन्निकर्षः भवति । तेन अयं घटः इति ज्ञानं जायते । एवमेव इतरपञ्चविधसन्निकर्षाणां अपि उदाहरणानि दृश्यमानानि ज्ञेयानि ।            

सन्निकर्षे द्वितीयः प्रकारः अलौकिकः सन्निकर्षः । अयं च त्रिविधः- समान्यलक्षणासन्निकर्षः, ज्ञानलक्षणासन्निकर्षः, योगलक्षणासन्निकर्षश्चेति । यथा अनुमानस्थले पर्वतो वह्निमान् इत्यनुमाने वह्नित्वेन निखिलवह्निनां धूमत्वेन निखिलधूमानां च ज्ञानं सामान्यलक्षणासन्निकर्षेण भवति । एवं सुरभिकुसुमम् इत्यत्र चक्षुरिन्द्रियेणैव सौरभांशः ज्ञानलक्षणासन्निकर्षेण ज्ञायते । एवं योगिना योगजसन्निकर्षेण भूतभविष्यत्वर्तमानकालवस्तूनामपि ज्ञानं भवति । एतेषु सन्निकर्षेषु विविच्य विवरणदानस्य अतिक्लिष्टत्वात् एते सन्निकर्षाः अलौकिकसन्निकर्षाः इत्युच्यन्ते । एवं लौकिक अलौकिकभेदेन सन्निकर्षः प्राधान्येन द्विविधः । एतादृश इन्द्रियार्थसन्निकर्षजन्यं ज्ञानमेव प्रत्यक्षमित्युच्यते । अतः  इन्द्रियार्थसन्निकर्ष एव प्रत्यक्षप्रमाणम् इति तार्किकैः कथ्यते ।

\articleend
