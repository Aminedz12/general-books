{\fontsize{15}{17}\selectfont
\presetvalues
\chapter{आर्यमर्यादा}

\begin{center}
\Authorline{वि~॥ उमाकान्त-भट्टः}
\smallskip

प्रांशुपालः\\
सर्वकारीयसंस्कृतपाठशाला, मेलुकोटे
\addrule
\end{center}

भारतवर्षे निवसतां जनानां जीवनपद्धतिरनितरसाधारणी विद्योतते प्राचीनकालादद्य यावदनुवर्तमाना इति सर्वेषामनुभवः~। भारते विशाले केवलं वसन्तो न भारताः ; किन्तु भारती सन्तत्या समुद्भूता भारता एव भारतीयाः~। ते तु स्वरूपतो भायाम् रताः, येषु स्वयं भा निरन्तरं निरता~। अत एव ते चिरन्तनार्षप्रज्ञयावाप्ताभ्यर्हितसंस्कारा इह परत्र च समानं जीवनमनुभवन्ति~। कदापि ते पारत्रिकीं यात्रां न विस्मरन्ति ; न वा केवलामैहिकीं विभावयन्ति~। परत्राविरोधेन इह सुखमीप्सन्ते ते~। एवं जीवनयात्रा निर्वहणे तेषामिहत्यः कालो देशः सन्निवेशश्चानुकूल्यमावहन्ति~। भारतभूमेः कणे कणे स्फूर्तिराविर्भवति तादृशी~। सर्वमिदमभिव्यनक्तीव वैष्णवी सूक्तिः 
\begin{verse}
उत्तरं यत्समुद्रस्य हिमाद्रेश्चैव दक्षिणम्~। \\
वर्षं तद्भारतं नाम भारती यत्र सन्ततिः~॥ इति~। 
\end{verse}
भारतस्योत्तरस्यां दिशि हिमालयो नाम नगाधिराजः पूर्वापरौ तोयनिधीवगाह्य पृथिव्या मानदण्ड इव स्थितो देवतात्मा औत्तरशीतझञ्झावाततत्सदृशशत्रुसमुदायादिभ्य इदं भारतं निरन्तरं संरक्षति सेनापतिरिव सज्जः~। पूर्वदक्षिणपश्चिमासु भारतभुवं परिगतो वारिधिः परिखेव जागर्ति~। मध्ये स्थितं  भूमण्डलं नदीनदपुलिननगगहनकासारकेदारादिनिसर्गतो वासयोग्यं रमणीयमुद्यानमिव देवानामपि नृत्यगीतलीलाविलासौत्सुख्यावहं दिव्यमेव विलसति~। कर्म भूमिरेषा सर्वेषाम् स्वर्गापवर्गास्पदम्~। देवमनुष्यगन्धर्वाणामयं देशो यथाधिकारं वैदिकं कर्मभूयो भूरि च कृत्वा नाकमेतुं सेतुभूतो जीवोत्कर्षविधानेन परमपुरुषार्थलाभाय प्रकल्पत इति प्राञ्चः स्म मुहुर्मुहुराहुः~। 

भारतीयानां जीवनपद्धतौ जागर्ति सर्वतः प्रशस्तार्यमर्यादा~। कर्मणामैहिकामुष्मिकञ्च फलं प्रतियन्ति भारताः~। ते नित्यस्यात्मनो देहान्तरप्राप्तिरूपं प्रेत्यभावमभ्युपयन्ति~। ते साधने\-श्वरानुग्रहादिवशात् तत्त्वज्ञानादपवर्गमपि रोचयन्ते~। तदर्थं तदावेदकवेदवाक्यानां निरर्गलं \-प्रामाण्यं निर्णयन्ति~। लोकयात्रायामपि वर्णाश्रमसामान्यविशेषादि धर्माणामनुष्ठानाद्य\-कर्तव्यत्वेन श्रद्दधते ते~। विश्वस्यान्यस्मिन् प्रदेशे न गोचरीभवत् करणत्रयसारूप्यं भारतानामेव विशेषः ~। ‘यद्धि मनसा ध्यायति  तद्वाचा वदति, तत्कर्मणा करोति’ इति त्रिभिरपि करणैः प्रवृत्ता सरूपा समपर्यवसाना यात्रा लोकस्य लोकानाञ्च क्षेमाय क्षमते~। यस्य परं करणान्येव परस्परं व्यभिचरन्ति स दुरात्मा~। एतत्प्रतिकोटिं प्रविष्टास्तु महात्मानः~। त एवार्या भारतीयाः येषाम् जीवनविधानं वेदप्रतिष्ठं पुरुषार्थपर्यवसन्नञ्च प्रशस्यते~। तदेवार्यत्वहेतुकं भारतानां सर्वात्मना महत्त्वं भगवान्मनुरब्रवीत् \-
\begin{verse}
एतद्देशप्रसूतस्य सकाशादग्रजन्मनः~। \\
स्वं स्वं चरित्रं शिक्षेरन् पृथिव्यां सर्वमानवाः~॥ इति~। 
\end{verse}
इह देशे सञ्जाताः जनाः विश्वस्येतरेषामग्रजन्मानो विद्यया, वाचा संस्कारेण जीवनविधानेन च~। एते विचारमभ्यर्हितं यथा मन्वते तथाचारमपि तत्समानस्तरेण~। अन्यत्र प्रायो ये विचारं प्रशंसन्ति विगर्हन्ते ते त्वाचारक्रमम्~। ये पुनराचारमाद्रियन्ते, ते विचारं तावत् नार्हयन्ति~। 
एतत्सर्वं मनसाकालय्य तत्र भवानाचार्यश्चणकात्मज एवं सूत्रयति स्म-
\begin{verse}
व्यवस्थितार्यमर्यादकृतवर्णाश्रमस्थितिः~। \\
त्रय्या हि रक्षितो लोकः प्रसीदति न सीदति~॥ इति~। 
\end{verse}
व्यवस्थिता आर्यमर्यादा यस्मिन् सह व्यवस्थितार्यमर्यादः आर्यजुष्टजीवन-सीमान\-मनुल्लङ्घ्यः वर्तमानः, कृता वर्णाश्रमधर्ममनुतिष्ठन् त्रय्या ऋग्यजुस्सामरूपवेदत्रयेण रञ्जितो लोकः स्वरूपच्युतिमानवाप्तो जनः प्रसीदति इह सुखं परत्र चानन्दमनुभवति, न सीदति न दुर्गतिं प्राप्य शोचति व्यवस्थितामार्यमर्यादामुपजीवतीति सारम्~। 

अधुना आर्या नाम के ? का च तेषाम्मर्यादा ? - इति किञ्चिदालोच्यते~। 

आर्या नाम काचिज्जनता या एष्याखण्डस्य मध्यदेशात् भारतमागता~। द्रविडा नाम अन्या काचिज्जनता पूर्वत इहत्यैव~। तयोः वर्षाणां पञ्चसाहस्र्याः पूर्वं महान् सङ्घर्ष आसीद्यस्मिन्निमे आर्या विजयशालिन इत्यादिकल्पना वैदेशिकानामितिहासविदां कुत्सितैव ; पूर्वाग्रहप्रयुक्तत्वात् , असङ्गत्वार्थत्वात् हेत्वाभासभूयस्त्वाच्च~। आर्या द्रविडा इत्युभयेपि इहत्या एव~। तेषां सङ्घर्ष आस्ताम् , तावता तयोरन्यतरे बाहीकाः~। आर्यता नाम न जातिर्न वा द्राविडी~। आर्यत्वं द्रविडत्वं च मनुष्याणाम् मनोधर्मौ तत्तज्जीवनविधानहेतू~। आर्याः शिष्टाः प्रामाणीका इति पर्यायः~। द्रवन्ति नियमसीमामुल्लङ्घ्य भोगेषु त्वरन्त इति द्रविडाः~। आर्याः कदापि न तथाभूताः~। इमे उभये यथापूर्वमद्यापि विद्यन्ते विद्योतन्ते च~। 

आर्यो नाम पूजार्हः~। भाषया भूषया मनीषया व्यवहारेण च योऽन्येषां सपर्यामर्हति स आर्य इति सम्मन्यते~। ऋ - गतौ धातोः विहिते ण्यत् प्रत्यये आर्यशब्दो निष्पद्यते~। “सर्वे गत्यर्थका ज्ञानार्थका” इति नियमेन ऋच्छति जानाति इदमस्य कारणमिति व्युत्पत्या कारणतत्त्वज्ञ इति तदर्थो विज्ञेयः~। कारणं ध्येयः इत्यौपनिषदवचनानुसारेण सृष्टिमूलपर्यन्तं विज्ञाय तदविरोधेन तत्त्वानामन्योन्यानुपमर्देन च जीवनं निर्वहन् पुनर्योगादिसाधनप्रणाल्या प्रकृतिमूलमुपेत्य ततो विमोचनञ्चानुवानः पुरुषः आर्यः~। यो हि ऋच्छति सः प्राप्नोति ; यो न ऋच्छति सः रिष्यते~। फलसाधनताप्रकारकप्रमावान् प्रमाणमात्रमुपजीव्य जीवन् प्रामाणिक इति विख्यात एव पूजामर्हति~। ईरयन्ति स्पुटमाचक्षते विप्रकृष्टमपि पुरुषार्थं ये त आर्या इति अस्मदाचार्याः पाठेष्वसकृत् बोधयन्ति स्म~। स चार्यः शिष्ट इति नवीनैः परिभाष्यते~। तत्स्वरूपं निरूपितमभियुक्तैः-
\begin{verse}
कर्तव्यमाचरन् कार्यमकर्तव्यमनाचरन्~। \\
तिष्ठति प्रकृताचारे स वा आर्य इति स्मृतः~॥  अन्यत्र महाभारते च\enginline{-}\\
आर्यता नाम भूतानां यः करोति प्रयत्नतः~। \\
शुभं कर्म निराकाशे वीतरागस्तथैव च~॥ इति~। 
\end{verse}
महाभारतदृष्ट्या तु सर्वभूतहितप्रवणत्वनिबन्धनमार्यत्वमिति विशेषः~। 

मर्यादा नाम सीमा~। स्वीयमर्यादायां सर्वे सुरक्षा भवन्ति~। सीमातिरेके रक्षा दुर्लभा समेषां चेतसाम्~। आम्भसिका मत्स्यादयो भूमौ न जीवन्ति~। भौमा मृगादयो नाप्सु जीवन्ति~। किमुत कतिपय क्षणेषु ते श्वसितुमपि न पारयन्ति~। एवमेव कालतो वस्तुतः सन्निवेशतश्च नाना मर्यादाः निसर्गे निगूढाः सन्ति~। एवमेव विज्ञाने विचारे संस्कारे संलापे प्रवृत्यादावपि सीमाः सन्ति~। विशिष्यावधेयाः~। तासु मर्यादासु काश्चित्स्वाभाविक्यः, काश्चिदन्याः वृद्धसंविदा सुनिबद्धाः~। एतादृशसीमान्तरेव प्रजास्तिष्ठेयुः इति व्यवस्थार्थमेव राजानः सृष्टाः , प्रभवश्चोद्भाविताः~। अत एव कविकुलगुरुणा कालिदासेन रघुवंशमहाकाव्ये दिलीपगुणवर्णनावसरे राजधर्मस्य मर्मैवं प्राकाश्यत -
\begin{verse}
प्रजानां विनयाधानाद् रक्षणाद्भरणादपि~। \\ 
स पिता पितरस्तासां केवलं जन्महेतवः~॥ इति~। 
\end{verse}
एतत्पर्यालोचनेन प्रायो जीवा जातमात्रेण द्रविडाः सन्तः शनैः शनैः संस्कारवशात् विनयमापिता आर्या भवन्ति समर्यादा इति तत्त्वमवधार्यते~। 

स्वाभाविकी सीमा मर्यादाशब्देन व्यपदिश्यते ; सा दैवतन्त्रा~। समयमूला तु सीमा व्यवस्थाशब्देन निर्दिश्यते सा पुरुषतन्त्रा~।  उभे सीमानौ जनानामन्दायावश्यक्यौ~। जनानां सीमान्तवृत्तिरेव स्वधर्म इति ख्यायते गीतायाम् \enginline{-} स्वधर्मे निधनं श्रेयः परधर्मो भयावहः~। स्वनुष्ठितादपि परधर्मादल्पानुष्ठितः स्वधर्मः श्रेयानिति गीतातात्पर्यम्~। 

स्वाभाविक्योपि सीमा इमाः सुसूक्ष्माः~। तासां प्रतीतिः किमालम्बा ?  इति जिज्ञासा सहजतयोत्तिष्ठति~। तस्याः समाधानमिदम्~। भारतीयमार्षं वायं समग्रमपि जीवनमर्यादायाः प्रबोधार्थमेव प्रवृत्तम्~। साक्षात्कृतधर्माणो ऋषयः~। ते निसर्गधर्मं साक्षादवलोक्य तदाभिमुख्यसम्पादनाय निसर्गसन्दर्शितमार्गानेव परमार्थतयोपादिशन्~। तस्मात् सर्वे वेदाः , स्मृतयः , इतिहासपुराणानि , काव्यनाटकादीनि आर्यजुष्टजीवनमर्यादामेव वर्णयितुमवतीर्णानि~। परन्तु तदन्तरवलोकने तात्पर्यग्रहणे चानुकूला तादृशी दृष्टिः गुरोरनुग्रहादवाप्तव्या~। इत्थं तत्र भवता कौटल्येन स्फुटं निर्दिष्टार्यमर्यादा न केवलं भारतस्य किन्तु विश्वस्य समेषां चेतनानां प्रसादरूपानुग्रहाय कल्पत इति विभावनीयम्~। 

\articleend
}
