{\fontsize{15}{17}\selectfont
\presetvalues
\chapter{शब्देन्दुशेखरः पुत्रो मञ्जूषा चैव कन्यका}

\begin{center}
\Authorline{डा~॥ वि.डि.हेगडे}
\smallskip

राष्ट्रप्रशस्तिद्वयपुरस्कृतः\\
निवृत्तः प्राध्यापकः - हिन्दिभाषाविभागः\\
मैसूरुविश्वविद्यालयः, मैसूरु
\addrule
\end{center}

नागेश आसीदपुत्रोपि पुत्री~। श्लोकद्वयं तद्विरचितस्य लघुशब्देन्दुशेखरस्य च समाप्तौ समुपलभ्यते -

\begin{verse}
शब्देन्दुशेखरः पुत्रो मञ्जूषा चैव कन्यका ~। \\
स्वमतौ सम्यगुत्पाद्य शिवयोरर्पितौ मया~॥\\
शब्देन्दुशेखरः सोऽयं फणिभाष्योक्तिभूषितः~। \\
सतां हृत्कमलेष्वास्तां यावच्चन्द्रदिवाकरौ~॥
\end{verse}

अत्र त्रयोंऽशाः (१.नागेशस्य न कोपि पुत्रो बभूव ; अत एव (२) काश्यां भगवत्याः श्री सिद्देश्वरीदेव्या निकटे स्थितं स्वगृहं नागेशो देशस्थब्राह्मणानां कुले परिणीतायै स्वपुत्र्यै प्रादात् ; (३) नागेशः शिष्टपरम्परानुसारेण शेखरे पुत्रत्वं मञ्जूषायां पुत्रीत्वं चारोप्य तौ ग्रन्थौ शिवयोरर्पयामास~। )  

अनुसन्धानदृशां जिज्ञासाम् उज्जीवयन्ति~। तदास्ताम् , नागेशपुत्रः शब्देन्दुशेखरो द्विविधो दृश्यते~। एको लघुशब्देन्दुशेखरः , अन्यश्च बृहच्छब्देन्दुशेखरः~। बृहच्छब्देन्दुशेखरस्य बृहत्वम् अस्मान् विस्मये निमज्जति~। 

शब्देन्दुशेखरस्याविर्भावः कथमभूत् ? अत्रास्ति सुधीर्घा कारणपरम्परा~। भगवान् पाणिनिः “वृद्धिरादैच्” इत्याद्यम् “अ अ” इत्यन्तम् “अष्टाध्यायी”नामकग्रन्थं जग्रन्थ~। ततः कात्यायनः वार्तिकाख्यं पाणिन्युक्तार्थपरिपूरकं दुरुक्तपुनरुक्ताद्यर्थनिरासकं च ग्रन्थमकरोत्~। ततो गच्छता कालेन भगवान् पतञ्जलिः महाभाष्यं चकार~। तदनन्तरं विद्वत्तलजेन भर्तृहरिणा वाक्यपदीयं प्राणायि~। जैयटात्मजकैयटेन हरिकारिकारूपां नावमवलम्ब्य भाष्याब्धेः पारं गन्तुं प्रदीपाख्या कुव्याख्या विरचिता तदनन्तरं वामनजयादित्याभ्यां सूत्रार्थप्रकाशिका काशिका प्रकाशिता~। जिनेन्द्रबुद्धिना काशिकाविवरणपञ्जिका प्राणायि~। “न्यास” इति तस्याः नामान्तरं~। हरदत्तस्य पदमञ्जरी काशिकार्थप्रकाशने महतीं भूमिकां वहति~। 

\begin{verse}
अनधीते महाभाष्ये व्यर्था सा पदमञ्जरी~। \\
अधीते तु महाभाष्ये व्यर्था सा पदमञ्जरी~॥\\
उक्तिरेषा रूढा शोभते पदमञ्जर्याः~। 
\end{verse}

एकतः काशिकादर्शितः पन्थाः~। तेन पथा गच्छतः न्यासपदमजञ्जर्यौ~। अन्यतः कौमुदीदर्शितः पन्थाः~। एतेन पथा गच्चति भूयान् ग्रन्थराशिः~। तत्राद्या प्रक्रियाकौमुदी या सञ्जग्रन्थे मनीषिणा रामचन्द्रेण~। एतत् ग्रन्थदर्शितपथा गतेन सुधिया भट्टोजी दीक्षितेन भाष्यार्णवं प्रमथ्य वैयाकरणसिद्धान्तकौमुदी व्यरच्यत~। प्राच्यां खण्डनाय आक्षेपदूरीकरणाय च प्रौढमनोरमा तेन स्वयं विरचिता~। तदुक्तदोषान् प्रदर्शयितुं कुचमर्दिनीनाम्नीं टीकां विरचितवान् पण्डितराजो जगन्नाथः~। जगन्नाथमतं उक्तानुक्तदुरुक्तचिन्तनेन खण्डयन् हरिदीक्षितः शब्दरत्नेन प्रौढमनोरमामलमकोरोत्~। ततो हरिदीक्षितशिष्येण नागेशेन शब्दरत्नोक्तन्यूनता परिहाराय कौमुद्या व्याख्यानभूतो लघुशब्देन्दुशेखरः प्राणायि - यथा 

\begin{verse}
पातञ्जले महाभाष्ये कृतभूरिपरिश्रमः~। \\
शिवभट्ट्सुतो धीमान् सतीदेव्यास्तु वंशजः~॥\\
याचकानां कल्पतरोररिकक्षहुताशनात्~। \\
शृङ्गवेरपुराधीशाद् रामतो लब्धजीविकः~॥\\
नत्वाफणीशं नागेशस्तनुतेऽर्थप्रकाशकम्~। \\
मनोरमार्धदेहं लघुशब्देन्दुशेखरम्~॥
\end{verse}

मनोरमालघुशब्देन्दुशेखरौ उभावपि उभावन्तरेण परिपूर्णतां न यातः - इत्यवगन्तव्यम्~। योऽधीती प्रौढमनोरमायां स एवाभिजानाति शब्देन्दुशेखरस्य महत्त्वम्~। 

बृहन्नसौ शब्देन्दुशेखरो “बृहच्छेन्दुशेखर” इति सार्थकनामा~। सः महाग्रन्थः १८८२ तमे शकाब्दे (१९६०तमे क्रैस्तवर्षे) डा.सीतारामशास्त्रिभिः सम्पादितो वाराणसेयसंस्कृत\-विश्वविद्यालयेन प्रकाशितश्च~। महाग्रन्थोऽयं त्रिषु भागेषु आहत्य २३२८ पृष्ठेषु विततः~। तत्र नामकरणविषये विसंवादवतां विदुषां विचारः श्रोतव्यः~। यथा बृहच्छब्देन्दुशेखर इति नाम ग्रन्थकृता न कृतम्~। श्ब्देन्दुशेखर इत्येव नामास्तु~। नागेशेन द्वौ शेखरौ निबद्धावित्यत्र प्रमाणं समुपलभ्यते~। लघुशब्देन्दुशेखरे पूर्वार्धान्ते, लकारार्थप्रक्रियान्ते, कृदन्तान्ते च विस्तरस्तु “बृहच्छब्देन्दुशेखरे” द्रष्टव्यः, विस्तरस्तु बृहच्छब्देन्दुशेखरे बोद्ध्यः” इति तल्लिखितं लभ्यते~। अत एव ग्रन्थकृतैवास्य ग्रन्थरत्नस्य बृहच्छब्देन्दुशेखर इत्याख्याप्यकारि - इति निश्चीयते~। प्रायो १८५० विक्रमाब्दे उदयङ्करपाठकेन निर्मितायां लघुशब्देन्दुशेखरटीकायं ज्यौत्स्नी इत्याख्यायां “बृहच्छेखर”, “बृहच्छब्देन्दुशेखर” इति नाम्नोः निर्देशोप्यत्र प्रमाणं भवितुमर्हति~। 

मनोरमार्धदेहं तत्त्वे शब्देन्दुशेखरम् इति नागेशेन शाब्देन्दुशेखरस्य प्रारम्भ एव निगदितम्~। मनोरमैव उमा + मनोरमोमा, सा अर्धदेहो यस्य सः = मनोरमार्धदेहः, तम् इति विग्रहः प्रदर्श्यते~। नागेशः स्थले स्थले प्रौढमनोरमायाः खण्डनं विधदत् मनोरमायाः अर्धं द्यति = (खण्डय़ति) इति मनोरमार्ध्रदा, तादृशी ईहा (= चेष्टा) यस्य सः मनोरमार्धदेहः (तथाभूतोऽयं मदीयो निबन्धः) = इति स्वाशयमप्यभिव्यञ्जयतीति विग्रहान्तरप्रदर्शनेन ज्ञातुं शक्यते~। 

बृहच्छब्देन्दुशेखरस्य सम्पादकैः डा. सीतारामशास्त्रिभिः नागेशं तत्कृतिश्च \break परिचाययितुं यल्लिखितं तदत्र कुतूहलं वर्धयितुं समुध्रियते - “अस्मिन्नेव समये (प्रायः १७५० वैक्रमेऽब्दे) स्वगुरु-हरिदीक्षितपादसेवन-लब्धवैदुष्यः स्वमतमण्डनपरमतखण्डन- सामर्थ्यवतीं नवनवोन्मेषशालिनीम् अनिर्वचनीयां कामपि स्पृहणीयां प्रतिभामदधानो विद्वत्तल्लजोऽयं श्रीभट्टनागेशो बहोः कालाज्जरठैः महावैयाकरणैः प्रचालयमाने आत्मनो वैशिष्ट्यमनन्य\-\break साधारणं शास्त्रार्थपाटवं ग्रन्थलेखनकौशलं च ख्यापयितुमिव, सिद्धान्तकौमुदीटीकामिषेण \-विरचितेन बृहच्छब्देन्दुशेखर इति नाम्ना प्रसिद्धेन महता ग्रन्थेन सह शास्त्रार्थरणाङ्गणवन्यगजभुवं केसरीकिशोर इवावततार -------------- ”~। 

नागेशो मञ्जूषां कन्यकां मन्यते स्म~। मञ्जूषा चैव कन्यका इति तेन सुष्ठूक्तं वैयाकरणसिद्धान्तमञ्जूषां परिचाययन्तः डा. सीतारामशास्त्रिणः एवं ब्रुवते- 

ग्रन्थेऽस्मिन् खण्डनखण्डखाद्यवत् स्वसिद्धान्तप्रतिपक्षभाजि सर्वेषां दार्शनिकानां मततमांसि नव्यभव्यतर्कार्करश्मिभिः साधुशकलीकृतानीति न तिरोहितमेतत्प्रेक्षावताम्~। व्याकरण\-शास्त्रं न केवलं प्रकृतिप्रत्ययमात्रबोधकम् अपि त्वेकं विशिष्टं दर्शनमपीति सिद्धान्तारविन्द\-विकासे अयमेव नागेशभट्टः सहस्रांशुरिव प्रचण्डप्रखरतरशेमुषीकोऽस्मिन् ग्रन्थे सर्वातिशायिनीं सफलतां पर्युपावृणोत्~। 

नागेशविरचिता मञ्जूषा त्रिविधा - गुरुमञ्जूषा, लघुमञ्जूषा, मरमलघुमञ्जूषा इति प्रसिद्धेति कथ्यते~। लघुश्ब्देन्दुशेखरस्य भूमिकायां नरहरिशर्मणा बृहन्मञ्जूषा इति नामोल्लेखः कृतः~। \hbox{लघुमञ्जूषायाम्} अर्थं परिष्कर्तुः नागेशस्य शैली समाकर्षति यथा  “द्रोणप्रस्थादिशब्दाश्चेयत्ता\-विशेषावच्छिन्नपलादिपरिच्छिन्नधान्यादिपरिच्छेदकत्वसमानाधिकरणजातिविशेषरूपद्रोण-\break त्वाद्यवच्छिन्ने शक्ताः~। अन्योन्येतरेतरत् परस्परशब्दानां क्रियाव्यतिहारविषयमेकमात्रावृत्यन्योन्यत्वम् अखण्डोपाधिरूपं प्रवृत्तिनिमित्तमिति सर्वेष्टसिद्धिः ”~। 

\section*{प्रातिपदिकार्थनिर्णयः}

नागेशपुत्रौ मञ्जूषाशेखरौ यावच्चन्द्रदिवाकरौ स्थास्यतः~। तावन्तरेण व्याकरणेऽधीती सम्पूर्णतां न याति~। तावदधीत्यैव सोऽधीती सम्पूर्णतां समाप्नोति~। 

\articleend
}
