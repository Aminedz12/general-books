{\fontsize{15}{17}\selectfont
\presetvalues
%\addtocontents{toc}{\protect\newpage}
\chapter{वेदान्तदर्शने शब्दप्रमाणं शाब्दापरोक्षवादश्च}

\begin{center}
\Authorline{डा ॥ शङ्कर नागेश-भट्टः }
\smallskip
 सहायकप्राध्यापकः\\
श्रीराजराजेश्वरी संस्कृतमहापाठशाला\\
श्री सोन्दा, स्वर्णवल्ली, शिरसि
\addrule
\end{center}

“दृशिर् प्रेक्षणे” इत्यस्माध्दातोः भावे करणे कर्मणि वा ल्युटि निष्पन्नस्य “दर्शनम्”\break इत्यस्य साक्षात्कारः, साक्षात्कारसाधनम्, साक्षात्कारविषय इति वाऽर्थः सम्पद्यते~। चार्वाक-बौध्द-जैन-न्याय-सांख्य-मीमांसादिषु तत्तत्तत्त्वप्रतिपादकशब्दराशेः “बौध्ददर्शनं”\break “सांख्यदर्शनम्” “न्यायदर्शनम्”  इत्यादिषु करणार्थे प्रयोगः प्रसिध्दः~। 

शब्दप्रमाणम्-आप्तवाक्यं शब्दप्रमाणम्~। आप्तस्तु सत्यद्रष्टा~। तेन उच्चारितं वाक्यं प्रमाणमित्यर्थः~। आगमप्रमाणमिति नार्थान्तरम्~। शब्दजं ज्ञानं (शाब्दं) कथमुत्पद्यते इति चेत् एवम्- शाब्दबोधे पदज्ञानं करणम् , (साक्षात्साधनमित्यर्थः) पदार्थस्मरणं तस्य करणम् (व्यापारः)~।  शाब्दबोधश्च फलम्~। वाक्यजन्यज्ञाने आकाङ्क्षा, योग्यता, आसत्तिः, तात्पर्यज्ञानं चेति चत्वारि सहकारिकारणानि~। 

\begin{enumerate}
\itemsep=0pt
\item तत्र पदार्थानां परस्परजिज्ञासाविषययोग्यत्वम् एकपदार्थज्ञानजन्यापरपदार्थज्ञानेच्छा- विषययोग्यत्वम् आकाङ्क्षा~। यथा क्रियाश्रवणे कारकस्य, कारकश्रवणे क्रियायाः,\break इतिकर्तव्यतायाश्च जिज्ञासा उदेति इति तेषां जिज्ञासाविषययोग्यत्वं आकाङ्क्षा वा अस्तीति ज्ञायते~। अत एव (आकाङ्क्षाभावादेव) गौः, अश्वः, पुरुषः, हस्ती इत्यादी\-नाम् अर्थानुभावकत्वं नास्ति~। 
\item तात्पर्यविषयीभूतसंसर्गाबाधो योग्यता~। वक्तुरिच्छाविषयीभूतस्य एकपदार्थे अपर\-पदार्थसम्बन्धस्य बाधाभाव इति तदर्थः~। “अग्निना सिञ्चति” इत्यादीनां योग्यताभावात् शाब्दबोधो न जायते~। 
\item अव्यवधानेन पदजन्यपदार्थोपस्थितिरासत्तिः~। घटः अस्ति इत्यादिदिनान्तरान्तरितात् वाक्यघटकशब्दकलापात् वाक्यार्थबोधो न भवति आसत्तेरभावात्~। 
\item वाक्यस्य यदर्थपरत्वं साधनीयं तदर्थानुभवयोग्यत्वं  तात्पर्यम्~। गेहे घटः इति वाक्यं गेहे घटसंसर्गप्रतीतिजननयोग्यम्, नतु पटसंसर्गप्रतीतिजननयोग्यमिति तद्वाक्यं घटसंसर्गपरम्, नतु पटसंसर्गपरम्~। 
\end{enumerate}

~\\[-1.7cm]

मानान्तराबाधिततात्पर्यविषयीभूतपदार्थसंसर्गबोधकत्वे सति वाक्यत्वम् आगमलक्षणम् इति फलितोऽर्थः~। एतत्सहकारिकारणचतुष्टये सति वाक्यात् वाक्यार्थबोधो भवति~। तथा च आकाङ्क्षादिसद्भावे वक्ता यत्तात्पर्यज्ञापकं यच्छब्दनिचयं प्रयुङ्क्ते, तत्तात्पर्यबोधकं तदेव वाक्यम्~। तच्च द्विविधम्- दृष्टार्थमदृष्टार्थं चेति~। यस्य वाक्यस्य प्रतिपाद्योऽर्थः चक्षुरादीन्द्रियग्राह्यो भवति, तद्दृष्टार्थं वाक्यम्~। चक्षुरादीन्द्रियाग्राह्यार्थवत् अदृष्टार्थं वाक्यम्~। “स्वर्गकामो यजेत” इत्यादिना स्वर्गाद्यप्रत्यक्षवस्तुविषयकं ज्ञानं लभ्यते, तस्य अदृष्टार्थत्वेऽपि आप्तवाक्यत्वात् तत्प्रमाणम्~। यद्यपि दृष्टार्थाप्तवाक्यस्य प्रामाण्यं प्रमाणान्तरेण ज्ञातुं शक्यम्, तथापि अदृष्टार्थाप्तवाक्यस्य प्रामाण्यं न प्रमाणान्तरेणेति आप्तवाक्यत्वादेव प्रामाण्यं स्वीकरणीयम्~। तत्र च न सर्वेषां वाक्यानां प्रामाण्यमस्ति, किं तु आप्तवाक्यस्यैव~। परमाप्तस्य वेदवाक्यस्य प्रामाण्ये किमु वक्तव्यम्? एतावदेव वक्तव्यं यत्- वाक्याज्जायमानं (शब्दात्)ज्ञानं कीदृश\-मिति~। क्वचिच्च शब्दादप्यपरोक्षं ज्ञानमुत्पद्यते इति वेदान्तिनामाशयः~। 

~\\[-1.7cm]

\section*{शाब्दापरोक्षवादः}

शब्देन निर्वर्तितं= शाब्दम्~। न परोक्षं= अपरोक्षम्~। शाब्दं च तदपरोक्षं च शाब्दापरोक्षं= ज्ञानम्~। शब्दजं सर्वमपि परोक्षमेवेति वेदन्तभिन्नाः सर्वेऽपि~। शब्दादपि अपरोक्षं ज्ञानमुत्पद्यतेति अद्वैतिनः~। न तु शब्दादेवेति~। ब्रह्मसाक्षात्कारेण खलु परमपुरुषार्थसिद्धिरिति औपनिषत्सिद्धान्तः~। अत एव आत्मदर्शनाय श्रवण-मनन-निदिध्यासनादीनि विहितानि~। श्रवणं नाम गुरुमुखात् वेदान्तवाक्यार्थश्रवणम्~। एवञ्च विचारजनितात् अपरोक्षरूपात् वाक्यार्थज्ञानात् मुक्तिरित्युक्तं भवति~। 

तदयुक्तमिति पूर्वपक्षः~। तथाहि- शब्दस्यैषः स्वभावः, यत्सः सर्वदा परोक्षमेव ज्ञानं जनयतीति~। लोके शब्दजन्यापरोक्षज्ञानाप्रसिद्धेश्च~। यदि वाक्यादेव साक्षात्कारः, श्रवणोत्तरयोः मनन- निदिध्यासनयोः विधानवैयर्थ्यं च प्रसज्येत~। विधीयेते च श्रवणोत्तरं मनन- निदिध्यासने “मन्तव्यो निदिध्यासितव्यः” इति~। अपि च “मनसैवानुद्रष्टव्यम्”, “ततस्तु तं पश्यते निष्कलं ध्यायमानः” इत्यादिश्रुतिभिः सावधारणं मनस एव अपरोक्षकरणत्वं दर्शितम्~। शब्दस्यापि अपरोक्षज्ञानकरणत्वे अवधारणवचनं भज्येत~। 

अत्रेदमवधेयम्- शब्दस्य परोक्षज्ञानमात्रजनकत्वनियमः नास्ति~। पारोक्ष्यं अनुमित्या\-देरपि साधारणम्~। पारोक्ष्ये शब्दजन्यत्वस्य अनवच्छेदकत्वात्~। (यत्र यत्र शब्दजन्यत्वं तत्र तत्र पारोक्ष्यमिति, यत्र यत्र परोक्षत्वं तत्र तत्र शब्दजन्यत्वमिति वा व्याप्तिर्नास्तीत्यर्थः) अतः अयोग्य- असन्निकृष्टविषयकज्ञानमेव परोक्षमित्यवश्यमङ्गीकरणीयम्~। विषपारोक्ष्यमेव ज्ञानस्य परोक्षत्वप्रयोजकमिति च~। नतु करणनियम्यत्वम्~। “अहं सुखी” “अहं दुःखी” इत्यादौ आपरोक्ष्यम्, योग्य-सन्निकृष्टविषयत्वात् सुख-दुःखादेः~। “अहं धार्मिकः” “अहमधार्मिकः” इत्यादौ च पारोक्ष्यमेव अयोग्यत्वात् धर्माधर्मादेः विषयस्य~। एवं च विषयपारोक्ष्य- आपरोक्ष्ययोस्तु आवरण- अनावरणे एव प्रयोजिके भवतः~। आवरणे सति अपरोक्षं ज्ञानं नोत्पद्यते, सन्निकर्षादिना घटादौ आवरणनाशे सति “अयं घटः” इत्यपरोक्षं ज्ञानमुत्पद्यते~। अनावरणं द्विविधम्- आवरणभङ्गः, तदत्यन्ताभावश्चेति~। सन्निकृष्टसामग्रीप्रयोज्यः आद्यः~। साक्षिणि अत्यन्ताभावरूपो द्वितीयः~। (साक्षिणि तदध्यस्तान्तःकरण-तद्वृत्यादौ अत्यन्ताभाव इत्यर्थः)

~\\[-1.7cm]

\section*{पूर्वपक्षः}

ननु प्रत्यक्षे चक्षुरादिषट्कस्य करणत्वं क्लृप्तम्, तद्व्यतिरेकेण शब्दस्यापि करणत्वे प्रत्यक्ष\-प्रमाणस्य वृद्धिः स्यात्, तदपेक्षया अन्यत्र क्लृप्तस्य पारोक्ष्यस्यैव सन्निकृष्टशाब्दबुद्धावपि\break स्वीक्रियते~। भ्रान्तिनिवृत्यनुभवनिर्वाहाय च तदव्यवहितोत्तरक्षणे मानसादिप्रत्यक्षं शाब्दबुद्धिसमानाकारं कल्प्यते, मनसः आत्मसाक्षात्कारकरणत्वस्य क्लृप्तत्वात् इति~। 

\section*{सिद्धान्तः}
\vskip -3pt

तदपि न युक्तम्~। एवं सति आत्म-तत्समवेत-सुखादिसाक्षात्कारे क्लृप्तस्य मनस एव\break सर्ववस्तुसाक्षात्कारेऽपि करणत्वं कल्प्यताम्, लाघवात्~। (चक्षुरादिघटितमनसः संयुक्तसम\-वेतादिसंनिकर्षान्तरकल्पनायाः अनवकाश इत्यर्थः) “चक्षुषा पश्यामि” इत्यादिप्रतीतेस्तु चक्षुषादीनां द्वारत्वेनापि निर्वोढुं शक्यत्वात्~। अपि च संयुक्तविशेषणतादि अनेकसन्निकर्षान् अङ्गीकृत्य, अनुपलब्धिप्रमाणं निराकुर्वतस्तव संयुक्तसमवायाख्यस्य एकस्यैव सन्निकर्षस्य अभ्युपगमेन प्रमाणपञ्चकपरित्यागे का लज्जा? (न काप्यनुपपत्तिरित्यर्थः) शब्दादेव अपरोक्षसाक्षात्कारेऽपि श्रवणोत्तरयोः मनन-निदिध्यासनयोः विधानं तु प्रमेयासम्भावनादि अप्रतिबद्धसाक्षात्कारजननायेति ब्रूमः~। प्रतिबद्धसाक्षात्कारस्य अविद्यानिवर्तनाक्षमत्वात्~। 

~\\[-1.7cm]

\section*{पूर्वपक्षः}
\vskip -3pt

ननु  यथा तत्तत्पुरुषगतसुखदुःखादेः तं तं प्रति अपरोक्षत्वेऽपि अन्येन शब्दलिङ्गादिना गृह्यमाणस्य पारोक्ष्यम्, तथा स्वतः अपरोक्षस्यापि आत्मनः शब्दलिङ्गादिना गम्यतायां पारोक्ष्यमेवास्तु (क्वचित् परोक्षमेव ज्ञानं जनयतः प्रमाणस्य अकस्मादेव अपरोक्षज्ञानजनकत्वं न युक्तमिति) इति चेत्-

~\\[-1.7cm]

\section*{सिध्दान्तः}
तार्किकैरपि वन्ह्याद्यनुमितिरूप-परोक्षज्ञानजनकस्यापि परामर्शस्य सन्निकृष्टविषये अपरोक्षज्ञानजनकत्वस्य स्वीकृतत्वात्~। परकीयसुख-दुःखादेः पारोक्ष्यं तु परं प्रत्ययोग्यत्वात् विषयस्य, न तु शब्दधीगोचरत्वादित्येव~। मानवैचित्र्येण वस्तुनि वैचित्र्यमाधीयते इत्येव~। प्रदीपेन प्रकाशमानो घटः नीलश्च, स एव तरणिना प्रकाशमानः रक्तश्चेति प्रकाशभेदात्वस्तुनि वैचित्र्यं वस्तुतो न भवति~। एवं प्रमाणानामपि वस्तुयाथात्म्यव्यञ्जकत्वा\-त्पारोक्ष्य -आपरोक्ष्यादि -वैचित्र्यव्यञ्जकत्वायोगात्~। एतेन- आघ्राणजं सौरभज्ञानं परोक्षमेवेत्यादि प्रत्युक्तं भवति~। (लौकिक-अलौकिकसन्निकर्षभेदौ, तद्धेतू कल्प्येयाताम्)

ननु प्रकाशकवैचित्र्यात् प्रकाश्ये वैचित्र्यं दृश्यते-यथा बालसूर्यमरीचिप्रकाशितस्य अभ्रखण्डस्य रक्तता, प्रौढप्रकाशितस्य शुभ्रता, यथा वा रक्तपटस्य चन्द्रिकायां नैल्यप्रतिभासः दृष्टः इति चेन्न- आरोपमन्तरेण वस्तुनि एतादृशप्रमात्मभानासम्भवात्~। वाक्यार्थबुद्धेस्तु अप्रमात्वे, वाक्यस्यापि अप्रामाण्यापत्तेः~। अतः-महावाक्यस्य प्रामाण्यान्यथानुपपत्यापि शाब्दापरोक्षसिद्धिः~। नहि “इदम्” इति ग्राह्ये “तत्” इति व्यवहारः~। पारोक्ष्य-अपरोक्ष्ये हि परस्परविरुद्धे, नैकवस्तुस्वभावौ भवतः~। (वस्तुनि विकल्पायोगादित्यर्थः) एवं च नित्यापरोक्षं चैतन्यं पारोक्ष्येण ग्राहयत् वाक्यम् अप्रमाणं भवेत्~। आकाशे रक्तताप्रतिभासः, रक्तपटे नैल्यप्रतिभासश्च न यथार्थः~। अपि च ज्ञान-विषययोः आपरोक्ष्ये करणानां का नियामकता? न च अन्वय-व्यतिरेकौ नियामकौ~। करणेषु सत्स्वेव अपरोक्षं ज्ञानम्, करणाभावे अपरोक्षज्ञानाभाव इति नियमासम्भवात्~। तत्र च न तावत् व्यतिरेकनियमसम्भवः, सुख-दुःखादिषु व्यभिचारात्~। (मनसः अनिन्द्रियत्वात्पराभिमत-विशेषगुणानां सुख-दुःख-इच्छा-प्रयत्नादीनाम् इन्द्रियवेद्यत्वम् औपनिषदैः (वेदान्तिभिः) नाङ्गीक्रियते~। ) मनसः इन्द्रियत्वाङ्गीकारेऽपि मनस्तादात्म्यापन्नानां तेषां (करणकोटिप्रविष्टानां) तद्वेद्यत्वायोगात्~। न हि दार्वादीन् छिद्यता परशुना स्वधर्माः छिद्यन्ते~। किन्तु सुखादयः साक्षिवेद्या एव~। अन्धकार- अज्ञान-स्वप्नपदार्थेषु  अपरोक्षत्वेन सर्वानुभवसिद्धेषु करणवेद्यत्वाभावात् साक्षिवेद्यत्वमेव~। (१. अन्धकारस्य भावत्वे आलोकसापेक्षचक्षुर्ग्राह्यत्वापत्तेः न भावरूपता~। अभावत्वे तु विशेषणताख्यसन्निकर्षनिरासेन अधिकरणालोकसंयोगरूपसन्निकर्षाभावात् (हेत्वभावात्) चक्षुषः अयोग्यत्वमेव~। २. अज्ञानेऽपि नेन्द्रिययोग्यता, रूपाद्यभावात्~। मनसोऽपि उपादानत्वाच्च~। ३. स्वप्ना अपि जाग्रद्वासनावासिताः सन्तः मनसः परिणामरूपाः मनसा ग्रहीतुमशक्याः~। लीनत्वात् साक्षिवेद्या एव ते~। साक्षी च आपरोक्षः~। एवं व्यतिरेकव्याप्त्यसम्भवः इति भावः)

नापि अन्वयनियमः, सत्यपि योगादिसन्निकर्षे चक्षुषा रूपादिरेव गृह्यते, न गुरुत्वादिकम्~। कस्मात्? अयोग्यत्वादिति चेत्- किं तद्योग्यत्वम्? न तावत् रूपादिमत्वम्, गुणादिषु अननुगतत्वात्~। न वा रूपमहत्वादिमत्समवेतत्वम्, रूपत्व-त्र्यणुकादिषु अननुगतत्वात्~। एवं च अस्मदभिमतम् अनावृतचिदभिन्नत्वमेव अनुगतमिति तत्प्रयोजकत्वं वक्तव्यम्~। (अनुगतजन्यज्ञानापरोक्ष्य्प्रयोजकयोग्यतायाः दुर्वचत्वादिति भावः) एवं च प्रत्यक्षे विषयस्य कारणत्वमङ्गीकृत्य, आपरोक्ष्ये योग्यतायाः प्रयोजकता दुर्वारा एव~। योग्यतया एव आपरोक्ष्यसिद्धेः, अयोग्ये सौरभादौ चाक्षुषत्वमङ्गीकुर्वतः स्फुट एव अन्वयव्यभिचारः~। अयं भावः- अपरोक्षता न इन्द्रियैः नियम्यते, किन्तु अभानापादक-अज्ञानावरणभङ्गे विषयावच्छिन्नचैतन्याभेद एव प्रयोजकः इति~। तत्रायं दृष्टान्तः-

तथाहि- कदाचित् गुरुणा ऋजुबुध्दयः दश शिष्याः कार्यविशेषे प्रेरिताः ग्रामान्तरं प्रति\break प्रेषिताः सन्तः मध्येमार्गं नदीमुत्ततारुः~। ततः सर्वे उत्तीर्णास्स्मः, उत कश्चिन्नः प्रवाहेणाऽपवा\-हित इति परिज्ञानाय तेष्वेकैकोऽपि स्वेतरान्नव पुरुषान् गणयित्वा, स्वात्मगणने व्यामूढो\break विषसाद~। इत्थं सर्वेऽपि दशमं प्रवाहापनीतं मन्वानाः शोकाकुलाः तमेवान्विष्यमाणाः\break नदीस्रोतसमनुजग्मुः~। तत्र च अन्यपुरुषमायान्तं दृष्ट्वा पप्रच्छुः- “किं चास्माकं नदीस्रोतसा प्रवाह्यमाणं दशममद्राक्षीः” इति~। स एतान् गणयित्वोवाच-“दशैव यूयं स्थ, न कश्चि\-दपवाहिः” इति~। तच्छ्रुत्वा एकैकोऽपि पूर्ववन्नवैव गणयित्वा “को दशमः” इति पप्रच्छ~।\break सः प्रत्याह- “दशमस्त्वमसि” इति~। एकैकोऽपि एवमेव तस्य वचनं श्रुत्वा, निपुणमवधार्य विशोको बभूव~। एवं चात्र शब्दादप्यपरोक्षं ज्ञानमुत्पन्नमिति प्रसिद्धम्~। एवं\break तत्त्वमसीत्यत्रापि~। नहीदं परोक्षमिति शक्यते वक्तुं, अनन्तरं तादृशदशम\-पुरुषसाक्षात्कारेच्छायाः अभावात्~। अन्यथा दशमदिदृक्षेच्छा नोच्छिद्येत~। न च भ्रमत्वं, “असौ मामेव दशमं\break दर्शयामास, अस्य वाक्यान्मे मोहोऽपगत” इति अनुभूयमानस्य परोक्षत्वकल्पने भ्रमत्व\-कल्पने वा गौरवात् , मानाभावात्, व्यवहारविरोधाच्च~। परोक्षज्ञानरूपधर्मिकल्पनातो वरं\break शब्दस्य साक्षात्कारकरणत्वरूपधर्मकल्पनं लाघवात्~। \textit{बहूनां ज्ञानानां भ्रान्तत्वकल्पनायां\break गौरवाधिक्यादिति च ज्ञेयम्}~। अपि च “तस्मै मृदितकषायाय तमसः पारं दर्शयति भगवान् सनत्कुमारः” इत्युपदेशः कृतः श्रूयते छान्दोग्यसप्तमप्रपाठके~। तत्र साक्षात्कारानुकूलव्यापारवान् हि भगवान् सनत्कुमारः “दर्शयति” इति णिजन्तप्रकृतिव्याख्यानेनोच्यते~। यदि बोधकवाक्यप्रयोगरूपः उपदेशः शिष्यस्य साक्षात्कारे हेतुः न स्यात्, तदा “दर्शयति” इति नाऽवक्ष्यत्~। तथा “आत्मा वारे द्रष्टव्यः श्रोतव्यः” इति साक्षात्कारमेवोद्दिश्य श्रवणविधानात्, मुख्याधिकारिणां श्रवणमात्रेण कृतार्थताश्रुतेश्च शाब्दापरोक्षसिद्धिः~। तथा च अखण्डशाब्दवृत्त्युदयसमकालमेव ब्रह्मावरकाज्ञाननाशात् न आपरोक्ष्यक्षतिरित्यर्थः~। एवं च श्रुत्यनुभवौ शाब्दापरोक्ष्ये प्रमाणम्~॥ टिप्पणीः =

~\\[-1.5cm]

\begin{verse}
अपरोक्षवस्तुविषयेति शब्दजाऽप्यपरोक्षतां न विजहात्यखण्डधीः~। \\
7निपुणं निशम्य दशमस्त्वमित्यादिवचनं न हि स्वमुपलब्धुमिच्छति~॥१॥\\
न च शब्दजन्यमतिमात्रतः स्वतो प्यपरोक्षवस्तु भजते परोक्षताम्~। \\
न खलु प्रदीपतरणिप्रकाशयोः प्रतिभासतोऽस्ति विषये पृथग्विधा~॥२॥\\
अपरोक्षवस्तुविषया परोक्षधीः न च मानभावमुपगन्तुमर्हति~। \\
स्वत एव शश्वदभातमात्मनः कुशलोऽपि न स्वमपि गूहितुं क्षमः~॥३॥\\
सुखदुःखरागयतनेष्वसम्मतेः तिमिरप्रसुप्तिविषयेष्वदर्शनात्~। \\
अपि हेतुरस्य विषयस्य योग्यते त्यपरोक्षता न करणैर्नियम्यते~॥४॥\\
अपिचाऽवृतार्थविषया परोक्षधीः वचसा च भग्नमनुभूयते तमः~। \\
उपदेशमात्रमुपवर्ण्य तत्पुनः श्रुतिराह दर्शयति पारमित्यपि~॥५॥\\
विदधाति यच्च परमात्मदृष्टये श्रवणं विमृष्टनिगमान्तगोचरम्~। \\
श्रुतिमात्रतीर्णतमसोयदब्रुवन् जनकोऽसि तारयसि पारमित्यपि~॥६॥\\
चैतन्यस्यापरोक्ष्यं स्वरसनिजदृशो मुख्यमन्याऽनपेक्षं\\ 
गौणं तद्गोचरेष्वावरणविरहिते तत्र तादात्म्यवत्सु~।\\ 
बाह्ये चैतन्ययोगं घटयितुमुचिता वृत्तिरक्षप्रसूता\\
वाक्योत्था मोहमात्रं विघटयति चितस्स्फूर्तिरन्याऽनपेक्षा~॥७॥\\
\hspace{5cm}-स्वाराज्यसिद्धौ कैवल्यप्रकरणम् 
\end{verse}

~\\[-1.7cm]

\noindent
वेदान्तपरिभाषायां-“नचैवमपि सुखस्य वर्तमानतादशायां त्वं सुखीत्यादिवाक्यजन्य\-ज्ञानस्य प्रत्यक्षता स्यादिति वाच्यम्, इष्टत्वात्~॥ दशमस्त्वमसीत्यादौ सन्निकृष्टविषये शब्दादप्यपरोक्ष-ज्ञानाभ्युपगमात्”१॥ इत्युक्तम्~। अयमत्राशयः- वेदान्तनये एकस्मिन्नेव ज्ञाने परोक्षत्वाऽपरोक्षत्वे स्वीक्रियेते~। “सुरभि चन्दनम्”“पर्वतो वह्निमान्” इत्यादावपि चन्दनखण्डांशे पर्वतांशे च अपरोक्षत्वम्  सौरभ्यांशे वह्न्यंशे च परोक्षत्वम्~। तथा सति तयोः जातित्वं नस्यादिति नैव शङ्क्यम्, तादृशजातेरनङ्गीकारात्~। तस्मात् शब्दादप्यपरोक्षं ज्ञानमुत्पद्यते इति सिध्दान्तः वेदान्तिनाम्~। अयमेव शाब्दापरोक्षवाद इत्युच्यते~। अयं च मतभेदेन चतुर्धा विभज्य प्रदर्शितः सिध्दान्तलेशसङ्ग्रहे अप्पय्यदीक्षितेन्द्रैः~। (विस्तरभयान्नेह वितन्यते)

\articleend
}
