{\fontsize{15}{17}\selectfont
\presetvalues
\chapter{आयुर्वेदः तर्कशास्त्रं च}

\begin{center}
\Authorline{डा~। एन् एस् रामचन्द्रः, {\fontsize{12}{14}\selectfont \eng{M.B.B.S}}}
\smallskip

न्याय-आयुर्वेदोभयविद्वान्,\\
चामराजपुरम् मैसूरु
\addrule
\end{center}

पुरा तपोपवासाध्ययनब्रह्मचर्यव्रतायुषां जनानां धर्मार्थकाममोक्षरूपपुरुषार्थाणां प्राप्तौ विघ्नभूताः रोगाः प्रादुरभूवन्~। तेषां शमनार्थम् उपायम् अन्वेष्ट्म् अङ्गिरा जमदग्न्यादयः अनेके ऋषयः शुभे हिमवतः पार्श्वे सम्मिलिताः चर्चाम् अकुर्वन्~। अस्मिन् विषये देवराजः इन्द्रः एव नः शरम् इति ते ध्यानचक्षुषा ददृशुः~। यतः ब्रह्मणा प्रोक्तं समग्रमायुर्वेदं प्रजापतिः जग्राह~। तस्मात् अश्विनौ अविन्दताम्~। ताभ्यां देवराजः इन्द्रः अयुर्वेदज्ञानं प्राप~। तस्मात् इन्द्रसमीपं को गच्छेत्~। इति जिज्ञासायाम् “अहं सहस्राक्षभवनं गच्छेयम्” इति भरद्वाजमुनिः प्रथमम् अवोचत्~। अन्ये ऋषयः अनुमेनिरे~। भरद्वाजो मुनिः सहस्राक्षस्य भवनं गत्वा तम् आशीर्भिः अभिनन्द्य ऋषीणां वचनं श्रावयामास~। तेन सन्तुष्टः देवराजः,भरद्वाजस्य मते तीक्ष्णताम् उपलक्ष्य अल्पैः पदैः विपुलं आयुर्वे बोधयामास~। भरद्वाजः ऋषिसभां प्रतिनिवृत्य इन्द्रेण यथा प्रोक्तं तथैव समग्रं आयुर्वेदम् ऋषिभ्यः अवोचत्~। कथमिति चेत् -
\begin{verse}
सामान्यं च विशेषं च गुणान् द्रव्याणि कर्म च~। \\
समवायं च तज्ज्ञात्वा तन्त्रोक्तं विधिमास्थिताः~। \\
लेभिरे प्ररमं शर्म जीवितं चाप्यनित्वरम्~॥ इति~। 
\end{verse}
पश्यन्नेवैतच्छ्लोकं एते तर्कशास्त्रे प्रसिद्धाः षड्भावपदार्थाः~। तर्हि आयुर्वेदः तर्कशास्त्ररूप एव इत्युद्गिरन्त्येव~। पदार्थानां निर्देशक्रमस्तु भिन्नः~। चिकित्सा शास्त्रस्य आयुर्वेदस्य ज्ञाने आनुकूल्यं वीक्ष्य पदार्थानां क्रमः विवरणानि च परिवर्तितानि दृश्यन्ते~। आयुषो वेदः- आयुर्वेदः~। 
\begin{verse}
आयुर्नाम शरीरेन्द्रियसहात्मसंयोगो धारि जीवितम्~। \\
नित्यगश्चानुबन्धश्च पर्यायैरायुरुच्यते~॥
\end{verse}
श्लोके चतुर्णां संयोगः एव आयुः इति कथितम्~। के ते इति चेत् – शरीरं प्ञ्चभूतात्मकम्, इन्द्रियाणि, श्रोत्र-त्वङ्नयनरसनाघ्राणरूपाणि~। सत्त्वं मनः आत्मा चेतनश्चेति~। शरीरवत् ब्रह्माण्डोऽपि पाञ्चभौतिकः~। पञ्चभूतात्मकं शरीरम् आयुर्वेदे त्रिदोषमयं परिगण्यते~। तद्यथा वातपित्तकफात्मकमिति~। वातदोषः आकाशवायुभूताभ्यां भवति~। पित्तम् अग्निभूतात्मकम्~। कफश्च पृथ्वीजलात्मकः~। एतेषां दोषत्रयाणां साम्यम् आरोग्यमिति कथ्यते~। तेषां वैषम्यं रोग इति~। तादृशस्य आरोग्यवतः स्वस्थस्य लक्षणमाह-
\begin{verse}
समदोषःसमाग्निश्च समधातुमलक्रियः~। \\
प्रसन्नत्मेन्द्रियमनाः स्वस्थ इत्यभिधीयते~॥ इति\\
पुरुषार्थाणां प्राप्त्यर्थं तादृशम् आरोग्यम् आवश्यकम्~। यतः –\\
धर्मार्थकाममोक्षाणाम् अरोग्यं मूलमुत्तमम्~। \\
रोगास्तस्यापहर्तारः श्रेयसो जीवितस्य च~॥ 
\end{verse}
स्वस्थस्य स्वास्थ्यरक्षणम् आतुरस्य रोगप्रशमनं च चिकित्सायाः प्रयोजने भवतः~। एतदर्थं पदार्थानां तत्वज्ञानम् आवश्यकं भवति~। एष एव विषयः, नाम- तर्कशास्त्रं कथं चिकित्साज्ञाने पूरकम्- इति अस्मिन् प्रबन्धे परिशील्यते~। 

चरकसंहितायाः सूत्रस्थाने प्रथमेऽध्याये षड्भावपदार्थाः आयुर्वेदोपयोगिनः निरूपिताः वर्तन्ते~। यथा -

\section*{सामान्यं –विशेषश्च -}

\begin{verse}
सर्वदा सर्वभावानां सामान्यं वृद्धिकारणम्~॥\\
ह्रासहेतुविशेषं च प्रवृत्तिरुभयस्य तु~॥
\end{verse}

शरीरस्थानां वातपित्थकफरूपाणां त्रयाणां दोषाणां, रसरक्तमांसमेदोऽस्थिमज्जाशुक्र\-रूपाणां सप्त धातूनां, मलमूत्रस्वेदादीनां मलानां च ये गुणाः कर्माणि च तादृशगुणकर्मवन्ति द्रव्याणि तत्तद्दोषधातुमलानां वृद्धिं जनयन्ति~। तत्र कारणं सामान्यं नाम-समानगुणकर्मवत्वम्~। तथैव रक्तं रक्तेन वर्धते मांसं मांसेन~। अत्र समाना जातिः वृद्धिकारणं भवति~। अतः अयुर्वेदे जातिभिन्नमपि गुणकर्मादिरूपं सामान्यशब्देन व्यवह्रियते~। तर्कशास्त्रे तु “नित्यमेकमनेकानुगतं सामन्यमिति” जातिरूपमेव गृह्यते~। अत एव आयुर्वेदे सामन्यलक्षणं “ सामन्यमेकत्वकरम्” इति वर्तते~। एकत्वकरमित्यस्य एकत्वबुद्धिकरमित्यर्थः~। 

एवमेव “अन्त्यो नित्यद्रव्यवृत्तिः विशेषः परिकीर्तितः” इति~। नित्यद्रव्यवृत्तिः अन्त्यः व्यावर्तकः एव विशेषः~। विशेषाः अनन्ताः इति विवरणं तर्कशास्त्रे लभ्यते~। आयुर्वेदे तु “विशेषस्तु पृथक्त्वकृत्” इति तल्लक्षणमुक्तम्~। नाम व्यावर्तकत्वमात्रं विशेषस्य लक्षणम्~। तेन जातिगुणकर्माणि यानि व्यावर्तकानि दृश्यन्ते तेषां सर्वेषामपि विशेष इति संज्ञा सिध्यति~। तादृशाः विशेषाः ह्रासहेतुर्विशेषश्च इत्युक्त्वा भावानां क्षये कारणानि भवन्ति~। यथा कफे गुरुत्वस्निग्धत्वादयो गुणाः वर्तन्ते~। तद्विरुद्धगुणवत्वात् हरीतकी कफं क्षपयति~। बृंहणकर्मवत् क्षीरं लङ्घनकर्मवन्तं वायुं क्षपयति~। सामान्यविशेषयोः अपरो विशेषः” सर्वदा तयोः प्रवृत्तिः युगपदेव भवतीति~। यथा हरीतकी सामान्यात् यदा वातं वर्धयति सममेव विशेषात् कफं क्षपयति~। क्षीरं सामन्यात् कफं वर्धयति सममेव विशेषात् वातम् पित्तं च क्षपयति इति~। अत एव

वृद्धाः ह्रासयितव्याः क्षीणाः वर्धयितव्याः समाः पालयितव्याः~। इति चिकित्सा सूत्रम् सम्पद्यते~। 

गुणाः – पदार्थानां गणनायां द्वितीयं स्थानं गुणानाम्~। यतः सामान्य विशेष निर्णये गुणाः प्रमुखं पात्रं वहन्ति~। 


समवायी तु निश्चेष्टः कारणं गुणः इति तस्य लक्षणं उच्यते~। समवायाधेयः कर्मभिन्नः कारणतावान् गुणः इत्यर्थः~। तर्कशास्त्रे चतुर्विंशतिः गुणाः प्रोक्ताः~। आयुर्वेदे तु एकचत्वारिंशत् गुणाः कथ्यन्ते~। यथा

\begin{verse}
सार्था गुर्वादयो बुद्धिः प्रयत्नान्ताः परादयः~। \\
गुणाः प्रोक्ताः ...............................~॥ इति~। 
\end{verse}
अर्थाः इन्द्रियार्थाः शब्दस्पर्शरूपरसगन्धाः पञ्च~। गुर्वादयः विंशतिः गुणाः यथा \ गुरु-लघु-शीत-उष्ण-स्निग्ध- रूक्ष-मन्द-तीक्ष्ण-स्थिर-सर-मृदु-कठिण-विशद-पिच्छिल-श्लक्ष्ण-खर-स्थूल-सूक्ष्म-सान्द्रद्रवाः~। बुद्धिः ज्ञानम्~। 

\section*{प्रयत्नान्ताः}

इच्छा, द्वेषः, सुखं, दुःखं, प्रयत्नः इति पञ्च~। परादयः

\begin{verse}
परापरत्वे युक्तिश्च स्ंख्या संयोग एव च~। \\
विभागश्च पृथक्त्वं च परिमाणमथापि च~। \\
संस्कारोऽभ्यास इत्येते गुणाः प्रोक्ताः परादयः~॥
\end{verse}
इति दश गुणाः~। इत्थम् आहत्य एकचत्वारिंशत् गुणाः~। एतेषु गुर्वादयः विंशतिर्गुणाः अतीवमुख्याः इत्यतः प्रत्येकस्मिन् भेषजद्रव्ये गुर्वादयः परिशील्यन्ते एव~। एवमेव संस्कारः, अभ्यासश्च चिकित्सायां प्रमुखं पात्रं वहतः~। 

संस्कारो हि गुणान्तराधानम्~। उदाहरणार्थं सर्वेषां मुख्याहारभूतान् तण्डुलानेव पश्यामः~। अन्नं तण्डुलान् जलेन साकं प्क्त्वा कुर्वन्ति~। तस्य भक्षणेन तृप्तिः उदरगौरवं च भवति~। तस्यैव पेया- {\fontsize{14}{16}\selectfont \kan{(ಪೇಯಾ ಎಂದರೆ  ಹೆಚ್ಚು ಗಟ್ಟಿಯಾಗಿಯೂ ಮಾಡದ, ತುಂಬಾ ನೀರಾಗಿಯೂ ಮಾಡದ, ಆದರೆ ಕುಡಿಯಲು ಯೋಗ್ಯವಾಗುವಂತೆ ದ್ರವಾಂಶವುಳ್ಳ ಅನ್ನದ ಗಂಜಿ)}} द्रवाधिका भवति~। सा तु लघ्वी जीर्णकरी च~। विषं जन्तून् मारयति~। किन्तु यदा विशिष्टः संस्कारः क्रियते, तदेव विषं रोगहरणं कृत्वा अमृतत्वाय कल्पते~। 

\section*{अभ्यासः}

अभ्यासः नाम शीलनं, अपरः मुख्यो गुणः~। क्षीरं अभ्यासेन सप्तधातून् वर्धयित्वा रसायनं भवति~। वृष्ट्या अभिषिक्ताः शीतबाधां ज्वरादिकं च लब्ध्वा परितपन्ति~। किन्तु कृषीवलाः नित्यं वृष्टौ चरन्तः अपि नीरोगिणः, वृष्टिवर्षं अभिनन्दन्ति~। तत्रापि कारणं अभ्यास एव~। अतः अभ्यासः विशिष्टगुणतया वर्ण्यते आयुर्वेदे~। 

\section*{द्रव्यम्}

गुणानामाधारभूतत्वात् गुणनिरूपणानन्तरं द्रव्याणि निरूपयति~। तल्लक्षणं तु –यत्राऽऽश्रिताः कर्मगुणाः समवायि यत् तद्द्रव्यम्....इति~। यत्र कर्माणि गुणाश्च समवायसंबन्धेन वर्तन्ते, यच्च कार्यसामन्यं प्रति समवायिकारणं तदेव द्रव्यम्” इत्यर्थः~। नव द्रव्याणि यथा \

खादीन्यात्मा मनः कालो दिशश्च द्रव्यसंग्रहः~। इति


खादीनि पञ्चभूतानि  आकाशः, वायुः, तेजः, जलं पृथिवीति~। आत्मा - चेतनः, मनः - सत्वं, कालः - क्षणादि दिशः - पूर्वादयः~। तर्कशास्त्रेपि एतान्येव नव द्रव्याणि उक्तानि~। अत्र वातपित्तकफानां प्राधान्यं सूचयति क्रमभेदः~। अकाशः वायुश्च वातदोषं जनयतः~। तेजः पित्तं जनयति~। जलं पृथिवी च कफं जनयतः~। अत एव आकाशम् आरभ्य पञ्चभूतान्युक्तानि~। 

चैतन्ययुक्तं पाञ्चभौतिकं शरीरमेवमुख्यं चिकित्सायां, अतः आत्मा मनश्च ततः निरूपिते~। अत एव
\begin{verse}
सेन्द्रियं चेतनं द्रव्यं निरिन्द्रियमचेतनम्\\
इति द्रव्याणि द्विधा विभक्तानि~। 
\end{verse}
भेषजानां, चिकित्साक्रियाक्रमाणां वमनविरेचनादीनां प्रयोगः योग्ये काले योग्ये च देशे एव सिद्धिं जनयति~। अतः कालः दिक्च ततः निर्दिष्टौ स्तः~। 

\section*{कर्म} 

द्रव्याणि स्वगुणानुसारं, स्वतन्त्रतया प्रभावेन च यत्कर्मकुर्वन्ति दीपनपाचनवमनविरेचनादीनि तान्येवकिल रोगं परिहृत्य आरोग्यं स्थापयन्ति~। अत एव द्रव्यनिरूपणानन्तरं कर्मनिरूपयति~। प्रयत्नादि कर्मचेष्टितमुच्यते” इति~। प्रयत्नः अत्र क्रिया एव~। अतः चेष्टरूपाः सर्वाः क्रियाः कर्म इत्येवाभिधीयन्ते~। तत्र कर्मलक्षणं तु-
\begin{verse}
संयोगे च विभागे च कारणं द्रव्यमाश्रितम्~। \\
कर्तव्यस्य क्रिया कर्म कर्म नान्यदपेक्ष्यते~॥ इति॥
\end{verse}
यः पदार्थः द्रव्यसमवेतः, उत्तरदेशसंयोगे पूर्वदेशविभागे च युगपत्कारणं भवति स एव कर्मेत्यभिधीयते~। 

तच्चक्रियारूपमेवेत्यर्थः~। 

\textbf{समवायः-} द्रव्ये गुणाः कर्माणि च समवायसम्बन्धेन वर्तन्ते इत्युक्तं किल~। तत्र समवायः कीदृशः इत्याकांक्षायां षष्टं समवायरूपं पदार्थं विवृणोति~। 
\begin{verse}
समवायोऽपृथग्भावो भूम्यादीनां गुणैर्मतः~। \\
स नित्यो यत्र हि द्रव्यं न तत्रानियतो गुणः॥ इति~॥
\end{verse}
अपृथग्भावः अयुतसिद्धिः~। एकं विना अन्यः न वर्तत इत्यर्थः~। यथा गन्धं विना पृथिवी, रसं विना जलं, न लभ्यते~। अत एवायं नित्यः सम्बन्धः समवायः भवति~। एवं षड्भावपदार्थान्निरूप्य चरकसंहितायां भगवान् आत्रेयः-
\begin{verse}
इत्युक्तं कारणं कार्यं धातुसाम्यमिहोच्यते~। \\
धातुसाम्यक्रिया चोक्ता तन्त्रस्यास्य प्रयोजनम्॥
\end{verse}
इति घोषयति~। एवं च षण्णां पदार्थानां ज्ञानं धातुसाम्यसम्पादने कारणम्भवतीत्यर्थः~। धातुसाम्ये मनसः प्रसन्नता, ततः आत्मनः परमात्मनश्च योगः नाम साक्षात्कारः भवति~। तस्मिन् परावरॆ दृष्टे सर्वॆषां दुःखानां निवृत्तिः नाम मोक्षः सम्पद्यते~। अत एव चरकसंहितायाः शारीरस्थाने प्रथमेऽध्याये अग्निवेशाऽऽत्रेययोः संवादः एवंरूपः दृश्यते \ क्व चैता वेदना सर्वाः निवृत्तिं यान्त्यदोषतः इति अग्निवेशेन पृष्टः भगवानात्रेयः
\begin{verse}
योगे मोक्षे च सर्वसां वेदनानामवर्तनम्~। \\
मोक्षे निवृत्तिर्निश्शेषा योगो मोक्षप्रवर्तकः~॥
\end{verse}
इति समाधानं दास्यति~। 

तर्कशास्त्रेऽपि पदार्थानां तत्त्वज्ञानात् एकविंशतिदुःखध्वंसरूपः मोक्षः सिद्ध्यतीति किल सिद्धान्तः~। अतः आयुर्वेदः तर्कशास्त्रम् इति शास्त्रद्वयमपि समानेनोद्देशेन प्रवृत्तौ पर्स्परं पूरकम् इति शम्~। 

\articleend
}
