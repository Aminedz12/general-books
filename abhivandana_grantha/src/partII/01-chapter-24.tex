{\fontsize{15}{17}\selectfont
\presetvalues
\chapter{अथर्ववेदे लोकोपयोगिनः विषयाः}

\begin{center}
\Authorline{वि~॥ डा. टि.वि.सत्यनारायणः}
\smallskip

अलङ्कार-व्याकरणशास्त्रविद्वान्,\\ 
निवृत्तः उपनिर्देशकः, \\
अद्यतन गौरवमार्गदर्शकश्च प्राच्यविद्यासंशोधनालयः, मैसूरु
\addrule
\end{center}
\vskip -10pt

जगतः सृष्टेः प्रारम्भे अग्नि-वायु-आदित्य-अङ्गिरा इति चतुर्ण्णाम् ऋषीणां हृदये समुद्भूताः ऋग्यजुस्सामाथर्वेति नामधेयैः ख्यातिं गताः चत्वारो वेदाः अधुना लोके समुल्लसन्ति~। चतुर्षु वेदेष्वपि अयमथर्ववेदः क्रमानुरोधेन चतुर्थः~। चत्वारोऽपि वेदाः समानपदभाज एव~। न हि तेषु महत्वविषयकं तारतम्यं पदं धत्ते~। चतुर्षु वेदेषु अथर्ववेदोऽसौ यद्यपि चतुर्थः अथापि चतुर्ण्णां वेदानां मध्ये कर्मकाण्डदृष्ट्या अतिशयितं पदमालम्बते~। कर्मकाण्डे तु यज्ञयागादिश्रौतस्मार्तकार्यनिर्वहणाय चत्वारः ऋत्विजः होता अध्वर्युः उद्गाता ब्रह्मा इति क्रमशः विद्यन्ते~। तेषु प्रमुखस्थाने अथर्ववेदज्ञः ब्रह्मा एव प्रमुखं विशिष्टतमं च स्थानं भवितुमर्हति नान्यः~। \textbf{यज्ञैरथर्वा प्रथमः पथस्तते} (ऋ.सं.१-८३-५) इति ऋग्वेदे एव उक्तत्वात्~। 

अथर्ववेदे २० काण्डास्सन्ति~। ७३१ सूक्तानि ५९८७ मन्त्राश्च अत्र विद्यन्ते~। एषु मन्त्रेषु द्वादशशतमिता मन्त्रास्तु ऋग्वेदेऽपि उपलभ्यन्ते~। अथर्ववेदस्य प्रायशः अशीतिमितानि सूक्तानि गद्यरूपाणि~। अस्य वेदस्य षष्ठांशमितो भागः गद्यात्मक एव~। 

\section*{अथर्ववेदे प्रतिपादिताः मुख्याः विषयाः}

अथर्ववेदे आयुर्वेदः मुख्यतया प्रतिपाद्यते~। अस्य ७३१ सूक्तेषु १४४ सूक्तानि आयुर्वेदविषयान् विविच्य वदन्ति~। २४४ सूक्तानि राजनीतिप्रतिपादकं राजधर्मं प्रकथयन्ति~। ७५ सूक्तानि समाजव्यवस्थासम्बन्धिनः विषयान् निरूपयन्ति~। ८६ सूक्तानि आध्यात्मिकविद्याविवरणानि प्रदास्यन्ति~। अवशिष्टानि १८२ सूक्तानि अन्यान्योपयुक्तविषयान् प्रवदन्ति~। 

\section*{अथर्ववेदे आयुर्वेदविज्ञानम्}

अथर्ववेदस्य आयुर्वेदविज्ञानप्रतिपादकसूक्तेषु स्थिताः मन्त्राः मनुष्यदेहस्य रचनातन्त्राणि \-शरीरान्तःस्थिताङ्गप्रत्यङ्गानां स्वरूपाणि सम्यक् वर्णयन्ति~। ततः राजयक्ष्म-ज्वर-कुष्ठ-\break कासादीन् दैहिकमानसिकरोगान् तैः रोगैः समुत्पद्यमानान् क्लेशान् पश्चात् तत्तद्रोगाणां \break निवारणाय समुचितचिकित्साः चिकित्सायै स्वीकृतौषधनामानि च अथर्ववेदमन्त्राः विशदतया वर्णयन्ति~। तत्र केचन मन्त्राः एवं सन्ति “\textbf{त्रीणि च्छन्दांसि कवयो वि येतिरे \-पुरुरूपं दर्शतं विश्वचक्षणम्~। आपो वाता ओषधयस्तान्येकस्मिन् भुवन अर्पितानि}” \-(अथर्ववेदः, १८-१-१७)~। मन्त्रेऽस्मिन् अनामयावनाय जल-वायु-औषध-स्वरूपाणि \-त्रीणि वस्तूनि परमावश्यकानीत्युक्तम्~। जले विद्यमाना रोगनिरोधकशक्तिः अद्वितीयेति मन्त्राः वदन्ति –
\begin{verse}
\textbf{अप्स्वन्तरमृतमप्सु भेषजम्~। अपामुत प्रशस्तिभिरश्वा}\\
\textbf{वाजिनो गावो भवथ वाजिनीः~॥} (अथर्ववेदः, १-४-४)
\end{verse}
जले रोगनिवारणशक्तिरस्ति~। मन्त्रजलनिषेचनेन सेवनेन च मानवाः अश्वा इव वाजिनः वृषभा इव बलिष्ठाः भवन्ति इति मन्त्रोऽयं वदति~। पुनरपरो प्रसिद्धो मन्त्रः –

\begin{verse}
\textbf{आपो हि ष्ठा मयोभुवस्ता न ऊर्जे दधातन~। }\\
\textbf{महे रणाय चक्षसे~। यो वः शिवतमो रसतस्य}\\
\textbf{भाजयतेह नः~। उशतीरिव मातरः~॥} (अथर्ववेदः, १-५-१, २)
\end{verse}

आपः मनुष्येभ्यः जीविभ्यः तेजः बलम् आनन्दं च दास्यन्ति~। एताः आपः देहारोग्यम् अभिलषमाणाः मातर इव उपकुर्वन्तीति भावः~। 

रोगचिकित्सायै सूर्यकिरणचिकित्साऽपि उपदिष्टा -

\begin{verse}
\textbf{अनु सूर्यमुदयता हृद्योतो हरिमा च ते~। }\\
\textbf{गो रोहितस्य वर्णेन तेन त्वा परिदध्मसि~॥} (अथर्ववेदः, १-२२-१)
\end{verse}

सूर्यकिरणाः हृद्रोगं कामलारोगं नेत्ररोगं च परिहरन्तीति एते मन्त्राः वदन्ति~। 

\begin{verse}
\textbf{आयुर्दा अग्ने जरसं वृणानो घृतप्रतीको घृतपृष्ठो अग्ने~। }\\
\textbf{घृतं पीत्वा मधु चारु गव्यं पितेव पुत्रानभि रक्षतादिमम्~॥}\\
\hspace{6cm}(अथर्ववेदः, २-१३-१)
\end{verse}

अग्नौ घृतेन वयं होमं कुर्मः~। घृतं पीत्वा सः अग्निः सन्तुष्टस्सन् अस्मान् पुत्रान् पौत्रान् सर्वान् रक्षति~। तथैव गव्यं घृतं मधु च पीत्वा अस्मभ्यं देहपुष्टिम् उत्साहं च दास्यति~। अतः अस्माभिः देहपुष्ट्यै आरोग्याय च घृतं मधु च सेवनीयमिति मन्त्रोऽयं भणति~। ओषधिविज्ञानं तु सम्यक् मन्त्राः वर्णयन्ति~। दिङ्मात्रमेकमुदाहरणम् –

\begin{verse}
\textbf{क्षुधामारं तृष्णामारमगोतामनपत्यताम्~। }\\
\textbf{अपामार्ग त्वया वयं सर्वं तदप मृज्महे~॥} (अथर्ववेद, ४-१७-६)\\
\textbf{पिप्पली क्षिप्तभेषज्यू३तातिविद्धभेषजी~। }\\
\textbf{तां देवाः समकल्पयन्नियं जीवितवा अलम्~॥}\\
\hspace{5cm}(अथर्ववेदः, ६-१०९-१)\\
\textbf{दशवृक्ष मुञ्चेमां रक्षसो ग्राह्या अधि यैनं जग्राह पर्वसु~। }\\
\textbf{अथो एनं वनस्पते जीवानां लोकमुन्नय~॥} (अथर्ववेदः, २-९-१)
\end{verse}

इत्यादयो मन्त्राः अनपत्यतासन्धिवातादिरोगनिवृत्यर्थम् अपामार्ग-पिप्पली- दशमूलाद्योषधीः खादयेत्~। तेन रोगाः दूरीभवन्तीति कथयन्ति~। 

कनकं सर्वदा सर्वैर्धारणीयम्~। तेन देहारोग्यं सिध्यतीति सर्वजनवादः~। तं समर्थयति मन्त्रोऽयमथर्ववेदे -

\begin{verse}
\textbf{नैनं रक्षांसि न पिशाचा सहन्ते}\\
\textbf{देवानामोजः प्रथमजं ह्ये३तत्~। }\\
\textbf{यो बिभर्ति दाक्षायणं हिरण्यं}\\
\textbf{स जीवेषु कृणुते दीर्घमायुः~॥} (अथर्ववेदः, १-३५-२)
\end{verse}

स्वर्णस्य सर्वरोगहारित्वं दीर्घायुर्दायकत्वं च अत्रोक्तम्~। अत्र पिशाचाः रक्षांसि च व्याधिवाचकाः~। 

अथर्ववेदस्य क्रिमिघ्नसूक्ते उदयतः अस्तंगच्छतः च सूर्यस्य किरणाः रोगान् निवारयन्ति क्रिमींश्च हन्ति इति अनेके मन्त्राः वदन्ति~। तद्यथा –

\begin{verse}
\textbf{उद्यन्नादित्यः क्रिमीन् हन्तु निम्रोचन्}\\
\textbf{हन्तु रश्मिभिः~। ये अन्तः क्रिमयो गवि~॥} (अथर्ववेदः, २-३२-१)
\end{verse}

दीर्घायुषः लाभाय अग्नेस्तेजसः सूर्यरश्मेश्च सेवनमवश्यं करणीयमिति एको मन्त्रः वदत्येवम् –

\begin{verse}
\textbf{उत्क्रामतः पुरुषमाव पत्था मृत्योः षड्वीशमवमुद्यमानः~। }\\
\textbf{मा च्छित्था अस्माल्लोकादग्नेः सूर्यस्य संदृशः~॥} (अथर्ववेदः, ८-१-४)
\end{verse}

प्रतिदिनम् अग्निदर्शनेन सूर्यदर्शनेन च अग्नेः सूर्यस्य च तेजांसि अस्माकं मृत्युं दूरं स्थापयित्वा बहुकालं वयं जीवितुं सहकारीणि भवन्तीत्युक्तम्~। शुद्धवायुसेवनेन आरोग्यं दृढीभवति स्थिरीभवति चेति मन्त्रोऽयमेवं वदति –

\begin{verse}
\textbf{आ वात वाहि भेषजं विवात वाहि यद्रपः~। }\\
\textbf{त्वं हि विश्वभेषज देवानां दूत ईयसे~॥ इति~। } (अथर्ववेदः, ४-१३-३)
\end{verse}

एवमथर्ववेदस्य अनेके मन्त्राः आयुर्विज्ञानप्रदर्शकाः चिकित्सापराश्च दृश्यन्ते~। शरीररचनावर्णनपराः मन्त्राः अपि अथर्ववेदे सुनिबद्धास्सन्ति~। तत्र किञ्चिदुदाहरणम् –

\begin{verse}
\textbf{केन पार्ष्णी आभृते पूरुषस्य केन मांसं संभृतं केन गुल्फौ~। }\\
\textbf{केनाङ्गुलीः पेशनीः केन खानि केनोच्छ्लङ्खौ मध्यतः कः प्रतिष्ठाम्~॥}\\
\textbf{कस्मान्नु गुल्फावधरावकृण्वन्नष्ठीवन्तावुत्तरौ पूरुषस्य~। }\\
\textbf{जङ्घे निऋत्य त्यदधुः क्व स्विज्जानुनोः सन्धी क उ तच्चिकेत~॥}\\
\textbf{चतुष्टयं युज्यते संहितान्तं जानुभ्यामूर्ध्वं शिथिरं कबन्धम्~। }\\
\textbf{श्रोणी यदूरु क उ तज्जजान याभ्यां सन्धिं सुदृढं बभूव~॥}\\
\textbf{कति देवाः कतमे त आसन् उरो ग्रीवाश्चिक्युः पूरुषस्य~। }\\
\textbf{कति स्तनौ व्यदधुः कः कफोडौ कति स्कन्धान् कति पृष्टिरचिन्वन्~॥}\\
\textbf{को अस्य बाहू समभरद् वीर्यं करवादिति~। }\\
\textbf{अंसौ को अस्य तद्देवः कुसिन्धे अध्या दधौ~॥}\\
\textbf{कः सप्त खानि वि ततर्द शीर्षाणि कर्णाविमौ नासिके चक्षणी मुखम्~। }\\
\textbf{येषां पुरुत्र विजयस्य मह्यनि चतुष्पादो द्विपदो यन्ति यामम्~॥}\\
\textbf{हन्वोर्हि जिह्वामदधात् पुरूचीमधा महीमधि शिश्राय वाचम्~। }\\
\textbf{स आ वरीवर्ति भुवनेष्वन्तरो वसानः क उ तच्चिकेत~॥}\\
\textbf{मस्तिष्कमस्य यतमो ललाटं ककाटिकां प्रथमो यः कपालम्~। }\\
\textbf{चित्वा चित्यं हन्वोः पूरुषस्य दिवं रुरोह कतमः स वेदः~॥}\\
\hspace{4cm}(अथर्ववेदः १०-२-१ तः ८ मन्त्राः)
\end{verse}

एषु मन्त्रेषु अस्माकं शरीरे नखशिखापर्यन्तं स्थितानामवयवानां नामानि उक्तानि~। एतादृशावयवविशिष्टं शरीरं कः सृष्टवान्~। प्रतिदिनं देहे एतैरवयवैः नानाविधानि कार्याणि प्रचलन्ति सन्ति~। एतानि कार्याणि कर्तुमेतानवयवान् इन्द्रियाणि च कः प्रेरयति इति तु आश्चर्यम् इत्युक्तम्~। 


\section*{सामाजिकनीतिदर्शनम्}

मानवः खलु सामाजिको जीवी~। विना समाजं यः कोऽपि जीवी क्षणमेकमपि जीवितुं न शक्नोति~। अथर्ववेदस्य तृतीयकाण्डे त्रिंशत्तमे सूक्ते समाजाभिवृद्ध्यै आदर्शभूतानां कतिपयमौलिकगुणानामतिहृद्यं निरूपणं दरीदृश्यते तद्यथा –

\begin{verse}
\textbf{सहृदयं सांमनस्यमविद्वेषं कृणोमि वः~। }\\
\textbf{अन्योऽन्यमभि हर्यत वत्सं जातमिवाघ्न्या~॥}\\
\textbf{अनुव्रतः पितुः पुत्रो मात्रा भवतु संमनाः~। }\\
\textbf{जाया पत्ये मधुमतीं वाचं वदतु शान्तिवाम्~॥}\\
\textbf{मा भ्राता भ्रातरं द्विक्षन् मा स्वसारमुत स्वसा~। }\\
\textbf{सम्यञ्चः सव्रता भूत्वा वाचं वदत भद्रया~॥} (अथर्ववेदः, ३.३०.१-३)\\
\textbf{ज्यायस्वन्तश्चित्तिनो मा वि यौष्ट संराधयन्तः सधुराश्चरन्तः~। }\\
\textbf{अन्यो अन्यस्मै वल्गु वदन्त एत सध्रीचीनान्वः संमनसस्कृणोमि~॥}\\
\hspace{6cm}(अथर्ववेद, ३-३०-५)
\end{verse}

वयं सर्वे सर्वदा सहृदयाः समानमनस्काः द्वेषविहीनाः भवेम~। यथा सद्यः प्रसूता धेनुः स्ववत्समभिगच्छति तं न परित्यजति क्षणमेकं तथा वयं सर्वे मिथः एकीभूय एकभावेन कार्याणि कुर्याम~। पुत्राः पुत्र्यः पितरं मातरं च अनुवर्तेरन्~। मातापित्रोः प्रीतिं सर्वदा निदध्यासुः~। पत्नी भर्तारं प्रति धवः पत्नीं प्रति च हृदयावर्जकानि शान्तिदायकानि सन्तोषदायकानि वचनानि वदेत्~। भ्राता स्वभ्रातरः मा द्विष्यात्~। अपि तु स्निह्येत्~। स्वसा स्वस्वस्रूः प्रीत्या शान्त्या अनुगच्छेत्~। सर्वे परस्परं वल्गूनि वचांसि वदेयुः~। कनिष्ठाः ज्येष्ठेषु गौरवं प्रदेयुः~। अनुकूलपराः भवेयुः~। इत्येवंरीत्या एते मन्त्राः प्रेम-सहानुभूतित्याग-सौमनस्य-सहृदयतायाश्च आदर्शं सुष्ठु कथयन्ति~। एवमीदृशस्समाजः स्वर्गोपमः यदि स्यात् तर्हि भूतले एव स्वर्गोऽवतरेत् खलु~। 


\section*{अथर्ववेदे गृहसूक्तम्}

अथर्ववेदे कानिचन सूक्तानि आदर्शगृहचित्रणं कुर्वन्ति~। अतः एतानि गृहसूक्तानि इति कथितानि वर्तन्ते~। सर्वसमृद्धिशालिनि देशे आदर्शगृहाः कथं भवेयुरिति केचनसूक्तमन्त्राः मनोज्ञं वर्णयन्ति दिङ्मात्रं केचन मन्त्राः उच्यन्ते –

\begin{verse}
\textbf{सूनृतावन्तः सुभगा इरावन्तो हसामुदाः~। }\\
\textbf{अतृष्या अक्षुध्या स्त गृहा माऽस्मद् बिभीतन~॥}\\
\hspace{6cm}(अथर्ववेदः, ७-६२-६)\\
\textbf{येषामध्येति प्रवसन् येषु सौमनसो बहुः~। }\\
\textbf{गृहानुप ह्वयामहे ते नो जानन्त्वायतः~॥} (अथर्ववेदः, ७-६२-३)\\
\textbf{उपहूता इव गावः उपहूता अजावयः~। }\\
\textbf{अथो अन्नस्य कीलाल उपहूतो गृहेषु नः~॥} (अथर्ववेदः, ७-६२-५)\\
\textbf{उपहूता भूरिधनाः सखायः स्वादु संमुदः~। }\\
\textbf{अक्षुध्या अतृष्या स्त गुहा मास्मद् बिभीतन~॥}\\
\hspace{5cm}(अथर्ववेदः, ७-६२-४)
\end{verse}

अस्मदीयाः गृहाः सर्वदा सत्यमधुरसम्भाषणयुताः भवेयुः~। शोभनाः सौभाग्यवन्तः हासप्रमोदशालिनः क्षुत्पिपासारहिताः एते गृहाः स्युः~। अमी सर्वदा सर्वथा निर्भया वर्तेरन्~। किञ्च गृहेषु पुष्कलं धनं मित्राणि बन्धवः प्राणिनः तिष्ठेयुः~। स्वादुहर्षवर्धकवचनानि सर्वे वदेयुः~। गृहे सर्वदा सर्वे नीरोगाः निर्भयाः सन्तोषचित्ताः विलसेयुः~। सर्वदा गृहेषु अन्नपानादिसामग्र्यः दुग्धदधिघृतसमृद्धिभाजः भवन्तु गृहाः इत्याशयः अत्र संदृष्टः~। सूनृतावन्तः सुभगाः हसामुदाः सौमनसः भूरिधनाः इत्यादयः शब्दाः गृहाणामुत्कर्षद्योतकाः नितरां जनानामवधानं समाकर्षन्ति~। 

एवम् अथर्ववेदस्य अनेकानि सूक्तानि एतेषु सूक्तेषु स्थिताः नैकविधाः मन्त्राः समाज\-जनोपयोगिनः आदर्शविषयान् प्रतिपादयन्ति~। सर्वसमाजोपयोगिनां विषयाणामवगमनाय सवैरथर्ववेदः अवश्यमध्येतव्यः~। 

\articleend
}

