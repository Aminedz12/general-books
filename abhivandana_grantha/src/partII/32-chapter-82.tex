{\fontsize{15}{17}\selectfont
\presetvalues
\chapter{मन एव मनुष्याणां कारणं बन्धमोक्षयोः}

\begin{center}
\Authorline{वि~॥ सुरेश-हेगडे}
\smallskip
अतिथि-उपन्यासकः, संस्कृतविभागः \\
मैसूरुविश्वविद्यालयः, मैसूरु
\addrule
\end{center}

\begin{center}
मन एव मनुष्याणां कारणं बन्धमोक्षयोः~। \\
बन्धाय विषयासक्तं मुक्तं निर्विषयं स्मृतम्१~॥
\end{center}
संसारसागरं संतितीर्षूणां जनाः हितमाचरन्ति~। उपनिषद्वाणी एव विराजतेतराम्~। ऋषिहृदयगह्वरात् विनिर्गता च इयं वाणी संसारपारावारपारीणानां ज्ञानचक्षुषां महर्षीणाम् अनुभवम् अभिव्यनक्ति~। एतस्याः वाण्याः याथार्थ्यं सर्वैरपि दर्शनकारैः अङ्ग्यकारि~। सत्वरजतमोमयस्य मनसः स्वरूपं तावत् प्रथममस्माभिः अनुसन्धेयं भवति~। 
\begin{verse}
चित्तं तु चेतो हृदयं स्वान्तं हृन्मानसं मनः२~॥
\end{verse}
चित्तादि अपरपर्यायं मनः मत्तद्विरदवत् दुर्निग्रहं मनुजानाम्~। अतः उपनिषदि पूर्वतनश्लोके अस्य मनसः द्वैविध्यमाह\enginline{-} 
\begin{verse}
मनो हि द्विविधं प्रोक्तं शुद्धं चाशुद्धमेव च~। \\
अशुद्धं कामसंकल्पं शुद्धं कामविवर्जितम्३~॥ इति~। 
\end{verse}
शुद्धमशुद्धं चेति मनः द्विप्रकारकं कथितं विद्वद्भिः~। अशुद्धं कामसंकल्पम्~। कामो विषयमात्राभिलाषः~। तस्मिन् संकल्पः तदधिकरणत्वं यस्मात् तत् कामसंकल्पम् साभिलाषमित्यर्थः~। शुद्धं च अभिलाषमात्ररहितः~। मनसः अशुद्धत्वे को दोषः ? शुद्धत्वे च को लाभः ? इति प्रश्ने- आगता चेयं वाणी मन एव मनुष्याणामिति~।   

कमपि आनन्दम् अनुभवितुमिच्छुः जनः लोके इन्द्रियविषयेषु प्रवर्तते इति विदितचरमेव~। आजनेः जनिजुषां सुखप्रेप्सयैव प्रवृत्तयः प्रवर्तन्ते किल~। शिशुरपि यं कमपि जनमवलोक्य मोदते~। जनेन सुखमाप्तुम् इच्छति~। क्रीडार्थं वस्तूनि अपेक्षते~। तत्रापि सुखलाभेच्छैव कारणं भवति~। यदा अपेक्षितं सुखं न प्राप्नोति तदा वस्त्वन्तरं कमयति~। एवं हि रजोगुणवशात् मनः अर्थादर्थं मृगवत् प्रवर्तते~। मनसः ईदृशी प्रवृत्तिरेव संसाराय कल्पते~। अर्थात् मन एव जीवात्मानं जन्म मरण चक्रेऽस्मिन् बध्नाति~। अत एव उक्तं मन एव मनुष्याणां कारणं बन्धस्य इति~। 

मनः संकल्पविकल्पात्मकम्~। यथा मनः अनुरक्त्या लोकं बध्नाति, तथैव विरक्त्या मोचयति~। अतः एकमेव मनः अनुरक्तिविरक्तिभ्यां बन्धहेतुः मोक्षहेतुश्च भवति~। यदा मनः इन्द्रियप्रणालिकया बहिः निःसरति तदा बन्धकं भवति~। यदा च विरक्त्या आत्मन्येव रमते तदा मोचकं भवति~। तथा च मोक्षः आत्मज्ञानसाध्यः~। आत्मज्ञानं च शमदमादि षड्गुणसम्पन्नमनसा साध्यम्~। एवं च आत्मज्ञानरूपमोक्षस्य मन एव हेतुः इति सिद्धः~। तदेव उक्तं विषयासक्तं मनः बन्धाय, निरासक्तं मनः मोक्षाय इति~। 

अत्र मनुष्याणां बन्धमोक्षहेतुतया मनः प्रतिपादितम्~। किं तन्मनः ? इति चेत्  सुखाद्युपलब्धिसाधनमिन्द्रियं मनः इति तार्किकाः~। संकल्पविकल्पात्मकं मनः इति सांख्याः~। द्वैते विशिष्टाद्वैते च ज्ञानक्रिया उभयसाधनमिन्द्रियं मनः इति~। अद्वैतिनस्तु संकल्पविकल्पात्मकम् अन्तःकरणं मनः इति~। एवं सर्वत्र मनः किञ्चन कारणम् इति सम्प्रतिपन्नम्~। तत्र ज्ञानसौष्टवं क्रियासौष्टवं च मनसैव साध्यते~। अतः लोके क्रियासंसिद्धये ज्ञानमवलम्बन्ते~। अत्र कर्मकौशलम् अपेक्ष्यते~। कृत्वापि न किञ्चित् कृतं यथा स्यात् तथा वर्तेत~। तत्र मनः उपकरोति चेत् एतत् साध्यं भवति~। अतः एतादृश पाण्डित्यं मनसैव संसाध्यम्~। 

अत्र गीता अनुसन्धेया 	
\begin{verse}
यस्य सर्वे समारम्भाः कामसंकल्पवर्जिताः~। \\
ज्ञानाग्निदग्धकर्माणं तमाहुः पण्डितं बुधाः४~॥\\
त्यक्त्वाकर्मफलासंगं नित्यतृप्तो निराश्रयः~। \\
कर्मण्यभिप्रवृत्तोऽपि नैव किञ्चित्करोति सः५~॥
\end{verse}
अपि च उक्तम्
\begin{verse}
शारीरं केवलं कर्म कुर्वन्नाप्नोति किल्बिषम्६~। \\
समःसिद्धावसिद्धौ च कृत्वापि न निबध्यते७~॥ इति च~॥
\end{verse}
एवञ्च कामकूपात्मकं मनः मोक्षप्रतिबन्धकम्~। निष्कामं मनः बन्धप्रतिबन्धकम्~। अतः स्वेनैव आत्मना संस्कृतेन मनसा सम्यक् कामविशिष्टं मनोरूपशत्रुं ज्ञात्वा तन्निवृत्यै योगः आदरणीयः~।“येन च वीतरागः पूतो भवति, आत्मानं च विन्दते”~। अतः आत्मलाभरूपमोक्षस्य मनः कारणम्~। आत्महानिरूपबन्धस्यापि मनः कारणं भवति~। तस्मात् सर्वदापि सर्वथापि च अस्माभिः निर्विषयत्वे यत्नः करणीयः इति अवगन्तव्यम्~। तदुक्तम्- 
\begin{verse}
यतो निर्विषयस्यास्य मनसो मुक्तिरिष्यते~। \\
अतो निर्विषयं नित्यं मनः कार्यं मुमुक्षुणा९~॥
\end{verse}

यस्मात् विषयाभिलाषशून्यस्य साक्षिप्रत्यक्षस्य अस्य मनसः मोक्षः अविद्यादिबन्धनेभ्यः अङ्गीक्रियते, अस्मात् कारणात् सर्वदा मोक्षावता विषयाभिलाषशून्यं मनः कर्तव्यम् इति शम्~।  

\section*{परिशीलनग्रन्थाः }
\begin{enumerate}
\item उपनिषदां समुच्चयः, आनन्दाश्रम मुद्रणालयः~। 
\item श्रीमद्भगवद्गीताभाष्यम्, मद्रास्~। 
\item श्रीमद्भगवद्गीता, गोरखपुरम्~। 
\item ईशाद्यष्टोत्तरशतोपनिषदः, चौकाम्बा~। 
\item नामलिंगानुशासनम्, सुविद्याप्रकाशनम्~। 
\item पातञ्जलयोगदर्शनम्, रामकृष्णाश्रमः~। 
\end{enumerate}
\articleend
}
