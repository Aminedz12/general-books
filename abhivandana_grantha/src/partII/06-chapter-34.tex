{\fontsize{15}{17}\selectfont
\presetvalues
\chapter{अपभ्रंशानां वाचकत्वविचारः}

\begin{center}
\Authorline{वि~। श्रीधर-भट्ट. के.आर्}
\smallskip

व्याकरणप्राध्यापकः\\
श्रीमन्महारजसंस्कृतमहा-\\
पाठशाला, मैसूरु
\addrule
\end{center}

विदितचरमेवैतत् समेषां विदुषां यत् शरीरावयवेषु प्रधानाङ्गं मुखमिव षट्स्वङ्गेषु प्रधानं व्याकरणं पाणिन्युपज्ञम्~। अनेनैव च इष्टस्य साधुशब्दस्य साधनात्~। तदभ्यधायि दार्शनिकप्रवरेण भर्तृहरिणा-
\begin{verse}
साधुत्वज्ञानविषया सैषा व्याकरणस्मृतिः” इति~। (वा. प. २-२४२)
\end{verse}
प्रकृतिप्रत्ययव्युत्पादनद्वारा तदनुगमोपायाः शिष्यन्त इति शास्त्रत्वमप्यस्य निर्विवादम्~। तत्र तावत् साधुत्वबोधनाय समस्ति द्वयी विधा-पदानां साधुत्वान्वाख्यानपरा, अर्थानुसन्धान\-साहित्येन तद्बोधनपरा चेति~। 

पदसाधुत्वनिर्णयाय प्रक्रियाज्ञानमिवार्थप्रत्ययमप्यावश्यकं मन्वानैः वैयाकरणमूर्धन्यैः पदे वाक्ये च कः प्रत्ययः? कः प्रत्ययार्थः? तयोः प्राधान्यं कस्य ? वाक्ये पदार्थद्वयस्य कः सम्बन्धः? कथमन्वयस्तयोः ? कश्च वाक्यार्थः ? कथङ्कारं वा शाब्दबोधः ? शब्दस्य अर्थावबोधने का शक्तिः ? तत्स्वरूपं कीदृशम् ? इत्येतत्सर्वं सम्यग् विचार्य निरणैषि~।  

तत्र तावत् पदानाम् अर्थावबोधने यत्सामर्थ्यं, सैवपदशक्तिरित्यभिहिता~। सा च शक्तिः वाच्य-वाचकभावरूपसम्बन्धान्तरमेवेति नागेशः~। अयमेवसम्बन्धः तादात्म्यरूपः शब्दबोधजनकश्च~। सेयं शक्तिः किं साधुष्वेव शब्देषु उत अपभ्रंशेष्वपीति जिज्ञासैवास्य लेखनस्य विषयः~। 

प्रवृत्तिनिवृत्तिद्वारा मानवमुपकुर्वाणमिदं वाङ्मयं चतुर्विधमिति मनीषिप्रवरेण दण्डिनेत्थं प्रावाचि काव्यादर्शे~। 
\begin{verse}
तदेतद्वाङ्मयं भूयः संस्कृतं प्राकृतं तथा~। \\
अपभ्रंशश्च मिश्रञ्चेत्याहुरार्याश्चतुर्विधम्~॥ ( काव्यादर्शः)
\end{verse}
वैयाकरणेन भर्तृहरिणा तावत् शब्दानामपभ्रंशविषये अन्यथैव न्यगादि -
\begin{verse}
दैवीवाग्व्यवकीर्णेयमशक्तैरभिधातृभिः~। \\
अनित्यदर्शिनां त्वस्मिन् वादे बुद्धिविपर्ययः~॥ इति (वा. प. २-२५५)
\end{verse}
अनया कारिकया निर्णेतुं शक्यते यत् वेदकालादेव वाणीपदेनाभिधीयमानेयं संस्कृतभाषा शनैःशनैः अशक्तैः अभिधातृभिः सङ्कीर्णा कृता, यतः बुद्धिवैपरीत्यं जातम्~। कोशकारोऽपि अर्थमिममनुमनुते~। 
\begin{verse}
शक्तिवैकल्यप्रमादालसतादिभिः~। \\
अन्ययोच्चरिताः शब्दाः अपभ्रंशा इतीरिताः~॥ 
\end{verse}
स्वभावतो भ्रंशात् अपभंशत्वम्” इति (वाचस्पत्यम्)~। श्रूयते च प्राचीनोक्तिः यत् ‘शब्दप्रकृतिरपभ्रंश’ इति~। शब्दः प्रकृतिः यस्येति विग्रहात् शब्दविकृतिरेवापभ्रंशः इति निर्णेतुं शक्यते~। गैरित्युच्चारणीये असामर्थ्यात् अज्ञानाद्वा गावी, गोणी, गोपालिका इत्युच्चारणे त एवापभ्रंशाः~। तेषामपभ्रंशानामपि वाचकत्वमस्तीति वैयाकरणाः~। यतः शक्तिग्राहकशिरोमणेर्व्यवहारस्य तत्रापि समानत्वात्~। 

ननु व्यवहारदर्शनं न शक्तिग्राहकम्, उत्पत्तिकाले तद्बोधकशब्दानामभावात्~। शक्ति\-ग्राहकाभावे इष्ट्साधनताज्ञानस्याभावे न बालानां तिरश्चां च प्रवृत्तिः न स्यादिति चेत् उच्यते - प्रवृत्तिमवलोक्य पूर्वजन्मानुभूतशक्तिस्मरणं तत्र कल्प्यते~। तस्मादेव हि सद्योगृहीतजन्मनां पदार्थज्ञानसम्भवः~। पश्वादीनां जडप्रायाणां तु स्वस्वजात्यनुसारेणैव प्रतिनियता काचित् प्रतिभाबोध्यते~। जन्मकाले एव शक्तिज्ञानोत्पत्तेः वक्तुमशक्यत्वात् तेषामपि अनादि वासनैव प्रवृत्तौ निवृत्तौ वा कारणां भवति~। 

नैयायिकास्तु साधुष्वेव केवलेषु शक्तिमङ्गीकुर्वन्ति~। असाधुशब्दानां बोधविषये प्रतिपादयन्ति प्रकारत्रयम्~। व्युत्पन्नाः असाधुशब्दश्रवणे, तेषां साधुशब्दान् संस्मृत्य बोधमाप्नुवन्ति~। तेषां बोधः साधुशब्दस्मरणद्वारा~। अव्युत्पन्नास्तु असाधून् यदा श्रुण्वन्ति तदा ते शक्तिभ्रमात् बोधं प्राप्नुवन्ति~। भ्रमश्च परम्परया इति तेषामभिप्रायः~। 

तत्तु न~। साधुशब्दस्मरणं विनापि बोधस्य सर्वानुभवसिद्धत्वात्~। अपभ्रंशबोध्यार्थवाचकसाधुशब्दमजानतां बोधो न स्याद्यदि साधुशब्दस्मरणरूप- कारणमवश्यमपेक्षितं स्यात्~। परं तद्विनापि बोधो भवति तादृशानाम्~। अव्युत्पन्नानां भ्रमात् बोधः इत्यपि न साधु~। बाधज्ञानं विना जायमानस्य असन्दिग्धज्ञानस्य भ्रमत्वायोगात्~। यतः यस्मिन् ज्ञाने उत्तरकाले बाधो दृश्यते, तत्रैव भ्रमसम्भवः~। गगर्यादि शब्दानां श्रवणानन्तरम् “इमे गर्ग्यादि शब्दाः वाचकत्वाभाववन्तः” इति ज्ञानं कस्यापि न जायते~। अतः भ्रमत्वं वक्तुं न शक्यम्~। अपभ्रंशानामपि वाचकत्वादेव स्त्रीशूद्रबालानाम् उच्चरिते साधावर्थसंशये तदपभ्रंशेनार्थनिर्णयः~। तदभ्यधायि च हरिणा 
\begin{verse}
पारम्पर्यादपभ्रंशा विगुणेष्वभिधातृषु~। \\
प्रसिद्धिमागता येन तेषां साधुवाचकः~॥ वा. प. २-१५३ इति~। 
\end{verse}
परम्परया संस्कारात् च्युताः एते अपभ्रंशाः लोकप्रसिद्धत्वात्  अपभ्रंशैः स्मारिता भवन्ति, न् तु तत्र साधुत्वभ्रमात् बोधः इत्याशयः~। अतःअपभ्रंशानां वाचकत्वमङ्गीकार्यमेव~। 

अपि च अपभ्रंशानां वाचकत्वे एव मीमांसाशास्त्रस्य आर्यम्लेच्छाधिकारिणां सङ्गतं भवति~। तथा हि- आर्याः अर्थात् शिष्टाः यवशब्दं दीर्घशूके (यवस्योपरि उभयभागे स्थिताः कण्टकाः) प्रयुञ्जते~। म्लेच्छास्तु अर्थात् अपभ्रंशवक्तारः यवशब्दं प्रियङ्गौ (धान्यविशेषे) प्रयुज्यते~। तमेव च बुद्ध्यते~। अपशब्दानां वाचकत्वाभवे यवशब्दात् म्लेच्छानां यो बोधः सः साधुशब्दभ्रमादिति वक्तव्यं भवति~। सः बोधः शक्तिभ्रमजन्य एव स्यात्~। यथा च शुक्तौ रजतत्वभ्रमवान् पुरुषः तेन रजते न पात्रं वा भूषणं वा कर्तुं न प्रभवेत्, यतः तस्य ज्ञानं किमपि साधकं भवति, तथा म्लेच्छप्रसिद्धिः भ्रममूलिकावस्त्वसाधिका~। यत्र च वस्तुसाधकयोः द्वयोः समवधानं भवति, अत्र च म्लेच्छानां प्रसिद्धिर्भ्रममूलिका, तेन वस्त्वसाधिका~। एवञ्च बलाबलविचारप्रसक्त्यभावे न आर्यप्रसिद्धेर्बलवत्वबोधकार्थम्लेच्छाधिकरणमसङ्गतं स्यात्~। उभयोर्वाचकत्वे तु न दोषः~। 

नन्वेवम् अपभ्रंशानां वाचकत्वे साधुत्वापत्तिरिति चेत्, न~। शास्त्रेण पुण्यपापयोः नियमः क्रियते~। तदुक्तं हरिणा-
\begin{verse}
वाचकत्वाविशेषेऽपि नियमः पुण्यपापयोः~। (वा. प. ३-३-३०) इति~। 
\end{verse}
अत्र विषये भाष्यकारस्यापि सम्मतिरस्ति~। पस्पशाह्निके समानायामर्थावगतौ शब्द\-श्चापशब्दैश्च शस्त्रेण धर्मनियमः” इत्युक्तत्वात्~। एवञ्च साधुशब्द एवाभ्युदयकारी भवतीत्याशयः~। इति शम्~। 

\articleend
}
