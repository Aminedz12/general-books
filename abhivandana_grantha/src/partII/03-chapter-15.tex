\chapter{शाक्ततन्त्रम्}

\begin{center}
\Authorline{वि । सूर्यनरायणनागेन्द्रभट्टः}
\smallskip
हित्लळ्ळि, यल्लापुरम्\\
श्रीगुरुः शरणम्
\addrule
\end{center}

आकाशवद् विस्तृतम्, सागरवद् गभीरं संस्कृतसाहित्यं विहाय परमात्मानं पूर्णतया केनापि ज्ञातुं न शक्यम्। अत्र साहित्ये सन्ति श्रुति – स्मृति – तन्त्र - काव्यादीनि। सन्ति कण्ठपाठनिपुणा ब्राह्मणा ये श्रुतिशाखामेकां करामलकवत् साक्षात्कुर्वन्ति कारयन्ति वा। सन्त्येव केचिद् ये श्रुतिस्मृत्यादीनां तत्तत्कालोचितप्रयोगेण संस्कारादीन् निर्वर्तयन्ति। सन्त्येव बहवो ये कालिदासादीन् अनुवर्तमानाः कविमार्गपरायणाः कवयन्ति स्वाभिमतं किञ्चित्। एवं विद्यमाने परिवर्धमानेपि साहित्ये, संस्कृतसाहित्येव शिष्टं किञ्चिदेव, नष्टमेव बह्विति वदन्तः सन्ति सन्तः।

सम्प्रति बहुविस्तृतं शाक्ततन्त्रमेवावलक्ष्य विचारयामः, कियदवशिष्यते, कियच्च नश्यतेति। पशुपतिः सकलं भुवनं तत्तत्सिद्धिप्रसवपरतन्त्रैः चतुष्षष्ट्या तन्त्रैः अतिसन्धाय स्थितः। पुनस्त्वन्निर्बन्धादखिलपुरुषार्थैकघटनास्वतन्त्रं ते तन्त्रं क्षितितलमवातीतरद् इत्युक्तं सौन्दर्यलहर्याम् (श्लोकः 31) –
\begin{verse}
चतुष्षष्ट्या तन्त्रैः सकलमतिसन्धाय भुवनं \\
स्थितस्तत्तत्सिद्धिप्रसवपरतन्त्रैः पशुपतिः।\\
पुनस्त्वन्निर्बन्धादखिलपुरुषार्थैकघटना \\
स्वतन्त्रं ते तन्त्रं क्षितितलमवातीतरदिदम्।।
\end{verse}
अत्र चतुष्षष्ट्या – चतुष्षष्टिसङ्ख्याकैः महामायाशम्बरादिभिः, तन्त्रैः – सिद्धान्तैः इति व्याख्यायते। महामायाशम्बरादीनि तु नाम्नापि नाम्नायते इति नास्ति स्थितिः। नाम्नैवाम्नायते –
\begin{verse}
महामायाशम्बरञ्च योगिनीजालशम्बरम् । \\
तत्त्वशम्बरकञ्चैव भैरवाष्टकमेव च।।\\
बहुरूपाष्टकञ्चैव यामलाष्टकमेव च । \\
चन्द्रज्ञानं मालिनी च महासम्मोहनं तथा।।\\
वामजुष्टं महादेवं वातुलं वातुलोत्तरम् ।\\
हृद्भेदं तन्त्रभेदञ्च गुह्ययन्त्रञ्च कामिकम्।।\\
कलावादं कलासारं तथान्यत् कुब्जिकामतम् ।\\
तन्त्रोत्तरञ्च वीणाख्यं त्रोतलं त्रोतलोत्तरम्।।\\
पञ्चामृतं रूपभेदं भूतोड्डामरमेव च । \\
कुलसारं कुलोड्डीशं कुलचूडामणिं तथा।।\\
सर्वज्ञानोत्तरं देवि महाकालीमतं तथा । \\
अरुणेशं मोदिनीशं विकुण्ठेश्वरमेव च।।\\
पूर्वपश्चिमदक्षञ्च उत्तरञ्च निरुत्तरम् । \\
विमलं विमलोत्थञ्च देवीमतमतः परम्।।
\end{verse}
अत्र अरुणेशम् ...इत्यर्धे पाठभेदो वामकेश्वरतन्त्रे दृश्यते –
\begin{verse}
महालक्ष्मीमतञ्चैव सिद्धयोगेश्वरीमतम् । \\
कुरूपिकामतं देवरूपिकामतमेव च।।\\
सर्ववीरमतञ्चैव विमलामतमुत्तमम् ।\\
पूर्वपश्चिमदक्षञ्च उत्तरञ्च निरुत्तरम्।।\\
तन्त्रं वैशेषिकं ज्ञानं वीरवलि तथा परम् ।\\ 
अरुणेशं मोहिनीशं विशुद्धेश्वरमेव च।।
\end{verse}
तथा च नामसङ्ग्रहेपि पाठभेददर्शनेनात्र कियदवशिष्टमिति सम्यगभ्यूहितुं शक्यम्। एषा कथा न केवलं शाक्तागमे। किन्तु शैवागमे पाञ्चरात्रागमे जैनागमे बौद्धागमे च समानमिदं विस्मरणम्।

लक्ष्मीधरेण सौन्दर्यलहरीव्याख्यायां चतुष्षष्टितन्त्राणां विषये किञ्चिद् विवृतम्। ततो नाममात्रं सङ्गृह्यते –
(1) महामायाशम्बरतन्त्रम् (2) योगिनीजालशम्बरम् (3) तत्त्वशम्बरम् (4) सिद्धभैरवतन्त्रम् (5) बटुकभैरवतन्त्रम् (6) कङ्कालभैरवतन्त्रम् (7) कालभैरवतन्त्रम् (8) कालाग्निभैरवतन्त्रम् (9) योगिनीभैरवतन्त्रम् (10) महाभैरवतन्त्रम् (11) शक्तिभैरवतन्त्रम् (12) ब्राह्मीतन्त्रम् (13) माहेश्वरीतन्त्रम् (14) कौमारीतन्त्रम् (15) वैष्णवीतन्त्रम् (16) वाराहीतन्त्रम् (17) माहेन्द्रीतन्त्रम् (18) चामुण्डातन्त्रम् (19) शिवदूतीतन्त्रम् (20)ब्रह्मयामलतन्त्रम् (21) विष्णुयामलतन्त्रम् (22) रुद्रयामलतन्त्रम् (23) लक्ष्मीयामलतन्त्रम् (24) उमायामलतन्त्रम् (25) स्कन्दयामलतन्त्रम् (26) गणेशयामलतन्त्रम् (27) जयद्रथयामलतन्त्रम् (28) चन्द्रज्ञानतन्त्रम् (29) मालिनीतन्त्रम् (30) महासम्मोहनतन्त्रम् (31) वामजुष्टतन्त्रम् (32) महादेवतन्त्रम् (33) वातुलतन्त्रम् (34) वातुलोत्तरतन्त्रम् (35) हृद्भेदतन्त्रम् (36) तन्त्रभेदतन्त्रम् (37) गुह्यतन्त्रम् (38) कामिकतन्त्रम् (39) कलावादतन्त्रम् (40) कलासारतन्त्रम् (41) कुब्जिकामततन्त्रम् (42) तन्त्रोत्तरतन्त्रम् (43) वीणाख्यतन्त्रम् (44) त्रोतलतन्त्रम् (45) त्रोतलोत्तरतन्त्रम् (46) पञ्चामृततन्त्रम् (47) रूपभेदतन्त्रम् (48) भूतोड्डामरतन्त्रम् (49) कुलसारतन्त्रम् (50) कुलोड्डीशतन्त्रम् (51) कुलचूडामणितन्त्रम् (52) सर्वज्ञानोत्तरतन्त्रम् (53) महाकालीमततन्त्रम् (54)अरुणेशतन्त्रम् (55) मोदि(हि)नीशतन्त्रम् (56) विकुण्ठेश्वरतन्त्रम् (57)पूर्वतन्त्रम् (58) पश्चिमतन्त्रम् (59) दक्षतन्त्रम् (60) उत्तरतन्त्रम् (61) निरुत्तरतन्त्रम् (62) विमलतन्त्रम् (63) विमलोत्थतन्त्रम् (64) देवीमततन्त्रम्

पाठान्तरानुसारेण कानिचिद् अन्यानि तन्त्राणि चतुष्षष्टिसङ्ख्याकेषु तन्त्रेष्वेव अन्तर्भूतानि -

(1) महालक्ष्मीमततन्त्रम् 	(2) 	सिद्धयोगीश्वरीमततन्त्रम् 	(3) 	कुरूपिकामततन्त्रम् 	(4) 	देवरूपिकामततन्त्रम् 	(5) 	सर्ववीरमततन्त्रम् 	(6) 	वैशेषिकतन्त्रम्

एतानि सर्वाणि तन्त्राणि कापालादीनि जनानां वञ्चनार्थं पशुपतिना निर्मितानि इति सौन्दर्यलहर्यां स्पष्टम्। अत्रान्तर्भूतं चन्द्रज्ञानं वातुलं वातुलोत्तरं कामिकं इत्यादि तन्त्रं शैवागमेष्वन्तर्गतं, येषां प्रामाण्यं कर्षणान्तप्रतिष्ठान्तविधिदर्शकतया तत्त्वनिरूपकतया वा दाक्षिणात्यशैवैः वीरशैवैरन्यैश्च स्वीक्रियते। कामिकादीनि वातुलान्तानि दशशैवभेदेन अष्टादशरौद्रभेदेन च द्विविधानि, अष्टाविंशतितन्त्राणि सदाशिवस्य पञ्चमुखप्रसूतानि दाक्षिणात्यशैवानां प्रसिद्धानि। तेषां सन्ति परश्शतानि उपतन्त्राणि। तानि शिवद्वैतपराणि कानिचित् अंशतः उपलभ्यन्ते। कानिचिद् नाममात्रावशेषाणि। तानि विहाय शिवद्वैताद्वैतबोधकानि शिवाद्वैतबोधकानि च तन्त्राणि तावन्त्येव सन्ति इत्यवगम्यते।

अष्टशाक्ततन्त्राणि शिष्टाचारगृहीतानि – (1) चन्द्रकलातन्त्रम् (2) ज्योत्स्नावतीतन्त्रम् (3) कलानिधितन्त्रम् (4) कुलार्णवतन्त्रम् (5) कुलेश्वरीतन्त्रम् (6) भुवनेश्वरीतन्त्रम् (7) बार्हस्पत्यतन्त्रम् (8) दूर्वासमततन्त्रम्

शुभागमतन्त्रपञ्चकमपि शाक्तसमये विद्यते – (1) वासिष्ठसंहिता (2) सनकसंहिता (3) शुकसंहिता (4) सनन्दनसंहिता (5) सनत्कुमारसंहिता

अनेन कृत्स्नं तन्त्रसाहित्यं न कथमपि कटाक्षितम्। न केनापि पूर्णतया तस्य सङ्गहः शक्यते कर्तुम्। केनचित् प्रबलेन कारणेन भारतीयैः सर्वं तन्त्रसाहित्यं तिरस्कृतमिति प्रतीयते। तथापि तान्त्रिकसाहित्यं तत्प्रभावश्च सर्वत्र भारते प्रचुरतयोपलभ्यते। सम्प्रति भारतं पूर्णतया न क्वापि शुद्धं वैदिकं स्मार्तं तान्त्रिकं वा द्रष्टुं शक्यते। सर्वत्र सर्वस्य प्रभावः दूरदर्शनादिना परिवर्धितः। व्योमयानस्थाने सङ्गीतमिव, शकटस्थाने चित्रमिव सर्वं सर्वत्र साक्षात्क्रियेत।

एवं सत्यपि नर्तकादिवत् श्रोत्रियः वैदिकः पौराणिकः आगमिकः तान्त्रिकः इति व्यवहारो विद्यते। अयं ऋग्वेदी यजुर्वेदी सामगः चतुर्वेदी वैयाकरणः मीमांसकः तार्किकः वेदान्ती शैवः वैष्णव इत्यादिव्यवहारो बहुत्र विधीयते। किन्तु अयं शाक्त इति कथनं नास्तिप्रायम्। अनेन ज्ञायते, शक्तिपीठानां द्विपञ्चाशत्सङ्ख्याकानां भारते सद्भावेपि पूर्ववत् शाक्ता न सन्ति इति। सर्वदर्शनसङ्गहे वेदान्तदर्शनानीव, नकुलीशादीनां शैवदर्शनानां सङ्ग्रहोस्ति। शाक्तदर्शनसङ्ग्रहस्याभावे कारणं स्मार्तानामेव शाक्तत्वेन सर्वत्र वर्तमानत्वम् इति वक्तव्यम्। अत एव श्रीशङ्करभगवत्पादैः सूत्रभाष्यस्य द्वितीयद्वितीये शैवानां भागवतानाञ्च निराकरणं कृतम्। शाक्तानां तु नास्ति निराकरणमित्येव न, सौन्दर्यलहर्यादि शाक्तसाहित्यमेव परिवर्धितं दृश्यते। अस्मिन् विषये  संशोधनार्थं प्रवृत्ताश्छात्राः पराक्रमन्तामित्युपरम्यते।

\articleend
