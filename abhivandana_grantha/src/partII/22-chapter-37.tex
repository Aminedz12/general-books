{\fontsize{15}{17}\selectfont
\presetvalues
\chapter{ईश्वरप्रामाण्यम्}

\begin{center}
\Authorline{वि~। जैरामभट्टः, उञ्चळ्ळी}
\smallskip
बेङ्गळूरु
\addrule
\end{center}
प्रत्यक्षैकप्रमाणवादिनो नास्तिकाः अप्रत्यक्षे दिष्टे न विश्वस्ताः~। आस्तिकास्तु संसारमहीरुहस्य कारणं किञ्चन दिष्टम् अस्तीत्यत्र दृढं विश्वस्ताः~। तच्च दिष्टम् ईश्वर इति नाम्नापि व्यपदिश्यते~। स ईश्वरः अस्तीति परमतनिराकरणपूर्वकं न्यायशास्त्रे प्रतिष्ठाप्यते~। 

ईश्वरास्तित्वे किम् मानम् ? न च तत्र प्रत्यक्षं प्रमाणम्~। अरूपि द्रव्यत्वात्~। बहिरिन्द्रियप्रत्यक्षे तु उद्भूतरूपस्यैव कारणत्वात्~। सः मानसप्रत्यक्षविषयोऽपि न भवति~। स्वात्मभिन्नद्रव्यत्वात्~। परेण मनसा परात्मनः आत्माद्यत्वात्~। नानुमानं प्रमाणम्~। ईश्वरस्य केनचिल्लिङ्गेन सहचारदर्शनाभावात् व्याप्तिज्ञानं न जायते~। ईश्वरसदृशस्य कस्यचिदभावेन सादृश्यज्ञानस्य अजननात् उपमानमपि न प्रमाणम्~। शब्दोऽपि न प्रमाणम्~। श्रुतीनां ईश्वरोक्तत्वेन प्रमाण्यात् अत्र तस्यैव सन्दिग्धतया तदुक्तवेदस्य प्रामाण्येपि सन्देहात्~। एवं न केनापि प्रमाणेन ईश्वरास्तित्वं प्रतिष्ठापयितुं शक्यते इति आक्षिप्यते~। 

ईश्वरस्य अस्तित्वे न्यायाचार्यः उदयनः अनुमानं प्रमाणम् उपस्थापयति~। तथा च उदयनोक्तिः-
\begin{verse}
\textbf{कार्यायोजनधृत्यादेः पदात् प्रत्ययतः श्रुतेः~। \\
वाक्यात् सङ्ख्याविशेषाच्च साध्यो विश्वविदव्ययः}~॥ इति~॥
\end{verse}
तदत्र कार्यत्वेन हेतुना ईश्वरास्तित्वमत्र परामृश्यते~। तथा च अनुमानस्वरूपम्- “\textbf{क्षित्यङ्कुरादिकं सकर्तृकं कार्यत्वात् घटवत्}” इति~। अत्र क्षित्यङ्कुरादिकमिति पक्षनिर्देशः~। तत्र क्षितिः, अङ्कुरः, आदिः इति पदत्रयं वर्तते~। क्षितिस्तु स्थूला पृथिवी~। अङ्कुरः द्व्यणुकम्~। आदिना तद्घटः ग्राह्यः~। अत्र क्षितित्वम्, अङ्कुरत्वं, घटत्वं चेति न पक्षतावच्छेदकम्~। किन्तु स्वरूपसम्बन्धात्मकं कार्यत्वं पक्षतावच्छेदकम्~। तादृशः पक्षतावच्छेदकावच्छिन्नः पक्षः क्षित्यङ्कुरादिकम्~। 

सकर्तृकमिति साध्यनिर्देशः~। सकर्तृकत्वं साध्यम्~। सकर्तृकत्वं च कर्तृजन्यत्वम्~। कर्त्रा जन्यं कर्तृजन्यम्~। कर्तृत्वं च उपादानगोचर-अपरोक्षज्ञानचिकीर्षाकृतिमत्वम्~। उपादानं नाम समवायिकारणम्~। अपरोक्षं नाम प्रत्यक्षम्~। कर्तुमिच्छा चिकीर्षा~। अस्य घटकर्तरि कुलाले एवम् अन्वयः~। यथा घटस्य कर्ता कुलालः~। अत्र घटसमवायिकारणं कपालद्वयम्~। तद्गोचरप्रत्यक्षज्ञानचिकीर्षा कृतिमत्वं च कपालद्वयविषयकं प्रत्यक्षज्ञानम्, घटो मत्कृतिसाध्यो भवेदितीच्छा, तद्विषयको यत्नः~। तादृशज्ञानचिकीर्षाकृतिमान् कुलालः भवति~। 

तथैव क्षित्यङ्कुरादीनां समवायिकारणं परमाण्वादयः~। तद्विषयकप्रत्यक्षज्ञानचिकिर्षा\-कृतिमत्वं च परमाण्वादिविषयकं प्रत्यक्षज्ञानम्, क्षित्यङ्कुरादिकं मत्कृतिसाध्यं भवेदितीच्छा, तद्विषयकः यत्नः~। तादृशज्ञानचिकीर्षाकृतिमान् ईश्वरः भवति~। 

कार्यत्वात् इति हेतुनिर्देशः~। कार्यत्वं हेतुः~। तच्च कार्यत्वं प्रागभावप्रतियोगित्वरूपम्~। 

\section*{अत्र आक्षेपः} 

प्रागभावप्रतियोगिनि या प्रतियोगिता अस्ति, साऽपि स्वरूपसम्बन्धेनैव वर्तते~। अतः कार्यत्व\-रूपो हेतुः स्वरूपसम्बन्धात्मकः~। पक्षतावच्छेदकं कार्यत्वमपि स्वरूपसम्बन्धात्मकमेव~। अतः पक्षतावच्छेदकहेत्वोरैक्यं भवतीति आपादनम्~। 

\section*{समाधानम्} 

संयोगेन घटाभावः अन्यः~। समवायेन घटाभावः अन्यः~। आद्ये प्रतियोगिता संयोगसम्बन्धा\-वच्छिना~। द्वितीये तु सा समवायसम्बन्धावच्छिना~। प्रतियोगिताभेदेनैव अभावभेदः\break समर्थनीयः~। प्रतियोगिता प्रतियोगिनि स्वरूपेण न तिष्ठति~। तस्यात् पृथक् अस्तित्वं वर्तते~। अत एव प्रतियोगितायाः पदार्थान्तरत्वमङ्गीक्रियते~। एवं पक्षतावच्छेदकहेत्वोः ऐक्यापत्तिः निराक्रियते~। 

\section*{द्वितीयं समाधानम्} 

कार्यत्वरूपं हेतुं विहाय, प्रतियोगितासम्बन्धावच्छिन्नप्रागभावस्य हेतुस्थाने प्रतिष्ठापनीयम्~। तदा न पक्षतावच्छेदकहेत्वोः ऐक्यम्~। तथा च क्षित्यङ्कुरादिकं सकर्तृकं प्रतियोगितासम्बन्धावच्छिन्नप्रागभाववत्वात् इति अनुमानस्वरूपं वक्तव्यम्~। अथवा प्रागभावप्रतियोगित्वरूपस्य कार्यत्वस्य पक्षतावच्छेदकत्वेन, सत्तावत्वे सति ध्वंसप्रतियोगित्वात्मकस्य कार्यत्वस्य हेतुत्वेन स्वीकर्तव्यम्~। 

एवं क्षित्यङ्कुरादीनां समवायिकारणविषयकप्रत्यक्षज्ञानचिकीर्षा कृतिमान् कश्चन कर्ता अनुमीयते न्यायनये~। स ईश्वर एव भवितुमर्हति~। तादृशकृतिमत्वं अस्मदादीनां तु न सम्भवति~। 

\articleend
}
