{\fontsize{15}{17}\selectfont
\presetvalues
\chapter{न्यायमते प्रामाण्यवादः}

\begin{center}
\Authorline{\textbf{डा~॥ राचोटि देवरु}}
\smallskip
अतिथि उपन्यासकः\\
संस्कृताध्यायनविभागः\\
मानसगङ्गोत्री,\\ 
मैसूरु ५७०००५
\addrule
\end{center}
प्रपञ्चेऽस्मिन् भारतं ज्ञानेन प्रज्वलितम्~। तत्र कारणन्तु भारतीयदर्शनशास्त्रादिज्ञानशाखा एव~। सा शाखा बहुमौल्ययुता~। भारतीयानां जीवतत्वचिन्तनं अत्यमूल्यम्~। दर्शनम् नाम \textbf{“येन दृश्यते तद्दर्शनमिति”} इत्यनेन यत् सर्वविषयकं ज्ञानं स्वगर्भे प्रतिष्ठापितं तद्दर्शनमिति व्याचक्षते~। भारतीयमहर्षिभिः स्वीय तपो द्वारा जनानां अज्ञानरूपतमो दूरीकृत्य सूर्यवत्, मनुष्यस्य बुद्धिरूपपुष्पं विकासयन् सूर्यकिरणवच्च भारतीयदर्शनशास्त्राणि प्रवर्तितानि~। 

भारतीयदर्शनं तावद् द्विधा विभज्यते आस्तिकं नास्तिकञ्चेति~। यद्दर्शनं वेदप्रमाण\-मङ्गीक्रियते तदास्तिकमिति~। वेदविरोधिनन्तावत् \textbf{“नास्तिको वेदनिन्दकः”} इत्युक्तत्वात् नास्तिकदर्शनमिति~। सांख्यमीमांसान्यायश्चेति आस्तिकदर्शनानि~। चार्वकबौद्धजैनानि नास्तिकदर्शनानि~। तेषु आस्तिकदर्शनेषु अन्यतमं न्यायवैशेषिकदर्शनम्~। ते द्वे दर्शने सिद्धान्त साम्यात् समानतन्त्रदर्शनमिति प्रसिद्धम्~। न्यायवैशेषिकदर्शनं विक्रमपूर्वादेव आगतम् इत्यत्र नास्ति संशयः~। तथापि वैशेषिकन्तु बुद्धात् पूर्वादेवागतमिति विद्वद्भिरभिप्रायः~। 

\textbf{“काणादं पाणिनीयञ्च सर्वशास्त्रॊऽपकारकम्”} इति लोकोक्तेः प्रसिद्धत्वात् भारतदेशे उत्पन्नानां सर्वदर्शनानां महदुपकारभूतं दर्शनं पाणीनिकृतं व्याकरणं, कणादकृतं वैशेषिकमिति लोके प्रसिद्धिः~। अतः न्यायवैशेषिकदर्शनं सर्वदर्शनानां मेरुदण्डप्रायः इति ज्ञानिनामभिप्रायः~। न्यायवैशेषिकस्य नामान्तरमित्थं वर्तते तर्कशास्त्रं, हॆतुविद्या, प्रमाणशास्त्रं आन्विक्षिकी इत्यादि नान्मा प्रसिद्धः~। 

\section*{न्यायवैशेषिकशास्त्रे प्रधानतत्वविचारः}

आस्तिक दार्शनिकसम्प्रदायेषु अत्यन्तं प्राचीनतमं दर्शनं सांख्यदर्शनम्, तदनन्तरं योगः,\break वैशेषिकं, पूर्वमीमांसा, वेदान्तः न्यायश्चेति दर्शनक्रमः दृश्यते~। न्यायशास्त्रस्य प्रमाणभूतः ग्रन्थः गौतमकृत न्यायसूत्रम्~। तत्तु पञ्चाध्यायसमन्वितम्~। एकैक अपि अध्यायः आन्हिकद्वययुक्तः अस्ति, आहत्य अष्टाविंशत्युत्तरपञ्चशतसूत्रात्मकोऽयम् ग्रन्थः~। वैशेषिकं तावत् दशाध्यायसमन्वितं एकैकस्यापि अध्यायस्य दौ आह्निकौ स्तः~। अत्र तर्क(वैशेषिक)शास्त्रे प्रधानतया धर्मधर्मिभेदमङ्गीकृत्य सप्तपदार्थाः स्वीक्रियन्ते~। अत्र धर्मो नाम वस्तुनि दृश्यमानः असाधारणधर्मः~। धर्मी नाम तदाधारभूतवस्तुरूपः~। वैशेषिकाः तावत् बाह्यार्थसत्यवादिनः~। अतः एतत्त् शास्त्रं मनुष्याणां साधारणगुणमाधारीकृत्य प्रवर्तते~। 

अत्र वैशेषिकाः भावाभावरूपेण सप्तपदार्थान् निरूपयन्ति~। तानि तु \textbf{“द्रव्यगुणकर्म\-सामान्यविशेषसमावायाभावाः”}1 इति सप्त~। नैय्यायिकाः तावत् षॊडशपदार्थान् अङ्गीकुर्वन्ति~। तानि -

\textbf{“प्रमाणप्रमॆयसंशयप्रयॊजनदृष्टान्तावयवतर्कनिर्णयवादजल्पवितण्डहेत्वा\-\break भासच्छलजाति निग्रहस्थानानि”}2 इति~। भाष्य़कारेण वात्स्यायनेन भाष्ये एवं उच्यते \textbf{“प्रमाणतॊऽर्थप्रतिपत्तौ प्रवृत्तिसामर्थ्यादर्थववत्प्रमाणम्”}3 इति~। अनेन ज्ञायते यत् इदं प्रमाणशास्त्रमिति~। अपिच प्रमाणमीमांसादर्शनमिति फलितो अर्थः समवगम्यते~। 

\section*{न्यायमते ज्ञानस्वरूपम् –}

ज्ञानं विभजते अनुभवस्मृतिभेदेन द्विविधम् इति~। \textbf{“सर्वव्यवहारहेतुर्गुणो बुद्धिर्ज्ञानम्~। सा द्विधा स्मृतिरनुभवश्चेति”}4 इत्यनेन~। अत्र सर्वव्यवहारं प्रति ज्ञानं कारणम् अपिच जानाति, इच्छति यतते इत्यादि प्रवृत्तिसामान्यं प्रति च ज्ञानम् कारणम् इति निगदितोऽर्थः~। स्मृतिस्तु \textbf{“संस्कारमात्रजन्यं ज्ञानंस्मृतिः”}5 इत्युक्तत्वात् संस्कारादिना जातं ज्ञानं स्मृतिः इति~। तल्लक्षणमित्थं भवति तथाच– “संस्कारेतरजन्यत्वाभावविशिष्ट-संस्कारजन्यत्व- विशिष्टज्ञानत्वं” स्मृतेर्लक्षणम्~। स्मृतिभिन्नं ज्ञानमनुभवः सः द्विविधः यथार्थाऽयथार्थभेदेन विभज्यते~। तल्लक्षणमुक्तम् अन्नंभट्टेन- \textbf{“तद्वति तत्प्रकारकॊऽनुभवो यथार्थः~। सैव प्रमेत्युच्यते~। यथा रजते इदं रजतमिति ज्ञानम्6”}~। अपि च अयथार्थ लक्षणमित्थं- \textbf{“तदभाववति तत्प्रकारकॊऽनुभवोऽयथार्थः~। सैवप्रमॆत्युच्यते~। यथा शुक्तौ इदं रजतमिति ज्ञानम्7”}~। यथार्थानुभवो नाम तन्निष्ठविशेष्यतानिरूपकत्वेसति तन्निष्ठ प्रकरतानिरूपकत्वे सत्यनुभवम्~। स एव प्रमा इत्यप्युच्यते~। यथा रजते इदं रजतमिति ज्ञानम्~। यथार्थानुभवः चतुर्विधः प्रत्यक्षानुमिति उपमितिशाब्दाः इति~। तत्करणमपि चतुर्विधम्~। तानि प्रत्यक्षानुमानोपामानशब्दभेदात्~। प्रत्यक्षज्ञानं प्रति इन्द्रियार्थसन्निकर्षम् आवश्यकः~। अत एव उच्यते \textbf{“इन्द्रियार्थसन्निकर्षोत्पन्नं ज्ञानमव्यपदेश्यमव्यभिचारिव्यसायात्मकं प्रत्यक्षम्”}8 इति~। अनॆन ज्ञायते यत् चक्षुरादीन्द्रियेण घटपटादिवस्तूनां सन्निकर्षोत्पन्नं भ्रमरहितं ज्ञानं प्रत्यक्ष प्रमा~। प्रत्यक्षं तज्जनकम् इन्द्रियं प्रमाणं भवति~। 

\textbf{“अनुमितिकरणमनुमानम्”} इत्युक्तमनुमानलक्षणम्~। अनुमानं द्विविधं स्वार्थानुमानम् परार्थानुमानञ्चेति~। स्वानुभावक हेतुकं स्वार्थानुमानं \textbf{“स्वार्थं स्वानुमितिहेतु”}रित्युक्तत्वात्~। \textbf{“यत्तु स्वयमेव धूमादग्निमनुमाय परप्रत्ययार्थं पञ्चावयव वाक्यं प्रयुंक्ते तत्परार्थानुमानं”9} भवति~। इत्थञ्च पञ्चावयवाक्यसमुत्पन्नं व्याप्तिविशिष्टं परार्थ साधकम् अनुमानं परार्थानुमानम्~। तदेव सूत्रकारैरुच्यते \textbf{“अथ तत्पूर्वकंत्रिविधमनुमानम्- पूर्ववत्शेषवत्सामन्यतोदृष्टञ्च”10} इति~। उपमानं सादृश्यरूपज्ञानमुपानमिति~। तत्र सूत्रकारैः \textbf{“प्रसिद्धसाधर्म्यात् सध्यसाधनमुपमानम्”11} इति~। प्रसिद्धवस्तुसादृशेन साध्यमानवस्तुसाधकमुपमानम् इति प्रतिपादितम्~। शब्दः \textbf{“अप्तोऽपदॆशः शब्दः”12} इति सूत्रकारेणोक्तत्वात् आप्तोऽपदॆश एव शब्दः भवति~। आप्तस्तु यथार्थवक्ता इति~। अनेन ज्ञायते यत् सर्वैरुक्तशब्दः न शब्दप्रमाणं भवति अपि तु आप्तोक्त शब्द एव शब्दप्रमाणमिति~। इदं प्रमाणचतुष्टयं यथार्थानुभवं प्रति करणं भवति~। अपरस्तु अयथार्थानुभवः~। सः त्रिविधः संशयविपर्यय(भ्रम)तर्कश्चेति~। प्रथमः संशयः~। पुरुषः गच्छन् सन् दूरस्थं वस्तु दृष्ट्वा तस्य ज्ञानं जायते, अयं स्थाणुर्वा\break पुरुषोवा इति~। तच्च ज्ञानं संशय इति~। तल्लक्षणमुच्यते अन्नंभट्टेन यथा \textbf{“एकस्मिन् धर्मिणि विरुद्धनाना धर्मवैशिष्ट्यावगाहि ज्ञानं संशय”13} इति~। द्वितीयः विपर्ययः- शुक्तिं दृष्ट्वा रजतमिति यदि ज्ञानं सः भ्रमः~। तृतीयः तर्कः – \textbf{“व्याप्यारोपेण व्यापकारोपस्तर्कः”14} इति~। पर्वते धूमं दृष्ट्वा वह्निं तत्र अस्तीति साध्यते यथा यदि वन्हिर्न स्यात् तर्हि धूमोऽपि न स्यात् इत्यात्मक अनिश्चयं उहः तर्क इति~। इत्थम् अयथार्थानुभवः व्याख्यातः~। 

स्मृतिज्ञानम् - ज्ञातविषयकं ज्ञानम् अथवा संस्कारादिना जनितम् ज्ञानम् स्मृतिः~। अत एवोच्यते भाष्यकारैः \textbf{“किञ्चित् क्षिप्रं स्मर्यते किञ्चित् चिरेण”15} अपिच न्याय मंजर्यां-
\begin{verse}
\textbf{क्वचिन्निद्रा क्वचिच्चिन्ता धातूनां विकृतिः क्वचित्~। \\
अलक्षमाणे तद्धातौ अदृष्टं स्मृतिकारकम्16}~॥इति~॥
\end{verse}
पूर्वतनम् अनुभवविषयस्मरणं शीघ्रमेव भवति~। अथवा विलम्बेन भवति सा स्मृतिः~। कदाचित् निद्रा क्वचित् चिन्ता, धातूनां विकृतिश्च एतद् सर्वं स्मृतिरूप ज्ञानस्य हेतुः भवति~। 

\section*{प्रामाण्यवादः}

जगदिदं ज्ञानाधीनम्~। लौकिकधार्मिकादि सर्वोऽपि व्यवहारः ज्ञानेनैव प्रचलति~। परन्तु तद्ज्ञानं न एकविधं भवति~। यथा एकस्मिन् समये यथार्थज्ञानं भवति अपरस्मिन् समये अयथार्थज्ञानञ्च भवति~। यदि यथार्थ ज्ञानं भवेत् तद् प्रमात्मकं ज्ञानम्~। यद्ययथार्थकं ज्ञानं तदप्रमात्मकंज्ञानमिति(भ्रमात्मकम्) इति व्यवह्रियते~। अत्र प्रमात्मकाप्रमात्मकज्ञानं प्रति किं कारणम् ? अथवा इदं ज्ञानं प्रमात्मकम्, इदं ज्ञानम् अप्रामात्मकमिति कथं निर्णीयते ? कदा निर्णीयते ? इत्यस्मिन् विषये अस्ति विप्रतिपत्तिः दार्शनिकलोके~। यथा-
\begin{verse}
\textbf{“प्रमाणत्वाप्रमाणत्वे स्वतः सांख्या समाश्रिताः~। \\
नैयायिकास्तु परतः सौगताश्चरमं स्वतः~। \\
प्रथमं परतः प्राहुः प्रामाण्यं वेदवादिनः~। \\
प्रमाणत्वं स्वतः प्राहुः परतश्चाप्रमाणताम्”17} इति~। 
\end{verse}
अत्र ज्ञानस्य प्रामाण्याप्रामाण्यं द्वयमपि स्वत इति सांख्याः~। ज्ञानस्यप्रामाण्याप्रामण्यम् उभयं परत इति नैय्यायिकाः~। बौद्धाः ज्ञानस्याप्रामाण्यं स्वतः प्रामाण्यमात्रं परत इति~। मीमांसकाः प्रामाण्यं स्वतः अप्रमाण्यं परत इति दार्शिनिकानामभिप्रायः~। अपि च तस्मिन् विषये प्रामाण्योत्पत्तिवादः प्रामाण्यज्ञप्तिवाद इति~। ज्ञानं प्रमा एवं भ्रम इत्युभयात्मकम्~। ज्ञानत्वात्मकम् यथार्थायथार्थोऽभयसाधारणम्~। प्रमात्मकमप्रमात्मकमिति ज्ञानविशेषः~। ज्ञानोत्पादकसाधनानि ज्ञानसामग्री इति कथ्यन्ते~। लोकेस्मिन् प्रत्येकस्मिन् कार्ये सामान्यसामग्र्यः कश्चन विशेषसामग्री च वर्तते इति अनुभवसिद्धम्~। यथा- वस्त्रं प्रति तन्तुः, समवायिकारणं~। तन्तुवाय इत्यादि साधारणकारणम्~। अपिच रेश्मवस्त्रं कार्पासवस्त्रं इत्यत्र तत्तत् विशेषसामग्री कारणं भवति~। तथा अत्रापि ज्ञानविषये ज्ञानसामान्यसामग्री ज्ञानविशॆषसामाग्री च भवति~। ज्ञानसामान्यसामग्री इन्द्रियप्रकाशादयः भवति~। ज्ञानविशेषो नाम प्रमा अप्रमा(भ्रमा) वा भवति~। ज्ञानस्य सामान्य सामग्री भिन्नविशेषसामग्रीणाम् अवश्यकता वर्तते वा नवा इत्यस्मिन् विषये प्रामाण्यस्य उत्पत्तौ स्वतस्त्वपरतस्त्ववादः~। उत्पन्नज्ञाने प्रमात्मकमप्रमात्मकं ज्ञानं कदा भवति ? ज्ञानग्रहणसमये एव तदुत्पत्तिर्भवतीति स्वतस्त्ववादिनः~। ज्ञानोत्पत्यनन्तरं अनुमानेन ज्ञानस्य प्रामाण्यं भवतीति परतः वादिनः~। अयमेव वैचारिकविषयः, ज्ञप्तिषुस्वतस्त्वपरतस्त्व प्रामाण्याप्राण्यवाद इति~। 

\section*{प्रामाण्यवादस्य महत्वम्-}

अनुभवस्य याथार्थ्यं प्रामाण्यमिति अयथार्थमप्रामाण्यमिति व्यवह्रियते~। शुक्तिं दृष्ट्वा इयं शुक्तिदिति ज्ञानं भवति चेत् तज्ज्ञानं यथार्थप्रामाण्यरूपत्वात् यथार्थं भवति~। यदि शुक्तिं दृष्ट्वा शुक्तौप्रकाशकत्वादिरूप सामान्य धर्मं ज्ञात्वा इदं रजतमिति ज्ञानं भवति चेत् शुक्तौ रजतत्व दर्शनात् अयं अप्रामाण्यरूपं अयथार्थं भवति~। प्रामाण्याप्रामाण्यज्ञानं न ज्ञानकाले ज्ञानग्रहण समयेवा उद्भवति~। अतः ज्ञाने गुणसत्वे प्रामण्यं गुणाभावे दोषसत्वे वा अप्रामाण्यमिति ज्ञायते~। ज्ञानोत्पत्यनन्तरमस्माकं प्रवृत्यादयः भवन्ति~। तज्ज्ञानं सफलत्वे प्रामाण्यमिति असफलत्वे अप्रामाण्यमिति निश्चीयते~। इदं जलमिति ज्ञानोत्पत्यनन्तरमेव वयं जलस्थान गत्वा जलं पीत्वाच पिपासा निवारणानन्तरमुत्पन्नं इदं जललिमिति ज्ञानं प्रमाणं यथार्थं भवति~। तथैव अप्रामाण्यमपि ज्ञातव्यं भवति~। सर्वेषु प्रमाणेषु प्रबलमिदं प्रमाणं प्रत्यक्षम्~। अथाऽपि प्रत्यक्षसिद्धविष्येऽपि विपरीतप्रवृत्तमनुमानं बाधकं भवति~। परन्तु तार्किकाः प्रत्यक्षॆण गृहीतविषयमपि अनुमानेन ज्ञातुं प्रयतन्ते इत्ययं कुतूहलकरः विषयः~। अत एव उच्यते वाचस्पतिमिश्रॆण- “प्रत्यक्षपरिकलितमप्यर्थम् अनुमानेन बुबुत्सन्ते तर्करसिकाः” इति~। अत्र अनुमानप्रमाणेन अयं विषयः ज्ञातव्य इत्यात्मिका सिषाधयिषा सत्वे अनुमानेन प्रमाणेन ज्ञातुमिच्छन्ति, परन्तु सिषाधयिषाभावे न ते प्रवर्तन्ते~। अतः ते एव उच्यन्ते \textbf{“न हि करिणि दृष्टे चीत्कारेण तमनुमिमते अनुमातारः”} इति~। प्रत्यक्षेण दृष्टॆ गजे पुनः चीत्कार श्रवणेन अयं गज इत्यनुमानेन कोऽपि ज्ञातुं न इच्छति~। इति ज्ञान प्रामाण्ये परतस्त्वं स्वीकुर्वन्ति तार्किकाः~। अतः विशिष्टबुद्धिविशेषज्ञानयोः कार्यकारणभावाङ्गीकारेण प्रामाण्यग्राहक विशेषज्ञानाभावेन, न ज्ञानग्राहकसामग्रीमात्रेण प्रामाण्योत्पत्तिर्भवतीति~। इन्द्रियसन्निकर्षेण वस्तुनः ज्ञानोत्पत्तिर्भवति ज्ञानोत्पत्तिक्षणे अस्माकं ज्ञानप्रामाण्यम् उत्पद्यते वा नवा इत्यत्र मीमांसकाः ज्ञानस्य स्वयंप्रकाशकत्वात् तत्क्षणे एव स्वतः प्रामाण्यं भवति इति वदन्ति~। परन्तु नैय्यायिकाः तन्नाङ्गीकुर्वन्ति~। तेषां मते ज्ञानस्य उत्पन्नक्षणे केवलार्थज्ञानं भवति नतु तत्प्रामाण्यज्ञानम्~। तदनन्तरमुत्पन्नेन ज्ञानेन प्रथमक्षणोत्पन्नंस्य ज्ञानस्य ग्रहंण भवति इति~। अत्र प्रथम क्षणोत्पन्नं ज्ञानम् व्यवसायात्मकं भवति~। द्वितीयक्षणोत्पन्नं ज्ञानम् अनुव्यवसायात्मकं भवति~। प्रथम क्षणोत्पन्नज्ञानविषयक ज्ञानमेव द्वितीयक्षणे मानसिकप्रत्यक्षज्ञानमनुव्यवसाय इति प्रसिद्धिः~। यथा \textbf{’इदं पुस्तकमि’}ति विषयाभिमुखज्ञानं व्यवसायात्मिअकमिति~। ततः पुस्तकमहं जानामीति प्रथमज्ञानप्रकारकमनुव्यवसाय इति~। इत्थञ्च अनुव्यवसायेन ज्ञान ग्राहकत्वात् प्रामाण्यं परतः भवति इति न्यायवेत्तॄणामभिप्रायः~। 

\textbf{“शिवं भूयात्”}

परामर्शनसाहित्यसूचिः
\begin{enumerate}
\item \textbf{न्यायवैशेषिकखण्डः.} 	सं. 	डा. 	श्रीकृष्णसोमवालः. 	 प्रकाशनम्. 	दिल्लि संस्कृत- 	अकादमी- 	२००१
\item \textbf{न्यायदर्शनम्.}	लॆखकः 	डा.के. 	नारायणभट्टः. 		प्रकाशनम्- 	प्रसाराङ्ग 	मैसूरु विश्वविद्यालयः 	मैसूरु.	२०१६	
\item \textbf{सर्चदर्शनसङ्ग्रहः,} 	 	अनुवादकः. 	डा. 	श्री 	शिवबसवस्वामिनः	प्रकाशकः 	जगद्गुरु श्री शिवरात्रीश्वर 	ग्रन्थमाले मैसूरु. 	द्वितीयमुद्रण- 	१९९९.
\item \textbf{न्यायदर्शनम्.} 	उदय 	नारायण सिंहः प्रकाशकः. 	चौखम्भा 	संस्कृत प्रतिष्ठान दिल्लि-२००४
\item \textbf{न्यायप्रदीपः.} 	प्रकाशः 	कर्नाटक प्रशासनः
\item \textbf{तत्त्वचिन्तामणिः.}	सं. 	प्रो. 	महाप्रभुलालगोस्वामी.	 	प्रकाशकः 	सम्पूर्णानन्द संस्कृत 	विश्वविद्यालय. 	वारणासी.
\item \textbf{अथर्ववेदभाष्य 	भाग १.} 	पं. 	शेष 	नवरत्न. 	वेदभाष्य 	प्रकाशन समिति, 	बेङ्गळूरु 	प्रथमा मुद्ण. 	२००१
\item \textbf{अथर्ववेदभाष्य 	भाग २}	पं. 	शेष 	नवरत्न	वेदभाष्य 	प्रकाशन समिति, 	बेङ्गळूरु 	प्रथमा मुद्ण. 	२००१
\item \textbf{अथर्ववेदभाष्य 	भाग ३} 		श्रुतिप्रिय	. 	वेदभाष्य 	प्रकाशन समिति, 	बेङ्गळूरु 	 प्रथमा मुद्ण. 	२००१
\item \textbf{अथर्ववेदभाष्य 	भाग ४.} 	 श्रुतिप्रिय. 	वेदभाष्य 	प्रकाशन समिति, 	बेङ्गळूरु 	 प्रथमा मुद्ण. 	२००१
\item \textbf{पातञ्जलयोगदर्शन-भाष्यानुवादव्यास-भाष्य 	तथा भोजवृत्ति सहिता}
\item \textbf{न्यायदर्श्नम्} 	 सं. 	उदयनारायण 	सिंह प्रकाशन. 	चौखाम्भासंस्कृत 	प्रतिष्ठाण नवदेहली
\item \textbf{उपनिषत् 	भावधरे.} 	सोमनाथानन्द. 	श्री 	रामकॄष्ण आश्रमः मैसूरु. 	
\item \textbf{माण्डोक्योपनिषत्.} 	स्वामि 	आदिदेवानन्द. 	श्री 	रामकॄष्ण आश्रमः मैसूरु. 	 	
\item \textbf{श्वेताश्वेतरोपनिषत्.} 	स्वामि 	आदिदेवानन्द. 	श्री 	रामकॄष्ण आश्रमः मैसूरु. 	 	
\item \textbf{कठॊपनिषत्.} 	स्वामि 	आदिदेवानन्द. 	श्री 	रामकॄष्ण आश्रमः मैसूरु.
\item \textbf{मुण्डूकोपनिषत्.} 	स्वामि 	आदिदेवानन्द. 	श्री 	रामकॄष्ण आश्रमः मैसूरु.
\item \textbf{माण्डोक्योपनिषत्.} श्री. 	सिद्देश्वर 	स्वामिनः ज्ञानयोगाश्रमः 	 विजापूर.
\item \textbf{ईशावास्योपनिषत्.} 	श्री 	मल्लिकार्जुन स्वामिनः 	ज्ञानयोगाश्रमः विजापूर.
\item \textbf{लिङ्गधारण 	चन्द्रिके.} 	(कन्नड) 	सिर्सि 	गुरुशान्तशास्त्रिणः
\end{enumerate}

\articleend
}
