\chapter{श्रीशङ्करभगवत्पादकृतः वैशेषिकमतविमर्शः}

\begin{center}
\Authorline{डा. गणेश ईश्वर भट्टः}
\smallskip

सहायकाचार्यः\\
राजीवगान्धीपरिसरः,\\ 
शृङ्गेरी
\addrule
\end{center}

श्रीशङ्करभगवत्पादैः परिपोषितं दर्शनीयतमं वेदान्तदर्शनम् अद्वितीयब्रह्मप्रतिपिपादयिषया प्रवर्तत इति नाविदितं दार्शनिकसिद्धान्तसारविदां विद्वन्मतल्लिकानाम् । सत्स्वपि प्रत्यक्षादिषु अन्येषु प्रमाणेषु, अद्वितीयब्रह्मणि श्रुतिरेव प्रमाणम् । यदाह भगवान् बादरायणः “\textbf{शास्त्रयोनित्वात्}” (ब्रह्मसूत्रम् १-१-३) इति । शास्त्रं श्रुतिः योनिः प्रमाणं यस्य तत् इति शास्त्रयोनिशब्दस्य विग्रहः । श्रुतिश्च “\textbf{तं त्वौपनिषदं पुरुषं पृच्छामि}” (बृहदारण्यकोपनिषत् ३-९-२६) इति । दार्शनिकानां मध्ये ये तर्केणैव स्वसिद्धान्तं स्थापयन्ति, ते तार्किकाः इत्युच्यन्ते । नास्तिकदर्शनेषु श्रुतिप्रामाण्यस्यैव अनङ्गीकारात् नास्तिकाः तार्किकाः एव । आस्तिकदर्शनेष्वपि नैयायिक-वैशेषिक-साङ्ख्य-योगाः श्रुतेः प्रामाण्यम् अङ्गीकृत्यापि श्रुतिम् आश्रित्य तत्त्वं न प्रतिपादयन्ति । अतः तेऽपि तार्किकाः । सम्प्रति तु न्यायवैशेषिकदर्शने तर्कशास्त्रनाम्ना व्यवह्रियेते ।

ब्रह्मसूत्रे तर्कपादाख्ये द्वितीयाध्यायस्य द्वितीयपादे सर्वेषां तार्किकाणां मतं निरस्तम् । तत्र द्वितीयं तृतीयं चाधिकरणं वैशेषिकमतसम्बद्धम् । प्रथमे, वैशेषिकैः वेदान्तदर्शने आक्षिप्ताः दोषाः परिहृताः । वैशेषिकाः सप्तपदार्थवादिनः । षोडशपदार्थानां सप्तस्वन्तर्भावेण सप्तविशेषपदार्थप्रतिपादककाणादसूत्रं तद्भाष्यमन्यं चैतदर्थप्रतिपादकमभिधीयते विदन्ति वा इति वैशेषिकशब्दस्य व्युत्पत्तिः । ब्रह्मैकमव सत्यम्, तदेव जगत्कारणम् इति वेदान्तसिद्धान्तः तेषां न सम्मतः । अतः एवं वैशेषिकैः आक्षिप्तम्- चेतनं ब्रह्म जगतः कारणं भवितुं नार्हति । यतः कारणद्रव्यसमवायिनो गुणाः कार्यद्रव्ये समानजातीयं गुणान्तरमारभन्ते । यथा शुक्लेभ्यस्तन्तुभ्यः शुक्लः पटः उत्पद्यते, तद्वत् । चेतनस्य ब्रह्मणो जगत्कारणत्वे अभ्युपगम्यमाने, कार्येऽपि जगति चैतन्यं समवेयात् । किन्तु जगति चैतन्यस्य अदर्शनात् चेतनं ब्रह्म न जगत्कारणम् इति ।

इममाक्षेपं प्रतिक्षेप्तुं “\textbf{महद्दीर्घवद्वा ह्रस्वपरिमण्डलाभ्याम्}” (ब्रह्मसूत्रम् २-२-११) इत्यधिकरणम् आरब्धम् । अत्र सूत्रभाष्यकृद्भ्यां वैशेषिकमतं तदीयया एव प्रक्रियया निरस्तम् । एवं हि वैशेषिकाः मन्यन्ते- परमाणवः कञ्चित्कालमनारब्धकार्या यथायोगं रूपादिमन्तः पारिमाण्डल्यपरिमाणाश्च तिष्ठन्ति । अददृष्टादिकारणसहिताः ते सृष्टिकाले द्व्यणुकादिक्रमेण कृत्स्नं कार्यजातमारभन्ते । द्व्यणुकादिकार्योत्पत्तिसमये कारणगुणाः कार्ये स्वसजातीयं गुणान्तरम् आरभन्ते । किन्तु परमाणुगुणविशेषभूतं पारिमाण्डल्यं द्व्यणुके पारिमाण्डल्यान्तरं नारभते । द्व्यणुके तु विद्यमानं परिमाणम् अणुत्वं ह्रस्वत्वं चेति तेषां मतम् । एवमेव द्व्यणुकात् त्र्यणुकस्य उत्पत्तौ द्व्यणुकगते अणुत्वह्रस्वत्वे त्र्यणुके सजातीयं परिमाणान्तरं नारभेते । त्र्यणुकगतं तु परिमाणं महत्त्वम् । तथा च यथा परमाणुगतस्य पारिमण्डल्यस्य द्व्यणुके अभावः, यथा च द्व्यणुकगतयोरपि अणुत्वह्रस्वत्वयोः त्र्यणुके अभावः, एवमेव चेतनात् ब्रह्मणः उत्पद्यमाने जगति चैतन्यस्य अभावः उपपद्यत एव।

एतदर्थनिरूपणावसरे भाष्यकारैः अपरः कश्चिदर्थः सूचितः । तैः “\textbf{यदापि द्वे द्व्यणुके चतुरणुकमारभेते, तदापि समानं द्व्यणुकसमवायिनां शुक्लादीनामारम्भकत्वम्, यदापि बहवः परमाणवः, बहूनि वा द्व्यणुकानि, द्व्यणुकसहितो वा परमाणुः कार्यमारभते, तदापि समानैषा योजना}” इत्युक्तम् । किन्तु एवंविधा प्रक्रिया वैशषिकाणां न सम्मता । तथापि भाष्यकारैः किमर्थम् एवमुक्तमिति चेत् वैशेषिकप्रक्रियायाः अप्रयोजकत्वद्योतनार्थम् । वैशेषिकोक्तरीत्या संयोगः द्वयोरेव परमाण्वोः इति नियमनं अयुक्तम् । भोगविशेषाभावात् अयं संयोगः न सम्भवति इत्यपि न, सामग्रीसम्भवप्राप्तस्य कार्यस्य प्रयोजनाभावमात्रेण निवारणासम्भवात् । स्पष्टं चैतत् परिमलब्रह्मविद्याभरणादिषु।

भाष्यकारैः विभिन्नेषु स्थलेषु वैशेषिकाणां नैकाः प्रक्रियाः दूषिताः । यथा अत्रैव सजातीयानां कार्यारम्भकत्वनियमः । कारणगुणाः सजातीयानाम् एव आरम्भकाः न भवन्ति । द्रव्यं प्रति समवायिकारणसंयोगः कारणम् इति वैशेषिकाः वदन्ति । तेन संयोगेन गुणेन द्रव्यस्य उत्पत्तिकथनात् सजातीयमेव उत्पद्यते इति नियमः भग्नः । अत्र वैशेषिकैः य आक्षेपः कृतः, तमपि भाष्यकाराः तदीयसूत्रोदाहरणेनैव निराकुर्वन् । यथा, कार्यं प्रति उपादानत्वात् ब्रह्म द्रव्यम् । तत् विलक्षणस्य जगतः उपादानं भवितुं नार्हति इत्याक्षिप्ते, तादृशं द्रव्यमेव उदाहर्तव्यम्, यत् विलक्षणकार्यं प्रति उपादानम् । संयोगः तु गुणत्वात् अत्र उदाहरणतां नार्हति इति वैशेषिकाणाम् आक्षेपः । तदा भाष्यकारैः प्रत्युक्तम् “\textbf{सूत्रकारोऽपि भवतां द्रव्यस्य गुणमुदाजहार प्रत्यक्षाप्रत्यक्षाणाम् अप्रत्यक्षत्वात् संयोगस्य पञ्चात्मकत्वं न विद्यते} (वैशेषिकसूत्रम् ४-२-२) \textbf{इति}” इति । अयमर्थः, अस्मदादीनां शरीरं पार्थिवम् । यदि तत् पाञ्चभौतिकं स्यात् तर्हि तस्य प्रत्यक्षत्वम् एव न स्यात् । यथा प्रत्यक्षायाः भूमेः अप्रत्यक्षस्य आकाशस्य च संयोगः अप्रत्यक्षः, तथैव प्रत्यक्षाप्रत्यक्षेषु समवायेन वर्तमानं शरीरं प्रत्यक्षमेव न स्यात् इति । अत्र वैशेषिकसूत्रकारेणैव द्रव्यस्य उदाहरणतया गुणः प्रदर्शितः । अतः वेदान्तिभिः अपि एवंविधम् उदाहरणं युज्यत इति वदतां भाष्यकाराणां महती सूक्ष्मेक्षिका दृश्यते ।

एतदनन्तरं परमाणुजगदकारणत्वाधिकरणे वैशेषिकाणां परमाणुकारणवादः निरस्तः । एवं वैशेषिकाः प्राहुः- पृथिव्यादिचतुष्टयस्य परमाणवः आकाशादिपञ्चकं च नित्यद्रव्याणि । पृथिव्यादीनां कार्यं द्व्यणुकादिकम् अनित्यम् । सर्वमपि कार्यं सावयवम्, स्वसमवायिकारणसंयोगजन्यं च । प्रलयकाले विभागेन स्थितेभ्यः परमाणुभ्यः सर्गादौ सृष्टिरारभ्यते । परमाणुद्वयसंयोगेन द्व्यणुकम्, त्रिभ्यः द्व्यणुकेभ्यः त्र्यणुकम्, ततः चतुरणुकम् इत्येवं स्थूलकार्यम् उत्पद्यते इति । अयं वादः औपनिषदस्य ब्रह्मकारणवादस्य विरोधीति सूत्रभाष्यकृद्भ्यां निरस्तम् । तत्र भाष्यकारैः पृष्टम् विभागावस्थानां परमणूनां संयोगे कारणीभूतं कर्म दुर्निरूपम् । तत्र यदि प्रयत्नोऽभिघातादिर्वा दृष्टं किमपि कर्मणो निमित्तमभ्युपगम्येत, तन्न सम्भवति । यतः शरीरप्रतिष्ठे मनस्यात्मनः संयोगे सति आत्मगुणः प्रयत्नो जायते । अभिघातः अपि सृष्ट्यनन्तरं वक्तुं शक्यः । शरीरादीनाम् उत्पत्तेः प्राक् तु एतादृशं कर्मणः निमित्तं भवितुं नार्हति । अदृष्टमपि न तत्र निमित्तम्, अदृष्टस्य अचेतनत्वात् । अपि च, परमाणुद्वयसंयोगः सर्वात्मना वा स्यात् एकदेशेन वा इति विचार्यमाणे, सर्वात्मना चेत्, उपचयानुपपत्तेः अणुमात्रत्वप्रसङ्गः । एकदेशेन चेत्, सावयवत्वप्रसङ्गः । तदेवं नियतस्य कस्यचित् कर्मनिमित्तस्य अभावात् अणुष्वाद्यं कर्म न सम्भवति । कर्माभावात् तन्निबन्धनः संयोगस्यापि असम्भवः । संयोगाभावाच्च तन्निबन्धनं द्व्यणुकादि कार्यजातं नोत्पद्येत । एवं महाप्रलयेऽपि कार्यजातस्य विभागोत्पत्त्यर्थं कर्म नैवाणूनां सम्भवेत् । अतः परमाणुकारणवादः अनुपपन्नः ।
\begin{verse}
\textbf{घटादीनां कपालादौ द्रव्येषु गुणकर्मणोः ।\\
तेषु जातेश्च सम्बन्धः समवायः प्रकीर्तितः ॥} (कारिकावली ११)
\end{verse}
इति वैशेषिकैः यः समवायः इति सम्बन्धः कल्पितः, सः भाष्ये नैकत्र निरस्तः । भाष्यकारैः एवं पृच्छ्यते- यदि समवायः सम्बन्धान्तरम् अनपेक्ष्य स्वयमेव सम्बद्ध्यते, तर्हि संयोगेन किमपराद्धम् । सोऽपि सम्बन्धत्वाविशेषात् स्वयमेव सम्बध्येत । अतः यथा संयोगः सम्बन्धत्वात् सम्बन्धान्तरम् अपेक्षते, एवं समवायस्यापि सम्बन्धान्तरस्य अपेक्षा दुर्वारा । तदा तस्यापि सम्बन्धान्तरम् इत्यनवस्था इति । तथैव परमाणूनां स्वभावः किं प्रवृत्तिः उत निवृत्तिः इति निरूपणस्य असम्भवः प्रदर्शितः । परमाणूनाम् अनित्यत्वापत्तिरपि दर्शिता । अयुतसिद्धयोः सम्बन्धः समवायः इति मतमपि विस्तरेण विचार्य निराकृतम् ।

अत्र प्रश्नः स्यात्- कुतः भगवत्पादाः वैशेषिकमतं निराचक्रुः? तस्यापि आस्तिकदर्शनत्वात्, तस्य तर्कप्रधानत्वेऽपि, तर्काणां वेदान्तेऽपि केनचिद्रूपेण आदृतत्वात् वैशेषिकमतं न अत्यन्तम् अनादरणीयम् इति । तत्रोच्यते- आस्तिकदर्शनं सदपि वैशेषिकदर्शनं न श्रुत्यर्थानुसारि । वैशेषिकाः स्वबुद्ध्यनुसारेण यत्किञ्चित् कल्पयित्वा श्रुतिविरोधरूपे महति दोषे आपतिते, गत्यन्तराभावात् कथञ्चित् श्रुतिवाक्यमपि स्वमतानुगुण्येन योजयन्ति । यथा आकाशनित्यत्ववादिनः “आत्मन आकाशः सम्भूतः” (तैत्तिरीयोपनिषत् २-१) इति आकाशस्य उत्पत्तिवादिनीं श्रुतिं गौणीं मन्यन्ते । अत एव भाष्ये समन्वयाध्याय एव निगदितम्- “\textbf{साङ्ख्यादयस्तु परिनिष्ठितं वस्तु प्रमाणान्तरगम्यमेवेति मन्यमानाः प्रधानादीनि कारणान्तराणि अनुमिमानाः तत्परतयैव वेदान्तवाक्यानि योजयन्ति । काणादास्त्वेतेभ्य एव वाक्येभ्य ईश्वरं निमित्तकारणमनुमिमते, अणूंश्च समवायिकारणम्}” इति ।

एवं श्रुतेः अनादरादेव वैशेषिकाः बौद्धादिनास्तिकापेक्षया न अत्यन्तं विलक्षणाः । इयानेव भेदः- बौद्धाः श्रुतेः प्रामाण्यं नाङ्गीकुर्वन्ति । वैशेषिकादयः तदङ्गीकृत्यापि तदर्थम् अन्यथा व्याख्यान्तीति । सूत्रकरेण सूचितमिममर्थं भाष्यकाराः एवं विशदीचक्रुः- “\textbf{वैशेषिकराद्धान्तो दुर्युक्तियोगाद् वेदविरोधाद् शिष्टापरिग्रहाच्च नापेक्षितव्य इत्युक्तम् । सोऽर्धवैनाशिक इति वैनाशिकत्वसाम्यात् सर्ववैनाशिकराद्धान्तो नितरामपेक्षितव्य इतीदम् इदानीमुपपादयामः}” (ब्रह्मसूत्रभाष्यम् २-२-१८) इति । भामतीकारा अप्याहुः “\textbf{वैशेषिकाः खल्वर्धवैनाशिकाः । ते हि परमाण्वाकाशदिक्कालात्ममनसां च सामन्यविशेषासमवायानां च गुणानां च केषाञ्चित् नित्यत्वम् अभ्युपेत्य शेषाणां निरन्वयनाशम् उपयन्ति, तेन ते अर्धवैनाशिकाः}” इति ।

वैशेषिकाः द्रव्यगुणादिरूपेण यं पदार्थविभागं कुर्वन्ति, सः अश्रौतः, अत एव भाष्यकाराणां असम्मतश्च । एकं ब्रह्मैव सर्वस्यापि प्रपञ्चस्य प्रकृतिः, अतः प्रपञ्चः ब्रह्माभिन्नः, कारणं ब्रह्मैव सत्यम् अन्यत् मिथ्या इति स्फुटं वेदान्ताः आघोषयन्ति । एवं सति अयं द्रव्यगुणादीनां अत्यन्तभेदः कथं शिष्टैः अङ्गीक्रियेत । अत्यन्तभिन्नानां तेषाम् इतरेतराधीनत्वं च कथं सङ्गच्छेत ।

तर्हि वैशेषिकमते वैदिकानां का दृष्टिः स्यात् इत्यस्मिन् विषये इदं सूत्रं सर्वथा स्मर्तव्यम् “\textbf{अपरिग्रहाच्चात्यन्तमनपेक्षा}” (ब्रह्मसूत्रम् २-२-१७) इति । भाष्यं च तत्र “\textbf{प्रधानकारणवादो वेदविद्भिरपि कैश्चिन्मन्वादिभिः सत्कार्यत्वाद्यंशोपजीवनाभिप्रायेण उपनिबद्धः । अयं तु परमाणुकारणवादो न कैश्चिदपि शिष्टैः केनचिदप्यंशेन परिगृहीत इत्यत्यन्तमेव अनादरणीयो वेदवादिभिः}” इति ।

एवमपि गौतमीयतन्त्रं भगवत्पादैः न अत्यन्तम् अनादृतम्, प्रत्युत क्वचित् समादृतम् । यथा समन्वयाधिकरणभाष्ये अविद्यानिवर्तकत्वात् तत्त्वज्ञानमेव मोक्षहेतुः, न कर्म इत्युपपादनावसरे “\textbf{तथा च आचार्यप्रणीतं न्यायोपबृंहितं सूत्रम्}” इत्युक्त्वा न्यायसूत्रम् उदाजह्रुः भगवत्पादाः “\textbf{दुःखजन्मप्रवृत्तिदोषमिथ्याज्ञानानाम् उत्तरोत्तरापाये तदनन्तरापायात् अपवर्गः}” (न्यायसूत्रम् १-१-२) इति । तत्त्वज्ञानं मिथ्याज्ञाननिवृत्तिद्वारा परम्परया दुःखं निवर्तयिष्यति इति नैयायिकैः यत् सिद्धान्तितम्, तद्वेदान्तिनां सम्मतम् । पत्यधिकरणभाष्येऽपि ईश्वरस्य केवलनिमित्तकारणत्ववादिनां मतं निराकुर्वन्तः प्राहुः “\textbf{प्रवर्तनालक्षणा दोषाः} (न्यायसूत्रम् १-१-१८) \textbf{इति न्ययवित्समयः}” इति । ईश्वरस्य जगत्सृष्ट्रौ प्रवर्तमानस्य रागद्वेषाख्यदोषाः प्रसज्यन्ते इत्यस्य अर्थस्य उपपादनाय इदं सत्रम् उदाहृतम् । बृहदारण्यकभाष्ये चतुर्थाध्यायस्य मैत्रेयीब्राह्मणं निगमनस्थानीयम् इति कथयन्तः भाष्यकाराः न्यगादिषुः “\textbf{अयं च न्यायः वाक्यकोविदैः परिगृहीतः हेत्वपदेशात् प्रतिज्ञायाः पुनर्वचनं निगमनम्}” (न्यायसूत्रम् १-१-३०) इति । एतावता नैयायिकानां मतं सर्वात्मना भगवत्पादानां सम्मतम् इति न मन्तव्यम् । न हि नैयायिकाः औपनिषदाद्वैतब्रह्मवादिनः ।

माण्डूक्योपनिषद्भाष्ये केचन वादिनः पूर्वम् अविद्यमानस्य वस्तुनः उत्पत्तिं कथयन्ति इत्यर्थस्य प्रतिपादनाय प्रवृत्तायाः कारिकायाः \textbf{अपरे धीराः} इति पदयोः अर्थं वर्णयन्तः भाष्यकाराः “\textbf{अपरे वैशिषिकाः नैयायिकाश्च धीराः धीमन्तः प्राज्ञाभिमानिन}” इति व्याख्यन् । वैशेषिका इव नैयायिका अपि असत्कार्यवादिनः । वादकथादीनां निर्वाहाय न्यायसरणिः उपयुज्यत इत्येव विशेषः न्यायमतस्य ।

“\textbf{नावेदविन्मनुते तं बृहन्तम्}” इत्यादिश्रुतेः, वेदान्तार्थज्ञानं विना परमपुरुषार्थः नैव सिद्ध्यति । अतः यथा “\textbf{न हि नहि रक्षति डुकृञ् करणे}” इत्युक्तरीत्या वैयाकरणानां धातुपाठः मरणकाले निरर्थकः, तथैव काणादगौतमीयानां अनात्मविषयाः विजिगीषुकथाः मुमुक्षुदृष्ट्या निष्फला एव । तदेवं वर्णितं भगवत्पादैः-
\begin{verse}
\textbf{घटो वा मृत्पिण्डोऽप्यणुरपि धूमोऽग्निरचलः\\
पटो वा तन्तुर्वा परिहरति किं घोरशमनम् ॥\\
वृथा कण्ठक्षोभं वहसि तरसा तर्कवचसा\\
पदाम्भोजं शम्भोर्भज परमसौख्यं व्रज सुधीः ॥} (शिवानन्दलहरी-६)इति ।
\end{verse}
न्यायवैशेषिकयोः मध्ये भाष्यकारैः वैशेषिकदर्शनम् अत्यन्तम् एव निराकृतम् । एवं निराकरणे कारणं तु वैशेषिकैः सिद्धान्तितः अवैदिकः परमाणुकारणवादः । परमाणुभ्यः द्व्यणुकादिक्रमेण जगत् उत्पद्यते इति वादः श्रुतिस्मृतिषु न क्वापि दृष्टचरः । प्रत्युत “आत्मन आकाशः सम्भूतः” “तस्मात् (ब्रह्मणः) कायाः प्रभवन्ति सर्वे” (आपस्तम्बधर्मसूत्रम् १-८-२३-२) इत्यादिभिः ब्रह्मकारणत्वप्रतिपादिनीभिः श्रुतिस्मृतिभिः अयं वादः विरुद्ध्यते । अतः भाष्यकाराः प्रस्थानत्रयभाष्ये असकृत् इमं वादं निराचक्रुः । यदि वैशेषिकदिमतोक्ताः तर्काः कथञ्चित् वेदान्तप्रतिपाद्यात्मज्ञानम् उपकुर्युः. तर्हि तावता अंशेन वैशेषिकादीनां मतानाम् आदरं कर्तुं भगवत्पादाः सिद्धाः एव । स्पष्टीकृतं चैतत् योगप्रत्युक्त्यधिकरणभाष्ये “\textbf{एतेन सर्वाणि तर्कस्मरणानि प्रतिवक्तव्यानि । तान्यपि तर्कोपपत्तिभ्यां तत्त्वज्ञानायोपकुर्वन्तीति चेत्, उपकुर्वन्तु नाम । तत्त्वज्ञानं तु वेदान्तवाक्येभ्य एव भवति}’’ इति ।

\articleend
