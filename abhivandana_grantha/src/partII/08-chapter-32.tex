{\fontsize{15}{17}\selectfont
\presetvalues
\chapter{भावभेदविवृतिः}

\begin{center}
\Authorline{वि~॥ चन्द्रशेखरभट्टः,शृङ्गेरी}
\smallskip

सहायकाचार्य:, व्याकरणविभाग:\\
राष्ट्रियसंस्कृतसंस्थानम् (मानितविश्वविद्यालय:)\\
राजीवगान्धीपरिसर:,मेणसे,\\ 
शृङ्गेरी -577139
\addrule
\end{center}

\begin{verse}
\textbf{वागर्थाविव सम्पृक्तौ वागर्थप्रतिपत्तये~। }\\
\textbf{जगतः पितरौ वन्दे पार्वतीपरमेश्वरौ~॥}
\end{verse}
अर्थबोधनाय हि लोके शब्दः प्रयुज्यते~। यावन्तश्चार्थास्ततोऽधिकतराः शब्दा एकैकस्यार्थस्य वाचकशब्दानां बहुत्वात्~। समेषामप्येषां लौकिकानां वैदिकानाञ्च शब्दानां व्युत्पादनं कुर्वत्पाणिनीयमेव भुवनतले प्रसिद्धम्, वेदाङ्गत्वेन च प्रथितम्~। शास्त्रेऽस्मिन् भाव इति शब्दः कथञ्जातीयकमर्थं प्रत्याययति, तद्बोध्यानामर्थानामस्ति वा किमपि वैलक्षण्यमिति मनागिवास्मिन् प्रबन्धे विचारो विधास्यते~। 
\begin{verse}
\textbf{भावः पदार्थसत्तायां क्रियाचेष्टात्मयोनिषु~। }\\
\textbf{विद्वल्लीलास्वभावेषु भूत्याभिप्रायजन्तुषु॥}
\end{verse}
इत्येवं बह्वर्थो भावशब्दः~। आलङ्कारिकास्तावत् चित्तवृत्तिविशेषं स्थायिसञ्चारिसात्विकाख्यं भावमाचक्षते~। धर्माधर्मज्ञानाज्ञान- वैराग्यावैराग्यैश्वर्यानैश्वर्याख्यावष्टौ भावशब्देन  व्यपदिश्यन्ते  साङ्ख्यैः~। द्रव्यगुणकर्मसामान्यविशेषसमवायाः षट् पदार्थाः भावशब्देनाभिधीयन्ते न्यायशस्त्रे~। ज्योतिःशास्त्रे च जातकफलनिर्णयायाश्रिताः तनुधनादयो द्वादश भावशब्दबोध्याः~। एवं स्थिते को भावो वैयाकरणानाम् ?

उच्यते, भावशब्दोऽयं त्रिष्वर्थेषु प्रसिद्धो व्याकरणे~। एको धात्वर्थभावः, इतरौ प्रत्ययार्थौ~। प्रत्ययार्थयोश्च कृद्वाच्य एकः, तद्धिताभिधेयोऽपरः~। “लः कर्मणि च भावे चाकर्मकेभ्यः” इति सूत्रीयो भावो धात्वर्थः~। साध्यक्रियाबोधक इति यावत्~। भावे इति सूत्र-उपात्तः कृद्भावः~। स हि क्रियायाः सिद्धावस्थां बोधयति~। तद्धितभावस्तु तस्य भावस्त्वतलौ इति सूत्रे~। स च प्रकृतिजन्यबोधे प्रकारीभूतो धर्मः~। 

\section*{धात्वर्थभावः -} 

“लः कर्मणि च भावे चाकर्मकेभ्यः” इति सूत्रेण धातोः कर्तरि, कर्मणि, भावे च लकारा विधीयन्ते~। लकाराः सकर्मकेभ्यो धातुभ्यः कर्तरि कर्मणि च स्युरकर्मकेभ्यो भावे कर्तरि च स्युरिति सूत्रार्थः~। अत्र सूत्रे धात्वर्थक्रियैव भावशब्देनाभिप्रेयते~। यौगिकोऽयं भावशब्दो भावनायाम्~। उत्पत्त्यर्थात् भवतेर्णिजन्तादेरजित्यच्प्रत्ययः~। एवञ्चास्य करोतिना तुल्यार्थत्वमायातम्~। अत एवायं भावना उत्पादना क्रिया इत्यादिशब्दैर्व्यपदेशमर्हति~। न चैवं भवतेरपि करोतिवत्सकर्मकत्वप्रसङ्ग इति वाच्यम्; फलसमानाधिकरणत्वेनाकर्मकत्वात्~। फलव्यधिकरणव्यापारवाचको हि सकर्मकः~। यथा - भावयति घटं कुलालः~। भवति घट इत्यत्र विद्यमानोऽप्युत्पादनार्थः फलव्यापारयोः सामानाधिकरण्यात् न प्रतीयत इति अकर्मकत्वं न हीयते~। अयमुत्पत्त्यनुकूलव्यापाररूपोऽर्थः सकलधातुवाच्यः~। यद्यपि सर्वेषामपि धातूनां समानार्थत्वापत्तिरेवम्, तथापि विक्लित्त्यादिफलस्य विभाजकस्य तत्र सत्वान्नानुपपत्तिः~। उक्तं च कौमुद्यां ‘भावो भावना उत्पादना क्रिया~। सा च धातुत्वेन सकलधातुवाच्या’ इति~। पच्यादयो धातुत्वेन रूपेणोत्पत्त्यनुकूलव्यापारात्मिकां क्रियामाहुः, पचित्वादिविशेषरूपेण तु विक्लित्त्यादितत्तत्फलांशमाहुः~। एवञ्च धातुना धातुत्वेन क्रियासामान्यस्य, पचित्वाद्यवच्छेदकविशेषेण विक्लित्त्यादिफलविशेषस्य च प्रतीतिरित्युक्तं भवति~। 

अथ पच्यादौ क्रिया इति कोऽसावर्थो द्रव्यव्यतिरिक्तः? यदि पुनरीहा, चेष्टा, व्यापार इति चेन्नेदं रमणीयमुत्तरम्, पर्यायोपादानमात्रेण स्वरूपस्यानिश्चयात्~। उच्यते, विषयेन्द्रियग्राह्या न भवति क्रिया~। तथा च भाष्यम्-  “क्रिया नामेयमत्यन्तापरिदृष्टाऽशक्या पिण्डीभूता निदर्शयितुम्” सा सावनुमानगम्या~। कोसावनुमानः ? इह सर्वेषु साधनेषु संनिहितेषु कदाचित्पचतीत्येतद्भवति कदाचिन्न भवति~। यस्मिन्संनिहिते पचतीत्येतद्भवति सा नूनं क्रिया~। अथवा यया देवदत्त इह भूत्वा पाटलिपुत्रे भवति सा नूनं क्रिया” इति~। फलानुमेया हि क्रिया~। इहस्थदेवदत्तस्य पाटलिपुत्रसंयोगेन गमनक्रियाऽनुमीयते~। ननु फलव्यापारयोः जन्यजनकभावग्रहमन्तरा फलेन कार्येण क्रियायाः कारणस्यानुमानं दुर्घटम् ? सत्यम्, धातुवाच्यसमूहस्य युगपदसन्निधानादप्रत्यक्षत्वम्~। परमवयवानां प्रत्यक्षे बुद्ध्या सर्वानवयवान् सङ्कलय्य पचतीति प्रयुज्यते~। एषा क्रियैव भावः धातुवाच्यः~। ननु क्रियाया धातुवाच्यत्वे तद्विशेष्यकबोधो न स्यात्, प्रत्ययार्थप्राधान्यस्य सर्वत्र क्लृप्तत्वादिति भावनायाः प्रत्ययवाच्यत्वं वदतां मीमांसकमतमेव रमणीयमिति चेदाहुः - प्रत्ययार्थः प्रधानमित्युत्सर्गः~। क्रियाप्रधानमाख्यातमिति रूपप्सूत्रभाष्यात्, भावप्रधानमाख्यातमिति निरुक्ताच्च तिङन्ते धात्वर्थव्यापारस्यैव प्राधान्यमवसीयते~। अत एव शृणु कोकिलो गायति, पश्य मृगो धावतीत्यादयः प्रयोगाः स्वरसत उपपद्यन्ते~। 

\section*{कृद्भावः} 

कृत्प्रत्ययाश्च केचित् भावार्थे विधीयन्ते~। ननु धात्वर्थः केवलः शुद्धो भाव इत्यभिधीयते, तत्कथं स प्रत्ययार्थः सम्पन्नः? उच्यते, धातुना भावोऽभिधीयत इति सत्यम्, परं सः साध्यभावः~। कृत्प्रत्ययाभिधेयस्तु सिद्धभाव इति तयोरस्ति वैलक्षण्यम्~। धात्वर्थस्य सिद्धरूपतैव घञादिप्रत्ययवाच्यो भावः~। उक्तं कौमुद्यां भावे इति सूत्रे “सिद्धावस्थापन्ने धात्वर्थे वाच्ये धातोर्घञ् स्यादि”ति~। उक्तञ्च भूषणे -
\begin{verse}
\textbf{साध्यत्वेन क्रिया तत्र धातुरूपनिबन्धना~। }\\
\textbf{सिद्धभावस्तु यस्तस्याः स घञादिनिबन्धनः॥} इति~। 
\end{verse}
साध्यत्वं हि क्रियान्तराकाङ्क्षानुत्थापकतावच्छेदकं यद्वैजात्यं तद्रूपम्~।     तादृशवैजात्यं धात्वर्थक्रियायां भासते~। तदेव च कारकान्वयितावच्छेदकम्~। केचिद्धि अत्र लक्षणे विप्रतिपन्नाः~। ते हि साध्यत्वमुत्पाद्यत्वमित्याहुः~। वस्तुतस्तु उत्पाद्यत्वमेव कारकान्वयितावच्छेदकं क्रियान्तराकाङ्क्षानुत्थापकतावच्छेदकञ्च~। इदमेव च ‘यावत्सिद्धमसिद्धं वा साध्यत्वेनाभिधीयते’ इति कारिकयोक्तं भूषणे~। किञ्च क्रियान्तराकाङ्क्षानुत्थापकतावच्छेदकत्वमिति साध्यक्रियाया लक्षणं न, किन्तु परिचायकं वाक्यम्~। अन्यथा क्रियाघटितत्वेनान्योन्याश्रयप्रसङ्गः~। लक्षणं तूत्पाद्यत्वमेव~। पचति, करोतीत्याद्युक्ते कारकाकाङ्क्षा जायते, न तु क्रियान्तराकाङ्क्षा~। न ह्युत्पाद्यस्योत्पाद्यान्तरापेक्षा, किन्तूत्पादककारकाकाङ्क्षा~। इदमेव च तात्पर्यं “नहि क्रिया क्रियया समवायं गच्छति” इति भाष्यस्य~। प्रत्ययवाच्यस्तु भावः कारकं नाकाङ्क्षति किन्तु स्वयमेव कारकत्वेन क्रियान्तरमाकाङ्क्षति~। क्रियान्तराकाङ्क्षोत्थापकत्वमेव च क्रियायाः सिद्धत्वं नाम~। यथा- पाकः, यागः इत्याद्युक्ते अस्ति, नश्यति, उत्पद्यते इत्येवं क्रियाविषयकाकाङ्क्षा भवति~। तथा च तिङन्तोपस्थाप्या क्रिया क्रियान्तरं नाकाङ्क्षति कृदन्तोपस्थाप्या त्वाकाङ्क्षतीत्युक्तं भवति~। तिङन्तवाच्यो भावः नियमेन कर्त्राकाङ्क्षः, कृदुपस्थाप्यस्तु न तथा~। न हि पाको राग इत्युक्ते नियमेन कर्त्राकाङ्क्षा दृश्यते~। धातुवाच्या क्रियाऽसत्वभूता~। लिङ्गसंख्याद्यन्वयस्तत्र नास्ति~। अत एवोक्तं वाक्यपदीये “असत्वभूतो भावस्तु तिङ्पदैरभिधीयते” इति~। घञादिवाच्यस्तु भावो द्रव्यायमाणः संल्लिङ्गसंख्याद्यन्वयी कारकत्वेन च क्रियान्वयीति विशेषः~। तथा च “सार्वधातुके यक्” इति सूत्रे भाष्यम् - ‘अस्ति खल्वपि विशेषः तिङभिहितस्य भावस्य कृदभिहितस्य च~। कृदभिहितो भावो द्रव्यवद् भवति~। द्रव्यं क्रियाभिनिर्वृत्तौ साधनत्वमुपैति, तद्वच्चास्य भावस्य कृदभिहितस्य भवति पाको वर्तत’ इति~। क्रिया क्रियया समवायं न गच्छति, पचति, पठतीति~। तद्वच्चास्य कृदभिहितस्य न भवति पाको वर्तत इति~। तिङभिहितेन भावेन कालपुरुषोपग्रहा व्यज्यन्ते, कृदभिहितेन पुनर्न व्यज्यन्ते~। तिङभिहितो भावः कर्त्रा सम्प्रयुज्यते कृदभिहितः पुनर्न सम्प्रयुज्यते~। लिङ्गकृतः सङ्ख्याकृतश्च विशेषः कृदभिहिते इति~। कृदन्ततिङन्तयोः क्रियात्वं समानम्, सिद्धत्वसाध्यत्वकृतो विशेषस्तयोः~। साध्यक्रियायामेव कारकाणामन्वयात् कृदन्तोपस्थाप्यभावे कारकाणामन्वयो न स्यादिति तु न युक्तम्, ओदनस्य पाक इत्यत्र कर्मषष्ठ्या एव मानत्वात्~। कर्तृकर्मणोः कृति इति सूत्रेण कर्मण्यत्र षष्ठी विधीयते~। किञ्च प्रत्ययेन सिद्धत्वमुच्यते चेदपि प्रकृत्या साध्यक्रियाया बोधात् तत्र कारकाणामन्वये बाधकाभावाच्च~। 

स्यादेतत्~। उत्पादनात्मकक्रियारूपो भावो धातुवाच्यश्चेत् कथं तत्रैव लकारविधिः स्यात् अनन्यलभ्यस्यैव हि शब्दार्थत्वात् ? अत्रोच्यते, यद्यपि भावे एव लकारो विधीयते, पाकादिशब्द इव प्रत्ययस्य सिद्धावस्थापन्नत्वमपि नार्थः तथापि धात्वर्थभाव एवानूद्यते~। यथा द्वौ त्रयः इत्यादौ प्रकृत्या बोधितोऽपि द्वित्वबहुत्वाद्यर्थः प्रत्ययेनानूद्यते~। 

\section*{भावलकारे पुरुषवचनव्यवस्था} 

ननु तिङ्वाच्यकारकसामानाधिकरण्ये सत्येव प्रथममध्यमोत्तमपुरुषाणां विधानात् भावलकारे न कस्यचन कारकस्य लकारवाच्यत्वाभावेन पुरुषवचनादिकं कथं स्यात् ? अत्रोच्यते, युष्मदस्मदोः तिङ्समानाधिकरणयोः मध्यमोत्तमौ विधीयेते~। तयोश्च तिङ्सामानाधिकरण्यं तिङ्वाच्यकारकवाचित्वमेव~। तस्मान्नास्ति सम्भवो मध्यमोत्तमयोः~। अतः शेषे प्रथमः इति प्रथमपुरुषोऽत्र भवति~। यथा त्वया आस्यते, मया आस्यते, चैत्रेण सुप्यते इति~। नन्वत्र प्रथमपुरुषोऽपि दुर्लभः, युष्मदस्मद्भिन्नस्यैव शेषपदार्थत्वेन तत्समानाधिकरणे देवदत्तः पचतीत्यादावेव प्रथमपुरुषस्य प्रवृत्तेरिति चेन्न, शेषे प्रथम इति सूत्रस्य युष्मदस्मदोरविषये प्रथमपुरुष इति व्याख्यानात्~। एवञ्च भावे प्रथमपुरुष एव~। 

घञादिवाच्यभावः सत्वधर्मं प्राप्नोतीति द्रव्यधर्मं लिङ्गसंख्यादिकम् लभत इति तत्र युज्यते द्विवचनबहुवचनादिकम्~। यथा- पाकः, पाकौ, पाकाः इति~। किन्तु तिङन्तोपस्थितभावस्य असत्वरूपत्वेन द्वित्वबहुत्वयोरप्रतीतेः त्वया मया अन्यैश्च आस्यत इत्येव प्रयोगः~। वस्तुतस्त्विदं प्रायिकम्~। अत एव उष्ट्रासिका आस्यन्ते, हतशायिकाः शय्यन्ते इति सार्वधातुके यक् इति सूत्रस्थभाष्यप्रयोगः सङ्गच्छते~। न हि तत्र कर्मणि लकारः येन बहुवचनमुपपद्येत धातुद्वयस्याप्यकर्मकत्वात्~। तत्र च उष्ट्राणाम् यादृशान्यासनानि, हतानाम् यादृशानि शयनानि तादृशानि देवदत्तादिकर्तृकाणि आसनानि, शयनानि इत्यर्थः~। सादृश्यवशात् आख्यातवाच्यस्यापि भावस्य भेदप्रतीतेः बहुवचनम्~। न चैवं तिङन्तवाच्यभावस्यासत्वरूपताभङ्गापत्तिरिति वाच्यम्; लिङ्गकारकयोगाभावमात्रेण असत्वरूपताया अहानात्~। 

\section*{कृद्वाच्यभावस्य क्वचिद्वैलक्षण्यम्} 

कृदभिहितो भावो द्रव्यवत्प्रकाशते इत्यंशो निरूपितः~। कृदभिहितस्य भावस्य लिङ्गसंख्याभ्यां योगोऽस्ति~। अत एव घञाद्यन्ते पाकः, पाकौ, पाकाः इति लिङ्गसंख्ययोः प्रतीतिः, तस्य कारकत्वेन क्रियान्वयश्च~। पाकं करोति, पाकेन निर्वृत्तम् इत्यादि यथा~। तत्रादिपदेन इनुण्, अप्, किः, क्यप्, क्तः (नपुंसके भावे क्तः) ल्युट् (ल्युट् च) स्त्रियां क्तिन्नित्यधिकारे विहिताः प्रत्ययाः इत्येते ग्राह्याः~। अत्र सर्वत्रापि प्रत्ययवाच्यो भावः सत्वरूपतामापन्नः~। तुमुनोऽप्यर्थो भावः, अव्ययकृतो भावे इति वचनात्~। परन्त्वयं भावः असत्वभूतः इतीतरस्मात्कृद्भावाद्विशिष्यते~। कृन्मेजन्तः इति अव्ययसंज्ञाऽस्य विधीयत इति अव्ययार्थे च लिङ्गसंख्याद्यन्वयाभावेन प्रकृत्यर्थापेक्षया प्रत्ययार्थे वैलक्षण्यं नानुभूयते~। तुमुना समानार्थे (तुमुण्ण्वुलौ क्रियायां क्रियार्थायाम्) विधीयमानोऽपि ण्वुल्प्रत्ययः कर्तुरेव वाचकः शब्दशक्तिस्वाभाव्यात्~। कृष्णं दर्शको याति~। तत्र कृष्णकर्मकभविष्यद्दर्शनोद्देश्यकं दृशिकर्तृकर्तृकं यानमिति बोधः~। एवं क्त्वाप्रत्ययार्थोऽपि भावः~। भुक्त्वा व्रजति~। तुमर्थे इत्यनुवृत्तिं स्वीकृत्य अव्ययकृतो भावे इति वचनस्य भाष्ये प्रत्याख्यानादयमपि भावः साध्यावस्थ एव~। यद्यपि सः धातुनैव लभ्यते तथापि तदनुवादकत्वं तस्य क्रियान्तरे विशेषणत्वं च~। 

\section*{तद्धितभावः –}

“तस्य भावस्त्वतलौ” इति सूत्रे निर्दिष्टोऽयं भावः~। नायं धात्वर्थभावः त्वतल्प्रत्ययान्तेषु तस्यासम्भवात्~। विशेषणीभूतो धर्म एव भावशब्देनोच्यते~। यथा गोर्भावो गोत्वम्, गोता~। त्वतलौ यस्मादुत्पत्स्येते तत्प्रकृतिविशेषात् व्यक्तौ बुबोधयिषितायां यद्विशेषणतया भासते स धर्मोऽत्र भावः~। गोशब्दाद्व्यक्तिबोधे जायमाने गोत्वं विशेषणतया भासते गोशब्दस्य च गोत्ववतीषु व्यक्तिषु गोत्वे च शक्तत्वात्~। सर्वासु गोव्यक्तिष्वनुगतं तदितरव्यक्तिभ्यो व्यावृत्तं कञ्चिद्धर्मविशेषं शक्यतावच्छेदकं पुरस्कृत्य गोशब्दः प्रवर्तते~। घटादिशब्दा अपि घटत्वादितत्तद्धर्मं पुरस्कृत्य प्रवर्तन्ते~। तथा च शब्दस्य प्रवृत्तिनिमित्तं भाव इति तात्पर्यम्~। तदेवोक्तं प्रकृतिजन्यबोधे प्रकारीभूतो धर्मो भाव इति~। अयं च धर्मः क्वचिज्जातिः, क्वचिद्गुणः, क्वचित्क्रिया, क्वचिच्छब्दः भवति~। गोत्वम्, शुक्लत्वम्, पाचकत्वम्, कुत्वम् इत्युदाहरणानि क्रमेण~। क्वचित् सम्बन्धोऽपि त्वतलोरर्थः~। तदुक्तम् - कृत्तद्धितसमासेभ्यः सम्बन्धाभिधानं भावप्रत्ययेन इति~। यथा पाचकत्वमित्यत्र कर्तृत्वरूपसम्बन्धः~। राजपुरुष इत्यत्र स्वत्वरूपसम्बन्धः, औपगवत्वमित्यत्र जन्यत्वसम्बन्धश्च प्रकारः~। 

इत्थं भावशब्दः धात्वर्थव्यापारे, सिद्धावस्थापन्नक्रियायां, प्रवृत्तिनिमित्तरूपे इत्येवं पाणिनीये नानार्थेषु सङ्केतितो वर्तते~। तद्विचारोऽत्र सङ्क्षेपेण निरूपितः~। इति शिवम्~॥

\articleend
}
