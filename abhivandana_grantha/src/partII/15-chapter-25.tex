{\fontsize{15}{17}\selectfont
\presetvalues
\chapter{जातिशक्तिवादः}


\begin{center}
\Authorline{डा~॥ नागपतिभट्टः}
\smallskip

सहायकप्राध्यापकः\\
श्री राघवेन्द्रभरती-सवेदसंस्कृत-पाठशाला\\
कवलक्कि, होन्नावर
\addrule
\end{center}

पदानां कार्यान्विते शक्तिरिति पक्षं सयुक्तिकं निरूप्य मणिकारः पदानां व्यक्तौ शक्तिः उत जातौ इति विचाराय प्रक्रमते~। 

\section*{प्राभाकरमतम्}

गवादिपदानां गोत्वादिजातौ शक्ति:~। यद्यपि गामानय इत्यादि व्यवहारात् गवादिव्यक्तावेव शक्तिः गृह्यते , न तु गोत्वादिजातौ तथापि न व्यक्तौ शक्तिः स्वीकर्तुं शक्यते~। तथाहि- ‘ गोपदस्य सकलगोव्यक्तिषु शक्त्युपगमे शक्यगोव्यक्तीनां आनन्त्येन शक्त्यानन्त्येन गौरवम्~। 

गोपदस्य यत्किञ्चिद्गोव्यक्तौ शक्त्युपगमे तद्विषयकोपस्थितिं प्रति तद्विषयक शक्तिज्ञानं कारणम् इति कार्यकारणभावस्य व्यभिचारः~। गोव्यक्त्यन्तरे शक्तिग्रहाभावेऽपि तदुपस्थितेः जननात्~। 

गोपदस्य मिलितासु सकलगोव्यक्तिषु शक्त्युपगमे ‘गां दद्यात्’ इत्यादौ सकलगोव्यक्तिषु दानकर्मतायाः बोधनीयतया तस्याश्च बाधितत्वेन अप्रामाण्यप्रसङ्गः~। 

गोपदस्य अन्यतमगोव्यक्तौ शक्त्युपगमे ‘गां दद्यात्’ इत्यादिवाक्यात् ‘इयं गौः गोपदशक्त्या उत अन्या’ इति सन्देहोदयेन निर्धारणाभावात् अप्रवृत्तिप्रसङ्गः~। 

गोव्यक्तिसामान्यं गोपदशक्यमित्युपगमे व्यक्तेरशक्यतया प्रतीत्यनुपपत्ति:~। तत्तद्रूपेण सकलगोव्यक्तीनां गोपदशक्यत्वे सर्वेषां तेषां रूपाणां ज्ञातुमशक्यतया शक्तिग्रहानुपपत्ति:~। शक्यतावच्छेदकभेदात् शक्तिभेदाच्च गौरवम्~। अगृहीतशक्तिकतया अज्ञातरूपावच्छिन्नासु गोव्यक्तिषु गोशब्दव्यवहाराभावप्रसङ्गश्च~। 

गोत्वोपलक्षितानां धानकर्मव्यक्तीनां धेनुपदशक्यत्वमिव गोत्वोपलक्षितानां सकल\-व्यक्तीनां गोपदशक्यत्वमित्युपगमस्तु न सम्भवति , गोपदशक्यव्यक्तिषु गोत्वस्य उपलक्षणत्वायोगात्~। यतो हि रूपान्तरेण विज्ञातमन्येन रुपेण उपलक्ष्यत इति नियमः~। यथा उत्तृणत्वादिना\break विज्ञातो गृहविशेषः काकवत्त्वेन उपलक्ष्यते~। गोव्यक्तिस्तु गोत्वादन्येन सास्नादिमत्वादिना न हि पूर्वंविज्ञायते~। गोपदश्रवणेन ‘गौः’ इत्येव प्रतीतेः~। किञ्च यद्विशिष्टे कार्यान्वयः तद्विशेषणम्, यदन्यविशिष्टे कार्यान्वयः तदुलक्षणमिति लक्षणानुसारं गोत्वस्य नोपलक्षणत्वम्~। ‘ गां पश्य ’ इत्यादौ गोत्वविशिष्टे दर्शनादिकार्यस्य अन्वयात्~। अपि तु विशेषणत्वमेव~।\break उपलक्षणत्वन्तु ‘ अयं वासस्वी देवदत्तशब्दवाच्य ’ इत्यादौ वाससः सम्भवति~। वासोविशिष्टे देवदत्तशब्दवाच्यत्वस्य अनन्वयात् , वाससि तद्बाधात्~। गोत्वविशिष्टानां व्यक्तीनां गोपदशक्यत्वम्~। शक्यतावच्छेदकस्य गोत्वस्यैक्यात् सकलगोव्यक्तिनिष्ठशक्तेरैक्यं निर्वहति इत्यपि न सम्भवति~। 

दण्डनिष्ठायां घटकारणतायामवच्छेदकीभूतस्यापि दण्डत्वस्य अकारणत्वमिव गोव्यक्ति\-निष्ठायां गोपदशक्यतायाम् अवच्छेदकस्य गोत्वस्य अशक्यत्वेन गोपदजन्यशाब्दबोधे तस्य प्रकारतया भानानुपपत्तेः~। तत्प्रकारकशाब्दबोधे वृत्तिग्रहाधीनतदुपस्थितेः हेतुत्वात्~। गोव्यक्तिषु गोपदशक्तेः ऐक्यानुपश्पत्तेश्च~। शक्यस्यैव शक्यानुगमकत्वनियमेन शक्यत्वे सति शक्यश्तावच्छेदकं यत् तदैक्यस्यैव शक्त्यैक्यप्रयोजकत्वात्~। 

यथा धेनुपदस्य शक्यानां धानकर्मकव्यक्तिविशेषाणाम् अनुगमकं गोत्वम् , अशक्यमपि ( शक्यतावच्छेदकम् ) शक्त्यैक्यप्रयोजकमित्युपगतं गुरुमते तथा गोपदस्य शक्यानां गोव्यक्तीनां अनुगमकं गोत्वम् अशक्यं सदपि शक्त्यैक्यं प्रयोजयति~। शाब्दबोधे प्रकारतया गोत्वस्य भानं तु गोपदशक्तिग्रहप्रयोज्यगोत्वप्रकारकसंस्कारसहकृतात् गोपदात् सम्भवति इत्यपि न युक्तम्~। 

यतो हि धेनुपदशक्यानां धानकर्मव्यक्तिविशेषाणाम् अनुगमकस्य गोत्वस्य शक्यत्वं न सम्भवति , अतिप्रसक्तत्वात्~। धेनुपदशक्यत्वाभाववति वृषभेऽपि गोत्वस्य सत्वात्~। 

गोपदशक्यानां गोव्यक्तीनां तु अनुगमकं गोत्वं अनतिप्रसक्ततया शक्यं भवेदेव~। तथा च आवश्यकत्वात् लाघवाच्च गोत्वस्यैव शक्यत्वं न तु गोव्यक्तीनामिति~। 

अस्तु गोत्वमपि शक्यम्~। तथा च गोत्वविशिष्टं गोपदशक्यम् उपगन्तव्यम्~। न तु गोत्वमात्रम् गोत्वविशिष्टे गोशब्दव्यवहारात्, गोशब्दश्रवणेन गोत्वविशिष्टस्य उपस्थितेश्च इति न युक्तम्~। 

विशेष्यभेदेन विशिष्टानां भेदात् विशेष्यीभूतव्यक्तीनामानन्त्येन विशिष्टानामानन्त्यात् विशिष्टस्य शक्यत्वे शक्त्यानन्त्यस्य तादवस्थ्यात्~। 

विशिष्टानां भेदेऽपि यावद्विशिष्टनिष्ठावच्छेद्यायाः ऐक्यं सम्भवति, अवच्छेदकैक्यस्य अवच्छेद्यैक्यप्रयोजकत्वात्~। अवच्छेदकस्य च विशेषणीभूत गोत्वादिजातेरैक्यात् इति न शक्त्यानन्त्यम् इति न च वाच्यम्~। 

व्यक्तौ जातेः विशेषणत्वेन ‘नागृहीतविशेषणाबुद्दि: विशेष्यमुपसङ्क्रामति इति न्यायेन जाते: वाच्यताया: अवश्यमुपगन्तव्यतया न व्यक्ते: वाच्यत्वं गौरवात्~। 

जातौ व्यक्ति: विशेषणम्~। गोत्वे गवेतरावृत्तित्वे सति सकलगोवृत्तित्वस्य विशेषणत्वात् तस्य च व्यक्तिघठितत्वात्~। तथा च नागृहीतविशेषणन्यायेन व्यक्तेरपि वाच्यत्वं सिध्यति इति न युज्यते~। जाते: स्वत एव व्यावृत्तत्वेन तत्र व्यक्ते: विशेषणत्वाभावात्~। अन्यथा जातिव्यक्त्योः ज्ञप्तौ अन्योन्याश्रयप्रसङ्गात्~। स्वतो व्यावृत्तत्वंच स्वाश्रयवत् स्वस्मिन् व्यावृत्तधीजनकस्वभाववत्त्वं नैयायिकमतेविशेषवत्~। जाते: स्वतो व्यावृत्तत्वानभ्युपगमे तद्व्यावर्तकस्य उपाधेः पुनरन्यो व्यावर्तक:, तस्यापि अन्य इत्येवम् अनवस्थाप्रसङ्ग इति~। 

जाते: स्वतो व्यावृत्तत्वेऽपि तत्व्यक्ते: विशेषणतया भानमावश्यकम्~। अन्यथा तन्मात्रावगाहिनो व्यक्त्यनवगाहिनो ज्ञानस्य निर्विकल्पकतया जातौ शक्तिग्रहानुपपत्तिप्रसङ्गः~। तथा च “नागृहीतविशेषणन्यायेन” व्यक्तेरपि वाच्यत्वम् आवश्यकमिति न युज्यते~। जातिव्यक्त्यो: द्वयो: वाच्यत्वे गौरवेण जातिमात्रस्य वाच्यत्वात्~। तथा च जातिमात्रं पदार्थ: न तु व्यक्तिरपि~। 

व्यक्ते: अपदार्थत्वे व्यक्तिबोधानुपपत्ति:~। तद्विषयकशाब्दबोधे तद् विषयकशक्तिज्ञ्~॥ नस्य हेतुत्वात् इति न च चिन्तनीयम्~। जातिविशिष्टव्यक्तिविषयकशाब्दबोधं प्रति जाति\-शक्तिज्ञ्~॥ नं कारणमिति कार्यकारणभावस्वीकारात् जातिशक्तिज्ञानेन व्यक्तिशक्तिज्ञ्~॥ नं\break विनापि व्यक्तिबोध उपपद्यते~। यथा न्यायमते व्यक्तिज्ञ्॥नेन संसर्गबोध:~। यद्वा जातिविषयकबोधं प्रतिजातिशक्तिज्ञानं कारणं, व्यक्तिविषयकबोधं प्रति जातिशक्तपदं कारणम्~। यथा न्यायमते अन्वये अशक्तमपि गोपदं स्वार्थस्य अन्वयं बोधयति तथा मन्मते व्यक्तौ अशक्तमपि गोपदं स्वार्थस्य आश्रयं बोधयति~। अथवा जातिविषयकबोधे जातिशक्तिज्ञानं कारणं, व्यक्ति\-बोधं प्रति जातिज्ञानं कारणम्~। जातिव्यक्त्यो: एकवित्तिवेद्यत्वनियमात्~। तथाहि यो येन विना न भासते तद्धीहेतुस्तम् अवबोधयति~। यथा ज्ञानं विषयं विना न भासते अत: ज्ञानधीहेतु: विषयं घटादिकम् अवभासयति~। यथा न्यायमते क्षित्यङ्कुरादिकारणीभूतस्य ज्ञानादे: साधक\-मनुमानमेव तस्य नित्यत्वं साधयति~। तथा जातिभासकसामग्री व्यक्तिमपि अवबोधयति व्यक्त्या विना जाते: अभानात्~। 

व्यक्तिज्ञानस्य जातिज्ञानहेतुकत्वे संविद्भेदात् तुल्यवित्तिवेद्यत्वानुपपत्तिः~। अहेतुकत्वे जातिहेतुकत्वे च नित्यत्वप्रसङ्ग इति न च वाच्यम्~। व्यक्तिज्ञानं प्रति जातिशक्तेः कारणत्वोपगमात्~। 

जातिधीहेतुः व्यक्तिभासकः इति न सम्भवति~। जातिम् अविषयीकृत्य प्रत्यक्षादिना\break व्यक्तिज्ञानोत्पादात् व्यक्तिधीसामग्र्याः जातिज्ञानसामग्रितो भिन्नत्वात्~। जातिविशिष्टव्यक्तिभानं तु उभयज्ञापकसामग्रीद्वयसमाजात् आर्थम्  इति न च वाच्यम्~। जातिव्यक्त्योः  प्रात्यक्षिकबोधे सामग्रीद्वयस्यपृथक् अव्ययव्यतिरेकग्रहात् जातिप्रत्यक्षहेतुः व्यक्तिप्रत्यक्षजनक इति नियमासम्भवेऽपि तयोः शाब्दबोधे पृथक् सामग्रीद्वयं निर्युक्तिकत्वात् नाङ्गीक्रियते~। तथा च जातिशाब्दबोधजनिका जातिशक्तिज्ञानादिसामग्र्येव व्यक्तिबोधजनिका नान्यसामग्री~। 

ज्ञानशक्तिवादे पदस्य यद्विषयक ज्ञानजननशक्तिः तत् शक्यम्~। तथा च गोपदस्य\break गोव्यक्ति गोत्वजातिएतदुभयविषयक बोधजननशक्तिमत्वात्तदुभयम् शक्यम्~। तथा च व्यक्ति\-शक्तिसिद्धेरिति न च वाच्यम्~। यद्विषयतानिरूपितशक्तिविषयता शाब्दबोधजनकतावच्छेदिका तस्यैव शक्यत्वोपगमात्~। जातिविषयतायाः जनकतावच्छेदककोटि-प्रविष्टतया \-जातेः शक्यत्वसम्भवेऽपि व्यक्तिविषयतायाः लाघवात् जनकतावच्छेदककोटौ अप्रवेशेन व्यक्तेः \-अशक्यत्वात्~। 

शक्तिविषयता नियतविषयताशालित्वमेव शक्यत्वं भवतु , लाघवात्~। तथा च व्यक्तेः  शक्यत्वं सिध्यतीति न वाच्यम्~। तथा सति मिति-मातृणामपि शक्यत्वापत्तेः~। 

\section*{कुब्जशक्तिवादिनः}

गोपदात् जातिव्यक्ति- उभयविषयकबोधजननात् तदुभयत्र गोपदस्य शक्तिरस्ति~। किन्तु जात्यंशे सा शक्तिः ज्ञाता उपयुज्यते, व्यक्त्यंशे स्वरूपसती उपयुज्यते~। अयमेव कुब्जशक्तिवादः~। नैयायिकेनापि अन्वयांशे कुब्जशक्तिः स्वीक्रियते~। इत्थञ्च व्यक्तेः शक्यत्वेपि वाच्यत्वं नास्ति, यतो हि ज्ञातशक्तिमच्छब्दजनितज्ञानविषयत्वरूपं वाच्यत्वम्~। 

शब्दजन्यज्ञानविषयत्वमेव लाघवात् अस्तु वाच्यत्वं , तथा च व्यक्तेरपि वाच्यत्वं       सिध्यतीति न वाच्यम्~। तथा सति लाक्षणिकगङ्गादिपदादेरपि वाच्यत्वप्रसङ्गात्~। 

जातेरिव व्यक्तेरपि पदादुपस्थितिः शक्तिसाध्येति व्यक्तिरपि पदश्क्या इति न शङ्कनीयम~। पदादुपस्थिते अनन्यलभ्ये एव शक्तिस्वीकारात्~। अन्यलभ्येऽर्थे शक्त्यकल्पनात्~। अन्यथा व्यायमते पदस्य अन्वयेऽपि शक्तिस्वीकारप्रसङ्गः~। गङ्गादिपदानां तीरादौ लक्षणायाः उच्छेदप्रसङ्गश्च~। तत्रापि शक्तिस्वीकारस्य दुर्वारत्वात्~। 

जातेः पदवाच्यत्वे तत्र व्यवहारेण  शक्तिग्रहः दुरुपपादः~। कारकोपरक्तक्रिया हि व्यवहारगोचरो भवति~। जातिस्तावत् न क्रिया , नापि कारकम्~। अतो न व्यवहारः इति नाशङ्कनीयम्~। यतो हि केवलव्यक्तेः न कारकत्वं सम्भवति~। न हि ‘गौःगच्छति’ इत्यत्र व्यक्तिमात्रं भातीति कस्यचित् प्रतीतिः~। किन्तु जातिविशिष्टया व्यक्तेः कारकत्वं~। तथा च जातेः कारकत्वेन व्यवहारगोचरत्वात् व्यवहारेण तत्र शक्तिग्रहः सूपपाद एव~। 

\section*{श्रीकरमतम्}

‘गौःगच्छति’ इत्यादौ व्यक्तेः अनुभवः औपादानिकः, जात्यनुभवः शाब्दः~। यद्यपि जातिव्यक्त्योः एकोऽनुभवः जायते~। तथापि तत्र जातिभाने पदं प्रयोजकं व्यक्तिभाने जातिः प्रयोजिका~। जातेः व्यक्त्यनुभावकत्वमेव उपादानार्थः~। 

गोव्यक्तेः गमनादिक्रियया आश्रयतासम्बन्धेन अन्वयः, गोत्वस्य तु अवच्छेदकतया यथा ‘अरुणया पिङ्गाक्ष्या एकहायन्या गवा सोमं क्रीणाति’ इत्यत्र गोः करणत्वेन क्रयणक्रियान्वयःआरुण्यस्य अवच्छेदकतया तदन्वयः~। 

इदन्तु न शोभनम्~। तथाहि जातिशक्त्यैव व्यक्तिधीसम्भवे व्यक्तिधियं प्रति जातेः कारणत्वस्य अक्लृप्तस्य कल्पनायां गौरवम्~। व्यक्तिलाभाय गौः गच्छति इत्यादौ जातिशक्त्यतिरिक्तप्रमाणाश्रयणे ‘व्रीहीनवहन्ति’ इत्यत्रापि लक्षणाश्रयणप्रसङ्गः~। 

\section*{ज्ञानशक्तिवादे व्यक्तिशक्त्युपपादनम्}

‘गोत्वज्ञाने शक्तं गोपदम्’‘गोज्ञाने शक्तं गोपदम्’ इत्याकारको वा शक्तिग्रहः, गोव्यक्तिं विषयीकृत्यैव गोत्वजातिमवगाहते, जातिव्यक्त्योः      तुल्यवित्तिवेद्यत्वात्~। तथा च शक्तिज्ञाननिष्ठकारणतायाः अवच्छेदककोटौ जातिविषयताया इव व्यक्तिविषयताया अपि प्रविष्टत्वात्\break व्यक्तेः शक्यत्वं दुर्वारम्~। नागृहीतविशेषणन्यायात्~। व्यक्तिमविषयीकृत्यापि शक्तिज्ञाने जाति\-भानस्योपगमे तुल्यवित्तिवेद्यत्वनियमस्य परिहृतत्वात् शब्दात् व्यक्ति प्रतीतिनिर्वाहाय तत्र\break पदवाच्यत्वस्वीकार आवश्यकः~। यद्धर्मवत्तया ज्ञाते एव यत्र यस्य ज्ञानं    स धर्मः तन्वि\-ष्ठायाः तस्याः अवच्छेदक इति नियमः~। व्यक्तिविषयकत्वेन ज्ञात एव जातिज्ञाने गोपदशक्तेः ज्ञानोद\-बोधात् व्यक्तिविषयकत्वम् जातिज्ञाननिष्ठायाः गोपदशक्यतायाः अवच्छेदकम्~। 

दर्शितनियमो व्यभिचरितः~। मितिमातृविषयकत्वेन ज्ञाते एव घटादिज्ञाने घटादिपदशक्यतायाः गृहणेऽपि मितिमातृविषयकत्वयोः घटादिपद- शक्यतावच्छेदकत्वा-नभ्युपगमात्~। एवं न्यायनयेऽपि आद्यव्युत्पत्तिसमये कार्यान्वितत्वेन ज्ञाते एव घटादौ घटादि\-पदशक्यतायाः ग्रहणेऽपि कार्यान्वितत्वस्य घटादिपदशक्यतावच्छेदकत्वानभ्युपगमात् इति नाशङ्कनीयम्~। यतो हि मितिमातृविषयकत्वेन कार्यान्वितत्त्वविषयकत्वेन च अज्ञातेऽपि घटादिज्ञाने घटादिपदशक्यताया: अवगाहनेन तयो: न शक्यतावच्छेदकत्वम्~। व्यक्तिविषयकत्वेन तु\break रूपेणाज्ञाते घटत्वादिज्ञाने घटादिपदशक्यत्वस्य अवगाहनं नैव सम्भवति, व्यक्तिमविषयीकॄत्य जाते: भानाभावात्~। 

उक्तनियमानभ्युपगमेऽपि व्यक्ते: पदवाच्यत्वं सिध्यति~। तथा हि घटादिपदात् जाति\-व्यक्ति-उभयावगा-प्रतीतिस्तावत् सर्वसम्मता~। तथा हि घटपदमेव तदुभयप्रतीतिकारणत्वान्यथानुपपत्या तदुभयविषयकत्वस्य शक्यतावच्छेदकत्वं कल्पयति~। प्रयोगस्तु-घटादिपदं जातिव्यक्त्युभय-विषयकत्वावच्छिन्नज्ञाननिष्ठशक्यताकं जातिव्यक्त्युभयविषयकशाब्द\-\break बोधजनकत्वात् इति~। जात्याश्रयत्वादे: कारणान्तरस्य व्यक्तिभासकस्य कल्पनायां तु \-गौरवम्~। 

\section*{व्यक्तिशक्त्युपपादने युक्त्यन्तरम्} 

प्रथमं व्यवहारेण शक्तिग्रहकाले घटादिपदानां घटादिव्यक्तौ शक्तिः अवधार्यते, न तु जातौ~। तस्याः व्यवहाराविषयत्वात्~। व्यक्तेरेव व्यवहारविषयत्वात्~। पश्चाच्च व्यक्तेः व्यावृत्यर्थम् अनुगमार्थं च जातिरपि व्यवहारविषयतां प्रमाणान्तरेण विनिश्चित्य तत्रापि पदस्य शक्तिः निर्णीयते~। तथा च जातिशक्तिज्ञानस्य व्यक्तिशक्तिज्ञानं प्रत्युपजीवकत्वात् जातिशक्तिज्ञानेन व्यक्तिशक्तिज्ञानस्य नान्यथासिद्धिः सम्भवति~। उपजीव्यविरोधात् इति~। 

अत्रास्वरसबीजम्- व्यक्तेः व्यावृत्यर्थम् अनुगमार्थमेव जातेः शक्तिकल्पनम् इत्युक्तं, लाघवादिविनिगमकस्य जातौ शक्तिकल्पकस्य विद्यमानत्वात्~। किञ्च, व्यक्तेः अन्यलभ्यत्वे प्रथमक्लृप्तापि व्यक्तिशक्तिधीस्त्याज्यैव~। व्यक्तेः अनन्यलभ्यत्वे तु तत एव व्यक्तिशक्तिसिद्धौ उपजीव्योपजीवकभावकल्पनं व्यर्थमेवेति~। 

\section*{जातिशक्तिवादे संस्कारेण व्यक्तिप्रतीत्युपपादनम्}

पदानां जातावेव शक्तिः, व्यक्तिप्रतीतिस्तु संस्कारात्~। तथाहि- जातौ शक्तिमवगाहमानो ग्रहः व्यक्तिमपि विषयीकतोति~। अतः सोऽयं शक्तिग्रहः नियमेन व्यक्तिविषयकसंस्कारं जनयति~। तथा च गवादिपदात् जातिविषयकस्मरणमुत्पद्यमानं व्यक्तिमपि नियमेन अवगाहते~। न च संस्कारस्य नियोतोद्बोधकत्वाभावात् शाक्तिग्रहजन्यस्मरणस्य नियमेन व्यक्तिविषयकत्वमनुपपन्नं इति वाच्यम्~। जात्यंशोद्बोधकस्यैव व्यक्त्यंशोद्बोधकत्वोपगमेन जातिस्मरणस्य नियमेन व्यक्तिविषयकत्वोपपत्तेः~। यथा संस्कारसहितात् इन्द्रियात् तत्राविशिष्ट इदन्तावगाहिनी प्रत्यभिज्ञा  समुदेति~। तत्राभाने संस्कारः इदन्ताभाने इन्द्रियं प्रयोजकं तथा संस्कारसहितात् पदावेव व्यक्तिविशिश्टानाम्  अवगाहमानो अनुभवो जायते , व्यक्तिभाने प्रयोजकः संस्कारः जातिभाने पदम्~। 

\section*{व्यक्तिशक्तिवादे शक्त्यैक्योपपादनम्}

विशिष्टानाम् आनन्त्येपि विशेषणीभूतस्य अवच्छेदकधर्मस्य ऐक्ये अवच्छेद्यैक्यं सम्भवति~। यथा विशेषणीभूतधूमव्यक्तीनां आनन्त्येऽपि धूमत्वेन कस्यचित् एकधूमव्यक्तौ वह्निव्याप्तिर्गृह्यते ‘धूमो वह्निव्याप्य’ इति ~। यथा गुरुमते कार्याणाम् आनन्त्येऽपि ‘कायर्म्शक्यम्’ इति लिङः क्वचित् अपूर्वे एककार्यत्वावच्छिन्ने शक्तिः गृह्यते, अवच्छेदकस्य कार्यत्वस्यैक्यात् \-शक्तेरैक्यम्~। यथा मीमांसकमतेऽपि 	

पश्वादिपदानां लोमवल्लाङ्गलयोगित्वलक्षणोपाधिमादाय क्वचित् पशु व्यक्तौ शक्तिग्रहः~। शक्यानां व्यक्तीनाम् आनन्त्येऽपि तत्तावच्छेदकस्य ऐक्यात् शक्तेरैक्यम्~। तथा गोपदस्य शक्यानां गोव्यक्तीनामानन्त्येऽपि शक्यतावच्छेदकगोत्वस्य ऐक्यात्, शक्तेरैक्यं सम्भवति~। { गोत्वेन रूपेण क्वचिद्गवि शक्तिर्गृह्यते ‘गौः शक्या’ इति } अथवा गोत्वसामान्यलक्षणया उपस्थितासु सकलगोव्यक्तिषु शक्तिर्गृह्यते ‘गौः शक्या’ इति~। न चैवं प्रमेयत्वेन रूपेण सकलप्रमेयज्ञानसम्भवात् सार्वज्ञ्यापत्तिः इति वाच्यम्  इृष्टत्वात्~। घटत्वादियावत् धर्मप्रकारेण ज्ञानं नेष्यते~। 

\section*{भाट्टमते व्यक्तिप्रतीत्युपपादनम्}

गवादिपदानां जातौ शक्तिः, व्यक्तिराक्षेपाल्लभ्यते~। न च जातिव्यक्त्योः एकवित्तिवेद्यत्वात् न आक्षिप्य - आक्षेपकभावः सम्भवति~। जातिः सामान्यधर्मः~। सामान्यञ्च समानानां भावः~। स च व्यक्तिं विना न भासते~। अतस्तयोः तुल्यवित्तिवेद्यत्वमेवेति वाच्यम्~। 

जातेः स्वरूपत एव शक्यत्वोपगमात्~। तथा च व्यक्तिघटितसामान्यत्वेन रूपेण जातेः अभानात् व्यक्तेः न तुल्यवित्तिवेद्यत्वम्~। जातेः स्वरूपतो भानानङ्गीकारे  निर्विकल्पेऽपि सा न भासेत~। 

न च व्यक्तिघटितेन सामान्यत्वेन रूपेण जातेः वाच्यत्वमावश्यकम्~। तथा सत्येव व्यक्तितो भिन्नतया जातेर्लाभेन व्यक्त्याक्षेपकत्वसम्भवात्~। व्यावृत्तत्वबुद्धिमन्तरा व्यक्तिविशेषाक्षेपकतासम्भवात्~। तथा जातिवित्तिवेद्यैव व्यक्तिर्नाक्षेपलभ्येति वाच्यम्~। 

स्वतो व्यावृवृत्तजातेरेव वाच्यत्वात्~। व्यक्तेः तद्घटितसामान्यत्वस्य वा व्यावर्तकत्वे अन्योन्याश्रयात्  अनवस्थानाद्वा~। 

न च व्यक्तेः आक्षेपापेक्षया जातिवित्तिवेद्यत्वमिति उचितम्~। तथाहि गोपदात् गोत्वधीः ततः क्रमेण व्याप्तिज्ञानं ततो व्यक्त्यनुमितिः इति ज्ञानपरम्पराकल्पनायां गौरवम्~। ‘गौः’ इति प्रतीतौ जातिव्यक्त्योः युगपत्भानोपगमे लाघवम्~। तत्र सूक्ष्मकालभेदेन जातिव्यक्त्योः भानं, न तु युगपत् इति तु नाशङ्कनीयम्~। युगपत्भानोपगमे बाधकाभावात् इति वाच्यम्~। 

शब्दस्यबोधकतायाः शक्त्यधीनत्वात्~। लाघवात् जातेरेव पदशक्यतायाः सिद्धौ व्यक्तिलाभाय उक्तज्ञानपरम्परानुसरणस्य आवश्यकत्वात्~। अन्यथा ( गवादि पदात् व्यक्तेः आक्षेपेन लाभो यदि नानुमन्यते तदा ) ‘चैत्रःपचति’ इत्यादौ आख्यातादपि कर्तुः आक्षेपेन लाभो न स्यात्~। तत्रापि कृतिकर्त्रोः तुल्यवित्तिवेद्यत्वसम्भवात्~। 

न च गोशब्दजन्यप्रतीतिःव्यक्तिविषयिणी जातिज्ञानत्वात् ‘इयं गौः’ इति प्रतीतिवत्\break इत्थमनुमानेन व्यक्तेः तुल्यवित्तिवेद्यत्वं सिध्यतीति वाच्यम्~। अप्रयोजकत्वात्~।  जात्यवगाहिनः प्रत्यक्षप्रतीतेः व्यक्त्यवगाहितायाः व्यक्तिधी हेतुसमाजाधीनत्वात् जातिधीहेतुसमाना\-धीनत्वाभावात्~। व्यक्तिधीहेतोः संयोगादिसन्निकर्षस्य जातिधीहेतोः संयुक्तसमवायादि सन्नि\-कर्षस्य च परस्परभेदात्~। गोत्वं गवादिविषयप्रतीतिविषयः जातित्वात् गोभिन्नभावत्वाद्वा\break इत्यनुमानेन व्यक्त्यनवगाहिन्याः जात्यवगाहिप्रतीतेः सिद्धिसम्भवाच्च~। 

न च व्यक्तिः जातिवित्तिवेद्या जातिपरतन्त्रकत्वात् यत्परतन्त्रकत्वम्, यत्र तत्र तेनैकवित्तिवेद्यत्वं यथा ज्ञानपरतन्त्रकत्वं विषये तत्र ज्ञानेनैकवित्तिवेद्यत्वम्~। इत्यनुमानसम्भवात् व्यक्तेः जत्या तुल्यवि- त्तिवेद्यत्वसिद्धिः निष्प्रत्यूहैवेति वाच्यम्~। 

परतन्त्रत्वस्य अत्यूनानतिप्रसक्तस्य दुर्निर्वचतया तद्घटितव्याप्त्यसिद्देः~। तथाहि पर\-तन्त्रत्वं तावत् न परसमवेतत्वम्, गन्धपरतन्त्रकत्ववत्यां पृथिव्यां गन्धेन एकवित्तिवेद्यताविरहेण व्यभि\-चारात्~। नापि परधीनिरुप्यत्वं, ज्ञानस्य परधीनिरुप्यत्वस्य असिद्धया दृष्टान्तासिद्धेः~। नापि\break परस्मिन् भासमाने एव भासमानत्वम्~। तस्य साध्याविलक्षणतया स्वरूपासिद्धेः~। नापि विशेषण\-त्वेनैव भासमानत्वम्~। ‘गौः’ इति प्रतीतौ जातेः विशेषणतया भासमानत्वेऽपि ‘गवि\break गोत्वम्’ इति प्रतीतौ विशेष्यतया भासमानत्वेन स्वरूपासिद्धेः~। न च स्वनिष्ठविशेष्यताऽनिरूप\-कत्वे सति स्वनिष्ठविशेषणतानिरूपकं यज्ज्ञानं तद्विषयत्वमिति विवक्षणे नोक्तदोषः, ‘गौः’ इति\break प्रतीतिमात्रस्य तथात्वेन आदातव्यत्वात् इति वाच्यम्, जातौ विशेषणत्वस्य दुर्घटतया\break स्वरूपासिद्धितादवस्थ्यात्~। तथाहि- पदात् जायमानं जातिस्मरणं निर्विकल्पकमेव~। तत्र\break जातेः स्वरूपत एव भानात् न विशेषणत्वम्~। जातिविशिष्टव्यक्तिविषयक संस्कारादेव\break गोपदेन जात्यंशोद्बोधे सति जायमानं स्मरणं जातिमात्रावगाही भवति~। तथा च जातिविशिष्टविषयकसंस्कारः जातिमात्रावगाहि स्मरणं हेतुरङ्गीक्रियते~। निर्विकल्पकस्य संस्काराजनकत्वेन जातिमात्रावगाहि संस्कारस्य दौर्लभ्यात्~। अथवा निर्विकल्पकस्यापि संस्कारजनकत्वमुपगम्य जातिमात्रावगाहिसंस्कारो जातिमात्रावगाहिस्मरणहेतुराङ्गीकर्तव्यः~। न च\break स्मरणं विशिष्टज्ञानात्मकमेव न् तु निर्विकल्पकमिति वाच्यम्~। तथा सति विशिष्टज्ञानात्मकस्मरणं प्रति हेतुभूतस्य अनुभवस्यापि विशिष्टज्ञानात्मकतायाः आवश्यकतया निष्प्रकारकज्ञानात्मकस्य निर्वि\-कल्पस्य विलोमप्रसङ्गात्~। तुल्यवित्तिवेद्यत्वाभ्युपगमपक्षेऽपि शब्दजन्यस्य\break गोत्वादिस्मरणस्य निर्विकल्पत्वम् अवश्यम् अभ्युपगन्तव्यम्, गोव्यक्तिवृत्तित्वादिवैशिष्ट्य\break अशक्यत्वेन गोत्वे विशेषणतया भानासम्भवात्~। न च निर्विकल्पकं न स्मृतिः इन्द्रियजन्यत्वात् यथा सविकल्पकम् इत्यनुमानसम्भवात् निर्विकल्पस्य स्मरणरूपत्वं न सिध्यतीति वाच्यम्~। स्मृतित्वं निर्विकल्पकवृत्ति ज्ञानत्वसाक्षात्व्याप्यधर्मत्वात् यथा अनुभवत्वम् इत्यनुमानेन निर्विकल्पकस्य स्मरणरूपस्य सिद्धेः~। 

आक्षेपात् व्यक्तिलाभ इति पक्षस्य विमर्शः

गवादिपदात् जातिमात्रलाभः, व्यक्तिलाभस्तु आक्षेपात् इति भाट्टमतमत्र विचार्यते~। ननु गोपदात् गोत्वस्मरणे जाते ततः आक्षेपात् उत्पद्यमाने व्यक्तिज्ञाने व्यक्तिः व्यक्तित्वेन भासते उत गोत्वेन उतरूपान्तरेण~। न तावत् व्यक्तित्वेन नापि रूपान्तरेण~। ‘गौ आनय’ इत्यादिशब्दप्रयोगोत्तरे गोत्वस्यैव प्रतीतेः तयोरप्रतीतेः~। नापि गोत्वेन व्यक्तेर्भानं भवितुमर्हति~। तथा सति आक्षेप्ये गोत्वविशिष्टे अन्तर्गततया गोत्वस्य आक्षेपकत्वसम्भवात् इति चेत् न~। गोत्वेन रूपेण व्यक्तिः आक्षेप्यत्वेऽपि गोत्वांशस्य नाक्षेप्यत्वम् अपि तु, विशेष्यभूतव्यक्तिमात्रस्य आक्षेप्यत्वम्~। तथा च गोत्वं व्यक्त्याश्रितं जातित्वात् इत्यनुमानं फलति~। अत्र हेतोः पक्षधर्मताबलात् गोत्वाश्रयव्यक्तिः सिध्यति~। अथवा अर्थापत्या व्यक्तिः सिध्यति~। यथा हि दिवा अभुञ्जानत्वे सति पीनत्वस्य रात्रिभोजनमन्तरा अनुपपद्यमानतया देवदत्तस्य रात्रिभोजनं अर्थापत्या\break कल्प्यते तथैव व्यक्तिमन्तरा गोत्वादिजातेः अनुपपद्यमानतया व्यक्तिः कल्प्यते इति~। 

\section*{आक्षेपात् व्यक्तिलाभपक्षस्य निराकरणम्} 

आक्षेपात् अर्थापत्या वा व्यक्तिलाभोपगमे गामानय इत्यादि वाक्यात् क्रियान्वयबोधो नोप\-पद्यते~। तत्र हि गोत्वविशिष्टस्य आनयनक्रिया बोध्यते~। तत्र च बोधे गोविशेष्यक गोत्व\-प्रकारकः ‘गौ’ इत्याकारको बोधः कारणम्~। स च गोशब्दादेव उपपादनीयः न तु आक्षेपात्~। आक्षेपात् ‘गोत्वं व्यक्त्याश्रितम्’ इत्याकारकस्यैव बोधस्य जननात्  न चास्य बोधस्य क्रियान्वयबोध हेतुत्वमात्रमिति वाच्यम्~। ‘गामानय’ इत्यत्र द्वितीयाप्रकृत्युपस्थापितस्यैव आनयनक्रियान्वये साकाङ्क्षत्वात्~। व्यक्तेः अतथात्वेन क्रियान्वये आकाङ्क्षा विरहात्~। यथा राजपुरुषमानय इत्यत्र द्वितीयाप्रकृत्युपस्थापितस्य पुरुषस्यैव आननयनादि क्रियान्वये साकाङ्क्षता न तु पुरुषसम्बन्धितया उपस्थितस्य राज्ञः~। तथा प्रकृते गोपदात् उपस्थितस्य गोत्वस्यैव क्रियान्वयः स्यात् न तु गोत्वाश्रयतया उपस्थितायाः व्यक्तेः~। 

\articleend
}
